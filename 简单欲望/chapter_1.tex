\chapter{第一卷 无心处淡看云起}

\section{第六章 从连到团}

一年过后,太行山深处。

帅子清连长看着笔直的站在自己面前的楚云飞,皱着眉头长叹一口气。

这个家伙在初来时就显得有点怪异,带了10几年兵的“帅哥”(大家都这么称呼帅连长)硬是没见过这种类型的兵。

参军的时候年纪才16岁,这很正常,不少城市兵还有个别农村兵都是这个年纪参军的,不过那些主儿都进了军分区、后勤机关或者部队企业什么的了,来野战部队的可没几个,这是一怪。

这家伙刚来的时候身体单薄,脸色发白,一副站也站不住的模样,可谁也没想到那单薄的身体内竟蕴涵着那么大的潜力,在训练的时候拼命上量,还经常自己单练,一副不想活了的样子,而且他还全都挺过来了。别人做3练习,他在暗处悄悄的做5练习6练习,想当杀手还是打手啊?和平年代这种训练态度算是二怪。

在3小时内背会《新兵须知》并且考试第一;文采惊人,经常负责板报文章和连内书面材料书写;汽车维修、通讯维护样样精通;聪明异常且从不提及参军前的事情,大家总是纳闷这个看来绝对能考上大学的家伙为什么会来参军。这是三怪。

感觉极其敏锐,在对抗练习中总能发现潜在的危险,他一潜伏,找他比在大海里找针容易不了多少。有人问起原因只会微笑,着急了就胡说是天性或者上辈子当过兵,比侦察兵还象侦察兵,这是四怪。

部队里有只狗,大小和牛犊相仿,皮毛油光发亮,人称“老油”,禀性善良但爱与人争斗,无聊者争相空手与其格斗,从来是人方输,但在不久前,身高已窜至170厘米但依旧销瘦的楚云飞打赢了人犬大战,并且成功的多次卫冕;同时他也是团内神枪之一,出枪奇快,枪感好,综合实力强大,见过点世面的“帅哥”感觉这家伙完全有能力参加参加全军大比武甚至获得名次。身手敏捷,实力超群姑且就算第五怪吧。

总之,除了年纪小点,天生的、绝对的优秀军人啊。

一怪两怪不算怪,不过有此5怪那绝对就是怪异了。

梁东民也是先阳兵,和楚云飞同一年入伍,身宽体胖,性情梗直。按说一个连队里100来号人有同省的老乡都该好好珍惜,何况是同市的,所以梁东民刚开始虽然和大家一样有点看不起这个比他小两岁的瘦弱少年,但还是表示出了老乡应有的热情;楚云飞倒是来者不拒,两人关系很快融洽。后来由于楚云飞的怪异逐渐表现出来,慢慢的赢得了梁东民的尊重,可是怪异越来越明显,梗直的梁东民实在按捺不住好奇,就追问起来。楚云飞倒是也没有把自己的身世看得过于珍重,实在顶不住就说了,梁东民“这才知道”原来他是先阳市著名不幸人士的后人,作为连里少数几个知情人之一他是经常开导楚云飞的。

自从知道楚云飞的遭遇,热忱的梁东民总是琢磨怎么才能帮帮不幸的老乡,在他的鼓动下,他家里不知道从哪里找了本没封皮的“武功秘籍”给他寄了过来,梁东民把这发黄的本本塞给楚云飞,自己是坚决的不参与修炼,估计是他心里也没底吧。

来者不拒是楚云飞最大的特点之一,何况是这种他最在意的物件儿?虽然没有任何前辈的指点,自信心十足的他还是按照秘籍尝试操作,修炼多日,居然没练出什么问题还隐隐感觉体内有“气”了,于是楚云飞热情马上高涨,拼命修炼,倒是把这本其实不怎么样的秘籍修炼到了一定高度。直接后果就是身体变敏捷,而且成了唯一能让“老油”躲着走的人类。

这个连队地处太行山区,周边10里只有两个小小的村落和其他部队,离最近的小镇义井都有60多里。

山民是比较贫穷的,来钱的地方不多,于是就有脑瓜灵活的村民想到低价收购部队的“剩余物资”自己用或去镇上倒卖,比如说战士穿的大头鞋、军装、帽子甚至军大衣军用被褥,凭良心说军队的“物资”质量是相当过硬的。同时战士中也确实有比较节省的或者缺钱的,就把多余的物资卖掉,甚至有个别战士偷盗与自己合不来的战友的“物资”去出售,不过后一种情况一旦被失主查获当事人的下场会是极其悲惨的。

贫穷的村民一旦有了来钱或者说省钱的途径,就不可能再回到从前安于贫穷的心态。于是问题就来了:永远没有人会嫌钱多,再于是就有眼红的村民想到了“盗窃”并积极的操作起来,久而久之,部队“失窃”现象就严重起来。于是只负责看军械看猪的“老油”们兼起了“抓小偷”的任务。

可俗话说得好:不怕贼偷,就怕贼惦记。有狗报警并不能杜绝盗窃。这天,连丢上衣和大头鞋的梁东民居然看到了自己做了暗记的失物穿在某村民身上,怒火冲天的他根本没想对方有5人而自己只是1人,直接冲上去就使用暴力。

村民成天劳作,身体肯定强壮,何况还是敢打狼和豹子的山民?彪悍的山民们在突来的咒骂声中惊见一个“村友”被打倒,血性发作,对肇事者进行了疯狂的报复,很快将肇事者击倒在地,而且并不因为梁东民已经躺在地上就终止对他的殴打。

恰好此时,楚云飞做完“功课”回营路过,听到异声蹑手蹑脚赶来,见到这种情况那是绝对不可能忍受的。军营一年,打架已是常事,何况以“老油”尚且回避的强悍。于是强大的怪异立刻体现,擅长“潜行”的楚云飞暴起一拳先击晕一个看起来最壮实的村民,趁对方尚未从殴打的快意中清醒过来之间又踢飞一个村民,几个回合过后,剩下的两个彪悍在正面的对抗中被更彪悍击倒。

楚云飞本想扶起战友就此回营,却发现红肿的“猪头”居然……居然是梁东民,这还了得?于是那几个村民下场之悲惨就不需要在这里赘述了。

中国的村庄其实和非洲的部落相似,那个村子的村民看到同“部族”被殴打的惨样,又知道人民的军队其实对人民下不了多黑的手这种常识,于是就有了相约而来,以全族之人力要求部队严厉惩罚打人凶手的景象。

惩罚梁东民其实没有任何说得过去的理由,何况他正躺在救护室里,所以“帅哥”此时正在为如何处理楚云飞给村民一个合理的交代而发愁。

楚云飞见“帅哥”迟迟拿不出主意,哪里知道首长正在回忆他的怪异,心情有一点点的不安,辩解道:“报告连长,虽然我做错了事,但是绝对不后悔,如果重新来一次,我还是会这样,您不用再犹豫了,该怎么处罚就怎么处罚我好了。”这也算是以退为进吧。

“帅哥”的思路被打断,点点头,“林指导员不在,不用开会决定了,这样吧,关三天禁闭,然后临时调你去义井团部避避风头,什么时候回来就等通知吧。”

把楚云飞当众请进“小黑屋”,“帅哥”回头对站在军营门口村民交代:“破坏军民团结,关他3天禁闭,然后送到团里接受处理,怎么样?”村民们不依不饶,“帅哥”的脸色开始狰狞,丢下一句“我的兵现在还在抢救!”掉头就走,有耳朵尖的村民隐约听见首长转身时嘀咕“偷东西还有理了。”

三天后,义井团部,楚云飞拿着连里的证明前来报到,负责接待的老兵王展翅挺奇怪这种待遇:“你是惹人了吧?”楚云飞把情况一五一十做了说明,王展翅点点头,“哦,是你呀,葛副团长说了,让你来了去见他。”

在葛副团长的办公室喊声“报告”,首长正在洗脸,“进来”,头也不回的喋喋不休,“解气,今天摔了老耿一个马趴,哈哈,——你是谁?”

楚云飞大声喊到:“报告首长,五连战士楚云飞向您报到!”

葛副团长点点头,顺手用毛巾擦干净脸,把毛巾搭起。拉开椅子坐下,抬头很有兴趣的看着站在桌前的战士:“破坏军民团结,你胆子不小啊。”

楚云飞一个立正:“报告首长,地下躺着我的战友,头脑发热,没想那么多。”

葛副团长的脸色变得好快:“这么说你还有理了?”

楚云飞不知道该说什么好,只好来个一声不吭,就当是默认。

看着小战士如此的冥顽不灵,葛副团长的脸色越发阴沉,“不吭气?不吭气就完了?别以为我不知道你怎么想的,你这种兵我见得多了,脸上一套,背后一套,是不是觉得该给你记功了?是不是感到很委屈啊?”

楚云飞知道不说话不行了,“啪”又是个立正,“报告首长,我没有。”

“没有?没有才怪?”葛副团长阴沉的脸上显示出“不详之兆”,“我们军队是人民的军队,人民是水,我们就是鱼,离开了水的鱼还能活么?”

“战友受伤?战友受伤就能对支持我们的人民群众下狠手么?人家都躺在地下了你还不罢休,年纪轻轻,好狠的心肠啊!”

楚云飞有点摸不着头脑,但直觉上他觉得应该表个态,又一个立正,“我错了,请首长处罚。”

葛副团长脸上慢慢阴转多云:“我觉得好象还是委屈了你了。”

又是立正,“报告首长,绝对没有。”

“恩”,葛副团长点点头,“先暂时相信你。这样吧,跟我来。”

葛副团长把楚云飞带到训练室,地上铺着厚厚的海绵垫子,“听说你和村民的冲突中使用了木棍等器械?实在太不象话了。”

想想初见到葛副团长时听到的话,楚云飞要是还弄不明白是怎么回事他就不是楚云飞了。他立刻表态,“报告首长,我没有用器械,——我可以证明。”

“哦?——怎么证明啊?”

“我擅长空手搏斗,我们帅连长可以证明,还有……还有就是您可以找个战士,我为您演示一下,保证速战速决。”试探的气球放出去了。

“哦,找个战士,——速战速决?”,葛副团长抿着嘴似乎沉思一下,“这样吧,临时找也不好找,我来吧。”

楚云飞决定配合一下葛副团长,同时再给自己弄点好处,做出一副迟疑的样子:“可是您是首长啊,我不能这么做。”

葛副团长大手一挥,“你不用顾忌。”

楚云飞压抑着自己的战斗欲望,坚决的摇摇头,开玩笑,不给好处凭什么跟你玩?

葛副团长的馋虫被勾引上来,底牌被小狐狸引了出来——“别就练了一张嘴吧?你要是能放倒我,那就不处罚你了。”

瞌睡的人如愿的收到俩枕头!!!

\section{第七章 高手的判断}

格斗开始了,为了尽快解决掉背上的包袱,楚云飞决定全力以赴。

两人的格斗使用的并不完全是军中流行的擒拿格斗着数,楚云飞有自己琢磨出的招数——秘籍上只有练气没有实战招数;而葛副团长却是正经科班——特种部队出身,不但着数多而且还有实战搏斗中领略的小技巧。

楚云飞越斗越吃力,越吃力越斗;葛副团长则是越斗越强,把自己的实力发挥的淋漓尽致,同时为楚云飞层出不断的新招暗暗叫好。

十几个回合过去了,楚云飞制服了葛副团长三次,僵持一次,其他就是孔夫子搬家——全是输(书),但第一个回合下来开始,两人就都很有默契的不提处罚,沉迷在男人的游戏当中了。

半小时后,终于停下来了,年纪奔四十的葛副团长大口的喘着粗气,“哈哈,过瘾。”小战士也拉着风箱讨好领导:“首长,呼~呼~,什么时候再来两盘?”

葛副团长坐在海绵垫子上汗流浃背,“你小子还真不错,帅子清没白给你说好话啊,——回头我闲了让他们喊你。”

葛副团长是特种兵出身,休闲时最大的爱好就是和人比赛格斗,其次是玩枪,属于典型的军人。爱好所至,本身条件也极佳,这样一来葛副团长的格斗术就是在特种兵里也找不出几个能跟他抗衡的。由于他酷爱比斗而实力强大,在团部里除了团长耿风是没有对手的,而和士兵比斗不但没有对手,而且那些懂事的士兵连力都不好好的出。如此遭遇让葛副团长时常惆怅的怀念起年轻时那些在特种兵营里度过的日子。

这次帅子清打来电话,汇报了楚云飞的怪异,也希望直率的葛副团长照顾好这个很有前途的士兵,没想到爱兵的葛副团长压根没把处罚战士当回事,第一个念头想到的也不是“照顾”而是“伸量”。

团长耿风是“龙扬”门传人,有练气的功夫和祖传的招式,在儿时就打下了坚实的基础,葛副团长实力虽然不错又年轻两岁但是远远不是耿团长的对手,只有在偷袭中偶尔能够得手。尤其让葛副团长郁闷的是,由于实力相差悬殊,基本每次和耿风的格斗中被压制的死死的,根本发挥不出来自己该有的实力。

于是,现在在葛副团长的眼里这个犯了错误的战士就很值得珍贵了。首先,小家伙确实有实力,制服他真的很不容易,要不是自己经验丰富,说不定输面更大些;其次,和小家伙格斗自己的实力能够全面展现,那种痛快和酣畅,过瘾;再次,小家伙没象别的战士那样缩手缩脚,不成个体统,幸亏自己先狠狠的唬他一下,要不真难说会不会有这个效果;还有就是,这家伙经常弄点新的花样,不但让人开眼还叫人期盼更大的惊喜啊。

最后小战士不知道天高地厚的约战让葛副团长不但没生气,还顿生“知音”的感觉,那个“帅哥”嘴里的“很有前途”的战士也真正的被葛副团长认为“很有前途”了。

殊不知葛副团长已经掉入小狐狸的彀中,当事人没有任何的觉悟,还在沾沾自喜。

“走吧,小……小……小楚”葛副团长还是记住了小战士的名字,“回去洗洗。”。。。。。。。。。。。。。

拿着楚云飞刚刚写好的检查,葛玉林副团长琢磨起来。

该怎么处置楚云飞呢?就事情本身而言,楚云飞并没有犯多大的过失,毕竟事情起因是村民盗窃财物。要说楚云飞当时没有阻止打斗是错误的话,那这个错误在相当程度上是可以理解的,谁不知道山民的彪悍啊?楚云飞如果真的那么脑袋不够用,上去调解的结果可以肯定是陪着战友一起躺在地上,只是他有可能不会象他的战友那么惨就是了。

那么从理智和负责的角度上讲,处罚楚云飞唯一的理由就是不该在对方丧失抵抗力以后还加以毒打,但是楚云飞的检查上也说了,本来是想忍气扶战友回去的,但是在看到是同一个城市的“远房表哥”(检查上原文如此,怀疑已串供)后才情绪失去控制的。显然,部队虽然不认可“狭隘的地域和亲情组成的小团体”那种关系,但是……也不能愚蠢到认为只有在战场上才能开始培养战友间的战斗友情吧?看到战友受那么重的伤害而热血上头不正是“战友情深”的体现么?毕竟是年轻人啊。

没错,这明显是影响“军民团结”的大事,不尊重人民的意见是不行的,虽然楚云飞已经被关过禁闭也写了检查,虽然梁东民同志遭受的毒打没人负责,但看来楚云飞同志在短期内是不能回连队那已经是铁板钉钉的事情了,想帮他的话只能让他在团部住下,做好打持久战的准备了。恩,就是这样,到会上也这么说好了。(老葛你真的没点私心么?)

事情一如葛副团长意料的发展,楚云飞在团部营房里搭起了床板被褥。他的入住其实并没有引起什么反应,毕竟是个团部,需要处理的人和事太多,他本身犯的事用“芝麻绿豆大小”来形容也不为过,所以虽然知道此事的人虽然不少,但是没人去操心。

楚云飞有点无奈的是:他虽然离开了连队来到团部,但是没有人知道他什么时候能或者会再回去,于是在团部中短期内没有属于他的位置。就象一个来部队探亲的家属一样,除了作为编外人员参加参加训练和时不时的跟葛副团长过过招外,他根本就无事可做。

于是,我们闲不住的年轻战士就在休闲的时候东游西逛,结识了不少领导和士兵。由于他喜欢琢磨新鲜东西,也有兴趣帮助别人,能力强脾气也不错,渐渐地大家也习惯了他的存在,并且口碑居然还相当不错。。。。。。。。。。

签完了汽车队士兵的报销条,耿风团长伸伸懒腰,恩,还行,今天居然没什么事了,休闲的时光难得啊,做点什么好呢?要不去山里打打猎?叫谁一起去呢?高政委,巩参谋,葛副团长?——算了,葛副团长就算了叫吧,别一去找他又被他缠住比武,那还怎么去打猎?

这葛副团长还真是个让人头疼的主,对格斗那叫个痴迷,比自己这个正宗“龙扬”门人还狂热,自从知道了自己的实力,每天上午下午的来登门讨教。自己也不是不喜欢比武,不过,这个频率太夸张了点吧?为了打击他的信心,每次必须绝对压制他的实力,要不他该俩小时来找一次了,要不是怕打击过分弄的自己也没对手玩,才不会偶尔的放他一马呢。

收拾心情,耿团长走到门外,刚要喊通讯员,却感觉到什么地方有点不对劲,会是什么地方不对呢?苦思冥想半天,葛副团长,对,就是他,老葛,老葛他……好久没来挑战了啊,怎么回事?太阳从西边出来了???

喊来了通讯员:“去看看高政委和巩参谋有要紧事没有,说我想叫他们去打猎,……对了,葛副团长这最近怎么不来了?”

“听说,葛副团长最近和一个士兵对上了,就是那个从五连来的打了人的那个。”通讯员明白耿团长指的是什么。

“哦?”耿风有点意外,葛玉林的实力他还是清楚的,看来打人的那个家伙真的身手不错啊。会上说他一个人空手打倒4个山民救了他的表哥,与会者为了保护自己的战士,并没有追究这事的真假,不过任是谁也能想到里面该是有猫腻的。莫非,这家伙真的没有用器械就打倒了4个山民?有功夫?

平常人是体会不到练武者对遭遇同行的那种喜悦的,尤其是水平差不多的同行,大家一见面一般就是先互相伸量伸量对手,倒不是非要争个胜负,而是武者的本能。

“算了”耿团长制止了通讯员离开的的脚步,“和我去找葛副团长吧。”

楚云飞刚刚和拿着图书室钥匙的老兵白为民说好一盒烟看5天书,正琢磨没事就去买烟吧,迎面撞上了葛副团长,葛副团长眉毛一扬:“来两盘?”

两个职位和年龄有相当差距的选手对练已经十来天了,葛副团长觉得身体恢复良好,有直追以前最佳状态的趋势,当然楚云飞收获也极大,毕竟是年轻么,水平上升得极快,十盘里基本上能拿下3盘了,反正现在俩“武疯子”一天不来那么十几二十多乃至三十来盘就浑身的不自在。

连输两盘以后,楚云飞“以其人之道还制其人之身。”头槌一晃,闪过首长闪电般伸来的右手,支撑腿倒至右腿,左腿佯动,窜至首长体侧,一个非标准的锁喉动作和反转擒拿将首长制服在地,为了防止首长的滑脱甚至是反制,正如教科书上所说,小战士极其敏捷地将膝盖狠狠顶住了葛副团长的腰眼。

耿风团长和通讯员进入训练室,一眼就看到了这令人发指的一幕,通讯员看到堂堂的副团长的遭遇在楞了两秒后飞快地跑了——当着首长们的面笑出来实在是有点不合适。

“继续,你们继续,别管我……”耿风乐呵呵的说——想不笑都难。

楚云飞认出了是大老板,想想平时葛副团长跟自己说的其人其事,料到了耿团长的用意,战斗继续。

可郁闷的是,尽管楚云飞使出全劲依然抵挡不住葛副团长的进攻,老葛也许是猜中了年轻人着急取胜的急躁心思,也许是想挽回刚才的尴尬,也许是有高手在旁边观战引起了他想充分显示实力的念头。总之,楚云飞连输3盘。

有领导兼高手在观战,两人都使出了浑身的解数,所以3盘过后两人已经没力气继续战斗了,葛副团长坚持站立着,对耿风说:“老耿,怎么样,呼~呼~,这家伙不错吧?”,说完慢慢坐在地上。耿风摇摇头:“奇怪啊。”

“奇怪什么?”葛玉林问。

耿风转向楚云飞:“你练过气吧?”楚云飞点点头,“是的,首长。”

“那就正常了,”耿风说:“我看你在格斗过程中把握不住呼吸节奏的时候总是下意识的通过闭气来调整,而葛副团长是通过急促的短呼吸来调整——只有经验丰富的士兵才能有他这种敏感和这种调整技能。”

在擒拿格斗中其实并不是很讲究呼吸和动作的配合的,但是擒拿格斗练得时间长了每个人就形成属于自己的习惯,什么动作吸气,什么动作呼气,什么时候呼吸转换,因为大家练功的时候多是单练,而对练的时候,双方也基本是按照套路来练的,所以这是久而久之下意识养成的节奏。而一旦节奏被打乱,练的时间短的人还感觉不到什么,象那种练了十来八年的士兵就会下意识的受到影响,轻的动作节奏配合会受到轻微影响,动作力度也会受到影响,重的甚至会因为换气不畅或者动作变形给对手以可乘之机。

而在实际的格斗中,一切都不可预测,也许这个动作没完就该变招另一个动作,呼吸节奏被打乱是再正常不过的事了,但是就是这个“正常”是绝对会影响招式威力的,大家可以想象一下让你在正常情况下全力一拳和呼吸紊乱时全力一拳哪个效果可能更大就明白了。

再打个比方,就是刚才楚云飞那很冒犯首长的一膝盖,在训练中,一般是要吸气提膝的,因为那样比较符合大多数人的习惯。一个习惯了这样训练的人在格斗中忽然在要吐气的时候有了这么个制服对手的机会,那是肯定会影响动作的敏捷和延续而使对手增加反制机会的,或者说动作强度不够而不能够彻底制服对手。

虽然在力量强弱分明的情况下这个问题绝对不是问题,但是在旗鼓相当的对战中它是绝对会带来变数的。

能注意到这个问题那绝对就是很高明的了,起码也得是练拳多年加身经百战,所以葛副团长居然能在打斗中着意调整某些比较离谱的频率就足以说明他实力的雄厚了。

而楚云飞能随时闭气则证明他具备了格斗过程中对全局把握的前瞻性——这点对如此年轻的人来说很不容易,并且对于明显经常格斗的人来说,养成闭气而不是换气的习惯者有可能是练过气的人,最重要的是,楚云飞的闭气是下意识的——这就证明这个年轻人不但可能练过气,资质好,而且发展潜力是绝对惊人的。

证明了自己的判断,耿风肯定的告诫葛副团长:小家伙用不了几天就会超过你,尤其是在现在你拼命陪练的情况下。

\section{第八章 有本秘籍}

首长之间的交流完毕后,耿风很感兴趣的看着楚云飞:“你练的是谁家的气?”

楚云飞很尴尬,“这个……我是从一本手写书上学来的,没有名字,那书是梁东民——就是我那个表哥送我的。”这家伙还真会利用形势。

“哦,书上没有招数,是吧?”

楚云飞脸有点红:“首长连这也看出来了,是,除了那几手擒拿格斗,我用的那些招式都是自己琢磨出来的,最近也从葛副团长那里学了点。”

有现在这个机会,楚云飞其实还想跟耿团长说说看能不能从他那里学点“龙扬”门的功夫,但转念一想,好多书上都写着多数门派的功夫是不外传的,虽然不知道这传说是真假,但是贸贸然向首长提这么个要求总是不太合适的。

现在的楚云飞明显还不是耿风的对手,所以耿风也没有了和楚云飞交手的兴趣,随便说了两句,带着一肚子的不自在就准备空手而归了。

眼看好容易有这么个机会就要眼睁睁在自己手里溜走,楚云飞可太不甘心了,“耿团长,我对那本书还有些不太了解的地方,你是高手,能不能给帮忙看看,要是方便的话就再指点指点我。”

楚云飞的算盘打得很快,练气者谁会对这东西不动心呢?先孝敬首长试试,反正这连名字都没有的东西是无主之物,现在归自己那就由自己做主了。至于团长那里要是不能教自己点“龙扬”门的东西,那这书他也不能白看吧?他可也算是觊觎了别家的东西了呢,指点指点自己一直不明白的东西不是也不错?唯一冒险的地方就是:万一团长对此书没太大兴趣的话那自己就冒失了。

其实耿风心里也在为难呢,看着这么一块好材料,自己却恪于祖训不能对他指教,遗憾是难免的。更让人难受就是这家伙居然有本关于练气的书,虽然心痒难耐很想看看,但是这个小战士自己连名字都不记得了,贸贸然张口总不合适自己堂堂团长身份,别人会怎么想啊?

惊讶的听到楚云飞如此一说,耿风差点笑出声来,好小子,懂事。不过,首长的矜持还是要有的,“方便么?”

楚云飞全明白了,抓紧机会弄好处啊:“报告首长,我表哥说这书是捡的,也没封皮和名字,我一直在拿着瞎练也没人教我,就当物主是我好了,应该没什么吧?”。

话里面有话呢,“报告首长,我表哥说这书是捡的,”——先把老乡绕进来再说,“也没封皮和名字,我一直在拿着瞎练也没人教我,”——起码你还不随便指点我一下?“就当物主是我好了,应该没什么吧?”——我“表哥”也算物主不是?首长你能关照就关照一下吧。

耿团长倒是没想那么多,反正事情是怎么回事刚才就知道了,只是要这小鬼肯主动拿出来那东西就是了,“哦,那我看看吧,书在你这里么?”

等等,楚云飞盘算得细得过分,还是先给梁东民抄一份吧,他前几天也想要来的,不过是因为繁体字太多实在看不懂才没当时拿走,万一……万一有个闪失对不住东民。想归想,话可没落下:“是啊,不过前几天表哥挨打以后说想看看,我就给他留下了,改天我要来给您送去好么?”

压抑,耿风有点老饕的感觉,闻得到吃不着还真难受,嘴里还得说“哦,不着急,有了你拿过来就行了。”回头边走边想:回去得安排通讯员问问这家伙叫什么名字,要不说不过去。

意外的是葛副团长对此书也有兴趣,虽然不合适抢在团长前面动手,但是“团长看完让我看看,反正是你的东西。”

三天后,楚云飞拿着秘籍去找耿团长,通讯员孟庆东已经知道这家伙是谁了,也知道团长对他有兴趣,倒是没难为他,“耿团长在接待人呢,先别进去。”楚云飞楞了一下:“那这个孟班长拿着吧,耿团长要的书,——来首长了?”“不是,是矿上的人。”楚云飞点点头走了孟庆东口里所说的“矿上”是团里的两个煤矿,关于这煤矿,情况还是真的是有点复杂。

不知道什么时候起,军队就有了办企业这种情况,前题肯定是因为军费常年的紧张这个缘故。13579团所在的地方土地贫瘠,交通不畅,商业、工业什么的也不发达,没什么太好的经营项目来办企业,但是这里有煤,部队就开了两个矿。

煤炭在那时并不是什么紧俏物资,附近小煤窑比比皆是竞争激烈,甚至有的小煤窑挖着挖着就跟别的煤窑挖通了,但部队办的矿附近还是很少有人敢于染指的,毕竟是军办矿,破坏“军队生产”是个可大可小的罪名呢。

部队办企业的优势是很多的:首先,不用上税;其次,在运输过程中军车是免检也是免费的;再次,不需要理睬那些工商、安检、环保、政府等各方面的闲杂人等,只要部队内部审查过了就成了;当然还有其他的优势这里就不一一说明了。

可部队毕竟是部队,不可能象企业管理那样方方面面全部管理到位,全面参与,因为那样我们的战士也不用训练下去挖煤好了,必须要有相关地方上的人打着部队的旗号来处理大部分事务,所以其实这些企业也就是打着部队的旗号来节省相当的费用而已,说白了就和承包相似。当然能承包了部队企业的人物肯定也不是一般的人物。

必须指出的是:部队办企业是个坏得不能再坏的现象。简单说说吧:军队经商会改变军队的行为逻辑;可能造成军内关系不和谐、军民关系不和谐、军政关系不和谐;会打断军队职业化的进程,破坏军队应有的专业性、责任性和统一性;使军队干政的可能性增加。试图用军队经商的方式解决军费不足问题虽能解燃眉之急,但会腐蚀国家柱石。

随着种种不合理现象逐渐表现出来,国内高层关于“中止军办企业”的方案已经开始研讨并实施在即。现在耿风正在接待的客人就是那些信息灵通的“不一般的人物”,这些人期盼能在军办企业终止前尽可能的获得最大的利益,很多人甚至谋划着利用这次全国性的“军办转民办”的机会为自己捞取更大的利益。

对于象耿风这种中级干部,这些人原则上是可以忽略他的存在的,不过这里毕竟是耿风的地盘,煤矿也是这个团名义下的产业,该走的过场还是要走的,做人嘛。

对于应付这种主儿耿风也是比较矛盾的,明知道有些东西似乎不太合适,但自己还不能明确拒绝,还得多方试探和打听那些不合适的东西是上级的意思还是这些家伙狐假虎威自己加上的;话又说回来,如果能处理好和这些家伙的关系,对自己也是绝对没有坏处的。

好容易打发走了客人。耿风松一口气,拿起那本书,信手翻了起来,看着看着就看进去了。为什么这么容易就能看进去?原因其实很简单,整本书就是个初级练气入门,是个练家子就能懂个八九不离十,可是话说回来,这本书也不是那么简单的。因为现在真正的练气者虽然不多,但门派可是不少,为了能在林林总总的门派中占有一席之地,各门派总是力求钻研自己门派的那点东西,精益求精,反而象这种练气的基础不是那么很受重视,久而久之,这种基础的练气方式反而在各个门派中没保留下多少,这也基本算是中国的特色了吧。

对于耿风来说,这本书肯定会让他受益的,但更重要的是让他开阔了眼界,原来,对于练气还有那么多的基础是被自己忽略了的,哈,确实是开眼了。总之一句话,基本上满足了耿风对这本书的期待,本来么,世界上哪里来的那么多惊天动地的武功秘籍啊?

楚云飞怀揣着刚从白班长那里弄来的《曼思坦因其人与闪电战》,正兴冲冲的往小花园走去,打算找片草地边晒太阳边看书,春光明媚,刚锻炼完身体躺在那里看看书绝对是惬意的事情。

可天不随人愿,迎面作训参谋柴旭东走了过来,“哈,小楚啊,这么巧,来,帮我把电话线接一下。”

柴旭东参谋才28、9岁,军衔是上尉,因为是普通的作训参谋,手下没有直属的兵,所以有事的时候只能喊别的士兵来帮忙。可这里是团部,参谋满天飞,就是老兵们嘴里所说“参谋不带长,放屁都不响”;再说团部里的兵刺头也多,新兵还好点,老兵们可都是油子,比如说几个老兵在聊天打牌,只要有人来喊帮忙,保准在一分钟内兵毛也找不到了,更有甚者会直接顶撞领导“忙着呢。”所以众参谋们需要帮小忙的时候,总是得看士兵的脸色。

不过楚云飞是比较好说话的,军营实在枯燥无聊而他又格外悠闲,和参谋们来往来往多少也能学点什么或者说弄两本书来看。这不,柴参谋一有事情就想到小楚了。柴参谋宿舍的电话线年久失修,接头众多,昨天又不通了,今天索性弄了两盘新线,叫楚云飞帮忙彻底消除隐患。

楚云飞的眉毛不引人注目的微微皱了一下,曼思坦因可是众多参谋的偶像,今天好容易弄到了这本仰慕已久的书,春光又是如此的明媚,自己还刚参加完对抗又洗了洗头和脸,而且来的人居然……又是老柴。难怪善良如楚云飞者也有点稍稍的不满。

不过也没太当成回事,楚云飞还是爽快的帮柴参谋放线拉线,爬高上低的,可能柴参谋自己也觉得过意不去,楚云飞又不抽烟,就没话找话的和楚云飞聊天,“小楚啊,安置下来没有?”

“没有呢,没地方想要我,要了我的话,万一我很快回去那里就缺员了,再要人就不知道啥时候了,反正一直在精简,人总是不够,可就没人要我。”

“哦,我听说矿上要有变动,可能需要抽点人去那里,不行你和老葛说说去那里吧,”柴参谋神秘的笑笑,一副心照不宣的样子,“那里可是个好地方哦”

楚云飞有一点点感动,好人总不会白做的,去企业肯定是个好差事,有外快可拿的,虽然自己未必介意但柴参谋无疑是真想告诉自己点事情的,“呵呵,我就是和葛副团长练练手,他可未必愿意帮我呢。”心道:去企业可和自己参军的目标越离越远了,我还想多在部队呆几年呢,再说,能帮的话到时候老葛肯定会问自己,不方便帮又何必去跟他张那个嘴。

又是一身的土,接完电话线楚云飞不得不再次去洗澡,正洗着呢,孟庆东来了,“小楚,团长让你现在去找他。”

\section{第九章 团长的徒弟}

耿风看着面前的小战士,有种说不出的欢喜,最近让小孟帮着打听了打听,这小家伙还不错,人聪明,为人也好。虽然祖训是不能违背的,但从一个武者的身份上讲,提点一个懂事的后辈他本门的功夫,那是一个前辈应该有的风范。

“小楚啊,你这本书不错,我看了看,都是些入门的东西,简单易懂,对你应该起到了很大的作用,我也知道你自己是摸索着练的,还好是这样的东西,要不你不出问题才怪,你倒是敢练。”

“可能你听说过,我们练内家功夫的一般来说是不能收外门弟子,也不能把本门的东西随便教人,这是规矩,我也没办法。不过你这本书虽然说是入门功夫,其实里面也不少东西和术语是普通人理解不了的,要是你想知道我倒是能跟你说说。”

楚云飞听得有点莫名其妙:“团长,既然是入门功夫,那还能说不错?”

耿风笑了笑:“比较正宗,自然不错了,你现在气感是什么样子的?”

遇到耿风楚云飞也算倒霉了,动不动就得脸红:“这个……我也说不清楚,就是……就是打坐的时候隐约能感受到有气劲在十二主脉里流动,在格斗中能冷静点的话,也能感受到它们的流动,不过是快了点。奇经八脉里没什么感觉,脉和脉都通不了,和书上写的不太一样,我也不敢使劲通它们,怕走火入魔或者书上没写对。您上次说我自己下意识闭气,我故意试了试,发现闭气的时候似乎皮肤能直接和外面的气交换,有点凉嗖嗖的感觉,一进身体就化开了。……还有就是开始每次练完放屁,现在是每次练完身体往外放气,也是凉嗖嗖的。”

哦,这就是没人指点的结果了,耿风点点头表示知道了,内气尚未通就已经能吸收外气了,有点本末倒置啊,等到外气凝实,再通内脉那可就不是一般的麻烦了,——等等,不对,也有门派……似乎“松涛”一派就是这么练的啊,不过,他们也会是用这样练气方式入门么?啧,会是怎么回事?

想想半天想不通,耿风就在那里呆住了。楚云飞看见首长半天没吭气,实在忍不住了,“团长?”

想来想去耿风觉得还是直说的好,“这个,我们‘龙扬’门和你的练气功夫类似,但是效果却是不一样的,别的门派里倒是有和你这情况类似的。可他们一般是从外气开始练的,先外后内,我们一直认为他们是先练六阳后练六阴的,——各家有各家的练法,这也就是门派为什么这么多的缘故。可是现在看来他们也可能是阴阳同练,先内后外的,可为什么会有这样的差别呢?”

楚云飞也楞了楞,然后提出自己的看法:“会不会是每个门派的创始人禀赋不一样的原因?”

“哦?”耿风相当意外:“你怎么会这么想?”

一种明悟在楚云飞心中油然而生,品味着这种异样的感觉,楚云飞缓缓的说道:“因为……因为每个人长得都差不多,可是每个人长得都不一样,肯定每个人的经脉多少和别人也不相同。就比如说‘大极’,大家都知道‘大极’其实只是一门功夫,总纲是一样的,可现在也分了陈、杨、吴、武、孙五大支派”

顺应着那种感悟,楚云飞在心中缓缓的放飞自己的思维“我想每个门派的理论是相对固定的,人和人总是有差别的,就决定了一种练气方式可能适用于张三但是未必适用于李四。对一个练气的人而言,也许最重要是怎么去选择适用于自己的修炼方式而不是选择门派的名气。”

看了看耿风,团长点点头示意他接着说。

“今天能有这么多的门派出现,我认为并不全是出师的门生想自立门户造成的。也许是每当有杰出人物出现的时候,他就会发现师门的理论未必全适合他,也许有些不适合他的内容他认为是没用或者基本没用的,而在某些方面因为自身条件比别人好或者机缘巧合他又会认为师门在这一方面发掘的不是很够。经过他发掘并完善后,再摈弃一些内容,那一个门派诞生就是早晚的事了。——这就是老话说的‘去芜存精’,但可以肯定的是,没准最合适修炼这个门派功法的只是他的创始人。”

厥词放完,感受着那种心灵的飞翔,好留恋……,猛然间,楚云飞才发现面对的是一脸木讷的耿风,自己是不是有点太放肆了?

耿风沉吟半天,若有所思的点点头:“你说的有你的道理,照这样说来一些门派发展到今天形成了有些对立的理论也可能是因为长久以来人为改动的原因,这些谈不拢的门派没准也是亲戚,只不过血缘关系比较远而已,是吧,哈哈。”

反应过来看着楚云飞,耿风才想起这是个如此年轻的战士,自己刚才的感觉有点象在和某个有自己观点的高手讨论呢。不过再回头想想这个家伙刚才讲述的练气感受,什么“放屁出气”的不专业用词,这两种感觉合在一起还真让人感觉有点滑稽和怪诞。

点点头,耿风决定了,“小楚,如果你的想象成立的话,那我就没办法怎么教你练了,参考上我的经验的话,没准教了你反而是害了你。看来只能把你不懂的术语跟你解说解说,……要不这样,再教你点招式好了,我也是跟别人学来的。”

享受着那种莫名感觉的余韵,楚云飞身心皆醉,对于学不到“龙扬”门功夫的那一点点芥蒂早飞到了九霄云外,何况还能从团长大人那里弄点拳脚上的功夫。

然后,全团部的官兵们都知道了团长收徒弟的消息,有心人甚至知道那人叫楚云飞是地方上打了架又经常和葛副团长对练的主儿。而当事的两个人并没有澄清谣言的自觉性,楚云飞那是不用说,做团长的名义上徒弟只会给自己增加方便而谣言反正又不是从自己这里传出去的;而团长大人不做声明的态度就有点耐人寻味了,尤其是这件事传出以后他不得不推掉几个抱着各种心态闻风来拜师的家伙。。。。。。。。。。

“楚云飞,快出来。”有人在院子外面喊。

楚云飞正在屋里看着妈妈来的信发呆,一年多来,心情好了一些,那些如噩梦般的往事似乎也在自己心灵深处的越行越远。但是每当看到母亲的来信,自己才能清楚的意识到那些东西从未离自己远去,仅仅是一时的蛰伏……对,蛰伏,不需要想起,但永远都不可能忘记……,可是,我现在能做些什么呢?

整理了一下情绪,楚云飞走出屋外,看到白为民在向自己招手。

在一开始白为民和楚云飞相处的并不是很好,这个老兵对楚云飞的态度一直是平淡中带点冷漠。慢慢的楚云飞才能感觉出并不是白为民对自己有什么意见,而是整个团部里士兵的关系基本上都是这样的。也许身居高处的人总是冷漠一些吧,在这里楚云飞确实感受不到在连队里习以为常的战友间那种真挚火热的情感。

其实原因很简单,楚云飞也明白,在这里的士兵大多数不是有自己的特长,就是有自己的关系,每个人有每个人的背景,都不是普通士兵,而且基本都是聪明人。正是因为都不普通,所以每个人都不愿意为自己带来什么麻烦,每个人也不愿意给别人找自己麻烦的机会,毕竟平淡的生活是人人都向往的。所以冷淡只是对自己的保护,对自己能力的珍惜,对自己身后关系的一种交代,或者还对他人还有种小小的看不起吧。战士们在平淡的忙碌中相互表面尊重、相互冷漠、相互提防、相互的心里看不起。

比如说在连队里,战士们可以为针头线脑大的事大打出手,还可能再招呼上老乡朋友相约到隐蔽之处来玩场大的。但是事情过去了就过去了,不影响过几天两个对头继续坐在一起吹牛打牌,除了个别民愤极大的基本上人人如此,年轻嘛,没有什么不可以的。

可是在团部很难想象发生这样的事情,如果真的不得不发生了争斗,那么这两个对头以后和睦相处的机会只有百分之五十。

在这种环境里能交成真正的朋友那就很不容易了,比如楚云飞和白为民。楚云飞爱看书,而且无所事事的日子实在苦恼,那以图书室管理员白为民作为自己刻意交往的目标就很正常了。而白为民又不想多事,所以在楚云飞提出进贡点物品后就比较痛快的答应了,毕竟是在他的权力范围内的事。

时间一长,两个都比较爱书的人就有了一些共同的语言,随着共同的语言越来越多,白为民不出意外的也被楚云飞的聪明和钻研精神所打动,真心实意的佩服起他来,两人的关系迅速的升温。

“哦,白班长,什么事啊?”楚云飞照军中的惯例称呼那些老兵为“班长”。

“这样的,现在有批弹药要报废,大家都到靶场去了,你不是总嫌打靶的机会少么?一起去吧。”

“哈”,楚云飞开心起来,“是啊,总是捡别人剩下的子弹打太不过瘾了,谢谢你通知我啊,不是又想从我这里勒索点什么吧?哈哈”楚云飞开着玩笑,打着哈哈,却意外的感受到白为民有点轻微的不自在。

靶场里人头攒动,人不少啊,团部里原来有这么多人啊?楚云飞知道现在这时候还轮不到自己这个“编外”上场,就老老实实站在远处看着。要销毁的弹药看来数量不少,型号也多,机枪步枪手榴弹,实弹空包弹曳光弹信号弹,什么都有。

看着人一轮一轮的上去,随着体力的不支,又一轮一轮的下去,楚云飞意外的发现葛副团长居然一直在场上,玩的还是班用机枪,牛人就是牛人啊。

时间一点点推移,太阳要落山了,看样子今天是轮不到楚云飞了,他摇摇头,遗憾的“啧”了一声,正要离开,却看见刚下场的白为民找过来了,“怎么不上去玩玩?”

“算了”,楚云飞苦笑,“你看轮得上我么?”

“那你和首长说说不就成了?”白为民道。

这不象是白为民啊,楚云飞有点狐疑地看着对方,白为民平时很少说及自己和俩团长的事情——这也是团部风气使然,“哦,不用了,改天打也一样,——白班长你有什么事想说就说吧。”

白为民有点尴尬,“咱们回去说吧。”。。。。。。

白为民来自黄土高原的一个小山村,父母是彻彻底底的“面朝黄土背朝天”的农民,家里有个妹妹,还有个因为中风半瘫在床上的奶奶,日子过得很苦。

因为没钱上高中,白为民17岁参军,他深知自身条件,人又聪明,在部队里积极学习,刻苦训练,在3年服役期满后,没有按期退役——退役不能在地方安排工作,转了合同兵,按照规定合同兵可以干5年,合同期满必须复员。

在合同兵期间,战士也可以报考军校,但是因为种种众所周知的原因,野战部队的考试指标很少。当然野战部队也有野战部队的好处,因为野战部队的战士接受的是正规训练,基础经验比较多知道如何治军,再加上作风硬朗什么的,一般都是进入野战系,所以这样的学生在毕业后得到指挥权的几率远远大于其他学生。而在知情人看来,在臃肿的机构中,做个连长也比做个普通的团参谋要强啊。

白为民因为是初中毕业又没什么背景,就算大家都承认他是有可能考上军校的,但是需要照顾的人太多而本身考试名额又有限,所以他一直没有报考军校的机会。他已经25岁了,今年就是他在部队里的最后一年。白为民是不甘心就此回去的,以他的情况就算回到原籍也得不到什么好的工作,而且他的家乡本身还是那么的落后。

白为民的计划是由合同兵转为士官,五年期满的合同兵转为士官的话还可以在部队里呆七年,前前后后就是总共当兵十五年,十五年期满可以得到一大笔遣散费,而且同样有得到工作的可能。期间士官还是可以考军校的,但是由于能考的早就考了而不能考的恐怕还是不能考,尤其是士官考军校就算考上了等毕业后年纪也大了没什么发展前途。

虽然合同兵转士官也需要考试,但是那只是走走过场,基本上能获得推荐就能批准了,何况以白为民的能力那几张试卷也不在话下。——现在的问题是,和报考军校一样,推荐名额也有限制!

白为民自家知道自家事,找不到人帮忙的话想得到推荐名额无异于白日做梦,可自己又实在没有可以倚靠的对象,就连处得还将就的那几个士兵也不会帮这个忙的——人情是用一次少一次的,人家凭什么白帮自己啊?

后来,从连队上来了个楚云飞,自己是眼看着他一天天的和葛副团长越走越近,本来这时候就该跟楚云飞套套近乎的,不过城市兵——都是滑头啊,这年头农村兵都学会当面一套背后一套了呢。再后来,他居然又成了耿团长的徒弟,不简单啊。再往后,才发现这个小家伙真的很聪明,又爱看书,倒是不知不觉关系近了很多,本来想关系再近点再试探试探看他肯不肯帮忙,可是时间,时间不等人了啊。

听完了白为民这番话,楚云飞先是有种被利用的感觉,不过转念一想,每个人都有追求上进的权利,谁也想让自己生活好点呀,人之常情嘛,再说以白为民的处事风格,他也不会深谋远虑或者心机深沉到这样的地步。可话又说回来,自己和两个团长的交往纯粹属于武术上的沟通,自己也没张嘴要求过什么东西,头一次张嘴就是这种别人的事明显手伸得长了点。说严重些就是不知自爱啊。

帮是肯定要帮的,但是怎么个帮法呢,楚云飞琢磨起来,发觉忽略了个情况,张嘴问:“这个推荐是六月吧?现在才三月啊。”

“六月倒是六月,可是我昨天听见王展翅的老乡说,他们团里大家都开始活动了,未雨绸缪啊。”白为民有点酸溜溜的感叹。

“哦”,这样啊,那还有段时间么。想起时间,楚云飞才发现快吹熄灯号了:“那我想想怎么帮你,你回去吧。”白为民忐忑不安的走了,双方却都没注意到小战士对老兵说话的语气有点不够恭敬。

\section{第十章 写作高手}

第二天,小雨,也许是下雨的原因,耿风难得的没什么事情,一个人在团部的院里静静的慢慢的走着,享受着丝丝春雨落在肩头的惬意,心情出奇的平静,通讯员也识趣的远远躲开。

耿风并不是个喜欢诗情画意的人,但是他喜欢下雨,更喜欢在这种细雨中品味那种细细的“沙沙”声中的静谥。

耿风“龙扬”门的功夫似乎听起来是比较威猛,有着一种动感极强的味道,但实际上并不是那样的。“龙扬”门还是比较注意练气里面的“练心”的,就是说讲究性情的冶炼,讲究中庸之道,门中开言大义中就有关于“养心”的说法。意思是如果不注意性情的冶炼,所练之气就会失于偏颇,性情急燥者气会偏阳,而性格懦弱者气会偏阴,阴阳失衡的结果就是练气者会难于进取获得更大发展,如果过分偏执强行修炼的话,所修内气会反过来影响修炼者的性格,这样发展下去的结果可就可想而知了。

所以耿风喜欢在雨里散步的意思就很明显了,这样的清新,这样的沉寂,这样平和,这样的空灵,自然会有助于他的修炼。

一边品味着这绵绵的春雨,一边体会着自己修炼状况。真的有了点进展,没有压抑自己的念头,任思绪信马由缰。看来是又有突破了,这可得感谢那个小家伙的话让自己开了窍,看来修炼中确实不必刻意去追求“按图索骥”,不影响大意的情况下又何必苛求自己去尽善尽美,练心如是,练气亦如是啊。——这个小家伙确实有意思,想想那天比试后小家伙不服气的样子,呵呵~楚云飞和耿风交手以后才明白什么是真正的高手,什么才叫高手的风范。团长大人往那里一站,楚云飞就感到一种压抑的感觉迎面向自己扑来,自己往常的那些灵动跳脱不翼而飞,而一向在格斗中运转异常迅速的大脑也慢得象钟上的时针,本来确定没有问题的四肢甚至躯干也有迟滞的感觉,而抬眼看看对方,仅仅是个小小的弓箭步微倾的身体和略扬的双手却给人一种“天动我不动,地摇我不摇”的感觉。

——不是对手!!!这就是楚云飞全部的感受。

然而年轻的牛犊怎么也不会被老虎吓住的,估计这就是内气外放吧,一种状态而已——必要的打气是应该的。

于是牛犊采用惯常使用的侧向出步——虽然有些示弱但在葛副团长身上无数次测试的结果证明那是一种理智,右手闪电般探出!虚招被识破!略收再出防招老!左向滑步!空间被阻!落地小跳步转身踢腿!踢空漏破绽!虚晃右肩右冲!屁股好疼!前翻!双腿不绞反踹,右翻滚!没人!鲤鱼打挺!封架!又有拳头!侧头!右肩好疼!左手探喉!右臂挡格!左手无法摔脱继续!脸着地!

这就是牛犊和老虎的头一战。

耿风笑嘻嘻的看着楚云飞,“不错,比我想象得好得多,尤其是反应不但快,而且出招不拘一格,很有全身都能用的味道,经验也不错,不过还是浮躁了点。”

打靶归来的军官们的喧嚣声入耳,打断了耿风的思路,要不去看看小家伙最近又有什么长进吧。。;。;。;。;。;。;。;。;。;。;楚云飞还是气喘吁吁的坐在垫子上,“葛副团长,看来耿团长教的这几下也没什么用啊。”

“还没什么用?别太不知足了,来10盘你能都赢4盘啦,这才几天?”

“照这么下去什么时候才能……呃,以您为目标,我想赶超的话很不容易啊。”

“哈,已经很不错了,赢我?”葛副团长眉头一扬,“再等10年吧,小毛孩子。”

是啊,楚云飞心想,自己优势是年轻,可最大的劣势是……太年轻,有些东西真的,真的是急不来的。

葛副团长能感觉到楚云飞的失落,但没往心里去:“耿团长就教了你这么几下?不至于吧?”

“除了这些倒是还有几招,可是不是军体拳和擒拿格斗的范围,不合适练习的时候用。”楚云飞的意思是不敢冒犯首长。

“好你个小鬼”,葛副团长非常生气对方小看自己,不就是些击打档部、喉部、太阳穴、膻中穴(俗称胸口,一击可致命)等致命部位的招数么?自己见得还少了?对练的时候能轻触就算对方输了,怎么不适合?“敢小看我?来,你来,要是这些动作能难住我,我让你连打3天枪”。开玩笑,是个高手就知道保护这些部位,很危险么?

楚云飞看葛副团长生气了,也不再保留,两人拉开架势继续,几个照面过后,楚云飞伸右手锁肩,葛副团长出左手成鹰爪外崩带擒拿,右腿作势前出,身子准备前倾,正是个过肩摔的预备,却没想对方收掌握拳,右拳重重击中了自己的右眼,正愕然间对方右拳化掌,轻轻掠过颈侧,宣告了比赛的结束。

无法抑制的怒意!“犯规”——这是葛副团长头一个概念,可是战场上怎么能有“犯规”这种事情存在?那就是作弊了,可是,生死相搏间,这作弊也正常啊。还有,他在哪里“犯规”或者“作弊”了?自己怎么会有这种感觉?

“老葛是不是感觉有点不对劲的地方?”门口一个笑嘻嘻的声音。

耿风来的时候,两人正在酣战中,没有注意到不速之客的到来。看到楚云飞出神入化的最后一拳,点点头,果然是自古英雄出少年啊。

看到葛副团长眯着一只眼睛,耿风实在憋不住笑意,四次了,虽然老葛现在的实力比楚云飞高,怎么每次见到他俩对练葛副团长都是一副吃蹩十足的样子。

看着葛副团长欲言又止的样子,耿风向他解说:“其实小楚这一拳你防不住是正常的,因为它已经超出了军中格斗的范畴,属于武术了。”

“为什么这么说呢?严格意义上讲,军中的格斗和普通的武术是有差别的。军中格斗术应该说是一种制人术或者杀人术,追求的是一种一击毙命的手段,招数简单而且实用,象你俩每次能练这么久已经是很奇怪的事情了。老葛你是特种兵出身,应该知道很多场合甚至不能给对方做出反应的机会,在战场上也是,必须尽可能快的去战胜对手才能对其他战友做出有效支援,确保胜利。”

“而武术就不一样了,最开始是原始社会……咳,扯远了。武术本身是对人体能力能够发展到什么程度的一种诠释,一种追求。哦,这是本意,可能后来有点变味——也不说这个,总之吧,武术追求的虽然也是制人,但是本身并不是以效率见长的。”

“而小楚这一拳,它的本意是激怒你,当然也算消耗你,这种效果对练军体格斗的人而言是十分显著的。因为军体格斗中不可能出现这种套路,就开始抓肩讲,在受到反击时应该选择收掌、变线握拳击打你的胸口或脖颈——这样容易受反击、与你的左臂对绞,但是不可能费劲变招那么远去打鼻子或眼睛——因为没效率,都不是致命的部位。你太习惯军体格斗了,这种没致命效果的招数自然不在你计算范围,所以很容易因为这意外的袭击而被制,其实你不去管这一拳而反击,小楚就被动了。”

看着忿忿不平的副团长,耿风的笑容又浮了上来:“其实你也不用太介意,小楚本来打你的鼻子效果会更好,估计是他没那个胆子。——奇怪,怎么你俩能练成这样?不是一直练的是军体格斗么?”

“超出范围”!这次葛副团长的蹩是吃的足又足,这满腔苦水还没法倒——自找的。

还好武人大都是比较直爽的,也算是楚云飞运气,葛副团长没在意:“这副样子让我怎么见人?小兔崽子下手还真狠,领导也敢打,哈”话锋一转,“这么说来,这武术也就是个花架子。”——这话恐怕是冲着一直让他郁闷的某些人了。。;。;。;。;。;。;楚云飞找到了白为民:“白班长,这个给你,你抄一份拿给我,原稿就不要了。”

“这是……”,白为民看着手里的稿纸,纸上赫然是“浅析‘电子战’在高技术条件下局部战争中的应用”几个大字和密密麻麻的小字,好家伙,有20多页吧?

“我这几天加班写的,你看看还有什么不懂的去查书,省得别人怀疑不是你写的,抄好了交给我,争取弄得轰动点,事情就好办了。”楚云飞一点都没有怀疑自己的眼光,这种比较“时髦”的东西应该是有市场的吧?

感激归感激,白为民还是有点疑惑,“可是我从来没写过这种东西啊?会有用么?”

“战士没事谁会写这种东西?”楚云飞也是战士,自然知道,“所以我怕你递不上去,我来想办法好了。——你总不是怀疑我的写作能力吧?”

“关键是你得弄清楚里面的内容,别到时候说起来出了差错,有时间的话最好再多看看这些方面的书。当然你也可以多抄一份让自己好好熟悉熟悉。”

感动——白为民的感觉,他也是实在没办法了才想到找楚云飞帮忙的,虽然楚云飞看来并没有为他向首长说什么,但是还是通过这一叠稿纸感受到了楚云飞的真诚——其实他还是看低了楚云飞的写作能力,反正已经没有什么选择了,而且楚云飞的计划看来还说得过去。

越抄白为民越惊奇,初中毕业的他并不是个自卑的人,不过看看这文章里犀利的观点,细致的分析,翔实的论据,浑若天成的结构,不服不行啊!看来自己还是低估了楚云飞啊,转念再想想楚云飞在举手投足不经意间流露出的那种顾盼自若……摇摇头,还是快抄吧。看看原稿上众多的圈圈,还有那么多不懂的东西要去消化呢。

…………

接过白为民递来的手稿,翻了翻,楚云飞笑笑,呵呵,还注明了“作者:白为民”,不错,想得周到。估计到了楚云飞在笑什么,白为民还是显示出了他质朴的一面:脸红了红,提都没好意思提注名的事,“那个,云飞,我回去查书了,事情就拜托你了。”说完老兵逃也似的跑了。

是啊,事情就拜托我了,楚云飞把稿件放在床上摊开,弄乱点,摆出正在看的样子,坐在板凳上开始琢磨起耿团长和葛副团长功夫的差异来。

“小楚,小楚”,葛副团长的声音在门外响起,楚云飞赶紧收拾心情,开始装模做样。

葛副团长走进屋来,看到楚云飞正在拿着一叠纸看得“全神贯注”,他也知道楚云飞爱看书,“看什么呢?这么入神?”

“哦”,楚云飞“发现”首长来了,站了起来,“是白为民写的一篇稿件,他想投稿,让我帮他审审文字有什么不通的地方,不过他写得不错,看着看着就入神了。”

“是么?咱们团里还有这种高手?”葛副团长半信半疑,从床上拿起稿件,“想弄点反映军营生活的稿子都得往部队里下任务呀。”

本来是想随便看看,结果看着看着葛副团长眼睛就发直了,恩,好,不错,军人出身的首长虽然不能完全看明白里面的内容,但是也知道确实写得不错,“别是抄的吧?”

心说不是抄的才怪,嘴上却回答:“这个我也不清楚,反正他是要往杂志投稿,这些事我也懒得操心。”

“恩,估计不会是抄的,要不让你帮他审什么?”葛副团长还是明白的,“杂志投稿,——杂志投稿?不是往咱们师部投稿?怎么回事?”

葛副团长的疑惑是有理由的,本来就是,为了抓好部队里的“精神文明建设”,各级首长都是把战士的投稿当作政治任务来完成的。通常是下发到每个连队相应的稿件数量,让连队的战士写写军营生活,抒发些理想什么的,多数属于凑字数的作品。象他手里的这种又有数量又有质量的稿子根本就见也没见过。

“这个我听他说来着”,楚云飞又开始信口开河,“好象是这种学术论文性质的稿子一般是咱们团部里参谋们写的,他估计战士就算写了也送不上去,好不容易有了这种思路,还不如拿出去投稿试试运气。”

这番话弄得葛副团长是又好气又好笑。

好笑的是:楚云飞翻过来的话基本上是正确的,参谋们做什么吃的?就是负责研究技战术配合,战略战术应用,分析军事动态那些事的。一个普通士兵冒冒失失弄这么个长篇大论出来确实是有点不知道天高地厚。

可也不能这么简单的认识问题啊。这就是葛副团长生气的地方了,我们的部队是什么部队?战士们完全可以畅所欲言的嘛,这么长的稿子,哪怕是里面有什么写的不对的地方又怕什么?战士是用心去写了的,这种自学精神本来就该提倡和发扬的。

“行了”,葛副团长把稿件叠巴叠巴塞进自己的口袋,“回头告诉他,就说我给他投了,想挣稿费自己再写一遍。——走,训练室。”首长没忘记来的意图。

\section{第十一章 初到兰山}

日子就这么一天天的过去,天气越来越暖和了,楚云飞做完练习,靠在单杠上懒洋洋的想着:又快清明了,得抽空悄悄出去买点纸了——心里又是一阵揪疼。

远远的孟庆东走了过来,“楚云飞,耿团长有事找你。”

跟着孟庆东走进团长办公室,“报告”,耿团长从文件中抬起头来,态度很亲切,“哦,小楚来了?”指指桌旁的椅子,“坐下说话。”又扭头对着通讯员说:“小孟,有人找的话就说我不在。”孟庆东乖觉的出去了。

“不错,又长壮了点,”上下打量着战士,团长流露出慈爱的目光。——对于楚云飞,耿风是真的有点喜欢,不过这样的小伙子谁又能不喜欢呢?真是个好苗子,人聪明,身手好,又肯钻研,要不把他调团部里来吧……算了,想什么呢,说正经的吧。

“最近你们五连有什么消息没有?”耿风指的是打人的事情。

“没有,”楚云飞有点郁闷,倒不是非要回连队,是总觉得有个事情挂在屁股上太不自在。

“哦,那这样吧,你在团部也没什么事情,安排你去个地方怎么样?不过可能有点危险。”耿风很严肃的说。

“报告首长,”楚云飞站起来了个立正,“我会坚决完成任务。”

“坐,坐”,耿风示意楚云飞坐下,“不用那么严肃。这事呢,不完全算团里的事,还关系到点私人交情。”

“请首长放心,”楚云飞又站起来了,不站不行啊,“我会坚决完成任务并保守秘密。”说完主动坐下。

小伙子就是聪明,耿风笑笑,“其实也没什么大不了的,团里有两个煤矿,现在军委限期转制,军办转民办。其中光华矿已经上交师部处理了——其实以前也是师部的摊子,咱们团不过挂个名就是了。现在是兰山矿要咱们团里配合办好移交。”

看看楚云飞没什么反应,耿风接着说:“所有军办企业转制要在五月底以前完成,现在正在做前期工作,时间还是比较紧迫的。矿上有咱们团一个排在那儿值守,排长叫杨首义,现在调你去主要是加强矿上的力量,保护咱们的军需物资。一定——要保证划分给咱们的物资不丢失,严防有人浑水摸鱼。……具体物资是怎么划分的你可以跟杨排长咨询。”

顿了顿,耿风组织了一下语言,“这次去,你找一个叫张志华的人,地方上的。跟他说是我要你找他的,然后听他的安排就行了,你的任务就是一定要保证他的安全,这是我的……朋友,必要时,”耿风停了半天,细长的凤眼眯了起来,“……可以不听杨排长的指挥!”

这次事情可不小,楚云飞感觉到了压力,团长这样慎重绝对不会是一时的心血来潮,看来“保护物资”只是个幌子或者是次要的。不过,问题还是要问清楚的,“要是有可能不听杨排长的话,那我用什么名义去呀?起码他也是个少尉吧?”——自己军衔低又在对方辖区。

耿风早就为他想好了,笑了笑说:“这不是问题,写证明的时候就说你是团部派去负责联络的,直接对团部负责,不过该出力的时候还是出出力为好。”感觉气氛有点沉闷,耿风打趣道“这下你可是钦差大臣了,不过谢主龙恩就免啦,哈哈。”

楚云飞也开心的笑了,不过心里可没当回事——当保镖的钦差大臣?

其实这次的事情大倒是没那么大,不过确实是非常复杂的。

这次军办企业的转制时间是军委决定的,为了防止时间一长军企人心涣散导致军队物资的流失,在可以控制的范围内是采取雷厉风行的措施。同时配合的还有当地各级政府,政府配合的目标主要是相关接手军办企业的公司和部门,为他们顺利接收军办企业而在相关的政策上开绿灯,在相关的手续办理上设绿色通道。当然关于军办企业里的当地员工的安置和协调工作也是要做的。

耿风所说的“军用物资”是在各企业中军队本身带去的物资,像军车、军需等就属于此类。有争议的是生产中所用的一些物资,象铁锨、水泵、电钻什么的这种本来部队能用得着的东西。还有就是一些专用器械,象吊车、气泵、采掘机这些本来是生产上用但不合适部队用的东西。

因为转制时间短,准备不是很充分,相关物资的分配双方得边干边谈,尤其还可能涉及到资金增减的问题,不可能所有物资一下划分清楚,而需要看管好的物资正是这没定下来的物资。但是很可能出现的情况是转让合同签了,还有相关物资没划分清楚,而接手方的人员并没有全部到位,那这没定下来的物资被盗抢的话军队处理起来就有点棘手了。

楚云飞的任务重点显然不在这些问题上,不过他甚至连上面说的这些情况都不了解,那就看年轻战士怎么表现吧。。;。;。;。;。;。;当楚云飞坐车来到兰山矿的时候,头一个感觉就是“真脏”,是啊,真脏,山路上厚厚的一层煤灰,空气中飘荡着煤粉,整个矿区笼罩在沉沉的黑雾中让人有阴天的感觉,甚至连路边的树叶都是——黑色的。

皱皱眉头,楚云飞下车走向站在门口的士兵,行个军礼,把证明递了过去。士兵瞄了一眼——团部的,回个军礼,“杨排长在路东那排红房子第二个房间里。”

进入黑忽忽的房间,一个壮实的少尉正在打盹,钦差大臣的“报告”声吵醒了少尉。

看完证明,杨首义黑油油的脸上露出了一丝困惑,估计是以前没接触到过类似的接待任务吧,不过他似乎很快就反应过来了,而眼睛里又换上了一种若有所思的神情:“哦,楚云飞,你就是那个团长的徒弟吧?”

没想到这事都传到这里了,楚云飞暗想,不过在这个时候可不能否认,“那是团长一时高兴,他可没教我几招。”这也算是实话吧。

杨排长站了起来,很高兴的伸出了手,“那欢迎你来到兰山矿,走吧,先把你住的地方安顿了,晚上给你接风,行李带了没有?”

杨排长带着楚云飞穿过煤粉飞扬的小路,绕过七八间小屋,还不停的跟路上遭遇到的人们打打招呼。楚云飞盯着少尉的背影,不但壮实而且个头很高,以楚云飞的个头才打到他的眼睛,看来有1米86、87的模样,估计手上功夫也可以吧?楚云飞暗自揣测着。杨排长在个小院门口停了下来,手一指院子,“喏,这里面就是咱们矿招待客人的地方,里面设施还是相当不错的。现在泉阳市政府和接手企业的工作人员都在里面住着呢。走,进去给你安排个房间。”

房间是个标准间,里面设施还真的不错,两张床,还有桌子,几张椅子,还有个很老旧的小电视,居然还有单独的卫生间,看样子不用去院子里挤公用水管和厕所。比楚云飞想象中的房间不知道好了多少倍。杨排长介绍完了说:“小楚你先熟悉一下这里的环境,过会儿我叫他们喊你来吃饭。”说完就走了。

楚云飞用了五分钟把行李搬进屋子,各归其位。看看没什么事可做,打开电视,才发现居然能收20多个台,看来这里也是安了卫星接收器的(俗称“锅”)。站起来关掉电视,走出房间。

站在院子里仔细打量,才发现这个院子其实不算小,只是那种狭长的布局而已,院子中间有长得很粗的一排柳树,树下还有一丛丛的冬青,树两侧还有停车的地方,停着两辆白色的小面包车和一辆6缸的“华夏”轿车,底盘比较高的那种。整个院子静悄悄的,没有杨排长说的那些人,看来都在办公区办理相关手续吧。

楚云飞也喜欢车,不但开得不错修车也拿手,不过在部队没机会接触什么好车。眼前这辆Ⅱ型的6缸“华夏”他就仅仅限于听说,只知道这车相当贵,大概有七八十万的样子,是内海市二汽生产的,曾经风靡一时,个头大,油耗高,马力强劲,内部设施也好,由于底盘偏高,舒适性稍微差点,近两年兴起“高速公路”的热潮,全国路况都在好转,这种款式车已经不多见了。

楚云飞正绕着车转来转去,背后有个带点南方口音的声音传来:“怎么样,这车挺酷吧?”楚云飞转头一看,一个中年人不知道什么时候站在了自己背后,这人个子不高,微胖,头有点秃,脸上挂着一种职业性的微笑。楚云飞心里有点纳闷,怎么这人用“酷”来形容车?点点头,“是啊,现在出的‘华夏’虽然舒适性强多了,但是玩车的话,还是这种Ⅱ型的过瘾。”

中年人有点意外,刚才自己看见一个士兵在自己车前转来转去,以为又是个看稀罕的,就出来随便和这个士兵聊两句,没想到这个士兵居然对车挺了解。“哦?你对车挺了解啊,是汽车兵么?我怎么没见过你呀?”楚云飞笑了笑,“我不是汽车兵,只是喜欢,我是刚调来的,你确实没见过我。”

两人就这么东一句西一句开始聊了起来,聊着聊着楚云飞就弄清楚了这人是接手企业的负责人,中年人也搞清楚了对方是团部派来这里负责联络的,看不出来啊,年纪轻轻,只是个小兵就能负责这个工作,别是有什么后台吧?

楚云飞因为知道的实在太少,所以总想打听打听这里目前的状况,包括各种台前幕后的消息;可中年人更介意的是这个士兵究竟有什么背景,来这里还有什么别的动机没有,毕竟是非常时期。所以两人渐渐的就说不到一块了,这互相猜忌的心一起,再沟通就不是什么容易的事了,可双方又不想就这么放弃,只好在有一搭,没一搭的闲聊中期待对方能在不经意间泄露点什么有用的信息出来。

\section{第十二章 身份微妙}

时间过得飞快,聊着聊着就见有人三三两两的从院子外面进来。过不多久,一个矮胖的中士从院子外面进来,向楚云飞走了过来,“你好,是楚云飞吧?我是3班班长许西林,杨排长让我来喊你吃饭,跟我来吧。”楚云飞向中年人打个招呼就掉头而去,同时留意到中年人笑意常存的脸上泛起了一股沉思。

“坐、坐、坐”杨排长站起身来,热情的招呼着楚云飞,“马上菜就来了,来来,我给你们介绍一下,这个是小楚,楚云飞,团里派来配合我们工作的;这个是侯副排长,侯喜才,这个是一班武班长,武涛,这三班长不用我介绍了吧?”

瘦瘦的武班长是个挺爽快的人,他先给大家散一圈烟,看到楚云飞摆手,一边把烟收回来,一边问:“小楚,喝点什么?白酒吧?平常我们这里可没人喝啤酒啊,男人嘛。”

楚云飞笑了笑,没说什么,看这架势就有种回到连队的感觉,大家在一起聚餐的时候也是这样使劲折腾,灌了这个灌那个,最后再把自己灌倒,没来由的一种亲切感升起。喝酒虽然不是他的强项,但是他有两个特点:一就是怎么都喝不倒,这话虽说有点夸张,但是可能由于他身体好的缘故,每次不管喝多少都不会影响头脑的清醒,也许身子已经不怎么听指挥了,但他总是能咬牙挺着回宿舍,至于多久能醒过酒来那就看喝了多少了;另一个就是自从他练气以后就只要愿意的话喝多少都不会吐了,一旦肚子里翻江倒海,只要默默的放松,身体内的气慢慢的运转一周,就能压制住那种恶心的感觉。除非是感觉喝太多了怕第二天难受,必须吐吐他才放任自己不去控制。

酒过三圈,男人间的距离自然的拉近,桌子上开始热闹起来。

“小楚啊,”说话的是武班长,“既然来了就是弟兄们有缘,反正这儿工作也停了,转制的事情跟咱们又没相干,没事多歇歇,看看电视,打打扑克,实在没意思咱们进山里面打兔子去。”

“喂,你这话说得不对,”侯副排长插话了,“最近事也不少,还是把该看的东西看好,别弄点纰漏就麻烦了,小楚还要跟团部汇报工作呢。”

武班长根本不理副排长的碴儿,“别听他的,反正都是咱管不了的,咱管得了的也没人敢动——部队的东西谁动动试试?用不着咱操心。”

看着俩人就要掐起来了,杨排长出来打圆场:“行了行了,你们两个,让自家兄弟看笑话——这样,小楚,明天让小许陪你到四处走走,了解了解情况,”顿了一下,“现在不说这个——不许谈工作,喝酒……”

楚云飞看看桌子上摆的烟,全是10来块钱的——属于那种能看10天书的烟,心想看来这里的战士就是过得滋润啊,不过这话不能说,就随便扯点别的吧,“怎么就咱们这几个喝酒,别的战士呢?二班长也不见啊。”

“牛班长现在带岗呢,”武班长接话,“那个人脾气怪,没事就喜欢吹笛子,要不就是锻炼和睡觉,不爱跟人来往,不合群。战士们打上饭回去吃了,要不就是去门口小卖部吃饭,今天给你接风,不叫他们了。”

“听说你功夫不错,因为打人才调到团部的?说说怎么回事……”侯副排长两只眼睛喝的通红。。;。;。;。;。;。;五个人喝了六瓶白酒,楚云飞晃晃悠悠的回去了。电视除了新闻以外他是不喜欢看的,在家的时候他就不爱看电视,因为他总觉得用电视消遣和了解东西都太慢,一本80万字的书他8个小时就能看完,可是要改拍成电视剧那怎么也得二十来个小时吧?而且还不能一鼓作气看完,实在是没意思。所以他没开电视,躺在床上琢磨着今天收集到的信息,不知不觉就睡着了。

第二天一大早,许西林就来找楚云飞,带着他吃了早饭,出去转了一圈介绍了一些情况,楚云飞又主动了解了一些,才基本上弄清楚了现在的物资存放及保管情况。可是楚云飞现在着急的还有怎么才能找到“张志华”这个人,但自己接触的这几个人里没有任何一个愿意提到现在兰山矿的具体情况。难道是真的事不关己没人操心?

想想耿团长总不会害自己,眼前的这个许班长感觉也是个比较谨慎话不多的人,楚云飞打算直接问问,“许班长,这里有个叫张志华的人么?”

许班长的脸上升起了一种很惊讶的表情:“你还不认识他?昨天和你说话的那个秃顶就是呀。不过……”许班长犹豫了半天,还是说了出来,“他们这些事情跟咱们不相关,最好离他们远点,尤其是张志华。”

看来其中就是有点问题,要不许西林不会直接警告自己,不过这些也在楚云飞的意料之中。耿团长的话已经能说明很多问题了,自己想保护好张志华的话说不准还会和杨首义他们起些纠纷,就算没有纠纷但指望不了他们的帮助那是基本上可以肯定的事了。转头回来想想,自己这“耿团长徒弟”的身份起码还是能让自己这一方面的人稍微有所顾虑的,看来这个徒弟也不是白当的啊。

打发走了许西林,回到小院里,“华夏”车还在那里,楚云飞又绕着车转悠开了,反正自己不知道张志华住哪个屋子,还不如等他来找自己。

张志华正在屋里闲得无聊呢,本来这里转制的事跟他是毫不相关的,但是集团里负责的人家中出了事情,他的哥哥——也就是宏达集团的董事长最近也遇到了一些麻烦一时抽不出人手,于是委派他来这里负责办理转制。他本来对这事情就不是很感兴趣,身份又在那里放着,不可能去和那些具体办事的人去处理那些琐碎事情,于是总是呆在院子附近转悠。

看到又是那个战士在自己车那里,张志华感觉有点不对劲,昨天你看车还可以说是对车感兴趣,今天又在那里转悠是什么意思?

走上前去,张志华脸上还是那种职业性的微笑,话可是就不那么客气了,“怎么,昨天没有看够?是不是想开开啊?”

楚云飞看到正主来了,根本没管他说什么,直接说自己的话,“你就是张志华先生吧?”张志华一楞,什么意思?看到他的表情,楚云飞肯定就是他了,干脆利落的说:“我是13579团耿风团长派来的,负责保护你的安全。我叫楚云飞,有什么要我做的么?”

张志华楞了一下,马上热情起来,“哦,耿团长派来的,走,屋里谈,屋里谈……”

在屋里坐下,楚云飞直接向张志华表示,耿团长派我来是保护你的,让我听你的安排,可是我根本就不熟悉这里的情况,也不知道你可能遭遇到来自什么方面的危险,你能不能把整个事情跟我讲述得清楚点,如果能够用其他手段解决了可能有的危险那不是更好?起码你也得让我知道该提防什么人吧?

张志华苦笑,“怎么说呢?这危险倒是未必,开始不过是我的老哥关心我,万事小心点总是没错的——所以专门托人和耿团长打个招呼,不过现在看来事情确实是有点麻烦。”

原来兰山矿在以前是由两股势力把持的,一股是集团军后勤某首长战友的儿子,他具体负责的就是生产,包括物资的采购,人员工资的发放,基础设施的投资等;另一股是军分区某副政委的侄子,他主要负责的是煤炭的销售和回款。两方由于各有各的背景、势力范围和来钱途径,多年以来,也是相安无事的。

可煤矿现在军办要转民办了,两方各使各的绝招,都想把兰山矿拿到自己的手里,没想到半路上杀出个宏达集团。

敢揽瓷器活,自然有金刚钻,宏达虽然是港资背景,但是由于很早就在大陆投资,并在香港回归时出了大力,所以在北京拥有雄厚的势力和人气。宏达一出手,在这次全国性的整改中就席卷了相当数量待转制的军办企业,由于怕树大招风也怕树敌太多,宏达吐出了不少属于鸡肋的企业,就是这样,目前还是得罪了属于北京的另两股势力,所以宏达集团在全力应付北京两股势力发难的时候,蓝山矿就只能拜托地方势力来支撑了。

兰山矿现有的两股地方势力是绝对不甘心就这么把多年经营的摊子拱手让人的,但实在是没实力和宏达集团斗的。既然不能正面对抗,只好在办理转制的时候多设置点障碍。让宏达不能那么顺利的接收,如果能达到让宏达不战而退的目的那是最好的,实在不行也要尽量搜刮兰山矿现有的物资和设备,在部队和宏达集团之间弄出大量不属于双方任何一方的物资来中饱私囊。

在矿上值守的杨首义排长期以来不但领取部队上的津贴,还有矿上给他们的补助,而且有时候拉煤的车来了为了出门的时候多拉那么一吨两吨甚至是偷过整辆车(某势力默许下的)也会对他们做些孝敬,所以日子还是过得满惬意的,自然对矿上的相关人等是抱有好感的。再加上天长日久,人和人之间总要有那么些点感情培养出来的。所以在这场宏达集团和地方势力的角逐中,指望他们排能对宏达集团做出什么有利的举动那是不可能的。反正企业转了民办以后这个排也不可能再留在这里了。

就这样,现在本来能够顺利办理的事情总是办得磕磕绊绊的,在计划中这个矿的转制的手续是该上个星期就办的差不多了,可是矿上的工作人员总是找各种理由来搪塞和刁难。刚刚把相关物资的划分办差不多了,最近又在转制后员工待遇上掀起了风浪,非要宏达集团满足他们的种种离谱的要求。可宏达集团就算不是本地的企业,但本地企业基本上员工待遇的情况还是了解的,这些所谓的“员工”的要求也离谱得太过分了,这实在是“是可忍,孰不可忍”——虽说集团财大气粗花几个钱不在乎,但是那些种种不合理的要求背后明显有一股野心太大的暗潮在涌动。

宏达集团在转制中计划买断的企业有二十多家,在手续办理过程中遇到些明的暗的势力背后阻挠是再正常不过的了——还有明明白白和他们抗衡的呢,象北京那两股势力其中一股和宏达结怨就是在地方上开始的。可是,由于宏达集团的背景相当雄厚,做事也比较灵活,付出一定代价后基本上就能摆平或者正在摆平那些小股势力了。于是,在兰山矿发生的种种不合理情况越来越引起了集团里的重视——毕竟是董事长的弟弟在这里呢。

其实正是因为张志华是董事长的弟弟,他才必须要起到表率作用,这就导致他不能在很多方面做出太大的让步——整个集团的员工都在看着他如何为公司争取权益呢,所以因为他的缘故引起了地方势力的强力抵抗那就很正常。可正常归正常,必要的心张志中还是要为自己的弟弟操的,于是就有了通过渠道给耿风打招呼的事情。

按说耿风是完全不想介入这么复杂的事情当中的,但是由于一来打招呼的是自己在军校学习时候的战友——不但关系不错,而且有点能力愿意向同学显示一下也是人之常情;二来就是和宏达集团这种在地方和军队都有相当实力的势力挂上钩的话,对耿风是绝对会有所帮助的——这也是耿风同学的本意,对同学谁不想求人帮忙的时候顺便为同学谋求点利益?所以,耿风团长的“徒弟”就在情况有些失控的时候代表另一股势力出现了。

以上的前因后果张志华肯定是不能完完全全告诉楚云飞的,但是楚云飞在了解了他该了解的东西以后也能意识到:这次事情很不简单,而自己能帮上的忙并不多,也只能扮演个保镖的角色了。不过自己似乎是被耿团长狠狠的算计了一下,估计他不否认自己的“徒弟”身份缘由就在此了。团长的意思很明确:不想在这淌混水里趟多深,但是绝对要保证张志华的安全,看来自己的身份还是很微妙的。

既然身份微妙,那就制造插手的理由吧,“张总,你的车能给我开开么?”

张志华看着他楞了一会儿,“真想开你就拿去开吧,喏,这是钥匙。小心点啊。”张志华从一个真皮手包里拎出钥匙递给楚云飞。

\section{第十三章 兰山的反击}

且不说小战士在这里怎么玩车,先说说矿上现在的情况吧,当事人之一魏翔也在为矿上的局面发愁。他本来是个乡镇企业的工人,因为家里穷,好容易娶的媳妇跟别人跑了,实在走投无路的情况下,才拼命怂恿父亲把他介绍给军里的陆战辉,陆叔叔倒是也没有亏待他,要他来这个矿上负责生产,几年下来,经济情况迅速改善,跨入了“成功人士”的门槛,当然,知恩必报是中国人的传统美德,在魏翔迈向成功的同时,他的谢意也在源源不断的向陆叔叔表达着。

这次改制对魏翔来说肯定是个机会,但是把握不好也可能就此滑入深渊,在他东奔西跑释放能量的同时,陶子辉找上门来。陶子辉的二叔在省军区做副政委,他在这个矿上负责销售,平时两个人的关系实在是说不上好,但是彼此也都知道对方身后的背景。虽说陶子辉平时不怎么看得起这个外地来的工人,同时也在为把兰山矿据为己有拼命活动,但是听到宏达集团有意插手兰山矿的时候,还是抱着“合则两利,分则两害”的理念来寻找魏翔做盟友。

现在困扰魏翔的是:陆叔叔并不赞成自己和陶子辉的计划,陆叔叔的意思是,反正宏达是抗不过的,适可而止就够了,别折腾的太厉害;而陶子辉是绝对不甘心就这么放弃的,信息灵通的他甚至已经知道宏达目前日子似乎也不好过。可目前的情况是转制手续已经没办法再拖了,员工待遇这最后一招还不知道能抵挡几天,要不和陶子辉再去计划一下?

陶子辉和魏翔坐在一起,大眼瞪小眼,半天没说话了,“要不,”魏翔还是有点心虚,“就这么算了,这事做起来实在成算不大。”

“那怎么行?”陶子辉的势力比魏翔大,胆子自然也大些,“做成现在这样,和宏达示好也来不及了,还不如弄个鱼死网破,宏达也未必愿意跟咱们死磕吧?”

“那事儿也不能弄得太大,万一杨首义他们插手怎么办?”

“他们?”陶子辉轻蔑的一笑,“他们有那个胆子么?还有,你亏待过他们还是我亏待过他们?再说他们插手我也不怕,他们又不能开枪,我找了两个高手,真要动起手来够他们招架的。”。。。。。。

楚云飞来到矿上已经五天了,除了定时给团部打个电话汇报一下工作,时不时的把“华夏”车弄出来兜兜风外,就是和张志华聊天侃大山,落在别人眼里的印象自然是“新来的小楚和张志华张总的关系不错”。

午觉醒来,楚云飞又在战士的营房前练了练单双杠,实在是无聊,正打算去给团部打个电话报报平安,却听见矿上办公楼处嘈杂声响起。

楚云飞走过去一看,却是有20余人围在办公楼的出口,在大声喧哗。楚云飞仔细辨认,认出有几个是矿上的工作人员,其他的估计也是只不过自己不认识就是了。站在那里仔细听听,大都喊着“强烈要求保证职工权益”,“把侵害军队财产的蠹虫赶出去”,“宏达集团,我们不欢迎你”之类的。市政府派来的几个工作人员手忙脚乱的在维持秩序。远远的有几个战士在看热闹。

看来有人要下手了,楚云飞赶紧扭头向招待所的小院走去。

张志华也得知了消息,正躲在屋子里坐卧不安,团团乱转。张志华的司机林海峰一根烟接着一根烟抽着,劝着张志华,“张总,要不咱们先回市里避避,安全第一……”张志华没好气的挥手打断林海峰的话,“怎么可能,先别说现在出得了出不了这个矿,只要咱一走,这里还不定闹成什么样子,后天,后天就是转制最后期限了,搞不定这里,有人要看笑话的。”

林海峰也知道张总说的对,刚收到兰山矿转制期限的通知,就有人闹事,而且闹事的居然都是矿上的中层干部,这些中层干部自然是以那两家的裙带居多,“一损俱损,一荣俱荣”的利益团体。看来是有人执意要破坏游戏规则了。

楚云飞敲门进来了,“张总,看情况有点棘手,估计闹到你这里来也是时间问题了。”张志华拧着眉头回答说:“是啊,可现在走也走不得,也没什么好办法,总不能让我答应他们那些不合理的要求吧?”

楚云飞笑笑,“为什么不能答应他们的要求呢?”

张志华有点恼怒的看着楚云飞,楚云飞则象什么话也没说的样子笑眯眯的和他对视着。慢慢的,张志华有点回过味来,“对呀,他们是以群众自发的形式来闹事的,主使人肯定不会露面的,……那么,就算答应他们什么也算不了数的,只要抓紧时间把合同定了,那以后……哈哈,好主意,好主意,小楚你看问题倒是很透彻啊。”

楚云飞笑笑没吭声,心说:知道你这老滑头肯定能反应过来,可你就没想想人家要是不理睬你的答复,执意闹事又该怎么办?自己还真是命苦,不出手恐怕都是不行的了。可话还不能这么说,你自己慢慢体会吧,该来的迟早都要来的。想到这里,楚云飞来了一句,“要是万一有什么事情,张总可是要记得首先保障我的安全啊。”

张志华楞了一下,是呀,事情万一不能顺利解决的话,别人是指望不上了,恐怕还得靠这个会武功的战士来保护了。既然小楚要出手对付那么多人,弄个手脚没轻重那是正常的了,后帐难免是会有的。人家毕竟是为自己帮忙的,这点要求实在不过分。想到这里,张志华点点头:“这个没问题,你是为了我们宏达嘛,别出人命就成。”他可没想到楚云飞的话里还有别的意思。

三个人正在这里商量不说,杨首义那里可是遇到了新的情况,陶子辉中午通知他,因为转制的事最近要告一段落了,很可能有部分“不安分”的职工会对军用物资等矿上的财产动歪脑筋,要求他们排合理安排好工作重点,配合地方站好“你们排最后一班岗”。这是杨首义排责无旁贷的事,所以他们“本来”,也是“应该”的脱离开了目前的混乱场面。。。。。。。

闹事的员工们远远没有张志华等想象的那么有耐心,没折腾了半小时,人群中就传出某人的声音:“跟这些小崽子说没用,找张秃顶去!”立刻就有人随声附和,“对,对,找张秃顶去。”“他肯定还在小院,走,大家都去。”于是闹烘烘的一群人向小院涌去,政府人员也紧跟在人群周围,严密的注视着事态的发展。

一群人涌进小院,“张秃顶”正和两个年轻人站在院中聊天,看到一下冒出这么多的人,张志华“意外”的皱起了眉头,做惯领导的大声说话还是很有气势的,“怎么回事?现在不是上班时间么?这么多人来小院搞什么?”

看到张志华态度悠闲,胸有成竹的模样,喧闹的人群有点意外,脚步停了下来,声音也小了下来,那几个年轻的政府工作人员赶紧跑上前,把张志华等三人和闹事的职工分隔开来。

一个粗壮的中年人说话了,“我们为什么你不知道么?我们是为了我们的合理待遇来的,你们宏达太过分了,我们要求属于自己的权益!”随之而来的就是一片响应:“我们要求合理待遇,”“宏达太黑心了”……喧哗之声又起。

“吵什么吵?”张志华一脸的凛然正气,他用手一指刚才说话那粗壮汉子,“你说怎么回事?”

那粗壮汉子看来也是很有经验的样子,并没有为张志华的气势压制住,以气愤填膺的表情回应着,“你们宏达!你们宏达太过份了,连我们的基本工资和福利待遇都满足不了,转制以后我们喝西北风去么?”

看着喧闹之声渐渐又起,张志华立刻弹压,“怎么可能?我们宏达的员工待遇怎么样你们还不清楚?怎么会连你们的基本工资和福利待遇都满足不了?”

这话说出来是相当有分量的,这20多人中明显有几个已经出现的迟疑的表情,是啊,看看宏达前来办谈判和交接的员工,挣多少未必人人清楚,但是光看人家身上行头、抽的烟还有随身物品就能够感到那种大公司的优越感,要是能享受上类似员工待遇,没准比现在情况还要好,自己还折腾什么劲呢。

不温不火的场面,让粗壮汉子也感受到了自己的团体有分化的趋势,那自然是不允许现在的场面就这么冷下去,“话都由你说了,我们可不知道宏达对员工是什么待遇,我们要求宏达满足我们自己提出来的要求。”于是跟风之声又起,“就是,谁知道你们宏达是怎么回事,还不是你们自己说的?”“宏达必须满足我们的条件。”……

张志华虽然早有准备,但心中怒火还是“腾”的冒了起来,好,好,既然这么给脸不要,以后自己也不用为出尔反尔自责了,怒归怒,脸上可不是那么回事,一脸和蔼,“满足你们的要求?好啊,你们毕竟是军办企业的老员工,满足你们的要求,也是我们应该做的——支持人民军队嘛。”说完,微微一笑,“我们正好可以参考参考你们的待遇要求,看看宏达集团有什么可以借鉴的,可以向咱们整个宏达推广才好呢。”

话一出口,场面当时就平静下来了,是啊,人家老总都答应了,还折腾个什么劲啊?总不能直接告诉人家“我们就是不想要你宏达接手兰山矿”。再说闹事的虽然以那两家的人为主,但是并不代表这些“内部人”在矿上以前能有多好的待遇,获利大的人永远是不可能多的。

张志华很满意现在这个静悄悄的场面,看到本来躲在办公楼里的员工得到消息后赶了过来,他甚至想和员工交代两句,做个明显的姿态出来——“看,我都吩咐他们答应你们的要求了”,但他深知现在的场面气氛来之不易,一旦说两句话打破这个微妙的局面,那就又增加变数了,也许内心深处他也不想表现得太过份省得事情结束后有人说他太无耻吧。

没人领导闹事行为的弊端这就充分体现出来了,张志华这突如神来的一手,远远超越了主谋者计划的范围。主使人早就安排了各种情况的应对方式:张志华躲避该如何应对,张志华拖延该如何,张志华态度强硬该如何,张志华耍赖该如何……独独没有想到的就是张志华全部答应了该如何应对。那主事的汉子更是楞在了当地:完全答应了!该怎么办?就此打住那是不可能的,可是,继续闹事就失去了闹事的理由,自己也是个做不得数的人,说话算话的主可又不合适出头,怎么办?怎么办?

憋来憋去,那粗壮汉子终于想出了一个可以说得过的理由——其实也是张志华本来的意图:“别听他的,他到时候说话不算数怎么办?”这一句话说出,人群终于有了反应,但他接下来的话又让所有在场的政府工作人员和宏达员工大跌眼镜:“把宏达赶出去,我们不需要它!”

事情可以这么想,话也可以背后这么说,但是,在这种场合这种气氛这么说就不合适了,毕竟人家的老总(副总?)已经答应了你们的全部条件,你再这么说不是明明白白说明今天这事本来就是对人不对事的么?不过,这似乎也是可以理解的,本来么,这世上有急智的人也并不是很多。这汉子能在短短时间内找出借口业已很不易,既然见招拆招有点困难,接下来彰显目的也就在所难免了。

兰山矿的员工们并没有多大惊奇,也许因为本身就抱着这种目的,所以没觉得粗壮汉子的话有什么不合适的,纷纷响应起来,一来二去,这喧闹就演化成了骚动,就有不冷静的员工冲上前来试图揪住张志华做个了断。当然在场的政府工作人员和宏达员工不会让他们就这么如意,双方就此撕扯了起来。

\section{第十四章 初逢高手}

由于兰山矿一方占了人数的优势,保护张志华的人就感觉吃力了,渐渐的,人墙吃不住冲击了,一个瘦高的小伙子冲了过来,紧跟着豁口处又钻过来几个中年人。

楚云飞一直站在张志华身边,看情形不妙,张开双臂挡在了张志华身前,“有话好好说,你们要干什么?”那几人却是根本不理他,直逼过来,嘴里还说着“没你事,走开。”揪扯间,只听张志华“哎呦”一声,回头一看,却是那个瘦高的小伙绕到了后面,张志华正痛苦的揉着肩膀,瘦高小伙却是满头雾水,用一种看见了外星人的眼神楞楞的看着他。

楚云飞明白该动手了,根本不给对方反应的时间就大喊起来:“你们怎么打人?”

一边说着话一边顺手叼住个小个中年人的胳膊直接把人扔了出去,跟着一个闪身晃到那个瘦高小伙面前,干净利落的卸开了他的肩关节。

瘦高小伙痛苦的弯着腰大叫,凄厉的喊声震慑了全场,所有人都是一楞,过后就是兰山矿职工滔天的怒火,“太过分了,打他们!”于是撕扯成了殴斗,连政府派驻职员也不可避免的卷入了混战。

既然遭到众多拳头的袭击,楚云飞也不再有所保留,招招都直指要害,指东打西,纵横群众,很快张志华面前就倒下了七八个人在那里哼哼,斗殴也进入了相持阶段。

看看没什么危险了,楚云飞就想按计划躺在地上装晕,可没人上来给他“致命一击”,正踯躅呢,又一个人冲了过来,动手吧,——女人?一个中年胖女人。不能打女人呀,还没等楚云飞想好该怎么处置,那个胖女人就伸出了保养得白皙细腻的粗壮十指,被染的红得发亮的指甲恶狠狠的抓向张志华的面部,楚云飞下意识的一个锁肩反扭,等到反应过来时忙不叠的顺势一送,松手,那女人就狼狈不堪的坐到了地上,又带起了一层淡淡的黑色灰尘。

淡淡的黑灰还没散去,黑尘中凭空冒出了两个身影,一个矮胖,一个瘦小,都四十多岁五十岁的模样,那瘦小者声音冰冷,一开口好象空气都被冻得铮铮作响:“我们本不愿多事,可一个武人居然对妇孺下手,小兄弟你是不是做得有点过分?”

楚云飞开始没反应过来对方在说什么,这两人出现的方式太令人震惊了,以楚云飞的眼神,也就是看到两人从院子的角落划过两条细线,带着风声和两溜残影掠了过来,却没有带起什么灰尘,这速度去奥运会参加短跑多好?高手,绝对是高手!

这种“浮光掠影”身法楚云飞听耿风说过,能练气的人才可以修炼的,不过会的人似乎不多,而且“浮光掠影”的速度跟使用这内外气修为程度有关,看着两人的速度,恐怕这一身功夫跟“团长师傅”有得一比了,天下练气的人这么多么?

那瘦小者名叫苏明亮,矮胖者名叫苏明辉,弟兄俩,苏明亮是哥哥,滇华省人,修炼的是家传的功夫。两人多年前在对越自卫反击战中杀了7个越南特工,当时陶子辉二叔陶政委正在被袭击的野战医院养伤,也因为两人的出手逃过一劫,后来就和陶政委结下了交情,时不时的互相走动探访一下。这次苏明辉正好过来找陶政委办事,却被陶子辉意外发现,知道属于难得一见的高人,刻意巴结,然后就要苏明辉帮忙来兰山矿坐镇,毕竟这种高人是可遇不可求的。

苏家兄弟本来就和陶政委交好,这次来求陶政委办事人家又很给面子,所以就答应了这个晚辈的要求,苏明辉还怕不牢靠,把自己的哥哥也拽过来了。

不过坐镇归坐镇,苏明亮还是提出了自己的要求:不能对平常人出手——反正大侄子你在这里人手足够了,如果杨首义排的战士介入此事,那倒是可以出手制止他们,关键是保证陶子辉别在这次风波中受到什么伤害。

所以闹事一开始,苏家兄弟只是站在远处观望,并没有插手的意思,后来张志华假装被打,弟兄俩也是看得清清楚楚,不过既然陶子辉发难在先,对方用点自卫的手段也很正常,并不是什么说不过去的事。再后来发现张总的“忘年交”——小战士楚云飞居然很能打,看来事情要黄,苏明辉就有点着急了,就跟哥哥说:“这个小战士也有功夫啊,咱们该出手了吧?这么看下去不是个事啊。”苏明亮却是有点为难,“咱们用什么借口出手呢?”正犹豫间,楚云飞给了他们可以出手的理由。

楚云飞反应过来对方说的话后,知道对方是高手,很恭敬的说:“两位前辈,我一下收不住手了,看到她要抓张总的脸,本能的做出来的反应,不是有意的。”苏明辉哪里肯就此放手,也是用那种冷冰冰的语气说:“本能?本能的杀人就不判死刑了?也别说那么多了,今天我代你师门教育教育你吧。”说罢,一拳带着风声击出。

楚云飞感觉又回到了和团长交手时的光景,对方逼人的气势让他处处感到凝滞,勉力招架了几招,被苏明亮一拳重重的击中头部,打着转飞出去了两米多,“扑通”一声重重的摔到了地上,同样的带起了一团轻轻的黑雾。

兰山矿职工们士气大震,刚才这个削瘦小战士的强悍已经让很多人暗自心惊,甚至有人已经打算逃跑了呢。看到这个家伙被个矮胖的“自己人”击倒,就有人大喊“别放过张秃顶,打他!”

楚云飞被这一拳击得脑袋发晕,无数颗星星在脑中环绕,难得的是他的头脑居然还很冷静,正琢磨着这下该“晕过去”了吧?却又听见有人不加掩饰的叫嚣着要打张志华,群众的情绪要被带起来这还了得?咬咬牙,一个鲤鱼打挺从地上弹起,“谁敢!想打张总的先过了我这关再说吧。”一时却没发现自己是在背对大家。

虽然楚云飞的站相有点不知所谓,但还是给了兰山矿的员工们很大压力,这家伙还真的彪悍啊,被打成这样还嘴硬,真不清楚这么削瘦的身体怎么能有这么强的抗打击能力。

苏明辉可是有点吃惊了,自己的功夫自己当然了解,刚才在旁观的时候已经琢磨出了这个战士的大致水平——练气时间看来不是很长,也没什么固定的门派招式,偶尔能看到一半招还算正经的架势,却是那种大杂烩,多的看来还是他自己琢磨出来的野路子,整个就是个江湖上所说的没门户的“浪人”。所以苏明辉才敢肆无忌惮的“替师门教育教育”楚云飞,也才敢不顾身份的下些重手。

苏明辉这一拳里隐含了他祖传功夫里的“震”字诀,他既不想打坏楚云飞,又不想让楚云飞再多生事端,所以只想用这个气诀把楚云飞震晕过去就算了,因为交手中感受到了小战士所练之气的水平,这有十足把握的一拳应当让这个战士爬不起来了,却没想到这个战士虽然看起来一时有点迷糊了,却还能生龙活虎的跳起来。

想归想,苏明辉可没放任楚云飞继续影响现场形势,嘴上还挺冠冕堂皇:“想不到你这家伙还是如此的冥顽不灵,还要恃强凌弱呀?只好再给你点教训了。”话音落下,不给楚云飞回话的机会,又是一腿弹出,矮矮的个头飞腿直接袭向楚云飞头部——给谁也看得出来此人是偏帮定了。

楚云飞这次是有点心理准备了,但是有准备是一回事,能不能却是另一回事,最多是多招架了几招而已,终于又被苏明辉的肩头重重的撞在胸部,整个人飞出去3米多远,又倒在了地上。还是老样子——又一团黑雾。

这次苏明辉用的是“弹”字诀,先卸后弹,而且籍着肩膀送了一股内气,楚云飞登时感到全身酸软无力,整个四肢和躯干就象要散开一样痛苦,五脏六腑也是翻江倒海说不出的难受,想要继续跳起却是不能够了,爬都爬不起来了!

苏明辉看到楚云飞倒在地上果真起不来了,可就悠闲啦,背着双手站在场外,不管哥哥站在那里微微摇头,嘴里还辩解着:“你们继续,我不管你们打架,我只是替武林教训个败类,别多心。”却是由于所修炼功法的原因,声音依旧是那么冰冷。

他说的倒是实话,可是在场的人怎么可能分辨出话里的真假?何况他的语气又是那么冰冷。想到这个胖子可能插手这场打架,和兰山矿员工们对抗的这些人心里都是凉冰冰的:反正这个胖子是来打偏架的,自己打输还好,打赢了这个厉害得离谱的家伙估计还是要伸手,自己可是没有那个小战士耐打呢。心里这么想,手上自然不自觉的就缓了下来;反观兰山矿那帮家伙下手可就更狠了。本来五五分的局面居然又倒向了兰山矿一方。

楚云飞躺在地上不能动,可眼珠还是能转的,场上形势看在眼里是干着急没办法,看着3、4个人慢慢围上了张总,本来还不是敢直接出手(谁也没打昏头),可张总的司机一着急,居然马上把张总的脑袋护在胸口死死的抱着,司机的忠心是可嘉的,但是这样一示弱,围着的人顺理成章的就拳脚相加了。

看着局面渐渐的恶化,直到张志华开始被人殴打,想想自己的责任,楚云飞急怒交加,不停的大口喘气,使劲挣扎着要起来,却没发现自己身体的内外气在急促的交换着,侵入自己体的内气也在内外气交换中慢慢的被淡化、同化着。

越看越气,越看越急,越看换气越快,越看越无法忍受,虽然只是短短的那么几分钟,在楚云飞眼中就象过了好几年一样的漫长,在他感觉再也无法忍受的时候,脑中“轰”的传来一声巨响,他居然又站起来了!

内气欢快的在体中循环着,流动着,奇经八脉、十二正经统统连贯串了起来,以一种不同往常流动方式的次序急速循环着。可楚云飞压根就没想在自己身上发生了什么事,一站起来就向那围殴张志华的人冲去。

那几个围殴者还没弄明白是怎么一回事,就觉得一种阴森森、冷冰冰的感觉铺天盖地般的压了过来,“刷”的一下从头到脚身上鸡皮疙瘩全起来了,扭头往来源一看,那个小战士居然红着眼睛爬了起来恶狠狠的冲了过来。

这下就该这几位品尝那种楚云飞经历过的慑人气势的味道了,这些少经战阵的人竟然就那么呆住了,有个机灵点的腿快,反应了过来,往后连退好几步,剩下的就没那么好命了,在楚云飞冲过来的时候活生生被他体外的强大气场撞了开来,怪的却是场中央的张志华和他的司机却没受到什么影响。

苏明辉可真是有点头大了,这小家伙究竟是怎么回事啊?这么抗打,自己给他的那一下够他在床上躺3天的,毕竟是六成功力的一撞呢,可他怎么能这么快又爬起来?不过恶人做都已经做了,这次苏明辉话都懒得说了,冲过去对着楚云飞又是一拳击出,顺手还带住个将要跌倒的兰山矿员工。

这拳头一出手,苏明辉马上就觉出来不对了,跟换了个人似的,对方的气势太强大了,这种感觉自己从未面对过,依稀就是记得年少时跟父亲过招的时候有过类似感受,这家伙、这家伙怎么能眨眼间变成这样?

看到拳头击来,楚云飞却是还没完全清醒过来,不过最近频繁的跟“武疯子”葛副团长过招,擒拿手下意识的全力迎了上去,可没注意那速度不知比往常快了多少倍。

一招被制!苏明辉既震惊又惭愧,实在是无地自容,这都哪一出跟哪一出啊?刚才在自己眼中的兔子突然就变成了雄狮,擒拿格斗自己见的多了,刚才和这个家伙交手自己也领教了几招,怎么会变成这个样子?怎么可能?怎么可能???

苏明辉自怨自艾的念头刚刚闪起,还有点迷糊的楚云飞却是根本没手软,擒住的对方右臂“咯啦”一声被他扭转,肩关节肘关节同时脱臼。

这下就不止是惊怒交加了,更添加了份痛苦在里面,苏明辉“啊~”的大喊一声,声音中却没有那份惯常的冰冷,不过还算不错,他到底是修炼之人,关节柔韧性很好的,平时也练过自行接卸关节的功夫,但是楚云飞这饱含内气的霸道一扭,却也是他无法忍受的,换个人早就晕过去了。

这“啊”的一声把楚云飞又叫清醒了几分,才发现自己居然用了全力扭断了对方手臂,没有丝毫的留手,还是这种真刀真枪的比试过瘾啊。

空中一个身影淡淡掠过,却是苏明亮出手了,看到弟弟不断的以大欺小,恃强凌弱,比楚云飞对女人出手还恶了几分,心下颇有些不以为然。以他的意思,略微惩罚一下,警告对方别插手这事就是了,谅来对方也不能反对,却没想弟弟居然把事情弄到了这一步,万一……万一对方身后真有什么人物支持,岂不是有损武林公义?坏了自家名声?凭空惹出许多事来?

虽然心里是这么想的,但是看到弟弟被对方扭断手臂痛苦的大叫,那点理智可就全跑到天外去了,也是没有出声警告就毫无保留的全力击出。

楚云飞虽然身体内外气怪异无比,但是基本的拳脚功夫还不是苏明亮的对手,更何况还有几分不清醒。苏明亮一晃而来,全力一脚踢出,等他反应过来想躲就来不及了,“嗵”的一脚重重击在他的右胸,由于两人都有强大的外气护身,这下相撞声音可是大得惊人,场中顿时沉静下来。

大家往声音传来处一看,果不其然,又是那个小战士搞的鬼,只见那战士站在地上,身前两道两米多长的划痕,该是被人硬生生的撞出那么远时在地上划出的“刹车印”了,“车印”周遭却是一大滩的鲜红,在一色的黑灰中分外刺眼,再看看小战士,嘴角边挂着的血渍说明了地上鲜红的来源,战士白皙的面孔因为那一抹红线显得越发的白皙和孱弱,那愤怒的眼神也由于脸色不佳而显得不是那么有力。

给了战士沉重一击的人呢?大家掉头一看,一个瘦小的身子正抱着右腿在地上来回的翻滚着,那矮胖却在旁边扶着他用冰冷的语气大叫:“哥,哥,怎么了?你没事吧?”

苏明亮这下可是有苦自知了,没想到这个战士练气功夫如此高深,自己那一脚就是3、4厘米厚的钢板也要变形的,居然踢他不倒,自己却被对方强劲反弹,因为是全力一击没有留手,犯了武者的大忌,足趾、足踝、腿关节全部脱臼,强力反弹下的痛苦和自己接卸关节的痛苦程度那纯粹是不可同日而语了。

\section{第十五章 回到团部}

大家还没从惊诧中清醒过来,那削瘦的小战士又恶狠狠的说话了:“来,你们谁再来?”说话间,咽了口唾沫和血,不让那抹鲜红继续扩散。

苏家兄弟自顾不暇,哪里还有心思应付他的挑战,做哥哥觉得自家平空弄出了对头来,正在剧痛并且郁闷着;做弟弟的却是被那早已忘却的吓人气势震慑得不愿意再多事了。兄弟俩虽说并不是什么心胸宽广之人,却也明白这事情怪不得人家。

兰山矿的员工们可是不愿意再斗下去了,他们此时的心理参照前一时对手的心理即可,谁愿意和这疯狂的家伙交手?于是就有人说话了:“算了,再闹事情更大了,要不这样,张总你给大家做个保证吧,保证以后会按照我们提的待遇要求来执行,我们今天就散了。”

张志华何许人也?就算比张志中相差一些,但兰山矿职工那点心思怎么可能和他玩?再说,对方那俩好手已经被骁勇的小战士打得一佛出世二佛升天了,场面又回到了自己的掌控中,想想刚才被人狠狠的在背上来了几拳,这样吃亏的事情自己都不知道有多少年没有遇到过了,一股怒气自然生出,一贯笑嘻嘻的脸上笑容不再:“妈的,刚才打我的时候怎么没人叫我张总?还要保证?我张志华还需要对人做什么保证?”扭头对着宏达的员工说:“把这帮人弄出去,你们该看病的看病,能坚持工作的继续工作。”说罢,不再吭声,束手而立,神情肃然。眼睛一直在人群中扫视,那粗壮汉子和瘦高小伙他自然是已经记住了,他正找那几个围殴他的人呢。打定主意肉体上的报复虽说未必要进行,但是怎么可能让他们继续在这里工作下去?。。。。。。

看着人群散去,院子里又剩下了刚开始的三个人,张志华皱着眉头扭头对司机说道:“小林子,你……”却看到林海峰蹲在地上痛苦的捂着腮帮子,还在不停的抹着眼泪,想说的话全憋回肚子了,摇摇头,怒其不争的长叹一声,他什么时候能有小楚这机灵劲就好了。

看看楚云飞还在怒视已经远得看不清的人群,张志华心生感动:多好的小伙子啊,能把他弄进公司就好了,哥哥这次没找错人啊,回头得专门找耿风好好谢谢他。

想罢,张志华向楚云飞走去:“云飞,这次可是多亏你了,不用看他们了,走,咱进屋好好聊聊。”边说边把手搭在了楚云飞的肩头,没想到,小战士顺着张总的手臂就倒了下去,这次楚云飞可真的是“如愿以偿”的晕了过去。

“小林子,还蹲在那里哭什么哭?是不是男人,快帮我把小楚弄进屋去。”

“咝……咝……,张总,我不是哭,是鼻子上挨了一拳。”

把楚云飞弄进屋里,张志华让林海峰洗洗脸马上开车去找医生上矿来,自己却跑到办公楼去打电话给耿风和自己的哥哥。

打给耿风的电话很简单,这里出了点小事,谢谢耿团长派来的战士,回头必当面谢,还有就是这个战士拼命保卫了宏达公司的尊严,现在却被两个不知道来历的好手弄得人事不醒,因为战士身份特殊,这个情况有必要让耿团长知道。

打给张志中的电话就详细得多了,把情况简单的介绍了一下,还有就是还是老大有眼光,居然能弄个不错的人来保护自己,此人现下正在昏迷中,再有就是弄个新的卫星电话来,这破山沟连手机都没信号,有个事情联系起来还真不方便。

在林海峰重金的许诺下,一位资深中年大夫带着简单的医疗器械很快的来到了矿上,仔细检查了楚云飞的情况,摇摇头,“奇怪,怎么大量运动肌体没产生多少乳酸,血没吐多少,可是血压已经跌到休克值以下了?心肺功能还算正常,身体也没什么不良反应,要是胸腔、脑或者脊髓内出血就麻烦了,现在过了多长时间了?”得到确切回答后,大夫决定暂时先别移动患者,10分钟检查一次,如果情况稳定就不会有太大问题,如果情况有恶化的趋势,那就得马上下山,为了观察方便,保险起见,先给患者输点葡萄糖,激素类地塞米松什么的就别先用了。

小林子在张总的催促下,先想办法在县医院预定了房间,又拜托那位大夫跟自己相熟的医护人员先打好招呼,到时候万一情况有变能用最高的效率治疗楚云飞。

忙完了这一切,楚云飞的状况还是那样,没什么显著的变化,看来暂时是没什么危险的。林海峰就想让大夫去看看自己的那些同事去,却被张志华兜头一顿骂:“人家小楚都这样了,你还不操操心,耽误了他怎么办?不是我说你,你要有小楚一半聪明就好了,再说,”下意识的看看四周,张志华小声说,“没准还会有什么事,咱们都得靠小楚呢,哪怕是个花架子也能唬人呀,张扬什么。”

苏家兄弟和陶子辉坐在一起,谁也不吭声,两人身上的伤都不是很要紧,接好关节休息一下就没什么事了,不过短时期内要动手的话怕是不太方便了。最后苏明辉还是开口了:“子辉,叔叔们对不住你,没给你把事办好,谁也没想到会有这么个战士出现,要不今天晚上我哥俩去给他下个重手?”陶子辉沉吟半晌,苦笑着摇了摇头:“算了,命里注定没有也不能强求,你们还不知道,耿风一个连的战士正在来这里的路上,地方上也有压力过来了,上面有什么动静还不知道呢,过了,这次做得过了。其实我根本没让他们打张志华,我又不傻。这帮笨蛋!唉~”又呆了半天突然想起了什么,“现在把柄全让人家抓住了,苏大叔,苏二叔,不是我撵你们,现在这情况你们再呆在这里会有麻烦,我马上安排人送你们走,这次你们为帮侄儿我已经付出很多了,回头我去滇华看你们去,有事记得联系我啊。”

楚云飞又在做那个已经伴随自己很久的梦了,依旧是那个白胡子老道,依旧是什么“玄青门”的事情,不过这次梦得比以往多得多,他梦到了自己在丹炉前炼丹,梦到了自己出山为山下村民把脉、开药,梦到了自己站在一座山峰,抬头仰望蓝天白云,任猎猎山风吹得自己的破道袍“哗哗”作响,任那种体察天地的明悟在心中升腾。

在第二天的上午,楚云飞终于苏醒了过来,其实这次他的身体并没有什么太大的问题,只是作为一个练气时间不长的普通人,因为种种的机缘巧合,初次进入一种很微妙的一般人也不可能理解的境界,脆弱的身体不能马上适应而已。加之他自己也不能很好的领悟和使用那种能力,硬生生的受了苏明亮的全力一脚,虽说下意识的保护了自己反击了对方。但受点轻伤那就难免了。

醒来的楚云飞一眼就看到了在他床前的林海峰,原来林海峰一晚上都在标准间的另一张床上呆着,时刻准备着开车送他兼陪视。早晨起来看没什么问题,还给他弄了几罐八宝粥放在桌上。

“张总没事吧?”楚云飞完全清醒过来头一句就是这话,林海峰听了心就有点发酸,虽然他不明白楚云飞和张总确切的关系,却也知道小战士是受人之托来保护张总的,这一点在楚云飞头一天开华夏车的时候他就清楚了。忠人所托,昏迷这么久开口就是这句话,这小楚也算条汉子了。“没什么,你昏过去的时候他们就散了。”

“好饿,现在什么时候了?”楚云飞放下心来,“哦,你昏迷是昨天的事了,现在是上午十点多,喏,这里是八宝粥,先垫垫吧,大夫说你醒了先吃点流食的好,我喂你吧。”

“什么话,我自己来就好了。”

看看楚云飞开始吃东西,“你先吃着,我去告诉张总你醒了。”。。。。。。

由于耿风派了一个连的战士来维持转制的秩序,又没有人再继续捣乱,所以在当天就办理完了所有该办的手续。张志华也明白这次物资流失了不少,但是这些事情都是兰山矿暗箱操作的,没几个人知道,说起来也不会对自己的名声造成什么损失,也就不多事了,商人么,本来就是以逐利为目的的,这点一次性的小损失对他来说不算什么。不过,他可是深深记恨住了某些人,现在形势不错,没必要节外生枝,只好暗自念叨以后别叫这些人撞在宏达的手里。

楚云飞坐着张志华的华夏回到了团部,张志华在耿风面前好好的夸了夸楚云飞,并且对耿风做出了承诺:耿团长的事就是我们宏达的事,以后耿团长用得着的地方尽管开口,千万别客气。弄的一向苛于笑怒的耿风也是笑声不断,很有些得遇知己的味道,只是在张志华走的时候还是没有送出门去,原因大家自然都是闭口不提的。

楚云飞送走张志华,又被耿风叫了去,这次却是很有些师傅的味道了。他问的自然是苏家兄弟的事,楚云飞把记得的过程和后来听说的事情逐一细细的说了一遍,听得耿风两眼发直:这样的高手?没见见还真可惜了,不过发生了这样的事,就算以后再见面那也肯定不会愉快了。

详细的问了过招的情况,耿风终于判断出,这兄弟俩应该出身于道门的某个分支,按照他们说话的声音应该是主修六阴经的,六阴经分布于四肢内侧和胸腹。所修炼功法自然是偏重练内气的,这样两个前辈居然能被小楚放倒那可也算异事了。

好奇心起,耿风抓起楚云飞的手腕,送入一股内气,细细体察了一下他的经脉状况,没发现什么异常,内气倒确实是比以往充沛了很多,只是还是在那里各自运行各自的,互不干扰。

琢磨了半天,耿风楞是弄不清楚在楚云飞身上到底发生了什么事,既然弄不清楚,就只好认为是如同很多书上所说,楚云飞在危难关头爆发出了身体内的潜能而已。不过这潜能确实也厉害得有些离谱就是了。

由于楚云飞身体还没有完全恢复,回来以后葛副团长并没有着急和他过招,倒是常去找他聊天,尤其是对那天楚云飞与苏家兄弟的一战分外的感兴趣,问了三四次了还是兴致不减,搞得楚云飞头大无比。

白为民听说楚云飞在地方上受了伤,也来看了楚云飞几次,还一反往常的节俭买了些昂贵的补品。等楚云飞问起他文章的事情,才知道这篇文章获得了团里和师里的好评,要上送军部了,如果能被军里选为战士作品代表上送的话,那一个三等功就稳拿了。不过因为这,现在已经有战士视他为转士官的最大对手了,甚至已经有人也开始学他找枪手写文章了,不过效果会如何那就不知道了。

\section{第十六章 绝版走私}

过得七八天,楚云飞又开始与葛副团长对练,葛副团长惊讶的发现楚云飞在体力、力量上有了明显的进步,尤其是已经很不错的反应速度又有提高,自己已经隐隐的不是对手了。失落的副团长决定认真的练练楚云飞手抄的“秘籍”副本,自己毕竟还没有老,不是么?。。。。。。

这天,耿风又把楚云飞叫了去,随便问了问他最近的生活状况和锻炼情况,拉了拉家常,然后一本正经的说:“小楚,你是个办事认真的战士,现在,我又需要你帮个忙了。”

因为自从兰山矿事件以来,耿风处处对楚云飞都非常的照顾,楚云飞也没有犹豫,“没什么,有什么事团长你说吧。”

“说这个事前,我先给你介绍个人,”耿风抬头对通讯员吩咐:“小孟,把他叫进来,你就不用进来了。”

楚云飞心里隐隐的兴奋起来,看来又有什么重要的事情了。

一个瘦长脸的高瘦男人走了进来,三十多岁的样子,肤色黝黑,张了双很帅气的丹凤眼,楚云飞却模糊的感觉到这个男人日子过得未必是很如意,也许是他手臂间偶尔露出的白皙说明了一些问题吧。

耿风先向楚云飞介绍:“认识一下,这是我家属的弟弟,也就是小舅子,沈文彬。”楚云飞赶紧伸出双手,“沈哥你好。”

耿风又掉头向沈文彬介绍,“这是楚云飞,小楚,一个办事能力很强的战士,这次让他配合你好了,不过你记住,就这一次,下不为例。”

饶是楚云飞再聪明过人,也没想到这次耿团长要他帮忙的竟然是——走私。

原本军车走私也是经济大潮中一种附属产物,因为军车运输的货物除了军队的稽查队别人是无权检查的,又有免费油料可加,如此保险的走私途径自然是令趋者若骛的。

而部队的稽查一般只是检验军车相关手续,如出车证、士兵证(军官证)、行车证等,目的也不过是为了查处挂军队牌照的地方车辆及防止军车私用等,如果手续健全而货物上再撂俩空炮弹箱子,一般稽查队才不会管你箱子下面还有什么东西。

可军车走私也是有一定风险的,当然相当规模和相当级别的走私除外,因为一般都是由地方上私人通过种种途径雇佣军车跑单帮的居多。由于是单帮,人手自然不会多,否则稽查一关也不好过;又由于是走私,押运者一般是没有枪支或者有枪支无弹药的——就算荷枪实弹也不敢开枪;而军车走私的安全性又注定了货物的高价值,所以就有深明底细者垂涎,专门打这种走私军车的主意,只要别去抢枪就成了,不把事情弄大,失主财物损失了也不可能声张,姑且算是黑吃黑吧,当然抢劫者眼光是不敢出错的。

沈文彬是个眼光很高的人,办事能力也不差,在上大学时也是一方才子,人长得帅气,学习也好,还是学生会的骨干,交际能力很强,最后成功的俘获了校花的芳心,两人毕业后就结了婚。

本来沈文彬也分到了一个不错的单位,但随着经商大潮的涌现,小家庭的生活也受到了冲击,比上不足的生活就让他的妻子不能满足了,夫妻二人都是很杰出很优秀的,为什么不能比别人过得好点?他本来就是个心高气傲的人,再受到妻子的怂恿,就决定抽出部分时间和精力做点“小买卖”。先用夫妻俩几年的积蓄开了个饭店,却是因为经营不善转让给了别人,做妻子的一着急就参加了“安益”的传销,可结果比开饭店赔的还厉害还搭进去了本职工作,丈夫也是忙于补窟窿,这心态一不好那是做什么赔什么,最后干脆被人狠狠骗了一笔预付款,搞得债台高筑。

耿风的妻子沈娟娟其实也是很疼这个弟弟的,但是却看不惯弟媳妇,她觉得弟弟走到现在这一步,他那个一直烟视媚行的妻子脱不了干系。

沈文彬的外债大头是他们单位的隋永义,那家伙人很聪明,最早是被沈文彬妻子的传销所惑,也加入了一支传销队伍,由于嘴皮灵光,敢于骗人,短短时间就积累了相当的财富,后来念及给自己灌输这一观念的恩人,在沈文彬张嘴的时候毫不犹豫的借给他15万。

当这钱连同沈文彬的老底全部打了水漂的时候,隋永义的脑筋就动到了耿风的身上,但是由于耿风做人一直恪守操节,沈娟娟因为弟媳妇的缘故也从不为弟弟吹什么风,所以尽管沈文彬向耿风提过好几次想用军车走私的事,耿风却根本没有理会。

这次军办企业的改革,让所有人都看到了军车走私的末日不远了,过了五月份肯定这事就没法再做下去了,沈文彬在隋永义的再三催促下,硬着头皮找到姐姐哭诉一番,姐姐也知道再不帮弟弟一把以后就没机会了,终于心软,枕头风于是刮起。

没当过兵的人是领会不到军人对家属的那份歉疚的,虽然这事性质严重,但是“忆往昔,榜样数量稠。”耿风还是答应了沈文彬的要求——只不过下不为例就是了。

这次沈文彬计划走私的是外国香烟,只要从桂龙省某沿海港能够接到货,即使卖给烟草批发商也有百分之一百五的利润,为了这次行动,隋永义准备了五十万的现金,沈文彬也砸锅卖铁的弄到了差不多十万,其中就有三万是姐姐的。

楚云飞和沈文彬坐在个小酒店听着他的絮叨,因为姐夫终于松口,沈文彬高兴的喝了不少,黑乎乎的脸上居然也能看到红光的闪耀。

楚云飞心下却有丝丝的感慨:人和人真的差很多啊,看张志华从没为百万以下的款项皱过眉头,可是自命不凡如斯的沈文彬眼中之“成功人士”隋永义一次也最多只能拿出五十万,怪不得人都说“官商”“官商”的,靠了“官”的才能真的算“商”啊。

沈文彬发现楚云飞半天没附和自己了,“小楚,想什么呢?”

楚云飞自然不能告诉沈文彬自己正在琢磨什么,而是提了一个细节:“可是现在我没有驾驶证呀,会不会耽误时间呢?”

沈文彬笑笑,一种万事在手的雍容浮现脸上:“这不是问题,我姐夫说一两天就办好了,以他的身份办这种小事还不简单?”

果不其然,耿风两天内就办好了楚云飞的驾驶证,出车证明也开了出来,给沈文彬和隋永义一人弄了身军服和相关证件,因为两人年纪明显偏大,居然还是搞了一个少尉,一个中尉的头衔。

第三天一清早,楚云飞开着“拉练”的军车载着两名走私者上路了,由于手续齐全,一路上虽然有过两次稽查的检查,却也没什么事情发生,只是第一次检查的时候稽查人员提示两位尉官注意扣好风纪扣而已。

经过辛苦的跋涉,终于在四天后的中午到达了桂龙省的防风市,理论上就是该在这里接货了。三人先找个旅馆住了下来,简单的休息了一下,在下午四点多的时候,换了便装出来转转防风市。

隋永义很方便的就联系上了他某个传销下线的亲戚,此人名叫杜爱国,也是个头脑精明的人,当时发展他做传销想开拓桂龙市场时,他很坚决的拒绝了,后来隋永义亲自出马也不行。两人见面谈话后隋永义才知道沿海城市和内地城市人们眼光和信息差距有多大,任他口舌生花,此人连“安益”的东西都没兴趣做,自然也不会接受他所做的这个小牌子了,用他的话就是“你骗我,我骗你的,有什么意思?太小儿科了。”

也正是因为那次谈话,隋永义才知道小小防风市四五十万人里居然有一半以上是做走私的,或者是做跟走私配套的服务。这个暴利的行业自然引起了隋永义的注意,结果隋永义没做成对方传销的上线,对方反而成了他走私的上线,两人约好,如果隋永义有兴趣做,杜爱国为他寻找上家。

杜爱国听到隋永义来了,很高兴,放下手头的活计,热情的引导三人在城市里转悠了起来,还介绍了不少当地不错的风味小吃给客人们品尝,宾主之间交谈甚欢。楚云飞三人也深深的体会到了有个当地人做向导的便利,这里的人虽然以走私为主业,但同其他占据了地势资源优势的地区一样,还是对外地人敌视得很厉害,也许正是一种明智的自我保护措施吧。

第二天将近中午的时候,杜爱国来找三人,他的理由很简单:上午是给过惯夜生活的人休息用的,不是用来办事的。不过关于上家昨晚已经搞定了,是个叫“牛哥”的二手贩子,以他们要的这点量最多也只能接受这种接待了,和“头家”打交道可不是普通的主可以享受的待遇。搞定的意思是“牛哥”愿意屈尊为这点量和他们交易,起码数量、质量和安全性上都不会出问题,不会被人“晒了咸鱼”。

三人和“牛哥”见面是在一家茶楼的雅座,牛哥长得粗壮得很,虽然个头低了点,但一看就是那种孔武有力的人,下颌处一道刀疤延伸到颈侧,看那架势不但身经百战而且命也很大。他身后还跟着四个大汉,却都是普通渔户的打扮,没有楚云飞想象的那么招摇。

茶点摆放了上来,这里的茶楼沿袭蜀山省一脉的风格,没什么小点心和小荤食,基本上就是一些瓜子花生、干鲜果品而已。

相互介绍完毕,几句客套话说过,话题转入正事,本来“牛哥”也没有兴趣在这种小事上耽误太多时间的。

“听说几位要办点货,不知道想要些什么盖子的?量好象不是很大的哦。”开始话“牛哥”讲得还是满客气的。

这种场面自然是隋永义来应对:“是啊,小弟头一次来防风办货,初次合作,数量是少了点,不过,来日方长么,哈哈。”这话很有水平,不说自己以前做过没有,只是说没在防风做过,既诚恳又含混。

“至于这盖子,只想要点白万和黄三,软硬各半就好了,不要专供的。”为了这趟走私,隋永义也是下了辛苦的,相关市场调查了一些,相应的行话也了解了一些,以求说出来不至于让行家笑话。

\section{第十七章 防风和牛哥}

万宝路和三五都是内地的畅销烟,尤其是白万宝路和黄三五,至于专供那是烟盒上印有“向中华人民共和国出口”的标记,表示是从正式渠道进入中国的正牌香烟。最近内地香烟市场上有传闻专供的烟没有走私来的烟味道纯正,所以非专供的烟销量更好些。当然也有地方因假烟猖獗更认可专供烟,各省行情不同而已,但是对于眼前这些走私者来说只是香烟的包装问题而已。

“牛哥”很痛快的开出了价码,自然还是按件来说的——一件香烟五十条。“恩,黄三二十四,白万二十三,软硬各半。”这么算下来就是三五是四块八,万宝路四块六。

隋永义和沈文彬细细一算,不由得暗吸口气,开口的还是隋永义,“这个,牛哥,这价钱不太对吧?好歹我们也是要办几十个的货(意为几十万),这价钱太高了。”

“哦”?“牛哥”饶有兴趣的看着他,“你觉得该是什么样的价位啊?”

面对态度暧昧的走私者,隋永义毕竟是个做过生意的,虽然心脏不争气的急剧跳动了两下,还是定下心神,“总也在十大几上吧?哪里能上了二十呀?”

行情是杜爱国早就透露过的,这俩牌子基本上一盒就在三块多,当然了,价格是会随着国际市场的行情和海关对走私的打击力度以及国家相关政策和文件的影响下略微浮动一些,一般也就在三块五到三块九的模样。

“牛哥”不屑的一笑:“行情是你们说的这样,不过,内地外烟的价钱涨了呀,你们还能不知道?”

内地烟价格涨不涨关你什么事呀?就算涨了你这里略微提一提也就好了,隋永义和沈文彬对望一眼,心里都是这个念头。

“牛哥”见两人如此的不开窍,只好挑明点,“本来说内地烟涨价跟我没关系,最多提一点价格,涨价毕竟说明风险大了,我这里没风险,那就说明你们运输上有风险了。”

看两人还是无动于衷,“牛哥”一点脸面也不留了,“风险出在哪里?因为马上军车就不能走私了。隋老弟,我知道你们是开着军车来的,这小小的防风有点风吹草动还有我不知道的?你以为我们做什么吃的?”

既然话全挑明了,“牛哥”也就不客气了,他已经有点不耐烦了:“我不知道你以前做过没有,我只知道你是第一次来防风做,量不大,还有可能再也不能来做了,我给你的这个价格已经很公道了,不信你可以去找别人做,除了晒你咸鱼的,你最多也就这个价钱拿。”

这下隋永义和沈文彬可有点傻眼了,自己的底细被人一下戳穿,人家说得也没错呀,对你这种明显的一锤子买卖,人家凭什么让你享受那种老客户待遇?

杜爱国本来在一旁静悄悄的,按规矩象他这种靠中介挣点小钱的主早就该在介绍完双方领头人物后就离场的,不过因为隋永义一方需要他在场偶尔翻译一下个别防风俗语才得以列席的。现在杜爱国也跳了出来劝戒二人:“对呀,牛哥这话直了点,不过也是事实,做这行买卖很讲究的,他要是这么说,那肯定是大家都商量过认可的。”

这笔钱该出多少,能挣多少,该在什么地方打点多少,机动费用多少等等这一系列的问题是隋永义和沈文彬早就盘算好的。眼睁睁的看着要损失四分之一以上的利润谁会就这么甘心啊?那损失的可是纯粹的利润啊,二人自是不甘心就这么损失一大笔,“牛哥,一盒差了有一块啊,你就高高手,让让好么?”

走私者看来也不都是穷凶极恶之徒,也许他们的凶恶也只是展现在特定的场合吧,“牛哥”还是很和蔼的回应两人的要求,“可以呀,给我一个让让的理由,不过——我可没多少时间。”

“让让还需要什么理由,不就是牛哥一句话么?”沈文彬确实是个书呆子,这话说得实在有点不上路,拍马也要分分场合啊。

“不需要么?”牛哥觉得好笑,他已经有九成的把握眼前这俩人是“初哥”了,不过自己现在左右是没什么事,就陪他们玩玩好了,“没理由我怎么跟同行解释?”

“需要跟他们解释么?这么小的买卖。”沈文彬微笑着继续书呆子式的拍马。

“不需要么?”牛哥似乎在学周星星,逗弄着眼前二位。

“牛哥,我能说句话么?”楚云飞隐隐的感觉出牛哥在逗弄自己的同伴。

“哦?”牛哥在心里纳闷了一声,这个年轻人一直不出声,看坐姿和走路是真正的军人,该是护车的汽车兵,所以牛哥一直没注意这个无关的家伙。不过,既然你这么恭敬的请示我,不让你说两句也不上路,看你能玩出什么花哨,如果你真的陪别的老客户来这里做过,适当给你个面子也无所谓,“小兄弟有话就说吧,你挺懂礼貌啊。”

“牛哥大人有大量,冒昧之处让您见笑了。”楚云飞笑笑,“我听说过个关于防风港的传闻,这次正好来了,想请教一下牛哥这事是不是真的。”

哦,也是新手,牛哥心里有点不屑,不过,玩得有点上瘾,就看看你能说些什么吧,“哦,说说看,这里我不知道的事还真不多。”这话说得站在牛哥身后的四位都有点搞不清楚大哥到底在想做什么了。

楚云飞脸上还是那种无害的笑容,“我听说防风港这个名字是这么来的,有人说咱们沿海地区深受台风之害,这里做为个能让渔船避风的港口,大家就叫它防风;也有人说人风的灾害远大于台风,这人风就是在海上伤天害理无恶不作的海盗,为了消灭他们,这里的渔民汇合各地来做生意的商人和天下豪杰,狠狠的清剿了海盗,为表彰这个义举,有个皇帝钦命这个港口为‘防风’,不知道是不是真的?”

牛哥心里对楚云飞有点好感了,国人对自己所在的地方总有种认同感的,防风港作为个小城市,知道它名字来历的人并不多,难得这么年轻的小伙居然能知道自己家乡名字的由来,一种微微自豪感在牛哥心里升起,“呵呵,没想到小哥你也知道我们这里名字的由来,没错,后来很有段时间海盗再不敢在附近生事了,我们这里人一直是很团结的。”这里一高兴,“小兄弟”升级成“小哥”了。

“是啊,就现在都能感觉到你们的团结。”楚云飞笑得越发灿烂,“不过,要成大事,也要天下豪杰的支持呀,人在江湖走,哪能不碰头,牛哥你说呢?”

牛哥又笑了,没想到玩了半天叫这个年轻人把自己绕进去了。不过,牛哥的心情还是不错的,身后那四位跟班也觉出了年轻战士做事的巧妙,既捧了防风人又为自己争取了好感,最后还反过来争取优惠。

“哈,这个小兄弟很会说话嘛,冲着你对我们防风这么了解,可以给你们让让,让大家好知道防风人是讲情谊的,不过——天下豪杰,是任谁都可以当的么?”

楚云飞知道不能再一味的软弱下去了,正好问题来了,现在的楚云飞又正好最不怕这样的事,“那牛哥的意思是?”

牛哥看着笑眯眯的年轻人,没由来的心中有点发悸:这家伙真是有点莫测高深啊,不过,牛哥自然也不会怕什么事,看了看那两个忐忑不安却强做镇定的买家,缓缓说道:“我的意思简单呀,你们可以跟我的人比比身手,身手好自然当得豪杰,”不对,牛哥看到两个买家明显的松了口气,面对的年轻人笑容中却掺杂了丝诡异,虽然只是若有若无的感受,见惯江湖险恶的牛哥眼里可没沙子,不等对方接话,口风立变,“不过远来是客,做买卖呢求个人气,大家就比比酒量好了。”

比酒量那就不太好分出胜负了,于是晚上隋永义买单,请大家痛饮了一翻,牛哥有事没参加,他的跟班来了三个,三对三,隋永义虽然不能喝,沈文彬喝酒却不含糊,再加上楚云飞这个无底洞,搞得最后杜爱国也加入了防风人的阵营,才是个皆大欢喜的场面。三跟班都喝了不少,却也没有过分。因为最近买卖太多,面临军车的最后一班辉煌,大家都是得加班加点拼命干的。

到末了,一个高姓的跟班拉住沈文彬嘀咕了两句,高跟班本来是想和隋永义说的,但是酒桌上发现楚云飞似乎和沈文彬比较惯,而隋永义和杜爱国的关系似乎更近点。

高跟班说的话是:你们这次来找杜爱国是个错误,严格讲他连走私团伙的外围都算不上,还好他是本地人,沟通还方便点,但是由于他没有固定的阵营,找上家的时候乱找了一气,这事就难免扎眼了点,怕是有那些专吃新手的人已经惦记上你们了,我们是做大买卖的,不可能晒你们咸鱼,但这一路上却是也未必太平了,不过,出了桂龙省应该就问题不大了。总之,路上小心吧。下次,如果有下次的话,来了让杜爱国直接找我们好了,或者你们直接来找,谈好甩他俩小钱就是了。

回到旅店,三人就合计了起来,牛哥这里看来让让是没什么问题的了,让多让少也就是人家操心的事了,反正内地的烟也涨了一些,不至于吃亏太厉害。现在的问题是,路上恐怕要有些困难了,在桂龙省要防吃新手的,出了省要防劫军车的,只能尽量走大路,上高速了,实在不行别去到处找军车加油站,自己加油好了。

第二天接货,牛哥还真是给了个好价钱:都是二十,昨天喝酒时大家也都知道了,最近接货的实在太多都赶这最后一茬呢,所以牛哥给出这么一个价钱也真是卖面子给他们了。

说点废话:这本书也上传一周多快两周了,7W多字无奈才不到40的推荐,这个……不是一般的伤心,俺有什么错误或者要改进的大家倒是说啊,不说我怎么会知道?

你说什么?没广告?哦,这书象我自己的孩子一样,如果可能的话,俺是不太想换个马甲拿他插标弄首的,好象没人要似的。迂腐?没错,就是迂腐,可俺真不忍心拿着它到处晃呀,再说,自家的孩子总觉得是好的,唉~~(要不,谁帮忙去做个广告?)

不过,说到底,还是俺写的少哇,贼痛心的说,要是等20W还没啥起色,咱这当作者的只好厚着脸皮去吆喝了,还没反应就只好练葵花了。一点废话,希望没影响大家心情,真要影响了,你在书评区骂吧。

\section{第十八章 新手的运气}

财货两讫,三人直接上路,一点都没有耽搁,军车一路飞驰,从中午十二点一直到晚上九点,三人连车都没下,逃命般的冲出了桂龙省。十一点,开出离桂龙有100多公里了,三人才在个县城找了个旅店住下,第二天早上六点就又起来赶路,赶到晚上,才松了口气,找了个大点的城市住下洗洗澡吃点热乎东西。

一旦离桂龙远了,三人就又弄不到一起了,隋永义的想法很功利:按照来时的计划,外烟沿途出售,等回去的时候,也就没多少存货了,当地市场马上就可以消化掉,至于军车被抢劫的可能,那什么时候都存在的,小心点就是了;沈文彬是跟头栽得太多了,觉得还是赶路回去的好,毕竟这么多的货物,闪失一下担当不起,回去以后烟可以慢慢卖,囤积一下没准还能再赶上什么好行情;楚云飞的想法是支持后者的,早赶回去早好,毕竟干的是很陌生的一行,变数太多了。

可楚云飞的任务是配合他们,他也不想给沈文彬下不来台,所以只好不吭气;沈文彬又欠着隋永义人情,这次走私隋永义早答应好:只要能弄来军车走私,那旧帐一笔勾销,沿途产生意外后果由他自负。

所以三人想法虽不同,却是很快达成了一致意见:就按隋永义说的办。

就这么着,军车一路走一路抛售走私烟,价钱是九块五左右,利润还是在百分之一百三十大几的模样。等到了秦岭附近,这烟也卖了差不多有一半了,本钱也都收了回来,剩下的自然就都属于纯利润了。

要过秦岭了,却发生了点小事,一个批发烟草的老板愿意出九块三的价格,一定要买下他们所有的烟。按说这烟零售十二三块,对方给的这包圆价钱也不能说很低,隋永义有点纳闷:你吃下这么多的货,不怕卖不动么?毕竟香烟是不能长时间囤积的。

对方却是笑话军人们不知道行情,这秦岭是什么地方?是汽车运输的交通枢纽中心,跑南北运输的车是都要从这里过的,货运量是远非其他地方可以比的,来上三五百万的烟几天就能出完。

谈到这里,隋永义他们自然更不愿意就这样出手了,既然别人从这小老板手里拿上货都能赚钱,那为什么自己不再多赚上点呢?

他们却不知道这个老板是兼卖假烟的,想要这批正经货也是用来装幌子充门面的,他的烟草批发量远远大于所说的那点,光靠卖正经烟能挣几个钱?

到最后,双方可想而知的没有谈出什么结果出来,看着老板那悻悻然的样子,三人心里都泛起点不安的念头,不过再想一想,这是什么地方?秦岭啊,车比蚂蚁还多的情况下,谁敢乱来?

没有塞车,军车很容易的翻过了秦岭,下山路上,隋永义兴致勃勃的和沈文彬商量着:“文彬,我觉得你的想法也不错,实在不行咱们把烟都拉回去吧,这沿路卖快是快了,可是挣的太少啦。泉阳市六十多万人,咱剩下这几万盒烟消化起来还不是轻轻松松的?再说还可以往周边卖呀,——咦?稽查?”

可不是,前方路上站着两个戴红箍的士兵,手拿小红旗,马路边还站着一个,看到楚云飞他们到了,小旗摇动,旗语指示:靠边,接受检查。

这次的稽查检查得格外详细,居然连炮弹箱子都翻起来检查,三个稽查很轻易的发现了大量的外国香烟:“中尉,你能解释一下这香烟是怎么回事么?”

隋永义的嘴皮还真不是盖的:“是这样,这次拉练,因为我们部队上的厂子马上要转制移交地方了,部队领导的意思是弄点东西回去发给大家做福利,也算是对职工和家属们的一番心意。”顺手又拿起烟来,“同志们都辛苦了,来来来,一人两条,是我们的小意思。”

楚云飞在旁边暗自点头:话说得不错,士兵也是人,只要你对他们的尊重到位了,很少有人专门找你茬的;再说了,军车走私也不是什么大不了的事,起码目前不是。士兵们也很少为这种犯不着的事无故去得罪同僚。

可三个稽查的态度耐人寻味:“少来这套,走,跟我们回分区去。”

于是,楚云飞驾车,副驾驶上是沈文彬和一个稽查,隋永义在车后座那半排上斜躺着,另两个稽查抓着门把手站在脚踏板上,向山下开去。

开了一个多小时,还没下山,副驾驶上的稽查示意:“前面那个路口,右拐。”楚云飞刚拐进去还不觉得怎么,走了二里来地:不对劲,一辆车不见,越走路越窄,这么小的路会是去军分区的?

一脚刹车踩了下去,楚云飞说话了:“班长,这是去分区的路么?不太象吧?军分区我是来过的。”——后一句自然是楚云飞诈人的。

“嘿嘿”中士稽查皮笑肉不笑,“这是小路,走着近,你不认识很正常。”

冲他这说话语气,楚云飞就知道上当了,稽查是什么样的人啊?怎么会跟你这待处理的士兵嬉皮笑脸?不再多说话,猛甩方向盘,想掉头回大路,不过这条路实在是太小了点,掉起头来太费事了。

至于这三个稽查,楚云飞可也没放在心上,心道:你们要是给我说不出个子丑寅卯来,我才懒得理你们,要是动手就来吧,就算你们是真的稽查,为了保护“部队福利”,动手也是可以理解的吧?

可三个稽查跟没事人一样看着楚云飞,同样感觉不对劲的沈文彬和隋永义正在纳闷这仨家伙怎么不制止楚云飞,谜底就在眼前呈现了出来。

由军车来的路上开来了一辆没牌子白色小面包车,车一停,里面跳下了七、八条汉子,手里都拎着铁棍。这么小的车,他们在里面挤着也不舒服吧?

一个看似领头的汉子喊道:“朋友,出门在外,求个平安,啥也别说了,东西放下,你们走你们的。”说完,一干人等就直接冲了过来。

这种假设过的情况一出现,楚云飞作为应对计划中的主力立刻出手,一个肘锤击中正要动手的假稽查腹部,那家伙当时就翻江倒海的吐了起来。

军车此时正横在路当中,楚云飞一侧的那个稽查接了根铁棍,恶狠狠的向车窗户砸来。

一下、两下,玻璃被砸穿,楚云飞出手如电,抓住了铁棍,腕子一扭,铁棍就来到了他的手上,顺手一棍敲晕了正在狂吐的假稽查,丢下铁棍,一边开车门一边说:“看好你们那边。”

车门开得很有技巧,先拽住把手开了,加速向外推去,那丢失铁棍的假稽查还没反应过来,就被狂甩到车头,得,又是个动不了的。

楚云飞才跳下车,三、四根铁棍兜头就砸了过来,没办法,楚云飞背靠汽车,对方虽然人多却也派不上用场。

躲开两根铁棍,轻抓一根顺势一挡,四根铁棍同时落空,趁两根铁棍碰撞时的震动楚云飞又夺一根铁棍,狠狠砸在那个手尚在发麻的汉子肩膀,那汉子再也把持不住,铁棍落地。

楚云飞一个前滚,躲过两根铁棍的袭击,手中铁棍又架住了一击,左手顺势又捡起一根铁棍。

双棍在手,楚云飞在人群中横冲直撞,人全去围攻他了,沈文彬和隋永义倒是没什么事情做了,两三分钟后,二人下车捡漏,把被打得七昏八素者敲晕,沈文彬居然还有心思评论呢:“他怎么只拿脚踢人呢?用棍子打多直接。”废话,手是两扇门,全凭脚打人啊,再说,铁棍直接打人,打残无所谓,打死人那麻烦可是大了。

沈文彬话音没落,就听见“砰”的一声闷响,正在激斗的楚云飞觉得右臂被狠狠的撞了一下,手中铁棍“咚”的落地,冷静的他没受太大影响,趁身体侧倾,飞左脚又踢倒一个,顺势又是个前翻,回头一看,却是沈文彬和隋永义正挥舞铁棍和两抢劫者对打,那两个抢劫者的身后,是那个看似领头的汉子,那汉子手中赫然拿着一把——一把传说中铁道游击队使用的驳壳枪!

一团混战中,那汉子其实是不敢随意开枪的,这枪也是平常抢劫时的道具而已,威慑的作用远远大于使用的作用,不过楚云飞实在是神勇得有些过分,那汉子一着急,就顾不了许多了。

手中枪还在指指点点,可是领头者再发现不了什么开枪机会,实在是太乱了,正犹豫间,楚云飞左手铁棒到了他的肩膀上,接着又是一脚,他也晕了过去。

等到领头的人苏醒过来的时候,实在是哭都哭不出来了,驳壳枪远远的在地上扔着,不过枪管已经被折成了“V”的形状,自己的人横七竖八的躺了一地,衣服全被剥得精光,所有人的手脚关节全部被卸掉,远处被开到山崖边的白色面包车正在熊熊燃烧着,下巴也被卸掉,他们连喊“救命”的份都没了,多久才能有人路过这条废弃的山路那就只有天知道了。山区的春天,有句话怎么说的来着?对了,是“春寒料峭”。

不知道是算楚云飞命大,还是老天爷不忍心再折磨沈文彬了,子弹只是擦伤了楚云飞的右臂,没伤到筋骨,用军车上的急救包简单包扎了一下,三人匆匆上路。沈文彬还不停絮絮叨叨:“可惜了,好好的车烧了做什么?”

对沈文彬,楚云飞一点都不客气,自己人嘛,“不烧?不烧等他们叫人开车撵咱们啊?”

“再说,他们虽然该死,但不能死在咱们手上,那路不定多久才有人路过呢,车一烧,起码有黑烟,没准就有人去看看怎么回事。”

果不出所料,有巡山的人发现黑烟,以为是山火,前去查探,结果发现那么多人躺在地上瑟瑟发抖着。发现的人不敢多事,报警了,警察来了一看,却发现两个榜上有名的抢劫犯,其他人也是这附近的游手好闲之辈,虽然是乡亲,但这事难免有点过大——那么多人伤筋动骨不说,居然还有枪支,更别说通缉犯了。严查之下,却破获了一系列的无头抢劫案,最后某批发烟草的大老板也被牵扯进去,花了好些钱才被取保候审。至于肇事军车,那是部队上的事,地方上是无权处理的,再说,如果把军车牵扯进来,这么大的案子功劳可就落不在这些警察身上了,于是,有关该军车的审讯记录后来不知道被弄到哪里去了。

三个走私者可猜不到是这种结果,只有楚云飞隐约猜出这事到了部队上应该不会产生太大的问题,他已经有过类似的经历了。

再也不敢耽搁了,三人如丧家之犬一般,一天一夜的时间,赶到了楚云飞所在师的辖区,才敢去修理修理打碎的车窗户,正规的给楚云飞包扎一下伤口。过意不去的沈文彬执意要仔细处理楚云飞的伤口,缝针居然没在外科而是去五官科缝的,虽然时间有点晚了,但是大把钞票花出去,自然有人愿意为他处理伤口。

\section{第十九章 团长的人情}

经此一事,耿风和楚云飞的关系就更好了,在楚云飞看来,团长把这么隐秘的事情交给自己处理,自然是对自己的信任,耿风却是因为“徒弟”两次为自己的事情负伤,不好好对他实在是说不过去。

一桩买卖下来,沈文彬不仅还清了欠款,还弄了十万块钱的赢余,心情自然不错,本来他还想重新老实做人,本分工作。可他的妻子却不愿意:这点钱够干什么的?还是花点钱自己重新回去上班,老公你就在外面闯荡吧。人心一旦放开了,再收回来确实不容易,沈文彬也喜欢上了这种充满挑战的经商生活,妻子一劝,自己就答应了。

楚云飞养伤期间,很多人前来探望:有自己连队的战友;有沈文彬和隋永义;有宏达集团的人……最让楚云飞难受的是:白为民又破费来看自己了。

楚云飞这次一出去就是小二十天,白为民来过之后才知道:那篇文章已经上送到军里了,三等功虽然看来是没跑了,但是这次转士官的人很多,关系硬的大有人在,据说名单已经内定,没有白为民的份。

踌躇了半晌,楚云飞答应白为民在时机允许的情况下为他说说情,至于效果怎么样是不敢保证的,毕竟不是他自己的事,不过,要是实在不行的情况下,自己可以为他找个还将就的工作,说到这里,楚云飞似乎又看到了张志华秃顶下胖乎乎的笑脸。

在白为民的催促下,这天,楚云飞和葛副团长练完之后,去找耿风。

小孟见来的是楚云飞,通报都免了,直接就让他进去了,进屋一看,沈文彬也在,耿风正和他有一搭没一搭的聊着,见是楚云飞来了,“哈,小楚来的正好,正有事想找你呢。”

“什么事?”

“这样的,小楚你的文化课好象学得不错,我给你弄了个考军校的指标,你该好好准备准备了。”耿风根本没考虑楚云飞的想法就为他做主了。不过这也怪不得他,带兵这么多年,还没听说过谁不愿意上军校的。

楚云飞可不这么想,他来军队其实只是为了躲开那个伤心的环境,或者还想学点格斗技巧去暗杀了那个杀害自己父亲的凶手,不过这个想法连他自己也不相信会成为现实,不过人活着总得有点希望的不是?

要是想在军队出人头地,楚云飞大可以多写点那些给白为民写的东西,或者报名参加军事比武什么的,但是就算在军队混好了,那能怎么样?总不能混成个团长就带上自己的团出兵巴基斯坦或者沙特吧?要是混成军委主席还有那个可能,不过……这个想法现实么?

楚云飞笑笑,“团长,我是有个事想跟您说说。”

“你说吧。”

“是这样的,管图书室的小白您知道吧?”

“恩,我知道,他不是最近还写了篇稿子么?挺轰动的,也算是咱们团的人才呀。”

“他那篇稿子是我和他一起写的。”楚云飞这么说是有原因的,他不能说这个稿子全是出自自己的手,因为在部队里,由于战争年代的理念一脉相传,冒他人之功是大不韪的事情,虽然是他自己愿意这么做,但会严重影响团长对白为民的观感的:贪他人之功的不是好人,更不是好士兵。

但是楚云飞还必须要在这件事里掺上一脚,一方面是说明自己和白为民关系好,早就在帮助他了;再说了,文章本来就是自己的写的,人总是有个虚荣心的,何况象他这样的年轻人,说的对象又是他的团长师傅。

可耿风明显的误解了楚云飞的意思,“这还了得?你现在把他给我叫来,我的团里不许有这种兵,还欺负到你头上了,我早觉得这家伙不是写那篇稿子的料。”

“我不是那个意思,”楚云飞有点不好意思,“其实我俩挺合的来的,经常在一起看书,所以才合作写了那篇稿子,因为他想转士官,我就让他拿那篇文章发表了。”

“哦,”团长师傅现在明白了,感情自己“徒弟”是来求情来了,“你想让我帮他转士官?”

就算和团长关系再熟,楚云飞还是满不好意思的,毕竟是手伸长了点,默默的点了点头。

“这个……”耿风沉吟了一下,似乎在组织语言,“不好办,要是你的事我就给你办了,我真要办点小事谁还不给个面子?”看着楚云飞失落的表情,团长进一步的解释,“搁在往年,根本不是个事,可今年不行啊,今年军委的大动作你也清楚不是?”

楚云飞知道团长在晦涩的指出军办企业的转制以及后来带出的一系列影响,点点头。

“这么多人裁撤下来了,总要安置啊,现在师里那点可怜的名额早让人争得快打破头了。你知道我给你争取这么个考军校的指标有多不容易么?”

听到军校指标,楚云飞又冒出了了希冀的眼光:要不让白为民去考军校?

耿风这人精哪里会不知道这小家伙在想什么,手一挥,“你想也别想,给他指标我不如去送人情呢,考军校指标可比士官指标难弄多了,要不是看你是前年兵,今年考军校最合适,我才不去费那个劲呢。”

“再说了,考军校的话小白的年纪也太大了点。”

虽然很失落,但楚云飞明白团长说的确实是实话:今年是不比往年,大家都能感觉到,而考军校的话先不说能不能考上,白为民的年纪确实大了点。

“你好象有点不愿意考军校?”团长的感觉很敏锐,“怕考不上?”

“不是的,考军校我还是有把握的,可是……”楚云飞真的有点迷茫了,考上军校以后呢?在部队里呆一辈子?自己到底该怎么做呢?向军委主席的位置奋斗这个念头他可是想也没想。

“可是什么?难道你来部队只是想混上三年然后回家?那么不求上进?”耿风可是有点生气了,当兵头年考军校是不可能的,再往后又可能年纪大了毕业后在部队里不好发展,毕竟有个“四十岁一刀切”在那里摆着,再往后还有“五十一刀切”。楚云飞今年考军校是最合适的了,况且他素质好又年轻,前途可是一片光明的,为了小伙子的事团长师傅可费老劲了,如今自己的劳动成果不被人领情,耿风能不生气么?

知道团长生气了,楚云飞低头寻思半天,终于决定实话实说,“团长,其实……其实我不是为了当兵而参军的。”

不是为了当兵而参军?什么狗屁不通的话?耿风更生气了,换个小兵他早骂上了,不过师傅毕竟还是对自己的徒弟有点了解的:徒弟有心事,还是那种不太合适说的事,“哦,感情你是为了当官才参军的?”

“有什么话就说吧,这里也没外人,就是我和你沈大哥,还扭扭捏捏的。”

楚云飞苦笑一下,“我倒不是扭捏,主要这话一下不知道该怎么说起,事情是这样的………………”

楚云飞把自己参军的原因从头到尾说了一遍,“……所以我都不知道自己下一步该往哪里走,心里背这么个包袱总不是滋味,想到那个马哈苏德还好好的活在世上吃香的喝辣的我就……”

说着说着战士的眼睛就红了,沉默半晌,又接着说道:“我离那家伙太远了,根本够不着他,我不知道是不是该考军校,就算考上军校我又该做什么,我真的不知道自己该做点什么。”

就是耿风和沈文彬比楚云飞多出若干的阅历和经验,初次听到这事还是呆了半天,楚振中的事情当时轰动全国,两人是听说过的也咒骂过的,但是那是离他们比较遥远的,那件事的发生给两人震撼甚至远远赶不上同在先阳的梁东民,更别说那大事的幕后内容了。

耿风和沈文彬对视半天,彼此都能看到对方眼中的不可思议和浓浓的同情。不容易,真的不容易啊,可怜的小家伙,一直背负着厚重的心理负担在踯躅独行,甚至放弃了自己的似锦前程,全国竞赛的优胜者啊。还好沈文彬虽然是书生意气,却也还算性情中人,没去想责备楚云飞这种把事情全揽到自己头上的想法。

至于楚云飞快意恩仇的观点,耿风更是强烈的支持,虽然峥嵘岁月磨平了团长不少的棱角,虽然有时候甚至可以用狡猾来形容他的为人处事,但不代表就不是个血性男儿,何况他还是个武者,更是个军人。

设身处地的想了想,耿风承认自己也拿不出什么更好的建议给对方,不过,军校还要他考么?军校,对了!军校!!!

“咳,”团长一声咳嗽打破室内的静谥,“这个,小楚,你的遭遇我很同情,我有个建议,你可以考虑考虑。”

“还是要考军校,为什么呢?我们国家也有蓝盔部队的,你知道吧?”

做军人的自然知道蓝盔,联合国维和部队嘛,楚云飞点点头。

“咱们师是乙种师,蓝盔里是不会有咱们这种师的成员的,要不就是甲种师,要不就是特种部队,还有军校生,起码这三种你是有希望的,你明白么?至于能派到哪个国家维和那就只有天知道了,不过,也算个希望不是?再说了你还年轻,去军校对自己毕竟是个提高。”好嘛,耿风的口气,根本就是楚云飞已经考上军校了,只是上不上的问题,不过事情似乎确实是如此,全国竞赛的优胜者呢。

“对啊,”沈文彬只是书生气浓,脑瓜可是绝对聪明的,最多算不太会为人处事就是了,“小楚你要是在军校表现优异,真没准能进入蓝盔,哪怕你在学校这几年蓝盔不招人,你毕业也可以选择去甲种师或者特种部队啊,以后还会有机会……”

耿风毫不犹豫的打断了小舅子的话,“军校生分配原则上是从哪里来回哪里去的。”

沈文彬也不客气的回敬:“分配是有分配的原则,那只是个原则,怎么争取还是看小楚的努力了,我们分配那会儿要不是我争取怎么能和家慧分到一起?”

“那倒也是,”耿风点点头,“不过,想进甲种师那可是真不容易,进特种部队还容易点。”

“反正军校生想参加蓝盔也不容易,这就都要看小楚自己的了,怕就怕到时候蓝盔要招人了,结果小楚还在咱乙种师混或者已经复员了,那可就错过机会了。”沈文彬也知道姐夫这次弄个考军校指标不容易,楚云飞进来前俩人还在说呢。

两人你一言我一语,楚云飞坐在那里仔细琢磨,反复思考以后决定了:考军校。

既然已经决定考军校,楚云飞肯定是要通知叶美一声的,顺便听听母亲的意见,“团长、沈哥,我回去给我妈写信去了,告她一声。”

“还写什么信啊?”说话的是沈文彬,“你团长这里有电话,想打多长时间还不由你?是不是,姐夫?”

“对对对,我还一直没想到这事呢,以后小楚有什么事就来我这里打电话吧,方便多了,别不好意思啊。”

团长既然这么说,楚云飞也不客气,拿起听筒就给家里打了个电话,接电话的居然是陈小军。

“表哥是你啊,我姨不在家,明天这会儿吧,你现在部队里呆着怎么样?能天天打枪么?……”

回到宿舍,楚云飞斜靠在床上仔细寻思:看来考军校确实是个不错的选择,就算没有维和部队这档子事,自己也要继续成长、生活下去的,不是么?何况还有可能加入这个出国维和的行列。

那么,该报个什么专业呢?楚云飞想来想去,还是报特种作战专业吧,因为蓝盔的组成里特种作战部队是肯定少不了的,至于军校生、甲种部队什么的都是未必一定有的;再说,学了哪个专业也未必能去相应的部队,以自己现在的身体素质,去那里应该把握更大点;而且去特种部队更能磨练自己,应该能多学点什么东西的吧?

给叶美打通了电话,母亲为孩子走出了情绪低谷而欣喜,更为孩子能重新上学而高兴,所以毫不犹豫的支持了。在做母亲的眼中,自己的孩子是天下最聪明的,能重回校园,不但对孩子的将来好,也对得起那早逝的丈夫了。

\section{第二十章 业余保安}

听到不能留部队的消息,白为民还是楞了半天,虽然是意料之中的事,但对这个已经在部队呆了八年的老兵来说,还是有点不能让人接受的。

就这样离开么?离开我的部队,离开这呆了八年熟悉的军营?离开这自己生活了如此长时间的地方?离开自己抛洒了青春和汗水,留下了无数悲喜回忆的地方????

与普通士兵不一样的是,一般志愿兵总是对军营充满了感情,白为民也不例外,想到这里,白为民甚至对楚云飞产生了一点点的嫉妒和怨怼。

不过,大家都是当兵的,部队里改天换日的大动作是谁也知道的,这种情况下楚云飞帮不上忙也在情理之中,毕竟军队企业转制对部队的冲击太大了。

白为民的苦恼楚云飞怎么能不了解,看着发呆的老兵,他甚至在有点恶作剧的想:这家伙会不会对送了我那么多东西肉疼?不过他还是很看不下去一个朋友这么痛苦的,“白班长,这样回去你能找到什么工作?”

得,又是一把盐撒在了正在巨痛的伤口上,白为民看上去快哭出来了,“能找到什么好工作?能找到好工作我早退伍了,我在老家根本就不认识几个人,听天由命吧,估计也就是个半死不活的小集体或者乡镇企业。”

“那我帮你想想办法吧,看能不能去矿上。”

去矿上?那感情好啊,白为民一下来了精神,“那可太好了,矿上现在可是归了宏达集团了,是个有前景的地方啊,云飞你上次去见过他们,是吧?”

楚云飞在兰山矿受伤的真正原因没几个人知道,大家也就是知道这家伙又和人打架了,反正也不是第一次了,就算有消息灵通点的知道楚云飞站在了宏达集团这边,可那也绝对应该是个偶然,没人去琢磨这件事背后的内容。

楚云飞点点头:“是,我和他们的副总关系不错,他说我有事可以去找他,你这事他该帮得上忙的。”

想到就做,楚云飞跑到团长那里就给张志华打电话,手机却是打不通,打到宏达总部,接电话的小姐却死活不肯告诉他“张总”在哪里,只留下了楚云飞的地址和耿风的电话。

两天后的早晨,楚云飞刚摊开新买来的参考书,打算恶补文化课的时候,小孟又来找他了:“小楚,团部有你的电话,团长让你快去。”

跑到团部,却看见严肃的团长大人正拿着电话眉开眼笑,“呵呵,张总不用客气,看你说的……”

看到楚云飞进来,团长点点头示意一下,然后对扎听筒说:“他来了,你等等啊。”

楚云飞上前接过听筒,“张总么?你好。”

“你好啊,小楚,身体恢复得怎么样了?”

“谢谢张总,早没事了。”

“真是对不住你了,还好你年轻啊,对了,前两天给我打电话了?”

“是啊,我打您手机不开机,打到宏达也不在。”

“哦,有什么事?”

“是这样的,”楚云飞扭头看看团长,团长正歪着脑袋看着他,可不说也不行啊,难道把团长撵出去不成?“我有个战友今年复员,想看看您那里有什么地方缺人,兰山矿就行。”

团长师傅皱着眉头摇摇脑袋,这傻小子,人情是用一次少一次,何况是宏达这样的大企业,你以为是求蹲在马路边下象棋的老头?

“哦,是这样啊,我们集团今年还负担着50个复员名额呢,可你张嘴了,那我还能说什么?”不过,该打的官腔那是还要打的,“人品没什么问题吧?”

老滑头,无非要我领情就是了,你那宏达那么大,哪里安排不下几个人?“他人不错,肯学习,就是学历低点,主要是他家里挺苦的,我有点不忍心,怎么也是朋友一场。”

电话那头的张志华沉吟了一下,挺苦的?不忍心?自己多久没听到过这种话了?见惯了商场的勾心斗角,看厌了政坛的跌宕起伏,那都是血淋淋赤裸裸的弱肉强食,这话听起来感觉自己又回到了二十,不,三十年前。

“恩,小楚你的事我还能说什么?是现在来么?”

“不,先跟您打个招呼,他按期复员。”

“哦,那我给他留个位置,到时候你联系我就行了。”

“那谢谢张总了,您跟耿团长还有什么话说么?”

“跟他没什么话了,跟你还有事说呢。”

“请讲。”

“是这样的,我们公司最近遇到了点小麻烦,保安部门人手少了点,本来也没想麻烦你和耿团长,谁知道你正好打过电话来,那就只好抓你壮丁了,哈。”

“那您跟我们团长说了没?”抬头看看耿风,团长点点头,说:“跟我说了,你去吧。”

“说过了,你们耿团长说你马上要考军校,反正在哪里复习都一样的,你带上书来我这里吧。”

…………

放下电话,耿风看了楚云飞半天,摇了摇头,“唉,你总惦记着个白为民做什么,有这心思还不如帮帮你那个表哥呢,团部里的兵……”长叹一声。

耿风居然能用一声叹息清楚的表达出“值得不值得啊?”这个意思。

“表哥?”楚云飞下意识的反问一声,却看见团长眯着眼睛,笑嘻嘻的看着他,慈爱的眼神表达出“小子,露馅了吧?”的意思。

楚云飞明白团长的意思,白为民严格的说还算不上真正意义上的战友,那些连队里的才是,以你楚云飞的那点能力,也不可能广洒甘霖,普渡众生不是?要帮也要帮最近的人,耿风都不想多管闲事,你凭什么管那么多啊?

楚云飞也不好多做解释,不好意思的笑笑,“那我表哥就请团长费心了,对了,这次我去北京——该怎么办呢?”

“怎么办?看着办呗,不过我感觉……”沉吟一下,耿风眼里的责怪没有变少,不过该提的建议还是要提的,“反正不会有什么太好的事,去了北京还是老实点,好好学学文化课,能不管的事尽量少管。——还不都是你自己惹出来的?”

耿风心里基本上能猜出来宏达发生了些什么事,但是,以楚云飞的地位,还是少知道点的好,上层的那些乱七八糟的事,不是个小兵该操心的。事情总是要过去的,以楚云飞爱多管闲事的性格,什么都不知道更好,还是安心做个打手吧,省得到时候死都不知道是怎么死的。。。。。。。。。。。。

楚云飞一身牛仔,脚踩白色冒牌“耐克”旅游鞋,一手拖着可折叠的旅游包,一手搭凉棚扫视着眼前的“宏达大厦”,五月的北京已经有点燥热了,时不时还刮过一阵热辣辣的风,可这风里的土未免多了点吧?

宏达集团不是一般的牛气啊,看这眼前20多层的金碧辉煌,耀眼的玻璃幕墙,楼侧的“宏达大厦”四个大字随便掉下一个来都能砸到十几个人,这是在北京,天子脚下啊!!!要是有这么个公司……人生至此,夫复何求??

打量半天,楚云飞刚要往楼里走,却突然的感到一阵的不舒服,那是一种悸动,就是练武人常说的“气机浮动”吧,一种在野战训练中,被枪瞄准的感觉。

楚云飞停下了脚,四处看看,没什么异常呀,怎么回事?

正纳闷中,大楼里施施然走出两个中年高个瘦子,高不是很高,比楚云飞高点,胖瘦和楚云飞倒是差不多。

两个练过气的!两人身上外放的气明显的在告诉同类:别理我,现在我很危险。

有情况?看样子是有点不对,内气外放是很费体力的,通常武者只有在比斗时才用的,这就意味着这二人随时在准备出手。

闹事的?楚云飞正在琢磨,却见张志华一手拿着有半块砖大的手机在说着什么,慌张的从楼里面出来了,身后还跟着三个人。

看到楚云飞,张志华明显的楞了一下,他楞了一下不要紧,那俩高个瘦子的眼光刷就转了过来,两股气势直冲楚云飞而来。虽然不是很强,只是一种戒备的味道,但楚云飞明显的感觉到了两人体内蕴藏的排山倒海的威压在一触即发。

也许是快想不起来了,张志华半天才反应过来:“小……小……小楚是吧?穿上这身衣服差点认不出来了,呵呵。”向后一回头,“阿强,这个是新来的保安,你安排一下。”再回头冲着楚云飞指指那巨大的手机,抱歉的笑笑,“小楚,我现在正忙呢,不好意思啊。”

张志华身后三人有俩比较年轻,大概是30岁出头的模样,还有一个就看不清岁数了,怎么也将近五十了吧,不过三人都很精神,楚云飞敏锐的感觉到这三人都不简单,应该跟前两个一样,都有功夫在身的。

那俩年轻人里过来一个比较矮壮的,浓眉大眼,寸头,皮肤很黑,眼角有颗豆大的黑痣,上下打量楚云飞一下,“跟我来,快点。”

黑痣年轻人把楚云飞领到地下室的保安部,冲着一个精瘦的汉子喊道:“黄经理,新来的保安,你安排吧,张总还等我出去呢。”说罢掉头就冲了出去。

黄经理打量楚云飞的时候楚云飞也在观察他,四十岁出头,个子不高,但是很结实,大额头,厚嘴唇,看上去象个厚道人,可两只小眼睛显得很精明。

黄经理拿出一张纸,“喏,填了它。”

楚云飞拿过来一看——联系册,联系册?要这个东西做什么?

管他,该怎么怎么办呗,楚云飞麻利的添好了联系册,黄经理拿过来看看:“啧,现在的年轻人,怎么就不知道好好练练字呢?外地人,才来北京啊?——还不到18岁?”

黄经理又抬头看看年轻人,个头还可以,有1米73、74吧?身子骨可够单薄的,做保安?夸张了点吧?要不是阿强亲自领来的,还真有点不相信呢。不过这肯定不是公司的关系户,要是关系户就该让自己去人事部领人了,哪轮到这家伙自己跑下来找保安部——感情黄经理以为是阿强下楼顺便带个新手认路呢。

算,管他呢,反正最近一直在招保安,也不差多个混饭的,不过这身板安排安排在大厦里显然没什么威慑力,虽然长相还是很排场的。

那去哪里呢?不行去“盛世年华”工地吧,那里最近不太平,建国这小子一直嚷嚷缺人呢。。。。。。。

北京郊区,“盛世年华”住宅小区工地,几个穿保安制服的年轻人蹲在地上抽着烟,有一搭没一搭的聊着天。

一个嘴上留小胡子的保安指着远处的一个美妙身姿,“哇靠,美女耶~~”

可美女离众保安过于遥远,于是有一个眉毛很淡的保安耻笑小胡子,“扯淡,那是大妈,二胡你什么眼神?”那小胡子姓胡,再加上他的小胡子,于是人称“二胡”。

最旁边坐在地上的突眼睛壮实保安开口了,听口气还是有点文化的,“我干,老子2.0的眼睛都看不清楚,你俩吹鸡巴的牛逼,都给我华丽的~”两根手指成“V”字向外一甩“爬开。”

其他保安哄起而闹之,在众人的注视下,“美女”越走越近。

等到大家都能看清,确认“美女”的确是美女的时候,就象有人突然关掉了满是噪音的电视一样,众人哑口无言。

一个穿着崭新保安制服,头上黄一绺,黑一绺的保安疑惑地左右看看,“兄弟们,这美女没那么漂亮吧?咋你们都抽抽了?”受气氛影响,这话声音不大。

“兄弟们”根本就没理他,等到那女子从远走近,又渐渐走远的时候,众保安才长出口气。

淡眉毛斜视黑黄毛一眼:“斑马你懂个屁,这是东三营村村长家闺女,她上下嘴唇一碰你起码住医院俩礼拜。”

“斑马”自然很不服气,“东三营不就是咱这里么?一个小小的村长闺女把你们吓成这样?咱这儿好歹也二十几号弟兄,再加上工人,丫的那点农民还敢闹事?不是找死么?”

众保安无言,果然是“无知者无畏”。

“二胡”皱着眉头,边琢磨边说,“其实……我觉得斑马这话也没错啊,东三营的这些农民是太狠了点,别的地方拉土方六块一方,他们跟咱们要十二还不叫外边人拉,欠揍不是?”

淡眉毛也搭腔,“谁说不是?在你的地盘你高高手,要个八、九块钱就是了,能照顾你本村人谁愿意生事去外面找车?这事情做得有点过,咱这宏达简直就是吃干饭的,是吧,黄哥?”

突眼睛的“黄哥”扫视大家一圈,伸俩手指出来做“V”状,还不停的勾动着。

老资格的“黄哥”又要开吹了,众保安都下意识的往跟前凑凑,“斑马”早把一根烟递到了“V”字开口处。

黄哥先点着烟惬意的深吸一口,又拎起身边的罐头瓶抿口茶水,把众人的胃口吊得足了又足,才慢吞吞的说了句比较让人意外的话:“东三营村算个球。”

扫视大家一眼,“黄哥”很满意大家的反应,“咳咳”两声开始白活,“以前都是东三营来求咱宏达的,求咱给他们点活,跟外边一样就行,可大家也知道,这东三营已经快算是城里了,给他们活他们干的太慢呐。”

“是啊,”淡眉毛在旁边附和,“他们已经算城中村了,闲惯的人干起活来那是慢点。”

“后来呢,这帮家伙招集了三、四百人来堵咱盛世年华的门,就跟前几天那阵势一样样的,”“黄哥”又喝口茶,清清嗓子“咱宏达能怕了这个?领导一个电话,来了五百多警察,干,那帮家伙跑得叫了个快,门口拉下的鞋都不下二十只。”

“后来啊,还是咱宏达好说话,咱总是在人家的地盘上不是?咱主动提出土方活都给他们了,还算九块一方,他们干不完才六块一方往外包,这价钱你们就都知道了。”

“那后来他们怎么这么不知道好歹,没事找事?”“二胡”替大家问了,“尤其是上礼拜又堵咱门,哇靠,不是我跑得快,脑袋差点让铁锹划拉半个下来。”

淡眉毛跟着搭腔,“是啊,也没见个警察,还不让还手,光知道答应人家条件,我他妈的都觉得丢人。”

黄哥左右瞅瞅,小声说:“现在东三营还是个球,咱现在这孙子样啊,因为……”手指指天,“上头有人整咱们。”

端人饭碗的自然希望锅结实点,淡眉毛也压低声音,“黄哥你的意思是?”

“意思?干,有屁的意思,”黄哥声音又大了起来,“咱宏达啥事没见过,这屁大的事能难了?我在宏达这么多年,大事见得多了,要咱真顶不住早撤球的了,谁还在这儿干挺?”

黄哥声音确实大了点,刚在门口下车的楚云飞都听见了,不过他不知道那些人在说什么,可跟他一起的高工听到了。

\section{第二十一章 不服水土}

楚云飞被安排到“盛世年华”工地,可他不认识路,而工地的总工高世杰正好在集团办事,就在回工地的时候顺路捎上他了。

高工看到保安在门口扎堆,眉毛一皱,“建国,不是让你们没事在屋里呆着么?又往外跑,操心刘经理再告你一状。”

这个黄建国就是保安部黄经理的侄子,若干年前甩开锄头来找他叔叔混进了宏达,有把子力气又没什么文化,就做保安做到了现在,人倒是不坏,只是没事爱占个小便宜,再顺便摆摆老资格。因为是老员工了,本职工作一直做得不错,又多少有点背景,所以宏达上上下下的人对他也是睁只眼闭只眼。

“高工,刘茂林说的是别出去乱晃,我们只在门口啊”,黄建国笑嘻嘻的说,上下打量了楚云飞一翻,看看楚云飞的行李包,“这是新来的保安吧?”

“恩,小伙子,这是保安黄队长,跟他去,让他给你安排吧。”

楚云飞跟着黄建国去了门口保安室,黄建国从一个破木头箱子里拿出来套旧的保安服,“喏,现在没新衣服了,这衣服别人穿了也没几天,洗洗还是新的,其他缺什么去门口小卖部买。”

大概楚云飞瘦瘦的身板让黄建国不太满意,他领着楚云飞站在院子里随便的指点了一下食堂厕所宿舍什么的,交代了开饭时间和值班时间就又去找人吹牛了。

楚云飞可是有点点失落,自己好歹也是以“特邀嘉宾”的名义来的,似乎待遇有点低吧?给件衣服居然还是别人穿过的?

不过,一路风尘仆仆赶来,确实是有些累了,看看表快到吃饭时间了,楚云飞甩开那些不愉快的念头,去小卖部买了饭盆和勺子,其他的东西没有买,他憋着劲下午好好问问黄队长怎么回事呢。

走进食堂,打饭的人很多,楚云飞懒得去挤,找个“小凳子”坐下——三块砖中间加了点水泥摞成的,周围打好饭的民工三三两两的走来,坐到类似的“小凳子”上,很快为数不多的“小凳子”就坐满了人,晚来的只好蹲在地上吃,不过看得出来保安的地位似乎比民工高点,总有民工主动为保安让座。

看看打饭的人少了一些,楚云飞正打算也去打饭,麻烦就找他来了。

淡眉毛保安走进了食堂,看样子很不屑和众多民工去挤着打饭,眼睛四下一扫,看到了坐在那里的楚云飞,就冲楚云飞走了过去。

楚云飞和淡眉毛打了个招呼,“来了?”淡眉毛却根本没有任何的礼貌可言,“你就是那个新来的?”楚云飞点点头,“有事?”

“去,”淡眉毛把手里的饭盆往楚云飞面前一递,“给我打一份。”

楚云飞有点发懵,看这位的腿脚很利索,不像有什么问题的,眉毛虽然淡了点,可这不影响他去打饭吧?

“去啊,”淡眉毛很不满意楚云飞的无动于衷,“还不把座位给我让开?”

哦,感情是自己坐了他的座位,楚云飞如是想,怪不得总见有人让座位,原来是来工地早的人早把这几个座位划分好了,后来的就只好蹲着吃了。

按理说楚云飞根本就不吃这套的,这座位又不是你家的。不过新来乍到,贸然挑战旧势力显然也不是什么好主意,最关键的是:是张志华叫自己来的,而背后还有个白为民的事要求人,为了朋友,也不能多事。

不过楚云飞的涵养肯定是不怎么到家,他气冲冲站起来,二话不说接过了淡眉毛的饭盆。

接过楚云飞递过来饭盆,淡眉毛以很专业的眼光扫视了一下内容,“操,全是白肉。”

楚云飞可懒得去想这话是对他说的还是对食堂打饭的师傅说的,他端着自己的饭盆溜达到个人少的地方就开动了,确实是有点饿了。

其实楚云飞把淡眉毛想得还是好了点,这家伙根本就是专门去欺负他的。

保安其实和警察差不多,都是欺软怕硬的主,而且以底层的保安尤甚,他们的欺软怕硬不但对外人是这样,内部也是这样。这点又有点类似监狱里的犯人,拳头大的就是爷爷,如果你够强,够狠或者后台够硬,那自然是没什么,否则你只好忍受“牛人”、“大神”的压榨,跟在别人屁股后面混。

楚云飞今天的遭遇,那就是大家习惯性的对新人的打压,换在监狱里那叫“服水土”,新人必过的一关。

尤其是楚云飞身材偏瘦,肤色白皙,又是一身的牛仔加旅游鞋,这样的学生崽那绝对是人见人踩,鬼见鬼害,扫帚见了都敢拍。

以楚云飞的博览群书,“服水土”他自然是知道的,可是以他的阅历,却想不到自己正在“服”保安版的“水土”。

当过兵的人吃饭是很快的,虽然楚云飞连打两次饭,但他吃完的时候大多数人还没有吃完,旁边的民工有两三个很诧异的看着他:这学生崽吃饭好快。

饭吃完了,一路的疲倦和满腹的怨气似乎也少了不少,看旁边有个有座位的吃完走了,楚云飞坐了过去,放松一下自己。

淡眉毛也吃得很快,吃完巡视一周,又看到了楚云飞。

楚云飞眼睁睁的看着淡眉毛走过来,心里的火气似乎又被勾起一些。

淡眉毛走到楚云飞面前,饭盆一伸,“给我洗碗去。”

楚云飞这次可没理他,过分了吧?眼睛眯起来了,放松的肌肉也在缓缓绷紧。

“去啊,”淡眉毛有点火了,“座位让开。”

楚云飞这下全明白了,火“腾”就起来了,操,原来让座位是这么个意思,你活腻歪了?

不过,楚云飞的性格是:越到紧要关头越冷静。他缓缓扫视了一下四周,发现有不少人在兴致勃勃的看着这边,更有几个一看就不是善茬的主在冲自己走来:是想离得近点看得清楚些么?

冷笑一下,楚云飞斜视一眼淡眉毛:“让我洗碗?咱俩很熟么?”先占住理再说。

淡眉毛根本无视楚云飞的表情,学生崽也会装酷?古惑仔看多了吧?

“老子认识你是个球,就是让你洗了。”

楚云飞慢慢站起来了,“欺负人?”

“切,”淡眉毛脸上一副“你很好笑”的表情,“老子就是欺负……”

“你了”两个字还没说完,楚云飞的饭盆兜头就砸了下去,接着饭盆一个横扫,淡眉毛就栽倒在地,因为被打到耳朵根部,直接晕了过去。

低头看看脸上挂着片菜叶的淡眉毛,再看看四周聚拢过来的人群,楚云飞手一松,“当啷”一声,已经成煎锅形状的饭盆掉地,猫腰捡起了淡眉毛的饭盆,对着饭盆“呼呼”的两口,吹去上面沾着浮土,冲着地上昏迷不醒的淡眉毛说,“这个算赔我了。”

楚云飞本来就不是循规蹈矩的人,从不认为一般场合下双方应该拉开架势再打,虽然他很能打,但先下手为强嘛,能省点事为什么不省点事?再说他自己也有被偷袭的准备:话那么多,你不先动手怪得谁来?

尤其是楚云飞原来所在的五连里,刚开始的时候很有那么几个战友就习惯“骂架”,双方站在那里什么“有本事你动动我”之类的话能对喊两三个小时,那种情况不先下手的话根本打不起来而且还得让人笑话。

所以楚云飞一般打架就是先让大家弄明白“事情是这样的”,然后直接开打,甚至有时候场面话都没有。

本来有几个保安在旁边打算看淡眉毛蹂躏新人的,可楚云飞的出手实在是太快了点,搞得周围的保安拉偏架都没来得及,等到他们反应过来,只看见楚云飞端着抢来的饭盆施施然走了出去。

楚云飞走了,食堂里可炸了锅,人声鼎沸,喧闹中几个保安把淡眉毛抬回了宿舍。

天大地大,人命最大,保安们虽然架打得不少,可并没有职业的系统的学习过人体解剖。倒是有俩退伍回来的保安,可他俩也没楚云飞那么了解人体结构,众保安你掐人中我屈膝盖他推后背,好半天才把淡眉毛弄醒。

黄建国中午是在小卖部吃小灶的,以他的“老人”身份是不屑去和大家挤食堂的,买单的自然是才来几天的新人“斑马”。等黄建国知道消息赶到宿舍的时候,淡眉毛已经悠悠醒转。

“黄哥,你可要给大家做主啊。”保安们七嘴八舌的向领导要求。

黄建国很快就弄明白了事情经过,淡眉毛想打压新人却没达到目的,这还了得?就算你是个狠人也要先躺下一回,这是规矩。

“服水土”确实是这样的规矩,你是狠人?你嚣张?以后的日子也许你大,不过就算以后你最大,“水土”你是必须服的,要不以后弟兄们不是随便你揉搓了?

“谁知道那小白脸现在在哪儿?干,弟兄们都带上家伙。”黄建国是老大,自然要负担起策划、组织和善后的工作,这小白脸看来不是善茬,你下手那么狠也别怪老子不给你留面子了。

楚云飞却是因为刚打了架,全身的兴奋劲儿还没过去,赶路的疲倦早就不知道哪里去了,午觉也不想睡,在院子里转悠半天,觉得太阳太毒,找了个阴凉地坐着。

远远看见七、八保安撸胳膊挽袖子的走了过来,手里还拎着凳子腿、铁锹把、墩布什么,淡眉毛最狠,居然双手握着一根一米多长的螺纹钢,报仇的迫切之心显而易见。楚云飞没怎么当回事——反正已经这样了,慢慢的站了起来。

这情况给谁也会觉得这小白脸要撒丫子开溜,保安们嘴里骂骂咧咧地就冲了过来,结果跑到跟前发现楚云飞没动,又讪讪的把举着家伙的胳膊放下,却是已经把楚云飞包围了,眼睛都看着黄建国。

黄建国斜着眼睛问楚云飞,“人是你打的?”

楚云飞本来还有点希望黄队长主持个公道,一听这话:明白了。他没回答黄建国的话却是把头扭向了淡眉毛,“咦,你脸上的菜叶子哪去了?”

是可忍,孰不可忍,这话肯定是在类似场合发明的,楚云飞话音未落,众保安手里的家伙纷纷落下。

既然已经被围住了,楚云飞就不想先动手了,那样太被动,这样乱哄哄的场合才合适他冲出人群,一脚踢开淡眉毛旁边的斑马,冲到淡眉毛身边狠狠一个肘锤,先冲出去再说。

先冲到身后的屋旁,楚云飞背靠墙壁开始左右移动,手脚并用,很快的的把一干保安打倒在地,只剩下了站在人群后面赤手空拳的黄建国。

黄建国自然是看得心惊胆战,干,好厉害的后生,是把好手!却看到后生把人挨个放倒以后又盯住了自己,“干,看个球,你还想打我?”

他要是好言相说,以楚云飞吃软不吃硬的脾气还真懒得动他:人家好歹是队长,也没拿着什么武器。

“你算什么东西!”楚云飞一个假动作,飞起一脚就把黄建国踢飞了。用非常不屑的口气说,“想找事儿的直说,玩儿阴的我也奉陪。”说完甩头故做潇洒的扬长而去。

\section{第二十二章 保安变保镖}

整人自然要有被人整的心理,可这楚云飞居然连队长也敢打,如此奇耻大辱,黄建国自然是不肯轻易罢休,否则他还怎么在工地混下去?

纠集所有保安再来一次群架?这似乎不合适,看看那小白脸,赤手空拳就放翻了七、八个人,不是一般的能打,要是人家拿个棍子什么的,所有保安加在一起恐怕也未必是对手。

话说回来,这样就算能把小白脸放倒,付出的代价也是可想而知的,闹成那样局面的话,宏达集团断然不会置之不理的,“服水土”毕竟是一种潜在规则,拿不到桌面上的。那到最后追究起来,于情于理肯定都是黄建国的责任。

玩阴的更不行,首先小白脸就不怕这个,看那样子还有点欢迎似的;其次黄建国知道自己做不出太阴的事来,可只要整不死人,后患必定是无穷的。

想来想去,也只有把楚云飞弄走,才能把面子维持下去,于是黄建国给叔叔打了个电话。

黄经理自然要为侄子做这个主的,于是打人事部的分机,“人力资源部么?我黄易啊,我这里有个新来的保安人品不好,我要开除他……好,等等我看一下他的名字……恩,楚云飞,对,没错,清楚的楚,云彩的云,飞机的飞。”

说完,黄经理把手里的联系册撕得粉碎,扔进了纸篓。

半小时后,人事部打来了电话:人事部电脑档案里没这么个人。

黄经理拿着听筒有点纳闷,“不是吧?你们上午才招的保安呀……对对对,应该是这名字,我确定……保安,没问题,怎么可能是文员呢?……那好,我再确认一下是楚云飞还是楚飞云。”

联系册已经撕了,算,再去纸篓翻吧,没人在?那只有自己去翻了。

翻着翻着黄经理一甩手,“孙彪,你给我滚进来!干,说你多少回了,怎么还往纸篓里吐痰?扣你两百奖金!!!”

高高壮壮的孙彪苦着脸把拼好的联系册递给了黄经理,于是黄经理又跟人事部联系,确定了不是文员楚飞云,要为侄子做主的叔叔才想起来,是张总的助理阿强把这个人带来的,为稳妥起见他必须要联系下阿强。

“阿强?我黄易啊,你上午带来的那个保安记得么?……我要开除他,怎么人事部没他的档案……好,我等等。”

放下电话,黄经理撇撇嘴,干,怪不得连建国也敢打,原来这小子认识张总,抬头看看孙彪还在,“干,建国这顿打怕是白挨了。”

阿强正和张志华一起赶路,汽车上也没外人,“张总,黄经理想开除你介绍的那个保安。”

张志华正躺在座位上闭目养神,实在是太累了,听到这话懒洋洋的问:“哪个保安?保安,哎,我忘记交代了。”

“告诉黄易,人不许他动,那可是救过我的人,小林子也知道,”张志华睁开眼睛,“小林子,楚云飞来了。对了,问问黄易为什么要开除楚云飞。”

没人的时候小林子话还是很多的,“小楚年纪不大,人可是挺仗义,功夫也好,强哥,你未必能打过他。”

阿强听见“楚云飞”这三个字就明白了,“哦,我知道了,兰山矿那个家伙呀,他能打过我?上次要不是我有事估计用不着他出手吧?”说着拿起手机。

“黄经理,张总说了,人你不能动,还有,你为什么要开除人家?”

黄易自然是要诉诉苦的,“我看他的身板没什么威慑力,就让他去盛世年华了,结果他一去就把建国他们十来个保安打了一顿,建国也挨打了,这样的人还能不开除?”

阿强会功夫,底下的这点事他还能不明白?“呵呵,我知道了,肯定是想让人家服水土,结果让人给菜了,那家伙可是会功夫的,还救过张总呢,你那亲戚可撞钉子上了。”

黄经理这才明白侄子惹了什么人,身为宏达的中层干部,自然知道这样的人才是宏达目前最需要的,“那你怎么不告诉我一声?我还当是个普通保安呢。”

“我也不知道啊,那小子刚来的时候张总在接电话,还要马上赶出去办事,随便和我说了下,我肯定是听到什么就怎么交代啊。”

“你这麻子不叫麻子,叫坑人,算,我把他喊回总部吧。”黄易恨恨的说。

挂掉手机,阿强和张总交代:“黄易把他安排到盛世年华了,他不是有个亲戚在那里?想给新来的下马威,呵呵,结果让人给菜了,打了十几个呢。”

下面乱七八糟的事张志华还是知道一些的,不过他可不象他的董事长哥哥,没精力管那么多,也懒得管,每个阶层和每个行业都有自己的规矩,在不危害公司利益的前提下,潜规则的存在自然有它存在的道理,何必去因为自己的好恶去强行干预?再说,什么都要管自己还活不活了?“活该,谁让他们不长眼?欺负别人自然也要有被人欺负的觉悟,哈哈。”

小林子也在旁边凑热闹,“我说么,小楚很明白事的人,怎么黄经理会开除他呢?”

“哦?”阿强在副驾驶座上瞥一眼林海峰,“你也算宏达的人呢,不觉得他这么做嚣张了点?”

“阿强,”张志华听出阿强似乎有点不服气,“我可告你啊,别乱来,那家伙好歹救过我呢,不过,”不知道出于什么心态,张志华又刺激了阿强一句,“我是怕你也被菜了,那多没面子。”

阿强让这话刺激得哭笑不得,张总你怎么说话呢?纯粹给我找郁闷呢。“不过这家伙算是把黄易得罪了,现在怎么安排他?”

张志华也想到这点了,“唉,都是这些破事闹的,我本来想的是让他给老黄当副手,随便挂个助理或者教练什么的,当保镖我怕他够戗,算,让我想想怎么安排他吧。不花钱的打手不用白不用。”

阿强听到向林海峰一扬眉毛,“听到没?当保镖都没资格,还和我比?”

林海峰也算是张志华的心腹,当私人司机的自然和老板关系好,所以他和阿强啥话也敢说,“切,那是以前,现在咱们这么多保镖里你能打过哪个?我看是小楚太年轻,也没啥名气,所以张总不太放心。你可没见着,那家伙发起狠来绝对是玩命的主,你想啊,打人能打到自己晕过去,佩服!”

狠人——因为打人能打到自己晕过去?好高深的观点,阿强正要反驳呢,张志华又说话了,“你俩还有完没完了?不错,就是小林子说的那个意思,现在住嘴,我要养养精神了,快累死了。”。。。。。。

违规者很快的被调回了总部,不但是黄哥保持住了自己的脸面,楚云飞也不用再去洗别人的衣服,该是皆大欢喜的场面了。可黄易还是比较郁闷的,因为他实在不想让这个打了自己侄子的家伙呆在保安部,由于黄经理是宏达从羊城市带来的老人,所以仗着多年苦劳,强烈要求把楚云飞安排在他视线之外。张志华也懒得和他计较,在百忙之中给楚云飞安排了个轻松活:为张志中的女儿张玉珊做保镖。

楚云飞十分不满意这样的安排,“张总,我给你侄女做保镖不太方便吧?”

张志华故意逗他,“不是吧?我们这些女方家长都没说什么,你居然还牢骚这么多?”

“行啦行啦,别苦着个脸,这不是也是为了你好么?我侄女一般不怎么出门的,你正好可以安下心好好复习一下,不是要考军校了么?要怪也只能怪你,不是你把黄经理的侄子打了,我还能没地方安排你?”

同一时间,北京城内著名会所“今夕何夕”的某包间内,一场重量级的谈判正在进行中。

一方是开国元勋张克诚的孙子张丰亚,一方是张志中的儿子张玉虎,两方正在为一个水利枢纽的工程承包内容僵持不下。

双方结怨起因就是军队企业的改制,在西北盛产棉花的“三线”中,有个军队被服厂也在改制企业名单里。最初是被宏达看上了,看上的原因很简单:这个企业运营状态良好,而且该企业常年接的是军队的定单,就算转制了,在一段时间中有巨大的惯性会继续承接军队定单,对于经营企业的高手来说,继续把它办成“二线军办企业”还是很有可能的,哪怕接手搭不上军队的线,也可以在惯性期内从容的对企业进行整合,另觅经营重点。

本来宏达接手这个企业就够不顺利的了,当地政府中很有些人对这个厂子有觊觎之心,好不容易借着军队的压力把干扰因素全排除了,可这个企业的一个副厂长不知道用什么路子联系上了张丰亚,于是张丰亚通过部队打招呼表示要让这个副厂长接手被服厂。

太子们虽然是很嚣张的,但是一般来说还是很少为类似的事开口的,这是当权者很忌讳的事:我们尊重老一辈革命家,也为你们的后代开辟了自留地,你们就不要随便插手到国家机构的管理项目中了。张太子也有自己的经营项目,按理说实在是犯不着横插一手,而且对手还是宏达这种企业。

所以张太子打招呼的时候,力度是非常大的,因为他知道宏达如果能听到这个声音,这个面子还是会给他的。他和宏达交情不深,为了保持身为太子的尊严,不合适专门托人去向宏达传话,再说了,一个外来的宏达,用得着么?

张太子想的不错,可宏达这里出了纰漏,纰漏出得也很能让人理解,好不容易把当地政府的一干人等排除了出去,剩下些不知死活的小螳螂宏达自然是不会放在心上的,于是工作重点就转移了。几个回合下来,当事情再次弄大的时候,宏达终于发现了来自北京的味道,但是张太子已经很不高兴了——给你脸了你居然不要?于是通过元勋的老部下强行将那个厂子拿下了。

负责西北这一块的就是张玉虎张公子,张公子是在哈佛读过MBA的,参加工作时间不长,虽然学历很高而且能力也有一些,但身为富豪子弟有点纨绔作风那是很正常的,人又年轻难免气度有点不足,董事长父亲才向他移交了一点点权利,他居然就稀里糊涂丢了其中一个厂子还惹了太子,太没面子了,张公子的自尊心严重受伤。

张太子你要做什么直接说一声就好了,虽然你招呼打的力度很大,可你不知道我们宏达工作重点已经转移了么?早不说话晚不说话,偏偏这个时候说,早点说的话,当地政府我都懒得应付,直接就走人了,这么做太不给我面子了吧?张公子显然没考虑到一个区区的副厂长要把事求到太子门下需要费多大的周折。

张公子生气了,后果自然很严重,同为京城上层,他自然知道张太子的一些东西。张太子家的自留地张公子是没能力也不敢去伸手的,可张太子还有些通过政府部门的过手油水,张公子就惦记上了,你惹我?那我回敬你。

张公子智商很高,自然做事也考虑周全,他也没明目张胆去打劫张太子,而是很隐秘的去拉张太子的后腿。听说张太子要包个水利工程然后转包,张公子就利用个外资公司去虎口夺食,由于张太子不便出头,自然张公子稳居上风。

可世界上的事确实没有绝对一说,张公子虽然事情做得高明,可由于属于私人恩怨(公子这么认为),不敢让他的张董父亲知道,年轻人做事不稳重,难免就有丝差池。

京城高明人众多,张太子感觉事情不对,大力去调查,这种局面下,恐怕就算是张志中出手也得露馅,何况小小的年轻张公子?

\section{第二十三章 谈判进行中}

张太子很轻易的就发现了是张公子在背后搞鬼,甚至太子都能打听出来这事根本就是公子的个人行为。

于是张太子真的生气了:面子是别人给的,可绝对是自己丢的。你不仁在前,就别怪我不义了。于是张太子动用各种能量,黑道、白道、红道、无间道一起出马,务必要给宏达一个惨痛的教训:太子,不是随便一个阿猫阿狗之流就能惹的。

于是宏达集团在各方面都受到了相当打压,虽然宏达海外的基业基本上没受到什么影响,虽然宏达根深叶茂家大业大,还支持得住,不过疲于奔命那是理所应当的。尤其是大多数的中间势力趁机拼命从宏达那里榨取油水:平时不方便怎么招惹你,现在可是该你出血的时候了。

事情于是就越发展越大,最后大到张太子也必须为某些事买单的时候,两败俱伤的结局就初现端倪了,太子和公子都不得不痛苦的承认:对手比想象中的难缠,事情不能再继续下去,否则就是白白便宜了看笑话的人。

不就是个小小的意气之争么?又不是有不共戴天的仇恨,在张志中张董的强烈建议下,两个年轻俊杰开始坐下来谈判。

既然双方都有和解的诚意,那和解就是个时间问题了。自然该有的步骤还是要有的,该做的工夫也是要做的,现在就是张公子和张太子的第三次谈判了。

张公子手里把玩着一个前清某亲王留下来的玉扳指,笑嘻嘻的说,“丰亚哥,其实我真的还是比较喜欢祖宗留下来的东西,非我族类,其心必异啊,你都不知道我上学时候怎么熬过来的,居然还说我黄皮白心。”温文尔雅的举止,尽显一代儒商风范。

张太子端起小茶壶,抿一口极品雨前雀舌,“呵呵,老虎,就算蒋介石带走了故宫的大部分文物,带走了那么多的名流高士,可大陆他是带不走的,中国五千年的文化底蕴,啧啧,哪儿是几条船几架飞机就能运得走的?所以我觉得你最多也就算得上喜爱,说精通可就不行了啊。”儒雅中透出丝丝豪气,端的不愧太子的形象。

“对于成功的商人来说,有喜欢的理由就可以做好,”儒商丝毫不为被怀疑而介意,“遗憾的是,小弟现在还算不上成功的商人,要不也不至于惹恼丰亚哥,自己找倒霉了。”

太子们衣食无缺,能量过人,讲的就是个面子,“哈哈,这件事哥哥我也有错,主要是咱们沟通不顺,沟通不顺,不许再说了啊。”

于是谈判一如既往的在友好的气氛中进行,两俊杰从伦敦的天气谈到赛马的配种,虽然圈子和经历不同,但所处阶级相同,共同语言确实是不少。

谈判的中心自然是那个水利枢纽工程,张公子已经下了大工夫和大价钱在上面了,自然不能轻易让出,那不是明白的告诉别人宏达惹不起张太子么?可张太子也不可能让宏达就这么把活拿走,要不太子的脸面何在?

不过两俊杰显然不能拿这个来讨论,讨价还价那是街头小贩做的事,如此高档的场合怎么能有这么粗俗的事发生?再说了,古人都说了——“功夫在棋外”,兄弟俩趁此机会交流一下感情才是重点,虽然没人说得清楚这里面的感情究竟是什么味道。

但是事情也不能这么一直拖着不谈,宏达那里不断的在鸡飞狗跳,太子这里不断的在受人耻笑。

谈到了赛马,儒商就稍微有了点咄咄逼人的味道,“丰亚哥,这就跟你说我的一样,赛马这东西,的确是西方的玩意,哥你作为个大陆人,说实在的肯定不够精通。”

张太子明显的收到了信号,“看看,书念多了不是?老虎,人家都喊我们太子呢,你真不知道太子的能量啊,哥哥又爱玩个马,你养的马绝对没哥哥的好,不服气么?”

确实不能小看任何人啊,儒商心里感叹,这么个花花公子这么敏锐的就收到信号回敬了回来,“那还真得看看哥的马了,要不咱们弟兄俩都把马拉出来比比?”

回过来个正确的反应!只要不做对手,张太子也愿意和聪明人打交道,“好啊,比比就比比,千万小心啊,到时候别说哥哥欺负你。”

儒商似乎为这句话说得伤了自尊,“哥你要这么说,咱们来点彩头?”

“彩头?”张太子似乎有点意外,“咱哥俩还说什么彩头,哥哥有的你看上了尽管开口好了,再这么说话就见外了啊。”

滑头,儒商显然吃了年轻和经验不足的亏,隐隐被太子压着一头,不过,强龙和地头蛇角力,那确实是有点束手束脚,“哥你别这么说啊,前些日子不是有点小误会来着?哥你要万一输了可得让让兄弟啊。”

“哦”,张太子做恍然大悟状,“对对对,啧,咱哥俩都忘了谈这档子事了,恩,不过确实是个小事,好,老虎你准备哭着回家吧。”

“不过我觉得还是有点占哥的便宜,咱大陆养马、驯马、配种这方面确实不行,骑师也是个问题,咱们再比两场国粹吧,三场定输赢好不?反正是个玩。”儒商觉得一场定输赢偶然性大了点,三场比较保险。

搁给个智商不够的太子估计就要生气了:你那么想赢啊?但是张太子看上去气度是绝对大的,绝对没有一丝一毫的介意,“呵呵,咱哥俩谁跟谁,老虎你这么爱玩,哥哥怎么能扫你的兴?”

玩,不但要玩得高雅,还要玩大点,这是上层人物的通病——不如此怎么能彰显自己的身份?于是两只高贵的狐狸就定下了三场比试:赛马——都拿出自己最好的马来,只比一场5000米,速度快的胜,马术就不用比了。

象棋——各邀高手一名,三局两胜,象棋爱好者众多,隐世高人也有,同时又便于观众参与和议论。

武术——各方出高手五人,五局定胜负,这就类似于门客的对抗了。

至于两主角高尔夫对决之类的项目,对抗味道太浓,观众的指指点点难免又和二人身份不合,自然不合适这个场合。。。。。。。

开着张志华的宝马车,林海峰把楚云飞送到了张志中家,那是栋别墅,院子有三百多平米,院中种满了花花草草的,还有草坪和一个小的喷水池,池子里还有个撒尿的小男孩雕像,类似于西方传说中的某个人物,可惜不是铜的是石头的,典型的欧式风格。

门口一个小屋子,里面是门卫,林海峰跟门卫打个招呼,“小亮,这是小楚,给小姐做保镖来的,张总安排的。”

“哦,欢迎欢迎,这么一来,兄弟我就能轻松点了。”可小亮上下打量楚云飞的眼神里,怎么也看不出“欢迎”的味道,这家伙能做保镖?

“曹妈,”小亮冲屋里轻喊一声,“有人来了。”

一个50多岁的微胖妇女应声而出,衣着干净整洁,气质也不错,不象个佣人老妈子之流,倒象个家政服务中心的老板娘。

“小兔崽子你再叫我曹妈我扒了你的皮,”曹妈微笑着威胁小亮,“告你多少回了要叫曹婶。”

楚云飞在旁边看得直想笑,“曹妈”这叫法听起来确实不怎么地道,恩,有歧义。

曹婶看到了来的两个客人,“呦,这不是二爷的司机么?小……小林子是吧?有事么?”

小林子在曹婶面前非常规矩,“曹婶,这是张总给小姐安排的保镖,楚云飞,人我带到了,后面你安排吧,我走了。”

曹婶上下看看楚云飞,“好腼腆个小伙子,象个学生,可不象个保镖。”

其实最近张家一直在招高手做保镖,也网罗到几个高人,不过曹婶敢当面这么评论的估计也就是楚云飞,可见他的形象确实属于“人见人踩”型。

“跟我进来吧,先放了东西,”曹婶扭头往屋里走,“小声些,小姐在睡觉,她一直神经衰弱,觉很轻,吵醒她可是要挨骂的。”

楚云飞下意识的抬头看看天,没错,是10点多了,小姐还在睡觉?

曹婶看见了楚云飞的动作,“没错,小姐是作家,睡得晚,起得也晚。”说着话就把楚云飞领进了别墅里。

“大爷打电话来了,知道你要来,地方我已经给你安排好了,你看,”曹婶指着一层门侧的一个房间,“那就是你的房间了,先把东西放进去吧。”

楚云飞带得东西其实很少,行李箱里光书就占了一大半位置,虽然北京买书方便,但买书也要花时间的,不如把要用的书带上,也能节省点钱。

看楚云飞摆放好了带来的东西,曹婶又带着楚云飞转了转房间,指点了洗手间、餐厅、书房、客房的位置,然后指点着二楼向楚云飞介绍:“那里是小姐的房间,那里是老爷的房间,小姐房间对面的这个是少爷的房间,那里是健身房,那里是小餐厅,反正二楼没事的话你最好少上去。”

三楼就是老爷和少爷的书房,小姐的写作室也在三楼,三楼另一半是阳台。

楚云飞看得自然是目瞪口呆,嘴里喃喃自语:“果然是大富之家,厉害,真厉害。”

也许是因为楚云飞人畜无害的长相吧,曹婶好象很喜爱楚云飞,忍不住透漏点小秘密,“这算什么呀,老爷在羊城的房子比这大得多得多了,还有游泳池、假山、小桥流水呢,吃的鱼有很多都是从自家湖里钓上来的呢。”

“好了,你都知道了,先熟悉熟悉吧,我去收拾屋子了。”

很奇怪,楚云飞并没有那么强烈的羡慕,年轻人在想的是:这么大的房子就这几个人住,不会觉得空荡荡的么?

\section{第二十四章 追根溯源}

等楚云飞一觉醒来,已经是下午三点了,看来是太累了点。昨天从工地回到宏达大厦,逛街逛得太晚了,结果不得不在保安室凑和了一晚上,没休息好。

推门而出,却看到沙发上有个长发披肩的年轻女孩在吃东西,茶几上一个盘子里装有半盘子水果沙拉,那女孩手里拿个很精美的小瓷碗,一个小勺,很优雅很认真吃的样子让人不禁想起“大家闺秀”四字。

听到楚云飞出来的声音,那女孩慢慢放下手中的小碗,抬头看看,“你好,你就是小楚么?”

声音细腻甜美,却带有些很自然的鼻音,给人一股很慵懒的味道。

女孩很漂亮,其实在楚云飞的眼中,审美标准已经很自然的严重降低,当兵一年多了,接触的除了男人就是动物,已经没有刚入伍时眼光那么挑剔了。

不过就算以楚云飞刚入伍时的眼光,眼前的女孩也算得上漂亮:肤色白皙,头发乌亮,长睫毛,双眼皮,嘴巴虽然大了点,但别有一种说不出的味道。

“哦,张小姐?”

“对,是我,家里情况曹婶都和你说过了吧?”

楚云飞点点头,“才起来?”

张小姐笑笑,眼角带起一点点的鱼尾纹,“不,起来一阵了,有个朋友要来,在这里等等她。”

面对富家子弟,楚云飞可不象一般人那么缩手缩脚,很自然的问,“你平时就吃这个?”

“是啊,”张小姐似乎很惊讶楚云飞这么说话,“吃这个不好么?”

“倒不是不好,”楚云飞老实的说,“我觉得这东西热量不是很够吧?”

“热量?”女孩更惊讶了,“我平时不怎么活动的,我是作家,需要的是营养,充分的营养。”

“切,”楚云飞很不以为然,“脑袋构思文章不需要热量么?光有营养有什么用?”

女孩白他一眼,“我还是女士啊,身材总得保持吧?你个小毛孩知道什么?”

“那倒也是,”楚云飞显然没多少和女孩打交道的经验,“那你在这里等人,我去看书了,有事叫我啊。”

英语书还没看几页,就听见门开的声音,一个清脆的女声传了进来,“玉珊姐姐,我来了。”

有人来了,于情于理楚云飞都必须过来看看,于是把书放下。

一个短头发的女孩双手环着张小姐的脖子,高兴的跳着,不过楚云飞的心思却全放在了她身后的汉子身上。

军人!肯定是军人,楚云飞马上就嗅出了同类的味道,而且从站姿和气势上,隐隐感觉出了彪悍的味道,保镖!楚云飞很快判明了对方身份,又是一个身份不普通的女孩!

不过楚云飞很快就释然了,周敦颐不是说过么?“往来无白丁”,大富人家交往的对象——简单得了么?

那汉子在第一时间也注意到了楚云飞,而且很敏锐的察觉到了楚云飞的危险性,两保镖就这么上上下下的仔细打量着对方。

两个女孩子可没注意到俩保镖的异常,张玉珊甚至没有发现楚云飞已经走到了客厅,对她来说,那个小毛孩只是一个很陌生的人,陌生到她还没有习惯接受对方的存在。

“琳琳又漂亮了好多啊,大美人了,江南的水土好养人,姐姐也想去住两天呢。”

大美人?楚云飞自然要回头看看,恩,确实不错,高挑个头,娃娃脸,大眼睛,和张小姐不相上下。

大美人高兴得快合不拢嘴了,可嘴上还在谦虚,“姐姐你少笑话人家了,哪儿赶得上姐姐呀?你才是大美人,我……我最多也就是个小美人。”

楚云飞听得啼笑皆非,这小丫头,别是智商有什么问题吧?

不过张小姐显然很习惯对方这么说话,继续打趣小美人。

大小美人寒暄半天,小美人才注意到客厅里的陌生男人,“这位是?”

看周琳琳挤眉弄眼的样子,张玉珊自然知道小丫头在想什么,微微一笑,“来的时候迷眼了?这是我爸给我找的保镖。”

周琳琳没想到这个常年在家里呆着的才女姐姐居然也有了专职保镖,张了张嘴想说些什么,不过张玉珊可不想提现在家里遇到困境。“琳琳怎么这么早就回来了?不是说要到六月的么?”扭头对楚云飞吩咐,“小楚,都是熟人,你回去吧。”

楚云飞掉头往房间里走,却听见周琳琳说,“恩,有个低我一年级的路菲菲被保送清华了,她在北京不认识人,我就陪她来了。”

路菲菲!!!毫无准备的楚云飞当时就处于石化状态了!!!

大小美人可都没防备,被低沉的吼声吓了一跳。彼此对视了一眼,还是张玉珊走到了石化的楚云飞身边,“小楚你怎么了?”

楚云飞半天才解除石化状态,却看见张玉珊站在自己面前怒视着自己,“张小姐?你怎么了?”

大家闺秀的气度绝对不是一天两天养成的,虽然很生气,但张玉珊的举止还是很优雅的,“你为什么学我们说话?”

“学你们说话?我有么?”楚云飞很莫名其妙,扭头看看,却看到周琳琳诧异的眼神和周家保镖戒备的神色。他自然不知道自己在失神状态下,咬牙切齿的喊了声“路菲菲”。

这个保镖该辞退了,张玉珊心里暗叹一声,可说话的声音却越发的轻柔,“哦,没有就算了,你下去吧。”

周琳琳可没这份顾忌,反正丢人的不是自己的人,“哎,这个保镖,你认识路菲菲?”

路菲菲?我怎么会不知道她?没她的话父亲也死不了!!!

不过楚云飞已经恢复正常了,“你说的是不是那个江南省的,全国物理竞赛排名第八的路菲菲?”

周琳琳惊讶的张大了嘴,“你真的认识她?”惊讶自然是难免的,一个小保镖会认识个小才女?

楚云飞努力的笑了笑,比哭好看不了多少。“我只是知道她。”

周琳琳可就更诧异了:“你居然知道她?她那么有名么?”

楚云飞真的懒得再说什么了,父亲的死路菲菲是绝对脱不了干系的,没她物理竞赛排名不会晚两天才出来,没有晚两天父亲又怎么会被绑架?不过这事怨也只能怨江南省教育系统那帮家伙,难道自己找路菲菲报仇去不成?

不过不回答客人的问题是不礼貌的,何况又是个美女,楚云飞无奈的笑笑,“好象她确实没那么有名。”

周琳琳可是越发的感兴趣了,“那你怎么知道她的?居然知道她排第八?你也是十中毕业的?”

楚云飞偷眼看一下张玉珊,大小姐脸上虽然微笑依旧,但小保镖已经感觉到了丝丝寒气,于是不再和小美人纠缠,“没什么,因为我排第四,没事我就回去了啊。”后面这句话是对张玉珊说的。

楚云飞对周琳琳抱歉的笑着点点头,算是招呼过了,扭头就走,背后传来小美人清脆的声音:“玉珊姐,这个人居然……是你的保镖?”

回到房间里,楚云飞的心情实在是难以平静,这天下的事,真是要多巧有多巧了,居然在这里能听到路菲菲的消息。其实这个人应该算是跟他八竿子都打不着的关系,可楚云飞在这两年里仔细回味了父亲遇难前前后后的关系,不但路菲菲和刘阳的名字被牢牢的记住了,赵学工更是被楚云飞定义为灾难源头,不是他的话楚云飞至于去参加物理竞赛么?当然恩师物理郭老师被楚云飞有选择性的忽略了。

对了,来北京了,要不要去见见王展强老师?楚云飞仔细琢磨了琢磨,虽然现在自己这副样子不合适去见老师,不过现在死活是学不进去了,就当出去散散心吧,估计王老师也很关心自己的近况。

于是楚云飞再次走出房间,咳嗽一声,等周琳琳停止说话的时候,对张玉珊说了声:“张小姐,现在既然有客人,我想请个假出去一趟见见我的老师,你看行么?”

\section{第二十五章 大小美人}

看着楚云飞放下电话,走出屋门,周琳琳纳闷的问张玉珊:“你真不知道这家伙叫什么?”

张玉珊似笑非笑的看着周琳琳,“是不知道,就知道他姓楚,连是清楚的楚还是褚遂良的褚我都不知道,你这是问我第二遍了哦。”

周琳琳的脸刷的就红了,“姐你的笑……怎么那么古怪啊?我可没有别的意思。”

“哦?”张玉珊很诧异的样子,“是么?那你刚才冲我挤什么的眼睛?”周家保镖也出去找小亮聊天去了,两姐妹自然就放肆多了。

“那人家不是不知道么?”周琳琳倚小卖小,“我是真的很好奇那家伙,全国第四?他在吹牛吧?”

“那我就不知道了,就听说这个保镖是请来的,不是雇的,也不用花钱,不用白不用的那种,要不你以为我有必要弄个专职保镖?”张玉珊说话条理非常清晰,不愧是作家,“不过听说这家伙今年好象也要考大学。”

“哦?那他不用在学校学习么?”周琳琳实在是个好奇宝宝,不过这种情况下似乎也不能怪她,楚云飞实在是怪异了点。

“这我就不知道了,”其实张玉珊对楚云飞的好奇心比周琳琳还大,以她作为一个作家所养成的敏锐观察力,再加上刚才楚云飞咬牙切齿的一声,小保镖身上应该有不少的故事,这无疑是她收集大量素材的好机会,可面对周琳琳,虽然很熟悉,却也不能太忽略自身的形象,“你回头问问罗菲菲,没准她知道呢,人都是分圈子的,他们之间没准也有个圈子。”

“姐你什么耳朵啊?是路菲菲,道路的路,不是罗菲菲。”

“好了好了,路菲菲,”这时候的张玉珊真的很象个大姐姐,“对了,第八能保送到清华,第四需要考大学?这又是怎么回事啊?”

“不是那样啦,路菲菲在全世界奥林匹克物理竞赛上拿了银牌的,”周琳琳反驳说,“虽然几块金牌没她的份,可也是银牌呢。”

“那这个小楚不是起码也是块银牌?”张玉珊问道,“第四应该比第八强点吧?你想想印象中有没有这么个人。”

“哎呀~”周琳琳拉长声音,显然对这个姐姐有点头疼,“路菲菲是我们学校的,其他人我们怎么会清楚?我都不知道有哪些人拿了金牌!”

“呵呵,”张玉珊笑了笑,“也是,学校宣传自己的学生还忙不过来呢,不过,这么说来路菲菲更该清楚他了,一起参加奥林匹克竞赛还能不知道?”

“好了,姐咱们不说她了,你最近写些什么东西啊?”

“随便写点杂文呗,另外在构思个中篇,想写两个大家族之间的恩怨,主要想写两对情人的感情纠葛,唉,没办法,现在闭门造车,只能写些自己熟悉的背景了。”

“姐你也该出去多走走了,”上下打量着张玉珊,“你快成宫廷文人了。”

“臭丫头,”张玉珊嫣然一笑,“宫廷文人是那么好当的么?出去采风我不是没想过,可是看来还要等等……”

话没说完,周琳琳就大惊小怪起来:“姐,你笑起来眼角有皱纹了啊!”

“什么?”张玉珊再也保持不住淑女形象了,尖叫一声,站了起来,“我要去照镜子!”

………………

周琳琳没坐多长时间就走了,虽然姐妹俩有说不完的话,但她刚回来一天,自然是不能回家太晚,再说她还面临着今年的高考,虽然以她的家庭条件随便去哪里都没什么问题,但总归还是要考试的,分数自然是能高还是高点好。

等到楚云飞回来的时候有八点钟了,曹婶很热情的问楚云飞吃过没有,没吃的话她再去做点。这显然不是客气话,楚云飞刚来北京,人生地不熟的,楚云飞笑着解释老师请他吃过了,不过由于争执,最后采用的是AA制。

等张志中回来的时候已经是晚上九点多快十点了,还好楚云飞忙着学习还没睡下,终于见识到了这个自己听闻了无数次的传奇人物。

张志中跟他的弟弟长得一点也不象,瘦高的个子,三角眼,染过的头发乌亮茂密,行走、站立间很有气势,楚云飞甚至暗暗猜测他是不是当过兵。

张志中对楚云飞非常的热情,热情到楚云飞有点受宠若惊的感觉,虽然总共也没说几句话,但张董的厉害之处岂可是一般人能够企及的?

张玉虎是在十点刚过回来的,他回家的时候张志中还在和楚云飞寒暄,张公子和战士打了个招呼就拉着他的张董父亲商量事去了,楚云飞也很识相的回他的房间继续学习。

第二天一大早,楚云飞在院子里锻炼的时候,张志中的菲亚特坐驾已经在院子外面等着了,看来浮华的背后总有各种的压力在驱策着。

接下来的日子,楚云飞复习他的文化课练他的功,张玉珊爬她的格子码她的字,倒也是两不相扰。

没事的时候,尤其是晚餐的时候,因为张家父子从不回家吃饭,而张家女主人在香港坐镇,总是两女一男的局面。

虽然张小姐恪守“食不言,寝不语”的古训,但饭后总有闲来无事消食的时间。虽然身份有些差异,但面对“采风”对象,张玉珊也没有计较太多,所以慢慢的楚云飞和张玉珊之间还是相互增加了不少的了解。

从聊天中楚云飞了解到了张玉珊今年24了,因高中时期估计遇到一场变故,没有念完,然后就在家中呆了几年,跟随家庭教师学习,后来去欧洲、美洲转了将近一年,回来把沿途所写的随想、心得和见闻写了一些出来,很是引起了一些关注,甚至在女性杂志《评文论梳》上开了个专栏成为版主。再后来就越发的收拾不住,名声鹊起,隐隐然有成为具备一代女“驴友”(爱好自助旅游的朋友)和“冷眼红颜”双重身份的高人,因身份尊崇,不堪受人骚扰,近年隐居家中。偶尔有两三张稿纸流露出去,却也维护了名声,没有被世人遗忘。

楚云飞自家的事自然也是交代了不少,可每当张玉珊问起来关于有关他父亲的这一系列的事时候他就不认真说了。没别的原因,只是楚云飞觉得这个女人对这件事过于执着,隐隐有让他不安或者说不爽的感觉。以至于张小姐一旦在这件事上进行试探,楚云飞就马上表示出对张小姐中学遭遇的事情有浓厚兴趣。

既然每个人都有属于自己的隐私,张玉珊又是气质雍容的大家闺秀,那话题自然就会转向别处了。

于是楚云飞就知道了那天来的周琳琳,是京城某高官的女儿,和张家谊属世交,感情自然很好。因为京城纨绔众多,衙内横行,不利于孩子的成长,周琳琳又是女孩子,于是就被送到了外婆家。在江南省上了三年高中,这次转学回来,应付即将到来的高考。

周琳琳还有两个哥哥,也是属于衙内系列的,也许正因为俩衙内的恶劣表现,才让周琳琳受那背井离乡之灾吧。

等到周末的时候,周琳琳又来了。

这次楚云飞就连出来的必要都没有了,探头一看是小美人,又缩头回去继续念他的英语。

大小美人继续女儿家的私语,可是很奇怪,若干话题过后,居然又谈到了楚云飞身上。

\section{第二十六章 好奇心过重}

“姐你那个小保镖确实挺有意思的,我问过路菲菲了。”周琳琳说这些话脸一点也不红。

“哦?”当姐姐自然要有个做姐姐的样子,只是应了一声表示感兴趣,没做太多的表示。

不过这一声已经足以让小美人继续说下去了,“路菲菲想了半天也想不起有个姓楚的,我还以为你的保镖吹牛呢,后来我一说他排第四,菲菲才想起来,这个人没去参加竞赛。”

“嗯,这个我知道,”张玉珊点头,“可我不知道他为什么不参加竞赛,这家伙也不告我。”

“路菲菲说他爸死了,”周琳琳因为比姐姐知道得多点,居然脸上有点兴奋,“好象他爸的死和他有点关系,他一气就不再上学了。”

好题材啊!张玉珊一下激动起来,不过想想不太适合自己的风格,难免又有点失落,“真的?你那同学是怎么说的?”

“她说她排第八本来没有参加竞赛的资格,和第七只差半分,就算去也是替补,气得她还哭了好几天。谁知道后来学校接到通知让她去参赛,等她去了才知道第四名去不了啦,老师们还去小楚的学校找过他,听说是他爸爸死了,因为跟他有关,他也退学了,其他的事老师们就不说了。”

“哦。”张玉珊没有得到自己想要的信息,这一声充满了遗憾。

“不过我挺好奇的,”周琳琳忽闪着大眼睛,“我琢磨了好几天,也问过路菲菲,她说楚云飞应该不认识她,就算可能在培训班见过,也肯定互相叫不来名字,不知道那家伙听到路菲菲的名字怎么反应那么大。”

“路菲菲长得好看么?”张玉珊又开始构想了,作家的想象力还不是一般的惊人。

“好看什么呀,”以周琳琳的资本,评论别人绝对不会客气,“站起来和躺下差不多高,眼睛上还有副大圈套小圈的眼镜。”

“算,你也别琢磨了,你的保镖不是在么?我问他。楚云飞,你给我出来!”

声音挺大,先是曹婶在二楼扶手处闪了一下,看没什么事又马上消失,跟着才是楚云飞不紧不慢的从房间里出来。

“周小姐,有什么事?”声音那么清脆,绝对是小美女在喊。

“没啥,我回去问路菲菲了,她说根本不认识你。”

“嗯,没错,是我知道她。”楚云飞脸上毫无表情。

“那我就奇怪了,你学习比她好,她长的也不好看,怎么上次说他你那么大反应?”

楚云飞笑笑没说什么,他能说什么?

周琳琳很不满意楚云飞的反应,眼珠一转,撒手锏飞出,“听说你爸的死和你有关?”好奇心上来,周小姐才不管别人的感受。

这种情况出现也在楚云飞意料之中,他已经听王展强老师说过了:老师们专门去找过他。

“是路菲菲说的么?”

“是啊,她还以为没准你是进监狱了。”这个猜想实在是不能怪路菲菲,竞赛当头,老师们自然不会就太多无关信息和学生们交流。

“操,”楚云飞口吐脏字,一点也不因为面前是俩女士而收敛,“不是她我爸爸会死?说我进监狱?”

哦?大小美人对视一眼,很默契的选择性的忽略了那个脏字。

“是么?怪不得上次你那么大反应。”周琳琳显然是因为遇到个好听的故事而激动,“那你说说到底是怎么回事?”

还有完没完?楚云飞还真的有点烦了,合着大小美女轮番上阵啊?“凭什么跟你说啊?那可不公平,除非你能告诉张小姐高中的时候怎么不上了,那我就告诉你究竟怎么回事。”

“你!”周琳琳气得差点跳起来,自己这天之娇女什么时候被这样拒绝过?“凭什么?信不信我有十种以上的办法让你说出来?”嚣张——不仅仅是男人的专利。

“琳琳!”张玉珊发话了,虽然很欣慰周琳琳维护自己的隐私,也恼怒楚云飞的不识抬举,但是毕竟不能坐看形势恶化,“注意点形象。”

楚云飞既然心情不好,自然是想到什么说什么,可也没想到周琳琳这么大的反应,不过再想想,也只不过就是个会投胎的家伙而已。嚣张?你那小样也敢到处嚣张,看在是个美女的份上,懒得跟你计较了。

想来想去,楚云飞居然觉得周琳琳实在很好笑,不由得嘴角露出了丝笑容,为了掩盖自己的笑意,马上向张玉珊请示,“张小姐,还有事么?”

大家都是人,谁又能比谁高明多少?楚云飞自觉掩饰得不错,周琳琳可全看在眼里了,气得小手直指楚云飞,“你,你敢笑话我?”

楚云飞脸上已经是波澜不惊了,周琳琳只好扭头向张玉珊,“玉珊姐,这个……我……你……你能不能把这个保镖送给我啊?”

虽然已经是社会主义初级阶段了,但是在上层,相互之间以下属做为交换和赠送的礼物这一现象也很常见,特权阶级自然是有自己的特权的,不过礼物自身有很大的自主权就是了。

“胡说什么呢?”经过几天接触,张玉珊也知道了楚云飞跟她家的关系。“他也算我们家的客人呢,两个小孩子,别瞎胡闹了。”

看着周玉珊气呼呼的样子,张玉珊冲着楚云飞微微抬下下巴,楚云飞就回房间了。

等到楚云飞进了房间,张玉珊悄悄的安慰周琳琳,“别说我没权力把他送你,就算送了你也拿不走。”

“他不走?由得了他?”周琳琳自然是不信的,“就这种小毛孩子,老刘一只手就能对付俩。”

“琳琳,你不小了,做事要动动脑子,”张玉珊佯怒,“他凭什么能做保镖的?”

看周琳琳要张嘴分辨,张玉珊赶紧打断,“好啦好啦,我知道老刘厉害,这家伙也厉害着呢。别看干瘦干瘦的,人家会内家气功,救过我二叔的。”

周琳琳还真是吓了一跳,这小白脸真有这么厉害?不过玉珊姐是不可能骗自己的,可自己也不能就这么让个小男孩耻笑吧?

小美女怀着这么个心思,心不在焉的和张玉珊聊着天,聊着聊着,张玉珊又问起周琳琳,“你看我现在眼角的皱纹还能看出来么?”

周琳琳装模做样的看了半天,虽然已经看不见皱纹了,还是来了句,“基本上是没了,好象还有点轻微的纹路哎。”

“是么?最近我也没熬夜了呀,还天天保养,”张玉珊天天看自己的眼角,感觉正常以后才敢这么问琳琳的,听到这回答,又是有点紧张,拿出小镜子照半天,“光线不好,你先坐坐,我回房间再看看。”

楚云飞刚念了没几个单词,周琳琳就进了他的房间,“很刻苦啊。”

楚云飞笑着点点头,继续念他的英语,小美女也没介意,凑到他身边悄悄的说,“我告诉你玉珊姐的事,你也把你的事告诉我好不好?”

楚云飞停了下来,“好啊,你说吧。”

“刚才不是我不说,当着玉珊姐,我怎么能那么不够意思?”周琳琳解释下自己的苦衷,“其实事情也很简单,玉珊姐得了白血病,白血病分很多种,她得的是那种需要保养的,保养好了就能基本康复的,不过这病不能受刺激,大家才不说。”

是这样么?楚云飞疑惑的看了周琳琳一眼,不过这事显然没办法求证;而自己的这点事情虽然说不说无关紧要,可对方处心积虑的要想知道真还让人有点接受不了。

再看看小美女忽闪的大眼睛,又有点心软,算了,自己还得学习呢,真个是阎王好见,小鬼难缠,唉。

就在同一时间,张玉珊照了半天镜子,还是看不出皱纹来,从房间出来了。隔着扶手向下一看,周琳琳不在大厅,思考了下,摇摇头,又进房间去了。

楚云飞把自己和路菲菲的恩怨源源本本的讲了出来,按周琳琳的计划就该拍手嘲笑他:哈哈,刚才我是骗你的。

不过小美女虽然有点点蛮横,毕竟不冷血,基本上就处于垂泪欲滴的状态了,“不好意思,说到了你的伤心事,都是我不好,不该这么逼你。”

楚云飞心里已经够烦的了,又看着美人大眼睛红红的,实在有点无奈了,“大姐,我最见不得女人哭,你能不能出去哭啊?”

“大姐?我有那么老么?”

\section{第二十七章 戏演全章}

楚云飞有点后悔当时用“阎王好见,小鬼难缠”来腹诽某个美丽的雌性“小鬼”。

因为在那以后的的日子里,周琳琳大小姐成了他挥之不去的梦魇,确实难缠。

开始自然是找他请教物理问题,有求于人态度自然要好,于是那个嚣张的小姐不见了,俨然一副大家闺秀的样子。请教完问题还顺便和楚云飞随便聊聊,因为两人年纪相当,还是有点共同语言的。而面对笑容可掬的美女,楚云飞也愿意多聊那么两句。

这么请教了几天后,周琳琳忽然良心发现:好象不能白请教老师。鉴于楚云飞在京城举目无亲,收入单薄,周小姐给楚云飞送来了大量的参考资料。

这些资料当然不是周琳琳专门给楚云飞买的,那都是周大小姐用过的,只是保养得比较好而看似新书就是了。不过楚云飞后来还是弄了把小刀,一边裁开“旧书”中偶尔没有被切割开的书页,一边感叹周琳琳读书的不求甚解。

单物理一门自然费不了楚云飞多大的事,很不幸的是:周琳琳又不小心发现了楚云飞在其他几门学科上造诣也很深刻,那么,有现成的老师能不用么?

谢师礼的内容自然也因之扩充了不少:老师英语发音似乎不太标准,那学生弄上一百来盘英语磁带总不是问题吧?什么?老师没有录音机?那正好,周小姐刚要淘汰个小录音机,借给老师用用算了,那台录音机碰巧也是保护得比较好的。

后来周小姐居然想学功夫了,因为她觉得楚老师肯定会答应,而老师的衣着恐怕不便指导她,于是周小姐又弄了两套衣服来。衣服当然是她两个哥哥买的,而买后才发现尺寸似乎有点不太合适。不过鉴于此礼出自他人,做老师的还是婉拒了。

楚云飞莫名其妙的就陷入了这种“你来我往”的烦恼中,鉴于来而不往非礼也这句老话,做老师的自然也要加倍努力学习,一来被学生的问题难住的话似乎没有面子,二来就是学生经常占用老师大量的时间,时间不抓紧是不行的。

张玉珊小姐似乎就有点小气了,周琳琳这么好的姐妹,不过在她家多吃了几顿饭而已,她好象就不太开心了,话也不怎么和楚云飞说了,倒是经常在周小姐在的时候还能记起自己有个保镖,指派她的保镖做点基本上没有什么意义的活。她似乎忘记了,周琳琳最近可给她提供了不少素材呢。

曹婶对楚云飞还是一如既往的热情,老人家还经常劝告楚云飞:没事的话白天可以多出去走动走动,熟悉熟悉北京城,反正深夜没人的时候学习效率高。。。。。。。

张玉虎一向给人的感觉是温文尔雅,但是这不代表他不会生气。

现在的张公子就很生气,完全没有了一代儒商气质,手拍椅子扶手大声谩骂:“什么东西,这种货色也敢号称高人?气死我了!下棋就下棋吧,还端个紫砂壶,穿个小马甲?以为是演戏啊?海涛这就是你们找了俩礼拜的高人?纯粹一江湖骗子!”

实在是不能怪张公子生气,自从哥俩定下三场的赌局,张太子那里就开始宣传开了,也没别的意思,无非借个消遣的机会大家聚聚,图个热闹而已。

娱乐内容自然是:张太子与张公子的友谊切磋!其实知道有赌注的早就知道了赌的是什么,不知道赌注的那永远也没资格知道。

等到俩人的赛马休整好要比赛的时候,起码来了六、七个号称“同好”的太子们观战,那些圈内混饭吃的闲人够资格的也基本上来全了,把个“英皇马术”的赛场填了一小半。

主菜也没那么快就上来,自然有饭前甜点和饭后水果的。

不出张公子所料:赛马确实是西方的玩意儿,公子的“黑杰克”以四、五个马身的巨大优势跑赢了太子的“野火”,众太子的嘲笑惹得张太子差一点点就掏出枪来干掉“野火”。当然大家也只是闹着玩的成分居多。

接下来象棋比赛按计划进行,这场比赛其实有众多人已经料到张公子必然放水。没别的原因:已经说好了三场比赛,张公子要是赢张太子个2:0的话,不但张太子的面子上实在难看,也是对众多观众的不负责任,尤其观众里还有个把口碑不太好的主,再说张太子还邀到了全国冠军、南国第一高手刘钦坐镇。

自然也纷纷有高人到张公子门下引见相与之人或者自荐,尤其全国亚军杨大华更是不请自来,可见虚名确实害人。

张公子却对这些人不感冒,纷纷好言劝说走了。不需要过多的解释:你能拿什么来保证赢棋?

刘钦摆明了是全国已知高人中的翘楚,想制他只能从不世出的高人中寻觅。最后终于在西南部山区某个小镇上找到个“让车无敌”的超级高手,经众多候补高手鉴定确实厉害。

大家都在认真的演戏,也都在认真的看戏,“传说中的高人”就是戏肉,有点脑子的都知道这世界哪里来的那么多“隐世高人”——商业化了的可不只围棋,就算有这种人,没个十年八年也是撞不到的。不过张公子执意这样,众多龙套也只能认为该主角无愧儒商称号——有点迂腐和异想天开。

那这场比赛结果肯定可想而知了,第一盘“让车无敌”似乎还没习惯“突然多出个车”来,左车未出就被兵临城下了,再往后只能稀里糊涂的缴械。

这么拙劣的棋局自然会引起几个真正高手的不满,经过他们一散布消息,张公子只能到处微笑着解释:“让惯车了,让惯车了,不是还有两盘么?”

张太子自然知道张公子的想法,委婉的劝说刘钦:下盘能让就让让,第三盘往死里掐丫的,大家不就图个热闹么?

于是第二天刘钦就存了让让的心思,可实在没想到“让车无敌”居然敢祭起“仙人指路”这种难以驾御的布局,下意识的攻击几招,高人已经溃不成军了,就剩下抱着紫砂壶一口一口灌水的份了,这种情况下刘钦就算想让都不行了,他还得要脸呢。

观棋室里的观众们刚为有个精彩开头而感到高兴,没想到战斗还没开始就已经接近尾声了。

0:2输了还是小事,更拙劣的棋局出现才是大事,张公子一盘没赢已经不太高兴了,再想想那些象棋爱好者,不发火实在是没办法了。

因为不是一个圈子的,众太子没有为难张公子的意思,而是再次聚拢在张太子面前继续嘲笑,“丰亚,你这戏演的实在太蹩脚了,要不哥哥给你介绍俩好演员吧?女明星也成啊,不过哥哥认识的好象还没你认识的多呢。”“张哥,打假球的可比你敬业多了。”

张太子可是心情不爽了,居然赢了也高兴不起来,上次不舒服还可以作势拔枪,这次连发泄都找不见对象,张玉虎啊张玉虎,亏我看你还算个人物,拜托你演戏也演得敬业点好不好?于是张太子虽然是笑着接受张公子的恭贺,可眼睛深处却有点只能让一人感受到的微怒。他可没想到张公子宁可自损形象也万万不敢出点什么纰漏。

虽然比赛有点缺陷,但毕竟是结束了。张公子的助手钱海涛背着沉重黑锅和太子的手下定下了下场比赛的日期。

至于张公子和张太子,想的都是:既然戏演砸了,下回唱出大的好了,争取演好吧。

\section{第二十八章 外行看热闹}

今天就是比赛武术的日期了,张丰亚已经憋足了劲今天要好好的干上一场,虽然这三场比赛的结果已经早在谈判那天定下了。

因为上场比赛实在是太没面子了,赢得差点吐血,张太子回家稍微一琢磨,就反应过来了:张公子实在是太慎重了。

下棋虽然有名次排名,但是同等级里的高手相遇,谁也不能说就能稳赢,小河翻船的例子都多得很,既然张公子暗示了三场比赛一定要输给自己,那自然最有把握的就是找个谁也没听说过的貌似高手来滥竽充数。只不过这“芋”有点过于“滥”就是了,自己实在有点冤枉张公子了。

既然“演戏”的名声已经传出去了,这次说什么也要真刀实枪的大干一场,一来挽回点名声,二来也叫张公子见识见识太子的威力,让他明白太子不需要谁来让。

于是张丰亚放弃了使用手里军队资源的设想,托爷爷的同辈人物色了三个真正的武林高手,再加上手里的俩军中高手,打造出一支超豪华的队伍。

张玉虎也有类似想法,上次丢人丢得自损形象,可没办法啊:输赢只是默契的承诺,拿不到桌面上,串通的机会都不可能有,要扳回形象也只能这次大打一场了。希望张丰亚能有如此默契,别随便找俩人凑个队伍就想拿走那么大个项目。不过以张太子的聪明这次应该遍寻天下,精英尽出了。

按照这样的推理,张公子这里也只能全力以赴了。否则又是极其不对等的话,赢的人输的人面子上都不好看。赢的还好说,毕竟只要是个高手就能弄两下花架子满足观众的,输的……那不是自找没趣么?

因为有张公子的承诺,起码在赌局结束前,张太子短期内压制了大部分方面对宏达的打压——确实有些方面也不是张太子说开始就能开始说结束就能结束的。既然宏达这里压力骤然减轻,张公子就有闲暇从容的调度高手来参赛。不过为了保险起见,张公子还是动用家族的力量找来了两个世外高人。

不过,既然这场比赛的胜利者注定是属于张太子的,那张太子只需要安排最强阵容就可以了,而张公子则需要精打细算务求不出差错。所以张公子组建的队伍居然有八人之多。

人太多了,怎么办?那好说,张公子在前两场比赛中没有约几个人来观看,这次多约几个同为纨绔的前来助阵就好了,谁能知道里面有张公子多少人?再说虽然纨绔子弟的素质也是参差不齐,但是谁都会明白这种场合还是能结交不少值得结交的朋友的,能被邀请到这种档次的场面来插一脚凑凑趣,实在也是一大乐事。于是接受张公子邀请的纨绔也不着实少。

其实张公子心头还有另一翻心思:上两次都是张太子被人围住耻笑,虽说不关自己什么事,但太子脸上多少是有点不好看的。这次如果效果不是很理想的话,自然也会有类似纨绔们来笑话自己,那样就不会让张太子专“丑”于前了。两人成为难兄难弟的话,谁还会介意这点无伤大雅的玩笑?

于是一大早来到张家的人实在是不少,不光有打手队伍,还有张公子几个处得很好的知交,大家寒暄一翻就要动身了。

有这么多高手来到张家,楚云飞自然不可能不知道,虽然晚上熬夜学习睡得很晚,但是军营养成的作息习惯还是让他早早的起来了。

看到这么多高手,楚云飞自然要在旁边观察观察,却没想到被张公子一眼看到了。张公子想起来这家伙似乎也很能打,虽然肯定不能算高手,但听二叔说挨打的功夫不错,要不把他也叫上?“云飞,你有什么事没有?”

楚云飞可没想到张公子这么客气的问他,说实话张公子不记得楚云飞姓什么了,不过能记住名字效果不是更好?楞了一下楚云飞才说,“也没什么事,就是复习呗,时间太紧张了。”

张玉虎也知道他要考军校,“天天学习不闷么?跟我出去玩玩吧,花不了多长时间的。”

楚云飞还没说话张玉珊就接上话了,“也是,小楚来这么长时间还没怎么出去过呢,哥你带他去走走吧。”得,已经有人做主,那楚云飞也没什么选择了。

一行人浩浩荡荡的到了赛场:黑郁金香拳击俱乐部。

没过多久,张太子一干人等也来了,倒是一眼能看出来:后面除了常年跟随的俩之外,还有五位,自然就是参赛的了。他一眼看到张公子就是爽朗的一笑,“哈哈,老虎,你这次来的人可是不少啊,别叫做哥哥的失望啊。”

来的人确实不少,而且还有不把场子挤满不罢休的架势。其实这个很容易理解:赛马,那是很高雅的西方玩意儿,爱好者实在说不上多;至于象棋,确实有点文了,造诣高的人也没几个,还有各种比赛可看;武术可不同了,基本上带点血性的男儿谁不喜欢看?再说,那都是货真价实的高手,可不是一般什么武术表演或者散打比赛可以比的。

规矩定的很简单:赤手相搏,穿布鞋,全身不能有铁器,兵器之类就更不允许有了。输赢判断参照拳击规则:出台者和倒地十数内不起者判输。

第一场自然是谁都无须考虑给谁留面子,于是张公子和张太子都派出了临时请来的高手,“先声夺人”那是必须的。

所以在观众的眼中,第一场是最没意思的,根本看不到什么精彩的招式,同为“大极”派的俩高手居然只是在一起推了推手,没见谁倒地也没见谁出台,战斗就结束了,张公子请来的“吴氏太极”的高手王大鹏获胜。

第二场自然也是无需什么顾忌,张太子这里还是世外高人,张公子那里可就是早些时候网罗来的好手。但是两人的差距并不是很明显,在观众的眼中也精彩了很多,战了多个回合,张公子这方举手认输。1:1第三和第四场双方出场都是平时网罗到的人才,虽然比斗质量不是特别的高,却是非常的好看,观众们看得更是大呼精彩。尤其张太子这里的两军中高手,不只多会些既漂亮又实用的武功招数,还都练过气,虽然局面上不太占优势,可两人非常的经打。第三场张太子方的高手就这么凭着不屈的斗志活生生把对方磨败了。

四场下来,2:2平。

\section{第二十九章 高手是废人}

形式明朗化了,张公子只能输不能赢了。

双方都很有默契的留了个最强的高手压阵,张太子方是一个40多岁的中年瘦子,碰巧的是王大鹏认识他,“陇西的三才派高手‘废人关’,这家伙厉害。”

张公子这方要出场的是蜀山剑门的严向东,一听到这话楞了下,“什么?他是‘废人关’?”

张公子很好奇,“怎么叫这么个名字?他不是很厉害么?”

严向东脸泛潮红,似乎有些激动,“厉害,这家伙名气很大,听说他手提不起三十斤的东西,所以大家说他是废人,可他散步的时候,能把靠近他的两百多斤的石头弹飞,你说厉害不?”

张公子肯定是听不明白这话是什么意思,但心可就放下了,“没什么,严叔,你尽力好了,小心别受什么伤就好。”

严向东脸更红了,不过这次是因为生气,“受伤?就凭他?要是有剑在手,他根本不够看的。我空手他也未必能捞到好处。”看来“武无第二”这话还是有一定道理的。

听说自己能赢,张公子的头“嗡”的就大了,不是吧?家里都给自己找了些什么人啊?想到这里,张公子都有点结巴了,“严、严叔,他打不过你?”

严向东以为张公子在为自己担心,“嗯,我有剑的话他绝对不行,不过只比拳脚和气功的话,胜负该是六四分,我六他四。”

按理说严向东说话可能有一定的水分,不过就算把水分抛去的话,这场比赛的前景似乎也不是很乐观,张公子有点头疼了,可面对高人,也不方便要人家让呀。

思来想去,张公子一咬牙还是小声说了实话,“严叔,不瞒你说,这场比赛咱……不能赢。”

严向东紧紧盯着张公子,过了半天才长叹口气,“玉虎,不是做叔叔的不帮你,实在是我和他比输不起呀!你们有你们的圈子,我们也有我们的圈子,……要不你换个人上吧,唉。”说完口气中竟是那么的遗憾。有高手在前竟然不能切磋两下,对武人来说确实是比较痛苦的事情。

张公子也比较痛苦,看来只能换人上了,还好后备选手也有,换谁呢?左边看看,右边看看,到底该换谁上呢?

赵一飞么?这家伙是野路子出身,都说他下手阴,会不会不太好看?陶永盛?身法倒是不错,不过似乎不太经打;蒋元?年纪是不是有点大?云飞?太单薄了,也不耐打,——等等,他好象很耐打?

想来想去,张公子冲楚云飞招招手,“云飞,你来,……这场比赛你上怎么样?”

楚云飞楞了一下,让我上?思索一下,明白了,小声问了句:“我上是没问题,输快点还是输慢点?要挺多长时间?”

张玉虎没想到楚云飞一下就弄明白他的意思了,至于时间?那自然要照顾观众的情绪,“时间越长越好。”

越长越好?说话当然容易了,你去试试?楚云飞很不情愿听到这个答案,“那……虎哥,我可是还要考试的。”

“考试?”张玉虎沉吟一下,“没问题,你事我包了。”估计这许诺里还包含这次的治疗费吧?

话说到这个地步,楚云飞也没什么好说的了,到时候他兑不了现,大不了逼着他弄张去沙特的护照吧。

双方往场上一站,门户一亮,彼此心里就有数了:不是一个数量级的。面对这种情况,楚云飞只能把得失的心态全部放下,什么都不去想,尽力使自己达到一种“空灵”的状态来减缓对方气势带来的压力。

他的这点变化对方立刻就捕捉到了,废人关对着他微笑着摇摇头,意思很明显:小伙子,没用的。然后就真的象散步一样慢慢向楚云飞踱来。

欺人太甚!楚云飞真的有点进入状态了,不顾重重压力,冲上去左腿就是一记横扫。

对方却只是微微一动,这脚的力气就不知道去了哪里,扫是扫住对方了,可就象清风掠过一样,没一丝力气,而且事情还没完,一股奇大的劲道顺着左腿送了回来,等到楚云飞感觉到的时候根本来不及反应了,在跌出去之前只有一个念头:这才是真正的“借力打力”么?

虽然劲道不是很足,但楚云飞还是摔出去有两米多远,虽没受到有什么严重的伤害,可战士心头还是泛起丝丝悲哀:这,才是我真正的实力?

楚云飞一个鱼跃,跳了起来,活动活动左腿,有些酸麻,髋关节更是有点涨涨的感觉。不过这点轻微的痛楚反而更加速了楚云飞大脑皮层的活动:攻击四肢看来效果不大,躯干部分似乎也就侧面效果好点,不过头部可能是个不错的选择。于是右腿前跨,左腿佯动,转小跳步左拳直击对方肩部,不等招式用老,右腿飞起直踢头部。好漂亮的一组动作!

可有时候仅有漂亮是不够的,楚云飞眼睁睁看着对方用肩部卸掉这一拳,因为没反作用力的支撑,右脚力度就有点欠缺,而对方的头部似乎确实是个有机可趁的所在,只可惜对方左手一揽,又不知道怎么一推,自己又不得不打着滚飞出。

看来对方在“力”的运用上已经相当高明了,楚云飞意识到这个问题后,放慢节奏,轻拍两掌出去试探,可对方反击回来的力道凶猛依旧。

废人关已经收到张太子的叮嘱:如果对手不是很强劲,可以适当放放水多来两个回合,增强比赛的娱乐性。虽然废人关的自身情况不允许长时间交手,但对这种档次的对手的话,他确实可以放出相当的水分。

于是观众们就有眼福了,大家看到一个瘦高的小伙子在不断的绕着一个同样瘦的中年人在不停的进攻,花样百出,非常的好看,但那中年人有如那坚实的钱塘大堤,任潮水汹涌而至,惊涛拍岸,却始终巍然不动。这就是“大巧若拙”了吧?

张太子的人那是不用说了,张公子的人也是看得不住的摇头,唉,没办法,差距确实太大了,还好这个小伙子真耐打,换个人早该撑不住了,不过换个人是不是场面不会这么难看?

时间慢慢的推移,打斗中的二人也渐渐的支持不住了,楚云飞实在是被摔的全身发麻,也没多少力气了,琢磨着这时间该差不多了吧?那就最后来一下吧。而废人关因为自身条件的原因,也不能再这么持续的耗费体力了,也想着是该给他下狠的了吧?

于是等楚云飞再次爬起的时候,双方不约而同的使出了狠招,楚云飞一个箭步上去,竟不是普通的武功招数,左手叼对方右手,右手也跟了过去,同时右腿抬起一个狠狠的膝撞,纯粹是不计后果拼命的招式。而废人关则是一如平常的后发制人,右手滑出对方擒拿,左手顺势一拨,右手狠狠一掌击在了了楚云飞的右后肩上,楚云飞直接的就飞出了台子,重重摔倒在地。

\section{第三十章 貌似高人}

“要结束了么?”恍惚中,楚云飞居然又品味到了那似曾相识的感觉,全身泛起一种轻飘飘的感觉,身体似乎已经不属于自己,可又偏偏能感受到身体内汹涌沸腾的内气,充盈到喷薄欲出,隐隐然觉得两肋凉风习习,头脑中也越发的空灵,竟有一种想要飞翔的感觉。

久违了,这种想要御风起舞的感受,楚云飞没想到自己在被百般摧残之后竟能再次登临这种境界,这不是人们常说的“潜能”,不是的,绝对不是,没有什么潜能能够给人如此清晰的感受,触摸得着,把握得住的感受。

真的不想让这种感觉在瞬间溜走,楚云飞小心翼翼的品味着,享受着,这次能不能从里面找到些什么规律?

谨慎地保持着这种状态,楚云飞慢慢的来到台上。虽然他可以确定自己是走上去的,但是竟然没有日常走动时的那种步伐频率,恍惚间竟有种自己是飘上台去的错觉。

楚云飞抬眼看去,却见废人关一脸惊骇的神色,嘴里还不停的嘟囔着什么,虽然声音很小,在观众如潮的喊声中却被楚云飞听得清清楚楚,“先天境界?会是先天境界?”

这个状态叫先天境界?那是什么意思?楚云飞正在考虑这个问题,却见裁判从一旁走来,把废人关的右手举起向周围示意,废人关神色也恢复正常微微向四周点头。

楚云飞这才想起:自己飞出台了,比赛结束了。心神微微悸动之下,头脑中不再空灵,那种令人痴迷的感受终于又如潮水般退去,不再出现。

奇怪的是,在楚云飞恢复正常的时候,废人关再次递来个意味深长的眼神。

比赛结束了,张太子一方获得了武术比赛的胜利,更难得的是,双方为在场的观众们献上了非常精彩、值得回味的一出争斗。

可这一切的一切,在张太子来看,可以品味的内容就太多了。自然前四场的比斗没什么可以琢磨的地方,最后一场的味道才古怪。

首先是蜀山高手严向东的问题,张太子这边也有人认出了这位高手,为了自己占据主动,张太子安排废人关先出场,意思就是告诉对方:你看着办吧。而对方显然也够谨慎,怕严向东这位能和废人关打成胜负五五开(引自废人关)的高手出什么闪失,终于没把这可怕的主安排上来。

可是另安排的这年轻人就更厉害得离谱了,居然可能已经达到无数武学大家梦寐以求的“先天境界”,而且居然还是这么的年轻!

然后这个年轻人采用相当笨拙和古怪的方式不停的被废人关击倒,这里面的良苦用心大家就都知道了,那比赛结束后年轻人终于显露高手真容的意思也就更明显了:我张玉虎确实遵守了许诺,你可千万别以为是真的比不过你。

想到这里,张太子摇头微笑:呵呵,终归是年轻气盛呀,有必要做得这么露骨么?光看那年轻人打不死的劲头,是个人就知道那是不露相的真人!!!

看着张太子笑容满面的向自己走来,张公子微笑着迎了上去,“还是丰亚哥厉害,小弟是服了。”

张太子回来的话也很中听,“哈哈,哪里哪里,做哥哥的才真是承让呢。”其中“承让”二字里的真诚谁也体会得到。

张公子终于轻松了,尘埃落定,“哥你可太客气了。”

张太子爽朗一笑,“哈哈,好了,过几天来哥哥家里坐坐,给老爷子引见引见我的好兄弟,对了,你这里最后出来的那位是?”

张公子楞了一下,这个是什么意思?不过他还是微笑着回答,“哦,那是临时喊来的,小妹的保镖,本来是那个中年人,严叔上场的,不过他突然有点不舒服,呵呵。”

这话听到张太子耳朵里,味道可就太多了:讨好?卖弄?示威?似乎都有那么一点点。还好两人已经真正说开,张太子也不再介意,“真是自古英雄出少年啊。”赞赏之意表露无疑。

张公子也是一楞,是在夸我么?估计不是,可……会是在说楚云飞?不可能吧???

于是张公子含混的一记马屁拍了过去,肯定是严丝合缝,“哪里哪里,象丰亚哥这样才是真正的人中龙凤啊。”——管你在夸谁呢。

同时哈哈一笑,两俊杰顿时仇泯怨消,取而代之的是惺惺相惜。

等到张公子回到自己这方,众纨绔一拥而上,少许嘲弄那是少不了的,更多的却是无谓的寒暄和套近乎。

只是宏达集团的名声就已经足以使众纨绔趋之若骛了,再加上这场比赛,是个人就知道张公子又结交下了众多太子,大家同在京城,相互间多点照应那绝对是应该的。

寒暄过后,张公子登车之际,却被严向东拽到一边,“刚才他们是不是认出我来了?”

张公子楞了一下,“他们没说,不过没准。”

“哦,”严向东有点遗憾,“那就算了,我还说认出我来的话,我和废人关再定上一场切磋呢。”

真搞不懂这些武人,把切磋看得那么重要?张公子笑笑,“那估计得过些日子了,现在要是你们说点什么,咱们可就成了打假球的了。”

严向东也早就想到了,黯然的点点头,忽然又想起什么,“对了,那个最后上场的什么飞,他最后的样子很奇怪,不过离得远了点,我说不出来那是怎么回事。”

比赛结束的第二天,楚云飞一大早就醒了,他终于明白了什么叫“痛不欲生”,全身酸疼,没有一处舒服的地方,而且不是那种锻炼过度的肌肉酸痛,而是深入骨髓甚至是内脏的疼。抬抬手说起床吧,却是连抬胳膊的动作都差点没能实现。

会是怎么回事?楚云飞费尽力气边穿衣服边琢磨,想来想去,想起了团长师傅和他讲过的“阴劲”,所谓阴劲其实也是内气使用的一种效果,在对手体内存留时间极长。一般只有极不对等的比斗中才能产生这样的后果。楚云飞不由得摇摇头,唉,烦恼只因强出头,不过,自己推得掉么?

没练过气的人只能等阴劲自然消去,一般时间是5天到15天不等,要是有那20天以上的,绝对是下手的人存心要取对方命。不过练气的人可以通过自身内气调理来把这股外来的气炼化。

于是楚云飞也只能老老实实的每天打坐练气,本来他还以为每天打坐上三两个小时就差不多了,没想到废人关的内气太厉害,每天不打坐12小时以上根本没什么明显效果,只能暂时把学习的事放下了。不过他也想通了,实在不能怪废人关下手狠,而是自己做事太勉强了,又没什么经验,那肯定是“强极则辱”了。

\section{第三十一章 关涛的烦恼}

废人关本名关涛,其父就是三才派的高手,可谓家学渊源。他从小身体条件就不是很好,虽然没什么大灾大病的,体质却是比不上别人。他天性很高,本身也喜欢习武,又不愿丢了家里的名声,于是痛下苦功,终于取长补短练成一套极具特点的功夫。以此在武林中闯下很大的名头。

可他的这套功夫不但要善用全身各处的肌肉,而且内气也需要能通畅的运至全身以做支持,所以对内气的精纯和浑厚方面的需求也远超其他人。也正因为是这样,楚云飞才吃尽了苦头。

现在的废人关内气修炼又到了一个瓶颈处,因为他的内气造诣本来就已经非常高深,在这个境界上再加以突破是相当不容易的。以他的能力,这辈子再难有所寸进也是正常的,于是就有了废人关遍寻天下寻找对手。可他找来找去,虽然也找到了几个对手,不过对方总是敝帚自珍,终于遇到个高手愿意指点,但很遗憾:修炼的方式并不适用于他。这个高手感叹现今武林门派驳杂,精髓缺失之余,给关涛指点了个方向,如果真想提升境界,只有找到修为臻达“先天境界”的高手,才能对他进行无视修炼方式的点拨,而且对那种境界的人来说,根本不存在门户之见一说了,因为他把你自身的东西帮忙修整下就可以了。

可找这种高人那就不是一般的费事了,有没有这种存在都是一个问题,指点废人关的高人也就知道似乎有两人可能已经达到这种境界,其中一人还是该高人见过,从那人外放的气势上判断出来的。

“先天境界”废人关自然是知道的,几乎所有门派的书中对这种境界都有详细的阐述,可也仅仅是阐述,现实中没几个人说得清楚那到底是一种什么样的境界。

于是,当废人关惊奇的发现居然有个年轻人能发散出符合门中书上所记载的“先天境界”种种征象,他没当场欣喜若狂,完全是仰仗了他那深厚的练气功底。

后来楚云飞对“先天境界”不着痕迹的收放自如更是让废人关心痒难耐了,要不是那种场合那种气氛,没准关涛当下就要向楚云飞请教了。

楚云飞从打坐中悠悠的回过神来,一眼就看到了坐在他面前的周琳琳,小美女正捧着一本参考资料发呆呢,发呆就发呆吧,怎么脸上还挂着那么一种暧昧的微笑?再看看桌上小闹钟,已经下午六点了。

“大姐,来了?怎么这么晚还不回家?”楚云飞一嗓子就把正走神的周琳琳喊了回来。

“要死了你,都告诉你不许这么喊了。”周琳琳反应过来,“怎么这两天一直在练功?伤得重么?”

“呵呵,”楚云飞笑笑,“重倒是不重,不过有内伤,不调理一下会很麻烦的。”

“哦?”周琳琳紧张起来,“要不要吃药?我……家好药可多呢。”

“这个……”楚云飞皱着眉头考虑半天,“现在这情况,没印象说该吃什么药,应该不用吧?”

周琳琳似乎想凑得近些仔细看看,可又停住了,“又是这个张丰亚,要不要我帮你收拾一下打你的那个人?”

楚云飞咧着嘴笑了,“咝~~,不用了,光明正大的输了,还找人家什么麻烦?”

两人正聊着,曹婶敲敲门进来了,“小楚,外面有个姓关的找你,说你认识他。”

姓关的?还是我认识的?楚云飞真想不起有这么个人来,扭头对周琳琳说,“你去找张小姐玩吧,我去看看。”

出门一看,这姓关的果然认识,不过这两天楚云飞都是咬牙切齿的咒他是“姓费的”,不是废人关又是谁?

废人关见到楚云飞,欣喜之情溢于言表,隔着老远就招呼上了,“楚师傅,在下关涛看你来了。”废人关在这称呼上很是琢磨了一阵,叫“楚兄弟”的话,不但冒昧,两人年纪差异也有点大;叫“小楚”自己又有点托大,叫“楚前辈”吧,虽然对方确实厉害,可年纪太小了点;叫“云飞”两人关系可还没到这份,又是刚恶斗过一场。于是依着江湖规矩叫了声“楚师傅”

虽然楚云飞是个不太爱计较的人,也明白这次比斗错不在废人关身上,可随便给谁“痛不欲生”的煎熬上几天,又见元凶出现的时候也难免有点抱怨的心思。

于是楚云飞一副公事公办的样子,“哦,关师傅,不知道这次来有什么见教?”

废人关还以为楚云飞依旧对那场比赛耿耿于怀,不过这也能理解,观看的人那么多,小伙子还要故意输给自己,心里自然憋屈。换了自己肯定也是一肚子不自在,他可没想到楚云飞是因为事后阴劲发作疼痛难忍的缘故,一拱手,“前几天关某得罪了,还望楚师傅万万不要见怪。”

这种局面楚云飞也不能再计较了,微微一笑,“关师傅不用客气,大家各为其主,都能互相理解的。不知道关师傅今天找我来有什么事情?”

废人关四下看看,楚云飞自然明白,“怠慢了,怠慢了,屋里请。”

把废人关领进了房间,楚云飞又喊曹婶拿来壶茶水来,两人才正式开始交流。

废人关说话还是很直接的,“楚师傅年纪轻轻,身手不凡,不知道师出何门?”

看上我的资质了?要收我做徒弟?楚云飞书看得多难免胡思乱想了一下,“说来惭愧,我实在是没什么师傅,自己胡乱练的,哪里算得上什么身手不凡?还有,关师傅可不敢这么叫,我担当不起,不嫌弃的话,叫我小楚好了。”

“哦,”废人关心里微微的有点失望,年轻人心胸还是不够开阔呀,不就是小小的一场比赛么,这次怕是又要空手而回了,“那就叫你小楚了,敢问你练气有多少年了?”

“练气?”楚云飞算算日子,“有十一、二个月了吧。”

废人关小时候是被人看不起,可他成名以后还真没见过有人如此的轻视他,敢在他面前这样的胡言乱语,当下脸一沉,“楚师傅,我是敬你的修为,但是做人还是厚重些的好,过分轻薄对修炼可实在不利的。”

楚云飞自然是满头的雾水,但马上反应了过来,“哦,你是说我那个‘先天境界’吧?”

这么直接了当的话,废人关的心思马上被勾起,那点小小的芥蒂也就不太在意了,“对,就是它,想你练气时间这么短怎么可能达到‘先天境界’的可能?”

楚云飞坐着沉思了半天,想想张公子既然和张太子已经化解了怨气,实话实说想来也没事,组织了下语言,缓缓的解释,“其实我也不明白那到底是种什么样的境界,我也是听你说才知道那叫‘先天境界’的。不过那是误打误撞出来的。”

废人关居然还能稳稳的坐在那里忍受这匪夷所思的话,不能不让人佩服他的养气功夫,可他的话里明白无误的表示出了怀疑:“练气十个月误打误撞到‘先天境界’?小楚你的命不是一般的好啊。”

楚云飞说了实话还被这样的怀疑,先前的那点怨恨不由得又被勾了起来,“‘先天境界’很厉害么?我还不是让你的阴劲折腾得好几天没睡好了?”

\section{第三十二章 闲话江湖}

废人关更摸不着头脑了,“你……你中了我的阴劲?”

“这个,”楚云飞其实也只是猜测而已,对练气这行他实在是不够精通的,“我觉得是阴劲,因为咱俩之间差距太大,不过到底是不是我也拿不准。”

“不可能吧?”废人关的心情已经不能简单的用惊讶来形容了,“我帮你看看行么?”

“看吧,”楚云飞把手伸了出来,废人关一把抓过他的手腕,疼的楚云飞直喊,“轻点轻点。”

废人关把了半天脉,眉毛都快拧到一起了,终于在沉寂半天后喃喃自语,“不是‘先天境界’,不是‘先天境界’……”

确实不是“先天境界”,楚云飞体内各脉将通未通,脉络也不是很宽阔,瓶颈众多,还积滞了好多废人关存留下的内气,唯一值得称道的也就是比同龄的练气者似乎脉络宽了点而已。

楚云飞好不容易遇到个高手愿意和他探讨下自身情况,自然要问个明白,“不是?那是怎么回事?”

废人关的希望本已破灭,却又被楚云飞这一问提起了兴趣,“是啊,那又是怎么回事?你再把那天的气势运出来我看看。”

楚云飞只能苦笑,“我平常也运不出那种气势,只在偶然间能达到这个效果,再说我现在实在是全身疼痛啊。”

废人关不好意思的摸摸脖子,“我倒是把这茬事给忘了,来,我给你敲打敲打,你配合着平时的方式运气就行,被打断也不要紧,继续运气就行。”说罢,真的在楚云飞身上敲打了起来。

这么敲打了有十来分钟,楚云飞越来越放松,又有点感觉要进入那种“伪先天境界”了,于是继续细细品味,空明身心,终于在废人关一次敲打中,“轰”的进入了那种状态。

每次都是来得那么突然,这次也不例外,楚云飞还真没感觉出这种状态是怎么来的,只是隐约地感受到不是单纯的气出丹田,似乎这些气一直在全身各处隐藏着,突然间就那么爆发了,丹田处不过藏得多了些而已。而且也许是因为太突然,脑袋没反应过来,楚云飞甚至觉得头部的气比全身的气加起来还要雄厚精纯得多。

感觉到废人关已经停止了敲打,楚云飞慢慢张开眼睛,却见废人关又是那么一脸惊讶,人在房门口站着,他坐的椅子和桌上的茶杯却已经都倒在了地上。

废人关终于有了反应,“对,对,对,就是这种状态,你保持住,我……”

话还没说完,曹婶推门进来了,“小楚你怎么了?丁零当啷的?”

这句话就象一针扎在了充满气的气球上,姓楚的气球马上就泄气了,还不能埋怨别人,“没事,曹婶,我们研究点功夫。”

曹婶看看楚云飞,又扭头看看废人关,摇摇头,叹口气,“你们小心点儿啊,别再弄出这么大动静。”说完掉头出去了。

废人关可没有被训斥的懊恼,满脸的兴奋,“你怎么做到的?这就是‘先天境界’啊,直接就能把我弹出去那么远。”

长叹一口气,楚云飞愣了半天,然后摇摇头,“不知道,就是你刚才敲打我的时候,突然来了感觉。”说完还伸伸胳膊踢踢腿,“不错,现在感觉好多了。”

“继续,”废人关也没多说,走上前去又开始敲打,楚云飞也只好配合。

这次就没那么灵了,废人关一直敲打了半个小时,楚云飞也没什么反应,只把个废人累得脸刷白。

“今天不行了,改天吧。”废人关终于明智地停止了无谓的工作。

楚云飞站起来走走,随便晃动晃动身体,居然感觉好了个八九分,“不错,关前辈确实功力深厚,你这随便推拿两下,居然没什么事了。”

废人关看看桌上闹钟,“时间不早了,出去吃饭吧,算我赔礼了。”他真舍不得就这么放过楚云飞。

楚云飞自然不肯接受,这算什么事啊?结果二人你争我让,声音难免就又大了点,等到曹婶推门进来,楚云飞立刻就下了决定,“好好好,关师傅,这顿就算你的了,下顿是我的。”

二人找个酒店,还特意要了个包间坐下,按道理两人并不怎么熟络,又都是练武之人,所以话题自然是离不开功夫。

一来二去,双方彼此的底细就知道得差不多了,于是两人又说到了共同关心的“先天境界”上。

虽然是练武的人,但废人关不喝酒,没被这些闲话把思考的中心转移,所以一直在琢磨楚云飞这种奇怪的状态“小楚,你能不能把你那本书,随便跟我说说?里面到底写了些什么?”

买单的不喝酒,楚云飞自然也不方便喝,可有菜无酒气氛似乎不太热闹,就点了瓶啤酒,一边品着啤酒一边回答,“其实也是本很普通的书,我师傅说是练气入门,比较正宗就是了,可我练的比较邪门,内气没通就通外气了,师傅说和他练的效果不一样。”

“你师傅?”废人关奇怪了,“你不是没师傅么?”

“我说我没师门,”楚云飞耐心解释,“不是没师傅,虽然也算不上真正的师傅,可在练气上指点过我的。”

于是楚云飞又把他和耿风在练气上探讨过的东西,又拿出来说了半天。

当废人关听到“没准最合适修炼某个门派功法的只是他的创始人”这论调的时候不住地点头,深有一种“于我心有戚戚然”的感受,是啊,这么多年来,自己不是走出了条自己的路?可要说谁都能这么练,那怎么可能?起码是达不到自己这种高度的。

有了这么通领悟,废人关觉得自己没白来找这个小伙一趟,“好了,小楚,我也决定了,继续混迹江湖,遍访高人,力争百尺竿头,再进一步。”

楚云飞听了这话,不由得羡慕起眼前这个人来,好逍遥啊,不过有个问题还是要问的,“那你这钱从哪来呀?打秋风么?”

废人关实在有点哭笑不得,“你这都说的什么呀?小说看多了吧?实际上,随便到哪里访个人,大家交流交流,我再指点一下他的门下弟子,自然就有人送上钱来,学武的哪里有穷人?像我这种身份,只会有人怕我不指点,不会有人学了舍不得出钱。”停顿一下,“再说,如果有人邀请管点什么闲事——只要不违背江湖道义,那钱就来得更快了。”

\section{第三十三章 严重不对等}

接下来的日子里,楚云飞又和废人关接触了几次,最后废人关离开京城的时候,楚云飞也混在张太子送行的队伍里尽了份后辈的心意。因为废人关并没有把自己走眼的事情说出去,所以张太子看到楚云飞难免又有点想入非非,但想想已经和张公子结为至交,资源是可以共享的,于是也没去理会这个年轻人。

周琳琳去张家是越发的勤快了。由于宏达集团的烦恼期已过,张家的人也都没那么忙了,张公子都撞到了几次周小姐的来访。

张公子很奇怪这小丫头不安心学习,不过鉴于小丫头已经在向大姑娘过渡了,也不好多张嘴。

终于到了楚云飞打背包回去考试的时候了,晚上张志华特意备酒席算是饯别,张玉珊说身体不舒服没去,周琳琳倒是参加了。

第二天,楚云飞一大早起来收拾好了行装,看看没什么事了,就在院子里打拳,拳打到一半,却被张玉珊叫进了屋。

“这阵子辛苦你了,又是帮我操心,又是帮我哥操心,还得辅导琳琳。”张玉珊轻抿一口果汁,一如往常的优雅。

“呵呵,”楚云飞笑笑,他真有点受不了张玉珊说话,你想说什么就直说吧,别总是这么扭扭捏捏的,“没什么,其实我挺空闲的,整天吃了睡睡了吃的。”既然你不说是什么事,我坚决不问,看你以后再这么说话,楚云飞带点恶意的想。

“这次回去,考上军校以后有什么打算?”张玉珊也知道她哥哥给对方承诺了点东西。

就这么一来二去的,两人没什么内容的话说了半个多小时,楚云飞有几次已经受不了啦,不过想想正可以通过这来修炼自己的养气功夫,居然又活生生的忍住了,到最后终于磨得张玉珊憋不住了。

“琳琳给你留她家电话和地址了吧?”

“是啊,”楚云飞回答的时候居然有点心虚,“她要我保持联系,有什么事也好交流。”

“是么?”张玉珊盯着楚云飞,“你俩除去功课还有什么交流么?”

“嗯,”看来情况不太对,楚云飞若无其事的笑笑,“朋友么,有高兴的事说说,有烦心的事也可以聊聊。”

“就这么简单?”张玉珊笑了笑,“琳琳可是我看着长大,看看你房间里的东西,她从来没对普通朋友这么大方过。”

“这……”楚云飞呆住了,其实这个问题他早就想过了,可是每次都是强迫自己不去想下去。不是他不喜欢周琳琳,而是觉得有点害怕,至于在怕什么他也不知道,是怕打破这份朦胧以后面对的尴尬?是怕周琳琳只是一时的热情而自己自作多情被拒绝?还是怕自己从此沉溺于其中,淡忘了心中的仇恨而导致终生都有个遗憾?反正他总觉得目前这个样子保持下去就不错。

可张玉珊这话也提醒了他,首先可以肯定周琳琳确实是对他有意思的,不是自己的错觉;再有就是,两人从此就要天各一方,有必要认真的考虑一下这个问题,否则就是真对不住那小美人如水的情意了。

想归想,楚云飞可没有马上承认的打算,是想再从张玉珊那里再套些话出来么?“这个,怎么说呢?我好歹也是放弃了自己的复习时间辅导她啊,再说,她也就是借给我,昨天都说了等我走了她就把东西拿走。”

“哼,”张玉珊很罕见的从鼻子里发出一声笑,“你是个明白人,知道我在跟你说什么,你觉得装那个糊涂有必要么?”

“这个……”楚云飞实在有点受不了张玉珊的咄咄逼人,“你说的这个问题我还真没仔细考虑过,等我回家好好想想吧。”

“是啊,”张玉珊长叹一声,好象是有什么感触,“你是该好好想想,认真的想想。”

楚云飞是什么人?这话哪里会听不明白?“玉珊姐你好象有点什么想法?方便的话说说成不?”

“好吧,”张玉珊等的就是这句话,“我早就觉得该和你认真说说了,我先问你个问题,你觉得,你俩……可能么?”

“可不可能这事谁也不能担保,”不知道为什么,楚云飞一说到感情方面就习惯性的犯迷糊,“相互了解总要有个过程的。”

这都什么和什么啊?张玉珊自认作为个作家,言语的条理性和清晰性还是有的,看对方如此回答,只好再解释清楚点,“抛开你俩的感情因素不说,你觉得你和琳琳最终能走到一起么?”

这不是废话么?楚云飞不吭声,盯着张玉珊等她的解释。

“唉,”张玉珊见对方如此冥顽不灵,摇摇头,“身份、地位,从这两方面讲,你不觉得你俩差得太多么?”

听着如此不留情面的话,楚云飞闭住眼睛,考虑了一下,张开眼睛坚定的回答,“我不认为这会是问题,关键还是得看两人的感情,有感情什么都好说,没感情说什么都没用。”

“哼,”张玉珊又从鼻子里发出一声笑,“感情?感情这东西又算得上什么东西?你以为琳琳的家人会同意你俩在一起么?不可能的,因为你没那个资格,没那个地位!”

楚云飞是绝对不吃这一套的,但他也只是笑了笑,“我相信这个世界还是有真正的感情存在的,也许是我年轻,但是我真的相信这世界上肯定还有些东西是真实的。”

张玉珊长叹一声,又轻轻摇了摇头,“我知道你会这么说的,其实我早该阻止琳琳和你接触下去的,可我又不忍心,不忍心看着琳琳不开心,女人,有的时候心确实太软。”

“不过,有一点你说对了,你确实是年轻,年轻到甚至想不到这个社会究竟有多么残酷。真情,我也不得不承认它可能是存在的,但是它只存在在对等的个体或者群体中,你明白么?”

“你俩的差距太大了,严重不对等的双方,怎么可能有什么结果?”

楚云飞知道张玉珊是作家,但是没想到这个气质雍容的小姐居然这么能说,而且渐渐的抛开了优雅的外套,露出了作家该有的锐利。不由得疑惑是不是每个人都必须这么伪装,他们不累么?

“我知道你是个很聪明的人,但是有些东西不是只靠聪明就能想到的,就算周家同意你俩来往,同意你俩结婚,他们难道不用考虑同一个圈子里的人会怎么看待、怎么评论这事么?他们不是生活在真空里的!”

楚云飞有点不耐烦了,“张小姐,你能告诉为什么和我说这个么?我俩好象还没什么呢。”

张玉珊笑了笑,稳定了一下心情,“为什么和你说?我还后悔没早说呢,象这种注定没有结果的感情,本来就该扼杀在摇篮里的。你俩都是很优秀的,实在不该在这件事上浪费你们青春。”

楚云飞实在有点想骂人的冲动,不过他内心深处也觉得张玉珊的话不是危言耸听,人毕竟不是生活在理想社会里的,“我没有这么做,怎么会知道结果一定是这样?不过,我不会因为别人的劝告就贸然放弃可能对我很珍贵的东西的。我会努力的去争取的。”

张玉珊没有生气,而是很平静的看着楚云飞,“我没有别的意思,”叹口气,又组织了一下语言继续说,“只是把将来的可能告诉你了,当然我也希望你能珍惜你该珍惜的东西,也希望琳琳能一直快乐下去,不过……成长总是要付出代价的。”

“如果硬要说我有什么意思的话,可能只有那么两点,一个是将来你想到我说的话的时候,仔细思考一下,可能就不会那么痛苦了;当然我更希望在这短短的几年里,你能有大的突破,如果你俩的差距能被缩小很多,那可能还来得及。总之一句话,愿你好自为之。”

楚云飞仔细的咀嚼着这些话,字里行间品不出任何的恶意,不过这并不代表他就认同了对方,“谢谢玉珊姐,我会认真对待你的建议的,真的非常感谢。”

就这样,楚云飞莫名其妙的就多了份牵挂,又增加了一份偌大的压力,可怜的小伙子背负着这自己找来的麻烦回家见久别的母亲去了。当然,肯定是要从首都给母亲捎点东西回去的。

\section{第三十四章 人算和天算}

楚云飞在本省当兵,考试就是在先阳市进行的,虽然这个城市留给了他不快的回忆,但毕竟是家乡,负面情绪被“游子回乡”的感觉抵消了不少。

于是虽然复习时间很短,楚云飞还是没有麻烦到张玉虎,凭借自己的力量考上了建业军事学院特种作战系。

建业是九朝古都,位于长江之畔,民风淳朴。得天独厚的地理位置,冠绝天下的钟灵之气,更滋润出一代代的如水佳人,养育出一辈辈的绝世天骄。

楚云飞是北方人,初来时有点不太适应这里潮湿的气候,对这里淮扬口味的饮食也有点不太能接受,不过日子一长,也就习惯了,后来吃饭的时候要是没点甜味还真觉得嗓子有点痒。

楚云飞入学的成绩不是很好,刚开始的时候没有什么人注意到他,在准军事化管理的学院中只是一名普通的小兵。但在新生进行的军训中,他就以优秀的表现获得教官的高度赞扬。

当同学们都把这个瘦瘦的家伙定义为“头脑简单,四肢将就”的彪悍型的军人时,接下来的文化课学习楚云飞又表现出了他那超乎常人的水准,于是随着时间的推移,他还是很正常的锋芒毕露了。

楚云飞所在系的学生基本上是以现役中的军人为主,能考进军校的军人一般还是很有点能力的,而且也有些军人世家的子弟通过正常或者非正常手段进来的,用“藏龙卧虎”这个词来形容这个系是不算夸张的。

一直以来,都有相当数量的好岗位提供给这个系的毕业生,所以这个系的竞争也是相当激烈的,当然也有那些不怎么求上进的主,反正只要毕业那铁饭碗是基本上可以保证了。

以楚云飞的能力和表现,他很正常的进入了“第一方阵”,虽然头上还有大二和大三的战友的压力,但军队里就是这样,只认能力。在学校能力的表现范围太小,只能用学习成绩和训练成绩做标准来衡量。

出乎楚云飞意料的是:纯以能力而言,同在“第一方阵”的居然还有两个出身军人世家的子弟。两人的所有成绩都不比他逊色多少,可要是再加上背景,楚云飞最多也只能和人家持平而论。

日子就这么一天天的过去了,楚云飞和周琳琳的感情也在不断往来的书信中升华,他意外的发现,其实很多的话说出来未免有点不好意思,但是写到信里却很正常,这就算“鸿雁传情”了吧?

到了大二的时候,学校里选修第二外语,出于众所周知的原因,楚云飞选了号称“世界上最难学的语言”阿拉伯语。也是在这个学年,第一个学期结束后,楚云飞并没有回家,而是跑到了北京,浪漫地陪着周琳琳度过春节和情人节。虽然大部分时间周琳琳并不方便外出,楚云飞也不得不长时间呆在周家大小姐借来的屋子里看电视,但该发生的事情还是发生了,两个热恋中的情人为对方付出了自己的第一次。

也是在大二,蓝盔部队确实来学校招人了,但只要大四的学生,这段海外的军营生活将成为他们的实习内容。不过除了特种作战系要了三个英语好的学生,步兵指挥系也要了两个。

到了大三的时候,纯粹从个人能力上讲,楚云飞已经成为系里名副其实的No.1,虽然上面还有大四的师兄,但大学最后一年大部分的学生都是在全国各地跑来跑去,不停的实习,对学校而言,基本上算是已经毕业了。

树大了自然要招风,虽然大部分的同学只是位于后几批方阵,对楚云飞只能仰视而动不起什么脑筋,可那几个同样和楚云飞位于第一方阵的同学却是从不肯停止追赶的脚步。

楚云飞开始大四生涯的时候,就有喜欢这个学生的老师跑来预定自己的研究生,系里也有老师来和楚云飞打招呼,想让他留校。楚云飞又一次面临人生的重大选择。

他先是和周琳琳写信商量,周大小姐自然知道这个家伙心里想的是什么,这也是楚云飞当时吸引她的原因之一。为父报仇这个情节虽然老套了点,但在现代社会这种悲情人物还是不容易见到的。可像这种事情周大小姐只是热衷于做个观众被感动一下,当她也成为剧中的人物参与演出时她还是希望能够改下剧本。

并不是说周琳琳就那么容易放下民族气节,虽然事实上女性对民族气节的敏感程度远远赶不上男性,而是她更关心的是这个自己愿意守侯一生的男人的安全,尤其是这个男人和她还有了亲密关系和相守一生的承诺。

但楚云飞肯定不愿意就这么放弃自己酝酿已久的计划,而且他真的不愿意让这重重的自责陪着自己走过一生,其实他写信给周琳琳与其说是商量,不如说他更多的是因为尊重而想给对方个通知。于是恋人间头一次有了重大的分歧。

有分歧自然会有争执,两人最后相互妥协的内容就是:暂时不去考虑考研究生,也不去想留校,而是看看还有参加蓝盔的可能没有,如果有,楚云飞可以参加,毕竟在哪里也是实习。可如果没有,楚云飞以后也不能削尖脑袋到处去找机会,毕竟将来家里会有老婆和孩子等他,他总不能年复一年的在国外出生入死吧?

于是楚云飞婉言推辞了老师们的厚爱,其实这并不奇怪,以他现在的成绩,只要能顺利保持,毕业时会有大量好的位置供他挑选。

可好的位置毕竟不会很多,而楚云飞也只是个人能力突出点,在军校这个注重出身和背景的大环境里,真要论起来他还真算不上具有绝对优势。

幸好对这一点,楚云飞也有清醒的认识,所以他去找那两个出身军人世家的佼佼者以求达成同盟,结盟的内容就是:如果大四有参加蓝盔的实习机会,希望对方能支持他,而作为回报,楚云飞将在毕业时为对方想得到的岗位让步。

两佼佼者一个叫成树国,一个叫杨华,杨华倒是很痛快的答应了,在他看来楚云飞实在对他构不成什么威胁,但是在帮助同学的同时能够给自己多留点机会何乐而不为呢?

成树国的态度就有点暧昧了,他二话不说先把楚云飞弄到军校外的小酒店里,干掉一瓶白酒以后两人才开始正经交流。

楚云飞这时候才明白:成树国也很渴望参加蓝盔,不为别的,而是这家伙有点过于热血,总是想着既然有机会出去走走,那就一定要扬我国威于异域,痛快的展示一下中华雄风给外人看看。而成树国的实力真的很强,从小家里的教育也是“中国人民从此站起来了”,他有这般想法确实是再正常不过了。

楚云飞当然不能因此而指责对方挡了他的道,这本来就是很飘渺的事情,楚云飞也不过是想让可能会来的事情变得更有点把握而已。

既然达不到目的,多个同盟也是不错的选择,反正名额应该不止一个吧?于是楚云飞和成树国结盟的内容就变成了:如果大四有参加蓝盔的实习机会,成树国在达到目的以后大力支持他,而作为回报,楚云飞同样将在毕业时为对方想得到的岗位让步。

机会总是会留给有准备的人,没过几天,蓝盔部队真的来招人了,楚云飞和成树国还有其他两个同学如愿以偿的加入了进去,不过去的地方可实在让俩人吐血:刚卡——中非东部的国家。

两人走了没多久,蓝盔又来招人了,杨华加入了这支部队,这支部队的驻扎地是:沙特!!!

大部分时候人算真的赶不上天算!

\section{第三十五章 初到刚卡}

全副武装的楚云飞从运输机上走下,扑面而来的风沙使战士的眼睛下意识的眯成了一条线:这沙土可比北京大多了,不是说撒哈拉沙漠在非洲北部么?

撒哈拉沙漠确实在非洲北部,但不是说有那么大的沙漠叫撒哈拉,它是由若干个大大小小的沙漠组成的,其中索度国的索度沙漠就和刚卡离得很近,严重的影响着刚卡的气候环境,这些是楚云飞后来才知道的。

中国蓝盔部队是在刚卡的首都摩沙下的飞机,摩沙的机场很简陋,感觉就象中国某个县的军用机场一样。虽然面积很大,但是整个机场就没有三层的建筑,而且所有的建筑物外墙都是蚀迹斑斑,唯一的高一点的空地勤指挥塔稍微好点,不过楚云飞感觉它象在团部附近所见过的抗日战争期间使用过现已遗弃的炮楼。

不过这还不算什么,当战士们坐上汽车往驻地行驶的时候,坑坑洼洼的地面和路边低矮的草房在使劲提醒着大家:这才叫落后!从机场到驻地两百公里的距离让大家在车上整整呆了八个小时。

看着路边随时蹿出的一群群孩子,楚云飞掉头问成树国,“你说这些家伙不知道计划生育也就算了,可他们怎么就弄不明白根本养活不了那么多孩子呢?”

成树国没有理他,他在不停地向一群孩子微笑着挥手,因为那些孩子在不停地冲着他喊:“KONGFULL!KONGFULL,CHINESEKONGFULL.”等孩子们不再追随汽车的时候,才掉头向一个战士故作严肃地说,“我发现我越来越佩服李小龙了。”

那个战士狠狠瞪他一眼,“你不是没事找事么?现在流行的是成龙。”

楚云飞笑了,“李大龙,你这名字在这里一亮,绝对会有黑人妹妹排队和你约会的,不过,啧啧,就你那功夫我怕会给你本家抹黑。”

李大龙的功夫显然比不上正和他开玩笑的这二位,恨恨的说,“笑吧,使劲笑,多吃点沙子有助于消化。”

按说一群人在车上吹牛聊天,时间会过得很快的,可等到到了驻地,强悍如楚云飞者也有点感觉受不了,天气实在是太热了,路又极其的难走。

以女士为主的卫生队那帮人就更惨了,有七八个已经吐得什么都吐不出来了,就剩下扶着汽车在那里干呕了。

营地也很简陋,刚盖起的一排排砖房还散发着泥土的香味,铁丝网围着的院子里更是杂草丛生,不过显然是因为气候原因,杂草没什么宽大的绿油油的叶子,居然以针状的居多,所有人都在疑惑:这就是联合国维和部队的驻地?

院子里站着三个人,那是先头部队,其中一个招呼大家,“大家辛苦了,进来吧,不管怎么说是到地方了,晚饭给你们准备好了。”

晚饭以罐头居多,都是“MADEINCHINA”的那种,水是矿泉水,酒是中国产啤酒,主食是方便面,关起门来没人会觉得是在国外。再有就是满身尘土的部队每人只有3升水来洗漱,厕所也是简易的露天厕所,条件实在不能说好。

第二天一大早,部队正在出早操,军营外逐渐围了三四百当地人,楚云飞望着名副其实的“黑鸦鸦一片”,很是纳闷发生了什么事情。有消息灵通的战友已经打听到清楚了:原来这些本地人都是来找工作的。

原来早有人惦记上这里了,因为营地显然没有完工,房子需要粉刷,院子需要平整,道路需要硬化,围墙也需要重新建设,而其他维和部队类似这样的活都是雇佣当地人来做的。

中国的维和部队来的时间晚,所选的位置也比较偏僻。前些日子营房施工的时候是政府的施工队来做的,当地人插不进手,可眼红那是肯定的。现在好容易驻军来了,后续工程即将开始,自然有消息灵通的想在这里挣点钱补贴家用。

看看外面的人群,楚云飞又有新问题了,“来干活就干活吧,怎么来的全是女人?男人都哪里去了?打仗去了?”

旁边就有战士答话了,“不懂了吧?这就是非洲,干活就是女人的专利,男人只管打猎和打仗,其他时候宁可睡觉喝酒也不能帮女人干活。”

楚云飞掉头一看,是高建军,这家伙是公安系统选送出来的,“呵呵,老高你倒是什么都知道啊。”

“那是,”高建军有点得意,“凭什么是哥哥我来这里啊?还不是咱对非洲这里了解点?”

楚云飞兴趣马上就来了,“那像那些非洲神话里说的,这里的男人有粮食先酿酒,要是女人想先给孩子做饭要挨打是不是真的啊?”

高建军还真知道这个,“你知道的也不少啊,以前就是这么回事,不过现在这里的人观念也有点改变了,多少还能留给孩子点粮食当饭吃,因为不能再乱吃了人嘛。”

大家正七嘴八舌的聊天,驻地的集合号响了,所有驻地官兵接到了来到非洲的第一个指令:完善驻地设施。

完善驻地设施就包括了以上所有当地人所揽的活,还包括架设工作设备、生活设备等等。这倒不是说中国的维和部队出不起那点钱雇佣当地人,主要是要发扬军队的优良传统:自力更生,艰苦奋斗。

当然中国蓝盔也有别的顾忌,所有的蓝盔部队都接到过类似的命令:尽量和驻地百姓保持一定的距离,一来可以防范某些别有用心的人对部队进行渗透;二来毕竟部队的性质是维和,保持公正性是很重要的,和当地百姓过于接近的话,难免会有失去公正的可能;还有就是军民关系处得过于接近的话难免会有什么磕磕碰碰的事情发生,会严重影响中国军队的国际形象。

这些东西都在出国时的短期培训中提到过,大家也都做好了来这里艰苦奋斗的准备,估计要是雇佣当地人才会让大家大吃一惊吧?所以部队开始了热火朝天的建设。

简易驻地休整起来其实很方便,大家都是年轻人,又有当地政府提供一些设备和材料,所以整个营地在短短两周内焕然一新:营房刷了,院子里和门口的路硬化了,通讯天线、卫星收发器都架了起来,连机井都打了一口,甚至还引进了当地很少见的一些灌木和花草对营地进行了绿化。

在建设的日子里,各种装备也先后运到,两星期后军营就开始了正常的运作。

\section{第三十六章 刚卡的祸源}

在这里先简单的介绍一下这支维和部队的组成结构:总人数有186人,其中隶属后勤部门的有17人,负责营房维护、设备维修、库房管理和炊事等工作。

还有支31人的卫生队,一如中国所有蓝盔部队一样,卫生队的人数比例极高,这不仅仅是因为维和部队的高危险性需要大量医护人员,卫生队还肩负着为当地群众看病和救治的工作,因为既然要维和,驻扎的地方总是不太平的,这种情况下当地群众的的治疗环境当然不会好到那里,所以卫生队就有了表现中国军队人性化一面的任务。

剩下的138人里有大队长和副大队长两人,其他就是三支分队,分队又分三支小队,另外每个分队还有机械小队,每个机械小队主要装备63C装甲运兵两辆,北京吉普大切诺基三辆。从人数上看基本上一个小队就是一个班的编制。

楚云飞和成树国都分到了第二分队里,由于来的军校生都是服现役的,语言交流上也有优势,所以比普通挑选来的士兵和警察级别稍高点,两人分别任第一和第三小队的副队长。

这部分行动部队肩负着保卫营房和驻地维和的任务,另外营地中还有大使馆派来的联络员一名。

驻地维和有多种方式,现在刚卡蓝盔的主要任务是监督停火协议和处理各方小规模冲突,像附近方圆将近两万平方公里的地区就交给了这三支分队。

刚卡是个资源极其匮乏的国家,不过铝的储藏量相当大,而且品质相当高,虽说有铝的地方一定有铁,但那些铁矿石的品质实在不敢恭维,基本上属于刚卡想卖但没国家想买的那种超级贫矿。

刚卡主要是由胡图族、西瓦族和图西族三个民族组成,还有少数的阿拉伯裔人,其中西瓦族占20%,是本地土生土长的种族;胡图族占65%,图西族占10%,两族都是在持续近二百年的民族大战中迁徙来刚卡的。

胡图族和图西族都是很大的民族集团,下面众多分支,覆盖了中非和北非好几个国家,但两族约两百年前由于不同的殖民者的支持,爆发了残酷的民族大战,每次战败一方都遭到残酷的屠杀和灭绝,这场周期性战争时至今日已经超过两百年,打打停停,停停打打,打到现在,两族已经忘记战争开始的原因了,只能用一个词来解释持续的仇恨:世仇。

在非洲独立运动开始后,两族也曾经有过短暂的合作,等到非洲大地各自势力范围、国家边界大局已定后,两族在各地的冲突又有愈演愈烈的苗头,其中更是有个中非小国因此分裂为两个国家。

本来那些高尚的文明国家是不屑去管这些黑人的死活的,更有战争贩子从中出售军火渔利,更绝的是为了长久的发财,这些战争贩子们居然想方设法的控制战争的规模,两族的冲突也因此限制在一定的范围之内,不能不说是个绝妙的讽刺。

这一切都在索度国发现世界上第三大油田后发生了巨大变化,巨大的油田吸引了众多文明国家的注意,甚至属于第三世界的中国也参与进了对索度的开发中。

索度国的图西族和胡图族人数差不多,基本上都是占总人口的20%左右,由于油田的发现,当地人的生活水平也提高了不少。人们都把一般的勇武说成“穷横”是有道理的,起码索度国内两族因为生活质量的提升,相互的敌意减少了不少。

可刚卡实在是太穷,又因为刚卡与索度相邻的地段是控制在胡图族手中,就有相当数量渴望幸福生活的胡图族人越界去讨生活。

这样一来,利益被分薄的图西人自然不肯答应,又因为两族在索度人口数量上的相对平衡被打破,于是战火又起,本来战争先开始是控制在刚卡国境,可随着战事的蔓延,渐渐就烧进了索度。

这种情况诸文明国家自然是不能接受,尤其在索度既得利益最大的是世界头号强国美国,于是联合国终于注意到了:在刚卡这么个贫瘠国家居然存在残酷的种族屠杀和种族灭绝。

经过联合国的正规表决,众多国家出兵索度和刚卡进行维和,其中驻刚卡的的维和部队目的就很明确了:在公平公正的前提下,以西瓦族为各级政府骨干,尽量帮扶人数不占优势的图西族。于是图西族的武器装备有了大幅提升,生活也开始变得稳定。

中国蓝盔的驻地就是在个西瓦族、胡图族和图西族三族交错居住的地段,巡逻任务就是在胡图族和图西族的交界处每天走走过场,当然如果有人真不给面子进行武装冲突那就地将双方缴械。

可这个地段也实在太复杂了点,三族人你中有我我中有你,不到两万平方公里的范围内有十几个大型部落居住点,还好唯一的中心城市坎塔卡的军营驻守着刚卡政府的一个正规团维护治安,要不中国军队绝对忙不过来。

在驻地完善之后,中国蓝盔就该执行维和任务了,当地政府派来了工作人员,经过仔细研讨,确定下了定期巡逻路线和不定期抽查范围。不过作为维和部队最高指挥官,商成钢少校觉得可用人手实在少了点:三个分队各负责一条路线,不捉襟见肘才怪,还好只是维和不是打仗。

第一次巡逻,在向导引领下,三个分队全是精英尽出,只留了少量的人员驻守营地,一来是要让大家熟悉线路,二来就是头次巡逻,不容有失。

在一比一百万的地图上随便划划是件很容易的事,可真要把这将近两百公里的路巡逻一遍那可真不是一般人能承受得住的,除去几个小镇的附近有所谓的公路,基本上部队巡逻时是不需要详细考虑怎么走的,方向对了就行了。

巡逻到一半部队开始向驻地折返的时候,楚云飞暗自庆幸:还好第二分队巡逻的路线不是三条线里最长的。可二分队第三小队副队长的庆幸还没持续多久,就不得不面对一个残酷的现实:在这里他又一次遭遇了“服水土”,不过是刚卡版的。

\section{第三十七章 运气真好}

二分队是从早上八点开始巡逻的,等到了中午十二点多,刚好巡逻到了折返点,由于刚卡接近赤道,十月的天气还是非常的热,大家吃了点随车带的东西,分队长吴海涛就吩咐大家找个阴凉地休息一阵。

等到了下午两点半的时候,休整过的队伍又开拔了,来回程的路线不同,还是成树国带辆吉普前面开路。

走到三点多的时候,二分队路过一个西瓦族的部落领地,尘土飞扬中一片黑忽忽的人群向二分队涌来,足有上千人,这次来的可全是男人,巡逻车队不得不停下了。

人群最前面站着五、六个看似领头的人,有老人也有中年人,后面是精壮的黑色汉子,他们手持各种各样的冷兵器:木棍、长矛、弓箭。而且在最外围还有四五十个拿着自动步枪的武装分子。

要说成树国的胆子还不是一般的大,带着向导和一个战士直接就上前去和对方交涉。西瓦族和他交涉的是个非常壮实的黑人,几个人乌里哇啦的说了一通,成树国回来向商分队长汇报。

原来这些西瓦人说这么大宗的武装车队要从这里路过,必须留下些见面礼,日后也好再相见,至于应该留下什么?自然所有的车辆和装备都要留下,看在是不同种族的份上,对随队的几个女军人就不做要求了。

吴分队长一听,头马上就大了,他扭头问带路的西瓦向导:“你们族有这么个风俗么?”

带路的向导是政府派来的工作人员,自然知道这个事件的恶劣性,“咝,按道理说是这样,咝,因为我们是正宗的刚卡本地人,咝,外族人总给我们带来大量的人口和战争,咝,所以我们一定要让外族人弄明白我们才是这里真正的主人,咝,交过礼物会给我们个他们部落的黑木杖,咝,有了这个以后见到大部分的西瓦部落都不用继续交了。”

吴海涛被他一口一个“咝”弄的头晕眼花,听到这个令人沮丧的回答,更是怒从心头起,“那你出来的时候怎么不提醒我们?”

那向导脸上就多了几分委屈出来,“咝,这是很古老的风俗了,咝,谁知道哪个部落还在用哪个部落没在用?咝,再说你们都那么有钱,东西给了他们不就完了?咝,美国人还给了呢。”

吴分队长被“咝”得快吐血了,掉头和围过来的指挥官们交代,“你们先商量商量,我去请示队长。”

第一小队长刘宁正热得虚火上头,脾气就不太好了,冲着向导就是一嗓子,“我说你说话能不能别‘咝咝’的?”

高建军也在这个分队,虽然他没有什么级别,不过和刘宁处得不错,扯了他一把“别说他了,这应该是部落间正规场合表示善意的附着音,估计他是用习惯了,这向导身份不低呢。”

成树国虽然职位低了点,可这家伙说起话来也很冲,“我们还就不给了,他们能怎么样?”

刘宁和楚云飞马上附议,不过第三小队的小队长孔繁茂是公安系统出身,考虑的出发点又不太相同,“要从大局考虑,事情激化了怎么收拾?”

第二小队的小队长韩斌也是军人出身,“操,那是你们警察,我们当兵的怎么能把武器交出去?”

刘宁马上大声赞成,“就是,要这么把武器交出去我家老头子知道了非枪毙了我不可。”感情这家伙也是军人世家出身大家用汉语正讨论得不可开交,“咝咝”的英语声又起,那向导有点着急了,“咝,你们快点商量,咝,等那个恰恰枝烧完他们就要动手了。”

大家听了,扭头一看,双方人群交界处,一条长有一米左右的灰色灌木枝正在阴燃,正像中国寺庙里烧的大香一样,不过烧的速度可是不慢。

楚云飞又动起了歪脑筋,跟向导请教,“我要是把那个树枝……恰恰枝弄灭了会怎么样?”

向导一着急,连“咝”的声音都发不出来了,又是高建军快速抢答,“部落间交往是很重大的事,你要随便破坏人家的示意,我估计……你会很惨。”

向导频频摇头表示赞同,“咝”的声音终于又出来了,“咝,他说得很对,咝,那样的话你就是他们全族的敌人,咝,交黄金都不行。”

吴海涛一脸严肃的走了过来,古副分队长迎上前去,“怎么样?”

吴分队长左右看看,尽量散发出种凛人的气势,“驻地正在和大使馆联系,商队长要我们见机行事,双方沟通时尽量做到言之有物、言之有理、言之有节,尽量避免过激行为,控制好事态发展的同时随时和驻地保持联系。其他两支分队还没遇到这样情况,正在迅速按原路返回,这次我们真是……”凛人的气势不见了,“我们真是太他妈的走运了。”

这时就显示出中国军人的素质了,竟然是全场鸦雀无声,大家眼睛都看着吴分队长,那意思再明白不过了:领导指示吧,打还是不打?只有向导还在好心的提醒,“咝,恰恰枝烧得很快,咝,你们快点决定吧。”他不着急不行,一旦恰恰枝烧完,他也是被攻击的对象。

吴分队长是武警部队出来的,既明白中国军人对武器的重视,也明白控制大局的重要性,急切间显然拿不出特别好的主意,“别傻了一样,上面是指望不上了,那个树枝还能烧十分钟左右,咱们四分钟时间自由讨论,两分钟集中意见和表决,剩下的时间机动备用。”

大家三五成堆议论纷纷,李大龙跟楚云飞嘀咕了句,“还好当兵的多,要是警察多就直接交枪了。”楚云飞瞪他一眼没说话,也没去找上级孔繁茂,而是拉住了高建军,“老高,你说要是两个部落一个要收礼物,一个不交礼物,那一定能打起来么?”

高建军也正在琢磨这个问题,不过肯定没楚云飞想得这么远,愣了半天才犹犹豫豫的说,“好象不一定,应该还有什么变通的方式吧?要不去找咝咝问问?”在这么紧张的场合下高建军居然能有心思给别人起外号,可见心理素质不差。

向导正被一堆人的奇怪问题弄得不耐烦呢。

“要是随便送他们点中国神药(清凉油)行不行?”——这是有经营头脑型的,“咝,凄凉有?那是好东西,咝,不过现在这种情况得听主人的话。”

“你就不能和他们解释下我们是维和部队么?”——这是说话不经过大脑型的,“咝,他们早知道了,咝,现在装不知道呢,咝,恰恰枝都点着说什么也晚了。”

“我们要是跑他们应该追不上,会有什么后果?”——这是以智取力型的,“咝,要跑可以,咝,不过你不尊重西瓦人的风俗,咝,以后是所有西瓦人的敌人,咝,别看,不算我。”

“不给我们面子,武力镇压他们行不行?”——这是……没大脑型的,“咝,别和我说这个,咝,我没听见。”

\section{第三十八章 勇士第一程}

楚云飞的英语很标准,又使用了西瓦人的风俗,所以虽然人多口杂,但他一说话向导就注意到了,“咝,要是两个部落一个要收礼物,一个不交礼物,咝,那一定能打起来么?”

在大家的笑声中西瓦向导明显的愣了一下,考虑半天才说,“咝,很久以前,咝,还有个折中的办法,咝,不过好久没人用了,咝,你可以替主人杀指定部落的人,咝,拿人头和人肉当礼物,咝,或者说走勇士旅程,咝,你可以别学我说话么?”

听到有别的解决方法,大家都不做声了,看着副小队长和西瓦人交流。

去杀人弄人头和人肉肯定是不现实的,楚云飞关心的自然是另一项,“勇士旅程怎么走?简单点说,哦,咝,我不是嘲笑你,咝,我是尊重你们的习俗。”

向导也知道时间不等人,马上解释,“咝,就是你们出三个勇士,咝,主人出十八个人拿弓箭射你,咝,四十八箭后还有勇士能继续战斗,咝,那就交一半礼物,咝,然后勇士和主人部落里的勇士打斗一场,咝,赢了就不用交礼物了。”

吴海涛也被楚云飞谈话吸引过来了,听完马上拍板,“那就选这个,大家表决一下吧。”

中国的部队里最不缺的就是热血男儿,当即大家就一致通过了这个决定,然后就是吴分队长向驻地汇报。

向导看大家决定了,也不再说什么,偷偷把楚云飞拉到一边,“咝,现在你要去把恰恰枝空手弄灭,咝,表示你们要走勇士旅程。”

楚云飞有点疑惑,“咝,为什么是我去?”

向导比他还疑惑,“咝,难道你不想做勇士么?咝,你又这么尊重我,咝,弄灭恰恰枝的肯定要走勇士旅程,咝,一般是最勇敢的勇士才有资格弄灭它。”

看着恰恰枝不弄都快灭了,楚云飞顾不上再说什么,飞跑过去,忍受着灼热,拇指和食指一用力,树枝灭了。

两边全被楚云飞这个动作弄呆了,中国蓝盔部队经过向导解释才明白这是个选择勇士旅程的示意;而西瓦族人却根本没想到以前的风俗,直觉就是想追杀这个蔑视他们的黄种人,幸亏有长者终于想起族中流传已久的传说,才压制住了蠢蠢欲动的人群。

怕引起对方误会,楚云飞弄灭恰恰枝后就站在原地,西瓦族人中走出个老者,英语可是非常标准,“勇士旅程?”

楚云飞点点头,“咝,勇士旅程。”想想动作不对,又马上摇摇头,“咝,抱歉,我们点头表示肯定,咝,我们选择勇士旅程。”

老者摇摇头,“咝,好的,咝,我们各自回去安排一下,咝,勇士你的尊重我收到了,咝,不过你不要再用咝了,咝,我和英国人打交道习惯了,咝,你那不标准的发音让我很难受。”

回到队伍里,经过向导的回忆,大家大致明白了勇士旅程的具体步骤。

先是自己这方出三人,每人只穿一条短裤以保证身上没有护具,然后对方出十八人以这三人为中心围成个直径约70米的圆,每人射三箭一共48枝箭,呃,不,是54枝。自己这方可以躲避,但只能在圆心附近很小的范围活动。

这关过后,三人中要有人还能继续战斗,就和对方选派的勇士徒手相搏,谁被压在地上起不来谁输。

既然大家都明白了下面要面对的是什么,那就是选派选手了,楚云飞自动出线,还有两人派谁去?

还好维和部队在来刚卡前经过了短暂的磨合和培训,相互之间也算知根知底,本来就实力很强的楚云飞自动入选了,那剩下的就是特种兵部队出来的杨国庆和郭平平了,成树国肯定是有点不服气的,不过他的身手确实和这三人有点差距,那也是没办法的事情。

虽然队伍里还有女军人,只穿短裤给人的感觉有点侮辱人,但是天气本身就相当热,又是西瓦族传统对事不对人的,再说西瓦族人也全是光着上身,所以这个小麻烦也就不算什么事了。

西瓦族的已经选出了他们的十八个射手,弓都不是很大的弓,一看五花八门的就知道不是什么制式兵器,地方也选好了,给三人划了一个直径5米左右的圆,“你们的勇士站这里。”

楚云飞和杨国庆、郭平平咬了咬耳朵,站进了圈子,又一条恰恰枝被西瓦老者点燃:战斗开始。

按道理说,西瓦射手们弓箭的最具杀伤力范围在20-25米,再近容易被猎物发现,太远的话不易瞄准,而且弓箭的威力也大大降低,再说也没那么好的制作材料。将近35米的距离就算是远的了,幸亏射手多,要不真不好射。

可看看对方的姿势,西瓦射手们更头疼了:恰恰枝一点燃,三人居然立刻都趴在了地上!

想想传说中的西瓦英雄舒云淡鹰,在同伴都被射倒的时候,咬牙从胸脯上拔下拔下利箭挡开了剩下的攻击,再看看这可恶的黄种人,这样的人也能叫做勇士?射手们抬头看看西瓦老者,老者摇头示意属于规则允许范围:看多了英国人打仗,他自然知道这种缩小被攻击面积的作战方式是大势所趋,自己的族人实在有点太不思进取了。

箭枝纷纷落下,西瓦射手们发现对方做出了更卑鄙的举动:他们不但在地上滚动,而且居然用手在地上搅拌起了大量灰尘!这叫人怎么瞄得准?老者也有点看不下去了,大声喊:“不许抛土!”

不许就不许吧,三人都这么想,反正不坚持让站着自己已经占大便宜了。于是三人不再扬土,边滚动边捡起掉落在四周的箭枝挡格起来。当西瓦射手们把箭射完的时候,只有杨国庆的肩头中了一箭,楚云飞和郭平平都完好无损!

西瓦人不但痛恨三人的卑鄙,而且更为自己射手的准头而恼怒不已:太丢人了!

于是西瓦人开始喧闹起来,那老者也觉得非常没面子,压压双手示意人群安静,然后一指楚云飞,“你,你和我们的勇士狂风比赛。”

因为大家都看出三人刚才的反应全是这个家伙出的主意,再说,郭平平比楚云飞的块头也大多了,肉搏的胜负可是需要压住对手才能取胜,老者自然想的是看你这个小瘦子怎么压住我们的狂风。

\section{第三十九章 狂风勇士}

出于对楚云飞实力的信任,大家自然对老者的意见没什么异议,事实上连咝咝向导也不清楚如果还有两个勇士幸存的话接下来的旅程该怎么走,尽管说三个勇士都幸存也不为过。

楚云飞自然也不例外,他个人如其他两人一样,觉得由自己出手那是最符合蓝盔实力的。

可自信归自信,该核实的还是要核实的,所以楚云飞谨慎的问了一句,“输赢的标准就是压住对方,爬不起来的就算输么?还有其他讲究没有?”

老者对自己的族人还是满有信心的,不过他对现代高科技的装备也有点戒心,“对,双方不得使用任何器械,只有真正不靠装备的男人才能算勇士。”

那就好办了,楚云飞对自己说。

尽管维和部队已经有了心理准备,但是狂风慢慢从人群中踱出时还是吓了大家一跳,不仅仅是因为他慢吞吞的出场方式摆足了威风,还有就是光凭他的身材也有足够的资格让大家揉揉眼睛了。

黑人里也有相扑运动员?这是在场所有黄种人脑中的疑问。

虽然可以肯定狂风不会是肯尼亚长跑运动员那种瘦小的体格,但是军人们的想法最多也就是按照拳王泰森那个级别来想象西瓦的勇士,没想到出来的居然是一个身高一米九左右,足有300斤的大号超级人型动物。

还好看起来比较笨拙,这个想法在军人们的脑筋中停留了还没有10秒种,就被对方的动作打破了:狂风全身上下扭曲了几下,传来了几声轻微的“啪啪”声,那是骨节的脆响。有经验的军人们都不由得抽了口冷气:好厉害的家伙。

看上去楚云飞的心理素质还真不是一般的好,只见他很自然的摆开门户,低眉顺目,一副荣辱不惊的样子,只有相对知道他多点的成树国才明白那是怎么回事:这家伙把打架得来的经验运用到这里还是不错的,表面看上去还算冷静。

老者从唇间发出一声怪啸,宣告了比赛的开始。

狂风先是看似随意的晃了晃身子,然后一个前跳冲到楚云飞面前,抬手就是一记左直拳,那敏捷的动作让人无法相信竟然是出自这么一个庞然大物。

楚云飞先向左一躲,然后不见腿部有什么动作就向后飘开两步,躲过了狂风合拢过来的双臂。开玩笑,和我比敏捷?

然后双方就这么你来我去的斗了起来,狂风始终想把和楚云飞的距离拉近,而楚云飞则是刻意保持距离。虽说狂风的手臂比楚云飞的长出一截,但保持距离基本就保证了楚云飞安全性。

为什么呢?因为狂风根本就不会用腿部进行攻击,楚云飞擅长用腿攻击,虽说他玩飞腿的技巧未必能赶得上杨国庆,可对付狂风那是足够了。飞腿也不需要用,就一些弹腿、横扫和侧踹就已经足够逼得狂风“哇哇”直叫了,没办法,谁叫他的胳膊没人家腿长呢?

于是大家就看到一座肉山在拼命追逐自己的猎物,而猎物无比灵活的左闪右躲,时不时再抽冷子敲打肉山两下,算是一场精彩而且各有所长的比斗了。

楚云飞在看到狂风体型的时候就知道自己肯定压不住这么大个家伙,一旦近身缠斗,自己还未必能稳占上风,所以“各打各的”就成了他的应对方式。

可对手的强悍也很出楚云飞的意料,打斗这么长时间,那家伙不知道挨了多少脚了,居然和没事人一样,看来黑人就是比黄种人体格好啊,他可没想到对方全身都快被踢得浑身发麻了,只是出于维持面子的心理才咬牙撑着。

打斗了有5分钟左右,狂风心里很清楚,自己的体力快支持不下去了,开玩笑,这么大的体格要保持那种敏捷,还要应变,得消耗多少热量啊?

不能再拖下去了,豁出去了,狂风一咬牙,一个虎跳,大张着双臂,整个身子象座山似的压向楚云飞:卑鄙的小瘦子!千万别让我挨到你,要不你就死定了!

楚云飞有点意外,他可没想到对方这么快就到了山穷水尽的地步,居然会在这时候舍命一博,于是马上后闪,可是慢了点,对方在惊天动地的倒地声中抓住了他的右脚腕子,于是他也被拖倒在地。

狂风心内狂喜,顺势把对方脚脖子一拧就想站起来先甩对手几圈,可他还没来得及站起就见对方的左脚踹向自己的头部,只好下意识的一歪头,另一只手也来帮忙招架。

是招假踹!等狂风明白过来的时候已经太晚了,楚云飞左脚变向踢向了他的太阳穴,虽然仓促架起的胳膊已经触到了对方的左小腿,但一切都太晚了,在昏迷前狂风脑中只有一个念头:这不是我想要的战斗方式!

看到肉山口吐白沫昏倒在地,大家都明白已经走完了勇士旅程,那么,下步该怎么交涉呢?各走各的路?

但接下来无数黑人看到了一幕令人发指的场景:那个卑鄙的黄种瘦子,他站起来以后,居然又趴在了狂风勇士的身上,装模做样的压制着昏迷的的勇士!居然还能微笑地问老者,“算我赢么?”

太过分了!西瓦人又骚动了起来,可在老者的示意下,骚动被控制在一定的程度。老者很清楚:瘦子是出于谨慎,因为勇士打斗胜负判断的依据就是压制住对手,看来对方是存心不想给自己任何漏洞来钻。

虽然也很痛恨对方侮辱勇士的举止,但是老者还是不得不宣布,“好了,你赢了,现在可以放手了。”

楚云飞一站起身,就有几个西瓦人过来救护昏迷的狂风,其中一个西瓦人向楚云飞伸出大拇指,“中国功夫,厉害,厉害!”楚云飞不好多说什么,微笑着向对方点头示意,想想不合适又马上摇头,可摇头也似乎不太合适,他就这么微笑地愣了一下。

楚云飞发愣不要紧,关键是他脸上还带着一丝微笑,这抹微笑落在西瓦人眼里,经受太多刺激的心灵绝对认为那是赤裸裸的嘲笑,于是终于有人忍不住了。

一个很壮实的黑人跳了出来,“你,中国勇士,我要和你打!”

楚云飞看看他,身高也有一米八五的模样,全身肌肉坟起,可又不臃肿,绝对是个敏捷型的好手。

掉头看向老者,“你们这又是什么说法?一定要打么?”

还没等老者答话,那黑人抢先说了,“我是岩石勇士,你只有打赢我,才能拿到黑木杖。”

\section{第四十章 岩石勇士}

“黑木杖?”楚云飞这才想起来似乎部队真需要有这么个东西,回头看看吴分队长,看队长向他微微点头示意,看来也是同意他的想法。

那老者先是愣了一下,才反应过来,先前并没有约定走完勇士旅程就要交给对方黑木杖,现在看来似乎是个不错的挑战借口,于是冲着咝咝向导很隐秘的把左手中指和拇指捏了捏。

这是西瓦族的约定暗语,通常在阴人的时候使用,大意就是:咱们是一起的,双方不要开口,协力干掉对方。自然这个手势现在的意思只是要求咝咝向导不戳穿谎话。

咝咝向导虽然做文明人多年了,但面对自己的同族人不起点偏袒之心确实是很难的,于是暗暗回了个中指和拇指紧捏的动作,意为我不参与但绝对会闭嘴不说话。

现在场上谁最苦恼?那个跳出来的岩石勇士最苦恼,因为他才是真正的狂风勇士。

狂风和岩石都是西瓦族勇士的头衔,本来象这种打斗应该就是狂风勇士出面的,岩石勇士一般是比赛力气的情况下才会出场的。由于楚云飞先前的表现有点过分,西瓦人出于强烈的报复心理才点名要他出战,结果等他站到那里西瓦人才想起来这家伙就是那个捏灭了恰恰枝的勇士。

既然这个瘦子是中国人里的第一勇士,那西瓦人出于保险的目的,自然不合适再让真正的狂风勇士出战了,于是老者显示他阴险的一面,让看来能克制对方的岩石勇士冒名出战。

可没想到打斗应了《地道战》里那句老话:“各村有各村的地道,各村有各村的高招。”岩石勇士被楚云飞克制得死死的,终于不幸落败。狂风勇士自然为有人替他糟蹋狂风的名声而气愤,待到见了楚云飞的微笑那就实在忍不住了,有样学样的冒充岩石出头挑战,值得幸运的是:他居然找了个不错的漏洞出来。

老者自然知道真狂风勇士心里想的是什么,反正再打一场自己这方也不会再输什么,没准狂风还真能把对方干掉。那不但能不给黑木杖,而且也能为族人出口憋了半天的闷气,于是冲楚云飞微笑,“你接受岩石勇士的挑战么?还是制服对手的人赢。”

楚云飞摇摇头,“好吧,我接受,不过我赢了真能拿到黑木杖么?”

老者还真有点受不了楚云飞西瓦式的表达方式,“你还是点头好了,如果你能取胜,我保证会给你们黑木杖。”

楚云飞还是要敲定一下,“这就是最后一场了吧?不会再有别的勇士和我比了吧?”

老者让楚云飞说得脸有点发热,“咝,那是自然。”

老者正待宣布战斗开始,却发现那个卑鄙的家伙又开始卑鄙了:他居然从西瓦人那里拿了根恰恰枝点了起来,盘腿往地上一坐。

这意思半天西瓦人才弄明白:这家伙要休息一会儿!所有西瓦人都快被他弄疯了,做为个勇士怎么可以这么斤斤计较?这是在破坏勇士的名声,可恶的中国人!

如果腹诽可以杀人的话相信楚云飞就在这前后半小时已经被人杀了5000多次了,可当事人却是毫不理会,若无其事的在地上打坐休息到恰恰枝熄灭。

中国勇士直到恰恰枝烧完还是一副精疲力竭、萎靡不振的样子,于是老者心情畅快的抓紧机会宣告了战斗的开始。

但是这家伙实在是太狡猾了,就在西瓦人都认为他还会象先前那样先守后攻的时候,却没想到楚云飞已经展开了凶猛的攻击,被他的伪装迷惑了的狂风勇士更是差点直接翻船。

楚云飞在休息的时间里有足够的时间去考虑这场战斗,在他的意识里,刚才那个肉山才是真正的麻烦,还好最终被他击败了,象眼前这个岩石勇士,楚云飞最擅长的擒拿格斗足以对付了。

西瓦人自然也听说过,起码是在电影里看到过中国功夫,但中国众多好手在中外对抗中取得胜利却不在拳脚上,而大部分是在抱摔上获得优势的。当然,象那种档次的对抗真正的高人是不屑一顾的。

虽然楚云飞也不太擅长抱摔,但他擅长擒拿,两者相加就给西瓦人造成了相当深刻的印象:残忍!

楚云飞先是和岩石(为方便起见,我们还是叫他岩石吧)勇士缠斗在一起,按理说双方都是敏捷型的选手,谁想完全压制对手都不是那么容易的事,可岩石勇士先被楚云飞的伪装迷惑了一下,结结实实的被楚云飞摔了两跤。

摔倒岩石后,楚云飞尝试着压制对手,却发现对手身手实在是过于敏捷,非常滑溜,力气也相当的大,那就只好对不起了。

于是又一幕令西瓦人发指的景象出现了:黄种勇士先是在缠斗中去扭动岩石的手臂,却被矫健的岩石勇士滑脱,岩石勇士趁机反扭对方手臂,却被卑鄙的黄种勇士狠狠一拳击在鼻梁上,岩石的鼻血和眼泪当时就流了下来,更让人气愤的是,卑鄙的勇士趁岩石的眼泪挡住视线之际,卸开了岩石勇士的左臂关节!

卸掉对方左臂,居然人家连哼都没哼一声,楚云飞在佩服之余更下了狠心,不服输是吧?左手叼腕子右手肘狠命发力,“喀啦啦”一声,岩石的右手手肘和右肩同时被拽脱臼!

就在岩石痛得死去活来,西瓦众人眦目欲裂的时候,楚云飞毫不手软,趁势一个抱摔,把岩石牢牢的压在了地上,还没等他说话,就听见老者急切的声音,“你赢了,松手吧。”

整场比赛居然用了还不到两分钟。

黑色人种虽说身体的延展性和灵活性都很强,但是西瓦族在漫长的和自然的抗争中还是经常面对眼下这种情况的,当时就有擅长治疗的西瓦人上来止住岩石的鼻血,但是似乎没有擅长接骨的人在场。

楚云飞看岩石咬着牙,浑身颤抖着被人往回扶,心下泛起一丝的不忍,“等等。”

岩石勇士现在心里那个悔恨就别提了:还怨人家糟蹋狂风的名声呢,自己这两下子不是更丢人?以后还会有人尊敬的叫自己勇士么?再加上关节脱臼那份钻心的痛楚,简直连死的心都有了。

听到楚云飞的声音,岩石勇士眼含无尽的怨恨向对方望去,却见那卑鄙勇士向自己微笑,“我能帮你把骨头接好么?”

\section{第四十一章 惊人一怒}

和普通的西瓦人一样,岩石勇士很讨厌那些廉价的同情,可他身体传来的剧痛却在提醒他:你快支持不住了,你的对手是个比你强大的勇士。

当岩石勇士低下眼皮,打算婉言拒绝对方的好意时,却不留神注意到了楚云飞的野战靴,他终于意识到了:对方是维和部队,有着强大的实力和背景!能不招惹还是不要招惹为妙!

于是明智的岩石勇士不得不接受了这个看来有点屈辱的建议,“那好吧,谢谢你。”

黑人的骨骼结构和中国人的没什么两样,但不同人种间还是有小小的差异的,他们的关节部位要比中国人的密实,骨节大点,弹性也大点。所以楚云飞必须静下心来,仔细揣摩肌肉纹理和骨关节的位置,他可不想“丢脸丢到国外”。

几声轻响伴随着岩石勇士的低沉的吼声传出,楚云飞直起身欣慰地搓搓手,“好了,活动活动吧。”

岩石勇士从地上站起来,活动了两下,感觉一切都恢复了正常,刚想说点什么,却听到对方阵营里穿来一声怒吼。“箭上有毒,杨国庆昏倒了!”

楚云飞不由得勃然大怒,一股惊人的气势迸发开来,顺手一拳击向地面,“砰”的一声闷响,激起一大片尘土,等到烟尘散去,西瓦人才发现这个瘦子就那么活生生的空手凌空把地面打出个直径超过半米的坑来。

这一拳把所有人都镇住了,场上一片寂静,只听见阴沉的声音从楚云飞牙缝间传出,“谁干的?给我站出来!”

一个瘦小的黑人从西瓦人中走了出来,就是他,刚才射出了为西瓦人挽回面子的唯一一箭。由于心急,他的话听起来有点结巴,“那是卡巴的种子,没有毒。”

被扬尘弄得灰头土脸的老者一听就明白了,马上给楚云飞和赶来的向导解释,“卡巴的种子,那个东西涂到箭上是为了麻醉猎物,是催眠,嗯,对,就是催眠。”

向导也听明白了,“咝,卡巴草的种子啊,咝,那只是麻醉药,咝,我确定。”

西瓦人也纷纷上来解释:没办法,眼前这人有点过于恐怖,先前还可以说他是卑鄙的小人,可那吓人的一拳可是实实在在的,来不得什么含糊。

于是维和部队就知道了,原来杨国庆只是被麻醉了,本来早该昏迷过去了,但估计是这家伙为了形象一直在硬挺,看到事情结束终于松了口气,于是就顺理成章的昏了过去。

等到所有事情尘埃落定,那根黑木杖终于到手的时候,大家都忍不住跑过来看看到底是个什么东西。看清楚以后楚云飞撇撇嘴,拉倒吧,就为这么个破木棍弄成这个样子?看起来也不是很难伪造的嘛。

由于在这里耽搁了近一个小时,等到二分队终于回到驻地的时候就快到晚上八点钟了,其他两支分队也早就回来了。

一分队回来得最早,接到回驻地的命令以后急忙往回赶,不到6点就回来了。

三分队回来得稍微晚点,在回来的路上,他们遇到了一起小事:二十几个图西族的人正在抢几个胡图族人的猎物。双方正闹得不可开交的时候三分队来了,于是两边停下开始对峙,三分队开路的副小队长按规定向天鸣枪,图西族人撒腿就跑了,剩下的胡图族人自然是心存感激,就想把他们打到的那头瞪羚送给三分队。

可维和部队有严格的纪律约束,三分队自然是不能收他们的礼物,再说,就算没有什么纪律,三分队善良的中国人也不忍心从几个只裹块破布的穷人那里接受那点食物。

当事的胡图族人可不这么想,在他们看来,如果没有这支队伍出现,自己几个能不能活都是个问题,把这唯一大点的猎物送出去自然就算不得什么了。

一个要送,一个不收,最后还是在西瓦向导的撮合下,三分队花钱买下了这只瞪羚,至于价钱是不是公道那就实在说不清楚了。

于是当天晚上整个驻地就有了野味可享用,鲜美的味道让大家回味无穷,当然小遗憾也难免有点,后来商队长居然就因为这个还向大使馆报告,希望下次送来的物资里能有些花椒和醋之类的调味品。

无疑二分队遇到的事情是当天最严重的,晚饭后驻地领导开会时,楚云飞作为当事人居然被暂时提拔半级,列席了这个会议。

针对这天的巡逻遇到的问题和因准备不足可能出现的情况,驻地向上级详细的做了汇报。相关的取证和论证又进行了将近一天,等到相应的应对措施颁布下来的时候已经是第三天了。

很显然最新的应对措施和以前宣布的有些许小的冲突,像以前颁布“驻地部队对突发事件经汇报后有自主应对的权力”,改成了“驻地部队需严格执行上级命令,临时决断也必须经过上级部门的许可才能实施。”

还有“驻地部队在维护自身形象的同时,执行驻地任务时应做到有理、有据、有节”,也被改成了“在驻地执行任务时,尽量做到以大局为重,严禁义气用事,违者视情节轻重给予相应的处分。”这句话明显说的就是二分队了。

还有就是“驻刚卡相关部门有为驻地部队提供尽可能详尽的信息的义务,尽量为驻地部队在执行任务时提供较全面的资料。”这就是人们说的“头痛医头,脚痛医脚”了。

当然还有其他的改动这里就不一一说明了,不过还好针对头次巡逻发生的些小事也没任何人受到什么奖惩。

远在非洲的部队自然不会知道,这些都是国内对外政策有不同理念的阵营交锋的结果,当然,就算知道了,也没人说得清楚到底哪个阵营的理念更为正确。

像在国内一样,驻地部队内部自然也有热衷于向上发展,长于揣摩政策的人,显然这些人已经敏锐的感觉到了一丝风向的变化。

楚云飞虽然脑瓜够机灵,却没把这事往心里去,在他看来,颁布的命令是用来指导行动的,至于为什么做改动他既不想也没兴趣知道。他在第一时间里考虑的是杨国庆的伤势,看过以后他实在有点哭笑不得:原来那箭是从杨国庆背后射进去的,估计本来的目标是他或者郭平平,还好射得不深。

\section{第四十二章 刚卡的风情}

随着新政策的颁布,驻地部队省去了很多事情,只要把相关的情况收集上去,自然有人来拿主意。于是楚云飞不由得又想起了张玉珊:自己现在干的这个工作恐怕也和“采风”类似了吧?

一个月的时间就在熟悉环境中度过了,现在驻地部队已经形成了相对固定的维和风格:三支分队各负责一条线路,每天有一个小队轮流巡逻。甚至到后来胡图人和图西人都熟悉了维和部队的巡逻时间表,于是两族又在巡逻队伍不会出现的时间制造些小冲突。

这种情况很快被维和部队探知,但是为了不犯原则错误,下情充分上传。而在上情尚未下达之前,驻地部队不会因此改变巡逻的方式和时间。毕竟都是小冲突,在巡逻队到达之前那两族人必须结束战斗,事实上也造成不了多大的伤亡,再说反正别人家的孩子是死不完的。

上头也很快传达命令下来:三支分队分别不定时的增派一支小队对可能发生冲突的地带进行巡逻。毕竟是在执行维和任务,不但要把握好微妙的平衡,也要考虑到维和的效果,不能让别国笑话。

既然不定时,执行抽查任务的小队就可能面临大点的风险,那63C装甲运兵车就有点不够用了,一个分队才两辆,经过紧急协调,大队又补充了两辆运兵车专门用来执行抽查任务。

驻地附近的百姓也逐渐习惯了维和部队的存在,还有个别人得到了为部队提供蔬菜水果的生意;当然来卫生队看病救治的也不少,居然还有不少人遭到的就是暴力伤害;更多的人是想和部队套套近乎,以谋得什么好处:眼光浅的就盯住了近在咫尺的水源、糖果、清凉油什么的小东西,眼光远的那想法就多了,居然还有相当数量的黑人少女想嫁个中国战士!

高建军就遇到了这样的麻烦,一个西瓦少女缠住了他非他不嫁,这少女倒是胸丰腿长,面容也还算娇好,皮肤黑点和嘴唇厚点那实在怪不得她。为什么缠的就是高建军呢?倒不是说他有多么高大英俊,那实在要归咎于这家伙对非洲的了解了。

高建军了解非洲那纯粹是个人行为,是多年养成的爱好。虽然作为维和战士他要严格遵守相关规定,但既然有兴趣,工作之余难免就愿意多接触两个当地人;再说他本身对非洲有相当的了解,对当地的风俗习惯要比其他人适应得多,反应也相当够水准,强烈的表现出了对驻地群众风俗的尊重。

这样的话蒂娜被高建军吸引那就很正常了,但部队有严格规定:不许和驻地群众谈恋爱!这不但为了维和部队的形象和公平性,也怕这事会为卖淫嫖娼开绿灯,那样的话可就和美国军队差劲的名声有得一比了。

所以高建军很明确的告诉蒂娜:我在中国有对象了,是个很漂亮的黄种人,当然高某人的意思并不是说黑种人就不漂亮,但作为个有绅士风度的男人只能非常遗憾地对西瓦少女说:抱歉,你确实是迟到了。

可非洲少女的热情怎么会这么轻易的泯灭?蒂娜依旧是天天来军营报到,部队也就渐渐的习惯了她的存在,哪天她要是没来,那一定是高建军巡逻的日子。高建军开始实在是有点招架不住,还专门去和著名的聪明人楚云飞讨教过应对手段,却意外的发现聪明人的情商似乎还低于普通人的水准。

随着日子的增长,高建军发现西瓦少女只会温柔的厮缠,实在是做不出什么过激的行动,慢慢的也就放下了防范的心思。是人都会有点虚荣心的,被少女纠缠并不会让男人丢人,再说又是具备了无害兼美丽的异国情调。

三个分队里没有巡逻任务的两个小队一般都驻守在驻地里,职守的职守,训练的训练,有时候还要应地方政府的要求帮助维持一下治安,当然大本营总是要保证相当数量的留守人员,谁知道什么时候会有突发事件出现?

所以,来到刚卡四十多天以后楚云飞才有个出去逛逛的机会。二分队第三小队和驻地其他岗位的工作人员开着五辆大切诺基去十公里外的坎塔卡采购点日常用品。

车队到达坎塔卡以后先去市政府做了必要的联系,接下来采购了些日常用品,带队的副大队长难得的发了发慈悲,给了大家俩小时自由活动的时间,但是必须三人以上组队才行。本来象楚云飞这样人缘不错的高手应该是被优先邀请组队的,但是很不幸,他被吕副大队长指定留下来看护车辆,同他一样倒霉的还有三小队的战士秦三宝。

这就是中心城市?楚云飞狐疑地看着四周低矮的建筑,也就是中国普通小镇的规格吧?还是不太发达那种,不过规模倒确实不小,看来能有20万人吧。

秦三宝撇着嘴,“女人就了不起啊?必须去逛街?”难怪他生气,本来是卫生队的石婷留下来看车的,但吕副队长耐不住小姑娘的死缠滥打,在人事安排上做了改动,秦三宝不得不服从命令。虽说女士绝对应该优先,但做人情的不是本人的话有点火气那是可以理解的。

楚云飞倒无所谓,挥手撵开两个试图靠近汽车的小孩,“跟女人计较什么?以后有的机会来这里,咱三小队算是最后来过这里的了吧?下次没准还是咱们。”

秦三宝点点头,“倒是,别人来过也没几个想再来的,实在没啥意思,不过,高建军除外。”说着说着就有点想笑,“我是气不过有人拿我做人情,呵呵。”

两人正在那里闲聊,走过来一个十四五岁的小女孩,看脖子上的饰物该是图西人,手里拎着两串捆着的“露露”,看起来是要卖的样子。

“露露”是种当地的植物,楚云飞也不知道学名叫什么,这东西地表上面部分没什么用,似乎还对人有点轻微的麻痹作用,牛羊什么的也不是很爱吃它,但它的地下的块茎里面有大量的水分,一般当地人走远路的时候总爱带上两个,不但削皮以后嚼嚼能止渴,着急的时候还能当干粮吃,虽然略微带点腥味。

露露分黄棕两个品种,棕色的不但味道不好还有一定的毒性,吃多了会让人腹泻。而通过地表上面部分观察两个品种实在不容易分出来,只有挖出来才知道,更由于黄色露露由于挖得人多,基本上不太好遇到,所以这野生的东西就成了商品,但是价格非常低廉。

当然这些商品兜售的对象绝对不会维和部队,因为那瓶装的矿泉水比露露可是好用多了。

这小女孩也没有跟楚云飞他们说话,就呆呆的站在离汽车不远的地方。没过多久,就有两人先后过去问价钱,但似乎谈不太拢,其中一人似乎有点恼怒,但是看了看楚云飞他俩,悻悻的走了。

又过了一阵,来了三个跌得东倒西歪的醉汉,看到了女孩,其中一个估计有点口渴,上前和女孩谈价钱。但是那女孩大概价格定得稍高,三人就和女孩吵了起来。吵着吵着,三人就把那两串露露强行抢了过去,小女孩掉头用求助的目光看着两个维和士兵,但楚云飞不确定那三人是不是胡图族的,实在不方便干涉。

那三人也看到了这俩士兵,但是似乎没什么顾忌,其中一个壮实的更是斜着眼睛用不屑的目光上下打量着他俩,然后重重一口唾沫吐在地上,“呸。”

秦三宝本来就不太高兴,看见这场面,掉头向楚云飞请示,“楚队长,要不要教训教训他们?”

\section{第四十三章 中国人好欺负?}

教训?凭什么教训人家?就为了人家离老远吐了自己一口?楚云飞无奈的摇摇头,“没借口啊,现在队里的情况你又不是不知道,不让咱们乱惹事。”

秦三宝自然是很窝火的,“操,这不是挑衅么?咱又没招惹他们。”

楚云飞只能苦笑,这事搁在新条例下来前自然不能就这么放过他们,不过,现在就只好装没看见了,军人,是以服从命令为天职的。

楚云飞一直紧盯着那三个男人,希望他们因不忿来和自己理论,自己当然就可以用“保护军车”的名义教训一下对方,但那三人虽然喝醉了,却没有主动来招惹这俩不怀好意的士兵,而是拎着抢来的物品骂骂咧咧地扬长而去。而那女孩也是站在那里哭了一阵,又坐下发了阵呆,也委屈地走了。

等到回到驻地,秦三宝把这事向分队长吴海涛做了汇报,吴分队长肯定了楚云飞的应对方式,“楚队长做得不错,就该这样啊,别整天就想无事生非的,上次颁布的条例你不记得了?”

秦三宝却是爱认个死理,“那他们朝我们吐唾沫呢,不是没事找事么?我们这也算丢脸丢到国外了吧?”不知道为什么,“丢脸丢到国外”这话现在在驻地很流行。

吴分队长很不满意秦三宝的觉悟,“这不是你丢不丢脸的问题,而是命令执行得彻底不彻底的事,别没事找事,小心以后不让你出去。”

秦三宝实在有点气愤难耐,又和别的战友说起了这事。才知道去过坎塔卡的维和部队也遇到过类似的事情,中国人这么招人恨么?

其实秦三宝的疑惑是有道理的,这里面确实出了点小问题。自从维和部队接到新条例以后,大家就减少了和当地群众的接触,毕竟是多一事不如少一事的。可接触一减少,部队表示善意的机会就少了,小东西也不怎么送人了,帮忙的事也做得少了,现在也就驻地周围的群众和部队还融洽些。

再有就是既然部队不怎么出头管那些小事了,自然在当地的威望就难免降低,很多人就发现对这支头戴蓝盔的部队可以采用无视的态度,再后来就有人发现在这些秉性内敛的中国人面前出点小格也招不来什么麻烦。于是就有爱卖弄的年轻人在适当距离上偶尔对部队小小挑衅一下以表现自己的勇敢。

当然驻地里的中国人并不知道这些,而刚卡人更没有义务去通知这些外国人,虽然当地政府的一些官员终于也知道了这些事,但在他们看来,这实在是和中国人天性爱忍让的特点有关——连电影上都那么演呢,再说像美国和英国那么嚣张的维和部队实在引不起当地官员的好感,他们自然不会希望中国人有样学样。

日子就这样一天天过去,天气渐渐转冷,楚云飞依旧隔三差五的去巡逻,蒂娜也是日复一日的来驻地,当地群众慢慢习惯了在规定的时间里来看病和打水。当两族冲突再起的消息从维和总部传到中国蓝盔的时候,成树国的那个小队居然成功阻止了将近500人参与的小规模冲突,获得了嘉奖。

这天,三小队去执行抽查任务,由于这次抽查范围比较大,很晚才回来。驻地门口聚了一堆人,蒂娜也在里面,楚云飞看看手表:五点二十,正是群众打水和看病的时间啊,发生什么事了?

原来这天有一分队二小队和其他工作人员去坎塔卡办事,办完事照例给了大家两个小时的活动时间,二小队的战士同时也是建业军事学院步兵指挥系的尤伟和两个女兵凑成了一组。

他们一行三人逛来逛去,终于在一个出售各种小装饰品的摊位前看好了几样小玩意,当打算掏钱购买的时候才发现价钱贵得离谱。买不起那只好走人了。

磨刀霍霍的摊主自然不肯就这么放三人离开,再说是人都知道中国人胆子小,威胁谩骂了半天才悻悻作罢。

麻烦的是其中一个女兵向朝霞长得相当漂亮又有点非洲人的特点:嘴唇很厚,眼睛很大,身材也非常好。三人又是面对摊主的侮辱没表现出什么火气,就有四、五个黑人盯上了他们。

等到他们三人走到人少的地段,这帮人就围上来调戏那两个女兵,尤其是向朝霞,被三人围住笑嘻嘻的打转,虽然黑人们还不敢对女战士进行直接的身体接触,但那股恶心劲是个人就受不了。

作为三人里唯一的男人,尤伟是肯定不能坐视女同胞被人欺负的,可又不敢开枪,于是和众黑人扭打起来,但他是步兵指挥专业的,虽说身手也不错但还是有所不足,向朝霞是公安系统选送的女警察,也能打两下,但最终尤伟还是被打了个鼻青脸肿,亏得向朝霞拿出小手枪向天鸣枪才没有更严重的后果发生。

无疑这是件很严重的事件,带队的吕副大队长也不敢自行做主,立刻把队伍带回了驻地。和吴大队长商量一番,几个领导都觉得现在部队有些过于忍让了,马上向上级做了汇报,等待指示的同时暂时中止和当地居民的频繁接触。

同个学校的战友受伤,楚云飞自然要去看看,虽然他跟尤伟不是特别熟悉,但看了尤伟的伤势还是恨得牙根直痒:那帮家伙下手实在太狠,打得尤伟两眼充血鼻梁下陷,两颗后槽牙也松动了。

一时间,驻地里群情激愤,一致要求领导们必须有所表示,当领导们把部队里存在的不稳定情绪再次向上报告后,没过多久,处理建议就下达了:当战士们受到人身攻击时,有反击的权利,但是禁止采取过激行为把事情激化。

操,这不是和没说一样?当吴大队长在骨干会议上一宣布上级指示,在座的都是这个反应。就有几个人把眼珠转向了以点子多出名的楚云飞。

楚云飞也不满意上级这么不咸不淡的指示,眼珠一转,有了!

当吴队长征求骨干们的意见时,楚云飞喊声“报告”开始发言,“我觉得我们可以合理的运用上级的指示,我们不主动惹事,但是可以诱使对我们不怀好意的当地人主动攻击我们。”

这主意其实很有几个人想到了,但建议是由楚云飞提出来的,这“好战份子”的帽子他可就戴定了,不过以楚云飞的性格,他才不在乎别人怎么说呢。

吴大队长很满意有人出头提出这样的建议,“这个……合理运用上级的指示,这个建议我看可以考虑一下,虽然我个人认为不太合适,但每个同志的意见我们都要尊重,大家表决一下吧。”

其实表决的话,谁都不会有什么风险,风险最大的反而是提出表决倡议的吴大队长,所以他必须认为这个建议“不太合适”。

表决的结果那自然就是楚云飞的建议被通过,毕竟是在部队,热血男儿实在是不少的。

\section{第四十四章 成功的反击}

其实大部分的事情总是第一步难走,既然有了应对的策略,具体细节就很好计划了。

经过维和部队骨干们的策划,若干执行方案就出台了,倒不是一定要抓住那几个行凶的家伙,而是要给所有人一个警示:千万不要招惹中国人。

主体思想就是示敌以弱。把部队里有数的几个绝对好手合理分配一下,让他们分别带个漂亮女兵单独行动,在遇到挑衅时先发制人,同时保持和大部队的联系。

也有人提出可以适当的利用下本地人,例如经常缠着高建军的蒂娜,找机会让他俩出去走走,同种的美少女被懦弱的中国人得手,肯定会有人看不顺眼的,不过这个建议似乎有点触犯其他纪律而被否决了。

成树国更离谱的建议居然会被大家采纳:鉴于高手楚云飞同志身材偏瘦,皮肤白皙,喉结也不明显,强烈希望他能乔装改扮,单独行动,以显我中华巾帼英雄形象,也为以后难保会落单的中国女军人多提供点安全保障。楚云飞不由暗自琢磨什么时候得罪了自己的校友。

不过某些人的便宜天生就不是好占的,楚云飞立刻就想起有个给部队卖菜的胡图人的小女孩很喜欢跟成树国一起玩耍,还骗去成树国不少小东西。于是他马上建议可以让成树国带女孩一起出去玩耍,女孩才5、6岁自然不会有什么恶劣影响。当然成某人身手稍微欠佳是以保护小女孩为主,同时需要再来个战士配合。最关键的是小女孩必须时常的哭哭啼啼以显示无助,这下看你成树国再琢磨着害人!

不过这次楚云飞倒是低估了成树国的觉悟,这家伙的“大汉沙文主义”倾向实在是很严重,居然很愉快的接受了这个建议,还表示自己有若干保证女孩持续哭泣的窍门。既然成副小队长都认可了,这个建议自然也就通过了,不过还是有人严正告诫成树国必须保证女孩的安全。

本来有人提议给这个行动起个“反击”之类的名称,但是大家都以看白痴的眼光看着他:这事方便声张么?你丫居然还要起个代号?

会议结束后,楚云飞悄悄的问成树国,“要是你自己的孩子你舍得么?”对方狠狠瞪他一眼,回答却是:“要是我的孩子就更好说了。”楚云飞无言,军人世家果然是够铁血的。

于是再次进入坎塔卡的时候,维和部队的人去的就多了那么几个,当然开始要办的事情照旧办理,然后就是杀气腾腾的俩小时自由活动时间。

虽然坎塔卡的居民很多,但是这将近二十个黄种人还是掀起了惊天的动静。其实参照我们国家的现状就可以知道:当有事情发生时,看热闹的绝对比参与的人要多得多。更何况刚卡人这种心态比起国人来还要严重很多。

这里在打个惊天动地。打完没多久另一处又打起来了,此起彼伏,叫一干闲人大呼过瘾。

自然也有那些有民族气节的黑人看不顺眼黄种人的嚣张,不过一打问总是自家人无理在先,再考虑考虑动手的是维和部队而不是中国游客,那肯伸手相助一同挨打的可就屈指可数了。

又有一些有心的刚卡人注意到:A处部队打的是胡图人,B处被打的是图西人,C处挨打的居然是西瓦族某官员的公子,绝对没有任何的厚此薄彼。居然就有人考虑如果中国蓝盔保持这种状态,是不是该诱导他们对某些地方重点照顾一下?

自由活动了一个来小时,行动就进行不下去了,当地只会敲诈勒索的警察实在是不能无动于衷,纷纷将战士们劝解回来,虽然有些警察也打着歪心思,但是吃公家饭的一般还是知道个轻重的,这事只能上报而自己实在做不得主。

市政府的一众黑人官僚自然知道自己对维和部队说的话分量有限,但大事化小的能力还是有的。于是够分量的官僚纷纷消失不见:他们实在是怕维和部队借题发挥,象那些臭名昭著的部队一样索取些什么特权。只有个类似副秘书长什么的官员来安慰“受到不公平待遇”的维和士兵。

在鸡飞狗跳中,中国维和部队度过了愉快的一天。而坎塔卡人这天中谈论最多的就是:中国功夫果然是名不虚传。

部队回到驻地,相关的领导自然还是要有相应的担心,既然战士没有遭到什么伤害,那剩下担心的就是刚卡政府的反应了。过得一天,坎塔卡反馈过来的信息一如所料:在为市民道歉的同时,还顺带慰问了驻地部队。那驻地向上汇报的时候自然也就知道该偏重什么方向了。

虽然驻地有个别人还是对部队目前的思想倾向不太赞同,也暗自向上级反应了真实情况以求得洗刷自己。但是维和部队的组成成分实在是有点复杂,对任何人而言,用微妙的平衡以求得中庸实在是个不错的选择。

令坎塔卡官员严重不安的是:中国维和部队似乎忽然间物资严重的短缺了起来,本来四五天才来一次的蓝盔部队现在基本上是隔天就来一回。过得几天,坎塔卡市无论是没长眼的白痴,还是为错过给黑人同胞出气的壮士,还有那些摸不清楚状况的外地人,都在经历了痛苦的亲身体验后得出了明智的结论:对于中国人,还是少招惹为妙。

当然这还是不能满足不了愤怒的士兵的胃口,用他们的话说就是毛主席都说过“打个五十年的和平出来”,实在是“宜将剩勇追穷寇,不可沽名学霸王”。于是第二阶段的计划开始执行,楚云飞开始化装后独立行动,而坎塔卡又似乎突然间多了几个喜爱拐卖黑人女孩的中国士兵。

这一试还真见效果,楚云飞的化装证实了部队的担心:还是有不少当地人对黄种人抱有敌意,甚至不排除是因为前一阶段的行动导致的报复心理的可能。对这种欺负单身女同胞的杂碎楚云飞自然不会留情,下手是相当狠的。而黑种男人的劣根性此时却显现无疑,居然大部分人是因为挨了女人的打而不好意思声张,只有在被人瞧见的时候才能使消息传开。等到大家都知道中国女人也会功夫的时候,那个假女士已经遭遇了五十多起“突发事件”了,而他的行动直接导致维和部队的女兵在人多的地段拥有充分的安全保证。

而成树国的行动效果就不算完美了,开始的时候还有众多黑人来管这“不平事”,成副队长一伙也不能下太重的手,虽然这里面肯定有借此泄愤的家伙。

日子久了大家才发现:什么时候总有正义感过剩的家伙。尽管每次小女孩都声明“黄叔叔”是自己认识的,但是成树国每次出现还是能“成树敌”——成功树立敌人。经过大家讨论,他们这一行实在已经没有继续“诱导”下去的必要了,因为没准会坏了部队名声呢。

又没过多久,楚云飞的任务也执行不下去了,尽管他每次化不同的妆,还经常的再拉上个很能打的女警察混淆别人的视线,但是终于有一天一个摊位的老板拽住他,要“她”帮忙教训几个人,“钱不是问题,我知道你很厉害。”

黄种人在黑人眼中实在是不容易分辨出张三李四的,但居然有人能认出“化不同妆”的楚云飞,那他的任务实在没有继续执行下去的必要了,当然该老板的请求楚云飞也没有答应,虽然存在是陷阱的可能,但楚云飞只是很不屑的斜视了他一眼掉头走了,他装女人自然不方便说话,但那一眼已经充分的表明了意思,“你以为维和部队是你能指使得动的么?”

\section{第四十五章 残酷的镇压}

为期将近一个月的“反击”行动结束了,换来的是当地人对中国军人无与伦比的尊重,从这点上来说,其实在很多地方“仁义之师”未必就能获得真正的尊敬。

行动带来的另一个效果就是很有些人跑来驻地拜师学艺,其中以少年居多,但也不乏年纪大点的,居然还有几个女人,那些女人学习功夫的理由却都是因为被丈夫打得忍受不了。

当驻地负责人好言解释之后,大部分人失望之余纷纷垂头而去,可依旧有那执着的在附近找到住的地方天天上门来骚扰,这种风潮一直持续到了维和部队的离开。人心都是肉长的,其中果真有那苦心人真的学到了一招半式。

成树国的任务结束得早些,所以等楚云飞回来的时候,他已经又开始了日常的巡逻任务,就在楚云飞结束任务的当天,成树国又遇到了立功的机会。

二分队巡逻的区域内有两个分别属于胡图和图西族的小部落结怨很深,两部族经常在隐秘之处爆发点小的冲突,虽然这些冲突大多是无心之举,也导致了双方各自缩小了各自的控制范围,但还是难免偶尔相遇。相遇之后最好的结果也是一方暴打另一方一顿,至于死人也是家常便饭。

这天胡图族的勇士勒勒瓦跟自己的新婚妻子大吵了一架,吵架的原因是勒勒瓦把自己妻子从娘家带来的一条精美项链换酒请朋友喝了。勒勒瓦的妻子是另一个小部落里有名的美女,家境也不错,还去刚卡的首都摩沙上过学,受环境影响,就多少有点女权的意识。

而勒勒瓦则是典型的胡图男人,虽说这对夫妻很有点郎才女貌,但意识不同,过日子就难免磕磕碰碰。今天的事真的把妻子气坏了,执意给丈夫点颜色,就赌气向族人说的“危险地段”走去,指望勒勒瓦能去喊她回来。

但勒勒瓦作为个勇士,怎么可能做这么没面子的事?自然是说了些“有种你真的过去”那样的气头话。等天色渐暗,勇士的气也消了,却依然不见妻子回转,那自然是要着急的。

勇士的嘴依然硬气,但是架不住同族人相劝,终于放下面子喊上一票人带上武器去寻找妻子。在寻找半天后终于寻到了妻子的尸体:美丽的妻子全身赤裸的躺在坚实的土地上早已没了气息,放火赶开妻子身上的蚂蚁,却见她身上还存有多个男人留下的斑斑印记和干涸了的液体,胸脯和眼睛被人残忍的挖去。

勒勒瓦是真心爱着自己的妻子的,如此奇耻大辱也不能就这么罢休的。至于是谁干的那是估计连狒狒都猜得出来:除了他们图西人谁还干的出来?

于是勇士带着一票人手持武器日夜不停在危险地带寻找机会,危险的地方自然是人迹罕至,但从而也有了值得冒险的价值,终于被胡图人抓住机会逮住了一个落单来偷打瞪羚的图西人。当然敢在这里单混的人也不是普通人,勒勒瓦因为追他差点被带麻醉的箭射住喉咙,还好带得人多没叫对方跑掉。

没有经过什么拷打,图西人就把自己知道的都说了,因为勒勒瓦答应痛快地杀死他。

原来勒勒瓦的妻子是被经常来打猎的一大帮图西人做掉的,领头的是勇士阿鲁达,阿鲁达的父母死于二十年前的两族冲突,从小就是孤儿的勇士阿鲁达实在是对胡图人有着刻骨的怨恨,否则的话勒勒瓦的妻子那么漂亮,真还未必会死。

事情发生就发生了,对阿鲁达来说,他杀了也不止一两个胡图人了,至于勒勒瓦妻子临死前的威胁他也没当回事:老子早就赚够了,勒勒瓦又能怎么样?我这一大帮人还怕他不成?阿鲁达根本就不相信对方敢在维和部队经常出没的危险地带搞大规模冲突。

勒勒瓦自然也听说过阿鲁达的恶名,同样也不服气他,但是勒勒瓦可是知道那一大帮人真的不少,起码有四十多号人。本来阿鲁达有意的话完全能组织上六七十个朋友来袭击他,但是在这一大片的平原上,又有随时可能出现主持公道的维和部队这种劣势方的希望,对于有丰富战斗经验的两族人来说,战斗绝对不会在五个小时内结束的,时间要是再长点都够维和部队来两次了。

于是勒勒瓦就去找自己的岳父借兵,两个大兄哥差点撕了他,还好脾气最坏的小舅子去摩沙上学去了。最终看在他是为妻子报仇的份上,两个大兄哥收下他的礼物,又喊了二十多个亲戚朋友前去助阵。

经过多天孜孜不倦的等候,勒勒瓦终于等到了阿鲁达一行人,双方就展开了激烈的战斗,子弹横飞,箭枝乱舞,等到眼看阿鲁达一方只剩下七、八个能活动的人的时候,刘宁和成树国带领的第一小队出现了。

按理说这时候双方能放下枪的话,双方只需要把武器上缴就可以了,除非接下来几天这里还有类似规模的战斗发生。可勒勒瓦的人死了七八个,一个大兄哥也胸部中弹,所以他不管三七二十一,在双方都停止射击的时候冲着阿鲁达又连开几枪,这几枪立刻吸引来了维和部队的注意:几颗子弹打得他脚前尘土飞扬。

其实勒勒瓦不打那几枪,地上二十多人的尸体和四十多的伤者也足以让维和部队的人把他们中间的带头者拘回去了,但中国维和部队一向是能中止的冲突就不加深双方的怨气,因为拘回去的话,肯定有人是会掉脑袋的。但勒勒瓦这枪开得太嚣张了,中国蓝盔想忽视都不行了,于是不得不把双方缴械,并且通过吉普车上的电台呼叫驻地的支援。等到支援部队到达,全部人都押回军营,第二天全部押解到坎塔卡,轻伤的都送去了,只留下十几个重伤员在驻地治疗。

由于很久没发生这么厉害的冲突了,坎塔卡政府处理得极其严厉,所有被押解过去的参与者七十多人均以“破坏和平”的罪名被枪毙,只有在中国军队驻地养伤的那十几个侥幸躲过了劫数。

虽然大家执行的是维和任务,但一次性这么多人的人头落地,还是让大部分中国士兵感到了现实的残酷。但是没办法,非洲很多国家都是这么对待种族冲突的,如果不采取如此的高压政策,很难想象现在的非洲究竟是什么样子。

\section{第四十六章 情报员蒂娜}

当然,作为士兵的中国人是不能为他们的职责抱怨什么的,再说他们的任务也只是制止冲突,具体应该怎么善后是不由他们负责的。

但是,该做的防范还是要做的,毕竟死了这么多人,虽说不是维和部队杀的,但是有人迁怒到中国人头上那也不算意外。

所以,在事情发生后的几天里,不但驻地全面警戒,派出去的巡逻队伍人数也骤然增加。本来每个分队都有装甲车和吉普车轮流休整和保养的,现在也全部派了出去,二分队还多派了辆装甲车,所有巡逻队伍电台也都是带了两台以备万一。

但是几天过后那两个部落并没有什么特别过分的举动,看来这次又如往常被扑灭的星星之火一样,可能产生的大规模冲突被残酷的镇压浇灭了。

于是巡逻队伍的人数又回到了以往的规模,但是做出这个决定的商大队长很快就后悔了。

这天又是一小队巡逻的日子,将近中午时分,营地里开始自由活动了,楚云飞刚想去洗衣服,却意外的发现蒂娜站在驻地门口焦急地东张西望。楚云飞有点纳闷,高建军是一小队的,蒂娜应该知道他不在呀。于是他走到门口问,“有什么事么?高建军今天巡逻。”

蒂娜却是眼睛一亮,慢慢凑到楚云飞跟前,“我想要两盒清凉油。”说着拉起楚云飞的手恳求着,却趁机把个小东西递到了楚云飞的手里。

尽管楚云飞知道她肯定有事,还是被少女的亲热劲弄得有点招架不住,不过心思马上又被手里那个小东西吸引住了:是个纸团,有问题!

惊愕只是那么一瞬间的事,楚云飞马上恢复了正常,很自然的向四周看看:附近的几个黑人没人注意到这里。于是双手借摸口袋的功夫把纸团藏起,又笑嘻嘻的冲着西瓦少女摊开双手,“我身上没了,你在这里这里等我,我回去拿。”

走进营房,楚云飞拿出纸团,是打印纸,摇摇头:肯定是蒂娜跟高建军勒索的,这个老高还真是够闷骚的。

摊开那一小块纸,纸上只有三个歪歪扭扭的英文单词,内容却是惊天动地:hutooarmyfight!!!

镇静!楚云飞先暗自对自己喊了声,然后大脑飞快的转动:胡图族的,还是有武装的,要打仗。

首先可以确定这个战斗是对中国军队而言,否则求救的话也轮不到蒂娜来张罗这事,其次,蒂娜这么神秘兮兮的来传送这个消息,肯定是怕别人知道,换句话说,也就是很有可能营地门口就有人知道这事。至于他们为什么不说,肯定是怕别人传出去对报信者不利——算,先不考虑这事。

胡图人的军队?现在刚卡军队都是掌握在西瓦人手里的,胡图和图西两族在部队的地位还不如阿拉伯人。外国胡图军队打来也不可能,那驻地早就该知道了,那就应该是胡图人的武装!

胡图人的武装要和中国军队战斗!袭击驻地还是袭击巡逻队?要是袭击驻地门口不可能还有那么多本地人,三族的都有,尤其还有可能不止一个人知道这事,谁会跑到不安全的地方送死?那肯定就是袭击巡逻队了。

再想想蒂娜的表情,联系一下前些日子发生的镇压,楚云飞再笨也弄明白了:胡图人要报复一小队!

楚云飞哪敢犹豫,也不走门,从窗户里就翻了出去,顺着营房后墙跑到商成钢的办公室下,刚要敲窗户,却想起来商大队长一大早就去坎塔卡了,于是又赶紧跑到冯副大队长办公室,探头一看,还好人在,赶紧敲窗户。

冯副队长正坐那里写着什么东西,听见窗户响,抬头一看是楚云飞,自然纳闷这家伙怎么不走门口喊“报告”进来,却看见楚副队长还在执着的敲着窗户。

这家伙,有急事也要讲究礼数才对,冯爱华抱怨着打开窗户,训斥的话还没开口,楚云飞已经抢先说话了,“报告冯大队,紧急情况。”说着就把手里的纸条递了过去。顺势翻进了冯爱华的房间。

冯爱华被他的话震得顾不上训人了,拿来纸条仔细一看,沉吟一下,马上反应了过来,“这东西哪里来的?”难道眼前这家伙在这里还建立了情报系统?

高建军和蒂娜的事整个驻地的人都知道,楚云飞很直接的汇报,“刚才温柔偷偷递给我的。”——“温柔”是驻地部队给蒂娜起的绰号,她的真名反而知道的人不多。

冯副队长主要负责的是政工方面的事情,思考了一下,“这商大队出去的还真是时候,说说你的想法。”

楚云飞的答复很利落,“我认为极大可能是胡图人要报复一小队。”

冯爱华有点手足无措了,真是怕什么来什么,有心把“温柔”喊进营地问问,却被楚云飞阻止了,“这样不好,她似乎很忌讳别人知道。”

冯爱华想想也是,平时除了做买卖的,没什么人会主动被叫进营房,出去找人家谈又不合适,“要不让卫生队偷偷问个病人,小楚你去卫生队找游队长进来。”

“不行,我怕别人看见我来你这里才没敢从门走。”

冯爱华马上跑到门口喊了个人过来,“你去把游队长喊来,快点。”楚云飞在屋里补充,“通讯站也喊个人来。”他却不知道门口站的是他的领导吴分队长。

卫生队游队长反馈的情况更说明了问题的严重:今天居然没病人来看病!不过通讯员倒是顺利地联系上了一小队,警报马上发了出去。

楚云飞从房间里出来,手里一上一下的抛着两盒清凉油,笑嘻嘻的走到蒂娜面前,少女伸手去拿,楚云飞却把手往身后一背,微笑着探头过去低声问,“是今天么?”

蒂娜明显地慌乱起来,“给我。”这情况看在别的黑人眼里却是中国人似乎在勒索蒂娜,中国人什么时候也开始占美女便宜了?

楚云飞又问了一遍,蒂娜看到四周黑人有人注意到了她,更慌乱了,语气变得生硬,声音也大了起来,“快把清凉油给我!”

楚云飞意识到不能再这样下去了,马上把清凉油递了过去,“开个玩笑,开个玩笑而已。”说完掉头就往回走,至于他开了什么玩笑,让西瓦少女自己去编吧,反正不会有人来问他。

事情其实很简单,前一天晚上,蒂娜不小心听到族里男人在喝酒时议论胡图族请了他们的游击队“哈伦”来报复中国人,西瓦少女敏感的意识到她心爱的人参与了上次的镇压,于是就悄悄的多听了几句,知道第二天游击队就要下手了,胡图人的消息确实准确:正是轮到第一小队巡逻了。

蒂娜很早就知道巡逻队是一早出发,她是见不到高建军本人的,也不敢那么赤裸裸的去通知中国人。西瓦少女心系心上人的安危,思来想去,少女的痴情还是战胜了对未知报复的恐惧。

蒂娜打好主意,却由于英文水平有限,只会说不会写,费劲了心机才写了三个单词,指望在门口能遇到两个熟悉点的人递进去,最好是那个“滑头瘦子”能注意到(语出高建军),没想到真的如愿以偿。

这里要说说“哈伦”游击队,这支胡图武装装备比较精良,人数将近10000,又各个都是心狠手辣之徒,是历次图西族大惨案的制造者之一,是刚卡当局严厉打击的非法武装之一。自从胡图族、西瓦族和政府三方宣布停火后,又因为会有维和部队相继进入刚卡,该游击队在上缴了部分武器后就消失了,大部分成员“藏兵于民”,只剩下少许精英分子藏匿了起来。由于行事变得低调,又有众多胡图部落为其做掩护,再者也没继续制造惨案,居然就再没人注意他们的行踪了。

至于为什么西瓦族能知道这些事,那就要从胡图族的人数说起了,自从胡图族在刚卡的人口数上占据绝对上风后,西瓦族就不得不与胡图人交好了,中立只是表面现象。虽然也有图西部落和西瓦族交好,但谁也不是傻瓜,去自找麻烦的招惹强大的一方。

这种倾向哪怕维和部队来了都不会因此而改变,原因没别的,维和部队迟早是要走的,而西瓦族还是要继续在刚卡生存的。

\section{第四十七章 猪头小队长}

主事的商少校去参加个西瓦高官的婚礼,陪同他去的还有5、6个战士。本来这种事情是应该冯副大队长出马的,可西瓦高官不是纳妾,是正妻死后的续娶,很正式的场合,又盛情邀请商大队长,不去似乎也不合适,起码不利于团结。

所以扮演“政委”角色的冯爱民就成了驻地的军事最高决策人,但此人搞搞思想工作,写写总结,做做汇报什么的还算拿手,可真要把偌大个队伍操练起来就有些勉为其难了,更别说遇到这种即使商少校在也要头疼的事了。

冯大队没什么决断力,但博采众家之长的能力还是有的,不但马上向上级汇报,同时持续的联系商少校,而且还招来了几个分队长和小队长火线开会。

这会不开还好,一开肯定是更不知道怎么办了,大家自然是众说纷纭:首先情报的来源就很值得怀疑,其次有那个新规定更制约了大家的主观能动性,再次就是到底这个袭击的重点是驻地还是巡逻队?重点是驻地的话出兵增援刘宁小队自然是自寻死路;重点是巡逻队的话,那该出多少人增援?人少的话显然不合适,逐次增兵更是军中大忌,要是出动人多了,谁能保证驻地不受袭击?

上级的指令显然是不太可能那么及时的传来,当然更不排除有人在更高的级别开会的可能。最后还是通讯员联系上了那个正在结婚的西瓦官员的随从,才得以联系上驻地部队的最高指挥官商成钢,而专门带了台电台去找领导的三个士兵也在那时踏进了婚礼现场。

商少校惊闻此讯,自然马上做了安排:先安排第二分队第三小队先行出动向第一小队靠拢,剩下的问题等他回去解决。

贵客匆匆向主人请辞,自然引起主人的不快,想到该事很可能升级为更严重的事件,商成钢很痛快的向主人坦承了遇到的问题,没想到主人也跟着一惊一乍起来。

接到命令的孔繁茂小队几乎在同时就出发了:大家已经等了有半个小时了!他们带了两部电台,一部在前面的吉普车上和一小队保持联络,一部在后面压阵的吉普车上与驻地保持联系,吴分队长在中间装甲车中,整个小队自然还是通过步话机联系。

三小队走了半个小时,电台里传来成树国的声音,“313,24这里又发现了小规模的冲突,人数30人左右,全是轻武器,我们要干预了。”

楚云飞是三小队的先锋,马上就骂了起来,“猪头,你要找死换个时间行不行?”这个成树国脑子进水了不成?人家是怕大平原上,你不路过那里专门勾你过去的,这都想不到?

成树国自然没那么白痴,“这件事我和刘宁已经沟通过了,决定不能因为不确定的情报就自动放弃军人的职责,夹着尾巴逃跑不是我们中国军队的传统,再说他们就在我们的正前方,难道你要我们拐弯不成?”

楚云飞气得大骂,“操,这么重要的事你们自己就做了决定?和驻地请示没有?”

成树国也骂起来,“操,就你有脑子啊?不请示我们敢么?懒得理你。”

成树国确实向驻地请示了,但以他和刘宁那惟恐天下不乱的性格,自然是把对方贬低了一通,又把远在自己北侧三百余米的人群挪到了“正前方”,两个队长都是军人世家出身,深知和平年代没那么多立功的机会。

而驻地商大队长那里也有自己的考虑,纵然情报属实,但和对方三十人左右的小部队接触一下还是很有必要的。一是可以获知对方的攻击重点来从容调整兵力,二是对方的装备、作战能力、来头都是必须要弄清的,既然这俩刺头执意要立功,那就由他们去吧,能弄到几个活口就更好了。反正按照人数来说,这支武装应该撼动不了一支小队,何况三小队也在支援途中。同时为保险起见,商少校还是下了“先轻微接触一下,如果情况不对马上撤退”的命令。

一切就在这阴差阳错中进行着,一小队为了保险起见,由装甲车带头赶到现场,向天鸣枪后对方果然没有缴械,所有黑人立刻趴到了地上,做出了战斗姿势,成树国马上呼叫三小队的支援,开着吉普车冲了过去。

“嗒嗒嗒”吉普车的车载机枪把地下打得尘土乱飞,对方不但没有接受这最后通牒,居然开始了还击,而且头一声沉闷的“嗵”声更说明了对方还有重武器。“枪榴弹”——成树国马上判断了出来,大喊一声,“跳车!”:这东西打不动装甲车,自然是冲吉普车来的。

不过对方的准头实在是欠佳,隔着不到一百米居然把枪榴弹打飞,在爆炸声中却有人“嗷嗷”直喊,中国军队的还击中,成树国回头看看是谁那么倒霉,却意外的发现中弹的居然是黑人:身后四十多米处,冒出了十来个黑人,从他们一身的尘土就可以断定是藏在土坑中的。

装甲车内的刘宁马上发现对方多了二十来人出来,果断的命令:“撤退,保持队形,我断后。”

等到成树国重新跳上车,却发现另一辆吉普车已经被另一颗枪榴弹的爆炸掀翻了,三个战士正在从车内往外爬,看来没人阵亡。成树国马上命令吉普车去救援,期间还要注意枪榴弹的声音。

装甲车不但用车载机枪进行火力压制,车内的战士也纷纷通过射击孔向外射击,以保证成树国的救援行动。

等到成树国把那三人拽到车上,却发现那些埋伏在土坑里的黑人纷纷拥了过来,于是一边机枪扫射一边观察突围路线,却发现黑人们扬手扔出了十来个圆盘,就落在了他们的退路四周:反坦克地雷!

怪不得看到有装甲车还敢袭击,原来是有这东西,成树国马上判断退路是走不得了,正想汇报,却听见刘队长的声音在步话机里响起,“操,反坦克地雷,居然扔我前面?是不是后面也有了?”成树国马上回答,“嗯,看来我们立功的机会到了,大家该努力了!”

以两辆吉普车和装甲车为支撑点,第一小队开始了艰苦的抵抗,很快有辆吉普车被枪榴弹击中,吉普车油箱也爆炸了,连环爆炸中,虽然一个冲在最前面的黑人挂彩,对一小队一方却造成一死两伤的伤害。

成树国马上命令车外的两个战士把伤员拉到装甲车内,自己掩护。没办法,装甲车外必须有人,电台一进装甲车就会被屏蔽,那装甲车的天线是给步话机用的!再说,装甲车里射击孔的死角实在多了点。后来刘宁也意识到了这点,又有两个战士被派出来支援成树国,其中高建军爬到了成树国旁边,另一个基本上是以扔手雷准著称的,看来战斗真的能促使人成长的。

没过多久,装甲车顶的机枪手也中弹了,刘宁露出身体接替了机枪手的位置:这挺双管机枪太重要了,现在可全靠它压制对方火力呢,成树国有样学样,把吉普车的机枪拆了下来,时不时的扫射一下,虽然地面不太平整,但强劲的火力还是不容对方质疑的,后来高建军专管为成树国压子弹。

车内的狙击手发挥了很大的作用,虽然找不出哪个是对方重要人员,还是清除了胡图人队伍中不少有威胁的火力点,其中还有两个试图投掷燃烧瓶的。

但是对方足足有6、70号人,人数实在是太多了,必须保持强劲的火力压制,否则被胡图人靠近大家只有死路一条。于是打着打着大家就发现:子弹不太够了。

本来维和部队巡逻惯例是带一个基数的弹药,后来发生了那件事后,驻地命令所有巡逻队伍必须带两个以上基数的弹药出巡。只是到了最近大家才又带成一个基数了,一小队两个队长都嫌麻烦,想跟大家看齐,可架不住高建军苦苦劝戒,目前携弹量还是两个基数,可目前这弹药消耗实在太厉害了。

成树国拿出手雷,对高建军说,“你别压子弹了,先帮忙看着。”说着丢下手雷拿起电台耳机,“快点快点,子弹不多了,再有半小时我他妈的只能喊‘向我开炮’了。”

电台里传来令人高兴的回答:“马上就到,我们都听到枪声了。”

\section{第四十八章 小酋长死了}

“哈伦”游击队的战术布置得很合理:由于巡逻队在平原上没有固定的出巡路线,不便于埋伏,就先用部分战士乔装成部落武装伪做冲突。等车队被吸引过来,用枪榴弹打掉两辆吉普车,打掉吉普车中国人就不能通过电台求援了。然后埋伏在附近的战士投掷反坦克地雷,困住装甲车。虽然游击队近期销声匿迹,并没有能击穿装甲车的重型武器,但是一旦困住了对方,就可以从容的用燃烧瓶等土制武器来消灭装甲车了。不过装甲车的射击死角虽然多,但真能构成火力网的话,那就通过不断的冲击一边消耗对方的弹药,一边冲到装甲车附近,看能不能用手雷炸药包什么干掉它,哪怕是只炸掉履带也是好的,却没想由于弹药匮乏和准头欠佳,带来的枪榴弹用完也才只击毁对方一辆吉普车,还好大致的战术设计还是实现了。

不过从另一个角度讲,中国士兵的素质也是很让“哈伦”游击队吃惊的。在一年多以前,同样是维和部队,孟加拉国的15名士兵被包围后虽然也没有举手投降,但是低劣的战斗素质让游击队很快就结束了战斗,有一辆吉普车载着六名士兵逃跑了,剩下的战士包括领队的排长被残忍的游击队虐杀了。可今天的中国士兵居然还有胆量在装甲车外策应,形成了严密的火力网,并不因为游击队人多就躲进“乌龟壳”里,都是黄种人,差距咋就这么大呢?

不过还好消耗对方弹药的计划基本上已经达到了,听着逐渐稀落的枪声,游击队又组织了一次小冲锋,结果对方的射击又密集了起来:不管怎么说,看来对方的弹药是不多了。

游击队正在为中国人的命运进行倒计时数数,远处矮树上的了望哨却传来了不好的消息:“自西面和西南面来了两支机械化部队,西面部队不多,看来是中国人的部队,西南面人数很多,象是政府军。”

现在这样子哪怕再来支中国巡逻队,游击队也难逃被击溃的厄运,何况还有大部队?看着包围圈里的装甲车,带队的游击队长叹一声:好运的中国人!阿罗神在上,要是能多给我点弹药,一定能弄死这些可恶的黄种人!

队长不再犹豫,“我们撤,把那俩西瓦人留下,其他的战士能走的马上走,不能走的就让他们为阿罗神献身吧。”

于是等到第三小队赶到的时候,本来要在外围侦察一下的,却只见第一小队在收殓牺牲的那个战士的尸体,还有两个黑人在中国士兵的枪口下瑟瑟发抖,四周残留着将近二十具胡图人的尸体,遍地的弹壳和弹头,还有爆炸留下的弹坑、冒着袅袅黑烟的草根和吉普车。

刘宁头部被颗子弹擦过,流淌下的鲜血已经在脸上干结,坐在那里抽着烟,看着战士们救护伤员,装甲车的那个机枪手此刻也是出气多,进气少了;成树国倒是活蹦乱跳的带着高建军和另一个战士在四周收集着那些地雷。

孔繁茂走过去,从刘宁口袋掏出烟来,给自己点上,深吸一口,“人呢?都跑了?”

刘宁点点头,本来他的脾气就不能说好,现在自己的战士牺牲,自然更没好气了,“这些杂碎跑得真快,现在追没准来不及了。”

孔繁茂很惊讶的看着他,“你现在还想去追?”这话意思很明白,你能保证不是个更大的陷阱?

刘宁脸上的肌肉不自然地抽动着,“我迟早要收拾这些杂碎们。”

楚云飞走了过来,“刚卡政府军到了,来了一个连。”

这支政府军的出现不是偶然的,他们就是驻守坎塔卡那个团的部队,虽说里面肯定也有人知道游击队要袭击中国维和部队,但怎么可能声张?当举办婚礼的那个西瓦官员把中国部队可能遇到袭击的事情汇报上去的时候,坎塔卡驻军就被调动了起来,进入紧急战备状态。同时又派出了一支机动能力较强的部队来支援中国人。

带队的西瓦连长上来和吴分队长打了个招呼,热情的问起了中国军队的损失,还询问需要有什么帮忙的地方没有,于是心急火燎的吴分队不但要张罗部队打扫战场、询问情况和汇报驻地,还要心不在焉地应付西瓦人。

西瓦连长呆了一阵,看到帮不上什么忙,刚要打个招呼带部队回转,却听见有人喊,“连长,这里还有俩西瓦人。”

这俩西瓦人是陪着自己的主人来的,他们的主人是个西瓦部落酋长的儿子,年纪只有11岁,很得酋长的喜爱。小男孩听说了“哈伦”游击队要偷袭中国人,仗着自己部族和胡图人的良好关系,一定要来看看,酋长怎么拦也拦不住。

倒不是酋长不疼自己的孩子,实在是由于长年的战争,很多部落里十一、二岁的孩子已经能抗着沉重的AK47上战场作战了,所以深受宠爱的酋长公子有这个念头倒也能被大人接受。

可酋长的孩子年纪小了点,又娇生惯养的没有什么作战经验,拿着一支步枪趴在那里傻乎乎射击。被一颗飞来的机枪子弹命中头部,大半个脑袋都炸得不见了,两陪同当时就傻眼了。

等到游击队要撤退的时候,实在是不放心那俩陪同跟着,这年头要没有防人之心死了也是活该,怎么可能带着他们去游击队的营地。杀了他俩吧,知道他们来的人不少,似乎也不合适,于是游击队和两人交代了一番说辞,拿走了两人随身携带的枪支任他们留下做俘虏。这两人也不敢乱跑,要这么直接跑回部落,下场不问可知,比较起来,还是做俘虏划算些。

等到刚卡政府军一来,两人就赶紧吆喝起来,指望能有人来拯救。当然会有政府军战士听到喊声来了解情况,中国维和部队又不象美国人一样那么霸道的自己划定禁区,于是俘虏就和政府军进行了沟通。

那俩俘虏自然是强调自己和袭击事件无关,只说自己路过,甚至为了掩盖自己部族和游击队沟通的事,居然来倒打一耙,说是维和部队是袭击胡图人在先。事实上负责沟通的政府军战士本不想多事,但是听到有酋长公子死亡却也不敢无动于衷了:大家都是土生土长的本地人,知道那个部落人数不少。

西瓦连长惊闻此讯,先是应俘虏要求向中国人提出了移交俘虏的请求,吴分队长向驻地请示后同意了。该连长请示上级后,因除去俘虏再没有活口,又以“事实认定不清”为由,提出要请一小队的负责人跟着回去调查的要求。

这个要求本来西瓦人也没指望能够实现,谁想驻地向上级请示后,马上就同意了。

这个兆头就有点不妙了,该让谁去?刘宁和成树国为此争了起来,副手一点都不把正职放在眼里,最后争吵的结果就是两人都去,也好相互有个照应。

还是吴分队长心思细,想两人都是有犯错误嫌疑的:起码目前说不太清楚——尽管有枪榴弹发射的痕迹和众多的反坦克地雷作证。那再安排个负责联络的人吧,该谁去?想来想去,吴海涛决定让综合素质超强的楚云飞跟随刚卡军队出发,还让他带了部电台。

\section{第四十九章 酋长要报仇}

刚卡政府军回去的时候带上了三个中国士兵,对非洲军队而言,这个连队机械化程度是相当高的,有两辆吉普车,还有两辆三轮摩托车,其他都是卡车,大部分的士兵都坐在卡车的车斗里。

楚云飞三人坐进了一辆破旧些的吉普车里,由于车里还有个排长,三人只能挤坐在后排,还好车里比较宽敞,中国军人也没有很肥胖的,舒适度还算不错,不过那台电台却被刚卡连长带进了他自己的车内。

楚云飞本不想那么轻易就给对方拿走电台的,但对方说部队里不允许有别的军队的电台,想要的话,等到了坎塔卡就移交给他,现在是暂时保管,而楚云飞的身份是联络员,自然不合适跟对方动手强行留下。

与中国军人同车的刚卡排长是个混血儿,除了非洲人的特点,似乎还有些白人和黄种人相貌的征兆。不过同其他刚卡人一样,排长也非常的健谈,与三人的沟通中,不但表示了对遥远中国的向往、对同在第三世界阵营的联合国常任理事国的羡慕,还很骄傲的介绍了自己一下。

原来这个排是刚卡的一个什么“闪电排”,有着比较悠久的名声和良好的军事素养,所以做排长的才能有辆吉普车座驾,虽然这车名义上是连里的东西,不过该排长把它当做自家物资倒也没人反对。

正说着,一辆摩托赶了过来,吉普车就停了下来,一个通讯兵向排长传来命令,“连长说城里出了点问题,要你带个班先把三位客人送回去,然后马上回军营。”通讯兵说完瞟了三人一眼。

排长很不解的看了通讯兵一眼,然后对三人说,“你们等等,我下去一下。”说完下车走到通讯兵跟前,通讯兵把张纸片交给了排长,估计是什么命令。

看完纸片,排长又回头警戒地看了一下中国军人,估计涉及军事机密,三名有教养的军人自然收回了好奇的目光,目不转睛的直视前方。

看到中国人不再向这里观看,排长低声和通讯兵交谈起来,用的却不再是英语。

阿拉伯语!楚云飞一下就注意到了,暗做个手势要成刘二人别吱声,自己却树起耳朵开始偷听。

前文说过,刚卡境内有少量的阿拉伯人,但由于人数太少,阿拉伯语又过于晦涩难学不好掌握,所以在刚卡是很少有人使用的。但这个排长本身是阿拉伯人和本地人的混血儿,从小就学会了阿拉伯语,而这个前来传信的通讯兵本来就是阿拉伯人,两人现在有阿拉伯语交谈的那自然就是秘密了,不过任他俩打破头也想不到这三个中国人中居然会有人精通阿拉伯语。

楚云飞从小就在语言方面有异常突出的天赋,后来阿拉伯语也学得非常好,当他知道巴基斯坦流行的是乌尔都语[注1]后,不但又去学了乌尔都语,还把能收集到的阿拉伯语的各种地方俚语都研究过了,所以这两人的谈话虽然口音很重、俚语众多,搁给个正宗阿拉伯人都未必听得懂,但还是让楚云飞弄明白了个大概。

怪不得要用阿拉伯语交谈,俩人正商量怎么杀死这三个中国人呢。

原来一小队杀的那个酋长公子来头非常大,该酋长所在部落历史悠久,人数也相当多,不少西瓦部落从理论上说起来都是从这个部落衍生出去的。这个酋长为人不错,和坎塔卡众多官员关系亲密,在西瓦人中的威望也相当高。

当该酋长知道自己心爱的小儿子死得那么惨,就强烈要求杀死凶手给儿子报仇。西瓦官员受不了压力,就试探性的要求中国蓝盔留下相关人员配合调查,没想到怕事的中国人那么痛快就答应了。

按这个酋长的要求,刚卡政府军要把犯人交到他手里,让凶手受尽折磨以后,再活祭自己的儿子。但坎塔卡的官员还没有白痴到那种程度,想也不想就拒绝了。

于是楚云飞听到的就是:连长命令排长把三人送到驻地附近,放三人回去,从背后开枪射杀三人,再给三人安个“抢劫枪支,试图逃跑”的罪名。那排长倒是心存不忍,辩解说有个人是联络员,似乎没有逃跑的理由,通讯员却回答他,“塔费拉酋长杀你全家也不需要理由。”

那排长无奈,只好喊了一个带班的班长过来,暗暗交代了事情,自己拦了辆卡车坐进了驾驶室,还和三个中国人交代,“我要赶回军营了,这个班送你们回去。”

看着那个不怀好意的班长坐进吉普车内,楚云飞装模做样的地喊,“我的电台,我的电台呢?”

排长自然不可能把电台交给楚云飞,再说那东西也不在他手上,“连长带走了,回头你们去坎塔卡找市政府要吧。”

吉普车重新启动,领着一卡车的刚卡士兵向中国维和部队驶去,那个班长坐在副驾驶座上一言不发。

刘宁和成树国看着楚云飞笑嘻嘻的样子,很是奇怪,成树国就问,“你怎么那么高兴啊?”

楚云飞笑着回答,“要回驻地了还不高兴?”一边说一边偷偷的指了指前排二人。这是楚云飞才考虑到的,刚卡人没想到自己会阿拉伯语,结果让自己知道了他们的计划,这个教训值得吸取,谁知道前面这俩会不会汉语?

刘宁正为自己的小队惨重的伤亡而愤恨、自责着,没注意到楚云飞的动作。可成树国却注意到了,他知道这个花样百出的家伙又有什么事了,想想刚才的情况,好象那个排长走以前说了一大通话,虽然听不懂,但似乎是阿拉伯语,阿拉伯语?楚云飞会啊!真是有事了!

想到这里成树国暗暗掐了刘宁一下,却没想刘宁“啊”了一声,虽然他随后马上反应过来不对了,却是有点欲盖弥彰的味道。

\section{第五十章 雪上又加霜}

刘宁不自然的喊声引起了坐在前排的班长的注意,他回头看了一下,发现三人没什么问题又掉回头去。

楚云飞把嘴凑在刘宁的耳朵边把前因后果说了清清楚楚,最后提出自己的建议:三人在下车后胁持对方班长做人质,能把副班长喊来一起劫持是最好的,然后尽快回到驻地,以后就只能走一步说一步了。

听完楚云飞的话,刘宁按照楚云飞的吩咐“淫荡”的笑起来,仿佛刚听完一个成人笑话,不过衬托上头部的绷带,再加上满是血痂的烟熏火燎的脸,这个笑容似乎用“狰狞”来形容更恰当些。

三人闲聊一阵,刘宁找个借口和楚云飞吵了起来,又掉头把这话悄悄传给了成树国,然后通过漫长、复杂的秘密交流,三人一致同意劫持人质。管他呢,反正对方有枪在手,指望“后发制人”还不如指望对方突然爱滋病发作来得现实些。

按照维和部队的规定,是不能随意激化各种矛盾的,三人的计划绝对是严重地违反了纪律。但此三人都不是循规蹈矩的主,再说,事关自己的生死,谁也没那么崇高,可以为了什么所谓的荣誉就白白的牺牲自己。要搁在以前,没准刘宁或者成树国的父亲会教训这几个人没组织性纪律性什么的,不过天长日久,环境在变,人也总是在变的。

所以,在这个信仰缺失的年代,我们实在不能指责这三个人不顾大局的计划,估计换谁也会这样的吧?

一个半小时后,吉普车开到了离驻地大约一公里的附近,刚卡班长叫几个中国人下车,“就这里吧,我们不能进你们营地了,部队有事,还要赶回去的。”

要是楚云飞没说过,成刘二人没准还真会有点摸不着头脑:在这里下车?不过,现在黑人班长的行为只会增加二人的杀心:看来实在是没有冤枉你们,人命关天,说不得也只好影响大局了。

于是刘宁依依不舍的同对方道别,成树国却似乎又突然想起来一事,“对了,你们副班长呢?也喊来嘛,我们中国有句古话,有运气的五百公里以外都能成为知己(有缘千里来相会),我一定要见见,大家以后就是朋友了嘛。”

刚卡班长拗不过他,只好把副班长也喊来,一帮士兵纷纷从远处走来“观摩”送别的景象。

刚卡人确实没想到“热情”的中国客人马上就变了脸,成树过和楚云飞一人一个制服了两个班长,居中的刘宁脸色也变了,下了两人的步枪,交给楚云飞一把,自己拿一把,成树国因为枪法和身手都略微差点,就拿个手雷专心控制人质好了,三人在大喊中一步一步的向后退,“不许开枪,要不你们班长就没命了。”

可三人万万没想到的是:刚卡士兵在短暂的惊讶过后,回答他们的竟然是激烈的枪声,两个班长看来人缘不是太好,当时就被打成了马蜂窝。7.62毫米口径的AK47穿透能力很强,楚云飞就被一颗透体的子弹擦伤了肋部。

楚云飞头一次在这么近的距离看到一个活人瞬间变成死人,但他根本没有感慨或者呕吐的心思和机会,尽管此人温热的鲜血甚至迸溅到了他的嘴唇上一滴。三人马上就地卧倒,两支步枪开始还击。

中国维和部队是从千军万马中选出来的,个人素质那绝对是没得说的,刘宁和楚云飞的枪法相当准。

当刚卡部队里四人中弹后,刚卡人终于发现这是场不对称的战斗,尽管自己这方人数上占着优势,但战斗力远弱于对方。意识到这一点后,刚卡军队顿时做鸟兽状跑了。

看着远遁的刚卡士兵,三人并没有追杀的欲望。站起来简单包扎一下,楚云飞伤在肋下,没法包扎,拿了件衣服裹住,刘宁在战斗时因动作剧烈,头部伤口又裂开了,还好出血不太多。再看看前方还在呻吟的刚卡伤兵,三人也没有前去报复的心思,更没有俘虏他们的兴趣,掉头向驻地走去。

驻地因可能遭遇来自外界的偷袭,派出了几个侦察兵在查探,听到枪声很快赶来,迎面撞上了三人。侦察兵先是被他们浑身的鲜血吓了一大跳,仔细问问,才知道大部分的血液来自于刚卡人,赶紧带他们回去,那时已经是晚上八点了。

本来商大队长已经在为部队出现了两名烈士而苦恼万分了,结果这三人回来让商成钢的脑袋又大了一号:情况是越来越糟糕了。

伤员先去了卫生队,刘宁的伤口重新包下就可以了,楚云飞的伤口居然还要缝针。等到他的伤口缝合好后,又被商大队长喊去了解情况。虽说一小队两个队长已经把情况汇报完了,但楚云飞作为联络人员陈述的过程应该是更客观的,何况他又是唯一听懂阴谋的的当事人。

商大队长听完楚云飞的陈述,又抓住重点询问了几句,又让三人在书记员的记录上签了字,就让他们回房间写材料去了,相关情况马上上报,兹事体大,来不得半点犹豫,和这事相比,死了两个战士都不是今天最重要的事了。

楚云飞也觉得事情不妙,暗暗找来高建军和李大龙吩咐了几句,却是在为可能出现情况做些准备。果然不出所料,当三人写完材料后,虽然已经是半夜十一点,商大队长还在和驻地领导们开会,而当事的三人已经被限制不许随便外出走动了。

当天夜里,整个营地就把战备状态提升到了最高的红色三级(作者杜撰)。第二天,所有驻地部队都取消了巡逻任务,全部集中在营地以防范可能发生的恶意袭击。

十点钟的时候,刚卡部队就兴师问罪来了,当然对于维和部队他们也不敢过分的嚣张。来了一营的人马,全部驻扎在驻地的正前方,带队的营长和坎塔卡派来的官员一同进入营地,在表示对巡逻队遭遇袭击事件的遗憾后,严令中国维和部队交出杀害刚卡军队两名班长和一名士兵的凶手。

\section{第五十一章 罪名是叛国}

当刚卡军队的汽车在远方出现的时候,楚云飞三人就接到了命令:在房间内呆着,不许出门,也不许其他战士和他们说话。

但这临时的措施怎么挡得住有心人的算计?我们的战士也是人,面对战友的无辜,谁又能不睁只眼闭只眼呢?所以谈判现场最新进展还是能传到楚云飞等三人耳朵里。

以商大队长为首的驻地领导自然是不肯承认自己的战士是杀人凶手,肯定要据理力争,可这事情各说各的理,中国部队又不能提出询问当事的刚卡士兵这种无理要求。由于事发当时在场的只有双方士兵而绝无第三者,在刚卡政府无理取闹的情况下,事件的性质实在是说不清楚,道不明白的。

于是商少校悄悄的向负责政委职能的冯副大队长抱怨:他们昨天要是能带回俩俘虏就好了。冯爱华却对商成钢的想法嗤之以鼻:切,人嘴两片皮,是非全在人说呢,有俘虏也起不了多大的作用。

后来刚卡方退而求其次,不再强调三人是杀人凶手,可又提出带走中国战士“配合调查”的要求。当然驻地领导谁也不肯答应,前车之鉴就在那里摆着,这样把三人交出去,有死无生的可能性实在是太大了。

于是驻地部队以向上级汇报的借口把事情往后拖,同时又按照上级的指示,建议双方组成个合议庭共同审理此案,自然西瓦人为什么要参与胡图人对中国部队的袭击也应该在审理范围之内。

刚卡政府在不断的讨价还价中逐渐的摸清了中国人的态度,在崇尚“丛林法则”的刚卡人看来:强者有权利通过暴力决定一切,而不是所谓的什么狗屁“仁义礼智信”和“温良恭俭让”。

所以刚卡人的理解就是:中国人是怯懦的,不敢像美国人一样肆无忌惮地胡作非为,甚至不惜他们自己受点委屈,也要把事情大事化小小事化了。他们却不知道驻地领导的态度实在是被那个狗屁的新规定影响了不少。

既然君子可以欺之以方,那不欺负你一下都对不起自己了,于是刚卡政府的态度水涨船高的强硬了起来,一定要中国军队交出凶手,至于双方共同审理的建议,刚卡政府会在向凶手取证后酌情考虑此事。

营地里,虽然中国军队也在剑拔弩张,却还没有走到与刚卡军队对峙的地步,于是就有有心的士兵发现三小队的李大龙一直在拿着块红手绢把玩,是怕冲突一起,再也见不到国内的对象么?不过这种场合也没谁有心思去取笑他。

维和部队是国内众多部门中选送的精英份子,无论是从思想觉悟还是军事素养上讲都是非常过硬的,所以除了相关人员,根本没有人想到有人会不服从组织纪律的,因此三个杀人嫌疑犯门口并没有人值守。于是当楚云飞听说领导不让他们在军营里和刚卡人对质的时候,马上就从窗户偷跑了出去,悄悄地联络上刘宁和成树国,三人一起商定出去躲躲,大不了回头一人背个处分,又死不了人。

本来成树国是不太赞同的,可情况实在是太不妙了,昨天要不是楚云飞自己的小命就被人阴了,而且估计连个“烈士”都混不上,现在还是对自己的小命负责点好。

刘宁的觉悟更高:做战士的自当是血染沙场,但要是窝窝囊囊地死得不明不白?对不起,门儿都没有!所以他自从昨天听到好友高建军传来的聪明人的建议时,马上认可了这个方案,甚至连躲藏的地方都让高建国选好了。

在驻地领导的一拖再拖中,三个肇事者早已远离了驻地,正在军营外将近两公里的地方拿着望远镜观看军营里“内应”李大龙发出的信号。

谈判还在持续着,情况在一次次地上报,上级部门终于做出了决定:人可以暂时交给刚卡政府,但是一定要刚卡政府保证士兵的安全,而且驻地要有副大队长以上级别的领导随行以确保士兵的安全。

既然上级有了指示,那随后的事情就好办了,刚卡人和驻地领导一同去房间带那肇事者,领头的正是这次指定的随行人选冯副大队长,待到进入房间,人们才发现:三个士兵跑了!

冯副大队长莫名其妙地松了口气,不过恐怕是为战士担心的比例还要小些,他自己不用再走昨天楚云飞的老路才是最重要的吧?

楚云飞和李大龙约定的暗号其实很简单,总共就是四种信号:情况好、较好、差、很差。现在的情况应该算是“很差”了,李大龙发出了信号。

远处的楚云飞看到令他心碎的信号传来,一股深入骨髓的凉意涌遍全身,眼泪止不住的流了下来,洇得望远镜里一片模糊,真的是这样么?我们……就这么被放弃了?被生我养我的祖国放弃了?与此同时,成树国也泪流满面,谁说“男儿有泪不轻弹”的?

刘宁可比二人坚强得多,看来做正职是有做正职的道理的,“哭什么哭,扯淡,天下大了去了,就凭咱哥仨还能怕了谁不成,操,虎行千里吃肉,老子们就是老虎。”说到最后一句时,正职的语气中也出现了一丝丝的哽咽。

营地里可乱了套了,各个地方都被驻地官兵翻遍了,也没找到三人的下落。商大队长多了个心眼,悄悄要孔繁茂清点枪支弹药和物资,果然不出所料:丢失步手枪各三支,子弹和手雷若干,还有望远镜等军需品。不过当孔繁茂把情况报告给商大队长的时候,少校严令他不得将此事声张,违者军法处置!

刚卡部队不能进入驻地,也在驻地四周搜查了起来,不过黑种男人的耐心实在是有限得很,他们很快就意识到:要是凶手昨天就开溜,恐怕再来十个营也搜不到人。于是他们断定凶手已经远遁了。

找不到人!刚卡官员这下可不答应了,一定要中国人给个说法,经过和上级的火速沟通,商大队长宣布:中国维和部队指战员刘宁、成树国、楚云飞被革去一切职务,同时对此三人进行无限期通缉!!!

楚云飞三人并没有走远,而是学着袭击一小队的胡图人的样子,找个隐秘地方挖了个土坑躲了起来。

而商大队长也没料到,军营里不但丢失了军用物资,民用物资也丢了一些,比如说药品和食物、淡水什么的,而且后来还在一直丢失,甚至有个别人清楚地知道了这些也当没看见。谁都知道这三人是冤枉的,所以连卫生队游队长都对药品统计结果睁只眼闭只眼。直到五天后整个维和部队因为这次事件和两名士兵的牺牲而全体撤回国去。

楚云飞三人在头次接受高建军带来的补给时才知道了自己被国家通缉,再也不能归队。虽然有一定的心理准备,但面对残酷的现实,三人的茫然心情不问可知。

更让高建军不忿的是:因为三人是不服从上级安排,没有顾全大局,严重地亵渎了中国维和部队的形象,自然也坐实了三人“杀人”的罪名。更因为也许是要给刚卡政府一个交代,所以罪名不是“逃兵”,而是:“叛国”!!!

奇怪的是,楚云飞听到这个消息,不但没有更加失落,反而激起了他的斗志:妈的,我的命是自己的,不是被别人拿来送人情的,叛国——那就叛国好了。

