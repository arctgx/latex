\chapter{楔子 蝴蝶扇动了她的翅膀}

\section{第一章 懵懂少年}

一切的发生都起源于一个初中教师对他学生的提问,一个再不能普通的问题。

“楚云飞,你到底是会不会啊?”执着教鞭的中年老师声色俱历地质问着讲台上的楚云飞。

可怜的楚云飞手拿粉笔,目光直勾勾地看着黑板上的题目——证明A角是B角度数的二分之一。他已经发呆3分钟了,惹得老师非常的不痛快,早知道就不让他上讲台做示范了。

听到老师的质问,楚云飞的脸“刷”就红了,要是他皮肤黑点倒也无所谓,遗憾的是——他的皮肤实在很白。在全班同学都看到红晕扩大到脖颈范围的时候,楚云飞小声回答了:“报告老师,我……不会”

赵老师心里这个失望,就别提了——因为是这么简单的题目,我才把你叫上来做的,没想到啊没想到,你居然连这种题都答不上来,我这个全市模范教师的脸都叫你丢光了,只好无奈的说:“看来你的感冒还没好彻底,那你下去吧,张月望同学上来解答一下”

一个高高瘦瘦的男生上来,一分钟就写完了答案。坐在后排观摩教学的其他数学老师不禁交换了一下惊异的目光,这么简单的题目是不会引起老师们太大的兴趣的,引起老师们兴趣的是——这个张月望证明题目的过程太简单了,很明显是引用了高中数学的思路,但他目前只是初三学生啊,看来赵学工赵老师的弟子还真是有点真才实料的,虽然那个姓楚的学生不知道发生了什么事情。

赵老师虽然也很满意自己的得意弟子解答得这么漂亮,但还是满肚子的不舒服:观摩教学是对自己工作的一项肯定,但也是对自己名誉的一项考验,在观摩教学过后被取消“模范教师”称号的老师也不是没有,偏偏这个楚云飞给自己丢了这么大一个人,要不是观摩教学过程中每个学生都要上讲台做示范;要不是因为自己观摩教学“从不事先做准备”的名声,要不是…………

说来这个楚云飞实在算得上是16中的一个怪学生,偏科偏得一塌糊涂,语文、英语还有政治什么的文科类的学习成绩那叫个顶刮刮,在年初举行的全市初中学生作文大奖赛中靠一篇“才情横溢”的文言文一举获得冠军,据说在该文中使用了几个冷僻字居然难倒了阅卷老师,翻查资料后在惊讶该生掌握文字之海量之余确定了此文的夺魁,又据说某阅卷老师专门前来找过他,想了解他的学习方法云云。

与之“相映成辉”的是:楚同学理科成绩的“乏善可陈”,简直不是乏善——根本就是白痴一个,再简单的题目到他手里都跟天书相仿,每次学校的理科考试基本上就是那么几分还是托了有选择题的光,当然偶尔他也能做对那么一道半道太简单的题目。很显然今天赵老师就是栽在了这个“偶尔”手上。

观摩教学结束的第二天,下学路上,楚云飞还是没从噩梦中醒来,心情就象这天的天气一样阴沉,赵老师的话如炸雷一样一次又一次的在耳边响起:“我没你这么个学生,丢尽了我的人!以后我的课你在教室门口站着,不想站也由你,随便鬼混去好了!!!”楚云飞的心里那个委屈就别提了,自己已经把学习时间的百分之九十花费在理科上了,谁也不能说自己没用心啊,结果赵老师居然这样说自己,再说这次观摩教学结束的时候光看那些老师眼中羡慕的神情就知道虽然有自己这么个“纰漏”,赵老师的地位肯定是又有提高啊。

回到家里,做护士的母亲一眼就发现了儿子的不对劲,“发生什么事了?”“没什么,赵老师说我不用心学习”楚云飞口是心非地说。

“那个数学老师?”

“恩”

“唉,不能怪人家说你,看看你每次都考什么分回来呀,真是……,你不知道义务教育只有9年?明年你怎么上高中啊?读一辈子高一?”随之而来的是没有休止的唠叨声。…………

郁闷啊!楚云飞此时的心情再也没办法平静了,脑袋里面乱成一团,越来越乱越来越乱,乱得都有些头疼的时候,老爸回来了。如果说有比母亲不停的叨叨还痛苦的事,那就是在遭受到“地毯式轰炸”时再加上时不时冒出来的“重磅炸弹”,老爸虽然没老妈口水多,但偶尔冒出来的两句绝对是致命的打击:诸如“你好好学学理科嘛”的怀疑论(我没有学?)或者“我现在除了失望就是失望”的心碎论(我拿回来作文冠军的时候怎么没见你这么失望啊?)

随着头越来越疼,晚饭也没心思吃了,楚云飞胡乱划拉了两口,回到了自己的房间,苦恼地躺在床上:要命啊,怎么现在什么都想不通畅,只有赵老师的恶形恶相、同学的幸灾乐祸、老妈的口沫飞溅和老爸拉长的脸,一幕一幕在眼前不断浮现,而且“屏幕”刷新越来越快,这几张脸也凑得越来越近……终于“嗡”的一声,楚云飞昏了过去。他没注意到的是:就在这时,窗外一道闪电掠过,随着“喀啦啦”一声巨雷,今年的第一场秋雨来了。

楚云飞又开始做那个很久以前总做的梦了:一个白须白发的老道在给自己讲述丹道,最后还是那句总结的话,“徒儿啊,你是我玄青门自开派以来最有天赋的也是最博学的弟子,千万记住,修道首先修身啊”,好了,梦做完了,能好好睡了——楚云飞都梦出经验来了,但好死不活的是:就在这时,窗外又是一声巨雷,于是楚云飞的命运掀开了新的一章。

“这是什么?”楚云飞茫然的望着手中长长短短不一的一把扁竹片,怎么和算命的签差不多?哦,对了,这不就是算筹么?圆已经32切了,离祖冲之所得出的圆周率精度还差很多,要继续下去么?是验证这个数据还是再对它的精确度进行提高?再下去就得耗费更多时间了,可还有那么多有待证明的构思啊,思来想去,难以取舍之间,楚云飞他——醒了。

“好奇怪的梦啊,算筹又是什么东西?”楚云飞昏昏沉沉地想着,“接着睡吧”

不幸的是:可怜的小楚同学再也睡不着了,无奈的他只好在床上翻来覆去,最后才发现毛病的根源——睡觉的时候没关灯!完了,这种状态还睡个屁啊?

爬起来先翻《辞海》,这个“算筹”是什么?哦,这样啊,是古代算盘还没发明以前的计算工具啊。那这个梦又是什么意思呢?楚云飞琢磨了半天也想不明白,算了,不想了!对了,作业还没做呢,今天是怎么回事啊?这么早就睡着了。现在几点了?不是吧?5点了?还有俩小时就得上学了,那只好把文科作业先放一边了,反正老师们也不会为难这个高材生的,先做理科吧,由于昨天理科只有数学和物理课,先做这两门吧。

不对,怎么这么快就能写完数学作业啊?往常都得和参考书上的答案相对,慢慢的凑出数字来才成的啊,至于证明题,以前曾经做出来过么?楚云飞一边嘀咕一边翻着参考书对答案:晕啊,怎么回事?这样也可以?明天赵老师又该说我了,“不骂你就不会好好学啊?”怀着美不滋滋的心情,楚云飞开始做物理作业了。

要说楚云飞在物理和化学上一窍不通那是冤枉他了,可楚云飞不得不面对的残酷现实是:尽管他知道物理题和化学题该怎么做,但是由于数学水平实在是太差,他依旧是得不出正确答案,所以在每次考试的时候,就算是做选择题,他还是习惯优先把自己得出的答案排除掉(偶尔……选择题中会有他算出的答案的),而且实践证明这样做能保证自己在选择题上的正确性。但今天可是真正的邪门,和参考书的答案一比较,居然和数学一样,楚云飞把所有的题目都算对了,哦,不是,有道题是错了,但是那是他没审清楚题,重新列算式以后再一算就万事OK了。

母亲已经开始起来做饭了,收拾好作业,楚云飞心情愉快的解决了早餐,加了件毛背心——一场秋雨一场寒嘛,向学校出发了。

昨天晚上的一场雨使得空气格外的清新,城市中常见的粉尘也没了踪影,城郊的龙山和凤山是那么的清晰,唯有躺在城市片片污水中杂七杂八的黄叶带给人三分萧瑟的感觉。由于起得早,加之心情不错,楚云飞没有象往常那样带着风走路,而是在慢慢品味或者说慢慢琢磨着早上发生的事情,晃晃悠悠的往学校走去。

一进教室,一个小眼睛,个子比楚云飞略高的同学就冲了过来,“云飞,听说没有?21冶要去沙特了,援建项目,去30个人呢”,这是王通,和楚云飞是从小玩大的朋友,他的父母亲和楚云飞的父亲都在中国第21冶金建筑有限公司工作,他的父亲是个处级干部,所以在他们21冶子弟中,王通算是个消息灵通的。

“没听说,老爷子昨天收拾我半天,也许是没他的份,心情不爽吧”楚云飞说,“对了,张月望,给你我的作业,今天我可是都做完了”

张月望推推鼻梁上的眼镜,直接打断了楚云飞的好心情:“楚云飞,赵老师跟我说,以后不许收你的作业。”

……

第三节课是数学课,赵老师站在讲台上,瞪着能点着烟的眼睛对着楚云飞:“我不是和你说了嘛?你现在给我出去,以后我的课你不许在教室”“赵老师,我……我以后一定好好学习”“说什么也没用,你不走我走,听见没有?”楚云飞不语。

赵老师抬头扫视一下教室,“大家都看到了,这个白痴不出去,那我只好对不起大家了,他不走我走。”说完作势掉头下讲台。“哄”,教室里一片笑声,老师这样骂人的还真少见,随之而来的是此起彼伏的交头接耳声。

再不出去得被别的同学埋怨了,楚云飞低头向教室门走去,边走边嘀咕:“操,什么模范教师,整个一个小心眼儿”,很显然赵老师注意到了他的不情愿,“楚云飞,你嘴里不干不净叨叨什么呢?”楚云飞的火气一下被点燃:“我说有人这么小心眼,怪不得赵东辉没妈呢”

“哄”教室里的响动比刚才大多了,赵东辉是赵学工的儿子,在隔壁3班,由于赵学工的不近人情,他的妻子和他离婚了,赵学工带着儿子生活。

楚云飞在说完话后直接窜出了教室,谁会傻乎乎等收拾?那可真成白痴了,至于在门口罚站?省省吧,反正以后也不用上数学课了。

日子一天天过去了,由于楚云飞所受惩罚实在有点夸张而且他在不上课的时间里东游西逛毕竟影响到了别人,学校里都知道了4班有那么个学生。所以楚云飞的班主任刘雨影找楚云飞谈话了。

刘老师是教化学的,最近楚云飞化学的作业完成的非常好,而且物理老师也对他的进步大跌眼镜。开始刘老师并不是很在意楚云飞受的处罚,但是由于事情一直没有解决而且眼看着楚云飞一天天的进步,刘老师决定帮帮这个“知耻而后勇”的学生。

“云飞啊,我和赵老师说说,你呢,明天上数学课的时候在教室外面好好的站着,等下课了和赵老师道个歉,虽然赵老师的方式不太正确,但他也是恨你不争气啊,你是学生,总不能叫老师先跟你认错吧?”

\section{第二章 成长的烦恼}

第二天,赵老师看都没看在教室外面站得笔直的楚云飞就进了教室,而楚云飞也只好在那里呆呆地站着,课间休息的时候,由于赵老师从来课间不休息,楚云飞就站在那里成了别的班同学的笑柄。

赵东辉上个厕所回来,看见了站在那里的楚云飞,“呦,这不是楚云飞么,你这个样子好傻哦~”看见楚云飞没反应,赵东辉又不甘就这么便宜了这个传说中侮辱过自己的家伙,可他也怕那个父亲的严厉,这不,眼珠一转,掉头走了。

1分钟以后,3班的刘凯、谢小亮晃悠过来了,刘凯是个壮实的家伙,而谢小亮则是个小不丁点,刘凯的哥哥刘勇是16中附近的混混,和着几个狐朋狗友号称“七金刚”,成天在这一带惹是生非。谢小亮以前因为个子小,经常被人欺负,于是费劲心机巴结上了刘凯,于是仗着刘勇的名气在学校里欺负了这个欺负那个。

“看这个傻B”说话的是谢小亮,“这样子还真象个白痴,真他妈的好笑啊。”楚云飞没理他们,知道这才是个开始,自己要是答话,或者有什么表情,麻烦会接踵而至。

但是看来刘凯和谢小亮并不想就这么放过他,谢小亮又开始说话了:“傻B,老子跟你说话呢,听见没有?”——没有反应,谢小亮有点恼了:“我操,挺拽啊,信不信老子抽你?”——还是没有反应,在谢小亮故意变脸的时候,上课铃响了。

又一节课下了,看到赵老师夹着讲义走了出来,楚云飞跑上去说:“赵老师,我错了,以后我一定好好学习”,赵老师好象没听见似的接着走他的路,楚云飞跑到赵老师的前方,又说了一遍。

赵老师皱皱眉头,“让开”,然后就目不斜视地走了,留下楚云飞呆呆地站在那里,耳边全是同学的哄笑声,其中刘凯和谢小亮的笑声格外刺耳。

带着一肚子的不爽,楚云飞和几个同学走出了校门,抬头却看见了刘凯和谢小亮带着几个同年级的学生挡住了去路,再往旁边一看,赵东辉远远的站在马路另一头,脸上一副“不关我事”的表情。同班的同学看见刘凯和谢小亮这俩家伙,知道没什么好事,远远的避开了,只剩下了王通站在楚云飞旁边。

“怎么回事?”王通问刘凯,谢小亮抢着说:“王通,没你的事,我们找他。”

王通一看架势不对,大声喊道:“操,差不多点,这可是我兄弟。”刘凯过来了,“王通,不是不给你面子,赵东辉要我们帮忙的,你也知道,我不想惹你,可你也别断我财路。”

王通知道刘凯是忌惮自己的老子,毕竟是个中层干部呢,可刘凯话都挑明了,似乎自己断他财路也不太不上路,但不管楚云飞怎么可能?他想了想,“这样,那给我个面子,单挑,挑完就事情就了结。”

楚云飞知道王通本身不是个招摇的人,这样的结果已经不错了,毕竟刘凯的背后还有个刘勇,再说自己心里正在不爽,打人一顿或者被打一顿而就此结束也是个不错的选择,“谁来?”那几个跟着刘凯的学生相互看了看,还没说什么,刘凯跳了出来:“来,楚云飞,试试老子的擒拿手。”

原来,“七金刚”曾经因勒索一个学生而被学生的哥哥——退役的特种兵陈群空手收拾了一顿,陈群看刘勇为人还可以,两人不打不相识,成了朋友,刘凯也就跟着陈群学了两招擒拿手。

随着俩人拳来脚往,两分钟后楚云飞被刘凯把手臂扭到背后,疼的眼泪都快出来了,可刘凯没打算这么放过他,也许是谢永亮喋喋不休的咒骂让他觉得惩罚必须继续,也许是楚云飞不卑不亢的神情刺激了他。在肚子上、胸上、头上挨了若干拳后,楚云飞不支倒地,他想就这么样吧,抱着头再挨几脚也该结束了,可随着恶狠狠的几脚踢来,一股莫名的倔强涌上心头,一种不屈服的热潮在胸中激荡,在谢小亮“操你妈,你再鸟(音同”吊“)呀,我日!”声中,这种冲动到了顶点,楚云飞……站起来了。

刘凯还真想的是再来几脚就算了,自己的拳脚轻重自己还是知道的,别出什么事情,可没想到楚云飞不给面子的站起来了,这不是说我打人的功夫不行么?楚云飞跟着来的“打完了?”一句又让他有点踌躇,该不该这样结束?可旁边的谢小亮不干了,“操你妈,白痴你真想找死啊?”

又一句这样的话,楚云飞真的受不了了,你骂我算了,你敢骂我妈?还一句接一句?“杂碎,你敢骂我妈?”他顺手捡起一块半砖,恶狠狠的朝谢小亮头上砸去,随着砖头和谢小亮脑袋的亲密接触,谢小亮栽倒在地上不动了。

一股疼痛从肩上涌来,砖头落地,楚云飞回头一看,红眼的刘凯正捏着拳头恶狠狠的看着他,“去你妈的”同样红眼的楚云飞暴怒的一拳打去,在拳头落在刘凯肩头的同时,“咯”的一声轻响,随后就是刘凯痛苦吼叫——他的右臂脱臼了。

然后的结果就是刘凯和谢小亮由那几个同学陪着去医院,王通陪着楚云飞回家,因为楚云飞没有遵守“单挑”的规矩,刘凯留下话来——这事儿没完!

回到家里,鼻青脸肿的楚云飞睁着眼睛胡说半天才让母亲相信自己确实是没有打架,虽然相对而言父亲不容易被欺骗,但令人高兴的是,父亲要出国了——沙特的援建项目。作为一个普通工程师,出国挣外汇这种好事并不是人人都能轮上的,所以楚振中并没有追究发生在儿子身上的事情,他只是一反常态的叮咛儿子要认真学习,好好做人之类的,让楚云飞意外的发现父亲的唠叨其实完全可以媲美母亲。

刘凯的哥哥刘勇不出意料的找到了楚云飞,并向他收取了1000元的“医疗费”,没有遭到毒打的楚云飞本来以为这事就此结束了,没想到后来的结果是谢小亮三天两头的带着一两个金刚向他勒索,没钱就是一顿胖揍,搞得楚云飞的压岁钱金额急剧下降。

时间过得飞快,一转眼两个月过去了,又是初三4班的数学课,楚云飞躺在操场的草坪上,眯着眼睛晒太阳,嘴里叼着草棍,无聊的想着:这全市数学竞赛明天要开始了,赵老师是肯定不能阻止自己参加的——他没那个权利,这些日子辛苦的学习应该没有白费,到时候可该好好出口鸟气了。可郁闷的是,这个被人勒索……什么时候才是个头啊???

江海中最近日子过得十分不顺心,电视台栏目又要整改了,辛苦工作快20年了,才是个《教育热点》栏目的副主编,本来教育类的节目就没什么搞头,无非播播教委的新闻,宣传一下各个学校类似校庆啦什么的活动,找两个特级老师讲讲题老师还不会认真给你讲,为什么?因为没报酬啊,谁不知道现在的老师随便带俩特长班都是按分钟收费啊?何况是特级教师?可是,居然《教育热点》栏目也要参加考评,这个栏目有收视率么?怎么现在人成这样了?看我这个副主编碍事也不能这么整啊,考评不过关恐怕这个副主编没了还是小事,弄个待岗(下岗的另一种称呼)也不是不可能啊。正在思索,“铃~~~~~”电话铃响,江副主编摇摇头让自己清醒了一下,提起话筒:“你好,哪位?”电话另一头是个少年的声音:“你好,江主编么?我是您女儿江娜的同学,我叫楚云飞,我作文竞赛拿第一的时候《教育热点》报道过我……”

三天后,数学竞赛的预赛结束了,楚云飞以89分的成绩高居16中第一,比第二名张月望多出了11分,这么个成绩对赵学工老师实在是个沉重的打击,顿时流言四起,尤其以嫉妒这个模范教师的老师和喜欢楚云飞但又不好公开和同事唱对台戏的老师(例如楚云飞的语文王老师和英语高老师,还有提出和解意见却不被人尊重的班主任刘老师)为传播主力:因为赵老师教得不好,所以楚云飞同学每次只能考几分,赵老师不好好反省自己,反而报复楚云飞同学,不惜以全班同学为人质阻碍他上课,楚云飞同学被逼无奈,辛苦自学,豁然发现其实赵老师真的是误人子弟,种种现象令人反思——赵学工凭什么当模范教师?自然其中也是不乏眼红特级教师待遇的人,谁让赵学工做人不成功呢?

而身为风头人物的楚云飞并没有陷入种种流言之中去,虽然落井下石并不违背他做人的宗旨,和别的老师做个小配合整整赵老师是他做梦都想的事,但是目前的他正头疼于应付来自学校外面的勒索——这个星期五下午他必须支付100元“第N次医疗费”,但是他坚决的要求从这次开始以后的缴费过程不能有谢小亮在场,他的要求得到了满足。

星期五下午,从语文王老师那里拿上王老师为他借的《奥林匹克数学习题集》,楚云飞兴冲冲的出了学校,根本看不出这是个即将遭到勒索的家伙。果不出所料,门口谢小亮冲他轻蔑的一笑:“黄楼后面,五哥六哥等你呢”

谢小亮口中的五哥刘哥就是“七金刚”里的老五黄强和老六迟志刚,长久以来,就是这俩人伙同谢小亮从楚云飞这里讹走不少钱,楚云飞皱起眉头,苦恼的轻叹一声。

过不多久,楚云飞灰头土脸的出来了,他走到斜着眼睛看他的谢小亮面前,用很古怪的表情说:“你的压岁钱有没有3000块钱?”说完,扬长而去。

回到家中,老妈已经在家,为他参加全市的数学竞赛复赛,老妈这段时间天天好肉好菜的做着,还时不时来点补品什么的,看着合不拢嘴的老妈,楚云飞真的有点纳闷,以前自己怎么就学不好数学呢?

晚上9点50,市第二电视台《教育热点》开播了,看着电视里的标题《怎么才能斩断伸向学校的黑手?——16中某学生遭遇抢劫现场纪实》,楚云飞的脸上漾起压抑不住的微笑。心道:还好老妈夜班,要不还真有点麻烦。

电视上,一个身穿16中校服的学生被两个明显大他一号的青年围在栋黄楼的墙角,那个学生佝偻着单薄的身体,缩着脑袋,时不时的摇摇头,却被那两个青年在头上左一下右一下的拍着,后来那两个青年明显的恼羞成怒,一顿拳脚把学生打倒在地,学生哆哆嗦嗦站起来,扭头,然后一个青年向学生扭头的方向的口袋中掏出什么几张类似钞票的东西。最后就是两个青年掉头愕然地望向镜头,然后头也不回地走了。镜头切回演播室,漂亮的主持人满脸都是正义:“这是我台记者在偶然间偷拍到的,为了真实的记录这一切,我们没有中止犯罪嫌疑人的行为………………最后,我们向广大观众承诺:我们会对该事件进行跟踪报道。”

回味了半天自己的作品,楚云飞站起来关掉电视,得意的心情还是无法平静:看来以后去演电影也是个不错的选择,哈哈!该去做题了,不过明后天看来肯定不能休息了。

\section{第三章 谁叫谁哥}

第二天中午,就在楚云飞等得都不耐烦了的时候,想象中的电话来了:“小楚么?”刘勇的声音有点软,“是我,你谁呀?”“我是刘勇,找你有点事情商量,什么时候方便?”,楚云飞嘴角扯了一下,懒洋洋的说:“哦,勇哥不是总叫我姓楚的么?今天怎么这么客气啊?”刘勇无奈的说:“别开玩笑了,真找你有事。”

楚云飞还是那种要死不活的声音:“我妈等我吃饭呢,勇哥你有什么事电话里说不成么?”“那出来吃吧,勇哥我请你”楚云飞笑了笑,“呵呵,勇哥你请我?不敢当……这么着,下午3点吧,什么地方?”“那就3点心情酒吧,哥哥我请你”。

下午3点,楚云飞准时出现在心情酒吧,其实这里只是一个很简陋小雪糕店,提供雪糕、冰激凌、小蛋糕还有啤酒什么的,离16中不远,离3中也很近,经常有3中的搞对象的学生在这里,偶尔16中早恋的小混混们也来这里请小女孩吃个冰激凌什么的。不过,在16中的初中生和3中的高中生眼里,这也算得上个酒吧了。

刘勇已经等了一阵了,他是中午听别人说的,电视里报道有俩勒索16中学生的青年被偷拍现行,而且定义是抢劫。讲述的人还说拍得不是很清楚,不过看模样应该是老五黄强和老六迟志刚。刘勇很清楚:抢劫和勒索这区别可大了去了,他马上联系老六和老五,这俩家伙倒是也承认得干脆,还说拿钱以后看见背后有个中年人鬼鬼祟祟在那里不知道做什么。刘勇马上意识到问题严重了,仔细问了他俩半天,才决定约楚云飞见面。

自然,为了这次见面刘勇想好了一套说词,还偷偷的带了个小录音机,在谈话中一定要说明这只是个勒索事件,实在不行,起码可以通过录音机的记录证明俩兄弟的无辜。

“噢,勇哥来得早啊,不好意思”楚云飞做了不好意思的样子,“先别说正事,勇哥先让酒吧放个迪曲成么?”

刘勇一下没反映过来,等回过味来,挫败、自责、失落,种种不祥的感觉涌上心头,看来自己的小录音机是用不成了,这事恐怕不是一般的棘手,小家伙不好对付啊。

楚云飞可笃定的很,放迪曲虽然有点扰民,但别的客人根本不可能反对,谁不认识“七金刚”里老二啊。

超重低音和打击乐响起,随着音箱的震动简陋的酒吧一颤一颤,让人有点房子会不会马上坍塌的担心。递过一罐啤酒,刘勇把嘴凑到楚云飞的耳边大声嚷嚷:“现在可以了吧?”楚云飞点点头,扯掉易拉罐的封口,大声说:“谢谢勇哥”。

“昨天电视看了没有?关于你和老五老六的事?”楚云飞点点头表示知道。

“我跟你要了1000块钱,就没再想找你的茬了,不过,和刘凯一起的那个小子总是撺掇老五老六去跟你要钱,我知道,但我也没管,虽然他们这么做有点不地道,不合规矩,可是我也不能挡弟兄们的财路,而且毕竟这事源头在我身上,现在,弄出事来了,他们不该没完没了的跟你勒索……”

“是抢劫!”楚云飞很不知好歹的打断了刘勇的话,喝了口啤酒,“我根本不认识他们,不过好象听说他们是什么金刚里面的人。”

刘勇瞪着楚云飞楞了半天,皱着眉头说:“这么着,你说吧,你要什么?怎么样才能证明他们是勒索?我喜欢痛快人。”说着说着刘勇眼睛有点红了。

“本来就是抢劫,能说点别的么?”楚云飞又一口啤酒。开玩笑,说是勒索?江叔叔不杀了我才怪,新闻要的就是震撼,和江叔叔可是有约定的。

“你~~~”刘勇眼睛更红了,腮帮子鼓起来了——他是在咬牙?

“勇哥听我说,除开抢劫和勒索,还有别的事可以商量的。”

刘勇盯着楚云飞看了半天,眼睛一点点变回原色,不过多了点忧虑,“小楚你是聪明人,有什么想法跟勇哥说说?”

“我可以说从来不认识他们,偷拍的效果也不是很好,根本就是无头案,我还可以不报案,就说我害怕报复,所谓民不举官不究,虽然这事肯定要有点影响,但是媒体最终注意的重点就是学生的普遍生存环境,个别案例并不重要,你们又是玩社会的,找俩认识的警察打点一下,五哥六哥出去躲俩月不就成了?”

刘勇倒抽一口气,操,刘凯怎么惹了这么个主?“还五哥六哥?让他们管你叫哥好了”。想了想,刘勇问:“开出你不报案的条件吧,我掂量掂量。”

楚云飞认真想了想:“我老子去国外了,家里没大人,以后有事就得勇哥你罩着我了。”刘勇一脸的诚意:“没问题,从今天起云飞你的事就是我的事了,还有么?”不好意思的挠挠头,楚云飞道:“还有就是这事主要是谢小亮煽风点火,就是那个小个子,这一阵搞得我总是提心吊胆,躲着五哥六哥,压力很大,又让他们弄走不少钱,我想跟他要3000块补偿,我只收2500好了。”刘勇无奈:这是让我替你勒索了,操,脏活我做,你拿大头,不过你没讹我封口费,弟兄们还有辛苦费拿。还算明白人。“好了,这事交给哥哥我了”“那就没别的事了。”

刘勇心里一下轻松起来:还是年轻啊,要是你让我自己开条件,那你不是大发了?

看看楚云飞已经走出酒吧,刘勇突然一个激灵,跑出去喊住楚云飞:“云飞,这事真的是偶然的?”楚云飞回过头一脸的茫然:“什么偶然?”刘勇心里叹口气:看来就是这样了,以你的聪明会不明白我的意思?“没什么,哥哥回去和老五老六打招呼去。”

看着楚云飞越走越远的身影,刘勇头皮一阵发麻:这家伙到底什么人啊?聪明得过分还是背后有人?得告诉哥几个,躲着他走为好。

…………

虽然是星期六,但是江海中今天可过得太充实了,一大早就被张副台长的电话骚扰,江副主编婉转的向领导解释了为什么发这种比较敏感的内容:我们栏目也要考核啊,又是在教育的范畴内没超出范围,时间太仓促没时间向领导请示毕竟新闻要讲究个及时性……说来说去就那么一个意思——收视率我是可以弄上去的,看你给不给政策了。随后又忙着接待各路神仙,打发完了日报的记者,看着雪片一样反馈回来的观众信息,江副主编幸福得脸皮发紧:怎么从来没发现居然有这么多人看《教育热点》?正在陶醉中,电话响起:“古田派出所?需要找我们了解情况?好的,我马上派人过去协助调查”放下电话,有点小小的担心:小楚你可得挺住啊!

楚云飞刚吃完丰盛的晚餐,正在家里练拉力器,他是典型的闲不住,吃完饭不能马上学习,但是总得给自己弄点什么事做吧。正在锻炼,门铃响起,打开门,一男一女俩警察走了进来,证件一晃:“古田派出所的,调查点事情。”楚云飞老妈过来了:“什么事?”“你是叶美,你是楚云飞?”在得到肯定回答后,两个警察对视一眼,“这样的,我们有点事情需要找楚云飞了解了解。”看到叶女士惊异的眼光,女警察话锋一转:“楚云飞你出来一下”,心道:看来叶美还不知情,那就把小家伙叫出来吧,省得有他妈在旁边纠扯,半天弄不明白情况。楚云飞冲着老妈笑笑:“没事,你儿子你还不放心?”说完就晃悠着跟俩警察出去了,留下老妈一个人在家里走来走去。

来到楼下,男警察一脸严肃:“昨天你遇到抢劫了?”楚云飞挺无奈,“是啊!”“那好,你跟我们走一趟,回去做个记录。”“等等,”楚云飞有点着急,“是不是做个记录就立案了?”“是。”“那我不去了,我怕报复,我还想正正常常上学呢。”男警察不乐意了,小家伙还挺刺头啊,“少废话,你有配合警察调查的义务,走。”楚云飞也有点火了,可是不敢发作,“我去过派出所好几次了,报告说学校附近有社会青年抢劫,可根本没人理我,”俩警察觉得好笑:废话,就你个毛孩子,谁会理你。楚云飞眼珠一转“这可是事实,而且那几个警察我也记得,你要我去,那就一定要把这话记录上,要不我不去。”两个警察有点头大了,这样恐怕会有同事涉及到失职了……现在孩子这么聪明了?女警察看情形不对,微微一笑,“那我们就在这里问问你好了,不过到时候抓住嫌疑人可是需要你指认的,这个没问题吧?”

在两个警察在确认楚云飞确实不认识那俩人之后,询问了两个人的体貌特征,楚云飞以吓昏头为理由坚决不好好说,看着两个警察悻悻的离去,回家接受老妈的审讯去了。

对付老妈也比较好办,说明警察只是调查学校的治安,否则自己铁定被带走了,不让你在场只是怕你问来问去耽误时间,他们还要去别的同学家调查呢。警察也是人,也想早点休息啊。还有,数学竞赛的复赛就要开始了。

叶美听了琢磨半天,看样子就是这么回事了,竞赛的事可是大事,孩子要能考好,不但能在自己的家人面前长脸,还能保送个好的学校,没准能上市一中呢。想到这里,挥挥手,示意楚云飞回房间学习去。

谢小亮这个周末可是过得不太如意,自己个子小体力差,本来以前经常受别人欺负,自从跟上凯哥也是指点江山激扬嘴皮粪土同学的,怎么扬眉吐气的日子没过多久凯哥的老大就给自己来一闷棍?五哥六哥平时拿钱的时候也是兄弟长兄弟短的,这一出事,不但得给那个楚王八3000,五哥六哥还刮了自己1000跑路费,自己的那点家当怎么够这么折腾的?关键是以后跟谁混好呢?万一凯哥不罩我了,我还不得被那些“翻身农奴”踩死啊?都是那个楚王八——?不对,这次收钱没我,逃过了这一劫,这还是楚云飞坚持的,不会这里有什么缘故吧?要不回头和云飞同学咨询咨询?似乎他来头也不小,跟着他没准也是个不错的选择。

又是一个阳光明媚的早晨,楚云飞心情愉快的来到学校,忙了整整一个周末,处理完了若干事情,挣到2500的“精神损失费”,又做出几道以前从来没有做出的数学题,现在的心情是非常放松的。

不过中国的成语里有个词叫“好景不长”,楚同学很快就收到一个令他不太开心的消息,一直关心自己爱护自己的班主任化学刘老师被调到了第一重点中学先阳市一中教高中去了,不过还好,接替的班主任是物理郭老师,要是赵学工的话楚云飞恐怕就得考虑换班了。

还有就是谢小亮这个另人讨厌的家伙居然来找楚云飞道歉,还说什么以前虽然对不住你,但是没想到云飞一直把我当同学,现在才知道同学情谊是多么的难得,搞的楚云飞暗自忖思:这家伙到底知道了多少?不过显然楚云飞不能直接告诉他:因为你在场的话就是勒索而构不成抢劫了。所以楚云飞只能真诚的告诫这个才被自己勒索一大笔的家伙,“其实真的没什么,我也知道你以前受的气很多,所以有刘凯给你做主以后难免嚣张了点,以后记得给别人留点余地就好了,过分的事情做多了总是不好。”简单的几句话说得谢小亮感激涕零,拍胸脯保证记住云飞同学的话了,请云飞同学看他的实际表现。

最让楚云飞挠头的还是纷纭而至的记者们,他有生以来第一次见识了记者的神通广大,所有的记者都能够肯定遭到抢劫的就是单薄的楚云飞同学,但是楚同学打定主意咬紧牙关就是不承认,只有在私下里才对两个缠得他受不了太敬业的记者承认了自己就是那个倒霉蛋,但是怕报复,怕影响正常的学习,母亲说了,一旦有哪个记者捅出此事,那么法庭是唯一的选择。

而显然楚某人目前只是个未成年的被监护人,监护人的警告在这些记者面前确实是道不好逾越的障碍。

\section{第四章 金牌战略}

楚云飞下学回家后,被母亲叶美女士逮住又是一通盘问,很显然母亲大人遭到了部分记者的骚扰,联想起警察上门的情景,母亲觉得有必要好好和儿子沟通沟通,孩子父亲远在异国他乡,做母亲不是更该操点心么?

不忍心让母亲再为自己担心,铁嘴钢牙的楚云飞没有继续否认下去,他告诉母亲:抢劫已经结束,儿子没遭到任何人身伤害,关键是警察说了,一定要控制传闻的散播,一来是为你儿子的人身安全着想,你不想儿子被报复吧?二来是宣传太厉害会影响警方破案,造成的后果不可想象,还有就是再有记者骚扰母亲大人母亲你可得记住用法庭去威胁啊。于是深为自己没能完成好保护孩子重任而自责的母亲成了儿子坚定的同盟。

三天后,全市数学竞赛成绩公布,在总分为250分的考试中,楚云飞以197分的成绩笑傲全市,33中刘丽丽同学以194分紧随其后,楚云飞同学再次登上风口浪尖,如果说上次16中第一不足以说明任何问题的话,这次16中的师生们再也不能视而不见了。

《教育热点》的记者们在一年内两次踏入16中,为的是表彰同一个学生,在16中建校以来的历史上,不能说绝后,但肯定是空前的。16中一把手蓝校长忙着跑前跑后,无怨无悔地在深冬的寒风中汗流浃背,在做完类似上次的学校简介和师资力量的夸大等工作后,有个漂亮的现场主持人完成了江副主编的授意:“我们想采访一下楚云飞同学,可以么?”蓝校长有点纳闷:怎么居然还有采访学生一项,不是说为了保证学生的正常学习尽量不让他们上镜的么?不过对于楚云飞这个怪学生蓝校长还是愿意网开一面的,再说,随随便便拒绝市台的记者也对学校没什么好处。

主持人:“楚云飞同学,你能讲讲你取得这么好成绩的经验么?”

楚云飞:“天道酬勤,我想努力学习是最主要的原因。”

主持人:“在年初的的作文竞赛和现在的数学竞赛都取得全市的第一,你认为仅仅靠努力就能够作到么?”

楚云飞(平淡状):“年初?那是过去的事情了。”

…………………………

主持人:“最后一个问题,全国初中生奥林匹克竞赛数学班暑假要开始集中培训,咱们市就你一个名额,能说说你的目标么?”

楚云飞:“不去了,其实我有3个月没上数学课了,我的基础很差,老师都不愿意教我。”

主持人惊讶得忘记了结束语,停顿一下,说了句不专业的话:“那先到这里好了。”

蓝校长回头看看,赵学工老师也在发愣,悄悄过去耳语:“不准说话,回头把你从那个班调走,还有,找块毛巾把头上汗擦擦,什么样子。”天知道就算允许赵老师讲话他能讲出什么来,蓝校长也算白操心了。

江海中看着录象苦笑:这个小家伙,胆子还真大,不过胆子小的也想不出上次那种吓人的计划。你随随便便挨顿揍,我这里忙的电话都快炸了,幸亏跟教委、公安局、市政府的协调和解释不用我负责,要不觉都没法睡了。

编辑看着副主编在那里发愣,大着胆子问了一声:“要剪掉么?”江海中给了他个“白痴,这还用问?”的眼神,编辑用无辜的眼神反驳“这可是你指定的人选”。

回到家里,老妈已经知道了,并且积极的开通了家里电话的国际长途,向远在沙特的楚振中报喜,遗憾的是楚振中在工地现场不在办公室,叶美10分钟就打一次电话让办公室里所有没去现场的人都熟悉了她的声音,在花费了120元长途电话费后才完成报喜的任务。后来交话费时叶美思索再三还是关闭了电话的国际长途权限。

第二天,叶美的妹妹叶丽来了,还带着自己孩子陈小军,陈小军比楚云飞小一岁上初二,是个调皮捣蛋的主儿,学习成绩不是很好,尤其数学相当糟糕。

叶丽的来意很明确:首先恭喜姐姐养了个这么出息的儿子,然后就是要求楚云飞辅导他的表弟陈小军,由于陈小军上学的19中离16中比较近,所以以后陈小军下午下学以后就到叶美家里来学习,星期六星期天再回家,反正兄弟俩以前接触也少,正好融洽融洽感情,再说楚振中也不在家,多一个人还热闹点。

叶美听了觉得妹妹的要求很正常,但还是向妹妹表示楚云飞已经初三正是关键时刻,会不会对孩子的学习造成影响?谁知道叶丽对楚云飞的信心比叶美还足:云飞愿意的话估计现在就能被保送到3中或者1中。于是这事就这么定了下来。

冬去春来,街道两旁的树叶由淡绿变成浓绿,大半年时间就这么匆匆过去了,楚云飞在这段时间里过得顺风顺水,先是放弃了参加奥林匹克竞赛数学班的培训资格,又在班主任郭老师的“关怀”下拿回一个物理竞赛全省第三,得到了奥林匹克竞赛物理班的培训资格,因为赵学工的缘故楚云飞拒绝了被保送到数学教学力量雄厚的第一重点市一中,而听从了郭老师的意见选择了以偏重物理出名又离家比较近的第三中学。

在谢小亮的努力下,楚云飞和他关系近了很多,不过这家伙还是少不了惹是生非,只是没那么过分就是了,惹出了几件不大不小麻烦,搞得刘凯都不好给他说话,还是楚云飞托刘勇帮他摆平的,楚云飞常笑他是要把“前10年受的欺负全找回来才罢休啊”?黄强和迟志刚也早回来了,不过每次遇到楚云飞的时候俩人都是远远绕开,能不碰头是坚决不碰头的。王通开始动家属院里某个凶巴巴小姑娘的脑筋,在楚云飞看来他的眼光实在不可恭维。陈小军的学习在楚云飞的帮助下也有所进步,他甚至和谢小亮也混的溜熟。

由于楚云飞不用参加中考,在中考冲刺的阶段他就不上学了。痛痛快快玩了几天,在感到实在闲得无聊后,办了个图书馆阅读卡开始“泡馆”。

中考结束了,王通和谢小亮拽着刘凯来找楚云飞庆贺,经过陈小军的指点在图书馆抓到了楚云飞,谢小亮一把拎起他看的书:“夸张了点吧?《道德经》?装神弄鬼的书?”楚云飞难得的脸红了红,“这种开放式的图书馆里也没什么太好看的书,而且我总是做梦梦见老道,没什么可看的就随便找几本这种书看看,考得怎么样?”“不说考试,找个地方喝啤酒去,你又不玩电子游戏。”王通插话了。

又瞎混了半个月,实在没什么可做的,楚云飞就想要不早点去北京好了,随便转几天培训班报名的时间就该到了。于是收拾行李,向北京出发。

到了北京先去培训班看看,这么早还真有接待的人,一个比楚云飞大不了几岁的小姑娘在那里负责登记,楚云飞有些怀疑她属于勤工俭学的学生,不过再想想吹皱昆明湖的春水也不关自己的事啊。

不想到时候临时找起来费手脚,楚云飞去培训的教室转了一圈,却意外的发现门是开着的,还有比自己来得更早的?探头一看,一个满头白发的中年人坐在那里发呆,虽然这么形容他有点不太厚道,但事实上确实这样,满头白发,可面色红润,极少皱纹,尤其是一双眼睛既黑又大且亮,没有老年人常见的那种浑浊的眼神。

楚云飞正在琢磨这位的年龄,那中年人却已经发现了他,冲楚云飞微微一笑:“参加培训的吧?你来的很早啊。”“是,我家是河东省先阳市的,没来过北京,早点过来好熟悉熟悉。”。

“哈”,那中年人露出很感兴趣的样子,“你别就是那个没参加数学培训选了物理培训的学生吧。”楚云飞难免有点疑惑:“我就是楚云飞,不过……我那么有名么?”

“唉,这你就不懂了,”中年人长叹一声“咱们国家‘金牌’战略你总该是知道的吧?金牌是唯一的目标。”停顿了一下,“为了能保证夺取最多的奖牌,你们所有即将参加培训的学生的考卷都被一一仔细的分析过,你本来就是很奇怪的例子了,错的都是比较基础的内容,复杂的反而难不倒你,本来你是被比较看好的。”

说着说着中年人眼睛一亮:“不过你居然转到了物理班。我们比较奇怪,数学竞赛的奖项要远远多于物理竞赛的奖项,所以物理竞赛不是很受重视,每个省只有前3名可以参加培训。——其实隔壁楼就是数学培训班,要不我也不可能知道你。”

楚云飞的好奇心一下被吊起来了,“那你对我的物理卷子肯定也研究过了,请问老师怎么评价?”

老师也许想回避什么,和楚云飞打起了太极拳:“其实你的思维存在个误区,数学水平是可以评价的,物理怎么能和数学相比?我不是说数学不重要,而是数学是手段,物理才是目的。”说着说着,老师发现楚云飞的眼中有异样的亮光,老师的讲话兴致被钩起来了,不自觉的就跑题了。

“数学是大部分学科的基础,通过它来发展各个学科,比如说物理,你想了解结果,就要通过数学方程式计算;你想做出假设,就要拟订理论来证明,也许还需要复杂的数学模型。就数学而言,对就是对的,错就是错的,暂时分不出对错的就是猜想,但是猜想最终也只有对或者错两种可能,只要数学水平能达到解决那个猜想的高度就成了,所以我个人认为数学只是一门辅助学科。”

楚云飞忽然有种豁然开朗的感觉,还有一种说不出味道在心中慢慢绽放。

“物理就不一样了,那是种全局性学科,我们对这个宇宙哪怕是这个这个小小的地球来说,未知的东西都太多了,太多的东西不能够用简单的对错来解释,因为我们知道的太少,甚至没有权力去判断它是对的还是错的,比如说一代伟人牛顿会在生命的最后20多年里试图用方程式去证明上帝的存在。”

“所以,要想真正学好物理,就先选择你的哲学吧。”

楚云飞听得目瞪口呆,脑中似乎有个阻碍思维的塞子在瞬间被拔去,天地间一片光明,以他的年龄和阅历来说,这些理论还根本不可能接触得到,书上也是没有的,以至于楚云飞觉得现在背上背包回家都可以了,真的没有白来。其实也正因为他还年轻很多条条框框没有定型,接触得早,才能够极大的拓展他的思维能力,不过这些就是后话了。

楚云飞发现自己还不知道老师姓什么,有点太没礼貌了,“请问老师贵姓?”

老师确实有大家风范:“我就是王展强。”

楚云飞打起了小算盘,和王老师套套近乎,没准能弄点小灶充实自己呢:“您就是负责培训我们的吧?”

王展强脸上表情奇怪,有点愤怒、有点尴尬、还有点哭笑不得:“我……呃,我的讲课方式不太适合‘金牌战略’的精神。”

接下来的几天,楚云飞也没兴趣出去玩了,天天和王老师混在一起。其实楚云飞对知识的渴求程度是相当变态的,尤其是他感兴趣的东西。但遗憾的是,再也没有了初听王老师说话的那种震撼了。其实这也很好理解,王展强在初次演讲的时候已经把总纲要说清楚了,后面那种类似“直线的终极表现就是圆”这种小小的有异议的理论已经引不起楚云飞太大的热情了。

同时楚云飞也搞清楚了王老师才40岁,满头白发属于天生,属于“海归”派,教书的能力是有的,但是他的讲课方式实在有点天马行空,叛经离道,明显不属于中国风格;他本人也对中国的教育方式也不是很满意。实在不好安排,就安排他在奥林匹克课题组做个审查教材的闲差事,其实那些全国筛选出来的优秀教师对教材的把握可比他要强。

培训开始以后,楚云飞才发现并没有想象中那么复杂,一开始讲课的时间还多些,到后来做题的时间要远远的大于讲课的时间,一个月时间在纸堆中飞快的逝去。

培训结束,楚云飞参加了考试,才知道决赛不仅仅是这次考试,还要综合平常的习题完成情况。前七名是要参加国际竞赛的,毕竟是“金牌战略”,全国的奥林匹克竞赛一定得是最强者,简单的一两次考试可不能作为全部标准,所以参赛成绩在9月中旬才能出来。

楚云飞背着沉重的资料回到了先阳市,开始了他的高中生活。王通也考进了3中,谢小亮考进了普通高中9中,刘凯上了铁路建筑学校。

\section{第五章 难以承受}

虽然只是高一新生,但在3中里楚云飞也是个小名人了,老师们就算不知道他的名字也知道有个学生没去1中来了3中还去北京接受了培训,而那些不太安分的学生可是知道有“七金刚”罩着的楚云飞来了,虽然他不怎么惹人但绝对是不能招惹的。

9月15日,全国奥林匹克竞赛的数学成绩和化学成绩公布了,但是物理成绩还没出来,晚上迫不及待的楚振中从沙特打来电话询问儿子成绩,楚云飞感觉和名次相比,父亲更介意自己能不能代表国家出赛。

17日全国奥林匹克竞赛的物理成绩才公布,楚云飞名列第4,忘年交王展强以私人名义打电话祝贺楚云飞,同时在楚云飞的追问下泄露个小秘密,原来是因为第七名和第八成绩极其接近,第七名是教育成绩普通的河西省的刘阳,第八名是教育强省江南省的路菲菲。参加国际奥林匹克物理竞赛的队伍是由决赛前7名组成,由于河西省俩名额,而江南省一个名额都没有,在教育界力量强大的江南省不甘心也有点挂不住,于是想通过一些途径来变通一下,虽然没有如愿,但是耽误两天成绩的公布就是很正常的事了。

晚上,楚云飞坐在家里美滋滋的等待父亲的电话,想起昨天父亲没听到自己成绩后的那种语气,一种浓浓的亲情绕上心头……遗憾的是,今天父亲没有来电话。

第二天中午,楚云飞独自在家吃饭,电话响起,楚云飞拎起听筒:“楚振中家么?我是21冶沙特分公司的经理李林,通知你们一个不幸的消息:楚振中同志昨天在外出途中,被不明身份者绑架,目前我们正在积极地营救……”

楚云飞木然地放下电话,呆呆地坐在那里,大脑在整整1分钟的时间里是一片空白,接着就是千万种感受翻江倒海一样地在脑袋里折腾起来,楚云飞觉得自己根本无法进行正常的思考了,千万个念头慢慢地浓缩成一个:是我害了父亲!是我害了父亲!!!

楚振中所在的工地只有一部电话还没有国际长途权,办公室里也只有经理李林的电话能打国际长途,这时手机并不是很普及,具备国际长途权的手机更是少有。所以当李林不在的时候,21冶想打国际长途的员工的只能选择等李林回来或者去附近的小镇打电话。

因为时差关系,楚振中在中午休息的时候给家中打电话是最合适的,家里肯定有人而且绝对还没有睡觉。由于中午休息时间办公室没人,楚振中只好去附近的小镇打电话,很显然因为15、16日一连两天同一时间出去打电话,引起了有心人的注意,所以17日在去小镇途中被绑架。21冶的同事们因忙于工作,在晚上才发现楚振中不见了,寻找了一晚上才从目击者那里了解到了事情经过,现在正在联系相熟的当地人打探消息,相关情况也知会了当地政府并上报大使馆。

下午,知道了消息的叶美也回家了,母子俩愁眉相对,呆在家里等待着可能来的电话。21冶也派出了相关的人员专门照顾他们母子,亲戚朋友同事们也纷纷来到家里安慰。

坏消息终于在次日中午传来,21冶沙特分公司得到确切消息:绑架者为巴基斯坦籍以阿卜杜拉。马哈苏德为首的恐怖分子,他们要求沙特政府释放关押在狱中的恐怖组织“基天”的头目阿卜拉欣。巴等11人,否则就处死人质:来自中国的工程师振中。楚和法国的女记者罗密。施奈德。

实在是坏得不能再坏的消息了,绑架者不是为钱来的。但是营救工作还是在努力进行着:通过巴基斯坦马哈苏德的部族施加压力,请沙特当地家族从中斡旋,提供多种赎换方案供绑匪选择……

两天后,更坏的消息传来,绑匪不耐烦了,限定48小时内沙特政府放人,否则杀害人质。绑匪声称:虽然人质所在国家中国和法国对巴基斯坦是很友好的,尤其是中国。但是为了追求的神圣目标目标,为了战友,为了拯救苦难深重的伊斯兰兄弟……他们——没有别的选择。

9月25日——楚云飞永远的记住了这个沉痛的日子,就在这一天,他永远的失去了慈爱的父亲,在当天中午,就在绑匪限定的最后期限到来之际,在沙特方面发动的解救人质的军事行动中楚振中不幸遇难,而另一名人质罗密。施奈德在失去一条胳膊后获救。

在得知绑匪的真正目的后,楚云飞整个人就陷入了恍惚中,他清楚的知道父亲可能会遭遇什么,沙特政府不会为此释放恐怖分子的,可是他不敢再想下去,在他的脑中,只能拼命捞住那可怜的两根救命稻草——中国和巴基斯坦关系很好,有关部门在积极营救。可看到每天以泪洗面,拼命烧香的母亲,除了一遍又一遍地对母亲说“中国和巴基斯坦关系很好,有关部门在积极营救”之外,他根本不知道还能做什么,因为——是自己害了父亲!

当噩耗来临,母亲晕了过去,楚云飞最后一根弦也彻底被绷断,为什么死的不是那个女人?为什么???他不停喃喃自语:“爸爸,是我害了你……爸爸,是我害了你……爸爸,是我害了你……”

当楚云飞从沉痛中略微清醒,心碎的母亲已经到达了遥远的沙特,然后,他开始慢慢地调整自己的心情。3天后,母亲携带着父亲的遗体归来,楚云飞沉静地陪着母亲送走了父亲。在那平静的面孔下,也许只有王通能稍稍了解那里埋藏着多少深深的自责……

父亲“二七”过后,楚云飞开始上学了,那出奇冷静的面孔使得善良的老师和同学们都不忍心再去怎么安慰他。然而,年少的无知使得同班的同学在很短的时间内就忘记了有些东西需要避讳。就是当天,在楚云飞听到有个知道内情的同学在向另一个同学说:“……他爸爸想快点知道他考了第几……”后,他立刻离开了学校。

那天晚上,楚云飞做了个梦,梦见了自己在硝烟弥漫的战壕中,身边的战友被飞来的炮弹炸得身体四处飞溅,被飞机的子弹打成一摊一摊,离自己最近的依稀叫“黑蛋”的战友大睁着眼睛,红白相间的液体从他额头正中的的洞中缓缓的向外弥漫……抖掉身上的泥土、铁屑和碎肉,望着远远在空中飘荡红色膏药,胸中有种感觉似乎要把肝肠压断……。。。。。。。。。。。。

江海中平静地接待了越发苍白的少年,发生在少年身上的事情他早已经知晓。正是因为这个少年的策划,他犯了个理智的错误;少年的后期合作,使得错误越发美丽,这个美丽的错误给自己带来了得到新的栏目和主编头衔的机会。

楚云飞向江主编坦白了自己的想法,他不想再在这个环境这样的下去,他想参军,也许没有亲手击毙那些魔鬼的机会,但是这样能让他稍微地解脱一些,起码能让他向这个目标靠得近点。在天上的父亲希望自己的孩子努力学习好好做人,可父亲也绝对会愿意让儿子过得开心一点的。虽然他的年龄不够参军资格,可他也知道有很多有办法的人能让自己穿上军装,遗憾的是:这些有办法的人里只有眼前的江主编还熟悉点。

江主编不能再对楚云飞做什么说教,他充分相信眼前这个少年的智慧,也能理解少年心中的沉重,更不想少年因为自己的拒绝而产生什么报复行为,那会影响自己前程的。虽然不认可楚云飞把错误都揽到自己身上的心态,但是他还是婉转的表达了自己的意见:我需要得到你的母亲的认可才能够帮助你。

母亲无奈的看着儿子,她真的不想再失去任何东西了,但是她太了解自己的儿子了,所以她感受到了儿子倔强和差到不能再差的状态。她没有哭天抢地,只是轻轻地劝说了几句,看到低着头的儿子一言不发,长长叹了口气,沉吟半天,她拿起了电话。。。。。。。。。。。。。。。

楚云飞甚至没有记完全同班同学的名字就穿上了军装,只是为了逃离这个环境和一个遥不可及的目标。其实他的愿望中还是希望母亲或者江主编能给他个不参军的好理由,不过很多年后谈起此事,所有的人都怀疑给他个理由他真的会留下来么?也许好的理由根本就不存在。

