\part{第一部 南楚状元}

\chapter{第一章 落魄书生}

显德十六年,随云欲科举,遂离江夏,往赴建业。

--《南朝楚史·江随云传》

南楚显德十六年,天下还在纷乱当中,但是局面已经清楚多了,长江以南大部分被南楚占据,江北则是大雍的天下,江夏是防守大雍的战略要地,而镇守江夏的镇远侯府乃是军机重地,所以时时刻刻守备森严,我这个西席虽然地位不低,但是也得乖乖的俯首听命,躲在书房里面尽量不要外出,免得惹祸上身。我一边翻着书本一边盘算着什么时候能够吃饭,没办法,镇远侯陆信乃是军方重臣,按照南楚的惯例,他的家人都要留在建业,只有十五岁的世子陆灿被陆信任命为侍卫留在身边,这个朝廷倒是允许的,陆灿虽然跟着我学文,但是武将世家的子弟自然也要学习军事,今天是江夏大都督陆信召开军议的日子,陆灿作为侍从被带去旁听,我就只好在书房等他了,原本说好了一起用饭的,不料今天的军议过了晌午也没完,而且所有参加军议的人都没有用饭,我这个小小的西席若是自己吃饱了,等陆灿回来一定得嫉妒的大喊大叫,然后又要找机会暗算我,我还是等他一起吧。想到这里,我摸摸扁扁的肚子,无奈的叹了口气。你说父子两人怎么差那么多,陆信慷慨大方,陆灿却是斤斤计较,上次他被陆侯爷责罚,我忍不住偷笑了一下被他看见,第二天就骗我出去散心,说什么我守孝已满三年,应该出去走走,结果把我骗进了烟月楼,要不是我见机溜的快,我的第一次就被抢走了呢。我一边胡思乱想一边无聊的翻着书本,唉,镇远侯府的书房虽然不错,但我这三年几乎都看完了,而且毕竟是武将世家,所以都是一些比较易见的书,我估计是让书铺把所有的书都送了一份,要不然怎么连黄历都有,可是没有什么真正的珍品啊。

我正在那里看着日影计算时间,这时陆灿的侍从陆忠来了,告诉我说,军议已经结束,陆信宴请下属,让陆灿也去作陪,让我不要等他了。我高兴的答应了,也不管饭菜已经凉了,就去狼吞虎咽起来。正吃的高兴呢,突然前面的大厅里传来一阵喧哗,开始的时候,我还没有在意,但是后来声音越来越响,只听见震耳欲聋的“抓刺客、抓刺客”的声音。我心里一震,糟了,这里有刺客,十有八九就是镇远侯遇刺,他现在可是我的靠山啊,可不能被刺客杀了啊。我知道自己没有本事保护镇远侯,还是躲起来的好,可是心里忐忑不安,从书架上拿起一具精巧的弩弓,这是南楚工部精制的弩弓,射程可以达到百步,可以连续射出五支弩箭,这原本是陆信送给陆灿的礼物,可是陆灿嫌弩弓不够光明磊落所以不喜欢使用,反倒便宜了我,谁让我不会武功,弓箭是肯定用不了的,这具弩弓才是我的最爱,将弩箭上好,把窗户打开一个缝向外看去,我呆的这个书房离前面的大厅不是特别远,只见外面刀枪如林,一大堆红衣军士正围着两个仆人装束的汉子厮杀,不一会儿,我看见镇远侯陆信在部将的陪同下赶来了,他的右臂缠着白布,血迹殷然,而经常在他左右的亲信侍卫陆平却不见踪影,只见陆信面上一片惨白,扶着他左侧的是陆灿,神色十分愤怒,见这样子,我猜到定是那两个刺客混进侯府,看情形可能是在陆信宴请下属时伪装上菜的仆役,然后突然行刺,我估计陆平八成已经尽忠职守了。

我正看得起劲,只见那两个刺客突然互相使了个颜色,突然从怀中掏出两颗黑色的圆珠子摔在地上,顿时白烟滚滚,片刻就将方圆十几丈的空间给遮住了,正在这时,我看见离陆侯不远处有一个身穿偏将服色的将领眼中闪过一丝凶光,一把匕首从袖口滑落到他的手里,我心知不好,连忙大叫道:“侯爷小心。”一边喊着,一边射出了一支弩箭,一声惨叫响起。等烟雾散尽,惊魂未定的众人看去,那两个刺客仍然被围在当中,而陆侯身后,一个偏将倒在地上,心口中箭,而他的手里仍然紧紧握着一把匕首,匕首的锋刃上泛着蓝光,而且离陆侯不到半步的距离。这情况就是瞎子也知道是怎么回事了。

看着那两个刺客难以突围,最后力战而死,陆侯下令部将善后,就把我召到了他处理军务的白虎堂。他神色复杂的望着我,问道:“多谢随云相救本侯一命。”我谦逊地道:“都是侯爷福德深厚,才能避过奸人陷害,晚生只是侥幸罢了。”陆侯疑惑地问道:“随云如何知道那人要行刺本侯呢?”这是他百思不得其解的问题。如何知道,当然是我看到的,可是我可不能这么说,这是我的防身法宝呢,我的六识天生异于常人,这么说吧,我的耳力,百步之内,可以听见落叶飞花,我的眼力,数里之内纤毫可见,我的味觉,什么东西,只要一沾唇,我就能分辨的一清二楚,我的嗅觉,只要一丝气味,我就能跟踪他十里八里,有的时候我都怀疑自己是不是人,不过我也知道,那些事情若给别人知道不免遭人嫉恨,你也不想有一个有一个人可以偷听你的私语吧,为了留作防身利器,这些事情我可是从来不告诉任何人的,除了我死去的父亲,没有任何人知道。所以我编了个谎话道:“说来也巧,晚生本来是拿着弩弓防身的,看见那两个刺客放出烟雾,不免觉得奇怪,想来不论那两个刺客如何本事,这种情形也难以脱身,放出烟雾一定是想给别人造成机会,所以晚生才会认为一定还有刺客藏身在侯爷左右,一时心急喊了出来,记得当时大人身后无人,想必刺客若要行刺,然后从那里来,所以胡乱射了一箭,幸好侯爷德厚,才能杀死刺客。”

陆信半信半疑的点点头,让我出去了。后来我听说行刺陆信的是大雍的刺客,他们收买了那个偏将,想刺杀镇远侯,然后趁着江夏群龙无首的时候来攻击,谁知万无一失的行刺计划却失败了,所以他们大军又退了回去。事后,陆信见我聪明多智,想让我进入他的幕府,可我一想,他这里和大雍隔江而望,经常要打仗的,如果一时不幸兵败,我可怎么办才好,而且,如果大雍知道了是我救了陆信,派刺客杀了我怎么办,所以我拒绝了,当然我不能用这个理由,就说是我父亲生前颇以没有功名为撼,所以我决定参加科举,这可是一个冠冕堂皇的理由,谁也不能挡着我上进不是。所以陆信不仅派人去我的原籍嘉兴为我取得了考试资格,还在恩科开考前的两个月,就送我盘缠,让我到建业赴试,为了我的人身安全,还让我跟着他们负责军需的人员一起走,无奈之下,我只好跟着那些人一起上路,好在路上我想了一个办法,说我偶感风寒,时间又还来得及,所以休息两天再走。于是我终于恢复了自由身,我又不是白痴,南楚在显德九年向大雍称臣,去帝号称国主,现在又有传言说国主想要恢复帝号,这样一来一定会惹恼大雍的,将来一定是兵祸连绵,我虽然不想去打仗,可是兵法我可懂得不少,人家大雍是兵强马壮,而南楚却是君臣醉生梦死,将校贪生畏死,就是有名的名将陆侯麾下,我听说也有不少胆小鬼呢,气得陆侯几次都要把他们斩了,可是碍于他们家族的势力,只能把他们养起来罢了。在这个时候考科举,我还不想作亡国之臣呢。

抱膝坐在一艘客货两用船上的甲板上,我舒舒服服的享受着夜晚清新的江风,这种中型船只,底舱都装满了货物,上面的船舱则隔成一些小房间供客人使用,绝对比那种专用的客船舒服,只是价格也贵上许多,不过,现在我腰里有几百两银子,怎么也够用了,所以我就奢侈上一回。看着清寒的明月,寥廓的星空,我不由诗兴大发,吟诵道:“细草微风岸,危樯独夜舟。星垂平野阔,月涌大江流。名岂文章著,官应老病休。飘飘何所似,天地一沙鸥。”正当我反复吟诵的时候,只听见身后有人拍掌叫好,我回头望去,只见一个青年站在那里,虽然月色昏暗,可凭着我的眼力,清楚的看到站在那里的是一个英俊威武的青年,虽然穿着便服,可是气势不凡,我怎么看都觉得比陆侯爷还要威严,而且他身上仿佛有一种惊人的魅力,令人如沐春风,有点自惭形秽的看看自己,身材普普通通,只是没有风吹即倒罢了,相貌虽然还算清秀俊美,可是怎么看都是一个文弱书生,现在兵荒马乱的,最吸引女孩子的还是文武双全的英俊公子,就是一个武夫,即使大字不识几个,只要稍微斯文一点,也比我能够吸引女孩子的眼光呢,问我怎么知道,当然是因为陆侯府上的那些侍女从来不正眼看我的缘故。

我站了起来,抱歉地道:“打扰阁下休息,真是抱歉。”

那个青年摇头道:“那里话,若非我没有休息,岂不是要错过公子这样的好诗,请问可是公子的作品么?”

我心里欢喜,面上却谦逊地道:“拙作难登大雅之堂,阁下见笑了。”

那青年上上下下打量了我半天,才道:“公子年纪轻轻,文才如此出众,真是佩服,在下李天翔,乃蜀王治下行商,这次到建业办事,请问公子尊姓大名,到建业何为?”

我心里嘀咕,这人虽然是蜀地口音,可是我听着总有一些别扭,但是别人的事情我管那么多,所以我客客气气地道:“晚生江哲,字随云,这次到建业是去赴考的。”

李天翔目中闪过一丝古怪的神色,道:“公子才华绝世,想必是蟾宫折桂,轻而易举了。”

我尴尬的笑了笑,如果不是为了圆谎,我跟本不想参加科考,反正我有办法避免中举,又让别人说不出什么来。李天翔见我窘迫,也不在说及科考的事情,感慨地道:“唉,这次从蜀中来,看到中原局势紧张,在江夏又几乎遇上战事,现在的生意越来越不好作了。前阵子南楚国主下旨增加关税,幸好蜀王国主遣使到南楚谈判,要不然我们的货船就要赔本了!”我随意地道:“其实蜀王国主根本不必费心,南楚、蜀国唇齿相依,只要把这层关系说透,国主一定会降低关税,甚至还会提供通商的优惠呢?”

李天翔微笑着问道:“这怎么说呢,在下可是不明白。”

难得遇到有人想知道我的看法,我得意地道:“这就要从当今天下的局势说起,当今天下,南楚和大雍对峙南北,但这只是表面的事情,不论军力民心,南楚都不及大雍,只能防守,无力进攻,所谓刚不可久,柔不可守,大家都知道这样下去,南楚迟早必亡,所以当今国主才会向大雍求和,去帝号,称国主,以求苟安,可是现在情势已经不同,蜀中在贵国治下,兵精粮足,虽然蜀国因为地理的限制,只能是一个偏安的格局,但是对我南楚,却是居高临下的强势,如果蜀国和大雍联合,大雍猛攻长江,蜀国临江而下,我南楚必然灭亡,单若蜀国严守蜀中,而我南楚和大雍北方的北汉联合,一旦雍军攻南楚,北汉从北面和南楚呼应,而大雍面临长江天险,只要守到三月以上,大雍必然退兵。”

李天翔面色肃然,良久才道:“若是这样,岂不是天下永难一统,只是苦了我们这些老百姓。”

我安慰他道:“我说的不过是理想中的情况,现在南楚君臣有些自大,认为长江天险可恃,危机隐伏,如果大雍有明智之士,还是有统一的可能的。”

李天翔似乎有些好奇,问道:“公子刚才不是说大雍难以为继么,怎么又说大雍还有可能一统天下。”

我理了理思路道:“虽然大雍处于百战之地,但是它的优势明显,上有明君贤臣,下有大军百万,只要战略正确,二十年内定可一统天下。现在天下的格局,蜀地才是关键,只是蜀中易守难攻罢了,若是想要夺取天下,首先便要结好北汉,安定后方,然后就要离间蜀楚。”

李天翔疑惑地问道:“结好北汉还是有路可循,蜀楚唇齿相依,如何离间呢?”

“这有什么难的,我听说近来南楚朝中有人想恢复帝号,如果大雍此刻表现的束手束脚,难以为战,南楚君臣必然迷惑,若是大雍再派遣细作,以甘言厚礼贿赂宠臣,促使南楚恢复帝号,那么南楚和蜀国之间的隔阂必然加重,到时候就连北汉也不免心中疑忌。到时候大雍暂时承认南楚称帝,两国划江而治,然后再和南楚联手攻打蜀国,南楚君臣短视,必然上当,虽然蜀中难攻,但是也难以抵挡两国攻势,到时候蜀国必然痛恨南楚,只要大雍策略得当,必然能够得到蜀中大部,然后大雍两面夹击,必然可以灭掉南楚。等到这时,就可以养精蓄锐,一举破汉,何愁天下不定。”

李天翔听得眉飞色舞,道:“看来只要我蜀中和南楚结好,就是大雍再大的本事,也没有办法,幸好江兄你不是大雍的子民,如果你去了大雍得到重用,我们蜀国可就危险了。”

我懒洋洋地道:“我才不去大雍呢,听说那里以军功为重,像我这种文弱书生,到了那里可是吃不开的,等过几年,我多挣点银子,到乡下买几亩地,娶个温柔贤惠的妻子,才是人生乐事呢?”

李天翔笑道:“那我就祝贺阁下如愿了,不过听你的计划,大雍应该不需要二十年的时间吧。”

我已经有了困意,道:“本来是不需要的,但是如果到攻下南楚为止,有个五六年就差不多了,可是我听说大雍的皇帝陛下春秋已高,太子李安虽然是储君,可是声望军功远不如次子雍王李贽,当初大雍立国的时候,雍帝李援因为次子李贽功高,所以用国号赐他封号雍王,原有立储之意,可是之后大雍典章制度一一齐备,李援又根据立嫡立长的制度立了李安为储君,所以萧墙之乱难免因此而起,搞不好大雍因此分崩离析也不一定,我说二十年还是在内乱不会范围太大的前提下呢。”

李天翔微微低下了头,良久道:“是啊!”

我不明白他话中的意思,也懒得去想,就告辞回舱了。第二天起来,我听说李天翔已经提前下船了,真是奇怪。

本来我的打算是不错的,可谁知道天意难料,我到建业的第一天就成了穷光蛋。

回想那时,我第一次看到建业,看到那虎踞龙盘的都城,真是瞠目结舌,所以在落店之后就出去游玩,在雍淮河畔的夫子庙,我遇到了一个命中的福星,当然当时对我来说,他就是我的灾星。

我正沿着街道溜达,突然看见前面聚了一堆人,忍不住好奇的钻了进去,却原来是一个小孩在卖身葬父,我一下子想起当初父亲去世,我囊空如洗,如果不是有机会进入镇远侯府,怕我也只能卖身葬父了,一时冲动,我掏出一百两银子给了那个小孩,他清秀的脸上露出感激的神色,恭敬地道:“公子,等小的葬了父亲就去伺候公子,请问公子住在哪里?”我尴尬的笑了笑,看看周围人群中射来的嫉妒眼神,心想财不露白的古训我已经犯了,难不成还告诉别人我住在哪里。也没答话,匆匆忙忙地就跑了,为了迅速回到客栈,我低着头飞快地走着,走到一个巷口的时候,只觉的身后有人靠了上来,我还没来得及回头,就觉得硬邦邦的东西顶住我的腰,于是我老老实实地被带进巷子,然后就觉得后脑勺被打了一棍子,等我醒来,我已经囊空如洗的躺在地上了,哭丧着脸回到客栈,万分庆幸当初存了十两银子在柜上,可是这点银子我顶多能住一个月,怎么办,怎么办?我辗转反侧了一个晚上,才想到唯一的解决办法,就是我认认真真的参加科考,然后取个名次,然后我就有官俸可以花,有官府给的宅子可以住了,想来南楚应该不会很快亡国吧,等我赚足了银两,我就可以辞官归隐了,到时候应该没有人和我这个没有官职的人过不去吧。

\chapter{第二章 金榜题名}

显德十六年六月,江哲入建业,八月,金榜出,江哲中一甲头名,赴琼林宴,宴未毕,雍使入朝,求联姻,以示盟好。

显德十六年十二月,雍长乐公主入楚,显德十七年戊辰元月,太子殿下赵嘉举行大婚,立长乐公主为太子妃。

长乐公主,年十五,母长孙氏,雍高祖贵妃,素得帝宠,长乐公主生时,逢雍高祖登基,故颇爱宠之,赐封号长乐公主。

--《南朝楚史·江随云传》

从会试考场出来,我伸展伸展四肢,唉,这几天可把我辛苦坏了,那个考棚又窄又小,我又没有银子打点,所以到了第三天,基本上屋子里面全是马桶的气味了,如果不是以前跟着爹爹流落他乡,吃了不少苦头,只怕我连饭都吃不下去,只怕我省吃俭用到了今天,身上就连一个铜子都没有了,离放榜还有半个月呢,这些日子我可怎么办呢,要不要去卖字画或者替人写书信,我认真的想着。

回到客栈,我计算一下,明天的房钱是没有了,所以拿着文房四宝,决定到夫子庙去摆摊,到了夫子庙,跟一个小茶馆的老板套了半天近乎,又答应替他写两封信,就在他的茶馆门口摆上了摊子,可惜生意不大好,到这里写信的人都是大字不识几个的,谁管你字写的怎么样。我等了半天也没有生意,正在愁苦的时候,一个青衣小妇人走了过来,我一看她的装束,就知道是个寡妇,可是年纪只有十八九岁的样子,真是可怜啊。她怯生生地道:“先生,奴家想写副状子。”我拿起笔道:“是什么状子,要告谁啊?”

她有些赧然地道:“奴家的丈夫不幸去世,奴家想要改嫁,可是公公不同意。”我又问了几句具体的情况,拿起笔写道:“十七娶,十八寡,公壮叔大,瓜田李下,嫁与不嫁?”她莫名其妙的看着我写得字,问道:“先生,这个几个字,太少了吧。”我得意地道:“你放心,这状子递上去,保证官府同意你改嫁。”她给我十个铜子,我满怀感激的望着铜子,心想,今天的晚饭有了,还得努力,明天的放钱还没有呢?接下来我又没有生意了。过了不到一个时辰,只见那个小寡妇喜气洋洋的回来了,一见到我就感激涕零地道:“先生,谢谢你的状子,大人一看到我的状子就准了。”我心想,那当然,现在的建业京兆尹是十分重视伦理道德的,寡妇改嫁,不过一人失节,若是发生乱伦丑闻,就是大事了。这个小寡妇一走,我的生意就好起来了,到了晚上一看,足够两三天的房钱了,当然我没有敢多写状子,如果有人来写状子,我总是变着法的劝他不要告状,不是为了别的,讼状写多了是要损害我的名声的。

在夫子庙写了几天信,我看差不多足够我在建业等到放榜了,就收了摊子,在小茶馆里面听人聊天说笑,反正一壶茶可以让我呆上一天,当然我虽然不作生意了,如果有人来找我写信,我还是干得,只是要多收几个铜子。反正消磨时光么。过了一两天,我一时手痒,用我学得一点易经给人测字算命,说句实话,我算命不大准,只是凭着一点易经心算,再加上我的观察能力,很快得就成了神算,当然我银子够花就行了,所以我一天只算三课,每天还奉送一课,说也奇怪,我这样倒是引起了不少人的好奇,所以银子如流水一般滚来。当然,为了掩人耳目,我改变了装束,又在相貌上做了点改变,也就是用药物涂面,使肤色发黄罢了。

这天快到午时了,我已经算过了三课,决定再算完免费的一课就收摊,这时一个小伙子匆匆忙忙地走来道:“先生,我是个行商,前两天收到同乡带来的口信,说我的妻子快要临盆了,可是身体不大好,我连忙赶回来,还没回家呢,不知怎么搞得,我心里很不安,您给我算算,这一胎是否平安,是男是女。”我将算筹摆了半天,才道:“没问题,小危则安,尊夫人本来有些凶险,但是你们夫妻平日积德行善,应该会顺产,你是子女双全的命格,老兄真是好福气。”问我怎么知道,我还真不知道,这种事情可是算不出来的,不过总不能说难听的话吧,把他急个半死怎么办,不过我看他相貌忠厚,身体不错,听他的口气,夫妻也颇为和睦,那么子女应该没有什么问题,至于他说妻子身体不大好,当然了,快要临盆了,丈夫还不在,心情哪里会好,这小子一回去,他妻子一高兴,一定会顺利生产的。至于是男孩是女孩我可没有明说,到时候也好搪塞。这个小伙子高高兴兴的就要给钱离去,我告诉他这一课是奉送的,他正在感谢我,一个中年汉子跑了过来,高兴地道:“老三,你可回来了,弟妹生了,一对龙凤胎啊,快回去,快回去。”那个小伙子一听,呆了半晌,突然狂奔而去。我吁了一口气,正在庆幸的时候,旁边的人都以崇拜的眼神看着我,看得我不好意思起来。

这时,一个坐在门口的灰衣人站了起来,走到我跟前,淡淡道:“先生给我算一课如何。”

我抬头望去,只见这人不过二十七八岁的模样,身躯挺拔矫健,年轻英俊的脸上透着沉稳的神色,他身后跟着一个青衣儒服的中年人和一个黑衣劲装的随从。我犹豫地道:“在下今天卦数已满,这个……”

那灰衣人淡淡道:“我也知道先生为难,只是我明日就要离京,所以请先生勉强为之。”

我看看这三个人,那灰衣人眼中满是命令的神色,想必是令出禁止的人物,而那个青衣人虽然有些不屑,却也有些期望,至于那个随从却是满脸的威胁。看到是得罪不起的,我算算日子,后天就要开榜了,就道:“也罢,在下恰好也要歇业了,这一卦就算是我的收山之作吧。”

那灰衣人有些惊异,似乎以为我是因为要给他算命才被迫如此,但是他心中疑惑难解,只得问道:“我即将远行,请问此行是凶是吉?”

我将算筹摆了半天,道:“坎卦上六,系用徽□,□于丛棘,三岁不得,凶。阁下此行怕是碍难重重。”说到这里我偷眼看看他的神色,心想,你这种人平日大概自信慢慢,既然你都犹豫不决的问卜,那事情必然棘手。那灰衣人神色灰暗,片刻又道:“请问先生,何处碍难。”这我怎么知道,我想了一想,心道这人从气度举止看起来应该是从军之人,见他身边这两人,一个应该是幕僚,一个应该是护卫,这人身份应该不简单,现在南楚有什么大事么,不管什么大事,我只要含糊其词就行了,想到这里我说道:“内有纷争,外有强敌,事情难办,若是阁下小心谨慎,或有可能。”我虽然说得含糊,可是却正好迎合了灰衣人的心理和朝局。灰衣人叹了一口气,转身离去了,那个青衣人取出一张银票放到桌子上,我等他们走远了,仔细一看,一千两,差点叫出声来,连忙塞到怀里,然后收摊,走人。

又过了几天,已经是八月十五了,今天是金榜出来的日子,我有些犹豫,如果是几天前,我当然盼望金榜题名,可是我现在囊中颇丰,倒是有些后悔可能会考上呢,所以我没有去看榜,在房内翻阅自己的诗稿,没有多久,听见外面响起噼里啪啦的鞭炮声,一名伙计和掌柜的兴冲冲的推门进来,高声报喜道:“恭喜老爷,贺喜老爷,恭喜江老爷高中一甲头名状元,小店真是蓬荜生辉,还请状元老爷得空给小店写几个字。”我有些迷茫的望着窗外,不知道前途如何。转念一想,反正我未必就赶上亡国,而且听说南楚翰林院的藏书楼藏书百万,是天下最大的藏书楼,我又高兴起来,听说南楚国主去年下诏收集天下图书字画,要建立崇文殿以传世,想必我会有机会参与呢。

当天晚上快到酉时的时候,我带着号牌到了会试院门口,门口聚集的新进士个个穿戴一新,神采飞扬,等我到了门口,却见所有人都以异样的眼神看我,有得还带着嫉妒的神色。我正奇怪呢,一个方面大耳的书生走了过来,问道:“这位兄台可是赴琼林宴的新进士么?”我点了点头道:“正是,请问有什么事情么?”那人闻言顿时露出尊敬的神色道:“原来是新科状元到了,失敬失敬,在下刘魁,真是本科的一甲第二名榜眼。”原来我来之前这里已经到齐了其他七十九名进士,只等我这个状元了,我这才明白为什么那么多人眼中带着异色。那些新进士一个个都过来寒暄,我正应付不来的时候。听见三声钟响,一个大官带着一些考官出来了,一个个检查我们的名牌,核实我们的身份,让我们排列起来随他入宫,我这个状元自然走在最前头,身后左右就是榜眼和探花,而其他七名一甲进士则跟在我们后面,另外七十名进士则七人一排的排成队列。走在往皇城的路上,道路两边都是看热闹的百姓,我们走过之处,欢声雷动,队伍在朝阳门进了皇宫内城,朝阳门是内城的大门,平日里除了皇上之外是谁也不能走得,除了皇上之外,就只有我们这些新科进士在赴琼林宴的时候可以走一回了。走进了内城,我不时看到假山花木之后有女子的嬉笑声传来,想必是那些宫女在偷看我们吧。

终于走到了琼林苑,我们在司礼监的官员安排下各自落座,所有的进士和主考官分别按照名次地位坐下之后,只听见司礼太监尖声道:“国主驾到。”只见一个身穿龙袍的老者在一群宫女太监的服侍下走了进来,我跟着众人跪伏在地,认真无比的喊道:“国主万岁万岁万万岁。”国主有气无力地道:“众卿平身。”我们站了起来,这个琼林宴总算要开始了。在按照礼仪一样样进行之后,我们终于可以放心的品尝御膳了,真是好吃啊,如果可能,我真想把御膳房的厨子弄回家做菜。酒过三巡,菜过五味,众人都有些放开了。

这时,赵胜放下筷子,对主考官说道:“史爱卿,为孤引见一下今科的前三甲吧。”主考官连忙起身行礼道:“臣遵旨。”然后指着我道:“禀国主,这位是今科会试的一甲第一名状元,嘉兴江哲。”我连忙离座跪倒道:“臣江哲叩见国主。”赵胜微笑着道:“好好,果然是年少英才,你的文章写得不错,尤其是那首《月下感怀》,孤已经命人重新谱曲,一会儿让大家都听听。”主考官又指着榜眼和探花道:“禀国主,这位是第二名榜眼江宁刘魁,这位是第三名探花淮扬伏玉伦。”赵胜一一赞叹了几句,然后吩咐我们归座。待我们落座,赵胜一摆手,不一会儿一队女乐从后殿飘出,有的吹箫抚琴,有得偏偏起舞,一会儿,一个女子曼声唱了起来道:“明月几时有,把酒问青天。不知天上宫阙,今夕是何年。我欲乘风归去。又恐琼楼玉宇,高处不胜寒,起舞弄清影,何似在人间。转朱阁,低绮户,照无眠,不应有恨,何事长向别时圆?人有悲欢离合,月有阴睛圆缺,此事古难全。但愿人长久,千里共婵娟。”正是我考试时的作品。殿中所有的人都沉浸在那美丽的情怀当中。

正在这时,一个太监进来禀报道:“启禀国主,丞相大人求见。”

赵胜漫声道:“什么事啊,孤正在这里举行琼林宴,有什么其他国务,就让他先处理吧。”那个太监道:“丞相大人说是有急事。”赵胜无可奈何地点头道:“好吧,让他进来吧。”不一会儿,一个穿着一品官服地老头子兴匆匆的走了进来,一见到赵胜就跪下道:“恭喜国主,贺喜国主,大雍遣使来朝,转达雍帝旨意,欲和我南楚结为姻亲。”赵胜面带喜色,有些不信地道:“此话当真。”那个老头子点头道:“正是如此,雍帝有一爱女,年方及笈,愿意许配我国太子为妃,从此两国和好,永不交兵。”赵胜大喜道:“今日真是双喜临门,我南楚新得栋梁之才,又和大雍结好。来人,速召雍使觐见。”说罢,赵胜起驾离去,我一生中唯一一次的琼林宴就这么虎头蛇尾的结束了,不过大家听到好消息都是面带欢容。我却有些疑惑,怎么大雍会突然结好南楚呢,难不成真像我策划的那样,不可能,我摇摇头。

之后几个月朝廷上下忙的要死,我则是按照惯例进了翰林院,高高兴兴的投进了藏书楼,只是隐隐听说,雍帝的女儿长乐公主容貌秀美,甚得雍帝宠爱,不过我想,一个刚刚十五六岁的小女孩能够多美丽,经过几个月的运作,完成纳彩、问名、纳吉、纳征、请期、亲迎的六礼之后,就在新春华旦之时,长乐公主正式和南楚太子举行了大婚,我作为新科状元有幸参加了婚礼,婚礼之后,太子殿下和太子妃接收群臣朝拜的时候,我终于看到了长乐公主的真容,当真是雍容华贵,绝色出尘,虽然年纪还小,不免有些稚嫩,但是当真是美丽啊。比较起来,旁边的太子殿下,虽然二十出头,但怎么看怎么觉得也别黯然失色。当然此时大家都在说什么“郎才女貌,天作之合”之类的鬼话。不过想来雍帝不会那么无情,用自己最爱的女儿来假意结好吧,我还是希望南楚不要和大雍打起来,虽然说长痛不如短痛,早点统一的好,但是我还是想多过几年舒心的日子,所以我诚心诚意的祈祷起来。希望大雍真的和南楚结好,让我过上几十年太平的日子。

在我诚心祈祷的时候,乐官开始奏乐,演唱的正是我这个刚刚出炉的翰林学士的新作《青玉案》。

“东风夜放花千树,更吹落、星如雨。宝马雕车香满路。凤箫声动,玉壶光转,一夜鱼龙舞。蛾儿雪柳黄金缕,笑语盈盈暗香去。众里寻他千百度。蓦然回首,那人却在,灯火阑珊处。”

乐声中宫女们翩翩起舞,我抬头望去,却看见长乐公主微微侧过头去,从她的眼角,一滴晶莹的泪珠无声无息的滑落尘埃。我心中一凉,这个孤独的少女从此就要在异国他乡度过自己的一生了,从此不能和父母家人相见,这还是从好的前景来看,如果,如果大雍只是假意结好,虽然我希望不是,可是我可不敢那么肯定,那么这个少女将要面临的是多么严酷的结局啊。这时,我看见太子殿下低头在公主耳边说了什么,虽然有些太远,声音又杂乱,可是我还是隐隐约约的听见太子殿下告诉长乐公主,这首《青玉案·元夕》是新科状元江哲的作品。长乐公主顺着太子殿下的目光向我看来,微微一笑,那笑容如同春花绽放一般,令我心中不由一颤,连忙低下了头,不知怎地,心里竟然生出一丝莫名其妙的感觉。

\chapter{第三章 翰林学士}

显德十六年九月,江哲入翰林院,依例授翰林院编修,职七品。

显德十七年元月,哲以博学多闻,特诏参与筹立崇文殿,历三年,哲精于鉴赏,明于考证,每每废寝忘食,手不释卷,闻者皆赞叹不已。未几,迁升翰林院修撰,从六品。

崇文殿典藏,均留存至今,卑人曾见之,十之六七均为哲校订品鉴,令人为之瞠目。

--《南朝楚史·江随云传》

真是幸福啊,我伸伸懒腰,拿起手里的孤本诗集,这些日子以来,我都在翰林院的藏书楼里边呆着,这里不愧是天下藏书之最,有很多我没有看过的书籍,我有过目不忘的本事,从前就看过很多书,基本上一本书只要看个一遍,就可以记住大概了,好的文章我还能一字不漏。不过我就是再大的本事,这上百万的书籍我也看不过来,所以找了一本藏书索引的册子,按照上面顺序拣一些没有看过一一看去,反正我在翰林院得呆个三五年,怎么也看的差不多了,当然我最留意那些注明孤本的书籍,要知道这样的书籍好多都是绝世之作。

这一天,我在书库里面正在找书看,无意中看见一本黄绫册子,看外表十分精致,想必是难得的精品,我随手翻开一看,差点没昏过去。首页血淋淋的八个大字“欲练神功,挥刀自宫。”我连忙合上,看看封面,却是什么《葵花宝典》,连忙扔到一边,我可还想娶妻生子啊。这时看到旁边有一本汉代的庄子《养生主》,连忙拿了起来,翻了几页,虽然和外面见到的文字差不多,但是眉批很丰富,密密麻麻的几乎写满了空白,我是很喜欢看别人的注解的,那里面凝聚着读书人的心血啊,看看旁边没人,我随手扯过垫脚的凳子坐了下去,到外面看多浪费来回的时间啊。这一看我可是着迷了,原来这个写批语的人可能是一个道士兼医生,写得都是一些养生的秘诀,什么时候该吃什么,该喝什么,几点起床,几点睡觉,如何在睡前打坐,如何在起床的时候练气,甚至连房中术都有,真是我的最爱啊,你可别笑我,我的最大愿望就是活的舒舒服服,无病无灾,娶个温柔贤惠的妻子,生几个可爱的孩子,房中术也很重要啊,你没见那些好色的人都经常短命么,就是不节制自己,不会养生啊。我正在高兴呢,突然想到,不行啊,我怎么知道他说的对不对,怎么办?想来想去,如果有疑惑就要自己解决。于是,接下来的一个月,我就在书库里面找寻养生方面的资料,有些互相矛盾,有些相互印证,我是谁啊,我是天才啊,终于让我整理出一套自己的养生要诀,并且开始付诸实施。

怎么做呢,首先,我每天一睁开眼睛,先静坐一会儿,练练养气之术,然后出去活动活动手足,练拳虽然不会,但是什么五禽戏还是可以的,然后吃上一顿清淡的早饭,再出门做事,中午若是没有什么事情,当然最好的就是回家,吃上一顿符合节令的滋补午饭,最好吃得晚一些,睡个午觉之后,喜欢干什么就干点什么,晚上若是有应酬一定要少喝酒少吃菜,等到回家之后,在睡前喝上一杯自己酿制的药酒清清肠胃,然后打坐半个时辰,再好好睡觉,而且平时坐卧行走都按照某种特定的姿势,当然看起来不能太明显。虽然我现在职位低微,这样的日子还不能保证,但是这是我要尽量达到的目标么。至于武功,我是不会练的,没听说过善泳者溺于水么,我若是会武功,难免会介入到一些意想不到的事情中去,搞不好还会英年早逝呢,反正我只想活到七十岁就可以了。

这么坚持了两个月,果然我的身体情况大有好转,以前经常有的小病痛也不见了,而且觉得思路明晰,读书作文更加得下笔如有神了。

这一天,我从书库里面走出来,准备去吃一顿好午餐,唉,我还雇不起好的厨子,只好自己做了。正在我盘算今天中午吃什么的时候,我的同年刘魁,就是那个榜眼笑嘻嘻的走了过来说道:“江年兄,怎么样,咱们一起去明月楼吧?”

“明月楼,干什么?”我好奇地问道。

刘魁惊讶地说道:“怎么,你不知道么,去参加长乐公主的琴会啊!”

“琴会,长乐公主。”我更加糊涂了。

刘魁道:“是啊,建业上下谁不知道啊,长乐公主远嫁我国,不免思乡情切,为了排遣寂寞,所以举行这个琴会,听说是想见识一下我南楚的士子风范,还听说长乐公主陪嫁的女伴是大雍有名的琴仙子梁婉,梁婉的琴技据说传自乐圣无忧子,超凡脱俗,若非长乐公主是她的至交好友,才不会陪公主远嫁南楚呢。还听说,梁婉有意在南楚择婿,你说,凡是未婚的才子,谁不想去试一试。”

我瞠目结舌地道:“可是,梁婉不是陪嫁来得么?”

旁边有人答道:“那不过是个名份,听说公主早就和太子说过了,梁婉是她的好姐妹,一定要嫁个志同道合的才子做正室呢。”

我回头一看,原来是探花伏玉伦,看他已经换上了华美的便服,腰间系着一支玉箫,想必是有心求凰了。不过他出身淮扬世家,应该有这个身份吧。我在心里窃笑,如果那个梁婉真的如此出色,想必太子殿下一定会扼腕叹息吧,不过他总不能不给长乐公主面子,反正他将来登基之后,三宫六院可以随便选妃,现在么,还是谨慎一点,毕竟长乐公主身份不同么。

本来我是没有什么兴趣的,我是很有自知之明的,我的相貌还算是不错的,但也不过中上而已,我的才华也不错,但是有才华没有好的背景,飞黄腾达的机会并不多,这年头,兵荒马乱的,会领军作战的将领要比我们这些文人强多了,南楚是比较重视文人的,所以它的国力就不强,就连偏安蜀中的蜀国都不如,如果不是水军比较厉害,大雍早就渡江了,综上所述,我江哲并非一个值得争取的目标,又没有强悍的实力防身,别说梁婉不会看上我,就是看上了,我敢娶么。但是不去也不好,让人以为我太不给太子、长乐公主面子,所以我决定就去这一次,反正我对那些琴棋书画并非十分在行,琴可以听听,棋可以下一下,就是很难赢棋,书法么,还不错,但是绝对算不上名家手笔,画画么,我勉强可以应付,但是我更擅长鉴赏,我有个表舅,是有名的朝奉,手里流过的珠宝首饰、古玩字画那是不可胜数,当年我曾经跟着他好好学过,这些年又博览群书,相信这方面可以混口饭吃,如果不是爹爹带我离开,我还真想去当朝奉呢。

一边胡思乱想,一边漫无边际的随口应付他们,我们一行人就这样来到了明月楼,明月楼原本是一个大官的别院,恰好和几年前新建的太子府毗邻,所以后来太子索性把它买了下来,因为喜欢它的小巧精致,所以没有把它和太子府连通,据说长乐公主来了以后非常喜欢这里,就要来做了她的休闲之处,现在梁婉在这里举行琴会,真是再合适不过了。穿过黑油油的角门,我左右打量着这个小园子,一潭碧水,十几株红梅,加上临波照影的二层精美小楼,真是神仙境界,怪不得长乐公主喜欢。我一边走一边想,这么一座小楼,能够容纳多少人呢?等我绕过潭边,却看见在小楼前面有一片空地,原本想必是种着花木的,现在却被人清理了出来,用松枝搭了一座花棚,棚子上面覆着厚厚的苫草,四周放着一圈红红的火炉,上面闻着美酒,棚子中间放了几排铺着厚厚的毛皮的座椅,南楚的冬天本来就不是特别寒冷,今天又凑巧下了一场轻雪,棚子里面一片暖洋洋的,有十几个穿着各色轻裘的贵公子坐在里面,一边赏雪品梅,一边喝着醇酿,真是南面王不易的美好生活。走近之后,我听见他们议论,原来长乐公主的琴会岂是什么人都能参加的,所以除了年轻的新贵之外,只有世家子弟才敢来参加,而且还有自负有些才名,否则岂不是自己来找难看,所以来得人并不像我想像的那么多。虽然有些后悔可以不来的,但是一看这种招待,我还有什么不满意的,连忙跳了一个犄角旮旯坐下,然后倒了一大杯温热的御酿,准备偷得平生半日闲了。

没等多久,小楼的楼门打开了,出来了十二个秀丽高挑的宫妆丽人,她们放下了门前的珠帘,不一会,里面传来环佩叮咚的声音,然后,隐隐传来沁人心脾的香气,其中一个宫女躬身向内施了一礼,然后转过身来用清脆的声音说道:“公主殿下有令,梁小姐在楼内抚琴,不论诗词文章,还是琴棋书画,如果有人能够令梁小姐青睐,梁小姐便出来和众人一见。”

众人立时断然稳坐,侧耳屏气。不过片刻,从楼中传来了梁婉的琴声,琴声初时微弱,令人非得侧耳细听,渐渐的,琴声宛转盘旋,如同穿花蝴蝶一般迤逦而出,琴音反反复复,音韵连绵不绝,恍若高山流泉,清新流畅,令人顿时生出荡气回肠的感觉。听到这里,我悄悄打了个哈欠,真是无聊,我还以为大雍来得琴师会很高明呢,却原来也不过如此,这样的琴艺在南楚也并非没有么。正在这时,琴声越发宛转低回,令人觉得有些昏昏欲睡,突然,防若银瓶乍破,铁骑突出,急促的音调好像千军万马一般纵横驰骋,琴声就在爆发之后变得浑厚沉着,杀机隐伏,豪迈悲凉,好一幅沙场秋点兵的景象。我凝神细听,这才是值得浮一大白的好琴音啊。接着琴声渐渐恢复平静,宛如大战之后的歌舞升平,让人在心旷神怡中沉醉。

一曲终了,掌声雷鸣,然后就是众人纷纷拿出自己的得意之作,想让梁婉中意,出来一见,偏偏,那梁婉大概心气极高,始终不肯出见,后来有些没头脑的众人的目光落到我身上,一个贵公子半是央求,半是命令的对我说道:“久闻江状元才华横溢,一首《月下感怀》惊动天下,还请江兄作诗一首,也免得我南楚士子无颜啊。”我倒是无言了,这些家伙,好像我拿不出什么好诗来,就是丢了国体一般,罢了,这小子是丞相大人尚维钧的独子,我也不能得罪他,刚好听了这样的曲子,我心里也很痒痒,于是,我也不要笔墨纸砚,高声吟诵道:“昵昵儿女语,恩怨相尔汝。划然变轩昂,勇士赴敌场。浮云柳絮无根蒂,天地阔远任飞扬。喧啾百鸟群,忽见孤凤凰。跻攀分寸不可上,失势一落千丈强。嗟余有两耳,未省听丝篁。自闻梁师弹,起坐在一旁。推手遽止之,湿衣泪滂滂。婉乎尔诚能,无以冰炭置我肠。”场中静默片刻,喝彩声顿起,几个人连忙吩咐拿笔墨,要将我的诗默下来。这里正在纷乱的时候,只听见珠帘飞扬,从楼中走出一个身穿素黄罗衣,披着浅绿大氅的女郎,我定睛看去,这女郎大约二十岁左右的年纪,和南楚女子大不相同的就是她那修长匀称、凹凸有致的美好身材,虽然因为天寒,衣着颇多,加上大氅的掩盖看不真切,但是那种隐隐约约的美感令人心生渴望。我向她的面上望去,却见她虽然未施脂粉,却是肤光如雪,两行入鬓的黛眉,配合那双清澈如冰泉的明眸,当真是绝世佳人。

梁婉目光落到我身上,微微一笑,款款下拜道:“这位就是南楚才子,今科状元吧,妾身很喜欢你的诗文呢。”我虽然有点昏淘淘的,但是心里可明白的很,连忙道:“拙作能够得小姐赏识,是随云之幸,其实我南楚才子如云,只是江某胜在才思敏捷罢了,小姐若是有兴趣,不妨和大家详谈。”那梁婉的美目流转,向众人看去,这下众人如蒙大赦,连忙围上前来,我则是不多说话,渐渐的,见梁婉已经和众人谈得十分投机,便悄悄的慢慢的溜了出去。就在我即将走出角门的时候,我下意识的回过头去,却看见小楼后面的窗子半开着,一双晶莹剔透的眼睛正在看着我。我推门走了出去,那是谁呢?不知怎么,我总觉得可能是长乐公主。

后来我听说,长乐公主将明月楼赐给梁婉居住,梁婉性情明朗,若是有人前去拜见,只要有拿的出手的诗词歌赋,或者精通琴棋书画,常常能够得到接见,不少爱慕梁婉的少年都是想方设法的见她一面,虽然不少人有心于她,却碍于长乐公主不敢用强,再说梁婉名气越来越大,就更没有人敢得罪她。到了后来,就是连赵胜国主也收了梁婉为义女,虽然没有列入宗谱,但是大家都开始称她明月公主,声名远扬。

我这个小小的翰林学士可不会去找这个麻烦,虽然梁婉几次下帖子请我,我都用种种借口回绝了,有人问我,我就说,书中自有颜如玉,别人虽笑我迂腐,却也乐得少了一个强敌,不过为了不大过分,我热切万分的投入到翰林院的藏书中去,这样我既自得其乐,又免得别人侧目,这样产生了一个令我欣喜若狂的结果,显德十七年元月,我被特诏允许参与了崇文殿的筹立。我这个过目不忘的年轻人很快成了其中的主力,也难怪,我既精通鉴赏古玩字画,又博闻强记,在整理藏书和字画的过程中十分得力,我又年轻力壮,不用我用谁呢?这是我人生中最快乐的一段日子,崇文殿从正式奉诏筹立到建成,一共经历了三年时间,我一直在其中,乐此不疲。

当然,在我沉迷书海的时候,发生了一件我隐隐约约觉得可能会发生的事情,就是南楚和蜀国发生了冲突,而且越演越烈。当然,我是没什么机会参与的,也没什么兴趣知道,除此之外,若是还有什么事情比较特殊的话,就是长乐公主怀孕了,可是却不幸流产,据说是因为年轻再加上水土不服,在这之后,长乐公主一直身体不大好,所以到建业西郊的莫愁湖行宫居住,当然,太子殿下是不会寂寞的,长乐公主陪嫁的宫女都是大雍的美女,而且个个擅长内媚之术,她们早就成了太子殿下的宠姬了。说给我听的人都是满脸的羡慕太子的艳福,我却是微微苦笑,在我看来,长乐公主恐怕是不大喜欢太子的,否则怎么会移居行宫呢,也是啊,人家金枝玉叶的大雍公主,为了和亲嫁到南楚,怕是没有什么心思讨好庸庸碌碌的南楚太子吧。我恶意地想,大雍陪嫁那么多美女,是不是存心迷惑太子,免得公主委屈呢?

\chapter{第四章 品画明冤}

显德十八年己巳,三月,我已经二十二岁了,刚中状元的时候,有很多人上门说媒,都被我婉拒了,用的理由是年纪还轻,想多多读书,好为朝廷效力,后来,这种事情就少了。因为明眼人都看的出来,我这个年少的状元完全没有飞黄腾达的欲望,完全沉浸在书海之中,甚至有一点痴迷,这样一个人,并不符合那些世家大族的要求,因此我得到了难得的清净。

这一天,我按照惯例来到翰林院准备工作,却看见一大堆人围在正堂上,我不由惊奇起来,要知道,虽然我也被称为翰林学士,但翰林院里边还有高下之分呢,我因为是状元,所以越过了最低的庶吉士、检讨,直接当上了正七品的编修,在这之上还有编撰,侍讲、侍读、侍讲学士、侍读学士、掌院学士多个级别,可是我看到那一堆人里边,上有掌院学士谢贤,下有和我同科的一个二甲进士,一个庶吉士,这就让我惊奇了,要知道,那些侍讲学士以上的很多人都是经常在国主身边伴驾的人物,怎么会围在一起呢。我走了过去,却看见尹学士和田学士正在滔滔不绝的争论着什么,而在他们中间的桌子上,摆着一卷古画,旁边摆着一章红字条,上面写着“青山居士临江图”七个字,原来他们正在讨论这副画的真伪。我这才明白过来,自从国主下诏筹立崇文殿之后,却是有不少人将珍藏的书籍字画送来,希望能够得到收录,只是真正的旷世杰作还是不大好找的。

尹学士一派雍容的说道:“这副画一定是伪画,青山居士前期的作品都是青绿山水,风格绚丽,后期因为参修佛道,所以作品大多是水墨山水,画风变得恬淡秀丽,这副画虽然是水墨山水,但是你看笔锋嶙峋,画中云雾仿佛扑面而来,江流奔腾,似有耳闻,所以我说这不是青山居士的作品。”

田学士也不示弱道:“你说得虽然有理,可是你看,这副画的纸质是精选的帘纹纸,虽然保存的很好,仍然可以看出应该是两百年前青山居士时期的画作,你看这副画上有青山居士五方印章,从题跋上看绝对没有问题。”

其他人各自支持两方,争吵不休,我来了兴趣,仔细看了半天,从记忆中搜索了半天,才终于作出了决定。这时他们也看到我来了,因为我这些日子以来都表现出对字画鉴赏的熟识,又是新人,所以两位学士不约而同的向我往来,掌院学士咳嗽了一声道:“随云,你的看法如何。”

我走到这副画前面,仔细的看了一看,开口道:“首先从款识来看,这副画的上款是‘柯子远兄雅玩‘,下款是‘元佑后二年甲申七月初九敬制‘,下面是名章‘蓝氏宁泉‘,画的四角都有青山居士的印章,左上角是‘宁泉画印‘朱文方印,左下角为‘临渊堂章‘的白文方印,右上角是‘奎章阁侍讲蓝‘的白文方印,右下角是‘青山居士‘的朱文方印,这四种印章在青山居士画作上基本都出现过,印章的鉴别,田大人是其中翘楚,必然是不会看错的。从考证上来看,青山居士原本是大晋名士,位居正四品奎章阁侍讲学士,后来西晋南渡,青山居士伤心国事,隐居蜀中临渊堂,据说当时居士贫不能自给,幸亏蜀中富商柯明接济,才度过那几年的战乱岁月,你们看画的右下角有柯氏的两方印章,可见此画是青山居士赠送给柯明的。”

我喘了一口气,接着说道:“这些印章都是有来历的,而且我曾读过青山居士的《蜀中纪事》,在第九卷里有记载‘至秋分,子远设宴,宾主俱欢,临别,柯氏执手相求拙作,感其意诚,为作临江图‘,后来我查阅柯氏的记载,虽然柯氏已经湮没,但是我记得在东晋末年陶开所著的《蜀志·石崇篇》里面提到‘石崇少微,为柯氏执役,柯氏薄待之,后石崇富甲天下,勾连内宦,污柯氏谋反,九族诛绝‘,你们看这副画左下角还有石崇‘金谷园密藏‘的印章,而且石崇后来身死族灭,他的收藏基本上都被没入官,你们看,左侧中部有‘长陵王印‘,长陵王,东晋末年王室,受宠于晋元帝,抄没石崇的正是元帝,所以这副画在长陵王手中的可能性很大。由此可见,此画的传承十分分明,所以我认为是真品。”

大多数人听的连连点头,只有尹学士不服气地道:“这些就算你说得都对,那么画风又如何解释呢?”

我一笑,道:“这一点是我的个人之见,如果有谬误还请众位指正,青山居士在南渡之前的画风明朗激烈,所以喜欢画青绿山水,但是在南渡之前那一两年,他的画风已经渐渐变得恬淡,基本上都是小青绿山水,以水墨勾皴淡色打底并施青绿等敷盖,间或已经有水墨山水出现,在蜀中几年,青山居士几乎没有作品传世,直到东晋平定之后,才开始专著水墨山水,但是初期仍然喜欢用浓墨渲染,笔法挺拔,从这些来看,我想蜀中时期想必是居士转变画风的时期,这也符合罕有作品流传的情形,毕竟不成熟的作品,经常可能会被主人焚毁,我在《蜀中纪事》的第七卷曾经见过青山居士焚毁画作的记录。”

听到这里,大家已经认可我的判断,目光也变得尊敬热切,毕竟像我这么博闻强记的人并不多见。

这件事之后,我有了更多的工作,最重要的一件事情就是到大内书库里面去整理御札,原来在筹建崇文殿时候,有人建议我南楚立国六十年,历经开国武帝赵涉和当今国主赵胜两朝,在史书的记载上却不够完善,希望能趁这次机会整理武帝的朱批和御札整理成册,供皇室子弟和勋贵学习,我虽然觉得很没意思,但是翰林院上下都十分认可,奏请国主之后,国主龙颜大悦,但是整理那些御札朱批可是一件非常麻烦的事情,我虽然是新人,但是因为我的能力非凡,所以掌院学士谢贤决定由最资深的侍读学士夏悚来负责,而我协助夏悚,夏悚实际上已经年过花甲,很快就要致仕退休了,所以我是实际上的负责人,而夏学士在跟我跑了几天之后就自动请假回家休息了。这项工作最麻烦的地方就是必须到御书房后面的藏书库工作,那里收藏着所有的文书,而且我不能自己查阅,必须要有管理书库的管事陪同,所以,我就在离国主不到百丈的距离处开始了我的工作,这大概就是近在咫尺远在天边的诠释吧。

管事的太监姓王,已经须发皆白了,每天坐上六七个时辰简直是要他的命,所以我第一天就聪明乖巧地劝道:“王公公,我们一起怎么也要待上十天半月的,您也不要客气,只要找个伶俐的小公公来帮忙,您就隔三差五的来看看就行了。”王公公年纪也大了,担任的又是闲差,藏书库虽然离御书房很近,可是司礼监的那些公公们都是年富力强的宠宦,所以王公公根本搭不上国主的边,既然没什么本事争宠,他年纪又大,谁会无端的和他为难,所以,他跟本不用太担心有人告发他不尽责。所以他就派了一个新收不到一年的弟子小顺子给我帮忙,因为这个小顺子聪明能干,而且读过几年书,胸中有个几百篇文字,这在太监来说已经很难得了,毕竟不是所有人都像司礼监的太监那样要接收专门授业的。

不过我看到小顺子就是一愣,因为如果我没记错的话,那小子就是我刚到建业的时候遇见的卖身葬父的小子,怎么现在成了太监了,不过大概是有什么伤心的事情吧,我也不好问他,反正他也没有认出我,我就把他当成陌生人算了,不过这小子还真的不错,不仅打点文房四宝十分得力,而且我只要说要找那一份奏折或者御札,他都能用最快的速度找到,所以我们合作愉快,原定二十天的工作量,按照现在的速度,看来有个十二三天就能差不多了。

第三天中午,我正在喝着饭后的一杯清茶,准备休息一下好继续,突然王公公怒气冲冲的在两个小太监的服侍下闯了进来,嘴里喊着:“小顺子,小顺子,你这个小奴才在哪儿?”我疑惑的看向他,这是怎么回事啊。

王公公看见我,换上笑容道:“江状元,你也在啊?”

废话,我不在这里在哪里,这里可不允许我回家午睡的。我心里想着,嘴里说道:“公公,怎么了,什么事情让您生这么大火。”

王公公生气地道:“小顺子这小兔崽子手脚不干净,偷走了我心爱的鼻烟壶,那可是先帝赏给老奴的。”

小顺子睁大了眼睛,普通一声跪在地上道:“绝没有的事情,奴才可没有那么大的胆子,敢偷御赐的东西。”他已经净身一年多了,十四五岁的年纪又是发育的时候,所以声音尖细,这时他心情惊慌,更是多了几分刺耳。

旁边那个小太监尖着嗓子道:“还敢强嘴,你当我们不知道么,你本来就是犯了强盗罪的罪人,王管事的东西丢了,我就想一定是你干得,公公到你房里一搜,果然就找到了。”

小顺子的脸色发青,他连连磕头道:“不是奴才,不是奴才干得,定是有人栽赃。”

王公公怒道:“你是说我栽你的赃,还是小福子栽你的赃。”

小顺子冷汗直冒,顿然转身扑到我身边,哀求道:“江大人,您是有学问的人,求你跟公公分辨一下,奴才这些天都在大人身边侍奉,哪里有时间去偷东西。”

我本来正在兴致勃勃的看着这幕好戏,那个小福子虽然是一个好戏子,可是我却听见他的呼吸有些急促,心跳加速,早就看出他在栽赃,只是小顺子来历不好,背景不清白,所以没法分辩罢了。我是不打算介入后宫的事情的,所以只是淡淡的看了他一眼,没有作声。小顺子急得什么似的。王公公见我不出声,厉声道:“你们把他给我捆了,送到敬事房去,把他给我活活打死,我让他敢偷东西,这在宫里头是大罪。”

我心一抖,不会吧,要打死他。小顺子吓得抱住我双腿哭道:“求大人看在小顺子伺候周到的份上,给奴才求个情吧,奴才实在没有偷东西。”

我一下子想起当初他卖身葬父的时候那种悲苦的模样,不由心软了下来,反正也不是什么大事,他又确实是冤枉的。眼珠一转,计上心来,我淡淡道:“王公公,我看这奴才哭得厉害,或许真是冤枉呢?”

王公公有些犹豫,半晌道:“东西是从他房里搜出来的。”

我笑道:“这小子这几天都跟着我,公公的东西是什么时候丢的。”

王公公想了想道:“昨天晚上还用着呢,今天晌午就不见了。”

我故意皱皱眉头道:“这确实难以分辨,这样吧,下官颇精易经,最能断人祸福,明人冤屈,我就算上一课吧。”

王公公这些太监因为人生坎坷,最是信命,他眼睛一亮道:“大人会卜算,好,老奴这就去取算筹。”

我摇手道:“小小的一课,就不用算筹了。这样吧,既然是断冤屈,凡是冤枉的人,心气必然正直,我这里有个法子,让小顺子和这个告发的小福子各自吃一颗我特制的金丹,待我祷告上苍,如果无罪,那人就没有事,如果有罪就会腹痛。”说完我从怀里掏出一个玉瓶,倒出两颗金光灿灿的金丹,递给两个小太监。

王公公笑道:“好啊,就让老奴见识状元公的本事。你们两个还不吃下去。”

小顺子毫不犹豫的将金丹吞下,小福子犹豫了一下,将金丹送到嘴边,一个小巧的动作,金丹就滚动到袖子里了。好本事,我赞叹不已。然后装模作样的祷告上苍,不到一株香的时候,突然小顺子脸色发白,哎呀一声跪倒在地,双手抱着肚子,痛苦不已。而小福子浑然无事。他得意地道:“果然是你偷的,状元公的祝祷真灵验。”

王公公犹豫的看了我一下,正要下令,我微微一笑道:“我虽然有些才能,可没有本事请动神明惩罚你们,这种金丹是我特制的,专门用来疏通肠胃的,昨天我听王公公说年纪大了,常常积食,这种药若是老人就着莲子汤吃了,恰好得力,若是血气正盛的少年人直接吃了,就会腹痛如绞,小福子,你的药呢,藏在哪里。”小福子吓得连连后退,只见王公公一个箭步走到他面前,轻轻捏着他手腕一提,小福子立刻痛得脸色发白,王公公轻轻松松得从小福子的袖子里找到了那颗金丹。然后松开手,小福子跌倒在地,吓得魂不附体。王公公淡淡道:“小顺子,还不去我房里,桌子上有一碗凉着的莲子汤。”

小顺子点点头,一下子冲了出去,不到片刻就回来了,满脸的清爽,王公公笑得眯了眼睛,道:“多谢状元公想着老奴。”说着几乎是把我手里的药瓶抢了过去。一边说着一边告辞出去,没一会儿,两个中年太监过来把小福子带走了。小顺子感激地跪在我面前,千恩万谢道:“恩公两次相救,小顺子就是作牛作马,也不能报此大恩。”我瞪大了眼睛,半晌才道:“你还记得我?”小顺子赧然道:“其实奴才一眼就认出状元公了,当初大人慷慨解囊,小的记忆犹新。”

我好奇地问道:“那你怎么不早说记得我呢?”

小顺子犹豫了半天,才道:“奴才,奴才当初卖身葬父是假的。”

我这下更是瞪大了眼睛。小顺子道:“奴才原本也是个书香门第出身,只是父亲亡故之后,叔叔为了夺产,偷偷把我卖给我一个戏班子,奴才从此就四处流浪,因为奴才受不了班主凌辱,所以和几个兄弟逃了出来,无以为生,就四处乞讨偷盗骗人。那次遇见大人,奴才正和一个老乞丐合伙,他扮亲爹,我当孝子,大人慷慨解囊,可是我两个同伴利欲熏心,偷偷尾随大人……”

说到这里,他更加不好意思,我立刻明白当初打晕我的人是谁了。不过我又迷惑地问道:“你们有了那么多银子,足够生活了,你怎么,你怎么?”我有些说不出口。

小顺子笑道:“或许是报应到了,我们几个被人胁裹去做盗匪,不料被官兵捉住了,我们劫的是一个宗室,又都是做惯了贼的人,所以判了死刑,我们几个年纪还小,判案的老爷说如果愿意入宫为奴可以免了一死,我那两个兄弟硬气,硬是上了法场,奴才胆子小,所以入了宫。”

我叹道:“你不是胆子小,你是有勇气啊,人生虽然多苦,但是我们却是要苦苦求生的,你能活下来,还能把往事当作笑谈,这才是勇士,轻抛生死的人大多不是勇士,而是逃避责任。”

小顺子突然再次跪倒抱住我的双腿,疼得我怀疑他要恩将仇报,然后我就觉得有水滴湿透了我的官袍。

这之后这小子服侍我更是尽心尽力,后来我听说王公公是个武功高手,小顺子正在跟他学武,一时心血来潮,再加上佩服这小子的坚忍不拔,所以我偷渡了一册《葵花宝典》的抄本进来。小顺子看了默不作声,只是郑重其事的收了下来。

半个月后,我离开了皇宫,带着整理好的御札,和一个最大的收获,我多了一个经常会深更半夜来拜访我的朋友。

\chapter{第五章 储君之争}

显德十九年庚午三月,赵胜薨,谥楚灵王,太子赵嘉灵前继位,下令沿用显德年号,立大雍长乐公主为后,大雍遣使祝贺,赠良马千匹,金帛无数。

中宫既定,朝野上下,咸思储君,谏议大夫罗文肃公进言,议立王三子赵陇为储君。

先,国主立长乐公主为王妃,王妃未有所出,乃遣陪嫁宫女侍奉太子殿下,殿下爱雍女美艳,多有宠幸,先后生三子四女,后灵王忧虑,立丞相尚维钧之女为太子侧妃,十四月,生陇,嘉登基,封尚氏为贵妃。尚氏出身名门,贤淑少妒,朝野以“子以母贵”旧例,请立其子。

王后闻之,大怒道:“哀家虽无子,焉知其后必无,况纵使终究无出,哀家昔日陪嫁宫女,皆大雍名门之女,至今已生二子矣,若论贵贱,岂不如尚氏,若要立储,立王长子可也。”

--《南朝楚史·楚炀王传》

显德十九年,国主死了,若是平常人死就死了,可是一个国主死了就是大事了,在国主晏驾前,我们翰林园将已经基本完成的崇文殿书目《崇文密藏》递了上去,国主大喜,虽然没有看到崇文殿的建成,但是他应该还算是瞑目的。

丝毫没有争议的,太子赵嘉在灵前即位了,然后就是改元、大赦天下这些事情,我们翰林院也忙得不亦乐乎,还有一些很重要的事情,我们这些小官员虽然没有大多插嘴的余地,但是也很关心的,就是立后和立储的事情。立后,是没有异议的,虽然长乐公主常年住在行宫养病,算不上尽责,但是南楚名义上是臣服大雍的,而且长乐公主又是先王所立的太子妃,所以长乐公主仍然顺利地接掌中宫。但是立储就麻烦了,长乐公主没有生子,而她虽然才十九岁,但是常年卧病,大家都怀疑她是否还会怀孕生子,而且国无储君,必然不宁,所以大臣们都希望先立一个太子,赵嘉已经有四个儿子七个女儿了,因为长乐公主遣宫女伺候太子,所以大多子女都是雍女所生,但是这一点引起朝中显贵的不满,幸好先王在两年前将丞相大人的女儿尚芷兰指婚给太子做侧妃,虽然因为太子宠爱雍女,但是尚妃肚皮十分争气,生下了王三子赵陇。在朝中大臣看来,若是长乐公主所出,那自然是尊贵的,但是其他雍女的子女在他们看来都是血统不够纯正的,所以众口一词要求立赵陇为储君。

国主虽然贪花好色,但是也是一个聪明人,自然知道在这一点上大臣们是对的,所以虽然他不是很喜欢尚氏,仍然把她封为贵妃,立赵陇为储君,他也是赞同的。但是长乐公主因此大怒,和国主大吵了一架,独自返回行宫了,这下国主可就焦头烂额了,虽然他和长乐公主聚少离多,但是长乐公主十分贤惠,不仅让自己陪嫁的雍国美女侍奉自己,而且还常常支持自己广选美女充实后宫,所以他对长乐公主是十分尊敬甚至有点畏惧的。况且,尚氏是南楚贵女这个理由是只能君臣心照不宣的,所以赵嘉暂时停止了立储,并且暗示朝臣,除非说服王后,否则不能立储。

可是这一点可就难为死这些朝臣了,长乐公主自从下嫁南楚之后,经常深居行宫,南楚那些朝臣命妇就是想巴结也找不到门路,那些公主亲近的宫女现在基本上都是国主的宠姬,她们的儿子没有立储的资格,她们怨恨还来不及呢,哪里还会劝说公主呢,渐渐的,所有人的目光都落在了一个人的身上--梁婉。

梁婉既是长乐公主的闺中好友,又是先王的义女,在南楚虽然择婿未成,但是和南楚文武俊杰交情非浅,按理她是最好的说客,可是她却拒绝了。所以多日来已经渐渐平静的明月楼又成了车水马龙的所在。

我就在这种情况下再次来到了明月楼,本来我是不想来的,可是梁婉突然下帖子请我,我虽然对她没有企图,但是幻想一下也是难免的,更何况她的帖子我拒绝的话未免有点失礼。

我施施然的走进院门,绕过碧波,现在的明月楼前面已经种满了梨花,现在四月,正是梨花的花期,满园的梨花如云似雪,深深的呼吸了一口沁人心脾的幽香,我向引路的侍女问道:“姑娘,请问梁小姐召下官来有什么吩咐么?”那个侍女俏皮地道:“那就要问小姐了,我一个小丫头怎么会知道,大人这样恭敬,奴婢愧不敢当。”我庄重地道:“俗话说,丞相家人七品官,梁小姐是先王义女,又是王后好友,怕是权势胜过丞相,那样说来,姑娘怎么也有六品了,下官才是从七品,自然要恭敬的。”那个侍女愣了愣,噗哧一声笑了,低声道:“奴婢听说我家小姐跟丞相大人讲,如果想劝王后,必须得大人出面。”

这回轮到我愣住了,什么时候我一个小小的翰林编修,能够说动堂堂的大雍公主,南楚王后了。半信半疑的走进明月楼,一眼就看到丞相大人和翰林院掌院学士坐在上首,梁婉在旁边作陪。我差点想转身就跑。但是明知道那是不可能的,所以还是恭恭敬敬的行了个礼道:“下官拜见丞相大人,掌院大人。”

丞相尚维钧连连点头道:“好,好,听谢大人说你十分得力,近日就要升迁,果然是国之栋梁,梁小姐,人已经来了,小姐前次说只有江翰林可以说服王后,到底是什么缘由呢?”

我立刻看向梁婉,我和她往日无仇,近日无怨她为什么这样陷害我呢。梁婉在我们三人的目光注视下好整以暇的品了一口香茗,才开口道:“说句心里话,妾身原是大雍人,众位大人议立王子陇为储君,其中深意就是路人也都知道,王后又岂会不明白呢,如今负气离宫,正是最恼恨的时候,妾身受公主大恩,又得公主视若姐妹,若是劝她依从国主和众位大人,岂不令公主寒心,到了那时,就是公主有转圜的余地也不能答应了,所以梁婉是万万不能相劝的,但是妾身受先王青睐,也是感激涕零,怎忍见他泉下辗转,所以竭尽所能也要从中转圜,思量再三,想起公主自至南楚,雅爱诗词,每日手不释卷,曾对妾身言道,昔日名家,皆已身归黄土,不能一见,而今日大家唯有南楚状元江哲,读其诗荡气回肠,又同在南楚,每思一见其人,但恐君臣分际,男女有别,虽咫尺不能相见,足为平生之憾。妾身想,若是江状元能够觐见王后,以偿王后夙愿,然后栽请状元婉转陈词,王后必然心动。”

我差点晕过去,我难道很像白痴么,我一个小小的状元,在王后眼里恐怕只是弄臣一类的角色,我凭什么去影响王后,切切的看向丞相大人,希望他能阻止这种不切实际的想法。可是我的梦想破灭了,尚维钧那老东西居然满面沉思,而掌院大人居然连连点头。就这样,我连反对的机会都没有就被梁婉押上了马车,向行宫驶去。

在路上我郑重其事的问道:“梁小姐,下官曾经得罪过你么?”

梁婉含笑摇头道:“没有。”

我又道:“那么下官得罪过大雍么?”

梁婉眼中闪过一丝轻蔑道:“没有。”

我突然怒道:“既然如此我既非你的杀父仇人也不是负了你的薄情郎,你非要害死我做什么。”

梁婉一惊,然后又露出如花的笑容道:“状元公生气了。”

我已经恢复平静,冷冷道:“我办事不利是小,只怕会连累梁小姐呢。”哼,我就是死也要拉你垫背,我心里恶狠狠地想。

梁婉眉目流转,嫣然道:“状元公误会妾身了,妾身这个法子十拿九稳。”

我不在和她说话,因为觉得为了一件已经形成定局的事情争吵毫无意义,刚才的发怒不过是模仿平常人的心态罢了,反正就算达不成任务,也不能说我有亏职守,最多官升得慢些罢了。梁婉见我不说话,反而多了几分敬意,这令我心里警惕,虽然这几年没有见过她,她的事情我却是知道一些的,从她的行为来看,实际上是大雍间谍的可能性很大,否则怎么三年没找到如意郎君,我看她长袖善舞,在南楚朝野如鱼得水,绝不相信她就是一个普通的女子。说句不好听的话,在嘉兴我虽然只因为上当去过一次烟月楼,但是烟月楼当家的花魁云燕就是一个秀丽如仙,又精通琴棋书画的美女,石榴裙下从者如云,我看梁婉的行径,也就是一个高级的妓女戏子罢了,大概不同之处,就是她往来的都是高官才子,后台又硬,而且没有卖身罢了。

梁婉不知道我在腹诽她,仍然有一句没一句的和我闲聊。大约过了两个多时辰,马车终于来到了莫愁湖行宫,在经过禁卫的盘查之后,我顺利的进入了行宫,来到面对着莫愁湖的临波轩前,梁婉也不让人禀报,扯着我就往里走,两旁的宫女大概都知道梁婉不好惹,除了急匆匆的进去禀报,就这样放任我们进去了。

一走进房间,我就看见长乐公主身穿素色宫装,斜倚在锦榻上正在翻阅一本书籍,她笑盈盈地抬头道:“婉儿姐姐来了。”一眼看见我,立刻满面羞红地道:“什么人如此大胆,敢闯哀家的寝宫。”梁婉放开我,上前道:“公主,你看妾身带了你最想见的人来,怎么你还发火呢?”

长乐公主一愣,心中想起一个人来,惊叫道:“难道是江哲江随云么?”

梁婉回头道:“江哲,还不来拜见公主。”

我一进门就愣住了,当年见到长乐公主的时候,她正是大婚之时,身穿大雍公主的服饰,又是红色嫁衣,所以虽然年仅十六岁,仍然是雍容华贵,今日她穿的却是素衣,没有半点妆饰,也未施脂粉,却是清秀文雅,楚楚动人,与大婚之时颇不相同,更何况这两年她颇经风霜,更多了一种成熟的丰韵,我的心跳越来越强烈,不知怎么,突然生出一丝邪念来,若是能够抱一抱她该有多好。

正当我胡思乱想的时候,梁婉的话提醒了我,连忙上前拜倒道:“臣翰林院编修江哲叩见王后千岁千千岁。”

长乐公主突然露出忧喜交加的神色,半晌才道:“江大人平身,哀家平日最喜欢江大人的诗词,今日相见,想有所请益,不知可否。”

我平静地道:“敢不从命。”

长乐公主似乎看出我有些冷淡,幽幽的看了我一眼,道:“这是哀家平日抄诵的诗词,江大人可知哀家最喜欢哪一首。”说着将手中的册子递给梁婉。梁婉微微一福,将册子又给了我。

我接过来一看,果然是一本手抄的诗词,一行行簪花小字娟秀非常,我翻开第一页,却是一首《锦瑟》。

“锦瑟无端五十弦,一弦一柱思华年。庄生晓梦迷蝴蝶,望帝春心托杜鹃。沧海月明珠有泪,蓝田日暖玉生烟。此情可待成追忆,只是当时已惘然。”我低声吟着十五岁的时候先父亡故时我写的诗,那时候父亲已经奄奄一息,他对着母亲的画像,时而低语,时而轻笑,更多的时候是淡淡的悲伤,确实是淡淡的,因为父亲就要去见母亲了,那悲伤中甚至带着一丝喜悦,就是因为这个原因,我没有强迫父亲吃那些苦涩的药,既然父亲的生命已经无法挽救,我又何必让他带着无尽的痛苦苦熬呢,我记着那天晚上我跪在父亲床前信誓旦旦的保证可以照顾自己,父亲欣慰的看着我,然后就没有了呼吸,他的神情是那样恬静。不由自主的,我的泪水垂落,今天我才知道父亲的去世带给我多大的伤痛啊。

长乐公主见我落泪,有些不安,抬头看了看梁婉。梁婉会意,递给我一块绢帕。

我拭去眼泪,微笑道:“王后见笑了,这首诗是臣在先父去世的时候写得,先父生前和先母恩爱非常,先母去世之后,父亲始终忧愁难解,到了临终之时,先父心情非常平静,只是因为将要和母亲见面了。所以臣写了这首诗,想不到公主这里也有。”

长乐公主柔声道:“哀家及笈之时,有人从南楚来,带给哀家这首诗,只是当时哀家还不知道江哲是谁,后来到了南楚,听到状元的《月下感怀》,觉得非常喜欢,一问殿下,才知道就是江状元的大作,从此之后,哀家请婉儿姐姐替我收集状元的诗词,这几年哀家深宫幽居,就是读状元的词才能稍解愁怀。”

我下拜道:“臣的诗能够得到王后赏识,是臣的荣幸。”

长乐见我已经平静,便问道:“这首锦瑟,哀家十分喜欢,只是哀家不懂,什么是‘蓝田日暖玉生烟‘,难道蓝田美玉,在日光之下,果然会生出轻烟么?”

我含笑答道:“这句诗是有出处的,昔日晋代司空图曾经说‘载叔伦谓诗家之景,宛如蓝田日暖,良玉生烟,可望而不可置于眉睫之前者也‘。”

长乐公主恍然道:“原来如此,哀家明白了。不知状元近日有什么新诗么。”

我略一思索,道:“臣这些日子忙于公务,诗词上倒是很少有佳作,若是王后不嫌弃,请容臣录一首游戏之作吧。”

长乐公主大喜,立刻召来宫女磨墨,我用旁边书桌上的文房四宝写下诗题“春日迁柳庄听莺”,然后又写道:“春还天上雨烟和,无数长条着地拖。几日绿阴添嫩色,一时黄鸟占乔柯。飞来如得青云路,听去疑闻红雪歌。袅袅风前张翠幕,交交枝上度金梭。从朝啼暮声谁巧,自北垂南影孰多。几缕依稀迷汉苑,一声仿佛忆秦娥。但容韵逸持相听,不许粗豪走马过。娇滑如珠生舌底,柔长如线结眉窝。浓光快目真生受,雏语消魂若死何。顾影却疑声断续,闻声还认影婆娑。相将何以酬今日,倒尽尊前金笸箩。”

长乐公主走上前来,低声诵读,良久才道:“南楚才子果然非凡,哀家读来,口齿流芳。”

我见长乐公主似乎有些倦容,便告辞道:“娘娘凤体欠安,臣不敢久留,就此告辞,请娘娘珍重。”

长乐公主微微一笑道:“多谢你了,梁婉,代哀家送送江大人。”

梁婉应声过来,领着我出去了,走出很远,梁婉突然站住,冷冰冰地道:“江大人,你是不是忘了什么?”

我一愣,才想起我跟本忘记劝娘娘立储之事了,但我转念一想,淡淡道:“梁小姐何必这样说呢,我劝与不劝应该没有什么关系。”

梁婉怒道:“怎么,你们南楚大臣都认为我们公主好欺负么?”

我看穿梁婉眼中的惊疑,却没有掩饰地道:“梁小姐应该很清楚,立储之时已成定局,王后心里也应该明白,只是若是轻轻答应,不免有损大雍的声威罢了。”

梁婉面色一沉道:“你胡说什么。”

我心想,与其让她以为我愚笨可以利用,倒不如让她明白我的厉害,敬而远之,免得她再来害我。因此,我用一种飘渺的语气道:“大雍公主远嫁南楚,本非情愿,所以王后根本就不奢望国主的宠爱,雍帝陪嫁如此之多的美女,不就是为了迷惑国主,免得王后还要应付自己不喜欢的夫婿么。至于梁小姐你么,长袖善舞,正是可以统领大雍在南楚的密探的好人选,小姐身份微妙,可以毫无顾忌的任意行事,若是公主负责此事,难免有人察觉公主的行为可疑,我想对大雍来说,公主只需要嫁到南楚就是尽了职责吧。”

梁婉虽然极力镇静,但是面色苍白,而且眼中闪过一丝寒光。

我连忙道:“下官不过是个小小的翰林,这些国祚大事,无从过问,也懒得过问,倒是小姐费心将下官牵扯进来,真是不智之举,若是下官平白无故有了什么意外,难免让人怀疑小姐的用心呢。”

梁婉又是一愣,片刻神色恢复正常,嫣然道:“王后喜欢大人的诗,以后每隔一段时间,妾身会派人去大人那里取大人的新作,大人想必不会不答应吧。”

我坦然道:“下官家境贫寒,还没有自己的府邸,只是在翰林院附近租了一间民宅罢了,小姐若是派人去,倒是经常找不到下官的,如果小姐不嫌弃,下官必然定时将新的诗文送到明月楼,请小姐转承王后千岁。”

梁婉赞赏的看了我一眼道:“好了,妾身还要回去相劝王后,车马已经准备好了,他们会送大人到丞相府回禀差事的。”

我恭谨地道谢,然后上车,离开。

深夜时分,我终于回到城内,一路平安,到了丞相府,对着满心忧虑的尚维钧,我“实话实说”道:“下官觐见王后,娘娘果然十分喜欢下官的诗词,问了很多这方面的事情,下官口舌拙笨,不知如何劝谏,后来娘娘累了,下官只得告退。后来梁小姐对下官说,她知道下官说不出口,她让下官去的目的不过是开解娘娘的愁闷,娘娘深明大义,早已明白立储大事需得如此,只是一时气恼难以改口罢了,梁小姐趁娘娘高兴再去劝谏,必然能够让娘娘回心转意,只是梁小姐说,还得国主亲自去一趟接娘娘回来,娘娘才好下台。”

尚维钧满心欢喜道:“好,好,江翰林果然是栋梁之才,我和你们谢学士已经商议过了,你筹立崇文殿有功,近日必有封赏,好了,你回去休息吧。”

满身疲劳地回到家,看见一灯如豆,知道小顺子来了,懒洋洋的走进去,倒在床上,问道:“今天怎么有空来,我记得你还得过两天才有假呢。”

小顺子轻笑着走过来,把我拉起来,帮助我宽衣解带道:“本来今天是我当值,但是我偷听到尚丞相跟国主说你去行宫的事情,所以跟别人换了班,来回这一趟可真累,我看你到了丞相府又出来,认为没有什么危险了,所以先回来给你弄些水沐浴,等你洗完了,夜宵也该好了。”

我的眼睛半睁半闭地被他拽到厨房,里面已经有一个盛了七成水的浴桶,炉灶上热着宵夜。我低声问道:“你没跟我进行宫吧?”小顺子扶我进了浴桶,淡淡道:“我的功夫还不行,行宫和丞相府守卫都很森严。”

我打了一个哈欠道:“在我枕头底下有一本剑谱,我不知道管不管用,你去看一看。”

小顺子淡淡道:“我已经看过了,剑法不错,不过对我没什么用,那需要阳刚的内气,我的内气却是最阴柔不过的。”

我已经几乎要睡着了,迷迷糊糊地道:“我知道了,我再去找找,你的武功越高,我越安全啊。”

小顺子回了一句什么,我没有听清。

半月之后,王后回宫,国主举行立储大典,百官皆有封赏,我越过了编撰的级别,直接成了侍读,从五品。

\chapter{第六章 雍使齐王}

先,显德十六年丁卯,德亲王赵珏奉密旨至横江,欲偷袭秣陵,事未成而泄密于雍,遂罢兵,未几,雍遣使至,许以长乐公主和亲,灵王惑之,乃息兵罢战。

胜将终时,召太子至榻前,谕之曰:“孤平生遗恨,未能善守祖宗基业,称臣于雍,尔若有半分孝心,当竭尽所能,恢复帝业。”太子指天立誓,灵王乃薨。

显德十九年庚午五月,大雍齐王来吊,齐王密商国主,许以重利,谋拟攻蜀,国主惑之,后雍使上下勾连,遂起攻蜀之议,南楚国本皆坏于此,然其时人不解其祸,亦不解重利为何,后有内侍闻国主泣告王后曰:“孤若能恢复帝业,不图尔为皇后,孤亦愿父事大雍,今齐王以帝业许我,望卿代孤婉转告尔父,南楚绝不负雍。”事乃泄。

齐王者,雍高祖六子,长乐公主异母兄也,少顽劣,后见雍王弱冠封王,功勋冠盖天下,乃悟,曰:“我当取而代之。”后以武勋闻名于世。

--《南朝楚史·楚炀王传》

显德十九年五月,大雍遣使来吊,我听说正使是雍帝李援六子,齐王李显,自幼深受宠爱,所以顽劣非常,每日里只知弄鹰射猎,不喜读书。自从七十年前东晋崩溃,中原分崩离析,李援之父李商趁势而起,自称雍王,几十年血战沙场,立国称雍,李商死后,李援即位,喜好声色犬马,不思进取,他的改变是因为他的二子李贽。

雍王李贽,幼时就有贤名,二十多年前,李援九岁的次子李贽在新春朝宴上白衣素服,直言进谏,指责李援抱残守缺,有负祖父遗愿,慷慨陈辞,令李援惭愧而退,不久之后,李援称帝,改元武威,随后厉兵秣马,鼓励农耕,在武威三年宣告天下,临行前,沥血告祭天地,立誓不平中原誓不休兵。李贽当时十二岁,随父出征,李贽虽然是天家贵胄,难得的是和兵士同住同食,又跟将领学习领军作战,他年纪虽轻,胆气却十分豪勇,常常身先士卒,冲杀破阵,据说有一次敌军袭营,李贽带着亲兵护送着雍帝重出重围,有士兵在后面高喊:“殿下不要抛弃我们。”李贽挥泪如雨,居然单人独骑冲回军营,将士感激涕零,拼死作战,居然逼退敌军,等到雍帝回营之后,李贽身受重伤,仍然穿着甲胄迎接父皇,雍帝流泪道:“此吾家千里驹。”李贽作战勇敢,又富于智谋,在几年之间积军功升为将军,更在大雍武威九年大破当时中原境内最强悍的反对势力夏王杨老生,为大雍的巩固立下了汗马功劳。所以雍帝封其为雍王,雍王李贽班师之时,雍都长安万人空巷,百官亲迎,当时李贽年方弱冠,如此荣宠,亘古未有,至于武威十年,南楚显德九年,南楚称臣,大雍成为名副其实的中原霸主,那是后话,当时李显在人群中看见李贽如此荣耀,心中怅然若失,对侍从说道:“我当取而代之。”当时李显十六岁,之后李显一改劣习,苦读文章,勤习武艺,并在两年后自请到北方边境从军。之后十年,李显在边关参与了和北汉的数次血战,李显虽然不如李贽那般英明神武,但是也是一员悍勇的猛将,这几年,大雍紧守边关,北方没有战事,齐王李显才回到长安,他和太子李安走得很近,在长安,李显是勋贵少年中的老大,经常无事生非,每日不是呼朋唤友,走马章台,就是弄鹰射猎,弄得长安鸡犬不宁,但他是雍帝爱子,又有军功在身,所以没人敢和他为难。

我认真地看着手上的情报,自从我“劝谏”王后成功之后,我就以侍读的身份开始伴驾,说是伴驾,其实就是提供建议供国主参详,这次齐王作为使者出使南楚,朝廷上下人仰马翻,我们人手一份关于齐王的情报,看来南楚在大雍的情报网也是很广的。这次齐王名义上是来吊唁,但是谁都认为不会这么简单,否则大雍没必要派这样重要的人来。其实要我来看,搞不好是因为齐王在长安玩得太厉害了,雍帝让他出来避避风头,我看情报上写着,就在一个月之前,齐王强抢民女为妾,被御史弹劾,虽然雍帝袒护爱子,也不免要略作惩罚,我看最后的处罚是罚俸一年,明显的袒护么,在这个当口,齐王出使避避风头也是可能的。不过那些大人可不这么认为,都认为雍帝派齐王出使恐怕是有什么重要的事情。

不过目前的情况好像挺倾向他们的看法,齐王在吊唁之后,就要求私下会见国主,现在他们正在御书房密谈。我今天当值在御书房伴驾,所以就在外面候旨。可不是我故意的,但是我的听力太好了,我把他们的对话听了八九成。

齐王李显一进门就单刀直入地道:“大雍希望和南楚联手,共谋蜀国,国主意下如何?”

赵嘉愣了半天才道:“蜀国和南楚一向交好,怎能无故相犯。”

李显笑道:“国家好恶,要看利益如何,蜀中虽然与南楚交好,双方通商频繁,如今南楚所需要的兵器战马大多需要从蜀中购买,我听说蜀国为此向贵国索取高价,几年前,贵国从北汉购买战马,想从蜀中运回,可是被蜀国截留,如果不是贵国灵王令人到蜀国贿赂,恐怕这批战马不能到手,而且还被迫答应以后不直接从北汉买马,可有此事。”

里面没有声音,但我可以想像国主的脸色必然青紫,那件事情我也听说过,还奇怪为什么蜀国如此目光短浅,结怨南楚。

又听见李显说道:“我大雍和南楚既是君臣,又是姻亲,皇妹长乐是我父皇爱女,如今已是南楚王后,我们两国休戚相关,如今蜀国仗着地利,既不对我大雍称臣,对南楚友邦又如此傲慢,不过是仗着易守难攻,已经和三国通商的便利。如今大雍和南楚开放通商,按照我国户部的统计,这两年我们两国的通商税收已经超过了和蜀国的通商税收,在本王看来,如今蜀中不过是日暮西山,苟延残喘罢了,如果我们两国联手攻下蜀国,父皇愿意和国主平分蜀国疆土,从此划江而止,永息干戈。”

赵嘉的呼吸变得十分急促,半天才说道:“兴兵作战不可不慎,何况蜀国易守难攻,如果久攻不下,不免劳民伤财。”

李显似乎有些犹豫,半晌才道:“本王临行,父皇秘密对我说,如果攻下蜀国,大雍边疆稳固,他也可以好好休息一下了,若是国主肯助我大雍攻蜀,事成之后,父皇愿意默许国主恢复帝号。”

听到这里,我心里一阵哀嚎,近年来朝野多有恢复帝号的呼声,我还听小顺子说,先王临死的时候还再嘱咐国主一定要恢复帝业,这个诱惑真是太大了。

果然,国主犹豫地道:“此事孤也一时难以决定,这样吧,孤还要征询一下臣子的意见。”

李显不悦地道:“如此大事,国主小心是应当的,只是此事非同小可,还请国主小心守秘,至于我父皇所说之事,还请国主格外小心,如果不慎流传出去,我大雍可是不会认帐的。”

赵嘉不顾李显话语的蛮横,连连道:“殿下放心,孤必然小心谨慎,此事事关重大,孤绝不敢掉以轻心。”

李显满意地道:“那么多谢国主的接见,本王这就告辞了。”

赵嘉连忙道:“王后与齐王殿下兄妹多年未见,急欲相会,不知齐王殿下何时有暇?”

李显朗声笑道:“本王早想见见皇妹,只是职责在身,需得先公后私,这就去求见王后。”

赵嘉喜道:“何言求见,就请齐王殿下和孤一起去见王后吧。”说着,传来脚步声,这郎舅二人向门口走来。我早已经听得心灰意冷,看来国主是一定会攻打蜀国了。

我决定要好好看看这个飞扬跋扈的齐王,这个人将要把南楚绑上大雍的战车。跟在国主后面的李显走了出来,今年二十六岁的李显有着英挺俊美的容貌,因为长期生活在军中,他的身姿峻挺如松,身上更是透出千锤百炼的杀伐之气,今天是正式朝见,所以他穿着大雍皇子的服侍,金黄色的锦衣,上面绣着蟠龙,更显得威风霸气。我打了一个冷战,这个齐王必然是一个心狠手辣的人物。

齐王在走过我身边的时候,突然看了我一眼,眼中透出冰雪一般的寒光,我连忙微微低头,避过他的目光,虽然他那包含杀气的眼光我曾经见过,但是没必然让他注意到我不怕他是不是。不过他注意我干什么,难道梁婉已经跟他汇报过什么,不过大雍还真是厉害啊,一个齐王已经如此威风,不知道在他之上的雍王又是什么样的风采。

李显注意到那个年轻人只是一个很特别的原因,他天生有一种野兽般的直觉,刚才在书房和赵嘉密谈,不知怎么,他总有一种忐忑不安的感觉,仿佛被人窃听一般,可是他又明明知道方圆二十丈内没有人影,超过二十丈,他们的声音若能被人听见,那人的武功就太厉害了,他相信那样的人南楚并不存在。走出房门,他状似无意的打量外边的官员和内侍,却发现虽然有几个武功不错的人,但是都应该是南楚大内的高手,而且他们的位置都不可能听见房内的声音,几个品级不等的伴驾官员虽然离得近一些,但他们明显都不会武功。当他的眼光落到江哲的身上,虽然知道这人不会是窃听的人,但是李显还是有些震惊,这个青年官员年纪虽然不大,但是气度雍容,神情淡然,李显是知道自己的虎威的,曾经在大雍,有一个官员得罪了自己,自己盛怒之下正欲发作,那个官员居然吓得晕了过去,其他的文武百官见了自己,总是有些神情不安,就是太子殿下在自己面前也常常陪着小心,除了那个人,李显想,自从自己加冠之后,还是第一次看见有人如此从容不迫地道。想到这里,他的目光不由变得更加威慑,那个青年官员微微低头侧目,避过他的眼光,这原本该是认输的表现,但不知怎么,李显觉得此人并不惧怕自己。

想到这里,李显站住问道:“你叫什么名字?”

我正用余光察看李显的动静,听到他的问话,又看见他停在我面前的靴子,只好抬起头来,看了一眼国主,用目光请示。国主笑道:“这是我南楚的第一才子,显德十六年的状元江哲,王后最喜欢他的诗词呢。”李显露出恍然大悟的神情,道:“原来你就是江哲,你的诗确实写得不错。‘东南形胜,三吴都会,钱塘自古繁华。烟柳画桥,风帘翠幕,参差十万人家。云树绕堤沙。怒涛卷霜雪,天堑无涯。市列珠玑,户盈罗绮竞豪奢。重湖叠嶂清嘉。有三秋桂子,十里荷花。羌管弄晴,菱歌泛夜,嬉嬉钓叟莲娃。千骑拥高牙。乘醉听箫鼓,吟赏烟霞。异日图将好景,归去凤池夸。‘这首《望海潮》就是你写的吧,令人对江南美景顿生向往,本王这次出使南楚,也是想看看南楚的风光啊。”

我偷眼看了看满面与有荣焉的国主,谦虚地道:“拙作简陋,幸得殿下赏识。”李显深深的看了我一眼,招呼国主离去了,我却觉得背心发凉,因为那种目光,是一种莫名其妙的痴狂,如同烈火一般的热情,我顿时怀疑,这位齐王除了喜欢拈花惹草之外,是不是也有断袖之癖啊,打了一个冷战,决定以后离他越远越好。

谁知道天不从人愿,第二天,我接到了旨意,国主命我在齐王殿下在南楚期间,负责领齐王四处走走。天啊,苍天不仁啊,我仰天长啸之余,决定问问小顺子,这些日子他能不能多抽点时间保护我。可恨的是,小顺子凉凉地道:“我很忙,反正齐王长得也不错,你就陪他多走走吧,说不定齐王会带你回大雍享福呢。”我气得差点晕过去,当即下定决心,我要用尽一切手段保护自己,绝对不能让齐王的恶毒念头得逞。

当我到驿馆向齐王报道的时候,看见齐王穿着淡青色的袍子,在还有些冰凉的春风里敞着怀坐在院子里大笑,在他旁边坐着一个白衣如雪的绝美少年,情意绵绵的望着他。我差点转身就跑,转念一想,这个白衣少年这样的相貌人品,就是许多绝色女子也不过如此,我一个相貌平凡的小翰林应该没有问题吧,于是,我恭恭敬敬的上前问好,然后表示奉了国主的命令前来伺候。

齐王闪亮的眼睛看了我半天,才道:“好啊,我听说建业的美女多得很,秦淮河的名妓谁最出色。”

我皱着眉头想了半天,道:“臣也不大清楚,请殿下容臣回去查一查,一定会将其中翘楚弄个清楚。”

齐王眼中满是笑意,道:“算了,你这一查,还不得传遍建业,说本王寻花问柳,若给父皇知道,我恐怕又得挨一顿训斥,走,今晚你陪我去看看,一定要找个出色的烟花魁首。”我大喜,心想,你喜欢去找女人就最好了。温柔乡是英雄冢,我绝对不介意你玩得英年早逝。一定要去找出最好的青楼,我心里盘算着,一会儿偷偷问问驿馆的官员,他一定知道。

快到黄昏的时候,我早就找机会问明白了秦淮河的深浅,若非齐王坚持要微服出游不许别人跟随,我还想拜托驿馆的官员领我们去呢。不过那个白衣少年人是谁啊,齐王也没有介绍,只说他姓秦,我叫他秦公子就可以了,不过,我怎么看都觉得这个白衣少年像一把藏在剑鞘里面的宝剑,匣剑帷灯,可怕的很,那像小顺子,如今好像是蔫萝卜一样无精打采的,我都怀疑他是不是武功越来越退步了,但是应该不会啊,他现在好像越来越神出鬼没了,大前天我刚从宫里回来,就看见他在我家里等我,说他今天白天不当值,所以跑到离这里将近七八十里的无锡去玩,给我带了那里的特产鲜肉小笼馒头和鸭血粉丝汤给我当宵夜,我看着还温热的馒头和鸭血汤发楞,虽然有食盒保暖,但是也不能超过一个时辰啊。想到这里,我又生起气来,这小子,明明知道我有危险怎么不答应来保护我呢,下次我再下厨做菜的时候,绝对不给他留一份。

我已经知道了,建业青楼最出名的是风月楼、潇湘院、怡红阁,飘香画舫,风月楼出名的是床上功夫,潇湘院靠的是歌舞伎,怡红阁是有名的赌场酒楼青楼大杂烩,而飘香画舫据说是因为当家的是秦淮第一名妓柳飘香,齐王既然是风月场中的常客,那么当然要让他去见见柳飘香了,想必这种皇室贵胄,就是逛窑子也不会喜欢太庸俗的地方吧。结果,我一说去飘香画舫,齐王就兴冲冲地道:“好啊,本王正想见识一下建业第一名妓的风采呢?”我当时差点没气歪了鼻子,他绝对是戏弄我,要不然还让我去打听,虽然那个驿馆的官员已经知道是齐王要去,但是还是用暧昧的眼光看我,我可还是守身如玉的奇男子啊。

\chapter{第七章 飘香画舫}

我陪着齐王殿下走在大街上,齐王兴致勃勃的问我四周景物,我对这些虽然不是特别熟悉,还是基本可以说出来的,但是为了到秦淮河必须经过风月最盛的秦淮大街,两边灯火通明,所有的青楼酒肆都大门洞开,门前都站着把门的龟奴,很多门前还有艳妆的女子莺声燕语招揽客人,我们一行人个个相貌不错,尤其齐王身穿锦袍,气度不凡,正是青楼的恩客模样。所以不少龟奴妓女都想来纠缠,可是我发现十几个平常装束的汉子有意无意的围在我们周围,将那些人推开,隐隐的保护着我们三人,这十几个人相貌都还平常,可是个个体格魁梧,单薄的衣衫之下隐隐可见坟起的块状肌肉,走起路来尘土凝而不散。我心里知道这些人必定是齐王的侍卫,就是么,一个堂堂的亲王出游,怎么会没有侍卫保护呢,既然他已经有了护卫,我就不用担心安全问题了。这一放松,就连道路两旁令我尴尬的景象都不能让我紧张了。

没走多久,就走到了秦淮河边,在这截特别宽阔的河面上,泊了十多艘大小画舫,其中一艘最是庞大,灯火辉煌,却没有像其他的画舫那样传出丝竹琴韵、猜拳斗酒的声音。我们走到河边,那里都是一些小快艇,我对着一个窈窕的船娘喊道:“船家,送我们到飘香画舫上去吧。”那个船娘抬头笑道:“几位爷来得晚了,只怕飘香画舫今日已经客满了,爷不见那画舫上已经开始挂起红灯,那是客满了,很快就要起锚了。”

齐王愠怒的看向我,我却平静地道:“我们已经预定了位子,多谢船家提点。”齐王面色变得缓和。我们三人上了快艇,接着十几个暗中保护的侍卫也都各自上了快艇,快艇在河中左穿右插,一会就到了那画舫前。登上画舫之后,一个极具姿色,打扮的艳丽火辣的中年女子热情地迎了上来,未语先笑,打着招呼道:“哎呀,原来是状元郎啊,奴家听说状元郎订下了一个舱房,还以为是有人冒名呢,谁不知道江大人最不喜欢我们这些风月场所。”

我把眼光从她胸前那抹雪白移开,笑道:“艳娘说笑了,我一个小小的翰林,平日哪有金银来飘香画舫啊,今日是我陪着贵客来这里见见飘香姑娘,艳娘可要好好伺候。”那艳娘早就看到李显,她阅人无数,一看到李显就知道来了难得的豪客,立刻眉开眼笑,曲意逢迎,到齐王面前飘飘下拜,道:“贵客远来,艳娘迎接来迟,还请贵客见谅,这位--”她的眼光飘向我,我识相地道:“这位是李公子,这位是秦公子。”艳娘娇声道:“两位贵客快请进,今日飘香姑娘心情不错,几位若是有幸,还可得到飘香青睐呢。”

我们三人被艳娘引进了一间宽敞雅洁的舱房,至于其他的侍卫都被引到附近的舱房,自有侍女相陪。这间舱房精美雅致,里面灯火通明,临窗处放了一张大圆桌,其他大半空间都是空的,看来是歌舞悦宾的地方,在舱房右边有扇小门,门上挂着珠帘,里面隐隐约约是一间卧室,看来这真是上好的舱房。房门两侧站着八个相貌娇俏的侍女,上来替三人脱去披风外衣,三人在桌前坐下,都坐在靠窗子的方向,接着那些侍女如同穿花蝴蝶一般往来,不一刻就在桌子上摆上了茶点美酒,然后三位相貌最美丽的侍女坐在三人旁边,原本那艳娘安排三人两边都有两个空位,让众人都可以左拥右抱,那位相貌绝美的秦公子却拒绝了她的好意,径自坐在了齐王身边,艳娘见多识广自然不会表现出什么异态,但我却心里一抖,这个不会是真的吧,那个秦公子是个娈童,以前不过是怀疑而已,这次我真的浑身恶寒。那个秦公子似乎察觉到了我的异常,冷冷的看了我一眼,眼中满是杀气。等到他回过头去,我才松了口气,下定决心以后一定要弄几个高手在身边,小顺子毕竟不是自由之身,可是到那里找忠心耿耿的护卫呢,真有这样的人也不会来听从我这个小翰林的命令吧。

我们在侍女的陪伴下慢慢的喝酒,等着飘香姑娘的到来,那几个侍女似乎有些不安,也难怪他们,齐王确实是风流倜傥,时不时的手眼温存,那秦公子神色冰冷,丝毫不理身边的侍女,不时的用凶恶的目光盯着齐王身边的侍女。我又只是温文有礼地敬而远之,让她们手足无措,正在尴尬的时候,舱门被推开了,一个绝美女子款款走了进来,那女子秀丽如同山川的俏脸未施脂粉,晶莹白嫩的肌肤带着淡淡的红晕,仿佛刚刚出浴之后一般,她那如同流瀑似的黑发光可鉴人,那双眼睛黑白分明,如同黑夜里最明亮的星星一样灿烂,她身上穿着一件宽宽松松的长袍,她的身材在南楚女子中也算是纤秀婀娜的,若论容貌气质,这女子虽然美丽,却还常见,但是最难得的那一种媚骨天生的姿态。

这女子柔柔的走了进来,坐在了三人的对面,美目流转,说道:“三位贵客初次来见飘香,飘香却来得这样晚,真是让三位久等了。”那声音听来令人销魂蚀骨,我和秦公子都不由面上一红,就是齐王李显的面上也露出异样的神色。那女子眼光在齐王身上停了一停,微笑道:“飘香听说齐王殿下是难得的英雄,更是怜香惜玉的豪杰,怎么今日这样腼腆。”我并不奇怪那柳飘香会猜出李显的身份,却想看看李显的反应。

李显初时有些惊疑,但立刻开怀笑道:“噢,你这小女子到时聪明,难道见过本王么?”

柳飘香见李显并不掩饰,眼中闪过赞赏的神色,答道:“殿下虽然穿着南楚的服饰,却大概不喜欢丝履,足上穿的还是大雍贵人爱穿的锦靴,再说王爷的风度气魄,这段时间,奴家早就听说齐王殿下来到建业,殿下若是不来,倒要让飘香自怜呢。倒是这位江翰林,可是难得一见,若非陪着殿下,只怕飘香至今还没有机会见上一见。”

我有些赧然,我曾经接过柳飘香的帖子,邀请我到飘香画舫拜访,可是我囊空如洗,所以就婉辞了。秦公子原本有些恼怒的看着齐王,此时却微笑着看了看我,似乎对我拒绝柳飘香很开心。我连忙道:“柳姑娘说笑了,下官家无恒产,怎么有资格来这里。”

柳飘香站了起来,款款站了起来,坐在我身边,抱住我的手臂道:“真是的,难道状元公就当我们这些青楼女子没有一丝真情,飘香就不能喜欢状元的才华,以身相许么?”我差点笑了出来,柳飘香若不爱金钱,怎么会成为建业第一花魁呢,我可是知道,建业许多达官贵人都是柳飘香的入幕之宾,不过我倒听说这柳飘香确实是一个奇女子,没有千万家财自然是得不到她的,但是有了金钱权势却也未必能够得到柳飘香,国主的叔父,韩王赵德隆曾经来到飘香画舫,当夜就要留宿,谁知柳飘香却不喜欢他,不论韩德隆如何讨好也不肯留他,最后赵德隆以权势相迫,谁知柳飘香却是宁死不屈,赵德隆不便用强只得离去,后来屡次想为难柳飘香,都因为柳飘香恩客众多而作罢。后来有人问她,韩王虽然年过五旬,但是相貌精力都还过人,你怎么不肯屈从呢?柳飘香冷笑道:“奴家虽然是下贱女子,却还是懂得什么是忠孝仁义,那赵德隆当年领军作战,自己胆小怕事打了败仗,他的部下拼死作战,救了他的性命,他却恩将仇报,反而弹劾他的部将不听将令,贻误军机,判了斩刑,这件事南楚谁不知道,只是碍着他的身份权势不敢指责他罢了,这样的懦夫小人,就是奴家这种青楼女子也看他不起。”这番话传了出去,人人鄙视韩王,却对柳飘香另眼相看,没多久,韩王就郁闷而死,因为这件事柳飘香名动天下,这才成了建业第一名妓,其实未必没有人强过柳飘香,只是没有像她这般爽直侠义罢了。

当初我听了此事虽然也觉得佩服,若非没有金银做缠头,所以才不敢来见她,如果早知道她肯不收我的银子,我说不定早就来看她了。大概是见我神态迷醉,那秦公子鄙夷的看了我一眼,那冰冷的目光立刻让我清醒过来,想起我是陪齐王殿下来的。所以我轻轻抽出手臂,恭恭敬敬地道:“多谢飘香赏识。”

柳飘香嗔怒的看了我一眼,怒气冲冲的站起来,走到齐王殿下身边,那种轻颦浅怒的动人神色,令得我们三人都不由呆住了,接下来那柳飘香再也不看我一眼,只是和齐王殿下谈笑,还不时的和秦公子说话,她手段高明的很,既显得热情亲切,也不会显得过于放荡,就连冷冰冰的秦公子也带上了一丝微笑。

柳飘香当真是绝代尤物,喝了几杯酒,她站起来喊了一声,从外面走进来一个绿衣侍女,手上抱着各色乐器,就在乐声中舞了起来,仪态万方,那仿佛燃烧生命的热情舞蹈令我完全沉醉,而当我看到柳飘香俏脸上的神情,就知道她是将自己的生命也投入到舞蹈当中,这一刻,我真的对她动了心。当柳飘香停下来的时候,我看见她也看向我,四目相对,柳飘香突然露出十分欢欣的神色,然后就走到齐王身边,懒洋洋地坐在他身边,那慵懒的美姿令人想立刻将她抱向床榻。

这时艳娘走了进来,笑着说道:“夜深了,请江大人、秦公子到旁边的舱房休息吧,若有喜欢的侍女,不妨请她们相陪。”

我心里有些酸酸的,连忙站起身来告辞,并请齐王殿下好好安歇,那秦公子愣了愣,突然站起走了出去。我连忙也跟着出去了。

遣走了侍女,我在一间舒适却不大的舱房里面和衣躺在床上,心里胡思乱想,满是柳飘香的倩影,听着窗外潺潺的水声,我慢慢的陷入沉睡。正在我半睡半醒的时候,突然感觉到有人伸手在解我的衣服,我心里一凛,不是齐王来偷袭我吧,连忙睁开眼睛,正要叫喊,却看见一张如花似玉的俏脸,却是柳飘香,我身子一软,立刻喊不出了,柳飘香见我醒了,嫣然一笑,纤手轻动,片刻就脱去了自己的衣服,露出了秀美娇弱的玉体,我缓缓伸出手,抱住她,但是有些犹豫的,我呐呐道:“齐王。”柳飘香噗哧一笑道:“你不知道么,那个秦公子是个女子,我们还没宽衣,她就忍不住了,冲了进来,我将房间让给他们了。大状元,你还等什么。”

我虽然学过房中术,可是从没真的碰过女子,一时不知该如何动作,柳飘香会意,反手抱着我,我只觉得脑子里面轰隆一声,不知不觉中,衣衫褪去,感觉到那柔软温暖的女子身躯将自己缠住,我终于完全迷失了,完全投入到男欢女爱中去。

当我在疲惫中睡去之后,柳飘香闭眼休息了片刻,做了起来,拿起丢在地上的衣衫披在身上,不一会就招了两名侍女进来,那两名侍女轻手轻脚的替我洗澡换衣,我虽然曾经醒了一会儿,却是半个指头也懒得动,等我醒来,已经躺在干净的床铺上,穿着熏香的睡衣,我看看身边沉睡的柳飘香,脸一下子红了,讷讷的说不出话来。柳飘香睁开眼睛,轻笑道:“状元郎,怎么不高兴被我这青楼女子夺了童子身么?”我更是面红耳赤,半晌才道:“你嫁给我好不好?”柳飘香先是嘲讽的笑了,但看到我认真的神情,叹了一口气,道:“不成的。”

“怎么,需要很多银子么,需要多少,我会有办法的。”我焦急的问道。

柳飘香抿嘴笑道:“不是的,我早就赚够了银子,赎回了自由。”

我黯然道:“那么,你不肯嫁我,是不是我不够资格。”

柳飘香惊奇地问道:“你是翰林学士,我就是嫁你为妾也不免影响你的仕途,你真的要娶我为妻么?”

我淡淡道:“那么什么关系,大不了我辞官好了,反正我也不是很田当官,这几年我还是有点积蓄,买上几百亩天地还是可以的,只是,我怕你不喜欢这种清贫的生活。”

柳飘香露出无意言表的笑容,道:“我知道你是真心的,而且没有一丝犹豫,我阅人无数,原本早有从良的意思,可是当我赚够了银子,突然想到,我能够嫁给谁呢,那些自命风流的色鬼,只是那副嘴脸我就恶心,若是老实的好人又嫌他呆板无趣,虽然有几个令我倾心的人,可是只要想到嫁给他之后,日后年老色衰,被他弃如破履的情景就不禁心寒。唉,今日见到你,你是真的欣赏我的舞姿,我看得出来,你知道我在舞艺上投了多少心血,所以我自荐枕席,幸喜君子真诚待我,可是不行啊,飘香性子轻浮,不能相夫教子,我就像江南的燕子,喜欢繁华,喜欢自由,再也不能被笼子关起来了,江郎,日后飘香或者阅尽天下男子,可是江郎要记得飘香心中最爱的始终是你,你可不能嫌弃飘香,偶尔来看看我好不好。”

我心里一痛,我听得出来,柳飘香说的是真心话,没有丝毫欺瞒,这样奇特的女子,真的没有男人可以留住她。握着她的纤手,道:“飘香名动京华,江哲虽然有个小小官职,若是常来相聚,不免惹出是非,今日一别,虽非永别,也是难得再见,飘香,飘香,你我相忘于江湖,胜过相濡以沫,若是日后相逢,你不要视我为路人才好。”

柳飘香娇躯震动,她知道这青年的心意,他不会满足和她暗通款曲,若不能娶她为妻,日后就不会再来找她,但她已经满足了,在虚情假意的人生中,她终于得到了一份真情。

当我走出舱房的时候,看见心满意足的齐王,和满面羞红不敢见人的秦公子,恭恭敬敬地道:“殿下,我们早点回驿站休息吧。”齐王看看我,笑道:“怎么样,昨夜可春风得意么?”

我心里嘀咕,他知道我和飘香在一起么?我只是淡淡一笑,一副神秘莫测的样子,齐王狐疑的看了我一眼,看来他昨夜忙于采摘鲜花,他的那些侍卫应该也在风流吧。在踏上河岸的时候,我不由回头看去,那飘香画舫沉静非常,那里埋葬了我的初恋。

送齐王他们回去之后,我急匆匆的赶回住处,看见桌子上摆着一张字条,上面写着“昨夜风流快活,不知有人虎视在旁,齐王此人,其心莫测,监视之人,我已处置。

我的手一抖,小顺子真的是忠心耿耿,只是不知我何德何能,得他这般看待。

就在这时,驿馆之内,齐王面沉如水,阶下站着一个面色惭愧的侍卫。齐王冷冷道:”你说你没有监视江哲,为什么?“那个侍卫满面惊惶地道:”殿下赎罪,臣原本奉命,在对面的舱房监视江哲,可是不知怎么突然睡了过去。“齐王神色更加严峻,却没有怪罪,只是让他下去。

坐在他旁边的秦公子淡淡道:”我已经检查过了,他是被人点了穴道。能够在这种狭小的地方神不知鬼不觉的点了他的穴道,这人的武功至少在我之上。“齐王疑惑地道:”可是我看江哲并不会武功,难道是他已经到了反璞归真的境界。“秦公子微微皱眉,想了半天道:”当今世上到了那种境界的只有家师、少林寺的慈真长老,以及魔门的宗主京无极三人,这江哲年纪如此之轻,我绝不相信他能达到这种境界。“齐王若有所思地道:”二哥和梁婉都要我注意这个江哲,本来我还不以为然,可是前日一见,就觉得此人深不可测,昨夜之事更令我难解啊。南楚俊杰果然不凡,幸好,幸好,此人韬光养晦,似乎还不会成为我们的障碍。“

秦公子低头道:”若是你觉得他麻烦,我可以帮你的。“

李显摇头道:”这样的人物,怎可轻易杀了,再说,我们也未必成功。“说罢他的眼中闪过耀眼的光芒。

\chapter{第八章 明月舌战}

显德十九年七月,德亲王赵珏归,国主问其攻蜀之事,其时丞相尚维钧力主攻蜀,朝野上下均附和之,德亲王力阻之,国主犹疑,七月十五日,灵王义女梁于明月楼设宴,邀请德亲王赴宴,其余同席者,丞相尚维钧、大雍齐王李显、齐王幕僚秦铮,江哲亦受邀,后世览此,或为不解,江哲官微,不知为何得以入席,以闻社稷大事,或曰,其人其时已有二心,然考之实据,似乎未必。

宴后,德亲王愤然归,江哲赶上,与德亲王数语,亲王沉默,之后朝会公议攻蜀之事,王默然不语,攻蜀之议遂成。或有人言,亲王不阻攻蜀之议,追根揭底,皆江哲之过也,罪莫大焉,然从亲王僚属处得知江哲所言,实一心为楚矣。

--《南朝楚史·江随云传》

德亲王赵珏回来了,纷纷攘攘的攻蜀之议平息了很多,因为赵珏一回来就直接去拜祭先王,先王薨逝的时候,赵珏镇守前方边境,不能回来奔丧,如今朝中政局已经平定,赵珏乃是军方重臣,攻蜀之议必须听听他的意见,所以才特意把他诏回。赵珏哭祭之后进宫觐见国主,在国主驾前直言不讳,力阻攻蜀之事。赵珏在朝中威望极高,所以立时有很多人就不在说攻打蜀国的事情了,但是更多的人却纷纷上门相劝,尤其是尚维钧一方的朝臣名士,但德亲王始终不肯答应。

七月十五日,明月公主梁婉下帖子邀请德亲王赴宴,并且同时邀请了齐王李显和丞相尚维钧,谁都知道这是什么意思,其实他们这些手握国家权柄的权贵之间的事情跟我没有什么关系,可是为什么我也要参加。我哭笑不得的看着齐王,我刚说我不过是一个小官员,没有资格参加。齐王殿下居然脸不变色地道:“不过是梁小姐召宴,你是国主派来接待我的,自然得参加。”我虽有心拒绝,可是当齐王殿下身边的侍卫都用满含杀气的目光看着我的时候,我还是答应了,谁说威武不能屈的,你让他们试试在这些久经沙场的侍卫面前说个不字。

齐王殿下是第二个到达的,这次的宴会是在明月楼上,如今正是盛夏,酷暑难耐,这小楼上将所有的窗户都敞开,四处都放着盛着藏冰的桶子,楼里面阵阵清凉,梁婉穿着一件淡黄的衫子,坐在主位,尚维钧一身丝袍,坐在左首第二张椅子上,他的下首坐着一个黑衫儒士,乃是尚维钧的幕僚年垣,尚维钧看到齐王殿下来到,满面堆笑的上前迎接,看到我,眉头一皱。我连忙趁机道:“下官奉旨陪同齐王殿下,既然大人在此,请容下官告退。”尚维钧露出满意的笑容,对我的识趣很是嘉许。我自以为得计,正想下楼。齐王带着坏笑,一把抓住我的胳膊道:“别走啊,尚大人,江翰林既然是国主派来的官员,又是翰林院的侍读,又是你们南楚的才子英杰,不如让他在这里旁听。”尚维钧皱皱眉,终于不敢得罪齐王殿下,只是给了我一个警告的眼神,让我不可多言。

齐王坐在右首首位,秦公子坐在他下首,我只得坐在秦公子下首,总不能坐在左边,毕竟是齐王坚持我留下来的。等了没有多久,就听见门外传来朗朗的笑声,走进一个身穿王爷服色的俊伟男子,因为灵王薨逝不到一年,所以他的冠带上戴着孝,正是德亲王赵珏,他身后跟着一个青衣中年儒士和一个黑衣佩剑的武士。我一看到赵珏,差点没叫出来,这人竟是当年我高中之前给他算过命的灰衣人,如果他就是德亲王,那么当时一定是要到横江驻守,准备要偷袭秣陵,怪不得他当时要我算凶吉,我当时答他“内有纷争,外有强敌”,现在想来居然暗合局势。这德亲王是灵王幼弟,军机重臣,想不到我曾经给他算过命,不知道他还记得我么?

赵珏的目光在屋内众人身上一一掠过,在我身上并未停留,应该是对我没有什么记忆。只是似乎对于我的身份有些狐疑。

赵珏坐在左首首席,那名武士站在他身后,而他那名幕僚则坐在了左首末席,因为我故意和秦公子隔了一个位子,所以那人正好坐在我对面,四目相视,我讨好的一笑,那人却用锐利的眼光探询的看了我一阵。

赵珏坐下,有侍女送上茶点,然后都退了出去。梁婉站起身道:“妾身奉了齐王和尚相之托,邀请德亲王赴宴,虽然妾身是不该介入军国大事的,只是诸位大人毕竟需要有人伺候,妾身不得已留下,此事事关我大雍和南楚,妾身生于大雍,又受南楚先王之恩,所以绝对不敢泄露只语片言。”

赵珏淡淡笑道:“梁小姐是先王义女,也可以算是赵珏的侄女,赵珏自然是相信小姐的,却不知齐王殿下和尚丞相有什么见教。”

李显看看赵珏,笑道:“久闻德亲王是南楚第一名将,都督南楚大军,今日一见,果然是雅致高量,风姿不凡,李显虽是亲王之尊,然而在军中不过是个将军,若是论起职位来,李显尤在亲王之下,见教二字,愧不敢当,只是德亲王力阻攻蜀之议,与名将之称不甚相符,还请德亲王示下。”

赵珏淡淡道:“蜀国不肯臣服大雍,虽然有罪,但是蜀国国主曾是东晋遗臣,与大雍虽然曾经同朝为臣,但是却没有君臣之分,如今我不知道大雍凭什么以蜀国不肯臣服为由,攻打蜀国,就是大雍认为理由充分,我南楚虽然称臣大雍,可从来没有受大雍调遣的本分。”

李显笑道:“德亲王此言差矣,我大雍君臣贤明,那蜀王割据地方,不肯称臣,此诚不可忍耐,如果蜀国早向我国称臣,我大雍也不会进攻蜀国,我听说天子之仇,九代之后还可以报复,当初蜀国趁我们大雍立国之初,出兵秦川,烧杀掳掠,令我大雍先帝闻之泣血,此仇不报,焉能为人。后来我大雍攻打南楚,蜀国再次出兵,虽然于南楚有恩,可是我大雍却损失惨重,三秦之地,千里废墟,生灵涂炭,就是事后,蜀国不也向贵国勒索了无数金帛女子。这样看来,蜀国是一个藏在暗处的恶狼,平时蛰伏不出,若见人有隙,必然出来咬人。现在德亲王替蜀国说话,只怕有一天会被这种毫无情义,只知道利益的友邦吞噬。”

赵珏冷冷道:“珏虽不才,也知唇亡齿寒的典故,只怕亡蜀之后,就是轮到我南楚了。”

李显顿时语塞,他心里明白得很,攻打蜀国之后,南楚就是下一个目标,只是没想到赵珏不惧得罪大雍如此单刀直入,作为大雍皇子,他不愿信口雌黄的说谎。这时秦公子接过话头道:“此言差矣,所谓唇亡齿寒,是要相互依存,同舟共济,如今蜀国屡次挑衅南楚,视友好如仇雠,如今是牙利如刀,啮唇见血,我不知德亲王所谓唇亡齿寒可是指此。”

赵珏淡淡一笑,他的幕僚青衫中年人,放下手中摇摆的折扇,开口道:“虽然南楚和蜀国小有纠葛,但是并非是奇耻大辱,显德九年,大雍平定中原,陈兵长江,若非蜀主相助,出兵秦川,大雍怎能罢兵休战。虽然如此,我南楚仍然向大雍称臣,此实在是切齿之辱,虽然如今两国和好,长乐公主下嫁我国主,两国结为姻缘之好,然而贵国在长江之北年年操练水军,南伐之意未息,不知齐王殿下如何解释。”

李显笑道:“两国虽然和好,然而贵国如亲王这样念念不忘两国之仇的人并非少数,我国若不练习水军,只怕贵国大军早就过江了,德亲王久镇长江,难道不知此中情况,何况,我国既然早已和贵国结好,我皇妹乃是父皇爱女,远嫁南楚,近年来不仅往来频繁,而且通商通婚,哪里像蜀国一样闭关锁国,我国早就有军议,不攻蜀以免心腹之患,就平南楚以求清卧榻之侧。”

赵珏冷笑道:“岂有此理,十年来,我南楚每年入贡金银财帛,可是贵国却从不肯出售兵器良马,若是真心结好,怎会如此,王后虽然是大雍公主,然后国家大事,怎么能顾忌妇人,郑武公为攻打胡国,先以爱女下嫁之事,赵珏不敢忘记。”

秦公子怒道:“德亲王如此侮辱我国,是可忍,孰不可忍,但是仔细想来,亲王所虑,也不是没有道理,请听在下为亲王解释。我国禁绝武器战马的出售,并非针对贵国,我国北方边境不宁,边军战士日夕枕戈而眠,如何敢出售战马兵器,何况贵国久据江南,江南都是河流湖泊,贵国若不想攻打大雍,为什么要战马,难道是想攻打蜀国么。”

赵珏语塞,尚维钧连忙转圜道:“王爷和秦公子都有些失言了,今日我等聚议,并非是为了意气之争,还请二位不要记恨。”

赵珏和秦公子双双举起茶杯喝了一口,表示放弃争论。

秦公子喘了口气道:“我国谋蜀,固然是因为蜀国执拗,不肯称臣,虽然结盟,却又履背盟约,最可恨的是,我国盐区产量不足,其余部分需要从蜀中购买,蜀国屡次提高售价,蜀中特产丰富,蜀国据宝地而聚敛,此事实在不能容忍,如果我们两国攻下蜀国,愿意与贵国平分蜀中人口土地,你我两国隔江而治,到时候南楚军力大增,我大雍还有边患,南楚据长江全境,还有什么可以担心的呢?若是这样,德亲王都不放心,认为不能抵抗我大雍,倒不如趁早弃甲投降,难道南楚只想偏安江南,生死受人主宰么?”

赵珏默然,却只是摇头,他心知南楚兵卒战力不强,若是攻打蜀国,只怕大部分土地人口都会落到大雍手里,什么平分战果,到后来还不时谁打下来的就是谁的。众人面面相觑,都看出赵珏脸上坚决的神色,看来不论如何舌灿莲花也不能改变他的心意,李显眼中闪过苦恼的神色,看了梁婉一眼。梁婉站起身来道:“今日大家都累了,若不嫌弃,请诸位到楼下用餐,妾身准备了消暑的酸梅汤,请诸位品尝。”

尚维钧站起身来笑道:“梁小姐的宴席一定要参加的,请请。”

赵珏站起身来,看看秦公子,问道:“请问阁下尊姓大名,在大雍身居什么官职?”

秦公子裣衽道:“在下秦铮,齐王帐下效力。”

赵珏笑道:“秦公子舌如利剑,赵珏佩服,只是有些事情就是说得再好,也抵不过实力和利益,我南楚自认没有资格和大雍分庭抗礼,若是大雍进攻蜀国,我南楚理应厉兵秣马,以求自保。”

秦公子看赵珏如此固执,苦笑道:“德亲王择善固执,非言词所动,秦铮孟浪,还请王爷恕罪。”

赵珏微微点头,道:“本王军务繁忙,就先告辞了,还请诸位恕罪。”众人没想到赵珏如此绝决,原本打算在酒酣耳热之后再良言相劝的,此时只得无可奈何的相送。几人都不时的交换眼色,我心里一动,突然站起身道:“诸位大人都已经劳顿,就由下官相送王爷。”齐王等人都没有情绪理会,尚维钧苦涩地道:“也好,也好。”

我跟着赵珏走了出来,赵珏有些疲倦,我仔细的看着这个年仅三十的亲王,这些年来他的压力一定很大,三年不见,他的两鬓已经微霜,而他的身上流露出坚毅不拔的气势,这是我南楚的擎天柱啊,我又是敬仰,又是替他难过,苦心孤诣不能为人理解,真是不明白为什么会有这样的勇气呢。赵珏察觉到我的目光,淡淡问道:“你是谁?”

我恭敬地道:“下官江哲,翰林院侍读,现在在国主身边伴驾。”

赵珏吃了一惊,问道:“你就是江哲,为什么会跟齐王坐在一起?”

我连忙解释道:“下官奉命接待齐王,今日齐王定要下官在场。下官有幸得以聆听王爷教诲,三生之幸。”

赵珏虽然有些奇怪,却也没有深究,苦涩地道:“我听过你的诗,写的真好,‘醉里挑灯看剑,梦回吹角连营。八百里分麾下炙,五十弦翻塞外声。沙场秋点兵。马作的卢飞快,弓如霹雳弦惊。了却君王天下事,赢得生前身后名。可怜白发生。‘”他似乎沉醉在那首我在江夏写的《破阵子》的意境中,无意地抚摸了鬓角片刻,良久,他淡淡道:“你认为我们应该攻打蜀国么。”

我见四周没有外人,便道:“在下官表示意见之前,请容下官问上三个问题?”

赵珏惊异的看了我一眼,道:“你问吧。”

我眼中闪过一丝悲哀,问题道:“其一,请问王爷,我南楚上至国主,下至庶民,可有人和王爷一样明白大雍的狼子野心。”

赵珏沉默半晌道:“没有几人,就是我的亲信属下,也都劝我攻打蜀国。”

我又问道:“其二,请问王爷,若是大雍自己攻打蜀国,蜀国求我出兵相救,我南楚敢出兵么?”

赵珏惨然道:“不敢,我国君臣必然坐视蜀国灭亡。”

我知道他的心痛,可是还是问了第三问道:“其三,若是王爷力阻攻蜀,而国主意旨已坚,只得另选将领,不知道我南楚还有人比将军更能够领兵作战么?”

我连续这三问一问比一问犀利,听的赵珏冷汗直流,他定定的看着我。

我低头道:“如今,我国已经不能自主了,若是王爷执意不肯,国主派了他人进攻蜀国,我国兵士本就不如蜀国和大雍,如果在攻蜀之时消耗太多,到时候,大雍欲破我南楚,势如破竹,如果王爷亲自进兵,能够得到巴蜀部分要害作为根基,在得到陇右关中作为缓冲,再稳守襄樊,那么大雍迫于局势,至少可保南楚数十年国祚,日后我南楚若能卧薪尝胆,未必不可以得到天下。”

赵珏面上先是露出悲怆,然后又恢复平静,接着眼中透出坚毅的神色,道:“江大人真是无双国士,若是我领军攻蜀,江大人可愿做我的幕僚。”

开什么玩笑,我可不想上战场,所以我淡淡道:“下官不通军略,不敢相从,若是王爷有所征询,下官必然知无不言,言无不尽。”

赵珏愕然地看了我一眼,似乎不明白我为何推拒这样的青云之路,他沉声道:“国家兴亡,匹夫有责,江大人是我南楚臣属,焉能不为我南楚尽力,你好好考虑一下。”说罢,带着人离开了。

我恼怒的看着赵珏的背影,恩将仇报的家伙,我刚刚指点了你,你就这样报答我,想让我上战场,真是岂有此理,怎么办,找谁帮忙让我不用从军出征呢,我苦苦的思索着。

注:仇雠(音仇),意思是敌人。

\chapter{第九章 军机幕僚}

显德十九年八月,南楚与大雍结盟,齐王代雍帝与国主歃血为盟,德亲王赵珏拜为大都督,领命出征,临行前,赵珏命江哲担任军中幕僚,参赞军机,时,国主心忧德亲王权柄过大,命内宦王海监军。

--《南朝楚史·江随云传》

该死的赵珏,真的让我从军了,我本来想求人帮忙的,可是赵珏如今是一人之下万人之上的大都督,所以我只得含着眼泪交割了翰林院的工作,从军征蜀,不过可以令我感到安慰的是,小顺子居然也随军出发,临行前,国主派了司礼监管事王海作为监军,虽然用宦官监军实在是败亡的内患,可是想到小顺子居然跟着王海一起来了,我就不由谢谢老天保佑,有了小顺子的保护,我应该不会遇上太多的危险,不过最好还是多找几个护卫,我准备和小顺子谈谈,等我看中人选,小顺子要帮我鉴定一下他们的武功,免得我找了一群酒囊饭袋。

这次攻打蜀国,南楚兵分两路,一路水路,由镇远侯陆心率领一万水军,出白帝巫峡,溯江而上,另外一路由大都督德亲王赵珏率领五万军队,从陆路杀奔巴州,双方约定会师雒城。我是德亲王帐下的幕僚,自然得跟着大军行止,不过我怨气难消,行军途中一直躲在监军王海的车驾上,王海和御书房藏书库的王管事是同族,所以对我还不错,路上还不时提起自从王管事服了我送的药身体大有好转。我自然识趣的答应替他配制一两种类似的药物。小顺子在旁边乖巧的伺候着我们两人,王海可心的看着小顺子,笑道:“这小子就是状元公曾经救过的奴才吧,小顺子什么都好,手脚勤快,口舌伶俐,识文断字,就是一点不好,一点也不上进,别的奴才为了一个差事能争得头破血流,恨不得围在国主身边,只有这小子,倒愿意抛弃那份好差事,跟着咱家到军中受苦。”

我不由看了小顺子一眼,有些愧疚,这小子都是为了我着想,小顺子乖巧地道:“公公说哪里话,公公和王老公公是亲戚,平日见了奴才总有打赏,这次公公得国主赏识担任监军,一旦得胜回朝,就是天大的功劳,奴才跟着公公也就沾了光,要不人家怎么都说富贵险中求呢?”王海笑得眼睛都睁不开了。我们三人正谈得开心。这时一个传令兵跑到我们车驾前,大声道:“江大人,王爷召您前去议事。”我无可奈何的下了车,从王海带来的大内侍卫手中接过马缰,晃晃当当的向前面驰去,我的骑术不是太好,还是这几天临时抱佛脚学的。好不容易来到停马等我的赵珏身边,我在马上抱拳行礼道:“王爷,下官奉命前来。”

赵珏看看我的狼狈模样,笑道:“江大人,你还是多学学骑马吧,否则很难随军的。”

我差点咬牙切齿,难道是我愿意随军的么。但是人在屋檐下,不得不低头,我只得恭恭敬敬地道:“下官遵命。有什么事情要下官去办,请王爷吩咐。”

赵珏催马慢慢前行,示意我跟上他,我手忙脚乱的催动坐骑。我们两人并肩而行了片刻,赵珏才道:“江大人还在怨恨本王么?”我皮笑肉不笑地道:“下官不敢,下官吃的是南楚的俸禄,怎么敢推辞朝廷的任命。”赵珏苦笑道:“不是本王为难大人,只是这次攻打蜀国,我们必须在取得最大利益的同时保存自己的实力。行军打仗是本王份内事,不会也不敢劳烦大人,只是平蜀之后,我们必然要和大雍商谈如何分配战果,到时如果没有江大人这样明白我们两国虚实而且明智果决的人士,只怕我们会吃大亏,所以只得为难江大人了。”

我忿忿不平的想:“不过是强盗成功之后的分赃罢了,不会等到你们打胜了在让我去么?”似乎看透了我的心思,赵珏道:“而且,我看先生如此才智,珏也想每日聆听教益,如今国家危难,还希望江大人多花些心思在军务上,好为国家出力。”听了赵珏的话,我仔细想来也有道理,既然我已经在军中不如趁此多了解一些军务吧,想到这里我低低欠身,表示接受他的意见。赵珏微微一笑,给马加上了一鞭,我的坐骑似乎有些也想奔驰,不耐烦的扭动着身躯,吓得我左右摇晃,幸好一个跟在一旁的赵珏的亲卫扶了我一把,我面红耳赤的道谢,发誓要好好学习骑马。

放下手中的笔,我揉揉肩膀,安营之后我就在处理这些军务,自从跟赵珏谈过之后,我就开始参与处理军务,从开始的磕磕绊绊到现在的游刃有余,我花的时间并不太长,从如何安营扎寨,如何编制军队,如何赏罚惩处,当然最主要的是文书处理和情报整理,这些军务的难度并不比我在翰林院的工作轻松。赵珏的幕僚当中以一直跟随他的黑山儒士容渊最受重用,经常跟在赵珏身边参赞,至于这些琐碎的军务则是其他的参赞处理的,我的加入减轻了他们的工作量,尤其是我没有多长时间就熟悉了其中的大部分文书处理方式,靠着我强大的记忆力和敏锐的判断力,很快就成了其中翘楚,尤其是情报分析工作,原本他们只是让我试试,不料从只言片语中考据查证本来就是我的强项,不需要笔墨记录,不论多么琐碎的情报,只要我看过一遍,就能够理清楚中间的脉络,所以后来那些幕僚索性将情报分析工作交给了我,由我整理出文书交赵珏批阅。直到这时,我才真正成了赵珏身边备受重用的参赞,除了容涣之外,我已经独占鳌头。

看看天色,已经深夜了,明天还要赶路呢,我将整理好的情报收集起来,准备送到容先生那里,觉得有些口渴,随手拿起小桌上的茶壶,却已经空了,我苦笑着摇摇头,这时,帐外轻轻传来一声咳嗽,然后小顺子走了进来,拿着一个食盒,淡淡道:“江大人,王监军知道你军务繁忙,托我送来夜宵,还要我谢谢你昨天给他的药。”

我一听小顺子的口气,就知道外面有人,于是笑道:“请替我多谢王监军,其实监军大人只是养尊处优惯了,这些日子过于疲惫休息不好,所以不免身子不爽,我的药物不过是让监军大人休息的好一些,快些恢复精力罢了。”小顺子将东西放在桌子上,道:“请大人趁热吃吧。”我摇摇头道:“我先将文书送过去,你先回去休息吧,明天还要行军呢。”小顺子将一张小字条塞到我手里,然后行礼退下。

我打开字条,上面写着一行娟秀的小字“军中来往不便,赵珏身边高手众多,容渊似乎对大人有些嫉妒,今天对赵珏进谗言,说大人与齐王来往密切,恐怕有所勾结,为了稳妥起见,尽量不要让大人接近要紧军务,赵珏半信半疑。”

我淡淡一笑,这样的事情总是难免的,我这样异军突起,也难怪容渊忌惮,不过若他进谗言成功,也没什么关系,反正自己也没有非要得到德亲王重用的理由。我走出帐篷,让帐下听命的军士陪着到了容渊处理军务的帐篷,将文书交给他,他收下,鼓励了我几句,满是信任赏识我的模样,真是人不可貌相啊。-我心里感叹着离开了帐篷。帐外此时月华如霜。

经过大半个月的行军,我们到了蜀国边境,之后攻城作战十分顺利,不过旬日就到了巴郡,我开始还奇怪为什么蜀国抵抗力为何这样软弱,后来问了人才知道蜀国毕竟兵员不足,所以除了要害险关之外其他地方并不布置重军,而巴郡,就是我们面临的第一道关卡,过了巴郡,前面都是艰险路径,连续二十多处关隘,都是易守难攻的格局,大战,就要开始了。

八月二十三日,南楚军到达了巴郡城下,我骑着德亲王特地为我选的温顺马匹,看着城高池深的巴郡城,城楼上刀枪如林,无数蜀军站在城上神情肃穆,一见可知是一支劲旅。德亲王微微带马,站在大军之前,冷冷的望着城墙。在城上众多军士之中站着一个身穿红色铁甲的将军,凭着我的目力可以看出这人大概五十多岁的年纪,相貌豪迈,身材矮小,虬髯满面。这人大声喝道:“南楚与我蜀国乃是盟好,为何无故撕毁盟约,前来偷袭。”

德亲王淡淡一笑,扬声道:“蜀国偏安一隅,割据天下,今日大雍龙兴中原,蜀国至今不肯称臣,是何心也,我南楚本大雍臣属,奉命来攻,一则尊奉帝命,二则雪洗多年来蜀国欺压之恨,我南楚子弟听了,蜀国仗着地势,常欺凌我边民,又趁通商之际擅抬物价,搜刮我百姓金银,今日我南楚兴兵,必要一战功成,报仇雪恨。”说罢,长鞭前指,南楚军齐声大喝,军鼓雷鸣,一个千人队开始呼喝前进,人人手持盾牌和环首刀,保护着着多驾云梯向城墙冲去,趁着城墙上箭手不能伸出头来向下射箭,南楚军将那些云梯靠在城墙上,开始向上攀登,另有二三十人推着冲车来到了城门下,巨大的撞击声压过了战鼓和号角的声音。还没有撞上几下,城上战鼓响起,滚木落石如雨而下,那些云梯也被拒杆推倒,南楚军士的身体从半空中坠落,血肉模糊,那冲车也被巨石砸得七零八落。

我看得心里忐忑,却看见德亲王和其他的将军幕僚都用淡然的神色看着战场,丝毫没有紧张的神情。接着鸣金声响,那些军士渐渐退回,我仔细看去,大多数军士还没有向上攀登,所以受伤的人并不是我想像的那么多,过了片刻,南楚军队第二波攻城开始了,城上也开始还击。

这一天,南楚军队一共进攻了二十多次,都是浅尝辄止,而城上的守兵也十分谨慎,并不滥用木石。到了将近黄昏的时候,南楚军队发起了猛攻,攻势如火如荼,军士们舍生忘死的向上攀登,竟然登上了城墙,在城上展开了血战,不过最后南楚军队仍然败退了下来。

我看着心神动摇,今天攻城应该死伤了两三千人,损失不是很大,但是那种可怕的气势令我久久不能平静。当天晚上我在营帐里辗转反侧,攻城损失如此大,听说下面还有那么多城池,每个城池都这样岂不是太凄惨了么。

第二天,攻城之战十分惨烈,太阳刚刚升起,军士们推着十几架投石车轰隆隆的走了出来,一声令下,一块块巨大的巨石腾空而起,重重的砸在城墙上,虽然因为巴郡城高池深,城墙没有动摇,但是城楼上碎石飞溅,城墙在呼啸声中颤抖,我的眼睛收缩了,看到了在巨石的砸击下的血肉横飞,接着那些城内守军冒着矢石也开始向下透石,城上投石机威势猛烈,砸向我军的战场,虽然因为难以瞄准的缘故,只击碎了半数我们的投石机,但是将我们前沿的阵地砸得七零八落,血肉模糊,尸骨成堆,投石之战持续了两拄香的时候,这短短时间我就手足冰凉,满眼里都是鲜血肉泥,我的眼力太好了,甚至看见那些军士死前惨淡凄厉的神情。接着大概是石块不足,双方的攻势都缓了下来,渐渐停止,南楚军推着箭塔,扛着云梯再次攻城,箭塔的高度虽然不及城墙,但是已经勉强可以抵挡城中的反击,双方锋利的翎箭在空中划过美丽的弧线,穿过健壮的肉体,飞溅出耀眼的血花,双方的鲜血就这样在城墙前面挥洒。当南楚军顶着箭雨再次向上冲锋,这次城墙上砸落的是滚烫的油和石灰,当焦头烂额的南楚军士坠落的时候,城墙上又丢下无数稻草和火把,城下顿时成了一片火海,只有少数身手敏捷的军士逃了回来,其余的军士都被火海包围,烧得惨不忍睹,火海中凄惨的叫声惊天动地。

我看到这里,真恨自己的六识如此灵敏,再也忍耐不住,连忙策马冲向后面,找了一个僻静的地方吐得淅沥哗啦,直到吐出了苦胆汁才停下来。等我直起腰来,看到穿着军士甲胄的小顺子站在我马前,他递给我一壶清水,让我漱口,等我心绪平静下来,我问道:“你怎么来了,你不是陪着王公公么?”小顺子低声道:“我跟王公公说不知道战场上情况如何,所以出来看看,王公公也担心得很,所以就同意了。”望望远处的战场,我心有余悸地道:“太可怕了,我还是回去吧。”正想策马,小顺子一下扯住我的马缰道:“大人,不可以,我虽然无知,也知道如果大人此时胆怯,以后在军中将领面前就再也抬不起头来了,而且大人今后还要上战场,难道次次躲避么。”

我听得有些羞愧,心想看来自己心志远远不如小顺子坚韧,感激地看了他一眼,策马赶回前线。等我再次回到赵珏身边,他身边的将领和幕僚都用赞许的眼光看着面色苍白如纸的我。赵珏嘉许道:“随云胆量果然不错,当初本王初上战场的时候,比你还要不堪,放心吧,多打上几仗就好了。”我在马上躬身行礼,问道:“王爷,下官不通军事,好像我们攻城不大顺利是么?”

赵珏苦笑道:“是啊,巴郡是蜀国重镇,不仅将领善战,而且军士骁勇,守城器械和粮草又充足,所以十分难攻,令本王心痛不已,幸好,若是攻下巴郡,下面的二十多个城池就容易多了。”

我又问道:“那么,依王爷所见,我们需要攻打几天。”

赵珏盘算了一下道:“我们若能在半月之内攻下巴郡,就不错了。”

我一盘算,大雍从阳平关经东川攻击葭萌关,也要经过几道险关,可是大雍兵精粮足,我们南楚若想抢先,就必须使用计策,我在脑海里面回想着曾经看过的战例,怎样才能解决当前的僵局呢?

一时间想不出来,我又回想着关于巴郡的情报,一条条的回忆,远眺城墙,那红甲将领正在城上指挥,只见他指挥若定,将巴郡防守地滴水不漏,我南楚稍有破绽,就被他一眼看穿,然后紧追猛打,毫不手软。

慢着,紧追猛打,毫不手软,我又想起关于守城将领情报:田维,制军严谨,英勇善战,善于守城,防守如山,尤其善于截寨。怪不得德亲王把营帐守得如此严密,原来此人善于截寨。慢慢的,一个诡计成形了,可是行得通么?想来想去,我策马到德亲王身侧,低声对他说了自己的看法。德亲王先是犹豫,渐渐的感起兴趣来,良久,微笑点头道,随即下令收兵。巴郡之战最血腥的一天终于落幕了。

\chapter{第十章 千里征程}

我对德亲王说的话很简单,“王爷,这附近崇山峻岭,未必没有小道可以绕行,就是不能绕行,我们也可以作出可以绕行的假相,引诱他们出战,我们怕的不是他们勇敢善战,怕的是他们死守不出,强行攻城,不如想想办法引诱他们出城,而且田维既然善战,肯定不甘心只是守城。”我说的只是一个原则,不过德亲王久经沙场,立刻心领神会,再说今天肯定是攻不下的,不如回去商量一下。

当然在之后的军议中,我没有发言,因为我对军务又不是很熟,我只是善于分析情报,并根据经验学识判断那里可以着手罢了,更何况现在容渊已经对我不满,我若太出风头必然会让他对我更加嫉恨,宁可得罪君子,不可得罪小人,这一点我可是记得很清楚,不过这些幕僚真厉害,我不过提出一种设想,他们就能够列出种种设想,然后查疑补漏,定出甲乙丙丁各种方案,最后列出可行的计策,我越看越是崇拜,可能我的表情太明显,他们都有些不好意思了,即使是容渊看我的眼神也柔和了许多。

第二天,德亲王派出军士四处打柴,寻找小路,然后命令剩下的半数军士在营帐中休息,其余的军士则站在远远的看着巴郡城,既不进攻,也不后退,只是不时派人佯攻,城楼上的守军若稍有反应,就退下来。过了中午,休息的军士和上午的军士换班。

第三天,南楚军在巴郡城前佯攻的军士开始忙起来,不是挖挖壕沟,就是练练拳脚,疏活筋骨,并且推了军中的战鼓到城前,每隔半个时辰就敲鼓呐喊。

第四天、第五天,城上的守军开始疲惫麻木了,毕竟南郡城中只有一万守军,南郡虽然是蜀国门户重镇,但是因为蜀国和南楚交好,所以军员并不足备。

第六天,城中蜀军开始不安了,而好消息传来,我们找到了一条小路可以绕过巴郡,这下第二步计划开始了。南楚军开始收拢军队,厉兵秣马,好像要进攻的样子,城中蜀军开始紧张,明显可以看到呈上守军增多,到了晚上,军队开始悄悄行军,这些被密切注意南楚军队的蜀国密探发觉了,他们自然而然的得出了南楚军队想绕过巴郡的结论,虽然对他们来说,南楚军队绕过巴郡不攻,等于是放弃了后路和补给的安全,但是田维个性坚强好战,这次坚守实在是因为兵力不足,南楚军队虽然只有五万,但是却是南楚最精良的军队,所以田维的压力极大,这几天看到情况不对,他和属下的将领商议很久,都觉得南楚军队必然是要绕过巴郡。商议之下,有将领提出,若是南楚军队真的绕过巴郡,巴郡若不从后袭击,那么将来就是南楚军队全军覆灭,巴郡将士也免不了受到惩处,这个阴影让所以将领都心里不安。最后,田维下令,趁着南楚军队还没有完全绕过,从后面袭击南楚的辎重队。

五万大军想要从小路行军,速度是极为缓慢的,田维没有多久就赶上了南楚大军的后队,田维挥动手中的大刀,大喝道:“南楚狗贼休走。”就在他的喊声中,田维带着的五千轻骑如同钢刀一般切入南楚的后军,南楚军队份散逃走,田维下令向粮车辎重上面投掷火把,霎时间火光四起,火光中,田维高声大笑,下令继续进攻,要把南楚军队击溃。就在这时,四散逃开的南楚军中露出一支身穿白色衣甲的步兵,他们向田维迎来,田维心里一寒,这不是德亲王殿下的亲卫军么,这只亲卫军本应该是扼守中军的,可是现在居然出现在这里,自己莫非上当了,田维在四顾看去,那些粮车的火很快就熄了,而在那支步兵之后,打出一杆赵字黄龙旗,田维心中又是担心,又是忧虑,若是自己真的中伏,那么必会败亡,但转念一想,眼前就是德亲王的亲卫,说不定德亲王本人就在不远,若是一举杀了德亲王。田维经不起诱惑,挥令前行。两军相交,田维的骑兵虽然占了优势,但是南楚的步兵擅于和骑兵作战,只见他们前排军士麻利的的跪下,将丈八长枪挡在马前,后排的军士拉弓射箭,借着狭小的地势,将田维挡住,田维杀了一阵,眼看没有可能取胜,下令撤军,他们的战马跑得飞快,不一会儿田维就彻底脱离了战场,蜀军马快,田维庆幸的想,不过怎么,也算一场小胜吧,快马奔驰了十几里路,还没有冲出多远,突然,从道路两侧冲出南楚军队,两侧夹攻,田维连忙吩咐众人不可停留,拼着伤亡,戮力突围,此时田维心里已经有了寒意,在短短的十几里山路上,不时的有南楚军突袭,他们数量不多,都是用弓箭从草丛树林或者岩石后面攻击,若非这里不是山谷,只怕,田维这几千铁骑逃生无望,就是这样当田维看到巴郡城墙的时候,已经花了大半个时辰,而且只剩三千残军了,临近巴郡城,田维却看到蜀军的火红旗帜从城头飘落,德亲王的黄龙旗从城头上冉冉升起。田维眼睁睁的看着城头上几个蜀军战士被砍倒在地,就在寒光四射的刀枪从中,田维看到了一个十分不协调的人,那是一个身穿青衫的青年儒士,他正带着怜悯的目光看着自己,在那血火之中,他的衣衫似乎没有染上半点污迹。他站在城墙上,却和其他南楚军士隔了一段距离,他仿佛是一个不属于战场的幽灵。

我在攻城还未完全结束的时候就上了城墙,这次南楚军在我的提议下留了一万人下来,这是军议之后我在就寝前看书的收获,德亲王认可之后,我们接着挖战壕的时候在战场附近挖了很多大坑,然后在佯装绕行的时候,将一万军队藏在大坑里,上面覆盖着油布,布上面盖着黄土,那些来查看的探子果然只注意到营地空空,却没有注意到咫尺之处的藏兵洞,在田维出兵之后,我们趁着守城士兵疏忽,立刻开始攻城,结果松懈的守军被我们击败,而我之所以登上城墙,是因为想看看结局,当然理由是蜀国回军的时候恐怕会报复,将我们这些留在外面的幕僚杀死,凭着这个理由,我们都入了城,当然守卫也很森严,免得我们被残兵所伤,然后我又说想看看外面的情况登上城墙,小顺子笑眯眯的派了两个御前侍卫跟着我,他们是保护王海王监军的,但是王监军已经知道小顺子功夫不错,他又跟我挺好,所以就答应派人来保护我,听小顺子说,这两个人功夫底子不错,至少可以保护我直到南楚军士来救我的时候。

我在血海之中走过,小心翼翼的不想沾染上血迹,可是脚下血流成河,没多久我的鞋子就浸透了鲜血,可是我的运气不错,至少身上没有沾血。当我忍着战场上的气味和惨叫到了城墙上的时候,仅剩的几个蜀军也很快的就被杀死,我向城下望去,恰好看见返回的蜀军。那么红袍将领呆呆的望着城上,在他身后,烟尘滚滚,我可以看见我军的旗帜,突然,那红袍将领大喝着向我军冲去,然后,我就眼睁睁的看着这支骑兵被我军围困、消弱、击溃,远远的,我看见那个红袍将军横剑自刎,临死前还在咆哮。

我的心一阵颤抖,战争,并不像我在史书上看到的那样轻描淡写啊,在巴郡蜀军万人眼里看来,我军是万恶的敌人,杀死他们的身体,夺走他们的城池,可是我们能够怎么办呢,这一刻我真的深深痛恨起这场战争来,就为了大雍和南楚的利益,蜀国就必须灭亡,用血流成河换取上位者的喜悦,这,真的值得么?

接下来,我病了,那血,那惨叫声,让我睡不好觉,吃不好饭,在急速的行军中,我的病情渐渐加重,后来,有一天晚上,小顺子到我营帐里,把我拖起来道:“我知道你为什么生病,收起你那廉价的同情心吧,我们双方已经成了敌人,我们在打仗,如果我们败了,就没命回家,什么仁义道德,什么礼仪廉耻,我只知道,我得活着,为了你,我得活着,那么你呢,你至少要为了我活着,记住,你救过我的命,如果不让我还了这笔债,我绝对不让你死。”

我恍恍惚惚的看见小顺子脸上的眼泪,淡淡道:“小顺子,兄弟,我知道你对我如同手足,可是我总是欺负你,你总是照顾我,保护我,可是我要走了,你不要难过,你不欠我什么。”

小顺子狠狠打了我一个耳光,道:“你以为我总跟着你干什么,你从来没有瞧不起我,你认为我小顺子是个人物,你教我读书,帮我学习武功,没有了你,谁还看我小顺子一眼,你若是死了,我就跟你死,下辈子我要做你的兄弟,让你永远得罩着我。”

我的泪水滚滚而落,是啊,我怎么能死,我还有一个兄弟呢,我若死了,小顺子岂不是孤孤零零的一个人,我从来都知道,小顺子总望我那里跑,是因为,我把他看成一个人,一个有血有肉的人,而不是一个小太监,一个没有面孔的人。哼,蜀国算什么,你们的人死了和我有什么相关,别说蜀国,就是南楚亡了,和我又有什么相关,这些日子以来,我病势沉重,除了小顺子和军医,我没有看见什么人,德亲王虽然来了两次,可是他后来也遗忘了我。我勉强起身道:“把我包袱里面白瓷瓶里面的药给我两粒。”小顺子连忙过去照办,我艰难的吃下药丸,道:“我要休息一会儿,明天早上给我准备丰盛一点的早饭。”

三天之后,我在昏睡了整整三天之后,终于吃上了小顺子送来的早饭,走出营帐,看看晴朗的天空,我伸开双手,深深的吸了一口气,对小顺子说,我的病刚刚好,跟王监军说,今天我搭他的车。

在我卧病的十几天,南楚军队的进展还是比较顺利的,南楚攻破巴郡的打击让这些小城池失去了坚守的信心,借助这种策略,强攻软骗,进军的速度超过了预计,大雍方面不知情况如何,没有情报传来。接下来的日子,我大病初愈,所以公务不多,常常在余暇的时候写写诗什么的,我可没有再多言,虽然德亲王曾经歉疚得来问我的病情,但是我不会原谅他的,从前对我这样看重,我一生病就把我丢在一边,所以,我总是不冷不淡的说上几句多谢,反正我经常和王监军在一起,也不用担心他会为难我,我就是这样小气,怎样。

就在行军作战中,南楚大军到了雒城之前,和已经提前赶到的水军会合,雒城是蜀国国都成都的屏障,此刻,这里已经聚集了蜀国五万大军,蜀国名将魏贤率兵两万在雒县前面依山立寨,作为护持,大将军龙步率领三万镇守雒城,南楚水陆大军以雷霆之势,攻击涪水关,守将血战数日,终究弃城而走,在涪城安下大营,他知道接下来的一战非得旷日持久不可,所以只是安排水陆守军布置周密,令水军游弋涪水,隔绝后援,雒县北门临涪水,南门外都是山路,德亲王借助水军运兵从东西两门攻打雒城。但是魏贤每每率军夹攻,数日之间,两军血战数场,南楚大军未占优势。德亲王见军士疲惫,索性收兵,除了不时派水军游弋之外,只是在涪城休兵备战,虽然距离南楚很远,但是靠着水运和蜀中的丰富物产,南楚大军补给并无缺乏。战局陷入僵局之中。

十一月二十七日,终于得到大雍的战报。

大雍由雍王李贽领军二十万,由于事先收买了阳平关守将,轻而易举破关而入,连战连捷,花了两月时间攻克南郑,东川虽然属于蜀国,但是繁荣锦绣都在西川,所以东川之人不免怨恨,李贽入川之后,大军秋毫无犯,四处荡平残军败将,扫清贼寇,不到三月,东川平定,李贽方陈兵葭萌关前,葭萌关一破,则成都东侧就再没有屏障。

蜀国国主孟昀两面守敌,捉襟见肘,紧急调派,葭萌关守军共达九万,又派了两万援助雒城,成都空虚。十一月十二日,两万援军在雒城三万守军和魏贤两万军队的支援下进入了雒城。

德亲王得到战报的时候,满脸青黑,因为大雍即使退兵,只要守住阳平关,则东川必然被大雍控制,而自己若是得不到雒城,则不能抵御蜀军,若是退到巴郡,他也舍不得,所以现在,南楚比大雍更着急攻打蜀国的事情。可是对峙这么多天,没有丝毫进展,怎不令人心焦。不过值得宽慰的是,南楚援军赶到了,现在南楚水陆两军一共九万,至少不会败退了,这样,在大雍和南楚双方攻击下,蜀国迟早必然败亡,只是事后未必是南楚得到成都罢了。

这段时间我过得比较悠闲,每天除了吃饭就是四处走走,当然为了小心蜀军的谍探和刺客,我不会走得太远,而且如果太悠然的话未免招人眼红,反正现在我也插不上手。容渊趁着我生病,剥夺了我参赞军务的权力,当然借口是我的大病。在他来说,我还在卧病呢。不过我也不计较,反正这场仗应该没有什么机会打败。

闲着也是闲着,我就跟小顺子说找个护卫的事情,小顺子想了半天,很是为难,他并没有认识太多的高手可以介绍给我,按照他的说法,他遇见的高手若是可以交手的都被他杀了,而且还需要忠心,这就更难了,他问我要不找个小太监做徒弟教他武功,然后保护我,被我拒绝了,一个原因是时间太长,另外一个原因就是太监没有办法经常出宫,小顺子想了半天道:“要不过几年,我诈死从宫里出来然后到你那里去。”

我本来想点头的,可是小顺子若是给人认出来,或者被人发现身份都有麻烦,后来我干脆道:“这样吧,我准备这次回去就辞官,你反正也不喜欢呆在宫里,不如我们两个从此浪迹天涯如何?”小顺子想了想,高兴地道:“这也不错,我早就想四处走走了,建业我都腻味了。那么我们去哪里呢?”我想了一想道:“反正蜀国灭亡了,如果大雍和南楚暂时打不起来,我们就到大雍看看,等到大雍和南楚打起来,我们再到北汉去看看,等到大雍若是和北汉打起来,我们就回南楚。几十年的时间,够我们四处游历了,如果什么时候厌倦了,就找个地方住下来。”小顺子满脸都是向往的神色。

就在我们两个憧憬未来的时候,突然小顺子毫无征兆的向庭院里面的花丛扑去。身影如同鬼魅,快捷无比,草丛中一个灰色影子弹身而起,两人身影一合而分,小顺子退了丈许,身影一折,凌空折转,再次扑击,那人仓促还击,却被小顺子一掌击中心口,顿时委顿在地。

我见小顺子对我点了点头,轻轻走到近前看去,那人却是一个二十多岁的青年,相貌平凡,是那种一走进人群就再以分辨不出来的相貌,穿着南楚的军服,但是我可以看得出来那件衣服有些不大合身,而且可以闻到淡淡的血腥味,这人是蜀军的探子,我心里想,原本应该交上去,一则立功,二来这是本分,可是想到我刚才的话若给德亲王他们知道,心里顿时生出杀机,对小顺子使了一个眼色,小顺子会意,一掌向那人头颅拍去。

那人痛苦的睁开眼睛,恰好看见小顺子的动作,他艰难的在地上打了个滚,小顺子冷冷一笑,掌势折转,继续向那人头上击去,我看见那人眼中满是悲愤的神色,不知怎么,开口道:“住手。”小顺子掌缘已经到了那人天灵,听见我阻止,猝然收回攻击,退到我身后。我郑重地道:“兄弟,我必须杀了你,如果你有什么遗愿,我可以成全你。”

那青年眼中闪过激动的神色,开口道:“求你放了我妻子。”

我愕然,什么时候我抢了他的妻子么,我好像没有干这种事情啊。

\chapter{第十一章 钩心斗角}

那个青年痛苦的咳嗽了几声,满眼期望的看着我,我无可奈何地示意小顺子把他扛到房间里,然后问道:“本官不才,也是读书士子,自信没有劫夺妇女的恶行,不知道你为什么认为尊夫人在我这里呢?”

那个青年疑惑的看了我一眼,道:“草民韩章虽是蜀国人,但是并非官员或者军士,只是一个普通的农夫,草民的妻子却是名门之女,相貌出众,身份高贵,三年前,拙荆因为不满家里订下的婚事而离家出走,因缘和草民成婚,几个月前,拙荆得知母亲染病,所以回去探亲,草民因为正值秋收,不便久留,所以自行返家,谁知碰上大雍和南楚一起攻打蜀国,拙荆的父亲田维是巴郡守将,不幸阵亡,拙荆和岳母被俘虏,我听到巴郡城破的消息日夜兼程赶去,探得她们被德亲王赏给了军中幕僚江哲为奴,所以又一路追踪而来。”

我疑惑的看看小顺子,小顺子露出恍然大悟的神情,道:“大人,那时候您在病中,德亲王见田维之女相貌俊秀,所以把她赏给了大人,用来奖励大人献策的功劳,只是大人一直昏迷不醒,所以奴才代大人作主,将她们留在了王公公那里,这些日子,奴才因为大人身体刚刚康健,想多伺候大人几天,见田氏服侍王公公十分周到,索性就安排她们继续伺候王公公,这样大家欢喜。”

我这才明白,怪不得这段时间小顺子总在我身边呢,我问道:“王公公待她们如何?”

小顺子恭敬地道:“大人放心,田氏聪明灵巧,王公公还想收她做义女呢,只是田夫人因为伤心田将军之死身体不大好。”

韩章听到这里,露出不可抑止的喜色,只是片刻就被痛苦的神情掩盖。我心想,看来这个韩章不是蜀军的探子,但是他听到我刚才的话,还要不要灭口呢?转念一想,也没有必要,难道他还能去向德亲王告密不成。在我犹豫的时候,韩章已经是奄奄一息,我连忙掏出一个针盒,从里面取出金针替他针灸,然后又给他服下伤药,他在药力的作用下昏昏睡去。我对小顺子说:“田维之死,我无能为力,两国交兵,死伤是难免的,但是他的妻子女儿又没有什么大罪,你安排一下,等我们攻下雒城,道路通畅之后,你就放了他们一家三口。”

小顺子道:“是,到时我跟王公公说清楚就好了,王公公不会不高兴的,不过有点可惜,这个韩章功夫底子不错,奴才不敢妄自菲薄,就是宫里的侍卫高手能在我攻击下活命的也不多,如果能把韩章留在身边做大人的侍卫就好了。”

我觉得不大可能,道:“我是南楚官员,他是蜀国将领的亲属,何况还有岳母和妻子,哪里能够做我的侍卫。你也有些异想天开了。”

小顺子道:“这也不是没有可能的,他的妻子现在是大人的奴仆,如果大人允许他留下来和妻子团聚,他不也得感恩么,只是我知道大人需要的是忠心的侍从,这人若是被迫留下,就不好了。”

我点点头道:“是啊,宁缺勿滥,若是不忠心,留也没用,不过我们若能攻下雒城,至少还得一两个月,这段时间他们没法子离开,就让他暂时做我的侍从吧,免得你我来往过密,惹人生疑。”小顺子同意的道:“也好,免得我总是担心大人的安全。”

等到韩章醒来已经是深夜了,他能够感觉到四肢百骸里面真气蓬勃,完全感觉不到曾经沉重的令他几乎丧命的内伤,他没有动作,但是能够感觉到自己是在一个小房间里面,他感觉不到周围有人,正要坐起来,一只冰冷的手掌轻轻的按住他的胸口,然后火光一闪,有人点燃了火烛,韩章借着微弱的烛光看去,看见那个打伤自己的少年正冷冷的看着自己,眼中满是杀意。韩章聪明的停止了动作,他不想莫名其妙的死去,尤其是在得到妻子平安无事的消息之后。

那人见他十分冷静,露出了一个冷冷的笑容,开口说道:“我叫李顺,你要找的江哲是我的主人,主人已经决定,等到雒城之战有个结果的时候,他会释放你和你的岳母妻子,但是在这之前,希望你暂时作他的侍从。”

韩章犹豫了一下,他毕竟是蜀国人,做侵略自己国土的官员的侍从,未免有些不愿。

小顺子仿佛没有看到他的神色,继续道:“说句实在话,你并非好人选,如果有人怀疑你是蜀军的探子,难免会给大人带来麻烦,但是既然大人已经决定,我也没有什么意见,明天大人会带你去见你的家人,然后会向监军大人禀明此事,监军大人许可之后,你就可以暂时留在我家大人身边,可是有一件事你要牢牢的记住。”小顺子的面孔变得阴森,他一字一句地道:“你没有听见我和大人的谈话,你不知道我和他的关系,如果你泄漏了一个字,我就是天涯海角也要杀了你,还要让你的妻子遭受人间最大的苦痛。”

韩章凛然道:“江大人和李爷对我恩重如山,今日之事,韩章至死不会对第二个人说起。”

小顺子收回了手掌,淡淡一笑,离开了。

第二天我带着韩章去见王公公,王公公听说此事倒是十分成全,反正田氏母女是我的奴婢,并且允许韩章暂时留在我身边,当然,他也知会了德亲王一方,让他们知道此事,免得误会韩章是探子,不过我想,暗中的监视是不会少的,所以告诉小顺子,暂时不要过来了。当然我也第一次看到了德亲王赏给我的奴婢,田氏名叫田素英,相貌俊秀,英气勃勃,不愧是将门虎女,听韩章说,田素英也会武功,而且不在韩章之下,这次分明是因为母亲才无法脱身,这让我吸了一口寒气,如果田素英刺杀王公公或者我怎么办,我问小顺子这件事情的时候,小顺子毫不在意的告诉我,别说王公公身边侍卫不少,而且他已经警告过田素英,如果敢行凶,必然杀了她的母亲,反正他们也逃不出涪水关。我立刻对小顺子另眼看待,这小子做事严密谨慎,如果他肯用心,何愁不能成为太监里面最大的总管,在我跟他说这个的时候,小顺子轻蔑地道:“服侍国主有什么好,低三下四,奴颜婢膝,若是稍微有个差错,还要担心人头落地,你就不同了,你要是真的生我的气,大不了骂我一顿,还得小心我受不了反噬。”我在小顺子幽冷的目光下顿时心生寒意,立刻盘算以前是不是有对他太过分的时候,但是想来想去,好像应该没有,不过不管怎样,一定要记住,这小子武功很高。

此时的成都已经一片混乱,朝中重臣丞相审峻带着大将梵虎、孟靼驻守葭萌关,大雍攻城十分频繁,令葭萌关守军几乎目不交睫,而大将军龙步和大将魏贤守巴郡,也是不敢松懈,蜀国中枢几乎已经没有一兵一卒,蜀王孟昀数月之间黑发成霜,他又是怨恨南楚背盟,又恨自己为什么得罪大雍。想来想去,却没有丝毫办法退敌,后来蜀国重臣法澜献计,说东川既然已经失去,不如向大雍媾和,如果大雍收兵,南楚必然不会独自攻打蜀国。计策虽然被国主接纳,但是派谁做使者呢,雍王李贽名动天下,若是派个普通人,只怕连话也说不上几句,后来蜀国狂生杨灿自请前去。杨灿日夜兼程到了葭萌关,葭萌关上下血火熊熊,杨灿好整以暇的休息了一夜,第二天出关到了雍营,递上国书求见。未几,雍王命令帅帐请见。

杨灿是蜀国有名的狂生,平日里恃才傲物,目中无人,但是看到雍王军容整肃,帐前虎赍雄壮非常,也不由心生寒意,他整理仪容,走进大帐,只见一相貌雍容,神态温和却隐隐带着森然气息的戎装男子坐在帅案后,雍王李贽今年三十一岁,常年征战沙场的他却丝毫不带杀气,他穿着黑色轻甲,外罩锦袍,神色间雍容安详,仿佛是在家中闲坐,而非在沙场领兵一般,他左手一方,依次站着十几个武将,个个气势沉稳凶悍,他的右手站着十几个或穿文官官服,或者身着布衣的幕僚,可见其麾下文武之盛。

杨灿入帐,立而不跪,高声道:“蜀国使臣杨灿拜见雍王殿下千岁。”

那些武将个个怒目圆睁,其中一个相貌粗豪的武将叱道:“小小使者,见了殿下为何不跪?”

杨灿扬声道:“杨灿虽是布衣,却是蜀国之民,殿下虽然尊贵,却是大雍之臣,今日灿奉国主之命前来出使,焉能下拜。”

一个相貌斯文,年仅五旬的谋士温文尔雅地道:“蜀国朝夕败亡,我大雍二十万大军,兵陈关下,贵国国主不思求胜,却派你这个使者前来,所为何事?”

杨灿欠身道:“我国国主自知得罪大雍,如今兵临城下,焉能不恐惧,但是我蜀国一日没有沦陷,身为蜀民,不敢有辱国体。若是大雍恕罪,允许我蜀国称臣纳贡,则灿虽狂妄,焉敢不敬上国重臣。”

一个年轻谋士,相貌平常,却是鹰鼻深目,冷冷道:“蜀国如今朝不保夕,葭萌关旦日即下,不知蜀国拿什么求和,我国即可全胜,又何必留尔等残生。”

杨灿昂然道:“现在蜀国虽然大败,但是葭萌关和巴郡仍然在掌握当中,未必没有苟安的可能,若是贵国执意要灭亡我蜀国,我国主宁可将蜀中全部送给南楚,到时南楚既得蜀中沃土,又据有荆襄,即使以大雍之强,从此也只能坐视南楚壮大,若是肯罢兵休战,我蜀国不仅向大雍称臣,而且葭萌关外东川之地也不敢索回。我主深恨南楚国主背盟负义,今后若是怀恨,只会向南楚报复,大雍得我半壁江山,又可坐视我蜀国和南楚相互仇杀,岂不快哉?”

众人都听得沉吟不语,连日来攻打葭萌关不克,令他们也多多少少生出撤军的想法,只是战略已定,不能修改,所以人的目光都落在雍王李贽的身上。

李贽微微一笑,问道:“不知蜀中人物如何?”

杨灿朗朗道:“我蜀中人物鼎盛,文有萧何之才,武有霸王之勇,谋有良平之智,我蜀中俊杰,皆是忠义之士,灿虽不才,敢效田横壮士,或有灿未知者,愿效聂政荆卿之行。”

李贽眼中闪过一丝不可察觉的寒光,继续问道:“现在蜀王驾下,如君者几人?”

杨灿道:“文武全才,智勇兼备之人,数以百计,如在下者,车载斗量。”

李贽问道:“既然如此,贵使身居何职?”

杨灿答道:“国主治下,物富民丰,我等野人,归于田园,朝夕享乐。”

李贽淡淡一笑,道:“贵使远来,必然疲惫,请暂回关,若是有所答复,必然遣使相告。”

杨灿再拜告辞,出帐不远,一个白衫儒士,细眉长目,气度风流,悄然出帐,问道:“杨先生蜀中狂士,为何先倨后恭?”

杨灿答道:“先前倨傲,为的是不屈心志,后来恭敬,为的是我蜀国社稷。”

白衫儒士默然,道:“在下大雍宣松,字常青,日后若有托付,可以送一纸书信与在下,只要不干系国家大事,常青必会尽力。”

杨灿谢过,自经葭萌关返回成都复命。

之后半月,雍军不再攻城,葭萌关压力顿减。

未几,消息被南楚密探千里加急送到德亲王赵珏手中,赵珏愤然,他这段时间不大好过,雒城久攻不下,龙步不愧是蜀中大将,常常趁着南楚军势变化的时候出城作战,常常让南楚不得不败退,而魏贤擅长截寨,三日一小截,五日一大截,让南楚军睡不安枕,龙步、魏贤两人交相呼应,南楚军队一月来没有寸进,后方粮道常常受到溃散的蜀军的侵扰,赵珏一时之间束手无策,正在烦恼的时候,又得到了这个惊人的坏消息,如果蜀国和大雍真的媾和,那么真是南楚的末日到了,这时他想起了江哲。这个年轻的状元个性实在有些古怪,虽然赵珏迫使江哲从军,在江哲因为战场受惊而重病期间又不大过问,但是这倒不能怪赵珏,前者,赵珏认为江哲乃是南楚的臣子,既然有才能怎能不报效国家,后者,赵珏却是因为当时军务太忙,忙于行军作战,连克城池,岂是易事。而江哲病愈之后对军务十分冷淡,赵珏一来是觉得江哲大病初愈未免懈怠,二来,他也察觉到心腹幕僚容渊对江哲的排斥,因为不想破坏和容渊的宾主关系,毕竟容渊军略上十分精通,是他不可缺少的左膀右臂,所以相比之下,对江哲不免有些淡然。两方面原因,让赵珏和江哲越来越疏远。可是到了今日,赵珏再次感觉到江哲的重要,江哲远胜众人的,不仅仅在于分析情报处理公文的能力,而更在于江哲对战略上的远见卓识,从攻打巴郡一战看来,江哲善于事先规划好作战的目的,并且能够从浩如烟海的情报中找到突破口,虽然实施上需要有谨慎细密的人来作,但是已经是难得非常。现在赵珏遇到决策上的疑难,他终于再次想起江哲,只是容渊又怎么办呢?

正在赵珏烦恼的时候,容渊前来拜见,一见到赵珏就双膝跪倒,口称请罪。赵珏愕然,连忙扶起容渊,问道:“容先生为何如此大礼?”

容渊惭愧地道:“属下心胸狭窄,排斥贤能,罪在不赦,近日来,属下每每想起如何破敌,总是想不出有效的方法,若是江状元在此,必然能够抽丝拨茧,订下大计,王爷请偕同属下前去,让属下当面向状元请罪,戮力同心,以破雒城。”

赵珏大喜道:“先生知错能改,善莫大焉,赵珏也有错,疏忽名士,我们两人一起去见江哲,必然能够得到谅解,好请江状元定计,破此僵局。”说着将手中的情报递给容渊,容渊一看,面色如土,他自然知道现在局势的凶险,如果蜀国真的向大雍称臣,那么一旦蜀国恢复元气,必然会以南楚为报复对象。想到这里,他连忙催着赵珏一起去找江哲。

此刻的我还沉浸在舒适的客居生活,知道田素英也会武技之后,王海监军立刻同意把田素英和田氏归还给我,他们一家团圆,自然喜乐,只是田素英对我还是不冷不热,毕竟我是南楚高官,又是出谋划策让她的父亲败亡的罪魁祸首。我还不知道南楚的天空上已经压了一片黑云。

就在我写下一首刚做的诗文的时候,门外有人问道:“江大人在吗?”

\chapter{第十二章 毒计连环}

显德十九年十二月,南楚大军战于雒城,时,雍王李贽战于葭萌关,未几,有谣言大雍与蜀国将媾和,德亲王赵珏不安甚,问策江哲,密谈良久,翌日,楚军攻城甚急,十二月十九日,雒城破,大将魏贤中伏死,翌日晨,再度攻城,大将军龙步误听谍报,出城追击,乃诱入重围,战一日,人困马乏,德亲王亲自阵前招降,龙某不从,愤而自尽,王叹息,亲收骸骨,礼葬雒城。

十二月二十五日,葭萌关得知雒城失陷,蜀军无战意,十二月二十八日,葭萌关陷,至此,蜀国再无屏障。有人云江哲曾于此时献破城、离间二策,后某偶遇故德亲王幕僚容渊,问其事,容某沉吟良久,曰有之,然二策详情终不肯说,未几,容某病逝,某往祭之,其子代白父语,江随云天下奇才,惜德亲王不敢用之。

--《南朝楚史·江随云传》

我放下笔,韩章已经走了出去,打开院门,看见外面站着一个身穿黄色金甲,披着白色锦袍的大将,这人身后则站着一个黑衫儒士,而在两人之后,则是一队身穿白色衣甲的亲卫,韩章在南楚军中已经待了将近一个月,怎不知道这人的身份,他心里忐忑,默然让开门口,站到一边。赵珏只是看了他一眼,便走进屋子里去,容渊打了一个手势也跟了进去,其他的亲卫立刻将江哲的书房团团围住。

我没有听见韩章通报的声音,正感觉奇怪,就看见赵珏走了进来,连忙按照礼仪站了起来,躬身行礼道:“王爷莅临寒舍,随云有失远迎,请王爷恕罪。”

赵珏先对了施了一礼道:“赵珏近日军务繁忙,没有前来探望江大人病情,还请见谅。”

我淡淡道:“王爷手持重兵,日理万机,哪里有闲暇顾及下官,不知今日到此,有何指教。”

赵珏看了容渊一眼,容渊连忙上前歉疚地道:“江大人,日前多有怠慢,还请大人海涵。”

我自然道:“容先生不必多礼,两位亲自莅临,必然有紧要的军务,还请直言。”

容渊惭愧的看了我一眼,道:“蜀国派使臣求见雍王李贽,请求媾和,李贽没有答应,但也没有拒绝。”说着递给我一大堆情报,十分详细,就连李贽和杨灿的对话都有。我看完之后,不由微笑,这个杨灿名字和我第一个学生陆灿一样,个性也差不多,又是威胁,又是利诱。只是太可惜了,我长叹道:“蜀国人物锦绣,可惜蜀主不能用贤才,如今江山危亡,这些贤才却仍不弃蜀国,难怪人说蜀人忠义。”

赵珏问道:“怎么,随云已经看出大雍不会接受蜀国的求和么?”

我笑道:“若是雍王想要同意媾和,就会问问有什么好处,但是雍王只问蜀中的人才,分明是打算收揽人才好治理蜀中,所以雍王不会同意求和。”

赵珏皱眉道:“那为什么雍王会让这样的消息流传出来,若是我南楚知道,未免有些……”

我淡淡道:“雍王果然是厉害,他这是故意让这样的谣言流传,恐怕就是要我南楚知道,据下官所知,这段时间我们南楚并没有认真攻城。我想王爷是希望雍王血战攻下葭萌关,到时雒城守军受到影响也不能再全力守城,到时我们就可以不费吹灰之力得到雒城。”

赵珏和容渊相互看了一眼,这本是他们两人暗中商议好的决策,想不到被我一语揭穿,两人都不语默认。我继续说道:“我想雍王也是心疼自己的损失,所以利用这个谣言迫使我们速战速决。唉,雍王真是可怕,就算我们看穿了他的心意又怎么样,到了后来,大不了雍王同意蜀国的求和,就可以让我们直面蜀国的怒火,大雍可以容忍蜀国的继续存在,他只要守稳阳平关,就可以占据东川沃土,而我们苦战至今,却只能得到艰险的蜀道,得不偿失,我们没有能力,也不敢拖下去,若是蜀国喘息过后,必然会向我们这背盟的南楚攻击,唯今之际,下官有上中下三策献上。”

赵珏听得心乱如麻,问道:“是哪三策?请大人详细讲来。”

我正容道:“这下策是仍然如旧,若是大雍忍耐不住,先攻下葭萌关,那么我们达到了原来的目的,只是若是大雍一怒之下,改了主意,接受蜀国的求和,那么我们就完全失败,此一策,胜败在雍王一念之间。”

赵珏黯然道:“胜负若是完全仰赖他人,岂不是人为刀俎,我为鱼肉,那么中策呢?”

我沉声道:“中策是我大军戮力攻城,只要我们攻破雒城,大雍就会立刻攻打葭萌关,到时蜀国自然灭亡,我们所得会超过预计,但是损失也会超过预计。”

赵珏皱皱眉,他就是不想损失太大才不想全力攻城的,他沉声道:“那么上策呢?”

我淡淡笑道:“我们用奇攻城,攻下雒城,损失不大,到时候在蜀中争夺战果的时候再克制一下,就能够达到全部的战略目的。”

赵珏眉头舒展,道:“可是如何用奇破城呢?”

我胸有成竹地道:“如今雒城守军分为两军,一军守城,一军在外扎营,交相呼应,如想取胜,必先断其外援,我建议我们加急攻城,然后在外面大营的敌军可以看到的方向点燃大火,那么外面的军队必然会以为雒城危急,前来救援,我们在途中设伏,全歼蜀军,外援斩绝之后,我们就可以专心对付城内的守军。到时,我们可以再一次攻城,然后让人穿了蜀军的衣甲假意袭击我们的辎重,让他们以为外面的蜀军仍然存在,我们表现的因为辎重被毁急忙撤退的样子,引诱蜀军出城追杀,安排伏兵断去后路,诱杀主将,到时雒城不破待何?”

听到这里,赵珏失手打碎了茶盏,这个计策如此精密狠毒,他用崭新的目光看着我,看得我莫名其妙,赵珏终于确认了眼前这个温文尔雅的少年状元,竟是一个心机深沉阴柔诡谲的人物,赵珏不由生出一丝寒意,他是一个光明磊落的人,虽然也曾用过计策,但是却没见过这样阴险的伏杀、诱杀连环圈套。他有些不大自然的向江哲告辞,前去安排作战。

十二月十六日黄昏,南楚军队开始猛攻雒城,不久,点燃大火,魏贤果然误以为雒城危亡,从山路急驰而来,却被南楚军中途伏击,魏贤拼死作战,被亲自参与伏击的赵珏亲手击杀。蜀军四散,楚军严守通道,避免消息被雒城得知。

十二月十七日清晨,南楚军队再次大举进攻,大将军龙步率领守军击退数次进攻,伤亡惨重,到了未时,南楚军队突然混乱,然后立刻收拢退兵,龙步在城上看见涪水关方向的南楚大营大火熊熊,这时,有赵珏亲信伪装蜀军斥候,前来禀报说是魏贤烧毁南楚辎重,龙步大喜,见蜀军败退毫无条理,率领铁骑从后面追杀,南楚军队大败四散,龙步追出二十里,却被隐藏在山中的南楚军队截断后路,南楚军收拢,形成十面埋伏,龙步带着七千精兵四处冲杀,苦战一夜,血染战炮,身背十几处重伤,终于冲不出去,身边亲卫全部阵亡,最后被南楚军队围住,赵珏亲自劝降,龙步大笑道:“我蜀国只有断头将军,怎有屈膝投降之辈。”说罢举剑自刎。赵珏为之叹息,命令厚葬。

十二月十八日,南楚军队撤回涪水关修整,十二月十九日,赵珏再次攻城,雒城失去大将,无力防守,于黄昏时开城投降。

十二月二十三日,葭萌关得到雒城失守的消息,军心大乱,成都已经完全裸露在南楚兵锋所指。同日,雍王李贽大举攻城,城中蜀军毫无斗志,十二月二十五日,葭萌关失守,丞相审峻被俘,至此,蜀国再没有可以据守的关隘。

蜀国孟昀得到战报,几度昏厥,在朝堂上泣道:“先祖开国,至今六十年矣,如今烟消云散,孤虽死不敢见先人。”问计于朝臣,有人提议投降,孟昀思之再三,掩面回去后宫。朝臣面面相觑,只得四散。

显德二十年辛未,元月新年,大雍、南楚朝野庆贺大胜,蜀国上下一片惨淡,等待正在修整的大雍和南楚两军入蜀中会师成都。

军中上下一片喜气洋洋,我在帅帐喝了几杯酒之后就告退了,回到住处,躺在床上,迷糊糊地想着将来,这次我献计只有赵珏知道,我还请他不要传扬出去,赵珏倒是答应了,想必是也觉得我这个计谋太狠毒,却不知我是因为即将离开南楚,若是给人知道我的计谋,恐怕我一辈子也不得安宁。胜利带来的喜悦并不多,因为我知道韩章昨夜在房中痛哭,虽然声音很低,甚至避开了他的妻子。

其实比起我的计策,雍王的计谋才是狠毒,堂堂正正的设计,让他人为他火中取栗,当时我还有一点没有告诉赵珏,雍王还有一个目的,就是趁此得到我南楚间谍网的第一手情报,想来,为了收集大雍和南楚媾和的情报,再把情报经过蜀中传递给德亲王,雍王应该趁机掌握了不少我南楚密探的身份行踪,到时候就可以轻而易举的铲除南楚的势力,这是何等心机啊。我并非有意没有告诉德亲王,这一点他们如果想不到,也未免太迟钝,更何况,我真的有点不敢得罪大雍,再加上赵珏对我的隐隐忌惮排斥,让我也失去了辅助他的兴趣。

唉,我本来是真的希望南楚能够有个中兴明主的,可是赵珏此人并非管仲乐毅之才,倒是有点像鲍叔,黑白善恶太分明,不知道合光同尘的道理。

南楚,真的没有希望了么,我沉痛的想着。

显德二十年的一月,双方大军都没有急于进取,各自修整,巩固已经占据的地盘,等待春暖花开的时节。对于蜀国来说,这是最后一个冬天。蜀国虽然失去了大部分军事力量,但是局势并不容易安抚,蜀民的骄傲和执拗让我见识了,短短的一个月,发生了七次叛乱,二十三次刺杀,当然没有人特意针对我,我不是一个出名的人。蜀中的名士也大多不肯降服,不能反抗则默默抵制,局势如此险恶,让赵珏的头发都白了几根。明明取胜,却如此艰难,最后统一的意见达成了,若是蜀王投降,应该会好一些,所以大雍和南楚经过鱼雁往来,终于决定提前在二月一日,同时向蜀中进发,兵指成都。

在出发之前,赵珏再次来见我,他现在喜欢私下来见我,而不是公开征询我的意见,想必是觉得我的计策太阴毒,不过我倒是比较喜欢这样的方式,比较安全么,没人知道是我的主意才好,何况这段时间我出了不少主意,都是有些阴狠的,所以我早就让韩章走了,他一个蜀国人这时候留在我身边,太危险了,万一他觉得杀了我可以抵得上他一家性命,我可就惨了。

为了保护我,小顺子费尽心机,居然在南楚军的大牢里面找到了一个好人选,那人叫陈稹,是蜀军的密谍杀手,城破之后他被俘虏,因为查出他多次行刺我军将领,原本是应该处死的,而且此人个性阴狠,天性凉薄,是个除了自己什么之外什么都不在乎的人物,若非这次破城太快,他早就逃走了。但是小顺子就是看中他天性凉薄好控制,让我管赵珏救了他,赵珏正觉得没有给我什么报答,所以就答应了,小顺子用秘传手法在他身上做了手脚,我还不放心,又让小顺子把我配制的一种慢性毒药给他吃了,当然只告诉他解药在小顺子那里,这样,我就有了一个可信的保镖,按照小顺子的说法,此人不会舍生取义的来杀我,爱惜生命又让他不会背叛,是最好用的护卫了。

赵珏坐下之后,忧虑地道:“我们即将攻占成都,到时没有了蜀国的缓冲,我们应该怎样应对大雍呢,国主传来密旨,让我们不可得罪大雍。”

这一点我早就有了腹稿,说道:“下官想,主要的矛盾会在成都,谁能够俘虏蜀王,谁才是最大的受益者,这一点我们不要和大雍争夺,争夺了也没有用,下官有个主意,让蜀王不会落到大雍手里,这就可以了,另外,我听说蜀王之所以远贤才,亲小人,是因为宠爱王妃金莲夫人和内宦张全,我听说金莲夫人美丽绝伦,我们将蜀国朝臣后妃全部让给大雍,若是雍王将他们送到雍都献俘,那么凭着金莲夫人的美色必然能够得到雍帝的宠爱,到时候我们就在大雍的后宫埋下了火种,若是雍王杀了他们,虽然有点可惜,可是雍帝得知必然心里恼怒,年老之人最爱美色,尤其是雍帝这种并非十分贤明的君主,不论事成事败,我们都离间了雍帝父子。当然我们也有必得的东西,入城之后,请王爷派容先生先去户部,收集典籍户口图册,这是我们将来治理西川的关键,金银珠宝之类当然也得抢夺,一则掩盖我们夺取文书的重要性,二来好犒赏三军,贿赂国主,至于其他,我们就不要管了,那些蜀国朝臣的府邸我们就让大雍去处理吧。”

赵珏听了连连点头道:“江大人提到离间雍帝父子,不知道可否多说一些。”

我想,反正我也要离开南楚了,就别藏着掖着了,所以说道:“雍王和大雍太子李安争夺皇储一事,天下皆知,大雍以武立国,李安必然处境尴尬,这次雍王破蜀,如此大功,李安必然恨得咬牙切齿,我们派密探到大雍去散布流言,说雍王要在东川自立……”说到这里,赵珏已经明白,深深的看了我一眼,满是惧意,他深深拜服道:“江大人良策,可保南楚数年平安,不知大人有何需求,赵珏必然全力相助。”

我心想,这是许愿封官了,淡淡道:“下官这次鞍马劳顿,染了病根,希望回去之后辞官回乡,若能得王爷许可,向国主进言,感激不尽。”

赵珏皱皱眉,心想,江哲此人心机深沉,若是不图权势倒是好事,可是他若走了,万一投了别国,那么南楚危矣。所以赵珏断然道:“江大人此言差矣,君才智过人,赵珏正要仰仗,怎可归隐,若是江大人不喜欢政务繁忙,本王当禀明国主,让江大人在翰林院恩养,无事不必处理公务即可。”

什么,我目瞪口呆,怎么事与愿违。当天晚上当我委屈地跟小顺子说的时候,小顺子拍着脑门道:“天啊,大人,你还是不明白那些皇家人的心思,你这样的人物,他们若肯放手,不怕你去投靠大雍、北汉么,谁让你这么露锋芒,看来你不仅不能辞官,从此以后还要韬光养晦,有了合适的机会干脆弃官而走。”

我赧然的看着小顺子,表示惭愧和拜服。

显德二十年二月十五,南楚和大雍会师成都,两军将成都围得水泄不通,德亲王前去拜会雍王李贽,我按耐不住好奇心,也想看看雍王是何等英雄人物,就跟着赵珏去了。来到雍王大营,看着虎踞龙盘杀气隐伏的大营,我就先是赞佩不已。雍王李贽在营门迎接我们。离得老远,我就看见了他站在营门口,他身穿亲王服饰,雍容高贵,虽然只是站在那里,我却觉得仿佛整个大营的气势都聚集在他身上。离大营百步,德亲王下马步行,我自然也照着做,离大营越来越近,雍王微笑着迎上,而我在这时终于发现了一件令我惊骇欲绝的事情,雍王李贽,我居然是认得的。

\chapter{第十三章 一曲催行}

我强忍着心里的恐惧低下头去,没错,是恐惧,那个李贽居然就是我在赴建业途中遇见的李天翔,天啊,我居然在大雍的雍王面前说了如何一统天下的大计,而且还说了大雍的内患,难不成,雍王真的听了我的建议,先破蜀,后破南楚,不会的,雍王文韬武略十分惊人,应该是他自己早有的主意吧。

这时雍王迎上前来,和赵珏以礼相见,雍王温和地道:“德亲王一路杀伐,途中辛苦了,破巴郡、陷雒城,只此两战,便可见亲王名将之姿。”

赵珏脸微微一红,道:“雍王如此赞誉,珏愧不敢当,今日我们两军会师,蜀国只剩成都孤城,不知雍王殿下如何打算。”

雍王道:“成都如今轻易可破,只是此城乃是蜀国都城,士民千万,繁华非常,若是我们两军破城,必然有害百姓,本王已经拟了一道劝降表,不知亲王以为如何?”

赵珏淡淡道:“劝降可以,只是这蜀王应该向大雍归降,还是归降我南楚呢?”

雍王理直气壮地道:“南楚为大雍属国,蜀王自然应该向大雍投降。”

赵珏心里早有准备,只是淡淡道:“既然如此,就请雍王殿下派遣使者前去说降,如果蜀王不肯归降,明日你我两军大举攻城如何?”

雍王笑道:“正该如此,苟廉苟先生是我帐下使节,我已请他出使,德亲王意下如何?”

赵珏忍不住看了我一眼,见我没有反应,便道:“苟廉先生跟随雍王殿下多年,据闻当年常常替殿下出使各方诸侯,想必定然能够劝降蜀王,珏静候佳音就是,只是珏军务繁忙,这就先回去等待消息。”

雍王李贽见赵珏同意自己的决定,便请赵珏留下一个亲信将军或者幕僚,好便于双方联络协商军务,赵珏想了一想,觉得也很有必要,只是看看身边的人,虽然都是亲信,但是传个话还行,若想能够和雍王商量军务,争取南楚的利益,就只有容渊和江哲两人,容渊是赵珏一刻也离不开的,所以他坦然道:“明日是战是和还没有一定,这位江参赞是我臂助,就由他留下吧,若有什么变化可以和他商量。”

雍王这才看了我一眼,似乎才看见我一般,我却觉得浑身发冷,赵珏这个白痴,雍王如此轻易就骗了他,我才不信商量什么军务呢,八成要我留下才是雍王的目的。眼看着赵珏离去,雍王请我跟他一起到帅帐叙谈,等待使节返回。我忐忑不安的跟着雍王进去,至于我的护卫陈稹早就被挡在帐外了。雍王坐在帅椅上,见我拘谨不安,笑道:“江大人怎么如此拘束,我们也算是旧识,还是不要多礼吧。”

我在心里痛骂了半天,才道:“当日下官多有得罪,不知是雍王微服出行,还请殿下恕罪。”

李贽见我坐下,才道:“何言恕罪,当时本王化装入蜀,查看蜀中军机民情,回程之时幸遇公子,听君一席话,胜读十年书,我大雍若能一统天下,江公子功在社稷。”

我差点气晕过去,我若是功在大雍的社稷,岂不是罪在我南楚的江山,这话若传了出去,岂不是要我的命么?我连忙辩解道:“雍王殿下胸藏锦绣,小臣的些许见识必然早就在殿下心中,殿下将这样的功劳推给小臣,随云可不敢当。”

雍王淡淡一笑,没有继续编排我,而是单刀直入地道:“当然听了公子的计策,又听说公子要到南楚出仕,本王原本想效强盗之行,将公子带回大雍,可惜恰好有人发现了本王行踪,欲图行刺,本王当时身边侍从不多,唯恐不能保护公子的安全,只得放过,如今公子已经成了南楚的臣子,真令李贽扼腕痛惜。”

我一听,心想,以他的身份,就是身份泄漏给蜀国和南楚,八成也没有人敢要杀他吧,那么想杀他的人自然只有一个了,想到李贽如此才华身份,却因为是次子,不能承继帝业,还要遭受兄长的妒忌和暗算追杀,倒也不由让人痛惜,不过我痛惜痛惜就算了,你就不要痛惜了,若是当日我被你带走,十有八九已经遭到池鱼之殃,死于非命了。心里想着,嘴里却道:“这也是小臣无缘为殿下效力,想必是天意如此。”

李贽看看我,眼中满是笑意,道:“当日你我有缘相逢,今日相见,江公子已经是德亲王的心腹军师,想必给德亲王出了不少好主意,德亲王和他手下其他的幕僚将军,都是比较正统的军人谋士,攻打巴郡、雒城这两战几乎都是用了诱杀和伏击的计策,想必是江公子的妙计了。”

我觉得身子有点僵硬,苦笑道:“小臣对军务上的事情哪里明白,只是说了一个原则,都是德亲王英明果断,定下计谋,才取得大胜。”

李贽郑重地道:“孙子兵法上面说‘夫未战而庙算胜者,得算多也;未战而庙算不胜者,得算少也。多算胜,少算不胜,而况于无算乎!吾以此观之,胜负见矣。‘,公子长于庙算,就已经是绝世之才,李贽能遇公子,如同周文王遇姜尚,汉高祖遇张良,南楚苟安江南,文恬武嬉,德亲王虽然文武双全,可惜没有帝王的气度,公子在南楚不过一文人骚客,若是归我大雍,必然是右弼之才。”

我心想照样招纳别国官员也未免太嚣张了吧,所以反问道:“听说石彧石子攸是雍王幕府首席谋士,雍王殿下每次出外,所有治下政务都由他一手处置,想必石先生就是殿下心目中的左辅吧。”

李贽显然有些不明白我为什么问这个,但是仍然答道:“子攸长于政务,有子攸坐镇后方军政,李贽才能用兵如神。”

我正色道:“若是石子攸也是别国臣子,其主并未薄待,一说而降,那么殿下还能这样重用他么?”

李贽一愣,苦笑道:“若是如此,李贽焉敢深信子攸。”

我笑道:“所以殿下明白小臣的苦衷了?”

李贽叹了口气道:“南楚并非梧桐,何缘栖得凤凰,南楚以凡人待汝,我以国士待君,随云还是不肯投我大雍么?”

我呆呆的望着李贽,其实我是真的有一点点后悔,如果当初李贽真的把我强行带走,我当时或许会很不高兴,甚至怨恨,可是也许现在就不用为了南楚费心,可是我既然已经做了南楚的官员,而且这些年来升迁顺利,又在翰林院学到了那么多东西,南楚待我不薄,我无论如何不能就这样投靠大雍,然后看着大雍灭亡南楚。想到这里,我黯然道:“南楚虽以凡人待人,我亦不该背叛,随云身为楚臣一日,就要为南楚效力一日。”

李贽轻声叹息,道:“若是南楚被我大雍灭亡呢,你会怎么办?”

我想了一想,道:“我自认没有覆雨翻云手,没本事绘出锦绣经纶图,若是南楚灭亡,若是大雍不加罪小臣,小臣自当浪迹天涯,与草木同朽。”

李贽淡淡道:“你在南楚攻蜀之时参赞军务,如此能力让人侧目,那赵珏虽然不能尽用汝才,但是想必日后也免不了用你参赞,到时,就算你想,大雍也不会放过一个你这样的人才。”

我认真地想了一想道:“若是小臣肯答应回到南楚之后不再出谋划策对付大雍,不知到时殿下可以放过小臣一条生路么?”

李贽微微皱眉,半晌问道:“你在蜀中仍有计策没有实施?你认为已经足以报答南楚君恩了么?”

我钦佩的看着李贽,雍王真是绝顶聪明,从我的一句话,就可以看出这些东西。我也不隐瞒他,道:“我替德亲王策划一谋,若是成功可保南楚数年平安。”

李贽突然眼中闪过一丝光芒,道:“如果我猜的不错,是和蜀王有关,蜀王若是归降我大雍,是南楚的心头大患。”

我也不掩饰,道:“正是如此,若是蜀王投降,我自有办法让蜀王死去,到时至少大雍占不到便宜。”

李贽面上露出深思的神色,道:“若是蜀王不肯归降,你我两军攻打成都,杀死蜀王或者蜀王自杀还都可能,若是蜀王投降,你真的有办法令蜀王死于大雍军中?”

我知道他不相信,但是却斩钉截铁地道:“正是如此。”

李贽站起身来,在帐中走了几步,道:“好吧,若是你真能如此,并且回到南楚之后再不替南楚设谋,只要我大雍破楚之时,你不在建业,本王就答应你让你平安度日。”

我大喜,这可是保命的谕旨啊,连忙上前拜谢,李贽意味深长地道:“如果蜀王平安到了大雍,又如何?”

我毫不犹豫地道:“若是如此,随云情愿为殿下效力。”

李贽大笑道:“好,好,你我一言而定。”说着伸出右掌,我心里一暖,也伸出右掌,两人击掌为誓。为了保险,我又道:“若是小臣取胜,回到南楚之后,如果殿下有和南楚无关的疑难,小臣可以代为参谋一二。”

李贽又是一愣,他原本想,若是我真的有本事在自己掌握之中杀了蜀王,那么自己将来又要放过他,但是是否要借助在南楚的力量先把我困住,想不到我又有这样一个提议,不由惊叹,默然良久道:“好。那我们先看看蜀王会不会投降吧?”说罢,回到帅案前坐下。

我也不知道继续说什么,也就坐在那里等着苟廉出使的结果。

等到日沉西山,苟廉回来了,向李贽禀报,蜀王明日正午将出城投降。我和李贽都是面露喜色,关系着我命运的赌注就要开始了。在和李贽商量过明天两军如何配合的细节之后,我要返回南楚军营,雍王亲自送我出营,让我受宠若惊。

第二天,蜀王白衣素服,带着文武百官,众位王子,出城十里投降大雍。纳降之后,我们两军分别从西门和东门入城,两军已经有了默契,基本上没有发生什么纠纷,只是在户部,容渊容先生和雍王的幕僚崔峦相遇,两人都奉命夺取户部文书典籍,对峙不下,在争论良久之后,雍王和德亲王亲自协商,决定异人一半,虽然可惜,但是总是比没有得到的好。赵珏暗中问我,蜀王投降,那么我们的离间计如何进行,而且蜀王归降大雍,对南楚统治西川也十分不利,我早就胸有成竹,告诉赵珏,只要在蜀王出发到雍都之前,举行一次宴会,让我参加就可以了。

经历了复杂的谈判和分赃之后,德亲王决定启程回国,雍王按照礼仪提出为德亲王饯行,这是顺理成章的事情,德亲王自然要赴宴的,而蜀王也要出席相送,在华丽的蜀王宫中,大雍和南楚的将军谋士坐在两方,饮酒作乐,蜀王坐在雍王下首,殿下坐着跟着蜀王归降的臣子,他们面色都不大好,尤其是蜀王,听说不到五十岁,可是相貌憔悴,须发皆白,说他是七十岁都有人信。酒过三巡,赵珏按照我的计划提出有酒没有歌舞太没意思,不如让被俘的蜀王女乐来歌舞助兴,大雍将帅虽然觉得南楚果然柔弱,但是也没有什么阻止的理由,就让蜀王的女乐前来助兴,蜀国琴乐,若浪激奔雷,蜀国宴舞,矫健婀娜,那些即将离开蜀国的君臣自然是强忍泪水,大雍和南楚的将领却是拍手叫好。

我看时机已经到了,对赵珏使了一个眼色,赵珏会意,起身道:“今日见了蜀中乐舞,十分动人,我南楚文雅风流,岂能没有歌舞悦宾,只是军中没有女乐,只好由在下操琴,以悦主人,翰林江哲,乃我南楚才子,为了今日之会,特意写了新词,请众位赏鉴。”

雍王李贽心里一动,这些日子以来,他派重兵保护蜀王,可是没有见到半个南楚杀手,今日蜀王即将赴大雍,他本就猜到我要有所动作,可是我只是要当场唱一首新词罢了,若是拒绝了赵珏亲自操琴,那么南楚君臣必然恼恨大雍无礼,所以虽然李贽明明知道不妥,仍然只得同意。

我站了起来,向众人施礼,赵珏坐下,轻抚琴弦,琴声悠扬清越,正是词牌《破阵子》的音律,我朗声唱道:“六十年来家国,三千里地山河。凤阁龙楼连霄汉,玉树琼枝作烟萝。几曾识干戈。一旦归为臣虏,沉腰潘鬓消磨。最是仓皇辞庙日,教坊犹奏离别歌。垂泪对宫娥。”

一曲唱罢,满殿寂静,李贽心里一寒,知道我已经出手了,向蜀王看去,蜀王本是麻木枯槁的面容上,露出悲痛欲绝的神色,而那些在殿下的蜀臣不是泪下如雨,就是怒目瞠视。良久,蜀王孟昀起身道:“小王酒后疲惫,请大雍雍王殿下允许小王暂回后宫小憩。”

雍王李贽面露苦涩,想要阻止,却偏偏无法出口,只得长叹道:“国主暂到后宫休息,请不要多虑,陛下必然不会薄待国主。”

孟昀没有答话,只是向殿中众人一一看去,当目光落到我身上的时候,我感受到他那目光中的绝望和怨恨,对于一个撕破你的美梦的人,还能有什么好感,然后蜀王离席而去,蜀国的朝臣都默默跪下相送。李贽苦笑着看向我,又是赞佩又是恼怒,遥遥举杯,一饮而尽。

片刻之后,几个内宦哭着到了殿前,下拜道:“国主饮鸩而亡。”

李贽大笑道:“好,好,江状元真是厉害,一曲破阵子,送了一位国主的性命。”说着淡淡道:“本王即将回国,军务繁忙,这就告辞了。”说罢转身而去。

赵珏和容渊都已经背心湿透,他们既是欢喜终于让蜀王自尽,又是担心过于得罪大雍。我则是哭笑不得,虽然逼死蜀王是很过分,但是也要他有羞耻之心,李贽临行的一句话似乎表示了对我的怨恨和不满,但是换个角度来说,我在南楚就可以安稳度日了,不过,他这一句话让我名扬天下,将来我岂不是难以隐姓埋名,这个李贽,这种情况还记得反击,真是可怕。

李贽坐在马上,终于处理完了蜀中的军政,他就要回大雍了,虽然蜀王自尽,但是蜀王妃和王储都在,足够献俘太庙的了,南楚大军已经在前日回军,按照两国盟约,东川归大雍,西蜀归南楚,实际上,葭萌关控制在大雍手里,雒城控制在南楚手里,蜀中却是两国缓冲之地,他的战略已经得到实现,只是,南楚占得便宜也不小,李贽苦笑,现在他可真是后悔当初没有冒险掳走江哲了。

他的幕僚谭说上前道:“殿下为何当日不阻止蜀王自尽,平白让南楚得意?”

李贽看了他一眼,他知道这些幕僚和属下将领对此都有疑问,淡淡道:“来不及了,若是蜀王在那种情况下还不自尽,只怕蜀中之人都会鄙薄他,他就是活着也是行尸走肉。”

李贽麾下猛将樊群怒道:“肯定是那个赵珏的诡计,居然让那个状元写词讥讽蜀王。”其他人都一一附和,不过有些幕僚也说,江哲的这首词真是绝世之作。

李贽微笑不语,心道:“你们怎么知道,那个江哲才是罪魁祸首,不过他倒干得巧妙,至少没有人猜到是他的主意。这个江哲,真是值得本王费心啊。”看看天色,扬鞭道:“我们快走吧,就让他们得意一阵子吧。”

附:

显德二十年二月十六日,蜀王孟昀白衣归降,蜀亡。

显德二十年三月二日,雍王贽为德亲王珏饯行,蜀王孟昀陪宴,席间不乏蜀乐歌舞,王乃亲自操琴,命哲演唱新词,哲歌《破阵子》,蜀王闻之,羞愧而退,乃饮鸩,殇,终年四十七岁。时人称江哲此作为《断肠词》,或为《绝命词》。

--《南朝楚史·江随云传》

\chapter{第十四章 玉碎珠沉}

远远的看见建业城,我真是心潮澎湃,终于回来了,离城三十里,国主带着文物百官前来迎接凯旋的功臣,我们都下马参拜国主,国主大喜,拉着德亲王的手道:“王叔功在社稷,孤已经备了酒宴,为王叔庆功。”当我随着大军入城的时候,无意中感觉到有人在御道左边的一座小酒楼上,一直的看着我,但我却没有觉得有什么恶意。

庆功宴后,我带着陈稹匆匆忙忙的赶回住所,这次攻打蜀国,我得到不少赏赐,所以早就决定另外在郊外买一座房子,反正德亲王也答应帮我通融,允许我在家养病,我就不用住在城里面那么拘束了,在我回来之前,小顺子已经跟着王海先回来了,他早就替我选好了房子,付了钱,得到房契了。在昨天晚上,他到驿站见我,告诉我房子的位置。我和陈稹按图索骥,没有多久就找到了那处宅院。那是一座清雅幽静的小农庄,亭台楼阁倒是应有尽有,小顺子已经雇了几个仆人,将上上下下打理得一尘不染。

我沐浴更衣之后,到了书房,里面小顺子已经把我的书籍都摆了进去,我拿起一本史记看了起来,这时,陈稹走了进来,禀报道:“大人,有人在外面求见,我一愣,我刚搬到这里,还没有到吏部登记,怎么会有人来拜访我。”

陈稹见我迷惑,解释道:“大人回来的时候是雇的马车,那个车夫回去之后有人问了大人的住处。”我心想,车船店脚牙,捉住就该杀,果然如此,一边想一边说道:“帖子呢?”

陈稹双手将帖子送上,坦白说,原本陈稹虽然听话,但是我总觉得他对我不大看得起,可是自从我一首词逼死蜀王之后,他的神情就变了,对我必恭必敬。我接过帖子打开一看,上面写着柳飘香三个字,我连忙问道:“那人还在么?”

陈稹答道:“小人已经让他们在门房等候。”我连忙道:“快让他们进来,不,我亲自去迎接。”说着,我连忙赶了出去,到了门房,我看见一个青衣书生,披了玄色披风遮挡了全身,戴着黑纱斗笠,看不清相貌,但是只看她的身材举止,我就不顾他身边两个乔装书童的侍女,冲过去握着他的双手,叫道:“你来了,今天是你在楼上看我么?”

一个侍女冷冷道:“自从状元公出征以来,我家小姐寝食不安,就连画舫也不去了,若非状元公今日回来,小姐还不会出门呢。”

我强忍心中的喜悦,握着柳飘香的纤手,道:“我就知道,我就知道,你也是喜欢我的。”

柳飘香摘下斗笠,露出苍白憔悴的容颜,我呆了一会儿,上前抱着她,道:“卿如此待我,随云粉身碎骨,也不能报答美人恩重。”

柳飘香淡淡道:“你出征之后,我日夜不安,总是担心你的安危,今日见你凯旋回来,我才放心下来,本来不该来见你,只是总想亲自问问你到底如何。”

我感激地道:“其实我想去看你的,只是总想着你未必希望看见我。”

那个侍女笑道:“好了,你们别酸了,奴婢可要累死了。”

我和柳飘香相视一笑,我扶着飘香走了进去,那两个侍女,自然有人照顾的。

深夜良宵,我看着柳飘香慵懒的睡姿,起床拿了纸笔,下笔如流水,这时,柳飘香醒来了,走过来,从后面抱住我,笑道:“状元公又在写诗了。”

我深情地望了她一眼,揽住她的纤腰,将她抱在膝上,让她看到我的新作。

她将秀发拢起,拿起诗稿,却是一首《鹊桥仙》,她低声念道:“纤云弄巧,飞星传恨,银汉迢迢暗度。金风玉露一相逢,便胜却、人间无数。柔情似水,佳期如梦,忍顾鹊桥归路。两情若是久长时,又岂在、朝朝暮暮。”

“啊!”她低声轻呼,然后用炽热的目光看着我,我哪里经得起这样的诱惑,抱着她走向床榻,一夜缠绵。等到第二天我起来,佳人已经不见影踪,我痛心地想,难道她还是不准备嫁给我么,可是她已经不再接客见客了,难道不是想嫁给我么?然后我就看到案头上墨迹尤新的一首小词。

“春日游,杏花吹满头,陌上谁家少年足风流。妾拟将身嫁与,纵被无情弃,不能羞。”

我万分感激的跪在地上祝祷道:“老天保佑,飘香真的愿意嫁给我了。”

什么清白,什么名节,飘香这样的奇女子如果能够嫁给我才是我的幸运,想一想,飘香不会是喜欢名利权势的女子,也不会太喜欢安定的生活,等我想办法离开南楚,就带着她云游天下,让她看看四海风光,美人相伴,游历天下,这样的日子就是神仙也不过如此,等到我们两个都倦了,就留在一个风景迷人的地方终老,这该是多么美好的前景啊。

我匆匆忙忙地赶到吏部,得知国主已经下诏升了我一级官职,我已经是翰林侍讲了,而且国主已经同意我暂时在家养病,办完了各种文书手续之后,我高兴的跑到一家珠宝行,看了半天,都没有中意的首饰,飘香见惯各种珠宝,怎么会喜欢这些俗物,后来我自己设计了样子,让他们为我打造一支金钗,一支金镯,他们看了我的设计图之后,要求可以使用这个样式,但是被我拒绝了,这是我要送给飘香的,怎么可以让他们仿制。不过我倒是答应给他们另外两张设计图,反正赚钱么,只要不传出去,都没有关系。他们十分高兴,说虽然我的设计需要名家精工制作,但是绝对不会误了我的期限。

也难怪他们这么郑重,我这根金钗不是普通的凤头钗,而是真正的凤钗,凤啄垂下的流苏上端,要有三颗三分径晶莹滚圆的珍珠,宝光四射的真正的南海珠。金钗、银珠、翠绿流苏,抢眼的程度是可想而知的,最惊人的是,我要求每颗珠都要由名匠毫刻一只凤凰,细小如粟,栩栩如生,位于珠孔的侧方,如不细心观察,不易发觉。金钗本身,凤嘴的吊环是所谓含环珠转球式的,可以任意八方旋转,这样的一支精美金钗,千金难求,若非赵珏私下里给了我大笔的赏赐,我哪有这个财力。

至于只手镯,我的设计是手镯的主体由十数条细巧的金丝按照螺旋的方式缠绕起来的,金丝上不规则地铸上铃铛,接口的地方是一朵莲花,每一个铃铛上还要雕刻上莲花的图案,这是我对飘香的赞誉,告诉她,在我心中,她仍然是一朵出污泥而不染的莲花。

忙了大半天,快到晚上我才志得意满的带着陈稹回家,刚到家门,却看到飘香的侍女扑到我面前痛哭,我愣住了,不知怎么一阵冰冷的寒意从心底生出,良久,我才听到我用僵硬的声音问道:“发生了什么事情?”

那个侍女哭诉道:“小姐今天早上回去十分高兴,准备遣散奴婢,从良嫁人,谁知艳娘派人来说,有贵客要见小姐,小姐不从,说是从今再也不见客人了,可是艳娘说,来人来头太大,求小姐救命,小姐想这些年来艳娘十分照顾我们,只是见一见,敷衍一下就可以了,等到小姐从良之后,就可以名正言顺的拒绝了。谁知,谁知,小姐一去不回。今天黄昏,突然有人送了小姐的尸体回来,说是小姐急病身亡……”

听到这里,我惨叫一声软倒在地,顿时昏了过去。等我醒来,看到小顺子焦急的容颜,我拉着他问道:“怎么会这样,飘香怎么会死?”

小顺子黯然道:“我将柳姑娘的尸身带了回来,仔细验过了尸体,柳姑娘是被人强暴之后,用阴柔的内力震断心脉而死,虽然做了清洗和掩饰,可是下体的伤痕和内力的痕迹瞒不过我。”

我痛苦的闭上了眼睛,如果飘香不是为了替我守节,何必如此,我继续问道:“是谁,是谁杀了她。”

小顺子道:“我已经查过了,艳娘说是梁婉派人来说有贵客要见柳姑娘,艳娘想梁婉不会为难柳姑娘,,大雍的贵客又不敢得罪,所以才勉强柳姑娘去了。我已经去探过明月楼,没看见什么贵客,不过我抓了他们一个下人拷问,知道,柳姑娘确实是在明月楼被害的,如果我没有看错,可能就是梁婉下的手,我试了试偷袭她,她的内力和柳姑娘的伤势符合。”

我惨然道:“梁婉,好,好。小顺子,扶我去见见飘香。”

我到了一间厢房,里面的棺木里面放着飘香的尸体,我看着她那栩栩如生的容貌,那带着愤怒和遗憾的神情,大哭起来,她真的死了,我心爱的女子,我要娶为妻子的女子,就这样被人杀害。

“梁婉!”我痛声高呼道。

接下来的日子,我如同行尸走肉一般麻木,好生安葬了飘香之后,然后,我真的病了,这一病就是半年,在蜀中留下的病根复发了,后来,我开始重新修炼养生的气功,渐渐的病体好转,容貌回复,只是却总是带着几分悲伤。

我病后不久,听说德亲王赵珏被国主封赐,许他剑履上殿,见君不败,也难怪,德亲王本来就是王叔,又是大都督,此刻真的封无可封了,我坚持着写了一封信,让陈稹送给赵珏,没有多久,赵珏就上表推辞,说自己本来就是王叔,地位已经十分尊荣,没有继续封赏的必要,如果国主觉得有功不赏未免有失国家体面,就请国主多赏些田地金帛,国主果然大喜,赏赐极厚,过了一段时间,德亲王自请镇守荆襄,国主也欣然恩准。

德亲王赵珏到荆襄镇守前,曾经来看过我,见我病重,还特意叮嘱太医院替我治疗,后来他在襄阳还多次送来药物和补品。不过小顺子说赵珏派了人留心我的行动,不必管他,反正我现在天天在床上养病,他不会留意我身边其他人的动静的,至于小顺子的行踪,现在也不是谁都可以发现的。

有一点倒是很令我担心的,国主本来想恢复帝号,不过大臣们都进谏说现在刚刚平蜀,兵力损失很大,还是等一段时间,国主本来很不高兴,后来接到齐王的信才黯然放弃,从此之后国主日夕迷于酒色,尤其迷恋从蜀国得来的一批女乐,在一班伴驾的文人墨客陪伴下,饮酒作乐,作诗填词,还把从蜀中得来的名家字画典籍登册收入崇文殿,除了这一点还比较令我欣赏之外,其他的都是昏君所为,他还把政务都交给丞相尚维钧处理,说什么外有王叔,内有尚丞相,孤可以旦夕宴饮了,在国主的带动下,很多朝臣也越发纵情声色,我派人收集了他们的诗词,都是些艳词,真是惨不忍睹。

南楚这般醉生梦死,大雍也不好过,雍王意欲自立的消息传到太子李安的耳朵里面,李安亲自到雍帝李援面前哭诉,李援诏回雍王,将他置闲,这半年来雍王留在长安,旦夕不宁,数次遭到刺杀暗算。我听到这个消息不久,有一个神秘人拜访了我的住处,他风尘仆仆,自称是雍王的护卫,我接过雍王的书信,上面说,他如今身背谗言,十有八九跟我的计策有关,当初我答应替他参谋,这件事和南楚无关,请问我该如何自保。我微微苦笑,雍王殿下真是会利用一切力量啊,想了一想,我回了一封书信给他,为了安全,我用左手写了一行字,没有抬头和落款

“欲取先予,外有强敌,内无忧患。”

雍王果真是聪明绝顶,后来我听说在雍帝召宴的时候,雍王李贽的酒中被人下毒,李贽饮后吐血不止,若非医圣桑臣恰好身在长安,只怕李贽已经死了,因为此时雍帝大怒,牵连甚广,李安这才收敛,过了不久,又听说北汉寇边,李贽立刻上书要求去抵御北汉,果然得到批准,雍帝也想暂时分开他们兄弟,让他们冷静一下。我知道这个消息,淡淡一笑,这对我来说是一举两得,雍王和北汉必然有数年交锋,太子李安在内掌握军需,必然百般为难李贽,这样就可以牵制大雍,令其无暇南顾,将来我若报仇,有雍王作靠山,只要我手段高明,没有人会特意来为难我。

我在病中的时候,小顺子亲自探察,最后告诉我说,如果要杀梁婉,他可以趁隙刺杀,可是我拒绝了,梁婉虽然罪无可赦,但是害死飘香的还有一个人,让梁婉为之拉皮条,除后患,这个人的身份一定非常特殊,是梁婉绝对不肯透露的,我知道这个女子艳如桃李,却毒如蛇蝎,我就是抓住了她,也不能让她乖乖说出凶手是谁,我必须让她处在一个就是死也不能瞑目的处境,才能迫使她说出实话,所以,现在不能杀她。

梁婉的确是狠毒,飘香死后,我为了掩人耳目,没有声扬,只是让艳娘悄悄的替她安葬,然后又示意陈稹,将飘香的积蓄给了她一部分,其余的都分给了飘香的侍女,安排她们离开建业,到别处生活,这些我都是透过陈稹暗中和艳娘联系的,艳娘知道飘香有了良人,却不知道是我,但见我这样慷慨,自然高兴,等她处理完一切之后,梁婉的杀手果然到了,梁婉派人监视艳娘,看她处理的井井有条,就没有着急下手,等到事情完了,她便派人杀了艳娘,我看她没有派人对付陈稹,确定飘香没有透露自己即将嫁人的事情。小顺子暗中跟着梁婉的杀手,亲眼看到了他向梁婉禀报说,一切线索都已经切断,那些飘香的侍女都已经远走高飞,对于梁婉来说是更好的处理方式,若是一并灭口,不免引人疑窦。

我听到小顺子说到这里,深深的吸了一口气,梁婉,你真的是该死至极,不管你是什么身份背景,我一定要你死无葬身之地。

过了一些日子,我的病情渐渐好转,一天夜里,我在后园里设香案祭祀飘香。想起两番恩爱,不由魂断神伤,默默祝祷道:“卿与我一见钟情,相知相爱,谁知天有不测风云,卿受难陨身,玉碎珠沉,倩影不留,残香难觅,卿若有灵,助我查出真凶,并帮凶梁氏,一并处死,以慰卿泉下冤魂。”

祝祷完,我拿起香案上的一个锦盒,里面是我本来想送给飘香的金钗和镯子,睹物思人,更加惆怅,锦盒里面还放着一枚玉指环,那是飘香被害那日特意找出来的,说是要送给我,飘香其他的首饰,我都作主给了她的侍女,只有这个指环我留了下来,这个指环原本是飘香自己买的,当时喜欢它碧绿的色泽和剔透的质感,只是大了一些,无法戴上,所以一直留在梳妆盒里面。我将指环戴在中指上,这是我心爱之人的遗物。锦盒里面还有两纸诗词,我拿出来,读到“妾拟将身嫁与,纵被无情弃,不能羞。”的时候,终于潸然泪下。

\chapter{第十五章 筹建秘营}

站在远处的陈稹见我伤心,走上前来道:“大人,节哀顺便,若是李爷知道大人这样难过,一定会怪罪属下没有伺候好大人的。”

我看了一眼陈稹,见他眼中带着浓浓的担忧,淡淡道:“你还记恨小顺子和本官么?”

陈稹坦然道:“小人从来没有怨过大人,当初小人身陷缧绁,命在旦夕,如果不是大人相救,小人早就被处死了,小人既是蜀人,大人是南楚官员,担心小人的忠诚也没有什么奇怪,虽然小人开始是有点不安,毕竟生死操之人手,可是这些日子以来,小人从来都能够按期得到解药,没有什么额外的要求和碍难,只要小人尽忠职守,必然不会受害,所以小人再没有怨言。”

我看了他一眼,他倒是精明,继续问道:“我献计连破巴郡、雒城,又逼死蜀王,你也不恨我么?”

陈稹跪倒在地道:“小人在蜀国只是一个谍探,出生入死不过是为了权势富贵,可是直到蜀国灭亡,小人依旧是一个生死由人的谍探,蜀国在时,小人没有背叛,蜀国灭亡,我们这些小人物还是要活命的,大人是南楚臣子,献计破蜀理所当然,小人虽是蜀民,却没有为蜀国复仇的责任,虽然是小人天生无情,但是国家既然没有能力庇佑百姓,也就没有存在的必要。”

我微微一笑,道:“你的性子和我倒是很像,其实南楚也不过是晚灭亡一段时间,到时你会怎么作?”

陈稹道:“我虽然不知道大人和大雍有什么关系,但是相信大人到时可以保全性命,陈稹不才,已经受过亡国之痛,到时只要能够安然度日,陈稹自信不会卖主求荣。”

我摇摇头,这小子倒是聪明,一句委婉的话都不说,应该是看穿了我的个性,如果在蜀国他也这样说话,估计早就没命了。随手取出一颗药丸道:“这是解药,你吃了之后可以解去全部毒性,以后就不用每月服药了。”

陈稹丝毫不犹豫的服下解药道:“属下愿意效忠大人。”

我见他这样爽快,而且胸有成竹,便问道:“你不会早就知道这毒药是我下的吧?”

陈稹笑道:“小人早就知道是大人下的毒,一般用毒的人都会很有自信,若是李爷精于下毒,就不会在我身上另外加上禁制了。”

我心想,这人这么精明,看来我还是坦诚一些好,便说道:“既然如此,我也不妨明言,如果只要一个护卫,只要你必须尽力保住我的性命也就够了,也不需你忠心,毕竟你不过是小顺子的替身罢了,若是想要用你办事,却非得忠诚可信才行。从今以后,本官身处群狼环伺之中,危机重重,动辄丧命,如果不是忠信之人,留也无用,你若不愿,明天我让小顺子解了你的禁制,你就离开吧,如果你真心相从,我必然待你如心腹,待我功成之后,自然会给你一个合适的安排,不至于亏待了你,但也未必会让你飞黄腾达。你意下如何。”

陈稹再拜道:“小人飘零无依,若是离开大人,不过能作些杀人越货的勾当,迟早必然受缚,我见大人凡事举重若轻,必然不会与草木同腐,若是大人不嫌弃,小人情愿为大人效力。”

我将他扶起,暂且相信他吧,我问道:“既然如此,我想问你,目前我们该如何行事。”

陈稹神色有些激动,道:“大人若想为夫人报仇,不管如何行事,都需要手中有一支绝对可以控制的力量,现在除了小人,李爷又不是自由身,力量太过薄弱,如果依赖他人,若是利益冲突,大人难免举止收到限制。”

我轻轻点头,蜀国谍探出身果然名不虚传,现在我至为紧要的就是建立一支属于自己的力量,保护自己,铲除敌人,可是要想建立武力,必须要足够的财力,这该怎么办呢?

接下来的几天,我躲在书房里想着该如何筹建这支力量,又如何维持它的生存,一边信手翻着书,一边胡思乱想,不能让这支力量过于庞大,既容易引起别人的注意,而且也耗费钱粮,又不能太小,起不到作用。最主要的是要有自己的财源。

过了几日,小顺子来了,知道我的想法之后,他建议先从小处开始,我和他将在蜀国得到的金银倾囊而出,秘密买下了离我住处不远的一个庄子,然后找了一些十二三岁的小孩子来训练,按照我的要求和他的想法,这些小孩子基本上都是无父无母,倔强顽强的小孩子,先由陈稹训练他们的基础武技,然后小顺子把我以前给他的一些武技整理之后,做了一个训练武技的计划,照他的说法,如果训练两年左右,就可以让这些小孩子有二流的身手,再加上特意训练他们暗杀刺探的绝技(这是陈稹的专长),那么就可以派上用场了。

我也想到了赚钱的办法,想当初我设计的首饰,不仅图案精美,而且可以由一流的匠人制作出来,所以才得到青睐,我虽然不是特别擅长这些手艺,但是我博览群书,看过很多奇淫技巧方面的书籍,所以我分批设计了很多各种图纸,有的是机关消息,有的是首饰服饰,还有一些精巧的玩物,最受欢迎的就是我改进了日冕利用摆线原理而制成的钟表,这是我读到大食来的书籍,上面提到摆线原理,我费尽心机制作而成的,为了便于匠人制作,我特意重新统一了度量衡等工具,按照图纸和我给的工具,就可以制作钟表,这些图纸,我都是以天机阁的名义找人合作生产,并索取他们利润的一成作为回报。至于出面的人叫寒无计,他是陈稹的同僚,在蜀国灭亡之后侥幸逃出了成都,因为大雍治理地方严密,他为了谋生到了南楚,只擅长杀人暗算,钩心斗角的他几乎没有谋生的能力,几乎贫病而死,当初陈稹奉命四处找寻合适的小孩子接受训练,恰好救了他一命,我见这人还算有骨气,没有作杀手强盗来求生,所以就让他担任实际上不存在的天机阁的总管。让他暗中使用我的设计和人合作,开始还需要他亲自找人合作,后来一有新作出现,他就暗中召开小型聚会,邀请有资格的商人来竞价,胜利者得到图纸等资料,只要保密严谨,可以独家生产。天机阁的名声就在南楚暗中传扬,没有人高声宣扬,毕竟那样就失去了竞价的机会,也就是失去了赚钱的可能。天机阁的请帖不仅成了实力的象征,也成了诚信的象征,因为如果没有良好的信誉,就是实力再强也得不到天机阁的请帖。

开始只是为了赚钱,后来我觉得很有意思,通过天机阁,我可以得到很多机密的情报,为了得到我的图纸设计,很多人愿意用各种机密来交换。当然我让寒无计更加谨慎小心,绝对不能失手,也不能被人跟上,寒无计做的很好,后来我手里的力量渐渐强大,我还特意派了一组十二个人受寒无计调遣,天机阁就这样成了南楚最有名的秘密组织之一。

过了一年多,我看看收益已经足够,就开始减少设计,只是每个月象征性的发出一张,而且只召集已经合作的商行竞价,后来他们那些商人索性组成了天机行会,意味和天机阁合作的行会,想要参加这个行会,必须得到三个推荐人,然后由天机阁同意。天机行会很快就成了南楚势力极强的行会。通过干股我能控制这个行会所有商家的一成利润,第一年我就得到了六十万两银子的收入,这些商行都是信誉良好,资金充足,影响力极强的商行,虽然我不能控制他们的经营,可是失去我会让他们损失的惨不忍睹这一点足可以让他们为我作造反以外的任何事情。

除此之外,我开始加入训练“秘营”的工作,秘营是我给这支将亲自掌握的力量所起的代号,我开始就是教他们读书识字,即使不能写诗作词,也要熟读我精心挑选的诗文典籍,因为我不可能让一群杀手类型的人物留在身边,所以他们必须学会这些礼仪进退、学会扮演可以在我身边出现的各种角色。

经过我和小顺子、陈稹三个人仔细研究讨论,我将秘营分为四组,第一组叫做虎组,这一组善于攻坚破锐,是杀伐的主力,他们既擅长江湖武技,可以搏杀武功高过自己的武士,又可以组成军阵,围杀敌人或者坚守待援,他们可以胜任保镖家将的角色;第二组叫做龙组,这一组人数较少,都是擅长特殊技能的少年,我将胸中所学列出传授,这些人都对某一两门十分感兴趣,而且下苦心专研,我也对他们特别传授,有人擅于占算布阵,有人擅于水底功夫,有人擅于建筑,这些人将来都是可以独当一面,适于单独行动的干才,他们基本上都会被我派出处理不同类型的外务,大多都在寒无计手下充任天机阁的成员;第三组称为暗组,擅于潜踪匿形,行刺暗杀,这一组我基本上不会让他们在我身边出现,只是执行我交代的任务,因为这一组比较没有前途,所以我跟他们约定为我效力十年,十年之内不能有牵挂羁绊,十年之后,他们将得到一笔丰厚的财产,让他们过上正常人的生活,当然那时候他们可以仍然替我效力,只是作一些不大危险的工作,每完成一次任务得到相应的酬金;第四组称为隐组,一个个都是训练有素的暗探卧底,基本上都可以伪装成各种人物探听消息,他们的特长不是武功,而是擅于伪装,擅于探听,完成训练之后,我在秘营里面精心挑选选择了八个人,他们都是各组的佼佼者,又都可以伪装我的仆人,这些人由我直接指挥,既是为了保护我,也是为了随时执行我的命令,为了便于任用,我让他们都姓江,名字依次叫做赤骥、盗骊、白义、逾轮、山子、渠黄、骅骝、绿耳,名字也就是他们的排名,如果有了损失或者汰换,那么顶替他们的人也叫这个名字。

这些孩子虽然年纪还轻,但是武功在小顺子的调教下都有了很大的成就,小顺子虽然不能教他们自己的武技,但是把我整理出来的武技教给他们之后,再和他们过招,这些孩子他们本身都是追求上进而又个性倔强,为了多接小顺子几招都刻苦用功,所以才能达到标准,其中有一些不符合条件的,或者动摇了的孩子,最后都被小顺子废了武功,然后用我提供的药物毁去了记忆。而且是当着所有人的面,因为当初小顺子就跟他们说得很清楚,如果达不到目标的处置方法,然后小顺子暗中安排这些孩子做了伙计等各种稳定的工作。在我和小顺子有计划的培养下,这些孩子只知道忠于我,他们没有对南楚和大雍的归属感,我终于打造了一支属于自己的力量。

力量建立之后,就是使用,我看南楚现在局势还是比较稳定,所以由我计划,由陈稹指挥,这些孩子轮流参加了不同的任务。让他们从稚嫩变得成熟,变得心狠手辣,变得冷静无情,其中最大的两次任务,一次是我的一个合作商行,利欲熏心,想要吞掉我的干股,为了以儆效尤,我让秘营出动,隐组负责收集情报,暗组负责清除商行所雇用的高手和商行的各级管事,而虎组最后雷霆一击,让这个商行上下三百多人死于非命,而龙组奉命用合法的契约,收回了我们应得的一切。这是一次我亲自策划的行动,冷酷无情、计划周密,而效果也很明显,没有人敢在欺骗天机阁,虽然很多无辜的人也死在里面,可是对我来说,他们的死更有威慑力,这样人们在选择得罪我或者背叛我的时候,就会考虑到后果了。

这次行动的最直接后果就是天机阁顺理成章的转入地下,人们不会因为它的神秘而忐忑不安,敢于作下这样的血案,那么天机阁本身就代表着血腥和残忍。期待着从我这里得到利益,惧怕我的报复,那么天机阁这块牌子才会站住脚。

第二次行动是公私两便,大雍的间谍网在南楚朝廷之前注意到了天机阁的价值,梁婉策划了一次行动,派人威胁利诱天机行会的一个商人,利用他进入天机阁的竞价会,想利用合作的机会控制天机阁,不过她太贪心了,这个商人虽然顺利得到了合作的机会,可是他们的试探和跟踪很快就被龙组的成员发觉,然后暗组和隐组布网查出了根源,我得到汇报之后,安排了一次约会,宣称天机阁主会出现,而得到消息的梁婉果然派了得力手下来参加,被我合围诛杀,这次小顺子蒙面出手,将梁婉手下的两个绝顶高手全部击杀,那个商人被我们取消了参加行会的资格,并且逼他交出一年应该分配给我的利润,这样一来,他虽然没有破产,但是失去信用和大量金钱的他很快就一蹶不振了。

我既保护了天机阁的声誉,再次表示出天机阁的超然地位和不受侵犯的决心,又狠狠的打击了梁婉的气焰,真是心满意足。

当我看到梁婉的损失情况,并且小顺子亲自去探听,得知梁婉收到大雍方面的斥责和处罚后,只是冷冷道:“这个女人,她忘记了了自己的职责,她是负责探听南楚军情民心的密谍,不应该擅自发展自己的力量,若非南楚朝廷太愚蠢而又软弱,她早就被捕获杀死了。如果不是我还要留着她的活命,只要一封信给德亲王,赵珏就会安排军方势力将她彻底铲除。

小顺子问道:”大人,你准备什么时候对付她呢?“

我淡淡看向远方,道:”等,时机很快就会来到,大雍已经坐不住了,小顺子,这次行动我们也损失了一些人员,你要加强他们的武功,我也会继续提高他们的才智,我们现在损失不起,我没有另外的一个两年可以浪费了。“

看看手里的情报,那是我派去大雍的隐组成员传回来的情报,”雍王在北汉边关作战顺利,很快就会凯旋“,”齐王勤于练兵“,”大雍兵部正在征兵“,”雍帝重新起用前任水军都督任海妄“,这一切消息虽然琐碎,但是我能够看到很多东西,看看远处天边的阴云,我知道,风暴很快就会来了,虽然这风暴如此猛烈,甚至我也会在其中覆顶,可是我就是拼了性命,也要替她报仇,看看右手中指上面那枚指环,我淡淡笑了。

\chapter{第十六章 大乱将起}

显德二十二年癸酉二月,雍王李贽再败北汉,然北汉主下诏,令威远将军龙庭飞夺情起复,龙惊才绝艳,力挽狂澜,力阻李贽于雁门关,李贽败退,然兵力未大损,同年三月,大雍北汉议和。

四月初,齐王李显南下,陈兵襄阳,时,德亲王珏镇襄阳,大败之。继而朝中有人间曰,德亲王兵权在握,时时练兵意在征北,大雍因而兴兵袭楚,国主信之,诏德亲王回朝,五月初四,齐王再次兴兵犯襄阳,国主大悔,命德亲王星夜兼程,奔赴襄阳。

--《南朝楚史·德亲王珏传》

我负手站在窗前,看看冷冷的月色,小顺子站在我后面,陈稹站在门口。小顺子道:“大人,雍王殿下的书信您准备如何回复,使者还在等着呢。”

我淡淡道:“你替我写回信,就说齐王必然不能取胜,有德亲王在,就是雍王亲来,也不是那么容易攻破德亲王镇守的荆襄的。我是南楚臣子,岂有避难大雍的道理。不过看来大雍即将兴兵,陈稹,你要派人好好监视梁婉,我想他们应该会有所动作。”

这时,门外传来敲门声,我点头示意,陈稹上前开门,一个十四五岁的大孩子走了进来,单膝点地道:“公子,传来急讯,大雍齐王李显进攻襄阳。”

我淡淡一笑,李显还是知道兵法的,荆襄若是落到大雍手里,那么蜀中和江南的联系就会截断,那么大雍就可以对南楚鲸吞蚕食了。不过我相信德亲王的本事,荆襄的防务是很严密的。

接下来的几天,朝中议论纷纷,大雍攻打南楚,让那些大臣又是害怕又是愤怒,有人愤怒的要求向大雍问罪,更多的人却在那里讨论怎么得罪了大雍,甚至有人说,应该立刻上表大雍,表示请罪,请大雍收兵。还是尚维钧这个丞相立场比较坚定,要求派使臣去质问大雍为什么无故相犯,这个提议虽然得到一致同意,满朝文武的心里却更是不安,所以连续几天有人暗暗拜访明月楼,想得到一些保证。这些我都没有阻止,连朝中大臣对南楚都已经失去了信心,我还能做什么呢?

我让小顺子拿出襄阳的兵力布防图,仔细研究,襄阳实际是由襄城和樊城组成,两城隔汉江相望,中间有浮桥相连,两城都是深沟高垒的大城,若是敌人分兵攻击必然减弱力量不能攻破,若是敌人攻击一城,两城士兵可以通过浮桥往来支援,再加上水军保护,所以襄阳易守难攻。当初德亲王到了襄阳之后,派人送了布防图给我,让我参谋一下,我没有明确答复,只是将一种浮桥的设计图给了德亲王,原来的木桥若是损坏很难修复,我授意在河中立起两列木桩,每根木桩都是用数丈大木锤入河底,上面穿以铁链,铺上木板,就是一道可以随时修复的浮桥,另外我又将一种带着铃铛的渔网捎了样品给德亲王,让他在作战时将渔网布在水下,避免水鬼偷袭破坏浮桥。我只是给了德亲王一张图和一张渔网,至于怎么布防是德亲王自己的主意,和我可没有什么相关。看来看去,还是觉得如果德亲王镇守襄阳,是不会轻易失守的,可是齐王难道不知道襄阳的易守难攻么。

四月十四日,齐王李显下令攻城,攻势如火如荼,大雍兵士不惧伤亡,拼死攻城,德亲王下令水军借助汉水用弓箭攻击齐王步兵,迫使他们退兵,齐王二次卷土重来,令人使用投石机逼退水军,大军趁势攻城,日以继夜攻击襄阳北门,德亲王见情势危急,亲率三千骑兵从南门出,袭击雍军侧面,雍军没有料到南楚军敢出城,阵脚大乱,齐王李显下令派出五千精骑迎敌,背赵珏引至东门下以滚木檑石击溃。李显大怒,派两万大军压阵,守住两翼,自己督促八万大军轮流攻击北门,北门岌岌可危,赵珏目不交睫在城上督战,终于在雍军疲惫之际,樊城守军从后偷袭,两方夹攻,李显见损失惨重,不得不退兵,赵珏追击三十里,雍军死伤累累,赵珏方才退兵,双方交战三日,雍军十五万大军死伤六万多人,南楚守军七万,死伤两万,这是一场惨胜。雍军退后,赵珏立刻遣人到朝中报捷,并请求援兵。

此时的朝堂上,赵嘉看着赵珏报捷的表章既是欢喜,又是忧虑,他开口道:“各位卿家,王叔虽然取胜,可是大雍军力胜我十倍,我们该如何是好啊?”

尚维钧禀道:“启禀国主,此次虽然大雍负盟,但我国兵力远逊大雍,不如趁此机会派人向大雍求和吧。”

众人听了纷纷道应该如此,就在这时,有人禀报,说派去大雍的使臣回来了,赵嘉连忙诏他上殿。这个使臣伏玉伦,是显德十六年的探花,现在在礼部任职,他跪禀道:“臣奉旨出使大雍,还未入大雍地界,就被齐王李显阻拦,他声称这次兴兵犯楚,是为了清君之侧,这是齐王给国主的信。”

赵嘉连忙让内侍接了过来,仔细一看,上面写着如下内容。

“大雍齐王拜上南楚国主,此次兴兵,非为别事,德亲王赵珏,狼子野心,坐镇襄阳,厉兵秣马,时时窥视我大雍边境,更有甚者,意图谋夺神器,此人不除,大雍南楚永无宁日,本王与国主郎舅至亲,焉肯加害,如不相信,请诏其还朝,必然推三阻四,不肯应承,昔日承诺,本王牢记在心,惟其权臣势大,一旦国主恢复帝业,那人兴兵作乱,我大雍亦不便插手,若是国主收其兵权,我两国和睦如初,若是国主信其谗言,本王将与国主会猎江南矣。”

赵嘉看了,遍体生寒,若要相信,怀疑其离间君臣,若是不信,自从赵珏攻打蜀国回来之后,屡屡索要军费钱粮,自镇襄阳,不肯回朝,莫非真的是有反意,再想起赵珏声威远胜自己,不由妒忌心起。淡淡道:“王叔取胜,也应该回朝受赏,传孤旨意,诏德亲王回朝。”

远在荆襄的赵珏收到谕旨之后,不肯回朝,上表称军情紧急,暂时不能回朝,原本赵嘉对赵珏的怀疑之心只有一分,见赵珏不肯回来,不由多了几分疑心,连下几道诏书,初时赵珏还以将在外军令有所不受为由而不遵旨,可是赵嘉的诏书言辞越来越锋利,最后,就连朝臣们也起了疑心,无奈之下,赵珏将荆襄防务交给容渊,自己带着一些亲卫返回建业。离建业还有几十里,一个相貌平平的汉子前来拦路,送给赵珏一封书信,赵珏打开一看,却是一行清秀飘逸的字迹。

“君初时不归已是大错,今日来归更是错上加错,唯今之际,不妨回转荆襄,拥兵自重。”

赵珏看了看,叹了口气,将信在火折上烧了,道:“替我谢谢你的主人,告诉他赵珏不是谋反之人。”

那人默然退去。

到了建业,赵珏到宫门求见,却被赵嘉一道诏书下狱了,赵珏虽然上表解释自己不肯回来的原因,但是无济于事,在赵嘉心中,若非担心齐王李显不肯依约退兵,早就将赵珏治罪了。就在赵珏下狱期间,突然有朝臣纷纷上表要求诛戮赵珏,但是赵嘉总算还没有糊涂到那份上,反而将赵珏从狱里放出,暂时软禁起来。

上表要求杀赵珏是我的主意,在我从陈稹那里得到赵珏不肯谋反的口信之后,我就想了这个办法,赵珏是个忠臣,也是一个愚蠢的人,他如果当初立刻回来,赵嘉必然会知道错怪了他,那么赵珏很快就可以回到襄阳去,既然开始没有回来,如今再回来,就显得做贼心虚了,赵嘉就是比较英明的人也不免生疑,更何况我认为赵嘉并不比白痴聪明到哪里去。赵珏被软禁之后,我实在是很为难,按照我的想法,其实如果赵珏就此出不来才好,这样我需要的机会很快就会到来,可是想到赵珏苦苦支撑南楚,却有苦难言的情景,我真的不忍心,就算南楚要灭亡,也应该是让热爱它的人尽力之后。所以我当时就写信给容渊,告诉他让他策动官员上表要求处死赵珏。我派出的使者速度很快,在赵珏刚到建业不久,容渊派来的人就到了,他派人四处挑动那些惧怕大雍的人上表,果然,赵嘉还没有糊涂到家,他对赵珏本来就还有一般信心,见到那么多人要求杀赵珏,反而惊疑起来,赵珏的命是保住了,现在就要看什么时候能够让他回襄阳,这就要靠大雍帮忙了。

果然,没有多久,齐王再次兵犯荆襄,这个齐王真是耐心太差,若是雍王的话,恐怕会多等等再说,容渊总算还能干,稳住了荆襄局势,襄阳的八百里加急文书到来,让国主立刻醒悟过来,连忙派赵珏返回襄阳。赵珏顾不得任何事情,立刻带了亲卫上路。到了城外不久,赵珏就看见一个清秀儒雅的青年坐在十里亭中,亭里的石桌上摆着一壶酒,两个酒盏,在他身后,站着一个相貌平平的中年人,在亭子四角,每处都站着两个十五六岁的半大小厮。赵珏微笑着甩蹬离鞍,在那青年面前深施一礼道:“承蒙随云搭救,赵珏感激不尽,今日又蒙君相送,真是惭愧。”

我站了起来,施礼道:“王爷福德深厚,那些鬼蜮伎俩自然是伤害不到王爷的,王爷此去荆襄,前途遥远,所以随云特来送行。”

一个小厮上前,替我们倒上两杯酒,然后悄然后退,赵珏见这小厮手脚伶俐,相貌俊俏,不由心生好感,道:“随云这几年养尊处优了,这几个仆人一见就知道是大家风范,还多了几分书香气。”

我淡然一笑,举杯道:“劝君更进一杯酒,此去荆襄愿路平。”

赵珏举杯一饮而尽,道:“可惜随云不肯和我去荆襄,若是有随云坐镇,荆襄才万无一失。”

我轻笑道:“王爷这不是低看了容先生么?”

赵珏起身道:“好了,送君千里,终需一别,荆襄军务紧急,我急于赶路,这就告辞了,等到击退雍军,你我再相聚畅饮,若是不幸,就请随云到我坟上祭奠一番吧。”

听到这里,我手里的酒杯几乎滑落,今日我为他起了一课,这两年我渐渐对卜算有了心得,可是今天早晨我沐浴焚香之后,为他起课占算,却得到一个凶卦,有中道夭折的意味,现在听到赵珏的话里有了凶信,更是心寒。赵珏上马正待离去,我突然道:“王爷,我有两个侍从,虽然年幼,但是颇通一些武术,就请他们代随云陪王爷到襄阳吧,也聊表下官不能随行的遗憾。盗骊、白义你们来见过王爷。”赵珏看看上前施礼的两个孩子,苦笑道:“随云,征途劳顿,还是不要为难孩子吧?”

我淡淡道:“他们弓马娴熟,不会误了王爷的行程。”

赵珏本要再劝,见我意思坚决,有时间紧迫,只得挥鞭告辞,纵马而去。

赵珏一路急赶,除了中途换马,就连吃饭和睡觉都在马鞍上,他原本担心江哲派在他身边的两个孩子支撑不住,但是每次看去,都见这两个孩子精神十足的模样,所以赵珏后来就不再担心他们了。眼看还有三百多里的路程,再换一次马应该就可以到襄阳了。赵珏在马上伸伸懒腰道:“好了,前面有座茶棚,我们在这里休息一下,吃顿午饭,然后一鼓作气赶到襄阳,怎么样?”大家都十分高兴,这几天的狂奔,真把他们累坏了,虽然接下来还要赶路,但是能够休息片刻也是好的。

盗骊和白义听到赵珏的吩咐,盗骊抢先下马,几步到了茶棚,吩咐收拾几张桌子都摆上热茶,这个茶棚虽然小,但是还有一些盐水花生之类的小菜,盗骊也让摆上,将那老板支使的团团转,不一会儿就收拾好了座头,白义却是自动去讨了开水铜盆,洗刷干净,从包裹里拿出方巾,等赵珏一坐下,就来服侍他洗脸拂尘,赵珏虽然是王族,但是多年来征战沙场,这些世家的享受早就可望而不可及了。见这一对小厮如此能干,不由心喜,等他坐了下来,喝了一杯热茶,就着盐水花生吃着干粮狼吞虎咽之后,却见盗骊、白义两人已经早早吃完了,正在那里督促老板给马匹上草料。赵珏不由道:“好一对能干的孩子,江状元果然厉害,将一对仆人训练到这样地步。”

他的一个亲卫笑道:“大人若是喜欢,回头跟江大人说一声,要了他们服侍也就是了。”

赵珏虽然知道别说两个仆人小厮,就是爱妾美婢拿来送人也是豪门常事,但还是摇摇头道:“君子不夺人所爱,这两个孩子可不是随便训练出来的。”

众人谈笑片刻,赵珏吩咐上路,就在这时,一个亲卫突然惨叫一声,众人看去,却见一支银箭射穿了他的背心。

众人都是军旅中人,立刻寻找障碍躲避,却听见一声朗笑,一个白衣人从林中缓缓走出,只见这人相貌俊美非常,修伟的身姿在白色武士袍的贴裹下卓然挺立,一张弧度几近完美的银白色大弓侧挂左肩,同色的箭壶斜系腰间,无论是服饰还是弓箭都精美异常,显然它们的主人是个相当考究之人。赵珏心里一寒道:“来得可是银弓浪子端木秋。”

那个白衣人笑道:“小人正是,听说德亲王到此,特来瞻仰,如蒙王爷不弃,请王爷到寒舍小憩。”

赵珏听他言辞温和,但内中含义却是极为傲慢,冷冷道:“本王军务繁忙,不敢拖延,阁下暗箭偷袭,想来是来刺杀本王的了。”

端木秋不屑地道:“本人从来不肯偷袭暗算,否则刚才这一箭就是要了王爷的性命了,至于那个军士不过是本人打个招呼,想来王爷不会见怪。”

赵珏冷冷道:“本王待属下一贯是视若手足,阁下如此轻贱士卒,怪不得没有在大雍军中效力,天下谁不知道金弓长孙,银弓端木,长孙将军在雍王麾下,率军作战,战无不胜,而你银弓端木,只能在江湖中好勇斗狠。”

赵珏的这番话想必刺痛了端木秋的心,他眼中闪过冰凉的杀气,冷冷道:“本人来此,不过是为了防止王爷逃走,如今看来,我不出手是不行了,却不知王爷能逃过本人几箭。”

一个娇纵的声音传来道:“本姑娘敢打包票,你射不死他。”随着声音,一个红衣的美丽女子走了出来,这女子相貌艳丽,但长眉入鬓,满身煞气,却是个女罗刹一般的人物。赵珏不由苦笑道:“原来你也来了,难怪,你们师兄妹本来就是形影不离。”

那女子冷冷道:“德亲王也认得本姑娘,倒是荣幸之至。”

\chapter{第十七章 忠魂渺渺}

王于途中遇刺,至襄阳,负伤苦战,齐王见德亲王归,大沮,乃退,未过旬月,国主诏下,责王作战不力,任雍军退去,王大恸,锥心泣血,夜半乃薨。三军缟素,以祭贤王。

--《南朝楚史·德亲王珏传》

赵珏平静地道:“金弓长孙,娥眉青衫,银弓端木,红妆罗刹,看到银弓在此,就知道火罗刹乔焰儿也必然在此,想不到你们都潜入了我南楚。”

端木秋轻抚弓弦道:“天下谁不知道大雍一统天下是迟早的事情,就是你南楚的武林豪杰不也都基本投靠了我大雍。”

赵珏怒道:“住口。”这本是他心里最为痛恨的事情,大雍崇尚军功,又不计较出身,所以很多南楚的江湖人物都投了大雍,而在南楚若想作高官必须是身世清白,所以南楚军中武力不如大雍远甚。

乔焰儿柳眉倒竖,道:“好大的胆子,竟敢呵斥我等,端木师兄,为我掠阵。”说着拔出背上长剑,如同一团火焰一般扑来,赵珏的亲卫拔刀迎上,这些亲卫都是擅长战阵搏杀的高手,但是乔焰儿却是一流高手,所以虽然是以一对六也毫无惧色,而且乔焰儿攻势如火如荼,不比这些亲卫的威猛刚烈逊色,端木秋的目光紧紧盯着战场,片刻,突然拉弓射箭,一道银影如同鬼魅一般穿透一个亲卫的咽喉。

赵珏眉头一皱,这两个人,一个擅长近身搏杀,一个擅长远攻,配合默契,自己只带了八个亲卫,恐怕会被他们一一杀死,看了看身后的亲卫,低声道:“我们去对付端木秋。”

那个亲卫点点头,两人同时向端木秋奔去,端木秋远远看见,银弓上弦,一弓两箭,又射杀了两个亲卫。这时赵珏已经到了他身前,长剑向他刺去,端木秋展开轻功身法,躲避赵珏的攻势,他的轻功十分玄妙,赵珏和那个亲卫始终伤不到他,但是端木秋也无暇放箭,只能用银弓抵挡,他的银弓是特制的,赵珏的宝剑也无法伤它分毫,众人缠战两处,端木秋的武技其实还不如赵珏,几次想要脱走都被赵珏困住,但是赵珏想要杀他也不能够。但是乔焰儿那一方却大占优势,如果等她杀光了那些亲卫,过来支援端木秋,那么赵珏就再无逃生的可能了,正在赵珏心焦如焚的时候,他眼角的余光看见两个身影,却是盗骊和白义,两人一个手持短剑,另外一个则拿着一具小巧的弩弓,正在悄悄接近乔焰儿,就在赵珏留意到两人的时候,突然盗骊手中的弩弓射出五屡寒芒,乔焰儿反映灵敏,竭力闪开,正在这时,白义手中的短剑雷霆一击,刺向乔焰儿的娇躯,乔焰儿眼中闪过烈焰,手中的长剑仿佛神助一般化作铜墙铁壁,白刃交击,白义踉跄后退,双手都是血迹,而乔焰儿小腹中了一剑,只见她玉面带煞,匆忙点穴止血,口中喊道:“师兄。”然后将长剑射向赵珏,赵珏闪身避开,端木秋趁势冲出,手中银弓连发五箭,将意图杀死乔焰儿的亲卫阻住,又回身一箭逼开赵珏,然后他已经到了乔焰儿身边,一把抱起师妹,飞奔而去。

赵珏送了一口气,看看盗骊和白义,笑道:“多亏你们了。”正在这时,赵珏突然看到众人脸上露出惊骇欲绝的神色,赵珏心思灵敏,立刻向前冲去,但是已经迟了,只觉得一柄利刃刺透了软甲,深深的刺进腰部,这还是因为赵珏及时闪避的原因。赵珏看见那些亲卫飞奔而来,最快的却是盗骊和白义,白义掠过自己身侧,身后传来了一声惨叫,而盗骊扶住自己,从怀中掏出一个蜡丸,捏碎蜡丸,将里面的药丸塞到自己嘴里,赵珏只觉得剧痛方才传来,不由痛呼一声,昏了过去。

等到赵珏醒来,发觉自己躺在茶棚的桌子上,盗骊、白义和其他的亲卫都愁眉苦脸的看着自己,而原本和自己联手对付端木秋的亲卫横尸不远处。他苦笑道:“想不到本王身边就有大雍的探子,他已经跟了我一年多了吧。”

盗骊上前道:“王爷,小人已经替王爷暂时止血包扎,并服下了灵药,一个月内,只要王爷心平气和,应该可以生命无虞,只是王爷伤势太重,如果能够回建业让我家公子亲自诊治,相信半年之内就可以康复。”

赵珏想了一想道:“他们半路行刺,想必是不愿让我回襄阳,我如果不能回去,只怕襄阳有险,还是去襄阳吧。”

一个亲卫苦涩地道:“王爷伤势如此严重,怎能上阵,还是回建业养伤吧。”

赵珏淡淡道:“不必多说,本王岂可惜身而轻社稷,立刻出发,到襄阳。”众人只得听命,盗骊和白义对视一眼,都是满眼的无奈和钦佩。众人还要相劝,赵珏一概不听,盗骊只得将赵珏的刀伤重新包扎裹紧,唯恐赵珏劳顿,众人原本不敢快马加鞭,但是赵珏心急襄阳安危,居然不顾伤势赶路,众人无奈,况且若是不回到军营,难免还会遇到刺客,也只能加快速度。等到赵珏终于进了襄阳城,已经是第二天黄昏,趁着夜色和容渊派出的接应,赵珏顺利的进了襄阳城。盗骊和白义商量了一下,盗骊继续留下来照顾赵珏的伤势,他跟着江哲所学的主要就是医术,虽然还不够高深,但是绝对强过襄阳的军医。而白义则返回建业向江哲复命。

当我从白义口中得知赵珏负伤之后,不由长叹,早在赵珏出征的时候,我就预感到不安,现在赵珏负伤出战,难道我的预感会成为现实么,想想赵珏对我虽然有些猜忌,但总算还是一个好上司,所以我犹豫很久终于决定去襄阳一次。

为了完全,我带了陈稹和赤骥他们七个人随行,我们在城外汇合之后,就尽快的赶向襄阳,因为我马骑的不好,所以弄了一辆马车,虽然颠簸一些,但是总比骑马舒服一些。一路上,我从秘营得到的情报,齐王李显大举攻城,但是赵珏亲自坐镇城上,所以雍军损兵折将,不能取胜,虽然因为无法接近战场,但是我还是得知了大概情况,连日来,雍军在襄阳损兵折将已经达到四万人,我想应该已经到了齐王的极限。

果然等我离襄阳两百多里的时候,我得知了雍军退兵的消息,而且根据秘营的回报,雍军应该是从南楚境内撤退,也就是说,我会迎头碰上雍军,为了逼开他们,我下令暂时留在一个小村子里面等候,据我所知,雍军一路行来秋毫无犯,应该不至于到这里劫掠。当天下午,雍军从村外经过,事前,雍军的前哨到村子里下令各家各户不许出门,我已经换上了青衫布衣,赤骥他们也都换上了农人的装束,所以没有引起什么注意,其实他们又不打算到村子里,所以只要将村子外面道路控制住就可以了。可是就在我等待的时候,突然听到外边一片混乱,接着有人来砸门道:“屋子里面的人出来,这里我军征用了。”

陈稹悄然走到我身边,用眼睛向我询问,我想了一想,微微摇头。陈稹装出慌乱的样子到门前拉开房门,哀求道:“军爷饶命,军爷饶命。”

砸门的是一个身穿黑色铁甲的军士,看装束佩刀不是普通的军卒,他看了屋子里一眼,道:“不用慌,我们用一下屋子,你们到厢房去呆着,不许走动,不许出声。”

我站了起来,带着赤骥向外走去,那个军士突然叫住我道:“你叫什么名字,可有功名?”

我平静地道:“晚生江随云,一介寒儒,没有功名。军爷有什么指教。”

那个军士眼中闪过一丝疑虑,继而突然醒悟过来,喊道:“来人啊,把他们抓起来,他们是奸细。”随着他的喊声,一队军士冲了进来,用刀枪将我们围住,陈稹往后退了一步,挡在我的身前,没有动手,他知道这时候不可冲动。

我用疑惑的神色问道:“军爷为何说晚生是奸细呢?”

那个军士眼中闪过莫名的寒芒,道:“我看你是不见棺材不掉泪,从你的举止气度来看,你绝对是常年养尊处优的人物,还有一种在人之上的气质,若非是奸细,为何说自己没有功名。”

我想不到这军士如此精明,忍不住上下打量了他半天,正想着如何应付现在的情况,一骑铁骑飞奔而来,马上那人喊道:“还没有准备好房子么,殿下急需疗伤之处。”这军士连忙道:“将军,我见这户人家有些可疑……”

话还没有说完,那位将军一眼看到了我,愣了一下,笑道:“我当是谁,原来是江翰林江大人,想不到我们会在这里见面。”我从那位将军一来就在苦笑,只得道:“原来是齐王殿下身边的黄护卫,想不到今日如此相见。”

那位将军正容道:“昔日殿下出使南楚,大人奉命随侍,礼数周到,黄某也十分感激,如今两国交兵,大人是南楚高官,为何在这荒郊小村出现。”

我心里一动,他刚才说殿下需要疗伤之处,便道:“实不相瞒,下官一位故友身患重病,药石罔效,下官颇通岐黄,所以前去为其治病。”

黄将军果然面上露出惊喜的神色,道:“原来江大人擅长岐黄,齐王殿下身中箭伤,军医无法救治,只得快马赶回大雍,如今途中病势加重,就请江大人为殿下看看吧。”

我欣然道:“医家有割股之心,下官敢不从命。”黄将军立刻吩咐人去请齐王殿下到这里来,那些军士露出古怪的神色,我听到他低声问黄将军道:“他是南楚官员,会为殿下真心诊治么?”黄将军也低声道:“当初我们在建业和这位江大人相视,他为人随和洒脱,不会拘泥身份的,殿下说此人胸怀锦绣,不可轻视,对他十分照顾,我想他不会不念旧情,更何况现在他在我军手上,谅他也不敢有什么异动。”

过了没多久,齐王殿下的车驾到了,黄将军等人将齐王抬到房间里,我看他面色火红,昏迷不醒,上前诊脉之后,沉吟一下道:“殿下中了我南楚的毒箭,那是从南蛮得到的毒蛇汁液,若非殿下内力深厚,体魄强健,又及时服下了一些解毒药物,早就不行了,如今是毒性加剧的症状,如果不得医治,三日之内必然不治。”

众人大惊,一个中年将军冷冷道:“既然如此,你可有救治的办法。”

我用询问的眼光看去,那位将军道:“本将军樊文诚,乃是齐王麾下近卫将军。”

我微微一笑道:“将军勿忧,下官来得及时,只要我替殿下针灸一番,再开个药方,定然保住齐王性命,只是今后齐王殿下需要休养半年。”

樊将军和黄将军脸上都露出喜色,我就在他们的监视下,为齐王针灸,让赤骥作助手,我花了一个时辰,完成了金针过穴的复杂过程,然后又开出解毒的方子,他们军中药物居然很齐全,我这个方子又没有什么特殊的药物。很快一服药下去,齐王的面色变得正常,睡得安稳多了。黄将军千恩万谢地送我到厢房休息。陈稹见他们出去了,低声问道:“大人,明天他们会放我们走么?”

我淡淡道:“没关系,我想齐王是个聪明人,如果他不放行,我自有法子取他性命。”

第二天清晨,齐王醒了过来,黄将军立刻召我前去为齐王诊脉。李显躺在床上微笑着看着我,等我宣布他体内毒性已经无碍,只要继续服用我的方子就可以清除余毒之后,李显笑道:“想不到今日陌路相逢,蒙大人救了本王的性命,干脆江大人就跟我回去算了。”

我淡淡道:“齐王殿下此言差矣,下官是南楚臣子,怎能投降大雍,若是殿下不念救命之恩,只管杀了下官就是。”

李显忙道:“大人不用生气,救命之恩焉敢忘记,如果大人不愿意,我不强迫就是。”

我心中一喜,我早就知道李显内心里仰慕雍王,雍王为人重情重义,这种情况下不会为难我,那么只要我这么说,齐王也不会作出忘恩负义的事情,所以我才会没有条件的替齐王治伤。

李显见我消了气,又问道:“我听说江大人是去为一位朋友治病,不知道本王可认识么?”我看出齐王眼中的疑惑,淡淡道:“这人殿下自然认得,就是我南楚德亲王赵珏。”

李显大怒道:“原来你竟是去替他治病,岂有此理,难道你以为本王会让你去治好他么?”

我冷冷道:“大雍南楚交战,德亲王中途遇刺,殿下阵上负伤,我既然为殿下治病,就不担心将来殿下再来犯境,我不知道殿下如此畏惧德亲王,居然要他死于刺客之手。”李显语塞,良久才道:“我料想赵珏就是伤势好了也不能挡我大雍铁骑,罢了,你去给他治伤吧,告诉他,我一定会让他死在我手上。”我微微鞠躬,表示遵命。

三天之后,齐王的伤势基本好转,他才放我离去,直到和我分手的时候,在马车上,他还道:“江大人,南楚迟早亡于我手,到时江大人可以来找我,本王必定保全江大人的身家性命。”我只是默然不语,至于他当我是默认还是不满,就随他了。

和齐王分手之后,我连夜赶路,终于到了襄阳,白义和守城门的将军认得,很快我就进城直奔德亲王的住处。可是我刚刚到了门前,就听到里面传来痛哭的声音,我愣住了,然后疯了一般冲进去。那些守门的军士基本上都认得我,等我冲进德亲王的卧室,看见容渊伏在地上嚎啕大哭,而在床上正是面白如纸的赵珏,盗骊站在一旁,面色悲凄,他们见我进来,容渊哽咽道:“随云,你来迟了。”

我失态地喊道:“盗骊,怎么会这样,你怎么会保不住他的性命。”

盗骊匍匐上前道:“大人,小人替亲王用药,效果明显,虽然亲王连日来上城督战,但是伤势没有过于恶化,谁知道,今日国主下诏斥责王爷,王爷见了诏书,气怒攻心,连连吐血,不到半个时辰,就,就去了。”

容渊垂泪上前道:“随云,你不要怪他,他已经尽了力。”

我冷冷道:“诏书在哪里?”

容渊长叹一声,指了指旁边的桌子,我走上前拿起黄绫诏书一看,只觉得胸口郁闷,口中一甜,哇的吐了一口鲜血。只见那诏书上面写着冰冷的言辞。

“王叔深明兵法,既统十万精兵,又据襄阳天险,为何久战无功,任雍军往来自如,莫非有通敌之事,望大都督体谅此国力疲敝之秋,速战速决。”

我放下诏书,推开要扶住我的赤骥和盗骊,看向桌子,上面摆着一份表章,我打开表章,容渊想要过来阻止我,却又站住了,我低头看去,上边字迹工整,却好几处有溅上的血迹。

“珏以王室之尊,庸碌之才,受知先帝,委任腹心,统率兵马,敢不尽心竭力,奈何微躯多病,大志未申,中道而陨,遗恨何极。方今大雍肆虐,南楚疲敝,此诚存亡之秋也,珏今将死,敢不忠言直谏,我南楚自和亲以来,朝臣每仰大雍鼻息,惧战求和,然虎狼之心,焉肯轻息,国主应亲贤臣,远小人,疏后宫,勤于政务,专心军事,远连北汉,近拒大雍,孰几可保南楚社稷平安。襄阳防务,至关紧要,容渊者,臣之腹心,多才干,精军务,珏之旧部,可归此人统领,请王命,诏此人代守襄阳,则可保襄阳无事,镇远侯陆信为人忠烈,临事不苟,可代大都督之职,珏临表涕零,不知所言,倘蒙垂鉴,珏死不朽矣。”

我放下表章,想到赵珏满怀悲痛,锥心泣血写这份表章的情景,泪落如雨,道:“王爷为何如此固执,若是当初听我一言,拥兵自重,何有今日。”

容渊上前道:“王爷临终,念及大人,曾想推荐大人镇守襄阳,但是思之再三,说道:‘随云雅量高致,天下无双,奈何明哲保身,必不肯以身相殉,容先生代我转告随云,若日后南楚灭绝,望先生看在珏之面上,为我南楚留一脉香烟。‘”

我默然良久,淡淡道道:“容先生尚请节哀,国主非是无情之人,见王爷表章,定会悔恨,先生镇守襄阳之事,应该可以办到,随云心灰意冷,即将辞官远行,他日相见,再叙别衷。”

说罢,我转身离去,到了门前,我掀开车帘上车的时候,听见远远传来炮声动地惊天,炮响十二记,主帅殁于军中。放下车帘,我淡淡道:“起程。”马车跑了起来,良久,我推开车窗,看看外面阴沉的天色,第一次真切的感觉到,南楚,完了。

\chapter{第十八章 南楚称帝}

显德二十二年七月,大雍遣使求和,南楚君臣颇畏征战,许之,未几朝野有人,称颂国主圣明,破蜀中,拒大雍,应晋帝位,国主惑于言辞,又信齐王当日所言,遂许之,于八月一日晋帝位,上表大雍,愿为兄弟之国。时,朝中明智之士上表劝谏,国主愤怒,贬斥极多,江哲亦在其中。先,江哲上《谏晋帝位书》,词深意切,语气激昂,痛斥国主之非,国主大怒,欲斩之,内侍劝曰:“江哲乃南楚才子之冠,不可轻易加刑。”国主乃息怒,诏曰:“迫令致仕,永不叙用。”江哲接旨,或劝之暂且隐忍,待日后相机劝国主收回成命,江哲唯言:“雷霆雨露,皆是君恩。”从容而退,人皆敬之。

--《南朝楚史·江随云传》

看我神色冰冷,陈稹一副欲言又止的神情,我淡淡道:“你想说什么?”

陈稹犹豫一下道:“大人,您与大雍颇有联系,但是为何又对德亲王的事情如此伤情?”

我沉默了良久,才道:“大雍人才鼎盛,军力强盛,又有明君贤臣,可以说天下一统的契机就在于大雍的发展,我南楚虽然人杰地灵,但是修于文略,疏忽武事,江南之人又多文弱,流弊难以革除,我从一开始就知道南楚必然亡于大雍,只是时间早晚的问题,所以我当初参加科考,并没有为南楚呕心沥血的打算,我一介寒生,在南楚根本不可能掌握权柄,就是我能够到了一人之下万人之上的地位,南楚也不是我可以大展宏图的地方,更何况我有自知之明,我文不能安邦,武不能定国,我所擅长的是出谋划策,决胜千里,如果没有明君贤臣,我也发挥不出什么作用,可是我终究是南楚人,让我看着南楚这样衰亡,我又不甘心,当初见到德亲王,我希望他能够是我心中的明君,可惜不是,他是个忠臣,不是枭雄,善善而不能用,恶恶而不能去,所以锥心泣血,殁于军中。大雍之人,我见过雍王、齐王,雍王殿下乃是王者风范,必然是一代圣主,齐王殿下虽然有些鲁莽,但是也是霸王之才,我没见过太子李安,但是想来能和雍王抗拒良久,那么也非同凡响。我是一个普通人,所以对雍王和齐王我始终不愿得罪,就是为了日后可以保全性命。”

陈稹道:“大人曾对德亲王和雍王分别献策,又是为了什么?”

我淡淡道:“这些我本来不需要对你讲,可是你既然甘心为我效命,那么我也不妨直言,我为德亲王献策,如今已经达到目的,破城之策就不必说了,离间之策如今已经见到效果,你以为这次为什么会是齐王攻打南楚。”

陈稹想了一想,道:“定是太子李安担心雍王功劳太大,无法控制。”

我闭上眼睛,道:“不错,当初我逼死蜀王,为的不是让大雍无法更好的统治东川,而是为了王妃金莲夫人,果然金莲夫人到了大雍,雍帝李援喜爱她的美色,将其纳入后宫,如果蜀王还在,雍帝必然不能如此做,雍王也不会因此直谏遭怒,否则你以为为什么接下来雍王会因为太子的攻击而狼狈不堪。”

陈稹疑惑地道:“可是没有听说过雍王进谏啊。”

我笑道:“这种事情,雍王怎么会当面进谏,可惜就是暗中的劝谏也不免遭到父皇的白眼。接下来的事情德亲王就不知道了,雍王派使者来求教,我让他假意中毒,拥兵边关,虽然保住了雍王的地位和安全,可是也让他更进一步的和父兄离心,这才是我离间策的全部内容。”

陈稹惊讶的看着我,道:“属下没有想到大人抱病替雍王谋划会是这个原因。”

我摇头道:“你也别太敬佩我,其实雍王和雍帝、太子之间的矛盾本来已经很尖锐,我只不过火上加油,而且雍王现在的困境对他来说并非是没有好处,等到雍王下定决心夺得皇位,那么大雍统一就是不可阻挡的了,我的所作所为不过是替南楚争取一些时间,如果南楚强大,那么雍王就不得不放慢脚步,南楚苟安上二三十年也不是不可能的,可是如今国主自毁长城,德亲王死后,南楚再也没有可以对抗大雍的将领了,容渊此人气量不足,陆信此人,愚忠而谋略欠缺,朝中重臣个个目光短浅,稍有才干者,不是沉迷酒色,就是息隐田园,陈稹,就是大雍内乱,我想南楚数年之内也会社稷不保了,但是也因为大雍内乱,我料南楚仍然会残余部分势力,在江南蜀中割据,大雍若想江南安康,没有十年以上的时间,是不可能得了。”

陈稹记下我的话,问道:“那么,大人我们下一步应该干什么呢?”

我淡淡道:“南楚再无可为,我回去之后会立即辞官,然后我们在建业等待,我想不久之后,我报仇的机会就到了。”

陈稹问道:“若是报仇之后呢,雍王和齐王对大人都十分器重,若是南楚灭亡,两位殿下恐怕都会来招揽大人,到时大人如何处置。”

我默然,然后道:“我曾以为自己会愿意投靠大雍,可是我发觉不行,南楚灭亡之后,我自然希望可以安度余生,如果雍王和齐王不肯放过我,那么我只好远离中原,如果不幸被他们所擒,我也不会为南楚殉葬,等到我报仇之后,我会将身边的势力暂时交给你掌握,对于大雍来说,我的势力太渺小,如果在我身边,只是会被注意,甚至遭到覆顶之灾,如果隐藏在暗处,或许还能救我一命。”

陈稹犹豫了一下道:“大人不如让李爷统领他们吧。”

我摇头道:“小顺子在我身边的用处更大,他武功高强,心思细密,是我的心腹益友,他若在外,反而会不够冷静,不能好好隐藏力量。”

陈稹心悦诚服的点头道:“既然如此,属下遵命。”

黯然的回到建业,我得知国主果然后悔,接纳了德亲王的遗表,封容渊为兵部侍郎,镇守襄阳,委任陆信为大都督,陆信回朝领受节钺的时候,我看到他风采不减当年,他的儿子,小侯爷陆灿,我的学生,已经是二十一岁的雄壮少年,我听说这些年来,陆灿已经成了陆信手下的先锋,作战勇敢,富于谋略,在南楚军中颇受好评。我回到家中不久,陆灿前来拜访,我毕竟曾经是他的老师。陆灿兴奋的对我说,我当年闲着无聊给他讲的兵法让他受益匪浅,他这次来想问我愿不愿意继续教他兵法,我看着他热情洋溢的表情,只能淡淡道:“当年我不过是纸上谈兵罢了,小侯爷还是多多向侯爷请教才是道理。”送走了陆灿,我心中一阵苦痛,这个当年在我面前受教的学生已经成了南楚的将领,想到不久之后他将面对的一切,我岂能不难过,想了很久,我把我整理的一些军阵让赤骥送去给陆灿,嘱咐他不要对外人说起,或许我的军阵能够让他在战场上多胜利几次,虽然最终结果可能只是多死一些人,但是这是他的命运,也是南楚的命运,这,也是我对南楚贡献的最后一点心力吧?

不久,有人上书说国主英明神武,在位数年,先破蜀国,今次又击退雍军,论其德能,应该晋位皇帝,和大雍分庭抗礼,赵嘉的耳朵太软,听了之后,居然也相信自己是天命所归,忘记了被他迫死的德亲王血泪斑斑的遗表,很快就下诏征询朝臣的意见,结果迷惑于胜利的朝臣大多附和,还纷纷上劝进表。

我听了之后,本来想先去辞官的我,沉思良久,写了一道表章《谏晋帝位书》,这份表章一递上去,国主果然大怒,我这份表章里面,明确的说明了当初攻打蜀国虽然取胜,可是大雍所得利益在我国之上,而且两国军队的强弱也十分明显,我也提到这次击退大雍不过是因为齐王领军作战过于强硬,襄阳又很坚固,如今德亲王殁于军中,我南楚再没有可以与之相提并论的将领,而大雍根基没有受到损害,如果国主称帝,那么大雍就可以以属国背叛的理由来攻打南楚,到时南楚理亏,只怕难以抵挡大雍的攻势。这份表章,我罕见的写出了自己真正的看法,因为这是我离开南楚前的最后一份表章,如果国主真的肯接纳,那么我宁愿将我的所有才智都献给南楚,即使死在战场上也不会后悔。

可惜,我预料的事情还是发生了,国主大怒,差点要立刻传旨将我斩首,总算我事先通过小顺子收买的内侍劝解得当,我被免去了官职。原本我是想正式辞官的,可是最后我上了这份表章做最后的赌博,果然我被免了官,这样,我和南楚再没有什么纠葛,恩怨两消了。当我神色淡然的听着来传旨的官员念诵的时候,我几乎想要笑出来,这样一来,大雍应该没有什么冠冕堂皇的理由来加罪我,也就不能用赦免我的理由让我归降了。传旨的是跟我同科的榜眼刘魁,他现在在国主身边听命,这份诏书就是他替国主书写的。满怀遗憾的,刘魁道:“江年兄,你不用消沉,国主虽然说永不叙用,等过几年事情淡了,我们为你进言,江年兄一片赤诚,为的是南楚社稷,到时国主必然会重新起用。”

我没有理会他的安慰,只是淡淡道:“雷霆雨露,都是君恩,下官怎敢有丝毫埋怨,前几年我从军蜀中,结果落下了病根,这几年一直在家养病,本来就不应该尸位素餐。”送走了客人,我淡淡道:“走吧,我们回家去。”

我带着陈稹等人还没有走出吏部的大门,就看见梁婉在一辆马车上向我示意,陈稹看看我阴沉的面色,低声道:“大人,不,公子,你别忘了……”

我拦住他的话,走上前去道:“原来是梁小姐,不知道有什么吩咐。”

梁婉笑道:“这里不好谈话,请状元上车一谈。”

我微笑着上车,对梁婉道:“也好,请小姐送我一程吧,到北门就可以了”

梁婉等我上了车,吩咐上路,笑着问道:“状元郎这次直言进谏,却落到这种下场,真是可怜,当初比干剖心,子胥沉江,虽是忠臣,却为天下所笑,都只为所事非人,如果状元郎不嫌弃,我在大雍颇有相识,愿意推荐大人到大雍任官。”

我微微一笑,道:“小姐如今是南楚王后心腹,又是先王义女,不为南楚费心,却为大雍效力,未免是有些心口不一。”

梁婉鄙夷地道:“谁希罕南楚的权位,状元郎聪明过人,齐王殿下多次赞颂,如果肯改弦易辙,想必是青云直上,前途不可限量。”

我微笑不语,左手一直转动着右手中指上的玉指环,那是我爱妻的遗物,良久才道:“小姐在南楚多年,虽然功勋卓著,不过是仗着大雍势力,如今南楚大雍绝交在即,到时候还请小姐珍重才是。”说罢,我吩咐停车,下车之前,我淡淡道:“临别忠言,还请小姐勿怪。”

梁婉迷惑的看着我离去,她不明白我为什么既不肯投降,又要劝她小心,想了半天,心道,莫非是他待价而沽,罢了,等到我大雍渡江之后,还怕你不投降么,便下令继续前行。

我下车之后,回忆着刚才近在咫尺的花容月貌,心里涌起一阵厌恶,这样的女子,真是应该碎尸万端,我想如果大雍真的只靠她统领江南密谍,那么我倒要怀疑大雍中人的智慧了。不过想到近年来的传言,都说梁婉不嫁是因为和国主有染,这次国主称帝,据我所知,梁婉的暗中运作,是不可缺少的因素,她确实是一个出色的间谍,收买朝臣,散布流言,我没有阻止她的行动,现在国主把她的话当成纶音,这么说来,大雍用人还是会看对手的,所以在我南楚的密谍首领,就用了这么一个美丽的女间谍。

在我之后,还有很多人进谏劝阻称帝,都被国主置之不理,例如翰林院掌院学士谢贤,谏议大夫罗大人,下场却是贬官的贬官,斥退的斥退,罗大人最后以死相谏,碧血染御阶,可惜国主没有醒悟,这些风波我都没有参与,我现在只是一个庶民罢了。

就这样八月一日,国主正式称帝,宣布改元至化,我想起当日国主继位的时候下令沿用显德年号,我还觉得奇怪,搞不好国主就是想称帝之后再用新年号,这样看来,国主还是有雄心壮志的,可惜志大才疏,没有恒心,这个至化年号,只怕会是个亡国的年号吧。

与此同时,大雍境内,雍王府,李贽看着手上的情报,道:“梁婉太嚣张了,她不知道谨言慎行的好处,如果不是因为她的师门,我绝不会这么纵容她。”

坐在他身边的一个相貌斯文,留着黑髯的中年人道:“殿下,凤仪门乃是大雍白道领袖,在大雍立国期间功劳卓著,现在她们的手伸得太长了,梁婉效命殿下,在南楚行事,却屡屡自作主张,还和太子、齐王的人走得很近,而齐王的准王妃秦铮更是梁婉的师妹,我怀疑她们准备支持太子继位。”

雍王冷冷道:“不用怀疑,我已经得到情报,凤仪门通过她们的弟子,父皇的宠妃纪贵妃向父皇进言,说我拥兵自重,若是继位,必然弑兄杀弟,而太子虽然才干稍差,但只要派贤臣辅佐,能够更好的治理天下,哼,不过是因为我不肯接纳她们的弟子做王妃罢了,一群女人,妄想控制天下,我李贽可不是木偶泥塑。”

中年人忧心忡忡地道:“可是凤仪门势力强大,若是极力阻挠殿下登基,那可怎么办呢?唉,属下不擅长策划,不能为殿下分忧。”

李贽目光一闪,道:“若是那人肯归我麾下,必然可以对付凤仪门,其实我并不惧怕凤仪门的武功,她们虽然武功高强,可是我已经结好了少林那些名门正派,至少可以避免凤仪门使用武力,我担忧的是她们长袖善舞,擅于挑拨离间,如果不能善用计策,让她们继续发展下去,我恐怕大雍江山落于妇人之手。”

中年人道:“总听殿下说起那江哲,属下十分渴望一见,只是殿下有把握让他效命么。”

李贽苦笑道:“怎么说呢,让他在我手下为官倒是并非很难,但是若要他忠心效命就难了,这人心思莫测,而且对荣华富贵、社稷民生都不甚关注,这样一个冷淡的人,我如何能让他倾心相投呢。我收到情报,他上表进谏,被南楚免官,看到他的表章,令我心惊,他对南楚大雍局势了如指掌,这样的人物,若是不能为我所用,真是李贽平生遗憾。”

中年人接过李贽递给他的表章,看了良久,抬头道:“殿下,你必须立刻派人去南楚,如果不能得到此人,我们大业难成,而且凤仪门不是瞎子,她们若是看到此人才干,必然会招揽他,他如果成了太子的幕僚,我们危矣。”

李贽微笑道:“我相信凤仪门没这个本事让他心悦诚服,凤仪门擅长的那套‘为国为民‘的表演感动不了他,李安也没有让他降服的可能,倒是齐王很有可能让他归顺,这次齐王传来密信,谈及在南楚遇到江哲,江哲救了他的性命,齐王虽然鲁莽,但是待人却是热诚,若是江哲随了他,齐王必定言听计从,那才是我们的一大危机,现在齐王养病,我已经禀明父皇,立刻攻打南楚,只要我先破楚,那么江哲必然落到我手。子攸,我们的确应该派人去南楚,不是为了说降,而是为了掌握江哲的行踪,想要说降,除了本王之外,无人可以成功。”

这时,门外的侍卫高声求见,进来之后,跪禀道:“殿下,陛下诏殿下入宫,商议伐楚之事。”

\chapter{第十九章 伐楚之策}

李贽随着内侍入宫,他前几日上书要求伐楚,但是没有回音,今日父皇终于召他入宫,不免有些喜出望外。议事是在御书房举行的,雍帝李援坐在龙书案之后,微眯着眼睛,神色不豫,而在他身侧坐着一个美丽出尘的宫装少妇,在书案左侧的椅子上依次坐着太子李安、丞相韦观、魏国公程殊,右侧除了第一个位子之外,坐着抚远大将军秦彝、齐王李显,李显仍然是面色苍白,有些病恹恹的,可是精神倒还不错。

太子李安,今年三十六岁,比雍王李贽大两岁,不过他没有练过武功,不像李贽这样英姿焕发,虽然因为保养的不错,看起来倒还不是很老,可是眉宇间总带着一丝疲惫,他看着从外面进来的李贽那种令人倾倒的英姿,眼中闪过一丝嫉妒。李贽径自走到龙书案前,拜倒在地道:“儿臣叩见父皇。”李援道:“贽儿,怎么来得这么晚?”李贽笑道:“儿臣来之前刚刚收到江南的谍报,所以整理了一下拿过来,好让父皇看看。”

李援奇怪的看了看李安道:“安儿,江南谍报你不是已经递上来了么?”

李安笑道:“想必是二弟还不知道,江南的谍报已经先到了我这里。”

李贽眼中闪过一丝冰冷的笑意,道:“太子殿下那里的江南谍报是梁婉传来的,儿臣这里的谍报渠道不同,所以想必有些父皇还不知道的事情。”

李安神色一凛,他千方百计将江南谍报网控制在手里,想不到李贽仍然另有情报,怎不令他嫉恨,冷冷道:“原来如此,前些日子,六弟进攻襄阳,如果二弟将那些情报也拿出来,想必六弟不会败得如此之惨吧。”他只顾自己快意和打击李贽,却忘了李显的心情,李显眼中闪过一丝阴蠡。

李贽不慌不忙地道:“臣弟是在六弟第一次攻打襄阳失利的时候才发觉我们在江南的谍报网还不完全,我们得到的襄阳军力布防图十分粗略,必然是襄阳守将在上呈兵部的时候做了手脚,可见梁小姐负责的谍报网已经被南楚有识之士留意,只是碍于南楚君臣的维护,才不敢清除他们,这样一来,等到我们正式和南楚开战,我们的谍报网必然会被摧毁,碍于这种情况,臣弟不得不重新布线,总算是颇有成绩,太子殿下不知详情,并非是臣弟阻拦,只是新的布线刚刚有了成效,所以没有及时支持六弟。”说得这里,李贽看了李显一眼,微微欠身表示歉意。李显微微摇头表示不介意。

从李贽一进来,就和太子李安唇枪舌剑,见他们暂时停止,除了雍帝、那位少妇和李安之外,其他人都纷纷站起来向李贽见礼,齐王李显本要站起来,却看到李安眼中的怒色,便又坐了回去。李贽坐到自己的位子上,向各人一一致意。那宫装美妇从李贽指责梁婉的时候就眼神如冰,等到李贽坐下之后,她开口道:“听殿下的意思,我婉师侄在江南含辛茹苦,居然还落了不是么?”见她开口,李安微微低头,嘴角带笑。

李贽欠身道:“贵妃娘娘,儿臣不敢妄自菲薄梁姑娘的功绩,当年长乐远嫁,父皇和我们都怜惜长乐,她的性子又是温和柔婉,所以贵妃娘娘派梁姑娘随长乐赴南楚,李贽也感激不尽,这些年来,我们在南楚如此顺利,梁姑娘功劳非浅,只是如今形势变化,梁姑娘几乎已经摆在了明处,所以儿臣不得不另外建立谍报网,免得梁姑娘被迫撤退之后,我们失去对江南的控制。”

少妇清艳的娇靥上露出淡淡的笑容,似乎接受了李贽的解释,那宛若雪后梅花的笑颜让书房里面的所有男人都不由心里一动,但是她既然是贵妃的身份,所以很快就都移开了目光。

李贽见气氛好转了,道:“父皇既然已经得到了太子殿下带来的谍报,想必是见过那份《谏晋帝位书》了?”李援从书案上拿起一份抄稿,道:“是啊,这个江哲果然才干不凡,太子和齐王都向我举荐过这个人,我见过他的诗词,尤其是那首破阵子,一曲小词,逼死蜀王,真是才华绝世,今天见了这份折子,我才相信这个人不仅仅是个才子,还是一个能臣,如果南楚重用了此人,可是大雍之祸,如今此人被免官,想必可以被招揽过来。”

李贽微笑道:“父皇说得是,此人才干的确不凡,儿臣在蜀中,六弟在南楚都见过他,可惜此人淡薄名利,又是南楚忠臣,只怕不肯归顺吧?”

李援点头道:“是啊,本王也忧虑这一点,见此人的表章,应该是南楚的忠臣,只是俗话说,贤臣择主而侍,我见此人诗词洒脱,应该不是固执之人吧?”

李贽听到他说到这里,知道李显没有把自己在襄阳遇到江哲的事情说给李安听,所以李援就不会认为江哲可能不会归顺,他看了李显一眼,李显神色有些不安,李贽微微一笑,继续道:“是啊,我这次因为得到江哲的表章,所以仔细查了一查,发觉此人和德亲王赵珏关系密切,在蜀中,他就为赵珏参赞,据说这两年多他在家养病,但是和襄阳书信不断,这次梁婉派人途中行刺,救了赵珏的正是他派去的仆人,而且还亲自到襄阳见了赵珏最后一面,儿臣又查到新任南楚大都督陆信和江哲也相识,当年江哲没有及第之前,曾是陆信之子陆灿的西席,所以儿臣想此人恐怕不会轻易归顺。”

李援听得津津有味,而李安和纪贵妃则暗中交换了一个眼神,看来他们本来对江哲并没有那么重视。李援看向韦观,问道:“韦相,你看呢?”

韦观答道:“陛下不必忧心,如今南楚疲惫,平定南方不过数年之事,到时候四海升平,贤士自然来归,江哲此人,看他的诗文不是固执之人,焉能不奉正朔。”

李援听了他的回答,不由开颜道:“韦相说得是,此人虽然值得重用,却不必太费心,等到南楚平定之后,朕诏他入朝为官就是。”

李贽看了看众人,发觉李安和纪贵妃眼中都是淡淡的神色,只有李显却是满眼讥诮,知道自己的目的已经达到,他在众人面前推崇江哲,正是为了隐藏自己对他的重视,想要暗中安排拉拢收纳江哲,是很难避开李安等人的注意的,倒不如摆明车马,表示对其的重视,那么其他人的目光就会集中在江哲表面的才华,反而不会真正了解江哲的重要性,也不会为了一个“普通”的名士和自己作对,能够看穿自己的计谋的只有李显,他同样了解江哲的才能,但是想必他也希望将江哲收归帐下,为了这个缘故,他决不会揭穿自己的所作所为,接下来,他就只需要和李显暗中争夺就可以了。

达到目的的李贽开怀地道:“父皇诏儿臣来商议伐楚的事情,不知道父皇有什么打算?”

李援道:“这次大雍在襄阳损兵折将,我担心南楚从此不受控制,准备派你领兵伐楚,如今南楚国主称帝,正好给了我们最好的借口,上次我们借口德亲王居心不良准备对我大雍不利,借口太牵强,现在我们伐楚理所当然,贽儿以为如何?”

李贽道:“父皇说得是,如今南楚军方混乱得很,按照儿臣本来的计划,应该大军重围,隔断荆襄和江南之间的联系,花上几年的时间,慢慢的消耗南楚军力民心,可是现在看来如果给了他们时间,他们的军队重新稳定下来,没有十几年的时间,就不可能攻下南楚,如果父皇允许,儿臣想要冒一个险,给南楚一个重击,让他们失去和我大雍对抗的决心,然后再一一平定反抗势力,虽然这样一来可能会旷日持久,但是在三年之内,儿臣可以保证将南楚收归大雍版图,然后再花上二十年的时间慢慢收复民心,父皇以为如何。”

李援听出了李贽的意思,按照他的想法,最完美的自然是将南楚一举荡平,但是如今看来南楚仍有可为,想到可以在三年之内将南楚征服,虽然代价是几十年的动荡不安,但是应该不会影响中原局势,而且到时候自己也已经不必操心了,建立功业的欲望超过了一切,他同意了李贽的意见。

纪贵妃眼中闪过一丝阴蠡,她知道这样一来,江南就会有多年的纷乱,黎民受苦,但是她没有阻止,因为她知道李援已经决定了,她再次认定,门主的决定是对的,雍王虽然雄才大略,但是比较起来,平庸的李安更加适合作大雍之主。

看李援已经同意,李贽提出了详细的计划,根据情报,现在南楚的军力分散,因为和大雍作战,南楚加强了在蜀中的防御,避免大雍突破蜀中,顺江而下,而襄阳两次收到攻击,兵员损失惨重,为了补充兵员几乎南楚兵部几乎捉襟见肘,还有漫长的长江防线,可以说南楚现在是外强中干的情势。李贽提出,首先从蜀中、襄阳两处展开攻击围困,让南楚专心两处战事,然后他自带一支精骑突破长江,进逼建业,按照常理,建业城没有几个月是攻不下来的,几个月的时间,足够南楚军断李贽后路,勤王建业的了,但是现在建业空虚,再利用大雍在建业的内应,李贽有自信可以在数日之内攻陷建业,然后将南楚王族和百官劫掠到大雍,到时南楚群龙无首,何况连都城都被攻破,国主都被俘虏,足可以大大打击南楚的士气,就算他们另外立了国主,也难以再和大雍对抗,然后大雍就可以以赵嘉的名义荡平江南。这个计划虽然要在实际上完全统治江南花的时间会多些,而且后患也会多些,但是李援更希望早些让南楚称臣,所以还是同意了这个计划。

李安虽然对军事不是很精通,但也知道这样的后患,但是想到如果真让李贽完全攻占了南楚,那么自己的储位怕是怎么也保不住了,李显这次进攻南楚失利,心想这样一来以后还有挽回掩面的可能,所以两人都没有反对,虽然魏国公程殊和抚远大将军秦彝都有些不赞同,但是他们也都了解其中的奥妙,知道反对也没有用,就这样,这么一个令后世诟病的不符合兵法的攻楚计划就这样通过了。除了李贽和石彧之外,没有人知道李贽最大的目的,就是为了得到江哲一个人呢。

众人商议已定,李援叹息道:“贽儿,这次你攻打建业,必须要保证长乐的安全,一定要把她安全带回来,为了大雍,她已经牺牲太多,朕对不起她啊。”

李贽微微叹息,长乐公主是父皇爱女,母亲长孙贵妃以贤德著称,长乐本人端庄温柔,所以长乐最受父皇宠爱,当初长孙贵妃所生的皇四子李贤为了保护李援而被刺客所杀,皇七子李晋又年幼夭折,所以父皇为了安慰长孙贵妃,答应长乐公主及笈之后可以自己选婿,而长乐公主已经有了心仪之人,父皇也同意为她赐婚,可是因为想要结好南楚,父皇又命令长乐下嫁南楚太子赵嘉,当时长孙贵妃在父皇面前哭诉,大雍和南楚迟早反目,若是长乐嫁了过去,将来如何自处。但是父皇还是下定了决心,长乐公主临别时那绝望的眼神令李贽至今不能忘怀,虽然他巧妙安排,让雍女争夺赵嘉的宠爱,避免长乐公主和赵嘉有太多的感情牵扯,可是当他知道长乐公主几乎隐居一般的生活的时候,还是痛惜万分,尤其是知道长乐公主怀孕之后,几经考虑毅然打掉孩子的时候,李贽几乎可以眼见长乐的悲痛绝望,她是明明知道这个孩子如果出生将来会面临的一切多么残酷的,所以才下了这个决心的。

想到这里,李贽断然道:“父皇放心,这次儿臣一定会接回皇妹,皇妹为我大雍牺牲良多,儿臣一定会保证她的安全,把皇妹接回来在父皇膝下承欢。”

李援叹息道:“接回来以后,过一段时间,朕要为长乐另外择婿,也免得她如此青春年少,就形如守寡。”

众人一阵犹豫,韦观开口道:“陛下心意随好,但是赵嘉若被俘虏来此,短时间内仍需借助他的名义,公主是南楚王后,若是陛下为公主公然择婿,南楚臣民必然切齿痛恨大雍。”

李援怒道:“难道让朕的女儿永远受苦不成?”

韦观语塞,在他看来,长乐公主幸福与否并不重要,但是这话他可不敢说。

李安打圆场道:“父皇,韦相说得也是有道理的,不如这样,我们先为皇妹选好夫婿,让他们先暗中订下婚约,等到过几年,南楚略为平定,赵嘉没有什么作用之后,再名正言顺的为皇妹完婚。”

李援微微点头道:“就这样吧,这件事情先不要传出去,等到长乐回来之后再说。好了,朕有些累了,你们去吧。”

李安、李贽、李显、韦观、程殊、秦彝都起身告辞,纪贵妃扶着李援走出了御书房。众人也各自离开,李显没有和李安一起走,反而故意留到后面,对李贽说道:“二哥,你以为江哲一定会归顺你么?”

李贽淡淡道:“怎么,六弟也想留他在麾下。”

李显摩拳擦掌道:“二哥,那个江哲,我一见就觉得投缘,你麾下文臣武将多如牛毛,这个江哲就给我吧。”

李贽微微一笑道:“你认为他不投我,就一定会投你么?”

李显道:“我看这小子有的时候还是挺识时务的,他若肯投我,我就拜他为老师,对他言听计从,他一定会答应的,只要二哥别和我抢。”

李贽苦笑,没想到李显竟如此折节下交,他不愿和李显争执,便道:“现在还不知道他肯不肯归顺大雍呢,我们争得太早了,对了,你和秦姑娘什么时候成婚?”

李显笑道:“我倒不急,反正名分已经定了,秦铮的师父和父亲都希望我快点,所以准备下个月大婚。”

李贽笑道:“那我赶不回来了,你呀,拖了人家好几年,亏得秦姑娘等着你。”

李显嗤道:“如果不是纪贵妃催父皇下旨,我还想再等等呢。外面美人如此之多,我哪里忙得过来,上次在南楚见过的那个柳飘香,真是一个天生尤物,若非是为了秦铮,我就可以到手了,二哥,这次到了南楚,你不妨去看看她,真是一个绝代佳人,像梁婉那种假惺惺的女子,还比不上她呢,女人么,干什么一脑子忧国忧民的。”

李贽笑道:“好好,我就告诉弟妹去,让她知道你瞧不起她。”李显连忙告饶不已。

李贽虽然面上带笑,心中却是冰寒一片,李安现在得到凤仪门支持,又有李显臂助,如果李显再成熟一些,那么李安真的就可以和自己分庭抗礼了,而不是凭仗父皇的偏袒,想到身边越来越危险的局势,李贽再次确定,必须得到江哲,他需要一个可以帮他冲破重重障碍的助力。

至化元年九月,雍王李贽献策平楚,率四十万大军南下,荆襄震动。

——《南朝楚史·楚炀王传》

\chapter{第二十章 趁火打劫}

当我听说蜀中和襄阳同时受到攻击的时候并没有什么奇怪,按照我的想法,想要攻打南楚,双管齐下是必不可少的,虽然花的时间长,但是只要夺了江淮,还怕南方不平呢,所以当我听说雍王带着两万轻骑直奔建业的时候,当时就呆住了,立刻翻出地图看了半天,越看越是糊涂,雍王雄才大略,怎么会这样做法,这样虽然可以一时攻占建业,但必定很快就会失去,就算南楚君臣落在他手里,必然会有人另立新君,甚至干脆取而代之,何况这样一来南楚必定陷入割据的局面,想要平定就得一城一池的厮杀,这样一来,没有二十年的时间,江南绝对无法平定。苦思了半天,我还是不明白李贽的用意。

要是换个角度呢,我突然想到,战争不过是政治的延续,那么李贽可以得到什么好处呢,可是我想来想去,不过是一个混乱的南楚会让太子李安不敢随意难为李贽,可是,如果李贽一举破楚,和李安真的翻了脸又有什么关系,我倒不相信李贽会斗不过李安,想来想去还是想不通,我万分疑惑的放下了手上的情报,不过这些,虽然出乎我的意外,但是我可以趁机实行我的计划,想到这里,我淡淡道:“赤骥。”替我整理地图的赤骥抬起头看向我。我下令道:“传信给你们的师父,今夜我要见他。”赤骥说了一声“是”,就转身出去了。

到了晚上,小顺子来得很快,我坐在书案后面,秘营八骏,也就是赤骥他们分别站在左右两侧,陈稹和寒无计分别站在左右两侧的首席,小顺子一进来就走到我身后,那里是他的位置,现在,秘营的统领是陈稹,天机阁的总管是寒无计,小顺子虽然没有明确的身份,可是人人都知道他是我的替身,可以替我发号施令,而且小顺子又是秘营弟子们的武技师父,秘营弟子对小顺子都十分尊敬,这就形成了小顺子崇高超脱的地位,可是他对我始终如同从前一般,甘愿作我的仆人侍从。

我见人到齐了,开口道:“诸位,我建立秘营、天机阁,等待的就是今天,时机已经成熟,今日我请诸位戮力同心,助我完成复仇大业。”

陈稹道:“公子,尽管吩咐,若非公子执意等候,我们拼了性命也早就杀了梁婉。”

其他人都只是静静的听着,按照我的规矩,不轮到他们是不能随便说话的,陈稹是秘营统领,除了小顺子,寒无计之外,所有人都是他的下属,小顺子没有必要是不会说话的,而寒无计的身份地位在陈稹之下,所以他也不会随便插话。

我看看寒无计,问道:“天机阁可一切准备妥当?”

寒无计躬身道:“公子放心,虽然因为雍军即将到来的消息传开,很多商人都开始逃难,但是公子事先吩咐的部分都在掌握之中。”

我点点头,说道:“从前我一直在等这个机会,只有南楚和大雍完全撕破脸,才会有我要的机会。那就是长乐公主,南楚王后,从一开始我就觉得大雍皇帝对这个公主确实十分爱护,你看他派了那么多美艳的宫女陪嫁,再看长乐公主多年来总是和国主若即若离,可见长乐公主只需要人在南楚即可,我想为了日后免得公主为难,所以大雍皇帝根本不希望公主和国主有太多的感情。”

听了我的话,小顺子等人先是迷惑猜疑,然后神情渐渐明朗,小顺子道:“公子说得不错,我在宫里知道,王后基本上不和国主共处,除了必要的时候,王后总是尽量待在行宫,就是待在宫里也总是落落寡欢,从不争宠,以前我还以为王后贤德,现在看来,正是公子说得那样,她跟本就无心留在南楚。”

我拍案道:“是啊,若非大雍皇帝爱惜这个女儿,完全可以不理会她的心情,让她好好笼络国主,才有更好的收获,既然他如此爱惜长乐公主,那么在大雍和南楚翻脸之前就一定要救出公主,而梁婉必然是主持这件事情的人,梁婉纵不畏死,长乐公主若有闪失,只怕她会比死还难过,所以只要我们趁她们逃出王宫的时候将他们困住,为了长乐公主的安全,梁婉就是想不招供都不可能。只要她招了供,她的生死就不再重要,我就可以快意恩仇,不过保护长乐公主的高手一定不会少,我们行事要万无一失,绝对不能让他们逃走,小顺子,这次你是我的主力,你有把握么?”

小顺子想了一想道:“公子放心,以我现在的武功,将她们抓住或许费劲,但是想要杀了她们不费什么气力,只要公子策划周密,我可以保证一定不会让她们逃走。”

我喜道:“好,好,骅骝、绿耳,你们两个率领隐组,一定要掌握好她们的一举一动,、白义、逾轮、山子、渠黄、你们四个率领虎组、暗组,是围困她们的主力,赤骥、盗骊你们率领龙组负责协调和善后,具体事宜由陈稹、寒无计你们指挥,现在立刻行动。小顺子,你先去跟踪王后,只要抓紧了这条线。梁婉决不可能逃走。”

在我紧张的阅读各种情报,好确定该采用那一种策略的时候,朝中已经一片混乱。国主赵嘉满眼都是红丝,愤怒地道:“每天总听你们说什么,我南楚兵精粮足,可是大雍就这么穿过防线,再过三个时辰,雍军就兵临城下了,你们说怎么办,怎么办。”

丞相尚维钧道:“陛下不用担心,雍军轻骑千里,到这里已经是强弩之末,建业虽然空虚,还有五万禁军,只要我们防守一段时间,勤王之师就会到达。”

这时一个大臣道:“陛下,尚丞相此言虽然有礼,可是雍军精锐,若是我们守不住建业,岂不是社稷危殆,依臣之见,陛下应该暂时移驾,到一安全之处暂避,等到敌军退后,再回建业重新整顿,陛下万金之躯,不可轻易涉险。”此言一出朝臣纷纷符合,这些人平日不是饮酒作乐,就是寻花问柳,自从赵嘉继位以来,贤臣大多疏远,小人却是越来越多,前次因为称帝的事情更是贬斥了一大堆贤臣,所以如今事情紧急,反而找不到可以共商国事的臣子了,尚维钧虽然平日庸碌,但这次倒是比较明智的,但是众怒难犯,最后只得折中道:“既然如此,陛下不妨暂时临幸他处,就由老臣率领禁军守建业,还请陛下允许太子监国。”赵嘉连连答应道:“好,建业就委托丞相了,只是太子才四岁,留下来恐怕没什么用处。”尚维钧心想,如果不留一个皇子在此,怎么抵挡雍军啊,只得再三请求,赵嘉对自己的太子本来也没有深厚的感情,但是现在他发现雍女之外的妃子只有尚妃生了皇子,自然多了几分关注,但是眼看雍军即将到来,赵嘉终于不愿耽误时间,匆匆忙忙带了一些亲信的大臣、妃子和几千禁军在雍军到来半个时辰之前就逃走了。赵嘉还没出城,尚维钧就下令派禁军去抄了明月楼,又派禁军围住中宫,将仍然留在后宫的长乐公主软禁,虽然赵嘉沐猴而冠的晋位皇帝,但是因为大雍和南楚交战余波未歇,所以还没有将王后晋封皇后,从李显第一次进攻襄阳,赵嘉就派人把王后接回宫中,只是惧怕大雍的强势,没有敢公然软禁,倒是长乐公主十分识大体,足迹不出宫门一步,如今的软禁也不过是做个样子,谁知禁军回复,明月楼已经空无人迹,而长乐公主也已经不见了,所有的宫女都被关在一间屋子里,尚维钧大惊失色,他知道失去了护身符,也顾不上检查防务,下令召来自己的亲信武士,让他们到后宫保护着尚妃和太子化妆成平民,立刻逃走。然后尚维钧立刻到城上主持守城。

与此同时,建业北郊的一处农庄里面却是白刃溅血的场面,梁婉一身青色布衣,手中拿着一柄短剑,剑身上仍然雪白如霜,但是梁婉却是额上见汗,在她身后的椅子上,容颜憔悴清丽,她也是一身布衣,身后站着一个秀丽的侍女,手上也拿着一柄短剑,在左右两侧站着十几个农夫装束的大雍密探,却是个个带伤,地上散放着一些带血的弩箭。

梁婉无论如何没有想到,自己刚刚带着长乐公主到了事先选好的隐蔽农舍,就被人偷袭,自己措不及防,只得带着人退入农舍,才发觉事先排在这里的两个人都被捆得严严实实,两人双脚都被砍伤,然后又妥善处理过,梁婉几次带人突围都被弩箭阻拦,一次梁婉仗着身上的软甲冲出去,谁知刚刚冲出院门就被四个手持长刀的蒙面人拦截,这些蒙面人的武功在梁婉看来不过是二流水准,但是他们勇猛善战,刀法凶狠,而且彼此呼应,组成刀阵,梁婉一时竟被困住,眼看弩箭招呼而来,只得拼死冲了回去,若非接应得当,只怕她的性命就留在外面了。如果不是有长乐公主在,她自然可以安排四散突围,凭她的武功逃出去的可能很大,只是现在却是进退两难,她心里越想越糊涂,围困自己的这些人是十分精锐的军士,至少不比大雍最精锐的军队差多少,而那些阻拦自己的高手更不是可以随便拿出来的,在如今的南楚,建业附近怎么可能有一支这样精锐军队,就算真是南楚的密谍,为什么到这里才动手,完全可以在自己将公主从宫里救出来的时候动手啊。梁婉始终想不通外面的是什么,但她很明白,必须守住,为了安全,她并没有通知雍军这个地点,如果等不到雍军来到,不仅她的命没了,就是公主也完了,如果公主出了事情,自己就是死了也难以平息雍帝的怒气,到时候承受怒气的就有凤仪门。

梁婉正在想着,一个人低声道:“梁小姐,他们醒了。”

梁婉心中一喜,他们留在这里的人虽然伤势得到处理,而且也没有死,可是却一直昏迷不醒,应该是服了什么药物。她走过去,急急问道:“怎么回事,是谁偷袭了你们。”

一个人舔舔干裂的嘴唇,道:“小姐,来得是一个人,黑衣蒙面,没有说话,武功高的出奇,只一招就伤了我们两个,那人本要杀了我们,却被一个后来的人阻止了,那人应该不会武功,因为他脚步虚浮,中气不足,他下令砍伤我们的双腿,然后我们就昏迷了过去。”

梁婉听了他们的说话,却没有什么帮助,这时外面传来冰冷地声音道:“屋子里面的人听着,我们已经不耐烦了,如果你们还不出来,一拄香时间之后,我们就用火攻。”

梁婉高声道:“你们若用火攻,不怕引起别人注意么?”她想试探来人的立场。

外面沉默了一会儿,那人又道:“南楚自顾不暇,大雍还得半个时辰才到,时间足够了,你们想的越久,待会儿我们的处置就更严厉,如果你们现在投降,我可以保证,至少你们不会死得太痛苦。”

梁婉冷汗直流,她第一次后悔自己没有带更多的人来这里。在她犹豫的时候,几捆稻草扔到了门口,一个火折子丢了过来,火焰升起,梁婉无奈,大喊道:“我们归降。”

两把钉耙将稻草扒走,一个身形不高不矮的黑衣蒙面人出现在门口,他双手空空,没有任何武器,可是梁婉却感觉到那人身上传来隐隐的压力,她左手按住腰间的飞刀,却失去了发刀的勇气,那个黑衣人用一种阴柔动人的声音道:“你们自束双手一个个走出来。”梁婉一震,这种声音她听过,那是太监的腔调,可是他们不应该是南楚的人啊。她鼓足勇气,丢下短剑,伸手整理了一下乱发,婀娜多姿的向那人走去,她知道这人很有可能是太监,就算不是也一定是练了极其阴柔歹毒的内功,那么个性也会是阴毒的性格,所以她不敢用美色惑人,而是极力表现出一种柔顺服从,她把双手背在身后,向那人走去,就在经过那人身边的时候,她的身躯仿佛毒蛇一般折转滑动,右手的飞刀向那人咽喉刺去,那是促不及防的一刀,但是那人的右手轻轻划出,梁婉只觉得手腕一麻,然后那只苍白冰冷的手捏住了自己的咽喉,梁婉只觉得那只手仿佛毒蛇一般的恶心可怕,然后她就失去了知觉。

等梁婉醒来,发觉自己在一片黑暗当中,她仔细聆听,却没有感觉到身边有人,她扭动一下身体,发觉自己的双手被牛筋紧紧的捆在身后,她的武功还在,身上也没有任何异样,她庆幸的吁了口气,她没有继续移动,毕竟她不想引起可能的注意,这时传来一个冰冷的声音道:“你醒了,公子要见你。”然后灯光亮起,梁婉不由自主的闭上了眼睛,然后两个人过来将她拖了起来,从感觉上看,这两个人都是年轻人,梁婉本能的想着。那两个人根本不想让她自己走路,将她拖到了一间宽敞的房间,看不到窗户,那是一间密室,四处燃着火把,在屋子中间的一张椅子上,坐着一个穿着黑色儒衫的蒙面人,而在四面的墙上,自己所有的属下都被五个铁环锁在墙上,他们身上没有受刑的痕迹,除此之外,梁婉看到那个黑衫人身边站着一个人,从他的双手可以认出,那人正是将自己生擒的高手,除此之外,屋子里还有六个黑衣人分别站在角落里。梁婉被一直拖到那黑衫人对面的墙上,那两个人熟练的将梁婉的手腕、脚腕用铁环拷住然后又将一条铁链拦住她的腰部,收紧,梁婉只觉得全身上下一丝也不能动弹,另外一个黑衣人拿来一桶凉水,泼在她身上,梁婉身上全部湿透,露出玲珑剔透发育成熟的娇躯轮廓,她又羞又怒,虽然已经二十七岁了,可是她还是处子之身,怎么能忍受这样的羞辱,那些黑衣人都以肆无忌惮的目光看着她,就是她那些属下也都偷眼看来。

梁婉怒道:“你到底是什么人,为什么和我大雍为难。”

那个黑衫儒生淡淡道:“在下并非和大雍为难,梁婉,我要的是你,其他人不过遭了池鱼之殃。”

梁婉心中一凛,想道,我这几年都在为大雍效力,怎么会有人找我报私仇,看着属下犹疑的目光,她有些羞恼,道:“你们把另外两位姑娘怎么样了?”

她不敢说明长乐公主的身份,可是那黑衫人却道:“你是说长乐公主殿下么,公主殿下与此事无关,在下也十分同情公主的遭遇,所以将她另外安排在一间厢房里,她那个侍女武功和你很相似,她想趁机偷袭,被我的属下误杀了。”

梁婉心中一恸,道:“你们真是狠毒,我师妹今年只有十九岁,想不到你们如此辣手。”

那黑衫儒生没有说话,他身后站立的那个人用阴柔的声音道:“我们错手杀了一个人有什么关系,如果你不肯回答我们的问题,我会让你生不如死。”

梁婉怒道:“你们究竟是谁,与我有什么冤仇。”

那个黑衫儒生冷冷道:“我只问你一件事,柳飘香是不是你杀的。”

梁婉顿时愣住了,她无论如何没有想到,居然会有人问她这个问题。

\chapter{第二十一章 得知真相}

我看着梁婉,为了抓住她,我费了多少心思,安排了多少暗桩,终于发现了她们要隐藏公主的地点,等她们入伏之后,我用军阵的方式围困,再用强大的武力和公主的安危威逼,终于将他们生擒,虽然似乎很简单顺利,但我花的心思却是太多了。为了迫使梁婉招供,我用这种方式让她觉得无力自保,只有让她失去所有的信心,才有可能让她乖乖招供,否则被她看穿我也不愿伤害公主,那么就惨了。

梁婉惨淡地道:“你是她什么人?”

我淡淡道:“飘香与我已有白首之约,那日她惨死那天得前一晚,她就在我的住处,可惜为了善始善终,她不忍拒绝艳娘的请求,所以死于非命。”

梁婉看着面前的那些人,飞快的搜索着记忆,想着和柳飘香有关的任何人,可是柳飘香虽然裙下之臣众多,却没有一个会符合眼前这人的行径,她又仔细的想着柳飘香临死前的情景,当时自己走进房间,看见柳飘香正在沐浴,她美丽的容貌上带着火一样的愤怒,看到自己,她冷冷道:“想不到明月公主不过如此,竟然欺骗侮辱我这样一个小女子。”梁婉还记得自己委婉的劝解,柳飘香却是神色冰冷地道:“你们位高权重,我也无话可说,就是告上了官府,也没有用处,你放心好了,我有自己的生活要过。”她明明是那样的表示忍让,可是自己却偏偏心生寒意,她不相信曾经敢当众凌辱韩王赵德隆的柳飘香会不追究这件事,想到只要柳飘香把这件事传了出去,自己的声誉就会化为乌有,如果失去在南楚的立足之地,那么自己苦心经营的一切都会被人占有,自己终于在柳飘香离去之前下了毒手。

我看到梁婉的思索,心中涌起滔天的愤怒,如果不是她杀了飘香,怎会这样深思,我冷冷道:“你想起来了么?”

梁婉看了我一眼,心道:“原来当日柳飘香之所以委曲求全,答应不向自己报复,却是为了和情人的团聚,看来她的情人身份应该不会太高,否则柳飘香不会答应不报复的。”

就在她继续思考的时候,那个到声音冰冷的黑衣人走到她面前,抓住她胸前的衣襟一扯,碎帛飞散,梁婉只觉胸前一凉,酥胸半裸,梁婉羞恼的叫了一声,知道这是对自己的警告,只得道:“既然到了这种地步,我相信阁下已经有了足够的证据,不错,柳飘香是我杀的。”

她承认了,我狠狠的看着梁婉,问道:“好,那么告诉我,那个欺辱了飘香又让你为他善后的混蛋又是谁?”

梁婉这才明白,原来自己仍然能保住性命的关键在这里。她本是智力过人的女子,如今有了可乘之机怎会不利用,她微笑道:“原来阁下想要知道这件事,这件事只有我一人知道,请问阁下,愿意付出什么代价来交换这个消息?”

我淡淡道:“早知你会这么做,但是若非有了足够的把握,我又怎会动手,梁姑娘,不论你身份何等重要,地位何等显赫,今日你落在我手里,我可以为所欲为,如果你肯说出那个人,我保证会让你死的安详,若是你不肯说,我有千百种法子,让你死不瞑目。”

梁婉冷冷一笑道:“我知道,对于一个女子,伤害她的方式有很多,你可以让这房间里所有的男子来侮辱我,你可以对我用尽酷刑,你还可以毁了我的容貌,可是你应该相信,我梁婉有着铁样的心肠,不论你如何伤害我,只要我抵死不说,那么最后死不瞑目的会是你,如果你肯和我公平谈判,那么我答应有一天会告诉你这个人的身份。”

我轻轻拍手,笑道:“好,不愧是大雍的密谍首领,你们说,我当初的谨慎是否有道理。”

陈稹冷冰冰地道:“公子果然才智过人,属下等拜服。”

我走到梁婉身前,冷冷道:“我早就知道你会这么做,你有必死的信念,我也相信你可以熬过种种酷刑,在下精于医道,可以让你尝到人生最大的苦痛和侮辱,这些人都是你的下属,我可以让你在他们面前婉转求欢,到时候你还有什么脸面作他们的首领。”

梁婉强忍心中的恐惧,道:“我知道你可以做到,听说有人擅于制作强烈的媚药,女子若是服了不堪设想,可是我只要记得是被药物所困,就不会因此抬不起头来。”

我冷笑道:“事后你更可以将他们杀了灭口,也就没有知道你的丑行了,是吗?”

梁婉淡淡道:“我怎会如此。”可是她目中带着惊骇,这正是她的打算。

我轻笑道:“你至今守身如玉,可我相信你不是一个洁身自好的女子,那为什么你没有情人呢,是你看不上天下的男人,还是你有了意中人,还是对你来说,处子之身十分重要。”

小顺子突然道:“公子,她所练的武功应该不会要求女子守身,我想她是有了意中人,或者她的目的是做某个人的妻妾,所以必须维持处子之身。”

我看看梁婉的神色,笑道:“或许真的如此呢,来人,拿酒来,给她喝下去。”

盗骊端着一壶酒和白义一起走了过来,白义捏住梁婉的鼻子,盗骊轻轻松松地将那壶酒给她灌了下去,他们手法娴熟,梁婉毫无反抗的余地,但是酒液仍然有小半流到胸前,梁婉等他们松开手,咳嗽了几声,只觉得胸前冰凉,喉中却是火辣辣的,脸上更是一片因为憋气导致的嫣红,梁婉只觉所有人的目光都落到自己身上,虽然羞愧,但她知道生死荣辱系于此刻,所以仍然坚强的抬起头,看向那黑衫儒生,心想,自己若是难以控制的时候便咬舌自尽,就是被阻止,那些人也会知道自己的绝决。

过了不久,梁婉没有觉得春心荡漾,却觉得神清气朗,灵智活泼,仿佛身在仙境一般快活,梁婉渐渐的有些慵懒,恨不得躺下来睡去,可是身躯一动,却被牢牢缚住。这时耳边传来一个温和的声音道:“梁姑娘,你可想休息了么?”

梁婉低低呻吟一声道:“我想睡一觉才好。”

那个声音又问道:“你在南楚这么久,想必收买了很多高官,手下有很多探子是么?”

梁婉神色迷蒙,回答道:“是啊,雍王殿下派我来保护公主殿下,后来又让我主持江南谍报,可惜我只能辜负他的厚爱了,师父说,太子殿下才是真命天子。”

“你的师父是谁?”那个声音还在问她。

梁婉不耐烦地道:“我师父当然是凤仪门主了。”

“噢,那么是谁要你去请柳姑娘到明月楼的?”

梁婉刚说出一个“是”字,突然清醒过来,她目射寒光,冷冷道:“我都说了些什么?”这时他的一个属下冷冷道:“你说,你背叛了雍王,投靠了太子。啊--”一个黑衣人的铁拳击中他的小腹,让他不能再说话。

我看看面如死灰的梁婉,道:“你连背叛的事情都说了出来,那么还有什么可以隐瞒的呢?”

梁婉冷笑道:“虽然我失言说了一些事情,大不了以后我明目张胆的效力太子,至于你想知道的人却是我唯一的筹码,所以你若不肯付出代价,我绝对不会说出那人的身份。其实你何必为了一个娼妓如此费心,天下好女子不知道有多少,我凤仪门中就有很多品貌非凡的师姐妹,若是阁下喜欢,梁婉愿意代为做媒。”

我淡淡道:“飘香虽然不幸落在风尘,但她的心却如九天明月,而梁姑娘虽然僭号明月,但是其行还不如风尘女子坦白。”

梁婉气得面色铁青,我却轻轻叹息了一声,梁婉果然是很难对付,我开始故意谈及媚药,因此人人都会以为我给她喝下的酒里面掺了媚药,我在酒中的确掺了药物,但是却是罂粟精练的迷魂药,这种迷魂药的最大缺点就是如果服用者有了准备就很难管用,我曾让俘虏来的大雍密谍服下此药,可是他们在没有准备的情况下仍然一言不发,所以我先让梁婉明白我的目的,这样她就失去了戒心,然后再让她服下她认为可以抵御的“媚药”,而服下迷魂药的梁婉果然说出了一些事情,遗憾的是,梁婉对生死相关的事情防备得太严,所以没能成功,但是我并不气馁,这原本就是我计划中的一步,到此为止,梁婉已经了解我对此事的关注,那么我使出杀手锏的时候,她才会答应和我交换条件。

我轻笑道:“看来梁姑娘真是不肯说了,既然如此,我就只好得罪了。”

梁婉傲然道:“我倒要看看你还有什么手段。”

我淡淡道:“我想请姑娘听一出好戏。”说罢,我挥了挥手,赤骥对我施了一礼,转身推开我的坐位后面的石门,就在石门打开的一刻,所有人的目光都看到了一面侧放的一人高的大青铜镜,镜子里灯光明灭,可以看到一张流苏帷帐的大床,在床沿上坐着一个素衣少女,正是长乐公主,从镜子的角度来看,长乐公主应该就在石门之后的房间里。赤骥走了进去,然后石门关上了。所有大雍的密谍都用一种可怕的目光看着我,看来他们已经猜出了我的手段。我一摆手,一个人将石门上隐藏着的一个铜管拉了出来,这时所有的人都听见从铜管里传来了声音。

“你是什么人,要对哀--我做什么?”

“不,你不要过来,你不要过来。”然后传来裂帛之声,和少女哭泣挣扎的声音。

“住手,住手。”所有的大雍密谍都在喊。只有梁婉仍是一脸的铁青,没有出声。

我示意合上铜管,虽然听不到声音,但是那些人更加的忧虑,他们开始拼命挣扎,有人开始叫骂。

我冷冷道:“梁姑娘,如果你不肯说出我想知道的事情,那么长乐公主会遭受到什么,你会明白,我想知道,如果大雍的皇帝陛下知道因为你的缘故让他的爱女受到如此折磨,他会怎么对你,太子会怎么对你,雍王会怎么对你。”

梁婉绝望的抬起头,她知道自己已经陷入了一个最深的陷阱,这个人如同魔鬼一般可怕,从他对付自己的手段可以看出,他是一个心思深沉的恶魔,他绝对做的出这种事情,只有一件事他不会作,就是伤害自己,因为他经不起她抵死不说的后果。

她苦涩地道:“让你的属下住手,如果公主没有受到伤害,你又答应不伤害我,那么我会告诉你。”

我淡淡道:“快些说吧,我的属下性子不急,你说出来,就还来得及。至于你的性命,我答应,今天不取你的性命,也不再伤害你。”

梁婉凄然道:“我只能相信你,那人是太子李安。”

我眉头一皱,冷冷道:“你在胡说么?大雍太子怎会到南楚来?”

梁婉镇定地道:“齐王许诺南楚国主可以称帝,但是破蜀之后,又要出尔反尔,如果没有身份更高的人来安抚,这件事情传出去岂非令大雍颜面无存,所以太子殿下秘密抵达南楚,除了赵嘉之外没有见任何人,临走之前,太子说听齐王殿下讲,柳飘香不可不见,我原想柳飘香不过一青楼女子,见了太子还不倾心相从,谁知柳飘香来了之后只是唱了一曲就要告辞,太子殿下一怒之下用了强,事后要我善后,我只得杀了柳飘香。”梁婉撒了一个小荒,李安虽然让他善后,却没有让她杀人,他认为只要多给些金银就可以了,偏偏梁婉畏惧柳飘香将这件事情传了出来,自己名声扫地不说,还会让太子殿下受到非难,所以才杀人灭口,对于梁婉来说,柳飘香的生死不过是一念之间罢了。

我看看梁婉,终于得知事情真相的我几乎万念俱灰,我要怎么向一国太子报复。梁婉似乎看出我的变化,道:“阁下,你若肯抛弃前嫌,梁婉保证你青云直上。”

我冷冷道:“你说得是真话么?”

梁婉冷冷道:“你只能相信我,若是你不信,当然可以出尔反尔的杀了我。”

我没有作声,再确认她说的是实话前,我不会杀她,梁婉也知道这一点,才会敢说了出来。

这时,一个大雍密探道:“阁下,你还没有放过公主呢。”

我没有说话,陈稹打开了石门,所有的人都看到在那面铜镜里,公主仍然坐在那里,只是姿势有了一些变化,赤骥走了出来,关上门。

我看了他们一眼,解释道:“诸位放心,公主殿下命运坎坷,在下怎会为难她,我这个属下精于口技,让各位见笑了。”

那些人都松了一口气,公主没有受到任何伤害,令他们十分欣慰,而梁婉却恶毒的看着我道:“原来是你,我知道你是谁了,江哲,你是江哲。”

她的话语如同寒风吹过一般,让所有的人都安静了下来,我的人自然是因为我身份的暴露,而大雍的人却是因为惊讶,他们都知道我这个状元才子的。

我冷冷道:“梁小姐怎会认出我的。”

梁婉傲然道:“你的声音,我终于想起来你的声音在哪里听过,还有,你在提及公主的时候,眼神温柔,充满同情怜悯,当初你觐见公主的时候,我见过你这个眼神。”

我赞赏的看了梁婉一眼道:“果然厉害,梁姑娘不愧是大雍密谍中的佼佼者,居然看穿了我这个不大接触的人的身份。”

梁婉神色有些古怪,冷冷道:“江哲,你挟持公主,犯下大罪,日后你若愿意,我可以引荐你进入大雍朝廷,到时候前程似锦,你何必为了一个女子和自己的前途为难。”

我冷冷一笑,道:“梁姑娘,你真的是太可怕了,所谓,青竹蛇儿口,黄蜂尾后针,二般皆不毒,最毒妇人心,我今日才信了,不错,我不杀你,我也不会伤害你,我只要你的记忆和才慧。”

小顺子走了过来,将一粒龙眼大的红色药丸塞进梁婉的口中,梁婉想要挣扎,可是小顺子冰冷的手让她失去了抵抗的勇气,我淡淡的看着她恐惧的眼神,道:“我没有杀你,也没有损害你的一丝一发,这粒药丸服下,你会忘了一切,我虽然不能确定你会忘掉多少,但我可以保证,你不会再记得今天发生的一切。”

梁婉恐惧的望着我,她以为我可能不会那么容易放过她,可是她怎么也不会想到我会用这种方式,她叫道:“我是骗你的,我告诉你的不是真的。”

我冷冷道:“梁姑娘,你若是要替人隐瞒,用得着拿太子殿下搪塞吗。”

梁婉只觉得一幕幕回忆从心底涌起,幼时的喜乐,少女时候的辛苦练武,第一次见到雍王殿下的惊喜动心,还有在南楚的种种钩心斗角,最后出现的却是柳飘香临死之前那种满含遗憾的眼神,然后一切的一切渐渐飘散。到了最后,梁婉脸上露出孩童一般的笑容,是那样的天真无暇。

我淡淡道:“你杀了我的妻子,我毁去你的人生,虽然不算扯平,但是也算你抵罪了,梁姑娘,若是我们没有再见之日,那么你就好好的活着吧,若是你我不幸,他日陌路重逢,我只好取了你性命,慰我爱妻在天之灵。”

我抬目望去,除了小顺子,所有的人眼中都是一片恐惧,即使是陈稹和赤骥、盗骊他们,他们都见过我用药毁去那些被送走的孩子的记忆,但是那时候我用的药量很小,所以只是让他们失去两三年的记忆罢了,那里见过今日梁婉这般的情景。我微微一笑,他们心里有所恐惧也不错,看看那些大雍的密探,我淡淡道:“你们知道了我的身份,抱歉,不能让你们这么离开了。”

一个人道:“你也要让我们服这种药?”

我摇头道:“这种药的价值胜过等量的黄金,我不会舍得随便使用的,你们的性命我要取走了,反正你们在南楚多年,我杀了你们并不为过。”

那些人眼中闪过悲壮的神色,其中一人道:“阁下是南楚高官,与我大雍上有国仇,下有私恨,你杀了我们原本没有什么关系,可是阁下既然怜惜公主殿下,还请阁下不要将公主交给南楚中人,请阁下将殿下送到雍王面前,我们虽死无恨。”

我看了那个汉子一眼,道:“今日之事,上不可告天地,下不可告父母,你们知道了这些隐秘,就是我不杀你们,你们也活不过太子的追杀,若是你们肯守信诺,我可以还你们兵刃,让你们送公主殿下到雍王那里,只是事后你们需要自杀守秘。”

那个汉子眼中闪过惊喜,道:“阁下肯相信我们。”

我正容道:“我相信大雍勇士的承诺,你们若是毁诺,只会让我瞧不起雍王殿下,你们见了我今日的手段,就该知道,我若想暗杀一个人并非难事,到时候雍王殿下就是你们背信的代价。”

那个大汉想了一想,道:“阁下手段如此冷酷阴狠,谋划又是如此严密,你若在暗中谋刺雍王殿下,果然有五分把握。好,我们的贱命,有什么要紧,完成任务才是最重要的事情,只是请阁下答允,我们想向雍王殿下禀告太子和梁婉勾结的事情。”

我淡淡道:“可以,只是,你们不能提及任何一件关于我们这些人和拙荆的事情。”

那大汉慨然应诺,我微微一笑,转身出去了,接下来的事情自有陈稹去办。小顺子跟在我身后,问道:“他们会守信么?”

我点点头道:“我不会看错人的,他们都是坚贞的勇士。”

\chapter{第二十二章 建业沦陷}

就在我在密室逼迫口供的时候,建业城已经安排好了防务,尚维钧连下谕令安排守城,虽然国主逃走的消息被人故意宣扬出去,所以城中禁军大多失去斗志,尚维钧下令连杀了数百散布谣言的“奸细”,这才勉强稳住了军心,尚维钧长久以来主持朝政,所以禁军将领都愿意听命,只是五万禁军对于守城来说并不足够,令尚维钧十分为难,后来只得驱使城中青壮男子上城作战,等到大雍前哨军队到来的时候,建业城已经可以一战了。

第二天,当东方的朝阳刚刚露出云层,在刚刚破晓的曙光中,千余黑衣黑甲的彪悍骑士由远及近,为首的一个黑衣将领提马立在一个小山坡上,远远的看着威严耸立的建业城,其他的骑士各自分散开来,片刻之间就都不见了踪影,只留下那黑衣将领和十几个亲卫,过了片刻,四野传来隐隐约约的号角声,那个黑衣将领接过亲卫递过来的号角,呜呜吹响,声音凄厉激昂,城上的守军都觉得心情异常紧张,恨不得大叫起来,虽然守城将领连连呵斥,仍然不时传来惊呼声,而远处的雍军骑士却是森然而立,毫无声息,过了片刻,远处传来大地剧烈颤动的声音,数万只马蹄践踏地面的隆隆巨响,震得人耳鼓生疼,片刻之间,从地平线处涌出成千上万的黑甲骑士,初时可以看到他们都是三五成群的散兵阵列,而就在他们冲向建业城的数里路程之内,可以明晰的看到他们由散列汇聚成密集而有序的战列的过程,那是一个宛如行云流水的过程,在离建业千步之外嘎然而止,接着战阵从中而分,一个金甲骑士缓缓走了出来,他身上披着黑色的大氅,在他身后,一个亲卫骑士挥开大旗,上面是血红的一行大字“天策大将军李”。旌旗展开的瞬间,那铁甲洪流中到处响起悠远豪迈的号角声,冲天的杀气,摄人的威严,让建业守军都不由心寒。

一个识文断字的禁军眯着眼睛看向那旌旗,叹息道:“威远大将军李,真的是雍王来了,听说他是大雍最厉害的王爷,咱们真的能守住建业么?”在他旁边的一个新招募来的军士忐忑不安的问道:“不是说是雍王领军么,怎么又是什么威远大将军?”禁军白了他一眼道:“你知道什么,威远大将军是雍王爷的官职,雍王是他的封号,听说雍王的旗子从来打得都是大将军的旗号,有人说是因为雍王觉得大将军是他一刀一枪杀回来的,所以才那么重视,另外还有一个金龙旗,那是只有安营扎寨或者打了胜仗以后才打的旗子。”新兵羡慕地道:“大哥你知道的可真多。”禁军得意地道:“那当然,老子当年攻打蜀国的时候见过雍王的军队,那时候咱们是友军。”

“啪,啪。”两声皮鞭着肉的声音传来,那个禁军惨叫一声仆倒在地,众人回首,看见督战队的一名军官虎视眈眈的看着自己,那名军官厉声道:“竟敢扰乱军心,若非大敌当前,本官先取了你的狗命。”那个禁军连忙爬了起来,道:“小人不敢,小人不敢。”看到督战队走远了,那个禁军吐出口里的血沫,恶狠狠的低声骂了几句,转过头看向城下。

尚维钧站在城墙之上,看着城下骁勇的敌军,心里盘算着,敌人虽然悍勇,但是只有两万,若是出城迎敌,能够捉到雍王的话,那么岂不是可以顿解危局,想到这里,他低声问身边的禁军副统领道:“敌人只有两万,我们是否可以出战。”副统领答道:“我们没有骑兵,还是守城的好。”尚维钧皱皱眉。这时城外的敌军高声呼喝讨战。尚维钧下令不许出城。只将檑木滚石准备好,等待敌军攻城。

远远的看着建业,李贽轻轻一笑,道:“我料他们不敢出城。”

他身边的亲卫统领司马雄问道:“殿下,我们只带了骑兵来,又该如何攻城呢?”

李贽笑道:“放心吧,我可没准备用骑兵攻城,建业虽然坚固,可惜军心涣散,我已经安排好了内应,今日我们就在这里看看就行了。对了,我想,我们派出去的人马会有些收获的。”

司马雄笑道:“是啊,殿下说南楚君臣可能会事先逃走的,所以安排陈将军他们先去拦截追击,密探来报,那赵嘉果然先逃了,若是我们绑了他们的国主在城下,不知道他们会不会乖乖投降呢?”

李贽道:“能不能捕获他们的国主大半得靠运气,不能依赖,还是想法子夺城要紧,那个尚维钧若非不通军务,咱们的探子还真的没办法安排内应呢,南楚将领凡是英勇善战的很难在建业立足,这可是千古奇谈,咱们大雍的禁军都是从军中精选的勇士呢。”

司马雄不赞同地道:“禁军虽然精锐,可是比起殿下的亲卫来说还差的远呢,虽然是因为太子排斥咱们的人加入禁军,可是谁不知道成为殿下亲卫的都是千里挑一的勇士。”

李贽微微一笑,没有反驳,他的亲卫三千铁骑,都是身经百战的虎赍死士,这次他带的两万人就是在亲卫基础上扩充的近卫军,精锐程度远远胜过大雍禁军,更不用说南楚的军队了。

这一天,李贽只命人在城下耀武扬威,尚维钧不敢出战,南楚军中更是消沉。到了黄昏,李贽命令到距离建业十里之外已经扎好的大营休息,尚维钧见李贽退兵,这才松了一口气,回到相府,他盘算着让尚妃带着太子先躲避起来是否太胆小了,想着明天是否接他们回来,胡乱吃了一些东西,尚维钧和衣在书房睡下,只是睡得很不安稳,从恶梦中惊醒了好几次。在睡梦中,尚维钧突然惊醒,他擦擦额上的冷汗,然后,他就听到由远及近传来的呼喊声和叫骂声,他坐起身来,房门被推开了,一个家人冲了进来,见他醒着,惊骇的喊道:“大人,不好了,城内的禁军反了。”尚维钧腾的站了起来,推开窗子,果然外面传来清晰可闻的叫喊声,有人大喊“敌军进城了”,有人大喊“国主都跑了,我们还卖命做什么”,大多的言语听不清,但是有人在叫骂,有人在喊着煽动的词语,尚维钧心寒如冰。就在这时,他看见城中四处开始起火,火焰冲天,尚维钧呆呆的看着火光,喃喃的不知道说些什么。

而在这时,建业西门的守军被偷袭,城门被打开了,雍军的铁骑闯关而入,建业的大街小巷没多久都是黑衣黑甲的铁骑,在冲天的火光中,他们的到来宛若鬼魅一般恐怖,街道上到处都是人喊马嘶,南楚军队开始还涌向西门,想要将敌军赶出去,但是在雍军残酷的杀戮下,很快就败退了,满街的残兵败将开始了逃命,甚至还有残军开始闯入民宅杀戮掠夺,建业城,在血火中颤抖呻吟。

天明之后,控制了建业城的雍军开始整顿城中的秩序,所有投降的南楚军被驱赶到城外营中监禁,趁乱打劫的乱军被杀死,悬挂头颅示众,所有的平民都得到闭门不出的谕令,火势也在雍军的指挥下被扑灭,然后控制了城门和城中要害的雍军开始盘查城内的住户,凡是南楚王族和三品以上的官员都被抓到天牢里面等候处置,其他人则被吩咐暂时闭门不许外出,大街小巷一片死寂,凡是擅自外出的人都会被问罪。尚维钧原想趁乱逃走,却被雍军俘虏,此刻正被关在天牢里面,其他敢于反抗的楚军都被斩杀。

到了午时,李贽入城了,看着血迹殷然的街道,李贽微笑道:“若非南楚君臣太过无能,哪里有这么容易就攻下建业的道理。”司马雄谨慎的看着周围,答道:“殿下,臣已经得到回报,王宫之内只剩下一些宫女和太监,咱们大雍送来的那些妃子都还在,不过那些皇子都被带走了,尚妃和太子都不在宫里,经过查问,可能是尚维钧把他们送走了。”

李贽想一想道:“那些雍女,你派人去问问,想要回国的,就让她们准备一下,如果赵嘉抓到了,就让她们继续伺候赵嘉,如果没抓到,就安排她们各自回家,尚维钧是尚妃的父亲,很重要,绝对不能让他自杀,好好看着他,把他带回去,其他的南楚官员不用管他们,等我们走的时候再放了他们。”

两人正在慢慢前行,这时一骑飞奔而来,那个骑士到了近前,禀报道:“陈将军斥候回报,已经抓到了赵嘉。”

李贽惊喜地道:“抓住了,在哪里?”

那个骑士道:“陈将军亲自带着人追击,根据内线的谍报,将他们一网成擒,陈将军回报,三千禁军被我们分散消灭,所有王族全部抓到,赵嘉束手就擒,估计明天就可以押送到建业。”

李贽下令道:“传令给黄将军,让他带人去支援,一定要把赵嘉安全带回建业。”

说罢,李贽笑道:“总算达成使命,若是捉不到赵嘉,我们这趟可就白来了。司马雄,记得我交代你的事情么,我有点不放心,你立刻亲自去一趟,一定要确保那里的安全。”

司马雄唯唯遵命,吩咐副手好好护卫雍王之后,他带着疑惑飞奔而去,早在入城之前,雍王就吩咐他派人去北郊一个地方,将那里严密的保护起来,他隐隐知道那人是一个南楚的官员,却不知为什么殿下把那人看的比什么都重要。

赶到北郊之后,司马雄老远就看见那个小庄子,外面有百多名骑士团团围住,水泄不通,司马雄来到近前,看到那庄子匾额上面写着“藏云庄”三个字,虽然司马雄只是粗通文墨,却也觉得那些字清秀飘逸的很。他策马到了近前,守门的段校尉迎了上来,挥刀行礼。司马雄问道:“情况怎么样?”

段校尉答道:“将军,我们围住这里之后,只有一个小孩子出来问是怎么回事,我只答他这是雍王的军令,他就回去了,之后里面就一点动静都没有。”

司马雄疑惑的摇摇头,他也不知道雍王为什么这么做,还让自己前来替他传话。他下了马,上前敲门,没有多久,一个十五六岁的清秀小厮开了门,神情冷静的看着他问道:“请问将爷有什么吩咐?”

司马雄道:“末将司马雄,奉雍王殿下命令,前来求见江哲江先生。”

那个小厮微微一笑道:“将军请进。”

随着那个小厮走进庄子,司马雄心里便觉得心神舒畅,这个庄子虽然不大,但也有几处亭台楼阁,楼阁之间或是流泉淙淙,或是藤萝松竹,一派清新雅致,那个小厮脚步轻快,带着司马雄沿着青石小路,片刻就到了一间隐蔽在绿竹林中的小阁,站在阁门之前的是一个相貌清秀带着几分阴冷的青年人,他含笑看着司马雄,道:“将军原来,我家公子本应亲自迎接,只是公子忝为南楚学士,不便降阶相迎,还请将军恕罪。”

司马雄听了这人声音,先是一阵寒意,继而凛然,立刻握住了腰间佩剑,这个人相貌清雅,声音却是阴柔尖细,司马雄常年在雍王左右,知道只有一种人有这样的特点。他惊疑地问道:“你是什么人,怎会在这里?”

那人目中寒光一闪,道:“奴才李顺,原是在南楚宫里当差的,因为和江大人交好,又不耐烦宫里面钩心斗角,所以前些日子脱身出来,就在公子身边伺候,倒叫将军动疑了。”

司马雄半信半疑的点点头,道:“请带我去见江大人。”

李顺转身,打开阁门,请司马雄进去。司马雄又看了他一眼,走进了小阁。一眼就看到一个相貌斯文俊秀的青年坐在一张书案之后,淡淡的看着自己,在他面前的书案上,摆着一本摊开的书籍,和一些文稿,沾这墨汁的羊毫笔放在笔架上,看来,在自己来之前,他正在写着什么东西。

司马雄看见这个青年,突然想起了这人他曾见过的,三年前,在蜀中,他曾在大营里见过他,当时他是和南楚德亲王一起来的,还曾经和雍王殿下密谈过一阵子,然后就是在饯行宴上,这个青年一曲《破阵子》迫死了蜀王,可惜自己只记得这人是江参赞,却不知道今日自己来求见的江大人就是他。

他反射性的行了一个军礼,这个人是他私下里很佩服的,虽然他还不大明白为什么蜀王听了他的曲子会自杀。他恭恭敬敬地道:“末将司马雄,忝为雍王亲卫统领,奉殿下钧旨,前来问候先生,殿下说今日他军务繁忙,想在晚间前来拜会先生,希望先生肯见他一面。”

我淡淡道:“江某如今是一介草民,又是形同软禁,还有什么资格拒绝雍王殿下的来访,却不知道在下身犯何罪,江某就是为官之时也不过四品侍读,听说三品之上才会被下狱,怎么我这个从前的四品也得下狱么?”

司马雄尴尬地道:“江大人言重,殿下对先生关爱备至,实在是担心先生被乱军打扰,这才派人前来保护,请先生不要见怪,若有不周之处,还请看在殿下面上不要怪罪我们这些粗人。”

我微微一笑道:“既然将军来了,小顺子,倒杯茶来,请将军在这里坐坐。”

司马雄连忙道:“先生不用多礼,末将焉敢打扰先生,如果方便的话,请随便准备一间厢房,容在下可以处理军务即可。”

我看了他一眼,道:“盗骊,你领这位将军到客居休息。”盗骊应声从我身后走出,向司马雄施了一礼,道:“将军请跟我来。”

司马雄看了一眼刚才几乎没有察觉到的小书童,向我告辞离去。

我微微一笑,自言自语道:“难怪雍王殿下名动宇内,一个亲卫统领都如此明礼仪,知进退。”

小顺子低声道:“没想到雍王对你这么关注,你看我们是不是现在就走。”

我摇头道:“天下迟早是大雍的,我若是这么走了,难免成为钦犯,还是等他来说个清楚吧。”

李贽来到楚王宫,命令将各宫殿都封闭起来,自己只留了一间偏殿用来办公,他一边处理军务,一边等候长乐公主的消息,幸好没有多久,一个亲卫前来禀报,说道:“殿下,公主殿下已经平安回来了,就在殿外等候。”

李贽大喜,一边走向殿门一边道:“长乐,长乐,你来了么?”随着他的喊声,一个素衣少女从殿外向他奔跑而来,他一把抱住妹妹的娇躯,笑道:“皇妹,你总算回到二哥的身边了,从今之后,你再不用怕任何事情。对了,护送你来的人呢,梁婉呢?”长乐公主眼中闪过惊恐的神情,道:“皇兄,梁姐姐疯了,其他的人都在外面。”李贽眉头一皱道:“传他们进来。”

随着他的声音,十几个身穿布衣,形容憔悴的大汉走了进来,他们走在最后面的两个人拖着哭闹的梁婉,见到雍王,他们眼中闪过欣慰的神色,下跪拜倒。李贽让他们起来,问道:“发生了什么事,梁婉怎么了。”

为首的大汉道:“殿下,梁婉投靠了太子殿下,而且还是凤仪门主的决定。”

李贽神色一寒,他已经猜到了这件事情,但是想不到凤仪门主如此嚣张,他问道:“你们是怎么知道的,还有,梁婉到底是怎么了。”

那个大汉心想,若是自己再说下去,恐怕会说出不该说的话,那个可怕的人,若是真的因此成了殿下的敌人,那么四面楚歌的殿下形势会更加危急,他再拜道:“殿下,属下在江南不敢丝毫忘记殿下的栽培,今日之事也是不得已,请殿下照顾小人的家人。”说罢,他拔出佩刀,挥刀自刎。在他拔出佩刀的时候,殿上的亲卫都以为他们要行刺,正要上前阻拦,谁知他竟会自尽,李贽惊骇万分,正要再问其他人,只见那些人齐声道:“请殿下代为照顾家人,殿下保重。”说罢一起挥刀自尽,一时间大殿上鲜血横流。长乐公主惊叫一声,掩面不敢回顾。

李贽迷惑不解,他愣愣的看着这诡异的场面,不知道该说些什么。问长乐公主,长乐公主却只知道自己这些人被人掳劫,而自己丝毫没有受到伤害,不久之后,自己和这些密探就被释放了,只是梁婉却疯了,而自己问那些密探,却都沉默不语。听了长乐公主的回答,李贽更加迷惑,到底发生了什么事情呢?

\chapter{第二十三章 归为臣虏}

至化元年十月,李贽突袭建业,借奸细之力,当夜破建业,尽拘百官。当日,长乐公主回宫,随行护卫者均死,至夜,李贽微服往藏云庄,许哲以高官厚禄,哲不从,第二日,国主掳归,李贽以军令掠劫建业,数日,勤王师将临建业,李贽已退,随行军中,尽掳南楚王族、文武百官,哲亦在其中,其时,哲已致仕。

--《南朝楚史·江随云传》

安顿好了长乐公主,李贽带着满腹的疑问,微服到了建业北郊的藏云庄,这次行军匆忙,他一个谋士也没有带,无人可以商议的痛苦让他更急于和心目中的子房相见。到了藏云庄,李贽的心情平静下来,他仔细的想着如何能够将江哲收归帐下,一路上他都在想这个问题,只是想来想去,无论什么法子都没有稳妥的把握,江哲此人,是罕见的没有可乘之机的人物,最后李贽下了决心,无论如何,一定要把江哲带走,否则自己不是白白来了建业。

平静下来之后,李贽走进了藏云庄,按照他的吩咐,雍军没有打扰藏云庄的主人,但是已经控制了庄中上下,在司马雄的引领下,李贽向后园的挽香苑走去,那里是江哲日常流连的地方,李贽可以看到隐在园中各处的雍军勇士。李贽有些担忧的看了司马雄一眼,问道:“江先生没有不满么?”司马雄低声道:“江先生仿佛对我们视而不见,庄子里面的下人很少,除了一个李顺,只有四个小仆人,不过名字奇怪的很,叫什么赤骥、盗骊、骅骝、绿耳的,这些仆人都很听话,没有惹什么麻烦,不过那个李顺末将怎么也觉得奇怪,他是个宦官。”

李贽的脚步顿了一下,道:“赤骥什么的,是穆王八骏的名字,看来江先生果然文采斐然至于那个李顺,本王隐隐约约知道这个人,我们在南楚军中的密探曾经说过有一个监军手下的太监和江哲此人关系十分密切,我原本以为只是一种私人情谊,现在看来这人和江先生的关系非同寻常呢,不过算了,一个内宦,我们也不必去为难他,免得得罪了先生。”

司马雄低声道:“那个李顺,末将总觉得不平常,见了他,就觉得心里发寒。”李贽看了他一眼,淡淡道:“噢,既然如此,你多留心一下就是了。”说着,两人已经到了挽香苑,在苑门外,赤骥和盗骊坐在门前的回廊上,正在低声谈笑,见李贽他们过来,两人站起身来,肃手而立。

李贽笑着问道:“江先生在里面么?”赤骥恭恭敬敬地道:“公子今日身子不爽,用过晚膳就休息了。”

司马雄一听,火气上涌,低声道:“殿下,末将已经告知今晚殿下会来拜访,此人真是太无礼了。”

李贽摆手阻止他继续说话,微笑道:“原来先生休息了,怎么先生身体一直不大好么?”

赤骥恭敬地答道:“公子从蜀中回来就一直卧病在床,前些日子本来已经好转,可是德亲王猝逝,公子上表又遭到贬斥,所以公子旧病复发,如果殿下有什么吩咐,小的就请李总管过来,请殿下训示。”

司马雄手按佩剑,怒气冲冲的看着赤骥,赤骥却是恭谨有礼,面带微笑,毫无畏惧。

李贽想了一想,道:“也好,本王就见见李总管吧。”说罢,李贽就在轩外不远处的小亭子里面坐下来,看着满园翠竹,怡然自得,盗骊和赤骥送上茶点,适逢十分周到,不多时,一身青衣的小顺子走了过来,恭谨的行了觐见皇子的大礼,道:“奴才李顺,叩见殿下,家主人因病失礼,不能前来侍奉,请殿下恕罪。”

李贽抬头看去,只见这个李顺相貌风度果然不凡,李贽在大雍没少见过内宦,但是不论他们地位高低,不论他们是嚣张驯服,他们都有相同的特点,就是他们眼中的自卑,而这个李顺的眼睛却是清冷而冷漠的,他的举止虽然谦卑,但是李贽可以感觉到他的骄傲,那是一种主宰生死的骄傲,李贽记得很清楚,他曾经见过这样的眼神,那是他第一次见到凤仪门主,当年他随父皇南征北战,一次行军途中,凤仪门主飘然而至,和李援一夕相谈,十分投机,不久之后,大雍就得到了白道武林的支持,而父皇身边也多了一个纪贵妃,李贽永远记得凤仪门主的眼睛,那是一双温柔慈悲、悲悯众生的眼睛,但是李贽也永远记得,当他率军攻打杨老生的时候,出手相助自己刺杀杨老生身边的大将之后,凤仪门主在一瞬间散发出来的惟我独尊的滔天气势,也就在那一刻,李贽生出了对凤仪门提防的心意。见到李顺的气质,李贽突然明白,这人一定是一个绝顶高手,而且是有望成为凤仪门主的对手的那种人物。

想到这里,李贽温和地道:“本王曾听说过关于李总管的一些事情,若是本王没有猜错,李总管也曾经参与过蜀中大战吧?”

李顺惊讶的看了李贽一眼道:“殿下居然知道小人一个奴才的事情,奴才和公子多年相识,承蒙公子经常照顾,如今建业混乱,索性就弃了那虚假的荣华,在公子身边吃碗闲饭,若是殿下要加罪奴才这个宫里面的人,奴才自然不敢反抗的。”

李贽摆手笑道:“两国交兵,干你们这些苦命人什么事情,何况如今李总管在江先生身边,日后本王还要李总管多多美言几句,看来江先生怒气很盛呢?”

李顺眼中闪过一丝好感,道:“公子虽然被迫致仕,可是毕竟为南楚效命多年,如今眼看江山社稷危亡,若是公子反而心喜,就是到了哪里也是说不过去的,而且殿下今次作战,意图不明,我家公子百思不得其解,若是殿下肯跟小人说说,小人转告公子,或许能搏公子一笑。”

李贽心里一动,莫非江哲对自己并非十分排斥,便坦然道:“这次攻打建业,若在江先生看来,可能觉得李贽胡闹,可是实在是祸起萧墙,李贽日日如履薄冰,如果不能得到江先生辅佐,只怕李贽性命不久,还请李总管代李贽转承心意,这次无论如何,都要请江先生随在下回大雍,若是江先生不肯眷顾,只怕李贽无福,再也不能恭聆教益了。”

李顺施礼道:“殿下如此器重公子,奴才代公子拜谢,请问殿下,我家公子只爱山川之美,既无济世救民之心,也没有建功立业之念,不知殿下凭什么要我家公子呕心沥血,却恐怕只能落得一个将来鸟尽弓藏、兔死狗烹的结局。”

李贽站起身来,诚挚地道:“我不敢说一定能够君臣相安,但是李贽绝不是妒贤忌能之人,也不是只能共患难不能共富贵的越王,本王知道江先生不爱富贵荣华,也不爱建功立业,但是若是天下纷乱,只怕江先生也不能平安度日,如今我大雍内患就在眼前,南楚群龙无首只怕很快就要陷入混乱,北汉虽然还算稳定,可是那里重武力,却不尊重士子,蜀中之人若是听了江先生之名,只怕报复之心胜过敬重之意,不是本王言辞威胁,若是我大雍不能一统天下,只怕滔滔乱世,再无净土。若是江先生肯助本王一臂之力,本王可以保证,将来先生可以在大雍安居乐业,贽与先生共享荣华。”

李顺想了一想,道:“殿下情真意切,奴才自会一字不差的禀报公子。”说罢,李顺躬身行礼,然后退了下去。李贽坐在亭子里,他心里充满了期望,从李顺的话里,李贽可以察觉到江哲并非完全拒绝,只是顾虑颇多罢了。

过了片刻,李顺回来了,道:“公子请奴才转告殿下,效命之事关乎公子一生荣辱,不能随意决定,如今殿下军务繁忙,还请殿下速回营中,公子说,殿下俘虏了尚维钧尚相爷,尚相爷是尚妃生父,不可慢待。如今太子和尚妃还在逃,若是殿下希望将来平南楚容易一些,还是不要过分追捕的好,国主出奔,若是殿下已经抓住了他们,那是最好。”

说到这里,李顺看了雍王一眼,李贽点头道:“明天赵嘉就会被送到建业。”

李顺继续道:“国主庸碌,昧于谗言,如今身陷囹圄,社稷不保,天下轻之,就是留在南楚也没有什么用处,若是带回大雍,性命不过数年,恐怕难以生还,只怕南楚臣民会因此深恨大雍,昔日楚怀王客死秦国,楚人大恨,曾有‘楚虽三户,亡秦必楚‘之言,日后大秦果然亡于楚人。”

李贽忧虑地道:“可是我这次兴兵建业,若不能将赵嘉和百官掳回,如何向父皇复命呢?”

李顺淡淡道:“公子也知殿下为难,所以又说,如果万不得已,必须将国主带回大雍,不可轻易伤害其身,应该立刻撤兵休战,和南楚谈和,让新君割地输诚,赎回被掠君臣,则一可以消减南楚国力,二可以免得和南楚结下不解深仇。”

李贽深思良久,才道:“多谢江先生良言,不论先生是否答应为李贽效命,李贽都对先生感激不尽。”

看着李贽的背影,李顺露出了一丝笑意,这是我特意让他代为接待雍王,让他用自己的眼光看看李贽是否值得跟随,他的答案是,值得。

听着小顺子详细的回禀,我放下手上的书卷,淡淡道:“看来,李贽对我是势在必得了。”

小顺子道:“公子,你的意见呢?”

我淡淡道:“雍王殿下有一句话倒是很让我动心,若是天下纷乱,我又哪里有可以安身之处呢?”

小顺子道:“何况还有太子李安,若是那人真是李安,公子要报仇不免要借助雍王的势力。”

我叹息道:“是啊,杀死李安未必困难,可是善后就麻烦了,可是我也不想就这么容易跟了李贽,当初我曾想为德亲王尽力,可是容渊却让我放弃了,李贽是明君,我还要看看他身边是否有贤臣。这样吧,我不会答应向他效力,就暂时这么拖着,我想我们就作为俘虏到大雍去吧。”

小顺子苦着脸道:“这也太屈辱了,公子居然要去作俘虏,座上客不做,要做阶下囚。”

我微笑道:“只怕现在做了座上客,将来就连阶下囚也做不成了呢?”

第二天,赵嘉被雍军带回了建业,一见到雍王,赵嘉连连苦求道:“孤对大雍从无反叛之心,望殿下看在王后之面,放孤一条生路。”

李贽只是温言相劝,只说父皇想念女儿女婿,想要接他们到雍都一家团聚。赵嘉苦苦相求,最后只得垂泪应允,最后要求见王后长乐公主,却被李贽说长乐公主受了惊吓,所以不便相见。

又过了几日,李贽将建业上下搜刮了一遍,载着国主、王族、妃嫔、百官离开了建业,当日南楚君臣痛哭失声,相送的百姓也是相顾流泪,可是在雍军的铁骑面前只能忍泪吞声。李贽坐在马上,看着两旁冰冷的眼神,苦笑道:“看来南楚民心还没有失去啊。”

随侍在侧的司马雄道:“是啊,不过他们可没有反抗的勇气,不然咱们只有两万人,他们就是一人来砍一刀,我们也完蛋了。”

李贽淡淡道:“南人阴柔,但是也不可小看他们的力量,如果我们威逼的太狠,只怕他们会拼了命和我们为难,他们擅长阴谋,到时候我们可是会处处荆棘呢。”

司马雄听到“南人阴柔”四个字,不由冷哼道:“南人真是心思深沉,殿下对那个状元江哲如此礼贤下士,可是他至今不肯答应归降,殿下如今将他作为俘虏带回去,看他还神气什么?”

李贽不由苦笑连连,他也没想到,从那日之后,他几次去求见江哲,江哲不是托病,就是匆匆一见就告退,始终不肯和自己深谈,自己屡次向李顺打听江哲的心意,李顺也是含糊其词,只是隐隐约约说,江哲不愿到大雍为官。最后迫不得已,李贽只得将江哲强行列入俘虏名册,带回大雍,他亲自去向江哲告罪,江哲却也只是淡淡一笑,似乎并不恼怒,等到上路的时候,江哲只带了李顺一个人,其他几个小厮都被他赠银遣散了,径自到了俘虏营中,他和很多官员都相熟,交情虽然不深,但是还算可以谈得来,他从容自若,倒是让不少忧心忡忡的官员心情好了很多。李贽很是担心彻底得罪了江哲,这几日真是寝食不安,可是南楚四方的勤王军队拼命向建业进攻,雍军已然有些抵挡不住,他必须尽快离开建业了。

长乐公主也随军北返,虽然收到了惊吓,但是长乐公主一想到可以回大雍,心情就开朗许多,只是这几日即将离开大雍,李贽便觉得长乐公主总是欲言又止,神色间有些怔忡,李贽几次相问,却被都长乐公主敷衍过去,但是李贽见长乐公主并非是关心赵嘉,也就没有过分关心,反正回去之后,自有长孙贵妃劝解。至于疯癫的梁婉,仿佛成了幼儿一般,每日不是哭闹,就是嬉戏,李贽军中没有凤仪门高手,只得让人严加看管,再派了一些宫女去照顾她。

李贽想着自己遇到的这些事情,真是苦涩难言,自己这趟攻打建业,是否走了一步歪棋呢,至少自己看到的眼前这些收获,将来可能都会变成自己亲自服下的无解毒药啊。

就在里边看着雍军离开建业的人群中,陈稹和寒无计冷冷的看着雍军铁骑,寒无计低声道:“其实,若是救出公子不是什么难事,可是公子却偏偏不肯。”

陈稹淡淡道:“你不知道,公子和雍王一直是有联系的,虽然是为了南楚居多,但我看公子对雍王还是很看重的,这次雍王求贤若渴,听赤骥传来的消息,根本是摆明了冲着公子来得,公子怎能不感激他的器重呢,只是公子还记挂着德亲王,对南楚还有几分情谊罢了,才宁愿作为俘虏随军。”

寒无计冷冷道:“其实公子就是心肠太软,当初公子为了南楚尽心竭力,若没有公子,我们蜀国不会败得那么容易,那个德亲王对公子也不是全心全意的倚重,偏偏公子就是放不下,当日还亲自到襄阳去救德亲王,可惜南楚国主庸碌无能,逼死了德亲王,令公子伤心失望。”

陈稹叹息道:“是啊,从襄阳回来,公子几乎旧病复发,还是李爷百般劝慰,公子才不再伤心。”

寒无计苦涩地道:“公子在南楚,和我们在蜀国,都是一样痛心啊,你平日虽然总是自诩冷漠无情,我不信你对蜀国就没有眷恋。”

陈稹沉默半晌,道:“蜀国待我刻薄寡恩,我如今想起来,也觉得有几分怀念,南楚待公子还算优容,也难怪公子始终不忍舍弃啊。”

\chapter{第二十四章 千里路遥}

经过几天的急行军之后,雍王和其他的雍军会合了,雍王十几万的军队步步为营向大雍境内撤退,其他负责阻截的军队让随后赶来的南楚军队不得不遥遥相送,所以接下来的行军是从容而舒适的,作为俘虏的我因为得到优待,不用和其他俘虏同住,雍王下令为我和小顺子单独准备了一个营帐,虽然是行军营帐,但是十分舒适讲究,地上铺着厚厚的锦毡,帐篷四周的缝隙都用毛皮紧紧地包裹起来,秋天的寒风一丝也不会吹进来。帐篷的一角放着一张大床,足可以让两个人安睡,帐篷的另一边放着一张松木方桌,两边摆着两把椅子,桌子上摆着一套紫砂茶具,而在帐篷中心放着一个精巧实用的铜火炉,现在上面放着一壶开水,使得整个帐篷都是暖洋洋的。

小顺子听水开了,熟练的替我泡上一杯热茶。我伸了一个懒腰,坐起身来,多年来几次事故,让我染上了病根,虽然我坚持练习养生的气功,但是还是会不时旧病复发,我也曾经想好好医治一下,可是心病难医,再加上医者难以自己医治,所以这几年我总是病恹恹的,虽说是托词养病,但是我的身体倒真的不是很好。小顺子服侍我坐起,抱怨道:“公子总是不肯好好休养,这次去大雍一路风尘,只怕公子又要犯病了。”

我叹了口气道:“这有什么法子,你也知道我的病是怎么来得,大半倒是心病,其实我现在已经好多了,只是这次行军让我又想起当年攻蜀的事情,可惜德亲王已经驾鹤西归了。如今我已经身在大雍军营,想起往事,不由令人扼腕。唉。”

这时,帐外传来朗朗的笑声道:“听说江先生身子不适,本王特来探望。”随着笑声,雍王李贽走了进来,他一身皇子服色,在他身后还跟着两个谋士,三个人走进帐来,我勉强要下床,李贽已经走了过来,按住我道:“先生不用起来,听说先生病了,贽军务繁忙,这才抽出时间来看望,真是失礼。”说罢,坐在我床边,担忧的看着我的面色。

我见那两位儒生也已经坐了下来,欠身道:“随云旧病复发,不能下床,还请诸位见谅,久闻雍王殿下身边人才济济,不知两位如何称呼。”

一个年纪已长、相貌清秀、五十多岁的中年儒士起身道:“北海管休见过江先生,先生文才誉满天下,管某曾读先生诗文,齿颊流芳,不忍逝卷啊。”

另外一个细眉长目,气度风流的白衫儒士也笑道:“当日先生一曲破阵子迫死蜀王,至今在下仍然心中念念,晚生董志。”

我淡淡道:“久闻雍王帐下谋士,北海管休擅长整顿粮草兵甲,洛阳董志擅长行军布阵,还有一位颍川苟廉,擅长出使四方,此三人并称三杰,今日一见果然名不虚传,可惜三杰只见其二,当真令随云叹息府薄缘浅。”

董志笑道:“苟兄如今不在中军,所以不得相见,他对先生也是十分敬仰,当日先生随故德亲王至大雍军中,我们三人恰好都不在军中,后来匆匆而别,也没有机会促膝详谈,如今先生也归了殿下麾下,想必日后可以把酒相谈了。”

我看看李贽,微微一笑,没有反驳董志的言语,免得他面上难看,只是淡淡道:“随云虽然多病,但是精神也还健旺,如果董兄有什么质询之处,尽可来问随云,随云敢不尽言。”

说了片刻,他们见我精神倦怠,便告辞而去,李贽频频嘱咐我好好休息,说已经安排了马车载我同行,又嘱咐小顺子好生照料,若是需要什么尽可向管休索取。

等到他们离去了,我靠在床上,笑道:“雍王这几个谋士倒是热诚得很,不过我看那个苟廉怕是有些量窄的,否则为什么雍王殿下没有带他同来呢?”

小顺子笑道:“这会儿公子倒是聪明了,那个苟廉也在营中,只是他性情不好,所以雍王没有邀他前来,免得立刻得罪了公子。”

雍王和两个谋士走出营帐,叹息道:“我原以为他是托病罢了,不料竟真的卧病不起,唉,他身子不好,我又迫他远行,怪不得他始终对本王冷淡非常。”

董志宽慰道:“殿下不必忧心,我见此人虽然卧病,但是精神很好,想必并没有因为殿下带他同行而恼怒,我虽然看不穿此人心事,但是我说他从了殿下,他也没有当面反驳,可见此人不是不可以降服的。”

李贽苦笑道:“江哲此人甚是随遇而安,我就是强行给他一个官职,他也未必会峻拒,只是若想让他真心效命,就是难事了,当初德亲王赵珏对他颇为看重,只是或者有些勉强,他便不肯再随军效力,德亲王还是南楚王叔,他就如此敷衍,我担心他也会这样敷衍我。”

管休道:“殿下安心,虽然此人心冷如冰,但是还是有一腔热血的,否则也不会上表直谏,只要殿下诚心相待,必然能够得到他的忠心,我听说当日他和德亲王疏远,倒多半是因为德亲王幕僚容渊的缘故,我倒是担心苟廉的性子,这人难得服人,总是要挑衅几回,只怕会惹恼了江随云。”

董志道:“管兄过虑了,我倒认为苟廉若是去了,恐怕会有意想不到的成绩,江哲此人外表虽然温文尔雅,但是内心倒是桀骜不逊的,和苟兄倒是性子相近,我看不会有什么不良后果的。”

就在三人在这里反复探讨的时候,他们担心的苟廉已经到了我的帐前,苟廉此人最是心高气傲,得知李贽到建业特地把江哲强行带了回来,又对他十分关爱,苟廉心里就已经不是滋味,这回李贽带着管休、董志去探病却不带自己,苟廉心里便是一阵不舒服,以他的聪明才智自然是知道李贽他们担心自己得罪了江哲,这让他更加不甘心,所以他趁着李贽他们离去不久,就来到我的帐前,我现在还是雍军的“俘虏”,虽然李贽下令不许人去打扰我,但是苟廉在军中的地位是很高的,所以看守我的军士也没有阻拦他,就让他施施然地走进了我的帐篷。

我一看到这个鹰钩鼻子的年轻人就猜到了他的身份,看他毫无礼貌的站在我面前打量了我半天,挥手阻止小顺子的怒火,我笑道:“请问可是舌厉如刀的苟廉苟永泉么?”

苟廉微微一怔,道:“想不到一曲送了蜀王性命的江随云也认得我这个小人物,真是荣幸之至。不知道昔日讽刺蜀王‘一旦归为臣虏,沉腰潘鬓消磨‘的状元郎是否早就知道今日之事,我见兄台形容憔悴,病体支离,应该也算的上‘沉腰潘鬓消磨‘吧。”

我淡淡道:“久闻永泉兄曾为故徐州将军张郴之幕僚,张郴不奉正朔,率兵割据地方,永泉兄当时在其帐下颇受荣宠,后来雍王殿下讨伐张郴,阁下奉命出使雍营,谁知折服于殿下威严,回去之后劝服张将军弃甲归降,日后阁下替雍王殿下出使四方,不辱使命,却不知是否因为最初替张郴出使,反而被人说降的羞辱,让阁下日后悬梁刺股,后来才有所成就呢?”

苟廉面上一红,他劝服张郴归降之事,虽然誉为美谈,但他自己总是觉得没有说服雍王退兵,反而成了雍王劝降的说客,未免有辱使命,想不到被人一针见血的揭穿。他赧然道:“雍王殿下龙凤之姿,雅量高致,岂是言辞可以动摇,在下铩羽而归也没有什么奇怪,而且在下挽张将军于水火,此功亦可补过,倒是阁下,既然知道大雍才是正统,为何不奉正朔。”

我笑道:“永泉兄此言差矣,我说张将军不奉正朔,乃是因为当日中原一统就在顷刻,人心归附,张将军倚仗兵势,不识时务,故而轻之,而我南楚虽然小国,然立国之久尤在大雍之上,随云曾是南楚状元,一甲进士,任职翰林院多年,深受国主重恩,焉能抛弃君上,改奉大雍,随云颇知廉耻,旧主尚在,怎能趋炎附势侍奉新主。”

苟廉眼珠一转,道:“阁下既然决心侍奉南楚,如今南楚国主已在我营中,赵嘉都屈膝侍奉我大雍,阁下为何如此执拗,何况我听说贤臣择主而事,赵嘉昏庸,迫死贤王,而我主雍王殿下虚怀若谷,礼贤下士,行事更是明决果断,仁义贤能之名布于天下,阁下为何抱残守缺,不肯归顺,以至为天下所笑。”

我冷冷一笑,道:“虽然贤臣择主,我未闻有旧主尚在,便侍奉新主的贤臣,昔日豫让侍奉智伯,是在中行氏亡后,中兴氏不过以凡人待之,豫让也未曾弃之,何况随云昔日所从,非是赵嘉一人,乃是南楚王室,先王加我翰林,德亲王用我参赞,恩情尤在眼前,焉能一见荣华富贵,便立投新主。”

苟廉正容道:“虽然阁下之言,句句金玉,然阁下早遭贬斥,何必如此痴心。”

我淡淡道:“昔日比干剖心,其志不改,屈原遭贬,闻楚怀王事,尤自沉江,随云并非痴人,不敢效法先贤行事,然而昧于荣华,投靠新主以求富贵,我不敢为此。”

苟廉听到此处,只得下拜道:“先生品质高洁,在下钦服,然而殿下有王者之姿,先生若是错过,未免可惜,但先生卧病军中,永泉不敢相强,至雍都千里路遥,永泉可否前来打扰,恭聆教益。”

我笑道:“永泉兄天下闻名,是随云应该多多请教,途中寂寞,若是阁下有暇,不妨前来屏烛夜谈,只是随云虽然博览群书,却对琴棋书画不甚了了,听说阁下于此颇有声名,还请阁下不吝赐教。”

李贽得知苟廉私自来见我之后,原本十分担忧,立刻派了人前来劝解,谁知那人来到,却见我和苟廉相谈甚欢,李贽闻之,不由喜形于色,从那之后,常常让帐下幕僚前来相陪,我也不会拒绝,多日促膝详谈,我对雍王帐下幕僚不由十分赞赏,管休对钱粮文案之事十分精通,董志精于兵法战阵,一谈起来便滔滔不绝,而苟廉博学多才,和我最是谈得来,只是他个性好胜,总喜欢和我辩论疑难,和这几个人日日相谈,我的心情倒也不错,再加上小顺子的仔细照料,我的病体在路上渐渐痊愈了。

我对他们的观感不错,他们对我也是十分钦佩。

管休擅长军务钱粮,是雍王亲信的主薄,可是他一和这个年轻人谈论起来,就发现不论自己说些什么,他都立刻心领神会,偶尔说上一两句,也都是切中要害,后来江哲无意中说曾在德亲王帐下处理过文书,这才让管休知道为什么这个翰林学士对这些琐碎的事情也如此了解,他原本以为江哲曾在德亲王幕府,不过是参赞军机罢了。

董志擅长兵法,可是和江哲辩论起来,却发现古今战阵,江哲无一不知,就是自己不甚了了的部分,江哲说起来也是头头是道,解释起来十分详尽,问他如何得知,这个青年笑着说曾在镇远侯陆府遍读兵书,后来在翰林院也曾经整理过兵书战策。董志原本想江哲不过是纸上谈兵,所以试着和他演习兵法,不料江哲用兵如天马行空,无迹可寻,每每从最不可思议之处而奇峰突起,但事后想来,却又入情入理,妙到巅毫。董志心服之后,也不免好胜,就和他辩论起作战的器械,不料江哲也能够说的条条是道,后来江哲虽然多是默然不语,但是若是偶一出言,就让董志想个半天,转天就去研究改进器械。

苟廉对江哲最是佩服,他原本自负博学,不料江哲在南楚曾经参与筹立崇文殿,所读过的书何止千万,每次争论文章,江哲往往旁征博引,让苟廉瞠目结舌,至于舌辩之术,虽然江哲不大常用,但是苟廉若是洋洋得意,不可自拔的时候,江哲往往一句话就让他心悦诚服。

令三人私下最佩服的就是,虽然江哲才华如此,为人却是恬淡自然,和他相谈的时候如同春风沐雨,只觉得其才华横溢,却不觉咄咄逼人,只有到了夜深人静之时,才会令人冷汗直流。到了后来,三人争胜之心越发急切,江哲却往往轻轻退却,让三人一腔热火化作春风,良久才会发觉江哲并未应战。

千里征程,虽然遥远,但是终有尽时,到了即将抵达雍都的时候,三人再次联袂求见李贽,要求他一定要把江哲收为麾下。苟廉最是激烈,道:“殿下若不能将此人收到麾下,真是可惜之至,此人之才,胜我等数倍,若是与之为敌,只怕我等尸骨无存。”

李贽苦着脸道:“众位先生,本王如何不知此人的重要,可是本王每次前去劝说,此人都默默不语,让本王毫无办法。”

管休道:“殿下不必着急,此人对殿下颇为敬重,对我们也没有什么敌意,应该不至于峻拒至此,这次回京,我们将此人送到雍王府软禁起来,慢慢劝解,总有办法的,何况石子攸宽厚仁德,一定能够开导于他。”

李贽叹息道:“也只有如此了,若是石子攸再不能说服他,本王,本王,唉,叫本王如何舍得。”

管休三人面面相觑,都知道李贽动了杀机。

“帘外雨潺潺,春意阑珊。罗衾不耐五更寒。梦里不知身是客,一晌贪欢。独自莫凭栏,无限江山,别时容易见时难。流水落花春去也,天上人间。”

我披衣站在窗前,这里是驿馆,明日就是我抵达雍都的日子了,我吟诵着新写的《浪淘沙》,心中无限寂寞,想起南楚迷人的风光,更是心中百转回肠。小顺子走到我身边,低声道:“公子,这些日子以来,你折服李贽的帐下谋士,对李贽却始终不肯青眼相加,如果李贽动了杀心,你该如何是好?”

“小顺子,你不明白,从前我不过是随遇而安,在谁那里为官都无所谓,就是在德亲王面前,我也不过敷衍罢了,可是雍王殿下心如明镜,我若投他,若不能推心置腹,那么雍王殿下不会满足,而且也解不了他的危局,若要我竭尽所能,那么我就要看看雍王的气度,我是存心逼他杀我的,如果他最终肯放手,我才当他是明君圣主,若是他最终动了杀机,那么他也不过是霸主雄才,与其日后我时时担忧他诛杀功臣,不如我今日试试他的胸怀,他若能终究放过我,那么我相信日后可以君臣善始善终,若是他--,我正好诈死脱身。”

小顺子面上露出焦急的神色道:“公子,雍王殿下势力极大,若是要杀你,如何能够脱身,我的武功虽然不错,也不敢保证可以救出公子。”

我淡淡一笑,道:“我想雍王殿下为了不伤天下名士的心,必然不会真刀真枪的杀我,用毒是最好的法子,我已经准备了一粒珍贵的毒药,到时我服下之后,僵硬如死,偷一个人困难,偷一具尸体还不容易么,待我脱身之后,隐蔽在雍都,等到可乘之机,我再趁机报了杀妻之恨,到时候,小顺子,你我就可以浪迹天涯,隐姓埋名,岂不快哉。人常说读万卷书,行万里路,我可是期待的很呢?”

小顺子宽心地道:“那我倒要期待雍王要杀公子呢,免得牵累公子去给他呕心沥血。”

我微微一笑,想让我呕心沥血,可不是什么人都有资格的,说句实话,我的这个试探恐怕没有人能通过,不为所用,必杀之,是那些英明君主不可言表的心思,可惜,雍王真是一个令我钦服的人呢,我有些遗憾的想着。

\chapter{第二十五章 初至雍都}

大雍武威二十三年(南楚至化元年)十一月二日,雍王得胜还朝,列南楚国主赵嘉、皇子、妃嫔、王族在前,列文武百官在后,献俘太庙。

--《雍史·太宗本纪》

我没有看到雍王被百官迎接进入城门的盛况,因为我如今的身份好说是一个客人,难听的说法就是一个俘虏,我既没有被献俘太庙的雅兴,也没有这份荣幸,所以我是和苟廉一起在大军入城很久之后才乘车进城的。穿过明德门,我将乘坐的马车的车窗打开,看见的是一条宽达四十丈的御街大道两旁,植有两行槐树,只是已经入冬,再也看不到绿树成茵,道路两边都有宽如小河流一般的排水沟,在和其他主要道路的排水沟交叉之处,均铺架石桥,如今虽然是寒冬,但是水沟之内热气腾腾,流水不绝,却令槐树之上积雪冰挂,充分显示出严冬的萧杀。

我低吟道:“山河千里国,城阙九重门。不睹皇居壮,安知天子尊。”

苟廉笑道:“秦中自古帝王州,长安文物荟萃,地势险要,南有秦岭中部为叠嶂,北有众山逶迤延绵,和秦岭遥遥呼应,泾、渭等八水环绕长安,八百里秦川自古以来就是帝王之资,大雍据长安为都城,正是王者气象,大雍一统天下,其势不可阻挡,南楚建都建业,建业天子气不足,建都于焉,常常一代而衰。”

我但笑不语,南楚的衰弱我心里很清楚,大雍的强盛我也很清楚,可是这并不是我必须投靠雍王的理由。苟廉眼中闪过一丝迷茫,他从未这样头疼,无论自己如何劝诱,这个青年或者赞同,或者微笑不语,但是始终不能让他答应投靠雍王,苟廉觉得是否自己太过着急,可是若是不能将他说服,若是雍王无法忍耐了,决定处死此人,岂不是太可惜了,苟廉曾经提出将江哲暂时软禁,慢慢相劝,可惜雍王只是苦笑不语,似乎时间很紧迫,这又是为了什么呢?

我指着窗外对小顺子道:“你看,这是朱雀大街,贯通长安城南北的第一长街,朱雀大街北端尽头,就是宫城和皇城,乃是大雍皇族所居,长安城内六部的官衙也在宫城之内,而我们现在所在的位置叫做郭城,长安郭城从左、右、南三方拱卫宫城和皇城。长安郭城共有南北十一条大街和东西十四条大街,纵横交错地把郭城内部划分为一百一十坊。其中贯穿城门之间的三条南北向大街和三条东西向大街构成长安城内的交通主干,而现在我们所在的朱雀大街就是长安最中心的街道。朱雀大街的尽头就是朱雀门,从那里可以进入宫城。”

苟廉笑道:“听江先生这样一说,我倒觉得仿佛阁下才是长安的地主呢?”

我淡淡道:“若是说起建业的情形,只怕永泉兄比在下还要了解呢。”苟廉再次苦笑。

我望着窗外熙熙攘攘的人群,这里的繁华比建业还要胜过几分,可是建业多得是纸醉金迷,士子淑女,这里却是慷慨激昂的儒生,雄姿英发的武士,到处流露着盛世气象。我笑了,这是真正的欢喜,江南虽好,又是故乡,可是我不会因此不喜欢这里,南楚,那是我记忆中的事情了。

马车很快就到了朱雀门,苟廉掀开车帘,手里是一面雍王府的令牌,守门的禁军看了一眼,恭恭敬敬的退下,苟廉正要吩咐继续前行。前面传来爽朗的笑声道:“苟先生,车里面可是皇兄的贵客。”苟廉抬头看去,却见前面驶来一辆华贵的马车,帘幕都是绣着金龙的锦缎制成,高挑的车帘后,一个英挺俊美的青年在两个娇美的侍妾服侍下半倚半坐,正在向自己招手。苟廉大为惊讶道:“齐王殿下,为何没有去参加庆功宴,反而要出城呢?”

李显在侍妾的搀扶下走出马车,道:“庆功宴么,还没开始呢,得等父皇告祭太庙之后才举行,本王早就告了病了,我听说二哥带了贵客回来,想着应该是本王的旧识,怎么也得来迎接一下,可是江大人么?本王是李显啊。”

我有些无可奈何,虽然明知此人会来搅局,但是这样急迫还是出乎我的意料,我探出头去,微笑道:“原来是齐王殿下,怎么来嘲笑我这个阶下之囚么?”

李显走到近前,朗声道:“什么话,江大人是绝世才子,别说皇兄,就是我父皇也不会让大人身陷缧绁,虽然皇兄呈上来的折子里面有大人的名字,不过父皇看了之后就划去了大人的名字,还说让皇兄好好招待大人,不可怠慢,过些日子,父皇还想召见大人呢。不过我跟父皇讨了旨意,若是江大人愿意,我的齐王府恭候大人上门。”

苟廉眉头一皱,心想,怪不得殿下心有苦衷,原来是知道有人会和殿下抢人,连忙道:“殿下,雍王殿下早有钧旨,命晚生好好接待,齐王可不能抢着作主人啊。”

李显蛮横地道:“就是皇兄在此,也不会和本王为难,江大人,昔日在南楚,你奉命招待本王,今次该轮到本王作地主了。”说着就伸手向我拉来。然后李显就觉得手腕被一只冰冷的手握住,然后他看到小顺子冰冷的笑容。李显识趣的收回了手,轻快地道:“既然苟先生这样坚持,本王只好算了。过几日江大人一定要到本王府上小住才行。”

我微微一笑,点头道:“若有机缘,自然要叨扰的。”苟廉忧心忡忡的看着我,欲言又止。

此刻大雍的金殿上正是一派君臣同欢的景象,今日献俘太庙之后,李援按照礼部制定的庆典依次完成了扫尘、大赦、接受百官朝拜、阅兵等等繁文缛节,总算到了金殿庆功的时候,李援在开宴之后,赵嘉和长乐公主被宣诏上殿,李援对着不停请罪的赵嘉只是淡淡的说了几句什么“翁婿之间,情分深厚,不会多加罪责”,便让赵嘉到驿馆暂时居住,至于长乐公主,李援一见便是泪流满面,等到长乐行礼之后,便拉了长乐的手,上下打量,看女儿容颜清减,混不似当初天真烂漫的模样,更是心痛,对长乐说道:“皇儿,你受苦了,父皇对你不起,你母亲他们都在后宫等你,你先去拜见,父皇晚些时候再去看你。”长乐公主在一干宫女内宦的簇拥下往后宫去了。

李援这才举起酒杯,高声道:“今日雍王得胜而归,朕虽然高兴雍王伐罪建功,却更喜他接回了爱女长乐,朕不胜酒力,众卿可要代朕多敬雍王几杯,今日君臣欢宴,不醉无归。”大殿之上群臣高呼万岁,同举金杯,喜笑颜开,雍王李贽已经洗去征尘,就在太子李安的下首席位上接收百官的敬酒,坐在上首的李安虽然笑语不断,但是目中的寒光却是连连闪动,他心中痛恨至极,原本安排齐王李显出征南楚,谁知损兵折将,无奈何只得让雍王李贽去啃这个硬骨头,不料雍王偷袭建业,掳回了南楚君臣,让李援欣喜若狂,却让李安气愤不已。

尤其令李安痛恨的是,他好不容易得到了南楚密谍情报网的负责人梁婉的归附,梁婉又成了白痴回来,自己在南楚所下的功夫化为乌有,岂不令李安沮丧愤恨。看着春风得意的李贽,李安恶狠狠的想:“若是本王得不到皇位,你李贽也别想如意。”

在李安切齿痛恨的时候,后宫之内也是乱纷纷的一片,皇后窦氏是太子李安的生母,长孙贵妃是长乐公主的生母,还有齐王的生母颜贵妃,以及纪贵妃四人聚在皇后宫中,不久之前,得报长乐公主的香车进了皇城,几人就在这里翘首以待。长孙贵妃这些年几乎泪眼哭干,几个儿子都没有留住,唯一的女儿又远嫁南楚,这次听说雍王接回了女儿,长孙贵妃早就坐立不安,没有多久,门外传来急促的脚步声,几个太监宫女进来禀报,公主已经在宫外候旨。皇后窦氏连忙道:“还候什么旨,还不让孩子进来。”

不过片刻,素衣素服的长乐公主走了进来,忍着眼泪拜见皇后,然后便仆到母妃怀里大哭起来,长孙贵妃更是哭得摧心断肠,她看着长乐公主憔悴的花颜,悲声道:“我的贞儿,你十五岁远嫁南楚,六年来娘亲每日焚香祝祷,既盼我儿夫妻和睦,又担心两国交战殃及孩儿,如今你总算平安归来,娘的心才安定下来,贞儿,你放心,你父皇答应为你另择佳婿,这一回娘亲为您作主,总要为你找个称心如意的郎君。”

皇后窦氏也一边流泪一边道:“好孩子,你在南楚受苦,哀家也是为你寝食难安,这一次哀家已经跟皇上说了,你为大雍已经牺牲良多,谁也不许再在你身上打主意,这次你若看中了什么人,哀家替你作主。”

长乐公主掩面道:“娘娘,母妃,长乐遵从皇命远嫁南楚,虽然如今回来了,但是总是南楚王后,孩儿就是再没有廉耻,怎能夫婿尚在就改嫁他人,还请几位娘娘替孩儿作主,就让孩子留在母妃身边,清清静静的待上几年,好好孝顺父皇母妃吧。”

几位娘娘面面相觑,想起来也真是为难,无论自己人怎么说,长乐终究是嫁了南楚国主,总不能这样安排他改嫁吧,长孙贵妃想起自己先后夭折的两个皇子,唯一的女儿又是这样苦命,更是痛哭不已。这时纪贵妃走到长乐身边,柔声劝慰道:“公主不用难过,皇上自然会安排的妥妥帖帖,绝不让公主难堪。”几个娘娘知道纪贵妃素来参与军国大事,见她这样说,都放了心,几位娘娘都是后宫妇人,什么阴狠毒辣的事情没有见过,既然皇上有心,那么赵嘉自然命不久长。长乐公主听了不由心里柔肠百转,她对赵嘉虽然没有什么情意,但是赵嘉对她倒是始终恭恭敬敬的,如今到了这种地步,自己成了陷害夫君的恶毒妇人,不禁泪如涌泉。

纪贵妃性子开朗,连连说笑,总算让长乐公主消去愁容,长孙贵妃也满脸笑容地道:“贞儿,娘已经将你从前住的翠鸾殿重新打理过了,来,跟皇后和几位娘娘跪安,咱们去看看你的住处。”

皇后等人也都笑着让长孙贵妃快去安顿长乐,纪贵妃道:“哎呀,就让姐姐一个人张罗,倒好像我们这些人不疼长乐,妹妹我年纪轻,就让我去打个下手吧。”

纪贵妃原本最是高傲,见她刻意奉承,长孙贵妃自然不会拒绝,三人辞别了皇后就向翠鸾殿走去,这翠鸾殿里面已经是焕然一新,长孙贵妃亲自挑选的宫女内宦早就等待主子的来临,长乐公主的行装早就搬了过来,在南楚陪伴长乐公主的得力侍女也已经将东西都安置好了,长乐公主扶着长孙贵妃,听着母亲唠唠叨叨的交待着事情,母女共同分享着天伦之乐。纪贵妃也在一旁,不时劝慰几句,她擅于言辞,倒也不令母女两人觉得有外人在侧不舒服。

过了一段时间,长孙贵妃有了几岁年纪,又是太欢喜,不免疲惫起来,长乐公主担心母亲身体,想要送母亲回寝宫,长孙贵妃体恤女儿辛苦,让她好好休息,自己回宫休息,纪贵妃却托词留下,长乐公主有些疑惑,但她在南楚为后多年,虽然深居简出,但养移气,居移体,自然也有母仪天下的风范,所以她静静的等待纪贵妃表露真情。果然过了没多久,纪贵妃遣散下人,郑重地问道:“公主,梁婉伺候公主多年,这次为何这个样子回来,我这个侄女奔波多年,落得这个下场,怎么不让本宫伤心。”

长乐公主心里一动,皇兄李贽就问了自己许久梁婉的事情,她早就听说这个纪贵妃出身江湖,也隐隐约约知道梁婉是纪贵妃推荐的,便也不隐瞒,将自己经历讲了一遍。

纪贵妃听得很认真,当她听到梁婉袭击那个黑衣人一招被擒的时候,脸上露出古怪的神色,问道:“公主,你是说梁婉没有还手的余地。”

长乐公主歉意地道:“本宫也看不明白,只觉得那人一伸手就制住了梁姐姐。”

纪贵妃问道:“那么这个黑衣人有什么特征呢?”

长乐公主陷入回忆,当日她满心惶恐的看着梁婉被擒,然后一个一个的密探被勒令束手,那个黑衣人走到自己面前,举手投足之间杀了意图刺杀他的侍女,站在自己面前,当时自己握紧了发簪,准备若是这人稍有冒犯便要自尽,却听见那个阴柔的声音淡淡说道:“王后,不用担心,我们不是南楚的人,请王后随我们去一个地方,事后我们会送王后去见雍王的。”说着便来搀扶自己,当时自己满眼都是侍女被杀的情景,而千金之躯更是没有被不相干的男人触及过,所以十分恐惧,那个声音在自己听来宛若魔鬼一样,自己颤抖着想要将金簪刺入咽喉,却被那人阻止,那人无奈地道:“王后宽心,家主人对王后并无恶意,我更是一个阉人,不会亵渎王后清白。”说着点了自己穴道,将自己眼睛蒙上,然后自己就失去了知觉。在被软禁在暗室的时候,来照顾自己的都是那个黑衣人,长乐公主能够确信那个确实是个阉人,甚至她可以从他对礼仪的熟稔知道这人是南楚的宫人。所以她并没有相信自己能够得到自由的说法,直到,那一天,自己见到了那些保护自己的密探,他们跪在地上向自己请罪,而在他们身边的是智力已经变成了幼儿的梁婉,在他们保护下,自己见到了皇兄,而且眼睁睁的看着他们自尽身亡,鲜血染红了金殿。

而她始终不知道发生了什么事情,在护送自己的过程中,无论自己怎么询问,他们都只是请罪,渐渐的,自己明白了,他们的自杀一定是那些黑衣人的要求,而他们为了保护自己答应了,按理说,她应该痛恨那些黑衣人,但是,奇怪的很,她并没有一丝痛恨,因为那些人始终没有对自己有一丝一毫的轻薄,他们留下自己的性命也是一件冒险的事情,至少自己听过他们的声音,还知道一个人是阉人,但是她没有告诉皇兄,因为虽然对方对她没有一丝要求,但是她终究是受了人家的不杀之恩。

纪贵妃见长乐公主想得入神,有些不耐烦,但她知道可能会让公主记起一些事情,随意耐心的等待,良久,公主用梦呓一般的声音道:“本宫只记得他们像军旅一样行动有序,纪律严明,对本宫恪守礼仪,其他的事情没有什么特别,那个黑衣人身材不高,眼睛很冷,就是这些。”

纪贵妃淡淡问道:“那些人是大雍人还是南楚人呢?”

长乐公主奇怪的看了纪贵妃一眼,道:“他们应该不是大雍人,因为我见的几个人都不像大雍人这样高大。”

纪贵妃露出冷冷的微笑道:“公主一路辛苦,请好好休息吧,本宫先告辞了。”

\chapter{第二十六章 余波未歇}

大雍武威二十三年(南楚至化元年)十一月三日,圣上下旨,加殊恩于齐王,人皆知其意在雍王也。

--《雍史·太宗本纪》

离开翠鸾殿,纪贵妃深吸了一口气,抒发一下心中郁闷,梁婉是门主梵惠瑶的爱徒,也是凤仪门重要的棋子,她在江南立功卓著,又和太子李安达成协议,不料这次竟然毁在了江南,怎不令人心痛,门主传来密信,要自己查清梁婉变疯的所有细节,自己知道,门主怀疑是雍王动了手脚,毕竟雍王对梁婉已经有了不满,要不然也不会派人另外建立情报网。可是从唯一亲身经历过那件事情的长乐公主口中,并没有得到一丝有用的情报。

纪贵妃微微冷笑,除了雍王,还会有谁呢,若是南楚人,一定不会平白放过长乐公主,除非是雍王的属下,才会对长乐公主这样礼待,可是没有证据啊,自己总不能平白无故的指责雍王李贽啊。想起皇帝的封赏,纪贵妃更是心冷如冰,今天的庆典上李援宣布因为雍王多年来战功卓著,近年来又先后灭蜀破楚,功高盖世,现有官职不能够表彰他的功劳,因此下诏封雍王为天策元帅,领大司徒,位在诸王公之上,赏食邑二万户,并赐衮冕一套、金辂轿一乘、玉璧一双、黄金六千斤、前后鼓吹九部之乐、班剑四十人,这是何等的荣耀,就是太子仪仗也不过稍胜一筹罢了。

更让纪贵妃心寒的是,皇上又下诏特许天策帅府自置官属,按照李贽上报的折子,计有长史、司马各一人,从事中郎二人,军咨祭酒二人,典签四人,主簿二人,录事二人,记室参军事二人,功、仓、兵、骑、铠、士六曹参军各二人,参军事六人。这样一来,李贽的天策帅府就成了一个麻雀虽小,五脏俱全的小朝廷。皇上会不会改变主意,立李贽为皇储呢?想了半天,纪贵妃摇头,虽然雍王功高,但是太子没有明显的失德,而且按照她对皇帝的了解,只怕今夜皇帝就会后悔给雍王的赏赐太厚了,估计过不了几天,皇上就会想方设法的消减雍王的势力。自古以来,功高震主,有几个会有好下场,想到这里,纪贵妃露出得意的笑容。

这时,一个绯衣宦官急匆匆的赶来,禀报道:“娘娘,皇上传了旨意,今夜要在娘娘那里歇息,请娘娘速速回宫,估摸着,再过小半个时辰,皇上就会到了。”纪贵妃心里大喜,她知道得很清楚,自己虽然容貌不错,但是论起感情和宠爱,在皇上面前并不突出,更何况自己一向都是淡薄恩宠的表现,更让自己很少得到爱宠,但是相对的,自己身为凤仪门和皇上的联系人的身份就更加突出,所以皇上经常让自己参与国事,今夜皇上要在自己这里留宿,看来是要讨论一下雍王的事情了,看来自己的想法没有错,皇上,已经对雍王十分忌惮了。想到这里,纪贵妃俏脸上露出了绽放如春花般的笑容。

有人欢喜有人忧,在盛大的庆功宴后也是如此,在金碧辉煌的太子府,李安愤怒的将书案上的文书全部拂到地上,狂叫道:“李贽,孤不杀你,誓不为人。”喊罢,他跌坐在椅子上,恶狠狠的看着书房门,仿佛雍王就要从那里出来一般。良久,他疲惫地道:“来人,请少傅来见孤。”

不过片刻,一个相貌平平的黑髯文士走了进来,他穿着太子少傅的官服,见了太子并不行礼,径自坐在太子左手的一张椅子上,笑道:“殿下怎么这样气恼?”

李安怒气冲冲地道:“李贽如今已经是天策元帅,老头子就差没有把我这个太子的位子给了他,你叫我如何不气恼。”

那个文士笑道:“殿下过虑了,皇上对殿下爱护备至,若是想立雍王为储早就立了,何必要等到今日。”

李安丧气地道:“少傅不知道,当初他的母亲是父皇的元配,我虽是长子,却是庶出,后来他母亲命短,早早归天,我的母后才立了正室,父皇称帝之后,追封他的母亲为孝贤皇后,所以若论嫡庶,我是不如他的,只是我占了长子的名份,母后又是当今皇后,才让我做了储君,如今,如今,我真的不知道该怎么办,若是父皇改了主意,我真是一点法子都没有了。”

文士目光一闪,道:“殿下是当局者迷,臣却认为太子的位子表面上危如累卵,实际上却稳如泰山。殿下想皇上对雍王宠爱,臣却以为皇上对雍王猜忌,想一想,雍王这些年来南征北战,我大雍的天下倒大半是他打下来的,皇上不免会觉得受了儿子的恩惠,如今雍王功高莫赏,若是皇上立他做储君,这是顺理成章的事情,可是皇上宁可特例加赏,也不肯更动太子的储位,这分明是偏心太子。臣以为皇上不是爱殿下,而是殿下的即位象征着皇上无上的权威,所以皇上无论如何不肯改变决定,只要殿下多在皇上面前表示孝顺皇上皇后,礼敬妃嫔,尊重雍王,兄友弟恭,皇上绝不会更换储君,更何况还有凤仪门的支持,殿下不会以为梁婉的倒戈就是因为她自己的决定吧。过些日子,皇上就会想到他百年之后,太子若是不能压服雍王,又该如何是好,他就会想法子打压雍王,只要殿下即了位,外有齐王辅佐,内有凤仪门助力,想要雍王的性命不过是易如反掌罢了。”

李安听了,良久,终于喜笑颜开,道:“少傅,多谢你开导孤王,依你之见,我们目前该作些什么?”

文士嘿嘿一笑,道:“多做多错,少做少错,殿下不妨合光同尘,倒是齐王那里,殿下要多多笼络,前些日子齐王战败,殿下给齐王不少脸色,这是太不应该了,若没有齐王襄助,殿下就没有日后擎天保驾的大将。”

李安站了起来,深施一礼道:“谨受教。”脸上露出暧昧的神色道:“六弟喜欢美女,我新近选了两个绝色的女子,原本是想送给父皇的,就先选一个送给他吧。”

那个文士脸上也露出暧昧的笑容,但又立刻扳起了脸。李安看了他一眼,笑道:“少傅在孤王这里还装什么正经,那个绝色不能给你,不过本王还有几个美人,送你两个如何。”

文士低下眼睑道:“那就多谢殿下赏赐了。”

李安大笑,笑声传出了书房,很远,很远。

带着醉意回到府邸的李贽服下解酒的药物,用冷水匆匆忙忙的洗了一个澡,然后一身清爽的来到了议事厅,大厅里面已经坐了一些人,正是石彧石子攸、管休、董志、苟廉几个谋士,武将们今日都大醉而归,李贽就没有让他们过来,李贽见他们正在低声讨论,吩咐司马雄到外面警戒,他走了进去,笑道:“让几位先生久等了,本王来晚了。”

几个谋士站起行礼,各自坐下,李贽看向石彧,问道:“你见过江哲了,觉得怎么样?”

石彧苦笑道:“江哲到了王府,一派泰然自若,好像就是自己的家一样,属下安排了最好的院子给他,他只是淡淡一笑,住进去之后,他对殿下安排的侍女仆人也没有任何异议,如果不是知道此人始终不肯归顺殿下,我倒要以为他已经效忠殿下了呢。我看若是殿下给他安排一个官职,他也不会拒绝,我看他似乎十分喜爱舒适的生活,至少不会以死相抗。”

李贽苦笑道:“这一点本王也清楚,若非如此,只怕本王还有些法子,他若是一心求死,以全名节,本王只要好好对待,细心照料,终有让他回心转意的一日,可是他这般随遇而安,本王就是给了他一个官职,只怕他也会尸位素餐,每天写写诗文,谈谈琴棋书画,只是本王真正需要的,他却吝于赐予,如今本王恨不得化身德亲王赵珏,赵珏虽然不幸,但是也曾经得他衷心相待。唉。本王最担心的就是齐王,齐王虽然鲁莽,但是却不是没有心机,他对孤说要待江哲以师礼。”

管休等人相视一笑,都道:“殿下过虑了,若是此人这么容易就被齐王感动,我们也就不用这么费心了。”

李贽转念一想,也觉得自己未免有些过虑,正要嘲讽几句,却见石彧若有所思,他有些担忧地道:“子攸,莫非你认为齐王有可能招揽到江随云么?”

石彧回过神来,笑道:“殿下,齐王这个主意倒也不错,不过未免有点谄媚,不过我们倒是可以借鉴,世子聪明颖悟,虽然年仅五岁,但是已经粗通文字,如果让世子拜他为师,那么他不就成了殿下的臂助,我想他总不会见了英才而宁愿失之交臂吧。”

李贽大喜道:“子攸真是好计谋,好,明日设宴洗尘,就让世子出来拜师,动作一定要快,我为了掩人耳目,已经将他的事情禀报了父皇,父皇要召见他呢,等到父皇召见之后,我们就不能软禁他了。”

虽然未必能够达到目的,但是总算有了法子,李贽顿觉浑身轻松,笑道:“对了,子攸,你说长乐公主遇劫的事情是怎么一回事,我派人查过,但是时间太短,查不出什么端倪,我派人去他们遇袭的地方勘察过,有些像是小型军队的手笔,但是在那个时候什么人赶去劫持公主呢?而且,本王不明白的是,那些返回来的密探为什么要自尽,公主安然无恙,无论如何,他们功大于过,就是畏罪自裁,也该跟本王详细说明事情经过啊?”

这些事情管休他们已经讨论过多次,李贽此刻提出只是想看看石彧的意见,石彧答道:“属下也想过这个问题,唯一的结论就是,首先,他们不是针对公主殿下去的,他们的目标就是梁婉,否则不会只有梁婉收到伤害,而那些密探自杀,属下觉得并非是因为畏罪,恐怕是一种协议,他们见到了劫持者,可能也知道了很多事情,可是他们能够安然带着公主回来,这一点除了说明他们对公主没有恶意,也说明他们确信不会泄露自己的秘密,公主始终什么都不知道,那么这些密探必然是许下了自裁的承诺。”

李贽道:“虽然如此,说句不好听的话,这些密探虽然是我大雍勇士,理应忠诚守信,可是已经回到本王身边,告诉本王真相应该胜过守诺的信义吧?”

石彧叹息道:“这就是最可怕的一点,除非他们认为自裁而死比告诉殿下真相更加对殿下有利。”

李贽神色一凛,道:“你是说那些人有足够的力量威胁本王。”

石彧点头道:“是的,听永泉说,殿下事后查验那些密探的尸身,发觉他们虽然受了一些刑罚,但是基本上都不严重,也就是说,对方并非滥施刑罚的人,而从梁婉来看,她的记忆全部毁去,这种手段十分诡秘,也就是说,对方的手段阴毒狠辣,我想那些密探心上所受的压力一定很大,最后甚至超过他们可以忍受的界限,才让他们遵守承诺自裁。”

李贽苦恼地道:“想不到暗中还有这些人在活动,子攸,你说这些人会是什么来历。”

石彧答道:“属下认为唯一可以猜测的是,那些人对我大雍并无敌意,否则公主殿下就不会平安归来,不过那些人针对梁婉,属下倒是认为,如果不是和凤仪门有关,就是和梁婉本人在南楚的所作所为有关,殿下不妨从这两方面着手。”

李贽连连点头,道:“子攸是本王的肝胆啊,若没有子攸,本王哪里还有斗志。”

石彧笑道:“江哲却是殿下的双翼,若是殿下有了此人,才是如虎添翼。”

众人相视而笑。

在这个不眠之夜,我也没有休息,站在窗前,看着满园的雪后美景,小顺子走过来,埋怨道:“公子,你身体刚刚好一些,又在这里吹风,也不知道爱惜身体,这里冷得很,我已经让他们准备了手炉。”说完,把一个手炉塞到我怀里,又把狐皮披风批到我肩上。

我笑道:“你放心,我的身子没有这么弱,怎么样,你有没有看过雍王府的防卫。”

小顺子笑道:“他们监视得很严密,我只是随便看了看,如果是我一个人倒没有什么,若是带着公子,就恐怕逃不出去了。”

我摇手道:“不妨事,我也没有打算让你救我出去,无论如何,我总是能保住性命的,只是不想为人卖命罢了,那些人杀来杀去,总有人能够一统天下,无论是谁都没有什么关系,何况雍王得胜算还是很大的。小顺子,看,又下雪了。”

小顺子顺着我的目光看向窗外,纷纷扬扬的瑞雪悄无声息的落下,寒冷的朔风扑面而来,不由笑道:“在南楚偶然下场小雪,公子便要赏雪饮酒,如今这里的雪这样好,公子可是又来了兴致。”

我点点头道:“是啊,明天你去跟他们要些上好的木炭,要些好酒,我看这雪明天也不会停,我要饮酒作诗呢?”

小顺子道:“这我可就只能替你温酒了,那些诗文我可不懂。”

我叹息道:“是啊,你啊,唯一令我不满的就是不能陪我写诗论文,不过若是没了你,我喝酒也不免少了兴致,良朋,美酒,飞雪,可是不能或缺啊,可惜,若是飘香尚在,唉。”

小顺子劝慰道:“公子,逝者已矣,莫要伤悲。”

我看向窗外的飞雪,再无言语。

第二天,果然飞雪连绵,李贽得到了一个消息,李援下旨,因齐王两次进攻南楚,苦战有功,又令南楚德亲王重伤而死,所以拜为大司空,也赐一套衮冕、金辂轿、双璧、黄金二千斤,前后鼓吹二部、班剑二十人。

得知这个消息,李贽并没有气愤,而是彻底的心寒,自己作战胜利,却是得到父皇猜忌的下场,赏赐齐王,不就是为了制衡自己么,他漠然的对石彧说道:“子攸,父皇待我何其薄也。”

石彧也是叹息不已,正要劝慰李贽,这时苟廉匆匆忙忙走进来道:“殿下,殿下,江随云的仆人去要了木炭美酒,要去赏雪,我已经让人引他到临波亭去了。”

李贽顿时转怒为喜道:“好,你办的好,走,咱们这就去凑个热闹,子攸,你安排一下,过半个时辰带世子去临波亭。”

此时的我,已经坐在临波亭里了,雍王府的后花园有一个两亩左右的小湖泊,据说是原本园中有一眼清泉,水量丰富,索性便挖了这个小湖泊,再通过长安的排水系统汇入永安渠,永安渠接通城北的渭河,供应长安一半的用水,又是水运交通要道,所以这个湖泊虽在皇城之内,却是活水。

小顺子一边温酒一边道:“公子,怎么这个亭子里一点都不冷呢?”

我笑道:“我也只在书上看过,你看这个亭子的顶上虽然只看得见厚厚一层苕草,其实这层草下面可是大有文章的呢,草的下面是一层油毡,再一层苕草再一层油毡,共有三层,然后再在最后一层油毡下搭了瓦片,这瓦片也是特制的,是空心的,所以盖在头顶上不怕跑了热气,再看这亭子的石料地板和边上围着的凳子,还有那几根铜铸的柱子,其实在柱子和亭子地下都点着火龙,就像老百姓家里的炕一样,再说这水,水最是冬暖夏凉的东西,水在流动,会把地里的热气都一起带进来,离水越近越暖和,所以这亭子里面怎么会冷,这是北方富豪人家为了赏雪专门建造的亭子,只要穿上轻裘,再抱上一个手炉,就不会冻着了。好了,你看外面飞雪连绵,乱舞梨花,遍地琼瑶,真是好地方啊。”

\chapter{第二十七章 赏雪赋诗}

站起来,我面向小湖,诗兴勃发,朗声吟道:“远眺寒山遮望眼,毗绝无际雪如莲。遥惜梅影映残月,暗叹竹魂写碧天。香冷何需邀众赏,花红独自缱缠绵。琼瑶罗绮玉人舞,素手轻拂泪管弦。”

吟诵一首之后,我不由欢笑起来,伸出手去,雪花落到手上,瞬息溶化。这时,有人在远处大声笑道:“江先生如此雅兴,为何不邀主人前来。”我回身望去,却见雍王李贽一身轻裘,几个谋士都在身后站着,几人都是笑意盈盈,走在后面的两个仆人一个手里提着一个大酒坛,另外一个提着一个食盒。

我微微一笑,道:“殿下公务繁忙,随云不过山野闲人,如何敢打扰殿下和几位呢。”

李贽走进临波亭,拂了拂身上的雪花,道:“我这世俗之人前来打扰先生雅兴了,这坛酒是父皇御赐的美酒,先生可不要错过啊。”

我淡淡一笑,道:“凡事总要有个先来后到,既然今日是随云先来了,那几位今日可要听我作主,小顺子,你来温酒,酒过三巡,诸位需得吟诗一首,题目便是《咏雪》,若是好诗,饮酒一杯,若是不好,需得罚酒三杯。”

李贽见我没有不满,欢喜地道:“既然先生定下了规矩,本王也不能不遵守,好吧,你们听着,若是写不出好诗,可要连饮御酒三杯,本王可告诉你们,这御酒醇厚香甜,若是多饮了几杯,听不到江先生的好诗,可是平生之憾啊。”

我们团团坐下,一个仆人将食盒中的几样下酒的果品点心放在桌子上,另外一个仆人将御酒的泥封打开,酒香扑鼻而来,芬芳醇美。苟廉闻了酒香,道:“若非是想听随云的大作,真想一醉方休啊。”

李贽挥手让仆人们退下,笑道:“好啊,赶明儿我送一坛酒给你,让你大醉一场。”苟廉连忙拜谢道:“殿下可不能后悔啊。”

说话不久,小顺子已经将温好的第一壶酒端了上来,给我们一一满上。我慢慢喝下这杯酒,顿觉齿颊流芳,四肢百骸都温暖起来,不由道:“真是好酒,我南楚的酒虽然绝佳,但是比起北方的酒不免淡了一些。”

石彧笑道:“既然随云喜欢,就多喝几杯吧。”

李贽微笑举杯,众人连喝了数杯,都觉得飘然如仙,气氛也热烈起来。李贽笑道:“我们刚才已经听到了随云的大作,那么理应我们先吟诗,永泉,你诗才最敏捷,就由你先来吧。”

苟廉站起身来,看看亭外的飞雪,高声道:“好,就由我先来,半壁雪原铺晚照,一湖暖玉涂云烟。览此佳境最得意,不羡桃源不羡仙。”

李贽首先道:“好,虽然意境平凡,却是和眼前盛境如此贴切,当饮一杯。”

我也笑道:“半壁雪原铺晚照,一湖暖玉涂云烟。永泉兄果然诗才敏捷,诸位与雍王殿下,外托君臣之义,内实亲如骨肉,上下并无嫌隙,在此冬日,饮酒作乐,果然是不羡桃源不羡仙。”

苟廉见有空隙,便道:“殿下待我等亲如骨肉,随云何不效我等一般,侍奉殿下,也品味一下不羡桃源不羡仙的心境呢?”

我微微一笑,道:“随云别无所长,只是擅长诗文,就先和诗一首吧,以偿先生盛情。枫染幽燕几时尽?名花淡荡宿枝轻。中庭鸟影扑寒翼,小宴炉云堆暖楹。三尺琴开梅着玉,四边歌动雾还晴。自称阔逸无萧瑟,万顷天空一掷行。”

董志拍手道:“好一个‘自称阔逸无萧瑟,万顷天空一掷行‘,可见随云心胸如朗月晴空,寥廓如此。当饮酒一杯。”

我接过小顺子递过来的酒杯,笑道:“随云当日在南楚,虽然职小位卑,尤自殚精竭虑,不敢稍有松懈,如今总算脱却樊笼,所谓‘复得返自然‘是也,永泉兄何忍心陷我于不忠,屈我于樊笼。”

苟廉语塞,只得苦笑。我却笑道:“从前和董兄论及军阵,今日却要领教董兄诗文了。”

董志拱手道:“献丑了,献丑了。”说罢站起吟咏道:“斗柄欲东指,吾兄方北游。无媒谒明主,失计干诸侯。夜雪入穿履,朝霜凝敝裘。遥知客舍饮,醉里闻春鸠。”

我听到这里,手一抖,一杯酒几乎倾倒在桌上,当年我入南楚为官,虽然原本没有侍奉明主,一统天下的大志,可是后来种种,却让我隐隐后悔当初的选择,若是当年我被雍王殿下带来了长安,可能就不会领受国破家亡的苦痛了吧。如今我做客长安,望不见南楚烟云,这种失群孤雁的悲凉,即使是半推半就抛弃了故国的我,也是满腹辛酸啊,举杯饮下美酒,酒入愁肠,愁更愁啊。

有些醉意的我,随手拿起一支银筷,一边敲击着酒壶,一边唱道:“把酒临波亭。看渊明、风流酷似,卧龙诸葛。何处飞来林间鹊?蹙踏松梢残雪。要破帽、多添华发。剩水残山无态度,被疏梅、料理成风月。两三雁,也萧瑟。佳人重约还轻别。怅清江、天寒不渡,水深冰合。路断车轮生四角,此地行人销骨。问谁使、君来愁绝?铸就而今相思错,料当初、咽尽肝肠血。长夜笛,吹裂!”

唱完一遍,我再度唱道:“铸就而今相思错,料当初、咽尽肝肠血。长夜笛,吹裂!”想起当年替德亲王筹划,每每深夜难眠,可惜却落得一个敬而远之,想起上表直谏,却落得永不叙用。不由悲从心起,泪落如雨。

董志连忙站起,致歉道:“是我不好,勾起随云心事,还请见谅。”

我摆手道:“多日悒郁,一扫而空,还要多谢董兄的好诗。”

董志也不敢再相劝,心道,看来他对南楚还是情深意重啊,这可怎么办才好。他看看雍王,李贽脸上又是赞叹,又是悲伤。

管休见此,连忙道:“我文才浅薄,还请诸位不要见笑。”说罢起身执酒道:“检尽历头冬又残,爱他凤雪忍他寒。拖条竹杖家家酒,上个篮舆处处山。添老大,转痴顽,谢添教我老来闲。道人还了鸳鸯债,纸帐梅花醉梦间。”

众人听了都不由大笑起来,苟廉更是被杯中酒呛住了,一边擦着眼泪一边道:“老管,从来不知道你这样风趣,我今日算是领教了。”

我也不由轻笑,举杯道:“管兄好词,随云自愧不如,自愧不如。”众人欢笑一阵,气氛变得活泼起来。

小顺子刚才见我伤心,不由暗中怒视董志,见管休一首诗词,令我开颜,心中不由大喜,连忙将刚温好的酒替管休倒满,眼中的喜色一闪而过,却被一直微笑旁观的石彧看在眼来,心道:“这是一个至诚忠心的下人。”

众人见我喜悦,这才松了一口气,他们又不是来气我的,而且后面还有文章,总不能让我早早就气走了吧。

石彧起身道:“江先生,石某和江先生相见太晚,可惜没有机会向先生讨教,这一杯酒敬先生,愿先生福体安康。”

我也站起来道:“石先生如此,随云愧不敢当,随云早就听说石先生是雍王殿下的萧何,殿下出征在外,先生为殿下打理后方,若没有先生,殿下恐怕腹背受敌,君之大才,随云一向万分佩服。”

石彧笑道:“随云如此推崇,倒令在下惭愧万分了。”

雍王起身道:“并非推崇,本王若非先生,焉有今日。”想起往日自己出征,太子总是在后面掣肘,如果不是石彧在后面替自己出面处理,自己焉能每战必胜,李贽举杯道:“今日本王敬先生一杯,聊表心中感激之情。”

石彧连忙举杯相谢,泪水盈眶,片刻之后,石彧道:“石某诗才不高,勉力为之,还请殿下和诸位不要取笑。”说罢,吟咏道:“长安雪后似春归,积素凝华连曙晖。色借玉珂迷晓骑,光添银烛晃朝衣。西山落月临天仗,北阙晴云捧禁闱。闻道仙郎歌白雪,由来此曲和人稀。”

我微笑拊掌道:“先生的诗,一见就是丞相气度,可惜随云不堪久居京华,否则一定可以见到先生领袖群伦的风采。”

石彧苦笑道:“随云若肯屈就,石彧情愿虚左以待。”

我微微一笑道:“江某闲云野鹤,不堪重任,先生若是这样说,岂不折了晚生的寿数。随云有小诗回赠,以谢先生美意。”

言罢,我从容歌道:“冻云深,淑气浅,寒欺绿野。轻雪伴、早梅飘谢。艳阳天、正明媚,却成潇洒。玉人歌,画楼酒,对此景、骤增高价。卖花巷陌,永灯台榭。好时节、怎生轻舍。赖和风,荡霁霭,廓清良夜。玉尘铺,桂华满,素光里、更堪游冶。”

歌罢,我笑道:“如今良辰美景,正好游冶,何必说些军国事,图增烦恼,昔日高人赋采薇,江某不才,不能不食大雍之粟,但也不爱大雍之禄。”

众人听了,一阵心灰,李贽站起身道:“先生志向高洁,本王佩服。”

我笑道:“殿下乃是这里的主人,也该赋诗一首,表明心志才是。”

李贽道:“那么先生见笑了。”说罢,李贽朗声吟咏道:“碧昏朝合雾,丹卷暝韬霞。结叶繁云色,凝琼遍雪华。光楼皎若粉,映幕集疑沙。泛柳飞飞絮,妆梅片片花。照璧台圆月,飘珠箔穿露。瑶洁短长阶,玉丛高下树。映桐珪累白,萦峰莲抱素。断续气将沉,徘徊岁云暮。怀珍愧隐德,表瑞伫丰年。蕊间飞禁苑,鹤处舞伊川。傥咏幽兰曲,同欢黄竹篇。”

我品味良久,敬服道:“殿下的诗沉健稳练,语壮意豪,一派帝王气象,这是天成,我等诗文,虽然优美,却是斧凿而成,随云佩服。”

李贽笑道:“我是皇子,这帝王气象四个字不敢自居,先生不要害我,总算没有丢丑,本王已经心满意足了,还请随云作诗一首,以做善始善终。”

我笑道:“再作下去,我就要江郎才尽了。”我已经带了七分醉意,更觉得身上发热,解开轻裘衣襟,走到亭边,临风长吟道:“有身莫犯飞龙鳞,有手莫辫猛虎须。君看昔日长安市,白头仙人隐玉壶。子猷闻风动窗竹,相邀共醉杯中酒。历阳何异山阴时,白雪飞花乱人目。君家有酒我何愁,客多乐酣秉烛游。谢尚自能鸲鹆舞,相如免脱肃霜裘。兴罢鼓棹过江去,千里相思明月楼。”

吟罢长诗,我回到桌前,拿起酒杯一饮而尽,醉意朦胧的我大笑道:“今日尽欢而散,随云多谢殿下了。”

李贽看着江随云,今日赏雪,在他不过是找个机会让世子来拜师,顺便和江随云亲近一下,没想到江随云诗兴勃发,暗里应对众人的劝说,滴水不漏,明里更是诗压全场,这般文雅风流,就是不知道此人有经天纬地的才能,也是不能放过。想到这里,其心更切。

这时,小顺子趁机到我身边,在我耳边低低说道:“有人来了,公子小心不可失言。”然后替我整理好衣衫,笑道:“公子身子不好,今日又多喝了几杯,可不能着凉了。”

我神志一清,耳中也传来低低的脚步声,却是四五个人的样子,其中一人脚步蹒跚,身子又轻,倒像是一个小孩儿。

清醒过来的我随手接过小顺子递过来的热方巾,擦了一把脸,道:“江某酒后失态,还请殿下和几位先生见谅。”

李贽笑道:“狂歌纵酒,名士风采,怎说失态,不过贵仆说得对,先生身体刚刚好转,不可着凉,还是多喝几杯吧。”

我坐回位子,接过温酒,慢慢品味了起来。眼睛余光却见李贽等人互相打着眼色,不由心里暗笑。

接着我就听到一个稚嫩的声音叫道:“父王,父王。”抬头看去,却看见一个小男孩高高兴兴的冲着我们摆手,那个男孩不过四五岁的年纪,相貌秀美,穿着黄色的王子服饰,身后跟着两个奶妈侍女和两个太监,此刻小男孩身上倒大半是雪痕,想来是跌了好几跤的缘故。

李贽见到男孩,满面喜色,道:“骏儿,你怎么浑身是雪,过来让父王看看。”

那个男孩连蹦带跳地走进亭子,依偎在李贽膝下,黑白分明的眼睛只在我身上打转。我微笑道:“草民见过世子。”

那个男孩走近来,拉住我的衣襟问道:“先生是谁,骏儿从来没见过你?”

我淡淡道:“草民江哲,字随云,是南楚人士,世子自然是没有见过草民的。”

李骏听了我的名字,念叨了半天才道:“我记得了,先生的诗写的很好。”看看外面的飞雪,笑道:“千山鸟飞绝,万迹人踪灭。孤舟蓑笠翁,独钓寒江雪。先生的这首《江雪》真是很好,就是太寂寞了,南楚的江上,真的这样寂寞么。”

我笑道:“南楚虽然人杰地灵,可是还是有很多没有人烟的地方,那里江河又多,所以真的有这样的地方,放眼望去,只有寒江冰雪,那一年我跟着先父远行,快到过年了,所以江上几乎没有舟船,大家都在家里团聚,先父自己驾舟,带着我在江上钓鱼,江水虽然没有结冰,可是到处都是白茫茫的。”

李骏眼睛放光,道:“先生的父亲真是太好了,我每次要父王带我出去玩儿,父王都没有时间,先生,若是有时间,你陪骏儿到渭河上去钓鱼好不好。”

我笑道:“世子千金之躯,怎能和我们这些草民一样,世子若是喜欢钓鱼,不如就在这个亭子里面垂钓吧,我看湖水里面有不少锦鳞,钓起来一定很有趣。”

李骏不依道:“在这里钓鱼有什么意思,若是钓不起来,那些下人恨不得把鱼给我挂在鱼钩上,而且我父王十几岁就在军中作战,我也要像父王一样,若是连大门都不出,将来怎么上阵杀敌。”

李贽脸上露出欣赏的神色,口中却道:“骏儿不得胡说,你将来要好好处理政务,不会像父王这样上阵杀敌,到时候我大雍一统天下,哪里还需要你去杀敌。”

李骏不赞同地道:“父王说的不对,我听先生们说要居安思危,若是将来又有了敌人,孩儿若是不会上阵杀敌,怎么捍卫大雍,所以政务要学,上阵杀敌也要学。”

说完,李骏露出不好意思的神情道:“所以,父王让孩儿出去看看吧,孩儿不会捣乱的。”

李贽笑道:“你这个小顽童,还是想去胡闹罢了,你若想上阵杀敌,就要学万人敌,首先更要熟读经史,父王上次给你选的师傅,怎么又被你赶跑了?”

李骏偷眼看看父王,道:“是那个师傅太没有本事了,我就是问他一个问题,他没有答上。”

众人来了兴致,李贽笑着问道:“你问了什么问题,让师傅没有答出来。”

李骏得意洋洋地道:“我那日听舅舅跟父王说起大理寺的一个案子,说是一个人的继母杀了他的父亲,他便杀了继母,县官判了他大逆灭伦的罪名,可是他不服上告。我问师傅,他说判得不错,这样的道理都不明白,所以我才赶走了他。”

李贽想起这个案子,也想起了自己告诉妻舅的判决,这件事情外人不知,果然是一个好题目。看了一眼江哲,李贽笑道:“那是你问错了人,除了父王,这里每一个人都能告诉你应该如何判决。”

果然,凭着李贽对儿子的了解,李骏的目光从几人身上一一越过,最后落到江哲身上,其他人他相信父王说得不错,但这个人呢?他拉着江哲的衣角道:“先生能告诉骏儿,该如何判决么?”

我淡淡一笑,道:“这些事情自有律令,在下一介草民,怎么有资格评论。”

李骏不依地道:“若是先生答了出来,骏儿就拜先生为师,若是答不出来,那么先生就做骏儿的随从。”

我看了李贽一眼,却见他也是一脸惊喜,看来并非他授意世子这样说的,不由笑道:“草民南楚罪臣,怎能做世子的先生。不过世子若是问我,我就说此人虽然杀了继母,但却是为父报仇,继母杀害亲夫,是自绝于夫家,那么此人杀继母只是杀了一个外人罢了,可以以杀人论罪,却不必以逆伦加罪。”

李骏欣喜地道:“先生果然是明理之人,我拿来问人,还没有人说的这般明白呢。”说罢,李骏跪在我面前道:“骏儿虽然年幼,但是也知道什么是一诺千金,骏儿愿意拜先生为师,先生可要带我去渭河钓鱼啊。”

我噗哧一声笑了,这个孩子这般绕来绕去,却不过是让我带他出去玩乐罢了。

这时,小顺子的声音在我耳畔响起道:“公子不可答应。”

我心中一凛,道:“世子说笑了,世子是金尊玉贵的身份,我不过是个亡国之人,这里的每个人比在下适合做世子的先生,江某可不敢应承。”说罢,我起身道:“随云不胜酒力,这就先告辞了。”

在我转身之时,我听到李贽失望的声音道:“江先生,你真的如此狠心么?”我的身躯微微一颤,终于没有答话。

注:本章涉及诗词,大多从网上摘抄,有些是网友创作,无法一一列举,谨此声明。另外,这一章我用了很多诗词,希望大家不会以为我是堆砌字数,我是很用心的选择诗词的,因为要通过诗词表示他们的心意。

\chapter{第二十八章 失望至极}

看着远去的背影,李贽手中的酒杯碎裂,鲜血从手心滴落,他从未像这样一般觉得心灰意冷,从少年时候,他就是众人的焦点,在多年的行军作战,领袖群伦的生涯中,他始终都是高高在上的王者,军士效死,百姓爱戴,群臣敬畏,皇室感佩,多少次,他只是用尊重之心礼敬贤才,就换得那些人的感激涕零,多少次他只是随意而为的一些小事,却成就了他平易近人的形象,渐渐的,他习惯了用自己的王者魅力去征服别人,用谦虚和平和去得到人心,今天,他真的遭到了惨痛的失败,无论自己怎样相待,那个人始终是微笑着远离,是的,自己可以将他留在身边作个官员,但是又有什么用,自己没有征服那个人,自己没有得到他的忠心,这一刻,李贽真的品尝到失败的苦果,多少次作战失利,多少次朝堂受窘,李贽从未如此失落,痛苦。

就在李贽不可自拔的时候,耳边传来了优雅的乐声,声音飘渺高洁,温和中正,李贽不由心中一动,心思渐渐平和下来,看了看身边谋士们忧虑的眼神,看看李骏几乎要被吓坏了的神情,他无奈的道:“本王累了,这就回去休息了。”说罢他起身离去。

石彧等人看着他的背影,感受到李贽的寂寞和悲伤,不由心情沉重。他们虽然不能明白李贽的心情,可是李贽受到什么样的打击却是心知肚明的。董志看人都已经散去,有些恼怒地道:“江随云也未免太过分了,殿下如此对他,他居然还是这样无情。”

管休叹息道:“无论他怎样过分,我们也不能怪责他,所谓忠臣不事二主,他不愿效忠殿下也没有什么奇怪。”

董志怒道:“所谓忠臣,若是不肯投降,宁死不屈也就罢了,可是他明明不是这样的人,却只是不肯效忠殿下,这样的明君不肯侍奉,难道去侍奉李安那样的人么?”

石彧若有所思地道:“我倒是担心殿下真的动了杀机,若是杀了此人,不仅是天下少了一个才子,还让殿下的声名收到损害,只是殿下担心的也有道理,这样的人才,怎能让他被他人所用,这些日子以来,他对殿下的事情又了解了那么多,就是殿下放心,我们也不能安心的。”

苟廉却道:“我觉得此人对殿下并非无心,只是却有一个我们不明白的碍难之处。”

六道目光立刻落到苟廉身上,苟廉能够出使四方,除了一张厉口之外,他察言观色的本事也起了不少作用,他既然这样说,自然是有几分把握的。苟廉突然微微一笑,一个仆人从远处走了过来,到了四人面前,恭恭敬敬的禀报道:“苟先生,小人问了送江先生回房的仆人,途中江先生突然摘了一片竹页,吹了一个曲子。”

苟廉挥手让他退下,看看三人,董志若有所思地道:“你是说刚才的乐声是江哲吹的。”

苟廉淡淡道:“我刚才听那乐曲技巧并不高明,只是曲调平和中正,发乎于心,而且又不是丝竹之声,所以派人去看一下,果然是江哲所为,此人能够猜到殿下愤怒欲狂,只是他的才智,他吹叶平复殿下的心境,却是他对殿下并非漠不关心,所以我说他必然有一个极大的碍难之处,才让他不肯侍奉殿下。”

石彧道:“可是问题在哪里呢,殿下醇和仁善,又是天纵英明,若要荣华富贵,不过殿下一言而已,若是有什么为难之处,殿下也必然能够替他排忧解难,殿下对他,难道还会不如南楚德亲王赵珏么。”

董志淡淡道:“我们若不能为殿下解忧,还有什么颜面留在王府,殿下如此重视此人,难道我们就不如他么?”

石彧长叹道:“我等所长,不过是济世安民、兵法战阵,虽堪称王佐之才,但是殿下此刻的大敌却不是我们可以解决的,细论殿下之敌,太子李安,其人外虽忠孝,内实阴狠,却偏偏占了大义名份,故而旗下既有胡作非为的小人,也有尊奉皇统的君子,这样一个敌人已经是难以对付,太子少傅鲁敬忠又是一个阴谋诡算,洞察人心的奇才,所以殿下始终不能撼动其储位,齐王殿下,外虽放浪鲁莽,但是从无过分之举,可见他实在是个干才,其人又能征善战,是太子之胆也,有了齐王襄助,太子就可以专心的对付殿下,不必担心日后大雍没有合适的统帅,还有圣上,不是我诽谤君上,圣上妒忌殿下才华,父子相疑已非一日,紧要时或许助殿下一臂之力,平日却是愿意看到太子打压殿下的,这些敌人虽然势力庞大,但是凭着殿下的风范能力,再有我们襄助,殿下还是有五成胜算的,但是最可怕的敌人却是凤仪门,凤仪门主,我曾有缘相见,其人上通天文阴阳、下知地理百家,谋划规断之道无不了然,虽是女子之身,却素有安邦定国的志向,更可怕的是她有自知之明,知道不能明着夺取天下,所以用尽手段控制我大雍朝野,皇上身边的纪贵妃、太子身边的侧妃萧兰、齐王妃秦铮都是凤仪门的弟子,暗中更是不知有多少人在我们身边,她们摆出一心辅佐社稷的姿态,使人尊敬她们的行止,不戒备她们的势力,如今她们已经摆明支持太子,有了她们的存在,圣上、太子、齐王是不可分割的整体,殿下如何对抗她们。”

三人听得心里凛然,他们原本不知道雍王的处境如此艰难,只是对于雍王坚拒和凤仪门的联姻知道一二,处于不同的理由,他们也都反对凤仪门渗透雍王的势力,想不到如今已经几乎是势不两立的局面了。董志深吸了一口气问道:“那么,这些事情,和江哲又有什么关系。”

石彧长叹道:“凤仪门主虽然有惊世绝艳之才,但是却有一个缺点,她毕竟是女子之身,行事不免有几分优柔寡断,有时过于谨慎,要想胜过此人,需要一个独立特行,非常之人,此所谓奇兵胜正兵者也,江哲此人,虽然外表随遇而安,但是内心却是傲然不群,清奇出众,观其为人行事,实在是毫无顾忌,天马行空,观其庙算用计,每每奇谋诡断,出乎众人意料,而且布局深远、思维缜密,行事又是阴柔狠绝,擅长险中取胜,殿下曾经说过此人与殿下几度交锋,殿下都是吃了不少亏,而且毫无反抗之力,当年他给殿下献策,虽然保住殿下平安,可是也成功的离间了殿下和皇上,可是殿下明明察觉他的用意,却没有法子阻止,只有江哲此人,才能胜过凤仪门主,不灭凤仪门,不仅殿下基业不保,我大雍迟早沦于妇人之手,所以殿下才会这样失态,还请诸位体谅殿下苦心,不要不满殿下对江哲的偏爱。”

董志惭愧地道:“多谢子攸先生教诲,志不能替殿下解忧,反而心生嫉妒,实在是惭愧之至。”

石彧起身道:“董兄言重,我们都是殿下的心腹,自然应该全心全意效忠殿下才是。”

在石彧消除了这可大可小的风波的时候,我半倚在床上,喝着解酒的香茶,满满的回想着今日的赏雪,自从飘香死后,我心中常常悒郁烦闷,今日之会,让我心情顿时爽朗,若非我主意已定,只怕就答应了雍王,想起当日在蜀中雍王大营相见之时,我心中还是有些敬畏的,如今我已没有了任何束缚,所以对雍王殿下少了忌惮,可是我却不得不承认,雍王殿下气量非凡,若是换了我,只怕早就这无礼的小子杀了。

可惜啊,无论如何,我都不能改变主意,我江随云从前可以随意的去科考,去献策,可是我现在更珍惜自己的自由,在能够抱住生命的前提下,我不会再将忠诚与人。微微一笑,虽然我好像从来没有过什么真正的忠诚。

在临睡之前,我又想到了雍王世子李骏,那个可爱天真的孩子,可惜啊,按照我对相书的了解,聪明外露而现夭徵,这个孩子只怕没有九五之尊的福气,转念一想,我又笑了,这个孩子虽然面相有些福薄,但是心性应该不错,又有雍王的福气罩着,至少也不会太短命,何况,我又替他可惜什么呢,一个金尊玉贵的皇孙,还有什么可遗憾的呢。

半梦半醒中,我也有些疑惑,在我看来,雍王不是死缠烂打的人,怎么这次这么反常呢,好像非要我臣服不可,这未免有些不合常理啊。

对于雍王李贽来说,恢复平静之后立刻就听到齐王来访的消息并不愉快,可是李显在他面前又是谄媚又是威胁地道:“二哥,你就让我见见江大人吧,当初在南楚我可就认识他了,父皇还说要给他封官,你不是把他软禁起来了吧。”无奈之下,李贽只得同意李显去见江哲。

一走进江哲所居住的栖凤轩,李显就嚷道:“随云,随云,看来二哥对你可是不错,这寒梅小筑是二哥心爱的园子,居然给了你住。”

我正和小顺子下棋,我的棋艺平平,小顺子却下得不错,据他说,下棋有助他练功,如果不是我大局观不错,再加上偶尔出几个怪招,只怕就要惨败了,所以李显一边吵嚷一边走进来的时候,我正皱着眉在想一步棋,小顺子看到李显进来,站起身施礼,道:“奴才参见齐王殿下。”然后轻轻推了我一下。

李显坐在小顺子的位置,见我还在冥思苦想,笑道:“别想了,你的棋艺我可是领教过的,真是臭不可闻。”

我被惊醒过来,看看对面的李显,愣愣道:“齐王殿下怎么来了?”

李显故意露出伤心失望的神情,道:“天啊,难道江大人才看到我这个七尺之躯么?”

我微微一笑,推开棋坪道:“小顺子,给殿下端杯茶来。”

小顺子端了一杯热茶过来,李显接过来,上下打量了小顺子半天道:“你是皇兄府里的内宦么,我怎么没见过,你是新来的么,怎么穿着这身衣服?”

小顺子淡淡道:“奴才是南楚人,曾在南楚王宫见过殿下,殿下自然是不记得奴才的。”

李显愣了一下,看着我道:“怎么江大人身边还有南楚的宫人。”

我笑道:“他是我一个旧交,这次雍王攻破建业,他趁乱离了宫,索性就不回去了。”

李显恍然大悟,说道:“原来如此,江大人身边还有这样一个奴才,大人可真是福分不浅,大人还是让他领个名份的好,若给人参奏你擅自使用阉人,是有罪的。”

我淡淡一笑,道:“江某不过一个草民,最么会有人参我,再说,小顺子是南楚人,难不成大雍还不许他们国破家亡之后另找出路不成。”

小顺子见气氛僵硬,连忙道:“公子,殿下也是一片好心。”

我这才脸色转晴,道:“殿下今日来看我,是随云的荣幸,不过殿下是无事不登三宝殿的,不知道有什么事情用着在下的。”

李显神色变得郑重,道:“江大人,我一见到你就觉得你是我李显最需要的人,别问我怎么知道,可是大人若肯做我的军师,我李显情愿将你当成师长看待,言听计从,绝不二话。”

看着李显殷切的目光,我不由苦笑,李显今年已经刚到而立之年,相貌俊伟的他带着森然的霸气,诚挚而又嚣张的气息让人又是敬畏又是亲近,如果不是有了雍王李贽,我倒认为李显更适合做大雍的君主,这人大事明白,小事糊涂,他选择支持李安,倒未必是李安有多么出色,或者对他如何器重,而是因为,李贽不需要他的能征善战,而李安离不开他的支持,对我来说,选择李显是不可能的,自从得知梁婉的身份以后,我让密营的人开始收集凤仪门的情报,在我到雍都之前,我已经得到了初步的情报,只是一些人尽皆知的消息,其中包括秦铮,齐王妃的出身,虽然是大家之女,却也是凤仪门的高徒,李显,是绝对不可能和凤仪门一刀两断的,而我,因为梁婉的缘故,已经成了凤仪门的敌人,我不敢说这件事永远不会泄露,天下没有不透风的墙,所以我不能投奔李显。

想到这里我正要严词拒绝,却突然想起我诈死的计划,便改口道:“殿下盛情,随云十分感激,只是雍王殿下不许在下离开此地,只能拒绝殿下美意了。”

李显惊讶地道:“怎么,二哥敢软禁你么,你恐怕不知道么,长乐这次回来,带着一本你的诗集,父皇看了十分喜欢,若非二哥说你卧病,早就要召见你了,干脆你就和我回去吧,本王谅二哥不会留难。”

我淡淡道:“殿下误会了,随云身子不好,途中感染风寒,这几日才有些好转,雍王殿下说,随云的身子不好,不许离开此地一步,实在是体恤随云啊,殿下不可误会。”

李显眼珠一转道:“既然如此,我回去派人来邀请你到我齐王府养病如何?”

我淡淡道:“我不习惯王府这种地方,太不方便,若是有什么清静的小府邸或者庄子,殿下不妨帮我看看,随云还有一些积蓄,买的起不大的住处。”

李显搓着手道:“这怎么成呢,我可是要你作师父的,怎么能让你住在外面。”

我故意道:“那就算了,改日我托雍王殿下想想办法,想必总有合适的园子,唉,就是殿下不答应,可就难了,谁让我承了雍王殿下的恩情呢?”

李显连忙道:“没问题,我一定替江大人,不,江先生找一处宅子,既清净优雅,又方便我去拜访的。”

我笑道:“那么随云就多谢殿下了。”

看着齐王兴高采烈的离开,我有些愧疚,齐王虽然有些鲁莽,但是对我倒是一片真心,可惜我终究要辜负他了,其实我最辜负的是雍王,他对我真的很用心,否则怎会突袭建业呢,这是我这段时间想通的,他突袭建业,恐怕我真的就是他所要得到的收获吧。

李贽送走了齐王,面色苍白的走进了大门,齐王的得意洋洋让他心灰意冷,石彧也是十分失望,他绝没有想到齐王如此轻而易举的就得到了江哲的认可,那么自己这些人又算什么。

回到书房,李贽淡淡道:“子攸,明日替我设宴,为江先生送行。”

石彧扑通跪倒在地上道:“殿下,不可放过此人啊。”他的声音颤抖而慌乱。

李贽的声音十分平静,他淡淡道:“替我准备藏锋壶,我要送他远行。”他的声音十分缥缈。

石彧身子一颤,道:“遵命。”他的目光充满了悲伤和绝望。

李贽抬起头道:“子攸,我做的对吗?此人若随了齐王,我寝食难安,不如杀之以绝后患。”

石彧凄然道:“毒杀此人,可以免除后患,不杀此人,我等死在顷刻。”

李贽泪落,黯然道:“可是杀了此人,本王于心难安,本王一向自负宽宏大量,如今却对一个不肯归顺本王的人下了毒手。”

石彧谏道:“殿下不可心软,此人惊才绝艳,若是放过,殿下大业危矣。”

李贽无力的摆摆手道:“本王已经下定决心,明日,就用销魂丹吧。”

石彧道:“是,这样一来,他会在十二个时辰后无病而终,不会有什么痛苦的。”

李贽没有作声。

\chapter{第二十九章 千钧一发}

我看着小顺子收拾东西,心里一阵茫然,明天就要恢复自由了,可是我却高兴不起来,良久,我狠下心来,对于一个明天绝对会鸩杀我的人,我何必还要费心。这时雍王府的仆人来通报,说是管休、董志、苟廉前来求见。他们是来尽最后的努力吧,心里一阵温暖,无论如何,他们都是不错的人,既然从今以后,再也没有同桌共饮的机会,不妨秉烛夜游一次吧。我笑着让仆人请他们进来。

管休他们都是聪明人,聊聊数语,就知道我的心意已决,便都不在多言,我们尽情的谈论着,一夜无眠。到了天明之时,我看看窗外的曙光,笑道:“天下没有不散的宴席,今日一别,他年相见,恐怕已成陌路了。”

苟廉凄然道:“随云既然知道如此,为什么还要投靠齐王。”

我微微一笑,道:“齐王殿下鲁莽直率,我不过在他麾下消磨几日,过一段时间,我就会离开长安,到时候,我们是友非敌,诸位就不必过虑了。”

董志低声道:“只怕齐王殿下也不愿放先生离去呢?”

我只是淡淡道:“几位请回吧,江某今日离开雍王府,殿下已经说过要为江某送行,随云总不能这样去见殿下,总要沐浴更衣,才好和殿下告别。”

管休起身道:“既然如此,我们也不打扰随云了,一会儿送行,我们就不去了,免得临别伤心,黯然销魂者,唯别而已。”

送走了三位谋士,我走进后面的厢房,这间厢房是专门的浴室,在房间中间是一个宽约五丈的浴池,整个池子是由青石铺成,进水口在浴池中央,上面是一朵出水荷花,在池底青石之下铺着铜管,将从园中引过来的清泉水加热之后,按动进水机关,温热的清泉水便从莲花喷头四散喷出。我进去的时候小顺子正在往池子里面放水,水雾四起,飞珠走玉,我微微一笑,皇家的享受果然不凡,每次我进来的时候都会这么想。

宽衣解带,走进浴池,享受着热水沐浴的舒畅,我笑道:“小顺子,你说,我以后也建一座这样的浴池好不好?”

小顺子没有回答我的话,我有些奇怪,回头看去,小顺子似乎在神游天外,我奇怪的摇摇头,不过我没有惊醒他,他在我面前是不会隐瞒心事的,我想很快他就有话对我说了。

沐浴之后,我穿上小顺子准备的衣服,这是我特意吩咐的,从最里面的内衣到最外面的儒衫,都是雪白的颜色,当我认真的穿上一件件衣服的时候,小顺子突然跪倒在地,悲声道:“公子,求你不要这样为难自己了。”

我微微一愣,正要接过他递过来的外袍的手停住了,问道:“小顺子,你在说什么?”

小顺子道:“公子一心要为夫人报仇,小顺子是知道的,请问公子,若要为夫人报仇,都有哪些计策。”

我看看他,淡淡道:“你我休戚相关,我不瞒你,早在知道罪魁祸首之后,我心里就有了上中下三策。”

小顺子道:“请问公子下策?”

我接过他手中的儒衫,缓缓道:“下策最为艰难,待我从雍都脱身之后,就要隐身市井,等待时机,所谓智者千虑,必有一疏,精心等待,终有机会刺杀李安,就是刺杀不成,千里之堤,溃于蚁穴,我游走天下,培植不满李安的势力,现在东川还未衷心顺服,南楚不日就会重新立国,借天下之力,再有雍王在侧虎视眈眈,我终有报仇雪恨的一天。只是杀害一国储君,不是一件小事,事成之后,我需要尽散部下家财,从此浪迹天涯,而且稍有不慎,就是败亡的命运。”

小顺子低声问道:“请问公子中策?”

我披上儒衫,淡淡道:“中策好一些,太子李安的左膀右臂是齐王,齐王虽然鲁莽,但是外粗内细,实在是当世俊杰,若无雍王,齐王为君也不错,我投靠齐王,替他出谋划策,挑拨他兄弟不合,到了适当时机,让他内乱萧墙,不管是便宜了雍王,还是便宜了齐王,我终究让太子折翼陨身,就算达不到目的,也可以让大雍内乱,一报国仇,二雪私恨。”

小顺子膝行向前,道:“请问公子上策?”

我系上衣带,笑道:“这上策最是光明正大,我归顺雍王,借刀杀人,令雍王殿下弑兄杀弟、逼父退位,不但我大仇得报,天下也得到一个明君圣主,一统曙光近在眼前,我江哲亦可留名青史,事成之后,或者归隐田园,或者安享富贵,这不是上策吗?”

小顺子严肃地道:“公子,这些年来,小顺子始终在你身边伺候,公子的心思小顺子怎么会不明白,公子明明知道投靠雍王是最好的选择,为何如此固执,公子的仇人也是雍王的敌人,只要公子归顺雍王,雍王登上大宝之时,就是公子大仇得报的时候,公子始终不肯归顺雍王,并且蓄意挑衅,迫得雍王定要杀公子而后快,其实只要公子顺从了雍王,等到报仇之后,公子便归隐山林,也能够博得一个安享余年,何必要这样冒险,公子虽然医术不凡,但是大雍皇族密藏的毒药未必就能解救,万一公子若是不幸,小顺子就是杀了雍王又有什么用呢?”

我淡淡道:“这些事情,我如何不明白,可是我平生行事,对敌人可以不择手段,却从来不会对亲近之人擅用心机,雍王殿下,旷代明君,对随云推心置腹,为了随云一人,用了多少心思,千里路遥,殿下解衣推食,随云并非铁石心肠,焉能不动心,可是我受南楚恩泽在前,与大雍结怨在后,已有隔阂在心。何况若是真心相从,便要尽心竭力为殿下设想,若无我筹划,殿下未必没有胜算,虽然惨烈,但是声名无瑕,若是我归顺殿下,随云乃是凡人,不免借机了却私怨,为我私心,伤害君臣大义,我若秉公,又如何对得起飘香泉下香魂,想来想去,既不愿害殿下青史上留下污名,也不愿愧对飘香吾妻,唯有舍易就难。至于中策,虽然无伤我心志,但是不免令雍王大受损伤,这样的明君,我不能为之效力已经愧疚于心,又怎忍伤害于他,所以只得采用下策。”

小顺子道:“公子不肯侍奉雍王,却是为了雍王着想,但又何必逼得雍王杀害公子呢,若是假意答应,过一段时间,逃出长安又有什么难处。”

我笑道:“我平生行事,小事上面或者不大谨慎,但是这等之事,却是绝不肯谎言欺骗的,当初我不肯为德亲王效力,也不曾谎言骗他,今日我既然不肯替雍王效力,也绝不会骗他,何况若不迫雍王杀我,我如何能够断绝归顺雍王的心思。小顺子,你记着,我今日诈死,确实有几分危险,所以我若是不幸,你记得,不可替我报仇伤害雍王,雍王殿下没有错,一个霸主,是绝对不能心软的。我只要你记着,有朝一日替我杀了李安,然后带着我的骨灰回南楚,将我和飘香合葬,你可答应么?”

小顺子俯首在地,良久才带着哭音道:“公子之命,奴才怎么会不听,若是公子不幸,待我杀了那李安之后,就回南楚,为公子守墓终生。”

我淡淡道:“多谢你了,其实我胜算很大,你也不必难过,过了这一关,天下就没有什么可以羁绊江某的了,就是报仇,我也不会牺牲自己余生的,你可以放心。”

小顺子默然不语,我知道他不信我,其实我说的是真的,我从来不会为了报仇而疯狂的。

在雍王的书房,李贽默默的看着书案上的一把银壶,石彧站在案前,忧心忡忡地道:“殿下为何不使用大雍密藏的鸳鸯壶,而使用这把这把藏锋壶呢?”

李贽淡淡道:“前朝秘制的鸳鸯壶虽然可靠,但是江哲熟读经典,精于鉴识,未必不认得鸳鸯壶,这把藏锋壶乃是本王在南楚的属下送来的,机关精巧,绝无破绽,还是使用这把壶吧,销魂丹不会让银壶变色,江哲不会察觉的。”

石彧多年跟随李贽,他能够感觉到李贽心里的悲伤,不由道:“殿下,刚才管休他们前来禀报,说江随云声称不会久事齐王。若是殿下不忍,不妨放过他。”

李贽漠然道:“你真的是这么想的么?”

石彧欲言又止,终于道:“都是属下之过,鼓励殿下求索贤才,可是如今殿下一不能平定南楚,二不能得到贤才,都是属下的罪责,但是这人,若是不杀,只怕属下日夜不安。”

李贽微微冷笑道:“没有你的事情,是本王太自信了,以为天下贤士都会效命于孤,罢了,就在前厅为江哲饯行吧,可怜绝世才子,从此黄土深埋,这是本王的罪孽,也是他的不幸。”

离雍王府不远处,一辆华丽的马车静静的等候,车内,齐王李显喜形于色,在他对面坐的是王妃秦铮,如今的秦铮不再是女扮男装,一身月白宫装,淡扫娥眉,天香国色,她淡淡道:“不就是那个翰林学士江哲么,怎么殿下这样看重他呢?”

李显眼中闪过一丝嘲讽,语气诚挚地道:“当年铮儿你舌厉如刀,也没有说服德亲王,可是此人三言两语就说服了赵珏,据说此人随同赵珏平蜀,我曾细细研究赵珏平蜀的方略,见其风格不同平常,可见江哲此人果然是有才华的,更何况我爱此人风采,已非一日,就连二哥都对他十分爱重,我折节下交又有什么不妥,不是我说你,铮儿你当世才女,家世容貌才华无一不是上上之选,可你唯一的缺陷就是少了谦逊容忍的性情,也难怪,你是天之骄女,本王有话在前,你若得罪了江先生,休怪本王无情。”

秦铮眼中闪过一丝怒色,当年自己奉命接近齐王,这齐王翩翩年少,又是一个风流倜傥的人物,不久便令秦铮倾心,在南楚自己因为嫉妒而中了齐王的圈套,一夕风流,自己成了齐王的未婚妻,可是从此之后,齐王故态复萌,不是走马章台,就是呼鹰逐兽,对自己若即若离,时而亲爱如蜜,时而冷淡如冰,自己还没有嫁入王府,就有了三四个庶出子女。可恨自己神魂颠倒,不能自拔,一直到最近才奉皇命成婚,可是李显虽然表面上对自己尊重非常,但是却在王府内院划下禁地,在里面声色犬马,自己也曾向师父和父亲哭诉,可是他们都说这是齐王风流本色,自己只能恪守妇道,用柔情羁绊,无奈之下,自己只能睁一只眼闭一只眼罢了,可是今日齐王变本加厉,竟让自己对一个南楚俘虏低头,但是她想来想去,终是不愿和齐王闹翻,只得隐忍道:“臣妾遵命。”

李显淡淡一笑,他很了解这个聪慧的女子,心高气傲,却是少了几分温柔,没有过多的劝导,他知道秦铮不会明里违背自己的命令。看着已经接近约定的时间,李显道:“好了,这就去雍王府吧。”

雍王府的大厅却是与众不同,不像一般王府一样富丽堂皇,只是宽阔深远、肃穆庄严,今日李贽也是一身素服,他原是心里存了哀悼之心,见到江哲,他就是一愣,江哲也是一身素服,趁着他清秀儒雅,略带憔悴的容貌,更是显得气度雍容。

他心中一阵惋惜,目光落到石彧身上。因为今日是要鸩杀江哲,所以李显只带了石彧相送。石彧目光冰冷,微微摇头。李贽不再犹豫,微笑道:“今日为先生送行,知道先生品性高洁,故而只能一杯美酒送行。先生不要推辞。”说罢,石彧端来一个黑木托盘,上面放着藏锋壶和两个银杯。

我的目光掠过藏锋壶,不由莞尔失笑,这藏锋壶是我亲自设计,通过天机阁出售,为了得到高价,只做了三把,每壶千金,想不到今日重逢在大雍,此壶壶底有一夹层,可以容纳一杯毒液,若是用此壶害人,只要将毒液注入壶底,倒酒之时只要按住壶把上面的莲花雕刻,那么壶底的毒液就通过壶壁流到壶口,从壶口旁边雕刻的莲花心倒入酒杯,这样用毒,神不知鬼不觉,就是杀了千人百人也不露形色,当然这毒药必须不让银壶变色,这样既可以免得什么人都可以使用藏锋壶害人,也是为了让喝酒之人不起疑心。想不到今日这藏锋壶用到了我自己身上,不知道这算不算是自作自受呢?

李贽拿起藏锋壶,先倒了一杯毒酒,又移开拇指给自己倒了一杯净酒,放下酒壶,他端起自己的酒杯,勉强笑道:“先生请满饮此杯,从此飞黄腾达,青云直上。”

我接过那杯毒酒,心想,若是我为此人呕心沥血,最后得到的也恐怕只是这杯毒酒罢了,看向雍王,见他强颜欢笑,淡淡一笑,想到此人从前恩宠,不由开口道:“殿下龙日天表,贵不可言,从今之后,只要外修兵甲,内修德政,太子纵然忌惮,也不敢轻易挑衅,至于其他事情,自有贤士为殿下谋划,哲今日辞别殿下,今日恐相见无期,愿殿下早日一统天下,令四海升平,百姓安乐,随云虽在江湖之远,也将为殿下日夜祈福。”我这一番话全是发自肺腑,我真的不怪他,他要杀我都是因为我逼他太狠,真龙自有逆鳞,想到今日之后不会再见,不免说了几句心里话,端起酒杯,我能够分辨出酒里面的毒药,我所配制的万毒降也是剧毒之药,但却能够护住心脉保住我的性命,今夜就是我诈死的良机。举起银杯,我就要喝下这一杯毒酒。

李贽从江哲接过酒杯,心中就是十分不安,他从未作过这种杀害贤才的事情,未免有些愧意,此刻听到江哲这一番肺腑之言,那有千钧之力的右手竟然颤抖起来,此时眼见江哲就要喝下毒酒,胸中血气翻涌,突然伸手按住了酒杯。

我疑惑的望着李贽按在酒杯上的手,看着他苍白的脸色,心中一片混乱,李贽虽然开始只是一时冲动,但是他很快就冷静下来,他拿走酒杯,淡淡道道:“先生虽是文士,可是胸襟气魄,不逊沙场壮士,当用大杯,不应该用此银杯,来人,拿我的金盔来。”

不多时,侍奉的仆人捧来了李贽上阵杀敌所穿的御赐金甲的头盔,李贽没有使用藏锋壶的机关,打开了壶盖,将壶中美酒全部倾倒在金盔之中,双手举起,道:“江哲,你虽是南楚繁华之地的才子,但你的心志品性却胜过我大雍的边关勇士,本王用金盔赐酒,望你一路顺风。”这一刻,李贽心中再也没有愤恨怨责,而是一片平和,他心想,不能让江哲为我所用,是我缺少德才,我若擅杀无罪贤士,就是帝位在我面前,我又有什么资格坐上去呢。

\chapter{第三十章 风虎云龙}

南楚至化元年十二月,江哲禁于雍王府,王虽倍加礼遇,但哲心志不屈,齐王显,颇爱哲才,促雍王赦之,雍王不得已许之,因哲品性高洁,乃以御赐金盔盛酒相送,哲乃感激涕零,遂降雍王。

--《南朝楚史·江随云传》

我几乎是下意识的接过金盔,脑子里满是李贽按在酒杯上的情景,他竟然放过我了,放过我这样一个屡次冒犯他的狂生,而且还是可以让他大业成灰的心中毒刺。不知怎么的,我的眼泪一滴滴坠落,落在金盔里,落在雪色的衣襟上,我几乎不能行动,想起当日德亲王一旦觉察我不可能忠心耿耿的效命南楚,就对我十分提防,我在建业养病,德亲王的密探始终在监视我,想起我最后一次上的谏表,一片赤心为了南楚,可是换来的只是贬斥,从前我以为对这些根本不在意,到今日我才发现这些都深埋在心里,这是连我自己都无法觉察,或者是不愿想起的悲凉往事。

我端起金盔,也不顾忌酒液溅落,一口气喝下了盔中美酒,心中暗想,这大概就是诸葛武侯为何鞠躬尽瘁的原因吧。美酒甘冽,我觉得胸中防若火烧一般,举起金盔,我拜倒在地,朗声道:“殿下深恩,臣虽肝脑涂地,不能报答万一,若殿下不嫌弃臣反复无常,臣江哲愿为殿下效力。”

李贽原本已经心灰意冷,不料我竟然突然归顺,一时之间,也不知该说些什么,还是石彧聪明,轻轻推了李贽一把,李贽连忙上前将我扶起,激动地道:“先生,你竟然回心转意,本王,本王真是不知道该怎么说,快,快,快起来。”

我这手无缚鸡之力的书生哪里有反抗的余地,被李贽给扶了起来,我心情已经渐渐平复,低吟道:“若使当时身不遇,老了英雄。汤武偶相逢。风虎云龙。”看看李贽,淡淡道:“殿下宽宏大量,饶臣性命,臣无以为报,只有为殿下鞠躬尽瘁,才能补偿这些日子对殿下的冒犯。”

李贽手一抖,震惊的看着我,他原本正在心里庆幸自己没有鸩杀江哲,否则岂不是失去了贤士,可是听我语气,我竟然是知道了他酒中下毒的事情。

我看着神情不安的李贽,微笑道:“殿下不必过虑,若非殿下手下留情,哲也不会甘心效命。”

李贽看看石彧,石彧早已经遣退了下人,这时听了我的说话不由心一颤,也看向李贽。

我也不隐晦,道:“殿下,石先生不必多心,哲从前愧对殿下恩典,殿下赐死也是理所当然,如今事过境迁,臣不会记恨,还请殿下不要见过这些日子臣的狂妄。”

我这样说,并非是揭短,既然我已经决定了效忠雍王,就要考虑到君臣相处之道,雍王想要鸩杀我,和我故意挑衅雍王,这些若是记在心里,将来不免成了嫌隙,现在我这样提出来,雍王就不会觉得愧对我,也不会记恨我对他的冒犯,将来自然君臣相安,可别说我心思太多,自古以来总有鸟尽弓藏的讥讽,但是明确说来,君王忌惮功臣是一个原因,臣子逾越臣道也有责任,所以我要为了今后留下后路。

李贽果然神色数变之后,终于开朗起来,道:“先生不怪罪本王就好,贽愿任命先生为天策帅府司马,和子攸同心协力辅佐本王。”

我再次下拜谢恩,李贽苦笑道:“先生不必这样拘礼,我视先生如同师友,先生可不要如此疏远。”

我笑道:“尊卑之礼不可轻废,随云岂可失礼,不过若是殿下不怪罪随云礼数不周,随云就不客气了。”这才是我的本意,我既然归顺了雍王,以后不免日日相见,若是总是恭恭敬敬,多痛苦啊,反正在雍王登基之前,我是不用太考虑礼数的问题的。

目的已经达到,我便正色道:“殿下,随云也想和殿下深谈,可是现在不行,请殿下遣人通报齐王,就说随云突然旧病复发,只得留下养病,齐王必然要亲来探望,随云斗胆,请殿下亲侍汤药,这是其一,其二,随云虽然对大雍之事略知一二,但是朝中势力纠结,仍然不甚明了,请殿下将现在所能收集到所有情报送来,待随云研究之后,今夜再与殿下详谈,其三,管休等人还不知今日之事,心中未免有些嫌隙,请子攸先生前去告知,不妨隐晦相告今日事情,以彰殿下仁德,且安谋士之心,此三事都是至关紧要。”

李贽听了我的话,眼中一亮,道:“随云果然思虑周密,本王立刻照办,本王陪同先生立刻回到客院,子攸,你先去通知齐王。”我和石彧相视一笑,石彧匆匆而去,我则做戏做到底,让雍王扶我出了大厅,在外面等得焦急无比的小顺子看我出来,连忙走了过来,冷冷的看了一眼雍王,道:“公子,发生了什么事情。”说着接替雍王搀住了我。我淡淡道:“小顺子,你有法子让我暂时生病的,我要见齐王。”

满怀欣喜的李显到了雍王府,却是一盆冷水当头浇下,当石彧告诉他江哲旧病复发,李显的第一个反应就是雍王故意强留,但他转念一想,雍王手段不会如此拙劣,不管如何,李显还是提出要见江哲一面,将秦铮留在车上,李显直奔栖凤轩而去,他心中满是恼怒,可是当他一走进房间,就看到江哲满面苍白的躺在软榻上,而自己的二哥,正在聚精会神的端着一碗热气腾腾的汤药,正在那里吹气,看到自己进来,只是微微一笑,道:“六弟,江先生昨夜和我府中几位幕僚秉烛相谈,今日我为他送行,先生多喝了几杯酒,竟然旧病复发,恐怕去不得了。”

李显看看江哲的面色,怒道:“怎会这样巧,他刚生病你的药就煎好了。”

李贽淡淡道:“江先生自从到了王府,几乎每日都要服药,这是常例,故而为兄吩咐随时都要备好汤药,幸好如此,今日先生突然发病,若没有此药,只怕先生又要卧病多日了。”

我艰难的睁开眼睛,在心里咒骂小顺子为什么这么认真,输入我体内的那缕阴寒真气令我浑身发冷,举动艰难,我有气无力地道:“随云自从在蜀中染病,就时常发作,不拘时刻,想不到偏偏赶在今日,真是愧对齐王殿下。”说罢,我咳嗽了几声。

雍王殿下轻轻尝了一口汤药,道:“好了。”说罢让小顺子扶起我,雍王亲自喂药。我服药之后,面色似乎好了一些,道:“两位殿下,随云服药之后,便得小睡,还请两位殿下不要见怪。”

雍王连忙道:“先生请好好休息,本王这就走了。”

我轻轻点头,用“感激涕零”的眼神望着雍王,然后似乎慢慢睡去。

雍王起身低声道:“六弟,我们不要打扰先生,到外面说话吧。”

出了栖凤轩,李显神色木然道:“天意如此,看来二哥你赢了。”

李贽笑道:“六弟多心了,等到江先生病好之后,自然会去齐王府的。”

李显冷笑道:“他病得好啊,堂堂天策元帅,雍王殿下,亲自侍奉汤药,他若再不动心,我倒要奇怪了。”

李贽心道,我从前也没少干,可惜他就是不肯归顺,口中却说道:“六弟多心了。”

李显拂袖而去,出了府门,也不上车,拽过一个侍卫的马匹,泄愤的狠狠抽了一鞭,那匹骏马嘶鸣一声,飞奔而去,李显不理会身后人的呼唤,愤然离去。

在栖凤轩中的我,让小顺子解开我身上的禁制,笑道:“我身上又是冷汗,又是酒气,快,我要沐浴。”

小顺子笑道:“早就准备好了,公子的脾气我还不知道么?”

我看看他,道:“你不问我怎么改了主意。”

小顺子淡淡道:“十几丈距离,我听得很清楚,公子的决定小顺子从来不会置疑,公子放心,只要小顺子在,谁也不能伤害公子。”

他说话的语气是那样淡然,又是那样坚决,我心里一暖,道:“那是当然,小顺子,你可要好好练功,在宦海之中我可以明哲保身,但是天下还有另一个世界,若是有绝顶高手刺杀我,可就要看你的了。”

小顺子眼中闪过一丝激昂的神色,口中却冷冷道:“公子放心,当初公子给我的剑谱,我都已经融会贯通,虽然有些人我胜不过,可是谁也别想轻易过了我这关。”

我点点头,小顺子一向不会虚言夸耀,但我又疑惑地问道:“我记得有些剑谱你说需要阳刚的真气,怎么现在也能用了么?”

小顺子淡淡一笑:“公子精通易理,难道不知道阴极阳生的道理么?”

我看着小顺子掩饰不住的喜色,虽然不甚明白,也知道小顺子的武功已经到了一个新的境界了,心想,我听人说过,练功得花上二十多年,才能登堂入室,怎么小顺子今年才二十出头,就这么厉害呢,莫非他真是练武的天才。却不知我的胡思乱想倒大半对了,小顺子天资聪明,性情坚忍不拔,练的武功又是合乎身体状况,再加上这些年跟着我,文理上也有了不小的成就,所以才能有今日的成就,虽然比起三大宗师来说还差的很远,但是已经远远超过了普通意义上的绝顶高手了。

换了一身青衣,我心情愉快的跟着石彧来到了雍王府的机密书房,这里位于王府右侧,守卫森严,在这间普普通通的书房里面,却收藏着王府的所有机密文件,除了雍王本人和石彧之外,其他人谁也不能擅自进入,照料书房的是四个十八九岁的书童,这些人个个举止得体,步履矫健,可见都是雍王的心腹亲随,换了一个时候,只要一道谕令,就可以成为雍王的得力干将,我暗自称赞雍王确实不凡,便走进书房,开始查询我需要知道的情报,虽然小顺子已经将从陈稹那里得来的情报告诉我,但是怎么比得上雍王收集的情报全面,留下来伺候我的书童十分得力,我按照目录索取文书,他都能立刻取来,虽然没有小顺子在身边伺候,有点不习惯,不过没关系,以后我会在自己的书房工作,这里的东西,我看过一遍就够了。

李信再一次偷眼看向那个二十多岁的俊秀青年,心中满是好奇。李信的父亲本是雍王的亲卫,在一次行刺中身亡,只留下一个孤儿,李贽见他孤苦无依,就将他收到府中照看,过了数年,他的勤奋好学和忠诚机敏得到了李贽的赏识,赐给名姓进了机要书房,在这里虽然行动受到严格的约束,但是能够参与机要,跟着雍王殿下身边,更是受益不浅,而且雍王早就说过,等到他们成年之后,就要让他们出去做官,李信很清楚这是一条青云之路,当然代价就是自己需要永远忠心,和怎样都不过分的谨慎小心,所以好奇是最大的缺点,曾经有一个书童一时好奇偷看了文书,犯了规矩,被雍王发现之后,一向和善的殿下勃然大怒,下令杖杀,李信永远都记得当时的惨况,所以他从来都不逾越本分,他明白不应该猜测这个青年人的身份,但是当他发现雍王殿下就在另一间书房等着这个青年人的时候,他还是忍不住好奇之心。

在另一间书房里面,李贽虽然在看着兵书,但是总是心神不安,他看看石彧,道:“子攸,你还是去休息吧,本王自己等他就行了,你不要太劳累了。”

石彧笑道:“今日江随云一归顺殿下,便雷厉风行,先让齐王放手,再让谋士安心,子攸十分叹服,所以也很想知道他会向殿下献上何等策略,急切之心,不在殿下之下。”

李贽笑道:“是啊,我真的很期待他的献策,目前的局势你很清楚,本王身陷罗网,越是挣扎,网子勒得越紧,我真的很想知道他有什么法子让本王脱出重围。想来真是吓了本王一身冷汗,我若真的鸩杀江哲,恐怕真是万劫不复了。”

石彧道:“是啊,多亏殿下仁德,否则江哲岂肯心悦诚服,属下想来想去,恐怕我们的心思都在他掌握之中,今日这杯鸩酒,恐怕就是江哲对殿下的考验。”

李贽疑惑地道:“可是本王若没有悬崖勒马,他难道真的会喝了那杯毒酒么?”

石彧苦笑道:“这个属下也无法得知他的安排,不过事情既然没有到了那个地步,殿下也不必费心了。”

李贽也笑道:“是啊,过去之事,本王还多虑什么呢,子攸,只怕他不会出来得很快,我们不妨下一盘棋吧,也好消磨时光。”

石彧道:“殿下既然有此雅兴,属下自然奉陪,还请殿下手下留情。”

两人相视一笑,摆上棋盘,对弈起来。

过了片刻,书童李忠走了进来,禀报道:“殿下,属下去送茶的时候,看见江先生似乎有些烦闷,先生还问他的下人在哪里。”

李贽一愣,看看石彧,石彧心思一转,道:“殿下,属下看江哲十分倚重他身边的那个李顺,似乎片刻也离他不得,而那李顺对江哲也是忠心不二,不如让李顺进去伺候,反正以后李顺也不免接触机密的。”

李贽想了想道:“不错,李顺此人,不是凡品,他们主仆之间必然情谊极深,李忠,你派人栖凤轩召李顺来书房伺候。”

过了一阵子,李忠又回来禀道:“殿下,石先生,江先生十分开心,李信回禀,那个李顺很守规矩,只是专心伺候,从不留心文书内容。”

李贽这才放下心来,道:“这就好了,子攸,该你了。”

石彧看看棋盘,笑道:“殿下神思不属,这盘棋看来属下要赢了。”

李贽苦笑着看看被困住的白龙,道:“是啊,本王输了。”

石彧道:“这是属下专心,殿下不用挂心那边了,下一盘可别让属下得胜了。”

李贽一边拣棋子,一边道:“好,看本王杀的你血流成河。”

两人渐渐投入进去,当第三盘棋局告终之后,石彧起身,看向窗外,此时正是黎明时分,窗外漆黑一片,李贽看看棋盘,道:“本王赢了半子。”

石彧笑道:“殿下棋力不凡,只要稍为用心,属下就一败涂地了。”

就在这时,李忠进来禀道:“殿下,石先生,江先生请见殿下。”

李贽一听,顾不得再拣棋子,跳起来道:“他神色如何?”

李忠道:“先生神色虽然有些疲乏,但是气度十分平和,还和小人说笑,说让小人去把殿下从寝宫拽出来呢?”

李贽悬了一夜的心终于放了下来,他长长的出了一口气,道:“还好,还好。”

石彧看向窗外,惊喜地道:“殿下,你看。”

李贽抬头望去,只见窗外,破晓的阳光已经穿过厚厚的云层,东方天空已经泛白。李贽笑道:“好兆头,走,我们去见江哲?”说罢向外走去,石彧看着李贽龙行虎步的英姿,不由放下了心中的一块大石,便也随后跟去。