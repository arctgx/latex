\part{第四部 烽烟再起}

\chapter{第一章 烈焰红妆}

原本应该今日晚上发文的,可是为了弥补两周以来的苦等,今天上午先发半章,两周以来,本想潜心写作,可惜工作太忙,至今存稿不多,想来不会让大家看得十分畅快了,虽然还是会每周更新五次,可是数量上若有不到之处,还请大家海涵,我的新书已经在欣辰出版社出版了,繁体版,还请多多支持。

————————————————————————

大雍武威二十七年,北汉边陲,秋风飒飒,从雁门关通往代州城的大道上,一个红衣骑士飞马奔驰,烟尘滚滚,只能够隐隐看见那是一匹浑身皮毛似血,红鬃如焰的胭脂马,那骑士周身虽然被大氅和上面的风帽挡住,看不见容貌体态,但是隐隐可以看见那人一身红色劲装,外面罩着同色的大氅,后肩斜背一张乌木檀弓,马鞍旁边挂着一袋白翎箭,腰间隐隐露出镶金嵌玉的刀柄,刀身被大氅掩住,看不见刀鞘何等样式,但是只见刀柄就知道这是一柄千金难换的宝刀。

那红衣骑士正在纵马狂奔,突然从两边斜次里冲出来五个骑士,都是披发左衽的蛮族骑士,冲向那红衣骑士,双方即将撞在一起的时候,那个红衣骑士迅速地张弓射箭,白羽箭如同流光闪电,一弓三箭,弓弦声响,有两个骑士料不到这个红衣骑士竟然能够在这样短的距离开弓射箭,翻身落马。可是一弓三箭对这个红衣骑士未免有些勉强,第三支箭便软弱无力,被一名骑士用刀拨开。剩下的三名骑士一边大声呼喊,一边狠狠杀来。那红衣骑士已经来不及发箭,只得拔出宝刀迎接。四个人都是马战娴熟的骑士,战得热火朝天,那个红衣骑士虽然宝刀锋利,骑术高明,可是那三个蛮族骑士也是勇猛的战士,渐渐的,红衣骑士开始有些招架不住。突然,那红衣骑士突然一声娇喝,喊道:“看毒粉。”左手一扬,一团粉红色的烟雾向两个蛮族骑士扑去。那两个骑士左右闪开,露出了一线空隙,那红衣骑士趁机催马,冲出了包围,向来路冲去。那几个蛮族骑士反身追去,谁知刚刚将要合围,那红衣骑士一提马缰,那匹胭脂马竟然前蹄高扬,反转马头,如同行云流水一般,速度丝毫不减的向代州城奔去。那几个蛮族骑士料不到那红衣骑士骑术也会如此厉害,不由滞了一滞,等他们翻身追去的时候,已经落后了许多。

红衣骑士苦恼地回头看了一眼,那几个蛮族骑士还是紧追不舍,红衣骑士银牙紧咬,她倒不担心安全,再往前二十里就是代州城,这几个不知如何混进来的蛮族骑士是绝对不敢紧追不舍的,但是若是被人知道自己独身出游却遇伏击,那么今后数月自己可就别想这样自由自在了。这时,她眼睛一亮,前面有一个灰衣骑士正在缓辔前行,那匹马也是百里挑一的骏马,马上的骑士也带着弓箭,这代州一地不论男女都是弓马熟稔的战士,这个骑士再不济也可以阻挡一下,两人联手,或者可以杀了那几个蛮子。想到这里,那红衣骑士高声喊道:“老兄,快放箭。”

那灰衣骑士愕然回首,眼中立刻闪过一丝寒芒,回身迎来。那匹黄骠马和红衣骑士错身而过,红衣骑士耳边听见弓弦响声,只听弦声,红衣骑士就知道这张弓力道并不强,在代州一般只有女子才会使用。可是听到弓箭破风之声,红衣骑士不由愕然,那分明是一弓五箭。她策马回身的时候,正好看见五支羽箭排列成前三后二的箭阵,其中一支羽箭射入了一个蛮人的咽喉,另外两支羽箭刚被另外两人挡开,后发的两支羽箭已经到了,两人虽然努力闪避,却是只避开了要害,双双中箭重伤。两人互相望了一眼,策马奔去,临去之时还带走了身死的同伴和无主的战马。

红衣骑士松了口气,上前拱手道:“多谢兄台救命之恩,林彤在此拜谢。”

那个灰衣骑士回过头来,眼睛就是一亮,只见这红衣骑士梳着三丫髻,包头的红色丝巾旁边插着一支金簪,不过十六七岁的年纪,肌肤如雪,双眉弯弯,一双乌溜溜的黑眼睛晶莹剔透,粉红娇嫩的樱唇嘴角微微上翘,显得调皮娇俏。

那红衣骑士却也看得呆了,那个灰衣骑士也是一个二十出头的少年,相貌清秀俊美,甚至带着几分文弱,可是眉宇间却带着几分看透世情的透彻和玩世不恭的闲散。这红衣少女平日所见多半是五大三粗的大汉,就是其中颇为俊美的男子也多半是英武俊朗的类型,哪里见过这样秀气的美男子,不由脸一红,问道:“你是什么人,我见你不像是我北汉之人,不会是奸细吧?”

那灰衣骑士已经平静下来,笑道:“这位小姐,这可不是报恩的道理啊,哪有把恩人当成奸细的?”

红衣少女林彤脸又是一红,道:“你救我的性命是一回事,是不是奸细是另外一回事,你如果不说,我可要送你去见官的。”

那灰衣骑士故意夸张地道:“哎呀,红霞郡主果然是了得,看来我可是救错了人呢?”

红衣少女不由愣住了,她乃是镇守代州和雁门一带的鹰扬将军林远霆的次女,林远霆乃是代州世家家主,北汉重臣,娶妻安庆长公主,长公主生了四子二女,四个儿子都是有名的虎将,长女林碧被当今的北汉主刘佑收养为义女,赐封嘉平公主,今年二十三岁。

林碧不仅美丽聪慧,更难得是武功军略出众,曾经多次迎击蛮人的进攻,立下了赫赫战功。迎娶林碧一直是北汉的勇士摩拳擦掌的目标,而林碧也立誓除非是志同道合的盖世英雄,否则终身不嫁。可是这样一个女子,有几人配的上呢。直到两年前,威远将军龙庭飞发妻去世,林碧才花落龙家。龙庭飞当时不过二三十岁,又是英俊威武,位高权重,战功赫赫,北汉主既重用他,也不免有些忌惮,为了笼络重臣,联姻自然是最好的法子,林碧既是才貌双全,又是出身皇室,自然是最好的人选,而且龙庭飞也是配得上林碧的豪杰,所以这桩婚事也就成了一时美谈。不过因为龙庭飞亡妻刚刚去世,又是忙于和大雍作战,所以双方商定暂不成婚。

这红衣少女出身如此,平日里虽然娇生惯养,可是却不是什么都不懂的千金小姐,若说是红霞郡主的身份,这代州一带凡是见到她的胭脂马和红衣,没有不认得的,可是这个少年明明不是本地人,却是一见就知道自己的身份,不由令她生出疑心。

疑心一起,她的语气中多了几分冷峻, 道:“你究竟是什么人,若是不实话实说,休怪我刀下无情。”一边说着一边握住了刀柄。

那少年一惊,连忙拱手道:“郡主休要动怒,草民姓王,单名骥,并非是奸细。”

林彤容色稍为缓和,上下打量了少年片刻,问道:“看你不像北汉人,你快把出身来历给本郡主说个清楚。”

少年苦笑了一下,道:“郡主,草民乃是南楚人士,后来流落四方,前年草民辗转到了北地,因为擅于医治马匹牲畜,所以多在蛮地行走,前些日子听说代州今秋设立榷场,所以特地来代州,想要看看榷场繁华,不料遇到了郡主,因为草民早已听说郡主这匹胭脂马乃是代州有名的宝马,所以认出了郡主,草民所说都是实情,还请郡主明鉴。”

林彤惊讶的看了少年半天,道:“王骥,你不会就是蛮人中口耳相传的‘伯乐神医’吧,听说你不仅善于医治牲畜,还善于相马?”

少年笑道:“不敢当郡主谬赞,草民在蛮地确实有些小小名气,想不到郡主也听说过。”

林彤道:“那是当然,我代州接近蛮地,每时每日都不敢轻忽蛮地的动静,可惜蛮地地广人稀,各部落逐水草而居,各种情报总是不够详尽。我原本以为人们所说的‘伯乐神医’一定是一个德高望重的老先生,想不到却是,却是这样年轻,王骥,本郡主问你,你本是南楚人,是从什么地方学来的兽医和相马本事,怎么又会跑到蛮地去呢?”

少年又是苦笑道:“郡主,总不能就这样说吧,堵着道路也不是办法吧?”

林彤惊觉道路上已经有了来往行人,便道:“本郡主还要去代州去逛榷场呢,你就跟我一起走吧,路上慢慢讲给我听,可不许你逃走,否则本郡主就要让爹爹出动大军追捕你。”

少年笑道:“草民不敢,郡主请。”

两人策马向代州兴去,因为某种莫名的缘故,两人都没有放马飞奔,只是缓辔而行,一边走一边继续说话。

林彤问道:“王骥,你还没有说跟谁学的本事呢?”虽然是同样的问题,可是目光中却是少了几分怀疑,多了几分好奇。

少年似乎是陷入了深思之中,直到林彤再次追问的时候才惊醒过来,微笑道:“说起草民的师父,乃是天下罕见的奇人,他一身所学神妙莫测,草民原本是他老人家身边一个侍童,不过是有幸学了些皮毛之术罢了。几年前,他老人家遣散了许多下人,草民也是其中之一。虽然草民也得了不少馈赠的金银,可是总不能坐吃山空,想来想去,草民也没有别的本事,只能靠着当兽医来谋生了。可是这兽医行当若是在南楚和大雍,也不过是能够混口饭吃,草民性子最不甘心落在人后,我想来想去,人生一世,总不能庸庸碌碌,所以就到了蛮地,那里牲畜最多,而且各种疑难杂症也是最多,我若是在那里出了名,自然是名扬天下,将来就用不着担心生计了。总算草民运气不错,行医数年没有犯什么差错,蛮人虽然骁勇嗜杀,可是对兽医却是最为敬重,所以草民在蛮地倒也是消遥自在,至于伯乐之说,乃是草民认出了几匹罕见的骏马罢了,消息以讹传讹,结果到了郡主耳中,不免有些夸张了。”

林彤眼睛转了转,问道:“看样子,你年纪虽轻,却是走过了很多地方,本郡主有些事情想问问你?”

王骥在马上躬身道:“请郡主垂询,草民如果知道,一定不敢隐瞒。”

林彤说道:“你连本郡主都认得,那么你一定听说过我姐姐嘉平公主林碧了?”

王骥点头道:“草民自然听过,嘉平公主乃是女中豪杰,率军屡次击退进犯的蛮人,北汉上下谁不知道公主的英名,听说公主已经许配给龙将军,正是一对绝世佳偶,天下谁不欣羡。”

林彤得意地道:“是啊,我姐夫乃是大大的英雄,天下也就是他能够配得上我姐姐。不过,我总是听说别人将一个什么长乐公主和姐姐并列,难道天下还有可以和我姐姐相比的女子么?我可是不相信,可是总是没有人肯告诉我这个长乐公主的事情,你不会说你也不清楚吧?”

看着林彤圆睁的杏眼,王骥不由噗哧一声笑了出来,直到看到林彤神色越来越气恼才止住笑容道:“也难怪他们不肯和你说,这位长乐公主的经历坎坷,而且又是大雍人,所以他们不肯说给你听。”

林彤兴奋地道:“那么你是知道的了,快讲给我听。”

王骥想了一想,道:“这位长乐公主的封号实际上是宁国长乐公主,他是大雍太上皇李援的长女,她的生母原是贵妃,三年前晋位皇后,如今已经是太后娘娘。这位公主殿下性子贤淑贞静,十六岁就下嫁给我南楚太子为妃,后来太子即位,公主成了南楚国母,若论身份自然是尊贵无比了。”

林彤疑惑地问道:“就是这个缘故,别人把她和姐姐并列么?”

王骥摇头笑道:“自然不是,这位公主虽然身份尊贵,可惜大雍和南楚乃是敌对,虽然南楚没有人敢对她不好,可是这位公主心里只怕没有一刻欢喜,常年隐居深宫,后来,南楚显德二十二年,那一年国主改元至化,不过这个年号已经被废弃了,显德二十三年,也就是贵国的荣盛十九年,当时的雍王李贽,带兵攻破了南楚都城,把长乐公主接回了大雍。”

林彤神色一喜道:“这样才好么,公主在南楚又不欢喜,虽然我很讨厌大雍人,可是大雍的皇帝这件事还是做的不错的。”

王骥一笑,道:“公主回大雍不久,国主就被放回了南楚,可是在路上就死了,所以大雍的皇帝要给公主另外找一个驸马。当时皇帝看中了三个人选,一个是大雍的抚远大将军秦彝的儿子秦青,一个是大雍丞相韦观的儿子韦膺,还有一个是大雍御前侍卫副总管夏侯沅峰,这三个人一个是武将,一个是文官,夏侯沅峰又是文武双全,貌如子都,素有大雍第一美男子之称,按理说,不管公主殿下眼界如何高,也应该有一个中意的了。”

林彤兴冲冲地问道:“那么公主看中了哪一个呢?”

王骥摇头道:“公主一个也不中意。”

林彤惊讶地道:“她都不中意,莫非是我姐夫那样的人她才看得中么?”

王骥笑道:“龙将军那样的人天下能有几个呢,草民也不知道公主殿下是否中意龙将军那样的英雄豪杰,可是最后大雍皇帝发了话,只要是公主殿下看中的人,无论是什么身份,都可以做驸马?”

林彤好奇地问道:“最后长乐公主选中了什么人呢?”

王骥叹了一口气道:“这驸马岂是可以随便选的,不论是哪一朝哪一代,所谓公主殿下,千金之尊,这幸福二字却是最难得的,不是嫁给功臣之家做了笼络臣子的工具,就是做了和亲的牺牲品。长乐公主和亲南楚,就是这样的牺牲品。虽然她侥幸归家,可是雍帝给她安排几个驸马人选,也都是名门子弟,说是让公主殿下随便选驸马,只怕长乐公主真是有了意中人,不是给雍帝杀了,就是给落选的几位公子暗害了。而且尚主一事虽说是荣耀无比,可是对于真的英雄豪杰来说,可能却会觉得温柔乡里英雄冢,不愿屈就呢。所以最后长乐公主就是一言不发,咬定牙关不肯选婿。后来雍帝下旨将公主许配给韦膺,可是长乐公主宁愿出家也不愿下嫁,最后皇帝也只能任由她守节不嫁了。当时有人传言,可能是长乐公主感激南楚国主恩情深重,是要为国主守节呢。”

林彤这次没有说话,可是目光中流露出不以为然。

王骥心知这是因为北汉地处边陲,青壮男子容易战死,所以为了维持人口,并不鼓励寡妇守节的缘故。他也不说破,继续道:“后来人们才猜测这位公主殿下目光如炬,一眼就可以看穿人的忠奸和前程,所以才不愿从那几个青年才俊选驸马。”

林彤忍不住问道:“这怎么说呢?”

王骥笑道:“郡主想必是不记得了,荣盛二十一年,也就是大雍武威二十五年,大雍曾经内乱,当时的太子李安犯上谋逆,后来被迫自尽了。”

林彤道:“我记得的,那一年姐姐跟姐夫订婚了,可是姐夫忙着出兵攻打大雍,婚事就耽搁到现在了。”

王骥道:“那位相国公子韦膺,参与谋反,后来跟着凤仪门逃走了,如今已经不知身在何处了,却连累了他的父亲自尽谢罪,如果不是大雍的皇帝顾念他父亲的功劳,只怕连九族都会遭殃。那位秦青秦将军却娶错了人,他的妻子靖江郡主李寒幽乃是叛逆,刺杀长乐公主未遂,这位秦将军却被妻子给杀了,而且据说凤仪门成功的逼宫谋反,也是因为这位秦将军上了妻子的当的缘故。”

王骥停顿了一下,拿起马鞍旁边的水袋喝起水来,林彤趁机问道:“那么那个叫夏侯沅峰的呢?”

王骥想了一下,道:“怎么说呢,这人如今已经成了新君跟前的新贵,虽然还是副总管的身份,可是外面流传雍王这几年在内廷设立了一个‘明鉴司’,这夏侯沅峰就是掌管明鉴司的人,草民不知道这明鉴司到底作些什么,不过只是听说大雍的文臣武将若是听到明鉴司的名头,多半都是要皱眉头的,想来这个夏侯沅峰无论如何也算不上好驸马的人选吧?”

林彤听到这里问道:“原来如此,这位长乐公主果然聪慧,可是若是凭这些就可以和我姐姐比肩,我可不服。”

王骥正要答话,突然神色一变,道:“郡主!后面有——”

林彤一惊,不由向后望去,只见就在数丈之外,一个身穿翠色骑装,身披一件绣着织锦凤凰的黄色大氅的女骑士正在微微含笑地望着自己,那女骑士大概二十多岁的年纪,相貌和林彤有七成相似,可是长眉入鬓,凤目含威,雍容华贵,气度风华却是远在林彤之上。女骑士身后二十丈外,有四男四女八名骑士,都是纹丝不动地策马立在那里。

那位女骑士见到林彤已经发觉到自己,便笑道:“彤儿,你又偷跑出来了。”

林彤惊叫了一声道:“姐姐。”便飞身从马上跃起,扑向那翠衣女子的怀中。那女骑士一伸手,恰好握住了林彤的纤手,林彤借力转了一个身,落到翠衣女子鞍上,轻轻巧巧地坐在女子怀中,脸上露出灿烂的笑容道:“姐姐,彤儿只是想去看看热闹罢了。”

翠衣女子微微一笑,凤目中闪过一丝溺爱,然后目光落到了王骥身上。

王骥心中一惊,连忙翻身下马,拜倒道:“草民王骥,叩见公主殿下。”

那翠衣女子,嘉平公主林碧伸手虚扶,和气地道:“免礼,想必是彤儿向你打听长乐公主的事情吧?本宫听你言辞风雅,如数家珍,想来定然是深知其中内情的了。”

林彤拉着姐姐的手臂道:“姐姐,他就是蛮人盛传的伯乐神医,是我缠着他问东问西的,方才他还救了我的性命,姐姐不可错怪他。”

听到林彤的说话,林碧眼中的神色温和了许多,可是却又带了几分疑虑,她在马上轻轻躬身道:“原来是王神医,听说王神医擅于医治马匹,本宫久闻盛名,那两个漏网之鱼已经给本宫擒获,他们原本是想趁着榷场人来人往的时候,截杀将领的奸细,多谢王先生相救舍妹。”

王骥恭敬地道:“草民不敢当殿下相谢,举手之劳,不足挂齿,若是没有别的事情,请容许草民告辞。”

林彤一听,有些焦急的扯扯姐姐的袖子。林碧不动声色地道:“王先生,方才听你说起长乐公主的事迹,本宫也很感兴趣,你不妨继续说下去,也让本宫听听。”

王骥苦笑一下,林碧何等身份,只怕长乐公主的轶事她是耳熟能详,不过她既然这样说,自己又能如何呢?当下只好纵身上马,一行人继续缓缓向代州行去。

\chapter{第二章 闲话秘史}

林远霆,父祖世镇代州,东晋贞渊十三年,远霆为代州刺史,仍尊晋室,时太原令刘胜割据晋中,建国称汉,以书招之降,远霆素忠义,不屈,胜率兵击之,奈代州军悍勇,不得胜。贞渊十四年,高祖援废黜晋帝,立国称“雍”,晋亡,远霆闻之,望长安遥祭,悲恸欲绝,乃归降北汉先主刘胜。胜感其忠义,慑其武勇,妻以爱女。远霆自尚主之后,克尽职守,抵御蛮人侵掠,数十年如一日,代州军民皆服膺。

嘉平公主碧,远霆长女,为北汉后主刘佑收为义女,军略过人,远胜兄弟,素为远霆钟爱,代州军民谓之“公主将军”而不名。

——《雍史·嘉平公主列传》

林彤见王骥眉头轻皱,坐在马上神不守舍,便高声道:“喂,别发呆了,我问你,你还没有说长乐公主后来怎样了呢?”

王骥身躯微震了一下,拱手道:“启禀郡主,这还是要从大雍那场内乱说起,当时太子勾结凤仪门将猎宫围住,雍王冲出重围,却无法和大军会合,这紧要关头,长乐公主挺身而出,说服了叛党暂时不要进攻皇帝的寝宫,在双方相持的时候,这位公主取得了皇帝的密旨和大将军的信物,然后想方设法传了出去,调动了勤王之军,救了所有人,还将叛逆歼灭,所以皇帝才给这位公主加了‘宁国’两字的封号。”

林彤怀疑地道:“真的吗,叛军作乱一定是将猎宫围得水泄不通,长乐公主又不是我姐姐,可以单人独骑杀出重围,怎么可能把密旨传出去,你不是骗人的吧?”

这时候王骥已经平静下来,恢复了方才的意态从容,笑道:“公主殿下威名远播,小人是十分佩服的。这位长乐公主却是一个手无缚鸡之力的弱女子,怎么可能杀出重围,这就要说到另外一个人了,这个人姓江名哲,字随云,乃是雍王麾下的心腹谋士,正是这人一手策划,不仅传出了密旨调动军队,还逼杀了天下三大宗师之首的凤仪门主。”

林彤眼睛一亮,道:“是啊,我听人说过凤仪门主已经死掉了,所以国师才下令门下可以随便进入大雍呢?不过国师那么厉害可怕的人,那凤仪门主不管现在有没有国师厉害,从前总是三大宗师之首,她真的死了么,那个江哲武功很高么,竟可以逼杀凤仪门主?他和长乐公主又有什么相干?”

她这一连串的问题让王骥苦笑道:“郡主,草民应该先答那个问题呢?”

林彤讪笑了一下,道:“你慢慢说吧?”

王骥道:“郡主,江哲此人也是一个文弱之人,小人听说他身子极弱,经常在生死线上徘徊,不过此人的谋略胆识却是天下无双。”

林彤嗤笑道:“这我可不信,一个文弱书生能有多大胆识,本郡主见过不少书生秀才,只要一见刀枪,不是吓得半死,就是屈膝投降,再说,这人就是再厉害,难道还厉害过我姐夫么?”

王骥为难的看了一眼林碧,林碧微微一笑道:“你不用忌讳,本宫也想听听外人的看法。”

王骥拱手表示致歉,这才道:“郡主,这可是不能比的,文人有文人的风骨,武人有武人的勇气,龙将军乃是三军统帅,又是当世数一数二的名将,这谋略胆识自然是过人的,可是若是设身处地,恐怕龙将军也做不出江哲江大人所做的事情了。”

林彤瞪大了眼睛道:“那我就认真听你讲,若是你言过其实,我可要责罚你。“

王骥微微一笑,道:“这位江大人本是我南楚的臣子,显德十六年,他年方弱冠,便一举成名,考中了状元,若论天下文章锦绣,没有胜过南楚的,这位江大人中了状元,可以说文章诗词冠甲天下了,而且当世之间,也无人能比得上他的绝世才华。”

林彤撇撇嘴道:“你欺负我不爱读书么?姐姐,你说,他说的是不是真的?”

林碧凤眼有些迷离地道:“他说的不错,这位江状元若论文才,确实是首屈一指,你继续说下去。”

王骥轻轻道:“若说他的诗词有多好,草民无学,也不是很清楚,可是草民最喜欢他早年的一首小词。”说罢,清了清嗓子,他朗声唱道:“一棹春风一叶舟,一纶茧缕一轻钩。花满渚,酒满瓯,万顷波中得自由。”他唱得十分动情,声音又是十分悠扬清越,众人彷佛也身在那碧波连天的江湖之上一般。

唱完一曲,王骥接着说道:“这位江大人做了几年翰林,得到我南楚贤王德亲王赵珏赏识,随军出征蜀中,草民也不清楚江大人献了什么计谋,可是有一件事情倒是脍炙人口,蜀国灭亡之后,雍王李贽要把蜀王押回大雍,这样一来,虽然是大雍和南楚平分蜀国疆土,可是大雍控制了蜀王,那就占了莫大的便宜了。当时大雍正是如日中天,南楚虽有千军万马也没有办法扭转这个局面。就是这位江大人,在酒宴之后一曲高歌,迫得蜀王自尽身亡,从那以后,朝野上下都称颂江大人是南楚第一才子。”

林彤不信地道:“我才不信,一首诗词就可以逼死一个国主,姐姐,他说的是真的么?”

林碧伸手抚摸这幼妹的秀发,道:“人人都有羞恶之心,那蜀王国破家亡,身陷缧绁,又被人当众讥讽,也难怪他要自尽身亡。”

林彤似懂非懂的点点头,道:“姐姐说是真的,就是真的,不过这个江哲可是真狠毒。”

王骥笑道:“或许是吧,可是江大人从蜀中回来就生了重病,将近两年都没有上朝,只是在家里养病,我想江大人心中并不会因为那些事情得意的。”

林彤又问道:“啊,我想起来了,荣盛十九年雍王攻破建业,这位江大人既然后来是雍王的心腹,想必是那个时候投降了雍王,他写诗讥讽蜀王投降,可是自己又屈膝投降,看来真是骨头不硬,这就是你说的文人风骨么?”

王骥神色一黯,道:“郡主这样说,草民也没有什么法子辩驳,可是在草民看来也不觉得江大人有什么不对。在显德二十二年,国主看不清形势,一定要晋帝位,江大人上书直谏,气得国主要将他斩首,可是总算是顾念江大人的才名和功劳,只是将他贬为庶民罢了。雍王入楚的时候,亲自上门礼聘,可是江大人坚持不肯投降,后来江大人是被雍王殿下强行掳回大雍的。草民听说雍王对江大人十分器重,用尽了法子劝降,草民想江大人最后投降或者是因为雍王心意太诚挚了吧?”

林彤不依不饶地道:“虽然说是贤臣择主而侍,可是我还是觉得谈不上什么风骨。”

王骥摇头一笑道:“郡主说的是。”虽然这样说着,可是神色间明显的有些敷衍,林彤正要继续进逼,林碧出言道:“彤儿,你不想继续听了么?”

林彤这才闭嘴不言,她最崇敬的就是姐姐和姐夫,所以对王骥认为江哲强过姐夫十分不开心。

王骥接着说道:“江大人到底为雍王谋划了什么,草民也是不知道的,可是雍王殿下对江大人十分尊敬爱护,如师如友,如兄如弟,江大人到大雍不久,故德亲王的贴身侍卫潜入大雍刺杀江大人,据说是德亲王临终曾经留下密令,如果江大人投了别国,就要杀了江大人。听说江大人受了重伤,险死还生,若非是雍王殿下用尽了种种名贵药物为他续命,根本等不到医圣桑先生救治了。雍王殿下因为这件事情大为震怒,从那以后,据说江大人身边的防卫要比雍王还要严密。”

林彤惊道:“你们的德亲王怎么这样无情,江哲虽然失节,可是毕竟情有可原,再说他人都死了,干么还操这个心呢?”

王骥叹息道:“当时很多人也都这么想,无论如何江大人总是为南楚立过功的,虽然他改事大雍,可是也是南楚先免了他的官职的,那个侍卫也未免太过固执了,再说江大人不过是一个文士,德亲王死前还记挂着他,也未免多事。可是后来那个侍卫逃回南楚的途中被江大人身边的一个仆人追杀,击杀在大江之上,这个仆人名叫李顺,原本是南楚宫中的一个宦官,不知怎么跟着江大人做了奴仆,原本没有人将他放在心上,可是这个李顺竟然有本事杀了这个侍卫,人们才知道他竟然是个罕见的高手,这么一个高手竟然愿意做江大人的奴才,这才有人想到,或者江大人是有几分本事的,不过大多人还是没有将江大人看在眼里。尤其是江大人遇刺之后,身体极弱,一年倒有半年在病榻之上,郡主应该知道,一个人若是身体不好,就是有十分本事也恐怕只能施展上两三分的。”

林彤听得入神,忍不住问道:“那江大人身体那么差,又是怎么给雍王出谋划策的,又是怎么逼杀凤仪门主的呢?”

王骥叹了一口气道:“草民也不知江大人如何为雍王出谋划策,可是听人说江大人帮助雍王平叛之后,已经是形消骨立,据说两鬓如霜,时常呕血不止,有人说江大人只怕是活不了多久了。”

林碧听到这里轻叹道:“鞠躬尽瘁,死而后已,可惜此人不在我北汉。你继续讲吧。”

王骥这才又接着说道:“当时江哲也在猎宫,而且身患重病,就是他即时识破了太子的阴谋,才没有让雍王枉死在小人之手。雍王突围之时就是谁都不带也要带上他的,可是江大人却主动留了下来,而收藏江大人的就是长乐公主。”

林彤眼珠一转,道:“长乐公主为什么会收留他呢,莫非他们有私情么?”

王骥犹豫了一下,道:“这个草民也说不好,长乐公主曾是南楚国母,江大人曾是南楚臣子,长乐公主常年居于深宫,江大人难得入朝,按理说两人是不可能有私情的,后来有人说,长乐公主入楚之后,最爱的就是诗词文章,江大人诗词冠绝天下,长乐公主最爱江大人的诗词,恐怕是因此对江大人心存爱慕。可是尊卑有别,君臣名份不能逾越,所以长乐公主才不肯向雍帝进言招江大人为驸马。可是猎宫事变之时,江大人前去求救,长乐公主自然是无论如何也要救他的,后来江大人运筹帷幄,由公主向雍帝取来了密旨信物,然后江大人派人将密旨送了出去,这次请来了勤王之兵。”

林彤好奇地问道:“江大人去向公主求救,莫非江大人也知道公主喜欢他么?”

王骥笑道:“这个草民也不清楚,其实无论如何,江大人也只能去求长乐公主帮忙的,长乐公主向来是中立的,若是大雍太子和雍王争斗起来,公主或许不会插手,可是涉及到皇帝,公主殿下父女情深,自然是不会坐视太子威逼父亲的。”

林彤又问道:“那么江大人就没有被搜到么,他又是怎么将密旨送出去的呢?”

王骥神色变得崇敬,道:“江大人躲到大概是很严密,再说可能是那些叛党也没留意这个文弱书生吧。至于传诏之人,这可是最令人意想不到的地方。夏侯沅峰本来是太子一党的人,也参与了谋反,据说他和太子少傅鲁敬忠十分亲近,所以万万想不到,就是这人借着替太子送伪诏的机会把真的密诏带了出去。夏侯沅峰本来不是雍王的人,这是人人都知道的,谁也不明白为什么这人会被江大人说服。他这一弃暗投明,可是立下了天大的功劳,如今更是深得宠信。可是江大人的本事才真的令人佩服,这样不可能的事情竟被他做到了。”

林彤点头道:“原来这样啊,那你快说江哲逼杀凤仪门主的事情,我还是不信他有那样的本事。”

王骥神色一振,道:“说起这件事,可是真的令人心服口服,江大人不过是一个文弱书生,只怕是凤仪门主一根手指就可以杀死他。当日凤仪门谋反失败,所有叛逆都被围了起来,眼看就要一网打尽,谁知道凤仪门主从天而降,单人独剑,在大殿之上一站,金殿之上除了皇帝亲王,就是重臣名将、江湖高手,可是在三大宗师之首,一个女子面前,竟然尽皆俯首,无人敢正眼相看,可是江大人一介文弱书生,又是奄奄一息,吐血将死之人,竟然声如金石,铿锵有力,宁为玉碎不肯瓦全,迫得凤仪门主只得同意自己留下做人质,换取弟子们的性命,这种气魄何人能及?”

林彤想要说话,可是想到自己有幸拜见国师的时候也是大气不敢喘,这样想来,江哲敢在凤仪门主面前不畏生死,直叱其非,果然是风骨嶙峋,便没有开口说话。

王骥又道:“接下来的事情知者不多,可是凤仪门主就在七日之日被少林寺的慈真大师带着门中高手和邪影李顺围杀,一代宗师,含恨而逝。”

林彤问道:“那么怎么说是江哲逼杀了凤仪门主呢?”

王骥道:“这个消息却是从少林寺传出来的,据说凤仪门主当初本就是受了伤的,她服了医圣桑先生的九转护心丹,暂时护住了心脉,可是江大人乃是医圣的弟子,精通医术,用了什么法子让凤仪门主在七日之中耗尽了生机,所以凤仪门主最后被迫得只能拼死一战,慈真大师也是宗师身份,邪影李顺也是绝顶高手,少林寺的十八金刚联手结阵,凤仪门主怎能不死呢,而且听说最后就是邪影李顺趁着两大宗师决斗之际,偷袭重伤了凤仪门主,才让一代宗师被迫*身死的。若是凤仪门主心中没有死志,只怕早就鸿飞冥冥,不知所终了。”

林彤神色十分震惊,半晌才道:“那这位江大人可真是厉害,不过那慈真大师也太没有宗师风度,联手夹攻,还要让人偷袭暗算,不过这样的事情他怎会传扬出来,多丢人啊?”

王骥摇头道:“草民听了只当是笑话传奇,可不明白慈真大师的心思。”

林彤抬头看向林碧,撒娇道:“姐姐你一定知道。快告诉我啊。”

林碧被她纠缠不过,只得笑道:“这有什么奇怪,那位江大人心机这样深沉,慈真大师将这件事情传了出去,自然是人人戒惧,到时候自然会对这位江大人多了几分提防,想来是慈真大师有些兔死狐悲吧?”

林彤似懂非懂地点点头道:“噢,那么王骥,江哲和长乐公主又怎么样了呢?”

王骥又道:“凤仪门主显身之时,金殿之上虽然皆是英杰,可是却尽皆低首,只有两人始终无畏生死,令人钦佩,一个是江哲江大人,他以文弱之身,直叱凤仪门主,令群英汗颜,另外一人就是长乐公主,当时江大人被凤仪门主内力所伤,吐血不止,长乐公主不顾凤仪门主剑锋所指,亲探江大人伤势,情之所衷,无视生死,让人怎不为之感叹。”

林彤“啊”了一声,道:“莫非长乐公主嫁给了江哲么,那也难怪旁人将长乐公主和我姐姐并列,那江哲果然可以和我姐夫相比。”

王骥都是微微一笑,知道这小郡主如此说法,就是承认江哲确实了得了。他也不说破,继续说道:“虽然雍王曾经请求皇上赐婚,群臣也被他们的深情感动,虽然觉得有违礼法,可是也没有人劝阻,可是雍帝却是不肯。”

林彤惊讶地道:“为什么,江哲立下这样的大功,他和公主又是两情相悦,为什么雍帝不答应呢?”

王骥笑道:“理由是因为江大人病体沉重,雍帝很担心若是江大人寿元不久,长乐公主本就一生坎坷,若是驸马早亡,岂不是雪上加霜,这个理由一说出来,就是雍王也不敢说不对的。”

林彤点点头道:“原来是这样啊,也对,那么是不是后来江大人身子养好了,大雍的皇上就为他们赐婚了呢?”

王骥笑道:“若是这样,也就谈不上传奇了,那位江大人立下这样的大功,眼看就要飞黄腾达,可是他却在雍王的立储大典之后就带着邪影李顺悄然远离了,据说这位江大人来去明白,将雍王的一切赏赐都封存起来,一介不取,就这样飘然远遁江湖了,他这样的才华功绩,却是丝毫不爱富贵权势,就是有人从前觉得他名节有亏,如今也不能不击节而叹。”

林彤眼中闪过一丝崇敬,道:“那这位江大人的人品才华可真是天下无双,不过他虽然厉害,你本来不是要告诉我为什么长乐公主可以和我姐姐齐名么,怎么跑题了呢?”

王骥心道,我就是真的跑题了,不也是被你引得么,面上却笑道:“郡主有所不知,江哲虽然是飘然远走,一介不取,可是却拐走了一个人。”

林彤瞪大了眼睛,道:“莫非,莫非,长乐公主竟然和她私奔了么?”

王骥拊掌道:“正是如此,长乐公主性子本就是外柔内刚,当初雍帝逼她另嫁,她就誓死不从,如今雍帝不许她嫁给江大人,可是江大人这样离去,叫公主怎能放心呢,这两人都是为大雍耗尽了青春心血的人,也就没有什么顾忌了,就双双远走天涯了,从此四海逍遥,做一对神仙眷侣,只羡鸳鸯不羡仙。这位公主殿下,本来已经被晋封宁国长乐公主,荣耀无比,母妃又晋位皇后,本是富贵已极,却是抛却一切,陪着爱侣隐遁江湖,这样的奇女子,应该可以勉强和嘉平公主殿下相比了。”一边说,一边瞧向林碧,眼中满是谨慎。

林碧摇头道:“宁国长乐公主忠孝两全,品貌过人,又是这样至情至性,不爱权势富贵,本宫怎比得上她呢。彤儿,你从前年纪小,爹娘担心你不懂得其中真谛,知道了反而不好,今日王先生讲给你听了,我看你倒还明白道理,也就不阻你了。”

说罢,林碧的目光落到王骥身上,意味深长地道:“王先生,你年纪轻轻,倒是见识广博,真是难得啊。”

众人的目光都落到王骥身上,都带了几分疑惑和提防。

\chapter{第三章 龙飞在天}

龙庭飞,出身北汉世家,惊才绝艳,军略武勇举世无双,号为无敌,后主托以军国大事,从无疑忌,庭飞亦以忠义报之。

——《北汉史·龙庭飞传》

王骥神色从容地道:“草民流浪四方,见识广博虽然谈不上,可是各种消息都知道一些,虽然大雍朝廷宣称公主因为猎宫受惊所以隐居休养,可是这民间早就流言纷纷,绘声绘色,不厌其详,事发之时草民正在大雍,听了不少传言,所以知道的较为详细一些,草民并非是奸细,还请公主明鉴。”

林碧目光一闪道:“王先生过虑了,先生精通相马医马,又是见识广博,正是我北汉渴求的人才,如果先生请屈尊,本宫必定尊先生为上宾。”

王骥犹豫了一下道:“公主如此看重,草民本应该从命,只是草民正要到东海蓬莱一行,只怕不能奉命。”

林碧微微一愣道:“你要去东海作甚?”

王骥恭谨地道:“草民近日见到故友,说是恩主身体康健,又有弄璋之喜,所以意欲前往庆贺。”

林彤好奇地道:“你的恩主不是老人家了么,怎么又会有儿子呢?”

王骥一怔,笑道:“郡主想必是误会了,草民的恩主尚在壮年,膝下除了一位小姐再无所出,近日才添了一位公子,下月乃是公子周岁,草民闻知此事,意欲前往祝贺。”

林碧神色一动道,这个王骥虽然年纪轻轻,可是言辞气度都十分不凡,他的恩主也必然也不是寻常人,而且自己不是就要到东海一行么,若是有机会见到他的恩主,说不定我北汉又多一位栋梁之材。想到这里,她开言笑道:“这倒巧了,本宫近日也要到东海一行,王先生可愿意和本宫同行?”

王骥一愣,问道:“公主乃是北汉重臣,怎会去东海一行,要知道虽然东海乃是东海侯的势力范围,虽然东海侯仍然独树一帜,可是天下人谁不知道近年来东海侯和大雍已经开始和解,颇有往来,公主若是要去东海,只怕是凶险不少。”

林碧笑道:“不妨事,先生想必还不知道,东海侯爱子成婚,喜贴已经遍洒天下,本宫主乃是奉王命前去祝贺的,而且同期将要举办的奇珍会也是一大盛事,本宫也想看看异国的奇珍。”

林彤一听满面欣喜,焦急地道:“姐姐,奇珍会么,我也要去看看。”

林碧微微一笑,伸手安抚地拍拍妹子,不让她插话。

王骥拊掌道:“啊,奇珍会,草民也听说过,近两年来,东海有富商海无涯,造大船往来高丽、倭国和南洋诸地,以中原江南所产瓷器、丝绸换取金银珠宝和各种特产,据说获利千百倍以上,想必这奇珍会就是海无涯举办的吧?”

林碧笑道:“正是如此,海无涯趁着东海侯爱子大婚之际举办奇珍会,而且现在天下谁不想和海无涯合作,独占远洋贸易的利润,这个机会自然是最适合的。”

王骥疑惑地道:“可是谁不知道这海无涯背后必定是东海侯支持,想要分享这里边的利润,只怕没有那么容易吧?而且各国都有巨商和海无涯合作,这才成了一个平衡的局面,若是公主想要独占利润,只怕大雍和南楚都是不肯的。”

林碧深深的看了王骥一眼,道:“王先生果然通达世事,这海无涯在东海立业至今已有五年,前两年不过是经营海运也还罢了,这两三年来组织了三次远洋商队,其中他自己新造的几十艘大船不仅载货多,航速快,而且配有种种新式的武器,随行的商船近百,还有东海侯的战船护航,若说是无能人支持策划,本宫可是绝对不信。不说别的,南楚、大雍和我北汉都有巨商和他合作,只因是三家得益,所以无人和他为难,这种心机胆识本宫就是十分佩服,可是有一利也有一弊,海无涯这般四面讨好,虽然如今还可以奏效,可是近几年来,战火益发肆虐,大雍、北汉绝没有共存的可能了,所以海无涯也要重新估量一下,与其中立,不如选一个主子的好。”

王骥听得心寒,这种机密事情给自己听了,只怕自己是绝对不能脱身离去了,他眼睛余光瞧去,只见那些男女侍卫骑士,都是手按刀柄,也只能当作没有看见,笑道:“公主说得正是,可是草民说一句不恭敬的话,海无涯可以和其他人不合作,可是东海侯确实抛不下的,没有东海侯的护航,远航的商船是绝对不可能平安无事的,若论物产丰富,大雍和南楚原本就占据了富庶之地,蜀国灭亡之后,蜀中的物产也被两国瓜分,海无涯若是只选一方合作,无论是选了大雍还是南楚都不意外,可是北汉未免少了几分优势,若是三方制衡,北汉倒还可以分一杯羹,若是想要独占远洋贸易的利润,只怕是得不偿失。”

林碧眼中寒芒一闪,道:“先生说得不错,本宫也是这么想,我北汉在远洋贸易中原本就处于劣势,先生也是误解了本宫的意思,想要独占远洋贸易利益的不是我北汉,而是南楚。进来南楚派来使臣,他们想要和北汉合作,威逼东海侯达成协议,三方联手控制远洋贸易,将大雍排除在外,朝廷派本宫前去东海,就是想趁机取利。若是东海侯同意此事,到时候就可以切断大雍在海外的收益,对我们自然是有益无害,若是不能,也要尽可能夺取更多的份额。”

王骥听得连连点头,道:“草民乃是南楚人,虽然去国多年,却知道我国最重商业,有这种打算也是理所当然的,却不知我国这次是谁负责此事?”

林碧微微一笑道:“本宫听说这次前去祝贺东海侯独子大婚的南楚使臣乃是南楚重臣镇远公陆信之子,大将军陆灿,大雍的使臣乃是庆王李康和礼部侍郎苟廉,双方都是不遗余力,所以我北汉的使臣也不能随便派一个人去,本宫就只好勉为其难了。王先生,你既然也是去东海,何妨和本宫同行,说不定本宫还有倚重之处。”

王骥恭敬地道:“草民有幸附诸骥尾,怎敢不从。”

林彤急忙道:“姐姐,我也要去。”林碧看去,只见幼妹眼中满是企求,又是滴溜溜转个不停,分明是下了决心若是自己不许同行,就要私自前去的主意。“宠溺的一笑道:“好吧,只要你听话,我就带你同去。”林彤大喜,双手合十,发誓赌咒自己决不会胡闹,林碧只是浅笑不语,心道,麻烦自然是少不了的,不过这次也不是什么生死攸关的大事,南楚想得倒是不错,说什么利益均沾,我可不想到时候看你们的脸色。这个王骥倒是一个人才,也不像那些平常南楚人一样好逸恶劳,若是此人没有什么问题,不管用什么法子也得留下来为我北汉效力才行。

王骥心中却是另有所思,是留在北汉作卧底还是回到恩主身边效力呢,他有些难以决定,想来想去,还是等到见到恩主再说,想到昨日看到的书信,字里行间喜气洋洋,想必恩主如今心情很好,这也难怪,红颜知己相伴,又是无拘无束,自在逍遥,如今又是新得贵子,想必恩主如今是不会再想出山了,可是这几年来,大雍和北汉相持不下,不知道恩主的逍遥日子还能过上几天呢?

在代州待了几日,王骥便跟着林碧、林彤二人踏上了前往东海的路程,虽然可以越过五台山从鲁地出海,可是林碧却是绕了一个大圈子,先去了沁州。

沁州乃是龙庭飞大将军所镇守,是北汉重镇,北控太原、南襟潞泽,太原既是北汉国都,又是兵家必争之地,东南方向有天门关、石岭关、赤塘关拒大雍军队于外,最是易守难攻。而沁州则是太原南面的门户,境内大半都是丘陵河流,龙庭飞的大军就是驻扎在此。而这两三年来,每至秋收时分,龙庭飞就提大军从沁州进攻泽州、潞州,甚至曾经入侵镇州,扰乱大雍边境的秋收,而经过武威二十五年的教训之后,齐王李显采取了坚壁清野的做法,派出大军严守关隘,建立保甲制度,在边境广设烽火台,监视北汉军的行动。若是北汉军进攻,便倾巢出动迎敌。谁知过了一段时间之后,龙庭飞改弦易辙,利用北汉军善于长途奔袭,来去如风的特点,使用游骑侵扰大雍边境。大雍明明兵力在北汉之上,却被龙庭飞压制的死死的。最后齐王索性坚守不出,将边境一带的居民全部迁移到防线之内,留下了百里左右的空白地带。这样一来,凭着大雍强大的军力和星罗棋布的堡垒军寨,总算是维持了一个平手的局面。

离沁州还有三十里之遥,王骥就看见远处烟尘滚滚,他仔细看去,只见烟尘凝而不散,就知道来的是一支精锐的骑兵,不过现在还在北汉境内,大雍骑兵现在固步自封,不用想也知道定然是友非敌,想必是沁州城派来迎接嘉平公主的吧。

不过片刻,烟尘中已经可以看清那是一支身穿红色衣甲的骑兵,为首之人也是一身火色战袍,头盔上面的面罩没有掀上去,看不清容貌,可是那矫健的身姿,如火如荼的气势,让人已经心折不已。林碧眼中闪过喜色,纵马上前,她身边那些时刻不离的侍卫却都一反常态的止步不前,就在林碧单骑上前的时候,那支骑兵的领袖,那个红衣将军也越众而出,两人距离三丈之时,同时跃起各自伸出一手,在空中相握,轻轻落在地上,然后火红拥住了翠绿,那种浓厚的深情和久别重逢的喜悦深深的感染了众人,都是默不作声,静静的看着那两人。

过了一会儿,两个身影分开,携手向王骥等人走来,王骥随着众人一起下马,垂手而立。两人走近众人,林碧面上喜气洋洋,分外的娇艳动人。而那个红衣将军也掀起头盔上的面罩,露出英俊得绝无瑕疵的面容,配合他那修长俊伟的八尺雄躯,深邃如同夜空的一双略带碧色的眼睛,威武中带着儒雅气息的雍容风度,让包括王骥在内的所有人都心悦诚服地拜了下去,齐声道:“属下叩见大将军金安。”

当然林彤可不会跪拜,而是高高兴兴的上前抱住了那红衣将军的左臂,兴奋地道:“姐夫,彤儿可是天天都想着你呢,你到底什么时候迎娶姐姐呢?”

王骥虽然早已经猜到那红衣将军正是北汉的擎天玉柱龙庭飞,可是仍然是忍不住心中一阵兴奋,能够见到这样的英雄人物,真是三生有幸,回想起这些年所见过的英雄人物,竟然觉得无人可以胜过这龙庭飞,就是那雍王李贽、齐王李显,虽然多了几分皇室君临天下的气概,但是比起龙庭飞来,却也不免逊色几分。或者能够和这人相比的,只有那个赐予自己一切的恩主吧,想起那个文弱清秀的青年,王骥不由心中一热,那个人就是有着那样的特质,不论在何人面前,你都会忍不住将目光落到他身上。

龙庭飞听了林彤的追问,神色不由有些尴尬,他和林碧订婚两年还没有完婚,却是原因众多,一来是因为发妻身亡不久,龙庭飞不愿这么快就续弦,再说这两三年龙庭飞都在忙着和大雍作战,殚精竭虑,自然是顾不上婚姻之事。更重要的一个原因却是,林碧虽是女子,却是极富军略才能,她的兄弟才能都不如她,这几年来,因为林远霆身体欠佳,代州的军政大权实际上掌握在林碧手中,为了严防北方蛮人趁机攻占北汉,林碧根本是不可能脱下战袍嫁作人妇的,所以这婚事就拖延了下来,只是这个理由却不好明着说出来。龙庭飞目光一转,落到了王骥身上,便笑着上前道:“这位就是伯乐神医王先生了,听碧妹说先生已经到了我北汉军中,龙某真是喜出望外,听说王先生善于相马医马,想必着养马之术也是非常出众的了,若是能够得先生襄助,我军中的战马定能更加精良。”

王骥再度拜倒道:“草民愧不敢当,所谓‘伯乐神医’,不过是虚名罢了,草民虽是南楚之民,却是心爱漠北风烟,若是将军不弃,待骥探亲归来,必定为将军效力,鞠躬尽瘁,死而后已。”

龙庭飞微微一笑,双手将王骥搀起道:“王先生肯为我北汉效力,乃是王上之幸,你多年流浪,想来必定惯于在外行走,此次你随公主往东海,还望你尽心竭力。”

王骥行礼道:“在下遵命,必定不负所托。”

龙庭飞又是淡淡一笑,笑容中满是欣慰喜乐,王骥不由心中一暖,若非是心中早有计较,只怕真会为这人效死命而不悔了。

一行人回到沁州城,林碧乃是公主身份,又是龙庭飞的未婚夫人,龙庭飞虽然为了不将林碧的行踪宣扬出去,只是招了亲信的将领设宴为林碧洗尘,但是将军府内仍然是热闹非凡,酒宴最热闹的时候,龙庭飞和林碧却没了影踪,众人只道这对未婚夫妻多日不久,想必是私下叙话去了,便都挤眉弄眼一番也就罢了。

龙庭飞和林碧果然在一间雅致的书房密谈,不过出乎众人意料的是,他们谈的不是什么儿女情长,而是在商议军机。

回到府内,龙庭飞早已经换上了一件天蓝色的长袍,容光焕发地看着林碧,道:“碧妹,你这次不用过于费心,只要旁观就可以了,如果南楚果然有本事说服东海侯排除大雍的势力,那么我们自然要分一杯羹的,而且还可以趁机大举进攻大雍,若是南楚失败,你就不要插手了,免得遭到池鱼之殃。不过希望你能趁机和东海侯达成盟约,若是能够得到东海侯的合作,大雍就是真的四面受敌了。”

林碧柳眉轻蹙道:“庭飞,可是你很清楚,大雍的兵力是最强的,你虽然占了上风,可是只要不能大败齐王,就不能真的威胁到大雍,你有没有什么打算呢?这样对峙下去对我们没有好处,我们北汉虽然军民骁勇,可是比起大雍的强大和南楚的富庶,实在没有统一天下的筹码。”

龙庭飞站起身来,走到墙边,指着上面挂着的地图道:“碧妹你看,现在齐王隐忍不出,我军虽然纵横无敌,可是大雍实力却没有什么损失,我军虽然勇猛,可是竭尽所能,也只有区区二十万,代州一带虽有十万军马,却是为了抵御蛮族,轻易不能调动的,所以想要取胜是不能这样按部就班下去的。如今大雍之所以可以跟我们对峙,全是因为齐王用兵老练,想来经过种种坎坷,李显已经长进了许多。若是能够铲除了李显,到时候我自信可以纵横大雍没有敌手,只要得到泽州和镇州,大雍就再也不能有力的遏制我国了。”

林碧皱眉道:“若是能够如此最好,若是除掉李显,除非李贽御驾亲征,否则大雍无人可以抵挡你的大军,可是若是李贽亲征,南楚就可以发难进攻,到时候大雍两面作战,形势更加岌岌可危。可是李显乃是皇室亲王,又得李贽信任,恐怕很难铲除他呢?”

龙庭飞笑道:“功高震主,天下有几个主君会不忌惮带兵的大将呢,就是李贽雄才大略,这疑心也是免不了的,更何况李显和李贽还有心病没有消除,当年李安勾结凤仪门逼宫叛变,李显虽然没有亲自参与,可是嫌疑也很深,他的王妃更是自尽身死,据我们得到的情报,当初李显本已被软禁起来,若非是我带兵进攻,李贽无奈之下,才不得已赦免了李显。可是这个李显脾气也未免太古怪了,他做了几件错事,其一就是拒绝了李贽的赐婚,年前李贽本想为李显另选王妃,可是却被李显拒绝了,其二就是原来的齐王妃所生的嫡子已经失去了世子的身份,若是李显聪明的,就应该对这个儿子不闻不问,可是他却把这个儿子带来了军营。这样一来,雍王不免心中有些不满,说起了这个李显也真是固执,这其中的深浅关节他不是不知道,却是总不肯低头服软。他这样自留破绽,我也不会客气,从去年开始我就散布流言,说李显坚守不出,乃是为了拥兵自重,这事若是换了旁人,以李贽的心胸和才识倒还不会太介意,可是若是曾有谋反嫌疑的李显,偏偏李显又是这样不识趣,你说李贽会怎么想。过去几个月,李贽连下了数道圣旨对李显加以抚慰呢。”

林碧想了一想道:“若是别的将军,这样抚慰只会让他感激,可是若是李显,这样的抚慰反而会让他觉得深受怀疑。”

龙庭飞道:“正是如此,李显也连上了数道奏章汇报军情,用以表明心迹,可是这种事情,却是欲辩无从,现在就连长安城内已经满是流言了。想来李贽也很为难,若是不召回李显,只怕流言传下去,李显心中恐惧,就是本无反心,也会生出反心来。”

林碧道:“其实若是李贽能够派一个够分量的监军,也可以稳定军心民心的。”

龙庭飞笑道:“哪里这样容易,这个监军既要有本事压制齐王,又要不能引发齐王麾下将领的愤怒,还要是李贽的心腹,现在大雍上下哪里有这样的人呢?只要这种情况继续下去,不论是为了保全齐王还是为了防止齐王反叛,召回齐王就成了不得已的做法了。到时候大雍方面没有大帅统军,必然有人忍不住进攻,我就可以趁势消灭雍军的有生力量了。”

林碧感叹地道:“希望你能够一举功成,我北汉是承受不住长久的对峙的。”

龙庭飞满怀信心地道:“碧妹放心,我攻占泽州、镇州之后,碧妹,你我的婚事也不该拖延下去了。”

林碧玉颜绯红,说不出话来,龙庭飞走到她面前,握住她的纤手道:“我知道你放心不下代州的事情,没有关系的,这件事情我已经想通,龙庭飞不是目光短浅之辈,只要能够令北汉国运昌盛,你我就是聚少离多又有何妨。”

林碧心中一阵感动,半晌才道:“等你大破雍军之后,就去禀明王上和我父亲吧。”

龙庭飞大喜,伸手将心爱的佳人拥入怀中,烛影摇红,映照着一对璧人的身影,此情此景,怎不令人心醉神迷。

\chapter{第四章 初到滨州}

姜永,父姜无涯,镇徐州,东晋时封为永宁侯,娶高祖姊宁华长公主为正室,大雍立国时,姜无涯与高祖争胜,遇刺重伤,殁于战场。永见敌势大,奉母携旧部远走东海为盗,袭父爵位为永宁侯,然叱咤东海,威震海疆,人乃称其东海侯而不名。高祖履招之降,永不至。

——《雍史·东海侯传》

离开沁州之后,林碧带了百余名侍卫日夜兼程走鲁南,这一带大雍和北汉的势力犬牙交错,所以林碧等人都是改换了装束,化装成客商旅人,一路上有惊无险,不过旬日之间,就到了滨州,滨州位于大雍和北汉的边界上,可是这里实际上却是东海侯姜永的势力范围,东海侯从前纵横海疆的时候,就是通过滨州得到补给的,而滨州的商人为了确保海上商船的安全,更是暗中和东海侯互通消息。尤其是近年来由李贽主持军政之后,大雍和东海之间的仇恨似乎渐渐消解,东海侯不再恶意劫掠大雍的商船,而大雍也不再严厉镇压倾向姜永的势力,所以东海的势力在滨州更是越发强大。尤其是在东海开创了远扬贸易的商道之后,滨州更是成了天下最大的港口之一,北汉和大雍通过滨州源源不断的将本国特产送上远行的商船,换取异国的金银粮食和各种特产。所以不论是北汉还是大雍都想控制住滨州,可是在没有绝对把握之前,却都不敢轻举妄动。

而南楚和东海之间的贸易却是通过杭州进行的,这次南楚想要迫使东海将大雍排斥在外,在北方和北汉合作,在南方和南楚合作,并非是什么好意,若是大雍采取玉石俱焚的手段,那么滨州就别想成为港口,到时候就只剩下南楚独占利润。所以林碧对于南楚的提议并不热衷,当然若是南楚真的成功了,林碧也会尽量想法子控制住滨州,虽然困难,可是也不是不可能的事情。

一进滨州,就感受到那种迎面而来的繁华气息,往来都是南腔北调的商贾,若非是秋风萧瑟,不免让人怀疑到了江南盛地。北汉在滨州名义上是敌国,所以自然没有馆驿,不过早有人为林碧在滨州最富盛名的平安客栈订下了一个独院。

平安客栈,这个名字十分平常普通,可是如今天下所有的平安客栈都是一个主人。在两年前,第一家平安客栈在南楚建业开张,之后很快就在天下各大都邑开设了分店,这平安客栈并非是以豪华见长,事实上这里的布置摆设以简朴清雅著称,客栈之中虽然服务周到,可是却也没有什么十分特殊之处,虽然可以做出天下各大菜系的名菜,可是比起真正的名家风味不免差了几分火候。按理说这样的客栈并没有什么值得重视的地方,可是当平安客栈开了多家分店之后,常常游走四方的商贾惊奇的发觉各处的平安客栈,居然十分相似,客栈的经营方式、房间的格局布置、饮食的口味,几乎是一个模子里面出来的一样。对于这些常年奔波在外的商人旅客来说,到处都有的平安客栈仿佛成了自己的家一样,在这里,他们总是能够得到熟悉的感觉。而且平安客栈还有一样好处,一旦你住进某一家客栈,数月之内,天下所有的平安客栈都会熟知你喜欢的房间,喜欢的食物等等,让你到处都有宾至如归的感觉。

当然未免有些人担心平安客栈会有问题,可是各处的平安客栈最多只有一两个管事真正属于平安客栈,其他的仆役都是从当地雇佣的,只是经过训练之后,这些仆役都按照那些管事带来的写满了各种规矩的小册子行事,若是有所违背,就会被辞退。所以才让各地的平安客栈既基本上相似,又在细节上有一些各自的特色。这种经营方式十分便于各地官府派遣间谍进入探察,可是也让他们很难探察到什么机密。所以至今平安客栈的后台老板仍然是一个秘密。

选了平安客栈居住,林碧自然不是因为这个缘故喜欢这个客栈,而是因为平安客栈还有一桩好处,它的每个房间都和其他房间之间用花木假山回廊之类的隔离开来,拥有隐秘和安全两种特质。如果租下一个院子,那么就更加安全了,院内错落有致的客房恰好控制了所有的要害地点,只要将各个客房安排妥当的人手,那么就自然而然地形成一个防护圈了,最适合带着保镖仆人远行的达官显贵使用了。只要住过一次,很多喜欢奢华的客人也会喜欢住在平安客栈的,而且平安客栈虽然不够奢华,可是布置陈设也是清雅淡然,也不辱没他们的身份的。

一住下来,林碧就派人拿了自己的帖子送到滨州知府黄炜府上去,黄炜名义上是大雍的官员,可是实际上却是东海侯姜永的家臣,这是人人都心知肚明的事情。姜永的势力虽然已经扩展到滨州,可是姜永本人却是不会在滨州的,想要赴喜筵,必须先递帖子过去,然后由东海侯派船迎接渡海前去。

林碧很想在寿筵之前和南楚使臣会一次面,可是这次南楚使臣却是从杭州从海路过来的,在寿筵之前双方根本不能会面,所以林碧也就听之任之了。

就在北汉众人各自休息之后,王骥却是躺在床上难以入眠,一路上林碧对他监视很严,他一直没有机会和自己人联系上,如今入住了平安客栈,如果他没有记错的话,这是唯一一个和自己人联系的机会。若是不能联系上,得到恩主的指示,那么他怎么去拜见恩主呢?再过三天就是九月二十八日,正是东海侯爱子大婚之时。而十月二日就是恩主爱子周岁喜筵,如何做呢,王骥心中十分犹豫。

正在王骥辗转反侧的时候,有人叩门道:“小人送来茶水,请客官开门。”

王骥扬声道:“门开着,你自己进来吧。”

房门应声而开,走进来一个青衣小帽的店小二,他一边将门关上,一边道:“客官,小店备有各地名茶,不知客官可有什么特别的喜好,小人擅自作主,送来的是龙井茶,若是客官不喜欢,可以随时更换。”

他口中这样说着,行为却是十分诡异,放下茶壶之后,就匆匆脱衣摘帽,王骥先是一惊,就看到那个店小二放在桌面上的一块玉牌,面色一喜,便也宽衣解带起来,口中却道:“龙井就很好,对了,在下要小睡片刻,你不可前来打扰。”一边说着,一边换上了店小二的衣服,将帽子向下压了压,两人身材相仿,面容隐藏起来之后倒有了七八分相似。那个店小二跳上了床,将被子盖着头装成入睡的样子。王骥却是带着茶盘走了出去。他对周围环境本就记在心里,也不多言,就向外面走去。果然刚走出院门,就看到另一个店小二在那里等候。王骥一言不发,跟在那人身后,转了几个圈子,走入了一间十分隐秘的客房。

那件客房中一人负手而立,闻声回头,四目相对,都是目中泪光隐隐,各自上前一步,把臂为礼。那人轻呼道:“赤骥,三年不见了。”王骥,不,应称他赤骥,他一字一顿地道:“绿耳,三年不见,你可是更稳重了,公子好么,众位兄弟好么?”

绿耳张口欲言,却觉得千言万语,不知从何说起,拉着赤骥坐下,整理了一下思绪,这才说道:“公子如今身体已经很是健朗,常常带着夫人驾舟海上,花前月下,好不令人羡慕,如今小公子已经将满周岁,柔蓝小姐活泼可爱,又有李爷和董总管、周尚仪服侍,正是其乐无穷呢。”

赤骥听后面上露出喜色,道:“那就好了,公子退隐之前,派我到蛮地行走,这几年漂流在外,只觉得如同身如飘萍,飘忽无依,如今总算是可以回到公子身边,又逢小公子周岁大喜,真是令我喜出望外。”

绿耳笑道:“谁说不是,这几年我奉命经营平安客栈,也是四海飘流,直到数日前才回到滨州,能够重见公子之面,只觉得心神立刻安稳下来。你被公子选去蛮地探听军情民心,我们原本还为你担忧,只怕是蛮人残狠,你性命堪忧,想不到你不仅平安回来,还博得一个‘神医伯乐’的美名,听说蛮人将你奉为神明,我还以为你会乐不思蜀呢,想不到你还是这样心心念着公子,公子若是知道定然也会感动,或许就不会赶你走了。”

赤骥拿起桌上的茶杯一饮而尽,淡淡道:“蛮人游牧为生,不事生产,每到秋高马肥之际便来劫掠中原,烧杀掳掠,无所不为,我们中原人看了自然觉得他们凶蛮残忍,其实我在蛮地两年,觉得那些普通牧民也是十分朴实善良,我在草原之后,曾经数次遇险,虽然保住性命,可是干粮马匹都失去了,都是被牧民所救。蛮人粗野不文,却是性情纯朴,爱恨都摆在脸上,我倒觉得和他们在一起要快乐的多。可惜草原上不仅有牧民,还有贵族。所谓的贵族多半是各个部落的首领和他们的亲属,这些人多半都是野心勃勃的枭雄,为了争夺女子金帛,他们不仅争着劫掠中原,还彼此互相征战。那些部族里的普通牧民实际上只比奴隶好些,平日辛苦劳作,战时还要上阵厮杀,若是胜了自然可以分得些许赏赐,若是败了,妻儿财产都被敌人夺走,所以他们无不骁勇善战,只因胜负关系生死荣辱。其实即使他们胜了,战利品也多半被贵族所得,他们自己不过是分到一些残羹剩饭罢了。”

绿耳奇怪的问道:“既然那些牧民如此堪怜,他们又是善战的勇士,为什么不肯反抗呢?”

赤骥苦笑道:“要想反抗谈何容易,草原之上生活艰苦,单身一人很难存活下去,这些牧民是离不开部族的,而那些贵族占有最丰美的水草,拥有精锐的战马和兵器,他们轻易地就可以收买部族中最勇猛的战士效死,那些受压迫最深的牧民如何能够反抗,而且不论是何时何地,只要能够存活下去,又有几个人愿意冒着必死的危险呢?”

绿耳犹豫地道:“我曾听说蛮人无恶不作,可是听你这样一说,我都有些同情他们,可是只怕公子听了却会恼怒呢?”

赤骥坦然道:“公子是何等人物,他是不会责怪我的,而且我心中疑惑也要问过公子,那些蛮人虽然是我中原血仇,可是我见他们也是有善有恶,我中原之地,争霸交战之时,手段也未必比他们慈悲到哪里去,所以我定要问问公子,为什么我们不能和平相处,却要互相残杀呢?”

绿耳道:“公子一定能够解开你的疑惑的。”

赤骥点点头,抛下了心中的苦恼疑惑,又问道:“如今你已经成了平安客栈的主人,家财万贯,自然是可喜可贺,可是我听说盗骊更加风光呢?”

绿耳笑道:“是啊,盗骊两次扬帆出海,此去何止千万里,带回的异国珠宝和特产真是令人眼花缭乱。其实最风光的倒是骅骝呢,这小子身份揭穿之后,秦勇将军和老夫人都没有怪他,这小子身份泄漏,又离不开京城,结果被雍王召到身边做了侍卫,听说现在已在明鉴司做了夏侯沅峰的副手,若论官职,倒是他最高了。可惜白义、逾轮、山子、渠黄他们四个如今还在忙着锦绣盟和天机阁的事情,就连这次公子也没有让他们回来。”

赤骥笑道:“你急什么,等到大雍一统天下,我们就可以悠闲自在了。”

绿耳目中闪过一丝憧憬,笑道:“是啊,我真的盼着天下一统,到时候我们就可以不用打打杀杀了。对了,赤骥,你怎么和北汉的人一起来了?公子见了密报,也觉得好笑呢?”

赤骥苦笑道:“我也不会想到会遇上林家的人啊?不过我这次倒是福分不浅,不仅见到了龙庭飞大将军,还见到了和公主殿下齐名的嘉平公主林碧,唉,他们可也称得上是一对英雄侠侣,可惜却是北汉的臣子。对了,公子可有什么吩咐么,龙庭飞和林碧想要我加入北汉军,若是公子有命,我愿去北汉卧底?”

绿耳摇头道:“公子说,林碧和龙庭飞都是不世出的奇才,这样的人不仅心志坚定,而且聪明无比,若非是天长日久,你是得不到他们的信任的,所以你就是在他们身边卧底也没有什么用处,公子让你陪他们参加过小侯爷的婚宴之后,就托词离去,对付这样的人,公子自有手段。对了,公子还让我嘱咐你,不可错过了小公子抓周呢?”

赤骥眼中闪过一丝喜色,道:“请回禀公子,就说赤骥谨尊公子谕令,一定会在十月初二之前赶到的。”

绿耳点点头,道:“我已经安排好了,过一会儿你的替身就会招呼店小二送去新的油灯,你就趁机和他换回来吧。”

赤骥点点头,满腔心事都已经放下,他笑道:“我可是带了一样珍贵的礼物给小公子,十月初二我一定会赶到的。”

绿耳笑道:“是啊,我也准备了礼物,只是恐怕谁也没有盗骊的礼物新奇,他可是刚从异国回来的。”

赤骥道:“这也没有法子,不过我的礼物也不会差到哪里去的,那可是我为蛮地实力最大的一个族长医治坐骑所得的谢礼呢?”

两人又谈了片刻,有人前来禀报说是时间已到,赤骥便拿了油灯走回住处,林碧虽然派了人守夜,可是却没有禁止店小二出入,赤骥顺利的回到房间,那个代替他躺在床上的店小二换回衣服,悄无声息地走了出去。赤骥躺在床上,没有多久便进入了梦乡。

第二天,林碧下令众人可以出去散心,只是不许招惹是非,不过王骥却被林彤拉上一起出去了,虽然不知林彤的心思,可是林碧的心思王骥却是明白的,现在林碧绝对不会让自己离开他们的视线的,果然,负责保护林彤的侍卫也被林碧放了假,而换上了林碧自己的两个亲信侍卫,这一男一女在王骥看来武功都很出色,王骥自知没有本事胜过这两人联手,若是他想趁机离去是绝对没有机会的,林碧行事果然是十分谨慎。不过王骥早已和自己人取得联系,所以也就无拘无束地陪着林彤在滨州城内游玩了起来。

这滨州城原本只是一个沿海小城,如今却已经是俨然大邑,城内商贾云集,各种店铺比比皆是,商铺之中更是琳琅满目,令人目不暇接。林彤兴奋地四处瞧看,不时被一些新奇的东西吸引过去。她身边的两个侍卫却是始终目光敏锐地留心着周围的情形。

走了几个时辰,手里已经堆满了盒子包裹的王骥苦恼地望着仍然兴致勃勃地林彤,也不知道为什么,这位小郡主偏要把所有东西都让他拿着,那两个侍卫却都只是笑吟吟的看着笑话。王骥自然知道他们不会主动帮自己提东西,免得妨碍他们的手脚,可是自己为什么要做这个小郡主的仆从。

正在王骥忿忿不平的时候,林彤已经一眼看到一家出售兵器的铺子,她虽是女子,可是自幼生长在战火之中,对于兵器战马是从心里喜欢,所以便兴冲冲地走了进去。这个铺子十分宽阔,四壁上挂着刀枪剑戟,都是上好的利器。在中间的一张长条桌子上,摆着一些精美的匕首短刀,其中有一些样式古怪,一见就知道不是中原打造的兵器。

林彤好奇地走了过去,拿起一柄弯刀仔细看去,这是一柄连鞘弯刀,绿色的鲨鱼皮鞘,温润洁白的象牙刀柄,手握之处缠着乌金细丝,刀身如同新月一般形状。林彤将刀抽出,只见刀光如霜似雪,心中便是十分喜爱。这时候,那个中年掌柜走了过来,挥手让接待林彤的伙计离去,笑呵呵地道:“小姐,这是从波斯买来的弯刀,可以切金断玉,最适合会武的小姐佩戴防身。小姐若是喜欢,小人愿意折价奉送。”

林彤拿着弯刀,走到试刀的木桩前,一刀劈下,那坚硬的老木被轻轻松松的削去了一角。林彤大喜,道:“这把刀多少钱?”掌柜连忙道:“这刀在波斯可是王室所用,小人不敢擅自抬价,只要三千两银子就行了。”

“什么?”林彤一惊,虽然早知道这把刀不会便宜,可是三千两也未免太贵了一些,她虽然出身名门,有郡主的封号,可是林家时代镇守代州,为了练兵,银钱本就如同流水一样花出去,而林家又以清廉著称,所以林彤可没有这么多银两。叹了一口气,林彤放下了短刀,若是自己真的花三千两银子买一把不能上阵杀敌的弯刀,只怕要被父亲责罚了。无精打采地向外走去,林彤忍不住回头了好几次,看向那把精美的波斯弯刀。

这时,一个小女孩蹦蹦跳跳的走了进来,她走得很快,偏巧林彤又在回头,两人撞在一起,那小女孩年小体轻,“哎呀”一声向后倒去。林彤是学武之人,立刻反应过来,伸手就将小女孩抱住,往下一看,只见这个小女孩五六岁的年纪,相貌秀丽娇俏,肤如凝脂,一双杏眼清澈明净,又带着一丝狡黠的意味,眉宇间的气质更是十分灵秀。林彤不由笑道:“小妹妹,又没有撞伤你?”

小女孩摇摇头,道:“大姐姐放心,蓝蓝没有伤着。”

林彤松开双手,那个小女孩冲到桌旁,拿起方才林彤喜欢的那把弯刀,兴冲冲地道:“掌柜伯伯,我带钱来了,把它卖给我吧。”

林彤的目光一凝,这样一把贵重的弯刀,这个小女孩居然要买,这是怎么回事?

那个掌柜也是有些尴尬,方才这个小女孩就是要买这柄弯刀,自己当然不信一个小女孩会有那么多银子,所以虽然小女孩要求自己留下弯刀暂时不要出售,自己却没有遵守约定,有些赧然的看了林彤一眼,他和气地道:“小姑娘,这可是要三千两银子啊。”

小姑娘得意洋洋地道:“我是带了银子的,不过给别人拿着罢了。海叔,海叔,你走快一些么?”

随着小女孩清脆悦耳的声音,一个浑厚的声音道:“来了,来了,小鬼头跑得这么快,海叔可追不上你。”声音还在耳边,一个青衫男子从外面走了进来,这是一个三十多岁的男子,相貌斯文俊朗,只是肤色古铜,脸上的皮肤粗糙,一看就是常年曝晒的结果,这个男子虽然衣着朴素,却是气度沉稳,神色间带着淡淡的威仪。掌柜的目光一闪,已经认出了这人身份,满脸堆笑的上前道:“原来是海爷来了,说什么买呢,小人这点生意都是托您的福,小姐若是喜欢,尽管拿去就是。”一边说着,掌柜的一边寻思,什么时候海爷身边有了这么一个宠爱的侄女呢?

那男子淡淡一笑道:“都是将本求利的生意人,我怎好占你的便宜,这个丫头是我一位至交的女儿,最是顽皮捣蛋,今次看中了这柄弯刀,花的也是她自己的零用,这是这丫头自己的事情,你也不用顾忌我,该多少就是多少。”

小女孩撅着嘴道:“海叔就是这样不讲情面,也不帮着蓝蓝侃价。”

男子微微一笑,道:“谁让你这样倔强,海叔手上什么珍贵的物事没有,你若喜欢尽管选了去,却偏偏看中了这把弯刀。”

林彤听这人口气很大,不由更加生出几分好奇,装作挑选刀剑的模样,留下了看起了热闹。

那个小女孩生气地道:“那怎么成,这可是爹爹答应的,让蓝蓝自己买一样礼物给弟弟,若是从海叔那里挑选,就不是蓝蓝送的礼物了。”

那男子失笑道:“你爹爹一向是不计较这些的,偏偏你这样倔强,好了,海叔不管你就是了。”一边说着,一边拿出银票递给那掌柜的,口中还道:“这下可好了,你这两年的零用钱和红包都搭上了,将来可别来找我借钱就行了。”

小女孩得意洋洋地道:“这个海叔就不用担心了,娘亲最疼我了,一定会多给蓝蓝零用的。”

这时候,那个掌柜已经将那柄弯刀用锦盒装好,恭恭敬敬的递给那个男子,并奉还了部分银票,道:“海爷,小人天胆也不敢在您头上争利,还请海爷笑纳。”

那个男子笑道:“这也不是什么大事,你们千里迢迢的带了货物回来,哪有贱卖的道理,我这个侄女喜欢这些精巧的东西,以后免不了打扰,你只要价钱公道些也就是了。”说着将那些银票又奉还给那掌柜。那掌柜的眼珠一转,道:“那小人就恭敬不如从命了,海爷,小人有样精巧的物事送给小姐赏玩。”说着他让伙计去后面拿来一个精钢制成的古怪物件,熟练的一拉一翻,那物件彻底打开,原来是一把精巧的手弩,精钢打造的弩臂用铰链紧密地结合在一起,一根极为结实的不知道用什么材料制造的弦丝,牢牢地系住弩臂两段。整把手弩完全打开,并不比手掌大多少,正好放到袖子里,用来防身最好不过。那个掌柜道:“这是小人无意中得到的,因为只有一件,威力也不是很大,所以没有拿出来出售,就送给小姐赏玩吧。”

小女孩眼中闪过兴奋的光芒,一把抢过手弩,翻来覆去看了半天,才道:“真的很精巧,海叔,蓝蓝很喜欢。”明亮的眼睛里面充满了恳求,那个男子微微一笑,道:“既然是人家的好意,你就留着吧。”说罢牵着小女孩的手向外走去,那掌柜的跟在后面相送,满面笑容,显然十分高兴那海爷收了礼物。

林彤想着是什么人让这掌柜必恭必敬,想必是滨州大有来头的人物吧?一边想着,不由眼光盯着那男子,露了形迹。那男子早已察觉到有人看着自己,但他身份非常,有人留意自己这是再寻常不过的事情,所以也不在意,不过出门的时候仍然顺便瞧去,谁知一看之下,他的面上露出一丝古怪的笑容,眼中更是闪过一缕寒芒。

等到那男子远去,林彤问那掌柜道:“这人是谁啊,你这样奉承他?”

那掌柜的笑道:“姑娘是外地人,或者不认识,这位就是我们滨州最大的船行老板,只手掌控远洋贸易的海无涯海爷啊。”

林彤惊叫了一声,出门瞧去,那海无涯已经没有了踪影。

这时,那个掌柜正对着伙计们喊道:“跟咱们打交道的海公子那是最精明的人,要想占点便宜比什么都难,海爷为人倒是慷慨大度,就是为人端谨,不喜欢应酬,是最难巴结的一个人,想不到今日这样巧,让我得了彩头,还不快去给东家送个信,过两天就请东家带着礼物去拜访海爷……”

林彤一跺脚道:“真可惜,若是姐姐在就好了。”说罢,林彤也没有逛街的兴致了,郁闷的向客栈走去。王骥淡淡一笑,眼中闪过一丝喜悦和憧憬。

\chapter{第五章 同舟共渡}

海仲英,号无涯,荆楚人,世代书香,英为庶出,性豪爽,不为嫡母所喜,后父母亡故,仲英携资材至闽境,组船队行商海上,颇豪富,仲英慷慨好义,人皆敬之。

武威二十三年,仲英赴南海,中道遇海匪,船货尽失,仲英仅以身免,时货主及船夫家人逼勒甚急,或劝其隐姓逃债。仲英道,我以诚信待人,今若逃,子孙后世不能见人矣。乃倾家荡产以偿债。后仲英东山再起于东海,商贾中人与其议价时,往往一言而决,皆服其诚信耳。

——《雍史·货殖列传》

林碧听了林彤的转述之中,安慰道:“彤儿,你也不用遗憾,海无涯在滨州乃是一言九鼎的人物,能够和他见面当然好,可是此人明显和东海侯关系密切,光是说服他也是没有用处的,东海侯若是不点头,谁也不能作主的。而且我们也已经打听清楚,若想说服海无涯,还不如说服他的侄儿海骊有效得多。海无涯至今未娶,两年多前,他的侄儿海骊从南楚前来投奔,如今已经成了他的左膀右臂。我们已经派人查过,海家在多年前因为洪水而毁于一旦,他这个侄儿流落在南楚,飘零多年,几乎什么都做过,直到两年多前,这个海骊不知从哪里得知海无涯是他的叔父,这才千里投亲。海无涯为人最是大度,全不计较昔年的兄弟纠葛,将这个侄儿收留下来。海骊此人年纪虽轻,却是心思细密,精明过人,海无涯的生意他倒是能够做上七分主的。想要完全排除大雍,我看恐怕是没有指望的,若是能够说服海骊倾向我们,那么我们的收获就很大了。”

林彤听了不由心想,既然海无涯只有一个侄儿,那么那个小女孩又是什么人呢,能让海无涯这样宠爱,她的身份一定是很不寻常吧。

不过她也知道这个问题是得不到答案的,便又问道:“姐姐,还有一件事情,我怎么觉得你对王骥十分提防,一点也不像你平日的举动。”

林碧轻轻一叹,道:“傻孩子,你当我和庭飞真的只想招揽王骥么?”

林彤一惊,道:“怎么,你们?”

林碧笑道:“我和庭飞都怀疑这王骥的主人的身份。王骥此人,不仅弓马出众,而且颇富文采,更有相马医马的本事,更难得的是他的气度,对着我和庭飞这样的身份,仍然是不卑不亢,一路行来,我见他对山川地理也十分熟悉,这样一个人,不论在哪里都不会被人忽视的,你说他在南楚和大雍都待过许多时候,为什么却没有加入军旅或者被人招揽。”

林彤争辩道:“他是兽医,或者是不喜欢从军或者做人家的下属吧?”

林碧又道:“我们一路几乎是行军一样的赶路,可是他不仅毫无疲惫之色,还常常说些笑话和见闻哄你开心。而且我见他对军中之事也不是很陌生,显然他不是从过军,就是受过这方面的训练,小妹,这个人的身份并不简单。”

林彤脸上红了又白,起身就要出去,林碧拉住她道:“你去做什么?”

林彤怒气冲冲地道:“我要去问他,为什么要做奸细,为什么要欺骗我——和姐姐。”

林碧摇头道:“我看他也不是存心骗你,一路上他并没有特意和你亲近,也没有探听军情,我想他遇见你乃是意料之外的事情,他,应该不是存心做细作的,我只是说他的出身必定有些问题。你看他对自己的恩主推崇备至。小妹,什么样的人可以有这样的奴仆呢,你有没有想过?”

林彤怔忡了半晌,想起王骥所说过的每一句话,然后她脑海中浮现王骥在谈到那个江哲的时候,眼中无论如何都掩饰不住的神采。不由吞吞吐吐道:“姐姐,你不会以为,以为,他的主子是那个人吧?”

林碧微微一笑道:“本来我也不会这样凭空猜测,可是他的主子偏偏在东海,这就更加引起了我们的疑心,当初江哲退隐之后,天下想要谁不想知道他的下落,这种人若不将他控制在手中,是没有人可以放心的。仔细想一想,江哲不是平常人,他是雍王的心腹谋士,又带着大雍的宁国长乐公主,长乐公主本是南楚王后。再想一想江哲的作为,南楚他不能去,原蜀国如今被南楚和大雍瓜分,可是他逼死了蜀王,他若是一个聪明人,最好今生今世都不要到蜀中去,现在蜀中局势并不稳定,锦绣盟打着复国的旗号在蜀中来去自如呢。他若留在大雍,普天之下,莫非王土,只怕是躲不过大雍官府的耳目的,若是来北汉,他就不怕我们将他捉起来么。这天下之大,只有一个地方是他可以藏身的,就是东海侯的辖地。东海侯现在和大雍虽然关系缓和,可是还没有归属大雍,姜永的性子倔强,只怕李援未死之前,他都不会归顺大雍的,而且根据我们得到的情报,东海侯之子姜海涛曾经身受毒伤,就是江哲医好的。你说,东海岂不是江哲归隐的最好地方,东海侯必然将他待为上宾,大雍也不会因此担心他被别国所用。只不过东海茫茫,海上交战,也不是我们北汉所擅长的,而且,江哲虽然是厉害,我和庭飞也不畏惧他,这件事情自然就放下了。可是这次遇到王骥,我就猜测恐怕他的主子乃是江哲,彤儿,你说若是江哲死在东海,会发生什么事情?”

林彤虽然年幼,很少参与军机,可是自幼耳濡目染,所以只想了片刻,就惊叫道:“只怕大雍皇帝会异常愤怒,东海和大雍之间会反目成仇,毕竟江哲是死在东海的。”

林碧好整以暇地道:“这个倒还罢了,雍帝李贽英明过人,迟早会明白东海乃是无辜的,虽然会有迁怒,可是也不至于因此影响最终的结局,东海归降大雍,是迟早的事情,可是李贽会千方百计追查暗杀江哲的凶手,我北汉和南楚就是最大的目标,到时候我们若是宣扬出去是我们做的,那么李贽就会下令齐王李显立刻进攻北汉,李显虽然兵多将广,可是现在君臣有隙,将士狐疑,我们北汉必定能够取得一场大胜,一举攻入大雍北方,居高临下,让大雍数年之内再无力和我们相抗。而南楚也可以趁机发难,彤儿,到时候我们就不用日日忧心国破家亡了。”

看着姐姐神采焕发的模样,林彤心中一阵悲喜交加,她自然知道这些年来父亲、姐姐和姐夫日日为国事忧虑,若是能够遂了姐姐的心愿,自然是最好不过,可是不知怎么,林彤想起了王骥所说过的江哲的事情,竟然不忍见那样一个人死在刺杀之下。

林碧似乎明白她的心意,握住她的手道:“彤儿,你如今已经及笈了,姐姐希望你能够明白,不是姐姐喜欢这样做,两国交战,谁不是用尽手段心机,这是半点慈悲都容不得的,咱们几个兄弟都是猛将、勇将,却偏偏没有一个可以帅才,你虽然年幼顽皮,可是我知道你才智不比姐姐差,彤儿,你要好好努力,过几年,等你可以担当大任,姐姐就可以安心的跟着你姐夫南征北战了。”

林彤愣了一会儿,突然落下泪来,抱着林碧,哭泣道:“姐姐,是我们不好,要不然就不会让你现在还不能嫁给姐夫,姐姐,你放心,彤儿以后再也不会贪玩了,以后等到彤儿做了大将军,带着千军万马镇守代州,让你和姐夫没有后顾之忧。”

林碧心中一酸,也抱住林彤,低声道:“彤儿,这是命运,我们林家从来没有不忠不义之辈,当年爹爹和娘亲本是两情相悦,可是外公起兵立国之后,爹爹宁可和娘亲永不见面,也不肯背叛晋主。我听几位叔伯说,当年先主大军将代州围住,城中已经粮尽,这时先主派人来告诉爹爹晋帝被废的消息,爹爹悲恸欲绝,虽然为了代州军民不得已归降了先主,可是爹爹却还是不肯在北汉做官,托辞养病,只在家中休养。后来蛮人犯境,代州危急,先主亲来相请,为了乡梓黎庶,爹爹终于重新披挂上阵,后来,爹爹就做了北汉的臣子。这么多年来,外公和舅舅都对我们林家信任倚赖,从无疑忌,彤儿,我们林家不能再看着家邦被人侵占了。身为林家的儿女,为了北汉,为了林家,没有什么不可以牺牲的,姐姐知道,你有些喜欢那个王骥,可是你要记着,他不是北汉人,而你是林家的女儿。”

林彤脸色变得苍白,她没有反驳姐姐的话,她真的是喜欢上了那个温文儒雅中带着坚强果敢的少年,她曾经以为,既然王骥已经答应姐夫留在北汉,那么或许就有可能将他留在身边。可是,现在林彤却终于明白,她那如同春花一般绚烂美丽的初恋,已经陨落在秋风萧瑟当中了。然后她听见林碧说道:“这次我带了明暗两批人手过来,若是发现江哲的踪迹,就要刺杀于他,所以王骥是万万不能放松,你要小心,不要让他传了什么消息出去,跟着他,一定能够找到江哲的。”

当王骥推门走出房间,想到客栈前面的饭堂用饭的时候,恰好看到林彤从林碧的房间走了出来,他正想和她打个招呼,却发不出声音来,那个娇俏可爱的小郡主周身上下焕发出艳丽无比的光芒,这样的她仿佛是另一个林碧一般。她的目光飘过,落在王骥身上,她微微一笑,那笑容是那样灿烂,可是王骥却觉得一阵心悸,林彤走过他的身旁,微笑道:“喂,你是要去前面用饭么,我也很想去前面吃呢,那里一定热闹多了。”王骥想要答话,可是却觉得口干舌燥,竟然无法说话,眼前的这个小郡主,是那样的熟悉,又是那样的陌生。

九月二十七日,在东海侯属下的引领下,林碧等人上了一艘大船,那是海氏船行特意准备的一艘大船,前面迎接参加喜筵的客人前往东海侯所占据的海岛。这艘客船只有相当身份的人才能搭乘。负责迎宾的是东海侯姜永的爱将罗横,他笑容可掬的在甲板上和客人攀谈,完全没有传闻中海上屠夫的模样。

林彤刚上船的时候还觉得很兴奋,可是船一动起来,便觉得头晕目眩,虽然舍不得海上的风景,却还是被林碧强迫着回去休息了。林碧却是站在船头,享受着习习的海风。用余光留意着船上的客人,船上的客人很多,身份各异,可是显而易见,多半都是商贾中人,能够坐上这艘船的,至少也是富甲一方的富商吧。

这时,身后有人说道:“草民海仲英,闻知公主殿下也在船上,特意前来拜见,还请公主恕草民冒昧。”

林碧回过头去,只见在自己的几个侍卫的防护圈外,站着一个身穿深蓝色衣袍的中年男子,相貌斯文俊朗,肤色呈现阳光曝晒之后的古铜色,他身后跟着一个少年,相貌清秀俊雅,肤色淡褐,显然原先的肤色应该是十分白皙的,应该是近年来被阳光晒成了这样的褐色。这两个男子相貌轮廓有七八分相似,显然有着血缘关系。

林碧心中一动,道:“原来是海无涯海先生和海骊海公子,今日相见,本宫十分荣幸。”一边说着,一边让侍卫放这两人过来。

海无涯笑道:“这无涯二字不过是大家送的别号罢了,因为冲犯了东海侯先尊的名号,所以如今已经不怎么使用了,公主殿下称在下一声仲英也就是了。殿下亲临东海,仲英本应前去拜见,只是殿下身份高贵,草民不敢亵渎,还请殿下见谅。”

林碧微微一笑,道:“海先生不用客气,怎么海先生没有前去帮忙东海侯料理婚宴呢,凭着先生和侯爷的关系,应该去帮忙的。”

海无涯眼中闪过一丝冷淡,道:“小侯爷的未来夫人乃是南闽越家的女儿,海某和越家素有旧怨,不愿破坏了气氛,所以没有去帮忙。”说到这里,海无涯似乎有些醒觉,掩饰地说道:“海某的奇珍会将在九月三十日举行,不知道公主是否有兴致,这次海某带了些海外奇珍,有些或者公主会感兴趣的。”

一边说着,海无涯伸出手去,一直微笑不语的海骊取出一张红色柬帖递给海无涯,海无涯将柬帖呈给林碧,道:“这上面有将会展示的一些贵重珍品的目录,若是殿下有兴趣,可以先看上一看。”

林碧接过柬帖,也不打开,笑道:“海先生果然是会做生意,南闽越家也是船业巨子,想来东海侯想要多个合作者呢?”

海无涯眼中闪过一丝冷笑,道:“殿下误会了,小侯爷的生母本就是南闽越家的人,这桩婚事也是亲上加亲罢了。”

这时,远处传来一个小女孩的笑声道:“海叔,海叔,你看蓝蓝射到了什么?”

林碧闻声望去,只见一个穿着粉色衣衫的小女孩正在蹦蹦跳跳的跑过来,她右手提着一具精巧的手弩,左手拎着一只被小巧的弩箭射穿了头部的海鸟。

在林碧的示意下,那些侍卫并没有阻拦小女孩,她高高兴兴地冲进海无涯的怀中,献宝一般地举高海鸟给他看。

海无涯宠溺地道:“好了,若是你爹爹知道,一定会很高兴的,不过大概他更喜欢你像个千金小姐吧。”

小女孩反驳道:“才不会呢,爹爹说蓝蓝喜欢怎样就怎样,以后蓝蓝还想跟着骊哥哥去看看那些红头发绿眼睛的夷人呢。”

海骊笑道:“这个我可不敢答应,谁不知道公子和夫人将小姐视若掌上明珠,我若是带你出海,公子最多不过是禁你的足罢了,我恐怕要被逐出门的。”

小女孩沮丧地道:“骊哥哥也不敢,呜呜,上次蓝蓝想托人给骏哥哥捎信,可是谁都不敢。”

海骊听到小女孩这样说,心中一凛,眼光担忧地瞧向林碧,只见她似乎没有察觉什么,只是满怀笑意的看着小女孩,才放下心来,歉意地道:“公主,小孩子顽皮,让您见笑了。”

林碧笑道:“不妨事,很可爱的小姑娘,叫什么名字,海公子和他的父亲有主仆名份么?”

海骊笑道:“她叫柔蓝,是海骊恩主的爱女,当年草民流浪四方,被恩主收留在门下,后来得知家叔的下落,前来投奔,蒙主人恩典,换海骊自由之身,只是旧日恩情不敢相忘,所以仍然以主仆相称。”

林碧看着柔蓝满含着好奇的大眼睛,伸手欲、将她抱起,海骊接过柔蓝手中的弩弓和海鸟,柔蓝双手得到了自由,自然而然的环抱着林碧的脖颈,林碧心中一暖,笑道:“小蓝蓝,你爹爹怎么不在这里啊?”

海骊一皱眉,正要抢着答话,却看到一个侍卫警告的眼神,这时候柔蓝已经说道:“爹爹不喜欢那么多人的,蓝蓝好不容易才求娘亲答应,让海叔和骊哥哥带着蓝蓝去看热闹呢?”

林碧又笑道:“那么蓝蓝姓什么呢?”

柔蓝的眼睛忽闪了一下,道:“这个,蓝蓝也不知道啊,爹爹就是爹爹,蓝蓝就叫蓝蓝,海叔,爹爹姓什么啊?”

众人听了都是会心地微笑,一个小孩子不知道父母的姓名是很平常的事情的,林碧也只能一笑了之。

看着蹦蹦跳跳远去的小柔蓝,林碧心道:“我或者太多疑了,怎么见到谁都想着和那人有关呢?”

这时,跑得飞快的小柔蓝和一个小男孩撞在一起,那个小男孩只有不到四岁的模样,可是却比柔蓝高一些,壮一些,两个孩子撞在一起,那个小男孩只是踉跄了一下,柔蓝却坐倒在地上。

海骊连忙走过去,将柔蓝提了起来,那个小男孩冷冷的看了两人一眼,就要转身离开,柔蓝大叫道:“喂,你撞到我了,怎么不赔礼就走。”

小男孩眼中闪过鄙视的神色,冷冷道:“你也有错。”

柔蓝只觉得脑子里轰的一声,她虽然年纪不大,可是平日遇到的人不是对她视若珍宝,就是必恭必敬,最差的也是颇为喜爱,从来没有人这样对她无礼,她的眼睛不知怎么红了起来,腾的一下跳了起来,一把拽住小男孩的衣衫,道:“快给我赔礼。”

小男孩本要挣脱,可是一眼看到柔蓝泪水盈盈的双眼,不由手上一软,却还是嘴硬地道:“你也有错的。”

柔蓝眼珠一转,松开手,道:“是我不好,不该乱跑的,对不住。”

小男孩一愣,还没有反应过来,柔蓝已经双手叉腰道:“我已经赔过礼了,该轮到你了。”

小男孩这下可是真的愣住了,半晌才呐呐道:“是我不好。”

柔蓝这才破涕而笑,露出得意的神情,这时候,传来一个豪爽的笑声道:“好本事,麟儿,可是很难看到你道歉呢?”

小男孩脸一红,低着头走到一个锦衣男子的身后,那个男子三十多岁的年纪,相貌英俊挺拔,幽黑的眼睛透着冰冷的寒气,虽然他在说笑,可是从他的神情却觉察不出一丝欢喜。这个男子周身上下都透着残忍冷酷的气息,可是举止之间却又是那样优雅从容,这个男子,仿佛是表面上驯服的猎豹一般,让人担忧他随时都有可能冲破樊笼撕裂敌人的胸膛。

小男孩孺慕的目光望着那个男子,可是那个男子却没有再望他一眼,而是淡淡的瞧着那个小女孩,小男孩眼中浮现出失望,低下了头。

林碧心中浮起警戒,这个男子绝对是一个危险的人物,那个男子的目光落到了林碧的身上,眼中泛起一丝笑意,林碧心中一寒,缓缓移步上前,她不愿在任何人面前低头,尤其是这个很可能是敌非友的男子。

这个男子淡淡道:“嘉平公主,初次相见,果然是闻名不如见面。”

林碧目光一闪,道:“想不到齐王殿下竟然会离开军中,当真令林碧惊奇万分。”

男子大笑道:“十年修得同船渡,本王真是万分荣幸,嘉平公主乃是女中豪杰,代父镇守代州,蛮人敬畏,本王微服至东海,原想着有机会见到公主一面,今日一见,足慰平生,龙庭飞虽然厉害,本王倒也没有放在心上,可是他有你这个未婚夫人,倒是让本王羡煞。”

林碧见他虽然言语放荡不羁,可是神色间却带着浓浓的阴郁之色,想到这人本是有名的风流浪子,可是两年前遭遇大变之后,不仅将府中姬妾几乎全部遣散,而且从此不近女色,为了亡妻如此情重,林碧心中油然生出怜悯之心。轻轻叹息一声,林碧淡淡道:“王爷过誉了,怎么王爷会到了东海,听说贵国这次的使臣乃是庆王李康呢?”

男子神色一黯,淡然道:“本王和东海侯乃是姑表兄弟,这次侄儿成婚,本王乃是私人身份道贺。赤骥,你怎么也在这里,你的主子呢?”

\chapter{第六章 生死无恨}

武威二十四年,仲英潦倒长安市,忽一日,有寒姓者登门造访,以重金偿其债,未几,仲英赴滨州,建海氏船行。武威二十五年,海氏得重资注入,造大船赴远洋,纵横海疆万里,遂得无涯别号。

——《雍史·货殖列传》

赤骥差点没有骂出声来,他万万想不到齐王竟然会和自己说话,就是白痴见到自己和林碧等人一起,也不会贸然说出自己的名字啊,虽然对齐王仍然记得自己感到奇怪,毕竟当初只是在南楚江哲为李显治伤的时候见过一面罢了。一边在心里面恶狠狠的诅咒,赤骥皮笑肉不笑地道:“回禀王爷,草民早就被恩主遣散,这次来东海贺喜,王爷若是有心,草民愿意引见。”

李显“哦”了一声,淡淡道:“你主子的性子也太古怪了些,放着荣华富贵不享,偏偏喜欢自找苦吃。”继而笑道:“碧公主,你我两国虽然敌对,可是这里是东海,本王也不想生出事端,难得可以抛开军务,我想公主不会拒人于千里之外,本王有意邀请公主同赏海景,不知道可有这个荣幸。”

林碧收回注视赤骥的目光,眼中闪过一丝寒芒,道:“能和王爷相见,林碧虽是女子,也不愿错过和王爷倾谈的机会,王爷请。”

李显露出赞许的神情,跟着林碧向船头走去,在李显走过柔蓝身边的时候,却停住脚步,笑道:“我的麟儿比你还要小一些呢,你肯不肯陪他玩一会儿呢?”柔蓝眼中露出迷茫的神色,对着这个似曾相识的陌生人说道:“好啊,不过我可是姐姐,若是他不听话,我可要管教他的。”李显哈哈一笑,眼中第一次多了一丝真正的笑意,道:“好啊,麟儿,你可听见了,若是你不听她的话,她可以替我管教你呢。”说完,走到船头林碧身边,两人侍卫将船上众人和他们隔绝开来,免得他们的谈话被不相干的人听到。

柔蓝得意的对着李麟说道:“听见了没有,你爹爹说了,要你做我的弟弟,太好了,我的小弟弟还不会走路呢,我还管教不了他,就先试试管教你吧。”

李麟脸上终于露出苦恼的神色,这一刻,他的神情才真得像一个小孩子,而柔蓝已经扯着他向后面跑去,大呼小叫地,好像是找到了新的玩具。

望着水天一色的茫茫大海,李显欲言又止,林碧轻挽秀发,道:“怎么齐王殿下不说话了,想来殿下有很重要的的事情要和本宫密谈,本宫不避嫌疑,与殿下坦诚相见,殿下怎么却矜持起来了。”

李显突然笑了起来,林碧一愣,立刻察觉自己的语意有些暧昧双关,脸一红,道:“若是王爷不肯谈正事,那么林碧只有告退了。”

李显淡淡道:“公主此行想必是身负重任,但不知公主可考虑过后果么?”

林碧面色一沉,冷然道:“不知王爷此言何指,本宫奉王命出使东海,不知有何后果可言,难不成东海是大雍管辖,容不得别人沾手么?”

李显叹息道:“我素来不喜欢多事,公主出使东海,乃是公事,我来参加喜宴,却是私事,所以不论公主想要做什么,我都懒得理会,可是公主此行随驾不少,本王得到密报,魔门宗主京无极的几个弟子,本来应该留在龙将军身边保护他的,可是这些日子都不见了踪影,我原以为龙将军担心公主安危,所以让他们随行保护,可是今日一见,公主身边却没有这几个人,向来是在暗处保护公主了。若非是公主有心作些事情,为什么要把他们隐藏起来呢?”

林碧轻轻侧过头去,不让眼中的杀机泄露出来,笑道:“殿下过虑了,或者这些人被庭飞派去做事了,说不定他们如今正在你们大雍境内作斥候呢?”

李显微微一笑,道:“公主既然这样说,那就是这样吧。滨州名义上属于大雍,实际上被东海侯控制,然而东海侯的势力虽然不小,却主要在海上,所以这滨州反而是东海侯势力最薄弱的地方,毕竟谁也不愿意在随时可能会失去的地盘上消耗实力,所以公主敢于带了大批人手来滨州,而且也有法子调动他们做任何事情,一击远扬,凭着北汉高手的骑射之术,自然可以让他们随时撤回贵国境内。本王只是想警告公主,有些人可以冒犯,有些人却是最好不要得罪。”

林碧心中一动,自己来到东海,所为何事只有自己心里明白,其余的人只是奉命行事,而且就是自己也只是得到“便宜行事”的指令罢了,怎么这齐王的语气,倒像是知道自己要对付江哲呢?自己虽然已经定下了刺杀江哲的计划,但要付诸实施却需要种种条件,自己需得找到江哲的藏身之处,而且还要有至少六七成的把握才能行动,就是现在,自己也不敢说这个计划定然可以达成,自己带了许多人手,倒是大半是为了应付南楚可能的行动的。

李显见林碧默然不语,不由心中好笑,自己此行本是为了求见那人而来,原以为东海茫茫,若不能得到东海侯协助,必然是无从相见,想不到昨日那人竟派了使者前来和自己相见,那人在滨州城设下耳目无数,大小事情无不了如指掌,就是自己这般仓促而来,仍然是避不开他的耳目,更别说本就令人瞩目的林碧了。

林碧想要刺杀江哲,这个李显倒是不觉得奇怪,当初江哲初入大雍,不就是遭到凤仪门和南楚的刺杀么,这样一个人活在世上,自然是有很多人寝食难安的,北汉和大雍多年交战,乃是生死仇敌,不论他们想要做什么都不奇怪。更何况北汉自有俊杰,焉能不会想到江哲正是唯一可以调和自己和皇兄关系的人呢?自己不就是为了目前的困境而来求助的么?不过,李显倒是很想知道为什么江哲不设下陷阱,将北汉高手一网打尽,反而让自己打草惊蛇,迫使林碧放弃刺杀呢。

看了林碧一眼,见她眼中杀气仍然隐伏,而且更添了几分忌惮疑惑,李显轻轻摇头,道:“殿下应该见过蜘蛛捕食,张开天罗地网,布下重重伏兵,只待敌人入网,就是必死无疑。公主心中忌惮之人,最擅长的就是布局,等你想到要对付他的时候,早已经深陷罗网之中,难有还手之力。他在东海将近三年,此地早已经是他的地盘了,公主一举一动都瞒不过他的。”

林碧心中一寒,此刻她终于明白传言不虚,自己的举动早已经落入那人计算当中,否则齐王怎会知道。可是心中疑虑又生,难道齐王和江哲早有秘密联络,否则齐王怎会知道这些事情,可是为什么那江哲明明已经占了先手,齐王却警告自己,这不是和江哲过不去么?越想越是觉得错综复杂,林碧勉强笑道:“多谢王爷指点,本宫只是担心舍妹安危,所以多带了几个属下罢了,东海虽然中立,可是和大雍却是日益亲近,王爷也不能怪本宫多加提防的。不过本宫倒是奇怪,想来王爷早就知道那人隐居东海,为什么大雍朝廷却任其流离在外呢,这样的人才若不善加使用,岂不可惜。”

李显见林碧眼中杀气已经消退,欣然道:“公主不必多心,若是公主见过那人,就知道他的性子实在古怪,本王也是来了东海之后才见到他派来的使者的。此人平生最爱就是明月清风,对于军政大事是能躲就躲的,东海茫茫,又有东海侯庇护,皇兄和我虽然都有心请他回去,可惜他却是神龙见首不见尾,始终找不到他的隐居之处,再说父皇也没有松口,皇兄也不好大张旗鼓的寻找,而且东海侯至今仍然对大雍耿耿于怀,皇兄也不愿惹恼了他。若非是本王被龙将军迫得狼狈不堪,也不敢这样鲁莽,连他隐居何处都不知道,就来求他襄助,本王原本是打算逼着东海侯引见的。不过托公主的福,本王刚来东海,就见到了他的使者。”

林碧心情已经渐渐平复,本来刺杀江哲就是不得已而为之的事情,既然已经被人识破,自然也没有必要勉强进行,倒不如即兴而为,或者会有更大的收获呢,有趣得看看李显,心道,若说起来,杀了这人或者更有价值呢。

李显见林碧笑容古怪,立刻猜出了她的心思,开怀大笑道:“公主不用这么狠心吧,说起来,我和龙将军也是惺惺相惜呢。能在战场上生死相搏,岂不是人生一大快事,那些阴谋诡计就是效果再好,也是流毒无穷,我等本是用性命争夺胜负的军人,何必还要在战场之外钩心斗角呢?那些事情就让那些文官去做吧,公主何不随龙将军和本王在战场上生死相见,那岂不是生也快意,死也无憾。”

林碧听了只觉心潮澎湃,这本是她心中所想,只可惜因为北汉以一州之力对抗中原,早已是捉襟见肘,若是再僵持下去,只怕就是胜了也是国力疲敝,更何况齐王固守,坚壁清野,欲胜无从呢?她看了一眼李显,只见他一扫方才的阴郁冷漠,眉宇间神采飞扬,笑容中带着睥睨天下的豪气,不由心想,和这样的人沙场血战,果然称得上是人生一大快事。想到这里,林碧心中也是豪气陡生,高声道:“拿酒来。”

林碧的两个侍卫闻言连忙拿了两个酒囊过来,林碧自己拿了一个,用目示意李显,李显了然,便也接过了一个酒囊。林碧笑道:“这里面是我北汉最好的烈酒,我们代州人有个习俗,若是见了最好的朋友或者最可敬的敌人,便要请他共饮美酒,若是朋友,从此就要肝胆相照,若是敌人,将来生死相见也不要彼此仇恨。王爷如此豪气干云,若是庭飞在此,必定要请王爷共饮的,碧虽女流,自觉不让须眉,就请王爷共饮烈酒,将来沙场相见,死也无恨。”

李显目光炯炯,半晌才道:“公主果然是巾帼奇女子,龙兄果然是好福气,好,这酒我喝了。”说罢,李显拔出酒囊的塞子,大口的喝了起来,这酒囊可以装得下半斤烈酒,李显仗着酒量大和内力深厚,一口气喝得干干净净,烈酒入腹,李显只觉得有些头重脚轻,却仍然倒过酒囊,示意已经涓滴不存。

林碧见了,微微一笑,举起酒囊也是一饮而尽,面上却只是略现嫣红罢了。她朗声吟道:“陌路相逢成知己,他年沙场见此心。”吟罢再不言语,转身走入船舱。

李显心中一震,觉得林碧这两句诗光明磊落,却又是意味深长,吟诵再三,只觉得心驰神往,更是盼着生死相见之际的重逢了。

这时,李显身后传来侍卫的呵斥声,然后一个清雅的声音说道:“海骊求见齐王殿下。”

李显没有回头,淡淡道:“让他过来。”

海骊走到齐王身后,恭敬地道:“草民海骊,在公子座下称作盗骊,给殿下请安。”

李显回头看了海骊一眼,道:“不必拘礼,怎么随云改变主意提前见我了么?”

盗骊答道:“公子传言,殿下既然来了东海,还是去见见东海侯的好,这次东海侯的喜事只怕不会顺顺当当的,殿下不要错过才好。”

李显笑道:“随云总是这般诡秘,罢了,能够这么容易就见到他,我已经很知足了,不过既然婚宴上会有事情发生,两个小孩子去是不是太危险了。”

盗骊说道:“殿下放心,公子已经有了安排,这次是最好的机会,让东海侯向大雍称臣,双方都有台阶下,而且公子说,如今已经是万事俱备,应该收网了,滨州原本是北汉对外的唯一通路,只要封闭此处,那么殿下就可以完成攻占北汉的功业了,这样的机会殿下不可错过。”

李显若有所思地道:“怎么,随云也觉得时机成熟了么,可是如今可是北汉正是最兴盛的时候啊?”刚说到这里,他看到了盗骊有些尴尬的神情,失笑道:“我倒忘记了,这里可不是军营,好了,你转告随云一声,我是服气了,想来皇兄的书信早就到了东海吧。”

又看了盗骊一眼,齐王道:“随云也是,你这样人才,不去搏个封妻荫子,却做什么商人,这又是何苦来呢?你若有心,我向随云提出来,让你去做官不好么?”

盗骊愣了一下,道:“殿下厚爱,草民铭感五内,只是草民如今虽然是白身,但是带着商船万里迢迢的行走异国他乡,觉得比什么都有乐趣,有没有官职倒也没有什么关系了,而且草民跟着公子,也就是为大雍效力,倒也不用去特意做官。”

李显听了心中一宽,只听这盗骊的口气,就知道江哲没有打着旁观的念头,看来这几年他虽然隐居不出,却是做了不少准备,那么请他出山调停应该是没有问题了,想到纠缠自己数年的苦恼可以烟消云散,李显也不由喜笑颜开。

这时,远处传来小柔蓝清婉动人的歌声道:“执手碧波上,极目海天明。心与孤帆远,身如一棹轻。浪花分日影,珊岛咽湍声。漠漠平烟外,翛然白鹭横。”

李显听了只觉心旷神怡,心道,柔蓝所唱,必是江哲新词,执手碧波,极目海天,想来长乐与他定然是绸缪情深,乐事无穷了。抬目望去,只见碧波如镜,白云悠悠,海天一色,心中也不由平静下来,他不怕沙场血战,却是恨透了朝野纷争,如今大雍上下流言纷飞,大半都是冲着自己来的,不是说自己要领兵造反,就是说皇帝要秋后算帐,虽然自己心中明白,就是李贽想要鸟尽弓藏,也不会赶在这个时候。可是这种流言,他李显可以不信,长安城里面的李贽可以不信,那些朝野重臣,军中的猛将却是半信半疑,令得军心浮动,后勤不稳,若是再这样下去,可就要被龙庭飞所乘了。这次他得知东海侯爱子大婚的消息之后,突发奇想,江哲隐居东海,乃是他和李贽都心知肚明的事情,虽然没有实信,可是隐隐约约还是可以肯定的。想来此人隐居了将近三年,也该偷懒够了,这个时候他若不出来相助,岂不是太无情了,不管怎么说,他如今可是李家的女婿,总不能眼看着兄弟閲墙,渔翁得利吧。

这时,远处传来了一个小男孩磕磕巴巴的歌声,想必是柔蓝逼着麟儿唱曲吧,可是只听了两句,李显就是心中一阵剧痛,脸色也变得青白起来。

“飞来双白鸽,乃从东南飞。十十将五五,罗列难成行。突然卒疲病,不能飞相随。五里一反顾,六里一徘徊。吾欲衔汝去,口噤不能开。吾欲负汝去,毛羽何摧颓。乐哉新相知,忧来相别离。躇踌顾群侣,泪落纵横垂。关关幽相远,哀哀鸣相啼,殷心伤泣血,泪目与诀别。见汝西北堕,吾何东南去。念卿旧日恩,幽恨不能语。”

那凄楚的歌声让李显几乎要疯狂了,那镇守边关的凄凉军帐,明月下泪尽时的悲歌,泪水刚要滴落,李显突然省悟,他走向后面的船舱。只见李麟唱着曲子,面上带着绝望和哀伤的神色,柔蓝正惊恐的看着他。

李显还没有走过去,柔蓝已经捂住了李麟的嘴道:“我不逼你唱曲子了,你唱得这样难过。”

李显心中一震,李麟小小年纪懂得什么,分明是看了自己平日情态才会这样模仿,强烈的悔恨从心中涌起,自己只想着将他带在身边,免得有心人谋害欺凌,却没有想到自己的悲苦全被这个孩子看在眼里,而自己平日忙于军务,为了保护这个孩子,又不免对他冷淡一些,而且,说句心里话,他也不知道应该如何照顾一个小孩子,想来这两年多来,苦的不仅仅是自己,最凄苦无助的就是这个失去了母亲,却得不到父爱滋润的麟儿。

这时李麟已经看到父亲,他不由缩到柔蓝身后,父亲对他来说是一个冷冰冰的暴君,而这个明明比自己还要矮小的小女孩,那软软小小的娇躯,那香香的气息,却让李麟觉得仿佛回到了那曾经有过的童年,母亲的怀抱一般。

李显大步上前,抱起李麟,和颜悦色地道:“麟儿不用害怕,都是爹爹不好,这次爹爹带你去见姑姑,你想不想留在姑姑身边。”

李麟眼中闪过一丝慌乱,道:“爹爹不要赶走麟儿。”他紧紧地攥住李显的衣衫,越发不肯松手。

李显笑道:“你这傻孩子,爹爹忙着打仗,没有时间照顾你,你的姑姑慈悲和蔼,一定待你如同亲生,而且还有一个小姐姐可以跟你玩呢。”

李麟疑惑的目光看向柔蓝,李显笑道:“聪明,不错,你以后便叫她蓝姐姐吧。”

李麟脸上露出罕见的灿烂笑容,李显心中一痛,更是紧紧的抱住了爱子。

刚走出舱门,林彤就看到远处怔怔站着的赤骥,她心中一痛,方才的事情他都已经知道了,这人的身份已经昭然若揭,就是自己想装作不知道也不可能了。她径直向外走去,好像没有看见赤骥一般。赤骥突然伸手拉住她的手臂。林彤脸色一寒,道:“你要做什么?”她的声音并不大,免得惊动旁人。

赤骥歉然道:“我不是有心欺瞒你的。”

林彤冷冷道:“你欺瞒了我什么,伯乐神医!”她的语气充满了愤懑和感伤。

赤骥沉默了片刻,道:“我没有说过几句谎言,只是没有说过我的恩主就是江哲江随云,而且答应龙将军为北汉效力也是权宜之计,我并没有想留在北汉刺探军情的意思。”

林彤漠然道:“我知道了,这件事情你没有什么错,两国交兵,各为其主罢了。”

赤骥被她冰寒的目光刺痛,不由松开了手,明明觉得自己没有做过什么过分的事情,却还是觉得愧疚涌上心头。

林彤走了几步,停住脚步道:“你没有欠我什么,是我脾气不好,迁怒于你,王骥,你以后会跟着主子攻打我们北汉么?”

赤骥愣了一下,斩钉截铁地道:“不会。”

林彤愣了一下,道:“你应该很适合做斥候的,而且你对北汉也很熟悉吧?”

赤骥低声道:“公子从来不会逼迫我们做任何事情,天下大的很,我自己还可以去做别的事情,而且,而且,我不想在沙场上见到你。”

林彤笑了,虽然赤骥看不到她的笑容,可是从她起伏的肩头可以看出她笑得很厉害,只是笑声中带着浓浓的悲凉,过了一会儿,林彤止住笑声,道:“你太懦弱了,像我姐姐和齐王李显那样多好,虽然惺惺相惜,可是仍然相约沙场相见,生死无恨,生死无恨,你若是也去和我们交战,我就在战场上杀死你,到时候我自然是不会恨你,你就是恨我又有什么关系呢?没有血性的匹夫,我林彤是绝对不会对你这样的懦夫手下留情的。”

赤骥没有说话,经过良好的谍探训练的他看得出来,林彤紧握的双拳,和她周身上下的紧崩代表着什么。可是他没有上前安慰她,因为他知道横在两人之间的是多么深的鸿沟,与其沉湎于美梦,不如就这样断绝情感的纠缠。这个美丽的如同火焰的少女,将会是他深藏心底的秘密。

他默默的向外走去,就在舱门将要关上的一刻,他听到了呜呜咽咽的哭泣声。可是他强忍着没有回头,也许他不留恋南楚,不留恋大雍,可是那个深沉如海,率性如风的身影,却是他永远也不能违逆背叛的主人。

在东海蓬莱岛的一隅,临海背山的一个小港湾内,建有一座清雅宜人的小庄园,名为静海山庄,山庄占地虽广,其中楼阁亭台却是寥寥无几,参差掩映在绿树丛中,宛如仙境。在半山腰的一座小巧红楼之内,一个青衣秀士正在临帖,雪白的宣纸上面留下了行云流水一般的字迹,这时,身后传来一个温婉中略带担忧的声音道:“蓝儿年纪还小,你也放心她去那种地方,你这作爹的不心疼,我这个娘亲还心疼呢?”

青衣秀士放下笔,满意的看看自己完成的字帖,笑道:“所谓慈母多败儿,此言不假,这件事情你就不要管了,难道我会不派人护着蓝儿么?”

珠帘轻动,一个娉婷多姿的月白身影从里间走出,娇嗔道:“你总是喜欢这样装神弄鬼,罢了,我也不和你争,若是蓝儿受了什么伤害,我可不饶你。”

青衣秀士放声大笑,伸手将那白衣女子揽入怀中,笑道:“好好,若是蓝儿受了什么伤害,我任你处置就是。”他这一抬头,露出了清秀儒雅的面容,这人年纪有些难以辨别,若单论相貌,大概只有二、三十岁的年纪,可是他的头发却是浅灰色,虽然光泽仍然不减少年,却是始终带了几许岁月的留痕,两鬓更是已经星霜点点,若是有人因此说他是四五十岁年纪,也未尝不可,而他的神情气度,宛若深山的潭水一般淡泊幽深,就是说他已经六七十岁,到了看穿世间冷暖的年纪,也不会有人怀疑。

那白衣女子看见他的面容,不由柔柔的叹息了一声,柔顺地依偎在他怀中不再说话。这时,身后突然传来婴儿的啼哭声,两人相视一笑,携手向内间走去。

\chapter{第七章 兄弟相见}

海骊,海氏船行二代家主,海仲英侄,年未弱冠,随仲英赴南海诸洲,后仲英无暇,骊自领商船下南行西下,海氏雄起,骊有力焉。骊擅工笔,亲绘海图十二幅,精确无疑,今犹用也。

大雍隆盛十七年,太宗以骊弘扬国威于海外,赐侯爵位,海骊虽进爵,行不稍改,年七十仍远渡重洋。大雍文宗昭宁十五年,骊于舱中小憩,忽梦故人,起而笑曰,吾当死也,乃焚香鼓琴,曲未终而殁,终年七十一岁。

骊为人,外虽亲切,内实疏冷,然信义为本,仲英死,数子尤在冲龄,人皆言骊必夺产矣,骊教诸弟如子,后十五年,择其佳者为嗣,人乃知其节。

骊喜读经,为居士,不婚不嗣,人皆异之。

——《雍史·货殖列传》

当赤骥茫然若失的走进自己的住处之后,却看见盗骊静静的望着自己。盗骊淡淡道:“一个小女孩而已,你怎会放在心上,很快你就会忘记她,她也会忘记你。”

赤骥心中一痛,道:“我也不知道为什么,本来我只当她是个麻烦的小妹妹,可是前日我见她从嘉平公主房间走出来的时候,她变得那样眩目,那样艳丽,我却忍不住心痛,凤凰浴火,虽然绝丽,可是那切肤之痛,却是何等难以忍受,那一刻,我才明白,一路上,我对她敷衍,甚至觉得她骄纵刁蛮,都是因为我知道终究会有分道扬镳的那一天,所以才不肯去喜欢她。我真的不想伤害她,可是如今她还是受了重伤,我却无能为力。盗骊,你不会明白的。”

盗骊漠然道:“不,我明白的很,当日我替公子办事,曾经留在一个小帮派里面,我也认识了一个天真善良的小姑娘,她喜欢上了我,我也对她动了心,可是最后我还是亲手杀了她的父兄。”

赤骥心中一动,记起盗骊曾经去做过一件大事,回来之后,数日不言不语,仿佛死去一般,当日他也曾去劝解,却觉得盗骊眼中全无生机,直到有一日公子秘密召见了盗骊之后,他才恢复了神采,而那之后,盗骊就被派到了东海。

他犹豫地问道:“那位姑娘,她,她也死了么?”

盗骊眼中闪过一丝不可遏制的悲伤,道:“当日我也想过,放过她一条生路,让她躲到穷乡僻壤去,就不会影响公子的大计,可是我清楚的很,如果她活着,那么很有可能会落到别人手上,成了别人对付我们的利器,而且她眼见我杀死她的父兄,这样的深仇大恨,我不知道她会作些什么。所以我亲手杀了她,我本是带着恶意而来,从一开始就知道这样的结局,可是我还是沦陷在她的绵绵情意当中,这是我的错误,所以我必须亲手结束这个错误。你也一样,只要你亲手杀了她,就可以消去心中的毒瘤,所以你一定要去北汉,否则你的一生都不会快乐。”

赤骥沉默片刻,道:“我明白你的意思,你亲手杀死爱人就是为了不想怨恨公子和同生共死的伙伴。你说得不错,她就和她的姐姐一样,都是女中英杰,她陨落之时,也一定像极了流星,在最灿烂的一刻死去,若是不能亲眼见到,我这一生都会懊悔。我会请求公子,从军征北,不过我不会让她知道我也在战场之上,这种苦痛我一人承受就可以了。”

盗骊淡淡道:“你明白了就好,如今你的身份已经暴露,明日你就跟在齐王殿下身边的,公子有些事情交代。”说罢递给他一个蜡丸。赤骥接过蜡丸,打开之后看过里面绵纸上面的指令,然后将它用火折子烧掉了,灰烬飘落在地上,赤骥露出了坚定的笑容。

当李显、林碧等人搭乘的客船到达东海侯的大营,一个无名小岛的时候,站在船首的两人都是眼中一亮。远远望去,这座小岛如同环抱的双臂一般,两侧都是峭壁林立,光滑的礁石根本无法攀登,没有可以遮掩的树木,让上面巡视的人可以一眼看见敌人。而小岛正中却是一个优良的海港,可以让大型的船只进去躲避风雨。东海侯乃是海上的霸主,前来祝贺小侯爷新婚的除了各大势力的使者之外,就是依靠海运为生的商人和劫掠海船的海盗。所以港口之内泾渭分明,各种势力之间彼此都十分戒备。而东海侯所属的战船将小岛周围围得水泄不通,这样的龙潭虎穴,就是京无极和慈真大师到了也难以为所欲为。

码头上站着几十个披红挂彩的大汉充任迎宾使者,一身大红喜服的小侯爷站在最前面,英姿勃发,喜气洋洋,病魔离体之后的姜海涛这两年在东海纵横无敌,不知歼灭招降了多少海盗,从前东海侯只是海上最大的势力,如今却已经成了所有海盗的司令人,能有这样的成绩,姜海涛功劳卓著,不仅姜永老怀堪慰,就是远在大雍的太上皇也曾为此大喜过望,这两年闲居下来,李援也很后悔当日对姐夫太不留余地了。

望见船头的倩影,姜海涛高声道:“姜海涛奉父命迎接北汉使者,嘉平公主殿下。”

林碧淡淡一笑,扬声道:“小侯爷不必多礼。”

说罢顺着跳板走到码头岸上,双方见礼之后,姜海涛的目光落到了随后下船的李显身后,脸上露出不可置信的喜色,喊道:“六叔。”雀跃地扑上前,抓住李显的手臂大笑道:“六叔来参加侄儿的婚礼,怎么不事先通知一声。”

李显也是微微一笑,道:“我是私下里来的,皇上可是不知道的,你别瞎嚷嚷。”

姜海涛激动地道:“六叔援手之恩,小侄铭感五内,今日六叔能够前来观礼,父亲一定是喜出望外。六叔,快去见见父亲。”

李显笑道:“也好,我和表兄多年不见,也应该先叙叙旧情。这是麟儿,我的儿子,你不认得吧?”

姜海涛看见李麟,心中一动,他也知道一些现在李显的情况,这个孩子一定是秦铮所生的,不过他是心胸宽广的人,这个孩子的母亲既然已经死了,他也不会再斤斤计较,便道:“原来是表弟,就让他到后面去见见我母亲吧。”

这时候一个娇嫩的声音不满地道:“蓝蓝也在这里呢。”

姜海涛这才发现站在李麟身边的还有一个小女孩,一看之下更是喜出望外,上前抱起柔蓝道:“蓝儿也来了,那么先生也来了么,父亲几次下帖子,先生都说不能来的。”

柔蓝得意地道:“我跟海叔来的,爹爹答应的。”

姜海涛眼中闪过失望的神色,他向齐王后面的海无涯和海骊打了一个招呼,放下柔蓝,引领着诸位贵客向远处的喜堂走去。这座岛屿是东海侯近年来常驻之处,从码头向上有着重重楼宇,其中半山处最是宽阔壮丽的大殿就是往日的议事厅,今日的喜堂。大殿两侧的偏殿里面都摆了上百桌酒宴,招待普通的客人,而当中的大殿之内,除了中间铺着红毡的花烛喜堂之外,两边也各自摆着十八桌酒席,招待贵宾。东海侯夫人据说体弱多病,今次没有出席,只有东海侯带着属下将领心腹,在大殿中喜笑颜开的招待宾客。人逢喜事精神爽,已经四十五岁的东海侯神采飞扬,还没有开宴,就已经连饮数杯。

这殿中客人,若论尊贵,自然是要数大雍和南楚的使者了。

庆亲王李康今年三十七岁,自从凤仪门覆亡之后,他的身份地位立刻上升了许多,论身份,他是李援第三个儿子,如今长子李安因为谋逆而赐死,次子李贽已经做了皇帝,若论身份贵重,庆亲王仅在父兄之下,而其他的几个还在世的年长皇子,五皇子宁郡王李祺自幼体弱多病,既不得李援宠爱,又不曾涉足军政,直到李贽登基之后才封他做了一个郡王,齐王虽然得到赦免,并且重领兵权,可是因为曾经涉嫌谋逆,爵位也由亲王降到了郡王,齐王之下的皇子公主都还没有成年,而李康却在这个时候因为守川有功,由郡王晋升亲王,此消彼长,掌握着益州军政大权的庆亲王就成了朝中一人之下万人之上的人物。这次奉了皇命出使东海贺喜,李康倒也很高兴,他和东海侯姜永早就暗中有所联络,若是能够趁机劝服姜永归顺大雍,可是天大的功劳啊。所以坐在首席的李康言笑宴宴,风趣热情,这位英姿勃发,如日中天的亲王这样平易近人,使得一桌子的客人都是如沐春风一般。

南楚的使者陆灿却是另一种模样,虽然年仅二十五岁,却已经是南楚大都督的陆灿神色从容淡漠,令人全然看不出他的心思,事实上,虽然说大雍派了庆亲王李康这样位高权重的使节,可是南楚派了陆灿过来仍然是件奇怪的事情。这几年,陆灿一边抵御着来自益州的侵扰,一边加强襄樊、长江防线,可以说是日理万机,作为大将军的陆灿,可以说是南楚武将第一人,这样的重要人物离开中枢,远赴东海,实在令人匪夷所思。不由令人怀疑南楚的政局出了什么变化。虽然陆灿神情冷静,没有流露出任何可以猜测的迹象,可是只看他旁边的副使伏玉伦全无顾忌,恣意谈笑的模样,就让人心中生出了各种遐想。谁不知道这个伏玉伦是南楚丞相尚维钧的女婿,南楚国主赵陇的姨夫呢,难道是南楚的两个顾命重臣,尚维钧和陆信之间发生了争端,陆灿出使东海是否是因为收到排挤?当今天下,战乱纷呈,谁不想多了解一些局势,免得收到连累呢。

正在堂上宾客谈笑的时候,负责迎宾的知客高声呼道:“嘉平公主、红霞郡主到。”

众人抬眼望去,恰好看见一个翠衣女子走了进来,为了参加喜筵,今日林碧并没有穿着平日为了方便领军作战而穿的胡服骑装,而是换上了符合身份的盛装,浅绿色的绣襦配上湖水绿的长裙,金碧色的外衫昭示着北汉公主的尊贵地位,腰间系着明珠宝刀,足上的鹿皮靴则提醒着众人这位公主的另外一个身份,北汉代州军的实际领军人。

堂上众人都起身相迎,就是敌国身份的庆王和礼部侍郎苟廉也不例外,不论是敌是友,这位领军抵抗蛮人,保护黎民乡梓的女将军,都是值得尊重的人。

林碧含笑和众人见礼,这时,一个清脆悦耳的声音响起道:“姐姐,那位是陆灿陆大将军啊?”

这时,众人才注意到林碧身后站着一个身穿红衣的少女,娇俏动人,明艳如火,只是众人方才都被林碧的风采所震慑,竞没有留心这个红衣少女亦步亦趋的跟着林碧,而且形迹亲热,不似侍女身份。此时听她说话,才想起方才知客通报的乃是两人。

陆灿听见那少女询问,淡淡一笑,多年的军旅生涯,这个昔日无法无天的淘气少年已经变成了沉默寡言的大将,他目光落到林碧身上,林碧也适时回以歉意的笑容,说道:“舍妹顽皮,还请大将军见谅。”

陆灿欠身道:“公主言重。”

这时红衣少女林彤好奇地道:“原来你就是陆灿,我听说你打仗很是厉害,让大雍铁骑不敢南窥,人家都说,北龙南陆,雍人见之而胆寒,想不到你还这样年轻。”

陆灿看了一眼庆王李康变得铁青的脸色,淡然道:“郡主谬赞了,龙大将军带甲二十万,压制大雍五十万边军,确是当世第一用兵大家,大雍和我南楚乃是友邦,并无战事,郡主的赞誉陆某可不敢当。”

陆灿这可是睁着眼睛说白话了,这几年来虽然南楚无力进攻大雍,大雍也无暇南顾,可是两国之间没少了小规模的战争,陆灿用兵如神,没有让大雍讨到半点好处,故而才有人将他和龙庭飞并称大雍的两大克星。可是毕竟名义上两国还是宗主国和藩属国的关系,两国又没有公开决裂,陆灿是绝对不会承认林彤的话语的。果然他这样一说,庆王的面色好转了许多。

林彤不满的嘟囔了几句,就在林碧警告的眼神中闭上了嘴,乖乖的跟着姐姐坐到席上,这一席已经坐了南楚和大雍的使者,加上林碧两人,还是空着许多位子,不过平常人可不会想坐到这一席上,当今天下三分,这三大势力的使者岂是可以攀比的。

林彤望了一眼庆王,恶意地道:“喂,你就是大雍的使者庆王么?”

李康冷冷看了林彤一眼,他可不想和这个小女孩争执,那样也未免有失身份。因此只是冷冷道:“正是。”

林彤笑道:“看你还算神气,可是比起齐王殿下真是差的很远,怪不得人家领着五十万大军镇守边关,你只能守着东川坐井观天。”

李康这下可是大为恼怒,叱道:“嘉平公主,请好好管教令妹。”坐在他旁边的苟廉却一皱眉,这个小女孩对齐王很熟悉么,按理说她不应该有机会见到齐王才对,虽然齐王正在和北汉对峙,可是王见王的机会应该很少会有的。

苟廉心中刚刚起了疑窦,知客已经高喊道:“大雍齐王殿下到。”

立刻满堂哗然,谁也不会想到齐王竟会到了此处,不说这次大雍的使者乃是庆王李康,大雍朝廷断断不会派了两个王爷前来,就说齐王身负重任,理应在军中镇守,就不该出现在此时此地。可是众人还在怀疑自己是否听错了,齐王冷峻的身影已经出现在门口,冰冷残忍的目光环视了堂内一周,顿时鸦雀无声。这样的威仪气魄,众人立刻相信,真的是齐王莅临东海。

虽然几乎是所有的人都避开了齐王凌人的目光,却有几个人不会畏惧齐王的威严。陆灿是其中之一,他听到齐王亲临之后,先是有些惊讶,然后又恢复了平静,只是淡淡瞧着齐王,眼神中透出评估和赞赏。

而庆王李康却是神色冰寒,他对齐王可是十分不满,从前齐王党附太子,对庆王从没看在眼里,而且他的王妃就是凤仪门弟子,这些已经足以让庆王恨之入骨了。可是更令李康痛恨的却是,这个桀骜不逊的六弟李显,即使在如今的情形下,也从来对自己低头。按照身份,自己是亲王,李显是郡王,自己是朝中红人,首屈一指的显贵,李显却至今带着谋逆嫌疑,可是就是这样,李显也从没将李康看在眼里。今年李贽登基,李康和李显都回去参加大典,李康本来想凭着兄长和亲王的身份和李显结好,谁知李显却连看也不看他一眼,更别说对他有所尊重。李康曾经因此秘密向李贽进谏,说李显太过桀骜不逊,可是李贽居然只是苦笑道:“六弟在朕面前也是如此,他就是这样的性子,三弟还是不要得罪他吧。”这一句话让李康立刻明白了,除非李显死掉,否则他绝对占不到李显的上风。看着那些军方将领和朝中重臣对李显必恭必敬,对着自己却是疏离淡漠。李康对李显的恨意越发深重。凭什么,这样一个大逆不道的狂妄之辈,可以理所当然的压在自己头上,这是李康埋藏在心中最深的怨恨。

不过当着这么多外人的面,李康自然不会表露出这样的恨意,就是在朝中他也只是微微流露一些不满罢了。他再次站起身来,强颜笑道:“六弟也来了,可是奉了皇上的旨意么?”

李贽看了一眼李康,冷冷道:“我是以私人身份来贺喜的,要什么旨意,三哥若有疑问,回去问皇上吧。”

他这般不讲情面,李康面色一寒,几乎就要当场发作,苟廉连忙打圆场道:“庆王爷不用担心,齐王爷也是亲戚情深,想来皇上也不会怪责的。”他这样一说,倒真的像是庆王兄弟情重,担忧齐王私自离军惹恼皇上一般。

李显看了苟廉一眼,倒是很给他面子,道:“三哥不用担心,回去我就给皇上写谢罪折子。”

说着露出了一个懒洋洋的笑容,然后大马金刀的坐了下来,这时候堂上众人才松了一口气,不由惊叹这齐王身上的煞气之重,真是天下罕见,同样是带兵的大将,陆灿一派神闲气静,儒将风范,嘉平公主则是令人倾慕的雍容沉稳,而齐王却是带着深重的杀伐之气。见这三人坐在一起,人人见到这般人物,真是不需此行。

虽然众人已经松懈下来,可是却还是觉得压抑,有齐王一人在此,满座之上,无人可以宽心饮宴,众人正觉得尴尬的时候,一个豪放的大笑声从后堂传来道:“怎么,六弟也来了么?”

众人一听,便知道是东海侯姜永到了,这东海除了庆王李康之后,就只有东海侯姜永可以这样称呼齐王李显了。果然从后堂走出一个身穿大红袍服的中年人,半百年纪,须发灰白,神情矍铄,肤色微黑,他行走起来仿佛带着风一般,身后的几个侍卫几乎都跟不上他的步伐。他走到席前,一把拉起李显道:“好六弟,你表哥可是盼着你来呢,若非你仗义,你那个侄儿别说娶妻,就是性命怕还保不住呢。来来,这次定要你的侄儿侄媳妇好好谢你的大恩,没有什么可以说的,别看你们大雍的使臣在我这里吃不开,你可是不一样,除了要我归降之外,只要你六弟有什么要求,尽管说出来,我姜永绝对不会给你打折扣。”

他这一番话可是吓坏了很多人,就连陆灿和林碧眼中也闪过忧色,若是齐王提出东海不能再和南楚北汉合作,这可如何是好。

还没有等到齐王回答,外面的知客不合时宜地道:“海氏船行,海无涯、海骊到!柔蓝小姐到!李麟少爷到。”随着声音,海氏叔侄含笑走进,而在他们身后,一个蹦蹦跳跳的小女孩扯着一个不情不愿的小男孩走了进来。除了林碧和李显等人之外,其他人又都一愣,这是怎么回事,什么时候知客会连小孩子都通报起来了。

小柔蓝滴溜溜的眼睛看着那些目瞪口呆的客人,不满地道:“你们瞪着蓝蓝做什么,麟弟,这些人好没有礼貌,帮我教训他们。”

李麟郁闷的看了看那些客人,冷冷道:“你是白痴么,你看我可以打得过谁?”

小柔蓝认真的看了一看,有些苦恼地道:“是有点困难啊,他们都比你高好多,如果骏哥哥在就好了,一定可以替我出气的。”

李麟不满地道:“你的骏哥哥好像也没有多大,我可不信他能替你出气,这样吧,你等一等,等我长大做了将军,就可以替你出气了。”

小柔蓝噤噤鼻子,嘟囔道:“骏哥哥就是很厉害么,爹爹欺负我的时候,他都会帮蓝蓝告状。”然后小柔蓝缓缓低下头,声音中开始带了哭音道:“呜呜,蓝蓝很久没见过骏哥哥了,爹爹都不许我给骏哥哥写信。”抬起头满怀憧憬地望着李麟,道:“你可以替我带信的,对不对?”

李麟气结,看着众人疑惑中带着好笑的神情,恶狠狠地道:“好了,我答应了,还不行么。”他的神情变得更加郁闷,方才柔蓝求了半天,他不好意思说自己不会去长安,没法子带信,只能铁了心肠不肯答应,没想到柔蓝却选了这个时候逼他答应,不喜欢别人瞩目的李麟只能答应下来,心里盘算着是否能让军中的信使帮忙带回去。

这两个小孩子这样一闹,众人的思绪都被引开了,不知道是谁先笑了起来,然后众人都开始开怀大笑,喜堂上气氛开始热烈起来。

李麟满面羞红,狠狠的看了柔蓝一样,柔蓝却是得意洋洋地上前扯着姜永的袍子,道:“姜伯伯,蓝蓝替爹爹来贺喜呢。”

姜永有些哭笑不得,道:“好,好,伯伯知道了,小蓝儿,要不要到后面去看看你的新嫂子。”

柔蓝连忙点头,姜永一挥手,两个站在边上的侍女连忙过来,领着小柔蓝向后堂走去,李麟皱皱眉,抬头看向父亲,李显轻轻点头,李麟便跟在柔蓝后面走了进去,众人只当他和柔蓝是一起的,全没留心,就这样让他跟了进去。

\chapter{第八章 南闽越氏}

南闽越氏,海运世家,历久不衰,海氏后起之秀,与越氏有旧怨,终不能解。

——《雍史·货殖列传》

静海山庄之内,红楼之上,我望着桌上的山川地理图,微笑道:“南闽越氏乃是天下海运第一家,已经传承数代,历久不衰,家族之中不仅能人辈出,而且姻亲遍及天下,自从东晋崩溃之后,越氏趁机掌控了南闽军政大权,在南楚立国之后,南闽仍然独树一帜,南楚迫于大雍的压力,根本就没有余力平定南闽,所以越家是实际上的一方诸侯,名义上南闽虽然是南楚的臣属,可是实际上就像滨州一样,并不受南楚的控制。不过越家也不会太过分,毕竟若是南楚铁了心,越家虽然可以通过向大雍臣服换取支持,但是短期之内就要退到海上了,那么越家在南闽的产业就会受到重大的损失,所以对于越家来说,最好天下就是这样四分五裂下去,他们才可以有更大的利益。”

原本坐在旁边的软榻上专心刺绣的长乐公主抬起头,若有所思地道:“当初表哥在东海蛰伏,越家主动支持表哥,又和表哥联姻,想来就是打着让表哥牵制大雍的主意了。”

我悠然道:“不错,越家虽然蛰居南海,没有逐鹿中原的本事,可是割据的野心确实有的,‘满堂花醉三千客,一剑光寒十二州’,这就是形容越家声威的名句,这十二州指得是福州、建州、泉州、漳州、汀州、南剑州、邵武、兴化和粤东的梅州、揭阳以及南澳,虽然南澳还称不上一州之地,可是此地素有闽粤咽喉之誉,商船云集,繁华更胜滨州,所以才说‘十二州’。虽然越家实际上只掌控了漳州、泉州、揭阳、南澳,但是这里乃是粤东南闽的精华之地,背山面海,南楚无能为力,大雍也是鞭长莫及。越家虽然低调,不曾争夺过霸权,也没有称王称霸,可是只从‘满堂花醉三千客’这一句就可以知道越家门客如云的盛况。想要维持这样的地位,除了向强者称臣之外,就是让乱世无休无止下去才有可能。这次姜、越两家再次联姻,就是越家主动的。”

长乐公主微微蹙眉道:“这越家如此用心,真是可恨,天下百姓的疾苦在他们眼中大概无关紧要吧。随云,既然如此,你为什么眼看着这桩婚事成功呢,这样一来,岂不是如了他们的心愿。而且,如今海氏在你的支持下从事海运,滨州已经成了仅次于南闽泉州的海港,而表哥的武力支持更加重要,如今越家恐怕也在打远洋贸易的主意,若是他们掌握了海氏造船的机密技术,岂不是如虎添翼,就是从这一方面看也不能让他们成功的和姜家联姻啊?”

我把玩着手中的碧玉镇纸,淡淡道:“越家虽然用心不好,可是让他们介入远洋贸易倒也是我的意愿,这世间之事就是如此,除了皇位只能一人独占之外,其他不管是什么,最好不要想方设法的一人独占,如今远洋贸易被海家独占,不知有多少人眼红呢,如今天下还没有一统,倒也罢了,等到天下一统,四海升平之后,只怕第一个想对付海家的就是天子。就算是看在我的面子上暂时不动海家,等到我百年之后,海家也是灭门可期。既然如此,还不如让越家来分一杯羹,这样一来,虽然也会有人想打击压制,可是只要本事够,就可以支持下去。”

长乐公主听到“第一个想对付海家的就是天子”这句话的时候,手一抖,绣花针已经刺伤了手指,听到后来却是平静下来,道:“这也说得是,皇兄虽然英明,可是这种事情也很难装作看不见的,夫君既然有此打算,姜越联姻之事,倒也不用挂在心上,只是越家本已是如此势大,又是倾向南楚,不肯臣服大雍,夫君如今就让他们插手远洋贸易,岂不是更加助长了他们的气焰?”

我意味深长地道:“哪有这样的好事,越家虽然可以参与进来,却不是现在,若是不将越家削弱,别说我不会放心,就是海兄也会不安的。我准备先给越家一个沉重的打击,再给他们机会参与远洋海运。”

长乐公主忧心地道:“可是越家既然是南海的霸主,夫君如何能够给他们太大的打击,毕竟现在南闽还是南楚的领土,若是激怒了越家,他们转而完全支持南楚,岂不是更加麻烦?”

我摇头道:“凡事都是盛极而衰,越氏如今已经传承十几代了,早已是隐忧重重,尤其令人诟病的是,越家做生意的手段太霸道了,对于生意上的对手常常是用尽手段打压,顺我者昌,逆我者亡,粤东南闽的商人都要仰其鼻息,仲英就曾经提过,当年他在粤东得罪了越家的一位执事,结果在出海之时遇到海匪,家业尽毁,后来仲英就怀疑过这件事是越氏所为,虽然没有证据,可是越氏和海匪之间素有往来,而且事后仲英原本可以将生意继续做下去的,那些债主原本并不想逼他还债,倒是希望他能够经营下去,好还上那些巨债的,也是越家从中作梗,最后仲英散尽家财,也还欠着很多债务,南闽又无法容身,才辗转到了大雍。说来也是很巧,无计掌管天机阁商务,看中了仲英的才干,便支持他东山再起,后来盗骊发觉他和海仲英乃是叔侄,我又隐居东海,才鼎力支持海氏,姜侯也对越家很是不满,这才形成了今日海氏后来居上的形势。越氏这样的行事作风,自然是树敌极多,平时还看不出来,若是到了关键时候就是群起而攻之的局面,而且越家内部也是隐忧重重。越氏家主之争如今已经是如火如荼,正是打击越氏的最好机会。”

长乐公主叹了口气道:“皇室夺嫡,固然是血腥重重,世家大族,家主之争,也是你死我活的惨事。”

我柔声道:“贞儿,你又想起猎宫之事了么?”

长乐公主眼中闪过一丝悲怆,说道:“这件事情我如何能够忘记,大哥谋逆赐死,六嫂自尽谢罪,皇后娘娘也是自尽身亡,这样的惨事贞儿真是不想回忆起来。”

我走到长乐公主身边,轻轻将她揽入怀中,道:“你也不要多想了,这也是他们罪有应得,而且,你我定情,也是缘于猎宫之变,不为别的,就为这个,你也不该如此伤情。”

长乐公主不由面上一红,虽然已经结缡近三年,想起当日猎宫之时,自己情不自禁当众失态,仍然是心中羞不可抑。我见她已经不再悲伤,这才道:“既然你不喜欢听越家那些家事,我也就不提了,这个时候,慎儿应该醒了,你去看一下吧,我还要看些文书呢,就不过去了。”

长乐公主收起绣品,埋怨道:“你这两年说是离开了朝廷,安心休养,却总是放不下这些事情,早知如此,还不如不离开呢,就连头发都变了灰色,你这又是何苦呢!”

我不由苦笑道:“贞儿,早就跟你说过了,我这头发也是无可奈何,当初那九转护心丹虽然保住了我的性命,到了东海,桑先生又是用心替我调理身体,可是那药性还是太烈了,这才让我的发色变成这个样子,这几年我可是平心静气,认真休养身体的,至于什么海氏、越氏那些琐事,不过是我闲着无聊弄来散心的,你可没有看见我废寝忘食吧?”

长乐公主白了江哲一眼,道:“好了,我信你就是,当初若非是帮着二哥,你也不会差点丧命在长安,以后可不许你那样拼命了,你当我不知道么,前些日子,二哥的信一到,你就开始忙起来了,看来这悠闲的日子就快结束了,我也不阻你行事,只是凡事总得张弛有道,可别像从前那样呕心沥血就好。”

我连忙道:“一定一定,妻命不可不遵,要不,我跟你一起去看慎儿。”

长乐公主忍笑道:“别胡闹了,当我不知道么,若是让你去看他,一定又会逗弄他,他可正是贪睡的时候。也不知道你这是什么性子,从前就听二嫂说过,你总是偷着欺负逗弄蓝儿,如今就连慎儿也不放过,真是不像个父亲。”

我不由缩了缩脖子,这个我可不敢辩驳,好几次把儿子逗弄哭了,都被公主抓个正着呢。

公主的身影消失之后,我收回了依依不舍的目光,上前检视那件公主留下的绣品,果然找到了上面的血迹,不由心中黯然,这几年来,我和公主虽然琴瑟和谐,可是心中却总是有些歉疚的。当日公主和我私奔到东海,在桑先生的主持下成了婚,别说什么公主下嫁那种种繁琐的礼节,就连基本的六礼都不具备,观礼的人更是寥寥无几,除了身边几个人之外,一个外人都没有。成婚之后,将近半年的时间,我都是在静养和服药中度过的,公主也不过担个名份罢了,可是公主全无怨言,尽心尽力的服侍伺候,并且担起了主持家务和照顾柔蓝的责任,虽然有董缺和周尚仪的帮助,可是一个天之娇女,将这些琐碎的家事料理清楚可是费了一番心血的。就是这两年我的身子已经大为好转,夫妻之间情谊虽好,闺房之中却是仍然不敢放纵的节制,公主却是一如既往,细心照顾我的起居饮食。为了调理我的身体,她更是拿了皇室收藏的药膳秘本向桑先生请教,如今我的饮食都是公主一手置办的,就连桑先生也不得不佩服公主在这方面的才慧。想起公主偶尔亲自下厨做的小糕点,我忍不住吞了一口口水,那种美味可是令人终生难忘啊。

公主如此情重,我本该就这样和她过些闲云野鹤的日子,可是如今我却不得不重新入世了,虽然不想抛下这种平淡安乐的生活,回到风浪险恶的俗世,可是这也是无可奈何之事,李贽前些时日让骅骝送了书信过来,说明了如今的局势,宛转地请我出去帮忙,不说李贽从前的恩遇,就是看在长乐公主的面子上,也不能不管,若是大雍皇室再出了什么惨祸,只怕长乐会受不住的。再说,这也是一个让长乐公主和太上皇重归于好机会,无论如何,当年公主私奔,总是让李援恼怒的,如今自己应皇帝敕令重出,正可以让他们父女修好,想必公主定会欢喜的。而且,我更是心知肚明,如今自己成了大雍皇室的女婿,我的命运已经和大雍息息相关了,若是大雍不能一统天下,那么自己也别想过上安乐的日子。

看着书案上的一叠文书,再次翻阅了一遍,我的脸上露出冰寒的笑容,轻轻念道:“东海、越家、北汉、南楚!”语气中渐渐带了肃杀之意。

同一时刻,在东海侯为爱子举行大婚的海岛港口中,南海越家送嫁的坐舟之上,一间十分隐秘的船舱之内,一个容貌秀雅,气质飘逸的青年也正在翻阅着文书,没有窗子的船舱内一盏银灯放射着昏暗的光芒,映射着这个将近三十岁左右的青年的脸庞,或许是灯光的作用,那青年俊秀的面容上带着一丝恶毒的杀机。

“东海,越家!”青年低声念道,眼中闪过不屑的寒光。放下手上的文书,青年拿起银灯走到船舱一角,那里的舱壁上挂着一张精致的地图,绘制的是原东晋的疆土范围,大雍、北汉、南楚现在所占据的领土都用不同颜色的颜料圈起。青年的目光落到北面的滨州和南面的泉州之上,露出一丝冰寒的笑意,然后他的目光又落到北汉和大雍对峙的沁州、泽州一带。他自言自语道:“北汉应该会趁机进攻大雍的,这样的良机他们应该不会错过,失去东海对大雍来说虽然不是致命的打击,却也是伤筋动骨的损失,而且控制东海还有一样好处,或许我能够抓到那个人呢。”

想到那个人,青年面上闪过深恶痛绝的神色,他狠狠地道:“江哲,李贞,我绝对不会放过你们,李贞,你以贞洁自许,百般不肯下嫁于我,这倒也罢了,可你竟然和江哲私奔,这样的不贞不洁,还有什么颜面活在世上。”

正在这个青年脸上露出残忍恶毒的神色的时候,有人在外面道:“首座,一切已经准备妥当,越无纠传来消息,如果没有意外,还请首座不用出手。”

青年脸上闪过一丝嘲讽,道:“进来吧。”

舱门打开,一个相貌清瘦的中年人走了进来,他恭谨地道:“首座,仪凰堂首座和凤舞堂首座都有书信到,请您指示何时发动。”

青年淡淡道:“急什么,等到他们两败俱伤之后在动手不是更好么?”

中年人微微一笑,道:“越无纠也算是一个精明人,这次居然这样就进了首座的圈套,也真是英名扫地了,首座英明神武,岂是那些商贾可以匹敌的。”

青年却是没有丝毫得意之色,道:“我从前也曾惨败过,吃一堑,长一智,我学到了两件事情,一件就是天下没有没有弱点的人,另一件就是事情若未成功,便不能松懈。越无纠不是一个蠢人,可是他的弱点也太明显了。说起来这也是越氏传承方式给了他太多的野心了。

说起来,这当初越氏的先祖倒也是颇有远见卓识的人,他知道富不过三代的道理,养尊处优的后代难以承担大任,可是又不想嫡系子孙被旁系取代,所以就定了这样古怪的规矩。每一代宗主都可以在子孙中选择一个贤能的继承人为下一代宗主,若是所有继承人都不肖,则宗主可以任选其一为代理宗主,然后指定宗族中最出色优秀的一人为总执事,族中大权由总执事掌握,同时,宗主会指定一个亲近之人为护法。这样一来,如果代理宗主的子嗣中有贤能的,就可以在护法的协助下,顺理成章地从总执事手中取回宗主权力,若是第三代也没有出色的继承人,那么总执事就可以继承宗主之位。这样一来,既给了宗族中旁系子弟夺嫡的机会,又给了嫡系最大的保障,试想,若是大权被剥夺的代理宗主,还不懂得好好教育儿子夺回权力,那么这一支被取代也是理所当然的事情。所以这个规矩定下之后,越氏传承十七代,嫡系虽然曾经失去过权力,可是最后又都夺了回来。这就是越无纠心中惴惴不安,和我们合作的原因。

如今的越氏宗主越无陵虽然庸碌,可是倒不是蠢人,他将亲妹子嫁给了东海侯姜永,就已经巩固了自己的权力,如今又要将爱女越青烟嫁给小侯爷姜海涛,他的长子越文翰更是雄才大略,你说这越无纠眼看着到手的大权又要送了出去,怎肯甘心,我们从这里着手,越无纠为了权势地位,哪有不上钩的道理。”

中年人犹豫地道:“虽然如此,越文翰很得越氏子弟的敬重,若是我们这样帮助越无纠,只怕越氏那些人不会接收越无纠作宗主的。”

青年笑道:“有些事情你不知道,这越文翰的确是雄才大略,可是他却做了一件最不该做的事情,他不该挡住了我们的路,不该有那么一位一心为他着想的好妹妹,更不该娶了那么一位妾室。”

中年人恍然道:“难道那位薛夫人竟也是仪凰堂的人么?”

青年犹豫了一下,道:“这倒也不是,从前这位薛夫人也是我们的旧识,她出身原本尊贵,就是如今,她的父亲也是官居一品的朝廷大员,一位堂堂的千金小姐,若不是行止差错,怎会做了人家的妾室。说起来,门主、纪首座和燕首座她们至今还觉得薛夫人太丢她们的面子呢。不过,不管怎么说,若没有薛夫人说服了越文翰兄妹,只怕他们早就自尽,也不会任凭我们摆布了。谁让这薛夫人好面子,不愿意从前的旧事给丈夫知道,若不是我们以此相胁,她怎肯就范。”

中年人道:“可是首座原本答应,事成之后,保住越文翰的性命,让他扶薛夫人为正室,这件事情越无纠肯答应么?”

青年冷笑道:“不答应也不行了,留下越文翰,是为了牵制越无纠,免得他气焰太嚣张,反正到时候越文翰也没有本事逃脱我们的手掌心了,他犯下的大罪,除了南楚和我们,谁还能护住他。”

青年说完这句话,舱中陷入了无比的静默,他下意识的回想起这几年的辛苦,原本是敌对的南楚并不容易立足,门中众人又是各有心思,经过两三年的争斗,好不容易让他重新组合了凤仪门,分组凤舞堂和仪凰堂,将凤仪门原来的势力分散,纪霞和燕无双分别统领两堂,两人之间因为理念不和常常暗中争斗,而自己组建辰堂,招纳外人入门,担任外围事务和冲锋陷阵的工作,表面上中立,却因为两堂互相攻讦,而让自己的辰堂成了最重要的势力,门主凌羽早已经给三堂架空,除了身边的一支亲卫之外再无别的力量。而自己也因为知道凤仪门终究不是自己可以夺取最终权力的所在,所以聪明的维持了凌羽的地位和门内的平衡,多么艰难的过程,才让自己终于完全掌握了凤仪门,可以开始自己梦寐以求的报复了。而他也终于说服了尚维钧和自己合作,对于尚维钧来说,一手掌控军权的陆家是太大的威胁,甚至胜过了大雍的南楚的威胁。自毁长城大概是南楚历代掌权人的爱好吧。

陆灿,青年眼中闪过一丝寒芒,若非是如今还要仰赖此人抵御大雍,他早就想法子让陆灿死于非命了,不是为了尚维钧那个废物,而是因为陆灿曾经是他的弟子。胸中好像有凶恶的猛兽在咆哮,在呼号,毁灭那人留在世间的一切,这已经是他——韦膺——心中唯一的执念了。

喜堂之内,重重帷幕之后,新妇仍在侍女仆妇的伺候下等候吉时,越氏乃是名门大族,越青烟又是宗主的嫡女,侍女如云,妆奁丰厚,前来送嫁的是新妇的嫡亲兄长,少宗主越文翰和越家总执事越无纠,当然此时他们已经在前面喜堂上了,后堂除了越家的女眷之外,就只有姜家的仆妇了。负责照顾新娘的却不是旁人,乃是越文翰的妾室薛夫人。

这位薛夫人嫁入越家已经将近两年,这位夫人乃是越文翰偶遇的一位小姐,据说是北地名门之后,因为命犯华盖,在南海普陀山紫竹庵带发清修,三年前越文翰到普陀山代过世的母亲还愿,无意间邂逅了这位薛夫人,颇为钟情,苦苦追求,可是这位薛夫人却是冷若冰霜,屡次拒绝。越文翰苦苦追求了一年多,才终于感动了佳人。按照越文翰的意思,想要娶她为正室,可是却遭到越氏长辈的反对,他们对越文翰冀望非浅,都将他当作未来的宗主,越氏宗主的婚姻是不能轻易决定的。薛氏虽然品貌双全,可是来历不明,是断不能为正室的。越文翰无奈之下,宛转向薛氏恳求,希望她下嫁自己为妾,等待合适时机再将她扶正。谁知薛氏闭门想了几日之后,竟然答应了,并说自己本不配做越夫人。越文翰虽然奇怪,可是他钟情已深,还是高高兴兴的娶了薛夫人。两人感情原本很好,可是自从小姐婚期议定之后,两人之间似乎除了问题,越文翰对薛氏突然冷淡下来,可是薛氏却是不以为意,反而热心的张罗着小姑的婚事。

柔蓝和李麟在姜家仆妇的带领下,走进后堂的时候看见的就是薛氏正在指挥侍女替新娘补妆。薛氏年纪也有二十六七岁,貌如春花,体态如柳,神情落落大方,气质雍容,室内虽然人多口杂,但是在她指挥下却是井井有条。不过柔蓝的心思全放到了新娘身上,仔细看去,只见那新娘越青烟不过十六七岁的年纪,弱质纤纤,眉目如画,秀丽清雅,虽然年纪还小,却已经是绝色姿容,若说有什么不足之处,就是这越青烟肤色过于白皙,几乎接近透明了,虽然美丽,却是过于苍白,显得气血不足。因此薛氏正在亲手为她施用胭脂,仔细的描画了半天,才勉强放手,薛氏想必精于理容,经她妙手,越小姐果然似乎多了几分血色,更添了几分艳丽。她一身红色绫绡嫁装,凤冠霞帔,更显得美丽不可方物,那领着柔蓝的仆妇惊叹道:“少夫人真是好容貌,小侯爷真是好福气。”

她的说话声惊动了薛氏等人,她笑道:“原来是李嬷嬷到了,这是?”她的目光落到柔蓝和李麟身上。

仆妇下拜道:“禀薛夫人,这位是蓝小姐,是小侯爷恩师的千金,侯爷让她来后堂见见少夫人。”

薛夫人眼中掠过一丝明亮的光芒,笑道:“原来是蓝小姐,青烟,你来见见。”

越青烟原本默然不语,听到薛夫人的说话,抬起头来,向柔蓝看来,明如寒泉的双眸闪过莫名的悲恸,轻轻欠身道:“蓝妹妹。”说罢伸出右手,示意柔蓝过去到她身边。那是怎样一只纤纤素手啊,冰肌雪肤,如同美玉雕成一般。柔蓝走到她身边,忍不住握住了那只纤手,触手一阵冰凉,柔蓝不由想道,难不成这个新娘子是冰做的不成么?不由打了一个寒战。

\chapter{第九章 花烛惨变}

柔蓝连忙抽出手来道:“好冰啊,姐姐的手怎么这样凉。”她奇怪地看着越青烟,心想公主娘娘的手总是暖洋洋的软软的,怎么这个新娘子姐姐的手却是冰的。越青烟歉意的一笑,道:“是姐姐身体不好,手足总是冷的。”

柔蓝眼珠转了一转,道:“姐姐身子不好么,我爹爹和公公都是神医呢,过几天海哥哥一定会带着姐姐去拜见爹爹娘亲的,到时候让公公给你看病好不好。

越青烟脸上闪过一丝无奈的笑意,低声道:“没用的。”她的声音十分低微,几乎接近呓语,就连站在她身边的小柔蓝也没有听清楚她说什么,可是站在柔蓝身后的李麟却是将她的神情看的清清楚楚。那是一种心灰意冷的绝望和无奈,李麟年纪虽小,却是看的明明白白,只因他早就看过这种神情,在大雍军中,李麟可不是养尊处优的少爷公子,虽然年纪不大,甚至还拿不动刀枪,可是李显几乎总是将他带走身边,李麟最经常看到的就是被俘虏的敌军谍探或者犯了军法的将士被自己的父亲下令推出去斩首。而每当这时,不论那人是苦苦哀求还是视死如归,李麟却都能从他们的眼中看见那种绝望无奈的眼神,就像是狩猎之时濒死的野兽的眼神。李麟知道,有这样眼神的人是最可怕的和最危险的,有一次他曾经因为同情一个将要被处死的军士,便走到他身边想要安慰于他,可是那个军士居然挣断了绳索,想要挟持李麟迫使李显放他离去,虽然最后军中的神箭手射死了那个军士,救了李麟性命,可是李麟从此对这种人便充满了戒心。他一把把柔蓝拽到自己的身后,用充满敌意的眼睛看着越青烟。柔蓝古怪的看了一眼李麟,不明白他要做什么,可是柔蓝却能够感觉到李麟的紧张的情绪和绷紧的身体,所以她也乖巧的一动不动。可是这个时候,正是越青烟此时正在强颜欢笑,伸手想要去拉柔蓝,李麟这样一来使得房内的情景变得十分尴尬。柔蓝轻轻的扯了一下李麟的衣服,李麟却是固执的不肯让越青烟亲近柔蓝,小小的心灵中只有一个念头,就是不许任何人伤害身后的这个小妹妹。小妹妹,当然是小妹妹,李麟固执的想,自己个子比她高,长得比她壮,虽然爹爹让自己称她姐姐,她也叫自己弟弟,可是在李麟小小的心灵里面,柔蓝就是自己的小妹妹。

这时候薛夫人走过来,熟练的将柔蓝抱了起来,李麟刚想阻止,但是薛夫人只是伸手轻轻一拨,就已经将柔蓝抱入怀中,李麟面上闪过羞恼的神色。薛夫人笑道:“蓝小姐,青烟脾气不好,想是让蓝小姐受惊了,这也是青烟有些紧张不安,谁让这是女子一生最重要的时候呢,过几日等到青烟去拜见令尊的时候,一定要让她给小姐道歉,小姐不如去看看侯爷夫人吧,她这些日子身体不好,就连婚宴也不能参加呢,若非是为了冲喜,我们还不会答应这么快就让青烟嫁过来呢。”

柔蓝眼中闪过迷茫,不论她如何聪明,毕竟还是一个小孩子,薛夫人这样絮絮叨叨的一番话听得她云里雾里,不过薛夫人这样说了半天,房内的气氛变得平和自然了许多。

这时,门外突然传来一个冰寒的声音道:“柔蓝小姐,老夫人请你过去见她。”那是一种如同山涧幽泉一般幽冷,声音中带着几分阴柔,动听而优雅,令人仿佛有热天吞下冰水一般的感觉。柔蓝大喜道:“顺叔叔。”然后就雀跃着向外面跑去。李麟一愣,便也跟着跑了出去。只见廊下一个青衣少年负手而立,冷若冰雪的面容上带着真心的微笑,柔蓝高兴的扑了过去,十分熟练的向上一跃,而青衣少年配合默契地轻轻一扶她的脚底,柔蓝借着这力道轻而易举的骑在了青衣少年的肩上。柔蓝欢欣地道:“顺叔叔,你怎么会在这里,你不是都不肯离开爹爹身边的么?”青衣少年淡淡一笑,道:“公子吩咐我来保护小姐。”他的目光落到了李麟身上,李麟只觉得那人的目光从自己身上掠过,仿佛可以看透自己的五脏六腑一般,不由后退了一步,可是强烈的被羞辱的感觉让他没有再退后,反而瞪着眼睛看向那个青衣少年。

这时,薛夫人的身影出现在门前,但是她没有走出房门,反而退了回去,她的面容上带了一些震惊,低声问道:“怎么这里会有男子在?”

姜家派过来的李嬷嬷看了门外一眼,道:“禀夫人,那位是蓝小姐家中的李爷,素来都在内宅行走的,并无妨碍,请夫人不用担心。”薛夫人眼中闪过一丝光芒,和一直站在屋角的一个侍女交换了一个眼神,那个侍女眼中掠过一丝杀机,似乎想要举起脚步,可是薛夫人递过了一个冷厉的眼神,侍女停住了脚步,眼中闪过一丝不满,然后侍女的目光落到了越青烟身上,那是带有征询意味的目光。越青烟轻轻点头,紧紧咬着嘴唇,还没有描画过的嘴唇本是苍白的全无血色,此刻却多了一丝血痕。她下意识的用右手抚向左手腕脉,在红绡喜服的掩盖下,她的左手腕脉处系着一条红绫丝巾。

吉时已经到了,在喜娘簇拥下,夫妻行了交拜之礼,拜了天地祖先,李显含笑站在一边,他的目光落到了喜堂的一角站着的两个人身上,一个是身材高大,神情倨傲的中年人,另外一个则是一个二十四五岁的青年人。引起李显注意的是,这两个人脸上的神情过于淡漠平静,这原本不是什么奇怪的事情,可是这两个人本是新娘的至亲,宗亲叔父越无纠和新娘的嫡亲兄长越文翰,在这样的大喜之日,就是他们和新娘之间感情淡漠也会装出欢喜之色的,更何况越青烟本是越文翰唯一的嫡亲妹子,而且据说兄妹之情十分深厚呢。李显的目光流转,看到了更多的不寻常之处,南楚的两个使者神情都有些古怪,副使伏玉伦神色有些紧张惶急,而正使陆灿却是神情悠闲从容,唇边带着淡淡的笑意。

就在新婚夫妻摆了天地父母之后,即将被送入洞房的时候,突然新娘的兄长越文翰高声道:“侯爷,小侄有一件事情想请您作个决定。”

东海侯姜永愣了一下,不悦地道:“文翰,不论是什么事情,总要等到成礼之后再谈吧。”

越文翰冷冷一笑,英俊冷漠的面容上露出讥诮的神色,道:“这件事情还是当众谈一谈的好,毕竟这件事情想必大家也都很有兴趣知道。”说罢他的目光从堂上众人身上一一掠过,有资格站在堂上观礼的人并不多,除了大雍、北汉、南楚的使者之外,只有东海侯的一些亲信属下和越家的人,就连海氏叔侄也因为身份不够而在堂外。这堂上众人都是身份显赫,久经战阵官场的人,怎会被他的气势压过去,若非是碍于东海侯的面子,只怕早就出声斥责了。姜永的神色变得冷沉,再也不是原本那个只是欣喜爱子成家立业的父亲,此刻的他已经变成了东海群盗的首脑,东海的霸主。他轻轻一挥手,所有参加观礼的东海众人有默契地控制了各处门户角落,将堂上众人隐隐包围起来。姜海涛原本喜气洋洋的神色变得十分冰寒,他甩开了手上的红绫,退到了父亲身后。可是这样的局势,处于弱势的越文翰却是似乎毫不在意,冷冷道:“越氏乃是以海运起家,要是有人作我们的对手,越氏自然也不会畏惧,可是海氏突然兴起,迫得我们越氏苦不堪言。海氏之所以占了我们的上风,不过是因为他们掌握了造巨舟的技术,而且还有姑夫的海上劲旅为他们护航,也难怪他们顺风顺水,姑夫不念昔日越氏暗中支持之恩,小侄也不敢挟恩图报,越氏也不贪求,只要海氏交出造船图和这几年绘制的海图就可以了,越氏自信还有可以力量可以保护船队。”

姜永没有作声,看了一眼姜海涛,姜海涛会意地道:“表哥这话可就不对了,做生意讲究的是各凭本事,海氏有本事造出大船,与越氏有什么相干,若是越氏想要和海氏合作,理应和海爷私下商量,为何却要搅闹小弟的喜事?”

越文翰脸上闪过一丝莫名的情绪,道:“天下谁不知道海氏船行的后台就是姑夫大人,海氏独霸海运只怕就是姑夫的期望吧,若是青烟和你完成了大礼,你们或许会看在亲戚的面子上给越氏一些好处,可是却绝不会平白将造船图给越氏,到时候小妹已经成了你们姜家的人,形如人质,越氏岂不是白白吃亏,还不如事先谈个明白的好。”

姜海涛怒道:“这算什么,这里是我东海,不是你们南闽,表哥若是想插手这桩生意,也应该拿着真金白银,和我们坐下来谈个清楚明白,这样子强词夺理,莫非越氏的生意一向是这么做的么?”

越文翰冷冷道:“所谓强权即是真理,只索要造船图和海图,这还是小侄看在姑夫重义,不肯轻易出卖盟友的情分上呢,若是按照总执事的意思,早就要请姑夫和我们联手瓜分了海氏,何必靠着人家的残羹剩饭过活,牢牢的掌控住发财的路子不是更好么?”

姜永脸色变了又变,听到这里冷冷道:“海氏是东海的盟友,你这是让我们姜家背弃盟约,出卖盟友么?想不到你竟是这样的人,罢了,看在你姑姑的份上,你们越家这就走吧,青烟你们带回去,我们姜家不敢要越家的女儿做媳妇。”

这时候,两家的争吵早已经惊动了整个岛屿,越家护送新娘的家将近卫都已经逼近了喜堂,他们早有准备,身上更是暗藏了兵刃,而姜家的属下负责保护整个岛屿上面的安全,也都是全副武装,双方在喜堂外面对峙起来,姜家乃是统兵之人,疏散宾客婢仆,安排贵宾们带来的近卫在两侧偏厅内暂歇,十分迅速明快,除了越家的人因为早有准备已经到了喜堂之外,其他的人都被软禁保护了起来。

越文翰对这样的局势仿若未见,反而冷冷一笑,高声道:“我越家的女儿尊贵得很,就是姜家想娶也未必可以娶得到呢,青烟,既然姜家看不中你,你就回来吧。”

一直肃立在一边默不作声的新娘微微欠身,然后一只欺霜赛雪的玉手扬起,摘下了盖在凤冠之上的红绫帕,露出绝美的容颜,那一双明澈如同秋水,冰冷如同寒江的眸子轻轻一转,已经将堂上众人看的清清楚楚,她低首敛眉,走到越文翰身边站定。

一直含笑不语的越无纠道:“侄女,既然姜家无情,我们也不用留手,还请侄女为自己讨个公道吧。”

众人听了心中都是一凛,若是越无纠下令让在堂外的越家随从进攻,众人倒是可以理解,可是越无纠却让越青烟出手,这可就匪夷所思了,越氏的女儿,那是名副其实的千金小姐,怎么可能会有攻敌的手段。不过他们也都提高了警惕,既然越无纠这样说,那么越青烟一定是有什么特殊的本事。

越青烟的目光转向越文翰,越文翰淡淡点头,越青烟眼中闪过一丝凄然,闭上了双目,就在这一瞬间,守在喜堂门口的那些东海的卫士,突然各自惨叫一声,软倒在地,昏迷过去。

姜海涛大惊,随手拔出一个卫士的长剑,扑向越青烟,口中道:“妖女敢在此地用毒,受死。”

姜永皱眉道:“涛儿不可鲁莽。”

但是这时姜海涛和挺身拦阻的越文翰交手起来,越文翰武功平平,姜海涛不过数招就已经将他逼开,他冲到越青烟身边,正要举手点了越青烟的穴道,越青烟睁开双目,那曾经明亮如同清泉的眼睛却已经变成了血红色,她露出一个冰冷的微笑,姜海涛只觉得五内如同针刺火烧,惨叫一声,跌倒在地。越青烟缓缓环视厅内,她的目光一落到某人身上,那个人就觉得头晕目眩,栽倒在地上。一身红衣的越青烟彷佛地狱烈火中的罗刹一般美艳,也如同罗刹一般令人魂飞魄散。

齐王李显突然一字一句道:“同心蛊,你用的是同心蛊。”

越青烟的目光落到了齐王身上,通红的眼睛带着哀莫大于心死的神情,然后她轻轻蹙眉,一滴冷汗从额头滚下。

李显冷冷道:“越姑娘不用费心了,同心蛊虽然厉害无比,可是本王身上有可以辟邪的珍宝,你的蛊毒是伤不到本王的。”

越青烟眉头又是一皱,道:“天下可以辟邪辟毒的宝玉并不多见,王爷身上的是‘辟邪紫玉’还是‘苦海菩提’呢?”

李显淡淡一笑,伸手从胸前拉出一条细细的乌金丝链子,链子上面系着一块紫色的巴掌大的佩玉,雕刻成辟邪的形状,宝气隐隐,玉色明净。

李康惊叫道:“父皇竟将此玉赏了给你。”他眼中闪过怒火。

林碧笑道:“早就听说辟邪紫玉功能辟毒驱邪,想不到齐王殿下竟然带在身上,看来我们还是有些福气的,越姑娘,你的同心蛊虽然可怕,可是在辟邪紫玉面前却无用武之地,而且同心蛊使用起来伤人伤己,越姑娘不若收了起来吧。”

越青烟的目光落到林碧身上,闪过一丝残狠,正要催动蛊毒,越无纠高声道:“二小姐,你怎会使用天下共禁的同心蛊,少主事先可知道么,宗主可知道么,怎么此事却不告诉为叔。”

越青烟露出嘲讽的笑容,停止了催动林碧身上蛊毒的动作,道:“不,爹爹不知道,大哥却是知道的。”

越无纠脸色一变,道:“属下身为总执事,真是无能失职,少宗主有意吞并东海,在下劝阻不成,只得从命,想不到公子竟然和小姐串通,使用同心蛊害人,属下虽然是臣属,也不敢服从乱命,公子小姐不若束手就擒,随属下回去向宗主请罪吧。”

他这一番话说的言辞恳切,那些挡在堂门口的越氏高手面面相觑,有人排众而出道:“少主,总执事所说极是,还请公子和小姐不要用蛊害人,随我们回去请宗主责罚吧。”

越文翰和越青烟脸上同时闪过一丝了然的神色,越文翰冷冷道:“你们都是越家的属下,这里没有你们说话的余地,青烟,若是有人敢不从命,你取了他的性命就是。”

越青烟微微一笑,本已经变成黑色的眼睛再次变得血红,同时,刚才排众而出说话的那个越家高手仆倒在地,面色狰狞,气绝身亡。所有的人都几乎吸了一口冷气。越青烟冷冷道:“所有人都放下武器,自束双手,违命者死,齐王殿下,你虽有宝玉护身,可是也只能护着自己,你的兄长属下却是一个都不能活,你若乖乖束手就擒,我还可以暂时留你们一条活命。”

林彤眼中满是惊惧的神色,轻轻拉着姐姐的衣袖问道:“姐姐,什么是同心蛊啊?”

林碧望向越青烟,越青烟别过头去,林碧叹息了一声道:“同心蛊乃是南疆蛊毒中最奇特的一种,这种蛊生性好洁好阴寒,只喜欢服食少女鲜血,别号蛊中之王,因为只要中了这种蛊毒,就再也没有挽救的可能了。想要养同心蛊,需要一个刚刚及笈的少女,每日里以鲜血和药喂食,还要将蛊王放到身边,日夕肌肤相亲,不能懈怠,时间需要三年至七年,这要看那少女的体质和资质了。数年之内,蛊王养成,此蛊就寄生在主人心口,人蛊合一,心灵相通,只要蛊主一动念,蛊王就可以在任何可以看到的人身上种下子蛊,之后只要那蛊主有心,对那中蛊之人,就可以主宰他的生死。这种毒蛊还有特异之处,若是中了子蛊之后,再服下蛊主的药物,中蛊之人就可以和蛊主心意相通,不论千山万水,都不能阻绝他们的心意相会,所以才叫做同心蛊。越姑娘想必是体质绝佳的鼎炉,只过了两三年蛊王就已经养成了,恐怕越姑娘废了不少心血吧?”

林碧说罢,惋惜的看了越青烟一眼,又道:“彤儿,此蛊最可怕之处就是不仅可以伤人,还会伤己,此蛊每日都需要服食主人的鲜血,分量与日俱增,而催蛊伤人之后更是需要数倍的鲜血。越姑娘气血不足,容貌如雪,想必就是这个缘故。这还罢了,要知道蛊毒虽然可怕,可是还有克制之法,若是越姑娘死于刀剑之下,那蛊王就会破体而出,将越姑娘身上精血全部吸食干净,然后这蛊王就可以自由自在的活在世上,它存身之处,方圆十里之内,绝对不会有人畜可以存活。就是越姑娘死的时候没有见血,蛊王没有机会破体而出,而是和越姑娘同死,那么所有越姑娘下过蛊毒的人也都会同时死去。这还是越姑娘可以控制蛊王的情况呢,若是越姑娘鲜血供养不足,那么这蛊王就会反噬主人,所以就是越姑娘也不能控制这种同心蛊的危害,这也是天下共禁同心蛊的缘故,只是这同心蛊早已经失传了,想不到竟还会有人修炼。”

越青烟脸上一片漠然,右手却忍不住抚向左腕,那里系着的丝巾之下,那是她每日用金针放血之处,伤痕宛在。

林彤痛惜地道:“哎呀,越姐姐,这同心蛊这样可怕,你,你有多少鲜血可以供养它啊,还是早些想个办法除了它吧。”

越青烟眼中闪过一丝暖意,她方才任凭林碧述说,原是因为想通过林碧的说话,让大家心中惊惧,这样也方便自己控制众人,可是林彤这样的关切,倒让她心中十分感动,心道,不论如何,我都不杀你就是了。她的目光落到姜永身上,淡淡道:“姑夫,你还不交图么?”

姜永眼中闪过一丝痛惜,道:“青烟,你本是千金小姐,为什么要练这种残狠的邪功,你可知道,就是你如今威风凛凛,可是却是昙花一现,终不久长,是谁,是谁让你练了这种功夫的?”

越青烟神色间露出一丝决然,道:“姑夫,对不住了。”说罢就要催动蛊毒,这时,突然门外金鼓大作,守卫在喜堂门前的那些越家侍卫惨叫连连。众人望去,只见那些侍卫身上都被翎箭射穿了要害。越无纠眉头一皱,到了门前向外看去,只见百步之外,重重叠叠的盾牌掩护下,一些身穿东海水军服饰的弓箭手正在引弓待发,一个大汉高声道:“里面的人听着,这里四下已经被我们围住,我们东海别的没有,若论弓箭可都是神箭手,如果你们还要放肆,休怪我们箭下无情。”说罢那个大汉举起长弓,射出了一支鸣镝,而千百支利箭随后而至,越无纠大惊,连忙踢上了厅门,避到一旁,只听见如同冰雹落地的一阵声音,那门板已经被利箭射穿。门外传来那大汉的喊声道:“侯爷,请你下令,若是有人敢不听从,一刻之后,我们就要放火烧屋了。”

越无纠神色惨变,道:“侯爷,听说东海水军每一队中都有若干神箭手,百步穿杨,取人性命,势如雷霆,今日一见,果然名不虚传,还轻侯爷下令让他们暂时退后,否则,若是伤了青烟,只怕我们都不能逃过蛊王的追杀了。”

姜永淡淡道:“这是东海,本来就不是别人可以作主的地方,青烟侄女,你是不是可以收手呢,先收回海涛身上的蛊毒如何?”

越青烟面色更加苍白,看了一眼越文翰,越文翰冷然道:“姑夫,事已至此,我们已经是无路可退,而且只要给青烟片刻时间,那些弓箭手也不会逃过蛊毒的暗算的。”

姜永笑道:“青烟若是催动蛊毒,可是需要耗费心力鲜血的,你不怕她被蛊王反噬么?”

越文翰淡淡道:“若是如此也没有什么不好,此间玉石俱焚,能够和这么多达官显贵死在一起,文翰和舍妹死也无憾。姑夫,你应该清楚,若是青烟身怀同心蛊的消息泄漏出去,只怕来兴师问罪的人车载斗量,不迫得舍妹投火*,是不会善罢甘休的,天下的人都是我们兄妹的敌人,多死几个也没有什么不好,若是姑夫不肯令属下放下弓箭,只怕小侄只能得罪了。”

李显突然纵身过来,一招就将越文翰拍倒,然后将腰间长剑拔出,指住越文翰的咽喉,他这几下十分迅捷,众人都在投鼠忌器,哪里想到李显这样大胆,虽然他有宝玉护身,可是这里的人各个身份贵重,若是真死了几个,只怕李显也不能交代的过去的。果然越青烟见状神色一变,立刻发动了蛊毒,庆王李康惨叫一声倒在地上。

李显却是神色不变,笑道:“越小姐可是糊涂了,什么时候你听说过帝王家还有亲情在,只要我李显一身平安,哪里管得别人死活,小姐兄妹情深,若是肯束手就擒,李某倒是可以保证,不会伤害你们兄妹,而且小姐就不想摆脱那蛊王反噬的命运么,若是小姐愿意,本王可以上书陛下,召集天下名医为小姐诊治,虽然可能只有一线生机,也胜过这样坐以待毙啊?”

越青烟神色有些动摇,可是转而又恢复了平静,冷冷道:“我不信你的话,你连兄长的生命都不顾,我怎知你会信守承诺呢?”

李显心中一喜,越青烟已经动摇,这就好了,他面上神色不变,道:“越姑娘,你既然和海涛有婚约,想必也知道我李显的为人,本王也没有什么别的好处,可是从没做过不守信诺的负义之事,只是本王的性子古怪,若是有人迫我,我就偏偏要和他为难到底,姑娘今日就是在我面前杀了我的三哥、表哥和侄儿,本王也不能低头求饶,可是本王立誓,姑娘若是下了狠手,我就会单身突围而去,到时候南闽越家终有落到我手上的一日,我也不诛姑娘的九族,只是将南闽越氏的族人全部贬为贱民,让他们生生世世,被踩在他人脚下,贱如泥土。”

越青烟神色渐变,她出身名门,读过律法,自然知道贱民男女,不可与良民婚配,所以贱民中的秀美男女往往沦为娼妓嬖童,李显的威胁是恐怖而直接的。这时,越文翰突然以咽喉向李显剑上撞去,李显手疾眼快,移开了剑锋,越无纠趁机将越文翰救了回去。

李显无奈的看看越文翰咽喉处的血迹,笑道:“看来还是你们赢了呢。”

越文翰站起身来,不理会越无纠的扶持,踉踉跄跄地站在越青烟身边,道:“齐王殿下,还请不要擅动,否则就不要怪我们动手杀人了。姑夫,请让你的属下弃械投降,否则小侄只好先取了表弟性命,再和姑夫说话了。”

姜永心中一震,无奈地高声道:“远新,暂时不要出手,等候我的命令。”

越文翰脸上露出绝决之色,众人都是心中苦笑,怎么这两兄妹都是如此悍不畏死,千金之子,坐不垂堂,他们这是何苦呢?

\chapter{第十章 东海波平}

今天更新一整章,因为写完了这么多,想到我苦苦等待爱看的书的时候的苦恼,所以全部发上来了。

————————————————

越文翰、越青烟交换了一个眼神,越文翰朗声道:“不论诸位如何本事,如今却都在我们兄妹掌握之中,虽然齐王殿下不怕蛊毒,可是齐王殿下应该也不想看着庆王殿下死在此处吧,到时候就是齐王殿下幸而逃走,只怕大雍的皇帝陛下也会置疑殿下是否借刀杀人吧。”

李显微微苦笑,就算不是为了这个原因,他也不能眼看着东海侯父子和三王兄死在此处,不说什么亲情道义,若是东海侯父子一死,这支眼看着就要落到大雍手上的强力水军定会四分五裂,到时候只怕海疆匪盗纷扰,而且将来进攻南楚,还需要有得力的水军统领,若论水战,只怕大雍还没有人可以胜过东海侯父子呢?再说庆王,这一点可是被越文翰说中了,自己可以不将庆王看在眼里,可是他身份上却是大雍地位最高的亲王,自己的兄长,若是任凭庆王死在这里,不说庆王的部下不会善罢甘休,就是朝中大臣也会怀疑自己别有用心,戕害国家重臣,皇室宗亲,到时候他们群起而攻,就是皇上也保不住自己,就算不被问罪,这帅印也得拱手让人,到时候谁能抵挡龙庭飞呢?更何况麟儿还在岛上,自己就是狠心绝义,难道还忍心牺牲发妻留下的唯一血脉么?

李显越想越是愤懑,他什么时候这样屈辱过,若是有人敢用人质威胁他,他最惯用的做法就是让敌人和人质玉石俱焚,可是这越氏兄妹握住了他的要害,用他牺牲不起的人质来威胁他,李显至此也只能苦笑连连了,不由心道,该死的江哲,若非是你让我来观礼,我现在早就在你那里盘桓了,也用不着陷入这样的困境,暗中诅咒了半天,李显突然醒悟,自己来此,江哲也是知情的,而且他的爱女柔蓝也在岛上,他又曾经派人通知自己喜宴必有波折,如今果然出了事情,那么他总不会没有准备吧。想到这里,他心中稍安,暗暗祝祷道:“江哲,江先生,好妹夫,我也不求别的,你快点施展手段摆平了那越氏兄妹吧。”

似乎是老天爷回应他的祝祷,有人淡淡说道:“越少宗主,越小姐,两位不用勉强做戏了,就在婚宴之前,陆某已经得到消息,令尊大人和越氏几位执事已经脱险了。”

越文翰和越青烟同时惊道:“什么,怎么可能?”

李显心里惊喜,转头看去,脸上的表情却凝固住了,这说话之人竟是南楚大将军陆灿,不论是什么解开僵局都有可能,李显却万万料不到是陆灿,虽然明知道这个陆灿是江哲的弟子,可是谁不知道江哲已经和南楚势同水火,死士行刺和臣娶君妻两件事情已经让江哲和南楚再无转圜余地了。

陆灿神色从容淡然,好像自己所说的只是平平常常的话语一样。他看了一眼神色陡变的越无纠,道:“陆某倒是十分佩服越执事的心机,利用越小姐觉得自己无用,希望帮助兄长夺位的心情诱使她养蛊,然后步步进逼,迫得越小姐下手杀了几个你的亲信,到了这时,越小姐修练同心蛊,杀害越氏同宗的把柄已经落在你的手上,你本来可以利用这一点迫使越少宗主放弃宗主之位。可惜越执事未免太过偏激,自己无情无义便以为别人也是如此,为了防止越少宗主有东山再起的可能,你又决定将忠于宗主的越氏族人全部铲除。可是你若想这样做,别说是其他的族人不肯等死,就是支持你继位的族人也不会愿意见到你独自称尊的局面。所以你痛下决心,决定在越氏内部进行一次大清洗,宁可将越氏削弱,也不容许有人可以和你争夺权力。本来一桩简单的家族争权,竟让你变成了涉及到天下大局的阴谋,说起来,陆某还真是不得不佩服你。”

陆灿的语气有些讥讽,他看向越无纠已经铁青的面庞,道:“越执事安排的真是一场好戏,囚禁了越宗主家中的其他长老执事,然后逼着越少宗主和越小姐按照你的计划演出这场好戏,只怕东海事了之后,人人都知道,越氏少宗主不顾禁令,指使越青烟小姐修炼同心蛊,杀害同宗不说,还为了夺取亲家东海掌握的船图、海图,滥杀无辜,不幸遇难者有大雍庆王李康、齐王李显、南楚陆灿、东海侯父子、海仲英叔侄。之后越宗主杀了子女,自尽以谢天下,越氏从此由大执事掌握,东海四分五散,海氏身死族灭,越氏掌握了远洋贸易,铁了心归附南楚。你的幕后指使和你都是心满意足,只可惜了无数冤魂。”

越无纠只觉得如坠冰窟,这是他和北辰堂首座所苦心安排的计划,只有他们两人才知道整个计划,怎会被这个自己必须杀死的青年如数家珍。那人也是南楚的势力,莫非自己竟然中了圈套么?他忍不住喊道:“青烟,杀了他,宗主绝不会有逃生的机会,你相信他的话么?”

越青烟眼中一阵茫然,她愕然道:“这是真的么,大执事,你不是答应过只要我听命行事,等我身死之后你就放过我的父兄,你真的早就决定将我们一家全都杀了么?”

越文翰听了之后,神色一变,冷冷道:“越无纠,你不是答应我只要事后我以死谢罪,就不会伤害我父亲和诸位长老执事和青烟的性命么,原来,你竟然是要这般斩尽杀绝,亏我和青烟还想无论如何也要夺到船图,这样我们虽然身死,但今后就无人可以撼动越氏的海上霸主地位。想不到你竟然要将族人尽皆出卖?”

越无纠眼中闪过一丝尴尬,道:“此人不过是胡说八道罢了,如今青烟的事情已经给这些人知道,你想她被活活烧死么,若是不将所有人都杀了灭口,我们就是想保青烟也是保不住的。”

越文翰冷静地道:“大执事,你乱了方寸了,你对我和青烟所说就已经有了不同,比起陆大将军来,你们所说的话谁更加可信就不用猜了。罢了,和你合作的那些人的心狠手辣我已经见识过了,斩草除根本就是江湖铁律,是我们兄妹太天真了,以为你还会念着越氏的祖宗,可惜我们却遇到了一个数典忘祖的叛徒。”

他轻轻握住越青烟的手,黯然道:“青烟,为兄怕是不能保护你了,你也不要担心,不论生死为兄陪你就是,你犯的错误,我都有责任。”他冷冷的看向陆灿,道:“陆将军,舍妹年幼无知,受人挑唆,若是陆将军真的就出了家父和诸位长辈,那么我们兄妹甘愿受死,不过将军也需答应文翰的一些条件,否则,我们也不会白白送死。”

越无纠怒道:“陆灿,你是怎么知道这些事情的,是不是他和你同谋,图谋越氏?”

他这句话起到了方才用来劝诱越氏兄妹的话语起不到的作用,越文翰眼中闪过疑问,他知道和越无纠同谋的幕后之人是南楚权贵,而且越无纠所说不错,陆灿知道的实在是太多了一些,若是陆灿是存心将越氏对立的双方势力一网打尽,然后控制越氏海运,那么对于南楚来说果然是更有益处的。

越文翰疑惑的目光让陆灿苦笑道:“大执事这倒打一耙的本事真是厉害,我知道此事都是因为一个人,陆某的恩师江哲江先生数日前传信给我,说有人想要趁着陆某出使东海加害于我,如果贤兄妹想弄清楚为什么陆某知道这些事情,不如想法子去问问江先生吧。”

李显闻言骂道:“果然是他搞得鬼,不过陆灿,他怎么会去找你帮忙,这是怎么回事,难不成他忘了南楚多得是恨不得杀了他的敌人么?”

陆灿微微一笑,道:“齐王爷,我也很想家师能够回归南楚,可惜家师恐怕是再不会和南楚有什么纠葛了,不过是越氏的人质被软禁的地方,别人不大方便动手罢了,而且大概家师也还顾及我这个弟子,不忍我被人害了吧?”陆灿心道,我总不能告诉你越氏的人质就被软禁在建业禁军的军营里面吧。

越文翰眼光有些犹疑不定,不论他如何怨恨越无纠,如果担心自己的父亲尊长,可是有一件事情他还是很明白,就算是越无纠最后得胜,他的身上也流着越家的血,可是若是让外人控制了越家,那么自己才真是罪无可赦,想了一想,越文翰心中突然一亮,不论陆灿所说几分是真,但是自己的父亲很有可能已经在陆灿手上,那么自己和妹妹既然已经报了必死之心,那么就要看看如何作法会让越家得到最大的利益了。不过不论如何做,都不能伤害东海侯父子,毕竟只有他们才有可能和越家有共同的利益。

想到这里,越文翰笑道:“青烟,事情既然已经这样,我们也不用违背自己的心意了,你先让海涛醒过来吧。”越青烟轻轻点头,眼中闪过一丝愧疚,无论如何,姜海涛都是她的夫婿。躺在地上的姜海涛发出低低的呻吟声,不过片刻就苏醒过来,他一醒过来就握住了宝剑。不过却没有攻击越氏兄妹,他不是鲁莽的人。

陆灿淡淡一笑,道:“越少宗主,你需得记得一件事情,同心蛊虽然无可解救,但是并非不能驱避,东晋末年,因为同心蛊而造成无数惨案之后,天下名医无不研究它的破解法子,齐王殿下的辟邪宝玉是天赐奇珍,可以保护殿下不受蛊毒所害,苦海菩提也有这样的功效,可是还有一个秘方,可以制成香囊,佩戴者也可不受蛊毒所害,虽然时效不长,可是若是权贵人家,配个十服八服还是很容易的。”说罢,陆灿从怀中取出一个香囊,虽然距离很远,可是越青烟还是皱了皱眉,往后退了几步。

东海侯神色一变,道:“陆将军,这也是江先生给你的么?”

众人一听,就知道他是生了疑心,这样的事情江哲若是一点也不告诉他,未免有些过分。

陆灿苦笑道:“我倒希望可以这样说,可惜不是的,我事先并不知道越小姐仗以制敌的本事乃是同心蛊,我刚才所说有些是先生告知的,有些是猜测的,这个药方虽然难配,可是对于诸位来说都不是什么难事,只是同心蛊已有多年没有出现过,所以诸位没有准备罢了。这个香囊本是伏大人身上的东西,我幼时顽皮,倒也学过妙手空空的本事,伏大人又太紧张,下船之前几次用手去摸,所以我一进喜堂就摸了来,方才听说是同心蛊,在下可是庆幸不已呢。

南楚副使伏玉伦对众人来说只是一个微末人物,竟谁都没有注意过他,此时看去,只见他瘫倒在椅子上已经有半天了,众人原本道他书生无用,也没有理会,听陆灿这样一说,才发觉伏玉伦竟然被点了穴道,眼睛睁的大大的,满眼都是惊惧。

陆灿将手中的香囊凑到鼻子跟前,笑道:“伏大人是准备送我的灵柩回南楚的吧,只怕陆某是不能让你如愿了。越少宗主,你们的死亡名单上至少有两个人不会死,只要我们活着离去,越氏的命运也就定了,为了越氏着想,我想两位会做出更好的选择的。”

越文翰叹息了一声道:“罢了,越氏多行不义,也难怪会有今日,越氏落到将军手上总比别人好些,青烟,收回蛊毒吧,我们也没有必要替人火中取栗。”

越青烟答应一声,那些中了蛊毒的人都渐渐开始苏醒。

越无纠脸色灰白,此刻他心中只有一个念头,怎么那些人都不出现呢,若是那些人出现,有了将与会众人一网打尽的实力,文翰和青烟也会顺从我的意思。他开始向后移动,一定要和他们会合,他心中这样想着。这时,从后堂走出一个明艳的少妇,她手中提着长剑,剑尖上仍有鲜血,她看着越无纠,冷冷道:“大执事,你不用妄想去和他们会合了,我已经杀了他们安排在喜娘侍女中的内应,现在他们还不知道你已经失败了,或许等他们束手就擒之后,你还会有机会和他们相见。

越无纠看到那个女子,苦涩地道:“薛氏,凤舞堂首座说你是她们的人,你怎会背叛的。”

薛夫人神色冷然地道:“不错,我从前的确是她们的人,可是她们却忘记了我早已经和她们恩断义绝,不错,他们说可以保住相公的性命,还说会让我做正室,可是我嫁的是一个顶天立地的男儿,不是一个苟且偷生的傀儡,越无纠,你也不用因为失败而痛心,她们本就不打算将越家全部给你,她们留着相公的性命就是为了找个机会除掉你。”

越无纠苦涩地道:“与虎谋皮,我自然早有准备,绝不会让她们有控制越家的可能,只要保住越家的根基,得到海氏的机密,那么将来越氏独霸海上指日可期,他们想要控制的产业对我来说本就没有什么重要。只是薛氏,你真得不怕我将你的身世公开么,一个下堂妇,一个意图投毒杀害丈夫子嗣的女人,有何颜面留在文翰身边。”

薛夫人神色不变,淡淡道:“我从前做的错事,早已经得到惩罚,而且相公早就知道我的事情,你们想用这个威胁我,真是愚不可及。”

越无纠看看越文翰,见他果然神色平静,不由道:“原来你们夫妻失和都是假的。”

越文翰冷冷道:“不,我们还没有做作到那种地方,这段时间我和秋雪的确有了分歧。”

越无纠脸色变得平静了许多,道:“想必这外援是薛氏你自作主张,没有经过文翰同意吧?”

薛夫人没有说话,眉宇间多了一丝惆怅,越文翰却道:“大执事果然对我了如指掌,不错,秋雪瞒着我写了一封信给她的前夫,这件事情才是我不能谅解她的缘故。”

越无纠不由苦笑,道:“原来如此,薛夫人不愧是凤仪门弟子,竟然想出这样的迂回求救的法子,夫人的前夫裴将军如今是雍帝心腹大将,督军江北,枕戈待命,令南楚上下无不忧心忡忡,不敢稍有轻忽。而且我听说当年凤仪门事变之后,若不是他抱病上书为令尊求情,只怕令尊官职不保,可惜我始终以为女子量窄,想不到夫人竟然肯向他求救,若是他得了书信,知道越氏将对东海下手,自然会有所举动,可是怎么我看东海却似乎不知情呢?”

这时薛秋雪也只能苦笑了,自从来了东海,她每日都在盼望有人和自己联络,却是一个人都没有,若不是今日见到了柔蓝和江哲的近卫邪影李顺,只怕她会在拜堂之前就崩溃了呢。

齐王嘟囔道:“是不是随云又故弄玄虚?”

这时,门外传来一个清雅冰寒的声音道:“殿下可不要冤枉我家公子,薛夫人的信到得太晚了,裴将军得知此事之后立刻禀明皇上,皇上想法子通知了我家公子,可是离小侯爷大婚只有半月之期,而且平白无故的就说越氏有歹意,只怕侯爷也不敢相信吧,而且薛夫人的信说得也不详细,越小姐有什么手段也没有写明白。所以我家公子才千里传书,请陆将军救下越氏宗主,行釜底抽薪之计,只要越小姐不受威胁,那么一切就可以平安了。这也是凤仪门余孽和越大执事太贪心了,既想得到东海和越氏,又想对庆王爷和陆将军动手,贪心不足,所以肇祸,若非是他们想要对付陆将军,只怕我家公子也没有办法摆平这件事情呢。”

众人抬头看去,只见门口站着一个青衣少年,容貌清秀阴柔,眉宇间却带着从容淡然的神情,他的气质阴柔中带着孤傲,仿佛如同春日的积雪,虽然冰寒,却是似乎虽然都可以融化成明澈冰洁,无处不至的雪水。

越无纠已是心灰意冷,一个名字浮现在脑海里,他脱口道:“邪影李顺!”话音刚落,青衣少年凌空虚点,越无纠只觉得四肢无力,软倒在地,他心中惊叹,隔空点穴,绝望的闭上了眼睛。

青衣少年淡淡一笑,道:“正是在下,越执事,在下方才已经去了送亲的船上,幸好侯爷有先见之明,曾经给过在下调用东海军士的权力,所以方才在下调动了三艘战船和千余名军士,将越氏船上的所有人都擒住了,当然可惜的是,凤仪门的余孽实在是诡计多端,竟然提前下了船,不过这里是海外孤岛,想来他们还应该在这里。”

李显笑道:“小顺子,我可不信你的主子把所有希望都寄托在别人身上,快说吧,他的杀手锏是什么?”

李顺欠身道:“殿下明鉴,我家公子自然不敢大意,关系着这么多人的安危呢,公子说,既然是让越小姐出手,那么恐怕不会是靠武功,下毒是最大的可能,如果只是平常的毒药,只要小心一些,不让越小姐下毒成功也就是了,不过公子说,下毒是很难控制的,而且南闽越氏也没有擅长用毒的习惯,所以公子就想到了邪术或者蛊毒,公子命在下带来了一些药物和破邪的东西,不过公子也没有想到越小姐用的是同心蛊,在下带来的驱蛊药恐怕是很难管用的。若说杀手锏么?”

李顺顿了一下,拿出一个精巧的小圆筒,道:“这是可以放出火焰的飞天神火,可以放出三次火焰,这里面的火药乃是精心调配,一旦着身,就不能扑灭,公子说,不论是什么毒术邪术,一烧了之,大半都可以管用。”说罢李顺将圆筒指向喜堂中的一张椅子,轻轻按动圆筒上面的机关,果然弹出一道白色的火焰,那张椅子在火焰中片刻就化为乌有,就连灰烬也没有,更奇特的是,离它不到半尺的另外一张椅子却一点事情也没有。众人见了不由心中一跳,暗道,好厉害的火啊。他们都是身份高贵之人,知道很多不为常人知道的事情,这同心蛊当年能够被扑灭,就是靠用烈火焚烧,火焰,本就是蛊毒的克星。今日越青烟能够占了上风,不过是因为事先没有准备罢了。

林彤看着李顺那俊秀的面容,心中生出寒意,低声道:“姐姐,邪影就这样可怕,他的主子一定更加恐怖。”

林碧微微苦笑,心道,我若早知道李顺不在江哲身边,早就派人想法子找到江哲的下落,将他刺杀了。

这时,李顺又道:“侯爷,外面的事情还需要善后,在下多有不便,请侯爷作主。”

姜永深深的看了李顺一眼,心道,我尊敬江哲原本是为了他救了我的儿子,今日才见了他的锋芒,看来果然是不能再和大雍继续敌对下去了,否则我父子的性命都得葬送在他们手上。他扬声道:“涛儿,你去安抚一下宾客,就说越氏的大执事犯上作乱,已经被擒。”他看了一眼越青烟,心中有些犹豫,爱子大婚,天下皆知,若是就这么算了,岂不是贻笑天下,可是越青烟身上有同心蛊,不仅性命堪忧,而且这姑娘忠于越氏,就是嫁了给爱子,只怕也会有麻烦。他这里犹豫,李显却是心思剔透的人,他笑道:“青烟,你过来,你既然和海涛拜了堂,就是我的侄儿媳妇,六叔也没有什么见面礼给你,这块紫玉就给你了。”说着,他摘下紫玉,塞到了低着头走过来的越青烟手中。越青烟一愣,明净的容颜上露出了不可置信的惊疑神情。

李显正色道:“青烟,我虽然不懂得什么蛊术,可是这块紫玉至少可以压制你的同心蛊一段时间吧,就是不能,这也是我给侄儿媳妇的礼物,你这孩子虽然有些糊涂,可是我倒是很喜欢你的脾气,为了兄长练这种伤人伤己的邪术,我想你当初虽然不知道这同心蛊的害处,可是刺血喂食蛊王,这种勇气至少本王没有,听你们刚才的话,你这孩子是准备牺牲自己的性命救父兄了,所谓在家从父,本王不说你错了,只是如今你已经是姜家的媳妇,出嫁从夫,以后可不许擅做主张了,我这个侄儿虽然单纯些,可是爱恨分明,以后你要相夫教子,恪守妇道,知道么?”

越青烟强忍泪水,低声道:“青烟不知道公公和相公的意思如何?”

李显看看姜永和姜海涛,姜永想了一想,心道这个媳妇倒是性子强韧,若是好好教导,一定能成为涛儿的贤内助,也免得涛儿将来宦海覆舟,不过不知道她身上的蛊毒能不能驱除,想来想去,他既不愿驳了李显的面子,也不想让老妻难过,便道:“堂也拜过了,这个媳妇我自然认可。”

姜海涛却是性子单纯,方才恨不得杀了越青烟,可是如今却是面色红红地道:“全凭父亲和六叔作主。”

李显朗声笑道:“好了,薛氏,你先送青烟去新房吧,越文翰,你也跟着海涛去料理一下外面的残局,其他的事情我就不管了,如今总算大局已经平定,不过让大家小心些,凤仪门的余孽还没有踪迹呢?至于越氏的事情么,陆灿,你怎么说?”

陆灿淡淡道:“越氏自然还是南楚的越氏,我们南楚的海运还仰仗越氏呢?不过海氏应该不会介意继续和南楚商贾合作吧?”

东海侯和李显交换了一个眼色,现在越氏的宗主可还在陆灿手心里呢,东海侯笑道:“陆将军不用担心,只要有生意,海氏是不会拒绝的。”

李显拊掌道:“好啊,那就赶快重新摆宴吧,外面的事情交给海涛去做,咱们还得多喝几杯才是,这可是大喜之日呢。”

众人听了李显的话各自反应不同,东海侯等人都是苦笑应命,齐王爷的威风毕竟压人,庆王苏醒之后就铁青了脸不说话,但也没有作声,苟廉比较幸运,一直冷眼旁观,而且也没有他插话的余地,陆灿只是微微含笑,而身边的伏玉伦却是小心翼翼地望着陆灿,神色十分紧张。林碧面上带着淡淡的苦笑,而林彤则好奇的望着李顺,这可是她久闻其名的人物呢。

第十一章    静海之会

姜海涛,东海侯哲嗣,善水战,性忠勇,太宗爱之如亲子,大雍隆盛元年率东海部众降雍,平楚役中履立战功,大雍隆盛九年晋封靖海公,元配越氏,有贤名,然性端严,人传公有河东疾,越氏富才略,或有人言,公一应奏章文书,皆越氏掌管也。

《雍史·靖海公传》

夜色朦朦,林碧站在客房窗前,望着黯淡的星空,她身后一个中年近卫正在向她禀告探察到的情报。

“在喜堂上变乱的时候,所有的客人和我们这些随从近卫都被东海侯的属下围得死死的,东海侯练兵果然有不凡之处,越氏的大船被东海侯的水军摧毁得很厉害,我们去看过,海面上都是尸体和船舵船帆的碎片,那艘船若是不好好修理,恐怕是不能用了。”

林碧叹息道:“这里毕竟是东海侯的地盘,除非是大军来攻,百多个人想要捣乱,不过是火中取栗罢了,如果不是越青烟使用了早已失传的同心蛊,恐怕根本就不可能占到上风,说到这里,本宫倒是很佩服设下计策的人,若是他们成功了,不仅控制了东海、海氏和越氏,还让大雍和南楚损失惨重,至于我们,虽然得不到什么实际的好处,但也没有什么损失,想来那些人还想我们趁机进攻大雍呢。好端端的一桩婚事,既是亲上加亲,又是郎才女貌,谁会想到新娘子会暗藏杀机呢?这幕后主使可真是够深沉的心机啊,若是庆王、东海侯父子、陆灿一起死了,只怕天下顷刻之间就会大乱,也难为他们找到敢养同心蛊的人,也难得越青烟这份资质,据说修练同心蛊,对于蛊主的要求是很苛刻的。不过最令本宫震惊的还是江哲的应对,不过是短短的半月之期,这人就调动了一切可以调动的资源,一个釜底抽薪,让越氏兄妹再没有必死之心,一管飞天神火,足可以应对最不堪的情况。东海来了这么多人,是敌是友难以判断,可是这人就有本事让我们都随了他的计策行事。修先生,你说我们可以做什么,才能摆脱这个人的威胁?”

那个中年近卫犹豫了一下,道:“殿下,今次师尊派了我们过来,本来是希望能够帮助殿下铲除异己的,可是如今的局势,东海已经被惊动了,我们恐怕很难下手,那个李顺我们也见到了,这人武功之高,不是我们可以抵挡的,除了师尊之外,只怕无人能够稳操胜券,而且就是勉强进行刺杀,只怕也不能杀死江哲本人,反而和他结怨太深,此人心机阴毒,若是他誓死报复,我们反而得不偿失。”

林碧叹息道:“我也知道这个道理,可是此人若是重新出仕,就是我们的敌人了,我很担心庭飞会中了他的诡计。”

中年近卫傲然道:“殿下放心,大将军军略无双,又有我们保护,不论什么阴谋诡计,只要我们不去理它,哪里还会上当。战场上面乃是堂堂正正的厮杀,这人能起什么作用,而且我看他们也不会好到哪里去,听郡主说,那庆王李康对齐王李显恐怕已经是恨之入骨了,兄弟不合的迹象十分明显,我们助他一臂之力,说不定能够让大雍自毁长城呢?”

林碧叹了一口气,正要说话,这时外面传来林彤的声音道:“小妹妹,你来做什么啊?”

林碧心中一动,侧耳听去,外面传来一个小女孩稚嫩的声音道:“柔蓝奉父亲之命,请嘉平公主、红霞郡主前往静海山庄做客。”

外面传来林彤有些犹疑的声音道:“小妹妹,你的父亲是哪一位?”

小女孩得意地道:“我爹爹姓江名哲。”

林碧心中没有震惊,反而觉得心中畅快,她早就怀疑这个小女孩的身份,可惜对于江哲的情报,北汉只知道一些重要的事情,对于江哲的私事却很含糊,所以林碧不能确定罢了。听到这里,她推门而出,笑道:“柔蓝小姐,林碧得到令尊邀请,不胜荣幸,一定会前去赴会的。”

柔蓝高兴地道:“那就太好了。”

林碧仔细瞧去,只见柔蓝手中还有几张帖子,便笑着问道:“小妹妹还要去送帖子么?”

柔蓝道:“是啊,还有陆灿大将军的帖子,齐王殿下的帖子和庆王殿下的帖子呢。”

林彤道:“柔蓝,你年纪这么小,怎么不让别人送过来呢?”

柔蓝歪着头道:“这是爹爹给蓝蓝的任务,蓝蓝当然不能让别人做啊。”

林碧看着小柔蓝一脸的天真稚气和认真,不由一笑,心道:“能够养出这样可爱的女儿,我也应该去见见江哲呢。”

同样的星空下,陆灿心中也是愁肠百结,伏玉伦如今已经被软禁起来,要杀此人不过是举手之劳,可是想到此人乃是尚相的东床快婿,陆灿便有些犹豫不决了。

在即将出发的时候,陆灿接到了江哲的书信,心中聊聊数语,告知南楚有高官意欲图谋东海,趁机陷害自己,让陆灿寻找越氏宗族被软禁之处,心中提到了几个可能的地方,而陆灿的属下果然在禁军大营里面找到了越氏宗主。多年征战,如今的陆灿已经不会是那么天真的人了,他并不会因为江哲而做出损害南楚利益的事情,当时他想来想去,都觉得虽然尚维钧有心谋害,可是自己既然已经事先知道,那么保住性命也应该不难,而且若是事情成功,那么南楚得到的利益也让陆灿十分心动。可是思之再三,陆灿却发现自己不得不做了江哲的棋子,既然江哲已经得到情报,那么必然会事先设下圈套,到时候南楚必然失败,触怒了东海,只怕反而会损失惨重。而且尚维钧仗着凤仪门余孽的力量,这两年来气焰嚣张,虽然凤仪门已经成了过街老鼠,可是那的确仍然是一支强大的力量,凤仪门对于南楚来说是一柄双刃剑,用得好,可以对抗大雍,若是用不好,只怕祸起萧墙,就是他们侥幸取得了成功,只怕对于南楚也是祸非福。所以陆灿还是按照江哲的建议救出了越氏的人质,虽然他们被禁军软禁,可是凭着陆家在南楚军方的力量,还是让陆灿将人救了出来,而且还将消息封锁起来。而且越氏现在的主事人越文翰也承了自己的人情,这越家是不会太轻易的立刻投靠大雍了,而且看在越氏的面子上,东海也不能对南楚过分敌对。虽然等到图穷匕现的时候,越氏还是靠不住的,可是投靠大雍,在南楚背后下绊子这种事情大概是不会做了。说起来南楚也没有吃亏,可是陆灿心中却是郁闷难安,先生的计策越来越如天马行空,将来大雍和南楚敌对之日,自己能够应付么?想了半天,陆灿低声道:“先生,你素来喜欢离间之计,不知道离间你和大雍朝廷有没有可能呢?

他站起身,走到旁边的客房,这件客房门口有两个近卫宿卫,正是软禁伏玉伦的所在。陆灿走进去的时候,看见伏玉伦脸色苍白地坐在椅子上,他一看到陆灿进来,连忙上前拜倒道:“大将军,下官都是奉了岳父的命令,求大将军饶命。”

陆灿脸色淡然,道:“起来吧,我知道你作不了主,不过事已如此,你说我该怎么处置你呢?”

伏玉伦惊恐地道:“只求大将军饶命,但有所命,下官无不从命。”

陆灿微微一笑道:“我要你回去告诉尚相,我陆灿没有和他争权夺利的心思,可是也不容人欺到头上,我知道凤仪门余孽隐藏在尚相身边,我也不管尚相如何做法,可是我希望你提醒尚相,凤仪门素有反骨,可以用,却不能不防,若是尚相利用他们铲除异己,只怕到头来南楚反而成了他们的天下。”

伏玉伦心中一喜,知道自己的性命终于保住了,连忙指天誓日的承诺必然会劝告尚维钧。陆灿心中一叹,心道,若是我杀了此人,只怕只有谋反一条路可以走了,虽然此人将来可能会报复,可是总不能现在就和尚相弄得誓不两立啊。

走出伏玉伦的房间,陆灿对身边近卫道:“好好照顾伏大人,不可让他和外人接触。”在回到南楚之前,陆灿并不希望有他人可以影响伏玉伦,使他改变了答应缓解陆家和尚维钧之间矛盾的承诺。

刚走出几步,就看到一个小女孩蹦蹦跳跳地走了进来,手里举着一张大大的红帖子,身后跟着两个东海的侍卫,她一看到陆灿就笑着道:“陆师兄,蓝蓝替爹爹送帖子来了。”她好奇地看着这个青年,她已经从别人口中得知这个青年是爹爹第一个弟子,所以就趁着送帖子来看看这个大师兄。

陆灿已经知道这个女孩就是恩师的女儿,虽然不明白恩师怎会多出了这个女儿,可是并不妨碍陆灿从这个女孩身上寻找恩师的影子。他温和的上前,伸手抱起柔蓝,仔细看去,这个小女孩灵秀慧美,虽然年幼,可是眉宇间却已经有了几分恩师的神蕴。柔蓝好奇地道:“陆师兄,你也是带兵打仗的将军么?”

陆灿露出真心的笑容道:“是啊,我也带过兵。”

柔蓝做了一个鬼脸,道:“我还以为大将军都像麟弟弟的父亲那样威风呢,可是碧公主那样美丽,陆师兄这样斯文,原来大将军没有特定的样子的。”

陆灿又是一笑,放下柔蓝,收起情怀,接过帖子,看了之后淡淡道:“请师妹转告先生,就说陆灿不便前去祝贺,还请先生见谅。”

柔蓝奇怪的问道:“陆师兄,你怎么不去呢?我的小弟弟很可爱呢,你不想见见么?”

陆灿微微苦笑,若是自己真的去了,只怕是会惹起无数非议,自己虽然不在意,可是若是在这个时候落下这个话柄,还怎么带兵呢,现在可还不是他能够解甲归田的时候,东海之事,尚维钧也是不能理直气壮地指责他的,毕竟凤仪门余孽名义上是不能出现在南楚的,可是若是自己去拜访江哲,这个通敌之嫌就解释不清楚了。可是这些事情他又怎么和这个小女孩说呢,所以他只能淡淡道:“请转告先生,灿谨祝小师弟福寿绵绵,请恕灿不便登门之罪。”

柔蓝乖巧地道:“噢,我回去会告诉爹爹的。”说罢,又是蹦蹦跳跳地离开了陆灿的住处。

陆灿望着柔蓝的背影,心道,先生邀人参加小师弟的抓周盛宴到底有什么目的呢?

露出淡淡的苦笑,陆灿心里明白,不论自己去还是不去,都不能消除尚维钧对自己的怀疑猜忌,自己不过是想尚维钧不能名正言顺的出手罢了,若非如此,他倒是真的想去看看江哲要做些什么,就算是进了圈套也好过什么都不知道吧。脑海中突然浮现一个想法,他隐隐知道尚维钧和北汉是有着暗中的同盟协议的,他从前并不过问这些事情,可是今次在东海遇到了北汉军方的重要人物,嘉平公主,若是自己能够和她达成共识,那么对于南楚和北汉应该都有好处吧,虽然深夜求见有些失礼,可是嘉平公主总不至于将自己拒之门外,而且不论结果如何,都会让人误会我和北汉军方已经有了协议,对自己是只有好处的。望着迷蒙的夜色,陆灿心中苦涩非常,从前只想着杀敌报国,尽忠职守,想不到我陆灿也有苦心孤诣,只为了苟全性命的一天。

另一间客院里面,李显身穿宽松的便袍,倚在软榻上,双手枕在脑后,状似悠闲,但是他的眉宇间却带着一丝愁容,他不是迟钝的人,庆王充满恨意和嫉妒的眼神他看得很清楚,这次在东海,自己压了庆王的风头。这个三哥性子是阴沉还是偏激,李显始终拿不准。当年行刺纪贵妃一事虽然显出了李康矢志复仇的决心和勇气,可是凤仪门的高手,堂堂的贵妃娘娘,这样的刺杀也未免有些儿戏,这件事情也显示了李康不够冷静和偏激的一面。可是李显心中却曾经怀疑,如果李康不进行这样一次鲁莽的刺杀,是否会得到镇守东川的机会,而且李康这样将自己和凤仪门的仇恨摆在了明处,因为他皇子的身份,凤仪门反而不便对付他,若是李康有个三长两短,那么凤仪门就是最大的嫌疑犯。所以多年来,虽然李康总被凤仪门压制,但是不仅安全无虞,而且势力还在稳定的增长。若是李康真的早就想到了这些事情,那么李康的心机可不是“深沉“两字可以形容的。

而且李显也明白自己现在的处境,若是皇兄李贽稍微动了一丝怀疑忌惮,那么一定是群起而攻的格局,到时候自己就是失去兵权,也还是轻的,恐怕只有圈禁至死的可能,这个时候,自己又大大得罪了三哥,庆王李康,现在这个朝廷中身份最尊贵的亲王。其实李显很明白,只要自己亲自去见李贽,认真请罪服软,那么扭转现在的困境不是不可能的,可是只要想到屈膝于李贽,李显心中就是一阵郁闷,那个自己追在他后面想要压过的皇兄如今已经是大雍天子,九五至尊,自己若是向他低头,岂不是也成了为了苟全性命富贵而奴颜婢膝的软骨头么?越想越是苦恼,李显心想,需得快些见到江哲,他隐隐感觉,唯一能够让他摆脱这个僵局的恐怕只有那个文弱的书生。

想起江哲,李显心中泛起一阵暖意,这个人啊,南楚初见,他对自己是冷淡而戒备的,可是不知怎地,他总是觉得这个青年文弱的体魄隐藏着某种令人惊惧的力量,第二次见面,这人和自己狭路相逢,他救了自己的性命,虽然十有八九是因为为了从雍军手中脱身。虽然自己知恩图报放过了他,可是心中的遗憾却是十分深重。然后江哲被皇兄带回了大雍,解衣推食,想要招揽他,江哲却答应了自己的招揽,当时自己是不可置信的惊喜,可是最后这还是一场闹剧,带着愤怒离开雍王府的时候,自己是恨不得杀了他的,可是接下来他遇刺重伤,可是自己想到的第一个念头就是要救他的性命。后来太子和雍王之间誓不两立,猎宫惨变,自己也被软禁,自己为了种种原因挟持了江哲,不管是为了保住他的性命还是将他当成人质,可是总归是救了他的性命,自己原没有挟恩图报的意思,所以事后被雍王软禁之时,他也从没有希望过江哲救自己性命。可是这人却是滴水之恩报以涌泉,先是让自己和他一起做凤仪门主的人质,使得自己有了“戴罪立功”的机会,然后北汉趁机进攻,也是此人留言推荐,自己才有机会重披战袍。李显心中早就将江哲当成了可以结交的好友,虽然此人心机深沉,可是却是一个重情重义的人,若是他将你当成自己人,那么就不用担心被他出卖。所以,这次他冒着被弹劾的危险到了东海,就是希望能够得到这个人的帮助,让自己摆脱目前的处境,在攻破北汉,平灭南楚之前,他李显绝不甘心就这么被陷害,大丈夫应该马革裹尸,死于沙场,怎能死在囚牢之中,小人构陷之下呢。

正在李显患得患失的时候,他身边的侍卫进来禀报道:“殿下,柔蓝小姐替江先生送来帖子,邀请殿下去静海山庄参加小公子的周岁喜宴。”

李显精神一震,总算来了正式的邀请了,他笑道:“让柔蓝进来。”

柔蓝走了进来,见到李显,乖巧地上前行礼叩见,上次船上见面,齐王的身份还没有挑明,自然没有人告诉柔蓝齐王的身份,而虽然过去曾经在大雍宫中见过齐王,但是当时柔蓝年纪还小,自然也不记得齐王的相貌,如今身份都已经明朗,柔蓝这次来见李显也就按照礼数拜见,她自幼就被雍王妃抚养,又多次进出宫廷,对于这些礼节自然十分熟悉,行礼叩头十分顺畅自然。

李显笑道:“柔蓝,快起来吧,你如今已经是长乐的女儿,也应该叫我一声舅舅,哪有这么多礼数。”说着,将柔蓝提起放到膝上,问道:“你爹爹和娘亲身体都好么,听说他们已经有了儿子,他们两个身子都弱,不知道你的小弟弟身子好不好。”

柔蓝兴奋地道:“小弟弟壮的很,而且都不喜欢哭,太爷说娘亲身子调养的好,小弟弟很健壮呢。爹爹和公主娘亲都很好,还常常驾舟出海呢,不过爹爹的头发都变成浅灰色了,听太爷说,是因为药力激的,不过以后爹爹就不用担心旧伤复发了。”

李显好奇地问道:“你的太爷是指谁啊?”

柔蓝忽闪了一下大眼睛,道:“舅舅不知道么,太爷姓桑的,爹爹和娘亲都将他当成祖父看待的。”

李显笑道:“原来是医圣桑先生,想来也是,随云离京之时,不说是病入膏肓也差不多了吧,果然只有桑先生才能救得了他。”

柔蓝摇头道:“太爷说,爹爹自己也可以医好的,不过会多花几年时间,而且效果也不会这么好。”

李显状似无意地问道:“你爹爹邀请了庆王没有?”

柔蓝道:“顺叔叔说,庆王殿下是陛下的使者,我去送帖子太不礼貌了,所以顺叔自己去了。”

李显会心的一笑,看来在江哲心目中,庆王不过是外人,想到今后就是庆王攻击自己,自己也有了有力的后援了。

这时,李显眼睛的余光看见一个小小的身影躲在内间门口犹豫着不敢出来,李显不由好笑,虽然麟儿表现出的态度有些冷淡不耐烦,可是看来他还是很想亲近小柔蓝呢,不过现在天色太晚了,柔蓝也得回去休息了,李显也只能装作没有看见,又问了柔蓝两句闲话,就让人送柔蓝回去了。送走了柔蓝,李显充满了期待,看来静海之会,自己会有心满意足的收获呢。

在新房之内,越青烟心中十分不安,喝完合卺酒之后,姜海涛就去料理善后了,而且越青烟也知道在自己蛊毒未解之前,是不能圆房的,可是她感觉到姜海涛在新房之内神色总是有些冷淡,不由心中忧虑。这时,薛秋雪走了进来,看到越青烟神色惶惶,笑道:“怎么了,这样紧张,我是来帮你卸装的,新郎今天不便过来,姑母说让我来陪你,免得你孤单。”

越青烟勉强一笑,在薛秋雪帮助下卸了钗环凤冠,她忐忑不安地道:“嫂子,你说相公是不是还生我的气呢?”

薛秋雪噗哧一声笑了,道:“傻孩子,小侯爷既然没有当面拒绝娶你,就是心中喜欢你,只是你还没入洞房,就让新郎昏倒在地,这面子上未免过不去,完成大礼之后,人已经娶到手了,新郎放下心了,就不免想起旧帐了,这些男子,没有不爱面子的,你哥哥不就是为了我向裴将军求救而跟我呕气么?”

越青烟羞涩地道:“嫂子,哥哥是吃醋呢,若非你想出法子,只怕我们一家骨肉离散,死于非命,哥哥不会和你闹别扭太久的,想起当初哥哥追求你的时候,可是就差掏出心肝给你看了。”

薛秋雪眼中闪过一丝羞涩和甜蜜,但是继而神色一正,道:“青烟,有件事情你哥哥让我嘱咐你,如今越氏的危机还没有过去,越无纠的死党虽然多半已经死在东海,可是越氏内部还是有他的人的,而且宗主他们落在陆灿手上,我们承了陆灿的人情,这人情迟早要还的,大雍和南楚终究不能和平共处,到时候我们越家还要有所选择,这次回去,你哥哥会接掌宗主之位,族内要进行清洗,南楚在南闽的势力也会增强,虽然你哥哥已经和海氏达成协议,得到了船图海图,可是也被迫将一部分海运的生意让给海氏,这样一来,十数年之内,我们都没有可能压过海氏,这一点你哥哥倒不担心,可是等到大雍和南楚起了战争的时候,海氏可以一心一意跟着东海,我们却是得左右摇摆,最后恐怕还是要壮士断腕,才能保全越氏,所以越氏今后的路可是艰难得很。”

越青烟深色焦急地道:“那么哥哥想让我做什么?”

薛秋雪坚定地说道:“青烟,你哥哥说,他只要你做一件事情,就是安心的当姜家的媳妇,不要为越家做任何损害姜家的事情,也不要为越家争取什么利益,这是男人的事情,你已经尽了做女儿,做妹妹的责任,现在你是姜家的媳妇,将来是孩子的母亲,你一切一切都要为姜家着想才行。”

越青烟明净的眼中满是泪水,道:“嫂子,可是我总不能看着哥哥受苦啊!”

薛秋雪安慰道:“傻孩子,你忘记了么,姜家和越家毕竟是姻亲,只要你得到他们的敬重喜爱,他们为了你自然会顾着越家,你若是失去了丈夫的爱重,那么你就帮不了越家了,所以记得,你只要做一个好妻子就行了,而且越家也不是那么容易就可以打垮的。”

越青烟狠狠的点头,道:“嫂子放心,青烟不会再被任何人利用,若是姜家以后怀恨,不肯帮助越氏,最多青烟和越氏同生共死罢了,青烟绝不会做出有违妇道的事情的。”

薛秋雪笑了,又道:“我们也已经接到邀请,参加静海山庄之会,到时候你也会陪着海涛去拜见他的恩师,而且你的蛊毒也要他想办法呢,所以你可以好好休息,那人身份超然,若是得到他的赏识,你在姜家的地位就会有很大的不同呢。”

越青烟眼中闪过一丝憧憬,道:“嫂子,我也很想看看江先生和长乐公主,听说他们是一对神仙眷侣呢?”

薛秋雪笑道:“那你很快就可以看到了,不过可别忘记了,那人智谋无双,称得上是天下最可怕的人呢。”

第十二章    有子足矣

大雍武威二十七年九月卅日,姜海涛的大婚虽然出了变故,可是毕竟顺利举行,为了不让客人败兴而归,奇珍会还是按期举行了,借着四方宾客如云的良机,奇珍会的成功自然会吸引更多的商贾投入远洋贸易,所以负责举办盛会的海无涯和海骊都是煞费苦心,难得大雍、北汉、南楚都有贵人在此,这邀请的帖子自然是早就送了过去,而且帖子后面还附着奇珍会上将要拍卖出售的珍宝的清单,其中不乏价值连城的异国珍宝,所以倒也引起了这几位在本国数一数二的重臣的兴趣。而且他们都接到了江哲的帖子,为了等候迎接的船只,也要待到十月二日的,这奇珍会若是不去参加,反而会让人以为东海之变对其有了特殊的影响,所以众人都参与了盛会。海仲英拿出来的异国珍宝果然是令人目不暇接,倒也令诸人觉得不虚此行。而参加了这次盛会的东海贵宾:齐王李显、庆王李康、嘉平公主林碧、红霞郡主林彤、南楚大将军陆灿和东海之主姜永、姜海涛父子,则是与会者中最吸引众人目光的人物,这些人都是举足轻重的各国重臣,他们的一举一动都有人留心在意,希望能够看到一丝端倪,毕竟谁都知道,当今天下,已经是战火熊熊,阴云密布的格局了,东海虽然暂时置身事外,可是一旦战起,这些身家都在各国的商贾,他们的身家性命可能就在这些人的一念之间了。

而十月初一日,南楚的坐舟首先离开了东海,陆灿在离开东海水军的势力范围之后,第一件事情就是到了船底的暗舱,去见一个本不应出现在南楚使节船上的人——韦膺。

神色冰冷,眼中带着阴蠡的韦膺看到陆灿走进,嘲讽地道:“陆将军很是谨慎呢,直到今日才来相见,不过不知道陆将军想如何处置在下呢?”

陆灿神色淡然地道:“韦首座不过是不敢惊动东海的人,所以才会束手就擒,而且难道首座不谢谢我的示警么?”

韦膺脸上露出一丝阴冷的笑容,道:“不错,我是应该谢谢陆将军的,陆将军遣人用本座和伏大人事先约定的信号,传来消息,所以本座带了属下避到南楚使节的船上,可惜等待本座的是陆将军的精卫,如今本座的属下都被陆将军杀的杀,擒的擒,如今船已离境,将军是来和韦某算帐的么,既然如此,还不如将本座交给大雍,这样一来,将军所得的好处不就更大了么?”

陆灿叹了一口气道:“首座何必说气话呢,这次的事情本将军也是身不由己,尚相准备借刀杀人,将陆某陷在东海,本将军确也想杀了伏玉伦和首座,也免得我南楚步上大雍的后尘,可是本将军清楚的很,我若是这样做了,就是和尚相翻脸了,尚相是国主的外祖,一手掌控朝中内政,若是将相不和,等不到大雍南下,我南楚也就完了,所以本座不杀你,你们对大雍心怀仇恨,我们南楚对大雍也是仇深似海,所谓同仇敌忾,若是你们想陷害于我,也要想想有没有人可以替我领兵上阵。”

韦膺沉默了片刻,道:“尚相想要自毁长城,我本是不赞同的,可是你是江哲的弟子,这一点尚相放心不下,我也不会忘记,而且凤仪门的事情,我做不了主,如果全部按照我的计划,绝不会让那薛秋雪有机可乘。”

陆灿正色道:“我和江先生虽然是师徒,可是我是南楚重臣,绝没有背叛君国的可能,而且说一句不客气的话,先生军略,我至少学了五成,我也不必妄自菲薄,这些年征战不休,我自信用兵不逊于任何人,我为将帅,至少可以抵御大雍锋芒,若是换了尚相的心腹领军,只怕南楚迟早覆亡,到时候你们再没有依托,如何向大雍复仇,今次相谈,我也不要你们支持我,只要你们不干涉南楚的军务,不起叛逆犯上的心思,其他的事情我也懒得过问。”

韦膺神色数变,道:“这件事情我一人不能作主。”

陆灿笑道:“我不急,如今我已经占了上风,所以你们可以慢慢考虑,其实以我的本心,是想将你们铲除的,只因你们虽然可以对大雍造成威胁,可是对本将军来说,你们更是南楚的乱源,可惜尚相对你们很重视,所以陆某也不能斩尽杀绝,这一次,我虽然杀了你属下多人,可是也是因为他们都是凶名在外的盗匪,我想韦首座也不会计较才对。”

韦膺淡淡一笑,对于这些被陆灿所杀的属下,他倒真是不是很在意,毕竟几个心腹都留了下来,那么就不算什么损失,只是这一点他却不便承认,免得落下一个薄情寡义的名声。

陆灿见韦膺已经心平气和,道:“不过本将军现在来见你,是有一件事情让你去办,这件事情你若是办得好了,也未必不能挽回损失。”

韦膺默不作声,只是露出询问的神色,陆灿压低了声音,说了一番话,韦膺纵是深沉,也是面色数变,良久才道:“陆将军果然够狠,这件事情若是成功,别说是你杀了我几个属下,就是你杀了伏玉伦,又有什么关碍,将军放心,这件事情韦膺必定拼尽全力,绝不敢有半点懈怠。”

陆灿眼中闪过一丝凄然,道:“既然如此,就请韦公子稍后下船,我已经准备好一切,只要公子赶到我所说的地方,将信物交给指定的人,或许就可以心愿得偿。”

韦膺露出了阴森的笑容,没有说话,可是面上却露出了得意和自信的神色。

十月二日,东海侯世子姜海涛亲领水军,护送齐王、林碧等人前往静海山庄,静海山庄地处蓬莱,路程并不遥远,清晨出发,不过两个时辰,就已经到了蓬莱,姜海涛站在船头,指着前面的小港湾对齐王等人道:“这里叫做眉月湾,以其状如新月而得名,这里水势平缓,就是海上起了大风浪,这里也不会收到影响,所以江先生特意拣了这里修建了静海山庄。六叔请看,静海山庄倚山面海,风景雅致,先生最喜欢凭栏观海,若是风和日丽的时候,还经常泛舟海上,小侄就曾经伺候过先生垂钓呢。”

这时,柔蓝拉着李麟走了过来,笑道:“舅舅,舅舅,爹爹最喜欢钓鱼,可是偏偏总是钓不起来,一直到现在,蓝蓝都没有吃过爹爹钓起来的鱼呢,就连蓝蓝都钓起过一条大鱼,这里可是四季都有好多好多的鱼虾的。”

姜海涛笑道:“是不是你被大鱼扯进海里的那一次,听说倒真是一条大鱼,不过不知道是人钓鱼还是鱼钓人?”

柔蓝一听气得双手叉腰,道:“涛哥哥最坏了,总是揭人家的短,啊,不跟你说了,爹爹娘亲在码头上呢。”说罢,柔蓝手舞足蹈地向着站在远处的小顺子冲了过去,熟练的在小顺子的协助下攀上了他的肩头,然后一边挥手一边喊叫道:“爹爹,娘亲,蓝蓝回来了,蓝蓝回来了。”

不过这时候,却没有留意她的激动兴奋了,所有的人目光都向岸上瞧去,就在山庄前面的小小的私人码头上,站着静海山庄的主人。

虽然距离尚远,可是众人几乎都是练武之人,大多人都能将岸上诸人的面貌看的清清楚楚。站在最前面的是一个青衣秀士,从面貌上看大概未到而立之年,虽然发色浅灰,两鬓星霜,可是只见他优雅从容的风采,眉宇间动人的光彩,就不会令人怀疑他已经接近垂暮之年,反而让他整个人流露出一种沉静幽冷的独特气质。而站在他身后半步的是一个风姿淡然如仙的清丽少妇,正是长乐公主。在长乐公主身后,站着一个年纪将近三旬却仍是未婚装束的秀丽女子,和一个十七八岁的少年,相貌灵秀中带着狡黠。

林彤的目光可没有去瞧江哲,虽然口中说着好奇,可是在她心里,那和姐姐齐名的长乐公主才是她最关心的人物,凭着敏锐的目光,林彤一直仔细打量着长乐公主,只见她相貌虽然清丽秀雅,可是比起自己姐妹来说却是逊色一筹,时近秋末,她身穿一身雨过天晴色的华贵衣裙,外罩秋香色披风,虽然只是站在那里,却是说不出的温婉高雅,乌黑的长发只用一根碧玉簪挽住,除了一对明珠耳饰之外,她周身上下再无一件首饰,华贵而素雅,正是她给人最深的感触。这时一阵冰凉的海风吹过,长乐公主柳眉一皱,回过头低声吩咐了一句什么,站在她身后的少年立刻将手中抱着的一袭玄色披风递给长乐公主,只见她上前一步对着那青衣秀士说了一句什么,距离还远,林彤自然听不见她在说些什么,只是见她柳眉轻蹙,微笑中带着嗔意,然后那青衣秀士接过披风披上,长乐公主露出淡淡的笑容,伸手替那青衣秀士系好披风。虽然只是简单随意的几个动作,可是那种平淡中蕴籍着的神情款款,却让林彤满腔敌意化成乌有,只觉得果然只有这样的女子,才配和姐姐相提并论。

站在码头上,我看着甲板上熟悉或者陌生的客人,心中涌起莫名的情绪,终于还是回到了天下纷争的战场上,虽然心中惋惜这段有生以来最平静快乐的日子的终结,可是我还是只能这样做。

我的目光从船上众人身上一一掠过,齐王李显,不仅丝毫不减当年的霸气,身上更是多了一些阴郁深沉和浓厚的杀气,看来这些年他还是十分自苦啊。而站在他身边不远处的男子,衣着华贵,相貌和李显有几分相似,神色疏离中带着高傲,这位一定是庆王李康了,在他身后目光炯炯,蓝衫飘飘的不正是数年不见的苟廉么。那两位身穿劲装大氅,身佩宝刀的女子,相貌一般的明艳,眉宇间更是英气逼人,这样的女中豪杰,定然是北汉的林氏姐妹。而站在姜海涛身边的少女,红衣似火,相貌如霜,也肯定是他的新婚夫人越青烟了。我将众人一一看过,然后目光落到了那个站在船头,肩上扛着大呼小叫的柔蓝的青衣少年身上,不由露出微笑,除了他之外,还有谁能够这么完美的完成这样的任务呢?

船停了,搭上了跳板,第一个下船的果然是柔蓝,几日不见,她似乎更加活泼,蹦蹦跳跳地就跑了下来,贞儿在我身后笑道:“蓝儿这个孩子就是这样顽皮活泼,说起来当初她可是皇嫂亲自教导抚育的,怎么性子还是这样急躁。”

我心虚的不敢搭话,这个十有八九是我调教出来的坏习惯,如果我不是总拿着各种零食逗她追着我跑,或许她会是一个小淑女吧。

这时候,蓝儿已经跑到身边,像小猴子一样蹦到我怀里,我勉力抱着她娇小的身躯,再次悲叹了一声,心中感叹,别人总说书生手无缚鸡之力,果然如此啊。我无奈而又苦恼地道:“蓝蓝,几天不见,你好像又重了。”

柔蓝小脸气得通红,报复地伸手来扯我的头发,我心里大叫糟糕,这时候贞儿给我解围道:“蓝儿,不要闹你爹爹了,还有客人在呢。”

柔蓝歪着头想了一下,不情不愿的从我身上跳了下去,站到了一边。

这时候李显已经一马当先地走到我和长乐公主面前,长乐公主上前一步裣衽行礼道:“六哥安好,不知道父皇和母后可康泰么?”

李显仔细打量了一下长乐公主,笑道:“父皇和太后娘娘身子都好,不过他们都很挂念你,你的胆子也够大的,堂堂一个公主殿下,就这么说走就走,可真让我刮目相看呢。”

长乐公主脸上飘过红云,也不理会这个调傥自己的六哥,又上前给庆王行礼,庆王和长乐公主几乎没有见过几面,亲情淡薄,虽然相互见礼,却只是礼数罢了。不过对于林碧姐妹,长乐公主倒是十分热情,她上前笑盈盈地道:“长乐久闻殿下声名,听说殿下在北汉镇守代州,战功显赫,乃是女子中的豪杰,长乐素来文弱,最是敬佩妹妹这样的女子,这次有幸邀请到公主参加小儿的抓周喜宴,真是荣幸之至。”

林碧也裣衽还礼道:“公主过谦了,碧亦久闻殿下侠骨冰心,蒙江先生邀请来到静海山庄,能够一见贤伉俪,才是碧的荣幸,匆匆前来,没有准备给令郎的贺礼,本是失礼之事,可是公主殿下和江先生都不是世俗中人,想必不会见怪。”

长乐公主忙道:“殿下不必客气,碧公主愿意前来,已经是随云和李贞之幸了。”这时长乐公主看见站在林碧身后的林彤,正打个一个呵欠,杏眼朦胧,似乎有些困倦。便道:“郡主可是有些疲倦么?若是不嫌弃,李贞可以安排郡主小憩片刻。”

林彤尴尬地点点头,她昨天晚上可是没有睡好觉呢,一心想着可以见到那对传奇的夫妻,在见到两人之后,兴奋之情一过,困意就涌了上来。

长乐公主微微一笑,道:“小六子,你伺候郡主先去休息一下,等到午间开宴的时候再请郡主过来。”

那个相貌灵秀狡黠的少年走了过来,伸手肃客。林彤不比林碧,一直在代州长大,将军府也没有宦官,又几乎没有去过北汉皇宫,看到长乐公主竟然让一个少年前来相陪,不由愣住了。林碧和长乐公主相视一笑,明白她一时懵懂住了。长乐公主轻笑道:“小六子是本宫母后亲赐的内侍,最是聪明伶俐,郡主若是有什么要求,只管问他就是。”林彤这才明白过来,赧然一笑,知道这个小六子是长乐公主从大雍皇宫里面带出来的太监,这才跟长乐公主和林碧行礼告退。

林碧虽然一直和长乐公主说着话,可是她眼睛的余光却是始终留心着江哲,毕竟那才是她最关心的人物。

我上前迎接两个大舅子的时候,心里满是尴尬,从前只觉得长乐跟我私奔只是我们两人之间的事情,毕竟我们两人都不欠大雍什么,可是今日见到齐王和庆王,明明应该是他们对我有所求才是,可是我却觉得浑身上下都不自在,完全没有了平日潇洒自若的心态。陪着小心,上前躬身一礼,道:“两位殿下莅临静海山庄,哲不胜荣幸。”

庆王露出温和的笑容,还礼道:“久闻随云才名,本王早就想见上一见,只可惜随云你效黄鹤杳然,令本王难觅仙踪,如今你和长乐已经成婚,等到回京之后就是驸马都尉的身份了,可不能再效范蠡子陵之行了,本王还想领教你安邦定国的才能呢?”

我微微一笑,心道,安邦定国自然有人可以去做,又不是非我不可,这庆王殿下未免有些太俗气了,不过碍着他的身份,我还是彬彬有礼地道:“殿下教诲,哲铭记于心。”

齐王却在旁边怪笑道:“好你个江哲,平日看你温文尔雅,一张口就是礼数,如今却拐走了长乐,连儿子也有了,本王可不知道是先给你一拳,替父皇和太后娘娘教训你一顿呢,还是先谢谢你让长乐容光焕发,再无昔日的苦楚辛酸。”

我含笑看看一听到齐王说出“给你一拳”就无声无息地站到了齐王身后的小顺子,道:“殿下饶命,哲的身体如今虽然康复,可是若是殿下饱以重拳,只怕哲的性命就没了,虽然我的性命殿下不用挂心,可是若是有人报复起来,只怕殿下就要吃苦头了。”

李显感觉到身后的丝丝凉气,连忙道:“开玩笑,开玩笑的,好了,外面海风太大,还是去看看我的小外甥吧,不知道是像你还是像长乐?”

我见李显服软,便也趁机下台,道:“哲在听涛阁安排了茶点,那里景致清幽,可以看海潮,赏日落,小儿的抓周之礼也在那里举行,定好了时间是午时,现在还有一个时辰的时间,就请诸位先到听涛阁品茗观海如何?”

这时,林碧已经跟着长乐公主走到我的身边,闻声笑道:“庆王殿下和齐王殿下乃是江先生的姻亲,若说客人,恐怕只有本宫算的上,本宫也正想凭栏观海呢。”

我的目光落在了林碧的身上,这位嘉平公主,身为北汉国主的甥女兼义女,世代镇守代州抵御蛮族的林家在北汉的地位十分崇高,身为当代林家的核心领袖,又具有公主的高贵身份,再加上和龙庭飞的婚约,这个女子可是关系到大雍能否将北汉纳入版图的重要人物,所以我才会邀请她来此,这次见面的机会,她会和我一样珍惜,能够有机会在这么近的距离研究自己的敌人,这并不是常有的机会,可惜我却没有机会先见到龙庭飞。

直到这时,姜海涛才有机会带着新妇前来拜见,我笑道:“虽然你是我的弟子,不过今日是来做客,就一起过去吧,端娘,你领着少夫人去拜见太爷吧。”这时候那个中年秀丽女子上前应诺,李显记忆力极佳,立刻认得这个女子就是从前长乐公主居住的翠鸾殿的尚仪,记得是姓周的,端娘大概是她的名字吧。越青烟来之前已经得知自己要去拜见的太爷就是医圣桑臣,能否重得生机与否就要看那人的医术了,不由十分紧张,从前她悍不畏死却是因为知道已无生机,如今却是曙光已现,自然是不甘心身死了。那中年女子似乎留意到了她的紧张,轻扶她的手臂,引领着她向山庄里面走去。越青烟心中虽然紧张,可是还是忍不住用眼睛的余光打量着静海山庄,毕竟这里的主人就是一手力挽狂澜的江哲啊。这一看不由心中更是多了些钦佩。越氏在南闽可以说是一方霸主,又是传承十几代的世家,自然是屋舍连绵,富丽堂皇,越青烟颇为擅长宫室布置,如今她用品鉴的目光看去,只见处处屋舍错落有致,亭台楼阁,花木扶疏,雅致清丽,薛萝藤蔓,青翠可爱,人行其间,只觉心旷神怡,无一处不动人。以微观著,这里的主人果然是非同寻常。

众人随着江哲夫妇沿着铺的平整的青石小路登上山顶,在广阔平坦的山顶上,一座飞丹流檐的二层六角形楼阁独自占据着这一方幽静,遗世而独立,孤高绝隐。这里就是听涛阁。

听涛阁是静海山庄地势最高的一处楼阁,四周百丈之内再没有可以挡住视线的树木和建筑,听涛阁外观端丽庄严,每个屋角都悬挂着黄铜风铃。一阵海风吹过,那些风铃叮咚作响,它们的样式位置都是经过精心设计的,各自有着微妙的不同,使得它们混合在一起的声音宛若天籁。

李显的目光落到了站在阁门口的一个蓝衫青年,这人相貌俊秀,肤色白皙晶莹,这人他是认得的,是江哲身边的侍卫董缺,不过两年多不见,虽然轮廓宛然,可是面貌却似乎有了许多细微的变化,李显差点认不出来他来了。对这个人,李显总是心中有些疑窦,虽然几年前曾在江哲身边见过他,可是他总是有意无意的避开自己,李显也曾经怀疑这人有些古怪,可是他军务繁忙,也懒得多费心思,今日重见,李显心中也只是闪过一个模糊的念头,便不再留意。

董缺上前禀报道:“公子,阁中一切均已经准备妥当。”

我满意的点点头,这个董缺这几年将静海山庄上下打理的妥妥当当,这个总管可是没有白当,小顺子现在除了我身边的事情几乎什么都不再过问了。带着众人上了听涛阁,因为今日有了外面的贵客,所以我自己的属下几乎都没有出现,只有盗骊、赤骥也跟着大家上了听涛阁,盗骊也还罢了,身为海氏的少主人,自然有资格入座,赤骥却是以我的旧日仆从身份来的,这样的身份原本是不能入阁的,倒是齐王将他当成侍从带上了听涛阁。所以听涛阁上除了静海山庄的人之外,就有了八个客人:齐王李显、庆王李康、嘉平公主林碧、苟廉、海无涯、海骊、姜海涛和赤骥。听涛阁二楼的花厅虽然十分宽阔,可是在中间摆着一张大木桌的时候,活动的范围就小了许多,所以诸位贵宾都更喜欢凭栏观海。

这里视野十分开阔,站在阁中可以俯瞰海湾内外的风光,海湾内侧风平浪静,碧波如镜,海湾外策却是峭壁如削,海浪湍急,这一座听涛阁可以同时看到碧海两种面貌,果然是一处绝好的观海楼阁。

不过在林碧心中却是想到,这座听涛阁可以将静海山庄上下景致一览无遗,若是在这里有一个武功高强的人物坐镇,那么就可以将整个山庄纳入保护之中了。

这时,董缺带着仆妇仆从送上来茶点,香茗配上精致的糕点,淡淡的香气立刻充满了整个楼阁,我向林碧施了一礼,道:“公主凤驾莅临,哲无以为谢,内子颇爱厨艺,听涛阁中所备茶点均是内子亲手准备,还请公主品尝。”

林碧含笑谢过,道:“江先生居住在这样的仙境,又有长乐公主相陪,这样的日子真是令人羡慕,怪不得先生不愿意理会世俗之事,其实碧真是羡慕先生,远离战争杀伐,不是什么人都有这样的福气的,敝履繁华,富贵浮云,真是令碧心中倾慕,我若是先生,是绝不会抛下这样的生活重新踏入红尘俗世的。”

我听了喜悦的一笑,道:“殿下还忘记了一件事情,所谓有子万事足,如今我儿女双全,这样的生活我可是不愿轻易放弃呢。”

李显一听面上变色,他此来的目的就是要将江哲请出东海,可是林碧这样说,分明是在暗中警告江哲不要介入大雍和北汉的争端,而江哲也似乎隐晦的表现了不愿脱离这样的生活的心意,虽然江哲是不可能和大雍撇清的,可是他也知道江哲对这样的生活似乎是十分喜爱,若是江哲不肯出山,就是李贽也不能过于迫他的。这样一想,李显不由更加苦恼,江哲特意邀请自己过来,不会是为了婉拒自己的要求吧?

第十三章    出卖爱子

这时,门外传来几个人轻微的脚步声,只听声音便知道不是练武之人,然后两个侍女推开了阁门,在几个侍女的簇拥下长乐公主抱着一个小婴孩走了进来。在她身后跟着的是柔蓝还有李麟,方才柔蓝陪着公主去抱孩子过来的时候,可没有忘记把他拉着。

李显第一个跳了起来,笑道:“我要看看这孩子是像长乐多些,还是像随云你多些。”当然除了见到小外甥的喜悦之外,他也想暂时避开这种尴尬的气氛,来日方长,大不了绑了人带走,李显烦恼地想着。不过他很快就把注意力集中在婴孩的身上。

虽然还不满一周岁,可是这个小婴孩却是精神十足,好奇的大眼睛滴溜溜的乱转,承袭了父母外貌的优点,虽然年纪还小,却可以看出将来长大也会是一个相貌清秀俊雅的少年。

李显越看越是觉得这个孩子的眼睛不知怎么像极了自己,忍不住伸手将孩子抱了过来,虽然已经有了几个子女,可是从来不会特意留心他们的李显本质上来说还不算是真正的父亲,所以抱着这个小婴儿对他来说简直比拿着刀枪还要艰苦。而且那柔软娇弱的婴儿身体也让李显手忙脚乱,唯恐力气过大伤到了他。不过这个小孩子似乎是精神十足,似乎也看出了李显的窘迫,咯咯地笑个不停。李显越发欢喜,忍不住伸手将他举得高高的。长乐公主惊叫道:“六哥,你不要吓到了慎儿。”谁知那个小婴儿不仅不害怕,反而欢声笑了起来。明亮的眼睛里面充满了兴奋和好奇。李显心中涌起一阵暖流,这个小小的婴儿第一次带给李显从未领略过的亲子之情。

帝王之家本来就是亲情淡漠,再加上昔日和秦铮并不和睦,所以对于他的嫡子李麟,李显从前并不关注,直到秦铮死后,李显心中愧疚,这才对李麟重视了起来,可是由于从前的疏远和李显心中的苦楚,对于李麟,李显更像是一个统帅、师长而非是父亲,他用心的教导李麟,希望即使不能继承王位,也能够让这个孩子承袭自己的衣钵,成为优秀的将军。可是对着这个小外甥,李显却是打从心里喜爱,一时间只恨这个孩子不是自己的儿子,这些年来,除了杀伐之外本已经是了无生趣的李显,却是第一次重新涌起对生命的渴望。

李麟怔怔的望着父亲,他从未见过父亲如此开心,这一刻他恨不得取代那个小婴儿,领略到父亲怀抱的温暖。这时,有一只手轻轻抚摸着他的头顶,他抬头望去,只见一个青衣秀士正含笑望着自己,眼光是那样的温暖,李麟只觉得泪水盈满了眼睛,他连忙侧过脸去,不想让人看见自己的懦弱。那个青衣秀士眼中闪过一丝怜惜,然后转过身去,笑道:“好了,殿下不要戏弄慎儿了,若是惊坏了他,贞儿会心痛的。”

李显依依不舍的将婴孩还给长乐公主,嘲笑道:“你不要装样子了,我可是听皇嫂和太子殿下说过,当初最爱欺负柔蓝的可是你吧。没见过这样的父亲,就知道欺负儿子女儿,不如把慎儿给我算了,免得受你这不良父亲的气。”

我一听可差点气歪了鼻子,这个齐王,从前就喜欢看我的好戏,每次见面一定是不忘了闹点别扭,双手怀抱,我冷笑道:“这儿子自然是不能给你了,不过好歹你也是他的舅舅,这样吧,你若是以后娶了王妃,生了一位嫡出的郡主,我就让慎儿叫你一声岳父如何。”

李显一听,脸色初时阴沉下来,他为了秦铮之事心有愧疚,不仅遣散了姬妾,而且还拒绝了,这是很多人都知道的事情,他可不会认为江哲不知道,心中自然有些恼恨。

但是不知怎地,过了片刻,他却渐渐觉得这个主意不错,若是这江慎做了自己的女婿,那么女婿也是半子,倒是会让自己心满意足,可是自己现在虽然有一两个女儿,都是庶出不说,年纪也比江慎大了许多,若是想要江慎作女婿,还真得再生个女儿出来。江哲让自己另娶王妃,生个嫡女,也不算是过分,毕竟江慎乃是长乐公主的长子,而且他的父亲又是这样的人物,这门亲事恐怕会有很多人惦记呢。

想到这里,李显心道就是为了这个女婿,也得娶个王妃才是。再说他也想到如今家中无人主持中馈,那些庶出子女也是无人管教,不过是让他们自生自灭罢了。若是有个显德的王妃替自己照顾,省去自己的麻烦不说,也不会耽误了那些儿女的将来。而且可能是看到江哲一家其乐融融,不由令李显有些愧疚,心道,所谓齐家治国,自己就连家事也是一团混乱,也难怪败给了皇上,长久以来因为夺嫡落败而郁结的心结居然有些松动。

心中执念消除,李显的脑子立刻灵活起来,立刻想到这倒是一个绝佳的好机会,连忙道:“那我们可说定了,若是我有了嫡女,将来慎儿可要做我的女婿。

我看看爱子,心道,儿子,你别怪我随随便便就定了你的终身,做了我的儿子,这婚姻之事恐怕是不能任凭你作主了,就是我不管,也会有人关心呢,齐王虽然性子执拗,可是倒是一个率直的人,他的女儿应该也会很出色呢。不过为了你的幸福,我就再多给你几个选择吧。想到这里,我又道:“那好,指腹为婚也不是没有过的事情,不过我也不想委屈了慎儿,这样吧,将来你多生几个女儿,让慎儿自己选择如何?”

李显也不在意自己将来的女儿被人挑选,道:“那你我就击掌明誓,约定此事,此事有这么多客人为证,长乐也在当面,这桩婚约你可不能抵赖。”

我微微一笑,心道,若是将来慎儿正是有了别的意中人,大不了将他逐出家门,让他自由自在也就罢了,他若不爱名利富贵,我只有高兴,难道还会怪他么,再说了,所谓青梅竹马,日久生情,将来慎儿和齐王的女儿有机会日日相见,若是那个女子还没本事让慎儿动心,那也怪不得我。这样想着,我举起手掌道:“当然不会抵赖,殿下若是有了嫡女,又和慎儿相配,这桩婚事就是殿下无心,哲还要登门求亲呢,除非慎儿不是我的儿子,否则这桩婚事就这么定了。”

李显虽然军略非凡,可是对于这等言语的细枝末节,自然不会留心,便也举起手掌,和我击掌为誓,约定了这桩指腹为婚的姻缘。

看着江哲和李显击掌明誓,阁中贵宾却是心思各异,林碧心中大叫不好,若是齐王因此和大雍上层和解,那么岂不是不利于我北汉,但她不露形色,只是微笑祝贺,李康心中觉得怒火熊熊,他可不想看着齐王又压到自己头上,就连对江哲也是生出了无穷的恨意,可是转念一想,这儿女之事岂是可以说有就有的,自己也未必没有机会搅散他们的好事,所以也没有露出什么痕迹。倒是苟廉真是心中欢喜,心道,齐王殿下虽然性子执拗,可是皇上对他倒是真的爱重,既然他答应娶妃,那么这可是一个弥补皇上和齐王之间感情的好机会,江哲果然是厉害,不过三言两语,就解决了这样一个大难题,若是皇上知道,不知道得多高兴呢。

不管众人什么心思,都是一派喜气洋洋,只有还不知道自己被父亲给出卖了的江慎好奇地看着那阁子中间摆着的大木桌上面形形色色的物件。不时伸手想去触摸那些东西,却是距离太远,没有办法碰到。忍不住,江慎脸上有些扭曲,眼看就要转化成倾盆大雨了。阁门一下子被撞开了,匆匆忙忙赶来的林彤高声问道:“开始抓周了么,抓到什么了?”小婴儿也被吓了一跳,眼泪还没有滴下就被吓回去了,忘记了哭闹的江慎,又是好奇的看向了林彤。

长乐公主微微一笑,她方才心中有些不快,心道,怎么随云也不和自己商量一句,就给慎儿指腹为婚。可是她毕竟出身皇室,自然知道越是身份高贵,越是没有可能自己择婚,不用说慎儿是自己的儿子,就是凭着江哲在大雍和皇兄心中的地位,搞不好就连自己夫妻二人也没有给儿子选择妻室的权力呢,如今江哲这样给儿子定了婚,倒也是未雨绸缪,若是能够让六哥回心转意,不再和皇兄对着干,这倒也是一件喜事。但见室内气氛密云不雨,庆王李康和嘉平公主林碧都是有些神思不属,正有些烦恼如何转圜,一见林彤鲁莽的闯了进来,便笑道:“郡主不用着急,还得过片刻呢,方才侍女已经去请郡主了,想必是和郡主错过了。

已经小睡了大半个时辰的林彤彻底清醒过来,尴尬地道了歉,退到林碧身后。长乐公主见时间已经差不多了,笑道:“随云,我看应该开始了,要不然慎儿可要着急了。”我看看慎儿好奇的目光,道:“那就让他去抓吧,我也很想看看慎儿会抓到什么呢?”

这抓周乃是流传已久的民俗,只要是稍微殷实的人家都会在子女周岁的时候遍邀亲友前来聚会,听涛阁中央的木桌上早就摆了许多东西,而江哲和长乐公主都不是寻常人,这抓周准备的物品也是十分精致贵重。

一个银盘里面放着一方金印,两个黑檀木盘,一个里面放着三本精装的书册,分别是《论语》、《老子》、《金刚经》,另外一个里面放着上好的湖笔、徽墨、宣纸、端砚,一个黄杨木盘里面放着算盘、元宝和帐册,一方红缎上面放着一具精心制作的白玉琴,长度只有半尺,一副墨玉水晶精制的围棋,价值连城,乌黑的铁盘里面放着一把短剑,一柄弯刀,都是绿鲨鱼皮鞘,金吞口,黄绒挽手,华贵非常。不过放在桌子最中间的却是一盒长乐公主亲自下厨制作的糕点,香气扑鼻,令人垂涎。

这些物品华贵非常,就是手掌权势富贵的齐王等人,也不免觉得有些过于奢侈,齐王看罢,笑道:“既然是我的女婿,那我也不能委屈了他。”说着从怀中取出一块晶莹剔透的紫玉兵符放到了桌子上面。

长乐公主惊道:“六哥,这可是你统率大军的兵符,这怎么好拿出来让慎儿抓取呢?”

李显笑道:“不过是应个景,就是慎儿抓住了,我也得收回来,不过是想看看这个孩子有没有带兵的命。”

我微微一笑,道:“殿下这么想恐怕要失望了,带兵之人,需得心狠,我看慎儿是个软心肠的人,恐怕是带不了兵的。”

李显挥手道:“这可不一定,谁是一生下来就心狠的,本王军中,很多勇士第一次上战场的时候连杀人都不敢,如今不也是杀人如麻,心狠如狼么?”

庆王眼中闪过一丝寒芒,笑道:“六弟这样热心,我这个三舅也不能不表示一下。”他从腰间解下一个有些陈旧的明黄荷包,上面绣着四爪金龙,荷包鼓鼓囊囊,却不知道里面是什么物事。

李显眼中闪过一丝迷惑,别人不知道,他可是知道这是什么东西,当年庆王的生母惨死之时,李显虽然有些瞧不起这个平素有些懦弱的兄长,却还是去安慰他,却看到李康抱着母妃的妆盒垂泪。李显虽然性子率直,也知道不该去打扰,便在暗中看着,当时庆王李康就从他母妃的妆盒里面取了一只玉镯放入身边的明黄荷包。而第二天李康就从皇宫消失了。多年之后李康再次出现在大雍朝廷上面之后,身边总是带着这个荷包,别人不去理会,李显却是记在心里,他也颇为感动庆王的孝心。只是一来他和庆王性子不合,二来,李显当时又是太子一党,所以没有庆王亲近,到了今日,两人之间已经是兄弟之情十分淡薄,难以挽回了,李显自然不会再提及当年想要安慰三哥的事情,所以李康也绝不会想到李显知道这荷包里面的物事。

我看着这个荷包,觉得有些奇怪,对于不明不白的东西,我是不会要的,因此说道:“不知道庆王殿下送了什么厚礼,若是太贵重,只怕小儿担当不起。”

李康笑道:“这件东西并不贵重,只是先母的一件遗物,若是令郎喜爱,说不定我们两家也有姻缘之份。”

我愣了一下,方才我刚刚说让慎儿做齐王的女婿,怎么庆王这就来提亲,这时,我看见庆王李康的目光落到了柔蓝身上,立刻明白过来,母亲的遗物自然是送给妻子或者儿媳的最好礼物,庆王竟然是想要柔蓝作他的儿媳妇。

心中的怒意再也不可遏制,虽然出卖了慎儿,小小年纪就给他订了婚事,可是这不代表我可以将柔蓝的婚事这样草草订下,在我心里,儿子自然是可以随便一些的,反正最不济将来可以让他逃家,女儿可是应该偏宠的,别说是庆王那不知好歹的儿子,就是大雍皇室任何一个子弟,也别想娶我的女儿。我的柔蓝将来要嫁一个爱她如同珍宝的男子,那些三妻四妾的皇室子弟怎配做柔蓝的夫婿。

脸上的神色变得漠然,我淡淡道:“殿下好意,哲心中感激,不过哲平生最疼惜这个女儿,她的婚事还要她自己愿意,如今蓝儿年纪还小,这婚姻之事还不便谈及。”

这番话可是丝毫没有给庆王面子,连我都有些担心他会翻脸,不过出乎我的意料,李康神色丝毫不变,笑道:“看来犬子是没有这个福气了,江先生的小姐,自然是金尊玉贵,理应有更好的良配了。”

这句话似是羡慕又似嘲讽,但是李康说来却是十分平和,我见他没有发作,心中也是暗暗松了口气,不由有些后悔将他请来,原本是为了他的身份,毕竟他是当朝的亲王,长乐的兄长,可是他这一来,不仅让我结了一个仇人,还使得大雍内部的矛盾落入外人的眼中,可惜我却不能碍着他的面子和表面上的和睦,就这样误了柔蓝的终身。看林碧眼中闪过若有所思的神采,也知道恐怕这次邀请她过来是有些得不偿失了。只是世间没有后悔药可以吃,心中暗道,罢了,以后总有法子弥补今日的损失的。我面上勉强露出笑容,道:“好了,贞儿,快让慎儿动手吧,我看他要忍耐不住了。”

长乐公主温柔的一笑,亲手将爱子放到木桌之上,任他自幼行动。江慎睁大了眼睛,露出欢喜好奇的神色,方才还急着想去拿那些有趣的物事,如今却是不肯伸手,只是仔细打量。

过了片刻,小婴孩开始移动,他迅速向中间爬起,拿起了一块香气扑鼻的糕点。。

我不由呻吟了一声,听说若是抓周的时候最先去抓糕点,代表着将来这孩子可能会好吃懒做,虽然客人多半会客气的恭维,说这孩子将来必定衣食周全。我原本想把糕点拿掉的,因为长乐的糕点连我都爱不释手,恐怕慎儿也抵不住诱惑,可是贞儿却说这是规矩,如今果不其然,慎儿第一个就去拿糕点。

这时,一直站在屋角沉默不语的小顺子突然笑了,我瞪了他一眼,道:“你笑什么?”

李顺笑道:“少爷果然是公子的儿子,公子不是跟奴才说过,当年公子抓周也是第一个就去抓糕点么?”

他这句话一说出来,屋子里面静默了片刻,然后李显大笑了起来,其他人虽然碍着我的面子,却也是笑容满面。我不觉有些尴尬,不过心想,这样一来,别人可就不会笑话慎儿了,小顺子虽然丢我的面子,可是保住了慎儿的面子,也算有功。

这时,慎儿已经放下了糕点,大眼睛滴溜溜转了一圈,伸手拿起了算盘,我心中一抖,然后笑道:“这也好,这也好,江某最是头痛管理帐目,如果不是有亲信属下帮忙管理产业,只怕江某早就一穷二白了。慎儿将来能够精明些,也免得败坏家业。”这些话倒不仅仅是安慰,我有本事创业,但是管理那些琐碎的帐目可是我最头疼的,幸好我一直扬长避短,不插手这些事情,若是慎儿精明一些,至少我不用担心他将来成了败家子。

然后慎儿丢下了算盘,伸手拿起了那柄精美的佩刀,我有些遗憾地想到:“明明还有一柄剑的,怎么不去拿呢,谁不知道佩剑之人往往文武双全,拿刀的可是鲁莽武夫居多呢。”

我有些心急地绕着桌子转了几圈,恨恨地道:“慎儿,你这小子怎么回事,当年为父可是第二个就拿起了文房四宝,你怎么对书本和笔墨一点都不中意。”阁中众人无不失笑,那些熟悉我的人还罢了,庆王李康和林碧、林彤心中都觉得好笑,想不到这才智过人的江随云,竟然也会有如此稚气的一面。不过我可没有留心他们的神色,一心一意地望着慎儿,希望他给我些面子。

这时,慎儿放下了佩刀,伸手向黑檀木盘伸去,我心中一喜,屏住了呼吸,生怕惊扰了他。慎儿的小手一扫,笔墨纸砚立刻乱成一团,他却伸手向另外一个盘子伸去,我心中暗喜,心道,若是拿了书本,也是极好的。果然慎儿伸手拿了一本书,然后小手好奇地撕扯起来。

我却只觉得脑子里嗡的一声,一把上前,拎着慎儿的衣领把他提了起来,大声骂道:“臭小子,你是怎么回事,当初为父我虽然也抓了一本老子,可是第一个可是拿了论语,你倒好,居然抱着金刚经不放,做什么不好,偏要去做和尚,岂有此理,快把这本书扔了,你就是一个字也不认得也没有关系,这和尚可是绝对不能做的。”

长乐公主哭笑不得,上前道:“随云,你不要冲动,拿了佛经也不过是和佛门有缘罢了,怎么就扯到做和尚呢,抓周不过是个仪式,哪有你这么当真的。快放手,别伤了慎儿。”

我赧然道:“是啊,是啊,是我太冲动了,谁让这小子不给我留点面子。”说罢我看看慎儿,担心他会不会受惊,可是不看还好,一看之下我是哭笑不得,只见慎儿虽然双手紧紧抱着金刚经,两只小脚却是晃晃当当,在那里荡起了秋千。

我悻悻地道:“家门不幸,怎么出了这么一个惫赖的小子。”

李显忍不住笑道:“随云,你也不用担心,我看这孩子顽皮活泼,将来可是习武的好材料。”

这时,小顺子突然目光一闪,向窗外望去,冷冷道:“何方高人莅临静海山庄,邪影李顺有礼了,还请现身。”

我心中一惊,这静海山庄虽然表面上看不出来,可是这庄中机关暗哨无数,怎会有人闯到这里还没有被人发觉呢?

这时,门外传来一声柔声的佛号,然后有人道:“李施主武功精进如此,真是令老衲佩服,方才老衲见江檀越正在驯子,不便打扰,还请恕罪。”

然后阁门被缓缓推开,一个身穿灰色袈裟的中年僧人含笑而立,我却觉得头皮一紧,赧然道:“慈真大师,江某可不是说做和尚不好,还请大师见谅。”心中暗叫倒霉,怎么偏偏我的话给这位宗师身份的高人听见,若是他以为我对佛门有偏见可怎么办。

慈真大师微微一笑,道:“老衲明白檀越心思,檀越年将而立,膝下只此一子,难免冀望甚深,不过老衲见这个孩子资质绝佳,若是檀越许可,老衲想收他为徒,不知檀越意下如何。”

我脱口就要拒绝,却看到小顺子轻轻摇头,我心中一动,这慈真大师断不会是想要我的儿子出家,慎儿乃是公主所出,又是我膝下独子,就是慈真大师再怎么爱才,也不会让慎儿去做和尚啊。

这时候,慈真大师又道:“裴云虽是我少林护法弟子,如今却是手握重兵,很多江湖上的事情都不便插手了,老衲见令郎品性资质都十分出众,所以心中喜爱,若是檀越首肯,老衲情愿将令郎收为关门弟子,还请檀越和公主殿下放心,看令郎面相,将来必是福寿绵绵,多子多孙的命,绝不会出家为僧的。”

我心中了然,或者慎儿资质非凡,不过我看这老和尚十有八九是为了握个人质在手,若是慎儿拜入少林,我和小顺子将来自然绝不可能和少林为难,这老和尚对我仍然是有所忌惮疑心呢。不过转念一想,慎儿总是没有一刻安静,看来是没有做文章的本事了,若是练武,除了慈真大师,天下可没有更好的师父了,少林的武功据说是天下最正宗的武学,练不好也不会练坏,拜了这样一位师父,将来还有谁敢为难慎儿呢?

这千种思绪一闪而过,我含笑道:“慎儿能够拜到大师门下,自然是他的福气,可是我们夫妻只有这一个儿子,只望他平安长大,承欢膝下,若是大师带他离去,岂不是太伤我们做父母的心肠么?”

慈真大师微微一笑,道:“老衲已经决定暂时在长安浮云寺挂单,如今令郎年纪还小,老衲也可暂时留在江先生家中施教。”

我心中大喜,道:“成交。”一言既出,才发觉失言,连忙道:“既然如此,哲多谢大师美意,不过大师怎会远来至此呢?”

\chapter{第十四章 心腹之患}

林碧站在甲板上,目光冰冷的望着渐渐远去的静海山庄,一个中年近卫走到她身后,禀报道:“殿下,不知道我们下一步应该怎么办?”

林碧轻蹙柳眉,道:“我初入静海山庄,仍然存了伺机动手的想法,可是静海山庄杀气隐伏,我便知道不可轻举妄动,原想既已知道静海山庄所在,或者会有良机,不料慈真大师竟会莅临东海,让本宫十分庆幸没有擅自发动,看来我们只有在途中刺杀了。”

中年近卫皱眉道:“可是慈真大师不是奉了大雍皇帝的旨意,前来迎接长乐公主和江哲回长安的么,他们一路上都会有重兵保护,还有慈真大师和邪影李顺这样的人物保护,就是师尊亲至怕也是无能为力吧,若是平白损兵折将,未免太可惜了。”

林碧没有回答他的疑问,反而轻拂秀发道:“萧护卫,你久在庭飞身边,又是国师弟子,眼力自然是非同反响。你对齐王和江哲两人如何看法?”

萧护卫虽然没有目睹听涛阁上面的情形,却也早已听过林彤绘声绘色的讲述,犹豫了一下,道:“齐王确是名将,但是比起大将军还差得远呢,行动举止未免过于嚣张,威势凌人,或者有可乘之机。至于江哲,属下觉得十分好笑,属下曾经听说此人才智过人,可是听了郡主所说,怎么觉得此人像个长不大的孩子,让属下都有些怀疑他是否真的是那个神机妙算的雍王首席谋士了。”

林碧淡淡一笑,道:“本宫初时也觉得好笑,未见江哲之前,我心中想着他是一个惊才绝艳,心思周密的奇才,海边初见,本宫觉得他飘然出尘,不类世间之人,可是听涛阁上却是让我开了眼界,这个江哲倒是赤子心肠,可是这正是他可怕之处。从前我只是对他戒备,如今却是对他恐惧。”

萧护卫奇怪地问道:“虽然他的举止有些好笑,可是公主若是说他善于掩饰,属下也不会觉得奇怪,可是为什么公主认为那是他的本色,却又认为他更加可怕,属下也曾学过兵法战策,都说为将者要冷静无情,才能战无不胜,我想这出谋划策也是一样,不是说智者无情么,若是江哲尚有感情上的弱点,怎么公主反而认为他更加可怕呢?”

林碧眼神变得幽深,道:“我林家时代为将,虽然称不上兵法大家,可是却也有些独到的心得。有人说带兵打仗需要冷酷无情,这倒也不错,可是根据我们多年领兵的经验,若是敌军主将完全的无情,只按照兵法和形势用兵,倒是十九必败的。主将若是过分无情,就会将麾下将士不当人,也就更加不会把敌军将士当成人,这样虽然可以几乎不被情感所误,可是打仗靠的是士兵,主将可以无情冷静,他麾下将士却是有血有肉的人,会恐惧,会仇恨,这样用兵,终究是众叛亲离的下场。

做谋士也是一样,谋士的等级可以粗略的分为三等,第三等的谋士虽然各有长才,但是也各有弱点,若是互斗起来,不过是各有胜负,这等人不需畏惧,第二等的谋士就是心性冷酷无情,他们心中只有利益的存在,这样的人物虽然可怕,却也有着可乘之机,毕竟人孰无情,这样的人虽然计策厉害,可是往往低估了被他们计算的对手的感情因素,自古以来,枭雄往往死于非命,精于谋算的人往往自作自受,就是因为他们忘记了对于某些人来说,利益权势抵不过忠义亲情,而且一个人若是心中只有利益,那么所作所为就是有迹可寻,这样一来,若是他们的对手富有智谋,就可以猜到他们的计策,只要力量充分,取胜就不难了。而最可怕的一等谋士,就是本身也有丰富的感情,可是出谋划策的时候却可以屏弃感情的影响,这样的谋士已经是凤毛麟角,难以对付了,可是这样的谋士也有弱点,他们的才华和心机往往让人心生忌惮,不愿和他们接近,所以他们往往会难以尽情发挥自己的才能,也难以让身边的人尽心尽力的执行他们的计策。这三等谋士虽然可怕,可是都还有可以着手的弱点。可是江哲却不一样,他已经超越了这个界限。

你也见识过他的计策,洞彻人心,如同弱水,无孔不入,最善于利用一切可以利用的外力,对人心洞如观火,可是今日我一见他,便知道这人最可怕之处就是他的赤子之心,不论他用计如何歹毒,可是他对身边的人却是一片赤诚,这样一来,他身边就不会有人掣肘,就可以完全发挥他的才能。你也听说了,不仅大雍皇帝李贽对他推心置腹,就是和李贽一向不和的齐王李显对此人也是十分厚爱,竟然不会因为江哲触动他的逆鳞而震怒。如今,他的儿子成了齐王未来的女婿,又是少林慈真大师的关门弟子,就连少林也不再将他视作潜在的威胁,这样一个人,既有惊天动地的手段,又有春风化雨的魅力,有他在,大雍就不会再有内乱纷争,你说,这人是不是可怕得很。说一句心里话,此人乃是我北汉的心腹之患,他一日不死,我一日不能心安。”

萧护卫眼中闪过杀机,道:“莫不如我们派出人去,想法子不惜一切代价暗杀了他如何?”

林碧不置可否,又道:“你认为齐王比起庭飞来说,军略孰高孰低?”

萧护卫惊道:“殿下怎会这样问,那齐王怎比得上大将军,不说他这些年来在大将军手上从来没有讨过好去,就是在南楚德亲王手上,他不也是惨败而归么?”

林碧叹了一口气,道:“其实用兵之道,说起来虽然复杂无比,可是实际上也不过是领精兵、知进退罢了,这就已经是难得的名将了,若是再能够偶出奇兵,当世之间也只有一二人可以做到。齐王也是军略不俗的人,当时名将,若论临阵指挥,只怕是无人能出齐王之右,而大雍铁骑精锐不在我北汉军之下,可是齐王始终被庭飞压制,就是在南楚也是落败而归,就是因为他天性执拗,争强好胜。这样的性子虽然有些好处,在落败之时,常常百折不挠,履败履战,终有胜利的一日,可是也常会当退不退,以至被人所乘。齐王个性高傲,轻易不肯服人,若是劝谏之人不是他心里敬佩之人,往往就会无功而返,所以他在北疆数年,也不能胜过庭飞,只不过齐王确实有将帅之才,而且经历夺嫡之变之后,性子也隐忍了许多,这才维持了大雍北方疆界的稳定。这次见到齐王,我原本并不担忧,因为他虽然气势逼人,可是却是性子依旧执拗难改,而且他心中死志胜过求生的意念,本宫原本想回去之后告知庭飞,让他可以从这个方向着手。

可是齐王见到江哲之后就不同了,那种执拗的心志变成了绕指柔,而且性子开朗了许多,甚至就连从前的死志也变成了生气勃勃,这样的齐王不是我想见到的,而我更担心江哲留在齐王身边,有这样一个齐王爱重的谋士替他出谋划策,庭飞的压力就太大了。”

萧护卫道:“殿下,那江哲不是要回长安么,我们想法子不让他到齐王军中就是了。”

林碧冷冷一笑,道:“我可不信江哲真的会回去长安,这种情形之下,恐怕他会直接跟着齐王去军中吧,齐王的性格很霸道,恐怕就是江哲想要陪着长乐公主回长安,他也不会放人的。”

萧护卫惊道:“不可能吧,慈真大师可是来传旨的,江哲难道敢违背大雍皇帝的旨意么,而且他就不担心雍帝怀疑他和齐王勾结么?”

林碧微笑道:“你可看到圣旨了么,不是就听到慈真大师的传话么,你怎么知道真正的旨意是什么,而且,江哲可是会凛遵圣旨的人么?”

萧护卫道:“那么殿下如何打算呢?”

林碧看看远处的天空,道:“我倒要看看江哲有没有这个本事进入大雍军营。齐王、江哲,你们都是我北汉的心腹之患,我是绝不会让你们轻而易举的到达战场的。陌路相逢成知己,他年沙场见此心。李显啊李显,你可还有生死无恨的胸怀么?”

曙光刚刚透出厚厚的云层,沉静的旷野就被清脆的马蹄声和车轮声惊醒,在空旷的官道上,一辆外面罩着青色布幔的四轮马车在四百多名骑士的拱卫下轻快的前进着,这四百多名骑士分为四队,一队开路,一队断后,另外两队则是左右护持着马车。他们的衣甲颜色也是分为两色,护着马车的两队骑士,一队都是黑衣黑甲,一队则是赤色衣甲,而前后的两队骑士则都是赤色衣甲。若是深知大雍军队详情的人见了定要惊疑不定,只因大雍军中不论是何人的军队,基本上都是穿着青色衣甲,青色近黑,但是除了一支军队之外,其他军队绝不会穿着纯黑色的衣甲,那支军队就是雍王的近卫军。除此之外,齐王近卫是赤色衣甲,秦彝军近卫则喜穿白色衣甲,禁军则是黄色衣甲。如今雍王登基,原先的禁军改称龙骧卫,仍旧负责拱卫皇城,但把从前的雍王近卫军则改称虎赍卫,拱卫宫城,龙骧虎赍并称禁军,虎赍卫服色仍然尚黑,并没有因为负责保护天子而改变从前的作风。如今大雍境内谁不知道,除了禁军之外,黑色衣甲不是谁都可以穿着的。那么这支将近百人的骑士就只有可能是大雍皇帝李贽身边亲信的虎赍卫了。可是和他们一起保护马车的却是齐王的近卫,不由令人惊疑这车中之人的身份了。

我笑着看向眉头紧锁的齐王,道:“殿下,这次我请东海侯襄助,将东海封锁了半个月,林碧定然没有机会提前传信回去,凭着我们这支力量,应该可以平安的回转殿下的大营了。殿下为什么还要忧心忡忡呢?”

齐王叹了口气道:“我也相信北汉没有办法将情报送出去,直到昨日我才知道你邀请林碧就是为了限制北汉的行动,再调动东海侯的势力襄助,就是为了防止北汉大军提前得到情报,在路上伏杀我们,毕竟这条路离边境不过几十里路,若是北汉骑兵在路上伏击,我们是很难逃生的,这一带接近边境,大雍的军力并不能占着绝对的上风,我又不能调动过多的军队来保护,免得打草惊蛇或者被人所乘。但是我带了三百亲卫,皇兄又派了一百虎赍做你的亲卫,有这四百骑兵,就算是遇上了敌人,我们也能寻机突围或者固守待援。再说,若是没有数日的时间策划,我可不信北汉有本事布下天罗地网。”

说到这里,齐王失笑道:“说起来你和皇上也够谨慎了,谁会想到虎赍卫竟然已经等在滨州了,而且慈真大师一到,林碧一走,你就立刻启程,恐怕现在林碧还落在我们后面呢,就是现在林碧已经传回去了消息,也来不及了。”

我不由问道:“既然如此殿下为什么还是如此忧心呢?莫非是担心李麟么,有贞儿照顾他,你还不放心么,麟儿年纪还小,就是再着急,也不能让他现在就上战场啊,这次让他跟着贞儿回京,你不用担心的。一路上可是有慈真大师保护呢。”

李显又是皱了皱眉头,道:“我知道长乐会好好保护麟儿,我原本就不担心,可是不知怎么,我总觉得我们忽略了什么?我们虽然暂时切断了林碧和北汉的情报通道,可是北汉可是有魔宗高手的,若是他们已经得到了消息,我担心他们会半路伏杀,随云你不通武艺,若是遇上敌军,我担心你的安危。”

我轻笑道:“这点险总是要冒的,这是最近的路,快马加鞭,五六天之内就可以到达大营,到时候有三十万大军保护,殿下就不用担心了,就是遇上了敌军,有殿下指挥拒敌,哲也是放心的,再说小顺子也随军同行,最不济也可以护着我杀出重围。”

李显的眉头略微舒展,说也奇怪,本来他也觉得江哲的计划十分缜密,可是一路上李显觉察到有些异样,这一带本来常有北汉游骑游弋,可是这次回程却是一个都没有碰上,这反而令李显感觉有一种山雨欲来风满楼的危险。

这时有人在车外禀道:“殿下,江大人,有些不对,派出去的斥候没有了回音。”

李显眉头紧锁,道:“前面地势平坦,正是骑兵发挥威力的战场,不可不防,派两个人,不要骑马,到前面去看一看。”两个齐王近卫下马,脱去沉重的铁甲,换上便衣,隐入路边的草丛里,转眼间就消失在我的视线里。

我隔着窗子向四下望去,只见深秋时节,道路两边的枯草高可过人,秋风一吹,波浪起伏,再加上这条道路地势颇高,两边则是斜斜向下延伸,就是藏了千军万马也未必能够看出来。我的心里也不由一寒。莫非真的有伏兵么,难道我设下那多么迷障,希望他们认为我正在准备返回长安的,却都落空了么?可是这么短的时间,他们怎么可能设下埋伏呢,虽然现在大雍的边境几乎对于北汉是不设防的,可是并不代表他们可以不经过周密的计划就进入大雍境内行动。

这时候前方突然传来尖利的铜哨声,我心中一抖,这是斥候传来的警报声,他们竟然没有悄然返回,看来已经遇到了极大的危险,看来不仅有伏兵,而且恐怕规模还很庞大。

齐王听到警报声,剑眉一轩,跳下马车,翻身上了一匹火红的战马,护卫的骑兵都是训练有素的军士,很快的摆开了迎敌突围的阵形。一直负责驾车的小顺子脸上浮现出了忧色,他检查了一下驾车的马匹,低声问道:“公子,马车速度太慢,恐怕跟不上,我们怎么办?”

身临险境,我反而平静下来,冷静地道:“你我不善于临阵指挥,这些事情自有齐王殿下操心,这辆马车乃是精钢制成,车厢四周都嵌着钢板,只要护住了马车,就不用担心我的安危,一会儿,你也听从殿下的吩咐,这种情况,恐怕若不出奇制胜,我们都会死在这里。”

这时,齐王已经下令改道向大雍境内奔去,看来他是想和这一带镇守的雍军会合,就在这时,四面八方号角声起,我已经将车窗里面的钢板挡上,透过车门上面的小窗子向外望去,只见视野里已经出现了风驰电掣的北汉骑兵,人如虎,马似龙,他们穿插纵横,很快就遥遥将我们这支军队包围起来。我用心算了一下,敌军至少有三千人以上,这绝对不是一次偶遇,北汉进入大雍境内游弋的军队一般都是百人左右的小队骑兵。这时,我看见正前方竖起的一面黑底红字的大旗,那上面写着一个大大的“石”字。

“飞虎将军石英,”我听到齐王高声喊道:“你竟敢潜入我大雍境内肆虐,可是不将我大雍看在眼里么?”

那大旗下面一个三十多岁,面庞瘦削,神情冷峻的一个中年将领也高声道:“北汉大雍乃是敌国,齐王殿下不知自重,轻身涉险,也怪不得石某,今日你是有死无生,若是下马投降,或者我看在你身份尊贵,不取你的性命。”那个将领身边有一个身穿北汉军衣甲的军士,面甲已经放下,看不到他的容貌,我可以凭着超绝的目力看见那个军士正在指着我的马车的方向说着什么。然后我就看见那个中年将军的目光也落到了马车上,那冰寒的目光扫过,我只觉得浑身冰寒。这时,那个中年将军大声喝道:“儿郎们,给我杀了李显,俘虏那辆马车,谁能给我带来李显的人头,赏金百斤,谁能给我俘虏那辆马车上面的人,赏金千金。”

然后,那个中年将军合上面甲,手中的马槊一举,带头冲了下来,而齐王李显则是冷冷一笑,手中的宝刀向前一指,高声道:“突围!”说罢,我就觉察到马车开始迅速跑起来,我连忙紧紧握住车厢壁上的把手,门上的小窗口也被小顺子从外面关上了。车厢里面一片漆黑,我看不到外面的战场,可是我能够感觉到四周震耳欲聋的喊杀声。

这一刻,*默默苦笑,我已经想通了很多事情,为什么林碧明明还在我们后面,可是伏兵似乎已经等了很久的样子,只因我错估了一个人,陆灿,只有可能是陆灿,他去见林碧,不是为了结盟或者别的什么,而是为了和林碧达成一个协议。南楚负责传递情报,北汉负责伏击,不论我如何足智多谋,对着千军万马也只有一个下场。林碧和陆灿倒是都明白以拙胜巧的道理。说也奇怪,我本来应该心中悲凉,我平生第一个弟子陆灿,就这样下了狠心,要将我这个师父送入黄泉。可是我心中却是有些隐隐的欢喜,在我看来,陆灿本就少了几分狠心和固执,如今的他才可以说是我的得意弟子啊。默默的听着外面的声音,我知道在这里我是派不上什么用场的,如果死在这里,不知道是否会是笑话呢?

第十五章    水深火热

又过了片刻,我的心情终于平静下来,仔细的盘算着如何应对现在的局面,四百对三千,双方都是精锐,可是我们这边有我这个累赘,恐怕是想逃也逃不掉,陆灿的事情以后再想不迟,现在逃命才是当务之急。我强迫自己忘记身处颠簸黑暗的车上这个事实,仔细的想着如何能够自救救人。过了片刻,我心中突然一亮,那个飞虎将军传令要生擒我,杀死齐王,看来对他来说,我和齐王的重要性并不一样,从赏金上来看,似乎我对他们重要一些,可是在我看来,却非是如此。对于带兵的将领,自然是齐王的生死对他们更重要,而我的重要性恐怕这些将领未必能够领会,对他们来说,可能我只是一个他们需要完成的任务才对,或许就是因为这个缘故,我的赏金才要高些,这是为了避免那些将士不要都只顾着去追齐王吧。就算是我估计错误,他们将我看得比齐王还重,对于我的计划来说,也没有什么大碍。

我正在这样想着,突然车门被人用力拉开,我看见衣衫尽是鲜血的齐王对着我喊道:“随云,我们必须分兵才行。”

我心道,果然是英雄所见略同,连忙道:“哲也正有此意。”探出头去,我看见原来我们到了一处狭窄的路口,两侧都是山岩,齐王命人阻住路口,暂时遏制住了追兵。我连忙从车上拿了那件特制的青色大氅,系在身上,然后吩咐小顺子道:“你快些换上衣甲,然后带我乘马,我们和齐王殿下分道而行,请殿下再分五十人给我,想来这样就可以让敌军力量分散了。”

齐王眼中闪过欣慰的神色,道:“正该如此,不过随云,你要小心,如果他们的目标是你,我担心你难以脱身。”

我笑道:“或许吧,不过对于北汉将士来说,恐怕在他们心目里你才是首要的目标,所以这次承受压力最大的可能就是殿下。”然后我和齐王匆匆研究了一些作战的细节,过了一会儿,守住路口的骑兵已经有些精疲力尽了。这时,小顺子已经准备好了一切。他先解下那两匹拉着马车的骏马,这两匹白马都是事先精选的,完全可以胜任战马的工作,小顺子将其中一匹马交给其他骑兵牵着,然后换上白色的精制衣甲,又从马车车座下面拿出两截长枪,将它们接起来,变成了丈二长枪。小顺子走过来先把我扶上战马,然后他自己也跳了上去,让我坐在他身后,他细心的用带子将我和他绑在一起。这时候,百名虎赍卫和五个齐王近卫也已经组成一队,我在马上看了齐王一眼,冷静地道:“殿下,臣先走一步了。”

说罢,小顺子一声令下,我们这支包括百名虎赍卫和五十名齐王亲卫的骑兵一马当先向旷野冲去,冲出数里之后,我回过头去,看见齐王已经带着另外一支骑兵向另外一个方向逃去,而我特制的那辆马车却被推翻在路口,阻挡住了追兵,我的马车颇为沉重,看来一时半刻他们是过不来了。

这次分兵我是经过深思熟虑的,不过齐王能够这么快就想通这件事情倒是更令我佩服,他可是一直都在作战呢,敌人的目标是两个,我们就是在一起力量也不会大多少,而敌人却可以全力以赴,现在虽然我们分兵力弱,可是敌人也会为了如何安排分兵而踌躇,总归我们是不会吃亏的。而且只顾逃命无力反击那是必败之局,如今齐王可以说是已经没有牵挂,就可以想法子反击了。

这时,前方的小顺子说道:“公子,后面大概有一千人的追兵,我们要怎么做?”

我心中一喜,果然对那些骄兵悍将来说,齐王这个大雍的统帅才是最重要的,不过一千骑兵也不是一个小数字,如果不能一举歼之,我们也不能脱身去帮助齐王。看看周围的地势,我道:“让呼延寿往荒野里面去,我身边带着二十支飞天神火。”

小顺子一点就透,立刻点头道:“我明白了,水火无情,果然是好办法。”说罢他便和统领骑兵的呼延寿研究起来,呼延寿本是我在雍王府的时候身边的近卫,这次皇上特意派了他过来,也是因为这个缘故,要不然堂堂一个虎赍卫的副统领,怎会屈尊如此。我听见他和小顺子说着如何引敌人上钩,果然是精于用兵的将领。我在心中默默祝祷可以一举成功,不然我的小命恐怕就要送在这里了。

这时,后面的追兵渐渐有些近了,没办法,小顺子的骑术虽然也很出色,可是和这些几乎成日在马上生活的骑兵来说还差得远呢。幸好呼延寿指挥的不错,转来转去,总算没有被后面的敌军给合围。又过了片刻,敌军已经给我们诱入了一片荒草离离的旷野,秋末时节,枯黄的野草干燥易燃。小顺子一见风向合适,一声高呼,众人加快了马速。而身后的北汉骑兵果然还是控制着马速追击,这是我们早就预料到的,在骑兵追杀敌人的时候,最忌讳的就是全力策马,这样一来,由于过于消耗马力,很容易被落下,所以一般来说,除非是已经合围或者敌军前进无路的时候,一般是不会全力策马的,基本上都会控制着马速,不急不缓地跟着敌军,等到他们人困马乏的时候,再发起猛攻,才能一举得胜。当然这是敌我双方骑兵素质差不多的时候的准则,如果敌军太弱,自然是不用这种手段的。因此小顺子他们加快马速,后面的追兵被落下了一些,却没有同样加快马速,免得被我们给拖垮了。

可惜这次他们如此做却是错了,就在双方相距超过将近两里路的时候,呼延寿一声呼啸,我军分成了十几个小队,四散开来,我能够听到身后的北汉敌军高声大笑,想必他们以为我军要分散逃跑了,这样一来,他们是必胜无疑的了,我甚至能够听出来他们的笑声里面带着可以狩猎猎物的欣喜。就在这时,小顺子突然策马回头,然后手中多了一个小银筒,连续按动上面的机关,从里面飞出火焰,迅速点燃了枯黄干燥的草原,若是平常的放火之法,恐怕还没有等到大火燃起,北汉敌军就已经突破了火焰防线了,可是这次小顺子使用的飞天神火非同寻常,只是顷刻之间,大火就已经蔓延开去。而就在这个时候,四散的雍军也从另外几个位置点燃了同样的大火,大火很快就连接成了一片,月牙形的火圈向北汉骑兵扑去,这里四下都是荒草蔓蔓,北汉骑兵想要绕过火圈来追击,却是已经来不及了。只得向后退去,可是他们的方向正是下风处,火焰带着黑烟追赶着他们,他们刚跑出七八里路,却绝望的发现,同样的大火阻挡了他们的归途。

我能够听到火海里面悲惨的叫声,心中凛然之余,也不由有些得意,幸好因为这飞天神火形状小巧,威力极大,所以我在马车上面带了二十支,如今虽然几乎全部用掉了,却破掉了一千铁骑追兵,也是物有所值。虽然我也知道全歼敌军是不可能的,不过至少可以灭掉他们大半的人马。

不过让我有些遗憾的就是恐怕派去后面放火的四个人恐怕是九死一生了。为了达到歼灭敌军的目的,我让呼延寿派了四个人在途中离开,迂回到两侧,见到前面火起,北汉军奔逃回来的时候,再加上两把火,这样火势就可以连起来,阻挡敌军的生路。可是飞天神火太厉害了,现在风又这么大,他们恐怕是回不来了。不过我心中很敬佩他们的勇气,虽然明知道留下来放火很危险,他们却是个个争先,让我不由有些汗颜。

不过这些事情也顾不上了,呼延寿收拢了军队,我们也得快些离开,现在离火场太近,如果风向一变,恐怕我们也得陪葬。

抛下了生死不知的追兵,我们赶向预定的会合地点,大雍在边境多有寨垒,齐王和我约定了会合的地点,到时候齐王就可以凭借堡垒固守,而我们就可以从后突袭北汉军。说来也是没有办法,二十支飞天神火想要对付三千骑兵只怕是不够的,所以我只能先诱使他们分兵,然后再歼灭其中一支,也幸好追我的骑兵较少,否则恐怕还要经过一番苦战呢。我一边听着耳边的风声,一边祝祷希望齐王殿下可以平安的赶到会合的地点,否则我可是什么办法都没有了。

而这时的我当然不会知道,大概半个时辰之后,当最开始着火的原野已经只剩下一片黑色的灰烬的时候,几匹被烧得焦黑的战马尸身被推开,从战马之下站起一个浑身都是黑灰的男子,他厉声喝道:“江哲,韦某与你誓不两立!”

这人正是韦膺,当日他奉了陆灿的命令,带着林碧的信物到了北汉军中,奉命接应林碧的飞虎将军石英得到林碧的军令之后,就带了三千骑兵,潜伏在齐王归途伏杀。而对大雍恨之入骨的韦膺也自告奋勇地参与了这次行动,而让他振奋的是,江哲果然也随着齐王同行。后来石英分兵追杀的时候,韦膺也选择了追杀江哲,可是却被火海所困。韦膺心机灵敏,他自知骑术平平,不可能逃出火海,便趁着混乱之时,仆杀了几名落后的北汉骑兵,杀了他们的战马,然后藏身在马腹之下,血水之中,这才勉强逃过了火海葬身的命运。他愤怒的诅咒了一番,然后踏上了回转南楚的路程,他可不会笨到再去追杀江哲,孤身一身去对付百多名骑兵,他没有这个勇气。

等到我们终于赶到固山寨的时候,虽然我是被小顺子带着的,可是仍然是筋疲力尽,两腿内侧都被马鞍磨破了,我今年年将而立,可是从来没有吃过这样的苦头,等我被震耳欲聋的喊杀声惊醒的时候,才发觉我们这支骑兵停在山坡之下,上面不远处就是山头,我能够听到山头那边的厮杀声。

小顺子把我扶了下来,道:“公子,前面就是固山寨了,齐王殿下被围在寨外,寨内的守军几次要想出来营救都失败了。”我心中一紧,咬着牙站了起来,道:“你扶我去看看。”

小顺子伸手揽住我的腰,也不见他如何动作,就已经带我上了山顶,躲在一块岩石之后,然后我就看到了战场。

固山寨得名是因为这座寨子建在一座小山头上,与其说是小山,不如说是一座岩石丘陵,而且寨子里面有一眼水量极大的泉水,顺着山势流下。修建宅子的时候,绕着寨子一周挖了两三丈深的沟渠,然后引泉水灌入,固山寨既有地势的优势,又有“护城河”拒敌,是一座颇为重要的寨垒。可惜因为寨子太坚固,所以里面驻守的军队大多都是步兵,只有三百骑兵而已。我向下看去,只见就在距离寨子千步之远的地方,齐王殿下带着百多名伤痕累累的骑兵冲杀不休,被一千多名骑兵困在阵中。而另外的七八百北汉骑兵则游弋在固山寨外,阻拦固山寨的援兵。我可以清晰的看到,在护城河边上,有大片的尸体和一些无主的战马在游荡。而在寨子的最高处,笔直的黑烟正在滚滚向上涌动。

这时呼延寿也跟了上来,忧心忡忡地道:“大人,方才寨内的军士曾经想出寨接应,可是却被挡了回去,虽然现在寨子用烽火通知临近各寨,可是没有一个时辰恐怕他们是到不了的。大人,我们必须救援齐王殿下才行。”

我惊叹地看着下面交战的双方,这可是我第一次如此近的看到精锐骑兵的交锋,虽然力量悬殊了一些,可是齐王一点也没有流露出怯意,每一次冲刺都是向敌军软肋而去,而指挥北汉军的飞虎将军石英虽然应变迅速,始终将齐王等人困在阵中,但是却是始终不能压制住齐王。我有些奇怪的问道:“呼延寿,虽然可能是因为齐王殿下战法高明,可是怎么我觉得石英有些名不副实呢?”

呼延寿道:“大人有所不知,北汉的几位将军长处各自不同,石英擅长千里奔袭,这次殿下身边的亲卫精锐只在石英所部之上,所以石英不能以急袭得手,这行军布阵的本事,北汉军中以鬼面将军谭忌为首,而在我大雍军中,临阵指挥骑兵,鲜有能胜过齐王殿下的,所以才会有这样的局面。

我心中暗道一声侥幸,若是这次追杀伏击我们的是鬼面将军谭忌,大概我就可以给齐王收尸了,当然这还是如果我有可能逃过一劫的情况下。这次北汉的安排不是不周到,可是却没有料到我会带着本来是为了东海之会准备的飞天神火,另外又忘记了如今秋高草长,乃是最是容易使用火攻的季节。而石英的战法被齐王克制却也是无奈之举,我想北汉也不能事先想到齐王会去东海的,一定是得到林碧的情报之后才匆匆派了在附近的石英前面,若是这些条件差了一点,今日就不是这个格局了。仔细观察了半天战局,我正色道:“小顺子,一会儿你跟着呼延将军顺势攻入北汉军中,你虽然骑术差些,可是应该勉强比得上一个普通骑兵了,你这几年练了姜家的枪法,应该用上的,如果能够取得石英的性命自然是最好,如果不能,也要让石英不能再如臂使指地指挥敌军。你们看这个安排如何?”

小顺子和呼延寿都微微皱眉,呼延寿先道:“大人,李爷武功高强,末将当年也曾亲眼所见,可是大人你的安危要紧,如果李爷也上阵杀敌,到时候若是被乱军伤了大人,我们可是担待不起。”

我苦笑道:“呼延将军,这也是不得已之事,你要指挥军队,恐怕斩将夺旗的事情你是腾不出手的,而且若是不能取胜,就是你们都在这里保护江某,也是无济于事。这样吧,你留几个虎赍保护我,只要你们速战速决,我应该不会有太大的危险的。”我可不好意思说呼延寿没有绝对的把握压制石英。

小顺子倒是没有出言,他是明白现在的局势的,也知道江哲令出如山,心道,只有自己快些杀了石英,然后马上回来保护公子,才是最好的解决法子,心中不免有些后悔没有让江哲多带几个心腹护卫过来。这时,下面齐王已经有些阵形散乱,看来是强弩之末了。我连忙下令道:“呼延将军,你快些行动,若是殿下受伤,只怕我们担当不起。”

呼延寿低声应诺,安排了几名武功高强的虎赍卫保护我,便回身上马,小顺子看了我一眼,也上了战马,这时我想起一件事情,连忙凑近喊道:“小顺子,还有一件事情?”小顺子脸上露出询问的神色,俯身低头,我在他耳边匆匆说了几句话,然后连忙退到一边。

呼延寿见众人都已经准备停当,一提马槊,无声的指向天空,然后猛然下挥,将近两百人的骑兵冲上了山顶,然后风驰电掣一般狂啸而下,站在一边的我只觉的地动山摇,碎石乱滚,差点跌倒在地上,幸好身边的几个留下来保护我的虎赍卫搀住了我。

这几个虎赍卫也都是当年在寒园保护我的近卫,这几年都已经升职,至少也是六品的武官了,不过前些日子他们一和我见面,就跟我诉苦,说是当年我出走之后,他们因为“保护不力”被当年的雍王,如今的皇上狠狠训斥了一顿,总算雍王知道他们委屈,没有责罚他们,反而因为他们在我身边待了几年,都给予了重用,可是还是很长时间都抬不起头来。幸好是他们,定然不会嘲笑我,当初在寒园的时候,他们可是都负着随时留心我的身体状况,一旦看见我面色不好,就得随时去请雍王府专门负责替我诊治的御医的。虽然我现在已经基本上恢复健康,可是在他们心目中大概还是那个随时都可能断气的药罐子吧。

都我站起来的时候,正好看见小顺子随在呼延寿身后冲入了北汉军的骑阵中,白马银枪雪战袍,威风凛凛,倒让我心中有些嫉妒,可惜啊,我是没有可能上阵杀敌了。黑红两色的铁流势不可挡,北汉军没有料到会有伏兵,一时间阵势大乱,而齐王所部声势大震,拼力厮杀,这时,寨内也已经惊动,寨门大开,仅剩下百多人的寨内骑兵也杀了出来,虽然大雍军力量仍然不如北汉军,可是内外夹击,三方猛攻,北汉军一片混乱。

石英万万没有料到会在这个时候身后出现敌军,事先他们已经清除了许多大雍的斥候,而且那些寨垒之内的雍军秉承齐王的严令,是轻易不会出寨的,所以他本来可以稳当当地围杀齐王的,而带着护卫“逃跑”的那个江哲也没有被他放在眼里,一个智谋出众的谋士可不一定会是能够领军作战的将领。如果不是林碧的指令中特意要求石英一定要擒杀江哲,那个南楚使节又是那样坚持,他跟本就不会派了一千人去追江哲,至于江哲能够脱身这一点,石英可是绝对没有料到的,所以他跟本就不会想到附近会有援军。而一眼看到黑红两色的衣甲,石英第一个念头就是想到了追击江哲的那些骑兵的安危,心中冰寒的同时,下令阻敌的命令也不免晚了一刻,就只这么一瞬之间,败局已成。

石英甚为果断,立刻下令撤军,自带亲军断后,北汉铁骑仗着人多,四散逃去,石英刚刚一槊将一个挡路的雍军撩倒,前面白影一闪,一个身穿白衣白甲的骑士挡住了自己的去路,面甲掩住了那人面容,看不见他的容貌,可是他的身材并不高大,石英冷冷一笑,自恃力大勇沉,一槊撩去,那个骑士也不闪避,一杆银枪从环辔间斜探而出,枪槊撞在一起,石英只觉得好像撞入了一团棉花,着力处似实还虚,不由身子一个踉跄,这时那骑士的银枪倏地裂开,散成满天枪影,枪尖激起的无数细小而冰寒的气流扑向石英。石英大喝一声,马槊当空一划,炽热的劲风挡住了银枪的攻势,“叮叮叮”一串兵刃交击的尖锐声响和暴起的风浪让两人身边数丈方圆之内再也无人能够立足。

石英乃是北汉著名的武将,在战场上虽然也遇过敌手,可是从来没有像今日这般艰苦,若非是他察觉到那人的枪法和骑术相差很大,利用自己骑术上面的优势,恐怕也不能和那人斗了一个旗鼓相当。双方斗了十几个回合,那人渐渐占了上风,突然银枪化作流星逸电,刺破了石英的防线,石英拼力闪躲,仍然被那人一枪刺穿了右肋,石英惨叫一声,不顾生死,手中马槊竭力出手,那人策马退了一步,石英转身逃去,他身边的十几个亲卫不约而同的挡住了那名敌将的攻势,银枪化作点点星雨,空中闪现朵朵灿烂的嫣红,当那十几个亲卫丧命在银枪之下的时候,石英已经在其他的亲卫保护下冲出了很远。那雪袍战将见已经追之不及,高声叫道:“石英,转告嘉平公主殿下,就说南楚可没有安下什么好心肠,他们不过是传传消息,你们却是损兵折将,这鹬蚌相争,渔翁得利的计策还看不透么?”

石英耳中听得明白,虽然明知那是挑拨离间,可是心中还是平白生出恼怒,不由怀疑起南楚的用心,据那使者所说,指使他的人乃是南楚陆灿,据说陆灿就是江哲的弟子,难道弟子还不知道师父的本事,莫非陆灿就是知道我们不可能轻而易举得手才传递消息给我们的么?

我在高处听到小顺子的喊声,面上露出微笑,陆灿和林碧联手害我,这个仇不能不报,北汉的军方领袖可是龙庭飞,若是能够让龙庭飞对陆灿有了戒意,那么就可以避免北汉和南楚勾结的太深,我也可以少些麻烦。

又过了一阵子,战场上已经平静下来,只剩下清理善后的大雍军士了,我这才在几名虎赍卫的保护下向山上走去。只有短短一段路,若是骑马转瞬就到,可是我双腿内侧早已是血肉模糊,实在不愿意乘马,走路虽然也很苦痛,也只得认了。走到山下,齐王带着亲卫迎了上来,他浑身上下伤痕累累,鲜血狼藉,十分狼狈,不过他可没有放在心上,一见我就大笑道:“随云,你好本事,以后干脆也指挥杀敌好了。”

我强忍着白他几眼的冲动,道:“殿下这可是为难我了,若是我都能上阵杀敌,那么就是南楚也是人人都可以从军作战了。

这时,寨内的守军将领也过来恭请我们入寨,我见小顺子正在和呼延寿他们一起善后,觉得现在也不会有什么危险了,便和齐王并肩走向寨门口处的吊桥,那里的尸体很多,还没有经过清扫,可是这里除了我之外人人都是久经沙场,谁也没有放在心上。我也只能视而不见地向寨内走去,心想,赶快沐浴更衣,睡上一觉,才是要紧的事情。

朦朦胧胧的,李虎睁开了眼睛,他是飞虎将军石英手下的一个小小的骑兵什长,在阻截固山寨援军的时候不慎被刺落马下,恰好头部撞击到岩石上,因此昏迷不醒。战时仓促,也无人注意到他还未死,他昏迷了许久,直到石英落败而走,这么长时间,也就没有人想到这里还会有活人。忙着清理战场的雍军还没有来得及顾及这里,只是简单地把挡着吊桥的一些尸体拖走罢了,然后就去打扫战场,救护战友,将伤重的北汉军补上一刀或者押到一边。所以李虎就这么躺在那里,无人过问。他睁开眼睛的时候,正好看到一个穿着皇族金色战甲,外披赤色战袍的将军和一个青衣文士并肩走向吊桥。李虎心中如同烈火焚烧,知道肯定是北汉军败了。眼光掠过,李虎看到身边有一柄不知是谁丢下的步槊,也无法多想,李虎拼尽最后的力量,伸手抓住步槊,然后突然坐起,将手中唯一的武器掷了出去。他见众人几乎都穿着战甲,又担心自己力弱不能一举得手,这一槊却是掷向了那青衣文士。

使尽了浑身力量的李虎只觉的眼前发黑,在看到那青衣文士后心被步槊刺中之后,身躯摇摇欲坠,在身边众人瞠目结舌中跌落桥下之后,李虎也没有力量抵挡冲过来按住自己的雍军,任凭他们捆绑殴打,他心中满是欢喜,放声大笑起来。

第十六章    我心依依

陆灿站在甲板之上,望着一望无际的碧海,明明是风和日丽的天气,可是他心中却是一片孤寂,虽然早就知道那人已经是大雍的重臣,深受大雍皇室的信赖,而且又娶了昔日的王后,大雍的宁国长乐公主,可是陆灿心中却无法产生对那人一丝敌意。他对那个人可以说是很了解的。昔日江哲在做自己的西席的时候,也只有十几岁的年纪,自然是不似如今这般深沉。陆灿深深的记得江哲平日最爱的就是偷懒,除了规定的时间之间绝对懒得监督自己读书,初时还经常跑出去逛街或者游玩,不过这人终究是好静的,到后来最经常做的事情就是拿了一本古籍,泡上一壶香茶,坐在树荫下津津有味的阅读。不过这人也很好诱惑,只要自己拿了什么新奇美味的糕点,多半都可以让他答应替自己写功课,或者作些别的什么小事。想到这里,陆灿不由失笑,可是笑容很快就消失不见。

他是知道的,自己这个师父生平最是没有大志,在南楚当了状元之后,除了曾经在筹立崇文殿的事情上十分用心,以及曾经襄助德亲王攻蜀之外,基本上就是尸位素餐了,所以后来江哲因为上书直谏而被贬斥的消息传来,陆灿第一个念头就是,师父不是想借机抽身了吧?可是没有,师父还是留在建业,当时陆灿还曾经惭愧的想,或许是自己想差了,如今师父已经是堂堂的翰林学士,怎能以从前的标准衡量。可是就在之后不久,雍王攻破建业,恩师被掳去大雍,而当陆灿得到准确的消息之时,一切已经事过境迁,恩师投效雍王,而且被南楚刺客重伤。这样的局势,让陆灿再也无法存有救回恩师的念头。因为陆灿已经明白,南楚已经永远失去了一个本有可能成为擎天玉柱的栋梁之材。

接下来,陆灿默默的注意着江哲的事情,始终默默无闻的江哲在猎宫之变中一鸣惊人,力挽狂澜,然后抛却荣华富贵,带着长乐公主私奔而去。虽然有些遗憾大雍终于被强有力的君主所掌握,可是陆灿还是默默的祝祷自己的恩师可以从此安享余生,因为他也得到过情报,知道恩师为了雍王,可真是鞠躬尽瘁,据说离去之时已经是病入膏肓了。

可是就在不久之前,江哲的一封信让他彻底明白,江哲不会在大雍没有一统天下之前归隐,江哲的生死荣辱已经和大雍皇室紧密的联系在一起了,所以前舱在心中已久的杀机终于爆发了,陆灿的心中只有一个念头,若是江哲继续为大雍效力,那么最后成为祭品的一定是南楚,陆灿不能眼睁睁看着家国覆亡,不论国主昏庸还是圣明,陆灿都不能让陆家三代效忠的南楚成为大雍铁蹄下的战利品,所以陆灿在自保的同时,下了决心,除去江哲。其实说服北汉伏杀江哲,陆灿并没有完全的把握,可是他也知道这是唯一的机会,只得尽力一试。他相信对付江哲最好的办法不是谋定后动,而是用最快的动作,用最猛烈的攻势,用直接了当的手段去攻击。虽然没有百分百的信心,可是早已察觉到江哲对自己并没有特别的戒心的陆灿,相信很有可能成功。

杀死一个敌人,甚至可以说是一个叛逆,原本应该是大快人心的事情,可是为什么心中如此之痛,陆灿仰天长叹。

同样的晴空,林碧心中却也是一阵怅然,她知道,按照时间推算,这个时候应该是齐王和江哲被石英伏击的时候了,一个是统帅大军阻挡北汉兵锋所指的大将,一个是智谋如海,手段通天的军师,这两个人一死,至少数年之内北汉可以安枕无忧,原本林碧应该兴奋期待,可是却总是有些不能释怀。这两个人给林碧的印象都很不错,齐王虽然有些杀气太重,性情也似乎有些暴戾,可是林碧能够感受到李显心中的悲怆沉痛,而且齐王本质上是一个性情中人,这让林碧心中对他多了几分好感和赏识,她甚至曾经将李显和龙庭飞比较,龙庭飞虽然明显胜过李显,可是林碧却隐隐觉得龙庭飞过于完美,令她在尊重倾慕之余也有些自惭形秽,她总觉得如果自己不是嘉平公主,那么自己根本配不起龙庭飞,这也是她这几年有意无意拖延婚事的一个原因。而李显就不同了,有过人之处,也有明显的缺点,反而让林碧觉得颇为可亲可爱,而李显不时流露出来的落寞更让林碧心中多了几分怜惜,之前林碧心中只当李显是敌人,所以还不觉得,可是在李显很可能丧命的时候,林碧却不由自主的想起了李显的音容笑貌。

而江哲呢,那个在传闻中心思阴毒可怕的谋士,带给林碧的却是一团迷雾,犹记得初见面时他气度闲雅,令人见之油然而生敬慕,更记得听涛阁上他稚气显露,童心犹存的另外一面,这个人,林碧隐隐觉得,或许很多人都误解了他,或者他本就是一个恬淡无害的异类,只有当你触犯了他的时候,他才会露出狰狞的面目。

还有温柔娴雅的长乐公主,林碧可以感觉到她的平安喜乐,从前坎坷的人生似乎在她身上看不到影子,可是林碧心中明白,这才是这个女子值得敬佩的地方,天下有几个女子可以坦然面对从前的伤痕累累,又有几个女子可以放弃唾手可得的权势富贵,跟着病弱的情郎携手共赴茫茫的前途呢?

还有柔蓝,那个受尽宠爱却是不显骄矜的小女孩,还有江慎,那个还不解人事,就被父亲“狠心出卖”的可怜男孩,林碧只觉得心中一阵剧痛,她是在毁灭怎样一些人的幸福啊!

痛过之后,林碧终于收拾起惆怅的心情,她告诉自己,不论那些人是怎样的可亲可敬,可是他们都是北汉的敌人,他们的死亡可能会换取无数北汉将士的生存,渐渐的恢复平静的心情,林碧低声道:“这是命运,如果失败的是我,那么我也愿承担所有的后果。”

在通往长安的路上,迤逦而行的公主鸾驾之中,长乐公主神色淡然地望着远处的天空,这次大雍朝廷可是给足了面子,在长乐公主在庆王李康的护送下进入大雍势力范围之后,太上皇李援和雍帝李贽就各自下了一道诏书,公告天下。

“武威二十五年十一月,朕尤在位,顾念宁国长乐公主孀居寂寥,赐婚天策帅府司马江哲,唯司马因国事卧病,不堪辛苦,朕心不忍,特许二人私下完婚,仪成六礼,礼部文书皆具。于今驸马病愈,朕甚思念,特诏还朝,钦此。”

“驸马都尉江哲,素有功于国,今赐封楚乡侯,食邑三千户。钦赐朕潜邸为宁国长乐公主府邸。公主世子江慎,赐封安国公,食邑五千户,长女柔蓝,赐封昭华郡主,食邑千户。钦此。”

这两道旨意不仅轻轻松松地掩盖了当日长乐公主私奔的事实,还封江哲为乡侯,更将年仅周岁的江哲长子江慎封了国公,这已经是外戚朝臣最高的爵位了,就连江哲的养女也封了郡主。如此封赏,就是再没有眼力的人也知道江哲夫妇深得皇室宠幸,绝对没人敢提及当年的事情了。

可是长乐公主心中却是十分淡然,当初出走之时,她就已经抛却了一切,若不是大雍局势不稳,就是再重的封赐也不能让长乐公主重回长安,更不愿让夫君重入宦海。可是长乐公主也清楚这其中的难处,如今夫君已经去了北疆前线,若是自己留在东海,先不说江哲会担心自己的安危,就是皇室也不免担心前线兵权谁属。自己若是不进京为人质,就是皇兄相信自己夫妇,那些大臣也不免会秘密进谏的。与其让那些人心中生出疑念,不如自动一些。所以长乐公主入京之事早就已经决定了。

轻轻叹了一口气,若是还有选择,长乐公主宁愿留在东海不问世事,可惜这却是不可能的事情。

这时柔蓝兴冲冲跳到鸾驾之上,问道:“娘亲,慎儿呢,看我给慎儿编了花环呢。”

长乐看了一眼那精致的花环,笑道:“编得很好看呢,是不是麟儿教你的,我看你方才和他在一起嘀嘀咕咕的。”

柔蓝眨了眨眼睛,道:“才不是呢,麟弟只会舞刀弄剑,怎么可能会编花环,是我跟尚仪学的,方才我不过是看麟弟很孤单,所以才去和他说话的,谁让三舅舅那么过分,不让麟弟和我坐一辆车,说什么我是郡主,麟弟虽然也是皇族子弟,却没有爵位,又说什么要避嫌,不让我们坐一起。”

长乐公主眼中闪过一丝冷然,淡淡道:“蓝儿,你去跟你三舅舅说一声,就说慎儿一直被慈真大师占着,我一人乘坐鸾驾很是寂寞,让麟儿和你与我一起乘坐吧。”

柔蓝大喜,道:“我这就去告诉他。”说罢跳下鸾驾,兴冲冲的跑向庆王的马车,身后自有侍卫紧紧跟随保护。

长乐公主心道:随云临行之前要我好好照料麟儿,我怎能看着他被人欺负。不由对久未蒙面的三哥添了几分恼意。

这时长空如洗,一行秋雁鸣呖而过,长乐公主听了不知怎么,觉得心中一紧,不由向北望去,不知夫君可到了大营没有?

“阿嚏”我打了一个大大的喷嚏,然后就听到齐王的窃笑声,狠狠的瞪了他一眼,若是我真得给那个北汉军一槊刺死,现在他想哭恐怕都哭不出来。说起来也是侥幸,因为想到上战场之后随时都可能有危险,所以我特意精制了一件护身的金缕衣,这金缕衣乃是古书上面所记载的奇物,乃是用云南苗疆特产的紫金沙混合异域乌兹炼制的软铜,熔炼之后抽成紫金丝,这种紫金丝细如毛发,柔韧无比,却是可也吊起千斤之物,用这种紫金丝混合西域金猩的毛发纺成的细线,编制成一件薄如蝉翼的内衣,穿在身上仿若不觉,却是可以刀枪不入,不说制衣的工艺十分复杂,就是为了得到那些原料,也是费尽心力,为了保命,我可是花了千万金银和无数心思啊,就是这样,我还不放心,又特制了一件青色大氅,夹层里面缝了三札牛皮,这可是制作皮甲的材料,虽然不如我的金缕衣那般刀枪不入,但是可以护住全身,总算是聊胜于无。虽然我费了不少心思和金钱,不过总算是物有所值,那一槊虽然刺中我的后心,将我撞落吊桥,倒是没有刺伤我,就是力道也消去大半,当然这也是因为那个北汉兵根本没有多少力气了。可是秋末时分,泉水寒彻,再说那护城河里面还有尸体血水混杂其中,我的水性也只是勉强可以浮在水面上,因此我落水之后着实吃了不少苦头,若非是小顺子远远看见,知道我应该没有受伤,连忙冲过来把我救了出来,只怕我没有被刺死也会被溺死,谁让齐王他们都以为我被击中后心,怕是死了,一时之间都反应不过来呢。不过吃了这样大的亏,从水里被捞上来之后又是吐得天昏地暗,在齐王面前,可是丢尽了面子,怎能让我不郁闷呢?更别说寒水一浸,我这身子终究不如常人,又感染了风寒,真是出师不利啊。

小顺子眼中闪过一丝忧虑,问道:“公子,是否多休息几日再启程,你身子素来不好,若是不好好治疗,属下实在放心不下。”

我懒洋洋地道:“不行啊,这里可不是什么安全的地方,虽然北汉军退走了,可是还要提防他们会有大军到来,还是快到大营好些。而且齐王殿下离开大营的事情本来是瞒着下边的将士的,如今恐怕已经是人尽皆知,如果殿下不回大营主持大局,恐怕于军不利,你放心,我不过是吃了点苦头,到了大营,也好休养,总比困在路上的好。对了,手炉热了么。”

小顺子连忙将准备好的手炉取来,我抱在怀里,紧了紧大氅,道:“我在路上就好好发一下汗,你们不用管我,等我到了大营,再叫醒我吧。”说完,我舒舒服服地躺在马车之上,闭上了眼睛。齐王有些好笑地看了看我,将自己的大氅解下,也盖在我身上,然后跳下马车,上了战马,看到脸色苦恼的呼延寿,便问道:“呼延寿,怎么了,从昨日就看到你一直苦着脸?”

呼延寿苦涩地道:“末将临行之时,陛下曾说,命我等好好保护江大人,还说若是江大人受了什么损害,就要重重降罪,如今大人不仅因为急行军而受了很多苦楚,而且又落入水中,受了风寒,只怕皇上若是知道,定会恼怒我等保护不力。”

齐王安慰道:“这个本王也没有办法,不过你们何必担忧,难不成皇上还会再派人来么,再说你们为了保护随云也损失了不少人,现在虽然随云受了些惊吓,但是也没有什么太大的损伤,无论如何总是总是有功的,再说皇上素来赏罚分明,将来你们多多尽心,让随云给你们多美言几句,难道皇上还能怪罪你们么?”

呼延寿听了心中稍安,不由感激地看看齐王,他方才是人在局中,不免糊涂,如今被齐王点透了关节,自然明白过来,心道,遇到敌军本是意料之外的事情,如今能够保得齐王殿下和江大人的平安,就已经是大功一件,陛下明鉴万里,赏罚分明,怎会凭白加以怪罪呢?

我在车上将他们的说话听得一清二楚,虽然距离远了一些,可是对我来说,自然是没有问题,心中不由叹了一口气,齐王李显,果然是对麾下将士关爱备至,即使呼延寿本是雍王亲信,只要做了他的属下,齐王也就一视同仁,难怪能够深得军心,引得朝中重臣忧虑呢?

若论才华气度,李显其实不弱于当今皇上李贽,但是他却有一样大大的缺憾,就是他的固执和偏激,这一点虽然是缺憾,却也算得上是优点,只因李显之所以能够成为今日大雍的武将之首,就是因为他百折不回的气势。自从李显带兵以来,不是没有落败过,可是李显却是败而不馁,再加上他精通战阵,生性勇猛,每次落败必带亲军断后,所以即使落败也不会伤筋动骨。而李显又善于从经验中吸取教训,卷土重来之时必然更加凶猛,令人头痛非常。多年征战,大雍虽然猛将如云,可是若是想要寻一个能够压得住军中骄兵悍将的人物,除了李贽之外,就只有齐王李显了。

兄弟两人比较起来,李贽思虑周密,攻无不克,战无不胜,可以说是大雍的军神和领袖,而李显却是大雍的利刃,军中将士的偶像,因为李显作战虽然有胜有负,但是他作战之时不屈不挠,领军作战身先士卒,落败之时亲自断后,无不令将士敬服,而李显的努力和进步更是人人都可看到的,对于仰之弥高的雍王,将士多是敬畏,而对于齐王,却是多了几分亲近。若论军心,雍王麾下自然是忠诚不二,可是齐王所部也不逊色,当日猎宫夺嫡之时,若是齐王下了决心,和雍王一博生死,那么雍王虽然最终多半仍会取胜,可是大雍国力必然因此衰退。这也是事先最令雍王和我头痛的地方,若非是连番变故,说不定在猎宫变故之前,我们就对齐王下手了。

齐王的固执和偏激让他在战场上成为敌军最头痛的敌人,若是对上雍王,基本上来说敌军多半已经是必败无疑,所以往往一战而定,也就没有什么好说了,若是对上齐王,虽然敌军可能取胜,可是只要不能在战场上留下齐王,那么就要面临*一般的反击和不死不休的报复,那种压力多半能够让敌将恨不得一开始就落败了。齐王能够抵挡天纵之才的龙庭飞,除了军事上面的才华之外,主要就是靠了他坚毅的心志,迫得龙庭飞无法一举功成,从起初的连战连败,到后来的平分秋色,齐王的进步人所共见。

可是这个明显的优点,在政事和家事上就成了很明显的缺点了,若非如此,齐王也不至于落得今日的窘境。根据我的调查和判断,当初齐王殿下为了能够占据军方首席的位置,铁心投靠太子李安,而他和凤仪门秦铮的联姻自然有政治婚姻的意味,可是李显对秦铮确实曾经动了真情,可是秦铮却偏偏和师门瓜葛不断,这就触犯了齐王的逆鳞,齐王此人,独占欲极强,所以为了掌握军中大权,明知李贽更应当继位,却仍然投效太子,也为了秦铮的软弱和摇摆而将其屏除在心门之外。若是齐王不那么固执,或许当日他就会效忠雍王,不会落得今日君臣相疑的格局,若是齐王不那么偏激,就不会疏远秦铮,若是他肯用心对待秦铮,或者很有可能让秦铮最后抛弃凤仪门,也就不会有晓霜溅血,夫妻永诀的悲剧发生了。

反过来说,若是齐王不那么固执偏激,一心一意地跟太子、凤仪门合作,不因为心中的鄙夷和芥蒂而疏远太子和凤仪门,猎宫之变,鹿死谁手还未可知呢?

就是因为齐王这古怪的个性,才有了今日他的窘境,我听闻齐王因为王妃秦铮之死而心中悲恸,不肯续弦,这也是皇上和齐王不和的流言能够到处纷传的缘故,可是在我看来,齐王对于秦铮,虽然有夫妻之情,却未必是真的如此深情难忘,倒是很有可能因为齐王心中存着昔日不该放弃和凤仪门争夺,以至秦铮泥足深陷,最后自尽身亡的愧悔吧。这样的心情或许才是齐王陷入不可自拔的死结的原因吧。

其实我总觉得齐王屡次拒绝皇上的好意,并非是存心不肯和皇上和解,恐怕还是心结难消,没有台阶可下,不过这不是长久之计,皇上毕竟是皇上,忍了一年两年,忍不得十年八年,再说皇上就算是能忍,那些重臣们也会屡屡进谏,日日就是,就是皇上相信齐王殿下不会有反意,也不能太过乾罡独断,到时候,恐怕齐王就不能领兵了,这样一来,岂不是让齐王更加怨恨,这样一个帅才,若是平白毁了,我可是不甘心的,再说齐王这个人若是和皇上和好,必然是铁了心效忠皇上,到时候大雍江山固若金汤,我也就可以安心归隐了。难得这次齐王终于退了一步,来寻我解围,这个好机会我怎能放过,皇上也是精明的人,和我虽然没有事先交流,却是想到了一起去,这次我们君臣再次联手,一定能够让齐王殿下心悦诚服地服软。而且也是机缘巧合,齐王这样高傲固执的人,居然对慎儿十分喜爱,甚至答应再娶正妃,只要齐王动了心,我就有法子化去他心中的寒冰,想到美好的前景,我不由轻轻一笑。等到他们兄弟君臣和睦,应该就没有我什么事情了吧,现在么,不过是他们之间少了一个台阶罢了,我就委屈一下,充当这个台阶吧。至于军务上面的事情我可不会插手的。

我正想得高兴,突然呼延寿叩动车门道:“公子,皇上的旨意已经到了大营,殿下问是否需要加快行程。”

我皱皱眉头,自从遇袭之后,齐王也顾不上什么隐秘了,不过是一夜之间,就传下数十军令,现在泽州、镇州境内是风声鹤唳,不说别的,如今身边的护军就有数千,而得了军令前来保护齐王的军队更是络绎不绝,这大军行动起来可是颇费钱粮,行军计划更是已经定下,如果现在加快行程不说影响到军事上的布局,恐怕还得轻骑赶路,这个苦我就吃不了。

这时,小顺子轻轻道:“齐王怕也不想急行军呢?”

我心中一动,仔细想了一想,果然如此,听齐王的口气,不过是不想落一个怠慢钦使的罪名,所以让我拒绝罢了,心中一笑,这齐王也是动了心思了,虽然是想拿我做挡箭牌,可是看在他也有心和皇上和解的份上,我就帮他一把吧。想到这里,便道:“请转告殿下,就说还是按照计划行程吧,钦使来传旨,恐怕我也有份,再加紧赶路,只怕我的性命倒要搭在路上了。”

果然我说了之后,齐王就没有再来打扰,若是从前,只怕齐王不是问也不问就加速行军,就是不理不睬,依然故我,如今的变化对我来说倒是可喜,至少齐王不会拗着性子做事了,不过想用我做挡箭牌,可是需得付出代价的,我总是要讨回来的。

第十七章    立威定策

大雍武威二十七年,十月十六日,哲初入泽州大营,任监军,杖悍将以立威,众军折服,军心乃安。

——《南朝楚史·江随云传》

数日之后,终于到了泽州大营,远远看着犄角相连,隐伏杀机的大营,心中不知怎地凭空生出骄傲的念头,上有雍王这样的明君,中有一干虎将,下有这样的雄兵万千,若是大雍不能一统天下,真是没有天理了。

齐王走到车前,笑道:“随云,这次你可不能坐车了,我命人准备了一匹性情温顺的战马,你应该没有问题吧?”

我微微一笑,道:“应该没有问题。”

说罢我在小顺子扶持下跃下马车,骑上了那匹齐王所说的温顺战马,虽然风寒尚未完全痊愈,但是已经大致无碍了,青衣飘飘,倒也是气度不凡,心里庆幸当日逃命落水的狼狈模样没有给太多人看见,我策马落在齐王身后一步向大营驰去。

离大营还有数里之遥,营门大开,衣甲鲜明的两列骑兵雁行而出,然后上百名品级足够的将军随后而出,策马亲来迎接,加上他们身后的亲兵,一个个气势汹汹,在我看来不像是迎接,倒像是上来挑战的一般。

那些将军到了我们面前,一个个挥刀行礼,然后高声道:“末将等恭迎大帅回营。”

我总算也在军中呆过,没有被他们的吼声镇住。眼光一闪,将这些将军面貌都看了清楚,有一些颇为熟悉,却是在雍王府见过面的,只不过我在雍王府也是深居简出,却是不怎么相识,不过站在众将之首的那人我是记得清清楚楚,正是我那个最不爱读书的弟子,荆迟,听说他已经做了齐王的副手,两年不见,他气质更加沉稳,少了几分鲁莽气息。还有一半将领颇为陌生,看他们看向齐王的目光忠诚狂热,其中有一两个人我记得在齐王身边见过,想必这些人都是齐王的亲信将领,这些将领隐隐分成了两派,中间隔着明显的距离,之间泾渭分明,我微微苦笑,不知道是不是齐王故意不去交好那些倾向雍王的将领,若是他肯用心,至少这些将领不肯明目张胆的拉帮结伙。

齐王回礼之后,高声道:“陛下钦使何在?”我自然知道齐王为何这样着急见到皇上的钦使,大雍军令,无武职者不得擅入军营,我如今没有武职在身,就是齐王也不便让我进军营的。

随着齐王的高呼,有人高喝道:“奉敕令,齐王李显、楚乡侯江哲接旨。”

我抬目看去,一个绯衣官员捧了黄绫圣旨从营门策马而出,李显和我连忙下马,香案早已经准备好了,荆迟带着众将簇拥着李显和我跪下听旨。

那名官员高声朗读了一遍圣旨,众将听得明白,却是任命楚乡侯江哲为监军,便宜行事。泽州大营上下都需受江哲监督。其实这些日子以来,这些将领心中都隐隐猜到了圣旨上面写得内容,任命监军,也不是什么特别的事情,只不过皇上和齐王之间的关系众人皆知,若是任命了别人,那些将领不免怀疑皇上是不放心齐王,准备对齐王对手了,可是任命江哲做了这个监军,可就不一样了。军中地位高的将领都知道这个江哲是皇上的心腹军师,对于江哲的事情知道得不少,雍王方面的将领自然知道江哲的厉害,相信若是他做了监军,那么齐王定然无法起异心,而齐王方面的将领却是知道齐王能够“戴罪立功”镇守泽州,就是这人向皇上留书推荐的,而且这人是齐王亲自请来的,,就是再笨的的人也知道齐王对他的敬重。所以军中将领虽然互相有隙,可是对这个监军却是都接受了他的存在。虽然江哲名声颇为响亮,可是这种文弱的书生,却是这些将领不愿亲近接受的一类人,再加上将领对监军身份的人物的忌惮排斥情绪也是难免,这些却是与江哲本人无关了。

圣旨宣过之后,谢过钦使之后,齐王下令升帐,这是军中的大事,一旦传令升帐,逾时不到是要斩首的,不过今次升帐却是比以前更加吓人,大帐之内,虎赍卫和齐王的亲兵两侧站立,虽然前日合力厮杀作战,如今已经不像一开始那样彼此戒备,可是还是存了一较高低的心思,双方都气势汹汹,那些解了兵器进帐议事的将领都觉得背后寒气四射,不由都是心中直打突。初时的惊讶之后,这些将领也都是从血火中杀出来的猛将,自然也都不忿这些亲卫的气焰,也都露出了杀气,弄得大帐之内气氛紧张,倒像是立刻就要燃烧一般。

李显心中苦笑,看了一眼坐在东侧上首的江哲,心道,我若是强行压制,只怕反而会激化矛盾,你的职责就是调解军中的对立情绪,怎么还是袖手旁观呢,一边想,一边使了几个眼色。

我看在眼里,心中道,若是他们打了起来,岂不是显得我无能么?我仔细看了众将一圈,目光落到荆迟身上,看来还是得拿他开刀才行。不过这也不是冤屈了他,泽州大营两派对立,他就是雍王一派的首领,倒不是这家伙存心争夺权利,偏巧他就是无遮拦的性子,平日行事不免懈怠礼仪,而且这人心直,对于皇上自然是不敢稍有放肆,对着昔日敌对的齐王却是不免有些大大咧咧,若是别人也就罢了,偏偏他是皇上的心腹将领,在泽州大营内可以说是除了齐王就是他了,他这样无心行事,别人却不免以为是皇上示意他掣肘齐王,所以雍齐两派将领的对立也就显露了出来,偏偏这个荆迟又是个极重情义的人,这样的人都有些护短,若是两派将领闹了起来,这荆迟总是带着亲信袍泽打头阵,结果让齐王越发难作。若是齐王置之不理,军心不稳无法克敌,若是齐王想要杀一儆百,偏偏这荆迟即是皇上爱将,又是无心之过。如今我若是不处罚荆迟,就不能镇住雍派将领,这也是我要拿他开刀的理由。

想到这里,我微笑道:“元帅,本监军初来乍到,还不清楚军中事务,不知道如今军情如何?”

李显一愣,心道随云怎么这么积极,前日我跟他说起军情,他还懒得听呢,总是到了大营再说,如今怎么主动问了起来。他正要搭话,我轻轻给他使了一个眼色,李显立刻住口不言。阶下众将,能够入得帐来的都不是有勇无谋的匹夫,所以虽然齐王没有答我,可是他们个个也是哑口无言。只有荆迟,数年不见,早就心痒痒地想跟我问候,可是一直没有机会,如今一见我出言询问,齐王又是默然不语,只道是齐王故意给我难堪,他又是除了齐王之外的第二人,便开口道:“禀告先生,末将——”

他刚要说话,我突然脸一沉,喝问道:“荆迟,监军和元帅说话,你为何胡乱插话?”

荆迟一愣,连忙辩解道:“先生,末将无心插话,只是元帅没有回答,末将才多言了?”

我冷冷道:“岂有此理,一军之中,帅位只可一人独据,我和元帅说话,元帅又没有许可你代为回答,你怎敢多言,难怪我听闻你飞扬跋扈,目无尊上,今日一见果然如此,若非你平日无所忌惮,今日怎有胆子抢在元帅前面答话。”

荆迟先是有些委屈,可是他早已习惯将我的话翻来覆去的想上几遍,这一想居然冷汗直流,想到数年来自己虽然无意,在军务上和齐王多有纷争,甚至有时迫着齐王改变主意,虽然有时自己说得对了,可是这样子无礼,难怪齐王一派的将领总是和自己为难,荆迟不是笨人,想到昔日离京之时,皇上让自己好好支持齐王,自己却是如此行为,怪不得江先生要出言斥责。想通了之后,心中委屈全消,反而是心惊胆战,他可是知道江先生手段厉害,心肠钢硬。扑通一声跪倒在地,荆迟战战兢兢地道:“末将知罪,请先生责罚。”

我心道,这荆迟果然是仍然畏惧我昔日的余威,拿他开刀可是选对了人了,目光一扫,只见雍王一派的将领人人面有不安之色,看来这几年都是没有少给齐王添麻烦,而齐王一派的将领却是人人欢欣。

我故意露出冰冷的神色,道:“本监军承皇命监督众将,荆迟犯上不敬,有害军心,罪在不赦,呼延寿,你给我将他推下去斩迄报来。”

阶下众将立刻哗然,雍派将领看着那面寒似水的监军,心道莫非是监军和齐王合谋要铲除荆迟,可是这监军乃是皇上钦命,总不会偏向齐王吧。那些齐派将领虽然恼恨荆迟,可是数年并肩作战,却也对他颇为了解,虽有敌意却也不能不承认这人乃是难得的大将,若是杀了也不免觉得惋惜。这时,呼延寿已经寒着脸带了两个虎赍卫就要将荆迟推下去。

雍派将领虽然心中疑虑,可是看到那些虎赍卫的服饰,都知道这是皇上的禁卫,心道莫非是皇上有心杀了荆迟不成,更是不敢阻拦,有的更是担忧起来,若是荆迟不肯凭白送命,搅闹起来,可就糟了,那样我们也没法子替他求情了。谁知出乎他们的意料,平日飞扬跳脱的荆迟居然只是苦着脸束手就擒。若是换了别人,荆迟自然不甘心这样被绑起来,可是当日在寒园我早就磨得他软了,在我面前,荆迟怎也鼓不起勇气反抗,再说我身后站着一个小顺子,荆迟可是深知小顺子的手段的,自然更加不敢反抗,就是冤枉也喊不出口,他可是知道我的本事,当年在寒园他可没有少因为强辩而被我惩戒,所以荆迟心中早就有了成见,若是不含冤,或者还会没事,若是强辩含冤只怕是罪加一等。想到寒园里面堆着的那些他抄过的书籍,荆迟就不寒而栗。

等到呼延寿将荆迟带了下去,李显心道,怎么人都带下去了莫不是随云真的动了杀机,而不是装个样子而已。忍不住看了江哲一眼,道:“随云,还未开战,就斩杀大将,未免有些可惜,不如饶了他这一次吧?”

我淡淡道:“军中铁律,轻慢主将乃是死罪,若是人人如此,军中岂不失了规矩。”

这时,阶下众将一看不好,这个监军是真的铁了心要杀人了,雍派将领连忙纷纷上前恳求,不过这次可都是先给齐王行礼之后再说话了,齐王一个眼色,那些也是心有戚戚焉的齐派将领也是纷纷求情。我这才脸色温和地道:“既然众将都为他求情,我就饶了他这一次,传令下去,将荆迟杖二十,而后若再有怠慢上位者,定斩不赦。”

军令传下,又过了片刻,呼延寿等人带了上身精赤,血痕宛然的荆迟前来复命,我这才收起怒容,淡淡道:“荆迟,杖罚你也受过了,以后可不许再犯,陛下命你为副,你怎可如此糊涂,扰乱军心,以前的事情到此为止,今后不许再擅自行事,否则就是齐王殿下不管你,我也不会放过你。”

荆迟虽然受罚,心中却想,既已受刑,看来先生不会生气了,便欣然答应。我见他这些神态,知道他虽然听命,但是还没有戒惧之心,灵机一动,便道:“荆迟,方才罚你,乃是军法,你好歹从我数年,也算是我的弟子,作为师长,我也要罚你不从上命,这个刑罚你若是不想受,可以断绝师徒恩义,我就不再管你。”

荆迟一听连忙道:“先生尽管责罚,弟子并无怨言。”他可是颇以身为我的弟子为荣,怎肯破门而出。再说若是真的断绝师徒恩义,不说如今我的身份,就是别人的耻笑也是受不起的。

我微微一笑,道:“你也知道,我门下虽有铁律,可是对你却只有一种惩罚,小顺子,你待会儿到他帐中监督他抄写军规百遍,不许他偷懒,找人代写。”

李显忍不住笑道:“早就听说随云你最喜欢罚荆将军抄书,如今一看果不其然。”

荆迟苦着脸应诺,看看齐王,心道:“我可再不敢和他作对,罚我抄写什么兵书军规也就罢了,若是先生恼怒起来,罚我抄写那些四书五经可怎么办呢?

接下来,齐王给我引见了军中众将,其中有几人我颇为留意,樊文诚、黄龄,齐王身边亲卫军的统领,夏宁、罗章乃是齐王麾下有名的猛将,这四人都是齐王的亲信,当年太子李安就是拿了兵符也调不动他们。雍王方面的将领我虽然也认得几个,可是如今长孙冀远在关中,裴云屯兵长江北岸,司马雄更是统领禁军,如今自然都见不到,剩下的这些将领我虽然多半听过,却也很难引起我的注意。之后齐王下令十日之后全军大比,命众将各自准备,言语中隐隐暗示大比之后就要出兵攻打北汉,众将这几年本就隐忍得难受,听了这个消息自是人人振奋,都想着在大比之中占先,也好出战之时打头阵。

等到众将退下,我本想去自己的营帐休息,却被齐王硬扯到了他的寝帐,既来之则安之,反正我的营帐也得小顺子他们整理好了才能入住,所以我就舒舒服服的倚在齐王那张大床之上,而齐王则是似笑非笑地盯着我,好像等我问他什么。

我却是装聋作哑,好像不知道他在等我问他出兵之事,其实仔细想来,如果不是皇上和齐王都想着出兵平汉,又何必这么紧张两人之间的芥蒂呢,更用不着皇上亲自写信相请,还要派了虎赍来催我前去,齐王也未必就这么急着去请我,否则我就是再休息几年恐怕也不要紧。

过了片刻,李显终于苦笑道:“随云,你不要装聋作哑了,还是快点说说你对这次出兵有什么看法吧?”

我故意惊问道:“殿下何出此言,大雍规矩,监军不可过问战事,这些事情殿下自该去问军中大将和幕僚才是。”

李显气结,他却是聪明,眼珠一转,道:“随云,你可知道镇守边关事关重大,不得圣旨不能回京。”

我愣了一下,道:“自应如此。”

李显露出狐狸一样的笑容道:“若是我们和北汉对峙,别说是一年两年,就是三年五年,我也有法子让你不能回京,却不知道到时候慎儿还认得你么?”

我听了仿若晴天霹雳,心道,糟糕,我怎么忘记了这件事情,若是北汉不能攻克,我就不能回京,想到贞儿、柔蓝和慎儿,心中更是焦虑,想了半天,不由失笑道:“殿下可真是随云的克星,当年在南楚的时候,我对殿下可是戒惧得很,殿下的侍卫手一按上刀柄,我便立刻屈服,如今殿下的杀气我却是不怕了,却又被殿下拿家室来威胁,让我做监军,却不知到底是让我压制殿下还是殿下压制我啊?”

李显苦笑,道:“那是你没有准备对付我,否则大概我就是被你卖了还在替你数钱呢。好了,快些想想,这次皇上的意思就是除掉龙庭飞,只要此人一死,北汉就是迟早覆亡的局面,可是龙庭飞领军作战从无败绩,本王虽然骄傲,也知道没有必胜的把握,若是和他拼兵力,恐怕会损失惨重,到时候大雍元气大伤,又如何对付南楚呢。”

我见齐王心诚,暗道,罢了,若是困在这里,也是没有趣味,要想报复齐王还怕找不到机会么,再说,我既然来了军中,若是不理军务,只怕皇上那里也说不过去,还是平了北汉要紧。

我整理了一下思路,道:“殿下和龙庭飞比较,谁的军略强些?”

齐王想了一下道:“本王擅长战阵,在战略上似乎不如龙庭飞,而且此人在军事上面的天赋确实出色,本王应该不如他,不过是靠着兵多将广罢了,不过本王倒也自信,这龙庭飞就是本事再强,也不可能让本王一败涂地就是了。”

我摇头道:“殿下所说只对了一半,龙庭飞军略确实强过殿下,这些年来,他屡次进攻大雍,都是得胜而归,最次也是全身而退,北汉军骁勇善战,龙庭飞麾下颇有几个大将,再加上明时势,知进退,所以大雍屡次败在龙庭飞之手。可是殿下若是和龙庭飞作战,却也不会弱过他,只是殿下心中只想着铲除龙庭飞,所以才不免被龙庭飞玩弄于股掌之上。”

齐王有些迷惑,道:“随云你不是也认为北汉有龙庭飞才是我军挫败的主因么?”

我笑道:“正是如此,北汉若没有龙庭飞支撑,早就被大雍所破,可是这并不代表我们对付北汉就是对付龙庭飞啊?”

齐王想了一想,道:“莫非你是想离间龙庭飞和北汉朝廷的关系么,只怕是很难,现在龙庭飞迫得信任,又是准驸马,就是想要离间也没有这么容易。”

我摇头道:“离间并不容易,现在的北汉主虽然不是什么明君贤主,但是却有一样好处,就是敢放手,敢信人,龙庭飞得侍这样的主君,也是他的福气,这离间一策,用在龙庭飞身上却是无用的。就是有用,只怕也耗时太多。”

齐王道:“那么随云你是什么意思呢?”

我微微一笑,道:“龙庭飞用兵虽然千变万化,可是万变不离其宗,他用兵喜欢奇正相辅,常常自率大军,然后遣一军为偏师,或者自领大军攻城破寨,或者令偏师袭我侧翼辎重,我雍军虽众,却往往落得一个被他恃强凌弱的机会。”

李显有些尴尬地道:“正是如此,他每次用兵或者派遣谭忌飘忽我大军左右,或者让石英千里奔袭,我为了对付龙庭飞总是不敢轻易分兵,就是这样,一有松懈,还往往被龙庭飞所乘,这些年来,北汉屡次进犯,用兵都是千变万化,让我不明白龙庭飞是如何如臂使指地指挥偏师?”

我轻轻一笑,道:“你这是把龙庭飞想得太高了,他就是再有本事,也不能分出分神指挥偏师,殿下不见龙庭飞常用谭忌另领一军,而石英虽然也会独自出击,却往往一击而退,不似谭忌一般飘忽难测,应该说谭忌也是一个将才,只可惜光芒被龙庭飞掩盖罢了。”

李显若有所思地点点头道:“你说的不错,龙庭飞就是三头六臂,如果没有得力的将领,也不可能履战履胜,这一点竟给我忘记了,只因大雍将领多半都败在龙庭飞手上,所以对他颇为忌惮,却忘了他身边的几个大将的重要性。”

我冷冷道:“龙庭飞是北汉军的魂魄,他麾下的将领就是他的羽翼手足,既然龙庭飞不可轻攘,那么我们就先断绝他的羽翼,折断他的手足,消磨他的心志,打击他的信心,这样连番打击,龙庭飞是苍鹰,也要陷入罗网,就是猛虎也要虎落平阳,殿下还怕他能够翻出大雍的手心么?”

齐王只听得一阵心寒,良久才道:“我们应该如何进行?”

我也不回答,站起身来,半晌才道:“若是殿下肯依从我的计策,一件件按照计划进行,我可以担保一年之内,龙庭飞授首,北汉称臣,不知殿下可愿遵从?”

齐王正容道:“先生之命,李显无不遵从。”

我又道:“此事不可外泄,否则若是龙庭飞防备到了我们的手段,又要多费手脚,所以除了我和殿下之外,任何人都不能知道此中真相。”

齐王笑道:“这是自然,君不密,则失臣,臣不密,则失身,几事不密则成害。本王自然知道守密的重要性。”

我满意地道:“既然如此,我便进行第一步,十日后的大比正是好时机,我要选一个人。”李显目光一闪,没有说话。

十日之后的大比热闹非常,这次齐王下令只比较战阵,各军选出千人来以木制兵器互相交战,这一次的大比的结果倒是令人万分惊讶,因为荆迟杖伤初愈(实际上是我不许他出战,他实际上已是副帅身份,)怎可与众将争锋),故而他的这一军是由参军宣松领军的,宣松虽然通晓军机,可是武艺不高,很少领军上阵,所以人人都道他必败无疑,谁知这宣松居然指挥有方,十几场厮杀,竟然一场未败,就是不能取胜也能得个平手。

这个宣松我也听过他的名字,此人投雍王之后不就,就被派到荆迟军中做参军,后来荆迟常年滞留长安,都是此人领军,想不到竟有如此手段。我一边惊叹,一边问齐王道:“殿下,这样的人才应该让他作将军才是,怎么还让他做参军呢?”

齐王尴尬地道:“军中同僚多年,谁不知道宣松可以领军,可是大雍的规矩,不能上阵杀敌的就不能作将军,宣松虽然通军务,可是他是幕僚出身,又是文人,所以不能让他领军。”

我忍不住笑道:“当日东晋之所以衰败,是因为轻视武人,用文人统军,以至于外不能御蛮夷,内不能平叛乱,后来局势纷乱,各方将领纷纷割据独立,这都是重文轻武的害处。如今大雍想必因此定下不许文人领军的律条,只是未免矫枉过正,这样的人才不让他领军,真是暴跈天物,怪不得我见这些年来的战报,荆迟这一军是攻如烈火,守如磐石,我还奇怪呢,荆迟的性子,若是让他进攻,那是无敌的先锋猛将,若是让他防守,只怕是力有不逮,却原来有这么一个枪手。这样的功绩却让他屈居人下,至今连入帐议事的资格都没有,真是可惜。”

齐王听了不觉面红耳赤,其实若是李贽还在领军,只怕早就破格将宣松升为将军了,只是李显虽然不会故意为难李贽的旧部,却也懒得为了提拔偏向李贽的将领而更改旧例。

我装作没有看见,道:“不过这倒也好,这次宣松正可以派上用场,这样的大功立下来,殿下也可以名正言顺的保举他提升将军,让他自领一军了。”

李显连忙道:“就依你,就依你。”

我轻笑出声,目光飘向远处,那里荆迟正扯着宣松说些什么,离得太远听不清楚,可见他得意洋洋地拍胸膛的模样着实好笑。小顺子不知何时回到我身后,传音道:“荆将军是跟宣参军说,他和您关系很好,一定有法子可以让宣参军自领一军去做将军。”

我不由动容,想不到荆迟竟有这样的胸怀和眼光,倒也让我刮目相看呢。

附录 十七禁律、五十四斩

其一:闻鼓不进,闻金不止,旗举不起,旗按不伏,此谓悖军,犯者斩之。

其二:呼名不应,点时不到,违期不至,动改师律,此谓慢军,犯者斩之。

其三:夜传刁斗,怠而不报,更筹违慢,声号不明,此谓懈军,犯者斩之。

其四:多出怨言,怒其主将,不听约束,更教难制,此谓构军,犯者斩之。

其五:扬声笑语,蔑视禁约,驰突军门,此谓轻军,犯者斩之。

其六:所用兵器,弓弩绝弦,箭无羽镞,剑戟不利,旗帜凋弊,此谓欺军,犯者斩之。

其七:谣言诡语,捏造鬼神,假托梦寐,大肆邪说,蛊惑军士,此谓淫军,犯者斩之。

其八:好舌利齿,妄为是非,调拨军士,令其不和,此谓谤军,犯者斩之。

其九:所到之地,凌虐其民,如有逼淫妇女,此谓奸军,犯者斩之。

其十:窃人财物,以为己利,夺人首级,以为己功,此谓盗军,犯者斩之。

其十一:军民聚众议事,私进帐下,探听军机,此谓探军,犯者斩之。

其十二:或闻所谋,及闻号令,漏泄於外,使敌人知之,此谓背军,犯者斩之。

其十三:调用之际,结舌不应,低眉俯首,面有难色,此谓狠军,犯者斩之。

其十四:出越行伍,搀前越后,言语喧哗,不遵禁训,此谓乱军,犯者斩之。

其十五:托伤作病,以避征伐,捏伤假死,因而逃避,此谓诈军,犯者斩之。

其十六:主掌钱粮,给赏之时阿私所亲,使士卒结怨,此谓弊军,犯者斩之。

其十七:观寇不审,探贼不详,到不言到,多则言少,少则言多,此谓误军,犯者斩之。

第十八章    苍鹰折翼(上)

大雍武威二十七年十月二十七日,刚刚举行过军中大比,泽州大营上下都得到军令,准备出征,就在一切齐备之后,前线传来敌情,在泽州东峪出现了北汉的前锋游骑,李显听了探报皱眉道:“随云,怎么龙庭飞会这时候出兵呢,虽然他每年都会出兵攻打泽州,可是基本上不是在春耕时分就是秋收季节,如今新粮已经入仓,他这时来进攻未免有些奇怪?”

我披着长衣,在灯下看着地图,淡淡道:“今年春天,龙庭飞曾经入寇泽州,所以秋天不来也没有什么奇怪,不过此人通晓军略,我们大雍这样大的动作,殿下你亲入东海,哲重入军旅,皇上和殿下又是忙着筹备物资,整顿军马,这种种征兆都说明大雍将有战事。龙庭飞手下肯定有人时时探听我方军情,如今沁州已经降霜,天气开始转为寒冷,但是距离大雪封道还有一月之期,也算的上是我们进军的一个好时机,龙庭飞定是担心我们进军沁州,劫掠杀戮,然后在降雪之前毁掉他们的存粮,这样接下来的冬天北汉可就难过了。我们兵强马壮,若是进攻沁州,龙庭飞就是天大本事也不能面面俱到,与其被我们困着攻打,自然不如以攻代守,先下手为强,等到他们退去之后,明年春天之前我们就不能进攻了,再说了,前些日子北汉军伏击我们的事情,就是我们没放在心上,你以为他们会相信我们不记仇么。”

李显其实心中也有类似的看法,他看向我道:“既然如此,我们就在泽州给他们一个教训好了,以逸待劳也没有什么不好,你觉得我们该如何进行呢?”

我指向沁水与十里河交界之处的庙坡道:“殿下,你觉得这里如何,这可是个瓮中捉鳖的好地方?”

李显看了一会儿,道:“你认为派谁去比较好?”

我轻轻道:“名义上镇守那里的应该是荆迟,可是实际上主持那里的人是谁就要看殿下你的器量心胸了。”

李显眼中闪过一丝寒光,道:“本王明白你的意思,你放心,这次本王就按照你的计策行事,什么功劳面子,只要最后我军胜了,本王就是天大的功劳,难道还会去和部将争功么?”转而又道:“不过本王担心他们不会入圈套。”

我微微一笑,道:“庙坡这里存放着泽州大营的所有粮饷,若是敌军都想不到攻击这里,那么除了和我们硬碰硬,就没有任何胜算,难道同等军力比拼之下,我们又是以防御为主,难道还会落败么,兴兵犯境,若是不能因粮于敌,那么就是打个平手,也是败了,而且还请殿下放心,龙庭飞就是再厉害,也不能轻易赢了这一局。”

李显这才放下心来,看着地图道:“按照斥候的回报,后日龙庭飞的主力就会到达秦泽,而我们正好在那里迎战。”我点头道:“秦泽地势平坦,虽有些荒原丘陵,但是还是很适合两军作战,也难怪这些年来大雍和北汉基本都会选在这里决战。”

李显感叹道:“是啊,秦泽荒原之上野草繁茂,都是浸透了两国战士鲜血的缘故啊,本王和龙庭飞在秦泽交战至今已有四次,本王都是只能勉强全身而退罢了,我大雍在秦泽埋葬了无数忠勇的将士,这次本王要让龙庭飞受到折翼之痛,希望他够聪明,可别让本王望穿秋水才好。”

我胸有成竹地道:“这个王爷不用操心,我们留下的线索,足够他们发现庙坡乃是我们存粮之处,而且北汉军将领自负得很,就是发现可能有问题,也会想尽办法来达成任务,可是这次他们却会发现撞上了铁板。”李显微微一笑,没有说话,除非是龙庭飞亲自带兵偷袭,其他的北汉将领还没有被李显放在眼里。

飒飒秋风,荒草离离,毗邻秦泽北面的十里原广袤荒凉,几个身穿暗黄色软甲的大雍斥候伏在丘陵之后盯着远处的天际,丘陵下面,几匹战马在那里悠然的吃着草料。其中一个斥候有些疲倦地揉了揉因为长期望着远方而觉得酸涩的眼睛,就在这时,他的同伴惊道:“敌军来了。”他连忙抬眼望去,只见苍穹尽处,碧蓝的天空背景上,突然浮现出一条深棕色的曲线,不过是呼吸之间,那起伏不定的线条越来越清晰,在黄色的原野上飞速的移动着,又过了片刻,已经可以看清楚那线条是由成千上万北汉铁骑组成的,而在一片深棕色当中,最耀眼的就是位于骑阵中心的一片火红。而几个斥候也看到了在头上的天空里翱翔着的几头苍鹰,这是北汉军用来探听军情的猛禽。而这些斥候都十分清楚,大雍和北汉的统帅都有一个共同的爱好,就是让自己的亲卫穿着红色甲胄,不过虽然都是红色,在战场上倒也容易分辨,不说盔甲的样式不同,齐王的亲卫使得都是马槊长矛,而龙庭飞的亲卫却都是硬弓劲弩。几个斥候知道龙庭飞果然已经亲率大军入侵泽州,自己的行踪恐怕也已经被头上的苍鹰发觉,若是再呆下去只怕是没命回营了。便悄无声息地上了战马,策马飞奔,回去报告军情去了。

又过了一阵子,北汉军已经到了近前,原本飞奔中势如潮水一般汹涌起伏的散列队形迅速的集中收缩,这一收缩阵线,那狂奔如雷的战阵的气势越来越凌人,带着令人窒息的压力,令人相信若是前方有人挡路,定会给这支铁骑撕个粉碎。在距离丘陵数里之外,速度开始减慢,然后在那些大雍斥候监视的丘陵下面停了下来。只有百骑左右的红甲骑士簇拥着一个穿着火红战袍的将领速度不减,直接冲上了丘陵,然后停住战马。

那红袍将领掀起掩盖住面容的面甲,露出一张英俊的绝无瑕疵的面容,深邃得犹如渊海的深碧双目露出湛然的神采,俯视着眼前那渗透了大雍北汉勇士鲜血的原野,那睥睨天下的傲然身姿仿佛山峰一样高大。他身边的近卫和身后的千万北汉军勇士望着他的背影,眼中都露出甘愿效死的狂热光芒。

这时,有四个近卫排众而出,同时高声发出了节奏轻扬的呼哨,天上盘旋的苍鹰都是一个俯冲,分别落到了四个近卫的左臂上。而策马而立的龙庭飞似乎没有丝毫留意,只是目光澹澹地看着将要展开血战的沙场。又过了片刻,整军之后的各军主将都策马上了丘陵,恭恭敬敬的立在龙庭飞马后。

北汉众将几乎都是身材雄壮,英气迫人,但是其中却有一个青袍将领与众不同,他身材瘦削修长,虽然也是身高八尺,却是没有什么凌人的气势。可是他面上覆着一个相貌狰狞的青铜面具,只露出一双如同冰雪一般冷酷无情的幽深黑眸。他马上挂着的兵器乃是一柄长戈,通体漆黑如墨,只有开刃处如霜雪一般寒芒四射。若是一眼望去,只觉得这人似乎落落寡合,和众将都隔着一段距离,除此之外也不觉得有什么异常之处。可是其他将领望着他的目光却都是有些畏惧退缩,仿佛这人乃是天地间最可怕的存在一般。

龙庭飞没有回头,可是他能够感觉到身后那种诡异的气氛,心中轻叹一声,他不是不知道麾下众将对鬼面将军谭忌的排斥和忌惮,可是谭忌是他不可缺少的臂膀,也就只好委屈众将了。龙庭飞麾下人才济济,虽然先锋将军苏定峦身死雍都,可是如今魔宗派来的三位弟子鹿伯言、鹿仲天、鹿叔函却几乎都可以和苏定峦比肩,每次出兵,磐石将军段无敌必然在后面严阵以待,稳稳地守住北汉军的退路,飞虎将军石英如同一柄利剑,可以轻而易举地刺透敌人的要害,可是他们却都不如谭忌来得重要。

鬼面将军谭忌,出身本是泽州和沁州交界处的山中猎户,住在深山之中,既不完粮也不纳税,心中也无国家之念,可是十四年前,北汉和大雍对峙之时,大雍的一位将军在领军偷袭敌营的时候,路过了与世无争的谭家寨,为了守秘,那位嗜血的将军下了屠杀令,除了带了所有积攒的毛皮,想到山外给心爱的女子买一支金钗的谭忌之外,全寨二百余人被杀得干干净净。而心急难耐连夜赶回的谭忌就在雍军大肆屠杀之际返回了山寨,可是无能为力的谭忌只能躲在山梁之后,眼睁睁的看着家园尽毁。自知无力报仇的谭忌一把大火焚毁了山寨,然后穿越一条只有寨中猎户知道的崎岖山道赶在雍军之前进了沁州。之后,那位千里偷袭沁州的将军被严阵以待的北汉军围歼,当时还是偏将的龙庭飞麾下多了一个亲兵近卫。

之后谭忌从一个小卒逐步升到了将军,虽有龙庭飞赏识,却也是艰难万分,只因谭忌惨遭巨变之后,性情大改,不喜与人交谈,又以严苛军法带兵,同僚部下都是畏之如虎,就是比他位阶高的也都对他心存忌惮。北汉军的习俗,喜欢勇猛狂放之士,这样一来不免多些骄兵悍将,可是谭忌麾下却是军律森严,稍有违犯,就是杖责鞭打,若有再犯,就是斩首,初时有人不服,以勇力抗之,可是谭忌虽然外貌文质彬彬,手段却是残狠无比,将那些反抗的军士全部擒拿处死,并加上叛逆犯上的罪名,连家人也受到牵连,这样一来,再没有人敢触犯谭忌。军规肃然之后,谭忌便用心训练了一支精锐的骑兵,这些骑兵都是远攻近战,马上马下都十分出众的勇士,长戈、马刀、硬弩是他们随身必带的武器。谭忌又从龙庭飞学习战阵,而且可以说是青出于蓝,沙场之上,可以凭着骑阵击败数倍之敌,或许是因为相貌不够威武,谭忌几乎是终日带着青铜鬼面,所以人称鬼面将军。后来谭忌又在军中选了一批资质过人的勇士,亲传戈术,以其中最出色的三十六人为亲兵,更让这些人都戴了和自己样式相同的青铜面具,称作鬼骑,这些鬼骑只带长戈,最擅冲刺,每逢战时,就奉了谭忌之命,或攻敌人软肋,或遇强用强,摧敌之锋锐,这三十六鬼骑乃是谭忌用来摧毁敌军战意军心的利器,谭忌又是不断训练候补,如有阵亡立刻补上新人,谭忌的威名倒是大半都是这三十六鬼骑替他睁来得。

不过若是仅仅如此,也不至于人见人畏,这谭忌最令人诟病之处就是嗜杀,虽然战乱纷呈,从军杀敌,没有不杀人如麻的,可是却也有个底线,杀俘不祥,因果报应,也是几乎人人都信的。征战之初,虽然也有杀戮平民,肆虐妇孺的事情,可是随着天下局势渐渐清晰,若是没有必要,这残杀无辜的事情也是能不做就不做了。可是谭忌却是分外的冷酷无情,兵锋所知不留俘虏,大军所过之处鸡犬不留。这样的残狠,就是同僚的北汉将领也是难以忍受的。幸好还有龙庭飞时时耳提面命,管制拘束,否则这谭忌恐怕就会更加过分。这样一个精通战阵,所向披靡的将领,又是心如铁石,冷酷无情的人物,怎不令人戒惧呢?

龙庭飞心中又是叹了一口气,其实他虽然器重谭忌,可是却不喜欢他。依着龙庭飞的个性,是不喜欢谭忌这种阴狠残酷的手段的,可是龙庭飞却又知道,除了谭忌之外,麾下众将还难以独当大雍铁骑,而且谭忌的残酷手段,也是北汉军能够在大雍境内肆虐的重要保证。若非是大雍边民畏谭忌如同鬼魅虎狼,北汉军恐怕会阻力重重,因此虽然谭忌多有不为龙庭飞所喜之处,却是始终得到龙庭飞的重用和信赖。

收起无奈的心思,龙庭飞微笑道:“谭将军,你说我们这次应该如何进攻呢?”

青铜面具之后传来幽冷的声音道:“大将军心中自然早有成算,末将无知,却也知道我军不过十万,敌军却有三十万,若不能攻敌之必救,摧敌之肝胆,就是有败无胜,若是将军许可,末将愿领一军,尽毁敌军根基。”

龙庭飞满意地道:“谭将军说得不错,我军远来,敌军占了地利人和,我们若是不能出奇制胜,就是有败无胜,萧桐,你来告诉大家我们得到的情报。”

龙庭飞近卫之中,曾经随着林碧去东海的萧护卫排众而出,朗声道:“大将军,诸位将军,属下奉了将令探察敌情,已经得到敌军动向,这次敌军统帅带了十五万主力,明日就会到达秦泽战场,据探马回报,领军的是齐王本人,监军江哲也随军同行,而敌军辎重大营安在庙坡,负责镇守的是敌军副将荆迟,大概有三万人马。另外泽州境内分散驻守十二万大军也已经开始集结。”

萧桐乃是魔宗爱徒,专司负责搜集军情,他手下多有魔宗高手,搜集到的情报自然不会有差,可是众人面面相觑,其中一个将军问道:“萧护卫,荆迟乃是敌军大将,怎么去守辎重呢,这不是杀鸡用牛刀么?”

萧桐笑道:“将军有所不知,这荆迟虽然是敌军大将,又是雍帝心腹,却是和齐王不睦,如今雍帝正一门心思和齐王修好,这荆迟自然要受点委屈,末将得到情报,监军江哲初入泽州大营,就杖责了荆迟,所以齐王就趁机剥夺了荆迟的职权,将他贬到后方去守辎重。荆迟原本带着七万大军,也被齐王拆分了,只让荆迟带着三万人去守辎重,另外四万人被齐王留在了中军。”

另外一个将军笑道:“李显还自称能带兵呢,这样趁机报复,也未免心胸狭窄了一些。”

谭忌冷冷道:“这也未必是李显心胸狭窄,你们不是也听说过,这两年多来,荆迟也没有少给齐王掣肘,这种良机,李显若不利用,也太可惜了,不过这李显还是手下留情的,若是他存心对付荆迟,就是让他去送死也未必不行。”

他这一说话,众将都是默然不语,场中顿时充满了尴尬的气氛,龙庭飞心中一叹,朗声道:“荆迟也算是出色的战将,他带着三万人马守护辎重,我们想要一举摧毁敌军要害,也是十分艰难,李显这样做也不算是大材小用,谭忌,你可有信心将荆迟部击败,尽夺敌军粮草辎重。”

谭忌冷然道:“荆迟虽然是一员猛将,可是性情如烈火,对付这种人,末将自有把握,请大将军放心,末将必定让敌军进退两难。”

龙庭飞满意地点点头,道:“好,谭忌听令,我给你精兵一万,命你在十日之内,败荆迟,破敌军辎重,之后你可自由行动。不过一定要在十一月月底之前返回本部,你可有异议?”

谭忌幽幽道:“末将领命。”,那声音带了几分狂热。北汉众将听了都是心中战栗,若是谭忌自由行动,泽州又该是血流成河了,多年征战,谭忌曾经三次攻破固县,两次血洗河西,甚至曾经深入到端氏镇和嘉丰一带,就连泽州重镇的晋城周边也曾经被谭忌洗劫过。对于泽州军民来说,谭忌是可以止小儿夜啼的魔鬼。

龙庭飞轻轻一叹,若不是北汉兵微,何必要用此凶人残杀无辜百姓,可是这种事情却总是要有人去做的,除了谭忌,还有谁能去担这个恶名。

纵马下了丘陵,谭忌被亲信的三十六鬼骑簇拥着走进自己的中军,原本谭忌自负戈术高明,并不喜欢这样的保护,可是两年多前,凤仪门的杀手就是趁着鬼骑冲阵之时,化装成亲兵模样,将谭忌刺成重伤,若非谭忌武功高明,再加上亲军舍命保护,只怕谭忌已经命丧战场,从那以后,谭忌就时刻留心自己的安危,三十六鬼骑若不上阵厮杀,就终日和谭忌形影不离,他们都和谭忌穿着相似的衣甲,再加上都是带着同样的青铜面具,除了亲信之外,根本无法迅速有效地分辨他们的身份。若是鬼骑冲锋之时,谭忌若是没有一马当先领着他们冲锋,就是以鬼骑的候补人选为近卫,其他人根本不能接近谭忌身侧,这样一来,若想刺杀谭忌,没有宗师本领,根本就是难逾登天,非是贪生畏死,在谭忌看来,就是死,也应该有足够的亡魂陪葬。

谭忌其实很明白龙庭飞对自己的观感,对于他这样敏感的人来说,龙庭飞虽然没有明确表示出来,可是那种隐隐的厌恶和排斥,其实谭忌心中了如指掌,可是他从来却不怨恨,是龙庭飞亲自传授兵法给他,也是龙庭飞请名师传授他武功,他也知道龙庭飞其实是看中了他的残狠和冷静,他只是龙庭飞手中的利器,甚至有一天,龙庭飞会在无法忍受他的作为之后将他弃如敝履,可是谭忌却是不能改变自己的做法,他明明知道,只要他肯约束自己的行径,不要超过众人忍耐的限度,就可以得到龙庭飞的真心信赖和众将的接受。可是对于谭忌来说,他的人生早就在家族尽毁的那一刻就已经死去,当他看着心爱的女子裸身躺在血泊之中,当他看着白发的祖父被生生钉在门板之上,当他看着敬爱的父亲死不瞑目,仍然张手护着年幼的弟妹,当他看着慈爱的母亲咬舌自尽的惨状,谭忌早就没有了任何对人生的眷恋。

可是他心中的怨恨是如此深重,让他即使看着仇敌死在北汉军的马蹄之下也仍然不能消解,所以他选择了从军,将手中的屠刀挥向曾经的乡亲,他恨那屠杀自己族人的大雍军队,恨那些为了保全财产性命全力支持雍军的泽州百姓,只有血火才能让他心中的悲痛暂时消解缓和。紧握手中的长戈,谭忌眼中闪过冰凉的杀机,就让这长戈沾染更多的鲜血,用来祭奠他父母亲族的亡灵吧。

十月三十日,在急行军之后修整了一夜的雍军从秦泽南面进入了战场,距离今年春季的那一次双方都很克制的交战之后,改变北疆局势的秦泽会战开始了,这一战,十五万雍军和九万北汉军,在方圆百里的秦泽尸堆如山,血流成河。

而在同时,带着万余精兵的鬼面将军谭忌,顺沁水而下,直奔庙坡,所过之处,因为雍军依然坚壁清野,并无人迹,可是谭忌仍然下令哨探齐出,若遇生人,尽皆斩之,就在十一月二日,谭忌已经遥遥望着庙坡辎重大营,开始筹划如何歼敌取胜了。而这时,秦泽之上,两军经过初期的试探布阵之后,战局开始展开。

策马站在高坡之上,齐王李显的王旗和帅旗在寒风中狂舞,火红色的铁骑将中军护得水泄不通,在帅旗之下,一个穿着金甲,骑着火红色的战马的大将左侧,我仍是披着那件特制的青色大氅,俯视着千军万马,在我身后,小顺子白马银枪,目光冷淡如冰,而在我身侧,一个身穿轻甲,外罩青色战袍的中年人手提马鞭,若有所思的望着下面的战局,他相貌儒雅斯文,细眉长目,文质彬彬,虽然穿着甲胄,可是除了腰间悬着佩剑之外,却是没有任何其他兵器。他不时传下各种谕令,由他身后那些赤色甲胄的齐王亲兵飞快的传下军令,指挥着前面的战事。而我的目光却是透过重重阻碍,落到远处敌军中那一片火红当中,在那迎风飘扬的赤龙旗下,有一个纵在千军万马当中也是佼然不群的峻拔身影。

这时,龙庭飞在指挥作战的同时,也在留心着敌军的中军大营,那大雍皇室的旗帜下面,那和自己敌对了数年,越战越是顽强的敌人,齐王李显,以及他身边那总在沙场之上,也是意态悠闲的青衣书生。这就是自己面前的敌人么,龙庭飞心中涌起强烈的斗志,可是转瞬他又冷静下来,他的目标不是尽歼敌军,而是尽量的消耗敌军的军力,在谭忌的配合下蚕食鲸吞雍军的实力,只有这样,他才能让北汉军越战越强,甚至可能让雍军再无力进攻本国。

可惜啊,龙庭飞心中涌起一丝无奈,按照他的计划,本已经在大雍朝野挑起了针对齐王的狂潮,可是这些在江哲出任监军之后就遭受了巨大的挫折。在中书令郑瑕、尚书右仆射石彧的主持下,联手压制了朝中对齐王的弹劾和攻击。这个江哲江随云,不过是小小的举动,就让自己一番苦心付诸东流,也难怪公主要想尽办法伏杀此人,可惜石英功败垂成,龙庭飞眼中闪过一丝沮丧之后,继而又鼓起信心,心道,这人就是智谋再高,只要我用兵没有差错,还怕他掀起什么风浪么?想到这里,龙庭飞微微一笑,道:“三位鹿将军,你们领本部下去冲杀一阵子,我见敌军右翼有些动作迟缓,良机不可错过。”

第十九章    苍鹰折翼(中)

鹿伯言、鹿仲天、鹿叔函本是一胞所生,一般相貌,一样勇猛,又是心有灵犀,被魔宗收为弟子,传授武艺,三人联手攻击之时,当真是所向披靡,是苏定峦之后北汉军最出名的先锋,他们观战多时,早已经心痒难耐,见龙庭飞下令,都是轰然应诺,各自策马飞奔到本部中军,准备厮杀。

雍军出动了五万步兵,弓箭手,长矛手,藤牌手参差错落,层层叠叠,摆了一个固如金汤的大阵,而七万骑兵隐在步兵阵后,钢浇铁铸的精锐骑兵纹丝不动地等待着中军的号令,除了偶尔有骑兵轻轻安抚一下被战场上面的惨烈气氛吸引得跃跃欲试的战马之外,几乎没有任何多余的动作。还有三万步兵按照中军的指挥随时准备替换疲乏的同袍,步军大阵之中杀气隐隐。

而北汉军都是骑兵,三万骑兵游弋在雍军阵外,强弓硬弩寻找着雍军的软肋,一层层的削弱着敌军的防守。这是一场拼实力的大战,没有丝毫取巧的余地。鲜血飞溅,染红了原野,满天飞舞的弓箭不时地带起血雨。

经过了半天的苦战,北汉军面对坚韧的敌军始终不能取得满意的战绩,龙庭飞也是将北汉军轮换上阵,双方几乎是在进行着消耗战。而到了午后,雍军的右翼因为被连续的猛烈攻击,终于有些支撑不住,北汉军的攻击过于频繁,让这一面再也无法换上生力军。就在这时,龙庭飞出动了鹿氏兄弟。

鹿伯言手持马槊,他身后的骑兵都是使用马槊长矛,这只骑兵主要就是担任攻坚的任务的,不过他们身上仍然带着小巧的复合弓,需要的时候也可以担任游猎的角色。鹿伯言手持马槊,高声道:“随我来。”说罢一马当先冲进了雍军的右翼,两军撞击在一起,将雍军的防线再次削弱,这时,鹿仲天和鹿书函也带着自己所部随后冲进了雍军右翼,他们三人配合十分默契,进攻的势头减弱之后便飘然远去,由另一人接替攻击,他们之间的交替攻击几乎是毫无缝隙,连续的猛烈强攻终于撕裂了雍军的防线。如同潮水一般涌入雍军阵内的北汉军开始了肆意残杀,血肉横飞。

这时雍军中军传来了号角长鸣的声音,雍军右翼如闻纶音,拼命抵挡北汉军的步兵向两侧分散。在他们身后露出了青色衣甲的大雍铁骑,马蹄如雷,他们硬生生地迎上了北汉军攻击最猛烈的骑兵。两军绞杀在一起,这一刻战场的重心就在这里。

鹿伯言已经和两位弟弟汇合在一起,三人同声高声嘶喊,他们都是越强愈强的勇将,一时之间竟然和大雍重骑斗了一个旗鼓相当。这时北汉中军传来高亢的号角指挥声,鹿伯言脑中一清,知道自己不该和重骑兵硬碰。他手一挥,高声呼道:“冲他们的中军。”说罢带着部下转向大雍中军的步兵,而他的两位弟弟也娴熟的接替他留下的空缺,骑阵变换自然流畅,北汉骁骑如同利刃一般切入了大雍的中军。

我在大雍中军帅旗之下将敌军的变阵看的清清楚楚,不由动容道:“好一支骑兵,江某早就听闻北汉骑兵骑战天下无双,今日一见果然名不虚传。”

那穿着金黄甲胄,面具放下的的骑士闷声闷气地道:“北汉先锋骑兵确实精锐,这还是换了统领之后的表现呢,虽然战术更加精良,可是比起从前先锋将军苏定峦带领这支骑兵的时候,气势已经弱了很多。不过我们大雍的铁骑也不比他们差,只是可惜他们都是轻骑,往来自若,我们的骑兵速度不如他们,泽州一地又是一马平川,最适合他们纵横,若是两军直接交锋,他们的轻骑还是不如我们的铁甲骑兵的威力大。大人你看,现在北汉骑兵不是已经避开了我军重骑的锋芒了么?”

我看得也是连连点头,道:“你说的不错,不过别忘了你现在是扮着殿下,可别乱说乱动。”

那个骑士嘟囔了一句什么,没有继续说话。

这时,宣松已经传下军令,大雍的中军彷佛化成了海洋,将那支北汉骑兵的洪流汇入其中。随着大雍连续投入兵力,我可以清晰的看到在他的指挥下,那支北汉骑兵越来越艰难的移动着,这时,北汉军也再次出动了两万骑兵,意图从外围击穿大雍的军阵,可是这军阵却是非常坚韧,抵挡着内外的夹攻,而大雍的重骑兵也再次发威。一次次的撞击着北汉军的软肋。接下来的作战简直是令我眼花缭乱,双方的用兵方式都是精准而无情的,不过我还是能够看出来,北汉军的进攻犀利而变化多端,宣松的用兵却是坚韧而平稳,双方几乎是有序而冷酷地消磨着生命和时间。直到夕阳西下,北汉军终于突破了大雍的军阵,在龙庭飞亲自断后下缓缓退去。宣松也趁势收兵,其实若是认真说起来,龙庭飞不是不可以早些让骑兵成功突围,只是那样一来未免损失惨重,也不会有现在的战果,而最后宣松也不是不可以强行阻止北汉军一段时间,只是这对于今日的胜负结果并没有什么帮助,只是会增多无依的损伤,所以最后双方可以说是颇有默契地各自退兵了。这一日,北汉军留下了将近六千具尸体,而大雍军则是伤亡两万五千多人。并非是龙庭飞的指挥强过宣松太多,而是大雍军今日乃是以步兵为主力,而北汉军却是来去如风的轻骑。这样的伤亡比例已经是不错的结果了。这也是没有办法的事情,双方的主将都没有犯什么过分的错误,就只能这样消耗生命和战力了,大雍铁骑虽然杀伤力更强,可是若是重骑轻易出动,不是被龙庭飞找到空隙,令我军损失惨重,就是龙庭飞不愿和我们硬拼,转而和我们游斗,这样一来,就失去了缠住北汉军的可能。

北汉军大都是轻骑,又是人带两马或者三马,行军速度比我们快得多了,我估计龙庭飞若不是想缠住我军主力,恐怕未必会和我们正面作战呢?而对于我军来说,若是不经过这样一场血战,就不能让北汉军相信我军的主力全部在此地。从前北汉军入寇,常常是四散侵扰,可是自从数年前齐王重镇边关,就建立了坚壁清野的防御体系,所以北汉军若是想要攻城拔寨,必然是艰难万分,而且还很容易被齐王大军断了归途,所以北汉军也就改了作战方式,龙庭飞常带大军和齐王盘旋,而另遣偏师入侵泽州内部,若是齐王想要严守不出,那么北汉军就可以从容地攻破外围的城寨,若是齐王前来和北汉军主力作战,那么偏师就可以自由来去,若是齐王想要先去堵截偏师,那么龙庭飞就可以率北汉军主力从后追袭,而且谭忌最善偷袭遁逃,石英又是行军迅速,虽然大雍军队强过北汉,却是被北汉军迫得应接不暇。所以这几年来,齐王多半都是带兵和北汉军主力大战一番,而那支偏师就只能依靠各地的防守力量,为此不断地收缩防线,泽州一带几乎是人烟散尽,都是这几年征战连绵的结果。

这次,齐王采纳了我的建议,以宣松为主将迎战龙庭飞,亲自带兵去迎战或者说是诱歼谭忌,这绝对是出乎意料的决定,大雍众将本来没有可以敌对龙庭飞的,谁会想到如今越来越有把握逼退龙庭飞的齐王会不亲自领兵呢。不过这也是幸亏还有宣松的存在。我本来是想实在不行,我就亲自领兵,加上众将的协助,至少可以勉强打个平手吧,如今有了宣松,我就可以放心了,毕竟我没有真的指挥过作战。

我佩服地看看宣松,称赞道:“宣参军果然是用兵老练,龙庭飞之意也不是在于决战,依我看明日他就不会这样猛攻了,对于麾下兵马的爱惜,他只有在我们之上。想要让龙庭飞没有多余的精力怀疑殿下不在军中,就要看宣参军的本事了。”

宣松望着江哲那张平静的笑脸,心中不由生出无限的感激,他本是文人,可是从军之后,他却越来越发觉自己更适合指挥作战,可惜大雍约定俗成的规矩,想要独自领军,必须能够上阵杀敌,若是武艺不精,就断然没有作将军的机会。这些年来,虽然宣松可以说实际上领着一军,可是却始终不能正位。初时,是因为荆迟不在军中,所以宣松代为主掌军务,后来荆迟重新领军上阵,麾下却是领了两军,这本是李贽为了加强荆迟的实力,而荆迟见自己颇有带兵的本事,索性便让自己自领一军,可是名义上他仍然只是一个参军罢了。直到日前大比,自己大胜众军,荆迟笑嘻嘻地说要替自己说项。当时宣松心中虽然欢喜期待,却也是惴惴不安,他自然知道江哲此人,虽然入雍王幕中比自己要晚,可是这人的身份可是不同寻常,乃是雍王最亲信的心腹,若是他能够替自己说一句话,那么自己多年来的期待就可以梦想成真。可是宣松也听荆迟说过,这位江大人似乎生性有些疏懒,无关之事从不插手,所以也不敢抱了太大期望。谁知当夜自己便被召入齐王大帐,并被授予临时指挥大军的重任,只要这次自己能够成功的阻挡龙庭飞的步伐,那么战后必然可以得到擢升,想要独自领军再非梦想,这一战关系重大,所以宣松始终战战兢兢。如今好不容易撑过了一天,宣松不由松了口气,擦了擦额上的汗水,在马上行礼道:“还要多谢监军大人,若非大人推荐,宣某焉有指挥全军的机会。”

我笑道:“这也是宣参军多年来厚积薄发,才有今日的成就,在下不过是多说了几句好话罢了。”

这时那身穿金色盔甲的“齐王”在马上伸了一个懒腰,苦恼地道:“大人,不若明日让乔祖做替身吧?不能上阵杀敌,还得披着这一身重铠,真是万分痛苦。”

这时他身后担任侍卫的乔祖不由求饶道:“大人,我哪里有殿下的风范,还是让马肃来扮殿下吧。”

我不由笑出声,道:“放心,你们一个都逃不了,这几日都要轮流做殿下的替身。”马肃和乔祖不由同时痛苦的呻吟了一声。我心中暗笑,心道,当日在猎宫你们四人奉了齐王之命将我从含香苑掳到齐王居处,虽然是救了我的性命,可是却也没有安着好心,后来还几次劝齐王杀我,免得留下祸根,虽然说最后齐王没有采纳你们的建议,可是此仇不能不报,陶林和庄峻在齐王身边,今次无法报复,你们落到我手上,哪有不报复的道理。今日我不过是让你们扮扮齐王殿下,虽然是得一天端着架子不能乱动,可也不算是太难熬,而且从今之后恩怨两清,你们还是占了大大的便宜,那两人说不定没有你们运气好呢。心中这样想着,嘴角不由露出得意的笑容。乔、马两人只觉得一阵心寒,心道难怪他指名让我们两个留下的时候,殿下那种笑容呢,又是吞吞吐吐的说什么江大人喜欢记仇,却原来这位江大人的性子是这般睚眦必报。想到这里,两人心中不知是喜还是忧,若是这样了结了过去的过节,倒也不错,就是不知道这十几日到底会给他怎样戏弄,想到这里,也不知道对两位随侍齐王的同伴是羡慕还是同情,毕竟他们迟早也会落到这位监军的手上。

这时,小顺子上前道:“公子,明日你还要在战场上待上一天么,我见你气色不是很好。”

我抱怨道:“这里风沙又大,坐在马上一天,累也累死了,若不是我得在这里替齐王殿下掩饰,早就让你驾了马车来了。”

这时,已经安排好退兵事宜的宣松走过来,关切地道:“大人明日不妨带了营帐来,可以在里面休息片刻,只要不时露个面,应该不会引起对方的怀疑的。”

我笑道:“不用多虑了,明日应该龙庭飞不会再这样拼命了,他这点家底若是拼光了,也不用我们忧心如何进攻北汉了,宣参军还是想想怎样和他周旋吧,只要撑过十日,齐王殿下那边应该就可以传来捷报了。”

当夜,我们在秦泽南面三十里之处扎营,到了晚上,我正睡得朦朦胧胧,只听见帐外突然传来喊杀声,我连忙起身,披上大氅,小顺子就睡在外帐,他见我从内帐出来,低声道:“是敌军偷营,公子不用担心。”

我有些紧张,虽然宣参军说过敌军可能会偷营,事先做了准备,可是我还是很担心被敌人得手。不顾小顺子的拦阻,我走到帐门外看去,只见黑夜之中,火光四起,无数阴暗的影子在营外旷野中中穿梭而过,夜色昏暗,过了片刻,北汉军大概是见我军营盘守得严密,便如潮水一般退去。而就在北汉军刚刚撤退的时候,从另一处营门暗暗掩出的雍军一部齐声呼喝,弩箭齐飞,不过北汉军也是早有防范,悄然隐入了黑暗之中,双方都没有过多的损失。

我心中刚刚舒了一口气,突然后营火起,却是北汉军二次来袭,这一次他们也没有入营,只是点了火箭射入营盘,宣松连忙下令救火,等到反击的人马出寨,北汉军已经退去了。一夜之间,北汉军数次前来侵扰,北汉军飘忽不定,我军可没有法子在夜里和他们缠斗,虽然没有损失多少,可是却是一夜无眠。到了第二天,日上三竿的时候我还是有些呵欠连天,倒是那些将军军士却是轮流休息,虽然精神也不好,却不像我这般萎靡。看来他们早就有这样的准备了,问过宣松等人才知道,北汉军最喜欢偷营,大雍军也曾想回敬过去,可是每次想要偷营,不是给人伏击,就是陷入重围,所以索性只是守稳了营盘,将靠近外侧的位置布置上重重岗哨罢了。我心中不快,心道,都是偷营,怎么他们就这么容易得逞,我们却是损兵折将,问过众将,才知道北汉军最善长使用鹰隼和獒犬,鹰隼可以在白日行军的时候查看敌情,獒犬却可以在晚上守夜,据说我军若是接近敌营十里之内,就难以避过獒犬的鼻子。我越想越是气恼,索性下令今日不要出战,命令将营盘外面三百步之内全挖成深达丈余的纵横交错的壕沟,让北汉军根本就无法接近营寨,然后在每处营门的位置都留下了一条完好的出路,这样一来,我军就可以出入自如,而敌军可别想随便过来偷袭。

宣松站在我身后,看着热火朝天的“工地”,犹豫地问道:“若是北汉军将出路封住,我们又该如何是好?”

我笑道:“这有什么关系,第一,我军有重骑,若是北汉军愿意用轻骑和我们硬碰,我可是求之不得,第二,我令众军挖壕沟的时候准备了许多木板,万一路途堵死了,只要将木板铺成一条通道即可,而且,我军还有一半步兵,对他们来说,这样的地形可是更加有利。”

宣松这才点头称是,其实这样的法子也不稀奇,只是偏偏大雍和北汉都是以骑兵为主力,又都是求胜心切,喜欢凭勇力取胜,以攻代守,在防守上未免有些懈怠,而且北汉军飘忽不定,连带的大雍军也不能固守一地,而且限制了敌方的骑兵,也不免限制了自己的出击路线,也就想不到这样费心费力地挖掘壕沟。不过对于我这个一心想要防守的人来说,这样子却可以确保安全,再说这次我也不信龙庭飞敢撇下我们去攻打别的地方,这几年齐王精心搭建的防御体系可没有那么多破绽可以利用。而且这样一来,至少不会再有人惊扰我的清梦了,就是真需要拔营,也没有什么要紧,这么多军士,让他们动动筋骨也是好的。。

我们这里忙着,小顺子突然走到我身边,低声道:“公子,远处有人窥营,是一个高手。”

我听了之后,一边转身和宣松等人说笑,一边打了一个手势,传下令去,过不了多时,穿着齐王金甲的乔祖从大帐中走了出来,一边走一边似乎很满意的点头,走到我身边之后,故意和我闲聊了两句,然后我们两人一起回转大帐。进帐之后,我连忙问小顺子道:“是什么人窥营,你可看清楚了么?”

小顺子道:“离得很远,属下没有看清楚,不过来人武功很高,看来是北汉军谍探中的好手。”

我也不为意,几个谍探而已,不过是看看今天我们怎么没有出战罢了,让他们回去却是更好的选择。不过我转念一想,有一个计划却是现在用最合适,不会引起北汉的疑心,便说道:“乔祖,齐王殿下曾许我使用死士营,你去找一个合适的人,武功要高强一些,我要用他做事。”

乔祖早就得到了齐王的指令,自然不会多问,吩咐了几个近卫,不多时,几个近卫带了一个军士进来,我仔细看去,这人也是形貌彪悍,气度沉稳,只可惜却是死士身份。齐王军中的死士营都是犯罪的军士组成,也有一部分本就是充军的囚犯,齐王将他们编入死士营,让他们执行一些九死一生的任务,凡是有立下大功的,就可以免去死罪,甚至可以恢复军职。这些人大多凶狠成性,武功高强,又都是犯了死罪,为了求生,执行起任务来都是十分用心,也只有这样的人才合我用。

我将这个军士打量了半天,才道:“本监军有件事情要你去做,这件事情十分危险,你若是能够成功回来,我就禀明殿下,免去你的死罪,恢复你的军职,你若是身死,也可列入阵亡名册,家人也可得到抚恤。不知道你可有胆量去做么?”

那个军士下拜道:“小人自知身犯死罪,蒙殿下恩典,许以戴罪立功,不敢推搪,但有任务,请大人吩咐。”

我将方才匆匆写好的一封书信递给他,道:“你将这封书信送到庙坡大营荆迟将军手里,他看了信就明白了,记着,信在人在,信亡人亡,听说你曾是江湖人身份,武功在一流之上,可要好好用心办事才是,若是丢了书信,会有什么后果本监军也不必多说。”

那名军士接过书信,他不是蠢人,知道这件事情若是容易,也不会特意从死士营选出自己来,他在营中武功已经可以说是数一数二的了,既然特意选了他,定是九死一生的重要任务。又磕了一个头道:“小人家中只有母亲和幼弟在,还求大人多多照应。”这却是军中传统,若是去执行几乎是必死的任务,都会在行前交待遗言。

我有些不忍地看了他一眼,道:“你放心吧,你的母亲兄弟,自有朝廷赡养。”

见这个军士就要退出帐去,我心中一叹,几乎是用耳语的声音道:“你只要让那封书信落到北汉谍探手中就行了。”我说的声音很低,那个军士已经去远,应该是听不见的,可是我见他身躯顿了一顿,似乎听见了我的说话,却没有回头,反而加快了步伐。

望着他的背影,我对小顺子淡淡道:“这人心性刚强,又是颇为聪明,我这样一说,他定然明白这一去需要牺牲性命才能更好的完成任务,毕竟他若逃生,那封书信的可信度不免差了一些。我这样一说,他定会心中感激,就是本可以逃生,恐怕也会甘心送了性命,我是否心肠太狠,定要迫他去送死呢。”

小顺子微微一笑道:“这不就是死士营存在的意义么,他若是立下大功,公子可以禀明殿下,对他的家人多加抚恤,想必这总比他身负死罪,屈辱而生好得多吧。”

我冷冷一笑,道:“心狠也得继续狠下去了,这人虽然是条汉子,但是我还是担心他会事到临头,贪生怕死,你跟着去看一看,若是他想要偷生,你就送他一程。不过可别露了形迹,凭你的武功,除非是魔宗亲临,想来不会有问题?”

小顺子轻轻点头,道:“公子安危需得当心。”

我失笑道:“这千军万马若是还保不住我的性命,就是你在也没有用了。”

小顺子莞尔一笑道:“那可说不好,若是我做刺客,就是千军万马,也可取得公子的项上人头。”

我不由摸摸脖颈,觉得好像有一股凉气从那里掠过。心知这小子是不忿我说他无用,故意来吓唬我的。

这时,数里之外,鹰目炯炯地望着大雍军营的萧桐心中千回百转,今日探营,他特意亲来,就是因为昨日一战令北汉军众将心中起了疑虑,虽然大雍军仍然是十分坚韧善战,可是怎么却是仿佛变了一个人指挥一样,齐王李显上阵作战的时候往往身先士卒,而且战风彪悍,这次用兵却是颇得“稳”字真谛。心中既有疑问,便要仔细查探,所以萧桐亲任斥候。不过见了大雍军在营寨外挖壕沟的举动,萧桐心中也相信了昨日众人商量过后的猜想,必定是江哲替李显出谋划策,若是李显,绝对不会想出这样的惫赖法子的。而且萧桐打从心里不相信齐王李显敢于放着龙庭飞不管,不在中军指挥。不过从昨日的用兵上看,那江哲虽然不错,但也算不上什么出类拔萃的奇才,行军作战虽然极有条理,但是却丝毫看不出什么奇特之处。这也难怪,那江哲虽然名冬天下,却不过是个谋士,这领军作战未必是他的长处。这样一来,萧桐更是不会相信齐王敢离开军营了。又看了片刻,萧桐正准备撤走。这时,萧桐突然看到从雍军大营的营门出来了单人独骑,向南面急驰而去,萧桐心中一动,这个时候,这个方向,定是齐王传令给后面的辎重大营,谭忌可正对庙坡虎视眈眈,若是得到什么情报,定会有些帮助,就是没有什么帮助,破坏敌人和后方的联络也是一件好事,虽然现在还不便使用大批侦骑,可是魔宗弟子最善江湖搏杀,对付一个信使自然不需费什么心思。想到这里,萧桐放飞了身边的一支黑鹰,那黑鹰一个盘旋,也向南面飞去,带去了截杀的指令。

第二十章    苍鹰折翼(下)

谭忌者,为大将军龙庭飞所重,拔于草莽,亲传兵法战策,由庶民而致将军,殊非易也。其为人,落落寡欢,不与同僚相近,大将军每燕饮众将,以励士气,忌虽勉强从之,然滴酒不沾,一人向隅,而满座不欢,数次后,大将军亦患之,不得已遣之。忌御下甚严,有犯军法者,虽勇士必斩之,故所部精练严整,每战必定不畏牺牲,军威之盛,天下罕见。忌虽位高,然不改旧日简素,不喜馈遗,每有赏赐,皆分赠部下,故虽严刚可畏,部下皆愿效死耳。

忌父母族人皆死于战乱,忌深恨焉,每出战,杀戮必重,屡有杀俘扰民之事,大将军劝止不听,然其用兵颇有法度,雍人畏惧,故大将军亦不能约束之。忌貌文秀,又兼身世凄苦,常有惭意,乃覆以青铜鬼面,终日不解,人皆以“鬼面将军”呼之,随身护卫皆效之,敌我上下,皆畏之。

——《北汉史·谭忌传》

天边苍鹰飞过,旷野青天,荒草漫漫,沁水呜咽,凄凉的鹰唳令人心中顿生人生寂寥之感。谭忌策马站在沁河岸边,目光中满是冷淡冰霜。

几个斥候飞马赶来,拜倒在地,其中一人高声道:“启禀将军,敌军辎重大营建在庙坡,粮草堆积如山,辎重大营的东营跨沁水,西营跨十里河,后营距两河交汇的秋风渡只有三里路,沁水上有四道浮桥,十里河上有三道浮桥,秋风渡共有水军船只千余艘,每次可以运送数日粮草辎重。辎重大营中军打得是荆迟的旗号,共有一万骑兵,两万步兵。”

谭忌没有作声,只是做了一个手势,侍立在他身侧的一个同样戴着青铜面具的侍卫,三十六骑之一,朗声道:“将军命你退下。”

几个斥候同时松了一口气,恭恭敬敬的退了下去。对着谭忌,是很少有人能够坦然自若的。

待他退下之后,谭忌寒声道:“罗蒙,你说,为什么堂堂一个大将,会被放到辎重营里,荆迟在大雍已经算是数一数二的骑兵将领,却被置闲在辎重营,从前齐王掌管军权的时候都没有这样做,换了雍帝的心腹来监军,怎就会有这种事情发生。”

那个侍卫犹豫了一下道:“将军,哪里没有权力纷争,齐王虽然权高,可是这荆迟明显是雍帝派来的钉子,齐王若是将他置闲,岂不是明目张胆和他的皇兄作对,如今既然换了人制约齐王,那么荆迟就不重要了,自然要趁着这个时候对他下手。这世道,有几人会顾念下属是忠是奸,还不是用的时候甘词厚币,不用的时候弃如破履。当年将军遇刺重伤,不就是有人趁机为难将军么?可没见大将军替您出头。”

这侍卫乃是三十六骑中跟随谭忌最久的,自然是心腹之人,所以才敢放肆直言。谭忌听了既不恼怒,也不惊讶,淡淡道:“人情如此,也无话可说,不过大将军待我恩重如山,不许你菲薄。石将军不过是心直口快,看不惯我的手段罢了,却不是存心和我作对,这种话以后不许再说。”

那侍卫连忙应诺,却又问道:“不知将军准备如何攻击敌军大营,荆迟也是我等劲敌,若是稍有不慎,只怕是有败无胜。”

谭忌冷冷一笑,道:“一个鲁莽之人,又是必然心存不满,有何惧哉,我已经有了计策,敌军依靠水运运送辎重,这本是好事,可惜却也给了我可乘之机。且看我手段,让敌军辎重粮草,尽化飞灰,我倒要看看,他们有什么法子继续作战。这也是他们想要大战,否则怎会将辎重大营设在庙坡,这里虽然方便运送,但是防备上却是不如高沟深垒的城池远甚。罗蒙,传我令谕,召集军中校尉,准备作战。”

罗蒙心中一喜,他可是知道将军神机妙算,鲜有落空的时候,这次立下大功,而石英上次却是损兵折将,自己等人就可以洗雪数年来常被石英等人压制的屈辱,虽然将军并不在意,可是那些人的排斥冷淡可都是他看在眼里的,因此罗蒙连忙下去传令,准备随着主将再一次破敌立功。

夜色深沉,雍军辎重大营内灯火通明,中军帐内,坐在主将位置上的却不是荆迟,而是换了普通青甲的齐王,这一次为了避过北汉秘谍的耳目,齐王和他的亲卫军都换了普通士兵的甲胄,更在辎重大营里面藏了两万骑兵,表面上看这里只有两万步兵,一万骑兵,实际上却是两万步兵,三万骑兵。营盘中搭建了帐篷,这些重骑兵藏在帐篷里面,轮流出去露面,因此瞒过了北汉军的眼睛。

坐在下首的荆迟振奋地道:“殿下,我们派出去的斥候都没有即时回来,看来谭忌果然已经来了,先生神机妙算,这次能够生擒谭忌的话,不仅龙庭飞失去左膀右臂,还可以振奋军心,那谭忌肆虐泽州多年,若是将他千刀万剐,也可消解民怨沸腾。

李显笑道:“还不知道能不能生擒活捉呢,听说此人生性严厉刚强,领军作战狡诈如狐,很多冷酷无情的人偏偏自己却是怕死得很,希望这谭忌不要让我失望。”两人正在闲谈,这时,突然营外士兵哗然,不过片刻,有人入帐禀报道:“启禀殿下、荆将军,有人从沁水上游放下火船,将沁水浮桥和两岸的辎重都点燃了。营前有千余北汉军正在攘战。”

李显精神一震,道:“果然来了,荆迟,你依计行事去吧。”

荆迟起身一礼,大踏步走出帐去,大声道:“快拿我的兵器来,我倒要看看什么人敢和老子作对。”

李显微微一笑,对身边的近卫庄峻道:“准备好,我们等到荆将军引走敌军之后再出营。”庄峻面上露出喜色,道:“殿下放心,我们早就准备好了,只等着上阵杀敌,这些日子可是憋闷坏了。”说着转身出帐传令去了。

谭忌远远的看见大雍重骑出了大营,万马奔腾,气势磅礴,不由叹息道:“这样的大将军马,却让他们守辎重,也真是可惜。”复又冷笑道:“我倒要看看平日冲锋陷阵的大将有没有法子固守营寨。”说罢,他一挥手,带着身边近卫向大雍军当头迎去。就在两军距离不到百步的时候,北汉军突然折转方向,避过雍军锋芒,从侧翼逼去,谭忌带着三十六骑冲入了大雍军阵。他手下这支骑兵乃是北汉军中最擅冲刺的劲旅,长戈挥动之中,血肉横飞,而跟在他们身后的骑兵却使用劲弩四面射去,大雍军阵为之动摇。荆迟带了七千铁骑出来,谭忌带了亲军冲杀了一阵,撕破重骑防线,耀武扬威地向远处遁去。荆迟又羞又恼,带着军士抢救辎重,虽然只是波及了岸边的一些营帐,可是也是损失不小。整顿到午后,却是从十里河上漂下火船来,这次雍军早有防备,可是却仍然弄得灰头土脸。荆迟策马站在营门,指天划日,将谭忌骂得体无全肤。这时,谭忌却又带着千余军士前来攘战。

荆迟大怒,带着铁骑就要出营,这时有参军装束的文官前来阻拦,进谏道:“将军,敌军只以一部挑战,分明是诱敌,还请将军谨慎。”

荆迟却是大骂道:“敌军有后援又如何,我们三万人被这几千人戏弄,传了出去,岂不是让人说我们大雍无人,再说我只带骑兵出营追杀,难道两万步兵还守不住大营么?”说罢带着骑兵出营而去。

这次两军初接,大雍军就发挥出了强大的战力,一时之间北汉军损失惨重,谭忌见强弱悬殊,带着亲兵退去,这次荆迟可是不依不饶,在后面舍命急追。谭忌带着亲卫亲自断后,就这样追追逃逃跑出了几十里路。谭忌虽然人少,却是精锐中的精锐,北汉军又是轻骑,稳稳的将荆迟军保持着一箭之地,若是荆迟军追得近了,就用弓弩逼退。荆迟也是精通骑战,索性不缓不急地跟在后面,只要前方北汉军稍有松懈,就要一举破袭敌军。双方这样一追一逃却是僵持住了。

追击了小半个时辰,谭忌已经到了沁水上游岸边,这里北汉军已经架起了数座浮桥,谭忌一声令下,带着众军向沁水西岸撤去。荆迟大怒,下令道:“给我追上去,不能让他们破坏浮桥。”

千余人不过片刻就过了浮桥,对面岸边乃是一座丘陵,眼看着北汉军转向丘陵后面去了。荆迟更是大急,可是一座浮桥对于近万的大雍铁骑来说实在是不够用。心中急了,也顾不上等待,荆迟带着亲军先追去了。转过丘陵,却是衣甲鲜明的七千北汉轻骑。策马奔上丘陵顶部的谭忌一举长戈,号角齐鸣。转瞬间将荆迟和千余亲卫铁骑包围起来,谭忌分兵两处,一半围住荆迟,一半阻截后面的援军,凭着丘陵拐角处的地利,生生挡住了后面的铁骑。

罗蒙兴奋地道:“我本以为荆迟会派先锋先过来探路,想不到他竟然亲自带军,倒让我平白拣了一个大便宜。”

谭忌冷冷道:“小心一些,事若反常必为妖,提防中了圈套的是我们。”

罗蒙笑道:“将军多虑了,必是荆迟不忿被人置闲,大人两次放下火船,他损失不小,将来若是齐王追究起来,他必然是罪责难逃,也难怪他如此气恼,再说荆迟是勇将,可没有听说过他擅长智谋。大将军不就是早就查过了么,他从前虽然战功赫赫,可是却从来冲杀在前,虽然他麾下似乎有个擅长防守的将才,可是这种时候,那人就是一起来了,恐怕也要留下镇守的。”

谭忌漠然道:“不可大意,而且我军虽然放火船烧了几个营帐,可是他们在营帐之间设下了防火之物,实际上损失并没有看上去那么惨重,荆迟几乎带出了所有骑兵,虽然很符合他的作风,可是我总是觉得有些蹊跷。

这时候,荆迟浑身是血,带着亲军居然冲破了北汉军的阻截,而号角高鸣之后,那些被堵截在后的雍军也如同潮水一般退回沁水东岸。谭忌不由皱眉道:“也难怪荆迟如此鲁莽,却原来战力如此,好了,我们去追荆迟,他现在孤军在外,一定要趁机除了他。”说罢,谭忌命人摧毁浮桥,断绝东岸大雍援军从后追袭的可能,然后向荆迟追去。

追了百里之遥,谭忌在斥候的指引下已经把握了荆迟逃亡的方向,却是准备迂回返回辎重大营。谭忌心中也不免生出争胜的意念,若是能够擒杀荆迟,这可是不小的功劳。而且追击了半日,经过斥候的报告,那些大雍援军早已成了无头苍蝇,根本无法对荆迟加以援手。谭忌大喜之下,更是紧追不舍。他对沁水西岸的地形早已经十分熟悉。在他不断的分兵阻截下,渐渐将荆迟围困在一个狭小的区域。不过谭忌皱了皱眉,这里离沁水东岸的辎重大营只有十里多路,虽然浮桥已毁,想要运送士兵过桥,没有半天是办不到的。不过谭忌还是担心会有意外,可是想要擒杀荆迟的想法却是越来越有可能实现,谭忌不由苦笑道:“这样的饵,就是有毒,我也舍不得放弃。”又仔细想了想,大雍诸将,比荆迟强的已经不多,若是大雍会将两个大将放到后方,那么自己就是落入陷阱也认了。决心既然下了,谭忌便下令集中全力,围歼荆迟。

伸手抹了一把脸上的血汗,荆迟苦恼地看着身边只剩几百人的亲军,心道,若是齐王想要借刀杀人,恐怕就会成功了。到了这个时候,还看不到援军,荆迟都有些怀疑齐王了,转念一想,就是齐王有心,也不会损害大局。又一马当先冲向前面拦截的北汉军,口中大声呼喝,鼓舞着亲军的士气。

谭忌站在高处,看着重重围困中挣扎的雍军,心中生出快意的感觉,大丈夫在世,若是不能快意杀伐,那么活着还有什么乐趣呢。

这时,谭忌眼角突然看到辎重大营方向烟尘滚滚,不由心中一动,距离太近,若是派斥候前去,只怕还来不及回报就被敌军击杀了,连忙命人驱使鹰隼去查看敌情。过了片刻,烟尘越发接近,谭忌不见苍鹰回报,而那烟尘凝而不散,想也知道是敌军援军到来,谭忌心中一惊,敌军这样快就渡河,除非是早有准备,荆迟出战之后就开始搭桥渡河,看来自己还是中了圈套,荆迟果有后援。不过谭忌很快就冷静下来,心道,敌军转瞬即到,荆迟还有数百勇士相随,气势不减,自己若是还想擒杀荆迟,必定会被敌军所乘,倒不如结成锋矢阵,舍命而战,若能击溃敌军的中军,就可以安然而去,压下敌军的气焰,就是不能杀死敌方主将,冲击敌军的中军,也可以让敌军促不及防,突围的机会就更多些,虽然危险,可是只有这样,才可能有一线生机。想到就做,谭忌立刻下令整军。那些北汉军虽然不明白为何眼看着敌军岌岌可危,主将却下令撤围,但是谭忌一向军令森严,他们也不敢迟延,片刻就排成了锋矢阵。阵形刚刚摆好,震耳欲聋的马蹄声就已经清晰可见,烟尘滚滚中,赤色衣甲的大雍铁骑人如虎马如龙,簇拥着一面金龙王旗,两翼伸张,隐隐有将北汉军合围之势。却是齐王命令部下都换回了自己的衣甲,来完成这最后一击了。

到了近前,铁骑也不稍歇,铺天盖地的向北汉军阵冲去。谭忌高呼道:“生死存亡,在此一举,随我来。”说罢当先向大雍中军冲去。他本是聪明人,一见王旗,就知道万万想不到的事情发生了,齐王竟然不在主力大军之中坐镇,那么这里绝对是一个陷阱,虽然不明白为什么齐王会舍本逐末,来对付自己这支偏师,可是谭忌知道,若不死战,那是别想生离此地了。

李显看着一身鲜血狼藉的荆迟,不由歉疚地道:“都怪本王不好,若不是想将谭忌麾下精兵一起留下,也不会让荆将军身入重围了。”

荆迟有气无力地瘫倒在马上,半晌才道:“殿下别忘了将皇上赏赐的那瓶御酒赏给末将就成了。”

李显失笑,荆迟也不由笑了起来,两人之间种种隔阂都在这一笑之间化为乌有。

这时候,荆迟看见齐王身后,一个穿着普通青甲,外罩白色战袍的青年相貌有些陌生,那人左肩侧挂一张银弓,相貌英俊,神态冷傲,眼神如电,却是十分威武出色,不由问道:“殿下,这位是哪位将军?”

李显笑道:“这是本王府上的客卿端木秋,金弓长孙,娥眉青衫,银弓端木,红妆罗刹,他就是银弓端木,前几天刚从京中来见本王,本王想到北汉的鹰隼十分讨厌,所以就让他留下了,方才就是他射杀了那两只黑鹰。端木虽然军略上并不擅长,可是若论箭术,可是不在长孙冀之下。”

荆迟和端木秋见了一礼,心道,这样的人物不从军真是可惜了。这时,谭忌带着三十六骑居然冲破了重重阻截,眼看着就要冲到中军了。荆迟心中一紧,道:“殿下,下令两翼前来救援吧。”

李显摇头道:“我们人虽然多些,可是敌军骁勇,若是放松围困,给他趁机冲出去,那可就是前功尽弃,再说。本王的亲卫军,难道比不上北汉的骑兵么?”最后两句,他却是高声说出,听到的齐王亲卫,都是心中羞恼,更是舍了性命作战,一时之间,就是最善冲刺的三十六骑也几乎是寸步难行了。

谭忌见到这种情况,仿佛又回到了当日眼看着父母亲族被人屠戮,自己却只能藏在岩石后面眼睁睁的看着的处境,那种屈辱和恨不得立刻死去的心痛让他不能自已。他高声呼道:“众君,我等和大雍结下血仇无数,若是被敌人俘虏,就是千刀万剐也不能偿罪,不若拼个一死,也免得落入敌手,受尽羞辱。”言罢,也不闪避对面刺过来的马槊,一伸手紧紧将那条马槊夹在腋下,一戈将那个大雍军士头颅削去,然后伸手将那人提到自己马上,将长戈挂在马上,然后双手将那人尸身高高举起,喝道:“有敌无我,死战求生。”然后双手用力,将那具尸身生生撕成两片,鲜血五脏溅落,将谭忌身上染成血红。雍军大哗,北汉军却是心中凶残之性尽皆激发出来,跟在谭忌后面,冲破了面前的阻碍,切入了中军。

荆迟心中一紧,连忙握紧马槊,却觉得手足无力,这时,齐王却已经长笑一声,策马迎上,左右近卫连忙随着冲上,想将齐王保护起来。可是齐王马快,却已经迎上了北汉军的锋矢阵之首——谭忌。

谭忌原本正在冲杀的顺畅,却觉得突然被人架住了长戈,抬眼一看,那人一身金甲,火色战袍,除了齐王不会是别人。想到若是杀死此人,敌军必然大乱,谭忌不由精神一震,连出杀招,而他身边的鬼骑也围了上来,一定要舍命拼下敌军的主将。可是齐王李显也是练武多年,既有名师教导,又是多次上阵,论武艺也不输谭忌,而且他身边勇士极多,齐王这一杀出,他们也跟了上来,双方一番血战,谭忌的攻势还是被暂时遏制了,若是往常作战也无关紧要,可是现在北汉军落入重围,结果就不同了,趁着锋矢阵暂时被阻挠的机会,其他雍军加强了攻势,北汉军两翼和后面的阵形渐渐散乱,不过片刻,就有蜂拥而上的雍军铁骑接替了齐王的位置,将北汉军彻底包围了起来。

退到大旗之下的李显深深的呼吸了一口冰冷的空气,这么多年上阵杀敌,虽然由于他的王爷身份,直面危险的局面并不是特别多,可是也不是没有在生死边缘徘徊过,可是方才谭忌和他麾下的鬼骑猛攻他的那一刻,李显还是真切的感觉到了什么是生死须臾。感激地看看荆迟,方才荆迟没有急着扑上来救人,而是迅速下令加强了攻势,让李显有机会退了下来。看看困兽犹斗的谭忌等人,李显心中不但生不出怒意,反而添了几分赏识,这些年来不是没有见识过猛将勇将,可是像谭忌这样有勇有谋的将领却是不多见,若不是北汉军一开始就走错了一步,也不会有机会将此人困住。又过了片刻,荆迟麾下那些骑兵也终于及时赶来,他们加入战场,终于确定了大雍的胜利,虽然北汉军已经结成圆阵固守,但是没有援军,败亡已经是迟早的事情,大局已定。

厮杀了半天,天色已经渐渐昏暗,李显担心谭忌趁夜突围,又调来了步兵,在四下点燃火把,将战场照得通明,北汉军已经只剩下寥寥的三千人,李显更是控制了进攻的节奏,不愿意破坏了全歼敌军的战机。北汉军残军摆了固守的圆阵,而大雍军也在外面摆了一个圆阵,满满的消磨着北汉军的生命。围困的战圈越来越小,李显更是命令雍军轮流上阵,北汉军不得休息,越发疲惫,只要圆阵一破,就是全军覆灭之时。可是在谭忌的指挥下,这支北汉军居然还未丧失战力。

立在阵心,谭忌嘴唇干裂,身边的鬼骑也只剩下十七人,自从他领军以来,还没有过这样的惨败。可是丛他的眼中却看不到失意和忧惧,只是如同往常一样的冰冷漠然。这些北汉军本就是骁勇成性,虽然濒临绝境,可是他们和大雍都有深仇血恨,虽然说阵上交锋,死而无怨,可是他们却是不同,死在他们手上的大雍平民数不胜数,历来谭忌麾下的军士落到雍军手中,几乎只有死路一条。可是如今他们心中却生不出对谭忌的怨恨,虽然是这人主导了对那些让他们绝无生路的屠杀,可是这些军士也明白,只有在谭忌麾下,他们才有可能在短短几年积攒下足够的金银,虽然他们丧命疆场,可是他们的家人早就有足够的金银可以过活。为了自己的家人,只有死战到底,只要北汉最后得以保全,自己的家人就会平安,这样的信念让他们虽然已经陷入必死绝境,却丝毫没有委屈求生的念头。

李显看得心中敬佩,道:“这样一支铁军,至今仍然不肯屈服,真是难得,就是我大雍也罕见这样的骑兵,荆迟,你说本王招降如何?”

荆迟犹豫了一下,道:“谭忌深为大雍军民所恨,只怕招降不宜。”

李显想了一想道:“我也知道一些事情,你也不用忌讳,这谭忌和大雍确实仇深似海。不说他父母亲族之死,就是这些年来他在泽州镇州杀人如麻,也是血债累累,不过本王实在爱惜他的人才,若是他肯归降,最多我将他调到南边去也就是了。”

说到这里,李显提高了声音,高声道:“谭忌,你已经身陷死境,若是肯归降,本王保证不伤你的性命,就是你的部下也可以一并饶过。本王言出如山,你可肯考虑一下?”

他的声音中蕴含了内力,虽然战场十分纷乱,众人却都听得清清楚楚,雍军也在将领们的示意下暂时放缓了攻势。

谭忌听得清清楚楚,他身边的近卫都听到青铜面具后面传来嘶哑的笑声,不多时,他高声道:“谭忌身为北汉将军,深受龙大将军厚恩,今日虽然落败,却是唯死而已,王爷不必费心,谭忌早已立誓,绝不会再受人屈辱。”

李显高声道:“你纵然不惜性命,难道你麾下将士的性命也不顾惜么?”

谭忌听了又是一笑,知道李显趁机打击北汉军的军心,想不到这齐王果然谨慎,都到了这种时候,还不忘打击敌军军心,他缓缓看看四周,笑道:“你们都是北汉之民,若有想要投降者,不妨说出来,本将军不阻拦你们求生就是。”众人听了都知道他并非想要骗出心志不稳的人杀之灭口,这是谭忌从来不屑去做的事情。过了片刻,众人齐声道:“愿随将军而死。”

谭忌叹了口气,目光落到一个个子最矮的鬼骑身上,道:“凌端,你今年只有十七岁,你的两个哥哥都曾是我的鬼骑,可惜却都死在战场上,半年前若非你武功确实出色,又是苦苦相求,我也不忍将你选入鬼骑,若是你想投降,我也不会怪你。” 那个鬼骑连忙跳下马跪倒在地,取下青铜面具,露出一张稚气犹存的英俊面孔,泣道:“将军何出此言,我们兄弟自幼无父无母,流落无依,若非将军传授武艺,如今还是人人得以欺凌的乞丐。端情愿和将军同死,请将军不要再说这样的话。”

谭忌听得只觉心中一暖,自从父母亲人亡故之后就已经冷若冰雪的心也觉得有些暖意,他淡淡道:“你起来吧,我不赶你就是。”见那个少年抹去眼泪,戴上面具,跳上战马。

谭忌仰面向天,拊掌而歌道:“天不仁兮生离乱,地不仁兮起狼烟;亲族父母兮化尘土,志摧心折兮可奈何;怨虽报兮恨不息,君恩重兮死亦难;杀人盈野兮吾且不悔,流血飘橹兮生灵涂炭;君执弩兮吾持戈,吾驱骑兮君相从;沁水寒兮葬吾躯,赴黄泉兮心意平;生死无惧兮慨而慷,逢彼旧人兮吾心伤!”

众军初时只是以声相合,后来便也跟着高歌起来,苍劲悲怆的歌声在天地间回荡盘旋,北汉军中杀气升腾,人人面上都是视死如归的神情。

见此情景,李显也不需再问,只是叹了一口气,传令道:“绝杀。”对于值得尊重的战士,本就只有让他们荣耀战死才能表达心中的敬意。

大雍骑兵在火光掩映下向北汉军逼去,这时候天上的乌云散尽,明月疏星无情地映照着残酷的战场。注视着北汉军最后的争斗。

\chapter{第二十一章 间其腹心}

忌纵横疆场多年,胜多负少,每独当一面,素为大将军所重。荣盛二十三年,大将军率众入泽州,与雍军主力战于秦泽,遣忌袭敌军辎重。不意雍军诡谋,齐王乔装离中军,设虎穴以待。忌不察,身陷重围。苦战一昼夜,弓矢尽,粮草绝,终以敌势过强,星陨沁水,三军皆从死,无一降者。时雍军主将齐王李显虽恶其多杀戮,仍惜其才,以使者劝降,忌拒之,高歌而绝,终年三十一岁。王亦叹息,不许戮尸,遣亲军送还北汉。大将军见之,痛彻肝胆,从其前言,擎其骨灰归葬故里。

——《北汉史·谭忌传》

第二天天亮,北汉军终于死伤殆尽,李显在侍卫保护下走入那片满是血腥的修罗场,战场上处处伏尸,每个死去的北汉军都是身背数处重伤,无一不是激战而亡。走到战场中心,那里正是战局最惨烈的地方,好几具尸体都戴着青铜面具,而在其中就有一个身穿将军服饰。李显仔细看去,只见那人张开双手,用身躯掩着一个较矮的身躯,右手仍然紧紧握着长戈,战袍破碎,尽是鲜血,在他身边,一匹背上仍然插着长矛的战马长声悲鸣,不时用力低下马首去推自己的主人,想要让他重新站起来。

也不需李显下令,自有人拖走那匹重伤将死,却仍然徘徊不去的战马,李显走上前去,俯身看去,只见那人的面上仍然覆着青铜面具,便伸手摘了下去。面具摘下,露出一张清秀的面容,虽然已经是而立之年,却是仍然俊秀斯文,常年不见日光,让他的肤色有些过于苍白,可是即使是闭上眼睛,仍然能够让人感觉到他浑身上下流露出来的悲凉气息。或者是有面具遮挡的缘故,虽然经过苦战,可是那人面上并无血迹,眉宇间甚至没有一丝濒临死亡的惊惧和愤怒,反而带着淡淡的笑容,仿佛走过长途的旅客终于放下了身上的重担一般,有一种如释重负的感觉。

李显轻轻一叹,方才招降,或者他也有扰乱敌军军心的用意,可是那一刻他是真的很想将此人收到麾下。此人虽然杀戮过重,可是军略勇气却是让人心折,只见他濒临绝境,他的部属却都甘心随他而死,就知此人虽然冷酷无情,但却不是天性暴戾之人,只是可惜了这样的人才。

李显正在惋惜,突然耳边传来低微的呻吟声,李显还没有反映过来,身躯已经自动地退了一步,而旁边的侍卫也都仗剑过来,谨慎的护着齐王。众人仔细听了一会儿,却再也没有声音,李显回忆了一下方才听到呻吟声的方向,目光落到谭忌身上,不,应该说谭忌身下护着的那个人。他令人将谭忌抬到一边,发现被谭忌压在身下的也是一个鬼骑,只是李显发觉那人虽然受了重伤,可是致命处的伤口却是很浅,想必是被谭忌以血肉之躯挡住了。

齐王身边的近卫陶林冷冷瞪了事先清理战场的人一眼,竟没有发现还有活人,若是有人趁机行刺岂不是糟糕。不过李显却是没有怪责,他上前摘下那昏迷不醒的鬼骑的面具,露出一张稚气犹存的面容,不由道:“想不到谭忌身边的鬼骑中竟有这样年少之人,小小年纪就上阵杀敌,还要担当冲阵之责,可真是不简单,来人,将他送到军医那里,给他好好治伤。”

众人面面相觑,和北汉征战多年,可以说仇恨似海,虽然雍军有着不杀俘的习惯,可是若在战场上看到敌军幸存的重伤者,多半都是一刀杀了,最多也就是弃置不理,怎还会给对方救治。李显微微一笑,他明白麾下将士心中的迷惑,可是想起临别之时那人板着面孔教训自己的模样,心中不由暗笑,朗声道:“从前我们和北汉仇深似海,自然是有冤报冤,有仇报仇,可是人谁无父母家人,杀其一人,却是一家皆哭。你们记着,皇上要得是天下一统,四海升平,他们今日是北汉的子民,将来就是大雍的子民,虽然沙场之上刀枪无情,死亦无恨,可是若是见死不救,岂不是等于残害自己的子民,本王在此传下军令,从今之后,擅自杀俘者处以死罪。”

众军轰然应诺,虽然有些人并不明白齐王的用意,可是军法如山的道理却是人人懂得的。这时一个部将出列道:“元帅,虽然如此,可是这个谭忌肆虐泽州多年,双手沾满大雍百姓的鲜血,我们多少袍泽都死在他手上,还请元帅准许末将等人将此人千刀万剐,才能消了心头之恨。”

李显正想应诺,但是目光落到谭忌的尸身上,看到他那平静的仿佛睡去的面容,叹息道:“我们大雍勇士快意恩仇,可是人死恨消,何必要和一个死人过不去呢?而且此人虽然对我大雍有害,却是北汉的忠臣,又是这样视死如归,本王也是心中敬慕,戮尸之举不是我们大雍王师应该做的事情。庄峻,你命人用棺木将谭将军装殓起来,等到战后送回北汉去吧。”

那将领面色有些羞惭,退了下去。李显看了他一眼,又高声道:“谭忌已经战死,不论什么大罪,一死也足够抵偿了。你们听着,我们也应该去会会那赖在泽州不走的龙大将军了,记恨一个死人也没有什么光彩,若是能够擒杀龙庭飞,才是我大雍男儿最大的荣耀。你们说是不是。”

众将听了,都是高声呼喝道:“杀龙庭飞,破北汉军。”初时只是众将高呼,后来四下军士也都是高声呼喝,方才因为齐王的军令而有些心中不满的将士再也没有半点怨言,是啊,戮尸或者残杀俘虏,这种事情怎是我们做的,自然是要将敌军主将一举擒杀,才能消去心中块垒啊。

李显见气势已经被自己挑了起来,又道:“传我将令,修整一日,明日我们去秦泽,看看龙大将军的威风。”这次众将都是欢声应诺,仿佛恨不得立刻上路似的。李显却是心中有些忧虑,不知道秦泽那里的战事如何了。

十一月七日夜,秦泽北汉大营中军帅帐,昏黄的灯光下,龙庭飞傲岸的身影被灯光映射得很长,他的目光一直没有离开帅案上那封书信,这是萧桐派出北汉谍探高手从一个大雍的秘密信使身上搜出来的。那个信使武功高强,性情坚韧,和北汉谍探在追逐了百里之后,身陷重围,却仍是死也不肯归降,临死之前还要毁去信件,却被魔宗高手夺去。这样一封信,必然是十分机密的事情,可是龙庭飞却宁愿这封信只是一个骗局,因为这封信虽然言词模糊,却是透着一种令龙庭飞不愿置信的信息。再次拿起信笺,龙庭飞用心看去。

“渠辈有信至,其意多有敷衍,言未随军,多有碍难,或者仍然意存观望,其为敌军主将腹心,若能动之,则北汉军必败也,故此战胜负事关要紧,若彼胜,恐再无可间之隙,若我胜,其必弃暗投明,此战之胜机不在秦泽,而在辎重粮道也,重任在肩,愿君勉之。”

这封书信既无抬头,也无落款,只是盖了一个私章,上面是寒园居士的字样,可是从口气上来看,那是雍军数一数二的人物所写,见这封信文字秀逸,龙庭飞心中隐隐觉得恐怕就是自己如今的对手,江哲亲书,而且听说江哲在雍帝潜邸的时候,就是居住在寒园之中,龙庭飞曾经见过那段时期江哲的一些诗文,确实曾经自称寒园居士。可是接信的人真是荆迟么,虽然这封信只是说明守护庙坡辎重大营的重要性,并隐隐说明有一个自己十分信任的部将起了叛意,只是还不坚定,要等这一战结束之后才会有决定。

龙庭飞不是没有疑心这是离间之策,虽然说江哲写信给被他有份贬斥的荆迟,稳定他的心志,也是理所当然的事情,可是这种事情若是江哲做来,怎不令龙庭飞心中怀疑这是阴谋离间呢?

所以当初第一眼看到这封书信,龙庭飞并未深信,只是暂时记在心里,不论如何,对这一战应该是没有影响的。可是这几日两军多次交战,虽然双方都无意决战,可是龙庭飞还是通过重重迹象看出了自己面对的不是过去的敌人,齐王的作战风格是炽烈而积极的,如同火焰一般无坚不摧,而自己如今的对手初时还有些窒碍,可是如今他的作战已经如同流水一般坚韧多变,水性至柔,然刚强莫之能胜,虽然龙庭飞和麾下众将都以为是江哲指挥。可是过了几日,龙庭飞心中却是疑心渐起,无论如何,江哲都是一个没有实际指挥过作战的文士,难道齐王会真的将指挥大权全部交给他?可是龙庭飞心中又是绝不相信齐王会不再军中,对着自己,难道还有主将敢擅离中军么?越想越是烦恼,龙庭飞终于下了决心,明日一定要揭开这个谜底,除非是齐王亲自领军上战,否则无论如何不能这样打下去了。

这时,同样的灯火昏黄,就在大雍中军帐内,宣松一边和众将商议军务,一边用眼睛余光去看坐在左侧上首的监军大人,只见江哲正倚在椅子上假寐,虽然他的姿态并没有什么明显的变化,一派好像正在沉思的模样,可是他很有技巧地将面孔躲在灯光照射不到的暗处,好不让众人看见他微阖的双目。宣松心中一阵感动和钦佩,这些日子以来,独自面对北汉名将的压力几乎都要让自己喘不过气来,可是这个总是懒懒散散的监军大人奇怪的却是总能让他觉得安稳,而且他也没有闲着,初时是替他压制不服的将领,后来总在私下提出军事上的建议,让自己在这短短数日之内,将过去所学融会贯通,如今他是真的有信心面对任何敌人了。而众将也渐渐对自己开始心悦诚服,可是若没有监军大人,这些可能会是他永远达不到的目标。

站在江哲身后的李顺看到了宣松的目光,微微一笑,轻轻用传音道:“公子不要睡了,军议就要散了。”言罢将一道真气送入江哲体内,过了一会儿,江哲缓缓醒来了,没有丝毫破绽地换了一个姿势,好像是听得累了,活动一下身躯一般。

我懒洋洋地看看众人,现在宣松已经可以完全指挥众将了,我对军议也就不大留心了,可是不出席又不好,毕竟宣松身份还差些,摸了摸茶杯,却是冷的,小顺子乖巧地给我换上热茶,我又活动了一下有些僵硬的手臂,心想,军议应该结束了吧。

这时候,突然外面传来低声压抑地兴奋呼声,不多时,乔祖高高兴兴地冲了进来,道:“启禀监军大人、宣参军,殿下有捷报传来,谭忌部已经被全歼,殿下已经回军,后日午时就会到达大营。”

帐内众将都是喜形于色,纷纷交头接耳,我也是喜上眉梢,我的第一步已经完满达成,站起身来,我笑道:“太好了,殿下那边已经取胜,这边也该收尾了,宣参军,我想北汉军可能数日之内才会得到战报,可是不论如何,今日我看龙庭飞用兵有些古怪,恐怕已经生疑,宣参军明日你也不用掩饰了,堂堂正正打出你的旗号,让北汉军知道大雍多有良将可以和龙庭飞抗衡,这样一来,北汉军必然士气颓废。龙庭飞为了调动士气,洗雪耻辱,必定大战一场,这一战只要你不败,对北汉军的打击就足够了,宣参军,明日就看你的了。”说罢,我向宣松做了一揖。众将也都起身,高声道:“末将等谨遵参军将令!”宣松心中激动万分,不过他毕竟非是常人,不过片刻就冷静下来,道:“多谢监军大人厚爱,诸位将军支持,明日,就让我们给北汉军一点颜色看看,让他们知道我大雍军的厉害。”众将轰然应诺,都是满面喜色。

翌日,龙庭飞望着大雍军的主将旗号,心中如同翻江倒海,虽然已经有了怀疑,但是见到这个情景仍然是心中惊怒非常。主将旗号换了一个“宣”字,除此之外,也再也看不见齐王近卫所在,这令龙庭飞立刻明白这几日和自己作战的根本不是齐王,那么齐王会在哪里呢,他可不信齐王会绕过自己去攻打沁州,至今自己和后方的联络并没有断绝。那么齐王只有可能在庙坡的辎重大营,为什么一个辎重大营在有荆迟这样的大将镇守之后,还要齐王亲自坐镇,除非是设网以待飞鸟自投,想到这里,龙庭飞心中一紧,若是如此,那么谭忌——

他高声道:“萧桐,你速派信使去庙坡,若是谭忌还没有进圈套,那么就让他撤回来,记得派你手下最高明的斥候前去,让他们带上信鹰,或许能够更容易找到谭忌。”

萧桐忧心忡忡地道:“属下遵命,只是将军,若真的敌军设下的诡谋,恐怕谭将军凶多吉少,而且谭将军用兵神出鬼没,行踪飘浮不定,除非是属下亲自前去,只怕很难找到谭将军。”龙庭飞黯然道:“我也知道,可是如今也只能尽人事,听天命,我身边需你掌管军情查探,所以你不能亲自去。唉,你也不用过于担心,谭忌很机敏,或者不会上当。”虽然这样说着,可是龙庭飞心中明白,这不过是安慰自己罢了,心中突然感觉到强烈的痛楚,龙庭飞皱紧了眉头,他真的很遗憾,这一刻他才发觉过去他对谭忌未免太过寡情了。

抬起头,透过重重的战阵和前方正在交战的混乱战场,龙庭飞隐隐能够看到敌军中军旗下,那正在指挥的青衣儒将挥斥方遒,而在他身边,一个青袍书生正在悠闲地望着战场。就是这两个人,将自己拖在了秦泽,而让自己的大将陷入罗网。忽然龙庭飞想到了那封言辞含糊的书信。

原本他还有些奇怪,那封书信语气含糊,有些像是安慰劝告,却又像是通报军情,龙庭飞本还有些疑心,若是此信真是江哲所写,似乎有些不合情理,江哲并没有必要一定在这个时候写这封信,毕竟荆迟也是大将,应不至于因公害私。只是虽有些疑问,但是有些事情总是宁可信其有的。如今已经清楚齐王很有可能就在庙坡,那么这封信就可以说得通了,若是齐王和副将荆迟都身在庙坡,必定不能放心秦泽这面的战局,江哲会写信给齐王通报军情,也就可以说通了。至于言辞模糊则根本是为了避免途中失信的可能,若是此信落入我方之手,也不会因此发觉齐王不在秦泽。而信上说及北汉内部有人想要叛变,则是真假未定,或者是真有其事,但是那叛徒心有犹疑,就是丢了此信,也不过是让我们心中警惕,而且可能还会让那人因为惊惶和压力而更快的屈服。当然也有可能是假的,不过那大雍信使拼命反抗,完全是假的可能性不会太大。萧桐不是说过这些日子,秦泽大营还有数个信使去庙坡么,虽然因为担心损失我军斥候而没有继续下令拦截,可是这也从侧面说明这封信确实是给齐王的。想到这里,龙庭飞心中一股怒火上涌,他绝对不能容忍有人背叛北汉,抬头看看远处的大雍中军,他更加不能容忍有人将自己如此戏弄。连连发下军令,既然齐王不在军中,那么他就要让雍军付出血的代价。脸上浮现出冷酷地杀机,若是能够让大雍在秦泽的主力遭受到惨重的损失,那么就是谭忌那边让齐王得了手,大雍也是得不偿失。

这已经是北汉第四次发起强攻了,我无奈地看着伏尸遍野的战场,心中哀叹,我是不是忽略了龙庭飞的决心,看来他是准备付出惨重的代价,也要取得大胜了,若是在这里的雍军主力惨败,那么我精心筹划的削弱龙庭飞羽翼的计划虽然成功了,却也失败了。若是龙庭飞大败雍军,这样一来,他的自信心必然高涨,不说我们兵力上的损失,只是惨败的事实就可以让北汉上下军民士气高涨了。

看看越发冷静,指挥若定的宣松,我松了一口气,或许他的指挥尚有些缺点,不过至少凭着将近两倍的兵力,至少可以打个平手吧。前些日子龙庭飞也是心存拖延,所以说用兵并不猛烈,这对宣松倒是一件好事,北汉军就如一块磨刀石一样,将宣松从一把利刃磨砺成了神兵,如今正是检验效果的时候了。若有选择,我也不会提前泄露齐王不在的秘密。可是这也是无可奈何的事情,只有通过这样的一战,龙庭飞无功而返,才能有效地打击他的信心,若是齐王带着大军在此,只怕龙庭飞绝对不会在秦泽决战。这次迎战北汉军,我可是打着一举三得的主意的,擒杀谭忌,折其羽翼,一封密信,间其腹心,再用宣松打击龙庭飞的信心。这些已经够他消受,更何况还有更多的后着等他龙庭飞消受呢。不过,我再次叹了口气,无论如何也要挨过这一战才行。

龙庭飞冷冷的看着前方的战场,已经六个时辰了,大雍军的阵线虽然有些软弱,可是始终没有崩溃的迹象,想不到这个宣松不过是个不知名的参军,居然有如此才能,大雍当真是英杰辈出。不过不能这样拖下去了,龙庭飞下定了决心,轻轻抚摸了一下百炼精钢打造的黑亮长戟,戟身上刻着细密的纹理,因为常年鲜血和汗水的浸润,使得那长戟黑色中透着暗红,唯有戟头利刃和长戟颈部的小枝以及其上的月牙弯刃仍然是雪亮晶莹。望着多年来相依相伴的兵刃,龙庭飞心中豪情顿起,纵声大笑道:“我北汉儿郎,个个都是英雄好汉,岂能被雍人所辱,众军随我去厮杀一场,让那些雍人看看我们的本事。”说罢一马当先,冲向两军混战之处。鬃毛如赤焰的神驹,在风中猎猎飞舞的火色战袍,以及那黑红的长戟,使得龙庭飞气势熊熊,彷佛无敌战神一般令人心悸神摇。

我几乎是屏着呼吸看着龙庭飞冲入军阵的,那如同烈焰燎原一般的气魄,纵横捭阖当者披靡的声威,让我也不由心中凛然。明明不过是数千近卫而已,但是那种强大的不可战胜气势却让战场上所有人都不由在这支军队面前有些退缩。眼看着大雍军阵被龙庭飞视若无物,我心中虽然有些苦恼,可是却是更加振奋,这样的龙庭飞才是迫得大雍数年来无法占据北汉寸土的无双名将啊。这一刻,彷佛整个战场只有那红色烈火在燃烧,在膨胀,而北汉军也似乎被主将的勇猛鼓舞,他们的攻势也变得如火如荼,整个北汉军仿佛都在燃烧。

这时宣松迅速的调动军马,采用了严守的策略,我心知宣松的长处不在进攻,所以他扬长避短,想用防守撑过北汉军的猛攻,毕竟刚不可久,只要撑到北汉军气势颓废,就可以趁机反攻了。这样的想法不错,可是如今的大雍军对宣松还没有彻底信服,在这样紧急的关头,不免有些迟疑,这样一来,整个军阵变得有些混乱,在龙庭飞的纵横杀伐之下,大雍军阵,一时之间,大雍军陷入了困境当中,若是再没有转机,只怕军阵即将崩溃。

宣松头上已经冷汗涟涟,他看向我,眼中露出迷茫和恳求的神色,我知道他希望我能够助他一臂之力,甚至希望我能够接过指挥权。我轻轻皱眉,这个时候我若是插手宣松的指挥,必然重重的打击宣松的信心,那样即使取胜也是得不偿失,我需要的是一个可以独当一面的大将,可是我若不插手,所谓兵败如山倒,虽然我军强大,可是恐怕也不能抵挡北汉军秋风扫落叶一般的攻击啊。

看了看有些混乱的占战局,我心中明白其实宣松的指挥并没有什么错误,不过是大雍将士对他仍有怀疑,仍然龙庭飞积威之下,众军不免有些忌惮,只要能够鼓舞士气,那么宣松一定可以稳住局面的。目光一闪,我看到了一边的战鼓,不由计上心来,回过头对小顺子说,你用内力助我,我要亲自擂鼓助威。

小顺子微微蹙眉,道:“不可太久,我的内力阴寒,并不适合助你。”

我笑道:“无妨,不会太久的。”

说罢我翻身下马,走到军鼓面前,挥手让那个负责击鼓的军士退下,拿起鼓槌,站在军鼓之前,小顺子站在我身后,右掌按在我的背心,我只觉得一股冰凉的气息透入我的体内,仿佛浑身热血都被这气息搅得翻腾起来,四肢百骸也是充满了力量。举起右手的鼓槌,我敲下了第一个鼓点。

正在混乱中的雍军突然耳边响起一声平地惊雷,都觉得心中一震,然后天地间响起了低沉而悠远军鼓声,那浑厚而沉着的鼓声绵密而流畅,如同缓缓流动的江水一般,那江心的巨石虽然壁立千仞,却也挡不住江流的前进,那破浪轻舟虽然可以纵横大江,却是不能摆脱江水的束缚。在这平稳的军鼓声中,雍军渐渐的冷静下来,阵势的变换也有了法度。

这时候,北汉军中响起了高亢的号角声,原本似乎有些被流水迟滞的北汉军又有了活力,开始了另外一轮猛攻,可是那军鼓声却也变得隐忍低沉,但也越发坚忍不拔,始终让每一个战场上的战士都听得清清楚楚。鼓声和号角声纠缠在一起,就像大雍军和北汉军的苦苦缠斗。那号角声越是高亢锐利的犹如烈日寒风,听到那鼓声,人人却都觉得仿佛看见了苦苦挣扎在寒风和烈火中的野草,无论如何艰苦,也不能阻止它们破土而出。

高亢的号角声和低沉的鼓声突然都变得微弱下去,但是天地间却充满了一触即发的杀气。突然,仿佛平地风雷一般,鼓声和号角声几乎同时响起,宛若东海潮涌,一浪高似一浪,一浪快似一浪,与此同时,龙庭飞和宣松几乎同时下令,两军混战在一起,血肉横飞,两支世间最强大的骑兵冲撞,厮杀,带着不与对方共存的决心展开了死战。

这时,那号角声直入云霄,越来越高亢,终于仿佛被拦腰折断一般没有了踪迹,而那脱离了重压的鼓声也有些慢了,却不停息,一声声震得人魂魄动摇,所有人都拼尽了全力厮杀,原野上绽开了无数的血花。夜幕渐渐降临,原野上两军开始点燃了火把,在深夜里面继续苦战,谁也没有后退。

而那战鼓声就如同来时一般突然,不知何时离开了血腥的战场,两军陷入了拉锯战似的苦战当中。

火焰明灭当中,宣松十分自信地指挥着雍军,而已经退回到中军的龙庭飞面色有些苍白,北汉军在他的指挥下虽然仍然占着优势,但是一时之间很难找到可乘之机了。而在不为众人注意的暗处,小顺子扶着近乎脱力昏迷的江哲缓缓走向临时搭建的营帐。而在北汉那面,一个周身上下用一件黑色披风遮住的黑衣人默默地看着手中断折的号角,终于长叹一声,隐入了黑暗,他的身影仿佛融入了夜色一般,很快就消失无踪了。

\chapter{第二十二章 内忧外患}

大雍武威二十七年,王从监军楚乡侯哲之策,以重兵当其偏师,斩谭忌,随后急行千里,往袭北汉军主力。其时龙庭飞知王离中军,戮力强攻,楚乡侯击鼓以励军心,当北汉军一昼夜。十一月九日,王率亲卫军距秦泽四十里。龙知难而退,王追击三百里,龙庭飞亲断后军,两军交锋十余次,互有胜负。十一月十五日,北汉段无敌领军接应,王以士卒疲惫,乃退回泽州。两军交战半月余,雍军伤亡六万,北汉军伤亡近四万,或曰此战无胜负,然此役后,北汉军再无余力寇泽州、镇州。

——《雍史·齐王世家》

雍都,长安,自从月初泽州传来八百里加急的军报之后,朝中群臣几乎都是忧心忡忡,这一次龙庭飞大举进攻泽州,虽然泽州大营兵多将广,可是并不代表有必胜的把握,不说龙庭飞乃是天下有数的名将,齐王这些年虽然可以勉强抵挡,可是却很难取得胜利,再说齐王和皇上的心结未解,泽州大营将帅不睦,重重隐忧令人头痛。这重重阴云并没有因为皇上派去新的监军——楚乡侯而消散,毕竟江哲不过是个文人,很多人都不相信他真能镇住齐王,就是他有本事调解齐王和众将之间的矛盾,对着龙庭飞也未必有胜算。

而且从北汉军甫入泽州,流言就在大雍各地出现,有人说这次龙庭飞倾全国之兵进攻泽州,大雍兵力已经不占优势,有人说雍军惨败,齐王不知生死,还有人说雍军内部发生变乱,不能抵挡北汉军的进攻,北汉军已经在泽州境内肆虐多日,杀死军民无数。当流言传入长安的时候,民心混乱。虽然多年来大雍的强盛让百姓心中较为自信,可是那流言说得绘声绘色,人心也不禁多了几分相信。没过多久,另外一种声音响起,说是大雍名将首推李贽,只有李贽御驾亲征才能扳回败局。

而在这种暗流潜伏的局势里面,长乐公主却起到了稳定人心的作用。长乐公主也是刚刚回京,在路上她就听见了这些流言,甚至庆王还曾私下里向她询问江哲是否有办法制住齐王。长乐公主自然只能微笑着劝慰庆王,说是齐王和驸马不会有什么纠纷,前方战事自有齐王负责。可是庆王似乎十分忧虑,虽然没有明说,但是却暗中派人加强了车驾的保护。长乐公主心中不是不担心前方战事,可是她相信江哲可以稳定泽州大营,她也相信齐王的军略,就算不能取胜,也不会大败,更何况江哲身边还有小顺子在保护呢。所以她仍然是神情从容,每日只是带着柔蓝和李麟观看沿途风景,当然有的时候还会抱着江慎,说起来,这三个孩子,倒似乎是江慎最好奇,若是想让他多睡一会儿,不让他看窗外,他经常都会哇哇大哭。

不过流言这样猖獗,长乐公主也觉得有些不对,而且在某日受到雍都的密旨之后,长乐公主便故意放慢了行程,绕道经过多处郡府,每到一处,她都主动接见当地高级官员的家眷。虽然她没有说过一句有关泽州战事和流言的事情,可是她那种平静愉悦的情绪感染了那些诰命夫人。人人都知道驸马楚乡侯身在泽州,如果泽州有事,公主怎会如此安详平静,这样的想法很快以更快的方式在中低级官员里面传递。等到长乐公主迟了多日回到雍都的时候,泽州虽然还没有战报传来,可是流言却几乎不会影响到官员了。这虽然是朝廷控制的缘故,可是长乐公主的功劳却是显而易见的。

十一月十七日,长乐公主的鸾驾终于到了长安,雍帝下旨,命太子李骏带领三品以上的官员郊迎三十里,凭着宁国长乐公主的身份,这并不僭越,而且京中谁不知道这次长乐公主回京一路上安抚人心的功绩。

撩开鸾驾上的珠帘,长乐公主眼中雾气朦胧,一段段回忆电闪而过,武威十七年,自己远嫁南楚,那时候的自己心中悲凄,只恨车驾走得太快,看不见长安烟云。武威二十三年,自己从南楚返回,虽然重回帝乡,却是心如古井,只想在亲人身边安度余生。之后自己虽然尽力闪避,却仍然被夺嫡憾事所扰,几乎不能在宫中安居,而这时,已经孀居的自己也心中波澜微起,可是心目中的良人却是咫尺天涯。直到武威二十五年自己不顾一切跟着良人离开长安,她才得到了从未有过的幸福安乐。如今自己重回长安,只怕是没有机会再回东海隐居了,心中有着和亲人团聚的喜乐,却也有着重新涉入世俗的无奈。

这时候,周尚仪带着几个宫女走过来,将几个孩子抱去,长乐公主平静了一下激动的心绪,露出淡淡的笑容,走下鸾驾,平静从容地看向迎接自己的众人。

已经将近十岁的太子李骏一大早就急匆匆地等着皇姑的车驾,说句实话,他和皇姑并不十分熟稔,毕竟没有见过几面,可是他可是很明白这位皇姑的地位。若不是宁国长乐公主,可能自己的父皇没有机会坐上皇位,而自己恐怕也早就没有命了,不过李骏当然明白最得自己父皇重视的一点却是,皇姑嫁给了楚乡侯江哲,用父皇的话说,这是把那个闲云野鹤的奇才绑在大雍战车上面最好的法子,而且还没有任何勉强和和隔阂。不过对于李骏来说,恐怕最重要的一点就是,那个多年不见的小妹妹这次也跟着皇姑回京了。想到这里,李骏不由气恼的想着当年他从幽州回到长安,本想着和柔蓝久别重逢,可是却是当头一个晴天霹雳,柔蓝居然被江先生给带走了,而且两年多来连封信都没有。心中有些惴惴不安,希望不会是柔蓝已经忘记我了吧。

终于等到了长乐公主的鸾驾,当李骏看到一身公主礼服的皇姑微笑着走到自己面前的时候,眼睛瞪得圆圆的,他可是还记得皇姑的模样,可是如今看来,明明外貌没有什么变化,却像是换了另外一个人,那种温柔娴雅,从容喜乐的神情,让人油然而生敬慕欣羡之情。

在郊迎礼毕之后,这时候,从站在鸾驾后面的宫女里面,一个娇俏可爱的小女孩冲了出去,一把抓着李骏的袖子,急切地道:“骏哥哥,你还记不记得蓝蓝。”

李骏看向那有些熟悉的小女孩,过去的回忆几乎立刻回到了脑海里,这一刻他忘记了一切礼仪,像过去一样伸手将小女孩抱了起来,高兴地道:“蓝蓝,你回来了,怎么这两年也不给我写信,我还以为你不记得我了。江先生,不,姑夫有没有欺负你,如果有,我去禀告母后,母后一定会替你讨还回来。”

柔蓝看着稚气消退,已经变得英俊玉立的李骏,突然大哭起来道:“爹爹欺负我,都不许我寄信给骏哥哥。”说罢,抽抽噎噎的柔蓝掏出厚厚一叠书信,都是写好之后却没能寄出去的信件。李骏只觉得不知怎么眼里有些水汽,这时候他已经想起不能在人前失态,努力抬高了头不让眼泪落下,郑重地接过那些信,道:“好啊,我一封封的看,蓝蓝就当成信在路上耽搁了很久吧。”柔蓝这才破涕为笑。李骏有些心虚地看看身后,还好,那些官员都很识趣得避得远远的,李骏这才送了口气,将柔蓝放了下来,一抬头,却看见长乐公主的笑容,不由脸一红,道:“皇姑,皇爷爷、皇祖母和父皇、母后都在等着您呢。”长乐公主微笑着牵过柔蓝的小手,道:“好,那我们就快些上路吧。”说罢领着柔蓝上了鸾驾,周尚仪也将慎儿送到鸾驾上面。如今,已经进了长安,就不方便让李麟也坐在鸾驾上面了,长乐公主眼睛的余光看见神色倔强的李麟,在上鸾驾之前低声和李骏说了一句话。

等到鸾驾启程之后,李骏走到李麟身边,温和地道:“你是麟弟吧,和我同骑如何?”

原本神色有些冷漠的李麟眼中闪过一丝温暖,方才因为柔蓝撇下自己去和李骏说话的酸意也渐渐消散了。不善言辞的他冷冷道:“我自己会骑马。”

李骏眼中露出惊讶的神色道:“你小小年纪就会骑马,真是厉害。”让侍卫牵过一匹御马,笑道:“这可是父皇赏赐给我的御马,性情很温顺,你骑骑看,可不要害怕啊。”李麟木木的点头。他年纪还小,但是这匹御马上面的鞍鞯都是特制的,所以李麟上马之后,很快就控制住了马匹,跟在鸾驾和李骏后面走向明德门。一路上,李骏不时问着李麟各种问题,觉得李骏有些罗嗦的同时,李麟心中也觉得越发温暖,看来自己在长安并不会太难过呢。

长乐公主走进太后居住的慈宁宫的时候,一眼就看见母后慈爱的眼神,她不由落泪,上前翩翩拜倒,长孙氏上前将爱女搀起,欣然地看到爱女容光照人,全不似从前憔悴模样,母女说了几句家常话,长孙氏挽着女儿让她坐在自己身边。长乐公主这才看到旁边还坐着颜贵太妃,连忙起身见礼。这几年颜妃虽然荣宠依旧,可是因为忧心爱子和当今皇上至今的隔阂,容颜之间带了几分苍老。宫中消息传得飞快,她早就得知长乐公主带了自己的亲孙儿过来,虽然有些恼恨秦铮连累了爱子,可是无论如何若非秦铮自尽谢罪,只怕事情会更加棘手,而李麟更是她的心头肉,若非实在不得已,她是不会让齐王带着李麟上战场的,这次听说长乐公主带了李麟回来,心中对长乐公主十分感激,而且她也听说了长乐的驸马去了泽州做参军,今后爱子一生荣辱可能就要看江哲夫妇的了,所以颜妃十分客气亲切地搀起长乐,道:“贞儿,听说你带了柔蓝和慎儿过来,姐姐早就想着外孙呢,还不快把他们带进来。”

长孙太后听了拊掌道:“妹妹,你说哀家是不是糊涂了,本来还想着让孩子们进来,可是一看到贞儿竟是什么都忘了。田尚宫,快些宣孩子们进来。”

不多时,周尚仪亲自抱着江慎,柔蓝和李麟跟在太子李骏后面走了进来,却是李骏一时舍不得和柔蓝分开,便也跟了来。

长孙太后却先是招手让柔蓝走到近前,将她抱在膝上,道:“小蓝蓝,可还记得哀家么?”

柔蓝眼中闪过兴奋的光彩,抓着太后道:“记得,蓝蓝很想娘娘,也很想皇帝爷爷。”

太后亲切地道:“如今你叫贞儿母亲,也该改口叫哀家一声外祖母了,太上皇这两年还时不时说起你,不过今日却又托词去打猎,唉,谁让他这么好面子,总记着当年不同意贞儿和你爹爹的婚事的事情,担心你们给他脸色看呢。”

众人听了都觉好笑,可是却都强忍,太后可以这么说,他们可不能嘲笑太上皇啊。

然后太后又道:“好了,快把慎儿抱过来,让哀家看看这个小外孙。”

长乐公主亲自接过爱子,抱到太后跟前,柔蓝乖巧地从太后膝上跳了下来,太后接过小娃儿,眼中泪花闪过,这是流着她骨血的孙儿,她自然心中爱极。江慎也精神得很,全不怕生,虽然就连走路也是踉踉跄跄,基本上还处在爬行阶段,可是并不妨碍他用小手去摸太后的凤冠。太后亲了半天,突然问道:“皇后怎么还没有过来,不是说今天一早就要过来么?”

田尚宫恭恭敬敬地道:“启禀太后,皇后娘娘本要过来的,可是段才人今晨突然腹痛,恐怕是要早产,皇后担心得很,所以派人禀告过了,要晚一些过来。”

长孙太后叹息道:“皇后果然贤德,皇上子嗣艰难,至今只有骏儿一个嫡子,若是有些什么意外,岂不是让皇上忧心么,如今朝中颇不宁静,边关又在打仗,也亏得皇后这个贤内助。四个月前,若非是皇后亲自过问,只怕段才人这个孩子就保不住了。”

颜贵太妃见长乐公主有些奇怪,便道:“这也是一件宫闱惨事,皇上登基之后,授意礼部裁撤后宫品轶,确定内廷主位,依次是皇后,贵、娴、淑、德四妃,昭仪、昭容、昭媛、修仪、修容、修媛、充仪、充容、充媛为九嫔,婕妤九名,美人九名,才人九名,其余主位全部裁撤。

雍王妃自然是正位中宫,赵氏和云氏都是做了多年的侧妃,又生了公主,所以赵氏封了贤妃,云氏封了德妃,因为后宫太过冷落,所以太后下懿旨选了一次秀女,其中最出色的就是司马修嫒,永和宫的主位,此女有些娇纵,不过也算是才貌双全,想不到却是心肠狠毒。永和宫里面的梨香阁住着段才人,段才人出身寒门,性情柔顺,皇上临幸了两次,就怀了身孕,这段才人不算受宠,又有些糊涂,居然没有留心,却被司马修嫒先知道了,竟在宫门下匙之后带着亲信闯入梨香阁,逼着段才人喝打胎药。永和宫诸殿本就是司马修嫒的天下,梨香阁又较为偏僻,居然让她肆意而为。可是这段才人也是外柔内刚,被灌药之后趁着防守不严,拖着性命逃到程婕妤居住的西配殿。程婕妤却是魏国公的远亲,家中也是将门,此女更是生就侠肝义胆,平素本就常常护着段才人和其他被司马修嫒欺凌的嫔妃,这次居然违背宫规,翻墙出了永和宫,连夜到坤宁宫求见皇后,禀明此事。这下事情可闹大了,皇后连夜赶去,下令软禁司马修嫒,又召御医全力救治,总算是段才人身子强健,又是拼命挣扎,只喝了大半碗药,这才保住了孩子和性命,可惜如今又是早产,也难怪皇后如此紧张,都顾不上来接你了。”

这些事情在宫廷中屡见不鲜,可是长乐公主仍然心中不乐,问道:“这司马修嫒是什么背景,竟然如此嚣张,这种事情别说一个修嫒,就是换了四妃也是不敢做的?”

长孙太后在颜贵太妃开始谈及此事的时候就让人将几个孩子带到外面玩去了,并遣散了宫人,此刻也是神色阴沉地道:“谁说不是,历朝历代,除非是皇上专宠到无法无天的地步,哪有妃子敢如此放肆,如今皇上对后宫疏淡得很,皇后又是震得住的人,这件事情哀家都觉得奇怪。后来皇后详查之后,这司马修嫒本是原蜀国世家之女,如今她的亲族仍然是东川第一名门,若非如此,就算她才貌双全,也不能进宫就做了修嫒。司马氏如今在东川也是庆王的最大助力,庆王更是亲自进宫向皇后求情,所以碍着庆王的面子,皇后只能下旨,将司马修嫒送入冷宫了事,程婕妤立下大功,封了充容,段才人无辜受害,不过因为如今不能起床,孩子也没有临盆,所以还没有封赏。”

长乐公主目中寒光一闪,又是庆王,对这个皇兄,她心中本是有些同情和敬佩的,可是这次相见之后,却见他处处和齐王为难,这还罢了,可能是因为从前凤仪门的事情让他心有余恨。可是这司马修嫒的事情未免有些蹊跷。长孙太后和颜太贵妃交换了一个眼色,她们对于这件事情十分不满,颜太贵妃自然是因为庆王是攻击爱子的主要人物,而长孙太后却是因为同病相怜,她几个儿子都没有活到今日,所以她最看不得戕害孩子的事情,司马修嫒触犯了她的逆鳞,可是虽然太后身份尊贵,长孙氏却是不愿意多管后宫的事情,毕竟皇帝不是她的亲子,她不想过于干涉皇后的权力。可是长乐公主就不同了,身为大雍皇室最尊贵的公主,驸马又是皇帝的心腹重臣,长乐公主若是出面,这件事情是谁也不敢多嘴的。

长乐公主眼中闪过一丝犹疑,她也对司马修嫒生出杀意,当年迫不得已亲手害死腹中娇儿,曾让她午夜梦回,泪湿罗衣,即使那是她不喜之人的骨肉。可是这样干涉皇家的事情,长乐公主不免有些担心,她是知道江哲的性子,本是最不喜欢惹麻烦的。

正在这时,突然门外传来急匆匆的脚步声,三人抬头望去,外面的尚宫高声道:“皇后娘娘到。”长乐公主站起身来,长孙太后和颜贵太妃也急切地向外看去,皇后高氏神色有些憔悴,身后跟着后宫主位妃嫔,进来给太后见礼之后,皇后黯然道:“段才人强撑着生了一位皇子,可怜她却抛下孩子去了,竟是连一眼孩子的面都没有见到。”

众人都是唏嘘不已,长乐公主心中生出怒气,上前给皇后见礼。高氏连忙扶起长乐,强颜欢笑道:“妹妹今日回来,本宫都没能去迎接,真是失礼。”

长乐公主劝解了皇嫂几句,抬眼看到妃嫔中一位婷婷玉立,面带英气的女子甚是悲凄,便用目光向皇后询问,皇后叹了一口气,道:“程充容,你也不要难过了,这都是命中的劫数,本宫知道你和段才人交好,身后之事,本宫不会亏待她的。母后,儿媳想段才人孕育皇子有功,就追封昭容吧。”

程充容却是上前拜倒道:“太后、贵太妃、皇后娘娘,臣妾原本没有资格说话,臣妾和段才人虽然交好,却也是泛泛而已,可是臣妾心中不平,那害人凶手虽然打入冷宫,可是却还活着,过几年遇上大赦,还可出宫还家,可怜段才人却是香消玉陨,还请母后和娘娘为她作主。”

三人都是有些难色,皇后用余光瞧了长乐公主一眼,道:“司马氏已经受到惩戒,这件事情本宫也很难追加罪责。”

程充容面色悲愤,含泪起身,皇后向太后施礼道:“母后,二皇子生而丧母,本来应该本宫抚育,可是本宫近来事情繁杂,不若将二皇子交给程充容抚养吧。”

太后点点头,道:“程氏,你是忠良之后,又是二皇子的恩人,可愿好好抚养他。”

程充容虽然难过,却也不由受宠若惊,道:“只恐臣妾不能尽职。”皇后温言劝慰,程充容终于坦然接受这样的恩遇。

皇后见事情暂时压了下去,便笑道:“时候也差不多了,本宫在坤宁宫设家宴为长乐洗尘,晚一些皇上也会过来,母后和太妃娘娘不如现在就过去吧,看看本宫准备的佳肴是不是合意。”

长孙太后和颜贵太妃都是笑容满面,在宫妃和女官的陪伴下出门而去,皇后故意落到后面,挽着长乐公主的手臂道:“妹妹,你的府邸本宫已经全部打理好了,你尽管住进去就行,不过今日可不能出宫。”

长乐公主心中一暖,反手握住皇后的手道:“皇嫂费心了。”然后她近似耳语地低声道:“皇兄怎么说?”

她虽然问得含糊,皇后却是立刻回答道:“皇上说,也该给庆王小小的警告,不过现在不宜重整东川防务,所以皇上和本宫都不好驳了庆王的面子。”

长乐公主心中明白,微微点头,不再说话。

当夜的坤宁宫灯火辉煌,太上皇李援终于还是忍不住对女儿的思念回来了,一见长乐公主便是喜笑颜开,看着女儿神采飞扬,没有什么比这个更让他高兴,慎儿年纪还小,自然不能入席,柔蓝却是被李援拉着坐到他身边。而随后而来的李贽则是让李骏和李麟分坐在他身侧。看得李康面色阴沉。

尽欢而散之后,当夜三更,长乐公主却是没有入眠,带了周尚仪、小六子和几个强壮有力的宫女太监闯入了冷宫,冷冷的看了那个原本娇纵美丽,如今却是形容憔悴的司马氏半晌,然后下令将其杖杀。那一夜,司马氏的悲嚎声惊动了整个冷宫。

第二日长乐公主当面向太后、皇后谢罪,太后刚刚假意训了长乐公主几句,闻风赶来的李援就出言开脱,这件事情就这么不了了之,就是庆王李康也不敢和自己的父皇抗议的。

三天之后,泽州捷报传来,而几乎同一时刻,南楚军情传来,陆灿出雒城,占领蜀中,兵压葭萌关,一路所向披靡,葭萌关告急,两国之间,再没有转圜的余地。

\chapter{第二十三章 万金家书}

文乡侯霍琮,出身寒微,太祖武威二十五年,为雍王府司马江哲救入王府。初时未蒙青眼,为寒园仆役。后太宗以潜邸赐宁国长乐公主,琮仍执役寒园,日常偷阅哲文稿书籍,为昭华郡主所察,郡主怜之,书告楚乡侯此事,哲闻之心动,传书公主,为其延师教读。后,哲自军中归,试其文章而喜,乃收归门下。

哲虽世称才子,文章锦绣,冠绝天下,然多涉猎,琮性谨严,唯读经史,青出于蓝,遂成文宗,然终琮余生,事哲如父。

——《雍史·文乡侯列传》

大雍武威二十七年十二月末,泽州大营上下一片喜气洋洋,在数年僵持之后终于取得了一次胜利,军中将士都是喜笑颜开,更何况皇上传旨重赏三军,所有的军士的荷包都是满满的,胜利加上赏赐令泽州将士扬眉吐气。

在龙庭飞退出泽州之后,齐王李显下令趁着雪降之前在沁州边境扎营,经过三十万雍军和从泽州征调来的二十万民伕一月奋战,修建了百里营盘,这一次,雍军是绝对不允许北汉军再次进入泽州的了。临近新年,泽州大营防守虽然森严,可是还是允许军士轮流出营,虽然附近没有城镇,可是逐利的商人早就在建立了临时的集市,临时搭建的房屋虽然简陋,可是却很温暖,酒店、青楼、赌场样样都有,还有各种各样的货物出售,齐王并不反对集市的出现,毕竟没有这些,冬天可就难过了,但是为了安全仍然派了军队将集市控制起来,免得北汉的间谍趁机入内探听军情。虽然没有军令下达,可是人人都知道,明春进攻北汉,已经是板上钉钉的事情了。

中军大营,我倚在软榻上看着家信,这次皇上派人来传旨嘉奖,顺便还带了家书过来,长乐和柔蓝都有信来,虽然很想看看柔蓝写些什么,可是对长乐的思念已经盈满心胸。所以我还是先打开了长乐公主的信。

长乐公主的信很长,居然写满了七张丝绢,从墨迹的新旧看来不是一次写得,可能是随想随写,每日都写上几行字,然后才随着使者而来。

“妾行程颇平顺,唯慎儿为慈真大师所占,妾终日难见数面,慎儿已能行步,然不能久,夫君归日,应能见慎儿独自行走矣。

……

三王兄对麟儿颇冷遇,妾虽不满,但兄妹多年不见,王兄又奉旨接妾身回京,不便劝止,只得令麟儿、柔蓝不离左右。妾心中忧虑,三王兄如今权势滔天,却对六王兄恨意不休,妾恐兄弟閲墙事重演。

……

得皇上密书,京中有流言说泽州兵败,妾知有夫君在泽州,必不至如此,然流言过处,人心惶惶,不得已妾身放缓行程,沿途接见地方官员眷属。

……

太子郊迎,礼重如此。妾身心有愧意,太子虽然年少,却是聪明仁厚,柔蓝和太子青梅竹马,重见仍然如昔日亲厚,麟儿虽孤傲,太子以诚相待,麟儿已兄事太子。皇兄下旨命麟儿为太子伴读,京中颇有非议。

……

有一事,妾心不安,司马修嫒,前蜀贵女,戕害怀孕才人,虽得皇后阻止,然才人产后而亡,二皇子可怜,生母卑微,无辜受害,所幸程充容仗义相救,皇嫂已令二皇子拜程充容为母。然司马修嫒戕害皇子,害死皇子生母,其罪非轻,按国法宫规,应杖杀之。

奈何三王兄亲来求情,言道镇守东川需蜀国世家襄助,司马氏功绩显著,若杀其女,恐东川生乱,皇嫂不得已赦其死罪。然母后众人皆心恨之,皇兄子嗣不昌,若如此姑息,恐后多生事端。妾身至京,母后相托,皇嫂暗传圣意,妾身乃于当夜杖杀修嫒于冷宫,虽是皇命,妾身仍难心安。惟恐三王兄记恨妾身,妾身得父兄爱护,谅无恙,唯虑波及夫君,望夫君志之。

……

妾身闻南楚军兵压葭萌关,皇兄已遣三王兄回东川,然妾心不安,夫君前番书信提及陆灿绝情之事,此子世代将门,又得夫君亲授兵法,妾身恐东川不敌,又三王兄与皇室裂痕宛在,妾身见其心思深沉,恐东川生变,军国大事,妾本不当过问,然若东川乱,北汉战事难息,妾不忍君久戍,故心实忧虑。

……

今日海仲英秘密入京,求见妾身,愿求周尚仪为妻室,妾身早闻两人钟情已深,然尚仪名字仍在宫中名册,海氏又常年出海,商人重利轻离别,妾身心有犹豫,故未挑明此事,今海氏意诚,妾身遂作主许之,前日已请准母后懿旨,消去端娘名字,定于年底完婚,然家事虽有董总管和小六子照看,内宅仍需女官,皇后已从内廷擢升良者为府中女官,此虽殊恩,然妾身不知夫君意下如何,未敢应允。

……

”

我看完书信,轻轻叹了口气,长乐是有些多想了,她是担心皇上想在我身边安排个人监视,其实公主府中家将侍女至少也有几百人,而且都是入京之前皇后亲自安排的,想要安插一个探子真是神不知鬼不觉,何必这么明着插人呢,再说皇后亲选的女官一定是精明能干,一定能够让长乐少费些心思,留这么一个人对我来说只有好处。就算是这人负着监视的责任也没有什么关系,我也没有什么一定要隐瞒的事情,再说,从司马修嫒的事情来看,皇上和皇后是将长乐当成得力助手了,这样一来,不免要有些秘密的消息传递,有这样一个人就可以留下传递消息的通道,更是求之不得的事情。

我抬起笔写了封回信,让长乐代我主持海仲英和周尚仪的婚事,另外皇后的好意一定要接受,至于庆王的事情我没有提,我并不想让长乐为军国大事烦恼,这些事情自然有皇上去操心,而我也不会去东川,开玩笑,我在那里的名声估计差的很,我可没有忘记蜀王的事情,不过锦绣盟在东川蜀中发展的不错,不过前段时间没有什么特别的情报传来,看来我应该催促陈稹一下了。

写完给长乐的回信,我又拿起柔蓝的书信,打开之后,刚看了几行字就几乎气歪了鼻子,这个小丫头居然在信里面得意洋洋地说道,她已经向皇后告状,说我不许她写信给太子,皇后答应她等我回京之后要好好教训我。

还好接下来都是跟我夸耀太上皇带着她微服出去玩乐的事情,看来太上皇对柔蓝可不是一般的宠爱呢。将其中欢乐描述得如同亲临其境,除此之外,就是等到太子从南书房回来之后,三个小孩子一起去玩的乐事。我心中有些酸意,这个小丫头总是处处如鱼得水,而且好像天生就是来克我的,玩得这么开心,居然还没有忘记告状。

最后面写得却是一件有些古怪的事情,柔蓝提及她溜到寒园去玩,那里因为曾是我的故居,里面至今仍然保留了许多文书和珍贵书籍,所以数年来都是有专人保护和整理的。因为李麟听了几日课居然嘲笑她不会诗文,柔蓝一怒之下想到我的书房去找一本少见的书来难为李麟,因为我还没有回京,所以寒园禁令仍然有效,柔蓝是偷偷进去的,毕竟寒园的防卫不可能像从前那么严密。可是柔蓝却发现了一件有趣的事情,有一个小男孩趁着没人看见偷偷看我的文集和藏书。柔蓝本想将这个小厮扭送到长乐面前,可是查了一下,却知道这个小男孩叫做霍琮,本是我带进雍王府的,这两三年一直在寒园整理花木,柔蓝想了想,若是这件事情被人知道,霍琮肯定会被赶出去,她虽然淘气,却是心软,不愿告密,就逼着霍琮给他讲解文章,据柔蓝说,霍琮讲得比我好,因为她能够听得明白。

看到这里,我不由陷入沉思,当年我路遇东海和庆王的属下,救回了一个孤儿,这个孩子有一双倔强的好眼,可是我当时一心一意都是夺嫡,根本就没有留心这个孩子,记得后来这个孩子就做了雍王府的仆役,而且因为他料理花木十分出色,有一次被我看见,就随口一句话让他进了寒园伺候花木。不过这个孩子我一直没有留心,想不到他还在寒园,而且听柔蓝说来,倒是一个好学上进的孩子。

想我江哲有才子之誉,可是我的几个记名弟子却都是武将,柔蓝不用提了,她若是对读书有兴趣,难道我还会不教她么,慎儿么,虽然年幼,可是怎么看都不像读书的种子,如今拜在慈真大师门下,将来做武林高手应该没有问题,若是说到文章,我就不抱什么奢望了。想来想去,我这满腹经纶居然没有一个传人,想到这里,我心中一动,拿出给公主的书信,让她先给那个霍琮请西席教读,心中想定,若是霍琮果然不错,我就收了这个弟子,若是我看不中么,栽培一个人才也没有什么不好吧?

放下家书,我又拿出皇上的密信,上面所说的正是如今的局势,南楚这次出兵事先全无征兆,拜当年皇上劫掠建业之赐,虽然南楚朝臣对大雍十分忌惮,可是却是畏惧多过仇恨,事实上如今南楚的政务掌控在尚维钧手中,这人怕是恨不得用金银财宝买的平安,这几年来,南楚每年除了例行缴纳的五百万两白银赔款之外,还要送上各种珍贵的贡品,女子金帛,我在南楚的生意这几年官府征收的税收已经是原来的三倍,虽然还有陆灿、容渊这样的武将,可是兵力却几乎没有什么增长,这也是没办法的事情,军队所需的辎重、粮饷何等巨大,无钱就别想养兵,不过我还是要佩服陆灿的,他这两年在蜀中屯田,并且通过长江水运和海运做走私的生意,所得金银众多,不仅练了一支精兵,还可以支援镇守荆襄的容渊。当然这件事情知道的人并不多,陆灿做的十分严密,就是南楚一手遮天的权相尚维钧也不很清楚,毕竟现在南楚军队可以说是陆家的天下,尚维钧若是逼得太紧,只怕还没有等到大雍南下,南楚就已经起了内乱。至于我知道这件事情,实在是因为天机阁和锦绣盟都有涉入,不过我倒不想阻止这件事情,不说这生意每年给我带来百万银钱,能够掌控南楚军队的财源就已经很令我得意了,只要需要,我可以随时切断南楚的走私路线,这样一来,没有了钱粮的南楚军队可就是捉襟见肘了,不过这样的利器自然是要在关键时候使用的,就是这次南楚兵压东川我也不想使用,毕竟大雍不可能两面作战,在北汉未平之前,还不能断绝南楚的希望。

将皇上的密书和兵部转来的军情再次翻阅了一遍,我心中突然生出奇怪的感觉,怎么会这么巧,北汉新败,南楚兴兵,庆王行径又是如此古怪,据我所知,这庆王有本事在东川经营多年,就连凤仪门如日中天的时候也不能把他怎么样,这样一个人,怎会轻易流露出和皇室的分歧,他恨齐王不要紧,可是却不该在齐王用兵北汉的时候生变,一个司马修嫒,虽然是前蜀贵女,可是毕竟是亡国之后,又犯了这样的大错,按理说,别说是赐死杖杀,就是问罪司马氏也是理所当然的事情,庆王只需要要求不问罪司马修嫒的族人,就已经是难得的人情了,一个女儿应该不会让司马氏做出和大雍朝廷决裂的决定,为什么我觉得庆王的做法有些过分嚣张呢?这三件事情中间必有联系,可是我却是一时想不出来。

想了许久还是觉得没有头绪,便放下文书,走出帐去,这时候已经将近黄昏,外面的空气十分寒冷,冷气扑面,我打了一个冷战,这北地的气候可真是难熬,虽然离开南楚已经多年,可是我还是不习惯北地的寒冷。冷风让我的头脑清醒了许多,我索性什么都不想,就这么漫无目的走来走去。走着走着,我突然看到小顺子正低头走进一个小营帐,顿时心中生出好奇,这几日他总是不见踪影,我本来还以为他是又在练什么新招式呢,想不到却在这里,四处看了一下,却原来我走到了监押重要俘虏的地方,可是小顺子到这里干什么呢?

虽然知道非礼勿视,可是我真的很是好奇,故意走到离那座营帐不远的地方,虽然这个距离还是挺远的,至少我身边的侍卫是听不见里边的说话的,可是我能听清啊,摆出陷入沉思的模样,好像还在考虑战策,可是我的心思全部用在耳朵上,仔细听着里面的情形。

凌端躺在床榻上,眼中满是冰寒和悲恸,他是鬼面将军身边鬼骑的唯一幸存者,他至今仍然清楚的记得,就在最后一刻,战马已经失去,只剩下几个鬼骑护着将军对着数不清的马槊和马刀,身边的同僚一个个失去了生命,终于战场上只剩下了将军和自己,事实上凌端至今不敢相信自己居然能够活到那个时候。雍军高呼着“生擒谭忌”围了上来。将军却将自己护在身后,他虽然能够暂时护着将军的后背,可是将军分明接去了大半攻势,那一刻,凌端发觉将军竟然是在拼命保护着自己,心中感激羞愧的凌端只能拼命防守,除非我死了,不能让任何人伤到将军的后背,这是凌端唯一的想法。最后一个大雍的武将似乎看出了自己是将军的弱点,转而猛攻自己,就在他的马槊将要刺进自己的咽喉的时候,将军竟然用手臂替自己挡住了那致命的一击。可是这样一来,局势更加险恶,万军重围当中,重伤无马,怎可能还有生还的希望。不过片刻,自己被刺倒在地,而将军就站在那里一步不动,长戈化作铜墙铁壁,护着自己不让那些杀红了眼睛的雍军顺手取了自己的性命。即使他身死之后,仍然用身躯将他护在身下。一动也不能动的凌端就这么近的看着他长戈飞舞,收取了无数生命,看着他被人围杀,自始至终,将军都没有说一个字,可是凌端分明看到将军的眼睛充满了鼓励,那是让自己保重的眼神。在谭忌仆倒在地的时候,凌端便晕了过去。

事实上,当凌端在大雍的军营内醒过来的时候,悲痛屈辱当中心中也有一丝喜悦,生命的美好他还没有完全领略,死亡毕竟不是他希望的事情,可是被俘之后的命运又会如何呢?他不会怀恨那些大雍将士,因为将军早就说过杀人者人恒杀之,当日他身为鬼骑,长戈之下,冤魂无数,今日虽然将军和同袍都死在雍军手中,甚至自己的两个哥哥都是战死沙场,不过凌端却也不会怨恨雍军,他只恨苍天,为什么天下要战乱,要让自己这些小民的性命贱如蝼蚁。当然凌端不恨雍军,却也不会感激雍军救治了自己,若是有机会,凌端还是希望能重新上战场杀敌,将军可是说过有什么仇恨,都到战场上面了结的。可是想要逃跑哪有这么容易,自己成了战俘就是不处死也要被送去做苦役,哪有可能回去北汉呢?

不提凌端心中所想,这座营帐却不是他一人居住的,所有俘虏都被监禁在军营当中,不论尊卑,都是十二人一个营帐,没有床榻灯火,只有少数身份比较特别的俘虏有较高的待遇,而凌端得到这样的待遇多半因为他是谭忌身边的鬼骑,可是另外一个和他住在一起的俘虏就有些奇怪了,那人是石英营中的一个什长,叫做李虎,这人虽然勇猛,却是性子鲁莽,职位又低,怎会被特别监押起来呢?可是这人是石英的部下,谭忌和石英最是不合,所以凌端也不愿意去理他,直到这人活转过来得意洋洋地说道,他大雍的监军楚乡侯给撞到水里,虽然没有成功的取了那人性命,可是李虎还是很得意,这下凌端可就明白了,带着同情的眼光看着这个笨蛋,虽然他并不十分清楚这位楚乡侯是什么人物,可是明摆着给这小子治伤是准备给他好看呢,就像杀猪之前总要养肥一样,不过想了想,他还是没有告诉这个少根筋的家伙渺茫的前途,毕竟自己这些人小命早已经不是自己的了,早知道也没有什么用处,还是让他多舒心几天吧。

正在胡思乱想,这时候有人走了进来,这人是一个青衣少年,容貌秀雅,带着几分阴柔,却又神情如冰霜,如同寒天飞雪一般孤洁,凌端只看了一眼就又躺了下去,那人这些日子常常过来,说来也奇怪,这人每次来都是只问两人伤势如何,然后说几句闲话就走了,态度虽然冷淡,却是没有一丝轻蔑之意,每次来都会带来上好的伤药,和一些精美的食物,凌端发觉,自从这人常常过来之后,监押自己的军士似乎更加多了,而且态度也都很恭敬。从这些凌端能够觉察出这人身份必然非同反响,可是问过外面的军士,却是一个个凛若寒蝉,谁也不肯谈及那人的事情。不过这人虽然亲切,凌端却是丝毫不愿意接近他,或许是多年沙场征战的缘故,凌端对于危险十分敏感,他能够感觉到那人虽然相貌清雅,神色中丝毫不露杀气,但是骨子里却是一个不将人命看在眼里的人。至于李虎,似乎也不大喜欢看见这个人,倒不是他有那么聪明,有一次凌端听见李虎嘟囔着什么“娘娘腔”之类的话,看来是他粗豪的性子犯了,看不得这种人的存在罢了。今日这人进来却和往日有些不同,双手空空,并没有带什么东西,虽然没有说话,可是凌端却能发觉他身上散发着从骨子里流露出来的冷意。不由心中苦笑,想必今日这人已经准备撕下面具,同情的看了李虎一眼,凌端能够感觉到,这人的目标不是自己。

\chapter{第二十四章 布局天下}

龙庭飞神色怔忡地坐在蒲团之上,默默的望着摇曳的灯火,已经七天了,自从泽州一战之后,边关暂且无事,龙庭飞便被北汉主召回晋阳,龙庭飞原本心中充满愧疚,只道要受斥责,谁知回到晋阳之后北汉主便把他召入晋阳宫,而接见他的却是北汉国师京无极。龙庭飞虽不是魔宗弟子,但是却多得京无极教诲,心中早已将他当作师长,若是京无极骂他几句,他倒觉得心里舒服许多,可是魔宗对战败之事却是一字未提,只命他在这空无一物的静室中面壁七日。

这七日,龙庭飞因着难得的安宁,仔细的思索着自己的过错,将泽州大战前后经过仔仔细细地想了无数遍,可是想来想去,龙庭飞却悲哀地觉得,这个圈套自己就算事先知道,也最多不过拼个惨胜罢了,难道自己的赫赫英名都是没有遇到敌手才得到的么,那么从未见过的江哲,莫非是自己的克星不成么。每想一次,龙庭飞就是越发心寒一些,七日之后,龙庭飞竟然觉得衣带渐宽,不由心中苦笑,但是却觉得心中明快许多,虽然知道了敌人的强大,可是龙庭飞心中反而宁静下来,他已经没有任何选择,大雍兵压沁州,最迟明年就会爆发大战,这一战,不是北汉亡国,就是大雍数年之内无力北上。

这时,有人推门进来,龙庭飞也不回头,仍然沉默不语,那人轻叹一声道:“宗主召你前去见他。”

龙庭飞这才起身,整理了一下衣衫,转身向那身形颀长地中年男子恭恭敬敬行了一礼,道:“庭飞见过段师兄。” 这中年男子乃是魔宗首徒段凌霄,龙庭飞虽然不是魔宗弟子,可是也曾得魔宗指点,段凌霄更是对他十分关爱,龙庭飞视之如兄,此时自是不敢失礼。

魔宗传承极严,绝没有广收门徒之事,虽然北汉很多高手将士都接受过魔宗的训练,可是最多也不过是一个记名弟子,京无极在北汉多年,门下也只有四个弟子,其余魔宗长老传人加在一起也不过半百之数。

京无极亲传四大弟子,首徒段凌霄,乃是魔宗多年随侍弟子,京无极常年闭关谢客,魔宗之事几乎都由段凌霄代掌,此人气度凝重,沉稳精明,武功也是极为出色,乃是下任宗主的不二人选,谭忌就曾经得他相传戈法武技。

魔宗次徒苏定峦,龙庭飞麾下四将之一,此人性情直率勇猛,最为京无极心爱,可惜已经身死大雍,英年早逝。

魔宗三徒萧桐,龙庭飞近卫,负责探察军情,为人狠辣果决,性情多疑,探查军情少有差错,是龙庭飞心腹之人,也是龙庭飞的左膀右臂。

魔宗四徒秋玉飞,本是月宗弟子,其师早年亡故,托孤于京无极,此子今年只有二十六岁,身兼日宗月宗两门之长,博学多才,精通音律,能以乐声伤人,武功天赋十分突出,此人天性不喜约束,最喜游荡,除了魔宗谕令之外,从不过问任何事情。外人虽然知道魔宗有四个弟子,可是却几乎没有人知道秋玉飞的形貌本领。

段凌霄微微一笑道:“庭飞,你也不要过于烦恼,宗主召见,必然有相助之策。”

龙庭飞心中稍安,苦笑道:“庭飞已经计拙,只盼着国师可以力挽狂澜了。”

段凌霄淡淡道:“宗主就算是有了计策,若没有你这大将军领军作战,也是无益于事,走吧,四弟已经回来了,也在宗主那里等你。”

离京无极居住的宫院还有一段距离,风中突然传来了铮铮琴声,只听琴声的出神入化,龙庭飞便知道是秋玉飞所弹奏,他微微一笑,说道:“玉飞的琴技越发进步了。”

刚说到这里,琴声一变,杀伐之声溢满天地,龙庭飞不由停住了脚步,这旋律似曾相识,龙庭飞也算是文武双全,听了片刻,突然记起这是秦泽决战之际敌军阵中传来的鼓声,竟被秋玉飞化入了琴曲。龙庭飞怅然而立,他怎会忘记那日,就是这鼓声让大雍将士稳住了心神,抵挡住了自己的攻击。他清晰地记得,自己遥望大雍中军的时候,那在帅旗之下,双手拿着鼓槌,站在高处奋力击鼓的瘦弱身影。就是那个文弱书生,让自己功败垂成。想到这里,龙庭飞突然明了,为何当日战场之上会有号角声相助己方,想必竟是秋玉飞到了秦泽,见江哲击鼓振奋军心,便以乐声襄助北汉军,可惜却没有成功。这些日子想必秋玉飞就是在揣摩如何将当日江哲的鼓声化入琴曲的吧,想必当日的败阵,即是自己的败绩,也是这高傲青年的奇耻大辱。

轻轻叹了一口气,龙庭飞再次举步,走上了玉阶,前面正是北汉国师京无极隐修之处——兰台。

兰台是一座三层高的楼台,雕梁画栋,美伦美央,晋阳宫本是东晋行宫,百余年来数次增建重修,宏伟壮丽,虽然两代北汉主都是不好奢华之人,除了必要的修缮之外,并没有增加什么建筑,可是仍然有着引人入胜的美好景观和富丽堂皇的华丽宫室,位于晋阳宫西侧的兰台就是其中之冠。这里本来是北汉主最爱流连的宫院,但是自从京无极封了国师之后,为了表示尊敬亲密之意,北汉主特意将兰台送给了京无极作为居处。自此以后,除非是京无极相邀,就是北汉主也不会擅自到此。

随着魔宗侍者走上兰台,兰台的第三层乃是露天修建,上有穹庐遮日,中有玉柱金梁支撑,地上铺着锦绣毡毯,四周以玉栏相护,从上而下垂着珠帘纱帐,层层掩映,仿若琼楼玉宇,不似人间。龙庭飞沿着玉阶走上兰台,只见兰台后侧中央,摆着一张舒适的软榻,一个蓝衫中年人倚在软榻之上,合着双目,似是小憩,软榻前方右侧一个黑衣青年席地而坐,面前放着玉几古琴,那青年正在一心一意地抚琴。在软榻左侧,一个香炉里面正冉冉升起淡淡的香烟,更是衬得此间仿若仙境。

龙庭飞看了一眼,走到台中的蒲团之上跪了下去,而段凌霄却是对着那蓝衫人京无极施了一礼,然后便坐了下来。

这时,“铮”的一声传来,却是断了一根琴弦,琴声突然嘎然而止,那黑衣青年抬起头来,那俊美无暇的面容上露出了一丝黯然。京无极坐起身来,叹息道:“玉飞,你的心乱了,看来这些日子的潜修还是不能让你从那日的打击中振奋起来。”

黑衣青年面上露出惭色,下拜道:“师尊,弟子平生别无所好,唯爱音律,自负天下没有敌手,可是那江哲只以战鼓仓促成曲,就胜了弟子,弟子心中绝不能服气,可是弟子竟然无法将那一曲谱入琴中,那江哲不过是三十岁年纪,又是多年卧病,弟子怎也不信他在音律上下的功夫胜过我多年苦修,难道世上真有人的天赋如此出色么?”

京无极看看龙庭飞挺拔玉立的身躯,笑道:“庭飞,你认为玉飞的音律果然不如那江哲么?”

龙庭飞犹豫了一下道:“弟子对音律所知不多,可是还是觉得似乎玉飞胜过江哲。”

京无极笑道:“玉飞,你这些日子斤斤计较音律上的胜负,却忘记了你和那人是在战场上相斗,你们的鼓声和号角声影响了军心,可是军心士气也影响了你们的乐声,如今就是让那江哲再次击鼓,也绝不可能重现那日的鼓乐,玉飞,你的音律之道天下无双,可是我北汉军却胜不过被激发了士气的大雍军,所以你之惨败,并不在于音律,江哲此人,善于因情生势,也善于借势生情,你若能体会到天人合一的妙境,武道必可突飞猛进,不可懈怠啊。”

黑衣青年秋玉飞眼中闪过了悟,下拜道:“弟子叩谢师尊教诲。”

龙庭飞听到此处只觉得玉面如同火烧一般,羞愧难当,京无极见了微微一笑,道:“庭飞你可是因为落败而含羞么?”

龙庭飞俯首道:“庭飞无能,辜负王上和国师的厚爱。”

京无极站了起来,走到近前亲手将龙庭飞搀起,道:“庭飞,你错了,能够带着二十万大军抵挡大雍多年,除了你世人有几人可以做到,整整十四年了,大雍在泽州最多时候曾进驻军五十万,四次攻入沁州,更有一次已经到了晋阳城下,可是从你镇守沁州之后,大雍再也不能踏上北汉的国土,你的功劳,王上知道,朝中群臣知道,本宗主知道,这北汉军民也都知道。大雍占据中原沃土,朝中名将辈出,当今雍帝李贽就是大雍军神,如今镇守泽州的齐王李显虽然不如乃兄高瞻远瞩,却也是当世名将,镇守泽州的雍军虽然只有三十万人,可是兵员充足,一旦有了损失,很快就可以补充上。而我北汉军虽然名义上有四十万,可是除了你这二十万全是精锐之外,其余的军队根本不可能调去助你。代州虽有十万军队,却是半军半民,抵御蛮人尚可,想要调动去对付雍军殊不可能,晋阳也有十万军马,可是还有负责北汉各地防务,你那二十万精锐已是竭尽全国之力,牺牲一人就很难补充。这样子的困境,若非你用兵如神,迫得大雍无力北进,只怕我北汉早已是国破家亡。你这一战虽然败了,可是巧妇难为无米之炊,也很难怪你的。”

龙庭飞神色惨然道:“都是末将没有看破他们的诡计,可惜了谭将军和无数战士。”

京无极苦笑道:“这也难怪你,别说是你,就是本宗,也没有料到那江哲竟有这样的胆量,竟然一个普通将领和你对峙,齐王如此信任江哲,这也是事先难料的事情,我们精心安排的流言又被大雍皇室所压制,谁会想到,一个娇弱的长乐公主,竟然就轻而易举的让许多地方官员稳住了心神,如今齐王和江哲取得这次大捷,今后要想再用离间,就是难如登天了。”

龙庭飞苦涩地道:“国师,虽然南楚拥兵东川,可是陆将军的说得很明白,若是想让南楚真的出兵并不容易,如今南楚上下几乎都寒了心胆,陆将军虽然心切一战,却是殊不可能。”

京无极牵着龙庭飞的手,将他拉到软榻前,示意龙庭飞坐下,悠闲地道:“有些事情,本宗已经经营许久,如今也应该告诉你了,本宗早知北汉的劣势所在,若是不能让大雍陷入内忧外患,我北汉根本没有取得天下的机会,所以这些年来本宗在南楚和蜀国都有安排,这次陆灿出兵东川,你以为是他一人决定的么,我魔门月宗一位师弟,如今已经是南楚军方领袖之一,虽然我们各事其主,可是这互利之事却是不会放过的。数年前我就已经和他联系上了,这次陆灿进兵东川,就是他的建议。虽然这一步棋不能改变什么,但是至少大雍不能悍然向泽州调兵,这样一来,你还有稳守沁州的把握。”

龙庭飞听得这样密闻,心中震惊,面上却不显露,道:“若是如此,弟子自信可以守住沁州,只是南楚军只能遥为策应,若是大雍下了狠心,泽州集结五十万军马还是可能的。

京无极笑道:“这个当然,南楚军虽然暂时不能出兵,可是等到局势变化之后,就是南楚朝廷不许,陆灿也不会放过良机的,这个先不谈。本宗在大雍内部安插的那根刺如今已经发挥作用了。庆王李康这次回到东川,立刻清洗了东川文武,将雍帝李贽的心腹全部软禁起来。若非不敢挑明叛旗,只怕早就将他们杀了。这件事情虽然大雍朝廷还蒙在鼓里,可是用不了多久,这庆王的反心就难以掩盖了。”

龙庭飞惊奇地道:“弟子曾听碧公主说过这庆王似乎和齐王不合,可是应该不会和李贽过不去吧,如今大雍朝廷新君已经坐稳了皇位,这个时候谋反可是有些古怪。”

京无极露出了意味深长的笑容道:“有件事情你不知道,庆王李康昔日得人传授武艺谋略,他心中对大雍怀恨极深,此子偏执桀骜,本就难驯,如今虽然名义上一人之下,万人之上,可是齐王在雍帝心目中的地位实际上却比他高的多,若非如此,此子或者会多隐忍几年,可是如今齐王眼看就要复爵,这李康就再难虚与委蛇了。不过此子心机倒也极深,他故意结好东原蜀国世家,笼络那些有心恢复蜀国的叛逆,他虽然是大雍皇室,可是凭着他的身世,居然使得那些人相信他和大雍皇室之间仇恨极深,这次雍帝后宫生变,就是这小子的诡计。他唆使司马氏送进后宫的贵女犯下大罪,然后迫使大雍皇室暗中杖杀那名妃嫔。为了庆王的面子,对外只说是此女暴毙,这样一来就给我庆王可乘之机,李康对对司马氏说大雍皇室不愿意接纳亡国之女为妃,故意残害其女,这样一来,故蜀世家心中怀恨,这次李康能够顺利掌控东川全局,也是这些世家襄助之功。如今雍帝李贽就算是得知此事,为了避免投鼠忌器,免得迫使李康索性勾结了南楚,也不敢轻易动手。这样一来,外有南楚、北汉为敌,内有庆王割据,大雍的局势可是不大妙啊。”

龙庭飞不由问道:“那传授庆王武功之人是谁,有没有法子通过他影响庆王,让他动作大些。”

京无极失笑道:“这倒容易,你去问凌霄吧。”

龙庭飞看了一眼段凌霄,见他微微含笑,目中闪过激动的神色,转而又有些苦恼地道:“国师果然高瞻远瞩,数年布局,今日才见成效,可是当务之急却是明春雍军恐会进攻沁州,现在南楚还在观望,庆王还没有竖起叛旗,我们若是首当其冲,只怕会损失惨重,就是胜了也难以得到什么好处。”

京无极叹息道:“这也是没有办法的事情,庆王虽然被我们影响,却也是因为他野心太大,若是让他现在反叛,等于是让他去送死,这种事情就是让他去做,也很难做到。南楚虽然有我们的人,可是毕竟上有国主丞相,还有陆氏父子权力大过他,他不可能做出更多的事情了,而且对他来说,南楚的利益才是最重要的,可是今次恐怕是最后一次遏制大雍的机会,若是让大雍脱出重围,一统天下就是指日可待。”

段凌霄插言道:“若是想阻止明春雍军出兵,只有一个法子,如今雍军北线主将乃是齐王,可是让北线稳如泰山的却是楚乡侯江哲,若是杀了此人,那么北线必然混乱,雍帝、齐王之间无人调艇,明春进攻必然外强中干,若是师尊允许,弟子愿意设法混入雍军,刺杀江哲。”

龙庭飞面上露出喜色,但是转念一想,无奈地道:“恐怕不行,碧公主说过江哲身边有一高手邪影李顺,段师兄虽然武功高强,可是此人有雍军相助,只怕师兄很难得手,若是失手,我们就再也没有机会,而且苏将军身死雍都,已经让龙某心痛万分,若是段师兄有什么损伤,庭飞万死难赎其罪。”

这时,秋玉飞突然起身道:“若是龙将军信任在下,玉飞愿意担此重任。”

段凌霄和龙庭飞都是大惊,秋玉飞醉心音律,武功虽然出色,却是比不上段凌霄,甚至还比不过常年疆场作战的苏定峦,如今正在军中效力的萧桐,他又是孤傲之人,这刺客可不是什么人都能做的。

京无极却是气定神闲,道:“玉飞可已经有了计策?”

秋玉飞道:“弟子已经想过,若是想要明刀明枪,恐怕弟子是不成的,那日和江哲比拼音律,弟子的号角被震断,自然是内力不如,可是那江哲却是靠别人的内力来和弟子比拼的,可见那人内力已经超过了弟子,就是大师兄去了,也是未必就有胜算,而且那人身在军中,身边甲卫如云,想要刺杀谈何容易,想来想去,只有混到那人身边才有可能寻机刺杀。我知那江哲乃是南楚才子,惊才绝艳,弟子也自负才学,我又听说那人爱才,今次那可以和龙将军交手的将领就是他推荐的,若是能够我进入雍军,凭着弟子的才学不难得到此人赏识,天长日久,等他戒心退去,弟子就可以从容杀之,如今天寒地冻,雍军困守泽州,正是最好的时机,数月时间,弟子或者能够完成使命,还请师尊许可。”

京无极凝神想了片刻,道:“也好,你如今对那江哲已经有了心结,若是能够将他杀死,应该可以回复你的心境,不过想要接近江哲并不容易,雍帝和齐王对此人都是十分爱重,不说他身边的邪影李顺,就是他身边的侍卫也都是雍帝亲自指派,想要接近他必须要有一个合适的身份,你的相貌身份虽然少有人知,可是想要顺利接近江哲,恐怕不易,三月时光,转瞬即逝,不能轻易浪费。”

秋玉飞微微蹙眉,这一点他的把握也不是很大,这时段凌霄道:“师尊,请让弟子来安排这件事情,弟子恰好有一个合适的身份让师弟借用。”

京无极知他稳重,也不多问,笑道:“既然如此,这件事情就交给你们了,虽然说刺杀不算是什么好计策,可是这个江哲乃是大雍皇室的女婿,又是雍帝心腹谋士,杀了此人,是一本万利的好事,你们不可不慎。”

秋玉飞正色道:“有大师兄相助,弟子一定可以得手,若是不然,弟子情愿身死以殉。”

京无极、段凌霄和龙庭飞都是眉头一皱,他们都从秋玉飞的话语中听出了不祥的征兆,段凌霄和龙庭飞同时看向京无极,眼中透出征询之意。京无极心思百转,终于说道:“你要小心行事,不可轻捐性命。”说罢转身走到栏边,负手望着天边寒云,心道,这也是他命中劫数,若是不能解脱心魔,终身难以寸进,不如一死也罢。

龙庭飞心中又想起一件事,道:“国师,弟子还有一件事请国师指点。”然后缓缓讲了那封密信的事情,他这次回到沁州,特意让萧桐留心属下将领的动静,可是这几日细思,总觉得似是而非,所以终于向京无极请教。

京无极犹豫了一下,却没有回答,半晌才道:“这一点本宗也无法答你,不过本宗不妨直言,白首相知犹按剑,本宗是绝对不会轻信任何人的。可是你是带兵的大将,若是疑心太重,恐怕会伤了属下之心,若是太过轻信,本宗又担心你被人出卖,这件事情,你不妨和王上商议一下吧。。”

龙庭飞听了心中一阵迷茫,竟然不知究竟该如何才好了。

离了兰台,龙庭飞想到自己这次回晋阳,只是和王上匆匆见了一面,理应前去述职才对。内侍通禀过后,后主刘佑在书房召见。

走进书房,一看到后主刘佑,龙庭飞只觉得心中一痛,还不到五十岁年纪,刘佑却已经是头发斑白,若非是面上仍然神采奕奕,哪里还有昔日的英姿雄风。龙庭飞上前拜倒,哽咽道:“末将有负王上厚爱,请王上重重治罪。”

后主轻轻一叹,伸手将他搀起,道:“龙卿乃我北汉栋梁,孤焉能随便治罪,胜败乃兵家常事,你不要放在心上,新年之后你就要回沁州镇守,孤望你不要有什么顾虑,尽力作战就是,我北汉立国二十三年,可我刘家裂土封侯却已经将近七十年,自问无负百姓。其实如今国士日衰,孤焉有不知道的道理,可是孤不能眼看着刘氏江山落入人手,只能累你呕心沥血了,龙卿受孤一拜,如今已是生死存亡之秋,孤将全国兵力托付于你,若是你不幸兵败,孤自会自尽以谢臣民。”

龙庭飞泪如雨下,匍匐在地,再也不能掩饰悲声,心中却再也不大算提及麾下将领或有叛逆之事,王上已经为国事如此忧心,他不忍再提,心中却是拿定主意,就是错杀一千,也不能放过一个叛逆。

君臣商议几句之后,龙庭飞正要告辞,后主却笑道:“还有一件事情,你和碧儿的婚事已经拖了很久,不如你们新年之前完成大礼如何?”

龙庭飞沉默半晌,道:“如今敌军压境,臣不愿落人口实,还是等到国事稍安之后再议吧。”

望着龙庭飞的背影,北汉主不由叹息道:“龙卿也未免太求全责备了,罢了,这些儿女之事孤也不便过问,碧儿,你说呢?”

屏风之后闪出林碧的身形,她黯然道:“庭飞心系国家大事,碧只有心中敬佩,只望他取得大胜,从此不再为泽州败绩耿耿于怀才好。”

北汉主也是叹息不已,望着神色有些憔悴的甥女兼义女,一个念头突然涌上心头,我这般苦苦挣扎,只为了保住自己基业,却让这些孩子这般痛苦,是不是有些自私呢?

\chapter{第二十五章 杀人灭口}

第二十五章章名修改,上半章做了小小的修改,所以这次将整章补发。

—————————————————

“阿嚏。”李虎打了一个大大的喷嚏,愤然的看向负手站在营帐前面观看雪景的江哲,再次痛恨自己怎会这般软弱,冒着大雪给敌人守卫,忍不住伸手向腰边摸去,还没有碰到刀柄,身后就传来一声轻咳。他愤然回头望去,只见凌端站在那里似笑非笑地望着自己,看到自己回头,凌端撇撇嘴,示意李虎留意一下站在不远处的几个虎视眈眈的卫士,李虎泄了气,随便一个虎赍卫都可以将自己擒拿,想要刺杀江哲真是自寻苦吃。

凌端看看李虎垂头丧气的背影,不由苦笑,自己又何尝不是身不由己呢,想到这里忍不住摸摸腰间短戈,继续琢磨如何能够刺杀江哲成功。

李虎和凌端两人的一举一动我都看得清清楚楚,忍不住唇边露出一丝笑意,收服两人的情景再次浮现在眼前,虽然这两人仍是心不甘情不愿,可是这无关紧要,只要能够达到我的目的,也就足够了。

营帐之内,李虎古怪的望着青衣少年,虽然他有些鲁莽,可是并不是白痴,这人今日流露出的冷厉气息让他浑身不舒服,忍不住道:“喂,今日谁给你气受了么,怎么脸色这么难看?”

李顺眼中闪过一丝杀机,道:“多日相识,两位想必还不知在下的身份,在下李顺,乃是楚乡侯家仆。”

凌端心中早有预料,只是微微苦笑,这时李顺有意无意地扫了凌端一眼,冰冷的目光让凌端心中一凛,想要提聚真气,可惜伤重初愈,根本无法行功,只得颓然坐倒。

李虎目光茫然,半晌才明白过来,道:“原来你小子是那个监军的属下,我就说么,怎么可能有人无缘无故这么好心,不过老子奇怪得很,你的主子若想杀我报复,当日一刀斩了老子就是,为什么这么麻烦,还要等到老子伤愈再动手。”

李顺神色越发冰冷,道:“我家公子身份不同寻常,多年来在下一手负责公子的安全,可是竟然让你在我的眼皮底下几乎伤了公子性命,这种奇耻大辱怎可不报,而且若是轻轻放过你,岂不是让他人以为我李顺好欺。李某生平最喜以牙还牙,可是当日你被俘之时,心存死志,我若是那时杀了你,平白让你快意,因此我令人替你治伤,对你倍加礼遇,等到你不想死了,我再杀你,这样才称我心意。不过一刀断首,却还是便宜了你,所以我给你两个选择,第一个选择,我给从北汉军俘虏中选出勇士,让你与他决斗,胜者生,败者死,你若能多胜几场,自然是可以多活几日。第二个选择,我为你准备了种种酷刑,你若能一一捱过,我就放你离去,你若是熬刑不过,自然是一死了之。”

李虎听得背脊直冒寒气,这两种死法可都不是什么好选择,不过他倒是颇为硬气,倔强地道:“老子既然落在你的手上,你要杀就杀,老子可没有闲心和你游戏,不过自相残杀老子是不会做的,你要动刑就动刑好了,看看老子能撑多久。”

小顺子微微一笑,笑容中带了一丝残忍的意味,正要说话,凌端却抢着道:“笨蛋,你若想死得痛快些还是选决斗吧,最多第一场就自己撞上对手的兵器,死得也算是痛快些。若是人家动了刑,等你求生不能求死不得的时候,不免哀告求饶,到时候将你带出去示众,你就是死了也是声名扫地。”

李虎听得如同身坠冰窟,可是却也有些不服气,道:“你怎知我不会熬刑而死,却会做出那番丑态。”

凌端苦笑,心道,我在将军身边多年,慷慨赴死容易,从容就义却难,就是钢浇铁铸的汉子,在酷刑之下也难以挣扎多久,将军也是善于用刑之人,一旦动了大刑,受刑之人不是寻机自尽,就是屈服求饶,熬刑而死的已经是千里挑一,能够熬刑到底的人我可还没有见过。虽然想多说几句,可是这时,李顺冷冰冰的眼神已经飘了过来,凌端也没有勇气再次提醒那只呆头呆脑的老虎,别过脸去,心道,你若不明白我也没有办法,我可不想生死两难。

小顺子眼中闪过恶意,心道,这凌端真是多事,要不要将他一起捎上呢?

李虎这下可明白了敢情两个选择不过是假相,面前这人就是要让自己死得痛苦屈辱,但是他生性不肯服软,反而笑道:“原来如此,你小子真是不地道,就连杀人也不愿给人一个痛快,老子多活这些日子也是赚到的,你想怎样处置就怎样处置吧。”说罢跳下床来向外走去,一边走还一边嘟囔道:“反正老子家中无亲无眷,就是留了污名又有什么关系。”

小顺子倒是一愣,他原本心想李虎会改变主意,求一个痛快,还在盘算如何及时出手,不让这李虎死得容易,可是李虎却还是选择了更痛苦的死法,只为了不愿同僚相残,这样一来,倒是让他有些过意不去。可是无论如何,这人在他心中已经是必死之人,他又是冷面冷心之人,转身便要出去安排。凌端却终是心中不忍,道:“这位兄台,沙场之上,生死乃是常事,贵长上如今春风得意,我们这些人却是阶下之囚,你们自然是可以随意处置,可是拖到今日来算旧帐,是否有些过分呢?”

小顺子停住脚步,回头看了凌端一眼,道:“你是谭忌将军近卫鬼骑,在下对谭将军颇为敬佩,所以就不计较你多嘴多舌了,不然我就让你和李虎同罪。谭将军为了己身之恨,屠杀泽州军民无数,这些人原本还是无辜的,想必你也没有劝阻过,这李虎险些伤害公子性命,此事焉能容忍,你说在下睚眦必报也好,说在下狠毒也好,这人却是一定要杀的。你还是顾着自己性命要紧,谭将军灵柩已经送去北汉,自然不会有戮尸之祸,至于你,若非齐王殿下宽宏大量,早就被千刀万剐以谢泽州军民了,还有心替别人抱不平么?”

凌端愕然,却不是为了这人嘲讽自己,那人分明说对将军心存敬佩,这怎么可能,别说是雍军中人,对将军理应只有仇恨,就是北汉军中,除了自己这些将军的直属部曲之外,其他将领军士对将军也都是忌惮不满得很。

这时,放慢脚步偷偷听完两人交谈的李虎知道自己终究不能幸免,有些垂头丧气地走出帐去,他倒是性情直率,也没有作出视死如归的姿态。谁知刚刚走到帐外,就看到不远处站了一个青衣书生,披着大氅,身后侍立着黑衣虎赍侍卫,李虎虽然当日只是匆匆看过江哲一眼,可是只看这样的架势,就知道来人身份,不由冷笑道:“原来是监军大人要亲自动刑啊,这样一来我李虎就是死了也是值得的,不过想起那日大人那副落汤鸡模样,想来还真是好笑得很。”说罢大声笑了起来,他却是想激怒江哲,最好惹他怒火上冲,一刀砍了自己最好。

小顺子这时也正在步出帐门,一眼看到含笑而立的江哲,不由惊叫一声,凭他的武功,本来不会忽略外面有人窃听,可是军营之中人来人往,江哲方才所站的的距离稍远,却被小顺子当成了无关之人,再说他也没有想到江哲会显身这里,虽然距离尚远,可是深知江哲底细的小顺子却知道自己方才所言已经都被听见,不由面红耳赤,上前呐呐道:“我不是想欺瞒公子,实在是记恨此人,还请公子恕罪。”

李虎本是义愤填膺,可是刚说了几句狠话,只见那青衣人目光温和地望着自己,不带丝毫恶意,甚至还带着几许激赏,不由有些手足无措,心道,想杀我的是那个李顺,也不关他的事,我这样恶言恶语是不是有些过份了。他有些赧然的站在一边,偷眼向江哲望去,怎么看都觉得这个青年瘦弱可怜,想起当日自己飞槊将他击入水中,当时只觉得意兴奋,如今想来却觉得有些惭愧,自己自负勇力,怎么却对一个手无缚鸡之力的书生下杀手呢?

他这边愣着,凌端在帐内听见“江哲”二字,不由心中一动,他已经知道都是此人计策,才让将军中伏而死,怀恨之余倒也想看看此人如何形貌。因此勉力出帐,凝神看去,虽然觉得江哲气度不凡,却也不是心中所想那种精明模样,虽然身在军中,又是高官侯爵,这人仍然是一领青衣,唇边含笑,目光柔和,行动举止中透着安谧宁静的味道,令人一见之下便生出可亲可近的念头。凌端不由茫然,这人就是害死将军的罪魁祸首,为何自己却竟然生不出一丝杀机呢?

见这三人都是一副尴尬模样,我不由摇头轻笑,虽然深知小顺子的脾气,不过见他报复之前还要事先想好将来如何搪塞我的借口,我虽然有些气恼,更多的却是觉得感动和好笑,看一看站在那边发愣的李虎,这个人从来没有被我放在心上过,当日落水,我只记恨齐王嘲笑,根本没有想过还要报复这人,想不到小顺子却是私下动了手,若非是被我撞见,大概这人就是到了黄泉也要诅咒我吧。另外那人我虽不认得,但是见他小小年纪就是神色如冰,杀气冲天,只是面色白皙,似乎少见阳光,倒是少见的英才。又看了那少年一眼,我看向李虎,笑着问道:“原来就是你送我去洗了一个冷水澡,不知这位是谁啊?”

凌端见我问他,偏过头去,不愿回答,小顺子冷冷看了他一眼,道:“这人叫凌端,乃是谭忌麾下的鬼骑。”

我动容道:“早闻鬼面将军身边的鬼骑勇猛,想不到你小小年纪就有这样的本事,真是难得,难得。”感慨了一番,我正想婉言劝解小顺子不用再理会李虎的时候,心中突然生出一个古怪的想法。

当日我在故意被北汉军截取的书信中提及他们后方有高级将领有心投降,可是并没有影射特定的人,至于其后如何加重龙庭飞的疑心我全部交给齐王去做了,只是给了一个原则,不要厚此薄彼,最好是人人都有嫌疑,人人都像是叛逆才好,免得诬陷错了人,让龙庭飞醒悟过来。可是看到这个军士,我却突然想到,相比其他将领,石英实际上是最合适的人选,当日他率军截杀我和齐王,能够侥幸逃生实在是运气居多,想必北汉军中也有人疑心吧,若是说石英在追杀我们的时候留了手,也说得过去,虽然我是很想先铲除段无敌的,可是石英比较粗心,似乎更容易落入陷阱。再想到我近日得到的情报,谭忌生前和石英十分不合,这次石英负伤留在沁州,谭忌带兵却是中伏而死,若是谭忌的亲卫说石英有谋反之心,只怕龙庭飞怎也会信个三分。想到这里,我又看了凌端一眼,不知道小顺子是有心还是无意,让他们住在一起,这样一来,我反间成功的可能性就更大了。

不过这件事情不能急躁,当务之急先要把这两人留在身边,否则怎有机会让他们知晓那样的“机密”呢?想到这里,我微笑道:“天气寒冷,总不能在外面叙谈,进去吧。”说罢,便向帐内走去,小顺子飞快的站到我身边,防备这两个俘虏向我行刺,其实不说他们伤势太重,难以行刺,有小顺子在我身边,就是他们完好无恙,也休想得手,所谓履险如夷,实际上多是心中明白没有危险罢了。

走进营帐,我拣了一张椅子坐了,李虎和凌端慢吞吞的走了进来,有些不情不愿,也有些好奇。

我将这两人又仔细打量了半天,方笑道:“小顺子,你也未免多事了,过几日他们伤愈之后,就要被送到苦役营,到时候外有重兵环卫,手无寸铁可恃,日日辛苦劳作,这两人都算是武艺出众之人,恐怕还要戴上脚镣,就是想要脱身都难,这些人都是俘虏,就是我们灭了北汉,数年之内也别想恢复自由之身,这般苦楚已经足够,你又何必还要寻机报复呢?”

李虎和凌端听了虽然黯然,却也知道按理应是如此,而且按照两军交战的规矩,像他们这种重伤的士卒,身份又不高,恐怕都会被打扫战场的敌军直接斩首,作俘虏也是轻伤的将士以及身份重要的将领才有这个资格的。就是成了战俘,像他们这种重伤,普通的军医也是无可奈何,恐怕是活不了多久的。说起来,李虎能够活到今日,还是因了小顺子想要报复而令人给他精心医治的缘故,而凌端则是受了谭忌的余荫,齐王特意下令命人救治,这才保住了性命。

我看到两人神情,心中生出一丝怜悯,人生来都是好生而恶死,这两人也是如此,若是为国捐躯,或者面临难以忍受的屈辱,想必他们不会贪生怕死,可是如今成了俘虏,如果没有什么意外,自然也是想活下去的,虽然如此,我却不会因此而轻视他们,若是我想迫他们归顺大雍,出卖北汉,那是绝无可能的事情,不过利用死亡的压力迫使他们暂时放弃一些尊严,应该还是可能的。

想到这里,我露出自认是十分诚恳的神情,道:“李兄,江某失察,至令你险些受辱,此事虽是下人胡为,却也是江某管教不严,作为补偿,李兄可愿暂时留在江某身边执役,等到战事结束之后,李兄就可自由离去。如果是江某本心,自然是想就此放了李兄,可是李兄也应该明白江某身为大雍监军,有些事情是不便做的,不过李兄也可以放心,江某身边的人多半不需要上阵杀敌,也不会让李兄和昔日同袍为难,不知道李兄可愿接受江某的好意。”

李虎瞪大了眼睛,说句实话,江哲的提议确实十分诱人,除了不够自由之外,几乎可以说是极为优厚,可是李虎刚刚受了教训,可不相信世上会有这样的好事,再说,这样算不算叛国投敌,李虎也盘算不明白,所以一时之间竟然不知道如何回答才好。

我又看向凌端,道:“齐王殿下对谭忌将军颇为推崇,江某也十分遗憾没有机会见到谭将军,凌少兄是谭将军麾下鬼骑唯一幸存之人,爱屋及乌,殿下也是不想留难,可是国有国法,军有军规,现在凌少兄也不能自由离去,殿下乃是皇室贵胄,三军统帅,不便留凌兄在身边,故而曾托江某照顾,若是凌少兄不介意,不妨也暂时留在江某身边如何?”

小顺子站在江哲身边,神色如冰,却是几乎笑出声来,什么时候齐王殿下托过公子来着,完全是公子信口雌黄呢,不过他是聪明人,见江哲这般神情,就知道必是又动了什么心思,自然不会拆台,反而故意流露出不满之色,道:“公子,您虽然答应过殿下照应凌端,可是凌端毕竟是敌人,将他们拘在营里也就是了,何必留在身边,若是这人忘恩负义,行刺公子该如何是好,还有这个李虎,公子不怪罪他已经是他的福分,何必还要留他在身边呢?”

他言辞中虽然满是不赞同,却是更加支持了江哲的说法,让李虎和凌端都觉得江哲确是一番好意。可是李虎和凌端两人却都无法答应,虽然留在营中绝对比去做苦役舒服多了,而且还可以很快就得到自由,不论大雍和北汉的战事如何,他们两个总能找到机会脱身的,可是会不会因此而一失足成千古恨,从此成了叛徒贼子呢?两人心中十分不安,原本除了为了打发无聊根本不愿意互相交谈的两人交换了几次眼色,可惜一个太粗心,一个不大擅长表示,险些成了闹剧,半天还是无法决定。

我心中觉得好笑,却也知道想要他们明确答应,是不可能的事情,利诱完了,自然该威逼了,便故意忽视他们拒绝的可能,道:“既然两位都不反对,小顺子,你安排一下,就让他们两个住到虎赍卫的营里,等到他们伤势再好一些,就让他们到帐前听用吧。”

说罢我也不看他们一脸不情愿,甚至准备拒绝的表情,三步并做两步,带着侍卫匆匆走了出去。李虎最是性急,大声道:“等一下,老子不……”话刚出口,却硬生生咽了回去,因为他看到小顺子挡在他前面,面上带着古怪的笑容,白皙的右手不知何时已经按在自己的肩头,一缕冰寒的真气从肩井袭入,李虎只觉得浑身冰寒,一句话也说不出来。凌端心中一寒,他清晰地看到了小顺子眼中淡淡的杀机,不由惊叫道:“贵上不是已经放过李虎了么?”

小顺子眼中闪过一丝犹豫,过了片刻,放下手道:“公子既然已经决定,我也无话可说,不过你们若是想要拒绝,我就立刻杀了你们两人,最多给公子责备几句,公子如此好意,你们若是不领情,就是不知死活,我杀了你们也不为过。”

两人心中都是剧震,这种情况下被杀,可真是有些划不来,凌端一咬牙,心道,若是我留下来说不定哪天可以杀了江哲,到时候就是死也值了,也顾不上这样的想法是否只是一种借口,凌端恨声道:“在下愿意从命,老虎,你呢?”李虎这时候也聪明起来,竟然看出了凌端的暗示,便粗声粗气地道:“我也是。”

小顺子眼中闪过一丝不可遏制的杀机,这一刻他真的有些愤怒,这两人盘算着什么他一眼就看了出来,让两个心存异志的人留在公子身边不是他所愿意的事情。可是他不得不强行抑止心中怒火,他明白这两人有这样的想法并不奇怪,这也是公子敢肯定他们会屈服的一个原因。走出营帐的时候,小顺子心中暗暗冷笑,可惜这两人太天真,人性是很古怪的,屈服一旦成了习惯,就会逐渐放弃自己的坚持,不论他们真心假意,这次的屈服都会让他们渐渐放弃仇恨和反抗的勇气,不过在他们彻底屈服之前,还是要时刻小心的,只不过虽然还不甚明白公子的计划,这两人只怕还没有屈服,就被公子彻底利用了。

自从那日之后,李虎和凌端就被迫换上了雍军衣甲,成了监军江哲身边的亲卫,两人心中无时无刻不想着刺杀江哲,只因若是不这样想,便会想起当日被江哲的“甜言蜜语”和小顺子的“威胁”胁迫而屈服的情景。可惜,并没有很好的机会,江哲虽然性子亲切疏懒,对待两人也似乎全无戒心,可惜他身边的侍卫却是小心翼翼,两人别说刺杀了,就是碰一碰兵器也会召来十几道目光的注视,更别说那个小顺子几乎总是在江哲身边,冰冷的目光仿佛虽然都可以穿透两人的心脏。说到这一点,两人就更加想不通,虽然留在江哲身边执役,但是江哲居然命人给了他们兵器,就连李虎都私下里说,这个监军大人是不是一个滥好人啊?这一点凌端倒是不会这样以为,至少每次齐王殿下来和江哲商议军务的时候,自己两人都会被隔离开去,看来这江哲并非没有戒心。不过这样一来,凌端倒是放下了心事,他不是白痴,跟着将军几年,也知道一些兵法,若是那江哲摆出完全信任自己的架势,凌端倒要认定江哲必然存了恶意呢。

李虎又是一个大喷嚏,按理说他是北汉人,常年生活在更加寒冷的沁州,本不应该如此容易受寒,可惜他如今是重伤初愈,元气大伤,自然是容易生病,倒是凌端虽然年轻,内力却练得精深,如今已经基本上行动如常。

这时,空中又开始飘下雪片来,那个两人最是忌惮的小顺子走到江哲身边,道:“公子,下雪了,还是回帐休息一下吧。”

凌端搓搓有些冰冷的双手,侧耳偷听江哲的回答,这样冷的天气,他也很想早些回去烤火呢。远远的从风中传来话语声道:“后日就是先父忌辰,可惜我飘零在外,无法回去上坟,你可知附近有什么寺院么,能够到佛前告祭一番,也是好的。”小顺子犹豫了一下道:“公子,离此六十里有一座万佛寺,本来是座大禅院,后来北汉军数次入侵泽州,这座佛寺才荒废了,近来我军大胜,泽州境内百废待兴,万佛寺也有了僧人主持,应该可以做法事的,而且泽州百姓都相信北汉军从此不能侵入寸土,所以从这里到万福寺沿途都已经渐渐有了村寨野店,而且入冬之前,道路也经过整修,公子若是前去,应该无碍,不过这几日连场大雪,恐怕路也不会太好走。”

刚听到这里,远处突然传来爽朗的笑声,道:“随云,顶风冒雪,何其自苦也。”凌端望去,却是齐王身穿便装,冒雪前来。

江哲也看见了齐王,却是一连的不愉快,道:“殿下到此,想必是又有军务,哲不过是个监军,殿下也不用事事和哲商量吧?”

齐王笑道:“由你这样的人才,本王若是不懂得利用,岂非太愚,本王确实有事情和你商量。”说完,扯着江哲向营帐走去。

凌端看了看李虎,耸了耸肩,一起向两人居住的营帐走去,一旦李显到来,都会有人让他们回去营帐休息,所以这次两人根本就没有等待命令,直接就准备回营。还没有走出几步,却看见齐王身边的近卫庄峻匆匆走来,凌端站住脚步,他是认得庄峻的,当初被俘,庄峻替齐王探视过他的伤情,所以凌端准备和他打个招呼。

庄峻看到凌端停住脚步,心中一喜,几步走到他面前,微笑道:“凌端,我有件事情和你说,让他先回去吧。”李虎听见他的说话,也不多言,便留下凌端自行回去了。凌端觉得有些奇怪,问道:“庄侍卫,有什么事情么?”

庄峻神色肃然道:“凌端,你一直和李虎住在一起,有没有听过他说起截杀殿下的事情?”

凌端有些茫然,道:“听他说过,不过他说得不大明白。”

庄峻面色更加深沉,道:“你都知道些什么?”

凌端心中一凛,戒备地道:“我知道的不多,只知道他们分兵追击,最后在固山寨被击败,怎么败得他都没有看见,所以我也不清楚,只听他说过监军大人落水的事情。”他并没有隐瞒,这些事情恐怕庄峻比自己知道的多得多。

庄峻似乎松了口气,笑道:“既然如此也就算了,好了,你我多日不见,趁着殿下和监军大人商议军情,我们聊一聊吧,你最近过得如何?”

凌端心中一动,见庄峻有意无意地望向自己居住的营帐,一个念头突然浮现,庄峻想将自己拖在这里,又问自己李虎都说过什么,莫非有些什么关碍,他心中一急,也顾不上和庄峻敷衍,转身向营帐跑去,却见两个齐王侍卫挡住了自己的去路。凌端一横心,短戈划出,虽然他伤势初愈,力道不足,可是习自谭忌的戈术果然不凡,不过三招两式,一个侍卫被迫错开了一步,凌端冲向营帐,此时庄峻喊道:“让他去吧。”

冲回营帐,凌端一眼看到李虎委顿在地,两个齐王侍卫正拖住李虎,准备出帐,凌端心中大惊,虽然知道无益于事,却还是挡住两人,手中短戈微微发抖,他很清楚,如果真是齐王要杀李虎,自己是无法可想的,可是这些日子,凌端早就将谭忌和石英的恩怨放到了一边,按照他的想法,石英再讨厌,也不关李虎的事情,这样一个直爽的汉子,让自己眼睁睁看他死去,心中怎忍。

这时,庄峻带着几个侍卫缓步走了过来,两边营帐居住的虎赍卫也都围拢过来,好奇的看着这古怪情景。

庄峻叹息道:“凌端,李虎的事情和你并不相关,齐王下了军令,他也在斩首之列,你还是不要过问了。”

凌端神色变得狰狞,气息渐粗,紧握短戈道:“我们本是俘虏,生死不能自主,你们自然是要杀就杀,不过想要带走李虎,就先杀了我吧,反正我早就想着随谭将军而去。”

庄峻冷冷道:“你想救人,就先过我这一关吧。”说罢一掌向凌端击去,凌端奋力还击,两人交手十数招,凌端已经气喘吁吁,又过了数招,便给庄峻一掌击倒。庄峻叹了一口气道:“今次的事情,我就当没有发生过,你回去休息吧。”说罢一挥手,两个侍卫拖着半昏迷的李虎向外走去。凌端眼眦欲裂,却是无法起身,他毕竟年少,两眼中居然有些雾气朦朦。

这时,一个虎赍卫脸色铁青,上前阻拦道:“庄侍卫,此来可有监军大人令谕,这两人乃是大人亲自收留,若无令谕,请恕我等不能任你们将李虎带走。”

庄峻拱手道:“殿下正在监军大人营帐,此事事关重大,大人必也不会阻拦。”

那个虎赍卫冷然道:“我已派人去通知大人,若是大人下了命令,我等自然不会过问。”

这时,一个虎赍卫从江哲的营帐匆匆跑来,在这人耳边低语了几句,凌端隐隐听见,那人说道:“截杀……不可外泄……杀人灭口。”虽然断断续续,可是凌端心中已经明白,看来李虎是因为某些机密之事,而被列入需要灭口的名单了。是什么事情,连这样一个小人物都要灭口,方才庄峻含糊的问话再次回响在脑海里。眼睁睁的看着李虎被带走,凌端心中剧痛,只觉眼前一黑就昏迷了过去。

\chapter{第二十六章 雪影杀机}

初,武威二十七年丁丑,太宗继位,高祖退位,尊为太上皇,以高祖尚在,下诏沿用武威年号。

年末,百官上书请更年号,以彰圣德,太宗许之。

——《雍史·太宗本纪》

隆盛元年戊寅,正月初七,雪后初晴,寒冷非常,十五之前,百业消停,路上更是行人寥寥。官道旁一座小小的野店却是酒旗招展,掌柜胡三往火炉中又加了几块木炭,无精打采地倚在柜台旁边打盹,这一个新春过的十分平顺,自从齐王在泽州大捷之后,泽州没有了明显的外患,从各地归乡的旅人络绎不绝,他的生意极好,本打算等到明年春天好好修修这座破落的店房,谁知初一去赌场玩耍,赌神菩萨不肯保佑,输掉了大半银两,老婆一气之下回了娘家,胡三后悔莫及,却又拉不下脸来去接妻子回来,只好愁眉苦脸地提前开业,希望能够碰上几个出手阔气的客人,或者还能赚上几两银子,好去讨老婆欢喜。

正被炉火熏得昏昏欲睡,突然耳边传来响亮的马蹄声,胡三精神一震,也顾不得彻骨透过来的冷风,推开店门向外看去,只见北面积雪飞扬,十二名骑士护着一辆马车奔来。胡三拼命看去,不多时,那些人已经接近数里之外,其中一骑脱众而出,快马加鞭,转瞬间飞马到了门前,马上的骑士用马鞭指着胡三问道:“有好酒么,店内可有闲人?”

胡三谄媚地道:“客官放心,小店的酒远近闻名,浓烈香醇,店内没有客人,就连一个小伙计也回去过年了,小店干净暖和,大爷在这数九寒天走远道,不妨进来喝上几杯,保管您舒坦。”

那个披着黑色大氅的骑士将风帽摘去,露出一张刚毅彪悍的面孔,他翻身下马,也不理会胡三,向店内走去,站在门口,看见里面十分宽敞,虽然桌椅简陋,却是颇为干净,满意的点点头,道:“我家大人要在这里打尖,你要好生伺候。”

胡三眼尖的很,早在骑士翻身下马的时候,就已经看清楚大氅之下乃是质地精良的黑色骑装,上身更穿着精美的黑色软甲,腰间佩着横刀,只看刀鞘就知道不是凡品,再加上足上的战靴,不用问也知道这是军中的将爷,再一听他有位大人要好好伺候,胡三心中大喜,来的既是达官显贵,那么只要自己伺候周到,银钱必然是不会少给。他十分利落的道:“将爷,小店后面的马棚宽阔得很,牧草都是上好的,小人去生上火炉,保管将爷的马匹不会受寒。”

那骑士挥手道:“快去吧,一会儿把好酒好肉都拿上来。”

这时,其他的人也已经到了,这个骑士快步走到马车前面,禀报道:“大人,里面可以打尖,请大人示下。”

马车里面传出来一个清朗的声音道:“路途辛苦,我们休息一个时辰,不过酒不能多喝。”那些骑士高声应诺,纷纷翻身下马。其中一个骑士从马上抛下血淋淋的野味,道:“掌柜的,马匹我们自己料理,你把这些野鸡兔子精心做几个小菜,给我家大人送上来。”胡三连连答应。

这时驾驶马车的青衣少年跳下车来,然后掀开车帘搀下一个青衣书生来。两人在胡三殷勤的引领下进了店堂,选了一张背风而又温暖的桌子坐下。而那些骑士迅速的将马车上的骏马和那些骑士的坐骑牵到马棚,也不用胡三插手,就连草料也是他们自己取用的。然后留下一个骑士在马棚守卫之后,其他的骑士才进了店堂,向那青衣书生见礼之后,才四散坐下。

胡三动作极快,这会儿功夫已经将准备好的熏肉大饼和烧酒摆满了桌子,胡三忙得满头是汗,不过看到那些护卫的将爷都是满面的满意神色,不由心中高兴。又过了一会儿,胡三用客人带来的野味做了几个小菜端到那青衣书生的桌子上,偷眼一看,只见那青衣书生面色微红,似乎是喝了几杯酒,不过自己送上来的熏肉却是几乎没有动过。而且他喝的酒也不是自己店内的烈酒,不知什么时候,桌子上多了一个青花瓷坛,以及一只似玉非玉,不知是什么材质的古朴酒觞,里面盛着澄碧色的美酒。除此之外还多一个食盒,里面装着一些精美的点心,食盒外面套着厚厚的毛皮,糕点上面仿佛还冒着热气。

胡三将野味放到桌子上,那坐在一边的青衣少年从身边的另一个盒子里面拿出银质的碗筷,放到那书生面前,对每一道菜都尝了一尝,才道:“公子请用。”

那青衣书生这才开始用餐,胡三看得瞠目结舌,他虽然也算是见多识广,但是毕竟只是守着一家小野店,还没有见过这种排场。

忙乎了大半个时辰,胡三终于闲了下来,那些骑士早就风卷残云一般将酒肉一扫而空,然后就慢条斯理的喝着酒低声聊天。而那个青衣书生用餐之后,则是拿起一卷书册看得入迷,胡三知道这些人大概还得休息小半个时辰,连忙又去捧了两坛酒过来,其中一个似乎是为首的骑士摇摇头,道:“不用了,若是喝醉了就不好赶路了,你把我们的酒囊都灌满吧。”说着将一个酒囊丢到桌子上,其他的骑士也都纷纷解下腰间酒囊放到桌子上。胡三一边灌酒一边盘算,每个酒囊至少能装两斤酒,只算今日的酒肉,就已经是笔大生意了。装完之后,胡三一算,却是只有十一个酒囊,心中奇怪之余不由偷眼望去,原来有一个骑士一开始就坐到角落里面,也不和其他的骑士坐在一起,胡三几乎忽略了他,一留神之下,才发觉那人竟然是一个十七八岁的少年,桌上的酒壶原样未动,竟然是滴酒不沾。胡三心中奇怪,北地严寒,人人都爱烈酒,怎么这个少年骑士竟然不喝酒呢,又多看了几眼,那个少年骑士似乎察觉到了他的目光,冷冷的望了他一眼。胡三只觉得心头巨震,那个少年神色冰冷,目光中更是带着逼人的杀气,胡三虽然不是军人,却也是在战乱中挣扎多年,那种目光他明白的很,那是一种带着刻骨仇恨和疯狂杀机的目光。

我缓缓的饮下清淡的美酒,过于醇厚的烈酒我可是消受不起的,说来也是有些惭愧,前些日子我想着父亲忌辰将到,想到万佛寺告祭,可是谁知还没有成行,朝廷就来了使臣,犒赏三军,我这个监军自然也脱不开身的。好容易过了新年,我才有了时间,也顾不上还不到十五,就带了小顺子和几个亲卫往万佛寺而去。齐王殿下倒是也想陪我去看看,却被我婉拒了。眼光掠过那暗处角落里面孤寂的身影,我心头一阵苦涩,可惜啊,就是简单的告祭亡父,我也不能不用上心机,这次特意带上凌端,就是要给他一个逃跑的机会。

多日前的剧变,李虎被齐王属下强行带走之后,凌端就变成这个样子,沉默,冷淡以及仇恨,可是这件事情我也是无可奈何,我不可能故意让他看见什么文书情报,这样子容易就是白痴也知道其中有诡计,只有这个法子,让凌端得知石英的旧部全部灭口的事实,这样等到他回到北汉,配合其他的事情,就会想到石英“背叛”的可能,这是我的计划中很重要的一步棋,想要铲除石英,这是必不可少的证据。

龙庭飞麾下将领之中,苏定峦、谭忌已经死了,只剩下石英和段无敌,我决定目标盯准石英,是因为段无敌善守,行事谨慎,必然是个精明人,而对于精明的下属,上位者可以倚重,却很难信赖,再加上我们得到的情报,石英的确是龙庭飞的爱将,这样一来,对付石英不仅是离间了龙庭飞的心腹,而且亲信的背叛也会更加严重的打击龙庭飞的信心。为了这个原因,我也不能顾惜凌端的心情了。

看着凌端,心中突然想起谭忌,齐王曾经将谭忌临终时候吟唱的一曲歌辞抄录给我,我吟诵再三,想起谭忌平生,也不禁深深叹息,这首歌辞虽然过于悲伤悒郁,却也是心血写成。在心中念诵了一遍,突然站起身来,向店外走去。

负责护卫江哲的呼延寿惊讶起身,正要动问,随后跟出的小顺子却一摆手道:“公子不过出去透口气,你们不用跟来。”他虽然这样说了,呼延寿却仍然招呼了另外一个侍卫跟了出去。凌端心中一动,也起身跟了出去,他自知虽然江哲对自己颇为优厚,那些侍卫却对自己十分戒备,所以站的远远的,看着江哲立在雪地当中,负手望天,不知道再想些什么。凌端摸摸腰间短戈,恨意更深,却是只能隐忍等待。

这时,江哲突然放声而歌道:

“天不仁兮降乱离,地不仁兮使我逢此时。干戈蔽日兮道路危,民卒流亡兮共哀悲。离离黄蒿兮枝枯叶干,累累白骨兮刀痕箭瘢。霰雪漫天兮心意寒,壮士碧血兮凝深川。日黯风悲兮边声四起,望断云山兮不见桑梓。万里飘摇兮身不自主,无日无夜兮不思我乡土。四海不平兮黎民多恨。我虽安居兮常闻唏嘘。乃从圣君兮多行不义,残人家国兮怨我者多,生不冀求兮南归雁,死当葬我兮楚江畔。”

凌端听得入神,虽然有些句子听不大懂,却也能够感觉到那歌声中流露出来的悲切苦痛,听到“乃从圣君兮多行不义,残人家国兮怨我者多”这两句的时候,凌端不禁泪落,想到将军和昔日同袍,想到那么爽直糊涂的李虎,心中的恨意煎熬几乎令他再也不能容忍那个清瘦的背影站在前面,伸手摸向短戈,眼中透出冲天的杀意,或者,就豁出命去吧,就是死在这里也好过这般痛苦。

就在凌端心志将乱的时候,旷野之中突然传来了一阵缥缈的琴声,若有若无,琴声铮铮,妙绝天下,清越激昂中又隐隐带着悒郁悲伤,幽恨重重,琴声虽然微弱,却是连绵不绝,人人都听得清清楚楚,不知何时,空中又飘起雪花,琴声渐渐接近,越来越悲怆的曲调令得整个天地间都仿佛充满了苍凉萧瑟的气息。

这琴声似乎充满了诱惑之力,令人心中凭空生出恨意和狂热的杀机,这时,其他的侍卫也步出野店,警惕的看向琴声传来的方向,不过众人都是心如铁石的沙场勇士,自然不会为琴声所动,反而都从目中流露出警惕的神色。

小顺子眉头轻皱,他能够听得出来,这琴声中蕴含着深厚的内力,这弹琴之人不仅精通音律,还是一位内家高手,他自然不会为琴声所动,却是担心的看向江哲,江哲可是不会武功的,不过只看上一眼,小顺子便松了口气。江哲虽然不懂武功,可是纯以欣赏的心情去听琴,倒也不会被琴曲左右。

我凝神听着琴声,不由击节而叹,我也会弹琴的,不过粗而不精,这曲子若是我来弹奏,好几处都会难以为聚,可是那人想必是指法精妙,居然自然而然的转了上去,我虽非音律大家,眼高手低这四个字几乎可以概括我在音律上面的本事了,也能听得出这弹琴之人果真是当世圣手。不过琴曲的讲究的是乐而不淫,哀而不伤,此人琴中愁苦太甚,心魔因之而生,这就有些不好了。

众人都无妨碍,只有凌端本就身世悲苦,至亲的兄长和最尊敬的将军都死在战场,新交的朋友又被杀了,自己屈身在敌人身边为侍从,心中本就是悒郁愤恨,方才又被挑起了心中魔孽,此刻被琴声所惑,神智渐渐迷乱,双目发红,面色狰狞,突然之间挥戈扑向那青色的瘦弱身影。

他的形迹早就落入呼延寿眼中,轻而易举的将他拦住,凌端势若疯虎,不管不顾,拼命杀来,但是呼延寿乃是虎赍卫中一等一的高手,凌端怎是他的对手,若非是凌端舍命攻击,只怕早就落败了。

听到兵刃撞击的声音,我也再无心听琴,回头望了一眼,只一眼便看出凌端乃是心神为琴声所夺,这可不是我预料中的事情,轻轻皱眉,我下令道:“小顺子将凌端制住,让两个侍卫去看看是何人弹琴肇祸,将他带来这里。”

小顺子身形如同虚幻一般,丈许空间仿佛一步而过,替呼延寿接过凌端的攻势,一指点在凌端额前,冰凉的真气化作千丝万缕没入凌端体内,凌端踉跄后退,跌倒在地上,眼神变得清明,惊骇的看着手中的短戈以及持刀冷冷望着自己的呼延寿,心中明白发生了何事,他虽然心有杀机,却不是逞强的蠢人,早知道刺杀江哲乃是不切实际的幻想罢了,心中念念,只是寻机逃走而已,见到这样的情景,不由骇然。

凌端自然知道这样的情形,恐怕自己会被当场处死,虽然天性的倔强和傲骨让他不愿哀告求生,但是人谁没有贪生之心,凌端心中惨然,长跪在地,低声道:“罪人冒犯大人,求大人饶恕。”之后便再不发一言。

我知凌端性情,这一句请罪对他来说已经是十分艰难,更何况我本就无心杀他,只不过也不能让他体会到这一点,所以我故意表现出犹豫不决。

凌端可以看到江哲面上的神情,但是若是再苦苦哀求,就不是他能够作出的事情了,于是干脆低下头去,等待那人发出斩杀自己的命令。这时,他却听到一声悠悠长叹,然后耳边传来温和的声音道:“凌端你跟随谭将军多年,心魔太重,我知道你心中对我仍有余恨,被琴声所惑,江某也不怪你,只是不可再犯,若是再有这样行径,我必将你斩杀。”

凌端心中一宽,心道,难得这次有机会离开雍军大营,若是有可能我必然脱逃,自然不会再犯。他恭敬地道:“凌端遵命,不敢再犯。”这才站起身来,抬目望去,只见那些虎赍侍卫望着自己的目光更加冷森,他却也不放在心上,只是退到一边。这时,远处一辆马车绝尘驶来,方才还在缭绕的琴声也嘎然而止,那马车两旁正是方才去寻找弹琴之人的侍卫,一左一右押着那辆马车过来。凌端也是心中好奇,仔细瞧去,不知道何人能够弹出这样的琴音。

那是一辆普普通通的马车,看上去只是寻常旅人所使用的,驾车的是一个半百老人,相貌清瘦,目光如电,一见便知有一身不弱的武功。马车到了近前,那个老人下车恭恭敬敬站在一边,车帘一挑,一个紫衣佩剑的劲装少女跳下马车,然后伸手相搀,扶下一个剑眉星目的英俊青年,这个青年身穿深黑色貂裘,腰间悬挂着名贵的宝剑,气度温文中带着高贵,神色从容自若,一见便知不是普通旅人。

一个侍卫引领三人缓缓走来,另一个侍卫则快走几步回禀道:“启禀大人,弹琴之人已经带到。”

那青年不卑不亢的上前一揖道:“草民高延拜见大人,不知召唤草民有何吩咐?”

我欣赏的看了这青年半晌,英俊的外貌,修长挺拔的身形,高贵儒雅的气度,礼数周到而又略带矜持的行止,这个青年绝对是世家子弟出身,我也不愿怠慢,微笑道:“在下江哲,于荒野之中听到高公子抚琴,只觉琴声如同天籁,令在下心旷神怡,故而邀请公子前来,侍卫鲁莽,或令公子受惊,哲代他二人向公子请罪,不知道公子为何来到泽州,如果有什么为难之事,哲忝为泽州大营监军,或可效劳。”

那青年眼中闪过一丝不易察觉的光彩,道:“草民惶恐,不知是宁国长乐公主驸马,楚乡侯在此,江侯爷名震天下,草民乃是高丽子民,因缘来到中原上国,草民在国内曾经见过侯爷诗篇,瑰丽无双,草民深为钦服,想不到今日有缘相见,高某幸甚。”

我叹道:“原来如此,高丽虽是外藩,却从无自外中原,这些年来虽然中原战乱不止,但是仍有使者晋谒天朝,哲于南楚为翰林时,曾为崇文殿之事整理旧岁文书,同元三年,也就是贞渊十年,高丽使者入朝,可惜遭遇狂风,不得已至杭州登岸,遂为南楚武帝赵涉滞留。大雍武威六年,贵国也曾遣使到长安晋谒,可惜当时中原正在混战,使者金桂民于回国途中为诸侯所害,为此朝廷出兵平乱,流血飘橹,以报此恨,可惜自从之后,贵国再无使者朝谒,甚为可惜。”

青年眼中闪过惊叹之色,道:“侯爷对敝国之事果然知之甚深,金公正是草民外祖,他殉职之事传回本国,父,敝国王上为此亲临祭奠,备极荣哀。自此以后,东海海盗猖獗,敝国和中原水路几乎断绝,更是无法入朝上国。直到数年前,海道畅通,敝国才重新和中原开展贸易。草民久仰中原文物,因此随船至滨州,原想追随外祖足迹,遍历中原名山大川,不料纸上得来终觉浅,草民走错路途,误入沁州,因两国交兵,不得已羁留年余,幸而月前贵国大捷,沁州惨败,急于扩军整装,草民趁隙偷离沁州,翻山越岭,多日辛苦,终于进入泽州,因此地仍为军管,草民又是来自沁州,为免被人疑心,因此买了马车,准备进入中原内陆,想不到在此地遇到侯爷,虽然此事有些难以说清,但是草民也不敢隐瞒,还请侯爷明鉴。”

我心中惊讶难抑,仔细打量这人,相貌上倒看不出有高丽血统,不过高丽贵族汉化极深,这倒也是寻常,目光落到他身后的老仆和侍女身上,如果他果真是高丽人,那么他的从人应该可以看出真假,举手招那老仆侍女上前。用高丽语问那少女道:“你家主人所言可是实情?”

我在滨州的时候,我曾经掩去本来面目和高丽富商谈过生意,因此高丽语还是会一些的,说起来也算是字正腔圆,那相貌秀丽的少女眼中闪过惊讶,脱口而出道:“正是实情。”用得果然是高丽语,话一出口,少女才醒悟过来,又改用中原话道:“奴婢主子,羁留沁州,本非得已,还请侯爷见谅。”说的还算是通顺,只是口音有些古怪,幸而她声音清脆动听,听起来也不觉得刺耳。

我微微一笑,道:“姑娘的汉话说的很好。不知道如何称呼?”少女面上一红,道:“奴婢金芝,因为公子喜爱中原典籍文物,令奴婢改说汉话,已有多年,只是奴婢愚笨,口音难改,侯爷见笑。”

我的目光落到那老仆身上,那老仆虽是仆役身份,但是气度也自不凡,只是一揖道:“老奴崔九成,汉话只能听不能说,请侯爷见谅。”他却是用高丽话回答,语气流畅自若。

我心道,虽然说两个精通高丽语的随从并不难找,可这两人很显然确非中原人,这样看来,这高延的身份应该疑问不大,不过虽然如此,也不能让他们就这样离开泽州,不如将他们留在泽州一段时间,等到确认他们没有问题之后再说。而且这个高延气度不凡,这样人物若是平白错过不能结交,岂非是十分可惜。想到这里,我带着歉意道:“江某辅佐齐王殿下镇守泽州,凡事不可不慎,高公子即是高丽贵客,泽州如今兵荒马乱,江某不便让公子自由来去,恐有不测,有伤齐王颜面,若是高公子不弃,不妨留在泽州一段时间,等到春暖花开之时,道路畅通,再往中原不迟,我见公子人品出众,若是得到殿下赏识,公子在大雍境内就可以自由来去,岂不好过这样处处为难。”

高延眼中闪过一丝异色,却是警惕的低头避开江哲的目光,片刻之后,才道:“侯爷好意,高延敢不从命。”

我欣然道:“本应立刻请高公子到军中歇息,只是江某有意往万佛寺拜祭先父,若是高公子愿意,可否随在下同往,若是公子想要急着休息,我当遣属下送公子至军营。”

高延道:“草民也是无事之人,万佛寺既然有此名称,必然是佛像众多,必有可供流连之处,草民生性喜爱风景文物,若是侯爷不觉得麻烦,高延愿随侯爷同往万佛寺。”

我笑道:“如此甚好,哲见公子马车简陋,哲所乘马车宽阔舒适,就请公子和我同乘吧。”

高延似乎有些惊讶,半晌才道:“多谢侯爷美意,高延从命。”

这时候,虎赍侍卫已经将马车备好,我请高延上了我的马车,高延很是知机,不等我们多说,就解下佩剑交给侍女送回自己的马车。我随后也坐了上去,不过这次小顺子可是不驾车了,他也跟了进来,一个陌生人和我同乘,他自然不会放心,呼延寿则亲自执鞭。侍女金芝从他们的马车上拿了琴囊过来,也在我的示意下坐进了马车。

我原本从滨州带来的马车早就毁于战火,这辆马车乃是最近才送来的,比那一辆更加宽敞,四个人坐在车内,仍然觉得十分舒适宽敞。马车里面分为前后两间,后面是一张软榻,榻下有柜子可以放置物品,前间则是两侧固定着锦凳,中间一张桌子,却是铁铸,上面铺着雪白的织锦,桌上的杯盘底部都是磁石制成,放在桌子上不会滑动。此刻桌子上除了茶具之外,只放着一些书卷。

为了抵御严寒,马车里面到处都铺着羊绒毯,四周也都用毛皮封得严严实实,除了两边的窗子为了取光而没有挡住之外,随手摸去,到处都是软软的毛皮,不过窗子上面使用的是半透明的琉璃,不会让寒风侵入,再加上桌子下面的黄铜火炉,马车里面暖洋洋的,一点寒意也没有。不过高延似乎并没有因为流露出惊奇,看来他的身份不简单啊。

\chapter{第二十七章 一见如故}

第二十七章一见如故

我看看坐在我对面的高延,笑道:“兄台的琴可否让在下赏鉴一番?”

那高延笑道:“自然可以,大人诗文名震天下,又曾经参与筹建崇文殿,想必精通鉴识,草民这具古琴能得大人赏鉴,也是幸事。”

说罢取了古琴出来,这具古琴长有三尺六寸六分,十三徽似木非木,似金非金。纹路精美流畅,乃是古桐木精制,外形古朴雅致,琴弦乃是天蚕丝混合精金所制,琴身断纹如梅花,必是百年以上的古琴,此琴千金难易,能够携有此琴,这个高延身份非同寻常。

我仔仔细细看了半晌,目光落到琴尾的一处断纹上,抚摸再三,才轻叹道:“好琴,这是东晋初年蔡氏精制的古琴,此琴名为‘洗尘’,先朝赐予高丽王室的珍品,高公子据有此琴,又是姓高,想必是高丽王室贵人,哲方才如有冒犯之处,还请见谅,不知公子真正身份为何?”

高延眼中闪过精光,道:“此琴虽然乃是琴中圣品,却是深藏馆阁,尘封多年,不意大人仍然一眼认出,看来大人也是琴道圣手,高某钦服,在下乃是高丽王第六子,只因大王兄和三王兄夺嫡愈演愈烈,在下不愿牵扯其中,因此带了随从远赴中原,此行乃是私自前来,还请大人见谅,不要张扬出去。”

我心中暗道,此人颇有王者风范,为何不谋求王位,反而远离风波,莫非世上真有这般不爱权势的王室子弟,心中虽然有些疑问,但是既然他话已出口,我也只好暂且相信,便笑道:“高公子所言极是,既然如此,我也不以爵位相称,免得招致物议。”

看了一眼几上古琴,我又笑道:“方才听到公子琴声,心实敬慕,此刻窗外飞雪,四野无人,不知哲是否有幸听公子抚上一曲。”

高延神色从容道:“大人品鉴即精,音律上必然也有独到之处,在下就抚上一曲,请大人指正。”说罢,神色一端,十指轻拂,一阵空灵的琴声从他指下飞出,琴声缥缈孤洁,听得人如痴如醉。一曲终了,我不近喝彩道:“好,状飞雪飘零之态,拟天地孤寂之形,公子琴艺当世无双。”

高延面上却没有喜色,只是淡淡道:“在下平生别无他好,唯爱音律,刻苦修习,惟恐不公,不知道大人可否指教在下一曲。”

我隐隐听出这人话语中不知怎么突然带了几分敌意,心中古怪之余,却也是兴致勃发,道:“哲从前随曾学琴,无奈哲性情疏懒,这琴学得十分粗疏,公子勿要见笑才是。”说罢接过古琴,神思一凝,十指按上琴弦。

琴声已经停止,呼延寿心中忧虑,虽然那琴声至美,却也无心理会,他心中十分不安,也不知道这三人究竟何等身份,大人竟然让那高延和自己同车,若是那人乃是刺客,就是李顺李爷武功绝世,也难保大人没有损伤,若是出了事情,就是大人不怪罪,齐王和皇上也断不会轻饶自己。想要多探听这几人来历,无奈只留下一个不会说汉话的老仆在外,呼延寿也是有心无力。正在思忖的时候,车内琴声再起。

这一次的琴声和方才不同,方才的琴声曲调华美,指法娴熟,就是呼延寿也知道是大家所弹,这次的琴声初时有些艰涩,指法也有些混乱,但是片刻之后这琴声却仿佛溶入了天地。方才的琴声,就是呼延寿听了也知道状拟飞雪,这次的琴声呼延寿却觉得琴声就是飞雪,飞雪就是琴声,过了片刻之后,这琴声仿佛和飞雪融合在一起,呼延寿甚至不知道自己听得的究竟是琴声,还是飞雪坠落那种若有若无的声音。一曲终了,不知何时,呼延寿已经忘记了驾车,幸好这马匹乃是走惯道路的好马,也不用他费心,这才没有出什么乱子。

高延怔怔的听着,眼中神色迷离,似是敬佩又似嫉妒,琴声停止了片刻,他才赞叹道:“虽然大人指法生疏,可是曲中意境胜过在下百倍,不知可否指点在下一二。”

我接过小顺子递过来的香茶,轻轻喝了一口,道:“公子过誉了,其实公子的指法和对琴曲的演绎都已经到了出神入化的境地,哲不如远甚,我和公子只有一点不同,公子爱得是音律,所以勿要求工,一心只想将琴曲弹得更好。哲则不然,琴棋书画,于我来说都是赏心悦目之事,不过是为了让自己开心罢了,所以我不求精,也不求工,只要能够抒发心意,曲调是否华美,指法是否严整,都不在我考虑之中。不过我这样弹琴,就是弹上几十年也就还是这个样子,不像公子,只要领略到更高的境界,就可以突飞猛进。”

高延定定的看着我,深施一礼道:“这样浅显的道理我却是如今才悟透,难怪我的琴艺数年没有寸进,今日得到大人赐教,在下感激不尽。”

我连忙伸手相搀,笑道:“我这个人疏懒惯了,用耳多过用手,希望以后还能听到公子雅奏,不过琴不可多弹,今日已经兴尽,不如你我小酌一番如何。”

高延笑道:“敢不从命,在下离开高丽的时候,除了此琴之外,只带了十几坛美酒,可惜如今已经全喝光了,只剩下一坛梨姜酒,一直舍不得喝,今日遇到知音,在下也不能再吝啬,金芝,你去将酒取来。”他侧头吩咐侍女,却没有留心对面的江哲神情微变,目中突然闪过一丝寒芒,却是转而化成笑意。

侍女金芝清脆的应诺,跳下车去,不多时捧了一个可以装五斤酒的小坛子来。小顺子从车中暗格里面取出两只酒觞,高延打开酒坛上面的泥封,将酒觞里面倒满金黄色的酒液。我举起酒觞,深深的吸了一口气,道:“好酒,贵国的梨姜酒以梨汁和生姜酿造,味道纯美,回味无穷,我在滨州曾经喝过,不过那一坛只是新酒,我看这一坛至少是十年陈的美酒,哲真是福气不小。”

高延举起酒觞,笑道:“我国无人不爱饮酒,虽然比起中原可能有些不如,不过这梨姜酒滋味独特,又有养生的功效,我素爱之,大人请。”说罢高延先饮了一口,我知高丽人虽然爱酒,却是不喜欢牛饮,一定要慢慢饮来才行,而我也不喜欢狂饮,因此也只是浅浅喝了一口。

有酒助兴,我们两人不由谈论起诗文音律来,这个高延果然是当世奇才,若非是我博览群书,只怕就要被他问倒。我们谈得高兴畅快,忘记了时间路途,不知过了多久,呼延寿禀道:“大人,万佛寺已经到了,方丈慈远大师在前面相候。”

我虽兴尤未尽,却也只能道:“绪之,我们且先安顿下来,等到我拜祭之后,不妨再详谈。”绪之乃是高延的字,我们两人谈得投机,已经用字相称,高延点头道:“随云之意甚是,拜祭令尊大人要紧。”

下了马车,我一眼就认出这个慈远大师,当初我在雍王府遇刺的时候,他曾经被皇上以裴云之名请到王府负责守卫寒园,事后我也曾经去拜谢过,他是少林佛法精深的高僧,想不到如今竟被派到这里做了方丈,想来也是少林寺有心在泽州建立堂口吧,不过这些不关我事,上前施礼道:“多年不见,大师一向安好?”

慈远大师不敢怠慢,上前合十行礼道:“侯爷莅临敝寺,老衲不胜荣宠,诸事已经备好,只待侯爷明日拜祭。”

我笑道:“大师不用这样客气,小儿如今已是贵门弟子,什么侯爷大人的不用提起,大师就称呼江某姓名即可,今日已经晚了,哲旅途劳累,请大师恕哲无礼,这就想要休息了。”

慈远大师笑道:“江檀越体弱多病,老衲心中志之,已经备好清静禅院,请。”说罢,慈远大师亲自将我们送到后面的一间别院,高延则被安排到旁边的客院,沐浴更衣,用过晚饭之后,我坐在窗前看着越来越大的飞雪,陷入沉思。

这时,小顺子已经打理好一切,道:“公子,所有先期派来的虎赍卫士,方丈大师都已经安排妥当,万福寺已在我们控制之下,不过公子今日太冒险了,这个高延来路尚没有查清,公子就和他同车同饮,万一他身份乃是伪造,意图行刺,如何是好。”

我轻笑道:“你过虑了,这样高量雅致的人物,就是想要刺杀也不会鲁莽行事,没有绝对把握刺杀成功并且安然离去,是绝不会随便出手的。这人身份是真是假自然有你们去查,可是无论如何,这样的才华人品实在令我动心,令我生出一见如故的感觉,这样难得的知音才子,我怎忍心放过。若是等到你们查清楚了,这人真是刺客,恐怕从此以后不能再这样畅所欲言,因此我才冒险和他同行同饮,当然,也是算准了他就是有些问题,也不会在路上动手。好了,你让呼延寿吩咐下面的侍卫小心行事,对了,暂时不要让凌端有机会逃走,等到明日再说。”

漫漫长夜,辗转难眠,高延,不,应该是秋玉飞几乎是一夜没有合眼,他心中千回百转,为什么自己心许的知音却是自己此番要刺杀的江哲呢?想起那人的才华气度,心中只有欣赏倾慕,可是数日之后,自己行刺于他,若是成功,自然是痛失知己,若是失败,必然也不会再有机会和他谈论琴棋书画,当真是万分惋惜。

秋玉飞使用的身份并非捏造,高延却有其人,却非是不想争权夺利离开高丽,而是力弱不能与争,被迫流亡中原,可是其兄派人一路追杀,幸得段凌霄相救,才能保住性命,段凌霄见秋玉飞意欲刺杀江哲,深知其中艰险,本门高手虽多,无奈和大雍多年征战,恐怕大雍秘谍多半都认得,因此只能秋玉飞一人前往。可是想要接近江哲谈何容易,大雍皇上亲选侍卫保护,又有齐王一力周全,身边高手如云,戒备森严,等闲人不可接近。所以段凌霄特意向高延借了两名仆婢,让秋玉飞扮作高延接近江哲。以高延的外邦王子身份,必然会令江哲失去部分戒心,段凌霄相信秋玉飞可以得到江哲赏识,只要准备妥当,不难寻到刺杀良机。秋玉飞本就和高延相识,常常共饮相聚,扮作高延竟是不费吹灰之力。可是秋玉飞却万万没有想到会在这种情况下结识江哲,而且两人还是一见如故,互相倾慕非常。

按照原来的计划,秋玉飞是准备被大雍军方怀疑拘留,这样只要报出高延的身份,那些将领官员自然不敢随便处置,泽州现在仍属齐王军管,秋玉飞自然会被押送到齐王大营,这样的特殊身份,盘问之际,监军江哲当然不会缺席,而且为了查明这个身份真假,秋玉飞自然会滞留军营一段时间,凭着秋玉飞的才华,自然有可能得到江哲爱重。谁料,秋玉飞还没有遇到盘问的雍军,就遇到了前往万佛寺告祭的江哲,秋玉飞自然不会拘泥计划,立刻就以高延的身份和江哲结交,而这其中,唯一出乎预料的就是,原本对江哲心存不服和恨意的秋玉飞发觉,江哲此人,竟是自己难得的知己良朋,造化弄人,莫此为甚。

翌日,我换了素衣,在大殿祭拜亡父,殿中除了僧侣之外,就只有小顺子、高延、呼延寿三人相陪。拈香告祭之后,我令那些僧人退下,淡淡道:“绪之可是疑惑我为何邀请你前来陪祭?”

高延心中早在疑惑,便道:“在下确实有此疑惑,不过我和随云相知,令尊大人也就是我的长辈,拜祭一番也是礼所应当。”

我笑道:“虽然如此,哲却不是自傲之人,今日邀请绪之同祭,实在是有一事相托。”说罢我伸手接过小顺子递过来的一卷黄绫册,十分慎重地双手递给高延,高延接过下意识的一看,封面上写着《清远琴谱》四字。他生性最爱琴艺,忍不住翻开一看,岂知越看越是震惊,这册上曲谱多为绝传古曲,也有几首并不知名,可是却也是十分典雅华美。这册琴谱对于爱琴之人,那是难得的珍贵之物,高延只觉得双手颤抖,兴奋地道:“随云,这琴谱,这琴谱是何人所修,能够一阅此书,在下纵是少了十年性命,也是值得的。”

我神色有些黯然,道:“此谱乃先父所亲书,先父在时,虽然从不执意进取,但是才华却是世间罕见,随云虽然自诩博闻强志,但是却是粗而不精,不如先父远甚,父亲也是雅爱音律之人,最爱抚琴,先母喜弹筝,两位大人常常琴筝唱和,恩爱非常,不过先父韬光养晦,世人不知先父琴艺可称大家。无奈自从先母不幸过身,父亲悲恸之余,断琴绝弦,再不抚琴,从此成为绝响。哲贪多不精,父亲曾言我不是习琴之人,所以琴艺并未传授,不过养病之时,父亲或者也不想一身所学没有传人,带病写成此书,其中大半是父亲整理出来的古曲,还有一些是父亲自己谱成的曲子。这些年来,哲深藏之,不为世人所见,只因世人多是贪恋荣华富贵之辈,我不愿先父心血为世俗所辱。不知是否天意,这次哲前往拜祭父亲,便特意带了此谱,想不到遇到绪之。绪之人品才华,我已经亲眼所见,绪之爱琴,我也已经了然,想来必是父亲在天有灵,假吾手传君琴谱。不过此谱为父亲遗物,我不忍舍之,只有请绪之自行抄录一本,想来绪之不会觉得烦难。”

高延怔怔良久,突然上前下拜道:“江兄恩惠,在下刻骨铭心,只恨不能报答兄长厚爱。”言罢已是双目微红,泪水滴落。

我将他搀起,道:“你若不是琴艺高手,我也不会赠谱给你,绪之不必如此,虽然日后你我可能再无相见之期,可是只要你能够将清远琴谱传承下去,先父在天之灵,也必然万分欣喜。绪之,这琴谱最后一曲,乃是先父最后所谱,乃是为了悼念先母所作,技巧繁杂,我不能弹,自先父断琴之后,我再也没有听过此曲,今日我拜祭父亲,能不能请你试弹此曲,以慰我心。”

高延长揖道:“敢不从命。”

当那华丽平和中带着无限凄婉的琴声在大殿响起的时候,我再次陷入了回忆,琴声初时优雅华美,如同春雨,千丝万缕般渗入泥土,如同春花,绚烂多姿,然后绚烂归于平淡,平和中带着款款深情,突然,变徵之声突起,秋风萧瑟,寒霜仆地,深情肇祸,鸳鸯折翼,然后曲调一变,变得缓慢悲切,那是一种刻骨的心伤。

泪水盈满双目,我低声吟道:“重过阊门万事非,同来何事不同归!梧桐半死清霜后,头白鸳鸯失伴飞。原上草,露初晞。旧栖新垄两依依。空床卧听南窗雨,谁复挑灯夜补衣。”父亲一生何其苦也,虽有满腹才华,却因为乱世之故,而宁愿隐逸终生,幸得佳偶,却又中道分离,最后抛下我这孤儿黯然离世。

琴曲终了,高延歉然道:“此曲深奥,仓卒之间,在下只能演绎出三四成的意境,请江兄原谅。”

我叹息道:“绪之何出此言,能够重温此曲,哲已是万分感佩,虽然世间擅琴者多,但是此曲乃是先父所谱,我不愿俗人弹之,上次听到此曲,已经是整整十七年了,多谢绪之为我抚琴。”

高延眼中闪过悲色,心道,我能为你所做的事情也只有这件事了,想起自己揭破身份,刺杀江哲的时候,必然要面对的难堪情景,高延心中越发苦痛。这时,他耳边却传来了犹如霹雳一般的问话道:“绪之,你认为大雍和北汉之战,孰胜孰败?”

高延心中一震,立刻清醒过来,自己面前这人不仅仅是一个对自己厚爱有加的知己,还是北汉的敌人,大雍的谋臣,他低下头,平息了一下震惊的心绪,道:“在下是外人,并不十分清楚这些事情,不过大雍带甲百万,占据中原,北汉却是局限一隅,兵力窘困,长此以往,必然落败,不过大雍南方尚有后患,若是四面受敌,北汉也未必没有苟延残喘的机会。”他这番话说得倒是情真意切,他知道当前的局势对北汉有诸多不利,若非如此,他也不会主动要求前来行刺江哲,这本不是兵家正道,而且他也知道,想要瞒过江哲眼睛,最好的办法就是说真话。

果然,江哲点头道:“绪之虽然来中原不久,不过对局势也算是有些认知,你说得不错,如今大雍正是处在关键时刻,若是能够一举攻下北汉,则天下一统,不过时间的问题,若是这次北伐失败,可怜天下百姓,还不知要承受战乱多久。”

高延心中巨震,他虽然知道来年必有战事,却没想到江哲将此事看得极重,竟然想一举功成,心中有些惊骇,却不敢流露出来,平静地道:“在下对军国大事知道的不多,大人乃是大雍重臣,所言必是没有差错。”

我微微一笑,道:“小顺子,再取一束香来。我要祭拜一个故人。”

小顺子递过一束香来,我拿着香火拜了几拜,然后小顺子将它插到香炉当中,我默默祝祷一番,才道:“绪之可知道我祭拜的故人是谁?”

高延微愣,他怎会知道,便答道:“在下不知,不过大人特意祭拜,必然不是寻常人物?”

我轻轻叹息道:“方才我祭拜之人乃是故德亲王赵珏,哲曾经在他帐下效力,德亲王品性高洁,忠贞贤良,哲深深敬慕,今日忆起前尘往事,故此祭之。”

此言一出,高延心中一震,若是真正的高延自然不知道江哲与德亲王旧事,但是秋玉飞却是知道的,他犹豫再三,终于忍不住问道:“在下曾闻太人与德亲王事,据说大人得德亲王赏识,从其征蜀,得胜而归,后德亲王殁于襄阳,大人还曾千里探望,可是后来大人上书被贬之后,又被如今的大雍皇帝掳入长安,遂降之。后闻有德亲王旧部尊王遗命刺杀大人,令大人九死一生,为何大人至今仍然深深怀念德亲王呢?”

我望着袅袅香烟,道:“德亲王殿下忠贞见疑,殁于襄阳,当时哲也在其身边,哲自幼生长南楚,若有可能,自然希望南楚能够一统天下,故而当日辅助德亲王攻蜀,心中虽知是奢望,也希望能够为家国尽力,可惜德亲王殁后,哲心灰意冷,对南楚再无一丝期望。当日雍王殿下将我掳入大雍,我心中实在不愿归降,故而着意为难殿下,不论南楚待我如何,我终究还是念着南楚之恩,无奈殿下之恩天高地厚,我一个俗人焉能不感激涕零,因此终于归顺殿下,虽然如此,我心中对德亲王仍感歉疚。可是那场刺杀却让我明白,对于德亲王来说,家国重于一切,我江哲不过是个棋子,若是对南楚有用,自然要好生笼络,若是有害,就一定要除掉,可是虽然我心悲痛,却也深深佩服他的忠心。”

高延有些茫然,不知道为什么本来说着北汉,江哲却突然谈到南楚。

这时,我又取了一束香拜道:“德亲王是我旧交,谭忌将军却是素未蒙面,这一束香却是希望谭将军能够瞑目九泉,当日德亲王身死,我是无能为力,今日谭将军之死却是我一手策划,谭将军忠于北汉如同德亲王忠于南楚,两位都是忠臣豪杰,也是哲心中敬佩之人,虽然哲所为之事,两位心中必然怀恨,可是各为其主,还望两位能够谅解。”

高延心中一震,想不到江哲竟然会祭拜谭忌,不由更加迷惑。却见江哲再次焚香祝祷道:“这第四束香却是求苍天宽恕,哲也知北汉龙将军乃是忠臣名将,本不应该勾连小人加以谋害,但是干戈一起,伏尸遍野,若是能够兵不血刃,哲情愿担此恶名。”

听到这里,高延几乎差点叫出声来,这是什么意思,此人的目光已经盯住了龙庭飞么,勾连小人是什么意思,莫非龙庭飞麾下有内奸叛逆确属实情,此刻他心中满是疑虑,几乎忘却了方才心中的感激和钦慕。但是他心思灵敏,莫非江哲实在趁机试探自己么?因此他故意流露出迷茫之色,似乎不明白江哲话中之意。

我直等到香尽,这才对高延道:“我已经命人准备文房四宝,明日就要起程回营,绪之恐怕不会有机会再看到琴谱,还是先去抄录吧。”

高延目光落到琴谱之上,几乎都忘记了北汉面临的危机,他心想,就是自己知道了什么,也不可能在这个时候回去警告龙庭飞,还是先抄了琴谱,剩下的事情以后再说吧。

看着高延的背影,小顺子低声道:“公子这是何意,对此人的探察尚未有回报,公子似乎已经将他当作清白无辜,又待他如知己好友,可是方才又故意误导他,奴才不知道公子心意到底如何?”

我叹了一口气道:“不需要情报了,我已经肯定此人必是北汉刺客无疑。”小顺子目光一闪,突然道:“公子既然肯定,奴才相信必有证据,那么公子是不准备杀他么?”呼延寿站在一旁,早就已经迷糊了,方才听到江哲祭拜德亲王所说的话,他心中十分不安,接下来的话语他更是有些不明白,江哲所行计策除了齐王之外,只有小顺子知道全部计划,呼延寿只是隐隐知道一部分,所以他也不知道江哲说得是真是假,这些事情他必须写成密折上报皇上,可是万一引起皇上对大人的猜忌,又该如何是好,呼延寿陷入了左右为难的窘境。如今听到江哲和小顺子的交谈,他终于明白至少江大人方才所说乃是误导高延的话语,可是为什么江大人这么肯定高延是刺客呢?

\chapter{第二十八章 步步为营}

屋内残灯如豆,我心中惆怅,难以入眠,小顺子推门而入,将手中一卷帛书递上,道:“这是和高延有关的情报,若非是公子已经肯定此人乃是北汉刺客,我也看不出其中有什么异常之处。”

我淡淡道:“是庄峻来了么,让他明日听用。这也是机缘凑巧,这高延本是真有其人,恐怕现在也是身在北汉,此人冒名而来,本来没有什么破绽,只可惜过犹不及,那‘洗尘’古琴就是最大的破绽。此琴虽然的确是高丽王室珍藏,可惜多年前被人盗出王宫,辗转到了江南,当初收赃的就是天机阁,我还曾经亲自鉴识过此琴,琴尾处断纹就是我亲自督工修整的,此琴被我暗中拍卖,世人罕有知晓,可是无论买琴者是谁,都不可能是真正的高延。我想此人本想利用这具古琴掩饰身份,可惜却留下这样的破绽。”

小顺子惋惜地道:“可惜此人才华,卿本佳人,奈何作贼,不过他既然是为了公子而来,公子就是想利用此人,又何必将琴谱相赠,岂不可惜?”

我轻叹道:“虽然我有心利用此人,可是赠谱之心却是一片赤诚,此人雅量高致,爱琴如命,这卷琴谱赠给他实在是再好不过,只希望此人不要过于固执,能够保留有用之身,不要辜负了我的琴谱,不过虽然不知道他的身份,这样的人才,应该不是普通人,我想,他应该能够带着琴谱回到北汉的。”

小顺子问道:“那么公子是否准备不再使用凌端呢?而且若是让高延行刺公子,也未免太冒险了,公子千金之躯,岂可轻易赴险。”

我笑道:“明日有你在我身边,又事先知道他要动手,难道还会被他所乘么,你尽管放心,明日依计行事即可。”

夜深雪寒,高延伏案急书,忙着抄录琴谱,此刻什么刺杀,什么北汉大雍,早就被他抛在脑后,直抄到半夜三更,才终于抄录完毕,高延又从头到尾检查了一遍,没有发现有疏漏之处,这才珍而重之的将抄好的琴谱收藏起来,又将江哲借给他的琴谱放好,准备明日归还。这些完成之后,高延轻叹一声,明日路上自己就要寻机动手了,若是真得跟到军营,就是刺杀成功也很难逃脱,原本他是拼着一死准备混进大雍军营的,如今难得有这个机会,江哲身边的护卫又不是很多,若是明日不能刺杀成功,恐怕自己真的很难脱身了。不过据说邪影李顺武功高强,自己如何能够瞒过他的耳目雷霆一击呢?而且就是刺杀成功,只怕自己也会遗憾终生吧,高延心中暗暗苦笑。

彻夜难眠的不是高延一人,这一夜凌端也是难以入眠,昨日到了万佛寺,他本有心趁夜逃亡,可是到了之后不久,才发现江哲身边的虎赍卫士先后到达,已经将万佛寺牢牢控制住,这还罢了,凌端相信还是有机会逃走,毕竟自己并没有得人重视,可是昨夜和自己同房的侍卫拿了一碗伤药来,自己因为白日和呼延寿交手,受了一些轻伤,也没有拒绝,可是不知那侍卫是否有意,药中加了些安眠的药物,竟然让自己安安稳稳睡了一夜。今日凌端偷偷将药倒去,伪装睡着,可是那个侍卫也在房中,凌端一时不敢动弹,惟恐惊动这个侍卫。可是他已经得知明日就要回程,若是再不想法子逃走,自己可真要没有机会了,他可不想什么时候像李虎一样被无缘无故的处死灭口。关于这件事情,他已经想了很久,只能认为和石英有关,却始终弄不明白李虎一个小小的士卒,怎会遭遇到这样的惨事。

终于夜深人静,凌端轻轻起身,走到那侍卫身边,正想趁着他熟睡将他杀了,但是转念一想,这个侍卫武功高过自己,若是不慎惊动他人,自己绝对难以逃生,而且自己若是这样做未免有些忘恩负义,这些日子,这个侍卫对自己十分照顾。想到这里,他只是轻轻点了那个侍卫的睡穴,让他不能醒来而已。

想了一想,凌端也不客气,将这个侍卫身上的金银一扫而空,他不是君子,知道无钱寸步难行的道理,穿上便装,披上大氅,他潜出房间,或许是因为他并未得到重视的缘故,这个房间可以说比较偏僻,只要穿过两道防线,应该不会有危险的,当然明日他们发觉之后,可能会派军队搜索自己,不过仗着对泽州地形的熟悉,凌端觉得自己有几分把握穿过群山回到沁州。

在凌端小心翼翼地按照白日的观察潜出古寺的时候,几双眼睛却暗中注视着他,呼延寿低声笑道:“这小子还算聪明,选得路途比较安全,当然这也是我们的布防主要是为了保护大人,才有这个空隙让他溜走。大人说今日凌端必然会逃走,果不其然。”

站在他身边的侍卫道:“还是大人手段高明,昨日一碗药摆平了这小子,明日又要回营,这小子若是不趁今夜逃走,还想什么时候逃走,这些日子他也够苦的,不过老赵可是倒霉了,被人打了闷棍不说,还被洗劫一空。”

呼延寿笑道:“明日按照计划传令捉拿凌端,能不能逃生就看他自己的本事了,不过你暗示一下,就说大人其实对他颇为怜悯,并不急着要他的脑袋,不过不要太留痕迹,这些事情你都明白,这个人还是让他逃回去比较好。好了,明日我们还有要事,大家都回去睡吧。”

负手站在窗前,秋玉飞神色漠然,今日就是生死相见之日,他要让心境空灵如往昔,才能完成刺杀江哲的任务,并且从重围中逃生,侍女金芝捧了水进来服侍他梳洗,他看着金芝,突然用高丽语道:“今日不论成功与否,你们两人都要殉死,你可后悔么?”

金芝警惕了看了窗外一眼,也用高丽语道:“主上受段爷大恩,无以为报,金芝和崔老都情愿赴死,请公子不必介怀。”秋玉飞再次叹息一声,从桌上拿起那本琴谱,轻轻抚摸着黄绫封面,神色无限惆怅。金芝见了,疑惑地问道:“公子,我见那位江大人温文儒雅,才华绝世,对公子也是推心置腹,公子如此动心,想必也是不愿杀他,为何定要勉强自己呢,金芝不是畏死,只是觉得公子失去这样的知己良朋,只怕一生都不会快乐。”

秋玉飞苦涩的一笑,道:“师门恩重,此事不能自主,昨日你不在大殿,没有听到他的话语,不论他是何等样人,有他一日,我北汉将士就难以安寝,其实我也知道大势如此,独木难支,可是哪怕能够避过今年春天的苦战,也能为北汉多留一分元气。”

金芝叹息一声,道:“既然如此,奴婢也无话可说。”

秋玉飞叹息一声,伸手去拿方巾,耳边突然传来若有若无的呼吸声,秋玉飞心中一震,莫非有人在外面偷听,可是方才怎么毫无所觉,那人既然能够瞒过自己的耳朵,为什么现在却又被自己发觉呢?他装作毫无所觉的样子披上外袍,道:“先去拜见江大人,你和崔老准备好行装,今日我们还要赶路呢。”

说罢他才装作不知道外面有人的样子推开房门,果然看到不远处站了一个小沙弥,神情似乎有些尴尬,见到秋玉飞出来,才松了一口气,道:“小僧静玄,奉方丈之命求见高檀越。”

高延心中一宽,知道那静玄是因为金芝在自己房中,不便出声求见,才在那里静候,不过这个小和尚武功倒是不错,他仔细打量了这个静玄一眼,只见他虽然不过十八九岁年纪,但是宝相庄严,气度凝重,已有高僧气象,不愿失礼,便道:“不知道方丈大师有何见教?”

静玄道:“今晨楚乡侯大发雷霆,正在责罚身边侍卫,这些事情本来不该佛门弟子过问,可是方丈大师忧心侯爷一怒之下,恐会开了杀戒,方丈心中不忍,想请公子前往相劝,侯爷待公子如同挚友,想必会给这个面子。”

这下秋玉飞心中倒是奇怪起来,怎么江哲会这般大怒,莫非是发生了什么大事么,他对静玄道:“在下和侯爷陌路相逢,蒙侯爷抬爱,视若知己,只是侯爷监察军务,恐怕其中涉及军机,在下不便插手,不过若是可能,在下也不会置身事外。小师父请头前带路吧。”

在静玄引领下走到江哲居住的客院,秋玉飞心中一惊,只见客院院门大开,百余侍卫将客院散立周围,虽然都是便装,却是杀气腾腾,威风凛凛,而江哲身穿轻裘,负手立在阶上,神色冰冷,几个侍卫跪在阶下。小顺子和呼延寿分别站在江哲左右,小顺子神色冷漠,呼延寿却是忧心忡忡。秋玉飞放慢脚步,想看一下情形。

这时,他听见江哲冷冷道:“赵维义,我曾命你用心监视凌端,你是如何用心的,居然被一个竖子制住,虽然那凌端所知不多,可是若是他逃回北汉,被有心人看破端倪,岂不是有害我军大业,来人,给我将赵维义拖下去重责三十棍,然后给我撵回长安,让皇上处置去。”

旁边的侍卫听命,如狼似虎一般将一个侍卫拖到一边,当庭杖责,那个侍卫虽然被打的血肉横飞,却是不敢呼痛,只是咬牙苦忍。

我早已发觉“高延”站在院门外,目光中神色十分复杂,心中不由生出遗憾,不是没有想欺骗自己,这高延却是高丽王子,可是先有“洗尘”的破绽,再加上昨日我赠谱之时反复试探,他虽表现完美,可是话语中终于露了痕迹,一个落难的高丽王子,一种爱琴的痴人,若非是与己身秘密切相关,怎会对中原之事这般关切,再高明的掩饰也瞒不过有心探察的眼睛。

故意装作没有看见“高延”,我的目光已经落到了另外几个侍卫身上,流露出犹豫的神情,似乎在思考要如何处罚他们。这时呼延寿的目光适时的落到了“高延”身上,露出隐约的喜色,道:“大人,高公子来了。”

我听到呼延寿的禀报,装作才发觉有人到来一般,抬目望去,看到“高延”之后,才让神色缓和下来,笑道:“原来是绪之来了,我在这里处罚侍卫,让绪之见笑了。”

秋玉飞上前行礼道:“在下惊扰江兄处理军务了,不知发生何事,让江兄这样恼怒。”

我示意他走到近前,神色有些懊恼地道:“绪之,有些时候妇人之仁真是要不得,前些日子齐王殿下在庙坡大破北汉谭忌,谭将军所部几乎全部殉死,只有一个鬼骑凌端幸存下来,我见他年纪不大,又是谭将军身边亲卫,不忍他在苦役营里煎熬,因此软硬兼施留在身边执役,这个孩子虽然总是不冷不热,我也没有放在心上,反而怜他忠勇,不愿加害,总是想着过上一两年,北汉平定之后放他自由就是。想不到这个少年也是不知好歹,竟然在昨晚摆脱侍卫的监控,私自逃走,虽然我有心提防,不让他接触军机,可是他毕竟在我身边多日,恐怕会知道一些不该知道的事情,你说,这些侍卫是否无用,让一个还未成年的孩子从他们眼皮底下逃走了。当日你我初会之前,此子为绪之琴声所动,竟然意图刺杀于我,若非我怜他心魔未除,早已将他赐死了,绪之或者还记得他。”

秋玉飞心中震惊,面上却不敢流露出来,当日他和江哲初会之时,确曾看到凌端跪地请罪的场景,但是他当时并未留意,此刻回想起来,那个少年神色倔强,跪在地上却仍然流露出不屈之态,想不到那少年竟是谭忌亲卫,更想不到江哲会将那少年留在身边。

秋玉飞镇定了一下,道:“在下确实记得那凌端,不过侯爷这样做法,在下以为不妥,侯爷乃是泽州大营监军,身份何等重要,凌端即是这等身份,侯爷就不该让他近身,如今责怪贵属下虽然没有什么不对,但是侯爷错失在先,依理不该过分责怪他们。”

我听了他的相劝,心中思忖,他倒是没有说错,若非是我本想利用凌端,这件事情本就是我错得更多,不过对这个“高延”更是生出爱惜之心,论事明白,言词委婉,善于劝谏,可惜却是北汉刺客,不能留在身边。自然而然地流露出被说服的神色,我放软了口气道:“绪之说得有理,这倒是我的错失了,罢了,赵维义虽然有错,三十杖也足以抵罪,就不用撵回去了,赵维义,你可心服。”

赵维义下衫皆是鲜血,被同僚搀扶过来,下拜道:“属下疏忽,让那小贼逃走,虽受责罚,也是理所当然,蒙高公子求情,大人宽恕,许属下戴罪立功,属下感激不尽。”

我看了一眼他身上血迹,有些愧疚地道:“我方才怒火攻心,倒让你受苦了,下去好好敷药养伤吧,至于缉拿凌端之事,虽然重要,但是也不用你们去做,一会儿派人回大营,请齐王殿下传下军令缉拿此人,不过此子虽然忘恩负义,我却怜他忠义,尽量还是生擒吧。绪之,让你见笑了,不妨和我一起用饭,一会儿就要启程了。”

秋玉飞俯身行礼道:“敢不从命,琴谱原璧奉还,请大人收下。”说罢双手郑重其事地递上琴谱。

我接过他手中的琴谱,心中也是感叹,知道从此刻起就要随时小心他的刺杀,因此琴谱一到手,我立刻将琴谱递给小顺子,小顺子也趁机靠近我身边,避免了让“高延”趁机刺杀的机会。

秋玉飞在将江哲接过琴谱的时候,下意识的握住了暗藏的兵器,但是一看见那双幽深淡然的双瞳,却是不禁手软,这一犹豫,小顺子已经靠近了江哲,自然而然的将江哲护住,秋玉飞心中叹息失去了一个机会,却又隐隐窃喜,他希望能够让江哲死得无知无觉,最好让不知道自己就是杀他的刺客才好。

我将琴谱收回,又伸出右手延请“高延”入内一同用早饭,见他有些怔怔地望着我,心中也是一动,我不忍杀他,看来他也不忍对我动手呢,便微笑道:“绪之在想什么呢?”

秋玉飞反应过来,正想为自己失神找个借口,突然远处传来快马奔驰的声音,众人都望向院门,不多时,四五个身穿火色衣甲的骑士在院门下马,一个威武的骑士匆匆走来,走到阶前下拜,双手过顶,举着一个装文书的锦袋,急切地道:“庄峻拜见大人,殿下有令,有紧急军情,请大人立刻回营商议。”

呼延寿取了锦袋上来,打开检视过后,将里面的两份文书递给江哲。秋玉飞眼光一闪,已经看到其中一份上面写着“高延”两字,另外一份却是只有上下款,虽然只是匆匆一瞥,却也看到是齐王写给江哲的书信。只见江哲先打开那封书信,看过之后,面上露出淡淡的喜色,虽是一闪而逝,却被秋玉飞看得清楚。江哲将那封书信折好递给小顺子,小顺子随手将那封书信放到怀中。而另外一份文书,江哲拿过来匆匆看了一遍,便向自己望来,秋玉飞知道必是雍军秘谍将对自己的身份调查情报送来,虽然相信师兄不会留下什么破绽,秋玉飞却仍然心中忐忑不安,面上却作出毫无察觉的模样。

我露出畅快的笑容,道:“绪之,我本想带你回营,不过大营已经送来情报,绪之你的身份料无问题,我就做一回主,给你身份文书,让你可以自由离去,虽然我更想和你多聚几日,可是兵危战凶,我也不想你涉险,如果你愿意的话,可以先到长安我府上暂住,多则两年,少则一年,我就会回京,到时候我可想听听你琴艺进步多少呢。”

秋玉飞心中剧震,眼睁睁看着江哲走入房间,不多时拿了一份墨迹尤新的文书出来,笑着对自己道:“有了这份文书,沿途官府不会为难,等你到了长安,可以去见内子,她自然会帮你安排住处,长安乃是帝都,繁华无比,绪之想必会满意那里的生活。”

江哲的神情是那样愉快,可是秋玉飞却是如坠冰窟,他怎会想到江哲竟会在自己身份得到“证实”之后立刻就遣自己离开,这虽然说明江哲对自己好感极深,才会如此轻易就让自己自由离去,可是这样一来,自己哪里还有机会刺杀呢?等他反应过来,那份文书已经塞到了自己手里,江哲却已经退开了。

将文书递给“高延”之后,我安全地退回小顺子身边,满意的心想,这下不会有太大的危险了,不过不敢流露出愉快的心情,我面上满是遗憾地道:“绪之,我要即刻启程了,如果有缘,我们定会再见的。”这时几个侍卫从房内出来,手里提着行囊,小顺子接过青色大氅,帮我系在身上,我又向“高延”行了一礼,道:“绪之珍重。”说罢就向外走去,小顺子和几个侍卫将我护在当中,向外走去。

秋玉飞知道这是最后的机会,再不想办法就没有了刺杀的可能,他情急智生,高声道:“江兄慢走一步。”言罢疾步上前,在江哲身后数丈处,单膝下拜道:“在下落难之人,得江兄厚爱,赠以琴谱,待如亲弟,在下无从回报,江兄请受某一拜,此后经年,应是相见无期。”言罢叩首下去。

我心中一震,明明猜到他是要诱我接近,可是心中却仍然是一片悲凉,我当然有不错的法子应对,只需背对着他,假惺惺的说上几句谦逊的话,再说些难堪离别之痛的虚言,就可以不去扶他。可是黯然销魂者,唯别而已,更别说今日分离之后便是仇敌,再无相聚论琴的机缘,回想数日来相聚,我虽也是真情流露,可是却是处处算计于他,他虽然是刺客,可是我看他用的真心倒比我多上几分。心下有些愧疚,不知为什么,我心头一热,再也不能保持冷静,便给他一个机会刺杀我吧,之后我就再不欠他分毫。想到这里,我转身向他走去,伸手相搀,道:“绪之不必多礼,今日不过暂别,他日自有相聚之期。”

就在江哲突然转身的时候,小顺子和知情的侍卫心中都是心中一抖,却又不敢拦阻,若是让“高延”看穿其中有诈,只怕是监军大人计策成空,这个罪责他们担当不起,可是江哲生命安全更胜其他,除了小顺子身份特殊,快步跟上,护在江哲身侧之外,他们也下意识地向江哲靠近,幸好秋玉飞心中激荡,也没有发觉这些侍卫的异常。

就在我右手搀向“高延”的时候,他抬起头来,我清晰地看到他眼中的绝决,然后便看见一个黑影龙蛇也似,从他袖中飞起,这样近的距离,我可以看清那是一条黑色的软鞭,此刻鞭稍蓄满真气,如同利箭一般刺向我的面门。明明心中早知会有行刺之事,可是我却听到耳边响起悲愤的叫声道:“绪之!”那明明是我自己的声音,为何我却不知是怎么喊出来的呢?

就在生死存亡之际,我觉得膝弯处一痛,双膝一软便要向下跪去,那黑色的鞭稍从我发髻上面拂过,然后一股强力从后面向我扯来,我仰面跌倒,双膝欲折,不由痛呼一声,却见眼前青影一闪,然后有人拖了我的双臂将我抢到一边。直等我清醒过来,才看到小顺子已经和那个“高延”缠斗在一起,而将我救到一边的则是呼延寿和另外一个侍卫。这下子我可明白了,定是小顺子用什么手法将我救下,不过这小子大概恼我轻身涉险,或者是没有别的好法子,才让我受了些苦痛,不过根据我对他的了解,原因多半是前者。死里逃生之后的虚弱让我心中暗暗发誓,以后绝对不能冲动,再不能做这样的蠢事,轻轻拭去不知何时出的冷汗,我高声道:“小顺子,给我将高延生擒活捉,我定要问问他是否还有良心。”不用装作,我的语气和神情是绝对的悲愤气恼。众侍卫将周边团团围住,方才江哲险些遇刺的情景让他们也是心有余悸,对刺客是刻骨痛恨,绝不容他逃生。

\chapter{第二十九章 舍命相搏}

鞭影翻飞,如同一条黑龙在云中飞舞,可是那如虚如幻的身影在重重鞭影中进退自如,每一指每一掌都辛辣凌厉,却又浑然天成,秋玉飞越斗越是心惊。虽然早知邪影李顺武功高强,可是今日交手才知道此人的确高明,若是大师兄在此,应该可以和他一战,自己若能撑过两百招就已经是难得的了,那些虎赍侍卫只是四处围住,想必是对邪影李顺信任非常,所以不插手他们之间的争斗,只是严防自己逃脱罢了。

交手十数招,秋玉飞已经出了一身冷汗,暗自庆幸自己从前虽然怠于学武,但是被师尊和大师兄监督着,武功倒是没有差得太多,正在这时,便听见江哲气愤的下达命令,要将自己生擒,秋玉飞心中一痛,索性不顾生死,拼命攻去,邪影李顺面上虽然闪过不豫之色,可是手上却是放松了许多,这一来此消彼长,秋玉飞居然占了上风。

被迫强行出手刺杀,本就是很难成功,秋玉飞也不知自己是否心中存了殉死之心,全然不顾临行之前师尊嘱咐自己的“伺机而动”要旨,但他心中明白,虽然他爱琴胜过一切,可是若是北汉覆亡,师门遭劫,他也情愿一死以谢,既然连生命都不顾惜,还顾什么情谊恩德,宁可自己身死,也要杀了江哲,这样疯狂的意念逐渐在他心中膨胀。

又交手几招,秋玉飞突然神色变得肃然,不避不让向小顺子扑去,小顺子一掌迎来,秋玉飞仿佛未见,软鞭如同毒蛇吐信一般绕向小顺子身后,前掌后鞭将小顺子困在其中,小顺子眉头一皱,他可不想和秋玉飞同归于尽,身形一转,间不容发地避过了鞭稍和掌风,这时秋玉飞突然侧头张口,一道血箭如同流虹掣电,射向小顺子要害,小顺子身法虽然变幻莫测,却也是难以应对,总算他已入先天境界,真气瞬间在体内逆转,那道血箭擦肩而过,小顺子只觉肩头剧痛,想来是受创不轻,而骤然逆转真气,就是他也不能全然无事,忍不住一口鲜血喷出,他怒火越盛,心中却是越发冷静,趁势一掌击去,秋玉飞使用的乃是魔门秘传的邪功“碧血箭”,以鲜血化成杀人利器,却是极伤元气,小顺子这一掌又是含怒而发,奇诡无比,秋玉飞眼看躲不过去,心中一横,硬生生受了一掌,冰寒的真气肆无忌惮地冲入秋玉飞体内,秋玉飞却是借力向后飘飞,虽然随着身形急速飞退,院中雪地上鲜血一路飞溅,却终于是脱身成功,直扑向江哲而去。

小顺子右掌击中秋玉飞,却觉得手下如击棉絮,无处着力,立刻心知不好,飞身追去。

我远远看见不过数十招之间,小顺子和“高延”就已经血溅当场,斗得惨烈无比,心中不由战栗,直担心小顺子是否不是对手,更后悔为何不早早将那“高延”用计谋困住。这时那“高延”又飞身向我扑来,我心中更是惊骇,幸而呼延寿等人将他阻住,虽然这些虎赍侍卫无人是他敌手,可是他一时也别想冲过重围,再看到小顺子也已经追击过来,看他无法脱身,我才放下心来。谁知刚刚松了口气,那效苍鹰扑击,在空中飞舞的英俊青年突然转头向我一笑,我见他玉面苍白,血迹宛然,心中凄然,还未等我心情平复,他已经再次借力飞纵,避过兵刃,两点金星从他袖中飞弹而出,透过人群向我射来。两个侍卫出刀拨打,却是落空,但是他们的身躯却挡在暗器之前,那两点金星却是穿过他们的血肉之躯,速度不稍减,向我射来。我只觉双腿发软,无力闪避,这时,一只苍白的手出现在我眼前,食指中指之间夹着一根乌黑的发簪,将那两点金星击落。却是小顺子心思灵敏,一见秋玉飞这般不惜牺牲进攻,便知道会有意外发生,对他来说,我的安全自然是最重要的,所以才及时赶回我身边,用我迫他留在身边的玄铁簪击落了那追魂夺命的暗器。这时,那两个被暗器穿过身体的侍卫才跌倒在地,痛呼不已,他们本是铁骨铮铮的汉子,如此痛苦,显然那暗器对他们的损害极大,鲜血汩汩而出,无法止住。

秋玉飞远远看见,神色一黯,这暗器十分歹毒,乃是京无极用在大漠时意外获得的一种奇异晶体磨制而成。这种晶体不惧水火,坚硬无比,可惜只有枣核大小,京无极令能工巧匠费了数年之力,才将这种晶体琢磨成梭形暗器,斜开尖刃,只要是用足了内力,可以透过精钢铁甲,更可以破去真气护身。这种暗器京无极也只有六枚,他自己并不使用暗器,又因为秋玉飞武功稍弱,所以赐给秋玉飞三枚防身,是秋玉飞救命的法宝,绝不轻易使用,想不到如今两枚齐出,却被小顺子拦住,他不由后悔方才暴起行刺的时候,若是使用暗器,或者已经成功了吧。

我深深打了一个寒栗,那暗器透过穿着软甲的侍卫身躯仍有这般威力,想也知道若是打在我身上会有什么后果。我俯身从地上捡起那两枚暗器,虽还不知它们的材质,却知十分珍贵,而且无毒,不由庆幸不已,,想必是“高延”十分高傲,不屑在暗器上淬毒吧。我高声道:“暗器无毒,用这瓶药替他们止血。”我从怀中取出一个玉瓶递给旁边的侍卫,他们连忙去救助那两个受伤的侍卫,不多时鲜血止住,幸好他们有意闪躲,没有射中要害,否则这种歹毒的斜刃,足以让他们身死当场。

这段时间虽然短暂,可是秋玉飞已经被六个侍卫联手结成的刀阵困住,这些侍卫都是精悍的沙场勇士,武功都在二流以上,如今又是不求有功,只求无过,他只觉得自己陷入罗网当中,无力自拔,但他秉性倨傲,虽然如此,仍然咬紧牙关苦战,幸好小顺子似乎是担心江哲的安危,没有加入战局,否则他早就支撑不住了。

我心中也生出一丝苦恼,这个“高延”也太狠毒了些,我原本是希望他知难而退的,他若一心逃走,再加上小顺子放水,未必没有机会,可是他这样拼命死战,看来只能将他生擒,再用不忍杀他的理由而将他拘禁起来,然后让他寻机逃走了。唉,世事不如人意者十之八九啊。

又过了几十招,小顺子有些不耐烦了,随手从地上掬了一捧雪,双手一握,真气外溢,不多时,雪化成冰,小顺子手掌一搓,十几块碎冰入手,他手指连续轻弹,那碎冰变成了神出鬼没的暗器,不过数招,秋玉飞闪躲不过,被一块碎冰击中麻穴,身子一滞,已经被呼延寿一刀背拍中后心,跌倒在地,立刻被两个侍卫反剪双手按在地上。一个擅长鹰爪手的侍卫上前,干脆利落地卸下他双臂关节。然后呼延寿带着几个侍卫将他带到我面前,强令他跪下,呼延寿亲手将他头发向后拽去,让他仰面向上。我清晰的看见他额头渗出滴滴冷汗,面色苍白如雪,却是不肯呼痛,神色漠然。

我心中苦苦盘算着如何能够不露破绽地放走“高延”,口中却是道:“高延,你真正身份为何?我想你不是真正的高丽王子。”

秋玉飞听见江哲问话,冷冷道:“我也不妨直言,在下乃是魔宗嫡传弟子秋玉飞,当日秦泽一战,我以号角相助北汉,却被你的鼓声所败,心中愤恨,因此前来行刺于你,你我两国仇恨似海,多说无益,要杀就杀,若是你恨我欺你,不论什么酷刑责罚,我都承受就是。”

我叹道:“原来你竟是魔宗弟子,唉,魔宗弟子果然是一身傲骨,贵国先锋将军苏定峦当日在雍都身亡,我虽没有亲见,但是皇上曾经数次提及苏将军的豪勇,秋玉飞你也不愧是魔宗弟子,我身边这许多高手还差点被你刺杀成功,你若肯归降于我,念你尚未造成大祸,我还可宽容,若是你再固执不降,休怪哲心狠手辣。”

秋玉飞神色冷然,道:“你既然知道我魔宗弟子身有傲骨,就不该劝降,几日来你待我恩厚,又以令尊琴谱相赠,我亦感激非常,但是两国交兵,各为其主,刺杀你虽非我所愿,却也是不得不如此,如今我落入你手,你若是仍有眷顾之心,就请给我一个痛快。”

我心中一动,掩面叹息道:“绪之,不,我应叫你玉飞,你我都是身不自主,我本应将你斩首,首级送去北汉示威,可是三日来相交莫逆,我心实在不忍,琴谱赠你,我也不愿收回,罢了,小顺子,你废去他的武功,然后将他送到营中软禁起来吧。”

虽然这样说,不过我在衣袖之后给小顺子使了几个眼色,想来他应该明白我的意思,谁知小顺子脸色阴沉,似乎没有留意我的眼色,走到秋玉飞身前,看看他惨白绝决的面色,伸指向他气海缓缓点去。我大惊,若是真的废了秋玉飞的武功,我还怎么让他逃走呢,可是这个时候我又不敢阻止,若是露了破绽,这秋玉飞恐怕就是非死不可了,这叫我怎么忍心。小顺子手指已经几乎点到秋玉飞气海,却突然停住了动作,缓缓起身道:“公子,此人伤势严重,若是此时立刻点破气海,只怕是病势缠绵,不久丧命,公子既然有心留他性命,不如等他伤势稍好一些再动手吧。”

我几乎是长出了一口气,心中明白小顺子仍然是记恨我今日的冒险,这才用这般举动来吓我,歉意地看了看小顺子,道:“竟然如此,我枉通医理,竟然忘记了你们习武之人真气被破之后,往往不如常人康健,罢了,暂时不要动手,你们将他关节接上,先将他带回营中软禁,对了,他还有仆婢在外,应该也是刺客一党,你们去将那两人擒来,带回营去好好盘问。”

秋玉飞从散功的威胁边缘脱身出来,心中也觉得侥幸,纵是一身傲骨,也不愿再出言冒犯,心道,我若能恢复一些功力,就有机会逃走,还是暂时不要惹怒他吧。这样想来,他神色平和了许多,也不说话,任凭那几个侍卫接上他手臂关节,一时没有绳索,几个侍卫面面相觑,对他们来说杀人比俘获敌人更方便,身上几乎从不带着绳索,只得点了秋玉飞几处穴道,将他放到阶上,准备一会儿上路时带走。

这时,去拘拿那崔九成和金芝的侍卫匆匆向院内走来,我一看他们双手空空,就知道人没有捉到,事实上,对那两个人我并没有放在心上,只看他们高丽话那么流畅,就知道十有八九可能真是高丽人,这两人若是逃走,对我来说只有好处,若是被俘,也无关紧要,只要我安全地回到大营,而秋玉飞途中顺利逃走,我这一局就已经布成,所以我并没有特意提前令人将他们拿住,现在看来,他们果然跑了。我只是淡淡对庄峻道:“庄侍卫,你先快马赶回去吧,请齐王下军令缉拿那两人和凌端。”庄峻一直护在我身边,他不知其中详情,但是见我遇刺也是出了一身冷汗,道:“大人放心,属下这就换马回去向殿下禀报,一定捉拿住他们。”他已经知道凌端的事情,只当多捉一个人而已,也没有放在心上。

我微笑点头,正要说几句嘉勉的话,毕竟他要立刻回去,未免辛苦一些,这时,突然院墙上显出两个身影,一个老态龙钟,一个婀娜多姿,却是崔九成和金芝,两人齐声尖啸,双手挥动,十多个小黑球从他们手中射出,在空中炸开,火焰飞散,毒雾缭绕,金针纷飞,这却是一种罕见的火药暗器,霎时间院中一片黑雾笼罩,视线不清,所有的侍卫都立刻找了遮蔽之处,幸好这些暗器虽然涉及面广,威力却是不大,这些侍卫都穿着软甲,只需护住面目即可。不过他们应该是不想伤害到秋玉飞,那暗器没有向石阶射去。

小顺子见状带着我跃到石阶之上,恰好站到秋玉飞身边,我心中并不害怕,那两个人武功应该并不高强,小顺子足以护住我。

这时,秋玉飞瘫倒在石阶上,虽然形容狼狈,但是他偶尔张开的眼睛却是闪现一丝寒光,魔宗有许多不为外人所知的独门武功,其中有一种心法最适合在这种时候使用,他仔细调整着呼吸,运起内力冲穴,虽然穴道被制,但是这种普通的手法对他来说作用并不大,趁着崔九成和金芝来攻,秋玉飞也顾不得可能被发现,一心一意的运行真气,等到邪影李顺带着江哲退到他身边不远处的时候,秋玉飞已经冲开了大半穴道。

他虽然仔细掩饰,可是小顺子武功远在他之上,虽然没有回头去查看秋玉飞的禁制,听见他的呼吸有异,便知其中有蹊跷之处,但是他也不露声色,心道,此人已经被俘,若是途中让他脱走,未免令人生疑,不如趁着这个混乱的时候,让他自行解穴,这回他总不会定要刺杀成功才肯甘心吧。

以小顺子本心来说,若是能够杀了秋玉飞才称心意,可是他也知道此人关系重大,乃是绝好的反间棋子,若是错失此人,不知道江哲是否还会轻身涉险,最好的解决方式就是完成公子的心愿,他本就是心思灵动之人,转瞬之间已经想出了一个主意。

这时,崔九成和金芝已经跃下围墙,两人手中都是一柄精光耀眼的短剑,向秋玉飞所在之处扑来,虽然他们的暗器歹毒,可是虎赍卫士毕竟是大雍最精锐的军队,不过片刻,这两人就被困军阵之中。小顺子故意站在秋玉飞和江哲中间,提防秋玉飞不顾生死再向江哲出手。

秋玉飞看见崔、金两人已经力竭,知道机会不再,也顾不上是否会被李顺发觉,真气逆行,忍不住一口鲜血喷出,拼着受了内伤,终于冲开了穴道。而小顺子的反应也果然如他预料一样,他出声的瞬间,小顺子已经带着江哲飘飞避开,秋玉飞翻身跳起,起足飞踢,积雪飞扬,向江、李二人身在之处袭去。而他自己却向院墙扑去。

同一时刻,崔九成踉跄后退,手中短剑被击飞,他跌倒在雪地上,两柄横刀下斩,他奋力翻滚避开,嫣红的鲜血滴落,金芝尖叫一声,手中短剑脱手而出,射向一个正要挥刀斩杀崔九成的侍卫,那个侍卫虽然看不到飞来的短剑,但是身后传来同僚的警告声,他不顾一切翻身避开,那柄短剑飞落雪中,这时候,崔九成艰难的坐起身来,双手抖动,黑色的暗器飞舞,侍卫们都不想和他同归于尽,自然而然的避开烟雾和毒针,崔九成用高丽语大声呼喝道:“你们快走!”,金芝和秋玉飞都听得清清楚楚。

这时,秋玉飞已经翻身跃到院墙之上,全力施展轻功的他不是那些侍卫可以阻拦的,更何况大部分的侍卫的注意力都不在他身上,而唯一能够阻拦他的李顺却不能脱身,因为,就在崔九成高喝的时候,金芝已经看见秋玉飞的行动,她将最后的暗器舍命掷向江哲所在的位置,虽然被那些侍卫和小顺子先后挡住,可是她也成功的让小顺子“不能”放心地去追击秋玉飞。

就在秋玉飞身形杳然之际,崔九成终于身中数刀颓倒在地,而金芝已经是手无寸铁,呼延寿十分恼怒,虽然他是得到小顺子暗中传音,让他不要安排阻拦秋玉飞的脱走,可是这么多侍卫却被三人逼得手忙脚乱,他心中仍然是十分窝火,看到崔九成已经伏诛,他的目光落到被众多侍卫围在当中的金芝身上,此刻的金芝只凭着小巧的身法躲闪,已经是气喘吁吁,钗横鬓乱,呼延寿满腔的杀机也不由有些消退,他高声道:“兄弟们先退下,金姑娘,你还不立刻投降,若是再负隅顽抗,只有死路一条。”

听到他的命令,那几个围杀金芝的侍卫退后一步,虎视耽耽地将金芝围在当中。

金芝只觉得浑身无力,双足一软,坐倒在雪地上,呼延寿的目光转向我,露出请示的意味,我叹了一口气,高声道:“金芝,你应该是高丽人,为何要插手中原的事情,如今秋玉飞已经逃走,你的任务想必已经完成,何不束手就擒,你一个弱女子,又是流亡异国,我也不想为难于你,只要你说出幕后主使和接应手段,我就放你离去如何?”

金芝无力地抬起头,用高丽语道:“殿下受秋公子师门大恩,将我和崔老转赠也是无奈之举,大人乃是中原贵胄,豁达海量,冤有头,债有主,请你不要怪罪六殿下,一切都是我们自己的主张。”说罢,少女的嘴角渗出乌黑的鲜血,娇躯一阵抽搐,软软地倒在地上,香消玉陨。

我沉默了片刻,道:“小顺子,你带着半数侍卫出去追捕秋玉飞,两个时辰后回来。”

小顺子皱皱眉,他自然知道我实际上是要他去做什么,可是若是放下我一人在此,他心中实在不放心,正在犹豫之际,一声清朗的佛号传来道:“阿弥托佛,李檀越尽可放心,老衲愿代檀越保护江侯爷一段时间。”

望着站在院门口的慈远大师和他身后几个神采奕奕的青年弟子,小顺子心中一宽,慈远大师武功在少林可以排到前十,这些青年弟子也都是少林的杰出弟子,有他们保护,短时间内绝对不会有问题的,事实上,如果他们早就在江哲身侧,秋玉飞未必敢出手刺杀,都是江哲故意安排,不让他们显身,才有今日这场虚惊。

望着小顺子他们的背影,我心中暗道,既然秋玉飞已经成功脱身,那么就要做戏做到十足,一定要让秋玉飞带回我准备好的毒饵,凌端、秋玉飞再加上沁州已经展开的杀局,不愁龙庭飞不入圈套。龙庭飞啊龙庭飞,羽翼折断之后,腹心又受重创,不知道你是否还有那样的勇气对抗大雍呢?

第三十章    绝地重生

秋玉飞并没有逃出很远,他深知自己的伤势很严重,如果不顾一切奔逃,只怕最终只能是死在雪中,他冲出万佛寺不远,便选中了一处小山坡,这里的背风处积雪足有丈余,秋玉飞小心翼翼地落到雪地上,轻软的积雪上只是微微下陷,秋玉飞觉得咽喉一甜,又强行将血水咽了下去,他强行使用踏雪无痕的轻功,就是为了不留痕迹,若是留下血迹岂不是糟糕至极。看看追兵还未出来,秋玉飞从腰间锦囊里面取出一颗龙眼大的蜡丸,轻轻捏碎外面的白蜡,里面是一颗朱红色的药丸,秋玉飞将药丸含入口中,红丸遇津而化,秋玉飞只觉得从丹田生出一股暖流,流向四肢百骸,他知道师门密藏的救命灵药已经起了作用,便轻轻躺在雪上,真气一凝,沉入积雪当中,随着他的下陷,周围的积雪簇拥过来,很快就将他存在的痕迹湮没。秋玉飞使用龟息心法,将外部的生机几乎断绝,开始进行疗伤。

借助药力和密藏的心法,秋玉飞只觉得身子好像处在温暖的水中,那种朦朦胧胧的舒适让他感觉似乎回到初生之前的那一片混沌的时光,多年的专心琴艺,厚积薄发,数日来的明悟,已经刚从生死边缘、情义两难的境界挣扎回来的强烈刺激,秋玉飞竟然奇迹一般的进入了那从未奢求的先天境界,外呼吸渐渐断绝,此刻的秋玉飞已经和莽原积雪融为了一体。

不知过了多久,秋玉飞的意识终于回到身上,仿佛从极度的深眠中突然惊醒,他能够感觉到周身气脉畅通无阻,不仅内伤尽复,而且真力尚有精进。他用六识探察周围情形,片刻,纵身破雪而出,抬眼望去,四野雪漫苍穹,身上积雪似乎比原先厚了许多。秋玉飞心知自己这次疗伤不知用了多少时日,远远望去,万佛寺依旧矗立,秋玉飞思忖良久,自己虽然已经功力精进,继大师兄段凌霄之后晋入先天境界,可是此地距离沁州数百里之遥,又是天寒地冻,若是不能得到补给,仍然难以飞渡。自己逃亡之时,除了伤药和那本琴谱之外,什么都没有携带,看来只有闯入这万佛寺索取了。他倒不畏惧寺中的少林高手,以他的武功,想要悄无声息地拿走干粮衣物并不困难。这次死里逃生,秋玉飞仿佛脱胎换骨一般,很多从前斤斤计较的事情,如今在他来说只是小事而已。

微微一笑,他举步向万佛寺走去,当日的黑裘如今已经成了破碎不堪的碎片,他却丝毫不觉得有什么不妥,走到寺门前轻轻叩门。不多时,一个小和尚前来开门,却正是他相识的静玄。静玄目瞪口呆地望着秋玉飞,讷讷道:“高公子,怎么你回来了?”

秋玉飞笑道:“我姓秋,名玉飞,慈远大师在么?”

静玄已经冷静下来,道:“七日前江侯离开敝寺,过了两日,齐王殿下传方丈至泽州大营,据闻殿下有意责难,因为当日公子行刺之时,敝寺上下并未出手相助江侯,至今仍无音信。”

秋玉飞歉然一笑,道:“这倒是在下连累贵寺了,不过在下看楚乡侯为人颇重情义,应该不会对贵寺有所责难。”

静玄引着秋玉飞向内走去,道:“公子说得是,当日公子两位同伴皆在寺中身殁,侯爷命敝寺好生安葬,现在骨灰都已经收好,若是公子有意,这次便可以带走。公子身边的事物侯爷皆令封存寺中,公子可要看看么?”

秋玉飞目光在静玄身上凝固了片刻,笑道:“少林弟子果然出类拔萃,小师父气度恢宏,方才我忽然起了杀机,想要除去未来的强敌,不过思之再三,有小师父这样的敌手,倒也是快意之事。”

静玄神色不变,回身道:“魔宗自经京宗主重整之后,凡是嫡传弟子,皆是一代人杰,秋公子历劫重生,前途不可限量,静玄不过是少林末学,焉敢当此赞誉。”

秋玉飞淡淡一笑,道:“你也不必曲意讨好了,我无心杀你全寺僧人,只要本公子离去之时,你们允诺不出寺门,我就不下毒手,小师父以为如何?”

静玄心中欣然,方才一见秋玉飞,他便知道此人已非吴下阿蒙,魔宗弟子又是心狠手辣,若是此人动了杀机,就算自己勉强可以逃生,寺中留守的几位师兄弟也绝难活命,因此一直曲意逢迎,虽然此举看来谄媚,但是在他来说,能够避免无谓的牺牲,也是值得的。

秋玉飞走入多日前居住的禅房,只见诸般物事仍然离去之时一般,只是十分洁净,看来有人常常打扫,他走到木几前,轻抚多日不见的爱琴,心中百感交集,轻叹道:“天命如此,夫复何言。”他知道江哲已经回到大军之中,再没有可能接近他进行刺杀,而且毋庸讳言,他对江哲的杀机已经被惺惺相惜的情感代替。将琴囊系在背上,秋玉飞道:“引我前去祭拜崔老和金芝。”

钟声缥缈,直入云端,站在大殿之上,秋玉飞心中默祷良久,才将崔九成和金芝的骨灰包好,这两人和他本来不过是陌路,却因为高延之命誓死相助,若非他二人,只怕他已经成为废人,被禁于雍军大营。不多时,静玄带着四五个年纪相仿的僧人走进大殿,手里拿着干粮和行囊。静玄上前道:“公子的马匹还在寺中,小僧想公子或者不想使用马车,所以已经备好鞍鞯,公子可以随时出发。”

秋玉飞目光一闪,道:“你倒是聪明解事!”看着气度沉稳的静玄,心中杀机不免又起,中原武林多一未来的栋梁,魔宗就是多一个未来的强敌,但是秋玉飞性情本就高傲,怎屑于杀一个对自己毕恭毕敬的和尚。终于轻叹一声,接过行囊走出了大殿,望望北方越来越厚的彤云,秋玉飞心道:“我还是迅速赶回沁州,刺杀虽然失败,但是数日相聚,我对江哲的观感或者对师尊和龙将军有所帮助,再说有些事情似乎很可疑,我也要想师尊禀明。”静玄在后面相送,秋玉飞面色一寒,道:“小师父应当知道轻重,你若是擅自离寺告密,秋某日后自然要来报复,雍军大营追缉秋某乃是必然之事,你也没有必要去锦上添花,还是在此安心念佛的好。”说罢举掌在静玄肩上轻轻按了一下,静玄面色骤然变得苍白,直到秋玉飞身影消失之后,才颓然到地。

几个小沙弥上前搀扶,惊问道:“师兄伤得怎样?”

静玄道:“无妨,只需数日闭关,再有你们相助,就会无事。”

一个小沙弥恨恨道:“若是师兄肯答应我们的提议,和那魔宗弟子拼了,也未必没有机会,这样含羞忍辱,这是何必?”

静玄淡然道:“师弟不知道厉害,我看此人功力已经大进,恐怕已经超越后天境界,师弟不知,到了那个级数,差之毫厘,失之千里,前些日子若非是有邪影李施主在,恐怕再多的人也未必拦得住他刺杀江侯爷,不过当日他应该还没有今日的水准,若非是李爷被江侯牵绊,只怕丧命当场的也会有这位秋公子。”

几个小沙弥听了虽然仍旧不服,但是他们素来信服静玄,也就不再多说,扶了静玄下去养伤。却不知静玄心中惊叹道:“江侯爷果然是天人,今日之事竟然被他料中。”

却原来当日虎赍卫四周追索二十里之后不见秋玉飞行踪,回来禀报之后,我思索再三,便找上慈远方丈,让他过几日等到大营军令到,就带了大部分弟子离开万佛寺,我料到秋玉飞伤重,必然走不远,只不过四野茫茫,魔宗之人必然擅长匿踪之术,找是找不到的,我也料到,此人定会事后重返万佛寺夺取干粮行囊,否则天寒地冻,他如何行走,若是万佛寺留人太多,我担心他会肆虐行事,这些和尚虽然厉害,可是真要是秋玉飞狠心起来,至少也要死上几个小和尚。我心中不想秋玉飞造此杀孽,和少林结仇,另一方面也希望他顺顺利利地回到北汉,所以只留个几个小和尚等着他。不过为了避免他杀人灭口,掩饰行踪,我又特意请慈远大师选一能屈能伸的弟子留守,好将秋玉飞送出门去。

而静玄就是被选中的知情人,他隐隐猜到万佛寺诸事恐怕都是江哲所策划的圈套,可是他在其中多方留心,也没有发觉什么破绽,只觉得一切都是顺理成章,秋玉飞乃是魔宗嫡传,也是静玄心中十分忌惮之人,可是却是落入陷阱而不自知,静玄心中戒惧的同时,也是谨言慎行,不敢稍露形色,幸而瞒过了秋玉飞的眼睛,保住了性命,完成了恩师谕令。他心有余悸的同时,也不由对楚乡侯江哲生出仰之弥高的观感。心中开始明白昔日在少林寺的时候,为什么方丈和慈真师伯对此人评价极高,又是颇为敬畏,更是处心积虑收了此人爱子为徒,这样的人物,只可以为友,不可为敌啊!

秋玉飞离开万佛寺之后,一路直向北汉边境而去,他地理颇熟,虽然雍军四处大索,可是他仍然能够找到一些小路通行,只是不能骑马了,这一段旅途虽然艰苦,可是秋玉飞武功刚刚突破瓶颈,在这种紧张艰苦的气氛下倒是更能稳定进境。

雍军的围捕并没有大张旗鼓,只是在各处关卡加紧盘查,秋玉飞能够感觉到他们外松内紧的局势,看来自己这次刺杀真得是令雍军很愤怒呢,不过对于功力大进的秋玉飞来说,虽然需要小心一些,绕过重重围堵倒是并不困难,若是从前的他,只怕是真要步步危机了。虽然如此,仍然花了十日才从莽莽群山里面进入沁州。

出山不远处有一处野店,原本是山中猎人常常聚集的地方,虽然简陋,却是烈酒香醇,野味丰富,秋玉飞走进野店的时候,店内除了掌柜夫妻之外,只有两个猎人正在那里喝酒,看到秋玉飞进来,都是面色惊异。虽然秋玉飞已经换上了普通衣饰,又因为翻山越岭而破碎不堪,可是容貌气度都是世间罕见,这几人怎不惊讶。秋玉飞也懒得理会他们,丢下一块碎银道:“有好酒拿一坛来,再上几个小菜。”

那掌柜连忙捧了酒坛过来,掌柜娘子则是端了野味殷勤送上,在这里可是难得见到这样的豪客。

秋玉飞放下心来,重回北汉领地,心中一宽之后,不免有些惆怅,这次败逃而回,颜面上可是有些过不去的,心中烦忧,忍不住借酒消愁,岂知酒入愁肠,更添愁思,醉意盎然中秋玉飞更是不愿赶路了,索性包下了野店唯一的一间客房,进去蒙头大睡。不知过了多久,秋玉飞才从睡梦中醒来,不由有些赧然,常年在外,何曾有过这样的失态放纵。起身从行囊里面取出干净的衣衫换上,准备出去吃些东西。谁知还没有走到店堂,就听见外面传来惊呼声。

秋玉飞心中一凛,向外望去,只见一个布衣少年倒在门口,掌柜的上前探视,神色惊惶地道:“这人气息都快没了,不会是要死了吧?”

秋玉飞见状,上前道:“让我看看吧。”说着俯身探视,片刻皱眉道:“此人是伤病交加,恐怕是几日没有好好休息进食了,掌柜的烧些热汤来给他灌下,先拿碗酒来。”

掌柜连忙倒了一碗烈酒端过,秋玉飞取出一粒固本培元的丹药给这人服下,将此人扶起,给他灌下烈酒,不多时,这人呼吸渐渐加粗,秋玉飞这才放下心来,目光落到少年面上,突然心中一动,此人似曾相识,想了多时,秋玉飞突然心中一亮,这人不就是那个当日伏地请罪的江哲侍卫,也就是从前的鬼骑凌端么?他比自己早一日逃走,想不到如今才到这里,想必是多日来费尽心力才逃出泽州,此人武功低微,能够逃生必然是受尽苦楚,若非是自己相救,只怕是会死在这里了,虽然对这少年并没有深刻的印象,可是想到两人同病相怜,都是在江哲手下幸而不死,心中不由生出好感,心道,自己不妨多留几日,带他一起回去吧。

将凌端扶到客房里面,秋玉飞再次仔细的检查了一下少年的伤势,觉得已经无碍性命,可是这样一检查,秋玉飞却发觉这个少年资质极好,而且所学心法也是魔宗旁支,不由心动,魔宗收徒讲究因缘,他对这少年生出亲切之感,心道,此子性情坚毅,若是学习日宗武功最好不过,虽然自己所习更偏向月宗,可是大师兄尚没有满意的门人,若是自己将这少年推荐给他,他应该会很满意。想到这里,就不能任由这少年昏迷下去,否则这少年功力必然大损。

凌端从昏迷中醒来,只觉得全身上下万分痛苦,不由呻吟出来,这些日子的逃亡已经耗费了他的全部心力,当看到那座野店的时候,凌端只觉得一切的辛苦都已经有了报偿,刚刚踏入店门就再也支撑不住,昏倒在地,此刻感觉到自己已经活了过来,凌端心中狂喜,他的身躯一动,身旁突然传来一个冰冷的声音道:“不可懈怠,起来我助你运功。”然后一粒药丸塞到口中,瞬间化作苦涩的寒流,凌端心中一惊,可是一只手已经按在他的背心,他的真气不受控制的运行起来,凌端心中一横,料此人是友非敌,便认真运功起来。初时,那人任由凌端自己行功,几遍之后,那人突然强行使用真气迫使凌端改变行功路线,凌端意欲强拒,可是内力却不受控制,那新的行功路线仿佛是真气本就该走的方向,凌端只觉得渐入忘我之境。不知过了多久,凌端悠悠醒来,只觉得四肢百骸真气畅通,他收功而起,只见一个布衣人负手站在窗前,向外看去。

凌端上前拜倒道:“弟子叩见前辈,前辈可是魔宗高人。”

那人没有回头,只是淡淡问道:“你知道我是魔宗之人?”

凌端谨慎地道:“弟子曾听将军说过,武功传自魔宗,前辈熟知弟子内功心法,所以弟子斗胆猜测,若有差错,还请前辈勿要责怪。”那人笑道:“果然是聪明过人,我是秋玉飞,魔宗嫡传弟子,我想,你应该还记得我。”说罢,那人转过身来。凌端骇然道:“高公子,你,你怎会是——?”话未说完,凌端已经明白其中始末,惊喜地问道:“前辈已经杀死江哲了么?”

秋玉飞叹了口气道:“别提了,能够生还已经是侥幸了,你能够逃回北汉,也是不容易,今后可有什么打算么?”

凌端露出遗憾的神色,但是他又警惕的看了秋玉飞一眼,担心他误解自己有嘲讽之意,见秋玉飞神色没有什么变化,才道:“弟子也不知道,本来弟子理应回军营,可是弟子心中有块垒难消,这次谭将军全军覆灭,弟子疑心有北汉人从中推波助澜,所以弟子想暗中查个明白。而且万人之中只有弟子生还,弟子也有些担心被人怀疑,经历了这许多事情,弟子不想再不明不白的死去。”说到这里,他的声音有些哽咽,想到莫名其妙被杀的李虎,他悲从心起。

秋玉飞轻轻拍拍凌端肩膀,他心中明白凌端心中忐忑,也知道北汉军内部有着隐忧,可是秋玉飞本就是魔宗当中的异类,身兼日宗月宗两门心法,却不喜欢战场厮杀也不喜欢阴谋诡计,除了音律之外,再无其他爱好,他也不愿过多涉及其中,便道:“你放心,随我回去见我大师兄,你若是福分够,可能会被师兄收为弟子,就是师兄觉得你资质不够,凭着我的面子,一个记名弟子还是没有问题的,到时候谁还敢加罪给你。”

凌端喜出望外,再拜道:“弟子叩谢前辈恩典,若能如此,弟子万幸。”

秋玉飞淡淡一笑,道:“好了,你去吃些东西,休息一天,明日和我一起启程,有些事情也要跟龙将军说个明白,我知道的不多,只是感觉大雍有什么阴谋正在进行,这些事情,萧师兄他们更加擅长,我就懒得过问了。还有,你也不用叫我前辈,我在门中排行第四,你叫我四公子或者四爷都行。”

凌端心中一寒,他知道萧桐负责军情探察,实际上还可能负责监视军中将兵,平日见到萧桐都是远远避开,这次要和他见面,不由心中惧意渐起。秋玉飞却没有留意这一点,目光飘向窗外,他也是心中不安,北汉的兴亡关系到魔宗荣辱,他虽然不愿过问军政,可是又怎能不担心覆巢之祸呢?

第二日,秋玉飞带了凌端出山找到哨所,借了马匹,急急赶向沁州,一路上马不停蹄,两日之后,两人终于到了沁州,还剩二十里路程,秋玉飞见凌端有些疲劳,就唤他下马在路边小店打尖。两人都是心事重重,缓缓用餐,却是无话可说。

突然,外面传来骏马奔驰和车轮滚滚的声音,秋玉飞无心理会,凌端却是听出这是训练有素的骑兵行军的声音,忍不住走出店门向外望去,只见远处一队骑兵押着一辆囚车驰来,囚车之中坐了一个相貌文雅,修眉长目的中年人,虽然身披枷锁,却是神态从容,毫无惧意。凌端一见,大惊非小,回身扑到秋玉飞面前,道:“四爷,怎么回事,段将军怎会被人用囚车押送?”

秋玉飞一皱眉,他疑惑地问道:“段将军,你是说我知道的那个段将军么?”

凌端点头道:“是段无敌将军,他难道犯了军法么,否则怎会被押起来,我看见押送段将军的是石将军的副将石钧,四爷,段将军素来得我们敬爱,为人又很严谨,怎会犯军法呢?再说,就是段将军犯了错,龙将军也不会这样折辱他吧?”

秋玉飞也是心中疑惑,可是按照魔宗的规矩,他没有军职,是不能直接过问军务的,可是心中疑惑难解,暗道,我私下问问总成吧?想到这里,秋玉飞出了店堂,这时,那队骑兵已经走到近前,秋玉飞挡住他们去路,冷冷道:“谁是负责之人,出来说话。”

那些骑兵勒住战马,将囚车护在中间,一个虬髯将领出阵,目光在秋玉飞身上转了一圈,却是想不起此人是谁,便高声道:“你是哪里蹦出来的小白脸,竟敢拦阻将爷执行军务,还不快退去,否则将爷就要问你一个劫囚之罪了。”秋玉飞面色一寒,身形一动,那个将领只觉得眼前一花,脸颊就被重重打了两记耳光。他恼羞成怒,道:“兄弟们,上,给我将他碎尸万段。”秋玉飞眼中杀机毕露,冷冷道:“你们真敢动手?”那将领大笑道:“我石钧说一不二,我既然不认得你,你又敢来拦路,十有八九是段无敌的相识,你若是劫囚,倒是一件好事,正好证明段无敌之罪。”秋玉飞神色越发冰冷,杀死几个士卒,对他来说不过是小事一桩,他正要出手之际,囚车之中的中年人扬声道:“石钧住手,你不看看对面的是什么人?四公子,末将身陷缧绁,不能见礼,请公子恕罪。”

秋玉飞看看中年人,淡淡道:“段将军,两年不见,你消瘦多了。”

中年人苦笑道:“四公子,末将每日殚精竭虑,如何能不消瘦,如今末将遭遇杀身之祸,还求公子在大将军面前替我缓颊,无敌感激不尽。”

秋玉飞在泽州留了多日,他眼见大雍军队那种从容自信的表现,战无不胜的气魄,心中隐隐觉得北汉军势虽也不差,却是少了些气魄,多了些悲愤,没想到刚刚回到沁州,又看到北汉军有数的名将遭到这样的折辱,怒火汹汹之余也有些心灰意冷,望望昏黄的苍穹,他心中突然生出不祥的预感,大势莫非真的是无法挽回了么。
