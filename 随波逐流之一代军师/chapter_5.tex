\part{第五部 纵横捭阖}

\chapter{第一章 惊闻巨变}

烛影摇红,帐外冰雪满天,帐内却是温暖如春,我披着长衣坐在桌案前看着案上的地图,心中踌躇难定,不知道凌端和秋玉飞是否能够回到北汉,虽然这两人都是坚毅不拔的性子,我又有意纵放,但是世事无常,若是他们一个也回不去,我可就白费了心思。

灯花绽开,惊醒了我的思绪,突然失笑起来,那边的计划进行的很顺利,就是秋玉飞和凌端都回不去,最多就是效果差些。我在十数日前就已经命令大雍在北汉的密谍,挑动石英和段无敌之间的不合,现在想必石英已经向龙庭飞告发段无敌的罪行了吧。看过有关石英的情报,除了作战之外,他实在是一个不通世事的人,如果不是龙庭飞的器重和保护,恐怕他不是死在战场之外,就是被人抛弃在战场上了,也只有他才会这样轻易地和龙庭飞另外一个心腹将领段无敌发生纷争。

突然生出奇想,若是和我的计划不符,龙庭飞过于相信石英,而秋玉飞和凌端又没有能够带回去不利石英的情报,龙庭飞麾下众将中最为沉稳端重的段无敌会不会成为牺牲品呢,若是能够做到这一点,倒是意外的收获,不过我可不敢这样奢望,段无敌作战可以用严谨少误来形容,这样一个人,很难将他入罪至死的,我并不贪心,而且留下段无敌也有好处,我不想北汉军失去战意,有这样一个防守出众的将领,是北汉军敢于勇猛作战的一个重要原因。

可能最后不如我的预想,石英逃过一劫,可是这期间已经足以造成将帅之间的隔阂和军心的动摇,说句心里话,占据了兵力的优势,我的计策不过是尽量减少我军的损失罢了,凭着齐王的用兵,和相对北汉军更加不利的局势,战胜北汉只是时间上的问题,只不过如果损失的太多,大雍统一的步伐会放慢很多,更重要的是,如果这仗打个几年,我可什么时候可以回家呢?

觉得有些疲累,我伸伸懒腰,准备上床休息,这时,呼延寿在外面禀报道:“大人,京中有信使来,是公主派来的,大人是否接见?”

我心中一惊,长乐怎会派来信使,她的书信都是通过驿站送来的,就是有些比较机密的事情,也经常利用传递军情的渠道送过来,莫非发生了什么事情,才让长乐派来信使。应该不会啊,虽然南楚有异动,东川也不稳,可是朝中人才济济,庆王殿下虽然不驯,可是也应该不会在这个时候明目张胆的抗拒雍都,我并没有将那些事情过于放在心上,毕竟我现在面对的敌人是北汉,皇上若是连这样的局势都不能稳住,也妄称明君了。

不管怎样我连忙召入信使,帐门一开,冷风透入,我打了一个寒战,一个相貌俊秀,肤色白皙晶莹的青年缓缓走入,却是董缺亲来,我心中更加担忧,董缺乃是我留在长乐身边的得力助手,长乐贵为公主,如今开府在外,若是没有董缺这样的人听命,必然会有许多不便。他亲自来此,必然是发生了极为重要的事情,而且可能是我们自己的事情。

董缺上前行了大礼,我轻轻看了跟在后面的呼延寿一眼,呼延寿很知趣地退了出去,虽然他负有监察之责,可是却知道有些事情最好不要去探查。他将要退出营帐的时候,我疲惫地道:“你去叫小顺子过来。”呼延寿连忙应诺,可是面色也有些忧虑,他已经察觉其中的异样气氛。

董缺见呼延寿出去,下拜道:“属下接到东川密报,事情紧急,不得不来向公子禀报。”

我挥手道:“不用多礼,等到小顺子来了再说,也免得你要说两遍,公主知道这件事情么?”

董缺道:“公主没有多问,不过命属下带来家书。”说着递上一封书信,趁着小顺子还没有到来,我展开书信,长乐并不知道天机阁和锦绣盟的事情,也从不会过问我身边的这些神秘人物,所以信上并没有什么特别的事情,只是提及霍琮学业进步很快,柔蓝跟他一起读书,已经不是从前那样贪玩,慎儿活泼可爱,家中并无事端。可是我能够感觉到字里行间的淡淡忧虑,毕竟夫妻数年,有些事情虽然没有和他说起,可是需要董缺亲自来见我,想必公主也知道发生了些不妥的事情。

我看过书信,心中已经平静下来,不论发生了什么事情,紧张都是没有用处的,过了一会儿,小顺子掀帘而入,前些日子秋玉飞行刺,虽然是我有心放水,可是小顺子还是很不满虎赍卫士应对绝顶高手的能力,所以这些日子一有空闲就在他们的营地和他们过招,就是晚上经常也给某些人特训,我常常看见身边的卫士鼻青脸肿,也有些同情,不过想到秋玉飞不过是魔宗小弟子,他上面还有高手,我就不说什么了,只是送去上好的伤药给他们。小顺子走到我身边,目光瞧向董缺,冷冷问道:“发生了什么事情?”

董缺道:“属下接到陈先生传书,庆王在东川铲除异己,除了倾向朝廷的文武官员之外,明鉴司在庆王身边的秘谍已经被揭穿身份,十四人被格杀,两人投降,只有一人逃走,庆王假称捉拿南楚秘谍,大索东川,那人已经被陈先生救下,不过我们和明鉴司并无合作,而且庆王封锁很严,陈先生不想冒险,只得将那人软禁起来,那人不知道我们身份,也不肯托我们相助送情报到雍都。而且庆王近来对原蜀国遗臣和反抗势力更加礼敬,还有使者找上锦绣盟,要求我们归顺,他答应扶立蜀王之子为王,重立蜀国。”

我皱眉道:“蜀王家眷不是都在雍都么?”

董缺摇头道:“庆王信使说当日蜀王归降之前,金莲夫人让两个怀孕宫女带着信物逃走,翼望为蜀王留下血脉,其中一个宫女后来果然生了男孩,据说已经落到庆王手中,庆王信使声称他们持有蜀王的身份信物,证明那男孩的身份。庆王信使许诺,庆王将立下血誓,绝不觊觎蜀国王位。”

我觉得有些头痛,虽然有些瞧不起庆王心胸狭窄,可是他真的谋反还是让我意想不到,毕竟他是大雍皇子,地位尊崇,就是谋反也该是夺取大雍皇位,想不到他却是去做蜀国的权臣,不过想了一会儿,我倒也佩服他的决心,他是准备拥立傀儡蜀王,然后和南楚、北汉一起发难,瓜分大雍,此人倒也放得下锦绣中原.。

叹了一口气,我在脑海中回想了一下庆王的情报,心中隐隐有了一个轮廓,看来庆王有此心已经多年,他也够隐忍,从前摆出和凤仪门誓不两立的架势,借助太上皇和皇上的同情,占据东川,坐拥兵马,现在又趁着大雍全力攻打北汉的时候暗中叛乱,看来对他来说,与其做大雍的亲王不如做一方诸侯,他对大雍的恨意不仅仅在于凤仪门,恐怕大雍皇室才是他心中痛恨的仇敌,想来,当日李援自认的补偿对他来说只是羞辱,他永远都会记得,大雍皇室为了凤仪门而贬斥放弃了他。

这些年来,他镇守东川,做得有声有色,刻意结好原蜀国遗臣,就是为了今日借重蜀人力量谋反,虽然从现在看来他还羽翼未丰,不会公然叛乱,可是若是稍微有隙,他就会向大雍腹地发起雷霆一击,东川的位置太重要了。现在想来,前些日子司马修嫒在宫中胡作非为,恐怕就是他的唆使,利用司马修嫒被杖杀一事,跳起蜀国大族的不满,如今为了庆王的颜面,皇上并没有将司马修嫒罪行公示天下,在庆王离间下,司马修嫒之死象征着大雍朝廷对蜀人的排斥,而失去抗争力量的蜀人就会依赖庆王。

想清楚整件事情,我不由庆幸当初让锦绣盟和大雍撇开关系,现在无人不知锦绣盟乃是神出鬼没的蜀人反抗势力,而且我特意让陈稹将那些心存复国之志的人物纳入盟中,用锦绣盟约束他们,总比让他们自行其事破坏小得多。

转念一想,我奇怪地问道:“明鉴司在东川的秘谍已经全被庆王控制?这样的话夏侯沅峰也未免太无能了,我觉得此人应该留有后手,他不是孤注一掷的人,不过庆王封锁消息很严密,若非锦绣盟控制的是本地蜀人的力量,这情报想必还传不出来,朝廷现在应该还不知道庆王谋反的事情么?”

董缺道:“这个我们也不清楚,我们对明鉴司是敬而远之的,但是庆王手段的确高明,正如公子所言,他切断了东川和关中的联系,就是明鉴司还有人手,也不能将消息传回去,我们通过蜀中,转道南楚天机阁将情报送到雍都的。而且陈先生估计庆王会让投降的秘谍继续传送假情报回去,这样一来,只怕雍都现在还不知道东川的事情。”

我站起身,示意小顺子取出东川的地图,沉吟再三道:“庆王谋反,现在还不是时候,我想明春我们和北汉苦战之际,才是他发难良机,这件事情已经是无可挽回,就是现在朝廷知道,也不可能改变这个局势了。董缺,你立刻亲自去见陈先生,让他同意归顺庆王,等到庆王谋反的时候,我希望锦绣盟成为庆王的最大助力,局势既然不可挽回,我们就要趁势而作,告诉陈先生,蜀国已亡,不可能在庆王手中重兴,我不过问他如何办事,我只要求他在我谕令传到的时候,可以一举覆灭庆王一党。”

董缺目中精光四射,他料不到江哲如此处置,又问道:“公子,我们难道不将此事告知朝廷么?”

我深沉的一笑,道:“夏侯沅峰不是常人,我不信明鉴司势力全部被铲除,虽然可能会晚一些,但是很快朝中就会知道此事,其实我更希望你们将消息截住,这些年来,皇上对东川始终存有戒心,在雍都和东川之间布有重兵,就是庆王起兵,也不能立刻奏效,我自信可以在一年之内灭掉北汉,就是不行,也可以让他们没有还手之力,到时候有锦绣盟作内应,庆王可灭,说不定还能饶上一些额外的甜头,董缺,你见到陈先生,也要弄清楚,如果他和寒总管都有心复蜀,说不得我也不能顾念旧情,白义、逾轮、山子、渠黄四人如今已经是锦绣盟和天机阁掌控大权的执事,若是有变,你就传我密令,软禁陈稹。”

董缺道:“公子放心,陈先生忠心公子,绝不会做出糊涂事的。”

我点头道:“我也只是防范于未然罢了,好了,你辛苦一些,连夜去东川吧,军营里面你不要多留,齐王不是好敷衍的。”

董缺默默点头,看向灯光下瘦弱的身影,心道:“这人总是没有轻闲的命。”

董缺走后,小顺子突然问道:“不告诉别人还可以,不告诉皇上恐怕将来皇上会怪罪公子?”

我苦笑道:“现在不行,若是皇上知道此事,我担心他会因为想保全庆王而急急行动,姑息养奸这种事情我是不做的,庆王不除,大雍难安,而且——”我停顿了一下,露出诡秘的微笑,道:“前日皇上密旨,将我狠狠训斥了一顿,说我不该轻身涉险,虽然他是好心,可是我什么时候受过这样的气,又被齐王嘲笑一顿,所以让他多忧心几日,就算是报复吧。”

小顺子苦笑,轻轻摇头。虽然主子已经是而立之年,可是还是不时会冒出孩子气来,总是让他啼笑皆非,不过这样一来,前些日子心中积怨却也烟消云散,他正色道:“公子,既然如此,北汉之事就需要快刀斩乱麻,不能拖下去了。”我点头道:“正是如此,我立意今年平定北汉,也是无奈之举,代州乃是抵御蛮族的要地,若是蛮族进攻北汉,我们不仅不能加紧进攻,还要缓下攻势,这是担心北汉国主不顾一切,放蛮人南下,只要是蛮人没有大举进攻的意思,北汉王室尚称贤明,必然不会作出这种为人诟病的举动来。”

小顺子若有所思地道:“公子派赤骥到蛮地去,莫非就是为了确认此事么?”

我淡淡一笑,道:“赤骥归来之后,向我禀明,今年秋天草原水草丰茂,蛮人各部都无心大举劫掠,所以代州今年只是略受侵害,并无大战,可是今年冬天蛮地遭受雪灾,这是我观看天象之后根据蛮地得来的情报确定的,明春蛮人必然大举进攻,可是我已经安排妥当,明春雪化之前,蛮地将遭瘟疫,牛马十不余一,这样一来,蛮人虽然有心进攻,可是碍于战力不足,代州足可抵御他们的侵扰。等我军进攻北汉的时候,如果北汉国主真的丧心病狂,想要利用蛮人和我们作战,那么首先代州林家必然坚决反对,其次蛮人势弱,我军灭汉之后也可以轻易将他们逐走。若是拖到明年秋天,蛮人恢复元气,为了弥补损失必然大举进攻,到时候我们若是再强攻北汉,就等于和蛮人呼应,一来有害大雍声名,再说也不利于大雍将来在这里的统治,所以这一年之内我们必须拿下北汉,为了这个目的,东川和南楚的事情都要放下。其实南楚主少国疑,庆王胸襟不广,只要皇上处置得当,不会影响北疆战事的。”

小顺子默默听着,良久道:“公子可要我去刺杀龙庭飞,他若一死,北汉再无回天之力。”

我正端茶欲饮,听到他的话一下子将茶水喷了出来,连忙道:“你别胡说,别说北汉有个宗师坐镇,就是没有也不用你去做这些事情,这种行刺的事情,多是势弱一方为了出奇制胜才用的手段,现在大雍兵力强大,不用你去做这种事情。而且——”面色渐渐沉素,我说道:“龙庭飞乃是北汉名将,北汉人最敬重勇士,事先削弱敌人无可厚非,可是若是不能在战场上将他们击败,北汉人绝不会心服大雍的统治,龙庭飞若死于暗杀,只怕数十年内北汉人都会争先恐后为他报仇,只有让他死在战场上,才会让北汉人彻底失去反抗的信心。”

小顺子无所谓地道:“公子既然这样说,那就算了,本来我是想着北汉人敢来刺杀公子,未免太过无礼,想要回报一下罢了。”

我露出古怪的笑容道:“想要报复,总会有机会的。”眼前突然闪过齐王可恶的身影,我心中突然生出一个想法,或者,我在向北汉报复这次行刺之事的同时,也还有机会报复一下这个克星的。

御香缥缈,九重深处,李贽坐在御书案后看着面前的折子,紧锁眉头,将折子递给坐在他左首一张椅子上的石彧,夏侯沅峰站在下面低眉顺目,神色恭谨非常。李贽叹了一口气道:“夏侯,你的明鉴司虽然迟了一些,但是总算是把消息传了回来,唉,三弟真是太糊涂了,他是天家贵胄,只要安分守己,就是数一数二的权贵,他却贪心不足,妄想谋反,难道他真的以为可以夺到皇位么,不论是名份还是功绩,他连六弟都不如,更何况是朕呢。夏侯,你在庆王身边已经没有了可以利用的人手了么?”

夏侯沅峰禀道:“臣死罪,除了一两名暗探之外,明鉴司人马已经全被铲除,有一人生死不明,但是臣想他绝无生还可能。”

李贽神色凝重地道:“东川生变,大雍的实力倒退到灭蜀之前了,李康这逆贼虽然还没有发动,可是明春泽州兴兵之时,他必然不会坐视,不过朕当日既然能够夺取东川,今日也不会畏惧于他,子攸,依你之见,朕是否应该暂时停止攻打北汉呢?”

石彧起身道:“陛下,臣以为万万不可,如今南楚、庆王、北汉将我大雍困在当中,若是一味防守,则只会削弱大雍国力,若是不能攻破一家,大雍危矣,齐王殿下、楚乡侯都有折子说北汉可攻,陛下不如对庆王加以安抚,同时小心戒备东川兵马,东川虽然有自立之心,可是庆王麾下都是大雍将士,蜀人也不见得深信庆王,庆王仓卒间绝对不可能大举进攻,陛下不妨缓缓图之,南楚暗弱,陛下可以甘辞厚币安抚南楚国主,到时候陆灿一人也不能擅自攻击大雍,南方可稳守,北方需强攻,陛下下密诏令齐王用心,有楚乡侯襄助,北汉可破也。”

李贽目光落到夏侯沅峰身上,见他神色中带着不赞同,问道:“夏侯卿可有什么见解?”

夏侯沅峰恭恭敬敬地道:“臣不通军事,然而也知攘外必先安内,南楚、北汉虽是敌国,不过是小患,我们不去攻打,他们也未必敢攻来,可是庆王谋反才是内忧,内忧不平,朝廷不安,臣的意见,不如暂缓北地攻势,安抚南楚,专心对付庆王。”

李贽微微一笑道:“夏侯说得不错,东川是要平定,但是如果朕一心纠葛于内乱,才是中了南楚和北汉君臣的下怀,夏侯,现在庆王也不敢明目张胆的反叛,你要想法子派进人去,策反、离间,这些事情不用朕教你。朕即位之后,在军部设立司闻曹担任刺军之责,朕将下密旨,组建西南郡司,负责东川、西蜀以及云贵的军情刺探,西南郡司暂时交给你署理,就把庆王当成从前的蜀王对待,大雍曾经做过的事情难道不能做第二次么。子攸,让苟廉出使南楚,安抚南楚国主的重任就让他承担,楚人畏惧大雍,一定要让他们不敢开战,陆灿一人之力焉能回天。北面么,我倒不担心,不过子攸代朕写封信给随云,朕不信他不知道东川的事情,让他也别藏着掖着,朕不会心软,让他拿个章程出来。”

石彧对这些事情只知道一个大概,但是他也隐隐知道江哲有些私下的力量始终没有交出,皇上对这件事情倒是默许的,因此点头称是。

夏侯沅峰听到这里却是心中一动,他对雍王夺嫡之前的事情很多都不清楚,但是听皇上的口气,似乎江哲有些私下的人手在东川,若是如此,那可就太好了,他本就担心急切之间不能妥善的重整东川的情报网呢。突然想到一件事情,他试探地道:“陛下,四日前,长乐公主府上的管家董缺突然北上,据说是去了泽州。”

李贽和石彧相视一笑,李贽摇头道:“这个随云,从来是云里雾里,难得坦诚相见。”

石彧笑道:“这也是陛下宽容,否则江侯爷这样的性子,还有谁有这个肚量用他呢?”

李贽神采飞扬地道:“朕平生最得意之事就是将江哲掌握到手中,子攸你用八百里加急将信送去,要不然,这人不知什么时候才会给朕一个准信呢?”

石彧含笑应诺,夏侯沅峰陪笑之余,再一次惊骇李贽对江哲的宠信,也再一次庆幸当初的选择。

\chapter{第二章 无敌之罪}

段无敌,祖父数代从戎,无敌少时,即有军略之才,十五从军,二十岁为禁军侍卫。时,晋阳有豪门何氏,为先主重臣,性跋扈,无敌不意得罪其家,贬斥至代州戍边,何氏尤不罢休,遣刺客杀之。段某幸脱大难。至代州,为林远霆所重,荐入沁州军,后为龙庭飞麾下名将,号磐石将军,长于守备,龙庭飞每出征,皆以段无敌守其后。

——《北汉史·段无敌传》

秋玉飞神色漠然,负手而立,凌端眼中闪着敬慕之色,段无敌虽然枷锁未除,却是下了囚车,三人站在路边枯树之下,石钧等人被赶出百步之外,不得近身。

段无敌神色平静,似乎不在意这一身枷锁,可是秋玉飞却能隐隐从他眼睛深处看出那种不愿为人探知的苦痛和委屈。他轻轻叹了口气道:“段将军素来得诸人敬重,龙将军也视将军如同左膀右臂,为什么会下令拘禁将军,将军不妨向我直言,待我设法为将军讨回公道。”

凌端连忙道:“是啊,段将军,谭将军生前对您敬重非常,若是将军在世,必然不会坐视您受屈含冤,小人虽然没有什么力量,可是也绝不会看着您受人诬陷。”

段无敌轻叹一声,道:“段某从前不过是对谭将军公平相待,想不到谭将军竟然如此推重,段某愧不敢当。”

凌端正色道:“当日将军遇刺重伤,我军颇受排挤,只有将军您不仅没有落井下石,还屡次额外送来钱粮,将军曾说,段将军您是可托以生死之人,凌端就是拼了性命,也不愿见将军受害。”

段无敌苦笑道:“谭将军谬赞了,说句公道话,这次段某乃是罪有应得,段某所犯乃是勾结商旅,走私货物,从中牟取巨利的大罪,数日前被飞虎将军石英查获,因此请了军令缚我到中军治罪。”

秋玉飞神色一变,他怎也料不到这平日端正恭谨,清白正直的段无敌竟会犯下这样的贪贿之罪,这样的罪行,轻些说是违反军规,贪赃枉法,重些说就是叛逆大罪。需要通过段无敌走私的货物,必然来自大雍或者东海,北汉国主有严令控制边关,除了少数商旅之外,其他人不许擅自和东海通商,而和大雍通商,罪同叛国。

秋玉飞心中恼怒,正要斥责段无敌几句,却见他神色平静,全无愧疚之色,心中不由一动,问道:“段将军可是受人诬陷?”段无敌平静地道:“并没有人诬陷,段某不必讳言,从三年前开始,段某经手十四次走私,得到银钱六十万,今次被石将军查获的货物价值三十万,段某可以从中获利十万。”

秋玉飞心中怒火熊熊,可是奇异的,一看到段无敌那双清澈如同明镜,深沉如同寒渊的眼睛,秋玉飞却是无法相信,这人会是一个不顾国法军规的贪渎将领。他深深吸了一口气道:“段将军不必再试探秋某,秋某相信将军所为必然有不得已之处。”

段无敌眼中光芒一闪,微笑道:“四公子身为国师弟子,虽然国师教徒甚严,公子也曾多受苦楚,可是公子怎会知道普通士卒的艰难,我军多年来和大雍作战,伤亡无数,这几年虽然胜多败少,可是大雍国势蒸蒸日上,我国却是越发艰难,公子想必不知道,从六年前开始,我军的粮饷就已经不足,能够拿到半数已经是难得的了,士卒重伤成残之后,抚恤也很难得到,所以军中流传这样的言语,宁可沙场战死,也不能成了废人。”

秋玉飞心中巨震,他虽然也是出身寒微,却是自幼就得到魔宗收养,比起几位师兄来,他可以说没有遭遇过太多的苦难,后来几位师兄或者主持魔宗事务,或者进入军旅,只有他终日弹琴练武,从不涉及这些军政要务,怎知北汉国事已经艰难至此。他的目光落到凌端身上,只见他面色隐隐带着悲痛,那是感同身受的神情。

凌端看见秋玉飞询问的目光,低声道:“四爷,段将军所说一字不差,当初我两位兄长从军报国,却是不许我和他们一起的,他们都说希望我能够成家立业,不要断绝了凌家香烟,可是我两位兄长战死之后,抚恤极少,家无余粮,我仗着学过武艺,也入了军旅,我从军杀敌虽然是想为兄长报仇,可是也是实在无力谋生,若非谭将军怜悯,我小小年纪怎可能成为将军亲卫,后面又蒙将军提拔,成了鬼骑的一员。四爷,打了这么多年的仗,谁家不是如此,所以我们都盼着可以攻下泽州,泽州沃土连绵,我们就可以靠着军屯养家活口。重伤成残的袍泽也可以有安身之所,不需为了担心连累家人而自杀,沁州,太贫瘠了。”

段无敌别过头去,可是秋玉飞看到他回头之际,清泪坠落尘埃,秋玉飞说不出话来,他从未想过,那些奋不顾身,拼命作战的军士居然承受着这样的苦难,比起他们,自己自由孤苦又算什么。他平静了一下心绪,道:“段将军所为莫非就是为了这些将士么?”

段无敌强颜一笑,道:“大将军为了弥补军饷缺额,下令允许将士在泽州劫掠,但是段某所部常年在后方防守,无法得到这样的好处,而且这两年齐王坚壁清野,我军很难有所斩获,不得已,我勾结巨商走私货物,一来从中优先取得廉价军需,二来索取重金补上军饷缺口,虽然此事有碍国法军规,可是段某也是顾不上了。”

凌端突然身子一颤,他跟在谭忌身边,隐隐知道这两年谭忌重伤不能领军,军中粮饷缺乏,这也是谭忌所部和取代谭忌出征的石英部下生出嫌隙的一个重要原因,凌端想起将军总是能够及时得到一些来路不明的银钱分发给将士,或者抚恤伤残,莫非,将军也参与了段无敌走私之事么?疑惑的目光望向段无敌,段无敌会意,却装作不见,其实走私之事,虽然段无敌竭力隐瞒,可是还是有人知道的,谭忌就是其中之一,还曾经派出亲信来相助段无敌,因为谭忌部下军饷总是连三成都很难拿到。这走私的事情,就是龙庭飞也未必不知道,只不过都是装聋作哑罢了,大概只有石英这个直肠子不知此事。不过事已至此,段无敌当然不会牵连旁人,所以对凌端的疑心视而不见。

秋玉飞也想到了这一点,他师兄萧桐掌管军中监察之责,这种事情若是一点都不知道,岂不是无能至极,萧桐若是知道,龙庭飞也必然知道,只是今次石英突然揭穿此事,就是龙庭飞也是无可奈何,必须将段无敌拘禁起来,这种事情是只能意会不可言传的,若是传出去龙庭飞支持走私,朝中刚正之臣必然要弹劾斥责,可是若想龙庭飞置身事外,段无敌就需要做这个替罪羊。想明白这一点之后,秋玉飞望向段无敌,眼中充满了无奈,道:“段将军,这件事情只怕在下难以求情,其实将军也是不得已,若是向大将军说明苦衷,大将军也会谅解,将军也可以戴罪立功。”

秋玉飞话中含义,段无敌心中明白,龙庭飞心有愧疚,自然不会重重加罪,可是这样以来,龙庭飞清名受损,北汉军心必然动摇,他摇头道:“四公子,末将只是在您面前才这样说,到了中军,末将只能自认贪贿,到时候大将军为了严肃军规,只能将无敌斩首或者下狱。无敌非是贪生畏死,这几年来,苏将军和谭将军相继殉国,无敌不是妄自尊大,若是没有在下防守沁州,大将军的压力九太大了,若是公子禀明国师,向大将军求情饶恕无敌性命,这样一来,虽然无敌要受些责难,可是一来不伤大将军公正廉明,二来无害军心,就是将末将贬为士卒,无敌也绝无怨言。”

秋玉飞心中一痛,道:“段将军忠义之心,玉飞感佩,请将军放心,我一定不会让大将军为难,也不会让段将军承担这样的罪名,我这就去见庭飞,先保住你的性命,再请师尊亲来求情,其实我想大将军也可能再设法赦你之罪,他不是无情无义之人。”

段无敌叹道:“大将军素来严正军法,末将不想害他蒙上污名,就是受刑而死,也是无所怨言。”

秋玉飞心中难过,却又转念一想,道:“石英是怎么回事,这种事情军中理应心照不宣,他怎会公然和你为难,将此事张扬出去,就是大将军也绝不会高兴他这样做的。”

段无敌无奈苦笑道:“这件事情末将也不明白,我和石将军虽然没有深交,也是多年袍泽,并无旧怨,前些日子,还曾请末将到飞雁楼喝酒,可是从那以后,石将军突然对末将冷言冷语,这次又突然发难,率亲信将商队截获,捉拿了末将的亲信卫士,然后便直接向大将军申诉,大将军传下令谕,召我去中军问罪,末将只带了几个亲卫前往大营,谁知石钧突然来到,说末将意欲私逃,将末将上了枷锁,打入囚车,末将也不明白为何石将军如此作为,石将军虽然爽直,却不是这样不通情理的人啊?”

秋玉飞听得出来,在说到飞雁楼的时候,段无敌语气有些古怪,他记下此事,心道,我去问问萧师兄,他必然明白其中关节,想到这里,道:“既然如此,段将军你们暂且缓行,我带着凌端先走一步,看看是否能够周旋此事。”

段无敌欣然道:“不论事成与否,末将都要谢谢四公子恩德。”

秋玉飞转身离去,上马之后直接奔向沁州城,他面色寒冷如冰,心中迷惑非常,石英和段无敌为何突然内讧,隐隐觉察到其中必有阴谋,说不定就是大雍间谍搞得鬼。秋玉飞心思百转,仔细回想在泽州所见所闻,当时他一心都在刺杀江哲上,虽然听到了一些事情,可是一来江哲等人言语含糊,二来他对沁州军情也不甚了然,所以只是如风吹过耳,并无痕迹。如今想来,却是有些异常之事。当日他行刺之前,齐王李显曾经写来书信,说有紧急军情,但是现在双方对峙,又是冰天雪地,根本不可能交战,会有什么军情这样紧急呢?突然,秋玉飞心中生出一念,按照时间推算,自己行刺之日前后,正是石英态度大变之时,莫非此事被雍军侦之,或者本就和雍军挑拨离间有关。

这个想法一生出,顿时如野火蓬勃,不能遏制,秋玉飞又想起凌端和他说过的事情,李虎被带走,据说随石英去截杀齐王、江哲的被俘军士全部被杀,凌端曾听到灭口之说,这灭得是什么口,莫非石英有变,想到这里,秋玉飞再也不能掩饰心中惊骇,又加了一鞭,他一定要赶去向龙庭飞说明此事,这件事情虽然他不甚明白,可是关系到两员大将,不能不慎重处置啊。

“朔风吹散三更雪,倩魂犹恋桃花月。梦好莫催醒,由他好处行。无端听画角,枕畔红冰薄。塞马一声嘶,残星拂大旗。”

沁州城内,最有名的烟花胜地飞雁楼中,大厅之内,客人众多,有富商贵胄,也有文人武士,最多的还是身穿便装的军中将领,一个高鬟如云的青年女子手抚琵琶,纵声高歌,虽然只是一个弱质女子,可是声如金石,坠地有声,清冽如冰。听得众人心醉神迷。

沁州乃是大将军驻军之处,自然是将领众多,飞雁楼乃是沁州第一风月之处,能够进入此楼的都是高级将领或者其他贵人,而此刻在堂上弹奏吟唱的歌女名叫青黛,数月前来到沁州,选了飞雁楼驻唱。这位青黛姑娘已经是花信年华,容貌清艳,长眉入鬓,即使是唱曲之时,神情也是冷漠如冰,曲终之后,从不多方索赏,与人交谈,也总是聊聊数语,气质更是孤傲高洁,令人不敢亵渎轻犯。她是北汉有名的歌女,歌声清冽,善唱名曲,一手琵琶,天下闻名,来往各处,每至一处都是倾动满城。此女与众不同之处就是精通剑术,身佩长剑,背负琵琶,独来独往,卖艺不卖身,若有浪荡子或者权贵想要轻薄,此女也是傲然不屈,曾因此剑伤数人,官府中人多怜她高洁,又有许多裙下之臣从中缓颊,方没有获罪入狱。青黛的身世不详,有人说此女原是世家之女,家族败亡之后不愿为人婢妾,宁可卖唱谋生,所以人颇敬之。

一曲终了,堂上掌声雷动,青黛对众人裣衽一礼,抱了琵琶离去,她素来如此,一曲终了便离开华堂。出了大厅,青黛将琵琶装入囊中,一个飞雁楼派来服侍青黛的侍女接过琵琶,低声道:“黛姐姐,石将军在小厅等你,您过去吧。”青黛点点头,冷冷道:“我卸妆之后就过去。”那个侍女连忙吩咐了另外一个小丫鬟,然后服侍着青黛回到住处。青黛歌喉出众,名声响亮,所以飞雁楼特意准备了一座小楼作为她的住处,因为青黛为人落落寡合,所以这座小楼位置较为偏僻,免得受人打扰。青黛上楼之后,对着铜镜卸去严妆,早有侍女准备好热水,她沐浴之后换上一件青色锦裘,从首饰盒中取出一支金步摇戴上,初次之外周身再无一件妆饰。她接过侍女递过来的红色大氅披上,向外走去,侍女连忙捧了琵琶跟上。走过一座石桥,苍松翠柏掩映下有一座华丽的花厅。厅前站着四个汉子,虽然也是穿着便装,可是只看他们的姿势和气度,就知道是军中勇士。见到青黛过来,那四人都是颔首为礼,青黛也轻轻裣衽,然后推门走入花厅。

这件花厅大概数丈方圆,十分宽敞明亮,一进门就可以看到一张暖炕,上面铺着红毡,暖炕上摆着一张红木炕桌,桌上摆着酒菜,地上放着一个大火炉,烟囱通向厅外,火炉上放着一个装酒的铜壶,而且火炉下面和暖炕相连,一边暖酒一边将暖炕烧得温热,室内温暖如春,石英坐在炕上饮酒,两个侍女一个烫酒,一个布菜,旁边的椅子上丢着大氅和佩刀。大概是室内比较热,石英已经除去外衫,只穿了中衣,面上带着酒气。

青黛走进了闻到浓烈的酒香,不由眉头轻皱,道:“石将军,你伤势未愈,还是不要饮酒了。”说着上前夺过酒杯,冷冷看了那两个侍女一眼,两个侍女知趣地退了下去,青黛闻到屋中酒气浓烈,走到窗前推开窗子,寒风扑面而入,顿时将酒气冲散了不少。

石英默不作声,任凭青黛拿去酒壶,他望向青黛的目光充满了炽烈的光芒,想起初次相见的情景,那时龙庭飞正率军在泽州作战,段无敌主管防务,他因为伤重不能随军,无聊之下到了飞雁楼听曲,他至今记得初见青黛,那坐在台上凝神唱曲的美丽女子,清艳中带着倔强的神情,虽然身处锦绣繁华,却是疏离冷淡得如同世外之人。虽然已经年过三旬,可是从无家室之念的石英沉沦在那双明澈幽深的眼眸中。他不顾一切向青黛求婚,愿意娶她为妻,并且誓言不会纳妾,可是青黛只是淡淡拒绝,自己追问多次,青黛终于向他说出拒绝的原因,而听闻原因之后,熊熊怒火立刻毁去了石英的理智。

青黛只是向他说说明,早在数年之前,她被强人掳走,失去了贞节,而那人的身份非同寻常,青黛拼着一死才逃出那人手中,可是虽然知道那人身份,却碍于不会被他人相信,所以青黛始终不曾说出此事。石英追问那人身份,青黛只是冷笑不语,石英无奈之下,只得常来探望,希望能够得到青黛芳心。

水滴石穿,深情感天,青黛也似乎有些软化,渐渐的,会和石英相聚小酌,神情虽然仍然孤傲,却是显得不那么拒人于千里之外了。直到前些日子,石英拉着段无敌一起到飞雁楼喝酒,谁知见到青黛之后,段无敌神情大变,忐忑不安,而青黛看到段无敌之后却是从未有过的震怒,拂袖而去,心中生疑的石英明暗探问,才从青黛口中得知,段无敌就是当日毁去青黛清白之人。石英大怒之下就要去向段无敌质问,青黛却扯住他不放,痛哭道:“妾身不过是个微贱歌女,别说此事没有人证,就是有了人证,难道还能将他怎样,别人不说我狐媚纠缠就已经是好的了,就算是大将军作主,最多不过让他娶了妾身,妾身虽然失节,可是也不愿服侍这样的恶人。”石英闻听之后,只觉得心丧如死,他想了许久之后,终于想到,若是自己设法杀了段无敌,那么青黛必然感激,这些日子以来,他看得出来,青黛于他并非无情,到时候自己诚心相求,青黛必然肯下嫁于他。当然在此之前,石英曾经试探过段无敌,可是每当他说及青黛,段无敌总是顾左右而言他,石英激愤之下,下定决心对付段无敌,而机会也很快就找到了。

看着青黛,石英欲言又止,此事还没有尘埃落定,他决定等到段无敌伏法之后再和青黛说起。两人刚说了几句话,突然有近卫进来禀报道:“将军,大将军招你前去。”这个近卫话没有说明,偷偷使了一个眼色,石英心中一动,知道段无敌果然已经被抓了回来,心中一喜,道:“青黛,军中有事,我先回去了。”

青黛微微一笑,道:“也好,不过你喝了这许多酒,去见大将军有些不妥,我方才已经让侍女去取醒酒汤了,你喝一碗再走,别忘了散散酒气。”石英听后,心中一暖,连连应诺。所以当他昂首离去之时,没有看见青黛眼中一闪而逝的寒光。罗网已经合拢,落网的猛虎再也不能脱身。

等到石英走后,青黛召来侍女,接过琵琶,十指一动,声如金石,却是名曲《十面埋伏》中的第六折,此曲虽然坊间盛传,可是能够弹得出神入化的只有聊聊数人,青黛弹了片刻,四周万籁寂静,只听得清冽的乐声回荡盘旋。青黛将第六折反复弹了数遍,方住手不弹。轻轻一叹,起身离去。

事有凑巧,飞马进城的秋玉飞恰于此时经过飞雁楼,青黛的琵琶声响遏行云,秋玉飞不由住马侧耳细听,他在音律上面才华无双,听了片刻,目中现出奕奕神光,低声道:“好一折《埋伏》,世上几人弹得,只是怎么杀气隐隐,似有绝决之意。”若依照秋玉飞本心,真想立刻去见那弹琵琶的高手,可是段无敌的事情还没有解决,他犹豫了一下,终于还是策马向大将军府邸奔去。

\chapter{第三章 有口难辩}

英得大将军宠信,千里奔袭,战功卓著,荣盛二十四年,英以私仇告发段无敌贪渎、勾结敌国商旅之罪,其时段无敌所为,乃大将军默许。英乃得罪。

——《北汉史·石英传》

大将军府内,龙庭飞负手站在堂上,心中怒火汹汹,这些日子以来,他在训练士卒、整顿兵甲的同时,也没有忘记监察麾下各将,在他心中,段无敌、石英最为可疑,这两人都是他亲信大将,石英擅长作战,于勾心斗角上面却不擅长,段无敌长于守备,虽然是北汉军最值得信任的后盾,可是不免少了些斩将立功的机会,这样一来,段无敌得到的赏赐和晋升是要落后一些的,而且段无敌性子深沉谨慎,龙庭飞本是有些怀疑他的,可是萧桐监视众将,却没有什么证据可证明两人已经和大雍有所勾结。

自从他回到沁州之后,段无敌就忙着四处调整防务,而一切的动作龙庭飞都细细留心,段无敌布下的防卫固若金汤,绝无破绽。石英本是除了打仗之外一切事情都懒得理会的,除了最近迷上一个有名的歌女之外,并没有什么特别。

那个歌女萧桐细细查过,乃是原晋阳名士苏锷之女。苏氏本来是东晋忠臣,不肯改仕北汉,在先主即位之后多有讽刺之语,最后先主一怒之下将苏氏抄家问罪,苏锷死于狱中,那是荣盛十年的事情。而青黛即是苏锷唯一的爱女,父亲死后,家产又尽被抄没,此女无依无靠,流落风尘,虽然如此,此女性情高傲,清白贞烈,颇为世人敬重。可以说此女对北汉朝廷怀有恨意,这可以从她平日行径看得出来,她几乎对北汉权贵豪门从不假以颜色,落落寡合,幸而敬重此女风范之人不少,否则她也不能安然卖艺。石英喜欢上这个女子,虽然有些不妥,可是只看她这样行径就知道她不会投靠大雍,否则绝不会放弃和权贵接近得到情报的机会,所以龙庭飞并未干涉石英和青黛之间的事情,更何况,在龙庭飞看来,石英也未必能够打动此女芳心。

两个嫌疑最大的将军却都没有反迹,龙庭飞原本已经怀疑自己是否中了敌人离间之计,谁知事情突然爆发,石英竟然突然指控段无敌勾结商旅走私,这件事情令龙庭飞颇感棘手,说句心里话,段无敌走私虽然隐秘,可是若是龙庭飞一无所知,也未免太无能了,可是段无敌所为之事,正是龙庭飞不便去做的事情,更何况所得款项全被段无敌用于补充军饷,所以龙庭飞不仅没有问罪,反而安排军需官和段无敌合作,使得那些银钱悄无声息地用于粮饷和抚恤。只不过这件事情,龙庭飞是绝对不能承认的,否则,镇守一方的大将公然违背律法,就是后主谅解此事,那些谏官也不会轻易放过他的。龙庭飞麾下众将,大多都知道一些,只有石英,一来是他性子直率,众人担心他不小心泄漏出去,二来石英不关心这些事情,所以很多人知道的事情,偏偏只有石英懵懵懂懂。所以石英突然以此发难,锋芒直指段无敌,令龙庭飞一时反应不过来,不得已只好下令拘禁段无敌。当然龙庭飞也有一点私心,在内奸未明之前,他也不介意暂时打压一下段无敌,毕竟若是段无敌谋反,那么对北汉军的打击就太大了。尽管如此,龙庭飞还是十分愤怒,因为段无敌之事揭露出来,那么就很难替他洗刷罪名,这样一来,不论段无敌是否背叛,龙庭飞都面临着会少掉一员大将的窘境,因此他对石英十分恼火,不免后悔从前过于宠信石英,纵容得他不知天高地厚。

萧桐走了进来,看着龙庭飞挺直的背影,犹豫了一下,上前道:“将军,玉飞回来了,他想立刻见你。”

龙庭飞身子一震,这段时间大雍防备森严,很难传出情报来,他还不知道秋玉飞行刺之事的结果,他从萧桐的语气中听出,刺杀并未成功,叹了口气道:“罢了,行刺一个堂堂的监军,本就是难事,玉飞平安回来就好,让他进来吧,他是否有什么紧急的事情要见我。”

萧桐道:“还是请他向将军禀明吧,这事关系到我军大将,玉飞素来和众将没有什么纠葛,他的话应该比较公正。”龙庭飞心中一惊,道:“快让他进来。”他心中充满了不祥的预感。

秋玉飞带着凌端走入大堂,凌端一望见龙庭飞,神色立刻激昂起来,他用激动敬慕的目光望着龙庭飞,在北汉军将士心中,龙庭飞本就是超越一切的神祗。他恭恭敬敬的下拜道:“小人凌端叩见大将军。”

龙庭飞目中闪过一丝疑惑,问道:“你是?”

凌端知道龙庭飞不会认得自己,毕竟自己出现在龙庭飞面前的时候都是带着面具的,想到这里,他不由又想起谭忌,忍不住泪流满面,道:“小人是谭将军麾下鬼骑近卫。”

龙庭飞惊讶地看了凌端半晌,上前将他搀起,道:“想不到谭将军还有近卫活着,凌端,你叫凌端,唉,你家将军的骨灰已经被我派人送回故里安葬,朝廷也有旌表封赏,只是可惜他不能上阵杀敌。”说到后来,龙庭飞语气中也带了悲凉,但是他很快就平静下来,又问道:“你怎么逃回来的。”

凌端看看秋玉飞,秋玉飞淡淡道:“你将一切事情都向将军禀明吧。”凌端点点头,将自己所见所闻一一道出,随后秋玉飞又补充了自己行刺之日的情景。龙庭飞听得眉头紧锁,他本是心中有所疑忌,秋玉飞和凌端所说虽然似乎没有什么特别,可是听在他和萧桐耳中,抽丝拨茧之后所显露的真相却是令两人骇然。毕竟比起段无敌来,龙庭飞更相信自己亲自提拔的石英,而萧桐也比较怀疑精明谨慎的段无敌。

对于秋玉飞他们自然全无怀疑,对于凌端却不能无疑,龙庭飞看看萧桐,萧桐会意,咳嗽了一声道:“凌端,你认为这些事情能够证明什么呢?”

凌端茫然道:“小人也不清楚,虽然石将军一向和我们将军不合,常常讽刺为难将军,可是若说石将军会生出叛逆之心,小人实在不敢相信,只是若非如此,为什么李虎他们都被斩首,比起他们,小人追随谭将军在泽州杀人如麻,就是要向泽州百姓交待,也应该斩了小人。而且江侯爷虽然不是主帅,可是小人见军中众将对江侯爷都是十分敬重,他说要将我们两人留在身边,就无人敢反对,就连齐王知道之后,也只是警告了我们几句,让我们不可忘恩负义。可是忽然之间,李虎就被强行带走处斩了,江侯爷也不阻止,我想若非是我什么都不知道,恐怕那日我也会被杀了。而且江侯爷宽宏大量,就连李虎险些杀了他都没有怪罪,如果不是干系重大,小人实在不明白他为什么这样做。”

龙庭飞和萧桐交换了一个眼色,从凌端的话中,他们听不出来一丝虚假,而且凌端的思绪有些杂乱,不像是事先编好的谎言,这说明凌端并非是投降了雍军,回来传递假情报的。秋玉飞看出两人心思,冷冷道:“我遇见凌端的时候,他已经奄奄一息,如果不是遇见我,恐怕他没命回来。”

龙庭飞和萧桐知道他的意思,若是凌端背叛了北汉,是绝对不可能落到那种境地的。就是苦肉计也要有个限度,秋玉飞既然说凌端曾经几乎死去,那么绝无虚假,如果凌端都可以瞒过秋玉飞的眼睛,秋玉飞也没有资格做京无极的嫡传弟子了。

这时,有近卫来禀报,石英已经在外面等候传见,龙庭飞心中有些犹豫,原本他招石英前来,是想弄清楚石英为何会突然向段无敌发难,可是现在他心中有了怀疑,反而担心打草惊蛇,他看了一眼萧桐,萧桐目光一闪道:“还是让他进来吧,总是要问一问的,师弟,你带着凌端先退到后面去吧。”

秋玉飞点点头,不过他淡淡道:“我在路上见过段将军,大将军、师兄,段将军虽然触犯军法,但念他也是一片苦心,还请两位给他一个机会。”

龙庭飞轻轻皱眉,叹了口气道:“我又何尝不知,不过这件事情恐怕不是这么简单,石英虽然鲁莽,但是这样的大事居然不向我请示就宣扬出去,我原本以为他是无心,现在却觉得他是有意,玉飞,我会好好处理的,凌端么,玉飞你可是有了安排?”他看得出来,秋玉飞对凌端十分亲切,所以特意问了一句。

秋玉飞道:“这小子资质品性我很喜欢,准备带他回去见见师尊和大师兄,如果大师兄也中意,我想让他拜在大师兄门下,若是不行,我就勉强收个弟子。”

听到他这样的回答,龙、萧两人都是神色一动,萧桐上前将凌端仔细打量了一会儿,笑道:“资质虽然只有中上,但是这孩子倒是坚毅不拔的性子,而且也不是过于刚直不知变通之辈,小小年纪就成了千里挑一的鬼骑,大师兄应该会中意,好,师弟好眼光。”

秋玉飞微微一笑,叫起凌端,带着他退到后面去了。

龙庭飞这才命人传石英进来,不多时,石英大步流星地走了进来,他已经逼去了身上酒气,进来之后恭恭敬敬行了军礼,道:“大将军传末将前来,可是有什么吩咐?”

龙庭飞深深的看了石英一眼,道:“石英,有一件事情我一直没有问你,如今段无敌也快被押来了,我且问你,你是怎么知道段无敌作那走私之事的,这样的大事,你为什么不事先和我商量,却在众将议事的时候当众说出,幸好段无敌没有畏罪潜逃,若是有了差池,岂不是你的罪过?”

石英犹豫了一下,道:“是末将的副将石钧无意中发觉,告诉了末将,末将愤怒之下,也来不及多想就在议事之时说了,这是末将的罪责。”说到这里,他面上露出了轻微的惭愧之色,为了报复段无敌,他跟本就没有想过私下向龙庭飞禀报,他虽然率直,却不是愚笨,这样大规模的走私,自己的属下都能查得出来,龙庭飞若是一点都不知道才怪,他只有这样做才能迫使龙庭飞斩杀段无敌。石英心中有数,虽然历来大将军对自己十分宠信,可是却更加倚重谭忌和段无敌,再说,若是从前,龙庭飞还可能严惩段无敌,现在兵势危急,想来大将军很可能会隐瞒此事,可是段无敌多活一日,青黛就一日没有欢颜,这些时日,看着她神色越发憔悴,石英已是痛彻心肺。

他神色变化虽微,但是龙庭飞和萧桐都是有心之人,两人都看的清清楚楚,龙庭飞心中一叹,道:“你在堂下等候段无敌对质吧。”石英应诺退下。龙庭飞神色一冷,道:“萧桐,石英他心中有鬼,你亲自去一趟他府上,搜查一下有没有什么不应该有的东西。”萧桐低声应喏,转身出去。龙庭飞心中大恨,一掌拍向桌案,桌上茶杯等物被震得跳了起来,茶水飞溅,堂下立刻有亲卫涌入,龙庭飞神色平静下来,道:“你们收拾一下,等到段无敌被押到之后,你们去了他的枷锁,将他带来见我,押送他的兵卒全部带到后面,不许他们胡乱行走,石钧是押送的将官吧,也将他一并带来。”

过了小半个时辰,段无敌终于被押到了。龙庭飞见到神色平静但是形容有些狼狈的段无敌,一时之间竟然不知道该说些什么,不论段无敌为了什么走私,不论自己是否默许,这件事情已经揭穿。若是说出真相,那么北汉所面临的窘境将人尽皆知,只怕军心不稳,而且违背国法的罪名也没有那么容易在朝野得到谅解,虽然国主信任器重自己,可是朝中还有许多对自己不满的势力,龙庭飞知道到了那时自己恐怕会被召回问罪。若是从前,龙庭飞倒不介意被问罪,只要自己还能领军作战,爵位和官职都不重要,可是现在是什么时候,大雍随时都可能发难,自己是一刻都离不开沁州。若是国主明显的偏袒自己,恐怕又会失去民心,对自己的声誉也有很大的影响。唯一的解决方法就是让段无敌顶罪,虽然只要自己一句话,段无敌定然会遵从,就是死也不会牵连自己,而且实际上自己也确实没有插手此事,可是让段无敌代自己受过,龙庭飞是无论如何作不出这种事情的。

段无敌心中明白龙庭飞所想,上前下拜道:“罪将叩见大将军,请大将军按照国法军规种种处置罪将,无论是何等处罚,罪将都是心甘情愿,只是如今国家在用人之际,求大将军留罪将残生,让罪将战死沙场,而不是死在刑场之上。”

龙庭飞身躯微微震动,良久才上前将段无敌扶起,深深一拜道:“段将军,这本是庭飞之过,却让将军担此污名,庭飞罪莫大焉。”

段无敌眼中闪过一丝激动的神色,肃容道:“大将军何出此言,这都是末将利欲熏心,和大将军何干。”

龙庭飞明白段无敌的心意,这件事情既然已经段无敌承担了罪责,就更不能牵扯到龙庭飞身上。他黯然直起身躯,道:“无敌,你现在一旁等候,如今还有一件事情更加重要,你在旁边听着。来人,传石钧。”

走进来的石钧神色十分不安,他颇为精明,自从路上遇到秋玉飞之后,他就不敢再为难段无敌,在最后一段路上,他心中一直打鼓。石钧本是石英族弟,少年时候就是好勇斗狠,乃是乡里有名的无赖,后来投奔石英之后,因为他心思灵巧武艺也不差,从一个小卒成了石英的副将,石英虽然骁勇善战,可是用人上面却是有些任人唯亲的,不过总算石钧知道自己的能力不足,便仗着石英信任,用小恩小惠结好军中勇士,也还勉强算是一呼百应。

前些日子,石英交待他探查段无敌的短处,石钧实在有些为难,不是因为段无敌威望身份,而是段无敌素来严谨,石钧无从下手,可是石英的命令是不能不遵从的。恰好段无敌麾下有一个将领因为犯了军法被段无敌降了职,那个将领心存怨望,寻机会滞留在沁州城,石钧得知之后便和他结识,拉着他去喝酒玩乐,这个将领对段无敌心存不满,在石钧贿赂下便露了一丝口风,说出了段无敌走私之事。石钧得知之后如获至宝,将这个消息报告给了石英。石英也是名将,既然知道这样的事情,用心之下果然不久就发现了证据,毕竟段无敌得到军中高级将领的支持和默许,所以并没有过分守密,而在石英策划之下,顺利的捉贼拿赃。整件事情都十分顺利,可是石钧有件事情却瞒着石英,在这个过程中,石钧“查到”了许多线索的情报,可是这些情报实际上不是石钧查到的,而是从一些神秘人手上得到的,如果没有这些情报,石英也不可能这样顺利的抓住段无敌的把柄。

可是现在石钧万分后悔自己的短见,想当初那些神秘人捧了金银上门,说是和段无敌勾结走私的商人和他们不和,双方在生意上面是敌手,所以想帮助石英打击段无敌,好铲除那些商人的后台,这是一个很合理的缘由,而且自己也需要这些情报,石钧就却之不恭了。可是路上的事情让石钧发觉自己的上司可能捅了一个马蜂窝,若是石英有什么不妥,自己的荣华富贵也就成空了,可是就是再后悔,也是无济于事。等到石钧押着段无敌到了大将军府,段无敌立刻就被卸了枷锁请了进去,反而石钧自己和那些军士被看押起来,石钧更是心中不安,心中盘算着如何应对。没过多久,石钧就被传去问话,他自然没有法子拒绝,只能硬着头皮走进龙庭飞召见将领的白虎节堂。一看到面色铁青,周身怒气杀机洋溢的龙庭飞,石钧只觉得几乎无法呼吸,上前几步扑通跪倒在地,身躯更是不由颤抖起来。

龙庭飞见到这种情状,心中更加怀疑起来,冷冷问道:“石钧,是你发觉了段无敌走私之事么?”

石钧小心翼翼地道:“正是末将。”

龙庭飞恨声道:“你是如何发现的,莫非你胆敢暗中监视大将么?”

石钧张口欲言,可是却无法出口,收买段无敌麾下将领和接受商人贿赂都不是可以明言的事情,若是自己说了出来,不说段无敌有罪没罪,只怕自己先被推出去斩首了。想到这里,不由额头冷汗涔涔,跪在地上,连连叩头,竟是不敢说话。

龙庭飞怒道:“你还不实话实说,若是有半句谎言,我就问你一个欺瞒主帅之罪,将你千刀万剐。”

石钧吓得面色苍白,连忙将自己如何从那名将领口中得到线索,又如何从神秘人那里得到贿赂和情报的事情说了。

龙庭飞勃然大怒,一脚踢出,将石钧踢飞到一旁,石钧口吐鲜血,却不敢擦拭,爬起来伏倒跪地,连连道:“末将知罪,求大将军饶命。”龙庭飞冷冷道:“将他带下去交给萧桐严刑盘问。”几个近卫将石钧拖了下去。

龙庭飞坐回帅位,疲惫地合上眼睛,仔细的想着石钧的口供,那些提供情报的人很可疑,他问段无敌道:“无敌,你可知有什么人会怀恨于你,而且可以得到你们走私的详细情报。”

段无敌皱眉想了片刻,道:“和末将勾结的商人都是国中大商贾,有资格做这种生意的不过两三家,末将和他们达成协议,按照一定比例共同合作,除此之外的商人就算眼馋,可是他们没有这个财力参与,而且也没有办法得到出货的情报。除非是和那些商人交易的东海商人,才可能得知我们出货的情报,不过他们怎有能力参与到北汉军务中?”

龙庭飞苦笑片刻,眼中闪过寒光,道:“怎有能力,我们都忘记了那人在东海待了将近三年,恐怕这件事情早就在他掌握当中了。”

段无敌脸色一变,他自然明白龙庭飞所说的“那人”是谁,不过他谨慎的问道:“大将军,这件事情未必如此,我们合作的商人都特意查过,应该不是大雍的人,而且我们还特意排除了海氏,就是因为海氏和东海太亲密了。那些商人身份并无问题,大多是南楚方面的人,幕后应该是南楚最神秘的天机阁,就算那人手段再高明,他也没有办法把手伸得那么长的。而且我们从南楚得到的情报应该不会有问题的,天机阁多年来多次损害大雍的利益,我们曾经怀疑天机阁的后台是南楚世家,现在我们和南楚同仇敌忾,他们不会在这个时候落井下石的。”

龙庭飞对段无敌的判断颇为信服,可是他仍然认为这次的事情必定有大雍的插手,除了大雍谁还会希望北汉军方大乱呢。想了一想,他说道:“你也说海氏和东海亲密非常,根据碧公主所说,海氏和江哲也有勾结,走私的货物都要通过滨州,海氏在那里一手遮天,恐怕蛛丝马迹难以逃过他们的眼睛,若是有心,也未必不能收集这些情报,唉。”

这次段无敌也默然,龙庭飞的判断很有道理,货物的进出果然是瞒不过海氏的耳目,莫非江哲早就在滨州布下了棋子,段无敌心中突然生出荒谬的想法,莫非江哲隐居东海,支持东海姜家和海氏将滨州发展成为远扬贸易的中心,也有引诱我们走私的用意,如今若是断了这条路线,只怕我北汉立刻陷入物资不足的困境,想到这里,段无敌突然觉得遍体生寒,却不敢将自己的想法说了出来,只能安慰自己道,江哲就是再精明,也不可能想得这么深远吧,东海之事尚是姜家作主,他不可能如臂使指的。

这时,萧桐面色凝重的进来了,他递上一个锦盒,打开盒子,里面装着两封书信。龙庭飞接过一看,顿时觉得万念俱灰。

两封书信都没有抬头和落款。

第一封书信上面写着“君之旧部,皆已灭口,君手下容情之恩已报,龙氏泽州惨败,君岂不悟,若待大军北进之际,君悔已迟,若弃暗投明,可许以侯爵之位,将军深思之。”

第二封书信上面写着“君知时势,顺天而行,乃幸事也,请先除段无敌,以表诚心,我将暗助将军行事。”

龙庭飞沉痛地道:“可问过他的亲卫了么,可是有人栽赃?”

萧桐苦涩地道:“属下仔细盘问,无人知道石英如何和大雍联系的,但是这锦盒是放在石英寝室的柜子里面的,这柜子只有石英有钥匙。而且有人留意到石英每晚睡前都会从锦盒,查看里面的信件。若是有人栽赃,至少昨夜之前那些书信不会在里面。”

龙庭飞手抚额头不语,神色冰冷阴郁,过了片刻,道:“传石英来见我。”

当石英走入堂上的时候,龙庭飞再也抑止不住心中愤怒,将锦盒和两封书信摔在石英面上,石英眼光一闪,看到书信,面上通红,道:“末将的私人书信怎会在大将军手上。”

心中存了万一之念的龙庭飞彻底失望了,他冷冷道:“那么你是承认这两封信是你的了?”

石英脸上一红,道:“正是末将所有。”

龙庭飞放声大笑,笑声中充满了悲凉之意,道:“我对你素来器重,你就这样报答我么,你可对得起王上和三军将士。”

石英心中迷惑,心道,怎么青黛亲书给我的诗词有什么关碍么?他下意识的拿起书信看去,一看之下,他目瞪口呆,再也说不出话来。

龙庭飞冷冷道:“原本我还相信你截杀齐王不成是意外,我还想你向无敌发难是为了看不惯这种贪枉之事,可是如今你要如何解释,北汉何曾亏负于你,你要叛国投敌。”

石英心中急切,想要解释,可是越是焦急却是越发难以分辩,拿着那两封书信竟是说不出话来,他伤势本就没有全好,急切之下,忍不住一口鲜血喷了出来。

\chapter{第四章 十面埋伏}

若是龙庭飞心境清明,必然会看出石英心有苦衷,可是这些日子来,他心中早已对部将起了疑心,石英向段无敌发难,又让他陷入两难境地,秋玉飞、凌端之言又让他深信不疑,所以见石英如此情状,反而越发觉得此人矫情作态,可恨至极。堂上另外两人,萧桐本就是负责监察军中将士,遇事总是爱生疑心,在他心中人不过分为两类,已经叛变的人,和将来要叛变的人,故而也没有察觉出石英心意。反而段无敌虽然遭遇这种难以翻身的境地,但他心中没有窒碍,看出石英之苦。连忙上前道:“大将军,石将军或者有些苦衷,还请大将军容他申诉,这两封书信若是雍人送来,石将军将他焚去就是,怎会留下作为证据呢?”

段无敌说得虽然有道理,龙庭飞和萧桐都是神情一动,偏偏石英心中怨恨已深,他对段无敌本就怀恨,如今龙庭飞又摆明偏袒段无敌,那两封书信也说不定是萧桐栽赃,所以他心中激愤之下,不仅没有趁势解释,反而怒道:“段无敌,不用你故作好心。”

龙庭飞听到此言心中更怒,厉声道:“将石英关入死牢,萧桐,立刻将石英亲信将领全部拘禁起来,一一盘查,看是否有人已经被他收买叛变。”他声色俱厉,段无敌和萧桐也都凛然不敢多言。石英只觉心灰意冷,看了看龙庭飞和段无敌,心道,我虽以私心告发段无敌,可是毕竟段无敌走私贪渎是实情,大将军不问此事,反而责问我如何得知,如今又拿着这两封不明不白的书信来向我问罪,罢了,既然大将军存心偏袒,我又何必还要辨白。石英本就是将生死看得极淡的人,想到这里也不向龙庭飞拜别,转身下堂,也不管身后跟上来的侍卫如何,心中满是苦楚悲愤。

龙庭飞见石英如此,心中越发气恼,但是他毕竟还是一代名将,虽然早已落入江哲彀中,心中还是隐隐觉得石英可能有些苦衷,便向萧桐问道:“萧桐,还是要仔细查一查,这段时间你应该对石英有所留心,可知道有什么人和他比较接近,说不定那两封书信真是有人栽赃也不一定。”

段无敌神色一喜,他虽然也怨恨石英无故向他发难,可是却不相信石英真的叛变投敌。

萧桐则是深思片刻道:“这件事属下也很奇怪,石将军往来之人并无嫌疑,除非是他的属下亲信有人通敌,才能替石将军和大雍交通消息,不过这也殊不可能,因为这种事情必然需要多次密商,传递消息之人必然需要常常外出,形迹必然会落入人眼,可是石将军属下没有这样形迹可疑之人,若是石将军派了数人分别传信,也断然没有可能,他纵然有了反意,也必定只能让一二亲信得知,绝不会如此不谨慎。所以这两封书信如何到了石将军手中还是难以推测。属下想来,不妨将飞雁楼青黛姑娘请来问话,今日石将军迷恋青黛姑娘之事人尽皆知,虽然属下没有发现她有什么问题,不过召她前来询问也应该有所收获。”

龙庭飞轻轻点头,一个歌女而已,无辜与否他自然不会放在心上,正要答应之时,段无敌已经神色慌乱地跪倒在地道:“大将军,青黛不会与此事相关,还请大将军不要为难于她。”

龙庭飞和萧桐都是一惊,段无敌怎会为青黛求情,两人心中疑虑,齐齐向段无敌望去,龙庭飞神色冰冷地问道:“无敌,你为何替她求情,莫非你和此女有些什么关碍,她不是石英钟情之人么?”

段无敌犹豫再三,终于说道:“末将不敢隐瞒大将军,末将在荣盛十七年被贬出京城,转戍代州,可是途中末将得罪的权贵派人追杀,末将重伤落水,性命垂危,幸得青黛姑娘相救,不避嫌疑,日夜侍奉,末将才能保住性命。这样的恩情末将不敢忘记,石将军钟情青黛,并非是她之罪,求大将军不要加罪于她。”

龙庭飞和萧桐对视一眼,萧桐似笑非笑地道:“段将军,荣盛十七年,你只有二十五岁,青黛姑娘其时也只有十七岁,莫非你们有了私情么?”

段无敌面上一红,他知道萧桐并非是盘问他的私事,只因青黛已经牵涉到石英案中,如今又和自己扯上关系,萧桐必定要盘根究底的,只是他和青黛之事乃是心中隐秘,他又担心说出之后害了青黛,不由踌躇难安,无法出口。

龙庭飞淡淡道:“你放心,我不会随意加罪于人,只要青黛姑娘并非大雍奸细,纵然受些苦楚,也不会有生命之虞的。”

段无敌心中越发担心,但是这样情景也不容他不说,只得道:“末将和青黛患难相交,日久生情,当时末将灰心仕途,我们有了婚姻之约,青黛因为家仇而对朝廷不满,所以要求末将随她隐居,最好是离开北汉,再不回头。可是末将伤愈之后遇到军中好友,他重责末将为了私情私恨辜负家国,末将乃痛悔前非,向青黛说明心意,之后我们两人发生了争执。末将希望青黛和我一起去代州,虽然代州艰苦,可是末将断然不会让青黛吃苦,而青黛也不是弱质女子,不会受不住风沙之苦。可是却被青黛严辞拒绝,她说与朝廷无恩,纵然不为敌,也不能反而为朝廷效力,坚决要求末将随她离去,也是末将忘恩负义,终于和她分道扬镳,青黛绝裾而去,从此我们两人恩断义绝。如今虽然青黛牵涉其中,可是末将承恩在前,负情在后,还请大将军看在末将面上,若是青黛与大雍无关,还请体谅她孤身飘零,不要怪罪于她。”

龙庭飞叹了口气道:“这也难怪,此女之事,我也略有所闻,她家破人亡,也难怪她对朝廷不满,若是她与石英背反之事无关,我也不会为难她。”

萧桐神色古怪地道:“大将军、段将军,我见石将军对段将军深怀恨意,近日石将军又对青黛姑娘钟情,莫非石将军知道了两位旧事,因此怀恨将军么,若是如此,石将军也未必是真的背叛,属下觉得青黛姑娘似乎有些不妥,还请段将军见谅,恐怕属下要对青黛姑娘严加盘问了。”

他这话如同冰霜一样让段无敌立刻心冷如冰,而龙庭飞却是心中一动,仔细想来,石英背叛的证据除了重重可疑迹象之外只有两封书信,若非是秋玉飞等人所见,加上石英向段无敌发难,恐怕自己也不会这般肯定石英背叛。但是这个念头一闪而过,龙庭飞心中早已相信身边有大将背叛,若不是石英,难道还是段无敌么,所以他只是冷冷道:“你去问吧,不过不要动刑,青黛既然以孤傲著称,那么必然不喜欢矫词掩饰,问清楚她是否受人指使给石英送过什么书信。”

萧桐应诺,正要出去办事,突然押送石英的两个侍卫冲了进来,高声道:“大将军不好了,石将军突然出手,将我们击晕,他逃走了。”

堂上三人都是听得呆了,谁也没有料到石英会在这时脱走,虽然龙庭飞下令将石英拘禁起来,可是毕竟还没有公开他的罪名,就算是石英真的反叛,也未必没有机会挽回龙庭飞的信任,这样突然脱走,就是龙庭飞原本相信他无辜,此刻也不会再有别的想法,更何况龙庭飞本就已经相信石英反叛之事。

龙庭飞深吸了一口气道:“传我谕令,四门紧闭,城内大索,一定要将石英生擒活捉。”

萧桐冷冷道:“大将军放心,属下和秋师弟一起出手,一定不会让他逃走。”

萧桐匆匆走下堂去,不多时,外面传来号角声,这是向四门传令,也是代表着沁州城此刻起进入军管,所有平民都必须闭门不出,三四年来,沁州从未有过这样的情势,满城军民不免人心惶惶。而在大将军府中,龙庭飞神色冰冷漠然,他真得觉得很疲倦,这些年来从军作战,他从未觉得像现在这样孤单和空虚。

苏定峦死于雍都,谭忌死在泽州,已经让他痛失臂膀,石英背叛,段无敌身陷缧绁,更让他觉得羽翼尽折,失去得力的心腹大将,龙庭飞第一次觉得再无杀敌取胜的把握。他沉默片刻,对段无敌说道:“我已决定,等到石英被擒之后,就说是他诬陷你入罪,这样一来此事谅可遮掩下去,现在正是用人之际,王上和朝中重臣也该知道轻重缓急,再说你的行事也是我默许,看在我的面上,不会有人追究此事,如今我身边四将已经只剩下你了,无敌,你不要辜负我的苦心,不可死在我的前面。”

段无敌只觉得心中一酸,泪如涌泉,虽然他不计毁誉,行那走私贪渎之事,都是为了北汉着想,可是却也知道一旦事情泄漏,自己不免要担上污名,就是不死也要失去军职,想不到龙庭飞竟然决定亲自承担罪责,这般维护爱重,自己就是一死也难以报答。他双膝跪倒,泣声道:“末将遵命,末将立誓舍身报国,捍卫江山社稷,就是粉身碎骨也不会后悔。”

龙庭飞眼中也不禁闪过泪光,他强行忍住,道:“如今时势危急,乱世见忠臣,庭飞世受国恩,龙家本是刘氏家将,如今拜将封侯,名扬天下,都是国主所赐,此恩此德,永世难忘。虽然大雍势强,可是龙家万万没有屈服的道理。而且我北汉和大雍多年交战,双方死伤无数,就是大雍几位宗亲将领,也都死在晋阳城下,一旦北汉败亡,只怕我国子民,世世代代都再也不能翻身,为人臣虏。无敌,你虽然出身寒微,又屡受挫折,可是国主、林大将军和我对你都是不薄,你不要辜负我的期望,若是有朝一日,我战死沙场,除了嘉平公主,北汉再也无人能够支撑大局,到时候你要全力襄助公主殿下,力挽狂澜,绝不能让我北汉子民死在大雍屠刀之下。”

段无敌心中一痛,道:“大将军不可这样说,虽然我国危急,可是也未必没有转机,大将军不可轻言生死,末将心中只有精忠二字,只有无敌在一日,绝对不会辜负家国。”

龙庭飞长叹一声,道:“你也去协助萧桐,一定要将石英擒回,我要知道他泄漏了多少军机出去。”段无敌应诺退下,龙庭飞手抚额头,只觉得身心俱疲。

飞雁楼中,青黛坐在厅中椅上,手持琵琶,不时拨动琴弦,却是始终断断续续,不成曲调,侍女也不敢过来打扰,只当她在谱曲,却不知青黛心中全无曲谱,她心中切切只是念着石英一人。

突然外面传来吵嚷声,侍女急切地道:“石将军,姑娘正在谱曲,说了不见客人。”话音未了,门外已经传来急匆匆的脚步声,然后门被推开了,石英神色平静的站在外面,但是青黛可以看得出他眼中深藏的灰心和绝望。

石英看向神色有些惊疑的青黛,朗声道:“青黛,我可以进去么?”

青黛本想拒绝,可是看到他的目光,不知怎么心中一软,轻声道:“将军请进。”

石英走进房间,毫无忌惮地看向青黛,室内温暖如春,此刻的她只穿着一件青色薄衫,婀娜修长的娇躯体态若隐若现,乌黑亮泽的秀发披散在肩上,越发显得娇美动人,可能是独处的缘故,她原本孤傲的神情也变得温柔缓和,使得现在的她失去了往日的冷漠傲然。石英心中悲凉,多少个夜晚心中苦思冥想,就是想见到青黛这般情态,如今得见,却是已经物是人非。

青黛轻轻簇眉,石英炽热而悲凉的目光让她心中不安,放下手中琵琶,她去拿挂在旁边的披风,可是她刚刚一动,石英已经到了她面前,然后她的娇躯就被石英紧紧抱在怀中,青黛心中一慌,就要出手反击,可是她的素手刚刚抬起,却又放下,因为她能够察觉石英心中并无情欲,石英只是紧紧的将她抱在怀里,她能够感觉到有泪水顺着自己的头发流淌。青黛素来守身如玉,虽然曾经说过自己失身于段无敌,可是实际上却仍然是处子之身,初时的紧张慌乱之后,青黛竟然觉得自己也沉迷在那强烈的男子气息当中,可是心中灵光一闪,青黛伸手推开了石英,两人之间既然如隔渊海,又何必让自己动心呢?这一次,石英没有反抗地被她推开了,他转过身去,回过头来的时候,已经看不出方才曾经流泪。石英轻笑道:“青黛,我即将远行,不知是否可以为我弹一曲琵琶。”

青黛淡淡道:“将军想听什么?”

石英心中却是从未有过的明晰,出了节堂之后,他突然想明白了很多事情,看了一眼那他永远舍不得伤害的女子,他从容地道:“青黛,我不知道为什么你亲书的诗词怎么会被人换掉,也不明白你和无敌之间有什么恩怨,甚至不知道你究竟是什么身份,可是我知道我对你不过是一厢情愿罢了,如今我已经无从辩驳,你就当是同情我,为我弹奏一曲如何?”

青黛神色一凝,轻轻拿起琵琶,却没有说话,面上神色冰寒,纤纤手指已经按在了琵琶的云头之上,那里藏着机关,可以射出毒针暗器。石英爽朗地一笑,道:“你不要多心,我如果有意伤害你,方才就会动手了,我不怪你,是我自己下了决定对付段无敌,无论如何,他走私贪贿总是实情,可惜我想不到大将军竟然偏袒他,两封书信就可以让他怀疑我的忠诚,青黛,我已经心灰意冷,临死之前,只想听你再弹一曲,这样你都不肯答应么?”

青黛眼中闪过凄然的神色,她淡淡道:“青黛愧对将军,愿为将军弹奏一曲。”

石英凝神看去,青黛神色冰冷中透着绝情,他心中一痛,知道这个女子对自己并无情意,可是只要看着那清艳如冰雪寒梅的容颜,他已经沉醉其中。

青黛坐在椅上,轻轻拨动琵琶,随着“轮拂”指法的运用,铿锵有力,激昂高亢的乐声溢满天地,动人心弦。石英轻轻叹息一声,他知道这一曲《十面埋伏》,当日他初见青黛,青黛就是弹奏此曲,也是那一面,让他从此钟情,不能自拔,青黛曾经为他讲解过此曲,所以石英心中明白这是第一折《列营》,果然是尽述人声鼎沸、擂鼓三通、军炮齐鸣、铁骑奔驰的列营情景。

继而旋律变得悠扬壮丽,令人仿佛见到军容整齐,浩浩荡荡的行军之景。之后节奏变得活泼跳跃,石英虽然只听过数次,却也知道进入了第三折《点将》。

沉醉在震动人心的乐声当中,石英仿佛不知今夕何夕,经历了《埋伏》和《小战》两折之后,终于到了此曲的精华所在,青黛十指如飞,技艺尽展,将千军万马声嘶力竭的呐喊和刀光剑影惊天动地的激战展现的淋漓尽致,石英坐正身子,这是他最爱的一折,每次听到这里他都要浮上一大白,忍不住四顾,看到窗前桌子上放着酒壶,他大踏步走了过去,也不倒酒,拿起酒壶痛饮起来。随手推开窗子,他看到几个身影闪到青松后面,他淡淡一笑,这些时候,来追捕自己的人应该已经到了外面,不知道自己能否听完这一曲。这时,曲声一变,变得阴沉悲凉,石英心中一震,这一折他从未听过,可是一瞬间他就知道这一折正是青黛从来不肯弹奏的《乌江自刎》。

青黛的性子古怪,这一曲十面埋伏,青黛从来都只弹到《九里山大战》这一折,下面那一折《乌江自刎》,青黛却是从来不曾弹过,她总是说《乌江自刎》后面的三折太累赘,她不喜欢弹,《乌江自刎》太悲凉,不吉利,所以她不肯弹。想不到今日青黛为他弹奏了此折,乌江自刎,青黛未免太抬举自己了,石英苦笑着将壶中烈酒一饮而尽。这时,石英眼中已经看到了萧桐的身影,而在他身后负手而立的黑衣青年,只看气度便知道必定是高手,不需要楚歌,已经是自知陷入了绝境。

乐声嘎然而止,青黛抬起头来,目光如同冰雪,望向石英,本以为是虚情假意,可是这个粗鲁爽朗的汉子竟然让自己真的动了心,曾经对那个负心人怨恨非常,这人是不是也会怨恨自己的负情负义呢?石英本是莽撞之人,可是此刻他心中却如明镜一般,看穿了青黛的心思,他走到青黛身边,握住她的纤手,笑道:“这不怪你,大将军本来就已经起了疑心了,否则也不会这么快就下了决定。”

青黛低声道:“刚强易折,你这又是何苦?”

石英心中一暖,知道青黛是劝自己向龙庭飞服软,解释清楚,虽然他很清楚青黛的无情,可是有这样的一丝心软已经让他心满意足。石英本性率直刚强,对他来说,龙庭飞的怀疑已经足以摧毁他的全部信念,而青黛的无情也让他再没有活下去的意志。

这时门外传来萧桐阴森的声音道:“石将军,大将军传你前去见他,你若不想连累青黛姑娘,还是自行出来吧。”

青黛心中一抖,她的手再次按上琵琶云头,如果石英改变心意,决定向龙庭飞屈膝陈情,那么自己擅自改变计划的后果就太严重了,那么唯一的办法就是当场刺杀了石英,才能挽回大局。石英却是微微一笑,朗声道:“我的事情和青黛无关,萧大人请进来说话。”

萧桐轻轻皱眉,找到石英并不困难,他跟本就没有掩饰行踪,直接就来了飞雁楼,若是此人负隅顽抗,于己不利,他不想轻身涉险,这时,房内突然传来女子的惊叫声,萧桐一惊,正要上前,身后的师弟秋玉飞已经越过自己,纵身入了青黛闺房。等到萧桐进入的时候,只见石英坐在椅上,一柄匕首深深的刺入了小腹,石英的右手按在匕首柄上。看到萧桐进入,石英微微一笑,用力一扳匕首,萧桐不忍地转过头去,他知道这样一来,石英的肺腑必然一团混乱,再无一丝生机。鲜血横流,石英沾满鲜血的左手指向青黛,道:“不要牵累她。”说罢,阖然长逝。

青黛面色苍白,从未想过这个男子身死,会让已经是无情无爱的自己,也觉得有些心痛悲伤,她拿起琵琶,十指轻动,房内响起悲怆缠绵的曲声,一曲终了,青黛拭去泪水,面色恢复冰雪一般的冷静。这时,萧桐走到她身边,客气地道:“青黛姑娘,石将军之事牵涉到姑娘,还请姑娘暂时和我们回去,如果姑娘并无牵连,我们会很快还姑娘自由之身。”青黛淡淡道:“妾身敢不从命,请容妾身更衣。”

\chapter{第五章 恩断情绝}

其时,英以叛逆之罪下狱,未入狱而脱走,大将军下令拘捕,英自戕死,大将军余怒未息,草草葬于沁州北郊。

荣盛二十五年,北汉亡,大雍齐王昭示天下,英无辜被戮之情乃为世人所知。

——《北汉史·石英传》

秋玉飞站在回廊之上,听着轩内如同行云流水一般的琵琶声,只觉得心旷神怡。

青黛姑娘被带进大将军府后,萧桐盘问之时,秋玉飞隐在暗中,他对青黛的才貌颇为爱惜,尤其是她这一手好琵琶,担心萧桐辣手摧花,故而暗中维护。不论萧桐如何软硬兼施,青黛只是冷冷应对,就是秋玉飞也能够看得到此女对北汉朝廷的恨意。对萧桐,她是冷淡疏离,提及段无敌,她是带着恨意,而提及石英,她的神情却是惆怅而歉疚,秋玉飞能够体会到她的心思,她对石英或者并无深情,可是石英的痴情却令她十分感动。这样一个女子若是大雍密谍,也未免太不称职了,只凭她的性子,就不适合做谍探。

在萧桐初步肯定此女无辜之后,却没有将她释放,一来是想仔细查清楚此女过往,另外龙庭飞也暗示他留下青黛,段无敌多年来军务繁忙,并没有成婚,见他昨日情急,便知道他对青黛并没有忘怀,如果能够让他们重归于好,也未免不是一段佳话。不过数日来,段无敌军务繁忙,石英死后,安抚他的旧部,处理走私一案的善后,都不是简单的事情,段无敌几乎没有时间来和青黛见面,可是秋玉飞却隐隐觉得两人之间怕是没有可能,因为青黛数日来除了弹奏琵琶之外就是静静发呆,从未要求和段无敌见面,非若是碍着段无敌,秋玉飞倒想和青黛在音律上探讨一番。

身后传来低低的脚步声,秋玉飞只听步伐,就知道是凌端来了,也不回头,耳边传来凌端的声音道:“四爷,酒来了,四爷听曲听得入迷,就不想去见见青黛姑娘么?”

秋玉飞回头白了凌端一眼,见他脸上带着古怪的笑容,伸手给了他一个蹦栗,凌端作出一副苦脸来,自从石英死后,凌端觉得谭将军和李虎的仇恨已经报了,心中再也没有挂碍,也恢复了从前的开朗。秋玉飞见他神色古怪,轻叱道:“胡说,君子不夺人所爱,段将军和青黛姑娘曾有婚姻之约,虽然中途分道扬镳,不过我看他们并未忘情,再说,我敬佩青黛姑娘的人品才华,可不是有心求凰。”

这时凌端远远的看见段无敌缓步走来,连忙拉了一下秋玉飞的衣襟,秋玉飞心想不便让他看见,连忙拉着凌端隐入假山之后。只见段无敌站在门前犹豫不决,几次伸手想要推门,却都放下了手。这时,门内传来一个清冷冰寒的声音道:“是段将军么,请进。”

秋玉飞微微一笑,转身离去,他可不想牵涉到人家的私情里面,凌端却是心中好奇,他年纪不大,也没有那么多顾忌,见秋玉飞已经远去,便掩到窗下偷听里面的谈话。若是从前,他的举动自然瞒不过里面的段无敌和青黛,可是如今久别重逢的两人都是心中激荡,全没留心外面有人在偷听。

青黛见到段无敌走进,并没有站起相迎,仍然手抚琵琶,不时轻轻拨动琴弦。段无敌站在门口,望着青黛,心中感慨万千,那时的青黛不似如今这般冷淡清艳,如果说如今的她如同冬日寒梅一般傲雪怒放,当日的她就像雨后的梨花一般孤洁动人。

青黛的目光落到段无敌身上,整整七年了,当日的青年将军如今已经是成熟稳重的中年人,那曾令自己动心之处仍然存在,可是两人之间却是已经如隔渊海,七年前,自己还只是一个茫然不知所措的少女,除了有着对北汉朝廷的深切恨意之外,就连如何报复也想不出来。当日遇到段无敌,她是真心想和他共携白首,可是此人心中终究是只有一个忠字,两人就这样分道扬镳,他去做他的北汉忠臣,自己却走上了另外一条道路。青黛,原本的北汉名门闺秀苏青,如今已经是大雍兵部司闻曹下辖的北郡司北汉谍报网的总哨,大雍武林盛传的四大青年高手——娥眉青衫,已经不可能和北汉的铁壁将军段无敌有什么私情存在了。

段无敌见青黛始终沉默不语,终于开口道:“青黛,多年不见,你受苦了,这么多年难道你没有遇到钟情之人,以你的才华容貌,理应早择佳婿才是。”

青黛别过头去,冷冷道:“石将军对青黛有意,不是已经被你们迫死了么。”

段无敌连忙道:“青黛,你听我解释,当日我见到石英钟情于你,就刻意避开,我知道你绝不会原谅我,石英性情率朗,你若嫁了给他,定然能够幸福,可是我也料不到他会叛国投敌,更想不到他会自戕。”

青黛冷冷拨动琵琶,道:“你不必多说,石将军对我青睐,并非代表我就要下嫁给他,不过他为人至情至性,比起你这种人来说好得多了。”

段无敌叹了一口气道:“你还怪我么?”

青黛漠然道:“曾经怪过你的,当日我离开你之后,只觉得人生无趣,因此闯入深山,只想默默死去,若非得到恩师相救,青黛早就死在野兽口中,后来青黛想通了,我恨北汉,你忠于北汉,这本是不可调和的矛盾,不是你错,不是我错,只不过当初我们忽略了两人之间的分歧。”

段无敌摇头道:“不是你错,是我的错,当日你很早就告诉我你的心意,我也答应了随你隐居,可是我出尔反尔,伤害了你,你至今未嫁,我心中万分愧疚,只是青黛,如今已经是这么多年过去了,难道你对北汉还这么怨恨么,那是国事,无关私仇,你又何必如此念念不忘。”

青黛面上露出讥诮的笑容道:“国事,私仇,我只知道我的族人死得死,散得散,都是因为国主的谕旨,我母亲死于贫病,我被迫青楼卖唱,都是因为北汉。我至今仍然留在北汉不肯离去,就是想看到北汉亡国的那一天,这才遂我心愿。”

“啪”一声清脆的耳光响起,段无敌出手之后,看到青黛素颜上面的红肿,不由愧疚地道:“青黛,抱歉,我不该对你动手,你不该说这些话,如果别人听到,你会被当作奸细的。而且你不该——”

青黛截住他的话语,道:“而且我不该当着你北汉将军的面说这种丧气话,是不是,这些年来,民间困苦不堪,除了少数豪强豪门仍然锦衣玉食,百姓也没有得到什么好处,更别说安居乐业,就是北汉亡了又有什么了不起。”

段无敌面色沉重,道:“青黛,这里是大将军府。”

青黛冷哼一声,别过脸去。段无敌道:“今日我只当没有听见你的说话,你应该清楚一二,如今上至王室,下至庶民,除了少数权贵之外,谁不是拼死一战。亡国奴的惨状,谁不清楚,大雍和北汉积怨已深,如果北汉亡国,那么我们的子民只怕是数代都不能翻身,这场战争必须打下去,就是我们最后惨败,也要让大雍损失惨重。到了那时,大雍就是灭亡了北汉,也不敢对我们的子民过分迫害,他会永远担心我们的子民揭竿而起。青黛,这些话我只对你说,北汉如今的确形势危急,不战是死,战可能也是死,可是我们不得不战。我们若能胜了最好,若是不胜,也要让大雍永远记得北汉勇士的可怕,只有这样,才能保住我们的子民不会被人屈辱,你也熟读经史,难道不记得东晋立国的时候,代州、晋阳、沁州归顺之后,整整百余年,我们这里的赋税要比别处重三成,蛮族时而入侵,东晋派来的官员刻意盘剥,直到百年后,状况才有所好转,青黛,你也想我们的乡亲受这样的苦么?”

青黛没有辩驳,若是北汉战败,将来大雍如何对待北汉的亡国子民,这不是她可以决定的事情,而且就是大雍善待北汉百姓,北汉王族和文臣武将也是下场堪忧,只凭这一点,北汉就不会轻易放弃作战。更何况以目前的局势,大雍也未必就能稳操胜算。不过她最感兴趣的是,是否段无敌真的这样悲观,如果北汉这样身份的大将都是这样的心情,那么大雍的胜算就又多了一些。想到这里,青黛不由心中苦笑,多年来的历练,让自己无时无刻都保持着冷静,就算是方才的“失态”也不过是加深自己在段无敌眼中的孤傲印象,凭着这样的印象,就可以让段无敌不会想到自己是奸细的可能。

见她不再说话,段无敌歉意地道:“青黛,我知道你不会原谅我的,过几日我会向大将军请求放你自由,这几日你先好好休息一下吧。”

青黛心中一惊,被滞留在这里并非是她所愿,她知道萧桐仍然没有放弃追查自己,虽然自己多年来谨慎小心,可是还是有些说不清的行踪,为了安全,自己还是应该尽早离开才行。想到这里,她冷冷道:“石将军可下葬了么?”

段无敌犹豫了一下道:“石将军葬在北郊,大将军很是恼怒,所以只命人草草安葬。”

青黛低头道:“石将军生前待我情深意重,我想去祭拜于他,不知道可不可以。”

段无敌心中一酸,虽然早已经不敢存着和青黛破镜重圆的奢望,可是见青黛对石英颇有情意,仍然让他心中有些不快,可是他毕竟早已放下此事,想了一想道:“也好,明日我应该无事,就让我陪你去拜祭石将军吧。”

青黛微微点头,有段无敌相陪最好不过,她重新拿起琵琶,十指轻抚,悲怆的乐声响起,段无敌知道青黛已经不想再和自己说话,他深深的看了青黛一眼,要将这个女子的容颜铭刻于心,然后转身走了出去,隔绝在两人之间的鸿沟是不可能填平了,他只希望大将军不会怪罪青黛,毕竟在现在的情势下,杀死一个心存恨意的歌女,这是谁也无法反对的。

望着段无敌的背影,青黛轻轻叹了口气,如果当初两人没有分开,或者不会有今日敌对的局面吧,自己怎能说无恨,若非是存心报复,自己何必擅自更改计划呢。原本上面传来的命令,让自己安排栽赃石英投敌的证据,然后放出段无敌走私军需,叛国投敌的流言,最后谨慎安排,将线索牵引到石英身上,这个任务虽然有难度,但是大雍军方在沁州暗藏的势力足以做到。可是当青黛亲自前来安排此事的时候,意外发生了,石英居然对自己一见钟情,而在飞雁楼邂逅段无敌之后,更激起了她心中怨恨,所以她选择了自己也难以控制的计策,故意挑拨石英对段无敌的嫉妒,然后安排石英得到她提供的情报,让他对段无敌开始攻击。原本上面的要求是要让石英蒙上嫌疑,段无敌名声受些损伤就可以了,可是自己的所为,让段无敌几乎被问罪,而石英也惨死在飞雁楼,如果不是石英性子果然如同上面所说,只怕自己此举必然失败,幸好最后还是成功了,可是自己也被软禁起来,如今想来还是后怕不已。

青黛不知道自己是否做的太过火了,只怕回去之后会受到责难惩罚。但是能够看到段无敌的窘境,却让她更是欢喜。不过这都是过眼云烟了,今日两人相见之后,青黛知道,自己真得不再恨段无敌,理念上的分歧本就不是情爱可以掩盖的,当初就算段无敌和自己一起隐居,也终有分道扬镳的一日。

幽幽一叹,青黛又想起了石英的音容笑貌,想起当初自己赴泽州大营向江大人述职的时候,那个温和淡然的青年一针见血的评价道:“石英此人,虽然是有数的名将,却是少受挫折,他从军不久就得到龙庭飞赏识,从此以后几乎是一帆风顺,在龙庭飞庇护之下,有很多阴暗之事,他都不甚明了,而且此人性子有一不好处,就是受不得委屈,尤其是不能容忍有人对他怀疑不信任,只要让龙庭飞怀疑他有投敌的可能,此人必然忿忿不平,只要稍加引导,就会做出些不可收拾的事情来,到时候,就是龙庭飞想不怀疑他,都不可能了。”那位江大人果然看人极准,若非是石英这样的性情,若是他向龙庭飞宛转陈情,只怕死得就是自己了。不过即使以自己如今的铁石心肠,也不免对他生出怜悯情意,这次虽然说是自己要想脱身寻的借口,不过却也是真心想祭拜于他,这样一个人,就是自己也不免动心的。

寒风萧萧,天地间一片苍茫,站在石英简陋的墓前,青黛心中有一种说不出的悲凉,焚化了纸钱之后,段无敌轻声道:“青黛,回去吧,天寒地冻,不可久留,你今日来看石将军,他在泉下知道,也必然瞑目。”

青黛微微苦笑,只怕石英英魂有灵,得知自己如何陷害欺骗于他,想要瞑目可就难了,她将特意带来的酒壶中的烈酒倒在坟上,心中默默祝祷道:“石将军,青黛害你英名受污,也是不得已,等到大雍一统天下之时,青黛必然想法设法为你洗清冤屈。”祝祷已毕,青黛取下背上琵琶,就在寒风当中弹奏起了几乎从来不弹的《十面埋伏》的最后一折——《回营》。

段无敌也没有觉得奇怪,再见青黛之后,他就发觉青黛似乎对于琵琶有着近似痴狂的喜爱,几乎不肯离身,而且她在石英坟前弹奏琵琶也是理所当然。可是就在乐声嘎然而止的时候,段无敌耳边突然传来呼啸声,他下意识地回头看去,身后的两个亲卫已经惨呼倒地,咽喉上插着黑色翎箭。而在三十丈外,十几个黑衣骑士都是黑巾蒙面,背负雕弓,冷森森地望着自己。段无敌心中一惊,怎会有刺客袭击,莫非是石英属下有人怀恨在心么,不由后悔只带了两个亲卫出来。他拔出腰刀,护在青黛身前,低声道:“上马,我们冲出去。”谁知青黛轻声一叹,段无敌只觉得一缕真气透体而入,强烈的麻痹感让他再也站立不住,软软倒在地上。然后一双素手将他扶起,让他倚着石英坟墓坐起,青黛那冷若冰霜的清艳面容落入他的眼中。

段无敌突然明白了很多事情,为什么石英会对自己如此愤恨,为什么他会死在飞雁楼,他厉声道:“青黛,你莫非已经投靠了大雍么?”

青黛眼中闪过冰寒的光芒,这时,一个黑衣骑士提着包裹下马走来,道:“小姐,请速速更衣,我们不能久留,必须赶在有人发觉之前离开沁州城。”他的声音清脆悦耳,再看他身形,就知道是一个女子。青黛将琵琶交给她接着,拿了包裹走到石碑之后,不多时已经换了黑色男式骑装出来,接过另外一个黑衣人递过来的黑色大氅。此刻的青黛,身穿男装,腰悬长剑,神色凛然,不再是青楼卖唱的歌女,而是统领千余密谍的北汉情报网总哨——娥眉青衫苏青。

她走到段无敌身前,漠然道:“七年前你绝情如此,令我险些自尽在山谷,可是我终于活了下来,既然你如此忠心北汉,我也没有话说,只有选择了这条路,北汉不亡,我今生不能瞑目,无敌,如今你我已经是陌路之敌,虽然知道不可能,我还是要问你一句,你肯不肯归降大雍?”

段无敌冷笑道:“你既然知我忠心,叛国投敌之事怎会去做,青黛,我为私情蒙蔽,如今想来,可是你挑拨石英向我发难,你是存心如此吗?石将军是真的叛变还是被你陷害。”

青黛轻轻叹气,早知道段无敌不会归降,既然此人不能杀,那么就只有继续诬陷石英了,她神色间流露出愤怒之情,道:“石英比你识趣得多,若非是他因我之故擅自向你挑衅,我大雍也不会失去这样的绝好内应。”

段无敌心中叹息,自己已经成了阶下之囚,石英业已自尽,青黛既然这样说,那么石英果然是叛国之人了,他勉力抬起头,道:“青黛,我不怪你投靠大雍,你心有仇恨,如此作为也是理所当然,不过我段无敌却是绝不会屈膝投降,你若看在昔日情分,就给我一个痛快吧。”

青黛冷冷道:“你放心,我本就没有想着将你擒去大雍,你的性子我清楚,左右都是死,何必让你多受一番屈辱呢?”

段无敌心中略安,道:“也好,既然如此,我昔日欠你的也可用性命偿还,从此你我恩怨两消。”说罢闭上双目,只待青黛动手。

青黛手抚剑柄,心中一痛,喃喃的道:“恩断情绝,也好,也好,终究有这一日。”说罢举剑向段无敌刺去。这时,那黑衣蒙面的女子突然拔剑出鞘,挡住了青黛的长剑。段无敌听得声音有异,睁开眼睛,看到这样情景,心中有些奇怪,神色却依旧从容自若。青黛见他神情,心中一软,昔日深情涌上心头,心道,就是无人拦阻,这一剑我难道真的能够刺下去么?

那个拦阻青黛的女子道:“小姐,你因为私心令石将军身死,若是能够将段将军带回去,或者还能将功赎罪,若是杀了他,未免太可惜了。”

青黛心中一动,虽然因为自己只能凭着琵琶曲调传出消息,启动事先约定的计划,所以自己的亲信助手只知道要保着段无敌性命,不过她所说的理由却非虚假,自己这次擅自改变计划,虽然结果更加圆满,只怕上面也会怪罪下来,可惜自己只能担着了。故意望望段无敌,见他神色间已经隐隐有了不安,知道他唯恐自己真的将他掳走。她心中微微苦笑,真是当局者迷,自己可没有本事带着一个俘虏返回大雍。但是戏还是要演完的。她故意按剑不语,片刻终于叹息道:“我既已犯下大错,也不奢望将功赎罪,此人毕竟是诚心诚意待我,若没有他相助,我恐怕会陷身沁州,不能生还,罢了,我宁可拼着一死也要偿还他的恩情,留他在此,我们走吧。”

另一个黑衣人策马出列道:“小姐,此人乃是北汉大将,若不杀之,日后恐怕此人会杀害我无数将士,小姐岂可因为私情纵之。”

青黛扬眉道:“这里的事情还论不到你来作主,此事我既已决定,上面怪罪下来,自有我一人承担。”

这时,一个黑衣人骑马奔来,高声道:“小姐,不好了,萧桐和秋玉飞快马向这里赶来,小姐我们快走吧。”青黛接了属下递过来的马缰,翻身上马,对段无敌冷冷道:“你我从此再无瓜葛,他日青黛若是幸而不死,和你沙场相见,你也不用手下留情。”说罢策马扬鞭而去,那个被青黛斥责的黑衣人悻悻望了段无敌一眼,也策马跟去。而那个黑衣蒙面女子却是最后动身,饱含杀机的目光在段无敌面上转了一转,终于离去,离去之前她的右手在身后弹出一枚双锋针,射入段无敌身躯。段无敌微微苦笑,听到马蹄声渐渐远去,然后他听到从沁州城方向传来的急促马蹄声,中针之处生出异样的麻痒,一阵头晕目眩的感觉传来,段无敌渐渐失去了意识。

\chapter{第六章 大战前夕}

同泰五年,元月,大雍使臣苟廉谒见,廉以重金赂群臣,时王年幼,丞相尚维钧把持朝政,廉数以密谈,尚相畏陆灿功高,乃约束其不许出战,致令坐失良机,此诚莫赦之罪也。

——《南朝楚史·楚愍王传》

望着手上的情报,我几乎是呻吟着将它看完,齐王可是拿着情报对我说道:“随云,没想到你的计策真够阴毒,这样就让龙庭飞麾下的大将一死一伤。”我只能苍白无力地辩解,这可不是我的安排,事实上,北汉总哨苏青的计策比我安排得更加狠毒更加凶险,而结果也更加完美,不仅达到了陷害石英、抹黑段无敌的目的,还顺便打击了龙庭飞的威信。如果不是苏青在带着一些密谍高手返回泽州途中被秋玉飞缀上,虽然靠着苏青出类拔萃的武功,和密谍高手的苦战,以及泽州派去的接应及时,终于逼退了秋玉飞,但是却付出了惨重的代价,这次的计划真的被苏青演绎的非常完美。

不过我心虚的想到,这好像不是苏青的责任,秋玉飞正是被我放走的,虽然不知道此人怎么突然成了先天级高手,可是好像是我的责任,才让苏青损失惨重的。说起来魔宗虽然是北汉的助力,可是京无极只能作个威慑力量罢了,像他这种身份的人物,若是亲自出手杀敌或者刺杀,只怕北汉军民都会觉得北汉大厦将倾了,而且京无极不动手,我们这边的宗师级高手也不会出动,所以不到紧要关头,京无极不会出手。比较起来,魔宗其他弟子对我们的威胁更大呢,就像秋玉飞,谁会想到他突然武功大进,晋入先天极数,这也怪不得苏青失误。小小的后悔了一番,不过秋玉飞终究是不能杀的,我也只得放下既成的事实,准备善后了。我决定将苏青召入中军,毕竟很快大军就要进攻北汉了,既然苏青身份已经泄漏,那么留在中军参赞更合适一下,这个女子,真的不简单,能够在北汉多年不漏半点破绽,这次身入虎穴,欲盖弥彰的手段用得炉火纯青,真是令我佩服的很。

将情报整理好,我吩咐呼延寿传苏青进来。换了一身青衫男装,虽然仍是婀娜多姿,却是如同冬日寒梅一般铁骨铮铮的苏青神色漠然地走进我的营帐,拜倒叩首道:“属下苏青,叩见楚乡侯监军大人,属下违背大人谕令,擅自更改计划,连累众多同僚遇难,还请大人治罪。”说罢轻轻咳嗽了几声,面色更加苍白如雪。

我赞叹的看了此女一眼,这是奇女子,六年前曾在大雍江湖上昙花一现,一身青色儒衫,却不曾掩饰女子身份,手段狠辣,却又光明磊落,不曾以真面目见人,短短半年就声名大振,然后便投靠雍王,自请赴北汉为密谍,功劳卓著,数年内就成了北汉总哨,不论才华忠心,都是密谍中首屈一指的人物,今次立下大功,但见她神色间既没有丝毫得意之色,也没有因为擅自违令而担忧失措,娥眉青衫,果然是非同一般。

苏青心中并非表现出来的那般冷静,其实也是忐忑不安,这位江大人虽然言辞温文儒雅,但是她身为北汉密谍总哨,自然对朝廷内幕知道的极多,此人手段如何,她心知肚明,若非是她和段无敌之间有纠葛,而且石英又意外迷恋自己,她是万万不敢擅自更改计划的。可是计划成功之后,她反而更担心自己的结局,智深者往往最恶事情脱出控制,自己所为只怕触犯此人逆鳞,他也不用网罗罪名,只凭自己属下精英被秋玉飞杀死杀伤半数,就可以加罪自己了。

我却不会想到她的心思,对我来说,属下之人能够随机应变,那是最好不过,不过既然有胆子改变计划,就要承担后果,若是败了自然要重重惩罚,若是胜了就当奖赏,苏青所得胜过所失,我自然要赏的。轻轻叹了一口气,我道:“这不是你的责任,虽然你擅自改变计划,可是却比我预想的效果要好,而且你牺牲良多,本侯怎会怪罪你,至于秋玉飞追杀之事,也是事先预料不到的,这次总算是得大于失,你也不用过分自责,我让小顺子送去的伤药你服了没有?”

苏青眼中闪过感激的神色,道:“属下多谢大人不罪之恩,伤药很有效。”

小顺子插话道:“苏总哨,等你伤愈之后,我要和你交手,看一下秋玉飞如今的身手如何。”

苏青爽快地道:“属下只接了秋玉飞百招,就落败受伤,属下无能,还请大人和李爷恕罪。”

我深吸了一口气,在小顺子猜测秋玉飞晋入先天之境后,我就心中不安,不过苏青一个女子,能够接下一个先天高手百招,这种武功已经不简单,可真是女中豪杰,只是至今仍然小姑独处,真是可怜可惜,我心中想着是否也可能替她说个媒,却不敢流露出这样的想法,免她以为我轻浮,只是道:“苏总哨,如今北汉必然全力清剿我方密谍,而且如今大战在即,你也不用回去了,等到我军进攻北汉之时,你再随军出发吧,指挥我方潜伏的密谍,掌控情报,我方的斥候营也交给你管理,你可愿意。”

苏青神色一喜,能够得到这样的重用,是她回来之前没有料到的,连忙叩谢道:“多谢大人厚爱,属下必定竭尽全力。”

等到苏青退去之后,我松了口气,对小顺子道:“事情如今已经安排的差不多了,大战在即,去请齐王、宣将军、荆迟过来,我们得商议一下如何进攻北汉了,还有,赤骥什么时候过来?东川和南楚有情报过来了么?”

小顺子道:“赤骥奉了公子谕令,去南楚整顿情报网,发觉这次之所以没有得到凤仪门异动的情报,实在是因为这次韦膺手段隐蔽,天机阁又不便过分插手的缘故,赤骥已经安排好了对凤仪门的监视,想来不会有这次的纰漏了,另外寒总管也没有因为东川的事情生出异心,所以赤骥已经动身赶来泽州了,预计这两三天就会到达。董缺已经到了东川,陈稹感激公子恩惠,他也不信庆王的承诺,而且他不像寒无计,对蜀国没有多少旧情,所以东川的局势已经稳定,现在已经和庆王达成了协议,相信很快就可以进入庆王势力的核心。不过若是庆王发动太快的话,只怕他们来不及控制庆王的要害。”

我淡淡道:“这个你放心,夏侯沅峰不是吃素的,他已经开始对庆王下手,让董缺和他联系,庆王依靠的力量损失惨重,才能让他更加依赖锦绣盟,如果庆王想见霍纪城,就说霍纪城不便出面,什么时候庆王扯起反旗,霍纪城才能出现,反正庆王也应该知道原蜀国的势力不会完全相信他的。”

小顺子噗哧一笑,道:“何止夏侯沅峰不是吃素的,皇上也不是吃素的,他让石大人写来的书信,就差没有明着说让你赶快献策了。”

我苦笑着道:“不知道我是不是前辈子欠了他们兄弟什么,我自负聪明,偏这两个人可以轻易看穿我。”

这时帐外传来爽朗的笑声道:“说什么呢,皇上若是能够看穿你,就不会总是吃瘪了,天下有几个做主君的像皇上一样,总是由着你的性子,什么事情,你不说皇上就不问,这样的宠信,让我都嫉妒呢。”然后齐王大步走了进来,挤眉弄眼地道:“随云,你对苏青很怜惜呢,怎么样,要不要我为你作伐,长乐贤惠得很,不会怪你的。”

我正色道:“殿下不要胡说,若是苏姑娘听见岂不是心灰意冷,她可不是以色事人之辈。”

李显被我硬顶了回去,赧然道:“我也是好心,苏青至今仍然孤身一人,一个女子这样苦撑,本王也看不过去,她这样心机手段,若非是你,谁能消受得起?”

我冷冷道:“我都不是殿下对手呢,何况是她,干脆我请长乐去向皇上禀明,将她许给王爷为妃如何?”

李显吓了一跳道:“别别,我只是开玩笑,这个苏青恐怖得很,我可不敢冒犯,再说如今她是三品的将军身份,可不能拿她开玩笑。”

我瞪了齐王一眼,也不知是谁先开的话头,不过我又奇怪地道:“我正想让小顺子派人去请殿下和宣松、荆迟呢,怎么殿下倒先来了,可是有什么事情么?”

齐王正色道:“也没有什么事情,不过是想和你商量一下进军的事情。”

我笑道:“在下也正有此想,等到两位将军到了之后我们再谈吧,不过这些事情,殿下足可应付,哲只能听听罢了。”齐王道:“我来的时候已经派人去传他们了,很快就会到了。”这时,帐外有侍卫禀道:“荆将军、宣将军求见。”

我和齐王相视一笑,大举进攻北汉迫在眉睫,决定大雍命运的一战即将开始,这一战若能速战速决,天下再也无人能够阻挡大雍一统天下的步伐,若是陷入长期作战的泥潭,那么就是大雍被群起而攻的局面,这一战,至关重要啊。

南楚,陆灿愤怒地将诏书掷到地上,本已计划好,一旦大雍北汉开战,那么自己立刻将蜀中的防务交给下属,自己亲率大军渡江攻击大雍,这是南楚唯一一次夺取天下的机会,错过这一次,没有了北汉铁骑牵制,南楚最多不过能够偏安江南罢了,可是雍使苟廉却用金钱和恫吓轻而易举的吓住了朝中群臣。望着那封阻止自己出战的诏书,陆灿真的觉得浑身无力。

这时,有人禀道:“将军,辰堂首座求见。”陆灿皱皱眉,心道,韦膺怎会前来,他伤势尚未痊愈,而且因为东海惨败,他的很多权力被凤舞堂和仪凰堂分割,如今正是韬光养晦的时候,他怎会前来和自己相见呢?不过虽然鄙夷韦膺的为人,但是对他的才华还是颇为看重的。陆灿传令让韦膺进来。

韦膺神色有些憔悴,毕竟从火海中脱身并不是那么容易的,一路上又遭遇大雍的追缉,能够安全回到南楚已经是非常不易了。他从容地向陆灿行了一礼,笑道:“陆将军想必是十分头痛,不知道在下可否有所谏言呢?”

陆灿淡淡道:“韦首座有何高见,朝廷已经有了旨意,本将军难道还能抗旨不成。”

韦膺笑道:“将军也太迂了,抗旨有什么要紧,令尊早已不问军事,南楚三分军权,将军掌握二分,荆襄守将容渊声威不如将军,平素也多听将军调遣,将军若是有心,我愿助将军清君侧,除去误国奸相,从此将军便可以大展宏图,膺也可以附诸骥尾,得报大仇。韦膺此心,天日可表,不知将军意下如何?”

陆灿拍案而起,斥道:“韦膺,你怎可出此无父无君之言,当初你们落难至此,若非尚相和王上恩德,你们焉能在南楚立足,如今刚刚得势,就像行此大逆不道之事,别怪我翻脸无情,绑了你送去给尚相,让他看看你们凤仪门的丑恶面目。”

他这一大怒,帐外的卫士拿着兵器冲了进来,陆灿的亲卫长冷冷的看了一眼韦膺,道:“将军,可是这人冒犯将军么,请将军示下。”

韦膺面上带着讥诮的笑容,道:“陆将军,要杀要绑也得等到在下说完肺腑之言啊,难不成将军不敢听在下的妄言么?”

陆灿面色一沉,挥手令亲卫退去,道:“韦膺,南楚不是大雍,本将军希望你好自为之。”

韦膺微微一笑,道:“将军可想知道苟廉和尚相密谈的内容?”

陆灿心中一惊,道:“你怎会知道这等机密大事?”

韦膺没有回答,模仿苟廉的语气道:“相爷乃国主外祖,警缨世家,此诚贵不可言,然国统存亡不在文臣,而在统兵大将,如今贵国兵权三分,陆公父子掌握二分,荆襄守将容渊掌握一分,相爷手中之兵不过可以控制建业一城而已,比起陆信陆公爷、陆灿陆将军和容将军来说,可以忽略不计,陆公爷虽然忠君爱国,但是总不会和自己的儿子为难,容将军也多听从陆将军之命,若是陆将军起意谋反,则贵国社稷顷刻颠覆,就是陆将军心无反意,相爷也要早做提防。如今我大雍有事北疆,陆将军少年轻浮,不惧螳臂当车之险,竟然意图渡江攻我,若彼败,我大雍皇帝盛怒之下,北疆事了,必定兴师问罪,到时两国交兵,血流成河,不免重现昔日惨状,何况贵国王上尚有兄弟在我国为质,若是皇帝震怒之下,改立新王,则贵国王上和相爷如何自处,若彼胜,不过是我大雍两面作战,不得已暂时退却,大雍兵甲百万,钱粮丰足,纵使一时落败,也不会伤害元气,而陆将军挟大胜余威,功高震主,即使陆将军本无反心,只怕到时也难免不生异心。相爷每每掣肘陆将军,到时候陆将军竖起清君侧的大旗,只怕南楚上下一呼百应,相爷不免死无葬身之地,就是贵国王室,恐怕也会遭到池鱼之秧。由此可见,两国交兵,不论胜负,于相爷都是无利可图,相爷不过是为了荣华富贵,一旦兵戈蔽日,相爷权势皆成泡影。为相爷计,莫过和议,昔日贵国战败,曾经立约年年赔款,至今贵国军民仍然深受其害,若是相爷以此为条件和我国和议,我国陛下为了北疆战事,必定同意减免赔款,到时候朝野必定赞誉相爷功劳,岂不胜过交兵之害。若是相爷同意,我国还可以与贵国重结秦晋之好,我陛下愿以爱女许以贵国王上,待公主及笈之期,两国便结姻亲之好。北汉兵强,没有十年八年,无法攻克,陛下心切北疆战事,更希望和南楚和议,不知相爷意下如何?”

陆灿初时还有些不明白,只听了几句脸色便沉了下来,等到韦膺说完,他叹了口气道:“尚相想必是答应了。”

韦膺冷冷道:“苟廉舌灿莲花,尚维钧昔日被大雍俘虏,早就心胆俱寒,只想偷安,更何况将军手握重兵,本就受尚相猜忌,陆公爷又卧病在床,如今和议已经谈成,将军除非是使用兵谏,否则绝没有挽回的机会。”

陆灿神色一动,道:“你今日来此,是你一人的意思,还是凤仪门的意思?”

韦膺神色有些焦躁,道:“她们畏惧大雍兵势,怎敢和大雍作战,只想施展阴谋诡计,沙场厮杀,她们早就没有参与的勇气了,这次是我一人的意思,不过若是将军肯起兵,我保证她们会选择支持将军。”

陆灿深深叹了一口气,道:“韦首座,我知道你今日乃是一片诚心,可是陆某身为南楚臣子,绝不能作出这种目无君上的事情,所以我不会起兵,你的心意我领了,也不会将今日之事泄漏出去,你去吧。”

韦膺失望地道:“你可知道今日若是妥协,再没有踏上大雍领土的机会?”

陆灿正色道:“不论将来如何,陆某不能做出不忠不孝之事,若是人臣都可以抗旨兵谏,那么朝廷威严何在,若是陆某做出这等事情,南楚从此王纲失统,与其如此,陆灿宁可将来苦战大雍,保住江南半壁江山。”

韦膺叹道:“你如此愚忠,怎是江哲的对手,罢了,是我瞎了眼睛,当你是可托付的主君,既然你下了决定,我也无话可说,只是从今之后,我可能会多有得罪,还请将军体谅。”

陆灿眼中杀机一闪,继而泄气地道:“我知道你想转而控制尚相,不过尚相虽然不明军略,那些钩心斗角之事,你未必是他的对手,无论如何,你若做得太过分,别忘记我手上还有大军。”

韦膺轻轻一叹,道:“我若是能够掌控凤仪门,必定除掉尚维钧,让你可以控制朝政,可惜这一点我无能为力,罢了,也是韦某命该如此,没有可能借助你南楚大军攻下长安。”说罢,韦膺转身走出,陆灿想要出声唤他,却终于没有出口,他既然不能做出不忠不孝之事,那么和韦膺决裂也是必然之事。深深叹息了一声,陆灿低声道:“纵是粉身碎骨,陆某也要保护着如画江山,只是这谋逆犯上之事,陆某却是死也不能从命啊。若是江先生在此,必定嘲笑自己我太过迂腐吧,昔日从他读书之时,先生就曾经取笑,唉,我终究是不如先生洒脱啊。”

走出陆灿大营,韦膺茫然地走了许久,良久才从彻底的失望心寒中恢复过来,身为丞相之子,又曾经做过高官,韦膺的军略才能绝不是泛泛而已,当今天下,大雍兵强马壮,南楚、北汉都无力与争,如今正是唯一的机会,南北夹攻,削弱大雍势力,只要大雍损失惨重,就是一时不能彻底灭亡大雍,它也无力再一统天下,若是天下一统,那么自己的仇恨就再也难以报复。凌羽、纪霞、燕无双这些人虽然也是略通军政,可是却是目光短浅,只想着让南楚偏安江南,对她们来说,大雍想要灭亡北汉,消化其地其民,没有十几年是不行的,而南楚虽然暗弱,但是毕竟占了半壁江山,只要守住长江,不惧大雍铁蹄南下,所以她们宁可用各种手段阻碍大雍的一统进程,却不敢正面对敌,生怕大雍索性先出兵南楚。在她们心中,有了十年的缓冲,足可以让南楚积蓄力量,至少几十年之内可保平安。而凤仪门主的仇恨,在她们来说,早已是昨日黄花,只要能够自身荣华富贵,她们不愿意豁出性命复仇。如今她们最想的是像昔日在大雍一样,暗中控制南楚朝政,而两国交兵,不符合她们的利益。这些愚蠢短视的女子,自己怎会和她们搅在一起。恨意重重中,韦膺清醒过来,他果断的放弃了无益的抱怨,既然不能利用陆灿向大雍出兵,那么自己就要借助凤仪门的力量,想尽办法控制南楚的朝政,然后集中所有的力量,向大雍报复,向江哲报复,为了这个目的,自己宁可付出任何代价。脸上闪过坚毅的神色,韦膺加快了步伐,他不能再浪费任何时间。

\chapter{第七章 阴云密布}

第七章阴云密布

流水一般连绵的琴声从龙庭飞府中的一处华轩传出,琴声宛若天籁,在仍然冰凉的微风中回荡,萧桐匆匆走来,隔着窗棂看到那黑色的身影,心中不由轻叹。一个多月前,自己无意中查到一些久远的几乎湮灭的情报,发觉青黛曾有一段很长的时间在北汉境内失去了踪影,心中生出不妥感觉的他立刻回来准备将青黛拘禁起来。可是却得知段无敌带着青黛出门了,而且不知两人去向。正在忙乱的时候,凌端说出了偷听来的消息,萧桐心中不安,请秋玉飞和自己一起前去寻找段无敌和青黛。而在石英墓前,两人看到的是被杀死的近卫和昏迷不醒的段无敌。段无敌是中了一种大雍密谍特制的剧毒,这种毒虽然不够强烈,不能让人立刻身死,可是却是很难治愈,中毒之人一两个月之内都很难恢复健康,常常被大雍密谍用来生擒目标。而段无敌清醒之后说出青黛所为之后。萧桐大受刺激,谁让他没有发觉青黛居然是大雍密谍呢?

为了弥补自己的错误,萧桐请秋玉飞前去追杀青黛,毕竟秋玉飞武功大进这一点他是看得出来的。可是秋玉飞居然婉拒了他的要求。萧桐素来是知道这个师弟对于战争和权势毫无兴趣,几乎从来不牵涉其中,可是这次秋玉飞去大雍刺杀江哲以及他出面替段无敌缓颊的事实让萧桐淡忘了这一点。因此两人之间发生了不大愉快的冲突,不过最后看在师兄弟的情分上,秋玉飞还是亲自出马了。而且在数百里的追杀过程中,秋玉飞亲手杀死杀伤了大半密谍,若非是大雍军方的接应及时,恐怕就连那个武功超出众人预计的青黛也不会活着回去。而回到沁州的秋玉飞十分不快,甚至立刻就要回晋阳,若非龙庭飞千方百计说服了他暂时留下,恐怕秋玉飞早就离去了。萧桐隐隐觉得,除了不愿涉入军务之外,师弟更可能怨恨自己迫他去追杀青黛,因为他从凌端口中得知,秋玉飞似乎对青黛也颇为青睐。

想起青黛,萧桐更是恨得咬牙切齿,多年打雁,却被大雁啄了眼睛,这个女子摆出对北汉朝廷痛恨的架势,却让自己完全没有怀疑她真的是大雍密谍,根据段无敌所见,此女身份极为重要,她能够接下师弟百招,这样的武功心机,很可能是大雍在北汉情报网的总哨,让她逃生真是万分可惜。虽然龙庭飞没有怪罪自己,可是萧桐却心中难安,所以更是要想法子留下秋玉飞,这个师弟武功突飞猛进,若有他相助,自己更可以放手而为了。

琴声终于停了,萧桐轻轻咳嗽了一声,走进了华轩,秋玉飞轻抚着琴弦,没有起身迎接师兄,他们师兄弟之间本就没有明显的身份高低,在魔门,武功和才华决定了很多东西,如今已经晋入先天境界的秋玉飞完全有资格冷落萧桐,即使萧桐是自己的师兄。

萧桐犹豫了一下道:“大将军需要一个人去东海,阻止东海侯在近期归顺大雍。”

秋玉飞淡淡道:“如何阻止,东海侯本是大雍外戚,而且江哲在东海数年,我想东海归顺大雍只是时间的问题。”

萧桐无奈地道:“你说得不错,可是我们需要东海的物资,虽然这几个月我们尽量的囤积物资,可是仍然不足够,如果东海归顺大雍,对我们来说打击太大了,我们希望东海仍然能够保持中立。”

秋玉飞剑眉扬起,道:“这恐怕不容易,难道大将军有什么对策?”

萧桐冷冷道:“当年东海与大雍为敌,若没有我国暗中支持,他们早就完蛋了,如今我们不求他支援我方,只要他保持中立,如果这一点他们不答应,那么姜氏父子忘恩负义,理应受到天遣。”

秋玉飞冷冷道:“你是要我用刺杀威胁他们么?东海是他们的势力范围,你不怕我死在海上?”

萧桐道:“以你如今的武功,至少可以逃出东海,而且有师尊作为后盾,东海绝对不敢轻易为难你,我们的要求并不过分,我想他们会同意的。”

秋玉飞轻抚琴弦,似乎有些犹豫不决,萧桐知道秋玉飞并非担心危险,而是在犹豫自己是否要介入这些事情。萧桐也不敢肯定他会如何答复,心中忐忑不安。这时,站在一旁侍奉的凌端低声道:“四爷,覆巢之下,焉有完卵,难道现在大雍还会将四爷当成无害之人么?”秋玉飞心中一凛,想起万佛寺刺杀,想起自己追杀青黛之事,终于叹了口气道:“好吧,我去就是。”

萧桐大喜道:“多谢师弟体谅愚兄难处。这也是师尊的意思,还望师弟多多用心。”

秋玉飞漠然,望着琴边那册琴谱,不由想起万佛寺之内那人对自己的厚爱,以及他得知自己乃是刺客之后悲愤的神色。想起那清秀儒雅,却是灰发霜鬓的形容,秋玉飞心中涌起无可言表的悲哀。人生难得一知己,可是自己却偏偏只能和他生死相见。

帅府节堂之上,龙庭飞对着麾下将领,冷冷道:“你们不用再说,我知道现在军心不稳,可是现在不是手软的时候,大雍齐王已经虎视眈眈,随时都会起兵攻打沁州。石英麾下的将领士卒必须重新编制,不能留下任何隐患,如今我北汉危亡在即,若是不用非常手段,不等大雍铁蹄进入沁州,我们就已经完了。传我谕令,沁州男子十五岁以上者均召入军中,此战之后,我自然重重赏赐抚恤,若是此战落败,社稷不存,还谈什么安居乐业。”

挥手斥退了麾下将领,龙庭飞疲倦地倚在帅椅上,这段时日他可是太辛苦了。石英自尽,段无敌中毒,他尽失臂膀,而石英背叛和段无敌走私的消息又不胫而走,为了安抚军心和应对朝廷,龙庭飞几乎费尽了所有心力。虽然如此,段无敌还是降了一级官职,石英在军中旧部也受到牵连,龙庭飞被迫进行了清洗,如今对着下面的将领,龙庭飞总觉得他们沉默中带着不满和反抗,可是却又无可奈何。想要重聚离散的军心是需要契机的。

目光落到帅案上面的一份文卷,那里面记载的全是大雍楚乡侯江哲的情报,龙庭飞将文卷拿起,再次阅读起来,读到最后,龙庭飞心中恨意渐起,都是这个人,自从他在东海显踪,自己的一切计划都遭遇到挫折,忍不住将文卷扯得粉碎,龙庭飞无力地叹了口气,莫非这人是我的克星么?心中苦闷之下,龙庭飞回到后宅,吩咐下人取来酒菜,独自一人饮了起来,酒入愁肠愁更愁,龙庭飞喝了许久,饶是他酒量不错,也是酩酊大醉。

“哎。”当龙庭飞从头疼愈烈中醒来之时,已经是正午时分,近卫送上热水面巾,一个近卫小心翼翼地道:“大将军,段将军在外面等了半天了。”

龙庭飞一惊,顾不上整理仪容,走出卧房,一眼就看见段无敌一身戎装,站在阶下,神情冷峻,面色苍白,龙庭飞连忙上前几步,急切地道:“无敌,你来做什么,你的伤势还没有痊愈。”然后又斥责近卫道:“你们不知道段将军身有毒伤,怎么不请他到旁边花厅里面休息,真是废物。”

几个近卫凛如寒蝉,呐呐不敢辩解,段无敌却是坦然道:“大将军不要责怪他们,是末将坚持在这里等候。”

龙庭飞愧疚地道:“无敌,都是我酒醉误事,对不住你,快,到我房中坐下。”段无敌眼中闪过一丝光芒,道:“末将正有事情和大将军商谈。”

龙庭飞亲自领了段无敌走进卧房,将近卫赶了出去,胡乱洗了两把脸,道:“无敌,你有什么事情就说吧。”

段无敌站了起来,正色道:“末将今日前来向大将军禀明军务,可是大将军居然没有出现,末将问过之后近卫才知道大将军酒醉,末将因此前来相谏,如今我北汉危在旦夕,大将军乃是军心所系,怎能贪杯误事,此时若是流传出去,岂不是令人心寒,末将狂妄直言,请大将军不要见怪。”

龙庭飞面上一红,继而颓然坐下,道:“无敌,你是我心腹人,我不瞒你,如今的局势我真的觉得无能为力,论军力,大雍是我数倍,论钱粮,大雍可以长年累月作战,我们若是打上几个月,只怕就辎重耗尽了,论将领,大雍拿出一个就是名将,可是我最信任的将领却是死得死,叛的叛,就连你也受了毒伤,我真得有些支撑不住了。大雍有李贽那种明君,李显那种大将,还有江哲那种谋士,我身上的压力你可明白?”

段无敌肃然道:“大将军对无敌推心置腹,那么无敌也不敢相瞒,我军窘况,无敌何尝不是心中明了,可是无论如何大将军不能流露出这样的心意,如今军中除了大将军,再也无人可以控制军心士气,如果大将军都放弃了,那么如何让麾下将士树立信心呢?大将军,你若是心意如此,那么我们不如不战得好,免得让将士白白丧命。”

龙庭飞被段无敌的言辞激得面红耳赤,望着神色苍白,额头上满是汗水的段无敌,如今段无敌身负污名罪责,在军中也是处境艰难,石英的部下对他很不谅解,很多下级军士也不明白他所做出的牺牲,可是他却仍然如此坚定不移。望着这样的段无敌,龙庭飞心中豪气渐起,北汉军中都是这样的豪杰,就是大雍再强大又能如何?龙庭飞恭恭敬敬地向段无敌行了一礼,段无敌连忙避过,龙庭飞大声道:“段将军忠言,庭飞谨记,就是粉身碎骨,也不能再这样灰心丧气。”

段无敌见龙庭飞恢复了往日神采,心中欣然,道:“大将军军略无双,我们沁州易守难攻,大将军也不用过分担忧。”龙庭飞已经恢复了信心,道:“段将军放心,除非是庭飞战死沙场,否则绝不会让大雍军攻下沁州。”

望着神采飞扬的龙庭飞,段无敌这才放下心来,道:“大将军请先更衣,末将告退了。”

龙庭飞笑道:“你先等我一下,看你已经可以起床了,有些事情我还要和你商议一下,若是撑不住,就在我府上休息,让你躺着养病可就太可惜了。”

段无敌心中一暖道:“末将遵命。”

同一时刻,南郑东郊一座古寺之内,李康站在大雄宝殿之内,望着庄严的佛像,陷入沉思。

虽然还不到二月,长安还是十分寒冷,可是南郑可是比长安温暖的多,东川富庶之地,李康在这里可以说是一手遮天,更何况他如今将朝廷安排的将领暗探一扫而空,更是没有掣肘之人,按理说李康应该十分欢喜得意才是,可是李康心中却燃烧着熊熊怒火。

就在方才,雍帝李贽的圣旨到了,不过不是朝廷使臣送来的,李康在使臣还没有进入东川之前就派出得力手下扮作山贼将使臣杀了,不过仍然将圣旨取了来。圣旨上面是命他严守葭萌关,不可懈怠。看了圣旨之后,李康本应该欢喜,因为这样看来朝廷还不知道东川已经被他完全控制,可是李康却还是十分恼怒,凭什么李贽可以对他呼来喝去。

李康从来都觉得自己是不幸的,出身微贱,自幼不得父皇宠爱,除了母亲之外,李康从来没有得到什么温情。多少次他眼睁睁看着李安、李贽,甚至李显、李贞,在父皇面前肆意邀宠,自己明明是三皇子,却因为母亲只是一个地位低下的嫔妾而不敢上前。若仅是如此,李康或者会容忍下去,可是唯一疼爱自己的母亲,却被纪霞那个贱妇生生害死,而父皇只是追封了事,一怒之下李康逃离了皇宫。

可是逃离了皇宫之后,李康才知道原来自己的生活已经是很多人梦寐以求得了,一个什么都不懂的皇子在乱世之中生存谈何容易,多少次被人辱骂殴打,多少次饥肠辘辘,凭着一点武技和心狠手辣,他终于活了下去,可是报仇却是遥遥无期的一件事。多少次他忍受不住外面的苦难,想屈服回宫,可是母亲临死之前的情景却让他终于坚持了下去。而直到他遇到那个改变了自己的命运的神秘人,李康才第一次觉得上天待自己不薄。而后他练成了高深的武技,回去行刺纪霞,却落败被擒,若非是郑暇仗义执言,只怕他这个皇子就要问罪下狱了,若是如此也就罢了,偏偏李援将他派去东川,无诏不得回朝,这种明是贬斥暗是保护的举动却让李康更加不平。明明自己是天家骨肉,却要让自己向凤仪门低头,李贽还不是明目张胆和凤仪门作对,可是凭着他的大军,谁敢和他为难。抱着这种心情,李康在东川整军尽心竭力,终于掌握了一支不小的力量,可是即使如此,干系大雍社稷的夺嫡之争,李康却没有丝毫机会参与,皇上、太子、雍王、齐王在这一点上似乎有相同的看法,所以李康的势力根本无法在雍都立足,就是最温和的李贽,也曾经写信阻止自己介入长安之事。难道我不是皇家的人么,这种屈辱让李康下定了决心,就是大雍颠覆,也不能任由人主宰欺压。所以超出了北汉魔宗的预计,李康决定反叛,而反叛的第一步就是清除身边的暗探。

东川数年,李康已经成功的有了自己的力量,而蜀国余孽也为了虚无缥缈的复国上了自己的船,再加上北汉魔宗的暗中支持,终于一举铲除了身边的暗探和卧底,这些人早就被李康监控起来,如今一网成擒,李康终于感觉到前所未有的轻松。然后汰换军中将领,更换官员,李康多年来的谋划终于付诸实施,东川已经是他一人的东川,而只要寻到适当时机,就可以出斜谷攻向长安,到时候大雍朝廷就在自己掌握当中。当然出兵的时机要仔细选择,要等到长安周边的军力被李贽调去援救泽州前线和荆襄长江一带之后,自己才可以如同匕首一般直接刺穿大雍的心脏。李康心中明白,如今虽然自己手上有十万兵马,可是这些兵马毕竟是大雍的军队,若是给大雍朝廷发觉自己的反叛,那么这只军队很可能会被朝廷分化招降,所以切断长安和东川之间的联系,隐蔽自己背叛的事实就成了最重要的事情。而想要达到自己的目的,凭着自己一人的力量是很艰难的,如果不能得到原蜀国势力的支持,自己只能功败垂成。而原蜀国势力除了那些想要恢复昔日荣耀的旧世家之外,至今仍然暗中反抗大雍的锦绣盟就成了他最想招揽的力量,经过多次谈判协商,今日就是锦绣盟主和自己会面的日子,霍纪城的谨慎很令李康叹服,他是辗转多次,才最后得知在此地和霍纪城相会的,为了安全,除了叶天秀和几个亲信侍卫,李康没有多带人马,他相信霍纪城也是很有诚意的,锦绣盟近些时日协助自己断绝长安和南郑通路,这就是诚意的证明。

将近黄昏时分,大雄宝殿的殿门突然无风自开,两个黑衣人站在门前,其中一人正是多次和庆王会过面的陈稹,而另外一人则戴着遮阳斗笠,青纱低垂,看不见形貌如何。李康欣然上前道:“陈副盟主,这位就是霍盟主吧,小王闻名久矣,今日得见,真是三生有幸。”

那黑衣蒙面人上前施了一礼道:“殿下礼贤下士,霍某也是闻名久矣,霍某不才,心中只有复国一念,多年来碌碌无为,真是惭愧,听陈稹说,殿下府上的蜀王遗子身份已经核实无误,殿下之恩,蜀国遗民无不感激涕零,霍某今日前来,除了致谢之外,也想和殿下商量一下合作事宜。”

李康道:“盟主太谦了,当初盟主刺杀南楚国主、害得李安户部事泄,就是凤仪门不也被盟主在江湖上狠狠打击了一番,这种种丰功伟绩,小王可是不敢忘记,尤其是洛阳一事,盟主义子少年英杰,凭着一人之力将洛阳两大世家几乎是天翻地覆,凤仪门在洛阳的影响力也削弱到了极点,这件事情长令小王击节而叹,不知可否有机会见见这位少年英雄。”

黑衣人轻声笑道:“小孩家的胡闹,倒是让王爷见笑了,霍离乃是我心腹爱将,又是我的义子,我素爱之,可惜此子身负重责不能脱身,若是王爷喜欢年少英杰,在下另一个义子霍义武功高强,办事放心,如果王爷不弃,请允许他替王爷效力。”

李康笑道:“好啊,贵盟英才辈出,本王真是羡慕得很,就让霍义到我身边作个亲卫吧,若是才能不凡,本王自然会重用他,霍盟主,关于我们的合作,不知道盟主意下如何?”

黑衣人沉默片刻,道:“王爷说得不错,这才是正事,在下冷眼旁观,王爷反意坚决,所以霍某才不畏陷阱的可能,来到南郑和王爷相见,可是王爷毕竟是大雍亲王,让在下怎么相信王爷会恢复蜀国江山,蜀王遗子身份虽然没有问题,可是这种使用傀儡的把戏也很常见,昔日霸王项羽不也拥立了怀王,可是最后怀王死在项羽手上。王爷凭什么让在下相信蜀国会真的复国呢?”

李康早有准备,坦然道:“小王也不说什么冠冕堂皇的大话,世上不会有这样的好事,本王起兵作战,却让小儿承受王位,所以这大权一定要在小王手中,拥立蜀王不过是个幌子,要让蜀国遗民支持小王的计策,可是本王也可以担保不会过河拆桥,毕竟如果没有蜀人的支持,本王也不可能割据一方。所以蜀国宗室我一定保全,甚至本王可以改奉蜀王宗庙,不过若是本王能够有所成就,这蜀王之位我是要定了的。诸位要得不过是荣华富贵,难道我李康就没有可以给你们的么,盟主不是愚忠之人,蜀王之位也不是他一家之物。”

那黑衣人虽然看不清神情,可是只见他身躯微震,就知道他心中激动,良久,黑衣人才道:“王爷说得不错,蜀王之位能者居之,王爷需要依靠蜀人,所以只要仔细筹划,二十年后,蜀人就会将王爷当成自己人。王爷如此推心置腹,霍某感激不尽,若是王爷说什么没有二心,倒让霍某小瞧了,好,若是王爷肯再答应霍某一个条件,你我盟约就在今日达成。”

李康大喜,他经过仔细揣摩,能够作出这种种匪夷所思之事的锦绣盟主绝非食古不化之人,所以料定霍纪城不会着紧蜀王遗子,果然如他所料,他稍微放心一些,道:“盟主请说,只要合情合理,本王一定答应。”

黑衣人斩钉截铁地道:“在下要得是权势。”

李康有些奇怪,自己要和锦绣盟结盟,这权势富贵自然是要给的,怎么霍纪城会特意提出,正要动问。黑衣人挥手让他不要说话,朗声道:“所谓权势多种多样,但是只有两种权势是不可轻易被夺取的,就是兵权和监察之权,皇权之所以至高无上,就是因为皇室掌握着压制一切的兵权和监察臣下的暗探,兵权我们锦绣盟没有兴趣,也没有这个能力掌握,所以我要暗探之权,锦绣盟可以成为王爷的耳目和杀手,只有这样,锦绣盟才能和王爷结成稳固的联盟。如果王爷不肯答应这个条件,那么锦绣盟绝不会和王爷合作。”

李康心中一凛,霍纪城果然厉害,虽然他是有心吸纳锦绣盟的力量,可是若是放手让他们掌管监察权力,那么自己就不可能和他们分离了,虽然有些犹豫,可是李康转念一想,自己不就是看中锦绣盟在这方面的能力么,只不过霍纪城要求的权力多一些,毕竟兵权在自己手上,只要掌握兵权,那么锦绣盟就不足为惧。而且这样一来,双方的盟约就坚不可摧,对于自己来说,完全掌控蜀国遗民的目的才有实现的可能。所以李康伸出手掌道:“一言为定。”

黑衣人眼中闪过激动的神色,两人击掌为誓,达成盟约。击掌之后,黑衣人就要告辞,道:“在下的名声有些不好,还是不公然出现比较稳妥,王爷现在也不想引起太多人注意吧,陈稹是我亲信,就让他和王爷商议合作的细节吧。”

李康眼中闪过寒芒,道:“这样也好,不过本王有个不情之请,本王对盟主早就感佩在心,今日相见,盟主却是不肯露出庐山真面目,不知道可否取下斗笠,坦诚相见呢?”

黑衣人默然,陈稹一直站在他身后,此刻似乎身躯一动,有些不安,可是殿门之外却响起不急不缓的脚步声,隐隐的杀气透入进来,而李康的身躯更是屹立如山,血腥的杀气冲天而起,显示出李康并非只是一个武将,而是双手沾满鲜血的江湖高手的身份。殿中的气氛在顷刻之间变得冷肃,杀机隐伏。

\chapter{第八章 古墓秘舵}

第八章古墓秘舵

就在大殿之内气氛一触即发之际,那黑衣人突然哈哈一笑,道:“好个庆王爷,王爷心中想必是早有疑问,只是若是说得早了,担心霍某心中生出嫌隙,不好叙谈,也罢,霍某遵命就是。”

李康微微一笑,他直到这时才提出要求还有一个缘故,若是协议达成,那么只要不大过分,霍纪城就不会过于记恨,可是此事十分重要,霍纪城多年没有露面,只闻其人,不见其容,李康总有些不放心。

黑衣人右手摘去斗笠,青纱飘飞,露出一张清瘦峻挺的容貌,虽然细目鹰鼻,令人一见便觉得他心狠手辣,但也算是仪表不凡,尤其是冷森冰寒的双目,令人一见胆寒。李康将这人相貌和大雍军方留存绘制的肖像比较了一下,确定这人正是霍纪城,方欣然道:“霍盟主果然气度不凡,能够和盟主合作,本王定可以宏图大展。”

霍纪城微微一笑,道:“王爷此言差矣,我锦绣盟怎敢说和王爷结盟,是王爷不弃,收留霍某和手下这些兄弟吃碗饱饭罢了,从今之后,锦绣盟和王爷君臣名份已定,王爷不必客气,不过我盟中不免有些固执起见的盟友,所以还请王爷暂时不要外泄此事,等到霍某将盟中料理干净,到时候想必王爷已经起兵,霍某一定来王爷帐下效命。”

李康笑道:“不妨,不妨,有陈副盟主在,就像霍盟主在一样。”

双方又寒暄片刻,殿内气氛渐渐和缓下来,而殿外的杀气也消失无踪,霍纪城和陈稹才寻机告退。

直到离开古寺二十里之外,陈稹才低声道:“董总管,多亏你设想周到,事先准备了这张面具,否则只怕我们的计策就失败了。”

“霍纪城”笑道:“其实陈兄也不是没有想到,只是这易容之术早已失传,也难怪陈兄没有办法,幸好这几年我和公子仔细研究,虽然不能持久,但是倒是惟妙惟肖,这次见面之后,基本上霍纪城就不用出现了,陈兄可以放心了。”说着话,黑衣人摘下斗笠,然后将一种药物抹在脸上,不过片刻,他脸上的皮肤仿佛干旱的土地一般开始龟裂,不过片刻,一些灰白色的薄皮剥落,露出一张俊秀白皙的容貌,星月沉沉,幽暗的光芒照射到他的面上,正是奉了江哲之命来到东川的董缺。他将斗笠戴上,笑道:“这面具就是有些不透气,将来我和公子再仔细研究一下,想办法做出更好更耐用的面具。”

陈稹道:“公子果然妙手,世间偶然流传的易容之术不过是改变一下外貌形象,像公子这种手法,可以模仿另外一个人容貌的易容术可是早已失传了。”

董缺道:“公子还在后悔,他说,若是当初杀了霍纪城的时候,将他的面皮剥下来制成面具,就方便多了,可惜这种手法还是近年才研究出来的,十分不成熟,公子也只是利用几个囚徒的面皮做了两次,虽然效果更好,可是制作手法还需要精研,可惜公子终究不忍心继续研究下去。”

陈稹开玩笑地道:“公子不忍心,将来董总管可以用心研究一下么,毕竟董总管在这上面已经费了许多心思。”他说出来只当是玩笑,董缺眼中却闪过一丝深思。

两人一边说着闲话,一边缓步行走,今日两人达成和李康的协议,心中都十分高兴,两人自信无人能够接近百丈之内,但是为了提防有人遥遥跟踪,仍然转了几个圈子,直到半夜时分两人才走到一座古墓前。两人四处转了一圈,确定没有人跟踪,陈稹走到墓前石碑之后,在石碑后面轻轻击了几掌,石碑悄无声息的移开,露出一条暗道,两人走下之后,石碑再次合上。这座古墓乃是前年陈稹从一个盗墓贼口中得知的,这座古墓足有几十间墓室,中以甬道相连,处处机关,十分严密。跟随陈稹而来的八骏之一山子对机关学得十分精通,众人费了许多心思,花了数月时间,将这座古墓清理出来,成了锦绣盟的总舵,能够进入这里的除了陈稹和秘营众人之外,只有锦绣盟的一些重要人物和陈稹在锦绣盟中的少数心腹。

两人走下密室,负责迎接两人的正是白义,他身材不高,肤色微黑,相貌神情有些憨厚,但是他却是秘营中的第一高手,搏杀之术超出众人之上,辅佐陈稹掌控锦绣盟,功劳非浅,当然在这里他的身份是霍纪城的义子霍义。董缺取下斗笠,接过白义递过的一个鬼脸面具,戴在脸上,在这里,他仍然是霍纪城,这里有些锦绣盟弟子虽然是陈稹心腹,但是他们也不知道霍纪城早已死去的事情,所以董缺仍然要以霍纪城的身份出现。

两人走入最大的一间墓室,这里是锦绣盟的议事厅,两侧都站了十几个形貌各异的人,董缺昂然坐上正中的位子,陈稹坐在他身侧,而白义站在董缺身后权充护卫。董缺用冰冷的声音道:“诸位请坐。”

那些人向董缺行礼之后,谨慎的坐下,他们大多都是蜀人,“霍纪城”很少和他们见面,大多都是通过陈稹或者使者传达各种命令,而他们对于霍盟主都是十分戒惧,不论是霍纪城从前的狠毒凶残,还是如今的诡秘阴狠,都让他们不敢生出背叛之心。

董缺冷冷道:“本座已经和庆王达成协议,我们将接管庆王的谍探监察组织,而相对的,我们也要支持庆王恢复蜀国,不知诸位有何高见?”

一个相貌豪迈的中年人站起身道:“盟主,此事不可,李康是大雍皇子,恢复蜀国还轮不到他。”

董缺冷哼一声道:“罗护法,你想清楚一些,凭着我们锦绣盟的力量,难道可能恢复蜀国么,如果没有庆王的大军,那只是镜花水月罢了,只要我们帮着庆王割据东川,再寻机出兵关中,等到我们蜀国的力量在庆王势力中占了上风的时候,还怕他心口不一么?”

那个中年人赧然坐下,他倒不会因为盟主训斥他而担忧,这几年来霍纪城的性情变化了许多,在他问众人意见的时候,大家是可以畅所欲言的,不过若是他已经作出了决定,就是绝对不许任何人违背他的命令的了。

众人商议了半天如何更好的控制庆王,气氛十分热烈,毕竟这些年来,这是最好的复国机会。董缺目光闪过,心中窃笑,公子的计策可真是高明,将这些心切复国的人控制起来,清除其中过于狂热的分子,将剩下的人约束起来,如今又可以利用他们的复国热忱消除庆王的疑心。不过当董缺目光落到一个沉默不语的中年人身上的时候,他却皱起了眉头。那个中年人叫做顾宁,在锦绣盟中声望极高,也是创盟元老之一,原先的霍纪城和他十分不合,曾经差点将他陷害至死。等到陈稹接收锦绣盟之后,将他放了出来,因为此人复国之志十分坚定,而且才华也颇为过人,又不是那种狂热分子,所以仍然许他高位,用他来招揽那些真正的复国志士,当然对他的监视也更加严密。幸好他和霍纪城并非十分亲近,瞒过他并不困难,否则就不得不杀死他,那可就是损失惨重了。董缺见他神情不对,便冷冷道:“顾护法,你可有什么意见么?”

顾宁心中一凛,当年他险些死在霍纪城手上,幸好陈稹加入之后,说服霍纪城赦免了自己,而这几年霍纪城心性成熟了许多,所安排的计策都是十分缜密周到,锦绣盟势力稳步上升,除了复国暂时无望之外,倒也没有什么不妥。可是顾宁心中却是有苦说不出来,他身边几乎都是陈稹派来监视自己的人,妻室子女都在这些人掌握之中,自己除了奉命行事之外再也没有别的选择,若没有陈稹的许可,自己的命令根本就无法传达下去。虽然自己的计策多被采用,可是随时都可能丧命的阴影仍然逼得他喘不过气来。

对于和庆王合作之事,他是不赞同的,蜀人想要复国根本就不应该借助他人势力,在顾宁心中,若是不能成功复国,那么宁可维持这样的状态,只要复国的火种传下去,那么总有一日可以如愿,这种急功近利的做法他并不同意。可是他深知霍纪城这样的态度,那么这个决定实际上是不能反抗的。可是眼看着蜀人无辜地陷入战火当中,他真的不情愿,侧头避过冰冷的目光,他沉声道:“庆王谋反,那是他们大雍的家事,不论谁胜谁负,我们都不可能真的复国,为何趟这混水,只怕是白白害死了众多盟友。”

陈稹眼中闪过冰冷的光芒,顾宁若是存有这种心思,难免会造成盟中众人离心离德,毕竟顾宁的声望摆在这里,锦绣盟从上至下只能有一个心思,陈稹不想留下锦绣盟分裂的后患。不能让致力于复国的蜀人脱离锦绣盟,这可是江哲定下的铁律。他站起身来,他冷冷道:“盟主,有一件事情属下早就想禀报,只是未到时机,我盟中有两名弟子生出异心,他们厌倦了复国之事,竟然想要退盟,如何处置还请门主裁决。”

董缺领会到了陈稹的意思,故作大怒,厉声道:“岂有此理,锦绣盟是可以随便来去的地方么,这两人是谁?传本座谕令,将这两个弟子给我处死,家人连坐。”

陈稹目光向下面众人一一看去,凡是接触到他的目光的人都不由低下头去,蜀中这几年来风调雨顺,庆王的治理秉承大雍朝廷的意旨,也算是颇为成功,百姓安居乐业,就是锦绣盟中也有一些年轻弟子生出了不想复国的念头,毕竟他们眷恋故国之心较为淡薄,心中明白陈稹定是要趁机发作某人,而且也知道多半目标不是自己,但是众人仍然心中忐忑不安。

陈稹眼中闪过一缕寒芒,恭谨地道:“是顾护法手下的熊暴和上官彦。”他这句话一说出,大部分人都送了口气,但是还有一些人露出忧虑的神色,熊暴是顾宁的外甥,上官彦是顾宁的义子,顾宁在盟中众人心中地位颇高,只是众人更加畏惧霍纪城和陈稹的手段心机,所以无人敢支持顾宁。

顾宁大惊,面色变得苍白,这两人都是他至亲之人,更是少年英杰,顾宁第一个念头是陈稹想趁机削弱自己的力量,可是转念一想,顾宁却觉得全身无力,这些时日熊暴和上官彦确实有些怨言,他们提出其实大雍一统天下之势已经不可扭转,与其谋求复国,不如让平民百姓安居乐业的好。顾宁心中也有同感,所以只是警告了他们不许说出去,可是想不到陈稹还是知道了。

无论如何,顾宁不能眼看着两个青年这样被处死,更何况家人连坐,那自己也会遭到波及,只得起身下拜道:“盟主,属下这两个晚辈只是胡乱说了几句闲话,他们对本盟忠心耿耿,绝无叛心,还请盟主原谅他们一时糊涂,请看在他们为锦绣盟履立功劳的份上,饶他们一死吧。顾某情愿代他们承受罪责。”

顾宁低声下气的恳求着,偷眼望去,只见盟主放在太师椅扶手上面的右手手指轻轻颤动,这是霍纪城动了杀机的习惯性动作,顾宁心中越发紧张,语气也渐渐急促起来。这时,盟主抬起右手,阻止了顾宁继续说下去,道:“既然顾护法求情,那么本座就网开一面,本座已经决定派霍义到庆王跟前效力,就让他们跟着霍义一起去吧,这件事情顾护法可有异议?”

犹豫了一会儿,顾宁终于颓然道:“属下没有异议。”想到了家人,他终于妥协了,为着复国大业,他可以牺牲一切,可是为了这种事情牺牲家人还是没有必要的,这几年霍纪城算无遗策,应该至少可以全身而退吧,顾宁这样想。

陈稹和董缺交换了一个眼色,特意模仿霍纪城的习惯动作,就让顾宁相信盟主动了杀机,无声的威胁让顾宁迅速屈服,一个铁骨铮铮的汉子被迫到这种地步,外人见了都会同情,可是陈稹和董缺都是铁石心肠,全无动容。董缺朗声道:“事情就这样决定了,不过本盟不能倾巢而出,防人之心不可无,由陈副盟主带一批人和庆王合作,本座仍然隐在暗出操纵大局。”众人同声应诺。陈稹和董缺又四目对视,两人心中早有盘算,将那些志切复国的盟友安排到庆王手下,让他们牺牲殆尽,正是最好的处置,而顾宁的冷静确实两人很欣赏的,而且江哲最终的目的是让锦绣盟中人淡忘复国的念头,所以顾宁就不用去了,至于熊暴和上官彦跟随白义去庆王麾下,却是为了寻机将他们控制起来,不让顾宁擅自行动罢了。

令众人散去之后,董缺低声道:“那个人怎么样?”

陈稹知道董缺问得是明鉴司被俘的暗探,也低声道:“仍在监押中,此人近来不安分,屡次想脱逃,若非他是明鉴司的人,早就死了十次了。”董缺道:“这个人应该放出去了,公子说让明鉴司和锦绣盟打一场,我们这边也好剔除一些不能教化的顽固之徒,至于明鉴司的损失,会让庆王相信我们的诚意,不过公子说了不能太过分,毕竟明鉴司是大雍所属,虽然那里面有些人是杀人放火的出身,而且公子也不想得罪夏侯沅峰,这个人不好惹。”

陈稹冷笑道:“夏侯沅峰不会心痛的,不过你说的有道理,还是要和他保持默契,不过这样的话,恐怕得你走一趟。”

董缺点头道:“我也这么想,不过不能太急促,公子的意思,将来锦绣盟还是要保留的,先把那个明鉴司的人放了,让他回去传个消息,夏侯沅峰心里也应该有点数的。”

陈稹道:“放心,就是审问的时候,我也是蒙面去得,他绝对不会知道这里是什么所在,锦绣盟三字他更是没有听到过。”董缺笑道:“现在也该让他知道一些了,这人是个好汉子,这么多日子不明不白的困着,还没有屈服,既然要放他,还是让他知道一些吧,这些夏侯那边也说的过去。”

董缺点点头,随着陈稹走到古墓深处,那里有几间机关密布的墓室,作为囚牢,而已经被软禁月余的明鉴司暗探裘山目前是唯一的囚犯。

裘山坐在石榻之上,面无表情,这间囚室十分整洁,石榻上面铺着稻草,被褥俱全,将他囚禁的这些神秘人虽然初时对他用刑逼供,但是不过数日就停止了,不再迫问他口供,还尽心尽力的替他治伤,可是这并不能让裘山生出一丝感激。见不到星月之光,只能凭着三餐来计算时间,一个多月的时间就这样荒废了,想到不能将情报送出东川,裘山心中万分愤怒,几次想要逃跑都功败垂成,若非是他心性坚强,只怕早就被这似乎漫无止境的囚禁逼疯了。忍不住摸摸身上的鞭痕,这是他上一次击晕守卫想要脱逃被俘之后,那些神秘人似乎下令打了他三十皮鞭,不过他们下手不重,否则只怕裘山现在别想起身了。

石门推开了,裘山眼睛都没有抬一下,虽然按照自己的饥饿程度,应该不是到了吃饭的时间,可是这种不明不白的囚禁和强烈的无力感,让他对很多事情都失去了兴趣。

一个清朗的声音传来道:“怎么,裘兄不想离开此间了么?”

裘山腾的一下站起来,面上却是一红,觉得自己表现的过于急切,抬眼望去,只见两个黑衣人站在面前,都戴着恶鬼面具,一个负手而立,另一个却站在门口,听这声音,裘山觉得有些陌生,赧然道:“请问阁下怎么称呼?”

站在门口的黑衣人开口道:“这位是我们长上霍爷。”

裘山心中一凛,他心思精明,对东川局势了若指掌,有本事将自己囚上一月,丝毫不露风声的组织并不多,一听见“霍爷”二字,他脱口而出道:“锦绣盟。”眼中立刻闪过警惕和疑惑的神色,锦绣盟和大雍的敌对他心中很清楚,有些疑惑就可以解释了,为什么这些人既不肯释放自己也不曾将自己交给庆王,可是另一个疑问又生了出来,为何这些人对自己这样礼遇呢?

董缺笑道:“裘兄好快的心思,不愧是明鉴司的人,在下霍纪城,忝为锦绣盟主。”

裘山面上露出冰冷的神色道:“原来如此,今日盟主前来相见,揭露迷雾,在下已经知道自己的结局了,多日来贵盟对在下的礼遇,裘山心中感激,不过在下没有什么可以说的,就请盟主赐在下一个痛快吧。”

董缺玩味地道:“看来你是认为我定要杀你了?”

裘山冷笑道:“锦绣盟是什么所在我心中清楚,盟主声名赫赫,在下也早有耳闻,不过看在贵盟多日来的照顾上,不妨劝盟主一句,大雍统一天下,乃是大势所趋,复国之望还是放弃的好。”

陈稹笑道:“你倒是好心,不过庆王谋反,恐怕大雍前途未卜,你怎知我们没有机会。”

裘山听出这是多次来探望审问自己的那人,冷冷道:“陛下圣明神武,我大雍带甲百万,庆王必定不会成功。”他说得斩钉截铁,董缺和陈稹相视一笑,心道此人果然意志坚定,那么让他回去最合适。

“楼船夜雪瓜州渡,铁马秋风大散关”,散关乃是关中四大名关之一,自古以来就是秦蜀之噤喉,东起陇首,西向终南,高峻雄险,在蜀国未亡之前,此处是大雍阻挡蜀国的要塞,虽然自从阳平关、葭萌关落入大雍手中,散关的地位降低了许多,可是大雍仍然在散关驻扎了足够的军力,而且当初李援和李贽都心中有些提防,所以庆王在散关根本就无法插手,守散关的将军叫做李宗勋,也是李氏皇族的子弟,只是血统偏远一些,他擅长守城,忠心又没有问题,所以特意选了他来镇守散关。而夏侯沅峰也在多日前来到散关,主持对蜀中的刺探,他带来了司闻曹西南郡司和明鉴司的人手,布置潜入东川的事宜,可是东川几乎是水泼不进,夏侯沅峰不知道这是锦绣盟暗中协助庆王的结果,对庆王的能力更是高看了一眼,心中越发苦恼。所以在夏侯沅峰得知裘山求见的时候,几乎是愣住了,原本以为早就死了的属下重新显身,这件事情足以让他震惊,而这次被夏侯沅峰带来协助自己的骅骝却是心中有数,虽然这几年他不再有机会接触江哲的势力,可是有些事情还是能够知道的,锦绣盟暗中被江哲控制,这件事他是知道的,所以裘山突然生还,骅骝很快就想到了可能的原因。夏侯沅峰心思细密,见骅骝嘴角露出笑意,立刻想起了李贽隐隐约约说过的事情,心中一宽,下令将裘山招了近来。

\chapter{第九章 高山流水}

初春的静海山庄,静谧而幽深,听涛阁外,碧海潮生,巨浪排空,一次次的撞击在岩石上,溅开似碎琼乱玉,又似风卷残雪,东海春潮,瑰丽万方。此时正是清晨,庄内的下人已经轻手轻脚的开始了一天的忙碌。而就在这时,听涛阁上突然传来激越的琴声,琴声如潮,激昂连绵,庄内众人都不由立住,侧耳倾听那动人心弦的琴声,恍惚之间,仿佛那气势磅礴的潮水已经越过峭壁,呈现在眼前一般。一曲终了,那些下人各自惊叹一番,又开始忙碌起来。而在静海山庄最高处的一间楼阁之内,一个白发如霜的老者放下手中的书卷,目光凝聚在远处的听涛阁上。这老者年过七旬,却是鹤发童颜,神情气度冷漠淡然,正是医圣桑臣。这时,门外传来清脆悦耳的声音道:“师祖,青烟给您请安来了。”

桑臣本是东海蓬莱人,而在他返乡隐居之后,江哲特意派了人建了静海山庄,接桑臣到此养老,桑臣虽然性情冷漠,可是对江哲却是视若孙儿,也就没有异议的住到了这里。江哲相助雍王夺嫡成功之后,扶病来到静海山庄,桑臣费了无数心思,才调养好江哲的身体,数年来,一家人其乐融融,桑臣对柔蓝和慎儿也是十分喜爱,倒是少了几分冷漠,多了几分温情。静海山庄风景如画,桑臣也有意在此养老,即使江哲夫妻已经离开,桑臣也仍然住在这里,不过膝下承欢的换了姜海涛、越青烟罢了。越青烟身上的蛊毒已经被桑臣除去,虽然数年内仍要用药物调治,但是性命已经无碍,而且越青烟虽然是女子,却是天资聪颖,对医道颇有见地,桑臣很满意她的灵秀和天资,将她留在山庄之内传她医术。姜海涛除了料理公务之外,也住在静海山庄,谁让他和越青烟夫妻和睦,不忍分离呢。所以静海山庄仍然是十分热闹,没有一分寂寞。

听见越青烟的声音,桑臣微微一笑,道:“进来吧,怎么海涛没有过来,昨日他不是回来了么?”

越青烟带着两个侍女走进房来,恭恭敬敬的行了大礼,数月时光,越青烟仍然是肌肤如霜雪,不过不同的是,两颊多了几许血色,让她显得越发清丽绝俗。听到桑臣的问话,她含笑道:“师祖,海涛也想给您来请安呢,不过方才先生的信使到了,海涛需要接待来使,所以恐怕得一会儿才能过来。”

桑臣点点头道:“弹琴的是谁,倒是好一手琴艺。”

越青烟道:“青烟听相公说,是北汉的使者秋玉飞,魔宗京宗主的嫡传弟子,公公已经将所有事情都交给相公处理,所以相公派人将他接来此地。”

桑臣轻轻蹙眉,魔宗,秋玉飞,他心中泛起涟漪,那是陌生而又熟悉的名字,六十年前,他桑臣也是魔门星宗宗主的候选,可是他对此却没有兴趣,最后因为他医圣的身份而失去了继承星宗宗主的机会。不过桑臣从未后悔过,他也不是多事的人,虽然身上的蛊毒早就被他化去,但是他从未想过泄露这个隐秘,星宗就这样成了他记忆中遥远的记忆,直到董缺的出现。一见到董缺,桑臣就知道此人必是星宗弟子,他曾隐隐暗示江哲董缺身份有诡秘之处,不过江哲只是笑道:“董缺心中有些隐秘,这个我知道,不过只要他忠心于我,我也不愿过问他的私事。”桑臣听后也不再过问,反正在他看来,董缺也没有恶意,不过是寻个安身之处罢了,不过为了以防万一,他还是把小顺子叫来,将一些自己参悟的绝学传授给他,这样一来,若是将来星宗和江哲有了冲突,小顺子足以对付星宗高手,他就不用担心江哲的安危了,不过从星宗的宗旨上看,他也不信星宗会和江哲对立。至于他自己,武功早就超越了魔宗的范畴,所以倒不忧虑董缺发现自己曾有的身份,更何况,就算他知道,又能如何呢?

董缺暂且不提,秋玉飞的到来却让桑臣心中微动。北汉魔宗和江哲可是敌对关系,秋玉飞来到东海,可不会存着什么好心,若是见见秋玉飞,应该可以了解魔宗现在的实力吧。虽然桑臣并不担心江哲的安危,有几十万大军和大雍的高手侍卫保护,又有得到他亲传的小顺子在旁,星宗的武功又是隐隐克制着日宗、月宗的武功,即使京无极的武功也已经超出两宗范畴,进入宗师行列,这种克制仍然是存在的,江哲应该不会那么容易遭遇危险吧?

听涛阁内,秋玉飞抚着爱琴,心中宁静许多,数日前他进入东海,就被东海来人接至在滨州的馆邑,等候小侯爷姜海涛的接见,直到昨日,才有人将自己接来静海山庄,在来之前秋玉飞已经听说静海山庄乃是江哲隐居之处,如今住在里面的是东海侯爱子姜海涛和他的夫人越青烟。想到自己即将踏进江哲的居所,秋玉飞心中不免五味杂陈。昨日更是一夜辗转反侧,难以入眠,到了清晨,他请庄内下人引路至听涛阁,想要观看海潮,到了阁中,海风清新,凭栏远眺,不由心旷神怡,因此抚琴抒怀,一曲终了,只觉得数日来的忧虑苦楚尽皆消散。秋玉飞站起身来,看着栏外的潮水,海风扑面而来,带着冰冷和清新,秋玉飞不由想到,若是江哲也在此处,两人一起观潮听琴,那该是何等的惬意啊。只可惜两人如今已是仇敌,只怕今生是没有这样的机会了。

正在秋玉飞心中惆怅的时候,耳边传来沉稳的脚步声,秋玉飞心中一动,来人龙行虎步,应该不是普通人物,他回到琴边坐下,等待来人。门外传来爽朗的声音道:“秋公子好兴致,观海抚琴,其乐无穷吧,不知道公子可喜欢静海山庄的景致。”声音未止,一个俊朗少年走了进来,正是昨日匆匆一会的姜海涛。

秋玉飞起身一礼道:“静海山庄风光如画,秋某十分喜爱,小侯爷特意来见,可是已经有了决定了么?”

姜海涛将一封书信放到琴旁,道:“今晨江先生的使者到了东海,这是先生给公子的书信。”

秋玉飞心中一震,虽然想到东海可能会将自己的行踪禀知江哲,却仍然不能消去他心中惊骇,看来江哲对东海的控制十分严密,若是自己的要求不被接受,莫非自己真要在东海大开杀戒么,这样一来,恐怕自己只能逃出东海去了。

打开书信,秋玉飞目光一凝,只见上面写着:“

玉飞贤弟如晤:

自万佛寺一别,闻君已平安归国,不胜庆幸,虽沁州之事害于贤弟,然各为其主,哲并无怨言。知君出使东海,哲有意留君暂驻静海。寒舍虽陋,却有藏书万卷,更有江海之胜,君若有意,或观海抚琴,或扁舟游弋,此乐何极,何必陷身沙场,致令双手血染,心境难平。东海风清月明,正合君心,屈君留此,望君远离俗世争端。若翌日重逢,望君前嫌尽逝,哲当与君琴歌唱和,再述别情。”

秋玉飞初时心中一宽,江哲并未怨恨自己,可是看到后来,他不由眉头紧锁,江哲竟然想将自己软禁在东海,真是岂有此理,他放下书信,冷冷道:“小侯爷可是自信能够制住秋某么?”

姜海涛摇手道:“秋公子过虑了,家父昔日曾受国师恩典,东海也曾收过贵国的钱粮,怎会恩将仇报,何况公子武功高强,海涛也无能囚禁公子,不过东海已经决定不参与此战,但是今次之后,东海于北汉再无亏欠,今后恐怕就不能再和贵国有什么牵扯了。”

秋玉飞心中一喜,疑惑地问道:“那么小侯爷凭什么自信可以留住秋某呢?”

姜海涛微微一笑道:“虽然昔日东海受过北汉的恩情,可是后来东海也有所偿还,其实双方早已扯平了,虽然昔日贵国雪中送炭的恩义未还,可是无论如何贵国也不会指望我们出兵相助吧。今次我方答应不出兵,而且贵国军方在此购买的钱粮,我方也愿意相助贵方运走,这样一来我方已经偿还恩义,两不相欠了。但是我方额外准备了一批粮草药物,都是贵方急需之物,只是贵方恐怕已经无力购买,海涛已经出资购下,贵国可以随时运走,补充军需,只是我方也有条件,就是秋公子留在东海,多则一年半载,少则数月,公子以为如何?”

秋玉飞沉默许久,他心中隐隐明白,江哲是决意将他滞留东海,甚至不惜付出资敌的代价,可是自己除了武功琴艺之外,再无所长,行军作战、出谋划策,自己都不擅长,可以说魔宗日月两宗的长处他都没有,而个人的武功强弱也无益军国大事,付出这些代价将自己留在东海,这值得么?江哲真的是为了私谊作出这种决定么?

见他迟疑,姜海涛道:“秋公子不用多心,先生对秋公子颇为爱重,不愿公子卷入世俗中事,才令海涛资助贵国粮草,交换秋公子留在东海,这样一来,秋公子在师门那里也可以说得过去。等到风平浪静之后,公子再回北汉不迟。”

秋玉飞叹了口气,姜海涛之言确实说到他心里去了,比起那批粮草来说,自己是否留在北汉,已经是微不足道的事情了,这的确是一个好借口,可是抛下师门不理,这自己能够心安么?

姜海涛见他神色,已经知道他的心意,又道:“如果秋公子不肯留在东海,那么姜某也无话可说,只是贵国别想从滨州取走一分钱粮,就是拼着担上忘恩负义的罪名,东海也会即刻归顺大雍,如何选择,请秋公子仔细思量。”

秋玉飞不由苦笑道:“小侯爷这样说,难道秋某还有别的选择么?”

姜海涛微微一笑,道:“静海山庄是先生居处,藏书极多,其中有不少琴谱可以供秋公子赏玩,内子在山庄养病,若是秋公子有什么需要,在下又不在的话,可以去向内子说明,另外,医圣桑先生在山庄隐修,先生说若有机缘,公子不妨去见见桑先生。”

秋玉飞微微一叹,道:“静海山庄人间仙境,玉飞羁留在此,料想不会有什么苦楚,不过小侯爷真的以为大雍必胜么?”

姜海涛含笑不语,娶妻之后,他的性子沉稳了许多,只是说道:“兵危战凶,这等事情怎能说得准呢?”不过他心中暗想,先生既然已经出山,那么北汉灭亡不过是时间的问题,但是虽然不知为什么先生一定要将秋玉飞留在东海,但是他却知道先生对秋玉飞十分爱重,而秋玉飞虽然不曾明言,可是对先生也似乎以知己相许,所以这种伤人的话是绝对不会说了。

秋玉飞见大局已定,心中反而清明起来,心道,不论江哲是何等用心,可是他却明白我的心意,知道我不愿跻身血腥战场,这两国相争,不论谁胜谁负,和我又有什么相干,再说就是大雍胜了,难道我魔宗不能及时抽身么,而且大雍虽然势大,北汉铁骑也有十余万,沁州又是易守难攻,我何必为此忧心呢,不如在东海小住,避开战事风波的好,想得通透之后,越发对江哲生出知己之情,忍不住抚上琴弦,一曲《高山流水》从弦上流出,巍巍如山,洋洋似水,琴声一起,静海山庄万籁俱静,人人听得心旷神怡,灵台明净。

一曲终了,越青烟从外走来,道:“秋公子琴艺无双,青烟敬服,妾身师祖请公子前去一见。”

秋玉飞微微一愣,不过医圣何等身份,就是京无极在此也不会矜持不去,秋玉飞起身道:“敢不从命。”

在姜海涛、越青烟引领下,秋玉飞穿过重重楼阁,走入桑臣居住的百草轩。还没有走进房门,秋玉飞心中生出不妥的感觉,明明知道室内应是有人,可是却又觉得那人仿佛不存在,秋玉飞曾有过这样的感觉,那就是在师尊面前,难道静海山庄居然有这样一位宗师级高手么?秋玉飞微微苦笑,只怕姜海涛在这里向自己说出决定,就是担心无人可以压制自己,若是自己凭借武功反抗,只怕会碰个头破血流吧,江哲行事果然是毫无破绽,自己落入他的彀中,是绝对没有机会脱身了,不过奇异的,秋玉飞反而更加心安理得起来,既然自己根本就没有可能离开东海,那么屈服留下也就是别无选择的了。忍不住抬头看看明净的天空,秋玉飞只觉得心境前所未有的宁静喜悦。

放下东海传书,我披上大氅,走出营帐,如今已经是二月初了,雪尽冰消,春耕在即,军中士卒每日晨练的时候甚至已经赤膀上阵了,不过我仍然觉得冰寒刺骨,唉,昔日的重病仍然在我身上留下了许多痕迹,不过少林的心法的确不错,至少我手足都是暖的,虽然力气不足,可是却也不会走起路来就气喘吁吁了,想必这次北伐,我不会过分吃苦吧,只可惜不能躲在东海或者长安休养,大雍若是不能一统天下,我怕是没有机会作个尸位素餐之人了。

远处传来脚步声,心中泛起齐王的影子,我也没有回头,道:“王爷亲来,莫非是有什么大事么?”

齐王闷闷地道:“随云,你是什么意思,东海已经宣布中立,而且还送了一批粮食军械给北汉,我可不信这是姜家的意思,你在东海数年,别告诉我仍然不能控制那里的局势。”

我微微一笑道:“这是什么话,哲在东海养病隐居,怎会想着去控制东海姜氏呢,姜氏和大雍皇室是姻亲,小侯爷又受了陛下和王爷的大恩,如何劝服他们归顺不是你们的事情么,而且数月前姜氏不就再和朝廷商量招抚事宜么?”

齐王道:“好给我说这些冠冕堂皇的话,东海归顺大雍是大势所趋,也无人可以改变,只是这次为什么会突然中立,还支持北汉和我们作对,别告诉我是你暗中算计,若是皇上怪罪下来,本王可不替你说情。”

我漫声道:“好啊,到时候就让皇上治我的罪好了,最好去了我的爵位,我带着长乐回东海隐居,你说随着海氏的船去海外看看好不好?”

齐王啼笑皆非地道:“好了,你就别气我了,是不是你和皇上有了什么共识,我总要给下面的将领交代清楚吧。”

我沉默了一会儿,道:“什么时候王爷需要向下面的将领交代了?我可以说给王爷知道,不过下面的将领还是暂时不要知道的好。”

齐王过来的时候,我们两人身边的侍卫都散了开去,将周围护住,不让我们的谈话外泄。我也就没有顾忌地道:“现在东川庆王有了反意,南楚虽然被安抚下去,可是还要担心他们的反复,若是东海现在归顺大雍,南楚、东川迫于压力,一定不顾一切向大雍挑战,现在东海表示中立,而且支持北汉粮草,不论天下人做何想法,都会暂时松口气,甚至认为大雍会陷入和北汉的苦战中,能够拖延一下南楚和东川的动作,这是第一个好处。另外,大战一起,我们就可以截断东海和北汉的通路,所以北汉还是会陷入钱粮不足的困境,而且,我们这次作战可不是准备长期围困的,北汉钱粮充裕与否并不重要。这件事情我已经托长乐向皇上陈词,等到北汉灭亡之后,东海再归顺不是锦上添花么?再说未虑胜先虑败,若是这次进攻不顺利,东海还可以继续中立,维持和北汉的关系么。”

停顿了一下,我淡淡道:“再说,这样做,我还可以趁机留下秋玉飞在东海,我不想他死在战场上,他的琴艺举世无双,这样的人不应该死在沁州。”

齐王古怪的看了江哲一眼,道:“本王可不信你会因为私情作出这样的决定,说罢,你这次准备如何利用秋玉飞,上次用他施展反间计还不够么?”

我有些恼羞成怒,瞪了齐王一眼,道:“你急什么,等到了最后关节你自然知道了。”这人总是揭穿我的险恶心思。不过我也不由汗然,比起秋玉飞来说,虽然他对我存了杀机,可是他确实真诚的多。转念一想,我也不过是在保住他的性命的时候,让他替我作些事情么,否则他一个魔门弟子,我怎么冠冕堂皇的保下他呢?

齐王倒也知趣,见我气恼,便岔开话题道:“随云,对于这次出兵沁州,你可有什么计策么?”

我懒洋洋地道:“出兵的日子早就定了,殿下准备这次怎么做?”

这可说到了齐王的痒处,他兴奋地道:“走,到你帐内去说。”说罢大步流星地走入我的营帐,我也跟了进去,亲自取出一张地图放到案上。

齐王指着地图道:“我已经让荆迟带五万人提前出发,从镇州经太行白陉攻壶关,我自带大军十五万北上,辎重随后军走沁水,两路夹攻,在沁州合兵,你看如何?”

我心中已经有了定计,道:“殿下带十万人足矣,留下五万人在泽州,而且要多张旗帜,做出十五万大军的样子,另外沿途请殿下派出斥候和谍探,截杀北汉军斥候谍探,绝对不能让他们穿过大军防线。”

齐王眼中闪过寒芒,道:“随云,皇上和你可是有了什么计策么?”

我微微一笑,低声指着地图将自己的全盘计划说了出来,齐王一边听一边点头,最后傲然道:“或许用不到这一步棋呢,我的十万大军加上荆迟的五万,难道不能拿下龙庭飞么?”

我轻笑道:“若是殿下能够立下这样的大功,那就更好了,不过龙庭飞不是平常人,这次北汉必定倾全国之力抵抗大军,殿下不可轻视。”

齐王一边看着地图,一边若有所思地研究我的战策,最后终于道:“好,不过这样一来你还要随军北上么?”

我叹了口气道:“我也不想冒险的,可是我若不显身,只怕北汉谍探会拼了性命到后方探查军情吧,我可不想这样,不过一想到骑马坐车,我浑身都觉得酸痛。”

齐王笑道:“我令人给你准备一艘快船,你沿沁水北上,让你免受路途之苦,沁州路途不好走,你的马车派不上用场的。”

我们两人计议已定,这时帐外有人高声道:“王爷、监军,皇上旨意已经到了大营。”我和齐王都是兴奋的向帐外走去,按照时间,皇上允许出战的圣旨应该是这几天到了。走出营帐,天边正是阴云密布,想来天地也知道将有一场血战,因而为此忧心忡忡吧。

\chapter{第十章 沁水初战}

隆盛元年戊寅,二月十六日,太宗下诏,遣齐王显、楚乡侯江哲攻沁州,雍汉战事乃起。

——《雍史·太宗本纪》

隆盛元年二月二十七日,沁州最南端的防线,凌垣堡,战云密布,大雍边境封锁一冬,就是最精明能干的斥候也没有办法传出消息来,但是人人都知道大雍不会这样罢休,战事将起。

一座城堡孤零零地矗立在小山冈之上,冈下就是沁水南流,每年初春时节,冰雪融化使得沁水高涨,沿河各地都要提防沁水泛滥,但是今年看来水位不高,应该无碍,这一带河面宽阔,水流平缓,土地肥沃,两岸有十数村庄,而山岗上面的凌垣堡就是北汉军驻扎之处,这里也是沁州最前沿的战线,过了此处五十里,就是冀氏县城,沿沁水而上,到处都是碉堡城寨,易守难攻,而安泽、沁源、沁州城就是其中最重要的关隘。

一队北汉士卒站在城墙之上,留意着南面的动静,自从年后,上面传下军令,让他们时刻提防大雍军进攻,所以他们丝毫不敢松懈。一个士卒大概是有些倦怠,回过头去想和同袍说几句闲话,但是一回头却看见同袍目瞪口呆地看着前方,他下意识地回过头去,只见地平线上突然出现了青黑色的线条,不过转瞬之间,那青黑色越发浓厚,虽然十分遥远,可是在那士卒眼中,仿佛已经看到了大雍的军旗,他声嘶力竭的喊道:“快敲警钟。”一个有些发愣的士卒清醒过来,三步并成两步奔到钟楼,将铜钟撞响,然后号角声在城堡里响起,从各处营房奔出许多披挂整齐的北汉士卒。一个身穿偏将服色的将领奔到堡楼上,惊怒地道:“派出去的斥候怎么没有回报,快去点燃烽火。”他的亲卫匆匆走到城堡最高处,点燃了烽火。滚滚的狼烟直直地指向苍穹,自从大雍武威二十二年之后,大雍军第一次踏上了北汉国土,一场关系北汉生死存亡的大战即将爆发。

大雍军先锋夏宁,齐王亲信爱将,望见远处狼烟滚滚,不由哈哈大笑,勒马扬鞭,指向前方道:“他们纵然发现我军又能如何,小小的一个凌垣堡难道还能挡住我们的兵锋所指。众军听令,一举拿下凌垣堡,奉齐王将令,大军清野。”说罢一马当先奔去,身着青黑色衣甲的雍军高声呼喝,随着夏宁冲去,小小的凌垣堡就是奋起反抗,也不过是螳臂当车罢了。不过半个时辰,凌垣堡已经被攻破,雍军四面围住,北汉军无一生还。凌垣堡本就是负责探察敌情的战线前哨,一旦雍军大举进攻,凌垣堡不可能固守,所以派到此地的军士都是心存死志,雍军初战,也没有劝降的意思,铁蹄之下,骨肉成泥。

夏宁见凌垣堡已经攻破,令人毁去城门和守城器械,然后大军向四面的乡野杀去,这一次齐王颁下严令,不能在身后留下敌人。一座座村庄被焚毁,虽然青壮男子大半从军,可是北汉民风彪悍,就是壮妇和孩童老人也都随时可能拿起刀剑攻击雍军士卒,所以在夏宁的命令下,雍军铁骑几乎是将这些村庄堡垒碾成了废墟,而幸存下来的平民则被刀剑驱赶着奔向端氏、安泽。大雍军没有轻骑突进,而是一步一个脚印的稳步前进,所过之处,留下荒废的村庄和无人耕作的田地。唯一令北汉平民庆幸的是,雍军统帅齐王军令,不得滥杀平民,所以只要不反抗,不仅能够保全性命,甚至还可以有机会带上一些财物,只不过,除了北上之外,他们没有别的方向可以去。

沁水岸边,一群衣衫褴褛的老弱妇孺相互扶持着艰难的向北走去,队伍中只有几辆破车,上面装着一些米粮,几个实在无力行走的孩童和老人坐在车上,神情满是凄惶,他们都是体弱无力之人,基本上在北上的流民中已经落到了最后面,而雍军铁骑更是已经过去了无数,他们经常会遇到往来搜索的雍军。而将他们逐出家园的雍军将领说得很清楚,如果三月十日之前,他们不能赶到端氏,那么就将被当作北汉军的奸细处死。凛冽的春风从河面上吹来,让一些衣衫单薄的老弱缩成一团,沁州的春天仍然是十分寒冷啊,前途茫茫,想到可能会被雍军当成奸细处死,队伍中一些老人已经是泪尽泣血。

谁会想到雍军会用这样的手段呢?六年前雍军也曾攻入沁州,却对沿途村寨秋毫无犯,如今却是一律踏平,几个老人私下谈起,都说这也难怪,昔日统军的是如今的大雍皇帝李贽,今次却是齐王李显,谁不知道李贽宽宏,齐王残狠呢?

一个坐在车上的小孩儿目光无意中掠过河面,他突然惊讶地指着河心道:“爷爷,那里有大船。”跟在车边踉踉跄跄行走的老人举目望去,也是呆住了,只见沁河中央,百余艘大小船只正溯流而上,其中一只楼船最是巨大坚固,船头树着一面大旗,上面是一个大大的江字。船上甲士林立,周围二十多艘战船将楼船护在中央,其后是装满雍军辎重的货船。老人的惊呼让其他人也都转头看去,看到雍军的水军快船和船上兵甲鲜明的士卒,他们几乎是再也无力行走,上次大雍军进攻北汉,可没有使用这么多水军,这一次,想必大雍是势在必得了吧?

这时,那只楼船船头似乎有些骚动,几个眼力较好的半大孩童清楚的看见从顶层的船舱缓步走出三个人,其中一人排众而出,站在船头,手抚栏杆,向岸边望来。这人一身素色衣袍,外披青色大氅,远远的看不见形貌,只看见那人发色浅灰,应该是不年轻了,除此之外众人只能看见一双清润冰寒的眼睛,虽然隔得很远,可是那双眼睛却几乎是看透了他们的五脏六腑一般,让他们心中生出莫名的寒意。而在人群之中,一个相貌朴实的中年农夫却在看到那只楼船的一瞬间眼中闪过冰冷的光芒,但是他又立刻低下了头,仍然是那副苦闷烦忧的模样,还不时摸摸右腿,那上面胡乱包裹着一些布条,应该是一条伤腿,难怪他落在后面。

这时,众人身后传来轻悄的马蹄声,虽然声音不大,可是地面的震动仍然让他们觉察到了危机,几个农夫拿起锄头镰刀,想要尽可能的保护自己的家人,那些雍军不知道什么时候会杀人的。落入他们视线的是一支不过二三十人的小骑队,领头的是一个身穿青黑色软甲的女将,虽然穿着无法分辨身份的甲胄,可是这女子清艳无双,长眉入鬓,令人一见便知道这是一个巾帼英杰,她披着一件黑色披风,腰间悬着长剑,背后挂着弩弓。而她身后的随从也都是身穿软甲,佩着弩弓,武器却是这种各样,几乎是无一类同。

那支骑队在接近这支被迫北上的流民队伍的时候,自然而然散开,隐隐将流民队伍围了起来,一个骑兵高声道:“你们为何还在这里流连,难道不知军令森严,只需过了明日,若是不能进入冀氏,就是你们的死期到了。”那声音清越动人,却也是一个女子。

一个老人踉跄上前道:“军爷,我们这里都是无力快走的老弱妇孺,因此误了行程,请军爷宽待一二。”

那个女子转头看向那为首的女将,那女将目光一一从众人身上掠过,目光冰澈刺骨,凡是被她盯住的人都觉得死亡的阴影笼罩过来。那女子的目光落到了那个受伤的中年农夫身上,嘴角露出一丝讥诮,提鞭指道:“你,出来。”

那个中年汉子犹豫了一下,一瘸一拐地走上前来,那女子的目光时刻不离地望着他,直到他走到马前,那女子才冷冷问道:“你是萧桐麾下的密探吧?”

那农夫神态茫然,似乎不知道那女子再说什么,只是惊惶辩解道:“小人不是奸细,乃是本分的庄稼人,只因腿摔伤了,才被村人抛下,落到了后面。”

那女子冷冷一笑,道:“我苏青乃是谍探中的好手,你如何能够瞒过我的眼睛?”说罢,手中长鞭仿佛毒蛇一般刺向那农夫咽喉。那农夫目光一闪,作出不及反应的样子,只是惨叫闭眼,那长鞭果然一触即回。那农夫已经浑身冷汗,吓得软倒在地。那女子居高临下,冷冷看了他半晌,回过头去高声道:“前线总哨苏青求见监军大人。”声音清冽,人人都觉得仿佛苏青就在自己耳边说话一样,虽然离河心很远,可是楼船上面也有些骚动,显然是听见了苏青的声音。不多时,一艘快船向岸边驶来,那女将带马向岸边走去,其他的骑士也都策马离去,却是沿岸前行,显然是不准备上船,而那个最先说话的女子却落到了后面。那中年农夫松了一口气,正要起身,却觉得一枚冰冷尖锐的异物刺入了自己的咽喉,在他挣扎着抬头看去,只见那落在后面的女子目光冷然地看着自己。农夫眼中闪过激烈的怒意和迷惑。

下马走到岸边,苏青目光平静似水,彷佛不知身后发生了什么,即使那些流民发出压抑的惊呼。直到那个青年女子策马赶到她身边,她才淡然道:“如月,宁可杀错,不可放过,你做的很好。”那个女子在马上行礼道:“多谢小姐称赞。”然后接过苏青抛过来的马缰。

苏青飞身跃上战船,对着那名穿着纯黑色甲胄的虎赍卫士道:“多谢接应,监军大人可好?”那名虎赍卫士笑道:“大人惯于坐船,没有什么不适,苏将军想必带来了军报,大人正在等候呢。”

我站在楼船之上,淡淡的望着岸上的流民,虽然春风凛冽,可是却无法穿透我身披的大氅,虽然只有区区五百步的距离,却是两种不同的命运,我是衣锦绣、掌重权的敌国高官,他们是性命贱如草芥的流民。生在乱世,又是从风光秀丽的江南辗转多年来到冰霜凝聚的塞北,这种情形早已是司空见惯,就是以大雍的兴盛,也难以避免这种情况的出现,更何况是连年征战的北汉呢。只看这些流民大多是老弱病残,就知道北汉的境况如何。

轻轻叹了口气,我将目光转向前方,我亲手制定的计策不能推翻,这些人若是不能逃到冀氏,就只有死路一条,我既然将他们推到死亡的边缘,又何必用廉价的同情来掩饰自己内心的罪恶感,还是让心底的怜悯被无情掩盖吧,只要大雍一统天下,我就可以不用看着这样的人间悲剧重演。

站在我身后的小顺子突然上前一步,低声道:“公子还是回舱去吧。”

我回头看了小顺子一眼,从他的眼神里面看得出来,他是不想我因为那些流民而心中难过,这世间虽有我尊敬爱重之人,但是只有小顺子才是我的知己,我轻轻一笑,低声道:“你放心,我素来自私怕死,你又不是不知道,怎会为了这些不相干的人动心。”

小顺子没有作声,站在我身后也没有退回去,我心中越发温暖,方才所说并非全是安慰的言辞,我不过是个平常的凡人,无力顾及天下苍生,除了我自己和我身边的亲人挚友,同僚下属,我也顾不得更多的人了。

呼延寿这时扬声道:“大人,前线总哨苏青苏将军求见。”

我点头道:“请苏将军上船。”苏青是一个我很赏识的将领,虽然是女子,却比大多数男子都冷静聪明,心思更是无情狠辣,这次我和齐王一致同意让她出任前线斥候总哨,负责探查军情,截杀北汉军的斥候谍探,这次想必是途经沁水,看到我的楼船,所以过来拜见我这个监军大人吧,这也是军旅中的不成文的惯例,而且按照我的估计,我军和北汉军还没有正面开战,应该不会有什么紧急军情的。

不多时苏青上得船来,果然如我预计一般,并没有什么紧要的事情,但是从苏青的语气中,我却听出她心中疑惑,为了大军清野的需要,十数日来仍在沁州边境徘徊,若是全力行军,只需两日就可以到达冀氏,可是为了将沿途碉堡民寨清除,大军至今仍然在这一带徘徊,所谓兵贵神速,也难怪她心中不解。不过她性情沉稳,并没有明着质疑,只是流露出对行军速度的不满。

我也无意对她解释,问道:“苏将军,派到流民中的我军谍探是否已经进入冀氏?”

苏青摇头道:“冀氏守将十分谨慎,将所有流民都挡在城外,并且让他们按照乡里编排安置,又设立了保甲连坐制度,我们的谍探虽然潜伏多年,因此没有被剔除出去,可是却是行动艰难,消息更是无法传递,攻打冀氏的时候恐怕是没有用处了,而且末将得到情报,冀氏已经得到命令,正在将那些流民和冀氏一带的平民迁入沁州腹地,只留下一些青壮男子帮助守城。”

我轻笑道:“北汉防守以段无敌为第一,想必是他的主意,他们想必已经决定用坚壁清野的,步步为营的方式迎战,这也不错,我们第一步本就是要清野,让两军战场之间没有平民的存在,他们这样倒是助了我们一臂之力,不过他们也是不得不尔,若不如此,不需我们大军进攻,冀氏就会被流民破城了。”

苏青犹豫了一下,终于问道:“大人,末将有一事不明,这些平民无害于大局,为何大人执意要先清四野呢,莫非是要胁民为前驱么?我大雍堂堂大国,为何使用这种手段,这样一来,对于大雍在沁州的统治恐怕会有很多障碍。”

我眼中闪过精光,想不到这个苏青还有这样的见地,并不仅是一个谍探的才能,赞赏地道:“苏将军能够看到这一点,可谓目光深远,驱民北上也是迫不得已,其中关键暂时还不能说给你听,我令齐王殿下严申军令,尽量不要滥杀无辜,这样一来,总有大半平民可以安然逃生,而且沁州历来是北汉和大雍对敌的前线,这里的民众也对大雍颇为仇视,所以就是他们更加怨恨我军,也顾不得了,就像泽州之民,对北汉何尝不是万分痛恨呢!”

这时前面突然传来一阵骚动,我下意识的看去,只见十余里之外河流转弯之处突然出现了悬挂着北汉军旗号的战船,不由心中一惊,北汉历来没有水军的编制,一支水军耗资无数,对于北汉来说,战马易得,骑兵易练,水军却是很难操练的,所以历来北汉军除了战时征用民船运送辎重之外,基本上没有使用水军作战的例子。不由看了苏青一眼,她在北汉多年,怎么没有发现水军的存在呢?

苏青也是脸色铁青,她负责在北汉的情报网,竟然没有发觉北汉军中有这支水军的存在,这不仅是重大的失职,也是莫大的耻辱,她冷厉的目光越过河面,这时候雍军前方的战船已经摆开了阵势准备迎敌了,雍军的水军虽然不如南楚水军那般善战,可是比起从未听说过的北汉水军来说,应该是颇为强大了。

北汉水军顺流而下,不过片刻就已经清晰可见,我看到那些战船,不由心中一叹,那分明是南楚水军常用的艨艟斗舰,造一艘战船少说也要一年半载,仔细看去,那些战船分明还是崭新的,想必是在去年泽州大战之前就在筹备水军了,看战船外形,应是南楚提供了工匠,如今通过海运,关山阻隔再也不是问题,难怪北汉也能筹建水军,不过想到其中耗费的人力物力,北汉军能够有这样的魄力可是不易的很啊。如今我军虽然有楼船一只,战船百余艘,可是比起北汉水军的艨艟斗舰,在速度和攻防上都落了下风,更何况我军还是在下游呢,事先没有预料到这种情况,泽州水军战力不强,看来我军要吃亏了。

沁河水道不宽,我眼看着那船首装着鹿角,船身涂以桐油的艨艟分成三列,向雍军战船撞来,不由叹了口气,想起昔日在南楚时候见过的水军作战的情景,犹豫着是否介入大雍水军将领的指挥。这时负责统领泽州水军的统领庄汝早已站到我身边,也顾不得向我请示,挥舞旗帜传下军令,我只看了片刻就放了心,看来这人指挥水军经验丰富,就是到了南楚也可以一战的,更何况只是新出茅庐的北汉水军呢。只见他下令让雍军战船分散开来,避开北汉水军的正面攻击,全力攻击两翼,沁水之上立刻弓箭如雨,水上作战,弓箭为先,更从战船上放下许多小型艨艟,利用船小高速的优势,身如北汉水军的防线。一时之间,沁水之上杀声震天,枪戈蔽日。

我望着两军作战,虽然船只优劣不同,将领战术也有参差,可是仍然有可观之处,看来都在水军上下了功夫,不知怎么我竟然想起了南楚,大雍和北汉都在发展水军,可见都有着南下的野心,可是南楚除了德亲王曾经力排众议建立了一支骑兵之外,仍然是以水军和步兵为主,据我所知,德亲王死后襄阳骑兵被南楚朝廷消减了不少,精锐程度大不如前,只看各国在军力上的投入,就知道南楚是落在最后面的了。

正在我心中隐隐惆怅的时候,庄汝过来道:“大人,末将要将敌军主力诱入包围,需以楼船作为诱饵,请大人暂时到舱中躲避,或者先到别的战船上面暂歇如何?”我淡淡看了他一眼,庄汝,二十七岁,面庞微黑,相貌平平,个子中等,身躯雄壮,性情沉静,乃是大雍寥寥无几的水军英才,唯一的弱点就是性情太过刚正,最看不起贪生怕死的文官,我甚至能够从他的眼睛里看见暗藏着的对我的轻视。他资历尚浅,可能对他来说,我不过是一个文弱书生,擅长阴谋诡计,运起又不错,得到皇室的青眼罢了,毕竟我的事情有很多都深藏云雾之中,不是他这种身份的将领可以知晓的。

故意不去理会他言语中暗藏的轻视,我淡淡道:“既要诱敌船来战,呼延寿,令虎赍卫士高声呼喊,就说是泽州大营监军,楚乡侯江哲在此。”

呼延寿略一犹豫,但是却被我淡然而坚定的语气震慑,传下令去,他带头高声呼喝道:“泽州大营监军,楚乡侯江哲在此,敌将若有胆量,可敢来战么?”

北汉水军主舰之上,一个身材高大的将领眼中闪过火热的光芒,振臂道:“儿郎们,生擒江哲,大破泽州水营。”随着他的命令,北汉水军攻势越发猛烈,两军都是拼死作战,只见战船往来交错,不时有战船倾覆沉没,过了片刻,北汉军三艘艨艟已经冲到楼船旁边,已经有敌军向楼船上面攀爬而来。我高声道:“呼延寿,你们皆听庄将军将令。”

庄汝眼中闪过一丝感激,连连传下军令,指挥楼船上面的水军和虎赍卫士作战,这些虎赍卫士虽然不擅长水战,可是他们个个都是武技高强的战士,而且已经能够在楼船上面往来自如,至少在比较风平浪静的沁河上是这样,所以北汉军除了少数勇士,根本无法攻上楼船。庄汝得空道:“大人,这里太危险,您先到舱中休息吧。”这一次他的语气十分诚恳。

我微微一笑,高声道:“江某虽然文弱,但是有我大雍诸位勇士保护,何惧北汉强攻,今日江某就在此处,看诸位大胜敌军。”那些水军和虎赍卫士都是精神一震,高声呼喊道:“大人信任我等,我等必要死战。”一时之间,大发神威,将那些攻上楼船的北汉水军逼退杀死。一艘艨艟上面指挥的一个英俊挺拔的青年将领厉喝道:“看箭。”弓弦声响,三支鹰翎箭快捷无比地射向我的面门,以我的眼力看去那羽箭也是快如流星,一些在我们两人之间直线上面的水军和虎赍卫士都是怒喝着想挡住羽箭,却都慢了一线,只有一个虎赍卫士横刀劈下,将一支羽箭斩断,但是羽箭前面的半截几乎是速度不减地射向我,而那个卫士却虎口巨震,横刀几乎脱手,双方距离不过二十多丈,也难怪他们无法阻挡。

就在那两支半羽箭将要临身之际,我面前突然出现一只白皙如雪的手掌,中指轻弹,三声脆响,那两支半羽箭被倒震而回。我早知道小顺子能够保住我的平安,面色丝毫没有改变,目光落到那射了我一箭的北汉军青年将领身上,我大声笑道:“若是有人取此人首级来献,赏黄金五十两,若是生擒此人,赏黄金百两。”

众人更是精神振奋,突遇强大水军的隐忧早就无影无踪,主帅既然要他们生擒敌将,看来自己一方已经稳占上风了。有几个大嗓门的虎赍卫士已经高声呼喊道:“那敌将还不束手就擒,百两黄金老子可是要定了。”那青年将领面色铁青,指挥麾下将士竭力攻打楼船,两军酣战不休,杀声震碎浮云。

第十一章    清野血战

隆盛元年戊寅,三月初九,大雍泽州水营与北汉沁州水营战于沁水,雍军辎重半毁,北汉水军副统领刘岱,瑾郡王第四子被俘。

——《资治通鉴·雍纪三》

无聊的抬头看看满天的羽箭,我从容自若地站在楼船之上,实在是因为这一带河流并非特别宽,小顺子足以在危急时候带我上岸逃走,所以我也就表现出冷静无畏的模样,若是真的有危险,只怕我早就让小顺子带我离开了。看看眼前混乱的河面,我站得有些累了,很想有张椅子坐下,不过考虑到鼓舞士气,还是得直直地站在那里。已经打了将近一个时辰了,近处应该有雍军过来支援,可是我抬头四望,却是没有人影,心中不由忐忑不安,莫非北汉军已经出来挑战了么,现在冀氏不稳,他们怎会在这个时候出战。

正在我心中盘算不停的时候,苏青在我身后冷冷道:“大人,末将仔细想过,这支水军应该是去年年初新建的,那个指挥水军的将领是北汉国主心腹将领吉盛,末将得到情报知道他在沁水上游建立新军,不过吉盛历来和龙庭飞不合,末将得到的情报是说他请旨训练新军,是为了和龙庭飞对抗,因此末将并没有特别留意,现在想来他们应该是利用沁水源头的湖泊训练水军。因为有魔门高手保护,我们派过去的斥候都无法渗入那里的防线,而且末将那时奉命在沁州一带主持大局,致有这样的疏漏,还请大人恕罪。”

我摆摆手道:“事已至此,多说无益,不过那吉盛原本应该不是擅长水军的吧,怎么会当上了水军统领。”

苏青想了一下道:“末将看北汉水军的战船,应该是南楚的制式艨艟,想必是有南楚水军将领帮助训练吧,吉盛虽然也是骑兵将领,但是他出身却是沁水渔夫,至少比别的将领合适吧。”

我指着那个方才射我三箭的青年将领,此刻他已经带了几艘船力图冲破阻截,去对付辎重船,见他骁勇善战,我不由颇为心动。苏青看了他一眼,眼中闪过寒芒,道:“此人乃是北汉王室宗亲,瑾郡王第四子刘岱,瑾郡王诸子大都不成材,只有这个庶子文武全才,原本有立为世子之意,不过郡王妃出身北汉名门,自然不肯让世子之位脱出手去,屡次为难刘岱,因此瑾郡王被迫将刘岱送到军中为将。想不到此人竟然已经成了水军将领。”

我惊叹道:“北汉王室果然人才辈出,这刘岱原本恐怕也是骑兵将领,学习水战不会太久,如今虽然仍有些不足之处,可是已经极为难得了,若是能够生擒此人,那么这一战我们就是小小挫败,也是值得的。”我看他几次冲击,都未能冲过我水军阻拦,去攻击后面的辎重船,不由心中一动,想了一下,对庄汝低声道:“可不可以将他放过去,然后拼上小半辎重,将他擒杀,此人乃是北汉宗亲,又是水军新秀,若是能够擒杀此人,北汉水军必然士气受挫,到时候沁水之上就是我军的天下了。”

庄汝为难地道:“若是辎重损失,只怕齐王殿下怪罪下来。”

我笑道:“只要擒杀此人,我一力承担就是。”

庄汝脸上露出宽心的神色,挥动手中旗帜,不多时,那刘岱果然顺利地冲破了大雍水军的防线,他惊喜地率军冲去,那船上水军都用上了火箭,一时之间江上烟火缭绕,好几艘辎重船都被点着了,我知道他的用意,要将那些辎重焚毁,重重打击我军士气,而且他烧尽辎重船之后还可以前后夹攻,攻破大雍水军的船阵。他冲杀得顺利,带动了许多北汉军战船也从那个缺口穿越过去,那些战船本来渐渐陷入雍军船阵,如今见到机会,都向后杀去。杀得顺利,北汉水军大都没有注意到,除了大半辎重船知机后退之外,还有十余艘辎重船在初时庄汝下令放开防线的时候就向两边闪开,隐隐将刘岱带来的战船围住。庄汝脸上露出杀机,一声令下,这些辎重船好像失去控制一样向中流冲去,船上水军点燃了辎重粮草,纷纷跳水逃生,十几艘火船将刘岱等人困住。

那青年将领一见之下,神色惨白,他是顺流而下,知道无法即时转舵回头,只得下令继续前冲,这时候,原本退后的辎重船有几艘在江心下锚停住,已经横阻在水面上,刘岱的战船冲过烟火之后正好撞在其上。那些辎重船上的雍军水军齐齐放出火箭,那些辎重船也是烈焰冲天,将刘岱那十几艘战船困在了火海当中。

这时候那北汉水军统领吉盛见雍军后方大火熊熊,视线被烟火阻隔,原本还在高兴刘岱烧了敌军辎重,谁知不多时从后面传来凄厉的号角声,吉盛一听只觉得心底冰凉,显然刘岱已经陷入绝境,虽然有心救援,但是眼看着雍军战船四面蜂拥而至,知道若是再战下去,必然无幸,只得下令退兵,北汉战船速度超过雍军,不多时成功地消失在雍军视线之外。

庄汝见敌军已经退走,连忙下令打扫战场,收搜俘虏,留下的北汉军几乎全部战死,他们的悍勇让雍军将士也心中感佩,只有死战到底的刘岱最后被几个水性好的雍军水鬼掀翻在水中,生擒活捉。这一战,雍军损失了十八艘辎重船,十九艘战船,而北汉军损失了七艘艨艟,十二艘斗舰,虽然比较起来,雍军还是败了,但是水军上下却都是一片欢声笑语。这次北汉水军毫无征兆地偷袭被击退,有了准备的雍军就可以争霸沁水了,他们有足够的手段让北汉水军无法南下,至于他们也无力取胜的事实并没有让他们担忧,毕竟泽州水军的主要目的就是运送辎重,而非是和北汉水军作战。而庄汝等人更是知道,生擒刘岱的事实,足以让新建的北汉水军失去信心,所以更是兴高采烈,至于损失的辎重么,他们就不会放在心上了,谁让我一力承担了呢。

我高兴地付出了百两黄金,让那几个生擒刘岱的水军自己去分配,让人将被江水灌得晕头转向的刘岱关入底舱。然后我回到舱房,苦着脸给齐王殿下写了一封信,向他说明损失辎重的情况,虽然我说同意庄汝牺牲一些辎重,可是十八艘也有点太离谱了,不过想到手上奇货可居的刘岱,我还是得意的笑了。

这时候呼延寿走了进来,神色凝重地道:“大人,援军到了。”

我一边奋笔疾书,一边问道:“怎么回事,我记得附近应该至少有千余骑兵的,他们不能水战,可是沁水河面不宽,他们可以在岸上使用弓弩射杀那些北汉水军的,怎么却来得这么晚,莫非没有看到我们求援的信号么。”

呼延寿悻悻道:“属下已经问过领军的将领,附近只有一些百人规模的小股骑兵,他们见到求援的信号之后,纷纷前来救援,谁知有人手段通神,居然连续狙杀了大半骑队的将领,这些骑兵被迫去追杀刺客,现在是一团混乱。”

我手一抖,一滴墨迹落在白纸之上,我看着被墨迹弄污的信纸,叹了口气,将那封未完成的书信随手扔到了船舱一角的火炉里面,放下羊毫,我面无表情地站起身道:“是一个人做的么?”

呼延寿黯然道:“是的,从行刺手法来看应该是一个人,而且我军清野多日,绝不可能有太多的刺客谍探留在这一带。”

我陷入沉思,抬头看向小顺子道:“你可有这样的手段?”

小顺子冷冷道:“那人武功不弱于我。”

我冷冷笑道:“你说北汉有几个人武功可以和你相提并论呢?”

小顺子想也不想地道:“应是段凌霄亲至,京无极不会出手的。”

我想了半晌,犹豫地道:“小顺子,你说段凌霄会不会继续留在这一带,如果他要刺杀我或者齐王应该都不容易,可是若是刺杀那些低级将领就易如反掌了。”

小顺子冷冷道:“段凌霄若是留在这里,只能是混在流民当中或者藏在野外,公子不妨立刻命令负责清野的骑兵以五百人为一队,互相呼应,将所见北汉人尽皆屠尽,让段凌霄无法藏身,就是段凌霄再想刺杀,也难以轻易接近我军,若是他勉强为之,那么五百骑兵足可以将他死死拖住,等到我军高手赶去之后,就是段凌霄武功再高,也难以逃生。”

我仔细的想了一下,道:“事情紧急,也不能禀报齐王殿下知道了,呼延寿,传我谕令,令我军提前清野,另外派人报知齐王殿下知道。”

我连忙写了十几封军令,盖上我的监军大印,然后令人传下去,我虽然是监军身份,不能直接调动军队,但是这种情况比较特殊,我只是要求提前行动,我的监军大印应该是好使的。而且我这也是为了那些中低级将领考虑,若是他们不爱惜自己的性命,那我也就顾不得他们了。当然我还是特意写了一封信向齐王通报,为了安全送到,我请苏青亲自送去,虽然她也不是段凌霄的对手,但是我总不能让小顺子去送信吧,毕竟我的性命才是最重要的。

荒草漫漫的驿道上,一支骑队疾驰而过,为首的正是苏青,身后则跟着一些身穿青色甲胄的骑兵,她奉命去向齐王禀告军情,因此快马加鞭,片刻不敢停留,此时,附近的军队都已经得到了江哲提前清野的命令,幸而北汉平民大多已经逃到冀氏,所以一路上走来,倒没有看到过多的屠杀场面,何况苏青心硬如铁,就是看到那种凄惨的景象也不过是一晒而已。她走得匆忙,除了她的亲信侍女如月之外,只带了江哲派给他的骑兵,那刺杀雍军将领的刺客应该还没有被擒杀,所以苏青一路上小心谨慎,丝毫不敢大意。

突然,苏青眼光掠见前方路面的歇脚亭里,一个灰衣人负手而立,苏青眼光何等敏锐,一眼看去,就已经将这男子形貌看的清清楚楚,只见他三十多岁年纪,身子峻挺犹如青松伟岸,相貌端方刚正,双目幽深,宛若夜空一般深邃,令人生出无法揣测的感觉。

苏青勒马而住,这些战马都是饱经训练,苏青一住马,那些后面的战马也都及时停住,原本狂奔的骑队静止下来,那些骑兵也都知道刺杀之事,心中都生出杀机,二十多人的杀气汇聚在一起,令得这一小块天地都仿佛凝固下来。那灰衣人目光闪过,也不由惊叹这支骑兵的精良,他缓缓上前一步,淡淡道:“姑娘可是大雍军营的总哨苏青?”虽然是疑问的语气,但是人人都觉得他心中早已这样认定,问这一句不过是为了确认罢了。

苏青冷冷道:“原来是魔宗首座弟子段凌霄亲至,段爷莫非不知道螳臂焉能当车,我大雍铁骑千万,阁下何必做这种无益之举。”

段凌霄微微一笑道:“姑娘说得不错,段某武功虽然高强,但是一人之力比不过千军万马,只是有些事情做了总比不做好,不久前姑娘在沁水岸边杀伐决断,段某十分佩服,段某的师弟萧桐曾经向在下详细述说了姑娘的丰功伟绩,段某不由想见见你这位女中豪杰。今日道左相逢,幸何如之,姑娘不如下马过来,我们叙谈一下可好?”

苏青眼中闪过热烈的光芒,道:“能够和阁下一谈,苏青深觉荣幸。”说罢翻身下马,向歇脚亭走去。她的侍女如月高声道:“小姐,他定是要截杀于你,怎可和他叙谈。”

苏青笑道:“段凌霄是何等身份,未来的魔宗宗主怎会出尔反尔,既然相邀苏青一谈,若是竟然不告而诛,岂非贻笑天下。”

段凌霄眼中闪过激赏的光芒,他自然不屑于和如月计较,只是冷冷看了她一眼,对苏青道:“苏总哨巾帼不让须眉,难怪萧师弟将姑娘视作生平大敌,我秋师弟对姑娘也十分仰慕,今日一见果然是闻名不如见面,苏姑娘,你本是北汉人,只为了私仇家恨,却替大雍张目,真是可惜可叹。”

苏青傲然一笑,道:“阁下是认定今日可以取苏青性命,所以才会觉得可惜可叹?北汉无恩于我苏青,就是为了报仇雪恨,苏青归附大雍也无不可,而且如今大雍据有中原,北汉南楚不过是苟延残喘,北汉魔宗纵然英杰无数,大势如此,又能奈何,若是阁下肯弃暗投明,必然位在苏青之上,何必还要抱残守缺,以至身死国灭。”

段凌霄眼中寒光一闪,道:“罢了,我也知道苏姑娘不会回头,只不过心中有些不忍,姑娘可知这一次为何雍军大肆驱赶屠杀平民,若是姑娘肯直言相告,段某可以不杀害姑娘属下的性命。”

苏青微微一笑,虽然知道段凌霄这样说是表示定要杀死自己,却不放在心上,道:“苏青不过是斥候总哨,这种军机大事如何知晓,阁下是问道于盲了。”

段凌霄冷冷道:“果然如此么?苏姑娘可知道我为何突然大开杀戒?”

苏青想了一下,神色凝重地道:“自然是不让这些骑兵救援水军,想必段大爷很希望我水军一败涂地。”

段凌霄淡淡道:“你说得不错,自从雍军入沁州之后,我便前来查探军情,这次雍军入侵,声势浩大,生死存亡在此一战,段某也不得不亲自出马。数日前看到大雍水军,得知楚乡侯江哲在水军之中,将消息传回之后,龙将军下令水军出战。能够一举攻破水军,断去雍军粮道自然很好,就是不能,若是趁机阵斩江哲,也是大功一件,为了此事,我不惜纡尊降贵亲自出手,刺杀了来援的各军将领,可惜大雍水军毕竟战力较强,结果只是差强人意。段某本想立刻离去,却又见到姑娘下船,想起姑娘的身份地位,想必知道很多机密,因此冒险前来阻截,若是姑娘肯将心中隐秘尽皆说出,段某可以不取姑娘性命,否则苏姑娘最好希望战死当场,若是被段某生擒,只怕种种酷刑会令姑娘悔不当初。”

苏青眼中闪过漠然的神色,道:“苏青早已经将生死置之度外,阁下如此威胁苏青,却也没有什么用处。”说罢,冷然退后,而那些护卫她的骑兵也已经纵马环绕在她身后,隐隐将她护在其中。说到这里,段凌霄和苏青都知道已经言尽于此,接下来只能凭武力说话了。

段凌霄轻叹一声,道:“苏姑娘如此人才,却是大雍之臣,真是可惜。”随着他惋惜的语声,天地间仿佛突然多了肃杀之气,人人都知道他即将出手,不由提气戒备,可是段凌霄却是没有一丝举动,只是从他身后涌出无穷无尽的杀气,迫得那些骑兵心中生出拼死一战和弃械投降两种念头,不过这些骑兵都是身经百战的勇士,虽然多半不是内家高手,却也都从战场习得比拼气势的技巧,也都将心中杀机肆意放出,一时之间,双方气势竟然旗鼓相当。

段凌霄眼中闪过一丝无奈,大雍有这样的精兵,难怪可以雄霸天下,比较起来,北汉的将士虽然勇猛凶悍,个人战力多半都在大雍勇士之上,可是若是组成军阵,却不免要逊色一些。不过他乃是先天高手,不过瞬息之间,就已经将心中杂念全部屏除干净,就连杀机也消退得无影无踪。那些大雍骑兵本来正竭力在那如同海潮一般的杀气中支撑,突然之间杀气消失殆尽,那些骑兵顿时失去了对手,都觉得心口一震,有几个战力稍弱的骑兵已经是面色苍白,更有一人,一口鲜血已经溅到马鞍之上。就在他们由最强转为最弱的瞬间,段凌霄已经出手。

苏青只觉得眼前一花,段凌霄的手掌已经拍向自己的面门,她翻身后退避开,寒光一闪,她拔剑还击,掌剑相交,却是声如金石,苏青只觉得虎口一麻,长剑几乎脱手,她深吸一口气,借力后退,段凌霄如影随形,两人战在一起,剑光雪影中夹缠着青灰两色的身影,令得那些骑兵无从相助,只能散开将两人包围起来,人人手上都取出了弩箭,准备适时射杀段凌霄。

苏青使出了浑身解数,剑浪一波高过一波,段凌霄却是如同海中巨礁,任凭风吹雨打也不低头,苏青遇到这样的强手,只觉得剑法从未施展得如此畅快,即使是上一次和秋玉飞交手也没有这样的感觉,因为秋玉飞武功灵巧机变,苏青速度身法都不如他,应接不暇之余,那里还能尽情施展剑法,反而是段凌霄的武功雄奇刚烈,让苏青更能发挥所长,使到酣畅处,剑影化作滔天巨浪,瑰丽中显露出杀机无数。段凌霄武功远远胜过苏青,虽然一时之间不能取了她性命,但是却是游刃有余,见到苏青这样的剑法身姿,眼中闪过异样的光芒。一声铮鸣,段凌霄袖中滑出一柄雪亮的短刀,无数声兵器撞击的声响震耳欲聋,硬生生接下苏青这一番猛攻,段凌霄的断刀化作流虹,一刀快似一刀,如同出水蛟龙一般穿破苏青的剑网。

苏青已经竭尽全力,猛攻之后的一丝破绽被段凌霄生生击破,她生性坚毅,间不容发之间右手长剑脱手向段凌霄射去,左手一柄匕首挡住了那柄断刀的锋刃,一声巨响,她的娇躯如同断线风筝一般向后坠落。段凌霄一声长啸,追击而去,这时,那些在外围掠阵的骑士同时高声呼喝,弩机齐响,几乎看不清影子的二十多支弩箭射向空中的段凌霄,段凌霄衣袖挥舞,那些弩箭如同遇到无形的墙壁一般停顿下来,反射坠落,这时,第二波、第三波弩箭已经射到,段凌霄身形如同风车一般在空中轮转,那些弩箭反射激回,两名骑士被反射的弩箭射落马下。但是段凌霄的行动也被延迟了片刻,这时,如月已经飞马而过,将苏青拉到马上,苏青吐出几口鲜血,大声道:“走!”如月带马向来路奔逃,那些骑士一边以弩弓阻拦段凌霄追击,一边策马追去。段凌霄眼中闪过一丝冷然,抓住苏青抛下的战马缰绳,策马疾驰追去,苏青的坐骑乃是千里挑一的骏马,段凌霄又是骑术高明,不到片刻已经追上了众人。

段凌霄冷冷一笑,凌空出掌,将最后面的一个骑士击落马下,骑马掠过他的坐骑时,随手取下他鞍边马槊,马槊闪过千百道幻影,两个骑士被他刺落马下,不过片刻间,他就已经追到了因为驮着两人而落在后半部的如月马后,苏青此刻正伏在如月肩上,似乎已经昏迷过去。

段凌霄眼中闪过寒芒,一槊刺向苏青背心,就在这时,苏青突然向侧面卧倒,如月则是俯下身去,苏青手中露出一具弩弓,弩机轻响,三枚弩箭同时射向段凌霄,此刻两人距离不过两丈,马槊又是长兵器,无法阻挡弩箭,幸而段凌霄骑术过人,他的身躯仿佛突然折断一般向后仰去,一支弩箭从他面门上掠过,一声凄厉的马嘶,段凌霄只觉身下一软,战马狂奔出十几丈路程,颓然倒地,段凌霄飞身跃起,身形向地上落去,同时马槊脱身而出,空中闪过一道奔雷掣电也似的乌光,射向已经从马上起身的苏青。苏青方才已经是用尽浑身之力才能完成仰身射箭这一举动,坐起身来,正是手足虚软有心无力之时,见到马槊射来,她再也无力闪避,苍白如雪的容颜上露出一丝令人心寒的微笑,她宁静的等待着马槊刺入自己胸口的瞬间。

第十二章    紫烟遗尘

隆盛元年戊寅,三月十二日,冀氏城破,雍军焚城,虽冀氏守将迁民安泽,然老弱不能走者不可胜数,冀氏死伤叠累,齐王显凶名益盛,然细察之,并无屠城之事。

——《资治通鉴·雍纪三》

就在千钧一发之际,一声厉喝传来,从道路两侧的荒草之中飞射出一柄投矛,正撞击在马槊中部,马槊偏离了一些方向,但是仍然向苏青射去,但是这短短的时间已经让如月行动,她抱着苏青滚落马下,跌倒在尘埃,这时,她的坐骑似乎也被风雷之声惊住,扬蹄人力而起,那柄马槊穿透马身,那匹骏马一声长嘶,向地上跌去,如月一落到马下,就抱着苏青向旁边滚去,那沉重的战马尸身只以毫厘之差,压倒在如月身边。

几乎是同时,段凌霄觉察到从道路两边涌来无穷的杀机,他下意识地纵身而起,当他身形跃到空中,无数弩箭向他射来,段凌霄狠狠吸了一口真气,身躯诡异地在空中折转方向,向旁边飘飞,那些飞舞的弩箭几乎是撞击在一起,同时他抢来的战马也在嘶鸣中倒地。段凌霄飘飞落地,从道路两边的岩石和深草中跃出十八个身穿黑色骑装,外罩软甲的青年战士,将段凌霄围在当中,这些青年人人手中都是横刀持盾,几乎每个人都是二十五岁到三十岁的年纪,个个神态沉稳,足下尘土凝而不散,眼中精光闪耀,一见便知是大雍军中千里挑一的好手。还有一人大约二十八九岁年纪,相貌朴实,但是双目寒光四射,浑身杀气隐而不露,只看神情气度就知道此人乃是为首之人。他手中也是一柄横刀,左手拿着精钢小盾,但是此刻横刀没有出鞘,腰间插着两支短矛,正是这人方才救下了苏青。

段凌霄叹息道:“你等是何人,怎会在这里拦截于我?”

那为首青年朗声道:“大雍皇帝陛下御前虎赍卫副统领,楚乡侯属下侍卫统领呼延寿奉楚乡侯之命,在此恭候阁下。”

段凌霄眼中寒光一闪,道:“这是江大人设局诱我入伏么,那么他也未免太不爱惜手下了,你们自信可以挡住我么。”

呼延寿高声道:“阁下不用挑拨离间,大人神机妙算,知道若是阁下仍然在此,十有八九会袭击苏将军,因此命我等暗中跟随,方才苏将军遇袭之时,已经将警讯传回,因此苏将军舍命向来路奔逃,将阁下诱入死路,我等新近学了一套刀阵,特向阁下领教。”

段凌霄淡淡道:“楚乡侯果然够谨慎,若是我不出手,他不过是多事罢了,若是我出手,他就可以寻到我的踪迹,不过他的心肠也够狠毒,若是苏青没有本事逃走,他不久平白损失了一员得力属下,苏姑娘这等人才,被他当作牺牲,岂非可惜得很?而且他派人设伏,却不让他的心腹手下邪影李顺前来,只让你们前来送死,这等心狠手辣,贪生怕死的人物也值得你们为他送命么?”

呼延寿眼中闪过怒色,冷冷道:“我家大人为人如何还轮不到阁下评价,心狠手辣,本就是好男儿的本色,若说大人贪生怕死,昔日也不会在凤仪门主面前傥傥而谈,何况李爷乃是大人近侍,本就不必上阵杀敌,我等武技都经过李爷指点,就请阁下指教一下如何?”

随着他的话语,那些虎赍卫士各自踏前一步,蓦然收缩的阵势气势顿时高涨,但是在颇精奇门遁甲阵法变换的段凌霄看来却隐隐露出不少破绽,不由微微一晒,这时,呼延寿已经拔出横刀,执盾上前,就在他入阵之后,这座刀阵却变得法度森严,所有的破绽都已经消失不见。段凌霄心中一惊,原本以为这刀阵是正反九宫合并而成的刀阵,想不到真正的人数却是十九人,原本的似是而非令他这懂得一些阵法的人心中轻视,而在呼延寿入阵之后,天罗地网已成,这种突然的打击足可以令被陷入阵中之人心志受挫,若是设阵之人乃是针对自己而来,那么他的心志可就太可怕了。段凌霄终于忍不住,在刀阵没有发动之前,出言问道:“这刀阵是何人所授,呼延将军最后入阵可是一贯如此?”

呼延寿微微一愣,本要下令厮杀的话语也被堵了回来,心道,你纵然想要拖延时间,也没有关系,此刻当有百余铁骑正向这里赶来,等他们到来,你就是三头六臂,也逃不出去,因此呼延寿答道:“阵法乃是江大人所授,刀术是李爷亲传,原本是为了保护大人安危,今日用来除奸,也是一桩美事。”

段凌霄听到这里微微一笑,他已用魔宗秘传心法,探听到方圆数里之内有两支军队从不同方向奔来,敌情已明,现在就可以逃走了,不过这刀阵非是短时间可以参透,最大可能是自己杀了大半虎赍卫士,却被雍军所困,生死两难,不过幸好他已经有了脱身的计策。

段凌霄就在刀阵之中朗声大笑,负手而立道:“奇怪啊奇怪,段某听说凤仪门乃是大雍叛贼,人人可诛,想不到如今却让我看到凤仪门的弟子在军中效力,苏青苏姑娘,你可是凤仪门主梵惠瑶的嫡传弟子,也不对啊,凤仪门主的嫡传弟子人人有名有姓,可没有听说过有姓苏的,不过姑娘这等武技,在凤仪门二代三代弟子中也可算是佼佼者了,不知道苏姑娘师承何人?”

他这一番话如雷贯耳,就是那些心肠如铁的雍军勇士也不由惊愕地望向苏青,而已经被侍女扶起的苏青本已经苍白如雪的容颜也被这番话惊得浑身一震,周身上下更是露出绝望至极的气息,就是再懵懂的人也明白段凌霄说中了苏青心中最不可告人的隐秘,就在这气息凝滞的瞬间,段凌霄已经捉住刀阵的一丝空隙,众人措手不及,飞身而出,身形化作流虹,转眼消失的无影无踪,风中传来他冰冷地声音道:“苏青,你武功来历已经泄漏,我倒要看看你如何在雍军待下去。”

场中一片静寂,无数的目光落到苏青身上,她傲然而立,彷佛寒冬雪梅一般铁骨铮铮,可是神情却是无比的凄艳悲凉,可见段凌霄所言并非是挑拨离间,她当真是凤仪门弟子。

凤仪门啊,那个从前风光显赫,如今已经令人避之不及的名字仍然深刻在所有人的心里。曾经掌控朝野多年,权倾天下,却又因为谋逆犯上而风流云散,凤仪门从前的弟子除了逃匿无踪的那些之外,剩下的多半都已经成了皇权斗争的牺牲品,有的为父母夫家不容,被迫离家远走,甚至青灯古佛聊度残生,有的得到家人庇佑,但是从此消沉下去,再也难见昔日容光,而军中更是将凤仪门的影响竭力排除,一旦和凤仪门扯上关系,就是不死也别想留在军中任职。可是,苏青,堂堂的大雍司闻曹所属北郡司北汉谍报网的总哨,三品将军,女中英杰,竟然是凤仪门弟子,传出去怎不令人瞠目结舌。

有几个见过方才苏青和段凌霄交手情景的骑兵心中忐忑不安,方才苏青剑如狂潮,华美瑰丽,果然有凤仪门剑法的影子,只不过他们从未有过这样的想法,因此没有意识到,定是段凌霄对凤仪门武功知之甚详,因此才发觉苏青的真实师承。不知不觉间,众人将苏青围了起来。

如月看着神情冷漠的苏青,突然大声道:“你们太过分了,小姐多年来为了大雍出生入死,不久前才从北汉死里逃生,今日若不是小姐拼死苦战,那段凌霄岂会落入埋伏,你们宁可信任一个敌人的言语,也不相信同生共死的同僚,这是什么道理?”说到后来,她已经是悲愤万分,抱着苏青泪如雨下。众人面面相觑,尤其是方才见到苏青血战段凌霄的雍军勇士,更是面露愧疚之色。

呼延寿咳嗽了一声,问道:“苏将军,那人所说是否实情,若是他假言构陷,请苏将军明言,我等自会替苏将军辨白。”众人知他心意,只要苏青说不是实情,那么他情愿隐瞒此事,众人心中也都这样想,不论苏青什么出身,他们只需知道这个女子和他们一样为了大雍不惜生命荣辱,那就够了。

这时远处烟尘滚滚,赶来支援的大雍骑兵终于赶到,到了近前,被诡异的气氛所震慑,他们自动停下坐骑,莫名其妙地望着众人,寒风吹过,千余人的包围之中,一个青衣女子站在那里,神情冷若冰霜,天地间一片沉默,除了风声和偶然有马匹呼着热气低声嘶鸣之外,再没有别的声音存在。

苏青挣开如月的扶持,走上前几步,走到呼延寿面前,微微一笑,那笑容犹如冬日里的一丝阳光一般灿烂,却也如同昙花一现的凄凉,她一字一句,声如金石,高声道:“段凌霄并非构陷,我苏青的恩师乃是凤仪门首座弟子闻氏讳紫烟,虽然苏青不过是恩师的记名弟子,但是师恩深重,苏青至今心中感佩,虽然迫于局势,不敢明言,但是我苏青从未忘记恩师救我性命,传我剑法的深情厚谊。不过,我苏青也从未忘记自己乃是大雍的将军身份,自认从未做过对不起朝廷袍泽的事情,今日事已泄露,终究是难以瞒过天下人的耳目,苏青一身在此,诸位如何处置,任凭尊便,不过如月虽然是我侍女,却不知道此情,我麾下众多兄弟,也无人知道我苏青的来历,还请诸位作证,替他们洗刷清白。”刚刚说完这番话,苏青只觉得头晕目眩,内伤加上心灰意冷,让她再也无力支撑,耳边传来如月的呼叫声,苏青只觉得软倒的身躯落入一个温暖的怀抱,她轻轻叹息一声,罢了,自己的命运就交给老天来决定吧,全然放弃之后的苏青陷入了最深的昏迷。

好温暖啊,苏青仿佛在做一个无休无止的美梦,好像回到了旧日那种受到保护,恣意轻狂的千金小姐生活,朦胧中好似幼年时候躺在母亲的怀抱,听着母亲低声吟唱着童谣,让自己心甘情愿陷入沉眠,不知不觉间,一滴晶莹的泪珠从眼角滑落,那再也寻不回的幸福生活,再也见不到的父母亲人。

朦朦胧胧的睁开双眼,苏青再次感觉到生命的存在,多年来在北汉日日殚精竭虑,就是睡眠中也是时刻提防着身边警讯,回到大雍之后,心中重担仍然存在,所以苏青很久没有这样酣然地睡上一觉了。她坐起身来,发现自己躺在一张温暖舒适的软榻上,罗幕低垂,空气中有着品流极高的熏香气味。苏青将被子扯落,只见自己身上穿了白色中衣,而且似乎是自己随身携带的换洗衣服,她挑开帷幕,发觉四周全是木质的板壁,地面轻轻晃动,没有窗子,但是房内空气并不污浊,这肯定是船上的舱房。目光掠过四周上下,只见房内并没有太多妆饰,但是桌椅书架一应俱全,床头放着香炉,壁上悬着书画,看起来十分清新雅致。苏青心中一惊,就是醒来发现自己身陷囹圄,她也不会这样吃惊,但是在战场之上受到这样的优待可就让她分外吃惊了。

她看见旁边一张椅子上面摆着一套青色军服和软甲,都是自己的衣服,只不过已经清洗缝补好了,她将衣服穿好,穿上战靴,在书案上面摆着自己的兵器和暗器,她也一一收好,看来自己还没有被解除军职,苏青心中略宽。整理好衣衫,苏青突然觉得腹中饥饿,也不知道自己多长时间没有吃东西了,只看自己内伤恍然若失,就知道至少有两三天的时间了。她正要推开房门,舱门从外面被拉开了,面上带着淡淡愁容的如月走了进来,一眼看到苏青,她欣喜万分的扑了过来,抱着苏青的身躯大哭起来。苏青心中一暖,也不将她推开,道:“傻丫头,我的衣服都被你哭湿了。”如月连忙松开手,一边抹着眼泪一边道:“监军大人说小姐快醒了,让我来看看,说是小姐如果醒了,请到前厅用饭。”

苏青心中一惊,神色有些苍白,有些事情终究是躲不过去,她强作笑颜道:“是么,我昏迷了几天了,怎么这么饿。”如月道:“当日小姐受伤昏迷,呼延将军将小姐带回船上,监军大人诊脉之后,说小姐内伤其实不重,只是有些过于劳累,再加上受到心灵上的打击,所以才会昏迷不醒。大人说让小姐好好休息几日,所以就在伤药中加了安眠的药物,如今已经是第四天了,这几日小姐除了服药,就是服用参汤,也难怪这样饥饿。”

苏青犹豫了一下问道:“如月,那件事情监军大人已经知道了么?”

如月偷眼看了一下苏青的脸色,道:“监军大人下了禁口令,不许将当日之事外泄,之后就将小姐留在船上养伤,其他的事情我也不知道。”

苏青心中忐忑不安,道:“领我去见江大人吧。”

走进江哲的舱房,苏青几乎立刻就被那一桌子丰盛的菜肴给吸引去了所有注意力,这个时候就是她的前途命运也胜不过食物的诱惑,她几乎是用尽了全身的力量克制住立刻拿起筷子的冲动。但是江哲的一个动作让她完全失去了控制,江哲将手指向桌面,这是一个寓意明确的动作,苏青几乎是连招呼也不大的扑到桌前,开始大快朵颐起来,直到她吃饱之后,才恢复了正常的思维。想起方才的失态,苏青面上一红,起身道:“末将失礼,请大人恕罪。”

我一直旁观苏青的举动,说起来此女不愧是名门出身,虽然狼吞虎咽,但是仍然保持着基本的仪态,只不过动作快了些,不过我能明白她的心情,饥饿加上心情的放松,会让人不克自制,若是她在敌手手中,绝不会这样放松,可以看得出来,她是真心将我当成了可以信赖倚靠的上司,所以才会这样松懈,至少,她对我并无敌意。这几日心中的犹豫突然烟消云散,我终于作出了如何处置苏青的决定。

我和站在桌边方才一直帮忙布菜,实际上是贴身保护我的小顺子交换了一个眼色,问苏青道:“苏将军,不知道可否将令师之事详细道来?”

苏青心知自己今后的生死荣辱就在此刻,丝毫不敢怠慢,道:“末将七年前和段无敌分手之后,因为心中悲愤欲绝,因此遁入深山,浑浑噩噩不知走了多久便昏迷过去,山中多有虎豹,末将其时已经存了死志,可是醒来之后却见自己处身山洞,身边是篝火野味,有人将末将救下之后安置在那里,救下末将的人正是先师闻紫烟。先师问过我身世之后,也是十分同情,见末将幼时学过武艺,就有意收留末将为徒。可是末将问过先师之后,知道凤仪门弟子需得遵从门主谕令,更不可能从军杀敌,末将死里逃生,心中已发下誓愿,一定要投入雍军,报仇雪恨,所以婉拒先师美意。先师知道苏青心意之后,十分赞赏,特意多留了十日,传授苏青剑术武技,不过先师为了避免被人知道此事,又见末将所学心法乃是道家正宗,所以并没有传授凤仪门内功给苏青,所传授的剑术也是先师自己参悟的杀招,彼此虽然有师徒之情,却没有正式名份。后来苏青练成剑术之后在中原创出名号,更成功的加入雍军,末将和先师的联络就更加隐秘,除了每年在我师徒相遇的山洞相见一次之外,就再也没有会面。先师说她受师门恩重,不论生死祸福,都要与师门不弃不离,弟子不过是学了一些粗浅剑招,她不想弟子陷身权势之争。所以凤仪门中无人知道末将和先师之事。先师猎宫之变前曾经在山洞那里留下她的剑术心得和一封遗书给末将,言道,她将从师门为大逆之事,若是事成也就罢了,若是事败,让末将不要记恨杀她之人,她是心甘情愿为师门殉葬。”说道最后,苏青已经是泪光盈盈,她起身下拜道:“大人,先师虽然做下大逆不道的举动,但是请大人念在先师实在是为愚忠愚孝所累,允许苏青前去祭拜先师。”

我听了苏青所说,只觉得心中凄然,道:“令师虽然做下错事,但是就是皇上也说她行军作战暗合兵法,性情更是刚烈无双,当日令师亲率大军追杀皇上,以少胜多,险些将皇上逼入绝境,之后令师和小顺子阵前决战,落败之后自尽身死,性情刚烈,皇上也为之哀叹。血手罗刹的确是凤仪门主最得意最出众的弟子,如今从你口中,得知昔日往事,闻女侠还是一位明辨是非的奇女子,只可惜被忠孝所困,致令身死名灭,苏将军,当日皇上也对闻女侠颇为激赏,所以令人将其秘密安葬骊山,日后你若去帝都,我会派人领你前去祭拜。”

苏青眼中闪过感激的神色,恭恭敬敬地磕了几个响头。这时小顺子突然冷冷道:“你不记恨我么?”

苏青看了小顺子一眼,淡淡道:“先师求仁得仁,又有何怨?”

小顺子看向我,默然不语,我知他已经同意我的决定,便道:“苏将军,你的事情我虽然下了禁口令,但是没有不透风的墙,所以我会上密奏禀明皇上,但是皇上宽宏大量,苏将军忠于朝廷,有功社稷,皇上不会怪罪,至今今后的安排,我也不能肯定皇上会如何处置,但是苏将军请放心,你至少可以看到北汉灭亡。”

苏青欣喜若狂,再拜道:“苏青心中唯一的愿望,就是看着北汉灭亡,能够遂此心愿,就是皇上判我重罪,苏青也是死而无怨,请大人允许苏青重回沙场,为大雍效力。”

我伸手虚扶,道:“齐王那里我会去说明,他不会反对此事,苏将军再休息一日就可以动身了,现在外面很需要苏将军负责斥候军机呢。”

苏青起身道:“多谢监军大人美意,苏青已经全部恢复了,现在就可以上阵了,不知道外面军情如何?”

我笑道:“昨日我军已经攻下冀氏,冀氏守将提前将平民撤到了安泽,他在据城死守一日之后连夜逃走,我军火烧冀氏,至今火焰仍未熄灭,齐王殿下领军直进安泽,水军也正朝安泽而去,不过前几日水军辎重受损,后续的辎重要在两日之后才能运到。”

苏青道:“安泽乃是段无敌亲自镇守,易守难攻,只怕是难以攻陷,不若末将派人前去散布流言,就说段无敌陷害石英入罪,众说纷纭,段无敌必然难以辨白,大人以为如何?”

我拊掌笑道:“正合我意,就是今日苏将军不醒,我也要传令下去这般进行了,安泽守军除了段无敌的嫡系之外,石英旧部也有许多,若是能够跳起安泽内乱,则我军可以轻而易举地攻下安泽。”

苏青谨慎地道:“大人,段无敌作战虽无赫赫之功,但是却令人无从下手,今日虽然用计离间他的军心,请大人禀告齐王殿下,不要轻视安泽守军。”

我点头道:“你说得不错,若非如此,段无敌也不会成为龙庭飞最依赖的左帮右臂,若说鬼面将军谭忌是龙庭飞的矛,铁壁将军段无敌就是龙庭飞的盾,如今矛已毁,盾已伤,我倒要看看龙庭飞如何指挥作战。”

苏青心中不期然闪过一丝哀叹,对着重如泰山的压力,段无敌会如何应对呢,我要灭掉北汉,你要守护北汉,不知道你我谁能够完成心愿,可是苏青心中知道,不论谁能够得偿夙愿,她和段无敌之间都已经是再无转圜的余地了。

第十三章    安泽败战

苏青,原北汉国人,少年时因家仇投军,果敢勇毅尤胜英杰,积功至北郡司总哨,素为属下爱重,沁州战中身世泄露,乃知为凤仪闻氏弟子,楚乡侯不以为忤,用其总领军中斥候,屡立功勋。

——《雍史·澄侯列传》

走出舱房,春日明媚的阳光让苏青不禁微微闭了闭眼睛,重见天日的喜悦让她忍不住唇边露出一丝微笑,不远处传来急促的呼吸声,苏青抬目看去,只见呼延寿站在那里手足无措,望着自己欲言又止,一个刚猛威武的大汉却是十分局促不安的模样。苏青心中一动,她久历风尘,知道呼延寿是对自己动了心,这时如月低声对她说道:“小姐,那日就是呼延将军亲手将你抱回船上的。”

苏青虽然是铁石心肠,也不由面上一红,想起那日自己最软弱之时朦胧中感觉到的温暖怀抱,原来就是此人,心中生出暖意,但是转念一想,苏青神色却是变得冷肃。虽然名义上呼延寿只是三品将军,而且实权尚不如自己,但是身为虎赍卫副统领,又被皇上派来保护楚乡侯,此人前途无量,而自己虽然军中地位颇高,但毕竟只是司闻曹所属。而且如今自己的秘密被揭破,就是皇上念着自己的功劳不予追究,但是削去军职也是很可能的,这些自己倒不在意,若是能够见到北汉覆亡,就是自己前途尽毁也没有什么关系,可是若是因为自己拖累他人就不好了,自己和此人绝无可能。

心中想到这些,苏青冷冷道:“多谢呼延将军照拂之恩,末将就要回军中去了,后会有期。”

呼延寿见苏青神色冰寒,满腔热情几乎都被冻彻,但是他想起数日前的情景,却仍是心动不已,那一日,他亲眼见到了这个女子最坚强和最脆弱的面貌,那种强烈的冲激让他至今仍然不能忘怀,但是转念一想,苏青不仅相貌清艳,而且武功高强,又是才能卓著,自己不过是一个禁军统领,如何能够配得上这样的奇女子,终于在苏青冷淡的目光下退了一步,强忍心中倾慕道:“兵危战凶,苏将军前途珍重。”苏青淡淡一笑,道:“多谢将军好意,苏青自会珍惜性命。”

由军中小船送到沁水岸边,那里正有苏青的属下焦急的等待着,见到苏青上岸,他们同时下拜道:“属下叩见将军。”苏青见他们个个神情肃穆中隐隐带着喜悦,知道这些下属对自己并未生出疏离之心,但是她却不愿流露出脆弱的情绪,只是冷冷道:“去安泽。”说罢接过他们递过的马缰,一马当先冲了出去。那些斥候秘谍相视以目,都是十分欢喜,对他们来说,苏青的身份来历并不重要,重要的是这个女子和他们一起出生入死多年,这种袍泽之情才是他们最重视的东西,更何况苏青的才能本领让他们打从心里佩服呢。

站在楼船舷窗前,我含笑看着下面发生的事情,道:“小顺子,你也和苏青叫过手,为什么没有发觉他和闻紫烟的关系?”

小顺子沉默了片刻,道:“这件事奴才早已经看出了端倪,苏青的剑术承袭闻紫烟,而闻紫烟的剑术和凤仪门众人其实有很多不同,更加辛辣无情,少了许多花哨的招式,不过奴才想闻紫烟此人刚毅果决,苏青个性和闻紫烟有许多相似,应该不会和那些凤仪门中人同流合污,因此奴才没有揭破此事。”

我笑道:“你是担心我斩草除根么?”

小顺子冷冷道:“斩草除根公子大概是不会做的,可是利用人利用个彻底,却是公子的本事,苏将军不是那种可以被利用欺瞒的人,奴才不想公子和她结下深仇,所以没有拆穿此事。”

我不由有些赧然,小顺子真是看透了我的为人,若是在此之前我知道了苏青的身份,一定会把她派到南楚去,现在我正觉得在南楚的控制有些不够严密,而且大概会欺瞒她很多事情,这是我用人的习惯,除了我的嫡系之外,其他的人我是不喜欢全盘托出的,可是如今苏青在这种情况,却让我只能在重用她和将她解职选择其一。

对我来说,苏青的忠诚没有疑问,而且她在秘谍中威望极高,对那些下面的将士来说,朝廷中的争权夺利实际上是一件比较遥远的事情,苏青和凤仪门的瓜葛并不能让他们产生不信任。当日那些知道苏青身份的将士之所以震惊,大多是担心苏青会因此遭受牵累,毕竟谋逆之罪是株连九族的,他们或者并不在意苏青的身份,可是却会在意军方上层的清洗,毕竟这会牵连很多人甚至是他们自己。

这样的情况下赦免苏青更符合大雍的利益,不过这只是我的想法,而我的能力也不过是让苏青在沁州之战期间不会被接触军职,最后的决定还是要让皇上来决定的,最终的结果不大好揣测,虽然皇上素来雍容大度,但是他毕竟是天子,天子最重视的就是皇位和社稷,当初凤仪门谋逆犯上,闻紫烟更是曾经几乎将皇上至于死地,虽然事后皇上表示出了对闻紫烟的敬重,可是最好的敌人是死去的敌人,闻紫烟若是死了,自然没有关系,闻紫烟活下来的话恐怕也会被枭首示众,所以苏青的命运还在两可之间。

我看看放在桌上的密折,其实我并不想现在就把折子递上去的,最好等到沁州之战结束之后再说,可是我不会设想军中没有夏侯沅峰明鉴司的人,而且虎赍卫也会有密折递上去,即使呼延寿明显的陷入了情网,这件事情与其瞒着不如我提早呈上去,至少凭我的面子,可以保住苏青的性命吧,这个女子巾帼更胜须眉,真是让我佩服得很,就连小顺子都有心成全,何况是我呢。

这时候呼延寿失魂落魄地走了进来,道:“齐王殿下那方面有军报传来,说是安泽守军十分凶悍,而且还动用了水军,要调水营去助阵。另外殿下请大人至中军观战。”

我轻轻皱眉,为什么北汉军会在安泽竭力抵抗,按照道理来说,沁源城高池深,易守难攻,粮道稳固,北汉军明显军力不如我军,与其这样消耗军力,不如趁势诱敌深入,在沁源固守,消耗我军实力,然后再用精锐骑兵和我军决战,这样才是更合理的做法。不过想不通的事情我暂且不去想,反正齐王他们都是沙场宿将,这些疑点他们不会看不出来,也不会不防备的。望着云山蔼蔼,这北汉可真是一块硬骨头啊,希望我的计策能够顺利成功,当然若是用不上就更好了。

冷眼望着城下蜂拥而至的雍军,段无敌神色肃然,不时的调动人马将城池守得稳如泰山,安泽城内守城的准备十分充分,兵力也颇为充足,段无敌守得十分严密,可是这仍然不能减轻他心中的疲惫,已经四天了,雍军兵力众多,轮流攻城,节奏严密而流畅,攻城日夜不停,他再擅长守城,也几乎是难以支撑。城上城下箭雨不断,投石车、弓弩机几乎没有停止过轰鸣,滚木擂石沸油铅水,将安泽城墙摧残的体无全肤,有些部分已经露出墙砖后面的黏土,这样下去,安泽城破只是时间的问题。段无敌疲倦的揉揉额角,上次中毒之后他的体力一直不够好,很容易疲劳。段无敌强行撑着身子向城下望去,雍军中军树着青罗伞盖,身穿金色战甲,外罩红锦战袍的雍军主帅齐王李显和一个青衣文士坐在椅子上正在谈笑甚欢,这种景象对北汉军的打击更胜过无休无止的攻城。

段无敌冷眼看了片刻,挥动令旗,沁州水军从安泽西面的水门冲出,绕到南面雍军的主攻方向,一阵机弩弓弦响动,正在攻城的雍军早就有了准备纷纷执盾躲避箭雨,可是这样一来攻城的力度自然弱了,安泽再次击退了雍军的这一轮猛攻,而雍军的水军战船出现的时候,根本不可能阻拦北汉水军的后退。在昨日泽州水营初至的时候,段无敌曾经用投石机击毁了一艘雍军战船,自此以后,雍军战船再也不敢接近安泽的水门了。

眼看着这一批攻城的雍军退下之后,另外一队雍军缓缓逼上,段无敌叹了口气,让守城的军士开始换防,他们已经连续作战半日,应该让他们下去休息一下了,抬头看看北方,段无敌心中想:“为什么大将军的援军还没有到来,大将军说只要我守住五日,就没有我的事情了,可是今日已经是第四日了。”正在他心中忐忑的时候,一个近卫匆匆跑来道:“将军,大将军信使到了,请将军依计行事。”说罢递上一封书信。段无敌连忙打开,只看了片刻,就心中狂喜,脸上露出不可掩饰的笑容,往往城下的雍军,段无敌眼中露出冰寒的杀机。

而此时,我在城下也是心中不安,事情反常即为妖,段无敌不是蠢人,龙庭飞更不是白痴,安泽这样的情形,根本阻不住我军锋芒,若是在沁源死守,就是一两个月我军都不可能攻下城池,在安泽,虽然段无敌防守的严密,可是安泽城墙的高度厚度都不足以坚守待援的,为什么他们不退呢,从安泽到沁源,中间山岭起伏,丘陵不断,若是他们逐步退守,凭借那些城寨,足可以拖住我们一月时光,事实上,我从来没有打算过用什么狡诈手段攻打安泽,甚至沁源,在这里,只能是我军靠着军力强攻才行。望望那似乎摇摇欲坠却屹立不倒的安泽,心中的疑虑再也难以掩饰,不由问道:“殿下,苏青可有军报传来?”

齐王皱眉道:“还没有,不过昨日又到了第二批辎重,另外还带来了几架神臂弩,明日攻城应该可以用上了。”

我轻轻点头,目光望向远方,夕阳西下,天色昏黄,夜里的攻城我就不看了,希望明日可见见到安泽城破,为什么苏青没有动作呢,我心中不由想到了一些不大好的可能。

苏青一身灰黄色的衣裤在山野间潜行,她重新回到战场之后,很快就发现了情况有些异常,虽然北汉军将雍军阻到安泽,而其后又安排了秘谍截杀雍军穿越安泽防线的斥候,可是苏青仍然凭着一身武功和对安泽地理的熟悉,混入了这一带,幸好这里的流民络绎不绝,仍然没有彻底撤到沁源。这种情况的诡异,让苏青暂时放弃了对安泽军心的离间,毕竟若是没有意外,安泽是守不住的,而她的职责就是让这个意外尽量不要发生。

她施展蛇行身法掠上那座防守严密的小山坡,仗着衣衫和泥土枯草颜色相近,总算是寻到了一个合适的地点探看军情,在小山之后,正是贯穿沁州的河流——沁水,苏青的眼睛突然露出惊骇的光芒,她看到了想要寻找的东西。那是一座水坝,下面没有什么异常,但是上面却有一些可以开阖的出水口,沁水穿过这些孔洞急速的下流,而在水坝旁边和沁水连通的,是一个数里方圆的大湖。苏青脑中闪过无数的思绪,在记忆中,这个湖泊并非原来所有,见湖泊四周都是火烧的痕迹,定是北汉军在冬日用火化去寒冰,然后挖掘而成的大湖,利用春日沁水涨水的时候蓄了一池的水,而水坝的设计十分巧妙,只要蓄满湖水,则沁水仍然可以顺流而下,这样下游就看不出来沁水的水位变化,毕竟这一湖水比起整个沁水来并不明显。可是只需将水坝上面的出水口封住一日,然后毁去水坝,借助地势和水力,足以形成能够湮灭千军万马的洪流,而在下面二十里,就是安泽,那里正是雍军和北汉交锋之处,一旦洪水流去,必然是雍军尽没,而有城墙保护的北汉军则不会有惨重的损失。

忍住心中惊骇,苏青缓缓的向下退去,十分缓慢,她不想在最后关头露出形迹,也是她运气不错,在数日前,这里还是重兵保护的所在,如今战事繁忙,这里又即将启用,所以没有太多的北汉谍探,他们大部分都到前面去探查军情,或者清除流民中的探子去了。这也是萧桐一时失误,在他意中,大雍秘谍中的佼佼者苏青应该正被拘禁甚至处死,其他的秘谍是很难有这个能力透过重重封锁到达此地的。终于安全回到了藏身处,苏青估计了一下时间,苦笑着施展浑身解数,向安泽奔去,这也是没有办法,这一带有不少北汉的鹰隼,信鸽是根本派不上用场的,别的斥候更是很难稳妥的传信回去,所以她只有拼命赶路了。虽然只有短短二十里的路程,可是为了突破重重封锁,苏青不敢奢望很快回到安泽,只是默默祝祷,希望可以在北汉军发动之前赶回安泽。

安泽城下,齐王怒气冲冲的望着安泽西面的水门,今日北汉水军屡屡出击,真是让他看了碍眼,眼看天将正午,居然没有一点破城的迹象,忍不住发了狠心,齐王终于下令先后两批到达安泽的水军主动出击,一定要让北汉的水军困守城中,不过出乎意料的是,北汉水军终究是新军,居然在大雍水军分进合击的战术下被截了归路,不得已往上游退去了,达到目的大雍水军也懒得去追击,索性堵住安泽的西水门,用船上的投石机和弩机向安泽西面的城墙发动攻击,一块块巨石向城墙砸去,一阵阵弩箭射向城头,碎石零落中将安泽守军的气焰立刻打了下去。见到这种情形,众军大喜,都是戮力攻城,一架架云梯井阑靠上城墙,开始有青黑色的身影出现在城墙之上,李显大喜,指着城头道:“若非安泽地势险要,后倚山崖,西临沁水,我们哪里需要这么多时间攻取。”

我微微一笑,心中反而更加忐忑不安,太容易了,段无敌是什么人,我见过关于他的情报,守住安泽十几日还是没有问题的,昨日齐王说想今日破城,我只是听听罢了,可是今日段无敌虽然锋芒四射,却全非旧日风范,守城就守城,频频出击实在有些不象话,而北汉水军的失误虽然合情合理,但是却未免有些让人心疑。我盯着安泽城想着心事,若是北汉军果然有阴谋,那么应该是如何着手得呢,北汉军力不如我军,我军攻城并无疏漏,敌军就是用什么手段,也不可能让我军伤筋动骨,除非是水火无情。想到这一点我心中突然一凛,我先前怎未想到这一点,或许是本就没有抱着取胜的心思吧。急急令人拿来安泽方圆五十里的地图,我仔细研究起来,目光落到了沁水之上,这一带地势陡急,若是在上游蓄水确实可以水淹雍军,虽然按照时间推算,这个工程应该很浩大,不可能在十天半月之内完成,而之前沁州仍在冰冻期,想要这样做也很困难,但是我军将要进攻北汉,世人皆知,未必北汉不能做到这些看似不可能的事情啊。虽然心里有了一些端倪,可是我不由皱紧了眉头,无凭无据的,我怎么让齐王撤军,这无法说服众将啊,就是想要说服齐王也不那么容易。正在我犹豫的时候,远处一骑绝尘而来,马上那人手持齐王军中的风行旗,那是斥候使用的信物,任何人都不敢阻挡他们的去路,在他前面的雍军原想阻挡,但是看到那人手中风行旗,都回避开去,那人飞马到了中军,下马急拜道:“殿下、监军大人,北汉军在二十里外飞云峡筑坝蓄水,恐怕今日就是放水之期。”

我心中虽然已经有了觉察,仍然不由惊咦了一声,仔细瞧去,那人正是苏青,只是如今形容憔悴,衣衫破碎,手臂上还有用衣襟包裹的伤口,可见是历经千辛万苦才到了这里。李显闻言也是大惊,突地站起问道:“可是实情?”我不等苏青答话,站起肃然道:“殿下,北汉军情况有异,臣也以为当是如此。”

李显为人决断,看了一眼苏青,又看了我手上的地图,断然道:“现在不知他们何时放水,我军不可贸然急退,宣松,你指挥攻城将士徐徐退下,我率亲卫断后,你们撤出沁水两岸,不可懈怠,令水军顺流而下,越快越好,随云,你不要跟着水军了,让虎赍卫护着你先到附近暂避。”

这时候我也顾不上客套了,小顺子扶着我上了战马,我低声道:“殿下不可轻身涉险,后面还有大局需要殿下掌控,这一次我们提前知道敌军诡计,就是损失重些,也不会翻不过身来。”

李显眼中闪过寒光,道:“你放心,我不是不知轻重的人,不会随便丧命的,你先走吧,等到军队开始撤退之后,我会及时离开的,现在走太走了,我担心乱了军心,苏将军,你知道此处地理,就保护江大人离去,等到水退之后,也好迅速和中军会合。”

苏青连忙点头,也翻身上了战马,我们一行百多人迅速离开了战场,我们原本就不是泽州大营的人,虽然走得突兀,也没有引起手下将士过多的注意,离去之时,我听到身后号角喧鸣,想来是齐王整军准备撤退了,心里祝祷齐王和三军将士可以安全退走,毕竟若是惨败在这里,那么我下面的计划就不可能实现了。

等到我离开安泽城将近二十里之后,耳边突然传来轰隆隆如同滚雷一般的巨响,我心中大叫“苦也”,想必是北汉军放水了,这么短时间不知道齐王来不来的及安全退走。但是我也顾不上那边的事情了,只能放马狂奔,谁知道那水能漫多远,我还是跑得越远越好。心里一边诅咒着龙庭飞和段无敌,一边诅咒自己为什么没有想到敌军会用水攻,我快马加鞭地赶着路,幸好这些日子在军中我还是练了练骑术,否则现在连逃命都困难了。

此刻的安泽城下,已经成了人间地狱,大水顺沁水河道直冲而下,原本还是天际的一道白线,没过片刻就已经露出了狰狞的真面目,那混浊的河水浪高数丈,彷佛受惊的猛兽,放肆奔流,天地间雷声滚滚,直可以震裂听者的耳膜,但是抬望眼却是晴空万里,洪水之威,乃至于此。洪水从安泽城西侧擦肩而过,转瞬将安泽城包围在其中,西侧的水门虽然早已关闭,但是河水顺着水门冲入城中,汹涌的狂潮在城中肆虐,段无敌早已经将城中军民般至高处,又已经安排好泻水的孔洞,却是在内城先开了门,留下外面的浮土没有凿穿,内里只用砖石堵塞,洪水一过,城墙立刻开了大洞,洪水穿城而过。即使这样,站在城楼上,眼看着城内洪水滔滔,段无敌仍然是心中忐忑不安,他可不想一城军民都替雍军陪葬,而且雍军不知如何得到消息,竟然提前撤退,若非是他用烽火传信,只怕那洪水就只淹了一个安泽城了。

安泽城内守军有城墙保护还可苟延残喘,城外的雍军可就损失惨重了。虽然因为及时得到消息,齐王下令让骑兵一马带双人离去,可是雍军在安泽城下有骑兵四万,步兵五万,虽然这几日多有损伤,骑兵又是竭力携带,仍有将近五千人的雍军只能步行撤退。双腿跑得再快也快不过洪水,他们大多是不识水性的旱鸭子,几乎尽皆损失在洪水当中。而大雍水军损失更加惨重,洪水波及下,大半战船辎重船毁于洪流之下,幸好上面的将士多半都会水,凭着过人的水性再加上抱着水中飘浮的船板,倒有大半人逃得性命,只是可惜了泽州水营的战船和雍军的所有辎重,几乎尽毁在沁水当中。

第十四章    胜固欣然

隆盛元年戊寅,三月十七日,雍军攻安泽,段无敌坚守不退,三月二十一日,龙庭飞决沁水淹雍军,雍军败绩,北汉密谍大索乡里三日。

——《资治通鉴·雍纪三》

站在残破的安泽城头,漠然地望着城下的水乡泽国,龙庭飞神色之间没有一丝欣喜,这一场水攻,虽然淹掉雍军无数,可是安泽城也是摇摇欲坠,杀人一万,自损三千,若非是万不得已,自己怎会作出这种决定。想到这场大水将会淹没沁水沿岸千万亩良田,多少北汉平民将要流离失所,龙庭飞心中就是隐隐作痛。这时,他身后传来段无敌和其他将领拜见的声音,龙庭飞不愿让心中烦恼感染到众将,让脸上的表情缓和了许多,甚至勉强露出一丝笑容,他朗声道:“这一次我军水攻取胜,但是雍军主力仍在,接下来还需苦战,诸君不可懈怠。”

段无敌在此间已是龙庭飞之下地位最高的将领,便首先开口道:“将军不必担忧,雍军虽然保全了大部分实力,但是水军几乎全毁,安泽和冀氏之间道路已经成了沼泽,车马难行,自此之后雍军粮道几近断绝,若是雍军主帅有自知之明,或者会退去也不一定,将军此计,败敌于顷刻之间,末将等尽皆拜服。”

众将也都连声称赞龙庭飞用兵如神,胜利的光芒让他们个个神采飞扬,几乎忘却前几日雍军大军攻城时候的压力和折磨。龙庭飞心中有些感叹,这些将领多半都是有勇无谋之辈,难以独当一面,可是他也只能强作笑容,接受众人的祝贺,毕竟他不能让众将泄气啊。他温和地道:“连日作战,辛苦非常,军务繁忙,诸将还是下去休息吧,今夜本大将军为诸位庆功。”众将都是轰然应诺,高高兴兴的退了下去。只留下龙、段两人在城楼之密谈,两人近卫都知机地站到远处,寒冷的春风吹过,偶尔可以听到片言只字,却是过耳即逝。

虽然心中有些凄凉,但是取得这样的战绩,龙庭飞心中其实也是十分高兴的,他感慨地道:“这一策我策划了许久,石英之事后,我令萧桐大肆捕杀雍军密谍细作,将安泽以北控制的十分严密,雍军密谍只会当我是因为石英之事而大发雷霆,浑然不知我是借机行事,而且秋四公子追杀百里,将大雍密谍重要领袖人物杀死大半,这数月正是大雍探察我军军情能力最弱的时候,趁着冰冻之期筑坝,雪化之时汇成一湖,万事俱备,终于水淹雍军。更令龙某欣喜若狂的是,在国师安排下,王上密练水军前来助阵,安泽五日苦战,将雍军水陆主力羁绊在安泽城下,这才能够一举功成。只可惜雍军水军强大,而我军水军避入支流也需要甚长时间,再加上关山阻隔,放水时机难以掌握。我原本是准备等到雍军较为疲惫的未时末再放水的,可惜不知如何终究被雍军发现端倪,幸好无敌及时举火通知,要不然只怕功亏一篑了。”

段无敌听到这番话,神色有些不安,他在安泽城头可以俯瞰雍军,苏青奔入军中报告军情的时候也落入他眼中,虽然距离颇远,但是段无敌眼力不凡,他对苏青又是敬佩又是歉疚,所以对她的身形记得清清楚楚,虽然距离遥远,但是还是给他隐隐约约认了出来。但是这种事情可就不便说出来了,毕竟自己和苏青曾有旧情,虽然如今已经恩断情绝,但是苏青在大雍立功越大,自己就不免越发尴尬。

他虽然不想多嘴,龙庭飞却想起了苏青,回头笑道:“无敌,你那位青黛姑娘的确是女中豪杰,若是她还在北汉主持大局,我们也未必有这么容易瞒住蓄水的事情,不过她大概也不可能在大雍待下去了。”

段无敌心中一惊,道:“将军何出此言,末将和苏青已经再无瓜葛。而且苏青在大雍颇得重用,为何将军说她在大雍不能容身呢?”

龙庭飞心中暗笑,心道这段无敌果然对那青黛不能忘情,不过他并没有因此恼怒,段无敌对北汉的忠心他是知道的,不计毁誉,舍弃私情,还有什么可以怀疑的呢?他微笑道:“前些日子段大公子到军中见我,曾说及苏青之事。雍军犯境之初,他正在冀氏之南,见我水军在沁水上拦截雍军水营,又恰逢楚乡侯江哲也在水军之中,若是我水军全力攻击,或能擒杀江哲,则我军可以士气大振。段大公子见此情势,为了让水军有更多时间作战,便去刺杀了若干前来援救的雍军骑兵的将领,可是让雍军大乱一场,可惜仍然功败垂成,水军还损失了一位宗室出身的副统领。”说到这里龙庭飞神色有些黯然,但是他转而又笑道:“段大公子见行踪已露,索性决定刺杀一位雍军重要人物,那江哲身边虎赍如云,又有邪影李顺这种高手保护,他就盯住了苏青苏将军。当时苏青可能是被江哲召见,我水军退后,江哲应该是知道了雍军将领遇刺的事情,特遣苏青去向齐王报信,这是段大公子从苏青的行踪上面判断的,因此他决定将苏青当作刺杀目标,苏青在北汉多年,熟知军情地理,若能杀之,价值最大。可惜那江哲果然料事如神,设下埋伏,大公子追杀苏青之时落入重围,不过大公子武功高强,还是被他脱身而走,也算是扫了江哲的面子。而且大公子还发现一件有趣的事情,那苏青的武功剑术竟然得自凤仪门真传,想来秋四公子应该是对凤仪门的剑术心法并不熟悉,所以才没有发现这件事情。若是我早知道此事,或许可以用计策反苏青,但是当时大公子为了脱身,索性将此事当众揭穿,哈,那可真是热闹的很,虽然大公子没有留下看后面的发展,但是我军在流民中的斥候有一两个侥幸逃生,他们亲见苏青昏迷不醒,被送到了江哲船上。哼,那江哲乃是雍帝心腹,与凤仪门必是誓不两立,凤仪门覆灭之后,凡是和凤仪门有关联的都被株连,虽然雍军中较为宽松,可是那苏青品貌才情都是十分出色,必定是凤仪门核心人物之一。凤仪门如今在大雍是最大的忌讳,那苏青恐怕是前程尽毁,即使念在她往日功劳,只怕也会被免去军职。其实我对苏姑娘也颇为痛惜,为了私仇家恨,她已对北汉不忠,如今身份被揭穿,她对大雍也有不忠,进退两难之际,或者会有回头的可能吧。若是无敌有机会重见此女,不妨出言招揽,若是她能够重归北汉,只要她能助我铲除大雍在北汉的谍报网,我可以免去她从前之罪。”

段无敌犹豫了一下道:“苏姑娘心志坚毅,不是随便改变心意的人,臣觉得她重归我国的可能不大,不过若是末将没有看错,昨日她曾快马入雍军大营,应该并未被解职。”想了许久,他终究不想因为自己的隐瞒有害大事,所以直言不讳。

龙庭飞眉头轻皱,片刻才开颜道:“我不信江哲会不追究此事,此人虽然外表温文儒雅,可是杀伐决断,更在常人之上,我听凌端说此人心狠手辣,御下严谨,就是那个邪影李顺,一旦他声色俱厉,也是噤若寒蝉,此人决不会轻易放过苏青,莫非是齐王的意思?齐王李显曾娶凤仪门女子为妃,倒是有可能余情未断,而且苏青可以说是他的直属手下,李显为人又是嚣张跋扈,不拘小节,即使屡遭挫折,仍然是性情不改,他若肆意妄为,江哲也难以阻止的。不过我可不信那雍帝李贽会将此事轻轻放过,凤仪门几乎夺了他的皇位,取了他的性命,他纵是量大如海也未必能够容得下苏青。此事事后必有后患,我会先派人去查一下,如果江哲果然因此事和齐王生出嫌隙,那么我们从中推波助澜,再将此事传入大雍朝廷,这可是最好的攻讦借口,有人不会错过这个机会的,到时候李显不死何待?不过这事也不忙,现在对敌才是大事,若是能够将李显留在沁州,这些计策不用也罢,齐王毕竟是难得的名将,死在战场上才是不负英名。”

段无敌虽然听得认真,可是并未对龙庭飞这番话生出多少共鸣,那些勾心斗角的事情他并不十分擅长,他是将领,非是阴谋家,若非是此事涉及苏青,他根本就没有兴趣仔细聆听。

龙庭飞看出了他的心思,不禁暗暗苦笑,目光扫过身后,那种空空荡荡的感觉让他心中一痛,曾几何时,他羽翼日渐凋零。想当初,谭忌、苏定峦和石英还在生的时候,他不论在何处都觉得心中十分踏实。谭忌虽然不喜言语,可是很多狠毒的计策都是自己和他一起研究出来的,而且此人虽然落落寡欢,嗜杀凶残,可是有他在自己身后,龙庭飞总是觉得心中十分安定。而苏定峦之死最令他扼腕,这样一柄无坚不摧的利刃就因为擅自参与行刺雍王的计划而丧命在长安,虽然如今鹿氏三兄弟可以替代苏定峦,可是龙庭飞心中仍有不足,鹿氏兄弟虽然勇猛不下苏定峦,可是却少了苏定峦那种气魄,苏定峦一人就可以让全军上下舍生忘死,强大的战力几乎是无坚不摧,而鹿氏三兄弟却似乎总是做不到那样的效果。

还有石英,这个是龙庭飞心中最深的痛,石英几乎是他一手提拔起来的将领,亲信更在其他三人之上,可是几乎是一夜之间,石英成了叛国投敌的逆贼,即使是现在,龙庭飞仍有不真实的感觉。当初他下令将石英囚禁,没有立刻斩首,也是心中隐隐希望能有转圜的余地,可是出乎意料的,石英居然自尽身死。龙庭飞初时心中松了口气,毕竟若是让他手刃这个素来爱重的亲信,还有些不舍,可是随着苏青身份的泄漏,龙庭飞心中不知怎地,有些怀疑自己是否误会了石英。可是证据确凿,而石英所做之事也确实令他头痛万分,所以他还是将这个心思深藏了起来。

想到身边大将连续身死,龙庭飞忍不住怒火攻心,目光落到城下,看着那残破的景象,他想起了一个可以出了心中恶气的法子,他恶狠狠道:“现在雍军无处安身,必定四散奔逃,而无敌既然说江哲最先离去,他和雍军大营必定会暂时分离,我已传令萧桐,派出我去密谍大索乡里,一旦发现江哲踪影,一定要千方百计将其刺杀,段大公子也准备亲自出手,若是能够杀了江哲,雍军必然士气大损,而且齐王也无法向雍帝李贽交代,至于苏青的事情毕竟是小节,若是江哲侥幸逃生再利用不迟,最好的结果,就是先将江哲狙杀。”

段无敌对此事却是并不重视,对他来说,刺杀敌人首脑虽然可以动摇敌人军心,可是若是不能最大限度的杀伤敌人,那么就不算是胜利,而且江哲身边有亲卫保护,刺杀未必能够成功,他当然不会扫龙庭飞的兴,只是岔开话道:“将军,雍军虽然落败,但是骑兵主力仍在,水退之后必然来攻,齐王李显生性猖狂,恐怕不会轻易退兵,不知道将军下一步准备如何作战。”

龙庭飞精神一振,道:“我正要和你商量,雍军虽败,但是没有伤筋动骨,若是你我在安泽和沁源之前重重布防,虽然雍军可能会付出惨重代价才能攻破这重重防线,可是大雍拒敌千里,带甲百万,就是补充个十万八万兵力也是易如反掌,我军却是难以为继。而且若是我们两国两败俱伤,可能会让外人拣了便宜,虽然你我都希望大雍四面受敌,可是这时机也是很重要的。更何况安泽已经残破不堪,若是守安泽不免太艰难,我的意思是在这些日子不妨多多挑衅,让齐王急于进攻,而我们退到沁源。到时候雍军想要进攻,就必须穿越眼前这几十里泥沼和将近四十里的山路,如今他们水军损失惨重,辎重粮草运送十分艰难,而我们固守沁源,不仅背靠坚城,而且粮草补给也方便得很,此消彼长,我军便占了地利人和,以逸待劳,便可徐徐作战,就是不能取胜,也可以拖住雍军,大雍还有内忧外患,只需拖上一段时日,雍军就会陷入绝境,我们则可以从容消减雍军实力,何乐而不为呢?”

段无敌点头道:“大将军此计使得,在沁源决战,一来可以拖长敌军的补给线,令敌军不耐久战,二来沁源深沟高垒,又有沁州城作为后盾,我军可以说已经立于不败之地。末将请命立刻将安泽军民撤到沁源,两地之间山路艰险,沁水两岸又成了水乡泽国,若是不速退,被雍军缠上,我们的损失就太大了。”

龙庭飞点点头,道:“无敌所说极是,不过我军密谍还是要多留一段时间,希望能够趁机搜杀一些雍军落单的将领,段大公子也会留下,可惜秋四公子被滞留东海,否则有他们联手,只要发现那江哲的行踪,就一定可以手到擒来。”

段无敌眉头深锁道:“末将对此事颇为不解,四公子前去东海只是希望东海保持中立,东海只是要求四公子留在东海,就可以严守中立,这未免有些太古怪了,何况他们还支援了我军一批粮草辎重。东海归附大雍恐怕只是时间的问题,雍军监军江哲在东海数年,东海小侯爷又是他的弟子,末将总觉得其中有些不妥,现在粮草已经到手,不如传言四公子,让四公子早日脱身归来如何?”

龙庭飞苦笑摇头道:“国师弟子毕竟是江湖人,首重信义,四公子尤其恪守信诺,就是国师令他提前归来,只怕他也会拒绝的,而且四公子性情冷傲,不习惯军旅生活,就是在这里也未必派上什么用场。何况大公子这次全力相助我等,四公子就是不在也没有什么关系,反倒是他若擅自离开东海,只怕东海大怒之下会和我国翻脸,不说别的,只要他们派上一支水军襄助雍军,我们就吃不消了。毕竟你也清楚,只需过几日,沁水水位就可恢复正常,到时候若是雍军有水军运送粮草,我们的如意算盘可就打不响了。”

两人正在商量军机,突然城楼下传来一阵喧嚣声,两人都是眉头一皱,段无敌叱道:“什么人在下面喧哗?”

只听见城楼下传来纷乱的脚步声,几个龙庭飞的亲卫扶着一个衣衫破碎形容狼狈的军士走了上来,那个军士嘶声道:“大将军,从十四日起,一支雍军突破太行白陉,猛攻壶关重地,刘将军亲自上阵,苦守关隘,可是攻城的雍军乃是雍军泽州大营副将荆迟,他带着骑兵三万,还有镇州守军四万相助,攻城日夜不停,刘将军已经令人向国主禀报军情,但是唯恐壶关不保,特遣小人前来向大将军禀报,求大将军速派援军。”

段无敌听得那人禀报,心中一凛,镇州和沁州隔着太行山,原本只要守稳了关口,就可以安枕无忧,而且这些年来,雍军每次攻打北汉都是从泽州入境,镇州从无动静,想不到这一次齐王竟然将手下的副将派去攻打壶关,壶关和沁源不到二百里距离,若是荆迟在十日之内攻破壶关,正可以和雍军主力前后夹攻北汉军,而国内兵力主要集中在代州、晋阳和沁州三处,晋阳军守卫都城,代州军担负着抵御蛮人的重任,都不能轻易调动,其余各处关隘也都不能轻易调兵,除非是从沁州派兵支援。想到这里,他拱手道:“大将军,末将请命去支援壶关。”

龙庭飞却是神色不变,冷冷道:“听斥候回报,说是不见荆迟旗号,我就想到可能他会走镇州,果然被我料中,壶关守将刘万利也是宗室将领,可惜只是中庸之才,若是他有无敌你一半的本事,我就不用担心壶关了。不过你不能去援救,雍军中也有擅守之人,擅守之人也必擅长攻城,若无无敌你在沁源,我军必败无疑。”

段无敌急道:“可是若是壶关被破,我国西南关隘守将都非是奇才,恐怕会被荆迟势如破竹,到时候我军和雍军主力陷入苦战,岂不是被他们前后夹攻,恐怕也不免落败的,何况荆迟还可以直指晋阳,若是都城危急,我们岂不是罪无可绾。”

龙庭飞微微一笑,道:“无敌你是过于忧虑了,只要传令各地据城而守,那荆迟就是攻破了壶关,难道还有精力一处处攻打么,他一定会直奔沁源。若是他发了疯去攻打晋阳,我倒要庆幸呢,晋阳城易守难攻,荆迟那几万人就是攻打上一两个月也没有可能攻破晋阳,不过据我估计,沁源才是荆迟的目标,毕竟消灭我军才是解决问题的关键。若是不知道荆迟之事,我军还有失败的可能,既然现在已经知道,我自然有法子将雍军泽州大营毁在沁州。”

段无敌皱紧了眉头,也想不出如何能够稳稳取胜,毕竟敌军有二十多万,而北汉军只有十余万,其中还有许多新军,对这如狼似虎的雍军,如何可以对抗雍军的前后夹攻呢?

龙庭飞却是神色自若,道:“我会向王上禀报,虽然这个计策有些冒险,可是若是我军战败,那就是国破家亡的结局,我想国主会赞同我的决定的。”说到这里,他这些日子有些憔悴的容颜突然焕发出耀眼的光彩,那双浅碧色的眼眸深邃粲然,伟岸的身形如同山峰一样峻挺,在这最艰难的时刻,他终于冲破了这些日子笼罩在他身上的重重阴云,恢复了他的骄傲和自信。

这时,那些闻知此事的将领正走上城楼,想探听龙庭飞的决定,见到龙庭飞那充满自信和勇气的身形,多日来心中的惴惴不安都如同阴云一般被阳光冲散,龙庭飞面上露出欣然的笑容,指着远处道:“诸位,雍军强大无比,诸位可有信心随我大破雍军?”

众将不由同时高声道:“末将等誓死效忠王上,跟随大将军血战到底,定要大破雍军,保家卫国。”

龙庭飞哈哈大笑,笑声爽朗而洪亮,令得城楼下忙着收拾残局的北汉军军士也都不由露出了自信的笑容。

见到龙庭飞如此神采飞扬,段无敌心中也终于安定下来,看到破出阴云的春阳,段无敌心道:“这是否我军大破雍军的征兆呢?”

龙庭飞这里自信满满,晋阳宫中却是一片愁云惨雾。

兰台之上,魔宗京无极正和后主刘佑隔着棋坪对弈,刘佑神色凝重,每下一子都要仔细想过,京无极则是随手应之,看去似乎并不认真,可是两人之间陷入窘局的似乎却是刘佑,只见他眉头紧锁,眉间满是愁苦之色,不似在下棋倒像是受刑一般。良久,刘佑推坪而起道:“孤已经输了,国师棋道高明,孤自愧不如。”

京无极微微一笑,道:“王上的心思不在棋中,却在沁州前线之上,焉能不败。”

刘佑苦笑道:“国师毕竟是世外之人,莫非竟对前方战势毫不关心么?”

京无极站起身来,走到玉栏旁边,伸手指向远处的崇德殿道:“金殿之上,文武重臣都在等国主前去议事,他们都对战势无比关心,为何王上不去和他们商议呢?”

刘佑走到京无极身前,看向崇德殿,那是他平日召见臣子议事之处,可是那殿中之人却无益大事,他叹了一口气道:“如今除了庭飞和碧儿,还有谁能派上用场,国师,若是你肯亲自出手,必定可以将大雍主帅刺于军中,到时候何愁他们不退兵呢,如今大雍已经没有凤仪门主,还有何人可以阻拦国师出手呢?”

京无极微微皱眉,道:“国主何不相信龙庭飞可以力挽狂澜呢,如今雍军主力被阻于沁源之南,雍军新近大败,若是无极出手,只怕会激怒大雍朝野。虽然凤仪门主已经身死,可是慈真大师仍然健在,他是佛门弟子,所以没有随军前来,若是他带领各派弟子到了沁州,我魔宗弟子毕竟不如他们人多势众,只怕反而会吃亏。何况凌霄、萧桐、玉飞都在为国效力,这已经足够了,何需本座亲自出手。”

刘佑眼中闪过焦急的神色道:“虽然如此,可是雍军偏师已经攻打壶关多日,一旦壶关被破,那么那支偏师就可以从背后攻击沁州,到时候沁州两面受敌,庭飞纵有再高的军略又能如何。代州军不能轻动,晋阳城中虽有十万军队,却非是骑兵,一旦壶关被攻破,就有社稷颠覆的危险,还请国师垂怜,亲自出手一次。”

京无极正要劝慰他,这时有内侍在台下高声道:“大将军有密奏至。”

刘佑闻之大喜,他知道壶关守将定会向龙庭飞求援,现在龙庭飞上了密折,定然是有了决断了,连忙道:“快将密折呈上。”接过龙庭飞亲书的密折,打开一看,刘佑脸色变化万千,良久,才将折子递给京无极。京无极阅后微微一笑,道:“庭飞果然有了计策,王上还要担心么?”

刘佑忧虑地道:“这也太险了,若是不如庭飞所料可怎么办呢?”

京无极冷冷道:“家国将亡,还顾虑那么多做什么,若是大将军战败,北汉亡无日矣,如果王上还有疑虑,不如问问碧公主,若是碧公主也支持此事,王上应该不会反对了吧?”

刘佑沉思片刻,道:“果然得去问问碧儿,不过纵是碧儿不同意,说不得孤也要勉强为之了,若是沁州战败,我国再无兵力可以对抗大雍,碧儿应该可以谅解此事吧?”

京无极默默点头,负手向远方望去,御花园中花木已经逢春,如烟如雾的烟柳当中,金壁辉煌的宫室越发壮美,若是沁州一战不能取胜,只怕是无边美景顿成断瓦残垣,而魔宗在北汉的根基也将被连根挖起,自己多年来的心血将毁于一旦。可是无论如何,自己绝不能亲自出手刺杀雍军大将。如今已经不是当年了,那时诸侯争霸,胜负未可预料,自己尚可以肆意妄为,如今大雍一统天下之势已经是难以阻挡,若是自己亲自出手,恐怕日后就会造成魔宗的覆灭,这是绝对不可以的。只要自己不出手,那么碍于自己的存在,就是北汉亡国,大雍朝廷也不敢过分逼迫魔宗,甚至还有可能保住北汉王室的一脉香烟。

轻轻叹了口气,走到兰台一角,那里放着一个装满了画轴的青瓷花瓶,他伸手抽出一卷画轴,轻轻展开,上面绘着一个白衣女子正在明月下舞剑,京无极自言自语道:“惠瑶啊惠瑶,若非你不肯退隐,不肯服老,又怎会有身死骊山猎宫的结局呢,却不知那迫死你的少年是一个怎样的人,若是凌霄将他狙杀,也算是替你报了仇吧!”

第十五章    败亦可喜

隆盛元年戊寅,三月十四日,大将荆迟率骑兵三万,镇州军四万越太行白陉,攻壶关甚急,守将刘万利急报晋阳、沁州,三月二十五日,壶关城破,荆迟率军奔沁源,势如破竹。

——《资治通鉴·雍纪三》

彤云蔽日,天空阴沉沉的,仿佛随时都可能滴下雨来,官道上百余骑士闷头狂奔,马蹄声如同奔雷,马上的骑士个个面沉似水,黑色的战袍上满是征尘,看上去就带着些狼狈,被这些骑士护在中间的一匹青骥神骏非常,上面却是坐着两个人,正是江哲和李顺。一口气跑出六七十里,马不停蹄,江哲骑术不精,为了加快行程,还是由小顺子和他同乘一骑,这匹青骥乃是千里挑一的神驹,虽然身上见汗,却是精神百倍。官道两边草深林密,小顺子一边小心地扶持着江哲,一边留心着四周的动静,在这种兵败逃难的时候,又是在敌国境内,他必须十分小心,这时右侧林中传来轻微的马蹄声和草木被穿拂而过的声音,小顺子抬起右手,百余骑战马同时停住,静悄无声,不愧是大雍最精锐的军队之一。不多时,苏青骑着一匹黑马穿林而出,她迎上众人,扬声道:“大人,今夜的宿处已经寻到,穿过树林十里处有个无名村庄,那里离官道很远,十分僻静,我在外面转了一圈,几乎没有看见人迹炊烟,里面的村民应该早就逃避兵灾去了,就是还有人家未走,凭我们的实力也可以一网打尽,不过为了避免打草惊蛇,我没有进去查探。”

我疲倦地道:“我军一到安泽就开始攻城,还没有进行清野,不过冀氏那边的消息过来,这一带的平民不是逃了就是躲进安泽了,这庄子没有人也不奇怪,不过大家还是要小心一些,一会儿将这庄子围住,里面若还有人,将他们关在一起。大家小心一些,我军初败,想要重整旗鼓至少也需数日时间,北汉军若是有余力一定会大索四乡,捕杀我军落单的将士,这几日最是危险,这藏身之地一定要小心防备,不能走漏风声。”

呼延寿提马上前道:“大人放心,苏将军前面带路,我们先围住庄子,然后再逐户搜索,不会让一人漏网。”我微微点头,这种事情他们绝对不会失手的,一个小小的村庄,别说可能没有人,就是有百八十人,对他们来说也是轻而易举就可以扫平的。呼延寿留了几个侍卫跟随保护我和小顺子,他们先赶过去了,我想着不会有什么问题,就让小顺子放慢了速度缓缓前行。林中道路崎岖,不能疾驰,小道两边枯草漫漫,几乎将道路都给掩盖住了。可见这是一个平日很少有人往来的村子,若非是为了逃避雍军,恐怕那里的村人还不会逃走呢,这也好,若是人太多,杀人灭口也未免太麻烦了,更何况杀害无辜,有伤天和。

走了半晌,眼前的道路突然宽阔起来,而且也平整了许多,露出光溜溜的泥土表面,这里应该是村人常来常往的地方了,我向前一看,果然已经到了密林的边缘,小顺子催马加鞭,策马走出林子。我只觉得眼前一亮,豁然开朗。密林之后是一片低洼的谷地,在谷地中心,有一个数亩方圆的小湖泊,湖水清澈见底,湖面上冒着蒸蒸热气,我能够感觉到这里比别处温暖许多,想必这个湖泊乃是温泉汇聚的。

湖边分散着三十多户人家,错落有致,屋舍之间阡陌交错,隐隐带着清逸之气。想来若是承平时期,必是鸡犬相闻,老死不相往来的世外桃源。只不过如今成了杀伐战场。四十多个虎赍卫将整个村庄四面围住,而在其中一座农舍前面,却是传来呼喝争斗的声音,我心中一惊,虎赍卫个个都是一流高手,怎么会在这个小村庄遇上对手,我的好奇心膨胀起来,连忙示意小顺子快些过去,小顺子大概也担心出了纰漏,策马片刻就到了那座农庄之前。

这座农舍占地半亩方圆,正房有三间,两侧各有三间厢房,房舍都是青石搭建,十分宽敞明亮,农舍四周篱笆稀疏,院内有一个小菜园,种着一些青菜,还有两垄菊花,可见这里的主人并非寻常农夫。虽然天气还很寒冷,但是可能是因为温泉湖水使得这里气温较高的缘故,青菜已经破土,菊花也已经有了绿叶。此刻院中两个虎赍卫士正联手和一个青年农夫交手,呼延寿负手站在院门处,十几个虎赍卫士将这座农舍围得严严实实。见到我停在院门之外,呼延寿连忙急趋走来,禀报道:“大人,庄子里面都已经清过了,这里的村人想必是早就离开了,只有这家有人住,还是一个高手。”

我点点头,仔细看去,只见那个农夫大概二十八九岁的年纪,相貌俊朗,鼻直口方,身材英伟,一见就知非是常人,他死死守在正房门前,手中一柄单刀,将两个虎赍卫士挡住,仍然是游刃有余,不过他面色有些苍白,显然已经看出形势危急。

小顺子看到这种情形,皱眉道:“怎么不让人从窗子进去,前后夹攻,快些将人制住,公子还要休息呢。”

呼延寿赧然道:“属下见这座农舍在整个村子里面最是格局开阔,景物也优雅,原本想请公子在这里休息的,所以不想破坏屋舍。”

我心中一动,这座农舍果然清幽,也亏得呼延寿想的周到,这时呼延寿大概是见小顺子脸色不好,连忙道:“大人稍待,属下这就亲自出手。”说罢便退了几步,转身拔刀向正房门口走去,他气度沉凝,那个农夫眼中闪过绝望的光芒,手上的招式也有些散乱。呼延寿果然是虎赍卫中数一数二的高手,他的刀法刚猛凶狠,将那农夫迫得捉襟见肘,不过数招,那个农夫已经是气喘吁吁,大概是久战力疲,那农夫一个失足跌倒在地,呼延寿一刀斩向那农夫,这样一个高手留着,只怕会有麻烦,所以他毫不手软,决定斩草除根。

这时屋内有人高声喝道:“刀下留人!”呼延寿原本也料到屋内可能有人,否则那个农夫不会死守正屋,不过那人声音沉稳威严,让呼延寿心中一动,手中的横刀骤然停住,刀锋停在那农夫脖颈上,那农夫已经是闭上了眼睛,但是觉察到刀锋停住了,虽然寒气袭人,但是似乎没有破皮见血,他睁开眼睛,怔怔地望着呼延寿。

这时房门被推开了,一个身穿灰衫的老者站在门前,他神色憔悴,几乎是骨瘦如柴,手里拄着一根拐杖,看上去大概五六十岁的年纪,但是此人虽然一副病入膏肓的模样,神情气度却是佼佼不群,颇有人上人的气度。

呼延寿冷冷望着那个老人,厉声道:“你是什么人?快将来历说来,如果稍有隐瞒,休怪本人刀下无情。”

那个老人漠然一笑,目光却落到院门外被几个侍卫护在当中的那骑青骥上,一个身穿青色大氅的文士骑在马上,神情带着淡淡的疲倦,两鬓微霜,发色灰白,看上去似乎是年纪很大,但是看他容颜,却是清秀儒雅,面白如玉,这种矛盾的形象让他周身上下透露出一种莫名的气质,还有一个青衣少年容颜似雪,神情如冰,牵着马缰侍立一旁,但是他气度清峻中带着森然,虽然神情恭敬,却不似一个普通的下人。

老人叹了一口气,道:“诸位想必是大雍贵人,何必为难我们这些乡野草民,小徒抗拒诸位将军,实在是因为诸位来势汹汹,还请大人恕罪。”

那青年农夫高声道:“你们要杀就杀我一人好了,伯父年迈,又病卧在床多年,你们总不能滥杀无辜吧?”

呼延寿将手中横刀向前一送,那青年觉得咽喉刺痛,呼延寿冷冷道:“不问你不许多言。”那青年眼中怒火熊熊,却只能闭口不言。呼延寿再次看向那老者,森然道:“姓名,来历?我不想再问一次。”

那个老者轻轻摇头,道:“老夫纪玄,将军想必没有听过。”

原本神情疲惫的我听到纪玄的名字,神情一振,朗声道:“纪玄,纪子城,北汉立国之前,曾是太原令刘胜帐前长史,熟读经史,精通易经算学,素为刘胜信重,刘胜立国之后,纪玄不满刘胜悖逆,遂挂冠而去,令刘胜扼腕不已,想必就是先生了。”说罢,我翻身下马,缓步走向农舍,向那老者深深一礼,道:“末学江哲,拜见纪老先生,晚生久闻老先生学问高深,高风亮节,今日一见,幸何如之。”

说完这番话,那倒在地上的青年农夫眼中闪过一丝诧异的神色,只不过被人用刀抵住咽喉,不敢出声说话罢了。而纪玄目中闪过幽深的光芒,道:“原来是南楚状元,大雍驸马,楚乡侯江哲,老夫虽然蛰居乡里,也听说侯爷声名,想不到侯爷竟会屈驾到此。”

我听他语气便觉得不善,这个纪玄只看他昔日因为不满刘胜立国,就挂冠而去,可见是一个恪守忠义之道的人,我虽有才名,却是先事南楚,后事大雍,又娶了长乐公主为妻,这个纪玄一定将我当成贰臣贼子看待,我看若非是为了那个青年的性命,这老先生还会把我冷嘲热讽一顿呢。

所以我很知趣地没有表示仰慕之情,转移话题道:“那位兄台称老先生是伯父,莫非是您的侄儿么?”

纪玄神色怆然道:“此子赵梁,字文山,乃是老夫挚友代州赵颐之子,老友夫妻死于战乱,这孩子自幼就在老夫身边长大,我和他父亲兄弟相称,这孩子便叫我伯父,实际上却是情同父子,前些日子闻听雍军攻沁州,沿途残杀平民,乡人恐惧不安,都已经北上避难,只有老夫身染重病,经受不起路途颠簸,只得留下待死,这孩子孝顺得很,坚持不肯自行逃去,还望侯爷看在小侄鲁莽无知和他的一片孝心份上,饶恕了他的性命吧。”

我看了那个纪梁一眼,心中倒是很敬佩,这人的确是个孝子,为了伯父不顾生死,见他方才一直挡着门口,想必是担心我们伤害他的伯父,而且他既然跟在纪玄身边,必定也是熟读经史,见他武功也是不错,倒是一个文武双全的人才,他们虽然是北汉人,可是纪玄对北汉王室应该没有什么忠心,耳濡目染,赵梁也应该不至于排斥大雍,这个赵梁倒是可以延揽的人才。想到这里,我便露出笑容道:“原来赵少兄是至孝之人,呼延将军,你退下吧,属下多有得罪,还请少兄见谅。”

呼延寿收刀退下,那赵梁站起身来,连忙走过去扶着纪玄,刚刚从鬼门关拣了一条性命,赵梁面色也是十分苍白,他恭恭敬敬地道:“侯爷大量,赵梁感激不尽,还请侯爷手下留情,不要伤害伯父性命。”

我正色道:“纪老先生乃是儒林大家,哲虽是后学末流,焉敢有加害之心,只不过我军新败,需要在此修整一段时间,还请赵少兄留在村中不要擅自行动,待江某离去之时,必定还两位自由。”

赵梁面上掠过喜色,我见他喜形于色,知他乃是城府不深之人,心中越发喜爱,又道:“本来村中空宅不少,可是我麾下多是武人,唯恐他们不知礼仪惊动纪老先生,再说我也喜爱此处清雅,不知道纪老先生可容江某在此寄居么?”

纪玄重重一哼,若非是担忧赵梁的性命,他怎会容许这样一个不忠不义之人留在自己家中,但是情势比人强,他也是无可奈何,冷冷说道:“侯爷有命,老夫焉敢不从,蜗居简陋,倒是让侯爷见笑了,梁儿,将东西收拾一下,我们到别处去住。”

我几乎要笑出声来,这个老先生可是真有趣,这是在嘲讽我鹊巢鸠占么,不过我心中倒是挺高兴,至少这个纪玄还懂得退让,我最是不喜欢遇见那种油盐不进的狠人,偏偏这种人都有不错的才能和响亮的声名,若是迫得我杀了纪玄,传扬出去岂不是难听得很。不过芸芸众生,毕竟是中庸者多,心志坚毅,外物不可撼动而又智慧高超的人却是难觅,虽然偏偏却让我遇上了好几个这样的人。

一个是小顺子,别看他少年时候似乎心性油滑,可是现在他可露出真面目了,他的心志可是无人可以动摇的,幸好老天保佑,他是一心一意守护我,将我当成知己骨肉。他绝对不容许任何人损害我的安全,包括我自己在内,否则那一次秋玉飞行刺于我,小顺子也不会因我自蹈险地而大怒了,让我吃了好几天的排头。

另一个就是陆灿,这个我昔日的弟子,他是下定了决心效忠南楚的,前几日有江南的谍报到来,陆灿竟因为尚维钧代替南楚国主赵陇所下的旨意而放弃了趁机攻击大雍的计划,这在我来说是不可想象的事情,可是他就这么做了,而且还心甘情愿被尚维钧软禁在建业,看来他是绝对不会做出违背臣节的事情了。虽然很高兴因为这个缘故而减轻了大雍南面的压力,可是我是绝对不会指望陆灿将来会投降大雍的了。

其实还有一个人就是齐王李显,他也是一个油盐不进的家伙,之所以现在对我言听计从,纯粹是因为他看我顺了眼,只看他当初一贯的作为,就知道此人若是拿定了主意,就绝对没有人可以改变,说起来我倒要庆幸万分,这人从来没有打算过自己去夺取大雍皇位,否则李贽就是取胜也是惨胜,以李显的心性,可以将大雍朝廷翻个底朝天的。狠狠的在心中诅咒了李显几句,原本已经心中有了警兆,可是无缘无故地就让李显退兵的话,他是不会听的,所以我就没有多嘴,结果害我落到这种地步。

敛去心中杂念,我叫住这就要进去收拾行礼的赵梁,歉意地道:“赵少兄且慢,老先生不要这样说,哲乃是末学晚辈,怎敢将老先生逐出住处,哲见两侧还有厢房,就借一间客房暂住,不知尊意如何?”

纪玄脸色缓和下来,我这样容让,他也难以恶言相向,便和颜悦色地道:“如此多谢侯爷海量,东厢客房梁儿常常清扫,就请侯爷委屈一下。”

我笑着答应,骑了半天的马,我几乎有些支撑不住了,揉揉额角,我勉强道:“晚生体弱,不堪风尘,就先告退了,请老先生也回房休息吧,明日哲还要向老先生请教呢?”

纪玄见我面色苍白,额头已经有了汗珠,其实他也沉疴在身,刚才说了这许久话也是仗着精神支撑,便拱手告退,回房去休息了。我则被小顺子扶入厢房,那间厢房果然雅洁,也不需整理,我除去大氅,倒在床上,几乎是一沾枕头就进入梦乡了。

一觉醒来只觉神清气爽,睁开眼睛,看见小顺子坐在窗前,手里拿着一本书卷正看得津津有味,我心中觉得很有成就感,能够让一个昔日看见书本就要睡觉的小子今日自觉地寻书去看,我还是一个很出色的先生啊。虽然我只是轻轻一动,小顺子却已经发觉我醒了,放下书卷,他拿了一杯热茶走过来,我灌下这杯热茶,觉得精神好了许多,腹中却饥饿起来。小顺子淡淡道:“厨房里面热着饭菜呢,我让他们端来。”

我起身披上外衣,懒洋洋地道:“也好。”小顺子出去吩咐一声,不多时,苏青端着一个木托盘走了进来,上面放着几样清淡的小菜。我一看是苏青,不由有些尴尬,埋怨道:“小顺子,怎么让苏将军做这样的事情,岂不是太失礼了。”

苏青倒是落落大方地道:“末将睡醒之后见到呼延将军一直不肯休息,问过之后才知道他一定要亲自值夜,末将想这几日不知何时会有苦战,不愿他这样辛劳,所以自请替他值夜,大人只将末将当成呼延将军好了,不用介意这些许小事。”

我这才松懈下来,想来苏青常年在军旅当中,恐怕也早不将自己当成女子了,拿起竹筷正要用饭,外面传来侍卫的轻叱声,我不由停住了筷子,苏青闻声走了出去,不多时回来道:“大人,是那位赵梁赵公子,他或许是得知大人醒了,想连夜求见。”

我心中觉得奇怪,道:“让他进来吧。”反正这个赵梁也翻不出什么大浪,我也就没有放在心上,谁让我身边有小顺子这个高手呢,若是那个刺杀苏青等人的段凌霄或者秋玉飞出现,我才会觉得危险吧。

不多时,赵梁走了进来,他一走进房门就跪在地上,连连顿首,我心中奇怪,想要上前搀扶,不过小顺子一道冷眼过来,我立刻自觉地缩回手,问道:“赵少兄为何如此?还请起来说话。”

赵梁没有起身,只是抬起头道:“草民有不情之请,恳求侯爷救我伯父性命。”

我心念一转已经明白了他的意思,纪玄沉疴缠身,我虽然没有替他诊脉,也知道病得很重,而我是医圣弟子的消息也颇有人知,这赵梁是求医来了。不过我几乎很少替人看病,只顾着照看自己的身体就够麻烦了。这不过是小事一件,我慨然应允道:“哲在此承蒙少兄款待,这件事情自然没有问题,等到明日哲会亲自替纪老先生诊脉,不过生死有命,医治不死病,哲也只能尽力而为,如果有不忍言之事,还请少兄见谅。”

赵梁喜道:“草民叩谢侯爷恩德,只要侯爷肯出手医治,不论如何,草民也只有感激涕零的道理,怎会怨怪侯爷。”

我看看桌上的饭菜,笑道:“如今已是深夜,少兄想必是久候了,恐怕也是腹中饥饿,我一人用餐也是无聊,少兄不妨和我一起用吧。”

赵梁焉敢和我同桌,不过我主意已定,一会儿,另外一副碗筷拿来了,赵梁只是象征着吃上少许,我则是一边用餐一边和他说话。果然不出我所料,这个赵梁果然是熟读经史,对于时事也是了如指掌,完全没有蛰居乡里的闭塞。我和他谈得开心,连小顺子将残羹剩菜撤了下去,换上了香茗我都没有留心,不过倒是习惯成自然地拿起来喝了一口,然后说道:“赵少兄如此人才,却屈居乡野,待我大雍入主沁州之后,不知道少兄可愿为大雍效力。”

赵梁神色数变,终于问道:“草民有一事不明,还请侯爷赐教。”

我品着香茗,嗯,山野清茶,果然是清新无比,口中应道:“文山有何事要问?”

赵梁肃容道:“如今雍军败于安泽,为何侯爷全无一丝烦恼,竟似胜券在握呢?莫非是雍军此败也在侯爷计算之中。”

我手一抖,茶水几乎溢了出去,用崭新的眼光看向赵梁,原本还以为他只是一个人才,现在看来这人是奇才,只从我片言只字,就看出了这许多东西,我放下茶盏,正色道:“此事涉及军机,文山可是真想知道么?”

赵梁心一抖,但是他十分明白自己的处境,既然江哲出言招揽,自己恐怕是没有脱身的可能了,若是不问清楚,雍军真的惨败而归,那么大雍一统天下就很有可能成了镜花水月,若是那样,自己岂不是平白担上了背国污名。所以赵梁坚定的点头道:“草民很想知道其中原因。”

我心道,这可不是我设下圈套,而是你自己上钩的,便坦然笑道:“虽然有些事情还不能说给你听,不过此败我并未放在心上,北汉军水淹安泽,那是两败俱伤的打法,可见北汉军已经后力难继了,我军虽然战败,可是因为撤退及时,主力并未受损,我想接下来北汉军最大的可能就是撤到沁源,诱使我军深入敌境,到时候我军粮道补给艰难,北汉军就可以从容对敌了。可是我军自始至终就没有抱着轻易取胜的心思,这场惨败只会让我军士气更加高涨,而且粮道虽然受阻,但是我军泽州水营还有几十艘战船,只要征用民船,就可以维系粮道,只要稳扎稳打,沁源并非难以攻下。更何况我军偏师应该已经在攻打壶关,只要壶关一破,二十万大军围攻沁源,城破只是迟早的事情。”

赵梁听了心中一沉,既然雍军监军如此深信必胜,那么雍军士气必然高涨,不论沁源能否被攻破,这一战都会让北汉损失惨重,虽然江哲没有说什么奇策,可是只需要堂堂正正汇集了足够的兵力,再有齐王李显这样的名将指挥,果然不需要用什么计策了。他虽是北汉人,可是既未出仕,受纪玄影响,也没有忠于刘氏的意思,所以投降大雍对他来说并非什么难以接受的事情。不过想到逃难的乡亲,赵梁又问道:“请问侯爷,大雍既然有一统天下的志向,为什么这次攻打沁州,却是沿途烧杀,驱民众北上,这等情势,实在令草民费解。”

我心道,清野之事事关军机,可不能告诉你,便只是轻描淡写地道:“沁州军民和大雍连年作战,几乎家家都有子弟死在战场之上,我军不希望留下后患骚扰粮道,所以才驱民北上,其实除了威慑之外,我军并没有大肆残杀平民,等到战平之后,我军自会出榜安民,如今却只能委屈他们了。”

赵梁心中仍有不解,但是他知道自己知道的已经足够,便起身下拜道:“若是伯父同意,赵梁情愿投靠大雍,只是赵梁乃是北汉国人,还请侯爷宽宥,允许赵梁不参与大雍和北汉之战。”

我连忙将他搀起道:“此事我可以作主,必不让少兄为难。”我心里盘算,将来让他安抚地方最好不过,当然不能让他在北汉军民眼中成了叛国罪人。

第二日我替纪玄诊治,幸好纪玄的病还可以治,只是如今药物不全,我便先用针灸和手头一些药物先替纪玄固本培元,等到回到军中就可以着手医治了。至于赵梁投效我的事情,纪玄只是叹了口气就不再过问,其实他也明白,若是我离去之时不杀他们灭口,只怕日后北汉军也会将他们当成叛国贼子杀了,赵梁就是不投降也没有别的路好走。我几乎想大笑出声,有了纪玄在手,将来北汉士子就会比较容易接受大雍的统治,我得到这两个人,对于皇上来说,恐怕比起攻破一座北汉的城池的功劳都要大得多呢。

接下来几天我见这里隐蔽,索性就留下不走了,反正一动不如一静,只需等上几天,就可以和李显会合,我也就不想出去冒险了。而且这里还有温泉,温泉可是可以令人延年益寿的。每天吃着粗茶淡饭,闲来泡泡温泉,手里拿上一卷古书,和纪老先生辩辩经义,真是神仙一般的日子啊。

第十六章    自投罗网

隆盛元年戊寅,三月二十四日,齐王李显重整大军至安泽,北汉军退守沁源。

——《资治通鉴·雍纪三》

舒舒服服的泡在温泉里面,我正眯着眼睛享受着难得的悠闲时光,突然从岸边传来苍凉的歌声道:“鸷鸟之不群兮,自前世而固然;何方圜之能周兮,夫孰异道而相安;屈心而抑志兮,忍尤而攘诟;伏清白以死直兮,固前圣之所厚。”我惊得几乎在水里一个踉跄栽倒,这个纪玄,真是太过分了,前两天和他辩经义的时候被他驳得体无全肤也就罢了,毕竟他是经学大家,我是甘拜下风。可是这老先生这两天脾气见长,没事就在那里吟诗颂赋,这也就罢了,文士雅好,无可厚非,可是他不能老在那里吟咏屈子的辞赋啊,什么“长太息以掩涕兮,哀民生之多艰”,什么“亦余心之所善兮,虽九死其尤未悔”,摆明了是讽刺我背楚投雍一事。好吧,我忍,等到和大军会合之后,我就不用和他待在一起,将来将他送到皇上身边,我一定小心避开他。恶狠狠地瞪了老头一眼,我再也没有沐浴的兴趣了,对岸边的小顺子说道:“扶我起来,我要更衣了。”

小顺子在我的熏陶下也是颇通诗文,对纪玄的明嘲暗讽也是心知肚明,不过我都没有法子,他也只能在旁边看着了,毕竟这位老先生不是穷凶极恶的敌人,只是一个好面子的老头,有赵梁在这里,这老头怎也不会作出太过分的事情,所以我小小受点委屈,小顺子也只是看笑话罢了。

看到小顺子暗自偷笑,我也只能心中郁闷,上岸之后穿上小顺子递过来的衣衫,我一边用方巾擦拭发上的水珠,一边道:“今日已经是二十三日,齐王应该已经重整旗鼓了,苏青前去探查军情,我想这两日应该可以和大军会合,到时候让齐王派人将他们师徒送到泽州去,眼不见心不烦,你觉得怎么样?”

小顺子眼光一闪,看了看那在不远处散步的纪玄和在纪玄身边神色尴尬的赵梁,冷笑道:“公子是自寻麻烦,纪老先生脾气执拗,若非是碍着赵梁也在我们手上,只怕他就没有这么客气了,这样的人将来若是得到皇上信重,这老先生再这样口无遮拦,只怕损及公子声名,若是照我的意思,将他们杀了就是,何必这么费心呢?”

我心中一抖,偷眼看去,见那一老一少应该听不到小顺子的声音,才低声道:“这怎么行,若是杀了他们,只怕我在北汉士子心目中的名声就要臭不可闻了,只要能够让他们为我大雍所用,我受点委屈也没有什么,再说这个纪玄秉承的是‘达则兼济天下,穷则独善其身’的信念,当初他不满刘胜立国,既没有上书直谏,也没有尸位素餐,而是弃官归隐,这就可知他非是愚忠之人。现在他嘲讽于我,即是宣泄心中不满,也是试探我的为人,如果我计较此事,岂不让他看轻了大雍君臣,所以万万计较不得。”

小顺子默默点头,没有继续劝我杀人,其实小顺子也未必不明白其中道理,不过他视我如父兄,不愿见人欺辱于我罢了。我心中暗暗苦笑,还有一件事我没有多说,在那些愚忠愚孝的士子心中,我的声名只怕已经是臭不可闻了,就是再加上一个纪玄又有什么要紧。

远远看见齐王的大旗,苏青心中终于松了一口气,策马上前对营门守军道:“末将苏青,奉监军大人之命,前来谒见齐王殿下。”

那个守将认得苏青,一听说是江哲派来,立刻眉开眼笑,这几日齐王忙着整军,虽然没有大发雷霆,可是总是阴沉着一个脸,让人见了就心惊胆战,而齐王殿下尊重监军大人已经人尽皆知,只要监军大人无恙,齐王必定欣喜,他们的日子也会好过得多。那守将一边派人去帅帐禀报,一边派副将引领苏青进去。

苏青走在营中,用目观瞧,虽然雍军新败,可是齐王所立的大营法度森严,营中毫无沮丧之气,齐王果然是当时名将之流,苏青心中称赞,面上却是平静无波,这还是她在身份泄漏之后第一次正式谒见齐王,她心中仍有不安,虽然先齐王妃乃是凤仪门弟子,可是齐王和凤仪门却是并不和睦,这个她是心知肚明的,齐王虽然因为监军大人的缘故并未对她另眼看待,可是苏青心中仍然惴惴不安。

走入大帐,苏青原本忐忑不安的心思终于平静下来,看到帐内正负手而立微笑着看着自己的齐王,苏青不知怎么,心中一宽,上前拜倒道:“末将叩见王爷,监军大人安然无恙,这是大人命末将带来的书信。”

李显看着苏青,面上虽然平静含笑,心中却是波涛汹涌,他的心腹侍卫曾经劝他将苏青置闲,甚至拘禁起来,免得再让朝廷对他生疑,可是李显却是想也不想地就拒绝了,他李显什么时候需要用别人的生死荣辱来洗刷自己的清白了。苏青的存在让他回忆起了许多往事,少年时候的秦铮,聪慧美丽,让他第一次全心投入,还有闻紫烟,那个冷漠如霜,却是凤仪门中他唯一尊重的人,李显本心就不想让苏青受到不公正的待遇。可是李显也清楚,自己的处境其实并不好多少,自己从前的所作所为,足够让李贽不需任何借口就可以将自己下狱问罪,真得要袒护苏青,对他的损害绝对不小。

幸好,泽州大营除了自己之外,还有江哲的存在,初时,李显知道苏青之事后,是有些担心的,江哲对凤仪门似乎是切齿痛恨,苏青即是闻紫烟弟子,就等于是凤仪门嫡系传人,江哲会不会放过苏青,李显并没有把握,而出乎他意料之外,又是他意料之中的是,江哲保住了苏青,这让李显对江哲更加尊重,也更加信赖。当然,对身为宗室亲王的李显来说,如果江哲的决定被朝廷接受,这将是一个明显的信息,即是朝廷将不再追究和凤仪门有关联之人的罪责,这将令许多人心安,虽然不知李贽会如何决定,但是李显能够感觉到其中的意义,他也相信李贽会做出明智的决定。

将心中所思隐藏起来,李显接过苏青手中的书信,说是书信,实际上却是一个龙眼大的白色蜡丸,李显随手从帅案上拿起一张绵纸,将蜡丸用绵纸包住,然后拿起放在书案上的一柄裁纸小刀熟练的在蜡丸表面一划,蜡丸被剖开之后,里面渗出残绿色的液体,很快就渗透了绵纸,李显从中取出一个小了一圈的蜜色蜡丸,用绵纸拭去上面的绿色液体,才隔着绵纸捏碎蜜色蜡丸,从中取出一张薄如蝉翼的丝绢,这期间李显的动作十分小心谨慎,绝不让那绿色液体沾在手上。苏青看得出神,眼中满是疑惑,不由问道:“殿下,这是怎么回事?”

李显头也不抬地道:“这是楚乡侯设计的,蜡丸内外两层,中间蓄满毒液,若是不知情之人直接用手捏破,不禁会被毒液所侵,还会浸毁里面的信纸,楚乡侯为人谨慎,想必这封信十分紧要,他担心中途被人夺去信件吧。”

苏青心中一凛,楚乡侯果然厉害,让自己送这封信恐怕也有试探之意,如果自己有心窥探机密,那么定然是中毒身亡,不过苏青心中倒没有不满,自己师承闻紫烟,还能被江哲付与重任,这种信赖已经是难得可贵,苏青只会敬佩江哲的手段,却不会生出怨怼。

李显看着薄绢上面密密麻麻的字迹,一会儿眉头紧锁,一会儿若有所思,半晌才轻轻摇头,叹了一口气,将薄绢放到了帅案上,这次战败,其实李显并没有放在心上,他从军作战以来,也不知败过几次,比这更加惨烈的败局他也收拾过,所以落败之后,他也就是忙着整编士卒准备再战,想不到江哲比他想得更深更远,明明是一次战败,他却想到了利用败局的计策,这书信上面所写真让李显看了心中陡寒,能够让这样的人忠心相事,怪不得二哥能够夺得皇位,李显此刻当真是心服口服了。

他看看神色冷然等待自己传令的苏青,笑道:“苏将军,你休息一日,明天去见楚乡侯,引他回中军,告诉他,他所托之事,我一定照办就是。”

苏青心中茫然,但是她从军多年,自然知道是什么是奉命行事,便凛然应诺。一夜无事,当然苏青并不知道当夜齐王八百里加急递上了一封奏折。

第二天苏青孤身上路,按照她的想法,其实最好带上千余骑兵,再去接江哲,不过齐王说北汉军密谍已经退走,此地已经尽在雍军掌握当中,所以就不用这么麻烦了,而且江哲身边也有虎赍卫保护,这样兴师动众,只怕江哲也不情愿。苏青自然不会有什么异议,她是和小顺子试过招的,可以肯定小顺子武功应该和段凌霄在伯仲之间,北汉就是有刺客留下,难道还会高过段凌霄么,所以苏青也并不担心,不过为了稳妥起见,苏青一路上还是小心翼翼,兜了几个圈子之后才回到江哲藏身之处。

和在外面戒备的虎赍卫士打了招呼,她走进江哲的居处,看见在庭院中摆着一张方桌,两张木椅,江哲正和纪玄在那里下棋,虽然药物不足,可是纪玄的身体还是渐渐好转。他生性喜欢下棋,往常病体沉重的时候还拉着赵梁和他对弈,如今更是忍不住了。尤其是江哲棋艺不过平常,经常是纪玄让他四个子还能够将江哲杀得一败涂地,既然不能以武力相抗,纪玄就更加喜欢在棋盘下打击这个他看不顺眼的后生晚辈了,偏偏江哲还不好意思推辞,只得苦着脸望着棋盘。

投子认输之后,我看看纪玄那张得意洋洋地老脸,不甘心地嘟囔了几句,但是他一个冷眼过来,我立刻陪笑着开始收拾棋子,不过说句实话,我心中并不气恼。虽然这老头子脾气古怪,常常给我难堪,可是初时的气恼之后,现在我反而喜欢上了这种感觉,良久以来,我身边之人不是对我敬如神明,就是对我畏如蛇蝎,虽有几个亲近之人,爱我重我,却唯独没有这样一个将我当成平常人的朋友。这老头虽然总是摆着脸,我却觉得他可亲可近,而且他虽然看我不顺眼,却没有什么强烈的敌意,要不让小顺子也不会容许他待在我身边,这个老头倒是一个很好的忘年之交,所以我也就甘心情愿得被他欺负了。

苏青走进院落的时候,看见的就是这么一副景象,她不由心中暗笑,上前禀报道:“大人,末将已经见过元帅,王爷说请大人速速归营,并说一切都按照大人的计策去办。”

我微微一笑,道:“纪老先生,请令侄助你收拾一下行装,我们吃过午饭就要上路了。”

纪玄手一抖,正在捡棋子的手一抖,一枚棋子落在棋盘上,一声轻响,他神色变得悻然,道:“老夫遵命。”

我知他心中不快,不过这个时候也不能和他多说什么,给在一旁侍立伺候的赵梁一个眼色,他上前将纪玄扶了下去,我笑道:“小顺子,去整理一下行装,记得一定要把纪玄老先生那卷孤本带上,老先生可是答应借给我看几日的。对了,去告诉呼延寿,准备离开这里。”

小顺子微微一笑,挥手召来两个虎赍卫士,让他们护在我身边,他的身形刚刚消失在厢房之后,我站起身来,道:“走吧,去湖边散散步,这真是一个好地方,可惜以后没有机会来了。”

一个虎赍卫士朗声道:“大人,李爷不在,还是小心一些的好。”

我不耐烦地道:“这里又没有敌人,担心什么,难道你们还保护不了我么,苏将军,你一路辛苦,先下去休息吧。”

不知怎么,苏青心中总是有种惴惴不安的感觉,她下意识地拒绝道:“大人,还是让末将随侍的好。”此言一出,她清晰地看到江哲的眼中寒芒一闪,露出一种颇有趣味的神采。

我看看苏青,心中不免怀疑她是不是猜到了什么,不过有她在身边并不妥当,我还是拒绝道:“不必了,苏将军先下去休息吧。”苏青见我话语中用了命令的口气,只得领命退去。

我走到温泉湖水旁边,看着清澈见底,如同一块明净的碧玉的湖水,在这穷乡僻壤,有着一湖温泉汇聚的碧水,造就了这桃源胜地,真是让人心惊这造化之起,我在兵败之后可以到此地避难,这大概是上天给我的恩赐吧,越想越是喜爱这个住了数日的地方,俯身下去伸手轻轻拨动那温热的湖水,碧波涟漪,将我的身影搅得粉碎,不由低吟道:“碧泉涌出半湖温,欲洗人间万古尘。”刚吟出两句,却听身后有人拊掌道:“好诗兴,闻听楚乡侯诗才冠绝天下,今日一见果然名不虚传,如此良辰美景,江侯爷就是死在此处也应是再无遗憾了吧!”

我微微一笑,心道:“你终于来了。”知道那人应该不会立刻动手,我站起转过身去,只见原本跟在我身后的两个虎赍卫士都是僵立不动,而在他们身后,一个身穿虎赍卫士衣甲,但是形貌却十分陌生的威武男子负手而立,神情气度佼佼不群,气魄更是有笑傲苍穹的威势,我朗声笑道:“原来是段凌霄段大公子亲至,哲未曾远迎,真是有失礼数。”这时远处身影闪动,那些发觉了我遇到危险的卫士正急急赶了过来,而一道青色身影最是快捷,转瞬之间就到了二十丈外,正要向前之时,段凌霄厉声道:“若有敢过此线者,我当立刻杀了楚乡侯。”言未罢而回身一拂,一道劲气透体而出,在十五丈之外的地面上划了一道横线,小顺子停在线外,双目透出冰寒的杀机,却终是没有越过此线。这时,呼延寿和苏青也已经赶到,落在横线之后,都是面色焦急,神色慌乱。

我却是心平气和,微笑道:“久闻魔宗首座弟子气度不凡,前次大公子杀我大雍将士多人,哲至今铭刻在心,今日大公子想必是为了取江哲性命而来的吧。”

段凌霄此刻却并不着急,他早就知道无论如何,只要自己杀了江哲,就必定惊动众人,所以也不着急脱身,凭他的武功,只要不硬拼,不陷入战阵,外面又有自己几个同门师弟接应,想要逃走并不困难,而且他距离江哲只有一丈距离,而武功最高的邪影距离自己却有十五丈,这样的距离,就是师尊亲自出手,也别想拦住自己杀了这个文弱书生,所以他也就不急着下手,毕竟对这个江哲,他也是有几分好奇的。

我见他暂不出手,却也心中略宽,若是一会儿打起来,可就没有机会这样心平气和地说话了,看看那两个僵立在一旁的两个侍卫,见两人都是怒容满面,大汗淋漓,却是无法动弹,便问道:“段大公子,阁下为何没有对这两个侍卫下毒手呢?哲心中虽然感激,却也觉得有些奇怪。”

段凌霄笑道:“我非是心慈手软,只是听闻楚乡侯精于用毒,昔日曾经以此将凤仪门诸人制住,而我又想和侯爷叙谈一番,所以留下这两人性命,希望能够让楚乡侯克制一下,不敢擅自用毒,以免伤害这两人性命。”

我目光一闪,道:“段大公子难道忘记了,他们都是我的侍卫,本就是为了保护我的安全才待在我身边的,我就是将他们一并害了,想来也没有人可以怪我,就是他们自己,九泉之下也会如此想。”

那两个侍卫眼中闪过热烈的光芒,看来对我的话语十分赞同,段凌霄虽然看不到他们的神情,可是仅凭他们呼吸的变化,就已经知道这两个侍卫果然是赤胆忠心之人,不过他却没有丝毫担忧,道:“若是旁人或者会如此做,但是段某觉得以江侯爷的性子,对敌人自然是绝情绝义,可是对自己人却是心慈手软,这是段某遍阅和侯爷有关的情报之后所得的结论,而且非若如此,玉飞恐怕也不能从侯爷手下逃出生天吧,如今若是说侯爷会不顾这两人性命而暗施剧毒,段某绝不相信。”

我一时语塞,虽然他说秋玉飞之事只是巧合,可是仔细想想,我还真的不大喜欢对身边人下毒手,不说别的,若是我没有事先安排好,今日遭遇这样的情景,我就是可以施毒,也断然难以下手的,毕竟这两个侍卫都是在寒园的时候就跟随我的旧人,这次又让他们置身险境,我已经是于心不安了。

段凌霄见我神色数变,知道说中我的要害,便不紧不慢地道:“江侯爷辅佐雍王殿下登上帝位,而后又抛弃权势隐居东海,段某本是十分佩服,只可惜侯爷终究抛不下君臣恩义,抛下隐逸生活助齐王攻我北汉,我虽深慕先生才华,如今也只能生死相见了,不过若是侯爷肯答应从此归隐林泉,再不为大雍出谋划策,段某今日可以放过先生一次。”

我轻叹一口气道:“荣华富贵于我不过是过眼云烟,只是江某生平最是贪生畏死,大雍若不能一统天下,江某今生也不能安居乐业,段大公子的好意我只能心领。不过大公子故作此言,是否希望削弱我属下众人斗志呢,其实在下手无缚鸡之力,大公子其实不必如此费心的。”

段凌霄叹息道:“侯爷过虑了,段某只是不忍四弟伤心罢了,他临去东海之前曾经传书与我,说及和侯爷相交之事,虽然当日他定要置侯爷于死地,但对侯爷却是十分敬慕。我知玉飞落落寡合,生平罕有知交,所以也不忍伤害侯爷性命,可是此番贵军虽然落败,仍是未伤元气,而侯爷在此时脱离大军保护,乃是我军削弱贵军的唯一机会。本来若是侯爷肯答应归隐,段某想请侯爷到晋阳休养,可惜我的好意不为侯爷所纳,如今只能是生死相见了。”

我不怒反笑,若是我方才贪生怕死,答应归隐,这段凌霄想必会接着提出让我随他到晋阳去,甚至还会以魔宗宗主之名起誓不伤我性命,可是我一个堂堂大雍监军,驸马都尉,楚乡侯真的被胁裹到北汉国都,我还有什么颜面去见长乐公主和皇上呢?这魔宗可真是好大的口气,可惜我江哲虽然贪生畏死,却也不是苟且偷生之人,当日我可以冒着生命危险去饮雍王的鸩酒,今日又怎会让自己陷入生死由人的境地,就是我没有事先设下罗网,等段凌霄自投,也绝不会甘心被俘的。

我恢复冰冷无情的心境,道:“段大公子,你可曾想过为何苏青一人前来接应?”

段凌霄心中一凛,眼前这青年瘦弱的身躯上突然散发出无穷的威严和杀气,令人刮目相看,他一边留心身边的动静,一边道:“想必是贵军以为我军斥候密谍已经全部退走了吧?”

我摇头道:“非是如此,哲平生最惯落井下石,所以不免以小人之心度君子之腹,也是哲心性骄傲,料想贵军必然要趁机刺杀在下,与其等待贵军来刺杀,不如引蛇出洞,我料来刺杀在下的必是段大公子,萧桐武功不如小顺子,他又是掌管军中斥候之人,不能轻易犯险,所以必是大公子出手,可是四野茫茫,我们隐蔽之处又是难以找寻,我若是大公子,也会盯着我军大营,因为我势必要和中军取得联系。所以我派苏青回去报信,一来阁下认得苏青,二来,有段无敌在安泽,应该可能知道苏青在我身侧,果然不出我所料,大公子跟踪苏青到此,我令虎赍卫布防着重于外围,阁下若想行刺成功,必须要等到我身边侍卫最少的时机,所以我遣开小顺子,只带了两个侍卫到湖边,果然阁下不出我所料,换上我身边侍卫的服饰之后,混到湖边欲图刺杀,不知道被阁下所制住的侍卫,是生是死?”

段凌霄心中一寒,自己连日来所为,这个江哲竟是如同眼见一般,他再次凝神细察,仍然不觉身侧两丈之内有人,他一边暗自思忖,一边漫声道:“自然是死了,不过虎赍卫果然厉害,我亲自出手,仍有一人几乎喊出声来,不过为了避免惊动众人,我只杀了三人,想来侯爷不会心痛的。”

我却当真是心中一痛,虽然早知必有牺牲,仍是让我心中愧疚,不由掩面道:“罢了,你们出手吧。”

第十七章    一线生机

第十七章一线生机

段凌霄早已心有准备,但是以他的身份若是我这么一说他就出手,那可就没有面子了,所以他仍然静立不动,想看清楚袭击从何而来,若是来自身后,那么十五丈的距离足以让他先杀了江哲,此刻他已经没有生擒江哲的想法了,这样的人物还是让他早些死去的好。

随着我的话音刚落,那两个侍卫前边和我只有半步距离的位置,突然泥沙飞扬,两道身影破土而出,转瞬间已经将我护在身后,段凌霄心中一寒,下意识地退了半步,怎么可能,这么近的距离自己竟然没有发觉有人潜伏。尘土飞散之后,段凌霄已经可以看清那两人身形,却是两个大约三十五六岁年纪的僧人,神光内敛,相貌平平,但是眉宇间却带着刚毅果决之色。段凌霄冷哼一声道:“原来是少林和尚?”

这时江哲清越的声音响起道:“法正大师、法忍大师乃是上一辈少林十八罗汉,数年前和凤仪门主一战之后,只有数人生还,两位大师经此一役,禅功精进,佛门武功最精吐纳收敛,所以才能瞒过大公子耳目,不过大公子放心,两位大师自承不是阁下对手,所以江某还另外请了高手前来助阵。”

段凌霄心中一沉,这两个和尚都曾经和凤仪门主生死交战,能够幸存下来已经不好对付,想不到江哲还有高手暗藏。这时他身后传来一个沉稳地声音道:“贫道张锦雄见过段大公子。”然后又是一个柔和的声音道:“峨嵋凌真子见过段大公子。”

段凌霄身形一闪,已经退后丈余,然后侧过身去,向身后两人看去,只见从那横线之后,两人缓缓走去,一个是青衣道人,相貌方正威猛,神完气足,双手空空,另一个却是一个淄衣女尼,相貌秀丽,神色恬淡,手中一柄拂尘。段凌霄不由轻叹道:“想不到江侯爷这次真是势在必得。”

我隐身在法正、法忍身后,闻言不由嘴角上翘,但是很快就收敛回去,这几个人可是我想了又想才选出来的,这次进攻沁州,为了防备魔宗弟子,齐王早就上书朝廷请动了江湖正派高手相助,各派最出众的高手往往都在本门潜修,这也难怪,武功练到了一定境界,沙场征战已经无助于心境的修炼,留在大雍朝中军中的高手往往不是绝顶高手,有几个武功绝顶的又都在皇宫,所以这一次我是特意请皇上征召了一些江湖高手在军中听用,当然此事十分隐秘,这些高手的身份可是很秘密的。

少林寺派来的高手最多,当年幸存的十二金刚就来了六人,还不算其他各代弟子。崆峒前次依附太子,虽然因为张锦雄的迷途知返而没有遭到牵连,可是也没有得到什么好处,这一次可是出了血本,让掌门弟子张锦雄带着十二名门中精锐弟子随军。张锦雄回到崆峒之后,因为经历大风大浪颇多,看破世情,出家做了道士,武功更是突飞猛进,又修炼了崆峒几种秘传绝学,如今武功已经是超一流水准,虽然还没有进入先天境界,可是也不过一线之差。峨嵋也不含糊,凌真子乃是峨嵋第一高手,虽然年过四旬,却是仿佛二十许人,峨嵋乱披风心法已经是炉火纯青,一手拂尘绝技天下闻名。

我写给齐王的书信,让他提前一天派来几个高手,按照信上的地图赶到此处,然后布下陷阱,等待段凌霄入伏,当然我也想到可能段凌霄不会来,但是在我计算中,至少有六成机会可以见到段凌霄,如今他已入伏,这四大高手虽然都未能进入先天境界,可是也基本上都是一线之差(这是小顺子评估后的结果),再有小顺子压阵,段凌霄可是插翅难飞了。

想到得意处,我朗声道:“若是段大公子肯束手就擒,江某愿意立誓不会相害,不知段大公子可愿意么?”

段凌霄深沉如渊海的眼中闪过一丝了悟,道:“生有时,死有地,此地清幽如同世外桃源,段某就是死在此处,也是无怨无尤,江侯爷手段通神,在下佩服。”

我听得此言,却是心中一动,一件从前被我忽略的事情涌上心头,我要杀段凌霄,实在是因为他武功太高,想到若非苏青探察到敌军水攻之策,只怕我也难逃水淹之祸,所以我越发担心段凌霄此人,他武功高强,若是将来被他发觉我的布置,岂不是功败垂成,所以我才不惜以身涉险,诱他入伏,准备将他击杀。可是目的即将达成之际,我却想到了另外一件事情,这些日子只想着此人的威胁,却忘记了此人乃是魔宗首徒,若是此人死在此地,那么京无极竟可以堂而皇之的亲自出手,我岂不是自找麻烦,只要段凌霄不死,京无极除非我们攻到晋阳,是不会轻易出手的,所以段凌霄不能死,甚至不能生擒,只能让他身负重伤而走,这样才符合我军的利益。

在我沉思的时候,段凌霄已经出手,身形直扑向我所站的位置,似乎想要一举狙杀于我,当然,法正、法忍早就严阵以待,临来之前,齐王曾有严令,若是楚乡侯有什么三长两短,谁也别想好过,方才隐在坑中听见江哲与段凌霄交谈,两人已经是心中忐忑不安,生恐江哲有个好歹,虽然知道江哲身穿软甲,而且两人又做好了阻拦段凌霄一击的准备,仍然不免心中惴惴,此刻那里会让段凌霄得手。就在两人出手相拦的时候,三股真气一触,段凌霄已经以比来势更快的速度退了回去,半空中身形一转,意图脱逃。而这时,唯一可能身法胜过段凌霄的小顺子却是不管不顾,抢到了我身边将我护住,眼看段凌霄就要脱出四人包围的时候,三抹红光一闪,恰好拦在段凌霄去路之上,段凌霄挥手一扫,红光穿破了他劲风,在他身前才缓缓跌落,饶是如此,段凌霄也是身形一滞,已经被法正、法忍、张锦雄和凌真子围在当中。那三抹红光却是张锦雄以崆峒秘传手法射出的三枚血蒺藜,可以穿破先天真气的绝毒暗器。。

段凌霄见唯一的机会已经失去,神色一凝,立稳门户,专心迎敌,五人战在一起,段凌霄固然是武功高强,而四人早就练习过联手合击之术,法正法忍内力高深,大开大阖,几乎承担了大部分攻击,而张锦雄武功走偏锋,狠辣歹毒,杀伤力最强,凌真子的乱披风心法最擅以柔胜刚,她也不急躁,仗着轻功身法拦在外围,只要段凌霄一想突围,就会面对她无孔不入的攻势,四人联手,果然威力无穷。虽然段凌霄不愧是魔宗首徒,应付得宜,不露败相,可是想要脱身也不是容易的事情,更何况还有一个小顺子在外面虎视耽耽呢。

小顺子护在我身边,看着这场龙争虎斗,却没有出手,一来是不放心我的安全,二来却是在研究段凌霄的破绽,希望一举克敌,他的心思可是瞒不过我的,我微微皱着眉头,想着如何处理现在的局面。这时,呼延寿和苏青带了二十余人回来,呈上六颗首级,呼延寿高声道:“启禀大人,随段凌霄来犯的六人皆已斩杀,请大人查验。这些人都是武功高强之辈,应该是魔宗弟子,不过我们也有三名弟兄受了伤。”

我微微一叹,唉,段凌霄入庄之时,我暗中埋伏下的哨探已经发觉跟随他来的这些人,所以段凌霄杀我侍卫,夺取衣甲的时候,我就已经知道他来了,只是可怜那几个侍卫了,年纪轻轻就死在敌人之手,我却无能为力,淡淡道:“呼延将他们的首级拿去祭奠勇士英魂吧。”

呼延寿知我心意,并不起身,道:“大人设伏之事,我等都早已知晓,其中危险人人尽知,就是丢了性命也是无怨无悔,请大人不必自责。”

我心中一暖,深深的看了呼延寿一眼,道:“若是我不得已需要放过此人,你们也不会恨我么?”

呼延寿心中一惊,但他很快就说道:“大人必是深思熟虑,才作出这样的决定,末将等人不会有怨言。”

我心中一宽,看看苏青,只见她目光炯炯,望着呼延寿,神色间有些惊异,见我望向她,才道:“大人神机妙算,如此决定必有深意,苏青支持大人任何决定。”

我这才放下心来,道:“段凌霄带来之人想必都是好手,杀此六人已经足以抵偿我军勇士的性命,你们先退下去吧。”

呼延寿和苏青退去,两人指挥虎赍卫将周边围住,摆好了苦练的刀阵,若是段凌霄脱出重围,也绝不可能轻易突破他们的刀阵。天罗地网已经搭就,段凌霄已是网中之鹰,再也难以脱身,只是我却心中难以决定,究竟是杀还是不杀。

又战了百余招,段凌霄心中清明如水,虽然围攻他的四人都是当时高手,可是和他比起来还是相差很远,先天后天虽然只有一线之差,却是天渊之别,若是只有这四人,拼着受些伤,段凌霄也自信可以将他们全歼,但是如今外面有百余虎赍云集,刀阵已成,他已是难以脱身,而站在江哲身侧的那个青衣少年,虽然没有出手,但是冰寒的目光彷佛可以穿透人心,段凌霄几乎用了五分心思来防备他,天罗地网即成,就是师尊在此,也未必可以全身而退。如今险恶局势,段凌霄却只觉心中兴起丝丝快感,生死一线的这种刺激对他来说已经是很难领略到了,这困窘的情势反而让他更加兴奋起来。

小顺子眼中突然寒光一闪,因为他已经发觉场中的战局有了隐隐的改变,虽然段凌霄仍然是以一敌四,而己方四人仍然是交错攻守,不论是进攻还是防守都是浑然一体,仿佛一个人长出了四双手臂一般,可是段凌霄似是胸有成竹,往来自如,虽然不能突破四人围攻,但是不论四人如何施展奇招妙技,都被他化解于无形。虽然此人乃是大敌,可是小顺子还是心中暗暗敬佩,他对江哲的情绪变化十分敏感,方才已经隐隐感觉江哲心中有些忧虑,所以低声问道:“公子,我需出手了。”

我一个激灵清醒过来,看看场中激战的段凌霄,神色从容,气度雍容,心道,若是我让小顺子生擒,恐怕会妨碍他出手,段凌霄是生是死还是看他自己的运气吧,最多我和魔宗对上就是,神色恢复如常,我冷冷道:“出手吧,小心行事,生死不论。”

小顺子轻轻点头,缓步上前,呼延寿和苏青则知机地站到我身边,将我护住,毕竟我的安危才是最要紧的,魔宗武功高深莫测,谁知道段凌霄有没有什么两败俱伤的绝学呢,若是给他寻到机会伤了我,就是将段凌霄千刀万剐也不能挽回这样的损失。

段凌霄心中凛然,他自然是看到了场外的变化,小顺子若是参与围攻,那么他就没有生出的可能了,可是他也知道这是一个唯一的逃走机会,若是小顺子要加入战圈,那么围攻自己的四大高手不免要让开一个空隙,而敌方的第一高手亲自出手,不论如何,其他人心中都会有些松懈,如果自己能够把握包围开阖的瞬间,就可以突围而出,错过这个机会,再也没有任何希望。可是如何把握这个机会呢?段凌霄心中生出死志,灵台一片空明,六识达到平生最灵敏的境界,他的这种变化虽然细微,而且出手也没有什么改变,可是围攻他的四人都是只差了先天境界一线距离的高手,心中顿生涟漪,也知已经到了生死存亡的关键时刻,凝神专注,准备在最合适的时机放开防线,让小顺子可以进入战局。这种无言中的紧张局势就连那些看不出其中奥妙的普通虎赍卫也都凝神屏气,不敢有丝毫松懈。

我虽然不会武功,可是在东海之时也常常看桑先生、小顺子和董缺等人切磋,凭着我过人的六识,更是将各人神态看的清清楚楚,何况江湖搏杀也是暗合兵法,我心中灵光电闪,突然明白了胜败关键。小顺子加入战局之时,正是我精心设计的陷阱最强之时,而在这变化之前的刹那却是最弱的一刻,只要渡过这一刹那,段凌霄就已经落入我的掌握。心中电转,看着缓缓接近正在交手的五人的小顺子,我心中盘算着如何襄助众人,破去段凌霄的一线生机,目光一扫,心中已经有了决定,对着身边的苏青低声说道:“你威力最强大的剑法是什么?”

苏青低声道:“师尊曾传苏青一招剑法‘玉石俱焚’,只是苏青练得还不到家,不能随意使出。”

她的声音快速而低微,没有丝毫犹豫,我心中一阵赞赏,果然是训练有素的军人,对上官的命令没有丝毫违逆之心,我也不和她客气,道:“用你最凌厉的剑法,等到小顺子加入合围的时候,阻拦趁机突围的段凌霄。”

这时候小顺子已经走到战圈外围,幸好他为了让四大高手做好自己加入的准备而缓行,否则我可没有时间安排苏青阻击了,而苏青也不愧是闻紫烟弟子,我虽然说得不甚明白,可是她却心领神会,趁着众人都注视战局的时候,轻轻移动到旁边,右手按在剑柄上,一双冰寒的美眸盯着段凌霄的一举一动,而呼延寿则移动一步,将苏青移开的破绽弥补过去。

就在这时,围攻段凌霄的四大高手,同时移形换位,身形快捷如电,众人只觉眼前一花,这四人已经变换了方位,而本来严密的防线也留下了一个空位,而小顺子身形如同鬼魅幻影一般,出现在那个空位上,五人动如风火,选择的时机几乎可以说是完美无缺,可是,果然如同天地至理一般,阵势在转为最强之前就是最弱之时,就在战阵开阖这一刹那,段凌霄的身形仿佛化成虚幻,如同惊雷掣电一般突破了重围,如同流虹逸电一般向湖水方向掠去。而这一刻,看到小顺子加入战局的众人果然都是本能的心中一宽,这一丝几乎难以察觉的破绽被段凌霄牢牢把握。他所选的方向也是经过精心挑选的,虽然这个方向似是绝地,可是江哲却正在这个方向,所以保护江哲的呼延寿和苏青必然都会首先竭尽所能保护江哲,以这两人的武功,自己绝对不可能一击取了江哲性命,而段凌霄也没有想过这样做,他只是希望凌波而过,隐入对面的密林当中。

他的计策本是万无一失,就在他从小顺子身侧掠过的时候,五人都是大惊,用尽浑身解数拦阻于他,两个少林僧人都是大喝出手,凌空直击,百步神拳击向他的背心,而张锦雄面色突然变得通红,吐气开声,一拳击向他右肋,这正是崆峒最高深的绝学——七伤拳,这一拳暗藏七种不同的劲道,若是击中人身,可令令骨骼经脉全部震断,外表却是看不出任何伤痕,凌真子则是一声叱喝,拂尘上千万银丝都抖的笔直,拂向段凌霄的后脑,而最具威胁的就是小顺子,他的武功本就和段凌霄相差无几,那玄铁发簪早已不需使用,一指凌空虚指,一道阴冷冰寒的真气如同利刃一般刺向段凌霄重穴。在这狭小的空间之内,各种劲力交错激荡,段凌霄身上所穿的虎赍卫软甲化作片片蝴蝶,在尖利的劲风呼啸中,段凌霄成功的突破五人围堵,身形化成一个弧线,准备避开直面江哲的方向,毕竟他还不想因为激怒众人而再度落入重围,而江哲若有生命之险,那是最能激怒众人的事情。

而就在段凌霄突破包围的时候,一声剑啸惊破长空,一道黑色身影凌空向段凌霄逃逸的路线扑去,剑光如同春云乍展,剑势更是充满了有我无敌,一去誓不回的气魄,剑光临身时,段凌霄心中长叹,一拳击出,拳剑相交,那柄百炼钢的长剑寸寸断折,苏青倒飞而回,段凌霄也是后退了半步,此刻他离湖水也不过三步之遥,可是咫尺天涯,生死相隔,小顺子面带严霜,已经挡在段凌霄身前,将段凌霄拦住,而四大高手也已经合围而来,五人将段凌霄困在其中,战阵已成,再无空隙。段凌霄一声长叹,知道自己唯一的生机已经生生断绝。他的目光穿越众人,落到了江哲身上,只见他面上带着淡淡的微笑,仿佛一切尽在其算中,而苏青则面色苍白地站在他身侧,可见方才那一剑也是令她损耗极大。虽然出剑的是苏青,可是段凌霄却知道苏青没有那样的心机察觉自己的突围时机,而最有嫌疑的自然就是可以指挥苏青的江哲了。想不到自己也会丧命在这个青年手上,段凌霄露出一丝苦笑。

看着被小顺子和其他四大高手联手迫回原处的段凌霄,我心中终于一宽,这下段凌霄是注定被留在这里了,就是想要生擒也未必没有机会了,方才他突围之际,必然受了重伤,小顺子和四大高手的拦截不是可以轻易避过的,如今小顺子他们心中不免羞恼,出手一定更加严谨,这样的情形若是段凌霄还能逃生,那么他只怕已经可以列入宗师一级了,不过在我看来,似乎是没有这个可能。不过我倒是真的佩服此人,小顺子武功可能和他差不多,但是在经验上可是差得多了,毕竟是年纪太轻了。不过经过今日一战,他应该更能精进一步吧。

又过了片刻,就是我这不懂武功之人也看得出段凌霄似乎已经没有还手之力,只是凭着意志苦撑罢了,小顺子等人却是配合默契,越来越得心应手。就在我心想是否让小顺子生擒段凌霄的时候,小顺子突然连出杀招,我只觉眼前一花,场中局势已经大变,小顺子和段凌霄两人硬碰硬地激斗起来,而其他四人则将两人围在当中,伺机袭击段凌霄的软肋。还没有等我反应过来,小顺子已经一掌击中段凌霄肩头,段凌霄身形踉跄后退之际,法忍、法正都是精通擒拿手的少林高手,趁机出手,将段凌霄绊住,段凌霄一声厉喝,一道碧血从口中激射而出,法忍法正都是少林高手,对魔宗密学颇有了解,都是极力闪躲,避开了内含段凌霄精血真气的“碧血箭”,段凌霄得到一丝空隙,但是张锦雄和凌真子却已经补上了空位,段凌霄低身避过凌真子的拂尘,却觉右膝一痛,却是小顺子一指虚点,指风击中他膝间委中穴,冰寒的真气侵入要穴,段凌霄几乎站立不住,他索性右膝跪地,一个翻滚,间不容发之际避过张锦雄掌风。段凌霄自知生还无望,他也看出敌人有生擒之念,否则刚才两个和尚就不会使用擒拿手了,心中顿时生出绝决之念,身为魔宗首徒,未来的魔宗宗主,焉能被俘受辱,段凌霄心中一叹,就要自断心经。就在千钧一发之际,众人耳边突然传来一声厉喝道:“统统住手,不然我杀了此人。”

第十八章    以命抵命

第十八章以命抵命

段凌霄本已心如死灰,但见小顺子五人都是收手后退,除了将自己围得更加严密之外,竟然都不再出手,不由抬目望去,只见那些虎赍卫士向两侧散去,露出两个人来,那两人一个是白发老者,一个是尤带稚气的清秀少年,那老者神情萎靡,手臂上胡乱缠着布条,鲜血渗出布条缝隙,更显得万分狼狈,而那少年左手架着那老者,右手执短刀抵住那老者咽喉,正站在江哲对面,相距遥遥。这时,那些虎赍卫中突然传出叱骂之声道:“凌端,你这忘恩负义之辈,竟敢用人质威胁我等。”江哲冷冷望了那虎赍一眼,冰冷的目光让他悻悻退下。

却原来那少年正是凌端,他跟随秋玉飞回到北汉之后就无意回到军中,毕竟对他来说,他的将军只有谭忌一人,何况秋玉飞有意引荐他投入魔宗,虽然秋玉飞没有来得及回到晋阳就去了东海,但是仍然给了他一封书信让他去见段凌霄,而段凌霄对凌端颇有好感,虽然还没有正式将他收为弟子,但也是迟早之事。凌端跟在段凌霄身边虽然不久,但是他的武功本是谭忌给他扎的根基,又得秋玉飞、段凌霄先后点拨,武功精进不少,虽然还不如这次段凌霄携带的几个魔宗记名弟子,可是已经勉强进入二流,他又是多年从军,对沁州、泽州地理十分熟稔,所以这次也跟随段凌霄参与了战后的搜杀行动。不过在跟踪苏青的时候,段凌霄是独自进行的,而其他接应段凌霄的魔宗弟子则是跟着段凌霄留下的标记赶来的,只有凌端因为武功不高,在十里之外就被众人留下看守马匹,这才逃过了虎赍卫的捕杀。可是凌端却不甘心留在后面等待,对他来说,江哲是他生命中最大的阴影,他最尊敬的将军,他同患难的朋友都是间接死在这人手中,所以他违背命令偷偷潜入村中。不过他来得晚了,此时虎赍卫已经撤下埋伏,在湖边困住了段凌霄,其余魔宗弟子纷纷授首,凌端来得迟了,却是保住了性命。

凌端自知没有本事救援段凌霄,心中只能企盼段凌霄能够自己逃走,可惜的是,段凌霄突围失败,凌端心中明白此番必是全军覆没,而唯一的转机就在于自己,因为似乎雍军没有发现自己的存在。虽然段凌霄尚未正式收凌端为徒,可是凌端心中已经将段凌霄当成了恩师,弟子为了救师尊性命,本就应该不吝牺牲,所以凌端作出了不顾生死的决定。

他潜入村中之时就发现了纪玄和赵梁两人,这两人被两名虎赍卫保护着,或者说是软禁着,不许他们离开住处,赵梁倒没有什么,赵玄却是在那里不住口的抱怨江哲,听得那两个虎赍卫苦笑连连。跟随了江哲一段时间的凌端知道江哲虽然性情随和,可是御下却很森严,他可是亲自领略过江哲手段的,而赵玄虽然怨言不断,可是凌端凭着直觉却能够感觉到这个老人语气中的亲切,他谈及江哲的语气倒像是知交和长辈的口气,而从那两个虎赍卫的神情上来看,也并未因此恼怒,这说明江哲对这个老人不是很尊重就是很容忍,不论是那一种情况,都说明了这个老人的重要性。想到这里,凌端便决定挟持赵玄要挟江哲,当然可能江哲根本就不在乎这个老人的性命,可是凌端绝不能眼睁睁看着段凌霄死在这里,他很清楚段凌霄的高傲,若是落败被俘,他是绝对不会苟活于世的。

可是不说那个忙着整理行装的青年武功不弱,就是那两个虎赍卫也不是自己可以轻易对付的,而且还不能惊动湖边的雍军,不过幸好凌端带了一筒袖箭,这本是萧桐给他的,这时北汉斥候使用的擒敌利器,箭头上淬了强烈的麻药,可以生擒敌人以便刑讯,凭着秋玉飞、段凌霄传授给他的密技,他顺利地将四人全部放倒。不过他并没有取这几人性命,这却不是他心软,他是担心若是杀了这几人激怒江哲,只怕会弄巧成拙。

我初时心中如同翻江倒海,怎么会有这样的事情发生,两名虎赍卫和赵梁保护着纪玄,凌端武功虽然出色,毕竟年纪还轻,不会是虎赍卫的对手,就是偷袭暗算,也不该无声无息地得手啊。苏青在我身边低声道:“大人,那人想必用了淬药的暗器,两军斥候都有这样的暗器,那是为了生擒敌人用的。”我心中恍然,怪不得纪玄一脸有气无力的模样,这样的手段我不是不知道,甚至秘营弟子手中的淬毒暗器都是我亲自研制的,不过我一直当凌端是一个品性光明之人,一时想不到他会用这种手段罢了。如今想来不由暗笑,毕竟凌端乃是谭忌亲卫,看来如今和魔宗关系也是非浅,这样的出身,怎会计较什么手段。

我看了一眼纪玄,见他神情委顿,心中不由微怒,道:“凌端,昔日之事江某也懒得提起,你视我待你恩义如同粪土,我也不怪你,今日你竟然想用人质威胁本侯,莫非你以为本侯乃是心慈手软之人么?”

凌端心中一寒,只见江哲神情冷淡,虽然是文弱书生,气度儒雅,但是此刻负手而立,单薄的身躯彷佛如同雪里青竹一般傲然,眉宇间更是带着淡淡杀气,想起昔日之事,只觉得思绪如潮涌。他苦涩地道:“大人手段,凌端不敢或忘,昔日凌端本已是待死之囚,幸而得大人怜悯,逃出生天。虽然大人后来杀了李虎,凌端心中怨恨多时,可是如今想来,我们的性命本就是大人捡回来的,就是大人再收去我们也是无话可说,当时大人若为稳妥,本应将我一并灭口,可是大人还是放过了在下。当日雪地野店中,凌端为琴声激起心魔,刺杀大人,又是大人开恩,饶了凌端性命。三番饶命之恩,凌端不敢忘记,可是凌端也不能忘记谭将军、李虎之死,而且如今段大公子乃是凌端欲拜恩师,恩师性命危在旦夕,身为弟子焉能坐视。凌端猜测大人对这老先生十分关爱,所以斗胆要挟,只要大人肯放过大公子,凌端情愿一死谢罪。”

我皱皱眉头,虽然杀死段凌霄不是我的意思,可是我也看出来了,若是想要生擒恐怕是没有可能的,这个段凌霄身份十分重要,见他性情才智,绝对不是肯忍辱负重的人物,可是这样放过他我又不甘心。下意识的望着小顺子,我用眼色询问他的意见。

小顺子眉头一皱,在他看来,自然是杀了段凌霄最好,那个纪玄如何比得上段凌霄重要,更何况若是有这样一个高手,终究是公子的威胁,可是他也知道自己不能擅自作主,毕竟公子眼光深远,很多决定当时看来十分不智,日后却是决定胜负的关键,所以他最后决定只将当前情形说明即可。思忖一下,小顺子传音道:“公子,段凌霄先后中了我两指,如今已经受了严重的内伤,我的内力至阴至寒,桑先生又曾经传我一种心法,可以克制魔宗心法,他的内伤如同附骨之蛆,若想恢复如初,就是有魔宗相助,没有数月时间也是不可能的,现在他不过是强行支撑罢了。”

听了小顺子的话,我心中略宽,既是如此,一个不能动手的段凌霄换纪玄,我就不吃亏了,不过便宜需要多占,也不能让凌端轻松得逞,否则以后有人效仿怎么办呢?故意将神情放得更冷,我森然道:“凌端,念在你也曾经在我身边听用,只要你放了纪老先生,我就饶你性命,否则我就先杀了段大公子,再和你周旋。”

凌端眼中闪过坚定的神色道:“大人,凌端既然敢要挟您,就没有将生死放在心上,若是大人令人继续向大公子出手,凌端只有先杀了这位老先生,然后陪着大公子死在此地,此人是生是死,大人一言可决。”

我心中一跳,想不到这个凌端如此坚决,不过他怎么会知道我定会交换人质?这时候,纪玄或许是药力渐退,勉力高声道:“老夫不用你江随云相救,要杀就杀,老夫岂是可辱之人。”我几乎咬碎了牙齿,这个纪玄,真是给我找麻烦,不过凌端若是误会我不想救他就麻烦了,连忙仔细查看凌端神色,见他神情越发自信,任凭纪玄高声呼喝,只是将短刀抵住纪玄咽喉,既不轻也不重,免得伤害了他,也提防他挣脱。见我沉默不语,凌端高声道:“大人,你若是再不决定,我就只好杀了他。”

我恨恨地看向段凌霄,道:“大公子怎样看这件事情?”

段凌霄方才一直调理自己的伤势,以便再出手时可以寻个陪葬,他并不能肯定江哲会为了一个老人放过自己,听到江哲向自己询问,淡然道:“端儿也是胡闹,大人乃是千金贵胄,怎会轻易受威胁,段某自信身价不低,端儿还是速速离去吧,至少这人换你的性命应该是够了。”

凌端眼睛一红,几乎要喷出火来,他自然也怀疑江哲是否会受自己威胁,虽然江哲似乎很重视自己手上的人质,可是段大公子乃是魔宗首徒,地位尊贵,就是换了自己,也绝不会轻易放过,可是只要有一线希望他也不愿放弃。望向江哲,他咬牙切齿地道:“大人,请你决定,若是不肯交换,在下只有杀了此人,也算讨回一些利息。”

我心中一凛,凌端生性孤傲乖戾,若是再逼迫下去,只怕他真的会杀了纪玄,那可就糟糕了,既然段凌霄已经受了重伤,就是放了也没有什么关系,反正只要他数月之内不能出手,我就放心了,等到他可以出手的时候,北汉已经大厦将倾,他武功再高也没有什么用处了。

我微微苦笑,心道,放过段凌霄也就罢了,可不能轻易放过你,眼珠一转,我冷冷道:“纪老先生虽是我忘年之交,可是段大公子乃是北汉国师首徒,地位何等尊贵,今日一见,也觉大公子乃是一代豪杰,就是放他走也无妨。可是你挟持人质要挟本侯,本侯若是将大公子轻轻放过,岂不是令天下人觉得本侯是可以要挟的,这样吧,若是你肯放了纪老先生,我允许你用自己性命交换段大公子的性命如何,一命抵一命,我已经吃亏了。”

凌端一愣,虽然他已经准备付出生命的代价,可是没有想到会是用这种方式,但是仔细一想,凌端反觉欣然,心道,挟持人质本来就是无耻之事,自己不过是一个小人物,大公子却是魔宗首徒,若是能够以命抵命,果然是自己占了便宜,想到这里,他冷静地道:“大人千金一诺,凌端从未见过公子有食言之事,以命抵命,凌端心甘情愿,只是请大人恕罪,大公子离去之前,凌端不能放开人质。”

段凌霄微微摇头,此刻他心知肚明,江哲或许并不想留下自己的性命,只凭方才江哲指使苏青拦截自己的手段,就知道江哲乃是心思缜密之人,也是狠毒之人,绝不会给敌人留下一条生路。他自问若是自己面对这样的局面,虽然有些危险,可是不是没有成功救下人质的可能,凌端的武功并不高。他也不会认为江哲真是信守承诺之人,只要杀了所有知情的外人,还会有谁知道他曾经不守诺呢。所以或许凌端是促成自己生还的人,可是若非江哲早有这样的想法,那么自己是绝对不可能得到这一线生机的。而江哲要凌端以命抵命,或者是因为报复凌端损害他的威严吧。可是如今段凌霄已经没有办法阻止这一切的发生了,除非他真得想死在这里,可是就是他甘愿一死,也是救不了凌端。他抬头向江哲看去,恰好江哲也正向他往来,那双清澈沉谧的眼睛仿佛带着一丝嘲讽,四目相对,段凌霄清晰地看到江哲面上闪过一丝惊诧,似乎他已经发觉自己看穿了他的心思,他不由露出苦涩的笑容,无论如何,自己的性命是一个魔宗后进弟子换回来的,这样的屈辱想必会跟着自己一辈子吧。

轻轻叹了口气,他淡然道:“端儿,放开纪老先生吧,江侯爷是什么人,岂是你可以威胁的,如今他既然答应了,就不会无故反悔,你也不要固执了。”

凌端心中茫然,他对段凌霄已是敬重非常,犹豫了一下,终于放开了纪玄,他自信大公子不会自寻死路,果然他放开纪玄之后,除了两个虎赍卫迅速扶走纪玄之外,江哲并没有下令攻击,甚至也无人来将自己制服。

我看了一眼神色茫然中带着死寂的凌端,知道这个少年是真的放弃了一切生存的欲望,不由心中怜惜,这时,一个虎赍卫匆匆赶来禀报道:“启禀大人,赵公子等三人都没有生命危险,只是昏迷过去了。”

我心中一宽,看看凌端,冷冷道:“凌端,你可知我为何一向对你优容。”

凌端抬起头,苍白的面上没有一丝血色,他咬紧牙关一言不发。

我冷冷道:“你不过是个普通士卒,我何需利用讨好你,若非你是谭将军亲卫鬼骑,你的生死我何需留意,当日本侯将你留在身边为侍从,可没有屈辱你,而你却忘恩负义,私自逃走,这也就罢了,念在谭将军面上,你忠心北汉也是无可厚非,本侯虽然令人缉拿,却没有真得对你如何,你侥幸偷生,就应该好生保住性命,可是你今日至此,恐怕也是为了刺杀本侯来的,见事机不遂,又胁迫人质威胁本侯,是可忍孰不可忍,来人,将他拖下去重责五十皮鞭。”

自有虎赍领命将他押了下去,凌端已是全无反抗之心,默默垂手走了出去,不多时,远处响起皮鞭着肉的声音。

处置了凌端,我看向段凌霄,微笑道:“大公子对我如此处置可有异议?”

段凌霄眼中闪过一丝庆幸,道:“侯爷慈悲,肯饶了凌端性命,段某感同身受,就是侯爷如今违背承诺,取了段某性命,段某也是死而无撼。”

我微微一笑,段凌霄果然目光如炬,只凭我责罚凌端,就知道我无心杀他,一来我曾经利用凌端,未免对他有些歉疚,二来,凌端的性情我很喜欢,既然他没有杀死被暗算的虎赍和赵梁,我也就网开一面了,当然最重要的原因是经过今日之事,凌端必然已经在段凌霄心中有了不一般的地位,将来必然成为魔宗的重要人物,有一个对我戒惧而又感激的人存在于魔宗之中,对我绝对是一件好事情,毕竟北汉魔宗是不可能覆灭的,不说魔宗传承自有独到之处,只凭着我的本心,就不会想要灭掉魔宗,毕竟皇上和我都不想看到少林寺这些名门宗派独大,江湖和朝廷一样,权力都需要制衡。

既然对段凌霄已经没有了杀意,我挥手令众人退去,只留下小顺子、呼延寿和苏青在身边保护,就连四大高手也让他们退到远处,段凌霄却没有趁机发难,他内伤非轻,小顺子却是全无损伤,再有苏青、呼延寿这样的高手在旁,段凌霄就是再自负也不会相信自己可以刺杀我,这样聪明果决的人岂会作出无益之事,所以我也摆出这种友善的格局,不过小顺子是不会让他离开了,千金之子,坐不垂堂,我可是很小心的,谁知道段凌霄会不会发疯呢?

我温和地道:“段大公子,凌端不适合再留在沁州,我会将他送到东海和玉飞一起,不知道大公子意下如何?”

段凌霄目光一闪,道:“多谢侯爷体恤,这孩子武功虽然不高,但是人品资质都是一流,我也不忍心他在战场上有什么损伤,玉飞对这孩子另眼看待,送去东海也是好的,侯爷对凌端果然是十分爱重。”

我轻轻一叹道:“哲平生遗憾,就是没有亲见谭将军一面,谭将军只有这么一个亲近侍卫留下,本侯怎忍心取他性命。”

段凌霄心中一动,见江哲语气诚挚,也不由叹息道:“谭忌孤傲绝世,心中满是仇恨悲苦,当日师尊曾有意收他为门下,可惜因为他心魔太重,所以只命在下代传武艺,谭将军身死,我亦痛心不已。”

我朗声吟道:“天不仁兮生离乱,地不仁兮起狼烟;亲族父母兮化尘土,志摧心折兮可奈何;怨虽报兮恨不息,君恩重兮死亦难;杀人盈野兮吾且不悔,流血飘橹兮生灵涂炭;君执弩兮吾持戈,吾驱骑兮君相从;沁水寒兮葬吾躯,赴黄泉兮心意平;生死无惧兮慨而慷,逢彼旧人兮吾心伤!”

段凌霄默默听着,神情间也现出怆然之色,默默回忆着谭忌的音容笑貌,心中悲意丛生,却又突然惊觉,他修炼玄功多年,本已很难情绪波动,想不到如今却是情不自禁,看来内伤之重尤在估计之上,他面色不露出丝毫异态,淡然道:“侯爷真是矛盾,谭忌虽然是死在齐王手中,计策恐怕却是侯爷定的,如今又何必为之感伤呢?”

我傲然一笑,道:“我虽然一介书生,却有些傲气,这世间之人虽众,却多是碌碌无为之人,而其中佼佼不群者却是凤毛麟角,我生平最爱豪杰,不论是敌是友,都不会怠慢,只是可惜我终究是世俗之人,碍于身份所限,纵然是心中爱重,也要除之而后快,谭将军、段大公子都是世间豪杰,所以谭将军必须得死,而大公子你虽然今日可以不死,但是焉知我不是为了今后的布局,只是到时大公子不要怪我才好。”

段凌霄朗声笑道:“江随云果然豪爽,你虽然是文士,却豪情不减当世英雄,雍帝有你辅佐,难怪这般得意,凌端不过是个后生晚辈,你不杀他也就罢了,不过玉飞曾经刺杀于你,你为何不杀他,反而不惜代价留他在东海呢,这却不是妇人之仁么?”

我微笑不语,秋玉飞虽然武功精进,但是他生性爱好音律,厌倦世俗,这样的人怎会对我造成威胁,留他下来,一来是我欣赏他,二来也是因为将来有用他之处,杀一个人不代表厌憎他,手下容情不代表慈悲,这些事情岂是可以对人解释清楚的,何况我也无心辨白,就让别人认为我有妇人之仁不好么?

见江哲不语,段凌霄也是默然不语,他自然知道两人终是敌对,不能交心,可是这些许时候相处,段凌霄却觉得江哲此人虽然是文弱书生,却有林下之风,相处之际时而觉得如沐春风,时而觉得如履寒冰,令人生出不忍远离也不敢亲近的矛盾感觉,只可惜此人却是大雍重臣。

沉默片刻,我也从自己的思绪中清醒过来,吩咐道:“呼延,去取酒来,我要为大公子送行。”

呼延寿警惕的看了段凌霄一眼,下去召唤一声,不多时亲自捧了一个木托盘过来,上面放着一个酒壶,两个酒盏,我亲手提起酒壶,将两杯酒倒满,自己端起一杯,呼延寿端着托盘走到段凌霄身边,段凌霄坦然一笑,也是端起一杯。

我举着酒杯道:“大公子,你杀我侍卫,我斩你同门,两国交兵,你我乃是仇敌,此地只有乡野村酿,不过今日相逢也是有缘,若是无酒难以尽兴,不知道大公子肯否赏光。”

段凌霄一饮而尽,道:“今日交手,我败你胜,可是贵军虽然强大,却未必可以取胜,希望阁下珍重。”

我不与置评,只是缓缓喝下杯中酒,道:“大公子可惜没有领军作战,以你的机智果决,用兵应该不在我国陛下之下。”

段凌霄先是一愣,又露出淡淡苦笑,自己身为魔宗首徒,需得维持超然姿态,怎能领军作战,再说一旦陷身军旅,武功就难精进,自己乃是师尊嫡传,为了维系师门声誉,更是不能分心世俗之事,只是这种缘故如何能够对人说起。

送走了飘然远去的段凌霄,我心中也是庆幸,幸好这个人不是我的对手,令人带过受刑之后的凌端,我也没有多说什么,只问他愿不愿意去东海见秋玉飞,若是愿意就自己上路,凌端目瞪口呆之余,点头应允,向来他也没有面子再和我作对了。不过他离去之后,我委婉地请张锦雄暗中跟踪他去东海,若是凌端果然守诺也就罢了,若是他途中逃走,那么就将他杀了,想来谭忌将军也不会介意我杀了这样一个无信无义之人吧。

第十九章    将计就计

戊寅,北汉龙庭飞决沁水淹安泽,大雍齐王兵败,楚乡侯江哲败走乡里,遇玄于野,时玄沉疴在身,哲乃强邀入雍军大营,施圣手起沉疴。

北汉亡后,玄奉诏觐见雍帝,帝许以厚禄高位,玄辞以忠臣不事二主,雍帝叹息良久,馈金帛田地以绾之。玄受金银而退,遂于灞上设帐授学。玄经学名家,求学者众,且不论门第,教无遗类,门人弟子遍及朝野。

时楚乡侯江哲性惫顽,每托病不入朝,且多谋善断,朝野皆畏之,然哲深畏玄。玄每登门,必严辞呵责,哲俯首无辩,时人甚异之。或谓邪不胜正之故也。

玄初为晋臣,奉帝命为太原令长史,刘胜甚重之,贞渊十四年,雍受晋禅,刘胜亦自立国主,玄叹之曰:“社稷崩坏,世无忠臣,吾不能改节而事诸侯。”悄然归乡里。后大雍得天下,以富贵招之,玄终不受,虽金银馈赠不绝于道,玄皆以助寒士读书,身故仅余赐第三进,藏书万卷,家无余财,殡葬无钱,人皆叹之。

玄以经学大家名动天下,然事东晋如一,至死不事二君,故立传于此书也。

——《东晋书·纪玄传》

送走了段凌霄和凌端,我立刻整齐人马上路了,险地不可久留,谁知道段凌霄会不会派出别的高手来截杀,再说我已经是满载而归,带回了纪玄和赵梁,让段凌霄铩羽而归,又没有留下不可冰释的深仇,此时不走,更待何时。纪玄受了惊吓,又在病中,不能乘马,我用了特制的药物让他昏睡过去,然后用村中唯一剩下的一辆破旧马车载了纪玄,赵梁则是随车侍奉,就这样赶奔齐王大营。

远远的看见中军大营旌旗密布,我心中就是一阵轻松,还没有走到营门,只见营门大开,兵马如潮水一般涌出,然后就看见齐王身着火色战袍,纵马而出,我心中一暖,不论齐王性情是如何高傲骄纵,但是待我却是始终不错,就是如今想起当初在南楚的时候,他总是有意无意戏弄于我的情景,也是觉得有趣胜过气恼,这样一个铁骨铮铮的男儿,我断然不容别人冤屈陷害了他。

齐王纵马过来,我则是缓缓骑马上前,小顺子早已下马避开,反正只要不纵马飞驰,我也不会掉下去的。两骑相近不到数丈,齐王策马停住,凝神看了我半天,才大笑道:“好,好,看来你跑得是很快,没有受伤,也没有吃什么苦头。”

我几乎是翻了一个白眼,说我跑得快是夸奖还是讽刺啊,没好气地道:“那是托了王爷的福,再跟王爷打上几年仗,只怕我就成了最擅逃跑的监军了。”

跟上来的众将相顾愕然,平日虽然齐王和楚乡侯总是喜欢开开玩笑,不过在大场面上还是客客气气的,想不到竟会在营门外笑谑了起来,幸而新败之后,本来就有些忧虑的将士不免担心朝廷是否会有处分,见这两人如此玩笑,倒是心放宽了些。

李显余光瞥见众将都是神情一松,心中一喜,他这些日子一来烦恼战败,二来担忧江哲安危,不免心情悒郁,结果令得军中也是气氛紧张,他今日借着迎接江哲的机会故意说上几句玩笑话,果然起了作用,军中气氛大变。他见目的达到,也不多耽搁,在马车扯着江哲披风道:“好了,我们进大帐议事吧,怎么样,路上可平安么,可有什么斩获?”

江哲让他派苏青一人回去,李显也知道江哲定是想吸引有心行刺的刺客,如今江哲平安回来,他自然想问问捕获了几个刺客,若是收获不小,江哲在大庭广众宣扬出来,也算是鼓舞士气。

我虽然明白他的心意,不过总不能说我放了段凌霄和凌端吧,于是只轻描淡写地道:“虽有几个刺客,也不是什么重要的人物,难不成我还带了人头回来么?”

说话间,我们两人已经策马走入营门,下马直入大帐,小顺子带了众侍卫去安排住处,安置纪玄和赵梁不提,呼延寿和苏青都有将职,跟着众将之后进了大帐,安泽战败之后第一次真正的军议开始了。

虽然刚刚经历了一场大败,众将不免有些颓然,但是毕竟北疆多年缠战,胜败乃是兵家常事,这次又没有伤及主力,所以众将倒也心平气和。我虽然不是军旅中人,可是对众将的心态倒也明白,虽然也为众将胜不骄、败不馁的气度心折,可是想到这是龙庭飞几年来的持续打击形成的结果,也不由心中苦笑。

李显笑道:“我军虽然在安泽大败,可是北汉军也不是没有损失,至少安泽城已经毁掉,而且段无敌所部也受了不少损伤,无家可归的流民更是十数万众,虽然北汉军将流民尽皆撤到沁源,坚壁清野,可是这么多流民,只怕北汉的粮草会消耗的极快,也不见得对他们十分有利。我军虽然落败,可是主力仍在,本王已经发书求援,只需一个月时间,水军援军就会到达,到时候我们粮道就会稳固,可以和敌军大战一场。如今敌军已经撤到沁源,那里是北汉主力所在,本王决定在沁源和龙庭飞决战,不知道众将以为如何?”

众将也都知道北汉军已经撤到沁源,若是不进攻难不成还守在这里么,自然也无异议,不过宣松心中却有忧虑,起身道:“元帅,所谓三军未动,粮草先行,虽然有水军援军,可是远水不解近渴,安泽和沁源虽然不到百里之遥,却是关山险阻,沿途山路崎岖,从陆路运输粮草消耗极大,如今军中粮草虽多可以用上半个月,后续的粮草只怕不能及时补给,不若主力暂时驻扎在安泽,派一二将领整修道路,阻截北汉军南下道路,等到援军到后再大举进攻,不知元帅以为如何?”

李显听了也知道宣松所说才是行军的正理,可是如今偏偏不能这么做,正盘算着如何措辞,我已经悠然道:“宣将军所说不错,只是我军和荆将军约定会师沁源,如今虽然不知战况如何,可是以荆将军用兵之快,只怕旬日之间就会兵抵沁源,到时候若是我大军不到,则不能成前后夹攻之势,若是被龙庭飞避重就轻先击败荆将军,那么这一战才是真得旷日持久,虽然如今粮草虽然有些困难,可是还是勉强可以支撑二十天的,至于粮道之事哲愿亲自负责,必不致令大军腹中无粮。”

宣松听了也觉有理,虽然仍然有些不安,倒是主帅和监军异口同声,他又是江哲提拔重用的将领,没有明确的理由,自然也不好反对,就这样决定了大军即日北上的战略。不知怎么,宣松偷眼看着江哲若有若无的慵懒笑容,心中泛起一种明悟,似乎有什么阴谋在展开吧,只不过自己还不够资格知道罢了。

遣走众将,李显皱眉道:“随云,我已经按照你的意思送上了求援的文书,这两日应该可以到皇上御前,可是我军不过小小挫败,为何你要我在奏折里面声称大败,并且大肆索要粮饷援军呢?”

我微微一笑,这个原因暂时还是不要告诉李显的好,散布假消息自然是引蛇出洞,不过李显还是不必知道了,这也是皇上的意思,我们都不希望李显分心旁顾,再说这种兄弟閲墙的事情参与一次已经够了,我想齐王也不想参与第二次吧。所以只是淡淡道:“这是皇上的意思,现在朝中有些人不稳,若是军情有变,这些人必定兴风作浪,与其让他们在紧要时候破坏我们的大事,不如让他们早些露出形迹,所以这次既然我们注定要败上几阵,就趁机递上报急的折子,岂不是正好,就是他们耳眼通天,也会上当受骗。”

李显心中一颤,朝中不稳,怎会如此,难道凭着二皇兄的手段还能坐不稳江山,朝中还有何人敢起波浪,秦程两家忠心耿耿,想来想去除非是自己起了反意。他心中浑没有将李康当回事,凭着东川那点人马,而且李康在军中威势远远不及李贽和自己,就是两人手下的许多大将也比李康出众。想来想去,李显还是想不出个所以然,虽然他知道皇上和江哲有过几次秘密的通信,可是他只当是皇兄不放心自己,所以江哲暗中报告军中事机罢了,既然相信江哲不会随便加害自己,所以李显只当不知,对于朝中事情他又是懒得理会,东川不稳之事又只有少数重臣知道,所以李显怎也想不出朝中有何变故。

我看出他心中疑虑,笑道:“也不是什么大风大浪,只是戾王、凤仪余孽罢了,还有人趁机攻击殿下,所以皇上不想殿下知道,免得殿下心中疑忌。”

李显听了此言倒是心中一宽,反正这些风言风语从他到泽州统军就没有断过,江哲既然这样说他也就放心了,只是悻悻道:“皇上信不信也无关紧要,只要不妨碍我攻打北汉也就罢了,等到攻下晋阳,随便皇兄将我撤职还是降罪就是。”

我暗暗苦笑,李显和皇上还是芥蒂难消,不过这个我可帮不上忙,如今能够让李显恢复昔日生气,已经是很不容易了,但是也不能不答话,心中存了些埋怨,我故意讽刺道:“哲还以为只有我一人不能看到征服南楚的壮举,想不到殿下也不想挥军南征呢?”

李显一愣,急急道:“什么,你说南征,莫非皇上已经有了这个意思?”

我奇怪地道:“这有什么,等到北汉平后,难道不用南征么,皇上的志向乃是一统天下,岂能让江南在卧榻之侧酣睡。”

李显恍然大悟,泄气地道:“原来如此,征南不知道皇兄会不会派我去,不过到时候也未必没有希望,至少可以让我带一支骑兵去攻打襄阳吧,毕竟那里我已经攻打两次了,至于南征主力,裴云希望大些,毕竟这几年他都在长江防守,还有,若是东海归降,海涛也有希望,不过随云你怎么不去呢?到时候恐怕皇上不会舍弃你这个大才不用的。”

我眼中闪过一丝无奈,道:“北疆若平,大雍基业已经巩固,灭楚不过是时间的问题,哲久已无心世事,若是皇上开恩,放我还山,我就回东海,若是皇上不愿意放我,长安也是不错的居处。哲背楚投雍,已经是有负故人,如果再率军攻楚,只怕将来无颜还乡了。”

李显不由暗骂自己糊涂,这种事情都想不明白,连忙道:“不去就不去,南楚暗弱,那里还用你出手。”

南楚暗弱,我微微冷笑,前些日子传来议和的结果,大雍已经同意南楚不再赔款,以江南的富裕,只要数年就可以恢复元气,若非南楚君昏臣暗,大雍也未必就可以轻而易举平了南楚,何况还有陆灿在,连我都在他身上吃了苦头,这个孩子可是不好对付呢。

李显觉出帐中气氛沉闷,转换话题道:“随云你这次自请督运粮草,可要小心谨慎,若是粮草跟不上,只怕你虽然是监军,也是死罪可免,活罪难逃。”

我心道,粮草不济,不过是活罪难逃,我若是也到了沁源,只怕败战之际,我就是想跑都跑不掉,还是躲在后面好些,不过这话我可不敢说,虽然齐王也认为我军还需要一败,可是在他本心,还是希望能够堂堂正正胜了北汉军的,我若是这样说了只怕他会气恼,其实我也很好奇,龙庭飞是否会按照我想的那样行事,我军胜负也在五五之间,不过最好还是落败的好,不然敌军缓缓后退,一城一城的和我们血战,只怕我军还没有攻到晋阳,李康就已经兵压长安,搞不好南楚也会趁机北上,所以若是龙庭飞真得从沁源败退,我就得重新策划战略了。

在帐内待得久了,觉得有些气闷,想着我的军帐应该已经安置好了,就和齐王告辞,走出大帐,看着昏昏暗暗的苍穹,我心中猜测着,那封告急的军情奏折是否已经到了长安,可是已经掀起了漫天的风浪。

“枕上独眠愁何状,隔窗孤月明。夜深云黯心意沉,寂寞披衣起坐数寒星。

晓来百念都成灰,剩有寂寥影。清泪滴尽梧桐雨,又闻声声更鼓摧人肠。”

长安深宫昭台阁内,一个容光绝丽的宫装女子轻抚银筝,低声吟唱这一曲幽怨悱恻的虞美人,虽然是锦衣玉食,珠围玉绕,却是孤寂无依,冷落深宫,那女子弹唱不多时,便已经是泪流满面。站在香炉旁边的秀丽侍女连忙递上丝巾,那女子用丝巾拭去眼泪,道:“婵儿,若是本宫没有远离故土,来到这不见天日的所在该有多好?”

那宫女听见主子抱怨,连忙转身走到门外,见其他的宫女都离得甚远,才回来低声道:“娘娘,不可多言,若是给人听到传了出去,对景发作起来,娘娘只怕吃罪不起,只要捱过几个月,等到皇上淡忘了那件事情,凭着娘娘的品貌才情,定可以东山再起。”

那女子闻言又是珠泪低垂,道:“想本宫也是世家之女,若是蜀国未亡,就是进了王宫也不会如此轻贱,如今被父亲送入大雍内宫,却是受此屈辱。皇上初时待我还好,一入宫就封了充仪,虽然是看在父亲的份上,可也是颇为恩宠。可是自从司马修嫒被杖杀之后,皇上迁怒我们这些东川世家送进来的宫妃,对本宫日渐疏远,前几日本宫卧病未能去向皇后请安,不知何人挑唆,皇上下诏责备本宫疏于礼仪,将本宫黜为充嫒,这本是无端的罪名,本宫想着若能消了皇上的怒气,也是值得的,可是自此之后数月都见不到皇上龙颜,就是宫中召宴,也有旨意不让本宫前去。如今这昭台阁冷落凄凉,比冷宫也不差什么,这种凄凉日子,让本宫如何煎熬,本宫倒是宁愿真得进冷宫去,等到大赦之日就可以回乡见见爹娘。”

那宫女婵儿眼中闪过一丝幽冷的光芒,口中却是劝解道:“娘娘不用烦恼,前日娘娘去给皇后请安,皇后不是暗示娘娘说,已经跟皇上进谏过了,说是皇上为了司马氏一事迁怒娘娘有失公正,或许这几日皇上就会回心转意了呢?”

那秀丽女子只是低声长叹,她出身世家,见惯种种争宠之事,怎相信皇后会替自己出面。主仆二人说一阵,哭一阵,正在肝肠寸断的时节,伺候昭台阁的内侍兴冲冲地奔了进来,在门外跪倒禀道:“娘娘大喜,皇上有旨,今夜留宿昭台阁,宋公公前来传旨,请娘娘准备接驾。”

那女子大喜,站起身来娇躯摇摇欲坠,低声问道:“婵儿,本宫没有听错吧?”

那宫女面上露出喜悦的神色,下拜道:“恭喜娘娘,奴婢早说皇上乃是英明圣主,必不会迁怒娘娘的。”

那女子连忙道:“婵儿,快陪本宫去接旨。” 说着接过那宫女刚刚用清水洗过的丝帕,胡乱拭去脸上的泪痕,匆匆走出去接旨。在昭台阁正殿之内,一个十七八岁的青衣太监正肃然而立,他就是皇上身边的亲侍宋晚。这个宋晚其实年纪不大,只有十七八岁的模样,相貌端正朴实,一副老实巴交的模样,但是只要想到他能够李贽登基之后不到两年之内,从一个原本根本见不到龙颜的洒扫太监成了皇上身边的红人,就知道此人绝不简单,更难得的是,这个宋晚性子沉稳端重,虽然受皇上宠爱已不在总管太监常恩之下,却是谨慎小心,绝不轻易得罪人,所以在宫中人缘极好。

宋晚见到黄充嫒走了出来,他恭恭敬敬地传了旨意,就要告退,对黄充嫒仍然有些杂乱的妆扮更是视而不见。黄充嫒虽然十分欣喜,却不敢失了礼数,接旨之后亲自送他出去,一边送着一边从腰间取下一块无暇美玉塞了过去,口中道:“公公乃是皇上近侍,劳烦公公亲来传旨,本宫心中感激,没有什么好东西,这块玉佩送给公公闲暇的时候赏玩。”宋晚接过玉佩,面上满是敦厚的笑容,黄充嫒这才心满意足的停住了脚步。那宫女婵儿却在旁边看得清楚,那宋晚眼神清澈,全然没有贪婪神色,心知,这宋晚眼光高得很,娘娘的玉佩也没有被他过分看重,不过她心中有数,宋晚近在帝侧,平日想要讨好他的人数不胜数,娘娘本心也不指望能够收买此人,只要他不作梗就已心满意足了。

当夜,李贽果然驾幸昭台阁,这位充嫒娘娘名唤黄璃,乃是东川黄氏的贵女,东川第一望族司马氏,排名仅在司马氏之下的就是黄氏,所以黄璃入宫之后就封了充仪,她相貌不如司马修嫒,但是擅于弹筝,通诗文,性情柔顺,所以宠幸不在司马修嫒之下,怎料一场大变,司马修嫒先被禁冷宫,后被宁国长乐长公主杖杀,黄璃也遭到皇上迁怒,降了品秩不说,还数月未蒙召见。她虽然性情柔顺,但是贵族女子的脾气还是有的,不免心中生怨。想不到皇后果然进了谏言,不过两日就蒙皇上召见,黄璃不由喜上眉梢,这一夜小心翼翼,唯恐服侍的不周到讨好,李贽似乎也心有歉疚,也是倍加温存,云雨过后,黄璃伺候着李贽用了汤浴,两人才相拥而眠。

四更天时,在外面值夜的宋晚突然匆匆走进寝宫,走到床前低声唤道:“皇上,皇上。”

李贽从梦中惊醒,坐起身道:“发生了什么事?让你这时候唤醒朕。”

宋晚低声道:“皇上吩咐过,若是有北疆紧急军报,不论何时都要立刻报知,方才是六王爷的八百里急报,我军在安泽大败。”

李贽听到此处已经是出了一身冷汗,连忙起身下床,披上长袍,接过宋晚递过来的军报走到银灯前仔细地看了起来,越看神色越是沉重,良久才道:“败已败了,如今也只能亡羊补牢,立刻召秦彝、郑瑕、石彧到文华殿议事。”说罢在宋晚服侍下匆匆穿上龙袍,正要踏出房门,李贽突然想起了什么,回身看向低垂的锦帐。他的神色有些不安,后悔地说道道:“哎呀,朕一时慌乱,竟忘了这不是乾清宫了。”说罢转身回到榻前,低声唤道:“爱妃,爱妃。”叫了几声,见黄璃仍然沉睡未醒,这才松了口气,道:“下次有事情的时候,若是有宫妃侍寝,记得提醒朕一声,尤其黄充嫒是蜀人。”说到这里,声音有些冰冷,宋晚连忙惶恐的谢罪,两人轻手轻脚地走了出去。

当李贽的身影消失在门外,黄璃睁开了眼睛,此刻她额头上满是冷汗,方才宋晚进来的时候她已经醒了,可是听到军机大事,聪颖的她连忙装作熟睡,幸而如此,否则只怕李贽会立刻将她软禁起来了,说不定打入冷宫都有可能,想到君恩薄如纸,黄璃不由暗暗饮泣。这时,宫女蝉儿走了进来,婵儿是她入宫时带来的侍女,一向忠心不二,所以黄充嫒也不瞒她,唤她过来将事情说了一遍,流泪道:“婵儿,皇上如此猜忌,本宫该如何是好?”那宫女婉言劝解道:“娘娘,天长日久,只要皇上知道娘娘的心意,就不会猜忌娘娘了。”黄璃仍是流泪不止,直到天色将明才昏昏睡去。她一睡去,那蝉儿眼中显出冰寒冷厉之色,趁着宫中宫女内侍忙忙碌碌的混乱,她径自走向御膳房,假意说黄充嫒想吃几道家乡的菜肴,和膳房交待之后,便回昭台阁去了,谁也没有留意,她塞给膳房一个老太监一个纸卷。

接下来几日,前方兵败之事被李贽君臣掩盖的严严实实,几乎是滴水不漏,长安城中都没有一丝风声,只是李贽秘密地调兵遣将,让一些有心人看在了眼里。而与此同时,透过不为人知的秘密渠道,安泽败战的消息已经传到了东川庆王耳里。李康正在焦急地等待时机,见到北疆兵败的情报心中不由大喜,可是小心谨慎的他没有立刻出兵,毕竟根据他多方收集到的情报,这次兵败并没有伤筋动骨。不过他趁机考验了一把锦绣盟的忠心和能力,就是要求锦绣盟调查这次兵败的详情。数日之后,锦绣盟呈上的情报让庆王十分满意,不仅将这次兵败的前后经过说得清清楚楚,而且还有一些就连李康也未得知的细节都查了出来。霍义禀明那些情报是锦绣盟透过在长安的暗探侦侧到的蛛丝马迹归纳出来的,毕竟齐王的大军将北疆隔绝得十分严密,根本无法潜进去探察军情。而李康另外从北汉魔宗得来了一份详细情报,两相对照,只怕世间没有人比他更了结安泽败战的详情了。李康更是证实了锦绣盟的能力和忠诚,也渐渐将重要的权力交给锦绣盟,毕竟在探查情报上面,锦绣盟有着绝对的优势和能力。

第二十章    惊闻密辛

放下手中的情报,李康满意的看向霍义,这个相貌平常,神态憨厚的普通青年虽然看上去只是一个没有心机的老实人,但是谁能够想到他乃是锦绣盟数一数二的人物呢,这些日子跟在李康身边,替李康办了不少事情,清除了不少倾向朝廷的官员,虽然对锦绣盟仍然有些提防,可是对于霍义,李康却已经是颇为信任了。

霍义,或者应该是白义,恭恭敬敬的站在下首,见李康已经看完情报,才说道:“殿下,属下已经得到消息,夏侯沅峰可能已经到了散关,这些日子,殿下拦截朝廷的钦使和文书,又以有盗贼出没为理由将散关通向东川的道路封锁,虽然表面上没有什么破绽,大雍朝廷忙着和北汉作战,对东川不免懈怠,可是李贽和他手下的臣子都不是等闲人,他们已经发现了端倪,若非不想在这个时候迫反王爷,只怕雍军已经入川了。不过夏侯沅峰已经亲自出手,近日本盟在散关之外抓住了十几个明鉴司的秘谍,不知道王爷准备何时动手,事不宜迟,若是等到大雍朝廷腾出手来,只怕我们就没有机会了。”

李康笑道:“你不用着急,现在李贽万万不敢和我翻脸的,而且我虽然摆出拥兵自重的格局,可是在李贽看来我最多不过是争权夺势,谁会想到我一个堂堂的大雍亲王会存心让大雍四分五裂呢?所以朝廷一定是尽量安抚,李贽连下几道诏书,嘉勉本王,不就是不想让本王明目张胆和朝廷作对么,他是想等到平灭北汉之后,挟着大胜余威再来对付我的,夏侯沅峰若是不来本王才觉得奇怪呢?不过现在时机还没有到,李显初败,力量还没有大损,凭着龙庭飞的本事,又占着地利人和,一定可以让李显遭遇惨败,等到那时我再出手不迟。”

霍义犹疑地道:“可是和北汉交手的是齐王李显,他乃是天下有数的名将,又有楚乡侯江哲辅佐,若是落败的是北汉可怎么办呢?”

李康摇头道:“江哲就是再聪明又如何,龙庭飞就是不能抵挡,只要一城一城的退守,就可以将齐王牵绊住,到时候久战不胜,我再收买朝中大臣进言,指责齐王空耗兵力,到时候内忧外患,说不得李贽得焦头烂额,别看大雍和南楚新近达成和议,到了那时,就是尚维钧再白痴也会落井下石的,其实我若是李贽,最要紧的不是攻北汉,而是先稳定东川才是,攘外必先安内,这是至理。”

霍义神思道:“或许大雍朝廷也是迫不得已,现在南楚观望,王爷虽有反意,但是却未昭彰,李贽想必是希望以迅雷不及掩耳之势先平北汉,到时候就可以从容对付我们了,只是他们没有料到经过泽州大败的北汉军还有这样的战力吧。”

李康点点头,道:“你们小心在意,我们发动的时机可是很要紧的,对了,在散关之外设下重重埋伏,绝不能让明鉴司的探子混入东川。”

霍义胸有成竹地道:“殿下放心,本盟马护法亲自坐镇,绝对不会让明鉴司得逞的。”李康微微含笑点头,他也有自己的心腹,自然知道在散关之外,锦绣盟已经或杀或擒了不少朝廷的密探,手段十分残恨激烈,自身也损失不小,可见锦绣盟的诚意和忠心。

告退之后,走到殿外,霍义的嘴角露出淡淡的浅笑,在外面等候他的是两个青年,一个温文儒雅,一个勇猛彪悍,都是二十五六岁的年纪,这两人正是上官彦和熊暴,他们面上神情十分冷漠,前些日子,他们被申斥之后就以戴罪立功的名义跟着霍义进了庆王府,虽然慑于淫威,这两个青年对霍义十分恭敬,丝毫不敢得罪,毕竟霍义是霍纪城义子,陈稹心腹,而他们的长辈家人还在锦绣盟手中,可是心中的排斥却是有增无减,即使霍义始终对他们客客气气也改变不了他们的心情。霍义见到他们的神情,心中微微一叹,只装作未见,吩咐道:“传信给马护法,加强对散关的监控,绝对不能让一个大雍秘谍混入东川。”

锦绣盟负责在散关之外阻截明鉴司秘谍的主事人马成今年四十多岁,乃是志切复国的中坚分子,这次陈稹特意派了他主持此事,就是因为他对大雍仇恨极深,而交给他的人手也都是锦绣盟中有数的好手,当然这些人都有一个特点,就是对于和大雍作对十分热衷,因为前几年锦绣盟韬光养晦而颇有不满,这次让他们出手,就像是猛虎出笼,所以这段时间他们成绩斐然。

在散关有两个人虽然也参与了这次行动,却是没有一点成就,其中一个是顾英,乃是锦绣盟大护法顾宁独子,前些日子锦绣盟主霍纪城决定和庆王合作,顾宁因为触怒霍纪城,被削去仅剩无几的权力,让锦绣盟众人再次见识了盟主排除异己的厉害手段。顾宁担忧自己的处境,就拜托好友马成照顾自己的独子,马成虽然也不是霍纪城的嫡系,可是素来更受霍纪城和陈稹器重,有他保护顾英,顾宁才能放下心来。而马成为了顾英的安危着想,即使接下了这样重要的任务,仍然将顾英带来散关,只是不许他出手罢了,毕竟顾英虽然武功不错,却只有十七岁而已。所以顾英只能看着别人动手。

而另一个人就不同了,他叫洛剑飞,乃是陈稹的心腹卫士。说起来,自从陈稹主管锦绣盟日常事务之后,盟中老人大半权力旁落,如今最受陈稹重用的就是盟主的义子霍义、霍山。霍义精明能干,武功高强,霍山精通机关消息,最善布局伏杀,这两人年纪虽轻,却是手握大权,杀伐决断,盟中众人无不敬畏。据说盟主还有一位义子霍离,曾经立下天大的功劳,如今已经销声匿迹,有传言说已经英年早逝,却是无人敢追究。除此之外,陈稹身边有一支神秘的卫队,这只卫队由一些年纪相仿的卫队组成,每一个卫士都是文武双全的俊杰,他们人数不定,行踪隐秘,除了陈稹之外恐怕没有人能够弄清楚他们的实力和编制,一旦盟中有大事发生,这些卫士常常是主事之人,所以无人敢轻视他们。盟中早有传言,这些卫士和霍义、霍山年纪相仿,气度相近,恐怕都是霍纪城亲自调教的,多半是霍纪城为了掌握盟中事务而派在陈稹身边的亲信耳目,而这个洛剑飞就是其中之一。

洛剑飞算是锦绣盟盟友较为熟悉的一个卫士,多次参与重要事务,和马成合作多次,此人相貌文秀,却是心狠手辣,有他出现的地方经常是血流成河,此人不仅对敌人狠辣,就是对自己人也是十分辣手,除了陈稹的之外绝对不听从别人的命令,就是霍义和霍山也不敢随便指挥他,这次陈稹派他来散关,就是想借助他的狠辣手段。马成隐隐知道他手中握着陈稹的密令,可以随时接管自己的权力,也就把他当成监军看待,更是不敢丝毫得罪,这人脾气古怪,白天就在秘舵中蒙头大睡,到了晚上就单人独剑到外面行走,几次回来的时候身上都带着血腥之气,甚至带了伤痕,可是却没有人看见他的俘虏,就连人头也没有一个,让人不知道他的战果如何。

要知道锦绣盟布下防线拦截散关出来的秘谍,毕竟是不容易的事情,大雍秘谍的身手都很不错,而且潜踪匿迹的本领也否出类拔萃。若是白日还好,只要派出眼线在高处仔细留心,就可以发现他们的行踪,在使用各种手段传信通知盟中高手截杀。若是晚上,视线不广,就只能派出高手在一些要道守株待兔,反正后面还有一道防线,那些秘谍就是过了这一关也不是那么容易混进东川的。不过晚上参与猎杀的多半是几人一组,只有洛剑飞喜欢一个人独来独往。

这一天晚上,月暗星沉,顾英悄悄离开了宿营地,跟在洛剑飞身后想看看他晚上都作些什么?他知道自己武功不如洛剑飞,所以远远的跟着,幸好洛剑飞并没有特意掩饰身形,所以顾英居然一路跟着洛剑飞到了一座山谷。这座山谷乃是从散关到东川的一条小路,因为路途崎岖,所以少有行人,却是秘谍来往的要道。若是白日,在山崖上俯瞰山谷,无人可以隐踪,若是晚上,则是漆黑一片什么也看不见,而山谷中没有合适的藏身之地,所以并不是合适的阻截地点,顾英心中奇怪,洛剑飞为什么选了这个地方,疑窦渐生。在他的注视下,洛剑飞登上两侧山崖,完全没有留在谷中潜伏等待猎杀机会的意思。

顾英犹豫了一下,也暗中跟着上了山崖,山崖顶上乃是一片竹林,竹林环绕着一座破旧的山神庙,山神庙之后有一块突出山崖的平坦巨石,顾英深知这里的地形,当初他是跟着马成到这里巡视过的。远远看见破庙中亮起了火光,在黑暗中一闪而逝,顾英知道洛剑飞是点燃了篝火,然后关闭了庙门,遮住了外泄的火光。便壮着胆子潜上山崖,绕到破庙后面,想看一下有没有机会进去,却又不会被洛剑飞看见。月光虽然昏暗,可是顾英还是隐隐约约能够看见眼前的景物,不多时,他发现墙角一丛乱草后面的墙壁似乎破了一个大洞,他无声无息地拨开那些枯草,那个大洞勉强可以让他钻进去,他轻手轻脚钻了进去,眼前一片黑暗,看不到火光,根据位置,他大致可以猜测那是供山神的供桌,至于看不到火光,看来是铺在供桌上面的锦幔仍然没有被偷走。蜷缩在狭小的空间,顾英一动也不敢动。

其实洛剑飞武功高强,原本不会这么没有防备,路上没有发觉一来是因为顾英小心,再加上他心切和人见面,所以没有留心,毕竟锦绣盟没人有胆子敢针对他,谁会想到顾英会初生牛犊不怕虎呢?等到他进了破庙之后,便忙着点燃篝火,清扫殿堂,顾英潜入供桌之下的时候,正是洛剑飞出去寻找干柴的时候,所以阴差阳错,就让顾英潜到了这个所在。顾英虽然年轻,但是武功是内家真传,洛剑飞虽然武功高强,终究只是一流,不能明察秋毫,所以竟没有发现顾英的存在。

轻轻将布幔露出一条小缝,顾英仔细看着明灭的火光和那个面色阴沉冰冷的黑衣少年。洛剑飞盘膝坐在火边,正在闭目养神,虽然年青俊秀,可是那种阴森的神情和周身上下流露出的淡淡杀气让他充满了威慑力,虽然火光照亮了他仿若刀削斧刻一般的俊秀面容,可是给人的感觉却是他随时都会消失在被火光驱散的黑暗当中。顾英想起马护法曾经对他说过,这个洛剑飞十有八九做过杀手,此刻他真的明白了马伯父的意思,这样的杀气,这样的阴暗,不是杀手才怪呢?

正在顾英觉得四肢有些麻痹的时候,突然庙外传来轻微的脚步声,顾英心中一惊,更是屏住了呼吸。庙门被推开了,寒风涌入,顾英打了一个冷战,只见庙门处站了一个高挑的身影,那人披着灰色的大氅,头上戴着遮阳斗笠,压得很低,看不清他的相貌,只见他左侧腰间露出剑柄,便知这人也是一个江湖人物。那人站在庙门前,静立片刻,伸手轻轻摘去斗笠,露出一张清秀含笑的面容,明亮如同夜空的寒星的眼睛隐隐带着泪光,定定地看着神色自若的洛剑飞,似是见到多年不久的亲人一般激动。

顾英心中一宽,心道,莫非是洛剑飞的故人,所以他没有明言,毕竟他虽然不喜欢陈稹一系的人,却也不想内讧。谁知刚刚送了口气,却见剑光一闪,那灰衣少年竟然合身扑上,大氅挥舞,带着巨大的风浪,将那篝火生生扑灭,顾英只觉眼前一黑,然后耳边传来兵器撞击的声音,顾英按住剑柄,侧耳细听,眼前漆黑一片,外面星月无光,他只能听着殿中两人苦战,更何况他是暗暗跟着洛剑飞来的,就是洛剑飞身死,他也不敢轻易出手的。过了片刻,顾英的眼睛渐渐适应了黑暗,透过帷幕缝隙,已经能够隐隐看见两人在大殿上激斗。这两人似乎都善于在黑暗中苦斗,剑气纵横,仿佛在白昼一般挥洒自如,顾英只能看见剑光和两人隐隐约约的身影。可是他却分辨不出那个洛剑飞,那后来的陌生青年不知何时已经丢下了大氅,两人都是劲装打扮,身材也是仿佛,就连剑法武功也有许多相似之处,倒像是一师之徒在那里较技,可是顾英分明觉得这两人都是凶猛绝伦,丝毫没有一丝留情之处。两人大概斗了百十招,其中一人稳稳占了上风,另一人却是只有招架之功,顾英心中忧虑,不知获胜的到底是谁。

这时那个落在下风的人飞身后退,笑道:“罢了,我服气了,你这几年武功进步的很快,想必是又得了李爷的真传吧?”顾英听这人声音陌生,知道是洛剑飞占了上风,心中一宽的同时,也不由生出疑虑,看来这两人果然是旧识,可是为何要在黑暗中交手,又是如此出手无情?

这时,火光衣衫,那个灰衣人点亮了火折子,将篝火点燃,随手捡起丢在地上的大氅披上,洛剑飞则是坐回原先的位置,示意那人坐在他身边,冰冷的面容上露出温暖的笑意道:“骅骝,多年不见,如今你已经是位高权重,想不到还记着我们这些故人?”

那个灰衣人叹息道:“若非是命运捉弄,我倒还想和你们一样在公子手下效力,如今赤骥在北疆为公子效力,盗骊在东海经营,绿耳的生意遍及天下,白义、山子在蜀中,逾轮、渠黄在南楚,其他的兄弟不论在哪里,也都是在公子羽翼之下,只有我,虽然做了官,近在帝侧,却是帮不上公子的忙,唉!”

洛剑飞微笑道:“你说什么呢,当初如果不是你帮着公子控制住了秦将军,只怕太子已经做了皇帝,现在你在明鉴司跟着夏侯沅峰,也是很重要的,若是夏侯沅峰有心对公子不利,你也可以即时发现么,李爷说过,若是皇上要杀公子,夏侯沅峰必定是最先知道,所以只要你盯住了夏侯,就等于盯住了皇上。再说,如今你舍得下你的义母和义兄么?”

顾英虽然见识不广,听到此处也是心中剧震,这个洛剑飞乃是锦绣盟的中坚,想不到竟然是大雍的秘谍,顾英心中当然不会想到陈稹也是其中一党,只想着如果将这件事情揭穿,那么陈稹就要无地自容,忍不住唇边露出笑意,继而又想到,这人在锦绣盟中卧底,恐怕不知放了多少大雍明鉴司的秘谍进去,可要快些禀报马护法才是。他毕竟年轻,心中焦虑非常,不知不觉间就连呼吸也重了几分。幸而那两人凑在一起低语,神情专注,似是没有察觉,顾英连忙又放轻呼吸,努力倾听。但是那两人声音很低,顾英只能隐隐约约听见一些零碎的断句,只是这两人不时提到“庆王”和“锦绣盟”的字眼。良久,那两人终于停止交谈,相视一笑,洛剑飞起身道:“好了,事情已经谈完了,你回去吧,一会儿若是天亮了,就不好行动了。”

那个灰衣人似乎犹豫了一下,道:“有一件事情,是夏侯大人托我转告的,他希望公子能够考虑一下,东川平后,将锦绣盟交到他手上。”

洛剑飞的动作似乎僵住了,半晌,他冷冷道:“夏侯大人是什么意思,锦绣盟是公子的利器,岂能随便给人,更何况我们凭什么让夏侯沅峰占这个便宜?”

灰衣人叹了口气,道:“夏侯大人说,普天之下,莫非王土,东川也是大雍版图,锦绣盟无论如何也是叛逆,他执掌明鉴司,不能容忍有这样的势力在朝廷掌握之外。而且公子如今已经封侯,将来还要步步高升,这些不光明的事情还是交给他比较好,如果锦绣盟还有存在的价值,那么也该由夏侯大人掌控。”

洛剑飞冷笑了几声,道:“你倒是大言不惭,你应该清楚,锦绣盟是怎么回事,如果是两年前,公子要将锦绣盟交出去,我绝对赞成,可是现在锦绣盟掌控着我们在东川和西蜀五成的生意,而且在南楚和天机阁、凤仪门余孽合作,锦绣盟对公子的重要你应该很清楚,这一次为了大雍,公子将牺牲锦绣盟七成以上的实力,想不到夏侯沅峰如此贪心,竟连剩下的三成也不放过,你竟然也替他说话,骅骝,你还记得是谁让你有了今日的荣华富贵么?”

灰衣人清秀的面容再也没有一丝笑意,他举起右手道:“我对天立誓,若有对不起公子的恶意,就让我死于非命,尸骨不全。”

洛剑飞听了他这番话,神情有些缓和,但是仍然带着怒气,道:“那好,我听你解释,你为何替夏侯沅峰说话?”

骅骝叹息道:“我刚听到夏侯大人这样说,也曾出言相责,可是夏侯大人说,从前东川在庆王掌握当中,所以公子掌握锦绣盟并没有不妥,可是东川平定之后,公子若再和有意复国的锦绣盟关系密切,只怕皇上那里也会多心。公子才华乃是天授,手中势力庞大,若说自保,未免太过,若是公子肯主动交出锦绣盟,那么一来表白忠心,二来也可和叛逆撇清关系,比起微不足道的损失来说,得到皇上的衷心信任,去除可能遭受猜疑的力量,并无不妥。我也觉得夏侯大人说得有礼,而且即使放弃锦绣盟,公子还有足够的实力自保,我们也可从锦绣盟脱身出来,集中力量卫护公子,所以我希望你能向陈爷他们说明此事,然后转呈公子知道,夏侯大人不想直接和公子商谈,这样若是不成,也不会生出嫌隙,你想我说得可对?”

洛剑飞神色数变,良久才道:“我会向陈爷说明此事,不过最终如何决定,还是要看公子的意思。”

骅骝道:“若是公子不同意,我将全力劝阻夏侯大人。”

洛剑飞微微点头,转身出了庙门,不多时,那个灰衣人也跟了出去。顾英这才发现自己几乎忘记了呼吸,这怎么可能,锦绣盟原来不过是别人的棋子,那个他们所说的公子不知何等身份,一手掌控着锦绣盟,却又和大雍明鉴司有瓜葛。顾英对时势不甚了解,若是换了他父亲或者义兄上官彦,必定能够猜到几分,他却是懵懂不知自己听到的事情乃是何等的骇人听闻。过了片刻,他估计那两人都应该已经走远,这才钻出供桌,准备回到马成身边向他说明今日所听到的密辛。谁知道他刚走出庙门,便觉得背心一麻,扑通跌倒在地,然后有人用足踏在他背上道:“果然我没有听错,庙中有人潜伏,剑飞,这人你可认得?”顾英只觉的浑身冰冷,他不是畏惧死亡,像他这种年纪,若是再大了几岁,领略过人生的种种乐事,或者会贪生畏死,可是如今正是年轻气盛,血气方刚的时候,最容易轻抛生死,他担心的却是父亲和其他叔伯亲人,自己这一死,只怕他们再也没有机会逃出生天。那个灰衣人一脚将他掀翻过来,顾英那张苍白的面孔落在洛剑飞眼中,他的瞳孔突然收缩,右手按上了剑柄。但是很快,洛剑飞的面上飘过挣扎的神情,那长剑,终究是没有拔出。

\chapter{第二十一章 兵出壶关}

壶关,乃是北汉扼守太行白陉的雄关,从镇州穿过白陉进入北汉疆界,群山环绕,而壶关正是咽喉要塞,其北有百谷山,其南有双龙山,两山夹峙,以壶口为关而得名,攻破壶关,雍军就可以长驱直入,而这一次雍军兵分两路,负责攻壶关的正是荆迟,这次他带了三万骑兵,再加上四万镇州军,从三月十四日猛攻壶关,守将刘万利也是有名的将领,带着七千守军坚守不退,雍军连攻七八日,却是难以攻破壶关。

三月二十一日,荆迟策马立在将旗之下,目光冰冷地望着那几乎被鲜血染红的城头,有些干裂的嘴唇显露出他内心的焦急,这一次军令很清楚,他必须攻破壶关,经上党至沁源,和齐王殿下会师,前后夹攻北汉军主力,北汉军兵力不足,只能扼守少数要塞,只要破了壶关,前面就是不设防的广大疆土。可是已经整整八天了,壶关在雍军的攻击下仍然屹立不倒,荆迟心中如同火焚一般,恨不得亲自上战场,可是骑兵若是用来攻城也未免太浪费了,齐王的意思很明白,镇州军攻城,而自己的骑兵是要千里奔袭的,万万不能在壶关损失太大,抬头看看天空,夕阳已经落到壶关城后,映照得城楼一片血红,他狠狠地道:“收兵。”然后策马回营,一定要想出办法,最多两日,若是再不能破城,哪怕就是自己亲自冲锋,也要踏上壶关的城楼。

三月二十二日,指挥攻城的镇州军主将林崖站在指挥作战的三丈高台之上,神色间满是忧思,这些日子冲车、弩车、云梯、投石机不知已经毁去了多少,壶关城下一片狼藉,护城河早已经被填平了,就是城门也早被雍军用桐油烧得稀烂,只是里面却被北汉军用石头砖木堵死,若是再不能破城,只怕贻误军机。只可惜那刘万利心狠手辣,一得知雍军即将攻壶关,就将壶关的所有青壮男子全部编成甲伍,相助攻城,采用连坐之法,令那些青壮彼此监视,大雍在壶关虽然有些潜伏许久的密谍,却始终没有机会里应外合攻破壶关,若非是其中有几个精明能干的利用丢滚木檑石林的机会丢下写着军情的木简,只怕现在都不知城中虚实。即使如此,壶关城墙坚固,两侧又有山峰相护,刘万利在两山之上各自立寨,三处互相支援,雍军损失惨重,却是不能得逞。今日林崖狠下心肠,将手下精兵良将全部派了上去,眼看着一架架云梯在烈火中倾倒,军中勇武的将士的鲜血涂满了壶关的外墙,纵然是身经百战,林崖也是太阳穴上青筋挑动,怒火丛生。

林崖正在指挥作战,突然感觉到脚下的木台颤动起来,不由向下望去,只见荆迟战袍左坦,散发披肩,双手抱着一具一人高的战鼓向上走来,走到台上,荆迟将战鼓立起,大声喝道:“取鼓槌来。”一个跟在荆迟后面上来的亲卫连忙将两个缠着红绸的鼓槌递给荆迟。荆迟大喝一声,舞动鼓槌,用力击起战鼓来,鼓声响彻云霄,如同天边连绵不绝的惊雷一般在整个战场轰鸣回旋。泽州大战之后,荆迟听说江哲击鼓助雍军大胜,就缠着江哲学习击鼓,江哲左右闲着无事,就教了他几日,虽然他不懂什么音律,可是他久经沙场,又是武将出身,他所击出来的鼓声虽然没有那么千变万化,却是更加威猛豪壮,鼓舞人心。镇州军听见那令人热血澎湃的鼓声,又得知击鼓的乃是荆迟之后,心中又是激动又是羞愧,都大叫道:“我等奉命攻壶关,好让荆将军去攻打北汉,可是如今我们苦战不下,让荆将军在这里苦苦相候,如今荆将军亲自为我们擂鼓助阵,如果我们不能破城,只怕这一辈子都在荆将军面前抬不起头来,镇州军的脸面也要被我们丢尽了。”镇州军将士彼此激励,这一轮攻城如火如荼,壶关也几乎在鼓声中动摇颤抖,天空中阴云密布,仿佛不忍见这地面上的血腥苦战。

站在城楼上面的刘万利满面灰土,他的眼中满是冰寒,没有援军,因为北汉军主力正在和雍军泽州大营苦战,其余的兵力不是在晋阳,就是在代州,晋阳不可以轻易调兵,而代州,刘万利呻吟了一声,当初林远霆归降之时,曾经和北汉主有过协议,代州军绝不出境,这或许是因为先主不希望强大的代州军影响北汉的政局,但是林远霆却是很高兴的答应了,他声称,代州军是为了保卫乡土,不是为了同室操戈,所以这些年来,代州军从来没有越出代州一步,当然出雁门攻打蛮人是不算的。所以防守壶关只有自己靠这支军队,可是整整八天了,刘万利很清楚壶关已经几乎快崩溃了,可是雍军却仍然是漫无边际,这一战,自己是有败无胜了。

刘万利的副将走了过来,他的嘴唇上面全是火泡,声音嘶哑地道:“将军,敌军又上来了,这一次他们推了四辆云车上来,恐怕是势在必得。”

刘万利轻轻一叹,壶关地势狭窄,一般来说,使用三辆云车恰好,若是使用四辆,不免太过集中,损失会更惨重些,但是相对的,对于己方的压力也会大一些,前几日,雍军一直不紧不慢得攻城,甚至每次只使用了两辆云车。他深吸了一口气道:“用火攻。”

副将得命,传下令去,为了能够多守几日,刘万利早就下令得等到敌军靠近再攻击,那几辆云车被镇州军退到关外之时,副将一声令下,守城的北汉军将收集的柴草打成捆,上面洒了油,用投石机抛到云车之上,然后用火箭射到上面,云车上面立刻火焰熊熊,这样雍军就不能攀到上面向城*箭。这时,城下的雍军却和往常不同,没有尽量攀上云车放箭,而是用力将云车推倒,四辆云车倾倒在城墙上,搭了一个斜坡,这时候,城下号角齐鸣,镇州军左右分开,一支五百人左右的雍军骑兵纵马奔上,铁蹄下灰尘滚滚,烟火四溅,竟然踏着倾倒的云车向城墙上面冲去。刘万利大声喝道:“射箭,射箭。”这时候北汉军也顾不得节省箭支,不要命地向雍军铁骑射去,这时候,雍军冲在最前面的一个将领已经大笑着冲上了城楼,碗口大的马蹄将两个北汉军踏在脚下,那将领手中的马槊挥舞,血光崩现,然后越来越多的雍军登上了城楼,壶关将破,刘万利心中浮现出四个大字,他几乎是有些绝望了,但是北汉人彪悍的血液让他几乎燃烧了起来,秘密传下军令之后,他指挥着城上守军拼命抵挡了一刻,然后大声喝道:“后退,后退,让他们上来。”此刻他颜面染血,彷佛恶鬼一般,城头的守军虽然迷惑,可是被他震慑,都是下意识地闪躲开来,还剩下四百余人的雍军骑兵几乎全部登上了壶关城楼,可是就在他们欣喜雀跃的时候,刘万利高声喝道:“放弩。”

接二连三的机簧声响起,五六十支乌黑的弩箭射入了雍军,几乎每一支弩箭都穿透了一批战马或者一个雍军骑士的身躯,狭窄的城头让骑兵无法散开,在退开的北汉军之后露出了三十多架神臂弩,这种弩是用来守城的,每支弩箭有四尺长,每次可以射出两支弩箭,却需要三个士兵协同使用,因为这种弩威力极大,百丈之内可以穿透铁甲,所以是最厉害的震关之宝,因为容易损坏,所以刘万利一直忍着没有使用,希望可以在最危险的时候出其不意占据上风,如今就是生死存亡之际,所以刘万利才会放雍军铁骑登城,然后暗中调了弩兵出来。现在弩弓大展神威,三轮攒射之后,雍军已经是伤亡惨重,这时候北汉军趁机合围,将滚热的沸油从云车上面倒了下去,将跟上来的镇州军逼退。云车终于在大火中燃烧殆尽,于是,城下数万的雍军只能眼睁睁看着登上壶关的铁骑被北汉军从容围歼,当真是肝肠寸断,壶关之上杀伐声渐渐减弱,突然一个嘶哑高亢的声音在城头高声唱道:“操吾戈兮披犀甲,车错毂兮短兵接。旌蔽日兮敌若云,矢交坠兮土争先。凌余阵兮躐余行,左骖殪兮右刃伤。霾两轮兮絷四马,援玉枹兮击鸣鼓。天时怼兮——”刚唱道此处,歌声突然断绝,城下雍军都是大恸。

荆迟丢下鼓槌,大踏步走下台去,取了自己的战马,也不穿衣甲,策马奔到壶关城下,望着关上泪水滚滚,这时候攻城的镇州军垂头丧气地缓缓败退,荆迟突然仰天高歌道:“天时怼兮威灵怒,严杀尽兮弃原野。出不入兮往不反,平原忽兮路超远。带长剑兮挟秦弓,首身离兮心不惩。诚既勇兮又以武,终刚强兮不可凌。身既死兮神以灵,魂魄毅兮为鬼雄。”雍军先是相顾愕然,然后便有将士跟着唱了起来,一传十,十传百,歌声越来越高,响彻苍穹,一种悲壮慷慨的气氛在雍军中高涨,歌声越来越响,唱了一遍又一遍,雍军再没有战败的气馁和悲观,烈焰一般的信心和杀气凝聚成了无坚不摧的锐气。

这一曲《国殇》乃是无人不知的战歌,不论是雍军、北汉军都是耳熟能详,就是不识字的也能硬记下来,城下雍军气势大振,北汉军也是心有戚戚焉,一时之间居然有些神色如土,眼看着雍军如此强势,想到战败之后的结果,都是心惊胆战。刘万利站在关上,一掌拍在城墙上,心道,好一个荆迟,竟然在失败之后用这种方式鼓舞士气,眼中寒光一闪,他低声道:“取我弓箭来。”一个亲卫连忙递上刘万利的铜胎弓,刘万利乃是骑射高手,可开五石强弓,五百步之内取人性命如同探囊取物,只是他腰间曾经受过重伤,力气不能持久,所以久已不能亲自上阵,如今他见荆迟赤膀上阵,心中动了杀机,尤恐他人箭法不如,乃亲自引弓。

荆迟一曲高歌,意犹未尽,指着城头高声喝骂,连日来的怒火让他恨不得将壶关守将生吞活剥,就在这时,一道几乎肉眼看不见的淡淡虚影从壶关城头射向荆迟,荆迟乃是雍军数一数二的勇将,骑射之术也是少有敌手,虽然没有听见弓弦响,也没有看清箭影,但是几乎是一瞬间,他感觉到了那种被人盯上的恐怖,几乎是下意识地转动身子,他双手空空,马槊也不能及时摘下,只能伸手抓去,白羽箭无巧不巧地穿过他的指缝,没入胸口。荆迟仰面向天,一声怒吼,如同小山一般的身躯跌落马下,左右雍军大哗,抢了荆迟向后退走,雍军中立刻传出鸣金之声,数万雍军如同潮水一般退去。

望着远去的雍军,刘万利几乎是不敢相信自己的眼睛,身边的将领亲卫高声呼喝,语气都是兴奋异常,刘万利却突然觉得腰间酸痛,不由苦笑连连,想当初北汉军的勇将,如今已经只能指挥守城,不能冲锋陷阵了。

副将拄着长刀,一瘸一拐地走了过来,狂喜地道:“将军神箭,那荆迟乃是雍军大将,将他射伤阵前,不仅雍军气势大弱,而且雍军失去了主帅,就是攻破壶关也没有什么用处,说不定明日他们就会退兵了。”

刘万利苦笑道:“若是如此最好,可是我若是敌军将领,攻城无功,主将被射伤,就是朝廷不会因此加罪,也会羞辱难当,必然不顾损失,死命破关,希望能够将功赎罪,只怕等到那荆迟生死一定,雍军就会再次猛攻,如今我们的底牌已经被人知晓,只怕接下来不过是捱一日是一日。”他说话的声音很低,毕竟不想打击正在兴奋激动的麾下将士,副将听了也是面色大变。

强撑着身体,安顿好将士布防之后,刘万利回到府邸,他的夫人早就忧心忡忡地准备了汤药热水,扶着他躺上榻去,替他敷药按摩,良久,旧伤带来的疼痛渐渐消去,刘万利才昏昏睡去。不知何时,刘万利忽然觉得鼻窦生痒,不由打了一个喷嚏,神智也清醒过来,睁开眼睛,却看见自己五岁的爱子刘淮拿着一根枯草往自己鼻孔里面插入。刘万利不由发出爽朗的笑声,伸手将爱子抱起,道:“小顽皮,怎么跑来打扰爹爹睡觉。”刘淮忽闪着大眼睛,奶生奶气地道:“爹爹这几天都不理淮儿。”一脸的不满神情。

刘万利心中一酸,心中有些愧疚,暗悔一年前不该心软,让夫人带着孩儿从晋阳来此,当时只道壶关稳如泰山,谁知会有今日的危局,如今敌军压境,破关只是时间的问题,可是自己乃是主将,若是偷偷将夫人和独子送走,只怕城中军民都要失去抵抗的勇气,可是若是不送走,一旦城破,玉石俱焚,雍军连日损失惨重,恐怕会屠城报复,只怕自己的夫人和爱子都要惨死在此。想到这里,刘万利不由身躯微微发抖,抱紧了爱子一句话也说不出来。

这时刘夫人捧着汤药走了进来,看到刘万利这种情态,多年夫妻如何不明白他的心思,她放下药碗,走到榻前跪下道:“相公,妾身本不该多言,可是如今局势如此,相公也要有所准备,妾身和相公结缡十二年,生死与共,休戚相关,情愿陪着相公赴死,可是淮儿年幼,又是刘家唯一的血脉,若是有了什么损伤,妾身到了九泉之下,也无颜见列祖列宗,求相公令人将淮儿送回乡下,交给妾身兄长照顾吧,妾身兄长乃是庶民,就是将来万一,万一风云突变,也不会连累到淮儿的。”

刘万利心中剧痛,他又如何不怜惜爱子,想他少年从军,和新婚夫人不过是相伴三日就上了战场,总算是老天眷顾,才能生还,多年来夫妻聚少别多,家中父母全由夫人照看,直到六年前自己重伤回家休养,才有了淮儿的出生,也让父母临终前没有留下什么遗憾。然后自己又被派到壶关镇守,那时正是大雍和北汉战势紧张的时候,壶关一夕数惊,他自然不敢将家人接来。想不到如今家人团聚却又遭遇敌军猛攻,而且壶关局势岌岌可危。可是若是将爱子送走,只怕会影响到守关,刘万利终于避开了夫人哀求的目光,低声道:“夫人放心,雍军主将今日被我射伤,我们定可等到援军。”说到这里,却是心中长叹,如今哪里还有援军呢?刘夫人也是珠泪滚滚,她不是寻常乡下女子,也是读过诗书,略通经史,又是常年支撑门庭,如何不明白丈夫的言不由衷。

正在刘万利和夫人肝肠寸断的时候,侍女匆匆进来禀报道:“将军,副将大人求见。”

刘万利立刻清醒过来,将爱子交给夫人,道:“你先进去吧,这件事情我会考虑的。”刘夫人心中一喜,连连点头,抱着刘淮匆匆走进后堂,临走还没有忘记嘱咐道:“相公别忘记服药了。”

送走了自己的夫人,吩咐请副将进来,刘万利拿起那碗已经有些温凉的汤药,慢慢的喝着,思忖着副将此来,会有什么事情呢?透过窗子看看外面,现在还不到黄昏,今日一战午时就已经结束了,现在守城诸事都应改已经料理妥当了,守城的事情他已经是驾轻就熟,如何处置应该不需向自己请示,自己旧伤复发,他也是知道的,怎么会在这个时候打扰自己呢?

年轻的副将匆匆走进房间,一见到刘万利就兴奋地说道:“将军,末将有个计策,可以解壶关之危局。”

刘万利心中一动,却是丝毫不露形色,就连端着药碗的手都没有丝毫颤动,淡淡道:“说吧,如今局势险恶如此,就是只有一分的希望,也不能轻易放弃。”

副将激动地道:“末将整顿防务的时候,派了关中最得力的斥候去探察敌军大营的情况,虽然敌军将荆迟的伤情隐瞒起来,可是营中军心不安,所有的军医都在中军大帐待命,众将也都在中军守候,可见荆迟伤势极重,就是不死也要脱一层皮。末将想,如今雍军士气大挫,对我们又不甚防备,他们是因为这些日子我们从未出关迎敌,所以看轻了我们,末将想若是我们今夜挑选精兵两千余人,趁着夜色深入敌军大帐,纵火焚营,,烧毁敌军辎重,杀他们一个措手不及,若是再有机会杀死几个重要的将领,到时候雍军主将不能理事,粮道穿越白陉,也是补给艰难,必定退兵,就是不退,也要暂缓攻关,我们也可以趁机飞檄各县,让他们征集丁勇前来襄助防守壶关,到时候壶关必定能够守住。”

刘万利毕竟多年征战,心中先是一喜,转而又有些担忧,雍军主将荆迟虽然受伤,可是镇州军主将心思缜密,未必想不到劫营的可能,再说雍军兵强马壮,自己这次逆袭未必真得能够达到目的。可是他的目光一闪,已经看到爱子遗落在床榻上面的那截枯草,心中突然一痛,若是这样下去,等到雍军稳住阵脚,壶关必破无疑,若是自己同意这个计策,若是能够逼退雍军,那么冒些险也是值得的。而且根据刘万利多年的沙场经验来看,这个计策倒是有五分机会,如今就是只有一两分机会也只得拼了。放下药碗,刘万利沉声道:“你去军中募集敢于效死的勇士一千五百人,再多就不行了,今夜我亲自率军偷袭。”

副将连忙道:“大人,你旧伤复发,如何能够率军袭营,还是让末将率军去吧。”

刘万利正要反对,熟悉的疼痛从腰间传来,他不由皱了皱眉,只得道:“那就拜托于你了,我军生死存亡就在今夜一战了。”

那青年副将拜倒道:“将军放心,若是有什么差池,末将情愿以身相殉,绝不偷生。”

刘万利心中涌起不祥的预感,想要出声阻止,可是想到如今的局势,心道,就是不成,也不过是早死数日罢了,如今不能再犹豫了。他伸手搀起副将,看着这个随自己作战多年的青年,眼中闪过悲痛之色,就是偷营成功,这种以卵击石的选择也可能是两败俱伤,可是自己却没有选择,只能眼睁睁看着这件事情的发生,从没有像现在这样痛恨苍天为何如此不仁,宁为太平犬,莫为乱世人,刘万利突然涌起一个大逆不道的想法,若是天下能够一统,就是北汉灭亡,那么似乎也没有什么关系吧?这个念头一生出,刘万利下意识地避开了副将的目光,心中暗道,无论如何,自己受王上厚恩,就是以身相殉也是理所当然,若是大雍一统真是不可遏制,那么就让自己成为大雍铁蹄下的血祭牺牲吧。

当夜,月光暗淡,壶关副将带着精心挑选出来的敢死勇士,远远的望着月光下虎踞龙盘的雍军大营,他身后是五百骑兵和一千步兵,士兵衔枚、战马勒口、棉布包蹄,虽然是许多人马,却是一丝声息也无,副将一挥手,百多人向他一拱手,隐入夜色当中。这百多人都是穿着黑色夜行衣,背负单刀,他们都带着引火之物,准备火烧雍军大营,而只待火起,副将就要带着众军冲入雍军大营,要杀他一个人仰马翻。

远处的雍军大营一片沉寂,除了负责夜间守卫的将士之外,几乎看不到人影,似乎所有雍军都在沉睡当中,想必今天白日的大变让他们心中疲惫不堪吧。副将心中也是忐忑不安,毕竟这一战他投入的都是壶关的精兵良将,一旦偷营失败,那可就是万劫不复了。

不多时,雍军营中突然火光四起,纷乱嘈杂的声音响起,火光明灭中可以看见四处奔逃的人影,副将心中大喜,一举手中的马槊,高呼道:“杀!”然后一马当先,冲向了雍军大营,顺着被潜入大营放火的斥候破坏的道路他首先冲进了雍军后营,两边都是烈焰,他用马槊左右挥舞,将已经着火的帐篷挑翻,顺便将它们丢到还没有起火的营帐上面,五百骑兵跟着他一路势如破竹的冲入雍军中军,而其他步兵则四处杀人放火,副将心中畅快,一路上除了将挡路的雍军挑翻之外丝毫不愿耽搁,若是一心想冲入中军,希望能够杀了雍军几个大将。眼睛余光看见大雍军营成了一片火海,他哈哈大笑着将前面拼命前来拦截的一个雍军刺道,高声道:“杀,杀个血流成河!”众军气焰大涨,也都是高声喊杀,就这样冲入了雍军中军,那悬着“荆”字的将旗的大帐。

\chapter{第二十二章 烈火焚城}

燕国公荆迟,出身寒微,太宗拔于行伍,骁勇悍猛,赤胆忠心,太宗每率军入阵,迟皆死命护之,太宗素重之。

荆某本庶人,少无学,不通文墨,太宗诫之曰:“不读书不能为将。”国公闻之诺诺,乃延师读,未两载,已粗通文字,然不通战策,唯行军作战暗合兵法,太宗亦无奈。

武威二十四年,太宗与戾王夺嫡之事急,迟奉命入京,为雍王司马江哲录为弟子,亲授经史兵法,迟性粗疏,得之少,然哲暗语太宗曰:“荆将军乃福将也,略通战策可也。”

隆盛元年三月,迟受命攻壶关,多日不下,遂诈伤诱敌军袭营,大破之,二十四日,破壶关,迟令尽屠城中士民,凶名大盛。而后,迟千里奔袭沁源,沿途若有阻碍,尽屠之,号曰,顺我者昌,逆我者亡。所过之处,血流千里,杀人盈野。北汉民风悍勇,亦慑于迟凶戾,不敢相阻。

——《雍史·燕国公传》

就在北汉军死士冲到雍军大帐之前的时候,副将心中突然一凛,在一片混乱中,雍军大营到处都是火光和往来奔逃的人影,可是眼前的中军大帐却是一片寂静,副将突然大声道:“后退,后退,有埋伏。”他麾下的将士都是神色茫然,目光都集中到他身上,副将一带马就要退走,可是仿佛呼应他的叫声一般,四周突然想起了连绵不绝的号角声和战鼓声,然后顷刻间大放光明,无数手执火把的雍军骑兵绕着大营高声呼喝,火光将雍军大营照得如同白昼一般,而原本大雍军营之内的火势却是渐渐减弱,而络绎不绝的雍军将士仿佛从暗夜中突然出现一般,将自己等人团团包围。副将心中惨然,目光在雍军中搜寻,希望看到设下这个埋伏的主事人。

这时,雍军大阵中分开来,一队身穿青黑色战袍的骑士奔到前面,为首的那人豹头环眼,虬髯如同钢铁,相貌粗豪,正是荆迟,而在他身边则是镇州军主将林崖。荆迟朗声大笑道:“哈哈,你这小子中了本将军的计了,还不快快投降,本将军念在你也有些本事,还可以饶你一死。”

那副将心中涌出绝望的浪潮,原本他以为可能是林崖看破北汉军可能袭营,所以设下埋伏,没有想到却是荆迟诈伤诱敌,可是这个荆迟虽然素有勇名,却没有听说他有这样的本事啊,他忿忿不平地道:“荆迟,你竟然没有受伤,莫非你早就有心诱我等袭营么?”

荆迟策马上前,冷笑道:“老子没有那么多心眼,说句老实话,你们那一箭可是够狠,老子也没有防备,幸好老子武艺不错,那一箭又是没有什么后劲,所以老子闪避的及时,只不过是一点轻伤罢了,老子根本不放在心上,也是你们运道不好,老子一中箭立刻就想到了可以引诱你们出城,省得你们学乌龟王八,打死不肯出壳。”

副将气得火冒三丈,高声道:“我等北汉男儿,顶天立地,怎可屈膝向人,我等今次袭营,已经是抱了必死之心,兄弟们,杀!”说罢带头冲向大雍阵营。这种小小场面,自然不需荆迟动手,雍军中号角迭起,北汉军如同水滴汇入大海,没有能够翻出更大的浪花。

火光照耀下,荆迟的面容带着无尽的杀气和狰狞,他高呼道:“这些北汉人,当真是死也不降,罢了,老子也不是吃素的,我倒要看看是你们的骨头硬,还是我的刀硬。给我将他们全部斩杀,所有的人头收集起来,摆在壶关之前,我要看看壶关还能守到什么时候?”林崖在一边听见,犹豫地道:“荆将军,这不大好吧,战场上厮杀也就罢了,将军这样做只怕会激起北汉人的抵抗之心。”

荆迟怒道:“难道老子手段慈悲,他们就不抵抗了么,一个壶关,就攻了这些时候,老子可是要和齐王殿下会师的,若是一路上北汉军都这样和老子纠缠,老子若是误了军机,要跟谁去说理。若是打上几十军棍也就罢了,如果再被先生罚去抄书,老子可就惨了,再说,若是真得误了大事,只怕老子就是想抄书也没有机会了,等到老子的脑袋被砍下来,难道这些北汉杂种会替老子掉泪么?听老子的,一会儿连夜攻城,若是明日壶关再攻不下,老子豁出去了,等到攻破壶关之后,给老子屠城,将来皇上怪罪下来,老子一人担着。”

见他这般凶神恶煞,林崖也只得唯唯称是,这会儿,潜入雍军大营的北汉军死士都已剿灭,荆迟手下的将士都是跟着他从刀山血海中杀出来的,一个个心如铁石,按照荆迟的命令丝毫不打折扣的将所有北汉军的人头都砍了下来系在马上。荆迟催促林崖下令攻关,林崖也知现在最是壶关虚弱的时候,也就从命,数万雍军逼到壶关之前,竖起火把,将壶关之下照得通亮,荆迟麾下将士将北汉军的首级丢在关下,堆成一个小山,荆迟策马在关下高声叱骂,雍军开始大举攻城。

三月二十三日清晨,刘万利站在城关之上,神色木然,不过是短短一夜,他的须发都已经变成了雪霜之色,昨夜副将出去偷营,他也没有闲着,令众军严阵以待,自己就在壶关之上遥望雍军大营,准备应变。副将中伏之后,刘万利也远远看出了端倪,等到舍命回来报信的斥候说明其中原委之后,刘万利只觉得如同冬日浸在寒水之中一般,冰冷彻骨,却也只能整顿军马,等待雍军攻关。

果然雍军很快就来攻关,或许是过于绝望,刘万利反而觉得自己心中前所未有的平静,指挥着几千残军死守城关,即使是眼看着昔日同袍的首级在雍军马蹄下化成肉泥,他的心思也没有丝毫撼动。如今雍军的攻势如同猛虎一般,有着不得手绝不停止的坚决,日夜不停的攻关,而刘万利就站在关上,几乎是粒米不进,却是觉得全身精力源源不绝,利用前些日子隐藏起来的神臂弩,巩固了壶关的防卫,死守不退。多日苦战,仇恨似海,每个北汉军士都心知肚明,一旦雍军破城,自己就是投降也未必能够活命,所以也没有丝毫懈怠。而雍军损失惨重,只有屠杀才能消解他们心中的怨毒,这一战的胜负关系生死存亡,双方都在殊死作战,谁也不敢稍为松懈。

无论壶关多么坚固,可是毕竟兵力不足,而且副将偷袭身死,损失的都是北汉军精英,所以虽然有神臂弩守关,可是到了二十三日晚间,壶关已经摇摇欲坠。刘万利立在关上,浑身战袍都被鲜血染红,他心中有着深切的悔意,袭营失败使得壶关的失陷至少提前了三日,此刻他越发后悔因为自己的私心而选择了袭营,这三日之差,可能会改变整个北汉战局,他自然明白荆迟深入北汉腹地可能带来的威胁。

夜深了,雍军疯狂而有序地攻着城,刘万利几乎是本能的指挥着手下的将士,可是经过一日夜的守城,壶关守军已经到了山穷水尽的地步,就连最为倚重的神臂弩都已经大半毁去,明日就是破关之时,刘万利心中已经了然,就在方才,已经有协助守城的青壮完全崩溃,口中高喊着愿意投降,想从里面打开城门,被刘万利命令督战队将他们全部射杀,可是壶关中军民斗志已经接近崩溃,刘万利很清楚已经完全不存在守住壶关的可能了。一团混乱的脑海中闪现出妻子和独子的身影,刘万利只觉得无穷的疲惫涌上心头。

三月二十四日,朝阳初升,林崖亲自指挥着一支精力充沛的雍军开始了最后的攻击,壶关的守军在雍军日以继夜的猛攻下终于完全崩溃,青黑色的身影终于冲上了血腥满地的壶关城楼,当雍军从里面打开城门的时候,荆迟带着铁骑一马当先冲入了壶关,他手下亲卫按照他的命令,四处高声喝道:“壶关守将顽固不化,令我军损失惨重,荆将军有令,尽屠城中军民,不得有误。”这一道血腥的命令使得苦战多日的雍军将士有了发泄心中愤怒的途径,在一片残嚎悲叫声中,鲜血流淌在大街小巷,血流成河。

在雍军登城之际,刘万利已经心如死灰,高声传令让北汉军自行逃走,沿途放火阻敌,他带着十几个亲卫奔向自己的府邸,一路上,溃散的北汉军四处放火,他们也都听到了雍军的屠城令,所以也都拼着一死放火阻敌,就是死,也不能让壶关白白落在敌人手中,北汉军这样的念头和雍军歇斯底里的残暴,终于将这屹立百年的险关毁于一旦。

不过刘万利对自己最后这道命令的后果也无心顾及了,他策马奔回府邸,将缰绳丢给亲卫,径自冲进了自己的府邸,家人侍女都已经四散奔逃,只有自己的夫人抱着爱子站在堂上,神色惨然,她一看见刘万利就是一声悲呼,而刘淮却是惊恐地大叫道:“爹爹,好多血。”

刘万利漠然低头,看见自己这一身鲜血狼藉,唇边露出一丝苦笑,对身边仅存的几个亲卫道:“你们都是刘某多年的好兄弟,如今刘某兵败至此,无颜逃生,只是尚有一事相求,不知道你们是否答应。”

那几个亲卫为首的叫做刘均,乃是自幼跟随刘万利的家仆,他下拜泣道:“老爷请吩咐。”

刘万利指着刘淮道:“我半生戎马,只有这一点骨血,你护着夫人和少爷去投奔舅爷,记得将来不要让这孩子替我报仇,两国征战,生死平常事耳,我只希望将来天下一统,这个孩子可以安守田园,娶妻生子,传承香烟。你可答应么?”

刘均闻言拔刀断去左手小指立誓道:“老爷放心,均就是丢了性命,也要护着主母和少爷逃出去,若是属下贪生怕死,就让我下一辈子做猪做狗,永世不得为人。”

刘万利心中一痛,躬身一拜道:“只要尔等尽力也就是了,若是淮儿终究不幸,也是他注定死在乱军之中。”刘均等人怎敢受他大礼,连忙闪身避开。刘万利又看向妻子道:“夫人,我累你半生辛苦,你快跟着刘均走吧,好好照顾我们的孩子,不要记挂于我。”

刘夫人眼中闪过晶莹的泪光,道:“那么将军你呢?”

刘万利颓然坐倒在椅子上,道:“我受王命守壶关,如今三军将士都殉国而亡,我有什么颜面苟且偷生?”

刘夫人镇静自若地将刘淮塞到刘均手中,然后从腰间取出一把匕首,抵住心口,众亲卫骇然惊呼,刘淮也大声哭泣起来,刘万利想要起身,却觉得双腿无力,这两日他全部精力都已经耗尽,一旦坐下,竟然无力起身,他抬手指向刘夫人,惊问道:“夫人,你要做什么?”

刘夫人悲声道:“相公,妾身不习骑射,如何能够随亲卫突围,与其母子死在一起,不如让刘均护着淮儿逃生,就让妾身陪着相公吧。”

刘万利心中大恸,知道夫人说得不错,他也是果决之人,挥手道:“刘均,带着淮儿走吧。”

刘均和几个亲卫都是泪流满面,跪倒拜了两拜,扯下战袍,刘均将刘淮捆在身前,带着几个亲卫冲了出去,外面到处都是喊杀声和马蹄踏地的震耳欲聋的声音,刘均几人的声音很快就消失在一片纷乱声中。刘万利只觉得浑身都已酥软,倒在椅子上,一个字都说不出话来,刘夫人却是十分冷静,将堂上帷幕扯下集中在一起,洒上灯油,然后将一个火把递给刘万利。刘万利只觉得肝肠寸断,一把抱过妻子的娇躯,道:“夫人,我对不起你。”

刘夫人微笑道:“相公,你我夫妻结发之日,就曾互许白首之盟,如今将军白发,妾身也自然要遵守诺言,你我夫妻同生共死,将军应当高兴才是。”

刘万利又是一声痛呼,扬手将那支火把丢到那堆引火之物上,火焰很快就蔓延开来,刘万利却是恍若不觉,只是抱着爱妻痛声悲嚎,刘夫人却是微阖双目,倚在丈夫怀抱中,面上露出愉悦的笑容,火光映照在她的玉容上,使得她的笑容越发明艳。火焰熊熊,很快将两人身影包裹起来,熊熊的火舌吞吐缭绕,和壶关四处纷起的火焰汇合在一起,整个壶关成了一片火海,黑烟滚滚,火光潋滟,壶关在火中颤抖崩溃。

被迫退出火海的荆迟狠狠地瞪着整个陷入火海的壶关,心中越发痛恨,在江哲的计划中,壶关是需要雍军镇守的关隘,只要守住壶关,北汉军就不可能真得切断荆迟的补给,可是如今壶关被大火所毁,想要守住这里就有了很多困难,心中大恨之余,荆迟更是下了决心,沿途一定要大肆杀伐,一定要让北汉军民不敢再这样反抗才行。林崖却是一脸苦涩,虽然他很不满荆迟如此决断,若非是荆迟摆出不肯纳降的姿态,北汉军也未必会誓死反抗,可是无论如何壶关被攻破,多半是荆迟的功劳,自己又能如何呢?

三月二十九日,沁源,北汉军帅帐之内,龙庭飞手里翻阅着军报,眉头紧锁,虽然早有预料,北汉军不可能阻拦荆迟的步伐,可是这样惨重的损失,仍然让龙庭飞触目惊心。

三月二十四日,荆迟攻上党,阵斩上党守将,守军尽皆坑杀。镇州军留一部守壶关,主力进驻上党。荆迟部越上党而不入,沿途十数城关,抵抗者尽遭屠杀。

三月二十六日,荆迟过潞城,声言若是不降,城破之后即屠城,潞城守将投降,荆迟穿城而过,直奔襄垣。

三月二十七日,荆迟火焚襄垣,襄垣守将殉国。预计,三月二十九日未时,荆迟可以到达沁源,雍军两部即将会师。

只有聊聊百余字,却蕴藏着无数的鲜血和惨痛,龙庭飞却只能坐视荆迟在北汉东南腹地纵横杀伐,他他将心中痛苦隐藏起来,很快就可以向荆迟索取抵偿,他暗暗的安慰自己。这时候,段无敌进来禀道:“大将军,齐王在阵前攘战。”

龙庭飞俊脸上闪过汹涌的杀机,道:“好,这一次是他自寻死路。无敌,传令下去,全军准备,待我阅兵之后上阵厮杀。”

段无敌觉察出龙庭飞身上突然迸发出来的豪气,也是心情激荡,虽然龙庭飞没有告诉他详细的布置,可是从萧桐这些日子几乎看不见影踪以及龙庭飞每天都专心研究地图的情况来看,看来龙庭飞已经有了必胜的把握,决战就在眼前,段无敌虽然也有些不满龙庭飞始终不对自己说明详情,但是即将到来的决战让他全然没有了怨怼,只要能够大破雍军,那么无论什么牺牲都是值得的。

比起龙庭飞来说,李显对全局的掌控并不那么准确,荆迟的动向他并不十分清楚,甚至不知道荆迟到了何处,毕竟这里是北汉的领土,荆迟的使者也无法穿破重重关隘,所以他只是按例来挑战罢了。

沁源之野,李显高据在战马之上,在他身后,四万雍军旌旗招展在他身后布阵,青黑色的方阵当中杀气冲天,而最耀眼的就是李显身后的三千铁卫,他们都穿着赤色战袍,春风吹拂中,战袍猎猎,使得他们如同春日遍山遍野的野火一般的嚣张无畏,而其他的雍军骑士则如同钢浇铁铸一般凝立不动,虽然是静止的战阵,可是却蕴藏着动静两种不同的气魄,无论是哪一种,都有着不可抵御的威势和霸气。

可是那个在阵前耀武扬威的李显,心中却是十分郁闷,虽然在安泽遭遇败绩,可是手上的兵力仍然十分雄厚,四万骑兵,还有后面将近四万的步兵,北汉军虽然号称十万铁骑,可是其中大概只有五万人才是精兵,其余的多半是这半年补充的新军,不论是武力还是训练都不如原先的北汉精兵。按理说,自己兵强马壮,还有荆迟的三万铁骑,不知道何时会到,双方大战起来,自己至少不会落败吧。可是江哲居然对自己说,让自己不用太坚持,等到落败之后后退即可,他会在后面整修道路,安排撤退,还让宣松带着步军在后面接应自己,难道自己一定会落败么?已经先后交战好几天了,哪一次北汉军占了便宜?李显愤愤不平地想,干脆自己将北汉军打个落花流水算了,什么务求全歼敌军主力,只要北汉军再大败一次,难道他们还能力挽狂澜么?

这时候,北汉军大营突然有了动作,正对着雍军的南面营门洞开,一支穿着火红色战袍的铁骑狂涌出营门,同时,东、西两侧的营门也是大开,络绎不绝的北汉军骑士潮水般涌出,北汉军和雍军不同,出营的时候并不列阵,如同狼群一般汹涌,也如同狼群一般没有秩序,可是当他们在空地汇聚的时候就如同河川汇入大海一般,很快就凝聚成了森严的战阵。不过片刻,至少数万的北汉军已经结成战阵,而其后还有无数的棕衣骑士正在结成新的战阵。

李显在马上一皱眉,看今天的形势,龙庭飞是想和自己决战了,这几日其实北汉军已经形成了局部的优势,但是李显虽然屡次挑衅,可是龙庭飞就是不肯和自己决战,怎么今日改了主意,莫非是军情有了重大的变化么?他心中打鼓,心道,若是真的决战,我军恐怕抵挡不住,还真用得上那条退路了,可是随云不是说龙庭飞不会轻易出动全力和自己决战的么?

这时候,从北汉军战阵中,数骑亲卫护着一人缓缓而出,那人掀起面甲,露出英俊的面容,深碧色的眼睛蕴藏着深沉的苦痛和悲愤,略现清减的容颜有些憔悴,只有那睥睨天下的风姿仍然如昔,龙庭飞轻轻抚着心爱的长戟,心中满是杀机,数月以来的种种屈辱让他早已心中怨毒无限。麾下四将如今只剩了段无敌,从前军中将士对自己无不心悦诚服,可是自从石英死后,他总是能够感觉到军中不满的情绪日益高涨,可是他只能暂时用武力压制。数日前在安泽水淹雍军,虽然付出的代价也是不小,但是毕竟战果惊人,军中将士对自己的信心才恢复如初。这一切都是因为那个江哲和眼前的李显,无论如何,过去的种种艰难都要过去了,只要今日大败雍军,就可以挽回大局,到时候自己就有机会重整军队了。

看着对面那个手持马槊的桀骜身影,龙庭飞眼中闪过烈焰,若非担心齐王不敌之后退到山中,配合步军阻挡北汉军,然后固守待援,自己怎会对着这几万人马始终不敢全力扑杀,今日终于可以将敌军全部绞杀,到时候北汉军可以像狼群捕猎一样,将入境的雍军一一消灭,雍军遭此惨败,数年之内再也不能北窥,数年之后,只怕大雍自己就自顾不暇了。高举手中长戟,龙庭飞高声喝道:“全歼雍军,生擒李显!”北汉军闻言精神一震,也都随之大声呼喝,一时之间气势大盛。

李显性子本就如火,一听到龙庭飞喊声,不由怒从心起,用手中马槊指向北汉军,笑骂道:“儿郎们,北汉人平日自称英雄,可是在安泽只敢用诡计水攻,这些日子又龟锁在营中不敢迎敌,这些胆小鬼居然要全歼我军,你们可信么?”

李显身边四大侍卫之一的陶林性子最是诙谐,高声应道:“殿下,龙将军大言不惭,你何必恼怒,等到咱们擒了龙大将军,让他给殿下行酒如何?”

雍军听了都是哈哈大笑,北汉军却是高声喝骂,反而李显和龙庭飞只是冷冷对望,主将的冷静渐渐感染了两军将士,不知不觉中,战场恢复了寂静,而那种满含杀机的寂静越发压抑凝重,人人都有一种喘不过气的感觉。然后彷佛是心有灵犀一般,龙庭飞和李显几乎是同时发令,青黑色和棕色的洪流几乎是同时涌动,然后撞击在一起,雍军和北汉军的决战开始了。

\chapter{第二十三章 沙场重逢}

隆盛元年戊寅,三月二十五日,齐王李显兵至沁源,与龙庭飞对峙沁源,北汉军十万,雍军四万,然北汉军多新军,龙庭飞隐忍不出战。

三月二十九日,龙庭飞列阵出,两军决于沁源。

——《资治通鉴·雍纪三》

马槊将一个北汉军挑落马下,李显将马槊交到左手,右手手腕已经有些发麻了,然后在亲卫簇拥下返回中军,这已经是他第三次率亲卫冲阵了,这样痛快淋漓的杀戮真让李显浑身都觉得爽快,虽然雍军在人数上少一些,可是北汉军也只是出动了六七万的样子,而且新军老军混杂,所以虽然已经战了半日,雍军还是没有露出什么败相,可是想要取胜却是休想。而且那个龙庭飞也有和自己相同的爱好,自己不过冲阵三次,他已经冲阵五次了,而且常常带着那些新军杀入雍军在转战中露出的空隙。经过几次的磨练,那些新军作战逐渐熟稔起来,李显能够感觉到压力越来越大了,是不是暂时后退呢?李显一边想着,一边传下军令,指挥雍军攻向敌军的破绽,两军都是百战余生的精骑,棋逢对手,都是陷入了苦战之中。

龙庭飞神色凝重地望着对面的敌军,雍军可真是不好对付,四万雍军,集结成三座骑阵,互相支援,常常是一支冲刺,另外两支压阵支援,雍军甲坚兵利,一次次撕开北汉军的防线,收割足够的性命之后便退去。北汉军由于去年泽州的惨败,无法有效地冲破雍军的战阵,所以龙庭飞索性散开战阵,用轻骑兵在雍军阵外游弋,用弓箭压制雍军的活动范围,调动精兵阻挠雍军冲破北汉军军阵的可能。

就这样双方陷入了僵局,雍军无法破阵,北汉军也无法彻底压制雍军,李显和龙庭飞心中都明白,这样下去,就是一方获胜也不过是一个惨胜。可是两人在临战指挥下水平相差不多,这种军力基本相等的情况下,谁也没有办法速胜,只能在生命的消耗中相持,谁犯的错误越少,谁就是胜利者。若是从前,李显和龙庭飞在这种情况下都会谋求避战,可是今日两日心中都有盘算,所以谁也不肯停手,而且两军缠战半日,双方都是苦战最酣的时候,这种情况更是谁也不敢冒着降低气势的危险退兵的。

李显皱紧了眉头,不对劲,龙庭飞的用兵他是领教过的,什么时候他会在这种结局不明朗的情况下陷入这样的苦战,若没有七、八分以上的胜算,龙庭飞不会大举出动的,死里求生是自己常做的事情,不过现在也很少做了,毕竟自己已经有了可以和龙庭飞对阵的自信了,那么他这样定是有阴谋。这时候苏青策马过来,高声禀报道:“殿下,荆将军已经在二十里之外,前锋已经和我军斥候接触。”李显心中大喜,在北汉境内龙庭飞的消息一定比自己灵通,那么龙庭飞应该是已经知道了荆迟将到的情报,所以想在荆迟到来之前消灭我军。心中计议已定,李显开始改变策略,尽量集中兵力,收缩防线的结果就是北汉军的战线扯地更长,攻击也更加猛烈,彷佛海潮无休无止的冲激着高耸的礁石。而李显也指挥着军队死力缠住龙庭飞,绝对不能让北汉军轻易撤退,只要缠住北汉军一段时间,就可以内外夹击,大破敌军。

二十里之外,荆迟带着铁骑正在向战场奔去,虽然一路上势如破竹,可是还是有不少北汉军民奋起抵抗,虽然被他一一歼灭,可是雍军也受了些损伤,就连荆迟也受了些轻伤。荆迟少年时,正值中原大乱,民不聊生,荆迟又是天生的狠辣性子,不愿在乡里受人欺辱,索性做了强盗,最惯的就是杀人盈野。后来大雍逐渐强盛起来,荆迟虽然性子粗豪,也知道作强盗不是了局,便去投了雍军,因为武艺高强,不到半年就成了军中有数的勇士,后来得到雍王重用,辗转成了雍王的心腹爱将,过去的事情自然无人提起了。李贽军纪严明,最不喜欢杀俘屠城之事,荆迟畏惧军法,所以也拘束住了野性。可是前些日子他独自领军,本就压力极大,再加上北汉人的顽强抵抗,越发触怒了这位强盗将军,索性大开杀戒,本来还不觉得什么,如今快要和齐王会师,荆迟却想起自己所作所为,不由有些烦恼,最后却给他横下心来,若能胜了北汉军,想来不会将自己斩首以正军规吧。所以他虽然知道北汉军兵力不弱,也没有丝毫畏惧,只是根据斥候的回报,判断着如何进军才好。前面探查军情的斥候飞马奔来三言两语说明白军情,又递上亲手绘制的草图。

荆迟令大军缓行,自己停在路边,一边在马鞍上看着斥候绘制的草图,一边低声嘟囔。他此刻形容实在有些狼狈,散发披肩,头盔早就被他不知何时丢落了,一身战袍早就破烂不堪,上面沾着斑斑点点的痕迹,有的是黄色的泥水,有的是红色的血迹,让身边的众将和亲卫暗暗好笑,却不敢多言。一路上荆迟的霸道和杀气可让这些戎马生涯多年的骄兵悍将心中戒惧忌惮的很。以前荆迟跟在雍王身边的时候,自然是不会流露出强烈的草莽气息,而在齐王麾下,荆迟心中一直存有戒心,更不会流露出破绽授人以柄,只有在今次独立领军而又一路杀伐之后,荆迟隐藏在粗豪表面下的真容才被众人熟知,故此都是多了几分畏惧,对着荆迟都是毕恭毕敬,更别说像从前一样开玩笑了。要知道几日前,荆迟就亲手斩了十几个醉心杀掠,忘记整军时间的军中悍卒。这种种变化,早就让众人见识了荆迟一直被压制住的霸道狠辣,所以任凭荆迟在那里专心研究地图而不肯及时出兵支援齐王,也没有人敢多问一句。

胡乱搔了搔一头乱发,荆迟终于抬头道:“好了,现在北汉军已经被齐王殿下缠住了,现在出兵最好,一定可以把北汉军阵搅得稀烂,到时候我们就可以狠打落水狗了。传我令,从敌军东侧直插中军,跟着老子的旗号,走。”说罢一声大喝,策马奔下山梁,他心中暗想,如今北汉军不知道自己到了才奇怪,不过想来他们也是没有办法脱身吧,老子一路上但凡遇到北汉军的探子都杀得干干净净,你就是得到情报也未必可以掌握老子发动的时间,不过就连撤军都撤不走,也真是无能,若非知道不可能有援军,老子可不敢全军出动。

传罢命令,荆迟一马当先奔去,众将都是精神大振,各自返回本阵,在行军中整顿军马,雍军铁骑都是百战余生的精兵,纵然在行进间队列也是丝毫不乱,马蹄声更是井然有序,千军万马倒像是一人一骑一般,荆迟抢先冲上一个斜坡,下面几十里平原,正是齐王和龙庭飞两军酣战之处,不远处就是沁源城,和春潮汹涌的沁水。荆迟一挥手,一个亲卫拿起号角,吹动起来,然后雍军军阵各处号角齐鸣,声音如同划破长空的迅雷,连绵高亢。荆迟振臂大呼道:“随我来。”然后一把从亲卫手中夺过一面将旗,左手高高举起,策马跃下山坡,身后将士不待他再次发令,也随之冲下,一道浑似黑水一般的洪流直插入北汉军东侧战阵。那军旗杆顶乃是锋利的枪头,荆迟挥旗一挑,将一个北汉军士刺倒,雍军铁骑如同钢刀一般,将北汉军东侧右翼划破。

就在雍军入阵的刹那,龙庭飞眼中闪过一丝寒芒,他厉声道:“无敌阻截齐王主力,我亲自去对付雍军援军。”然后又低声道:“无敌只需支持两个时辰即可。”然后带着亲卫迎向从右翼猛攻向中军的荆迟。段无敌眼中闪过一丝了悟,接过指挥权,接下了齐王越来越猛烈的攻击。

北汉军右翼以新军居多,荆迟选了这里切入,也是因为得到斥候回报,对于富有经验的斥候来说,新军老军一看便知,而对荆迟来说,虽然是内外夹攻,但是毕竟两军数量相差不大,想要取胜自然只有从敌军最弱处动手。而情况也似乎十分顺利,北汉军右翼居然轻而易举地被荆迟击穿,荆迟心中大惑。左顾右盼间,眼前红光迸现,一支身穿红色战袍的北汉军挡在了前面。荆迟心中一惊,但是此刻已是有进无退,荆迟一咬牙,将旗丢给身后的亲卫,马槊一指,直向北汉军帅旗攻去,不过瞬息之间,雍军荆迟部已经和北汉军最强大的武力碰撞在一起,北汉军右翼则开始用弓箭射击荆迟部的中后部,而龙庭飞挺身而出,强行止住了雍军的前进,战场上一片混战,两军交缠在一起,鲜血渗透了大地,汇入了沁水,那呜咽的血红色河水向下游淌去,带去无数人的性命和一切。

齐王和荆迟都知道胜负在此一举,若给北汉军重整旗鼓,只怕就是旷日持久的苦战,所以两人都是尽展所能,雍军几乎是不顾一切的猛攻,但是龙庭飞屹立不退,遏制了荆迟的攻势,段无敌则是通过严密的防守,将齐王主力压制住,眼看着战局又进入僵局,虽然李显和荆迟渐渐占了上风,毕竟更善于突袭猎杀的北汉军在大规模骑战上少些优势,可是荆迟和李显心中都涌起强烈的不安。只是隔着重重阻隔,两人无法沟通,更是不敢轻易退去,若是自己一方先退,只怕所有的压力集中在另外一方上面,就有大败之虞。虽然雍军似乎渐渐控制了战局,一心苦守的北汉军却是士气渐渐消退,两人却都是一脸的苦涩和疑惑。荆迟两次三番带着精兵猛攻龙庭飞亲卫,有一次荆迟甚至亲自冲入北汉军阵,更是和龙庭飞亲自交手,可是龙庭飞的画戟舞动起来如同黑豹出林,流畅敏捷中带着浓厚的杀机,荆迟反而被他击退,不得不牺牲了十数亲卫逃回本阵。

李显心中越发不安,无意中抬头,突然看见空中两只苍鹰反复盘旋,李显心中一凛,高声道:“端木,给我射杀那些苍鹰。”他的声音变得尖利凶狠,担任李显亲卫的端木秋如今已经比较熟悉军旅生涯,听到李显传令,摘下银弓,引弓成满月,三支鹰翎箭如同如同流虹一般划过长空,一只苍鹰哀鸣坠落,另一只苍鹰却是一箭擦过翅膀,摇摇欲坠地向远处飞去,弓弦再响,一支鹰翎箭透过苍鹰身躯。李显心中没有丝毫愉悦,到底龙庭飞准备了什么杀手锏。突然之间,李显脑海中灵光一闪,他苦笑连连,此刻他才明白为何江哲会说自己必然大败,自己怎会忘记北汉存亡之秋,区区约定又怎能抵得过骨肉之亲,夫妻之情。几乎是立刻之间,李显下令吹动撤军的号角。心中也有了不妥感觉的荆迟也是立刻收缩阵线,准备抢先冲出北汉军的包围。

几乎是那两头苍鹰陨落的瞬间,一处隐蔽的山谷之内,身穿深绿色甲胄,外罩金凤织锦大氅的林碧负手而立,望着哀鸣滑落的爱鹰,凤目中露出一丝冰寒之色。她冷冷道:“众军听令,出发。”那些原本闲散的坐在地上,倚在马鞍前的,看上去和气懒散的军士几乎是在顷刻之间褪去了伪装,上马,整理兵器,立刻变成了杀气凛凛的战士。林碧翻身上了战马,也不招呼一声,便策马冲出了山谷,丝毫不用她吩咐,二十多名男女亲卫如影随形一般策马跟上,将林碧护在当中,而那些原本看上去散漫混乱的代州骑士更是丝毫没有犹豫,虽然从衣甲上面看不出他们的军职高低,可是他们自然而然的按照心照不宣的次序策马跟上,似乎松散而实际上严密的骑兵战阵本就是代州军的特色之一。

这个山谷中聚集了一万五千代州军,和北汉军主力不同,代州军穿的是各色各样的皮甲,看上去似乎十分混乱,这是因为代州军几乎是父死子继,兄终弟及,往往一副上好的甲胄流传数代,就连兵器马匹也往往是自备,这是代州军独一无二的传统。

东晋文弱,即使在中兴之时,朝廷也无能抵御蛮人,而林氏为了保护乡土,便私自招募乡勇御敌,因为代州不论男女,为了抵御蛮人都是苦练骑射,所以代州军都是土生土长的乡人。至于自备兵器马匹,乃是因为代州人虽然深受蛮人侵掠,却也被蛮人的习性所染,在代州,若是稍有资财,家中如果生了一个男孩子,第一件事情就是准备一块精铁,然后每年锤炼一次,等到这个男孩子成人,就将这块精铁铸成兵器,百炼精钢铸成的兵器自然是得心应手。而一般在这个男孩子稍微长大的时候,就选一匹小马驹让他亲自喂养照看,这样等到男孩子长大之后,就可以得到一匹心灵相通的爱马。即使后来代州军成了名正言顺的官兵,这种习性也没有改变,所以代州军看上去总是有些像乌合之众。可是只有和他们做过战的人才知道代州军的可怕之处。

因为常年和蛮人作战,几乎每一个代州军士都有单枪匹马被蛮人追杀的经历,所以他们的战力绝对是出类拔萃,而一旦他们组成骑兵,又是另外一番景象,代州军是靠着血缘和地域组织起来的劲旅,所以一旦上了战场,这些骑兵的协同作战可以说是天衣无缝,为了亲人的安危,他们作战悍不畏死,这样一支骑兵可以说是天下无双,只是将近百年来,代州军从来没有过出境作战的例子,所以除了蛮人和曾经和代州军苦战过的北汉军,无人真正知晓代州军的可怕之处。这一次北汉王室动之以情,终于说服了代州出兵,而林碧在代州军心目中是下一任统帅的不二人选,也是看在龙庭飞乃是林碧未婚夫婿的份上,代州军才会同意到沁源助阵。

就在李显和荆迟心有默契地想要退兵,却被龙庭飞率北汉军苦苦缠住的时候,远处突然响起号角声,那号角声和雍军、北汉军常常使用的曲调皆不相同,充满了苍凉和野性,令人一听之下就觉得心胆俱寒。而且在李显、荆迟的耳中,可以听得出来那号角声快速前进,几乎是风驰电掣,能够以这样的速度,保持骑兵冲锋的阵形,两人都自认没有这样的本事,不由心中更是忧虑。那号角声从西北方向逼来,却在即将接近战场的时候突然转了方向,向李显后阵绕去。李显心中大惊,连声催动麾下将士变阵,加强后面的防御。

可是几乎就在李显的将令传递到全军的时候,努力变换阵势的雍军遭到了重击,代州军的战马虽然看上去毛色混乱,可是有一个共同的特点,就是都是上好的战马,毕竟在战场上想要保住性命,马匹的精良是必要的条件,而且代州接近蛮地,虽然年年交战,可是闲时的互市也不会错过,代州人有更好的途径获得蛮人的良马。所以林碧带着代州军几乎是没有任何迟滞地冲入了雍军后阵,然后就是雨点似的箭矢落下。准确而无情的消灭着后方的雍军。

若论骑射之术,中原没有军队可以胜过代州军,为了和蛮人作战,代州不论男女,都是自幼学习射箭,就是一个小女孩,也可以轻而易举的百步穿杨。而在战场上,骑马射箭有三种境界,最平常的就是“骑射”,要求可以在战马上可以坐稳射箭,要求百米靶十中五,七十米靶十中七,五十米靶十中九。当然不要说代州军,就是雍军和北汉军的精兵在“骑射”上也可以做到百米靶十中八九。第二种境界就是“奔射”,要求骑士在高速奔跑的战马能全方位射击,并且命中率最起码要达到骑射的要求。还有一项要求是,在战马奔驰的一起一伏中,骑士必须抓住这瞬间各射出一箭。凡是能够做到这一点的骑兵已经是天下有数的精兵,就是雍军和北汉军中也只有三成军队可以完全达到这样的目标。第三种境界就是“飞射”,要求在任何状态下都可以射中固定的靶子,这已经不是普通骑兵能掌握的技术,能够有这种本领的骑士通常是军中有数的神箭手或者出色的骑兵将领。而代州军可怕之处就在于几乎所有人都能够达到“奔射”的境界,还有一成左右可以达到“飞射”的境界,这样的水准,就是以骑射为谋生技能的蛮人也不过如此。

眼睁睁看着代州军在雍军后阵中纵横来去,近处用马刀,远处用弓箭,轻而易举地摧毁了后面的防线,李显只觉得心中剧震,此刻他已经明白败局已成,若是换了别人,不免不服或者颓丧,可是李显不知道在龙庭飞手下吃过多少次亏,吃败仗早已成了习惯,此刻想也不想发出将令,带着雍军向北汉新军的方向冲去,这时候荆迟已经穿越阻碍,和李显会师,李显一见到荆迟,也不容他反对,厉声道:“荆将军,你为先锋,率军冲阵,向安泽方向败退,本王亲自断后。”说罢带着亲卫军闪在一旁。让后面的雍军先通过。

荆迟略一犹豫,就策马冲在了前面,他也是深知李显的脾气,知道这个时候若是自己争着断后,只怕会被李显一刀砍了,自己若想李显平安,唯一的法子就是尽快冲破重围。而他主攻的方向都是北汉军的新军,对着凶神恶煞的荆迟,不由有些怯然,荆迟几乎是没有费多少力气就冲破重围,向安泽方向退去。而李显带着亲卫军断后,几乎是承担了代州军的全部压力。明明数量远远不及雍军和北汉军,可是代州军的攻击如火如荼,几乎让李显忽视了龙庭飞正在从两侧猛烈攻击雍军的两肋。可是坦率的说,雍军和北汉军交手多年,彼此对于对方的战术都很熟悉,所以应对北汉军的攻击,虽然雍军损失不小,可是倒也是应对的十分顺手。而代州军却不同了,只见他们交错着射箭,准确而有效地消灭着落后的雍军,丝毫不显得急躁,始终紧紧黏在后面,从容自若而又冷酷无情的猎杀令人心中陡然生出寒意。李显虽然亲自断后,可是仍然只能勉强挡着代州军的攻击。

李显心中焦急非常,若是不能迅速和敌军脱离,雍军恐怕要惨败溃散了,李显心一横,策马扬鞭向代州军前锋冲去,他身边的亲卫迅速跟上,而紧紧跟着李显的一队亲卫都拿着皮盾替李显遮挡箭雨,而端木秋则紧跟在李显身侧,引弓待发。代州军稍微停滞了一下,似乎有些诧异雍军为何反而迎头冲上,可是几乎是立刻间代州军阵放缓了速度,前锋形成了一个半圆,仿佛要将雍军反攻而来的这支劲旅围住,而箭矢却更加密集,想要尽可能的消灭这支敌军。虽然李显亲卫执盾相护,可是仍然有不少赤衣骑士坠马陨命。

这时候,端木秋一声厉喝,弓弦迭响,每声轻响都有九支羽箭如同幻影一般射入代州军阵,端木秋号称银弓,箭术自然是已经到了炉火纯青的地步,就是以骑射见长的代州军也罕有人能及,一时之间,不少冲锋在前的代州军勇士中箭坠马,代州军是绝对不会和敌人争一时之锋芒的,所以代州军又放缓了一些速度,而就在这时,李显已经冲入了代州军前锋,马槊横扫,鲜血迸现,即使是个人战力极强的代州军勇士,也是有所不敌。一时之间,代州军的攻势被强行遏制了,虽然这只是暂时的,代州军的反攻将更加悍勇,可是战场之上,生死往往在一线之间,任何迟滞都可能造成不可挽回的后果,所以代州军的主将林碧动了。

刚将一名代州军士刺落马下,李显耳中传来清脆的銮铃声,然后他便看到雪亮的枪尖刺向自己的咽喉,那一枪突如其来,枪上的红缨被劲风激荡,直立得宛如钢针,李显手中的马槊向上格挡,那银枪顷刻间化成千百条幻影,李显只觉得马槊没有碰到丝毫阻碍,一种力道落空的无力感从心中涌起,然后便觉得双手虎口剧痛,马槊被一个强劲的力道向上挑起,如虚似幻却带着无穷杀机的枪尖从两臂之间刺向李显的胸口。银枪带出的劲风带着无坚不摧的威势,若被这一枪刺中,虽然有甲胄的保护恐怕也会重伤。不过李显毕竟是久经沙场的虎将,他将手中的马槊向前抛出,身子在马上扭转,枪尖擦过他的左肋,两马错镫之际,李显长身而起,右手抓住从空中坠落的马槊,顺势刺向敌人,银枪毫不示弱的架住了马槊,瞬息之间,撞击数次,却是平分秋色,李显忍不住抬头望去,那人也正向他往来,四目相对,两人都是有些愕然。虽然是敌对的主将,可是战场上主将交锋乃是罕见之事,两人交手之前竟是谁都没有想到会遇到彼此。

林碧目光闪动,对面的敌人面甲并没有放下,她一眼就认出这人正是雍军主帅李显,和上次相见不同,那时的李显危险而压抑,仿佛虽是都会择人而啮的猎豹,可是如今的李显神色坚毅果决,虽然是战败之际,却仍是没有一丝灰心沮丧,那一种泰山崩于前而面不改色的气度,让林碧也不由心折,那一身火色战袍已经被鲜血浸透,更衬出李显的英勇彪悍。

李显看着对面的敌人,银枪黑马,深绿色甲胄,虽然面甲没有掀起,看不到容颜,可是那双隔着面甲仍然湛然幽冷的凤目,以及婀娜矫健的英姿,再加上身后绣着织锦凤凰的大氅,都显露了对方的身份。他无声地道:“嘉平公主。”

几乎是同时,两人想起了东海波涛之上,两人对饮的情景,当时曾有生死无恨之语,虽然有知己之感,可惜两人却是敌人。李显和林碧都是心志坚毅之人,几乎是一失神之后,又都立刻清醒过来,银枪和马槊分开,两马错身而过,两人几乎是同时强行策马回身,一声清鸣,马槊和银枪再次交锋。这时,两人亲卫已经蜂拥而上,将两人分隔开来。李显仰首长啸,这番冲杀,已经暂时抑止了代州军的攻势,达到目的之后,李显立刻向雍军后阵追去,在雍军将领的接应下,飞也似的逃去。或许是逃得多了,虽然马速极快,战阵却是丝毫不乱。

林碧怅然低吟道:“陌路相逢成知己,他年沙场见此心。”然后高声道:“随我追,就是追到冀氏,也要取了李显性命。”代州军闻言也随之高呼道:“杀了李显,杀了李显。”代州铁骑径自向雍军追去。龙庭飞心中暗暗计算,方才一战,虽然已经大胜,可是雍军主力仍然存在,而且若是李显不死,自己这一战也不能说是大获全胜,于是也扬声道:“诸君,公主带着代州军前来助阵,我们岂可落在人后,杀。”北汉军将士轰然应诺,也向雍军追杀而去。

\chapter{第二十四章 战事如棋}

两军对峙,未分胜负,雍将荆迟千里奔袭,猛攻北汉军后军,龙庭飞率亲卫迎之,荆迟不能胜。

战正酣,嘉平公主率代州军攻齐王后军,代州军骁勇善战,齐王不敌,乃竭力突围。王亲自断后,全军而退。

是役也,齐王部折万五,荆迟部折九千,龙庭飞军折万人,代州军几无所损,遗尸遍野,沁水尽血染。

——《资治通鉴·雍纪三》

三十里之外,沁源与安泽之间的群山中,一处修整过的山梁上,千余雍军在倚山而建的寨垒中严阵以待,而在寨垒最高处,一个青衣书生和一个青袍儒将正在对弈。一枚黑色的围棋子轻轻落在一片白子的边缘,将白色的大龙困在其中,宣松微笑着看向愁眉苦脸的监军大人,若论弈棋,这位监军大人可远远不是自己的对手啊,不过也只有在下棋的的时候,这位江大人才会流露出一些孩子气吧。不过宣松心思也不在棋上,这次齐王兵锋直指沁源,监军大人却说服殿下将所有步兵留下,整修道路,修建工事,从冀氏到安泽、沁源之间的群山,布下了多重防线,若是问他为何耗费兵力防守,他却只道“未虑胜,先虑败”。众人只觉得监军大人过于谨慎,但是念及前些日子的败阵,再说齐王已经同意,也就无人反对,宣松心中最是迷糊,原本和龙庭飞对峙最需大将,江哲却是将自己留在此处,前几日还令自己安排防线,这几日防线粗成,索性就拉着自己下棋,倒像是无所事事一般。可是宣松却不能像江哲这样轻松,但是他生性深沉,知道纵然自己焦急万分,也不能让这位监军大人交出底细,所以索性在棋盘上将他杀得七零八落。

我看看一败涂地的棋盘,心里盘算着是否让小顺子传音给我,然后大胜个几盘,可是想来想去,棋风不同,太容易被人看穿了,终于还是作罢,这时候一骑绝尘而来,马上是一个少年骑士,正是前几日才赶来的赤骥。我让他留心前方的军情,现在他快马赶来,想必是设想中的变化已经出现了。我微笑着丢下棋子,赤骥下马走到近前,躬身道:“公子,前方军报传来,荆迟将军已经和齐王殿下会师,若是苦战下去,我军或会惨胜。不过我们果然发现了代州军的踪迹。”

我挥手让赤骥退到一边,看向皱眉苦思的宣松,道:“宣将军可知道天下最强的骑兵是哪一支?”

宣松苦笑道:“这个也不好说,我大雍铁骑和北汉骁骑似乎相差不多,南楚、蜀国就不必提了,除非是塞外蛮人的骑兵,可以说得上是天下最强。”

我对小顺子道:“撤去棋盘,将地图拿来。”

小顺子上前将棋盘收好,交给赤骥拿了下去,将一张地图放到了方桌上,轻轻铺开。

我指着上面一个明显的标志道:“天下骑兵最强的就是代州军,不论是奔袭还是冲锋,天下少有能够敌得过的,这些年来,蛮人年年铩羽,都是因为代州军越来越强大,可是木秀于林,风必摧之,可知代州军为什么能够安然无恙?”

宣松皱眉道:“北汉国主和代州林氏乃是姻亲,林氏既无反心,北汉国主怎会加害?”

我摇头道:“虽然也有这个缘故,但是还有最重要的一个缘故,就是代州军有最大的缺陷,这个缺陷注定林氏不可能以代州军为根基成就霸业,所以不论是东晋后期,还是北汉立国,最后都默许了林氏割据代州。”

宣松正容道:“愿闻其详。”

我笑道:“其实宣将军也未必不知道,只是可能不够充分罢了,代州军兵力虽强,但是却十分排外,代州军以血缘和忠义维系,所以若不是代州人,绝没有可能在代州军取得高位,而且代州军只对守家卫土感兴趣,所以不论是蛮人侵掠还是北汉军进攻,代州军都是誓死反抗,可是若想让代州军出境攻击,那大半将士都是敬谢不敏的。所以只要不侵犯代州,那么代州就是最好的朋友,这就是北汉国主最后竭力结好代州林氏,而又许诺不调用代州军的缘故。只因代州军本就是不可能被轻易调动的。所以北汉虽然拥有代州,但是世人都不将代州军当成北汉的战力,只因代州军不出境,已经是人们心中的固有的印象。”

宣松皱紧了眉头,只因他听不出江哲说这番话的原因。

我叹了口气道:“说到这里,我就不得不佩服北汉的国主,自从代州降服之后,不仅恪守诺言,绝不调用代州军,还对代州百般结好,几次代州有了灾情,他都动用国库赈济,每年赏赐代州军的金帛都十分丰盛,十几年前,中原多家势力混战,数次侵入北汉,甚至兵锋直指晋阳城,北汉国主都没有调动代州军,因为那时候中原还没有平定,只要守住晋阳,那么入侵的势力都必定没有后力,不得不退走。所以到了北汉生死存亡之际,厚积薄发,代州和北汉朝廷的关系已经到了最密切的时候,所以才可能说服代州军出兵相助北汉军围歼我军。”

听到此处,宣松已经是面色铁青,他沉重地道:“代州军虽然强大,但是毕竟一州之力,有限得很,未必可以起到什么作用。”

我指向地图上面的雁门,道:“代州军不会倾巢而出,只因蛮人南下的时间快要到了,这一次蛮人虽然因为雪灾受到很大的打击,可是侵掠定然会更加凶狠,虽然后力不足,可是初时的攻击一定是非常猛烈,所以两万五千的代州军最多只能有一万五千人南下,而能够担任主将的只有嘉平公主,她既是北汉公主,又是代州军心目中的统帅,更是北汉军主将龙庭飞的未婚妻子,只有她才能够和龙庭飞配合歼灭我军。我早已料定,代州军必然出战,如果不出战,那么龙庭飞种种布置无从解释。”

宣松腾地站了起来,道:“监军大人既然早知道代州军会出兵,为何不告知殿下,殿下只有四万铁骑,加上荆将军最多不过七万,北汉军原本已经有十万军队,再加上虎狼也似的代州军,殿下岂不是败局已成,大人坐视此事发生,是为何故?”

我淡淡的看了宣松一眼,继续道:“宣将军可知道敌我两军所求者何?”

宣松强忍心中愤怒,道:“自然是战胜敌军,我军与北汉军已是誓不两立,北汉军若败,就是亡国之危,我军若败,数年之内无力北窥。”

我摇头道:“宣将军所说并不完全,北汉军想要取胜,可是他们不想要一场惨胜,大雍势强,北汉国力不足,我们若是败了,不需数年就可以东山再起,北汉军就是惨胜,二十年之内恐怕也无力南下,如今天下争霸已经到了最关键的时候,北汉若是国力骤降,就是我大雍亡了,也有别人来攻,所以北汉国主和龙庭飞想要的是一场大胜,而且还要损失越少越好。所以我军在安泽败后继续北上,就是踏入了龙庭飞预定的战场,他要在沁源歼灭我军主力,最好是将齐王殿下俘虏或者杀死,这样大雍伤筋动骨,北汉国力无损,他们就可以眼看着我大雍陷入和南楚的缠战之中,而他们可以休养生息,等到大雍国疲民弱,北汉军就可以南下西进,攻取大雍领土。”

宣松听得连连点头,道:“所以龙庭飞才会调动代州军,只因他手上的十万铁骑不能稳胜我军。”

我说道:“不仅如此,荆将军行踪龙庭飞焉能不知,他是故意不留后备军力,全军攻击齐王殿下,诱使荆将军不顾长途跋涉之后军队疲惫,立刻加入战局。”

宣松问道:“若是荆将军猜透龙庭飞诱使他攻击呢?”

我摇头道:“先不说荆迟是否能够看穿龙庭飞的心思,若是荆将军不进攻,齐王殿下必然损失惨重,到时候就是两军会师,也不能稳操胜券,所以荆将军是一定会攻击的,再说晋阳军不能轻动,而且步兵居多,荆将军也想不到会有一支强力的骑兵作为北汉军后援。所以这个陷阱荆将军是一定会踏进去的。”

宣松眼中闪过迷茫,道:“末将不明白,既然监军大人早知如此,为何不改弦易辙,稳步作战?”

我笑道:“这就要说到我军的作战目的,我军兵力强大,若是强攻北汉,虽然不免损兵折将,但是北汉终究是不敌我军的,代州军虽然骁勇,可是一来不能久离代州,二来毕竟只有万余人,所以我军如果稳步作战,不是大胜也是惨胜,这都无关紧要,可是北汉和蜀国、南楚不同,蜀人偏安,一旦亡国,就很容易安抚,虽然会有些不自量力之人想要复国,但是若不能得到强大力量的支持,他们是翻不起什么大浪的。楚人暗弱,一旦亡国,只要不损害他们的利益,他们多半不敢反抗。唯有北汉,国主尚称贤明,军民上下一心,若是我军贪求速胜,只顾夺城拔寨,就是我军攻下了晋阳城,控制了北汉王室,也不能压制各地兴起的义军。所以皇上不担心我们落败,若是败了再战就好,若是不能全胜才是麻烦。若是敌军主力仍存,必然一城一城的据守,这就已经是不解之局,有些事情你不清楚,我们没有那多么时间,就是北汉军主力溃散了,只要留下一两成的余孽,那么将来我们面对的就是所有北汉人的反抗,那些逃散的北汉军就是火种,而且若是有龙庭飞之类的人物逃生,别说三年五年,就是十年八年,我们也难以征服北汉。所以我军要胜,就必须要将北汉军一网打尽,还要将北汉军的首脑人物全部成擒。想要做到这一点就必须将敌人诱到我们的战场,可是龙庭飞、林碧和北汉将军们不是蠢人,若想让他们入彀,就必须有足够的牺牲。所以齐王殿下必然会在沁源战败,然后才可以败退诱敌。而北汉军为了取得满意的战果,一定会紧追不舍,只有这样,我军的目标才会实现?”

宣松听得目眩神迷,良久才道:“原来如此,殿下可是已经知道其中关节了么,只是可怜我军惨死的勇士。”

我叹息道:“齐王殿下知道一些,但是并不完全,整个作战方略只有皇上和我清楚全盘关节,我以殿下将会战败相激,殿下作战之时,必然奋勇无比,这样才会让龙庭飞中计,但是到了将败之时,殿下久经战阵,又是胜不骄败不馁的性子,所以必然能够尽量保全实力撤退。宣将军,一局棋若没有两个国手对弈,总是难得精彩,北汉这一局棋,正因敌手高明,才会中我计算,若不是龙庭飞知道必须擒杀了齐王才算功德圆满,又怎会被诱入我们准备好的战场。这一迷局,北汉就是再有聪明的人也看不穿,身在局中,有几人能够超然物外。”

宣松已是心悦诚服,道:“请监军大人示下,末将应该如何行事?”

我指向地图上的一点道:“敌军追击,必然是凶猛无比,我军败退,也要做得十分严密,宣将军只需用出手段来,接应齐王和荆将军退到此处,就是大功一件,将军需要记得,敌军主将乃是非凡之人,将军败退之时越是尽心尽力,敌军越不会想到我军还有后手。”

看到我所指之处,宣松眼中闪过热烈的光芒,道:“原来如此,怪不得,怪不得。”

我微微一笑,又道:“北汉军水淹安泽之后,道路被毁,我连日令人整修道路,就是为了接应我军,一来是为了减轻伤亡,二来我们的准备越充分,北汉军就会以为我们求胜之心越强烈,就更不会想到我军败退会有什么别的意图。”

这时候,小顺子递过我的大氅,我接过披上,道:“既然宣将军已经知道局势,在下就要先告退了,江某无才,经不住战阵之苦,就先到后面等着诸位,齐王殿下身边有法正大师和法忍大师率各派高手保护,宣将军不必忧心,纵然是有些危险,他们也能保住殿下平安。”

宣松脸上露出古怪的神色,想不到监军大人将临阵脱逃说得如此理直气壮,不过知道齐王应该不会有生命之险,还是让宣松松了一口气,如今这里就是江哲官职最高,他要先走也是无人能够阻拦,或许这就是江哲强行留下自己的缘故,只因自己可以在他脱身之后率军接应齐王吧。

我当然知道宣松的心思,不过为了不再领略逃跑的痛苦经历,我是宁可临阵脱逃了,带着小顺子和赤骥以及那些神情不满的虎赍卫,他们多半都想上阵杀敌,我向准备好的马车走去。临上马车之前,我忍不住抬头看看苍穹,再过小半个时辰应该就是日落了,想必一更时分,齐王就可以败退到第一道防线,不过这几百里的败退路程并不好走,不过这一点我就无能为力了,行军作战,夫未战而庙算胜者,得算多也,如今种种布局已经如我所料,若是我军仍然落败,也只能说是天意如此,非人力所能挽回。不过我却也不必忧心,北汉国力军力摆在那里,最多我们胜得辛苦些,留下的后患多些,难道还能让他们翻天么?忍不住想到龙庭飞,看他行军布阵,也是一等一的人才,可惜却是我的对手。忍不住低声道:“剪其羽翼,断其枝叶,缚其手脚,困其意志,此谓四面绝网,纵有翻天覆地之才,安能脱我掌握?”不知怎么,难言的疲倦涌了上来,这些日子殚精竭虑,仔细安排种种布局,唯恐有些什么事情改变了大局,如今总算是乾坤已定,接下来的事情已经不受我控制,我几乎是昏昏沉沉的上了马车,临上车前,我突然回头,对宣松道:“吩咐苏青,一定要尽全力截杀北汉军密谍,绝对不能让北汉军发现我军的布局,北汉军中段凌霄已经不可能亲自出手,秋玉飞也被拘留东海,剩下的人中应该苏青可以应付,就是有些不能应付的,齐王身边的高手也可襄助,急着,绝对不能让他们识破。”

宣松几乎是小心翼翼地道:“末将遵命,大人可是身体不适,还是快些休息吧。”

我抬头,看见小顺子、赤骥、呼延寿等人眼中都是闪过忧虑之色,我笑道:“怎么了?都是大惊小怪的模样?”

小顺子突然一声轻叹,将一粒药丸塞到我口中,我只觉得身心渐渐松懈下来,甜美的梦境向我袭来,很快就昏睡了过去。

宣松心惊地道:“大人面色为何如此苍白,可是旧病复发么?”

小顺子冷冷道:“公子为了此战,殚精竭虑将近半载,如今诸事已经尽在算中,公子松懈下来,不免有些倦怠,宣将军,此战胜败,你关系重大,若是因为你的缘故让公子功败垂成,我定不会饶你。”说罢抱着江哲进入车厢,赤骥忧心地望了车厢一眼,坐上车夫的位置,挥起了马鞭。

望着远去的马车,宣松心中一阵愧疚,方才他还在腹诽江哲临阵脱逃,却全然没有想到令敌我双方按照他的布局行动,需要耗费江哲多少心思,他断然道:“立刻出发,我们去接应齐王殿下。”自有亲卫奉上甲胄马匹,宣松换了衣甲,策马扬鞭,向沁源方向奔去。

远方的战场上,李显几乎是一边断后压阵,一边低声暗骂,自己怎么会如此之蠢,当初想来想去,居然就没有想起代州军,林碧会来助阵,他倒是想到过,可是代州军会来一半以上,他可没有想过,毕竟代州军不出境,乃是人们心中的常识,而且谁都知道北方蛮人蠢蠢欲动,谁会想到林碧会如此大胆,带了大半军力南下呢?不过他骂得最多的还是江哲,全盘的安排李显还真得不大清楚,所以他心中有些没底,不知道后面的安排是否妥当,不由后悔自己当初被江哲三言两语激得只想和北汉军拼个你死我活,没有详细追问。这时候,荆迟已经从前军转来,前面自有雍军宿将开路,他也跑到后面相助齐王断后,策马奔到齐王身边,荆迟有些沮丧地道:“殿下,咱们妄称英雄,竟然被一个女子打得落花流水,这下可怎么办,回去之后怎么见人啊?”

李显也懒得和他解说,反正到时候荆迟自然就知道了,努努嘴道:“别愣着了,代州军又上来了。”

只见远处烟尘滚滚,凝而不散,代州军逼近雍军后阵,却不冲锋,只是游弋往来,不时用弓箭猎杀猎杀落后的雍军骑兵,偶尔还有胆子大的勇士冲入雍军军阵,厮杀一番再退去。搅得雍军不安宁,李显眼中寒光一闪,提着马槊亲自到了阵后,有了他压阵,雍军胆气立壮,也开始凌厉的反击。两军就这样纠缠不休,却都没有放慢速度,日影西沉时候,雍军前锋已经进入宣松布下的第一道防线。

两山对峙的山谷开口,是沁水的河道,河道两边是可以容得下骏马奔驰的崎岖山道,寒水幽鸣,两侧怪石嶙峋,这一带的群山都是石山,山上植被稀疏,岩石坚硬,难以穿凿,无法修建固定的寨垒,两侧悬崖峭壁,距离沁水足有数十丈的高度,虽然临水,却是取水困难,难攻可也难守,所以当初北汉军没有在这里固守,与其在这狭窄之处消耗实力,不如在平坦之处更可以发挥骑兵的实力。不过如今防守的是雍军,雍军的步兵用来防守临时搭建的工事最好不过,虽然因为种种限制,不可能长期固守,但是只要每一处守个一日半日,就可以拖延北汉军的进攻速度。而这一点也正是龙庭飞担忧的,他不希望当自己苦心孤诣地攻破雍军防线之后,却遇到雍军大量的援军。按照正常的方式估算,从兵败消息传到泽州,泽州集结兵力到发援军,至少也需要半月时间,这是事先有所预备的情况,但是也不无可能,因此龙庭飞带着北汉军主力匆匆赶来,和林碧汇合,若是不能将李显留在此处,就需要加速攻击,一定要在十日之内将雍军迫到安泽,这样才有可能完成全歼雍军的目标。

一个青衫儒将站在一侧的山峰上,山谷外早有严阵以待的雍军用弓弩压阵,接应雍军骑兵入谷,井然有序,全无一丝紊乱。这时候北汉军业已觉察到时间紧迫,他们的攻击也越来越猛烈,若非李显和荆迟两人亲自殿后阻截,只怕雍军后阵早被攻破了。血红的夕阳在天际欲沉还止,晚霞好似艳丽的血花一般凄艳,两军竭尽所能得苦战着,全然不顾牺牲,无数勇士的生命谱写成最壮丽的战火画卷。

雍军已有三分之一进入了山谷,就在这时,沁水上游出现了北汉水军的艨艟斗舰,顺着湍流的河水飞速冲下,船上的水军都是执盾携弩,显然是准备利用沁水冲入山谷,使用弩弓截断雍军的后路。远远望见水军的旗帜,北汉军都是声威大震,攻击也越发得心应手,雍军虽然有些不安,可是毕竟是百战雄狮,初时还有些不安,但是很快就稳定下来,只是退兵的速度似乎加快了许多,对北汉军的抵抗也不免松懈了一些。

就在为首的三艘战船将要接近谷口的时候,那在山峰上指挥的青衫将领挥动旗帜,那三艘战船船首似乎撞在了什么阻碍之上,前行无力,船身不由被水流冲得倾斜过去,不过片刻,那三艘战船就将河道堵住大半,战船上面的北汉水军毕竟不是久经水战的楚人,不由混乱起来,这时候,谷口的雍军军阵中推出几十架弩机和投石车,箭矢和巨石如同雨点一般袭去。北汉水军中军传出号令,那些水军奋勇还击,但是船只不能移动,船身倾斜也让北汉军无力反抗。过了片刻,水军传出撤军的号令,那三艘战船上面的水军纷纷跳水或者乘坐小船退走。

龙庭飞剑眉深蹙,不多时有斥候回报道:“将军,雍军在河面上安了拦江铁索,方才水军冲锋之时,雍军用铰链将铁索拉起,挡住我军战船。”

雍军缓缓进入山谷,龙庭飞目视雍军大旗消失在视线中,不由恨声道:“雍军手段果然高明,在退路上花了这么多心思,想不到数日之间,竟连拦江铁索也打造了出来,可惜,否则若是我水军阻住山口,雍军休想逃走。”

段无敌在一旁劝解道:“将军不必忧心,虽然不如我们预计,可是从另一方面说,雍军也是后援无力,否则他们何妨将我们放过山去,在安泽以逸待劳,大破我军,现在他们守得严密,正说明实力不强,想利用地利消耗我军实力,可是这一带我们比他们更熟悉,只要尽快攻破他们的防线,利用我军擅于冲锋追猎的长处,一定可以将雍军消灭,李显生性顽强,绝不会弃军而逃,我们还有机会将他留下。”

龙庭飞眼中闪过绝决的神色道:“若不能擒杀李显,我们虽胜尤败,传我将令,放火箭毁去堵路战船,铁索可以用火烧溶,让水军去做,就是将三十里山川化成火海,我也要让雍军没有容身之处。我从前令你准备黑油和硝石,只需将黑油倾倒在沁水上,一把火就可以逼退山谷中的雍军。我给你两天时间,你可能作到。”

段无敌心中一凛,这黑油乃是古怪之物,不沉于水,易燃,火势经久不息,只是燃烧之后黑烟缭绕,被黑油渗透的土地寸草不生,龙庭飞此举虽然狠毒,可是这三十里荒山和沁水下游,必然受损严重,只是如今却也顾不得了,他躬身道:“将军放心,末将必不辱命。”

\chapter{第二十五章 火烧沁水}

雍军败退,以铁索拦江,阻住北汉水军,山势险要,难攻难守,两军争夺谷口两日,不分胜负。

四月初一,龙庭飞命麾下段将军以黑油沉江,烈火焚之,雍军败退,死伤迭见。后三十年,山中不见寸草,越明年,沁水乃清。

——《资治通鉴·雍纪三》

三月三十日清晨,李显从军帐中走出,虽然已经是春天,但是清晨的温度仍然很低,江风清冷,雾气蒙蒙,沁水寒凝,李显凝神苦思,这一处山谷中可藏兵近万,是距离北汉军屯兵的沁水谷口最近的军营,昨天晚上,雍军就在沁水沿岸的十几个这样的山谷里面扎营,从今天开始,就要在步兵的支援下退兵了,这一带山谷并不是好的拒敌地点,虽然用步军防守北汉骑兵很合算,可是李显从来不喜欢这种没有胜利可能的牺牲,所以退兵是唯一的选择,而且谁知道北汉军会想出什么法子攻打呢,毕竟这样的山谷对雍军铁骑也是一种束缚,最重要的一点,想要胜利,就不能在这里据守,只不过退兵的时机要巧妙,不能让北汉军看去自己根本就没有打算据守山谷,当然损失也要越小越好。一边仔细想着如何应对目前的战局,李显负手走向不远处的营帐,那是宣松的营帐,李显愤愤的想,昨日太忙了,只听说江哲先走了,一定要问清楚宣松,这个家伙是如何临阵脱逃的。

走近宣松的营帐,帐内却是空无一人,想必是出去安排防守了,李显也没有在意,径自走了进去。宣松身为大将,营帐自然是颇为舒适,内外隔着帷幕,内间是行军床榻,外间是桌椅,地上铺着厚厚的毛毡。李显坐在椅子上,心里想着如何才能撤退的干净利索,这时,他听到外面传来脚步声,一个不急不缓,脚步清越,一个龙行虎步,威猛沉重。李显听出这两人乃是并肩而行,想也知道是宣松和荆迟一起前来,他突然心中一动,这两人都是皇上心腹,又是多年同僚,想必有不少知己话要说,自己何妨听听他们私下里面说些什么呢?

李显心意一定,就掀开帷幕走进内帐,他的身形刚刚隐入帘幕后,军帐的帐门就被荆迟挑开,他大步流星地走了进来,径自坐到书案边上,将书案上的茶壶倒了一大杯清茶出来,一口喝个干净。宣松在后面跟了进来,看到这种情景,摇头道:“将军还是喜欢这般牛饮,真是可惜了这上品的贡茶,这可是前些日子监军大人下棋输给我的好茶啊。”

荆迟一听到“监军大人”四个字,一口茶水立刻喷了出来,哈哈笑道:“原来是下棋赢得,那可就容易得很了,当年天策府上下谁不知道江先生才华虽然绝世,偏偏就是棋艺平平,有一次输得惨了,便吟了一首七绝谢绝对弈,我虽然是老粗可也还记得。那首诗是这样的,‘平生事物总关情,雅谢纷纷局一枰。不是畏难甘袖手,嫌他黑白太分明。’”

李显在帐后几乎笑出声来,这件事情他却是知道的,甚至他还知道荆迟之所以记得这么清楚,实在是因为那日荆迟在旁边随侍,忍不住嘲笑了江哲几句,江哲便罚他将这首诗抄了百遍,昔日雍王府关防虽然严密,可是凤仪门仍然在雍王府中有些探子,这些事情就是李显从秦铮那里看到的,不过后来雍王府那边却是越来越森严,到了最后,竟是很难得到什么有用的情报了。

宣松自然不知道这段隐秘,倒是长叹道:“楚乡侯性情随和,淡泊名利,却是忠心王事,鞠躬尽瘁,昔日曾闻江大人因为劳顿而几乎病重不起,松本来只是耳闻罢了,想不到昨日才见到颜色,江大人昨日离去之时,几乎不能亲自上车,想必是疲累已极,我等只能尽心竭力完成江大人定下的计策,否则上负皇恩,下负江大人苦心。”

李显闻言身躯一颤,当然猎宫之变,他可是亲眼所见,晓霜殿上,江哲形销骨立,病骨支离,两鬓星霜,几乎是奄奄一息,而当他在东海重见江哲,虽然江哲已经恢复了健康,但是那一头灰发,两鬓微霜,仍然让李显心中黯然,这些日子以来,江哲虽然表面松懈,可是李显却是知道江哲经常阅读各种情报直到深夜,更是亲自处置安排了许多看上去莫名其妙的事情,不过李显却深知江哲布局的本事,自然不会以为江哲是在偷懒。昨日听到江哲先退走,李显也不过是有些轻微的怨气,毕竟他也知道江哲的身体恐怕经不起溃败的路途,所以并没有真的恼怒,可是闻听江哲临去之时竟然如此虚弱,心中不由忐忑不安,若是江哲旧病复发该如何是好,不说自己心中难安,就是皇上和长乐公主那里也是交待不过去的。

他心思一乱,气息立刻沉重起来,外间的荆迟听到江哲身体状况有些不好,原本也是愁眉不展,听到内间有声息,心中一惊,伸手按住刀柄道:“里面什么人,为何在此偷听?”

宣松本是儒将,武功平平,听到荆迟喊声,立刻起身向帐门移去,若真的有刺客或者密谍,那么他自然不想拖累荆迟出手的,却见内帐帘幕一跳,齐王李显走了出来,面上神情冰寒,淡淡道:“宣将军,立刻令我军整顿行装,按照计划开始撤退,本王没有心情和北汉人耗着。”

宣松和荆迟都是一怔,但是见到齐王神色不快,再说上下之分摆在那里,也不能指责这位王爷听壁角,连忙应诺,下去安排军务,原本计划是要在这里守上两三日,再大举撤军的,但是齐王既然要改变计划,宣松又觉得影响不大,便也没有谏言。

这时候,日头已高,前面谷口之处,北汉军已经开始挑战,为了不让雍军疑心,北汉军在谷口连番攘战,而且在外面造攻击的器械,全没有露出一丝破绽。若是换了平常,李显或者会亲自上阵和敌军对峙,但是他听闻江哲生病后,便是心情郁闷,也懒得上阵,只让荆迟带军出去对敌。

北汉军阵上,龙庭飞和林碧并马而立,望着在谷口对峙的两军,神情都有些失落,良久,龙庭飞黯然道:“雍军昨日大败,可是不过一夜,就再也看不到颓废气象,雍军心志之坚,我军不及。”

林碧心中也有同感,道:“大雍如今上有明君,下有良将,将士用命,皆愿效死,只可惜我北汉屈居一隅,虽然上下一心,却是力不从心。”

龙庭飞笑道:“碧妹也不必如此,只要我们这次擒杀李显,大雍损失惨重,数年之内别想进兵沁州,到时候,我们再用合纵之策,和南楚、东川联盟,到时候,大雍再也不会有今日的威势。”

林碧微微一笑,她知道龙庭飞不过是劝慰她罢了,大雍岂是那么容易崩溃的,她心中有更深的忧虑,这次代州出兵她是答应了父兄的,一定要在四月二十日之前赶回代州,蛮人蠢蠢欲动,代州只有一万骑兵,虽然代州军民已经夜夜枕戈,但是大哥、二哥都是猛将,而非大将,父亲又卧病在床,自己怎能放心得下。

谷口两军交战正酣,荆迟麾下一个青年偏将最是骁勇,几次冲入北汉军阵,舍生忘死,全身而回雍军都是大声为他鼓劲,龙庭飞眉头一皱,正要吩咐派人将敌军这个偏将斩了,萧桐匆匆赶来,低声禀道:“将军,让鹿氏兄弟上阵,那个偏将乃是我们的人,他定是有急信要传。”

龙庭飞神色一动,高声道:“伯言、仲天、叔函你们率军上去,一定要把这个偏将给我擒杀。”鹿氏三兄弟早就跃跃欲试,连忙同声应诺,萧桐早已退到一边,在鹿叔函身边说了几句吩咐,鹿叔函眼中寒芒一闪,跟在两位兄长后面出阵而去。

很快三人就冲到了前面,鹿伯言和鹿仲天亲自迎住带军冲杀的荆迟,鹿叔函则是有意无意的带军挡住了那位偏将,敌我两军都道这是北汉军立威之举,也没有觉得有什么古怪,那偏将举起马槊冲来,人借马势,也是势不可挡,鹿叔函乃是不下于当年先锋将军苏定峦的猛将,冷冷一笑,马槊一挑,那偏将一声惊喝,手中兵刃脱手,鹿叔函一声厉喝,马槊横扫,正好击中那偏将的腰肋,将他扫下马去,但那偏将却不服输,人落马下却是纵身跃起,鹿叔函却举起马槊向下刺去,眼看着就要将那偏将的咽喉刺穿。那偏将凌空翻转,马槊擦过他的面颊,刺入泥土当中,那偏将也是站立不住,跌坐在地上,但是那偏将却一扬手,一柄霜刃飞刀如同流星电闪一般射向鹿叔函面门,鹿叔函闪躲不及,却是张口一咬,将那飞刀截住。就在这一瞬间,那偏将已经被冲上来的雍军救走。

两军混战,处处都是厮杀,但是两人这一番干净利落的交锋仍然让众人看在眼里,两军本都是铁血男儿,最尊重勇士,何况鹿叔函攻得猛烈,那偏将也是矫捷如同灵狐,虽然落败却也没有丢多少面子,所以不论雍军还是北汉军都是同声喝彩。这时,北汉军也已经挽回颜面,两军缠斗已久,眼看着日上中天,双方各自鸣金,都是缓缓退去。

回到北汉军大营,摒退众将,鹿叔函将那枚飞刀交给萧桐,萧桐轻轻旋转刀柄,那刀柄却是中空的,里面塞着一张纸卷,上面写着慢慢的蝇头小字。

“军中传言,楚乡侯旧病复发,已返泽州,齐王决意提前退兵,今日午后开始。”

看完上面的情报,龙庭飞神色忧喜交加,他无声地将纸卷递给林碧,手指轻轻敲击着书案,似是陷入了沉思之中。

良久林碧抬起头道:“若是楚乡侯病发属实,那么这就是最好的机会,雍帝和齐王之间全靠此人缓冲和解,楚乡侯卧病,此刻齐王必定心中不安,所以才会加速退兵,这样一来,雍军不免军心不安,行军急躁,我军若想取胜就会容易许多。”

龙庭飞皱眉道:“可是此事很难判断真假,而且雍军加速退兵,我们火攻之策就不免效果差了许多,萧桐,你说这份情报可否属实?”

萧桐恭谨地道:“此人乃是我魔宗旁系弟子,他是北汉人,父母亲族都在晋阳,两年前我军劫掠泽州的时候,血洗了一个村落,属下令其取代了其中一个被杀的村民的身份,两年来从未动用这颗暗子,所以属下相信此人身份绝对不曾泄露。而且他冒险传回的情报事关重大,却是简略粗疏,也符合他的身份,昨日荆迟才和雍军会合,这些事情此人绝对不可能知道得很详细,此人聪明果敢,若是虚实难辨,是绝不会这样冒险的。”

龙庭飞和林碧都是默默点头,两人四目相对,都是心意已决,龙庭飞起身道:“传令无敌,虽然黑油尚未全部送到,可是也顾不得了,今夜开始火攻,然后我们追袭雍军的时候,不妨散布些流言,就说楚乡侯故意陷害齐王落败,如今又临阵脱逃,到时候雍军必然心乱,说不定李显也会这样想呢。”

夜深人静,沁水之上,千余北汉军都穿了深色夜行衣,轻手轻脚地将一桶桶黑油倒入沁水,夜色深沉,星月无光,幽深的沁水上面盖了厚厚一层黑油,黑油向下游淌去,丝毫没有引起谷中雍军的注意。龙庭飞和林碧站在岸边,两人都是神色凝重,据他们估计,一日之间,雍军至少已经撤走三分之一,若是再不巧被巡夜的雍军发觉河内黑油,那么胜算就更加小了。

段无敌走近两人身边,低声道:“根据水流的速度,大概四更左右可以遍及三十里河道,公主、将军,我们需在那时点火。”

林碧轻轻点头,微微一叹,她在代州虽然也是杀伐决断,但是却多半是刀锋喋血,这种火烧水攻的手段却几乎没有用过,心中未免有些忐忑,毕竟代州英杰,最喜欢光明正大的沙场交锋。龙庭飞却是神色从容,道:“好,希望这一把大火可以烧毁雍军的勇气和信心。”

山谷之中,雍军大帐灯火通明,今日在李显的坚持下,撤走了两万步兵和万余骑兵,李显、荆迟和宣松三人正在彻夜商讨如何撤兵,所以直到深夜仍然没有休息。全然不知沁水中杀机隐藏,水流湍急,今夜风势沿河而下,那些黑油又经过处理,所以并没有刺鼻的气味,因此无人发觉这潜在的危机。

三更天,大雍军营已经几乎没有了声息,除了防守谷口,唯恐北汉军趁夜偷袭的守军之外,所有人都在沉睡,这时候,从一座小帐篷里面走出两人,这两人都穿着青色甲胄,但是营帐旁边的火光映射下,却看出这两人身姿纤弱,原来却是两名女子。这两人正是苏青和她的亲信侍女如月。

苏青多年来出生入死,能够履险如夷,虽然多半凭着武功智谋,可是还有一项长处人所难及,就是对于危险的敏感,有些事情虽然还未发生,甚至端倪还没有显露,苏青往往就能有所警觉,虽然往往只是心中不安甚至心悸,可是却几乎是次次灵验,这也是她能够凭着一个女子之身纵横北汉的关键所在。今夜她半夜便被噩梦惊醒,起来之后发觉浑身都是冷汗,因此立刻唤醒如月,穿上甲胄,走出营帐,虽然不能凭着自己的心绪而示警,但是至少她可以去查一查,是否有什么不妥之处。

她在军营中缓缓行走,巡视的军士见到她都是躬身行礼,苏青一一还礼,心思却是不知飞到何处,她专心致志地四处打量,希望能够找到让自己心生警兆的踪迹。但是她能够感觉到的只有凝重和沉静,心中渐渐涌起一丝焦躁,苏青转身走向沁水,在夜里坐在河边听听水流的呜咽,应该是涤清心中烦恼的最好的法子吧。走到河边,苏青深深的吸了一口气,冰冷的空气吸入肺腑,苏青突然一皱眉,空气中有一种淡淡的熟悉气味,刺鼻而辛辣,她眼中立刻露出冰寒的光芒,目光闪动,渐渐落到了河面上,苏青的脸色突然惨变,想也不想地回身向大帐走去,不能惊惶,不能惊动大营上下,否则黑夜之中会有炸营的危险。

齐王的大帐灯光已经熄灭,苏青走到帐外,看见在外面守夜的是齐王心腹的侍卫庄峻,她匆匆上前,低声道:“殿下何在,末将有紧急军情禀报。”

庄峻眼中闪过一丝惊异,不明白为何苏青神情如此凝重,但是他知道苏青乃是斥候好手,所以连忙冲进营帐,不多时,齐王披了战袍走了出来,火把的光芒照射到苏青面上,清艳的容颜苍白如雪。当听苏青禀明军情之后,李显眼中闪过炽热的火焰,他立刻令人层层传令,所有雍军立刻撤走。他们不知道北汉军什么时候发动,但是苏青说得很清楚这种黑油燃烧的烟是有毒的,就是避入两边的山谷也是难逃危险,而且等到北汉军攻入的时候,恐怕是瓮中捉鳖,死在绝地,所以不论如何,只有一个逃字。

幸好雍军这两天都是人不解甲,马不卸鞍,所以不到半个更次,就已经全军准备好了,而动作最快的一批已经上路了。李显望着那些神色迷惑的步兵,他们很难及时撤退的,原本留下他们是为了更好的防守,也是为了不让北汉军发觉撤军的内在意图,可是这些大好男儿却要屈辱的死在这里,虽然不知道北汉军什么时候发动,但是天明之前,这些人很难逃出山谷,道路,太狭窄了。可是,李显心知现在不能说明真相,如果给这些军士知道陷入必死绝境,恐怕会有一场混乱,到时候惊动了北汉军,只怕一个人都逃不出去。李显心中一横,道:“宣松,派个人率领他们在谷口等候,就说等到黎明时分偷袭北汉军营,如果火起,就带着他们冲出谷口,攻击北汉军,你挑一个肯赴死的去。”

宣松心中一痛,却知道非得如此,上前施礼道:“殿下,这些军士末将指挥多日,不如由末将亲自率领他们出击,也免得选错了时机,平白牺牲。”

李显怒道:“胡说,你是军中大将,本王正要倚重,焉能为此必死之事。”

宣松道:“殿下倚重末将,不过是为了阻截北汉追兵,殿下从前也擅于败退,末将并非必要的存在,倒是为了和北汉追兵血战,需要荆将军这样的武将,而且如今若无大将殿后,只恐军心生变,末将乃是最适合的人选,何况这一次失策,都是末将没有察觉敌军阴谋,末将理应留下戴罪立功。”

李显听后,只觉得心痛难忍,但是他深知若无宣松这样的大将殿后,果然是军心易乱,眼中闪过痛惜之色,他低声道:“也好,荆迟,我们出发。”说罢上了战马,头也不会策马奔去,荆迟略一犹豫,也只得跟了上去。敌军欲用火攻之事,只有齐王和少数将领知道,所以雍军没有丝毫混乱,只道齐王决定连夜撤军罢了。策马行了一段路,李显突然转身奔回,指着宣松道:“宣将军,此间之事,由你便宜行事,不可轻言殉国,若有差池,皆有本王担待。”宣松身子一震,知道齐王暗示他紧要时候可以投降,好保住性命,虽然这不是他所能作出的事情,但是他仍然俯身下拜道:“末将遵命。”语气中隐隐带了悲声。

当齐王的身影消失在夜色中之后,宣松恢复平静的面容,道:“黎明时分准备袭营,现在传令下去,三军开拔。”这时候夜色仍深,宣松令三军衔枚,然后又让众人用浸透了山谷中清泉水的巾布裹住口鼻,又让心腹亲卫走在河边,再加上光线黯淡,竟然无人发现河中玄机。虽然一些机灵人已经察觉不妥,但是军令如山,此刻若是宣扬起来,不免立刻成了刀下之鬼,也只能不声不响,跟着大军行动。不多时雍军已经到了谷口,宣松令心腹亲卫出去查探,那亲卫回来已经是面无人色,低声禀道:“将军,敌军大营离此不远,我看见很多人影在河边。” 这个亲卫已经知道实情,自然知道其中凶险。

就在这时,突然谷外火光乍起,顷刻间身边的沁水上已经是烈焰滚滚,含有毒性的黑烟向岸边涌来,山谷之中黑烟弥漫,对面难见人影。宣松令人击鼓,鼓声沉沉,犹如被陷入绝境的野兽悲嚎,此刻就是没有宣松的军令,面对身后的死亡,也是只有一条道路好走,雍军按照军令向谷外冲去,但是山谷狭窄,只能鱼贯而出,即使到了此刻,雍军仍然颇有章法,没有彼此拥挤,可见训练有素。不多时,前面响起惊呼声和兵刃撞击的声音,宣松眼中闪过泪光,这是自杀之举,两万雍军步兵对着十万北汉铁骑和代州军,那是必死无疑。他口中低声道:“楚乡侯,末将辜负你的期望,没有看穿敌军火烧沁水的阴谋,若是末将早些发觉,不论如何还有法子应对,如今却是只能以死赎罪了,希望你的计策成功,为我大雍男儿报此深仇。”抬起头来,拔出腰间长剑,他在亲卫保护下向前冲去,奔向前方的死亡之所,即使是死,他也更希望能够死在北汉军军阵之中。在他身后,沁水上面的火势转瞬数里,还在飞速的向前蔓延,下面是寒水,上面是烈焰,黑烟滚滚,毒气朦朦,两侧的草木被大火烧着,火势更加凶猛,岩石被黑烟熏得漆黑,若有人在此,绝无生还希望,三十里山川变成了修罗场,烈火将一切生命吞噬。

北汉军火烧沁水,除了先撤走的万余骑兵和两万步兵,齐王、荆迟麾下仍有骑兵三万众生还,只有千余人未几逃出,被火海吞噬,因出发及时,再加上黑油不足,所以雍军主力幸存,然两万步兵自杀性的袭击,除了造成千余北汉骑兵的死亡之外,全军覆没。至此,雍军北伐军十三万步骑,只余半数残军,虽然主力骑兵尤存,但是北汉军已然占据了绝对的优势。

\chapter{第二十六章 四面绝网}

夜寒如水,冀氏之野的一座小山村,村民早已被逐走,只留下空荡荡的屋舍。数日前,这里有了临时的主人。村中最宽敞的一间农舍之内,烛影摇红,灯花乍碎,简陋的木床上铺着华丽的卧具,一个青衣书生倚在榻上正慢慢喝着一碗散发着清香的药汤。

将药碗递给榻前侍奉的青衣少年,我一声长叹道:“人算不如天算!岂料北汉将领如此辣手,宣松之事,真令我痛心疾首,小顺子,后来战事如何?”

小顺子低头道:“龙庭飞对我军突围之举早有防范,我军从谷口突围,用投石车和弓箭封住谷口,拦截我军,谷口狭窄,难以穿行,仅数千人冲出谷口,死于北汉军重围之中,余下众人皆被火焚而死,焦骨遍野,我军斥候没有探明宣将军生死,但是想来恐怕已经死在乱军之中。”说到此处,见江哲容颜惨淡,他劝慰道:“公子本不是前方将领,这并不是公子的责任,何必愧疚。”

我苦笑道:“并非我自寻烦恼,宣松乃是难得的人才,难得的是能攻能守,千军易得,一将难求,损失此人,纵然大败北汉军,也不过是两败俱伤罢了,叫我怎么不心痛。唉,我虽然也想到敌人可能用火攻,可是沁水河谷树木稀疏,水流充足,火攻并不容易,所以我也没有提醒他们注意,可是想不到龙庭飞会用黑油倒入沁水,作为助燃之物,若非苏将军发觉,只怕全军覆没,龙庭飞果真不同寻常。”说到后来,我越发心中郁闷,不由轻咳了几声,小顺子连忙捧过茶杯,我就着茶杯喝了一口水,觉得舒坦了许多,又问道:“殿下如何应付下面的战局的?”

小顺子看了一眼手上的薄绢,道:“齐王殿下亲率大军在沁水河谷谷口伏击,四月二日,谷中火熄之后,龙庭飞留段无敌镇守沁源,亲率北汉军出谷追击,被殿下伏击得手,北汉军兵力强大,两军缠战半日,殿下退向安泽。四月三日,殿下利用安泽地势不利于骑兵作战的条件,使用步军再次和北汉军交锋,并无胜负,四月四日,殿下到了冀氏之北,正在阻击北汉军追兵,好让步军可以撤回泽州,两军对峙已经有两日了,虽然北汉军损失惨重,但是殿下也是损失非轻,明日殿下就会全军撤退,全速行军,不再和敌军纠缠。”

我眼中闪过一丝喜色,道:“大胜之后两次遇挫,想必北汉军不会轻轻放过我军的。”

小顺子淡淡道:“公子说得是,我听说北汉军战得很凶狠,齐王殿下两次撤退都几乎被敌人合围,这一次撤军,敌军不仅会追击,还是不死不休,就是追入泽州,也不会轻易放过。”

我闻言拊掌道:“齐王殿下果然明白我的心意,龙庭飞本是心性高傲之人,昔日泽州败战之后,又被我设计消磨其心志,如今借助大胜,挽回了荣耀和信心,齐王殿下不顾兵力处于弱势,摧敌锋锐,龙庭飞必然不能容忍,这一番追击势不可挡,却正是入我彀中。不过若非齐王殿下心志坚毅,百折不回,谁能够完成这艰难的任务呢?”

这时,赤骥进来禀报道:“公子,长孙将军在外求见。”

我淡淡道:“请他进来吧。”目光却望向不可见的远方,现在正是最重要的时刻,如果龙庭飞生出疑心,撤军而回,我军可就是白辛苦了一场。这时的我自然不知道“楚乡侯病重”这个被夸大的情报带给北汉军的影响,它让北汉军上层几乎没有任何怀疑地冲进了陷阱。

伸手抚摸战马被汗水打湿的鬃毛,李显抬头望向后方,北汉军暂时没有见到影踪,抬头看看,日正中天,想必敌军是准备休息一下吧,这几日他可是万分辛苦,挑衅的后果就是敌人的拼死追击,即使已经将到冀氏,五十里之外就是泽州边境。不过虽然只有五十里的道路,却比前面的路程都要艰险,之前逃亡的时候,可以迂回转进,虽然敌军有两倍以上,可是想要围攻还是比较困难的,只要自己灵活一些,敌军想要合围是不可能的。可是接下来的五十里,就只能快马奔驰了,若是再四处流窜,只怕会被敌军发觉一只脚已经踏入圈套。

匆匆喂过战马,李显看见后面烟尘再起,振奋精神道:“我们一鼓作气,回泽州去,不用列阵,大家自己逃吧。”说罢扬鞭策马冲了出去。荆迟在后阵得知军令,看看烈日,愁眉苦脸地道:“走吧,谁若是落在后面,可就被敌人合围了。”

这些日子,李显和荆迟两人充分利用了齐王旧部和雍王旧部之间的不合,交替充当冲锋断后的角色,因此冲锋者往往不顾生死,犀利狠辣,断后者也是浑身带刺,令敌人不能轻易接近。两人都是明里暗里的示意下属,如今败是败了,若是再输给对头,那么可是面子里子都没了。所以虽然连遭惨败,军中士气倒是越来越高涨,若非敌人也是非常的强大,又有代州军助阵,恐怕混杂半数新兵的北汉军还会被反咬一口呢。不过尽管如此,兵力上的差距仍然让雍军不断后退,如今已经进入了最后的逃亡阶段,李显又下了溃逃令,所有雍军都是自顾自地开始溃逃,虽然多年行军作战的习惯,让雍军仍然保持着一定的军阵,可是几乎是漫天遍野的零散军阵,让敌人没有了一定的目标,这也就增加了敌军在追击过程中合围的难度。

追上来的龙庭飞和林碧,看着溃逃的雍军,都是发出由衷的笑容,五十里路程一马平川,若是不紧紧追击,只怕会被雍军逃回泽州去,不过两人对于骑兵作战都是心中有数,也知道这是敌军最后的手段,溃逃令可以令逃跑的军队拥有最快的速度和最不可预测的逃亡方向,可是一旦下了溃逃令,就是只能逃跑不能反击了,想要全歼敌军,这是最后也是最佳的机会。龙庭飞眼中闪过坚毅的光芒,道:“碧妹,代州军马快,你亲自率军绕到敌军前面去,我率大军在后追击,如今敌军已经溃逃,不可能有反击之力了,我们只需留住敌军大半,就可以达到目的。到时候若是齐王逃了,我们最多直接攻入泽州去。”

林碧轻轻点头,全歼雍军是北汉军将士的一致要求,不说雍军在沁州的大肆烧杀,只凭着水淹安泽、火烧沁水两战,北汉军虽然大胜,可是却是牺牲了己方重镇和境内山川,北汉军上下都是恨恨不平。而四月二日,当北汉军穿过余烟未熄的沁水河谷,本以为雍军已经远逃的时候,却被齐王当头一棒,损失不小,接下来更是被齐王左冲右杀,迂回挑衅,弄得颇为狼狈,军中上下都想生擒齐王,取得最辉煌的胜利,若是现在退兵,只怕是士卒生怨,将士离心,所以追击成了唯一的选择,也是最好的选择。

林碧接了军令,带着代州军绕开雍军奔逃的方向,从侧面向沁州、泽州边境赶去,代州军战马精良,又都是骑术高明的战士,速度要比雍军和北汉军主力都快些,正是最适合围追堵截的军队,前番若不是李显所选的战场巧妙,又仗着兵力远远超过代州军,几次强行突破代州军的防线,而林碧在仍有足够的机会全歼雍军之下,也不想损失过重,恐怕雍军早就被围歼了,即使如此,代州军铁蹄之下,也留下了无数雍军勇士的尸骨,代州军马,天下无双。

李显策马狂奔,现在不需要顾惜马力了,护在他身旁的亲卫却都是眉头紧锁,他们尚不知道泽州方面的接应如何安排,自己败退沁源之后,他们和后方的联系就人为的中断了,所有消息往来,只有李显一人知晓,在溃逃之际,前途的茫然最令他们心忧,荆迟则是带着亲卫处于溃逃雍军的尾部,他手上有一支三千人的精骑,维持着比较完整的编制,如果北汉军追得过于接近的话,他就可以发动反击,不过北汉军合围在望,也不想平白消耗军力,所以一路上两军都没有发生交战。而在荆迟身边多了一个较为陌生的面孔,是一个叫做戴钥的年轻偏将,上次沁水河谷北面谷口一战,戴钥和北汉猛将鹿叔函交战,虽然是大败而归,可是他的敏捷和机灵到让荆迟颇为赞赏,因此将他留在了身边。此刻的荆迟自然不知道自己留下的是危险的敌人。

经过艰苦的跋涉,李显知道已经接近了泽州边境,他心中一边嘀咕,怎么没有看到接应的军队呢,一边埋头狂奔,这时候,前面突然有雍军匆匆奔回,惊道:“殿下,不好了,前面有代州军拦截。”李显停住马匹,心中暗暗苦恼,想不到代州军马这么快,想必他们是沿着雍军溃逃的外围赶过来的,自己已经几乎是在雍军的前锋了,还是被代州军截住,这样若是没有援军,岂不是要全军覆没。他可不想奢望在这里冲破代州军的拦阻,这里不是沁河谷口,阻住谷口就可以挡住北汉军出来,这里也不是安泽,那里道路泥泞,马速被拖累得相差不大,这里可是除了秦泽之外,泽州和沁州交界处最适合骑兵作战的原野啊。李显心里暗中诅咒江哲道:“姓江的,你若是没有准备好伏兵,就等着给我收尸吧,本王还没有嫡出的郡主,你的儿媳妇还没有出世,若是本王死在这里,作鬼也要咒你儿子一辈子娶不上媳妇。”口中却是懒洋洋地道:“好了,就在这里汇集军队,本王去见见那位嘉平公主。”说罢向前奔去,心道,反正等不到后面的追兵到达,代州军也不会轻易发动,我不如去见见林碧,说几句闲话拖延一下时间吧。

林碧站在阵前,代州军虽然阻到了雍军前面,可是也是刚刚列阵完毕,全军上下更是马困人乏,所以也无心在此时立刻出战,看到雍军往后退去,也是并不追赶,林碧休息了一会儿,觉得精力已经恢复,就静静等待着决战时刻的到来。这时候,她眼中看到一支红色的骑兵,齐王在亲卫簇拥下赶到了,隔着百余步距离,确保可以随时逃跑之后,李显大声笑道:“嘉平公主殿下,你率大军来相助龙将军,就不想想代州安危,若是蛮人南下,只怕代州将成血海,那么公主可是得不偿失了。”

林碧面上神色一黯,高声道:“大雍攻我疆土,清野血洗,屠城破关,不比蛮人好到哪里,若是不能留下王爷,代州军绝不还乡。”她的声音清越如同银铃,即使是充满了杀机,也是令人怦然心动。李显肃容道:“公主何出此言,这些年来,我们两国征战不休,你们打过来,就要血洗泽州,我攻过去,自然也要杀人报复,但是代州军历来不曾参与两国征战,只是守护大好河山不被蛮人侵扰,何必介入这争权夺势的无益之战呢?”

林碧面上一红,这种想法她也有过,代州军上下都对雍军和北汉军之间的征战毫无兴趣,可是代州军受北汉国主重恩,如何推却国主的请求,自己又是国主义女,龙庭飞未婚妻子,怎能拒绝这出兵的要求。见她不好答话,从军中飞马奔出一个青年将领,正是林碧兄长林澄山,乃是林远霆第三子,代州军将领,他冷冷道:“两军作战,王爷何必多言,若是不想交锋,王爷只需下马受缚,想来以王爷身份尊贵,国主也不致相害。”

李显微微一笑,心道,我李显岂是受缚之人,再说若是随云安排妥当,成了阶下囚的还不知道是谁呢?也不再言语,策马向后,退入雍军之中。雍军便在距离代州军二里之外开始集结,代州军虽然知道,但是一来还没有恢复过来,二来若是急急进攻,担心李显脱逃,所以只是守稳了去路,等着北汉军主力到达。

双方对峙了不到小半个时辰,雍军已经集结了大半,代州军开始了零星的游猎,不允许雍军列好军阵。双方缠斗了片刻,代州军骁勇,雍军虽然也不差,但是很多军士还落在后面,散漫的军阵也造不成足够的威胁,当后方荆迟也赶来之后,雍军开始向代州军猛攻,只是被代州军侵扰之下,战阵散乱,不免攻击软弱。在林碧的指挥下,雍军很快就不得不再次退后重整。就在这时,后方传来号角长鸣声以及铁蹄踏碎山河的轰鸣声,虽然隔着很远,可是林碧却一眼就看到了那猎猎飞舞的龙庭飞帅旗,代州军高声呼喝,不多时,从北汉军阵中也传出来相互呼应的长啸声,号角声,北汉骑士的呼喝声溢满天地,北汉军,终于合围了。

龙庭飞望见李显的帅旗,终于放下了心事,冷冷道:“传令,围歼!”随着他的一声号令,决战开始了,代州军和北汉军配合默契,将雍军围在当中,虽然北汉军不过是雍军的两倍,但是代州军擅长游弋猎杀,他们在外围转动,一旦有雍军冲破北汉军的空隙,就用弓箭射杀,有效地阻止了雍军突围的意图。雍军虽然苦苦支撑,可是活动的范围却是越来越小。这时候,李显已经暗中痛骂不止了,若是再这样下去,自己可真要全军覆没了。突然一个古怪的念头涌上心头,这不会是江哲故意的吧,或者他是奉了皇兄之命想要消减自己的军力吧。

就在李显心中惴惴不安的时候,荆迟遭遇到了危机,荆迟素来喜欢亲自冲阵,这一次也不例外,可是不同的是,他身边多了一个心怀不轨之人。

那名偏将戴钥,在作战时紧紧跟在荆迟身边,旁人只当他新得升赏,感恩涕零,一心保护荆迟 罢了,却不知他是想趁机暗算。对于一个卧底来说,他虽然成功地混入了雍军,而且成了一个不大不小的将领,麾下也有两千骑兵,可是他还是一个失败的卧底,因为这次作战,不要说他,就是军职再高些的将领,也不清楚实际上的安排,所以他并没有得到什么有价值的情报,而且雍军斥候总哨苏青十分厉害,让他根本没有什么机会传递情报。而他唯一一次冒险送出去的情报让龙庭飞提前了火攻时间,确实有些价值,可是里面却混杂了江哲病重的假情报。当然戴钥现在还不知道这一点,但是李显夜里提前撤军,仍然让戴钥明白自己的情报再次落到了空处。如今他的任务即将终结,在雍军全军覆灭之后,他自然不需要留在荆迟身边,这样算起来,他在此战中基本上没有立下什么功勋,懊恼之余,他想到不如趁机杀了荆迟。若是能够阵斩雍军的大将,一定可以让正在奋战的雍军失去信心和斗志,虽然有被荆迟亲卫围杀的危险,但是想必主将遇刺的震惊会让他们短时间内失去反应能力吧,所以他一边埋头作战,一边寻找着暗杀荆迟的机会。

此刻唯一没有将心思放在战场上的,只有林碧和萧桐两人,林碧令人将萧桐召来,忧心忡忡地道:“萧大人,我方才令军中斥候刺探泽州方向是否有援军,可是却是没有回应,就连探查军情的黑鹰也无影无踪,虽然时间还短,可是我心中始终不安,是不是你亲自派人去看看。”

萧桐心中也是一凛,自从过了安泽,虽然雍军已经是日暮途穷,可是萧桐还是派出了不少斥候,原本没有异常,可是过了冀氏之后,行军太快,斥候几乎都来不及回报,所以已经有些时候没有消息了,如今想来,萧桐心中生出不祥的预感。可是,真的会有不妥么,看看被围的雍军,雍军连番惨败,主帅齐王屡次断后,连番遇险,若非他身边的亲卫十分高明,中间更有一些江湖高手保护,只怕早就被擒杀了。就是有什么诡谋,也不需要敌军主将亲自担任诱敌之人吧,萧桐心中犹疑,决定再派出得力的斥候四下打探。

萧桐放心不下,吩咐自己亲信的斥候再去刺探,那人从他视野中消失不久,突然泽州方向传来刺耳的警示声,萧桐骇然望去,只见刚刚离去的心腹斥候一边策马狂奔,一边挥舞着手臂,接着,萧桐感觉到大地开始震荡,远处天边出现了一条黑线,如同雷鸣一般的声音滚滚而来,然后,萧桐看到斥候的身躯从马上软软栽倒,可以清晰的看见他背后插着一支利箭。

几乎所有的人都呆住了,包括心知肚明是怎么回事的李显,他刚刚心中生出猜忌,便见到援军到来,不由又是愧疚又是欣喜。他顾不得嘲笑麾下众人目瞪口呆的拙样,高声喝骂着重整军阵,和北汉军迅速脱离,向一侧让开战场,免得被北汉军胁裹住。

那条黑线越来越清晰,很快就可以看清最前面战士的面孔和前方飘扬的旗帜。黑色为底,上面书着“长孙”两字的帅旗几乎是第一时刻落入众人眼中,那如狼似虎的雍军铁骑浩浩荡荡,带着从容的杀气。在距离战场五百步之外,雍军铁骑轰然而止,一员身穿黑色甲胄,外覆同色披风的大将在亲卫簇拥下策马出了军阵,他举起右手,手中是金光粲然的长弓,众目睽睽之下,他抽出一支鹰翎箭,引弓射箭,两只正在战场上盘旋的苍鹰恰好身影重叠,利箭贯穿了一只苍鹰的身躯,余势仍在,又贯穿了第二只苍鹰的身躯,两鹰应声而坠。那员大将掀开面甲,露出一张俊伟的面容,长眉凤目,白面微须,温雅如同儒士,却透着森然不可侵犯的凛然气势,战场上一片寂然,除了战马喘息和伤兵呻吟的声音之外再也没有任何声响。

那大将高声喝道:“末将长孙冀,奉大雍皇帝陛下谕令,前来讨伐北汉贼军,若有弃械投降者,可免死罪,若是顽抗,唯死而已。”

李显终于松了一口气的同时,却扼腕骂道:“这个江随云,真是口风够紧,本王还以为你不过安排了本王留下的十几万大军,想不到皇兄的老底都掏出来了,居然是长孙冀亲至,这次若是不能全歼北汉军,可就是千古奇闻了。”荆迟也是一片茫然,搔搔乱发道:“长孙也来了,怎么搞得,这里什么时候有这么一支伏兵?”戴钥见势悄悄收起了暗器,此刻再刺杀只能是自寻死路。

龙庭飞深吸了一口气,发出了撤兵的命令,鹿伯言正在他身侧,焦急地道:“大将军,何必退兵呢,敌军虽然人多势众,我军也是相差不远,只要我等拼力苦战,未必会败。”

龙庭飞微微苦笑,道:“伯言,我也希望如此,可是若是别的将领领军,也就罢了,我只会以为是齐王求得泽州援军接应,可是竟是长孙冀亲至,此人乃是雍帝亲信爱将,本来是拱卫雍都的重臣,如今竟然到了泽州,想来我们是中了敌军诱敌之计了。李显够狠,他连番苦战就是为了将我们诱到此地,堂堂一个大雍亲王,不顾生死到了这种地步,也真令我佩服得五体投地。若是我所料不差,雍军攻入沁州之初,采用清野之策,就是为了布下这些伏兵,如今我们虽然只见到雍军一部,但是恐怕身后也已经有了敌军,唯今之计,只有迅速撤退,希望雍军来不及合围,让我们退回沁源,否则我军将要全军覆没。”

鹿伯言醒悟过来,面上露出戒惧之色,道:“雍军果然够狠,安泽水淹,沁源苦战,沁水火烧,两次伏击,敌我两军大战连场,竟然只是为了诱使我军入伏,大将军且宽心,就是后面有伏兵,凭着我们十万铁骑,未必没有机会突围返回沁源。”

龙庭飞也只能接受他的劝慰,这时候,林碧令信使传信过来道:“敌军必然四面设伏,代州军善于攻击,愿为前驱。”

龙庭飞微微一叹道:“希望碧公主能够来得及突围,我亲自断后,伯言你们兄弟跟在代州军之后,若是有敌军就全力攻击,若是不能返回沁源,我们都要死在雍军合围之中。”

北汉军的反应极快,几乎是没有任何犹豫就开始撤退,长孙冀仿佛未见,策马上前到了齐王近前,在马上躬身一礼道:“长孙冀拜见王爷,请恕末将甲胄在身,不便大礼参拜。”

李显如今已经是大大松了口气,淡淡道:“长孙将军,伏兵可都已经安排妥当?”

长孙冀恭敬地道:“王爷放心,左右各有八万大军,冀氏之南,有十万精兵阻住北汉军归路,我军步骑三十六万,布下天罗地网,敌军休想逃脱。”

李显状似无意地道:“好啊,长孙将军困住龙庭飞、林碧两军,功劳可是大的很,本王十几万大军却只落得一个惨败而归,倒让本王汗颜。”

长孙冀十分聪明,自然知道这位王爷有了不满之意,连忙道:“殿下何出此言,若非殿下以身涉险,诱敌深入,岂能困住北汉军主力,皇上早有吩咐,末将等全部听从王爷调遣,请王爷尽管吩咐。”

李显面上露出一丝淡淡的笑意,他虽然不是争功之人,可是若是全歼北汉军的机会给长孙冀夺去,那他可就大大不平了,要知道这些日子以来他受尽战败的屈辱,屡次遭遇被敌人擒杀的危险,最希望的就是亲手报仇雪恨。见到长孙冀这样识相,李显心中十分满意,但是他不是不识抬举之人,既然长孙冀如此大度,他也就不急着争夺军权,只是淡然道:“我军疲惫不堪,正需修整,长孙将军自去合围即可,不知负责在冀氏阻击的是哪位将军,可要提防北汉军强行突围啊。”

长孙冀恭敬地道:“是樊文诚、罗章两位将军,王爷将他们留在泽州,他们早已摩拳擦掌,末将因为两位将军和北汉军交战多年,熟悉北汉军的战术,所以请他们带了十万泽州军在冀氏拦截。”

李显满意地点点头,道:“好了,你去安排合围吧,随云在何处,本王要和他商议军务。”

这时候荆迟噗哧一笑,撤退的一路上,荆迟已经不止一次听到李显暗中嘀咕,说是要和江哲算帐,什么商议军务,不过是借口罢了。他这一笑,可让李显生出恼意,上下打量了荆迟半晌,看得荆迟心惊胆战,李显才缓缓道:“荆将军也和本王一起去吧,荆将军这次厉害得很,将北汉境内搅得翻天覆地,屠城血洗,杀人如麻,不知道你的江先生听了怎么想?”

荆迟一听立刻面色苍白,当日江哲传授军法,曾经说过,最不喜没有理由的屠杀,自己这次任性而行,坏了大雍军规,将来叙功的时候不免要受到朝廷责难,不过这毕竟是以后的事情,如今却要先面对先生,不知道这次会否让自己抄书抄到白头,想到这里,不由满面愁容。李显却不管他,令长孙冀派亲卫引路,自行离去了。荆迟垂头丧气地想要跟上,目光落到长孙冀身上,突然露出得意的笑容。

送走了齐王,长孙冀的面上神色风情云淡,从容发出军令,他率领的雍军开始向前逼近,若是此刻有人能够从苍穹俯视,便可看到,在北汉军两侧,两支雍军正在向中心逼近,而从冀氏方向,一支雍军堵住了北汉军退兵之路,百里方圆之内,三十六万雍军不急不缓地合拢,并且开始缩小包围圈,北汉军已经陷入了罗网,虽然仍有一战之力,却是再没有任何生路。

\chapter{第二十七章 杏花疏影}

四月初七,雍军溃逃,代州军轻骑挡前路,龙庭飞将大军尾随不舍,至泽沁边境,两军战未酣,雍军伏兵尽出,则长孙冀奉雍帝命,隐踪迹,藏将旗,潜伏于此多日,三十六万雍军困北汉军于野。

——《资治通鉴·雍纪三》

乍暖还寒时候,最难将息,我临时寄居的小村庄已是春意盎然,满村的杏花已经是含苞绽放,红的、粉的、白的,一团团,一簇簇,娇艳清新,最动人杏花疏影。

我令小顺子在村口的亭子里面铺上锦毡,四周围上锦幔,一个火炉放在旁边,上面温着一壶上好的汾酒,这大铜壶可以装上十斤酒,最适合聚饮了。我裹着大氅坐在铺着一张黑熊皮的太师椅上,温暖舒适的皮毛让我有一种可以完全放松的感觉。

呵口气暖暖有些冰凉的双手,对着槛外杏花,不由生出酒兴,望一望那大铜壶,我还没开口,小顺子已经了然,取出一把小银壶,从铜壶中取酒注满,然后又从银壶里面倒出一杯热酒,用白玉杯盛了递给我,望着原本清澈明晰的汾酒在品质绝佳的白玉杯中呈现出琥珀之色,我满意地啜饮了一小口。这时,耳边传来疾驰的马蹄声,我抬起头,看见绝尘而来的一队骑士,为首的人正是征尘未洗的齐王李显,身后则跟着一干亲卫。到了近前,李显丢了缰绳,大踏步走进亭中,我放下酒杯,起身恭迎道:“多日不见,王爷可安好。”

李显望着我半天,眸中神色变幻万千,良久才道:“随云,你所料的没有差错,我连战连败,若非你事先已有安排,设下大军埋伏,只怕今次真是惨败而归,不过随云,我虽然料到你会从别处调兵,要不然我早就知道你的安排了,还是想不到皇兄这次会这么大手笔,难道你们不担心帝都的安危么,可别瞒我,现在南楚仍有威胁,李康在东川蠢蠢欲动,我都知道,你们不怕有人趁机作乱么?”

我笑道:“王爷过虑了,大雍江山稳如泰山,皇上早有安排,不过哲需向王爷请罪,方才得知北汉军入伏,臣已经令人送了八百里加急的折子上去,说是我军沁水河谷惨败,请皇上速发援军。”

李显神色一变,继而大笑道:“原来如此,原来如此,随云你心中果然是自有丘壑,在你心里北汉战局不过是棋盘上的一角之地罢了,想必你已经为老三设下了陷阱,就等着我这边大局抵定,好请君入瓮了。”

我含笑道:“这些琐碎事情,王爷不必挂心,倒是王爷这些日子辛苦非常,哲已备好美酒为王爷接风洗尘,王爷也该先饮一杯才是。”

李显大马金刀地坐在椅子上,大笑道:“随云你的本事我是领教了,也怪我先前自大,只说放手让你施为,绝不多问,结果本王成了你的棋子,这些本王都不怪罪,不过这次本王几乎丧命,你也该有些补偿才是。”

我淡淡一笑,一摆手,小顺子取过一个锦盒递到李显面前,李显好奇地看着锦盒,正要伸手打开,我却笑道:“盒中之物不好给人看见,王爷回去再看吧。”李显本也不甚关心,便挥手让一个亲卫收了,接过小顺子递过的酒杯,一饮而尽,懒洋洋地道:“本来本王还想和你较较劲,若是我能够一路取胜,势如破竹,你有何安排都是徒费心思,想不到龙庭飞如此厉害,本王始终不如,落得一个惨败而逃的下场,若非事先知道你有所安排,本王按照你的吩咐诱敌入伏,恐怕今日本王就成了大雍的罪人。”

我见李显有些颓丧,正色道:“王爷此言差矣,北汉军强大世人共知,王爷只带了十万步骑,荆将军也仅有三万步骑,地利人和皆为敌军所有,王爷能够保全骑兵主力,又在沁水河谷惨败之后,不屈不挠,连番苦战,引诱敌军入伏,此乃是名将所为。王爷不顾毁誉,不顾危险,亲身诱敌,若无王爷,龙庭飞焉能一路南下毫无戒备,接下来战事,不过是以强凌弱罢了,此番北伐,王爷乃是首功。此是哲肺腑之言,请王爷明察。”

李显心中一暖,这一次他可是吃尽了苦头,虽然达到了预定的目标,表面上却是大败亏输,他心里不免有些窝囊,但是听了江哲苦心劝慰,他心思渐宽,微笑着举起玉盏,我见状连忙亲自把盏,将酒杯注满。李显笑道:“罢了,不论是胜是败,能够让随云亲自行酒,也算是不枉此行了。”

我见齐王已经消去胸中块垒,心中略宽,其实对于损失如此惨重,我也是心里有些黯然,虽然是准备战败诱敌,可是龙庭飞如此辣手,真让我瞠目结舌,这一次与其说是诈败诱敌,倒不如说是趁着败退诱敌,不过如今既然大局已定,此事不说也罢,免得齐王难堪。又劝了几杯酒,我自己也陪了一杯,苍白的面容上带了一丝红晕,李显见状,忙道:“随云,你病体如何?可是旧病复发么?”

我一怔,继而笑道:“没有这样严重,只是哲不耐疲累,如今大局已定,剩下的战事自有王爷安排,哲可以静养些日子,很快就会痊愈的。”

李显放下心来,道:“你可不能偷懒,接下来应该如何安排,你还得出谋划策,龙庭飞、林碧是杀是擒,接下来我军该如何动作,你可有打算?”

我抬头望望天际浮云,轻笑道:“这些事情王爷何需问我,只是林碧关系代州军的动向,不可随便处置,若是可能,还请王爷尽量生擒,交给皇上处置。倒是有一件事情,宣松是生是死,王爷可有消息?”

李显皱眉道:“河谷伏击之时,我令人特意生擒了一个北汉将领,但是他却声称不知,不过龙庭飞心狠手辣,当日我军勇士几乎都葬身火海,恐怕宣松也是难逃此阶。”

我叹息道:“得知宣将军失踪之后,我曾卜算一课,卦中有死里逃生的意味,故而我总是心存侥幸,如今龙庭飞兵困于此,沁源必然混乱,需派谍探去查一查,如果宣将军得以生还,也好搭救。小顺子,这件事情你去可好?”

小顺子眉头轻皱,却不言语,他深知江哲为宣松之事常常心中愧疚,这次病体颇为沉重,也有这个缘故,可是若是要他离开公子身边,他却是百般不愿。

李显道:“宣将军之事,我也不能放下,这样吧,就让苏青带着营中好手前去,她很是能干,必然不辱使命。”

我摇头道:“苏将军虽然出色,但是段无敌也不是易与之辈,从前他败在苏将军手上,乃是为旧情所困,如今恐怕苏将军很难得手,再说沁源若有魔宗高手,苏将军独木难成林,宣将军之事事关重要,小顺子若不前去,我不能安心。至于我的安全,张锦雄已经归来,就让他负责护卫吧,峨眉凌真子也可相助。”

小顺子见我心意已决,只得道:“公子既然心意如此,我这就亲自去沁源一趟,公子安危,还请王爷多多看顾。”

李显道:“你放心,我重立中军大营之后,就让随云回营。”

见事情已经商量妥当,我笑道:“怎么不见荆迟呢,听说他也无恙?”

李显噗哧一笑,道:“这家伙担心你罚他,最后扯着长孙冀不放,说是要去看龙庭飞被围之后的惨状,说什么也不和本王来见你。”

我淡淡一笑,道:“他可是怕我怪他屠城之事么?”

李显眼中闪过一丝讥讽,道:“不知随云你怎会收他为弟子,若是他聪明一些,便知道你不会怪他非常之举,他偏师远袭,若不是杀伐决断,只怕会陷入苦战,只是你这人虽然心狠手辣,平日里却是温文儒雅,浑让人忘记你乃是心硬如铁之人。”

我不理会齐王对我的评价,从容道:“我虽不怪他,但是却不能不罚他,想来皇上也会给他些惩罚,大概这次的功劳是没有了,毕竟将来大雍是要安抚北汉民众的。”

李显微笑摇头,道:“这些事情我懒得理会,自有皇兄斟酌,随云,林碧既然不可杀,可有什么法子动摇代州军的军心么,这些时日我可是见识了代州军的厉害,这样的铁骑若是杀得性起,我军只怕损失不轻。”

代州么,我漫声道:“却看胡马,揽尽雁门*,旬日之内,蛮人将会进攻代州,代州骑兵只余万人,对着蛮人铁骑,必然是心有余而力不足,如今代州林远霆卧病,留在代州的林澄仪、林澄迩勇猛有余,智谋不足,幼女林彤从未领军,恐怕是凶多吉少。只需将这个消息传扬出去,代州军哪里还有死战之心,十日之内若是不能决战,只怕林碧也不能控制代州军的行动了。”

李显正要点头,耳边传来杯盘粉碎的声音,李显闻声望去,杏花从中,一个二十许年纪的少年人矗立在一树粉红的杏花之下,神情怔忡,面色苍白,在他脚下,一个青瓷盘子摔得粉碎,地上散落着干果糕点,李显愕然,这个少年他认得,正是随云的属下侍从赤骥,也曾有数面之缘,却不知他因何事如此惊惶。

小顺子眼中寒光一闪,冷冷道:“赤骥,退下去面壁思过,不经允许,不得出门。”

李显心中觉得古怪,但是见到小顺子如此直接地惩罚那个少年,全无让自己得知其中缘由的意思,也只能一笑了之。孰知那少年竟然扑到亭子前面,俯身拜倒道:“求公子恩典,允许赤骥去代州一行。”李显心中一震,目光落到江哲面上,却见江哲神色从容自若,只是神色间多了几分肃然。

赤骥直到跪倒在地,才明白自己说了什么,但是他没有一丝后悔,即使说出这番话的结果可能是被拘禁,可能会失去自己目前所有的一切,但是他却全然没有一丝悔意,这一刻,他心中只有那个红衣的娇俏少女,自从东海归来,令他魂牵梦萦的倩影。虽然当初盗骊警告过自己,既然已经错放深情,便要勇于面对,可是他终于发觉自己只是一个懦夫,他逃避了这一切,随着公主回到长安,奉了密令去南楚整顿天机阁情报网。最后他终于按耐不住,接了公子谕令来到北汉,他以为自己可以狠心的看着那个美丽的少女死在战场上,或者死在屠刀下,可是当他知道代州陷入绝境的时候,他竟然还是崩溃了,此刻他只想去代州,和她一起并肩作战,即使是死。

我叹息道:“长沟流月去无声,杏花疏影里,吹笛到天明。昨日夜里我听见你弄笛,便已觉得其中情思缠绵,你随我已将近十年,应知我的脾气,我素来不喜欢强人所难,你若是从此离我门下,我便放你去代州。只是代州就是抵住蛮人侵扰,也抵不过大雍铁骑的践踏,你和小郡主之间不过是镜花水月,赤骥,你真要放弃锦绣前程,去和她同生共死么?”

赤骥泪水悄然滑下,道:“公子收留赤骥在身边,赤骥今日所会的一切本事都是公子所赐,属下也曾想过和她生死相见于沙场,只是如今知道她将要和蛮人作战,我实在难以放下,与其日后和她一决生死,我情愿为了保护她死在雁门关外,若是公子开恩,允许赤骥去代州助她,蛮人退后,就是赤骥仍然苟延残喘,也情愿一死以谢公子,决不会泄漏公子的任何隐秘。”

我轻轻摇头,半晌才道:“你从东海之后,便喜欢上了弄笛,今日就吹一曲给我听,若是我觉得好,就放你离去。”

赤骥眼中闪过迷茫,但是他素来对江哲只有崇敬戒惧,取出一支黄色竹笛,长跪在地上吹奏起来。赤骥本是楚地流浪的孤儿,吹笛本是寻常之事,也无所谓喜爱不喜爱,后来飘泊天涯,转瞬生死,早就没有弄笛的雅兴。可是东海之后,他心中常有悒郁,忍不住捡起童时喜好,弄笛疏解心中愁闷,他本是聪明之人,也曾跟着江哲学过音律,虽然只有数月时光,笛子已经吹得颇为动人。昨夜他弄笛之时,乃是满腔相思,故而吹奏的是一曲江南盛行的笛曲《梅花落》,曲调缠绵悱恻,婉转动人,今日江哲要他吹曲,他心中一动,却吹起了一曲尚不十分熟悉的曲子《折柳》,这是他在代州之时听到的曲子,当时无意中记下了曲谱,后来回到南楚,闲暇时候整理了出来,也曾练习过几次,今日吹来,虽然还有些晦涩,可是曲中之情正合他的心事,笛声清冽,吹彻云天深处,离愁别绪中更有金戈之声,刀枪之鸣。

他这番吹笛不要紧,却令有心人肝肠如焚,不远处,一行人牵马步行向这里走来,为首的正是拖延许久终于不得不来的荆迟,他缠着长孙冀想要留在军中,长孙冀忍笑之余劝他还是早去拜见江哲的好,不论是负荆还是谢罪,终究是个了局,所以荆迟最后带着十余亲卫去见江哲,随行的众人中也有戴钥,他故意流露出渴见之情,荆迟这几日和他相处的也是很好,对他颇为赏识,便带了他一起同行。还没有走近村子,荆迟心中忐忑不安,说是怕不恭敬,便亲自下马步行,戴钥和这些亲卫也都只好随之步行。一行人还没有走到村头,便听见笛声洌洌,忍不住驻足细听。戴钥本是北汉人,这首曲子除了在代州,在北汉其他地方也是颇为流行,戴钥听了之后,只觉乡愁顿起,想到如今北汉擎天柱已经被雍军困住,国家倾覆就在转瞬之间,心中苦痛难以言表,若非他训练有素,只怕早就露了形迹。

那曲声回旋往复,连绵不绝,众人也已经走到近处,荆迟整整衣冠,径自向那坐着听曲的两人走去,戴钥正要跟上,却被荆迟亲卫扯住,戴钥心中一惊,只道自己心中杀意泄露,那亲卫已经低声道:“不可接近,楚乡侯大人身边是不容生面孔接近的,你不见虎赍卫正盯着我们么,除了荆将军,我们还没有资格接近江大人。”戴钥仔细一看,果然在那亭子周围,都有虎赍卫把守,就是齐王的亲卫也站在远处,不能接近亭子百步之内,戴钥心中生出懊恼之意,面上却神色不变,侧头问道:“怎么这位江大人这般高傲么?”那亲卫笑道:“这你可就怪错江大人了,江大人性子随和得很,这是皇上的意思,我听将军说过,从前江大人遇刺重伤,几乎丧命,自此之后,江大人身边的侍卫一直是皇上指派的。”戴钥点头示意明白,心中却生出古怪的念头,若是大雍的皇帝想杀这位江大人,岂不是易如反掌,刚想到此处,他只觉得亭中一道冰冷的目光从自己身上掠过,不由心中一寒,他忍住心中惊惧,过了须臾才将脖颈转了回去,抬头望去,只见一个貌如冰雪的青衣少年站在杏花影中,手执银壶,虽然做着下人之事,但是见他气度却全无一分奴颜婢膝之态。邪影李顺,这个名字立刻涌现在戴钥的心头。

戴钥正在思忖,笛声休止,只见那个长跪弄笛的少年俯首叩拜,沉默不语,戴钥心中觉得奇怪,却不敢多问,只是暗暗留心,只见那亭中灰发青衣之人,缓缓站起,走下石阶,将那少年搀起,叹息道:“你的心意我已明了,你要去代州,我不阻你,只是你不可轻言牺牲,我希望待雍军平定代州的时候,你能够回来见我。放心,我不是要你做什么,我只是要你尽量活下来,回来见我。”那少年起身之后,用衣袖拭去眼泪,恭敬地退去。戴钥虽然莫名其妙,但是这个少年将要去代州,这一点他却听得清清楚楚,心中不由生起疑云。

这时候,荆迟已经面色古怪的上前施礼道:“末将拜见先生,不知先生可安好。”

我心中暗暗偷笑,望着面色不安的荆迟,道:“怎么荆将军有暇来见我了么?”

荆迟苦着脸道:“末将知罪,请先生责罚。”

我淡淡道:“我罚你做什么,你是朝廷重臣,军中大将,千里奔袭,就是没有功劳还有苦劳,我虽有一个小小的爵位,但是荆迟你封侯也不过是早一日晚一日的事情,若论职位么,江某这几日身子不好,已经上书辞去监军之位,虽然还没有旨意,仍然得尸位素餐,不过可不敢责罚你这位带着重兵的悍将。”

荆迟听了这番诛心之言,吓得魂不附体,只当江哲真得生了恼意,连忙拜倒道:“先生休要发怒,荆迟不是存心怠慢先生,只是此番带兵多有不到之处,唯恐先生怪罪,因此来迟了些时候,求先生不要动气,先生正病着,若是伤了身体,末将也是寝食难安。”

戴钥远远看着心中骇然,他可以隐隐听见两人语声,平日跟在荆迟身边,见他豪爽粗直,此次行军,又见他血腥镇压,心中早将荆迟当成了杀星,想不到他竟在一个文弱书生面前如此卑躬屈膝,让戴钥心中一惊,莫非是这个老粗竟是尊师之人,还是这青衣书生有着让人不得不畏惧尊敬的实力。魔宗之人,本就是尊敬强权实力,最瞧不起那些仪仗权势地位盛气凌人之辈。戴钥怎么看也不觉得那青衣人有什么威势,为何方才那少年和荆迟在他面前都是战战兢兢,甚至连邪影李顺这等不可揣测的高手甘愿做他的奴才呢?他心中疑惑难解,更是留心看下面的发展。谁知,一个虎赍卫过来,低声吩咐他们到村中休息,戴钥不得已跟着众人离去,却是故意放慢脚步,竭力听去。却是越来越听不清晰,耳边传来一句破碎模糊的话语道:“屠城之事你也无甚大错,何需歉疚……”,那声音温柔淡雅,却说着这般无情之语,令戴钥心中寒冷非常。

“星星白发,生于鬓垂。虽非青蝇,秽我光仪。”一身戎装,站在庭中最中央的那株粗可怀抱的老槐树之下,林远霆朗声吟毕,开怀大笑道:“诸君,老夫虽然年迈,仍有上马挥戈之力,蛮人虽然凶狠,但是我代州男儿难道会畏惧他们么?”

左右站了两排的代州军将领同时喝道:“代州男儿,以死于沙场为荣,怎会畏惧蛮人,请将军下令,将蛮人逐出代郡。”

林远霆哈哈大笑,本来有些青黄的面容上露出不减昔日的雄风豪气,他向身后望去,代州军的将领都在庭中,有五六十岁,满身伤痕的白发宿将,也有春秋正盛的中年猛将,还有仍然带着稚气的少年将领,而自己的两个儿子林澄仪、林澄迩也在其中,只是可惜,这些将领勇猛有余,智谋不足,此番蛮人来势汹汹,若是只凭着这些将领殊死血战,只怕是两败俱伤。他眼中闪过一丝悲怆,却很快消退,作为代州军现在的主将,他不能流露出心中的悲凉。

林远霆歉然道:“为了国主之令,碧儿率我军主力前去沁州,致令代州局势严峻至此,远霆惭愧。齐兄弟,你本已解甲归田,如今又要披挂上阵,为兄对你不起。”

一个须发皆白的老将上前抱拳道:“将军休要这样说,国主对我代州恩情深重,如今国家危亡,迫不得已召代州军南下,也是情有可原,此事乃是我代州军公议,不关将军和郡主的事情。犬子有幸随郡主南下,孙儿年纪还小,蛮人入侵,我齐家焉能没有上阵之人,末将虽然年老,但是武艺却没有放下,将军不要小看了末将。”

林远霆心中一暖,道:“多谢兄弟体谅,不过你乃是宿将,不可轻易上阵,你若能在中军指挥得当,已经是最大的功勋,这一次我发出征召令,代州十五岁以上的男儿皆要准备厮杀,他们年轻气盛,需你主持大局,至于上阵厮杀乃是年轻人的事情,你可不要和他们争功才是。”

那老将面上先是露出不豫之色,但见林远霆神色坚决,也知自己最应该做的事情就是将沙场经验传授给年轻人,所以应诺退下。

林远霆微微一笑,道:“好,诸将听令,雁门之外的村民皆已经迁回关内,我等需要严守关隘,这一次我们兵力不足,不能像从前一样在雁门之外和敌人主力交锋,但是闭关自守却是寻死之道,这一次蛮人遭遇雪灾,必然不顾性命地来攻击代州,若是我们只顾稳守,蛮人就会从代州防线的空隙渗入进来,所以还是得出关决战,可是我们只能派精兵和他们周旋,就让澄仪和澄迩带兵前去,你们以为如何?”

众将都知林氏兄弟虽然年轻,却是猛将,虽然不及林碧足智多谋,但是也是中规中矩的将领,实力在其他青年将领之上,所以也都没有异议。林远霆正要下令点兵,从内宅走出一个红衣少女,火红的甲胄,红绸披风,弓箭佩刀,一样不少,正是林远霆幼女林彤。此刻林彤面如寒霜,凛然含威,但是那双眼睛却带着火一般的战意,东海归来之后,这个女孩仿佛突然长大了一般,从前的娇俏调皮消失无踪,代之而起的是火一般的炽烈和凤凰一般的眩目。短短时间之内,她的骑射兵法进步到只差乃姐少许的境界。但是这一次出兵,林远霆仍然没有想过让她上阵,毕竟,林家四子二女,已有五人在战场上驰骋,对这个最小的女儿,林远霆毕竟是存了些私心。

林彤走到庭中,单膝下拜道:“女儿请命,随父亲上阵杀敌,驱除蛮人,卫我家园。”

林远霆怒道:“你一个小小女子,怎出此狂言,上阵杀敌,自有父兄担当,你还是在府中护卫你母亲才是。”

林彤凛然道:“父亲此言差矣,女儿虽然年幼,也已经十七岁了,姐姐也是十五岁就上了沙场,女儿知道年轻识浅,也不敢奢望领军作战,只需能够随父兄杀敌报国,已经心满意足。而且姐姐为了国家存亡,去了沁州和大雍作战,就让彤儿替姐姐上阵,将蛮人赶出代州去吧。”

林远霆面上神情又是欣慰,又是哀伤,面上神情变幻万千,这个女儿的性子他很清楚,就是不让她随行,只怕她也会私自混在民团中上阵,而且,看到女儿如此刚烈,他心中也是欢喜非常,终于,林远霆叹了口气道:“此次上阵,你暂时担任为父的亲卫。”

林彤叩首再拜,站起身来,走到父亲身后,她的目光仿佛穿透云山,到了那沁水之畔,若是我战死在沙场之上,或许就不会见到你和我的家人生死相见吧,此刻,她的脑海中浮起一个清秀俊雅,洒脱可亲的少年身影,深沉的哀痛从心底涌起,一滴珠泪滚落尘埃。

\chapter{第二十八章 安排香饵}

四月初十,雍都得军报,仅言雍军沁水河谷惨败事,太宗闻讯怒,率军征北汉,留太子监国,亲赴潼关。

——《资治通鉴·雍纪三》

沁源城,处理完繁杂的军务,段无敌站起身来,活动了一下有些僵硬的身躯,自从上次毒伤之后,虽然伤势已经痊愈,但是仍然有气虚体弱之感,这一次他奉命留守沁源,整日忙着情理沁水河谷,以防万一兵败之后可以退守此地,所以他这几日几乎是目不交睫,前线的军报每日送达,段无敌知道北汉军衔尾追击,雍军已经溃逃,只是今日到了这番时候,怎么却不见军报传来,段无敌心中忧虑万分,只是这里距离冀氏足有百里有余,虽然他已派了斥候前去探察,但是若果真前方出了问题,自己也不可能在明日清晨之前得到消息。

在书房里面转了几圈,段无敌心中终究是有些不安,灵光一闪,他想起一个人来,这人身份不同寻常,或许对这种迷雾中的战况有些独到的见解,虽然这人绝不会轻易说出来,但是还是有机会套出一些口风的。想到这里,他唤来亲卫,向太守府后面的地牢走去。

段无敌沿着青石甬道向下缓行,两侧的墙壁阴冷潮湿,在接近地面的地方甚至长了青苔,除了火把明灭的光芒之外,看不到一丝天光,这里是监押重犯的所在,内外戒备森严,就是一只老鼠,也难以逃脱出去。走到甬道尽头,是一扇精钢的铁门,只是或许是时日久了,上面有一层斑斑的铁锈。守门的两个军士躬身一礼。

段无敌低声问道:“犯人情况如何?”

一个军士答道:“启禀将军,他自从醒来之后就沉默不语,不过不曾反抗,现在已经可以起身,但是不能行走。”

段无敌点点头,令他们打开铁门,门一开,一股浓厚的药材气味混杂着潮气冲了出来,段无敌微微皱眉,走了进去。囚牢大概两丈方圆,只有一张石床摆在正对面,上面铺着厚厚的稻草,散发着潮气,墙壁上延伸出一条铁链,末端的镣铐将坐在石床上的那人手脚锁住,令此人行动难以超出铁链的范围。那人身上一袭粗布囚衣,身上有不少布条包裹的伤口,显然是身负重伤,他的长发散落在面容前,看不到相貌,可是从发隙中可以看到他的左脸也裹着白布,这人形容狼狈,但是他坐在那里,却仍然是身姿挺拔,更带着从容不迫的气度,虽然身处囚牢,却全然没有一丝戒惧和颓丧。

段无敌轻轻皱眉,此人身受火伤,这地牢之内实在不适合他,只是此人乃是雍军大将,自己也不便优容于他。走到床前,段无敌说道:“宣将军,伤势可好转了些么?”

那人抬起头来,抬起右手拨开覆面的长发,露出一张憔悴的面容,左侧面颊包着白布,但仍然可以看到烧伤的痕迹,但是相貌宛然,正是宣松宣常青。他微微一笑,道:“原来是段将军,在下伤势并未恶化,多谢将军遣军师诊治。”

段无敌轻轻一叹,当日雍军奋不顾身地想冲出谷口,却被大将军下令以弓弩封住去路,万余雍军尽死火中,打扫战场的时候,却发觉宣松被十数亲卫压在身下,以身躯鲜血护住,这等身份的雍军将领被俘乃是近年来罕见之事,故而龙庭飞下令将其囚禁起来,并且命令军医替他诊治。宣松苏醒之时,龙庭飞已经率兵出发,段无敌本也有心从宣松口中得知一些雍军军机,可是宣松醒来之后几乎默然不语,虽然没有寻死之意,可是也全然没有屈服之心,段无敌又是军务繁忙,宣松又是伤势未愈,也就没有在这上面下功夫。可是如今军情不明,就不容段无敌心慈手软,需得想法设法从宣松口中得知雍军的机密了。

宣松淡淡的望着有些出神的段无敌,他心中明白此人来意,虽然在这个囚牢之中不见天日,可是根据饮食的次数可以知道大约的日子,再加上自己重伤昏迷的时间,想必如今北汉军已经入伏了吧,看来现在段无敌尚未得到准确的情报,只是发觉不妥罢了。从战场上死里逃生,宣松心中除了痛惜赴死的军士之外,全无殉死之心,只因齐王临去之时那一句话,若是能够重回雍军,纵然受些屈辱也是值得的,不过若是北汉将领想从自己口中问出什么军机,那可是休想,自己虽然翼求重新上阵作战,但又岂是贪生怕死之辈。想到此处,宣松开口道:“段将军可知道宣某为何苟延残喘至今?”

段无敌心中一动,道:“段某想宣将军不是屈膝投降之人,必然是想重见大雍旌旗。”

宣松微笑道:“宣某自幼熟读兵书,只是武艺平平,大雍军中原本最重骑射武艺,因此宣某虽然很想领军作战,但是苦无机缘,也是宣某运气不错,先在荆迟将军麾下为参军,荆迟将军性子豁达,不计较权力分散,允许宣某领军,后来又得到监军大人和齐王殿下赏识,秦泽一战,宣某名动天下,这才做了将军。这番功名来之不易,宣某心中长存感怀之念,因此当日龙大将军火烧沁水,宣某明知九死一生,仍然率军赴死。”

段无敌皱眉道:“其实当日你们的齐王殿下已经率军远走,你们赶不及撤退,何妨投降,可惜宣将军执迷不悟,至令两万勇士死于火海之中,宣将军于心何忍?”

宣松淡淡道:“段将军此言差矣,虽说当日尚可屈膝乞命,但是我大雍勇士岂是贪生畏死之人,若是如此,只怕虽然苟活于世,却是再无面目见人。有些事情就是如此,难道段将军身处绝境之中,就会为了顾惜手下军士的性命而投降么?”

段无敌无语,若是他能够如此,又何必和大雍苦苦作战,明明知道局势不利,却仍要千辛百苦极力周旋,有些事情看似只是退让一步,但那一步却是终究退让不得。他也明白宣松言下之意,是不要奢望从他口中问出什么军机,但是这是唯一的途径,让他如何能够轻轻放弃,想来想去,唯有旁敲侧击,希望能够多了解一些端倪。想到此处,段无敌恭敬地道:“是段某孟浪了,宣将军乃是忠义之人,断不会自污,段某也不愿自寻没趣,不过此地是在不适合养伤,段某之意,请宣将军到舍下养伤,不知尊意如何?”

宣松知他不过是想要迂回行事,自己就是不愿,也难以阻止他的好意,何况他不是迂腐之人,因此只是笑道:“如此宣某就多谢了。”

段无敌心中微喜,令亲兵将宣松扶持出了地牢,送到自己住处,寻了一间关防严密的居室让宣松养伤,不论是否能够软化此人心防,只是心中的敬意,已经足以让段无敌如此做了。

可惜坏消息来得太快了,当斥候回报冀氏之南出现雍军大军,龙将军已经被围之时,段无敌几乎是惊呆了,坐立不安地将所有能够得到的情报翻阅一遍,段无敌无奈地发觉,北汉唯一的机动军力已经被困,而自己手上只有数万步兵,守城尚可,想要救援却是无能为力。他只觉得浑身上下似乎所有的气力都被这坏消息击溃,怔怔想了片刻,他下令封锁消息,立刻令人密报国主此地军情,增强沁源的防卫,再将一切他可以做的事情做完之后,他走进了宣松被软禁的居处。

此刻的宣松已经换了干净的衣袍,倚在软塌上静养伤势,段无敌走进去的时候,他正拿着一本古籍看的津津有味。听到段无敌的脚步声,他抬起头,看见段无敌面色凝重,眼中透着冰寒的杀意,心中一动,猜到可能是北汉军被困的军情传回,放下书册,宣松淡淡道:“段将军神色不安,可是前方有不妥之处?”

段无敌深深地望了宣松一眼,道:“宣将军乃是军中大将,又得楚乡侯信任,莫非不知今日之事么?”

宣松淡然道:“楚乡侯智深勇沉,胸中藏有百万甲兵,他的计策我焉能知晓,不过若论庙算,北汉国中控无人是他敌手,大将军虽然用兵如神,可惜限于兵力局势,纵然十战九胜,这最后一败已可倾国。”

段无敌只觉心中一痛,原本仍然存有的一丝不切实际的幻想破灭无踪,他按住腰间佩剑,恨不得一剑将眼前之人杀死,可是良久,他终于消退杀机,冷冷道:“大将军带十万铁骑,又有嘉平公主辅佐,虽然被困,但是也不是轻易就可以吃掉的,战局未必没有转机,宣将军还是不要高兴过早的好。”

宣松眼中寒光一闪,道:“大将军轻骑远袭,身边最多不过是两日粮草,不知道能支持几日?”

段无敌眼中闪过一丝侥幸,距他得到的情报,在雍军合围之前,负责运送辎重粮草的水军已经进入了包围圈,并且和龙庭飞大军汇合,虽然水军不可能突出重围,但是龙庭飞身边至少有半月粮草,若是节省一些,可以再拖延一些时间,虽然北汉军被困,可是未必没有突围的希望。只是这些事情他当然不愿对宣松明言,不过为了继续套出一些情报,段无敌嘲讽地道:“大将军身边粮草是否充足不劳宣将军费心,只是雍帝大军轻出,虽然至今方露端倪,可是如今已经是人尽皆知,只怕雍帝会后悔莫及。”

宣松知他暗指南楚虎视眈眈,以及东川不稳之事,只是这些事情如何处置却非他所知,因此只是笑道:“代州军南下,不知雁门局势若何?”

段无敌一滞,代州局势紧张,这他也不是不清楚,只是此事他也无能为力,想到此处,段无敌不由微微苦笑,想及自己不过是一个普通将领,难以掌控大局,如今局势糜烂至此,自己更是回天无力,唯一能做的就是向国主求援,以及尽力守住沁源城罢了。

望着段无敌离去之时略现悲凉的背影,宣松淡淡一笑,他明白此人的心思,只是北汉大厦将倾,又岂是数人之力可以力挽狂澜的,只是不知道自己是否有希望生还,说不定北汉朝廷为了坚定不妥协的心志,会下令将自己阵前出斩也不一定吧。

大雍帝都,昭台阁中,黄充嫒黄璃喜上眉梢,一针一线绣着明黄色的龙袍,这些日子皇上对她颇为宠爱,屡屡临幸,她本是没有什么主见心机的女子,早就从前苦恼抛却,每日里只是费尽心思讨好李贽,希望能够多获一些宠爱罢了。

正在她凝神刺绣的时候,她的心腹侍女婵儿捧着茶点走了进来,见到黄充嫒专心致志的神情,她眼中闪过一丝鄙夷,却转而化成笑容,上前施礼道:“娘娘的绣工越发出神入化了,这云龙当真是要破衣而飞,皇上见了定然是十分欢喜。”

黄充嫒轻笑道:“我这点绣工比不上表姐的一点皮毛,表姐乃是旧蜀绣工第一人,她绣得龙袍才是活灵活现呢。”

正说到此处,门外传来一个爽朗的笑声道:“是么,爱妃是否太谦了,你的绣工朕看着已经是很不错了。”

黄璃欣喜地抬起头,正看见李贽走了进来,身后紧跟着宋晚,她连忙上前行礼,被李贽一把搀起。李贽拿起绣到一半的龙袍,一边看着上面精美的绣工,一边道:“怎么,你的表姐绣工比你更出色么?”

黄璃眼中流光溢彩,道:“那是当然,天下四大名绣,苏绣第一人乃是南楚顾绣娘,湘绣第一人乃是大雍薛绫衣,闽绣乃是南闽越青烟,蜀绣第一人就是臣妾的表姐宋影,臣妾少时曾经跟着表姐学过刺绣,只是天分才情远远不如,若是表姐在雍都,臣妾必定求她替皇上绣一件龙袍。”

李贽若有所思地道:“闽绣,越青烟,可是东海侯新妇么?”

黄璃眼中闪过迷茫之色,道:“臣妾不知,只是听人说南闽越青烟,最喜欢仿绣字画,笔意画风宛若原作,只是越小姐乃是名门闺秀,作品极少,若是能够得到一件,往往珍藏不露,所以臣妾竟然是没有见过。”

李贽笑道:“若真是朕所想之人,倒也容易,将来必然让她送一副刺绣给你,不过你的表姐也是名绣,不知道如今何在?”

黄璃脸色一变,偷眼望了李贽一眼,低头道:“臣妾的表姐原本是蜀主尚衣女官,蜀亡后遣散回家,两年前为庆王爷纳入府中。”

李贽的眉头不经意轻皱了一下,道:“原来如此,宋晚,庆王的正妃侧妃中可有此女?”

宋晚望了黄璃一眼,道:“禀皇上,并无此女,想必此女只是庆王殿下侍妾身份,所以并没有禀明宗人府。”

李贽点点头,笑道:“不妨事,改日朕下旨给宋氏侧妃的名份就是。”

黄璃大喜,下拜道:“臣妾代表姐叩谢皇上恩典。”

李贽将她搀起,见她容光艳丽,欢喜无限,心中也是一柔,将她轻轻揽入怀中,黄璃身子软弱无力,面色羞红,宋晚和婵儿识趣地推了出去。正在两人情意绵绵之际,宋晚突然神色紧张地冲了进来,叩首道:“皇上,泽州有八百里加急军情禀告。”

李贽脸上的懊恼立刻被惊容取代,松开黄璃,也顾不上还是在妃嫔寝宫,上前接过军报,一看之下,身躯摇摇欲坠,面色更是苍白如雪,半晌拂袖而出,宋晚匆匆跟上。黄璃大惊,连忙跪送李贽离去。等到李贽离开之后,婵儿惊惶地走了进来,问道:“娘娘,怎么皇上气冲冲就走了,莫不是娘娘伺候不周?”

黄璃摇头道:“不是的,皇上突然接到了泽州的折子,就这样走了,看皇上神情,想必是前方有什么事情惹恼了皇上。”

婵儿神色一动,道:“娘娘,皇上这样烦恼,娘娘不妨去打听一下,以免言语中不小心触及皇上的心事。”

黄璃苦恼地道:“可是本宫如何打听呢,这种事情若是本宫过于用心,恐怕会被皇后娘娘责备。”

婵儿笑道:“这有何难,娘娘不是感激皇后的爱护么,不妨现在去见皇后娘娘,就说是皇上突然怒气大发,您担心皇上气坏了身子,求皇后娘娘去探问一下,等到事后再问皇后娘娘是何事不就行了,皇后娘娘慈悲和蔼,一定不会瞒着娘娘的。”

黄璃心想也是,起身道:“你伺候本宫梳妆,本宫这就去向皇后娘娘请安。”婵儿大喜,连忙上前帮助黄璃梳妆,只是黄璃却看不见婵儿嘴角的恶毒微笑。

等到黄璃从皇后宫中回来之时,已经是愁容满面,她对着婵儿抱怨道:“这颗怎么好,泽州又打了败仗,听说是代州军出现了,齐王殿下败退三十里,又被一把大火烧得惨败,好像还有一位将军独立断后以至生死不明,齐王殿下不是有数的名将么,还有那位据说才智过人的江驸马相助,却败得这样惨,皇后娘娘说,皇上正在召集重臣,准备亲自出征了,唉,皇上乃是万金之体,何必要亲征呢,朝廷又不是没有将军了。虽然前些日子长孙将军被派出去防着南楚,可是不是还有秦将军他们么?”

婵儿劝慰道:“娘娘,皇上从前乃是大雍第一名将,若是亲征,必然是马到功成,娘娘不若将龙袍快些绣好,若是赶得及让皇上出征的时候穿上,那该多好啊。”

黄璃听了连连点头,连忙拿起未完成的龙袍开始飞针走线,婵儿见她专心致志,顾不上自己,便悄悄走出去,托词去了御膳房,当夜,李贽即将亲征的情报传去了东川。

文华殿之外,自从方才几位朝中重臣进去之后,所有内侍和宫女都被逐出殿外,这些人都是战战兢兢,谁不知道方才皇上在殿内大发雷霆,若是此刻触怒了皇上,只怕性命堪忧,即使是在明君圣主眼中,他们这些人的性命也不过是贱若蝼蚁罢了,天子之怒,非同小可。这些人却万万想不到,文华殿之内的气氛并不像他们想象的那般紧张。事实上,李贽是面带笑容的坐在龙书案之后,看着一封密折,那是齐王李显和楚乡侯江哲联名上的密折,是通过最隐秘的渠道递上来的。

郑暇、石彧、董志、管休、苟廉,还有秦彝和程殊都被李贽召来殿中,这样的格局更让人相信前方的确出现了紧急军情,就是秦彝和程殊被特旨召来的时候也是心中不安,直到得知内情才放下心来。

李贽放下密折,喜悦地道:“六弟和随云果然不负朕望,如今北汉军已经入伏,大局已定,六弟不畏艰险,舍生忘死,朕心中甚是安慰。”

石彧笑道:“陛下为北汉之战筹谋良久,长孙将军虽说是托词支援裴将军,但是三十万大军无声无息地赶赴泽州,陛下可是费尽了心思,如今总算是将北汉军主力困住,凭着齐王殿下的用兵手段,龙庭飞就是在用兵如神也不可能突围,而且代州军主力也陷入重围,这对将来取得代州甚是有利。”

秦彝皱眉道:“代州林远霆我也见过,此人英勇豪迈,刚烈忠义,若是想要降服此人甚是为难,可是代州林氏有功于黎民社稷,在代州的名望声威如日中天,若是林氏坚不投降,只怕是陛下要为难了。”

苟廉道:“信国公所虑虽然极是,不过代州林氏虽然声名赫赫,却是因为他们时代守卫代州,抵御蛮人,对他们来说,守卫乡梓乃是最重要的事情,所以当初虽然他们不满北汉先主自立,最后仍然降服,就是因为他们不愿两面树敌,只需将代州和晋阳分隔开来,等到攻破晋阳,北汉亡国,林家终究会屈服的,或许他们会抗拒大雍的统治,但是却不会和朝廷为敌。”

李贽点头道:“虽然如此,朕更希望林家能够心甘情愿的归顺大雍,林家世镇代州,抵御蛮族,功劳卓著,将来大雍一统天下,还需良将镇守代州,林家乃是不二人选,朕已传书齐王,令他一定要保住嘉平公主林碧的性命,对代州军也要以迫降为主。”

郑暇恭敬地道:“陛下圣明,代州林家虽然有割据之嫌,但是代代都是忠心王事的良将,且无野心,若能招抚,定然是北疆屏障,不过若想林家归降,最好的法子还是迫降北汉王室之后,令北汉主写书劝降,若是以大军压境,代州军必然奋起反抗,若是两军交战损失惨重,不利于将来对代州的安抚。”

李贽道:“朕意也是如此,这次朕决意亲征,虽然也有诱敌之意,但是首要的目的还是平定北汉大局,齐王虽然英勇,但是对于政务从来漠不关心,随云体弱,不堪劳累,平汉之后诸般事务千头万绪,都需朕作主才行。”

对于李贽亲征,郑暇等人并不反对,不说李贽本就是大雍的军神,出征得胜乃是理所当然之事,就是为了齐王,李贽亲赴北汉战场也是利多弊少,这次作战虽然齐王战绩并不显著,可是若非他以身涉险,诱使北汉军投入陷阱,也不会有现在的局面,等到齐王歼灭北汉军之后,就可以北上晋阳,攻破北汉都城,这样的功劳,对于齐王来说太重了。若是李贽亲自指挥平定北汉的最后一战,这不论是对大雍还是对齐王,都是更为合适的处置方式。更何况李贽亲征还有诱蛇出动的作用,与其让东川庆王在大雍最脆弱的时候发难,不如让他在朝廷选定的时间发难更为稳妥。

正在李贽和诸人商讨亲征事宜的时候,宋晚悄无声息地走进殿内,承上一封密折,李贽接过之后,剑眉一轩,道:“是夏侯的折子,他那里早已经安排妥当,随时可以发动,这是向朕请示来了。”

听到夏侯沅峰的名字,众人都忍不住轻轻皱眉,虽然这几年夏侯沅峰已经成了雍帝的亲信,可是这个昔日丰神如玉的英俊青年在众人心中早已经成了黑暗中的阴影,夏侯沅峰阴险狠辣的手段也令众人多有诟病,但是明鉴司在李贽心中的地位众人是知道的,而且夏侯沅峰身后还有江哲的影子在。虽然江哲并未插手明鉴司的事情,可是夏侯沅峰昔日本是通过江哲投效雍王的,而他的副手刘华正是江哲旧日的心腹侍从,夏侯沅峰又是明里暗里对江哲十分尊敬,所以众人早就隐隐将他当成了江哲一系的势力。

虽然如此,听到最大的心腹隐患即将被清除,众人面上都露出了满意的神情,李贽放下密折,心中却有着淡淡的忧虑,夏侯沅峰的密折里面暗示,将要趁机接管江哲在旧蜀的秘密势力。在李贽本心来说,当东川落入他的掌握之后,他也不希望还有独立于他的控制之外的势力存在,而锦绣盟,无论江哲对这个力量掌控程度如何,毕竟还是一个叛逆组织,李贽唯一担心的就是,这是否会引起江哲的不满呢?

\chapter{第二十九章 壮士断腕}

四月十五日,太宗出潼关,旌旗所指,无不望风而遁,势如破竹。

同日,庆王于南郑誓师起兵,立蜀王遗腹子孟旭为国主,立誓恢复蜀国,旧蜀遗臣数百,皆涕泪俱下,俯首拜服。

四月十六日,庆王破散关,天下震动。

——《资治通鉴·雍纪三》

散关城上,庆王李康望着城内衣甲鲜明的军士,不由发出由衷的微笑,这些年来的经营,加上威逼利诱,终于将这支大雍的军队牢牢控制在手中,再加上东川豪门集结私兵组成的五万大军,拥军十五万的东川,足可以占据大雍的根基所在——关中,昔日大雍选择攻蜀,很大的因素就是因为蜀国占据汉中地,据阳平关,只需攻破散关,就可以进入关中。这样的威胁让大雍朝廷时刻觉得头上悬着一柄利剑,虽然蜀国王室一心苟安,也不能消除大雍的戒惧,如今自己轻而易举得到了散关,西有散关,东有葭萌关,掌握东川肥沃之地,胜可以得关中,奠立帝业之基,败可以退守东川,冷眼旁观诸侯纷争,比起作一个永远与皇位绝缘的大雍亲王,这才是自己梦寐以求的成就。

正在李康浮想联翩的时候,身后传来一个绵软甜美的声音道:“王爷,春寒料峭,怎不披上妾身送您的披风。”

李康心中一暖,回过头去,果然见到一个素衣少妇向自己走来,虽然因为在军中的缘故,这少妇身上的衣着十分简约素雅,青墨一般的乌丝绾着云螺髻,只用一枚金环束在底部,身姿婀娜,行动如柳,容颜秀美,宛若池中之莲,天然美态已足倾国倾城。那少妇嫣然一笑,裣衽一礼,李康伸手将她搀起,笑道:“卿也太小心了,本王身子强健,这小小春寒,哪里需要什么披风呢?”少妇嗔道:“王爷军务繁忙,目不交睫,妾身无能相助,自然只有尽心竭力,照料王爷的身子,王爷乃是千金之体,若是受了风寒,岂不有碍大业。”说罢,从身后一个劲装侍女手中取过一袭白色蜀锦的披风亲手替李康系上,那披风上刺绣着金色的貔貅,栩栩如生,李康微笑着任凭这女子施为。那女子系好披风,无意中一抬头,看见李康眼中满溢的柔情,玉颜飞红,低头道:“妾身告退,请王爷珍重身体。”言罢转身离去,李康虽然很想她陪在身边,但是现在军务在身,而且出征带着侍妾已经是颇为不妥,若是自己再儿女情长,只怕是有碍军心,所以他只是目送爱妾离去。

就在那少妇即将步下城楼的时候,一个相貌平常的青年匆匆走上,看见那少妇,青年避过一旁行了一礼,少妇微笑颔首,带着侍女走了下去,那青年这才走到庆王面前,禀道:“王爷,散关之内已经全被我军控制,所有被俘雍军都已经关押起来,不过末将审讯之后得知,散关守将李宗勋在关破之时已经逃走,也没有见到明鉴司的踪影,请问王爷是否需要派兵追杀,散关副将献关有功,尚在等待王爷召见。”

李康眼中闪过一丝遗憾的神色,道:“可惜了,李宗勋也是一员良将,对散关又是了如指掌,若是将他击杀,能省下不少麻烦,明鉴司最擅驱利避害,逃走也不稀奇,不过这次你们收买内应,里应外合破了散关,明鉴司必然受到重责,这也够了。”对于锦绣盟的成绩,李康十分满意,先是截断关中和东川的通路,令自己稳稳地将东川大权掌控在手中,又通过威逼利诱,收买了散关副将,使得自己不费吹灰之力就得到散关,这样的功劳终于让李康放下了对于锦绣盟的最后一丝戒心。

这时,叶天秀匆匆赶来,他是李康的心腹,这次被李康任命为刺奸,专司监察军中将校,现在庆王麾下的军队由旧蜀豪门的私兵和大雍军队组成,矛盾丛生,军心也颇有不稳之处,所以叶天秀十分忙碌,庆王原本的密谍人员几乎都用在这上面,一来是李康毕竟更信任自己一手选拔的人员,二来这样也可以让锦绣盟相信李康的诚意,更加尽心,再说对外的情报探察本就是锦绣盟的长处,当然李康也保留着一支针对长安的秘密情报力量。除此之外,李康心知肚明,在这乱世,只有手握军权,才能稳如泰山,所以他全力控制军队,只有军权稳固,就不担心旧蜀势力和锦绣盟有什么不妥之处。

李康听叶天秀将军中情形汇报之后,满意地道:“天秀你辛苦了,现在我们起事的情报只怕已经传到长安,虽然李贽亲征去了,虽然父皇已经不理事,可是还有李骏监国、石彧辅政,更有秦彝和程殊这些老将在长安,我军只能稳扎稳打,我已经决定亲自率军攻陈仓,现在北汉那边战局对大雍不利,我倒要看看雍庭如何两面对敌。”

叶天秀听到李康以雍庭称呼大雍朝廷,知道王爷已经是彻底和大雍绝情绝义,其实叶天秀心中并不希望李康如此做,身为大雍亲王,权势富贵已经是天下少见,何必还要起兵谋反,不过他深受李康知遇之恩,也就顾不得什么大义了,李康话音一落,叶天秀便道:“陈仓守将阴囹乃是李贽心腹爱将,用兵谨慎,擅于守城,陈仓只怕难攻。”

李康笑道:“不妨事,锦绣盟刺客已经混入陈仓,只要等到陈仓被我们攻得筋疲力尽之时,就可寻隙将阴囹刺杀,到时候陈仓必然混乱,我们就可以攻破坚城。再说现在雍庭的心思只怕大半放在北汉,这里只怕顾不上呢,倒是我们攻下陈仓之后,进兵渭南之后,拱卫三秦的那几十万大军恐怕都会压过来。”

叶天秀道:“恐怕信国公秦老将军会随军而至,秦老将军身经百战,甚得军心,我们只怕难以取胜。”

李康冷笑道:“秦彝已经老了,自从秦青死后,此人锐气全消,已不足虑,再说龙庭飞用兵如神,轻取李显,就是李贽去了,难道还能力挽狂澜,我们只需多耗上些日子,必能有所斩获,就是我们最后不得已退回陈仓,也是足可告慰。”

听上官彦密报之后,霍义心中生出淡淡的嘲讽,李康打得如意算盘,螳螂捕蝉,不知黄雀在后,他怎知身边一切已经被我们所渗透,北汉方面明鉴司成绩卓著,将晋阳和东川的情报截断,即使偶然有些消息传了过来,也被自己凭着锦绣盟在庆王身边的力量截获,长安方面庆王的情报渠道更是已经落入明鉴司监控,源源不断的假情报让庆王已经有些得意忘形,浑然忘记自己的对手是多么可怕的人物。

上官彦望着霍义略带嘲讽的微笑,心中一阵冰寒,前些日子他从义父那里得到讯息,义弟顾英突然失踪,他和熊暴想来想去,都觉得义弟恐怕是落入了陈稹等人的控制,所谓失踪不过是为了更加严密的控制顾宁的势力罢了,他曾经旁敲侧击问过霍义,却是只得到意味深长的微笑,无奈之下,他更是不敢违背霍义的命令。义父只有这一个亲生爱子,若是有所损伤,让自己如何可以安心,所以即使霍义的命令再古怪,他和熊暴也不敢违抗,即使是让他在担任侍卫的时候监视庆王的举动。望着霍义若有所思的面容,上官彦只觉得心思渐渐沉入悲哀,什么时候他可以摆脱这些可怕的人物,什么时候他能够恢复平静的生活,复国这种镜花水月的事情为什么要自己付出一切,现在所谓的复国不过是将蜀人绑在了大雍内讧的战车上,他不知道这有什么意义。

霍义遣走上官彦,面色又变得阴沉下来,虽然现在一切都很顺利,可是想到陈稹传来的消息,他心中忍不住生出杀意,夏侯沅峰凭什么提出这个要求,没有锦绣盟,明鉴司在东川能这么顺利么,现在倒好,他居然要过河拆桥,若非不知公子意下如何,他早就想和夏侯沅峰翻脸了。强压下心中怒火,霍义再次将心思放到庆王身上,无意中目光一闪,看到一个素影向城头走去,想必是那位宋夫人去请庆王下去用饭吧。

想到这位宋夫人,霍义心中生出烦躁之意,其实说起来这位宋夫人贤淑温婉,又有一手出色的刺绣技艺,庆王对其宠爱非常,虽然因为宋夫人尚无子女,没有晋位侧妃,可是庆王将这位宋夫人时刻带在身边,就是出兵也是如此,就知道庆王对其的爱宠。而且这位宋夫人全无一般女子的矫揉造作,对待他们这些庆王的下属礼数周到,落落大方,可是霍义却始终觉得这个女子带给自己很沉重的压力。她那双盈盈秋水一般的明眸望向自己的时候,总是带着信赖和恳求,似乎希望自己尽心竭力辅佐庆王,而她的一言一笑都是那样楚楚动人,却让霍义心中平白生出危险的感觉。若是动手之时,需要先杀了宋夫人,这是霍义心中的决定,他始终觉得,宋夫人将是自己最大的阻碍。

宋影抬头望向城头,看到李康神采飞扬的模样,不由停住了脚步,虽然已经年尽四旬,但是因为学武的缘故,李康的容貌仍然如同三十许人,只是多了几分历经沧桑的深沉,俊朗的容貌更令人心中生出倾慕之心。从未想到自己会倾心爱恋一个男子,宋影唇边露出淡淡的笑意。十五岁及笈之时,便因为绣工出众而被选入蜀宫做了尚衣女官,蜀王宠爱金莲夫人,对自己丝毫无意,而自己也瞧不起暮气沉沉的蜀王,就这样似水年华空流逝。原本以为一生就这样度过,谁知道蜀国灭亡,雍王下令遣散蜀宫宫女,自己得以还家。摽梅已过,不愿为俗人妻妾,故而自己选择了孤身一人,可是就在姨夫的盛宴上,自己见到了庆王李康。至今仍然记得初相见时,李康那灼灼的目光,之后李康更是想法设法和自己相见,只为求得自己允诺下嫁。一见已将心相许,这般珍爱终于让自己动了心,动了情,虽然李康碍于局势,不便将自己立为侧妃,以免落下和东川世家联姻的话柄,但是无数次在枕前耳边倾诉衷情,却让她越发沉醉。

宋影望着那峻挺的身影,心中暗道,这样的人本应该立在千万人之上,即使前方的路再险阻,也要陪他同行,不离不弃。见李康转过头来对自己轻轻一笑,宋影也露出嫣然的笑容,向心爱的夫婿走去,李康似有所觉得抬头望来,两人双手相握,再不分开。

此刻的陈仓城内,气氛是紧张而炽热的,这里的将士在闻知庆王谋反的消息之后,都是发自内心的震怒,庆王是什么人,皇室贵胄,掌握东川军政大权,十万铁骑,可是居然在这种时候谋反,现在北汉战事不利的消息也已经隐隐传到了陈仓军中,皇上亲征,长安空虚,庆王的谋反如同雪上加霜,这令所有将士都生出不可遏制的恨意,一定要借助陈仓坚城,不让叛臣贼子东进一步,这是所有将士的心愿。

和陈仓将士的紧张和愤怒相比,在陈仓太守府的后宅之中却是一番从容景象,这里早就被明鉴司征用,成了夏侯沅峰发号施令的地方。

在一间花厅之内,夏侯沅峰站在窗前,含笑看着窗外的新柳碧桃,在他身后,一个灰衣文士正在奋笔疾书,处理着一些公文,房间那弥漫着一种紧张而又从容的矛盾气息。半晌,那灰衣文士捧着文卷走了过来,道:“大人,请您过目。”夏侯沅峰接过文书,浏览一遍,回到书案前签押盖章。那灰衣文士将文书交代下去,回到厅中,见夏侯沅峰仍然神思不属,忍不住问道:“大人,卑职有一事不明,可否请教?”

夏侯沅峰微微一笑,道:“子岳请讲。”

这灰衣文士乃是他的心腹幕僚,自然不会有什么顾忌,坦然道:“大人,锦绣盟乃是江侯爷手中的势力,从现在我们掌握的情报来看,这个组织实力强大,控制的地域也很广阔,无论如何,江侯爷必定对其十分重视,大人借机索取锦绣盟的掌控权,岂不是大大得罪了江侯爷。在皇上心目中,侯爷的分量比起大人要重要许多,难道大人不担心江侯爷为此发难么?”

夏侯沅峰笑道:“子岳,有些事情你不明白,这位驸马爷的手段,最擅长借势,从锦绣盟就可以看出来,他令心腹之人控制了锦绣盟的核心层,但是锦绣盟大部分的力量还是由心存反意的蜀人构成,我不得不佩服他的手段,能够让一个这样的锦绣盟为其所用。可是这样一来也有一个坏处,一旦事机败露,锦绣盟必然会不受控制,江侯爷固然可以让其毁灭,可是玉石俱焚,两败俱伤对任何人都没有好处。所以想要完全控制这样一个组织,实力强大的明鉴司比江侯爷更适合,这一点他会心知肚明。而且这一次锦绣盟配合我们平定东川叛乱,将来便只有两条路好走,其一,锦绣盟被我们控制的消息外泄,不是自行毁灭就是归附大雍,其二,锦绣盟功成身退,但是经过这一次,锦绣盟反迹昭然天下,从此需得和大雍作对到底。我想江侯爷的意思是继续控制锦绣盟,让他成为敌对力量,吸引所有对大雍不满的蜀人,将他们控制起来,还可通过锦绣盟和南楚控制的西蜀交通消息。这本也是一个好主意,放长线吊大鱼,可惜江侯爷忽视了一件事情,从前东川在庆王控制之下,皇上自然不会介意锦绣盟的存在,毕竟这可以让皇上更好的掌控东川的局势,可是一旦东川完全归于皇上控制之下,那么这样一个强大的反叛组织存在,就不利于大雍在东川的统治,也容易引起皇上猜忌。而且军略上可以使用权谋,理政却是只能遵循正途进行,所以这一次锦绣盟必须和庆王一起消失,当然其中江侯爷自己的力量可以全身而退,但是其余的力量只能落入我们的控制,宁可多费心思,重建被我们控制的地下势力,侵入西蜀,也不能让反迹昭然的锦绣盟成为蜀人心目中的英雄,且继续存在。”

灰衣文士皱眉道:“大人所说极是,只是江侯爷可能明白大人苦心,卑职观其用计,环环入扣,令人入局而不自知,可是往往阴谋为体,阴狠绝情,若是他因此怀恨大人,又如何是好?”

夏侯沅峰笑道:“你过虑了,此人虽然用计狠毒,可是为人倒是不喜欢多事的,而且他生性闻一知十,只需知道我的要求,就会明白其中深意,此人行事果断得很,一旦他觉察出来,锦绣盟已经成了他的隐患,他的手段会比我还要激烈,若是由他亲自动手,只怕锦绣盟会成昨日云烟。所以我才要求接手,当然也是我舍不得锦绣盟所控制的情报网和实力,若是没有好处,我又何必出头呢?你看着吧,这两天刘华就会前来见我,转达江侯爷的决定。”自从夏侯沅峰提出接收锦绣盟的要求之后,刘华就几乎避开和夏侯沅峰的每一次见面,即使在放弃散关徉退的大事上,也是派了属下前面商讨。

灰衣文士点点头,正要说话,这时,有人在外叩门,灰衣文士推门出去,不多时走了进来,眼中满是惊佩,道:“刘大人求见。”

走进花厅,骅骝心中带着淡淡的不满,可是公子的既然已经有了决定,那么自己就不得不来见见这位夏侯大人,强忍心中的怒气,骅骝行了谒见之礼。夏侯沅峰全无半分得意之色,相反地却是礼数周到,令骅骝也无法流露出更多的怨言。

平静了一下心中情绪,骅骝淡淡道:“夏侯大人,这是锦绣盟盟友以及所有产业的名单,其中有些人特别标注过的,是可以招纳之人,公子命我转告大人,庆王之事结束,锦绣盟就由大人随意处置。”

夏侯沅峰的瞳孔突然收缩,他从心底察觉到丝丝的寒意,虽然他方才说过江哲若是行事,必然是果断非常,可是他也认为江哲不过是交出锦绣盟盟友名单也就罢了,但是锦绣盟控制的产业却会被他收入囊中,对于这一点,夏侯沅峰早已决定不会过问,不仅仅是因为这是江哲理所当然应该得到的报偿,还有一个原因,若是江哲占有这些产业,那么通过锦绣盟中人的口供,夏侯沅峰可以确信自己能到得到锦绣盟大部分产业的名单,那么通过监视这些产业,就可以对江哲本身真正的实力进行监控,这并非是夏侯沅峰存心和江哲为难,而是顾虑到将来可能的需要,夏侯沅峰并不希望在大雍有任何势力可以逃过自己的眼睛。可是他万万想不到,江哲竟连所有的产业一并放弃,蜂虿入怀各自去解,毒蛇噬臂壮士断腕,他竟然丝毫不留下任何可以让自己渗透的空隙。这样的绝决,让夏侯沅峰甚至有些后悔自己从前的决定,莫非江哲看透了自己的私心,却看不透自己的好意么,那样岂不是平白结下了不可匹敌的大仇。

锦绣盟密舵之内,陈稹和董缺正在意态悠闲地品茗,陈稹道:“夏侯沅峰一定十分吃惊公子的决断。”

董缺道:“公子传信说,夏侯沅峰提醒了他,锦绣盟确实不便再保留在手中,公子的意思,让我们将所有产业可以周转的现银全部拿走,至于锦绣盟的人手,让我们过滤之后全部留给夏侯沅峰处置,不过我却不甘心这样便宜了夏侯沅峰,总要给他一些麻烦才能够补偿我们的损失。”

陈稹缓缓道:“锦绣盟里面我们自己的人手自然要撤走的,那些顽固不化的盟友也可以全其忠义,可是顾宁这些人怎么办,他们虽然也有反意,可是毕竟是比较温和的,有他们存在也可以更好的控制蜀国的谋反势力,而且他的几个晚辈也都有放弃复国的意思,如果一并杀了,只怕反而弄巧成拙。你想给夏侯沅峰留些麻烦,可有什么主意,公子可同意么?”

董缺笑道:“公子怎会不同意呢?我见公子字里行间虽然语气极淡,可是却有不满之意,必然是想给夏侯沅峰一些教训的,公子可是最不喜欢被人威胁的,至于报复的手段么,我倒有一个想法?”说到这里,董缺放低了声音,说了一番话,陈稹听得眼中寒光四射,半晌才道:“好主意,这样一举两得,既可以牵绊那些复国势力,让他们不敢妄自出头,二来也可给夏侯沅峰造成一些麻烦,将来这些事情还不是得落到他头上。”

两人计议已定,陈稹笑道:“陈仓那边需我主持大局,我今夜就要动手,至于南郑,就要看你的手段了?”

董缺淡淡道:“你放心,我自会料理。”

陈稹正要说话,门外传来一个冰冷的声音道:“顾宁求见盟主、副盟主。”陈稹和董缺相视一笑,眼中流露出相同的意味,这不是说曹操曹操就到么?

董缺迅速拿起一个鬼面具戴上,只露出一双冰寒的眼眸,陈稹见他已经准备好,便开口道:“顾护法可有什么事情?”

石门洞开,顾宁大步流星地走了进来,面色苍白如雪,他也不行礼,冷冷望着两人道:“顾某一身在此,不论两位如何处置都无怨言,只求放我几个孩儿一条生路。”

董缺心中明白,知道这是其子顾英失踪的消息终于传到了顾宁耳中,说来顾宁在锦绣盟毕竟是根深蒂固,陈稹已经下令将这个消息隐瞒,但是顾宁仍然得到了风声。他和陈稹四目相对,都觉得这是最好的威逼时机。陈稹故作不解道:“顾护法何出此言,令郎无端失踪,本座也曾下令仔细搜查,只是没有消息,令甥和顾护法的义子在盟主义子霍义身边,安全无忧,顾护法这样说是什么意思?”

顾宁已经是万念俱灰,他颓然拜倒,语气中毫无生气,说道:“副盟主何必还要这样说,顾某心知肚明,盟主自从一开始就对顾某心存不满,不过是记恨当年顾某力阻盟主掌控大权罢了,当日顾某也是丝毫没有私心,只是见盟主所为过于急进,伤害了无辜百姓,这才屡次阻止门主所为,虽然盟主将顾宁羁押准备处死,顾宁也是无话可说。后来盟主自大雍归来,开恩放过顾某,顾某全家都是感激不尽,后来更见盟主策划得当,锦绣盟蒸蒸日上,顾某也是由衷欢喜,虽然盟主因为旧怨将顾某闲置,顾某也是心甘情愿。前些日子我不同意盟主和庆王合作,也是并无私心,盟主下令将我几个孩儿分别调开,顾某也是只能认命,可是我的英儿自幼丧母,全靠我一人抚养长大,今次盟主对他动手,想必也不会放过彦儿和暴儿,顾宁情愿代他们一死,只求盟主开恩,让他们自生自灭去吧。”

陈稹淡淡一笑,心道,你怎知道顾英乃是听见了不该听的东西,若非我下令给洛剑飞让他留意顾英,不能让他脱离控制,也不能让他丧命,洛剑飞不得已剑下留情饶了他的性命,你现在来求情也是晚了,不过却可以利用这个机会迫他去做一件事情。对董缺使了一个眼色,示意他开口,董缺会意,冷冷道:“顾护法,你多次和本座为难,本座也不怪你,你若是能够做一件事情,我就饶了你几个孩儿的性命。”

顾宁微微苦笑,道:“盟主请吩咐。”

董缺道:“你也知道,现在庆王尊蜀王遗腹子孟旭为主,自己任摄政王,不过是虚应故事,只有那些腐儒才会相信庆王的诚意,庆王的意思,希望等到他回来之后,不要再见到那个傀儡,免得落下弑君之名,我会安排你接近孟旭,然后杀了他,我可以保证,你的晚辈都会活的好好的。”

顾宁愕然,脸上的表情变得阴沉,青筋迸动,眼中闪过挣扎的神色,半晌才道:“属下遵命。”

遣走了顾宁,陈稹笑道:“你说,一个一心复国的忠义之人,会做出弑君的事情么?”

董缺淡淡道道:“这有什么关系,不论他如何做,和我们有什么相关?”

两人相视而笑,都露出阴谋得逞的神情。

第三十章    生离死别

北汉军被困于野,苦战十数日,欲突围,皆为雍军死战而阻,然雍军急切间亦不能破北汉军阵。

四月十八日,北汉军粮尽,乃杀马为食,天明之际,分兵突围,战乃定。

——《资治通鉴·雍纪三》

什么是英雄陌路,什么是绝境,龙庭飞轻轻叹了一口气,多年征战,从未有过如此险恶的境况,可是龙庭飞惊奇地发现,他的心绪竟然已经没有丝毫波动,从发觉自己被雍军围困的那一刻,他就清晰地听到心中的那根紧崩的弦断裂的声音。他真的太疲倦了,这些年来,几乎是以一己之力支撑着北汉的大局,对面的敌人源源不绝,且坚韧不拔,胜不骄,败不馁,几乎是硬生生地磨去了他的棱角和斗志,倚为臂膀的心腹将领死得死,叛的叛,如今他已经是孑然一身,更是亲手将缔结鸳盟的爱侣拉入了绝境,自己的道路怕是已经走到了尽头,龙庭飞心中明白,这一次不会再有任何逃生的希望。

雍军的伏兵加上已经重整旗鼓的齐王铁骑,四十余万大军将十万北汉军困住在荒野,双方战力并没有绝对的差异,不付出惨重的牺牲,绝对无法突围。沁州地势狭窄,想要突围只能向冀氏和泽州两个方向才有可能,可是若是向泽州方向突围,龙庭飞等人自知怕是没有机会重回北汉了,敌方占据了强势,己方的选择又极为有限,在这种情况下,十几天来,龙庭飞和林碧亲自策划了数次突围,可惜因为意图全军而出,每次突围都被雍军所阻,空留下无数战士的血肉,沁水呜咽,血流成河,在雍军越来越缩紧的包围圈中,就连泥土都被鲜血浸透。

席地坐在简陋的营帐里,火把昏暗的光芒映照在龙庭飞消瘦憔悴的面容上,比起从前的英姿勃发,如今的龙庭飞神情中带着漠然和寂寥,唯有那双略带碧色的双眼,仍然闪现着光芒,只是有心人可以看出,和从前睥睨天下的傲气不同,他双目之中的光芒充满了对世情的明悟和莫名的悲怆。

帐外传来脚步声,龙庭飞没有抬头,仍然看着萧桐亲自绘制的简图,上面记录着军中斥候舍生忘死探察来的雍军布防图。有人走进营帐,站在他身前,火光将来人的身影拖得很长,阴影挡住了龙庭飞面前地图。龙庭飞微微皱眉,抬起头,明灭的火光映射到他眼瞳深处,也将来人的身影映射到他眼中。深绿色甲胄,织锦金凤的大氅,那人正是林碧。

林碧也憔悴了许多,曾经明艳的容貌多了风霜之色,衣袍之上血迹斑斑,金枝玉叶的身份,如今却是血染战袍,龙庭飞心中一阵悲凉,他淡淡道:“碧公主可有什么事情?”

林碧轻轻摇头,坐在龙庭飞对面,将螓首埋在双手之中,良久才道:“方才雍军用弓箭射来书信到我营中。”

龙庭飞淡淡道:“想必是劝降吧,这些日子我营中也接了不少这样的书信,若非我多方设法鼓舞士气,只怕我军难免军心大乱。”

林碧眼中闪过寒芒,道:“不是劝降,是告诉我军,蛮人入侵代州,声势浩大,我二哥林澄迩率军出击,不幸中了蛮军诡计,二哥拼死杀出血路,身背十余箭死在雁门关外,家父旧病复发,军中群龙无首。”

龙庭飞只觉得心头剧震,好狠毒的心计,不论这信中说得是真是假,代州军军心必然动摇,他软弱地道:“这或许是敌人诡计。”

林碧淡淡一笑,笑容却满是悲恸的意味,她寒声道:“我也希望是敌人阴谋,可是就算是阴谋,也已经得逞,如今我营中将士已经是人心惶惶,就是我三哥澄山,四弟澄渊也是战意全失。何况这消息恐怕是真的,这封信是齐王李显特意写给我的,和其他的信不同,上面将代州之事说得很是详细,李显是不会用假言来骗我的。”说罢,林碧将一封书信递给龙庭飞。

龙庭飞接过书信,一目十行的阅读了一遍,上面果然将代州军情写得十分清楚详细,若是连林碧都觉得没有破绽,那么很可能是真的,他颓然放下书信,道:“你可是有了决定,若是代州军想要投降,我并不会怪你。”

林碧霍然而起,寒声道:“代州军从未做过背信弃义之事,今次出兵乃是公议所决,岂会临阵生变,自从我代州军建立以来,只有同归于尽,从无屈膝投敌之事,即使昔日归顺北汉,也没有说过一个降字。”

龙庭飞的神情变得肃然,也起身道:“我早已料到公主心志坚定,方才不过是试探之语,我乃是统兵大将,军心最是要紧,还请碧妹恕罪。”

林碧神情有些和缓,道:“但是事已至此,我们也需有所应对,必须下定决心不计牺牲地突围了,若是再耽搁,只怕我也不能控制军心了。”

龙庭飞眼中闪过冰寒的光芒,道:“我也正想邀你过来商议突围之事。这些日子多次厮杀,碧妹应该清楚,雍军是绝不会放过我的,每当我率军冲阵的时候,雍军都是不顾牺牲阻挡我军,若是代州军独自冲阵,雍军则以诱敌深入之策应对,若非碧妹果决,只怕早已陷入敌军围困。由此可见,雍军的目标主要在于龙某和沁州军主力,而对于代州军却是留有余地。所以我精心策划了新的突围计划,需要碧妹你全力协助。”

林碧没有言语,龙庭飞所说她又何尝看不出来,但是代州军纵然再英勇,也只有一万五千人,纵然雍军有所容情,想要趁机冲破雍军军阵也是不可能的,缓缓抬头,她的语气淡然而明悟,说道:“你可是要我代州军掩护沁州军突围。”

龙庭飞淡淡一笑,道:“代州军一军之力,想要掩护沁州军突围也是不可能之事,雍军只需五万精兵,就可以阻挡代州军冲阵,若是我趁机带主力突围,雍军必然全力围堵,如果力有不殆,就算是放了代州军出去,雍军也不会让我军有突围的可能。碧妹应该明白,对于北汉的忠心,我军远胜贵军,所以雍军才会以沁州军为主要目标。”

林碧没有说话,她静静地听着,等待龙庭飞的解释,龙庭飞继续说道:“所以我决定这次突围分为三波,你率代州军第一波冲阵,从东北方向突围,雍军必然采用从前的做法,竭力将代州军诱入包围,将你我两军分开,然后我率两万精骑,多张旗帜,从正北方向冲阵,雍军必然竭尽所能阻挡于我,之后,鹿氏兄弟将率我军主力从西北突围,其间将分兵至沁水,毁去雍军阻挡河面的强弩投石机,助水军出困。”

林碧心中一寒,道:“你是要以自己为饵,引诱雍军主力围攻,好让沁州主力突围。”

龙庭飞肃容道:“唯有如此,才能保住沁州军主力,龙某作战不力,连累三军将士,若是再惜命偷生,还有何颜面去见王上,雍军四面合围,北面兵力最多只有十余万,只不过一旦我军陷入苦战,其余三面便从后攻击,这才令我们始终不能突围,这一次我亲自冲阵,诱使敌军主力全力困我,凭着鹿氏兄弟的勇猛,突围的机会很高,而一旦雍军误以为代州军乃是为了掩护我突围,对碧妹的围困必然减弱,代州军突围的机会也很大,以龙某一人性命和两万亲卫军的牺牲,换取我军主力突围,这值得。不过碧妹率先突围,损失也必然惨重,所以我要先和你商量。”

看着龙庭飞说及自己生死时候的漠然神情,林碧娇躯摇摇欲坠,眼前这人乃是自己的未婚夫婿,无奈家国危亡,两人各自都是带兵的大将,因此聚少离多,每次见面除了军务就是军务,几乎很少谈及私情,可是林碧早已将他视为终身伴侣,如今却要中道分离,让她如何能够承受。这一刻,她不再是代州军民景仰的“公主将军”,只是一个将要失去爱侣的苦命女子。

强忍眼中清泪,林碧低声道:“你这般慷慨赴死,那么我呢,你可还记得你我大婚之期,就在今年年末。”

龙庭飞神色一变,眉宇间流露出黯然销魂的神色,这次要求代州军出兵,林远霆额外提了一个要求,就是龙庭飞和林碧的婚事不能再拖,国主作主订了日期,雍军若退,今年年末就是两人大婚之期,当日龙庭飞心中也是暗自欣喜,若能够退去雍军,那么自己也有面目迎亲。只是如今看来,两人竟然是有缘无份,再无结缡的可能。

龙庭飞狠下心肠,道:“碧妹,非是庭飞负约,只是为了家国社稷,庭飞不敢贪生。”

林碧掩面踉跄而退,倚在营帐壁上,身躯微微颤抖,虽然没有哭泣出声,可是那强自抑制的呜咽声却更是令人心碎肠断。龙庭飞纵然是心如铁石也是无法消受,他大步上前将林碧揽入怀中,林碧螓首埋在龙庭飞胸前,细碎的哭泣声回荡在营帐之中,龙庭飞能够感觉到胸前战袍上一阵温热,他心知乃是林碧珠泪渗透衣衫,心中剧痛之下,紧紧抱住林碧娇躯。这时,火把燃尽熄灭,帐内一片黑暗,狭小的空间里只有两人的呼吸声和林碧低低的啜泣声。黑暗之中,龙庭飞这在人前从来是神采飞扬的一代名将,也是黯然泪落。

良久,林碧轻轻挣脱龙庭飞的双臂,轻声道:“既然已经决定,我这就回去安排。”龙庭飞没有说话,他听着林碧挑开帘幕出帐,听着林碧远去的足音,握紧了双拳,寒声道:“大丈夫在世,上不能全社稷,以报君父之恩,下不能护妻子,至令其血染战袍,尚有何颜面苟活于世。”

忽而,龙庭飞耳边传来细弱的歌声,不多时,那歌声越来越响,已经可以听得十分清晰,龙庭飞仔细倾听,歌声却是从代州军军营中传出来的。

“黑云压城城欲摧,甲光向月金鳞开。角声满天秋色里,塞上燕支凝夜紫。半卷红旗临易水,霜重鼓寒声不起。报君黄金台上意,提携玉龙为君死。”

这首战歌乃是代州军最爱唱的曲子,代州军和蛮人作战,多在秋高马肥之际,执干戈以护乡梓,据雁门而抗胡骑,此时唱来虽然与时地不合,但是却让代州军重新激起战意。

歌声初时喑哑艰涩,想必是代州军多日血战,早已是口干唇裂之故,但是唱到后来却是越来越响亮,初时只有百余人在唱,后来附和的人越来越多,最后除了代州军,沁州军也开始随之高歌起来,如同千江万流汇入大海一般,歌声汇聚成气势磅礴的洪流,歌声中多日来士气渐弱的北汉军重新凝聚成无坚不摧的劲旅。

龙庭飞面上凄然之色一扫而空,缓缓的将周身甲胄束好,战袍如火,俊面如冰,走出帐去,决战之期,就在明日,哪里还有儿女情长的时间。

走出帐外,龙庭飞放眼望去,漆黑的苍穹下星星点点的篝火,空气中满是血腥的气味,除了遍野的歌声之外,还能够隐隐约约听见军士忍痛呻吟的声音,一边仔细盘算着突围之策的成败几率,一边听着众军苍凉豪迈的歌声,犹有寒意的春夜透着冷寂肃杀,龙庭飞心中空明非常,他知道必是林碧令代州军吟唱耳熟能详的军歌来激励士气,心中感佩非常,更是希望明日林碧能够突围而出,他心中明白,林碧所面临的危险只比自己低些,最大的可能,明日两人都会死在乱军之中。

这时,萧桐走到近前,不过十数日之间,他已经是形容消瘦,神色憔悴,除了辛苦刺探敌军虚实军情之外,他心中愧疚非常,自从今次雍军攻沁州以来,他屡次铩羽,手下秘谍死伤无数,此次中伏未能即时发觉敌军动向也是原因之一,萧桐无数次痛恨自己无能失职,以至有今日之危局,内外煎迫之下,才令萧桐形容减损如此。

他走到龙庭飞身侧,忐忑不安地道:“将军,方才属下见到公主,说您已经决定突围了。”

龙庭飞淡淡道:“不错,你辅佐鹿氏兄弟最后突围,详细的安排待会儿军议的时候我会说明。”

萧桐道:“将军,您是我军主帅,如何能够自蹈险地,诱敌之事还是让别人去做吧,不妨从军中选取身材和您相近之人,穿了您的衣甲充做诱饵,再让代州军担任突围的主力,将军有很大的机会趁机突围。”

龙庭飞淡淡道:“我是三军主帅,若不当先,如何能够激励将士赴死,至于让代州军充做牺牲,此事再也休提,代州军本不需出兵,如今却因相助我军而陷于死地,我们若是做出忘恩负义之事,还有什么颜面去见代州父老。”

他的语气虽然淡漠,但是一字字犹如钢刀刻在岩石之上,萧桐听罢,知道其心已决,竟然是再无转圜的余地,他也知道龙庭飞所言句句皆真,也只有他亲自出马,才能诱使雍军主力出动,暗暗叹息,萧桐下拜道:“请将军允许属下随您突围。”

龙庭飞望了萧桐一眼,道:“这又何苦呢,今次你虽然屡次遭遇挫折,但是那是因为敌军斥候总哨确实厉害,我北汉军中若论谍探,以你为最佳,若是换了别人,只怕我们早就成了聋子、瞎子。你也不要过分愧疚,这次战败不关你的事情,是我根本就没有想到敌军会是诱敌入伏之计,庙算已然输了一筹,才有今日之败。萧桐,这次你需听我命令,随鹿氏兄弟突围,他们三兄弟军略平平,我很是忧心,你在我身边多年,耳濡目染也有些长进,有你相随,才能保证他们可以顺利突围。”

萧桐默然,良久顿首道:“属下遵命。”他心中已经有了决定,戴罪立功,留得有用之身,全力相助鹿氏兄弟突围,就是以死相谢,也需等到日后风平浪静之时。

龙庭飞见他已经答应随雍军主力突围,欣然道:“好了,看天色已经快三更了,你吩咐下去,三更造饭,五更突围,先让各军主将来见我。”

萧桐心中一跳,道:“将军,我军已经粮尽,因为将军一直在帐中思索军机,所以属下没有禀报。”

龙庭飞冷冷一笑,这样事关军机的大事却不禀报,哪有这样的道理,他在军中威望甚隆,早有军士密报于他,沁州军诸军将领私下密议之事,若非如此,自己也不会断然决定明晨突围,原本想敲打萧桐几句,但是看到萧桐惴惴不安的神情,想到明日就是死别之期,他也不愿过分斥责,只是淡淡道:“知道了,受伤的战马和多余的战马全部杀了,让众军食用。”

在龙庭飞清冷淡然的目光下,萧桐只觉得出了一身冷汗,喏喏退下,晚餐之后,各营都已粮尽,众将私下商议,明日必须突围只有牺牲一部分人冲阵,才有可能突围成功,而沁州军和代州军之间毕竟感情淡漠,所以他们都想迫使龙庭飞同意牺牲代州军,以保证沁州军主力可以突围,可是担心龙庭飞不肯,才想趁着军中无粮相迫,却再也想不到龙庭飞竟会痛下决心,以自己为牺牲,为沁州军主力和代州军争取突围的机会。

一匹匹受伤或者完好的战马长声嘶鸣,铜铃大的眼睛透出不相信的神情,长刀砍落马颈,鲜血泉涌,当战马沉重的身躯倾倒尘埃,挥刀砍死战马的北汉军军士突然丢下长刀,扑在马尸之上痛哭起来,几个军士将他扯起拉到一边,可是他们眼中也是泪水滚滚。对于身为骑兵的他们来说,战马是他们最亲近的朋友,为了养好战马,和战马建立默契,他们几乎是战马吃睡在一起,杀死战马是多么不可理解的事情,一般来说,只有当一匹战马重伤到无法挽救的地步才会将它杀死,而吃马肉更是不被允许的。可是如今他们却要杀死大批的战马,这些战马有的受了轻伤,有的甚至完好无损,只是失去了乘坐的主人,对于要突围的北汉军来说,只需要保留足够的战马就可以了,剩下的战马只能是杀死食用。马肉割取下来,除了让众军饱食一顿准备突围之外,剩余的全部制成干粮,毕竟突围作战的时间并不确定会有多久。整个军营里面充满了惨烈的气氛,亲手杀死心爱的战马的刺激,让所有北汉军的眼睛都变得通红,里面是烈焰,是悲恸。

吃过很有可能是最后一餐的战饭,北汉军开始整军,望着虽然履遭挫折,但仍然整齐有序的大营,龙庭飞策马立在营前,他身后是各军将领,已经都结束完毕,只等着将令就要出发。龙庭飞神色宁静,仿佛不是去赴死,只是去赴一场好友的邀宴。耳边传来熟悉的马蹄声和清脆的銮铃声,龙庭飞剑眉一轩,微笑转头,果然是林碧在代州军亲卫的簇拥下策马过来。

林碧来到龙庭飞马前,想要说些什么,却发觉自己无话可说,仿佛所有的言语都在昨夜说尽,她近乎放肆的凝望着龙庭飞清瘦英俊的面容,不知不觉间,一滴珠泪垂落。龙庭飞一眼便看到林碧有些微红肿的凤目,他想伸出手去安慰于她,却终于没有这么做,只是在马上行礼道:“今次突围,需仗碧妹武勇,庭飞感激不尽。社稷危亡,碧妹乃是公主之尊,还需殚精竭虑,为王上分忧。”

林碧侧过脸去,良久才有比较平静的声音道:“将军保重,突围虽然危险,但是将军神武,若是苍天见佑,或者我们还可相见。”

龙庭飞微微一笑,道:“将近黎明,碧妹乃是第一波冲阵之人,还请准备出发。”

林碧策马奔离,高声道:“林碧遵命,将军珍重。”当战马转向代州军军阵的时候,林碧借机回头望去,虽然距离已经很远,可是林碧却一眼便看见龙庭飞浅碧色的双瞳,那深沉如海的幽深眼瞳蕴含着悲恸和祝福,她从未见过那双眼睛里面流露出这么多情感,而在四目相对的瞬间,那种种深情却突然消失无踪,林碧身躯一颤,若非她身边的女亲卫适时地扶了她一把,她几乎要坠落马下。

她还没有从那双浅碧色的眼瞳中挣脱出来,已经看到了代州军猎猎的军旗,林碧心头一震,顷刻间抛却了所有杂念,摘下银枪,林碧振臂长啸,清亮如同凤呖九天的啸声在天空中回荡,代州军将士大为振奋,也随同高声长啸,排山倒海的呼啸声震碎了黎明前的最后一丝黑暗。

\chapter{第三十一章 三路突围}

荣盛二十四年戊寅,庭飞为雍军围困于冀氏之南,血战十余日不得出,时,代州为蛮人侵扰,势危急,雍军以箭书告之,欲乱军心,且汉军粮尽,众将欲以代州军为牺牲,求突围之机,庭飞察之,不得已亲定突围之策。

——《北汉史·龙庭飞传》

策马站在矮坡之上,李显目光如炬,似笑非笑地望着远处严阵以待的雍军军阵,经过几日的修整之后,他已经重新接掌了大权,负责对北汉军的围歼,因为冀氏是北汉军突围的主要方向,所以他亲率大军阻断北汉军归路。连日厮杀,兵强马壮的雍军硬生生的将北汉军的攻势阻住,而长孙冀则在后面负责压迫北汉军的生存空间,协助李显从后打击北汉军,北汉军几乎突围失败,不得不撤退,都是因为长孙冀的作用,当然李显硬朗的作风也是北汉军始终不能突破重围的重要原因。多年征战,只有今日李显才体会到一切尽在掌握的美妙感觉。

不过李显却仍然觉得郁闷,也不知道发生了什么事情,这些日子江哲似乎心情很不好,对军务漠不关心,每日里不是读书就是练字,每次看到自己总是冷着一张脸,似乎对自己颇为恼怒,不,并非只是针对自己,长孙冀得空时曾去求见,他也是这样不冷不热的模样,就连荆迟都被他撵出门去,偏偏自己还不知道到底是什么让这位一向温文儒雅的青年如此不近人情。摇摇头,李显屏弃心中的杂念,看向前方,昨日自己得到代州的情报,心中一动,便用箭术传信给林碧,想来代州军必然军心不稳,根据斥候的回报,北汉军这一两日就会断粮,想必北汉军突围就在今明两日,而黎明时分正是最紧要的时候,所以他才亲自在此坐镇。

忽然,前面的军阵有些变化,李显精神一振,抬头望去,只见在清晨的第一缕阳光之下,代州军正如同利箭一般向雍军大阵冲来,那为首之人手举银枪,身披织锦金凤的大氅,正是嘉平公主林碧。这一次林碧虽然仍然戴了头盔,却没有将面甲合上,露出秀美如玉的绝色面容上,马如骄龙,人如飞凤,只是面寒如冰,不免减弱了几分魅力。李显只觉得心头剧震,那一刻,他眼中只有那鲜明动人的飒爽英姿。就在李显略一犹豫的瞬间,林碧已经一马当先冲入了雍军的东营,银枪飞舞,当者披靡,在她身后,代州军高声呼喝,后面的军士张弓射箭,前面的军士则是挥舞着刀枪冲入雍军的阵营,那些如同暴雨一般急促的箭矢似乎长了眼睛,懂得避开代州军的身体,却无情地收取着雍军的性命。李显一惊,连忙下达军令,令旗挥舞,鼓号齐鸣,雍军东营开始有序的后退着,其中两翼退得慢些,欲将代州军包围,这是这些时日一贯的做法。

林碧久经沙场,自然知道此刻应该控制攻击的速度,免得陷入敌军三面包围,但是这一次林碧有了不同的选择,她高声呼道:“家乡父老稽首相盼,弟兄们,杀!”然后几乎是不管不顾地冲进了雍军的中军,代州军仿佛一柄尖刀一般刺入了雍军的胸膛。

林碧一声清叱,银枪挑开一柄马槊,直接了当地刺入一名雍军骑士的咽喉,那濒临死亡雍军骑兵满眼血红面容狰狞,大吼一声丢下手中马槊,血淋淋的双手拽住银枪,死也不肯松手,林碧在马上一转身,左手拔出腰间宝刀,刀光一闪,斩断那人双臂,银枪平划,将一个疯狂攻来的雍军咽喉划破,宝刀回旋,斩下一名雍军的首级,然后宝刀归鞘。转瞬之间杀了三人的林碧此刻如同修罗一般残恨,然而绝艳的容颜却如同绽放在战场的狂花,令美丽的春花也失去了颜色。在她疯狂的厮杀激励下,代州军发挥了最强的个人战力,陷入包围之后,他们几乎每个人都是面对着数个敌人,可是凭着他们精湛的马术和功夫,竟然丝毫不落下风,代州军好像变成了浑身是利刃的刺猬,一层层削减着雍军的包围。

李显一皱眉,原本预料代州军军心会涣散,想不到林碧以返乡杀敌号召代州军,如今看来反而更加增强了代州军的死战之心,看来东营未必能够支持得住,可是若是此刻支援东营,接下来的所要面对的沁州军可就难对付了。自己原本预料沁州军有可能会和代州军产生矛盾,因为代州军是最适合作为突围先锋,转移雍军视线的,可是代州军却未必愿意这般牺牲,想不到林碧居然肯心甘情愿地替龙庭飞打头阵,难道她不考虑代州军的损失么。

事到如今,多想无益,对东营前来求援的军士冷冷道:“告诉罗章,没有援军,他五万大军若是还挡不住代州军,也不用来请罪,自己抹了脖子吧。”

这时,代州军已经撕破雍军东营的第一道防线,林碧耳边传来沉闷的鼓声,几百面大鼓同时发出隆隆巨响,令人心中仿佛压着厚厚的阴云,林碧抬目望去,九个雍军步军方阵正严守以待,每个方阵都是由三千人组成,最前面是一人多高的巨盾,后面是密密麻麻的长矛,然后是刀斧兵,再然后是弓箭手。最后面还有一个方阵,里面竖着雍军的将旗,上面是一个龙飞凤舞的“罗”字。

林碧眼中闪过寒芒,一举银枪,指向雍军方阵,喝道:“放箭!”代州军并未放慢马速,第一轮奔射的箭矢射入雍军方阵的时候,距离尚有两百步,第五轮箭雨,两军相距已经只有五十步,百余步*出五箭,代州军箭术足以称雄天下,精准的箭术压迫得雍军无法抬头,几乎是躬身缩颈避在盾牌之后,气势不免稍弱,就在这时,代州军已经冲入了雍军的军阵,战马撞击在盾牌上,长矛刺入人体,两军都没有放松射箭,暴雨一般的箭矢在天空飞舞,雍军的弓箭手拼命地放着箭,想要阻挡代州军的前进,而代州军则如同鬼魅一般,一箭一箭地还击,他们在马上做着各种各样的动作,闪躲,挥刀,枪刺,槊挑,但是却仍然能够在各种情况下射箭杀敌。第一个军阵被突破了,第二个军阵被突破了,就在这时,代州军身后喊杀声再起,那些刚刚被代州军突破防线的雍军骑兵重整旗鼓,从后面攻上来了。代州军后面的骑士反身射箭还击,两军胶结在一起,代州军的攻势受到了遏制。

就在这时,地平线上出现了北汉军的帅旗,旌旗招展,铁骑如风,经过一顿饱餐之后的北汉军气势如虹地冲向雍军的中军大营,看到飞舞在战场上的“龙”字大旗,李显精神一振,立刻连连下令,调动军队上前迎敌,龙庭飞冲阵虽然是势不可挡,不过李显早已有所准备,随他阻击龙庭飞的都是从沁源败退的沙场余生的勇士,本就是武勇过人的精兵,心中的屈辱感又是十分强烈,他们几乎是用性命和北汉军拼杀,绝不能让一个北汉人从这里突围,这是这支军队的唯一信念。两军硬生生撞击在一起,一方舍命突围,一方立誓雪耻,这一场厮杀堪称惨烈。一个雍军刚将敌人挑落马下,被马槊贯穿身体的北汉军士惨笑着紧紧抱住敌人的兵器,另一个北汉军士趁机将他刺倒,另外两个雍军左右包抄过来,两柄马槊几乎是同时刺入这个军士的身体,不远处一个浑身是血的北汉军士,瞪着血红的双眼按动手中的强弩,弩箭穿透了在马上摇摇欲坠的北汉军士和两个将他刺杀的雍军军士的衣甲和身躯。

龙庭飞冷眼看着两军混战的战场,即使是破釜沉舟的北汉军勇士也不能轻易突破雍军的防线,他深深呼吸了一口春日微凉的空气,空气中除了泥土的芳香和青草的气息之外,就只有浓浓的血腥气息,他合上面甲,举起手中长戟,大喝一声道:“随我来。”便冲入了军阵,在他身后,身穿赤色战袍的亲卫高声呼啸着挥舞着兵刃,如火如荼的攻势立刻吸引了所有人的目光,北汉军自动地向两侧分开,火红色的洪流形成锥矢阵,楔入了雍军的中军,其余的北汉军自动附在锥矢阵的尾部,洪流越来越庞大,雍军的军阵开始动摇,开始动荡。

李显见状冷冷一笑,多年征战,他和龙庭飞不知道多少次沙场交锋,早就看惯了龙庭飞的嚣张气焰,虽然心中不免佩服,可是想要让他俯首认输却是休想,马槊一举,号角声破空而起,李显刚要策马上阵,身边的侍卫庄峻上前相阻道:“殿下,如今龙庭飞已经是虎落平阳,束手就擒只是时间的问题,殿下乃是千金之躯,不应该再披挂上阵,如果有什么损伤,岂不是功亏一篑。”李显大笑道:“主帅若不亲身赴险,如何能够激励士气?本王与龙庭飞交战多年,今日怎能不送他一程,你闪开。”马槊轻挥,迫得庄峻闪开,李显已经一马当先迎上了北汉军的前锋,他身边的亲卫训练有素地随之冲上,将李显护在当中,两团火焰在战场中心碰撞交缠,战马的嘶鸣声和战士声嘶力竭的喊杀声以及勇士身死之前的痛苦呻吟声交织在一起,几乎每一个人都被血腥和杀气冲昏了头脑,疯狂的气息弥漫了整个战场。

龙庭飞和李显的目光在战场上交缠在一起,虽然两人中间隔着许多亲卫,令他们根本无法当面交手,可是两个人的目光始终落在对方身上,手中的兵器只是本能的将身边的敌人清除,多少次沙场上相逢,虽然两人始终没有机会面对面的厮杀,可是却已将彼此的身影刻在心头,今日终于到了生死相决之时。几乎是同时发动,两人穿过自己的亲卫的阻碍,长戟划过一个半圆,马槊则是直刺,两件兵器交击在一起,又迅速的分开,两人的亲卫几乎潮涌般冲来,想重新将自己的主帅保护起来,可是两人的兵器荡起的劲风蓄满真气,让那些亲卫无法靠近,两人猛烈的战在了一起,龙争虎斗,谁都没有退后的意思。

挡开刺向自己咽喉的长戟,李显眼中满是热烈的火焰,就是这个人,让自己一次次饱尝失败的苦痛,一次次死里逃生,这几年身上添了不少伤痕,都是这人的赐予,可是奇怪的,李显却不觉得这人可恨,或许是从前拜此人所赐,让自己每每在生死关头挣扎,消磨了自己心中伤痛的缘故吧。这一生,他输给了皇兄李贽,虽然没有在沙场上见高下,可是很明显的,夺嫡的失败让自己永远成了皇兄的手下败将。而另一个战胜自己,让自己无能为力的就是眼前此人,败退冀氏将其诱入重围虽然是一大胜利,可是扪心自问,李显宁愿在沁源堂堂正正的胜了他。可是除了心中的敬意,李显心中还有一丝莫名其妙的妒意,明明这人陷入重围,生死已经不能自主,可是李显却觉得自己情愿是龙庭飞,情愿战死在沙场之上。狠狠的骂了自己一声莫名其妙,李显奋力地挡开刺来的长戟,反手一槊刺向龙庭飞的胸口。

就是这个人,明明屡次战败,可是却败而不馁,一次次前来迎战,始终保持着旺盛的斗志,龙庭飞有的时候觉得自己仿佛一块试金石,将眼前这人磨砺成了最锋利的兵刃,每一次见到眼前这人舍生忘死的冲锋陷阵,悍不畏死地断后血战,龙庭飞心中总是生出一丝敬意,不是所有的人都会像眼前这人一样,明明是皇室贵胄,千金之子,却不惜性命拼死作战的。心中轻叹,如今眼前这人百炼成钢,而自己却要折戟沉沙在沁水之畔。抬眼望去,看到李显那双满是火焰和杀气的幽深双眼,龙庭飞微微一笑,长戟横扫,若是能和此人并骨沙场,倒也算是值得吧。

两军主帅在战场上单挑,这可是难得一见的奇观,不过两军亲卫都是浑身冷汗,若是让主帅死在自己前面,可是他们身为亲卫者的奇耻大辱,虽然龙庭飞和李显越战越猛,罡风四逸,迫得周围之人不得不退到数丈之外,可是这些亲卫仍然在两人周围厮杀起来,同样颜色的衣甲混杂在一起,虽然样式不同不至于让他们看错了敌人,可是在远处的两军将士看来,却是很难分清敌友,所以箭雨不再向这里覆盖。

苦战了几十回合,龙庭飞和李显两人都已经额头见汗,两人都是万人敌,马上功夫都是出类拔萃,相差有限,所以拼杀起来越发耗费真气体力,不过明眼人已经可以看出,龙庭飞已经隐隐占了上风,毕竟他曾受过魔宗指点,武艺比起李显来说要略胜一筹,而李显的优势在于他的坚韧,数年来苦战连连,李显不知道多少次以身赴险,武艺在杀伐之中锻炼得炉火纯青,最是坚忍不拔,虽然龙庭飞占了上风,可是李显也是守得森严非常,就是再战上百十回合,也不会落败。

两人缠战许久,龙庭飞已经觉察出来己方的攻势变缓,雍军却是越来越稳,若非是眼前有机会杀了李显,只怕龙庭飞已经要抛开李显继续冲阵了。心中有些急躁,龙庭飞开始有些不顾一切,几乎每一招都是两败俱伤的杀招,李显却是丝毫不畏惧,反而和龙庭飞抢攻起来,这样一来两人都是频频遇险,看得双方亲卫心惊胆战。

这一刻,庄峻终于忍不住了,高声道:“保护殿下。”说罢举起马槊冲了过去,再也顾不上是否会被李显责怪。就在他冲出的瞬间,九支羽箭如同幻影一般穿越凝结的杀气,穿越交错的人影,射向龙庭飞,龙庭飞长戟划了一个圆圈,九支长箭仿佛泥牛入海,但是龙庭飞也是连人带马后退了三步,长箭里面蕴藏的真气让龙庭飞的身躯摇摇欲坠,长戟荡开,露出了身前要害。那是端木秋射出的箭矢,身为齐王亲卫的他除了箭术之外,并非特别擅长马上功夫,所以故意落在了后面,此刻他发挥了他的箭术的最高水平,成功的钳制了龙庭飞的攻击,让李显取得了良机。李显策马上前,马槊毫无怜悯之意地刺向龙庭飞心口。一个北汉骑士目眦欲裂,左手短刀狠狠的扎在马臀之上,战马一声长嘶,疯狂地向前冲刺,正好挡在李显马前,人立而起,李显的马槊狠狠的穿透那匹战马的马首,马上的骑士在翻身落马之际短刀脱手而出,射向李显的咽喉。李显这一槊几乎用尽了浑身力气,明明见到短刀飞射而来,却是无力闪躲,他的双目突然变得雪亮通彻,淡淡望着将要夺取自己生命的暗器,神情却是冰一样的冷静。就在千钧一发之际,他的亲卫已经赶到,一声响亮的佛号震耳欲聋,“阿弥托佛”,一个亲卫翻身飞掠,转瞬间越过数丈空间,一掌劈去,那柄短刀斜斜擦过李显的脖颈,那名亲卫力竭飘落,他的战马恰好跟上,亲卫落在马鞍之上,高声道:“殿下不可轻身涉险。”这名亲卫却正是法正大师。他话音刚落,齐王的亲卫已经蜂拥而上,将他保护起来,李显无奈地一笑,抬头望去,只见龙庭飞正俯身将那名落马的军士救起,那名军士翻身坐到龙庭飞身后,龙庭飞正策马远离,当李显看去的时候,龙庭飞似有所觉,回头一望,四目相对,两人眼中都是倾慕之色。李显又是一笑,高声道:“杀!不可放走北汉军一人。”龙庭飞已经冲入雍军军阵当中,原本有些混乱的北汉军自动跟随在他身后,锥矢阵再次形成。

李显知道身边的侍卫是绝不会允许自己再次上阵厮杀了,也只得开始专心的指挥军队消磨北汉军的锐气和力量,两军交战最酣的时候,雍军临近沁水方向的西营突然喊杀声震天,李显心中一震,目光望向龙庭飞,方才一番冲阵,李显已经有所发觉,龙庭飞身后旌旗虽然显示的是全军,但是仔细看来似乎只有两三万人,李显心中一阵激荡,明白龙庭飞以己身为饵的真意,可是这一方向的主力都在自己大营之内,负责西营的是荆迟,手下只有四万人,恐怕会让北汉军突围成功。唇边露出玩味的笑容,李显心道,荆迟也是大雍的一员虎将,有他阻挡,北汉军也没有那么容易突围,长孙冀可不是吃素的,前后合围,北汉军也只有死路一条。更何况,李显心道,只要杀了你龙庭飞,就是跑掉几万人又有什么要紧。想到这里,李显也不打算增援西营,反而继续下令围歼龙庭飞。北汉军的后面,长孙冀已经率军逼近,这次北汉军摆明了要决战,没有被北汉军趁机从后方突围的可能,所以长孙冀也开始露出了危险的锋芒。

雍军西营,荆迟指挥着军队抵抗着北汉军原来越强大的攻击,将近六七万的北汉军在局部战场上占据了优势,荆迟完全是死守营地,他早已得到消息,知道林碧和龙庭飞正在东营和中军大营冲阵,只要自己能够死守营地,那么等到另外两营取胜,自己就可以得到支援,东营或者比较难于脱身,但是齐王那里有六万骑兵,两万步兵,应该可以稳胜。整个冀氏方向的防线,除了合围时候的十万军队之外,齐王将所有泽州大营的败退军队都集中到了这里,这样的兵力,加上长孙冀会在后方收缩包围,绝对不会让北汉军突围成功。

此时若有一双眼睛在苍穹俯视,必然可以看到,北汉军三路突围军队,都陷入苦战之中,作为多年的对手,泽州军早已经习惯了和他们的苦战,兵力占优,后面又有己方大军的他们完全没有顾忌的用尽了一切战力,将北汉军死死挡住,若是没有意外,龙庭飞的突围大计便成了泡影。然而龙庭飞何许人也,若没有十足的把握,怎会定计分兵突围,这样的战势他早已想到,若非是齐王必定会亲临他突围的战场,他又怎会定要以身为饵,自始至终,他突围的主要方向就在西营,不仅仅是因为那里靠近沁水,可以顺便接应水军突围,另一个原因就是,那里的守将乃是荆迟,而在荆迟身边有一个魔宗弟子潜伏。

就在荆迟专心致志指挥的时候,突然耳边传来亲卫们惊恐欲绝的叫声,荆迟几乎是下意识地闪身,身躯在马上收缩,尽力减少可能会被袭击的范围,即使如此,他仍然感觉到锋利的刀刃刺入自己身躯的冰凉感觉,剧痛袭来,荆迟圆睁双眼,看见身后偷袭自己的人正是近日颇得自己宠信的偏将戴钥,此刻他面上带着淡淡的微笑,在他身后,几柄横刀刺入他的身躯,五六支马槊将他刺穿,但是所有的一切都来不及阻止他将一柄匕首刺入荆迟的肋部。荆迟的身躯开始摇晃,在他即将跌落马下的时候,几个亲卫扑过来将他抱住。戴钥眼中闪过明亮的神采,用尽最后的力量,高声喝道:“王上,宗主!”然后缓缓合上双目,他的生命之火就这样悄悄熄灭。

这时,北汉军阵中的萧桐轻轻侧过脸去,虽然戴钥的喊声没有能够传到他耳中,但是大雍军阵的混乱已经说明了一切,神色有些黯然,他沉声道:“三位鹿将军,可以突围了。”北汉军中号角迭起,开始了势不可挡地冲锋,骤然失去主将的雍军开始混乱,终于,雍军的防线被突破了一个口子,北汉军蜂拥而出。

雍军阵中,荆迟的亲卫将他抱到安全之处,军医连滚带爬地被几个亲卫架来,卸衣甲,拔出匕首,上药,鲜血从伤口泉涌而出,很快的就渗透了包扎的布条,军医欲哭无泪地道:“属下无能,将军,将军的伤势恐怕……”就在众人心灰意冷之时,荆迟突然清醒过来,他勉力道:“颈下,锁片里面。”一个亲卫立刻伸手,将荆迟衣领撕开,原来荆迟颈上挂着一个金锁片,亲卫打开锁片,里面是一枚龙眼大的蜡丸,白色的蜡衣上有一行细如蚊足的小字“寒园秘制”。军医眼睛一亮,一把抢过蜡丸,轻轻捏碎白色的蜡衣,一缕清香沁人心脾,露出一颗红艳如火的药丸,军医将其塞到已经浑身冰冷的荆迟口中。药丸入口即化,几乎是转瞬之间,荆迟的体温开始转暖,然后伤口的血流渐渐减少,在军医敷上数倍的伤药之后,伤口不再流血,荆迟的呼吸开始趋于平稳,虽然再度陷入昏迷,但是任何人都可以看得出来,他的性命保住了。

一个亲卫看看混乱的战场,北汉军已经大部分突围出去,只有六七千人被接替指挥的副将生生挡住,满目都是雍军狼藉的尸体,他颤声道:“怎么办,怎么办?”另一个亲卫高声道:“快去禀报殿下这里的情况,咱们先作一个绳网,将荆将军送到楚乡侯大人那里,监军大人医术通神,免得咱们将军伤势变化。”这个亲卫乃是多年跟随荆迟的心腹,他的话很有道理,众人立刻分开行事,用四匹马中间拉上一张绳网,将荆迟放到上面,免得受到震动,加重伤势,亲卫们护着荆迟离开了战场。

西营的剧变同时传到了李显和龙庭飞的耳中,龙庭飞松了一口气,笑道:“诸君,我军主力已经突围,现在就看我们自己的了,就是不能生还,也需拉上几个陪葬,杀!”随着他的命令,北汉军开始了肆无忌惮的冲杀。而李显则是面色铁青,迅速传令道:“令西营副将暂理军务,追杀阻截北汉军主力,立刻传信长孙将军,让他全力北上,绝不能让北汉军这样轻松地返回沁源。”然后李显肃容道:“事已如此,也不需后悔,全力围歼龙庭飞,若是再有差池,我们还有什么颜面见人。”众军也都是愤怒欲狂,扑向了面前的敌人,绝不能再让龙庭飞突围,这成了每个雍军将士心中唯一的念头。

\chapter{第三十二章 碧血忠魂}

代州军为先锋冲阵,庭飞自率亲军突围吸引雍军主力,汉军主力从西北出。雍人素惮庭飞威名,以大军阻其冲阵,庭飞冲杀一日夜,马疲力尽,为雍军所困,身被十余处伤,不能行。大雍齐王爱其勇烈,亲赴前敌招之降,庭飞严辞拒之,托以后事,乃自尽,时庭飞年仅三十三岁,其亲卫数百尚存,皆殉死,将军爱马,投沁水而亡。王令筑将军墓于野,又铸“忠义坟”、“义马冢”相伴,后乡老筑祠于墓后,春秋祭祀,凡忠义之士,入祠而拜,往往见其灵异。

——《北汉史·龙庭飞传》

四月十九日,当清晨的曙光再次穿透云层的时候,战场上已经只剩下千余北汉军被雍军团团围住,昨日北汉军主力突围之后,龙庭飞冲阵数次,见没有机会突围,便结圆阵固守,雍军四面猛攻,北汉军却是报了必死之心,双方缠战直到日暮,李显大怒,令人举起火把连夜苦战,直到深夜时分北汉军阵才开始崩溃,但是分散的北汉军组成一个个小的圆阵,顽固地做着无谓的抵抗,很多饥肠辘辘的北汉军士就在战场上渴饮马血,生吃马肉,也不肯弃械投降,直到清晨,李显才终于肃清了除了龙庭飞和其亲军之外的所有残余,几乎没有俘虏,所有的北汉军几乎都是至死方休,有些北汉军在无力作战之后,便自尽而死,也不肯被俘受辱,仅有的几百俘虏不是伤重地无法自尽,就是力竭晕倒,没有机会寻死。

李显脸色铁青地望着被困在重围之中的龙庭飞,双手握拳,气愤非常,这时,身后传来清雅的声音道:“殿下为何面色如此难看,眼看敌酋就要授首,殿下应该高兴才是。”

李显也不回头,嘲讽地道:“原来是监军大人来了,怎么不生闷气了么?”

我忍不住摸摸鼻子,缩回颈子,尴尬地笑了一下,暗自后悔前两日不该得罪了齐王。不过说起来也不能怪我啊,我虽然产业遍天下,但是却是摊子大利润微薄,平白地损失了蜀地的生意网,怎能不让我痛心疾首。

说起来我手上的产业主要分为四部分,第一部分就是南楚天机阁,天机阁暗中掌控着江南商业中的三成,可是这三成却不是我能够全部控制的,其中大部分股份属于我的合作者,另外一部分被我分给了秘营弟子,只有一部分还在我直接掌握之中,可是按照我的计划,天下一统之后,我将把全部产业分散出去,也就是说以天机阁名义控制的产业,我不能随便变卖,也不能过分支取金钱,而且为了支撑在南楚的情报网,我所应该得到的这部分利润基本上是见不到的。

第二部分就是绿耳负责的平安客栈,这是我完全掌控的产业,负责我和其他产业的联络,还是我情报的一个来源,想要控制这样一个庞大的产业,所需要耗费的精力和金钱难以计数,总之,现在仍然处于收支平衡阶段,虽然将来会有细水长流的收益,可是至少目前,我还指望不上。

第三部分就是我在海氏船行的股份,这部分可以说是暴利,也是我目前的主要金源,毋庸多说。若没有海氏提供的源源不断的金钱,我哪有可能有一座人间仙境的静海山庄,更别提建立平安客栈了。

而第四部分就是锦绣盟控制下的产业,当初我本来是为了让锦绣盟那些盟友有个托身之所,也免得他们每天只想着复国报仇,想不到却是财源滚滚,这些锦绣盟中人多半都是颇有才华人脉的俊杰,如果不是这等人物,焉能有心反抗大雍,在这些地头蛇的努力下,锦绣盟的产业可是蒸蒸日上,每年看到收入的帐目我都乐得合不拢嘴。当初我当局者迷,不想放弃锦绣盟,就是为了舍不得这些收益,可是在得知夏侯沅峰的要求之后,我的脑子清醒过来,无奈地发现,我需得放弃锦绣盟,为了不让夏侯沅峰通过锦绣盟的产业渗入到我的势力当中,我痛下决心放弃了所有产业,让陈稹他们将九成以上的流动资金全部通过天机阁送到绿耳手中,虽然我已经尽力减小损失了,只留下店铺、货物和不动产给锦绣盟负责管理这些产业人,在无知中等待夏侯沅峰的强行接收,可是我还是很心痛,想到以后我每年的收入都少了四成,怎不让我捶胸顿足。

什么,你对我说富贵如浮云,简直是胡说,我江哲虽然不爱权势声名,可是钱财还是爱的,若是没有金银,我拿什么养家糊口,难不成要我贪污受贿么。想当初不就是因为小顺子打了我的闷棍,才害得我去考了状元,虽然因此过了几年安逸的日子,可是却也改变了我的一生,若是我当初就有家财万贯,或许如今还在那个山明水秀的地方隐居,每日里看书品茗,赏花钓鱼,其乐无穷,虽然会平淡些,但是却能无忧无虑地度过这一生吧。再说了,凭我现在的身体,虽然勉强称得上健康,可是若没有足够的金钱让我可以使用各种名贵的药物调养身体,再让我为了赚钱而去奔波劳苦,不知道我能不能活到柔蓝和慎儿成亲的那一天。想要过上舒心的日子,哪里不用钱啊,我喜欢的名人字画要钱,我喜欢的孤本珍本也要钱,就是写字用的纸墨,弹琴时候焚的清香,满园的奇花异草,不都是金钱堆起来的么。

这样想来,今次的损失足可以让我痛彻心肺,想来想去,都是因为大雍皇室的缘故,既然李贽是皇上,我不敢迁怒,长乐是我心爱之人,我不忍迁怒,自然只有迁怒眼前的李显了,而长孙冀和荆迟他们,谁让他们是李贽的心腹爱将,所以我就一并迁怒了。这些日子借着养病对军中之事一概不理。当然迁怒归迁怒,我也是觉得李显足可以挡住龙庭飞、林碧,作战的事情我又不是十分精通,所以也就没有理会,怎会想到如今战势成了这个模样,不过现在的局势我还是颇为满意。

龙庭飞被困,迟早就缚,林碧虽然带着代州军趁着雍军无力增援的机会,突破了西营的包围,带着七千代州子弟突围而出,可是代州军实力大损,而且根据我得到的消息,林碧的突围已经不可能影响北汉的大局,而她的生还,也让大雍和北汉王室、代州林家之间尚有转圜的余地。而最出人意料的就是荆迟遇刺,使得沁州军主力突围成功,若非昔日我在寒园的时候给他一粒保命的丹药,只怕他性命难保,这一点显然超出了我的预计。不过由于李显当机立断,令长孙冀不必担心被围的龙庭飞和代州军,而是专心去追杀逃跑的沁州军。虽然沁州军突围成功,还趁机杀了封住沁水的雍军,救出了北汉水军的残余力量,可是在长孙冀的追杀之下,还是只有三万残军逃回了沁源,如今长孙冀已经封锁沁水河谷,陈兵沁源城下,可以说预期的目标皆已达到,虽然不是十全十美,荆迟重伤,李显也觉得面子过不去,可是这还是一次决定性的胜利。

看看李显冰冷的面孔,我叹了口气,歉意地道:“臣前几日小病,不免有些思念妻儿,所以对殿下多有得罪,还请殿下恕罪。”

李显心中知道江哲所说不过是托词,可是他却能够听出其话语中的歉疚和修好之意,再一听到江哲提及妻儿,他脑海里立刻浮现出慎儿娇憨的模样,心中一软,怒意渐渐消散,再想想虽然早已指腹为婚,可是将来婚事是否能够顺利,还需江哲成全,李显脸上露出似笑非笑的神情,也放弃了和江哲的小小过节,笑道:“本王也知道其实已是大胜,只是想到这般窝囊,不仅让林碧突围出去,还放了几万残军到沁源,不免有些美中不足,再说荆将军遇刺重伤,也令本王气愤难忍。”

我见李显已经有了缓和,也笑道:“殿下,如今敌酋已在掌握之中,若能生擒龙庭飞,献俘阕下,这也是难得的荣耀。”说出这番话我原本以为可以得到李显的赞同,毕竟生俘敌军主帅这样的功劳可是足以令李显扬眉吐气的,也可以弥补一下他今次损失的面子。出乎我的意料,李显不但没有附和,反而皱眉道:“很难啊,本王和龙庭飞交战多年,知道他的为人,此人性情高傲,又是北汉军神,若是战败,他是宁可一死也不会被俘受辱的,不说别人,就是本王,若是有落到敌人手中的可能,也只有一条路可走。”

我心中一震,用崭新的目光看向李显,在经过屡屡的挫折和打击之后,这位昔日飞扬跋扈的齐王殿下,在不改昔日高傲性情的前提下,心思也已经深沉如渊海。目光转向战场上,看到那陷入重围的龙庭飞和其亲卫,每个人脸上都是宁静非常,手上的杀戮好像完全无法影响他们的心绪,那是真正的勇士面对必死之境的神情,我轻轻叹了口气,枉我自认擅于把握人心,对于这种沙场勇士还是有些偏差,龙庭飞是不可能被俘虏的。想起曾有人对我说过,当日猎宫之变的时候,皇上被闻紫烟迫得陷入绝境,曾有意赴死,如今想来,李贽、李显和龙庭飞虽然身份地位相差极大,可是有一点却是相似的,那就是他们都是真正的将军,对于他们来说,可以战死,可以战败,却是绝不能被俘受辱。忽然之间,我对血腥的战场多了一分敬意和关注,就让我这个心性不坚的软弱之人,亲眼目睹绝世名将的最后风采吧。

这时,李显叹了口气道:“虽然没有可能,不过本王也不能就这样放弃,若是龙庭飞能够投降,对北汉军心的打击无法估算。”言罢,李显传令停战,如今战场的局势已经完全在雍军控制之下,所以雍军停下攻击,只是将北汉军残余围在当中,而早已濒临绝境的北汉军也没有继续攻击,而是停下来希望能够恢复几分气力,重整一下几乎崩溃的圆阵。战场上突然变得安静下来,除了沉重的呼吸声和战马的哀鸣声之外,天地间一片寂静。

李显策马上前,朗声道:“龙将军,如今你已经身陷绝境,除了这几百个亲卫之外,再无一兵一卒可以调动,本王敬你忠心耿耿,更是佩服你军略无双,若是你肯弃械投降,本王保证,必然待为上宾,就是对你麾下将士,也不会有丝毫轻辱。将军以身为饵,血战一日夜,碧血忠心,天人共鉴,就是如今你放弃抵抗,北汉国主当也不会苛责,何必还要死战,难道将军不爱惜这些对你忠心耿耿的战士么?”

被亲卫簇拥在当中的龙庭飞闻言,缓缓向四周望去,只见不过数百人的亲卫,都已经是人困马乏,战袍破碎,鲜血渗透赤色的战袍,让人分不清哪里是血迹,哪里是战袍的本色。弓箭早已折断,钢刀也已经砍钝,每个亲卫眉宇间都是深深的疲倦之色,眼中除了绝望便是漠然,这里的每一个人都早已知道死亡随时都会到来。龙庭飞微微一笑,道:“诸君闪开,让龙某和齐王殿下说几句话。”

那些亲卫神色不动,迅速的分开一条道路,从圆阵的缺口处,龙庭飞和李显再次面对面的见到了彼此,虽然隔着一段距离,但是已经足以看清对方的容颜,那些亲卫没有丝毫犹豫,反正已经是必死之局,就是齐王趁机攻击又有什么关系,而且他们虽然对敌军主帅恨之入骨,却也知道那人也是当世豪杰,绝不会作出出尔反尔的事情,真正的英雄豪杰,本就只有通过沙场血战才能相互了解。

龙庭飞的目光落到李显身后,那个一身青衣,形容憔悴,却是意态悠闲的书生身上,这一次自己之败,是败在了李贽和李显联手之上,若非自己没有料到李贽会在这种危险的时候出动大军协助李显对付自己,焉能有此惨败,而能够让李贽和李显顺利合作,在其中穿针引线之人,就只有这个青衣人——楚乡侯江哲。不过他的目光一闪而过,终于还是落在了李显身上,不论计策如何周详,若无此人苦战,自己也断不会落入重围。

摘下头盔,随手丢落马下,龙庭飞笑道:“齐王殿下,你也是一军主帅,焉能不知主帅被俘,乃是奇耻大辱,龙某不才,也是一员大将,我龙家世代受国主大恩,付与重权,妻以公主,外托君臣之义,内结骨肉之恩,焉有束手就缚的道理。”

李显道:“本王也知道龙将军大义凛然,绝不会甘心束手,但是将军可以甘心赴死,难道你的麾下将士也都该死么,这样吧,本王可以全君忠义,龙将军何妨下令,命麾下将士投降本王,本王可以保证他们的性命无恙,将来皇上大赦天下,本王保证会让这些将士解甲归田,与其让他们随将军而死,不若将军放过他们,让他们可以娶妻生子,安守田园,难道将军不想为北汉留下一些壮士豪杰么?”

龙庭飞淡淡一笑,从容地道:“齐王殿下说得也不错,龙某既然已经四面楚歌,也不必拖他们和我做伴,诸君,你们已经为了王上,为了龙某,付出的已经够多,今日龙某陷你们于死地,你们仍然拼死作战,于情于理,你们都已经尽到职责,忠义无愧于心,龙某现在下令,你们可以弃械投降,这是龙某的命令,将来若有机会重见国主,你们可以禀告于王上,就说龙某所言,你们并非贪生怕死的懦夫,而是我北汉擎天立地的勇士。”

这些亲卫听到龙庭飞这番话,都是眼含泪水,沉默不语,他们自然知道眼前的情景,主帅已然声明不会投降,却让他们弃械,龙庭飞这番心意,他们自然可以领会,可是弃主偷生,如何能够让他们安心。一个二十出头的青年亲卫突然掩面大哭,他面上都是血迹,泪血混合,越发狼狈不堪,他的哭声仿佛是一个信号,一个亲卫黯然低头,手上的钢刀坠落尘埃,接着,一个又一个的亲卫开始哭泣,他们的兵刃开始脱手,显然已经接收了接下来的命运。

李显没有传令让雍军前去接受俘虏,只是静静的看着这一切。

龙庭飞露出灿烂的笑容,道:“齐王殿下,你我交战多年,也算是神交知己,有一事托付于你,不知道你可肯答应。”

李显慎重地道:“本王与将军,惺惺相惜,非是一日,只要李显能够做到,必然尽心竭力。”

龙庭飞的目光变得温柔幽远,他思索了一下如何措词,才开口道:“龙某青年丧妻,并无子嗣,后事自然无需担心,至于族中父老子弟,都是北汉忠臣,生死祸福也无需龙某忧心,他们自会与北汉共存亡。只有一事,龙某放下不下,就是嘉平公主林碧,龙某的未婚妻子。”

李显愕然,林碧乃是北汉公主,龙庭飞纵然不放心,也不应该和自己说起此事啊。他神色古怪地道:“将军不必担心,嘉平公主已经突围成功,如今应该已经回到了沁源。”

龙庭飞淡淡一笑,道:“非是龙某矫情,北汉若是能够不被大雍吞并,此事提也无用,若是不幸,纳入大雍版图,虽然碧公主乃是王室成员,但是她也是代州军的统帅,代州军百多年来捍卫疆土,御胡蛮于雁门,功在社稷,除非大雍想要尽屠代州之民,否则终究是要安抚代州的,若是杀了碧公主,只怕代州永无宁日,所以请殿下相机进言,保全林氏,龙某可以保证,代州林氏一旦归顺,就不会有二心异志。”

李显犹豫了一下,终于道:“此事事关重大,本王不敢保证,但是必然尽力一试,我皇兄英明神武,必然不会轻易加害忠勇之士。”

龙庭飞眼中闪过一缕宽慰的神采,又道:“还有一事,若是大雍一统天下,碧公主又是平安无事,龙某希望殿下能够代我照顾于她。”

李显身子一颤,若非及时抓住缰绳,几乎要滚落马下,仿佛是心底的秘密被人揭穿,他涨红着脸道:“龙将军,你胡说什么?”

龙庭飞似乎是看穿了李显的心意,凝重地道:“龙某非是胡言,我与碧公主虽然名份已定,可是尚未大婚,我两人虽然是有缘无份,可是毕竟人人都将她当作了龙夫人,只怕纵然是碧公主有意另择佳偶,也是无人敢有求凰之意。碧公主乃是女中豪杰,我不忍她担此虚名孤苦一生,王爷乃是当世英雄,龙某也是敬重万分,碧公主提及东海相遇之事,龙某相信两位也有知己相惜之意,若是有可能,龙某希望王爷能够好好照顾她。”

李显更是满面通红,良久才道:“碧公主才貌双全,又是当世名将,女中豪杰,李显却是风流纨绔,声名狼藉,焉能配得上碧公主,何况……”说到这里,李显突然停住了话语,只因他突然发觉了心底深藏的秘密,东海一会,他竟然已对林碧钟情,只是碍于罗敷有夫,以及敌对的身份,才从来不敢多想,如今突然有了一个光明正大的机会让自己追求林碧,他心中自是不愿轻轻拒绝。

龙庭飞见状不由莞尔,道:“若是将来碧公主也有许可之意,不知道王爷可愿答应这桩婚事?”

李显狠狠心,顾不得身后那些目瞪口呆的亲信,道:“若是碧公主首肯,李显绝对不负所托。”说完这句话,李显松了口气,但是心底却是苦笑不已,大概自己没有机会生个嫡出的郡主,招慎儿为女婿了。

龙庭飞神色一松,笑道:“龙某自然希望我北汉国运昌隆,但是也衷心祝愿王爷诸事顺遂,虽然有些矛盾,但王爷应知龙某一片诚心。”

李显面色赧然,说不出话来。龙庭飞也不再理会他,低声道:“碧血黄沙,忠魂深埋,龙庭飞今日一死,犹有余恨,若是死后还可为国主效忠,该有多好!”说罢,龙庭飞长剑出鞘,寒光一闪,碧血横流,众人惊呼声中,身躯跌落马下。两军将士原本见他谈笑宴宴,虽然是嘱托身后事,可是却自有一种从容气度,竟然都生出他不会求死的错觉,谁知方见他俯首低语,却突然引剑自绝,都是措手不及。龙庭飞的坐骑也是难得的龙驹良马,此刻浑身皆是血染,浑不见昔日英姿,见到主人跌落马前,那战马一边哀鸣,一边不时低头拱一拱主人渐渐冰冷的身躯,嘶叫声哀凄悲怆,令人闻之断肠。

李显黯然,正欲下令善后,龙庭飞一个亲卫突然大声喝道:“将军平日待我们恩重如山,如何可以令将军孤身上路。”这个亲卫原本兵器已经丢弃,但是他作战之时本已受了重伤,一支利箭穿透手臂,箭身虽然截断,但是箭头仍然深深扎在肉中。那亲卫此刻一腔悲愤,竟然不顾一切伸手拔出箭头,带出一团血肉,那亲卫不管不顾,箭头直刺咽喉,立刻气绝身亡,仆倒在地。本来正在哭泣流泪的另一个亲卫见状,大吼道:“将军!”俯身捡起丢弃的佩刀,自尽身亡。他们的举动感染了众人,那些亲卫本就是听了龙庭飞之命才弃械的,如今正是满腔羞愧,悲痛难忍,见状都是高呼一声“将军”,各自自绝。

李显高声道:“不可!”但是却已经来不及了,不过转瞬之间,数百亲卫竟然都已经自尽身亡。李显颓然放下手去,心中不由怅然,竟然一个人都没有救下,北汉勇士,果然是个个忠义。战场中心,龙庭飞的坐骑突然一声哀鸣,向东方奔去。雍军谁也想不到拦阻此马,放开防线,任凭那战马脱逃而去。

我在后面冷眼旁观,龙庭飞此举虽然意外,却也不是不可理解,想必他心中也知道,无论他是否能够突围成功,北汉都已经是日暮西山,所有才有托付后事给李显的举动。不过他将林碧托付给李显倒是我料想不到的,这件事情已经如何解决,是有利还是不利,我开始暗中盘算。

接下来李显下令打扫战场,我也一直跟在李显身边,想看看他如何安排。李显亲自令人在冀氏之野为龙庭飞造坟安葬,又令人将殉死的亲军葬在旁边,铸成一座大坟,称为忠义坟。下葬之日,有雍军回报,龙庭飞战马奔至沁水,于沁水岸边哀鸣泣血,继而自沉其中。李显闻听,唏嘘不语,我也是心中怆然,便提议将战马尸首运来,葬在龙庭飞坟侧,李显立刻答应,令人照办,这座战马的坟墓被李显赐名“义马冢。”

我军北上之前,再次来到龙庭飞墓前,虽然只有数日,可是我却看到墓前有香花供养,不知是何人前来祭奠,我亲酹酒于坟前,祝祷道:“龙将军,虽然是我害死你的,不过这也是无奈之事,你的遗愿我必然助你完成,希望你九泉之下不要责怪于我,你英魂有灵,还应庇佑一方水土,可不要厉鬼作乱,来索我的性命才好。”不知怎么,我觉得坟前有些阴风阵阵,打了一个哆嗦,决定还是立刻离开的好。

\chapter{第三十三章 代州烟云}

红霞郡主林彤,代州侯林远霆幼女,嘉平公主之妹,郡主素得爱宠,父母兄姐视为珍宝,然主爱武妆,常独出,携弓刀射猎。大雍隆盛元年,北汉荣盛二十四年,嘉平公主赴沁州助战,蛮人攻雁门甚急,时远霆病笃,二兄澄迩战死,代州无主,主挺身而出,率众御蛮人,主虽年少,然威仪勇烈不逊父姊,遂得众人拥戴为将军,以抗蛮人。

——《雍史·红霞郡主传》

林彤一身红衣,站在雁门关城头之上,飞快的传下军令,下令抵御猛力攻城的蛮人,虽然他们没有足够的攻城器械,可是凭着勇猛善战以及人数上的优势,还是给雁门关造成了巨大的压力,为了有效地杀伤敌人,林彤精准地选择着投下滚水擂石的时间。敌人的攻击越来越猛烈,虽然蛮人以骑射见长,可是和代州军鏖战多年,他们也学会了攻城的技巧,云梯、投石车的使用让他们有了更大的可能破关,甚至有擅长套索的蛮人用绳索登城。林彤能够感觉得到蛮人这几日兵力越来越雄厚,想必整个草原的蛮人部落已经集结起来合作攻城,攻破雁门关,长驱直入,劫掠一空,好渡过今春口粮缺乏的难关。终于,损失惨重的蛮人开始后退了,林彤松了一口气,她知道不用多久,蛮人就会重新集结兵力,前来进攻,虽然如此,总算得到了短暂的休息时间,也足以告慰。

苦战多日,林彤已是玉容清减,但是神情却是镇静非常,为了鼓舞士气,她已经连续三天三夜没有下城楼一步,她那一身红衣如同火焰一般,始终燃烧在城上,激励着众军血战。自从兄长出城遇伏,在关前中箭身亡之后,父亲便一病不起,长兄林澄仪只会厮杀,军略粗疏,又生性冲动,军中众将引以为忧,不得已虚尊林彤为主将。这原本是权宜之计,可是谁知道林彤却是以纤弱之躯撑起了大局,指挥作战条条是道,不逊于百战宿将,所以不过数日,代州军民就已经将林彤当成了可以接替林碧的主帅。

说起来林彤从前虽然没有指挥过作战,但是她天性聪颖,喜欢骑马射箭,对于沙场征战之事本就十分感兴趣,虽然父母兄姐都很有默契地不让她经历战事,可是她平日来最喜欢跟着林碧到处走动,所以耳濡目染,在军略上已经是颇有见地。东海之事后,林彤蓦然成长,更是在军略上十分用心,再加上前几日陪着林远霆在雁门关指挥,天赋见识再加上虚心,林彤在短短时间内成了合格的统帅。即使有些小小的疏失,在代州军叔伯兄长们的帮衬下,也足以弥补,而且林彤生来机敏,对于战场的把握十分恰当,这才成就了红霞郡主的英名。当然此刻林彤完全无心计较这些,更是没有意识到众人已经将她当成了姐姐的替身,只是努力地想着如何对付蛮人。

拖着沉重的步伐,林彤不顾疲倦,在城上巡视,察看防务,对受伤的军士加以慰问,直到处理完军务,她才寻了一个跺口,倚着城墙坐下,将披风裹住身体,双手抱膝,准备小睡一下。不多时,林彤已经进入梦乡,此刻,她自然不知道有一双眼睛默默地注视着她。

守关的军士和民壮分为两轮,这一轮都已下去休息,而轮换上来守关的军士和民壮开始接受防务,代州民壮也是以军队标准训练,编成甲伍,一切都有条不紊地进行着。在这其中有一支队伍有些不同,他们的动作明显有些散漫,这是代州军征用的外郡民团,每年蛮人入侵的时候,代州军都会将外郡到此的青壮征召入伍,用兵法约束,一来是担心其中有蛮人奸细,二来是为了增强战力,这些人会被编成军旅,由代州老军任伍长什长,有勇力者上关御敌,软弱无能者在下面担浆送水,负责指挥监视他们的代州老军都是经验丰富的沙场勇士,这些人可以怯懦贪生,却绝对没有机会行使奸细的职责。

这只大约有百人左右的民壮乃是这次征召的青壮中颇富勇力之辈,对于上阵杀敌也无戒惧之心,所以才会被派到关上协助代州军民防守,负责指挥这百人的队史名叫林远崇,今年三十九岁,乃是代州林氏的旁宗子弟,若论辈分,乃是林碧、林彤的叔父,虽然军略平平,但是多年血战余生,乃是出色的下级军官,为人又很细心,最是适合指挥监视这些颇为悍勇的外郡之人。他指挥着众人开始布防,虽然有些紊乱,但是仍在可以接受的范围之内,再说这些人都是好手,一会儿守关可以起到不小的作用,所以他还是比较满意的。目光无意中落到一个相貌平平的少年身上,林远崇轻轻一皱眉,这个少年王大郎乃是他最为注意之人,虽然数日来他的表现可圈可点,虽然骁勇,但是并不能和代州勇士相提并论,对于杀伐既没有过分的惧怕也没有兴奋冲动的异常表现,但是凭着多年征战的知觉,林远崇总觉得他身上有一种淡淡的危险气息,令他每次接近此人身边,都有一种压抑的感觉。

不着痕迹地暗中留意这个少年,其实仔细看去,这个少年的五官都是清秀俊逸,可是不知怎么组合在一起却变得平淡寻常,而且还有几分垂头丧气的感觉,面色白皙,似乎有些文弱,但是略现粗糙的皮肤和矫捷有力的肢体让人知道他非是弱者。虽然平日不显山不露水,可是作战时常常有出色的表现,遵守军令,协助同伴,能够力克敌军勇士,这都是有过军人生涯之人的特点。平日沉默寡言,可是关键时候一句话常常有振聋发聩的作用。这一切都让这个数日前以寻访亲友的名义来到雁门而被征用的少年,蒙上了一层迷雾。

当然林远崇绝对不会相信这个少年乃是蛮人的奸细,只见他杀敌时候的辣手,协助自己指挥众人的从容不迫,除非蛮人都是傻子,否则绝不会将这样的人物派来卧底,而非让他领军攻关。见那少年抱着横刀,微闭双眼坐在那里休息,这又是和他身份不符之处,只有久经沙场的战士,才懂得在任意闲暇都需尽力保持体力,而非像另外几个雏儿一样紧张地向外张望,担心敌人前来攻击。林远崇收回目光,不论这人身份有什么蹊跷,只有他不是蛮人的奸细,那就没有关系,至于今后的事情,也要将蛮人逐走才有余暇去考虑。

虽然微合着双目,但是周围一切都映照在心中,更是从那一丝露出的双目缝隙中注视着心切之人,赤骥并非表面上那样沉静。只是使用了一些小小的易容手段,对五官稍微修饰,就让原本俊秀的容貌失去了光彩,刻意不露锋芒,虽然为了作战,难免在这支百人团队中露些颜色,但是相信指挥所有雁门守军的林彤不会留意到一个小小的外人。赤骥就这样混入了代州军,林彤的身边,他自然知道并非无人对自己生疑,只是他对代州有些了解,知道只要不表现出可能是蛮人奸细的迹象,就不会有人对自己详加盘问,微微一笑,等到蛮人退去之后,就是代州军想要秋后算帐,也已经无关紧要。若是林彤那时候还活着,就算将自己杀了,自己也是心无遗憾,若是林彤死了,赤骥心中一痛,相信自己也必然随她而去。既然如此,自己何须处处谨慎小心,反正虽然公子希望自己能够活着回去见他,赤骥自己却是没有这样的奢望。强自来到代州,自己可以说在某种意义上已经背叛了公子,身为八骏一日,将要将公子的意愿当作自己的意愿,在他选择了来和林彤并肩作战的一刻,他八骏之首的地位就已经动摇。何况,大雍不会放任代州的割据,雍军绝对会兵压代州,而赤骥他自己,绝对不希望自己的剑上,沾染了心爱之人和其亲人的鲜血。

过了一会儿,赤骥被人唤起,轮到他上去监视敌情了,他站在关上,双目灼灼地望着远处,双手却在反复做着一样工作,将身边箭囊里面的利箭取出,从腰间接下一个葫芦,然后取出一块方巾,又从怀中取出一副鹿皮手套戴上,接着从葫芦中倒出黑色的液体,浸湿方巾,用方巾擦拭箭头,他的动作灵敏而轻巧,一支支箭矢被他处理过之后,箭头显出灰黑色,而在他做这件事情的时候,他身边的几个青壮默契地挡住其他人的目光,直到他完成这些工作。

刚刚将葫芦系回腰间,身后传来一个悦耳中带着些许沙哑的声音问道:“你在做什么。”赤骥心中一颤,动作却是丝毫没有迟滞,转身拜倒道:“小人正在往箭上淬毒。”

林彤凤目中露出疑惑的神色道:“何必淬毒,我军勇士,谁的箭不是可以立取敌人性命,淬毒费时耗力,用处却不大。”

赤骥用变换过的口音道:“小人非是代州人,虽然也会射箭,却是力道不足,往往穿透敌人皮甲就再也无力致人死地,所以在箭上淬毒,也好增加杀伤敌人的可能。”

林彤恍然道:“原来如此!”她颇有兴趣地道:“你是什么人,怎会制毒,像你这样淬毒十分麻烦,可有法子大量制毒,迅速制作毒箭。起来说话吧,不要跪着了。”

赤骥闻言,平静了一下情绪,站起身来,垂首道:“小人王大郎,乃是游方郎中,也会一些医术,这种毒药乃是小人配制,见血封喉,只是使用起来也很麻烦,淬在箭矢上毒性不能持久,所以小人才会现在才淬毒。郡主守关,需要大量箭矢,制作毒箭确实费时费力。不过据小人所知,代州弓箭作坊比比皆是,其中都有大量的漆,漆中自有毒性,郡主若是令人将成捆的箭支箭头浸入漆中,然后晾干,这样的箭支若是射伤了人,伤口必定麻痒肿胀,而且很难愈合。”

林彤听得心中一动,仔细向眼前的少年瞧去,只见他虽然说话不卑不亢,可是却是垂首低眉,一眼也不偷望自己,似是十分拘谨之人,可是说出来的话语却是带着淡淡的杀机恶意,令人心中陡寒,忍不住道:“你抬起头来。”

赤骥缓缓抬头,林彤望向他的面容,眼中闪过一丝迷惑,眼前的面孔有些熟悉,可是自己却偏偏想不起来,她正欲再问话,身后的亲卫禀道:“郡主,齐老将军过来了。”,林彤对这位父执辈十分倚重,转身准备前去迎接。走到半路,她心中突然灵光一闪,已经想起这少年的相貌竟然和自己心中的那个人九成相似,只是神情气度,以及眼角眉梢的差异,让自己竟然一时想不起来,相貌如此相似,总不会那人就是赤骥吧,林彤脚步一顿。片刻,林彤嘲讽的一笑,怎会是赤骥呢,大雍虎吞山河,楚乡侯正是风光荣耀,他必定在主子身边效力,前程似锦,怎会来到这危机四伏的代州和蛮人作战,再说,那人既然有本事在北汉蛮地厮混,必然会些奇巧之术,怎会摆着一张九成相似的面孔出现在自己面前,而且连姓氏也不改,自己何必胡思乱想。

犹豫了一下,林彤停住脚步,回头问道:“王大郎,你可有同胞手足?”

赤骥流露出似乎有些迷惑的神情,道:“回郡主,小人并无兄弟姊妹。”

林彤怅然道:“是么。”转身继续向前走去,她加快了脚步,扬起笑容,几步迎上齐老将军,笑道:“齐伯伯,可否请你主持,将箭矢的箭头涂上黑漆么?”

望着林彤的矫捷的背影,赤骥嘴角露出一丝淡淡的笑意,这次出发之前,公子曾经告诉自己,自己若是上了战场,必然无法随时随地留心易容后的容貌,与其被人识破易容,将自己当作奸细,不如只改变一些相貌的细节,然后刻意改变一下语气和举止。果然这样一来,就连代州军最熟悉自己的林彤,也不过是起了疑心,而且立刻就因为自己的“破绽”太多,而不会想到自己的身份。虽然若是长期相处,林彤很容易就会认出自己,但是赤骥相信,林彤对自己恐怕怀恨不已,应该会刻意避开自己。虽然有些淡淡的得意,可是赤骥心中却也有着淡淡的遗憾,咫尺天涯,还有什么比这个更令人失意的么。

过了半个时辰,当淬过漆的毒箭准备了一半的时候,雁门关外出现了蛮人遮天盖日的身影。赤骥发出警讯之后,眼中闪过一丝寒光,从这次的蛮军的图腾和装束来看,大草原上八大部落竟然已经全部到齐,这次,蛮人是准备开始总攻了。蛮人按照部落各自排开,其中一个部落突然树起了绘着黒狼图腾的金色大旗,大旗下一个身穿黄色汗王服饰的英俊青年举起手臂,然后雁门关外传来惊天动地的呼声,“大汗万岁,大汗万岁!”,千万人同声高喝,震得雁门关上众人都是面色苍白。金色狼旗,大汗万岁的呼声,这说明了东晋初年被中原大军击溃草原汗廷之后分崩离析的各部重新一统,新汗王的出现,说明了这一次蛮人对代州已经是势在必得。赤骥可以估算出眼前的蛮军足有六万人,想起自己在草原上奔走各部的时候,各部果然已经有了和解的倾向,而英俊青年原本是格勒部酋长完颜纳金,他在草原上声威显赫,素以英明果决,骁勇善战著称,可是其他各部的酋长多半和他的父亲同辈,赤骥绝对没有想到他竟然能够一统草原。如今蛮人汗廷重建,代州只是他们的第一步目标罢了,赤骥正在紧张地思索,身边传来兵刃跌落的声音,却是和他同伍的一个大汉面色苍白,被蛮人的声威吓得魂不附体。

赤骥一皱眉,看向周边,就是代州军也不免神色仓皇,正想着如何鼓舞士气,林彤轻身一跃,已经跳到一个墙跺之上,指着蛮人王旗高声道:“你们都害怕了么,这些蛮人把你们的胆都吓破了吧,你们听着,雁门关之后,是我们的家人骨肉,站在这里挥刀的代州勇士们,你们的父母妻儿都在后面看着你们,如今朝廷正在和大雍争夺疆土,我们代州外无援军,内里空虚,除了我们,再也没有人能够保护自己,若是让蛮人冲破雁门关,代州将化成人间地狱,难道你们这些男儿还不如我一个初次上阵的小女子,就是死也是我们先死,总好过看着父老乡亲死在屠刀之下。”

林彤那烈火一般的怒气和发自肺腑的言语让众人面露羞愧之色,齐老将军振臂高声道:“郡主尚且如此勇烈,我们堂堂男儿,难道还会贪生畏死。除非我代州男儿死得一个不剩,否则蛮人休想攻破雁门。死战不退,有我无敌。”众人都是精神大振,也都高声呼道:“死战不退,有我无敌。”城上突然高涨的气势让正在高呼万岁的蛮人面面相觑,不由停住了呼喊。

这时,那王旗之下,信任汗王完颜纳金,一抬手,一个亲卫递过一张一人多高的巨弓,完颜纳金策马出阵,众人只觉眼前一花,完颜纳金已经独自出阵,策马奔到接近雁门关五百步的位置,呼吸之间张弓射箭,三支狼牙箭首尾相连,如同虚影一般射向站在高处的林彤。几乎是一刹那,第一支狼牙已经接近了林彤,林彤翻身下落,避过第一支狼牙,拔出腰刀,想要挡住第二支狼牙,那狼牙力道极强,林彤只觉得手臂一麻,那支狼牙竟然射穿了那柄百炼钢刀,但是第三支狼牙距离林彤不到十步,林彤却是再也无法移动身躯,眼看那支狼牙就要穿透林彤的娇躯。

众人惊呼声中,仿佛穿越了无尽的时光,攸然而现的一支羽箭射中了那支狼牙箭,但是力道显然相距甚远,那支羽箭反弹而落,众人热望成空,不由同声哀叹,谁知就在第一支羽箭反弹的瞬间,略略有些偏差的狼牙被第二支羽箭射到了箭身,接下来,第三支,第四支,直到第五支,五支羽箭几乎是相差一丝地距离依次射中那支力道强劲的狼牙,水滴石穿,那支狼牙箭终于被改变了方向,从林彤脸颊旁边掠过,带起一缕血丝,深深地扎入后面的城墙。

这五箭虽然力道不强,可是准头和速度都是世所罕见。不仅代州军中响起如同雷霆一般的叫好声,就是雁门关外的蛮人中也传出来了赞誉之声。林彤飘落在地面上,几个亲卫已经拿着重盾将她护住,林彤也顾不得玉颊上面的些微伤痕,怔怔地望着几十步之外引弓待发的少年,一弓五箭,这一次无论他有什么改变掩饰,林彤已经认出他的身份,两行清泪滴落,转瞬被雁门关上的风吹干,林彤柔声而又坚决地叫道:“赤骥!”

赤骥微微苦笑,身份泄漏之后,他也无需再加以掩饰,随手从腰间百宝囊里面取出一粒丹药捏碎,在面上一抹,去掉那少量的易容药物,然后从容自若地笑道:“红霞郡主,多日不见了。”平添了几分俊秀的容貌,以及潇洒俊朗、略带些玩世不恭的笑容,让他顷刻间脱胎换骨,鹤立鸡群。众人都不由惊咦一声,这样的鱼龙变化可是让他们生出如梦如幻的感觉。

只有林彤,毫不惊异地道:“为什么你会在这里?大雍占尽上风,何需你来做卧底,你的主子安着什么鬼心思?”

众人骇然望向赤骥,原本心中的感激立刻化作疑惑,他是大雍的密谍,现在碧公主正在和大雍作战,这人岂会安着什么好心。站在赤骥身边的那些被征用的青壮向后退去,代州军则慢慢地围了上来,可是这人刚刚救了林彤,那些人心中犹豫,也不愿立刻动手,都向林彤望去。

这时,已经被亲卫接回本阵的完颜纳金眼中精光一闪,虽然隔着里许距离,可是站在雁门关城头,孑然独立的那人,分明是自己相识之人,他高声道:“本王以为是谁,原来是伯乐神医王先生,你虽然也是中原人,可是却在我草原扬名,昔日在茫茫草原之上,各部酋长均待你如上宾,你不是北汉人,与其在上面被人当成仇敌,不如来本王帐下效力,本王愿待你如兄弟手足,荣华富贵,女子金帛,任你随意而取,你意如何?”

他这样公然招降,语气中隐隐带了挑唆之意,就是原本敌意不强的那些外郡青壮,也不由握紧了兵刃,虎视耽耽地望着赤骥。

赤骥微微苦笑,转身向下望去,高声道:“完颜酋长,昔日在下到你格勒部,受你厚待,我替你治好心爱良驹,你授我骑射之术,你我朋友相交,情义非浅。然而私情不能害公义,我本是南楚人,如今更是大雍之民,本与北汉不相干,可是不论是大雍、北汉还是南楚,都是中原一脉,汉家正统。今日若是你汗王到我中原游历,在下必然以礼相待,视若贵宾,可是你如今挥军南下,侵我汉家土地,就是我不共戴天的仇人。不过在下念着昔日情谊,劝汗王一句,如今我中原即将一统,汗王虽然勇猛,却非是我大雍之敌,若是汗王果然为草原各部着想,不如息兵罢战,以免壮志成灰,草原血流成河。”

完颜纳金冷笑道:“中原分崩离析已非一日,如今又是内战连连,哪里还有力量挡我大军南下,本王也不贪心,只要取了代州,让你中原之人无力阻我铁骑即可,你若不降,休怪本王手下绝情。”

赤骥冷冷一笑,取出一支羽箭,折为两段,高声道:“今日我折箭为誓,你我恩断义绝,汗王尽管来攻打雁门,我就是死在汗王箭下,也是死而无怨,只是汗王若是死在我手上,也不要怪我负义。”

完颜纳金剑眉一轩,高声道:“你自寻死路,也怪不得本王,开始攻城!”在他一声令下,蛮军向雁门关扑去。

赤骥说完这番话,回头望去,他心中忐忑,不知道这些人是否能够接纳自己和他们并肩作战,一回头,一袋羽箭塞到他手中,他看到林远崇热情洋溢的笑容,抬头四顾,众人眼中都是一片温暖,赤骥只觉得热泪盈眶,却是无法说话。众人都看向林彤,毕竟赤骥能否留下,还需林彤决定。林彤别过脸去,淡淡道:“还不去守城,蛮人要上来了。”赤骥心中一阵激动,紧紧握住弓箭,热泪滚落。

这时,完颜纳金轻声叹息,对于那个王骥他颇为了解,昔日相识之时,就觉得这人才华过人,可惜当时他虽有野心,却碍于力量不足,不能公然强留草原上人人敬重的伯乐神医,只能以情义接纳。今次他趁着各部受灾严重,趁机利用囤积的粮食控制了各部,逼迫他们歃血为盟,重建汗廷,恢复昔日完颜家族的荣耀,可是当时王骥已经消失无踪。方才王骥救下林彤,破坏他立威之举,他心中愤怒之余,想要借着他和雁门守军的矛盾毁了此人,免得对自己攻取代州的计划造成不好的影响,可惜却是功亏一篑。中原人不是最喜欢内斗的么,完颜纳金有些郁闷地想着。

\chapter{第三十四章 势定收官(上)}

北汉国主闻沁州兵败,集重兵拱卫晋阳,四月二十二日,嘉平公主率残军返晋阳,民皆惧雍人报复,扶老携幼避难北上,日行三十里,故龙将军爱将段无敌者,素以擅守闻名,自请断后,护民北上。

太宗入汉境,闻汉主退守晋阳,笑曰,当先断外援,乃舍晋阳,绕道轻取楼烦关,陈兵于忻、代间。

——《资治通鉴·雍纪三》

沁源城的将军府,一间雅阁之内,指着棋坪上面黑白相间的棋子,我谆谆善诱地道:“一局棋粗略的分,可以分为三个阶段,布局、中盘和收官,若以战争喻之,布局就是战前双方集结明里暗里的力量,互相试探,布置兵力的过程,若是布局有所差池,则等于是授敌于柄,所以下棋布局不可不谨慎,就如这次攻北汉,初时表面上只是我大雍泽州军与北汉沁州军之间的交战,可是北汉外结南楚为援,又挑动我大雍内部生变,除了沁州军之外,又调动代州军行雷霆一击,布局不谓不深远,手段也是狠辣激烈。可是朝廷利用南楚内部的矛盾,断去这个外援,对于内部变乱,则是采取手段,控制其发展,不令其影响大局,至于正式的作战,除了泽州军之外,又密遣长孙将军来援,我军不论事先的庙算,还是兵力的集结都胜过了北汉,所以才为取胜奠定了基础。

至于中盘,则是双方绞杀的过程,可以说大部分战争胜负在中盘可以就可以决定,这次我军和北汉军在沁州的作战,可以说就是双方博弈的过程,稍有不慎,就是一败涂地,安泽、沁源、沁水河谷,我军可以说连败三阵,但是由于情报及时,再加上殿下身先士卒,苦战断后,才能够将敌军诱入合围,若非如此,只怕我们设下的埋伏就成了最大的笑话了。

而收官则是结束作战的过程,如今我军已经控制了大局,但是如果不步步为营地作战,还是有失败的可能,或者被敌人拼个鱼死网破。”

如今已经是四月二十三日,我军已经攻下了沁源,不过与其说是攻取沁源,倒不如说是北汉军主动放弃了沁源,四月二十日,林碧带着代州军残军和沁州残军会合,被段无敌接应回沁源。根据我军谍探探听到的消息,北汉国主已经有了命令下来,让林碧撤回晋阳,这也是无可奈何的事情,如今北汉若是再分散兵力,只有被敌人各个击破的结果,若是集重兵于晋阳,还可保全实力。而且晋阳乃是北汉国都,地势险要,若是不能攻下晋阳,大雍将来纵然攻城略地,也是很难守住这些城池的,因此撤退已经成了唯一的选择。可是我军当然不能任由敌军就这么轻松地撤退,所以大军开始以雷霆扫穴之势步步推进。这种时候,我自然不需要随军而行,就留在了沁源坐镇,当然我不是一个人,还有荆迟也留在沁源养伤。这次他受伤极重,虽然保住了性命,可是没有半年时间的调养,根本不可能重新上阵,至于军务自然有别人去操心,我闲着也是闲着,就拉着荆迟陪我下棋,荆迟性情粗率,对围棋并不感兴趣,不过我自然有手段让他乖乖来学,也趁机教些军略给他,免得他只知道杀伐,想要为帅独当一面,他现在可还差的远呢。

坐在我对面的软榻上,面色苍白,但是神色还算不错的荆迟看着棋盘,见我讲得兴起,他却偷偷打了一个呵欠,被我瞪了一眼,他尴尬的一笑,想要敷衍过去,便问道:“先生,我军是如何收官的呢?”

我轻轻摇头,孺子不可教也,我还是说些现在的情况吧,至于能够领会多少就看他自己的了。

收起棋子,整理好棋坪,令呼延寿取来一张北汉地图就放在棋坪上面,然后将几枚白棋子放到晋阳的位置,道:“晋阳如今集结了北汉的大部分军力,除了原本是十万守军,还有各地汇集的五万军队,这些军队战力参差不齐,但是仍可一战,而沁州败军仍有三万,段无敌手中也有数万兵力,再加上嘉平公主的代州军,至少可以集结五万人回到晋阳。所以现在北汉已经将全国之力都集中在晋阳,他们是希望守住晋阳,晋阳百万军民,城高水深,粮草可以支持一年以上,再有精兵强将守城,可以将我军拖在北汉境内。晋阳乃是百战之地,若是不能攻取,就算我们攻下了北汉其余国土,也是不能据有其地。所以这收官之战也非是轻而易举,朝廷想要的是完胜,而不是两败俱伤,鱼死网破。所以现在我军收官的第一步就是压缩敌军的生存空间,断绝其外援。”

荆迟听了目光立刻落到了代州,便指着雁门道:“先生,前几日军报不是说蛮人叩关么,难道代州还有能力援救晋阳么?”

我笑道:“代州如今局势非常紧张,现在蛮人八部已经重新重立汗王完颜纳金,猛攻雁门关,代州军主力又被林碧带走,一旦雁门关被攻破,蛮人必定长驱直入,劫掠无度,甚至还会占据代州,窥视忻州、晋阳。如果代州可以抵御蛮人,我们尚可任之由之,可是现在这种情况,若是代州最后不保,必然将军民撤到忻州,在北汉两面受敌的情况下,代州军会和晋阳合兵一处,到时候不仅晋阳得到强援,还让蛮人侵入国境,恐怕到时候直接面对蛮人的就是我军了,若是北汉王室再有人提议和蛮人媾和,用金帛土地诱使蛮人和我军为敌,我军可就是真的陷入困境了。而且嘉平公主军略不在龙庭飞之下,她现在已被推选为主帅,率军返晋阳,若是有她主持晋阳军务,想要攻下晋阳可以说是难如登天。”

荆迟看了地图半晌,道:“嘉平公主得知代州的军情,恐怕会日夜兼程,返回代州吧,怎会有心镇守晋阳。”

我笑道:“你能想到这一点也算不错,不过现在林碧不可能回代州了,皇上出了潼关之后,没有直接到晋阳,而是绕到楼烦关,现在代州已经和忻州、晋阳隔绝开来,按照我原来的计划,只要大军守住楼烦关,就可以将蛮人挡在代州,我军可以坐视代州军和蛮人两败俱伤,等到晋阳平后,再从容收拾残局,到时候蛮人必定已经攻取代州,我军可以趁势消灭八部主力,这样一来,蛮人十几年之内元气难复,而代州遭此重创,也可便于日后的统治。”

荆迟听得心里发冷,道:“先生也太狠心了,这样一来,代州勇士岂不是死的干干净净,虽然老子被他们追得屁滚尿流,可是老子还是很敬佩嘉平公主和代州军的。”他心中不满,口气也有些异样,若是往常,定然不敢如此。

我瞪了他一眼,道:“不消减敌人的实力,难道和敌人硬拼么?”

荆迟吞吞吐吐地不敢反驳,可是眼中却是明明白白的反对,我见状一笑,道:“你不用这副表情,皇上已经驳回了我的计策,皇上考虑得更深远,蛮人是不可能一举消灭的,以后代州仍然是抵御蛮人的重镇,若是代州元气大伤,对于日后抵御蛮人,必然有很大的影响。而且代州林家世代镇守边疆,对权势富贵都不甚重视,林氏虽然在北汉地位显赫,可是据说家无余财,所有俸禄金帛都用在军费和抚恤上了,而且他们并不完全听从晋阳的命令,虽然北汉国主和代州乃是姻亲,可是除了今次北汉生死存亡之际,代州军从未出境相助,即使这一次出战也不是因为两家的姻亲关系,而是因为北汉王室这些年对代州的援助让他们生出报恩之心。这样看来,林家并非北汉忠臣,他们的忠义只对着社稷黎民,并非是对着哪个朝廷,这样的林家乃是纯臣,所以对于林家,皇上不仅不想剿灭,还想保全林家的力量。皇上说,林家有功于代州乡梓,更是北疆铁壁,不可轻易撼动,若是按照我的计策,不仅可惜了林家,自毁长城,而且也会让代州人对我大雍恨之入骨,不利于将来的统治,所以皇上决定对林家进行招抚,就是对北汉王室,皇上也不想斩尽杀绝。”

荆迟听得大喜,脱口道:“我说么,这样的阴毒计策皇上才不会采用呢,皇上平生最是爱才惜才,对于忠义之士更是礼敬有加,若是沙场上针锋相对,就是杀了林家一门也不出奇,可是利用蛮人对付林家那可不是皇上做得出来的。”

说完这番话,荆迟只觉得脖颈后面突然有些寒毛倒竖,立刻想起来自己这番话可是将江哲骂得痛快淋漓,忍不住偷眼望去,只见江哲神情似笑非笑,状似不在意地玩弄着手中的几枚棋子,不过荆迟怎么看都觉得那笑容中带着缕缕杀气,有些畏惧地向后移动了一下,荆迟讷讷道:“那个,先生,我不是在骂你。”

我笑道:“我又没有怪你,你看,现在齐王殿下和长孙将军已经开始分兵进攻,齐王殿下追击沁州军,而长孙将军负责平定四方,在我军晋阳会师之前,要将北汉的所有反抗力量消灭压服,或者驱逐到晋阳,不过你是不能参加了,谁让你如此轻信,让北汉魔宗的密谍近了身旁,害得自己重伤不说,还让北汉军冲出了五六万人。等到将来战后论功,你初时入壶关一路奔袭,杀伐极重,就是皇上心里不介意,也不免要重重罚你,一来安定民心,二来以儆效尤,从沁源到冀氏,你虽然一路上断后苦战,可毕竟是败仗,最多是将功补过,真是可惜啊,围歼北汉军这样的大功劳,你又因为遇刺而失职,看来这一次你是只有苦劳,没有功劳了。”

荆迟只觉得十分憋气,听着那似是惋惜实是讥讽的话语,越发郁闷,却又不敢不听,幸好江哲很快就停止了嘲讽,开始指着地图继续讲了起来,荆迟心中一宽,他对江哲的脾气略知一二,既然他立刻嘲讽了自己一番,那么就不会再记恨了,也就放心地听着江哲继续讲解如何“收官”。

用棋子标示出敌我两军的位置,我指着沁州城道:“沁州城乃是沁州首府,龙庭飞帅府所在,现在北汉军正在这里整顿军马,准备继续撤退,为了逼迫敌军进一步分兵,齐王殿下令人散布流言,说是一路上雍军将要遇城屠城,现在我军进军路线上的民众都在涌向沁州城,沁州百姓多年来支持龙庭飞和我军作战,本就心中惴惴,而且龙庭飞一死,他们信心全无,所以才会扶老携幼,想要北上逃亡,沁州城被流民涌入,根本无法防守,除非林碧等人可以狠心将流民逐出城去。但是这种事情就是北汉将领做的出来,也难以安抚和沁州民众有着千丝万缕联系的沁州军,所以不论是为了王命,还是为了生存,北汉军只有一个选择,就是北返晋阳。原本我不过是希望北汉军失去民心罢了,想不到还有痴人,段无敌已经主动留下断后,现在流民一日只能行数十里,他带着本部不到两万人徐徐断后,现在应该快被殿下追上了。对了,知道为什么北汉人这么相信我军会屠城么,齐王令人打了你的旗号在前锋,说是你不过是轻伤罢了,现在已经负伤上阵,准备报复屠城呢。”

这下子荆迟可是瞪大了眼睛,委屈地看着我,我却是哈哈大笑,这下子方才那口气可是全出了。

过了片刻荆迟都督囔囔地说道:“反正就是我倒霉,若是真的让我去屠城也就罢了,偏偏只是担个虚名。”,我面上神色不变,却是强忍笑意,他虽然说得小声,我可是听得清清楚楚。看看荆迟已经有了倦意,让他好好养病,我返回自己的书房。

这件书房原本是段无敌所使用的,书房里面仍然留着许多段无敌不及带走的书卷文稿,他虽是武将,倒是颇通经史,看他留在书房里面的笔记和一些文稿,虽然文字有些粗浅,但是意境倒是颇为深远。我取过昨日还没有看完的一本笔记,接着上文翻阅起来,里面多半是他读书时候记录的心得和一些随笔,还有一些类似记事的文字,这可是了解一个人最好的途径,尤其是想要收官,他可是其中一个关键啊。对于荆迟,我只是说了军事上面的一些事情,还有一些事情,他是不必知道的。

段无敌这次负责断后,他手上可是有一个重要的人质的,就是宣松,我已经得到小顺子和苏青的消息,得知宣松仍然活着,只是受了伤被拘禁着,虽然找到了人,可是就是小顺子再厉害,也没有办法从重围中将宣松救出,而苏青虽然千方百计的设法,但是沁源被段无敌管制的如同铜墙铁壁一般,别说救出宣松,就是想联络上他,也是费了九牛二虎之力。尤其是林碧进入沁源之后,想救人更是休想,原本小顺子和苏青都已经有些放弃了,谁知道北汉军撤退到沁州城之后,林碧第二天就率军北上晋阳,段无敌自请断后,却将宣松暗中留了下来。说起来也是很巧,这宣松被俘一事知道的人不多,而知道的人除了林碧、萧桐和段无敌之外几乎都已经死在了冀氏,所以再取得林碧的默许后,宣松就被段无敌留作人质。得知这个消息,我自然猜到段无敌的用意,不过是希望通过宣松换取一些条件,但是想来他也不会过分,而且我早已经安排妥当,绝不会便宜了他,这一次,段无敌是注定没有机会回到晋阳了。

北汉战场这边大局已定,所谓的收官却不仅仅是指这里,东海那边我前几日传书过去,让他们放了秋玉飞,等到秋玉飞回到北汉,大局已定,而我就可以通过他和魔宗谈判,这样好的一个中间人,我怎会不用,否则当初又何必费尽心思留下他的性命,我可不会为了惜才的缘故而让自己置于危险,若非我有用他之处,怎会放纵自己的情感和他结交为友。还有,东川也应该平定了,想到这里,我踱步走到窗下的一局残棋前,将一粒棋子轻轻放在棋盘上的西南角上,一子定乾坤,从此西南无事,不知道一个人从最高处陨落的感觉是什么样子的,我有些不怀好意地偷笑起来。全然不知站在书房外面守卫的呼延寿打了一个冷战,心道,不知道又有谁要倒霉了。

此刻的南郑城中,昔日的蜀王行宫,庆王府邸,现在已经是刚刚“复国”的蜀王宫,新任蜀王孟旭不过是个小娃娃,正在母妃和一群侍女内侍的照看下玩闹。如今的蜀国王太后戚氏不过是个二十多岁的青年女子,昔年本是金莲夫人的侍女,因此有机会得到蜀王宠幸,怀了身孕之后,也还没有晋位妃嫔,若是蜀国不亡,她最多不过是后宫一个普普通通的妃子,她的儿子也不过是一个地位低微的小王子罢了。可是如今,她却成了旧蜀遗臣复国的旗帜,不论这件事情对她母子是幸运还是不幸,她都是无能作主的傀儡罢了,所以虽然贵为太后,她仍然是神情忧郁,只有在看到爱子的憨态之后,才会偶然露出一丝笑容。

孟旭在宫女帮助下,终于折下一支桃花,连跑带跳地拿着桃花扑进母亲怀中,高高地举着花枝要母亲拿着,戚氏心中涌出强烈的喜悦,一把抱紧爱子,心道,若是能够和爱子无忧无虑地度过平静的一生,该有多少。就在这时,戚氏耳边突然传来几声闷哼,戚氏抬起头,正好看见最后一个内侍被击晕在地,而出手之人却是一个穿着侍卫服饰的中年人,那人相貌儒雅,神色有些阴郁,戚氏惊呼道:“顾侍卫!”

戚氏仓惶四顾,只见一左一右两个中年侍卫已经将其他的侍卫宫女全部制住,这两个侍卫一个满面虬髯,相貌威猛,一个鹰目薄唇,相貌森严冷峻,却是没有见过。她抑制住呼救的冲动,强做镇静地望着这几个心存恶意的中年人。自从蜀亡之后,她奉了蜀王和金莲夫人之命逃出王宫,后来被侍从出卖给庆王,虽然庆王为了利用他们的身份而没有加害,可是戚氏也已经历经劫难,早不是昔日的无知女子,她知道若是胡乱呼救只能让眼前这三人痛下杀手,因此不仅不敢呼救,还伸手将孟旭紧紧抱在怀中,还捂住孟旭的嘴,不让他惊叫出声。

其他两个侍卫已经退到顾侍卫身后,戚氏知道这顾侍卫乃是三人之首,她隐隐记得,这人叫做顾宁,身份颇高,虽然来到宫中不过几日,可是侍卫中很多人都对他极为尊敬。而且这人平日礼数周到,从来不曾因为她母子的傀儡身份流露出轻视之意,但是为何这人突然痛下杀手,她用戒惧的目光望着顾宁,道:“顾侍卫,你要对本宫和王儿做什么?”

顾宁轻叹一声,手按刀柄,缓缓走到戚氏面前,拜倒道:“草民奉命前来取王上性命。”他奉了霍纪城之命进入蜀王宫,为了行事方便,只带了两个结义兄弟,章函和何匀,这两个兄弟都对复国大业无甚兴趣,只是为了兄弟之情才和他共同进退罢了。

戚氏面色苍白,道:“是奉了庆王之命么?现在他应该还不敢杀死我们才是。”

顾宁听到此处心中一动,心道,这个道理就连这妇人都知道,盟主又如何不知道,他为何迫我冒犯王上,莫非他有什么诡计,可是无论如何,自己终究是难以逃出那人控制。

他黯然道:“太后,臣也是不得已,还请太后恕罪。”说罢起身拔刀,犹豫了一下,挥刀下斩。

戚氏虽然无力反抗,可是身为母亲的本能让她尽全力将爱子抱在怀中,用身躯挡在钢刀面前,就是死也要死在爱子前面,而且她心中仍有些许翼望,从这人的口气中可以听出,他心中杀意不重,似乎也是被迫而为,若是这人杀死自己,心中不忍之下,或许杀意更会消退,说不定爱子还能留得性命。

钢刀蓦然停住,距离戚氏不过一线之差,顾宁额头青筋暴起,那一刀无论如何也劈不下去,他本是忠义之人,如何能够对王室中人痛下杀手,就算戚氏母子不是这样的身份,身为侠义之士,他又怎能对妇孺下此毒手。

戚氏见状连忙跪倒在地,泣道:“顾侍卫,求你刀下留情,饶了我母子性命么,妾身母子终身感激不尽。”

顾宁的目光犹疑不定,面上露出挣扎的神色,这时,那个鹰目薄唇的中年男子冷冷道:“顾大哥,你别忘了彦儿、暴儿还在霍义手上,英儿更是生死不明,你若不遵从盟主之命,孩子们怎么办?他们母子不过是庆王的傀儡,难道你还真的当他们是什么王上,太后么?”

戚氏闻言连忙哀求道:“顾侍卫,妾身和旭儿身份并无虚假,但是妾身不敢以此奢求饶命,只求顾侍卫看在我们孤儿寡母的份上饶过我们性命,若是有所不便,只要能够饶过旭儿性命,就是将妾身千刀万剐,妾身也无怨言。”她听出顾宁似乎也是因为子侄被执不得已才要取自己母子的性命,所以婉转以母子之情感动其心。

顾宁听到此处,终于长叹一声,放下了钢刀,黯然道:“姑且不论这孩子乃是先王骨肉,只论江湖道义,难道我顾某可以借着杀死人家母子来救自己的骨肉么,几位兄弟,我已经决定离开锦绣盟,盟主心性乖戾,迟早会将我们一一杀死,若是你们愿意,就和我护着他们母子离开吧,不论是庆王还是盟主,都是心狠手辣之人,我不忍先王遗腹子死在那些野心家的手中。”

两个中年人面面相觑,那个虬髯大汉问道:“大哥,那么几个孩子怎么办?”

顾宁痛苦地道:“盟主手段狠毒,我只能试一试去救他们,你们带着王上母子先离开,我去散关,想办法救回彦儿和暴儿,至于英儿,只怕是没有可能救出来了。”

那个鹰目薄唇的中年人叹息道:“我本就是因为与大哥的兄弟之情才留在锦绣盟,否则那霍纪城虽然手段厉害,又怎能驱策于我,既然大哥已经决定和锦绣盟恩断义绝,我自然没有异议。你们可愿意和我们一起离开?”最后一句话却是去问戚氏,戚氏心中忐忑,虽然这几个人原本想要杀自己,可是看起来他们倒是并非恶人,其实对于庆王,她也没有信心,再说若是不答应,只怕这看起来就心狠手辣的汉子就会杀了自己母子,所以戚氏连忙点头道:“妾身母子就拜托几位侠士了。”

那虬髯大汉道:“大哥,我和你去散关,让老章带着他们母子先走吧。”顾宁心中感激,三人之中若论武功就是这大汉最高明,乃是锦绣盟中数一数二的高手,有他相伴,救出几个孩子的机会便大了许多。

那个鹰目薄唇的中年人皱眉道:“大哥,三弟他武功虽然高明,但是性子粗疏,救人需得靠心机和手段,还是我去吧。而且盟中兄弟有很多都受过大哥的恩惠,大哥可以让他们先隐瞒一下消息,这样我们还是有很大机会救出几个侄儿的。”

顾宁知道自己这个二弟章函虽然有些略嫌狠毒,可是却是心机深沉,颇富智谋,若非此人眼中只有自己,以他的才华,早就得到了霍纪城的重用了,他的计策必然有着较高的成功可能,所以他长揖到地道:“多谢兄弟助我。”章函笑道:“谢什么,当初若非大哥救了我的性命,只怕世上早就没有章函这个人了,而且说句实话,我也厌倦了这样的生活,能够隐居田园总好过朝不保夕,两年前我就建了一处秘密的庄子,这次我们就去那里种地打猎,过些逍遥的日子不是很好么。”

顾宁叹息道:“只看庆王行径,就知道他不是真心助我蜀国复国,霍盟主又是野心勃勃,复国无望,我们却能救出先王血脉,令先王宗祀不绝,也算是尽了忠义之道了。”

戚氏听到这里,才真得放下心来,她是个知道进退的女人,成为别人扶持的傀儡,并非她的意愿,若能够和儿子隐居乡野,倒也是心满意足,只是对这些人她心中仍有疑虑,不敢流露出心中所思,于是仍然沉默不语。当下何匀带了戚氏母子,在几个亲信弟子和不知真情的锦绣盟弟子协助下逃出了王宫,而顾宁和章函则直奔散关。其实虽然这三人努力掩饰,但是这种大事如何能够瞒过众多耳目,不过在三人走后,自然有人助他们将痕迹抹去,将消息隐瞒,不过这些就不是三人所能知道的了。

\chapter{第三十五章 势定收官(下)}

趁着天色将晚,攻打陈仓的大军陆续回营的混乱时机,私下里和章函见面之后,上官彦忐忑不安地回到和熊暴合住的营帐,虽然两人如今实际上是人质的身份,可是霍义并没有亏待他们,让他们两人住在一起,平日对他们也是没有丝毫折辱,若非是头上隐隐悬着利剑,对于精明能干的霍义,上官彦倒是感激尊敬居多。可惜他很清楚,只需一道令谕,这看似对自己两人颇为照顾的少年,就会毫不犹豫地处死自己两人,所以上官彦始终不敢掉以轻心。尤其是章函告知自己如今情况的变化,自己和熊暴需得立刻脱逃,他更是忧心忡忡。霍义虽然没有明言,但是自己和熊暴必须有一人随时在他身边听用,不能离开他的视线范围,如何能够两人都安然脱身呢,上官彦努力地想着。不过不论如何,现在他需要和熊暴说明此事,现在正是军中晚饭之前的休憩时间,熊暴应该已经从霍义身边离开返回营帐,而自己在晚饭之后还要到霍义身边听用,虽然只有半个时辰的时间,但是相信可以和熊暴说个明白,这样一旦事情有变,熊暴也不会随便落入别人的陷阱中。

走入营帐,上官彦只觉得心头一震,他看见霍义负手站在帐中,却是不见熊暴,莫非义父等人到此的消息已经走露,上官彦心里想着,却不得不上前施礼道:“属下见过公子,公子怎会到这里来,莫非是有什么紧要事情么?”

霍义朴实的面容上露出一丝冰冷的笑意,道:“盟主有谕令传下,今夜你们都需留下听用,若有违背,不仅你们自己要受重责,还要连累家人。”说罢,右手开始把玩着一块玉佩,眼中流露出浓厚的威胁意味。上官彦仔细一看,只觉得心中一冷,那块玉佩他认得,正是刚刚和他分手的章函身边之物,这块玉佩乃是章函四十寿诞时候,上官彦送给他的贺礼。因为章函平日对上官彦多有青睐,上官彦为表示心中感激之情才特意买了一块据说可以辟邪的汉玉献上,章函感于上官彦的孝心,几乎是玉不离身,就是方才,上官彦也看见他腰间悬挂此玉。如今这玉竟然在霍义之手,难道不过片刻之间,章函竟然已经身落虎口,想来熊暴也已经被拘禁起来。他忍不住按住腰间剑柄,一腔热血涌上心头。

霍义仿佛不知他心中变化,仍然笑道:“对了,令弟我们已经找到了,他毕竟年轻,竟然私自去截杀明鉴司的密谍,结果被人俘虏了,幸好明鉴司想要从他口中问出我方机密,才没有杀害令弟,这次洛剑飞率人捣毁了明鉴司一处密舵,结果救出了令弟,他虽然受了些委屈,但是精神还好,两位可以放心了,不需数日,你们一家就可团圆。”

这番话仿佛一桶冰水从头浇下,上官彦恢复了冷静,心中一阵悲哀,想不到自己等人终究是逃不出锦绣盟的控制,多日来苦苦支撑的意志终于崩溃,他颓然道:“公子还有什么吩咐请直说无妨,只是盟主这样对待我们这些盟友,实在是令人心寒。”

霍义淡淡一笑,那朴实的面容似乎多了几分狡黠,他对愤愤不平的上官彦说道:“其实也是你们一直不肯甘心听命,若是你们心中没有敌意,盟主和副盟主又何必和你们过不去,如今你的义父几个人也在我们监控之下,只需一声令下,就可以束手就擒,对了,他们劫持王上和太后,就是我将他们凌迟,也无人会替他们喊冤。”

上官彦大怒道:“若非是你们逼着我义父去弑君,义父怎会救出王上和太后,你们要杀就杀,何必还要陷害义父。”

霍义噗哧一笑,道:“原来你果然见过了章函,看来我没有猜错。”

上官彦一愣,这是怎么回事,章二叔不是已经被他们抓住了么?突然之间,他恍然大悟,看向霍义手中的玉佩,霍义一笑,将玉佩抛了过来,上官彦接住玉佩仔细一看,果然是一块仿制的玉佩,虽然惟妙惟肖,可是上官彦仍然从一些轻微的差异看出这不是真品,方才他只是一时之间急火攻心,才没有识破。识破机关之后,上官彦并没有轻松多少,从这块仿制的玉佩看来,霍义甚至是陈稹、霍纪城对自己等人都是早有戒心,一旦发动就是雷霆一击,绝对不容许失败,若非如此怎会仿制这块玉佩。如今霍义既然当面试探,那么定然是已经准备妥当。到了这种时候,上官彦反而心中一松,他心中明白,霍义绝对不会浪费心机在无用之人身上,他既然对自己使用手段,那么就是还有转圜的余地。上官彦不是不服输的人,叹了一口气,他颓然道:“不论智谋武力,我等都是甘拜下风,请霍公子明言吧,无论如何,只要上官彦力所能及,必定竭尽所能,只希望盟主能够手下容情,放过我义父和两位叔叔。”

见他如此,霍义微微一笑,上官彦果然聪慧,可惜不够狠辣,这也是自己找上他的缘故,这样的人比较好控制,虽然要放纵顾家的人,但是不能让他们脱离控制,所以必须在顾家安排下钉子,而上官彦就是最好的人选,他够精明,也识时务,只要保全顾家上下的性命,他就会俯首听命,而想要瞒过顾宁等人的眼睛,也只有上官彦有这个本事,熊暴粗率,顾英血气方刚,都不是好人选。

霍义拉着上官彦,让他坐在一边,道:“其实盟主本心,是不会为难你们顾家的,这次陈仓事毕,锦绣盟也将烟消云散,你们顾家也可以归隐山林,不问世事,不过顾家带走了蜀国王室余孽,这终究是祸患无穷,所以盟主之意,是要你随时监视,只要你们顾家没有借着蜀王余孽复国之心,在下可以代替盟主保证,绝不会伤害你们顾家一人。”

听到此处,上官彦心中一震,虽然他对锦绣盟霍纪城等人颇有微辞,可是从未想过霍纪城等人会和大雍有什么勾结,可是听霍义言辞,竟然隐隐透露出这令他难以相信的信息,他愕然望着霍义,不知道该说些什么。

霍义淡淡一笑,道:“这些事情你也不用多想,若是我们有心将旧蜀余孽一网打尽,也不会留着孟旭不管,只要你们顾家从今后安分守己,就可以保住平安,日后如何联系我会详细告知,现在你先和我去办一件大事,等到此事完成,你就可以带着熊暴逃离,至于顾英,我会告诉你去何处救他。若是不遵我命,就是阖家皆死,若是从了我命,最坏的情况也可以晚死几年。你放心,我不会让你去做些多余的事情,更不会利用你引诱那些复国志士入彀,我们主上有令,锦绣盟之事我们今后不会再过问,留下你这条线索,不过是为了防备一二。具体的情况,以后你可以去问顾英,只是不能再让别人知道。”

上官彦心中迷茫,他自然不知道陈稹等人的心思,将锦绣盟交到夏侯沅峰手中,明鉴司就可以借此掌控旧蜀的所有反抗势力,为了不让夏侯沅峰过分得意,陈稹和董缺设计让顾宁去杀孟旭,事实上,两人都早已猜到顾宁十有八九难以下手,而救走蜀王母子就成了唯一的选择。这样一来,夏侯沅峰虽然达到了平定庆王叛乱的目标,却让蜀王余孽逃走,有功有过,功劳不显而过错昭彰,必然会受到不知详情的人的攻击。夏侯沅峰虽然掌握了锦绣盟,却也接下了追查蜀王余孽的重担,陈稹等人相信顾宁自有办法逃过夏侯沅峰的追索,毕竟顾宁在锦绣盟中地位极高,人缘又好,在旧蜀又是根基极深,再加上陈稹等人的秘密相助,顾宁逍遥法外的可能性是很大的。当然为了以防万一,他们还是决定在顾宁身边留下一颗棋子,而上官彦就是最好的人选。虽然也有可能上官彦会在今后想尽办法脱离他们的掌控,可是这已经无关紧要,随着时光流逝,孟旭的重要性会逐渐下降,而上官彦等人本就已经没有复国之心,所以陈稹等人并不担心将来难以控制顾家。至于顾英,则是因为他已经知道了真相,又不便杀他,所以索性将他也算在其中。

随着霍义走出营帐,上官彦心中一片茫然,他自然不知道此刻,在几十里之外,顾英正被刘华谆谆善诱地说服,成为控制顾家的第二颗棋子。这并不困难,自从顾英被刘华软禁在身边之后,刘华就用种种手段将这个血气方刚的少年驯服。死亡的威胁,再加上刘华本是八骏之中隐组魁首,最是具有亲和力,轻而易举地就让顾英将他当成了兄长知己,又经刘华婉转说明保全顾氏的好意,顾英怎会不入彀呢?

庆王坐在帐中,想起今日攻打陈仓又是无功而返,情绪全无,锦绣盟虽然答应伺机刺杀陈仓守将,但是连续数日都是毫无动静,反而因为连续的攻城折损他不少心腹将士,他心中颇为不满,可惜叶天秀被他留在散关镇守,若不然李康真想让叶天秀前来暗中查一下,锦绣盟是否有意拖延,好提出一些过分的要求。可是散关那里刚刚到手,那个副将虽然投诚,但是毕竟还需小心,若非不愿意留下杀降的名声,也不想动摇军心,为了稳妥起见,李康本想杀了那个副将的。正在李康忧心忡忡地胡思乱想的时候,帐外有人道:“王爷,陈仓有喜讯传来。”

庆王一抬头,只见霍义匆匆走入,身后只跟着上官彦,手中却拿着一个还在滴血的圆形青缎包裹。李康心中大喜,几乎是不敢相信地道:“可是大事已成。”

霍义上前拜倒道:“王爷,盟主亲自出手,已经取了陈仓守将阴囹首级,现在陈仓城中一片混乱,请王爷立刻点兵,进攻陈仓,必可一举而下。”

李康强忍心中喜悦道:“将人头拿来我看,阴囹我是认得的。”

霍义膝行上前,捧着人头递上,满面喜色中带着激动的神情,李康心想他定是因为即将攻下陈仓而兴奋,在和锦绣盟的盟约当中,如果锦绣盟刺杀陈仓守将成功,那么锦绣盟将要收获的权势非同小可,而霍义就是获利最大的人之一。不过李康仍然保持冷静,在起身接过首级的时候,仍然保持着若有若无的警惕,就如同平常一样。就在这时,李康一个亲信的将领冲进帐内兴奋地道:“王爷,陈仓城内灯火通明,一片混乱。”李康安排他随时监视陈仓城情形,如今他亲来报告,更是证实了锦绣盟果然成功地刺杀了阴囹。

李康这才放下心来,双手接过那包裹,一手托着,一手去解包裹,当看到那发结披散的人头,他丝毫没有厌憎之心,而是伸手拨开覆面的乱发,那双目紧闭,神情狰狞的容貌,正是他记忆中的阴囹,李康大喜。就在他心情一松的时候,单膝跪在地上的霍义已经暴起扑上。李康本能地将手中的人头击向霍义,合身向后退去,双手肤色突然变成金色,霍义手中擎着的匕首如同惊虹贯月,将那枚首级绞成粉碎,但是也就是一线之差。当霍义匕首直刺李康小腹的时候,已经被李康右手牢牢捉住锋利的剑刃,声如金铁,虽然李康是赤手空拳,但是手上却是丝毫血迹也无,李康目中寒光一闪,左手一拳击出,迫得霍义弃了匕首向后退去。只见霍义手中射出一枚弹丸,弹丸发出轻微的爆裂声,帐内立刻青烟滚滚,李康心中一惊,唯恐烟中有毒,向后疾退,左手反手一划,立掌如刀,寸许后的帐幕被他破开一个大洞,倒退而出。虽然他的视线被青烟所蔽,但是仍然察觉那霍义并未追击而来,反而耳中传来一声闷哼,他听得出是自己的爱将被人所杀的声音,那人竟连一声惨叫也没有发出。李康心中一痛,他对霍义和上官彦的武功颇为了解,知道这两人都不可能一招杀死那个将领,必是两人联手。李康虽然交手经验不甚丰富,但是他立刻想到霍义不追击自己,必然是另有伏兵,否则自己若是召来侍卫,他们必然死无葬身之地。

这番想法说来话长,其实不过是灵光一现,李康正欲移开身形,一枚尖锐之物刺入他脊背重穴,李康只觉得真气一泄,向下仆倒,还没有落到地上,一人贴地掠过,将他接住,穿越裂开的营帐,将他又送回了营帐。李康只觉得身体僵硬,再也不能移动分毫,不由一声轻叹,正欲高声呼救,那挟持他的人已经一掌切在他咽喉,李康只觉一阵剧痛袭来,再也无法喊出声来。这时候青烟已经渐渐消散,李康用目观瞧,只见自己的心腹将领已经倒在地上,右手尚按在剑柄上,肋下鲜血崩流,而上官彦站在帐门之处,手中佩剑鲜血淋淋,而那个将领咽喉处有明显的指痕,竟是被人用掌风切断了他的咽喉。这时,李康身后那人将他放到椅子上,走到他面前,那人正是陈稹。

李康只觉得嘴里发苦,虽然知道问也是无用,却还是勉强出声问道:“为什么?”

这一次陈稹没有阻止他说话,因为他知道李康这次是不会高声喊叫的,他微微一笑,道:“霍义,拿着王爷的令箭,召集军中众将到大帐候命。”

霍义微微一笑,走到书案上拿起一支金批令箭转身走了出去,上官彦眼中闪过一丝莫名的神采,望了陈稹一眼,从容地将剑上的鲜血在那已死的将领战袍上面拭去,跟着霍义走了出去。

陈稹拖了一张椅子坐到李康对面,从怀中取出一个玉瓶,从里面倒出一粒药丸塞到李康口中,李康无力抗拒,那药丸一入腹,李康只觉得一身真气仿佛春雪消融一般,渐渐失去。他断了暗运真气逼出背上暗器的念头,眼中闪过痛苦之色,再次问道:“为什么?”

陈稹淡淡一笑,道:“殿下何必多问,想来大雍的君臣也想问殿下,为什么好好的亲王不作,却要起兵谋反。”

李康仿佛没有听到陈稹的反驳,继续问道:“我自问对你锦绣盟仁至义尽,若非如此,怎能让你这样轻易制住我,我若失败,对锦绣盟有什么好处,难道你们不想复国么?”

陈稹眼中闪过讥讽,道:“复国,是你们这些王公贵族津津乐道的事情,陈某不过是个平平常常的江湖人,若是有安乐茶饭,谁愿意去做那些枉费心机的大事,大雍一统天下,其势已不可绾,你就是谋反成功,对你是有好处,对蜀国王室或者也有好处,可是对我们这些人有什么好处,荣华富贵可以让众人折腰,但是对于生死之间挣扎求存的人来说,不过是镜花水月。”

李康怒道:“不对,你们锦绣盟如此作为,既然不是为了复国,定然是和李贽有所勾结,否则何必如此,只是李贽能够给你们的,本王也一样可以给,为什么你们要背叛本王。”

陈稹听着大营里面渐渐响起的嘈杂声音,道:“王爷何必追根究底,今日之后,你我再无相见之期,王爷乃是天家骨肉,是生是死不是小人可以作主的,若是王爷仍然保得性命,小人说得多了岂不麻烦。”

李康惨然道:“你又何必如此谨慎,罢了你不肯说我也终会晓得,李贽总会让我死个明白,不过你要对本王手下都做些什么,可否说个明白。”

陈稹笑道:“闲着也是闲着,既然王爷想要知道,小人就多嘴一些,王爷帐外的护卫都是因为在饮食中被我下了秘药,方才我潜到帐边的时候,正是他们药性发作之时,若无解药,他们是绝对醒不过来的,所以也就不能保护王爷。那颗人头乃是用了易容之术,真正的阴将军自然还在陈仓严阵以待。方才霍义去召集军中将领,然后明鉴司夏侯大人将亲自动手,将王爷心腹将领一网打尽,至于军中将士,本就是大雍子弟,只需安抚,就可让他们归顺。对了,明鉴司刘大人将在散关动手,和那位献关的副将里应外合,散关到手之后,明鉴司将以雷霆之势扫清东川叛逆,只需旬日时间,就可以平定东川。”

李康只觉心头剧痛,口中一甜,一口鲜血已经喷出,他狠声道:“你们锦绣盟竟然是李贽的走狗,好,好,想不到名义上谋图复国的锦绣盟竟然是大雍的鹰犬,霍纪城想必是李贽的亲信,否则怎会将锦绣盟尽皆葬送,我明白了,昔日霍纪城必然是受了李贽指使,才故意和李安勾连,害了太子性命,李贽好狠的手段,好狠的心肠,好一个霍纪城,只可惜他这样的功劳却是不能公告天下,难道霍纪城就不怕兔死狗烹,鸟尽弓藏,恐怕将来天下人都会笑话姓霍的,说他目光短浅。”

陈稹神色不变,笑道:“殿下过虑了,一来此事和皇上毫不相关,二来霍纪城早已是死人一个,已经用不着兔死狗烹了,至于身后留下丑名,无言见人的也是霍纪城,和陈某有什么相干。”

李康误解了陈稹的意思,厉声道:“原来你是犯上作乱之人,莫非你杀了霍纪城,暗中投靠了大雍么?”

陈稹懒得和他多说,淡然道:“或者是这样吧,殿下还是多为自己考虑一下,不知道皇上会如何处置你这个落井下石的兄弟,对了有件事情王爷或者还不知道,北汉军败之事乃是谣言,齐王殿下在冀氏围歼北汉军主力,龙庭飞陨身沁水,如今北汉已经是日薄西山,只待皇上亲征晋阳,就可以一战功成。”

听到此处,李康只觉得眼前一黑,竟然是气得晕了过去,他素来自负,只道屈居东川,乃是因为父皇偏心,若是自己有机会成为皇储,必然胜过李安、李贽,想不到竟被这些草莽之人玩弄于股掌之间,一时气急攻心,竟然昏迷过去。

陈稹冷冷一笑,这时有人走进帐来,笑道:“陈兄果然厉害,舌锋如刀,心志深沉,若是陈兄有意,明鉴司尚有空缺,在下虚席以待。”进来之人却是夏侯沅峰,他一身轻袍绶带,素净的衣衫上却染着几处殷红的血迹,让他俊雅的容颜上带了隐隐的杀机。陈稹瞥了他一眼,道:“夏侯大人想必已经控制了军中大局,若是没有什么事情,在下就要告退了。”

夏侯沅峰上前一揖道:“陈兄,虽说是荣华富贵如浮云,但是大丈夫不可一时无权,你真的放得下一呼百诺的权势么,如今锦绣盟已将成为过眼云烟,陈兄今后不过是江侯爷身边一个侍从,冷冷落落,有何趣味,不若效命皇家,博得一个封妻荫子,也不枉人生一世。”

陈稹神情淡漠,默然不语,自从江哲将秘营交给他调度,他便将仅有的忠心给了那人,若是翼图荣华富贵,以那人的显赫身份,轻而易举就可以给自己一个锦绣前程,但是陈稹昔日就已经厌倦了瞒上欺下的密谍生涯,而在江哲手下,只要能够完成江哲交给的任务,其间却是可以任意而为,他自问不会有更好的主上,所以对于夏侯沅峰的话语,他是丝毫没有兴趣。

见他如此,夏侯沅峰无奈地一笑,道:“接下来的事情自有在下接手,陈兄可以随意了,若是还有什么事情交代,不妨现在直言。”

陈稹看了夏侯沅峰一眼,他心知此人心机深沉,若是自己流露出什么牵挂,只怕将来难以脱身,所以无心多言,只是漠然道:“大人尽可以动手,公子属下明晨即将离开东川。”说罢他拂袖而出,再也不看夏侯沅峰一眼,对于夏侯沅峰胁迫江哲一事,他仍是耿耿于怀。

第二日清晨,陈稹、董缺、白义(霍义)、山子(霍山)四人策马站在陈仓城外,望着雍军将东川庆王的军队进行整肃,霍义面上神情有些不安,山子见状笑着问道:“白义,怎么了,莫非舍不得锦绣盟么?”

白义道:“怎会舍不得呢,只是我忧心一件事情,骅骝有消息传来,他居然让叶天秀带着宋夫人逃走了,这终究是后患无穷。”

山子道:“不过是一个弱女子和一个剑手,普天之下,莫非王土,他们能够逃到哪里去,最多你让骅骝多派人手,将他们缉拿归案不就成了,倒是锦绣盟那边,我担心会有余孽漏网。”

董缺淡淡道:“怕什么,凭着名单和你关于密舵的机关图,夏侯沅峰足以将锦绣盟重要人物一网打尽,就是有几个运气好的人逃走,难道他还能找到咱们的踪影么?对了,剑飞的事情办的怎么样?”

陈稹道:“剑飞的事情很顺利,上官彦和熊暴已经救出了顾英,顾氏一门已经隐入深山,剑飞足可掌握他们的动向,不过等他们安定下来,剑飞就会离开,毕竟夏侯沅峰不是吃素的,如果他通过剑飞找到顾氏一门,我们的计划就白费了。”

众人相视一笑,都是觉得心满意足,不约而同策马离去,他们的方向乃是长安,他们将在那里等候江哲的到来。

隆盛元年四月末,陈仓城下,庆王叛军突然烟消云散,此时离庆王立誓恢复蜀国,不过短短旬日,庆王束手,叛乱的将领俱被擒杀,南郑城中,蜀国遗臣尽皆抄斩,血流成河,蜀国复国势力锦绣盟也遭灭顶之灾。这种种巨变,让主持其事的大雍明鉴司威震天下,只手平叛的夏侯沅峰也成了众矢之的。这一场复国谋逆闹剧便这样匆匆落幕。然而令心有余悸的蜀人略为宽心的是,新任的蜀王孟旭也消失的无影无踪。在这种种纷乱当中,自然不会有人留意到,庆王的一个侍妾宋夫人逃匿无踪,不过和她同时消失的庆王心腹亲卫叶天秀倒是有百两黄金的赏格。

自然也不会有人注意到,就在同时,大雍后宫之内也经历了一次秘密的清洗,别说几个内侍宫女被处死这种小事,就是昭台阁黄充嫒因为父族涉及叛乱而被打入冷宫这样的事情,也不过是风过无痕,转眼就无人再加以关心。

\chapter{第三十六章 忠贞见疑(上)}

荣盛二十四年,北汉兵败沁州,嘉平公主退守晋阳,雍军以屠城相胁,平民皆北上避战祸,烟尘蔽道,道路艰难,老幼皆号哭,无敌乃自请为后军。雍军煎迫甚急,然为无敌所阻,终因力竭为雍军所困,无敌以雍将俘虏宣松为质,胁雍帅解围,方生还。

然无敌未至晋阳,道路喧嚣,皆言其归顺敌军,北汉主不察,下诏赐死,时流言蜚语无数,无敌无可辩驳,唯嘉平公主知其冤,令其远走以避。

——《北汉史·段无敌传》

平遥城东三十里,荒村寂寥,渺无人烟,一队雍军斥候如同旋风一般沿着大路北上,离村子还有数里之遥,十几个雍军策马出阵,进村子转了一圈,回到队中,向为首的军官禀报道:“村中已无人烟,屋舍完好,可作扎营之处。”

那军官点头道:“不可小心大意,北汉贼子连日来多次偷袭骚扰,我军已经颇为疲倦,你们随我将村子好生搜查一遍,绝不能留下任何隐患,虽然中军自会关防,但是若是被他们发现有什么差池,只怕我们吃罪不起。”

那些雍军轰然应诺,除了十余人仍然在村外按刀戒备,其余人都是入村搜查,丝毫不放过任何可疑之处,为首的军官更是先捡出几间较为整齐的屋舍,里里外外检视了一遍,然后亲自坐镇,等待中军到来。

过了半个时辰,夕阳下金龙旗迎风招展,雍军中军到达荒村,随后大军开始在村外扎营,而雍军主帅齐王李显则是进了村中休息,早有侍卫将屋舍打扫干净,虽然不过是临时的住处,但是床榻换上李显行军所用的锦绣被褥,所有的用具器皿都是军中所携,就连窗子也覆上锦幔,原本简陋朴素的农居,不过片刻就变成了舒适华丽的行馆。

李显召众将一起用膳之后,便围着银灯商议军机,隐在屋角百无寂寥的正是邪影李顺,他神情阴郁,似是十分不快,只因不得不留在齐王营中,所以便被李显充做护卫,若非如此,他只怕早就寻个僻静的所在练功打坐去了。

李显有些恼怒地道:“这个段无敌,真真是油烟不进,本王猛攻,他便择险而守,本王稍有松懈,他便来偷营袭寨,要不然就来夺本王的辎重,这些日子,本王可是被他骚扰的苦了,明日我军就可以攻打平遥,此地乃是北汉有数的坚城,段无敌据城而守,只怕是又要耽误本王数日,你们可有计策,让他早些弃城,哼,只要等到本王到了晋阳城下,我看他还能翻出什么花样。现在长孙将军四处剿灭北汉各地的零星反抗军队,进展迅速,若是本王得他相助才能攻到晋阳,可当真是丢人得很。”

齐王爱将夏宁摩拳擦掌地道:“殿下,段无敌虽然难缠,但是只要他肯和我们正面对敌,还怕他作甚,殿下,请让末将攻城,不需三日,一定可以破城。”

樊文诚嗤道:“若是戮力攻城,还用得着你么,我们谁不可以指挥,殿下是想减少些损失,毕竟这次我们泽州军损失非轻。”

众将纷纷出谋划策,但是李显越听眉头皱得越紧,段无敌有平遥坚城为后盾,手中又有近万兵力,想要强攻必然损失惨重,他虽知段无敌的弱点乃是爱民,若是胁裹百姓攻城,或者用其他手段迫使段无敌不得不放弃平遥都是可能的,毕竟段无敌的目的不过是拖延雍军的行程。但是不说现在所经之处北汉民众几乎早已逃得影踪不见,就是能够捉到足够的平民,他也不愿在即将灭亡北汉之际加深和北汉平民之间的仇恨,虽然借着荆迟的嗜杀名声迫使沿途民众大肆逃亡,可是李显并不想真得屠城灭寨,他李显并非凶残成性,若是没有必要,可不想牵连无辜的平民。

李顺站在房间的暗影当中,忍不住轻轻撇撇嘴,若非公子曾经下过命令,对于宣松生要见人,死要见尸,他现在早就去了沁源服侍公子,何必赖在这里不走,还被齐王当成苦工,谁让宣松仍在段无敌手中,自己却寻不到机会救人,只有留在李显身边相机救人呢。见众人讨论的越发热烈,什么歪门邪道都开始盘算出来,李顺悄无声息地飘出房间,想呼吸一下冰冷的空气。外面的空气十分清新,李顺觉得心情舒畅许多,忍不住在暗淡的星光和明灭的灯火中漫步起来,将心神沉浸在天地之间,李顺静静地品味着无尽的黑夜。突然,李顺觉得一阵心悸,他若有所觉的向远处望去,隔着千军万马,铜墙铁壁,黑暗深处透着隐隐的杀气,那是一种熟稔的气息。

自从和凤仪门主一战之后,李顺获益良多,东海苦修,让他的先天境界更进一步,当世除了数人之外,再无对手,如今他已经掌握了“锁魂”之术,武功达到一定水准的人物,只要接近他一定距离之内,他的心灵上都能够有所警觉,这个距离并不固定,和双方的武功深浅密切相关,若是对方是平常之人,除非是刻意留心,否则很难在他心灵上形成警兆,若是对方是未进入先天境界的高手,就是十余里内,只要那人情绪波动稍为剧烈,他都能有所感应。若是对方也是先天极数的高手,那么变数就多了,若是对方修为胜过他,或者精于收敛之术,就很难发觉对方的存在,例如当日段凌霄行刺江哲,虽然是事先有所安排,可是在段凌霄出手之前,李顺确实没有明确的感觉到段凌霄的存在,如果对方就像黑暗中那人一般,晋入先天境界不久,修为尚浅,还没有达到锁魂境界,对李顺来说,这种先天高手比寻常存有敌意的高手更容易在他心湖上留下痕迹。

当然若是到了凤仪门主和慈真大师那种级数,彼此之间无论如何都无法掩饰存在,所以昔日在雍都,两人虽然不曾相见,但是对彼此的心绪变化和举动都是如同目睹一般,若是在那两人面前,李顺自知绝没有可能掩饰自己的心绪情感,幸好,那种宗师身份的人物,轻易不会出手。

李顺略一思索,已经从那熟悉中略有陌生的气息中有所猜测,且那人有杀气而没有杀意,身份更是昭然,他冷冷一笑,向暗处掠去,转眼间穿越连营,到了大营之外一处荒僻的山冈。只见残月疏星之下,一个黑袍青年立在冈上,神色淡漠中带着寂寥。在他身边站着一个黑衣少年,身后背着琴囊,神情也有些惨淡。李顺见到这两人,唇边露出淡淡的笑意,朗声道:“原来是秋公子回来了,东海风光如何?”

秋玉飞漠然道:“你当我是来行刺的么?”

李顺摇头道:“你是个聪明人,应该知道不可能,不过公子怎么这么快就放你出来了?若非公子手谕,你是别想从静海山庄脱身的。”

秋玉飞深深地看了李顺一眼,道:“你家公子行事,布局深远,放我出来,自然是有用我之处,只是我也未必让他如愿。这次本想去见见他,问他几句话,可是听说你在李显大营之中,想来就是我去了,他也不会见我。你倒也不用担心我会行刺于他,我若是敢出手,只怕桑先生不会放过我,桑先生的境界我不敢揣测,但就是师尊,也未必能够取胜。我已经传书晋阳,魔宗是不会有人去行刺楚乡侯的,有桑先生做后盾,就是师尊也不愿擅动杀机,更何况,北汉局势糜烂至此,就是师尊出手,也不能挽回什么,我魔宗不会做这等狗急跳墙之事。”

李顺拊掌道:“秋公子说得好,若是当初你有这样的聪明才智,只怕公子也难以利用阁下行离间之计。”

秋玉飞面色数变,半晌才道:“果然当日我是中了奸计,前些日子接到楚乡侯的书信,信中多有歉意,我就已经有了疑心,反复猜想,再经桑先生指点,才知道昔日我是受了蒙骗。”

李顺微微一笑,他早知江哲心意,必然会在这个时候透露出石英受冤屈的真相,用来打击段无敌,而秋玉飞突然返回北汉,他便料到江哲会将真相让他知晓,试探之下,果不其然。

秋玉飞轻轻一叹,转身欲行,却又顿住脚步道:“当日随云与我中道相逢,我虽是存了歹意,却仍视他为知己,不知他可是始终虚情假意?”

李顺肃然道:“公子纵然心机深沉,若非阁下才华横溢,人品脱俗,公子焉能以清远琴谱相赠,那琴谱乃是公子亡父心血,公子若是虚情假意,焉能忍痛割爱,阁下若是仍然因为敌对之事怨恨公子,倒也悉听尊便,只是却不可怀疑当日公子的一片诚意。”

秋玉飞默然良久,举步离去,那少年正是凌端,跟在他身后亦步亦趋,不多时两人就已消失在夜色当中。

李顺眼中闪过寒意,目光仿佛穿越重重黑暗,望向平遥城,如今苏青应该已经安排妥当,现在想必从平遥到晋阳,都已经流传着龙庭飞中了离间计迫死石英的传言,如今龙庭飞已经死去,那么昔日有关之人便要受到更大的压力,段无敌在这件事情起了不少的作用,必然会受到北汉上层的苛责。就算是嘉平公主等人明白段无敌的无辜,只怕他也难以谅解自己的行为。

想到当日受命之时公子神神秘秘塞给自己,让自己在齐王进兵之时交给苏青的锦囊,李顺也是不由心折,在黯淡的月光下,他从已经拆开的锦囊中取出一张短柬,上面写着寥寥几行字。

“令苏青散布流言,提及昔日离间石英之事,以乱段无敌军心,段心地仁厚,不肯负人,必然惭愧欲死,举动若有差池,则乘机间之,其在朝中无人,值北汉生死存亡之秋,易为所乘。”

李顺淡淡一笑,右手轻搓,那张短柬灰飞烟灭。

第二日,李显开始攻打平遥,完全是中规中矩的作战方式,凭着雍军雄厚的兵力和连绵不绝的攻势,进展颇为顺利,到了未时,李显亲自指挥攻城的一面的城墙防守开始有些崩溃的征兆,在投石机的猛攻下,城墙一角突然崩塌,雍军立刻高声欢呼起来,顺着城墙的缺口,无数雍军借助云梯等攻城器械开始向内攻入,缺口附近的北汉军死命抵住,但是却仍然阻不住雍军的攻势。

这时,段无敌冷静的下了军令,他身边的亲卫几乎是不可置信地望着他,但是素日的威严让他毫不迟疑地传下命令,守在那缺口的北汉军听到号令,立刻让出了一条通路,当前面压力骤然减轻而攻上了城头的雍军欢呼之时,机簧响动,早已严阵以待的北汉军发动强弩,这些强弩上面都缠着黑火药硝石等引火之物,点燃之后射入雍军当中,接二连三的爆炸让雍军立刻大乱,这时,原本避在一边的北汉军蜂拥而上,将他们击溃杀死,趁着雍军攻势受挫的瞬间,北汉军将火油倾倒下去,然后丢下火把,城下火焰熊熊,城上血光迸流。

当最后一个雍军被斩杀的时候,段无敌走近城墙,双手按在两侧被鲜血浸透的墙跺之上,向下望去,只见雍军开始撤退,如同海水退潮一般的迅速,那其中隐隐的压力和威势让段无敌面上神色越发冰冷。回望城头烟烧火燎的残破景象,遥望数里之外连绵数十里的敌营,段无敌心中一阵冰冷。

虽然逼退了敌军,可是他心中没有丝毫轻松。虽然雍军是今日才开始攻城,可是从前日起,城中流言四起,虽然这城头上没有人敢于当面议论,可是段无敌知道那谣言是说自己走私贪赃被石英告发,自己则在龙庭飞面前进了谗言,构陷石英入罪,并迫害石英致死。他身边亲卫忿忿不平,几次请命要将散播谣言的人查出来杀了,却都被段无敌强行压下。他不是不知道军心稳定对于守城的重要性,可是他却不能严厉追查此事,只因他手中的军队除了他自己的旧部之外,还有三成是石英的旧部,而传播谣言的大多是石英旧部。他们倒也不是存心如此,哪一个军人不希望自己的将军爱兵如子,作战英勇,若是在一个身负污名的将军麾下,那种耻辱只怕一生都无法洗清。昔日石英死后,声名尽毁,这些旧部不知因此受尽多少屈辱,如今得知自己的将军乃是被人迫害致死,怎能不互相传告。对于他们来说,“受蒙蔽”的主帅龙将军既然已经死了,那么需要为此负责的自然是“进谗言”的段无敌。这样一来,石英旧部人人心怀怨恨,就连自己的部属中,有些也生出疑心。可是对于这种情况,段无敌却又无能为力,若是自己想要肃清谣言,必然要波及许多无辜将士,只怕还没有等到敌军攻城,己方就已经自相残杀了,无奈之下,段无敌只有借着当前严峻的军情暂时压制众军,若是能够回到晋阳,还有机会挽救军心吧,他只能这样安慰自己。

这时,在两个北汉军士的“保护”下,宣松走上了城头,他的伤势已经渐渐痊愈,虽然面上疤痕宛然,但是已经可以行动自如,自从北汉军从沁源撤军之后,他便在段无敌军中,段无敌对他颇为客气,只要不是在行军或者作战的关键时候,监管虽然严密,但是并不苛刻,所以他才能在这个时候上城来。望着城头残破的情景,宣松心中有些黯然,他已经从北汉军口中得知了方才的血战,当然,这是因为那些北汉军将士很想打击一下他这个大雍将军,他自然知道城头的斑斑血迹代表了什么,但是他没有露出悲伤的神情。身为大雍将士,本应有战死沙场的觉悟,悲伤和同情能够有什么用处,难道他可以为了减小伤亡,不让雍军攻城,难道他可以说服北汉军放弃抵抗。只有天下一统,才可以让这种无所谓对与不对的血战不再发生。

看到段无敌的背影,宣松心中生出敬意,就是这个人,多日来连续苦战,阻碍了雍军的进攻,让数以百万的北汉军民得到了撤退和逃亡的机会,宣松清楚,虽然大雍也是军纪严明,可是这并不能保证不会伤害无辜的北汉平民。此人忠义爱民,若是能够说服他投降,大雍可得良将贤臣,想到这里宣松朗声笑道:“若论守城,天下无人能够胜过段将军,齐王殿下一日之内数次猛攻,都被阁下击退,只不过雍军兵力雄厚,将军外无援军,城中军心不稳,粮草困乏,不知道能守住几日?”

段无敌也不回头,平静地道:“再守两日即可,嘉平公主传来军令,晋阳一带百姓都可进城了,到时候晋阳内有百万军民,粮草军械都不缺乏,就是守上一年半载也是易事。”

宣松叹息道:“纵然如此,北汉又能支撑多少时日,虽然无人和我说起,我却知道,如今的局势对你们来说是何等不利,不说龙将军殉国之事,只见嘉平公主下令收缩防线到晋阳,就知道你们已经没有取胜的希望了,只能凭借晋阳的地利死守,保留最后的生机,除非我大雍最后不得不撤军,否则北汉亡国已成定局。段将军,你纵然不爱惜自己的性命,难道也不爱惜自己麾下将士的性命么,如今,雍军已经包围了平遥,齐王殿下不过是担心你在后面袭击粮道,加上时间充裕,所以才戮力攻城,否则只要留下几万人围着平遥,大军就可继续北上了。你想要多守两日,只怕是再也没有机会返回晋阳了。”

段无敌没有反驳,这些日子他和宣松数次详谈,虽然双方都存了戒心,不过是想多套取一些情报罢了,可是彼此对于对方的才能都颇为敬重,两人都是善于防守的将才,所以宣松只是这么看了几眼,便知道城中虚实。宣松所说一字不假,而且有些事情段无敌已经知道,却没有透漏给宣松,比如说,雍帝李贽亲征的消息,以及李贽的大军已经截断了代州和忻州道路的消息。对于这件事情,段无敌心中分外不安,虽然因为代州军归家无路,已经被迫留在了晋阳,甚至嘉平公主也已经正式接受国主的诏令,成了北汉军晋阳主将,可是段无敌隐隐觉得,这恐怕是雍军很重要的一步棋,可能将令北汉土崩瓦解。只可惜他是一个军人,有些事情他还是不甚了解,对雍帝的这个举动,他只是近乎本能的觉得危险,却不知其真意。

宣松见段无敌默认了自己的说话,又道:“再说段将军的处境似乎也不大好……”刚说到这里,段无敌举手阻止了他的下文,沉声道:“亦予心之所善兮,虽九死其犹未悔。”宣松身躯一震,望向段无敌坚毅端凝的面容,终于叹息道:“段将军既然此心不悔,宣某也不愿玷辱将军清名,只是信而见疑,忠而被谤,此乃千古之悲,贵国王上虽非昏庸之主,然而值此危亡之时,也难免过分谨慎,希望若是到了不可挽回之时,将军也不要愚忠到底才是。”

段无敌终于回过头来,淡淡道:“若是我放宣将军回去,阁下何以相报?”

宣松早有准备,若非是有利用自己之处,不是早早一刀杀了,就是将自己交给嘉平公主带去晋阳,何必要费力留在军中,望向段无敌憔悴而又平静的面容,他笑道:“陷敌之将,本无自主之权,阁下若有此意,不妨派使者去见见齐王殿下。”

段无敌从容道:“总要再守一日,方有讨价还价的余地。”

宣松不由苦笑,想不到自己竟然成了货物,和段无敌目光相对,宣松的苦笑渐渐褪去,他能够看得出来,对面那男子眼中深沉的悲哀,自己所说的一切,他都很清楚,若论才干,段无敌绝对在自己之上,只是自己有幸做了雍臣,而此人不幸却是汉将,“虽九死其犹未悔”,能够吟出这名句,可见其人心中早已经有了明悟。他深深一揖,道:“若是宣某回到雍营,而殿下又不怪罪的话,必然会率军和将军作战,若是将军不幸受困,还望将军不要一心求死,倒是宣某必然向殿下求情,保全将军性命颜面。”

段无敌先是有些气恼,但是见到宣松无比认真的神情,他神色变得和缓,道:“昔日段某曾经听闻,宣将军深慕忠义,在蜀中与狂生杨灿一面之缘,便倾囊赠金,使其妻儿得以安居,段某知道阁下一片好意,虽不能受,也当感激不尽。”

虽然被段无敌婉拒,但是宣松心中并无气恼,只是更添了几分惋惜,转身离去,宣松心中一片痛惜,自从和北汉军交战以来,便深为这些豪勇忠义之士而叹息,就是灭亡了北汉,真的能够得到这里的民心么,宣松第一次觉得攻打北汉,或许会陷入泥潭。

接下来的两日,李显竟然不再攻城,段无敌十分迷惑,但是他忙着安抚军中的暗流已经是焦头烂额,也顾不上深思了,第四日,雍军已经云集平遥,段无敌虽然拖延了雍军进攻晋阳的时间,可是自己却陷入了无法后退的僵局。站在城头,段无敌想着,不知道派去雍军的使者能够达成任务,虽然用人质胁迫不免有些难堪,但是若能救出麾下将士,倒也值得。他很清楚,宣松虽然在雍军中地位重要,可是毕竟不是主将,所以他的要求并不苛刻,只要求雍军不追击撤退的北汉军,平遥城将完好的交到雍军手中,他也承诺不烧毁城中粮草辎重。他相信这个要求有可能成功,因为对于雍军来说,自己这一支兵力无足轻重,而宣松素得军心,若是齐王不顾及宣松性命,只怕是雍军军心必然生怨,在付出不多的情况下了,他相信齐王不会作出这种亲者通,仇者快的蠢事。

接到段无敌的书信,李显哈哈大笑,这两日他停军不攻,为的就是这封书信,那日军议之后,他私下招了苏青过来,问明白散布流言的情况之后,他便明白了江哲的用心,之后又收到了江哲的书信,更是让他心如明镜。为了让流言更加逼真,他干脆不再进攻,这样一来,就可以放出段无敌见局势险峻,有心投降的谣言,众口铄金,李显相信段无敌支撑不了多久。而且就算没有其他好处,能够救回宣松也已经值得,想起当日中夜诀别,李显仍觉心中痛楚,所以他不仅立刻答应了段无敌的条件,还派出使者前去平遥。这个使者,正是苏青。

\chapter{第三十七章 忠贞见疑(中)}

望着满面风霜却越发清艳的苏青,段无敌只觉得心中一片平静,昔日爱恨如风消逝,他微笑道:“贵国殿下可是已经答应在下的要求?”

苏青心中涌起莫名的思绪,只是从这一句话,她就知道眼前这人已经将自己当成了陌路之人,这不是自己早就想到的么,昔日沁州城外恩断情绝,也就注定了今日。抬起头,她从容道:“殿下应允将军的要求,只要宣将军安然无恙,殿下答应,一日之内,不追击贵军。”

段无敌眼中闪过欣然的光芒,原本只是搏上一搏,想不到果然收效,他笑道:“不过贵军强大,而我军弱小,我不能不防殿下失言,不知道贵使有什么打算?”

苏青冷冷道:“齐王殿下一诺千金,岂有反悔的道理,不过将军不信,也是情理所在,若是将军愿意,可以先将宣将军送回雍营,苏青愿为人质。”

段无敌其实并无怀疑之意,不过是为了安抚军心罢了,所以便道:“既然如此,那就委屈贵使了。”

苏青微微一笑,就如寒梅绽放一般美艳,担任人质是她自请,段无敌若是聪明的,应该赶快逐走自己才是,只不过只怕直到今日,在这个男子心中,自己不过是走错了道路的迷途孤雁罢了,自己的危险尚未被他获悉吧?

当宣松走到雍军辕门,心中生出近乡情怯之感的时候,只听军中号角响起,辕门大开,李显带着众将大张旗鼓地出迎,宣松只觉眼中湿润,上前几步拜倒道:“罪将辱没军威,尚请殿下惩处。”

李显急步上前,伸手相搀,阻住宣松下拜,他满面歉疚,道:“宣将军何出此言,当日是李显不察,以致于此,当日若非宣将军慷慨赴死,本王曾经有言在先,若有差池,皆有本王担待,你幸而生还,本王若再加以怪罪,岂不是太苛刻了,你放心,今日之辱,你定可一一讨还。”

宣松感激涕零,半晌才平静下来,连忙道:“殿下,不可拘泥小义,段无敌乃是最擅长防守的将才,若是他回到晋阳守城,对于我军未免威胁太大,还请殿下奋起直追,擒杀段无敌。”

李显笑道:“早知道你会这样说,不过你不用担心了,段无敌断无可能回到晋阳的,再说苏将军还在他军中为质,现在也不适合进攻。”

宣松愕然道:“苏将军怎会去做人质,她虽然精明能干,但是毕竟是个女子,又和北汉结下深仇,恐怕就是段无敌恪守信义,也难免遇到危险。”

李显低声道:“你放心,自然有人接应苏将军,那段无敌毕竟是个君子,又有本王大军在此,苏青不会有事,只怕他还会后悔莫及呢。”想到得意之处,李显忍不住哈哈大笑。还有什么比胜券在握更加令人兴奋。

两人携手走进中军大帐,让宣松坐在左侧首席,众将一一入座,李显道:“宣将军,你历劫归来,本应该让你好好修养,可是如今军情紧急,段无敌擅长败退,步步为营,这也是你的长处,只好让你辛苦一趟了,等到明日此时,你率军衔尾追击,如何进退你便宜处置。”

宣松心中大喜,他不是没有担心过会暂时被搁置,想不到李显对自己如此信赖重用,连忙起身道:“末将遵命。”

李显见状不由微笑,其实现在并非一定需要宣松领军作战,他不过是通过这种方式表示他对宣松的器重,避免有人借着宣松被俘之事兴风作浪,不论是在什么地方,小人都是难免的。

北汉军从平遥撤退之后,几乎是全力行军,一日之间便已经到了阳邑,当安排好防务之后,段无敌走入亲兵为自己准备好的住处,一走进房间,他停住了脚步,只见外间坐着一人,苏青坐在椅上,玉手托腮,含笑看着自己。一旁的梨木衣架上面挂着青黑色的披风,室内几乎是一尘不染,而苏青面前的方桌上放着香气四溢的饭菜,一旁的椅子上还摆着铜盆方巾,盆内清水仍然冒着滚滚热气。

跟在段无敌身后的两个亲卫都是下意识地按住了刀柄,但是继而又露出迷茫的神色,显然这种温馨的场面让他们生出疑惑。就连段无敌也是一阵迷茫,若非是苏青身着劲装,腰间佩剑,明丽的笑容中带着些许讥诮和冰冷,他几乎要错认自己是回到了家中,而面前的男装丽人便是自己的妻子。他眼中恢复清明,冷冷道:“你为何会在这里,监视你的军士在哪里?”

苏青望望段无敌身后的亲卫,淡淡道:“你要在他们面前盘问我么?”

段无敌没有作声,挥手遣走侍卫,然后在桌子的另一边坐了下来,静静的看着苏青。苏青眼中闪过莫名的神色,她神色淡漠地道:“军中有些石将军旧部,他们还认得我,有些人寻机前来质问当日之事,我便告诉他们当日石将军并不知道我的身份,当日我不过是利用石将军在沁州城栖身,虽然做了些推波助澜的事情,不过却也料不到龙将军会深信石将军叛变,唉,石将军过于刚烈,若是当日他肯向龙将军辩白,未必没有机会洗清冤枉。”

段无敌只觉得口中发苦,道:“你所说可是实情?”

苏青回想起当日石英愤然自尽的情景,纵然是铁石心肠,也不由黯然神伤,她淡淡道:“自是实情,有些时候事实往往更能将人诱入歧途,不过你也不必后悔,石英虽然并未暗中投降大雍,但是他确实是存心针对于你,只因我告诉他了一些关于你的谎言。还有,当日石英自尽之时,已经猜到我的身份,但是他并没有告诉你们,而是甘心赴死。”

段无敌怒不可遏,右手猛然捶在桌面上,杯盘被震得砰砰作响,他怒视着苏青,但是怒火很快就平息下来,只因他看到苏青平静而又冷酷的神情。他松弛下来,微微苦笑,自己不是早已决定只将这个女子当成敌人的么,既然如此,又何必为她的所作所为生出怨恨呢。

觉得从未有过的疲倦,段无敌冷冷道:“好手段,昔日迫得石将军自尽,如今又用来污蔑我,苏姑娘,你够狠,只是你为何对我明言?”

苏青意味深长地道:“今日你与我在此密会,明日就会传得沸沸扬扬,用不了多少时间,就连晋阳都会知道你寻了个冠冕堂皇的借口放走了宣将军,而且还和昔日的未婚妻子密谈,你说晋阳会怎样想?”

段无敌默然不语,苏青站起身,拿起披风系好,道:“时间已至,你若是现在将我杀了,还可挽回这一切,若不然,我可能就有机会替你收尸了。不过你若是能够想通,齐王殿下等你弃暗投明。”

段无敌默然不语,虽然苏青陷害他至此,可是他却没有丝毫怨恨,彼此各为其主,不论做什么都是理所当然之事,只是苏青仍然给自己留了一条生路,这已经足以令他感激,只可惜,那条路却是他宁死也不愿去走的,在苏青即将走出房门的时候,他低声道:“多谢你,很抱歉。”

苏青娇躯一震,虽说在沁州城两人恩断情绝,但是这又岂是可以轻易办到的,不论是恨,还是爱,她心中仍然有着段无敌的影子。她今日来此,既是为了让段无敌更加有口难辩,也是希望段无敌能够答允投降,免去杀身之祸,但是她纵有此心,也没有指望这个男子能够明白,事实上,她已经做好了准备,从今之后,这个男子只会当自己是毒如蛇蝎之人,可是这个男子却将自己心意看的清清楚楚,却又明确得告诉自己不会接受。苏青不由心中酸楚,她低声道:“昔日你我两情相许,我从未后悔,纵然后来我被你伤得体无全肤,也仍然当你是铁骨铮铮的好男儿,只是既然你我已经分道扬镳,就再没有重聚的可能。不过,你当真要为北汉殉葬么?”

段无敌沉声道:“昔日之事,其咎在我,你的选择,我亦无话可说,你不需为我费心,求仁得仁,我死而无怨。只是我曾经听说你和凤仪门有些关联,原本还在担忧你再不能得到大雍接纳,到时天下虽大,无你容身之处,可是如今看来,齐王果然是非同常人,仍然重用于你,据闻雍帝器量仍在齐王之上,想来你不会因此受到牵连。只不过有一件事情我始终牵挂,你至今仍然小姑独处,或许是我自大,但是终究是我误你终身,若有可能,希望你能早结良缘,也可告慰你的双亲在天之灵。”

两行珠泪滚滚而下,苏青走出房门,她没有回答,也没有再回头,亲手陷害曾经的未婚夫婿,很有可能将他送上断头台,心中怎不痛楚,何况他纵然到了绝境,仍然没有一丝怨恨之心,又怎不让她愧疚。走出门外,苏青迅速拭去泪痕,取了坐骑扬长而去,骏马在风中疾驰,苏青心中只有一个念头,无敌,你若因此而死,我也只能用独身终老来向你赎罪了。

浑浑噩噩不知奔了多久,苏青突然听到马蹄声响,她立刻清醒过来,抬头一望,立时愣住,只见对面两匹马绝尘而来,马上两人她都认得,前面骑着一片黑马的正是秋玉飞,而后面骑着黄骠马的则是凌端。双方都不约而同地放慢了马速,然后停住坐骑,默默的望着对方。

苏青先醒悟过来,在马上一揖道:“原来是秋四公子,当初被公子一路追杀,现在末将还记得当日的苦楚呢,听闻公子出使东海,想不到今日归来,此去莫非是要去阳邑么,段无敌段将军就在阳邑,再过一两日,只怕我雍军主力就会到此了,公子虽然武功出众,但是毕竟只是一人,为了公子着想,还是请公子速速返回晋阳吧。”

秋玉飞微微一笑,眼中闪过一丝倾慕混合杀机的复杂情绪,对于这个女子,他是深深佩服的,弱质孤女,只身蹈虎穴,立下赫赫奇功,当日自己一路追杀,只有这个女子可以和自己一战,武功高,心机深,智慧高,再加上精通音律,相貌清艳,怎不令须眉汗颜,只可惜却偏偏和北汉仇恨似海,不惜舍弃家国爱侣,为敌国效命征战。是否杀了她以毁去齐王得力的臂膀呢?只是现在三人都身在旷野,那女子的战马也是千里挑一的良驹,若是一心逃走,自己也未必能够得手。

正在秋玉飞犹豫是否出手的时候,身后烟尘滚滚,当先一骑是一个青衣少年,容颜如雪,正是邪影李顺,秋玉飞微微一叹,对苏青还礼一揖道:“陌路相逢,只是没有时间叙谈,姑娘的琵琶绝艺,玉飞仰慕非常,他日若有机缘,还当请教。”说罢策马急急而去。

苏青只觉得背心冷汗涔涔,直到秋玉飞远走,她才觉得方才笼罩在身上的沉重压力消失不见,这时小顺子已经到了近前,他淡淡道:“公子书信到了,调在下和苏将军前去听命,公子说,是要我们准备接待一位佳客。”苏青眼中闪过疑惑的神色,是什么佳客要楚乡侯亲自迎接呢?一个念头突然如同星火一般在她心头闪现,她的容颜突然变得苍白,很多事情都可以想通了,例如为什么秋玉飞会出现在这里,想得越清楚,苏青对江哲此人的心机越发觉得心寒,如今想起来,自己昔日擅自决定,改变了他的计策之事,未免是有些过于冒失了。

夜色深沉,段无敌望着手中绘制完毕的晋阳防务图,心满意足地放下了笔,这两日谣言四起,就连他的大部分旧部也对他生出疑心,若非是他用强硬手段压制,只怕这些士卒早就哗变了,虽然也有亲信的将领和亲卫仍然相信自己,可是他们除了徒劳地替自己辨白之外再也无能为力,而且大概只需晋阳一道旨意,自己就将孤立无援了吧,毕竟自己从未刻意笼络过下属,众叛亲离并非只有暴虐的首领才会遭遇到的窘况。送走苏青的当日夜里,晋阳有紧急军令到来,命自己固守阳邑,段无敌心知这是晋阳也对自己生出疑心,事已至此,他也无意辨白,所有的谣言可以说九成都是实情,只是增加了一些子虚乌有的细节,可就是如此才让他百口莫辩。想来晋阳应该有所决定了吧,他心中泛起淡淡的苦涩。

这时,有人在外冷冷道:“段将军,你为何还在这里?”

段无敌愕然抬首,一人推门而入,段无敌化惊为喜,上前施礼道:“原来是四公子,东海一行想必多有艰险,公子能够平安归来,国师必然大喜过望。”

秋玉飞望着段无敌黯然道:“我进城之时已经得知如今情形,你的处境未免太艰难了,纵然是我,若非昔日和你有相交之情,也会怀疑你的忠诚,而且说句实话,就算是你从前忠心耿耿,如今这样地剪迫,只怕你也难以继续忠于北汉,所以我虽然传书师尊,希望他为你缓颊,但是恐怕没有什么用处,唯今之计,你不若走了吧,就是去投了大雍,只要你不替他们来攻打晋阳,我也不会怪你。”

段无敌微微一笑,道:“公子何出此言,段某问心无愧,焉能畏罪潜逃,公子信任段某忠诚,段某感激不尽,若是我真的逃走,只怕是弄假成真,龙将军殉国之后,只有嘉平公主独力擎天,她待我不薄,我不能辜负她的信任。”

突然,外面传来自己亲卫惊怒交加的呵斥声,这些亲卫都是跟着段无敌出生入死的亲信,自然知道自己的将军受了何等的冤屈,只是他们纵然辩白也无人愿意相信,如今他们突然这样混乱,必然是晋阳前来查办自己的使者到了,段无敌微微一笑,道:“想必是晋阳使者到了,公子在此或有不便,若是不嫌弃,请到内室暂避,不必以段某为念。”秋玉飞一声长叹,身形隐入内室,通往内室的房门无声关闭。段无敌站起身走到书案之前,静候使者进来。

不多时,房门被人推开,段无敌一眼便看到了神色憔悴的林碧,竟然是嘉平公主亲至,这是怎么回事,林碧如今应该在总领晋阳防务,段无敌不由神色数变。林碧走到书案后面径自坐下,看向案上墨汁淋漓的布防图,神色一黯,道:“段将军仍然为晋阳防务忧心么?”

段无敌肃手站在案前,道:“末将曾在晋阳卫戍,晋阳防卫本是固若金汤,不过天长日久,难免有些缺失,末将曾经仔细研究过如何补救,只可惜不得兵部接纳,这几日末将凭着记忆重新绘制了一张布防图,其中有些地方是防务上的薄弱之处,若是能够按照这张图加强守卫,或者会好些,还请公主过目,若是公主觉得可行,不妨一试。”

林碧望向段无敌神色坦荡的面容,道:“你可知王上下了严令,将你立刻明正典刑,我多次苦苦相劝,王上仍然固执己见。国师之意,也说你纵然本无二心,如今也不能保证你不会投敌,因此支持王上的决定。”

段无敌平静地道:“末将早已料到如此,敌人的计谋虽然简单,却是狠辣非常,段某也有错处,不论是为什么,末将昔日走私贪贿都是罪证确凿,而且石英将军若果真冤枉而死,末将也是罪魁祸首,再说为了性命放纵俘虏,为了私情放走苏青,这都是真的,段某知道自己罪不容诛,王上只令斩首,已经是法外施恩,公主不必介怀。”

林碧面上露出痛惜的神情,道:“庭飞当日曾对我说过你的事情,你不计毁誉,为了北汉做了许多事情,这种种罪状却都是冤屈了你,用宣松交换你和将士们的性命,这是我默许的,放走苏青,也是理所当然之事,难道我北汉还能杀害使者么?只是朝中群起攻讦,我多替你声辩几句,便险些被国主逐出大殿。唉,昔日朝中重武轻文,如今那些文官个个言辞激烈,好像若不杀你,社稷必亡,朝中勋贵武将虽多,但是庭飞昔日喜欢提拔寒门出身的将领,唯才是举,令他们颇有微词,如今庭飞殉国,他们便也趁机攻讦于你,哼,大敌当前,他们不想着如何对敌,还在排除异己,好像若有他们带兵,就可以挽回危局一般,不知自量。段将军,林碧无能,不能保住你了,只能争取亲来阳邑处置你,这样也可保全你的体面。”

段无敌下拜道:“多谢公主殿下相信末将忠心,事已至此,公主不要为了末将生死和朝廷决裂,若是没有公主担任主将,只恐晋阳难守,末将纵死也不会怨恨王上和公主,就请公主下令将末将阵前斩首吧,若能够保住社稷黎庶,末将就是遗臭万年也无怨恨。”

林碧掩面道:“忠贞见疑,朝廷对你不起,你,你去吧。”

段无敌再拜叩首,然后举步向门外走去,他刚走到门口,门外的林碧亲卫要上前将他缚住的时候,林碧突然高声道:“且慢。”

众人都是一愣,向林碧望去,只见林碧神色坚毅非常,她断然道:“段将军,有我林碧在此,断不能让你无辜遇害,你立刻离开北汉吧,现在国内一片混乱,很多地方我军已经撤退,而雍军尚未进驻,你有很大的机会逃出去。去滨州吧,现在那里名义上还不是大雍所属,而且现在大雍也顾不上缉拿你,从滨州转道南楚,这是你唯一的生路,将来若能逐走雍人,你还有机会重回北汉的。”

段无敌听到这里,竟然呆住了,他万万没有想到林碧竟有如此担当,人若有一线生机,又怎能不牢牢把握,方才秋玉飞劝他,他不想林碧疑他,因此不肯离去,如今林碧劝他,他心结既解,越想越是觉得可行,若能留得有用之身,还有为国效力之日,若是一死了之,不过是亲痛仇快,而且现在除了林碧,也无人可以支撑危局,林碧只需说自己先行逃走,想来国主也不会怪罪林碧。

林碧见他情状,不由一阵辛酸,但是想到此人忠心为国,不计毁誉的壮举,仍然令她决定承担放走“叛逆”的责难,她上前道:“段将军,此地不可久留,国主或许会再派使者,到时候你就不可能脱身了,我知你一向廉洁,家无余财,这些金珠你带着路上使用。”说着将一个钱袋塞到段无敌手中,这个钱袋里面是些轻巧的金珠,价值不菲而便于携带,临行之前,林碧鬼使神差地带在身上,或许当时她就有了这种想法吧,只是在方才她才终于下定决心。

段无敌接过钱袋,忍不住热泪盈眶,他也知道林碧担了天大干系,更是知道这已经是自己唯一一条活路,虽然前途茫茫,说不定会落入雍军之手,或者被北汉军当成叛贼杀死,但是他仍然是感激涕零,双膝跪地,段无敌泣道:“公主恩义,末将永志不忘,若是日后无敌侥幸逃生,必然传信回来,公主但有所命,无敌无不遵从,殿下宽心,若是无敌不幸落入敌手,绝不会苟且偷生。”

林碧珠泪欲落,她心中是有些顾忌,若是段无敌落入敌手,恐怕终会归顺雍军,所以来时也是宁愿屈杀了段无敌,见段无敌如此许诺,她心中一宽之余,也不由有些愧疚。林碧背过身去,轻轻挥手,示意段无敌离去,段无敌顿首再拜,终于转身离去,此一去或者再无相见之期,怎不令豪杰扼腕。

段无敌的身影消失之后,一直在内室听着外面动静的秋玉飞面上露出欣慰的微笑,方才林碧要将段无敌推下斩首,他已经下定决心要去劫法场了,如今见到林碧放走段无敌,他才心中一宽,本想出去和林碧相见,但是突然,他心中一动,城外有一个他熟悉的人的气息陡现,杀机隐伏,冷冷一笑,他的身影化成虚幻,从内室的窗子跃出,趁着城中的混乱,向段无敌离去的方向追去。

阳邑城外,站在山冈之上的萧桐望见段无敌策马出城,不由一顿足,师尊得知林碧亲来阳邑之后,思索再三,令他赶来此地追杀可能会被林碧放走的段无敌,如今果不其然,他正要策马追赶,突然耳边传来清冷的声音道:“师兄,你当真要赶尽杀绝?”

萧桐愕然,抬头望去,却见秋玉飞负手而立,他苦笑道:“师弟,这是师尊的谕令,不论段将军是否冤枉,他若落入敌手,都是很大的威胁,你不能心慈手软。”

秋玉飞冷冷道:“段将军对北汉忠心耿耿,虽然如今谣言满天,但是我相信终有水落石出的一日,我和碧公主一样,都不相信段将军有了二心。就是师尊亲来,我也不会任由师尊动手。”萧桐只得苦笑,他知道若论武功,自己不是这个师弟的对手,看来追杀段无敌已经是不可能之事了,只得道:“你既然已经回来了,就去晋阳见见师尊吧。”秋玉飞淡淡道:“好,我们一起上路吧。”萧桐忙道:“我还有军务在身。”秋玉飞冷眼看去,萧桐连忙解释道:“你放心,我对魔尊立誓,若是我去追杀段将军,就让我死后沦陷在魔尊血狱,永世不得超生。实在是军情紧急,我尚有要事在身。”秋玉飞默然不语,既然萧桐立下天魔血誓,就必然不会违背。他转身离去,倏忽不见,萧桐仰头苦笑不已,自己这个师弟数月不见,修为更是突飞猛进,真让自己这个师兄汗颜。罢了,既然碧公主和玉飞都对段无敌如此信任,想来段无敌当真是忠义无双,自己何必去做小人呢?

\chapter{第三十八章 忠贞见疑(下)}

在渺无人烟的官道旁边,一片郁郁葱葱的小树林之后,清澈见底的小河流蜿蜒而出,这片小树林十分稀疏,一条可容一辆马车行走的道路深入林中,林外挂着酒幌,一眼就可以看到林中隐隐有四五间宽阔的茅屋,门上也插着酒旗,这里想必是旅客中午打尖的好去处。虽然是战乱时节,可是林中酒香隐隐,看来生意没有停业,不过说来也并不奇怪,这里并不是雍军进军的主要方向,所以很多人的生活仍然是一如往常,只不过多了些许忐忑不安罢了。平民百姓就是这般,只要不是刀斧临头,就得照常营生,否则这一年生计可如何支撑。

段无敌已经换上了行路旅人的便装,外面罩了披风,头上戴着顶信阳斗笠,这种斗笠乃是行道中人常备之物,遮风避雨,颇为方便,四面有垂纱的可以遮掩面貌,北汉境内春秋风大,就是男子也很喜欢用来遮挡风尘。他一路疾驰,顾不得爱惜马力,这一带虽然雍军尚未驻兵,但是有不少斥候常常往来,他也只能尽量避开罢了,此刻他心中不免凄惶,埋头赶路,尽量让自己无心去感叹前路茫茫。看看天色,已经快到午时,他觉得有些困乏,座下战马身上也是汗水涔涔。他不由向远处张望,一眼看见路边的酒旗,他心中一动,自己匆匆而出,干粮也没有准备,不如进去休息一下,顺便购些干粮,装些村酒,以备路上食用,错过这里,前面恐怕很难寻到打尖的所在了。想到这里,他策马走入树林,不多时走到野店门前,只见店门大开,里面几张方桌十分洁净,里面已经有了几个客人,坐在最右侧的桌子旁边,一个四十多岁的中年店主正在笑呵呵地端酒上菜。见到那种闲适的气氛,段无敌心中一宽,将马系在店前的树上,走入店堂,高声道:“来些好酒好菜,待会儿我还要赶路。”说罢,拣了最左面的桌子坐了,随手在桌上丢了一块碎银。

那店主连忙上前抹桌子,左手灵巧地将银子笼入袖中,倒上热茶,热情地道:“客爷一路辛苦,小店虽然偏远,可是山珍野味还是有的,还有上好的陈年老酒,客爷稍待。”说罢对着里面喊道:“小三,快端上好酒好菜。”随着他的喊声,一个满面憨直的青年端着酒菜从里间走了出来,这个青年二十多岁,虎背熊腰,只是神色呆傻,显然是智力不足,他傻呵呵地将一盘花生米和一盘猪头肉放到桌上,又从店房一角的大酒缸里装了一壶老酒放到段无敌面前,然后就回到里间去了,接着便听见锅铲作响,不多时,几个野味小菜端了上来,一桌子荤素俱全,香气扑鼻。

段无敌只觉得饥肠辘辘,但他警惕仍在,有意无意地向对面看去,只见对面共有四人,上首坐着一个商贾装束的中年人,似是主人,左右两人都是保镖装束,相貌豪勇,还有一个青衣人背对着自己,虽然看不到相貌,但是发色浅灰,想必是年纪不轻,但见他背影并无苍老之态,想来应是五十许人,他只用一根玉簪束发,除此之外再无修饰,身穿青衫,想必是帐房先生一流的人物。略一打量,这些人看上去都不似军旅中人,确定这些人应该不是追兵,段无敌松了一口气,开始埋头狼吞虎咽起来。

匆匆离开阳邑,他已经大半天没有进餐,饥饿交加,吃相也自然难看起来,吃个七八分饱之后,他开始松弛下来,这店中的老酒虽然是乡村野酿,却是甘冽辛辣,意犹未尽,他又想倒一杯,谁知已经涓滴不剩,他皱了一下眉,忍不住又要了一壶,他平日很少饮酒,非是酒量不好,而是不愿贻误军机,如今落到这步田地,自然也少了几分拘束,他连饮数杯,只觉得身上轻松了许多,困乏渐渐消去。酒之一物最能令人意乱神迷,人一松懈下来,不由开始胡思乱想,想到自己忠心耿耿,却落得一个叛逆的罪名,被迫仓皇出走,忍不住悲从心来,酒入愁肠,神色间更是多了几分悲愤和落寞。浑不知自己情态俱落在对面数人的眼中,那青衣人虽然背对着段无敌,但是一把特制的小铜壶将段无敌的身影映射其中,那人看在眼中,面上闪过悲怜之色。

多饮了些酒,段无敌只觉头重脚轻,酒意上头,忍不住高声吟道:“帝高阳之苗裔兮,朕皇考曰伯庸。摄提贞于孟陬兮,惟庚寅吾以降。……”这首屈子名篇乃是他生平最爱之作,他虽然不甚通经史,但是对这首《离骚》却是爱不释手,倒背如流,他声音因为多日心中熬煎,不免嘶哑低沉,但是吟来情真意切,令人感叹不已,吟道“亦余心之所善兮,虽九死其犹未悔”一句之时,他反复吟咏,却是再也吟不下去,拭去泪痕,再次举杯一饮而尽。

就在这时,只听有人接着这一句开始吟诵起来,那人声如金玉,意韵悠长,段无敌听得入神,住杯不饮,那人吟到“屈心而抑志兮,忍尤而攘诟。伏清白以死直兮,固前圣之所厚。”一句,段无敌心中越发痛楚,直到那人吟道最后一句“乱曰:已矣哉!国无人莫我知兮,又何怀乎故都!既莫足与为美政兮,吾将从彭咸之所居!”的时候,段无敌才突然清醒过来,乡村野店,商贾中人怎会有人吟诵屈子诗篇,他抬目望去,只见对面仍然是那几个客人,其他三人都在默默饮酒,想必吟诵之人是那个背对自己之人。

或许是感觉到他的目光,那个灰发人转身过来,笑道:“在下见将军痛心疾首,不能吟完整篇,一时见猎兴起,替阁下吟诵完全,想必是打扰了将军饮酒,还请恕罪。”

段无敌心中一跳,这人如何知道自己身份,他仔细瞧去,只见这个灰发人两鬓星霜,但是相貌却是儒雅俊秀,丰姿如玉,仍然是青年模样,而且气度闲适,令人一见便生出敬慕之心。这人的身影自己竟然有熟悉之感,心中灵光一现,段无敌只觉得口中苦涩非常,将杯中烈酒一饮而尽,他平静地道:“段某何幸,竟然劳楚乡侯亲至。”

我对段无敌识破我的身份并不觉得奇怪,毕竟我这种少年白发的形貌也太容易辨认,扮作商贾和两个保镖都是这次随军的白道高手,他们身上没有军旅中人的气息,这才瞒过了段无敌的耳目,如今见我身份泄露,立刻站起身护在我身边,而里间的门帘一挑,李顺缓步走出,在他身后,扮作店主和伙计小三的两个密谍也恢复了彪悍的神情,店门处更是多了两个身影,正是苏青和呼延寿,店外隐隐传来压抑的呼吸声和兵器出鞘的声音,显然这一座野店已经成了天罗地网,而段无敌正是网中鸟雀,再无逃生之路。

段无敌心中也明白如今的局势,事到临头,他反而沉静如山,只是缓缓替自己又倒了一杯酒,举杯相邀道:“自从侯爷东海复出以来,我军屡次遭遇挫折,谭将军、龙将军先后殉国,石将军被迫自绝,段某落得一个叛国罪名,却又落入侯爷陷阱,侯爷智谋果然是惊天动地。只是侯爷乃是千金之躯,为何孤身涉险,若想取段某性命,只需一队骑兵,或者几个侍卫即可,何必亲临险地。”最后一句话隐含讥讽,但是他的神色却是十分冷静,似乎并未身处陷阱。

我心中没有丝毫得意,反而有些隐隐的挫败。我重重布置都是为了逼这个男子出走,从他离开阳邑的一刻,至少有数百人监视他的行踪,算定了此处必然是他打尖之所,将这里控制起来等他自投罗网,原本是希望给他一个下马威,挫折他的心志。可是这个男子纵然是落入我掌中,仍然是这样平静淡漠,仿佛早已料到这一幕似的,这样心志坚定之人,我可以摧毁他的生命荣耀,却不能摧毁他的意志,心中隐隐有了失败的预感,我只能暗暗叹气,准备不计成败的试上一试。

微微苦笑一下,我道:“江某虽然设计陷害将军,却是因为我料嘉平公主必然不会残害忠良,不过公主也不能和北汉上下这许多人相抗,只能让将军远走高飞,将军想要逃脱,只有往东海一行,东海虽然迟早归附大雍,但是毕竟是一条生路,以姜侯的为人,就是知道将军的行踪被他察知,也会装作不知道。所以江某特意在此恭候将军,这般用心拳拳,将军纵不领情,也不应如此冷淡,岂不是辜负在下的诚意。”

段无敌心中电转,早已想通许多问题,道:“秋四公子原本陷身东海,这一次却平安归来,是不是侯爷早料到四公子会来保护段某性命?”

我心中暗赞,这人一针见血,说破我的心思,道:“不错,从前我将玉飞软禁在东海,只因他已是先天高手,我不想他参与此战,不过如今大局已定,我尚有用他之处,所以特意将他请回,不过还有一个目的就是为了将军,否则至少他还要在东海呆上半个月。玉飞性情中人,昔日石英之事,他也身涉其中,我以此事冤枉将军,别人纵然不相信将军忠义,玉飞断然不会怀疑将军叛国,他身份超然,又是独立特行,就是嘉平公主不得不要加害将军,他也会出手救助将军。玉飞虽然行踪缥缈,难以追踪,可是毕竟沁州一地可以说已经尽在我军之手,冀氏拜祭龙将军,平遥窥视齐王大营,赶赴阳邑救助将军,我都心中有数。段将军恐怕不知道,萧桐奉命前来,以防嘉平公主放你逃生,他本欲追杀于你,就是玉飞拦住了他。”

段无敌目中闪过感激之色,道:“秋四公子救命之恩,段某感激不尽,只是恐怕没有机会当面谢过,侯爷若是再见他之时,请代段某致谢。”

我皱皱眉,刻意忽略他隐隐透漏出来的死志,道:“北汉诸多将领,江某最仰慕将军的为人,将军忠心耿耿,且不计毁誉,不计荣辱,将军之才,尤在龙将军和嘉平公主之上,只是可惜出身寒门,无人依傍,才没有机会担任主将。若是将军肯投效大雍,皇上和齐王殿下必然欣喜若狂,宣将军虽曾受辱于将军手中,可是对将军也是十分赞誉,若是将军肯归顺大雍,必然不失封侯之位。若是无意画影凌烟,将军素来爱惜百姓,若肯为大雍效力,必然可以周全北汉将士平民,只是不知道将军可肯为北汉民众继续牺牲自己的声誉么?”

段无敌微微一笑,举杯一饮而尽,只觉得如同烈火入喉,他按住腰间佩剑,道:“不论阁下如何花言巧语,也不能动摇段某心志,背叛就是背叛,段某乃是北汉臣子,不稀罕大雍君王赏赐的富贵。至于说到周全北汉百姓,这不过是个借口,这世上少了段某并没什么要紧,若是北汉当真亡国,大雍天子肯善待我北汉百姓自然最好,若是不能,自有义士揭竿而起,段某虽然不爱惜自己声誉,可是却断然没有投敌的可能。侯爷也说段某身上污名多半是侯爷所赐,既然不是真的,难道段某还会破罐破摔,真的屈膝投降么?侯爷今日高高在上,不知道午夜梦回,想起南楚是何种感觉。”

我微微苦笑,段无敌心志坚定,我本以为在有国难奔,有家难归,且自身陷入困境的情况下,此人心意或者会有所动摇,不料他竟然如此执拗。或者是见我被段无敌顶撞地无话可说,李顺冷冷道:“我家公子好生劝你,你如何这般无礼,岂不知你身陷死地,只需公子一道令谕,就是惨死之局,事后我家公子再宣扬出去,说你已经投降大雍,你纵死也是身败名裂,就算你赤胆忠心又有何人知晓,只怕就连嘉平公主和秋四公子也当你真的叛国。”

段无敌淡淡一笑,手按剑柄道:“不需侯爷下令,段某自绝可也,至于身外荣辱,段某本就不放在心上,纵然千夫所指,只要段某问心无愧,又有什么要紧,再说有些事情纸包不住火,终有水落石出的一天。”

李顺眼中闪过凌厉的杀机,冷冷道:“在我面前你要寻死也未必可以做到。”说着踏前一步,双目紧紧盯着段无敌。段无敌面色一寒,按剑的右手作势拔剑,就在众人目光集中在他的右手的时候,他左手闪电般从腿侧拔出一柄匕首,向小腹刺去。就在他拔出匕首的瞬间,苏青手中一枚双锋针将欲射出,但是她心中闪过一个念头,与其让他受尽屈辱,不若让他死了吧,她垂下眼帘,没有发出原本想要射伤段无敌手腕的一针。可是当她耳中传来痛苦的呻吟声之时,惊讶地抬头,却看见李顺左手捏住段无敌咽喉处,匕首已经到了李顺右手。苏青心中一紧,目光流转之处,却看到一双温润的眼睛饶有兴趣地看着自己,心中一震,双锋针坠落尘埃。

收回目光,将方才那有趣的一幕藏在心底,我挥手让李顺退下,温和地道:“段将军,属下无礼,请勿见怪。”

段无敌颓然软倒,酒意和方才呼吸中断让他头晕目眩,任凭李顺解去他腰间长剑,然后一杯烈酒灌入他的口中,他再次清醒过来,微微苦笑,抬头看去,却见那俊雅青年站在自己面前,手中拿着一块丝巾,而在他身后一双冰寒的眼睛冷冷看着自己,段无敌只觉得心头发寒,就如同被毒蛇盯上的青蛙一般,不敢擅动。他心知自己稍有不妥举动,便当真会陷入生死不能的窘境,接过丝巾,拭去面上污痕,他心中清明,想要摆脱这种景况,只有一个方法。

望向江哲,段无敌沉声道:“我曾和秋四公子促膝详谈,对侯爷为人略知一二。世人虽道侯爷狠毒,我却认为侯爷乃是性情中人,南楚德亲王待侯爷凉薄,但是侯爷却始终没有恶语相加,侯爷为了大雍天下鞠躬尽瘁,死而后已,这种种情事,天下皆知。想来侯爷昔日面对凤仪门主之时,也有不计生死毁誉的勇气。段某不才,纵然是求生不得,求死不能,也有面对的勇气,绝不会屈膝投靠,只是侯爷既然对段某颇有爱惜之处,又何忍迫段某如此,若能成全段某忠义,段某九泉之下也当感激不尽。”

我微微一叹,望进段无敌双目,只觉他目光坚忍,毫无惧意,我心中越发苦涩,知道这一次当真是徒劳无功了。这时苏青上前一步,语气有些凄楚,道:“侯爷,末将请您成全了他吧。”此言一出,段无敌忍不住望向苏青,目中满是感激之色,苏青心中越发伤痛,侧过头去,不愿见此情状。

我轻轻摇头,退后几步,转过身去,李顺心中了然,将长剑递还,也退后几步。苏青心中一痛,知道此意乃是让段无敌自绝,不忍旁观,她轻轻后退一步,侧过脸去。呼延寿见到,轻轻平移半步,遮住苏青大半身形,他心中忐忑,方才苏青履有不当之举,他担心若是段无敌自绝之时,苏青若有什么强烈反应,会遭到江哲猜忌,所以才将她身形挡住。

段无敌心中半是欢喜半是伤悲,起身一揖道:“多谢侯爷恩典。”目光在呼延寿和苏青身上掠过,他本是心思细密之人,一眼便看出其中蹊跷,微微一笑,他面向晋阳方向拜倒,凄然道:“无敌生不能卫护社稷,死后唯愿魂归故里,护佑乡梓。”言罢举剑就喉。

我不知怎地,心中一热,断喝道:“且慢。”李顺早有准备,弹指发出劲气,段无敌只觉手一麻,长剑坠地,他心中一惊,愤然道:“莫非侯爷想要出尔反尔,戏弄段某不成。”此刻他真是愤怒至极,腾的站起,虽然立刻被人拦住去路,避免他暴起发难,但是他怒火汹汹,双目都几乎变成血红。

我微微一笑,道:“将军放心,我绝不会改变主意,只是想给将军另外一个选择,若是将军不愿,就请自行了断,江某绝不拦阻。”

段无敌望望李顺等人,知道自己就是想不听都不成,只得怒道:“侯爷有话请讲”。

我一字一句道:“我欲放将军离去,不知将军意下如何?”

段无敌心中巨震,但是他很快就晒笑道:“侯爷想是说笑,段某不才,若是今日处在侯爷的位置,也绝不可能放走笼中之鸟。”

我走到桌前坐了下来,挥手示意除了李顺之外众人都退去,然后请段无敌坐在对面,段无敌略为犹豫,便走了过来,他早已将一切置之度外,索性放纵起来。

我笑道:“江某不必讳言,昔日背离南楚,投靠雍王殿下,乃是失节之举,如今又娶了宁国长乐公主,臣娶君妻,更是大大的不忠不义,后世必然对我有微词,就是遗臭万年也有可能,但是身外浮名我毫不在意,只因当日的选择是我心甘情愿,并无半分勉强。”

段无敌见江哲突然说出这番话来,只能默默听着。

我想起往事,面上露出怀念的神色,道:“其实江某虽然当初也不是没有忠义名节的顾忌,段将军应该知道当初江某是被我大雍当今皇上俘虏到了雍都的。”

段无敌点头道:“末将知道,侯爷当日已是布衣,其时雍王殿下亲自相请,侯爷不肯效命,方为雍王殿下虏去雍都,据说殿下对侯爷解衣推食,敬爱备至,才终于感动了侯爷,改节相事。”说到最后一句,讽刺的意味已经极浓。

我却毫不在意,淡淡道:“其实那些所谓的礼贤下士的举动如何能够动摇我的心志,天下的君主谁不是这样,创业之时,将臣子当成骨肉至亲般看待,一旦事过境迁,便是鸟尽弓藏,兔死狗烹,有些昏庸的君主,甚至大事未成就先斩羽翼。当日江某虽然有些俗事牵挂,可是却也用不着替人效命,所以我下定决心,不肯效命雍王,甚至百般刁难,逼得雍王殿下不得不放手。殿下雄才大略,自然不肯轻轻将我放走,不得已下了决心赐我一死。”

听到此处,段无敌深吸一口冷气,得悉这样的隐秘,他也不由生出兴趣,问道:“那么侯爷又怎会投效了雍王殿下。”

我傲然道:“江某当日自然有保命的妙策,世间霸主,对人才多半是顺我者昌,逆我者亡,我迫使雍王赐以毒酒,就是想假死逃生,到时候天地任我逍遥,待我凡尘事了,若还留得命在,便寻一个清净所在,了此残生,此乃人生快事。”

说到此处,我不由露出感慨神色,继续道:“不料我江哲自信可以料尽世人心事,却终于输给了雍王殿下,殿下竟然千钧一发之际,倾去毒酒,金盔盛酒壮我行色,江某不才,也知道世人少有能与我抗衡者,殿下却能轻轻放过,如此仁爱之主,我焉能为了小节辜负大义,所以我终于称臣于殿下,从此君臣相得,如鱼得水,以至于今。”

段无敌眼中闪过一丝倾慕,但他很快就道:“大雍天子虽然仁爱,但是毕竟非我北汉之主,若是侯爷以为如此可以说服段某投降,请恕段某不识抬举。”

我摇手笑道:“非是如此,将军心志之坚,当时无双,我知道将军断然不肯负了北汉社稷百姓,我也知道将军请自绝,是因为不相信我会放将军离去。”

段无敌默然不语,这本是显而易见的事实。

我淡淡道:“的确,将军乃是名将之才,对北汉又是忠心耿耿,若说我肯放过将军,实在是无人肯信,可是江某方才想起昔日之事,皇上当日爱才惜才,饶我性命,也是断无可能之事,我深慕将军为人,今日放过将军,又有什么不可以的,所以只要将军答应我一件事情,我就放将军离去。”

段无敌目中露出怀疑和期望混杂的神色,却仍是默然不语。

我再次肯定道:“江某此心天日可表,将军只需答应我一事,我就放将军离去。”

段无敌犹豫了一下,问道:“请侯爷吩咐,不过有些事情段某是不会答应的。”

我心中明白,道:“你放心,我必然不为难你,我知道你此去是想从滨州转道南楚,你若是答应不去南楚,我就放你离去。”

段无敌皱眉道:“东海迟早将属大雍,段某怎可留在敌国境内。”

听他这样说,我知他已经动心,又道:“虽然如此,可是除了南楚还有许多可去之处,近些年,常有中原人士随船出海,或至高丽,或至南洋诸国,不一而足,将军若是肯离开中原,自然不能再和大雍为敌,我就是纵放了你,皇上和齐王殿下那里也说的过去,不知道将军意下如何?”

段无敌沉默半晌,若是北汉亡国,就是到了南楚又能如何,若是北汉不亡,自己纵在海外,又有什么紧要,想到这里,他点头道:“末将答应这个条件就是。”

我微微一笑,道:“既然如此,将军就请自行去滨州,寻海氏船行的少东主海骊,他自会安排将军离开中原。”

段无敌疑惑地问道:“侯爷用计,往往不留一丝余地,为何今日竟然宽纵在下,难道只是为了我令侯爷想起昔日之事么?”

我站起身,小顺子替我系上一件青色披风,走到门口,我停住脚步,淡淡道:“我素来用计,都是利用了别人的短处,只有今次,却是利用了将军的忠义和仁爱之心,或许是这个缘故,才会对将军十分歉疚,今后你远离中土,漂流无依,这种生活比起死亡也不过是略胜一线罢了,这也算不上宽纵。只是将军需记得,若是你妄想利用我的好意,江某的报复也将令将军后悔莫及,苏将军虽然与你断恩,但是她今日替你求情,仍有昔日情谊,你若不想连累了她,就在海外待上几年吧,到时候北汉已经消亡,你若愿意回来,也无妨碍。”

段无敌呆立店堂之中,耳畔传来远去的马蹄声,他心中五味杂陈,缓缓捡起长剑还鞘,那黑暗中的一线光明,是否另一番天地呢?

坐在马上,我眼角余光掠过,苏青一路低头不语,想来她和段无敌仍有情义,只是两人中间隔着国仇私恨,只怕是鸳梦难温。微微一笑,我望向北方,这几日,皇上已经连下四道密诏,让我去忻州见驾。如今大军即将合围,只需代州事了,就可开展晋阳攻势,泽州大营这边将帅已经和睦非常,再无内忧,我的职责已了。数年不见,也难怪皇上心急,召我去见,抗旨之事,再一再二,不可再三再四,我还是应该快快启程才好。抬头看天,只觉风清云淡,令我心旷神怡,只是不知赤骥那傻小子现在还活着么?

\chapter{第三十九章 狭路相逢}

古道漫漫,旌旗如火,一支衣甲鲜明的铁骑护着一辆马车在官道上行进,道路两旁黍麦离离,却是渺无人烟,非是这一带的百姓皆已逃走,事实上,雍帝李贽闪电奇袭,这里的百姓根本没有逃走的机会,现在无人只是因为在一个时辰之前,已经有人奉命将这里道路扫清,以免发生任何意外。

我坐在马车当中,两侧帘幕挑起,沐浴在北地和煦的春光之中,在五千铁骑的保护下,我跟本不担心会有人来行刺,反而饱览沿途风光,悠闲如同春日出游。在我启程北上之时,李显和长孙冀已经合兵一处,大举向晋阳推进,现在北汉根本没有办法派出一支千人以上的军队越过雍军的重重封锁,只需代州事了,大军合围,就可以开始最后的攻势。更何况东川事了,大雍可以全力对付北汉,强弱悬殊,胜算可期,想到此处,就是我也不免有些志得意满。

这时,耳边传来轻叹之声,我回头一瞧,李顺面上露出淡淡的愁容,不由瞪大了眼睛,这家伙就是和凤仪门主交手,也没有露出发愁的神色,今日却是怎么了,似乎是看出了我的疑惑,李顺忧虑地道:“公子,从前两军胜负未分,魔宗宗主自然不会轻易出手,如今大局已定,京无极岂会再袖手旁观,慈真大师在皇上身边护驾,齐王殿下身边也有少林高手保护,而公子身边却只有我一人,就连张锦雄他们公子也没有带在身边,而魔宗弟子如段凌霄、秋玉飞者也都是先天高手,若是他们一起出手,别说公子身边只有五千铁骑,就是再多上一些,也难免会被他们近身攻击,其实公子就是再抗旨几次又有什么关系,总好过这样涉险。”

我不以为意地道:“你过虑了,魔宗是何等人物,就是想要刺杀,也是对着皇上和齐王殿下,毕竟如今想要挽回局势,除非这两人出了什么意外,我如今已经没有那么大的价值了,行刺我就是成功了,最多也是激怒皇上和齐王罢了,除非是纯粹泄愤,否则行刺我全无道理。”

李顺苦笑道:“公子,有些人行事是没有道理的,魔宗这样的人做出事来,怎会次次被人料中。”

我正要劝解于他,突然耳边骤然响起三声琴音,琴声铮铮,犹如惊雷入耳,我只觉心头血涌,身形一颤,李顺的手掌已经按在我的背心,真气渡入。

接踵而来的连绵不绝的琴音,丝丝如缕,明明声音不高,却是清晰入耳,从何而来,只是仿佛弹琴人就在身边一般,琴声明丽中透着隐隐愁绪,仿佛冻结的冰河,阳光下晶莹剔透,美不胜收,河面下却是杀机隐隐,凶险暗藏。琴声越来越激越,大军驻足不前,人人都觉得这琴声排山倒海而来,明明己方是重兵环绕,却觉得如同沧海孤舟,无依无靠。

就在这时,那一辆被重重保护地马车上传出了如泣如诉地乐声,非丝非竹,却是清越缠绵,那琴声激越高亢,那乐声却是一丝不绝,缠绕在琴声之上,遇强愈强。

不多时琴声渐渐停止,然后从古道旁田野深处,清晰可闻地传出几声“仙翁、仙翁”的琴声,虽然众人多半不通音律,可是却分明听从琴中相邀之意。

我面上神情微变,这琴声是何人所弹,我一听便知,可是令我意外的是这琴声中隐隐带着的另外一重含义,那弹琴之人分明是身不由主,所以才会愁绪万千。挑开车帘,我淡淡道:“且在这里稍住,小顺子、呼延寿随我一同前去拜见魔宗。”

李顺和呼延寿面上都闪过惊容,但是他们也心中有所预料,并未提出什么疑问,呼延寿正色道:“魔宗深不可测,两国又是敌对,大人不可轻身涉险。”李顺虽然没有说话,可是满面都是不赞同的神色。

我不容反驳地道:“我就是想要改道也是迟了,就算有五千铁骑,也不过能够自保罢了,再说魔宗何等人物,既然邀我相见,就不会妄下杀手,好了,我意已决,你们不用说了。”

呼延寿神情一震,这平日温文儒雅的青年眼中突然闪现坚毅神色,言语中更是透出不容辩驳的威严,他心一横,暗道,若是大人有所损伤,最多我陪葬就是。下定决心之后,他亲自选了虎赍卫武功最强、配合最严密的十八人随行,又传下军令,令三军将前方的田野团团包围,一旦里面有什么不妥迹象便要发起攻击,玉石俱焚。

在呼延寿安排人手的时候,我却是不慌不忙地把玩着手中折扇,对面色冷如冰霜的李顺视若未见,虽然有些突如其来,但是和魔宗的相见早在我计划之中,只不过原本以为会在晋阳合围之后罢了。三大宗师,凤仪门主不必说了,慈真大师不愧是得道高僧,却不知这位北汉国师,魔宗宗主又是何等样人?见他几个弟子,段凌霄气宇轩昂,勇毅果决,不愧是魔宗嫡传,萧桐精明能干,虽然屡次受我所欺,不过是失了先机,当年身死雍都的苏定峦也是刚烈忠勇,令人心折,秋玉飞虽然孤傲淡漠,但是人品才华堪称绝世,不愧是名门弟子,就是如龙庭飞、谭忌、凌端等人,只是接受过魔宗指点之人,也都是当世英雄豪杰,有徒如此,魔宗必然不致令我失望吧。

见呼延寿已经调度完毕,我缓步当车,向琴声传来之处走去,方才呼延寿已经令两个虎赍去探过道了,有他们领路,自然是直捣黄龙,不过我不会武功,足上丝履每每陷入松软的泥土中,行走起来颇为艰难,李顺几次想要伸手搀扶我,却都被我婉拒,去见魔宗宗主啊,当然要抱着虔诚之心,形容上狼狈一些正显诚意么。

穿过田间小道,绕过一个小山坡,背风处的矮坪早已被人平整清理过了,一座营帐扎在其上,和可以遮风避雨的军帐不同,这营帐的帐幕都是白色丝幕,在阳光的映照下几乎可以一眼看穿,帐门处未有遮挡,可以清晰的看到帐内情景。数丈方圆的营帐内,地上铺着厚厚的华美温暖的羊毛地毯,只见厚度就知道下面铺着厚厚的地毡,足可以将地底的寒气隔断,帐内没有椅子,只是有四五个锦缎为面的蒲团,和几张样式古朴大方的矮桌,营帐一角,青铜香炉中正升起袅袅幽香,虽然陈设简单,可是每一件都是精美非常,透出这里的主人不同于流俗的气度。

呼延寿等人可全然没有欣赏的心思,虽然碍着帐内主人的威势,他也不敢令虎赍卫接近营帐,但是却是四散开来,将营帐隐隐围住,我微微一笑,虽然知道此举纯属无用,但是却也不愿出言劝阻,就让他们心安一点不好么。走到帐前,我看看里面华贵的地毯,再看看满是泥土的丝履,微微一晒,索性丢掉鞋子,径自走入帐中,对着那坐在正中主位,相貌儒雅斯文,气度雍容的蓝衫中年人深深一揖,道:“末学江哲,拜见宗主,晚生仰慕前辈已非一日,今日陌路相逢,蒙前辈宠召,当真是幸何如之。”

京无极的目光定定的落在眼前这青衣青年身上,一袭普普通通的青衫,衣衫下摆尚有泥土的痕迹,丝履已经脱在帐外,头上未戴巾冠,只用一根玉簪绾住灰发,哪里像一个身份贵重的大雍侯爵,驸马都尉,倒似是山野书生,无拘无束,明明面对着自己这个举手投足之间就可以取其性命的强敌,但是容色淡淡,似乎全无生死之念,仿佛他只是来拜会一个至亲长辈一般随意自然。

唇边露出一丝微笑,心中却是微微叹息,京无极伸手虚搀,道:“江先生不必多礼,贵客远来,风尘仆仆,京某不过是略尽地主之谊罢了,请坐。玉飞,请江先生用茶。”

我直起身,拣了一个蒲团坐了,李顺则是第一时刻站到我身后去,虽然不谙武功,可是我能够感觉到他身上的剑拔弩张的气息。轻轻用手肘撞了他一下,感觉到他身上的紧张气息突然消失不见,恢复成往日的平静淡漠。就在这一瞬间,我感觉到京无极略带赞许的目光掠过。防若未觉,我抬起头,看向一身黑衣,端着茶盏单膝跪在我面前,神色端凝的秋玉飞,笑容满面地道:“玉飞贤弟,多日不见了。”说罢双手接过茶盏,却是丝毫不敢怠慢,秋玉飞这样的人物,若非今日我是他师尊的座上宾,焉能如此大礼,不说我爱他重他,只凭他的身份地位,就不应轻慢于他。

秋玉飞眼中闪过莫名的情绪,这个人曾经是自己深深相负之人,可是如今却又知道自己多半是他手上的棋子,觉得恩怨两清之后,心头涌起的便只是当日的惺惺相惜。回到晋阳之后,自己去向师尊请罪,谁知师尊只是一笑了之,翌日就带着离开晋阳,想不到却是要在途中拦截江哲,他心中知道自己绝不会违背师尊的意愿,可是若是师尊决意要取这个青年的性命,自己又如何是好?心中的挣扎琴中表露无疑,想不到江哲仍然来此相见,而不是迅速带着大军逃去,这一会面是否生死相见,秋玉飞心中殊无把握。

京无极看向微笑品茗的江哲,目光落到他的两鬓星霜之上,叹息道:“江先生未过三旬,便是早生华发,当真是可叹可怜,雍帝能有先生这样忠心耿耿,呕心沥血的谋士,难怪所向披靡,不过大局初定,就解去先生监军之职,不知先生可否介意,又不顾关山路遥,召先生前往相见,不知是否君臣情深,迫切想和先生相见呢?”

我恭恭敬敬地道:“宗主过誉了,哲生性疏懒,尽人皆知,所谓呕心沥血,不过是少年识浅,不顾惜身体罢了,以致少年华发,贻笑大方。至于说到天子爱重,君臣情深,就更谈不到了,天子乃是万民之主,君臣名份攸关,安能有偏爱私情。且哲体弱,皇上不忍加以重担,担任监军不过是不得已而为之,如今将帅同心,哲再无用处,故而免职一事理所当然,至于千里相召,乃是关系代州军务,不便相告,还请宗主见谅。”

京无极眼中闪过一丝惊疑,道:“久闻先生外柔内刚,昔日对着凤仪门主尚且傥傥而谈,毫无畏惧之心,今日却为何对京某这强敌如此坦诚,知无不言,莫非先生不畏凤仪,却畏京某么?”

我淡淡一笑,道:“宗主何出此言,哲有问必答,乃是因为宗主是玉飞贤弟的师尊,哲与玉飞不打不相识,虽然昔日有些不快,可是哲却仍然视玉飞如同知交,这样一来,宗主也是哲的长辈,长辈有所询问,只要不关系我军机密,怎可不回答呢。”

京无极似笑非笑地道:“原来如此,只是江先生为雍帝、齐王出谋划策,坏我大事,北汉上下无不切齿痛恨,若能取先生首级,必能够鼓舞士气,且乱大雍军心,本座来此也是存了杀意,先生如此临危不惧,是以为本座心慈手软,还是以为你这几千铁骑,身侧亲随可以保住你的性命,还是以为我会看在玉飞面上饶你不死呢?你放玉飞归来,是否想让他劝阻本座,好保住自己性命呢?”

这番话宗无极说来虽然是轻描淡写,但是听在李顺、呼延寿、秋玉飞等人心中却是觉得字字诛心,声声震耳,且不论呼延寿手心见汗,就是李顺、秋玉飞两人,本已都晋入先天境界,仍然是心中一乱,李顺自然是一心提防京无极的发难,秋玉飞却是心中犹豫难决,营帐内外气氛顿时变得凝重沉滞,令人几乎喘不过气来。只有一人仍然是神情如常,便是那免冠铣足的江哲。

我当着帐内敌友,一位宗师,两位先天高手之面,舒展筋骨,大大地伸了一个懒腰,然后也不再保持跪坐的姿势,而是换成箕坐的姿势,笑道:“方才是晚生拜见朋友的长辈,自然要礼数周到,恭恭敬敬,如今宗主既然已经说明是敌非友,那么哲也不必拘束了,还请宗主勿怪,哲平日懒散惯了,实在不耐烦那些礼数。”

我这么一说,却见秋玉飞面上露出啼笑皆非之色,而京无极面上也是神色和缓,虽然看不到身后李顺的神情,可是多年相伴,只从他气息的变化上也知道他心中也是敌意稍减,他对我十分了解,自然知道我不会拿性命开玩笑,这样做必然是有所仗恃。

我当然不会过分放肆,正色道:“宗主此来,只携玉飞一人,若是有心要刺杀在下,怎会琴声邀客,五千铁骑并非虚设,若是宗主和玉飞行雷霆一击,尚有得手生还的可能,如今哲虽入罗,但是外有大军围困,内有小顺子相护,若是宗主此时出手,取江某性命或者易如反掌,但是想要生出此地却是艰难非常,就是宗主无妨,玉飞也绝难逃脱,玉飞贤弟对宗主尊敬孝顺,想必宗主尚不会置其于必死绝境。”

我说到此处,见京无极虽然不曾言语,但是神色间颇有许可之意,便继续道:“更何况宗主自入北汉一来,对于行刺之事已经不甚看重,这也难怪,北汉民风豪勇,不喜阴谋诡计,行刺这等事情若是偶一为之尚可,若是经常做来,不免令魔宗在北汉民众眼中沦落为阴险小人,宗主身份尊崇,更是不能轻易出手行刺。玉飞和段大公子行刺在下,一来我素有阴柔诡谲的名声,非是英雄好汉,让北汉军民觉得行刺我尚可接受,二来,兵危战凶,江某乃是关键人物之一,行刺我一人得益不浅,所以才无人反对,如今江某已经解去监军之职,已经不是这战局中的重要人物,宗主地位又远远胜过段大公子和玉飞,所以宗主行刺我非但不能激励北汉军心,反而降低了自己的身份,而且除了激怒我军之外又得不到什么实际的利益,所以宗主此来当不是行刺。再说,宗主邀我相见,若是骤下杀手,岂非贻笑天下。”

京无极眼中闪过笑意,淡淡道:“你说了这许多理由,却都不是我不杀你的理由。”

我心中一喜,总算得到一句实在话,看来性命无虞,连忙恭恭敬敬地道:“请宗主示下。”全然忘记我无礼的坐姿和可以说是狼狈的形容。

京无极微微一晒,道:“京某既然已经下了兰台,便是抛却国师身份,若要杀人,哪里还会有什么顾忌,纵你有无数的理由,我要杀你也不会皱一下眉头,何须考虑玉飞心意,更不会顾忌什么地位身份,至于有没有利益更是不必考虑,只凭杀你可以泄我之愤,便无人能够改变我的心意。今日不取你性命,本座唯一的理由就是不想杀你。”

我听得浑身冷汗,好险,好险,从京无极说话之时那种情真意切的神情,便知道他所说绝无虚假,他当真只是不想杀我罢了,虽然不知为什么,但是能够保住性命当真是老天爷保佑。

想到这里,我连忙恢复跪坐的姿势,摆出最有礼貌的姿态,道:“多谢宗主不杀之恩,且不知宗主此来有何指教,哲若有效劳之处,无不应命。”

京无极心中微叹,江哲之名他早已耳闻,他与凤仪门主虽然曾决生死,可是两人之间却是没有一丝敌意,反而生出惺惺相惜之念,此后虽然关山阻隔,却是一刻都没有忘记当日白衣染血的绝代丽人。自闻梵惠瑶身死猎宫之后,京无极便千方百计将前后经过一一探察,虽然有些事情无人知晓,没有外传,但是其中轮廓已经知道十之八九。迫死凤仪门主,就是眼前这个青年一手而为,可是奇怪的,京无极却全然没有生出憎恨之心,只因这个青年实在已经将能够运用的力量都发挥到极至,他只是存了有朝一日在智慧上将这青年击败之心,就是派秋玉飞、段凌霄两次刺杀,贯彻其中的也是双方的斗智斗勇,非是全凭强横不可抵挡的武力,可惜终究是功败垂成。东川事败的消息已经传到,北汉局势几乎已经是无可挽回,虽然晋阳尚有一战之力,也不过是苟延残喘,这失败的非是别人,正是他京无极自己,布局天下已成虚话,就连自己的心爱弟子也个个败在江哲手上,这一次魔宗虽然力量未损,却是一败涂地,怎能不让他动心,想亲眼见一见这个将无数豪杰玩弄于股掌之上的文弱书生呢。

岂知闻名不如见面,今日一见才觉得这青年果然是名不虚传,明明当着自己的面,这青年忽而恭敬,忽而放纵,种种变化令他也生出不能捉摸的感觉,可是却偏偏有一种自然而然的味道,令人觉得他实在是诚心诚意,且无丝毫惧意戒心。对之如饮醴酒,如沐春风,忽而惊觉,才发觉自己身陷绝境,秋玉飞当日万佛寺的处境京无极此刻才能全部领会,对心爱的弟子投以同情的一瞥,京无极道:“今日逆旅相逢,已属难得,楚乡侯对我魔宗处处留有情面,想必定有话和本座说,是么?”

我淡淡道:“宗主既然说到这里,哲也不敢隐瞒,若是哲对魔宗有恶意,当日就绝不会放过宗主首徒,段凌霄段大公子,当日我们尚属敌对,且胜负未可断言,所以哲也没有多说什么,今日宗主亲来,正好谈谈此事,其实就是宗主不说,等到晋阳合围之日,哲也要拜托玉飞贤弟代为引见。”

京无极冷冷道:“你是想要劝降,是么?”

我微微一晒,道:“宗主是何等人物,焉能屈膝请降,这劝降二字再也休提,哲只是代皇上提出一个建议,晋阳一旦合围,就是北汉覆亡之时,昔日宗主中原一败,遂遁入北地,皇上只是希望北汉亡后,宗主不要再去南楚。”

京无极若有所思地道:“雍帝之意,京某明白,天下一统契机已现,京某若是去了南楚,对于雍帝来说虽然终有解决之道,却是不免太麻烦了。”

我笑道:“其实这个条件不说也罢,宗主是何等样人,北汉国主尚称贤明,对宗主尊敬有加,这才博得宗主青睐,南楚民风柔弱,君弱臣暗,怎配栖得凤凰,只要宗主答应,大雍千万里山河,任由宗主来去,魔宗弟子一旦解甲归隐,就不会被当成北汉余孽看待,虽然白道中人或者会对宗主不谅,但是魔宗弟子,个个英雄豪杰,怎会对此有所戒惧。天下一统,宗主也当过过悠闲轻松的日子了。”

京无极眼中闪过一丝凌厉,道:“条件倒是优厚非常,可是你也说了,国主待我魔宗不薄,京某不才,焉能此时抛弃国主和无数将士。今日一见,不过是想见识一下江先生的风采罢了,至于方才所谈之事,不过是本座早已料到你有些话要说,故而令你明言,只因今日一别,来日就是生死相见,本座不想到了雍军兵临城下之时,你还要利用玉飞对你的知己之情,难道你当真以为本座会贪生畏死么?”

我早已预料到京无极会这样说,肃容道:“此言实在是江某肺腑之言,江某和皇上多次传书密谈,都提及魔宗之事,皇上常言,宗主与凤仪门主都是一代宗师,凤仪弟子只知道在朝中和后宫兴风作浪,全不似魔宗弟子浴血沙场,换取荣耀和功名,虽然当日宗主落败,但是今日却是宗主远胜凤仪门主了。魔宗弟子不会抛弃同袍,这一点皇上早有预料,虽然如此,仍然有此建议,只因皇上当真是对魔宗弟子另眼相看。今日之言,只需请宗主记在心中,今日一别,该如何厮杀就如何厮杀,皇上不会有怨恨之心,不论到了何时,这个建议都不会失效。”

京无极听到此处,也不由动容,自己这次突然生出想和江哲一见的念头,又这样阻道相见,如今不知道是庆幸还是后悔,自己听到雍帝这样的厚待都忍不住动心,更何况魔宗弟子呢,一旦他们有了退路,是否还会拼死血战,或者这样的差别将改变北汉的命运,可是无论如何,京无极心中也有一丝感激,魔宗不会因为得罪了可能一统天下的雍廷而彻底消亡,这已经是他听到的最好消息。

想到这里,京无极缓缓闭上双目,道:“时光不早,江先生应该上路了,玉飞当奏一曲为侯爷送行。”

秋玉飞低声领命,走到帐幕一角,将那“洗尘”爱琴放到膝上,十指轻动,清越的琴声响起,意境清远高阔,种种离愁别绪,化作天外烟云。

我起身一揖到地,今日相见,已经达到我的目的,此时也该是告别之时,走出营帐,套上丝履,这次我可不会走回去了,小顺子搀着我很快就回到马车上,呼延寿一声令下,五千铁骑迅速北上,全无逗留之念。

直走出三十里,我才突然想到,方才怎么竟然没有生出将京无极围杀的念头,虽然若是我这样做了,难免损失惨重,就是将自己的性命搭进去也有可能,可是我并非是经过深思熟虑觉得胜算不大而放弃,而是根本就没有生出一丝恶意杀机,心中恍然,魔宗果然是当世之雄,仅凭举止言谈中隐约可见的威势已经让我心折,这样的人物,岂是凤仪门主可以比拟的,想来若是两人今日一战,胜得必然是魔宗宗主吧。忍不住看看小顺子,他是否也会受到压制影响,这样一来岂不会有伤他的修为么?谁知我一眼看去,小顺子面上宝光隐隐,静默不语中带着深深了悟,看来他的修为不仅没有受到什么损伤,还有了一些进步,我心中一宽,看向道路两边的青青黍麦,露出心满意足的笑容。

第四十章    雁门喋血

第五部还剩一章完结,明日一次更新7000字,特此告知!

————————————————

满眼都是血红的天地,天空,泥土还有战士的衣甲,都是猩红的颜色,绝望的情绪潮涌一般袭来,敌人的狰狞面目仿佛就在眼前,自己不论如何挣扎,都无法摆脱林立的刀枪和如同暴雨一般的箭矢。就在他最无助的时候,灰暗阴沉的天空突然出现了一缕阳光,透过层层彤云,带来了温暖的希望,然后就在那血海当中,出现了那个他熟悉敬慕的青色身影。“公子!”赤骥高声叫道。然后他就被人粗暴的推醒了。

睁开眼睛,毫不意外地看到林彤满是怒气的俏丽面容,林彤怒道:“你能不能把你的主子先抛到脑后,这已经是你第十四次在梦里叫着他的名字了。别忘了你在雁门,不是在你主子身边,就算是你的主子再仗义,现在不也任你在这里拼死拼活么,有那个精力,还是想想如何对付蛮人吧。”

望着林彤轻嗔薄怒的神情,赤骥只觉得心中一甜,他能够听得出林彤话语中的微微酸意,就是身边那些经过的代州军勇士,望向两人的目光也是充满了笑意,连续五天五夜,蛮人几乎是不停息的进攻,两人初时并肩作战,不知多少次从敌人手中救下对方,到了后来,赤骥表现出了颇为惊人的军事才能,所以他和林彤开始轮流指挥军队御敌,这之后的整整三天,两人就只能在叫醒对方的时候说上几句话,可是却丝毫不觉的孤单,仿佛对方就在自己身边一般。在这生死不由自主的时地,两人都刻意忘记了之间的重重阻隔,除了林彤总是嫉妒赤骥对江哲的极度崇拜之外。

赤骥坐起身来,侧耳听去,并没有喊杀声,想必蛮军还没有攻城,伸出手臂揽住林彤的纤腰,轻轻用力,林彤促不及防,被他拉入怀中,北地民风豪爽,周围的军士不以为忤,反而都高声打起呼哨了,林彤满面通红,一州撞在赤骥的胸口,赤骥一声痛呼,林彤立时想起前日赤骥胸前受了箭伤,不由心中一软,赤骥趁机将林彤紧紧抱在怀里。林彤婴宁一声,埋首在那充满男子气息的胸膛上,羞赧难言,混不似可以指挥千军万马的女将军,赤骥心中一颤,原本的调笑之意转为一腔柔情。

这时,林远崇从远处跑来,高声道:“郡主,王兄弟,侯爷请你们过去。”赤骥和林彤都是慌慌张张地跳了起来,林彤几乎没有面对身边的长辈和同袍的勇气,低着头一路小跑,不一会儿就消失得无影无踪。赤骥却是有些犹豫不安,代州侯林远霆是什么人物,镇守代州多年,令蛮人不能南下一步,虽然如今年老多病,但是虎老雄威在,更何况他是林彤的父亲,赤骥心中忐忑不安,望着林远崇,就是没有勇气走出一步。

林远崇笑道:“哎呀,怎么骁勇善战的沙场勇士如此腼腆呢,放心,我族兄豁达得很,不会计较你调戏彤儿的事情。”

赤骥望望城外血流遍野的惨况,吞吞吐吐地道:“这个,郡主现在去见林侯爷,万一蛮人现在进攻,我还是留在这里吧。”这时,强而有力的巨掌重重地拍在他肩上,一个苍老中透着矫健的声音道:“小子,放心去吧,有我这把老骨头在,守上一两个时辰还是没有问题的。”赤骥露出苦笑,没有回头也知道来人正是代州的齐老将军,上上下下谁敢和这位戎马一生,浑身是伤痕的老将军争辩,可是真的要去见林远霆么,赤骥心中犹豫难决。

林远崇眼中闪过寒芒,冷冷道:“怎么,你不想去见侯爷,莫非你对郡主只是逢场作戏么?”

赤骥打了一个寒战,低声道:“就是侯爷同意又能如何,我违背公子训诫,虽然公子开恩,放我来到代州,但是日后公子若是召我回去问罪,我亦不能反抗,而且蛮军势大,雁门危殆,就是退了蛮军,对着雍军又怎么办呢?”

他的声音很低,但是齐老将军和林远崇都听得清清楚楚,两人眼中都闪过迷茫之色,这何尝不是两人心中几乎不敢去想的隐痛。林远崇望望赤骥,想起这个少年的主人就是令代州局势糜烂如此的罪魁祸首之一,心中涌起迁怒之意,但是看看这个连日苦战,形容憔悴的少年,却是一句恶语也说不出来,代州勇士,本就是恩怨分明之辈。轻叹一声,林远崇道:“走吧,侯爷在等你,难得今日他清醒过来。”

雁门关内一件静室,仿佛隔绝了血腥的战场,室内溢满浓厚的汤药气味,没有一丝奢华的房间和代州普通平民的居室没有什么不同,宽大的木榻上,一个老者坐起身来,正在林彤的服侍下缓缓喝着一碗苦涩的汤药,虽然形容枯槁,满头霜发,可是仍然可以看出昔日的儒雅轮廓,可见这老者当年必是一个俊朗英武的美男子。进到房中,赤骥反而平静下来,上前拜倒道:“晚辈王骥,拜见侯爷。”

那老者眼中闪过凌厉的光芒,仔细的打量了赤骥片刻,道:“你就是楚乡侯的侍从,伯乐神医王骥,这名字是真的还是假的?”

赤骥只觉得那老者目光如同利剑一般,穿透了自己的心扉,不由感叹难怪此人可以镇守代州多年,果然是名将气度,他恭恭敬敬地道:“晚辈本是孤儿,除了知道自己姓王之外,并没有名字,昔日我家公子收留晚辈在身边,赐了赤骥这个名字,后来晚辈便为自己取名王骥,并非是假名。”

林远霆淡淡一笑,道:“彤儿,你二哥的灵柩是否已经运回去了了?”

林彤眼圈一红,道:“是的,等到蛮军退后,还要父亲主持,将二哥的灵位送入祠堂。”

林远霆爱怜的拍了拍林彤的肩膀,对赤骥道:“贤侄见笑了,彤儿这孩子心太软,其实伤心什么呢,百余年来,代州林家死在沙场的不计其数。我这一辈兄弟五人,只有我一人活了下来,几位兄弟都死在战场上,没有一个善终,如今又轮到他们这一辈,唉,澄迩已经去了,碧儿和澄山、澄渊都被阻截在晋阳,一旦雍军合围,也是九死一生,澄仪性情粗暴,彤儿年轻识浅,今次林家就是烟消云散也没有什么奇怪。我林家有规矩,只有战死沙场的族人的牌位才有资格进祠堂享受后人供奉,百多年来,不能进去的也不过寥寥数人,本来老夫以为数年边疆平静,大概是要终老病榻,没有机会进祠堂了,想不到今日又有了机会,彤儿,为父决定冒险一次,拼掉蛮军的主力,虽然这样一来雁门守军恐怕会全军覆没,可是蛮人也是元气大伤,就有法子将他们逐出代州。”

林彤“哇”的一声痛苦出声,扑在父亲怀中泪如泉涌,林远霆这是在交待后事,她心中怎不明白,赤骥上前欲伸手安慰他,却被林彤避过,赤骥心中一痛,朗声道:“侯爷,郡主,若有什么重责请交给赤骥去做。”他心中只有一念,便是死在林彤之前,林远霆心中了然,望向赤骥的眼神多了几分嘉许,说道:“贤侄人品才华都和彤儿相配,只可惜彤儿既然身为林家的后人,就没有舍弃代州军民逃生的理由,彤儿,你可怨怪为父么?”

林彤擦干眼泪,道:“爹爹何出此言,若能战死沙场,女儿也可进入祠堂,这是何等荣耀,女儿怎会怨怪父亲,请爹爹吩咐,我们该如何做?”

林远霆欣然一笑,道:“好,我林家果然没有贪生怕死之辈,不过你们也不可轻易舍弃生命,此战之后或能留得性命,你们也不可轻言牺牲,彤儿,我昨日已经令你大哥带了降表去见雍帝了。”

林彤大惊,道:“父亲你说什么,请降,这是为什么,你将母亲和姐姐,还有三哥四哥置于何地?”

林远霆抬手阻住林彤说话,淡淡道:“林家是为了代州而生,不是代州为了林家存在,我已经想得很清楚,雍帝的大军截住代州和晋阳的通道,代州已经成了孤军,只能独自面对蛮军,这次我虽然可以设下计策,破去蛮军主力,但是四分五裂的蛮军一定会更加猖狂狠毒,代州主力被阻截在晋阳,对着十数年来最猛烈的一次侵扰,代州已经是无能为力了,唯一的办法就是投降大雍。雍帝乃是贤明圣主,怎会不知道代州的重要,之所以没有攻入代州不过是碍着我们林家罢了,如今我令你大哥去请降,又将仅剩的兵力消耗在雁门关战场,雍帝就再没有任何顾忌,必然会星夜前来援救,代州几十万百姓就可以免受蛮人残害。”

林彤泪如雨下,她明白父亲是要用林家的牺牲换取代州的生存,她抽出腰间佩刀,在左臂上一划,鲜血泉涌,血泪交映下,林彤肃容道:“女儿明白父亲的意思,林家只可以为代州牺牲,若是女儿侥幸生还,也会向雍帝请降,绝对不会让代州军民为了我林家的私事和大雍铁骑为敌。”

赤骥听到此处也是心痛如死,这两父女所说他全然不能辩驳,昔日离开公子的时候,公子就曾经暗示就是代州胜了蛮人,林家也难逃覆灭的结局,因此希望他能够即使脱身,甚至就是带走林彤也可以,保住一人还是可以的,那是公子未曾言明的意思,可是此刻他却明白,自己心爱的女子果然是巾帼英杰,是断然不会苟且偷生的。他扑通跪倒在地,道:“侯爷,晚辈对郡主情有独衷,希望侯爷将郡主许配给赤骥,赤骥情愿和郡主同生共死。”

林远霆眼中闪过欣慰的神色,但是却摇头道:“贤侄,你近日来助我代州军民守卫雁门,已经是犯了贵上的大忌,如今何必还要蹈此死局,楚乡侯圣眷正隆,贤侄你日后前途无量,何必要为小女放弃一切。”

赤骥不语,接下腰间竹笛,吹奏了起来,那笛声高亢激越,林远霆虽然出身将门,却是娶了一位曾有才女之称的公主妻子,对于音律也不陌生,听了片刻,拊掌唱道:“将军百战身名裂。向河梁、回头万里,故人长绝。易水萧萧西风冷,满座衣冠似雪。正壮士、悲歌未彻。啼鸟还知如许恨,料不啼清泪长啼血。谁共我,醉明月?”,词曲勇烈,令得室外守卫的将士也都侧耳倾听,心中满是赴死的豪情。林远霆叹息道:“想不到你也能领会铁血金戈,生死一掷的豪情,好,好,你果然配得上彤儿。”这时,笛声一变,却是缠绵悱恻中带着义无反顾的激烈,林彤心中一颤,沉迷在情郎用心血演奏的曲调当中,甚至不知曲声何时停止,只听见赤骥一字一句道:“舍却残生犹不悔,求侯爷将郡主许配给我。”

林远霆看向林彤,淡淡道:“彤儿意下如何?”

林彤眼中泪光盈盈,面色羞红中带着凄然,明知马上就要以身赴险,九死一生,让她如何能拒绝情郎甘愿陪她赴死的一片情意。她侧过脸去,道:“全凭父亲作主。”

林远霆剑眉一轩,道:“好,既然你们两人情投意合,本侯就成全你们,王骥,我的女儿出嫁也不用选什么良辰吉日,你若愿意,就在雁门关城头,本侯面前,代州军万千勇士的面前,你们拜了天地,结为夫妻如何?”

赤骥大喜,叩首道:“王骥叩见岳父大人,一切全凭岳父作主。”

雁门关下,前几日攻城的失败让所有蛮人的心中都是怒火熊熊,完颜纳金见雁门关内守将的力量越来越弱,打定主意这次定要成功,当众歃血,折箭立誓之后,蛮人联军再次聚集中关城之下。完颜纳金和其他各部的酋长指点着雁门关商量如何攻打的时候,只听关上突然鼓乐喧天,众蛮军都是极目望去,只见雁门关正门之上,刀枪剑戟上结着红色彩绸,衣甲鲜明的代州将士分立两侧,个个都是喜气洋洋,一队身穿喜服的新人正在一个相貌清峻的老者面前对拜结亲。三拜之后,关上欢呼声四起,众蛮人侧耳听去,那些人却是在高声呼唤道:“郡主和郡马爷百年好合,白首偕老。”

完颜纳金大怒,马鞭一指,道:“这些人竟敢轻视我们大军,两军阵前居然张灯结彩拜上了天地,立刻开始攻城,本王要让他们喜事变丧事,林远霆就在上面,这些年来我们多少父执兄长死在这人手中,谁能取他首级,就是我草原第一勇士,赏金千两,美女一名。”

这时有人高声道:“汗王,谁不知道林家有一对姐妹花,不如这样,谁能杀了林远霆,就将城上的新娘子赏给他。”完颜纳金举目望去,却是白狼部的酋长莫尔干在那里喊叫,他微微一笑,高声道:“传本王之令,谁能杀了林远霆,红霞郡主就是他的爱妾,不过诸位可要生擒这位新婚燕尔的郡主娘娘才行啊。”另一个蛮人将军大笑道:“新婚燕尔,老子最喜欢抢别人的新娘子,林远霆,快些洗干净自己的脖子等老子来砍吧。”

城上的代州军听着下面的污言秽语,个个面沉似水,却都沉默不语,耻辱是要用鲜血才能洗清的,原本带着如在梦中的喜悦的赤骥面色铁青,却只是脱下新郎袍服,露出一身鲜明的衣甲,而林彤冷冷地瞧了下面一眼,素手一分,那红绫嫁衣化作蝴蝶碎去,露出一身火红的软甲,两人站在林远霆身侧,恰似一对金童玉女,误落凡尘。

林远霆坐在椅上,他的力气已经不足以长久支撑他的双腿了,朗声道:“完颜纳金,你来吧,你的父亲叔叔都是死在雁门关下,看看你有没有这个本事攻上来,不过你堂堂的汗王,想必没有心情和从前一样上阵杀敌了吧。

强烈的讥讽让完颜纳金面色数变,蛮人本崇尚武勇,想起这几日完颜纳金始终不曾亲自上阵,不免暗中说些言语。完颜纳金本是极为自负的一个人,狠声道:“林远霆,你等着,本王定要亲自取你首级,掳回你的宝贝女儿为奴。”

此言一出,城下哗然,城下的代州军也忍不住叫骂起来,完颜纳金手一挥,号角声起,蛮军开始了最猛烈的一次攻关之战。令完颜纳金等人欣喜的是,这一次代州军的力量减弱了许多,想来是多日的苦战让他们消耗太多的缘故,但是他们仍然顽强的抵抗着,箭射完了,用刀砍,刀锋钝了就用拳头和牙齿,甚至有些再无力气的军士干脆抱着攻上城头的敌军滚下关去,有些军士就是死后也紧紧咬着敌人的咽喉,明明雁门关已经岌岌可危,可是就是攻不上去。这一日黄昏,完颜纳金终于按耐不住,将特意保留下来的格勒部最精锐的军队雪狼军派了出去,雪狼军乃是完颜纳金亲自挑选训练的劲旅,个个都是草原上千里挑一的勇士,格勒部就是靠着雪狼军才力压群雄,扶持着完颜纳金登上汗王之位。一声令下,雪狼军顺着云梯攀上,每个人的动作都是快如闪电,城头的守军已经疲惫不堪,几乎是一瞬间,雁门关城头之上就已经被雪狼军占据,完颜纳金大喜,令人吹起进攻的号角,众蛮军耀武扬威,只待雪狼军从里面打开关门,就要一拥而入,血洗雁门关,然后踏上中原沃土,进行杀戮和掠夺。

冲上城头的雪狼军本已养精蓄锐多日,城上的疲军怎是他们的对手,几乎是一转眼的功夫,他们已经冲破了重重防线,向着坐在高处指挥作战的林远霆扑去,擒贼先勤王,斩杀林远霆乃是完颜纳金之命,他们自然都想争夺这个功劳。

林远霆苍白的面上露出一丝红晕,手一挥,在暗处隐藏了一日的伏兵冲了出来,截断了雪狼军的退路,为首的正是林远崇,这支伏兵乃是整个雁门关中最精锐的勇士组成,这一日不论关上如何苦战,他们都只能隐在暗处不能援手,眼看这同袍亲人惨死,早已令他们生出誓死雪恨之心,就在他们冲出的一瞬间,早有军士将事先准备好的黑火药点燃,剧烈的震颤和轰鸣之后,已经将雁门关所有上下通行的道路封死,这是林远霆准备的死局,要将格勒部赖以威慑各部的武力铲除,这样蛮人将再度分裂。与此同时,雁门关的城门缓缓打开,露出了不设防的软肋。

面对着眼前的盛宴,蛮人各部酋长大喜,只道是雪狼军已经成功地夺取了关门,就连完颜纳金也忽略了城头上的异常,一马当先的冲入了雁门关,对城门处拼死血战已经被蛮军逼到绝境的代州军看也不看一眼,径自挥刀想杀上城头,可是一眼看到碎石堵塞的蹬道,完颜纳金心中一寒,也无心去想为什么代州军将城头和关下隔绝,大声喝道:“退,退。”可是他的声音淹没在蛮军兴奋的高呼声中,完颜纳金再也无法如臂使指的指挥被胜利冲昏了头脑的军队,被身后的军队胁裹着前冲了将近几百丈,完颜纳金近乎绝望地看到了一支整装待发的铁骑,策马站在最前面的正是赤骥和林彤,伴随而来的则是疾雨一般的箭矢,蛮军和代州军多次交战,每次若是中了代州军的圈套,就往往损失惨重,更何况如今主持雁门关军务的就是他们心中最畏惧的林远霆,不由有些慌乱,前面的蛮军拼命向后退,想回到他们占据优势的平原,而后面的蛮军尚不知道前方的变化,仍然向前冲杀。

就在蛮军陷入混乱的时候,在亲卫保护下后退的完颜纳金耳边传来弩机的声音,他下意识地俯下身躯,想避过随之而来的弩箭,可是混乱的战场上突然响起一串高亢的呼哨,他座下的战马闻声突然扬蹄而立,完颜纳金促不及防,身形暴露在弩箭的攻击范围之内,剧烈的疼痛袭来,他才听到弩箭穿透自己甲胄的声音,耳边传来亲信部将的惊呼声,近距离的强弩攒射,乃是白发百灵的阎王帖子。只觉得往事在脑海中接踵而来,完颜纳金不甘心地高吼道:“苍天无眼!”然后这刚刚登上蛮人最尊崇的宝座,满是野心,一心翼望可以重现昔日汗廷荣耀的青年汗王,就这样跌落尘埃。

失去了首领,原本慌乱的蛮人反而被激怒了,他们开始自然而然地组成小股骑兵,向代州军开始反攻,不需要强行合作,蛮人反而更容易发挥自己的战力,雁门关内外只听见杀生四起,不论是代州人还是蛮人,都忘却了一切地拼死厮杀。弓箭早就不知何时失落,赤骥手中的长枪犹如蛟龙,死死护住林彤的侧翼,此刻他万般庆幸昔日跟着李顺学过马上厮杀的枪法,这几年又下过一些功夫。林彤乃是武将世家出身,若论枪法更在赤骥之上,银枪如雪,影似梨花,血肉飞溅中更显得这一对璧人英武如玉。

只是代州军力量太薄弱了,虽然他们拼命苦战,换取了数倍的蛮人生命,可是越来越多的蛮军冲入关内,代州军却是没有援军,战局越来越倾向蛮军。见到这种情形,林彤无奈地发出了撤军的命令,这是林远霆的意思,到了这个时候,残余的代州军只能沦为敌人铁骑下的冤魂,既然已经达到作战目的,与其让他们战死此地,不如为代州军多留些种子。

听到撤退的号角,所有的代州军勇士几乎是含着泪退走,他们无力顾及被封锁的城头上的战况,甚至无力顾及他们年轻的统领,赤骥和林彤带着林家的死士断后,他们用鲜血和生命确保着代州军勇士撤退的道路的畅通无阻,军令如山,而且若是自己撤退的及时,或者郡主和郡马尚有生还的可能吧,每一个代州将士都奋力奔逃,许多受了重伤不愿拖累同袍的将士干脆挥刀自尽,还有一些战马受伤或者不能骑马奔逃的将士则是跟着林彤一起断后,几乎不到一拄香的时间,代州军的残部就已经突围而去,只有林彤、赤骥仍然带着百余人不能离开,这倒不是两人存心一死,虽然这样的念头早就深埋在心,可是他们都不情愿让这么多代州勇士陪葬,只不过蛮人已经将他们彻底包围,再没有突围的可能了。

林彤心中没有丝毫后悔和绝望,身为林家之人,就是女子也有舍身沙场的觉悟,她心中唯一的牵挂就是在代郡的母亲,不知道母亲会如何打算,托庇于雍军对这位外柔内刚的北汉公主来说,或许是不能接受的决定吧。耳边传来赤骥沉重的呼吸声,林彤侧过脸望去,只见那原本清秀洒脱的少年,如今已经是浑身浴血,身上更是伤痕累累,心中涌起不可遏制的感激和甜蜜,这个抛弃了青云之路,选择了和自己共赴黄泉的少年,已经是自己的夫婿,虽然只有短短的一日,但是林彤却觉得两人仿佛已经结发多年,再无彼此。仿佛是心有灵犀,赤骥也转头向林彤望来,四目相对,都是深情无限。然后两人几乎是同时出枪,将袭向爱侣的敌人刺倒。四周的蛮军望不到边,就像波涛汹涌的海浪,转眼间就可以将这支仅存的代州军淹没。但是两人却都仿若未见,就在这时,林彤的战马终于颓然倒地,身中数箭,创伤多处,这匹战马能够支持到现在已经是很难得了,赤骥连忙伸手一拉林彤,林彤借势飞起,轻盈如燕地落在赤骥身前,回眸一笑。赤骥左手紧紧握住林彤的左手,揽住她的纤腰,还以笑容,两人全然没有夺取无主战马的打算,多活片刻又能如何,还不如生死都在一起。

赤骥只觉得从没有像此刻一样心绪空灵,和心爱之人在战场上相拥,即使越来越近的蛮人凶恶的面容也不能让他心中生出一丝涟漪,握紧了银枪,他等着最后时刻的来临。恍惚中,他突然感到大地传来猛烈的震动,那是只有受过严格训练的骑兵全力疾驰才能产生的震动,莫非是我糊涂了么,赤骥苦笑,但是他很快就看到身边的林家死士和外面猛攻的蛮军眼中也都流露出相似的迷茫,那些蛮人甚至放缓了攻击。他还没有反应过来,耳边就响起了熟悉的号角声和越来越响的轰鸣声,赤骥落下泪来,哽咽中,他甚至无法开口回答林彤满眼的疑问,只是抱紧了林彤的纤腰,仿佛一放手,就会失去他心中的挚爱。

\chapter{第四十一章 遥望林泉}

五月二十日,代州使者入晋阳,嘉平公主闻凶讯,恸哭泣血,言曰:承父训,非以代州事林氏,以林氏事代州耳,乃令两兄率代州军出城降雍,后主闻之,唯叹息流涕,不肯阻,且遣人语主曰:可出城降之。主曰:受王深恩,死且不悔,焉能背离,乃止。

雍帝闻公主不归,感叹莫名,遣使入晋阳劝降,络绎不绝,后主感雍帝意诚,乃降。

——《资治通鉴·雍纪三》

就在这时,外围的蛮人开始奔逃,仅存的十几个林家死士抬头望去,一支青黑色衣甲的骑兵正在大肆屠戮着兵败如山倒的蛮人,铁蹄雷震,旌旗如海,正是雍军的前锋到了。烟尘弥漫中,冲到林彤等人身边的雍军骑兵流畅地左右一分,一个雍军将领策马奔来,而他身边一个身穿代州军甲胄的高大青年一马当先奔来,高声道:“彤儿,彤儿,父亲呢?”

林彤心中,死里逃生的喜悦和前途未卜的迷茫混杂一处,见到这个青年,种种情绪都化作乌有,她高声悲叫道:“大哥,大哥,爹爹在城上,早已没有了声息,只怕,只怕……”

那青年一声怒吼,转头扑向那已经被封堵住的蹬道,那个雍军将领轻轻一叹,一挥手,一些雍军随那青年而去,那将领肃容道:“末将李榷,忝居大雍威武军副将之职,奉陛下之命,救援雁门,不知诸位可还有余力为大军指引方向,追杀蛮军。”

林彤拭去珠泪,断然道:“我是林彤,愿为将军引路。”

李榷皱眉道:“郡主久战余生,只怕难以支持,而且郡主难道不想去看看林老将军的情况么?”

林彤断然道:“林彤的性命早已不是自己的,能够活到如今已经是上天庇佑,父亲是生是死,林彤已经无能为力,可是若让蛮人全军退走,林彤纵死也无颜去见代州父老,请将军放心,林彤尚可支撑。”

李榷仍然有些犹豫,赤骥出言道:“李将军请宽心,在下王骥,愿和拙荆一起为大军引路,在下熟知雁门关外的地理,当会有助大军追敌,请将军不必担心我们夫妇。”

那李榷目中闪过一丝耀眼的光芒,他在马上拱手道:“原来是楚乡侯门下的赤骥公子,失敬失敬,末将曾在寒园侍奉过先生,临别之时楚乡侯曾经托末将留心公子的下落,见到公子安然无恙,末将也心中安慰,且有公子引路,想必定然可以让蛮人欲逃无路。”

赤骥发出低呼,忍不住问道:“我家公子也到了忻州么?”林彤闻言心中生出恼意,正好有雍军牵来战马,她闷声不响地手肘一撞赤骥小腹,赤骥忍痛不已之时,她已经上了新的战马,策马向蛮人逃去的方向奔去。赤骥也顾不得和李榷多说,连忙追了上去。引得那些劫后余生的林家死士都是会心一笑,几个自负尚有余力的也策马追去,在前面为雍军引路。

李榷也是暗暗好笑,其实他也没有见到江哲,从十几日前,他就奉命进入代州,代州人都知道林家和大雍之间乃是敌对,如今雁门关血战正酣,竟是无人忍心将消息送去雁门,他们都担心林远霆若是知道大雍攻入代州的消息,牺牲了自己成全一州百姓,因此便自发的组织起来,阻挡雍军的攻势。虽然李榷已经多次声明欲救援雁门,那些民众仍然以为大雍是要趁火打劫,在不能伤害代州平民的情况下,雍军可以说是举步唯艰,往往是一夕数惊,好容易才到了代郡。这时候代州民众都以为李榷欲攻代郡,那里是林氏的宗祠所在,代州侯夫人安庆长公主如今就在代郡,李榷几乎是寸步难行,就在他苦不堪言的时候,遇到了准备去向雍军请降求援的林澄仪。而几乎是与此同时,江哲的信使也到了李榷面前,向他说明了赤骥在雁门协助林家守关之事。虽然不明白怎么江哲的门人会在雁门,但是曾经在寒园守卫的李榷也只能惊叹江先生的神机妙算罢了。有了林澄仪的指引,雍军前锋几乎是毫无阻碍地赶赴雁门,李榷心知皇上对代州林家十分器重,所以一路狂奔,尤其在遇到从雁门逃出的残军之后更是心急火燎。到了雁门,从千钧一发的危局中救下了林彤和赤骥,他心中也是十分庆幸,看来林远霆已经是凶多吉少,而林彤如今已是林远霆亲命的代州主将,有了她的合作就可以安定代州,这一点林彤恐怕比林澄仪更加重要,只看林远霆最后将大任交给幼女而不是长子,就知道这一点了,更何况和林澄仪同行一日夜,他也已经看出林澄仪虽然骑射高明,性情直爽,却是没有作为将帅的潜质。

这时,城头上突然传来了痛彻心肺的哭喊声,李榷轻轻一叹,就见林澄仪从蹬道冲下,翻上战马就向关外冲去,李榷见他泪痕满面,双目如血,心中更是怜悯,使了一个眼色,一个接近林澄仪的亲卫趁他无备,一剑柄将他击晕搀扶下去。这时,一个偏将从从城头下来,到了李榷马前,摇头赞叹道:“将军,代州军果然是英雄豪杰,城上简直是修罗场,三千雪狼军和所有代州军几乎全战死了,不过代州军一名将领叫做林远崇的仍然活着,还有几个代州军将士也只是身负重伤,虽然都不能说话和移动,但是性命应该无碍,属下已经令军医救治,林远霆已然战死,身边都是雪狼军和代州军的尸首,依末将所见,定是他以身诱敌,在身边设下埋伏诱杀敌军。”

李榷也是心中叹服,道:“好了,我们也去追敌吧,别让人将我们威武军瞧得扁了。”说罢策马扬鞭向雁门关外奔去。

在相隔两百年之后,中原的铁骑终于再次踏上了蛮人的土地,这一次足足追袭三百里,在代州军指引下,李榷将蛮人的主力击溃,此后的二十年,重建的代州军多次袭入草原,将蛮人各部打得七零八落,格勒部更是几乎灭族,自那以后,足足有五十年之久,蛮人偃旗息鼓,不敢窥视雁门关。北疆一地,固若金汤。这是后话不提,雁门大胜之后,当务之急就是如何面对已经控制了整个代州的雍军了。

如今的代州,残军不过千余人,主将乃是红霞郡主林彤,虽然兵力微薄,可是从李榷进入代州以来的经验来看,如果林家不顾一切发动代州民众抵抗雍军,这绝对是一场苦战。林远霆在雁门关苦守无援,一来是因为按照惯例,代州各郡县的乡民团练主要是为了保护乡梓用的,一般不会参与大战,二来雍军进入代州也给了各郡县不少压力。

在林彤扶柩返回代郡之后,李榷很想催促林彤去忻州觐见雍帝。但是他又不敢犯了众怒,如今蛮人已退,代州各地得知林远霆战死的噩耗,都是纷纷前来吊唁哭祭,代州一地放眼望去,满目都是孝衣如雪,这种情形下李榷怎敢催逼林彤。安庆长公主得知丈夫和爱子战死的消息,再加上雍军入境,所以一病不起,林远崇已经可以扶杖而行,以长辈身份主持丧仪,林澄仪和林彤、赤骥都在守灵,众人都下意识地将觐见雍帝之事抛到脑后,就是赤骥,也不愿当真去面对李贽,谁知道最后会如何处置林家呢?在这种情形下,李榷也只能无可奈何地回报给雍帝,等候谕旨行事。

五月十四日,拖着沉重的脚步,走向灵堂,赤骥越发觉得疲乏,丧仪本就十分繁复,何况林远霆身份尊贵,种种礼节更是不能轻忽,林氏兄妹都不擅长处理各种琐事,只有赤骥熟稔外务,他只能以女婿的身份四处奔走,反而是林澄仪和林彤,除了在灵堂守孝跪灵,接待前来吊唁的宾客之外,没有更多的事情要做。方才有军士前来禀报,说是驻扎在代郡之外的雍军突然有了异动,赤骥苦笑,如今难道还有什么法子对付强大的雍军铁骑么,再说就是有法子,难道自己还能和大雍为敌不成。

走入灵堂,只见容色憔悴的林彤怔怔地望着堂前的灵柩和牌位,林澄仪则是木无表情地跪在上首,堂下都是代州军仍然存活下来的将领和代郡的官员,各郡县来吊唁的军民几乎都已经祭拜过了,这两日灵堂已经不再那么忙碌了。这些将领和官员都在下面窃窃私语,有些事情终究是要面对的,可是却无人能够忍心去和林氏兄妹说及此事。赤骥微微一叹,走到林彤身边,柔声道:“彤儿,你这些日子太辛苦了,到后面休息一下吧。”林彤抬起头来,眼中闪过悲色,道:“骥郎,明日我就带着众将去忻州觐见,正式递上降表,答应父亲的事情,我不会反悔,你也不用担心我会和大雍为敌,无论如何,代州能够守住,都有雍军的功劳。”

赤骥没有说话,只是轻轻拍了拍林彤的香肩,他能够说什么呢,即使明知这少女说出这番话时心痛如死,却也只能看着瞧着。

正在灵堂上众人听闻林彤的话语,都在黯然神伤的时候,门外有军士来报,说是有客人前来吊唁,林彤皱眉道:“不是早就有令么,凡是前来吊唁的皆可直接入内。”那军士道:“启禀郡主,来人不是我们代州人,属下见他们颇不寻常。”林彤淡漠的一笑,道:“怕甚么,难道现在我们还有什么顾忌么,请客人进来吧。”军士唯唯应诺,退了下去,不多时一行人直向灵堂而来。

代州众人都是用目瞧去,设祭已经多日,代州各地凡是有些名望声威的人几乎都已经亲自前来拜祭或者遣人代祭,怎么这时候还有人前来祭灵,目光落到来人身上,人人心中都生出不同寻常之感。来人共有四人,走在最前面的一人身穿素衣,大概三十五、六岁的模样,相貌威武雍容,气度恢宏,大步流星,有龙行虎步之姿,令人不敢正视,而在他身后半步随行的则是一个灰发男子,两鬓星霜,却是相貌儒雅俊秀,素衣儒服,洒脱不群。在两人后面并肩而行的是一个相貌平平的中年人和一个相貌清秀阴柔的少年,皆是穿着青衣,从衣着和位置来看,恰似两个仆从,可是在代州众人看来,那青衣中年人走起路来点尘不惊,双目神光隐隐,一对上他的目光,便觉得五脏六腑似乎都被看透彻了一般,那青衣少年虽然看上去似乎不会武功,但是只是看他一眼,便觉得仿佛数九寒天被人浇了一头冰雪一般浑身冰冷。众人面面相觑,都不知这四人来历,这时堂上传来一声惊呼,众人看去,却是林彤和赤骥双双所发,赤骥神色满是震惊和慌乱,林彤也是满面惊容。

这时,那为首的中年人上香之后,对着灵位行了一揖,他并未下拜行礼,可是不知怎么,代州众人都觉得理所当然,林澄仪、林彤和赤骥也都下拜还礼,只是赤骥神色仍然惶恐,林彤则是珠泪盈眶,神情震动。

然后那素衣书生上香拜祭,还礼之时,赤骥却是退了一步,以示不敢受礼,林彤望了赤骥一眼,轻叹一声,也是退了一步,和赤骥双双还礼。代州众人几乎都已经知道赤骥身份,心中均涌起一个不可思议的念头,望向两位前来吊唁的客人的眼神也变得惊疑不定。

这时,两个青衣人也依例拜祭,礼毕之后,那为首的中年人长叹道:“朕素闻代州林氏世代镇守边关,勇烈无双,只可惜晚了一步,不能亲见林老将军一面,今日亲来拜祭,也是稍减心中遗憾之意,少将军和郡主尚请节哀,今后朕尚需倚重林家镇守代州。”堂上众人无不哗然,竟然是大雍之主李贽亲来吊唁,如今代州已经落入雍军掌握,人为刀俎,我为鱼肉,想不到李贽竟然如此礼敬林家,怎不令众人感激涕零。也有人目光落到那灰发青年身上,青年华发,气度闲雅,又得赤骥、林彤如此礼重,除了楚乡侯江哲还会是何人。既然知道李贽和江哲两人身份,不用问也知道那两个青衣人必是随行的高手,而那相貌阴柔秀雅的少年,多半就是天下闻名的邪影李顺。

既然已经得知来人身份,众人都望向林彤,雍帝亲临,如今林彤乃是代州主将,理应上前叩见以示忠诚,只有这样,才算是正式归降大雍,可是林彤年轻气盛,人人都担忧她不肯屈膝请降,若是惹怒雍帝,只恐林家将要遭遇覆顶之灾。不料林彤神色冷静非常,膝行上前一步道:“陛下白衣吊唁,林氏满门皆感激不尽,父亲遗命臣等归降大雍,罪臣林彤暂代主将之职,今日便在父亲灵前立誓,代州军民从此归顺,绝无异心,只是两位兄长和姐姐尚在晋阳,他们尚不知此事,罪臣也不能勉强兄姐行事,尚请陛下恕罪。且家母身份不同,如果陛下有意加罪,林彤自请代母承受。”

众人听林彤如此说,虽然是实情,却都心中不安,担心雍帝震怒,李贽却是微微一笑,道:“嘉平公主亦是巾帼英杰,代州军陷于晋阳者,朕自有处置,林卿不必忧心。至于令堂,虽然是北汉长公主,然而与军国大事并无关联,且是林侯遗孀,朕岂会无端加罪。”到了此时,林彤方觉得浑身一松,诚心诚意的叩首道:“陛下宽宏大量,臣林彤率代州将士,叩见皇帝陛下,万岁万万岁。”众人皆拜,行了三拜九叩大礼,不多时,消息传出灵堂,只听见外面代州军民皆呼“万岁”,声音惊天动地,由近及远,初时还只有林府附近的军民高呼,到了后来,满城皆是呼声,声音直入云霄,直到此刻,仍然在代郡之外严阵以待的雍军将领们,才终于放下了心中大石。至此,代州终于彻底降了大雍。

赤骥只觉得多日紧张的神经终于松懈下来,想起当日辞别公子前来代州之事,几乎是恍若隔世,想不到自己竟然活了下来,代州林家也没有遭到雍军清洗,自己和林彤居然顺利地成了夫妻,令他有一种如在梦幻中的感觉。忍不住向江哲望去,一触到那双温和沉静的幽深双眸,赤骥觉察得到江哲的目光中透着的丝丝暖意和赞赏亲切之意,热泪忍不住滚滚而下。

五月二十日,代州遣使入晋阳,其时晋阳已经被雍军四面围困,林碧得知父亲战死的消息,哭拜于地,代州军三军缟素,后主刘佑下旨亲设灵堂,遥祭英灵。其后,林澄山、林澄渊奉了林碧将令,率代州军出城归降雍军,北汉朝中有人言欲不许代州军出城,以免乱了军心民心,被后主所阻,代州军顺利出城,林碧则辞去代州军主将之职,留在晋阳,欲与晋阳共存亡。

雍军围城不攻,至六月十五日,雍帝五次遣使入城说降,许以保全北汉王室宗庙,其时北汉唯有晋阳尚存,军民困守其中,虽有林碧主持军务,然雍军无机可乘,且代州已降,北汉军上下皆疑其终将降雍。后主询问重臣,皆无以答对,乃问计国师京无极于兰台,两人密谈终宵,余人皆不能与闻。

六月十八日,后主遣使递降表至雍营,翌日,携宗室百官,白衣出降,至此北汉亡国,享国二十四年。李贽下诏,赐封后主为永定郡王,送回雍都安置,北汉宗室皆降爵迁至雍都,唯嘉平公主林碧,李贽嘉许其忠贞善战,仍赐封公主。代州林氏,林远霆所殁,仍赐封代郡侯,令其长子林澄仪袭爵,令其女红霞郡主林彤掌代州将军印,镇守雁门。

其后李贽任宣松为晋阳节度使,擢布衣赵梁为晋阳令佐之,又在晋阳新立平北军,荆迟为主将,统军二十万,节略原北汉各州郡,且受宣松节制,北地略平,大雍朝臣多次上书,催促李贽还朝,七月初二,李贽班师返回长安,齐王李显、嘉平公主林碧、楚乡侯江哲皆随驾西入长安。

御辇之上,李贽举杯笑道:“随云,多年不见,你的棋艺毫无长进啊。”

我看看七零八落的棋局,耸耸肩道:“臣的棋艺不是没有进步,只是陛下的棋艺越发精湛了。陛下这次和齐王殿下想必已经是前嫌尽逝,不知道臣提及的喜事陛下如何看待?”

李贽笑道:“若是六弟真有这个本事,朕为其主婚就是,总之不能委屈了碧公主,倒是赤骥和林彤的婚事朕没有想到,此子是你门下俊杰,居然舍得抛弃青云之路,去和小郡主同生共死,还得到林远霆亲自允婚,有他在代州,朕也放心许多,林家纵然桀骜不逊,朕也有了拴马的笼头。”

我淡淡道:“这是赤骥用自己的性命换来的,当日我虽放他离去,心中却不是不恼怒,不过总算他还是心里有我这个主子,所以就给了他一个机会,若是他死在雁门,自然也就算了,若有重逢之日,我就成全他的苦恋。否则,就算他已经是代侯的女婿,我要取他性命也不过是易如反掌。”

李贽瞧了我一眼,摇头道:“你就别嘴硬了,你上书给朕说什么让朕坐视代州苦战,不就是想激朕快些决定救援代州么,你给李榷的信是怎么回事,只怕你比谁都担心赤骥的安危,让他在雁门苦战,不过是给他一个博取美人芳心的机会罢了,总算这小子够胆量,没有辜负了你的期望,朕已经封了他将军之位,就让他在代州给朕看守边关吧。”

我赧然一笑,不再多言。

李贽将御酒倒了一杯,递给我道:“随云,全凭你苦心孤诣,让北汉王室失去了最后的依靠,不得不请降于朕,若是最后真得凭着血战夺取晋阳,不仅我军损失惨重,数十年之内,晋阳也难以恢复元气,如今北汉降服,大雍尽得其士卒钱粮,只需数年养精蓄锐,就可以南下攻楚,卿功莫大焉,请满饮此杯。”

我接过御酒一饮而尽,笑道:“皇上,北汉已经平了,东海的降书已经到了朝廷,南下攻楚之事也用不到微臣,是不是允许臣暂回东海休养一段时日呢?”

李贽闻言,板着脸道:“这可不行,不说朕绝不许你离朝而去,难道你和长乐结缡数年,还不去拜见岳父岳母么,太后正等着你前去拜见呢,她总担心你身子不好,担心长乐吃苦,不见一见你绝不肯放心,至于父皇么,我离京之时,已经被柔蓝那丫头甜言蜜语哄得心软了,决定不再怪罪你了,你若是错过今次,可别想让父皇接纳你了。再说,你不想见见长乐、柔蓝和慎儿么,父皇和母后可是一个都不肯放的,除非你肯独自一个回东海去,否则这辈子你别想离开长安。”

我苦着脸,最后的希望随风飘去,想想我那舒适恬静的静海山庄,真是可惜啊。

见我脸色苦闷,李贽也觉得不忍,正想安慰几句,这时候外面传来匆匆的脚步声,有人在窗外诚惶诚恐地禀道:“陛下,有八百里加急军情。”

我和李贽都是眉头一皱,李贽接过文书,只看了一眼,便发出叹息之声,深深地看了我一眼,道:“随云,你的弟子没有一个是善与之辈。”

我心中一震,这是什么意思,连忙抢过情报一看,也不由发出苦笑,这上面写的很清楚,六月二十七日,陆灿轻骑夺取葭萌关,从此东川和蜀中之间的门户已经落入南楚掌握,想要攻打南楚,一是从蜀中顺江而下,一是渡江作战,如今荆襄之地已经固若金汤,长江天险又为双方共有,陆灿这小子够厉害,表面上被尚维钧压制得什么都不能做,却趁着大雍疏忽之时突然进军东川,这小子定是勾结了庆王余孽,才能兵不血刃地攻下葭萌关,如今南楚稳稳占据了半壁江南,天下一统遥遥无期,我什么时候才能归隐林泉啊。

忍不住深深的叹口气,我举起酒杯,缓缓饮下清冽的御酒,目光透过薄薄的纱幕,看向御辇之外的广阔天地,天下事每每不能尽如人意,我又何必为此烦恼呢?

在写完第三部的时候我曾经说过情节已经告一段落,如今这句话我又要再说一遍,北汉烽烟平息,东川叛变结束,东海也即将归顺大雍,至此天下已经成了南北对峙的格局,当我写完第三部的时候,真的考虑过能不能继续写好后面的章节,第四部、第五部的写作可以说是很艰难的,人的惰性和写作上的困难都让我曾经却步,速度也慢了许多,唯一可以告慰的就是还是保持了比较稳定的更新频次,无论如何总算完成了这两部五十万字。

说句实话,这两部有些这样那样的问题,很多人都批评的就是主角的戏经常被其他的角色抢走了,可是这也是没有办法的办法,主角的文弱造成了他不可能东奔西跑的窘局,所以我只能将主角的构想由他的手下来实现,他的手下可以说是受了他的熏陶,从他们的手段就可以表现出江哲的厉害之处,尤其是八骏,他们可以说是江哲的直系弟子,他们的行为更能够表现出江哲的才华和性格。

而且这两部我主要想表现的是英雄和英雄的对决,无论是北汉还是大雍,都没有什么错误,只是他们存在于一个时代,我不认为北汉的龙庭飞、林碧、段无敌、谭忌、段凌霄、秋玉飞等人会因为大雍的强大和优势就放弃抵抗,明知是悲剧,但是他们决不会轻易妥协,所以我在这里做了“杀人的随波”,龙庭飞、谭忌、石英被我一一送入黄泉,而段凌霄、秋玉飞、段无敌虽然没有死,可是他们经受的也是如同烈火焚身的苦痛。还有一些出场不多的人物,我也都尽情地描写了他们的死亡。

例如,壶关的守将刘万利,他是一个比较平庸的人才,甚至在他殉国的时候,还在牵挂妻儿,在“兵出壶关”和“烈火焚城”这两章,我描写了刘万利的挣扎和刘夫人的坚贞,在*的那一幕,我特意写了刘夫人的冷静和刘万利的软弱,就是想写一个普通的将领的殉国,我不认为所有人都是像龙庭飞、谭忌那样能够慷慨赴死的勇士,还有一些人或者没有多少闪光的亮点,甚至也会软弱,也会害怕,可是就是这样,他们的殉国才会令我更加感动,而刘夫人的形象则是一个反照,我最不喜欢例如慈航静斋那种仙子形象的人物,所以我写了凤仪门,可是我又认为世间有很多女子不惭须眉,刘夫人就是其中之一。

还有刺杀荆迟的魔宗刺客戴钥、战死在雁门关的林远霆,他们都是出场很少的人物,可是他们才是北汉的脊梁所在,所以我也不吝于笔墨。

而且在这两部里面,我塑造了苏青的形象,其实原本并没有这个人物的构想,只是我写完了前三部之后,唯一的遗憾就是闻紫烟,那个并不美丽,却有着勇烈气质的女子,感叹她必死的命运,所以我写了苏青,闻紫烟的弟子,也从侧面描写了闻紫烟对于自己的愚忠并非没有挣扎,只是她不能背叛给了自己一切的师门,我从来不喜欢大义灭亲的做法,有时候我觉得大义灭亲更是心性凉薄的同义词,所以闻紫烟死了,可是她留下了一个苏青,我一直觉得苏青的纵横疆场,快意恩仇,这才是闻紫烟的理想。不过如今也给读者留下了疑问,苏青究竟*,坦白的说,我还没有想法,除了绝对不会让她嫁给某人做妾之外,大家不妨讨论一下,嘻嘻,我不保证会令大家如愿以偿啊。

总的来说,这两部我写了很多各种各样的人物,大雍和北汉的战争实在太残酷了,因为陨落的都是英雄,可是没有这样的血战,就没有一统天下的可能。

最后,或者北汉战场的匆匆落幕可能会令很多人失望,觉得应该再写一些例如水淹晋阳的大战,可是我从一开始就没有打算过这样写,不论是庆王那场可笑的叛乱,还是北汉的灭亡,我体现的正是江哲的战争思想,破国为下,全国为上。所以江哲令庆王得意忘形地发起了叛乱,然后在最高处陨落,将庆王和蜀国的反对势力连根拔起,所以江哲将北汉国主麾下的名将和支撑势力一一拔除,龙庭飞等将领死的死,走的走,主力被雍军击溃,军队再无胜利的信心,安抚魔宗,让京无极没有决死之心,代州的投降,让北汉国主唯一的依靠林碧失去了战意,终于如同压垮骆驼的最后一根稻草一样,让北汉失去了所有希望,所以最后北汉王室的请降也就合情合理了。善战者无赫赫之功,希望我表现出来了这种思想。

接下来的南楚之战,其实可能几章就结束,也可能拖了十几章,甚至二三十章,我也不敢确定,但是绝对不会超过三十章,其实是否写到这里结束呢?无论如何,我要休息了,放松放松,好好构思一下下面如何写,或者是写还是不写。所以下周就不会有更新了,当初第三部完成,我休息了两周,这次只休息一周,应该不算什么吧?再见,一直支持我到现在的读者们,如果有什么意见,请在书评区发表,我会时常上来看看的。
