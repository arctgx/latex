\part{第六部 天长地久}

\chapter{第一章 少年不知愁}

“春桥南望水溶溶,一桁晴山倒碧峰。秦苑落花零露湿,灞陵新酒拨醅浓。青龙夭矫盘双阙,丹凤褵渉隔九重。万古行人离别地,不堪吟罢夕阳钟。”

大雍隆盛七年甲申,仲春时分,春意融融,风和日丽,通往长安的驿道上车马如流,络绎不绝,往来客商何止千万,自从隆盛元年北方一统之后,便和南楚议和,双方划江而止,虽然暗流汹涌,双方并不因为表面的和平松懈,可是毕竟还是过了七年的太平日子,大雍朝政清明,政通人和,国力蒸蒸日上,长安也越发繁华,尤其是这几年大雍致力于西域商道的开拓,尤其是几条驿道的修建更是方便了各地的商旅,长安已经成为天下的商业中心。

在络绎不绝的商旅中,有一支并不显眼的小商队不紧不慢地赶着路,这支商队是由一些小商旅临时组成的,长路漫漫,再加上大雍统一北方不久,难免会有一些盗匪出没,所以结伴而行,也图个平安。这支商队主事的是一个宋姓商人宋俭,他四十出头年纪,在大江南北奔波行商多年,精明能干,性情豪爽,所以被众人推举出来主事。看到灞岸隐隐约约的柳色,他举鞭指着前方兴奋地道:“伙计们,前面就是灞桥了,咱们赶一赶,今天日暮之前就可以到栈中休息了。”这些商旅都是十分兴奋,也都随声应和着,他们在长安都有固定的合作商栈,只要到了商栈,自会有人帮助他们安顿,眼看目的地就要到了,就是最沉稳的人也不免有些激动。其中有一个有一个十三、四岁的少年最是兴奋,两眼放光地望着前方的烟尘。

宋俭见状不由微微一笑,这个少年叫做云路,并非是商旅,而是在路上遇见到旅人,当日他们贪赶路程,在途中遇见山贼,虽然商队中也有保镖打手,可是那些山贼仗着弓箭封住了道路,正在危急之时,这个少年骑马经过相助他们击退了山贼。这个少年年纪虽然不大,可是如同乳虎一般的身躯力量无穷,居然可以使用三石的强弓,箭法惊绝,连珠七箭,射杀了数名悍贼。逐走贼人之后,众人得知这个少年是要北上到长安寻亲,便在他的要求下带他同行,反正多带一个人并不费什么事情,而且这个少年的箭术还可以派上用场。一路上这少年跟前跟后,十分勤快伶俐,性情又是开朗明快,虽然只有月余时间,却已经成了商队中最受欢迎的人物。

不过宋俭毕竟是世事练达,他早已看出这少年不同寻常之处,虽然这少年颇为聪明能干,又能够吃苦耐劳,可是从他初时经常犯些小错误来看,明显是没有做过这些事情的,而且他手足上虽有老茧,可是却像是练武所致,而且他虽然年少,却是通晓文字,虽然一看就是初次出门的雏儿,可是一路上自己为他指点沿途风物,只需三言两语,他就了然,甚至还能追根究底地提出一些详细的问题,若不是这少年年纪轻轻,自己倒要怀疑这少年是南楚派去大雍的秘谍了。不过看着这个少年好奇地神情,宋俭笑了笑,南楚就是再无人,也不会派这样一个小孩子去探听军情吧,多半是哪个世家的子弟离家出走吧,而且见这少年文武两途都有些成就,他的家世一定不凡。不过这些事情也不用他们操心,只要这个少年不是谍探,就不会影响到他们的生意。

望着的灞岸风光,云路心中十分欢喜,那是长途跋涉之后,终于到了目的地的喜悦,可是一种说不清道不明的情绪让他几乎忍不住叹息出声。自幼生长在江南繁华之地,看惯了吴风楚月,草长莺飞的江南风光,一路北上,却见北地春光也是旖旎动人,且更有一种奋发向上的生机,两地*或者是不相上下,可是比起江南春雨中一步三叹、伤春感怀的书生,他倒是更喜欢那些在北地春风中纵马驰骋的少年豪杰。一路上接过的城镇乡村无数,云路总觉得这些大雍人豪迈武勇,或许他们的生活不比江南人安逸,可是他们神情中却有着强烈的自信和傲然。怪不得父亲每每感叹不已,每次提及北方的强敌便嗟叹不已,明明才三十多岁的年纪,却已经鬓生华发。自己以前总在奇怪,为什么在南楚有着数一数二的权势地位,凭一己之力不让雍人南下牧马的父亲,私下里却总是愁眉不展,江南虽然富足安逸,却是军民贪安,若是对上厉兵秣马的大雍,必然是一场苦战。想起建业城里刀枪都已经生锈的禁军,再想想一路上看到的大雍各地驻军和乡兵团练,这些应该只是大雍二三流的军事力量,若论武力已经在南楚大部分军队之上。比较起来,大概只有父亲和镇守荆襄的容将军、镇守葭萌关的余将军麾下的军队才可以和大雍对敌,也难怪父亲虽然和那个老狐狸不合,却在和大雍议和之事上面始终意见一致。

云路真正的身份乃是南楚大将军陆灿长子陆云,当年陆灿虽然顽皮捣蛋,可是对于婚姻大事却是毫无自主之权,十八岁就奉命完婚,翌年就生下了陆云,十四年之内,已经有了三子一女,当然陆灿最为钟爱的就是长子陆云,陆云不论是相貌性情和父亲几乎是一个模子里面出来的,虽然生于繁华锦绣当中,却是最爱弓马刀枪,几乎是刚学会走路就跟着家将学习武艺,十岁出头就可以箭射猛兽,枪挑盗匪,是有名的将门虎子。

像他这样的身份,本不应该偷偷潜来大雍,这次离家出走却是刺杀一人,说起来自从隆盛元年(同泰十一年)陆灿趁着大雍北汉缠战,庆王叛乱刚被平息,东川人心混乱之际,袭取葭萌关之后,陆灿在南楚已经成了名实相符的军方领袖,就是权倾朝野的尚维钧也要顾忌他三分。南楚朝中那些争权夺利的小人见正面不能撼动陆灿的地位,便百般从侧面攻击陆灿,而陆灿曾在江哲门下受教的事实就成了最好的把柄。

曾经为南楚翰林,却投降大雍,又迎娶了曾为南楚王后的长乐公主,这样不忠不义的江哲早已成了南楚朝野攻讦的对象,在有心人的挑拨下,江南士子就是酒酣耳热之后,也不免骂几句贰臣贼子江随云,而身为江哲弟子,且从来不曾当中宣称和江哲割袍断义的陆灿也不免遭到池鱼之秧。虽然因着陆灿捍卫社稷的功劳,以及他手中的军权,还无人敢当年指斥,可是暗地里还是诽谤不断,甚至还曾有狂生上门投书,劝谏陆灿“大义灭亲”。这样的情形持续了很长时间,即使江哲如今已经是大雍朝廷的重臣,堂堂的郡侯,驸马都尉,深得雍帝李贽信重,也不能消灭南楚对他的责难风浪。而陆云无论如何也不能理解,为什么父亲宁可受人议论指斥,也不肯和那人割袍断义,甚至直到如今,仍然每年遣使前去问安,纵然那人在大雍权高位重,也不应如此委屈苟且啊。

强烈的不满本已沉积在陆云心中,在今年新春华旦,陆云随着父亲入宫参加宴会,却在花园中被尚维钧的长孙尚文带着几个臭味相投的豪门子弟围住,当着他的面辱骂他的父亲私通大雍,陆云大怒之下将这几个纨绔子弟打得头破血流,这下可惹了大祸。当陆灿责问他的时候,他只是沉默不语,被陆灿用家法责罚,躺在床上养了半个月的伤,又被禁足闭门思过。可是陆云生性勇烈,想到若是自己去刺杀了江哲,那么就无人可以责备父亲了。所以趁着父亲去巡视长江防务离家出走。他年纪小,平日陆灿管束又严,所以认得他的人不多,竟然被他混过了重重关卡,一路北上到了长安。看着遥遥可望的长安城,他心中又是激动又是慌乱,如何能够在重重护卫下刺杀那个叛国的逆贼,为自己的父亲洗清污名呢,而且绝对不可被人知道自己的身份,就是再无知,他也知道刺杀堂堂的大雍驸马,雍帝重臣,会掀起什么风浪,他不想连累父亲,或者效仿古时的聂政一般,行刺成功就毁容自尽,就让陆云这个人消失得无影无踪吧。狠狠地握住双拳,陆云策马跟着商队向长安走去。

刚刚过了灞桥,正当满心杀机的陆云也沉醉在明媚的春光中的时候,突然后面传来了急促的马蹄声,陆云曾经在父亲训练骑兵的时候旁观,一听便知道这是训练有素的骑兵在奔驰,而且从整齐有力的马蹄声可以听出,这是一支十分精锐的骑兵,就是父亲麾下最精锐的骑兵也不过如此,忍不住回头一看。只见远处一支衣甲杂乱不齐的骑兵飞驰而来,陆云忍不住吸了口气,这次骑兵气势汹汹,如狼似虎,虽然衣甲各异,可是却都是上好的精铁战甲,只见他们的姿势就知道这是一支经过千锤百炼的骑兵。陆云定睛看去,只见这只骑兵最前面的一人执着风行旗,火焰一般的旗帜上面有一个鲜明的“林”字。

陆云和商队众人退到路边,几乎是转瞬之间,这支骑兵就已经从身边疾驰而过,陆云看的清清楚楚,被众人簇拥在中间的是一对青年男女,男子身穿青色便装,大概是二十八九岁的年纪,相貌相貌俊朗,面上带些风霜之色,可是眉宇间带着儒雅的气息,而那女子大概是二十五六岁的年纪,一身火红的劲装大氅,身佩长弓白羽箭,娇艳如花,气势如火,明丽妩媚中带着飒爽英气。在双方擦肩而过的时候,那个青年男子似乎无意中目光一转,落到了陆云身上,似乎微微一怔,陆云心中一震,那个男子的目光温文中有一种不可言表的威严,周身上下带着隐而不显的杀气,这是出色的将领才有的气质。似乎是感觉到那个男子的分神,那个女子也随之一瞥,陆云再次觉得震撼,那个女子的气势更加凌人,那是统领千军万马的气度威严。

转瞬之间,那支骑兵已经远去了,可是留给陆云却是深深的震惊,难道大雍的将领都是这样的风采么,难怪父亲会因此愁眉不展了。

这时,耳边传来同伴的议论声。

“原来红霞郡主也到了长安了,一定是来祝寿的,太上皇过世已经好几年了,这次是皇上四十五岁大寿,长安传来的消息都说要大举庆祝,难怪代州也派了使者过来祝寿。”

陆云心里想着这位红霞郡主是什么人,却是一时想不起来,忍不住问宋俭道:“宋大叔,这位红霞郡主是什么人啊,怎么看上去如此威风凛凛?”

宋俭笑道:“小路,你没有来过大雍,不知道,这大雍朝廷和咱们南楚不同,女子也可以上阵杀敌,方才过去的那一位是代州将军林彤,她原来是北汉的红霞郡主,代州归降大雍之后,雍帝对林家十分礼遇,仍然保留了她的郡主名位。这位郡主可不简单,当年带着代州军死守雁门,战到最后一兵一卒,死也不退,林老将军阵亡之后,她遵从父命投了大雍,现在虽然林家的家主是代郡侯林澄仪,但是代州军民都只遵从红霞郡主的命令。她旁边那人想必就是郡马王骥将军,王将军本来也是咱们南楚人,他是楚郡侯的门人,跟着江侯爷到了大雍,和这位红霞郡主在东海一见钟情,只可惜各为其主,只能鸳鸯折翼。后来大雍和北汉交战,蛮人却又趁机入侵雁门,这位王将军得知心上人在雁门死战,便抛弃一切去了代州和郡主同生共死,后来林老侯爷在决战之前给他们在阵前完婚,原本王将军是准备和红霞郡主一起战死的,幸好大雍皇上器量宽宏,及时派去援军,要不然他们恐怕就死在雁门关了。”

陆云听得出神,道:“怪不得这样的气度,原来是抵御蛮人的名将,我听说这些年大雍每年都要派军到蛮人草原上面作战猎杀,想必就是红霞郡主和王骥将军主持,怪不得他们身上带着这样浓厚的霸气杀机。”

宋俭点头道:“说起来,大雍的女将军可不少呢。不说别人,这位红霞郡主的长姐嘉平公主,那可是和宁国长乐长公主齐名的女中俊杰,一文一武,都是只手可以撼动朝野的人物。当初嘉平公主配合龙将军和大雍作战,将大雍多少能征善战的名将都打得落花流水,当初大雍四十万大军围困,还让这位公主殿下杀出了重围。大雍人都说,当初皇上定要招降林家,对北汉王室又是如此礼遇,多半也是看在这位公主殿下的面上。你知道么,听说当年龙大将军自尽之前,向齐王殿下托付后事,后来此事传得沸沸扬扬,齐王殿下也是对嘉平公主十分倾慕,可是这位公主殿下就是不肯答允。还是这位齐王爷苦苦追求了两三年,终于感动了公主殿下,点头允婚。三年前,嘉平公主和齐王殿下大婚之时,雍帝赐婚,太上皇和永定郡王,就是原来的北汉国主亲自主婚,那可是轰动了大江南北的盛况啊。大雍皇室、朝廷的所有重臣全部参加了不说,原来北汉的许多重臣、将领也都前来参加婚宴。北汉的民风就是这样强悍,当初北汉灭国之后,这些人不是解甲归田,就是弃官归隐,都不肯屈膝事敌,可是那场婚宴之后,这些人都纷纷重新投入军旅了。”

陆云面色有些沉重,这件事情他却是知道的,当初父亲得知此事后,曾经长叹不已,当日他还不明白,如今听到宋俭这样说才想通了,齐王和嘉平公主的婚姻,代表着大雍和北汉上层的融合,大雍国事鼎盛,对南楚自然是雪上加霜,难怪父亲要担忧不已了。而且齐王殿下本已经是父亲的劲敌,再加上这位嘉平公主,父亲就更加吃力了,更何况还有那位和父亲隔江对峙多年的裴云裴将军呢。陆云一点也不怀疑嘉平公主的本事,不说那种种传闻,只见她的幼妹红霞郡主如此英姿飒爽,就知道嘉平公主必然更加出色。

这时,宋俭又道:“云路,若是到了长安,你可能还会见到另一位传奇人物呢,就是澄侯苏青,这位苏将军本来也是北汉人,不过她为了报家仇投靠了大雍,在北汉做了多年的谍探,据说立下无数奇功,不过后来她身份泄露,竟然是凤仪门叛逆之后,据说她的师父曾经追杀了大雍皇帝几百里,差点得手。此事传开之后,很多人都说就是大雍皇上再大度,这个苏将军也得被削职为民,谁知道真是天子量大如海,雍帝不仅没有加罪,还赐她侯爵之位,现在这位苏将军是虎赍卫副统领,负责大内禁卫之责,甚得皇上皇后的信赖重用。你看看,这北汉女子当真不寻常,这三人哪一个人都可以翻天覆地,却都投了大雍,这样一看,大雍的文臣武将更加了不得,若非是我们南楚还有陆将军,只怕雍军早就渡江南下了。”

陆云听到此处只能深深叹气,父亲肩上的担子何等沉重,他又有了更深的了解,可是还有人暗中诽谤指责他,自己定要杀了那害得父亲受尽屈辱的江哲,不论他是何等的位高权重。

就在陆云暗自发誓的时候,耳后再次传来迅疾的马蹄声,还有清脆如同银铃一般的笑声随风飘来,陆云忍不住顺着笑声望去,只见另外一条岔路上七骑骏马飞驰而来,陆云看到上面的骑士,忍住揉眼睛的冲动,他瞪大了眼睛仔细看去。

这七匹骏马都是千里挑一的良骥名驹,前面三骑的骑士都是十几岁的少年孩童,后面四骑则是护卫的武士,显然是长安豪门少年游春归来。

中间骑着一匹白马的是一个相貌秀美非常的少年,柳眉杏眼,肌光如雪,穿着一袭淡黄的衣衫,神采飞扬,陆云听到的笑声正是这个黄衣少年发出的。而在这少年左侧一骑的骑士是一个十六七岁的俊秀少年,虽然穿着骑装,却是儒雅斯文,纵然是骑马飞奔,也不带一丝跋扈之气。在那黄衣少年另一侧的黑衣少年则是大不相同,虽然看上去只有十几岁年纪,可是却是面色冰寒,冷峻森严,眉宇间更带着丝丝杀气,令人一见便心惊胆战。

陆云的目光凝滞在那黄衣少年身上,无论如何也不能收回来,这少年仿佛春日里最明媚的阳光一般那样耀眼,他的笑声是如此的欢快,觉察不出一丝的烦恼忧闷,只要看到他,便觉得天地间是那样的宽阔,人生是那样的美好。那样的明快耀眼,让陆云忍不住生出淡淡的嫉妒,自己是怀恨而来,十有八九还会将性命葬送在这里,可是同样的天空之下,却有一个和自己年纪相仿的少年,这样的快乐洒脱。

——————————————————————

在起点的同意下,我提前解禁,虽然公众版的更新不会一周五章,可是我会保持一定的频次,只是尚未决定按照何种规律,大家可以留心一下书评之类的信息。

\chapter{第二章 青梅如豆}

公主自归雍后,随永定郡王西入长安,郡王初时每忧惧朝廷加罪,公主旦夕侍奉不稍离,王乃安。

太宗待公主厚,每召宴,必邀公主至,无论皇室贵胄、文武重臣,有轻慢者皆论罪。然公主英姿端谨,见者无不肃然,莫敢轻也。

时,齐王解兵权,归京参赞军事,倾慕公主忠烈,宛转致意永定郡王,欲求公主为偶,郡王畏其权柄,授意公主允婚,公主怒,仗剑入齐王府,王长跪谢之,近侍告以先龙将军遗言,公主怒稍解,乃弃之去。

——《雍史·嘉平公主列传》

就在陆云痴痴凝望着那黄衣少年的笑黡之时,那三骑骏马已经擦身而过,就在这时,那黑衣少年突然“咦”了一声,猛地勒马收缰,那匹黑色的乌锥马仰首长嘶,居然当时便止住了步伐,可见马是良骥,这黑衣少年的骑术也是十分精绝。旁边两骑却是抢出了几丈之后才停住坐骑,可见骑术逊色许多。倒是后面紧紧跟随的四名护卫,几乎是悄无声息地勒马停住,那几人都是手按刀柄,隐隐护住前面的三个少年。

那黑衣少年高据马上,用马鞭指着陆云问道:“你是什么人?从哪里来?到长安做什么?”

陆云心中一震,不知自己可是露了什么破绽,但是他毕竟是将门虎子,勇气非凡,当下不卑不亢地道:“小可姓云,名叫云路,是南楚人,这次是跟着商队到长安寻亲的。”

这时候,那两骑少年也策马走了过来,陆云趁机仔细打量这三人,方才三人都是策马狂奔,距离颇远,倒是没有看仔细,如今相距不过丈余,陆云已经可以清楚的看到三人相貌体态。

那黄衣少年身量尚未长成,面容秀美,雪肤花貌,仔细看来应该只有十一、二岁的模样,这还是陆云根据他的骑术判断的,毕竟一个若是未满十岁的孩童就有这样的骑术的话,也未免有些惊世骇俗,因为这少年肌肤如同凝脂一般娇嫩,神态又是娇憨动人,就是说他只有九岁或者十岁也是有人会相信的。此刻这黄衣少年把玩着手中淡绿色的精美马鞭,一会儿看看陆云,一会儿看看那黑衣少年,一双乌溜溜的明眸透出强烈的好奇意味。

而在自己面前用怀疑的目光望着自己的黑衣少年,虽然气势汹汹,口气老气横秋,一派可以当家作主的模样,但是陆云仔细看去,这少年相貌颇为稚嫩,应该和那个黄衣少年年纪仿佛,至少不会比自己更大,只是他眉宇间带着浓厚的煞气阴云,让他神情有些沧桑,再加上他身量颇高,所以显得年纪大些。

而策马站在后面那个骑装少年却最令陆云警惕,那少年看上去十六七岁年纪,相貌平常,气质倒是斯文儒雅,座下的骏马虽然名贵,但是身上的衣衫和手中的马鞭却都是平常之物,无论怎样看去这少年都不过是一个普普通通的少年,可是他却和这两个一见便是出身不凡的少年并骑而行,而且神态自若,毫无一丝怯懦不安的神态。陆云记得,父亲曾经警告自己,这样的人最是危险,定要留心。

那黑衣少年对陆云的回答似乎并不在意,顿了一下,又用马鞭指着陆云背上的弓箭道:“你这是上好的铁胎弓,应该有三石之力,若能使用这样的强弓,就是一个八尺大汉也可以参加军旅了,你真能使用这弓箭么?”

陆云心中一宽,却原来是自己的弓箭引起了这少年的注意,他沉声道:“小可自幼好武,力气还算过得去,勉强可以使用这张铁弓,原本也颇为自傲,只是小可一路走来,见大雍各地都有许多少年勇士在校场上练习弓箭,很多人也可使用这样的强弓,想来倒是小可少见多怪了。”

那黑衣少年听出陆云略带些嘲讽的语气,是在暗示自己不必大惊小怪,他心道,这南楚少年既然敢携带三石强弓防身,可见对自己的力气箭术必然十分自信,大雍少年虽然好武成性,但是这般年纪的武士,在校场使用三石强弓还可以,真得用来作战防身,却是一般都只能使用二石的弓箭,南楚少年若论先天体质,本就不如北人强健,这少年却可轻而易举使用三石强弓,可见身份必定不同寻常。想到这里,他冷冷道:“我见你身份不明,很有可能是南楚奸细,你可随我回府接受盘询,若是你果然身份清白,我自会放了你,若是你身份有鬼,可别怪我处置了你。”

陆云暗自惊心,但是他也是傲气之人,冷冷道:“这位公子未免强词夺理,小可虽然出身草莽,也知道什么是律法,公子年纪轻轻,想必不是官府中人,凭什么要拘禁小可,再说,小可来去明白,公子胡乱加以罪名,莫非大雍就是这样对待他国之人的么?”

那黑衣少年剑眉一轩,道:“你倒是能言善辩,可惜却是寻错了对象,我乃是嘉郡王李麟,如何不能查问于你,你是自己跟我走还是我让人将你擒回王府,若是你敢违命逃走,本王爷便传令让禁军追缉你,到时候就不是这般对你客气了。”

陆云大怒,忍不住握紧双拳,无论自己身份若何,可是这黑衣少年毫无证据就要将自己带回府去,岂不是仗势欺人,转念一想,他想起这少年自报的身份,竟然是一位郡王,虽然不明白他到底是什么身份,但是却是宗室无疑,听他语气对自己虽有疑心,却并不肯定,若是自己得到他的信任,或者会有机会接近楚郡侯江哲吧。

这时,见他怒气冲冲,却敢怒不敢言的模样,那黄衣少年心中一软,开口道:“麟弟,算了吧,他年纪也不比我们大多少,怎会是奸细呢,你不是看人家用的强弓力量大,见猎心喜,想迫他留在你身边做侍卫吧?你若胡作非为,我便去向齐王舅舅告状去,就是舅舅不管你,舅妈也不会放过你。”

陆云心中一动,抬头看去,只见那黑衣少年脸上闪过可疑的红云,别过脸去道:“父王和母妃才不会怪罪我呢,反正他身份确实可疑。”

这时,那黄衣少年大怒,一手叉腰道:“李麟,你若是再这样不听话,我便去寻骏哥哥,让他重重责罚你,要不是我求骏哥哥让你出来,你现在应该陪着骏哥哥读书呢。”

这少年声音清婉,虽然在叉腰怒骂,可是那种娇嗔的动人神态却让陆云觉得心神一荡,竟然是目眩神迷,再也不能移动目光。这时,原本听了那少年叱骂,有些气馁的李麟一眼看到陆云痴迷的神色,心中一团怒火腾的燃起,狠狠一鞭向陆云抽去,陆云心神大乱,全没有防备,那一鞭狠狠地抽在他肩上,刹时衣破血溅,陆云一声痛呼,伸手握住弓臂,怒视那黑衣少年。这时,那几个护卫同时策马上前,虎视耽耽地望着陆云,陆云心中一凛,强压怒火道:“不论你是什么亲王郡王,也未免太欺辱人了。”

李麟见他神色激愤,也不免心中不安,也不由觉得自己有些过分,毕竟自己的同伴相貌气质都是上上之选,这南楚少年不过是多看了几眼,自己又何必生气,可是方才自己也不知怎么就是心头火气,但是无论他如何歉疚,毕竟他的出身性情,不能让他轻易低头道歉。偏偏这时,那黄衣少年见到陆云身上的血迹,叫得惊天动地,说道:“李麟,你太过分了,我要让齐王舅舅禁你的足。”然后那少年跳下坐骑,走到陆云身边,从怀中取出一块帕子,对陆云说道:“你别在意啊,我麟弟就是这样的脾气,他没有什么恶意的。”说罢,从腰间锦囊里面取出一瓶伤药,替陆云裹起伤来。

陆云原本心中徨然,不忍推拒,偏偏一个护卫走近来道:“郡主,还是让属下帮这个小兄弟裹伤吧。”陆云心中一颤,这少年竟是一个小女孩,怪不得相貌如此灵秀娇柔,再想起那个护卫称呼这小女孩作郡主,想必也是大雍皇室之人,心中一团混乱,不知是惊惶还是失望,陆云猛地将黄衣女孩推开,骂道:“不必你猫哭耗子。”那少女被推的一个踉跄,差点跌倒,她自幼受惯娇宠,何曾如此委屈,若非是想替顽劣的“弟弟”道歉,怎会给这陌生的少年裹伤,想不到这少年如此无礼,一时间忍不住珠泪盈盈。李麟原本冷着脸站在一边,想着如何讨好挽回,一见陆云这般无礼,更是怒火难耐,马鞭一指,道:“这小贼竟敢冒犯昭华郡主,给我将他绑了,带回府去治罪。”

陆云原本也正愧疚自己不该这般对待那好心的少女,一听李麟所言,只觉得如同晴空霹雳一般,昭华郡主,这个名字他可是知道的。为了刺杀江哲,他行前偷阅父亲书房的文书,知道楚郡侯江哲有一义女,名唤江柔蓝,甚得皇室爱宠,赐封为昭华郡主,眼前这少女竟然是江哲之女。也就是自己父亲的小师妹,纵然不论师门名份,这少女的父亲乃是南楚叛臣,是自己想要刺杀的仇敌,不知怎么,他心中一片空空落落,就连那两个护卫过来捆绑自己也忘了反抗。

这时李麟又对柔蓝吼道:“看吧,就是你这样心软,这小贼分明是南楚奸细,还有跟他同行这些人,也都给我送到京兆尹去,好好盘问一下,看看他们有没有什么问题?”

这时早已经心中叫苦的宋俭等人只得上前求告道:“郡王爷,我等都是奉公守法的商人,这位小兄弟也实在不是什么奸细,还求郡王爷开恩宽恕。”

李麟冷着脸不理会他们,几个护卫互相看看,无奈地摇摇头,其中一人拿出号角,准备发出警讯召唤附近巡视的禁军。

这时原本被李麟责骂的泪水涟涟的柔蓝高声道:“李麟,你有完没有,若是你再这般胡闹,我就再也不理你,分明是你先挑衅别人,惹得他对我无礼,怎么如今你却变本加厉欺辱人。”

李麟也是大怒,指着柔蓝道:“我是替你出气,你却不领情,他们是你什么人,要你这样费心,莫非就因为他们是南楚人,你便这样留情,可别忘了,姑夫是南楚人,你可不是,你是大雍人。”

柔蓝闻言掩面大哭起来,一边哭一边道:“你,你胡说八道,分明是你不讲理,喜欢摆郡王架子,我不愿你胡作非为,你却骂我,呜呜,以后再也不理你了。”说罢翻身上马,策马就要离开,李麟慌了神,策马拦住柔蓝去路,张口想要道歉,却是众目睽睽,说不出口,只急得汗如雨下。

这时,那个一直在旁边冷眼旁观的少年淡淡道:“别吵了,也不是什么大事,在这里闹小孩子脾气,没的让人笑话。蓝儿,嘉郡王也是想为你出气,不是有心气你,郡王的性子你还不知道么,只要这位小兄弟身份没有问题,是不会随便为难他的,最多委屈他几日,你若不多事,郡王也不会这般恼怒。”柔蓝怔怔地听着,最后低头无语,面上怒色渐渐褪去。

那少年又对李麟说道:“嘉郡王,蓝儿性子和善,不喜欢见你欺辱别人,这也是她当你是手足至亲,长安这么多权贵子弟,你何时见过柔蓝这般多事,去管别人的闲事。”

李麟听后,神情渐渐和缓,低声道:“霍大哥,是我不对,不该见猎心喜,和这人为难。”说罢一挥手,让护卫将陆云放了。

陆云轻揉手腕上的绳子痕迹,仿佛身在梦中一般,这时,那霍姓少年策马上前道:“这位小兄弟,虽然是嘉郡王有些过分,可是你也未免太傲了,虽然说人不能没有骨气,可是你孤身在外,怎可任性,再说我家蓝儿对你始终以礼相待,你也不该迁怒于她。这里是二十两银子,给你养伤压惊,你别拒绝,这是礼数,也是人情,你来长安既然是寻亲,必然有些难处,若是有什么不便,可以去宁国长乐长公主府上寻我,我叫霍琮,皇城你恐怕进不去,只要将口信告诉朱雀门的侍卫就行了。”

陆云心情已经平静下来,虽然不知道这少年是何等身份,他和昭华郡主如此亲密,却又对李麟以郡王相称,而李麟又称他大哥,他的身份越发扑朔迷离,但是既然他住在江哲府上,定和江家有着极深的关系,而且他三言两语就平息了李麟和江柔蓝的争执吵闹,对自己这一番话也是有礼有节,若是自己没有存了歹意,定会怒气全消,就像父亲所说,这样的人当真非常可怕。

他躬身一揖道:“多谢兄台教诲,也是小可不明世事,对郡王爷、郡主多有冒犯,还请三位恕罪,云会在长安多日,若是郡王爷、郡主有所征询,尽管令人传唤小可就是,若有差遣,小可定当效命。”

那霍姓少年目中闪过一缕光芒,笑道:“如此最好不过。”说罢,翻身上马,含笑一揖,这时,李麟已经不耐烦地策马而去,柔蓝紧紧跟随,临行前仍然对陆云一笑,她面上尚有泪痕,但是这一笑却如春花绽放,再也看不出方才的不快。那霍姓少年和几个护卫也是纵马追去。

那些逃过一劫的商人或者抱怨,或者相劝,陆云却都没有放在心上,此刻他心中正在盘算着如何利用今日的偶遇。这几人必然都和江哲有着密切的关系,那嘉郡王李麟一见便是果决狠毒之辈,若是他察觉自己有些异状,恐怕不等到掌握真凭实据,就会将自己囚禁起来严刑逼供,而那个霍琮,恐怕也是心机深沉之人,且不说江哲身边的护卫,只是这两个少年已经让他十分警惕,倒是昭华郡主江柔蓝,她是受尽宠爱的天之骄女,又是这般善良天真,必然不会成为自己的障碍,或许还能成为自己的助力,让自己寻到接近江哲的良机呢。心中这般想着,陆云突然对自己厌憎起来,自己什么时候成了这样阴险的人,竟然要利用那一个少女去刺杀她的父亲。

且不论陆云心中自我谴责,那三个少年少女快马回到皇城,李麟只将柔蓝送到家门口就头也不回的落荒而逃,他可不愿见到柔蓝当着自己的面告状,只需想到姑夫那带着笑意的诡异目光,就让他从心底生出寒意。说起来,自己这位姑夫的性子也真奇怪,明明皇上伯父那般爱重,他却宁愿常年告病隐居在寒园,常常迫得皇伯父和父王去寻他问策,这也罢了,那毕竟是军国大事,他也懒得理会,反正将来也不需要他操心。唯一令李麟难受的是,这个姑夫最大的爱好就是欺负自己的一双子女,江柔蓝和江慎,而且这么多年乐此不疲。如今蓝儿仗着皇后和太子替她撑腰,已经没有那么烦恼,江慎么,小小年纪就知道躲在浮云寺不回家,若是一回家总是往自己家里跑,尤其是妹妹李凝出生之后,这小子更是不愿回家了。可恨的是,姑夫欺负不到自己的儿女,不知怎么又瞄上了自己,每次自己去他那里,都会被他寻个借口戏弄,这次自己气哭了柔蓝,他一定不会放过机会的。想到这里,李麟恨不得从未见过这个姑夫,奇怪,自己当初怎会觉得姑夫和蔼可亲的,定是年少无知的缘故。

————————————————————————

以后周一和周五解禁一章怎么样?

\chapter{第三章 知是故人来}

隆盛四年,公主除孝服,王亲赴永定郡王府拜谒求婚,公主展颜许之,太宗闻之大喜,亲为赐婚。

时,高祖尚称康健,自齐王鳏后,每常忧虑,闻婚事大喜,亲为主婚,于席上执郡王臂曰,两家世为姻亲,乃以端仪公主许永定郡王世子。

端仪公主,高祖十四女,昭仪段氏所出,贤淑沉凝,美姿仪,年十五岁,永定郡王世子刘和,性纯良,淡泊知礼,年十九岁。秦晋既成,刘氏遂安。

——《雍史·嘉平公主列传》

从侧门走进齐王府,李麟将坐骑交给侍卫,正想回自己的住处沐浴更衣,却被侍卫叫住道:“麟殿下,王爷吩咐你一回来便去见他?”

李麟犹豫了一下,对于父亲他始终抱着仰慕和畏惧混杂的感情,而李显如今每日不是忙着朝政,就是围在自己那对孪生弟妹的身边,根本就没有时间管自己,如今召唤自己,莫非自己犯了什么错么。当下他不敢犹豫,匆匆走到内宅的花厅,还没有走近门边,就听到厅内传来爽朗的笑声,正是自己父亲的声音。李麟悄悄走到花厅一侧,透过半开半阖的窗子向内悄悄望去。一瞥之下,李麟的身躯突然僵住了,怎么会这样,坐在自己父亲对面,神态悠闲、星鬓朱颜的不正是姑夫江哲么,两人正在对弈,只见父亲如此开心,大概又在棋盘上杀得姑夫落花流水吧。什么时候这个连上朝都不愿意的楚郡侯会跑到自己家里窜门了,总不会他已经知道自己气哭了柔蓝吧?李麟一边胡思乱想,一边犹豫着是否偷偷溜走,只当自己没有回来,这时,和姑夫形影不离的邪影有意无意地对着窗棂一笑,李麟垂头丧气地发觉自己没有可能偷跑,只能缓缓向花厅的厅门走去。

轻轻一笑,我装作不知窗外李麟正在那里探头探脑,说起来也真惭愧,我自己的儿女都聪明得很,知道如何避免我的欺负。柔蓝是仗着皇后娘娘和太子殿下为她撑腰,不说皇后娘娘当初亲手抚养柔蓝长大,将她当成亲生女儿一样,就是太子殿下,又何尝不将她当成亲妹妹一样看待,太子殿下还罢了,虽然他是储君,但是毕竟我是他姑夫,他也不敢对我失礼,可是皇后娘娘哪里是我惹得起的,若非太上皇已经崩猝,只怕我连教训柔蓝都不敢。至于江慎么,这个惫赖小子不提也罢,一年倒有十个月在和尚庙里面称王称霸也就算了,居然为了躲我,没事就跑到他未来岳父家里骗吃骗喝,尤其是他的小未婚妻李凝出生之后,这小子基本上除了年节是看不到人影了。既然齐王拐跑了我的儿子,自然我要报复回来,李麟这小子比较倒霉,就成了我的开心果。至于李凝的孪生弟弟李卓,如今的齐王世子,我可不敢欺负,齐王妃,嘉平公主林碧的厉害我可是清楚的。当初齐王去永定郡王府求婚,是我撺掇的,林碧仗剑闯入齐王府的时候,我可是也在场的,若不是我给李显出了一个下跪请罪的主意,只怕林碧早就一剑杀了李显,然后自尽谢罪了。若是真得如此,只怕好不容易迫降的北汉就会重新竖起叛旗,想要在数年之内消化北汉的国土和民众,那就是痴人说梦了。幸好我早有准备,借着这个机会说出了龙庭飞的遗愿,总算让林碧消去了怒气,还让齐王有了一个追求佳人的借口和良机。经过三年的苦心孤诣,总算让齐王得偿素愿。

其实也不是我想冒险,这也是无奈之举,北汉王室归降之后,大雍内部不是没有斩草除根的呼声,可是却被李贽否决了,说起来李贽也真是明智大度,北汉王室虽然灭国请降,可是刘氏在北汉的影响已经是根深蒂固,若是刘氏不安,则北汉不安。斩草除根虽然是比较容易的做法,可是后患也是无穷的,不说林家会因此不满,生出叛意,就是那些在北汉请降之后解甲归田的北汉将领战士,还有已经退隐的魔宗,他们都不会因为北汉王室的覆灭而放弃抵抗,反而会让他们不屈不挠地和大雍为难。可是若是任由刘氏的影响力继续存在,对于大雍的皇权也是一个隐患。

最后我另辟蹊径提出了融合北汉王室的计策,既然北汉王室声威显赫,素得民心,那么就将他们融入大雍皇室,凡是刘家的女儿,便娶入皇室做皇妃、王妃,凡是刘家的子孙,就让他们娶皇室的宗女,这样下去,最多三代,刘家就和皇室成了不可分割的血亲,到时候一荣俱荣,一损俱损,他们还反什么,北汉的骄兵悍将难道还会和自己的旧主为难么。纵然两国军民仇恨绵绵,可是只要鼓励他们通婚,让他们的血脉融合在一起,再深的仇恨在血脉相连之后消逝。

而要实现这个计策,最重要的就是齐王和嘉平公主的联姻,齐王率兵灭了北汉,虽然最后收网的是陛下,可是对于北汉人来说,李显才是罪魁祸首,而嘉平公主,身兼代州林氏的精神领袖和北汉王室的支柱两个身份,她又是龙庭飞未过门的妻子,可以说是北汉军方唯一认可的领袖,只有让她嫁入皇室,才能彻底让大雍皇室放心,也让刘氏安心,又能够笼络林氏。可是想要达到这个目的,就不能让林碧有一丝勉强,被迫下嫁和两情相悦可是两种效果。为了这个目的,我在齐王身后出谋划策,终于让林碧点头下嫁,这可比当初我设计灭掉北汉还要艰难,李显枉称风流,在追求林碧的时候,什么拙态都被我看到了,幸好,最后还是如愿以偿。就在李显和林碧的大婚上,太上皇完成了最后一击,将刚刚及笈的端仪公主许配给了永定郡王世子,原本的北汉王储刘和。刘和性情纯良,对于权势并没有什么兴趣,若是北汉尚在,他作为王储实在是有些不大称职的,可是作为永定郡王世子,却最合大雍皇室的心意。这两桩婚事一成,效果立刻就显露出来了,很多原本不肯为大雍效力的将领官员,也都纷纷出仕或者加入军旅,有了北汉勇士的加入,征讨北汉时候受到重创的雍军元气也渐渐恢复。

当然在这其中被我和李贽计算的还有齐王李显,为了追求林碧,李显颇为识相地放弃了军权,无论谁也不能让这样一对夫妻手握军权的。尤其是李显大婚之后,他几乎不再涉足军旅,这让皇上有机会在重整军队的时候将军权全部收回,大雍内部再也不存在可以和皇权对立的力量。虽然对李显有些过分,不过一个愿打,一个愿挨,为了美人放弃江山权力也不是他一个。再说我虽然设计夺了他的兵权,可是他在军中的影响仍在,而且和林碧的婚姻,也给了他最切实的保障,除非是想颠覆社稷,否则不论是谁坐在大雍的皇位上,都不会轻易对他出手。再说,等到征讨南楚的时候,也少不了他一份,能够先后灭掉北汉、南楚,这样的战功无论是谁都应该满足了。

这样平衡的局势被我费尽心机促成,可谓劳苦功高,可是李显也太不讲义气了,林碧尚未河东吼,当今朝中一人之下、万人之上的齐王爷就为了讨好佳人,将我彻底出卖,弄得我现在一见到林碧便有些心虚。唯一庆幸的就是北汉众人没有将龙庭飞等人之死都算在我头上,毕竟对于他们来说,败在一个文弱书生手上总是有些丢面子,所以这个黑锅自然由李贽、李显替我背了,反正无论如何,最后出手的人又不是我。

不过在觉得有些吃亏的同时,我也寻到了出气的法子,就是欺负一下李麟,不过说句心里话,若非我对这小子疼爱怜惜,也不会去戏弄他,毕竟由于我的缘故,他失去母亲,自幼在军旅长大,且随着李凝、李卓的降生,齐王世子的地位也彻底与他无缘,和那些本来就不受重视的兄弟们不同,原本身为嫡子的李麟更加凄惨些。为了弥补这个孩子,我向皇上提议封他一个郡王的爵位。且现在他是太子李骏的伴读,没有意外的话,将来也会是李骏的左膀右臂,这样应该足以补偿他的损失了。

正在我一边品茗一边胡思乱想的时候,李麟已经走了进来,这么长时间,就是乌龟也爬到了,他低着头走进来给李显见礼之后,便要往屋角躲去,我笑道:“麟儿,你躲什么,就不过来给我这个姑夫见礼么?”

李显闻言皱眉道:“麟儿,你这是做什么,一点礼数都不懂。”

我轻摇折扇阻住李显话语,道:“麟儿,你不是犯了什么错,不敢见我吧?”

李麟忙道:“没有,没有,我没有把柔蓝气哭。”一句话出口差点咬了舌头,也不知怎么,一见到江哲似笑非笑的神情,他就心中慌乱。不由偷眼看向两位长辈。

李显一瞪眼道:“什么,你将柔蓝气哭了,怎么回事,还不给我说明白,然后去给我闭门思过,晚饭就不要吃了。”

李麟苦着脸不敢应声,这时我却一笑,道:“我当是什么事,柔蓝那丫头娇纵得很,有人气气她也好,免得让她越发跋扈,六哥你也别跟皇后娘娘一样,将这丫头宠得含在嘴里都怕化了。麟儿,说说是怎么回事,若是这丫头无理取闹,回去我责罚她。”

李麟差点没有落下泪来,幸好不是柔蓝的过错,若是被江哲抓住机会责罚了柔蓝,只怕事后自己就要受家法了,然后可能还会被皇后娘娘训斥一顿,最后么,八成太子堂兄大概就会把自己拘在身边十天半月了,在皇宫里面,处处都是规矩,别提多闷了,自己可受不了。看着江哲虎视眈眈的目光,李麟连忙将今日的事情避重就轻地说了一遍。

李显听后眉头一皱,他倒不是责怪李麟仗势欺人,反正他也知道李麟不会太过分,最多也就是给那南楚少年一点苦头吃罢了,他少年之时比李麟还要霸道嚣张呢,他若有所思地道:“你说这少年十三四岁模样,可以使用三石强弓,若论弓箭,就是最擅长骑射的代州,这也是千里挑一的了,不知道他箭术怎么样?这也难怪你留意,麟儿,替我传令下去,将那个少年给我带回来,我要试试他的身手。”

听到这里我不由一笑,有其父必有其子,慎儿虽然不像我,可是李麟倒是像极了齐王,见李麟就要下去传令,我阻止道:“等一等,这么一件小事,你这堂堂的亲王插手也未免太惊世骇俗了,孩子们的事情就让他们自己解决吧,麟儿,你虽然年少,但是已经是朝廷钦封的嘉郡王,这件事情就交给你了,只是不许你草菅人命,如何处置你自己作主吧。”

李麟大喜,他心中仍然念念不忘那南楚少年,只是碍着柔蓝不敢再生是非,如今既然有江哲作主,那么自己就可以为所欲为了,心中痒痒,恨不得立刻就去将那少年擒回府中。

李显见他如此急迫,骂道:“一点定性也没有,急什么,这人既然自称是来寻亲的,难道还会这么快离开么,再说就是他逃了,只要一道军令传下,还怕他逃回南楚么?今天你小姨母他们要来拜见你母妃,今天晚上的家宴,你母妃说了,谁都不许缺席。”

李麟只得凛然遵命,却偷眼看向江哲,这下他可知道为什么姑夫会在这里了,小姨母的仪宾王骥将军是姑夫的门人,若是来到长安,到兵部报到之后一定要先去拜见姑夫,必然是自己的继母想先见到妹妹、妹夫,所以迫着姑夫也到自己家中等候。忍不住低头偷笑,自己的姑夫虽然威风八面,就是在皇上伯父面前也是漫不经心的模样,唯有在嘉平公主面前却是战战兢兢的模样,当真是好笑极了,真想不通当初他是怎么将北汉君臣将士玩弄于股掌之上的。

我此时已经无心理会李麟的小动作,精通箭术,小小年纪可开三石强弓,云路,陆云,哼哼,这样的儿戏手段也想瞒过我的耳目,却不知道他来大雍做什么,但是肯定不是来拜见师祖的,再说听说陆灿对这个长子陆云十分钟爱,想必是那少年自己的主意,我还得知会骅骝一声,让他不要将这少年当成奸细下狱才行。既然这少年已经来了,我也应该尽尽长辈的责任,就让我给他一点小小教训吧,嗯,就让李麟和柔蓝去应付吧,再有霍琮把握大局,应该不会有什么出人意料的变故了。

想起霍琮,我不由露出心满意足的微笑,这个霍琮是我最得意的弟子,将来必定青出于蓝,我心性浮躁,所学博而不精,且虽然有心隐忍,却总是忍不住显露锋芒。而我其他的弟子,各自有着不同的缺点,陆灿心性过于光明忠直,终究会因此受害,荆迟性子粗率,有时冲动难以控制,我虽爱他朴实无华,只可惜终究难成名将,八骏各有所长,但是限于资质经历,虽可独当一面,却不能总揽全局。至于我那双儿女,柔蓝虽然聪明灵慧,如今不过是我刻意让她没有机会面对残酷的现实罢了,一个女孩子,我并不希望她太出色,只想她幸福的度过一生,慎儿么,不提也罢,我的聪明才慧他或许继承了三分,可是我的惫赖懒散却继承了十分,我都替慈真大师觉得惋惜,这样一个糊涂小子,能够担任护法之责么,不过傻人有傻福,他这性子,或许会一生如意呢。

排指算了一遍,只有霍琮才是我最得意的弟子,坚忍不拔,心胸广阔,有自己的主见又能够通权达变,博览群书却又专心经史,最难得是他甘于平淡,擅于隐忍。我不过是被拘禁在富贵荣华中的囚徒,虽然枷锁是人世间种种美好的情谊,却终究是不得自由,而他却是真正能够大隐于朝的隐士,也是唯一可以继承我衣钵的弟子,所以我明明知道他的身份有问题却将他留了下来,一来是爱才,二来这样的人才若不留在身边,可就有些危险了。

这时,齐王身边的四大侍卫之一的陶林匆匆过来道:“禀王爷、江侯爷,郡主和王仪宾到了,公主有请。”

我和李显对视一笑,并肩向王府的银安殿走去,刚刚走入大殿,便见到雍容华贵的嘉平公主拉着林彤的手正在絮絮低语,而赤骥则站在一旁肃手而立,在林碧面前,他始终有些拘谨。一眼看到我,他连忙过来拜倒见礼,口中道:“见到先生容颜如昔,赤骥心中方安,这次途中遇见盗骊,他托我向先生问安。我原本想先去见先生的,不过入城的时候却听萧总管说,先生也在齐王殿下这里。”

我忍不住一阵憋气,这小子怎么哪壶不开提哪壶,嘴角露出一丝阴笑,我笑道:“没什么,今日过来和齐王下棋罢了,赤骥,怎么样,听说你半年前受了伤,如今没事了么?”

林彤闻言忧心忡忡地道:“先生,骥郎他的箭伤虽然痊愈,可是一遇到刮风下雨仍然觉得疼痛,我正想拜托先生替他看看呢?”

我笑道:“无妨,无妨,这是经脉受了损伤,让他到我府上,我给他针灸几次就好了,顺便也将这套针法教给他,若论医术,还是赤骥学得好些,虽然后来转行做了兽医。”心中却暗自想到,我的夺魂金针可是天下无双,除去赤骥的病根绝对没有问题,只不过那套金针本来是用来行刑的,或许会痛一些。当然凭着我的本事,面上自然不会露出一丝破绽。林彤高兴的点头称谢,正在我暗自得意的时候,却见林碧向我淡淡一瞥,目光中带着淡淡的警告,我心中一惊,连忙避开她的目光,暗道,谅赤骥也不敢告诉她们实情。

这时候家宴已经备好,林碧拉着林彤向外走去,李显跟了出去,我见赤骥神情有些古怪,似有隐情要禀报,便故意落后了一步,果然,赤骥在我耳边低声道:“先生,盗骊托我禀告,段将军已经回到中土了,按照先生从前的命令,他已经令人将段将军送往南山别业。这几天应该就会到长安,到时候会有比较详细的信息。”

我心中一震,段无敌么,当年北汉请降之后,我曾想将他招回,谁知他已经出海去了,从此后影踪全无,想不到今日终于回来了,对于这个我颇为歉疚的敌手,我应该如何对待他呢?

\chapter{第四章 射柳金谷园}

嘉郡王麟,齐王显第三子,生母为王正妃秦氏,秦氏因谋逆之罪自尽,郡王遭连坐失爵。时王受命镇泽州,携其从军旅。武威二十七年,郡王随父至东海,见宁国长乐公主,公主怜其无辜,乃携郡王返长安,太宗嘉其有父祖之风,令其为太子伴读。

隆盛五年,齐王妃嘉平公主林碧生子卓,立为世子。太宗以齐王功高,赐封其第三子郡王爵。

——《雍史·嘉郡王列传》

事实上,当接到嘉郡王李麟的帖子的时候,陆云毫不意外,到了长安之后,陆云便设法打探了一下嘉郡王的来历,这件事情并不是什么隐秘,事实上颇为市井中人津津乐道。

嘉郡王李麟,齐王李显第三子,本来是先齐王妃秦铮所出嫡子,显贵无比,只可惜秦铮涉入叛逆之事,虽然自尽谢罪,保全了齐王父子不受牵连,可是子以母贵,李麟这世子之位也是不保了,且齐王原本对于这个嫡子并不关心,所以当时人人以为李麟再无出头之日,不仅他的异母兄弟,就连王府中的奴仆也敢欺凌他。孰料齐王竟对这个儿子重视起来,就连去泽州大营镇守也将他带在身边,几年之后,李麟又在东海遇见了江哲和长乐公主夫妻,这下可是时来运转,随着长乐公主回京之后不久就被皇室重新接纳,成了太子的伴读,这可是青云之路的开端啊。即使在齐王迎娶嘉平公主林碧之后,李麟的地位也没有受到影响,虽然齐王世子为嫡子李卓所有,可是雍帝随即下诏赐封李麟为郡王,这样一来,虽然李麟不能承袭齐王的亲王之位,却也远远胜过那些若无功绩只能封个闲散侯爵的庶出王子,而且如今李麟深得太子器重,将来的仕途必然是一帆风顺,所以李麟虽然年少,却已经成了大雍朝野不能不关注的权贵人物。

不过令长安百姓最是津津乐道地却是这位嘉郡王的独立特行,虽然只有十一岁,在平常人家还是个不懂事的孩子,可是这位郡王却已经名动长安,每日里除了陪伴太子读书之外,就是带着侍卫在长安内外游荡,最喜欢惹是生非,长安亲贵子弟见了他就像老鼠见到猫一样,也有御史谏官上书弹劾,可是皇上闻后却是哈哈大笑,说此子颇有齐王当年的风范,将奏折留中不问,这样一来,长安更是无人敢得罪嘉郡王。幸好这位郡王虽然飞扬跋扈,却是不喜欢欺凌弱小,往往还有抱打不平的举动,所以长安人对他倒并不反感,时间长了,反而觉得嘉郡王脾气虽然不好,心肠却是不坏。

而这位嘉郡王最大的爱好就是招揽武士,若是遇见武艺高强之人,必定想尽办法试探那人的实力,若是出类拔萃的,往往推荐到各军从戎,或者留在身边做侍卫,他年纪虽少,眼光却是十分精准,凡是被他看中的几乎都是俊杰,到了后来,嘉郡王一封荐书比兵部的文书都管用些。所以虽然嘉郡王往往会无事生非的和人为难,有心者却都知道这是良机而非麻烦。

这些事情宋俭等人虽然久在长安,却也不甚清楚,反而是那些长安本地的地头蛇所知甚详,在他们听说陆云冒犯了嘉郡王之后,反而恭喜连连,说只要陆云身份清白,那么很有可能得到晋身良机,不过也有人替他担心,因为嘉郡王虽然平日跋扈飞扬,可是那日的举动还是有些不同寻常。在得知那日和郡王同行的乃是昭华郡主江柔蓝之后,那些人都是神色暧昧,陆云追问了许久,那些人才隐晦地告诉陆云,昭华郡主深受皇室爱宠,据说当朝的太子殿下和嘉郡王都对她言听计从,若是陆云经历多些,自然明白其中含义,可是他有生以来不是在家中读书练武,就是到军营流连,所以听后只觉云里雾里,不明所以。

可是无论如何,陆云却得出结论,嘉郡王绝不会放过自己,不论是好意还是歹意,而自己唯一的办法就是等待嘉郡王出手,若是不幸,自己身份泄漏,自然是有死无生,但若是运气好了,或者可以趁机接近刺杀的目标。所以接到李麟的请贴,陆云虽然为上面命令式的口气以及前来邀请的几个侍卫那种你不去就绑了你去的神情恼怒,却仍然同意前往拜会。

沿着朱雀大街策马而行,两侧的建筑壮丽雄伟,令人目不暇接,陆云却是无心观赏,眼看就要进入大雍的皇城,这让他心中又是激动又是惶恐。路上不时见到来往巡视的禁军,陆云知道这是因为雍帝大寿在即,京城加强了防卫的缘故。到了朱雀门,那些侍卫都是每日进出惯了的,和那些守门的侍卫禁军谈笑风生,却仍需递上令牌核对,陆云心中又是一阵黯然,建业的皇城守卫的松散他可是曾经亲见的。

走入朱雀门之后就是和承天门相连的承天街,街道两侧是三省六部各种衙门,都是禁卫森严,气度恢宏,承天街走到一半,那几个侍卫引着陆云转向东侧,那是景风门大街,穿过景风门走了半晌才进入安兴坊,齐王府占据了安兴坊几乎四分之一的面积,嘉郡王尚未开府,自然仍然住在父亲府中。陆云并不知道,其实凭着李麟的令牌,是可以从皇城的角门直接走胜业坊、崇仁坊之间的街道到达齐王府的。

不过让陆云从这条路进来却是李麟特意安排的。一来,陆云毕竟身份有些不清楚,不想让他接触到那些捷径,二来,也是想通过朱雀大街两侧的森严气氛给陆云一个下马威,顺便看一下陆云的气度,当然在李麟心目中,是针对南楚少年云路,而非是南楚大将军陆灿长子陆云,所以他没有想到陆云虽然也颇有感慨,却丝毫没有受到威慑,因此当他看到陆云神情仍然是那么平静冷漠的时候,也不免有些惊异,毕竟对于一个平民来说,皇城的威严是足以让他心灵受到威慑压制的。

李麟接待陆云的地方是他的住处金谷园,这里是相对独立的一处园林,原本是齐王李显的居处,当初李显和王妃秦铮有嫌隙,所以不愿意在内宅居住,反而在金谷园下榻,齐王本是李援最为宠爱的皇子,当初在他开府之时,李援赏给他的皇庄产业就是最多的,就连他的王府也比李安、李贽的王府宽阔豪华。李显在军中虽然可以和将士同甘共苦,但是却还是喜欢奢华之人,所以他多年居住的金谷园当真是繁华锦绣,富丽堂皇。李显和林碧大婚之后,夫妻和睦,自然就搬回内宅去了,李麟封了郡王之后,虽然因为年幼尚未开府,可是住在内宅也有所不便,所以李显就将金谷园给了李麟。李麟性子比李显更加狂放,对于这些园林景物殊不在意,所以从未改变过园中布局,倒是太子李骏和昭华郡主江柔蓝过来游玩的时候,各自挑了喜欢的地方下榻,然后迫着李麟照他们的心意改建过几次。

陆云走入金谷园之后,也不由目眩神迷,陆家虽然也是世代将门,不愁吃穿,但是历代家主都是清廉自守,所以家中陈设园林不比普通官员强到哪里去,不过毕竟陆云也见过世面,再说对荣华富贵又不甚贪恋,所以很快就定下心神,随着侍卫走到了碧云阁。

金谷园中有龙首渠通过,汇聚了一池碧水,整个园中的建筑九成以上都是临水而建,池畔堆石成山,假山高约二三十丈,峭壁林立,占地数亩方圆,山上有一座飞丹流檐的二层小楼,只有一条乌石铺成的蜿蜒山路可以上下通行。只需一队禁军在山下守护,纵然是一流高手,也别想随便进出。

沿着山路前行,陆云心中反而平静下来,凭他出身将门的见识,自然知道这里即是易守难攻的绝地,也是软禁囚犯的好地方,不过想必自己一个平常少年,是不会有人这样费心的,所以想必是李麟在自己的住处召见他,这也是一种厚待。走上山顶,一眼便看到碧云阁孑然独立,四周寥落,空空荡荡,假山上面虽然铺了厚厚的泥土,却只种了一些低矮的常青灌木,一眼看去,丝毫不见*盎然,反而觉得有一种深秋的阴郁,那楼阁就和负手站在朱栏之前,俯瞰碧波的少年一般孤傲跋扈。

陆云走到李麟身后,下拜道:“草民云路,叩见嘉郡王殿下千岁。”

李麟却不令他平身,冷冷道:“本王召你前来,你可知道是何用意?”

陆云不卑不亢地道:“草民得罪殿下,殿下若有惩处,草民也无怨言。”

李麟回过头来,噗哧一笑,冷峻的气质立刻被稚气的笑容破坏无遗,他过来亲手搀起陆云道:“看来是吓不住你了,当日本王也有些激动,不免屈辱了你,不过你当日竟对昭华郡主失礼,也难怪本王恼怒,今日本王邀你前来,一来是想给你赔个礼,二来么,本王也想见识一下你的箭术。”

陆云纵然是心存敌意,也觉得心中一暖,心道,难怪这嘉郡王小小年纪就有这样的声名,不愧是大雍皇室名将齐王爱子。他起身一揖道:“请郡王爷吩咐。”

李麟目光一转,道:“我这里没有校场,昭华不许我在这里修建,不过只是看看你的箭术,去父王的校场又太麻烦了,你可能射中那棵树。”说罢指向远处临水的一株柳树,那里距离假山有一百五十步之远,又是高低悬殊,若想射中柳树,必须是一流的箭术才行。

陆云的弓箭已被侍卫拿走,正要向李麟讨取,只见一个侍卫捧了一副弓箭过来,弓是犀角弓,描金箭囊里面是二十支上好的雕翎箭,陆云一见此弓便目光一亮,上前拿起来拂拭良久,爱不释手。拉弓空弦使了几次之后,他取了三支雕翎箭,引弓而射,只见三缕乌光一闪而逝,三只雕翎箭居然射在同一根柳条之上。百步之外接连三箭都射中风中飘拂的柳条,这样的箭术已经可以称得上是神箭了,李麟目中精光四射,自叹不如,见陆云将犀角弓放回盘中,仍然是满目留恋,李麟笑道:“好,云兄你的箭术果然非凡,这张弓乃是工部精制,千里挑一,也只有这样的宝弓才配得上你的箭技,本王就将这副弓箭送给你,你可不能推辞。”

陆云心中十分喜爱这弓箭,且他也有心接近李麟,所以便躬身一揖道:“谢郡王爷赏赐,草民愧领了。”

李麟见他如此豪爽,心中大喜,道:“你这样的箭术,如何沦落江湖,听说你故乡已经没有亲人,何妨留在本王身边作个侍卫,我大雍素重武勇,你在这里前途似锦,也免得去给南楚的昏君奸臣做奴才。”

陆云心知李麟是从商队中查问过他伪造的身世,故意露出犹豫的神色道:“草民是南楚人,故土难离,再说恐怕因为出身有些妨碍。”

李麟笑道:“你过虑了,我大雍海纳百川,从不计较这些出身来历的小事,别说你是南楚人,两国虽然交过兵,却也多年交好,就是原来的北汉军将士,多半手染我大雍军民的鲜血,如今还不是照样得到重用。”

陆云装作心中块垒消除的模样,欣然道:“如此草民就多谢郡王爷赏识提携。”

李麟道:“这也是你自己的本事,你也年纪不大,现在也不方便从军,这样吧,你就留在本王身边作个侍卫,过几年若有战事,随本王出征,也好搏个功名,一会儿你将身世履历写清楚交给我的侍卫总管,等到兵部司闻曹有了回文之后就是登记在册的侍卫了。”

陆云心中一凛,这少年郡王虽然爱才,却不是轻信之人,不过他暗想,若没有一段时间,根本无法证明自己的身世真假,而且他的身世虽然是伪造的,可是也不是全然胡说八道。他称自己是江夏云桥村之人,父母双亡,有一个叔父多年前背井离乡,据说在长安有人见过他,所以前来寻亲。这江夏云桥村确实是有的,虽然跟陆云没有什么关联,他自己的祖籍是吴郡,江夏是他祖父多年镇守的地方,所以对于江夏乡里的情况,陆云并不陌生。再说南楚这些年和强邻毗邻,边境村人迁入大雍的比比皆是,所以他的身世倒不是全无根据,在江夏云桥村未必没有这样一个寻亲离家的少年。而且陆云声称当日在家中因为没有冠礼,并没有名字,只有一个乳名叫做二郎,在南楚乡村,这个乳名若是一叫出去,只怕十个人里面会有五六个人答应,所以陆云并不担心会被发觉自己的真正身份,就算是发觉有些问题,据他估计,李麟也不会一定要将自己当成奸细杀了。再说这段时间过去,自己纵然不能得手,也有机会逃脱的,所以陆云便俯首称是,并没有露出一丝慌乱。

李麟见他顺服,却也没有觉得奇怪,虽然当日陆云表现的十分冷傲,可是毕竟身份悬殊,自己以礼相待,他自然也不该过分矫情的,这样的表现倒是理所当然,想到自己可能招揽了一个出色的少年侍卫,他笑道:“云路,你也不用过于拘礼,我们府上规矩没有别家森严,等到你的身份核实之后,本王带你去见父王,他也很想看看你的武艺呢。”

陆云心中一凛,齐王的声名在南楚可以止小儿夜啼,当初他在荆襄两战,杀人无数,如今又平了北汉,在南楚的传闻中,齐王就是屠夫的代名词,当然在陆云心目中,齐王是父亲的对手之一,若有机会见到,他倒也十分期望。

接下来的日子,陆云便被李麟留在碧云阁,碧云阁乃是李麟寝居,本来不当让资历浅薄的陆云留在这里,不过这里并没有什么机要文件,所以李麟向来将陆云这样身份的人先安排在这里,既可以起到软禁的作用,又有信赖器重的意味。

适逢雍帝大寿,朝廷上下都很忙碌,李麟更是几乎每天都要入宫陪伴太子,陆云身份尚没有查清,自然不能入宫,虽然李麟不在,可是他身边总有侍卫相陪,更是婉言劝阻他离开金谷园,陆云心惊之余却也无可奈何。又过了几日,乃是雍帝大寿,普天同庆,李麟更是被太子留在东宫,陆云只能坐困愁城,恨不得放弃刺杀逃出去,只是齐王府戒备森严,陆云根本无法随便走动,索性破罐破摔,留在碧云阁不出去了,想来最多是身份存疑,失去接近目标的机会罢了。

雍帝大寿之后的第三天,陆云被从宫中返回的李麟召去,陆云走进去的时候,只见一个中年官员肃手而立,而李麟坐在主位上看着手中的绵纸。这一次李麟不是穿着平日常穿的黑衣箭服,而是穿着郡王服饰,杏黄袍服,头戴金冠,他虽然年少,但是身量已经颇高,看上去威风凛凛,颇见皇家气象。看到陆云进来,他笑着将手中的绵纸递给陆云,道:“虽然不是十拿九稳,不过你的身份大致已没有问题了。”

陆云忍住心中的惊讶,接过那张绵纸,上面写着一些蝇头小字,记录了一个南楚江夏陆村的少年家世。父亲是受伤退伍的低级将领,母亲是书香门第的淑女,父母都已经因病亡故,族人星散,有一位叔父下落不明,少年自幼习武,精于箭术,三年前远走他乡,尚未加冠,乳名二郎,不过因为没有族人,所以不知道年龄。陆云差点惊呼出来,想不到真有一个这样的人存在,虽然和自己的描述有些参差,但是基本上可以含糊过去,心中庆幸身世将不会造成阻碍的同时,陆云不由暗中拜谢上苍。

李麟去过那张绵纸道:“难怪你箭术出众,原来是克绍其裘,既然你的身份已经没有问题,今后就在我身边行走吧,正好一会儿我要去送红霞郡主和王仪宾回代州,你跟我一起去吧。”

陆云心中一动,若是替红霞郡主送行,齐王和嘉平公主必然前往,能够一举见到这么多名将,忍不住露出期盼神色。

\chapter{第五章 水流花谢}

郡主入雍后,镇守雁门二十年,屡率军入蛮地掠敌,蛮人见之魂断,呼之曰血罗刹。

郡主仪宾王骥,本楚人,失父母,流落建业,入江哲门下,列为八骏之首,后奉哲命赴蛮地探军情,以伯乐神医之名声震边塞,偶遇郡主于代州,钟情于东海,惜各为其主,凤泊鸾飘。后郡主血战于雁门,骥闻之,泣告于哲,求赴代州同死,哲不得已许之,骥乃舍青云之路,至雁门助郡主守关。雁门将破,远霆感骥痴,阵前以郡主许之。郡主降雍后,骥奉旨协守雁门,为郡主之副。

初时,主无出,或有劝骥纳妾传宗者,骥不许,曰,我无亲族,毋忧绝宗祀。主闻之涕然,终不忍王氏无后,乃亲为选良家女,骥愤然出,半月不归,主乃止。

——《雍史·红霞郡主传》

灞桥柳如烟,行人欲断肠,送行的官员早已经离去,长亭之内,林碧却仍然握着妹妹的手低声嘱托,这一别不知何年何月能够再见,林碧心知自己终生也不会有机会重回故土,再也无缘见到雁门*,所以对承继自己衣钵的幼妹,更加牵肠挂肚。长亭之外,赤骥正和齐王低语,他们很有默契地留出了让林氏姐妹话别的空间。而李麟和其他几个兄弟站在一边肃手而立,这场合没有他们说话的余地。陆云立在李麟之后,小心翼翼的打量着那些闻名已久的人物。林彤和赤骥他都已经见过了,而齐王的豪迈爽朗和林碧的雍容威严让他油然而生一种倾慕之情。他自然不知道七年前的齐王,却是一柄寒光四射,杀气不能自抑的利剑,伤人也伤己,而今日,宝剑已经藏于匣中,虽然锋利不减,却是更加莫测高深。

亭中,林碧低声道:“彤儿,你要小心一些,这几年你们多次深入蛮地,也未免太危险了,你是代州主将,若有闪失影响极大,也该让后辈多带带兵了。听说你经常和妹夫吵闹,这不大好,虽然他是你的副将,可是毕竟也是你的夫婿,又是江侯的心腹,你不要和他生出嫌隙,还有,你和妹夫成婚多年,还没有子嗣,这件事情就连皇后都问过,你们夫妻准备怎么办?若是你听我的话,还是替他纳妾才是。”

林彤瞥了赤骥一眼,也低声道:“姐姐,我和骥郎吵架不过是习惯罢了,若是几日不吵,便浑身不舒服,你可别以为我凶悍,分明是他变着法子喜欢惹我生气。这次进京,骥郎请侯爷替我们诊过脉了,侯爷说,我们都没有问题,没有子嗣或者是天意,其实我也问过骥郎的意思,不过骥郎说他早已没有亲族,也不担忧无后不孝,我倒是肯委屈些让他纳妾,还替他张罗过,是他坚决不肯,还和我生了半个月闷气。”

林碧听了不由一笑,用余光忘了赤骥一眼,道:“妹夫也是至情至性之人,难怪当年肯陪你赴死,罢了,你们的事情我也不管了,只要你们夫妻和顺,我也就放心了。”

林彤却是忧心地道:“姐姐,这次我来长安,看到江侯爷在你面前好像总是战战兢兢的,不是你为难他吧,这样是不是不大好,江侯爷是骥郎的恩主,这个人很可怕的,你看骥郎不过在他身边待了几年,便是这样难缠,你是不是还怨恨他从前设计害了姐夫,不,龙将军。”

林碧淡淡一笑,目光宁静而平和,她轻声道:“两国征战,哪里有那么多仇怨,李显亲手迫死庭飞,我尚且不再怀恨,何况是江侯呢。若说他惧怕我,这可是你误解了,他对着凤仪门主、魔宗宗主尚且不惧,我一个败军之将,有什么可怕的。这人性情就是这样,越是亲近之人他越是喜欢欺弄,你看他总是欺负柔蓝、麟儿这些孩子,难道会以为江侯当真讨厌他们么,在我面前,他既然不敢欺弄我,自然只有惧怕我了,这人性情就是这样别扭古怪,越是他重视的人,就越是不知道该如何相处。恐怕这世上只有长乐公主和邪影李顺,能够见到他最真实的一面吧。”

林彤听得眼前一亮,想起王骥说起在江哲面前总是吃苦头的往事,忍不住低笑起来,姐姐当真是明察秋毫,一眼便看穿那个有着神鬼莫测之机的男子,不过是一个不善于表露真情的腼腆之人。

正在她们姐妹执手低语的时候,远处烟尘滚滚,马蹄如雷,却是十几骑骏马绝尘而来,众人抬眼望去,为首的两人一着青衣,一穿黄衫,正是霍琮和柔蓝带着侍卫前来送行。

林彤露出微笑,她对柔蓝也是十分喜爱,方才还在埋怨这丫头无情无义,不来相送,一声欢笑,她走出长亭,招手道:“蓝蓝,怎么还记得来送我啊。”

柔蓝勒马收缰,下马奔来,一把搂住林彤的颈子道:“彤姨,你好没良心,我被太后娘娘召去陪她了,要不是我记着你今天就走,求娘娘让我出宫来送你,现在我还在长乐宫看戏呢。”

林彤伸出两指捏住柔蓝雪白娇嫩的脸颊,笑道:“就你会找理由,当我不知道么,你的公主娘亲这几天就在宫里面陪太后呢,怎么不见你爹爹,这次骥郎要去给你爹爹辞行,居然都没有见到,怎么皇上寿筵之后就看不见他了呢?”

柔蓝挣开林彤的手指,香舌轻吐道:“这个我可不知道,爹爹不在家,我欢喜还来不及呢,霍哥哥,你一定知道的吧,爹爹对你比对我和慎儿都好些。”

林彤望了一眼霍琮,这个少年虽然平凡普通,可是不知怎么,林彤就是觉得在他面前不敢放肆,或许是他那种平和宁静的气质让人不愿失礼吧,她微笑问道:“霍公子,你知道先生在什么地方么,骥郎原本想当面辞行的,这一去也不知什么时候能够再来长安。”

霍琮施礼道:“禀郡主,先生前日从宫中赴宴归来,就去了南山别业,似乎有什么事情要处置,他说让我替他给郡主和赤骥师兄送行。”

林彤失望地叹了口气,不再追问,而隐在侍卫当中,原本正忍不住看向柔蓝的陆云却是心中一动,南山别业,江哲去了南山别业,那就是不在皇城之内,身边的侍卫不知道会否少些,或许自己会有机会刺杀吧,只是不知道那别业在什么地方,而且自己不知道能不能抽身去寻,再说那人身边定有侍卫保护,还有邪影李顺在侧,恐怕难以得手。

这时,赤骥走到霍琮身边低声道:“师弟,有件事情请你转告师父,我见嘉郡王新收留的那个侍卫面貌有些像一个人,虽然觉得不大可能,可是还是要请你禀告一声。”

霍琮神色不动,微笑着侧耳倾听,仿佛赤骥和他说的不过是些家长里短的琐事,口中却道:“这件事情先生已经知道了,师兄不必挂怀,先生说,师兄临行之前,可以将段将军的事情告诉公主,想必公主也是想和段将军重见一面的。”

赤骥闻言心中一动,对于江哲已经知道那南楚少年之事,他倒不觉得奇怪,这少年相貌和陆灿有四五分分相似,精通箭术,双臂力大无穷,就是他也生出疑心,江哲若是见到必然心疑。可是将段无敌之事告诉林碧,他担心先生又准备给人下套,若是别人或许自己只会帮忙蒙住那人眼睛,可是林碧乃是林彤亲姐,他有些担心后果。

霍琮见状,低笑道:“师兄放心,先生也是好意,希望公主能够说服段将军为朝廷效力罢了。”

赤骥心中一宽,道:“我知道了,师弟,这次前来,见先生对你青眼有加,我可是又羡又妒,你有这个福气留在先生身边,定要代我们这些不肖弟子尽心侍奉。”

霍琮点头应是,心中却隐隐泛起一丝惆怅,师恩如山,先生待自己如此之好,自己却不得不隐藏心事,欺瞒于他,若是有一日那件事情泄露,自己又当如何是好,除非是血溅寒园,否则生有何欢。

无论是如何不舍,林彤和赤骥终于还是踏上了旅途,望着远去的背影,李显走到泪光隐隐的爱妻身边,道:“碧儿,回去吧,最多过两年,再让他们进京述职也就是了。”

林碧黯然道:“没什么,你不用担心,姐妹分离这是迟早的事情,我只是有些难过不能回去看看罢了。”

李显默然,这件事情他也帮不上忙,有些事情也是无可奈何,就像他用放弃军权换取和林碧结合,林碧想要刘氏和林家的安泰,也只能放弃返回代州的期望。见他如此,林碧反而笑道:“其实这也没什么,长安也很好,再说有你和孩儿在,哪里不是家呢,倒是你娶了我,牺牲未免大了些。”

李显见她释然,笑道:“孤王不爱江山爱美人,这有什么不对。”林碧面上一红,就要转身离去,却被李显揽住纤腰不肯放手,她心中一甜,对自己没有固执仇恨放弃这令自己心动的男子的决定,再也没有一丝悔意。想起方才赤骥偷偷告诉自己的消息,或许自己应该去见见段无敌,前尘往事,应该是不需挂怀了,纵然自己又是中了江哲圈套,能够让一个心存黎民社稷的忠义之士不至于沦落江湖,也是值得的。

李显和林碧在这里情意绵绵,却让齐王几个儿子在一边十分尴尬,都是低着头不语,除了李麟之外,其他几个王子没有一个和李显个性相似的,从前李显对他们不闻不问,他们对李显也是只有畏惧之心,直到林碧加入齐王府之后,重立家规,对这几个庶子颇为照顾,这几个少年对林碧自然十分尊重,当然不敢看到李显轻薄她的景象。李麟胆子大,别过脸去重重咳嗽了几声,林碧一惊,连忙推开李显。

李显只得松开手,望望几个儿子,道:“你们都自行回去吧。”然后狠狠的瞪了李麟一眼,挽着林碧上车走了。

李麟哭丧着脸,自己可是好意,却得罪了父王,大概回去之后,父王就会寻个理由拉自己去校场了,想到很可能今天晚上会浑身疼痛,难以入眠,李麟心情当然不会好转,他那几个兄弟给了他一个自求多福的眼神,都各自上马走了。

这时,霍琮含笑道:“郡王爷,这几日先生和公主都不在府上,你不如过来小住几日如何?”

李麟一听大喜过望,连忙道:“好,好,多谢你了,霍大哥。”

陆云眼中掠过喜色,想不到这么快就有机会进入江哲的府邸,虽然江哲现在不在,但总归是个收获不是么。

他全未发觉,在邀请李麟的时候,霍琮的目光在他身上停留了一瞬,他自然也不知道,那份对他的身世调查的文书就是霍琮伪造之后通过司闻曹送到李麟手中的,否则世间哪里会有那么巧的事情,真有一个云二郎的存在。

第二天清晨,陆云睁开眼睛的时候,天已经大亮了,他不由十分奇怪,昨日他跟着李麟到了江哲府上,李麟住在栖凤轩,他作为李麟的侍卫自然也得住在那里,江哲的府邸据说本是雍王潜邸,在陆云看来,虽然也是富丽清幽,却比齐王府小的多了,也没有那么多亭台楼阁。身在仇人的地盘,他本来以为自己昨夜会很难入眠,却不料一夜无梦,真让他费解。

走出房间,他一眼看到李麟正在院中练剑,几个侍卫在旁边相陪,陆云脸一红,站在一边,等到李麟练剑之后,他上前谢罪道:“属下不小心睡过头了,还请殿下恕罪。”

李麟笑道:“你是第一次来这里,不习惯也是有的,本王有时会在这里小住的,以后你就习惯了。好了,陪我去寒园吧,霍大哥让我们去他那里一起用早膳。”

陆云眉心一跳,忍不住道:“属下在南楚就听说寒园乃是楚侯运筹帷幄之处,想不到竟然已经给了霍公子居住。”

李麟突然诡秘的一笑,道:“你说得错了,寒园至今仍然是姑夫的居处,虽然现在姑夫的寝居在内宅,但是一个月总有十几天,姑夫仍然住在寒园,而且那里还是姑夫的书房,不知道多少计策是在那里拟定的,就是皇伯父要向姑夫问策,也是在寒园的。”

陆云有些疑惑,明明霍琮是住在寒园的,他如今已经知道那青衣少年乃是江哲弟子,也就是少主人之一的身份,怎会没有自己独立的住处。带着重重疑惑,陆云跟着李麟走向寒园,一路上他仔细留心,江哲府上的侍卫果然个个非是等闲,防卫森严远胜齐王府,想要行刺当真是十分艰难。

走到寒园门口,李麟让其他侍卫下去休息,拉着陆云道:“你和他们不同,本王当你是朋友,和我一起进去吧。”

陆云心中一暖,他自然知道李麟待自己与众不同,朋友的意味倒是比下属多些,但是眼看就要进入江哲经常流连的地方,他心中十分紧张,也就顾不上体味李麟的心意了。

一走进寒园,陆云便是一愣,这里面的清幽冷落让他想起父亲的书房所在之处,也是这般冷寂,就连明媚的春光在这里似乎也减去了几分颜色,外面森严的戒备和里面的萧条冷落,真是对比鲜明。不过让陆云更加奇怪的是,在初升的阳光下,霍琮一身布衣,正在那里修剪花木,他是那样的认真尽责,就连自己这些人进来他都没有察觉。

李麟上前叫道:“霍大哥,你还没有完工啊,早膳不是还没有准备好吧,这是云路,霍大哥还记得吧,这次我带他一起来的,也让柔蓝见见他,知道我没有欺辱他。”

霍琮闻言抬起头,露出一个淡淡的笑容,将手中的花剪放下,拍去上面的泥土,道:“听郡王爷说,你已经在他身边任职,虽然多半是郡王爷相迫,你也不要怪他,他也是一片好意。”

陆云连忙道:“并非是王爷相迫,小可流落长安,寻亲不遇,也不是了局,留在郡王爷身边,尚可有个落脚的地方。”

李麟皱眉道:“云路,原来你是这个心思,难怪当日这么容易就留下来,本王还生过疑心呢?”

陆云心中一宽,就是想到李麟可能会怀疑自己留下的缘故,毕竟当日在驿道上,自己表现的十分桀骜,这般轻易屈服未免有些儿戏,所以今天他趁机弥补了一下,果然消去了李麟的疑心。

霍琮眼中闪过一丝笑意,道:“原来如此啊,好了,柔蓝一会儿就会过来,你们先去花厅等着,我去换件衣服。”说完他转身走去,李麟拉着陆云走向花厅,嘟囔道:“寒园就是这点不好,不许留仆人伺候,幸好早膳还不用自己去取。”

陆云心中疑惑,忍不住问道:“霍公子很喜欢照料花木么,为什么他会住在这里,这里不是军机重地么?”

李麟笑道:“你可知道霍大哥的身份?”

陆云道:“属下听说霍公子是侯爷的亲传弟子。”

李麟举起食指道:“有件事情,你却不知道,霍大哥还是寒园的仆役,负责照看这里的花木。”

陆云愕然,良久才道:“可是,霍公子不是侯爷的弟子么,怎么侯爷还让他做仆役,这未免有些太离谱了。”

李麟笑道:“我这个姑夫的性子就是这样古怪,所以霍大哥才会住在寒园,却又不是寒园的主人。”

陆云还是大惑不解,这时耳边传来一个平和的声音道:“这是先生用心良苦,先生常说,每个人都应该有自己的位置,江家不留无用之人,琮若想留在府上,就要以劳力换取食宿,所以琮虽然拜在先生门下,却仍要做仆役维持生计。不过成了先生的弟子,总是有些好处的,寒园的工作并不繁重,那些耗费时间的工作都有别人去做,我只需照料花木即可。”

陆云回头望去,只见霍琮换了一身洁净的青衫,站在门口,清晨的阳光映射在他的背后,让陆云觉得他的面孔有些模糊,可是他仍然能够看到霍琮平静安详的神色。

他听到霍琮继续说道:“有些人将轻抛权势富贵当成美谈,有些人身份低贱,却以布衣傲王侯自得,先生却不以为然,他常说富贵权势不仅仅是权利和享受,也是一种不可推卸的责任,既然手握大权,就应该尽忠职守,不负苍天爱重,若是出身寒微,操持贱业,也不当以为羞辱,应该安之如素,只要无愧于心,就不负平生。”

陆云只觉得心神撼动,什么样的人能够说出这番话,这样的人怎会卖国求荣,辜负君父。花厅之内一片寂静,就连李麟也在深思霍琮所言。

这时,门外传来少女清脆悦耳的声音道:“霍哥哥,麟弟,我来了,麟弟,听说你带了云路过来是么,云路,麟弟没有迫你吧。”随着语声,陆云只觉眼前一亮,一个穿着鹅黄衫子的少女站在门口,肤若凝脂,容貌秀美,尤其是那双黑亮剔透的明眸,总是滴溜溜转个不停,让人越发觉得这少女顽皮娇俏。她也没有过分的妆饰,只是用一枚金环束发,那金环浑似花枝环绕,相连处打造成含苞欲放的一朵寒梅,这般姿容相貌,虽然年幼,却已经仿佛神仙中人。

陆云心中一颤,初次见到昭华郡主的女装模样,他只觉的心中慌乱,却又隐隐带着痛惜伤悲,一时间情绪无比低落。

霍琮和李麟却是常常见到柔蓝俏丽模样,习以为常,李麟抱怨道:“怎么总是不相信我,我哪里是强迫别人的恶人,云路可是自愿留在我身边的。”

柔蓝明眸流盼,道:“云路,是这样么?”

陆云这时也已经清醒过来,躬身道:“属下得郡王器重,确是自愿留在郡王身边的。”

柔蓝嫣然一笑,道:“那就好,霍哥哥,今日难得爹爹不在,我们吃完早膳一起玩好不好。”

李麟高兴地道:“好啊,太子今日不会召我去的,我们正好出去游春。”

霍琮笑道:“游春什么时候都可以去,倒是先生不在,不如在府里玩乐,岂不是更好。”

李麟和柔蓝听了都是连连点头,柔蓝道:“还是霍哥哥聪明,我们就去临波亭吧,虽然现在无雪,可是临波亭赏花也很好,内宅云路不便去的。”

霍琮点头道:“临波亭很好,你们或许不知道,当初先生就是在临波亭赏雪赋诗,压倒了雍王府的所有幕僚呢,一会儿到了那里,我将当日先生他们所赋的诗都抄录下来给你们看。”

柔蓝和李麟虽然年少贪玩,可是对诗词歌赋也不是一无所知,更何况是江哲的旧事,霍琮既然要给他们讲诗,也定会告诉当日之事,这些事情江哲从不跟他们说起,却对霍琮并不隐瞒,有机会得知江哲过往,两人都是连连点头,就是陆云也心中向往,此刻他对江哲的恨意不知不觉中已经消退了许多,更想知道他的事迹,毕竟在南楚,众人除了漫骂之外很少提及江哲的传闻。

四人匆匆吃过早膳,联袂来到临波亭,霍琮果然录了那些诗词给三个少年讲解,又将昔日之事讲给三人听,谈兴正酣的时候,突然有侍卫前来禀报道:“郡王爷,太子殿下急召你入宫。”

柔蓝和李麟都是一脸的扫兴,李麟无奈地道:“看了今日只能半途而废了,云路不能跟我进宫,霍大哥,就让他先跟着你吧,等我晚上回来你再接着讲好不好。”

霍琮笑道:“你去吧,太子说不定有什么急事,我等你回来再接着讲,反正先生后日才能回来呢。”

送走了李麟,柔蓝无精打采地坐在亭边,望着湖水发呆,霍琮则是取过棋坪自己打起棋谱来,亭中气氛有些沉闷,陆云想要告辞离去,却又有些不舍。霍琮见陆云神情无聊,笑道:“郡王爷在这里就和自己家一样,你也不要拘束,其实你年纪还轻,还是应该多读些书才是,兵书你读过没有?”

陆云心道,若是我说读过,未免有些不符身份,便道:“没有读过。”

霍琮道:“你既然跟着郡王,将来难免征战沙场,要想作个将领,兵法是不能不读的,这样吧,我回去取一本书给你看。”说罢转身离去,亭中只留下柔蓝和陆云两人,附近的侍女侍卫早就被霍琮遣走,亭中一片寂静。

望着柔蓝的背影,陆云心中突然生出恶念,这可是一个良机,自己有机会取走江哲爱女的性命,江哲令自己的父亲痛苦万分,自己若是杀了柔蓝,必定可以让江哲痛不欲生,与其等待可能永远也不会出现的刺杀机会,眼前的少女是更好的选择。

抬头看看四周无人,陆云终于按耐不住心中杀机,心中的仇恨和多日来不能自主的委屈驱走了他心中的朦胧爱意,若是没有了制约,就是最良善的人也会萌生恶念。

站在柔蓝身后,他轻轻拔出藏在靴子里面的匕首,就要向柔蓝背心刺去,只需一剑,就可以取了这少女性命,然后他可以等到霍琮回来,偷袭刺杀了他,霍琮看上去不会武功,柔蓝也不高明,自己应可得手,之后就可以凭着嘉郡王侍卫的身份离开这里,只要他安排妥当,直到他离开皇城,也不会有人发觉尸体。

可是当他站在柔蓝身后,少女娇小的背影让他心中一软,这一剑再也刺不下去,自己的仇人是江哲,和这少女有什么相关,霍琮对自己颇好,自己如何可以恩将仇报,就在陆云心中犹豫不决的时候,柔蓝不知怎么失去了平衡,一声惊叫,向水中倒去,陆云微微一愣,只见柔蓝已经落入池中,一边喊着救命一边伸出手胡乱挥舞。她的声音传得很远,陆云可以看到远处有人影闪动,想必是侍卫们听见柔蓝的呼救声,正在向这边赶来。

看看水中挣扎呼救的少女,他心中一颤,和衣跳入水中,不过片刻便抱着柔蓝爬了上来,这时候侍卫们已经纷纷赶到,陆云熟练地帮助柔蓝吐出腹中清水,柔蓝清醒过来,抱着刚刚赶来的霍琮大哭起来。霍琮谢过陆云,匆匆抱着柔蓝走去内宅。看着柔蓝苍白的面色,以及凌乱的衣衫,陆云心中不知是什么滋味,救起柔蓝,并不是为了掩人耳目,他跳下水去的时候竟然是全无一丝悔意,目光落到地面上遗落的束发金环,陆云心中越发慌乱。

他当然不会想到,霍琮抱着柔蓝进入后宅,将她送回卧房之后,正要让侍女前来伺候,柔蓝拉着他的衣袖,冷冷道:“霍哥哥,你搞什么鬼,这个云路是怎么回事,为什么他要刺杀我。”

霍琮不动声色地道:“他想杀你么?”

柔蓝怒冲冲地道:“我从水中倒影看得分明,他想要从后面用匕首刺杀我,我知道不是他的对手,所以装作失足落水,这样他就不便下手,我却可以呼救。你可别说你不知情,骏哥哥怎么会出尔反尔,派人来召麟弟进宫,我可不信这个时候会有什么大事牵涉到麟弟,定是你从中作梗,故意遣走麟弟,还有你怎么将他和我单独留在临波亭,就连一个侍卫都不留,这不是你的作风。最关键的一点,是谁让侍女通知我今天里面穿上金缕衣的,你有什么瞒着我,那云路是不是南楚奸细,若不是我担心他刺杀不成露了破绽,可能反而会破坏了你的计划,我何必要装作落水呢,反正他的匕首也不可能刺穿金缕衣。”

霍琮微微一笑,道:“这个你就不用过问了,这是先生的意思,其实我看云路还是狠不下心的,再说暗中有侍卫保护你呢,绝不会让他得手的,今日之事你不要说出去。”

柔蓝怔住了,此刻的霍哥哥,面上的神情像极了爹爹平日捉弄自己时候的模样,她打了一个寒战,决定由衷的同情那个方才想要杀害自己的少年。

\chapter{第六章 惊鸿照影}

陆云怔怔望着手中的金环,呆若木鸡,方才有侍女前来寻找郡主失落的金环,他却下意识地将金环藏起,心中不免有些愧悔,纵然明知那少女对他来说犹如水中仙,梦中花,他却为何深陷下去,错过了唯一报复江哲的机会,罢了,罢了,柔蓝不过是江哲的义女,自己怎能如此无耻,对江哲无可奈何,就将目标放到一个小女孩儿的身上。

正在这时,远处突然传来李麟的怒吼声道:“什么,你说柔蓝落水,差点淹死,这怎么可能,你竟敢诅咒她,本王要砍了你。”

陆云心中一凛,他对李麟已经是颇为忌惮,唯恐他问多了,发觉自己的不妥,连忙闻声赶去,还没有绕过花丛,便听到一个清朗含威的声音道:“麟弟不可鲁莽,这侍女说的或许过分些,但定无恶意,你不也是听说柔蓝落水,才匆匆赶回来的么,我们还是去内宅看看吧,这丫头平日胡闹惯了,说不定是怎么回事呢?”

陆云心中一动,透过花丛望去,只见前面花径上,李麟怒气冲冲地站在一个侍女面前,那个侍女吓得魂不附体,跪在地上磕头如捣蒜,而在李麟身后,站着一个身穿明黄服饰的少年,大概十五六岁年纪,相貌俊秀温文,双目幽深,如同深潭也似,神态从容磊落,此刻正拉着李麟相劝。不必多想,见这少年服饰以及对李麟的称呼,陆云心中翻江倒海一般,自己竟然和大雍的太子殿下李骏距离不到数丈,忍不住握住了匕首。目光落到那少年太子的面上,见他神态温和,含笑解劝,虽然有着尊贵无比的身份,却令人有如沐春风的感觉。听闻这位太子殿下小小年纪便代李贽镇守幽州,素有仁孝之名,如今一见,果然是气度不凡,再想起南楚国主赵陇,明明年纪相仿,更是一国之君,却是只知吃喝玩乐,平庸无能,心中更是一痛,不由气息一乱。侍立在李骏身后的一个青年侍卫眉梢一扬,上前一步,挡在李骏身侧,喝道:“什么人在花丛后面鬼鬼祟祟的。”他的语气并不凌厉,可能因为这里是长乐公主府,公认防备最森严的府邸之一的缘故。

陆云心中一震,绕过花丛,向李麟单膝下跪道:“属下云路叩见郡王爷。”他故意表示不认识李骏,这样即使问罪,也会轻些,不知者不罪么。果然他偷眼望去,那侍卫神情和缓,退到了李骏身后。

李麟粗声粗气地道:“原来是你小子,是不是见本王发脾气,不敢过来了?”

陆云心中更加安定,低首敛眉地道:“属下不敢。”

李麟摇手道:“算了,来拜见太子殿下,皇兄,这是我新收的侍卫,我见这小子还不错,过几年准备送到东宫去给你做侍卫,不过现在还不行,明鉴司和司闻曹盯得紧,这小子身份不甚清楚,我若送了去,只怕要遭弹劾的。”

李骏微微一笑,他自然知道这个道理,自己身边的侍卫都是精挑细选的,身世来历、武功人品,都要经过考核,不过李麟既然如此重视这个少年侍卫,想必人才难得,他上前一步,亲手搀起陆云道:“平身吧,你是麟弟的侍卫,以后也不免和孤常常相见,不必如此拘礼,也不要听麟弟胡说,孤东宫的侍卫都是父皇指派,人数有限,所以不免条件多些,你今后跟着嘉郡王,也是前程似锦,过几年在沙场上搏个功名,封侯拜将,岂不是胜过在孤身边委屈。”

陆云唯唯诺诺,眼中闪过倾慕之色,这位大雍太子,果然是帝王气度,只是简简单单几句话,便听得人心中温暖,若自己果然是无牵无挂的云路,只怕从今后舍命相报也是可能的。

李骏又仔细看了陆云片刻,见这少年年纪虽轻,神态也颇为恭谨,可是举止不卑不亢,眉宇间带着一丝傲气刚强,果然是人品难得,也不由心中喜爱,看了李麟一眼,嘉许地道:“王弟的眼力果然不凡,我见此子有长孙将军的气度。”

李麟露出得意洋洋的神情,他年纪尚幼,露出这样的神情不但不令人生出恼怒之意,反而令人觉得他稚气尤存。李骏摇头微笑,道:“好了,我们去看柔蓝吧,她吃了苦头,一定会在我们身上讨还的,若是去的晚了,只怕要受她几日冷落了。”

李麟神情一变,愤愤道:“柔蓝最是偏心,每次见了你都是眉开眼笑,见了我就是横眉竖眼,明明你三五天才能来看她一次,我几乎每天都陪着她,可是她对你总是那样厚待。”

李骏大笑道:“谁让你不是看着她长大的呢,想当初我还是雍王世子的时候,可就尽心竭力帮着她逃脱姑夫的毒手,你呢,东海初见就和她争执,还逼着小丫头叫你哥哥,后来又被姑夫骗了,当他的帮凶欺负柔蓝,活该你今日受报。”

李麟顿足不语,脸上一会儿黑,一会儿红,望望四周忍笑的侍卫,喝道:“都滚开,这里是姑夫府上,还用得着你们在这里看戏。”两人的侍卫面面相觑,不知是否该听命。李骏笑道:“罢了,除了冷恢之外,你们都下去休息吧。”除了站在李骏身后的青年侍卫之外,其他侍卫都各自散去,陆云心中一叹,也准备离去。不料李麟叫住他道:“云路,你跟在霍大哥和柔蓝身边的,听说是你救起了柔蓝,是么?”

陆云面上一红,想起自己本来是准备取柔蓝性命的,不由有些惭愧,低声道:“属下恰好在场,因为略通水性,所以只得冒犯了郡主。”

李骏惊咦了一声,看向陆云的目光更是多了几分赏识,轻轻点头,然后向内宅方向走去。李麟摆摆手,示意陆云不必跟随,然后匆匆赶去,陆云愣了片刻,终于轻叹一声,无精打采地向栖凤轩走去。

谁知他刚刚走到栖凤轩,李麟又怒气冲冲地奔了进来,喊道:“气死我了,都跟我回府。”众侍卫见他大怒,也不敢问他发生了什么事情,只得匆匆跟上李麟出了长乐公主府,李麟取了马匹,恨恨地一鞭下去,竟在皇城之内纵马飞奔,那些侍卫大惊,在后面连声呼唤,他们不敢在皇城纵马,这可是大罪,虽然眼看着李麟的背影远去,却也只能心焦不已,匆匆向齐王府赶去。

陆云心中奇怪,向一个较为熟悉的侍卫问道:“王爷这是怎么了,发这么大的脾气?”

那个侍卫左顾右盼了片刻,小声道:“定是又吃了昭华郡主的排头了。整个长安城谁不知道,咱们郡王天不怕,地不怕,就怕楚侯和昭华郡主,尤其是郡主,他们两人若在一起,一定是三天一大吵,两天一小吵,到了最后,不是郡主去向咱们王爷王妃哭诉,就是郡王去长乐公主殿下那里告状,初时两家长辈都还又是相劝又是责罚,可是转过脸去,他们两个又和好如初了,如今可是谁都懒得管了。不过今天也真奇怪,本来太子殿下和霍公子若是在场,总能劝住郡王爷和郡主的,今日不知是出了什么事情,这两位的话都不管用了。”

陆云听得有趣,忍不住低头暗笑,不论身份何等尊贵,昭华郡主江柔蓝和嘉郡王李麟都终究是孩子罢了,不过他还真得难以将如今这个孩子一般稚气的少年和那个在金谷园召见自己之时气度森严的嘉郡王联系到一起。

过了一会儿,众侍卫回到齐王府,一眼便看到李麟在门前大步流星地走来走去,看到这些侍卫,他怒道:“怎么这么慢,父王让我随侍母妃去南山,你们还不快去准备。”众侍卫一听,也顾不得辩解是郡王爷速度太快,匆匆去准备行装了。陆云心中大喜,自己正在烦恼如何撇开嘉郡王去南山寻找刺杀江哲的机会,想不到嘉郡王也要去南山,不知是否苍天庇佑。众侍卫都已去了,只剩下李麟怒气冲冲地站在王府门前,对着下马石一脚一脚地踢着,发泄心中的怒火。

望着陆云的背影,李麟气得又是一脚踢去,方才去看柔蓝,岂料她对自己冷嘲热讽,说自己眼力差劲,居然留了一个刺客在身边,这怎能怪他,明明是司闻曹没有查清楚,再说她本来不是也对云路颇为赞赏么,怎么如今责任都到了他身上。又气又恼的他本来想立刻出去就云路杀了,谁知却被霍琮拦住,反而让他将云路带到南山别业去。眼看着那个欺骗自己的少年却不能出手责罚,他心中怒火难以消退,索性违反禁律,纵马飞奔返回齐王府。不论他何等气恼,却知道霍琮的意思就是姑夫江哲的意思,一路上想着如何将云路带到南山别业去,还得寻个理由,不能让他生疑,这可是霍大哥交待的。谁知一回到府上,就得知齐王妃林碧要去南山,本来是让自己的庶出大哥李景随行的,他便抢了这个差使,心里知道十有八九又是姑夫的算计,否则母妃怎会莫名其妙地独自去南山呢,母妃如今又怀了身孕,父王几乎是一刻不离的。想到这次自己出了纰漏,多半几个月之内都会被姑夫和柔蓝嘲笑,他便又是气恼又是沮丧,对云路更加恼恨,若非强自隐忍,只怕目光都能将云路刺穿了。

心中满是疑惑,陆云不明白为什么齐王妃会轻车简从去南山,他在齐王府多日,已经知道齐王府在南山并无别业,据说李显性子古怪,说什么不喜欢终南隐士,所以他在西郊和东郊都有别业,唯独没在南山修建别业。但是他也懒得多想,反正有机会去南山倒也不错,他心中盘算着如何寻找江哲的别业,如何混进去行刺,全然没有留意李麟偶尔望向他的森冷目光。

南山距离长安足有五六十里,加上又需绕行西郊,李麟又奉了父命,不许林碧劳累,当夜在杜曲安顿,直到第二天午后,才终于到了南山别业。南山林壑幽美、气势雄伟,皂水、沣水、灞水、浐水、滈水,俱由南山中源出,北流入渭,林碧要去的南山别业就位于南山北麓,一道溪流蜿蜒而下,沿着溪流修建了数处水榭,两侧则是怪石嶙峋,草木丰盛,并无道路通行,若想出入别业,只能乘舟渡水。溪水在山脚汇聚成池,池中停着一只轻舟。云路这些侍卫是最后登舟的,逆水行舟,那青衣仆人却是驾轻就熟,将几个侍卫送到最下面的那座水榭,安顿他们之后便离去了。这座水榭想必就是为了安顿侍卫仆从的,宽阔朴素。

到达之后,陆云便知道这正是楚郡侯的别业,欣喜之余,就在考虑如何寻找行刺的机会,陆云挑了一个临水的房间居住,这房间位于水榭一角,狭小局促,无人和他争夺,却正合他心意。打开窗子,下面丈许处就是溪水,溪水清澈见底,溪底乱石嶙峋,尖锐的碎石之间,可以看到鱼虾在嬉戏。陆云顺着溪水向上望去,视线所及,已经可以看到两座水榭。水榭之间虽然都有虹桥连接,可是陆云知道若是自己走上去,肯定是立刻被擒住,所以他的目光落到了溪水上,若是夜里,自己应该可以溯流而上,寻到江哲的寝居吧。

吃过晚饭之后,陆云只说自己一路骑马疲倦,早早就去安眠了,也没有引起别人的注意,他自己住一个小房间,所以也不用担心别人发觉他的行踪,将房门栓好,等到二更时分,天色已经变得漆黑之后,他就换上一件黑色夜行衣。这件夜行衣乃是精制的,轻薄光滑,可以在水中暂时替代水靠,而且最难得是体积极小,便于收藏,他带在身边许久,都没有人发觉。

打开窗子,他警惕地看了一眼,除了几处水榭之外,并无光亮,他翻身跃出窗子,吊在窗下,伸手掩上窗扉,然后纵身入水,他的水性极佳,动作轻灵,不仅没有发出声响,就连水花也没有溅起半分。入水之后,他逆流游去,水势颇为湍急,颇费力气。游了一会儿,到了第二座水榭,他攀上临水的窗子向内望去,里面也是一些侍卫,看服色是虎贲卫,应该是江哲身边的人。再向上游去,第三座水榭还没有接近,便听到李麟大笑的声音。陆云抓着岸边的岩石休息了一会儿,继续向上游去,转过一个拐角,前方还有四座水榭。第四座水榭黑暗无光,没有声息,他游到第五座水榭,发觉这座水榭比起前面四座有些不同,距离水面只有尺许高度,邻水的房间外面是一处平台,平台的一半是凌波悬空的,三面以朱栏相护,从这里溪水渐宽,水流也缓慢许多。陆云心中一动,正想攀上平台探听一下,手指刚刚抓住一根栏杆,便听到房门打开的声音,然后灯光从门内溢出,将整个平台笼入了昏黄的光芒之中。陆云心中一寒,身躯轻轻沉入水中,只是攀着水中支撑平台的柱子,侧耳倾听。

然后他的耳边传来一声叹息,那是一个男子的声音,然后,头上的灯光亮了许多,想必是那人点燃了平台一角高挑的风灯,这下子四周水面都被照亮了,无法潜行,陆云心中烦恼非常,却只能隐忍等待。过了片刻,那人还没有回房的意思,山风冰凉,月色星光都极为黯淡,不知这人怎么会有赏玩景致的心情,陆云心中暗暗痛骂,却是无可奈何。

这时候,那男子突然轻咦了一声,陆云心中一紧,随即听到了一个女子的叹息声,这个声音清冷而悲凉,陆云只觉得心神一颤,忍不住仔细听去。

只听那女子说道:“无敌,这些年你在异域飘荡,还过的好么?”

那男子的声音平淡清雅,他答道:“多谢你的关心,也说不上好不好,日子还算平静,只是总会想起昔日的同僚,和沁州的风烟,所以终究是忍不住回来了。故土难离,大概就是如此。听说你已经封了侯爵,颇受重用,我也替你高兴。”

那女子淡淡道:“其实皇上对我也是过于厚待了,凭我的微薄功劳,做虎贲卫副统领尚且可以,封侯却是赏赐太重了。”

那男子道:“你当得的,而且大雍重用于你,那些和凤仪门有关联的人就会放心许多,知道大雍不至于因为出身的缘故摒弃他们,想来这几年凤仪门的余孽在大雍的活动应该越发艰难了。”

那女子沉默片刻,道:“这些事情我无需过问的,自从北汉灭亡之后,我心愿已了,除了虎贲卫的事务,我已经不过问别的事情么,护卫皇室责任重大,我不敢松懈。”

那男子叹息道:“我知道你其实对于权势名利并不重视,只是如今你纵然想脱身也是不可能了,若是离开大雍朝廷的庇佑,你在天下可能会是寸步难行,毕竟现在北汉王室虽然已经降服,可是怀恨你的人一定还有很多,就是凤仪门,也不会放过你的。听说你还没有成婚,呼延将军呢,他这次应该是陪你一起来的吧?”

那女子顿了一下,道:“呼延他这次定要陪我来,甚至还去司马大人那里请了假,我也没有办法,只好由得他了。其实我现在过得很好,无需殚精竭虑,无需钩心斗角,有些事情你说得很明白,我只需安分守己,就可以安享富贵,这样的日子是我最期望的,这么多年,我苦苦挣扎,早已经筋疲力尽,当日觐见陛下,我曾提出辞官归隐,陛下说我结怨太多,又是颇有功劳,不愿我在民间消沉,所以给了我一个虎贲卫副统领的职位。我若有心,自可以做一番事业,我若无心,也可以安养度日,皇上待我恩重如山,所以我虽然知道他们也是想利用我的身份安抚人心,却仍然留在长安。如今我一无牵挂,唯一觉得对不起的就是你,所以听侯爷说你也到了长安,终于还是前来看你,你,还恨我么?”

那男子笑道:“恨,谈不上,十三年前,你我分手之时,就已经分道扬镳,各为其主,你虽然投了大雍,倾覆了北汉江山,可是我不恨你,这是你的选择,只要你无悔,别人还有什么可以指责你的呢?七年前,我身陷求生不得、求死不能的窘境,我知道你有心相救,又替我求情,这份情谊我绝不会忘记。可是青妹,我怨你,石英之死,虽然是多种因素造成的,可是你是起了主要的作用,而且我知道你是利用了我们之间的事情,石英虽然和我不合,可是他是堂堂正正的好汉子,刚烈无比。这件事情我永远不能原谅你,你不仅让他百口莫辩,自尽身死,还污蔑他的名节,虽然这是两国征战的手段,我不恨你如此作为,可是身为旧交,我不能不怨你。”

那女子沉默许久,突然笑道:“我明白,今日听到你这番话,我才觉得终于释然,石将军之死,这些年来我每每想起,都是觉得不安神伤,今日有人为此事怨我,我反而可以抛下心事了,谢谢你,无敌,解去我心中死结。这些年来,我始终等着和你重逢的机会,你别笑我,虽然当年在石将军墓前,是我断情绝义,可是直到今日我知道你必会终生怨我,我才能放下心事,觉得不再亏负你。”

那男子沉声道:“我明白,海骊曾对我说,若是不给你机会了断你我之间的缘分,你这一生都不能安乐,否则无论如何,我也不会到长安来的。呼延将军这些年来对你情深意重,当年初见,我便知道他的心意了,你半生凄苦,若有他陪伴,我也能够放心许多。”

那女子的语气多了几分温柔,道:“其实来这里的途中我已经答应了他的求婚,你愿意留下来参加我们的婚宴么?”

那男子喜道:“恭喜你了,江侯爷已经答应,过几日就放我离去,只怕没有机会参加你们的婚宴了,替我转告呼延将军,就说我祝你们白首偕老,永结同心。”

陆云在下面听得目眩神迷,此刻他早已听出这两人身份,大雍澄侯苏青,龙庭飞麾下四将仅存的段无敌段将军,这两人的事迹他也听父亲说过,想不到却在今日听到两人的密谈。若非是强行隐忍,他真想露出头去看看两人的风采。

这时,耳边传来远去的脚步声,想必是苏青离开了,那男子一声轻叹,叹息中却带着喜悦和宽慰,这时,冷月无声,影沉寒水。只听那男子低吟道:“伤心桥下春波绿,曾是惊鸿照影来。”语声凄凉,陆云虽然不甚明白其中深意,也觉得为之黯然神伤。

\chapter{第七章 何处是青山}

隆盛二年,青奉诏入长安,钦封侯爵,雍史女子封侯,自青始也。

隆盛七年,青下嫁虎贲卫副统领呼延寿,帝亲赐诏书许婚,因新人无亲族,令太子亲临主婚。

——《雍史·澄侯列传》

时间缓缓流逝,灯光依旧明灭,陆云等得焦心,这时,耳中传来轻叹声,平台有了轻微的震动,那男子似乎正向房内走去,陆云心中一喜,却听到一个女子惊喜地道:“段将军,果真是你?”然后陆云便感觉有人走上平台,而且听脚步声,似乎是两个人,陆云差点想抱头痛哭一番。

这时他听到那男子声音冷淡地道:“公主殿下,好久不见,萧大人,别来无恙。”陆云心中一震,这才听出那女子竟是齐王妃林碧,那个萧大人是不是今日跟在王妃车驾旁边的那个萧总管呢,听嘉郡王的侍卫说,那个萧总管原本是北汉人,是随着王妃娘娘进入王府的,据说武功十分高明,只是不大管事,也不怎么抛头露面。

林碧叹了口气,道:“我来之前,便知道你会这个样子,可是怨恨我没有坚持到兵败人亡么?”

段无敌冷冷道:“其实大家早已知道,晋阳不过是苦撑罢了,国主请降,倒也成全了千万军民,我们作臣子的,也只能接受罢了,虽然转眼间大家都高官厚禄,荣华富贵,忘记了曾为社稷牺牲的沁州军民,这也是人情之常,更别提有人忘却旧情,嫁与仇敌,去享王妃的尊荣。”

林碧没有说话,只是一声长叹,声音充满了惆怅,另一个男子的声音响起道:“段无敌,你太过分了,你可知道公主殿下的一片苦心,若没有公主委身下嫁,国主焉能安享荣华,我们这些人也将惴惴不安,公主正是为了我北汉军民宗祀,才毅然下嫁,再说龙将军临终之前也有遗言交待,你怎可如此无礼。”

段无敌的声音变得嘲讽讥诮,他扬声道:“是么,我去沁州祭拜将军,却是听到人人传唱俚曲‘昔日汉公主,今日齐王妃,遥望故将军,佳城郁郁深。’”

台上突然没有了声息,可是陆云却能够感觉的那上面凝滞的气氛,沉闷的令他几乎喘不过气来。不过他心中十分矛盾,明明觉得这位段将军原来非是那样软弱和气,而是绵里藏针,刚烈果决,却又觉得嘉平公主非是段将军所说的那般不好。忍不住用心听去,等待接下来的发展。岂料就在这时,一阵风吹来,那平台上的风灯突然一闪而灭,河面上顿时一片漆黑,陆云心中大喜,也顾不上继续偷听,迅速潜入水中逆流游去,几下子已经离开了平台的范围,身后灯光重新点燃。陆云回头望去,只见平台之上站着三人,林碧一身王妃服饰,明黄色的披风猎猎作响,神情却是惆怅感伤,在她身后,果然是那个消瘦阴森的萧总管,而在两人对面,站着一个布衣中年男子,相貌儒雅,满面风霜,虽然只是那样随意的站着,但是身姿笔挺,犹如青松白杨,面上的神情冰寒震怒,威势凌人,怎也令人想不到,他方才会有那样温柔的语气,宽容的态度。陆云顾不得多想,时间不多,他奋力向上游去。

平台之上,林碧神情平静下来,淡然自若地道:“段将军责备的是,有些话林碧可以和你说个明白,虽然本来没有必要,可是你是庭飞生前的心腹之人,我也当你是自己人,所以不想瞒你,不错,我林碧的确委曲求全,下嫁杀夫仇人,这件事情无论如何掩饰也是没有用处的。可是我却不曾后悔,当初国破家亡,我可以自尽殉夫,也可以誓死不嫁,我相信没有人敢逼我成婚,可是林碧不是一个人,我是北汉的公主,代州的主将,我一死事小,大雍和北汉却要仇恨绵绵,难以化解,你也想我北汉百姓像东晋初年那样受尽排斥凌辱么?有件事情你并不清楚,庭飞为何当日战场许婚,不是他瞧不起我林碧,以为他死了我就没有幸福可言,非要托付给别人才放心,而是他当日就已知道北汉大厦将倾,唯一保全社稷黎民的法子就是请降,而且,他或许已经看穿了大雍迫降之意,也看穿了王上终会投降的结局,所以他留书给我,安排身后之事,要我不可为了仇恨放弃责任,这桩婚事是庭飞的心愿。”

段无敌怒道:“我不相信,我不相信龙将军会这样做,他留了什么书信给你,拿给我看。”

林碧淡淡一笑,从怀中掏出一个泛黄的鸳鸯荷包,那上面仍有未褪的血迹,她将荷包递给段无敌。

段无敌双手颤抖,接过荷包,他知道这是女子送给情郎的定情之物,当年苏青也曾送过给他,只是十三年前绝裾而去之时,那荷包也被她丢入火中焚毁。荷包中往往会放上一绺青丝,以示千里随君之意。他打开荷包,果然看到一绺青丝,然后他便看到一幅白绢。取出白绢抖开,上面是一幅血书,钢筋铁骨,正是龙庭飞的字迹。

“卿见此书,庭飞业已为国舍身,死虽无恨,仍念汉家江山,身后无人,唯有托付于卿,卿且忍辱负重,不可为私仇情恨,断绝君臣之义。”

段无敌手一抖,白绢飘落在地,林碧上前拾起,望着那白绢,眼中闪过悲凉之色,道:“这封血书是庭飞暗中交给萧大人,令人在适当时候交给我的,舅父请降之后,萧桐将血书给我,当时我尚不明白他的意思,后来得知庭飞临阵许婚之事,我才明白。庭飞他或者曾经被蒙蔽了许久,可是临危之际,他却心地清明,他已看穿了一切。他很清楚舅父请降之后可能会面对的困境,解决这个问题只有联姻,而我林碧不幸身为北汉公主,舅父唯一的支柱,我若不嫁入皇室,就没有可能消洱仇恨,我不知他是否太狠心,为了王上的安全,北汉宗祀的延续,他忍心要我另嫁。如今想来,当日庭飞自绝,不是为了不肯受被俘之辱,而是为了彻底的尽忠,他,他竟是早已知道非死不可。”

段无敌抬起头,悲声道:“将军!”然后又道:“殿下,此事还有何人知道?”

林碧摇头道:“此事有关庭飞声誉,除了我与萧桐,没有别人知道,我原本要焚去血书,只是想到你或者会回来,所以留给你看,你是庭飞麾下四将仅存之人,若不能得到你的谅解,我心难安,庭飞泉下也势必不能瞑目。”

段无敌黯然道:“殿下委屈至此,我却出言相责,请殿下恕我之罪。”

出乎他的意料,林碧摇头道:“不,你骂得没错,虽然我答应婚事,是为了庭飞的嘱托,为了北汉的安宁,可是若不是李显颇有令我倾心之处,我也不会嫁他,我林碧若要嫁人,大雍皇室多得是好男儿,就是我想嫁入宫中,也少不了一个贵妃名号,我接受李显,只是因为他是个不逊于庭飞的豪杰,这些年我并没有委屈,李显待我情深意重,我从来没有后悔过。”

若是她方才这样说,段无敌只会冷笑而已,如今她这样说来,段无敌却是心中安慰,林碧嫁入皇室和亲,已经是定局,能够嫁给一个好汉子,英雄人物,已经是不幸中的大幸。

林碧取下风灯的纱罩,将血书在火上烧了,然后道:“无敌,事已至此,也无需多谈了,如今总算风平浪静,大雍陛下没有亏待我们,你之才华人品,庭飞和我素来敬重,何妨为朝廷尽一番心力,也算不负此生。昔日舅父冤屈了你,你若能在大雍封侯拜相,我也能心安许多。”

段无敌神情已经恢复平淡随和,躬身一揖道:“公主厚爱,无敌明白,只是无敌早已意冷心灰,更何况权势富贵本非所愿,一路行来,见到国泰民安,无敌已经很是满足了,所以我准备回沁州去,龙将军自尽殉国,苏将军身死雍都,谭将军战死沙场,石将军冤屈而死,昔日沁州众将,只有无敌一人尚存,无敌纵然厚颜,也不愿侍奉大雍皇帝。无敌一身,无牵无挂,不似公主,尚需担负万千军民的荣辱安危,所以无敌决意回沁州隐居,此事尚未得楚侯正式允准,尚请公主代无敌求情一二。”

林碧轻叹一声,说到这种地步,她自然知道段无敌心意不可更改,其实她也不想拦阻段无敌归隐,她只是担心江哲肯不肯放手,江哲此人,对敌之时狠辣非常,绝不给敌人生路,段无敌若是隐居乡间,在如今天下尚未一统的情况下,很有可能是隐患,她不知道江哲能否放过段无敌,若是段无敌等到十年之后再回来,或许就不会如此烦恼。可是林碧也清楚,去国离乡之苦,她身在长安,仍然常常想起雁门夜月,更何况段无敌是漂洋出海呢。

最后她轻叹一声道:“我会去向江侯说及你的事情,他应该会卖我一个面子吧,无敌,你今后准备到沁州何处隐居?”

段无敌淡淡道:“沁州认识我的人太多,我不想招惹是非,从前谭将军归葬故园的时候,我曾亲往送葬,那里是个好地方,当年我便说过有朝一日会去那里隐居,这次途中遇见几位旧部,他们已经解甲归田,我提及想到谭将军故里安居,他们已经先去了,如今想必正在披荆斩棘,重整田园。”

林碧又是轻叹一声,这几年来她的叹息也不及今日之多,谭忌死后,虽然北汉也有封赏,可是谭忌并不得北汉重视,他的身后事已经可以算是萧条了,北汉亡国之后,大雍对于北汉为国牺牲的将领,也都有所追封,可是谭忌因为曾在泽州大肆杀戮,所以被置之不理。想必谭忌的坟墓早已没有专人照顾,曾经为北汉出生入死的将领,身后却是凄凉非常,只是死者已矣,来者可追,这件事情关心的人并不多,毕竟谭忌的为人过于偏激,想不到段无敌仍然念念不忘,怎不让她心中愧疚。转身离去,林碧留下了一句斩钉截铁的话语道:“段将军且放心,有我林碧在,万万不能让人难为了你,谭将军墓前,每逢清明,请代我焚一拄香,是刘氏和我林碧对不住谭将军和你。”

费尽千辛万苦,陆云终于到了最后一座水榭,在第五座水榭,他在冰冷的水中浸泡了半天,早已经是手足麻痹,这最后的一段路,让他几乎支撑不住,看看和第五座水榭相似的格局,他终于笑了,第六座水榭里面,他看见了齐王妃的侍女,那么这座水榭,一定是江哲的住处了。看看没有完全关好的门扉和透过门缝的昏黄灯光,他警惕地打量一下四周,并没有看到什么侍卫,轻轻攀上平台,他伏地而行,贴着门缝向里面望去。

地上铺着毛毯锦毡,四周是垂纱帷幕,檀香轻飘,棋坪琴台,满架书香,隔着一扇锦绣屏风,后面隐隐是锦帐低垂,这是一间华贵舒适的居室,一眼陆云便确定,这一定是江哲的住处,只是室内寂然,似乎无人。他本来觉得这水榭没有一点戒备,若是躲入室内,应该可以等到江哲归来,骤下杀手,不免暗中欣喜,转念一想,若是自己这样登堂入室,必然留下水痕,江哲归来之时,侍卫稍一巡视就会发觉,可若留在门外平台上,若是有巡视的侍卫经过,恐怕一眼便会看到自己,想到此处,不由皱紧了眉头。

这时,陆云无意中目光一闪,看到屏风后面一张春凳上散落着一些衣衫,他心中一动,除下夜行衣,拭去身上水痕,将夜行衣塞到门口地毡之下,然后走入室内,拣了一件衣衫穿上,这件衣衫十分不起眼,想来一时片刻,不会有人发觉丢失。然后他转到屏风之后,闪身躲到床底,握好尖刀,等待江哲归来就寝。

过了不多时,另一面的房门打开了,两个人走了进来,陆云只能看到那两人的腿,前面那人身穿青衣,似是下人装束,后面那人却是青袍曳地,衣衫华贵,两人都没有走入屏风后面的内室,那衣衫华贵之人坐在锦墩之上,道:“公主已经和段将军谈完了么?”陆云心中一颤,知道这人正是江哲,他的声音清雅,语气温和随意,全然没有掌握重权之人的傲慢口气。另一人恭恭敬敬地道:“公主令萧大人传言,想和您见面详谈。”这个人的声音冰冷无情,但是又带着一丝温和,仿佛冬日里的一丝和风,陆云猜测这人定是邪影李顺,更是放缓了呼吸,不敢露出一丝声息。

那人站起身来,道:“公主相召,我们过去吧,想必段将军已经有了决定。”

这时门外有人冷冷道:“不必了,江侯爷,我林碧已经来了。”说罢两人推门而进,只听声音,陆云便知道是林碧和萧总管。

双方见礼之后,林碧开门见山地道:“江侯爷,我想请你网开一面,放过段将军,不知道你意下如何?”

江哲不紧不慢地道:“殿下有故旧之情,哲心中明白,只是段将军昔日乃北汉大将,皇上和齐王殿下对其都有留心,当日我宽释段将军之事,皇上得知之后虽然没有怪我,可是也是叹息不已,说这等名将,却被我放过了。”

林碧冷冷道:“当日你就是强留下段无敌,最后也不过是留下一个心死之人,他是绝对不会归降的。”

江哲淡然道:“我清楚此事,沁州军皆是龙将军部将,忠于刘氏,且和大雍结下深仇,段将军又是择善固执之人,当日是绝对不会投降的,所以我终于放了他一条生路,幸好他也是守诺之人,没有辜负我手下留情的美意。”

林碧语气软弱了一些,道:“既然如此,今日你何必还要为难他,他是不会和大雍为敌的,他所求的不过是隐居田园。”

江哲笑道:“若是如此,只怕可惜了段将军的本事,他若肯归降,必能封侯拜相,何乐而不为呢?”

林碧无奈地道:“段将军本是无心功名之人,他有意在谭将军故里隐居,你若不放心,最多安排些人监视就是,他如今心灰意冷,就算你强留他在朝中,也派不上用场的。庭飞和麾下四将,如今只有他一人尚存,他是不可能归降的,你应该清楚,沁州、泽州两地军民之间的仇恨,想要化解不是十年八年的事情,段将军既然无心和大雍为敌,你若强行软禁他,只恐不妥。”

江哲似乎思索了许久,终于道:“既然公主殿下为他缓颊,我便再放纵他一次,不过殿下却要保证段将军不会生出反意。”

林碧淡淡道:“我们都已经降了,难道他还会树起叛旗么,他只是想寻个安身之地,他乡虽好,不是故乡,他这次冒险回来,想必没有料到这么多年,你还记着他的存在。”

江哲叹息道:“忠臣义士,永铭人心,我怎会忘记。段将军想到谭将军故里隐居,这样也是好的,谭将军身后萧条,有段将军照顾他的坟茔,最好不过。”

林碧闻言冷冷道:“当日将谭将军从武庙春秋祭祀中除名,你不也是赞成的么,若是当初你肯进言,焉能至此。”

江哲淡淡道:“谭将军为人我素来仰慕,那些朝廷的春秋祭祀虽然珍贵,可是谭将军的性情怎会看重,与其让人怀着恨意和不敬去祭祀他,倒不如让他在一个清静的所在好生安眠。”

林碧默然,只觉得此人所言倒也极有道理,时间已经太晚,既然段无敌的事情已经解决,她起身告辞。临去之时,林碧突然问道:“江先生,南楚陆灿是你的弟子,有朝一日,两国交兵,你将如何待他?也会是这般斩尽杀绝么?”

江哲似乎犹豫了一下,道:“我自然希望保全他的,只是我这个弟子心性坚毅,只怕是死而后已,我虽然希望他至少能够像段将军一般归隐,恐怕也是没有可能的。”他并没有正面答复,可是其中的含义却很清楚,陆云心中一寒,更是握紧了匕首。林碧闻言微微一笑,转身离去。

\chapter{第八章 绿杨芳草}

隆盛七年,无敌归中原,隐于沁、泽群山中,于谭将军忌故里结庐,终老不出。

——《北汉史·段无敌传》

忌殁后,社稷倾覆,雍帝令礼部录北汉殉死众将名姓,准入武庙,享春秋祭祀,忌凶名过甚,礼部上书请除其名,雍帝许之。

——《北汉史·谭忌传》

林碧两人离去之后,陆云听到那冰冷的声音道:“碧公主似乎也知道了。”

江哲笑道:“想必是李麟这小子口风不严,跟碧公主抱怨过了。无妨的,你去吧。”

然后陆云便听到有人推门而出的声音,他心中大喜,江哲一人在此,可真是天赐良机,又过了片刻,见江哲并未安寝,他轻手轻脚地从床底钻了出来,只见江哲背对着自己坐在那里,灰发青衣,一手放在旁边的小方桌上,另一手拿着书卷。陆云缓步上前,正欲下手刺去,不知怎么他突然看到江哲的那只右手,食指在桌上轻轻敲击,十分悠闲自得的模样。

心中灵光电闪,陆云突然丢下匕首,拜倒在地,朗声道:“陆云拜见师祖安好。”

江哲正在敲击方桌的手指突然停住了,他缓缓回过头来,道:“起来吧,你这一路上辛苦了。”

四目相对,陆云一眼便看到那双温和平静,却幽远深邃的眸子,他甚至看到了这星鬓朱颜的男子唇角的一丝笑意,心中只觉得如释重负,自己果然没有料错,这人分明知道自己的行踪。

我看看陆云身上不合体的衣服,微微一笑,扬声道:“小顺子,可以进来了。”

水榭的房门再次打开,小顺子走了进来,岁月的流逝在他身上并没有留下明显的痕迹,如冰似雪的容颜和七年前并没有什么分别,只是那双眼睛越发深沉冷静。他冷冷地望了陆云一眼,道:“公子何必还要对这小子留情,他竟敢筹划刺杀公子,罪不容恕,就是公子不想将他送给明鉴司,也当让他尝尝公子夺魂金针的味道。”

见到陆云神色尴尬,我笑道:“小顺子,就别吓唬他了,若不是你暗中相护以真气相护,他哪里能在段将军居住的水榭藏身那么长时间,凭他这点武功,不说苏侯和萧大人,就是段将军和碧公主,他能瞒过谁的耳目。”

小顺子微微一笑,道:“虽然被冷水浸了半天,可是这小子倒是听到了不少隐秘,要不是后来的事情我觉得他不适宜听到,又看他急得可怜,也不会熄灭灯火,让他可以脱身了。”

陆云惊骇地看着小顺子,他虽然知道自己落入江哲算计,可是也万万料不到这人竟然一路上跟着自己,怪不得那灯火熄灭得那么及时,想到若是自己被平台上面的人发现,那些人说不定会杀了自己灭口,毕竟他们所说的事情肯定不希望别人知道。他可不认为那些人有一个是心慈手软的。想到这里,连忙对这小顺子拜谢施礼,小顺子微笑受了。

拜谢之后,陆云忍不住问道:“师祖,您是什么时候知道晚辈的来意的?”

我笑道:“那么你又是什么时候知道我是设下陷阱等你入彀的呢?”

陆云恭恭敬敬地道:“晚辈经常听父亲说及师祖往事,父亲曾说,昔年师祖闲暇时候最喜欢戏弄于他,初时父亲屡屡上当,后来却十次能逃过七八次。”

我想起往日,那可是一个不解之谜啊,那小子明明笨得很,可是我偏偏不能随心所欲的戏弄他,虽然因为我碍着西席身份,不敢太过火,可是那小子定是有些秘诀的,心中好奇之念涌起,我装作不甚在意的模样问道:“哦,原来是你父亲传了你秘诀,却不知我露了什么破绽?”

陆云自然不会在这个时候卖关子,连忙道:“父亲说,您每次若是想捉弄人,若是手放在桌面上,都会忍不住用食指轻叩书案,所以只要留意一下,就不会经常上当。”

我愣了片刻,原来如此,当日我若是在书房和陆灿较量,怪不得总是让这小子逃过去呢,还是我当年年纪太轻不懂掩饰,若是现在可就不会这样容易了,至于陆云这小子发觉破绽,纯粹是因为我今日根本就没有将他看得很重要。心中释然之后,我笑道:“你想效仿刺客行刺,还太嫩了些,你刚入长安就露了破绽,姑且不论这些,我和你父亲相识在二十年前,他当时年龄和你相仿,你和你父亲的相貌现在虽然只有五六分相似,可是和他少时却是一模一样,就是你如愿以偿的接近了我,只需一眼我就会看出你的身份。你是陆灿之子,又素有武勇之名,大雍明鉴司、司闻曹早有你的画像存档,若非是我令人替你掩饰,只是金谷园那一关你便躲不过去。”

陆云惭愧地低头不语,此刻他可是知道自己的幼稚了。

我继续打击他道:“你也是将门之子,如何为此荒谬之事,一个小孩子,妄想刺杀大雍重臣,你若失败自然是命丧雍都,就是成功了,难道不会挑起两国战火么,到时候纵然你父亲在南楚可以一手遮天,也不能护你平安,莫非你以为南楚现在愿意和大雍一战么?”

陆云头上冷汗涔涔,他此刻才知道自己犯了多大的错误,若是大雍利用这个借口挑起战火,自己就是南楚的罪人,父亲也要受自己连累。

我叹息道:“你以为你的父亲当真是为了私情对我这般恭敬么,你可知道我的性命曾经险些葬送在他手上?你父亲不过是希望我能够看在故旧之情,不在大雍铁骑南下之时出谋划策罢了,留得一分情谊,总好过撕破脸皮。当日我便猜到你来此定是为了替你父亲除掉我这个背叛君父的师父,便觉得你年轻气盛,将来定会给你父亲惹来无数麻烦,因此便设下三重考验,你若能够通过,可见你还有些长处,我便饶你一次,你若当真是鲁莽无能,我拼着你父亲怨恨,也要取了你的性命。你父亲在南楚如履薄冰,若是你再不能体谅他的苦衷,不死何为?”

陆云如梦初醒,从前的种种疑惑都有了答案,父亲之所以对眼前此人那般恭敬,不是为了旧情,而是为了歉疚,想起自己从前对父亲的误解和指责,当真是痛悔交加,忍不住伏地痛哭起来。

我见这少年已经知道自己的错误,语气放缓了许多,道:“我安排了三次考验,第一次便是李麟,他在金谷园召见你,你若是不中他的意,便是武技平平,还敢前来行刺,便是庸才,杀了最好,免得连累你的父亲,不过你果然算得上少年英杰,百步射柳,在你这般年纪,箭术已经是很出众了,这第一次考验你过得很顺利。第二次考验就是临波亭之内,我原本想看看你会不会心狠手辣地要伤害柔蓝,若是你这般狠毒,霍琮便会奉命将你处置,可惜柔蓝毕竟是我的女儿,为了避免和你直接冲突,居然自己跳入水中,所以这第二关也勉强算你过了。第三关就是今夜,你要是想不到溯流而上寻到我的寝居,就是才智不足,我也要治你的罪。你既然有胆量来大雍行刺我,若是武功、才智、品性说不过去,我杀你也没有什么可惜的,不过你运气不错,三关皆过,如今你的性命是保住了,总算不愧是陆灿的爱子。”

陆云止住泪水,满面通红的听着,不由庆幸自己当日没有来得及伤害柔蓝,不过另一种情绪涌了上来,他忐忑不安地道:“师祖,莫非他们都知道我的身份了么?”

我笑道:“怎么了,没有颜面和他们相见了么,虽然当时不知道,不过如今都知道了,要不然你以为为什么李麟昨日发那么大脾气?”

陆云心中又是惭愧,又是难过,虽然今日之前他还是将柔蓝等人当成仇敌,可是不可否认的,对于霍琮、柔蓝,甚至李麟和李骏,他都是好感多些,今日既然行刺已经彻底失败,他也就放开胸怀,不免有些担心这几人瞧自己不起。我见他神情便知他心意,不由暗暗欣喜,我之所以费心让几个孩子主导这个圈套,就是希望影响陆云的观感,影响他的心志,甚至是陆灿的心志,这种微妙的感情对于国仇家恨或者没有什么作用,可是一旦到了烟消云散的时候,往往会起到决定性的作用。我特意让他有机会见到段无敌,就是希望能够在最后关头影响陆家的选择,我是不指望陆灿弃暗投明,只是希望最后能够保全陆家的血脉。这点私心我当然不会说出来,只能通过潜移默化的方式着手。

陆云羞愧难安,他原本是怀着恨意而来,可是来到长安之后,才发觉江哲或许不是南楚流传的那样无耻,他若是那样的人,为什么那么多人对他都是那般敬重,就是畏惧中也存了敬慕,还有若是江哲果真是流言所说的那般贪图荣华富贵,为什么从柔蓝、霍琮身上却看不到丝毫纨绔子弟的缺点,事实上,他对江哲的仇恨早已淡化,只是他一直没有发觉罢了。方才准备行刺的时候,若非是他心中杀意不浓,又怎能发现江哲的小动作。

可是望着江哲儒雅风流的身影,陆云却是难以表露孺慕之情,毕竟这人是大雍重臣,他在李麟身边多日,隐隐感觉到大雍可能很快就会南征了,到时候凭这人显露的狠毒手段,只怕自己的父亲即将万劫不复,心中一痛,陆云突然再次落下泪来,这一次他却没有哭出声,只因心头仿佛刀割一般,望着江哲的目光模糊迷离,却是什么也不能说。

我轻叹一声,知他心中矛盾,但是各为其主,两国交兵,这件事情我是无能为力,就是陆灿也是无能为力,更别说陆云一个小孩子了,将手一伸,小顺子立刻将一个玉瓶放到我手中,我上前搀起陆云,道:“你今日受了寒气,若是不好生拔除,将来必有后患,这瓶药可以固本培元,你每天晚上服一粒,连服一月即可,剩下的药物你就留在身边,若是受伤初愈,服用此药,必有好处。前日你爹爹已经派了家将来见我,知子莫如父,他也猜到你会前来行刺我,所以派人一路寻来,他们在我府上等你,你见了他们就回去吧,别让你爹爹为你忧心。两国征战的事,你一个小孩子插不上手的。”

陆云心中一宽,他不是没有担心眼前这人利用自己胁迫父亲,虽然知道父亲定然不会屈服,可是必然会有人利用这个机会打击父亲,更何况父亲必然会因此难过伤心,若是如此,他纵死也不会安心。抬头看向那双充满慈爱的眼睛,他扑到江哲怀中啜泣起来。

我怀抱着这个少年,心中感慨万千,我不是不可以利用他在长安的事情施展我最擅长的计策,可是一点私心终于还是让我放弃了,希望大雍铁骑犁庭扫穴之后,这个孩子能够留得性命,能够想起长安还有他的依靠。

第二天,林碧最先离开了南山别业,李麟自然随行而去,陆云却被留在江哲身边,他也想寻个机会向李麟致歉,可是李麟根本就不理会他,奉着林碧的车驾扬长而去,陆云也只能黯然失落罢了。

苏青和呼延寿是第二波走的,陆云寻个机会,他很想见见这位名扬天下的女侯爷。当他看到苏青的时候,即使是他这般年少,也不由惊呆,遇雪尤清,经霜更艳,那是霜雪摧残后傲然挺立的寒梅的风姿。而她旁边那位将军,不论相貌还是气质都有些黯然失色,陆云不由有些奇怪澄侯苏青为什么会选择这样一个夫婿。直到他无意中看到苏青转头和那位将军说话的画面,那男子面上的神情是那样的专注,那是呵护至宝的神情,而苏青的神情是那样的柔和平静。虽然不甚明白,可是陆云却已知道,只有这样的男子,才能最好地保护一个半生凄苦的女子。

陆云没有看到段无敌离开,因为当日下午,他就跟着江哲离开了南山别业,回到江哲府上,陆云见到了父亲秘密派来的家将,含羞带愧地被两个看着自己长大的家将委婉地教训了一顿,第二天他的行装就已经准备好了,临行之前,除了霍琮执意送他到灞桥之外,他没有见到柔蓝和李麟的身影。

看到陆云怀着期望而又有些愧疚的神色,霍琮微微一笑,折了一支杨柳递给陆云,道:“你别介意,他们年纪轻,不免气盛些,其实主要是觉得被你瞒过了,所以不开心,其实他们并没有怪你。”

接过柳枝,陆云叹了口气道:“总是我的不对,这些日子多谢霍大哥照料了,本来嘉郡王送给我的那张弓我想亲自交还他的,如今只能拜托霍大哥了。”

说罢,陆云将当日李麟送给他的弓箭递给霍琮,霍琮叹道:“你这又何必呢,嘉郡王不会这样小气的。”

陆云坚持地道:“请转告嘉郡王和昭华郡主,陆云欺骗他们并非本意,此去千里,可能再无相见之期,郡王厚爱,陆云无以为报,只能归还弓箭,郡主那里,请替陆云致歉。”

霍琮正要说话,突然远处烟尘滚滚,霍琮心中一动,转头一望,笑道:“有什么话,你去和他们亲口说吧。”

陆云心中一震,举目望去,那策马而来的不正是李麟和柔蓝么,他心中一热,几乎要落下泪来。两骑骏马停在长亭之外,李麟和柔蓝纵下马来,将马缰一甩,便双双走到陆云面前。

李麟看了一眼陆云手中的犀角弓,恶声恶气地道:“本王送出去的东西什么时候要往回收了,一张破弓而已,难道你都不敢拿么?”

陆云看了李麟一眼,终于将弓箭交给家将,然后上前一揖道:“这些日子多蒙郡王照顾,陆云多有欺瞒,还请郡王恕罪。”

李麟苦笑了一下,道:“罢了,如果不是有人帮着你,本王怎会上了这么长时间的当,这不关你的事情,谁让有些人就知道助纣为虐。”说罢,他瞪了霍琮一眼。然后有些遗憾地看了一眼陆云,李麟继续道:“你怎么偏偏是陆将军的儿子呢,若是换了另外一个人,本王一定将你留下来,我皇兄对你可是颇为赏识呢?有些事情我不说你也知道,说不定将来在沙场上我们还能碰面呢,到时候你若败在我手上,可不能寻死啊。”

陆云苦笑了一下,他怎不知道当前的局势,大雍的贵胄都在这里摩拳擦掌,可是南楚却是文恬武嬉,大部分都在醉生梦死,可是他是陆家的嫡长子,焉能屈服,他抬起头昂然道:“王爷此言差矣,我南楚虽然势弱,可是尚有半壁江山,大雍铁骑若敢南下,我陆云定然披甲上阵,就是死也不会看着社稷颠覆,陆云虽然有愧郡王爷厚爱,可是将来若是沙场相见,也断然没有相让之理。”

李麟面上露出愤怒和敬佩混杂的神色,正要再说些什么,这时柔蓝抢上前来,推开李麟,伸出右手,巧笑倩兮地道:“还是没有影子的事情,别吵了,陆云,本郡主的金环丢了,想来想去都是被你拣了,如今你要回去了,还不还给我。”

陆云面上一红,望望李麟闻言突然露出的怒容,以及霍琮了然的笑容,恋恋不舍地从怀中取出金环,那仍然沾着他体温的金环在阳光下眩目耀眼,陆云一狠心,将金环向那只纤纤素手中放去。柔蓝接过金环,突然噗哧一笑,这一笑让陆云立刻忘记了身在何处,这时柔蓝又将金环塞到他手中,道:“算了,一只金环罢了,听说你还有个妹妹,今年也有七岁了吧,这金环你替我送给她吧。”

陆云接回金环,不知道该说些什么,这时候,家将催促道:“少爷,我们还要赶路呢。”

陆云心中一震,将金环塞到怀中,对三人抱拳一揖,道:“诸位珍重,陆云拜别。”说罢转身上了骏马,也不去看三人的神情,扬鞭策马而去,耳边风声作响,陆云只觉得迎风的双眼一阵迷离,忍住心中悲伤,他心道:“爹爹,我回来了,回来和你一起守护家国,死且不悔。”

\chapter{第九章 处处烽烟起}

大雍隆盛七年甲申秋,雍帝责南楚久不朝贡,诏南楚国主觐见,南楚国主陇闻之,惊惧莫名,数日不朝,辞以疾。雍帝闻之怒,誓师南征,三路大军齐发,再起刀兵。

——《资治通鉴·雍纪三》

同泰十一年,雍军南下,云未之行。

——《南朝楚史·江随云传》

大雍隆盛七年,南楚同泰十一年,十月初二,南楚江夏大营中军校场之上,军士们正在练习骑射,不时传出彩声如雷。

“咻、咻、咻”,连珠三箭射中了靶心,校场之上再度响起一阵欢呼之声,那射箭之人身材不高,身穿银甲,坐下的黄骠马乃是千里挑一的骏马,飞马奔射,箭箭中的,这样的箭术确实值得众军士欢呼,更何况那骑士正是他们尊重爱戴的主将长子。

直到射完了一囊箭矢,那个骑士才停了下来,二十四支箭矢将靶心掩住不留一丝空隙,他摘下头盔,露出犹带稚气的面容,擦拭了一下头上的汗珠,策马走到校场边上,跳下战马,磨娑了爱马片刻,才对围上来的军士笑道:“好了,射一轮箭舒服多了,将军还没有升帐么?”

几个军士笑道:“少将军,你的箭术越来越出色了,大将军凌晨时分才回来,今日可能不会升帐了。”

少年闻言一皱眉,道:“最近那边动作频繁,大将军这次去建业不知道情形如何?”

一个军士闻言道:“少将军不如私下去问问杨参军,大将军不肯告诉你,或者杨参军会露些口风的。”

那少年斥道:“胡说,若是杨参军那么容易套出口风,大将军哪里会这样信任他。”

另一个军士突然道:“对了,韦先生方才来了,如今已经去见大将军了。”

少年一皱眉,韦先生,他怎么来了,此人一向是无事不登三宝殿,想到这里,他也顾不得满身的汗水尘土,匆匆和几个军士交待了一声,便向父亲的营帐奔去。不多时,跑到了父亲的营帐,外面的亲兵见了他正要出声召唤,却被他摇头阻止,拉了一人低声问道:“大将军和韦先生在里面说话么?”

那亲兵点头道:“是啊,来了半天了,大将军也是的,对这人何必这么客气呢?”

少年瞪了他一眼道:“你知道什么,若没有这人从中斡旋,大将军和那老狐狸早就闹翻了,再说他在大雍消息灵通,若没有他帮忙,想等到兵部将情报送来,哼,只怕雍军都过了江,情报还未来呢。”

那亲兵低声嘟囔了几句,这少年虽然是少将军,可是素来和他们打成一片,所以他才敢和这少年说出心里话,他也知道这少年虽然责备自己,却没有恶意,也不会说出去,所以只是抱怨了几句,毕竟在他看来,那韦先生乃是犯上逆伦之人,他虽是粗人,却是无论如何也瞧他不起的。

少年在门口转了半天,还不见父亲出来,终于忍耐不住,凑到营帐门口,侧耳听去,那些亲卫互视一笑,挤眉弄眼,只作不见。那少年顾不得理会他们,只是极力捕捉帐内飘出来的微弱语声。

营帐之内,陈设甚是简朴,除了简陋的行军床,一张方桌和两把椅子之外,几乎是空空荡荡,除了桌案上面放着几卷书册之外,这营帐和普通的低级将领的住处几乎没有什么不同。

一个三十出头年纪的男子负手站在帐中,望着悬挂在营帐壁上的一副地图,神色沉重。这男子相貌英武,气质斯文,可算的上是俊逸人物,只是两鬓微霜,神色间带着沧桑之色,若非是他一身戎装,真让人不敢相信他是南楚军方的第一人。另一人相貌雍容俊雅,看去上不过三旬年纪,神色间带着淡淡的嘲讽,见他风采气度,断然不会想到他已经是三十五岁之龄了,而那个戎装男子明明小他三岁,却是显得比他苍老些。

见那戎装男子沉默不语,雍容男子冷笑道:“你还看什么,这次雍军定是不会放过这个机会,除了你的国主之外,天下谁不知道大雍这次是趁机寻衅,准备南下牧马,北汉灭亡已经整整七年了,大雍已经消化了北汉的领土人力,李贽的年纪也不轻了,难道他不想在有生之年看到天下一统了,卧榻之畔,怎容他人酣睡,就是南楚没有丝毫违逆过错,大雍也不会放弃南下的意图。前些时候,少将军从北面回来,不是说得很清楚,大雍连一个少年郡王都盼着上阵厮杀,南侵之意昭然若揭,你还不省悟么?若非见你还有几分气魄,七年前敢于背着南楚君臣袭取葭萌关,我怎会替你尽力,现在凤舞堂燕首座和仪凰堂纪首座,和尚维钧那老狐狸合作的甚是默契,虽然不便明目张胆地登上朝堂,可是已是逐渐权倾朝野,若非是尚维钧尚存一丝戒心,又有我辰堂替你张目,只怕你这大将军也很难坐稳位子了。”

戎装男子叹道:“韦兄厚谊,灿心中明白,若无你周旋,只怕也不能和那些人共处朝堂,前些日子,她们提出联姻之事,被我拒绝,然后尚相便故意拖延粮饷,若非韦兄相助,只怕这一关我就过不去。”

那雍容男子闻言叹道:“其实这与我无关,你掌握着南楚七成以上的兵力,尚相如何不清楚,我只是给他们寻个台阶罢了,其实你不肯让少将军和她们结亲,也是对的,她们在大雍的所作所为谁不清楚,就是我也看不过眼,对外一塌糊涂,内斗倒是一把好手,你说我助你,其实若没有你的支持,我的辰堂早就被她们压制住了,毕竟经济大权被她们掌握了,我们也是互利罢了。陆大将军,你若肯起义兵,清君侧,我便助你一臂之力。”

戎装男子苦笑道:“韦兄,若是再说下去,只怕我只能送客了。”

那雍容男子大笑道:“知道你不会答应的,你若有江哲的五分心狠手辣,也不会被尚维钧逼得离开建业了。”

那戎装男子微微一笑,道:“这几年韦兄似乎对家师的恨意少了许多,提起他的时候,也不会咬牙切齿了。”

那雍容男子冷冷道:“庆王覆灭、北汉灭亡,虽然是大雍兵多将广,李贽深得人心,李显英勇善战,可若不是此人运筹帷幄,哪里这样容易,我自知不可能和他匹敌,想来唯有一个法子向他报复,他不是叛楚投雍么,我便投了南楚,他不是想要助李贽一统天下,我便要让南楚割据半壁江山,纵然不能亲自取他性命,也要让他不得安宁,若非如此,我何必和你合作,只凭你和他的关系,我就应该和你为难才是,只是南楚却无人可以替代你,我也只能将就了。”

戎装男子不以为忤,只是淡淡一笑,面前此人,也只有自己敢于重用他,既然有着同样的目标,那么这人就是可信的,即使他人品有些缺憾,为了南楚大局,他也不会介意了。

雍容男子或许是发泄了一阵,轻松了许多,又道:“这次大雍遣使斥责,说南楚三年不曾朝贡,我已经查过了,说起来真是啼笑皆非,伏玉伦也当真是胆大包天,同泰九年他奉命去雍都进贡,途中被盗匪劫持,那些盗匪夺去贡品,却给了他伪造的回书和一半赃物。此人畏惧加罪,居然瞒过此事,接下来两年更是食髓知味,和那些盗匪内外勾结,分了贡品,伪造国书。而大雍三年来往来文书从来不提及此事,却于今年发难,要国主去雍都谢罪,若是这其中没有阴谋,我可不信。”

戎装男子淡淡道:“伏玉伦有才无德,软弱贪财,又仗着尚相的权势胡作非为,不过这种事情,如无人挑唆威逼,他也不敢做的,一旦上了贼船,更是没有办法回头,想必大雍也是费尽心思布了这个局,筹措三年,就为了今日东窗事发,兴师问罪。”

雍容男子叹了口气道:“谁说不是呢,偏偏伏玉伦在尚相庇护之下,他截下的贡品,倒有一半给了尚相,还有一成给了纪首座,若非碍于纪首座和燕首座,我怎会如今才知道此事,也不会一点准备都没有。妇人误国,古人诚不欺我,为了这些蝇头小利,居然无视大局,恐怕她们原本还在得意可以损害大雍的利益呢?也不想想,这种事情,难道大雍会视而不见么?我今日方知被仇恨和欲望蒙蔽了眼睛是多么愚蠢,若是我当年有此见识,或许不会到了今日,有家难投,有国难奔,孑然一身,形影相吊。”

戎装男子皱眉道:“我去建业见尚相,国主已经数日不朝,我请尚相斩了伏玉伦向大雍谢罪,尚相却不肯答应,只是贬去伏玉伦官职罢了,这等时候还要护短,唉。”

雍容男子神色讥诮,没有说话,良久才道:“杀与不杀,都已经晚了,这次是难得的良机,大雍不会错过的,尚相已经遣使送去谢罪书,但是我看也没有什么用处,说不定现在大雍就在誓师出兵了。”

戎装男子正要说话,帐外突然传来喧哗之声,他眉头一皱,便已听到很多人匆匆而来,还高声喊道:“大将军,大将军,葭萌关信使求见。”

戎装男子闻言一叹,掀起帐门,向外走去,外面的亲兵都是躬身施礼道:“大将军!”戎装男子向躲在亲兵后面缩头缩脑的爱子瞥了一眼,冷冷道:“陆云不尊军令,私窥营帐,拉下去重责五板。”

那少年正是陆云,闻言吓得跪倒在地,道:“属下知罪。”其他的亲兵也是凛如寒蝉,不敢替陆云求情,他们也有防范不严的罪名,若是大将军将他们一并责罚,不说丢人现眼,难道让别人保护大将军么?

陆灿也不理会他们,迎上匆匆赶来的参军杨秀和一个风尘仆仆的信使,信使上前拜倒道:“属下奉余将军之命前来禀报军情,九月二十三日,汉中节度使秦勇督众猛攻葭萌关,八百里加急早已上呈兵部,可是兵部至今没有回书,余将军命我前来请示大将军。”陆灿神色不变,但是眼中闪过一丝厉芒。

正在这时,一个斥候飞马入营,跌跌撞撞地扑到陆灿前面,道:“大将军,容将军有书信至,长孙冀大军前锋已到南阳,徐州军也已经南下,请大将军及早定夺。”

营中众将都已匆匆赶来,听到斥候所说,都纷纷上前道:“大将军,朝廷还在争吵不休,如何治罪,如何议和,如今雍军已经南下了,大将军难道还要等待国主的旨意么?”

陆灿环视四周,他那双本来显得疲惫沧桑的双目,仿佛顷刻间爆发出凌人的气势,接触到他目光的将领军士都是不由躬身施礼,陆灿朗声道:“大雍图江南之心由来已久,自显德二十二年,李贽劫掠建业,掠先王百官,夺子民金帛,血流成河,生灵涂炭,十一年来,雍军时刻窥视江南,令我江南军民,无一日可以安寝,今日陆某决意一战,众军可愿随我戮力死战,以保社稷百姓!”

众将闻之,皆振臂高呼道:“雍人残暴,十年旧恨,永生难忘,愿随大将军死战!”

陆灿大笑道:“如此击鼓升帐,杨秀,代我传令各军,从此刻起,各地军情先送到我这里,还有替本大将军上书国主,请旨迎战。”说罢,陆灿一挥锦袍袍袖,向中军大帐走去,众将都是满面喜色,连忙跟在后面而去。

那雍容男子走出陆灿的寝帐,露出了阴冷的笑容,心道,陆灿啊陆灿,不知你是聪明还是愚蠢,平日谦冲退让,战时却又独断专行,对尚维钧的胡乱指挥置之不理,只是今次大战非同寻常,等到雍军退后,只怕你就是想要和尚维钧和平相处,也是不可能了,只是不知你的忠心能够持续多久呢?

等到受过军棍之后的陆云赶到大帐,军议已经开始,他也是陆灿的亲兵,又是陆门嫡长子,自然可以旁听,悄然溜到大帐一角,他仔细倾听起来。这时参军杨秀正在慷慨陈辞道:“大将军,这次雍军分三路进攻,汉中秦勇猛攻葭萌关,秦勇此人,乃是雍帝亲信,雍军秦程一系如今的主要人物,曾有救驾之功,为人又是沉稳持重,对大雍皇室忠心耿耿,四年前,雍帝将其任命为汉中节度使,在南郑设立行辕,就是为了重夺葭萌关,进攻西蜀,然后顺江而下,取西陵、荆门等地,但是这一路关山险阻,雍军纵然势大,也不能一蹴而就,余将军定可守住,这一路,我们便不需担心。第二路,乃是长孙冀,此人乃是雍帝未登基前的爱将,能征善战,北汉设伏围困龙庭飞就是此人手笔,虽然龙庭飞以身做饵,再有代州军为先锋冲阵,逃出生天,可是北汉最精锐的沁州军大半毁在他的手上。此人既已到了南阳,那么这次必然主攻襄阳,容将军自德亲王之时便镇守襄阳,地利人和无不占据,也必然能够抵挡长孙冀。第三路裴云,大雍势力最盛的时候,此人曾在淮南和大将军对峙,其时若非襄阳、江陵皆在我手,只怕此人早已心存渡江之念。同泰五年,雍军泽州大战取胜后,开始反攻北汉,当时大雍东川不稳,北线胶结,此人方退到淮北,坐镇徐州。此后七年,大雍养精蓄锐,但是此人在徐州日日操戈,雍帝更是亲封其为淮南节度使,如今大雍大举南下,裴云对淮南十分熟悉,只怕会是势如破竹,大将军若想破坏雍军南征攻势,必须迅速击败徐州军,然后驰援襄阳,到时候雍军两路皆退,则汉中之敌不战自退。”说完之后,杨秀和陆灿交换了一个眼色,杨秀坐到陆灿右侧下首,等待众将提出意见。

众将听了杨秀之言,都是连连点头,一个五十多岁的老将起身道:“大将军,江夏大营和九江大营如今皆在大将军直接指挥之下,余将军和容将军也遵从大将军号令无疑,若是裴云走淮南,我们自然不惧,可是若是裴云顺汴、泗而下取淮东又该如何,淮东守军乃是尚相心腹骆娄真统率,素来和大将军不合,此人庸碌无为,绝不是裴云对手,若是裴云攻取淮东,侵掠淮扬,继而攻取建业,末将恐南楚再次承受昔年之辱。”

这老将是陆灿父亲昔年部将,陆灿素来敬重,在他起身时便示意他坐下慢慢讲,听完之后更是眉头紧锁,其他将领则是有的气恼,有的无奈。这骆娄真乃是尚维钧亲信的将领,昔年陆氏掌控军事大权,尚维钧本就心中不安,后来陆灿趁着大雍东川不稳,不顾尚维钧阻挠,悍然夺取葭萌关,尚维钧虽然事后也很欢喜,可是心中更加忌惮,镇远公陆信病逝之后,尚维钧想要夺取江夏军权,未能得逞之后便趁着雍军收缩防线,在淮东安置自己的亲信,骆娄真就是其中最得尚维钧信任的将领,如今是尚维钧的侄女婿,镇东将军,职位和襄阳容渊同列,还在葭萌关余缅之上。其实骆娄真此人吹牛拍马还行,若是论起行军作战,还不如江夏大营一个普通将领,若是裴云攻略淮东,还真是一件麻烦的事情。

陆云凝神想了片刻,道:“唯今之际,雍军南下已成定局,尚相无论如何也不会在这个时候和我为难,待我写封书信给骆将军,交待他一些事情,若是他能够照着做,淮东尚可以安稳,若是他不从良言,我也只得请了旨意去淮东接管他的军权了。”

众将面面相觑,虽然这是唯一应对徐州军入淮东的办法,但是对手中那点军权看得死死的尚维钧,能够允许这种情况发生么?

\chapter{第十章 帐下犹歌舞}

隆盛七年十月六日,徐州大营主将裴云自汴、泗南下,袭泗口。

——《资治通鉴·雍纪三》

十月五日,淮东楚州大营,夜色已深,中军帐内却是欢歌笑语,歌舞升平,南楚淮东主将骆娄真正和众将宴饮,大帐之内,十几名舞姬正在翩翩作舞,舞姿曼妙,轻薄的纱衣,隐约露出的雪白肌肤,都让帐中醉醺醺的将领和帐外守卫的军士看得目瞪口呆,嘴角流涎。坐在上首的骆娄真左拥右抱着两个十六七岁的娇美少女,不时的仰头大笑,两个少女媚笑着替他倒酒布菜,不时一个少女会用红唇渡酒,骆娄真来者不拒,醉意盎然地随着舞曲打着拍子,很少有人能够注意到,其实他的目光只是偶尔在那些舞姬身上掠过。对于这些任人采撷的女子,骆娄真并没有什么兴趣,他的注意力大半时候都在那些将领身上。几个高级将领身边也有花枝招展的少女相陪,那些中低级将领则是全部心思都放在那些艳丽的舞姬身上。骆娄真得意的一笑,他有俊逸的容貌,有高强的武技,唯一不具备的就是军略上面的才能,在得到尚维钧支持一日三迁,掌管楚州大营之后,为了巩固权位,他废了不少心思,用金钱美色笼络那些骄兵悍将,那些真正有才华的将领被他排挤出去,以免危及他的地位。靠着金钱美色和手中的兵权,楚州大营倒也是铁板一块,至少肆虐淮东,无人可挡。

骆娄真初时倒也有些自知之明,知道若是和世代将门的陆氏相比,自己根基太浅,对于尚维钧意欲扶持他对付陆家的心意虽然了然,却从来不敢真得得罪陆氏,除此之外,只是牢牢控制住淮东,对尚维钧惟命是从,尚维钧也知陆家不可轻与,因此骆娄真正好在淮东安居。至于大雍的威胁他本来倒也挂在心上,可是七八年没有动静,再加上周围围满了善于吹捧的小人,早已经飘飘然,基本上他已经忘却了大雍铁骑的厉害之处。

正在骆娄真觉得有些兴尽的时候,一个亲兵匆匆跑进来道:“启禀将军,陆大将军有书信到。”

骆娄真懒洋洋地道:“能有什么事情呢?让信使进来。”

亲兵犹豫地看了一眼大帐之内的糜烂景象,不敢提出异议,大将军陆灿乃是南楚职权最高的将领,骆娄真这般轻忽他的使者,这实在是有些失礼,再说听说大将军治军极严,若是给使者见到这种场面,也是不甚妥当,不过这亲兵知道自己若是说出来,多半会被骆娄真责罚一顿,所以也就只好领命引使者入见。

过了片刻,使者大踏步走入,一眼看到帐中景象就是眼中一寒,他施了一个军礼,道:“末将陆群,奉大将军之命送上书信,请骆将军查收。”跟在他身后进来的一个少年军士神色不动,随之行礼。

骆娄真一招手,一个亲兵上前接过书信,呈上给骆娄真,骆娄真看了哈哈一笑,道:“大将军也太过虑了,这可不是八九年前了,如今我军据有江淮蜀中,又有长江天险,雍军想要再像从前一般往来自如,那是异想天开,大将军的心意本将军领了,请回报大将军,末将奉了王命主管淮东军务,不敢有丝毫懈怠,至于大将军信上所说之事,本将军明白了,不过说到增援么,倒是不必了,我淮东七万之众,难道还不能应对雍军的进攻么?”

那使者乃是陆氏家将,见骆娄真这番话不冷不热,带着调傥轻视,忍不住火从心起,有心发作,身边那少年军士轻轻扯了一下他的战袍,那使者忍怒道:“既如此,请将军赐还回书,让末将带回。”

骆娄真不耐烦地对酒席上面一个文士道:“黄参军,你替我写封回书给大将军,写完了让他带回去。”说罢一指那使者,神态甚是倨傲无礼。这一次那少年军士面色也是一变,目中闪过杀机。

取了回书,使者和那少年军士转身出帐,直到出了辕门,仍然听到营中传来的缥缈乐声,那少年军士冷冷道:“回去需得告诉爹爹,若让骆娄真守淮东,雍军必定长驱而入,还是让爹爹准备收拾残局吧。”

陆群叹息道:“少将军放心,大将军早已知道骆娄真的为人,这次我们过来传信不过是尽尽人事罢了,后面的事情大将军定有解决之道,少将军和亲兵会合之后便去寿春吧,守寿春的石观将军生性严谨,大将军的军令是要你在十二日之前到达,若是违了军令,只怕他会打你板子的。”

少年军士忍不住神情微动,克制住去抚摸受刑之处的冲动,刚刚受了军刑,就骑马数日,这种滋味并不好受。

此时楚州大营之内,骆娄真逐走使者,正是兴致甚高,见席上将领已经心痒难耐,便大笑道:“罢了,歌舞已经尽兴,诸将同乐吧。”这正是众将期待已久的事情,见骆娄真在两个少女扶持下向帐外走去,一个早已忍耐不住的将领向一个舞姬扑去。当高级将领纷纷抱着艳丽的侍女走出营帐之后,本应是处理军机大事的中军帐内传来了*之声。

骆娄真满意地回到自己的寝帐,胡天胡地一番,便昏昏睡去,刚过三更天,突然有亲兵匆匆跑进来道:“将军,相爷的使者求见。”从睡梦中醒来的骆娄真吓得出了一身冷汗,虽然说逢场作戏是人之常情,但若是给尚维钧的使者见到自己这般情态,回去说上几句,必然下了相爷的面子,他的权势富贵皆是尚维钧所赐,又娶了尚维钧的侄女,是万万不敢得罪尚维钧的。连忙让亲兵将两个少女藏到别的营帐,自己匆匆用冷水洗了一把脸,亲自去将使者迎入。不过那个使者根本就没有理会骆娄真的满身酒气和其身上的胭脂花粉的香气,将尚维钧手书交给骆娄真之后便匆匆告辞而去。

打开书信之后,骆娄真只觉得仿佛一盆冷水从头泼下,那上面分明写着近来雍军可能进犯淮东,让他稳守淮泗口,不得浪战,退敌可也。

其实尚维钧写来这封信时仍然不认为雍军会大举南征的可能,这七年来,雍军固步自封,让尚维钧生出了错觉,据有江淮荆襄,蜀中防线也是固若金汤,再加上有长江为后盾,比起当年的一夕数惊,现在尚维钧完全相信南楚四十万大军可以保住江南半壁江山,北进中原的念头他是不敢有的,可是大雍断然难以南下的想法已经根深蒂固,不仅是他,就是建业百官,也多半没有戒心。因为尚维钧不仅对陆灿的上书毫无赞同之意,甚至还有反感之心。

前些日子,陆云失踪多日的事情早已经被尚维钧察知,甚至陆云在长安的所作所为尚维钧也知道了大半,本来有心趁机要挟陆灿,进一步夺取军权。但是心腹都劝他此事没有确凿的证据,不若暂时搁置,等到拿到陆氏通敌的罪证之后再发难不迟,所以尚维钧只是增强了对陆氏的监视而已。不过若非接下来陆灿深居江夏大营,几乎寸步不出,对南楚朝政噤口不言,就连陆云也被直接送到了营中,尚维钧是绝对会拿此事作些文章的。

在尚维钧看来,既然陆氏和长安暗通消息,往来不绝,若是大雍今年真的有意南征,陆云和陆灿的两个心腹家将根本不可能从长安平安归来,以己度人,就是自己也会留下陆云胁迫其父的,所以陆灿这般危言耸听多半是为了争夺军权。就是现在雍军在葭萌关下猛攻,在尚维钧看来,也不过是威慑罢了,毕竟贡品一事,确是落了大雍的面子,而且和江淮荆襄不同,葭萌关那里这些年来虽然没有大战,可是也不甚平静,再说,说不定余缅是奉了陆灿之命假传军情也不一定,纵然不是,凭着葭萌关天险,还挡不住雍军么?何况现在南楚的主力军队,葭萌关余缅麾下三万人和襄阳容渊麾下五万人,再加上江夏大营、九江大营各六万人,总共二十万都在陆灿直接控制之下,淮西五万守军虽然名义上不属于陆灿管辖,可是主将石观乃是陆信提拔的将领,对陆门一向十分尊敬,就是建业十万禁军,其中也有四万禁军倾向陆灿,剩下的那六万禁军战力不强,若没有淮东七万军队,就是改朝换代也不是不可能的事情。

不过为了谨慎起见,尚维钧仍然写了一封书信给骆娄真,毕竟有备无患也是好的,凭着淮泗口的地利,若是雍军果然攻淮东,将雍军攻势阻住应该不成问题,他还特意提醒骆娄真,若是战胜雍军也不可追击,免得激怒大雍,惹得大雍全军来袭,那可就是胜亦尤败了。

尚维钧的本意骆娄真自然不知道,相反的,因为对于陆灿的本事尚有些了解,再加上尚维钧的威势,让他立刻相信了雍军可能南征的消息,他想了半晌,大雍淮南节度使裴云坐镇徐州,本来就是针对淮东多些,从徐州顺汴、泗而下,首当其冲就是自己的楚州大营。想到这里,他怒道:“大将军的信呢,你们丢到哪里去了,快给我拿过来。”

有亲兵连忙将原本骆娄真弃之不顾的书信呈上,骆娄真颤抖着手打开书信,很快就看到了自己最关心的一段,原本的陈词滥调成了金石之言。

“守江必先守淮,淮东以楚州、泗州、广陵为表,可翼蔽扬州、历阳,两地若失,则建业危殆,将军大营镇楚州,北营镇泗州,南营镇广陵,则淮泗口本已无忧,唯泗口一地,乃泗水入淮之要冲,在楚州之侧,雍军南下,若不经泗口,无以侵楚州。将军若听吾言,分重兵镇泗口,略可保淮东平安。”

看毕书信,骆娄真大喝道:“立刻升帐,升帐,本将军要调兵。”

亲卫惊道:“将军,周副将、黄参军和诸位偏将,皆酒醉未醒。”

骆娄真焦急地挫了挫手,有心痛骂,却也知道自己才是罪魁祸首,想了半晌,道:“去找孙定来。”亲兵愣了一下,骆娄真已经是一脚将他提出寝帐,高声道:“还不快去。”那亲兵连滚带爬的去了。那孙定本是一个颇有才能的将领,只因性情耿直,屡次冒犯骆娄真,骆娄真将他从偏将贬为校尉,但是骆娄真毕竟还是有些眼力,知道此人才能,始终没有将他逐出淮东军,只不过对其不闻不问罢了,甚至有时还抚慰几句。今次到了紧要关头,他自然想起这人来。

过了不多时,孙定入见,此人不到三十岁年纪,相貌英伟,不似江南人物,只是在淮东数年,郁郁不得志,所以神情冷淡,进入帐内,他对骆娄真身上的酒气香气视而不见,躬身施礼道:“孙定叩见将军,请将军吩咐。”

骆娄真强作镇定地道:“本将军给你五千人,你立刻率军到泗口,接管那里的防务,提防雍军入侵。”

孙定一愣,他是校尉,只能率领千人而已,如何骆娄真竟然给他五千人。

骆娄真又道:“事情紧急,本将军暂且晋你偏将之位,等待查明雍军动静之后,本将军自会上禀朝廷,让你名实相符。”

孙定听了心中明白,定是雍军有了动向,骆娄真临阵无人,才想起自己,不过他也不介意,若有机会立下战功,何乐而不为呢,这骆娄真虽然妒贤忌能,但是倒有些好处,就是自己的战功被他夺了,至少这偏将之位是跑不掉了。所以孙定立刻凛然领命,出营点了五千军士,这五千军士有一营是他自领,素来训练严格,另外四营也都勉强可以使用,楚州大营没有骑兵,孙定带了五千人马披星戴月向泗口而去。泗口因为骆娄真的轻忽,只有五百人驻守,若是一旦雍军入侵,绝无守住的可能,孙定想到此处,也是心急如焚,急急赶去泗口。

将近泗口,已经可以看到南楚军在此的驻军营房了,这时候正是黎明时分,黯淡无光,孙定先令亲兵去通报泗口驻军的都尉,看到亲兵被营外巡视守夜的军士拦住盘问,孙定突然一皱眉,心中生出疑念。本来若是驻军之地,有军士巡夜最合理不过,可是孙定却偏偏知道现在守泗口的胡都尉是一个贪生怕死之辈,军纪松散非常,若非轮防泗口,更易提升军职,且七年来大雍从无举动,此人是万万不会到这个险地来的,若是他的营盘,凭自己这个心腹亲兵的本事,只怕走到营门,还不会有人发现呢,看看不远处的泗水,淮水,再看看沉寂森严的营盘,孙定突然生出一个古怪的念头,他轻轻传下军令,让军士们整理好甲胄兵刃,然后自己带了十几个武功出色的亲卫,缓步向那营门走去。

还没有走到营门,一个穿着什长服色的英俊青年带着五六个军士匆匆走来,迎上孙定道:“您就是孙校尉大人吧,我们都尉昨日受了风寒,现在还不能起身,属下田成,奉命前来迎接校尉大人。”

孙定目光落到那青年身上,口音、服饰、说辞没有一点问题,可是他心中却越发生出寒意,若是胡都尉手下有这样的人才,他倒要庆幸万分了,还有这青年面上的神情,是一种自傲、自信的神情,绝不是在淮东军随处可以见到的麻木和茫然神色,更重要的一点,这青年身上有淡淡的血腥气,这是孙定绝不会忽略的。他深吸了一口气,尽量平静地道:“既如此,请带路。”

那个青年转过身去正欲起步,孙定突然拔刀砍去,这一刀如同惊鸿掣电,又是背后偷袭,本来那青年是万万难以躲过,不料那青年似乎早有防备,身子向后便倒,急猛非常,但后背离地不足一尺之时,突然停止,仿佛斜插在地上一般,孙定挥刀下斩,那青年的身躯便直直挺起,同时拔刀反击,“铮”一声刀鸣,孙定被震退了一步,那青年已经脱出他的刀势控制,另外几个军士则是散开一些,将孙定和几个亲卫隐隐围住。

孙定叹息道:“好一式铁板桥,乃是少林正宗秘传,阁下是淮南节度使裴云裴将军麾下何人?”

那青年眉峰一扬,朗声道:“既然被你识破,我也不妨直言,我乃白衣营杜凌峰,裴将军乃是在下师叔。”

孙定虽然早有预料,仍然是神色一惨,白衣营乃是裴云亲手创建,江湖中人往往有桀骜不逊,不甚习惯军规国法的,裴云便建立白衣营招纳人才,入此营者拘束极轻,只需告知裴云一声便可解甲归田,若是有心功名,也可正式从军。此营中人身手都在一流以上,最多时也不过十八人,因为裴云身份的缘故,倒有大半是少林或者其他名门正派的杰出子弟,若有他们出现,便说明裴云对泗口是势在必得。这些人必是受裴云之命,暗中除去泗口守军,准备接应雍军南下,孙定心中苦涩非常。

但是孙定毕竟也是出色的军人,他立刻想通了一件事,既然杜凌峰有意诱使自己入营,那么说明雍军此地兵力不足,那么自己还有机会得回泗口。想到这里,孙定振臂高呼道:“杀!”

随着他的喊声,南楚军向营房攻来,那英俊青年亲自断后,退回营去,从营房里涌出数百人,列阵相迎,对这五千敌军,还敢列阵,孙定也是心中佩服,不过若是他们不出营就更好了,自己只需围住营房,使用火攻,便可取胜。

呼喝声中,两军开始了血战,泗口的重要,双方都是心知肚明,谁都没有丝毫犹豫,这一交锋,孙定不由更是担忧,他这边除了自己那一千军士,其余四千基本上战阵不熟,武艺不精,难以派上什么用场,人数虽众,却不能有效地压缩敌阵。而敌军虽然人少,却是个个骁勇善战,更有杜凌峰武勇过人,连杀数名南楚勇士,一时之间,战况居然胶结在一起。孙定担心雍军援军将到,不由一皱眉,本想速战速决,想不到反而被缠住了。他想了一想,便调出两百自己那一营的军士,让他们在外围射箭,这些军士熟习水战,弓箭自然是不弱的,这样一来,雍军渐渐势弱,正当孙定催动军士,准备消灭这支雍军的时候。被围的雍军突然高声欢呼,那呼声越来越高,仿佛从远处传来,孙定一惊,抬头一看,天色已经发白,下意识地向泗水一看,只见旌旗招展,舟船蔽江,那船头锦旗招展,上面正是一个大大的“裴”字。

\chapter{第十一章 烽火扬州路}

淮东将军骆某遣校尉孙定率军五千援泗口,孙定军至,泗口已陷,时雍军主力未至,定起兵攻之,未果,雍军已至,泗口遂为雍军所夺,孙定困重围,士卒皆乞降,孙定不能阻,雍军俘之。

——《资治通鉴·雍纪三》

楚州大营,骆娄真坐在大帐里面愁眉不展,花了一上午的时间,才勉强完成楚州大营的备战,这让他更加忧虑,这样的情况如何迎敌呢,若是有三五日的时间,自己便可做好准备,只是不知道雍军什么时候到来,最好尚相和陆大将军都是杞人忧天。不过仔细想想,淮东本就是重地,雍军攻淮南,不是取淮西寿春,就是攻取淮东扬州,而想要取扬州,楚州、泗州、广陵就是雍军必夺之地,若是雍军有意取淮东,自己定是首当其冲。

骆娄真看了黄参军一眼,不耐烦地问道:“怎么样,派去泗州和广陵的信使可回来了么?”

黄参军神色不安地道:“尚未回来,不过两地距离也颇远,一来一回,总得要到晚上才能回来。”

骆娄真怒道:“都是废物,陆大将军的信使可以数日之内从江夏赶到楚州,咫尺之地的泗州、广陵,也要花那么多时间,还有孙定这厮,我让他到泗口接管防务,怎么这么长时间也没有派个信使回来禀明情况。”

黄参军见他怒气勃发,紧张地道:“或许是军务繁忙,想必下午就会有消息的。”

骆娄真心中稍安,道:“传令下去,让周副将等诸将不可懈怠,若是楚州有失,我的性命保不住,你们也别想好过。”

黄参军打了一个冷颤,道:“将军,是否通知楚州郡守一声,那里还有五千守军,虽然战力不强,可是有所防备也是好的。”

骆娄真一皱眉,他和楚州郡守不合,只是碍着那郡守是南楚世家子弟,自己根基尚浅,所以不愿得罪,但是此刻他也知道唇亡齿寒的道理,自己在楚州西南立营,若是雍军来攻,自己守不住大营,就只能退入楚州守城,若是不趁现在打好招呼,恐怕连个后路都没有,思忖片刻,他冷冷一笑,道:“派人去通知顾元雍一声,就说让他即日闭城,以待敌军。”黄参军连忙答应,骆娄真和顾元雍不和,主要是因为楚州大营的军士在楚州胡作非为,骆娄真又不甚约束的缘故,但是因为骆娄真的后台太大,顾元雍无奈之下也只得想法子讨好于他,虽然骆娄真不甚领情,可是黄参军等人也是沾光不少,对顾元雍自然有些好感,所以黄参军才会想办法及时通知楚州军情。

黄参军刚刚离去,亲卫进来禀报道:“启禀将军,孙校尉的亲卫回来了。”

骆娄真大喜道:“快让他进来。”

不多时,走进两个军士来,前面那人骆娄真认得,乃是孙定的族人孙方,现在是孙定的亲兵头目,后面那人却是有些战战兢兢的,进帐之后始终不敢抬头,显然是心中畏惧,骆娄真只道那人也是孙定亲兵,便没有理会,问那孙方道:“孙校尉已经到了泗口吧,情形怎么样,雍军可有动静。”

孙方神色有些紧张,道:“启禀将军,校尉大人令我回报,雍军暂无动静,不过校尉大人已经派出斥候沿河而上,探听军情,若是有消息,必定飞报大营。”

骆娄真心中一宽,望望孙方身边那人道:“此人是谁,怎么也带他进帐了?”

孙方有些惊惶地道:“他是我们营中数一数二的高手,校尉大人担心雍军斥候已经潜入淮泗,所以令他和属下一起前来。”

骆娄真笑道:“理应如此,孙校尉果然细心,你叫什么名字,既然孙定都说你的功夫不错,想必定然是千里挑一的勇士,怎么胆怯得像个娘们,来人,赏他一樽酒,不要这么紧张,本将军又不是杀人魔王。”

那军士闻言似乎心中一宽,身躯放松了许多,抬起头来,双手接过酒盏,上前一步道:“多谢将军赐酒。”说罢一饮而尽。

骆娄真仔细看去,只见这军士看上去二十八、九岁年纪,面庞棱角分明,俊朗英武,神情沉静淡漠,笔挺的身姿宛如白杨一般峻挺,双目开阖间寒光电闪。骆娄真心中一震,这样的气度,就是大将军陆灿也不过如此,若是他曾经见过此人,怎会没有一点印象,他站起身来,高声道:“你绝不是楚州大营的士卒,你是何人?”随着他的喊声,帐外他的亲兵蜂拥而入,将骆娄真护在其中。

骆娄真正欲令人将孙方和那军士拿下,就在这时,帐外一阵喧哗,声音越来越大,一个浑身浴血的斥候跌跌撞撞冲了进来,扑倒在地,声嘶力竭地道:“将军,不好了,雍军夺下了泗口,前锋已经向大营而来。”骆娄真抬起头,面上神色满是绝望,恶狠狠地望着孙方和那个军士,怒声道:“你二人定是雍军奸细,来人,给我将他们斩了。”

孙方已经是吓得魂不附体,那军士却是神色不变,淡淡一笑,道:“骆将军,在下淮南节度使裴云,特来向将军致意。”帐内众人都觉得耳中轰然,这怎么可能,雍军大将,掌握徐州大营十五万大军的裴云怎会出现在此地。就是带着裴云一起进来的孙方也是上下牙直打架,他被俘投降后奉命带这人混入楚州大营,他一直以为这人乃是白衣营高手,怎知竟是裴云本人,毕竟裴云已经是三十五六岁年纪,怎料他看起来如此年轻,也难怪无人能够想到这军士的身份。此刻众人脑海中都浮现出裴云的身世,少林高徒,武艺精深,曾闻佛门心法有修身养性的好处,如今看来果不其然。

就在众人心旌动摇的时候,裴云身形一晃,已经向骆娄真扑去,骆娄真心中也生出凶念,若是生擒此人,那么雍军说不定会大乱,到时候淮东安保,自己的功劳可是非小。他厉声道:“不许放箭,给我擒下此人。”他不许属下放箭,是担心若是杀了裴云,激怒雍军,在淮东各地肆虐一番,自己的罪责还是不小,甚至可能会葬送自己的嫡系军队。

就在他的话音未落之际,帐内惨喝声起,十数名扑上去的亲卫滚成一地,裴云的双手金光隐隐,他的无敌金刚力已经是炉火纯青,一掌下去便是有死无生,转眼间他已经突破亲卫拦阻到了骆娄真身前。骆娄真拔剑刺去,这一剑风雷之声大作,若是常人必定先要躲避,裴云却是挥掌相迎,剑掌相交,却发出金石之声,骆娄真被他的掌力震得后退一步,这时,裴云又是一掌击来,这一掌势如泰山压顶,骆娄真又是被迫后退一步。掌风激荡,大帐之内劲风狂啸,裴云只是缓慢从容地向骆娄真一步步逼去,一套平凡的少林拳在他手中使出却是威风八面。那些亲卫就连插手也插不上,更别提围攻裴云了。骆娄真的剑术本来是颇为出众的,可是他沉溺酒色,内力受了很大的影响,眼看着裴云步步逼近,他却连一剑也不能反击,本想高声呼唤亲卫放箭,却担心牵连自己,更是没有出声的力量,这一刻,楚州大营虽有三万大军,骆娄真却觉得自己只是孤单一人。

“砰”,骆娄真的后背撞上了营帐的后壁,这时候,一营军士已经冲到大帐帐门处,黄参军厉声道:“射死他,不要伤了将军。”

骆娄真大喜,脸上露出了狰狞的笑容,只要自己再抵挡几招,就可以反败为胜,他可不相信血肉之躯可以抵御弓箭的攒射,自己只需趁着裴云当箭之时,划破营帐逃出即可,后面黄参军必定已经安排了接应。就在这时,骆娄真看到裴云淡漠的面上露出一丝嘲讽,心中电转,骆娄真猛然挥剑向裴云斩去,这一剑他用尽了所有力量,如同匹练一般的剑气摧枯拉朽,裴云眼中闪过一丝赞赏,拔刀出鞘迎击而上,刀剑相击,剑吟刀鸣,骆娄真的身躯不可避免地再次撞在了营帐壁上。就在这时,一柄长刀破壁而入,正好将骆娄真的身躯穿透,鲜血飞溅,骆娄真发出一声惊天动地的惨喝,裴云已经一刀斩落,骆娄真的人头飞起。

黄参军的声音带着哭腔,他高呼道:“立刻放箭。”

那些军士见到主将惨死,早已经是心惊胆战,一听到黄参军的命令,都是下意识地引弓放箭,只是心志混乱,这第一轮箭毫无威力,不过早已缩到大帐一角的孙方仍然遭到池鱼之秧,身中数箭而死。裴云则一脚踢开骆娄真尸身,拎着他的人头,一刀挥去,营帐中分,缺口处露出一个手执长刀的南楚军士,地上满是尸首,正是黄参军安排的接应军士。裴云破帐而出,第二轮箭矢才追袭而至。可是那个军士刀化长虹,将所有箭矢都统统挡住,等到第三轮箭矢射出的时候,裴云和那个军士已经冲出了十余丈,没入了南楚军营之中。营中传来两人的大喝声道:“骆娄真已死,骆娄真已死。”

营中一团混乱,不知多少人慌乱地奔跑,惊叫,也有将领们极力约束部下的喝骂声,斥责声。就在这时,四野号角声起,鼓声阵阵,有南楚军高声喊道:“不好了,雍军来了。”身旁传来千军万马的奔驰声,地面的震动说明了来的是一支骑兵。黄参军回头望去,只见辕门处,身穿青黑色衣甲的雍军铁骑如同潮水一般涌入楚州大营,混乱的南楚将士在雍军铁蹄践踏下骨肉化泥,那些雍军手中都是长达三尺二寸,需要双手握持的绣春刀,一刀斩下,就可将人砍成两段,他们在营中左冲右突,所向披靡。

如何可以抵抗这样的军队,几乎所有南楚将士的心中都涌上这样的念头,有人开始舍命从别的营门逃走,有人茫然无措地躲在营帐中等待末日的来临,当然也有人声嘶力竭地组织着反攻,黄参军就是其中之一,他已经发觉了来的这支雍军其实人数并不多,大概只有数千人,所以他开始下令指挥军士反击,本来应该担负起这个职责的李副将早在看到雍军入营的一刻,就已经带着百余亲卫从后面逃走了。

南楚军的反抗开始有了效果,三万大军毕竟不是这么容易就崩溃的,无论如何,淮东军原本也是精锐之师,这些年来虽然被骆娄真害得锐气全无,但是到了生死关头,还是可以一战的。雍军的攻势开始受到遏制,已经不能自如地攻击了。

正在这时,那原本在乱军中失去踪影的裴云出现了,这些许时候,他已经换了衣衫,身穿黑衣黑甲,身后的黑色大氅在秋风中猎猎作响,在他身后跟着十余亲卫,这些人都是寻常的青黑色衣甲,不过他们身上却都披着白色大氅,衣甲上面也没有表明身份的标志,这正是裴云麾下白衣营的标志,其中一人正是暗中夺取泗口的杜凌峰。这些人就在乱军之中安步当车,向中军大帐走来。

黄参军正在营前指挥楚军反击,他虽然是文官出身,平素又是怯懦非常,可是毕竟有些军事才能,群龙无首的楚军只需有了首领,就可以勉强对抗数量远远不如他们的雍军。他看到裴云带着亲卫在乱军中缓缓走来,心中大惊,若是让此人杀到这里,只怕再没有机会守住大营了,他连连下令阻截裴云这些人。可是裴云身边这些人的武力强悍非常,不需裴云动手,他们刀斩枪挑,已经开出了一条血路,在他们前面,南楚军开始崩溃,开始逃窜,黄参军也不能让他们继续听命。

裴云就这样走到中军帐前,他不去看面色苍白,被军士护在其中的黄参军,抬起头看向中军帐前飞扬的大纛,神色异常淡漠,抬步向大纛走去。负责守护帅旗的军士们舍命抵抗,但是在裴云身边的白衣营的刀剑下,他们的抵抗成了微不足道的挣扎。走到大纛之下,裴云一声厉喝,挥刀斩去,一道绚烂的光芒闪过,大纛的旗杆从中而断,营中的南楚军看到帅旗倒地,仅存的斗志终于完全崩溃了。有些胆量大的脱营而走,有些干脆丢了刀枪,跪伏在地,完全放弃了抵抗。楚州大营旌旗倒伏,残破狼藉,三万军士除了逃走和战死的之外,尚有一半束手就擒。望着全线崩溃的大营,黄参军呆若木鸡,良久他拔出佩剑,欲向颈上抹去,但是手足颤抖,竟是不敢下手。还没有等他鼓起勇气,裴云身边的一个亲卫已经策马过来,一刀背打在他背上,将他劈晕在地。至此,楚州大营大局已定。

看着在雍军威逼下弃械投降的南楚军士,杜凌峰高声笑道:“师叔,怎么南楚军这么稀松,若是他们的战力都是如此,恐怕用不了半年,我们就可以灭掉南楚了。”

裴云淡淡地看了他一眼,道:“骆娄真昏庸无能,只知道用金钱美色笼络部将,不知道整军经武,南楚淮东军战力不强,你若看到陆灿麾下的军队,就知道南楚也有英雄好汉了,若是你这般轻敌,我可不敢再让你做先锋。”

杜凌峰一伸舌头,道:“是,属下知错,绝不敢轻敌,将军可不要把我留在后面。”

裴云淡淡一笑,也不理会他,对着另一个白衣营勇士,一个相貌清峻的中年人道:“卫平,你带五百人留在这里看守俘虏,我要立刻突袭楚州。”

卫平忧心忡忡地道:“将军,你是一军主将,不应身先士卒,独自入营斩杀骆娄真可以说是因为将军武功胜过我们这些人,可是突袭楚州,事关重大,请将军三思,若是将军有什么不妥,我们如何向三军将士交待。”

裴云笑道:“你放心,取了楚州之后,我想以身犯险都没有机会了,张文秀领军攻泗州,旦夕可下,然后合击广陵,等攻下扬州,我们便要和陆灿交战,到时候我哪里还有出手的机会。”

杜凌峰闻言问道:“师叔,陆灿一定会来救援淮东么?”

裴云点头道:“若是扬州落入我手,我军就可以陈兵瓜州渡,威胁对岸的京口,若是我们不取京口,沿江而上至燕子矶,就可威胁建业,所以陆灿是绝对不能容许我们在淮东耀武扬威的,尚维钧虽然擅权,可是关键时候也会放手,虽然会拖延一些时间,可是我们要先清除南楚军的残余,就是速度再快,想要攻到扬州,也得一月时间,到时候陆灿必定已经在长江严阵以待。”

杜凌峰道:“既然如此,不若我们奔袭扬州,一路马不停蹄,让陆灿没有时间赶过来如何?”

裴云淡淡一笑,道:“这一战势在必行,没有躲避的可能,你不要多问了。”杜凌峰神色茫然,却也不敢再问。

这时卫平道:“将军,此地还有万余俘虏,我军哪里有余力看管他们,请将军示下如何处置?”

裴云道:“杀俘不祥,何况这些南楚军心志已丧,不足为害,你将他们禁于营中即可,若是有变你们脱身就是,再过一个时辰,何郢就会到了,将楚州大营交给他即可,你分兵两万去楚州接应我。”说罢,裴云便向外面走去,此刻随他前来袭取楚州大营的先锋营已在列阵,等候他的到来。

到了未时,雍军主力的步兵在一个中年将领的带领下终于赶到了楚州大营,看到的场景却让他瞠目结舌,万余南楚军都在营帐中静坐,只有五百雍军来回巡视镇压。见到卫平之后,那中年将领何郢立刻令三万大军接管楚州大营,卫平则带着两万步骑向楚州而去。

在楚州大营陷落两个时辰之后,泗州大营被五万雍军猛攻,由于骆娄真信使途中被白衣营截杀,泗州大营毫无准备,总算这里的守将平日尚且留心军务,直守到第二日清晨,泗州大营才陷落。之后张文秀领军攻泗州,泗州郡守怯懦不敢迎战,开城投降,而楚州昨夜已经易主,至此南楚淮东守军只剩下广陵大营一部,大雍破楚之战的序幕终于正式揭开了,淮左名都,竹西佳处,风月无边的扬州路,已经俱是战云烽火,铁骑踏碎了南楚的苟安美梦。

\chapter{第十二章 孤城血未干}

淮南节度使裴云,轻取楚州、泗州,亲斩南楚淮东主将骆娄真,淮东各镇,皆闻风而降,唯淮东军副将蔡临,收溃兵,守广陵,雍军攻而不下,裴云令何郢部绕道袭取高邮,渡水侧击之,广陵败绩,援军久不至,蔡临知势不可绾,时,裴云以箭书招之降,蔡临遂引军出城,自绝阵前,广陵众将乃降。十月二十九日,雍军至扬州,扬州守军不战而溃。

——《资治通鉴·雍纪三》

楚州名胜,以城中的镇淮楼、韩侯祠和城郊的漂母祠、韩侯钓鱼台最为出名,楚州郡守顾元雍本来最是喜爱镇淮楼,不仅常常在此处召宴城中名士,昨夜更是在此指挥楚州守军抵抗雍军的进攻,可是一夜之内,再次来到镇淮楼,他却已经是阶下之囚,虽然身边监管的雍军军士没有丝毫失礼,可是他心中的苦涩和恐惧却是怎么也摆脱不掉。

昨天黄昏时分,城外来了丢盔卸甲的楚州大营溃军,自己方得知原来雍军已经攻陷楚州大营,骆娄真已经战死,他连忙打开城门让这些败军进城,为首的那人正是黄参军,此人经常帮自己在骆娄真面前缓颊,所以他并没有生出疑心。不料进城的却是煞星,黄参军竟然是被雍军逼着来赚城的,原本尚可勉强一战的楚州就这样莫名其妙地陷落了。总算顾元雍尚存了一分戒心,虽然被雍军进了城,可是他在亲兵的保护下退守镇淮楼,和雍军开始了巷战,雍军战力强横,但是楚州守军毕竟是熟悉地理,两军缠战许久,胜负未分。但是当日夜里,雍军的援军两万人涌入楚州城,顾元雍最后的一点希望也破灭了,眼看着楚州城内满是雍军的旌旗,剩下的千余守军被围在镇淮楼下,无奈之下他只能举城请降。之后他就被迫领着雍军四城安民,到了天明时分,楚州城就已经切切实实被大雍据有了。

一夜未睡的顾元雍又被雍军主将裴云召来镇淮楼,走上原本自己最熟悉的顶楼,他便看到裴云站在窗前,负手而立,俯瞰楼下的景致,在他身后两侧,左右各站着两人,都是青黑色衣甲白色大氅的白衣营高手。顾元雍虽然不知道这些亲卫身份的特殊性,也能够看得出个个气度凌厉,不似寻常军士。他神色苦涩地上前一揖到地道:“南楚降臣顾元雍拜见节度使大人。”

裴云转过身来,伸手相搀,待他起身之后,裴云微微一笑,道:“裴某奉我大雍皇帝陛下之命攻略淮东,于楚州百姓多有冒犯,昨夜血战,难免伤及许多无辜,大人既然已经弃暗投明,还请大人多多安抚才是。”

顾元雍诺诺答应,心中却是生出期望之心,莫非雍军并不准备将自己处死么,自己抵抗了雍军将近大半夜,黑夜之中,攻城的雍军损伤也是不小,总有千人左右,他原本以为只要等到楚州平定,自己就会被秋后算帐呢,若非是担忧楚州城被屠城血洗报复,他也不会投降,不料这位淮南节度使,雍军主将似乎没有怪罪自己的意思。

顾元雍从前没有和雍军作战的经验,自然不知道在雍军眼中,敌军若是抵抗才是正常的,若是不抵抗就请降,倒会让他们觉得奇怪呢?

裴云对顾元雍抚慰了几句,言辞温和,让顾元雍渐渐安下心来,这时候,杜凌峰怒气冲冲地走上楼来,对这裴云施了一礼,道:“将军,那楚州长史太无礼了,属下奉命去收缴文书图章,他竟然不肯交出,还将您大骂了一通,说您使用诈术赚城,是阴险小人。”

顾元雍心中咯噔一下,那楚州长史荆长卿是同泰二年秋闱二甲九名的进士,四年前到楚州上任。此人是嘉兴世家子弟,本来按照他的背景才华,应该有更高的官位,至少也可以进翰林院的,可是他却仕途坎坷,多年来在各地任职参军、司马之类的职务,始终不得晋升,与他同科之人都已经金堂玉马,唯有他年届不惑才被任命为楚州长史。他到任之后,顾元雍仔细留心,此人行事有理有节,进退得宜,克尽职守,清正廉洁,的确是良才,他曾问及其仕途坎坷的缘故,这人只是叹息不语,这其中自然有隐情,可是顾元雍生平不喜欢探查别人的隐私,所以也就只是放在心里罢了。不料今日此人竟然如此执拗,若是触犯雍军,岂不是没了性命,他妻妾子女都在楚州城内,弄个不好,全家灭门也是可能的,想及此处,他不由心中暗暗焦急。

裴云神色不动,淡淡道:“凌峰,你如何处置了?”

杜凌峰道:“我一气之下,已经让人将他绑到了楼下,请将军允许属下将此人斩首示众,以为敢和我大雍为敌者戒。”

想及荆长卿平日的好处,顾元雍连忙上前作揖道:“将军恕罪,将军恕罪,荆长史生性刚正,或者有所冒犯,将军宽容大量,还请饶恕他的性命。”

裴云笑道:“将他带来,我要见见这个强项长史。”

杜凌峰大喜,传令下去,不多时亲卫押着一个人上来了,这人四十岁左右年纪,相貌斯文,气度平和,只是此刻他浑身是土,官帽已经不知掉到哪里去了,额头上还有血迹,可见一路上吃了不少的苦头。

上得楼来,那人立而不跪,只是怒目而视,杜凌峰见他如此,怒道:“见到我家将军还不跪下请罪。”

那人冷冷道:“荆某是南楚臣子,为何要拜大雍的将军?”

裴云闻言笑道:“顾郡守已经率楚州官员投降我大雍,你如今是降臣,为何不跪?”

那人怒道:“郡守请降,我长史没有请降,尔等侵我国土,伤我黎庶,南楚百姓无不恨之入骨,如今虽然迫于局势暂时屈服,待王军北上,犁庭扫穴,绝不令尔等逃出淮东。”

杜凌峰大怒,上前就是一记耳光,将那人打翻在地,指着那人骂道:“南楚百姓恨之入骨的不知道是谁呢?谁不知道骆娄真在淮东肆虐,抢掠民女,强征粮饷,今日我军贴出告示,提及骆娄真伏法之事,楚州百姓无不欢欣鼓舞,你既然这样硬气,怎么没有胆子和骆娄真相抗,我平生最讨厌你这等腐儒,既然你不肯归降,那你就是我军的囚犯,我也不杀你,将你在郡守府前枷号三日,看你还有没有力气大骂。”他这一拳极重,打得那人半边脸都肿了起来,口角溢血,那人似乎也豁出去了,痛骂不已,虽然口齿不清,但是杜凌峰却听得怒火更盛,他拔出佩刀,指着那人道:“好,你既然自己寻死,我就成全你。”

裴云原本只是淡淡瞧着杜凌峰行事,见他真的要挥刀杀人,才阻止道:“算了,他也是个忠义之辈,杀之不祥,将他关入大牢算了,不要过分难为他的家人。”

杜凌峰喜道:“属下遵命。”说罢拖了那人向楼下走去。

顾元雍吓得冷汗直流,杜凌峰虽然是在殴打责骂那个不恭的长史,可是其余几人的眼光明明在自己身上打转,分明是杀鸡儆猴的意思。眼看着得力的下属官员被那个嚣张跋扈的雍军军士凌辱,顾元雍心中生出屈辱之感,恨不得也将这些人大骂一通,然后让裴云下令将自己拖出去斩首,这也算是为国尽忠了。他面上神色一阵青,一阵红,自然被裴云看在眼里,但是如今最重要的是威慑楚州官员,让他们不敢反抗才是,所以他装作没有看见顾元雍的面色,南楚在淮东的高级官员都是南楚世家子弟,就是请降,也是绝对靠不住的,裴云只等攻下广陵之后,就要清洗淮东,将之作为大雍进攻南楚的前线,现在不过是暂时隐忍罢了。

过了一日,裴云留下卫平带着五千人镇守楚州,自己率着大军会合何郢部向广陵而去,与此同时,成功夺取泗州的张文秀部,也向广陵会合。

广陵是扬州的最后一道门户,此地本来属于扬州管辖,而扬州古称广陵,东晋末年,此地设县天长,后改广陵为扬州,改天长为广陵,到如今已经有数十年,人们早已习惯了这种叫法。将广陵当作扬州北面的屏障,夺取广陵,扬州就可一举而下,所以南楚在此地设立了广陵大营。

广陵大营的副将蔡临虽然也是尚维钧一系,可是此人倒是生性正直,他是尚维钧的外甥,若非是和尚维钧不合,只怕这淮东主将的位子也不会落到骆娄真身上,所以骆娄真对其敬而远之,将广陵大营交到他手上便不闻不问,蔡临练兵颇有独到之处,约束士卒,从不扰民,还多有扶危济困之举,所以在广陵一带声名极好,楚州大营和泗州大营溃败之后,都有不少残军逃到广陵,被他收入营中,整顿之后,倒也有三万多人。他将军情上报建业之后,便领军进驻广陵城,他心里有数,若想正面对抗雍军,必然是惨败之局,所以准备依靠广陵城抵挡雍军的攻势。他有自知之明,知道自己不可能战胜裴云,只盼着能够守到南楚援军到来。

十月九日,裴云大军到达广陵,十万雍军陈兵广陵城下,一眼望去密密麻麻,雍军的大营犄角相连,气度森严,只是望去就已令人生出不能取胜之感。蔡临指着雍军大营道:“若是广陵失守,雍军便可以长驱直入扬州,威胁京口、建业,尔等若不戮力苦战,淮东军威名尽丧,本将军已经呈书建业,向尚相和陆大将军求援,我们只需守个十天半月,就可等到援军,诸君可肯效死。”广陵大营将士都是深受蔡临恩泽,闻言都是高声道:“愿为将军效死。”

啸声远扬,城下雍军听得清清楚楚,裴云一皱眉,对身后的何郢、张文秀道:“看来广陵城不好攻取啊!”何文秀是一个相貌俊朗的青年将领,他朗声笑道:“将军何必挂虑,广陵纵然难攻,还能挡住我大雍铁骑么?”众将士也都高声道:“请将军下令攻城,不克广陵,誓不为人。”

裴云闻言挥鞭指着广陵城道:“既然如此,何郢,你这次尚未立下战功,就让你先上如何?”

何郢大喜,一路上裴云抢着做了先锋,反而是他只能带着大路人马跟在后面,早已求战心切,闻言他凛然尊令,策马向军前走去,不多时,号角声鸣,雍军的第一波攻城开始了。

谁也没有想到,这一攻,就是整整半个月。

蔡临在广陵可谓甚得民心,他又不似骆娄真那般无能懈怠,这些年来备战充分,广陵城内的粮草辎重十分充足,在他的率领下,广陵城毫不动摇地撑了半个月,城上城下,皆是一片狼藉,雍军的投石车、箭楼不知道损坏了多少,南楚军不知道射出了多少箭矢,泼下了多少沸油金水,滚石檑木更是数不胜数,到了后来,靠近城墙的房屋皆被拆毁,石头木料都用来守城了。雍军几次派出敢死队攻上城去,都没有成功。最接近成功的一次,是十月十九日,裴云派出了所有的白衣营侍卫,整整十六人带着三百敢死勇士登城,蔡临带着亲卫亲自迎敌,苦战半日,若非是从广陵城东的高邮湖上突然来了援军,只怕广陵城已经失守,这场恶战,白衣营死了两人,三百勇士无一生还,蔡临身边的亲卫也死伤殆尽。可是落日余晖下,浴血的广陵城仍然屹立不倒。

裴云的神情有些冰寒,虽然并没有准备几日就攻下广陵,可是现在的情形却是太不利了,必须要随时都可以结束此战才行。杜凌峰神色疲惫地走了过来,他虽然年轻,但是武功在白衣营中也是数一数二的,两人又是师叔师侄的关系,所以裴云对他十分关切,见他浑身是血,裴云皱眉问道:“怎么样,伤重不重?”

杜凌峰道:“我只是挨了两刀,没有伤到筋骨,可惜了这些兄弟,蔡临身边的亲卫武功高明得很,当初骆娄真身边的亲卫要是这样高明,只怕师叔和我都会葬送在楚州大营。”

裴云叹息道:“建业蔡氏在南楚是有名的世家,自然是有些高手护卫的,蔡临又是蔡氏嫡子,也难怪如此。”

杜凌峰道:“将军,高邮守军居然有胆量前来救援广陵,是不是南楚的援军已经准备过江了。”

裴云摇头道:“司闻曹传来的消息,现在陆灿正在建业要求接管淮东军权,尚维钧仍然推辞不肯。”

杜凌峰愕然道:“尚维钧难道不知道现在淮东已经是岌岌可危了么?”

裴云笑道:“这件事情倒是有些蹊跷,似乎有人截断了淮东和建业的消息往来,广陵的求援书根本就没有到达建业。”

杜凌峰茫然,但是他很快就将此事置之脑后,道:“师叔,那么现在怎么办,高邮守军竟然敢出城作战?”

裴云正欲答他,一个斥候过来禀报道:“将军,已经探查清楚那些人不是高邮守军,而是高邮湖水匪,首领名叫官枫,此人水性过人,在高邮首屈一指,因为抗拒骆娄真强征粮饷才被迫入湖为匪,平素劫富济贫,深得高邮民心,不过他和广陵大营蔡临是生死之交,若非是蔡临缓颊,只怕骆娄真早就调动水军来清剿高邮湖了,今日正是他率了部众救援广陵。”

裴云笑道:“此人倒也讲义气,只可惜不过是螳臂当车罢了,何郢,你明日去取高邮,凌峰,去楚州传我军令,调一营水军到高邮待命,到时在水军护翼下,何郢渡水袭取广陵东侧,促不及防之下,广陵旦夕可破。”

众将轰然领命,十月二十日,何郢袭取高邮,十月二十一日,一营水军到了高邮湖,原本在攻取扬州之前是不准备使用水军的,所以水军是在楚州洪泽湖待命的,如今情形有变,只好调一营水军到高邮湖对付水匪。

十月二十二日,广陵的决战开始了,这一次雍军有备而来,在官枫出城攻击岸上的雍军的时候,大雍水军突然出现,大雍在江淮和南楚对峙多年,水军精锐不比南楚差多少,和这些乌合之众的水匪比较当真是天壤之别,一番苦战之后,水匪全军覆灭,除了官枫侥幸逃回广陵之外,无一生还。雍军本已切断了广陵和扬州之间的通道,如今东面的高邮湖也落入雍军掌握,何郢借助水军屡次攻击东城,这一次,广陵真的成了孤城。

十月二十三日,在雍军连续的猛攻下,广陵城终于失去了抵抗的能力,虽然雍军将士都强烈要求裴云一举攻下广陵,最好是屠城泄愤,但是却被裴云阻止,令人向城中射去箭书招降。

旬月之间已经是老了十余岁的蔡临望着手上的箭书,他的神情是异样的淡漠,看看身前众将,都已经是疲惫不堪,更是几乎人人带伤,如今广陵城内只有万余残军,整整两万军士死在城头之上,广陵军民死伤叠籍,真是再也打不下去了。反而是城外的雍军,靠着充足的攻城器械和强悍的战力,虽然是攻城一方,却只是损失了一万五千多人,主力依然无损。为什么援军还没有来?蔡临可以从麾下将士的眼中看到这样的疑问,城防残破,外无援军,士卒疲敝,就是名将之姿也难以继续守城,更何况蔡临自认只是平庸之才,微微苦笑,他黯然道:“明日出城请降。”

看到众将如释重负的神情,蔡临知道他们并非是因为可以保住性命而欢喜,谁也不知道雍军会否因为损失惨重而报复,可是只要能够从无休无止的攻城战中解脱出来,这已经足够了。无必救之兵者,则无必守之城,广陵军民心志已经崩溃,当真是没有守住的可能了。

众将离去之后,站在屋角的一个古铜色肤色的青年走过来道:“蔡大哥,你当真要投降么?”

蔡临看了他一眼,道:“官贤弟,你对蔡某已经是仁至义尽,趁着今夜,你从高邮湖逃走吧。”

那青年愤然道:“蔡大哥,昔日若不是你援手,我爹娘早就被官府所杀,二老临终之时尚命我舍命相报恩情,我岂能独自脱身,你若身死,我还有什么脸面去见爹娘之面。”

蔡临黯然道:“我当日不过是举手之劳,你何必放在心上,况且我是托你去求见陆大将军,请他早日在京口准备迎敌,我明日不过是请降,以裴云的声名为人,是不会为难我的,此事十分紧要,更胜我的性命,你拿着我的信物去吧。”

官枫犹豫再三,道:“既然蔡大哥如此说,我便去见陆大将军,大哥放心,等我见了陆大将军便回淮东,想法子救你出来。”

蔡临笑道:“好,我会等你来救我,你晚上就走吧,我很累了,准备好好休息一下,这些日子难得有一天晚上不用担心雍军袭城,我也该好好休息一夜了。”

官枫见他神色憔悴,便告辞道:“大哥珍重,那么晚上我就不来辞行了,你放心,最多五六日我就能回来,到时候一定会来寻你,在江淮,我一人可以来去自如,绝不会被雍军发现的。”

蔡临点点头,转身回内室去了。当夜官枫趁着夜色离开了广陵,大雍水军只有一营,防范得并不严密,所以官枫顺利地潜入高邮湖,游了一夜,登岸向南而去。

十月二十四日,蔡临酣睡了一晚之后,修面整饬之后,沐浴更衣,换上了一身青衣,他本是出身名门,也曾有过功名,虽然改了武职,却仍是不脱文人气度,穿上青衫,不似是浴血守城的武将,倒像是游学的文士一般,混不似前几日的狼狈模样,望望铜镜里面消瘦但是精神奕奕的容貌,他微微一笑。用过早饭,众将和广陵官员已经在外等候,他望了众人一眼,笑道:“诸位不必担忧,率众顽抗者,是蔡某一人,雍军若要问罪,自有蔡某当之。”众人都是面面相觑,见蔡临如此神情坦荡,众人也都放心许多。

巳时初,蔡临令人开了北城门,自己率众将和广陵官员步行至雍营请降,此时,裴云早已得到禀报,对于这个抵抗大军半月之久的南楚将军,他心中颇为敬佩,为了表示敬意,他也带了众将列阵出迎,双方相距二十丈才停下脚步。雍军众将望着蔡临,都是暗暗称奇,这人看上去倒像是一个白面书生,想不到竟然能够在雍军猛攻之下苦守孤城半月。

蔡临望望前面气度森严的雍军军阵,淡淡一笑,他本是世家子弟,书香门第,从来都是崇文轻武,只有他读书不成改学剑,更是违背父命进了军旅,只可惜自己才能平平,以至于兵败至此,还有何颜面请降苟活。他一举手,止住南楚众将步伐,独自上前,距离雍军军阵数丈,他方站住,望向雍军主将裴云,朗声道:“裴将军,蔡临痴心妄想,率众抵抗贵军,半月之间,血溅孤城,将军如有怪罪之意,蔡临一身担之,尚请宽宥广陵军民。”

裴云也高声道:“两国征战,理应如此,裴某不才,也不会因此事报复广陵军民。”

蔡临朗声一笑,拔剑出鞘,副将黄城只道他要献上剑印,表示投诚之意,便捧了将印过来,孰料蔡临引剑就颈道:“蔡某乃是南楚之臣,没有请降的道理,今日以死谢罪,身后之事,便由黄副将作主。”说罢,在裴云“不可!”声中引剑自绝。鲜血滴落,蔡临身躯仆倒于地。

南楚中人都是惊愕万分,黄副将更是大叫一声,扑到蔡临尸身前痛哭流涕。雍军众将纵然原本心存恨意,此刻也是怨尽恨消,望着蔡临尸首唏嘘不已。

良久,那副将泪尽而起,取了蔡临血剑和剑印上前拜倒道:“末将南楚淮东军广陵大营副将黄城,谨代广陵军民,向淮南节度使裴将军请降,唯请将军宽恕士卒百姓,末将等皆任凭将军处置。”

裴云下马上前,接过剑印道:“大雍淮南节度使、平威将军裴云,谨代吾皇接受广陵军民归降,将军且宽心,裴某不会妄杀广陵军民泄愤。”

那副将叩首道:“末将叩谢将军宽宥。”在他身后,广陵众将和官员都拜倒谢罪。至此,淮东之战最血腥的一幕终于过去。

裴云宽慰广陵投降众将官员之后,返回大营,正准备安排进军扬州,这时候有楚州信使送来卫平的书信,裴云打开一看,眉头紧皱,将信件交给众将传阅。

杜凌峰随侍在侧,也看了书信,他性子最急,惊叫道:“怎么可能,荆长卿明明已经下在大牢,尚有家眷牵累,居然一家人都消失无踪,这怎么可能呢?”

张文秀、何郢和其他将领也是面面相觑,裴云淡淡道:“一个荆长卿倒是不算什么,不过此事说明我军过于急促了,传我将令,何郢随我先取扬州,文秀负责将淮东各镇都清洗一遍,凡是和南楚关系紧密的人都要盘查清楚,不可再留下后患,不妨留下一些空隙,让那些忠心南楚的官员百姓南逃,这样淮东也清静些,皇上的意思,是要稳守淮泗口,即使不能顺利渡江,也不能再失去淮东。”众将轰然应诺。

雍军在广陵修整三日之后,裴云率军赴扬州,十月二十九日,雍军兵锋到了扬州郊外,扬州郡守弃城而逃,雍军兵不血刃攻取扬州,至此,淮东全境陷落。

\chapter{第十三章 冷月无声}

淮东消息断绝,南楚大将军陆灿自请主淮东,主政尚维钧不许,雍军据扬州,虎视京口,军报入建业,尚维钧惊恐莫名,乃许陆灿军权,陆灿督九江大营三万众,舟船两千五百艘,陈兵京口,对峙雍军。

——《资治通鉴·雍纪三》

十一月初二,雍都,长乐公主府邸,临波亭之内,进入十一月,长安的深夜已经是非常寒冷,更何况前几日还下了一场雪,可是江哲却偏要临湖赏月,怎不令小顺子头痛,一大早他便令人将临波亭里面的火龙烧得暖暖的,当江哲从寒园来到临波亭之时,亭内已经是温暖如春,不过看着江哲寂寥黯淡的神色,小顺子不由一阵苦恼。自从大雍南征开始之后,江哲便是隐居在府中,哪里也不去,除了在寒园读书,就是在临波亭发呆,这些日子,不仅婉拒了李贽的召见,就是李显、石彧等人也一概不见。小顺子自然明白江哲为何如此,大雍南征乃是迟早之事,只是众人都没预料到,一旦成真之后,江哲竟会如此消沉。

良久,江哲突然吟道:“久为劳生事,不学摄生道。年少已多病,此身岂堪老?”

小顺子听得心中一惊,诗词中涉及生老病死,往往易成诗谶,江哲早年殚精竭虑,以致华发早生,几乎吐血而死,可不是“久为劳生事”么,“不学摄生道”虽然略有偏差,这些年他也开始修练一些养生的功法,可是碍于天资,实在是进步不大,“年少已多病”自不待言,若是“此身岂堪老”这句再应验了,岂不是一语成谶,想到这里,小顺子只觉得出了一身冷汗,连忙上前道:“公子何出此言,若是公子觉得在雍都不能安居,不若我陪公子回东海去吧?”

我淡淡道:“这一次皇上攻略江南,并未和我商量进军之策,你可知这是为何?”

小顺子眼中闪过利芒,道:“莫非皇上对公子已经生出疑忌之心,所以才故意将公子排除在外,这次大军征南,理应设立平南行辕督管各军,若是如此,齐王殿下乃是众望所归的平南行辕元帅,可是皇上也没有下旨设立,莫非皇上对齐王殿下和公子的交情生出不满了么?”

我摇头道:“皇上是否疑忌齐王还未可知,但是就连齐王也没有提议筹建行辕。至于对我,皇上若是真的生出疑忌之心,是断然不会露出这样的形迹的,他只是担心我留恋故国,不愿难为我罢了。更何况平汉之后,皇上心中已经生出骄矜之心,他以为灭楚易如反掌,三路大军五十万人马齐头并进,江南不过二十万精兵可以和大雍一战,自然是一战成功,玉石俱焚。不仅是皇上,就是齐王殿下和诸位将军,也不免存了轻视江南之意。我之忧虑,俱在于此。”

小顺子拊掌道:“公子对江南之事了如指掌,莫非这一战大雍将会损兵折将么,既然如此,公子为什么不向皇上说明情况呢?”

我苦笑道:“有些时候,事情若不摆在眼前,是很难让人相信的,皇上和诸臣商议平楚之事时,即使以石彧的稳重,都说出‘南楚内有权臣擅权,且有凤仪余孽为患,将相不和甚矣,虽然有大将如陆灿者,也断无立功于外的道理,我军循序而进,纵然不能一战平楚,也可攻略淮南,占据蜀中,夺取襄阳,令南楚只能倚长江苟延残喘。’这样的话来,可见大雍上层已经失去了冷静。反而是南楚,虽然陆灿受制于权臣,却是上下同仇敌忾,戮力同心,这一战,大雍必然败于南楚。我已经上了密折给皇上,说及此战胜败尤在两可之间,劝其不要急于兴兵,可惜皇上将密折留中不问,显然是不同意我的意见,或者还会以为我是不忍见故国兵燹,所以危言耸听,其实大丈夫岂可瞻前顾后,我受大雍十余年恩养,又受皇上如此厚爱,又怎会蛇鼠两端,不知抉择。”

小顺子疑惑地道:“公子,且不说石相所说是否能够实现,但是南楚将相不和,又有凤仪门从中作梗,这的确是事实,陆将军虽然军略出众,可是尚不能掌控全部军权,难道这样也可战胜么,秦将军稳重老练,长孙将军深沉多智,裴将军勇毅果决,三人都是可以独当一面的将帅之才,陆将军一人如何可以取胜。”

我叹息道:“尚维钧的确是误国之人,可是南楚国主赵陇是他的外孙,他将南楚江山当成自家之物,所以一旦局势危急,他定是全力支持陆灿,至于战胜之后的倾轧排挤,那倒也不必细说,只不过那时对大雍来说已经太迟了。说到凤仪门,我颇有后悔之处,当初放纵凤仪门余孽,实在是因为她们成事不足,败事有余,可是我当真不该放过韦膺,只是碍于当时局势,不得不尔。此人虽然心狠手辣,被名利仇恨所羁绊,以至于家破人亡,流落南楚,可是此人毕竟是韦观之子,又受凤仪门主看重,当真是才华过人,目光如炬,他竟在痛定思痛之后选择了陆灿作为辅佐的对象。这些年来,若无他从中转圜,以陆灿的为人品性,早已和尚维钧两败俱伤。陆灿和我不同,我喜欢以权谋用人,凡是我的属下,就算是对我尊敬爱戴,也要将他生死完全掌控,一旦生出违逆之心,便可断然处置,陆灿以诚信用人,纵然是属下心中有自己的打算,只要无害忠义,他也就用之不疑,所以韦膺可以为他所用,有这样一个人替陆灿消灭政敌,排忧解难,陆灿才能在南楚屹立不倒。”

小顺子惊讶地问道:“韦膺此人,果然这般厉害么?”

我微微一叹,道:“此人厉害之处,还在你我想象之上,自从图谋东海不成之后,此人不知怎么和陆灿达成了某种默契,这些年来,尚维钧和凤仪门都对陆灿用过手段,俱是被他化解,兵部司闻曹多次使用计谋,想通过南楚内部的权势斗争陷害陆灿,也都被他消灭于无形,此事大雍上层尚不清楚是韦膺所为,是我遍阅司闻曹的文书和天机阁的密报,才从蛛丝马迹中发觉的。唉,陆灿能够任用韦膺,此诚为我所不及,韦膺能够不介意陆灿和我的关系,也是我预料不到的。”

小顺子想了片刻,道:“公子,昨日皇上令人送到寒园的军报,葭萌关和襄阳都已经开战,虽然尚无进展,可是这两地守军绝对无暇旁顾,淮东大局已定,而南楚朝廷才有应对,陆灿调动九江大营镇京口,不过一月之间,南楚已经失去淮东,这样的战局公子尚觉得不安么,若非南楚朝廷掣肘,陆灿怎会如今才领兵出战,如今淮东屏障已失,陆灿纵有回天之力,怕也是无可奈何。”

我移开望向冷月的目光,回过头道:“你可知道,这一次陆灿没有及时出兵淮东,并不在皇上意料之内,陆灿用军之时,往往会临阵决断,将在外君命有所不受,这也是他当初有胆量袭取葭萌关的缘故。你说他为什么会甘心在建业拖延时日,为什么裴云禀报说淮东和建业之间消息断绝?”

小顺子心中一惊,道:“公子曾说陆灿心性光明。”

我淡淡道:“为将者必要心狠手辣,陆灿对敌对友的确光明正大,可是他的手段也未必慈和多少,否则当年也不会安排截杀我的计划,更何况还有韦膺在他身边。”

小顺子思忖片刻,轻轻一叹。我继续道:“淮东地势险要,河流交错,最适合水陆作战,南楚水军熟知地理,擅于用舟师在江河中来去奔袭,若是陆灿和裴云在淮东交战,必然是胶结之势,战势也将对南楚有利。只是这样一来,南楚军想要放弃淮东也不是易事,兵戈相连,断不能轻易退却,若是如此,就合了我军之意。将陆灿牵绊在淮东,则淮西、九江、江夏无备。徐州大营水军步骑十五万,为何有三万军队不知去向,长孙冀二十万大军,难道都准备在襄阳滞留么?南楚其他的将领尚不能独当一面,葭萌关余缅不过是萧规曹随,襄阳容渊若是离开襄阳,也不过是离水之鱼,失群孤雁,南楚的弱点便是只有陆灿一人可以支撑大局,尚不如当初的北汉,龙庭飞殁后,还有嘉平公主、段将军可以继承他的遗志。所以裴将军在淮东稳步攻掠,就是为了诱使陆灿入淮东,只可惜裴云不能太过火,最后功败垂成,以致两军对峙于瓜州渡。南楚虽然失去了淮东,可是倚仗长江天险,陆灿可以游弋往来,灵活自如,这一点上,我军的意图已经遭遇到了挫折。可是淮东的一帆风顺,也不免让大雍上下对南楚戒心更弱,此消彼长,你可明白大雍目前的处境了!”

小顺子听得一身冷汗,可是他又反驳道:“虽然如此,陆灿一时在京口动弹不得,江夏大营不能轻动,其他诸军皆不能救援,公子之意,我军有意淮南,淮南守将石观虽然善战,也不能胜过大雍百战余生的勇士,凭着淮西弱旅,如何能够对抗大雍铁骑?”

我叹息道:“此事我一时也想不清楚,但是有些时候,人力可以胜天,我想十五之前,必有军报传到,到时便清楚陆灿如何应对了。我只希望这一战大雍不要损失过重才好。”

小顺子默然不语,良久才道:“公子还是不必忧心的好,裴将军、长孙将军都是能征善战之辈,必然不会惨败到不可收拾的地步。公子,陈稹昨日有消息至,您的表兄荆长卿在楚州被俘,吃了不少苦头,不过山子和渠黄已经利用天机阁在淮东的秘舵,将他们一家送回嘉兴了。”

我微微一笑道:“表兄生性固执,舅父有意迁居长安,只有他坚决不肯,恪守忠义之道,这次可是吃了苦头了,裴云想必不知道他和我的关系,否则怎也不会为难他?”

小顺子笑道:“公子和嘉兴荆家早已断绝往来,就是舜卿表少爷也早已被荆老爷赶出了家门,也难怪裴将军没有留心此事,不过这件事情恐怕明鉴司的人已经知道了,虽然陈稹安排的十分周密,就连荆氏也不知道他们的身份,可是我担心会被明鉴司发觉天机阁和公子的关系。”

我点头道:“这件事情不可不防,不过上次蜀中之事,夏侯沅峰受了不小的教训,因为葭萌关失守一事,许多大臣怪罪他办事不利,我们手中又有蜀王遗子,夏侯沅峰不敢过分得罪我的,再说南楚平后,天机阁也该销声匿迹了,这些年,绿耳的成就和海氏的利润已经足够支撑我们的生活,倒也不必过分担心天机阁的存亡了,让他们小心些,不要被陆灿和韦膺发觉破绽。平楚之战,我尚有用天机阁之处。”

小顺子低声应诺。

这时,远处传来踏碎积雪的声音,我眉头一皱,怎么这个时候会有人来临波亭打扰我,抬头望去,只见几盏宫灯掩映下,长乐公主只带着两个侍女和小六子向这边走来。心中涌起一阵暖意,十年夫妻,相敬如宾,这个女子仍然像当日我们携手离开长安之时那样深情不减。

为了观赏雪景,我特意不许人将临波亭周围的积雪扫去,石径上也是如此,见她在侍女扶持下踏着深雪跋涉而来,我忍不住上前相迎,一走出临波亭,寒风扑面而来,我不由打了一个冷颤,更是心中一痛,紧走几步握住长乐的素手,道:“这么晚了,你还出来做什么?”说着,连忙拉了她走入临波亭。

临波亭内,灯光如雪,我忍不住望向长乐恬静清丽的容颜,这么多年风风雨雨,即使是在回到长安之后,她也经常需要在宫廷之内周旋,应对各种明枪暗箭,为我争取一个安乐自在的空间,可是不论是时光如何流逝,她的风姿却是没有丝毫减损,虽然眉目之间已经留下了岁月的痕迹,可是却只能让她更加动人,犹如一眼沁人心脾的清泉,虽然沉默幽静,却是甜美怡人。握住她冰冷的双手,看向她被寒风吹红的玉颜,我一声轻叹,已经轻轻吻住她的樱唇。

长乐的娇躯轻轻挣动了一下,即使多年夫妻,她仍然不习惯在人前这样的亲昵,不过她也没有推开我,任凭我恣意爱怜。感觉到她的娇羞,我放弃了继续进攻的打算,笑道:“我没有事,你放心吧,不用为我担忧。”

长乐此刻的玉颜越发嫣红,迅速望了一眼在临波亭外眼观鼻,鼻观口,口观心的四人,温婉地道:“我知你定有打算,我也不想问你,只是雪夜寒冷,你也应当加件衣衫,小六子,拿过来吧。”

小六子抱着一个包裹走了进来,长乐公主抖开明黄的包袱皮,取出一件雪貂皮大氅道:“这是皇嫂今日令人送来的,是幽州今年的贡品,最是轻薄暖和,我不管你是赏雪还是赏月,总要加件衣裳才是。”

我任凭她替我系上大氅,然后再度握住她双手,满意的点点头,她的手已经恢复了暖意,伸手挽住她的纤腰,我笑道:“既然来了,就陪我一起吧,看看这波心冷月,天上寒星。”

长乐抬起头,不去看天上的星月,却是看向我,不语嫣然。我只觉得心中平和安乐,真希望时光永远停滞在这一刻才好。

这时候,小顺子等人都已经识趣地退得远远的,只留下我们夫妻二人月下絮语。挽着长乐,暂时抛却心中烦恼,专心致志地陪着她叙谈,心中一个念头涌起,又转瞬消逝,这样的月夜,长江之上,是否也有人在品味着无声的冷月呢?

千里之外,隔着浩荡江水,雍军的大营和南楚军的水营正在对峙,新月黯淡,明星一片,站在楼船之上,陆灿望着江心冷月,酹酒祝祷道:“唯愿苍天佑我,驱逐大雍强敌,护我社稷百姓。蔡将军英魂有灵,当谅我苦衷。”言罢,他看着手中蔡临的信物,不由唏嘘不已,日前,有人执蔡临信物前来求见,之后那人便要返回淮东去救蔡临,在自己坦言相告蔡临已经自尽殉国之后,那人当时便痛哭昏倒。想到自己舍弃淮东之举,纵然无人责备,也是于心难安。

他身后一人冷冷道:“大将军何必挂怀,是韦某先斩后奏,断绝淮东与建业的消息往来,若不如此,如何能够让尚相交出全部兵权,如今大将军已经掌控南楚全部军力,可以全力对抗雍军,牺牲淮东一地又算得了什么,更何况淮东军软弱不堪,又是尚相嫡系,他们损失重些对将军只有好处,不是么?”

陆灿苦笑道:“韦兄何出此言,此事我亦是同谋,虽然淮东消息断绝,可是我怎不知裴云之能,淮东诸将,无有可以对抗之人,只不过为了大局,我只能伪作不知,和尚相在建业纠缠不清,以致淮东沦陷,蔡将军从容就义,唉,这是我的罪过,韦兄不过是为了我军着想罢了。”

韦膺神色一动,却只是淡淡道:“韦某所为何尝是为了你,不过是想你打个大大的胜仗罢了,你可有把握?”

陆灿但笑不语,道:“淮西一个时辰前军报至此,南阳大营崔珏部已经向寿春进发,而徐州大营这次没有在淮东露面地董山已经到了钟离,长孙冀亲领南阳大营十四万大军围困襄阳,淮西只有石观将军三万人马,雍军之意了然,是要迫我首尾难顾,我已传令钟离,守住三日之后便可退到寿春,若是实在不能安然退去,总是请降也无妨碍,这样一来,就可以将雍军两部都吸引到寿春。”

韦膺皱眉道:“你当真以为寿春可以对抗雍军么,石观之才不过中上,雍军却是兵多将广。”

陆灿肃容道:“守城之要,关键在于军心民心,石将军定能稳守寿春无碍,更何况云儿是我长子,又是镇远公世子,有他在寿春,则军民心安,寿春断不会失守。”

韦膺道:“可是只是倚城固守,终究是难以持久,更何况江夏大营也是水军为主,虽有三千骑兵,也是杯水车薪,你总不会让水军去和大雍的铁骑交战吧,那岂不是舍本逐末,九江大营又在这里和雍军对峙,裴云只需牵制住我军,寿春迟早不保,难道你就不担心爱子的安危么?”

陆灿淡淡道:“身为陆氏之子,他当有舍身为国的打算,更何况此战我已经有所准备,这次雍军主要是针对淮西而来,淮东是陷阱,襄阳和葭萌关不过是可有可无的目标,只可惜,雍军既无人统率大局,又没有出动东海水军,此必是雍帝轻视我南楚将士之故,陆某当给雍军一次重击,令雍军铁骑再不敢窥伺淮南。”

韦膺闻言,默然不语,这一刻,他可以清晰地看到陆灿身上爆发的战意杀机,或许选择支持这个男子,当真是他一生中最正确的决定,既然如此,自己便要为他考虑周到,定不能让他受权臣奸佞所害。

想到此处,韦膺试探地问道:“扬州郡守胡成可是已经在大将军营中?”

陆灿眉梢一扬,道:“不错,此人弃城而逃,舍弃扬州千万军民,着实该杀,渡过江来,此人还妄想回建业去安享荣华富贵,却落入我的手中,我已经决定渡江作战之前,用他的人头祭旗。”

韦膺叹气道:“此人虽然无耻,可是他乃是尚相亲选的郡守,据说用了三十万金买这个郡守的官位,这次回到南楚,又遣家人贿赂尚相二十万金,尚相的文书明日就会到了,令你将他送回建业处置。”

陆灿眉宇间闪过怒色,道:“好一个贪官,怪不得他在扬州公然走私海盐,原来是想挽回损失,尚维钧当真是糊涂了,这么一个人居然去做扬州郡守,怪不得扬州不攻自破,既然明日文书才到。”他沉吟了片刻,朗声道:“来人。”一个亲卫从外面进来肃手听令。陆灿冷冷道:“你回大营,传我军令,立刻将胡成斩首示众。”那军士应诺去了。然后陆灿似笑非笑地望着韦膺道:“韦兄也是想为胡成求情?”

韦膺淡淡一笑道:“不过是想大将军早些动手,免得和尚相冲突罢了。”

陆灿一怔,摇头失笑,望望对面江岸上大营中的火光,道:“韦兄可敢和我去窥营么?”

韦膺笑道:“大将军召我上船,不就是为了去察看敌情么。”

陆灿微微一笑,令军士驾着楼船向对岸驶去。此刻满天繁星,江心月冷,天地间除了寒风呜咽,便只有楼船渡水的声音。

\chapter{第十四章 问是谁家子}

雍楚两军对峙于瓜州渡,皆按兵不动,三日,淮西告急,长孙冀麾下崔珏部攻寿春,徐州大营董山部攻钟离,钟离五日乃陷,郡守朱某,都尉陈某被俘不屈,皆殉死。两军合攻寿春,寿春乃淮南重镇,欲得淮南,必得寿春。时,陆灿长子云奉命助石观守寿春,云年十三,武勇过人,淮西军民闻云在,皆曰,大将军必不弃吾等,乃戮力死战,雍军寸步不能进。

——《资治通鉴·雍纪三》

钟离城终于拿下了,可是董山完全没有一丝欢喜,整整五天,仅有三千守军的钟离城让他饱尝了碰壁之苦,三万大军日夜攻城,明明显得那么软弱的钟离,却是始终不曾屈服,外城陷落了,退到内城,内城陷落了,便逐寸逐寸的巷战,这小小的钟离城,几乎吸干了雍军的鲜血。坐在钟离郡守府衙的大堂上,望着被士卒连推带搡押来的钟离郡守,董山深吸了一口气,道:“你抵抗大雍天军,罪在不赦,若肯归降,本将军便暂且饶你性命,若是不降,休怪我用你的人头祭奠我麾下将士的英灵。”

钟离郡守是一个三十多岁的中年人,他大笑道:“朱某乃国主头开恩科的探花,深受国恩,焉能屈膝降敌,要杀就杀,何必多言。”

董山大怒,道:“将他拉到门前处斩,成全他的忠义。”

那些军士推着那郡守去了,到了官衙门前,将那郡守按倒在地便要行刑,这时候,一个头盔散落,狼狈不堪的将领被雍军捆绑着送到此处,见到那郡守将被处斩,那将领嘶声问道:“郡守大人,你为何定要死守不退,又不肯从上命归降?”

那朱姓郡守道:“我受朝廷之命牧守钟离,岂能弃城而逃,且雍军攻势猛烈,若是存了求生之念,钟离早已陷落了,想要退守,谈何容易,何况这么多将士已经先行一步,本郡守如何能够让他们久等。大将军虽然宽宥,但是你我都是南楚臣子,怎能不为国舍命。”说罢,那朱姓郡守引颈受剹。

那将领听了叹息道:“郡守大人一介文士尚且以身殉国,何况是陈某这等武人呢?” 他被俘之后,本来存了投降之心,见到郡守殉死,再也不能贪生,进到堂内,董山虽然出言劝降,他却是一言不发,董山不耐,也下令将他处死,那将领至死再无一言。

在钟离修整一日之后,董山带着徐州军赶到了寿春,距离寿春还有二十里,南阳大营崔珏已经派出使者亲迎,这次攻打寿春,南阳大营才是主力,不过因为南阳大营将士对淮南地理不熟,所以朝廷才决定由裴云派出一部人马支援长孙冀。不过董山和崔珏倒是旧识,两人都曾在齐王麾下效力,数年前才各奔前程的。那个前来迎接的亲卫是崔珏族侄崔放,也是董山旧识。他策马上前,上下打量了崔放片刻,朗声笑道:“好小子,几年不见,你已经这么大了,怎么样,战况如何,你叔父身体如何?”

那年轻亲卫也笑道:“董叔,我叔父身体很好,战况很激烈,寿春守军几乎是不要性命的抵抗,叔父正觉得兵力不足,你们来了可就好了。”

董山心中一震,看来寿春这里也不轻松啊,随即他肃然道:“徐州大营副将董山奉淮南节度使裴将军之命前来听从崔将军调遣。”

那信使见状也正色道:“南阳大营平远将军崔珏,奉长孙将军之命攻寿春,属下崔放,奉将军命迎接董将军。”

两人说罢相视一笑,董山传令让麾下将士先去扎营,自己带了几个亲卫跟着崔放去阵前寻找崔珏去了。

寿春城前,烟火弥漫,三十余岁年纪的崔珏皱着眉望着前方,他本是一个相貌端正的男子,可惜容貌却被面颊上的一道刀疤破坏无遗,董山策马来到阵前的时候,正见崔珏用马鞭指着寿春城上道:“令敢死营登城,从那里上,那里必然有敌军大将,否则守军不会如此顽强。”军令传下,不多时,一营带着肃杀之气的青甲军士向寿春城奔去。董山自然知道这些是犯了军法的军士,或者干脆就是充军的囚犯,若是能够立下大功生还,便可恢复自由之身,所以作战之时都是奋勇争先,最是勇猛不过,雍军各军中都有这样的建制存在。

这时崔珏已经发觉董山来了,回头笑道:“钟离已经攻破了?我可还在这里焦头烂额呢。”

董山在马上一揖道:“崔大哥,一向可好,你就别打趣我了,一个小小的钟离我攻了五日,结果连一个重要的俘虏都没有到手。”

崔珏奇怪地道:“怎么,守将和钟离郡守都战死了么?”

董山惭愧地道:“本来都被我俘虏了,却是我一时火起,将他们都斩了。”

崔珏微微一愣,笑道:“这也不算什么,裴将军也不会因为这件事情责怪你,多半还会替你掩饰一二,不过淮西的南楚军果然是骁勇善战,你的军队先休息一下,明日和我一起攻城,也不知敢死营能不能将那里的守军重创。”说罢他提鞭指向寿春城,董山也向上望去。

只见敢死营的军士已经顶着箭雨滚石登上了城头,似乎没有什么阻碍,董山一皱眉,道:“看起来似乎很容易。”

崔珏也疑惑地道:“奇怪,这些天我攻城多次,每次从这个方向都十分艰难,就是上了城头,也是没有一人能够生还,怎么今次这样轻易。”

两人眼看着敢死营勇士的青甲消失在跺口,都生出莫名的感觉,这一次的攻击定然不会成功。就在这时,寿春城头突然传来混杂着惨叫的厮杀之声,而在那里的跺口又出现了南楚军的身影。

崔珏和董山面面相觑,崔珏苦笑道:“想不到这一次他们却是用了请君入瓮的诡计。”

董山叹息道:“想必是他们也知道敢死营的厉害之处,所以索性让他们攻了进去,慢慢歼灭他们,我们看不到实际的战况,若是想根据那里的战况决定下一步的攻势,所作出的任何决定都可能是错误的,守卫那里的将领必然是自信十足且颇富计谋,可是我见帅旗不在那里,想必是个寻常将领,寿春城也当真是人才济济。”

崔珏知道这次敢死营恐怕是自投罗网了,但是毕竟敢死营必定还在苦战,胜败未可预料,所以还是调派重兵趁机抢城,传令下去之后他苦笑道:“谁说不是,裴将军在淮东势如破竹,我们在淮西却是步步艰难。”

董山安慰他道:“这可怨不得你我,淮东军糜烂已非一日,裴将军数年来派了无数斥候到淮东探查军情,对于淮东将领早已了如指掌,若非如此,裴将军怎会孤身涉险入楚州大营行刺敌军主将呢。”

崔珏一边留意着寿春城头的情形,一边笑道:“我可是听说,皇上下了旨意申斥裴将军,不许他再涉险行事,差一点就将他独自夺取楚州大营的功劳也给抹去了。”

董山不为意地道:“将军才不会放在心上,不过暂时想必是不会再轻身涉险了。”

两人说着闲话的时候,城头上厮杀之声已经消失了,崔珏微微苦笑,知道自己赋予重望的敢死营已经全军覆没了,便传下军令,缓下攻势,这一次的攻城又失败了。

城头之上,陆云喘着粗气坐倒在地上,看着重围中横七竖八的雍军敢死营尸体,再看看手上已经卷刃的钢刀,身上血染战袍,地上血流成河,方才这场厮杀可是让他从鬼门关打了一个转,若不是两个军士拼着一死替他挡住了敌人的刀剑,只怕他已经人头落地了。虽然他是将门之子,又是内外兼修,双臂神力,可是和这些悍不畏死的军士比起来,还是差些气势,想到此处,不仅有点后怕,自己这请君入瓮之计差点成了引狼入室。可是这有什么办法,明明知道来敌是敢死营的勇士,若不将他们围起来歼灭,而只是抵抗敌军的强攻,只怕会被敌人攻破防线的。

将战场清扫了一下,负责防守这一带的将领陈明走了过来,笑道:“少将军,果然好计策,我们从前也和敌军的敢死营做过战,若是没有三倍以上的损失,是不可能消灭敢死营的,这次我们损失少了一多半。”

陆云脸上一红,道:“都是大家拼力死战,我不过是出个主意罢了。”

陈明拍拍他的肩道:“不愧是大将军之子,我们将军派人请你过去一趟。”

陆云犹豫了一下,道:“现在方便么,敌军还在攻城呢?”

陈明笑道:“没关系,雍军已经势弱了,这一天又可以顺利撑过去了。”

这时一个军士高声喊道:“不好了,敌军打出了徐州大营的旗号,钟离完了。”

陆云和陈明都是一惊,几步跑到城跺前向下望去,只见雍军的中军帅旗旁边,又多了两面大旗,一面是徐州大营的旗号,另一面旗帜上面有一个大大的“董”字,陆云浑身一震,明明知道钟离陷落是迟早的事情,可是真的知道仍然是这样难以接受。

这时,雍军中有人高声喝道:“我军已经攻陷钟离,钟离郡守和钟离都尉的人头在此,寿春守将听着,若是不降,一旦城破,尔等也将悬首城门。”说着有人用旗杆挑起两个人头立在阵前。

城上的守军一片哗然,士气一时间滑落了许多,许多将士涌到城墙边,向下望去,看见高挑的人头,虽然看得不甚清楚,可是城头上已经是一片愁云惨雾。

这时,陆云身边突然传来牙齿咬得咯咯作响的声音,陆云偏过头望去,只见陈明满目怒火杀气,望着雍军中军的“董”字大旗,脸上的神情悲恸莫名,眼中一滴滴落下泪来。他心中疑惑,向左右看去,一个军士低声道:“钟离陈都尉是陈大哥的兄长。”陆云一声惊叫,黯然地看向陈明。这时候只见陈明跃上城跺,高声道:“城下的贼子听着,你们杀了我兄长,我陈明拼着性命也要报此血仇,兄弟们,干什么垂头丧气,朱郡守和陈都尉已经为国尽忠,难道我们还要让他们在阎王爷面前笑话我们贪生怕死么?”

从寿春正面的帅旗下,一个低沉有力的声音道:“誓死守城,杀敌雪恨。”寿春守军闻声也随之高呼道:“誓死守城,杀敌雪恨!”声音惊天动地,再也没有方才的悲恸消沉。

城下的崔珏和董山相视一眼,打击敌人士气的计策失败了。崔珏一皱眉,对一个亲卫使了一个眼色,那个亲卫是有名的神箭手,在长孙冀麾下,擅长箭术的将士本就特别多些。他领会了崔珏的意思,策马上前,在几个军士的掩护下,一箭向城上射去,这一箭如同流星电闪,几乎看不清箭影,三百步距离转瞬穿越,向仍然站在城跺上的陈明射去。陈明仍在望着兄长首级流泪,丝毫没有留意雍军的暗袭,城上众军都是大声呼叫道:“小心!”

但是比起他们示警的叫声更快地是两道箭影,从陈明身后和帅旗所在之处分别射出,这两道箭影几乎是同时射中那支偷袭的箭矢,那支箭矢断成了三截,那两道箭影也是反弹而回,可见力道上要差一些,城上的守军都是高声叫好,城下雍军却也高声叫道:“好箭法!”雍军本来就不吝于对敌人的赞誉,不过他们的战意不仅没有削弱,反而更加旺盛起来,都是跃跃欲试。

崔珏和董山都是露出苦笑,城上敌军士气正旺,己方虽然也被挑起了战意,可是若是这个时候继续攻城,除了增加损失之外,是绝对不可能成功的,看看天色,两人同时决定收兵。

望着缓缓退去的雍军,陆云放下弓箭,心中感叹道,怪不得大雍多年来可以在群雄环伺下屹立不倒,只见这些军士竟替敌军喝彩,而又丝毫不曾减弱气势,反而更加生出斗志,就知道即使是父亲麾下的精兵也比不上他们,终究是缺少这般的信心和坚定。这些雍军,只怕失去了主将仍然能够进退有序,而若是父亲出了什么意外,江夏大营和九江大营都会群龙无首,慌乱失措。

在陈明的谢意和其他将士的赞颂声中,陆云好奇地问道:“不知道方才是谁和我同时发箭的,我怎么不记得石将军身边亲卫有这样的高明的箭手呢?”那些将士听了,突然都露出诡秘的笑容,陈明已经从丧兄的悲痛中挣扎出来了一些,强笑道:“少将军,反正我们将军正在那边等你呢,你何不过去看看呢?”

陆云心道也是,就向那边走去,不多时走到帅旗之下,只见淮西主将石观正在那里吩咐整修城墙,准备明日的作战。陆云的目光却是一下子就落到了站在石观身边的一个少年身上,那个少年年纪和他相仿,相貌和石观有七成相似,只是眉宇间秀气许多,石观本就是相貌堂堂,那少年自是俊美端秀,虽然不如陆云雄壮,可是腰间佩剑,肩上挂弓,一身剑气隐隐,英姿飒爽。

陆云一见这少年便觉得惺惺相惜,心中觉得定是这少年射了方才那一箭,但是不便先和他说话,上前对石观施礼道:“将军传唤,陆云姗姗来迟,请将军恕罪。”

石观看了陆云一眼,笑道:“云侄果然是年少英杰,箭术超群,用兵也颇有章法,不愧是大将军虎子,你也不要过于客气了,我在镇远公老将军麾下多年,和你父亲也是兄弟相称,如今虽然权位悬殊,不过想来你叫我一声世伯还是应当的。”

陆云原本是因为这位石将军严肃可畏,一直不敢使用这样亲切的称呼,只是按照军中的规矩称呼他将军,今日见石观神态和蔼,心中一宽,下拜道:“侄儿陆云拜见世伯。”

石观伸手相搀,指着那个俊秀的少年道:“这是我的女儿石绣,自幼顽劣,被她祖母、娘亲当成男孩养大的,比你大一岁,你就叫她姐姐吧。”

陆云瞪大了眼睛,这怎么可能,这个少年虽然俊秀非常,可是眉宇间英气勃勃,完全没有一丝女孩儿家的娇柔温婉,怎么可能是个少女。

石绣见状冷冷一笑,上前就是一脚踢去,正中陆云的小腿,陆云痛得一个踉跄,差点叫了出来,石绣怒道:“瞪着眼睛看什么,还有,不许叫姐姐,若是你敢乱叫,可别怪我砍你十剑八剑。”

石观只装作没有看见,撇开两人继续安排军务,他这个女儿自幼男装,哪有半分女孩子的模样,若非如此,怎会明年就要及笈了,却还没有许人,就连自己麾下的将士也都乖乖叫她少爷或者少将军,有些人甚至都不知道石绣原本是一个女孩儿,不过他总不能对陆云说自己有个儿子吧,而且这几日通过对陆云明里暗里了解,他心中倒有一个想法,只不过不知道是否高攀,所以一上来就说明了石绣的身份。

这两个少年少女自然不明白他的心意,见石观忙着处理军务,石绣扯着陆云到一边去,威胁利诱,不许他以姐姐相称。

石绣上面本来有一个兄长,只是年幼夭折,所以石绣出生之后,石观为了安慰母亲和妻子,就将石绣当成儿子教养,石绣也是性子像极了父亲,女孩儿擅长的女红之类一概不通,对于弓马武艺却是一学就会,后来又拜了一位从蜀中避难而来的峨嵋高手学习内家拳剑,小小年纪,武功已经出类拔萃。她性子刚强,不喜欢和那些同龄少女一起做女红,只喜欢使枪弄剑,走马射猎,一见陆云也是小小年纪便武艺高强,心中生出意气相投之念,相谈片刻,两人已经是言笑宴宴,和乐如同手足。

第二日,崔珏和董山重整旗鼓,再次攻城,这一次两人也不理会什么攻心和士气的事情,只是中规中矩的攻城,抓住每一个破绽,捕捉每一个时机,在如同细水长流的攻势中,不时发起*似的攻击,夜袭、突袭,无所不用其极,石观也是毫不示弱,守城时稳如磐石,夜里也趁机偷营截寨,整整十二天,两军几乎是将所有攻城守城的手段一一演练了一遍。借着坚城的保护,寿春守军可以说和雍军实力相当,战力上面,雍军虽然强些,但是淮西军也不是弱者,可以说双方拼得就是士气和毅力。这方面寿春守军也不欠缺,陆云这些日子几乎是敌军从哪里主攻,他就到哪里去守城,从初时的稚嫩,到后来的成熟,他成了南楚军千里挑一的勇士,就是下面攻城的雍军,也知道寿春有一位年纪不大的神箭手,少年勇士。这样的陆云成了寿春军民心中的支柱,只要陆云在这里,那么就一定会有援军,陆云小小年纪就这样勇猛,陆大将军一定是名不虚传,只要援军一到,就可以击败雍军。这样的念头让每一个淮西将士都悍不畏死,也让寿春成了雍军心目中收割人命仅次于襄阳的修罗场。

石绣也没有丝毫示弱,对于陆云,她有着极强的较量意识,她的宝剑雕弓,收取的性命不比陆云少多少,而且不知是有意还是无意,两人都穿着同样的盔甲,身量相仿,有着同样出神入化的箭术,虽然一使刀,一使剑,可是在雍军眼里,他们被当成了同一个人,所以寿春的少年勇士瞻之在左,忽而在右,成了雍军心目中颇为神秘可怕的眼中钉。

十一月二十日,酉时,雍军终于停止了攻势,再次毫无所获地退走了,陆云望着远去的雍军,这些日子,因为南楚军的袭营,雍军已经将大营挪到了十里之外。陆云疲惫不堪地活动了一下麻木的四肢,将手中的横刀丢落,他自己的钢刀早已毁去,这柄刀是从攻城的雍军手中多来的,用得卷了刃自然丢掉即可。这时候,石绣大踏步走了过来,她身上的戎装也是尽被血染,在守城或者袭营的时候,两个人颇有默契地不在一个地方出现,但是冥冥中似乎有无形的力量让他们彼此牵绊,即使隔着千人万人,似乎也能够感觉到对方的存在。

-------------------------------

石绣上前对陆云道:“云弟,今晚还去劫营么?”

陆云摇头道:“玉锦,今天不行,连续劫了三日,今天雍军一定会有防备,我已经跟伯父说过了。”在雍军和南楚军彼此偷营袭城的过程中,陆云表现出了十分机敏的直觉,选择劫营时机十分恰当,而且敌军若有埋伏,陆云总能在斥候探查之前便生出不妥的感觉。就连陆云也觉得奇怪,是不是在长安上了太多的当,让他变得这般敏感。至于称呼石绣“玉锦”,则是因为石绣不许他称呼姐姐,直接称呼名字又觉得失礼,所以陆云索性称呼石绣的表字,这是半年前石绣的师父离去之前赠给她的字。

石绣点点头,无所谓地道:“好吧,那么咱们回去吧,这一身血衣穿着多不舒服。”说完不耐烦地耸耸肩,这个姿势若是别的女子做来必定粗野难看,可是石绣做来,却有一种洒脱不羁的感觉,更何况她本就穿着男装,活脱脱一个少年将军,哪里有半分女儿情态。

这本是陆云看惯的动作情态,可是不知怎么,今日陆云心中突然一颤,竟然想起了原本已经在记忆中深藏的昭华郡主江柔蓝。初次相见,柔蓝也是穿着男装,可是和石绣不同,她虽然穿着男装,却是那般的娇俏端丽,她的气质纯净,如同清泉一般明晰,或许是身份的缘故,她的光芒是那般耀眼,虽然没有娇纵之气,甚至可以说是善解人意,天真无邪,可是陆云总觉得柔蓝有一种仰之弥高,望之弥远的气质。可是眼前这个少女,却让陆云有一种亲切的感觉,如手足,如骨血,不可分割,两人相处之时,几乎不需言语,就可以沟通无碍。石绣看看莫名其妙发呆的陆云,习惯性地一脚踹去,陆云下意识地想避开,可是不知怎么看到石绣带着嗔意的目光,身躯便移动不了,结果被踢得结结实实。陆云一声惨叫,引得众将士掩嘴偷笑,这样的好戏这些日子总在上演,他们早已经看得熟了。

这时,石观身边的亲卫奔过来道:“少将军,少爷,将军召你们过去。”

陆云和石绣奇怪地互望一眼,然后陆云不再揉腿,直起身来,和石绣一起向石观所在的方向走去,到了石观处,见他左臂上停着一只灰羽红睛的信鸽,陆云心中一动,上前惊喜地问道:“伯父,可是反攻的时候到了?”

石观微微一笑,将手中的一张细绵纸递给陆云,陆云拿过一看,只见上面绘着只有一个铁划银钩的“战”字,下面盖着南楚大将军陆灿的金印,除此之外字条一角还有一个小小的“丙”字,陆云只觉得心中狂喜,再也说不出话来。石绣在旁边看的迷糊,索性抢过字条,翻来覆去地看着。

陆云向石观施礼道:“伯父,陆云也想随伯父上阵杀敌,请伯父准许。”

石观微微一皱眉,守城的时候陆云自然可以参加,偷营的时候也不妨事,可是反攻在即,战阵之上,刀枪无情,若是陆云有个闪失,自己可怎么向大将军交待?见他犹豫,陆云连忙道:“伯父,您也知道,我是迟早都要上阵杀敌的,这些日子我的武艺您也见了,这次上阵我一定紧跟着伯父,绝不会擅自冲杀。”

这时候石绣将字条看了半天也不明白其中含义,便又还给了陆云,陆云这时正在满怀热望地望着石观,却是极为顺畅地接过字条,见到两人之间的小动作,石观不由一笑,心道,我这丫头终于可以嫁出去了,罢了,这小子迟早也要上阵的,跟着我总比跟着别人好,便道:“好吧,你准备一下马匹武器,到时候跟在我身边护卫。”这下石绣可听明白了,原来是要出城作战了,连忙道:“爹爹,我也要上阵杀敌。”

这次石观可不答应了,怒道:“胡闹,一个女孩子,马上就要嫁人了,也不知道学些中馈之事,就知道舞刀弄剑,这次不行,乖乖呆在城里。”

石绣扯着父亲战袍道:“爹爹,我哪里比云弟差,他都能上阵,我为什么不能,最多我也呆在爹爹身边护卫就是了,再说我可不嫁给那些娘亲选的官宦子弟,要嫁便嫁给能够和我一起上阵杀敌的英雄好汉。”说到最后一句,她的脸上也有了一丝羞意,可是双目目光炯炯,竟是没有一丝退缩。

陆云被她神光所摄,不由道:“伯父,玉锦武艺那样出众,就让她一起吧,在战场上我一定会好好保护她的。”

谁知石绣不领情,飞脚踢去,道:“谁要你保护,我武艺比你差么。”陆云不敢闪躲,只是苦着脸硬受了这一脚。

石观忍住狂笑的冲动,再看看石绣一副你不让我上阵,我便自己跟去的模样,心道,也罢,还是留在自己身边放心些,便道:“好吧,你们两个一起都去,不过不许离开我的左右。”

陆云和石绣都是十分欣喜,自然而然牵着手跑去整理马匹和兵器,浑然没有察觉应该避嫌。石观眼中闪过喜悦的神色,然后面色沉静下来,又看向那张字条,“丙”,那么至少已经失落了“甲”、“乙”两份传书,雍军的防范很严密啊,不过就算是字条落入雍军之手又有什么关系,这张字条不过是个信号罢了。

第二日,陆云和石绣都是全副披挂,偏偏一日都没有任何意外,雍军和南楚军都已经熟悉了对方的战术,几乎是敌军一动,便知道如何应对,厮杀虽然惨烈,却是全无新意。日落时分,崔珏随手丢去手上的两张字条,道:“果然是无稽之谈,定是南楚军有意迷惑人心,陆灿就是天大的胆子,现在也不敢离开京口。”一阵风吹过,那字条在风中翻转,露出上面的金印。

十一月二十一日,石观仍然令将士披挂好,准备随时出战,更是抽出一部精兵,让他们养精蓄锐,双方战到午时,太阳移到南面的天空,今日是难得的晴朗天气,虽然冬日天气有些寒冷,可是城上城下的将士都是汗透重衣,双方都已经是强弩之末,几乎全凭毅力在苦斗,十几日毫不间断的攻守,实在是消磨人的体力和意志。

崔珏和董山对望一眼,都看到对方眼中的忧虑,董山犹豫地道:“裴将军和陆灿在扬州对峙,我们攻略淮西,这本是既定之策,可是淮西战况这样艰难,真是始料未及。”

崔珏道:“那也没有办法,反正寿春没有援军,总归是我们占优势。罢了,再猛攻一次,趁着中午守军疲惫加把力。”

董山点点头,这本是惯例,这一次攻击若是不能得手,便会撤退休息到未时,然后再一鼓作气攻击到日暮。

崔珏催动三军,开始攻城,换下来的疲军几乎是倒地便睡,连日来的疲惫不仅仅在身体上,也在精神上,看着这种情况,崔珏动动嘴唇,终于没有下令让那些军士警戒。

这一次的攻势似乎效果很不错,寿春的防守有些软弱,在雍军不遗余力的猛攻下有了溃败的迹象,崔、董两人都是心中一喜,交换了一个眼色,派出最精锐的敢死营,准备给寿春守军决定性的一击,或者今日就可攻破寿春,这不仅是两位将军的想法,就是攻城的军士也感觉到了城头守军的力竭,都是拼命攻去。

就在这时,数里之外的山坡林木之后,一双眼睛闪现出杀机,轻轻举手,身后传来有些带着紧张的呼吸和战马轻微的喘气声。然后那人断然挥手,一马当先绕过缓坡,绕了一个弧形,向雍军后阵冲去。

“杀!”高亢入云的喊声、震耳欲聋的马蹄踏地的声音以及战鼓隆隆的声音同一时间响彻云霄,崔珏和董山心中一惊,向侧面望去,只见远处烟尘滚滚,一支骑兵正在袭来,一时之间看不出人数,但是总在五千之上,那些骑兵皆着银甲,衣甲映着明亮的阳光,令人几乎无法睁开双眼。

怎会这样,两人心中都是惊骇莫名,南楚长于舟师,对于骑兵并不十分重视,据他们所知,如今整个南楚,除了襄阳的九千骑兵,江夏大营的三千骑兵之外,整个南楚几乎再也寻不出一支有足够战力的骑兵,这些骑兵多半是当年德亲王打下的底子,可是这支骑兵是从哪里来的?千万种思绪一闪而过,两人都是同声高呼道:“退,撤退。”

可是这时候那支银铠骑兵已经冲入了雍军后阵,雍军本已疲惫不堪,又在促不及防的时候,一触之下,雍军立刻陷入了混乱和崩溃的局面,那支骑兵肆无忌惮的冲杀着,仿佛利刃一般将雍军切得四分五裂,就在这时,寿春原本已经从里面封住的城门开了,这原本是雍军的期望,可是如今却是雪上加霜。站在城门口高据马上的大将正是石观,在他左右,两个白衣白甲的少年将军一左一右相护,两人手中都是一杆银枪,背上挂着雕弓,马上悬着箭囊,就连两人的战马也都是极为相似的白龙马,面甲都是放下的,看不到两人相貌,虽然身材有些不同,可是在战甲掩盖下看不出来,这两人竟似是一对双生兄弟,许多看到的雍军心中都无端生出“原来如此”的念头,脑海里闪过这些日子活跃在寿春城头的少年勇士的形象。

只是这些雍军马上就看到那将领挥刀前指,城内的五千生力军冲入了雍军前阵。寿春守军并没有成建制的骑兵,除了石观身边这支百人左右的亲卫之外,再无战马,可是他们的战力并不弱,而他们的出战让雍军心灵受到的重创并不弱于后面冲阵的骑兵,原本困在网中的鸟雀破网而出,那么猎人的心情可想而知。

在南楚军两面夹攻之下,六万雍军岌岌可危,攻城的损耗太大了,崔珏和董山对视一眼,目光交汇之处,已经是争吵了无数次,然后董山一抱拳,高声道:“随我来。”然后便向南楚军迎去,崔珏目中闪过悲色,也高声道:“随我来。”然后向东南方向冲去。随着两人的分头行动,徐州军下意识地跟随着董山断后,南阳军则随着崔珏突围。

天地间杀声震耳,南楚两军仿佛是两只铁拳,相互呼应着杀戮着雍军,而雍军毕竟是百战精兵,在董山的拼死断后下,崔珏终于成功地带着三万多人杀了出去,转道向北而去。南楚军没有追击,而是专心致志地消灭董山部,留下断后的一万七千徐州军和没有来得及逃走的一万余南阳军虽然舍命相博,但是养精蓄锐的精兵对着久战之后的疲兵,又是占了先机,胜负已定。当太阳西垂的时候,战场上已经只剩下数千残军。而南楚军却是越战越多,城中休息过的淮西军也加入了战场,两万多淮西军加上来援的九千骑兵,将雍军困在阵中。

董山只觉得鲜血蒙住了眼睛,忍不住用袍袖擦拭,定睛瞧去,南楚军的骑兵虽然骑射出众,武艺高强,可是仍然能看出一丝生疏,这是经过良好训练,但是没有真正上过战场的军队,只不过今日之后就不同了,这场胜仗将让他们成为真正的雄兵。耳边传来同袍的微弱的呻吟声和低沉的咒骂声,董山的目光落到了一双并肩作战的少年将军身上,他们手中的银枪上下翻飞,一刚一柔,配合得天衣无缝,一个如同蛟龙出海,一个幻化出点点梨花,在他们身后,留下的是一片血海。

这时,南楚军中竖起的“石”字帅旗下,一个中年将领高声道:“董山,你们已经陷入死地,何不弃械归降?”随着他的喊声,南楚军开始放缓攻势,却又加强了包围。

董山传令让雍军向自己靠拢,高声道:“大雍男儿,岂有归降的道理。”

这时,南楚军中一个低级将领高声喝骂道:“董山,你杀了我兄长,陈某正要寻你报仇,你不降最好。”

董山冷冷看了那将领一眼,笑道:“董某在战场上厮杀了十年,杀过的人数不胜数,谁知道你的兄长是哪一个,想要报仇,就拍马过来,何必惺惺作态。”

那将领大怒,但是他没有骑马,自然没有可能向一个骑兵将领冲杀,只恨得眼眦欲裂。

这时候,那从乱军中返回石观身边的两个白袍小将,其中一人掀起面甲,高声道:“董将军,你或许不将自己的生死放在心上,难道不爱惜你的将士,难道你要让麾下将士全部死绝么?你若肯放下兵器,我保证你麾下的将士会得到应有的礼遇,我军绝不会残杀虐待他们。”

董山目光炯炯地望着那个少年,看上去不过十三四岁年纪,却是英气勃勃,好一个少年英雄,他哈哈一笑,道:“若要董某归降,那是不可能的,这样吧,你们若有勇士可以在战场上胜了本将军,本将军在此立誓,不论我是生是死,我麾下将士皆会弃械归降。”

石观的目光和那支骑兵为首的一人交换了一个眼色,他们并不是心慈手软,只是担心这支雍军临死之前的反噬让己方骑兵损失太大,那就不值得了,可是若论单打独斗,又有何人有把握可以胜过这个大雍将领,若是败了,又如何面对同袍和陆大将军。两个人的目光不约而同落在了陆云身上,陆云是陆灿之子,若是他和董山一战,不论胜败都可交待,毕竟他只有十三岁,可是两人又都担心陆云有了什么意外,那可就糟了。

见南楚军迟迟没有回应,董山仰天大笑道:“江南果然没有好汉,竟然没有人敢和我一战。”

他的狂言却惹恼了一人,石绣原本还在担心自己杀昏了头,早就忘记了留在父亲身边的约定,一会儿要被父亲责骂,此刻一见董山的放肆狂妄,她柳眉倒竖,掀起面甲,高声道:“董山,别说江南没有英雄好汉,就是我们这些小孩子,你也未必胜得过,你若有胆量,我和他一起向你挑战,我们两人年纪加起来也大不过你,你可敢应战。”

董山一怔,不过他想起两个少年方才的骁勇,倒是不觉得受到侮辱,心道,他们小小年纪,就上阵杀敌,倒也算是英雄,若是死在这样两个少年英雄手上,倒也不算侮辱,若是杀了他们,更能铲除两个祸根,当真是合算得很。所以他不容石观等人反对,策马冲出雍军军阵,朗声道:“好,我董山接受你们的挑战,报上名来,让本将军知道杀的是谁。”

陆云闻言,心中豪气顿生,早就忘了反对,朗声道:“家父忝居大将军之位,我名陆云,董将军可要记住了。”

石绣却是聪明,女孩子的名字怎可随便让人知道,她虽然不忌讳,若是母亲知道必然恼怒,便扬声道:“家父淮西主将,我名石玉锦,董将军不可忘记。”

董山虽然早已料到这两个少年身份不同寻常,却也想不到一是陆灿之子,一是石观之子(他没有看出石玉锦是个少女),朗声笑道:“好,原来是两位少将军,果然是将门虎子。”

说罢扬槊冲上,陆云和石绣对望一眼,双双策马冲上,石观连忙下令调动弓箭手,一旦董山有可能伤及陆云和石绣,他是无论如何也要放箭救人的。

三马盘旋,两条银枪和一条马槊在尘沙中奋战不休,青黑色的衣甲和白色的衣甲交错混合,这一战并没有像大多数人想得那样一面倒,董山虽然是大雍悍将,可是陆云和石绣也是武艺不弱,再加上两人心有灵犀,配合严密,董山又是筋疲力尽,居然战得平分秋色。

一个回合,十个回合,一百个回合,当战到百合之后,三人都已经人困马乏,董山在马上摇摇欲坠,只是石绣和陆云也好不到哪里去,陆云毕竟是男子,这些日子又服用了江哲所送的丹药,固本培元很有益处,尚能支撑,石绣却是气喘吁吁,已经是汗透衣甲,手中银枪似乎也握不住了。董山见状,奋起余力向石绣攻去,不再避让陆云的银枪,虽然在他来说陆云更有价值,可是自恃力量不足的他,选择了更好下手的石绣。一槊刺去,透甲而入,石绣的银枪脱手,翻身坠马。

陆云只觉肝胆俱裂,一声断喝,悲愤让他全力催枪,银枪化作虹影,向董山背后刺去,但是就在银枪即将着体之时,董山的身躯在马上诡异的扭动,那一枪只是透过了右肋,陆云用力过猛,身躯前倾,董山却是微微一笑,马槊刺向陆云咽喉,全然不将身上的伤势看在眼里。

几乎是顷刻之间,局势突变如此,南楚军一片哗然,石观想要传令放箭,却是身躯僵硬,只是望着爱女向下坠落的身躯,一个字也说不出来,一个动作也做不出来。

眼看董山的马槊将要刺穿陆云的咽喉,董山面上露出欢容,能够在临死之前杀死南楚两位未来的英杰,便是死也值得了,谁知胸前一痛,他缓缓低头,看见胸前透出的银色枪尖。马槊锋利的尖锋即将临喉,陆云濒死的一刻,眼前突然闪现出石绣怒目圆睁,银牙紧咬的俊秀容颜,几乎是疑在梦中,可是透过董山胸口的银枪,和减缓的马槊刺击速度让他立刻醒悟过来,一个蹬里藏身,翻身落马,银枪收而再吐,这一枪刺中了董山小腹。受了致命的三枪,董山眼中的生命光芒终于消散,他留恋地望了一眼北方的天空,身躯从马上滑落。

陆云听不见耳边传来的南楚军震耳欲聋的欢呼声,也听不见雍军痛彻心肺的悲呼声,他翻身上马,怔怔望着对面的石绣,两人隔着失去主人的空鞍战马痴痴相望。

方才董山一槊刺中石绣的之前的瞬间,石绣便清醒过来,她心中灵光电闪,便徉做中槊坠马,其实那一槊只是留下了一道不深的伤痕,只是董山已经疲倦不堪,手感麻木,完全没有察觉那一槊根本没有击实。当他回身反噬一击的时候,石绣已经翻身而起,崩飞的银枪正如她预计的一般落入手中,她拼尽全力一击,刺出了致命的一枪,才让董山手中力道减弱,陆云得以死里逃生。

耳边欢呼声依旧,两人眼神渐渐恢复了生机,都已经感觉到生命重新回到自己身上,想起方才的生死一线,两人都是不由打了一个冷颤,策马转身向石观走去,两人的目光始终不曾分离,生恐眼前见到的只是虚幻,对方早已死在董山之手。

这时候石观已经清醒过来,悄悄抹去眼中的泪水,他策马迎上,两手各自抓着两小一臂,高声呼道:“天佑南楚,赐我少年英杰。”南楚军高呼道:“天佑南楚,赐我少年英杰。陆云、石玉锦,陆云、石玉锦!”呼声连绵不绝,震撼人心。在南楚军的欢呼声中,一个雍军军士黯然丢下手中兵刃,其他的雍军将士似乎是受到了感染,兵器坠落的声音络绎不绝。

\chapter{第十五章 楼船夜雪}

初,灿驯精骑于蜀中,隐秘不为人知,雍军崔、董部合攻寿春,石观坚守不退,灿密令精骑潜行赴淮西,二十一日,雍军猛攻疲敝,至午时,南楚精骑突出,大破雍军于城下。雍军以董山部断后,崔珏部突围而走,然折损十之四五。

董山,陇西天水人,少无父母,好勇斗狠,为亲族所恶,乃从军行,初为齐王部将,隆盛五年,转任徐州,为淮南节度使裴云部将,隆盛七年,奉命入淮西,取钟离,攻寿春,寿春大败,董某自请断后,为南楚军所困。时,楚军欲招降,为其言辞挑之,出陆云、石玉锦与其死战,陆、石阵斩董山,雍军乃降。

——《资治通鉴·雍纪三》

十一月二十一日夜晚,京口瓜州,大雾垂江,陆灿立在楼船之上,望着滔滔江水,在他身后,九江大营的水军已经做好准备,利用这个机会渡江偷袭,和雍军对峙了二十余日,陆灿虽然表面平静,但是心中却是忐忑不安。

他并不担心对岸的裴云,对岸雍军虽然将近十万之众,但是水军却只有两万余人,舟船不到千艘,这样的兵力,想要渡江攻取京口殊不可能,当然,即使他有意夺回扬州,凭着五万水军也是很难成功,在瓜州渡口,两军都没有必要的胜算,这也是这些日子两军都没有主动挑战的缘故。只不过裴云可以安之如素,自己却是牵挂着数处战局,淮西能否按照自己的计划取得胜利,襄樊能否稳如泰山,葭萌关是否能够安然无恙。而在这其中,最重要的就是淮西之战,淮西若败,从此淮南不属南楚,雍军便可从容截断襄樊和江陵之间的联系。这样一来,荆襄孤立,在长江下游,雍军又可兵临长江,除了长江之外,再无缓冲的余地,到了那时,就是孙武再生,也不可能挽回大局了。

在得到雍军的动向和各路的兵力布置之后,陆灿看得出来,对于淮西的重要,雍军也是心中有数,蜀中和襄阳都以大将主攻,这是为了牵制两地,不令他们分身,否则这两地都是易守难攻的所在,且负责守备的南楚将领也不是凡品,雍军若真心攻取一处,至少兵力要增加到一倍以上才行。淮东局势糜烂,裴云单刀直入,原本雍军可以将此地当做突破口,可是想必雍帝也看出来淮东水网纵横,更加有利南楚军的攻防,所以虽然裴云轻取淮东,却仍然不曾妄进,甚至有意诱使自己陷入淮东争夺的泥潭。所以对于雍军来说,真正的目的还是淮西,虽然大张旗鼓,用三路大军的攻势掩盖雍军的真正目的,可是兵锋所知只能是寿春。

不过陆灿虽然看出了这一点,却也是无可奈何,余缅、容渊若是稍有松懈,雍军趁势大举进攻也是极为可能的,而京口如不防范,裴云也必会渡江取建业,一旦十万雍军步骑过了江,以建业禁军的实力,只恐昔年旧事重演。所以纵然以陆灿之能,也只能看着雍军取淮西,若是雍军派出大将重兵攻略淮西,那么陆灿也无能为力了,偏偏大雍朝野弥漫的轻敌之心让李贽没有派出大将督军淮西,只是由长孙冀和裴云各自派军组成联军攻寿春,这一来,陆灿就有了反败为胜的机会。

为了取得淮西的胜利,陆灿可以说用尽了全部心力,淮西主将石观,虽然不是什么奇才,但是却坚韧冷静,足可信任,为了迷惑雍军,不让雍军派出可独当一面的大将攻淮西,陆灿故意“疏忽”了寿春战局,不曾派援军救淮西。然后他不吝惜爱子性命,让陆云到寿春辅佐石观,这实在是一件十分危险的事情,稍有不妥,即使淮西大捷,陆云的性命也会葬送在寿春。可是如果不这样做,就不能稳定寿春军民之心,也就不能将雍军拖到精疲力尽的地步,更不可能凭着九千精骑大破雍军。最后,陆灿调动了一直以来雪藏的飞骑营。

南楚并不重视骑兵,这是因为地势所限,也是因为南楚自立国以来就缺乏北上的信心,所以在和大雍的战争中,南楚历来处于弱势,以至于屈居藩属,这一情形的改变是在德亲王赵珏主军的时候。赵珏对于南楚军事上的不利情况痛心疾首,在他坚持下,南楚终于拥有了自己的骑兵,靠着不到两万人的骑兵,赵珏阻住大雍南下的铁蹄,攻破了蜀中,齐王李显两次攻襄阳,都是这支骑兵配合城内守军出击,才能取得最终的胜利。可是在德亲王薨逝之后,受到重击和奇耻大辱的南楚君臣,不但没有卧薪尝胆,谋求报复,反而绥靖势力抬头,当时接替德亲王主管军务的镇远公陆信,却又是水军出身,对骑兵不甚重视,所以这支骑兵不但得不到扩充,反而渐渐被削弱。若非是德亲王旧部力争,只怕也难以维持襄阳骑兵的编制。

在陆灿承袭大将军位之后,他决定重新发展骑兵,可是在尚维钧等人的阻挠下,江夏骑营刚刚有三千人,就再也得不得朝廷的支持,甚至有朝臣攻讦,指责陆灿耗费军饷,筹措无用靡费的骑兵,甚至有人指责陆灿是借着训练新军有意培养自己的嫡系。当然陆灿尚不能和尚维钧对抗,不得已放弃了筹建骑营的举措。不过陆灿并未放弃,在他取得葭萌关大捷之后,便在蜀中秘密训练骑营,余缅对陆灿十分尊重,惟命是从,陆灿在南楚军方的势力也几乎可以一手遮天,蜀中又多所以这支骑营的存在不仅大雍密谍一无所知,就是南楚朝廷也不清楚。

战马的获得主要有三种来源:德亲王建立的骑营被消减的时候,裁撤下来的军马便被陆灿秘密送到蜀中建立马场;从海上偷运北汉战马,这一条路线并不理想,大雍在东海势力极强,海运十分艰难,战马很难支撑,而且又要千里迢迢运到蜀中,不过蜀中马场的许多优秀的种马都是从这条路线进入的,只不过北汉灭亡之后,这条路线基本上用不到了;除此之外,陆灿甚至曾经派出亲信到吐蕃买马,其中艰险不问可知。在陆灿苦心经营下,终于有了今日的飞骑营九千骑兵。

骑兵的选拔是陆灿借着种种机会,从南楚军中选拔出的勇士,训练的将领有的是蜀中的降将,有的是德亲王的旧部,蜀中的降将倒也罢了,德亲王的旧部是如何到了蜀中的呢,这却是因为襄阳主将容渊的缘故,容渊此人,才略出众,只是心胸不够宽阔,在他接任襄阳将位之后,将一些素来不合的将领排挤出去,当时总督南楚军务的陆信不愿得罪他,便暗中将这些将领安置起来,这其中有不少骑兵将领,到了后来,这些人又被陆灿说服训练骑营。

十年生聚,终于让陆灿掌握了一支精锐的骑兵,且不为人知,而这支骑兵就成了南楚获胜的关键。在淮西之战开始之前,陆灿就已经密令这支骑兵潜行到江陵,蜀道虽然艰难,雍军密谍虽然耳眼通天,可是从蜀中至江夏,陆氏经营多年,再利用江夏骑营的掩护,这支骑兵终于悄无声息地到了江陵。淮西之战白热化之后,这支骑兵又趁着乱局到了寿春,趁着夜色,马蹄包上厚布,人衔枚,马摘铃,悄然到了寿春城下,隐蔽起来等待出击的机会。而雍军疲敝之下,又担心南楚军袭营截杀,所以没有在晚上派出斥候查探军情。就这样,飞骑营给了雍军重重一击,取得淮西大捷。

当然陆灿此刻尚未得到淮西军报,自然不知道自己已经成功,只是他早已下定决心,不论淮西之战如何,都会在今日发起决战,淮西若胜,自是最好,淮西若败,那么自己更是应该尽快在淮东取得一场胜利,夺回扬州,用以遮蔽京口、建业。至于如何接应淮西、淮东两处战场,他已经托付给杨秀,杨秀这次一直在江夏大营掌控大局。

大雾越来越浓,几乎伸手不见五指,陆灿轻叹一声,道:“出击。”

随着陆灿的一声命令,南楚水军向对岸袭去,隆盛七年大雍南征决定最终胜负的一战掀开了序幕。

瓜州,雍军旱寨之内,裴云本已入睡,虽然今夜雾锁寒江,但是多日来对岸南楚军的消极防御,让他也不免有些懈怠,虽然令雍军巡夜军士仔细留心江上动静,可是裴云并没有想到今日南楚军会大举进攻。

所以直到南楚水军到了雍军水寨边缘,才被雍军哨探发觉,一时之间,水寨旱寨金鼓齐鸣,雍军也是训练有素,纷纷出帐迎敌,大雾弥漫,岸上也是一片白茫茫的,只听见南楚军的喊杀声,以及被南楚军用火箭点燃的营寨升起的熊熊火光。

火光驱散了部分雾气,这时,已经披挂上阵的裴云令所有雍军都点燃火把,虽然火把的光亮成了南楚军的最好箭靶,但是在雍军的防范下还是很快稳住了阵脚,瓜州上下,火光通明,江岸上的大雾被驱散了六七成,可是江中依旧迷雾蒙蒙,雍军可以说处于被动挨打的地步,裴云只得下令严守旱寨水寨,令三军以弓箭还击。半夜苦战,到了天明时分,雍军已经击退数次南楚军的抢摊,但是水寨之内一片狼藉,裴云心中怒火熊熊。

天明之后,大雾渐渐散去,已经可以看清楚南楚军的战船了,这一看更是令裴云心中一惊,只见两千多艘舟船摆开水阵,在江中往来如飞,似乎迷雾根本不能阻碍他们的前进。把心一横,难得楚军肯出战,裴云下令大雍水军出寨迎敌,当然因为大雍水军只有千余舟船,两万之众,所以裴云下令己方不能越过江心,最好将南楚水军引到江边来,让岸上的雍军用弓箭相助。

一时之间,江中舟船横冲直撞,两军开始了激烈的水战,这些年来,雍军的水军也在江淮鏖战,精锐程度也是不减南楚水军,可是毕竟南楚水军势大,而且熟悉水文,战局很快就向南楚一方倾斜,裴云见状便下令己方水军暂时退守水寨。果然,在岸上雍军的威胁下,南楚水军并未继续进攻,而是返回南岸去了。

过了午时,吃饱喝足,休息之后恢复了精力的南楚水军再次出击,战势胶结,南楚水军攻不上瓜州,大雍水军也不能渡过江心。裴云站在江边,望着江心处迎风招展的陆灿大纛,心中越发不安。到了未时,水战越来越凶狠,南楚军放出许多小型战船,那些战船船头包着精铁,一撞之下,可以让雍军战船受到重创,这些小战船在南楚军艨艟斗舰的掩护下,如同狼群一般撕咬着雍军的战船,不时看到两军的战船覆没在江中,落水的将士几乎没有被拯救的可能,因为敌军的箭矢会无情的射穿他们的身躯,江水皆被血染,战船的残骸顺着江水东流而去。大雍的水军已经放弃了战胜的可能,只是紧紧地防守着水寨,不让南楚军破寨而入。南楚军在水寨之前有些无可奈何,雍军的步骑虽然不能水战,可是在旱寨里面射箭还是可以的。眼看战局只能这样僵持下去,裴云松了口气,本就没有胜过南楚水军的打算,这样的结果他并不觉得意外,只要南楚水军不能登上瓜州,那么局势就不会发生什么变化。

到了申时末,残阳如血,彤云密布,寒风渐渐凛冽起来,南楚军却是越战越勇,丝毫没有退兵的打算,裴云心中忐忑不安。就在这时,江心楼船之上,陆灿接到了一封军报,合上军报,陆灿眼中露出粲然的光芒,高声道:“诸君,淮西大捷,我军大破雍军,斩首近三万,俘虏雍军四千人,阵斩敌将董山。”楼船上众人听了,都是高声欢呼,声音越来越响,这个消息仿佛长了翅膀一样传遍南楚水军,几乎所有将士都是欢呼着扑向雍军水寨,前仆后继,淮西胜利的激励,让他们不顾生死。他们的欢呼声,让雍军将士心中迷惑,但是也只能顽强地抵御着南楚军的攻击。过了小半个时辰,彤云更加浓厚,夕阳已经难以看到,天地间一片萧索昏暗,南楚军经过一天的苦战,攻势已经渐渐减弱,雍军都是精神一振,知道只要击退这次的进攻,今日之战就该结束了。

岂料就在这时,南楚军中再度传来欢呼,雍军都是大骇,四下环顾,一个雍军军士突然指着西边叫道:“敌人有援军。”凡是听到的人都向西面望去,只见天水交接之处,遮蔽江面的楼船艨艟正向瓜州而来。南楚援军到来的消息如同寒风一般迅速传开,雍军将领极目望去,那些舟船越来越近,几乎是可以看清楚上面站着的南楚军士的面庞,只是船上的旗帜被狂风吹得猎猎飞舞,看不清上面的字迹。可是裴云心中豁然明了,除了江夏大营,南楚哪里还有可能有这样庞大的水军。战,还是不战,裴云眼中闪过坚毅的神色,高声道:“准备迎战!”

在夜幕低垂之际,江夏大营赶到瓜州,向雍军水陆大寨开始了猛攻,生力军的加入,让雍军的命运陷入了不可知的黑暗,此刻,积蓄了一天的力量,飞雪终于飘飘洒洒地落向大江,雪夜寒江,楼船艨艟,战火鲜血,绘制成了最绚丽的图画。

陆灿立在楼船之时,望着节节败退的雍军,终于露出了欣慰的笑容,忍不住望向手中的淮西军报,在文书之后,分明有一封石观的私人书信,上面写着这样的文字。

“少将军身先士卒,奋勇作战,深得淮西军民之心,且与绣儿联手阵斩董山,虽然颇有少年意气,以致险遭不测,然大将军有子如龙虎,乃是南楚之幸,陆氏之幸。”

十一月二十二日,清晨,淮东雍军终于全线溃败,裴云率白衣营亲自断后,南楚军重夺扬州。然而淮东的局势仍然没有更好的变化,骆娄真在淮东的暴虐,让淮东平民对南楚缺乏信任,所以裴云得以退守楚州、泗州,虽然其他府县都被南楚军收复,可是雍军仍然掌握着侵略淮东的前沿重镇。而南楚虽然取得两场大捷,兵力也是损失惨重,所以陆灿只得留大将守扬州、扼广陵,在淮东成了两军对峙之局。而在淮西,虽然南楚军借机收复了钟离,可是崔珏退守宿州,淮西军力不足,无法进一步威胁徐州。

隆盛七年的大雍南征,双方都损失了十万以上的士卒,勉强可以说是打了个平手,南楚惨胜,雍军惨败,淮东重镇楚州、泗州的陷落,是雍军占了上风,可是裴云被牵制在淮东战场,南楚淮西军随时可以进犯大雍控制的宿州、徐州,这里又是南楚占了上风。这一战获得最大利益的便是南楚大将军陆灿,夺回了淮东的军权,淮西、瓜州渡两场大捷,让陆灿的声名如日中天,南楚军方自此只有一个声音,加以时日,不难稳固江淮,到时候大雍南征再无希望,天下即将陷入南北分治的僵局。

\chapter{第十六章 三顾频烦}

隆盛七年十二月,大雍惨败淮南,淮南节度使裴云、靖北将军长孙冀上书谢罪,雍帝叹曰,二卿无罪,皆朕之过也,乃下诏罪己,斋戒祭天,以告英魂。

——《资治通鉴·雍纪三》

“江夏大营十一月四日东下,沿途戒备森严,声言因淮西告急,九江空虚,将至九江防范雍军渡江。”

在寒园之内,明亮的灯光之下,霍琮捧着文卷朗声读着,而江哲正倚在软榻上悠闲自在地把玩着晶莹剔透的墨玉棋子,小顺子则是坐在棋坪对面的椅子上,皱着眉看着面前的棋盘,盘面上白棋一条大龙眼看就要被黑棋合围,这本是很难出现的情况,若论棋艺,小顺子虽然不能称是国手,可是要胜过江哲那是轻而易举的,所以霍琮明明在那里读着兵部转来的军报,仍然是不时偷眼观瞧。

当霍琮读到江夏大营加入瓜州渡口的大战之时,我把玩棋子的动作停了下来,抬起头道:“陆灿果然大有长进,也够胆量,九江空虚不就是他一手造成的么,不与裴云在淮东争锋,而是将九江大营调到京口,造出南楚中部防线不稳的迹象,然后借口九江空虚,又调动江夏大营到九江,似是拆了西墙补东墙,实际上却是迷惑我军耳目,一来不让我军想到会有骑营驰援寿春的可能,二来也令我军忽视了江夏大营会合九江大营,在扬州决战的可能。不过陆灿此计也是极险,淮西战局胜负未分,荆襄又有我军游弋,一旦寿春失守,或者长孙将军绕过荆襄,直入荆南,那么南楚军都将陷入万劫不复之地。不过想必陆灿已经心知肚明,这一次我军的主攻方向不是襄阳,长孙将军又是稳扎稳打之人,不会冒险突进,只有淮西之战,陆灿的确是冒了险的,不过此举已经有名将之风,淮西之战若有三成胜算,这么做就是值得的。嗯,琮儿,念念淮西的军报,我要看看那里陆灿是如何安排的?”

霍琮寻出淮西的军报,按照次序详细念了一遍,当他念到陆云和石观之子石玉锦阵斩董山的时候,我的手一抖,但是面上神情没有丝毫变化,反而笑道:“好啊,陆灿做的不错,雉鹰若不赶出巢去,也不能振翅高飞,陆灿将亲子放在险地,怪不得淮西军如此顽强,否则崔珏、董山也是难得的猛将,也不会在寿春被阻。其实也是皇上轻敌,若是派上一员谨慎小心的大将,再多派几万人马,严防敌军增援,断不会使大军因为久战疲敝,落得一个兵败如山倒。其实这也难怪,陆灿这支骑营如此隐秘,司闻曹全无所知,恐怕就是南楚朝廷也是不知道的,既不知寿春将有援军,也难怪崔珏、董山二人懈怠。不过董山被两个不到十五岁的少年联手击杀,倒也是颇为让人意外,我记得他是一员猛将。”

霍琮道:“根据司闻曹事后的调查,董将军断后苦战,那时应该已经是强弩之末,而陆少将军和石少将军都是难得的少年勇士,所以才能取得这样的战绩,听说当时的战况十分危险,两位少将军也是险些丧命。”

我轻轻一叹道:“经此一战,淮西军民士气高涨,陆云虽然年少,却已经成为南楚军方不可忽视的力量,陆灿定会趁机在淮西扩军备战,加强对淮西的控制。等到淮西军力强大之后,就可以向东北攻宿州、徐州,或者向西北攻取豪州、睢阳,想来数年之内,陆灿都会从淮西屡屡出兵,攻略淮北,训练士卒。”

霍琮疑惑地道:“先生,虽然陆灿已经掌握江南军权,可是大雍拥甲百万,这次战败并未伤筋动骨,陆灿理应休养生息,防备大雍南征才是,怎会主动挑起战事呢?”

我轻笑道:“陆灿虽然掌握了江淮兵权,可是心却还不够狠,禁军仍有大部分掌握在尚维钧手中,建业仍然是尚氏的天下,陆灿手中的兵权越重,就越会有些自诩忠臣的文官担忧他仗恃兵权谋反,所以尚维钧的支持者反而会越来越多。等着吧,等到论功行赏之后,就会有人想尽办法消弱陆灿的权力。所以他若想自保,只能主动出兵,边境战乱不休,才能保全他的身家性命。”

霍琮眼中寒光一闪,道:“功高莫赏,本就是不赦之罪,陆大将军会不会索性自立为王,到时候江南便是铁板一块,再无可乘之机。”

我扬声笑道:“琮儿,你以为兵变是那么容易的事情么,不错,陆灿手掌重兵,一旦兵发建业,就可以犁庭扫穴,控制南楚朝廷,甚至自立为王。可是有些事情却不是只靠军队就可以实现的,一旦陆灿起兵反叛,那些因为陆家忠义声望而为之效命的将士就会失望,甚至还会有人起兵勤王,别忘了襄阳容渊、淮西石观、葭萌关余缅虽然都尊陆灿为首,而且他们和陆氏也多有牵绊,可是他们更是南楚的忠臣,若是让他们随陆灿反叛,恐怕还不能够。而且尚维钧掌控朝局多年,与南楚各大世家之间有着盘根错节的关系,现在南楚朝廷的官员,十之六七都是尚氏一党,若是陆灿清了君侧,这些官员怎么办,都杀了,南楚朝堂一空,政局立刻陷入混乱,若是不杀,这些人难道会真心尊奉陆氏为王么?陆氏的力量主要集中在军方,根本没有办法控制整个南楚的朝廷,恐怕到时候朝政会被趁虚而入的世家势力掌控,到了那时,各大世家为了争权夺利,必然彼此攻讦,只怕南楚的局势会更加糜烂。所以陆灿不能用兵变的方式解决即将面对的压力,唯一的办法自然是挑起外患,只要江淮战事还在进行,尚维钧等人就不敢随便加害陆灿和他手下的将士。而且大雍南征之心是不会消除的,与其坐着等大雍来攻,还不如主动出击,还可以利用这些小规模的战斗磨砺士气,训练士卒,让南楚的边境稳如泰山,这样一举两得的事情,陆灿何乐而不为呢?”

霍琮听得入神,良久才道:“先生,尚维钧畏惧陆灿军权,必然不敢轻举妄动,而陆灿与其去争夺朝中的权力,倒不如掌控大军在外一呼百应的好,只是这样一来,江南局势稳定,大雍就不可能顺利的平灭南楚,天下难以一统,岂不是兵燹永难休止。”

我瞥了他一眼,道:“陆灿这个人忠义之心极重,他之所以争夺军权不过是因为不愿见到大雍铁骑南下罢了,对他来说,他主军,尚维钧主政,那是最好不过。当然日后他位高权重,会不会有不臣的心思尚未可知,可是在我看来,这个人没有谋反的可能。陆氏世代将门,忠义之心已经根深蒂固,陆灿也不例外,虽然他的手段厉害一些,行事少些忌惮,可是他没有自立之心。只是他虽然用心是好的,尚维钧却是不会认同,现在不过是暂时的妥协,这种军政分离的情况终究不能持久,除非是南楚国主有足够的威望收回军政大权,或者尚维钧甘心雌服,只是这两点都不现实。南北对峙,终究不能长久,此消彼长,必有一方灰飞烟灭,两国相争如此,两个权臣相争也是如此。纵然陆灿委曲求全,或者用些雷霆手段压制这个隐患,可是一旦爆发出来,就是惊天惨变。只不过南楚君臣若不是太愚蠢的话,维持几年平衡局面应该还没有问题。不过,琮儿,你问这些事做什么,莫非也想和陆灿较量一番,看看谁才是我门下第一人?”

霍琮脸上露出尴尬的神色道:“弟子怎会有此意,是嘉郡王托我试探先生的口风,想知道先生是否已经有了平楚之策,或许是奉了齐王殿下的命令吧。”

我冷冷一笑,道:“多管闲事,李麟既然是郡王之尊,费些心思也就罢了,你不过一个白衣,何必这么多事,你只要读好你的书就行了,对了,明日你将兵部送来的军报整理之后交回去,就说江某乃是闲散之人,对于这些军报不感兴趣。以后若再有这样的文书送来,就说我正在养病,无暇理会身外之事,不许你再擅自接下这些军报。”

霍琮心中嘀咕,你方才不是听得很认真么,还振振有词地分析局势,如今怎么又改口了,口中却连忙道:“都是弟子擅自作主,请先生恕罪。”说罢恭恭敬敬地退了出去。

看着霍琮的背影,我的嘴角露出一丝冷笑么,哼,什么齐王的意思,嘉郡王多半是奉了太子之命,太子多半是奉了皇上之命,不过是想试探一下我的心意。看来这次攻楚的惨败,让大雍君臣头脑清醒了许多,自然想到了我当日的上书,看来皇上已经明白非是我眷恋故国,而是他们轻敌了。如今局势变化至此,这些人定是都想听听我的判断。可是我江哲岂是召之即来,挥之即去的人物,既然他们曾经怀疑过我,我便索性不介入雍楚之战,这本就是我的希望,反正他们君臣都是身经百战的名将,步步为营,这种情况下,有个二、三十年的努力,攻下南楚应该没有问题吧?毕竟南楚内部还是隐忧重重的,陆灿若是没有进取之心,我料他四五年之内就会遭遇剧变,南楚现在的国主赵陇,应该还有几年就要加冠了,到时候理应亲政,那可是尚维钧夺回军权的最好的机会啊。不过陆灿这些日子的手段带着阴狠,不似他的风格,一个人行事的作风是很难改变的,多半是韦膺的谋划,这两人合作如鱼得水,对于南征十分不利。罢了,我怎么又在盘算平楚之事,不是想好了置身事外的么?

侧过脸看着小顺子还在冥思苦想,我偷偷笑了,日前得到一本国手的棋谱,上面有几个玲珑棋局,特意摆了一个,总算是把他难住了,也让我扳回一些面子,想起从前被他杀得冷汗直流的惨状,我得意地望向小顺子,希望看到他认败服输的场面。岂料正在我得意洋洋的时候,小顺子眉头突然舒展,放下了一粒白色的水晶棋子,顿时盘面局势扭转,原本陷入困境的白棋奇兵突出,反败为胜,和黑棋对峙起来。我叹了一口气,知道又没有难住小顺子,随手从玉枕之下取出那本棋谱,扔给他之后,有些赌气地推开棋盘,仰面躺在软榻之上,身下是温暖柔软的被褥,空气中带着淡淡的芬芳香气,我有了一丝倦意。为了不想长乐替皇上说话,所以这些日子我准备留宿寒园了。

小顺子微微一笑,将棋谱打开翻了一遍,收到怀中,然后一边收拾棋子,一边道:“公子,你和皇上斗气好么?毕竟他是君,公子是臣。”

良久,江哲始终不语,就在小顺子收拾好棋子,以为江哲不会回答的时候,江哲淡淡道:“遇事要防微杜渐,这次皇上可以对我不信任,那么将来呢?我不能留下隐患。而且我若是表现的大度宽容,凭着皇上的才智,怎会看不出我已经对他生出疑虑,只有我凭着本性和他为难,他才会相信我并没有因此事改变对他的观感。”

小顺子默然,他没有继续问下去,例如江哲心中是否对皇帝真的生出不满?是否江哲真的依旧留恋南楚,所以才不愿献策平楚?一旦江哲作出决定,不论是多么不合情理,他都不会反对。将棋坪收好之后,他往香炉中加了一些安息香,然后拿了毯子盖在已经昏昏入睡的江哲身上。做完这一切,他便坐在一旁的蒲团上打坐调息,对于他来说,睡眠已经是一件不很重要的事情了。

过了片刻,他突然轻轻皱眉,看了一眼仍在沉睡的江哲,他转身推开房门,走了出去,一眼便看见一行人正向这里走来,其中一人披了大氅,遮住了面容,可是隐约露出的明黄色袍服以及他身边的侍卫仍然令小顺子一眼便认出他的身份。那些人走到近前,那遮住面容的中年人道:“随云可已入睡了么?”

小顺子低首敛眉地道:“公子已经入睡了,近日公子很难入眠,所以点了一支安息香,只怕公子明晨之前是不会醒过来的,而且公子近日身体不适,恐怕不能接驾。”

那人微微苦笑,抬起头,兜帽滑落,露出年华已去,却依然气度雍容的面容,事先令兵部送来文书,又通过霍琮试探,原本就是为了表示他的致歉之意,可是如今看来江哲并不领情,这个人,还是当年的性子,至今没有改变,想到此处,李贽更是为自己前些日子对江哲的疑心觉得歉疚。看看挡在自己面前的李顺,虽然姿态是那样的谦卑,可是李贽却知道,那种顺服只是外表的伪装罢了,他相信自己若是要强行进入,邪影李顺可不会顾忌自己的身份,一旦事情到了那种地步,可就没有挽回的余地了。无奈之下,李贽只得转身离去,盘算着这次如何说服江哲,应该不会比当初说服他投效自己更困难吧?

接下来在大雍君臣忙着为战败善后的时候,一向深居简出的楚郡侯成为大雍朝臣瞩目的对象。一个流言在雍都百官中悄悄流传,皇上几次亲临长乐公主府,居然被江哲拒于寒园之外,除了当初见识过江哲刚烈一面的石彧等人之外,其余的朝臣是不敢相信这件事情的,事实上,这也不过是捕风捉影的臆测罢了。这种丢脸的事情,皇上不会说,他身边的侍卫内侍不敢说,就是长乐公主府里上上下下,倒有大半是皇上皇后精心安排的,所以这件事情原本无人外传。可是再隐秘的事情也是有迹可寻的,皇上几次三番造访长乐公主府邸,却总是败兴而回,种种蛛丝马迹通过宫人口耳相传,真相就被勾画出来。又被有心人传播出去,街谈巷议中都有涉及。只是这件事情,就是最刚直的谏官也是缄口不言,不说楚郡侯暗中的势力有多大,只凭皇上对其的信宠,也知道此人若是不能一击致其于死地,最好不要得罪。而且这件事情若是无人知道,皇上还可留些颜面,若是流传出去,只怕反而会让皇上恼羞成怒,到时候挑起事端的官员可就麻烦了。

这个流言尚未平息,又过了一些日子,又有新的流言传开,有人说楚郡侯江哲之所以不肯出谋划策,不肯见驾,是因为留恋故国,而且现在南楚赫赫有名的大将军就是他的亲传弟子,江哲与南楚陆氏至今藕断丝连,多有往来。这个流言说得有根有据,很多官员百姓都相信了,就是朝中重臣也不免信了几分。

听到这个流言,李贽恼怒非常,到了如今,他自然不会仍然怀疑江哲会为了南楚撇开大雍,可是他也知道江哲的性子最是执拗,现在本来就在和自己斗气,若是再给他知道这个流言,说不定一怒之下反而真的会缄口不言,那岂不是糟糕至极。所以他下令明鉴司追查流言的来源,又下了严令,不许人将消息传到江哲耳中。只是流言蜚语满长安,想要追查却没有源头可寻,李贽不免龙颜震怒,雍都的气氛变得异常紧张。

过了几日,李显轻身简从的到了寒园,他是奉了李贽的旨意前来求和。这一次南征李显并未上书请命出征,一来是没有将南楚江淮军力看在眼里,在他看来,这次攻略江淮无需他坐镇,等到江淮平定,需要渡江作战之时他再请命不迟,二来也是因为林碧临盆在即,他也有些舍不下娇妻爱子,所以李贽无意让他南征,他也便没有主动提起,只是在制定南征计划的时候在旁边参赞罢了。当初江哲上书反对这次的南征,他也和李贽一样,以为江哲不免有故国之情,所以两兄弟合作默契的将此事隐瞒了起来,免得有人趁机攻讦江哲。不料南征惨败,江哲所言字字珠玑,李贽和李显都是从战场上面杀出来的大将,自然不是寻常人物,很快就意识到了他们轻敌的错误。十年的休养生息,恢复国力的不仅仅是大雍,南楚也不再是从前的疲敝景象。可是虽然意识到了这一点,局势的变化已经不可挽回,陆灿掌控了江南军权,这样一来,江南半壁江山难以颠覆,陆灿在他们眼中成了大雍南征的最大障碍,想要平楚,必须除去陆灿,想要除去陆灿,那么有一个人的意见最为重要,这人就是江哲。不论陆灿如何出色,不能否认此人的本事多半和江哲有关,既然如此,除了江哲之外,谁还能够制定出平楚之策呢?李贽和李显都不希望两国对峙几十年的时间。

既然江哲不卖李贽的面子,那么李显也就责无旁贷的前来相劝了,不过虽然是有求于人,李显的性子还是那般嚣张,一路横冲直撞,长乐公主府上的侍卫都不敢阻拦,虽然主人说过不见客,可是李显一路直闯寒园,却是没有一人敢阻拦。李显刚走到书房门口,就听到江哲暴跳如雷的痛骂声。李显心中好奇,这么多年相交,好像没有见过江哲这样骂人,不由停住脚步,侧耳听去。

我看着跪在那里老老实实的慎儿,心中怒火汹汹,这个臭小子,明明在那里罚跪,可是你看他眼珠转个不停,就知道他分明是在胡思乱想,哪里有半分悔过的意思?忍不住又骂道:“整天只知道练武贪玩,我亲自教你读书,你居然给我偷溜,一本论语念了半年居然还背不下来,听着,今天罚你将论语抄上三遍,若是交不上来,就别想吃晚饭。”

慎儿今年已经八岁了,生得眉清目秀,聪明可爱,偏偏是一副笨肚肠,让他读书比什么都困难,也不知道是像谁,我在他这个年纪早就熟读经史了,他的娘亲也是聪明之人,怎么就他这样蠢笨,可是那慈真老和尚居然说他是武学奇才,真是没有天理了?

我刚说完惩罚方式,慎儿一下子跳起来道:“爹爹,那我就去抄书了,不过爹爹,我背不下来论语不关我的事情,都是爹爹你教的不好,一篇文字,爹爹偏要东拉西扯,扯上一大堆有的没的,姐姐也说了,若是想要读书,跟着霍哥哥要好的多。”

我听到这番话气得差点晕过去,拿起戒尺就要打他的手心,不料江慎转身向外逃去,敏捷非常,如同一缕轻烟一般转眼消逝在门口,我大吼一声道:“小顺子,给我把他抓回来,我要把他的手心打烂。”话音未落,就听到慎儿一声欢呼道:“岳父大人。”

我心中一凛,立刻改口道:“慎儿,慢点跑,别摔着。”绝对的慈父口吻,原本在旁边站着的小顺子露出有趣的笑容,当然笑容在我暴走之前已经消逝。

然后我便看见李显拉着慎儿走了进来,面色极为不善,我忍气吞声地上前施礼道:“原来是六哥来了,让你见笑了,慎儿太顽皮了。”唉,自从李显回到长安之后,就几乎霸占了慎儿,每次慎儿从浮云寺回来,还没有在家待上两三天,就会被他接走,我若想不答应,就要面对他的冷森面容,也就是他追求林碧那几年好一些。等到李凝出生之后,齐王可就是变本加厉,先拐了慎儿叫他岳父,然后堂而皇之的领了去。倒是我这个父亲,难以管教自己的儿子。不过,我摸摸鼻子,如果不是我从小就喜欢欺负慎儿,这小子也不会这么快就见异思迁吧?

李显犹豫了一下,他将慎儿当成亲生儿子一般看待,一听说江哲要打慎儿手心,心中便不高兴,可是他此来是为了替皇兄求和来了,总不好给江哲脸色看吧,犹豫再三,终于道:“随云,我看你还是给慎儿请个启蒙的先生吧,要是不愿意,就让霍琮教他也行,听柔蓝说,你一讲书就喜欢引经据典,也难怪慎儿听不懂。”

慎儿聪明得很,听出岳父的口气有些软弱,立刻变得老老实实,眼巴巴地看着我,道:“爹爹,是慎儿太笨了,都听不懂你讲书,不像霍哥哥,闻一知十,你还是让别人教我吧。”

我见状不由心中苦笑,这个孩子到底像谁呢?

这时,齐王又道:“其实,慎儿将来也用不着十年寒窗,将来作个将军不好么,我看这小子武功根基扎实得很,胆子又大,有几分像我。”说罢有些得意地抚摸着慎儿的脑袋。慎儿也是一脸得意洋洋的模样,倒好像李显才是他的爹爹一样。心中生出一种酸溜溜的感觉,我语气不善地道:“小顺子,送慎儿到他的书房抄书,论语抄一百遍,你看着他,如果他敢偷溜回浮云寺,你就把他抓回来,替我打他的板子。”

慎儿一听犹如五雷轰顶,立刻呆住不动,直到小顺子上前一把将他拎起,走向门外的时候,他才大叫道:“顺叔叔饶命,脖子很痛啊,岳父救命啊,娘亲救命,霍哥哥救命,姐姐救命。”片刻,惊天动地的呼救声渐渐远去。我不由汗颜,这个小子,真是丢尽了我的脸面,狠狠的瞪了李显一眼,都是他宠坏了慎儿,所以今天不论他来干什么,我都不会让他如愿。

李显何等聪明,一见便知自己还是捅翻了马蜂窝,这江哲分明是准备公报私仇了,不由露出一丝苦笑,这次前来的目的是绝对没有可能实现了。

九重宫阙,干百楼台,金殿辇路,玉砌雕栏,御书房之内,李贽愁锁双眉,看着一书案的密折奏章,却是无法静下心来披阅,宋晚轻手轻脚的走了进来,禀报道:“皇上,齐王殿下在外面候旨。”

李贽连忙道:“还候什么旨,他什么时候这么守规矩了,快宣。”宋晚走了出去,不多时领了李显走进书房,然后不需吩咐,便带着书房内伺候的宫女内侍退了出去,留给两兄弟密谈的空间。

这些人的身影一消失,李显立刻故态复萌,随手扯了一张椅子坐在下首,抱怨道:“皇兄,这件事情我可办砸了,随云根本不听我劝解。”

李贽丝毫不以李显的嚣张行径为忤,笑道:“你临去的时候不是拍着胸膛说定可以成功的么?”

李显赧然道:“这个,实在是不凑巧。”说罢李显将今日的情形说了一遍,李贽听了连连苦笑,李显正色道:“皇兄,看来随云不过是一时意气,等过些时日定会回心转意的,你也不用着急,现在随云和我们在一条船上,他是不会看着我们翻船的。”

李贽苦笑道:“时间不等人啊,若是再过几个月,只怕江淮防线固若金汤,我们就更加没有机会了,若是在拟定平楚之策的时候,没有随云的意见,我实在不放心,现在的南楚不是从前的南楚,我不想这一仗打下来,打得两败俱伤,民生凋敝,所以必须说服随云参与这一战,事实上,我准备年后就建立江南行辕,由你亲自坐镇,总督荆襄、江淮的战事,随云我也有意让他随军参赞,所以需要快些说服他,随云的性子,也真是太执拗。”

李显听到江南行辕之事,只是眉梢微扬,却没有作声,但是听到最后一句,却笑道:“随云乃是国士,皇上以国士待之,才能让他甘心效命,天下除了皇兄之外,还有谁能驾驭他,我想他不过是一时气恼罢了,其实我看他气已经消了,只是没有台阶下罢了,若不是我今日去的不巧,说不定他现在已经跟我进宫了。”

李贽也是微微一笑,他在长乐公主府上耳目甚多,自然知道这几日江哲的心情已经恢复如初,要不然也不会让李显前去劝解,只是如今李显被顶了回来,应该让谁去劝解呢,盘算了半天,满朝重臣,居然没有几个可以和江哲说上话的,这些年来,江哲在雍都竟是大隐于朝,并无知交,就是和昔日雍王府的属官也都鲜有往来。更何况这种事情也不能让太多人知道,李贽不想给人留下江哲恃宠而骄的印象。一时之间,兄弟两人坐困愁城,竟是没有了主意。

这时候,宋晚再次进来禀报道:“启禀皇上,夏侯沅峰大人求见。”

李贽没有言语,只是一摆手,宋晚退了下去。李显知道夏侯沅峰乃是李贽的心腹,担负着监察百官的重责,不免有些隐秘的事情,自己还是不知道为好,便起身要告辞。

李贽笑道:“不妨事,六弟不用回避,是我让夏侯查一下最近是谁在散播流言,想要离间我们君臣至亲,想来他是有了结果了,你听听也无妨。”

不多时,夏侯沅峰走了进来,虽然已经是三旬出头,又在官场历练多年,添了几许风霜之色,不似当年俊雅无双模样,但是岁月仿佛没有在他身上留下多少痕迹,夏侯沅峰仍然是风度翩翩,俊逸优雅,不负美男子之誉。

进到书房之内,夏侯沅峰上前施礼道:“启禀皇上,臣仔细盘查之下,散布流言者恐怕和南楚有些关联。”

李贽倒也不惊奇,如今南北对峙,若说有人想要离间自己和江哲,自然是南楚之人其心最切,他淡淡道:“这件事情不便宣扬,你将名单呈上,日后对他们仔细监视,一旦有异动便控制起来。”

夏侯沅峰将写好的折子呈上,就要转身离去,无意中望见李显烦恼的面容,心中一动,道:“皇上和齐王殿下可是为了楚郡侯之事烦恼?”

李贽闻言苦笑道:“夏侯,你可有什么法子解决此事?”他不过是随便问问,夏侯沅峰和江哲一直有些宿怨,李贽根本不会相信夏侯沅峰能够有什么办法说服江哲献策。不料夏侯沅峰上前恭恭敬敬地道:“臣子之责便是要为君父分忧,臣愿前往说服楚郡侯。”

李贽一惊,上下打量了夏侯沅峰片刻,才道:“你去试试也好。”夏侯沅峰含笑而退,似乎劝服江哲是件极为容易的事情,这令李贽和李显也生出了期望之心。

飞雪连天,彤云密布,坐在临波亭之内,我静坐抚琴,琴声拟出飞雪凌空之态,浑然一体。良久,我推开玉琴,轻轻叹息,树欲静而风不止,这些日子长安的暗流汹涌怎能瞒过我的耳目,虽然皇上有意维护,可是我又怎会不知这些攻讦我的流言的存在。抚摸着琴身的断纹,我便想起秋玉飞,自从北汉亡后,魔宗隐退,不过段凌霄等人自然不能随便抽身,段凌霄就在大内隐居,萧桐随在林碧之侧,其他魔宗弟子或者从军,或者留在大内做了侍卫,虽然魔宗弟子比较桀骜不逊,可是他们的能力手段出众,现在魔宗已经隐隐成了可以和少林等门派相抗衡的力量。这其中也只有秋玉飞置身事外,带着凌端隐居在我送给他的静海山庄。可以常年领略东海风光,或者一叶扁舟,凌波独海,或者月下抚琴,逍遥自在,只恨我却被红尘羁绊,不能离开雍都一步。接过小顺子递过来的温酒,我一饮而尽,绵软香甜的琼浆让我生出沉醉之感。

一个侍卫踏雪而来,小顺子走出亭去听他禀报了什么,转身回来道:“公子,夏侯沅峰求见。”

我微微一愣,怎么夏侯沅峰会来我这里,自从东川之事后,这人总是躲得远远的,倒好像我是鬼怪一般,心中好奇,我笑道:“请夏侯大人到这里来。”

不多时,夏侯沅峰随着侍卫迤逦而来,雪色轻裘,临风玉树,明朗如月,这人若是看外表绝对想不到竟是血染双手的明鉴司主事。

我站起身来,在亭中相迎,亭外飘雪如织,我自然不会去领教其中的寒气袭人,伸手肃客,请夏侯沅峰入座,我笑道:“不知道夏侯大人怎么有空前来造访,大雪漫天,有佳客登门,不可无酒,小顺子,取一坛御酒来,这壶‘凝春’太香艳,夏侯大人是不会喜欢的。”

夏侯沅峰笑道:“侯爷不必费心,久闻长公主殿下采百花之精酿造的‘凝春’酒,香醇绵软,饮之如琼浆玉露,下官早有意品尝其中滋味,只是不得门而入,今日有幸亲见,岂能错过美酒。”

我眼中闪过一丝光芒,道:“这‘凝春’酒乃是长乐亲酿,其中除了百花之精,还加入了许多滋养身体的药物,常年饮用可以令人耳聪目明,身轻体健,只是过于绵软香醇,不大适合雍人口味,想不到夏侯大人却能领会之中妙处。”

夏侯沅峰恭敬地道:“长公主殿下深情感天,为了侯爷康泰,才酿制此酒,那些外人怎知长公主之心,如何能够领略此酒深意,况且那些凡夫俗子也没有资格品尝这绝世美酒。”

我听到此处已经知道夏侯沅峰的来意,用长乐的深情提醒我不要忘却自己和大雍皇室不可斩断的牵绊,只是他够聪明,利用这寄托长乐情衷的“凝春”酒抒发心意,倒是让我生不出恼意。

示意他坐下,我亲手斟了一杯酒给他,道:“既然夏侯大人深知此酒的珍贵,就请喝上一杯,此酒每年只酿造二十四坛,除了送给太后娘娘、皇后娘娘、齐王妃殿下的几坛之外,再无流传。”

夏侯沅峰坦然落座,道:“拙荆蒙皇后恩典,赏赐了一壶‘凝春’,才有幸尝到这难得的佳酿,今日若是侯爷大度的话,不如让下官多饮几杯。”

我已猜知夏侯沅峰的来意,不过却也好奇他凭什么相信自己能够说服我,所以故意不问他的来意,反而殷勤劝酒,对着亭外茫茫飞雪引经据典,和夏侯沅峰讨论起诗词歌赋来,认识夏侯沅峰这么多年,只知道他心思细密,腹有权谋,武功过人,心狠手辣,可是今日一谈论,才发觉此人果然是文武双全,初时本是有意为难考较,谈论了许多时候,反而觉得和此人聊天十分愉快,不由渐渐淡忘了他的来意。

夏侯沅峰见气氛融融,心中暗喜,把酒道:“下官听说侯爷在北汉时曾经和诗一首,以抒心臆,其中有两句‘生不冀求兮南归雁,死当葬我兮楚江畔’之句,不知可是实情?”

我心中一动,知道他终于开始进攻了,他所提及的诗句,是我感于谭忌绝命词的悲恸,所和之诗,这件事情夏侯沅峰也知道,倒也不甚意外,他是明鉴司主事,当时我身边的侍卫都是虎贲卫高手,必然有人将这诗词送到御前,夏侯沅峰得到李贽宠信,这又不是什么隐秘,他知道也是可能的。不过他是要指我留恋故国么?嘴角露出淡淡的嘲讽微笑,我淡淡道:“故国之思,人之常情,夏侯大人敢是要上折子参我么?”

夏侯沅峰却又转移了话题,道:“这一次裴将军在淮东轻身涉险,计取楚州,虽然立下大功,可是未免太冒险了。”

我随口道:“裴将军性情如此,喜欢亲自上阵厮杀,不过若是到了紧要关头,他不会这么做的。”

夏侯沅峰笑道:“不过裴将军倒是胆子极大,镇淮楼公然折辱侯爷至亲,想来也令下官佩服。”

我心中一凛,目光低垂道:“荆长卿虽然是江某表兄,但是他是南楚忠臣,道不同不相为谋,裴将军此举并无不当之处。”

夏侯沅峰摇头道:“裴将军不过是没有留意罢了,若是他早知道那人身份,必然不会轻辱,不过侯爷对贵亲十分关爱,没过几日,荆长卿就从楚州大牢里面失踪了,听说已经回到了嘉兴,虽然这一战南楚胜了,令表兄不免有弃职私逃之嫌,不过想来没有人会为难荆氏,毕竟陆大将军如今权倾朝野,看在他的面子上,也不会有人对荆氏下手。”

我露出一丝冷笑,当初在东川,夏侯沅峰就想夺取锦绣盟的权力,虽然我让他如了愿,可是也给了他一个教训,如今他又想插手我在南楚的势力么?莫非他此来不是为了说服我和李贽和解?

站起身来,走到琴台之侧,轻抚琴弦,琴声铮铮,透出肃杀之意,我淡淡道:“夏侯大人还有什么要说的么?”

夏侯沅峰毫不理会我逐客之意,又饮了一杯酒,道:“陆灿长子陆云,少年英杰,阵斩董山,名扬淮西,此子据明鉴司所察,曾经在长安滞留多日。”

我眼中闪过嘲讽的神色,陆云之事我早知道难以瞒过明鉴司耳目,否则我何必将霍琮、李麟和柔蓝都牵扯进来,除了历练这几个孩子之外,就是让某些人投鼠忌器。但是转瞬,我眼中的神色变得悲伤,原本希望能够保住陆云,可惜他在淮西的所作所为,已经让我的努力成了泡影,谁会想到一个十三岁的少年可以有这样的本领成就呢?

夏侯沅峰或许察觉到了我心意的变化,又道:“侯爷出身南楚,对故国多有眷恋,更有亲友弟子在彼,战事一起,难免玉石俱焚,侯爷或有周全之意,然而若是侯爷置身事外,将来以何功勋为荆氏、陆氏缓颊,下官曾听说,侯爷曾承诺德亲王保全南楚一脉香烟,若是侯爷不肯献策平楚,将来拿什么向皇上陈词。猎宫之变,侯爷昔日有功于大雍皇室,然而皇室以长公主下嫁,可谓无亏侯爷,侯爷有平汉之功,然而侯爷如今身为郡侯,子为国公,女为郡主,一家荣宠备至,平汉之功已经得偿。难道等到了南楚覆亡之际,侯爷要以这些旧日功绩换取皇上的恩典么,到时候就是皇上不说什么,侯爷能够无愧于心么?而且若没有侯爷亲赴江南主持平楚之策,只怕侯爷的一番苦心都将成为泡影,下官放肆,但是句句都是肺腑之言,请侯爷明鉴。”

我眼中闪过莫名的神色,看向夏侯沅峰,这个人当真不简单,这一番话已经彻底将我说服,而且这番话也只有他能说,李贽、李显若是这样说了,反而会让我觉得他们有意要挟,若是石彧那些大臣说来,不免会变得冠冕堂皇,反而像是以大义相责,更令我生出逆反之心,只有夏侯沅峰这个心中只有功利之人说来,我才觉得情真意切。

夏侯沅峰微微一笑,又道:“还有一事,侯爷不知可否知晓,韦膺如今就在陆灿幕中担任客卿,此事虽然隐秘,可是也瞒不过司闻曹、明鉴司的耳目。”

我神色淡淡,这件事情我早已知道,在夏侯沅峰这个聪明人面前,我也懒得惺惺作态。

夏侯沅峰心知肚明,道:“韦膺对侯爷十分憎恨,他才智过人,手段阴狠,陆大将军又是军略出众,与侯爷又是少年相交,师徒投契,对侯爷十分了解,这两人联手,必是侯爷强敌,侯爷才智过人,遇到这样的对手,难道不想和他们较量一下么?陆灿掌握南楚军权,侯爷也可左右大雍平楚策略,不若在江南之地对弈一番,看看是侯爷才智无双,还是陆将军青出于蓝,这岂不是一大快事?”

听到此处,纵然是我也不免生出好胜之心,忍不住笑道:“夏侯大人的口舌之利,不亚于苏秦张仪,哲今日领教了。”

夏侯沅峰面色不变地道:“侯爷谬赞,下官愧不敢当,只是昔日对侯爷多有得罪,所以今日前来相劝,也是希望侯爷功成之日,能够记得下官的一番好意,不要仍然记恨下官才是。”

我终于忍不住大笑出声,道:“好,好,夏侯沅峰果然不愧是夏侯沅峰,想来你也急着回去复命,我就不留你了,禀报陛下一声,明天上午我会入宫觐见。”

夏侯沅峰笑道:“侯爷之意,下官一定禀明,不过不知可否送下官一壶‘凝春’呢,此酒下官实在喜爱得很。”

我向夏侯沅峰面上瞧去,怎也看不出他有半分虚情假意,这“凝春”酒香甜绵醇,但是并不合北方男儿的口味,所以此酒除了送给太后、皇后和齐王妃之外,长乐是不送给别人的,就是齐王妃林碧,我猜她也多半更喜欢北地的烈酒。忍不住轻轻摇头,我淡淡道:“小顺子,让人送一坛‘凝春’到夏侯大人府上。”

夏侯沅峰含笑致谢,然后告辞离去,望着茫茫飞雪中他俊逸的风姿,我心中生出敬佩之心,要留心啊,这个人从前我可以将他玩弄于股掌之上,多半是他甘心如此,若不小心提防,只怕将来吃亏的会是我吧。

\chapter{第十七章 平楚策}

同泰十一年甲申,雍军犯境,大将军陆总督江南军务,雍军惨败而归。

雍帝亲谒寒园问策,哲感帝诚,献平楚策,君臣促膝倾谈,终夜不寐,后人有言,南楚覆亡,皆始于此。哲于雍功高盖世,于楚则罪莫大焉。

——《南朝楚史·江随云传》

送走了夏侯沅峰,我坐在亭中继续赏雪,既然已经被他说服,决定向皇上献策平楚,我也该想想如何陈述所思所想,这些日子原本有许多想法,只是隐忍不言罢了,可是不知怎么,我一用心思索,却觉得心烦意乱。虽然心中早有了计策,可是这些计策本是纸上谈兵,一想到真要付诸实施,将会带来的血腥和兵燹,竟觉得心中悲恸难当。

仔细想来,我竟然真如那些流言所说,依旧留恋故国,想来李贽当日也并非冤屈了我,我若不是仍然心存故国,就应该向皇上据理力争,说明大雍不利的局势何在,并且提出解决的计策,而不是凭空说大雍将败。大雍战败之后,我因为李贽曾经疑我,而不肯和他和解,向他献策,并非是我一向的执拗脾气犯了,我竟是不愿让自己的献策覆灭故国。

我这算什么,鼠尾两端么,早已背国离乡,如今又何必假惺惺的留恋故国呢?大雍待我仁至义尽,我却想眼看着大雍将士在南楚失去性命荣耀,越想越是难过,忍不住连连饮了几杯。

“凝春”酒虽然香醇,但是后劲却是十足,我已经先后饮了十几杯,不免酒意上头,有些头昏目眩,原本刻意忽略的情绪涌上心头,越发觉得心中苦痛,忍不住走到槛外,雪花扑面而来,散入轻裘透锦衣,寒气袭人,素来畏惧寒冷的我却怔怔站在雪中,想到阔别南楚多年,如今终于有机会回到故国,却是要与之为敌,不由泪水滑落,立刻被寒风化成薄冰贴在面颊上,我却丝毫不觉寒冷。

小顺子原本在一旁看着江哲饮酒,此时看出不对,连忙上前半扶半抱,将江哲搀回临波亭,拿了一件大氅将他裹住,却见江哲神智昏昏,不由低声叹息道:“公子,你这是何苦呢?你若决定献策平楚,就要下定决心,不可再眷恋故国,你若决意不再献策,又何必为了那些无情无义之人多费心思!”

我已陷入醉意朦胧之中,倚在亭中舒适的躺椅上,对小顺子的话充耳不闻,只是清泪忍不住长流,有些事情一旦面对,终究是不能心如铁石,虽然我早已决定不再留恋故国,可是到了这个时候,仍然不能控制自己,罢了,今夜再放纵一次吧,明日就要用尽心力对付南楚了。不知道若是当初我不曾投靠雍王,今日会不会好过一些。

小顺子有些惊慌失措,这么多年,他从未见过江哲如此失态,他不明白,南楚还有什么值得留恋之处,公子这些日子不是忧心大雍胜过南楚么,怎么答应了献策平楚之后却是如此悲恸。

正在小顺子手足无措的时候,远处突然传来脚步声,小顺子心中一惊,来人脚步声他十分熟悉,抬头看去,果然是李贽带着侍卫正在向这边走来,公子这个模样不能让人见到,小顺子正欲扶着江哲暂避,目光闪处,却看到李贽身边竟然跟着冷川和段凌霄两人,别人也就罢了,自己带着江哲,绝对是瞒不过这两人耳目的。略一犹豫,李贽已经一边笑着一边走进临波亭道:“随云在么,朕可是等不急了?”刚说到这里,他的语声停住了,只因他看到江哲躺在躺椅上似乎已经醉倒了,而且口中喃喃低语,不由放低了声音。

小顺子强压心中忧虑,道:“公子多喝了几杯,已然醉了,不知道陛下亲临,还请恕罪。”

李贽笑道:“是朕太心急了,不关随云的事,罢了,今夜朕就在寒园留宿一夜。”说罢,他坐在江哲身边,正要看一下江哲酒醉的情形,但是目光一闪,却看到江哲眼角的泪光,然后耳边传来江哲的醉语,只听得两句,他已经是面色微变。心中震动之下,无意中抬头看向小顺子,发觉小顺子紧握双拳,目中闪烁着紧张的光芒。心思千回百转,他却是微微一笑,道:“小顺子,今夜朕要和随云抵足而眠,你安排一下。”

小顺子蓦然抬头,露出惊异的神色。

寒园之内,今夜戒备森严,望着寝居之内暗淡的灯光,小顺子忍不住在房内踱来踱去,若非李贽十分坚决,江哲又沉醉不起,不论付出何等代价,他也要避免这种情况的发生,他很担心江哲再说出什么不该说的话,惹恼了李贽。这时有人在外面叩门,小顺子没有去开门,只是冷冷道:“段大公子么,请进。”

门开了,走进来的果然是段凌霄,他笑道:“邪影李顺也有如此不冷静的时候,真是难得一见。”

小顺子冷冷道:“正如我也想不到段大公子会随驾而来。”

段凌霄不以为忤地道:“段某留在雍都为人质,这是事实,段某也不需掩饰,不过今日段某前来,就是想看看楚侯爷如何献策平楚,覆灭故国,想不到却见到他借酒消愁,倒也觉得不虚此行,只不知陛下会如何处置,想必这也是你如此不安的缘故吧?”

小顺子冷冷道:“不关你的事。”

段凌霄笑道:“自然不关我的事,不过四弟来信相询,我不过是想替他看看罢了。”

小顺子眼中闪过烦恼的神色,又望向寝居之内迷蒙的灯光,考虑着如何应付可能会来临的*。

卧室之内,我呻吟了一声,虽然“凝春”并不会让宿醉之后的人头痛,可是我仍然觉得有些不舒服,真是喝多了,不由叫道:“小顺子,给我倒杯茶。”耳边传来有人走动的声音,然后有人端了茶杯过来,我闭着眼睛喝了一口,觉得舒服了许多,翻了一个身准备继续入睡。但是朦胧中突然觉得有些异样,那送上茶水的人动作生疏,而且足音刚劲有力,这分明不是小顺子,我惊出一身冷汗,起身道:“谁在那里?”昏暗的灯光中,只见一个男子负手而立,我一看到那人面貌,吓得立刻酒意全消,爬起身来,也不顾身上只穿着中衣,下床拜倒道:“臣江哲叩见陛下,请恕臣失礼之罪。”

李贽上前一步将我搀起,叹道:“是朕错了,你若是不愿参与南征之事,朕可以不为难你。”

我心中一惊,抬头看时,发觉李贽面上并没有恼怒之色,而且他一身明黄中衣,似是十分随意模样。一时不知道该说些什么,李贽已经拉着我坐到软榻上,自己坐在我对面,感慨道:“想起昔日朕将你强行带回雍都,就是在这府上,朕费尽心机要将你收为己用,往事历历在目,犹如发生在昨日一般。”

这时,我已经平静下来,想必我的失态李贽都已经看在眼里,不论接下来会发生什么,我已经有了准备,因此只是淡淡道:“臣也记得,临波亭赏雪赋诗之事,记忆犹新,就在雍王府内,臣屡次辜负殿下厚爱,迫得殿下左右为难。”

李贽长叹道:“不仅是左右为难,朕是动了杀机,准备在你告辞之时鸩杀于你。”

我身躯一震,这件事情虽然我们君臣两人心知肚明,可是谁都没有捅破这张灯笼纸,想不到今日李贽竟然说了出来,觉得脑后有点凉风,莫非李贽是准备跟我算总帐么,想着这些年自己做的这些事情,有多少可以算的上是欺君之罪,一数之下不免汗颜。

似乎是察觉到我的不安,李贽笑道:“不过总算朕悬崖勒马,才没有犯下大错,留下了你这位国士,这些年来,若没有你出谋划策,朕焉有今日,其实朕也想过不能因为平楚之事难为你,可是到头来还是让你为难了,随云,你若真觉得不情愿,朕答应你从今放你还山,你若不想留在雍都,朕不阻你回东海。”

我听得心中一震,想起昔日君臣际会,龙虎风云之事,忍不住心潮澎湃,望着李贽疲倦中带着诚挚的面容,我终于俯首拜倒道:“陛下何出此言,陛下待臣之厚,亘古未有,如今大雍南征受阻,陛下烦恼难安,臣焉能去过闲云野鹤的日子,陛下,臣心中已有平楚之策,只需三年五载,定能一统天下。”

李贽闻言大喜,道:“随云果然已有良策,快说给朕听。”说着又将我搀起。

君臣二人相视而笑,都觉得前些日子生出的芥蒂烟消云散。

我整理了一下思路,道:“这次我军战败,其实是轻敌之故,若是当时遣大将攻淮西,或许不会遭遇惨败,只是如今情势已经不同,南楚军权皆在陆灿之手,从江淮防线攻入已经殊不可能。

大雍和南楚南北对峙,长江乃是天然的防线,上通巴蜀、中经荆襄、东连吴越,上下游之间相互呼应,若是失去长江,则南楚覆亡只在朝夕,然而如今长江防线尽在南楚控制之下,南楚以淮南为长江之蔽,我军则据淮北重镇,江淮之险,双方共有,以陆灿之能,必然在淮南布置重兵,时刻窥伺淮北,陛下需以重兵拱卫淮北,如此一来,双方在江淮形成对峙之局。

前人曾言‘欲固东南者,必争江汉;欲窥中原者,必得淮泗。有江汉而无淮泗,国必弱;有淮泗而无江汉之上游,国必危。’天下之势,荆襄、青州为江淮两翼,荆襄足以屏护江汉上游,青州足以屏护淮泗上游。如今南楚据有荆襄,则可以固守江淮,但是青州在我之手,南楚也别想北上夺取中原,我军虽不能胜,但已可保证不败。

由此可知,大雍若想南征,荆襄方是关键,荆襄不失,纵然我军得到淮南,也不稳妥,只是荆襄重镇,南楚经营多年,易守难攻,且有江陵、江夏为根基,欲取荆襄,难如登天,这也是屡次大雍南征,往往绕过荆襄,从江淮袭入的缘故,只是这样一来,纵然一时取胜,终究不能持久。且荆襄若在南楚之手,一旦大雍疲敝,南楚可命一大将,出襄阳,攻取南阳,一旦南阳落入南楚之手,则淮北危殆。所以说,若想平楚,襄阳不可不取。”

说到此处,李贽已经是连连点头,却又皱眉道:“随云所说,令我心中豁然,只是荆襄之险,天下罕见,大雍几次攻襄阳,都是无功而回,如今镇守襄阳的容渊,乃是德亲王旧部,熟知军机地理,有他在一日,襄阳不可轻取。”

我笑道:“江淮、荆襄不可取,那么何不另辟蹊径,昔年陛下和德亲王联手攻蜀,不就是因为旧蜀北据汉中,东据三巴,北可以威胁关陇重地,东可以顺水直下江陵,直取东南么,如今我大雍据有汉中,阳平关也在我手中,何不攻取葭萌关,自巴蜀东出,则江淮防线再无用处,如今陛下每每佯攻巴蜀,实在是浪费了大好的战机。”

李贽皱眉道:“巴蜀之重我也知道,只是欲从汉中入蜀,迂回取荆襄,葭萌关、涪城、成都、巴郡、万州、夔州,一路而下,处处险阻,这条路也并不容易走。”

我淡淡道:“巴蜀虽然险关处处,可是若是自西向东,并非十分艰难,而且我们还可以在东南牵制敌军主力,令巴蜀空虚,陛下,何不令东海水军南下,在长江入海口的定海、岱山、普陀等地建立水营,时时窥伺杭州湾,稍有懈怠,则沿长江侵入内陆,南楚为了保全东南各府县的安全,必然将水营重兵置在吴越之地,如此一来,南楚之兵力都集中在吴越和江淮,西面巴蜀自然空虚,我军正可趁虚而入。”

李贽听到此处,不觉站了起来,在室内负手转了几圈,兴奋地道:“好,好计策,朕怎么从没想到可以这样使用水军,原本朕准备在据有荆襄、淮南之地后,调动大雍所有水军渡江而战,却从没想到可以调动东海水军牵制南楚兵力,这样一来,我专而敌分,不论南楚在蜀中、荆襄、淮南、吴越何处露出破绽,我军皆可趁虚而入。”

我也站起身道:“虽然如此,江南防线毕竟稳固,若是陆灿择几处紧要之地死守,我军缓急难攻,故还需用计,不论何等坚固的防线,守备之人若有缺陷,就是可乘之机,巴蜀余缅,守成有余,进取不足,唯承陆氏余威,不足为惧,一旦南楚朝中有变,则巴蜀必定有隙,姑容图之,襄樊容渊,虽然有才有识,只可惜心胸狭窄,此次陆灿立下盖世奇功,他却是苦守襄樊,心中必然生出不满,若令人趁机间之,使其生出怨怼争功之心,则襄樊可乘,即使不能一举夺下襄樊,也可毁去襄樊主力,令容渊再无支援江淮之力。淮西石观,此次立下大功,必然被视为陆氏一党,陆氏若败,此人定受牵连。如今陆灿虽然掌控军权,可是朝政仍在尚维钧之手,且南楚国主即将亲政,素闻赵陇才能平庸,必然会被尚氏利用对付陆灿,而陆灿虽是忠义之人,却并不迂腐,为了保全南楚战力,必然会作出一些令赵陇、尚氏不满之事,文武不合,君臣相忌,南楚倾覆不过是指顾间事,只是其中变化莫测,需小心经营才是。”

李贽连连点头道:“随云一向谋定而后动,其中细节倒也不必详述,朕决意筹立江南行辕,令齐王为帅,督军南征,随云随军参赞,不知道卿意下如何?”

我坦然道:“敢不从命,只是陛下不如令太子殿下为副帅,总督辎重粮饷一切事务,一则为齐王分忧,二来历练太子。”

李贽眼中精光一闪,心中隐忧被江哲解开,不由笑道:“也好,当初朕和六弟都是冲龄从军,骏儿如今已经十六岁了,也该历练一下,就是麟儿,也不妨随军出征,过上几年,朝中又多一员大将。听说那南楚陆云、石玉锦都是十三四岁的少年,却能够阵斩朕的猛将,果然是英雄出少年,想来骏儿和麟儿也不会比他们逊色。”

我脸色微变,一揖到地道:“臣死罪,纵放陆云,还请陛下惩处。”

李贽摇头道:“这也不是什么大事,朕听骏儿说过了,我大雍猛将如云,难道还会忌惮一个小孩子么,就是将来平了南楚,卿若想保全什么人尽管和朕说就是。”

我黯然道:“陛下厚谊,臣心领就是,只是陆氏忠义,臣早已心知肚明,恐难保全。”

李贽也是长叹不已,窗外仍然漆黑一片,我和李贽就在灯光之下,细细的探讨着平楚的种种计策,浑然不知时光流逝,窗外飞雪无声无息地洒向大地,天地间一片肃杀之气。

不知何时,我和李贽谈兴还未淡去,窗外已经是东方发白,宋晚进来催促李贽回宫,李贽一边着衣一边笑道:“随云,记得昔日赏雪赋诗,随云才惊四座,如今窗外飞雪连绵,卿何不赋诗一首,以抒心臆。”

我的心情此刻已经是豁然开朗,只觉得如织飞雪都透着丝丝春意,不由逸兴横飞,推开窗子,望着满园飞雪高声吟道:“连空飞雪明如洗,忽忆清江水见沙。夜听疏疏还密密,晓看整整复斜斜。风回共作婆娑舞,天巧能开顷刻花。正使尽情寒至骨,不妨桃李用年华。”(注1)

李贽拊掌道:“好一个‘夜听疏疏还密密,晓看整整复斜斜’,朕也有一诗咏雪。”说罢推开房门,走向园中,朗声吟道:“五丁仗剑决云霓,直取天河下帝畿。战罢玉龙三百万,败鳞残甲满天飞。” (注2)

我听后不由高声道:“陛下此诗,英风豪气,胜过臣百倍。”

李贽朗声大笑,踏雪而去,已经在外面伺候的侍卫内侍,皆是匆匆追去。只有段凌霄仍然站在窗前,望着李贽背影,道:“若非此等人杰,焉能驾驭江随云这般奇才,段某今日方知,我们败得理所当然。”在他身后,小顺子微微冷哼,转身出了房间,自去服侍江哲去了。

——————————————————————————

注1:黄庭坚《咏雪诗》

注2:张元《雪》

\chapter{第十八章 冠盖满京华}

隆盛八年乙酉三月,雍帝下诏,任齐王显为江南行辕主帅,任太子骏为副帅,总督巴蜀、襄樊、江淮、东海大军百万,南征伐楚,任楚郡侯江某为行辕参赞。

——《资治通鉴·雍纪四》

南楚同泰十二年乙酉元月十三日,南楚国都建业,元宵佳节将临,城内城外都是一片喜气洋洋,年前南楚军在淮西和瓜州渡口的两场大胜,让南楚上下陷入了狂热之中。

十余年前雍王李贽劫掠建业,掳走国主和百官,对南楚的打击超过很多人的想象,虽然此事早已经事过境迁,南楚有了新的国主,又已经重新巩固了江淮防线,可是几乎所有的南楚人都有一种朝不保夕的感觉,随时担心大雍的铁蹄会将眼前的繁华锦绣踏碎,所以,这些年来,江南多了许多矢志雪耻复仇的狂生,更多了许多醉生梦死的轻薄浪子。这一次陆灿取得了淮西大捷和瓜州大捷,不仅洗雪了当年的耻辱,还重建了南楚军民的信心,而陆灿也不再是那些文人攻讦的对象,而是成了力挽狂澜的名将,可以带着南楚军民对抗大雍百万大军,保全江南锦绣繁华的英雄。

这一次的元宵节,正是在大胜之后,所以不论是士绅百姓,都有意借着庆祝佳节表示心中喜悦,所以今年的花灯比起往年更加热闹,满城灯火辉煌,宛如仙宫玉阙一般。秦淮河上更是飘着千万盏莲灯,仿佛天上的星河落入人间,所有的画舫游船都是高高挑起各色花灯,有如琼楼玉宇,更有歌女舞姬穿着霓裳彩衣,在画舫之上载歌载舞,歌声嘹亮,犹如天籁,舞姿婀娜,犹如天仙。火树银花不夜天,此情此景,令人心醉神迷,浑然忘记了人间何世。这还只是十三上灯,若是到了上元日,建业城内外必然更加繁华。

冠盖满京华,斯人独憔悴,在这普天同庆之际,却有人有苦难言,在丞相府的书房之内,此刻却是一片阴云密布。权倾朝野的尚维钧坐在书案后愁容满面,书房内或坐或站还有三个人。一个神色拘谨的中年人站在尚维钧身后,他正是尚维钧独子尚承业,才能平庸,遇事全无主见,尚维钧屡次想要提拔他到要职上,却都不得不放弃,所以他只能在吏部担任一个闲职,在这个书房之内也没有他的座位。其实他在外面也是恣意轻狂的人物,只不过在父亲面前却是战战兢兢,不敢放肆。左首一张太师椅上坐着一个细眉长目的中年人,他正是户部尚书尹端华,尚维钧的门生,也是他的心腹党羽。而在右首坐着的是一个老儒生,他是尚维钧的谋主宁谦,尚维钧多年来在宦场上与人钩心斗角,往往仰赖此人毒谋。

沉默了许久,尚维钧终于忍不住道:“宁先生、端华,你们可有什么主意么,本相已经将封赏之事一拖再拖,可是后日就是上元,无论如何也该封赏大军了。可是陆灿已是镇远公,又是大将军之尊,若是再要封赏,就是王爵之位,异姓不封王,这是金科玉律,可是若不如此,又如何封赏?如今淮东军权已失,南楚军权尽在陆氏之手,一旦陆灿生出不满,只怕我等都要死无葬身之地。”

尹端华忧虑地道:“是啊,陆灿前几日上折子要求扩军备战,他已经掌控了几乎全部军权,却还要扩充军队,这不是存心不轨么?”

尚维钧摇头道:“你过虑了,扩军也是必须的,这次淮东军几乎全部葬送,若不扩军,无法巩固江淮防线,而且若是扩军,我们也有机会安插自己的人进去。”

那老儒生眼中闪过寒光,道:“相爷虽有此意,可是若是任由陆灿征兵,只怕这些新军都会惟陆氏之命是从。”

尚维钧摆手道:“这也是没有办法,我们之中并无可以带兵之人,那个骆娄真将我在淮东的努力全部葬送,唉,不提也罢,还是商议一下如何封赏吧。”

那老儒生捻着胡须道:“相爷不如和陆灿交换一下条件,他不是想要扩军么,此事必须通过朝议,相爷答允支持他征兵备战,但是要他放弃这次的封赏,相爷可以随便给他增加一些采邑,但是不提升他的爵位,这样一来岂不是皆大欢喜,而且面子上也过的去,想来陆灿会放弃爵位换取相爷的支持的。”

尚维钧连连点头,道:“宁先生说得是,扩军不是一件小事,若没有朝廷的粮饷,是不可能顺利进行的,陆灿虽然可恶,可是倒也不是不识抬举之人。这样吧,他的儿子不是立下战功了么,这次就给他一个六品校尉的军职,算作补偿。”

尹端华道:“这倒是便宜了陆氏父子,不过其他有功的将士该如何封赏呢,封赏轻了这些人要闹事的,封赏重了,这些人也多半只是对陆灿感恩,有几个人会想到是国主和相爷的恩典呢?”

宁谦迷着眼睛不语,他不甚赞同尹端华这番话,可是看到尚维钧在那里若有所思的模样,他便没有出言反对。

这时候尚承业出言道:“其实军方也不是铁板一块,这一次陆灿、石观立下大功,可是余缅和容渊虽然守土有功,可是毕竟功浅,父亲不如重重封赏石观,却对余缅和容渊一带而过,余缅倒也罢了,那容渊可还不是陆灿的死党,此人心胸又是有些狭窄的,必然因此嫉恨陆灿,父亲不妨私下对其多加抚慰,此人可是有真才实学的,又是德亲王的旧部,本是忠君爱国之人,说不定会投入父亲麾下呢。”

此言一出,不仅尚维钧目光一亮,就是尹端华和宁谦也都连连点头。尚承业在这种场合素来不多言,今日突然献策,却是如此妙计,令尹、宁二人刮目相看,连连赞誉。尚维钧却是知道这个儿子的深浅,惊奇地问道:“你今日倒是言之有物,不知是谁的主意?”

尚承业脸一红,道:“父亲,是我新结识的一个朋友,是个寒门书生,无心科举,只在烟花柳巷里面给那些歌女作曲填词,虽然人在万花丛中,却是洁身自好,孩儿见他气度高华,所以折节下交。前些日子和他一起喝酒,无意中说起大将军如今权威之重,已经胜过父亲,他便笑着说陆灿仍不能一手遮天,若是如此这般,必能有效。”

尚维钧目光闪动,道:“你可仔细查过此人身份,以你的身份,交友不可不慎。”

尚承业赧然道:“孩儿只是和他诗酒相交,所以并不了解他的身世,不过此人雅量高致,才华横溢,只可惜看破世情,无心功名,父亲若是有意,孩儿可以试着延揽他到父亲幕府。”

尚维钧摇头道:“先看看吧,用人不可不慎,不过这人如此才具,倒是不可轻忽,你先好好笼络他,若是身份没有问题,倒不妨招揽进府。”说罢,尚维钧犹豫了一下,又道:“还有一件事,本来我有心将义女灵湘许给陆灿长子,若是能够联姻,也可多些控制陆氏的筹码,可惜却被陆灿拒绝,你们看可有挽回余地?”

宁谦皱了一下眉,他自然知道这个灵湘是何许人,她是凤仪门仪凰堂首座纪霞的义女,却又拜了尚维钧为义父。事实上,宁谦也知道纪霞和尚维钧的暧昧关系,虽然凤仪门的种种传闻尚维钧也清楚,可是一个曾经是大雍贵妃的女子的吸引力太大了,所以尚维钧还是陷入到了凤仪门的柔情陷阱之中。这件婚事被陆灿拒绝早在宁谦意料之中,若是陆灿不拒绝才奇怪呢,陆氏未来的家主,自然该娶一位南楚名门的淑女,怎能娶一个出身不明的女子为妻。犹豫了一下,宁谦婉转地道:“相爷,若是有意联姻,不妨考虑一下淑宁长公主。”

“淑宁长公主!”尚维钧喃喃低语,淑宁公主是当今国主赵陇同父异母的妹妹,今年十五岁,品貌乃是上上之选,只不过母亲早已经亡故,在王室并无地位,尚维钧更是没有留意到她的存在,如今听到宁谦提醒,他心中一动,若是许个公主给陆氏,这不是最好的笼络么,毕竟还是需要依靠陆氏抵抗大雍的。而且若是陆氏有了反意,淑宁长公主也可以起到平常人起不到的作用。

就在尚维钧和亲信在书房密谋的时候,奉命回京接受封赏的陆灿等人已经入城了。不愿惊扰百姓,所以陆灿乃是微服入城,望着满眼的富贵升平,他一声轻叹,虽然这次取得淮西大捷和瓜州大捷,可是他没有忘记淮东重镇楚州、泗州已经落入雍军之手,而且雍军随时可以调动大军南下,到时候南楚面对的压力只能更大。而且最关键的是,大雍遭遇如此惨败,雍帝必然起用江哲,只恐大雍再度南征之时,自己的恩师就会随军南下。

不过他心中的苦恼显然没有感染到身后两个少年身上。石绣东张西望地看着道路两边的花灯,俊秀的面容上满是惊讶憧憬的神情,陆云则是为她一一指点着沿途的景物,像极了最好客的主人。这次两人都是奉诏入朝受封赏的,虽然石绣本是女子,按例不在封赏之列,可是两人如今已经是南楚人人传颂的少年英雄,又因为军报的含糊,以及建业的失误,使得石绣也得到了入京受赏的旨意,虽然石观上书说明此事,但是最后建业为了激励军心,还是决定将错就错,对“石玉锦”进行封赏,只不过在旨意里面含糊其词,没有说明石玉锦是男是女罢了。

望着街道两边的绚烂灯火,陆云心中也是有些忐忑不安,当初他不辞而别离开建业去了雍都,从长安回来之后又被父亲直接送到了江夏,然后又去了淮西战场,算起来离家已经有将近十个月,想必娘亲必然是为他操碎了心,这次恐怕会被娘亲重重责罚,虽然罚跪挨板子都不算什么,可是若给弟妹看到可是太丢人了。转念一想,不如想法子让几个弟妹在娘亲面前替自己求一下情吧,不过这却需要先贿赂一下几个小家伙。盘算了一下,二弟也喜欢骑射,自己就将嘉郡王送给自己的犀角弓给二弟吧,大雍工部精制的弓箭可是上上之选,而且自己也不好意思使用李麟送给自己的宝弓去射杀大雍的将士。小弟么,年纪还小,就把自己在路上买的面人、木偶送给他就行了。至于小妹么,陆云心中一跳,想到了怀中那枚金环,然后他便想起了昭华郡主亦喜亦嗔的娇颜,那本已模糊的娇俏少女形象再次鲜明起来。

这时候石绣不耐烦地高声道:“云弟,你在发什么呆呢,那是什么灯啊,好漂亮啊。”

陆云顿时惊醒过来,脸一红,转头看向石绣,看到这个和自己并辔作战的少女面上带着灿然的光彩,被寒风吹得通红的面庞是那样的动人娇艳,这一刻,他忽然觉得自己身边的原来是个女孩子,突然心念一动,从怀中取出金环递给石绣道:“绣姐,这个送给你。”

石绣原本大怒,正要纠正陆云的称呼,一眼却看到那枚花枝盘绕的金环,无论如何,她终究是一个少女,一双明亮的大眼睛弯成了月牙,接过金环爱不释手。陆云心中发虚的想到,石绣和自己情同手足,将金环送给她也说的过去吧,虽然昭华郡主原本说送给自己的妹妹。这时候石绣却是依依不舍地将金环递了回来,低声道:“这太贵重了,你还是收回去吧。”石绣虽然素来不留心这些细务,可是这支金环如此精美绝伦,想必千金难买,她怎能收下这样贵重的礼物。

陆云目中闪过一丝光芒,低声道:“这也是朋友送给我的,你就当替我保管吧。”

石绣本想拒绝,却不知怎么说不出口,只是低头把玩着那支金环,无意中目光一闪,看到金环相连之处的寒梅花蕊之中有两个细如米粒的小字,石绣凝神看去,却是“昭华”二字,不由心中一动,笑道:“那好,我先替你收着。”

陆云只觉得放下了心中大石,笑道:“等到十五那天,我带你出去逛灯会好不好,现在不过是走马观花,有许多好玩的地方你还没有见过呢?”

石绣闻言眼中一亮道:“好啊,听说秦淮河很好玩儿,水上都是莲花灯,而且还有杂耍和歌舞可以看。”

陆云连连点头答允,石绣面上露出甜美的笑容,两人在马上凑近低语,商议着如何去玩耍,这一刻,两人可不是名扬江南的少年英雄,只是一对没有长大的孩子罢了。

两个孩子的低语都被陆灿听得清清楚楚,他心中烦恼稍解,想到石观隐隐透出的结亲之意,更是不由微微一笑,再想起年余不见的妻子儿女,心中生出无限柔情,加了一鞭,加快了马速,向前走去。

镇远公府在建业城南,府邸庄严肃穆,今日中门大开,门前张灯结彩,家主战胜归来,阖家上下自然都要出来迎接,为首的中年女子端庄秀丽,正是陆灿之妻。在她身后一左一右站着两个小孩,左边的男孩十岁左右的模样,和陆云相貌相似,只是略显秀气一些,他是陆灿次子陆风,右边的女孩只有七八岁模样,年纪虽小,却是已经如同仙露明珠一般清丽,此刻正倚在母亲身边偷偷打量着众人,她是陆灿独女陆梅。在三人身后,还有一个中年妇人抱着一个两三岁的小男孩,这个小男孩生的虎头虎脑,十分可爱,却是陆灿幼子陆霆。

石绣站在陆云身边,不知怎么心砰砰跳,她早知陆夫人是名门出身,定然是四德俱备,她却是假小子一般,这两年娘亲没有少教训自己,若是陆夫人也那样罗嗦可怎么办。

这时候陆夫人带着众人向陆灿见礼已毕,陆云忐忑不安地上前给娘亲见礼,陆夫人一看到长子,眼中顿时一片朦胧,拉起爱子上下打量了半天,确定爱子完好无损才放下心来。这时候轮到石绣上前见礼,石绣偷眼看了陆云一眼,上前拜倒见礼。

陆夫人早就接到丈夫的书信,知道了石绣之事,也知道丈夫有意联姻,更知道这个男装少女英武非常,在战场上和爱子并辔杀敌,心中早已存了好感。上前搀起少女,轻轻将她抱入怀中,道:“你就是绣儿吧,好孩子,多谢你了,若不是你拼了性命,我的云儿只怕就没命了。”

石绣闻言满脸通红,她知道陆夫人所说却是自己在战场上诈死之后,暴起刺死董山的事情,虽然在效果上救了陆云性命,但是实际上却是两人联手之功,她正要解释,却看到陆云偷偷给她使眼色,不由住口不言。陆夫人一见这个少女不安的模样,心中更是欢喜,拉着她的手道:“你也不要拘束,到了这里就是到了家一样,我待你和云儿一样。”一握住少女的手,便觉得那只纤手刚劲有力,而且皮肤有些粗糙,显然是常年练武留下的痕迹,心中生出怜惜之意,再看看陆云紧张的神色,突然觉得有这样一个儿媳也不错,本来尚存的一丝疑虑也消失无踪,含笑拉着石绣的手向内走去。

陆云只觉得心中一宽,轻拍胸膛,觉得没有那么紧张了,然后他便看到二弟陆风和小妹陆梅闪亮的眼睛,两人一左一右拉着他,陆风恶狠狠地道:“大哥,你骗我替你偷盘缠,结果害得我被娘亲罚跪。”陆梅却是眼泪汪汪地道:“大哥,以后带梅儿一起偷跑好不好?”陆云只觉得一股暖流流入心湖,伸出双手将弟妹抱住,久别重逢的激动之情让他几乎说不出话来。

在镇远公府的大门缓缓合上的时候,在街道对面的一家酒楼上面,临街的包厢之内,一个青年微笑着饮下一杯酒,望着紧闭的朱红大门,眼中闪过一丝寒光。

\chapter{第十九章 依稀旧人影}

这个青年大约二十八、九岁年纪,是一个青年儒生,穿着一身洗得发白的青衫,腰间系着一支斑竹箫,似乎颇为落魄,但是他相貌清秀儒雅,气度高华,仿佛对清贫的生活毫不在意。这青年手中始终把玩着一柄折扇,折扇摇摇,忽开忽阖,隐隐约约露出扇面上面的美女影像。这柄折扇华美名贵,和他清寒的衣着形成鲜明的对比,而且轻浮的美女扇面和他清冷的神情更是不甚相称。可是奇异的是,这种种的不协调,却透出一种莫名的和谐,让这个青年越发显得风姿俊逸。

那青年又饮了数杯酒,低吟浅唱道:“惆怅梦余山月斜,孤灯照壁背窗纱,小楼高阁谢娘家。暗想玉容何所似,一枝春雪冻梅花,满身香雾簇朝霞。(注1)”

他的声音有些低哑,可是这一曲唱来却是宛转低回,深情相寄,这酒楼中本是高朋满座,他的歌声一起,竟是满座寂然,他的声量并不高,却是人人听得清清楚楚,都是侧耳倾听,更是有人和着曲调轻轻打着拍子。刚唱到第二句,楼中响起清丽动人的笛声,笛声伴着歌声,越发的令人心醉神迷。

一曲唱罢,笛声却没有停止,然后楼中便又响起一个女子澄净透明的歌声,那女子却是将青年所唱的曲子重新唱了一遍,虽然是同样的曲调语句,细节处却是多了许多变化,且那女子的歌声百转回肠,将那词中深意演绎的淋漓尽致,令得楼中众人浑忘今夕何夕。

那青年微阖双目,品味着那美妙绝伦的歌声,良久,歌声消散,有轻盈的足音在厢房门口停住,他睁开双目,叹息道:“定是如梦姑娘亲临,唉,姑娘的歌舞千金难买,如今却在这小小酒楼之内展露歌喉,若是给建业风流子弟知道,定然是捶胸顿足,长叹不已。”

竹帘一挑,一个身披红色昭君套的女郎飘然而入,在她身后则是一个青衣侍女和一个彪悍雄壮的大汉。这女郎入得厢房,那青衣侍女帮她脱去昭君套,那女郎长身玉立,穿着一身朴素无华的白缎子曳地长裙,仿佛一朵白莲无声绽放。那女郎大约二十出头年纪,相貌秀丽清雅,姑且不论她肤若凝脂,柳眉如叶,只是那一双清澈明晰的秋波明眸,流转处便是万种风情。她上前翩翩下拜道:“妾身柳如梦,见过宋逾宋先生。”

那青年微微一笑,起身道:“如梦画舫柳姑娘,素以歌舞清议闻名江南,宋某不过是个寒门浪子,如何当得起姑娘大礼。”但是他眉宇之间却是傲气不减,没有一丝一毫自卑之意。

那女郎轻轻一叹,眉宇间露出淡淡的愁容,明眸流转,更觉愁肠百结,她低声道:“妾身在秦淮以声色娱人,却是时时受人排挤欺凌,这一次南楚大军击退雍军,秦淮所有青楼画舫共议,上元日要在玄武湖举行花魁大赛,选出三人分称状元、榜眼和探花,从今之后,只有这三人能够称得上花魁娘子。从前大家都是各自为政,只需捧场的人多了,便可被同行尊为花魁,这一次却和以往不同,众位姐妹需要当场献艺,再由满湖贵客品鉴,胜者名扬江南,败者从此无颜。”

那青年淡淡道:“如梦姑娘色艺双全,秦淮谁不知晓,何必担心此事。”

柳如梦眼中似乎闪过泪光,道:“妾身一向独来独往,不受拘束。秦淮青楼如今却隐隐是双雄对峙,万花楼和月影轩互不相让,这一次为了争夺花魁,双方都是费尽心思,万花楼倒还罢了,他们推出的头牌秋雁姑娘,色艺不在妾身之下,那月影轩的萧二娘却是百般设计逼迫妾身加盟,妾身不允,他们便施展诡计,偷去了妾身为这次盛会求得的新词,若是妾身在玄武湖盛会之上,只能唱些陈词滥调,别说花魁之位得不到,恐怕还会被人耻笑。妾身想来想去,只有宋先生才可助我,还请先生垂怜。”

那青年闻言皱眉道:“你应知道,我虽然常常替人写些诗词,却是多半都是替万花楼旗下的姑娘效劳,我与万楼主也算是交情不浅,这一次事关重大,我若是相助于你,岂不是得罪了万楼主,而且秦淮谁不知道月影轩的秦二娘心狠手辣,我若坏她大事,只怕在秦淮再也不能安身,如梦姑娘,你应知宋某苦衷。”

柳如梦掩面道:“若没有四五首新词,只怕难以支撑,急切之间,妾身到何处购得这许多华美新词,唉,难道妾身这次真要一败涂地,罢了,我柳如梦终究是不如柳飘香,想当初飘香姑娘舞姿倾城,在秦淮河上独树一帜,想起她笑傲公侯,痛斥韩王的传说,如梦每每觉得荡气回肠,总想着效仿飘香姐姐英姿,如今看来,不过是痴人说梦。”

那青年闻言眼中闪过最深沉的哀痛,转瞬消逝,继而叹息道:“如梦姑娘有这样的志气,宋某佩服,若是姑娘不嫌弃,宋某情愿相伴妆台,为姑娘填词作曲,却不知道姑娘缺不缺琴师,宋某的琴技也是颇有可观之处。”

柳如梦原本见最后的希望断绝,不由说出内心之言,想不到宋逾却突然答应为她写词,更是愿意进一步做她裙下之臣,不由喜出望外,放下衣袖,秀丽的面容上珠泪盈盈,此刻破涕而笑,越发显得美丽不可方物。她上前扯着宋逾衣袖道:“哎呀,宋先生若肯屈尊,如梦情愿拜先生为师,恭聆教益。”

宋逾见她惊喜交加的神情,只觉得心神一荡,竟是不能自持,他混迹青楼烟花之中,本是为了麻醉自己,对于那些莺莺燕燕,不过是逢场作戏,最放纵的时候也只是手眼温存,虽然身在百花丛中,心却如古井无波。柳如梦虽然一向闻名,但是他心中有结,一听说此女姓柳,便故意避开,至今从未见面,怎也想不到今日一见,这柳如梦不论品貌才情,都像极了他心中倾慕已久的佳人,怎不让他心醉神迷。

宋逾,本是南楚寒门之子,本名宋敏,十二岁时已经中了秀才,被乡里誉为奇才,却不料家遭回禄,不得已流落建业,贫病濒死之际为名动江南的名妓柳飘香所救,并留他在飘香画舫上做了一个小厮。其时他虽然年少,但是却对柳飘香生出倾慕之心,为了心中痴情,他甘心情愿留在画舫之上操持贱役,虽然根本没有机会接近佳人,可是柳飘香的一颦一笑却都是他心中最珍贵的回忆。因为他时刻留心,就连柳飘香和江哲的私情他也略知一二,虽然也为柳飘香得以匹配良人欣喜,但是心中之痛也不能稍减。在柳飘香飘然离开画舫之后,他便伤心离开,因此避过了之后降临的灭口屠杀。其后他因缘际会加入了秘营,却又惊骇地得知柳飘香已经香消玉陨。为了替心上人报仇,他专心苦练,虽然练武的资质不过中上,可是在他不懈的努力下,终于晋位八骏,得江哲赐名逾轮。

秘营八骏,龙组,赤骥最得江哲重用,有将才,重情义,盗骊性情坚毅,处事冷静,却是外冷内热;虎组,白义外表朴实,却有领袖之才,统率着秘营的主要战力;暗组,山子精于机关暗器,甚至后来为之荒废了武功,但是秘营暗组的刺杀计划,却往往依赖于他的支持,渠黄,相貌平平,令人过目即忘,往往在敌人濒死之前,才会察觉他的存在;隐组,骅骝,外表平和,容易亲近,可是心思缜密,虽然经常会因为情义手软,可是真正需要的时候,他可以冷酷无情到极至,绿耳,外表爽朗亲切,实则精明能干,善于经营。

而逾轮则是八骏中最特殊的一个人,他本来是虎组之首,位在白义之下,可以说他的武功在秘营之中是出类拔萃的,本来也应该和霍义一样明火执仗地杀人,可是他却更喜欢做刺客,原本江哲因为他相貌气度过于出众,认为他不适合进入暗组,可是到了后来,却人人都不得不承认,他是最出色的刺客。他手中的折扇便是他的武器,折扇的扇骨乃是精钢所制,中藏钢针暗器,可以在对敌之时直取敌人要害,死在这柄折扇下面的高手数不胜数。不过逾轮却多半是采用暗算偷袭的方法制敌,他筹划严密、布局细致,出手从不落空,善用计谋,体察人心,时有神来之笔,往往在不可能的情况下取了敌人性命,却无人知道是他动的手。而他从一出道的时候,就用放荡不羁的行为来掩饰自己的真面目,再加上他才华出众,写诗填词一挥而就,稍有余暇就流连于烟花柳巷之中,这种种放纵举止,便成了他最好的掩饰。表面上,他是气度高华的书生,形迹放荡的浪子,却无人想到他会是铁石心肠的刺客。

秘营弟子于南楚显德二十二年元月正式出师,大雍隆盛六年元月,也就是两年之前,按照当初的十年之约,秘营弟子都可以获得自由,去过自己想要的生活,甚至在这之前,赤骥、骅骝都已经正式脱离了秘营,而盗骊的精力也是更多的投入到了海氏船行之中。虽然得到了自由,可是秘营众人却是几乎都选择了继续效忠江哲,毕竟不论想要得到富贵还是财富,跟着江哲都不难得到,更何况他们对江哲的忠心本就根深蒂固。逾轮几乎是唯一的例外,身列八骏之一,他已经是江哲的记名弟子之一,大雍国势正盛,江哲如日中天,有这个身份,他几乎可以得到梦寐以求的一切。可是他却选择了脱离秘营,回到南楚国都建业度过往后的人生。逾轮不知道江哲是否有过将他灭口的打算,可是最终他平安地回到了建业,而且过上了想要的生活。比较而言,八骏之中,他对江哲的忠心是最淡的一个,离开秘营和江哲,不是为了南楚和其他什么原因,事实上,如果江哲强迫他留下,他也不会反抗,他只是想回到最初的开始罢了。

离开了秘营之后,逾轮的生活很快就陷入了困境,他在秘营所学的都是杀伐阴谋,独独没有学过如何谋生,毕竟他不是暗组、隐组之人,多年的高高在上,他也不再习惯低声下气,更别提靠气力谋生了。他唯一的才能就是杀人,却连如何联络刺杀生意都不知道,除此之外他还会的就只有写诗填词,可是他又不屑以诗词换取金钱,更何况他在秘营之时也不重钱财,有了金银也往往很快就挥霍一空,若非是临去之时得到了一笔盘缠,恐怕他只能两手空空的离开了。

摆脱了羁绊之后,逾轮几乎是直接就到了秦淮河,他气度不凡,相貌俊秀,再加上文采飞扬,囊中多金,很快就成了秦淮河上的佳客。每日里流连于风月之中,倚红隈翠,醇酒歌舞,闲来便是吟诗作对,他的诗词清雅动人,缠绵悱恻,寻常歌女唱熟一首,也能够红上半月。后来他囊中金尽,若非是时常有青楼中的红牌向他求取诗词,然后以金银相赠,只怕他早已囊空如洗。

即使是这样,没有多久他就已经一贫如洗,从锦衣玉食、一呼百应的地位落到这种窘况,若是常人不免苦恼悔恨,逾轮却是甘之如饴,这样清贫的生活过了整整一年半。直到渠黄有一日到建业办事,知他隐居在此,特意来看望他,见他贫苦如此,渠黄几乎惊呆了。结果素来沉默寡言的渠黄不由分说扯着他去酒楼对饮一夜,然后留下身上几乎所有的金银便消失无踪。一月之后,渠黄再次出现,却是带来了一个刺杀任务。从那之后,逾轮的生活有了改变,每隔一段时间,他会从天机阁或者秘营手中得到各种各样的任务,这些任务都集中在建业附近,而且多半颇为艰难,其实天机阁在建业颇有一些产业,而且秘营在建业的活动也很频繁,只是逾轮离开秘营之后,不清楚其中的详情罢了,每次完成任务,所得的酬金足以让他过上一段时间的豪奢生活,这才让他不至于贫无立锥之地。

逾轮没有犹豫就接受了这样的改变,虽然从昔日的秘营主事变成了今日被驱使的工具,他却没有丝毫怨言,也没有丝毫悔意,他生命的火焰仿佛早已在十余年前燃尽,只有在秦淮风月之中,逾轮才能感觉到平安和喜乐。其实有的时候,逾轮自己也不明白,为什么会像扑火的飞蛾一般无怨无悔,每当他想弄清楚的时候,眼前总是泛起那永远不能忘记的明艳面容。

直到今日,在这座普普通通的酒楼之上,他遇到了柳如梦,才感觉到生命似乎重新有了波澜,这个女子相貌和柳飘香没有任何相似之处,可是在她倾述衷情之后,逾轮却发觉,这个女子的气质风情,竟是像极了他梦萦魂牵之人,也只有这个缘故,才能让他答应留在这女子身边,浑然忘记三月前接下的任务是多么的凶险难测。想到此处,他看向柳如梦的目光越发凄清伤恸。

柳如梦心细如发,自然能够觉察出来他情绪的变化,对于这个青年宋逾,她早有耳闻,秦淮河上很多姐妹都对她提过此人,只是不知何故,始终两人不曾相见,她也想过是否宋逾有心避开,可是却觉得殊无可能。姐妹们都说宋逾为人古怪,虽然每日里不是长歌当哭,便是买醉秦淮,又在风月场中左拥右抱,挥金如土,任性放纵,对着高官文士也往往白眼相看,但是对着自己这些卖笑为生的女子却没有半点傲慢,而是以友朋相待,全不似那些在秦淮寻欢作乐的那些男子,纵然是满面堆笑,也是心中鄙夷。一位心细的姐妹曾说,这位宋先生虽然身处花丛,却从不曾真得开心,纵然是脂香粉腻,也遮不住他冷落风华,纵然是欢声笑语,也掩不去他眼中痛楚。柳如梦原本半信半疑,今日一见才知道果然如此。只是不知道他未过而立之年,缘何心伤如此,以至于明珠蒙尘。

不过宋逾身上的隐秘可以慢慢去发掘,柳如梦施礼道:“先生既然允了如梦,不若现在和如梦回去吧,唉,月影轩素来蛮横无理,若给他们知道先生相助妾身,恐有不忍言之事。”

逾轮收回目光,淡淡道:“月影轩的人我还不放在心上,姑娘请先回去吧,明日我自会到画舫相见。”

柳如梦欲要再劝,见宋逾神情冷冷,眉宇间流露出不可违逆的肃然气息,心思千回百转,翩翩下拜道:“既如此,妾身就在舫上恭候先生。”

逾轮背过身去,举杯邀月,心中一阵酸楚,忍不住低声道:“昔日的多情公子,如今恐怕眼中只有新人颜如玉,哪里还记得建业城古坟凄凉。柳姑娘,原以为世上除了我再无人记得你,想不到今日风尘之中你竟还有一位知己。”

正在逾轮回肠九转之时,有人大笑着挑帘而入,道:“宋兄弟,这次为兄可是露了脸了,多谢你的主意,怎么这样的好日子你却在这个小地方委屈,怎么样,和我一起去月影轩痛饮几杯如何?”

逾轮眼中闪过一丝冷意,笑道:“尚兄言重了,我不过是随便说说,那些国家大事自有人去操心,何必我们这些小民多事呢,喝酒可以,不过尚兄可不要再说那些败兴之事才是。”

那人正是尚承业,他虽然是尚维钧独子,身份贵重,然后平庸驽钝,平日所遇之人不是谄媚讨好,就是表面尊重,实则鄙夷,尚承业虽然愚笨,时间久了,也知道身边之人多是虚情假意,唯有这风月场中结识的好友,虽然时常冷言冷语,却是只将他当作一个寻常人看待,相处起来自在如意。所以闻言之后,不仅不恼怒,反而笑着上前拉起逾轮向外走去,一边走一边道:“这有何妨,军国大事自有我爹他们理会,快走吧,今次一定要一醉方休。”

逾轮微微一笑,任由他拉着向外走去。

————————————————————

注1:韦庄《浣溪沙》

\chapter{第二十章 恩重爱深}

同泰十二年上元日,时人未解兵燹将临,且庆淮南扬州大捷,乃起盛事于玄武湖,百花争艳,以夺魁首。其中最佳者为柳姬,众以状元呼之,其时烟水尤寒,柳姬舞于湖心,雾生足下,烟笼娇姿,凌波飞舞,水过无痕,疑似画中仙,见者皆醉,后二十年,无人能胜。

柳姬者,本姓乔,小字素华,母曰乔姬,乔姬名霞,善博有侠气,华为其养女,亦侠而慧,颇知书,十四岁待客舫上,唯洁身自好,欲觅知音,豪贵爱其色艺,虽千金不至。不意遇薄幸子,愤而自经,救而复苏,乔姬恐其复寻死,令侍婢守之,柳姬笑曰:儿死而复生,乃悟世情冷暖,母毋忧。乃改其行,设锦帐于河上,以声色歌舞娱人。柳姬雅擅歌舞,言辞便利,每于舫上召宴,席间顾盼生姿,众皆目眩神迷。

姬为人豁达,不重金帛,有人缓急求之,虽千金不惜,且不畏强横,遇事则仗义执言,常有义举,秦淮众妓多受其恩义,不论年岁,皆以姊呼之。姬平素读书,最喜前贤“人生如梦”句,且慕秦淮故妓柳氏飘香之行,乃改柳姓,自名如梦。

——《南朝楚史·柳姬传》

上元日,建业城内的气氛到了最热烈的顶点,将近未时,玄武湖上面的花魁大赛也已经进入了最后的高潮,在玄武湖湖心搭建的高台之上,每个想要夺取花魁的女子都可以在上面表演才艺,表演之后还要乘着画舫游湖一周,让一湖之人都可以看得清楚,所过之处,宾客可以将手中珠花投到船上,以珠花多者为胜。如此进行三轮比试。第二轮珠花数目最多的三人便是江南花魁,不过这三人还要经过第三轮决赛,这一场便是最后的博弈,要决出状元、榜眼、探花的名次,虽然都是花魁,可是名号的不同将决定谁是江南第一花魁,所以这一轮的比试只会是更加惨烈。

至于珠花乃是秦淮青楼赌场所制,是用黄金混合铜铁打造成的,形似一朵盛开的牡丹花,一朵珠花售价一两,湖中四处都有小舟游弋,向观看比赛的宾客出售珠花。如今前两轮已毕,已经稳占花魁之位的三人都是名头不小,万花楼的碧烟姑娘,媚态天生,舞姿曼妙,月影轩的灵雨姑娘清丽如仙,精通音律,最后一人,就是在秦淮独张艳帜的柳如梦。万花楼和月影轩都是江南最有名的青楼之一,更是暗中控制了江南七八成的青楼赌场,他们参赛的人选进入最后的决赛也是理所当然,倒是柳如梦一向独来独往,能够入决赛实在是众望所归,不少平日只能在两大势力之间苟延残喘的秦楼楚馆的主事人都是暗中相助,希望柳如梦能够夺得状元,也好扫扫两大势力的脸面。

前面两轮处于弱势的碧烟决赛中第一个出场,她的歌喉略逊其他两女,倒是舞姿十分出众,所以这一次她表演的是“胡旋舞”,白色纱衣、长袖如云,绿色绫裤、红色锦靴,腰间缠着轻纱彩带,身上佩着珠玉琳琅,走到台中锦毡之上,美目流转,风情万种,虽然只是站在那里,却已经展现出天生的娇媚艳骨。

台下画舫之中,富有西域风情的弦鼓声破空而起,碧烟两脚足尖交叉、左手叉腰、右手擎起,已经在乐声中飞旋起来,随着乐声的越来越急促,她的飞旋舞姿也越发迅疾,转眼之间,已经看不清她的容颜体态,只看见长袖回旋似飘雪,彩带轻纱似飞蓬,身上所佩的珠玉更是相互撞击,发出清脆的金玉之声,和乐声暗合。这样罕见的歌舞,以及碧烟婀娜刚健的舞姿令得湖上众人纷纷喝彩。

更有人从记忆中回想这种舞姿的来历,却是想不起来,还是有些博学多闻的人猜测到这是东晋时候从西域传来的胡旋舞,不由都佩服万花楼的苦心,连已经失传的胡旋舞都发掘了出来。原本三女之中以碧烟声名最弱,多半都认为她虽然娇媚,却少了几分才艺,今日在湖上一舞,霎时减弱了她以色事人的印象。

不知道碧烟在台上旋转了千次还是万次,就在众人看的眼花缭乱,激动难抑,高声喝彩的时候,乐声嘎然而止,碧烟停住身形,对着四面贵宾一一施礼,在台上顾盼生姿,神采飞扬,博得阵阵喝彩之声。

当碧烟游湖一周,满载而归之后,月影轩的画舫接近高台,众人平静心情,等待夺魁呼声最高的灵雨出场,灵雨姑娘是月影轩的当家花魁,冰清玉洁如白莲,楚楚动人如弱柳,琴艺无双,许多琴中圣手都自愧不如,更难得是,她至今守身如玉,尚无人可以攀折这朵名花。画舫停住之后,众人都看着舱门,等待灵雨出现。孰料灵雨身影始终不见,一缕琴音却从舱中幽幽飘出,如同春露花雨一般的点点滴滴渗入人心,又似飞雪飘舞透着清冷孤洁之意,轻易地将人引入如梦如幻不可自拔的神秘之境。一曲终了,一扇窗子无风自开,露出一个翠衣女子的侧影。灵雨姑娘在月影轩当众抚琴之时,也是白纱覆面,只有被她延入香闺之人才能见到她的面容,今日虽然只是半面对人,但已经是引得众人全神贯注地凝视,几乎是大气也不敢喘,都希望能够见到这位出水青莲也似的佳人真面,更何况虽然看不到花容月貌,但是那灵秀的轮廓,如雪的肌肤,如云如墨的青丝,已经引起众人无限美好的遐想。

此刻,远处的如梦画舫之上,柳如梦秀眉轻颦道:“好一个月影轩,这般安排真是独具匠心,若非是先生相助,如梦此番必定输给了她。”

逾轮负手站在窗前,望着月影轩的画舫道:“宋某虽然混迹青楼,只可惜囊中空空,无缘见到灵雨姑娘真面,灵雨姑娘琴艺无双,也不需要靠宋某的诗词招揽客人,不过宋某几次听到她的琴声,都觉得纵然是最欢乐平和的曲调,在她手中也是别有一种幽愁暗恨。”

柳如梦叹息道:“我曾和灵雨妹妹有缘相会,只觉得她心中隐隐有着不可排解的苦恨,说来也难怪,灵雨妹妹品性高洁,怎堪忍受青楼生涯,这样的生活,实在不是她那样的柔弱女子可以承受的。”

逾轮听得出来,柳如梦的语气是真诚的,而且毫无自怜之意,就像当年的她一样,心中闪过一丝喜悦,他笑道:“如梦姑娘可不要为了同情她而放弃今日的比赛啊?”

柳如梦面上神采焕然,笑道:“同情归同情,我可不会放水。”这时,灵雨已经退场,柳如梦站起身来道:“也该轮到我了。”言罢,向舱外走去,她此刻穿的是粉色绣缛,荷叶曳地长裙,行动之间宛若荷花凌波,动人至极。逾轮目中闪过一丝悲凉,取下腰间的斑竹箫,轻轻抚摸,诸般乐器,他最爱的就是竹箫,只因箫声幽怨,可以将他的心事尽情倾诉出来。

欣赏过碧烟和灵雨的出色才艺之后,众人的目光都集中在如梦画舫之上,毕竟前两场柳如梦凭着两曲新词和动人的歌喉赢得了第一,不过这一轮比赛两女都已经尽展所长,若是柳如梦不能别出机杼,恐怕只能屈居探花了。在众人热切的目光中,如梦画舫向湖心荡去,不过令众人奇怪的是,还有四艘小舟随在画舫之后而行。到了高台之下,从画舫舱门走出二十四个彩衣女子,各自捧着各色乐器,婀娜多姿地登上小舟,四艘小舟围住了高台。一个抱着琵琶的端丽女子玉手一拨,铮然的琵琶声铁骑突出,随后那些女乐开始弹奏起来,曲调缠绵清越。

湖边众人议论纷纷,虽然说柳如梦这样安排也不算违规,可是三女这等才艺,已经不是寻常的乐师舞姬可以改变大局的了,正在这时,有人指着湖心惊叫道:“起雾了?”众人凝神看去,只见从四艘小舟溢出白色的轻烟薄雾,今日湖上原本有微风,那些烟雾却凝而不散,瞬间将高台遮住。就在众人迷惑之时,那些小舟也被烟雾裹入其中,身形若隐若现,这时,一缕如同天籁一般的歌声从雾中飘出。

“碧荷生幽泉,朝日艳且鲜。秋花冒绿水,密叶罗青烟。秀色粉绝世,馨香谁为传?坐看飞霜满,凋此红芳年。结根未得所,愿托华池边。”(注1)

众人听得如痴如醉,比起柳如梦前面的两曲,这一曲更多了一种足以令人销魂蚀骨的意味,恍惚间,众人只觉那雾中定是有天上的仙子正在顾影自怜,轻歌漫唱,自己这些人便是无意偷听到天上仙音的凡夫俗子。

一曲终了,正当众人意犹未尽的时候,台上的轻烟渐渐沉落,也消散了许多,露出了翩翩起舞的身影,仿佛天上的仙子云端起舞,水袖挥舞,在她周围扬起了一片粉红纱幔,柳腰折转,举手投足之间满是奔放的美、撩人的风情。这时,雾中传来歌女们柔婉的歌声,伴着清新宛转的乐声,缥缈虚幻,若有若有。

“灼灼荷花瑞,亭亭出水中。一茎孤引绿,双影共分红。”

随着那歌声,一缕箫音不知从何处飘来,清丽的箫音不似人间所有,而在高台之上,轻烟渐渐散去,露出了湖中高台的真貌,那在台上随着箫音歌声飞舞的身影吸引了所有人的目光。那快得令人眼花缭乱的繁杂舞步,由她踩着却是那么轻盈,似乎婀娜的娇躯没有丝毫重量。不盈一握的足尖在锦毡上轻跃回旋,她的舞姿宛若凌波仙子,又好象迎风摇拽的荷花一般出尘。此时别的笙管乐声皆已消散,只余一缕箫声在湖上若隐若现,箫音舞姿融为一体,不可分割。正在众人目眩神迷的时候,轻烟薄雾再次涌起,漫过高台,掩去荷叶罗裙。

“色夺歌人脸,香乱舞衣风。名莲自可念,况复两心同。”(注2)雾中的歌声越发旖旎,台上的舞姿也越发飘逸。白雾再次笼罩了高台,歌声渐歇,众人眼看着那绚丽的舞姿在雾中渐渐隐去,都生出十分不舍的心情。直到什么都看不见之后,仍然极力瞩目,盼着再见到那样的仙姿。这一刻,花魁状元由谁获得再无悬念。

与此同时,岸边一辆马车之内,一个女子捏碎了手中的茶杯,冷酷的杀意从目中一闪而逝,这个女子艳妆华服,明艳动人,若是不认得她的人,必然不敢相信这样一个雍容华贵的贵妇人竟是月影轩的主事人。

同时,一艘轻舟之内,另外一个相貌斯文和善的华服中年人也是一声轻叹,他把玩着手中的酒杯,神色间有几分惆怅,在他旁边的青衣儒士低声道:“楼主,那宋逾也太忘恩负义了,这些年若无楼主照顾,只怕他早就骨肉化泥了,如今竟然相助柳如梦夺魁,楼主可要给他一个教训。”

中年人却是轻轻一叹,道:“这也不是坏事,我们和月影轩不论谁取胜,都必然占据压倒性的优势,这样一来反而会失去应有的平衡,柳如梦获胜对我们并没有什么不利。你也知道如今柳如梦和月影轩之间已经结下仇怨,而柳如梦虽然独立特行,可是秦淮河的青楼女子,有几个没有受过她的照顾恩惠,这次月影轩急功近利,竟然仗着权势逼迫于她,现在不知有多少人暗自怀恨,不过是畏惧他们的后台,敢怒而不敢言罢了。这次柳如梦取得花魁状元的地位,那些分散的青楼画舫必然隐隐以她为首,处于中立地位,我们和月影轩两强相争,本已渐渐处在弱势,如今柳如梦必然暗助我们一臂之力,这对我们只有好处。至于宋逾么,虽然他这次有些过分,可是却不能伤害他,陈兄托我留意他,他的生死我们不能擅自决定。”

那青衣儒士知道楼主所说的“陈兄”十分重要,那人即是楼主故交,当初楼主筹建万花楼的时候,也得了那人倾力相助,在财力和人力上都得了不少支持,才有今日的局面,所以只是苦笑一声,这次他准备让碧烟夺得花魁状元,为此费尽心力令碧烟习得早已失传的胡旋舞,想不到却是这样的结局。这时,一个绸衫汉子掀帘走入舱中,在万楼主身边说了几句话。万楼主面上露出了玩味的笑容,道:“看来宋逾有麻烦了。”

当柳如梦终于夺得花魁状元之后,宋逾的眼神恢复了冰一样的清冷,寻个机会离开了画舫,乘着小舟自僻静处上岸,他可不会认为万花楼或者月影轩会善罢甘休,虽然碍着柳如梦已经夺得状元之位,他们不便对柳如梦出手,可是自己这个“帮凶”却定已经成了他们的眼中钉。月影轩一向以飞扬跋扈闻名,手段也相当的狠辣,这次自己坏了他们的好事,必然不肯善罢甘休,至于万花楼么,宋逾眼中闪过一丝愧疚,他在建业穷困潦倒之际,万楼主屡次施以援手,这样的恩情他还没有还报,若是万楼主派人前来问罪,他真不知该怎么应对。不过他想到的首先是不要牵连到柳如梦身上,所以特意离开画舫,也就是想给对方一个出手的机会,这种事情只要应付得当,应该不会造成太大的麻烦。

当宋逾走到人烟稀少的地方之后,果然觉察到身后有人跟踪,而且跟踪之人似乎无意隐瞒行踪,宋逾淡淡一笑,更是着意向隐蔽之处走去,转过一个弯,他在林中小道上停住身形,等待身后跟踪过来的人,他轻轻把玩着手中折扇,想着要不要一举杀了跟踪之人还是留下他们的性命,免得和月影轩生出不解之仇。

轻微的脚步声即将接近宋逾选定的战场,他目中闪过冰冷的杀机,轻摇折扇,那个身影终于出现在他眼前,宋逾手中的折扇突然停住了,他怔怔地望着那个面容阴冷的中年人,一句话也说不出来。

那个中年人微微一笑,道:“逾轮,不认得我了么?”

宋逾回过神来,举目四顾,只见身后多了几个熟悉的身影,这些人都是他昔日的同僚,其中更有一两个是他的下属,如今他们都正处在一生中最颠峰的时刻,和两年来堕落沉沦的自己不同,他们身上的气势沉凝而自信。他轻叹一声,道:“不知道陈爷突然来寻逾轮,可是有什么吩咐?”他没有提及自己已经退出秘营之事,若是那有用处,不说也无妨,若是没有用处,他也不想给任何人嘲讽自己的机会,尤其是当着旧日同伴的面。

陈稹看着逾轮平静的神情,道:“两年前你欲离开秘营回南楚的时候,我曾向公子提出你知道的太多了,应该将你灭口,或者将你拘束在我们可以控制的地方,可是公子却没有同意,不过李爷暗中下了命令,你若是有不妥之处,准许我便宜行事。”

逾轮没有丝毫意外的神情,抬起头道:“我知道,虽然当初有十年之约,可是公子能够允许我离开,更允许我自由自在地回到建业,临行更赠以重金,让原本已将多年积蓄挥霍一空的我不至于寸步难行,逾轮至今感激涕零,我也没有想到公子会如此宽宏大量,不过我知道公子素来谨慎,所以我知道身边一定会有人监视。”

陈稹叹息道:“你既然知道,又何必要说出来,如果你不知道身边有人监视,我还可以对你宽容一些。”

逾轮眼中闪过嘲讽的神色,道:“对着陈爷和昔日的兄弟,我没有必要掩饰什么,我若是想不到身边会有人监视,恐怕才会让陈爷瞧不起吧?”

陈稹道:“半年前渠黄来看你,知道你境况如此艰辛,虽然恼你不自爱,却也为你担忧,回去之后他便提出将一些任务交给你,这件事情我想来也没有什么不好,至少可以保证你在我们的控制之下。不过三月之前那个任务本来不该由你这种已经脱离秘营的人来做,可是渠黄替你力争,我也就答应了,毕竟你本来已经有了很好的机会。这个任务并不是我们迫你的,对不对?”

逾轮黯然道:“不错,这个任务我知道它的重要,也知道它的危险,之所以肯接手是因为事成之后,想必身边就不会再有你们的人监视了。”

陈稹道:“既然你接下了这个任务,就不应该因为私事坏了大计,可是你为了一个柳如梦居然和月影轩为敌,你难道不知道月影轩是谁的势力,因为今日之事,你可能失败,也可能被迫中途脱离,无论如何,都会影响到公子的大计。公子的规矩你应该清楚,因为私情而害大计,罪不容赦。”

逾轮额头渗出冷汗,他不是没有想到其中的危险,可是为了柳如梦他还是冒了险,他也想过事后补救的难度,也想过失败之后的下场,可是这些在柳如梦的倩影面前都化为乌有。他低声道:“逾轮既然身犯不赦之罪,任凭陈爷处置就是,只是我想不到陈爷会这样快就知道此事?”

陈稹冷冷道:“我本是为了别的事情而来,想不到却在这里见到你的手段,将一个无依无靠的柳如梦捧上花魁之位,也难为你的本事,只是如今我只能取你性命,现在建业有很多人知道月影轩对付柳如梦之事,你不是还说给了尚承业听么,如果你死了,尚承业想必会以为是月影轩下手,这也是不错的结果。”

逾轮冷冷一笑,道:“陈爷何必强词夺理,秘营何时会牺牲自己人成就大事,不如说你早就有心杀我吧。”

此言一出,四周将逾轮围住的众人都是面色微变,目光轻轻瞥向陈稹。陈稹却是神色不动,道:“第一,你已经不是秘营之人,牺牲你也无妨碍,第二,我从不否认有杀你之心,只是你不该让我抓到机会。逾轮,你若现在肯回归秘营,我便放过你,你答应么?”

逾轮抬起头,面色越发冰寒,一个青年低声道:“四哥,你何必如此固执,回到营中有什么不好,你若不想再过这样的日子,只需提出来,便可到大雍繁华之地安居,若是想要荣华富贵,也有进身之阶,都好过你在建业沦落。”

逾轮轻轻摇头,道:“我不想和兄弟自相残杀,我一个人也不是你们的对手,所以陈爷可以动手了,我做出的决定绝对不会改变。”说罢,他丢下折扇,负手而立,身姿孤傲如青松,等着陈稹下令,他不是真的不想反抗,可是他真的不能对昔日同生共死的兄弟出手,而且,他也知道,早在他被陈稹震慑之时,围上来的诸位兄弟已经将他的所有生路都封住了,既然一定要死,何必还要拖他们下水呢?死就死吧,他对生命早已不再在乎。只是为什么这一刻,眼前却浮现出一个朦胧的倩影呢?

看着神色淡淡,摆明了不会反抗的逾轮,陈稹眼中闪过一丝悲伤,这个青年也曾是他训练出来的精英,可是自己却要亲手将他处死,神色渐渐恢复冷酷,这是一定要做的事情,他早已发觉逾轮望着江哲的目光有的时候会带着怨恨,也曾对江哲提过,只是江哲却是但笑不语,但是如今,他既然把握了机会,就绝不会放过这个隐患,即使他的死亡会带来难以估量的损失也是如此。想及此处,陈稹淡然道:“杀!”

那些青年都没有丝毫犹豫,虽然面前是他们生死与共的同伴,可是上命绝不可违,这是秘营的铁律。

就在千钧一发之刻,有人高声喝道:“住手!”

所有的人都停了手,那是白义的声音,在赤骥、盗骊相继离开秘营之后,白义已经是秘营之首,虽然陈稹是他们的师傅,也是他们的统领,可是对他们来说,白义才是他们的首领,更何况他们本心也不想杀逾轮。

陈稹一皱眉,但是奇异的,他心中也有如释重负的感觉,望向声音来处,一个风尘仆仆的青年站在那里。他冷冷道:“白义,这件事情应是由我作主。”

白义上前施礼道:“陈爷,属下怎敢违背谕令,不过这是公子的手令。”说着,他递上一封书信,陈稹看后轻轻一叹,双手一搓,书信化成飞灰,望了一眼逾轮,他淡淡道:“你好自为之吧,公子对你太宽宏了。”说罢转身而去,那些青年都对逾轮施以抱歉的眼神,然后匆匆跟着陈稹离去。

纵然早已无视生死,但是死里逃生之后的感觉仍然让逾轮觉得身躯有些发软,看向白义朴实敦厚的面容,他微微苦笑,索性坐倒在地,道:“白义,你又何必如此呢,这下你可得罪了陈爷了,何况你救得我了一次,救不了我第二次,从前两国休战,我留在建业还是无所谓的,如今两国开战,秘营一定会有很多行动,留下我这么一个人在建业,就是公子也必然不会放心的。”

白衣轻叹道:“你既然知道情势,为何定要留在建业,你若不想再过杀戮阴谋的日子,只需有意,不论是赤骥、盗骊、绿耳还是骅骝那里你都可以去的,就是都不想去,东海也可隐居,你却偏要留在建业,也难怪陈爷猜疑,其实我至今不相信公子竟会放过你。你以为渠黄为什么要设法让你参与这个任务,只是想不到,陈爷终究不肯放过你的。”

逾轮默然,良久才道:“是你去信给公子取得手令的么?”

白义淡淡一笑,渠黄在三月前力排众议举荐逾轮执行这个任务的时候,那时他就已料到这个举动难以阻止陈稹的杀机,所以暗中传书寒园求得手令,两日前他知道陈稹将亲至建业,便已想到今日之局,所以日夜兼程前来阻拦。不过他没有多说什么,只是道:“逾轮,公子对你已经仁至义尽,我希望你能够好好想清楚。”

逾轮沉默不语,可是眼中闪过坚毅的神色,他早已尽尝离开秘营之后的艰难处境,也知道有更宽阔的道路可走,可是自从柳飘香之仇报复之后,他就已经没有留在秘营的理由,而这世上除了建业之后,还有什么地方可以让他留恋呢?纵然是死,他也不想屈服。只是他心中也有疑问,公子对自己这般宽容,只是为了昔日主仆师生之情么?莫非公子竟然知道自己的身份?这不可能的,自己从未和公子见过面,只是自己暗中见过他的容貌罢了,若非如此,怎会知道那位令公子矢志复仇的柳夫人就是飘香姑娘。

白义看出已经无法说服逾轮,只得摇头道:“罢了,人各有志,你小心行事吧,我不知你怎会为柳如梦出头,可是你要小心些,万楼主是陈爷旧识,你在建业的行踪就是他传书给陈爷的,而且月影轩的底子你心里也有数,这次我们不能出面助你,你要小心了。尚承业那边你也要加快动作,荆家的处置现在正是时候。”

逾轮轻轻一叹,果然是万楼主,这两年万楼主对他颇为照顾,他心中便有猜疑,所以方才才会这般肯定陈稹在自己身边安排了探子,果然如同所料,不过这样一来,万楼主这次应该不会和他为难,他只需对付月影轩即可,想来倒也放心许多。

白义转过身去,道:“月影轩派来跟踪你的人,陈爷已经令人解决了,这件事情万楼主会认下来,你不必担心,逾轮,你好自为之吧。”他欲言又止,终于没有再说下去,这一次的相助已经是令他费尽心思,下次陈稹若再要动手,恐怕他也无能为力了。轻叹一声,他的身影消失在密林之中。逾轮没有作声,只是望着他的背影出神,眼中闪过泪影,白义不忘十年手足之情,那么自己呢,当真可以忘却十年恩义?

————————————————————

注1:隋杜公瞻《咏同心芙蓉》

注2:唐李白《古风》其二十六

\chapter{第二十一章 一夜鱼龙舞}

玄武湖上的花魁大赛虽然鼎盛,有兴趣的却多半是官宦子弟,富商豪门,但是当夜的灯会,却是老少咸宜,这一夜,不论是达官显贵还是平民百姓,都是锦衣夜行,普天同庆。建业城内流光幻彩,各色各样的绮丽花灯争奇斗艳,灯光夜色交相辉映,街道上更是熙熙攘攘,车水马龙。富贵人家更是费尽心思夸显华采,竞奢赛富,金银、琉璃、珠玉装饰成宝光四射的华贵灯盏,更有许多人家在门前高台,令人在台上表演百艺杂耍,精彩纷呈,引来人潮如涌,还有人家在门前摆了彩棚,里面悬出灯谜,摆了锦缎金银作为彩金,引得无数男女皱眉苦思。

在人群之中,陆云和石绣携着手走在街上,两人今日在朝堂上受了封赏,都封了六品的校尉军职,虽然现在只是虚职,不可能让他们真的领兵,但是这毕竟是难得的荣耀,两人自然不知道这封赏不过是朝廷的敷衍,也是弥补陆灿应得的封赏的补偿罢了,自然欢天喜地,所以相约出来看灯。两人都是天不怕地不怕的性子,再加上武艺高强,所以也没有带上家将,就偷溜出镇远公府。石绣初次来到建业,对这里的街道不熟悉,陆云担心她迷了路,路上的人又太多,所以便一直牵着她的手,不让她走失。

走了一阵,石绣正在目不暇接的时候,耳边突然传来几个男子唉声叹气的谈话声,却是说起有一富户在门前摆下擂台,据说彩头是一盏八宝琉璃灯,若是有人能够箭射金钱,便将此灯相赠,据说若是年貌相当,还会将女儿许配给夺擂之人。这些男子都是会些弓箭,所以上去试试运气。石绣对于招亲之事自然不感兴趣,可是一听到射箭夺灯,便竖起了耳朵,听了片刻,她便对陆云道:“云弟,我们去试试吧,猜谜我们又不会。”陆云听了也是颇感兴趣,便带着石绣向那些人所说的方向走去。走了不到一拄香的时间,果然看到了箭擂。

那是一家高墙深户的豪门,门前辟出一块空地,距离大门百步之外树着一根旗杆,旗杆上面挂着一盏红灯,灯下悬着一枚金钱,正随风飘荡,在大门旁边搭着彩棚,用纱幔隔成内外两间,外间是一个气度不凡的中年华服人主持,棚内放着一张长桌,桌上放着雕弓翎箭。至于作为彩头的八宝琉璃灯正悬在大门上,那是一盏八角宫灯,宫灯是由六十四片琉璃晶片构成的,串连其中的都是金丝银线,更有明珠碧玉妆饰,红烛摇曳,越发显得晶莹剔透。只是宝灯顶部的那一枚鸽卵大小的璀璨明珠,就已经价值连城,怪不得有许多人在旁边摩拳擦掌。虽然南楚崇文轻武,但是射箭也是读书人的六艺之一,倒也有很多人敢于上前试射,不过试射需要先拿出十两银子,这就让许多人止步了。

陆云揣测了一下,那旗杆是特意准备的,足有十丈高,那枚金钱轻薄小巧,只用红色丝线悬在灯下,随着高处的寒风飘来荡去。若是自下向上射箭,这样的距离,这样的靶子,果然是十分艰难,就是自己也不敢保证可以射中金钱,不过彩棚上面的告示说明三箭有一箭射中金钱即可,那么自己倒是有七八分把握。

这时,石绣已经双眼发亮地道:“云弟,你带了银子没有?”

陆云正要劝石绣不要去出风头,但是四目相对,石绣那双明眸之中的粲然光芒,却让陆云心中一软,道:“你先试一下,如果不成我再试一次,一定可以夺得宫灯的。”石绣白了他一眼道:“我若射不中,你就能射中么?”陆云顿时语塞,两人箭术本在伯仲之间,石绣这样说并没有差错。于是他苦笑一下,将一块银两塞到石绣手中。

石绣接过银两,走向彩棚,围观众人都是眼睛一亮,石绣身穿白色衣衫,相貌俊秀,眉梢眼角都带着自信,这般英姿年少,若非是她年纪看上去还不大,只怕那些难得出门的名门闺秀也会心动心慌。她上前取了雕弓和三支羽箭,丢下银两,走到白线之后,眯缝着眼睛瞧了一下那随风起舞的金钱,弯弓如满月,凝神搭箭。围观众人都是屏气观瞧,想看看着俊秀少年是否能够箭射金钱,过了片刻,石绣仍然没有发箭,人群中有些人开始说笑,开始松懈,都觉得这少年不过是虚张声势罢了。就在这时,弓弦一响,一支羽箭电闪而没,一声低微的轻响,羽箭已经穿过金钱方眼,众人还没有反应过来,第二支羽箭已经划过长空,红色丝线从中断绝,金钱向地上坠落,就在这时,第三支羽箭破空而来,正将那枚金钱穿在箭矢之上,余势未歇,贯入其后的旗杆之上。

周围一片静寂,在这上元之夜,这样的寂静显得分外古怪,石绣微微一笑,收起弓箭,微红的面容上露出得意的微笑,四周惊天彻地的叫好声响起。石绣对着众人施了一个罗圈揖,转身看向那正捻着胡须发呆的中年人,笑道:“那盏八宝琉璃灯应该归我了吧?”

那个中年人心中苦涩难言,正在他犹豫的时候,身后帘幕之中传来银铃一般的语声道:“高总管,既然这位公子箭射金钱,自然该将宫灯相赠。”

石绣微微一愣,虽然早已看到帘幕后影影绰绰有数个身影,却想不到发话的竟是一个女子,想到方才听来的闲言闲语,这家设下箭擂,也有招亲的意思,想必帘后之人就是这家的小姐,不由觉得有些尴尬。她虽然好穿男装,也不将自己当成女子看待,可是她毕竟是个正常的少女。忍不住回头望向陆云,陆云也正在为石绣的箭术暗暗喝彩,这些日子没有少切磋,不过今日才看到石绣的真实本领。看到石绣求助的目光,他上前笑道:“既然主人都这样说了,这位总管怎么还不去取灯?”

陆云一站到石绣身边,围观众人的目光又都是一亮,陆云虽然不如石绣俊美,可是身世经历再加上父亲的熏陶,让他气度卓然,同样的一身白衣更是衬得他英武不群,陆石二人站在一起,相互映衬,越发显得两人的不凡。

那中年人尴尬的一笑,吩咐家人去取宫灯,正要上前搭话,帘幕一挑,一个十五、六岁的锦衣少女走了出来,她穿着轻裘锦靴,衣衫华贵,娇艳明媚如春花,目光流转处如春波含情,令得众人都是深吸了一口长气。

她上前对着陆、石二人轻施一礼道:“小女子纪灵湘,见过两位公子,不知道两位如何称呼,我这宫灯虽然要送,却也要送给清白人家,若是落入歹人之手,岂不是明珠投暗么?”她这一番话说的极快,却又字字清晰,让人听来只觉得如同珠落玉盘。就是石绣身为女子,听了也是心中一动,纵然觉得她有些强词夺理,也不愿和她争辩。

陆云却是神色如常地道:“小姐悬灯之时可没有说过还要问身家,既然我们已经射下金钱,此灯就该归我们所有,若是小姐想违约,只怕诸位父老乡亲也不答应。”此言一出,那些围观之人纵然被少女丽色所迷,却也议论纷纷,还有人轻薄地道:“这位小姐,说话不能不算数,你问人家身份,不是看中了这位小公子吧?”

锦衣少女脸色一变,她相貌美丽,又有颇富权势的后台,所以一向都是要风得风,要雨得雨,从无人对她无礼,今日陆云抢白了他,又引得无赖嘲弄,不免心中大怒,眼中闪过一道寒芒杀气。

其实陆云虽然年少,又是血气方刚,怎会对美色毫无感觉,可是他却结识过昭华郡主江柔蓝、石绣这样的少女,所以对于纪灵湘,他心中丝毫没有生出波澜。若论相貌,江柔蓝和纪灵湘不过在伯仲之间,可是若论气度,却是天壤之别,柔蓝身上,既有着温柔善良的天性,也有着皇室中人睥睨天下的骄傲,那种骄傲不是形之于外的表象,而是深入骨髓的自信自尊,纵然是娇柔如水,水面下也是暗藏着波涛汹涌,那便是江柔蓝。虽然陆云对柔蓝尚未真正了解,可是几次相见,就已经让他心中映下了柔蓝的倩影,虽然如明月一般可望而不可及,也难以摒去倾慕敬爱之心。石绣虽然相貌不如纪灵湘,可是她豪迈英勇,全无女子软弱拘泥之态,却是另有一种傲骨风姿,何况并肩作战多日,两人早已不知不觉间有了血脉相连一般的情感。相较之下,纪灵湘虽然美丽娇艳,却不免有些骄纵倨傲,气质不如柔蓝,情义不如石绣,若是寻常少年或许会为她的美色目眩神迷,但在陆云看来却是如同泥塑木偶,全无生机可言。

这时,那总管已经捧了宫灯过来,那宫灯十分精巧,取出火烛之后,可以轻易的折叠起来,那总管用红色锦盒装了,双手递给石绣。石绣接过之后,满心欢喜地向外走去,陆云跟在她后面也是笑容满面,两个人都没有对那锦衣少女多看一眼,径自说着话向外走去。

围观众人见宫灯已经被人夺走,便都各自散去,只留下那锦衣少女仍然银牙紧咬地站在彩棚之前,她脸色变得青白,在此设下箭擂,本是为了吸引陆云前来,这是早已制定的计划,在发觉陆云出府的一刻开始启动,为此特意令人用言辞吸引陆云和石玉锦到来。谁知人虽然来了,下场夺灯的却是石玉锦。这锦衣少女并不知道石绣乃是女子之身,只知道她是和陆云齐名的石玉锦,其实在她看来,风度翩翩的石玉锦更符合她的心意,只是师父的命令是让自己借着箭擂夺灯接近陆云,所幸陆云才貌也不算差。可是令她万万想不到的是,陆云对她视若无睹,这样的屈辱令她将陆云恨之入骨,也暗暗担忧师父会否责怪自己。

见她神情黯然,那高总管低声道:“三小姐不必担忧,此事纵然不成,首座也不会责怪你。”

纪灵湘轻轻一叹,道:“如果大师姐那边能够顺利一些,能够夺得花魁状元,师父欣喜之下,或者不会责怪我,如今师父正在十分恼恨,只怕今次不好过了。”

那中年人低声道:“三小姐放心,首座已经下令除去那坏了我们大事之人,柳如梦不过是一个无依无靠的弱质女子,迟早会落入我们掌握的。”

纪灵湘没有作声,她虽然年轻,却并不幼稚,也不认为这件事情会这般容易解决,何况不论结果如何和他并没有什么关系,她只是担忧自己如何能够渡过眼前这一关。

“法轮天上转,梵声天上来。灯树千光照,花焰七枝开。月影疑流水,春风含夜梅。燔动黄金地,钟发琉璃台。(注1)”,明月楼高,灯火辉煌,下面就是车水马龙的御街,从半开半阖的窗内,传出动人的歌声,纵然是在这样喧嚣的夜晚,那歌声也是这般清晰可闻。

在楼上雅室之内,一个云鬓高耸,身披轻纱的美丽少女抚琴低唱,歌声如梦如幻。在室内一角,两个男子微笑聆听,他们身边各有两个娇艳少女相陪。一曲终了,一个中年男子拍掌道:“好歌,好词,宋兄弟果然好文采,怪不得助得柳姑娘夺得花魁之位,只是恐怕却得罪了别人?”另一个神色清冷的青年醉眼朦胧地道:“尚兄多虑了,若是真的有人为此小事而怪罪我,最多我避开一段时间,想来事过境迁,应该不会有多少人还记得此事。倒是尚兄今日的心情似乎很好,莫不是有心看我的笑话吧?”

那中年男子哈哈一笑,将怀中的美女推开,对那青年说道:“宋兄弟,多亏了你的计策,近日家父召集幕僚议事的时候,对我常有勉励之辞,凭你的这些功劳,你放心,别的不敢说,月影轩那边,我定能劝服她们不要和你为难。”

逾轮闻言淡淡一笑,道:“其实令尊也是望子成龙,所以昔日才对尚兄多有鞭策,尚兄是执掌朝纲的相阁之才,为相者若能采纳良言,临机决断,就已经是良相,我想令尊就是觉得尚兄能够接受小弟愚见,且能相机应用,所以才对兄台多有勉励吧。而且陆大将军毕竟是南楚的擎天玉柱,令尊不过是想对其稍加约束,免得他走上歧途罢了,我那点浅见,恐怕还不曾看在尚相眼里。”

尚承业神色飘飘然,得意地道:“那是当然,家父可还不会将那陆门竖子放在眼里,而且此人和大雍重臣,我南楚的叛臣江哲多有勾结,若非念在此人尚得军心,只怕家父早就将其治罪了。”

逾轮心中一动,故意道:“噢,尚兄是说那位娶了大雍公主的楚侯爷么,虽然宋某也觉得此人无甚气节,可是他能够有今天的成就,想必也不是寻常之辈,听闻此人曾助雍帝夺嫡,又助齐王平汉,这样的本领才能,天下罕见。陆大将军能够以一己之力退去雍军三路大军,这样的本事才能,也是极不寻常。怪不得人人都说,他们两人曾有师徒之谊,不过陆大将军身为南楚大将,理应大义灭亲才是。”

尚承业拊掌道:“就是啊,那江哲辜负君恩,为了荣华富贵叛国投敌,又臣娶君妻,当真是大逆不道。陆灿虽然在他门下受业,可是陆氏乃是南楚世家,理应大义灭亲才是,可是陆灿不仅对江哲多方维护,甚至还让自己的儿子前去长安,颇有通敌之嫌,若非是碍着他这次的功劳,这件事情家父绝不会放过。还有那嘉兴荆氏,乃是江哲母家,父亲有意除去荆家,陆灿也是从中作梗,当真岂有此理。”

逾轮笑道:“这想必是相爷太心急了,陆大将军素以赏罚严明闻世,无端灭人满门他定然不会同意,不过尚兄,荆氏虽然和江哲已经绝了往来,可是毕竟也是江侯的母家,难道相爷不畏得罪了此人么?”

尚承业鄙夷地道:“若非是看在陆大将军面上,家父早就对荆氏下手了,那江哲虽然威名赫赫,可是多半是大雍皇室为了长乐公主的面子吹嘘的吧,当年此人家父也曾见过,若是果然有才,怎会看不出来,此人或者有些阴谋诡计,当初夺嫡之事可能确是出力不小,可是若说他能够相助李显灭掉北汉,我可是不相信,他一个手无缚鸡之力的文人能做什么呢,恐怕只是替雍帝监视一下齐王李显吧。”

听到此处,逾轮已经知道南楚上层对江哲果然是不甚瞧得起,他也猜得到,这或许是尚维钧等人通过贬低敌人,来维持士气的手段,但是只看连尚承业也不甚了然江哲的才能本领,就知道尚维钧等人也未必多瞧得起江哲,他昔年受教于江哲,自然知道这等轻敌之念的害处,不过他自然不会想要扭转尚承业的观感,只是笑道:“既是如此,若是相爷令人缓缓为之,想来定有成效,荆氏也是世家,必然有不肖子弟,若是发现一人有过便处置一人,陆大将军纵然有意维护,难道还能为了一两个人和相爷为难么?”

尚承业眼睛一亮,思忖起这个方法的可行性,想了许久,露出得意的笑容,想来用这种手法不仅可以满足父亲的心意,而且还可暗暗打击陆氏,父亲若是知道,一定会十分满意。

逾轮见状已经知道尚承业已经入彀,便故意转移话题,他对音律诗词都十分精通,说起一些奇闻轶事也是头头是道,尚承业也很快就忘记了方才的插曲,只是专心玩乐起来。

夜深人阑,就是外面的街道上人烟也渐渐散去,尚承业早已不胜酒力,扶了佳人入内室寻欢去了,逾轮却是把酒站在窗前,望着西沉的明月,神情黯淡,夜深人静之时,他总是难以排遣心中的寂寞,所以平日他往往都是纵情声色直到天明,可是今夜却不同,他知道暗处有人在窥伺自己,而且那些人已经开始驱赶过往行人,免得自己有机会混入人群逃走了,而他也就是要给她们一个机会。随手从腰间取出一粒醒酒药服下,暗暗运功数次,觉得精力心神已经稳定下来。他轻轻一按窗棂,身躯如同飞雁一般落到街道上,如同落花坠地,轻悄无声。

暗处传来轻咦之声,不多时,茫茫晨雾之中,显出一个青衣女子的身影,那女子面蒙轻纱,虽然只是缓缓之行,却有一种高贵雍容的气质,在她身后两个劲装侍女紧紧跟随,这两个女郎都没有遮掩面容,露出如花似玉的娇艳面容,一看便知道不会超过二十岁,可是她们一身凌人的剑气却让人不敢相信这两人未到二十芳华。

逾轮向那三个女子望去,俊逸的面容上露出玩世不恭的笑容,道:“原来月影轩还有这样美丽的女剑客,宋某当真是佩服,却不知几位姑娘身价几何?”

那两个女郎面上都露出凛然的杀气,那站在中间的女子虽然面容隐在轻纱之下,可是眼中也是透出冰寒的杀机,她冷冷道:“宋逾,你既然喜欢油嘴滑舌,那么本座若是杀你也不算滥杀无辜了。”

宋逾微微一笑,正要说话,却见那青衣女子手一挥,那两个女郎已经仗剑扑上,剑光闪闪,透着无穷的杀机,这两个少女剑法出众,而且配合的十分默契,一时之间宋逾有些手忙脚乱。那两个少女精神大振,更是连出杀手,迫得宋逾连连后退。那青衣女子轻轻点头,似乎颇为满意两个侍女的剑法。就在这时,局势突变,宋逾一个踉跄,向后倒去,那两个少女同时挥剑下斩,就在千钧一发之际,宋逾的身形仿佛游鱼一般,从两人剑下滑了出去,同时他手中折扇轻指,两道乌光电闪同时没入两个少女的咽喉,两个少女娇躯同时一颤,向下仆倒,宋逾则已经若无其事的站在一旁。那青衣女子神情一震,目光在两个少女身上一转,冷冷道:“好毒辣的暗器,含笑杀人,阁下好狠毒的心肠。”

宋逾面上露出淡淡的傲气,冷笑道:“宋某杀人无数,从无怜香惜玉之心,这两个丫头就是前车之鉴,姑娘可还要和宋某一战?”

那青衣女子冷冷道:“阁下好狂妄,本座成名之时,你恐怕还没有出师呢。看剑。”声音未歇,一柄利剑已经指到了宋逾胸前,宋逾的身躯随剑飞退,两人之间仿佛是配合了前次万次一般,人剑竟是没有一丝空隙。剑势将尽之时,宋逾手中的折扇突出,这一招妙到峰巅,那青衣女子措手不及,回剑阻拦,宋逾趁势攻去,两人在轻雾中苦战起来。青衣女子剑法神妙,映着西沉的月光,剑光如雪,耀眼的流光飞虹将两人的身形都笼罩在其中。而宋逾的身姿轻盈,在剑光中飞舞不休,手中的折扇忽开忽阖,每一个动作都是那样的清晰流畅,潇洒飘逸,不带一分杀气,可是只要那青衣女子稍露破绽,他的招式就会变得狠毒无情,无声无息地穿过青衣女子的剑网,直取要害,迫得她回剑相护。拼了百十招,两人仍是旗鼓相当,那青衣女子眼中杀机越浓,她早在十余年前就已经扬名天下,想不到今日竟会被一个小自己七八岁的青年迫成平手。

正在这时,另一侧的高楼之上,传来一声轻喝道:“住手。”然后一道紫影飞掠下来,正将青衣女子和宋逾两人分开,两人凝神一看,来人却是一个紫衣老者,他相貌清峻,神情威严,他虽然没有带着兵器,可是一双手白皙如玉,十分刺眼。宋逾脑海中灵光一现,已经想到这老者的身份,这人正是尚维钧亲聘的高手绵里藏针欧元宁,据说此人武功深不可测,据说已经接近先天之境。他是尚维钧的亲信,想不到竟会出现在此地,想到此人的身份,宋逾做出恭恭敬敬的神态,一声也不敢出。那青衣女子秀眉微蹙,似乎有些难以决定。

那老者淡淡道:“谢姑娘,这人乃是尚公子挚友,相爷对其也颇有了解,大家都是为了相爷效力,何必自相残杀呢?你将我的意思告诉纪首座和燕首座,她们会明白的。”

那青衣女子终于长叹一声,收剑回鞘,裣衽一礼,然后转身离去,不多时,几个中年女子出现,将两个少女的尸体带走。那老者轻轻一叹,道:“卿本佳人,奈何作贼,想不到昔日名门弟子,今日沦落到这种地步,当真是可惜可叹。宋逾,老夫已经察知,你以无情公子之名,在南楚境内做下无数大案,有人称你是江南第一杀手,直到数年前才销声匿迹,想不到你竟会在建业隐居,你接近我家公子有何目的?”

宋逾心中毫不惊慌,面上却做出被揭穿身份的慌乱和杀意,他戒备地道:“欧前辈是要惩恶扬善么?宋某虽然是曾以杀人为业,如今已经是金盆洗手,至于和尚公子结交,却非有意。”他能够感觉到老者的目光紧紧盯在自己面上,若是自己稍露破绽,定会招致老者的雷霆一击。不过他所说没有一分虚假,他和尚承业的交往的确是无意之举,只不过如今被他利用完成任务罢了。至于杀手身份的泄露,本就是有心为之,这样正可解释他十余年来莫测的行踪。

果然那老者笑道:“老夫可不管这些闲事,只是觉得有些可惜,宋敏,你本是少年才子,可惜沦落成为杀手,如今改邪归正,也算是迷途知返,老夫已经查问过了,你和公子果然是无心结识,不过就算你是有心接近公子,求个进身之阶,也不算是什么错处,相爷对你颇为重视,已令人将你的案底抽去,从今之后不会有人发觉你就是无情公子,你就是想从正途得个功名也不是什么难事。”

宋逾面露古怪之色,似乎因为自己少年之事被老者查了出来,有些尴尬,也似是对尚维钧的恩情十分感激,他深深下拜道:“晚生汗颜,辜负了先严教诲,只是宋某浪迹天涯,早已没有功名之念,还请前辈向相爷转呈晚生心意。不过尚公子对晚生视如手足,所以晚生有心替公子尽些心力,若是相爷觉得不妥,晚生不再和尚公子见面就是。”

那老者目中神光一闪,继而变得柔和,淡淡道:“原来如此,你既已无心功名,老夫也不相强,不过你要安分守己才是,不可再这般出手无情,今次看在老夫面上,她们放手而去,若是知道你已经不在尚相庇护之下,你必然遭遇惨烈的报复。你和尚公子既然有缘相识,就好生把握吧,你要好自为之。”

宋逾闻言,心中冷笑,知道这老者是逼迫自己替尚氏效力,若是自己想要脱身离去,只怕就会遭遇杀身之祸,不过这种情况他早有预料,故意流露出惶恐神情,俯身一拜,道:“多谢前辈教诲,宋逾拜谢。”等他再次抬起头,紫衣老者已经杳无人影。宋逾微微一笑,但是一缕惆怅却又涌上心头,他接下任务,接近尚承业,通过此人影响尚维钧的决定,这个任务的危险不问可知,可是当初他是孑然一身,自然无所畏惧,可是如今他却有了牵挂,只望不要连累柳如梦才好。

宋逾怎也想不到,就在这时,一个雍容男子正透过珠帘看向他,直到宋逾的身影消失之后,那人才一声轻叹,对身后一个中年汉子道:“这么一个人在建业滞留,为什么我们没有发觉。”

那中年汉子诚惶诚恐地道:“首座,这也是没有办法的事情,我们辰堂在建业的势力被仪凰堂压制住了,自然消息不灵,若非是我们的探子发觉纪首座请了谢护法出手,还不知道这件事情呢?”

那雍容男子正是韦膺,他冷冷道:“这个宋逾气度不凡,心机深沉,只见他有本事帮着柳如梦夺得花魁之位,就知道此人才智过人,这样的人应该招揽才是,纪首座却要杀人泄愤,真是鼠目寸光。”

那中年男子不敢接口,只是沉默不语,韦膺冷笑道:“只可惜这人还是入了尚维钧掌中,我便只能将他当成敌人了,派人留意他,时时回报。”中年男子连声应诺,韦膺目中寒光连闪,他总觉得这青年会给自己带来很大的麻烦,可是若是出手杀他可能会触怒尚维钧,他还不想和尚氏翻脸,只能轻叹一声,道:“敌人已经蠢蠢欲动,这里却还只是钩心斗角,当真令人心寒,唉!”

————————————

注1:隋炀帝《元夕于通衢建灯夜升南楼》

\chapter{第二十二章 激宕波澜惊}

隆盛八年乙酉元月,雍帝密诏靖海侯姜某率东海水军南下,二月初八,东海水军大破定海军山。同日,南阳大营长孙冀困襄阳。

——《资治通鉴·雍纪四》

尚维钧满意地放下手中的案卷,这是嘉兴府的文书,刑部已经批复了斩立决,回文已经在路上了,只需数日时间,这文书就会到嘉兴。这本是一件极小的案子,不过是一个弃职私逃的官员被判了斩刑,原本用不着堂堂的丞相关注,可是尚维钧却相信陆灿一定会阻挠或者前来求情。他盘算着是坚决不允陆灿求情,杀了那荆长卿,还是给陆灿一个情面,让他多些让步,可是不论怎样,自己都是占了上风。承业孩儿果然越来越长进,这样的法子都想得出来,只是不知是否那宋逾的功劳。

正在他沉思之时,宁谦匆匆走进来禀报道:“相爷,大将军陆灿在外求见。”

尚维钧精神一振,道:“宁先生,陆灿神色如何?”

宁谦忧心忡忡地道:“他面色冷肃,虽然看不出心情变化,可是显然十分愤怒不满,相爷要小心行事。”

尚维钧挥手道:“不妨事,这次本相占了道理,他可是将弃职私逃的胡成在军前斩首的,我不过是要杀一个荆长卿,而且仔细追究起来,这人说不定是怎样逃生的呢,就是判他一个通敌之罪也不是不可以,本相不杀荆氏满门已经是十分宽容了。好了,你随本相亲迎大将军吧。”尚维钧起身向外走去,这次他可是礼数周到,绝对不给陆灿借题发挥的借口。

书房阶下,陆灿负手而立,他的神情冷峻,仿佛千年不化的寒冰。尚维钧心中泛起得意之情,前些日子被这后辈压下的气势重新回到他身上,他似笑非笑地降阶相迎,道:“不知道大将军来此有何贵干,可是军饷有什么差池,若是如此,本相必然责成兵部、户部的官员尽心竭力。”

陆灿目中闪过冰寒的光芒,他自然知道尚维钧的心意,只可惜自己却没有时间为了一个人和尚维钧牵扯不清了,他冷冷道:“尚相可知如今雍军已经入境了?”

尚维钧身子一震,脱口道:“怎么可能,雍军刚刚大败而归,怎么这么快就卷土重来?”

陆灿眼中闪过嘲讽的光芒,道:“一刻之前,陆某接到谍报,南阳大营的雍军已经再次兵临襄阳,这一次来势汹汹,不似佯攻,这还罢了,襄阳有容将军镇守,谅可无碍,可是另一道军报却言大雍水军已经攻下定海,余杭水营兵力不足,只能稳守钱塘水道,不让雍军深入内腹。若给大雍水军控制了杭州湾,则吴郡、越郡迟早不保,到时候有何种后果,相爷可明白了。”

尚维钧虽然不甚通军事,却也知道东南沿海的吴越二郡为南楚钱粮重地,若是被大雍水军侵掠,则南楚根基浮动,纵有江淮之险,也将被敌所制。想到此处,已经是面色青白,他艰难的问道:“为何雍军不攻宁海,却取定海。”

陆灿淡淡道:“宁海军山乃是长江入海的咽喉要地,若是此处有失,则泰州、扬州都会危急,若是雍军逆流而上,建业将遭兵燹,但也正因此故,宁海军山的水军不敢稍有懈怠,又占了地利人和,所以雍军不取宁海。而定海军山虽然蔽翼杭州湾,却是久无战事,军备疲敝,也难怪雍军舍难就易。”

陆灿语气虽淡,尚维钧仍然听出他话语中的冰寒,宁海、定海两处军山乃是南楚武帝设立,本是防御海寇的重要军镇,一向由建业直辖,近年来吴越并无战事,尚维钧嫌两处军山耗费糜重,几次消减军费,虽然陆灿曾经多次进谏,他仍然不为所动。只是两处军山却非是平等而待,宁海军山主将赵群乃是王族,所以尚维钧只是不闻不问罢了,而定海军山所得的粮饷几乎已经是仅够温饱,就连整修舰艇也无法进行。想不到如今雍军竟然攻破定海军山,岂不让尚维钧面目无光,若非如今是他自己秉政,这样的罪责足以让他丢官弃职了。他犹豫片刻,道:“雍军攻定海,这也是始料未及,大将军此来,定有见教,不知应如何对敌?”

陆灿冷然道:“定海失守,杭州湾已经成了不设防的所在,唯今之际,需要严守余杭,避免大雍水军入钱塘,否则吴越必然不保,其次,会稽、余姚、镇海、嘉兴、海宁、平湖都需要分兵防守,这一次入侵的雍军定是东海水军,他们本就是海寇出身,海战上无人可敌,我军只能稳守沿海,不许雍军侵入,才能有些胜算,只是这样一来,吴越两郡将耗费粮饷兵力无数,请相爷下令减免两地税收,令各郡组织义军守土抗敌,只有如此,才能减少我军在吴郡、越郡的压力。”

尚维钧听得一阵心痛,吴越之地,富庶丰裕,就是减少一厘的税收,也将是令人心痛的损失,但是如今这般危急,也只能如此。若不组建义军,靠着那些软弱无能的守军,吴郡、越郡必然不保,若是不肯降低税收,那些百姓又哪有精力整军经武呢?想来想去,吴越之地的官员多半是世家子弟,能干的极少,还需将他们调回来,若是他们失城失地,或者死于兵燹,自己也要麻烦连连。想到此处,他只得道:“一切由大将军决定,本相这就将余杭水营和定海军山的军权交给大将军掌管。”虽然局势如此,尚维钧还是刻意留下了宁海军山,现在宁海军山尚安然无恙,他自然不愿将这样一支水军交给陆灿。陆灿明白他的心意,只是冷冷一笑,便告辞离去,留下愧悔交加的尚维钧在那里不安徘徊。

越郡杭州湾入海之处,有岱山、定海、普陀诸岛,武帝赵涉于定海置县,设立军山,总辖岱山、普陀水营,定海军山势力最大的时候,平湖、海宁、余姚、镇海都曾经在其管辖之下,直到尚维钧秉政之后,因为海疆无事,对定海军山屡次消减粮饷,以致水营糜烂,士卒疲敝,才会被东海水军一举攻下岱山、定海,普陀虽然尚且在南楚水军之手,却已经是岌岌可危。

我站在高崖之上,遥望天际,穿过眼前这片碧海,就是越郡镇海,而从此地向西北渡海,就是吴郡平湖,平湖之西就是海宁,而从海宁登陆,快马加鞭,不需一日,就可到达嘉兴,那里曾是我出生之地,也是娘亲埋骨之所,想起当初父亲在江夏病故,我差点要卖身葬父,根本无力将父亲灵柩送到嘉兴和母亲合葬。后来我中了状元,可是和荆氏并未和解,也就没有移灵,毕竟母亲的墓地也是荆氏所有,父亲是不会想寄人篱下的。想到母亲孤坟凄凉,我不免心中怅然,轻轻长叹。

小顺子上前道:“公子,高处风大,还是回去吧。”

我淡淡道:“琮儿跟在海涛身边可还称职么?”

小顺子见状只得叹道:“定海军山虽然荒废多年,可是一切文书图籍都还在,只是都已经尘土深埋,琮少爷跟在您身边多年,整理这些文书十分得力,姜侯多有倚赖。”

这时,有个青影向上行来,小顺子也不需回头,便笑道:“琮少爷来了,想必文书已经整理完毕了。”

我还未答话,霍琮已经匆匆到来,深施一礼道:“先生,弟子已经将全部文书都整理好了,其中有杭州湾的精密海图,姜侯请先生前去商议下一步的战事。”

我又望了一眼碧海,只可惜云山遮断归途,望不见家山乡梓,轻轻一叹,我转身向下走去。山下的虎贲卫士除了数人之外,都已经是新面孔。这么多年来,当日曾随我平汉的虎贲卫士多半都已经高升了,不过这些新的卫士武力只有更强,当年我所传授的刀阵已经被虎贲卫精益求精,现在就是小顺子,急切之间也不能讨到他们的便宜。不过这一次呼延寿仍然是我的亲卫统领,想来是皇上的安排,也够委屈他这位大统领的了。

霍琮跟在我身边,兴奋地道:“先生的计策令弟子拜服,历来南北政权争夺天下,都是在江淮争胜,想不到先生竟然别出机杼,从海上攻取吴越,纵然不能摧枯拉朽,也定然可以动摇南楚的根基。”

我淡淡道:“这个计策却不是我首先想到的,此策本是南楚武帝谋划,却被我反过来利用了。”

霍琮大惊,露出疑惑的神情,就是小顺子也露出感兴趣的好奇之色,我见状笑道:“昔年,我曾奉旨整理御札,其中便有武帝御批。武帝十分勤政,御批极为丰富,更是涉及到许多军政大事,例如,他对定海、宁海两处军山就十分关切,亲自规划水营寨垒,又多次追加粮饷,更令人精心绘制各地海图,我见他字里行间都流露出霸气,绝非偏安之辈,便仔细阅读他历年御札手书,终于推测出他有心将两大军山建成攻防利器。平日可以防止海寇和大雍水军,到了关键时候就可以沿岸北上,侵蚀青州、幽冀沿海。自古南北之争,往往都在江淮决胜负,武帝却认为南人暗弱,不及北人勇猛,与其在陆地血战,不如从沿海侵袭,夺得海疆之后,再通过河流向内陆侵袭,以及之长,攻敌之短,胜过从陆路劳师远征。这样的战策前所未有,我见之后也十分感慨,便是受了武帝影响,才会献策攻取定海军山,侵袭吴越。只可惜武帝去得太早,以至于无人承继大业。后人只知两军山护翼海疆,不可轻动,却不知其原本设立的目的,甚至定海军山还被南楚朝廷消减军费,以致如此疲敝,平白便宜了我们。”

话音尽处,我们已经下了山崖,呼延寿一个手势,那些虎贲卫士已经将我们三人翼护起来,定海初平,难免岛上会有些余孽或者南楚军的谍探,所以对于我的安全,呼延寿是一刻也不敢放松的。我们沿着荒草漫漫的道路走向定海都督府邸,定海水营这些年来无钱整修,就连岛上的道路也被野草遮蔽,水营更是已经残破不堪,还可一观的就只有定海都督府了,依然雕梁画栋,富丽堂皇。看到一片荒凉之中的豪华府邸,小顺子不由笑道:“这里的主将这般糊涂,怪不得定海水军一攻即破,全无战力。”我也是心有戚戚焉,连连点头,就是有心贪污些军饷,也犯不着花在府邸上面吧,这不是存心激起士卒的恨意么,真让我怀疑定海的主将是不是大雍的密谍。

还未走到府门,姜海涛带着部将已经匆匆迎上,如今他也是年近三旬,自从七年前东海归附大雍之后,东海侯姜永舍弃大雍的高官厚禄,飘摇出海去了,东海水军便由姜海涛统率。他虽然有些直率,不甚熟悉官场之事,可是有一位贤内助善加辅佐,再加上他统率水军的本领出众,又有雍帝李贽和齐王李显的照应,倒也没有什么麻烦阻碍。这一次雍帝令他南下攻略吴越,这对他来说并无什么问题,唯一令他头痛的就是,江哲居然随船而行。倒不是不愿意江哲在他身边指手画脚,只是担心江哲若是出了什么问题,他可是担待不起。

到了近前,姜海涛就要下拜行礼,我和他虽有师徒名份,若论爵位,他尚在我之上,他以师徒之礼拜我,岂不是让他麾下将领为难,所以我连忙阻止道:“你若要行此大礼,私下里再说,难道还要让你麾下的将领都跟你一起行大礼么?”

姜海涛一回头,看向身后众将,不由赧然,上前躬身一揖道:“先生,现在定海局势已定,我想听听先生的意见,我们应如何攻取吴越。”

我随着姜海涛向府内节堂走去,一边走一边道:“你定然已经有了打算,不知道你想如何做?”

姜海涛道:“若是能够攻破余杭水营,杭州湾就再无敌手,只是余杭一向极重水营,恐怕不能得手。我有意先取沿海州府。”

我说道:“近日建业将有举措,尚维钧一向最会贪功诿过,这次定海被我军攻取,他定会将定海军山交给陆灿,但是宁海军山的军权他却不会放过,所以我们不用担忧宁海水营会南下攻打定海,反而应该提防陆灿的反攻。余杭水营既然不易攻取,我军便不必急着攻余杭,会稽、余姚、镇海、嘉兴、海宁、平湖都是吴越重镇,却又军备不足,我军趁着现在陆灿还未到越郡,先将这些重镇的粮饷府库洗劫一空,因粮于敌,之后纵然越郡重被陆灿夺回,我军也有了立足的本钱。而且你还可劫掠沿海的青壮,将他们置于孤岛,可迫使他们在岛上耕种,用来弥补我军钱粮的缺口。这样一来,纵然宁海水营能够阻止我军从青州获得补给,也无济于事了。只要立足稳固,吴越迟早落入我军手中。”

姜海涛闻言笑道:“这本是我们作海盗之时常有的举动,掳劫钱粮人口,损敌而利己,想不到今日还要如此作为,普陀之地,最适宜拘禁俘虏,原本我准备过些日子再去攻取,如今看来却是应该快些着手了。请先生放心,十日之内,越郡沿海的青壮都会落入我的掌中。等到陆灿来了越郡,也只能黯然长叹,坐视吴越之地被我洗劫。”

我摇头道:“那倒也未必,到时候多半还是相持之局,他没有足够的兵力将你们逐出定海,你也没有足够的军力占领吴越,不过你放心吧,陆灿不能在越郡长久待下去,长孙冀奉命攻襄阳,这一次必有斩获,到时候陆灿自然不能再留在越郡和你对抗了。”

姜海涛若有所思地道:“先生放心,这些日子,我定让陆灿陷在越郡,也好呼应襄阳战事。”

我微微一笑,这小子一谈到行军作战便十分机灵,我稍微露点口风,他就知道这一次主要的目标是在襄阳。想到我这次坚持要随水军南下,借口是想看看海战,实则是我想趁机回一趟嘉兴,解决荆氏的问题,顺便拜祭一下母亲,不知道他有没有这个胆子放行呢?想到此处,我露出诡异的笑容,走在我旁边的姜海涛一个冷颤,错过脸去,心中生出不祥的预感。

此时,陆灿正在乘舟直奔余杭,这一次他带来九江水营的一万士卒,决定将他们充实到余杭水营,若没有一支战力足够的军队,就是组建起义军也将没有用武之地,而且只有先将雍军逼退,才有组建义军的可能。也无心去看两岸景色,陆灿心道,只需给我三年,我就可以在吴越之地练成一支精兵,重新夺回定海,将雍军逐走。但是心中一缕隐忧涌起,这次雍军困襄阳,真的只是佯攻么,这一次东海水军寇吴越,已经出了他的意料,若是襄阳这次有什么变化,恐怕局势堪危,轻轻一叹,陆灿知道自己别无选择,吴越之地,素来尚维钧不许自己插手,若不是这次雍军寇吴越,尚维钧尚不会允许自己接掌吴越军政大权,而这次自己若不亲赴吴越,只怕那里将成为资敌之地。而襄阳,毕竟还有容渊在,应该可以支撑得住吧,在心中安慰自己片刻,陆灿终于将全部心思放在了如何完善越郡防线,避免雍军入寇内陆上面。

\chapter{第二十三章 乡音无改}

同泰十二年,雍军东海水营寇吴越,哲随行军中,二月十二日,雍军入嘉兴,哲潜行祭母,会荆氏,尽逝前嫌,然莫为世人知。

——《南朝楚史·江随云传》

嘉兴烟雨楼本是东南名楼,最多士子游人,尤其是二月初春,碧柳如烟,清波荡漾,渔船帆影,往来如梭,最是景色怡人。只可惜如今虽是赏景之时,楼中之人却都愁眉深锁。早在数日之前,就已经有传言说及雍军攻下定海,但是这消息并未引起他们过分的惊骇,吴越之地,几乎很少遭遇兵燹,在他们心目中,雍军很快就会被余杭水营击退。可是事情的演变令他们措手不及,几乎是转瞬之间,雍军如火如荼的攻势就已经席卷了吴越之地。前日雍军已经攻下了平湖、海宁,据两地传来的消息,雍军并没有大肆屠杀,只是将当地军民拘禁城中,不令自由行动。虽然不解雍军用意,但是因此之故,嘉兴军民也不免有些放心,雍军攻越郡只是仗着出其不意,一旦南楚军反攻过来,雍军必定会被迫退回海上,只要雍军不杀害人命,那么就是损失些金钱粮饷也没有什么大碍。

楼中众人都是嘉兴各大世家的年轻子弟,也有嘉兴一地知名的寒士,如今雍军前锋已经到了嘉兴城郊,这些青年子弟不愿困在家中,都在烟雨楼聚集,希望得知最新的战况,也只有这些尚有血气之勇的青年才有胆量在这个时候聚集起来。这些年轻人中有一人神情有些不同,那是一个弱冠年纪的少年,青衫儒服,相貌俊秀,气度深沉,他坐在窗前俯瞰南湖景色,似乎有意和众人隔离开来。满楼众人也是有意无意地避开他,但是却都暗暗用目留意他的神色。这个少年名叫荆信,他是荆氏嫡长孙,荆长卿之子。

和各地攻讦江哲的风气不同,嘉兴一地的世家盘根错节,为了荆家的面子,众人多半都是缄口不言,而且内心深处,这些世家反而都暗暗羡慕荆氏旁宗出了江哲这样的人物。家国天下,在这些世家眼中,家族的荣耀才是最重要的,虽然不免将大雍的勇士当作蛮子,认为他们不及南人诗词风流,但是大雍的威势仍然让他们心有余悸。所以即便是为了留条后路,嘉兴世家对荆氏一向是不敢轻忽的,这也是尚维钧想要铲除荆家,却不能顺利进行的一个缘故。当然荆氏也不是全然不会受到影响,碍着朝廷的颜面,嘉兴世家表面上对荆氏还是会冷淡一些的。荆信身为荆家的继承人,自然对这种情形深有体会,若是大雍和别国开战,众少年在烟雨楼论战之时,往往将他围在当中,若是大雍和南楚作战,众人则是有意无意地将他孤立起来,当然,却也不会对他视而不见,甚至对他的论断更加留心。久而久之,荆信便习惯了这种对待,所以今日他便刻意和众人保持了一定的距离。

望向窗外的湖水,荆信心中并没有表现出来的那样平静,对于这个表叔江哲,他从未见过,也没有任何印象,可是对于江哲之父江寒秋,他却有些了解。昔年江寒秋离开嘉兴的时候,带走了自己的全部文稿,但是在荆氏的书房之内,却留下了几本笔记,上面有他读书的心得,荆信自从得知江哲之事后,便特意去看那几本笔记。虽然江寒秋籍籍无名,可是他的笔记可以说是包罗万象,极有见地。荆信每次读后,都有新的收获,不由叹息,有这样的父亲,怪不得江哲可以名动天下。

对于江哲,荆氏之内是有两种倾向的,有如荆舜荆一般索性去了大雍,依靠江哲的支持重立家业的,也有如荆长卿一般忿忿不平,将其当作乱臣贼子的。荆信心中明白,这些年来,祖父已经渐渐倾向二叔,甚至族中也对自己的父亲不满,想要让二叔接任家主,只是碍着二叔在大雍行商,不便张扬罢了。在荆信心目中,他自然不赞同父亲这般固执,不念亲情,可是若是依附江哲投向大雍,他也不甚情愿。荆氏为何要依靠外人立足呢?这便是他心中所思。

这时,一个少年奔上楼来,大声道:“糟了,嘉兴守军不敢出城迎敌,已经溃散逃去,雍军已经入城了,正在沿途戒严,不许居民上街行走,再过片刻,就要到烟雨楼了。”

这些青年大哗,心中都生出恐惧来,虽然还没有雍军屠城的消息,可是这种人为刀俎,我为鱼肉的情形并不好受,一个英武少年怒道:“都是尚维钧那厮,只知道搜刮聚敛,这吴越文武官职都是他鬻爵卖官的本钱,贤达充任下陈,庸碌之辈反而金堂玉马,否则怎会被雍军直入吴越内陆。”众少年闻言都是齐声喝彩,平日碍着尚维钧秉政之威,纵有不满,也只能私下里议论几句,今日这少年当众指斥,嘉兴又遭遇变乱,人人都觉得心神畅快。但是纵然如此,也已经无济于事,众人不免黯然叹息。一个矮胖青年看向荆信,见他神色沉静,不由讽刺道:“荆兄却是可以安枕无忧,纵然雍军屠戮嘉兴,也不会为难荆氏,令尊于兵荒马乱之中,还能够安然从淮东返回,何况如今呢?”

荆信本是心思深沉之人,闻言也不由勃然大怒,荆长卿在楚州遇险,幸好有人暗中相救,才将荆长卿一家送回嘉兴,荆信若非留在家乡侍奉祖父,也必然遭此劫难。那相送之人丝毫不露声色,来去无踪,但是想来也知道能够在淮东战乱之际救出荆长卿的,必不是寻常之人。这件事情荆氏本来不愿声张,想不到却被朝中秉政之人严令追究,将荆长卿下狱问罪,甚至已经下了斩首文书。可是在这个时候,却传来雍军攻破定海的消息,就是嘉兴官府有再大的胆子,也不敢在这个时候将荆长卿斩立决,反而将文书藏起,让荆长卿取保出狱,这件事情虽然别人不知,但是嘉兴各大世家都是知道的。此事既是荆氏隐秘,也是荆信心中禁忌,这矮胖青年一说出口,也觉得自己失言,但是看到荆信阴沉的面容,又觉得自己说得没错,露出桀骜之色。

这时,另外一个沉稳青年道:“事已至此,嘉兴已经为雍军所得,我们还是各自归家去吧,也好和家人同甘共苦。”这些青年闻言,也知道自己全无扭转局势的力量,便趁着烟雨楼尚未戒严,一一离去了。

荆信却是站在楼上低头不语,神色冰寒,想到父亲在楚州受辱,一路上逃难也是十分艰难,可是在嘉兴世家子弟看来,不过是装腔作势,真是令他痛恨不已,心中突然生出一个念头,若是自己从军作战,将雍军逐出吴越,想来应该不会有人再指责荆氏通敌了。这个念头一生出来,便如烈火燎原,一发不可收拾。这时,楼下传来纷乱之声,他走到另外一扇窗子,向下望去,街道上到处都是慌乱失措的民众,雍军如同青黑色的铁流一般正从四面八方涌入,在他们的强势威逼下,这些无力自保的南楚平民纷纷闭户归家,整座嘉兴城已经渐渐落入雍军的控制。

荆信正欲转身下楼,趁机归家,还没有走下楼梯,只见几个步履沉凝的黑衣军士护着一个青衣少年走上楼来,荆信心中一惊,还未作出反应,一个军士已经一把将他推到一边,按着刀柄问道:“你是什么人,为什么这个时候还在烟雨楼流连?”那军士杀气隐隐,显然荆信若是回答不当,就要将他一刀杀死。

荆信微怒道:“晚生本来在此赏玩湖景,贵军入城,不及闪避,若是你等要因此加害,晚生也无话可说。”

那军士笑道:“你这书生倒是盛气凌人得很。”言罢回头问道:“霍公子,可要将他监押起来么?”

那青衣少年走上前来,笑道:“这倒是我们失礼了,烟雨楼本是人人都可以来此赏玩的胜地,兄台在此也没有什么奇怪。在下霍琮,请问兄台尊姓大名,我见兄台气度不凡,这般时候还在外面流连,想必是嘉兴青年俊杰。”

荆信凝神瞧去,这青衣少年不过十六、七岁,容貌平平,不甚出众,却是神色淡然,而那几个黑衣军士一眼便可看出非是普通军士,荆信虽然对军务不甚了然,但也知道雍军服色以黑为贵,能够穿着黑衣黑甲的,必然是雍军猛士。这少年如此年纪,就可以指挥这些黑衣军士,必然是雍军重要人物,虽然知道此人乃是南楚的强敌大仇,但见他和颜悦色,荆信心中却是生不出丝毫厌恶仇恨之感,再见他眉宇之间自有一种雍容淡漠的气度,更是不敢怠慢,躬身施礼道:“晚生荆信,不敢当俊杰之称。”

那青衣少年闻言神色一动,笑道:“原来是嘉兴荆氏的才子,听说荆兄十四岁时已经中了举人,若非近年来闭门读书,不求功名,只怕已经名登金榜,成了南楚的栋梁之材了。”

荆信听他语气,似乎对自己的荆氏身份并不留意,心中反而一宽,但是听到他这般恭维,却生出一缕寒意,自来两国征战,对敌国的人才不是据为己有,就是杀之而后快,这少年虽然是淡淡几语,却可能是决定自己生死的判词。但是对待这种情况,他也只能微笑道:“霍公子年纪如此之轻,却显然深受贵军勇士敬重,想必身份地位必然紧要,这般人物,方可称得上是栋梁之材。荆某无心功名,平日里只是读书饮酒,闲来便浏览南湖风光,殊无雄心壮志,怎称得上是栋梁,都是霍公子谬赞了。”

那青衣少年闻言淡淡一笑,道:“荆兄过誉了,我不过是附骥之人,并无可取之处,今日和荆公子有缘相见,霍某有意请公子共饮几杯,不知公子意下如何?”

荆信微微苦笑,看了一眼那几个按刀而立的军士,道:“敢不从命。”

那青衣少年邀请荆信入席,楼中伙计在雍军军士监视下,战战兢兢地送上酒菜。荆信本是心中忐忑不安,但是几杯酒之后,见那青衣少年不曾提起荆氏和江哲的关系,也不曾有意招揽,他心中才平静下来,虽然不免有些自嘲,看来自己的才学还不入人眼,但是言谈举止之间已经是挥洒自如。那青衣少年自称初次来到嘉兴,便向荆信问及嘉兴名胜。

荆信已经略带几分酒意,指着楼前的湖水道:“嘉兴南湖,素有东南奇秀之称,此是滮湖,嘉兴西南名秀川,有鸳鸯湖与此湖相接,两湖并称南湖。滮湖为众流所汇,停蓄演迤,揽其形势,实为灵秀所钟,鸳鸯湖中隔一长堤,堤上有一座石桥,名叫五龙桥,桥东的湖泊叫东湖,桥西为西湖。古人曾有诗言‘东西两湖水,相并比鸳鸯。湖里鸳鸯鸟,双双锦翼长’(注1),就是描述鸳鸯湖美景,西湖又称里湖,旋称蠡湖,为后人附会而称作范蠡湖,湖边建有范少伯祠,用以祭祀贤良。‘槜李城南范蠡湖,野桃花落点春芜。湖中种得杨池藕,得似西施臂也无。’(注2),此诗就是吟咏西湖美景的,西施臂即是西湖莲藕之名。”

霍琮听得入神,微笑看去,只见荆信神采飞扬,气宇风流,想及此人身份,心道,不愧是先生亲眷,把盏敬酒道:“荆兄果然才华过人,小弟也记得几首前人词句,尽述烟雨楼胜景。不知道荆兄可听过么?”言罢他从容吟道:“细雨前汀,菱花开过苹花断。倚楼客倦,雨远更烟远。平底船轻,柳外渔歌缓。风吹散,鸳鸯飞遍,只是无人见。”(注3)

此诗吟罢,荆信心思电转,眉头深锁,沉默不语,他在祖父书房之内曾经见过一张条幅,就是这几句词,落款是清远居士,清远居士正是江哲之父江寒秋的别号,这首词流传不广,至少荆信不曾见过嘉兴还有别人知晓,这少年却吟咏出来,莫非此人和江哲有什么关联么?他心中生出疑念,神色便渐渐变化,那青衣少年问他三句,他也难以回答一句,一时之间烟雨楼上的气氛变得尴尬起来。

这时,一个中年将领步上楼来,对这青衣少年抱拳道:“霍参赞,嘉兴已经全部控制住,请参军下令。”

青衣少年起身道:“方将军不必拘礼,霍琮只是暂领虚职罢了。”

那中年将军却是神色恭敬,道:“侯爷有令,这次行事要听从参赞之命,请霍参赞尽管吩咐。”

那青衣少年微微一笑,道:“如此霍某擅专了,请方将军将嘉兴世家家主、名士贤达都请来烟雨楼吧。”

这中年将军正是方远新,乃是东海数一数二的将领,能征善战,本来不会听从一个乳臭未干的少年命令,可是这霍琮自从到了定海,便奉命整理定海军山遗留的文书图籍,这些文书都是关系定海军山的机要,到了后来,这霍琮对定海和吴越沿海地势军情了若指掌,就是靖海侯也要仰赖于他。东海水军在定海所立的大营便是他根据图籍完善的,甚至何处该修寨垒,何处该设哨所,他也一清二楚,最后靖海侯授他参赞一职,却是无人反对。更何况他是楚郡侯弟子,和靖海侯师兄弟相称,所以这些将领也不敢轻视于他。这次姜海涛阻止不了江哲前来嘉兴,便特意让霍琮负责劫掠越郡之事,又让方远新统军,就是为了江哲的安全着想,否则虽然霍琮才能出众,姜海涛也不会让一个少年主管此事。

荆信在一旁听见已经是神色大变,他虽然猜到这少年身份重要,却也想不到嘉兴军民生死皆在此人掌握之中。有心想要告辞,谁知尚未出口,那青衣少年已经笑道:“荆兄才具,霍琮心中敬服,还请荆兄多留些时候,一来替在下引见嘉兴贤才,二来在下也想和荆兄多盘桓些时候。”抬头看去,却见那青衣少年神色从容,毫无威凌之意,纵然心中不满,也难以出口。大雍才俊如此,南楚焉能久存?荆信一叹,身不由己,自己又能如何呢?

鸳鸯湖畔,有一处梅林,梅林之中有一处数丈方圆的坪子,就在梅花疏影之中,掩映着一处坟茔,墓前一块青石墓碑,上面的字迹已经十分模糊,更被青苔所掩,难以看清文字。可是墓碑虽然残破,那坟茔却似有人照料,墓草青青,更有香花供奉,坪子上更是足迹成蹊,显然有人常常在此徘徊流连。对比梅林之外的荒草漫漫,当真是古怪得很。

时近午后,这里的沉静被人声惊碎,一个披着青色大氅,头戴信阳斗笠的男子正缓缓向梅林走来,在他身后,一个容颜如雪的青衣少年迤逦而行,两人左右身后,则是一些黑衣军士紧紧护卫。梅林之外,更是早有一些黑衣大氅的军士将梅林团团围住,林外青草已被摧残得七零八落,那男子见状眉头轻皱,不由庆幸为免打扰亡者安宁,事先下了不许这些武士进入梅林的谕令。

走到梅林之前,那青衣少年走入林中,他虽然不甚留意足下,可是所过之处青草不折,可见他的轻功高绝,不多时,青衣少年出林道:“公子,可以进去祭奠老夫人了。”那男子轻声长叹,轻轻除去青色大氅,摘下遮住面容的斗笠,露出华发朱颜,白衣素服。他举步向梅林之内行去,那青衣少年接过一个武士手中提着的香烛纸钱,随后入林。那些黑衣护卫都是小心谨慎地留意四周,大雍驸马都尉,楚郡侯江哲亲身至此祭奠亡母,纵然嘉兴已经落入雍军手中,也不能大意,若被隐秘行踪的南楚谍探盯上,岂不是麻烦至极。

我望着梦中依稀仿佛的梅林,记起当日拜别母亲坟茔的情景,不由泪洒黄土,在墓前拜倒,顿首膝前,泪水无声的滑落,若非娘亲亡故,父亲怎会和舅父生出嫌隙,因此离开故园,流浪江南,若不是旅途劳顿,父亲怎会旧病复发,又怎会因为痛惜娘亲之死而心伤难愈,以至于留下我这人海孤雉。父亲心碎而死,我飘零半生,都是因为娘亲亡故,想及此处,怎不令我肝肠寸断。

不知哭了多久,颈后有冰凉的真气侵入,我浑身一个冷颤,方才清醒过来,心中明白是小顺子见我过于伤心,才用真气唤醒我,免得我悲恸过度。我望了跪在我身后的小顺子一眼,眼中透出一丝暖意,然后接过他手中的纸钱香烛,在娘前墓前焚化。目光一闪,看到那被青苔蒙蔽的石碑,心中一痛,伸手除去青苔,露出碑上俊逸清雅的字迹,石碑上面书着“江门荆氏之墓”,落款是“寒秋泣立”四字。

看到碑上的父亲墨宝,心中原本生出的戾气渐渐消散,耳边传来苍劲的足音,由远及近,小顺子走出梅林,不多时转回道:“是荆氏老家主前来,被呼延统领阻住,公子是否要见他?”我略一犹豫,道:“请舅父进来吧。”

不多时,一个华服老者拄杖走入,这人已经年过七旬,须发皆白,容颜苍老,神情冷肃,不过见他身姿,便知道仍是身轻体健。他走入梅林,也不瞧我一眼,径自走到墓前,望着坟茔,良久方道:“哲儿你离开嘉兴多年,这次应是头一次回来拜祭你娘亲。”

我叹息一声,终于下拜道:“舅父大人康健如昔,甥儿江哲叩见。”

那老者也不上前搀扶,淡淡道:“你的口音尚有嘉兴余韵,想来未曾忘记乡梓,不过你又何必行此虚礼,你应知道我对你父子的恨意。我和你娘亲的生母早亡,继母不良,父亲又醉心仕途,令我兄妹二人在家中受尽孤苦,若非有小妹时时劝慰,当初我早已离家而去,根本不会有机会继承家主之位。你娘亲身子不好,我不愿她嫁给薄情宦游之人,所以亲自为她择婿,你爹爹无心仕途,才华横溢,故而被我看中,说服父亲将小妹许配给他。”

我站起身来,默默听着他的话语,他语气激动,显然这些心事埋藏多年,无人可以述说,今次才对我说了出来,这些往事我不甚清楚,今日听到舅父说及,自然是专心倾听,听到此处,我插话道:“父亲在世之时,曾言昔日和娘亲结为鸳侣,多蒙舅父从中斡旋。”

那老者冷哼道:“总算他还有些良心,哼,小妹和你父亲成婚之后,倒也是举案齐眉,相敬如宾。只是过了不久,她便怀了你,其时她常常晕厥,我召来良医为她诊治,那医士说你娘亲先天不足,若是生育便有性命之忧,当时若用药物流去胎儿,尚还不晚。我便劝你爹娘答允,若是你父亲忧虑没有后嗣,最多我送他几个侍妾就是。岂料你爹爹竟然不肯答允,结果小妹生下你之后,险死还生。其后数年都是缠绵病榻,若非如此,怎会在瘟疫爆发之时受到波及而死。都是你父子害死了她,你今日回来祭拜也还罢了,但你若想将江寒秋的灵柩送回来合葬,除非我死了,否则绝无可能。”

闻言,我昔日模糊的记忆渐渐回来,想起少时虽然常见爹爹娘亲花间唱和,琴筝合奏,但是娘亲果然总是那般苍白神色,虚弱体态,想起爹爹过去隐约透露的一言半语,忍不住清泪垂落,泣道:“舅父难道不明白,这决定乃是娘亲之意,爹爹不过是不愿违逆娘亲苦心。”

那老者身子一颤,望向江哲的面容,心中浮起亡妹的倩影,发觉甥儿的相貌轮廓和亡妹颇为相似,当日小妹也是这般清泪滚滚,向自己哀求定要留下胎儿,良久,他才叹道:“你说得不错,若非小妹坚持,我又怎会屈服,只是我失妹之痛,难以平息,只得迁怒于你父子。”说出这句话,仿佛是多年支持他的仇恨支柱崩溃一般,他的神情多了几分颓废,似乎身姿也疲软了许多。

我心中也觉得苦涩非常,舅父虽然害得我父子飘零天涯,可是却是出于对娘亲的兄妹情深,梅林之中,足迹成蹊,显然舅父常来祭拜娘亲,却故意让父亲立下的石碑被青苔遮掩,却是因为他对父亲怨怼之情始终不减,当初我中了状元之后,荆氏族人颇有欲和我和好的,最后却不了了之,虽然是我无意,但是也多半是因为舅父反对,这也是舅父迁怒于我。但是,归根结底,却也是因为他对娘亲不能忘怀,我又何必还要和他作对。想到此处,我上前深深一拜,道:“舅父,我爹爹离开嘉兴之后,也是思念娘亲成疾,因为不愿令爹爹伤怀,我也不敢多问娘亲的事情,舅父如今在此,何不向甥儿说一说娘亲的风采,也好让哲心中多些可以追念的往事。”

老者闻言,也不由开怀,笑道:“你娘亲小字梅娘,生平也最是爱梅,少年之时,若是梅花含苞待放,便彻夜不寐,等候梅花开放,偶然有梅花早开,便定要前去赏梅,纵然冰雪未消,也不顾及。曾有一次她正在病中,闻说园中梅花初放,便不顾侍婢劝阻,披衣进园,踏雪折梅,结果受了风寒,大病一场,连日昏昏。自她嫁给你父亲之后,常和你父亲琴筝唱和,更是做了一首《梅花落》的筝曲,尽述梅花清华孤傲之姿,你可还有印象?”

我略一思索,已经记了起来,轻声唱道:“中庭多杂树,偏为梅咨嗟。问君何独然?念其霜中能作花,露中能作实。摇荡春风媚春日,念尔零落逐寒风,徒有霜华无霜质。(注4)”

老者闭目聆听,歌尽方道:“那一年嘉兴遭遇瘟疫,你娘亲本就体弱,不幸染病,临去之时,对我和你父亲说,她虽然不愿离去,无奈却终究不能抗拒天命,你虽年幼,自有你爹爹照看,谅无妨碍,只是不能再看一眼梅花飞雪,却是死有余恨。故而你娘亲殁后,我便选了这处梅林安葬于她,让梅香疏影,常伴芳魂。”

我忆起娘亲过世之时,我还年幼,又因为瘟疫横行,被送到别处安居,竟不能见到娘亲最后一面,忍不住泪落,道:“舅父其实不必为娘亲伤恸,娘亲少时有舅父照拂,出嫁后又和爹爹夫妻情深,虽然不幸早逝,但是想必娘亲其时心中定是平安喜乐,只因有舅父和爹爹这般爱她,她纵死也不会觉得此生虚妄。”

不知何时,夕阳已经西沉,晚霞映入梅林,染了轻红的薄雾载沉载浮,再有那若有若无的梅香相伴,梅林之内宛似仙境瑶池,坟中沉眠的又是我们两人至亲,梅林之中一片静默,空气中凝聚着祥和安宁的气息,令我二人都不愿言语。那老者更是似乎陷入回忆之中,眉宇间现出温柔怀念之色。

良久,夕阳的余晖渐渐黯淡,老者清醒过来,淡然道:“你这次前来,准备如何对待嘉兴世族,又准备如何对待荆氏?”

我轻轻一叹,终究是要回到正事上来,仇怨和家族存亡相比,孰重孰轻,舅父心中也是明白的,更何况我们终究是至亲,抬头微笑道:“舅父何出此言,哲此次不过是趁着我军攻占嘉兴的良机前来祭拜娘亲罢了,至于军务上的事情,我却不便插手。”

老者眼中寒光电闪,道:“以你楚郡侯的身份,怎会轻易到嘉兴来,就是你不惧危险,大雍皇帝也未必放心,而且你若仅是为了祭拜亡母,何必遣人密送帖子到荆家,想来这一次你是要和荆氏作个了断了,若是我今日不来,只怕荆氏也将烟消云散。数日之前,朝廷下了公文,判了长卿死罪,你想必已经知道?”

我目光流转,道:“此事我的确知情,今次已是最后的机会,雍军退后,再无人能够维护荆氏,舅父难道不念族人安危,何况今后吴越将是战场,荆氏在嘉兴也难安居。”

老者叹道:“故土难离,只是我也知道没有选择,长卿经此一事,已经心灰意冷,说服他已是不难。”

我早已料到如此,两国大战在即,我不想在南楚留有我的软肋,对于荆氏,我既然难以完全忘怀,就只有迫使他们归属大雍。对舅父轻轻一拜,道:“舅父如此明理,哲心中感佩,明日雍军将清洗嘉兴,凡是青壮男女,士子工匠,皆在劫掳之列,我已转托负责的将领,他会对荆氏加以关照,等到适合的时候,舅父可以随船去大雍安居。”

老者身躯轻颤,良久才道:“好狠毒的手段,夺取吴越人口钱粮,弱敌而资己,虽然是海盗手段,却是极富实效,我纵然不答应归顺,你也会令人将荆氏掳去定海,是么?”

见舅父一眼看穿我的心意,我倒也是心中赞佩,却不便说什么,只是深深一拜。老者轻轻一叹,举步向外走去,我心中怆然,背过身去,不愿见他苍老身形,风中却飘来他苍劲的语声道:“哲儿不必为难,你对荆氏已是仁至义尽,谢谢你对长卿和舜卿的提携救助。”

闻言,我心中一宽,放下了心中大石,荆氏的事情终于处理妥当,我便可以安心离去了。对着娘亲坟茔再拜叩首,徘徊良久,终于依依惜别。

这一次我费尽心机说服姜海涛,让他允许我亲到嘉兴一趟,除了想拜祭母亲之外,最重要的却是要和荆氏和解,毕竟嘉兴荆氏是我母族,先天上已经有争取的可能,这次我献策图谋吴越,掳劫世家平民填定海,是为了削弱南楚,可是我并不准备真得残害吴越之民,一来不符合我的性子,无利之事我从来不做,二来也有损大雍荣耀,三来将来统一江南之后,吴越之地必然因此久久不肯降服,所以最好的办法就是在被掳的吴越之民中选出一些人来,通过他们管理俘虏,这样一来,外严内宽,以吴越之人温和隐忍之民风,才不会造成大雍统治上的困难。而这样的人选不可轻易选择,又需有治理内政的才能,所以嘉兴世家就成了我的选择,人谁没有私心呢,我也不会例外。只不过当日我只和海涛说了一半缘故,我来嘉兴尚有别的缘故,只希望他得报之后不会捶胸顿足吧?

——————————————

注1:宋张尧同《嘉禾百咏》

注2:清谭吉璁《和鸳鸯湖棹歌之十》

注3:清冯登府《点绛唇•烟雨楼秋泛》

注4:南朝宋鲍照《梅花落》

\chapter{第二十四章 金蝉脱壳}

二月十三日,东海水军掠吴越之地,青壮钱粮尽归定海,余姚、镇海、嘉兴、海宁、平湖皆无幸,唯余杭、会稽得水营翼护,无所伤。

——《资治通鉴·雍纪四》

烟雨楼上,诸世家家主皆被召来,还有嘉兴名士数人,都被雍军强行请来,原以为是雍军大将相召,孰料主人竟是一个十六七岁的少年,原本这些家主心中都存了轻视不忿之心,孰料这少年言辞得体,对嘉兴众人底细均了如指掌,言谈之中,更是流露出敬仰之意,不过片刻,就令众人放下敌视之心。那少年便令摆下酒宴,向众人询问嘉兴地理黎庶,众人既在篱下,焉敢不答,再说也有心一挫这少年锐气,寻机出言问难,结果烟雨楼便成了高谈清议之所。这少年虽然没有什么明见卓识,却是气度从容,侃侃而谈,极善调动气氛,竟令楼中其乐融融,直到日落黄昏,这些家主名士也是意犹未尽。那少年又令秉烛继宴,众人竟也没有十分拒绝。

荆信虽然是嘉兴世家青年俊杰中佼佼者,原本却也没有资格参与这样的谈话,但是荆氏声言家主卧病,不便前来,奉命而来的却是荆信的三叔荆逊卿,荆逊卿本来忧虑这样一来难免会得罪雍军,但是见到荆信在此,而且霍琮对荆信似乎十分器重,荆逊卿灵机一动,假传荆长卿之命,让荆信替家主赴宴。霍琮闻后十分高兴,更是特意让荆信坐在身边。若论荆氏地位,在嘉兴虽然颇为显赫,但是可以和其相提并论的就有两家,霍琮这般对待荆信,固然是殊荣,但是荆信只觉得众人看向自己的目光都充满疑惑,众目睽睽之下,坐立难安,所以在席间也是沉默寡言。但是他越看越是惊心,霍琮虽然谦抑平和,却隐隐控制着大局,嘉兴世家已经尽入其彀中而不自知。

夜色渐深,那些家主开始有些不安起来,一场宴会到了这个时候未免拖得太长了,可是往主位看去,那霍姓少年仍然神采奕奕,兴致正浓,这些家主开始忧虑起来,再想想四周充做侍从的雍军军士,个个都是虎视眈眈,心中不免担忧起来,他们也知道这少年将自己召来定是有所借重,可是不论是想要如何,到了这个时候也应该宣布了,怎么却拖着不肯散席。这样一来,众人不免开始胡思乱想,但是这些人又多半是老奸巨猾之人,自然不敢让气氛变得尴尬,更是费尽了心思寻出些话题来交谈,困得呵欠连天也不敢表露出来。

直到第二日清晨,霍琮才起身笑道:“晚生和诸位贤达一夜长谈,真是受益匪浅,只可惜天下没有不散的宴席,长夜漫漫,终有尽时。”

嘉兴世家中颇富盛名的君氏家主强行睁着红通通的眼睛,起身道:“能与霍参赞共饮,是我等之幸,参赞年少英杰,若有指教,尽管畅言,我等必然尽力为之。”他却也是忍不住了,与其不识抬举等到雍军翻脸,还是主动询问价码吧,在他心目中,若是送上金银钱粮,应该可免杀身之祸,雍军是不可能在嘉兴多留的。

霍琮早已得到回报,先生已经离开嘉兴,而一夜之间,雍军已经将嘉兴世家平民全部登记在册,只待自己下令了,所以他也不虚言矫饰,肃容道:“霍某奉靖海侯之命,取吴越之民填定海,诸位皆是嘉兴贤达,尚请戮力相助。”

此言一出,众人先是茫然,继而眼中露出惊骇之色,瞠目结舌地望向霍琮,都露出不敢置信的神色,这和善平凡的少年在他们眼中顿时成了毒蛇猛兽。霍琮笑道:“诸位族人,皆已束装上道,嘉兴车马舟船已经尽被我军征用,各位一路上当不致辛苦。”

荆信本是沉默不语,听到此处也是怒火填膺,起身扬声道:“雍军自称王者之师,如何行此不义之事,掳民入海,此是盗匪行径,扰民至此,何以对天下之人?”

霍琮平静地道:“两国征战,无所不用其极,若是尽屠吴越之民,也可达到同样的效果,只是我大雍天子仁厚,不愿残害黎庶百姓,取吴越之民填定海,已是定局,两害相较取其轻,荆兄应当谅解才是。”他语气虽然平淡,但是目光中寒芒闪现,却似乎动了杀机,荆信一滞,荆逊卿已经轻拉他的衣袖,阻止他继续说话,荆信只得颓然坐下。

这一次雍军侵入吴越,本已在南楚朝野预料之外,但是纵然定海被夺,吴越两郡的世家官员也并不觉得雍军会登陆作战,毕竟雍军在吴越之地全无根基,若是效仿海盗上岸劫掠,也未免有失大国风范。孰料东海水军主事之人本就做过海盗,再有一位不拘礼俗的楚郡侯为谋主,竟然定下了取吴越之民填定海的决策,用以和南楚长期对抗。若是换了大雍别的将领来主持定海,或者会换一种方式作战,但是姜海涛既对江哲信服,又秉政海盗作风,再加上他投雍之后,被雍帝赐以侯爵之位,却是承袭父荫,未立战功,这在大雍来说也是特例,所以他也很想用战绩证明自己,所以才会不遗余力地采用这种可能会受人非议的战策。

片刻之后,烟雨楼下传来嘈杂之声,荆信闻声不顾雍军军士执刃在侧,到了窗前向下望去,只见街道两旁都有雍军进入民居,按照名册将一些青壮男女用绳索缚住向外赶去,老弱妇孺跟在后面啼哭,却被雍军执利刃逼退,嘉兴城内一片混乱,荆信只觉心中茫然。这时有人高声唤他名姓,他回过头去,只见烟雨楼上已经只有那些垂头丧气的世家家主和雍军军士,那青衣少年霍琮已经影踪不见,唤他之人正是一个军士,却是催促他整装上道。

南楚同泰十二年,大雍隆盛八年,对于吴越之地的世家百姓来说,可以说是一场浩劫,余姚、镇海、嘉兴、海宁、平湖被掳走五十万青壮,其中包括了各地世家宗族,寒门名士,各类工匠,雍军的手段可以说十分果决狠辣,五府县人口近三百万,却被雍军掳走六分之一,其中包括近五万世家族人、寒门名士,十万工匠,其余皆是青壮男女,按册索人,百不余一。待到陆灿率领九江水营经江南运河至嘉兴之时,雍军离开不到六个时辰,陆灿另遣部将前往接管余杭水营,自己率军追击雍军,无奈雍军早已计划周详,行动迅速,陆灿直追到盐官,却只能眼看着雍军从容渡海而去,只余下陆灿扼腕叹息,也不禁惊叹雍军主事之人手段狠辣高明,要知道雍军撤退可不是轻身离开的,随行的既有劫掠的钱粮也有被胁裹的民众,雍军居然能够毫不拖泥带水的撤入海中,怎不令陆灿惊佩。

站在岸边,望着雍军扬帆远走的船只,陆灿恨声长叹,却也无可奈何,而此时,得到他谕令的余杭水营才姗姗来迟,陆灿知道余杭水营向来自成一系,而且耽于安乐,早已没有了出海作战的勇气,却也只能轻轻责备几句,事已至此,重整余杭水营还需这些将领协助。接下来的日子,陆灿只能一边整编水营,一边重整沿海寨垒,防止雍军再度登岸劫掳,吴越之地遭此重创,留下无数残破门户,失去亲人的苦痛和担忧亲人遭到报复的吴越之民,对于组建义军并不支持,若非陆灿声威赫赫,又劝服吴越幸存的世家自保,更有武林侠士振臂一呼,全力协助,只怕组建义军一事将事倍功半。就在陆灿着手吴越海防的时候,一个消息传入他耳中,令他双眉深锁,这消息便是大雍楚郡侯江哲竟然身在定海,而且曾经亲赴嘉兴祭拜亡母。

一石激起千层浪,消息不胫而走,不过数日已经流传开去。江哲前往嘉兴祭灵,此事虽然隐秘,但是也并非是水过无痕,事后有见到蛛丝马迹的人一参详,便发觉了此事,更何况还有暗藏的南楚谍探,他们更是将江哲来去的行踪都掌握了,只是不敢出面阻拦暗杀罢了,毕竟雍军势大,江哲身边的侍卫又十分厉害。

虽然南楚上下,对江哲是异口同声地指斥辱骂,但是其实暗中却有几种不同的看法,有将之视为无君无父的贰臣贼子的,也有暗中羡慕他得此富贵荣华的,但是总的来说,能够知道江哲厉害的人却不多。一来南楚上层刻意瞒去江哲之能,二来江哲虽有侯爵之位,多半人都以为是雍帝酬其夺嫡之功,或者以为是长乐公主的缘故,纵有明智之士,也因为得不到足够的情报,不能正确评价江哲的才能。可是对于南楚军政核心人物来说,却不会轻看江哲,就是执意采用愚民之策的尚维钧,也不会轻视于他。如今江哲现身嘉兴,显然是在东海军中参赞军机,这样一来,雍军的主攻方向一定是吴越,否则江哲怎会在定海,纵然是陆灿,也不会相信江哲会为了祭拜亡母而至定海。

当然这个消息传开之后,南楚军政各种势力并没有立刻确信,都是全力收集相关情报,江哲身份不同,他若出现在定海,将显现雍军的下一步战略,谁都能想到,江哲重入军旅,必定是雍帝之意,若非是为了南楚之战,还有什么能令这位在大雍地位超然的寒园隐士来到江南呢?陆灿首先便是令人在嘉兴寻找线索,抽丝拨茧,终于确定了江哲的确曾经出现在嘉兴。不提嘉兴荆氏族人全部消失,曾有村人看见一些黑衣雍军来去,而烟雨楼的伙计掌柜幸存下来,更是将烟雨楼中发生的事情全部相告,虽然不知那少年参赞是什么人,可是只听他所作所为,陆灿就已隐隐想到此人身份,通过情报得知这少年参赞名霍琮之后,陆灿更是心中了然,霍琮年纪尚轻,大雍又是人才济济,除非江哲亲至定海,霍琮随行,才有可能让这少年一展长才。

另一方面,南楚从大雍内部得到的消息也确定楚郡侯江哲已经消失许久,而雍帝亲赴寒园相请之事更是沸沸扬扬,甚至有消息证实江哲的确去了东海,综合各路消息,陆灿终于确定江哲果然是随东海水军来了定海。

等到尚维钧得到同样的情报之后,随即传来密令,暂时令宁海军山接受陆灿调遣,要求陆灿全力剿灭占据定海的雍军,当然还有一个要求,尚维钧严令陆灿铲除心腹之患——江哲。尚维钧平日虽然明里暗里指责陆灿对江哲有师徒故旧之情,不过是为了争权夺利,实际上他内心深处并不认为如此,陆氏数代辅佐赵氏王族,绝无背国的可能。对于江哲在大雍的地位,尚维钧也是心知肚明。尚维钧虽然争权夺势的私心,可是他毕竟不是全然无能,对于江哲的厉害之处他清楚得很,若非如此,从前也不会对嘉兴荆氏留情,如果不是如今已经没有挽回的余地,他也不一定会对荆氏下手。

如今他既然认定了大雍的主攻方向乃是吴越,也就顾不上宁海的军权了,虽然只是允许陆灿调动宁海水营,而非是将军权全部交付,但是对他来说已经付出了巨大的牺牲。陆灿既不能辜负尚维钧的“好意”,而且他也有相同的看法,想到雍军在吴越劫掳的手段,不似东海水军原有的鲁莽粗率,而是精密狠辣,陆灿也相信江哲定是在定海指挥吴越水战。既然如此,就不能按照原来的计划放任雍军占据定海,若是拖个三年两载,只怕自己的精兵还未练成,雍军已经占据吴越两郡了。

因为江哲一人,原本可能暂时陷入僵持局面的杭州湾掀起了滔天战火,尚、陆两人再次捐弃前嫌,一心对外,余杭水营和宁海水营联手向定海发起了猛攻。

碧海之上,刚刚结束的一场恶战留下了无数的战船残骸,海面上浮尸处处,随着海流向外海漂去,敌我双方的船队向两个方向驶去,不过旬日之间,双方已经大战连场,却是未分胜负,若论水战,能与吴越水军对战的本就只有怒海求生的东海水军。

站在船头,感受着冰凉的海风,霍琮青衣飘飘,面色有些苍白,作战之时的颠簸疾行对他来说未免有些难耐,毕竟他不是常年在海上作战行船的东海军士。远处天际之下,海鸟掠波飞过,海浪滚滚,掩去了方才海战的痕迹,霍琮心中感慨万千,想及行踪不明的恩师,又是涌起无限烦恼。

劫掳吴越本是一件十分成功的壮举,可是回到定海之后,霍琮便挨了当头一棒,差点被坏消息击懵了,本来早应该返回的江哲居然影踪不见,只有百余名虎贲卫垂头丧气地回到定海,姜海涛和霍琮盘问之下,才知道究竟发生了什么事情。

却原来江哲离开嘉兴之后,不仅没有返回定海的意思,还准备由嘉兴北上,经江南运河至震泽湖,再经运河至京口,渡江穿越南楚控制的淮东,转道徐州,奔赴襄阳战场,这如何能让虎贲卫接受,此去千里迢迢,而且一路上多半都是南楚的势力范围,若是江哲的身份被南楚发觉,只怕性命不保。呼延寿出面谏止,却是无济于事。江哲说得很明白,若是呼延寿想要强行阻拦,他就要让邪影李顺带着他独自上路。争论纠缠了半天,最后呼延寿知道阻止不了,只得退让一步要求随行保护,恳求了半天,江哲才答应带上五个虎贲卫士,呼延寿只得选了四个武艺高强的侍卫和自己一同随行,而其他的虎贲卫士则被迫返回定海掩护江哲的行踪。

得知详情之后,姜海涛和霍琮差点气晕,尤其是姜海涛,当初江哲要先随水军南下,雍帝已经是颇为担心,临行之前曾有书信给姜海涛,让他保护江哲的安全,想不到初到吴越,就被江哲摆了一道,若是江哲有什么三长两短,他如何向李贽、李显和长乐公主交待。霍琮也是头痛万分,但是他毕竟是江哲最得意的弟子,倒是觉得江哲不是轻身赴险之人,这样决定必有缘故,所以反而劝姜海涛不要担忧。

那些虎贲卫奉命暂时留在霍琮身边,并带了江哲书信回来,江哲信上嘱咐二人,将他身在定海的消息传出来,不要让南楚军发觉他不在定海,而且说明消息传出之后,南楚军将对定海发起猛攻,让姜海涛小心。二人思索再三,只得遵行,为了作出江哲仍在定海的假相,甚至霍琮曾经染了鬓角,扮作江哲模样在船上出现。

而南楚军的猛攻也让他们吃尽了苦头。幸而宝剑越磨越是锋利,几次海战,南楚军都没有占到身边便宜,毕竟南楚水军多半都在内陆江河作战,对于海战,还是不如东海水军。双方便这样僵持住了,幸而定海已经在普陀建立了补给根基,又夺取了吴越钱粮,虽然宁海军山阻断北上归途,却也占不到什么便宜。虽然陆灿也曾有意取普陀,夺回吴越之民,但是一来普陀难攻,二来东海水军屡次在其攻击时从后袭击,三来就是攻下普陀,想要将五十万吴越之民运回陆上,在东海水军窥伺下也殊不可能,所以最终陆灿放弃了这样的做法,只能以海战为要,茫茫碧海,化作血火战场,东南局势,俱被东海水军牵制住了,陆灿虽然善战,也无法分心襄樊战事,只能全部托付容渊负责。

\chapter{第二十五章 却泛扁舟}

雍军退,哲嘉兴祭母事泄,世人皆知,人皆言哲献策掠吴越,皆责其戕害乡梓。然雍军虽劫掳,不曾虐杀黎庶,或言乃哲之功也。嘉兴父老畏雍军再往,翼骨肉重返,不敢取荆氏寸土。

——《南朝楚史·江随云传》

就在南楚水军和大雍水军在海上对峙之时,我已经在震泽湖上饱览无限风光,作为激化吴越局势的罪魁祸首,我可是没有一丝悔意,战争已经是必不可免的结局,吴越战局越激烈便越能转移南楚朝野的视线,也便于蜀中、襄阳战役的进行,至于我临阵脱逃么,咳咳,东海现在不是也用不到我么。

轻摇折扇,坐在画舫前舱之内,卷起珠帘,绶带锦袍,品着香茗,惬意地眯着眼睛享受春日的阳光,我摆足了南楚贵公子的派头,若非舟中没有歌女舞姬,倒是像极了游春的世家子弟,我又特意将灰发染成黑色,容貌也略加修饰,避免因为华发朱颜被人识破身份。吴郡虽然已经陷入了战乱,可是尚未波及到震泽湖周边的州府,吴郡人的和顺性情也让此地仍然处于平和安乐之中。毕竟陆大将军已经来了吴越,那么他们自然就不必担心了。我在湖上住了三日,八百里震泽,三万六千顷湖面,湖中有湖,山外有山,春光明媚,游人如织,丝毫看不出战乱近在咫尺的迹象。

珠帘轻动,呼延寿走了进来,他面上的神色十分不好,走到我面前躬身一揖道:“公子,险地不可多留,还请公子示下,我们何时动身?”

我抬起头看了他一眼,心中生出笑意,他相貌朴实敦厚,虽然多年位高权重,却没有染上颐指气使的脾性,只不过将近八尺的身高已经俊挺的身姿实在是很扎眼,再加上双目神光奕奕,双手虬筋纠结,怎么看都是一位威风凛凛的将军,可是却被我迫着穿上家仆服饰,还真是有些古怪啊。这也难怪,呼延寿可是虎贲卫的副统领,堂堂的一品将军,怎也不像一个平常的仆役。就是他带来的五个侍卫,我也看不出哪里像家仆。不过只要他们几个人别站在一起,倒也不是过分显眼,北地口音虽然重些,平日不说话也就成了,总有办法混过去的。不过,要不是呼延寿一口一个皇命,我又不想让李贽因此对他生出不满,才不会将他留在身边呢。至于他催促我赶路,也没有什么奇怪,要知道我在南楚境内待得越久,他的责任也就越重。更何况我们此次来震泽湖,路上可是和陆灿擦肩而过的,当九江水营急急南下的时候,我正在支流上面好整以暇地看着南楚水军的艨艟呢,我倒是没有什么,不过呼延寿可是一脸的铁青,唯恐被雍军发觉我的存在。只可惜他虽然是一片好意,我却不能成全他,留在震泽湖可并非是无事生非,我可是有为而来。

微笑着喝了一口香茗,我懒洋洋地道:“呼延,别那么着急么,难得来到震泽湖,不欣赏一下东山、西山的美景,岂不是太可惜了,何况现在南楚军正在从长江向余杭调动,与其现在上路,冒着遇到南楚军的危险,还不如等过几日,水道上比较平静之后再赶路不迟。”

呼延寿愣了一下,也觉得有些道理,可是留在楚境过久也是不妥,想到这次未能阻止江哲行动,回去之后已经难免被问罪,若是江哲再出些意外,自己怕是没有颜面回到长安了,想到此处正欲再劝,湖面上传来一阵琵琶之声,清越缠绵,应和湖波,声声入耳。

琵琶之声一起,我心中便是一动,闭目细听,那如泣如诉,如怨如慕的乐声几乎近在耳畔,诉不尽离情别怨,道不尽百转愁肠,一曲琵琶奏来动人心魄,好一曲昭君怨。听到一半,我睁开双目,轻轻一叹,昭君怨虽然是离别宫怨之词,却暗藏着“思汉”之意,缠绵悱恻中,乃是去国怀乡之沉痛,繁华退尽之喟叹。弹奏此曲之人,虽然弹出了绕指柔的意境,但是隐隐有落拓大方的气度,想必是忧心国事的才子。南楚繁华,江南烟水之间,不知有多少俊杰,只是南楚朝廷以诗词歌赋考较才能,纵然是皓首穷经,也难免黯然落第,而且就算是进了仕途,若无世家看重,也是没有一展长才的可能。就是陆灿,素以招纳贤才为名,也不能摆脱这种影响,他军中将领参赞,多半都和陆氏有着斩不断的渊源。想要凭借一己才能,在南楚立足并不容易,这弹奏琵琶的圣手想必也是报国无门之人,所以才会在曲中蕴藏这许多悲愤。

无意中一瞥,却见呼延寿也站在那里听得入神,心中不由奇怪,他什么时候也欣赏起琵琶了,倒是难得,心思一转,我几乎失笑起来,澄侯苏青精擅琵琶,已经是人尽皆知的事情,呼延寿既是她的夫婿,想必耳濡目染之下,也能领略一二。

这时,琵琶声一变,却是变得激昂壮烈,宛若铁骑突出,银瓶乍破,琵琶声中,我只觉得心跳加速,气血翻涌,面上顿时没了血色,珠帘飞起,原本在后舱入定的小顺子突然现身,飞身掠到我身后,一掌按在我背心,一缕真气渡入,片刻,我才长出一口气,平静了下来。呼延寿则是面色一寒,向外走去,显然是查探敌踪去了。

小顺子目中寒光四射,望向琵琶传来的方向,周身透出隐隐的杀气,这时,湖上传来一个男子引吭高歌的声音道:“醉里挑灯看剑,梦回吹角连营。八百里分麾下炙,五十弦翻塞外声。沙场秋点兵。 马作的卢飞快,弓如霹雳弦惊。了却君王天下事,赢得生前身后名。可怜白发生。”

我微微一愣,这原本是我在江夏见陆信练兵所作之词,后来为德亲王所获,他十分喜爱,每于军中吟唱,我的词风并不以豪迈为主,这一首却是苍劲雄浑,只是自从德亲王殁后,我又投了大雍,虽然我的诗词仍然在南楚流传,但是这一首却很少有人传唱,或者是觉得我不配写出“了却君王天下事,赢得生前身后名”这样的句子吧,尤其是现在,我已经公然领军攻吴越,还有人敢高声吟唱这首词,倒也难得。想到此处,方才险些被琴音所乘的恼意渐渐散去。

一曲未终,呼延寿已经回舱禀报道:“公子,三里之外有一艘游船,乐声是从那里传出的。”

我闻言透过珠帘向外望去,以我的目力,一眼便看到一艘没有船篷的小舟正在湖上随波起伏,舟上只有两人,一个是布衣儒服的男子,一个是黄冠的道士,那道士手中拿着撑船的竹竿,在船尾临风而立,双臂较为颀长,那男子却是高据船头,手执琵琶,背上背着长剑,正仰头向那道士说着什么,从我的方向只能看到二人侧面,但是也可看出二人气度便觉不凡,吴越乃是江南繁盛之地,地灵人杰,英才辈出,只是不能尽为南楚所用罢了。而且这两人能以琴歌震人魂魄,若非有小顺子相护,我恐怕已经受伤了。

想到此处,我兴奋地道:“这样文武双全的人物,可不能不见。”话音刚落,还不等呼延寿出言反对,身后已经传来一声冷哼,我身子一抖,回头对小顺子笑道:“下不为例,仅此一次。”眼巴巴地望着他,只怕他出言反对,这次出走可是我费了许多力气才说服小顺子的,各种理由摆了半天,才让小顺子勉强点头,但是一路上也是闷闷不乐,我在画舫小住,他始终在后舱入定,就是和我斗气呢,否则他历来都是在我身边伺候的。

小顺子心中本来是很不高兴的,本不愿江哲再惹是非,但是见到公子神采焕然,举止间更是多了放纵逍遥之意,再想到公子身在雍都,纵然是繁华深处,天伦之乐,却也掩不住淡淡的倦意,只有在暂时摆脱红尘琐事之后才能如此开怀,心中生出不忍,叹气道:“见就见吧。”

我闻言心中一喜,令呼延寿出去吩咐一声,将画舫靠近游船,挑帘走出船舱,扬声道:“这位仁兄弹得好琵琶,道长一曲高歌也是惊破世间闲鸥鹭,在下嘉兴云无踪,相请两位过来喝杯清茶,不知道两位可肯赏光么?”

那黄冠道士偏过脸来望了我一眼,冷笑道:“我们是贫寒之人,不配作世家子弟的嘉宾,阁下既是祖籍嘉兴,当知日前嘉兴遭劫之事,可是贫道不见阁下有悲愤难言之态,却在这仲春时分,嬉游湖上,当真是没有心肝之人,这等薄情寡义,怎配和我们说话。”

呼延寿闻言大怒,双目炯炯望着那道士,双手紧握,指节发出轻响,似猛虎将欲择人而噬。那道士冷冷一笑,一双利眼毫不示弱地迎上呼延寿的目光,周身透出沉凝的杀气。

那布衣儒士略一皱眉,放下琵琶,也向画舫望来,他身上一缕剑气冲天而起,却不是和那道士的杀气汇合,而是将两人暗斗阻断,虽然如此,呼延寿也是面色苍白,似乎受到重击,不过他心志坚毅,又是常常面对宗师级高手的气势凌逼(小顺子的特训),眉宇间丝毫没有示弱,反而更是露出敌意。那道士被同伴剑气阻挠,他对这同伴素来尊重,却没有生出恼意,但是见到呼延寿竟也能不减威势,倒是心中佩服,眉宇间缓和了许多。

那布衣儒士温和地道:“阁下请勿见怪,敝友性直,多有冒犯,不过我等江湖野人,不便和世家豪门相交,还请阁下见谅。”言辞和缓,虽然暗藏疏远拒绝之意,听起来却不那么刺耳了。

说话之时,那布衣儒士也是目光炯炯地望着对面画舫上面的锦衣公子,心中暗暗探究这人来历。这艘画舫乃是吴州最大的绣庄“撷绣坊”所有,“撷绣坊”几乎垄断了江南五成的苏绣,南楚名绣顾绣娘七大弟子,“撷绣坊”便请到了四名,“撷绣坊”东主姓氏不详,乃是近十余年才兴起的,据说坊主只是一个不到而立之年的青年,眼前这锦衣公子莫非就是撷绣坊主么?可是这人相貌清雅,举止洒脱飞扬,虽然自己的同伴恶言相向,那人却是没有一丝怒容,神色上反而透出宽容谅解之意,从容恬淡之处,不像是斤斤计较的商贾气相,更没有撷绣坊东主鲸吞蚕食的枭雄气度。

这时,那锦衣公子微微一笑,目光从黄冠道士身上移开,转向那布衣儒士望来,这儒士心中一震,这锦衣人双眸有些黯淡,显然神气不足,只是平常人模样,但是凝神看去,却觉得他双眸渊深似海,沉静幽冷,更透着看破世情的恬淡神采。目光流转,这人的面容顿觉生动起来,配合他清秀白皙的容貌,令人生出难辨他真实年纪的感觉。

这布衣儒士本是南楚武林出类拔萃的人物,剑法出众,又是满腹经纶,足智多谋,在南楚可以和他相提并论的不过是数人罢了。他的见识深远更非是常人能比,四目对视,只是一瞥之间已经觉出这锦衣人的不凡之处,眼睛余光望去,自己的同伴似是没有察觉,面上都是不耐之情。布衣儒士心中越发震骇,自己的同伴比自己年长许多,更是饱历世情,竟未看去这人真正的神采,若非是这人隐晦光芒,只是在和自己对视之时才流露出来,就是这人的气宇风标,若非智慧阅历到了一定的层次,根本无法领略。想到此处,他心中不由生出歉意,觉得自己断然拒绝,未免有些失礼。

正在他目中闪过犹豫挣扎之色时,那黄冠道士已经不耐烦地道:“话也说过了,可以走了吧,真是可惜,好好的兴致,都被这些纨绔子弟打扰了。”

布衣儒士眉头一皱,正欲出言阻止同伴恶语,那画舫之上的锦衣公子突然扬声笑道:“等一等!”

那黄冠道士一挑眉,正欲说话,却已经被布衣儒士阻住,他对着画舫一揖道:“同伴鲁莽,多有失礼,尚请海涵。”这一次他眉宇间一片诚心诚意,全然没有方才淡漠疏离的意味。

此时两人相貌皆已落入我眼中,那道士大概三十六、七岁,相貌清奇,但是眉宇间似有深愁,那布衣儒士年过三旬,剑眉星目,英俊儒雅,气度风流,这两人都是气度不凡,这样的人物,纵然是无礼些,我也舍不得不告而杀。方才那声“等一等”非是阻止这两人离去,而是阻止我身后舱中的小顺子出手,小顺子素来对我敬爱,见那道士屡次拂逆,早已生出杀意,只是他早已可以将杀意收敛自如,泄漏的一丝杀意若有若无,除了我这极为熟悉他的人之外,别人多半难以察觉。

向前行了一步,我淡然自若地道:“却是在下失礼了,贸然相邀,既无名贴,也无引见之人,只是在下生平最爱豪迈风流之士,阁下琵琶之声尽述忧国忧民之意,这位道长所唱更是故德亲王最爱的词章,国难思良将,可知道长胸怀。在下虽是庸碌之人,却也感佩两位拳拳之心,故而前来相邀,只是想不到两位如此峻拒,听道长语气,似是不满世家子弟崖岸自高,但是如今看来,想来我们三人之中,崖岸自高的是两位忧心国事的义士,而非是我这只爱安乐的俗人。”

那两人默默听完,那道士面上满是尴尬惊怒,继而又变得有些灰心丧气,反而那布衣儒士目放奇光,面上露出倾慕之色,抱拳一揖道:“阁下说得是,是我们太拘泥了。不过敝友也是情有可原,近日陆大将军欲在吴越练义军,巩固海防,缺少军资,在下和这位兄弟有意说服吴越世家捐助义军,昨日方从无锡返回,却是人人推辞,个个退后,费尽心力,也只募得三成之数。所以我这位兄弟心中烦恼,看到阁下画舫锦衣,便有迁怒之意。”

我闻言略略一惊,想不到这两人竟是陆灿的助力,与他们盘桓会否泄漏身份呢?心思一转,我笑道:“原来如此,两位果然是侠士之风,为国为民。看样子两位想必是准备去吴州募款吧,在下与吴州首富‘撷绣坊’周东主乃是故交,在下之言,他总能听从,若是他肯带头捐资,想必对两位会有所帮助。这样一来,两位总不至于还要拒绝我的好意吧?”

那两人温言目中都是闪过喜色,那道士更是面红耳赤地作揖道:“若是如此,贫道向公子致歉,公子有为国之心,贫道代大将军多谢阁下慨然解囊。”

我笑道:“谢不谢的就算了,两位若是看得起在下,还请过来一叙。”

这一次两人都没有拒绝,也不需跳板,都是轻身纵上画舫,自有船夫去将小舟系在画舫之后,我伸手肃客,将两人请入前舱,自己随后跟入,给呼延寿一个眼色,让他回到后面去,免得他露出破绽。

\chapter{第二十六章 茶香留客饮}

走入舱内,目光闪过,我便是一愣,那站在舱中一角的青衣小厮看身形分明是小顺子,可是容貌却变了许多,虽然只是眉梢眼角的轻微改变,但是却仿佛变成了另外一个人,而且气质也变得平庸,宛若明月被乌云遮掩,旁人绝对看不出他是当世先天高手之一。我知道小顺子是用内力改变面上的肌肉,变了容颜,虽然变化不多,甚至不会让外面的船夫发觉,但是若是认识他的人见了,绝不会认出他是邪影李顺。他为什么这么做呢?转念一想,心中豁然,这小子在江湖上面的名气不小,说不准有谁认得他,不改容貌太危险了,他的心思总是比我细密许多。

目光从小顺子身上一扫而过,只当没有看见他一般,我坐在桌旁,笑着问道:“还未请教两位如何称呼?”

那布衣儒士歉然道:“在下东阳丁铭,这是敝友苦竹子道长。”

闻言我眼睛一亮,这两人我都知道,苦竹子么,曾听小顺子提过,这人本是南楚秘谍,当年小顺子千里追杀毒手邪心,曾放过他一马,后来他无颜再留在大雍,回到南楚之后便销声匿迹,想不到今日竟在这里见到,怪不得小顺子要这么急着改变容貌,这些年来小顺子容貌没有什么大的改变,恐怕此人一眼就能认出他来。至于这个丁铭么,我也是知道的。江南武林之中有四个第一,江南第一杀手无情公子,天下第一神秘人天机阁主,天下第一用毒高手申如晦,最后一个就是吴越第一剑丁铭。曾有人言他的剑法足以称得上江南第一,只是他却谦逊不肯承认。

想来想去,这四个第一,倒有两人和我有关,无情公子是已经离开秘营的逾轮,不知道他现在还能否保有第一杀手的实力,天机阁主不就是我自己么,至于这吴越第一剑丁铭,曾经屡次阻挠过大雍秘谍意图控制江南武林的举动,已经是司闻曹登录在册的人物。凤仪门虽然迁至江南,但是由于过去和江南武林的纠葛,失去了梵惠瑶、闻紫烟这样的高手,且名声尽毁,在江南武林立足十分困难,最后是凭着武力女色掌控了一批黑道高手,才勉强恢复了部分实力,更别想像在大雍一般领袖武林,江南白道上,只有这人才称得上领袖人物。

真是太巧了,居然让这么两个人物上了我的船,我露出热诚的神色,拱手道:“相逢也是有缘,两位都是朱家郭解一流的人物,今日得见,三生有幸,李二,去取周东主刚送来的那坛惠山泉,再取那包新茶过来,我这位家人的茶道可是极为出色,又是新采的吓煞人香,才敢请两位品尝。”

丁铭含笑道:“震泽湖所产的吓煞人香已是好茶,且有天下第二泉之水,听来也令人觉得心旷神怡,云兄这般活法却是逍遥自在,在下枉称逍遥,却是俗事羁绊,不能自拔。”

我自然知道丁铭话外之意,大笑道:“丁兄这是嘲讽我了,孰不知人生如梦,若是坚要清醒度日,最是痛苦难当,方才道长责我不为乡梓遭劫忧心,却不知我纵然肝肠寸断又有何益。天下一统,乃是大势所趋,所差之处无非是以南统北还是以北统南罢了,不论谁人登上至尊之位,受苦者还是我们这些平民百姓。何况纵然战国鲁仲连在世,也不可能说服雍帝放弃南征之心,更是不可能说服南楚君臣束手就擒,无论如何,战乱兵燹已是难免,我非贤哲,只能随波沉浮,无力抵御尘世骇浪,这次雍军不曾血洗嘉兴,已经是不幸中的大幸,想来还是我那位同乡尚念故土之情,否则只怕吴越繁华之地,将成修罗血海。”

那道士闻言神色一冷,厉声道:“俱是你们这般世家子弟,豪门富商,只知有家,不知有国,否则我南楚坐拥半壁江山,有蜀中、荆襄、江淮之险,又有宁海、定海两大军山水营,岂会落到今日四处受敌的下场。云公子可知道,我南楚水军与雍军在杭州湾已经大战两场,皆是未分胜负,而荆襄局势也十分紧张,南阳军再度围攻襄阳,蜀中雍军也是蠢蠢欲动。而我南楚世家却仍是醉生梦死,上元日天机阁在建业举行竟宝大会,一方水晶龙璧竟以二百万两出售,君臣上下,豪奢成风,坐视民间疾苦,南楚若亡,俱是尔等之过。”

丁铭一皱眉,他知道苦竹子自从昔日返回南楚之后,便被解除军职,流落江湖,心性不免偏激许多,平时倒也罢了。但是此刻却不妥当,这神秘云姓公子想必在吴州有着暗藏的影响力,如果得罪了他,吴州募捐将成泡影。足下轻踢了苦竹子一下,歉然道:“云兄深明时势,豁然通达,想必这天下之争在公子来说只是无谓之事,我等都是世俗之人,实不忍见雍军铁骑,踏碎江南半壁,如今两国南北对峙,若论兵力,南楚不如大雍远甚,可是若论疆土财力,南楚并不逊于大雍,若是能够划江而止,倒也是一件幸事。何况我南楚虽然暗弱,却也有大将军这样的擎天玉柱,淮西、扬州两战,便令雍军重创,如今虽然雍军再度开战,可是若有大将军树起帅旗,南楚军民戮力助之,胜算可期,公子有意资助吴越义军,不也是心怀国事的表现么?苦竹子,云公子非是那些平庸之辈可比,还不谢罪。”

苦竹子闻言只得起身谢罪,我也是起身还礼,笑道:“苦主道长所说也无甚差错,水晶龙璧长二尺,宽高皆是一尺,上面雕刻了一百零八条蟠龙,若置于灯火之下,璀璨夺目,群龙活灵活现,仿佛将要破壁而出,更有晶璧之中的细纹,宛似重重祥云,这样的龙璧,乃是无价之宝,在下曾得一观,也是难舍难分,只可惜如今已经被人购下,如今想必已经深锁重楼,不能再见天日,当真可惜可叹。”我一边打趣苦竹子,一边不由佩服这丁铭之才,先是委婉地指责我不关心国家兴亡,然后又暗示苦竹子我向义军捐资便是好的征兆,当真是面面俱到,南楚俊杰之多,当如群星闪耀,只可惜却为浮云遮掩,若是南楚朝廷政治清明,当真不可攻打啊。

苦竹子听得一阵郁闷,却不愿再说什么冲撞的话,倒是丁铭目光一闪,能够有资格参与天机阁竟宝大会的,必是南楚有名的富商世家主事之人。

这时候,小顺子已经取来紫砂茶具,两包茶叶,以及一坛密封的泉水,我便转移话题道:“品茗不可无乐,今日既有嘉宾,就让在下抚琴一曲,以助雅兴如何?”

丁铭也正想暂时转移一下话题,便道:“正欲闻阁下琴音,尚请赐教。”他进来之时,便已看到舱内有琴台,他也是雅擅音律之人,自然知道乐声即心声,他本已觉出此地主人神秘莫测,故而也有心探测。

我虽然知他心意,却不担忧,走到琴台之旁坐下,抛去俗念,一心只去想着淙淙流水,十指轻拂,琴音响起。丁铭仔细听去,只觉那琴音似是细细的雨滴自天际而降,继而流入山间清溪,漫过山石,越过树根草茎,如织的溪水汇成河流,河流汇聚成湖泊,应和着舱外湖水激荡,融为一体,不分彼此,令人听来只觉是天籁,不似丝弦之声,琴声中更是透着洒脱不羁,自在逍遥之意一听可知。

这时,小顺子便在一旁慢慢地烹茶,每个步骤都作的精致无比,仿佛也是应和着琴音一般,每一个动作都是那样分明,优雅从容,待到琴音终止之时,茶香袅袅,已经溢满舱中,小顺子分了三盏茶,用晶莹剔透,几乎透明的雪色瓷杯盛了送上,趁着杯色,茶汤便似无瑕玉珀,或而绿或而深绿,深淡之中,烟雾如织、茶香泄泄,当中的茶叶却有的卷,有的呈片状。

丁铭端起茶杯,便是微微一愣,他是吴越之人,又是常年四处游走,震泽湖东山碧螺峰所产的吓煞人香并不陌生,这种茶叶的特点便是条索纤细、卷曲成螺,满身披毫,银白隐翠,香气浓郁,滋味鲜醇甘厚,汤色碧绿清澈,叶底嫩绿明亮,可是如今这盏茶中却显然混入了另外一种名茶。心中生出好奇之意,将茶水一饮而尽,只觉滋味变幻莫测,更有一种香醇滋味。细细想来,那种茶香却是有些陌生,不由簇眉深思。

苦竹子虽然今日多有心浮气躁,但是他本也是南楚秘谍中的魁首人物,听到丁铭暗示之后也变得冷静下来,他本是黄冠道士,平素多有品茗养性的时候,又是曾经走遍大江南北,天下名茶,他倒是知道不少,饮下茶水,思索片刻,道:“这是信阳毛尖混和了吓煞人香,好茶,好心思。”

我也饮去杯中茶水,笑道:“李二最善烹茶,天下名茶,他见过十之八九,今次的吓煞人香采得过早,刚过春分而已,所以不免多些轻浮之意,故而他才以信阳毛尖相辅,道长能够一语道破,也是茶道中人。”

丁铭目光在小顺子身上一转,只觉得这仆人面容平凡,虽然沉默寡言,但是双眸清冷冰寒,烹茶奉茶都是娴熟干练,凡是世家豪门,多有这种佳仆,甚至是世代主从,不离不弃,云公子身边既有这种仆从,显然身世不凡,而且他和撷绣坊主既是故交,理应有着相近的身份,但是嘉兴未听过有云姓大族,心中更添了几分疑惑,便出言试探道:“云公子既然是嘉兴人,想必见过如今正在攻略吴越那人,不知道公子觉得他是怎样一个人?”

我笑道:“这倒是难为我了,我虽生于嘉兴,但是自幼家境贫寒,族人寥落如寒星,江哲其人,据说也是自幼离乡,且是荆氏旁宗,这样的身份地位,纵然同在嘉兴,又哪里有相识的机会。丁兄若想知道他是怎样的人,也不需问我,只需听听街谈巷议也就知道了,不过在我看来,他是一个有福气的人,娶得如花美眷,深得雍帝信任的,这样的好运世间几人能有?”

丁铭眸中寒光电闪,道:“原来云公子也是出身寒门,想来今日能有这般成就,必是经过千辛万苦,只是公子身家基业想必都在江南,却不担心在战火中付之一炬么?”丁铭心中思量再三,这位云公子听他语气竟不是名门世家子弟,此人的气宇风标,绝不是庸碌之人,见他排场,又是豪富之人,那么这人身份就有趣得很了,不能轻轻放过。更何况他久在吴越,却不曾知道这么一个人,又怎会甘心含糊下去呢。

我淡淡一笑,道:“不惜身家基业的又何止我一人,南楚数代国主,除了武帝陛下之外,都是最不惜基业的人?”

丁铭沉声道:“公子何出此言?”

我望向窗外,淡然道:“晋朝立国以来,朝廷选士以德行门第为主,所谓德行,皆是世家吹捧,所谓门第,更是将寒门庶人拒之门外,结果国力日益衰退,为蛮人破了国都,帝后皆*死。太子南渡,立建业为陪都,苟延残喘,人称其后的晋廷为东晋。如今的南楚王宫,多半仍是当日修建的陪都皇宫遗址。虽然最后中原将士将蛮人逐了出去,国都迁回长安,但是选士的方式仍未改变。其后不过百年,东晋便四分五裂,武帝陛下承袭了江南沃土,立国称帝,改以科举制度选士,选拔将领更是不拘一格。可惜为了大业,武帝被迫和江南世家妥协,放手部分权力,换取世家支持,但是以武帝的雄才大略,那些世家不敢过分阻挠,其时南楚朝中皆是俊杰,不拘出身来历,不问道德文章,乃是南楚最兴盛的时候。可惜武帝立国不到七年,便不幸崩逝,灵王继位之后,世家势力重新抬头。之后三代国主,皆是浑浑噩噩,只知平衡世家之力,以保王位不失,科举选才变成形式,更将以策论选才,变成以诗词歌赋争胜。而且就是高中金榜,若无世家支持,纵有惊人才能,也不能晋身朝堂,朝中人事更替,多半都是世家争雄的结果,贤能列为下陈,庸才却为高官,南楚人才凋零,多因于此。国主尚且不知奋发以守基业,何况我们这些普通百姓呢?”

丁铭眼中闪过黯然之色,他本是寒门士子,读书不成方学剑,虽然成了有名的剑客,但是在世家眼中不过是个武夫,虽有报国之志,却无进身之阶,但是他仍然说道:“国主年幼,尚未亲政,尚相秉政,虽然才具平平,但是朝局尚称平稳,尚有陆大将军选贤任能,以保疆土,若得大贤相辅,未必没有转机。公子真知灼见,世所罕见,若肯为国家效力,必是一代名臣,为何还要沉埋民间,韬光养晦。”

我冷笑道:“丁兄若真是这样想的,那么在下倒是要送客了。若说国主年幼,只是未亲政之过,丁兄想必不知道,水晶龙璧如今就在大内藏宝阁内。且自从显德二十二年建业被李贽攻破,朝中秉政世家皆遭兵燹,只有尚氏因祸得福,一统朝纲,这十年来朝廷上岂止是风平浪静,根本就是尚氏的一言堂,只可惜尚氏才能不足,目光短浅,不知趁机执行新政,削弱世家在地方上的影响,唯才是举,加强国力,反而任人唯亲,不问贤愚。当年朝堂上还有可观之人,如今除了一二人之外,不是尚氏附庸,就是碌碌无为之辈。陆大将军虽然如你所说,选贤任能,可是兵部掌握在尚维钧之手,在军中想要升任校尉,便需兵部文书,陆灿虽然有心,可是这些年来又有几人能够从士卒升为将军。而且陆灿也不过能够在他亲领的军中这般选拔人才,就是陆氏嫡系将领军中,升迁也多半和家世派系相关,这一点就是陆灿本人也无力改变。若非如此,丁兄这等豪杰,为什么胸怀报国之志,却不曾投身军旅,效力疆场,只肯在野襄助呢。”

丁铭叹息再三,终于不语,这锦衣公子所说之言无一不真,却是没有办法辩驳,只能道:“国家兴亡,匹夫有责,公子所说虽然有理,但是现在局势紧张,我等也不能坐视雍军南下,尤其不能眼看雍军肆虐吴越,离散无数骨肉。只可惜吴越世家商贾未受波及者却多半畏惧雍军,不敢捐资筹建义军,当真可惜可叹!”

我见他如此,便顺着他的口气道:“丁兄这却是不知道世家商贾之人的心思了,这些人心中只有利益二字,若非如此,怎会私航贸易成风,朝廷律令在吴越之地多半是一纸空文,就是尚维钧,不也是想尽办法将心腹之人安排到吴越主政,暗中进行私航贸易么?这些人心目中利益比什么都重要,若是出资筹建义军,义军再被朝廷控制,则吴越再不能像从前一般不受建业政令约束,这才是他们心中的忌讳。而且吴越世家最大的利润来自远洋贸易,余杭正是吴越之地最大的港口,如今却被雍军堵住,吴越世家在两军胜负未明之前,自然不愿过分得罪雍军。”

丁铭心中原本只有社稷黎庶之念,对于这些世家商贾的私心自然考虑不周,但是他也是聪明之人,略一思索,已经明白其中道理,他蹙眉道:“可是定海为雍军所夺,远洋贸易必然中断,吴越世家理应有心逐走雍军,重开海运才是?”

我笑道:“若是南楚可以在短期之内取得大胜,吴越世家自然会大力支持,但是东海水军名扬天下,一旦占据定海之后,纵然陆大将军有天纵之才,没有数年也不可能取得决定性的胜利,这样一来,未来数年的僵持局面不可避免。这于一来,吴越海航也将受到极大影响,余杭海运断绝之后,吴越中小世家、普通商贾便要欲哭无泪,但是势力庞大的世家商贾却可以通过宁海进行私航贸易,当今天下两大船行,海氏乃是大雍势力,南闽越氏却仍然归属南楚,越氏自然会乐于和吴越世家合作贸易,就是海氏也不会拒绝这样的私航贸易,毕竟吴越所产的货物在大雍朝野也是极受欢迎的,而且因为货物数量的减少,价格反而会上涨数倍,对于那些人来说,利润并不会降低多少,反而有了垄断商道的可能。只是私航贸易不论是北上高丽,还是南下南洋诸国,都需经过雍军控制的水域,与雍军秘密修好,便成了重中之重,这种情形下,却让他们怎敢得罪雍军呢?”

丁铭听到此处,心道,这位云公子必是出色的商人,才能对其中关节一清二楚,这些事情我却是闻所未闻,而且此人与“撷绣坊”关系非浅,见他气宇风标,那周东主又如此巴结,远远地取了惠山泉送来也就罢了,尚未到最佳时候的吓煞人香也赶着送来,说不定这人就是“撷绣坊”的后台。心中起了这样的想法,他越发有意问道:“那么以公子之见,应如何说服吴越世家支持筹建义军呢?”

我毫不犹豫地道:“商人既然逐利,便需以利动之。陆大将军势必不能久留吴越,一旦他离去,若是没有义军协助楚军巩固吴越海防,雍军必然再度登岸劫掳,若是雍军在吴越连连得手,纵然肯开启私航贸易,吴越世家也只是为人作嫁罢了。敌对双方合作,一方若没有足够的实力,就不能在合作之时占据上风,所以对吴越世家来说,只有将雍军逼退到海上,才有商谈的可能。而且吴越世家本就各自有家将私兵,若是担心义军被朝廷控制,伤及他们的根基,何妨将私兵混入义军之中,这样义军就可以在吴越世家控制之下,不至于成为朝廷肃清异己的工具。”

丁铭皱眉道:“这样一来,虽然义军能够成功筹建,可是却不免沦为吴越世家的私人武力,将来必有后患。”

我笑道:“丁兄既然有意相问,我不过是随便说说罢了,这不过是应急的策略,若不如此,难以令义军迅速成形,至于能够控制义军不过是说服吴越世家的借口罢了,真得实施起来,却有许多微妙之处可以斟酌,却不知到头来是谁占了上风。以在下想来,若是组建了义军,纵然人心不齐,凭着大将军的本事气度,想必也难不倒他。而且陆大将军文韬武略,都远胜于人,或者有更好的办法吧!”

丁铭暗暗点头,觉得云无踪所说极有道理,抬眼望去,这位云公子轻摇折扇,神色淡定,眉宇间透着坚定自信的光芒,显然对自己的判断确定无疑,对陆大将军陆灿也是十分尊重敬佩,这样看来他对南楚并不是像他所说的那般失望透顶,若是用大义相责,或者能够说服他替国家尽力,最不济也可得到他的帮助指点。而且此人如此气度才能,若是埋于草莽岂不十分可惜。想到此处,正想出言劝谏,只见云无踪眸中满是笑意,挥扇从容问道:“以丁兄之见,吴越之战,雍军和南楚谁的胜算高些,我那同乡可真有本事鲸吞吴越之地?那人虽然是名声远扬,但是却多半都是阴谋诡计,这堂堂正正的征战,只怕他也没有什么法子吧?”

丁铭闻言,越来想要说的话却咽了回去,心中涌起无限感慨,叹道:“云兄对朝廷弊政看得一清二楚,对大雍的强盛想必也是心中了然,大雍素来国力便在我国之上,七八年之前那场平汉之战,虽然交战双方也是死伤叠籍,但是大雍却没有伤到元气,事后又将北汉国力全盘消化,就连当初的嘉平公主,也成了如今的齐王妃,大雍国力有增无减,而趁势谋反,想要夺取天下的东川庆王,却成了最大的笑话,那一场莫名其妙的平叛,如今想来也是让人觉得匪夷所思。谁会想到锦绣盟竟然在阵前倒戈,锦绣盟在旧蜀之地一向神出鬼没,就是大雍和我南楚多次清剿,也是毫无结果,更和两国都结下深仇大恨,这一点人人深信不疑。可是这样一个声威赫赫,极其严密的组织,却是早已被大雍明鉴司渗透掌控,轻而易举将庆王李康制住。声势浩大的锦绣盟转眼间烟消云散,明鉴司主事夏侯沅峰名扬天下,就连蜀中也为之震动。若非陆大将军趁着东川尚未平定之时袭取了葭萌关,只怕几年前雍军便已攻入蜀中了。蜀中如今虽然安稳,襄樊、江淮之地却是时刻悬着利剑在头上,大雍带甲百万,淮西、扬州两场大败并未损伤筋骨,一旦雍帝将从前驰骋北疆的猛士调到江淮来,只怕就没有这么容易对付了。更令人头痛的是,雍军却又别寻蹊径,从海上攻来,吴越危殆。我南楚徒有人口千万,半壁江山,却是处处都要设防,处处都有敌军,我虽无甚军略,也知道什么是备多而军分,武学中也有柔不可守的道理,久守必失,还击却又无力,如之奈何?大雍南楚孰强孰弱,已是昭然若揭之事。

至于公子问及江哲江随云其人,其实就是在下不说,公子也知道此人厉害,虽然朝廷民间一味轻辱贬低此人,可是只要是有识之士,怎会忘记昔日攻蜀之时,此人献策献计,襄助德亲王连克坚城,最后更是逼死蜀王,除去蜀中隐患。虽然因为事后他卧病隐退,令人渐渐忘记他的光彩,但是天下谁又敢忘记他?我曾见过他因之被贬的《谏晋帝位书》,策中尽述南楚之危,其中便涉及吴越,指责吴越守军不修甲兵,吴越世家不奉建业律令,一旦有事无以对敌,只是若非今日之变,南楚恐怕无人能悟其中真知灼见。以在下之见,德亲王最失策之事,就是身后遣刺客刺杀此人,若非如此,这人或者还会顾念南楚,而不是今日带兵来攻吴越,毫无故国之念。”

丁铭说及此处,已是不假思索,此言一出,舱中一声脆响,众人看去,却是苦竹子捏碎了手中茶杯。丁铭欲言又止,这时,小顺子已经提着刚刚煮沸的泉水准备前来续水,对苦竹子损毁价值不菲的茶杯的举动,他连眉毛也不曾稍动一下,只是又奉了一杯茶过来,这却是方才特意多分出的茶汤,还顺手递过方巾,苦竹子赧然一笑,用方巾擦去手上茶末,眼中露出歉意,小顺子却径自替众人续水去了。

丁铭见状心中一宽,又接着道:“姑且不论此人军略如何,只是他一人在定海,便牵制了陆大将军不敢轻易离开吴越,这等威势,就是平常人也知道其中深浅。”

我微微一笑,目视第二泡的茶汤,其色愈加莹碧,口中却道:“既是如此,吴越之地,多得是轻锐敢死之士,为何不仗剑除奸。此人曾在翰林院待了多年,又是博闻强知之人,想必对南楚各处地理郡治军事一清二楚,观此人行事,指顾之间翻云覆雨,又得雍帝信重,若是杀了此人,岂不是消去莫大隐患。”

丁铭叹道:“谈何容易,此人虽然是文弱书生,却有一先天级数的高手侍奉左右,”说到此处,他看了苦竹子一眼,见他神色黯然,却没有冲动之意,方继续道:“更有雍帝亲派的虎贲侍卫保护,出入之时,前呼后拥,关防严密,岂有行刺的机会?”

我看了一眼他身后佩剑,道:“虽然这人身边防范严密,但是若有人甘心赴死,效仿聂荆之流,也未必没有机会,那人身边虽有高手,但是南楚也未必没有可以匹敌之人,就如丁兄,一身剑气,含而不露,若是殚精竭虑,行博浪一击,也未必没有机会。”

丁铭苦笑道:“我等学剑之人,首要诚心正意,此人虽然投了大雍,可是无论怎样看来,也没有什么过错。且不论他投雍是在免官之后,又是被俘虏至雍都,身不由己,就是别种情形,一个才华绝世的谋士,遇到雍帝那样的明君圣主,解衣推食,推心置腹,怎能不感激涕零,心悦诚服。这人投了大雍,在下反复想来,竟是想不出一丝可以责备他的理由,纵然是那人站在我面前,我也无法问心无愧地向他行刺。更何况若论武功,在下虽然小有成就,却也不敢和邪影李顺相提并论。我虽然习剑多年,但是却不曾转战天下,徘徊生死,如何能比得上那些历经生死的真正高手。江南武林无甚风浪,这些年来竟是没有先天高手出现,怎比得北地高手如云。那人身边,纵然没有邪影李顺,虎贲侍卫,难道就没有少林高手,魔宗弟子么?想要行刺此人只是痴人说梦。”

我垂下眼帘,饮去杯中茶水,道:“丁兄果然是俊杰之才,行刺敌酋多半是想要以弱胜强的无奈之举,如今两军对峙杭州湾,若是陆大将军能够以堂堂正正之兵攻破定海,就可以消除祸患,这才是光明正大的战策。丁兄为国为民,乃是侠之大者,却令在下深深敬佩。”

丁铭起身一揖道:“云公子既然也这样觉得,为何不替国家效力,陆大将军为人谦抑,礼贤下士,若是知道有公子这样的人物,必然倒履相迎。”他目中满是期望之色,令人几乎不忍心拒绝。

我摇头微笑道:“在下本是闲云野鹤,生平不问国家大事,平日往来大江南北,惯了对月饮酒,临风听琴,若能遇到丁兄这样的人,品茗清谈,就已经是人生最大快事,至于那些征战杀伐之事,我实在无心理会。南北之战,不论谁胜谁负,都是一家一姓之争,和我们这些平凡百姓没有什么关系。丁兄心意,我虽感佩,请恕我不能介入军国之争。不过我在江南还有些力量,若是丁兄缓急之时,可以前来求助。”

丁铭心中黯然,举目望见,只见这锦衣公子神色淡漠,飘逸清雅之处宛似谪仙一般,心道,这样人物,果然不该牵涉红尘之事,罢了,能够得他一诺,已经是难得至极了。转头看去,苦竹子似有不悦之色,连忙使个眼色让他忍耐,自己却道:“是在下鲁莽了,还请公子见谅。”

我见他知情识趣,更是生出好感,笑道:“丁兄体谅在下苦衷,在下甚感宽慰,只是还请丁兄不要对人说及在下之事,在下不愿多生事端。”

丁铭微微一愣,这个要求虽然合理,可是这人神秘莫测,若是自己隐去此人之事,未免不妥,因此只是唯唯道:“在下自然不会对人说起。”苦竹子知他心意,只是默然不语,他们两人的小动作我自然看在眼里,我也不甚在意,这样的局势早已在我料中。

故意露出欣然愉悦之色,站起身来,接过小顺子手中水壶,亲手替两人续水,滚泉入杯,虽然不如小顺子手法精湛,却也不致于水溅茶飞,然后更是亲手捧了茶杯递给丁铭和苦竹子,两人都是起身双手接过。

虽然双方心中都有各自的机谋,但是此刻三人对视,却也是觉得今日一会,甚是畅意自在,相视一笑,各自饮茶。我们残茶入腹之后,小顺子开始撤去茶具,舱中颇有曲终人散的意味。我走到琴台之侧,轻拂琴弦,琴声铮铮,尽述离别之意。虽不言语,丁铭素擅琵琶,精通音律,自然听得出琴中送客之意,站起身来,正欲出言告辞,却突然觉得手足再无一丝力气。

他目中闪过骇意,连忙运起真气,却是一丝也提不起来,只觉得浑身百骸如浴春风,有一种暖洋洋软绵绵的感觉,如饮醇酒,不能自拔。双足一软,跌倒在椅上,只觉得浑身的力量都在逐寸逐分地散去。勉强回过头去,只见苦竹子不知何时已经晕倒在椅中,面色微红,似是好梦正酣。

眼中神光电闪,丁铭却想不出自己是如何中了毒的,困倦之意涌上,他恨不得立刻睡去,但是心中却明白自己是受了暗算,无论如何也要问个清楚,不能这样不明不白地晕睡过去。他勉力咬破舌尖,一口鲜血喷出,额头渗出滴滴汗珠,脑中一清,他艰难的问道:“云兄,你这是何意?”

那背立抚琴之人回过头来,眼中似有惊讶之色,笑道:“丁兄何必这样苦苦支撑,只要放松自己,便可安然入梦,再无辛苦。”

丁铭一手紧紧握住椅臂,道:“云兄是何时下毒的,为何在下并未发觉。”说到回来,疼痛的感觉渐渐消散,晕眩之感再度袭来,他睁大眼睛不肯合上,只怕一闭上双目,就会沉沦不起。

只见那云无踪淡然道:“今日相逢本是偶然,品茗谈心也是平常之事,只是你我言语投契,在下不免多说了一些不该说的话,若是往日,你离去之后,我便可以束装上道,纵然阁下想要追踪,也是有心无力。但是今日不巧,我尚要留此一夜,若是阁下有心探测我的行踪,不免多了许多麻烦。为了解决这个难题,在下在最后一杯茶中下了些安眠药物,请两位在画舫之上酣睡一夜,等到明日红日高起,两位便可回到人世间了,丁兄苦苦支撑,又是何苦来由?”

丁铭只觉得意识渐渐向黑暗沉沦,他勉力向那锦衣公子看去,心中隐隐觉得,此次一别,恐怕再也没有机会见到这神秘莫测的云公子,更是不愿错过最后的机会了解此人。只见云无踪轻叹一声,怅然道:“今日一别,后会无期,丁兄人品出众,意志坚强,在下心中敬佩,在下承诺之事,绝不会失言背信,只是丁兄若是将我的事情到处宣扬,在下恼怒起来,可就不知道会发生什么不愉快的事情?为了丁兄着想,今日之事还请保密才是。”听到此处,丁铭终于再也支持不住,朦胧中只见那人缓步向自己走来,耳边传来那人淡漠惆怅的语声道:“天意从来高难问,相对陶然共忘机” 然后,丁铭便陷入了最深沉的黑暗之中。

\chapter{第二十七章 还如一梦中}

还未睁开眼睛,丁铭便觉出异样来,昏倒之时本在画舫中,但是此刻却觉得湖风轻拂,身上冰凉,耳边就是湖水激荡之声,身下更有飘忽不定之感,他不敢轻动,先将身体调整到可以随时出手的状态,更是用六识去感受身边的情形。但是除了湖水之声,就只听到不远处传来一个均匀平缓的呼吸声,确定身边并没有危险的存在,他缓缓睁开眼睛。只见自己躺在原本的轻舟之上,对面缩在船尾酣睡的便是苦竹子,撑船的竹竿仍然在他手中横握。而自己却是伏在船头,琵琶放在身边,佩剑仍然系在身上。丁铭心中生出莫名的感觉,好像昨日并没有人邀请自己两人到画舫上品茗,更没有人和自己争辩谈论。自己两人不过是在湖上睡了一夜罢了,那天籁一般的琴声,香气四溢的新茶,还有那优雅睿智的神秘云公子似乎都并未存在过,恍恍忽忽似是黄粱一梦。

他翻身坐起,忍不住舔舔干涩的嘴唇,却觉得一阵刺痛,却原来是不小心碰到了咬破的舌尖,虽然鲜血早已凝固,但是仍然有疼痛之感,直到此刻,他才相信昨日发生的一切并非是梦境。运起真气,行功一周天,他能够感觉到身上并无任何异样,真气如珠,流畅自如,更是没有丝毫窒碍。而且他也丝毫没有中了迷药之后的头昏脑涨,反而觉得神清气爽,若非可能受了一夜寒风,伏地而睡的姿势又不甚妥当,只怕就连腰酸背疼的感觉也不会有。他伸展一些有些麻涨的四肢,准备去叫醒苦竹子,却有一物掉落在甲板上,发出一声清脆的响声。他仔细看去,却是一块晶莹润泽的白色玉佩。

丁铭下意识地拿起玉佩一看,只见玉佩正面是雕功精美的图画,绘的是云海茫茫中隐约矗立的仙山楼阁,而在玉佩背面,更有两行铁划银钩的小字,“天意难问,机深虑远”。丁铭心中一动,回忆起自己昏迷之前,听到那云无踪所念的两句诗,反复吟咏数遍,丁铭心中突然一动,眼中放出光彩。云无踪如此人物,岂能默默无名,想不到自己竟然有幸见到江南武林最神秘的天机阁主。

天机阁纵横江南已经有十余年了,其势力却如冰山一角,令人永远难以揣测它的深浅,也只有云无踪这样的人物,才配得上天机阁主的身份,而自己竟然有幸和这样的神秘人物品茗清谈,更得他承诺相助,丁铭心中激动难抑,只觉得天地间豁然开朗。对于云无踪使用迷药将自己制住,更是没有一丝怨言,就是自己身为天机阁主,也必会如此做的,虽然揭示了身份,却绝不会将自己的安全交给别人掌握。

这时苦竹子也已经醒了过来,他却是不似丁铭那般生出错觉,曾经身为秘谍的长处显现出来,一睁开眼睛,他便森然道:“我们中了暗算了,丁兄。”

丁铭笑道:“何止是中了暗算,我们简直是被人玩弄于股掌之上呢?”

苦竹子一愣,丁铭说出这话时,面上却是笑意盎然,完全没有一丝怒意,他也是精明之人,目光一闪,便已落到了丁铭心中紧握的玉佩之上,丁铭将玉佩递了过去,苦竹子目光闪动,不久,用略带试探的语气道:“莫非是天机阁中人?”

丁铭也是颇为佩服苦竹子的心思灵敏,道:“我想定是如此,那云无踪十有八九就是天机阁主。”

苦竹子想了半晌,只觉得那云无踪身上种种谜团都迎刃而解,既是天机阁主,能有这般豪奢享受更是理所当然。自称非是世家出身,却有着不亚于世家子弟的气度,身边有训练有素的忠仆侍奉,又有气度森然的高手护卫,能够被“撷绣坊”周东主奉若上宾,曾经见过水晶龙璧,对其下落了如指掌,这种种令人难以揣度之处,只要认定这人是天机阁主,便都是理所当然之事。而且此人气度见识,当世罕有能够匹敌之人,却又默默无闻,殊不可能,若是他是天机阁主,那么若没有这样的本事,反而令人怀疑他的身份了。最重要的一点,云无踪言谈之中,对于时事了如指掌,却对两国之争无甚兴趣,不偏不倚,这也符合天机阁的形象,天机阁历来不甚关心国家之争,虽然表面上倾向南楚,但是对于大雍似乎也没有过分的排拒。

想通之后,苦竹子脱口而出道:“这件事情应该告诉大将军。”他这样说却是因为,早年他仍为秘谍之时,就曾经奉命探测天机阁之秘,毕竟天机阁巧夺天工的机关暗器,种种匪夷所思的奇妙构思设想,都是令人垂涎三尺的,就是南楚和大雍的军方也不例外,可是十余年来,天机阁仍然时隐时现,纵然一时被人占了上风,损失了一些力量,但是接之而来的惨重报复,足以令任何人胆寒警惕。结果纵然有人发觉了天机阁的一些行踪线索,或者是不敢打草惊蛇,或者是投鼠忌器,都不敢随便出手,往往在极短时间之内,线索就会被人斩断。事实上,在无法将天机阁势力一网打尽之下,任何势力也不敢对天机阁动手。更何况天机阁虽然实力强大,却并不专横,也没有独霸某种行业的野心,与之合作,能够得到发展壮大的机会,与之为敌,却是家破人亡的下场,这种情况下,还有多少人能够鼓起勇气和天机阁为敌。在南楚,天机阁就是这样独特的存在。

可是如今却有机会将天机阁控制住,那从未露面的天机阁主居然露了真相,换了旁人或者没有能力对付,但是若是陆灿,南楚军方势力最大的将领,却有力量对付一个不再神秘的人。

但是苦竹子话一出口,丁铭却断然道:“这万万不行,一旦如此,只怕就有祸事了?”

苦竹子露出疑惑的神情,丁铭见状叹道:“苦竹子,你毕竟出身世家,虽然现在成了江湖人,但是有些事情你还是看不穿,对于天机阁主这样的人来说,自身安危是最重要的,他既然已经要求过我们不能说出他的事情,若是我们违背了他的意思,只怕他就会成为我们最大的仇敌,你也应该能够看出来,他对大雍并无恶感,如果他一怒之下投了大雍,只怕对南楚来说便是雪上加霜。”

苦竹子反驳道:“可是天机阁一向不问身份来历,昔年有几份重要的兵械设计图便被大雍方面的人购去,与其留下这样一个难以控制的中间力量,不如将它牢牢控制在掌中。”

丁铭摇头道:“苦竹道兄,小弟冒昧地问一句,是否昔年之事对你的打击太重,以至于你不能清醒地认识当前的局势呢?”

苦竹子仿佛被人当头一棒,神情变得骇人,眼中冒出怒火,丁铭凛然道:“道兄当年死里逃生,却被容渊以此理由逐出军旅,这些年来,道长心结始终不去,我们这些朋友也不愿意伤害你,可是今日小弟要问道兄一句,天机阁主能够声色不动地将你我迷昏,若是他下的是剧毒,你我岂不是早已丧命?天机阁主若是那么好对付,又怎能纵横江南多年。若是我所料不差,只怕他早已鸿飞冥冥,更是换了身份姓名,甚至相貌也未必还是这个模样,否则他怎能多年来保持隐秘的身份。他若不防范你我会对他生出歹意,就不会用药物将我们迷昏了。”

苦竹子的面色渐渐变得僵硬,昔年往事一幕幕从眼前闪过,最后浮现的是那个月光下容色如雪的少年,他颓然倒在船上,良久才疲惫地抬起头道:“小丁,谢谢你点醒我,我当真是被心魔所困,是啊,天机阁是什么样的势力,这种时候想要舍本逐末去对付它,岂不是自寻死路,不说别的,有了天机阁的策应,只怕吴越再无海防可言,吴越世家只怕倒有大半和天机阁有着生意上的往来呢。”

见他已经醒悟,而且用当日初见之时的口吻唤他,丁铭心中一宽,笑道:“我们这就去吴州吧,我想撷绣坊周东主应该已经有所准备了。”苦竹子爽朗的一笑,将心中烦恼抛去,拿起竹竿撑船准备向吴州而去,但是他却突然惨叫起来。丁铭一惊,抬头道:“怎么了?”

苦竹子哭丧着脸道:“这些没有天良的家伙,把我们丢在船上也就罢了,怎么却不将小舟系住,现在我们到底被湖水冲到了哪里,我却是也不知道了?”

丁铭闻言,先是愣了一阵,继而哈哈大笑起来,那笑声中满是愉悦之情,他心道,多半是那天机阁主故意而为,说不定就是惩罚苦竹子出言不逊。望向苍天云际,眼前再次浮现出云无踪的洒脱可亲的形容,“天意难问,机深虑远”,这虽是天机阁的来由,可是在那云无踪眼中,却恐怕真正的含义还是“天意从来高难问,相对陶然共忘机”吧。

“阿嚏”我打了一个大大的喷嚏,摸摸鼻子,莫非有人在背后骂我么?不知道是姜海涛还是霍琮,他们两个骂我倒是理所当然的,尤其是霍琮,不过十几岁年纪,就被我丢到战场上,说起来自己也觉得过意不去。或者是呼延寿,从昨天晚上他的脸色就不大好,这也难怪,除非是我到了雍军大营,否则他的脸色绝对不会好看的。或者是小顺子在腹诽我,从昨天晚上我不让他杀人灭口之后,他就一直用冷冰冰的目光盯着我,如果不是我郑重警告他不能瞒着我下手,只怕那两人性命早就没了,现在他只是瞪着我,这已经是很客气了。

这时候,我乘坐的轻舟正向无锡驶去,昨夜,我在南楚的属下全部到齐,就在震泽湖心之中密会,这也是我离开南楚之后唯一的一次,陈稹、寒无计自然在场,秘营弟子除了逾轮之外,也是全部到齐。早在今年年初,我便传令陈稹、寒无计,让他们安排这次会面,并特意说明了我会到场,当然时间和地点都故意含糊其词,更是趁机考验所有弟子的忠诚,这些事情他们本是驾轻就熟,全不需我费心提醒。结果也是令我欣慰,虽然这些年来几乎难以见面,但是他们的忠诚却是未减。

和众人相见之后,我对接下来数年之内天机阁的宗旨策略给了明确的解释,这便是我一定要留在震泽湖数日的原因。虽然天机阁是我一手缔造,秘营更是我最可靠的力量,可是久离必疏,又是大战在即,我不能忽视任何微妙的因素,只有用自己的双眼确定他们的心意,当面说服他们接受我的决定,我才能确保可以如臂使指地控制天机阁,既能够对我有所助力,又不会损害到天机阁的根基。今后数年,两国之间必然是势成水火,消息往来将变得非常艰难,为了安全起见,我将无法像从前一样给他们详细的指令。所以这一次见面,我一定要他们明白我的用意,而这些事情,光用信件是说不清楚的,所以我才要亲自前来。

在我的决定下,天机阁在大雍和南楚相争其间,将要维持中立,甚至可以稍微偏向南楚一些,并不需要他们给大雍提供什么情报,更不用他们做内鬼里应外合,就连原本准备让他们挑动吴越世家支持陆灿组建义军这件事情,现在也有了接手之人,他们只需推波助澜就可以了。等到大雍步步推进的时候,他们只需主动一些合作即可。

这样的决定令陈稹和白义他们都十分惊奇,甚至白义犹豫之后,委婉地说明他们并不介意楚人身份的问题,他们只忠于我一人,但是他们的心意我虽然感动,却不会改变我的决定。

这样的决定,不是因为怀疑他们的忠诚,虽然他们几乎都是南楚人,可是却几乎没有得到过朝廷乡梓的善待,当初我从孤儿之中选拔秘营弟子,就是不希望他们有太多牵绊。这些年来,他们也没有因为我投了大雍有所不满,始终忠心耿耿地为我效命,所以我并不会认为他们会因为故国而生出叛逆之心。但是,即使这些弟子并没有什么想法,我却不能不顾及到天机阁的局限之处。

无论如何,天机阁的根基还是在南楚,若说和敌国有些生意往来,或者想做些不利于朝廷的事情,这对一个神秘莫测的组织来说都是理所当然的,就是和大雍关系密切一些,对于以利益为重的商贾来说也没有什么特别。可是,如果我想让天机阁全力和雍军合作,这就会导致天机阁根基的浮动。天机阁能够神出鬼没,是因为产业众多,盟友遍及江南,可是这些产业中的掌柜、伙计多半都是楚人,那些盟友也多半是楚人。天机阁弟子可以不顾虑南楚故国,可是那些楚人却不能不顾虑,他们可能会在雍军面前屈膝,却还不会铁了心投效敌国。与其令天机阁后院起火,还不如让他们继续在天机阁控制之下,这样也比较容易诱导他们接受大雍的统治。如果弄得天机阁烟消云散,声名扫地,就像锦绣盟一样,我可舍不得,天机阁的产业可是我这些属下弟子安身立命之处,无谓的损失可会令我心痛的。

当然,最重要的一点,超出本分的事情不能做,收集情报,收买敌国重臣将领这些都是司闻曹的职责,我若插手,岂不是越权行事,我可没有打算和司闻曹争功。就像当初锦绣盟的事情,现在想来,我却是有些多事了,监察官员是明鉴司的事情,我却让锦绣盟去多事,虽然结果不错,但是若是因此引起了李贽的不满,可就得不偿失了。而且锦绣盟的事情夏侯沅峰替我背了黑锅,这次若是天机阁成为众矢之的,难不成司闻曹会替我背黑锅么?想来想去,天机阁还是安稳一些好,不显山不露水才是真正的赢家。

正在我浮想联翩的时候,一个蓝衫青年走入舱中,恭恭敬敬地禀道:“公子,无锡飞鸽传书到,一切已经准备妥当,只等公子一到,就可上路。”

我醒过神来,笑道:“山子你在机关暗器上的成就已经不在我之下,这次更是亲自出手,我自然是放心的,断不会误了我的行程,也不会露了破绽,不过上船的时候还是要安排一下,既要避人耳目,又不能让人疑心。”

那蓝衫青年眼中闪过惊喜,对于我的赞赏十分激动,不过接受过的教诲却让他强行抑制心情的波动,应诺告退,临去之时,目光在呼延寿身上一扫而过。

一直在旁边沉默不语的呼延寿心中一叹,这蓝衫青年相貌沉静冷肃,武功显然不弱,见他气度言语,也是出类拔萃,听侯爷对他的称呼,想来也是八骏之属。昨夜天机阁之会,至今想来也是如梦如幻,他虽然没有资格出席,可是却也冷眼旁观到秘营弟子出入。今日想来,仍是赞叹不已,江南之地,果然是地灵人杰,群英荟萃,若是南楚国主也是明君,能够举贤任能,大雍根本没有取胜的可能。

舟行两日,终于到了无锡一处隐秘的船坞,走出船舱,我望着装满粮食的那艘特制货船,心中生出惆怅的感觉,上了此船,就意味着这短短的逍遥时光已经逝去,好梦由来容易醒,唉!

\chapter{第二十八章 乐在相知心}

我要乘坐的货船是从震泽湖出发,沿着江南运河北上京口,这是从无锡向淮东运送粮草的船只,去年秋天在淮东的一战,正是秋收将临之际,因为雍军犯境,以致颗粒无收,淮东被南楚收复之后,两军对峙,更是急需粮草,至少在夏收之前,淮东粮草都要靠江南调度。所以从去年年底开始,从吴越至淮东的运粮船就络绎不绝,有官粮也有私粮,其中从无锡起运的粮船占六成之上。粮行这样的生意多半在世家控制之下,但是这并不妨碍天机阁控制的商行跑一次龙套,在吴越买上十船八船的粮食,运到淮东出售,这是一件很平常的事情,运河上这样的船只络绎不绝,自然不会有人知道其中一艘特制的货船之内,多了几个不该存在的偷渡客人。

这艘货船表面上和普通货船没有什么不动,但是却在设计的时候动了手脚,在舱中加了一个密室,可以装载一些价值不菲的私货,现在,我就是被夹带的偷渡之人,小顺子则成了粮船管事(山子)身边的小厮,他只需改变相貌即可,世间能够看出他深浅的也不过寥寥数人,不必担心有人会识破他的身份。而呼延寿和其他四名侍卫,全被小顺子封了七八成的武功,然后丢到船上去做苦力了。反正换上船夫的粗布衣衫之后,目中神光黯淡,除了身材高大一些,怎也看不出是身居武功的军人。随着东海水军南下的时候,这些人都已经度过了晕船的难关,这一次,我特意先派人训练了他们半天如何行船,只要不胡乱说话,充做船夫杂役倒也勉强可以。这些侍卫都是克尽职守、精明能干的军士,否则也不能被选入虎贲卫,他们若是下起功夫来,等到下船的时候,一定已经是最好的船夫之一了。其实我倒不是不顾及呼延寿的面子,才让他也去做船夫,只是船上的密室小了一些,住一个人还可以,若是再加一个就太拥挤了。

这个密室只有两丈方圆,室内只有一张床榻,一桌一椅,除此之外就只有一小块空地可以供人活动一下筋骨,虽然通风还算不错,甚至还有一个相通的小房间可以盥洗,但是毕竟不够舒适,尤其对我这个享受惯了的人。可是我也是无可奈何,淮东不比吴越,我若是抛头露面出了什么纰漏,想跑都跑不掉,所以只能委屈一下,躲在密室里面了,这也是小顺子当初答应我潜行南楚的条件。想到我需要在这里闷上十天半月,就是叫苦连天,呼延寿他们虽然可怜一些,但是至少还可见到天日,而小顺子更是可以自由自在的在外面游荡,凭他的武功,就是在岸上逛一圈再回来,也不会被人发觉,这样的强烈对比真是令人郁闷啊。

看看嵌在舱壁上的夜明珠,心中生出一丝庆幸,这种密室通风虽然还不错,但是若是长时间点起灯火,却也难以忍受,可是这里没有天光,若是不点灯火,便是伸手不见五指,若是别人藏在里面,自然只能忍受一下。但是山子精灵得很,临时在壁上加了一个小机关,可以嵌入几颗夜明珠,这样一来,室内珠光明亮,虽然不及天光,但是视线无碍,就是想看看书,也不会觉得光线太暗,若非如此,这十几天我可怎么煎熬呢?

放下书卷,我再次轻叹一声,真是寂寞啊,或许是习惯吧,我从前最是喜欢清静的,可是现在却觉得分外不能忍受寂寞。小顺子也真是的,抛下我独自去逍遥了,说来也奇怪,若是他在我身边,就是一天不说一句话,我也不觉得孤单,在榻上翻来覆去了几次,终于忍耐不住,跳下床在地上踱步,转了几圈,越发觉得气闷,恨不得出去透透风,可是想到和小顺子有约在先,途中不能离开密室,便只能黯然神伤。正在我烦恼无比的时候,密室的小门无声滑开,小顺子躬身钻了进来,手上提着一个食盒。

我心中大喜,等小顺子将食盒放在桌上,准备出去的时候,拉着他道:“和我一起吃吧,吃完再出去不迟。”小顺子瞥了我一眼,却没有理会我,只是将食盒里面的菜肴和碗筷拿了出来,我一见却是大喜,竟有两副碗筷,小顺子果然够义气,知道我闷得很,所以特意陪我吃饭,想到此处,连忙拿了两个茶杯放在桌上,又殷勤地提壶倒茶,准备讨好他一下,全没留意小顺子眼中闪过的一丝笑意。

吃完饭后,我见小顺子在那里收拾碗筷,想到他又要出去闲逛,我却是作茧自缚,心中涌起强烈的郁闷感觉,往榻上一躺,翻身向内,瞪着墙壁发呆。过了没多久,便听到小顺子离开的声音,心中越发腹诽起来,他若想离开绝对可以做到无声无息,怎么偏偏弄出这样的响动,不是存心气我吧,不过想想我不许他杀了丁铭二人,却不跟他说原因,也难怪他这样气我。正在胡思乱想,身后传来小顺子冷淡的语声道:“下一盘棋如何?”

我喜出望外,连忙翻身坐起,就连上一次被小顺子杀得汗流浃背,立誓不再和他下棋的事情都忘得一干二净,匆匆道:“不许反悔,至少三盘。”

小顺子微微一笑,已经恢复真容的清秀面容上露出温暖的表情,这可是这些日子罕见的表情啊。

一局棋才下了一半,我便又皱起眉来,看着被小顺子杀得七零八落的盘面苦笑,抬起头来,见小顺子神色和气,我壮着胆子道:“下棋也没有意思,我们随便聊聊天吧?”小顺子目光一闪,淡淡道:“说些什么呢?”

我笑道:“什么都可以,你想问什么,或者想说什么都可以,难得这样清闲,身边又没有外人。”

我心中想着,只要小顺子问起,我就可以和他说明这些日子肆意妄为的缘故,也免得他心里不快。谁知小顺子想了一想,道:“公子当初向皇上提出随水军南下,皇上问公子何故,公子只说想令楚军误会我军主攻方向乃是吴越,今日想来,公子真正的理由不仅如此,一来是想和荆氏和解,二来是分担姜侯的压力吧?”

我捡起一枚棋子,在手中把玩着笑道:“想和荆氏和解倒是真的,虽然就是别人来,也可对荆氏手下留情,可惜我却知道舅父他老人家固执强硬,我若不能和舅父化解心结,荆氏是万万不能为我军所用的,只是皇上必不会放心我回嘉兴,所以我便没有提起。至于分担海涛身上的压力,这话又如何说呢?”

小顺子淡淡道:“东海水军自从归顺大雍一来,这是头一次出战,胜负战绩十分紧要,吴越乃是南楚精华之地,纵然一时得手,也难免遭遇挫败,而且以王者之师,行海匪之策,恐怕易遭攻讦,纵然现在无人说什么,等到日后发作出来,便是一桩大罪。公子相携南下,首议劫掳吴越之策,这样将来若是有人想要以此责难,就要考虑到公子的立场。公子这样做,岂不是替姜侯分担压力么?”

我微笑不语,小顺子继续道:“其实若非东海水军最擅登陆劫掳,纵然公子定下计策,准备了吴越的精确地图,也不可能在短短十余日之内完成这样的战策,若是姜侯没有准备这样做,也不会备下那么多近海战船,劫掳的过程也不会这样干净利索。如今公子虽然得了献策之功,但是姜侯将战策执行得如此完美,已经是不世之功,而公子却将可能的攻击揽于自身,还不知将来是福是祸。”

看了我一眼,小顺子又道:“公子自然也考虑过这样做的后果,将来公子若是失了帝心,也难免会有人以此攻讦公子,可是这些事情公子自然不会放在心上,反而是姜侯,他年轻气盛,若是因此和大雍离心,却是可惜了这支纵横四海的水军。而且只要姜侯无事,海氏船行就不会受到波及,我们便有后路可退,所以公子便顾不得声名了,而是一力承担献策的责任。”

听到此处,我也不由一笑,道:“狡兔三窟,这也是自全之道。”

小顺子微微一笑,又道:“公子若仅是想要留条后路,自可留在定海,等到风平浪静的时候再北方返回中原。可是公子却决意孤身穿越楚境,前往淮东。”

我面上一红,道:“这个我不是解释过了么?”

小顺子道:“公子的确和我解释过了,今年三月江南行辕就要筹建,公子还需去赴任,而一旦南楚军知道公子在定海,宁海水营必定阻住北上之路,短时间之内,公子无法北上,纵有水军护送,也难免遭遇宁海突袭,若是公子滞留定海,不免贻误军机,令皇上对公子当初决意南下的事情不满。为了赶时间,也为了安全起见,不如从陆上走,在天机阁掩护之下,反而安全一些。”

我笑道:“就是如此,我可没有说谎。”

小顺子瞥了我一眼,道:“公子自然没有说谎,只是避重就轻,你要离开定海非是为了江南行辕的军务,而是为了姜侯,有公子在定海一日,姜侯的一切功劳都不免打个折扣,姜侯与公子名为师徒,侍奉公子却是如父如兄,公子自然不愿损及姜侯声威,所以匆匆离开定海。至于留下琮公子,一来是为了造成公子仍在定海的假相,二来也是让琮公子辅佐姜侯。琮公子虽然年轻,但是心性沉稳,姜侯虽然骁勇善战,却是有些气盛,若和陆灿相较,恐怕有些不如,但有了琮公子辅佐,必然可以稳住定海局势,纵然小挫,也不会受到大的损伤。”

我轻叹一声,道:“还有一个理由,你却没有猜到。”

小顺子眉梢一挑,道:“公子是说这次也是为了考验琮公子么?”

我微微一愣,笑道:“这一点你也想到了?”

小顺子道:“琮公子身世不明,偏偏最得公子爱重,总是不忍强行逼问,只是这几年琮公子甚得太子、嘉郡王器重,将来也必会成为大雍重臣,以琮公子的本事才华,就是想要权倾朝野也不是什么难事。这本来也没有什么,只是公子心中担忧他与大雍有隙,这一次特意将他一人留在定海,不似从前一般始终将他约束在身边,他骤得自由,难免会流露出心中所思,公子想必在虎贲卫中已留下暗子,监视琮公子的行径,一旦发觉有什么异样,就可以请姜侯将他软禁起来。定海孤绝海外,琮公子就是做出了什么不妥当的事情,也难以影响大局,而且纵然有事,还可令姜侯相助掩住真相,不令外泄。公子这样行事,既是为了试探琮公子,也是为了万一之时,可以保护琮公子。只盼琮公子能够体谅公子心意,不要做出亲痛仇快之事。”

我闻言喟然长叹,琮儿之事,我已经拖延多年,但是现在却不能继续不闻不问了,太子已经开始涉入军政,若是琮儿果然有些不妥之处,我也要在太子重用他之前弄清楚才行。

小顺子却又有惊人之语道:“这些事情很容易便可明白,只是公子与那丁铭、苦竹子二人相交之事,却令我苦思不解,只是今日突然想明白了,所以也想问问公子是否正确?”

听到此处,我却是大感兴趣,这几日我都以为小顺子为了这件事生气,想不到他却在替我想理由,倒要听听他是否明白我的心意,坐直了身子,面上流露出洗耳恭听的神情。

小顺子淡然道:“初时公子只是见猎心喜,想要和才俊之士一会罢了,谁知两人上船之后,公子得知他们的身份,便有意借重,我本来担心公子这样人物,当世少有,他们若是仔细想去,难免会想到公子真正的身份,所以主张杀了两人,可是公子却不许我动手,只是暗示我在第三次泡茶的滚水中加入迷药,然后亲手续水,将两人迷晕,又留下信物,暗示公子天机阁主的身份。我这才明白公子深意,天机阁主神秘莫测,乃是传奇人物,他们知道公子乃是天机阁主之后,不论是什么蹊跷破绽,在他们看来都是可以解释的,自然就不会想到公子真正的身份。公子亲手续水,是为了让他们误以为是公子亲自下毒,可是他们自然看不出端倪,便会以为公子深藏不露,这样一来,更是不会想到公子是江哲江随云,世人可是都知道公子是文弱书生的。可是我却不明白公子为何费心留下他们的性命,莫非只是为了丁铭那一番肺腑之言么?”

我淡淡一笑,眼中透出狡黠之意,既是为自己灵机一动想出的计策自豪,也是暗笑小顺子只是看到了表面的文章。谁知小顺子也是微微一笑,继续道:“所以这几日我都在冥思苦想,终于被我想通了整件事情事,只因他们要做的事情也是公子要做的事情,而且他们做来更是事半功倍,所以公子才宁可冒着泄漏身份的危险也要放过这两人。只不过手段虽然相同,目的却是天壤之别,他们是要维护南楚社稷黎民,公子的目的却是为了削弱铲除吴越世家。

公子生于嘉兴,天机阁产业在吴越的就有四成,虽然公子流离在外,却始终不曾忘记乡梓,这一次公子献策劫掳吴越,恐怕很是有人诟病公子不念乡梓,却不知公子一片苦心。

在公子心目中,吴越世家实在是最大的障碍,南楚的衰落,一个主要的原因就是王室和世家的相争,对公子来说,世家掌权有害无益,如今南楚其他各地的世家多半凋零,只有吴越之地,反而因为远离战火和远洋贸易,世家的力量越来越大。公子既然投了大雍,自然不希望大雍将来也重蹈覆辙,因此吴越世家必须要被清洗。可是大雍一统天下后,吴越世家必定望风归附,不论真心假意,到时候若是再清洗,只怕江南民心不稳,皇上乃是英主,必然不会纵容吴越世家,吴越世家不肯屈服,必定挑起民变,这样一来,锦绣河山,必将成为血海屠场,舞榭歌台,将成断瓦残垣,几十年之内吴越之地恐怕也难以恢复元气。所以公子苦思之下,才定了劫掳吴越的战策。

这条计策,表面是只是为了削弱吴越的抗拒力量,也是为了定海可以长期和吴越对峙。实则还有三个好处。其一,吴越世家为了担心雍军再次登陆,最后必定组织义军私兵对抗雍军,这样在作战中可以消除吴越世家的武力;其二,双方交战时日一久,就会结下深仇,战况惨烈,死伤叠籍,等到大雍南下之后,却可以用吴越世家抵抗王师的理由对其进行清洗,覆巢之下,焉有完卵,此举光明正大,吴越世家想要挑起民变,也会得不到厌倦战事的平民的支持;其三,公子掳劫嘉兴世家到普陀,可以在数年之内破坏其世家体系,令其成为符合大雍需要的力量,等到大雍一统天下之后,将这些人迁回吴越,他们就成了大雍统治吴越的根基和助力。这样一来,公子既可以达到清洗吴越世家的目的,又保住了吴越千万军民的身家性命,若不是念及乡梓,公子何必这样费尽苦心,甚至不惜担上恶名。

就是公子有意让呼延寿看见天机阁的力量,也是为了通过他警示皇上。吴越之人,虽然性情和顺,骨子里却有轻锐敢死的本质,自古以来,最多刺客剑侠,大雍纵是灭了南楚,可是想要江南稳固,没有十年时间安抚镇压,也是不可能的,公子想必是担心皇上因为吴越的反抗暗流而采用强硬政策,所以才有意无意地警示皇上。只是这样一来,公子岂不是又给自己多了一个阴蓄死士的罪名,又揭示了隐藏的实力,这让我始终觉得有些不安,若是皇上有意鸟尽弓藏,公子何以应对。”

我只觉得心中畅快非常,这些心事我虽然在脑海里想过千次万次,却是不能上不能告天地君父,中不能告妻子亲朋,下不能述与鬼神,只能自己一人苦苦盘算,小顺子虽然亲密,我却不愿乱他心思,这些日子以来,当真是苦涩难言。一路北上,虽然没有见到多少外人,但是也隐隐听到有人议论雍军劫掳吴越之事,提及之人多半将我当成叛国背乡之人,痛加辱骂,这一点虽然在我意中,心中也是凄苦难安。想不到小顺子不需我明言,就能知我心意,他素来除了武学之外,少有关心世事,这一次费心苦思,定是觉察出我心中苦闷,所以才揭穿我的苦衷,用以安慰于我。

强抑心中狂澜,我尽量平静地道:“这也没有什么,天下一统之后,天机阁也该成昨日黄花,其实那些产业早已都分给秘营弟子了,只是现在还挂着天机阁的牌子罢了。这些力量给皇上知道,也没有什么关系,等到大雍一统天下之后,我纵有再强的力量,难道还能胜过朝廷么?与其私蓄武力自保,还不如散去这些力量,这样才不会引起皇室猜忌。再说皇上性情,也不是那样刻薄寡恩之人,鸟尽弓藏之语今后不要再说了。”

站起身来,负手仰望,珠光辉映之下,只觉得心境渐渐平和,想到世上终有一人知我深心,而这人又是朝夕相随,亲如骨肉的小顺子,越发觉得心中欢馨喜乐,就是这窄小阴暗的密室,在我眼中仿佛也成了贝宫珠阙。嘴角忍不住露出一丝笑容,我道:“好了,你出去吧,若是给人发觉你这个小厮总是不见踪影,想来山子也没有法子替你遮掩过去。”

小顺子目光一闪,垂下眼帘,转身离开密室,还未合上暗门,便听到身后传来轻笑之声,看到公子愁闷全消,他也是心中愉快,想来接下来的日子公子不会觉得难熬了吧。想到此处,他也是难掩唇边笑意,步履轻快地向舱外走去。

\chapter{第二十九章 吴钩霜雪明}

隆盛八年二月二十六日,正是风和日丽的好天气,立在镇淮楼上,站在窗前俯瞰城下风景,裴云看似平静的面容下面隐藏着一丝烦闷,淮东战场失利,虽然占着楚州、泗州,也不能让他心中好过一些。偏偏这一次他奉了旨意,只在淮东牵制楚军,不能趁着陆灿陷在吴越主动出击,更是令他气闷。想到襄阳烽烟弥漫,长孙冀的南阳大营已经增兵至三十万,自己却未得到兵力补充,现在徐州大营尚不足十万兵力,想要发起一次大的军事行动都没有多少余力,这怎能不让他气闷呢。

另一件让他气闷的事情便是新任楚州郡守罗景。当初他原本准备等到局势稳定之后就将顾元雍撤换,免得根基不稳。谁知这顾元雍从前在骆娄真掌控楚州的时候有心无力,处理政务每有疏漏,可是自从投了大雍之后,居然如有神助,将楚州政务打点的头头是道,当初裴云从扬州败退,能够稳守楚州、泗州一线,实在是多有仰仗顾元雍的助力。裴云原本是赏罚公正的人,见顾元雍十分得力,就有心让他继续留任,可是这时候朝廷却已经派来了罗景担任楚州郡守,虽然不甚甘心,可是这也是说得过去的,毕竟楚州的位置很是重要。可是那罗景虽然能力出众,性情却甚是桀骜,治理楚州的手段雷厉风行,惹得楚州百姓怨声载道,若是换了别处,裴云也不会和他作对,只是楚州乃是前线重镇,又是新降,需要安抚才是,所以曾向罗景暗示。可是这新任郡守自恃才高,却不肯稍做让步。若是换了别人,裴云多半先给他一顿军棍,然后将他赶回去,毕竟楚州仍是军镇,需受裴云管辖。可是这郡守后台极硬,乃是当今皇后内兄高融的爱婿,高融乃是雍帝重臣,曾有幽州辅佐太子李骏的功劳,在皇上心目中的地位极高。裴云虽然不惧高融,但是他现在乃是败军之将,自然不想轻易得罪了高融,只是这样文武不和,如何能够全力进逼淮东呢?这样的烦恼之事怎不让裴云心中气闷。

裴云站在那里静默不语,立在他身后的顾元雍却是心平气和。作为一个降臣,他早已经有了充分的准备,至于家族的安危,他却并不担心,衡阳顾氏世代传承,断不会因为一个不肖子弟而灭族,现在他只需担心自己的身家性命即可。他是一个识时务的人,从前他是南楚世家子弟,便悉心读书,考取功名,为家族取得荣耀,为官楚州,立于虎狼之策,他就明哲保身,纵然为了楚州军民和骆娄真相争,也是控制在骆娄真可以忍耐的范围之内,更是着意结好楚州大营的军官,留下求救求情的后路。雍军攻下楚州,他便黯然投降,裴云委他重任,他便尽心尽力去做,如今免去他的官职,他也没有什么忧虑,只是筹划着是寻机回乡,还是继续等候雍廷的任命。在顾元雍心目中,他自认只是庸碌之辈,无力与强权相争,只要不过分侵犯他的利益,做雍臣还是楚臣倒也没有什么不同。当然若是现在南楚反攻回来,他可不会立刻就投降回去,毕竟好马还不吃回头草呢,只是若是大雍有人迫他做些丧心病狂之事,例如让他说降族人投雍,里应外合对付南楚,这他也是绝对不肯做的。顾元雍本就是这样的人,所以,裴云有意留他在楚州,他也就顺理成章地留了下来,施施然跟在裴云身边行走,而那新任郡守自然不知道,他许多不合楚州民情的律令,都是在这人示意下,指令楚州官员阳奉阴违,瞒上欺下,才没有挑起变乱的。

裴云立了许久,终于无奈地摇头道:“罢了,不想这许多烦心事,顾大人,我们换身衣服,出去走一走,散散心也是好的。”顾元雍闻言笑道:“将军平日军务繁忙,对这楚州城只是走马观花罢了,今日既想散心,就由元雍做陪,观赏一下淮安风光。”裴云微笑点头,回头看了一眼杜凌峰,道:“今日出去只是闲游,不许你随便惹事。”杜凌峰连忙应是,面上却是一红,他生性好斗,总是喜欢惹是生非,若不是这个缘故,也不会至今不肯正式进入军旅。

裴云虽然想出去散散心,但是毕竟三人过于显眼,裴云今年虽然已经三十四岁,可是自幼修习佛门心法,内力精深,使得他看上去还不到三旬年纪,加上相貌气度都是人中之龙,就是穿了便装也是人人瞩目,更何况往来遇到的巡视军士见到他都不免行礼,而顾元雍本是楚州郡守,更是无人不识,杜凌峰无事就在城中闲逛,认得他的人也是极多,众目睽睽之下,想要游玩也无法尽兴。裴云自嘲的一笑,目光闪出,看到街旁有一座小酒楼倒还清雅,便举步向内走去。

那酒楼的伙计几乎是跌跌撞撞地向内肃客,掌柜的三步两步就奔到近前,低头哈腰,迎了三人上楼,这楼上只有六七付座头,临窗的三付座头都用屏风隔开,外面挂着淡黄的竹帘,倒是清雅别致。顾元雍虽然在楚州多年,可是这座小酒楼却没有来过,如今一看的倒是觉得颇有遗珠之憾。三人坐了下来,要了些酒菜,便饮酒闲聊起来。裴云推开窗子向下看去,街上人来人往,比起镇淮楼下生人勿近的冷落自然有趣多了,越发觉得微服出来却是对了。

这时,掌柜又引了几个客人上楼来,那掌柜本想今日楼上不招待客人,但是杜凌峰聪明得很,知道裴云今日出来乃是散心,就是多些人气才会高兴,所以早已警告过掌柜不要泄露楼上有贵客,让他照常对待。那掌柜虽然不敢不依,但是却也留了小心,带到楼上的客人也是先揣测一下有无妨碍。今次的客人共有六人,明显是远道出行,颇有身份的人物,所以他才放心地将人请上楼来,其中两人径自走向裴云左手的座头,另外四人却是在外面楼梯旁边择了座位,显然是主从分明。掌柜刚要转身下楼,只见两个俊逸书生正在上楼,这两人相貌相似,只是一个高些,一个矮些,差着一两岁年纪。一看之下,这掌柜心中大惊,这两人乃是兄弟,兄长周明,弟弟周晦,素来都在他楼上饮酒,周明为人最是狂放不羁,一向都有些悖逆的话语,平日倒也罢了,无人告密外传,今日楼上却有贵客在。想到此处,那掌柜刚要上前阻拦,谁知周明已经大笑道:“老杜,你上次说青梅酒今日就可以开坛了,我们兄弟特意前来痛饮几杯。”

那掌柜心中一叹,知道已经来不及阻止了,只得含含糊糊地道:“那青梅酒又酸又涩,也只有你们兄弟喜欢。”

周明闻言又是大笑,那周晦却只是微微一笑,周明道:“这青梅酒乃是老杜你用夏日摘取的七分熟的野生青梅混合寒冬冰雪所酿,味道虽然酸涩,却是别有一种风味,岂是俗人可以领会,岂止我们兄弟喜欢,文浦也是最爱此酒,只不过今日他却不能来了。”说到最后,语声却是有些唏嘘。

掌柜又是心中一惊,连忙岔开话题道:“不是还有两位公子来品酒了么,小人这就去取酒,两位公子请先坐坐。”说罢,他便凑到两人身边正要低语,耳中却是传来一声冷哼,他身子一颤,察觉到从竹帘之后透出冷厉的目光,只得下楼去了。临去之时悄悄回头,却见周氏兄弟毫无所觉,似乎那一声冷哼并未听见,心中觉得古怪,却也只能黯然伤神。这时帘内的裴云却是淡淡一笑,便是他传音警示那掌柜,但是心中也生出忧虑,想到楚州百姓对大雍的抵触之心有增无减,不由轻叹。

那周氏兄弟径自走入临窗最右面的座头,似是熟门熟路,那周明一边走一边对弟弟说道:“前年你我送青浦兄远走高飞的时候,曾经有约,今日在此重逢,共饮老杜新酿的青梅酒,只可惜如今楚州已属大雍,往来道路断绝,青浦兄今日定是要失约了。”

周晦道:“这也难怪,楚州已经不属南楚,青浦兄虽然是千金一诺之人,却也只能望青梅而生叹,有家难回,有国难奔了。”

周明笑道:“其实这也未必,青浦兄文武双全,一向有心为国效力,只是看不惯朝廷昏庸,所以才浪迹萍踪,无心仕途,不过如今淮东由陆大将军主事,说不定青浦兄就在扬州、广陵呢,虽然两军对峙,但若他有心,凭他的本事也未必不能回来。而且青浦兄从无失诺之事,所以我今日才要在此等候,否则若是他冒险回来,我们兄弟却躲在家里不敢出头,岂不是愧对良朋。”

周晦却道:“兄长慎言才是,以小弟看来,青浦兄还是不来为好,他视华先生如父,若是得知噩耗,必然不肯罢休,但是那罗贼乃是楚州郡守,手握重权,青浦兄若是有意寻仇,只怕反而误了他的性命。”周明闻言也是长叹不已。

裴云本没有理会楼上其他的酒客,但周氏兄弟又没有刻意放低声音,所以他听得一清二楚,回头看了顾元雍一眼,眼中流露出疑问。顾元雍也听见了两兄弟的话语,心中正为他们担忧,看了裴云一眼,踌躇难言,倒是杜凌峰低声道:“这两人将军想是忘记了,年前我军败于瓜州渡,那周明写了诗文讥讽将军,还当众说陆灿必能夺回楚州,本来这样狂生理应问斩,只是师叔却没有在意,只是让顾大人管束他们。罗大人上任之后,和城内的士子寒生多有争端,更是派人监视这些人,一旦有不妥言语,便要下狱问罪,现在城中士子多半闭门不出,以避灾祸。只怕现在楼下就有罗大人的暗探呢。至于他们所说的华先生想是城中名士华玄,至于那个青浦兄,想是两年前因为打伤骆娄真麾下军士而出走的楚州才子庄青浦,庄青浦乃是楚州士子的领袖人物,和周氏兄弟相交莫逆。”

裴云这才想起那件事来,只是淡淡一笑,对于这些狂生文士的攻讦,他从来不放在心上,只要大雍节节取胜,时日一久,这些人自然不会再胡言乱语。倒是那个华玄的事情很是麻烦,那人学问精深,城中儒士十之六七都在他门下称弟子,自雍军入城后华轩就闭门不出,罗景有意迫他入仕以收士子之心,却被他严拒,罗章人一怒之下将他关入了大牢,还是顾元雍亲向裴云求情,裴云下了一道手令令罗景放人,这才令那老先生脱了囹圄之灾,结果华玄年老体弱,在狱中又受了凌辱,出狱不到半月就病故了,若非顾元雍从中调停,裴云又及时增派军士坐镇,到华家祭灵的楚州士子们差点闹出事情来,罗景事后还上书弹劾裴云纵容轻慢,令裴云差点气晕,但是裴云生性沉稳,虽然已经怒极,却不显露出来,只是上了一道折子自辩。想到罗景这般强势压制,岂不是更加容易惹出是非,一旦乱了民心,自己如何稳守楚州呢?想到此处,裴云心中越发惆怅,心道,若那庄青浦果然来了,就将他带回营中去,免得他向罗景寻仇,可惜了一个人才,微微摇头,裴云又向窗外望去。

顾元雍却是暗暗皱起眉头,庄青浦乃是江淮名士,性情义烈,文采过人,又擅剑术,乃是楚州难得的佳子弟,他父母都已亡故,族中乏人,若非华玄爱他资质,收到家中照顾,恐怕难以成人,他若知道华玄死讯,只怕真会向罗景寻仇。庄青浦在楚州士子中声望极高,若是他一呼百应,掀起变乱,岂不是天大的麻烦。他不知裴云心意,更是担心庄青浦今日会冒险而来,苦苦想着如何可以引走裴云,或者想法子私会庄青浦,劝服他不要闹事。但是见到裴云在那里饮酒赏景,全无起身之意,他又不敢露出形迹,更不敢暗示周氏兄弟,心中越发焦急起来。

这时候,掌柜已经抱了一个小酒坛上来,一打开酒坛上面的泥封,便溢出酒香缕缕,香气中已经带着孤绝之意。周明倒了一盏淡青酒液,轻啜一口,朗声道:“晓雾锁秦楼,又添离愁。临风把盏倾金瓯。阳关唱遍也难留,此恨悠悠。”反复吟咏数遍,声音满是惆怅。

裴云听得微微皱眉,他虽然不甚通诗词,也知道这应是一首《浪淘沙》的上半阙,那周明既是才子,怎会续不出后面半阙。

这时,却从楼梯上传来一个清朗孤傲的声音续道:“青梅撷满袖,疏疏雪片。经年酿作杜家酒。饮罢孤寒立轻舟,一醉方休。”

周明和周晦两人都是惊喜交加,周明更是冲出竹帘,望向楼梯,失声问道:“青浦兄,竟是你回来了么?”

裴云心中一震,想不到这庄青浦果然来了,姑且不论他如何穿越城关,但是此人重诺守信之处,已经令人惊叹。裴云从帘内向外望去,只见周明和一个书生把臂对视,周明竟是满面眼泪,显然十分激动。那白衣书生也是颇为激动,但是神色间却有一种冷静决然的意味。裴云仔细望去,只见那书生剑眉星目,风姿飘逸,犹如临风玉树,当真是貌如子都,风标绝世,只是周身上下都笼罩着孤傲清绝之意,少了几分亲切意味。那书生一身白衣如雪,宽袍绶带,大袖飘飘,腰间悬着三尺青锋,非是那种轻飘飘突具华丽外表的饰剑,而是古朴沉凝的黑鞘黑柄的长剑。可见这书生竟真是文武双全的俊杰。

裴云心中惊叹,目光一扫,落到了那书生面上,只见那书生虽然神光未减,但是面色苍白,印堂有一道黑影,太阳穴上更是隐隐有着暗红印迹,裴云心中一颤,不由黯然轻叹道:“可惜,可惜!”

岂知从左侧座头之内,也传来一个清雅的声音道:“可惜如此人才。”

裴云心中一动,目光向左侧望去,隔着屏风,看不到那边客人的相貌,但是那语声有些熟悉,一时之间却想不起来是何人。杜凌峰见他神色,便知究竟,在他耳边低声道:“那四个人和他们一起来的。”说罢伸手轻指,裴云望去,却是四个青年坐在那里低头进食,裴云只是一眼,便看出这四人气度沉凝,目中神光隐隐,身姿笔挺,衣履看似平常,兵刃也都用布裹住,像是寻常富商护卫模样,可是现在楚州境内哪里还会有寻常客商出没,何况这四人一见便知是身手不凡。越看越是觉得古怪,裴云不由剑眉微皱,现在楚州关防极严,这样的人物在楚州出现,为何自己没有得到禀报呢?

这时,那白衣书生的目光也扫视了楼上的酒客一周,淡淡一笑,随着周氏兄弟走入座头,道:“当年分别之时我写的词你还记得这样清楚,看来今日我若不来,你一定会骂死我了。快倒酒来,我等着今日已经许久,这些年飘零江湖,最盼的就是老杜的青梅酒,如今得偿夙愿,便是立刻死了也是不枉此生。”

周明心中皆是狂喜,只道他狂放,连忙取了一个大酒盏,倒了满满一杯青梅酒递上,那白衣书生一饮而尽,原本苍白的面色也多了些血色。周明喜道:“青浦兄还是这样爽快,老杜一年只酿十坛百斤青梅酒,这一次我已经全部买下,你我兄弟来个一醉方休,尽述离情别绪,待到酒醒之后,不论青浦兄如何吩咐,小弟都是欣然遵命。”他不便问友人是否已经得知恩师死讯,所以这样隐晦道来。却听的隔着屏风的顾元雍心焦如焚,恨不得高呼示警。

那白衣书生却是一笑,道:“为兄可没有事情相求,今日前来只是为了昔日诺言和这青梅酒罢了。”说罢取过席上酒壶,自斟一杯饮了,酒色染上面容,越发显得飘逸风流。周明犹豫了一下,欲言又止,终是不愿出口相问友人是否已经得知华玄死讯。

这时,淡黄竹帘被人挑起,走进来两个青衣人,前面的那人灰发霜鬓,相貌儒雅俊秀,气度从容洒脱,后面的那人似是仆从身份,低首跟随。周明一愣,见那人形容陌生,神韵奇秀,若是从前,见了这等人物,他自然是着意结交,可是想到楚州已是大雍所属,虽然这人看上去颇有楚人风姿,但必是雍人无疑,因此怒道:“阁下为何擅自闯席,未免太过无礼。”

那人目光一闪,道:“我闻三位盛赞青梅酒,也想尝尝这清绝孤寒之酒,若是诸位愿意,在下愿以此物交换一坛新酒,不知三位意下如何?”

说罢张开右手,手心中是一粒龙眼般大的蜡丸,周明正要相问,那人已经捏碎蜡丸,露出一粒红如火焰的丹药,厢房中立刻溢满香气,周明只是闻到那香气便觉得神清气爽。读书人有言,不为良相,便为良医,他虽然医术平平,却也知道这是极好的续命丹药,只是自己三人似乎用不上,正在犹豫的时候,庄青浦已经冷冷道:“多谢阁下,一坛青梅酒换取这粒药丸,未免太不值得了,阁下若爱此酒,我令掌柜送去一坛就是。”周明心中茫然,却下意识地唤掌柜取酒,不多时,杜掌柜果然另外提了一坛青梅酒送来。

那青衣人轻轻一叹,道:“是我太多事了,早片刻,晚片刻却也没有多少分别。”说罢用力一捏,那粒药丸变成粉碎,厢房中香气大盛,红色药粉飘落地上,那青衣人取出丝绢,拭去手中药粉,转身走了出去。周明心中一惊,觉得万分可惜,那药丸必是救命良药,却化成灰烬坠落尘埃。一眼望去,无意中却见到那青衣人右手之上戴着一枚玉指环。指环本是女子饰物,男子戴来略显轻薄,那青衣人气度不凡,却如何有这脂粉气,周明心中生出轻慢,目光中露出不屑之色。孰料那青衣仆人此时方要出去,一眼看到周明神情,目中闪过一丝寒芒,冷冷看了周明一眼,向外走去。这一举动,周明没有留意,却被坐在边上的周晦看到。那青衣仆从看上去二十多岁模样,相貌清秀白皙,只是一双眸子竟似寒泉一般幽深清冷,周晦心中一惊,生出不安的感觉。

此时的裴云却是愣愣地坐在椅子上,心中溢满惊喜,却又不敢相信眼前所见竟是真情,只是透过竹帘看到那两人面容,已经令他心中巨震,再听了几句话,越发确定自己的判断,恨不得立刻出去相见,只是想到自己若是一出去,只怕惊动楼上众人,反而不敢轻动,只是却坐立不安,深怕轻慢了那人。这时耳中传来冰寒的声音道:“公子请将军暂且不要过来相见。”裴云心中一宽,这才平心静气下来,心思潮涌,想着如何利用这一机缘,摆脱自己的为难窘境。

这时,那庄青浦也似是觉察出了酒楼上面的气氛有异,起身笑道道:“酒已饮过,人已会过,我这就要走了。”周明惊道:“青浦兄难得回来,如何这就要走?”庄青浦眼中露出不舍之意,神色间有些碍难。

周晦却是已经看出一些不祥的征兆,起身一揖道:“青浦兄若有什么难处,还请言来,在下兄弟纵然粉身碎骨,也不负所托。”

庄青浦知道周晦素来细心,便笑道:“哪里还有什么事情,只是希望没有连累了两位才好。”说罢起身一揖,然后举步向外走去,周明起身欲拦,庄青浦却已走到了楼梯口,正要举步下楼。周明想要喊他,周晦却拉住他轻轻摇头。周明也是聪明人,突然心中明了,脱口而出道:“莫非青浦兄已经去过华家了?”周晦还没有回答,耳中传来呼喊奔逃之声,周晦顾不得向兄长解释,已经扑到窗前。

街道上两侧烟尘滚滚,楚州雍军铁甲在烟尘中历历可见,已经将四面八方都封锁起来,街上的百姓四散奔逃,一个锦衣大汉带着百余身穿灰色衣甲的卫军冲了进来,指着街道两旁的宅院道:“有人看见那刺客在这里出现过,必然已经逃到两侧的宅院店铺里面了,你们给我挨家进去搜查,若有反抗杀无赦。”

周明此刻也凭窗向楼下望去,他认得那锦衣大汉乃是楚州卫军校尉高秉。按照大雍军制,各州郡都有卫军编制,战力较弱,兵源主要来自被裁撤下的军士,平日协助郡守维护地方安靖,楚州卫军编制有三千人,只不过现在楚州乃是淮南节度使裴云镇守,所以编制不满,只有一千二百人。那高秉乃是国舅高融的族人,在此任卫军校尉,其意不问可明,此人一向都是楚州郡守罗景的亲信爪牙,周明对其恨之入骨。心道他来捕捉什么刺客,莫非有人刺杀罗景么?他素来思维敏捷,立刻就联想到庄青浦方才的言词,听他语气,竟是心事已了,再无牵挂,想必那罗景必然已经授首,而且下手之人正是庄青浦。想通这一点,周明只觉得如坠冰窟,心中丝毫恶人受报的喜悦,也无心去想庄青浦如何有法子刺杀了堂堂的一位郡守,只是想到庄青浦就在楼下还未出门,这团团重围之中,庄青浦如何逃得出去?

楼下的高秉也是浑身冰冷,想起一个时辰之前的事情,仍然觉得恍如梦中。当时突然有一书生前来求见,说能够劝服楚州士子出仕雍廷,罗景自是欣喜,因为华玄之事,他陷入十分被动的局面,虽然他借着弹劾裴云暂时避开了风头,但是一旦朝廷得知此事真相,前途只怕尽毁,所以罗景急急召见。那书生入见之时腰悬长剑,除此之外并无暗藏兵刃,罗景和高秉都只道这是士子习气,并未介意,但是为了安全起见,仍是让他解剑入内。

来求见的书生自称庄青浦,乃是华玄门生,这个名字罗景听过,知道这人在楚州士子中名声不小,虽然鄙夷此人忘恩负义,不顾恩师之死,前来投靠,但是罗景也知若有此人相助,笼络楚州士子的大事十有八九可成。所以对那庄青浦十分礼遇。庄青浦侃侃而谈,他对楚州名士了如指掌,谈及如何笼络这些人更是头头是道,罗景听得兴起,不再疑心。罗景虽然骄横,但是才学也是不浅,否则也不能做到郡守,见庄青浦才学气度都十分出众,也有心招揽,便和他详谈起来,一谈之下,更觉投机,谈到酣处,庄青浦起而作剑舞,折柳为剑,长歌当哭,其中有“何言中路遭弃捐,零落飘沦古狱边。虽复沉埋无所用,犹能夜夜气冲天。(注1)”之句。罗景见他狂放风流,更无疑心,笑曰剑舞不可无剑,乃令人取来庄青浦的佩剑。

庄青浦接剑之后,再作剑舞,果然是剑如流虹,寒芒若霜雪。剑舞之后,罗景上前致意,却被庄青浦暴起行刺,高秉救之不及,只能围魏救赵,一掌击向庄青浦后心要害,只盼庄青浦避让一下,这样便不能一举杀死罗景,庄青浦的剑术虽然绚丽,却并非一流身手,只需有一线空隙,高秉便有信心救下罗景。谁知庄青浦也自知机会不再,竟然甘受一掌,一剑穿心,取了罗景性命,然后向外逃去。高秉本来自信这一掌可以击碎刺客心脉,可是庄青浦居然还有余力逃走,再加上罗景身死的冲击,高秉愣了片刻,等他清醒过来,熟知郡守府地形的庄青浦已经无影无踪。

高秉气怒攻心,令卫军追缉,更是令人向裴云求援,调动军队,封闭所有街道,缉拿刺客。高秉不是庸才,城中雍军虽然不受高秉指挥,可是也知捉拿刺杀郡守的刺客关系重大,通力合作,虽然楚州百姓都是不甚合作,却仍然发觉了庄青浦的行踪,确定他就在这条街道的范围之内。那些雍军尚未得到将令,便封锁住四面通路,让高秉自行带着卫军进去搜捕。而高秉想到无法向国舅高融交待,心中戾气上升,一进来便下令卫军强行搜查,一时间街道两边的屋舍都是人仰马翻,哭叫连天,更是不时传来卫军鞭打百姓的暴戾喝骂之声。

周明急得团团乱转,他既不想庄青浦被捉住,又不忍见百姓受到牵连,再说雍军定会上楼搜查,如果得知庄青浦曾经来过,必定受到株连,他虽然胆气豪壮,但是想到楚州郡守遇刺身死的严重性,再想到昔日裴云攻楚州时候的杀戮鲜血,心中也是寒气直冒,却是无可奈何,不知如何应对。

楼下的庄青浦神色黯然,他自然知道情势的严重,他未回楚州之前便已经知道恩师身死的噩耗,虽然他在楚州的人脉让他混入了城池,又让他未见罗景之前已经知道他的性情,设下了行刺之计,而且一举功成,甚至逃出郡守府之后,还有法子换下血衣离开险地。可是他也知道自己是绝没有机会再混出城去的,出城的盘查本就十分严厉,而且行刺之后,雍军必然封城。更何况他若一走,雍军恼怒之下,必会大索全城,连累无辜,所以他本就无意逃走,更何况他还有难言之隐。如今迟迟不出去,不过是不愿落入高秉手中,在死前还要受辱罢了。这时,几个卫军已经冲入酒楼,其中一人一眼便看到站在门口的庄青浦,高声喝道:“刺客在此。”

庄青浦微微一叹,举步向外走去,那几个卫军正欲上前将他缉拿,但是见他气度从容,竟是一愣,让他走到了街道上,几人怔了一下,执刀跟出,拦住庄青浦的退路。

庄青浦毫不在意,站在道中,高声道:“庄青浦在此,尔等何需扰民。”

高秉一见大喜,他一眼认出庄青浦,厉声道:“将他拿下,本校尉要将他碎尸万段。”想到前程可能尽数毁在这人手里,他当真是切齿痛恨。庄青浦冷冷一笑,宝剑出鞘,寒光一闪,迫退几个上来擒拿的卫军,道:“若想擒我,你就亲自上来吧,这些军士奉命行事,我还没有杀他们的兴趣。”

高秉大怒,上前一步,正欲亲自出手,心中决意要将这庄青浦狠狠折辱,这时却听有人高声喝道:“且慢。”高秉回头看去,只见隶属裴云白衣营的卫平立在街口高声喝止,卫平常常奉命和高秉打交道,高秉自然认得他。见他阻止,高秉心中微怒,正要讦问,却见卫平一挥手,精悍的雍军军士四面涌来,迅速控制住四周,强弓利箭,刀枪如林。

高秉闻言怒道:“此人行刺罗大人,理应交给我卫军处置。”

卫平高声道:“现在两军对峙,此人突如其来,刺杀郡守,我怀疑此人乃是楚军秘谍,需要交由将军处置,刺客听着,你若束手就擒,无所隐瞒,我必向将军求情,给你一个痛快,还不放下兵刃,立刻投降。”卫平得知此事之后,他担心罗景之死会牵连裴云,所以决定将刺客控制在手中,便匆匆赶来,却不知道裴云就在旁边的一家小酒楼之中。

庄青浦闻言却是哈哈大笑,虽然是肆意欢笑,却是不减俊逸风流,片刻,他止住笑声,道:“庄某本是寻常书生,虽有报国之志,却无青云之径,当日因为得罪那骆娄真被迫出走,昨日归来却得知恩师死在那罗景手上,且不论国仇,恩师教养我成人,我尚未膝前尽孝,却见恩师灵柩,今日行刺乃是我一人之事,无关他人,庄某今日唯死而已,万万不会落入你等手中。”

卫平一皱眉,道:“有我在此,你想死也不容易。” 说罢一挥手,人群中走出两个白衣营勇士,一人提着红缨枪,一人背上乃是宝刀,两人左右逼近,庄青浦擎剑微笑,两人正欲上前动手,却听旁边酒楼上有人朗声道:“下去吧,堂堂白衣营勇士,对着一个将死之人,何需如此多事,庄青浦,裴某念在你为师报仇,孝义双全,今日不为难你,你去吧,本将军保证不会随意株连。”

庄青浦闻言一惊,抬头望去,只见自己方才下来的酒楼之上,中间的那扇窗前,站了一个黑衣青年,气度沉静从容,俊朗英武,一见便觉心中折服,他离开郡守府的时候,心脉已经尽断,不过他剑术虽然不精,内功心法却有独到之处,尚能凭着意志和秘传心法支撑罢了,只需心神一泄,便会立刻死去。他心中念念不忘当日之约,所以临死之前也要来喝一杯青梅酒,又担心亲故受自己牵连,所以不肯舍生而去。

方才那青衣人送药给他,就是看出他伤重将死,虽然闻到那良药香气,也觉精神一震,但是庄青浦自知无药可救,也不想平白欠下人情,所以不肯接受。却是想不到裴云也在酒楼之上,更是想不到这位裴将军也是一眼看出他伤重将死,不愧是少林嫡传弟子。

原本为了罗景之事,他对大雍深恶痛绝,但是看到裴云这样气度心胸,却也心服口服,这些白衣营武士的厉害之处他自然可以看出来,出动两人不过是不让他有自杀的机会罢了,若非他已经命悬一线,真的动起手来,只怕他临死之前还要受辱。若非心中仍有牵挂,放心不下亲朋故旧,也不会忍死相持,如今听到裴云无意株连,心中一宽,心旌摇动,只觉四肢无力,竟是再也难以行走。他仰头高声道:“多谢裴将军海量宽宏,不罪无辜。”言罢,双目微阖,却是立住不动。卫平上前一看,仰头道:“将军,他已死了。”

街上雍军和楚州百姓都是动容,尤其是那些百姓,素来知道庄青浦的声名,更有人跪下磕头,低声祝祷。裴云一叹,从楼上纵到街心,负手看了庄青浦遗体片刻,躬身一揖道:“裴某从无虚言,绝不会因一人之事为难楚州父老。”声音方落,庄青浦尸身已经坠落尘埃。

裴云微微一叹,看也不看高秉一眼,对卫平道:“立刻传我将令,封闭城门,全城戒严,擅自行走者以奸细罪名处置,罗郡守已经已经遇刺,便由顾元雍暂代其职,高秉护卫郡守不利,暂免军职,卫军交由你统领。”

高秉本已怒气冲冲,听到这里喝道:“裴云,你如何这样胡作非为,本校尉乃是皇命钦封,岂是你说免就免的,那刺客行刺郡守,你竟容他从容自尽,又令南楚降臣接任,莫非这刺客是你主使的不成。”

裴云闻言面色一冷,森然道:“高秉,你不过是个卫军校尉,本将军却是淮南节度使,楚州乃是军镇,又受本将军统管,不要说你一个小小的校尉,就是换了偏将、副将,若有像你这等行事,贻误军机的,我也是先斩后奏。来人,将他带下去。”高秉想要反抗,但是看到就是自己麾下的卫军也全然没有遵命的意思,只得束手就擒,被几个军士带了下去。他素来仗势横行,见他被禁,街上一片欢声。裴云微微一笑,向酒楼之内走去。

卫平急忙上前道:“将军,有人到镇淮楼求见,手中拿着皇上御赐金牌,属下是来请将军回去的。”

裴云道:“我已知道了。”微笑不语,心道,我若非知道那人莅临楚州,也不敢这般肆意妄为。举步向楼上走去,他心中满是疑惑,正要向那人询问。

这时楼上,周明掩面不语,泪流满面,眼看好友身死,自己却是什么也帮不上,音容笑貌,犹在眼前,斯人已逝,遗恨无穷,周晦也是黯然不语,但是他想的更多,想到裴云方才就在旁边,那么一切他自然看在眼里,却不知会否为难自己兄弟?

这时,顾元雍挑帘而入,两人看见,都是起身一揖,周明呜咽难言,周晦则恭敬地道:“尚请大人周旋,允许我们兄弟安葬庄兄。”

顾元雍闻言一叹,道:“你们兄弟虽然性情一冷一热,却都是重义之人,放心吧,裴将军为人言出如山,绝不会更改,他方才下楼之时已经让我转告你们兄弟,令你们厚葬青浦,这件事情他不便出面,无论如何青浦刺杀了大雍郡守,这是死罪,不牵连旁人已经是裴将军法外开恩,你们不可因此生出怨怼之心,也不要想着为他报仇,青浦求仁得仁,想来也是死而无怨。”

周明、周晦闻言下拜致谢,周明道:“大人放心,我们兄弟不是不识进退之人,不会把青浦之死怪在裴将军身上,今日之事,就是裴将军将我们两兄弟立刻杀了,也未必说不过去,更何况裴将军还允许我等安葬亡友。”

顾元雍扶起两人道:“你们这就去吧,楼中尚有贵人在,关于他的事情你们不可多言,若有违逆,就是裴将军也救不了你们。” 两人闻言都是骇然,却只能凛然遵命。

————————————

注1:唐郭震《宝剑篇》

\chapter{第三十章 画角金饶}

隆盛八年二月,杨秀奉陆灿将令督军淮南,窥伺淮北。

——《资治通鉴·雍纪四》

周氏兄弟辞别顾元雍,便要下楼,但是酒楼之上却是气象大变,所有的闲杂人等都已离去,那四个原本坐在最外面座头的青年已经双双拱卫在最左侧竹帘之外,渊停岳峙,气度沉凝,接过四人身前的时候,周明、周晦只觉八道冰寒的目光在自己身上一闪而过,便已汗透重衣,这等威势,必不是寻常人物。而且两人眼光瞥去,已经看到杜凌峰立在帘外,神色恭谨中带着淡淡的戒惧,便知道这帘内那个灰发霜鬓的青衣人就是顾元雍所说的贵人,只是却想不到会是何许人罢了。两人不敢窥伺,匆匆下楼,周晦心中却无端想起那青衣仆从的幽冷双目,只是奇异的,却是想不起那人形貌。

我站在窗前向下望去,看着周氏兄弟招呼街上父老,帮忙安排庄青浦的后事,不由指着他们道:“我未免太多事了,其实南楚俊杰无数,一旦到了国破家亡之际,便此起彼伏,层出不穷,无需我费心警示,皇上也会知道平楚的艰难。楚人便如水一般,看似软弱可欺,但是若是真得激怒了他们,便会面临无孔不入的反击。如今我们占了上风,不过是尚未逼近楚人心目中的底线罢了。若不能让楚人彻底失去对南楚王室的信心,纵然铁蹄踏碎江南山河,也只能得到断瓦残垣,荞麦青青。”

小顺子答道:“公子之意,也是为了能够多留下一些南楚俊杰,免得损及天地灵气,一片悲悯之心,苍天也必然见怜,怎会怪公子多事。”

我微微一叹,想到这些日子蛙居舱中,到了广陵之后,舍舟登陆,一路上餐风露宿,分外艰辛,南楚淮东军并不轻与,想要穿越重重防线,若没有熟悉地理的秘营弟子带路,只怕我们这么多人没有可能无声无息地到达楚州。不过我们所走的路途虽然艰辛,却也是两国秘谍往来之途,一路上没有少遇到那些往来秘谍,都是靠着小顺子的指点,避过这些人的耳目。

进楚州城却是使用呼延寿等人携带的虎贲卫令牌,我一路辛劳,便让呼延寿去见裴云,自己在路边寻了一个酒楼准备休息一下,不料竟看到这样的场景,庄青浦上楼之时,我便看去他已命悬一线,以我的医术也已经无望回生,心中不忍之下,便以丹药相赠,随不能绾回他的性命,却可让他多活几个时辰。只是这庄青浦却是择善固执,竟然不肯接受,虽然说不过是几个时辰的区别,但是人谁不是贪生而畏死,他如此绝决倒也令我倾慕,只可惜天妒英才,不能挽回。

这时,帘外传来裴云清朗的语声道:“淮南节度使,徐州大营主将裴云请见。”

我微微一笑,指着帘外道:“都进来吧,哪里还要这么多礼节。”

裴云此时早已化去身上酒气,闻言整理了一下衣衫,对着这个就连自己的恩师也是十分敬重的人物,他丝毫不敢轻慢,更何况这人昔年对自己尚有恩情。见江哲这样吩咐,便带了顾、杜二人一起走了进去。

进得帘内,裴云单膝下拜道:“末将拜见侯爷,不知侯爷竟会到此,未曾远迎,尚请侯爷恕罪。”

我上前搀起裴云,笑道:“你如今已经是堂堂的节度使,何必这样多礼呢?我是私行至此,皇上想必还不知道呢?”

裴云心中暗道,不论你如何前来,若没有你在此,我也不便轻易解除高秉军职,去了内患,若非罗景已经遇刺,有了这人支持,自己也可将罗景免去,只是想到此人一来,许多为难之事便不再成为麻烦,这一拜他就是心甘情愿。

我隐隐猜知他的心意,微微一笑,目光转向顾元雍,见他神色惊骇,想必已经猜到我的身份,正在奇怪我本应该在定海,为何竟会到了楚州吧?

上前一揖,道:“这位想必就是顾大人吧,本侯表兄在楚州任职,多蒙大人照顾,在下待他致谢。”

顾元雍心中茫然,不知所措,江南江北音讯隔绝,荆长卿那等小事自然不会流传过来,见他茫然,我给小顺子使了一个眼色,小顺子上前淡淡道:“嘉兴荆氏乃是公子母族,现任家主荆长卿便是公子表兄,曾任楚州长史,蒙大人青眼,心中感佩非常。这一次公子途经嘉兴,荆长史托公子转呈谢意。”

裴云、杜凌峰和顾元雍都觉得脑子里面轰然,他们自然不知小顺子这番话真真假假,荆长卿和江哲一向有隙,这次嘉兴之行,两人根本没有见面。倒是顾元雍首先清醒过来,他不似裴云和杜凌峰一般担忧已经得罪了江哲,倒能够冷眼旁观。见江哲眼中满是笑意,并无责怪之意,而且此人既然声名显赫,必是喜怒不形于色之辈,若是真的因此生出怨恨,岂能侃侃直言。如今他得裴云之命,代理郡守之职,一荣俱荣,一损俱损,他的身家性命倒多半系于裴云身上,所以自然不愿看他难堪,便出言道:“荆长史精忠耿直,在下一向钦佩,就是裴将军,虽然为了立威,将他囚禁,却也对他看重得很。”

裴云这时候已经醒悟过来,不由庆幸自己当时没有直接杀了那个强项长史,见江哲没有怒容,再想到荆长卿的离奇失踪,也不由笑道:“裴某本来以为是麾下将士过于疏漏,才被人劫了囚牢,如今想来,就是他们目不交睫,想来也没有法子看住人犯吧。”

这番话却是暗含奉承之意,却又不露痕迹,就是我听了也觉得顺耳,原本有意吓裴云一吓,免得他平白借了我的威势,此刻也是不想了,指着那坛青梅酒道:“罢了,罢了,这酒果然不错,我明日就要离开楚州,就让掌柜再拿来两坛,你我小酌一番如何?”

裴云心中一宽,知道那件事情并未让江哲心生不满,目光一闪,看到杜凌峰神色不安,便道:“侯爷有此雅兴,末将怎会推辞,凌峰,去取两坛青梅酒过来。”

杜凌峰心中狂喜,连忙匆匆施礼退下,心中暗暗赌咒,明日这楚郡侯离开之前,他都不会再靠近江哲一步,对于江哲的畏惧,却不是因为那种种传言,对于一个血气方刚的青年来说,任何权威的力量都不能让他们却步。只不过杜凌峰在少林寺练武之时,曾有一次慈真大师带着关门弟子江慎回到寺中,在慈真大师忙着和寺中长老谈论佛经武学的时候,江慎便交给那些下辈弟子轮流照看,其时江慎不过四岁,却是淘气至极,让众人都是头痛欲裂。一天轮到杜凌峰照顾江慎的那天,江慎尤其顽皮,一眼照看不到,就不知道跑到哪里去了,杜凌峰性子有些急躁,趁着别人不注意,将江慎狠狠打了一顿屁股,接下来江慎果然老实了半天。结果等到杜凌峰中午午睡醒来,抱着江慎要把他交还给师伯祖慈真大师的时候,却是人人见了他都目瞪口呆,然后便是掩口偷笑。杜凌峰醒悟过来,一照镜子才发觉自己的眉毛竟被人剃去了,之后半年时间,羞得他都不敢出门,再见到江慎也是退避三舍。在他想来,有其父必有其子,江慎那样的小魔星,他的爹爹必定也不好惹,自己偏偏得罪了江哲,自然是离得越远越好。

片刻,两坛青梅酒被杜凌峰亲自捧了进来,然后他便趁机退下,顾元雍见裴云和江哲似乎有意密谈,便也识趣地退了出去。

酒过三巡,裴云开始步入正题,出言问道:“侯爷不是随水军去了定海么,前日传来的谍报仍说侯爷趁夜袭取镇海甬江口,烧毁楚军船只百余艘。”

我闻言笑道:“这是夸大了,明州甬江口港湾为淤泥所阻,一千石以上的船只就不能进入,陆灿最多在那里留下一些快船,用来监视定海动静,传递军情,若是现在陆灿还让东海水军有机会取得重大胜绩,他也不会是堂堂的大将军了。”说到这里,我又转头对小顺子道:“琮儿还是不够稳重,这种小事也要出面,这可不符合我的性子,只怕再有一两次这样的举动,就是我没有露出行踪,陆灿也会知道定海那边是个替身了。”

小顺子淡淡道:“就是他知道了,也要说服别人。”

裴云自然已经听出,定海那边江哲留下了替身,江哲所说的“琮儿”之名他虽没有听过,但是想来是江哲弟子门生一流的人物,想到江哲将南楚君臣的目光诱向吴越,自己却脱身来了楚州,这等神龙见首不见尾的行止,当真令人敬服。

裴云心中疑云重重,朝廷既没有旨意,也没有得到任何相关的谍报,虽然他还不会以为江哲有什么问题,但是当前最紧要的就是保护好江哲,其次就是上书皇上,说明此事,可是江哲方才说明日要走,他若真的随便放了江哲离去,只怕将来有什么差池,皇上定会怪罪下来,所以出言问道:“侯爷履险如夷,自吴越潜来楚州,末将佩服,现在战事连绵,虽然淮北尚在我军掌控,但是南楚的谍探也经常深入过来,侯爷不如留在楚州一段时日吧?”

我冷笑道:“若是留在楚州,只怕会被敌军生擒了,裴将军这样放心楚州的防务么?只怕就连徐州都未必保得住了。”

裴云心中一震,谨慎地道:“侯爷此言何意,末将在楚州、泗州阻住南楚军北上之路,淮西楚军虽然上次取得大胜,但是也是损害极重,又有崔珏崔将军守宿州,为何徐州也会失守?而且陆灿又为侯爷计策羁绊吴越,难道还有法子分身北上攻打我军么?”

我轻叹道:“皇上和齐王,甚至我,都还是轻视了陆灿,我军年前战败之后,楚州、泗州、宿州防线仍然稳固,淮北尚有你和崔珏两部军马,更有十几万精兵,在我们心目中已经可以守住淮北,姑且不论南楚君臣是否又胆子挑衅开战,齐王殿下即将率军南下,在汝南设立江南行辕,总督南征军务,呼应南阳、徐州,所以虽然淮北兵力不足,我们也没有放在心上。

谁会想到陆灿竟然有这样的胆量,今次我经淮东北上,发觉楚军征调粮草的数量超过了淮东楚军正常所需,而且杨秀现在就在广陵坐镇,广陵乃是北上要道,现在正在厉兵秣马,我本来不以为意,只凭淮东军马,绝对不可能一举攻破泗州、楚州,直到我到了楚州,才发觉这里居然文武不和,民怨沸腾,怪不得陆灿有胆量进攻楚州,若是我所料不差,只要杨秀一进攻,淮西守军便会配合飞骑营北上,夺取宿州,进逼徐州,一旦徐州失守,向北可以威胁青州,向西可以威胁南阳,楚军不仅稳据江淮,还可占有进攻大雍的主动。

现在想来也真是天佑大雍,东海水军攻吴越,损及南楚赋税根本,陆灿不得不亲赴吴越督军,杨秀虽然也是人才,却少了几分决断,为了求稳,延缓了进攻的时间,否则若是十日之前,他们就开始发动,只怕楚州百姓就会揭竿而起,到时候楚州就危险了。”

裴云听到此处已经是脸色铁青,不由暗悔自己爱惜前程,放纵罗景胡为,仔细想来,越想心中越是生出寒意,现在长孙冀又在围困攻打襄阳,虽然佯攻的成分居多,但是也必然没有余暇顾及江淮战事,而淮北防线似安实危,若是楚军有意北进,目标必然是指向徐州。我见裴云已经知道目前形势的严峻,又道:“这也怪不得你,南楚军从未主动北上,如今你已经知道消息,应该如何防守你便去安排吧,只要不丢了楚州,就是泗州失守,也不算我军战败。”

裴云站起一揖道:“末将多谢侯爷警示,请侯爷放心,只要裴某在楚州一日,就断不会让楚州失守。”

我点头道:“这样就好,虽然江南行辕尚未筹立,但是我任参赞一事已经定下,你不需担心会有什么罪责,一切我皆可担待。本来淮北危殆,我应留在此处才是,只是襄阳战事按照原来的计划,未免有些保守了,所以我要去见长孙冀,你给我通关文书,再给我一个向导指路,还有凡是知道我来楚州的军民,你都要小心防范,我还不想别人知道行踪。”

裴云点头道:“末将遵命,方才侯爷见到的杜凌峰是我师侄,他道路极熟,可以为侯爷向导。今日见到侯爷的人,末将会将他们控制起来,断不会让此事外泄。”

我点点头道:“一旦楚州遇袭,你要严防城中生乱,顾元雍算是个人才,只要你还有取胜的希望,他就不会反叛,此人在楚州颇有声望,你这次让他接替罗景,却是对了,你要好好笼络他,才能稳住楚州民心。那个高秉成事不足,我看他败事倒是有余的,若是有什么不妥,就把他斩了,不要手软才是。”

裴云肃然道:“末将遵命。”

我站起身道:“好了,就让那个杜凌峰替我们安排食宿吧,你的军务要紧,明日我离开也不必你相送,免得露出什么风声。”

裴云道:“侯爷所需文书,明日凌峰会呈上给侯爷,末将现在便要去城外大营点兵,请侯爷恕末将轻慢之罪。”

我淡淡道:“快去吧,我还想在这里喝上几杯酒。”

裴云起身告退,毫不犹豫地向楼下走去,没过多久,我便听到楼下的马蹄声响起,渐渐远去。

我轻轻一叹,道:“这一次真是侥幸,若不是路上呼延寿发觉粮船的数目远远超过应有的规模,又有你这样身手的人去做谍探,还不能发觉这一次南楚的声东击西的计策。说来也真是好笑,我将楚军目光诱到吴越,陆灿却也因势利导,趁机收复淮东,进逼徐州。这一次我们两人倒是一个平手。”

小顺子淡淡道:“无论计策如何周密,既然已经泄漏,就没有那么容易成功,否则公子怎会这么放心去襄阳呢?”

我闻言笑道:“裴云乃是少林护法弟子,他的性情既有刚毅果决的一面,也有通权达变的一面,前些日子他纵容罗景,便是不想得罪权臣,以致使得楚州局势不稳,但是如今他既已知道南楚军有进攻之意,便会杀伐决断,纵然楚州血流成河,也不会让南楚有机可乘的。”说到此处,我又叹道:“若是我早来一日,只怕此刻裴云已经将罗景赶走了,那么就不会有今日之憾了。”

小顺子冷笑道:“公子这却是说糊涂话了,只怕这庄青浦和南楚也有瓜葛,否则他凭什么穿过两军防线,回到楚州,再说他行刺罗景,不也是对南楚有利么,裴将军和罗景尚未达到水火不容的境地,若是楚军袭来,一个铁腕郡守恐怕比起一个降臣要可信的多吧?而且若非庄青浦重伤将死,纵然裴将军怜惜于他,也不得不将其擒拿处斩,到时候城中士子必然对大雍更加怨恨,内忧外患一起发作,只怕楚州城就没有那么好守了。”

我听了之后,低头想了片刻,道:“你说得也有道理,不过庄青浦已死,这件事还是不必提了,无论如何这人死得也是可惜了,若是杨秀真要牺牲这样一个人,我倒要笑他目光短浅呢。”

这时,耳中传来熟悉的脚步声,一听便知道上楼的正是呼延寿,我突然笑道:“呼延娶了苏侯,别的好处不说,这监察敌情的本事却是突飞猛进了要不然只怕楚军兵临城下,我们才会知道南楚还有胆子进攻淮北呢?”

小顺子闻言一愕,纵然以他的冷面冷心,也不由莞尔。

\chapter{第三十一章 三千里地山河}

二月二十六日,酉时,襄阳。

落日斜阳之下,雍军渐渐退去,容渊轻叹一声,只觉得心中无比惆怅,自从德亲王死后,自己因为亲王的遗折保举,成了襄阳将军,镇守重镇,可是这些年来,他却从来没有一丝开怀。对南楚君臣来说,他容渊不过是个寒门书生,虽然有些守城的本事,却也谈不上名将,所以十余年来只能枯守襄阳。他很想取得几场大胜,扬眉吐气,然后进入南楚的军事中心,可是无论他如何努力,始终只是一个守将罢了。更令他郁闷的是,大雍自从齐王两次攻襄阳惨败之后,就再也不曾将重兵放在襄阳上。每次大战一起,都是派出十万八万的兵马来围困襄阳,这样一来,襄阳虽然安枕无忧,可是功劳却也谈不上了。就像刚刚结束的大战,陆灿、石观受了种种封赏,他和葭萌关余缅却是连一纸褒奖都没有。想到自己纵然没有大破敌军的战功,可是死在襄阳城下的雍军也是数不胜数,而且只凭襄阳一城,便牵引十万以上的雍军,这本身已经是不小的功劳。可是大战之后却没有得到丝毫认同,以容渊的心性,怎堪忍受这样的屈辱。

望着退走的雍军,容渊愤怒的一掌拍在城墙的石跺上,长孙冀这狗贼,简直把襄阳城当成了练兵的地方,每日轮流派出军队攻城,磨合他们的战力,全没有勇气孤注一掷,难道雍军不知道若是不得襄阳,则无法威胁江陵、江夏,甚至就是夺得了淮南,也会立足不稳么?

二月二十六日,亥时,宿州。

夜色朦胧,凉风习习,一间朴素无华的寝室之内,烛火摇曳,榻上睡着一人,面上刀疤宛然,纵然是在睡梦中也是愁眉深锁。在门外,两个守护的亲卫目光如鹰隼,即使是在千军万马的保护之下,也仍然没有片刻松懈。将近子时,换班的亲卫匆匆走来,他们走到门口,两个原本守门的亲卫相视一笑,轻手轻脚地向外走去,准备换防。其中一个亲卫无意中目光掠过那个亲卫面容,却是一张陌生的面孔。他心中一惊,停下脚步正要动问,便觉得眼前寒光一闪,然后一只手已经捂住他的口鼻,鲜血涌入他的喉咙,他极力想要呼喊,却是无法出声。而另外一个亲卫几乎是完全没有防范,只觉眼前一黑,便失去了知觉。那两个假扮的亲卫迅速将两人放到门口,让他们倚着墙壁站着,残月之下,若是从远处看去,只会以为两人偷懒打个瞌睡罢了。然后这两人其中一人推门而入,另一人却掩到窗下,手中寒光如雪,却是一柄匕首。

崔珏眼眦欲裂,眼睁睁看着多年好友浴血断后,眼睁睁看着他战死在沙场,不由冷汗涟涟,羞愤难当,然后他便从梦中惊醒,他坐起身来,睁眼望去,却见昏暗的灯光下,一条黑影正向自己扑来,他几乎是想也不想地翻身滚下床榻,血光崩现,一条手臂落在地上,崔珏一声痛呼,高声叫道:“有刺客。”声音撕破了寂静的夜空。那刺客原本想要暗暗行刺,孰料这本已睡着的目标竟会突然暴起,结果只是砍下崔珏左臂罢了。而崔珏的一声惊喝,外面立刻一片沸沸扬扬,灯火喊声向这边涌来。那刺客略一犹豫,已经碎窗而出,会合外面的同伴,向外冲去。但是崔珏身为将军,身边的亲卫极多,若非崔珏一向自负武艺,不喜欢太多的亲卫随侍,两人根本就没有机会,如今既然已经惊动了人,这两人如何能够逃得出去,在杀了数人之后,一个刺客战死,另一个刺客被那些亲卫活捉。推到阶前。这时候崔珏已经面色苍白地坐在一把椅子上,旁边是军医替他裹伤,骤然断了一臂,崔珏伤势极重,如今已经是强行支撑着盘问刺客了。

那刺客缄口不言,崔珏问了几遍见他不肯说话,也失去了耐心,正想让人将他关押起来,远处突然传来惊呼声和喊杀声,然后便是北门燃起熊熊火焰。崔珏心中一惊,站起身来,却是一个踉跄,这时,一个军士奔了进来,扑到道:“将军不好了,是南楚军来攻城了,北门被奸细打开,现在楚军已经入城了。

崔珏恨声道:“好狠毒的手段,楚军只是占了北门,传我将令和敌军巷战。”说罢伸手去拿兵刃,却只觉头晕目眩,一跤跌倒在扶持他的族侄崔放怀中。这时,城中众将多半都已冲到了崔珏的住处,却只看到崔放抱着崔珏大哭。崔珏的副将见状大声道:“将军已经受了重伤,我军又没有防备,如果和敌军缠战,只怕数万军马都要葬送在宿州,何不弃城而走,退到萧县防守,然后再向徐州求援。”崔放连连点头,扬声道:“副将军请暂代将军传令,我护送将军先走一步。”那将领闻言慨然道:“由我亲自断后,诸位将军都快些召集人马撤退,敌军来自南面,却封了北门,为了稳妥起见,我们从西门撤退。”

崔放闻言也顾不上别的,抱着崔珏上马,在亲卫营保护下向西门冲去。刚出府门不远,只见长街之上,一队骑兵正向这边冲来,为首的便是两个白袍小将,两条银枪如银龙飞舞,收取着雍军将士的性命。转瞬之间,他们的身影被涌上的雍军淹没,崔放不顾一切冲向了西门,将要冲出城门的时候,无意中一回首,身后已经是一片火海。崔放抹去眼角热泪,投入到茫茫的夜色之中。

这一战直到天明方才结束,宿州三万军马,倒有半数葬身火海,副将战死城中,飞骑营在陆云、石玉锦统率下追出二十里,大破雍军,雍军败退萧县,崔珏伤重昏迷。

二月二十七日,寅时末,泗州。

天光未晓,雾冷水寒,滔滔淮水之上,尽是渡舟,在黑暗中向对岸驶去,悄无声息地向泗州城摸去,泗州城距离淮水只有两里远,船上的军士都是穿着和夜色相近的灰暗衣衫,天光黯淡,雾锁淮水,直到那些灰暗身影到了泗州城下,仍然没有被雍军发觉。

到了城下,十几个黑影掩到城下,手足并用向城上爬去,这些人身手敏捷,只凭着城墙的些许凹凸不平,就能够如同猿猴一般向上攀去。还未接近城头,城上便有人低呼道:“你们来了。”言罢放下绳索,这些黑衣人借着绳索,不多时已经登上城墙,没入黑暗之中。过了不到一拄香时间,泗州城内突然火光四起,然后城门之内传来纷乱的喊杀声,不多时,城门洞开。

伏在暗处的南楚军将领望见,心知里应外合大破泗州的战术已经成了一半,挥动旗帜,杀声震天,南楚军士向城门冲去,那将领一马当先,直入城中,只见眼前烟火弥漫,引路之人很快就消失在演武之中,那将领一皱眉,喊道:“不可深入,控制城门。”

就在这时,两边突然传出喊杀声,那将领一愣,只见雍军从两侧涌上,身后的城门则是轰然关闭,那将领心知不好,大叫道:“中计了,跟我杀出去。”却还没有跑出两步,就已经被利箭射杀。

淮水对岸,原本遥望着泗州的杨秀心中生出不祥的预感,已经过了小半个时辰,尚未得到回报,正在他心焦的时候,只见河对岸泗州城门突然洞开,一个雍军将领纵马到了河边,朗声大笑道:“多谢你们的厚礼,本将军笑纳了。”说罢,他手一挥,身边的军士丢下几十颗人头,那将军高声道:“张将军有命,凡是私通楚军,意图谋夺泗州的叛逆均已正法,首级令我送给杨大人。”说罢,那支雍军快马奔了回去。此刻河上的烟雾刚刚散去,露出湍流的淮水,以及对岸固若金汤的城池。

杨秀心中一阵剧痛,知道辛辛苦苦联络的内线和派去夺城的勇士都已殉难。

此刻站在泗州城头的张文秀也是一手的冷汗,若非昨日得到密报,城内世家有不稳迹象,而黄昏时分又得到裴云密令,让他不顾一切,收押城内豪门,才发觉南楚军里应外合的阴谋,若非如此,只怕泗州城即将不保。如今他手中的五万军队,分别扼守泗州和徐城,南楚军则在对岸的都梁扎下大营,淮东楚军主动北上,这一战的艰苦,不问可知。

三月二日,襄阳城内。容渊望着手上的密报,几乎是咬牙切齿,这两日雍军突然放缓了攻势,容渊心中不安,遣人出去查探,却发觉城外雍军竟然走了大半,只剩下几万人在那里佯攻。疑惑之下俘来一些雍军军士拷问,才得知江淮战场大战已起,裴云的求援书信已经到了襄阳,长孙冀留下两万人在这里虚张声势,自己带着主力去淮北了。容渊得知之后心中大恨,这样的大事,自己竟然全不知道,陆灿也是未免欺人太甚。

发动了所有人手暗探,容渊很快就得知了江淮的情形,这一场战事波及两淮,战事激烈非常。

二月二十六日,崔珏遇刺,宿州失守,崔珏退守萧县。

二月二十七日,杨秀谋泗州失利,渡淮水攻徐城,两军在泗州、徐城之间交战数场,互有胜负。

二月二十八日,杨秀留部将攻泗州,自率水军自里运河攻楚州。

二月二十九日,楚军破徐城,决洪泽之水灌泗州,张文秀被迫退往楚州,为杨秀截住去路。

三月一日,张文秀苦战一昼夜之后,裴云出楚州,接应泗州军,两军退入楚州,杨秀困之。

三月二日,五日猛攻之后,萧县城破,淮西军及飞骑营尾追雍军,九里山中伏,陆云、石玉锦率军突破重围,退守萧县。

可是,这场场大战,却和襄阳军没有丝毫关系,容渊每想到此处,都觉得心如刀割,妒火膺胸。他本是量窄之人,前次陆灿大胜,他却连苦劳都没有,此事早已在别有用心的人口中变成了陆灿妒贤忌能的铁证。如今陆灿丝毫不考虑襄阳军,自行发动江淮之战,甚至他本人还在吴越忙着海战,只将战事交给杨秀、石观,还有乳臭未干的陆云、石玉锦,全没有看到襄阳军的战力。这等轻慢,令容渊生出争功之心。

三月六日,岘山之顶,赏玩着前朝乃止更早的摩崖石刻,我心中平静如水,正在仔细研读那些模糊的文字的时候,呼延寿匆匆走来,禀道:“侯爷,容渊果然已经向南阳去了。”

我闻言轻轻一叹,道:“容渊此人,乃是我的旧识,此人才学过人,只是过分量窄,前次他未得建业封赏,已经心中嫉恨,这一次陆灿兴兵又没有他的事情,怎不令他恼恨,所谓利令智昏,只需设下计谋让他以为长孙将军真的去救援淮东,他必会寻机出战,建立大功,和收复淮北的大功相比,若能夺到南阳,就有进攻武关,直逼关中的机会,这样的大功他若不心动,也就不是容渊了。”

呼延寿笑道:“侯爷的计策厉害之处就在于所有的消息都是真的,只不过设法让容渊知道的多了一些,长孙将军减兵增灶之策,让那容渊对南阳军东进全无疑心,所以生出贪功之心。可惜长孙将军已经在南阳布下重兵,只怕容渊他去得来不得了。”

我淡淡一笑,道:“容渊去攻南阳,也只是想得些功劳,一路上必然狐疑进退,若是稍有风吹草动,说不定他就跑回襄阳了,所以必须将他诱到南阳才行,只有在南阳受挫,他才会急急返回襄阳,到时候我军便在途中设伏,方可拦住他的归路。容渊袭取南阳,必是轻骑北上,襄阳城中仍会留下守城兵力,所以我大军便需困住襄阳。若是毁去容渊带出的主力,则襄阳从此没有出击之力,若是趁势攻下襄阳,则是大获全胜。到时候只要徐州还在我军手中,就是丢了整个淮南,也不是什么要紧的事情。”

呼延寿敬佩地道:“侯爷攻心之计,最是难以防范,事前怎也未想到容渊竟会出襄阳北上。”

我闻言道:“岂止你没有料到,按照我原先的计划,只是利用流言激使容渊出战,让他连胜几场,然后诱杀襄阳骑兵主力,可是想不到江淮战事竟会提前爆发,我才想到可以利用这个机会和容渊的狭窄器量,骗他劳师远攻,而我们趁机夺取襄阳。此举不论成与不成,襄阳都不再是大雍咽喉上的那根利刺。”

说完之后,也不理会呼延寿在那里深思,站在岘山顶远眺,襄阳城和汉江、渔梁洲,及汉江对岸的鹿门山都是历历在目,想到再过半日,这里便是烽烟再起,失去了主将的城池,不知是否还能够固若金汤。

接下来的半个月局势的变化异常迅速,当初怎也不会想到雍楚第二次大战竟会这么快就开始了。

三月七日,长孙冀遣将莫业攻襄阳,断绝襄阳、南阳通道。

三月八日,容渊破新野。

三月九日,容渊攻南阳不克,得知长孙冀并未驰援江淮消息。

三月十一日,容渊在新野与长孙冀交战,战势不利。

三月十二日,容渊损失惨重,突围成功。

三月十三日,樊城陷落,容渊阻于汉水。

三月十四日,襄阳守军出城接应容渊不果。

三月十五日,容渊、长孙冀再战唐白河,长孙冀小挫。

三月十六日,容渊绕道樊城西侧,欲渡汉水入襄阳,为莫业所阻。

三月十七日,襄阳城破,容渊见势不可为,携残军渡汉水败退宜城,途经风林关遇伏,只余三千步骑脱走。

在襄阳鏖战之时,江淮战事也是分外激烈。

因崔珏不能上阵,裴云于三月四日,遣张文秀援崔珏部,至三月十九日为止,淮西军与张文秀于萧县、九里山之间共交战十七场,萧县屡次换手,双方皆损失惨重,张文秀兵力耗尽,不得已退守徐州。淮西军猛攻两日不克。

三月二十二日,大雍江南行辕先锋大将荆迟至徐州,败飞骑营于徐州城下,南楚淮西军连夜退守宿州。

三月二十四日,荆迟攻宿州不克,转道楚州,其时裴云稳守楚州已将近一月,楚州危殆,得荆迟援救,士气大振。

三月二十五日,杨秀得知襄阳失守,徐州援军到达,不得已退守淮水,然大雍在淮南只余楚州一城。

至此,历经一个月的雍楚大战终于结束了,但是南楚的厄运并未停息,据有江淮,而失襄阳,姑且不论是得是失,但是蜀中的巨变才是更震骇人心的。

早在年初,便有流言提及余缅因为未曾受赏而有意背离,虽然这流言被陆灿驳斥,尚维钧却心中不安,便在上元日之后,派去内侍为监军,此是南楚惯例,陆灿虽然不满,也是无可奈何。岂料那内侍索贿不成,屡次进谗言指责余缅有二心,虽然皆因陆灿之故而没有起到作用,但是尚维钧的疑心也是越来越重,最后将葭萌关守军的粮饷交由那监军控制,结果那内侍贪污大半粮饷,令得葭萌关守军无粮无饷,人心浮动。陆灿得知之后,上书建业,要求招回内侍问罪,那内侍得知,畏惧加罪,暗中投降大雍,里应外合,三月二十九日,秦勇袭取葭萌关,余缅退守剑阁。

或许唯一能够令南楚朝廷放心一些的便是,在陆灿亲自督军之下,吴越义军稳固了海防,雍军再不能轻易进入杭州湾了。但是吴越的小小胜利,抵不过襄阳和蜀中的失利。四月中旬,齐王李显大军抵达徐州,江南行辕的建立,更令南楚朝廷惴惴不安。陆灿其时已得军报,吴越战事委于部将,赶至江夏指挥作战。

李显到达徐州之后,遣长孙冀自襄阳而下,沿汉水河谷向江陵进攻,四月二十一日,容渊得陆灿军令,弃守宜城,死守竟陵,长孙冀连攻不克。陆灿自江夏出兵,沿汉水援竟陵,败长孙冀于城下,长孙冀败退宜城,容渊急躁,不奉军令追击,长孙冀弃宜城北返襄阳。容渊追至风林关,不意雍军故技重施,再度设伏。容渊败退。陆灿援军赶至风林关,再次突袭,雍军措手不及,风林关破,雍军遭重创,退守襄阳。陆灿知襄阳不可攻,乃止。

其时,秦勇久战剑阁不下,乃绕道阴平道,欲经龙安、江油至绵阳,余缅得陆灿千里传书,分兵扼守龙安,秦俑久攻不下,退守葭萌关。

裴云得援军相助,猛攻淮东,杨秀凭水军往来淮水、运河,雍军步履艰难,不能过淮水半步,淮东陷入僵持。淮西石观亲守宿州,雍将荆迟猛攻月余,城破,石观退守钟离,临去火烧宿州,只留下焦土一片。雍军久战疲惫,钟离防线稳固,又有飞骑营助战,雍军不得入淮西。

雍楚缠战半载,皆疲惫不堪,东海水军更是屡屡劫掠吴越,虽然余杭水营得义军相助,没有重大的损失,但是临海三十里之内,再无平民敢于居住,吴越商业损失惨重。雍军虽然多有取胜,但是楚军也是稳扎稳打,战线胶结,均不能取得决定性的胜利。

金秋十月,尚维钧遣使徐州议和,大雍君臣也苦于南楚坚韧难攻,同意停战议和,议和之举持续四月,大雍要求南楚割地求和,尚维钧意动,陆灿坚决不许,争执数月,议和失败,翌年,战端再起,秦勇自米仓道入蜀,经巴中而夺巴郡,蜀中虽为陆氏经营多年,但是终究是旧蜀之地,明鉴司夏侯沅峰亲至巴郡,数月之内,巴郡稳固,期间南楚夔州军和剑阁余缅双面夹攻,皆为秦勇退去,蜀中与东南道路断绝。

隆盛九年八月,李贽接受江哲建议,提出和议,以剑阁、成都各地交换巴郡及余缅所部楚军,九月,和议成功,南楚失去了占据多年的蜀中大半领土。陆灿力排众议,令余缅守巴郡,并于夔州设重兵为巴郡后援。

之后一年,雍军再无进取,余缅守巴郡毫无疏漏,雍军没有得到顺江而下的机会。江陵、江夏也是稳如泰山,雍军几次攻竟陵、随州,都未成功,淮西、淮东虽然战线时时变动,但是雍军始终也不能尽得江淮之地。连续三年的大战,南楚军在陆灿指挥下越战越强,再有江淮之险,水军之利,战事陷入僵持阶段。

\chapter{第三十二章 腐鼠成滋味}

同泰十四年丁亥八月,国主陇大婚,册蔡氏为王后,司徒蔡楷次女也,贵妃纪氏,同日册立。

八月十八日,国主亲政,御金殿受贺,加恩内外,罪非殊死,咸赦除之。

——《南朝楚史·楚愍王传》

同泰十四年七月,江南流火,热浪滚滚,江水东流,时值正午,就是江面上也是行船寥寥,而在江边一棵大柳树下面,却坐着一个绿衣少女,虽然看上去还不到豆蔻年纪,但是清丽绝俗,动人之处宛若仙露明珠,她身上衣衫正是江南寻常少女爱穿的夏衫,朴素无华,但是只见她明眸善睐,容颜如画,便知道非是寻常小家碧玉。她抱膝坐在青石上,一双清澈明净的明眸望着在江边踱来踱去,全然不顾及头上烈日的少年,眼中尽是疑惑。那少年英武俊秀,十三四岁年纪,虽然相貌稚嫩,但是已有英姿勃发的气度,不过他此刻却在江边踱步张望,神色焦急紧迫。那绿衣少女终于忍耐不住,扬声道:“二哥,你不是雇了船么,怎么现在还没有来?”

那少年苦着脸道:“明明说好了今天在这里见面,船资也预付了一半,怎地这般不讲信用。”

那少女抱怨道:“都是你了,一定要拉着我去寿春看望嫂嫂,还不告诉娘亲知道,若是不然,我们就可以跟着义叔一起上路,也不会在这里晒太阳。”

少年眼中闪过一丝无奈,却迅即掩去,道:“可是你说的,娘亲不会让你去寿春的,我本来是要去钟离见大哥,好跟着他上阵杀敌,如果不是你强要跟着我,我就可以大摇大摆地上路,也不用私下里在这雇船了。”

那少女俏脸气得通红,她本是温柔娴雅的千金小姐,虽然也曾幻想外面的广阔天空,但是却没有勇气离家出走,若不是这个二哥一边冷嘲热讽,一边暗暗怂恿,自己哪有胆子跟他出门,甚至瞒过了娘亲。想到此处,想要大骂一通,偏偏她生性温柔,最是不习惯骂人,一时间气得说不出话来。

这时,那少年突然指着江面道:“太好了,船来了。”少女闻言也是大喜,站起身向江面望去,只见一艘小型客船凌波而来,不多时停在岸边,那少年对站在船头的中年船夫道:“顾大叔,你怎么才来啊?”

那中年船夫道:“陆公子,今天小三突然闹起肚子,不能上船,小人一人不能驾舟,只得临时找了个侄儿做帮手,这才误了时间,还请公子见谅。”

少年脸色缓和下来,道:“原来如此,三哥没事吧?”

中年船夫笑道:“没什么,想必是吃了什么不干净的东西了,公子请上船吧。”

少年向船尾望了一眼,那把舵的青年肤色古铜,精壮憨直,这才回头道:“梅儿,上船吧。”

那绿衣少女闻声答道:“知道了。”说罢走了过来,她虽年幼,却是秀美非常,那中年船夫虽然见多识广,也不由暗赞一声,搭上跳板,让那两兄妹上船。两个少年少女,谁也没有留意到,那把舵的青年微微低头,掩去眼中暴射的精芒。

上得船来,一叶小舟逆流而上,骄阳似火,江风也带着熏人热气,两个船夫驾驶小舟前行了十余里,便转向驶入一条小河流,这条河流水面宽阔,八面来风,两岸绿柳如荫,枝叶蔽天,映在江面上,笼罩出一片清凉,乃是夏日过往船只休憩的最好去处,如今河内已有十余艘大小客船或是货船,其中更有一艘华丽的画舫,黑木描金,秀丽狭长的船身宛似江南少女纤细的娇躯,船头上悬着数盏宫灯,虽然现在没有点燃,可是灯上龙飞凤舞的四个大字仍然清晰可见。少女一眼望见,低声念道:“如梦画舫。”面上露出羡慕之色,道:“二哥,好漂亮的画舫啊,要是能上去看看就好了。”

旁边那少年听见,嘴角露出苦笑,他可不像妹妹一般大门不出,二门不迈,平日游走建业城内外,自然知道如梦画舫的事情,为难了片刻,道:“梅儿,那不是你该去的地方。”少女眼中露出奇怪之色,望向二哥,道:“二哥不是骗我吧?”面上神情满是怀疑。少年想要辩解自己从不骗人,却发现说不出口,毕竟自己从前对妹妹所说的话,十句里面往往有九句是假的,只得赧然道:“梅儿,那是江南第一花魁柳如梦的画舫。”

那少女虽然年幼,却也听过柳如梦的声名,虽然尚不懂风月之事,也隐隐知道其中含义,不由面上一红,正想避入舱中。这时,从那画舫之上传出清丽凄婉的箫声,那动人的旋律宛似寒水一般流淌入人心,那炎热的夏日仿佛也失去了威力,接着便从画舫之上传来了一律天籁也似的歌声。

“守得莲开结伴游,约开萍叶上兰舟。来时浦口云随棹,采罢江边月满楼。花不语,水空流,年年拚得为花愁。明朝万一西风动,争向朱颜不耐秋。(注1)”

少女听得入神,对少年道:“好美的歌声,好动人的箫音,二哥,今日难得有此良机,让我去看看柳姑娘好不好?”说罢,眼中流露出期盼之色。

少年眼中露出犹豫之色,但是见到少女神色,心中一软,终于叹息道:“好吧,柳姑娘名动江南,你就是见她一面,爹爹知道了也不会过分责怪。”说罢让那中年船夫向画舫驶去。

不多时,小舟靠近画舫,画舫上面一个秀丽的船娘望见小舟,脆声道:“这位小公子,你有什么事情?”

那少年叹了口气,看看妹妹眼中祈求的神色,道:“请禀告柳姑娘,陆风、陆梅途经此地,听到姑娘仙音,想登舫一见。”一边说着,一边按向钱袋,心道也不知道银子够不够。

那船娘噗哧一笑,道:“小公子,你这般年纪,别是开玩笑吧?再说我家姑娘不过是在此休憩,并无会客之意。”

少年脸上一红,看了一眼妹妹,道:“不敢相瞒,实在是舍妹听了箫歌,心醉神迷,因此想要见见柳姑娘。”

那船娘微微一笑,看向陆梅,眼中神光一闪,走到舱门低声说了几句话,不多时转回道:“我家姑娘说了,既是知音之人,就请上船小憩。”

少年陆风心中一宽,对着船夫低声嘱咐了几句,带着陆梅上了画舫,从舱中走出一个秀丽侍女,挑起珠帘,两人走了进去,便只觉得舱内一阵清凉之气扑面而来。

陆风定睛瞧去,只见舱内十分宽敞,陈设素雅高华,内侧摆着一张藤床,上面放着一张小方桌,桌上摆着银盘,盘内是冰镇的西瓜,舱内更是摆着冰盆,怪不得清凉无限。一个女郎就倚在桌前,一身素衣,全无锦绣,青丝如墨,垂在身前,虽然是淡扫娥眉,却别有一种妩媚明艳。而在舱内还有一个青衣男子,站在窗前,凝神望着珠帘之外的烟柳江岸,青衫及地,腰悬竹箫,自有一种漠然高华的风姿。

陆梅却无心打量舱内陈设,几步走到藤床之前,欢喜地道:“你便是柳姐姐么?你的歌唱得真好!”

柳如梦本无心见客,但是方才琴师宋逾示意她见一见两人,所以才相邀陆氏兄妹上船,但是见到陆梅这般毫无心机的欢喜赞美,也不由心中一动,浅笑道:“如梦本就是靠着这些谋生,小姐这是谬赞了。”说罢伸出纤纤素手,拉着陆梅坐到身边,秋波流转,已经将这少女上上下下打量清楚,只觉得这少女清丽秀美,年纪虽小,却是一个天生的小美人,若是长到十三四岁,必然是绝色,更令柳如梦动心的便是,这少女纯真无华,更有一种从骨子里透出的灵秀娴雅气质,一见之下,便知不是寻常人家的女儿。

越看越是喜爱,柳如梦笑着问道:“你叫陆梅么?果然是人如其名,我见尤怜,这是要去什么地方啊?”

陆梅望了一眼陆风,见他微微摇头,便道:“我和二哥去看大哥和大嫂,路过这里,听到姐姐的歌声,所以求二哥带我来见姐姐。”

柳如梦自然没有错过两人之间的微妙神情,但是以她的心机,自然知道装作不知道好些,道:“你也喜欢唱曲么?”

陆梅点点头,羞涩地道:“我唱得不好……”

陆风心中不耐,目光落到那青衣男子身上,上前几步道:“这位想必就是宋逾宋先生,久闻先生之名,今日相见,幸何如之。”

逾轮闻言回过头来,淡淡道:“陆二公子乃是将门虎子,怎会留心我这么一个小人物?”

陆风心中一震,他虽然年轻,却是聪明过人,对建业的人物多有知晓,自然知道这个宋逾的才名,更知道此人乃是尚承业的心腹谋士,这几年尚承业得他襄助,在朝堂上大有斩获,已经非是从前碌碌无为的勋贵子弟。方才陆梅想要上船来见柳如梦,陆风便想到吹箫之人必是宋逾,此人放荡不羁,除了偶尔给尚承业献策之外,几乎常年都在柳如梦身边。

他的神情变化逾轮也是看在眼里,心道,传闻陆氏在建业的暗势力倒有大半掌握在这少年之手,如今看来果然是真的,要知道宋逾乃是尚承业谋士一事,十分隐秘,除了少数人物之外无人知道,而这陆风能够知道,可见他能够深入到陆氏在建业暗藏的力量内部。

得到这个答案之后,逾轮再不多问,转头看向窗外,神色冷漠,似是对身后之人全无兴趣。陆风心中却在苦思冥想,今日道左相逢,莫不是中了圈套不成,不由隐隐生出悔意。

过了将近一个时辰,日光西斜,江面上热浪消减,陆风便带着陆梅告辞。陆梅临别之时,神色依依不舍,这一个时辰,柳如梦教她许多音律歌舞上面的知识,令她生出感激之心,且柳如梦善于言辞,令人如沐春风,不忍分离。但是陆风早有去意,这一个时辰他可是度日如年,他有心探听宋逾深浅,不料此人言辞冷淡,不愿和他多说,令他十分冷落难堪,此刻自然匆匆告辞。

望着两人临去身影,逾轮眼中闪过一丝悲色,柳如梦走到近前,吐气如兰,道:“这两人你认得么?”

逾轮淡淡道:“这许多时候,你还不知道他们的身份么?”

柳如梦柳眉轻扬,道:“我才懒得多问,何况这小姑娘温柔可人,我也不愿用什么心机,反正不过是萍水相逢罢了。”

逾轮漠然道:“那是陆灿的次子陆风和爱女陆梅。”

柳如梦微微一怔,道:“竟是大将军的子女,倒也难得,那陆梅的身份,就是公主也未必比她尊贵,她倒是没有一丝傲气,真不愧是名门之女,只是这样的千金小姐,怎会孤身随着兄长离家呢?”

逾轮淡淡道:“名门之女又如何,也逃不过争权夺利,近日国主就要大婚,大婚之后便要亲政。这大婚一事极为重要,立谁为王后更是重中之重。”

柳如梦恍然大悟道:“原来如此,以大将军的身份地位,莫非这位陆小姐要做王后么?不过她似乎还不到十三岁,是不是小了一些?”

逾轮冷冷道:“年纪有什么关系,若非是陆小姐尚不足十三岁,未到待选之龄,只怕现在已经列入选后名册了。这次立后,朝廷上下争论不休,尚维钧虽然有意将族女立为王后,但是却被陆灿上书谏止,毕竟尚氏已经有了一位太后,若是再出一个王后,未免有些过分。”

柳如梦若有所思地道:“若是大将军有意令陆小姐立为王后,为何现在陆小姐却在外面游荡呢?”

逾轮漠然道:“大将军可没有这个意思,前些日子太后便示意陆夫人,有意将陆小姐选为王后,如今看来陆氏是不愿意了,只不过大概不想公然反对,所以才让陆小姐离开建业吧。”

柳如梦美目流转,道:“太后既有这样的意思,却被陆氏暗拒,大将军岂不是得罪了太后。”

逾轮冷笑道:“这也没有办法,你可知道尚相心意,是绝对不愿看到陆梅为王后的,一旦陆氏成了国戚,只怕尚相就是梦中也会惊醒,所以他以陆梅年幼为由阻止,主张册立蔡氏女为王后,但若是陆氏不和皇室联姻,尚相也会忐忑不安,所以他竟提出册立陆梅为贵妃的荒唐主意,偏偏太后心志不坚,既希望和陆氏联姻,却又屈从尚相的心意,想要委屈陆梅为贵妃。也难怪陆氏放纵陆梅逃离建业,陆灿如今在南楚的地位何等崇高,他的女儿若是进宫,若是不做王后,岂不是面子全无。”

柳如梦思之再三,叹道:“这样一来,不论如何,陆氏和尚相都要结下仇恨,传闻昔日尚相曾经有意将义女配于陆云,却被大将军拒绝,后来又有意令陆云尚淑宁公主,却被大将军以陆少将军已经订婚为由婉拒,如今陆小姐又逃避选后,只怕太后和国主会以为大将军轻视朝廷,这件事情终究是后患无穷。”

逾轮闻言,眼中悲色越发浓厚,道:“于今腐鼠成滋味,猜疑鲲鹏议不休,大将军岂是贪慕权势之人,更无攀龙附凤之心,只是尚相这样的人是不会相信大将军的心志的。”

柳如梦也是轻声叹息,良久才道:“你不如设法请尚大人向尚相解释一下,如今大将军统军在外,对着大雍百万铁骑,若是朝中生了什么变故,只怕大厦将倾。”

逾轮一声长叹,没有言语,心中想起昨日受到的指令。那上面熟悉的字迹令自己心中巨震。

“赵陇即将亲政,大婚立后迫在眉睫,陆氏独秀江南,尚氏必欲陆灿之女为后妃,灿性高洁,必不肯卖女求荣,其间必定生隙,可说服尚氏,若灿为国丈,必有谋逆之心,以此断绝联姻之意。”

逾轮心中默念多遍,暗暗苦笑道:“先生,在你心目中,若是成了你的敌人,你便不会有任何慈悲么?那陆灿本是你的门生,如今你却要将他置于死地,只是你却为何对我这般纵容?”

再想到三年来得到的三封指令,逾轮心中只觉冰寒刺骨。

同泰十二年襄阳失守,消息传到建业,尚维钧惊恐万分,想要将襄阳守将容渊下狱问罪,那人传来第一道指令,让自己献策,趁机散布流言,说是陆灿有意令朝廷问罪襄阳将士,却让尚维钧出面收服容渊,此举不仅让容渊对陆灿更加怀恨,更是让尚维钧拥有了军方的支持力量,也让自己得到了尚氏的信任。

同泰十三年巴郡失守,余缅固守剑阁,成都守军也是死战不降,两军胶结,大雍提出和议,若是南楚放弃剑阁、成都,就将已经被困住的楚军交还给南楚,并且愿意将巴郡还给南楚,陆灿决意不许,要遣水军入蜀援救,自己得到第二封指令,通过尚承业劝说尚氏,一旦水军入蜀,长江防线必定空虚,若是久战不下,一旦定海雍军趁机发难,只怕会危及建业,与其分兵苦战,不如扼守巴郡,免得雍军顺江而下。和议成功之后,自己又按照指令趁机劝说尚维钧加罪余缅,陆灿大怒,和尚维钧当面相争,终于令余缅继续镇守巴郡,却是更加增加了尚维钧对陆氏的疑忌。

如今再加上这第三封指令,逾轮心知肚明,尚维钧对陆氏的猜忌将要到达顶点,随着赵陇亲政,三年来按兵不动的江哲,只怕即将展开反攻了。

————————————

注1:晏几道《蝶恋花》

\chapter{第三十三章 沧海两茫茫}

隆盛十年丙戌,吴越有异人助义军,于沿海乡镇建地道寨垒御雍军,雍军虽势强,不得其门而入,吴越渐安。

——《资治通鉴·雍纪四》

碧海潮生,彤云密布,眼看就要下雨了,可是箕坐在海滩岩石上面的青年却是神色沉重,完全没有回去避雨的意思,他是吴郡镇海人,同泰十二年东海水军上岸劫掠,他的父兄都是出色的铁匠,所铸的兵刃吴越闻名,因此被劫掠带走,只留下老母兄嫂还有两个侄儿,他当时不在家中,所以幸免于难。后来他加入了义军,只盼再也不让雍军上岸劫掠,更深的期盼却是能够见到父兄之面,只是不知父兄如今可还活着,想到此处不由痛心疾首。

正在他眼中渐渐朦胧之时,无意中目光一闪,却见海上几艘轻舟乘风破浪而来,船上皆是身穿软甲的雍军,他大惊失色,起身高叫道:“雍军来了,雍军来了。”但是今日眼看就要下雨,巡视这段海岸的义军都懈怠未来,那青年虽然高声叫喊,却没有人听见。跑出没有多远,耳中听到风声,青年向侧边扑去,身后传来一声惊咦,一刀斩空,那人顺势横斩,青年闪身避开,却被另外一个雍军军士一脚踢倒,那挥刀攻击的军士趁机用刀指住青年的咽喉,冷冷问道:“寨中有多少义军?云子山在何处?”

青年闭口不言,眼中露出倔强的神色。那雍军军士微微一笑,也不多问,挥刀便要斩落,那青年突然开口问道:“你的刀是谁铸的?”刀锋一顿,蓦然停住,只是将那青年颈上划破一道血痕。这时候,除了驾驶海舟的军士仍在船上之外,其余雍军已经陆续上岸,其中一人衣甲略有不同,显然是首领身份,他听到青年问话,上前笑道:“你不知道么,我军从吴越掳走许多工匠,这些人被编入定海匠造营,他的刀便是你们镇海最有名的铸剑师公孙墨所造。”

青年眼中闪过一丝难以抑制的喜色,用颤抖的声音问道:“他还活着,那么他的儿子呢?”

那执刀军士眼中闪过意味深长的神色,道:“你是说公孙般么,他铸的刀也是不错的,不过他更擅长制造弩机。”

青年忍不住落下泪来,爹爹和兄长都还活着,终于得到亲人音讯的喜悦让他难以自抑。耳中传来那军士冷硬的声音道:“你和公孙墨有什么关系?寨中有多少义军,你若老实招供,我便饶你一死。”

青年眼中闪过利芒,道:“你们掳我骨肉,侵我乡土,在下便是一死,也不会告诉你们义军的情报。”说罢挺身而起,咽喉向刀刃上撞去,那军士眼明手快,迅速收刀,却仍然在那青年颈上划破了一个大大的伤口,鲜血泉涌,青年的视线开始模糊,心中生出强烈的遗憾,若是能够告诉娘亲父兄尚存的好消息,自己就是死了也没有什么关系,只是如今娘亲却要承受更多的悲痛了。

望着陷入昏迷的青年,为首的军士眼中闪过寒芒,道:“是条好汉子,给他一个痛快吧。”

那执刀军士却目光一闪,在那为首的军士耳边低语了几句,那为首军士闻言沉思片刻,道:“就这样办吧,他伤得不重,替他裹好伤势,让他自生自灭就是。”

那为首的军士略一思索,道:“好主意,就这么办吧。”说罢举步海滩上走去,前面便是防海堤,越过防海堤不远便有义军军营,登陆偷袭已经是东海水军驾轻就熟的作战手段,义军虽然骁勇善战,不过却也是防不胜防。在这军士身后,雍军军士自然而然的结成战阵,向前走去,凝固的杀气冲天而起。

当那青年被雨水浇醒的时候,只觉颈上疼痛难当,他挣扎着爬起,回头四顾,却是没有一个人影,自己躺在防海堤上,颈上已经被人包扎妥当。他踉踉跄跄地站起,向营垒奔去,不知道摔倒了多少次,身上皆是泥污,等他奔到营垒,却是呆若木鸡,只见营帐内外,皆是七零八落的尸体,大雨汇成河流,雨水混合着血水,从营帐内外流淌。青年俯下身去,只觉心中悲愤欲绝,良久,他站起身来,内外巡视了一圈,虽然面上皆是血泪,但是眼中却是多了几许神采,低声道:“太好了,没有全死,没有全死。”他数了一遍,这里只有三十余人的尸体,这里原本有百人驻守,看来大部分的人应该是逃走了,就是最坏的结果,也不过是被雍军俘虏去了定海,凭着今日所知,那些兄弟也不是非死不可,想到此处,他心中宽慰许多。但是他突然想起那些雍军盘问自己的话语,他们是冲着云先生来的,若是那些同伴落在雍军手中,大刑之下招了供,说出了云先生的下落,岂不是糟糕至极。云先生主持沿海村寨的地道涉及修建,劳苦功高,岂能让他受到伤害,想到这里,他振作起精神,决意去向云先生报告此地发生的事情,让他暂时躲避起来。这时,天空中雷声轰鸣,电闪连连,大雨倾盆而下,天地之间皆是雾水蒙蒙,数丈之外,几乎是看不到人影,青年踉跄的背影很快就消失在雨雾之中,却不知身后跟上了两个黑暗的影子。

海浪滚滚,在壁立千仞的山崖之下汹涌激荡,崖下乱石嶙峋,惊涛拍岸,宛若千堆雪,碧涛之中藏着无穷杀机。雨后初晴,荆信立在崖上,心中轻叹,离开嘉兴已经整整三年了,想到渡过茫茫碧海,就是日日思念的故土,他心中越发生出悲意。

耳中传来轻健沉稳的足音,荆信没有回头,只是淡淡道:“霍兄今日怎么有空过来?”

霍琮微微一笑,这三年来荆信对自己仍是耿耿于怀,也不在意,站到荆信身边,道:“先生有令,命我去江南行辕见他。”

虽然只是淡淡的一句话,荆信却是身躯一颤,良久才略带嘲讽地道:“恭喜霍兄,这几年霍兄困在海上,恐怕不比荆某自由多少,如今蛟龙出海,再不需困在浅滩,想必公子定是万分欢喜吧?”

霍琮闻言,眼中闪过一缕笑意,道:“荆兄言重了,在下留在定海,不过是因为海路被阻,陆路难行,且靖海公尚有借重在下之处,所以才留在定海。而且靖海公在普陀周边数以百计的大小岛屿之上,安置了五十多万从吴越掳来的平民,地域广阔,岛屿众多,户口繁密,在下受命,暂代普陀县令,政务繁忙,不啻一县之主。管理五十万心怀疑忌敌意的俘虏,还要为大军提供粮草辎重,这样的重任,却交给在下一个未曾加冠的少年承担,已经是十分重用,怎谈得上龙困险滩呢?”

荆信闻言冷笑道:“以霍兄之才别说是一县之主,就是作个知州、郡守也是绰绰有余,困在普陀管理我们这些被俘之人,岂不是大材小用。”

霍琮却笑道:“荆兄这却是太看轻了这个县令之位,这几年荆兄帮我做了不少事情,开荒屯田,钱粮刑名,这些庶务看起来简单,做起来却是千头万绪,荆兄难道还不记得我的狼狈模样么?”

荆信不由噗哧一笑,顷刻间尴尬的气氛消失无踪,想到三年来这少年带着被俘虏至此的吴越民众,修建房屋,屯田渔猎,将荒凉的普陀群岛变成了可以安居乐业的乐土,虽然尚有雍军兵戈在外,又不时征用岛民至定海服役,但是总算没有更可怕的事情发生。不过霍琮所说的确属实,那些琐碎庶务,原本荆信也没有看在眼里,可是被这少年拉在身边一起处理政务,几乎忙得他昏天黑地,才知道就是一个小小的县令也不好做,尤其是两手空空,白手起家的县令。

见荆信开怀,霍琮心中却生出淡淡的惆怅,虽然在普陀这三年他大有斩获,可是这并不能说明荆信所言非是真情,事实上,以霍琮的聪慧,早已发觉了虎贲卫之中有暗中监视自己的人,甚至从姜海涛的眼中也看到了些许的猜疑提防。他早已明白,先生果然是将自己软禁在了普陀,只不过拘禁自己的是茫茫碧海,而非是刀戈武力罢了。否则虽然定海水营阻住归路,但是私航贸易越来越盛行的今日,哪里寻不到机会让自己返回大雍呢?是否先生知道了一些什么,霍琮曾经这样想过,甚至生出自暴自弃之心,若是自己刻意作些什么,或者先生一纸令谕,就可以取了自己性命,也免得自己心中为难。可是之后不绝于途的书信却让他生出愧疚之心。

大概是因为道路阻隔的缘故,有的时候十天半月也收不到一封书信,有的时候却是一下子受到好几封,有的信中解释一些自己回信中提到的疑难,有的信中给自己讲解军政大略,每封信中都蕴含着浓厚的情谊,更令霍琮心中不安愧疚。

先生信中虽然没有说明为何将自己留在定海,却让姜海涛任命自己为普陀县令,并要求自己踏踏实实作一个地方官吏。虽然管辖的不是普通百姓,而是吴越俘虏,但是政务却是更加繁重,兢兢业业做了三年县令,深知为政之难,霍琮心中明白江哲苦心,但是却还是无论如何也忘却不了江哲将自己弃在定海的举动,并派人暗中监视的举动。目光瞥向荆信,心中暗暗苦笑,虽然荆氏仍然是俘虏身份,但是却在普陀担任了许多内政职务,荆氏老家主更是已经随着南闽越氏的商船去了长安休养,只要南楚平定,这些普陀俘虏回到吴越,必定会先被任用,可谓前途无量,倒是自己,虽然现在掌握着他们的生杀大权,却不知下场如何。

过了片刻,霍琮终于平静一下心情,对荆信道:“我奉命去见先生,所以想将这县令之职交给荆兄接任,不知道荆兄意下如何?”

荆信先是一惊,继而平静下来,普陀政务一向由被俘民众自行管理,只是县令一职却由霍琮担任,并控制着岛上唯一的一支武力,用来镇压可能的反抗,如今霍琮离去,这个职位自然需要有人接替,自己虽然是楚人,但是这几年辅佐霍琮,也算是十分得力,再加上姑夫的缘故,就算是自己仍然想要忠于南楚,只怕也没有人会信了。想了许久,他终于道:“罢了,我又何必自欺欺人,这县令一职我接任就是。”

霍琮微微一笑,知道三年时光,岛上的吴越士子终于开始屈服软化了,荆信本就是他们的领袖人物,有他继任县令,更可以安抚岛上掳民。想到先生之命自己终于完成,便是前途茫茫,也觉得心中无限欢喜。

离开普陀,乘上海舟,霍琮放下心事,这艘海船的统领和他素来交好,见霍琮站在船尾望着普陀,似乎十分留恋,便上前笑道:“霍参赞何必这样伤怀,今次楚侯召您前去,想来将有重用,我们这边不过是小打小闹,到了那边,才是金戈铁马,痛快淋漓呢?”

霍琮勉强一笑,道:“在海上待了三年,只是有些舍不得罢了,难怪先生总是对东海念念不忘。”

那统领不知霍琮心事,只是寻些有趣的事情和他叙说,霍琮虽然随口应对,心思却已经飞到了千里之外。

过了小半个时辰,霍琮回到了定海,如今的定海已经非是三年前那般残破,岛上的军营庄严肃穆,到处都可看见阡陌交错的情景,后岛匠造营内,叮叮当当的声音终日不绝,船坞之内也有吴越工匠配合着东海工匠修补船只,若是降服便可得到善待,若是反抗便会被处死,被掳来的吴越平民早已经大半默认了雍军的统治。当然,尽管吴越掳民降服者众,但是能够上得定海的也都是经过精挑细选的,免得他们趁机作乱。这一切的兴盛场面,都有自己的汗水渗透其中,霍琮心中生出自豪之意,迈步走向中军大帐,在他身后跟随着四名虎贲卫士。

当年江哲脱走吴越之后,这些虎贲卫士几乎都被留在了定海,后来战事胶结,这些人除了半数有机会去了雍楚前线护卫江哲,其余都被江哲强令留在了霍琮身边,不过霍琮自认没有使用虎贲卫士护卫的身份和必要,最后在靖海公斡旋之下,双方达成协议,除了霍琮身边随时都要留下四个虎贲卫士护卫之外,其余的虎贲卫士都跟着东海水军上岸劫掳吴越,免得他们的刀都钝了。这样的结果倒是皆大欢喜,有这些武功高强的虎贲卫士加入,对付吴越义军中的武林高手倒是多了许多保证,而霍琮也不会觉得如坐针毡,不说这些虎贲卫士中有奉了江哲之命监视自己的人物,就是没有,他一个尚没有正式入仕的少年,怎敢使用皇家的铁卫为护卫呢?

中军大帐之内,姜海涛得知霍琮将到,也是颇为高兴,这三年来这少年相助自己不少,只是江哲令虎贲卫士暗中传书自己,让自己留意霍琮行止,甚令自己生疑,初时尚以为不过是先生考验弟子罢了,但是后来却传书让自己将霍琮困于普陀,虽然是重任,却是羁绊岛上,不能北返,姜海涛虽然率直,也知其中定有文章,却是不忍多问,毕竟霍琮十分得他赏识。想到即日霍琮就可回到江哲身边,想必江哲已经回心转意,他心中欢喜,不亚于隆盛九年承帝命晋升公爵之时。

霍琮走入帐内,向姜海涛行礼之后,姜海涛将一份文书递给霍琮道:“我军海船若是北上,难以避过宁海的阻截,不过恰好有南闽越氏的海船北上高丽,这是你的身份文书,安全北上应该不会有问题。”

霍琮自然知道这几年虽然两军交战频繁,可是吴越许多大世家却和宁海军山的将领勾结进行私航贸易,因为参与私航贸易的两家船行海氏和越氏都和姜家有着密切的关系,所以定海也是睁一只眼闭一只眼,甚至从中获利不少,当然对于姜海涛来说,最重要的是通过这种贸易,可以获得短缺的物资粮草,这对于被宁海军山截断归途的东海水军来说十分重要。至于利用两家船行,传递一些情报,护送往来信使,这更是不可言传的好处。对于参与私航贸易的世家来说,从中获取的暴利足以让他们忽视这样做产生的资敌后果。若非是为了维持平等的合作地位,这些世家暗中支持吴越义军不遗余力,早有人会对他们下手了。

交待了一些公务之后,姜海涛正色道:“还有一事也颇令我为难,还请你转告先生,这半年多来,吴越沿海许多村镇请了高人,在村内挖出地道躲避我军,我曾收买其中一些人,得知那些地道宛若蛛网,若无人带领,十有八九都会走入歧途,被暗藏的无数机关毒烟所伤。我军还未进村,村内乡民已经躲入地道,甚至连粮食钱财都藏了进去,令我军徒劳无功。”

霍琮已经有段时间没有接触军务,听了十分好奇,道:“不知是何人想出了这个主意,可有什么线索么?”

姜海涛苦笑道:“倒是有一点线索,前几日我得到消息,得知那人正在镇海附近主持修建地道寨垒,便遣出好手突袭,他们上岸之后便先歼灭了一支巡哨义军,又留下活口,令其不知不觉中引路前往,果然见到了那个云子山,可是他身边有许多高手护卫,在我军数百勇士的围剿下居然还让这人逃了出去,当真是令我军颜面无存。根据俘虏的口供,只知道那人是吴越第一剑丁铭的好友,身份不明,却是最擅长机关暗器。你见到先生之后,将我的麻烦跟他禀明,若是没有什么好办法应对,只怕这样下去,我军在吴越劫夺的钱粮会越来越少,现在我军的粮食还不能自给,若是不能从吴越获取相当的数量的钱粮,麻烦可就大了。”

霍琮听了,陷入沉思当中,表面上看来只是吴越出了个麻烦人物,为什么他心中会隐隐觉得这其中有些蹊跷呢?

\chapter{第三十四章 欠东风}

陆灿,江夏人,镇远侯嫡嗣,祖父平,武帝时为大将,忠勇以闻,父信,督军江夏二十年,沈厚精忠,朝野共钦。公少失恃,随父入军营,十余岁,能挽三石强弓,有神力,虽百战勇士不能敌。信每谓左右,曰:“此子功业必在吾上。”

公自幼好武厌文,因国中崇文轻武,信为之忧心,延师教读。公性顽劣,履驱西席。显德十一年,信聘嘉兴江哲为西席,时哲仅十五岁,或虑公不能安,然公改颜相事,执礼甚恭。

显德二十二年,哲被掳入雍,降之,未数年,雍帝赐封楚乡侯,又尚大雍宁国长乐公主,国人闻之愤然,昔日同僚旧友皆诟厉之,唯公默然,或有讦公,公曰:一日为师,终身为父,焉能因不得已之事而绝之,讦者闻之,愧而退。

——《南朝楚史·忠武公传》

隆盛十年八月初,从海州通向徐州的驿道上,行人络绎不绝,刚刚下了一场大雨,驱除了炙人的炎热,从海面上吹来的风带着淡淡的腥气,也带着无比的清新。这时,远处烟尘滚滚,辚辚车响传入耳中,连绵不绝的辎重车队在雍军军士护卫下从海州方向走来。路上的客商旅人都纷纷向路边让去,这样的情形几乎每隔十天半月就会上演,所以他们不需要等到军士下令就自动避开。大雍和南楚开战数年,耗费粮饷辎重无数,虽然雍军也在当地屯田养兵,可是还是需要从大雍各地运来钱粮辎重,而从幽冀运来的钱粮主要就是通过海州云台港转运徐州的。

在这支浩浩荡荡的军队中,却有一个未穿甲胄的青衣少年策马缓缓前行,他正是霍琮,两日前他从云台登陆,本应快马加鞭赶赴徐州,可是上岸之后,他心中便生出忧惧之意,便故意拖延路程,又和运送粮草的军队一起上路,名义上是为了沿途安全。护卫他的虎贲卫士虽然对他的心思旁观者清,但是却也不忍揭穿,毕竟数年相从,他们和霍琮之间已经情谊非浅。

将近午时,押运粮草辎重的将领下令众军在路边休憩,那将领过来道:“霍公子,前面有个野店,末将往来此间经常在那里打尖,公子若是不嫌弃的话,就让末将请公子小酌一番可否?”

霍琮虽然心中忧虑,但是面上却是一丝也不会显露出来,那将领有意结好,他自也不会拒绝,便笑道:“将军好意,在下愧领。”说罢翻身下马,和那将领一边说笑一边向那野店走去。几个虎贲卫士则是自然而然的分出两人先去了那野店查探,这次霍琮离开定海,按理来说那些跟随霍琮留在定海的虎贲卫士再也没有理由留在定海,可是他们中的大多数都在东海军中效力,许多都已经担任了中级将领或其他重要职务,若是一下子抽离,不免影响东海军的战力。所以在江哲召回霍琮之前,上书雍帝,干脆将那些侍卫转入东海水军之中任职,除了四个常年跟在霍琮身边的虎贲卫士之外,其他人都留在了定海。那押送辎重的将领并不清楚霍琮的身份,可是只见这少年身边竟有虎贲卫随从护卫,也知道霍琮身份的重要,所以一路上毕恭毕敬,十分礼遇。而霍琮也趁机打听了许多徐州的情形。

自从隆盛八年江南行辕在徐州立下大营之后,几十万援军将淮北守得固若金汤,三年来数次大战,江淮之间血流成河,双方将领都是殚精竭虑,战场之外,谍探往来南北不绝于道,就是徐州也难以避免南楚谍探和江湖义士的渗透,而徐州更有齐王李显、太子李骏坐镇,所以刺客更是层出不穷。所以徐州早已进入军管,戒备森严。而令霍琮牵挂的恩师江哲,此时却不在徐州,虽然江哲身为江南行辕参赞,却似乎不甚在意军机大事,三年来不仅数次返回雍都,平日也多半往来荆襄淮北山水之间,或荡舟微山湖上,或登嵩山访佛寺,或流连于汉水岘山,竟是罕有过问军情大事。不过雍帝对江哲的纵容也是前所未有,不仅没有降罪,反而升了他的爵位,如今江哲已经是楚国侯之尊了,这令许多人眼红不平。就是霍琮,虽然知道江哲晋爵是因为隆盛八年的大功,可是江哲这般放纵也是令他颇为不解,授人于柄并不是自己这位恩师会做的事情啊。

霍琮心中千回百转,面上却是神色不露,和那将领谈笑宴宴的走向路边宽敞整洁的野店,掀帘走入店门,那将领正要高声招呼掌柜,目光一转,却是身躯一震,呆住不动。霍琮走在后面,见那将领举止有些不对,目光却被那人身躯所阻,看不见店房内有什么不妥,却是下意识地退了一步,而跟在他身后的两个虎贲卫士则是跟上一步,隐隐将他护住。

若是店内出了什么意外,事先进去的两个虎贲卫士应该会发觉示警的,霍琮心中疑惑,目光炯炯向内望去,这时候那将领竟是匆匆向前两步,拜倒在地道:“末将薛全忠叩见侯爷,不知侯爷在此,请恕末将擅闯之罪。”

听得此言,霍琮只觉得脑子里面轰隆一声,身体竟似僵住一般,目光越过那拜倒的将领,他向内望去,只见店房正中的座头上,坐着两个自己熟悉无比的人。那个容颜洁如冰雪,比起三年前容颜虽然有几分变化,却依旧华年如昔的青年,不正是先生时刻不离的侍从邪影李顺么。而那个青衫及地,灰发霜鬓,容颜上又多了几分风霜之色,双目却是越发温润深邃的男子,不正是阔别数年的恩师么?

那男子伸手虚扶,令那将领起身,然后目光望向店门处,笑道:“琮儿,三年不见,你不会是认不得为师了吧,真是枉费为师亲自来迎你的心意了。”

望着那双满是赞赏欣慰的深眸,霍琮只觉得心中纠缠多日的忧惧如同见到烈日的冰雪,转瞬间化去无踪,再也忍不住激动的心绪,扑到那男子面前,拜倒在地,哽咽道:“弟子叩见恩师,恩师一向可好。”语声未歇,滴滴泪水已经滴落尘埃。

见到霍琮双肩轻颤,却是强自抑制激动的模样,我也是心中震动,这一刻,我也不由生出歉意,想到这几年刻意委屈这个心爱的弟子,他小小年纪,就要承受这样的压力,也真是难为了他。上前将他搀起,挽着他坐下,笑道:“好了,这几年虽然苦了你,不过寻常人可是很难有这样的机会,像你这般年纪就牧守一方的,海涛传书来,说你助他作战十分得力,牧守普陀也是殚精竭虑,还要荐你正式任官呢。不过我却替你婉拒了,这几年不过是让你历练一番,也让你熟悉一下庶务,若是出去任官却是不必了,在我身边再学几年,到时候就可以直接辅佐太子殿下理政了,若是现在有了官职反而麻烦。”

听了恩师谆谆善诱的一番言辞,霍琮原本心中暗藏的不安渐渐淡去,拭去泪痕,这才发觉店内已经只剩下了江哲、李顺和自己,其他不相干的人都已经无声无息地退了出去,留下了一个独立的空间让他们师徒叙谈,至于李顺,霍琮自然知道此人与恩师本如一体,他留在此地并无挂碍,平静了一下心情,霍琮将心中久藏的疑问提出道:“先生,弟子在定海得知战报,心中长有疑惑,孙子有言,兵者,国之大事,死生之地,存亡之道,不可不察也。先生深通兵法,应知战事胶结,有害社稷黎民,若是能胜,理应速战速决,若是不能胜,也应偃旗息鼓,厉兵秣马,以待时机。先生得皇上器重,为何不尽心竭力,或者谏言皇上罢战,或者一鼓作气,平定南楚呢?”

我闻言微微一笑,道:“琮儿,天下有识之士都说南楚暗弱,为何大雍履攻不下?你可知其中缘故?”

霍琮正色道:“南人多半都存苟安之心,爱慕荣华,无心进取中原,若论两国战力,除了大将军陆灿麾下各部之外,其余多半战力不强,我军精锐可以以一当十,所以南楚无力对大雍产生威胁,此南楚之暗弱。虽然如此,江南富庶,沃土千里,又有江淮阻隔北方铁骑,更有蜀中扼守江水上游,利于防守,自古以来,扼守江淮割据江南半壁江山的诸侯数不胜数,南楚国主只要拥有民心,稳守江淮天险,再有一二名将扼守要地,军心如一,就可令大雍望长江而叹。如今南楚抚有江南数十年,虽然如今权臣秉政,但是政局尚称稳定,捐税并不沉重,平民尚可勉强安居,民心仍然依附,更有陆大将军这般的名将阻我军南下,所以战事胶结数年,履攻不下。”

我暗暗点头,霍琮这几年果然大有长进,又问道:“既如此,你看如今局势,双方谁占了优势呢?”

霍琮早已将这些事情想得通透,不加思索地道:“襄阳在我军手中,南楚军便没有北上荆襄,进兵南阳,威胁关中的可能,徐州固若金汤,南楚淮南军便没有北上青徐的机会,蜀中大半已经落入我手,南楚军只能据巴郡、夔州自守,如今南楚军只能被动防守,优势再何方不问可知,只是南楚军仍然能够自保,而且这几年兵锋磨砺,南楚军的战力也渐渐加强,若是再拖延下去,此消彼长,说不定优势就会转到南楚军手中。”

我欣慰地道:“你能够看穿这一点,果然没有荒废时光,不错,现在南楚似危实安,而我军虽然占据优势,却是外强中干,陆灿非是不思进取之人,三年前他趁着我军没有及时增援的机会,突袭楚州、泗州,若非我军先在定海发难,只怕已经被他趁机夺取了空虚的徐州。虽然我因势利导,利用襄阳守将容渊的心结,夺取襄阳,反而占了一丝上风,可是陆灿雄心却是展露无遗。如今南楚虽然处于弱势,可是却被陆灿趁着连年苦战,尽收江淮兵权,练就一支不逊于我军的精兵,只待我军稍现疲态,他就会奇兵突出,攻我军之不备,将大雍平楚的努力化为乌有。”

霍琮听得心惊胆战,低头苦思良久,才道:“陆灿为战,虽然常以防守为主,但是每每在敌军懈怠之际,突出奇兵,袭取要害城关,趁东川之乱取葭萌关是一例,趁我军败后修整之时,遣石观取宿州,杨秀袭泗州又是一例,如今两军僵持年余,只怕陆灿已经在谋划进攻我军重地了,只是不知他会将目标放在何处?”

我轻轻点头,叹道:“琮儿可知若想攻取南楚,最好的时机就是在武威二十三年,那时候北汉新败,蜀中尚没有完全平定,而南楚却是贤王驾鹤,君暗臣昏,朝野分崩离析,所以陛下可以率大军破建业,俘国主,全身而退,若是那时大雍可以一鼓作气,定有机会一举平灭南楚。只可惜那时候大雍朝中夺嫡之忧迫在眉睫,陛下虽然掌握大军,却不敢全力攻楚,军心不一,以致错失良机。等到朝中平定之后,北汉已经恢复了战力,北方战事再起,东川隐忧也是渐渐浮出水面,而南楚地广人稠,局势已经稳定,若是一旦南征,必是旷日持久,所以不得已定下先平汉,再灭楚的策略。等到北汉平定之后,为了消化北汉国力,又因为失去葭萌关,所以陛下又不得不休养生息,就在这期间,陆灿已经成为南楚军方第一人,虽然南楚朝政尽在尚维钧把持之下,可是军方却是没有人可以和陆灿抗衡,这是几十年来南楚军方少有的一统局面,我们已经失去了灭楚的良机。

若依我的意思,隆盛七年,就不应起兵平南,要知道当时尚维钧和陆灿一问一武,把持军政,若是大雍南征,纵然尚维钧心存恶念,也只能倚赖陆灿,大雍铁骑兵临江南,反而会让两人抛却嫌隙,共同对外。可惜陛下心切一统大业,终于决意平楚,以至于成全了陆灿,让他尽得江南军心。战事既起,我受皇命南来,原本有意利用定海牵制吴越,再在江淮、荆襄和楚军对峙,并不准备立刻启衅大战,不料陆灿却是主动进攻,更是利用战事连绵加强自己在南楚军中的地位。看到江淮、荆襄兵燹绵绵,我才确定陆灿心意,他不甘心苟安江南,竟有中原之志,虽然大雍有明主在位,又有名将雄兵,急切不可攻,可是只要陆灿夺去了北窥中原的门户,据守不让,等到南楚明君在位,就可以北上中原,虽然那可能是几十年之后的事情,可是却非是不可能的梦想。”

霍琮闻言,目中闪烁着寒芒,良久才道:“先生既然已经看穿陆灿心意,想必已经有了应对之策,这几年先生流连于山水之间,莫非是让陆灿不再着紧先生的举动么?”

我淡淡一笑道:“两军交战,斩将夺旗,非是我所长,就是我在军前,也起不到什么作用,若想对付陆灿,还需从南楚朝中着手。陆灿虽然有雄心,却是看不明局势,南楚朝政糜烂,国主赵陇刚刚亲政,就忙着选纳美女,大兴土木,修建宫室,不是明君所为,而尚维钧忌惮陆灿已久,只是碍着陆灿手中兵权,又因为大雍虎视眈眈,又没有借口,才隐忍不发,自古以来,朝中有昏君奸臣,大将岂有立功于外的机会。陆灿身遭疑忌如此,却不能以非常手段排除异己,掌控朝政,已是自蹈死路,我所需的只是一个局势,就可以陷陆灿于必死之地,何需和他沙场交锋呢?”

霍琮心思电转,转瞬之间已经将数年之间的事情回想了一遍,虽然他不知江哲暗中的许多布置,但是只是他知道的事情已经令他心中生出寒意,偷眼望了江哲一眼,他问道:“容渊莫非是先生安排给尚维钧的利器?”

我点头道:“容渊失守襄阳,乃是大罪,南楚朝廷竟然不曾问罪,只是降了他一级军职,更让他领兵将功赎罪,纵然是陆灿有心维护,若没有尚维钧首肯,焉能如此?容渊此人心胸狭窄,忌惮陆灿声望功业已久,陆灿也有错处,容渊是德亲王故将,性情又有固执偏狭之处,这样的人若不用之就需除之,免得他生出是非,偏偏陆灿因为不喜容渊排除异己的手段,不愿用之,却又任其主掌襄阳,以至于将帅失和,令我军趁隙取了襄阳,致令容渊不得已依附尚维钧自保,一旦尚维钧对陆灿动手,容渊就是操刀之人,陆灿却因为心中执念,不愿斩尽杀绝,反而有心弥补,任用容渊为将主江陵军事,岂不是错上加错。不过若非早知陆灿性情,必定不会落井下石,我又怎会放容渊逃生,昔日容渊仓惶南逃,我令人在风林关设伏,若非网开一面,岂会让容渊脱走,只因留下容渊此人,尚维钧才有对付陆灿之力。”

霍琮又道:“陆将军一心都在战事上,不免疏忽朝中之事,而且陆将军生性高洁,不喜欢争权夺利、谄媚事君,所以必然不得君心,尚相秉政之时还罢了,尚维钧不能随便寻个理由处置陆将军,但是一旦国主亲政,情势就不同了,雷霆雨露,皆是君恩,就是国主赵陇想要毫无理由的免去陆将军军职,陆将军也只能黯然从命,只不过因为战事胶结,这个命令也不能随便下达罢了。”

我叹息道:“大将在外,每有临机独断之事,陆灿为人更是刚毅果决,袭取葭萌关,用兵淮东,皆是独断专行,所以我大雍密谍虽然深入南楚朝野,却是没有得到兴兵的征兆,这样的举动本就是人臣大忌,纵然主上是明君圣主,也是杀身之祸,更何况南楚国主还算不上中兴之主,秉政的尚维钧又是权相之属呢?前些日子,南楚尚太后有意将陆灿之女陆梅选为王后,虽然受阻于尚维钧,仍有意选陆梅为贵妃,对陆灿来说,将陆梅送入宫中为妃本是最好的处置方式,一旦和王室联姻,陆灿就有机会掌控南楚政务,渐渐排除尚氏的影响,可惜陆灿却不是权臣,他也不愿出卖爱女换取富贵,我得到消息,陆梅在陆灿次子陆风护送下到了寿春,路上更有辰堂高手暗中护送,这样一来,赵陇必然对陆灿心怀不满,一旦情势变化,赵陇决不会想到要维护陆灿。更何况……唉!”

霍琮眼中露出悲意,接道:“更何况掌兵大将本就是君王猜忌的对象,陆将军手握重兵,又不愿谄媚王室,赵陇必然怀疑他的忠诚,自古以来功臣名将本就难免厄运,更何况陆将军如此耿介,一旦局势稳定下来,陆氏必然遭遇劫难。再有奸臣小人趁机进谗言,陆将军想要解甲归田也殊不可能。”

我淡淡道:“这样的情势,发展下去,陆灿唯一的生路就是起兵谋反,但是陆氏忠贞,天下共钦,他若真的起兵谋反,从前清名尽化乌有,江南必然大乱,到时候就是我军的机会,若是陆灿终究不反,必然难逃昏君奸臣的毒手,到时候江南柱石倾覆,还有何人可以抵御我军南下。”

霍琮低声道:“虽然隐忧重重,但是陆将军手握重兵,又在和我军激战,想来尚维钧尚不至于在这种情况下自毁长城吧?”

我眼中闪过一丝哀恸,道:“尚维钧不是蠢材,自然不会贸然动手,他若下手,一来是战事平定,二来是陆灿要有把柄落在他手中,只是我三年谋划,就是为了今日,如今万事俱备,只欠东风,数月之间,南楚即将大变,我召你前来,就是不想让你错过这决定南楚命运的变乱。”

霍琮只觉心中剧痛,三年前在吴越和陆灿也曾交手数次,虽然从未蒙面,也能觉出其人风采性情,实在是当时豪杰,想到此人即将死于阴谋之下,不由黯然难言,良久方道:“先生既言只欠东风,却不知东风何指?”

我目光一闪,道:“这东风便是襄阳,襄阳为陆灿必取之地,只是他攻取襄阳之时,就是南楚栋梁倾折之始。”

\chapter{第三十五章 襄阳恨}

公初为将,代父镇守蜀中,虽无盛名,然将士父老皆服其德,后主军机,屯兵江夏,督军江淮,北骑不得南下。

时,尚相秉政,不思进取,灿唯默然应之。同泰五年,灿不请上命,趁大雍东川变乱,轻骑袭取葭萌关,绝雍军入蜀道路。尚相闻之,怒责其矫命出兵,公侃侃道:“灿承父荫,有顾命重责,朝政尽付相爷,然军机大事,乃灿之事也,若待朝廷命下,事机泄矣!”尚相闻之,遂改颜相向,然心实忌之。

同泰十一年,雍帝以细故兴兵,三路大军,分取荆襄、淮西、淮东,淮东陷敌手,雍军据扬州,窥视江南,公亲率水营守京口,且遣长子云赴淮西寿春助石观部守淮西。雍军果如公所料,趁隙攻淮西,寿春激战十余日,军民闻云在,皆曰:大将军必不弃吾等,死守不退。雍军久战疲敝,为飞骑营所破,淮西遂安。淮西大捷,公趁势增援扬州,雪夜大破雍军于瓜州渡口,大战连捷,遂复淮东。公以一己之力,挽狂澜于绝境,后数年,雍楚大战,兵燹绵延千里,雍军虽强,终不能渡江水,公转战千里,百战百胜,世人评天下名将,列公为第一。

飞骑营,始建于同泰五年,初,公有意进取,唯虑江南少精骑,不能敌雍军,欲在江淮建骑营,为朝臣所阻。公不得已,欲借襄阳秘练精兵,渊疑公欲夺襄阳军权,阴阻公行事,两人遂生隙。后,公袭得葭萌关,蜀中皆入掌握,乃于其地秘练精骑,称飞骑营,淮西一战,扬名天下。公甚重飞骑营,骑营统领皆亲选,每休战,皆令将士被重铠习骑射,赏罚皆重,虽亲子不能免。飞骑精兵,不逊大雍铁骑,淮西鏖战,赖飞骑营多矣。

——《南朝楚史•忠武公传》

霍琮心中一亮,离开定海之时心中生出的疑惑豁然而解,出言问道:“先生,那在吴越相助南楚义军修建寨垒地道的云子山莫非是先生所遣?”

我但笑不语,扬眉示意他继续说下去,霍琮越发确定自己的判断,道:“弟子从靖海公处得知吴越有奇人襄助,心中便觉有些异样,先生在江南颇有力量,若非如此,也不能轻易往来吴越江淮,若是吴越果然有人精通土木建筑,先生不会不知,吴越战事,乃是先生一手挑起,若知有人阻碍先生大事,必然不会坐视此种事情发生。以先生在南楚的潜势力,绝不会让那云子山坐大到如此境地。所以弟子猜测那人和先生有些关联。

先生对门下事历来讳莫如深,旁人只知王骥、海骊、刘华、陆迩之名,皆为先生寄名弟子,却鲜有知晓这四人本名赤骥、盗骊、骅骝、绿耳,穆王八骏的典故凡是读书人多半读过,所以弟子猜测先生门下如赤骥者,共有八人,想来云子山就是其中第五人。先生虽然不曾告知弟子详细情况,弟子却知先生在机关土木之学上造诣非浅,想来那人就是承袭了先生这方面的衣钵吧?”

我微微一笑,道:“你这话若给别人听去,岂不是会以为我背了大雍暗助故国,这个罪名可是不浅。”

霍琮笑道:“欲先取之,必先与之,先生令那位师兄暗助义军,虽然令东海水军再吴越难有斩获,却也消减了义军的斗志,若是人人都躲在地道中避战,岂不是让我军往来自如,而且既然修建地道之人乃是我方之人,只需一纸地图就可以令我军按图索骥。不过我想先生未必是存了这样的心思,吴越战事应该不放在先生眼中,先生既然将襄阳当作诱饵,想必令吴越义军占据上风,就是为了让陆将军放心北上吧?”

我闻言轻叹道:“我用了三年时间,迫使陆灿进入我的局中,如今他唯一可能突破僵局的地方就是襄阳,陆灿决计想不到吴越的僵局是我设计,没有后顾之忧,他必然要锐意进取,江淮有齐王坐镇,他纵然有惊天手段也不可能取得太大的战绩,只有荆襄之地,虽有长孙冀镇守,却略现薄弱,而且容渊自失襄阳之后,切齿不忘这般屈辱,陆灿若取襄阳,容渊必然奋勇争先,而且南北之争,襄阳乃是军事重镇,陆灿纵然看穿我的手段,也不能不取襄阳,若不趁此北上,恐怕再没有这样的良机。”

霍琮疑惑地道:“可是弟子却不明白,襄阳如何成为先生的东风呢?”

我瞧了他一眼,淡淡道:“跟在我身边,你自然会知道什么是祸福相依的。”

霍琮闻言却黯然道:“弟子却宁愿终生都不会看到先生和陆将军师徒相残,先生纵然取胜,只怕也不会有丝毫欢喜。”

我本来正欲伸手去取桌上的茶杯,闻言手一颤,茶水飞溅,良久,我才淡淡道:“你还是不明白陆灿的品性,若能取我性命,他不会有丝毫犹豫,可是他对我的敬爱之心却也不会稍减半分,我既然决意南来,就不会对他手下留情,只是他始终也是我心爱的弟子。琮儿,你若叛我,我必亲手杀之,可是你若有什么苦衷,只要你说了出来,我都会替你担待。”

霍琮闻言心中一震,面色变得苍白,却是缄口不言,面上露出倔强的神色。

小顺子在我和霍琮谈话之时,已经起身避过一边,虽然数丈之内,不论我们两人声音多么细微,他都可以听得一清二楚,可是面子上还是要给师徒两人促膝私谈的空间,此刻见霍琮竟然不顾公子心意,执拗不言,他面上闪过一丝杀意,店房之内的空气都似乎冰冷沉凝了几分。霍琮本是心思灵透之人,只觉后颈寒毛倒竖,便知是小顺子动了杀机,可是他也是性情坚忍不拔之人,虽然压力滚滚而来,却是强自支撑,不肯露出丝毫示弱。

我见状一叹,这孩子终于还是不肯说出自己的心事,明明知道我一句话,就可以将他再次流放到偏远之地,甚至取了他的性命,却还是这般倔强,虽然有些遗憾这少年对我没有丝毫信心,但是见他如此,我终究是狠不下心为难他,只得微微一笑,道:“罢了,这些事情以后再说吧,你还是随我去襄阳吧。”

霍琮只觉身上一松,潮水般的杀气蓦然消褪,他忍不住拭去头上冷汗,目光望向江哲,心中暗道,或许过不了多久,自己便再也没有机会随侍恩师,只是不知道到时候恩师在处置自己之时,是否也会像对陆灿一般心存师徒之情,下手却是毫无怜惜。

几乎是江哲与霍琮师徒重逢的同时,在江陵城外,汉水之上,一艘楼船之上,南楚军方两位大将正在密谈,其中一人正是陆灿,另一人却是江陵守将容渊。距离襄阳失守不过三年,容渊却是苍老憔悴了许多,虽然对着南楚军方第一人,他的神情却是淡漠而疏远的,陆灿的神情从容冷静,但是目中却闪烁着热切的光芒。

容渊沉默良久,终于抬头冷然道:“夺回襄阳,乃是容某梦寐以求之事,大将军既有这样的决心,容某敢不从命,只是这种大事将军也要瞒着朝廷,难道就不担心国主怪罪么?”

陆灿叹道:“我岂不知此举定会引起非议,但是朝中情形容兄也应该知道,若是我真的请命而行,只怕雍军已经知道我军目标,况且将在外君命有所不受,陆某既然主持军机,就只能勉力为之。襄阳易守难攻,我会尽力将长孙冀诱出坚城,容兄趁机攻取襄阳,其间若有变故,容兄可相机行事。”

容渊眼中闪过寒芒,道:“大将军可知,若是这次不能取下襄阳,尚相必会问罪将军,如今国主亲政,将军顾命之权已经被朝廷收回,若是将军独断专行,必将授人以柄。”

陆灿淡然道:“若能够夺回襄阳,陆某就是担些罪名也无妨碍,敌我两军已经僵持年余,此时正是雍军懈怠之时,而我军却是卧薪尝胆,寻求报仇雪耻的战机,江淮、吴越战事胶结,正可以趁机进兵荆襄,襄阳乃是南北相争的军事重地,若不得襄阳,江陵、江夏都会受到威胁,我军也没有威胁敌军的本钱。”

容渊闻言肃然道:“末将必会全力以赴,不夺回襄阳,绝不收兵。”

陆灿心中略宽,容渊虽然和他性情不合,如今又已经依附尚维钧,但是他相信若有夺回襄阳的机会,容渊便会不顾一切的从命行事,而若想夺回襄阳,若不得容渊支持,希望就小得多了。想到此处,他转头向容渊望去,恰好容渊也正向他望来,两人目光相对,都觉出对方眼中的热切和战意,攻取襄阳之举,两人心志如一,因此之故,从前的嫌隙这一刻似乎也消失无踪了。

八月十二日,陆灿自江夏率军溯澴水而上,出义阳,义阳之南有三关,分别是武胜关、平靖关、九里关,武胜关、九里关在楚军掌握之中,平靖关则在雍军手中,三关互为犄角,皆是易守难攻,故而两军多年激战,鲜有在此的时候,陆灿却是从数年前便着手于此,多年谋划,大军压境,数日前攻破义阳,义阳守将战死。

八月十五日,陆灿出义阳,西略宛、邓,势如破竹,此举突如其来,在陆灿意中,长孙冀必然亲自率军前来迎战,大雍众将,若论武略,南阳一带,只有长孙冀可以和陆灿相较,襄阳城高水深,易守难攻,南阳却是略为空虚,长孙冀除非是不顾根基,否则必会回师南阳。孰料长孙冀仅遣部将莫业迎敌,两军战于河内,莫业败绩,退守南阳。陆灿遂南下,攻襄阳腹背。莫业率军从后击之,灿于新野设伏,莫业察知,不敢进,陆灿留大将守新野,自率主力南略襄阳。

和陆灿的一帆风顺相比,容渊却是步履艰难,八月十四日,他出竟陵北上,欲取襄阳,不料长孙冀竟然不顾陆灿的威胁,亲率大军守宜城,两军在宜城、竟陵之间缠战十数日,容渊得知陆灿已经迂回袭取襄阳腹背的战报,心中大怒,率军猛攻宜城,长孙冀暗遣军士于黑夜躲在乡野,第二日容渊猛攻宜城之时,伏兵四起,大破楚军,容渊败绩,退守竟陵。长孙冀反攻竟陵,容渊严守六日,

八月二十七日,竟陵危急之时,长孙冀突然退兵远走,容渊探得军情,襄阳竟然已经被陆灿攻陷,容渊得知这个消息之后,怒火攻心,本已在守城之时受了重伤的容渊,竟是吐血不止,卧病不起。

八月二十九日,容渊怒返江陵,并上书南楚朝廷,弹劾大将军陆灿不奉王命,轻易出兵,陷麾下将士及友军于水火,悖逆狂妄,独断专行。

陆灿攻陷襄阳,也是十分意外,襄阳的守备居然十分稀松,不过九日,就被楚军攻下,陆灿询问俘虏,方知八月七日,江南行辕参赞江哲亲来襄阳,和长孙冀密谈之后,暗中分兵三万,不知去向。也因此故,襄阳城才会城防空虚,以至于被陆灿所乘。陆灿心知江哲计谋百出,心中忧虑,便遣侦骑四方探听雍军军情,在他心中江哲一人抵得上雍军十万精兵,分心之下,便没有及时出兵从后攻击长孙冀,驰援容渊,在他想来,容渊守竟陵坚城,纵然不胜也无妨碍,却忘却了容渊心结,数日延误,终于导致无法挽回的憾事。

八月二十六日,陆灿得报,江哲屯兵谷城,思索再三,便留部将守襄阳,亲提兵赴谷城,率兵攻城。谷城虽然城池不大,却是扼守汉水中游的军事要地,又有重兵把守,急切之间也无法一举攻下。

我站在城头,轻摇折扇,看着城下衣甲鲜明的楚军,微笑对站在身后面色沉静的霍琮道:“你在吴越也见过陆灿用兵,可否猜猜谷城能够守到什么时候?”

霍琮微微苦笑,看了一眼站在城楼上指挥守城的将领常谅,心道,幸好先生的说话那人听不到,却只能开口答道:“吴越海战,陆将军和靖海公数次交战,弟子也曾旁观,陆将军用兵如神,靖海公每每叹息,若非东海水军长于海战,难免遭遇败绩,只看这一次他别寻蹊径,出兵义阳,迂回攻襄阳腹背,如此作战当真如天马行空,我大雍虽多有名将,却未必及得,若是没有外力,只怕谷城守不到十日。”

我忍不住低声嘟囔道:“这虽然是实话,不过你也太不给我留面子了,不管怎么说我也是陆灿的师父,难道我就一定会败么?”

霍琮闻言不敢出声,小顺子却是冷笑道:“公子从未指挥作战,能够守到十日还是常将军的功劳,若是有你插手,只怕还要少几日。”他的声音虽然不高,可是在我身后不远处护卫的呼延寿和几个侍卫都听得清清楚楚,都是强忍笑意,不敢出声。

我无奈地摇摇头,小顺子的话我可不敢驳回,望了城下一眼,叹息道:“只可惜他没有十日时间了。陆灿为人光明磊落,又是世家出身,对于人心险恶终究知道的太少。我猜知近期他就会出兵襄阳,他的本心是想趁着赵陇亲政未久,他尚可自行其是的时候夺取襄阳,而为了更有把握一些,他必定会和容渊合兵进攻,所以我令长孙冀厚此薄彼,阻住容渊。容渊对于失去襄阳切齿不忘,陆灿用他做偏师,就是因为他必然戮力死战,陆灿声名在外,按照情理长孙冀应该亲自迎战,这样一来容渊就可趁虚而入,攻取襄阳。这样一来,不仅达到了他的目的,还可弥补和容渊的嫌隙,可谓一举两得。我却偏偏让长孙冀去阻容渊,将收复襄阳功劳让陆灿夺去,在陆灿来说这是不得已,总不能放着襄阳等待容渊来取吧。可是容渊本就器量狭窄,又和陆灿有隙,这一次合力出兵本是为了因为襄阳之仇压过旧恨,一旦襄阳被陆灿所取,容渊心中的怒火足以令他做出不理智的事情,南楚变乱将起,陆灿哪里还有可能安心作战呢?”

霍琮虽然已经心知肚明,仍然一阵心寒,犹豫了一下,问道:“先生既然早有利用将帅不合的内患对付陆大将军,为何隐忍三年不发?”

我低声抱怨了一句道:“我难道不想早些平定南楚么?”然后才答道:“时机未至,纵然隐患爆发出来,也不能伤筋动骨,三年鏖战,以一己之力抵抗雍军数倍之众,陆灿如今已经是南楚的军神,深得军心民心,只有这时候发难才能最大限度的消减南楚军民的斗志,若是动手早了,纵然陆灿一死,南楚军方也不过是陷入四分五裂的境况罢了,却不会放弃抵抗我军,战火将会连绵十余载。而且尚维钧和陆灿顾命之时,若是用了此计,尚维钧纵然有心对付陆灿,陆灿也不会甘心俯首,可是如今就不一样了,赵陇已经亲政,他的旨意是真正的王命,除非陆灿有意谋反,是绝不敢公然违抗的。”

霍琮轻叹道:“陆大将军虽然有捍卫社稷的功劳,可是在尚维钧和南楚国主的心目中恐怕只是一个手握重兵的权臣,唯恐其动摇赵氏王权,若是两国相安无事,武将无用之时,只怕大将军也难逃鸟尽弓藏之祸,只是如今两国战火汹汹,南楚朝廷应该不致于自毁栋梁吧?”

我目光一闪,道:“自然有让南楚君臣安心的法子,目前却无需多言,先提防着别让他取了谷城吧。”

小顺子闻言冷冷道:“公子既知守城之险,为何定要留在谷城面对大军,若论行军作战,陆灿乃是数一数二的名将,公子可是认为他会手下留情么?”

我长叹道:“陆灿若是会手下留情,就不是陆灿了,不过这个险却不能不冒,若不如此,怎让陆灿有口难辩呢?”

小顺子神色稍缓,道:“敌军开始攻城了,公子还是到城中避避吧,刀枪无眼,险地不可久留。”

我听着城下传来的喊杀声,看到城上军士严阵以待的模样,微微一笑道:“我虽不是主将,却是侯爵之尊,如何可以避入城中,小顺子,取来我的古琴,让我在城楼上弹奏一曲,好为三军将士助兴。”

说罢挥袖走上城楼,小顺子叹了口气,终于捧来古琴,我居高临下,望着从容不迫攻城的楚军,以及千军万马中身着锦袍金甲的峻挺身影,数年之间,他的容色苍老了许多,可见心中之苦,说起来我们已经有十三年没有见过面了。轻抚琴弦,若有若无的琴声飘下城楼,琴声宛若流水,流水不绝,宛似别愁,我将眼前战乱,心中阴谋尽皆抛去,只是一心抚琴,也不去想如何用琴声挑起己方军士的士气,如何散去敌军的战意,就好像是在寒园之中,对花弹奏,也像是在江水之上,临风抚琴。

城下指挥攻城的陆灿双眉紧锁,琴声淙淙,溢满天地,丝丝缕缕,皆入耳中,他心头惊异,不问可知,这个时候还有闲情逸致抚琴的,除了先生之外再无别人,只是先生虽然通晓音律,却没有内力,如何能让这琴声凝而不散,溢满苍穹。

只是他也没有心情顾及此事,令军中士卒敲响催战鼓,鼓声隆隆,响彻天地,想要掩去琴声,可是那琴声便如清风过隙,流水浸沙,虽是若隐若现,却始终不曾断绝,声声入耳,陆灿心中生出颓意,只觉得仿佛眼前这片天空尽在那弹琴之人的网罗之下。

这时候汉水之畔,两个身影默然立在那里,远观那如火如荼的战事,其中一个男子,白衣如雪,剑眉星目,风姿飘逸,负手而立,神情淡漠,另一人则是一个黑衣青年,英姿飒爽,神色冰寒,他手中捧着琴囊,目光炯炯,望着血花飞溅的战场,周身上下洋溢着浓厚的战意杀机。

那雪衣青年听着琴声,沉吟良久,才道:“若论弹奏技巧,随云远在我之下,可是他的悟性却是这般出众,不需倚靠外力,便可以深入心魂,纵是雷霆铁壁,也难以阻绝遮掩,我也是两年前才达到这般境界,想不到他竟也能够弹出这样的琴音。凌端,拿琴来,我要和随云一曲。”

凌端一撇嘴,虽然如今魔宗也已经是大雍臣民,但是对于凌端来说,那个江哲仍然是最可恨的仇人,并非是因为那人设下的计策,让自己最尊敬的谭将军战死沙场,马革裹尸,本就是谭忌夙愿,也不是因为那人利用自己害死了石将军,虽然知道石英之死乃是大雍阴谋,但是对于石英的恶劣印象并没有消退,对他来说,始终念念不忘的便是李虎,那个鲁莽的笨蛋,却因为那样可恨的缘故被江哲杀了,自己这些小人物的性命在江哲心中,大概就连蝼蚁都不如吧?这些年来,他随着四公子见过江哲数次,却是一句话也不愿和他多说,甚至刻意避开那人,只怕自己忍不住质问那人关于李虎的事情。

虽然心中恼恨,却不敢违背秋玉飞之命,恭恭敬敬递上“洗尘”古琴,秋玉飞盘膝坐下,将古琴放在膝上,轻抚琴弦,一缕孤绝的琴声从指下溢出。琴声宛似奇峰凌云,清绝激昂,却又和谷城之上传来的琴声拍拍相合,两缕琴音一若行云流水,一如嶙峋孤峰,流水绕奇峰,其中有清商,虽然分明听出两缕琴音的不同,却又觉得流水孤峰山水相互辉映,交融一处。

此时此刻,不论是城上的雍军,还是城下的楚军,都仿佛失魂落魄一般,沉醉在琴音之中,战场之上的杀伐之声渐渐消散,戾气也化为祥和,陆灿在楚军阵中不由摇头长叹,今日楚军再无战意,一曲古琴,散去七万楚军斗志,这等事情当真让他有苦难言,黯然下令鸣金收兵,免得己方被城中雍军所乘。

楚军听得鸣金,都是满脸的不舍,却不敢有违军令,渐渐退去,军中部将正欲簇拥陆灿离去,陆灿一咬钢牙,挥手令亲卫递上自己的神弓,纵马出阵,会挽雕弓如满月,一箭向谷城城楼射去,他所站的位置距离城楼足有五百步之遥,那一箭却是见光不见影,瞬间穿越漫长的距离,射向城楼上抚琴的江哲咽喉。城上雍军看到陆灿张弓射箭,开口欲呼,那一箭却是已经到了江哲面前丈许之处,只是那箭矢却也没有更进一步的机会,一只宛似冰雪美玉调成的素手挡在箭矢之前,手指轻弹,那一支势如雷霆逸电的鹰翎箭已经被弹落在地,小顺子面如严霜,眼中露出无穷的杀机。

陆灿本是双臂神力,上阵杀敌之时,常以弓箭射杀敌将,虽然不如大雍长孙冀等人的神射,但是五百步之内也是箭无虚发,只是后来他身为大将军,鲜有亲自上阵的机会,又因为他颇通经史,有儒将之誉,所以勇武之名反而渐渐被人淡忘。不过陆灿这一箭却非是想要泄愤,或是要取江哲性命,他自然知道江哲身边有人可以拦下此箭,这一箭不过是表示师徒绝决之意罢了,所以一箭射出,他就连结果也不看一眼,便策马奔入军中,被亲卫簇拥着远去了,不论是城下楚军还是城上雍军,凡是看到这一箭的,都是黯然,师徒反目,故人长绝,本就是人生憾事。

城楼之上,江哲却是微阖双目,只顾抚琴,似乎根本没有留意到方才险些被箭矢射杀。琴声一变,便如海浪退潮一般,重重叠浪,正迎合着楚军退兵之势,而那从汉水之畔传来的琴声也是随之一变,便如海浪之中千年屹立的巨礁,纵然狂风海浪消磨,依旧傲立狂澜之中,亘古不变,青山绿水化作碧海礁崖,却是一般的丝丝入扣,亲密无间。

当楚军的背影消失在视线之内的时候,两缕琴声似有默契一般地嘎然而止,我推琴而起,淡淡道:“玉飞若是来了,琮儿请他到县衙见我。”

霍琮闻声不由道:“先生,陆将军那一箭并非是真的要杀先生。”

我眼中闪过一丝怅然,道:“他就是真心想要射杀于我,也没有什么不对。”说罢,我转身向城内走去。

霍琮望着江哲的背影,眼中透出淡淡的苦涩意味。

过了些许时候,秋玉飞带着凌端已经到了谷城之下,只是敌军不知何时来攻,城门却是不能轻开,城上放下绳索竹篮接两人入城,秋玉飞和凌端都是熟知战事的人,自不会以为是轻辱,秋玉飞便让凌端坐在竹篮中,不多时上了城头,那些军士正要再放下竹篮,却见眼前白影一闪,一个雪衣青年已经站在他们面前。那些军士目瞪口呆,古城城墙虽然不甚高,也是高约十余丈,竹篮只能承载一人,这雪衣青年却是不需借力,便这样轻轻巧巧的上了城楼,不由庆幸这人非是敌人。

霍琮却是丝毫不曾惊慌,他自己虽然只是略略学了些寻常武功,却是曾经见识过小顺子的本事,秋玉飞的身份他十分清楚,魔宗嫡传弟子有这样的武功也不奇怪,上前一揖道:“霍琮拜见四公子,先生在县衙等候四公子。”

凌端闻言冷笑道:“江先生真是客气,还记得遣人相迎,当真看得起故人。”

霍琮能够察觉出凌端话语中的敌意,他也略知凌端之事,微笑道:“凌兄言重,我家先生与四公子琴音相酬,知己于心,四公子乃是世外之人,素有林下之风,先生不曾亲迎,一来是因为尚有公务待理,二来也是不愿用这些世俗礼数来辱没四公子。”

凌端想要出言争辩,连张了几次嘴,却都想不出该说些什么,只得哑口无言,忿忿不平地站到了一边。

秋玉飞原本含笑看凌端和霍琮说话,琴音相和,彼此心照,他自然不会误解江哲轻视于他,凌端借题发挥,他却也不阻止,只是想看看霍琮如何应对,这少年他虽然不认得的,但是魔宗消息灵通,江哲身边最心爱的弟子是谁,他怎会不知,只看霍琮相貌气度,便已知道他的身份。

虽然知道江哲弟子必是才俊,但是霍琮轻描淡写的几句言语就令凌端铩羽,却也令他动容,仔细瞧去,这少年虽然相貌寻常,但是气度神采却有五分颇似江哲,只是少了几分懒散狂放,多了些凝重端厚,只是多看了几眼,秋玉飞又是眉头一皱,这叫霍琮的少年的面上竟有心气郁结之相,显然心事重重,江哲精通医术,怎会看不出来,又怎会让自己的弟子苦恨如此。但是他只是暗暗记在心中,笑道:“好了,凌端不要乱说话了,霍琮带路吧,随云想必还在等我呢。”

霍琮引着两人走向县衙,县衙这时已经是楚国侯江哲的官邸,戒备森严,四周守卫的皆是身着黑衣黑甲的虎贲卫,三人刚走入县衙之门,凌端目光闪动,打量着周围地势,这却是他的习惯,谁知目光一闪,却看到了一个黑衣大汉立在阶下,凌端霎时间目瞪口呆,几步奔到那大汉身前,结结巴巴地道:“李虎,你怎么还活着?你怎么成了虎贲卫?”

那大汉神色迷糊地摸了摸脑袋,道:“凌小子,是你啊,怎么你不知道我还活着么?”

凌端气得大骂道:“我怎么知道你还活着,当初你被庄大人带走,不是说已经被灭口了么,怎么现在你还活得好好的,既然活着,这么多年怎么不知道给我传个消息,难道患难之情你就一点没有放在心上。”骂到后来,凌端已经是怒火丛生,方才见到故人的狂喜也消退了几分。

李虎眼中闪过迷惑,道:“什么灭口啊,当初我和那些兄弟都被押到了别处,做了一年多苦役就被放出去了,兄弟们多半都领了银钱回乡了,我也没有地方可去,正不知道怎么营生才好,谁知道呼延统领来问我要不要去长安,我想着石将军也没了,就跟着统领进京了,先是在虎翼营中待了几年,呼延大人经常来指点我武艺,四年皇上亲临营中大比,选拔虎贲卫,我本来差了些落选,但是皇上听说我就是一槊把江侯打下水的李虎,就把我选入虎贲卫了。三年前又被派来保护江侯。不过,我听说你跟着秋四公子去了东海静海山庄,托人给你写过信,你没有收到么?”

凌端看着李虎迷茫的神情,知道这傻大个心中懵懂,对当日之事糊里糊涂,这些年来竟是只有自己时刻忍受着仇恨折磨,举目四顾,秋玉飞和霍琮早已不见身影,就是旁边的虎贲卫也都避开了,多年的恨意猛然落到了空处,他心中又是欢喜又是茫然,喃喃问道:“你托什么人送的信啊?”

李虎搔首道:“我不知道静海山庄在什么地方,就请呼延统领帮忙,转托侯爷给你传个消息,心想你什么时候来长安,可以来找我喝酒。”

凌端哭笑不得,这下他可知道问题出在什么地方了,但是想到故友竟然健在,心中的欢喜混合着说不清道不明的情绪,让他忍不住泪下如雨。李虎看着昔日患难好友这般模样,只急得手足无措,在凌端身边直转圈子。

秋玉飞在小顺子引领之下走入内堂,只见江哲负手立在堂前,背影有几分萧瑟。秋玉飞叹道:“莫非随云在记恨那一箭么?”

我也没有回头,道:“两国交战,岂有恩义可言,更何况我不过是叛国负恩之人,他如此相待已经是仁至义尽了,当初我在陆府为西席,心怀丧父之痛,虽然是因为他不爱读书,所以立下各行其是的约定,可是实际上也是因为当时跟本没有心情教他读书,若不是他赤心相待,我也不能那么快就振作起来。而且我虽然腹中颇有才学,但是毕竟年轻识浅,教他读书之时多有疏漏,若不是他和我针锋相对,辩论探讨,我也没有今日的成就。陆府五年,我是举目无亲,他虽是侯府世子,陆侯练兵,常年不在府中,他又是幼年丧母,诺大的陆府,不过是我们两人相依为命,与其说是师生,倒不如说是朋友手足。虽然他少年性情,常常与我玩笑胡闹,可是却是真心将我当成亲人,我爱读孤本奇书,他便替我搜求,我贪看江上雪景受了风寒,他亲自侍奉汤药,当初我有意离开南楚之前,便是最放心不下这个亲如手足的弟子。可是如今却偏要亲自设计让他落入陷阱,别说他射我一箭以示恩断义绝,就是他真的要杀我,我也无法怪他,若非是陛下待我恩重如山,我纵然眼看战火再连绵三十载,也不会插手此战。”

秋玉飞觉出江哲语气苍凉,便故意调侃道:“随云或许不恨陆灿绝情,只是若说不怪他我可不信,凌端不过是当年挟持人质救了我师兄一次,你便故意瞒了他十年,让他终日怀恨不休,思念亡友,若非这次你有求于我,怕他从中作梗,恐怕还不会让他知晓真相吧。”

我闻言不由一笑,回头道:“江某记仇量窄你也不是今日才知的了,何必取笑我呢?”

秋玉飞见江哲露出欢颜,心中一宽,举目望去,数年不见,只觉得江哲两鬓星霜更多了几分,灰发也浅了几分,不由叹道:“听说随云这几年浪迹山水之间,对于军务都不甚留心,我还以为随云必定神采奕奕,怎么如今看来却是憔悴了许多?”

我轻轻一叹,道:“岁月匆匆,容颜渐老,这也是无奈之事,倒是玉飞风采如昔,令哲既羡又妒。这次哲千里传书相请,实在是有一件大事相托,想来想去,就只有玉飞能够助我一臂之力,只是此事颇有为难处,若是魔宗不许,或者玉飞不便,哲也不敢强求。”

秋玉飞心中一动,已经猜到江哲所托之事,坦然道:“随云既有请托,玉飞敢不从命,我魔宗如今已经是大雍之臣,此来更是先去拜见过师尊,师尊已经许我便宜行事,若是事情紧要,我即日便可南下,只是你这一番苦心,只怕也是无济于事。”

我欣然道:“不论成败,总要尽我心意,多谢玉飞慨然相助,只是如今还有些时间,你我何妨相聚数日,等到南楚兵退再说。”

秋玉飞叹道:“这倒也是。”继而又笑道:“随云琴艺大有进境,我正要请教呢。”

我笑道:“正合我意,小顺子,这几日我就不到城上去了,就让琮儿跟着常将军去迎战吧。”小顺子闻言转身出去传令。

秋玉飞目光一闪,道:“随云对那一箭断绝师徒情谊的陆灿尚有顾念之情,这霍琮也是你的弟子,为何你却对他不甚顾惜,否则他怎会郁结于心呢?这样的人才,你若不喜爱,不如将他送了给我吧。”

我意味深长地道:“天将降大任于斯人也,必先苦其心志,劳其筋骨,饿其体肤,空乏其身,行拂乱其所为,所以动心忍性,曾益其所不能。”

秋玉飞闻言轻叹,再不多言,两人相视一笑,并肩走入后堂。

接下来整整十日,两人只在后面抚琴论曲,将外面的战火视若未见。任由霍琮跟着常将军抵挡陆灿的强攻。

八月二十七日,长孙冀回师襄阳,攻城甚急。或有部将劝陆灿先返襄阳歼灭长孙冀,陆灿思忖再三,只令部将死守襄阳,不容长孙冀援救谷城,然后便是下令猛攻谷城,因为攻打襄阳之时,投石车和床弩都已经用完,二十六日江哲和秋玉飞双琴合璧,散去楚军战意,陆灿退军之后便令军士赶造投石车,二十七日开始,日夜攻城不停,他虽然从未在襄阳领兵,但是当年却曾令人将襄阳周边城镇强弱虚实都打听得清清楚楚,谷城距离襄阳不到一百五十里,快马一日可到,所以他对谷城城墙的弱点一清二楚,投石车发出的巨石全部冲着那些薄弱之处招呼,不到一日夜,谷城城墙已经残破不堪。霍琮向江哲求教,却被拒之门外,无奈之下,他心一横便自作主张,令军士造了几架小型的投石车搬上城头,用烘干的枯草捆成草球,里面放入引火之物,点燃之后投掷到敌阵上,烧毁了十余架投石车之后,楚军的攻势便难以为继了。

八月三十日,陆灿得知容渊退兵的消息,又通过数日攻城,发觉谷城之内绝对不到三万人,最多只有五千人,判断其余雍军必然暗中调往他处,说不定已经回师襄阳,若是襄阳失守,自己的后路便会断绝,但是陆灿也知道,如今自己孤军在雍境,纵然退守襄阳,也是内外交困,所以他便继续攻谷城,存心要以谷城诱使雍军来援,又派多人潜回南楚,用大将军令调动江夏留守的水营增援。

此时,到了谷城之后,便被江哲下令,经由老河口转道邓州的雍军疾驰回襄阳,会合长孙冀断绝襄阳道路,按照江哲事先谕令,只顾攻打襄阳城,却不去救谷城。

九月二日,长孙冀得知陆灿掘水灌城,被守军在城内挖掘城壕,令河水汇入地下,担忧谷城不能守住,派遣一万军士援救谷城,距离古城三十里之时,斥候回报,谷城浓烟滚滚,援军将领误以为谷城失守,奋不顾身快马加鞭前去救援,被陆灿部将途中伏击,万余军士死伤叠籍。长孙冀闻报令人猛攻襄阳,襄阳楚军只有万余军士留守,雍军弃城之前已经将城中粮草辎重带走大半,守城本来极难,但是虽然雍军三年来善待襄阳父老,襄阳人仍是不忘故国,闻知是大将军陆灿取襄阳,皆不顾生死,舍家拼命,相助楚军守城,雍军急切难以攻下。

九月四日,陆灿令军士挖掘沟渠,引走谷城城下的积水,这时候城墙在大水内外浸泡之下,已经根基浮动,陆灿令军士掘地道入城,被霍琮以城内积水灌入地道,破去楚军攻势。

九月五日,陆灿命军士以柴火架在地道外面烧城,日以继夜,通宵达旦,这次不像九月二日那般堆火生烟,诱骗援军,而是欲毁城墙,霍琮令军士修补城墙,苦不堪言,但是到了九月六日早晨,在城外响了一日夜的战鼓声中,谷城南面城墙崩塌,就在霍琮计穷之时,却发觉城外楚军并未趁势进攻,令斥候出城查探,楚军军营之内只有二十余只山羊被蒙了眼睛倒吊起来,前蹄击鼓不休,楚军竟是已经趁夜走了。

九月六日凌晨,陆灿率军突然出现在襄阳城外,昨夜斥候回报,陆灿仍在攻谷城,长孙冀未料陆灿回师,因为襄阳守军无力出城作战,因此并未提防,更何况其时已经是黎明,正是楚军沉睡未醒之时,陆灿率军马踏雍营,长孙冀仓卒之间遭遇大败,整军不及,幸而雍军精锐,大半逃生。陆灿重入襄阳,破去雍军重围。再度遣使往江陵、江夏调派援军。

在陆灿在谷城、襄阳挥军苦战之时,建业却已经一片混乱,九月一日,容渊的弹劾表章到了建业,尚维钧方知陆灿出兵之事,震怒之下召集心腹议事,如今国主亲政,虽然朝政仍在尚维钧掌握之中,但是毕竟名义上多了一个国主,而且尚维钧虽然贪权,却没有谋反之意,对自己的亲外孙更是只有维护逢迎之心。而陆灿,手中兵权越来越强,在隆盛八年,更是借着御敌之名,分去江淮荆襄四品以下官员的黜陟之权,尚维钧早已是对其戒惧不安。在尚维钧来说,有几十万大军守江淮,又有长江天险,十余年来重新经营的江南防线固若金汤,纵然没有了陆灿,只要放弃一些战事频繁的无用城池,稳守重镇,即使雍军大举南征,也不可能再渡长江。反而是陆灿,拥兵自重,在国中又是深得军民之心,一旦他起了反意,便是灭顶之灾。本来在赵陇亲政之后,尚维钧就有意借着国主名义,缓缓收回陆灿军权,想不到陆灿依然故我,又像从前一样不告而战,尚维钧心中下了决心,若是陆灿取下襄阳,大败雍军,也要将其招回建业,以封赏之名留住他。商议了一夜,设下如何诱骗陆灿回转建业的计策之后,尚维钧便令司徒蔡楷为钦使,至江夏迎候陆灿,一旦陆灿得胜之后,便招陆灿回京受封赏。蔡楷乃是新王后之父,堂堂的国丈,又是朝中重臣,声名赫赫,素以名儒闻世,蔡后得力,陆灿也有功劳,蔡楷前去相召,必然不会让陆灿生出疑心。

谁知不过数日,传来楚军被困襄阳,陆灿却猛攻谷城以及江哲正在城中的消息,更有陆灿召集援军的命令,尚维钧虽然担心陆灿战败,损伤南楚元气,却也欣慰陆灿能够大义灭亲,甚至亲自传书令容渊救援襄阳。容渊以重病不能领军推辞,再度上书,称陆灿拥兵自重,无视朝廷,为己身功业,不惜将士性命。

九月六日起,江南流言四起,皆说陆灿孤军守襄阳,不退也不进,是因为陆灿有意割据江淮称王,又指陆灿不破谷城,是因为不愿得罪大雍皇室,因为一旦陆灿自立,则江淮两面受敌,所以暗中向楚国侯江哲屈膝,表示和解之意,破长孙冀,取襄阳,不过是掩人耳目,否则为何雍军迟迟不再攻打襄阳呢?

九月十二日,仪凰堂首座纪霞向尚维钧呈上得自民间的一首短歌,“鹫翎金仆姑,燕尾绣蝥弧。陆王扬新令,千营共一呼。”(注1)

尚维钧一见便觉心如寒冰,诗中所指陆王,除了陆灿还能是何人,以军功扬威,一呼百诺,一令既下,千营一呼,除了陆灿还有何人,细察诗中之意,陆灿竟有称王之意。他犹自担心纪霞有心构陷,又令亲信暗访,却发觉数日之间,无论是江淮、荆襄,还是吴越,长江南北尽是歌声,就是三岁小儿,也在呀呀学语,唱着“陆王扬新令,千营共一呼”。尚维钧也是通晓经史之人,自然知道什么是谶谣,如果不是陆灿有意谋反,怎会传出这样的反诗,若非是陆灿这样的地位权势,如何能令一首歌谣数日之间传遍江水。

疑念既起,尚维钧心中忧急万分。恰在这时,尚维钧之子尚承业进言道:“陆灿拥甲兵,据重镇,往往不请命而自出兵,虽然功高,却非是纯臣,姑且不论他有反意的消息是真是假,朝野民心,知有陆灿,不知有国主,更不用说父亲了。若是陆灿振臂一呼,恐怕江南立刻便会易帜,到时候,不止王上身亡国灭,我们尚氏也会烟消云散。若是襄阳之战,陆灿大胜而归,朝廷必然要重重封赏,据闻军中已有怨言,万不能像前几次那样敷衍过去,可是此人已经位极人臣,身为南楚大将军,总督江南军事,爵封一等公爵,若是再要加封,就只能封王了。异姓为王,这是谋反的前兆,纵然陆灿现在没有反意,天长日久,也难免不会被部将胁裹称王。为父亲计,与其坐以待毙,不如先下手为强,除去陆灿和其心腹之后,再安抚他手下的将士,这些将士的亲眷都在江右,而且群龙无首,如何反叛,到时候从军中选一二和陆灿有嫌隙的宿将,让他们安分守己的防守雍军即可,父亲想必也没有中原之望,何何必定要倚重那陆灿呢?”

尚维钧虽然心许,但是依然犹豫不决,正在这时,前方军报再度传来,陆灿放弃唾手可得的谷城,回师襄阳,大败长孙冀,回书求援。尚维钧听到这样的消息,却是精神一振,若是陆灿在襄阳大胜,自己可能便无法挟制陆灿,如今陆灿急待援军,自己便可趁机迫使陆灿回军,没有襄阳,最多是失去了夺取中原的可能,可是陆灿若是谋反,却是破家亡国的大事,所以他立刻进宫,请赵陇下旨,令蔡楷为监军使,以王命阻止江夏大营出兵,更调动容渊至江夏,声称等待王命,合兵北上襄阳,却暗中让容渊封住江水,不许江夏军北上。

赵陇虽然已经亲政,但是却沉迷酒色之中,对于国事漠不关心,对于外公主张毫无反对之意,便下了旨意送往襄阳,命陆灿退兵,在他看来,孤军北上,谋夺中原,实在是一件没有必要的事情,据有半壁江山,放眼望去,宝殿生辉,室中尽是奇珍异宝,触手之处,满是冰肌玉骨,水晶帘下,脂香粉腻,这般福分,终老江南足矣,何必以卵击石,多生事端。

九月十八日,圣旨到了襄阳,陆灿拒不接旨,以“将在外君命有所不受”为辞拒绝退兵。

陆灿抗旨之事传到建业,赵陇大怒,他冲龄继位,虽然从未掌权,但是也无人违逆过他的命令,陆灿对他来说不过是个平常臣子罢了,竟然违背王命,一怒之下,再度颁旨召还陆灿,贵妃纪灵湘故意微辞讥讽,说是陆灿不会遵从旨意,赵陇担心在爱妃面前失了面子,两日内接连下了七道退兵诏书。

九月二十五日,第二道诏书到了襄阳,陆灿愤而不受,可是建业依次来了七名钦使,皆是宣旨令陆灿退兵。纵然如此,陆灿本也不愿放弃襄阳,可是陆灿虽然决意进取,江夏援军却为容渊所阻,江淮军马又无法调动,粮将尽,孤立无援,雍军却是大军重整,眼看即日就要进攻襄阳,且将襄阳周边坚壁清野,不容楚军因粮于敌。陆灿立在襄阳城头,临风而泣道:“大业未成,而中道南渡,从今后再无中原之望。”

不得已之下,陆灿下令退兵,襄阳人得知楚军将要退兵的消息,都是大为惊慌,拥在陆灿帅府之前,皆道:“我等助大将军守城,一旦雍军夺回襄阳,岂不是要问罪众人,大雍法令森严,我等唯死而已,求大将军救命。”

陆灿闻言叹道:“陆某不能北望中原,却也不能害了襄阳父老。”然后便下令先让襄阳民众南迁,过随州,到江夏安居。

陆灿亲自提兵断后,守襄阳不退,长孙冀得知襄阳民众南迁的消息,惊怒之下,挥军攻城,陆灿严守七日,襄阳城头染血,雍军难以攻入,十月三日,陆灿纵火焚烧襄阳,然后趁乱从襄阳城西门突围,向随州而去。

在陆灿离开襄阳城十余里之后,却听到耳边传来如同雷霆轰鸣一般的声响,连绵不绝,仿佛雷神发怒,陆灿心中一动,面色却变得苍白如纸,只听声音的位置,便知道是从城墙的位置传来,定是城墙之下掘出坑道,中藏火药,此番被大火点燃,才发出这般响声,陆灿心思灵透,立刻猜知这定是雍军安排破城的暗着,这样的法子,绝非守城将领可以想到。而雍军攻城这些时候,却不用这暗着破城,陆灿便知自己定是已经陷入了圈套,纵然自己生出襄阳,也难免受国主猜忌,想来那火药不过是某人为了防范于未然而设下的后手罢了。苦涩的一笑,陆灿策马向随州而去,月余苦战,烽火襄阳,留下的尽是士卒鲜血,将军遗恨。

——————————————

注1:卢纶《塞下曲四首之一》改

\chapter{第三十六章 长城空自许}

同泰十二年初,雍军掠吴越,公奉上命督军余杭,练义军护海防,人皆以公不能兼顾江淮战事,公乃暗命参军杨秀袭泗州、楚州、淮西将军石观进军宿州,雍军未料公先启衅,失宿州、楚州,淮北危殆,赖大雍淮南节度使裴云死命拒之。

三月,襄阳将军容渊闻战事,怪公轻己,不以告,乃自领军取南阳,中雍军诱敌之计,反失襄阳,风林关遇伏,连战连败,退守宜城。公欲加罪,尚相阻之,容渊遂附权相,恨公欲行军法罪己,深恨之。

雍楚大战月余,于江淮两军互有胜负,吴越则僵持胶结,然失襄阳,乃失荆襄屏障,战未平,葭萌关为内奸所卖,朝廷欲问罪余缅。公曲护余缅,上书自请罪,谢以主军不利,尚相温言慰之,不敢加罪余某,然心疑公左坦心腹,益忌之。

四月,大雍齐王督江南,公与之战,自蜀中至吴越,战火皆汹汹,公请朝廷曰:“战事无常,进退不定,诸府县皆需军管。”尚相不得已从之,江淮、荆襄四品以下官员,许公得自黜陟,虽暗怒而不言。

十月,大雍求和议,欲得随州、竟陵,许以息兵,尚相阴许之,公闻,当廷斥之曰:“若失竟陵、随州,则江陵、江夏不保,武帝辛苦取之,岂可轻易弃于虎狼。”和议乃止,尚相惭愧,然忌意愈深。

同泰十三年,公连挫雍军,竟陵、随州皆安,然汉中节度使秦勇自米仓道取巴郡,公急令部将扼夔州。

八月,雍军遣使,欲以困剑阁、成都楚军及巴郡,交换成都、剑阁等地,公许之,仍命余缅守巴郡,尚相以余缅丧师辱国,欲斩之,公力辩不可,尚相遂止,此时已生杀公之念,因公战功卓著,不敢轻动。

——《南朝楚史·忠武公传》

南楚同泰十四年九月十七日,安陆城,夜色昏昏,街道上满是神色肃然的军士,悄无声息地往来巡视,城中军民都是悄然吞声,只因今天午后,从襄阳退兵的楚军到了安陆。安陆乃是隶属于江夏的大县,楚军若是北上襄阳,必要经过此地,陆氏多年经营,这里的人心皆属陆氏。陆灿对于安陆人来说,并不仅仅是南楚大将军而已。以往陆灿经过安陆,都会驻留一日,与城中父老把酒言欢,可是这一次却是有些不同。入城之后,陆灿便径到别业休养,过了些时候,安陆父老才得知陆灿竟然卧病不起。安陆军民闻知,都是心中焦虑,几乎家家焚香祝祷,泣告上苍,翼望莫要夺去南楚栋梁。

陆氏在安陆的别业,不过是座宽敞的宅院,虽然气度森严,格局广阔,既没有清幽的景致,也没有奢华的陈设,除了有几个仆人负责照看之外,再无下人。现在这座别院内外已被陆灿亲卫围得水泄不通,绝不容任何人打扰,在这些将士心目中,害得将军重病的朝廷钦使便是最不可放行的人物。

在内室之中,陆灿身穿宽袍,负手站在窗前,望着天上明月,俊朗的容貌上露出淡淡的倦容,看上去全无重病的模样。夜色渐深,更鼓声声,从远方的黑暗中传来,一声声摧折人心。这时,一个亲卫进来禀道:“大将军,韦先生在外求见。”

陆灿眼中闪过一丝寒光,道:“请韦先生进来。”

那亲卫犹豫了一下道:“将军,是否增派一些人手,韦先生的武功……”

陆灿淡淡道:“不必。”

那亲卫不敢多说,连忙退了出去,过了片刻,引进一个雍容男子。陆灿转过身去看着他道:“韦先生,我想你这两日也该到了。”

韦膺一看到陆灿,便觉心中一惊,不过是数月未见,陆灿两鬓星霜多了数分,虽然从容冷静的气度没有什么改变,身上却明显多了几分倦怠。不过这已经在韦膺意料之中,他神色肃然,上前一揖道:“韦某拜见大将军,大将军一路辛苦了,不知道大将军对于将要发生的事情,可有什么安排?”

陆灿微微一笑,道:“韦先生是以凤仪门辰堂首座的身份来见我,还是以陆某幕中客卿的身份前来的呢?”

韦膺目光一闪,道:“自然是大将军客卿的身份前来,在下没有能够阻止种种不利于大将军的事情发生,还请大将军恕罪。”

陆灿摇头道:“你不是不能阻止,而是根本没有想过阻止。”

韦膺低头道:“大将军何出此言,在下实在没有料到容将军会上书弹劾,更没有料到流言四起,更有那些不知厉害的妇人女子从中作梗,以至于大将军被迫退兵,但是韦某一人之力,实在不能和尚相、仪凰堂、凤舞堂相提并论,所以才会束手无策,令大将军处于此种境地。”

陆灿淡淡道:“今年王上亲政,立后之时,你曾劝我设法令梅儿为后,被我拒绝,后来太后想要梅儿进宫为妃,消息还没有外泄,风儿便已经知道了,我留在京中的些许力量,不过是探听一些朝廷动向,并不能深入内宫,得悉这样的隐秘,风儿也只是名义上的首领,不过是因为这种事情需要一个陆家人来负责罢了,可是风儿却提前得到了这个消息,又瞒着他娘亲,唆使梅儿出走,一路上却是你暗中派了高手沿途护卫,按照你的性子,若是梅儿入宫为妃,既可以弥和陆氏和王室的嫌隙,也可以和纪贵妃相抗,对你有诸般利益,可是你却暗中相助风儿,这却是什么缘故?”

韦膺抬起头来,神色坦然道:“将军为南楚重臣,梅小姐也是德容兼备,若是太后和国主有意立小姐为后,这是理所当然之事,纵然是将军也不能直接拒绝,只不过将军不愿以小姐终身幸福,换取荣华富贵,这也是父女情深,无可厚非,之后太后更是想要屈小姐为妃,若是大将军真的答应,岂不是贻笑天下,所以在下没有请命便协助二公子将小姐送到寿春,不过将军也是看轻了二公子,我虽令人将消息泄漏出去,但是二公子却是从别的途径知道这件事情的,在下也想不到二公子有这般胆量,竟然立刻骗了小姐北上投奔大公子,至于沿途护送,那也是分内之事。”

陆灿扬眉道:“陆某岂羡椒房之宠,梅儿生性柔顺,我怎忍让她到那不见天日的地方和人相争,否则我若有心,就是想要梅儿立为王后也非是不可能。可是自古以来,女为中宫,父为权臣,鲜有善终的,所以我不愿和王室联姻,就是云儿,我也不愿他尚主,淑宁公主虽然不错,可是我更喜欢可以和云儿并肩作战的玉锦为儿媳,更何况这也是云儿的意思,我陆氏从无谄媚事主之辈。这件事你虽然有些私心,我也要谢谢你,若是一旦太后将立妃之意挑明了,若再拒绝,就不免正面冲突,那非是我的意愿。不过容渊之事,你却出我意料,若是按照你从前的习惯,就是我不同意,容渊第二封弹劾的奏章也是绝对递不上去。”

韦膺面色一沉道:“大将军若是这样看待在下,在下也无话可说,不错,我的确可以设计刺杀容渊,或者中途劫走奏章,可是这却要和凤舞堂作对,这一次凤舞堂首座燕无双亲自出马,保护容渊的安全,第二封奏折更是仪凰堂谢晓彤亲自送到建业的,韦某岂能出手,莫非大将军以为韦某和她们作对是理所当然之事么?”

陆灿淡淡一笑,道:“若非是知道韦先生两年前便和她们决裂,我也不会信任将军先生如此,也不会轻易落到今日的地步,莫非先生要说是我陆灿轻信了你么?”

韦膺闻言心中一震,他万万料不到两年前自己和纪霞、燕无双在凌羽面前的那场争执竟然也被陆灿知晓,心神一阵恍惚,陆灿那一句淡淡的话语,对他来说如同天上惊雷,自从离开大雍之后,内心深处他将自己早已看轻了自己,甚至常有自暴自弃之心,若非尚有仇敌活在世上,很可能他早已不能这般苟延残喘下去,可是陆灿却待他一如常人,好像他不曾叛国谋逆,也不曾连累亲族,这些年来更是对他信任重用,不知不觉间陆灿在他心目中已经重于一切,他有些慌乱地道:“大将军请听说解释,实在是,实在是……”却觉得无话可说,原本想好的推诿之言却是再也说不出口。

陆灿也不看他,转身看向窗外,冷冷道:“我退守襄阳之时,江南流言四起,这几年你的辰堂得我支持,势力大增,难道就一点法子都没有么,杨秀不便公然出面,你为什么毫不动作?”

韦膺勉强道:“大将军也应知道大雍素来在我南楚境内多有秘谍,而且江南多有畏惧雍军的软弱之人,若非如此,大将军也不会屡次出兵都不肯事先告知建业,若非投鼠忌器,只怕大将军先就会在建业血洗一番,而且这次司闻曹的主事必是换了人,手段比起从前越发隐秘狠辣,那首短歌更是辞意皆美,寻常百姓只道是赞誉将军,全无介意,我纵然全力搜捕,只怕也难以将大雍密谍一网打尽,反而会暴露了辰堂的实力。何况大将军遭朝廷猜忌已非一日,纵然平息流言,也免不了今日之事,与其做些无用之功,不如以谋后图。”

陆灿闻言轻轻一叹,道:“韦先生可是想要劝说陆灿起兵反叛么?”

陆灿出兵襄阳之后,因着容渊一封奏章引发的诸多事件虽然也令韦膺有些为难,可是若是他真心出力,至少也不会到了这般境地,只是他心中也有私心,所以才隐忍不肯轻动,如今被陆灿挑明,他露出尴尬神色,却知再不能虚言搪塞,上前拜倒道:“大将军恕罪,非是韦某不改旧日之恶,只是韦某流离江南至今已有十二年,想起前尘往事,一点恨意终究不能消去,只是韦某也知道与仇人已有天渊之别,他是大雍驸马,如今已经是国侯爵位,更得李贽信重,身边又有邪影保护,不论是文武手段,我都无奈他何,唯一的报复手段就是在战场堂堂正正的厮杀,若是能够挥军攻入雍都,毁去他安身立命的一切,才是真得报仇雪恨。

只是大雍如日中天,北汉已降,李康也已经一败涂地,病死在雍都,南楚又是这般情况,昏君权相只知苟安,凤仪门上上下下,多半都已经忘却了昔日仇恨,只想在江南苟延残喘,根本不敢提起报仇二字。我本已心灰意冷,可是大将军却令我看到了希望,初时我只是希望阻止雍军南下,只要不令大雍一统天下,这已经可以令大雍君臣遗恨无穷。后来膺得知将军也有中原之志,便决定一心效忠大将军,韦某并非是欲求荣华富贵,只要有朝一日,大将军能够马踏中原,我的仇恨便也报了,纵然大将军念师徒之情,曲护那人,韦某也无怨恨之意。

可是大将军纵然军略无双,却是无心政争,已将军手中之权,纵然除去尚维钧,一掌朝廷大权,也是轻而易举之事,可是大将军却甘心受那权相压制,韦某也知历代史实,自古以来没有内有权臣,而大将可立功于外者,若想席卷中原,便需清君侧,涤清朝纲,攘外必先安内。可是韦某也知大将军忠义,从无权位之念,所以这一次我便没有暗中阻止局势的恶化,只希望大将军被迫起兵,将朝中奸臣一扫而空,待到朝中平定,大将军统军北伐,再无窒碍,可立万世功业。

若是大将军担心清流抨击,韦某可以保证那些人没有法子惹事,如今朝中早已是奸佞横行,清流隐退,而将军奋战多年,护得社稷黎庶平安,军心民心都早已归附,如今昏君奸臣又蓄意加害将军,此是起事良机,只要大将军暂时不废去国主之位,那些清流必会称赞大将军铲除奸臣的功业。若是大将军不能当机立断,只怕不仅大业难成,将军也会遭到杀身之祸,到时候覆巢之下,焉有完卵,不仅将军家人会遭到牵连,就是将军麾下的将士也不能幸免于难。到时候名将黜退,功臣身死,大雍铁骑必会趁势南下,南楚社稷颠覆,将军纵然身死九泉,怕也不能瞑目吧?”

陆灿默然良久,道:“我幼时曾随先生读史,古来名将多半没有好下场,能够马革裹尸已经是苍天护佑,多半都会死在朝堂之上,其时先生便对我说,我陆家世代为将,要学孙武功成身退,不可学韩信居功自傲,更不要学李牧孤忠而死,我却对师父说,若是太平无事,不妨学孙武明哲保身,若是战事不休,我便不会轻易隐退,纵然做了韩信、李牧,我也不悔。

灿祖父为武帝擢于行伍,起于草莽,而为大将,生前恩宠,死后陪葬王陵,恩遇之深,世所罕见,本应忠心以报,可是先王昏庸,奸佞弄权,贤王陨命,良臣斥退,父亲忧虑自保,缄口不言,以至于眼看国都险入敌手,君臣被掳。父亲率勤王大军进入建业之时,看到昔日花遮柳护的都城皆是断瓦残垣,便曾亲谒武帝陵寝,泣血请罪,此恨此辱,父亲至死难以忘怀,更是自惭不曾犯颜直谏,以护社稷,临终之前,更是对陆某谆谆教诲,不可顾惜身家性命,也不可顾惜声名权势。所以这些年来,陆某不顾权臣讥讽,独断专行,屡忤尚相,今次更是得罪王上,都是为了社稷安危,可是若是陆某借朝廷之失,以清君侧之名谋反,岂不是令父祖在地下蒙羞,坏了陆氏忠义之名。”

韦膺闻言起身急道:“大将军,你怎能为了忠义之名,就辜负了将士之心,若是雍军渡过长江,灭亡南楚,大将军你纵有忠义之名,又有何用,难道将军不念江南亿万百姓安危,忍见战火兵燹,摧毁楚地山河么?”

陆灿微微一笑道:“我纵然反了,难道就可挽救社稷黎民么?那你就太看轻了雍帝和先生了,先生用计素来考虑周全。我若起兵谋反,江南大好河山,立刻便陷入内乱战火,虽然尚维钧手中兵力远不如我,可是只需我和容渊在江夏大战旬日,雍军就会趁势南下,纵然江夏无事,江陵也必不保。宁海水军仍然在尚相手中,而且宁海主将赵群乃是王族,必会起兵勤王,到时候东海水军趁势进攻,宁海军山也将不保,到时候将有何种结果,你该心知肚明。纵然宁海、江陵无事,一旦内乱纷起,支持尚相的世家必然起兵相抗,到时候战事必然一发不可收拾,还有什么力量抵御雍军南下。我若一反,便是倾覆社稷的罪人,陆某岂是愚忠之人,只是大丈夫有所为有所不为,为了身家性命谋逆犯上,此事万万不行。韦膺,你莫非还不明白么,先生便是利用了你的复仇之心,若非如此,恐怕这离间计策还不会这般成功呢。”

韦膺只觉心中巨震,身躯摇摇欲坠,踉踉跄跄退了几步,陆灿起兵可能会面临的情势,他也有些预料,令他受创深重的乃是陆灿所言,自己举动竟在江哲意料之中。若是别人这样说,韦膺只会嗤之以鼻,可是陆灿不同,多年来和陆灿相交,韦膺深知陆灿才智,而且陆灿曾是江哲弟子,对于江哲自然颇为了解,他若这样说,必是十拿九稳,被仇人利用的屈辱和恐慌令他几乎难以自持。这时候,他耳边传来陆灿淡漠的声音道:“陆某虽无权臣之心,却有权臣之实,平日却是轻忽朝廷猜忌,和容将军之间的嫌隙也是由来已久,所以才会中了先生圈套,今日的结局,其错在我,以先生为人,必然还有后续手段,想来陆某性命不久,韦先生虽然略有私心,但是却始终无负陆某,这次我已经不可能返回江淮,道路消息也必定已经被尚相断绝,所以有些事情只能请韦先生相助了。”

韦膺艰难地道:“大雍铁骑仍在虎视眈眈,若是朝野上书进谏,大将军再向尚相示好,未必没有生机,尚相还不是糊涂之人,终有挽回的可能的。”说出这番话来,他自己也是不信,若非是相信陆灿非反不可,他又怎会轻身来见陆灿,而且江哲的手段他也见识过,若说江哲的计策会有这般明显的漏洞,他也不会相信。

陆灿微笑摇头道:“能否活命姑且不说,这次尚相既然准备动手,也必定不会只对着陆某一人,诸多旧部倒也罢了,尚相必然不会一网打尽,若不留下一些将领,如何可以对敌雍军,但是淮东杨秀、蜀中余缅、和淮西石观必是难逃池鱼之殃。这三人之中杨秀虽然是我亲信,却是旧蜀之人,在江南并无根基,我修书一封,你代我告诉他,委屈他投效尚相,若有他相助,尚相便可掌握淮东大军,尚相必会接纳于他。余缅是我旧部,近年来屡次遭遇败绩,但是我却不怪他,蜀中精兵几乎皆被我抽空,他能靠着数万士卒对抗雍军二十万之众,已经是十分不容易了,可是尚相若是掌管兵权,绝对是放他不过的。余缅的性子我知道,他对尚相早已是十分寒心,又非是世家出身,所忠的不过是陆某罢了,若是我一死,他恐怕就会投了雍军,若是他真的有了反意,必然先会逆旨不遵,一旦有了这样的迹象,你便派人将我随身佩剑封送给余缅,他自会知道该如何做的。石观之事,有些难为,其女玉锦和云儿结缡不到一年,玉锦更是已经有了身孕,性子又是贞烈无比,恐怕不肯合离,不过石观应该明白其中利害,我也只能听之任之,你只要告诉云儿我的意思即可。”

韦膺已是肝肠寸断,纵然是昔日晓霜殿上功败垂成,他也没有这般痛悔,伏拜在地道:“大将军,若是起兵尚有生机,难道大将军就不顾及夫人和几位公子小姐么,云公子年纪虽轻,却是勇猛善战,更是新婚不久,少夫人更是有了身孕,再过五个月就要临盆,难道大将军想让自己的孙儿连父亲之面都见不到么,风公子虽然年少,却是聪明颖悟,梅小姐和小公子都尚未成年,大将军何忍他们同遭劫难。”

陆灿眼中闪过一丝泪光,却偏过头去,黯然道:“尚相为了收拢陆某旧部,必然不致于将陆某明正典刑,更不会立刻便对陆某妻儿动手,云儿从军数年,颇有威名,尚相或者不会放过,可是风儿、梅儿和霆儿都还年幼,若是我所料不差,尚相会将陆某家人迁徙南疆,不过想必会在途中加害。韦兄,你虽然相助陆某数年,可是毕竟仍是凤仪门所属,若是辰堂被毁,凤仪门也是势力大减,所以只要韦兄不明着和他们作对,尚相还是容得你的,我身死之后,若是能够顾念旧情,尚请设法援手,也不必托付给陆某旧部照看,寻个荒村,让他们安身立命。”

韦膺闻言面如死灰,知道陆灿心意已决,定然是不会起兵谋反的了,陆灿竟将身后之事都已经安排妥当,只为了军中不起变乱,不让大雍趁势南侵,想到若非自己私心作祟,也不会让陆灿没有丝毫应对机会,而陆灿直到此刻,仍然以后事相托,全不介意他的污名错失。心中渐渐有了主张,他紧咬牙关,丝丝鲜血渗了出来,起身再拜道:“将军放心,韦某就是舍了性命,也定会尽力护住将军家人。”

陆灿释然道:“我相信韦兄必会不负所托,你我相交多年,今日一别,可能再无后会之期,本不该相促,但是钦使已在路上,不便让人看见韦兄此刻还在这里,只能请韦兄连夜动身了。”

韦膺轻轻点头,双手接过陆灿递过的佩剑和书信,心中又是剧痛阵阵,忍着伤悲,转身向外走去。刚走出房门,便听到外面人声喧嚣,隐隐听见“钦使”、“圣旨”的词句,心中已知是建业的旨意到了,那亲卫早有准备,引着韦膺从侧门离开了别业。

走出院门,韦膺忍不住绕到前面暗中看去,只见被军士堵在门口的钦使正怒气冲冲地向着守门的将士大骂,而韦膺一眼便看到那钦使身后身穿内侍服色的几人,那面容竟是十分熟悉,虽然面容略加修饰,衣裳中也作了手脚,看不出是女子所扮,可是却瞒不过他的眼睛,不由心中暗恨,昔日堂堂的名门弟子,如今竟沦落如此,在南楚苟且偷安也就罢了,还只知道排除异己,不过是因为陆灿不接受她们的示好,便不惜摧折栋梁,这般目光短浅,当真令人痛恨。

就在韦膺咬牙切齿之时,门内走出陆灿亲卫,传下军令,放了那些钦使进去,韦膺心中一冷,知道事情终于不可能再挽回,这时候,暗中走出两个中年汉子,都是恭恭敬敬地躬身行礼,其中一人急急道:“首座,接下来我们该怎么办才好?”

韦膺抬起头来,眼中皆是绝决之意,道:“知己之恩不可忘,我们先去淮西见陆少将军,厉鸣随我一起走,崔庠调动辰堂所有人手,听我号令,我若能说动少将军起兵,大将军还有一线生机,若是不能,我便去淮东见杨秀,无论如何,总不能这般听天由命。”

\chapter{第三十七章 斩草除根}

同泰十四年八月,公练兵精熟,乃与容渊订约,合取襄阳,容渊遇强兵相阻,不得进,阻于竟陵,公出义阳,进宛、邓,破襄阳,闻楚国侯江哲守谷城,乃挥军攻之。哲于城上抚琴,公闻之而退,叹曰:“吾师不可轻犯,稍待一夜。”

竟陵兵退,容渊闻公取襄阳,怒急,连上二表诬公拥兵自重。时民间流言起,歌曰:“陆王扬新令,千营共一呼。”尚相疑公有自立之意。

公不知江南生变,攻谷城十日,将下,公知襄阳危殆,弃谷城回师,败雍军于城外,虑襄阳无援,请援兵于朝中。尚相闻之,更疑公暗通雍人,乃促国主下诏召还,公辞以将在外,国主闻之而怒,连下七道退兵诏书,公外无援军,内乏粮草,不得已而返。临风泣曰:“大业未成,而中道南渡,从今后再无中原之望。”

公班师,襄阳父老阻马道:“我等助大将军守城,雍军以军法治襄阳,必不赦之。”

公闻言泣下,乃缓行,候民南迁。雍军闻之怒,苦攻不退,公守七日,焚襄阳而归。

九月,公班师至安陆,钦使至军中,促公轻身入京,部将或劝其反,公曰:“岂可负忠义。”乃抱病就道,三军皆泣下。

——《南朝楚史·忠武公传》

韦膺知道此时淮西主将石观在寿春坐镇,陆云却是在钟离统率飞骑营和雍军作战,这些年来陆云在宿州和萧县之间往来纵横,避敌锋锐,击敌软肋,已经是极富盛名的少年将领,尤其是前两年,陆云和石玉锦两人常常一起上阵,瞻之在左,互焉在右,搅得敌军人仰马翻,若是能够得到陆云支持,振臂一呼,至少淮西军便会鼎立支持。父子连心,或者可以逼得陆灿不得不反。甚至不必竖起反旗,只要故意挑起边衅,和大雍开战,战事一起,尚维钧必然不敢轻易害死陆灿。想到此处,韦膺便不顾辛劳,连夜向钟离赶去,他知道一旦陆灿束手就擒,朝廷的钦使也会到淮西去,所以定要快马加鞭,敢在那钦使的前面。

九月二十二日,一身风尘的韦膺赶到寿春,本来已经不准备入城,而是直接赶到钟离去见陆云,岂知便在城门处看到一个身着银甲,披着血红大氅的少年将军率着十余亲卫,从城门处杀出,那少年将军手提银枪,枪影闪处,那些守城的军士都纷纷逃开,让那少年一行人冲出了城门。

韦膺避在路旁,极目望去,只见那少年将军身前似乎坐着一人,更用大氅将那人牢牢裹住,那般英姿飒爽,令人一见心折。可是韦膺见了便觉心中一寒,那少年将军虽然一身戎装,他也认得出正是陆云之妻石玉锦。石玉锦不同寻常女子,这几年一直与陆云并肩作战,为飞骑营副将,悍勇刚烈之处,更胜男子,上阵之时,每着银甲,和陆云形容仿佛,雍军皆知陆石之名。她即是南楚极负盛名的少年将领,又是石观之女,怎会从寿春城厮杀而出。韦膺正在犹疑之时,那些人已经从他身边如同风驰电掣一般掠过,大氅被风吹起,露出石玉锦身前那人容貌,竟是一个清丽娇美的少女。而令韦膺心惊的便是,那少女竟是陆灿独女陆梅。石玉锦本已怀了五月身孕,否则也不会离开钟离,回到寿春休养,却在这个时候策马狂奔,莫非是朝廷钦使已经对淮西动手,还是石观有什么举动。韦膺心中尚未想通此事,便看到城内涌出一支身穿禁军服色的军士,竟是耀武扬威地向石玉锦一行人追去。

韦膺差点没有跌下马来,这队禁军也未免太嚣张了吧,竟在淮西追杀石观之女,石观只需暗示一下,便会有人将他们围歼,最多将责任抛给雍军就是了,心中疑念顿起,莫非石观这么快就投靠了尚维钧,所以要加害陆梅,而石玉锦违背父命,救走了梅儿。继而,韦膺看到一队淮西军骑兵也冲出了城门,心中越发焦虑,此刻韦膺更不想进城去见石观了,若是石观果然已经投向了尚维钧,那么自己就是出手救援石玉锦,也是全无作用,若是没有,那么自己就更不用多事,还不如立刻赶到钟离,让陆云心中有些准备的好。只是韦膺心中已经涌上失败的阴影,难道忠义如陆氏也不能得到苍天见怜,徒让那阴险狡诈之人逞凶么,莫非自己真的一点机会都没有了么?

石玉锦隐在头盔下的面容已经是一片苍白,数月不曾骑马,只觉已经生疏许多,更何况隐隐的不适之感让她总觉得有些头晕目眩,可是她仍然坚定的坐在马上,不愿露出一丝疲惫。紧紧抱着梅儿,她心中满是激愤,十余日前得知公公陆灿被人谗言加害,她便心中不安,催促父亲上书替公公辩白,却如石沉大海。更令她惊心的是,昨夜父亲身边的亲卫偷偷跑来告诉于他,尚维钧派来了使者,说是大将军已经被擒拿入京,更要将在淮西的陆氏三兄妹秘密擒回建业,而父亲竟然已经同意了,只是要求保住自己一人。

石玉锦痛恨父亲负义,也不耽搁,立刻就去寻到陆梅,只带着身边亲卫矫命冲出寿春城,她一心想要去钟离和陆云会合,也顾不上身体不适,更顾不上向梅儿说明事情真相,只是一心赶路,幸好守城军士都不敢和她交手,才让她轻易冲出了城门。离城不久,她便发觉身后有禁军追来,心中一横,索性率着亲卫回马杀去。

那些禁军这几年虽然也经过训练,可是比起经年厮杀的淮西军精骑来说,不过是初生牛犊,虽然他们毫无畏惧地迎了上来,但是却被石玉锦一行人轻易击溃,石玉锦一马当先,一枪没入那为首的禁军将领的胸口,石玉锦正欲奋力将那尸体挑飞,却觉手中一软,力道一散,鲜血飞溅了过来,她一身银甲皆是鲜血,幸而陆梅已经被她用大氅护在胸前,才没有沾染上鲜血。石玉锦深吸了一口气,银枪向四散奔逃的几个禁军士卒指去,高声道:“一个不留。”

正在这时,远处烟尘滚滚,却是一个中年将领带着百余淮西军士赶了来。那些淮西军士两翼延伸,如同双臂伸张,将那些逃向他们方向的禁军卫士护了起来,为首的将领高声道:“少将军,将军有令,请少将军和陆小姐立刻返回寿春。”

石玉锦怒道:“陈明,你竟敢来拿我,难道忘记了当初是谁替你报了杀兄大仇,你也算对得起云弟和我。”

那中年将领面上露出惭色,却忐忑不安地道:“少将军,军命不敢不从,将军命我转告少将军,天下之大,哪里又有逃生之处,与其苟延残喘,不如搏个忠义之名,而且将军定会上书保奏,未必没有生机可言,还请少将军体谅将军的苦衷,不要担上不忠不义之名。”

石玉锦本就是性如烈火,提起银枪指着陈明骂道:“我不管什么忠义,若论忠义,还有何人可以胜过大将军,可是国主一道旨意,就可以将公公困入牢狱,我可不会让云弟、二弟和梅儿去建业送死,你回去告诉我爹爹,当初这门亲事也是他促成的,我们石家更是陆家提携起来的,若是他忘恩负义,帮着那奸相来为难我们夫妻,我就是一死,也不认他做爹爹。”

陈明闻言眼中闪过异色,道:“少将军既然这般说,那么末将就只能冒犯了,上,将军有命,不许伤了少将军和梅小姐。”

石玉锦闻言大怒,想不到陈明竟然真敢出手,正要提枪上前,几个亲卫抢出,高声道:“少将军先走,我们断后。”

石玉锦一愕,若是从前,别说是让部下断后,就是自己冲锋慢了一步,还要懊悔几日,可是想到自己如今的状况,再想到怀中的梅儿,与其陷在这里,不如先走,更何况彼此非是仇敌,只要自己逃走了,那些军士自可弃械投降,想来陈明也不会难为他们,想到此处,她厉声道:“陈明,你若杀了他们,迟早必死在我枪下。”说罢策马狂奔而去,尚有八名亲卫随之而去,一半亲卫自动留下阻住追兵。不过片刻,石玉锦等人的背影已经消失无踪,那些亲卫死命厮杀抵挡,陈明被阻了片刻,已经是追之不及,叹息一声,道:“少将军已经走了,你们还不弃械投降,跟我回去见将军请罪。”

那些亲卫都是石观旧部,只不过被石玉锦选去做了亲卫,若非是为了少将军,也不会和陈明作战,闻言都是心神一泄,先有两个亲卫被击落马下,另几个亲卫见状也是苦笑着丢下兵刃,任凭陈明麾下的军士将他们捆绑了起来。

岂料这时,一个禁军拿着钢刀上来就是乱劈而下,陈明等人均未料到,眼看着一个亲卫倒在血泊当中,那个禁军才被其余淮西军士制住,那禁军仍然不依不饶地道:“这些叛逆贼子,个个该杀,陈校尉若是袒护他们,也是同罪。”

陈明眼中闪过一丝凶光,心念一转,想起将军严令,终于强忍愤怒地道:“他们犯了军法,自然有将军处置,却不用阁下多事,这里是淮西,不是建业。”那禁军终于发觉众人眼中的怒火,想到如今自己不过寥寥数人,若是被人杀人灭口,却连“冤枉”二字都喊不出来,还是回去见到钦使大人再添油加醋一番吧。想到这里,他的气焰立刻降了下去,目中更是露出惧色。陈明冷冷看了他一眼,高声道:“回营!”说罢自己上前抱起那被杀的亲卫尸身,上马狂奔而去。其余淮西军士相视一眼,纷纷斩断那些投降亲卫的绳索,让他们自行上马回去,免得又被那些禁军残害,掉头不顾而去。那些活下来的禁军都是心中暗怒,却也顾不得同伴的尸身,只是策马跟着淮西军离去,免得落单之后死个不明不白。

石玉锦策马奔出许久,才想起看看陆梅的情况,喝令众人停住坐骑,掀起面甲,打开大氅,检视一番,见陆梅身上并无伤痕,这才放心,耳中却传来呜咽之声,惊讶地看去,却见陆梅清丽如仙的面容上满是泪痕,感觉到石玉锦紧张的目光,她抬起头来,鼓起勇气问道:“大嫂,发生了什么事情?为什么他们说爹爹被下狱了,为什么石伯伯要抓我们?”

石玉锦心中一痛,道:“梅儿,你不用担心,父亲虽然有些碍难,但是想必不会没有转圜余地,我爹爹负义,我也瞧他不起,不过想来他也不会斩尽杀绝,我们还是先去寻你大哥,到时候有飞骑营相护,想来也没有人敢对我们动手。”

陆梅明眸中珠泪隐隐,她低声道:“我知道大家都不愿意告诉我真话,太后想要让我入宫作贵妃,我也不愿,可是二哥骗我来寿春,却不告诉我实情,如今大嫂也是这样,都是梅儿没有用,不能帮忙大家,还要拖累嫂嫂。”

石玉锦越发酸楚,低声道:“傻丫头,你是陆家的掌上明珠,若是还要你去操心战场厮杀、朝廷争斗的事情,还要我们这些人还做什么,你不要担心,我就是拼了性命,也会护住你平安,最多我和你大哥双枪杀出淮西去。”

陆梅闻言更是珠泪滚滚,倚在石玉锦胸前哽咽不语,八名亲卫也都是黯然失色,其中一人恨声道:“将军素重信义,这一次如何依附权相,竟连少将军也不顾惜。”话一出口,便觉失言,只见石玉锦面上越发苍白,竟是一口鲜血奔出,陆梅不由一声惊呼,伸手扶住石玉锦,众人都知道石玉锦素来争强好胜,此番逃出寿春的奔波劳苦却不如父亲的所为令她伤痛。那亲卫愧悔难当,狠狠打了自己一记耳光。石玉锦睁开眼睛,淡淡道:“不关你的事情,罢了,我们先去钟离吧。”

此言一出,众人齐声应诺,就在这时,却传来一个幽冷的声音道:“钟离路远,恐怕诸位是去不成了,还是让本座送石少将军和陆小姐去黄泉路吧。”

众人闻声望去,却见左侧小径上,百余丈外款款走来一个青衣女子,看似动作极慢,但是转瞬之间便已到了近前,足不沾尘,青衣飘舞,风姿秀丽,虽然眉梢眼角带些岁月痕迹,但是动人之处,不亚于二八少女,她一身上下,除了背上一柄青锋剑外,再无旁物,越发显得朴素无华。

石玉锦眉头紧锁,望着那青衣女子,她也曾学过峨嵋武技,并非只会战场厮杀的武功,一眼便看出这女子双目寒光四射,一身剑气凌人,乃是少见的高手,若是战场厮杀,自己还有几分机会,若是江湖搏杀,自己必然是一败涂地。

轻轻拍了拍有些微微颤抖的陆梅,石玉锦高声道:“你是什么人,竟敢拦阻本将军的道路?”

那青衣女子眼中闪过一丝嘲讽,淡淡道:“本座凤非非,想来少将军也未必听过这个名字。”

石玉锦心中有些茫然,觉得有些熟悉,却想不起这个名字在哪里听过,不知怎地,石玉锦却觉得那女子讥讽的神色并非是针对自己,更像是一种自嘲。不过此刻她也顾不得考虑这些,使了一个眼色,一个亲卫策马过来,低声道:“得罪。”然后伸出双手将陆梅抱了过去,放在了他的马上。陆梅虽然有些不安,但是那亲卫已经有三旬年纪,倒像是她的长辈一般,动作又是小心翼翼,陆梅心中又担心石玉锦,所以也就没有流露出异样的神色。

石玉锦将陆梅送到一边,心中一宽,提枪指着那青衣女子道:“不管你是何人,想要取本少将军的性命,还要问我的银枪答不答应。”

那青衣女子凤非非冷冷一笑,石玉锦只觉眼前一花,漫天剑影已经到了身前,石玉锦也顾不得分辨剑势来处,心中涌起强烈的危机感觉,一声厉喝,银枪平平刺出,直入剑影中心,这一枪充满沙场血战的气魄,已是两败俱伤的的招式,一声脆响,如雪剑光中传来一声惊咦,但是剑光丝毫没有减弱的迹象,便如潮水一般扑了过来。石玉锦只觉眼前皆是剑影,就连青衣女子的身影都看不到,她索性微阖双目,也不去看那耀眼的剑光,便凭着心中灵悟,只是将银枪抖开,枪影如梨花,散落如雪。凭着千万军中纵横往来的枪法,竟是将那剑光挡住,但是石玉锦心知自己不过是凭着不顾生死,以及沙场血战的经验拼了平手,若是再斗下去,最多不过三十招,自己便会伤于剑下。石玉锦是沙场骁将,不是江湖女子,想到此处,也顾不得什么规矩,高声道:“大家一起上,围杀此人。”

众亲卫早已严阵以待,一听石玉锦号令,除了两名亲卫留下护着陆梅之外,其余亲卫已经提枪举槊而上,六人结成战阵,相互呼应,向那青衣女子背后杀去。那女子剑法虽然高明,但是在石玉锦和六名亲卫围攻之下,也是陷入了守多于攻的境地,更何况六人还有马匹相助。

凤非非有些恼怒,冷笑道:“素闻石观之女年纪虽轻,却是沙场骁将,英勇善战,如今看来也不过倚仗人多势众罢了。”口中不停,剑势也越发凌厉,丈许方圆之内,皆是剑浪雪影,滚滚如潮。

石玉锦也不理会她,战场上若是斤斤计较什么,哪里还有取胜的可能,一柄银枪越发出神入化,剑浪之中飞腾纵跃,宛似蛟龙戏水,一招一式已臻化境,这一刻,她渐渐忘却了危机四伏的处境,数年沙场血战,生死一线的危机,加上心灰意冷,漠视生死的心境,竟让她奇迹一般地晋入了枪人合一的境界,只觉得手中银枪仿佛有了自己的生命,自动挡去敌人攻击,刺向敌人要害,枪剑交击的清脆响声不绝于耳,凤非非虽然武艺高强,但是宝剑毕竟不如长枪一般利于攻远,只觉得内腑连连受到震荡,不由心中一寒,心中有了脱身之念。

偏偏就在此刻,石玉锦突然觉得腹痛如绞,她这般奋力厮杀,已经是动了胎气,忍不住一声轻呼,手中银枪一颤,露出了一线破绽。凤非非乃是江湖上一等一的剑术高手,趁机一声厉喝,手中银光暴射,血花飞溅,数声惨喝,几个亲卫已经捂着咽喉向马下栽倒,凤非非竟然趁着难得的良机,将在后面助攻的六个亲卫一并杀死,剑光一敛,凤非非已经退出数丈,面色显得有几分苍白,这一剑她也是竭尽所能,消耗极大。

石玉锦只是手中一缓,几个陪着她沙场血战的亲卫就已经当场身死,不由心中大恸,可是腹中剧痛再次传来,她不由惊骇万分,这时,凤非非已经合身扑上,石玉锦再也不敢接战,惨然道:“快走。”声音未落,已经策马向荒野奔去,那护着陆梅的亲卫也随即扬鞭追去,而另一名原本执刀护着陆梅亲卫却策马向那青衣女子冲去。三人两骑还未奔出多远,便听见身后传来惨呼之声,那名仅存的亲卫回头望去,只见自己的兄弟人头飞起,尸身正被那青衣女子踢落马下,那女子已经落在马鞍上,正欲策马追来。而前面马上,石玉锦已经是伏在马背上,似乎已经陷入昏迷,若非是习惯和直觉让她紧紧抱着马颈,恐怕已经坠落马下。那亲卫心中一惨,铁青的面色上露出狰狞之色,他高声道:“梅小姐,你护着少将军。”说罢纵身离鞍,落在地上,立在道中,迎向飞来的追骑。

陆梅一声痛呼,但是她虽然年幼识浅,却也是将门之女,知道此刻生死攸关,两人三命皆在自己手中,幸好她也会些骑术,虽然不精,但是此刻心中尽忘一切,策马飞驰,居然追上了石玉锦,此时,石玉锦已经失去知觉,身躯摇摇欲坠,陆梅心一横,飞身扑去,全不顾生死,居然给她跃到了石玉锦身后马鞍之上,握住已经松落的马缰。觉出出了一身冷汗,陆梅暗中庆幸不已,原本她跟着二哥练习这一招的时候,十次倒有九次会坠马,幸好有家将护卫,才没有折断脖颈,后来便被娘亲禁制再练习这样危险的招数,幸好这一次侥幸成功。略略冷静下来,她生恐那青衣女子追来,手中没有马鞭,她心中一狠,拔出腰间用来自卫的匕首,向马臀刺去,那白马一声长嘶,发狂一般向前方奔去。陆梅只觉耳边风声阵阵,早已看不清两边景物,只能紧紧抱着石玉锦,拽紧马缰任凭那骏马狂奔。

却不知身后凤非非正在切齿痛骂,哪里还能追来,那最后拦阻的亲卫武功在她看来并不足道,岂料那人口中发出长短不一的呼哨声,那些战马听了,都是四散扬蹄奔去,就连她身下那匹战马也是发狂一般,极力想将她甩落。她一个失神,便缰绳脱手,幸好她轻功过人,飞身而起,没有被惊马伤到,眼看着可以用来追敌的战马失去,她只能一剑刺死那亲卫泄愤。不料那亲卫竟然拼死抱住她的右腿,她虽然已经三十多岁年纪,却还是未嫁之身,心中不由慌乱,连连砍了几剑,才将那亲卫双手斩断,脱身出来。看到那亲卫睁得滚圆的血红双目,她心中怒火上涌,狠狠地挥剑将那亲卫尸身斩成十七八段,才终于消去怒火。看看远方,也不知道那两个目标已经逃到何处,她只得轻叹一声,准备先去钟离守株待兔。身躯方动,却觉得背心一痛,继而麻痹的感觉从脊背向全身蔓延,她艰难地想要提剑,却是手一松,长剑落地,然后她的身躯便向前仆倒,且感觉到身体一分分失去知觉,她勉力喝道:“是谁,偷袭暗算,非是英雄。”

一个清冷的声音从身后传来道:“凤仪门的三姑娘,如今却成了追杀忠臣名将家眷的刺客,莫非这就是名门弟子么,在我看来还不如这些忠心护主的将士,我晚来一步,真是可惜了这些英雄男儿,凤姑娘,九泉之下,不知道你有没有颜面去见尊师。”

凤非非能够感觉到生命的逝去,她的目光渐渐灰暗,嘶声道:“你是谁,我要知道你是谁?”

身后那人漫声吟道:“落花流水两关情。恨无凭。梦难成。倚遍阑干,依旧楚风清。露滴松梢人静也,开宝篆,诵黄庭。(注1)将死之人,何必还要知道那么多事情,莫非你还想托梦给你的师姐妹们么?”

风非非脑海中泛起模糊的影像,少女时候父母双亡的凄苦,拜入师门之后风光荣耀,一心练剑博得师父欢心的辛苦,师姐妹们闲来谈笑的情景,一幕一幕回想起来,渐渐的,一切皆化作过眼云烟,她的身躯渐渐停止了挣扎,双目失去了神采。

那人将凤非非的尸身翻了过来,目光落到她青灰色的玉容上,叹息道:“你虽然只知人云亦云,可是这些年来也算是洁身自好,没有过分辱没师门,如今你既然已经死了,我也不愿你多受屈辱,卿本佳人,奈何作贼,今日归于黄土,也莫要再生遗恨。”说罢,那人将手中玉瓶之内的药物倒在风非非身上,不过片刻,红粉佳人便已化作一滩清水,渗入地下,只余下一些零散物事,那人皆用黄土埋了,然后便循着马蹄印走去,不多时已经没入荒野之中。

九月二十三日,钟离城内,刚刚从宿州战场返回的陆云和等在钟离一夜的韦膺一起得知了石玉锦、陆梅失踪的消息,韦膺心中悔恨没有保护二女一起到钟离,陆云却是神色沉静如水,毫无一丝激荡,似乎并不在意,可是韦膺分明能够觉察得出来,这少年身上深沉的悲哀。劝慰了陆云几句,韦膺开口相劝陆云起兵救父,陆云却只是摇头不语,在旁边早已是泪流满面的陆风目中闪过光芒,厉声道:“大哥,你就是不恨他们害得大嫂和妹妹失踪,难道也不顾及爹爹的性命么?”

陆云收回淡漠的目光,道:“我早已立誓和爹爹一样尽忠报国,死且不悔,爹爹尚且束手就缚,不肯反叛,我焉能败坏爹爹的忠义之名。”

陆风怒道:“难道为了忠义之名,就可以不顾亲人生死么,他们是要斩尽杀绝,不仅是要杀了爹爹,恐怕还要杀你,甚至还要杀大嫂,杀梅儿,就是娘亲和小弟也逃不过一死,凭什么我们陆家要死尽死绝,才是忠义,狗屁!”

陆云面上闪过怒色,挥手一个巴掌,将陆风打倒在地,指着陆风骂道:“你若有此心,就不是我陆家的子孙,爹爹平日的教诲你都忘记了么。”

陆风吐出口中鲜血,惨然道:“爹爹平日总是说陆氏子弟,纵死不能负忠义,为家国不可惜身,为黎民不惜荣辱。可是我不甘心,永远也不甘心。”

陆云冷冷道:“你既然记得,如何敢出此狂言,若是爹爹肯反,岂会自缚入京,爹爹尚且如此,我岂能谋反,我若提兵杀回建业,只怕正好做了雍军前锋,到时候那昏君奸相便可名正言顺的将爹爹杀害,身为人子,岂可陷尊长于不忠不义。更何况爹爹不反,自是不愿见江南亿万黎庶死于内乱,我也是这般想法,我们一家人就是都死了,若能免去内乱灾祸,也是死得其所。”

陆风眼中滴下血泪,嘶声道:“难道娘亲、大嫂、梅儿和小弟的性命,大哥就一点也不顾惜了么?”

陆云眼中闪过一丝痛楚,他柔声道:“二弟,娘亲和小弟现在建业,我若起兵,必然是先害了他们,玉锦和梅儿虽然失踪,但是总算还没有见到尸身,说不定还有生还的可能,爹爹和我为国而死,无怨无悔,你却不能留在这里。现在你立刻更名换姓,远走高飞,为我陆家留一脉香烟,这便是你的功劳。”

陆风闻言泣道:“大哥,不,你和我一起走吧,与其给他们杀了,不如我们一起走吧。”

陆云背过身去,淡淡道:“陆氏一门,除了爹爹之外,便只有我在军中,我若逃生,那奸相必然加罪诬陷爹爹,更何况我在外一日,奸相始终不能安心,必然不会放过娘亲和小弟,我若身入囹圄,他们才会放松对玉锦、你和梅儿的追缉。你也不要担心,爹爹和我未必就没有机会生还。”

陆风大哭道:“不,我也要和大哥一起去建业,要死我们死在一起。”

陆云叱道:“糊涂,你若也死了,将来玉锦和梅儿,甚至娘亲和小弟还能倚靠何人?”说完这句话,颜色稍缓,又道:“还有一件事情,你要记住,当年我去雍都刺杀师祖,谁知连动手的机会都没有,丢尽了面子,却也结识了几个朋友,如今他们多半已经上了战场,无论于公于私,你若见了他们,他们必然会庇护于你,就是师祖也曾说过,将来若有危难,可以投奔于他。可是我陆氏子弟,怎能投靠敌国,所以你要记得,纵然陷于生死绝境,也绝对不可投靠大雍,更不可和南楚为敌。”

陆风知道兄长言出如山,颇有父风,不敢再违逆,只是默默点头,一滴滴血泪落在尘埃。

陆云也不回头,语气中又多了几分悲凉,继续道:“你去吧,若非淮西军尚未出动,只怕朝廷钦使已经到了钟离了,若是,若是还能见到玉锦,替我转告她,要她别怪岳父大人,岳父的苦心,她终究会明白的。”

陆风心中悲愤,想到若非石观这么快就投靠了尚维钧,也不会让自己一家陷入这样处境,正要破口大骂,却听见水滴落地的声音,看到兄长肩头轻颤,再也不愿让兄长痛心,大哭着向外奔去。

良久,陆云回过身来,对着默然站在一边的韦膺一揖道:“韦伯父,让你失望了,爹爹的托付还要请你多多费心才是。”

韦膺只觉心中剧痛,强忍悲怆道:“少将军不愧是陆氏嫡长,想来大将军业已料到,就是韦某违背他的意愿,也是无济于事。”

陆云低声道:“云有负伯父厚望,将来若是伯父见到拙荆,还请转告他,岳父大人也是不得已,他这样做也不过是想迫着拙荆远走高飞罢了,拙荆性情刚烈,若是岳父不这样做,拙荆绝不会离开淮西避难。”

韦膺叹道:“膺再无话可说,这就去淮东见杨参军,转呈大将军之命。”说罢转身黯然离去。

离开钟离,韦膺一路狂奔,赶向广陵,那里是淮东军大营所在,刚刚进入淮东境内,韦膺便得知了一个消息,雍帝李贽因为襄阳战事大发雷霆,齐王李显、太子李骏、襄阳主将长孙冀受到申斥,而始作俑者江哲更是被降爵罚俸,原本已经是国侯爵位的江哲,再次成了乡侯,据说若非看在宁国长乐公主面上,恐怕侯爵之位也保不住。而且李贽因为战事不利,已经下令雍军退缩防线,甚至有大雍重臣上书提议休战和议。这个消息若是放在数月之前,那是绝对的好消息,可是现在,却是催魂夺命的阎王帖子,韦膺闻讯,一口鲜血终于忍耐不住,吐在尘埃,这一刻,他再度领略了江哲狠辣周密的计策,绝不会给人留下一丝一毫的机会。

————————————

注1:宋张继先《江神子》

\chapter{第三十八章 君恩九鼎重}

公既就缚,权相命捕其党羽,以诸将皆握兵权,且缓图之。

公长子云,年十六,从石观战于淮西,素以勇武著称,观多得其力,甚爱之。观有女字玉锦,年十七,亦善战,每着银甲,骑白马,提枪携弓,与云并肩出,不分轩瑾。

同泰十三年,太后欲令云尚淑宁公主,主贤淑以闻,人皆羡之,云独不愿,语父曰:“愿娶志同道合者为妻。”公与观早已心照,遂许之。

钦使至寿春,时公爱女避祸寿春,观欲将其交付钦使监押,玉锦闻之震怒,不顾身重,抱女出城去,义烈堪敬,钦使遣兵追之,死伤殆尽,两女亦无所踪。钦使畏惧,恐云不肯就缚,促令观提军至钟离。

观故迟之,过五日乃起兵,至钟离,云久待矣,闻诏旨,曰:“陆氏忠心,天人共鉴。”乃坦然就缚。时云领飞骑营,精锐冠于江淮,众军欲截之,云饬令归营,皆不敢相阻,声威至此矣。

——《南朝楚史·忠武公传》

十月初三,楚州。

裴云立在镇淮楼上,心思郁结,眼前的秋色都失去了光彩,荆襄战事的结果早已到了他耳中,战事的扑朔迷离令他瞠目结舌,陆灿兵出义阳,趁虚而入攻取襄阳,以及之后的谷城鏖战,襄阳对峙,种种变化都令人侧目,襄阳的一失一得更是令人不解,直到得知陆灿被南楚国主赵陇解除兵权,召入建业的消息之后,裴云才隐隐明白荆襄血战、襄阳易手都是为了一个陆灿。可是即使想通这一点,裴云心中却是越发惊骇。

兵家有言,荆襄乃是天下要冲之地,长江横贯东西,连接吴蜀,由大江入湘、入赣,亦无不便捷;汉水由江夏逶迤而北以至西北,自襄阳西北行入汉中、关中,北行入南阳、洛阳,或水或陆,皆有通道,欲得天下,必须据有荆襄,每至天下四分五裂,诸侯割据之时,荆襄更是首当其冲的战场。荆襄境内,襄阳、江陵、江夏,皆是军事重镇,而襄阳更是最重要的军镇,南楚据有襄阳,可以北上中原,大雍据有襄阳,可以威慑荆襄。早在大雍立国之初,就时时窥伺襄阳,可是那时襄阳在德亲王赵珏镇守之下,稳如泰山,雍军在襄阳坚城深垒之下屡屡受挫,不知多少勇士折戟沉沙,襄阳乃是大雍将士心中之恨。直到隆盛八年江哲设下计谋,利用杨秀攻淮东的机会,诱敌北上,才趁隙夺得了襄阳。襄阳一入大雍之手,南楚就再无反攻的机会,虽然陆灿将江南守得固若金汤,可是却也无力危及大雍的根基。

以襄阳的重要,纵然是雍帝御驾亲征,也断然不敢轻易舍弃如此重镇,可是江哲居然将如此重地当作诱饵,轻轻放手,虽然最后收回襄阳,可是大火之后,只留下残破孤城,襄阳之民又纷纷南渡,数年之内襄阳难以恢复旧观,姑且不论江哲的手笔之大,更令裴云忧心的是,根据他从少林得到的消息,这一战雍帝李贽事先竟然毫不知情,江哲乃是矫命为之。姑且不论这一战的惊险之处,只是江哲的胆量就令裴云心中惊骇欲绝,若是雍帝责问下来,恐怕是难以绾回的重罪。若是旁人,或者还会冷眼旁观,江哲恩宠之重,早令许多人不满,他在战事胶结之时,仍然嬉游于山水之间,不问军务,便令雍帝案上多了许多弹劾的奏章,如今犯下这般大罪,恐怕就是宁国长乐公主也护不住他。或许有人会想趁机落井下石,可是裴云却不能这么想,姑且不论江哲之子江慎乃是恩师关门弟子,就是他这几年也多得江哲照应。三年前杨秀攻楚州、泗州之战,裴云可以说是败了,而且事前楚州郡守罗景遇刺,此事又是大大的得罪了国舅高融,再加上扬州战败,朝中多有大臣上书,欲令雍帝降罪裴云,若非得到江哲支持,雍帝又念昔日救驾之功,只怕裴云如今已经是缧绁罪臣。这几年,裴云养精蓄锐,徐州大营战力全复,正是求战心切之时,若是江哲遭贬,裴云深恐自己也遭到连累,一旦丢了兵权,岂不是再无洗刷败战之辱的机会,所以比起寻常人来,裴云心中最是忧虑江哲的处境。

心中忧虑重重的裴云,就连杜凌峰上楼的足声也未听到,直到耳中传来杜凌峰的声音,他才反应过来,只听见杜凌峰禀报道:“将军,徐州有书至,皇上下了旨意,申斥齐王爷和太子殿下,以及长孙将军,江侯则被降了两级爵位,后来又下诏将侯爷江南行辕参赞之职也免去了。”

裴云心中一震,但是却将心中忧虑隐藏起来,面沉如水地道:“圣上如此震怒,也是难免的,只是朝中难道就没有人保奏么,无论如何,襄阳还在我军手中。”

杜凌峰犹豫了一下道:“从长安传来的消息说,皇上得知战报便是勃然大怒,虽然石相和诸位大人多有缓颊,但是明鉴司夏侯沅峰却趁机上奏,攻讦江侯怠慢职守,更将江侯三年来的行踪一一奏明,皇上这才龙颜震怒,下旨申斥,更要将侯爷除爵免职,若非是石相苦苦求情,只怕就连乡侯爵位也保不住了。”

裴云心中轻叹,目光一转,却见杜凌峰面上也有不安之意,便笑道:“你自从上次随侍江侯去襄阳之后,就是提起江侯的名字也是战战兢兢,如今江侯获罪,你理应欢喜才是,怎么倒是这般情状。”

杜凌峰赧然道:“这也怪不得凌峰,师叔不知道,上一次随江侯去襄阳,现在想起来也是心有余悸,当时荆襄还是南楚所属,江侯竟然在岘山流连多日,弟子心中时刻忧心,若给楚军发觉,江侯有所损伤,别说性命难保,只怕还要连累师门,偏偏江侯却丝毫不体念我们这些护卫的人,甚至还去远眺襄阳城楼,就是呼延将军和几位侍卫大人也都是战战兢兢,唯恐出事,怪不得人家都说江侯性情古怪,凌峰只盼一辈子都不用再服侍于他。不过如今江侯获罪,弟子却又觉得心中忐忑,倒不是为了师叔着想,师叔素来对功名富贵看得极淡,皇上对师叔也是颇为看重,纵然连累到师叔,想来也不至于有大碍,只是不知怎么,弟子总觉得江侯若是被贬,只怕更是危险。”

裴云心中一动,想不到这个素来直爽,心机不深的师侄竟也有这般灵思,当年师父慈真大师便曾说过,江哲此人渊深智海,心机深沉,阴柔诡谲,身边又有邪影李顺这样的高手随侍,若是没有羁绊,任他自由自在,只恐他一念之差,就会生出惊天变乱。幸而此人为雍帝所用,虽然可怜了天下英雄,但是能够促成江山一统,也是不世功业,而且此人有皇权约束,也可消去许多隐患。方才他得知江侯被贬,心中便有忧虑,若是江哲因此疏离雍廷,甚而遁入湖海,恐怕不是天下之幸。想不到杜凌峰竟也隐隐想到此处,看来多年历练,这个师侄已经不是从前的鲁莽少年,微微一笑,裴云道:“这几日晚上到我那里,我要看看你的进境。”

杜凌峰闻言大喜,心知师叔准备指点自己的武艺,不由摩拳擦掌,裴云看了心中暗笑,道:“好了,我也有些乏了,一起去杜家楼喝杯酒吧。”自从三年前楚州惊变之后,杜家酒楼便名闻江淮,庄青浦为师报仇的义举和杜家楼的青梅酒一起传颂江淮,就是裴云如今也是深爱此酒,只是他声威显赫,不便常去酒楼罢了,今日他心中郁闷,便想到杜家楼去散散心。

杜家楼虽然已经名闻江淮,却已然是旧日模样,并未进行扩建,青梅酒也不曾比从前多酿几坛,那杜掌柜虽然是商贾之身,却是颇有林下之风,若非是一时才俊,纵然出重金也难以购买到一坛青梅酒,若是倜傥风流之士,纵然身无分文,也可获赠佳酿。这样一来,青梅酒名声越发响亮,许多喝不到青梅酒的平常人,也多半会喝上几盏杜家陈酿,杜家楼几乎是门庭如市,若非事前订下位子,必然会被拒之门外。不过裴云自然不必忧心,楼上有一付座头终年闲置,就是为了提防有裴云这样的人物,或者是江淮名士偶然莅临,却无座位的情形。

换了便装,走在大街上,裴云倒也觉得心情好了许多,到了杜家楼,杜掌柜闻讯出来迎接,面上却露出一些古怪神色,裴云也未留心,刚刚走上二楼,便听见一个清朗温润的声音道:“晓雾锁秦楼,又添离愁。临风把盏倾金瓯。阳关唱遍也难留,此恨悠悠。青梅撷满袖,疏疏雪片。经年酿作杜家酒。饮罢孤寒立轻舟,一醉方休。庄青浦这首词意境深远,可见其才,可怜他英年早逝,当真是可惜可叹。”

裴云微微一愣,庄青浦虽然得楚州人敬爱,但是毕竟是刺杀郡守之人,所以很少有人这般当众赞他,免得落入雍军耳中,生出事端,而这人说话的语调一听便觉是长安人,既是雍人,为何如此毫无忌讳的称赞庄青浦呢?

心中生疑,足下不由一滞,耳边却又听到一个熟悉的声音道:“子良此言虽然没有什么不妥,但是也要慎言才是。”

裴云闻言更是大惊,这人刚刚被贬,如何又到了楚州,目光一转,发觉楼上除了一些目中神光隐隐,一见便知是高手精锐的侍卫散坐四周之外,再没有本地酒客,越发觉得震惊,整理了一下衣衫,他上前对着传出语声的厢房一揖道:“侯爷屈身来此,为何不曾相告裴云,也好让末将设宴为侯爷接风洗尘才是。”

帘中传出江哲清雅的声音道:“江某如今已经解去参赞之职,若非陛下隆恩,只恐爵位也不会只降了两级,裴将军何必这般多礼,今日来此,不过是想起此间青梅酒罢了,幸而老杜还留了几坛,不知让我空劳往返。”

裴云挑帘而入,笑道:“侯爷宠辱不惊,末将佩服,不过想来陛下终会体谅侯爷苦心,能令陆灿失去兵权,纵然是丢了襄阳,也未必得不回来,何况襄阳还没有失去呢。”心中不由暗暗猜想那被江哲叫做“子良”的是何方神圣,怎么听起来江哲的声音中透着几分尊重。走进厢房之内,裴云便是一惊,只见和江哲坐在一起品酒闲谈的竟是一个十八九岁的少年,相貌俊秀,虽然是一身平平常常的黄衫,却显得气度不凡,威势含而不露,而令裴云震惊的是,那少年竟是太子李骏,江南行辕的副帅。

心中千回百转,种种思绪一闪而过,裴云单膝下拜道:“末将叩见太子殿下千岁,千千岁,不知殿下驾到,未曾亲迎,还请殿下恕罪。”

李骏起身,伸手虚扶道:“裴将军平身,将军镇守楚州,令南楚淮东军不能北上青徐,劳苦功高,孤一向深知,心存感佩,还请不要多礼。”

江哲却是神情疏懒,坐在席上纹丝不动,却也不见李骏有什么异色,裴云想起曾听人说,太子李骏和江哲亲厚非常,如今看来果不其然,再看到江哲全无被贬之后应有的挫败神情,又有李骏微服相从,心中忧虑一扫而空,起身坦然道:“殿下与侯爷微服至楚州,必有教诲,末将厉兵秣马三年,只待军令一下,便要南下洗雪当日战败之辱,还请殿下训示。”

我忍不住打了一个呵欠,心道:“这几年大概是把裴云闷坏了,蜀中、荆襄、淮西都是年年恶战,只有淮东几乎是风平浪静,一见到李骏便要请战,还真是性急。”望了一眼在那里和裴云说着一些不深不浅的话语,却言辞恳切的李骏,心中越发郁闷。这一次设计离间南楚将相君臣,更是设下计策要将敌对势力大大的消耗一番,却也有激流勇退之心,所以才故意隐瞒了一些关键的事情没有告诉李贽,更是在过去的三年里面放荡不羁,果然这次襄阳之战后,弹劾我的折子便如雪片一般,李贽也果然大怒,贬了我的爵位军职。这本来在我意料之内,正好可以让南楚昏君权相放心的去对付陆灿。至于失去君恩的打击么,反正接下来的事情也用不到李贽的支持了。我还一心想着今次事后,便要趁势退隐,也免得见到故国败亡呢。不料刚刚心满意足的听到贬斥的旨意,暗中却接到了嘉奖的密旨,李贽竟全然不怪我擅自行事,还说什么南楚折损陆灿一人便可胜过十座城池。眼看着脱身之计又成了泡影,怎不让我心中气苦,若非是还念念不忘南楚未了之事,真恨不得立刻脱身事外。只是不知道那边的事情,已经进展的怎么样了,想必一两个月之内,就会有结果吧。

十二月五日,建业。

逾轮走出尚承业的私宅,已经是子夜时分,白天纷纷扬扬飘洒了一日的轻雪已经不知何时停了,晦暗的夜空,全然看不见一丝星月光芒,手中的灯笼在这迷蒙的夜色中也只能驱散开丈许方圆的黑暗,宋逾只觉得自己的心灵,便也如这黑夜一般黯淡。不知茫然走了多久,逾轮停住脚步,眼前已经是一扇黑漆木门,门上挂着一盏绿色宫灯,灯光并不十分明亮,可是在宋逾心中,却觉得这便是黑暗之中唯一的一线光明。这里,便是柳如梦在建业的住处柳园。入冬以来,寒气倍增,柳如梦便弃了画舫,住到城中来了,柳园虽然不大,却是清幽雅致,常令人有不思归去之感。伸手想要敲门,逾轮却突然生出怯意,一只手伸在半空,就是无法再向前一分。

恍恍忽忽的记起今日临行之前,柳如梦手执红色纸伞,一身素衣立在雪中相送,轻启樱唇道:“先生,如梦虽然是风尘中人,也知大将军忠义,先生和尚大人交好,若能劝他向相爷婉转陈词,免去将相之争,实是国家之幸,若是芝兰凌霜,玉柱倾颓,岂不是自毁长城,徒令亲痛仇快。”

可是自己又是如何做的,当尚承业忧心忡忡地向自己说出尚维钧至今也是犹豫不决,自己却道:“陆大将军是否谋反已经不重要,只是尚相这次这般得罪了大将军,不知道大将军会不会忘记此事,这一次大将军束手就擒,更是谕令部将不得闹事,却不知下一次是否还会这般不惜生死荣辱,任凭相爷加罪。”只看尚承业若有所思的神色,逾轮便知道陆灿距离死亡又近了一步。

不到两月时间,世事却已经是翻天覆地,不提大雍自从襄阳之战后,齐王、太子皆遭申斥,就连一向深得帝宠的江哲也是降爵罚俸,没过几日更是传来消息,江哲军职已经被雍帝解除,甚至雍军还有收缩防线的迹象,种种征兆都表明持续数年的战事有可能休止,可是这样一来,外患将去,南楚内部的矛盾越发尖锐了。

自从陆灿被解到建业,国主赵陇只是匆匆见了他一面,就将陆灿下狱,陆灿留在建业的妻子幼儿也被软禁府中,就连在淮西领军的陆云也被缇骑锁拿入京,只有陆灿此子陆风、三女陆梅和长媳石绣影踪不见。但是陆风、陆梅都未成人,而石绣又是石观之女,看在石观识趣投效的份上,尚维钧自然也不会太过分,只是下令缉拿罢了。不过他虽然不甚在意,凤仪门却是高手频出,搜索三人行踪,逾轮不知凤仪门为何如此紧张,过了些时日才从尚承业口中得知原来凤仪门的一位高手去淮西相助钦使捉拿陆氏众人,却生死不知,消失无踪,尚承业提起此事只是有些幸灾乐祸,逾轮却是心中暗自揣测,不知是否秘营出手?

不知茫然多久,逾轮突然惊觉一缕剑气从暗处袭来,久经生死的经验让他立时清醒过来,身形一闪,身形已经如同鬼魅一般避开剑气,身形如同一片枯叶般贴在墙壁上,目光炯炯向暗处望去,眼中满是警惕,虽然那剑气并无杀意,但是逾轮却是丝毫不敢轻忽,右手的折扇虚指向前方,冷冷道:“是何人在此窥伺?”

暗巷之中传来一个清朗的声音道:“宋先生见谅,在下在此久等先生归来,想要登门拜访,不料先生在门前久立,在下唯恐先生受寒,因此用些法子惊醒先生,还请先生不要怪罪。”

逾轮此刻已经恢复冷若冰雪的心境,低垂眼帘默然不语,知道方才自己心神不宁,没有留意到暗中有人,不过那人必定也是高手,否则不会这般轻易瞒过自己的耳目。心念百转,逾轮冷冷道:“宋某不过是一个轻薄浪子,阁下有何见教?”

那人沉默片刻道:“先生和尚相之子交好,建业无人不知,如今大将军被诬入狱,不知生死如何,且尚相将大将军拘于何处也是无人知晓,所以在下冒昧前来动问,先生雅量高致,不贪权势,建业无人不晓,纵然那尚承业也不能将先生收入幕中,想来先生也心知大将军忠义,还请先生不吝赐教。”

逾轮心中一冷,这人知道自己和尚承业交好不奇怪,可是他凭什么知道自己能够得知陆灿被囚之处,知道自己能够影响尚承业极深的人并不多,是什么人出卖了自己呢?想到身后院中的柳如梦,便是知道的人之一,而且两月来,更是屡次劝自己为陆灿尽些心力,莫非是她出卖了自己。心中生出不可遏制的怒气,目中闪过不屈之色,他厉声道:“阁下想要问的事情我的确知道,可是若想我说出来却不可能。”说罢身上涌出冰冷的杀气,灵觉中察觉到暗中共有两人,其中一人剑气凌人,另一人也是内力深厚,虽然觉出这两人若是联手,自己难有胜算,可是他却越发坚定了心思,生出以死相拼之心。

那暗中之人似乎察觉到了逾轮气势的变化,轻叹一声,走出暗巷,移步到门前,昏暗的灯光照射在他斯文俊朗的面容上,这人却是一个布衣儒士,身佩长剑,一身剑气凌人,双目神光隐隐,盯在逾轮面上,目中隐隐带着惋惜之色。

逾轮上前一步,手中折扇轻摇,扇上美人似隐似现,逍遥的身姿中却带着孤傲意味。

那布衣儒士抱拳道:“宋先生可是误会了什么,在下并无恶意,只是想知道陆将军的情形罢了。”

逾轮冷冷道:“大将军生死,乃是朝廷之事,与你何干,不过是一介布衣,既未食君禄,又不是世家子弟,何必管这些闲事呢?”

那布衣儒士叹道:“先生此言差矣,两月来大将军陷入狱中,南楚上下,皆为之忧心,不仅文武官员纷纷上书保奏,就是布衣士子也纷纷为之鸣冤,国家兴亡,怎说不干我们的事情,先生无心富贵,浪迹风尘,我闻先生为人,也是心中敬重,为何却不肯相告实情,莫非一心维护那误国奸相么?”

逾轮冷笑道:“阁下却是自欺欺人,大将军虽然有功于国,却是秉性忠直,南楚世家和文武官员敬他的多,忌他的更多,你看那些上书鸣冤的可有几个是三品以上的官员,就连他的心腹部将又如何?杨秀沉默不语,不过是上了几封奏折辩解,更是一手揽去淮东军权,暗中和尚相结好。石观不仅将自己的女婿交给了尚相,更是甘心攀附权贵。余缅倒是想要出兵,可惜容渊铁索拦江便将他逼了回去,有始无终。还有那个容渊,原本也是忠臣名将,如今却连上三封奏折弹劾大将军,最后一封更是直接指斥大将军通敌,以至南楚叛臣死里逃生,襄阳失而复得,这两条罪名更是狠毒,说大将军欲在江淮称王,不过是没影儿的事,这两条罪状却是解释不清的。不提这人,如今南楚这些权贵世家,谁不是想着害死大将军,好抢夺他留下的兵权。纵然有你这样的人物为大将军费心,可是又有什么用处?阁下也不过能够欺宋某孤身一人罢了,就是宋某告诉你大将军被囚之处,你有什么本事救他出来?”

那人沉吟未语,暗处之人却是按耐不住,走到灯光下冷冷道:“你这等浮浪子弟怎知道大将军心胸,若非是大将军压制,只怕南楚已经是烽火四起,只是若是大将军真的被害,只怕那些忠心大将军的将士就再也不能忍耐了,只要你说出大将军被囚何地,我们绝不为难你。”灯光下看的分明,这后来之人却是一个黄冠道士。

逾轮冷冷一笑,有意无意地折扇轻摇,似乎要继续和那道士争辩,岂料折扇开阖之间,一道乌光突然从扇骨中射向黄冠道士的咽喉,这一下突如其来,那道士想不到逾轮出手竟会这样狠辣,促不及防,眼看那暗器就要取了他的性命,不料剑光一闪,那道乌光被击落一旁,那布衣儒士手持长剑,眼中皆是怒色,道:“你如此手段,必是心狠手辣之辈,看剑。”声音未落,一道匹练一般的剑光已经袭到逾轮面前。

逾轮闪身飞退,手中折扇摇动,将剑势挡去,剑扇相交,逾轮面色微变,这布衣儒士的内力平和深厚,强过他许多,一剑已经险些让他失去折扇。探出敌人深浅,逾轮便展开身形,只是四处游走,寻机出手,那布衣儒士的剑术光明正大,守得森严,攻得稳健,便如名将率正兵攻城略地,毫无缝隙可言,逾轮心中发愁,这样的剑术对付他刺客一流的武功,最是合适不过,除非是自己趁他不备,否则很难有得手的机会。逾轮心中烦恼,那布衣儒士也是心惊不已,这青年的武功诡谲狠辣,游走于自己的剑势之中,挥洒自如,可是只要自己稍露破绽,他便如鬼魅一般袭向自己的要害,只斗了几招,那布衣儒士心中便生出异样的感觉,这个青年必是双手沾满血腥的杀手身份,否则不会有这样的身手和杀气。不过这儒士心中虽然有些不安,剑势却是越来越沉稳。

两人交手不到百招,虽然表面上平分秋色,但是逾轮隐隐觉出自己的武功已经被对方的剑法压制,心中生出强烈的杀意,索性施展开两败俱伤的招式,不惜生死,也要和那剑士一决,不知怎么,他心中隐隐觉得,柳如梦若是出卖自己,十有八九定是为了此人,所以越发对他生出恨意。

那儒士眉头深皱,他得到情报,这个宋逾知道许多自己想要得知的消息,而且此人出入都是形单影孤,性情又颇为高洁,应该可以用情义动之,所以才来相询,想不到这人不知为何竟然动了拼死之心,虽然自己终会取得胜利,可是若是杀死这人,一来失去了探听消息的机会,二来也会打草惊蛇。心念轻动,他皱眉道:“宋先生,若再不肯住手,只怕在下兄弟就要得罪了。”说罢,连展剑势,将逾轮迫得越发窘迫,连连后退。就在逾轮退出第三步的时候,那黄冠道人飞身而起,手中显出一柄拂尘,径自向逾轮后心点去。这两人心有默契,只想点了这青年的穴道将他制住。岂料逾轮似乎早有所料,就在那道人堪堪点到他背心重穴的时候,他的身形仿佛狸猫翻转过来,竟是不顾长剑穿心的厄运,手中折扇射出三缕乌光,道人料不到他竟会和自己拼命,眼看即将死在暗器之下,不由一声怒吼。

就在这时,寂静黑暗的夜色中传来三声裂帛一般的琴音,仿佛来自幽冥的利刃一般,穿越十几丈空间,逾轮射出的乌光竟然从中折断,与此同时,布衣儒士手中的长剑和黄冠道士手中的拂尘都是被无形之力震得一偏,只是毫厘之差,已经避免了两败俱伤的惨剧,一时之间三人都是惊得呆住了。

这时,从暗中走出一个黑衣青年,面上蒙着黑纱,走到近前躬身一礼道:“宋公子,多有得罪,请看在素日相识的份上不要见怪。”

这人虽然蒙着面,可是逾轮却是一眼便认出他的身份,面上露出惊疑之色,忐忑不安地道:“这是怎么回事?白,白兄。”

那人一揖道:“请宋公子恕罪,丁大侠欲为大将军尽力,无奈不知囚所,难以下手,而且若非昏君奸相下手谋害,也不便擅自出手搭救大将军,为了得到准确的消息,丁大侠和阁中有旧,故此相求,阁主知道宋公子可能知晓内情,为了大义,不得不违背昔日承诺,指引丁大侠来寻公子,若有得罪,尚请见谅才是。”

逾轮面色数变,眼中渐渐清明,望望眼前旧日同僚,又向黑暗中望了一眼,欲言又止。

那人又上前道:“宋公子,你和阁主本是旧日相识,阁主也知违诺相烦,未免过分,可是还请公子看在陆将军乃是南楚栋梁,不容摧折的份上,畅所欲言。”

逾轮眼中闪过无奈凄苦之色,道:“我受阁主大恩,无以为报,纵然身死,也无所顾惜,既然阁主相询,在下知无不言,陆将军便囚在城中乔家废园,只恐数日之内,就会生死分明,我也慕陆将军为人,陆将军赴死之时,我定会亲自前去送行。阁主欲知陆将军生死,不妨留意在下行踪就是。”

那布衣儒士和黄冠道士都是大喜,上前拜谢,逾轮却只是冷冷一笑,不理不睬。这时候暗中传来几声琴音,隐隐有劝慰之意,逾轮心念数转,面上露出悲喜交加之色,也不敲门,纵身跃入柳园之中。继而暗中传来一缕箫音,声音凄楚,似有无限幽恨,转瞬消失在风中。

布衣儒士乃是知音之人,听出箫音隐含的惆怅之意,心中不由生出疑问,向那蒙面青年问道:“请问白兄,这位宋先生和天机阁有何牵扯,若是他有勉强之处,只怕大事会毁于一旦。”

蒙面青年笑道:“丁大侠不必担心,宋公子和本阁关系非浅,只是数年前已经退隐江湖,按照敝阁规矩便是再无牵扯,这一次阁主不得已毁诺,想来他心中不满,不过阁主待他恩重如山,他又是重情重义之人,只要阁主吩咐,他定不会相负的。”

布衣儒士放下心来,一揖道:“请代在下谢谢阁主大恩。”

那青年肃然道:“皆是为天下百姓尽力,何谈恩情,在下告辞,若有什么事情,请转告寒总管知道即可。”

说罢那青年悄然隐入黑暗之中,黑暗中琴音响起,有相别之意,片刻杳然。

布衣儒士面上露出倾慕之色,道:“天机阁主果然是世间奇人,若非得他相助,我们哪有可能相救陆将军。”

那黄冠道士面上露出疑惑之色,道:“天机阁主始终以白纱覆面,就连身形也隐在宽袍之下,丁兄真的肯定他就是我们在震泽湖上所遇之人?”

布衣儒士道:“相貌身形虽不可见,但是听他琴音,定是当日相遇的云公子,不过像他这样的人物,是绝对不会当面露出真相的,不过能够得他相助已经是苍天庇佑,我们也就不要追根究底了。”

那黄冠道士听了也是连连点头,却又忧心忡忡地道:“动手劫狱,终究是不臣之举,还是希望国主能够体念大将军捍卫社稷之功,若能下旨赦免,才是最好不过的。”

那布衣儒士喟然道:“只盼君恩如海,能够体念忠臣之心。”说完自己也觉得这是妄想,只得轻叹一声,隐入夜色之中,转瞬消失不见。那黄冠道士叹道:“君恩九鼎重,臣命一毫轻。当初王爷因此而死,大将军又凭什么能够幸免于难,我也是贪求了。”说罢,也随后没入夜色之中。

此刻的南楚深宫,赵陇看着尚维钧承上的密折,撇撇嘴,不过是杀个臣子,干什么这样慎重,又是深夜呈递,还要秘密赐死,明明是谋反重罪,却只将家人判了流刑,心中生出想要加重刑罚的意念,但是想了片刻,还是懒得多事,便批了一个“可”字,然后随手将折子丢在桌子上,迷迷糊糊地向后殿走去,那里还有等待他的美人呢。

\chapter{第三十九章 丹心坚似铁}

公坐系两月,尚相以襄阳事构之,令刑部主审,公坦然辩,诸官皆无言。尚相患之,转诬公长子云谋起兵救父,刑逼甚急,体无全肤,或谓云曰:“尚相必欲将军父子死,纵不肯屈,亦不能免,何妨虚应之,略免其苦。”云怒曰:“死且死矣,岂可留污名于世。”

狱不成,公部将皆得命,安抚军心,上书保奏而已,唯余缅闻公入缧绁,起兵欲救之,阻于江陵。尚相以此责公,公乃亲书劝之,余缅得书,黯然而退,尚相亦不敢加罪,虑公部将终为乱,欲赦之。

幕客宁谦闻之,阴劝尚相曰:“大将军在,诸将皆倚之,大将军殁,诸将眷属均在江南,又无首领,胡敢反。”尚相子承业亦劝之:“擒虎易,纵虎难,既已成仇,不可赦也,不然,我父子死无葬身之地也。”

尚相乃决,深夜入宫求密诏,国主不察,许之,乃以鸩酒赐公死,时年三十五岁,国中闻者皆哀痛,服孝私祭者不可胜数。

——《南朝楚史·忠武公传》

十二月七日,朔风飘雪,这一年江南的冬天倍加寒冷,建业城内一片萧瑟,在城内一隅荒废已久的“乔氏园”中,气氛更是冰冷肃杀,园中虽有十数处亭台楼阁,可是多半都是四处透风的破旧屋舍,冬日的寒风肆虐其中,纵然点起熊熊的火炉也不能逼退刺骨的阴冷。

在其中一间最为宽阔的楼阁之内,同样的冰冷阴沉,却连一个火盆也没有,寒风透过木板的缝隙吹入,令得房内宛如冰窟一般,可是居住在这里的男子却是宛似不觉,虽然身上只穿着一件灰色的半旧棉袍,但是刺骨的寒冷似乎并不能让他稍有瑟缩。而他的身上还戴着十余斤重的枷锁镣铐,稍一动作,便是叮当作响,手腕脚踝上更是有着红肿伤痕,可是这男子神色淡然,似乎浑不在意,目光流转之中,看到雪片丝丝缕缕从破损的窗棂飘入室内,这男子突然露出一丝笑容,走到窗前,伸手推开两扇残破的窗子,淡然望着飞雪如织的废园。任凭飞雪扑面而来,丝丝缕缕渗入衣襟发际之中。在他推窗观雪之时,不知有多少目光瞩目在他身上,直到发觉他并无异动,那些目光中才消去了警惕之色。

这时,门外有人轻咳一声,继而一个紫衣老者推门而入,在他身后则是一个青衫书生,一手提着一个食盒,另一手提着一个酒坛。那男子仍然目视窗外,毫不在意来人是谁。那紫衣老者见状心中生出敬佩之情,若是寻常人在这种地方拘禁月余,只怕已是奄奄一息,何况此人原本是大将军之尊,纵然不是锦衣玉食,又何曾受过这样的苦楚,可是这人却仍然是铁骨铮铮,不曾听他说过一个苦字,也不曾见他恶言向人。若非是相爷授意,恐怕自己也不愿这样折磨于他。那书生的目光望向临窗观雪的男子,眼中闪过复杂神色,将手中的食盒放在一旁,从中取出一席丰盛的佳肴,然后取出一个精美的银壶,和一只酒觞,倒了满满一杯放在桌上。那紫衣老者恭谨地道:“大将军,请用膳吧。”

陆灿转过身来,虽然数月囚禁,令他形容消瘦,面上也带了几分病容,但是双目却依然炯炯有神,全无英雄末路的悲凉之色。他望了一眼丰盛的酒食,目光在陌生的青衣书生面上掠过,笑道:“欧先生今日亲自来送酒食,又一改往常,非是寒透的囚粮,想必尚相已经有了决断,今日可是陆某陨命之时。”

紫衣老者欧元宁面上露出惭色,陆灿自下狱之后,也曾受过酷刑迫供,但是陆灿不肯屈招,朝野又有不满声浪,尚相便将他囚到乔氏园,改而向陆云迫供。尚维钧却也是心思狠毒,知道对于陆灿这等位高权重之人,一些不露声色的折辱更能够消减他的意志,虽然未必能够迫得陆灿屈服,但是能够折辱这位素来铁骨铮铮的大敌,也是心满意足,只可惜事与愿违,陆灿虽然受尽苦楚,但是除了目光越发淡然之外,竟是没有丝毫屈服之意。

欧元宁轻轻一叹,心中生出不安之意,道:“大将军目光如炬,国主已经下旨,今日便是大将军辞世之日,一个时辰之后,赐死诏书便会送到,尚相有谕,大将军乃是朝廷重臣,临去不可轻率,故令在下置酒相送。”

陆灿面上并无惊怒之色,看向宋逾道:“你是什么人?为何会在此地?”

宋逾一怔,料不到陆灿闻知大限在即,却无愤怒不平,反而还有兴趣问自己的来历,上前一揖道:“草民宋逾,与尚相公子乃是知交,闻听将军将去,故前来送行,且将军虽入囹圄,建业城中不知有多少人想要搭救将军,从前大势未定,这些人还不敢轻易动手,如今赐死诏书已下,难免会泄漏消息,尚相恐有人知大势不可绾,前来劫狱,故此令欧前辈亲来设伏,草民虽然武艺平平,但幸得尚相、欧前辈赏识,故此应命前来。”

欧元宁一皱眉,虽然宋逾所说并无虚言,尚维钧正是因为担心有人劫狱,才增加了许多高手守卫乔氏园,这宋逾正是因为这个缘故才来到此处的,可是却也不必毫无遮掩,侃侃直言吧。

陆灿听了却是觉得此宋逾性情直率,毫无拘泥之态,笑道:“即是如此,你就陪陆某小酌几杯,等候诏书前来吧。”

宋逾目视欧元宁,欧元宁心道,这宋逾功夫绝佳,有他在此,纵然有什么变故,也可先杀了陆灿,自己还需安排园中防务,凤仪门中人终究是外人,难以信任,还是自己亲自巡视一番的好。想到此处,他笑道:“大将军既然有此雅兴,宋逾理应从命。”说罢取出钥匙亲手替陆灿除去镣铐,道:“大将军请慢饮,老朽先下去了。”说罢给宋逾使了一个眼色,宋逾微微点头,欧元宁才转身走了出去。

陆灿除去镣铐,身上轻松许多,走到桌前举起酒觞,一饮而尽,道:“好酒,你也坐下吧,饮酒不可无伴,一个人未免太寂寞了。”

宋逾看了一眼屋内,取了一个缺口的茶杯过来,到了满满一杯酒之后,又替陆灿斟满一杯,举杯道:“能得大将军赐酒,草民荣宠备至。”说罢也是一饮而尽。

陆灿微微一笑,把酒啜饮,笑语从容,缓缓问及宋逾的身世经历,宋逾却也不隐瞒,除了身属秘营之事不曾外泄,就连曾为杀手的事情也是侃侃而谈。不过数语之间,宋逾便觉得眼前这位大将军和蔼可亲,言辞恳切,令人有如沐春风,如饮醇酒之感,陆灿却也觉得这青年虽然常有激愤消沉神色,却也是才华过人,问及军略,言语间颇有卓识,人品气度皆有可取之处,不由劝道:“宋公子才华过人,理应为国效力,怎能屈身草莽,沉沦风月,如今宋公子得尚相器重,理应从军报国才是,想来尚相也会首肯。”

宋逾目中闪过惊异,道:“大将军被尚相诬害,国主下诏赐死,难道竟然连一点怨言也没有么,竟然还要劝草民为国效力?”

陆灿淡然道:“我非圣贤,岂能无怨,但是怨则怨矣,陆某尽忠报国之心却不稍改,我死之后,尚相必定排挤打压陆某旧部,我见宋公子颇有大才,又得尚相信赖,若能领军上阵,倒也是国家之幸,将士之幸。”言罢,话语一转,却是说及自己从前领军作战的一些心得。

宋逾心中越发惊佩,想到自己秉承江哲之命,数次进言暗害,此人到了今日地步,自己难辞其咎,不由心中愧悔难当,耳中听见陆灿娓娓道来,竟有传授兵法之意,终忍不住拜倒在地道:“大将军如此厚爱,在下惭愧难当,陷大将军于死地,草民其罪非轻,何敢再聆教益。”

陆灿闻言有些惊愕,这青年虽然虽得尚维钧看重,但是恐怕并没有资格献策进言,如何这般说法?

见陆灿神色,宋逾越发痛悔,张口欲言,却想起自己纵然说给此人知道,也不过是伤口上洒盐,有害无益,神色一颓,道:“大将军且饮酒,草民在外恭候。”

陆灿神色一黯,道:“既然如此,你去吧。”他也是心思灵透之人,隐隐间已有所觉,见宋逾走出室外,他苦涩地一笑,举目望向窗外,不过些许时候,窗外飞雪越是迷离,随风飘舞,如幻如梦,恍惚间不由想起旧日往事,一桩桩,一件件,皆是难忘。

突然之间,雪影迷离之中,突然传来一缕琴音,琴音便如飞雪,千丝万缕,无孔不入,孤傲清冷,变幻莫测,陆灿只觉心神皆随着琴音起伏,气血上涌,心中一震,几步走到窗前,任凭雪花扑面,这才冷静下来,目光炯炯向园中望去。却见茫茫雪雾之中不时有血花飞溅,宛若红梅绽放,此起彼伏的厮杀声,惨呼声,和兵刃撞击的声音却随之而来,搅乱了这片静谧的雪景。

陆灿心知是有人前来劫狱,心中生出疑虑,所有旧部均得到他的严令,绝对不许来建业生事,会有何人前来劫狱呢,方才宋逾所言,他只当是尚维钧多疑,想不到竟然真的有人劫狱。仔细听去,只觉杀声从四面八方传来,进攻之人颇有章法,不似乌合之众,只是进展艰难,显然尚维钧在此地也是布下重兵,有意将来人一网打尽。陆灿心思电转,突然生出不祥的预感,莫非有人从中左右,欲令南楚豪杰皆丧身在此。唯今之际,只有自己出面,令那些来劫狱之人立刻退去,才能免去此劫。

想到此处,陆灿跃出窗外,纵身向杀声最响之处而去,此刻他除去枷锁,虽然元气因为数月囚禁而大伤,但是却仍然身手矫健。岂料他刚刚落入雪中,便有一人挡在他面前,一柄折扇忽开忽阖,挡住他的去路。陆灿望向那神色冷厉的宋逾,喝道:“让开,本将军绝不能让我南楚俊杰自相残杀。”

宋逾心中虽然佩服陆灿这般快就看出其中玄机,更没有被求生之念蒙蔽,但是想到自己得到的严令,就是将陆灿留在此处,绝不能让他阻止这注定两败俱伤的惨剧,目中闪过厉色,道:“草民奉命,不许大将军离开此间一步,国主诏书到此之前,还请大将军就在房内饮酒,外面的事情,却不需大将军费心。”

陆灿眼中寒芒一闪,叱道:“你究竟是楚人还是雍人?”

宋逾心中一颤,却昂首道:“宋某生于南楚,长于南楚。”

陆灿却是识破他话中隐含之意,冷笑道:“可是你却不当自己是楚人,可对,若非如此,你为何阻拦陆某平息干戈的好意。”

宋逾心中一横道:“大将军若是此刻前去,必定难逃毒手,若是留在此地,若是来人得胜,大将军尚可生还,岂不是两全其美,何必自寻死路。”说罢挥扇攻去,陆灿对于这种江湖技击之术,并不精擅,被宋逾困住,不能脱身而去,心中越发生出寒意,想到自己纵然舍身一死,也不能免去内乱之祸,拳掌之间,越发生出拼死之念。

数十丈外,欧元宁立在雪中,双手紧握,对着那白衣蒙面,端坐抚琴的身影,眼眦欲裂。就在片刻之前,袭击突如其来,欧元宁几乎是眼睁睁看着这人势如破竹,破众而入,幸而此人似有独来独往的意味,只是他孤身一人冲进乔氏园中。欧元宁令众人拒守,自己亲自追来,岂料那人竟然如此狠辣,留守在园中的十余侍卫都被这人轻易取了性命,更可恨的是,这人居然坐在雪中抚琴,琴音便如利刃,声声似乎要割断自己的肝肠,地上的伏尸之中便有他两个弟子,本是青春正盛,如今却已经惨死在眼前。欧元宁屡次想要出手,但是明明见那白衣人坐在雪上抚琴,全无防备的模样,却觉得那人周身上下,全无破绽,自己全无把握,不由心中大恨。欧元宁一边思索着这人到底是谁,江南从未听过有这般高手存在,一边寻找着出手的机会,心头越发郁闷,目光一闪,忽然发觉周围丈许方圆之内的雪花都随着琴音舞动,和数丈之外的飞雪变化迥异,顿时明白过来,那人的琴音已经结成罗网,将自己锁住,若是自己再不出手,便是唯死而已。

心中生出死志,狂啸一声,欧元宁身上劲气潮涌,那些诡异的雪片霎时间四散飞扬,顿时觉得身上压力一轻,再不犹豫,一掌击出,向那白衣人扑去,掌风激荡中,雪花飞溅,那人一声长笑,舍琴而起,起身迎上,欧元宁耳中传来一个若有若无的声音道:“琴音伤敌的功夫终究还是未成,就看你这老儿可以接我几招吧!”声音未息,欧元宁便觉得那人一掌到了眼前,长袖飞舞中,一只白皙的右手隐在袖中,欲发不发,这等后发先至的本事,也令欧元宁一惊。轰然一声巨响,双掌隔着那人衣袖相交,那人衣袖便如片片蝴蝶一般碎去。欧元宁只觉得那人内力虚无飘渺,这一掌似乎击在空处,那人却也惊咦而退,道:“好个绵掌,似阴柔实刚强,一掌竟有九重力道,不愧是绵里藏针。”

欧元宁心中略定,这人武功虽然匪夷所思,但是却未必强过自己多少,只不过他武功古怪,身法莫测,所以才令自己一时失措,落了下风罢了,此刻心中有数,信心大增,便又向那人攻去。耳中隐隐传来宋逾的声音,想来正在阻拦陆灿,若是自己失手,让这人救走陆灿,岂非是大祸临头。想到此处,他全无隐晦,倾力向那白衣人攻去。

这一次交手却是和方才不同,竟有平分秋色之势,其实那白衣人虽然境界见识都胜过欧元宁,但是欧元宁内力精深,老而弥坚,此消彼长,白衣人想要取胜也不是一件容易的事情。掌风拳影激荡之中,飞雪随之飘舞,两人的身影纠结在一起,除了欧元宁的紫衣尚可看见一线影子之外,那白衣人身影早已和飞雪交融在一起,不分彼此。

雪影迷漫之中,白衣人耳中传来错落有致的哨音,心中一惊,知道随自己来攻的江湖豪杰已经伤亡过半,自己不能再和这老者纠缠下去了,深吸一口气,本来扑向欧元宁的身形突然生生停住,凌空一掌,飞雪扑面而来,欧元宁一愕之间,便看见雪花中金星隐现,竭力闪去,却是已经躲避不及,只觉肋下剧痛,伸手摸去,只觉鲜血泉涌,这时,那白衣人袖中突然飞出一道黑影,宛似蛟龙旋舞,瞬间缠住欧元宁脖颈。欧元宁大喝一声,心恨这白衣人无耻暗算,不顾生死扑去,一掌拍去,这一次他拼上了全力,白衣人也是未能完全闪开,那一掌拍在白衣人肩上。白衣人趁势后退,便如流星闪电一般,欧元宁为长鞭所拽,只觉呼吸不畅,也是被向前拖去,那人后退不过数丈,已经到了一棵大树之下。欧元宁心中大喜,也顾不得颈上鞭索越发收紧,拼尽全力一掌向那白衣人击去,岂料那白衣人身形急停,贴着树干径直而上,飞身掠过横枝,急急坠落。霎时间化动为静,欧元宁高大的身躯在风中摇曳,四肢软软垂下,颈骨折断,竟被生生勒死在园中树上。

那白衣人一声轻咳,掀起面纱,一口鲜血吐在雪地上,嫣红如同梅瓣,他叹息道:“此人果然是好对手,只可惜我没有时间和你好好切磋,这般死了想必你也不会甘心吧。”说罢收回长鞭,欧元宁的尸首坠落在地上,激起雪尘漫天。那白衣人走回原处,抱起几乎被积雪掩盖的古琴,看也不看倒在周围横七竖八的尸体,举步向园内走去。

陆灿只觉胸中血气上涌,气喘吁吁,这些日子以来的折磨,让他再也无力和这青年宋逾相抗,不过百余招,他便已经不能支撑,见这青年依然是神采奕奕,他不由轻声一叹,退出战圈,倚在墙壁上,道:“你究竟是何人,若真是尚维钧心腹,现在就应该杀我才是,看你并无杀意,莫非真如我所料,你竟是雍人细作。”

宋逾淡淡道:“大将军过虑了,我非是雍人细作。”口中说着轻描淡写的话语,他的目光却仿佛透过无尽飞雪,看向那不可测的深处。

这时候白衣人已经到了近前,他的目光在陆灿身上一掠而过,在宋逾身上停留了一瞬,宋逾心中一颤,悄然退到陆灿身后,虽然不知道这人是谁,可是他却知道此人既然能够冒充天机阁主,必然是先生知交心腹,所以不由心中惊惧,此刻反而是陆灿更能够令他安心。那白衣人却是不曾说些什么,身影忽然疾退,转瞬消逝在飞雪中。陆灿目中闪过惊疑,回头看了宋逾一眼,见他神色沉默中隐隐有些不安,陆灿心中微动。

乔氏园之外,率众阻拦前来劫狱的义士的,除了尚维钧的心腹武士之外,还有一些劲装女剑手,她们的首领有两人,这两人都是轻纱覆面,一人华衣盛妆,一人青衣素服,剑气如霜,往来纵横,进攻一方,不知有多少人死在她们手中,直到丁铭以一人之力拦下这两人之后,才稳住了局势。丁铭很快辨认出了这两个女子的剑法,凤仪门在江南数年,丁铭也见识过她们的剑法,不过今日一战,丁铭才真得见识到了凤仪门的厉害。两个女子双剑合璧,剑势宛然游龙惊鸿,纵横捭阖,华美狠辣,若非是丁铭也是剑术高手,当真是难以匹敌。

战了两刻时间,丁铭发觉自己这一方死伤惨重,若非是仗着在吴越战场磨练出来的战阵,对着这些豪门鹰犬,还真是难以取胜,而且现在敌方援军未到,一来是乔氏园偏远,二来是禁军中也多有敬重陆灿之人,被丁铭安排的人手暗中说服劝阻,故意拖延,但是时间若是太久了,只怕就不能阻住援军了。就在他心焦之时,一个白衣人从园中缓缓而出,也不见他如何动作,便身如飞絮一般,飘向那华服女子身后,一掌击去。那女子觉出身后掌风如利刃,倾力闪躲,虽然避开这一掌,但是再也不能和同伴联剑对敌,那青衣女子原本专心致志地和同伴联手,这一下却是露出了大大的破绽。丁铭一声轻叱,剑如流虹,血光飞溅,那青衣女子娇躯一抖,鲜血瞬间渗透了衣衫,仆倒在地。

丁铭毫不犹豫,身剑合一,接着飞身向那华服女子扑去,那华服女子见到同伴委地,一声惊呼,转身逃去,但是丁铭这一剑摧枯拉朽,一去不回,竟是生生刺入那女子背心。那华服女子一声痛呼,反手一剑,便如电闪一般,丁铭只觉眼前剑光一闪,那一剑已经奔心口而来,他弃剑急退,那剑势却如附骨之俎一般,眼看就要刺入他的心口,却是嘎然而止,竟是一条黑色长鞭缠住了剑身。丁铭松了一口气,顺着长鞭看去,却见正是天机阁主出手相救。这时候,那华服女子娇躯才缓缓倒在地上。丁铭心中一寒,心道,只看这濒死一剑,这女子的剑术其实不弱于自己多少,若是她肯鼓起勇气和自己交手,绝不会败得这样快的,凤仪门的女剑手果然名不虚传。

丁铭心中正在胡思乱想,耳中传来裂帛一般的琴音,他神思一震,却见那白衣人指着园中,虽然看不到神情,却明显流露出不豫之色,丁铭不由有些惭愧,也顾不得外面还在缠战,跟着那白衣人向园内奔去。临来之前,有约在先,丁铭需要去劝陆灿答应和他们离开建业,只是被阻在外面许久,丁铭几乎忘记了这件事情,连忙过去拔起长剑,转身向园内走去,那白衣人目光一闪,看外面仍是相持之局,便随之走入园内。

在丁铭随着那白衣人走入园中的时候,凤仪门的女剑手已经看到两位首领倒在地上,两个劲装女子抛下交手的敌人,仗剑奔了过来,那华服女子已经浑身冰冷,没有气息,那青衣女子却只是昏迷了过去,当两人匆匆给她裹伤服药之后,那青衣女子终于缓缓醒来,她的目光在那华服女子身上停了片刻,眼神中满是哀痛和绝望。一个劲装女子低声道:“七姑娘,要不然我们赶快退走吧。”话语中满是惧意。青衣女子摇头道:“我们已没有回头路可走,先将二姐的尸首抬到边上,你们都去,别放过一个来犯之人,施放二姐身上的求援信号,召城中弟子前来相救。”那女剑手闻言泪落,走回那华服女子身边,从她身上取出一个桑纸包裹的小球,震腕向空中投去,那小球受到震动,火花飞溅,从中分裂,一道火焰冲天而起,在半空中化成一只彩凤模样,更是发出凤鸣也似的声音,惊彻寒夜。青衣女子微阖双目,珠泪滚滚而下,低声道:“二姐,三姐,你们都这样去了,我为何还要这样辛苦地活着。”寒冷渐渐袭来,青衣女子的意识缓缓散去,珠泪已化成两行冰霜,凝在如美玉一般的面颊上。

陆灿立在雪中,尽管身上已经积雪甚厚,他却没有拂拭的意思,宋逾站在他身后,似乎是保护,又似是监视,听到耳中隐隐传来的厮杀之声,陆灿心中觉得茫然,知道自己已经不可能阻止眼前的血战,陆灿便静静地等待着结束的时刻,也等待着赐死诏书的到来,只要自己留在这里,那么无论什么人的阴谋,都不能顺利展开。

过了片刻,果然见到两人踏雪而来,其中一人走到近前便下拜道:“丁铭叩见大将军,请大将军随我们出城,城外有甲士接应,已经备好车马,沿途都有护卫,便可直奔军中。”

陆灿的目光只在丁铭身上一扫而过,却是看向一身白衣,面覆白纱,就连眼睛也用轻纱遮住的那人,淡淡道:“阁下是何人,为何参与此事。”虽然飞雪障目,可是陆灿也知道若无此人杀了欧元宁,丁铭等人绝对不可能闯入园中,所以方追问白衣人的目的。

丁铭心中一惊,担忧白衣人恼怒,岂料白衣人只是淡淡道:“丁兄与我有旧,苦苦相求,我便出手搭救,否则大将军纵然有功于社稷黎民,又与我们这些江湖草民有什么相干。”

陆灿闻言却觉得心中一宽,心道,他若不是存心来救我,倒也不虑他有什么阴谋。转目望向丁铭,他叹道:“丁大侠何必如此费心,陆某生死无关紧要,你却是吴越义军的首领,若是有所闪失,岂不让定海占了便宜,你还是速回吴越去吧,不要牵涉这些朝廷大事。”

丁铭高声道:“大将军此言差矣,丁某不过是个江湖人,我若死了自有别人可以统领义军,可是若无大将军指挥若定,如何可以抵御雍军铁骑,大将军岂能坐视雍军南下,甘心被那奸臣所害。”

陆灿苦笑道:“丁兄,你是一片好心,只是陆某生死已经无关紧要,纵然我可以逃出建业,也将成为叛逆,到时候尚相必然下令清洗我的旧部,南楚内乱将起,丁兄难道要我率军谋反么?与其引起内乱,自相残杀,不如陆某服法而死,有诸位义士舍身为国,南楚尚可平安无事,再过些年,或有更胜陆某的人能够北上中原,令雍军从此不能南下。”

丁铭听得泪落,道:“大将军为国为民,鞍马劳顿,舍生忘死,今日仍念着社稷百姓,那奸相所为实在是令人发指,大将军若是离开建业,避入军中,再上书求赦,或者也可免去内乱,大将军若是不走,我们情愿死在这里,也不肯这样离去。”

陆灿微微一笑,道:“陆某一人生死事小,家国安危事大,尚相必然已经在陆某旧部之中安插了刺客心腹,一旦陆某脱逃,只怕他们都会遭到戕害,而且军中士卒的家眷都在江水之南,一旦尚相疑心他们谋反,他们便是家破人亡的结局,岂可为陆某一人,害了麾下这些将士。丁兄不要再多说了,你去吧,陆某是绝对不会逃出建业的。”

这时,那白衣人冷冷道:“何必这样废话,将他打晕了带走就是。”话音刚落,只见陆灿幽深双眸中射出寒光,原本平和淡凝的气势瞬间变得酷厉凌人,那是一种沙场血战中养成的可以匹敌千人万人的大将气度,而他面上的神色却是那样淡漠,双手背负而立,陆灿冷冷道:“阁下当真以为凭着武功高强就可以为所欲为么?”

白衣人心神一颤,目光透过轻纱,在陆灿面上凝注片刻,见他眉宇间皆是宁为玉碎,不为瓦全之意,轻轻一叹,道:“大将军不欲令南楚内乱,却只是梦想罢了,无论如何,这内乱都是不可免的,大将军只需答应一声,我必然可以带着大将军离开建业,到时候不论是回到军中起兵,还是远遁江湖逍遥,我都可以实现大将军的愿望。大将军难道就不为家人着想么,覆巢之下,焉有完卵,纵然大将军甘心赴死,尚维钧也绝不会放过大将军的家人。”

陆灿的目光没有丝毫软弱,白衣人的言辞虽然犀利,却并未在他心湖之上留下印痕,这一切早在十年之前,他就已经想得清清楚楚了。他却也不辩驳,只是露出坚定淡漠的微笑,然后举手,食中二指便如利刃一般刺透了胸膛,鲜血涌出,虽然手指只刺入了一分,并未伤及要害,可是他的意思却是明明白白。

顾不得惊讶陆灿的指力,丁铭几乎是立刻起身退去,连退了十余步,目中满是悲恸,颤声道:“大将军,丁铭遵命就是。”

陆灿淡漠的目光望向白衣人,白衣人目中光芒闪烁,陆灿微微一笑,指上用力,鲜血泉涌而出,白衣人能够感觉到丁铭恳求的目光,他也知道若是立刻出手,或者可以阻止陆灿自戕,但是陆灿心意已决,纵然是救了出去,结果也不会有两样,更何况若是任他背负叛逆之名死在外面,还不如让他死于此处,也算是全了他的忠义。更何况那人原本就说过,要自己给陆灿留下选择的余地。轻轻一叹,白衣人的身形隐入雪中,就如来时一般无影无踪。

陆灿心中一宽,知道局势终于已经在自己控制之下,望向丁铭,他淡淡道:“丁兄去吧,不要再多添伤亡,切忌不可自相残杀,徒令雍人快意,更要留心身边之人,雍人最擅用间,你要小心在意。”他心中虽然也想警告丁铭小心身后的宋逾和那来历不明的白衣人,但是却也知道若是自己说得过分明白,只怕丁铭也不能生出建业,与其如此,不如让他心存警惕就好,也免得吴越义军失去领袖。

看着丁铭掩面而退,飞雪之中突然传来一缕琴音,琴音凄楚,隐隐有诀别之意,陆灿心中突然生出一个古怪念头,这琴声自己必然听过,或者不是这首曲子,但是那琴中深藏的孤傲清冷意蕴却是一般无二,想到此处却是不由失笑,自己对于音律并不精擅,怎能听出琴音异同。将手指拔出,任凭鲜血滴落,拂去身上积雪,陆灿走入室内,倒了一杯酒,举杯道:“了却君王天下事,赢得生前身后名。可怜白发生。只可惜我还没有完成心愿,就要身名俱裂。宋逾,你为何不一起走,莫非以为我没有看穿你的伪装么?若非你是他们的内应,只怕那白衣人或者丁铭先就要杀了你。”

宋逾淡淡道:“大将军何出此言,宋某奉命守护大将军,力阻大将军离开此地,后来也是大将军求情,才令那些人没有下手杀我,大将军舍生不逃,想来也是顾念在下克尽职守的缘故,才多有眷顾吧?”

陆灿听了不觉失笑,也不顾鲜血流淌,举杯道:“说得好,你这般才智气度,倒是难得,说吧,你和我的恩师江哲有何关系?想来也只有先生能够作出这样的事情,将陆某的生死利用的这般彻底,你这般人才,只怕也是先生的门人吧?”

宋逾神色微动,看向陆灿磊落的神色,低声道:“我是先生不肖弟子,早已经叛出门墙,承蒙先生开恩,不曾取我性命,今次奉命数进谗言,加害将军,于心有愧,将军纵然将此事说了出去,我也不怪将军。”

陆灿轻轻皱眉,道:“我听你语气似有怨恨,莫非你怀恨先生,可是若是这样,你又为何奉他之命行事呢?”

宋逾目光向外扫去,方才凤仪门的求援信号他也已经看到,知道很快就会有人进来查看,便低声道:“我和先生本有旧怨,只是先生不知,但是仔细想来,却也怪不得先生,又蒙先生恩德,同僚厚谊,所以不能拒绝先生的命令,只是却害了将军,我心中十分不安,将军为人忠义,性情又如光风霁月,逾轮此生也觉痛悔难当。”

陆灿叹道:“这也不关你的事情,先生不过是火上添油,纵然没有他的计策,再过数年,也免不了这一劫,只是原本我以为可以先完成北上中原的夙愿,令雍军铁骑不能窥伺江南,只恨这一日终究来得太早了。我现在才明白,当日谷城之上,先生抚琴一曲,非是为了退敌,而是为了诀别,一曲之后,再不复见,这才是先生的意思。”

这时,宋逾耳中已经传来足音,他连忙轻咳一声道:“将军,要不要裹一下伤势?”

陆灿目光一转,道:“你今后还要留在建业么?”

宋逾心中明白,低声道:“此事已了,在下再无牵挂,绝不会再涉入南北之争。”

陆灿微微一笑,点头道:“那就好,我相信你并未虚言,否则纵然是你对我这般诚恳殷切,我也只能取了你的性命了,想来我若说上几句话,尚维钧还是宁可信其有的,若是再见到先生,请替我说一句多谢。”

宋逾低声道:“多谢大将军宽宏,若有机缘,必定转告。”正想再说些什么,眼角余光看见身影闪动,他默然不再言语。

这时候,援军已经进了园中,走在最前面的却是尚承业,他身后皆是带甲军士,想必是亲自带着援军前来乔氏废园,毕竟陆灿的生死,和他们父子的关系最是密切。在尚承业身后,便是几个绯衣内侍,手上捧着圣旨鸩酒,却是路上相逢,一并赶了过来。一眼看到陆灿坐在那里饮酒,尚承业便松了一口气,停步不前,看了一眼宋逾,眼中露出赞赏之色,示意他退出来。

宋逾掩去眼中悲色,走出房间,站到尚承业身后,只见那绯衣内侍尖声宣旨,宋逾神思不属,恍恍忽忽只听见“赐死”、“弃市”这样的字眼。然后透过洞开的房门,他便眼睁睁地看着陆灿含笑倒了一杯鸩酒,明晰温和的目光环视众人,在自己身上更是多停留了一瞬,然后不顾前胸血迹斑斑,举杯而饮。宋逾眼中一片模糊,悄悄地退了一步,只觉得自己的生命仿佛也随着陆灿自尽而逝去了一般。

\chapter{第四十章 洒泪今成血}

公南归时,已知难免,尽遣心腹部将,尚相欲安将士之心,故殊少牵连,唯公长子云,判令弃市,籍公家赀,徙家南闽。公殁时,飞雪漫天,似彰公之孤忠,尚相畏人知,率重兵围乔氏园,有义士杀入,欲救公出逃,公拒之而死,忠义若此,而奸相鸩之,此诚天地不容。公既死,尚相不安,令缇骑即斩云于狱,使者至天牢,见狱吏军士皆茫然若梦,惊视狱中,则云已杳。公之爱妻幼子,并婢仆家将共四十六人,次日即南徙也。

——《南朝楚史·忠武公传》

同泰十四年,忠武公殁于建业,主淮东军事,参军杨秀闻凶讯,设祭帐于军中。哲闻之,悲恸欲绝,曰:“皆我之罪也。”乃着素衣,渡淮水祭之,诸将皆知其设计害忠武公死,欲杀之,哲欲祭而后死,诸将乃许。哲奏琴灵前,众将闻之皆泣下,不能举刀,哲乃还楚州。

——《南朝楚史·江随云传》

丁铭等人离开乔氏园,早有人暗助逃出城去,到了城外数里,风雪之中显出一行身影,却是百余骑士护着一辆马车,这些骑士都穿着没有标记的衣甲,彪悍威武,显然是百战余生的猛士,为首的是一个青袍将领,面上覆着青纱,见到丁铭身影,他眼中先是闪过喜色,但是目光一转,却没有看到那熟悉的身影,喜色变成了失望。

丁铭快步上前,对他青袍将领一揖,悲痛地道:“大将军不肯随我等出城,只怕如今已经……”话音未息,已经是落下泪来。

那青袍将领闻言默然,良久才道:“大将军性情我素来知道,只是也不免抱着万一之念,如今事已至此,你们已经尽了全力了,我不能离开军中太久,只能立刻赶回去了。”

丁铭俯身拜道:“石兄高义,丁某佩服,淮西尚赖兄镇守,还是请石兄速行,日后若有所命,丁某绝不会推辞,纵然大将军殉难,南楚江山也不能容许雍军肆虐。”

那青袍将领叹道:“丁兄忠义之心,石某深铭五内,我得大将军厚爱,却不能救他性命,已经是惭愧至极,若是再不能守住淮西,除了一死,也没有别的法子赎罪了。”

说罢,那青袍将领告辞离去,一行人在风雪之中,策马远去,丁铭望着青袍将领苍劲的背影,心中涌起悲意,因为陆灿的缘故,这人他也是相识已久,两人一见之下颇为投缘,彼此更是引为知己。原本他也憎恨此人负义,只为了自己的地位官职,竟然将爱女女婿全都舍弃,可是这人却遣使请他赴建业搭救陆灿,更是不惜一切亲自接应,原本丁铭心中还有疑惑,可是建业城外相见之后,丁铭便相信这人非是虚情假意。擅离中军,这不是小罪名,若被尚维钧知道,最好的结果也是解去军职,可是这人全不顾及,想来他当日负义之举也是迫不得已的吧。

石观纵马在雪中飞奔,不知什么时候,泪水已经滑落,纵然是当日他狠心舍弃女儿,也没有落泪,当初陆灿尚未被召回建业,他和陆云便已知道局势不妙,两人暗中商议如何应对,石观在数年前就曾经忧虑这种情形,向陆灿提出谏言,当时陆灿便要求他纵然有什么变化,也不能为了私人情谊乱了军心大局,而陆云更是不惜一死,也不愿坏了父亲忠义之名,两人心意相通,却都是最担忧石绣。以石绣的刚烈,纵然石观能够保住她的平安,她也会不惜一死。无奈之下,石观便和陆云商量,石观故意迫使石绣保护陆梅逃走,再让陆云以弱妹和未出世的孩儿相托,这样一来,石绣就只能活下去,不能轻易殉夫。这样做法,即可保住陆氏一脉香烟,也可让石观得到尚维钧的信任。不料石绣却在去钟离的途中失踪,生死不明,石观暗中令人寻找,却始终不见女儿踪影,这已经令石观心痛不已。如今他违背陆灿心意,联合丁铭欲救陆灿脱险,却也功败垂成,再想到爱婿也断不能保住性命,怎不让石观悲愤欲绝。

一行人策马狂奔,视线为风雪所阻,又都是乍闻噩耗,心神振荡,不免失了几分警惕,就在石观策马经过一个弯道的时候,道路狭窄,前后的亲卫都错开了位置,防守严密的骑阵露出了空隙,正在这时,堆积成丘的积雪突然四散飞扬,一个白色身影凌空而起,手中寒芒乍现,那道匹练也似的寒光,便如天上的星河一般流光溢彩,生生的刺入了石观后心,石观一声怒喝,挥拳击去,掌风便如雷霆一般,那人硬生生受了一掌,却是一声不吭,趁势掠向雪中,后面的亲卫都是惊恐地大声怒喝,几乎是同时射出了夺命的箭矢,那人身形刚落在地上,便纵身向远处扑去,身形奇快,那快如流星闪电的数十支箭矢深深地射入了那人身后的地面上,第二轮,第三轮箭矢几乎是追着那人的身形,却都以毫厘之差错过,转瞬之间,那人身影已经消失无踪。这时,石观的身躯才缓缓倒下,被两个甩蹬离鞍滚下马来的亲卫死死抱住,其中一人颤抖着伸手探视,汗水泪水涔涔而下,忍不住高声痛呼道:“将军死了,将军死了。”

这些军士都觉得如同五雷轰顶一般,将军死在此地,不要说无法向军中同袍交代,就是对朝廷也说不过去,毕竟石观本不应该在建业城外出现的。充满杀意的目光向那刺客遁去的方向望去,一个为首的亲卫道:“一半人送将军回寿春,立刻送信给杨参军,请他设法到淮西主持大局,另一半人跟我去追杀那刺客,不报此仇,绝不回寿春。”众亲卫悍然应诺,迅速分成两拨,更是分出两人直奔淮东而去,转瞬之间,他们的支柱已经崩塌,此刻在他们心中,恨不得死去的却是自己。

此刻,石观的尸身静静躺在亲卫怀中,漫天的飞雪落在他惊怒悲愤的面容上,仿佛是哀悼着这位淮西军主将的猝逝,也像是哀悼着南楚又失去了一位大将。

和丁铭等人分手之后,那丁铭心目中的“天机阁主”却没有出城,而是径自返回天机阁在建业城内的隐秘住处,这是一座富商的宅邸,只是最后一进却单独辟出来做了天机阁的密舵。走入温暖如春的楼阁,白衣人轻轻一叹,换下已经狼狈不堪的衣衫,走进屏风之后,那里已经备有沐浴香汤和崭新的衣履。不多时,白衣人已经换了一身浅黑色的锦衣出来,英俊沉郁的面容上带着淡淡的倦意,倚在软榻上随手拿起一本琴谱慢慢看去,但是目光却有些涣散,看来并没有用心在琴谱之上。这白衣人,所谓的天机阁主,正是魔宗嫡传弟子秋玉飞。

当日他得到江哲传书,请他到荆襄一会,秋玉飞便知江哲定是有事相求,虽然对于江哲的请托,可以答应也可以不答应,但是念及两人的交情,秋玉飞自然不会拒绝,更何况途中他去拜见京无极,向他请教之时,京无极也有意让他到江南走一趟,所以秋玉飞才欣然而来。在谷城相会之后,秋玉飞才得知江哲竟然要他冒充天机阁主,这却令秋玉飞豁然开朗,立刻想明白了当初为何江哲会识破他的身份,也不由暗惊江哲的潜势力之大。为了一探天机阁的深浅,秋玉飞也就甘心做一次江哲的替身和杀手了。

不过只可惜江哲所托的第一件事情就没有成功,陆灿还是慷慨赴死了,而自己堂堂的魔宗弟子,竟在陆灿面前落了下风,这令秋玉飞心中郁闷的很,更何况见到陆灿这样的名将陨落,秋玉飞心中也不好受,想到昔日在北汉时眼见之事,越发深有感慨。放下琴谱,不由轻叹,江哲的手段也未免太阴毒了,不知道他是用了什么法子,让江南的武林中人自相残杀,想来天机阁从今之后必会推波助澜,令江南越发混乱吧。

不知过了多久,凌端闯了进来,面上满是喜色,一见到秋玉飞便道:“四爷,得手了,大概所有的高手都到乔氏园去了,天牢里面几乎没有什么防范,而且我们还使用了‘迷梦’,这种迷药可真是厉害,那些狱卒和军士明明还有知觉,就是懵懵懂懂,就像梦游一般。”

秋玉飞淡淡道:“那陆云有没有和你们为难?不会也不想离开天牢吧?”

凌端嘻嘻笑道:“我可忘了问他,反正他也中了迷药,我和白义直接就把他带出来了。”

秋玉飞微微苦笑,道:“我看你还是告诉白义一声,直接将他迷晕了事,将他交到随云手中再救醒也不迟,免得多生是非。”

凌端惊讶地道:“四爷真是有先见之明,我来的时候就听见白义让人去拿准备好的‘千日醉’,那可是能够让人睡上三年的好东西,想来白义是不会让那小子醒来吵闹的了。”继而有些疑惑地问道:“不过四爷怎会知道这小子不会顺服呢,莫非是已经有了经验,哎呀,难道四爷没有救出陆灿么?四爷不是说他若不答应,就直接打晕了事么?”

秋玉飞瞪了凌端一眼,冷笑道:“你现在的武功也不错了,若是现在见到你的谭将军,你可有胆子为了救他将他击晕?”

凌端打了一个冷颤,道:“这我可怎么敢,谭将军一双眼睛只要看你一眼,便会觉得从心里往外都是寒意呢。”

秋玉飞也懒得和他多说,道:“据说忠义之人鬼神不敢近,我不过是个寻常江湖人,可没有鬼神之力,陆将军尽忠全节,此诚为天下人所钦服,只是随云若是得知这个消息,恐怕还是要悲恸难当的。”

凌端见秋玉飞这般悲叹,却是心中冷笑,虽然对于江哲的怨恨已经消散许多,可是却不意味着他已经原谅了那人过去所做的一切。

或许是觉得心中烦乱,秋玉飞突地起身,丢下琴谱道:“我出去走走,你不要到外面生事。”说罢也不等凌端叫苦便走了出去,这时候夜色已深,雪下的越发大了,街上却处处可见禁军往来的身影,秋玉飞衣着华贵,在雪中缓缓而行,更是着意避开那些禁军,凭他的武功自然是轻而易举,建业城里面的混乱局势皆被他看在眼里,更是不由惊叹江哲的手段,虽然未能如愿救出陆灿,可是丁铭等人和尚维钧、凤仪门的仇恨是万万化解不开的了。入夜时分,雪势渐渐小了许多,已经可以隔着数丈看清人影,秋玉飞有些倦了,正想回去休息,目光一闪,却看到一个轻盈婀娜的身影在夜空飞雪中纵越,不由心中一动,悄悄跟了上去。几乎传过了小半个建业城,他看到那个身影没入了一座灯火辉煌的华丽庭院之中,听到院中传来的乐声歌声,熙熙攘攘的人声以及门前车水马龙的情景,秋玉飞眉头一皱,猜出这身影的身份。不过他可没有必要作些额外的事情,正欲转身离开,一缕琴音从一座楼阁之中传出。

秋玉飞脚步一凝,风尘女子抚琴悦宾是常有的事情,可是这琴音却大不寻常,竟是一曲《猗兰操》,幽怨高洁。秋玉飞细细品味,弹琴之人手法轻柔,曲中自怜身落风尘之意,便如香兰生于荒野,不得其时,不论是指法还是心境,都将此曲演绎的完美无缺。秋玉飞本是最爱音律之人,听得目放奇光,也不顾此地乃是敌人重地,便如一个寻芳客一般走入了月影轩的大门。

不需多费唇舌,凭着秋玉飞的品貌和重金,轻而易举地便走入了月影轩灵雨的香闺,刚刚在前厅献艺,便需待客的灵雨神情柔婉,灵秀动人的姿容,楚楚可怜的气质,都让人目眩神迷,绝不会后悔花了重金,却只能喝一杯茶,说上几句话而已。可是秋玉飞却能够感觉到灵雨眼眸中深藏的淡漠和倦意,这个女子,并不像她的身份所代表的势力那般跋扈,琴音舒心臆,或许她也是污泥中的一朵白莲吧。

心中存了这样的想法,秋玉飞完全抛却了来建业之前看到那份情报关于这个女子的评介,微笑道:“灵雨姑娘可以说是当世数一数二的琴师,不知道在下能不能再听姑娘奏上一曲呢?”

灵雨眼中闪过一丝惊诧,面容几乎是立刻之间变得生动起来,真正的仔细打量了秋玉飞一眼,心中一动,道:“四公子想必听过大家抚琴,不知道小女子的琴艺有什么缺憾之处?”

秋玉飞见灵雨一开口便是询问音律,心中越发觉得这女子不俗,若是说到音律,当世之间已无人可以胜过他,灵雨的琴艺虽然出众,在他看来也有可以推敲之处,当下便取过灵雨古琴,弹奏起方才那一曲《猗兰操》。

琴声一起,灵雨便是精神大振,凝神听着琴音变化,全不知晓,秋玉飞已经用真气隔绝了琴音,除了她之外,月影轩上下并无人能够听到琴声,毕竟秋玉飞还不想引起凤仪门的注意。

一曲终了,灵雨已经心中狂喜,便取回古琴,重新弹奏,秋玉飞见她如此痴迷,心中更是欢喜,索性站在她身后,不时指点她的指法和技巧。

等到灵雨完全贯通之后,已经是将近子时,若是平常,早有人前来促驾,可是灵雨并没有暗示逐客,而凤仪门上下正为惨痛的损失而忙乱,所以竟无人前来打扰,当然后来,秋玉飞也无需隔绝声音了,反正只有灵雨在练琴,若是那样做反而容易引起别人怀疑。

灵雨意犹未尽,正想继续请教,突然看到秋玉飞若有若无的笑意,才想起自己全然忘了这人乃是自己的客人,不由玉面通红,翩翩下拜道:“灵雨怠慢四公子了,公子精通音律,灵雨当真想随公子学琴,只可惜身不由己,不知道公子明日还来么?”

秋玉飞目光如炬,看出这灵雨姑娘纯然一片求教之心,不由轻叹道:“姑娘如此苦心孤诣,难怪能有这样的琴艺,只是在下即将离开建业,想来真是遗憾,不能和姑娘再次探讨琴艺。”

灵雨闻言,目中闪过波光,想到自己本是书香门第的小姐,无奈家破人亡,沦落风尘,又不幸成了凤仪门弟子,竟然连赎身的自由也没有,她身世坎坷,除了寄情音律之外再也没有别的意念,就是师父教她武功,她除了勤练内功,以便增强弹琴的力量之外,对于轻功剑法都是不甚用心,若非看在她的才貌和琴艺出众,只怕师父也不会继续将自己留在门下吧?原本庆幸可以摆脱清白遭污的厄运,如今灵雨却恨不得是个寻常女子,可以要求赎身,随着这琴艺更胜自己的四公子离去,可以自由自在的学琴抚琴。忍不住珠泪滴落,她一手拉着秋玉飞的衣袖,哽咽不能言,良久才道:“四公子既然要走,就让灵雨再为公子抚琴一曲。”

说罢,灵雨拭去泪痕,再次抚动琴弦,这次奏的却是一曲《高山流水》,这一曲本来是知音相惜之意,灵雨弹来,却是多了几分哀怨悲切,更有知音匆匆离别,自己却不能相随的恨意,灵雨全神贯注地弹奏完一曲,抬目看时,却见那俊逸多才的青年公子已经不见踪影,只在琴台上多了一块玉佩。

灵雨拿起玉佩,却是一块羊脂美玉雕刻成古琴模样,心中微痛,将玉佩按在心口,轻阖双目,泪水滚滚而下。他却不知,秋玉飞离去之时,却是心中暗道,只为了这个灵雨姑娘,我也要多留几日。原本秋玉飞已经准备即刻动身返回东海,可是此刻却下定了决心帮着江哲完成铲除凤仪门的大计,以他的聪明,自然看得出灵雨乃是被迫留在凤仪门罢了,并无选择的余地。

我坐在棋坪前,看着黑白分明的棋局,淡淡道:“石观竟然已经死了?是谁下的手?淮西军由谁接管了?”

霍琮闻言心中一寒,自从先生得知陆灿死讯之后,便始终是这般淡然自若的模样,似乎死去的只是一个不相识的外人,竟连一丝悲色也无,可是不知怎么,霍琮却觉得越发蹊跷,先生绝非凉薄之人,按理来说绝不会毫无所动,江哲这般模样却比放声大哭更加令霍琮忧虑。这时候江哲的目光已经向他望来,似在催促他回答,望着那双幽深淡然的眼睛,霍琮不由低下头去,低声道:“先生事前已经预料到石观非是负义之人,所以令司闻曹留意石观行踪,不过下手的却不是大雍刺客,而是凤仪门的燕无双,司闻曹借刀杀人,凤仪门的反应也很快,还不能确定燕无双是事先设伏,还是跟踪丁鸣寻到石观,但是燕无双居然在石观归途上暴起行刺,一举取了石观性命,石观亲卫舍命追杀,四十人全军覆没,被燕无双个个击破,不过燕无双也受了重伤,回到建业城后就卧病不起。至于淮西军的新任主将,乃是南楚王后兄长蔡群,此人乃是国戚,又得尚维钧信任,最重要的是,他和凤仪门关系密切,而且此人垂涎纪霞首徒灵雨已久,据说纪霞已经许诺,等到蔡群在淮西立足之后,就将弟子灵雨送给蔡群为妾。”

我若有所思地道:“蔡群此人才能如何,可曾领军作战?”

霍琮道:“蔡群虽然是世家子弟,倒也勉强算得上是文武双全,蔡氏倒是的确出了几个不错的子弟,此人倒颇有些高傲,在余杭任将军,能力中上,颇为胜任,只是性情高傲,又兼风流成性,赵陇亲政之后,他因为是国舅,而被诏回建业为禁卫军副统领。此人为淮西主将,若无大战,倒也胜任。”

我又问道:“尚维钧没有趁这个机会清洗淮西军?”

霍琮道:“行刺石观的事情想必尚维钧并不清楚,按照司闻曹得到的消息,石观的尸体被亲卫带回淮西之后,杨秀的信使就到了淮西,按照他的意思,淮西军以石观重病身亡的名义上报南楚朝廷,尚维钧也不愿惊扰军心,多生是非,对他来说,石将军死了最好,免得留下后患。”

我叹道:“这也好,若是石将军死在司闻曹的秘谍手上,将来若是见到云儿夫妇,也不好交待,不过燕无双果然狠绝,当年她便是除了闻紫烟之外,凤仪门弟子中最擅长刺杀的一人,现在看来她的武功有进无退,幸好如今她已经重伤,这样一来我们铲除凤仪门的时候就容易多了。对了,乔氏园一战,伤亡如何?”

霍琮偷偷的瞥了一眼江哲,只见先生依然是若无其事的样子,可是站在一边的小顺子神情却是罕见的凝重,犹豫了一下,他说道:“乔氏园搭救大将军,按照先生的意思,除了四公子之外,我们的人只是暗中协助,这一点已经得到丁铭等人的谅解,所以我们并无伤亡,尚维钧的心腹第一高手欧元宁被四公子缢杀,凤仪门萧兰、谢晓彤阵亡,参战的剑士死伤过半,尚维钧的势力也是损失惨重,丁铭带来的吴越高手也只有三成生还,而且白义师兄趁机救出了陆云,这一次先生的目的已经全部达到。事后尚维钧大怒不已,凤仪门果然趁机撺掇尚维钧利用陆夫人和陆霆等人南徙的机会,故意放出风声,要在途中杀害陆氏满门,准备将同情陆氏的江湖中人诱入罗网,然后一网打尽,不过白义师兄本来想要逾轮师兄向尚承业进言的,却被逾轮师兄拒绝。”

江哲点头道:“当日不救陆氏满门,一来是人太多,难以相救,二来也怕陆夫人和陆灿一样的忠烈,反而会让我们的人陷入泥潭,三来我也是断定凤仪门会如此做,这一次凤仪门先后损失了三大高手,必然痛彻肺腑,若不利用机会削弱江南武林,也就不是凤仪门了,事先我便说过一定要杀死凤仪门一两个高手,他们倒是做的超出我的预计。对了,让他们把这个消息透漏给韦膺,不论他是继续和凤仪门同流合污,还是改弦易辙,继续忠于陆氏,都不能让他置身事外。”

霍琮疑惑地问道:“先生,弟子不明白为何要在这时对付凤仪门,凤仪门成事不足,败事有余,弟子认为若是任其所为,反而有利于我军南征。”

我冷冷道:“从前南楚有陆灿独撑大局,那么凤仪门的存在自然是我军最好的助力,如今陆灿已逝,尚维钧一手掌握大权,若得凤仪门相助,便可掌控将帅,铲除异己,陆灿虽然已死,可是他临去余威尤在,众将敬他忠义,不敢起反意,尚维钧便可以顺利掌握权柄。如果凤仪门毁去,尚维钧的实力又大减,不能威胁南楚将帅的安危,陆灿旧部以及其他将军都会为了自保各自保留实力,这样我大军便可横扫江南,所以凤仪门已经不该存在这世上。传令陈稹,让他设法让江南武林的自相残杀越演越烈,然后联合司闻曹将他们斩尽杀绝,凤仪门尤其不能放过,不过那些秉承忠义的江湖势力不妨给他们留条生路,也免得江南武林一蹶不振,这有违我保留江南元气的意思,毕竟草莽之中也多有俊才。对了,明鉴司不是已经将手伸入江南了么,在敌国活动虽然是司闻曹的管辖范围,可是也不要便宜了夏侯沅峰,将他一起拉下水,敢带头弹劾我,也别想袖手旁观。”

霍琮唯唯应诺,问道:“董总管传讯来,向先生请示淮西之事,还有陆氏一门可要带回大雍安置?”

我想了一想道:“淮西还算安全,石玉锦将要临盆,就让她在淮西待产吧,先别告诉她外面的事情,让董缺好好照顾她和陆梅。等到我军下淮西的时候,让荆迟将她们送到我这里来,陆氏的事情看他们的意思,如果陆夫人坚持要奉旨南徙,就让越氏好好安顿他们,否则就将他们送到大雍来。还有陆风,他现在行踪不明,应该是在韦膺的保护之下,这件事情不能放松,一定要将他找到,我已经害死了陆灿,绝不能让他的家人有什么闪失。”

霍琮心中一震,这是先生听到陆灿死讯之后唯一一次说到自己的感受,偷眼瞧去,江哲的神色依旧是那样平静淡漠,仿佛这些话并非是他说的一般,见他言词无碍,思路清晰,计策也是从前那般狠辣,本应该放心才是,可是霍琮心中突然涌起强烈的不安。然后,他耳边便传来江哲斩钉截铁的声音道:“听说杨秀不惧南楚朝廷的责难,在广陵为陆灿设了祭帐,可有此事?”

霍琮心中一惊,刚想要说没有,却发觉江哲的目光冰寒刺骨,看了一眼神色木然的小顺子一眼,终于无奈地道:“这,听说是的,司闻曹回报,巴郡、江夏、九江、寿春、广陵、余杭,各军都设了祭帐,就是南楚朝廷也不敢明令阻止,淮东军更是全军缟素,每日里都是哭声震天。”

我闻言释然道:“这才对了,若是这些人连祭帐都不敢设,也枉费陆灿的孤忠和良苦用心。小顺子,我明日想去广陵拜祭灿儿,你觉得如何?”

霍琮大惊,连忙看向小顺子,希望他像以往一样阻止先生不当的举动,不料小顺子眼中闪过挣扎的神色,良久才道:“是,我会保护公子去广陵,绝对没有任何人可以阻拦先生的路途。”

听到小顺子肯定的回答,我宽心的笑了,道:“是啊,我怎能不去拜祭灿儿呢,只可惜他的尸身在建业,要是能够见见他多好。”

小顺子毫不犹豫地道:“公子放心,等到攻下南楚之后,我陪着公子去建业,替大将军重修坟茔,到时候公子便可以祭奠大将军灵柩。”

我含笑点头,道:“好啊,你去安排吧,呼延寿是肯定要跟的,其他人么能免就免了,对了,裴云身边那个杜凌峰我很喜欢,如果他有兴趣,让他一起去吧。”

小顺子应诺道:“是,我会安排好的,公子不如好好休息一下,明日还要赶路,公子可是不能劳累的。”

我闻言点头道:“也好,我去躺一躺。”

小顺子小心翼翼地扶着我走到床前,我不由暗笑他这般多事,好像我是容易摔碎的瓷人一般,躺在床上,我几乎是立刻进入了梦乡,梦中仿佛见到久违的陆灿音容,唉,这小子急什么,我不是很快就要拜祭你去了么?也不用这么快就托梦给我吧,放心吧,你的家人我都会好好照看的。

我却全然不知道,走出房门之后,霍琮脸色铁青地抓住小顺子,道:“先生不对劲,顺叔,不能去广陵,先生的离间之计瞒不了南楚人这么久,杨秀只怕会把先生生祭在陆将军灵前的。”

小顺子眼中露出少见的惶恐和悲痛,良久才道:“公子要去,谁也不能拦阻,走,跟我去见太子殿下和裴将军,公子去广陵的时候,要让裴将军大军在淮水严阵以待,如果公子有什么三长两短,就让裴将军渡过淮水,将淮东军全部屠杀干净,为公子报仇就是,可是就算公子会死在广陵,这次也不能阻止他去,谁也不能。还有一件事,你要记着,若是你敢背叛公子,我必将你碎尸万段,让你死无葬身之地。”说罢,小顺子露出酷厉冰寒的神色,甩开霍琮,径自走去,霍琮只觉得一股寒意从心底涌起,他忽然明白了一切,明白了小顺子为何不顾先生安危,同意他置于险地,但是明白过后,心中的重压却几乎令他不能呼吸,不能思索,小顺子的威胁更是让他明白,无论如何,先生都不会平白无故地伤害自己,只因对于先生来说,若是伤害自己心爱的弟子,就跟伤害自身一样痛苦,忍不住泪水滂沦,霍琮艰难地移动步子,走到江哲的卧房之前,跪倒在地,从房内传来江哲均匀的呼吸声,显然他睡得很熟,可是霍琮却是越来越伤悲,转瞬之间已经泣不成声。

淮水南岸,如今已经是一片缟素,在得知陆灿死讯之后,杨秀纵然是奉了陆灿遗命,也再不能抑制心中的悲痛,更何况军中皆是悲声,便不顾尚维钧的猜忌在广陵设下祭帐,想来法不责众,尚维钧也不能利用这个理由为难淮东军。军中将士,皆是白衣戴孝,黑纱缠臂,人人皆是悲愤欲绝。却在这时,突然有斥候回报,雍军集结在淮水北岸,泗州城前,磨刀霍霍,竟似有趁机攻击之意,杨秀不由大怒,乘人之丧而攻之,自古以来便是不义之举,众将士也是怒不可遏,纷纷振臂高呼,欲和雍军血战。岂料雍军却是遣使渡水传讯,大雍楚乡侯江哲意欲至广陵吊祭,众将面面相觑,虽然众将未必能够识破大雍的离间计,可是陆灿被赐死的罪名就是勾结大雍意图自立,这江哲实在是害死大将军的罪魁祸首,当下群情愤然,都是声言要将江哲杀死在灵堂之上,以祭陆灿英灵。

众将士可以快意恩仇,杨秀却是不能轻易决断,若是江哲真的前来祭灵,于情于理,都不能杀害大雍吊祭的使者,但是若是任凭江哲来去自如,只怕军中的怨恨就会集中在自己身上,军中本已有了怨言,只因自己不曾起兵相救大将军,他本是蜀人,若无陆灿支持,根本难以在军中立足,如今能够统帅淮东军,也多半陆灿余威和自己这几年的经营,若是伤了军心,只怕就是尚维钧不动手,自己也不能掌控淮东军队。更何况雍军拥兵淮水北岸,所为何来,不用问也知道,一旦江哲陨命广陵,那么雍军必然渡水作战,现在并不是和雍军大战的好时机。所以思之再三,杨秀婉拒了江哲前来吊祭的要求。

可是这年轻的使者却肃容道:“杨参军,你我两国虽然是敌对,可是忠臣义士人所共敬,陆大将军和楚乡侯更是少年之交,份属师徒,情同手足,虽然不幸中道分离,各为其主,以至于生死相见,可是私情不害公谊,还请将军不要拒绝楚乡侯一片诚心,想来就是大将军泉下有知,也会乐于见到侯爷亲来吊祭,人死如灯灭,想来大将军也不会怀恨昔日恩师的。”

杨秀思索再三,终于叹道:“江侯爷居然有此心意,我若坚拒,反而令天下人觉得我南楚将士心胸狭窄,只是在下不妨直言,若是江侯轻身来此,会有什么后果杨某也不敢肯定,不过杨某定然尽力阻止淮东将士复仇之心。”

那少年使者端重地道:“我大雍上下皆相信南楚将士不会迁怒于我家侯爷,若有意外,想必也与将军无关,只是我大雍太子殿下也在楚州军中,殿下有令,若是侯爷有什么短长,必要血洗淮东,才能向陛下交待,请杨参军谨记此事,莫要等到刀兵一起,以为我军不教而诛。”

杨秀眼中闪过厉色,冷冷道:“使者是在威胁杨某么?”

那少年使者平静地道:“纵然在下不说明,莫非将军还想不到我军拥兵泗州城下是为何么?我大雍行事素来光明正大,故而太子殿下令在下向参军大人明言此事,却并非是有意威胁,我们两国之争,已是不死无休之局,纵然今日不战,将来也是要战的,太子殿下并不认为拥兵淮水就可以威胁将军。”

杨秀闻言眼中闪过异色,道:“好个大雍太子,素闻贵国太子殿下自幼便有贤孝之名,想不到行事也是这般刚毅果决,好,杨某就静候楚乡侯前来祭灵,不过并不保证他的安全就是了。”

那使者也没有惊怒之色,只是行礼想要告退,杨秀却止住他,目光在这看上去平凡普通的少年使者身上凝注了片刻,问道:“还未请问贵使尊讳?”

那使者神色仍然是冷冷淡淡,道:“在下霍琮。”

杨秀目光一寒,良久才道:“原来是你,好,送客。”

待霍琮离开大帐之后,从内帐走出了韦膺,虽然只有数月时间,韦膺的形容憔悴了许多,尤其是陆灿死后,他在短短几日之内,竟连两鬓都有了星霜,这让原本十分擅长保养的韦膺仿佛苍老了几岁。他目光幽冷地道:“杨参军,你想不想为大将军报仇?”

杨秀知他心意,淡淡道:“大丈夫就是想要报仇,也不能用这种手段。”

韦膺冷笑道:“你以为那人会是真心前来吊祭么,只怕他离去之时,就是尚维钧动手之时,你就不怕尚维钧以此为借口为难你么?”

杨秀从容道:“两军交战,尚且不斩来使,何况是前来吊祭的使者呢?我就是这样禀明朝廷,我朝素重礼法,想来尚相也不能以此为借口,韦兄,你对大将军的心意我是感激的,可是这次却不能任你动手。”

韦膺听出杨秀话外之意,却是怀疑自己想要报私仇,其实他虽然未必没有趁机报复之意,可是却实在是想替陆灿报了江哲陷害之仇,但是望着杨秀淡漠的神情,却是没有再多言,转身黯然离帐,心道,这世上也只有大将军一人敢于相信我,他如今已死,南楚军中也不是我久留之地了。

走出大帐不远,厉鸣匆匆走来,目光中满是不可置信的神色,韦膺见他神色古怪,正欲动问,他已经走到韦膺身边,低声向他说了几句话,韦膺眼中也闪过匪夷所思的神色,厉鸣见状又低声道:“崔庠传来消息,门主已经同意对陆氏下手,传书请首座回去,门主许诺既往不咎。”韦膺目光沉凝下去,良久才道:“等我见过江哲之后,我们便回去。”说罢又冷笑道:“这场猫哭耗子的好戏怎能不看呢?”

翌日,大雍前来吊祭的车马渡过了淮水,一行人皆着素衣,在南楚军士虎视眈眈之下,来到了广陵大营。

我坐在马车上,静静地想着心事,这次随行的除了小顺子和呼延寿之外,虎贲卫是一个不拉的全部跟来了,本来是不想带他们的,这么多高手勇士,不是挑衅么,可惜他们居然说什么若是不能保护我,有违皇上旨意,我也就只好认了。除此之外,随行的还有霍琮和杜凌峰,霍琮昨天自请出使也就罢了,这次还要和我一起来,罢了,这小子要是不怕死就让他跟吧,至于杜凌峰,我实在是觉得他在我面前如坐针毡的模样十分有趣,原本只是一提罢了,并不准备让他跟来的,谁知这小子居然咬着牙跟来了,想想也觉得好笑。不过也不知道小顺子是怎么说服了李骏和裴云的,我原本还担心得让小顺子背着我跑到广陵来呢。

马车停了,小顺子在外面请我下车,我伸了一个懒腰,这一路真是折腾人,路不大好走啊,连年征战,道路损毁,等到拿下淮东之后,应该纠工整顿一下道路。走下马车,觉得外面的阳光有些强烈,忍不住迷了迷眼睛,眼前一片缟素,不论是地上的积雪,还是南楚军士手中的兵刃,都映射着明亮的光芒,令我几乎睁不开眼睛。

霍琮已经站到我身边,扯了我衣袖一下,上前引见道:“先生,这位就是杨参军杨大人。”

我看了杨秀一眼,这人我还记得,便上前施礼道:“杨参军,多年不见,风采却是如昔,不知道还记得江某么?”

杨秀凝视江哲良久,上次见面的时候江哲重伤初愈,神色憔悴,全无光彩,他其实没有看出此人有什么奇异之处,十余年不见,这次见面,杨秀只觉得这人神色恬淡,目光幽深,灰发霜鬓,岁月的流逝让这人变得越发沉凝,只是眉宇间总是带了几分散漫,令杨秀心中疑惑的是,江哲面上丝毫没有悲色,在杨秀想来,这人不论是真是假,理应面带戚容才是。

犹豫了片刻,感受到身后诸将的骚动不满,杨秀冷冷道:“楚乡侯前来吊祭,可知我军上下深恨阁下,阁下恐怕来得去不得!”

听了他包含威胁的话语,呼延寿、杜凌峰和虎贲卫众人都是面露怒色,呼延寿更是上前一步道:“要想伤害侯爷性命,还得看我们答不答应。”

霍琮却是沉默不语,目光中只是多了些忧虑,而小顺子则是面如寒霜,就是怒气填膺的南楚军士也能够感觉到空气中多了几分寒意,尚未吊祭,帐前便凝滞住了。

杨秀目光望向江哲,想看看他如何应付这局面,若能让这位大雍楚乡侯在这里受挫,最可以振奋军心的,只是不杀了他,便不会失了道理。

我烦恼地皱紧了眉头,这些人怎么回事,在这里吵闹什么,耽误我的时间,想来灿儿等我已经很久了,冷冷道:“就是要动手也得等江某拜祭之后。”说罢我也不理会众人,便向祭帐走去。

杨秀一愣,暗中打了一个手势,站在祭帐之前的两行白衣白甲的军士同声高呼道:“楚乡侯进帐拜祭大将军!”便同时拔刀出鞘,两两相交,举在头顶,在帐前摆下了迎客的刀阵。雪亮的单刀映射着日光和雪光,刀柄上系着的素绸随风飘舞,每个军士眼中都露出耀眼的杀机。

我见这些阻道的南楚军士终于让出了通道,满意的一笑,便向祭帐走去,只是怎么眼前总有些雪色素绸在脸上拂来拂去,不耐烦的皱皱眉,懒得伸手去拨开这些素绸,径自向帐内走去,走入雪色的祭帐,一眼便看到盛着陆灿衣冠的灵柩和摆在上面的灵牌,我只觉得浑身的力气似乎消失殆尽,走到灵柩之前,双腿已经有些发软,也不顾及什么礼仪,便抱膝坐在灵柩前面用作跪拜的蒲团上面。

凝望着灵牌许久,我放声吟道:

“记得相逢一笑迎,剪烛西窗夜谈兵。

结恩深处胜骨肉,不因孤零欺馆宾。

无奈寒霜摧庭兰,羁旅承恩拘闲云。

人生南北多歧路,君向潇湘我向秦。”

一诗吟毕,尤觉不足,不假思索,再度吟道:

“廿载征尘如一梦,中原北望气如山。

才兼文武无余子,功到雄奇即罪名。

太息反目成仇雠,割袍绝义中道违。

君归黄泉无所恨,洒泪苍天可告谁?”

吟完两诗,觉得心中畅快许多,眼前仿佛见到陆灿的音容笑貌,又想起秋玉飞和逾轮的传书,他临死之前仍要谢我,我们早已经恩断义绝,纵然明知他若能杀我也不会轻轻放过,我却知他始终不曾忘记昔日旧情,只不过私人情谊抵不过两国仇恨,才有今日的结局。

不过呆了多久,目光瞥见霍琮怀中抱着的古琴,随手一挥,霍琮将琴递过,我盘膝坐下,轻拂琴弦,心中想起少时在江夏渡过的时光,如今想来,那竟是我这一生最快乐的日子,琴音不知不觉间响起,我心中只想着那段平和安乐的日子,想起和陆灿抵足而眠,想起他在校场练习射箭,迫着我也陪他在烈日下面流汗,想起我替他伪造功课交差,想起和他偷溜出去游春,却被陆侯爷捉个正着的尴尬,想着想着,唇边不由露出微笑,琴声也越发活泼灵动。

杨秀立在祭帐之外,神色凝重地望着被阳光映射得几乎透明的白色帐幕之后的单薄身影,摆开刀阵迎宾原本只是想要摧折江哲的勇气,可是这文弱书生竟然眼睛也不眨一下地走入祭帐,其中好几次他头上的钢刀做势下移,他都没有丝毫理会,这一刻,杨秀真的相信了这人胆量包天的传言。

听到那人朗声吟诵的两首悼词,杨秀纵然觉得这人定是虚情假意,却也不由闻之摧心,想到大将军战功赫赫,一片忠诚,却死于内争而非战场,竟连马革裹尸都不能够,不由暗自伤痛。

可是当琴声一起,杨秀面上神色大变,那琴声中竟没有一丝悲意,反而是充满了欢畅,不说杨秀颇通音律,就是那些原本虎视耽耽的将士,初时也觉气恼,可是只听了片刻,杀气便渐渐消退,反而不约而同地忆起少年时候结交的玩伴,想起那铭刻在心,没有利害关系的友情。琴声越来越平和喜乐,可是不知何时,杨秀却觉得脸颊已经润湿,仿佛身陷在不愿醒来的梦境中一般,等到杨秀清醒过来,身边已经泣声一片,明明是欢喜至极的琴音,可是却无人不觉悲从心起,这一刻,杨秀当真相信江哲乃是真心诚意前来拜祭。

当琴声终止,江哲仍然是神色淡漠地从祭帐之内走出,匆匆一拜便扬长而去,这时候,淮东军上下竟然没有人想要留难他,他们已经忘记了这人的身份,只记得他是大将军的少年好友,如此而已。

小顺子和众人护着江哲车马,几乎是毫不停留地渡过了淮水,能够这般容易回来,许多人都想不到,看到雍军大旗的时候,纵然是悍不畏死的虎贲卫士也是忍不住低声欢呼,只有小顺子、呼延寿和霍琮都是忧心忡忡,不时留心江哲的神色。

我望见策马前来迎接的李骏,不知怎么,心中似乎有什么断裂了一般,我伸手拉着小顺子,艰难地问道:“小顺子,陆灿他死了?”

小顺子无视众人望过来的惊异目光,目中露出坚决的神色,狠心地道:“是的,陆灿已经死了。”我这才觉得天昏地暗,这几日以来,陆灿的死讯虽然入了我的耳,却未曾入我的心,直到此刻,我才突然明白过来,陆灿真的死了,死在我的手上,一阵撕心裂肺的疼痛凭空袭来,只觉喉中一甜,一口鲜血已经吐在了小顺子的衣袖之上,素衫鲜血,越发刺眼,抬头望见小顺子忧惧的目光,我只觉得眼前一黑,便向下栽倒,只觉得有人扶住我,在我耳边呼喊,我却什么都不想听,只是任凭泪水滑落,意识也渐渐沉入黑暗。

众人的惊呼声中,李骏已经冲到了江哲身边,只见江哲已经昏迷过去,苍白的面容上一丝血色也无,紧闭的双眼却是泪水直流,那泪水竟是淡淡的红色,李骏惊叫道:“先生怎样了?”

这几日一直脸色沉郁的小顺子却长出了一口气,道:“好了,好了,总算是哭出来了,这下可以放心了,殿下,立刻将公子送回楚州,召军医诊治。”心中却是一阵后怕,想到江哲得闻凶讯之后不正常的冷静,他便担心江哲悲痛过甚,虽然之后江哲似乎头脑清醒得很,可是小顺子却从蛛丝马迹中觉察出异常,为了让江哲将心伤释放出来,才不顾一切纵容江哲去广陵拜祭,终于令江哲清醒过来,纵然为此伤病,却也不妨了。

霍琮愣在那里,看见小顺子欣慰的神色,欢喜和悲伤两种情绪同时袭来,一时不觉涕泪交流,连忙用袍袖胡乱擦拭,跟着众人的脚步匆匆向楚州而去。

\chapter{第四十一章 行路难}

公虽殁,余威尤在,于百姓亦有遗恩。

初,公自襄阳南返,随公归者,不绝如流,公于途中奏以长沙闲田处之,未果,公以谋逆罪死于囹圄,尚相以安陆、云梦荒地处之,又疑中有细作,拘束甚严,民皆苦,泣曰:“不若死于军法。”

尚相闻之怒,阴令心腹屠戮之,有公旧部暗告众人,曰:“大将军救诸人,今尚相欲杀无辜,我不能忍,请即行。”民皆泣号,不知所为,其人乃以公书信令牌授之,令众人乘夜返襄阳,奉令者闻之,追杀不舍,道路诸将,皆公旧部,见令牌皆释之,民得返襄阳者十之八九。至襄阳,民皆泣告城下,愿受军法,雍将长孙冀不忍,犹豫未决,民以公书信呈上,长孙冀览信而叹,请旨皆赦之。至今襄阳之民,皆奉公之灵位。

——《南朝楚史·忠武公传》

“泻水置平地,各自东西南北流。人生亦有命,安能行叹复坐愁。酌酒以自宽,举杯断绝歌路难。心非木石岂无感,吞声踯躅不敢言。(注1)”

山路崎岖,蜿蜒难上,一个中年美妇带着两个女剑士攀山而上,听到迤逦传来的悲歌,这中年美妇面上先是露出一丝嘲讽,但是继而神色变得怆然,耳中听到水声潺潺,便加快了脚步。绕过一道绝壁,眼前一亮,豁然开朗,半山处却有一块半亩方坪,右侧峭壁林立,削若笔管,左侧绝壁之间,一线飞瀑若断若续,便如玉带流碧,瀑下乱石嶙峋之间却是一方深潭,流瀑溅在碧潭中心润白如玉的一方巨石上,阳光映射下宛如珠玉。一个青衣男子坐在潭边青石上,脱了鞋袜,双足浸在潭中,似乎全不觉得冬日积雪汇成的潭水的刺骨寒意。中年美妇望见了男子身边连鞘的佩剑一眼,冷笑道:“韦膺你可后悔当日定要依附陆灿,和我们作对的决定?”

韦膺也不回头,淡淡道:“这世间可以后悔的事情太多了,我若要后悔这件事,还不如后悔猎宫之事,这些日子,你们的损失可是比我惨重,我虽然没有了靠山,可是你们却损失了中坚力量,莫非你不后悔么,贵妃娘娘?”

那女子面上露出浓厚的戾气,原本美艳的容貌几乎也变得扭曲了,良久,她才平静下来,冷冷道:“不要这样叫我,什么贵妃,什么娘娘,我不过是师姐的棋子罢了,窦皇后、长孙贵妃、颜贵妃才是李援的贤妻爱妾,我纪霞又算什么?不过这个身份也有好处,否则凭着尚维钧权倾江南的势力,又怎会入了我的罗网呢?这一次我们的损失的确很重,萧兰、凤非非和谢晓彤都死了,非非和晓彤也还罢了,她们除了有一身剑术之外,平素行事束手束脚,萧兰却是可惜了,月影轩一直是交给她打理的,她这一死,我便失去了助力,这倒是头痛的很。”

韦膺冷冷道:“如今凤舞堂、仪凰堂已经只剩下你和燕无双两个首座,实力空虚,所以你才会说服门主,和辰堂和好如初,甚至不计较我襄助大将军之事?”

纪霞扬眉道:“正是,我不仅希望与你合作,还希望你助我夺权,燕无双为了挽回面子,亲自刺杀石观,如今重伤卧病,凌羽一向不理事,若是你我合作,就是得到门主之位,也不是不可能的。”

韦膺回头道:“你这却是痴心妄想了,凌羽能够稳占门主之位,一来是因为有梵门主遗命,二来也是因为当初闻师姐训练的那些女剑手,尚有半数听从她的命令,她隐忍多年,默认自己被咱们架空的处境,却非是怯懦,绝不会任你行事。而且如今我们三堂虽然都是势力大减,可是百足之蛇,死而不僵,燕首座刺杀石观成功,令我们得以插手淮西军,这份功劳可谓不小,韦某虽然失势,可是若没有辰堂作为外围力量,你们也别想在南楚立足稳固,反倒是你,乔园损失的力量主要都是仪凰堂的,若不能成功完成这次诱敌入彀的计策,虽然你们笼络了尚维钧、赵陇,可是仪凰堂也将从此沦为附庸,若我是你的话,就不要想着自相残杀,还是想想如何将拥护大将军的江湖势力一网打尽吧?”

纪霞听了韦膺的冷言冷语,不但不懊恼,反而笑道:“好,好,你能够坦然直言,可见还当我们是自己人,门主,你可听见了,可不会怀疑韦首座的忠心了吧?”

韦膺眼神微微一变,目光落在了纪霞身后的两个女剑士身上,这两人都是三十五六岁年纪,神色木然,剑气凌人,看不出有什么异常,可是就在韦膺目中露出异色的时候,其中一人突然朗声道:“师叔说得不错,韦首座果然是一片忠心。”说罢走到潭边,伸手到流瀑之中,鞠了些水洗去面上药物,露出天然国色的丽容,嫣然笑道:“还是师叔的手段高明,不过是些脂粉药物,便瞒过了韦首座的眼睛。”

轻轻一叹,韦膺从容不迫地整理衣着,穿上靴袜,起身淡淡道:“原来是门主有意相试,韦某虽然效命大将军,却也不过是为了本门着想,莫非门主以为韦某还有什么别的选择么?”

凌羽露出惭色道:“却是本座多心了,韦兄与我等既有同门之谊,又同是天涯沦落人,岂会有二心,这一次我等定要戮力同心,才能让我凤仪门在南楚大展宏图,还请韦兄不要怪罪本座存心试探才好。”

韦膺心中轻叹,这个多年来黯淡无光的女子一鸣惊人,将三堂多年来的努力一并接收,凤仪门主选了她为继任倒不是仅仅为了势力的平衡。虽然心中感叹,但是面色却是丝毫没有变化,只是淡淡道:“这也是理所当然之事,门主重整三堂,自然应该确认我等的忠心的。”

凌羽虽然神色淡然,此刻也不免眼中露出喜色,欣慰地道:“韦首座能够这般想就最好不过,这次我们设下罗网,定要将那些不识相的江湖中人一网打尽,到时候我们凤仪门便可在江南独霸天下,再加上我们的力量已经渗入朝廷和军队,数年之内,定能恢复昔日荣光。”

韦膺没有言语,心中却是冷笑。

见他神色漠然,凌羽反而更加放心,她深知韦膺心计深沉,如果他并非真心回归,必定不会这般冷淡,既然如此,她更需好好笼络韦膺,在她看来,韦膺的才能更在门中诸人之上,若不能得到他真心的支持,凤仪门想要在朝野立足必然分外艰辛。想到此处,凌羽对纪霞笑道:“师叔,请您再去巡视一下,这件事情也只有师叔亲历亲为,才可以令我放心。”

纪霞裣衽道:“属下遵命!”说罢孤身向下走去,另一个女剑士则是退到山路的转弯处,按剑护卫,纪霞走了片刻,知道自己已经走出了那女剑士的视线所及,才缓缓停住脚步,面上露出黯然的神色,想到自己一生任人摆布,出走到了南楚之后,为了夺得权力甚至不惜一切,可是只是数日之间,一切努力都化为泡影,让扮猪吃老虎的凌羽坐享其成,想到此处,纪霞便觉得无比疲惫。良久,她的神色振奋起来,虽然凌羽重掌大权,可是她不相信韦膺会甘心听命,而且自己的三个弟子都颇为争气,小弟子纪灵湘已经是贵妃,宠爱冠绝后宫,二弟子灵剑虽然相貌不甚出色,但是剑法之精在新进弟子中首屈一指,至于大弟子灵雨,想到她,纪霞皱了一下眉,这个弟子对于剑术不甚用心,只是醉心音律,这倒也没有什么,凭着她的才貌,若肯用心拉拢朝中显贵,却也不错,却偏偏她竟是死也不肯,若非是她的冷淡性情更令众人倾心,自己早就不会容许她这般放肆了,不过这一次却不能再放纵她了,笼络蔡群不仅是凌羽决定的,也是她争夺权力的重要手段,所以这次回去定要迫服这个小妮子。心中思绪万千,纪霞再次举步向下走去,毕竟目前最重要的是即将开始的大战。

韦膺目光从流瀑上收了回来,道:“纪堂主手中实力不可小视,门主不应对她如此轻忽的。”

凌羽目光流转,笑道:“这是自然的,却不知韦兄可是仍为陆氏之事怀恨我等?”

韦膺冷冷道:“韦某为大将军效力也不过是为了报仇的私心,如今大将军既然已经死了,我也不会为陆氏殉葬,可是你们这等短视,推波助澜,自毁长城,难道就不担心雍军南下,南楚若亡,你们纵然权倾朝野又有什么用处呢?”

凌羽叹道:“这也是不得已啊,如果陆灿肯和我们合作,本座也不希望这样做,可是你清楚,陆氏父子对我们凤仪门全无好感,若是他掌了大权,只怕我们就没有容身之地了,如今虽然没有了陆灿,可是这几年南楚军力强了许多,至少可以守住长江,只要能够守住江南,总有我们存活之地,所以虽然时机不大恰当,但是还是不得不下手了。”

韦膺轻轻一叹,再无言语,凌羽见状笑道:“这一次你选定了此地作为伏击之处,当真是最合适不过?”

韦膺淡淡道:“自越郡至南闽,有两条路,一条是从衢州常山走分水关大路,一条是从衢州江山走仙霞岭小关,自江山青湖至浦城,一路上要经过仙霞岭、丹枫岭、梨岭、仙阳岭,几百里山路处处皆是死所,其中又以仙霞岭最险,峭壁峻岭,高三百六十级,共二十四曲,长二十里,沿途隘口数处,宽度不到一丈,居高临下,一夫当关,万夫莫开,险峻之处,不亚于蜀中剑阁,我们途中设伏,自然百无一失。”

凌羽目光一闪,道:“陆氏流徙之人虽然不少,可是除了陆夫人和陆灿幼子陆霆之外,别的人生死都无需在意,不过尚相之意,那救走陆云之人必然也会前来救援陆氏遗孽,为了一网打尽,还需诱蛇出动,我想让你的辰堂先动手,引出暗中保护之人后,再集中门中全部力量,雷霆扫穴,你看如何?”

出乎凌羽的意料之外,虽然这个计策明显有消弱辰堂实力的意味,可是韦膺却一口答应道:“自该如此,辰堂虽然人多势众,但是大半都是碌碌之辈,纵然损失惨重也无妨碍,不过陆氏母子的性命还是要紧的,若是他们死在混战之中,那么前面救援的人就会退缩,不如令辰堂外围之人和尚相派来的精兵先行攻击,再由我带着堂中高手扮作救援之人,然后护着陆夫人和陆霆固守待援,这样一来,那些暗中保护的人就会如他飞蛾扑火一般自行投到,等到适当时机,门主便可发动全部力量,斩尽杀绝。”

凌羽心中暗喜韦膺的计策狠绝,又道:“既然如此,就烦劳韦兄了,不过据我所知,陆灿次子陆风应该在你手中,此子也不能留,韦兄可不能心慈手软。”

韦膺心知凌羽定在自己身边有细作,而且这人身份还不低,否则不会知道这样隐秘的事情,不过此刻他已不在意这样的事情,所以只是扬眉道:“此子生死有何要紧,不过韦某素来谨慎,提防着有了万一的变化,还可利用他拉拢大将军旧部,要杀他也要等到这边成功之后,否则岂不是太可惜了?”

凌羽闻言苦笑道:“韦兄说得太迟了,我已经派了朱师叔去杀他,不过想来这边也不会失败吧?”

韦膺的双瞳瞬间收缩了许多,却状似无意地低头掩去眼中杀气,道:“我派去监视这小子的人只怕不会轻易让朱长老动手,只可惜了我苦心收服的四个护卫。”

凌羽笑道:“韦兄放心,我已经请朱师叔小心在意,不会随便伤了你的人的,朱师叔当初随着师尊转战天下,虽然已经退隐多年,可是余威犹在,一身剑术更是老练狠辣,应该可以制住那几个护卫,不需伤了他们的性命。”

韦膺目光低垂,暗暗沉吟,凌羽能够一举夺权,除了仪凰堂、凤舞堂实力大损之外,朱长老这些人也是原因之一,她们多半都是凤仪门主同辈的师妹或者昔年的侍女,早已经封剑归隐,甚至当年猎宫之变也没有参与,却因为池鱼之秧而被迫一起流亡南楚,如今她们不甘寂寞,重出江湖,却也难对付得很,不知道陆风能否保住性命?不过不管陆风生死如何,自己如今却也顾不得他了。

说到此处,两人都觉无话可说,各自沉默下去,目光望向下面的山路,未过多久,韦膺身边的亲信崔庠匆匆走了上来,那女剑士轻叱阻拦,未等韦膺出言,凌羽便已下令放行,韦膺目光一凝,却未多说什么。崔庠上前一揖道:“启禀门主、首座,再过半个时辰,陆氏流徙众人就可到达此地,请示下。”

韦膺转头看向凌羽,凌羽微微一笑道:“辰堂的进攻就由韦兄自行安排,本座也要去安排妥当,等到适当时机,便会出手。”说罢凌羽飘然而去。韦膺知道凌羽对自己仍然存了一分戒心,恐怕要等到辰堂牺牲惨重之后才会真的相信自己,暗暗一叹,他从容道:“你率堂中众人攻击,我会率辰堂血卫闯进去保护陆夫人和陆公子,我们都会蒙面行事,你们也不能露出身份,不要让他们知道实情,这样一来彼此厮杀都不会留情,便不会露出破绽。”

崔庠闻言惊道:“首座,这样一来辰堂力量大损,恐怕有害无益,还请首座仔细思量。”

韦膺冷笑道:“辰堂所属虽然众多,但是多半都是软硬兼施强迫收纳的,其中忠于本座的不过十之二三,,其他人多是趋炎附势,本座如今失势,只怕他们早就心存反意,这一次正好借刀杀人,清除堂中败类,就是全死了也没有什么可惜,本座的血卫足可自保,你也不必担心我的安危,把我们当成仇敌就行了,只要小心一些,别自己送了性命就成了。”

崔庠心中冰寒,虽然韦膺素来杀伐决断,可是今日这般狠毒,仍然是让他瞠目结舌,这次堂中前来担任伏击者乃是多年来收纳的高手,占了堂中实力十之五六,一旦折损,辰堂势力必然大减,可是韦膺却毫不顾惜。转念又想到这些年来韦膺每从堂中招纳高手组建血卫,这些血卫不仅武艺高强,而且对韦膺忠诚不二,人数虽少,却占了堂中实力十之四五,只是血卫负责攻坚,常有折损,至今人数仍不足五十人。这次韦膺将血卫几乎全部带了来,原本以为是要最后雷霆一击的,想不到韦膺却要让这些血卫和辰堂主力自相残杀,一旦两败俱伤,岂不是自折臂膀,越想越是觉得韦膺疯了,崔庠愣愣地站在那里,却是说不出一句遵命行事的话来。

韦膺心冷如冰,见到崔庠这般模样,却毫无怜悯地道:“你还不去,莫非是想抗令么?”

崔庠觉察出韦膺身上的冰冷杀气,心中一寒,猛然想到厉鸣踪影不见,素来韦膺便更信任厉鸣,这一次却不带他前来,是否奉了韦膺之命在暗中待命呢,所以才会不惜折损辰堂实力,想来就是为了要清除内部的隐患,想通之后心中豁然开朗,这正是韦膺素来用人的手段,轻易不会让人知道他真正的目的和计划,便欣然道:“属下遵命。”

韦膺望着崔庠离去的背影,目光寒冷如冰,表面上看来他身边的心腹是厉鸣和崔庠二人,崔庠更是受到重用一些,但是实际上,他却对崔庠有些不信任。此人有本事将辰堂投效来的牛鬼蛇神压制得服服帖帖,武功出众,平日行事也是十分得力,这样的人才却甘居下陈,自己对他又无多少恩惠,怎样想来也觉得不安。

只不过韦膺本就不甚相信这些被武力财富所胁迫的属下,所以才将辰堂大半力量交给崔庠统领,只是冷眼旁观其中动静,任凭这些四分五裂的江湖高手明争暗斗,自己却从中选取一些可用之人,收服其心,编入血卫,而这些真正忠诚的血卫则由他自己亲领,任何人都不能插手,反而是位在崔庠之下的厉鸣,因为得到信任能够知道一些机密。方才和凌羽一席谈话,韦膺便知道辰堂这些人中必有凌羽的人,而凌羽心气极高,崔庠很可能便是她的目标,方才又见凌羽对崔庠这般态度,韦膺便更加疑心,此刻崔庠又坦然答应率众自相残杀,丝毫也不顾惜属下生死,心中更是生出杀意。若非崔庠这般行事暗合了他的心意,只怕韦膺已经要骤下毒手了。

强自压抑心中杀机,想到一切事情很快就会有个了断,韦膺再度将目光投向飞瀑,只见一线流泉击在石上,飞琼碎玉,溅雪如烟,心中生出无限凄怆,举目望烟霞,苍烟无际,眼中雾气浮起,陆灿的音容笑貌宛在眼前,想起自己苦心保护的陆风有可能已经被杀,心中痛楚,再也难当,数滴清泪没入潭中,转瞬无踪。

蜿蜒的山路上一行人马缓缓而行,最前面是一队禁军,此刻都小心翼翼地走着,生恐落入驿道一侧的深谷中去,身上都是衣甲齐备,虽然攀山过岭,十分辛苦,却完全没有卸甲轻身而行的打算。中间行走的四五十人却形貌各异,却都是形容憔悴,风尘仆仆,更夹着一些老弱妇孺,其中有一个中年女子步履十分艰难。这女子虽然是粗衣囚服,却依旧雍容风姿,只是容颜皆被汗水尘沙遮盖,她身边两个青年女子各自背着一个包裹,虽然也是艰苦无比,但似是仍有余力,不时地搀扶这中年女子前行。除了这三个女子之外,还有五六个妇人,年纪多半在二三十岁上下,身边多有男子扶持,一见便是夫妇模样,更有一些男女童子,聚在一起,彼此相携,奋力攀登,更有一个五六岁的小男孩实在不能独立登山,被一个中年男子缚在背上前行。除此之外,便是二三十个男子,年纪仿佛,都在三十岁上下,虽然都穿着囚服,但是行动之间隐隐有杀气威势,隐隐结成军阵,护在妇孺外侧。

在他们身后,又有一队禁军,他们在攀登之时仍然小心翼翼地监视着前面的囚犯,唯恐出了什么变乱。本来就是有个把人途中脱逃,或者出了变故,也不算是什么大事,最多报上疾病而死即可,可是这些都是钦犯,别说逃走一个,就是死了一个,上面恐怕也会怪罪下来。

更何况这些禁军都知道自己押解的是什么人,大将军陆灿威名赫赫,旧部无数,肯为他效死的义士更是数不胜数,事过境迁,陆灿鸩死乔园之日,有人欲要救援的事情早已沸沸扬扬,更何况本已被判了斩立决的陆云被人劫走,若说不会有人路上劫囚,这些禁军是绝对不信的。仙霞岭山路崎岖,却拦不住江湖中人,若是有人趁机救走了陆夫人或者小公子,这可是灭门的大罪。

当然后面这队禁军为首的都尉段约心中更有别的烦恼,他也是个世家子弟,虽然家族势力不大,却也能勉强在建业立足,虽然他并非嫡子,却也得承家族关照,做了个禁军都尉,统率千余军士,驻在建业城外,本以为这一生也就这样浑浑噩噩的过去,想不到这次却接下一个烫手的差使,居然得到谕令,让他押送陆灿家人到汀洲定远,那里可是蛮荒瘴疠之地,姑且不论能否活着回去,只是想到一路上的艰险就足以让他裹足不前了。更何况他除了担心会有人前来劫囚之外,更担心另外一件事情,虽然在尚维钧的高压之下,并无多少人敢私下议论,可是尚维钧有意斩草除根的流言蜚语早就暗中流传,自己非是尚相心腹,想来也不会暗中示意自己途中下手,但若是真的得到密令,他也很怀疑自己有胆子下手。大将军生前威名显赫,旧部无数,若是自己真的做了帮凶,十有八九就会被当成替死鬼,就是尚相不灭口,那些骄兵悍将也放不过自己,就算侥幸无事,在军中也别想抬起头来,担上这样的罪名,就算是在尚相嫡系的禁军之中,也难免遭到排斥。

更令段约头痛的是,直到离开建业,他也没有得到什么密令,这样一来便有两种可能,一来是尚相并无意为难大将军家人,这自然是最好不过,只要自己安全将钦犯送到定远,就没事了,想来大将军的旧部也未必愿意冒了叛逆之名中途劫囚吧,就算是劫囚,只要自己识相一些,倒也未必就死了,回到建业最多是除去军职,在家族的斡旋下,性命应该无碍。可是如果尚相是准备另外派人截杀,自己这些人全做了陪葬牺牲,那可就一丝生机也无了。

心中存了这样的想法,段约一路上不仅小心翼翼,更是不愿对陆氏一门众人有所失礼,心想若是真得遇到敌袭,说不定还可得到助力,他可是知道这次被流徙的除了陆夫人母子和一些婢仆之外,还有一些陆氏的家将,多半都在战场上面厮杀过,比起这些没有经验的禁军,更有些用处,若是能够安全抵达定远,纵然暗中得罪了尚相,倒也不是没有生机可言。

韦膺远远地望见陆夫人一行,虽然还有数里之遥,在他看来却是如在眼前,虽然因为山路转折,那些人影忽隐忽现,但是他的目光却几乎透过层层山岩,落到陆夫人的身上,仙霞岭的山路虽然修建的颇为不错,路面皆是从山崖上采集的青石铺成,平坦齐整,只是山势险要,五步一转弯,三步一上岭,一边是峭壁,一边是山涧,不能骑马坐车,只能步行攀登,就是寻常男子也会苦于路途,更别说像陆夫人这样的女子,想到此处不觉心中怆然,大将军身后如此凋零,情何以堪。目光一闪,又看到被一个陆氏家将背负的陆霆身上,想到这幼童兄姐多半生死不明,心中只觉微痛。

正在韦膺心神渐乱之时,前面的禁军都已经到了山势较为平缓之地,那些提心吊胆的禁军都是心中一宽,纷纷避到路边蔓蔓青草之上,或坐或倚,各自休息。韦膺见状微微冷笑,他立在高处俯瞰下面山道,那些禁军竟都没有发觉,想到从前见过的雍军和陆灿麾下楚军,行军之时何曾如此轻慢,从怀中取出一方青色绢帕,将面目掩住,只露出一双眼睛,然后退了几步,避免给陆氏家将发觉,这些家将必会留心周围,难免会看见自己的身形,这时,从绝壁之后走上三十个身穿劲装的蒙面人,都是身携兵刃,步履沉稳,见到韦膺之后,俯身下拜,韦膺示意他们不要出声,仍是向下面望去。

没过多久,山崖之下传来纷纷攘攘的人声,却是后面众人也都到了,段约见此地宽阔平坦,故而下令停止前进,已经是正午时分,正好休息片刻。所有的军士和陆氏众人,都取出干粮饮水各自吃饭。那些禁军以往都在建业繁华之地,如何受过这样的苦楚,纷纷抱怨不休,陆氏众人却是默默无言,两个青年女子扶陆夫人坐在路边青石之上,陆霆被那中年家将解了下来,抱到陆夫人身边。、

那家将名叫陆康,本是陆信的近卫,对陆氏忠心耿耿,只因性情耿直,又不愿离开陆信,所以始终没有独自领军。陆信殁后,陆灿对他十分敬重,又因为他已经年过四旬,所以将他留在府中统率家将。陆康今年已经有四十六岁,妻子前年过世,又没有子女,所以对于陆灿诸子皆是爱如亲生,尤其是陆霆最得他疼爱。今次陆氏遭劫,陆康随同陆夫人流徙,仙霞岭道路艰难,陆康唯恐陆霆年幼失足,所以将他缚在背上,就连别的家将想要背负陆霆,他都不能放心。

陆霆虽然被背负而行,可是小小年纪数月来经历种种惨变,又得知父亲身故,哭泣不休,上路时已经是有些不妥,这些日子道路艰难,更是水土不服,形容消瘦,双目青黑,令人看了心痛万分。陆夫人抱过陆霆,柔声喂他喝水,又让他吃干粮,陆霆只吃了两口,便再也吃不下去。陆夫人心中担忧,却也无计可施。她身边的两个青年女子虽然名为婢女,却将陆夫人当成姐姐一般看待,其中一个叫做陆贞的侍女劝解道:“夫人,等到到了浦城,我们请段将军在那里停留几日,请个大夫来给小公子诊治,入了闽境,尚维钧的势力就不那么大了,段将军一路上颇为照顾,想来是不会拒绝的。”

陆夫人轻叹道:“也只有如此了,云儿、风儿、绣儿和梅儿都是下落不明,若是霆儿再有些三长两短,我纵然死了也难以去见他们的父亲。”说罢,又将干粮掰碎,迫着陆霆吃下。见她如此,两个侍女都是珠泪低垂,她们两人都是被陆夫人收留的孤女,更曾经跟着家将学过武艺,这一次陆氏遭劫,事前陆夫人便有了察觉,更是将家中婢仆散去,如今留下的任,都是受过陆氏重恩,坚决不肯离去,这两个侍女一向是陆夫人身边的宠婢,又有些武力,所以坚持不肯离去,一路上若没有她们两人照顾,陆夫人只怕会更加艰难。

正在这时,本来倚在山壁上闭目休息的陆康突然眉头一皱,低声道:“大家小心,我听见有人从后面数里赶来,来人步伐纷乱急促,想来不是寻常商旅。”

陆氏的家将都知道陆康从军多年,最擅地听之术,都是心中一惊,目光看向陆夫人,陆夫人不知军事,却看向陆康,陆康轻声道:“若是大将军旧部前来援救,多半是军旅中人,这些人绝对不是,虽然听说有些江湖义士参与乔园之事,但是夫人若能平安到了定远,却也胜过匿踪逃刑,所以这些人多半不是来救我们的人,不过禁军无用,我们不如想法子趁乱夺取兵刃自保的好。”

众家将都是深恨禁军,不由都流露出赞同之色。正在此时,段约带着两个军士走了过来,众人见状各自微微移动身形,以防范突变,段约丝毫不觉,朗声道:“陆夫人,末将也料不到路程这样艰难,等到了岭下的仙霞驿站,不如雇一乘轿子,明日就让夫人和小公子乘轿而行如何?”陆氏众人闻言都是大喜,陆夫人却淡淡道:“妾身多谢将军好意,只是深恐犯了律法,累及将军。”段约见陆夫人并没有严拒,心知定是陆夫人担忧爱子,所以才有意接受,便笑道:“夫人言重,末将没有什么别的本事,手下这些兄弟还管束得住,只要不让旁人知道,到了仙阳岭平缓之地,夫人再步行就是。”

陆夫人闻言也是心中略喜,想到若有软轿,至少可以让爱子得以休息,望了陆康一眼,点头示意,陆康心中明白,上前道:“陆康代夫人多谢将军。”然后又低声道:“将军小心戒备,后面有不速之客。”

段约闻言大骇,怔怔地望了陆康一眼,匆匆向后走去,想到若非自己觉得上了仙霞岭之后,就无需担忧尚相耳目,所以好意提出替陆夫人雇佣轿子,那家将也未必会告诉自己这件事情,不由大叹好心有好报,连忙低声传令,让一些军士堵住后面隘口,又令一些军士到前面探路。这些禁军训练不精,一时间山道上情势混乱,看得陆氏家将都是皱眉嗤笑不已。

正在这些禁军纷乱之时,山路前面却突然放出惨呼,段约一惊,转头看去,只见一个禁军踉踉跄跄地跑了回来,刚刚出了隘口便一跤跌倒,背上的衣甲已经中分,鲜血迸流,显然是有人一刀砍裂了衣甲,伤了他的性命。段约心中一寒,攻击竟从前面而来,莫非陆康竟是误导自己么?还未想得清楚,身后山路上已经传来手下军士喝骂之声和兵刃相撞的声音,转回头来,段约看见那狭窄的隘口正有一些黑衣蒙面人攻来,幸好山路狭窄,被禁军军士死死挡住,这些军士虽然不善战,却也知道若是失去此处隘口,只怕没有命在,所以倒也不惜生死,堵住了山路。段约心中一宽,连忙下令前面的禁军阻住前面的隘口,此处山道两端隘口若被敌人占据,中间地势广阔,最适合激战,到了那时,只怕真是一线生机也无,所以段约连连下令,迫手下军士死守。这时候,前后敌踪都已暴露,过了片刻,段约便从军士口中得知前后各有敌人百余人,依次来攻,而且都是擅长武技的江湖人模样,正适合在狭窄的地方激战,若非自己带了几具强弩,恐怕早被那些人攻进来了。段约忧心忡忡,口中却高声道:“尔等何方盗匪,竟敢劫掳禁军,速速退去,尚可留尔等性命。”

闻言,那些黑衣人都是哈哈大笑,更有一人一刀将眼前的军士人头砍落之后,大笑道:“你们这些禁军皆是无能之辈,杀就杀了,谁还顾惜你们的性命,若说要杀我们,也得你们有这个本事,难道你们是大将军的麾下么?”

段约闻言更是惊骇,心道这些莫非是来救陆氏一门的江湖人物,再度高声道:“你们若是大将军的旧部,应该知道前来劫人有害无益,陆夫人和公子虽然流徙南闽,但是将来也未必没有遇赦还乡的机会,你们若是胡作非为,劫夺钦犯,到时候陆氏一门就真的不见天日了。”

那些黑衣人却又是出声嘲笑,反而加强了攻势,更有人出言说些污言秽语,虽然不曾辱及陆夫人,但是言语可憎,令陆氏众人也是簇眉不已。

段约心中叫苦,这些人既不是寻常盗匪,又不是陆将军一方的人,那定是截杀陆氏一门的刺客了,想到此处不由生出同仇敌忾之心,转头向陆夫人哀求道:“夫人,这些匪徒定是不怀好意,能否请夫人下令让府中家将相助末将。”

陆夫人闻言,想了一想道:“这些人绝不是先夫故旧,如果将军落败,我等的遭遇恐怕更加难堪,确实是并肩作战的好,将军不如将前面的防卫交给陆康指挥,将军专心后面的战事如何?”

段约心中大喜,连忙同意,分了一些兵器给陆氏的家将,陆康留下五个家将保护陆夫人等妇孺,自己率着二十多个家将到了前面,这些家将都是善战猛士,再加上陆康指挥得当,不到片刻就稳住了前面的危局。

可是虽然如此,那些攻击的黑衣人都是武艺精熟的悍匪,兵器又十分精良,虽然不善于战阵,但是因为山路隘口狭窄,所以武力便成了关键,他们一人几乎可以抵上数个军士,所以双方实力此消彼长,不到一个时辰,禁军已经死伤叠籍,若没有陆氏家将的战力,只怕已经被攻破了隘口了。

陆康心中焦急,心道这些悍匪在此地动手,定是看准了此地易守难攻,虽然他们不容易攻进来,可是我们也不容易攻出去,这是要将我们一网打尽啊,可是虽然想通了这一点,却也无可奈何,陆氏的家将虽然武艺精熟,可是比起那些悍匪来说,近身搏斗并非所长,若非仗着力量和配合,只怕早就被这些黑衣人攻进来了。

正在陆康心焦之时,突然听见侍女陆慧高声喊道:“康叔,上面有人下来了。”

陆康闻言抬头望去,只见从山崖之上,放下五六条长索,正有些黑衣蒙面人援绳而下,心中大惊,正欲令人用弩弓射杀,只见其中一人手一举,却是一块玉牌,然后轻轻掷来,陆康下意识的伸手接住,却是陆灿令牌,凭此可以出入大将军府邸,陆康仔细瞧去,只是片刻已经看出这人身形宛似韦膺。可是他心中犹豫,韦膺虽然是大将军心腹之人,可是毕竟也是凤仪门中人,凤仪门勾结尚相,谗言加害大将军已经不是什么秘密,韦膺此来到底如何他也不敢确定。只是陆康心中一犹豫,已经有十余个黑衣人落在地上,抛出玉牌那人也不解去面纱,只是向臂上一指,却是一方血色丝巾。然后便拿着兵器向前面走去,那些禁军本想分出人来厮杀,却被陆康阻住,那人也不管众人疑虑,走到前面,一剑便刺死了一个趁隙闯进来的黑衣悍匪。

陆康见状大喜,高声道:“这是自己人,大家不必担心,说着又示意众人留心臂上红巾。”众人这才放下心来,全力迎敌。而这些黑衣人已经全部下来,分头向两侧支援。这些黑衣红巾的蒙面人个个武艺高超,悍不畏死,有了他们相助,那些蒙面悍匪攻势渐渐被遏制,只是这些人皆是江湖人手段,厮杀起来旗鼓相当,损失也是越发惨重,双方都是狠辣非常,就是被刀剑所伤,也是没有丝毫惊惧,只是舍命攻杀,不过片刻,两边隘口都已经尽是鲜血,只是道路狭窄,若有重伤者或是战死者往往立足不住,跌落山道,要不然只怕已经被伏尸阻住道路了。

只是被困在山道上的众人虽得援军,但是两侧敌人也是人多势众,苦战了许久,众人都是渐渐力竭,反而是敌人轮换来攻,仍然龙精虎猛。陆康拭去面上鲜血,目光落到那已经退了下来,站在自己身边调息的蒙面人首领,低声道:“韦先生前来救援,大将军泉下有知定然感激不尽。”

韦膺觉得浮动的气息渐渐平稳,也没有回答陆康的话,只是淡淡看了一眼山道对面的山岭云霭,道:“我不过是来赴死的。”

陆康心中一震,正要说些什么,只见后面传来吼声如雷,更有一个清朗的声音直入耳中,却是有人运气高呼道:“丁铭在此,陆夫人、陆公子不必忧心。”然后耳边便传来书生惨叫,却是强援到了,陆康大喜,连忙对韦膺道:“韦先生,能否请你迎接丁大侠,里应外合,定可除去后面的敌人。”

韦膺目中闪过寒芒,道:“你放心。”

说罢连声厉喝,那些黑衣红巾的蒙面人如今还有十六人幸存,九人在前面隘口,七人在后面隘口,听见韦膺厉喝之声,前面便又分了四人过来,随着韦膺冲到后面隘口,那些残余的禁军都依着段约之命退下,只留下陆氏家将配合韦膺等人,两面夹攻,那些悍匪前后遇敌,不过两刻时间,已经纷纷死伤殆尽。韦膺一剑刺倒一个蒙面悍匪,那人拼死一刀还击,却只是削落了韦膺面巾,在他英俊的容貌上留下一道刀痕。那人心中早已存有的疑虑在看见韦膺容貌之后终于得到答案,指着韦膺厉声道:“你——”话音未落,已经被韦膺一剑封喉,踢落山道。这时,韦膺眼前一花,只见一道剑芒划破长空,等韦膺定睛一瞧,却是一个布衣儒士转过隘口,手中长剑光芒四射,两个悍匪正掩住双目痛呼,跌跌撞撞地向山崖坠落。

丁铭瞧见韦膺,便是一惊,虽然知道此人和陆灿的关系,却也想不到这人竟然有勇气前来护送陆氏赴闽,就在他一愕之间,韦膺已经扯了一块衣衫,将面孔蒙住,转身带着剩下的九个血卫奔向前面隘口,陆康却过来高声道:“是丁大侠么,那些臂上戴着红巾的是自己人。”丁铭心中豁然,举步跟着韦膺等人向前面走去,在他身后,数十名风尘仆仆的汉子随着苦竹子走来,留下数人守住隘口,还有些人负责监视禁军,提防他们动手,毕竟他们在尚维钧心目中已经是敌人了。

丁铭和韦膺也曾相识,只是他看不起韦膺昔日叛国之事,所以两人并没有什么深厚的交往,如今他却紧赶几步,走到韦膺身边,和他并肩而行,感慨地道:“韦兄不畏奸相权势,当真是大将军知交,丁某素来多有得罪,还请韦兄见谅。”岂料韦膺没有作声,只是淡淡瞥了他一眼,便仗剑前行,丁铭一愣,却非是奇怪韦膺的无礼,而是他分明望见韦膺一双寒光四射的眸子中,竟然有着绝决之意。

只是数步之间,两人赶到前面隘口,形势已经岌岌可危,留下的五个血卫只有一人还在浴血苦战,禁军更是死伤殆尽,陆氏家将也是死伤惨重,韦膺和丁铭同时冲入敌群,剑光闪闪,连杀数人,才遏制住局面。这时,在那些黑衣蒙面人后面指挥攻打隘口的崔庠心中越发惊疑,他方才听到韦膺事先约定的喝声,知道是让他趁机猛攻,他便派上了手下最精锐的高手,如今却又被首座阻住,首座这般做法究竟是想做什么?

还没有等到崔庠心中想明白,山崖之上突然飞起焰火,继而传来银铃一般的笑声,崔庠心中惊疑,抬头望去,只见山道绝壁之上不知何时已经站了八九十个女子,其中有荆钗布裙的老妇,也有仪容华贵的中年美妇,更有许多三十岁左右年纪的雪衣女子,还有些十八九岁年纪的娇美少女,却都是相貌冰冷,腰悬利剑,被众女如同众星捧月一般簇拥着立在绝壁之上的是一个霓裳女子,天姿国色,宛若仙子。

崔庠心中立刻明白,自己等人是让那些来援救陆氏的人相信并非陷阱的诱饵,虽然还不明白为何首座要这般冒险,不仅牺牲自己率领的辰堂下属,还要牺牲他心腹的血卫,更是连自己也舍命厮杀,但是崔庠已经知道若想活命,此刻就该逃了,连忙下令撤退。还未等崔庠率众退走,只见绝壁上那些雪衣女剑手都取出弩弓,同声齐喝,三道乌光射向对面的山崖,轻轻巧巧没入石壁,只隐隐听见响动,丁铭等人仔细看去,那些乌光却是一些特制的弩箭,一触到石壁箭矢便张开形成飞抓,稳稳地抓住了突出的岩石,铁抓削铁如泥,都是深深扎入石壁之中,而以丁铭的目力更是发觉那些飞抓之后都漂浮着一根几乎肉眼难以看见的丝线。还未等丁铭想明白,崖上那些雪衣女剑手已经顺着斜飞的丝线飘落到地面上,轻如落花,落地无声。

从崖上最先跃下的几人一到便是挥剑杀去,将一些瞠目结舌的禁军刺杀在地,不过丁铭不仅剑术精通,也知军略,连连下令,收拢防线,等到这些女子全部下崖之后,阻住道路之时,丁铭已经率众将陆氏众人护在山壁之下,而韦膺和他麾下的血卫都是苦战多时,筋疲力尽,也被护在后面。

凌羽飘下山崖,见状心中暗喜,却不露声色,上前道:“这位想必就是吴越第一剑丁铭丁大侠,当日在乔园,本座的二师姐和七师妹想必就是死在丁兄剑下的吧?”

丁铭闻言叹道:“卿本佳人,奈何作贼,这位想必就是凤仪门的凌门主,昔日梵门主虽行悖逆之事,却也不会为奸臣张目,残害忠良,门主这样做岂不是有辱师门。”

凌羽面色一寒,道:“只需将你们斩尽杀绝,今日之事还有何人知道?”

见凌羽面上杀机毕露,丁铭冷笑道:“若要人不知,除非己莫为,凌门主自欺欺人,却不知天下谁不知道凤仪门党附尚维钧,构陷忠良的丑事。”

凌羽大怒,传令道:“给我将他们全部杀了,本座要用他们的鲜血,祭祀姐妹亡灵。”话声未落,突然岩壁下传来陆夫人惊叫,丁铭等人都是大惊失色,回头望去,只见韦膺手中抱着陆霆,长剑横在陆霆颈上,他身边皆是黑衣人相护,正和陆氏家将对峙,陆夫人头发披散,舍命挣扎,便要扑过去夺还孩儿,却被两个侍女死死抱住。

丁铭也顾不得凌羽在前,剑指韦膺厉声道:“你要做什么?”

韦膺除去面巾,冷冷一笑道:“韦某舍生忘死,不过是为了诱使你们入伏,如今已经达到目的,自然不愿和你们并骨青山,你若放开道路,让我带了小公子出去,纵然是你们都死在这里,还可留得小公子性命,若是不然,韦某和门主内外夹攻,纵然本座死在此处,你们也别想活命。”

陆康见状大骂道:“韦膺,大将军对你器重亲厚,你却这样翻脸无情,方才我还感激你不顾生死救护夫人公子,想不到你竟是这般狠毒心肠,丁大侠,绝不能放他出去,公子落在他身上,必死无疑,若他留下公子,倒可放他出去。”

丁铭闻言深以为然,也道:“韦膺你乃是叛国逆伦之人,如今又辜负大将军厚爱,当真是死有余辜,本来以在下之见,纵然死了也要拖你上路,可是你若肯将小公子留下,我就暂时留你性命,放你出去。”

韦膺放声大笑,手中长剑轻轻颤动,陆霆颈上渗出血迹,虽然他病恹恹,神思昏昏,却也痛得大叫,陆夫人见状一声惨叫,螓首低垂,竟是昏迷过去,韦膺敛去笑容,冷冷道:“韦某乃是一片好意,不过想替大将军留下一脉香烟,你若想小公子陪死,还不如我现在就杀了他。”

丁铭众人面面相觑,难以决定,这时陆夫人悠悠醒来,一双明目便如清水也似,惨然道:“丁大侠,放他去吧,韦先生,你若念大将军半点好处,也不要伤了霆儿性命。”

韦膺望见那双满是悲伤恳切的眼睛,心中一颤,道:“夫人放心去吧,除非我死,否则绝不许任何人伤了小公子。”陆夫人微微点头,颜面而泣。丁铭见状黯然,终于令人让开道路。

韦膺也不理会众人仇恨鄙夷的目光,抱着陆霆走向凌羽,道:“韦某苦战许久,想先下去休息,不知门主可否允许?”

凌羽目光一闪,道:“你真的想救这个孽种么?”

韦膺目光一闪,低声道:“我在广陵见到江哲拜祭大将军,知他当真是伤痛彻骨,若能留得陆氏一子在手,必然有些用处,只是门主已经令人去杀陆风,我只好留下陆霆的性命。”

凌羽微微一笑,终于相信了韦膺的诚意,道:“好了,你去吧,辛苦了,等我将这些人都杀尽了,再来和你商量这件事情。”

韦膺微微一笑,抱着陆霆走向通往浦城方向的隘口,陆霆大哭起来,伸手向韦膺面上抓去,但是他此刻病弱无力,又是小小年纪,韦膺仿若不觉,转瞬之间,韦膺的身影已经消失在山路之后,只听见陆霆的哭声隐隐传来。

——————————————

注1:鲍照《行路难》

\chapter{第四十二章 悔已迟}

丁铭心中一痛,仗剑前指道:“就让在下见识一下名震天下的凤仪门剑法吧,你们还不动手么?”

这时候凌羽身边一个灰发妇人冷笑道:“既然你们想死,我就成全你们。”说罢挥剑而上。

凌羽微微皱眉,但是这人乃是自己的师叔身份,性情如火,也不便说她什么,故而笑道:“诸位姐妹,给本座取了这些人的首级,以报大仇。”凌羽一声令下,这些女子挥剑冲上,霎时间剑影如山,剑光如雪,杀向这些义士和陆氏家将。

一时之间,血光迸现,杀声四起,丁铭心中一叹,若非得知陆夫人一行被困在山道上,自己也不会全无留手的赶到这里救援,想不到却是中了凤仪门奸计,自己一死也还罢了,连累这许多义士,又害了陆夫人性命,当真悔恨不已,只恨那韦膺如此奸猾负义,又叹天机阁主这次未允前来,此时丁铭心中再无生还之望,手中长剑势如长虹,如同龙翔凤舞,生生挡住几个年纪已老的女子,这几人都是剑术高手,昔年纵横中原的女剑客,却被一个后辈挡住,都是心中恼怒,剑法也是越来越凶狠,若非丁铭也是以命搏命,只怕已经被她们冲破防线杀进去了。

见到丁铭等人在强大的攻势下岌岌可危的模样,想到从此之后,凤仪门便可独霸江南,凌羽唇角露出笑容,更添了几分丽色,越发显得容光照人。

抱着陆霆的韦膺带着仅存的十个血卫,走出了隘口,他的目光淡凝,任凭陆霆哭喊挣扎,就连面颊上已经凝结的刀痕被陆霆抓破,鲜血一滴滴落下,也没有让他眼神发生一丝变化。

走过二十余丈,崔庠已经独自等在那里,其余的人都被他遣到前面去了,再没有得到韦膺命令之前,他实在不敢让双方碰面,一旦有些人怒火攻心,向韦膺发难,那可就麻烦了。韦膺却看也不看他一眼,径自走向事先驻扎的营地,沿着山道前行不远,韦膺便施展轻功,掠入岭上密林,左传右折许久,才到了一个平坦的谷地,三面都是峭壁,外面则是竹林,中间可容数百人休憩,正是辰堂选好的营地,不过现在营帐虽然还在,却是只有七八十人还在这里,更是大半伤痕累累。

他们一看到韦膺抱着陆霆过来,本来各自起身相迎,可是这些人也是老江湖了,很快就发觉不对,目光落在韦膺臂上红巾,以及他身后浑身带血的血卫身上,种种疑惑顿时明了,他们中本就有人已经怀疑,这下子疑团顿解,有些人顿时喝骂起来,全然不顾韦膺在前。崔庠心中焦虑,正要上前阻止,却见韦膺一声冷笑,身后一个血卫挥手一扬,一个骂声最响的大汉眉心中了一柄飞刀,顿时身亡,这些人顿时鸦雀无声,想起韦膺素日的手段,都是心中一寒,虽然目中凶光四射,却再也不敢多言。

韦膺冷笑道:“你们这些蠢材,死去些废物有什么要紧,又不是你们的亲人故旧,若是不这样做,我们岂能置身事外,得到下手的机会,莫非你们很想被那些妇人女子一辈子压在头上么?”

这一次众人的目光都有了变化,凶光渐渐褪去,他们素来都是凶狠成性的悍匪,岂甘心被些女子占了上风,只是韦膺既然同意辰堂听命于凤仪门主凌羽,他们也没有什么法子,凌羽的势力在那里摆着,他们也不敢出言反对,如今听到韦膺语气,似乎有些转机,立时都忘了死去的同伴。

韦膺见状更是嘲讽地道:“若是你们有胆量和本座一起动手,将这些女人一网打尽,将来南楚境内还有谁敢和我们作对,还不快些准备一下,等到他们两败俱伤,我们就要出手了。”

其中一人犹豫地道:“首座,她们人多势众,而且武艺高强,我们实力大损,恐怕很难得手吧?”那人说完便悄悄后退了一步,担心韦膺恼羞成怒对他出手,果然这句话一说出来,场中又是议论纷纷,毕竟辰堂力量大损就是韦膺一手造成的。

韦膺却毫无气恼的模样,冰寒的目光环视一周,人人都觉得他的目光中充满了自信,虽然没有多说什么,但是这些人却平静下来,焦急地等待着韦膺掀开底牌。

韦膺冷眼看着这些狰狞的面孔,只觉得心灰意冷,想到自己当初为了报仇,急功近利地组建辰堂,以至于堂中多半是些见利忘义的盗贼匪类,虽然自己利用武力和金钱将他们牢牢控制在手中,甚至利用他们替陆灿做了许多事情,可是这些人却仍然没有多少长进,就连自己命令他们截杀陆灿遗孤,这些人也完全没有异议,除了自己挑选出来的这些血卫尚有一些忠义血性,眼前这些残存下来的恶徒都是该死之辈。想到此处,最后一丝怜悯也渐渐消散,韦膺冰冷地道:“将箱子抬上来。”

两个血卫早从隐秘之处抬了一个樟木箱子上来,其中一人打开箱盖,露出许多拳头大的红色弹丸,韦膺指着箱子道:“这些是本座用二十万两银子向毒王申如晦买来的一百枚‘阎王笑’,阎王笑内藏火药剧毒,只要用得好,一枚就可以取了几十人性命。现在凤仪门正在和江南武林中人激战,我们只要封住前路和上方,就可以将她们消灭十之八九,本座亲率血卫上崖,将凤仪门留下的警哨除去,然后诸位便可为所欲为。这瓶中乃是解药,凡是有胆量跟随本座去的人就服上一粒,富贵险中求,等到大事一成,我们便是生死兄弟,将来必定同享荣华,若是胆小的人不妨留在这里,只要不随便行动,本座也不怪罪你们,这里只有五十粒解药,价值千两黄金,去的人少了,本座还可以省下几粒珍贵的解毒药物。”

众人闻言多半惊喜交加,有的争着上前,有的怯懦后退,最后选出了三十五人参与此事,剩下的解药则是韦膺和这些血卫使用的,定下计策之后,韦膺又下令众人先饱餐一顿,恢复精力,自己则抱着陆霆走入营帐。陆霆一路上昏昏沉沉,此刻早已含着眼泪睡着了,韦膺怜惜地看着他虎头虎脑的可爱模样,面上的冰冷神色渐渐软化,将他放在床铺上,替他盖好被子,轻轻拍着他促他入眠。

过不了多久,一个血卫走入帐内,低声道:“首座,所有不愿去的人都已经处置了。”

韦膺恢复冰冷的神色,淡淡道:“可有引起变乱?”

那血卫禀道:“首座放心,我们在那些人的饮食中下了迷药,现在他们都已经昏睡了,说是提防他们通风报信,其他的人也很谅解,毕竟谁都不想和凤仪门真刀真枪地敌对,等到我们离去之后,留下一个兄弟将他们全杀了就是。”

韦膺轻轻点头道:“好,雷九,你可是觉得我心太狠么,就连自己的属下都不放过?”

雷九寒声道:“这些人都是无义之辈,大将军乃是国之栋梁,被奸臣陷害而死,就是我们这些杀人如麻的恶人也觉得不忍,这些人却是毫无动容,将他们除去理所当然。不过——”说到最后两个字,雷九偷眼望了韦膺一眼,又道:“首座这般计策,将凤仪门和陆夫人、丁大侠他们一并害了,属下还是觉得心中不安,虽然丁铭那些人和我们素来是对头,但是毕竟他们也是大将军知交,还有陆夫人在内,首座这般做未免太狠了。”

韦膺神色冷冷道:“大将军殁后,南楚军政尽被奸相掌握,凤仪门便是奸相的左膀右臂,若有她们在,一来大将军旧部时刻不安,二来大将军家人难逃死劫,所以不论为了什么缘故,凤仪门都是必需除去的。若能铲除凤仪门的势力,别说牺牲一个辰堂,就是再加上丁铭那些人的性命也是值得的,再说韦某本就是叛国逆伦的恶人,再加上一条残害忠良的罪名又有什么关系。至于陆夫人,唉,却是我无能为力,她们母子若不留下一人,纵然我辰堂势力折损许多,凌羽也不能相信本座,更不会任由本座离开,想来陆夫人若是知晓内情,也会要求本座带走小公子吧。只是本座有些对你们不起,你们这些血卫不仅对本座忠诚不二,这些年来也是为国为民做了不少大事,如今却令你们折损许多,我心中十分不安。”

雷九斩钉截铁地道:“首座不必如此说,雷九本来是一个穷途末路的盗匪,若非得到首座器重,至今还在江湖上浑浑噩噩的挣扎求生呢,可是这些年来雷九却可以堂堂正正的活着,更是能替大将军效力,为国尽忠,就是现在死了,也觉得不枉此生,可以去见父祖之面。今日虽然死了许多兄弟,却是为了保护陆夫人而死的,死有何憾。只是,只是若能救出陆夫人,纵然我们这些人全死了,属下也觉得心甘情愿。”

韦膺闻言黯然道:“我在四年前和天下第一用毒高手,毒王申如晦偶然相逢,侥幸帮了他一点小忙,所以这次才能从他那里购得这些毒药,阎王笑内藏剧毒十分厉害,中毒百息之内若不能得到解药,就是必死无疑。随本座前去的共有四十四人,还有五粒解药没有使用,留一粒给小公子,以防万一,另外四粒若能给陆夫人等人,倒也可以救几个人,只是一旦发动起来,只怕就来不及了,就是因为这个缘故,我才没有多想此事。”

雷九也是苦笑不已,是啊,那剧毒发作如此厉害,纵然有人可以在发动之后到崖下送药,却也没有法子在百息之内令陆夫人等人相信并服下解药,怪不得韦膺不考虑此事,雷九也是心狠手辣之人,事已至此,多想无益,便出言道:“时候应该差不多了,是否让他们准备动身呢?”

韦膺点头道:“我想丁铭他们勉强可以支撑到天黑,现在是该去了,雷九,你就不要去了,小公子我就交给你保护,如果我能够生还,自然罢了,若是我死了,陆夫人安然无恙,你就把小公子交给陆夫人,如果陆夫人也死了,就交给杨秀杨参军,万不得已的时候,也可以将小公子送到大雍楚乡侯江哲手上,他虽然是大雍重臣,可是和大将军私谊深厚,想来是可以庇护小公子的,只是此事有违大将军之意,若非不得已,还是不要这样做的好。”

雷九惊道:“属下岂可临阵脱逃,不如让崔护法去吧。”他不知道韦膺对崔庠的疑心,仍然将崔庠当成韦膺的心腹。

韦膺怒道:“这怎是临阵脱逃,若非厉鸣尚有要事,不能脱身,我也不会让你做这件事情了,崔庠若是现在走了,我担心那些人生疑,你应知道现在大将军身后凋零,小公子若有什么意外,只怕,唉!你是血卫之中随我最久的了,若非是信任于你,我怎敢将小公子相托,这件事情不容置疑,你想抗命么?”

雷九闻言不敢相抗,只得唯唯听命。韦膺放下心事,起身走出营帐,望着暮霭渐沉的山林,只觉一阵疲惫,其实这一次虽然有毒药暗器相助,可是凤仪门的剑术武功也是不同凡响,更有许多灵丹妙药难以揣测,最大的可能就是两败俱伤,凤仪门纵然全毁,自己也别想全身而退,这一去便是再无回头之路,纵然以韦膺之心狠,也觉得心中怅然。

可是渐渐的,韦膺眉宇间现出戾气杀机。回头之路?自己早已经没有了回头之路了!自己从堂堂的相国公子成为今日的叛国逆臣,青云之路断绝,更是飘零江湖,与草木同朽,归乡不得,复仇无望,只留下满腔恨意。侥天之幸,自己得到陆灿信任,便一心助他征战疆场,希望把握这唯一的复仇机会,可是这一切却又被凤仪门这些目光短浅的女子毁去。既然自己已经再没有复仇的可能,甚至就连立足之地也快没有了,何必还要留恋人世,世间千百种苦楚,自己已经一一尝遍,生死早已经成了无所谓的事情。可是纵然有心一死,心中的恨意也不能丝毫减弱,只是恨得却不是江哲,而是凤仪门。一步走错,步步错,至今自己再无回头路可走,这一路上蒙蔽了自己灵智的不就是凤仪门么,自己就是要死,也要拖上凤仪门陪葬。想到此处,韦膺周身透出无穷杀机,看向已经整装待发的辰堂所属,冷冷道:“成功失败,在此一举,若想搏得富贵荣华,就随本座舍命一拼吧。”说罢便大步流星向岭下走去,众人都连忙随在身后,有的幻想着唾手可得的荣华富贵,有的紧张地想着如何可以在混战中保住性命,还有的知道其中凶险,却暗自下了狠心不死无休,数十人各有心思,随着韦膺走向修罗场。

雷九黯然望着韦膺背影,直到众人身影都已没入暮霭之中,这才提了一把刀,走入那些被制住的辰堂所属的帐中,丝毫没有怜悯之意,一刀一个,杀得帐内血流成河,将留在营地的四十余人全部杀了,这时候他身上已经全是鲜血,新鲜的血液溅在白天苦战时留下的血迹之上,雷九也觉得不很舒服,想到若被陆霆看到,恐怕惊吓了小孩子,便走到营地后面的泉水旁边,洗去身上血迹,然后换了一身衣衫,又走回营帐,准备按照韦膺吩咐,先带着陆霆躲避起来,等到大势已定之后,再决定如何去做吧。

岂料刚掀帘走入帐内,雷九便觉得身子僵住,只见一个剑眉星目,英俊无比的雪衣人坐在床铺上,正伸出两指替陆霆诊脉,在他身后站着一个黑衣青年,背负琴囊,也是俊秀人物,眉宇间的神色便如利刃一般刺目,这两人突如其来,相貌气度又都是出类拔萃,雷九心思千回百转,也想不出江南还有这般的人物。若非是看见雪衣人似乎对陆霆没有恶意,只怕他已经要肝胆俱裂了。即使如此,雷九仍然伸手按向刀柄,厉声道:“你们是什么人?到这里做什么?你想对小公子怎样?”

听了他连声质问,那雪衣人防若不觉,那黑衣青年却冷笑道:“我们是什么人,却也不必告诉你,这孩子也当真可怜,被你们这些匪类害成这般模样,我家四爷看了喜欢,要将他带走呢!你是他什么人?如果非亲非故,就不要多管闲事。”

雷九大怒,挥刀砍去,刀光如同匹练,狠辣非常,这一刀乃是他的杀手锏,纵横江南多年,也鲜有人能够全身而退,岂料那黑衣青年徒手迎上,雷九只觉眼前一花,便觉腕脉一麻,钢刀脱手,他反应极快,左手一扬,一柄飞刀射向那青年要害。那青年身形又是一闪,一掌拍去,那飞刀折向弹去,那青年却是一掌拍向雷九胸口,掌风寒气四溢,虽未及体,也觉得不可相抗。雷九却是大惊,顾不得那一掌的凶险,舍命向床铺扑去,却只能眼睁睁看着飞刀向陆霆刺去,口中惨叫道:“小公子!”

这时,那雪衣男子袍袖一拂,拦下飞刀,目光落在雷九惊恐悲愤的面容上,也不拦阻,任凭他扑到床前,一挥手,令随后追击而来的黑衣青年退下。雷九看到飞刀被击落,这才觉得心中石块落地,不由自主地检视了陆霆周身一遍,一抬头,正看见雪衣人那双清如寒江的眸子,心中便是一震,想到这人身边一个随从便可将自己轻易击败,心中涌起无力反抗的软弱感觉。但是他想到首座托付的重任,只得忍住羞辱,拜倒在地道:“请阁下放过小公子,在下奉命照顾于他,若是给阁下将人带走,在下无法向首座交待。”

雪衣人目光一闪,道:“此子身染疾病,又受了惊吓,若是再给你们这些粗人照料下去,只怕性命难保,本座偶然经过此地,爱惜此子根骨,有意将他收留在身边,这也是一番好意,你也不是他的亲朋长辈,有何资格阻我将他带走?”

雷九欲言又止,不知这人何等身份,小公子身份又不能随便泄漏。

见他如此,那雪衣人抱起陆霆就要向外走去,雷九大惊,欲要上前,却被黑衣青年拦住,雷九知道自己不是对手,只得颓然道:“小公子是陆大将军幼子,在下奉命照看于他,小公子的兄姐都下落不明,大将军在世上恐怕只剩这点骨血,求阁下高抬贵手,不要强行带走小公子。”

那人脚步一凝,目光闪动,许久才道:“他是陆灿幼子,此刻应该随着陆夫人迁徙南闽,如何会在这里?”

雷九唯恐他带走陆霆,想到韦膺此刻应该已经动手,倒也不必完全隐瞒,因此便轻描淡写、避重就轻地说了一些经过,原本只是希望这人听后可以留下陆霆,最不济留下姓名,让自己可以知道小公子是被谁带走,将来也好有个找寻的线索。岂料那人听后却是长叹道:“原来如此,我便觉得韦膺所作所为有些不合常理,想不到他也有这般心志,我倒是轻看他了。”

雷九心中一震,顿时明白这人竟是对自己这些人所知甚详,方才却是有意套问,不由大怒,也顾不得一切,捡起方才落在地上的钢刀便向那人攻去,岂料身形刚动,那雪衣人袍袖一挥,雷九便觉几处穴道一麻,已经跌倒在地。眼睛余光只看见那雪衣人抱了陆霆离去,大声道:“不要带走小公子,你们究竟是什么人?”耳边只传来那黑衣青年的声音道:“陆霆留在我们四爷身边,安全无虞,你不用担心,见你也是血性汉子,凌某就放你一条生路,不论是凤仪门还是韦膺,今次都是唯死而已,你还是逃命去吧。”

听到这几句话,雷九只觉得脑中轰然,一个不可思议的念头浮上,心中狂喊道:“他们定是雍人,他们定是雍人。”霎时间气急攻心,却连一个字也喊不出来,雷九就这样昏迷了过去。

丁铭一剑刺死刚刚杀死自己一名同伴的雪衣女子,然后迅速后退两步返回己方战阵,追袭而来的利剑被他身侧的两柄长剑合力挡住,与此同时,一支弩箭穿过阵形开阖时露出的缝隙,虽然被敌人击落,却成功的逼退了敌人。拭去头上汗珠,无意中一回头,只见一个十一二岁的少年拿着弩弓,目光炯炯的望着外面的凤仪门剑手,寻找着房间的机会,心中惊叹之余,也不由焦虑起来,虽然自己及时布下圆阵固守,可是凤仪门的实力果然深不可测,还不到一个时辰,自己带来的人已经只剩下一半,如今陆氏家将已经只剩下五六人,其余全是妇人孺子,至于禁军虽然还剩下二十多人,却是已经胆寒,只是因为凤仪门毫无留手之意,所以才不得不死战罢了,眼看已经很难守住,丁铭生出突围之念,只是凤仪门将上下左右都困住了,却是没有一丝生路。

这时,一个陆氏家将被一个高鬓灰发女子一剑刺杀,被丁铭等人护在后面,站在陆夫人身边的一个中年女子一声惨叫,顿时昏厥过去,同时,那拿着弩弓助阵的少年也悲声叫道:“爹爹!”丁铭心中一颤,身形一闪,再度越过战圈,一剑便如星河影落,将那灰发女子刺死,然后纵身飞退,数道剑光如影随形而来,丁铭知道若是再退,就会被敌人攻破圆阵,便停住脚步,以一己之力抵抗如山剑影。

凌羽看得清清楚楚,下令道:“不许放他回去。”随着她的命令,几个原本仗剑观战的雪衣女子也上前助阵,凤仪门众人都知道只要杀了丁铭,被围困的这些人就再也无力反抗,所以皆是全力以赴,剑气纵横,血影飞溅,丁铭知道已经到了生死关头,也顾不得留下气力回旋,竭尽全力施展剑技。

交战双方却都没有留意到在陆氏的园阵之中,一个禁军军士目光突然有了变化,这个军士原本只是寻常禁军,若说有什么不同,就是凭他的微末武技,竟然一直活到现在,此刻他正在协助一个江湖高手抵挡一个雪衣女剑手的攻击,可是他耳中突然传来节奏分明的鸟鸣之声,随着声音的变化,他的神色渐渐有了变化,突然之间,他手中的钢刀横挑,这一刀异常的狠毒,别说是对面的凤仪门女剑手,就是和他并肩作战的那个吴越义军的高手也是一怔,就在这一瞬间,这一刀已经切入了那女剑手胸腹,然后他已经顺势夺过那女剑手的长剑,剑光暴射,便如流星电闪,切断了另一个凤仪门女剑手的咽喉,然后也不顾身边众人的异样目光,他已经疾退向陆夫人的方向。谁也没有料到一个寻常禁军竟有这样的身手,几乎是被他势如破竹地冲到了陆夫人身边,一声清叱,护在陆夫人身边的两个侍女同时挥刀阻拦,那军士手中剑光一闪,已经击落她们手中的钢刀,厉声道:“陆夫人,我是江侯弟子。”

陆夫人和她身边的众人都是露出迷惑惊骇之色,几乎就在同时,绝壁上传来叱喝之声,同时无数红色弹丸从空中掷落,爆炸开来,霎时间白色的烟雾滚滚卷向交战双方,这时候日已西垂,暮霭重重,血红的霞光映射在白雾上,令得朦朦白雾也多了几分妩媚,可是这般美景却没有几人可以欣赏,白雾中传出惨呼惊叫之声,从山崖上露出数十黑色身影,接二连三的抛下弹丸,下面颇为封闭的空间尽是白烟滚滚,不见人影。

几乎就在白烟弥漫的瞬间,凤仪门众人都已经觉察出烟中剧毒,这种阎王笑剧毒虽然炽烈,可是若是闭住呼吸,仅是皮肤上沾染到毒烟,倒可以多支撑片刻,几乎大部分人都争先恐后地向上飞纵,而在这时,山崖下不仅砸下更多的毒药弹丸,烟雾中更是夹杂了弩箭暗器,最先冲上去的凤仪门女弟子都纷纷坠落下去,白雾中传出人体撞击在山石上面的声音,直到上面不再有毒烟弹丸抛下的时候,才有十数条身影穿云破雾一般借着丝索之力跃上山崖。山崖上面毒烟稀薄,可以看出冲出来的都是凌羽、纪霞这样内力精深,而且经验丰富的高手。她们几乎都是一开始就闭住了呼吸,然后隐忍到最后再飞身冲起,既无同门阻碍,上面也再没有弩箭暗器袭击,所以才能顺利登上山崖。她们经验都很丰富,几乎是登上山崖的同时就挥剑斩杀,虽然白烟障目,可是扑上来拦阻的七八个悍匪都被她们斩杀。不过等她们登上崖顶,崖下已经是一片雾海,只能隐隐听见下面传来的呻吟声,能够脱身的竟然不到十五人,陆氏一方更是一人也未冲出。

凌羽将目光从崖下收回,冷冷望向对面负手而立的韦膺,美丽的容颜上满是杀机,眼中也有惊惧之色,她万万料不到韦膺竟有如此手笔,这些毒药毒性十分强烈,必然贵重无比,更别说韦膺牺牲了辰堂十之八九的力量,想到凤仪门的实力在这毒烟之下几乎全部折损,自己重建凤仪门荣耀的心愿瞬间成了泡影,凌羽神色变幻莫测,最后只是一字一句仿佛迸出来一般,恨声问道:“为什么你要这样做?”

品味着凌羽话语中隐藏的刻骨仇恨,韦膺却微笑道:“这样不好么,青山寂寂,寒水澌澌,正是埋香葬玉之所,对了,我将辰堂掌管的生意已经暗中卖了,所有的银两都变成了这些毒药,只为了杀死凤仪门上下百余人,韦某这般慷慨,门主准备怎样报答韦某呢?”

凌羽拔剑出鞘,剑芒如雪,吞吐不定,她冷冷道:“韦膺,你这叛贼,当真辜负了师尊教诲之恩,只凭我们几人,就可以将你葬送在此地,你既然自己寻死,本座就成全了你。”

韦膺淡淡道:“不错,韦某清楚得很,你们几个人足以将韦某等人杀死在此地,可是只凭你们女子难道还能在江南立足么,若没有辰堂之力,你们便是瞎子聋子,只能听凭尚维钧摆布,哼哼,韦某纵然死了,你们也是很快就会来陪我的,可别忘了大将军之死和你们有多少干系,就是南楚没有人敢向你们寻仇,江哲江随云岂会放过你们。至于说韦某是叛贼么……”韦膺的声音一顿,继而放声大笑道:“十三年前韦某就已经是个叛贼,叛国叛君,叛父逆伦,如今再背叛你们又有什么要紧?”

凌羽闻言大怒,心中怒火高涨,仰天长啸,啸声宛如凤鸣九天,也不见她如何动作,已经剑化长虹,身剑合一,匹练般的剑光向韦膺当心刺来,韦膺仿若未见,负手望天,眼中满是淡漠,竟是无意还手。

韦膺无视生死,他身边的血卫可不愿坐视主上被杀,其中两人纵身迎上,岂料凌羽身形仿佛轻烟一般,剑光左右一闪,那两个血卫已经跌落下去。这时,那些均是面如寒霜的凤仪门弟子已经各自展开身形扑来,她们心中都是同样的惊怒,只见剑光闪闪,那些想要救援韦膺的血卫和想要逃命的辰堂属下都被笼在了灿如烟霞的剑光之中。能够逃出毒烟的除了凌羽之外,都是和纪霞同辈的凤仪门弟子,更是曾经杀人无数,绝不会有丝毫手软。其实若非方才她们自恃身份,没有向丁铭等人出手,否则恐怕也等不到韦膺来袭击了就得手了,当然韦膺原本也是料定了她们不会随便出手,而是会令新进弟子出手历练。此刻她们恨意如山,都是全力以赴,更是结成剑阵,顷刻之间就将辰堂众人都圈在了崖上,却要一个一个杀死,不放一人漏网。

韦膺本来已经闭目待死,岂料身前响起惨喝声,声音十分熟悉,睁开眼睛,却见两个心腹血卫被凌空扑来的凌羽斩杀,虽然早已心灰意冷,也不由生出恨意,拔剑还击,只是却已经太迟了,只是勉强接下了凌羽一剑,便被震退数步,眼前一花,凌羽手中的利剑已经指向他的咽喉,虽然距离还有丈余,可是韦膺只觉那一剑威势已经将自己所有后路全部阻住,不由苦笑,想不到自己竟连凌羽一剑也没有接下。正在这时,却见一人舍下自己的对手,猛然扑在韦膺身前,身形还未冲到,便被他的对手,一个四十多岁的中年女子顺势挥剑掠过背脊,顿时鲜血横流,可是那人却是悍不畏死,竟是张臂向凌羽冲来。那人身上皆是鲜血,形容狼藉,凌羽生性爱洁,纵然恨极韦膺,也不由闪身避开,反手一剑,剑芒如虹,刺穿了那人胸口,那人再也支撑不住踉跄跌倒,凌羽正欲补上一剑,眼前剑光一闪,只得退后避开,抬眼看去,却是韦膺满面寒意地站在那人身侧。

韦膺目中透出古怪之色,低头看向那人,冷冷道:“你为何要舍命救我?”

那人却正是崔庠,他艰难地答道:“我知道首座素来对我有些疑心,今日更是看得明明白白,只是崔庠自认从未有过异心,却无以自白,唯有一死明志,还请首座保重。”话音方落,已经瞑目长逝。韦膺怔怔地望着崔庠,目中露出愧悔之色。耳边却传来凌羽嘲讽的话语道:“韦膺,你的胆量哪里去了,莫非只能说些大话,或者让别人替死么?”

韦膺心中涌起杀意,缓缓抬起头,对于四周的惨叫声仿若未闻,冷冷道:“韦某原本想着早死早超生,反正凤仪门也已经日暮西山,便也懒得和你们这些妇人女子动手,不过现在韦某倒想再多一个人陪葬,不知道凌门主可有兴趣和在下并骨仙霞,也为人间留下一段佳话。”

面上露出暴戾之色,凤仪门弟子本就最恨别人将她们当成无用女子看待,凌羽心中越发恨意滔天,更恶韦膺至今仍然言语轻薄,不由冷冷道:“你也配和本座同归于尽么,你放心,我定不会随随便便杀了你,待本座将你生擒之后,将你千刀万剐,若不让你死的凄惨无比,我也枉为了凤仪门主,师尊传人。”

韦膺心知自己本就不是凌羽对手,这些年来自己沉迷仇恨,虽然武艺精进许多,但是比起埋头苦练剑术的凌羽,必然不值一提,只是此刻他却毫无惧意,长剑一举,神色穆然,周围尽是剑光血影,烟霭沉沉,惨红的夕阳照在他面上,越发像是血色,韦膺面上露出似笑非笑的神色,朗声笑道:“那么就看凌门主有没有这个本事了!”他话音未落,凌羽已经挥剑刺来,剑气如霜,人美如玉,剑势更是灿如晚霞,华丽庄重,纵然是韦膺也觉得目眩神迷,虽然他剑术不如,可是也看得出只怕凌羽剑术已在门中公认第一的燕无双之上,越发明白这女子的隐忍狠毒,想来若非到了今日境地,这女子还会继续隐瞒自己的造诣吧,淡淡一笑,也不忧虑生死,移步出剑,他的剑术也曾受过凤仪门主指点,虽然不如凌羽嫡传,可是若是有备之下,倒也不会一败涂地,两剑相接,瞬间已经交击数次,铮铮剑鸣,便似龙啸凤吟一般,剑华如练,倒似是旗鼓相当。

对于山崖下面的事情,此刻双方都已经无心理会,只顾互相厮杀,一番苦战之后,韦膺手下死伤殆尽,凤仪门弟子却也又死了三人,只有韦膺仍在和凌羽激战之中,不过凌羽已经占了上风,只是见其余仇敌都已伏诛,便故意放缓攻势,只是寻机在韦膺身上刺上一剑,却不伤他要害,剩下的十余凤仪门弟子对这种残虐手段也不觉得过分,这样的事情从前也不是没有做过,更何况韦膺还是毁去凤仪门根基的死敌,所以只是将四周围住,提防韦膺舍命突围,竟是存心要把韦膺折磨至死。

身上皆是剑伤血污,再也没有昔日贵公子的气度风采,韦膺目中却始终宁静平和,仿佛周身剑伤并不存在一般。不过他心中也隐隐有着疑惑,按照他的判断,当日乔园之事恐怕也有江哲插手,否则不会是这样的结果,尚维钧、凤仪门和南楚义士两败俱伤,欧元宁被神秘高手所杀,凤仪门死了两大高手,在他想来若是大将军肯逃生,恐怕已经鸿飞冥冥了,再加上后来石观的“重病身亡”,陆云的神秘获救,怎么想来都觉得只有江哲占了便宜。而且和江哲作对多年,韦膺更是隐隐觉得这其中有江哲行事的风格,只恨自己却无能插手,也无法插手。不过若真的如自己所想,韦膺更是确信江哲不会任凭陆氏母子陷入绝境,所以他在未竟全功之后也没有沮丧,只因他相信江哲定然安排有人窥伺,绝对不会放过铲除凤仪门的大好机会,可是直到如今仍未见影踪,莫非自己猜错了么?想到不能亲眼见到凤仪门彻底覆灭,韦膺心中一冷,再也不愿苦苦挣扎下去。

这时候,凌羽正一剑点向韦膺小腹,却只准备轻伤他一剑,孰料韦膺目中寒光一闪,竟是挺身而上,那利剑瞬间插入他腹中,凌羽大惊,只道韦膺有心求死,连忙抽剑,提防韦膺速死,岂料竟被韦膺用左手牢牢抓住,不由露出惊容,韦膺却抬头一笑,血污的面容竟显得飘逸非常。凌羽心中一寒,韦膺已经如影随形扑了过来。凌羽毕竟养尊处优多年,一时之间想不到弃剑后退,只是一怔之间,韦膺已经贴身抱住凌羽。围观的凤仪门弟子同声大哗,剑光一闪,韦膺左臂已经被斩断,可是韦膺却舍命向崖边冲去,避开了斩向右臂双腿的剑光,只是在上面留下了三道深深的剑痕。被他紧紧抱住的凌羽大骇,拼命挣扎,但是她毕竟是女子,先天力弱,更何况就在韦膺冲到没有人把守的悬崖边上的时候,凌羽觉出韦膺腰间突然多了尖锐之物,没入自己体内,却是被韦膺腰带上暗藏的突刺利刃所伤,不由尖声痛呼,失去了壮士断腕的机会,只是扎眼之间,韦、凌两人已经投向山崖下面去了。凌羽耳边听到风声阵阵,五官七窍都感觉到毒烟侵入的异样,然后便是狠狠撞击到山道后,周身筋骨折断的剧痛如同海浪一般滚滚袭来,令她立时失去了知觉。

崖上凤仪门弟子面面相觑,想不到韦膺竟能咸鱼翻身,拖了凌羽陪葬,不说山崖之高,只是下面的毒烟就可葬送凌羽的性命,纪霞见状,厉声道:“别着急,等到烟散之后,我们再下去寻找门主尸体。” 此刻众人之中,只有纪霞身份最高,众皆默然点头,见状纪霞心中一喜,但是想到凤仪门势力尽毁在此,却也不禁惆怅难言,正欲下令寻个地方暂避,四周渐沉的暮色中突然传来冷笑声道:“贵妃娘娘,好久不见了。”

纪霞大骇,闻声望去,暗处突然有人点燃了火把,然后火光一点点亮起,或远或近,却将此处隐隐围住,不多时四周皆是一片光明,纪霞一眼便看到明亮的火焰下,一个相貌俊雅的男子负手而立,一身锦衣,玉簪束发,风姿翩翩,火光下越发显得俊美如玉。四周更是身影重重,将逃生之路全部挡住。

纪霞骇道:“夏侯沅峰,你怎会在此,这不可能!”

看着纪霞歇斯底里的模样,夏侯沅峰微笑道:“贵妃娘娘,不,娘娘的封号早已被除去,应该称您纪夫人才是,下官乃是奉了圣命,不辞辛苦深入南楚,若是凤仪门不除,皇上始终不能安枕,昔日之事,你们不会忘记,皇上也不会忘记,所以我虽忝掌明鉴司,也不敢在长安享福,只能前来送娘娘一程,只是想不到已经有人先动手了,倒是省了本座许多时间。”

纪霞只觉心灰意冷,手中长剑几乎跌落,但是转念之间,她便振奋起来,厉声道:“大家随我突围,现在是晚上,他们要想一网打尽,没有这样容易。”

说罢举剑冲上,她素来知道夏侯沅峰明哲保身的性子,所以索性便向夏侯沅峰冲去,想要迫他闪避,好趁势冲出去,岂料还未冲出三步,耳边便响起连绵不绝的弩弓响声,她全然不顾一切,向前扑去,那些弩箭几乎是追逐着她的影子而飞舞,就在她将要冲到夏侯沅峰身边的时候,火焰下白影一闪,一个雪衣人站在夏侯沅峰前面,一掌向前轻拍,纪霞苦战大半日,早已经是强弩之末,方才不过是最后的余勇,几乎是没有任何反击的机会,便被那人一掌切在了心脉上。纪霞缓缓倒向地面,难以形容的松弛感觉袭来,她突然想到,若是早知道死亡并不可怕,自己是否还会挣扎求存这么多年?已经听不见同门的惨叫声,纪霞唇角露出一丝疲倦的笑意,缓缓沉入黑暗深渊。

过了片刻,夏侯沅峰借着火光一一监视十几具尸体,有的是被弩箭射死,有的是死在刀剑之下,其中更有五人几乎破阵而出,却被雪衣人一一击毙,不由露出满意的笑容,转身向那雪衣人一揖道:“多谢四公子援手之恩。”

那雪衣人英俊的面容却有几分无趣,淡淡道:“想不到竞没有费多少力气,早知如此,秋某也真不必跑来这一趟。”

夏侯沅峰笑道:“四公子过谦了,若非四公子这样的身手,谁能一路上将各方势力的动静探听得一清二楚,方才我们岂能这般轻松地围歼凤仪门余孽,四公子之功,在下定会禀报皇上知道。”

秋玉飞冷冷道:“我也不希罕什么封赏,你别多事就行了。”说罢转身向黑暗中走去,转瞬身形消失不见。夏侯沅峰目光闪动,似乎有些不解秋玉飞的话中之意。良久,他神色平复下来,下令道:“山风已经驱散毒烟,你们下去将凤仪门的尸体全部验过,还有别忘了将韦膺的尸体也捡出来,他这次可算是立下了大功,若没有他,凤仪门也不可能这么容易被全部歼灭,而且他也是皇上留意的人,生死都要有个回报。”

想到若非韦膺用诸般计策,将凤仪门诱入死地,若是仅凭自己施展手段,必然很难避过凤仪门的耳目,将她们一网打尽,心中存了感激之意,决定将韦膺尸首好好安葬起来。

明鉴司众人见下面毒烟果然已经散尽,便拿了火把下去检视,不多时,有人上来对夏侯沅峰禀道:“大人,陆夫人一行和那些南楚江湖人物有二十余人不见了。”那人目光闪烁,担忧受到重责。孰料夏侯沅峰这才放下心来,他得到江哲传信,让他派人和司闻曹一起南下铲除凤仪门,他觉得这是难得的功劳,所以就借口司闻曹忙于军务,自行率人南下,果然立下大功,将凤仪门全部铲除。这里发生的诸般事情他都已经从秋玉飞口中得知,只是为了一举成功而迟迟不出手,一想到陆夫人可能死在毒烟之下,若是江哲怪罪下来,虽然不是自己所为,也不由心中惴惴,直到此刻他才放心下来,猜测定是江哲属下所为,不由惊佩万分,想不到如今已经病倒在楚州的那人,竟还有如此通天手段。

这时,另外一人匆匆上来,在夏侯沅峰耳边低语几句,夏侯沅峰心中一动,疾步走下崖去,绕到下面山道,也顾不上火光下修罗场一般的景象,目光落在了被几个属下抬过来的男子身上。那人身上皆是剑伤,皮开肉绽,血污满身,右臂已经被砍断,就连双腿也是软软下垂,显然腿骨已经折断了,但是夏侯沅峰仍然可以发觉那人胸前仍有起伏,竟然还有一丝气息未绝。

思索片刻,夏侯沅峰轻轻一叹,取出一粒丹药,塞到那人口中,又接过水囊灌了他几口水,过了些时候,那人一声呻吟,竟悠悠醒转过来。夏侯沅峰又是一声轻叹,道:“韦兄,多年不见了,你可还记得小弟么。”

韦膺睁开眼睛,只觉得周身剧痛难当,身体四肢似乎都已经不是自己的了,面目双眼更是被鲜血蒙蔽,视线模糊,根本看不清面前火光下站立那人的相貌,可是一听到夏侯沅峰的声音,他几乎是立刻辨认出来说话之人的身份,忍住痛楚,他平静地道:“有水么,扶我起来。”

那人一声轻叹,俯身将他搀起,韦膺勉力移动了一下右臂,虽然疼痛,但是感觉却渐渐回来了,他伸出手,那人倒了清水在他手上,他掬水洗去面上血污,露出清雅俊秀的面容,虽然面上仍有刀痕剑伤,更是有许多岁月的痕迹,可是当他微笑着看向夏侯沅峰的时候,夏侯沅峰只觉得眼前仿佛出现了幻影,眼前这个韦膺好像非是垂死之人,却还是昔日先帝面前雍容俊雅的相国公子。想起从前御前演武之事,恍如昨日,夏侯沅峰面上不由露出迷茫怀念的神色。良久,夏侯沅峰叹息道:“韦兄可还有什么心愿未了,只要不和天意相违,在下必会尽力。”

韦膺游目四顾,淡淡问道:“陆夫人可死了么?”

夏侯沅峰目中闪过惊异之色,道:“没有,陆夫人影踪不见,想来已经脱险了。”

韦膺露出笑容,松了口气道:“这当真是我能听到的最好消息,这样我纵然死了,也不会无颜去见大将军了。”抬头看向夏侯沅峰,双眸映着火焰,越发流光溢彩,全不似将死之人的黯淡,笑道:“十三年前朱雀门外演武,我、你还有秦青便是其中佼佼者,只可惜秦将军死在猎宫之变,我如今也要去了,只有你仍然活在世上,却也是不能堂堂正正站在朝堂之上,想到你我三人光彩,皆被一人夺走,你可还有恨意。”

夏侯沅峰见韦膺气息渐弱,也不拖延,坦然道:“怎么不恨,我夏侯沅峰素来自负,当年大雍才俊,除了韦兄之外,别人都不放在眼里,可是江随云一到大雍,我们便都逊色许多,怎会不嫉恨于他。可是我素来识时务,那人若论才智手段,可算是天下第一人,当断则断,当留情处便留情,这般心志机谋,我自愧不如,所以自然也就服气了,或者还有些嫉恨,可是我却不会破坏自己的锦绣前程,和他作对。”

韦膺闻言笑道:“好,好,我当初若看得透,也不会有今日的下场,你我也算旧友,既然你有这样的心意,我也不会矫揉造作,韦某此生做下许多错事,回想起来往往痛悔不已,如今葬身异乡,也是咎由自取,与人无尤,拜托你将我的尸骨焚化成劫灰,一半带回长安,我无颜葬入韦氏祖坟,请你将我埋在可以望见先父陵墓的山岭之上,让我可以在九泉下替父亲守陵,以赎我不忠不孝的罪愆。”

夏侯沅峰默默点头,道:“这件事情没有问题,韦兄你虽然犯下不赦之罪,可是你今日痛改前非,和凤仪门同归于尽,又只是要求归葬故土,皇上就是知道也会默许的。那么韦兄你另外一半骨灰要如何安排呢?”

韦膺眼神渐渐涣散,他沉声道:“韦某叛国逆伦,世人不齿,只有南楚大将军陆灿信我用我,此恩此德就是粉身碎骨也难以报还,如今我辜负了他的厚爱,就要葬身仙霞,请将我的另一半骨灰洒到大将军坟上,韦某就是死了,也不忘他的恩义。”

夏侯沅峰闻言愕然,良久叹道:“陆灿能够得到韦兄这般忠心以报,定是当世英杰,可惜我竟未能亲见此人一面,只怕会留下终生遗憾。”说罢他缓缓摇头准备离去,韦膺此时气息将绝,他知道此时韦膺已是听不见自己的声音,更是看不到自己的面容了。

韦膺眼前已经是一片黑暗,他知道死亡即将到来,可是他心中却再没有一丝怨念,不由放声高歌道:“泻水置平地,各自东西南北流……”他意中是在高歌,但是实际上声音却微弱非常,刚唱了两句,声音便已突然断绝。

夏侯沅峰不由回头望去,只见韦膺气息已绝,面容却是分外的平静祥和。

\chapter{第四十三章 所恨不相识}

呵呵,今天好不容易抽出时间找了一个可以上网的地方,将这一周的分量都解禁了,在外面的日子真辛苦啊!不过还要继续请假啊,因为还不知道什么时候可以回去。

——————————

既返,乃卧病不起,以楚州战事将起,舆送徐州将养,经年乃愈,自此无心俗务,上书请骸骨,雍帝不许。

——《南朝楚史·江随云传》

就在韦膺和凌羽在崖上苦战的时候,崖下山道上已经是一片寂静,崖上众人都以为毒烟肆虐,再无劫余,所以全没留意下面动静,却不知道未散去的毒烟中别有洞天。当那扮成禁军军士之人冲到陆夫人面前自曝身份的时候,正是韦膺发动毒袭之时,毒烟四溢,遮天蔽日,尚未波及崖底,陆夫人这些体弱的女子已经摇摇欲坠,那军士也顾不得解释,从怀中取出一支玉瓶,倾出一些药丸来,急急道:“这是寒园秘制的药物,可解百毒,夫人快些服下。”

陆夫人此刻已经想得明白,这人定是江哲派来保护自己母子的高手,虽然身为南楚贵妇,可是陆夫人却是经常听到丈夫谈及江哲,所以对于江哲并没有过分的排斥,可是心念一转,想到若是服了解药,便是承受了大雍之恩,目中闪过犹豫之色。

这时陆氏众人虽然都接过了解药,目光却都看着陆夫人,等她之命,这时候毒烟已经弥漫过来,诸人皆是摇摇欲坠,但便是几个未成年的幼童,也不曾服下解药。那军士见状心中也是感叹不已,却不曾出言相劝,他正是八骏之一的渠黄,素来知道江哲和陆灿之间师徒情谊最为深厚,虽然中道分道扬镳,仍然互相牵挂,这次江哲更是为了陆灿之死一病不起,故而渠黄虽然也是敬重陆灿为人,却也心生妒意,所以他也故意不多言,有心相试陆夫人。

陆夫人目光一闪,眼中露出痛色,将解药纳入口中,见她接受,陆氏众人才各自服药,却有两个幼童已经无力服药,在旁边同伴相助下才服下了解药。

渠黄见众人都服下了解药,心中一宽,这种解毒药十分珍贵,就是八骏等人,身边最多也只有十粒八粒防身,这一次先生却令人额外送来二十粒备用,原本自己还以为没有必要,想不到真的用上了,要不然自己身上的解药可是绝对不够用。看看瓶中还剩下的七粒解药,渠黄微微摇头,便欲收起。

这时候丁铭已经到了近前,练武之人若遇危险,第一反应就是自保,毒烟一落,他便屏气相抗,又立刻服下了身上常备的一些解毒药,只是却不甚对症,收效极微,眼看身边血战余生的同伴中毒而倒,丁铭却无能为力,幸好这时候凤仪门中人也已经慌乱起来,丁铭便发出突围的命令,等到他率人退到山崖之下,想勉励支撑带着陆氏众人冲出去,却见到陆夫人等人安然无事,方才后面的变化他没有留意,此刻见到这般古怪情形却是一愣,心神一震,便决有些摇摇欲坠。

服下解药的陆夫人虽然仍觉有些恹恹,但是却已经没有胸闷昏眩之感,见到丁铭等人来到,连忙问渠黄道:“请问阁下可还有解药么?”

渠黄目光一闪,见到丁铭和身后数人强自抑制呼吸的神情已经微黑的面色,想到他们不谋求脱身而是先来救人,心中生出敬意,想到并未得到命令定要将他们一起葬送,轻叹一声,将剩下的药丸递了过去。丁铭见陆夫人安然无恙,也知道这药丸有效,虽然不知道这军士如何有解药,却连忙接过分给众人,只是药丸已经只剩七粒,包括丁铭在内,却有八人撑到现在,丁铭略一犹豫,便迅速将最后一粒解药纳入身边一个已经接近昏迷的同伴口中,自己却因为屏气过久,已经面红耳赤,支持不住,忍不住呼吸了半口毒烟,顿觉天旋地转,冷意涌上周身。身形一软,却被一人扶住,继而一粒药丸塞到他口中,过了片刻,他渐渐清醒过来,只见那相貌平平的禁军正目光迥然地望着自己,不由低声道:“多谢阁下救命之恩,阁下是什么人?”

渠黄轻轻一叹道:“丁大侠不要怪我才是,解药已经没有了,我给你服下的是以毒攻毒的药物,不论是什么剧毒都可以压制一些时日,只是事后若不得名医诊治,只怕性命是保不住了,我的身份也不怕告诉你,在下渠黄,乃是江侯记名弟子,这次奉命保护陆夫人一家南下,因为不便让陆夫人知道,所以在下设法让一个禁军不能前来,顶替他混入押解的禁军之中,如今迫不得已泄漏了身份,丁大侠需念同舟共济之情,等到度过难关再计较此事如何?”

丁铭心中虽惊,却隐隐觉得理应如此,楚乡侯江哲广陵拜祭之事江南皆知,如今陆灿已死,江哲与陆氏敌对之势已经不再,那么出手维护陆氏后人也是理所当然,虽然对这自称渠黄的军士深藏不露的手段仍有戒心,可是当前却也顾不得这许多,低声道:“上面正在厮杀,我们护着陆夫人先离开此处如何?”虽然听到崖上语声,他已得知韦膺同归于尽之意,可是想到韦膺不分敌我的行径,心中仍有余恨,也不愿上去相助,更何况他虽然暂时压制了毒性,但是气力不济,众人虽然已经解去剧毒,仍有气血翻涌之感,更是无法作战。诸人商议一定,便在渠黄引领下潜行离开此地,这时候山道上都是仆倒在地不知生死的凤仪门弟子,丁铭等人心中暗惊,若非有渠黄相助,只怕他们也不可能逃出毒烟加害。到了此时众人虽然仍有戒备,却也不便流露出怀疑之心,跟着渠黄走去。

走了许久,这时候天色已经几乎全黑了,山路艰险南行,一片黑暗之中,虽有丁铭等人护持,也难免失足,渠黄见已走出很远,便从怀中取出一串夜明珠,珠光不甚明亮,若是在远处必然难以察觉,可是却可照亮身边丈许方圆,只是这串夜明珠已经是贵重无比,更何况那串明珠每一颗都一般大小,浑圆晶莹,毫无瑕疵,当真是价值连城。丁铭等人初时都觉目眩,但是毕竟众人都是心志坚毅之辈,否则也不能生出绝地,清醒过来,却不明白这人为何取出明珠炫耀。渠黄似乎全没留意众人神色,扯断珠串,将夜明珠分与众人,然后当先走去,众人才明白渠黄之意。走在最后面的丁铭心中感叹,虽然只是借出明珠照明,但是有这般豪气雅量,就是自己见惯南楚英雄,也觉得心折,这人寂寂无名,却有这般气度,大雍能够席卷天下,想来也是理所当然。

走了没有许久,渠黄便带着众人走入一个山谷之中,只见那里已经立下了数座营帐,泥土痕迹仍新,显然是刚刚搭建好的,恐怕还不到半个时辰,营帐之中,已经备好寝具热水,和热腾腾的食物,却是连一个人都没有。渠黄便请众人入内休息,丁铭皱眉不语,此人竟在此地准备妥当,莫非自己的遭遇都在这人计之内中,但是此时却不便多问,任凭渠黄指挥调度,只觉这人相貌平平,看似寻常,可是见他气度从容,指挥若定,看来他自称是江哲弟子,其中并无虚言。

这时陆夫人帐中突然传来惊喜的呼声,丁铭心中一震,顾不得大防,急步过去,掀开帘幕,只见陆夫人怀中抱着陆霆,泪流满面,陆霆气色好转许多,正在用小手擦拭着娘亲面上的泪痕。

丁铭心中也是惊喜万分,却急忙退了出去,正好见到渠黄微微而笑,正欲相问,这时候苦竹子冷笑道:“莫非你们又和韦膺联手了么,难怪韦膺要和凤仪门火并呢?”

渠黄面色一寒,别有一种冷峻气势,淡淡道:“阁下说哪里话,韦膺乃是叛国臣子,我家先生怎能和他合作。只不过先生派来的人极多,早就缀上了韦膺,不过是寻机将陆公子救了出来罢了,若非在下得到同伴传讯,得知毒袭之事,也来不及救下诸位,陆公子之事也是路上才得到的消息,已经有人替他诊治过了,药方就在帐内书案上,药物也已经备好,可以令陆夫人侍女煎药给他服下,想来可以免去陆公子水土不服之苦。”

苦竹子愕然不语,丁铭叹息道:“江侯爷果然手段惊人,难怪我总是听到路边崖上有鸟鸣之声,更隐隐觉得暗中有人窥伺,想来此处都已经在阁下掌控之中了?”

渠黄冷笑道:“南楚江湖中人,最爱以小人之心度君子之腹,侯爷是何等样人,岂会乘人之危,你们这些人还不在他眼里,侯爷苦心孤诣,在下以身犯险,不过是为了陆夫人母子的平安罢了。”

丁铭默然,心知这人所说皆是实言,自己这些人何曾会被大雍重臣如江哲者看在眼里,但是若是陆夫人母子被雍人控制,必然会影响南楚士气,可是他却也不能提出什么异议,一路以来的生死挣扎,足以令任何人裹足不前。

这时,两人身后传来一个温婉坚定的声音道:“江侯爷好意我母子感激不尽,只是先夫早有训诫,未亡人也不能为了性命投靠敌国。”

两人闻声回头,只见陆夫人站在不远处,神色平和,彷佛所说的只是寻常言语,而非是将生机轻轻放过一般。

渠黄心中越发生出敬意,上前施礼道:“夫人,南楚已经不是乐土,定远更是瘴疠之地,夫人和小公子都是千金之体,岂能淹留险地,侯爷已经安排妥当,只要夫人愿意,便可扬帆直上北地,侯爷亦可许诺,绝不会利用夫人和公子的身份做出有害南楚的事情。”

陆夫人淡淡道:“侯爷金诺,未亡人自然是信得过的,想来如今大雍也不需利用孤儿寡妇招揽人心,只是陆氏乃是南楚的臣子,便是死也要死在南楚,朝廷虽然辜负忠良,可是陆氏绝不会辜负朝廷,定远虽然是险恶之地,可是既是朝廷之命,未亡人也不能违背旨意。”

渠黄肃然道:“陆氏忠烈,在下敬服,只是南楚昏君奸相自毁长城,不念忠诚,夫人又何必对这样的朝廷念念不忘呢,侯爷并非是希望夫人背叛故国,只是念在古旧师生情分,不愿大将军身后凋零罢了。”

陆夫人裣衽为礼道:“阁下不惜生死,冒险犯难,搭救未亡人与妾身幼子,这等恩情妾身感激不尽,便是阁下要未亡人以死相报,妾身也不会有何怨言,唯有此事万万不能,先夫为了忠义二字,不惜以身相殉,妾身不敢说继承先夫遗志,但是却也不能舍弃家国,苟安于世。”

丁铭闻言,上前一揖道:“夫人之言,仿若醍醐灌顶,大将军殁于奸相之手,我等都觉心寒,更有许多义军志士弃军而走,今日听到夫人之言,才知我等都不如夫人深明忠义之理,在下如若生还吴越,必将夫人言语传示众人知道,纵然死在沙场,也绝不会放任雍军铁骑南下。”

陆夫人目中隐隐有泪光,道:“先夫若知丁大侠这般想,定当瞑目九泉。”

渠黄面上神色变幻不定,良久才道:“丁大侠可知道性命尚在我等掌握之中,纵然在下任凭大侠返回吴越,阁下身上的剧毒仍未解除,能够医治阁下的岐黄圣手多半都在大雍,不需我们多费心思,阁下也是性命不久。”

丁铭坦然笑道:“能够多活这许多时光,已经是阁下厚赐,虽然人多贪生而畏死,可是若是阁下以死相迫,却是小瞧了在下了。”

渠黄闻言微微一笑道:“人不畏死,奈何以死惧之,丁大侠却也太小觑了在下了。此间事了,在下便要回去复命了,段约虽死,他身上的文书我已经取来,丁大侠便可以护送陆夫人到定远替他交差,至于阁下身上的隐患,在下一时也没有什么法子,不过若是阁下有暇,不妨到南闽越氏试一试。”说罢从容一揖,便向外走去,转眼之间便影踪不见。

丁铭和陆夫人都是一愣,两人都知道这人费了许多心思,都以为他不达目的不会罢手,事实上两人都已决定坦然面对任何结果,想不到这人说放手就放手,这般绝决洒脱,令人心折。两人相视一眼,眼中都有忧色,良久,陆夫人轻轻一叹,回帐去了。

渠黄的身形便如飞鸟一般在夜色中纵越,没过多久便看到前面昏黄的珠光,心中一喜,便加快了脚步,走到近前,只见一个衣衫破旧的青年立在山岭上,若非是手中的明珠闪耀,就是说他是个叫化子也会有人相信的。

渠黄见到那青年俊秀憔悴的面容,心中生出怜悯,停在那人身后,道:“逾轮,你何必这般自苦,既不肯返回秘营,又非要跟着我们南下保护陆夫人母子,难道你不怕陈爷顺便将你也杀了么?”

逾轮也没有回头,低声道:“陆夫人已经平安了么?”

渠黄耸耸肩道:“已经平安无事了,想来陈爷已经去和夏侯沅峰交涉去了,免得他趁机多事,还要为难陆夫人。逾轮,你今后有何打算?”

逾轮叹道:“我答应了大将军不再涉入两国之争,若是我留在建业,便不能避免此事,所以我索性南下护送陆夫人母子,若能护得他们平安,也算是不枉和大将军的一面之缘,如今既然已经没有事了,请替我将明珠交还给白义,我这就走了,也不和他道别了。”

渠黄叹息道:“你这人还是这样古怪,以前你说要回建业,所以不肯留在秘营,现在你也不回建业了,为什么还不肯回来呢?”

闻言,逾轮面上突然露出尴尬之色,渠黄和他十分熟稔,心中一动,上前道:“逾轮,你有什么心事,不能跟我说么?我们可是多年的手足兄弟,你不如说给我听听,说不定我能替你拿个主意。”

逾轮犹豫半晌,终于呐呐道:“我原本以为只是将她当成替身,可是这些日子我心中总是想着她。”

渠黄心中一乐,道:“原来你这浪子也动了心了,可是那位柳姑娘,你在她身边做了许久琴师,原来是情之所衷,不能自已。既然如此,为什么不去快向她求婚,窈窕淑女,君子好逑,柳如梦乃是江南花魁,品貌才艺世间少有,你的人品才华也是罕有匹敌,正是天生一对佳偶,若是觉得没有丰厚的聘礼,不敢出口,我们这些兄弟助你一臂之力,绝对让你风风光光地迎娶佳人。”一边说着,一边在苦思,逾轮所说的替身是何意。

逾轮不知他心思,黯然道:“我纵有此心,也不敢说出口,如梦她最慕忠烈之人,大将军便是其中之一,若给她知道我也有份陷害大将军,只怕她不会原谅我的。”

渠黄神色一动,展颜道:“你这是当局者迷,陆灿之死,还是尚维钧所为,你不过是推波助澜,还是奉命行事,这罪责与你何关,反而你也曾出手相救陆灿,如今又南下保护陆夫人母子,柳姑娘若是知道,只会敬佩于你,更何况你献策之事除了尚维钧父子也没有别人知道,只要你不说,谁会知道呢?”

逾轮神色郁郁,只是摇头道:“若要人不知,除非己莫为,终究是瞒不过人的。更何况我当日见到大将军自尽,便常想着,若是我和他原本相识,无论如何也不会进言害他。”

见他这般悒郁,渠黄叹道:“当真是可惜得很,我在江南多年,虽见过其人,却不曾真的相识,不过是一面之缘,你便为他愧悔伤怀至此,想来这人定是当世第一流的人物吧!”

逾轮淡淡道:“若论才能本事,自然不及先生,可是若论胸怀心志,当世无人能及。”

渠黄神色微变,良久才道:“先生已经决定不再过问世俗之事,天机阁也将烟消云散了,你若还要留在江南,只怕我们也很难护着你了。”

逾轮没有作声,目光中满是冷淡漠然。

\chapter{第四十四章 风流云散}

十五年,春夏之交,雍军攻巴郡甚急,余缅内惧尚相加害,外苦雍军势强,乃生降敌之意,使节往还,渐泄于人,事未成,有密使呈公故剑并书信,余缅览书而羞,愧悔无地,拔剑欲捐生,为心腹所阻,乃绝雍使,自誓与城偕亡。九月,巴郡为雍军所破,缅乃伏剑而死,以全其誓。公之余威至此矣。

十六年冬,雍军尽据江北之地,挥军欲渡长江,求和未许,国中皆惊惧,国主下罪己诏,欲得将士死力,诸将皆力白公冤,慷慨陈辞,直入禁中,国主悔之莫及,乃除维钧相位,诏复公爵,以礼改葬,建庙于江夏,谥忠武。

公元配吴氏,大家女也,忠烈端肃,持家严谨。公受诬入缧绁,夫人先得讯,乃散仆婢,从容若定。

即公殁,家人远徙,夫人以弱质入瘴疠之地,持家教子一如平常,十六年春,定远流疫肆虐,夫人采药制丹,不辞辛苦,遍走乡里传方救人,赖夫人赠药而生者以万千计,人皆呼以“娘娘”而不名。

十七年春,楚亡,雍帝感公忠义,乃遣使赴闽,诏夫人赴长安恩养,夫人拒之曰:“先翁先夫皆楚臣,妾亦楚臣,不敢受大雍诏令。”帝叹息不已,乃止,亦不加罪。

夫人居闽几二十年,卒于汀洲,及逝,诸子奉灵柩返江夏,并公合葬。闽人念夫人恩义,立衣冠冢于定远,至今香火不绝。

论曰:自晋亡后,诸国争雄,天下纷乱,其中佼佼者,唯雍、楚、汉也,求善战名将,多不胜数,求其文武全器,忠义并举者,一代岂多哉。公以弱冠少年,履挫强敌,千里转战,鲜有一败,战法军略称雄足矣,此仍不足为公誉。公北上欲还襄阳,战未成而受诏班师,泣于风中,忠贞之言,出于肺腑,而王上不察,论以逆罪。时,公掌虎符而御三军,威势冠于群伦,而束手就缚,从容赴死,此诚难矣!且公一门皆忠烈,及楚亡,雍帝选俊才入仕,楚人从者如流,皆忘故恩,帝以显爵诏陆氏入朝,公诸子皆不仕,忠义若此,而愍王杀之,呜呼冤哉!呜呼冤哉!

——《南朝楚史·忠武公传》

寒风瑟瑟,虽然已经初春时候,但是犹有残雪未融,陆风坐在毒龙泽湖边青石之上,抱膝枯坐,神色一片茫然,自从他被兄长相迫从钟离逃出之后,只觉天下之大,自己却是无处可去,所以韦膺派人寻他的时候,他并未反对韦膺的安排,辗转数处之后,他便被送到了这几乎与世隔绝的所在。

毒龙泽本是淮水下游的一座湖泊,绵延十余里,养育了一方沃土,可是数十年前,发生了黄河夺淮的洪灾,毒龙泽不再有淮水汇入,渐渐便被淤泥堵塞,如今已经成了沼泽地,方圆二十余里之内又都是沙土地,五谷不生,也就渐渐没有了人烟,正是因为这个缘故,韦膺才在距离毒龙泽数里之外建了秘舵,又在毒龙泽之内准备了藏身之处,为的就是一旦发生变故可以避敌其中。

陆风被送到此处之后,若有闲暇便在泽边练习剑术,这是韦膺特意留给他的剑谱,或者是担心他无所事事吧,陆风也知道将来道路艰难,所以练剑倒也是十分用心,何况若不找件事情来做,让他如何排遣心中苦痛,父亲被害,亲人零落,自己却无能为力,这种境况非是寻常人可以承受的。

可是陆风却真的什么也不能做,纵然想要起兵报仇,一来父兄有命,不许他这样做,二来他年纪尚轻,在父亲旧部中并没有什么威望,若是兄长陆云自然不同,振臂一呼,必会从者如云,心中的无力感让陆风渐渐憔悴消瘦,明明是青春年华,却是暮气沉沉。

不知待了多长时间,天色渐渐昏暗,寒风愈冷,陆风站起身向住处走去,离那几间茅屋还有几十丈远,陆风突然觉出风中有淡淡的血腥气味,心中一凛,握紧了佩剑,放慢了脚步,仔细瞧去,平常这时候,茅屋里面应该有炊烟升起,可是今日却是不见,而且堂屋的房门虚掩,未曾紧闭,这也是有些异常。

陆风深吸了一口气,状似不知情的模样走向茅屋,口中高声叫道:“赵叔,我回来了。”好似没有戒心一般地推门向堂屋内走去,就在他挑帘而入的瞬间,眼睛余光瞥见一缕剑芒无声无息地袭来。陆云心中早有准备,向下仆倒,翻身向上,右手一挥,三支袖箭射向偷袭之人。那人一声惊咦,长剑回挽,三支袖箭皆被拨开。陆风已经纵身而起,盯着那人。

那人是一个女子,虽然相貌端丽,可是鬓发星霜,眼角鱼尾纹清晰可见,虽然难以揣测,可是陆风可以肯定这女子年纪肯定已经不小了。那女子目光炯炯,淡淡地瞧着陆风道:“好机灵的小子,你既然知道有了变故,为什么还要冒险进来呢?”

陆风深吸一口气,道:“我发觉异常的时候,已经在你视线范围之内,若是我当时逃走,虽然可能免得一死,却是没有机会知道是谁要杀我,所以我才冒险回来,可是你武功这样高,看来我是自投罗网了。”

那女子冷冷一笑,道:“若非是那四个废物还有几分本事,迫得我见了血,也不会被你发觉有异,不过你进不进来都没有什么关系,只是这样却免了我的奔波,见你还有几分聪明,我就给你一个全尸吧。”说罢,那女子手中长剑轻轻刺来,虽然剑势缓慢,可是陆风却觉得那长剑仿佛将自己的逃生之路全部封住,这一剑他认得,韦膺给他的剑谱上面有这一式“不战而屈”,越是精通剑术之人,往往生出不能反抗之感。若是这女子用了别的招式,陆风或者只能拼死还击,可是这一招韦膺给他的剑谱上面却有破招。

韦膺的武功虽然不如凤仪门嫡传弟子纯正,但是当初为了掩人耳目,凤仪门主将自己精研出来的一些散手剑式秘授给他,这些剑式多半奇诡狠辣,有失气度,因为不合凤仪门剑法华丽堂正的风格,所以除了韦膺之外,并没有别人得到传授。而韦膺乃是相国公子,平日结识了许多奇人异士,更在大雍御书房之内遍阅许多剑法的秘笈,后来在南楚主持辰堂,也是笼络了许多高手,留心请谊,若论剑法之博,天下无人能及,他给陆风的剑谱上面,就记录了他这些年收集的精绝剑招,还有他的一些心得,虽然杂乱无章,却是几乎尽得天下剑法精粹,所以陆风才能看到可以破解这一式的剑招。若是韦膺能够专心在剑法上面,绝不会在凌羽剑下全无反抗之力。

却说陆风心中一喜,长剑斜挑,举重若轻,便如奇兵突出。这一式“履险如夷”乃是韦膺机缘偶得的剑式,便是觉得可以破去凤仪门绝招,才记录在剑谱上,因此被陆风记在心中。那女子并不认得,若是韦膺自己和她交手,她必定小心提防,不会让韦膺轻易得手,可是陆风小小孩童,那女子全没放在眼里,这一大意之下,陆风的一剑已经击破这女子的剑势,撞碎了窗子,冲出茅屋去了。那女子顿时愣住了,她虽然已经多年不曾轻易出手,可是剑术日益精进,自负罕有对手,可是竟被这少年破了剑式。

不过她虽然失手,却立刻清醒过来,出了茅屋,便看到那少年向来时的方向狂奔,她施展轻功追去,陆风这些日子早在韦膺指点下苦练剑术内力,轻功也是大有长进,道路又是十分熟悉,那女子一时之间倒也追不上他,不过两人距离却是越来越近。

陆风只觉得胸口痛涨得厉害,却只能舍命狂奔,毒龙泽终于出现在眼前,几乎是连滚带爬地扑进了沼泽之内。就在他纵身而起的时候,耳中传来剑啸之声,然后便觉背后剧痛,当他跌落在一块坚实的空地的时候,已经痛得几乎昏迷过去,可是他也顾不得一切,一个翻滚纵起身来,向沼泽内冲去。

那女子眉头紧锁,觑着那少年的落足之处追踪而去,这少年只顾闷头奔逃,却是熟悉道路,在这随时都可能覆顶的险地往来自如,她自然不知道韦膺当初派人仔细侦测过泽中道路,陆风来此之后,几乎每天都要花些时间按照地图熟悉地形,并且随时修正地图,为的就是应对今日这种情况,每一处可以立足的地方他都记在心上,所以才能纵跃如飞。

虽然如此,没有跑出数里之路,那女子便看到那少年突然失足跌倒在地,露出冷笑,知道这少年乃是伤势过重,不能支撑了,飞身掠去,准备取了那少年性命,岂料身形刚落,耳边便传来崩簧响声,右足被什么东西夹住,那女子一声惨呼,向下软倒,就在这时,原本伏在地上生死不知的陆风已经一个鲤鱼打挺,飞纵而起,落在了数丈之外,奔逃而去。

那女子用目瞧去,却见脚踝被一个兽夹夹住,血透衣衫,稍微一动便是痛彻骨髓,知道腿骨已经被夹断了。她虽然内力精深,剑术高明,却毕竟是个女子,虽然也曾浴血转战,可是养尊处优多年,早已不能经受这样的折磨,几乎痛得昏迷过去,好不容易取下兽夹,放眼四顾,只见荒草蔓蔓,泥水泥泞,杳无人迹,只得寻了两根枯枝将断骨绑好,又找了一根树枝做拐杖,沿着来路走去,虽然只有一足便利,可是她毕竟轻功超群,倒也不至于寸步难行。幸而追进来的时候,她就硬记下路途,又有足迹可以辨认,再加上小心试探,走了大半路程,倒也平安无事,虽然断腿之处痛彻心肺,但是若不能出了沼泽,只怕就是死了也无人知道,因此她只能勉力支撑,只是越发懊悔,想不到自己竟会在阴沟里面翻了船。

正在这时,那女子突然觉出足下有异物蠕动,下意识地看去,却是高声尖叫起来,只见旁边的沼泽中竟有无数毒蛇游动,而自己足下正踩着一条毒蛇,女子畏蛇乃是天性,她吓得向旁边跃去,却忘记了这里乃是沼泽,脚下一软,已经陷入泥中,这时候她若冷静些,尚有机会逃出,可是放眼望去,却到处都是毒蛇耸动,惊骇的手足酥软,只是这样一迟疑,已经被毒蛇所啮,毒液攻心,行动不便,陷入淤泥,她的命运再也无法改变。

此刻,站在远处的陆风冷冷望着那女子拼命挣扎,渐渐昏迷,缓缓向泥中沉去,他忍着伤痛将那女子诱到自己设下兽夹捕捉泽中野兽的地方,令其重伤,脱走之后,又绕到回去的路上,掩去真正的路途,留下了伪造的足迹,将这女子诱入毒蛇聚集之处,毒龙泽的名字岂是随便叫的,终于将这女子杀死在沼泽之中。凝神瞧了许久,直到那女子没顶之后,陆风才向外走去。

虽然利用沼泽杀了强敌,但是他心中没有丝毫轻松,虽然只是交手一招,但是他已猜出这女子是凤仪门所属。他不会以为韦膺要出卖他,韦膺若想杀他,只需暗中下令给保护他的几人就行,自己必定不会防范。想来韦膺必然已经落入进退两难的窘境,想到韦膺对自己百般爱护,更是将一身所学记录成册传授自己,想到他可能的危难,陆风不由泪落如雨。好不容易走回到茅屋,寻到厢房,看到里面血迹斑斑的四具尸体,陆风更是悲从心起,这四人多日来将他照顾得无微不至,却死在那女子手中。虽然心中悲痛,但是想到敌踪不知何时会再至,陆风也不敢耽搁,寻了伤药敷了伤口,将几个血卫埋葬在屋旁,将藏在暗格中的金银秘笈带在身上,便离开了短暂的安居之处。虽然前路茫茫,但是陆风却已经有了决定,他要寻地隐居,苦练剑法,天下大势不可绾,既不能率军征战沙场,报仇雪恨,那么不如仗剑行走天下,或者还有快意恩仇的机会。

孤灯焰已昏,斯人独憔悴,燕无双倚在软榻之上闭目养神,绝丽的容颜上略带病容,面色苍白如雪,不时地轻咳几声,在旁边伺候的侍女并非凤仪门弟子,这一次南下事关重要,所以她将全部实力交给了凌羽,不是不知道凌羽夺权之心,可是若能恢复凤仪门昔日声威,她倒也不介意牺牲一些权力。当初凤仪门众弟子,便以她和凌羽最得凤仪门主器重,都有继承大位之望,但是最后凌羽得到了门主之位,燕无双心中不忿,便和纪霞、韦膺联手,分割凌羽的权势。但是比较起来,燕无双仍然是众人中最忠于凤仪门的,之所以和凌羽争权夺利,却也是为了她不信服凌羽能够撑起大局,这一次凌羽便是以大局为重的理由说服了她,才让她决定亲自出手刺杀石观,更将所有人手都交给凌羽指挥,自己留在月影轩后面的密室养病。

耳中传来脚步声,来人步履分外的匆忙慌乱,就在燕无双疑惑地睁开眼睛的同时,一个十八九岁的绝艳女子走了进来,虽然对她自己来说已经是尽力遮掩身份,可是不论是头上钗环,还是玉腕上钏镯,以及衣履裁剪质地,都可以看得出来人的身份尊贵无比,只是如今她的面上惊惶无比,扑到榻前悲声道:“师姐,不好了,大事不好了,师父他们全都出事了。”

燕无双只觉得娇躯如坠冰窟,支起病体,一把握住那女子皓腕厉声道:“灵湘,你说什么?”

纪灵湘泪流满面,将从南闽得来的消息一一说出,虽然凤仪门众人全部葬送在仙霞岭上,无人返回报信,可是陆夫人一行到了浦城之后,向官府说明了途中遇匪,禁军皆没的事情,这样的大事,自然是六百里加急报到了建业,纪灵湘身为南楚贵妃,长侍君侧,几乎是很快就得到了消息,她自己可以从字里行间猜知真相,若是凤仪门还有人在,绝不会让陆夫人一行平安到了浦城。忧心忡忡地等了数日,又从尚维钧那里得到确讯,仙霞岭上积尸如山,堆成了京观,惊骇了无数行人。纪灵湘得知凤仪门全军覆没的确切消息之后,便趁着今夜国主赵陇宿在王后宫中,私自出宫来向燕无双禀报。

燕无双只觉心痛如绞,不能自持,张口欲言,已经是一口鲜血吐出,纪灵湘连忙取了桌上的茶杯,上前服侍燕无双,燕无双略略平静下来,就着茶杯喝了两口温热的香茗,正欲抬头细问,突然胸腹间剧痛无比,愕然下望,只见一只素手紧握短剑,那短剑的剑身全部没入自己的胸口。燕无双一掌击出,纪灵湘被她推出,撞击在房门上,半晌才站了起来,口角溢血,花容如纸,大笑道:“还好,还好,师姐的伤势不轻,要不然这一掌便可取了我的性命。”

燕无双神色漠然地道:“为什么你要这样做?”

纪灵湘绝美的容颜上满是戾气,狠狠道:“因为我要活下去,我不想做你们的棋子,我纪灵湘如今已经是堂堂的贵妃娘娘,可是在你们前面却只是一个寻常卒子,我不甘心,可是我也不敢反抗,我知道你们若要我死,那是轻而易举的事情。可是如今不同了,师父和门主她们都死了,再也不能威胁我了,唯一令本宫寝食难安的就是燕师姐,你们这些人和我不一样,你们才是凤仪门嫡传弟子,一旦师父她们的死讯传回,这凤仪门就是你的囊中之物,你若想重振凤仪门,必然会难为于我,你若不想振作,也可据有千万金银。荣华富贵,谁不喜爱,我纪灵湘不想和你们这些穷途末路的人一起走上不归路,也不想放弃这诺大的财富。只要你死了,凤仪门就只剩下我和灵雨,灵雨那妮子一心只扑在音律上面,武功平平,又无权势,我要对付她易如反掌,到时候这一切都是我的。手中有这许多财富,又有义父支持,更为王上宠妃,想如何就如何,我不杀你,怎对得起自己呢?”

燕无双惨然笑道:“好,好,你够狠,不愧是凤仪门弟子,只可惜南楚江山岌岌可危,我却要看看你可以横行到几时。”说罢拔出插在胸口上的短剑,鲜血狂涌而出,燕无双玉手一挥,电闪流虹,掠过纪灵湘面颊,透入房门,纪灵湘只觉面上一凉,伸手摸去,纤指上皆是鲜血,不由大骇。凝神瞧去,只见燕无双已经闭目而逝,这才敢走到铜镜之前,仔细察看面上伤痕,幸好只是一线血痕,若是敷上宫中秘制的伤药,旬日可愈,这才放下心来。铜镜中略嫌模糊的丽人影像露出粲然的笑容,然后便是一道寒光闪过,一柄飞刀射入了躲在屋角瑟瑟发抖的侍女体内,室内传来一声短促的惨叫。

檀香袅袅,春风入罗帷,灵雨凝神抚琴,一曲《猗兰操》从指下淙淙流出,一曲终了,灵雨轻轻叹息,又忆起那自称四公子的英俊男子指点自己琴艺的情景,低吟道:“幽植众能知,贞芳只暗持。自无君子佩,未是国香衰。白露沾长早,青春每到迟。不知当路草,芳馥欲何为。(注1)”

有意无意地拂动着琴弦,忧虑从心而起,她虽然幽居楼中,不问世事,可是仍然能够感受到月影轩内外的不平静,师门长辈已经许久不见,昨日她照例去向燕首座请安,却得知燕无双已经离开了月影轩,她知道燕无双伤势很重,心中不免疑惑,轩中打理琐务的管事也都是神神秘秘的,凭她的身份,虽然一向不管轩中之事,可是若是开口相问,管事也应该回答一二,可是昨日她诘问之时,却被那些人敷衍应付,没有得到任何答案,这等诡异情况,令她也心中不安起来,今日便索性不出去待客了,避在楼中弹琴自娱。

正在这时,灵雨身边的侍女鸾儿跌跌撞撞地跑了进来,叫道:“小姐,不好了,万花楼的人来了,说是月影轩已经卖给他们了,姑娘们已经乱成一团了。”

灵雨惊愕地站了起来,走到门外,凭栏望去,只见园中果然是一片混乱,到处都是穿着万花楼服色的大汉来回穿梭,灵雨不知所措地转了几个圈子,竟想不到可以去向谁询问,想来昨日那管事吞吞吐吐的模样,定是他已经知道今日之事,茫然走入房间,跌坐在绣墩上,良久才道:“鸾儿,你去请万花楼主事之人过来,就说我有事相询。”

鸾儿慌忙应了,正要出门,门外传来一个温和的声音道:“不必请了,万某已经来了,灵雨姑娘乃是花魁之尊,万某自然应该亲自来请。”话音未息,一个华衣中年人走了进来,满面笑容,倒似是一个和气生财的商贾,绝不像是一个掌控江南风月半壁天下的大豪。

灵雨站起身,裣衽为礼道:“灵雨见过万楼主,只因心中有些疑惑,不得不请来相问。不知月影轩如何会成为万楼主的产业,虽然二娘已经过世,可是月影轩自然有人接管,应该不会落入外人之手?”

中年人叹息道:“灵雨姑娘想必还不知道吧,月影轩的真正主人已经葬身闽越边境的仙霞岭,此事已经传遍江南,月影轩已经是无根之水,万某花了五百万两银子买下了月影轩名下的全部青楼,姑娘也是其中之一,灵雨若是不信,可以看一下这些契约。”

灵雨只觉娇躯摇摇欲坠,虽然她对凤仪门诸人并无深厚的感情,可是毕竟是多年相处,若是没有凤仪门,她便只是一个人海孤女罢了,纵然早已生出疏离之心,也不会毫不动心。鸾儿连忙上前将她搀扶住了。灵雨强自冷静下来,裣衽道:“妾身失礼了,请让妾身验过契约文书,若是果然是真,妾身自也不能阻楼主入主月影轩之事。”

万楼主将一卷文书放到窗下书案上,灵雨上前仔细检视,发觉契约文书皆是真品,她虽然不理轩中事务,也知道能够拿到这些东西的人并不多,心中一叹,若是果真是三师妹所为,那么师尊死在仙霞岭之事就定然是千真万确的了。更令灵雨心惊的时候,竟然看到了自己的卖身契约,她当初本就是萧兰买回来的,可是在她被纪霞收入门下的时候,这契约便没有了作用,而且她也不敢相信凤仪门会放过自己,更没有留心卖身契的事情,想不到纪灵湘如此狠心,竟然将自己也卖给了万花楼,岂不是让自己任人摆布。想到此处,心中焦虑如火,只觉得娇躯一软,已经昏倒在了鸾儿怀中。其实这也是灵雨素来不以江湖中人自居的缘故,完全想不到可以用武力解决问题的缘故,否则纵然她武功不高,想要逃走却也不是不可能。

不知过了多久,灵雨悠悠醒转过来,耳边传来一个清脆悦耳的声音道:“万楼主,这却是你的不是了,风月场中自有规矩,当初举行秦淮花魁大赛的时候,便已白纸黑字说得明白,需得是已经自赎其身的姐妹才能参与,否则若是身不由主,怎配做烟花魁首,更何况自古以来,能够艳冠群芳夺得花魁的姐妹,也没有为人挟持的道理。这卖身契就是真的,也应该扯了才是,再说这也未必就是真的。若是万楼主不顾规矩,凭着这纸契约要想为难灵雨妹妹,只怕寒了姐妹们的心。我们这些误落风尘的女子,谁不盼着有一日清清白白的作人,若是灵雨妹妹这花中榜眼尚不能得到自由之身,只怕姐妹们都要死了从良的心了。”

灵雨听得声音熟悉,睁开眼睛望去,只见自己躺在内室软榻上,隔着珠帘,隐隐可以看到一个婀娜身影正在侃侃直言,坐了起来,却见鸾儿在一旁泪光盈盈地看着自己,便低声道:“这是怎么回事?”

鸾儿泣道:“小姐晕倒之后,万楼主便令婢子伺候小姐歇息,婢子知道小姐心思,却向轩中姐妹求救,大家都没有法子,还是月蓉姑娘说如梦姑娘侠骨柔肠,一向替姐妹们排忧解难,而且如梦姑娘在万楼主面前也可以说上话,若能求她出面,或者会有转机。婢子虽然也知道咱们月影轩一向和柳姑娘过不去,但是几次琴会相见,如梦姑娘对小姐都是很赏识的,所以便想法子送了信给柳姑娘。”

灵雨心中涌起暖流,勉力支撑着起了身,见身上衣衫还算得体,便扶着鸾儿走出珠帘,只见万楼主和柳如梦正对面坐着。柳如梦今年已经是二十六岁年纪,若是别的风尘女子,多半已经人老珠黄,可是柳如梦却是不同,比起当日夺得状元之时,风姿丝毫不减,只见她身穿一袭雨过天青色的曳地长裙,青丝绾在脑后,便如流瀑一般,身姿如细柳婀娜,容貌秀雅如春花,一双明眸流转,顾盼生姿,满室生光。

灵雨和柳如梦平日相知不深,只有几次琴会见过,月影轩和柳如梦多有嫌隙,却是柳如梦大度,对她们却从没有冷言冷语,故而有些交往,想不到自己今日落入窘境,却是并不熟识的柳如梦前来相救,反而是自己的师妹将自己出卖,不觉悲从中起,只叫得一声“柳姐姐”便哽咽不能语。

柳如梦站起将灵雨揽入怀中,柳眉倒竖,对万楼主道:“如梦一向敬重楼主行事,今日若是楼主定要为难灵雨妹妹,如梦虽然人微力薄,却也不能坐视此事,若是楼主肯网开一面,想来日后若有请托,如梦和灵雨妹妹都不会拒绝。”

万楼主心思百转,若是柳如梦振臂一呼,只怕自己旗下这些青楼的姑娘都会响应,秦淮河上的姑娘多半受过柳如梦好处恩惠,纵然自己可以高压逼迫这些女子屈服,可是这样一来她们必然心中不情愿,难免生出事端,再说自己若是落下刻薄无情的声名,只怕得不偿失,想到深处,他笑道:“如梦既然这样说,万某岂能不给姑娘颜面。”说罢便将灵雨的卖身契在火上烧了,又道:“灵雨姑娘从今之后便是自由之身,当然若是姑娘愿意留在万花楼,万某也会以礼相待。”

灵雨只觉心中狂喜,几乎不能言语,柳如梦见状将她放开,轻轻推了她一下,她才记得上前下拜道:“多谢楼主恩德。”犹豫了一下,她又问道:“请问楼主,仙霞之事可是真的?”

万楼主意味深长地道:“若非是真的,只怕在下也没有胆子来接收月影轩,姑娘与她们非是同路人,不过是偶然相逢,同舟共渡一段时日罢了,从今之后,姑娘也应抛却过往,过些自由自在的日子才是。”

灵雨闻言只觉一身轻松,她对凤仪门本无忠诚,仅有的一些留恋也被纪灵湘的绝情打破,月影轩她已经是不想多留,只是前路茫茫,无处可去,却又觉得有些为难。

柳如梦见状笑道:“妹妹不必烦恼,我那里虽然简陋,却还可以住得,妹妹不如到我那里歇息几日,等到过些日子再做决定不迟。”

灵雨感激地道:“多谢姐姐,小妹只好叨扰了。万楼主,鸾儿服侍我数年,我舍不得她,若是楼主答应,灵雨愿以百金赎取鸾儿。”

万楼主笑道:“灵雨姑娘言重了,鸾儿既是姑娘侍婢,万某怎会留难,区区百金,在下还不曾放在眼里,姑娘随身一切,可以慢慢收拾,万某会令手下送到柳姑娘处。”

灵雨再度裣衽为礼,万楼主含笑还礼,便径自离去了。

当灵雨随着柳如梦离开月影轩的时候,却不知道,万楼主正和一个青衣儒士在暗处看着两人。那青衣儒士犹豫地道:“楼主,陈爷托你照看灵雨姑娘,你任她离去,岂不是得罪了陈爷?”万楼主笑道:“不妨事,我探过了口风,是有贵人中意了灵雨姑娘,不过是托我照顾一下,免得有人趁机欺凌于她,如今她被柳如梦接走,既合她的心意,也不会违背了陈爷的意思,咱们只要派人盯着些就行了。再说你别忘了,柳如梦身后的宋逾,虽然他和陈爷之间有些恩怨,可是看起来仍是有些情分的,只要护住灵雨姑娘平安,我们便只有好处,没有坏处的。”

当灵雨走入柳如梦香闺的时候,一眼便看到墙壁上挂着的一幅字,却是醉后狂草,逸兴横飞,笔走龙蛇,灵雨也是琴棋书画皆通的才女,见那字写得好,便是眼睛一亮,低声念道:“银城远枕清江曲。汀洲老尽蒹葭绿。君上木兰舟。妾愁双凤楼。角声何处发。月浸溪桥雪。独自倚阑看。风飘襟袖寒。(注2)”下款却是“烟波散人”,不由道:“好凄清的词,烟波散人想必就是姐姐身边那位宋先生的雅号,怎么不见他的人影呢?”

柳如梦闻言微笑道:“他一个七尺男儿,怎会长久羁绊在温柔乡中,前些日子,他便辞去了琴师之职,离开建业了。”言辞虽然淡漠,可是只见她微蹙柳眉,愁锁花容,灵雨心中便知秦淮谣传并非虚假,柳如梦果然钟情了那位宋逾宋先生,那位宋先生数年来留在柳如梦身边,显然也是有情的,只是不知为何竟然凤飘鸾泊,中道乖分。愈要相劝,却无端想起那位四公子来,心中也是一阵怅然,不由暗暗祝祷道:“弱女自知微贱,不敢奢求,若能再遇四公子,从他学琴,纵然折损一生福寿也不会后悔。”

——————————————

注1:唐崔涂《琴曲歌辞•幽兰》

注2:陈允平《菩萨蛮》

\chapter{第四十五章 一见心相许}

公主闻哲病笃,乃请旨南下探视,雍帝许之,乃携昭华郡主、安国公至徐州侍疾。哲病将痊,有御史进谏,以哲督军在外,公主不可离京,雍帝留中不问,未几以太后微恙,懿旨诏公主回京。

——《南朝楚史·江随云传》

暮春四月,芳菲渐近,绿树成茵,正是人间好时节,可是自钟离至寿春的驿道上却是惨淡冷清,路边常见枯骨伏尸,林间树上每见鸦雀哀鸣。突然远处传来蹄声如雷,鸦雀惊飞,却是两军在旷野交战,一支是楚军飞骑营旗号,一支却是黑衣黑甲的雍军骑兵,两军相互绞杀,战得如火如荼,仔细看去,却是雍军占了上风。

从大雍隆盛十一年二月起,大雍再次发动了猛攻,这一次却是几路大军齐头并进,秦勇攻巴郡,长孙冀攻江陵,荆迟攻钟离,裴云攻泗州,战火连绵,更生从前,而南楚却失去了军方第一人陆灿,各处战场几乎是各行其是。别处也还罢了,淮西最是危急,石观已死,新任主将蔡群才能平庸,只知死守寿春,而他对陆灿嫡系的飞骑营又是心存忌惮,每每迫令他们和雍军主力接战。飞骑营虽然精锐,但是毕竟只有不到万人的骑兵,如今又失去了主将陆云和石玉锦,对着曾经纵横北疆的大雍铁骑,更是难以取胜,只是两月时间,就已经折损了大半实力,三月中旬,钟离便失守了,飞骑营却奉命阻碍雍军进兵,越发损失惨重。

这一支正在和飞骑营对敌的骑兵也不是寻常骑营,在大雍黑衣黑甲不是寻常军士可以穿的,这支骑兵乃是嘉郡王李麟的亲军,雍帝亲许使用黑甲,今次雍军攻淮西,李麟便是雍军的先锋将领。其实隆盛八年,李显督军江南之时,李麟便随父南来,跟在军中见习军务,可是虽然他很想上战场,更想和陆云交锋,却被李显一瞪眼给否决了,用李显的原话来说,莫非我们大雍没有人了么,让你这个小娃娃上阵杀敌,而军中的将领听了居然都是一脸赞同的神色,让李麟郁闷不已,只能暗中腹诽,当初皇伯父和父王不都是十几岁年纪就上阵杀敌的么?

直到今年春天,已经满十五岁的李麟终于得到了齐王允许领军上阵,而皇伯父李贽更是下旨准许他的亲军穿着黑甲,以示荣宠。李麟虽然是初次上阵,可是他在军中历练多年,只是几阵下来,荆迟便放心地让他做先锋了。只可惜陆云已经不在钟离了,就连淮西军中那个据说比陆云还出色的少年将领石玉锦也无影无踪,不能和他们一决高下,却让李麟扼腕不已。

不急不缓地驱使战阵,追在飞骑营后面,绞杀飞骑营落后的骑兵,将飞骑营数次反攻一一化解,飞骑营主将觉得不妙,便停下列阵,准备迎战。雍军见状,两翼伸展,隐隐欲将楚军包围,战阵列好之后,李麟提槊纵马出阵,大声笑道:“本王素来听说飞骑营飘忽善战,今日看来真是闻名不如见面,你们还是弃械投降,看在你们的陆云陆将军份上,本王自会善待尔等。”

见这黑衣少年将军如此嚣张,飞骑营上下都是义愤填膺,但是他们孤军奋战,敌军又是百战铁骑,这少年将军虽然言词狂妄,指挥起战阵来却是如臂使指,得心应手,心中都生出死意,为首的将领正欲出阵应答,突然风中传来一个冰冷悦耳的声音道:“是何人说飞骑营名不副实,便让我石玉锦领教一二。”飞骑营闻声几乎等呆住了,若是这时候雍军进攻,必能打个措手不及,只是雍军的主将也愣住了,全没想到下令攻击。

飞骑营将士静默了数息,继而高声欢呼起来,战阵便如潮水一般从中而分,一个白马银枪的少年将军从容策马穿过战阵,威武英俊,雄姿勃发,虽然只有十八九岁模样,但是只见他气势沉凝,杀气隐隐,便知是善战宿将,在他身边还有一个十一二岁的布衣绝丽少女,骑着一匹枣红马跟随,那少女怀中竟抱着一个婴孩,高据骏马,虽然衣着寻常,形容甚至有些狼狈,但是气度从容,明眸流波,浅笑嫣然,就像是游春的千金小姐一般。这一双金童玉女也似的人物出现在战场上,怎不令人瞠目结舌。

那少年将军一双冰冷的眼睛冷冷在李麟身上扫了一眼,道:“就是你大言不惭,竟敢要飞骑营请降么?”

李麟目光炯炯地望着那少年将军,眼中满是赞赏之色,心道,难怪这人的声名还在陆云之上,果然是南楚俊杰,心中生出争胜之念,他提槊上前道:“阁下便是石玉锦石少将军么?若是少将军觉得本王说得不对,可敢和本王一决么?”

此言一出,李麟身边的亲卫都是哗然,他们多半都是李麟亲自拣选提拔的勇士,对嘉郡王忠心耿耿,更何况又得了太子和齐王的严令,就是死也不能让嘉郡王涉险,石玉锦乃是楚军中出名的少年勇将,曾经阵斩雍军大将,这些年来在淮西更是威名赫赫,若是嘉郡王有了什么短长,就是一死也不能赎罪,偏偏又是李麟自己提出决斗,就是想阻止这场决战也没有借口,所以不等石玉锦出言同意,几名亲卫猛士已经策马冲上,口中喊道:“想要和王爷交锋,先过了我们这关再说。”

李麟眼睁睁地看着亲卫冲了上去,气得火冒三丈,却不便斥责他们,免得削弱了己方士气,只见石玉锦放声大笑,摘下鞍前银枪迎上,飞骑营将士都是发出长啸助威,丝毫不觉得石玉锦以寡敌众会有什么危险,双方战马交错之际,只见银枪疾点,便如梨花影动,瑞雪纷纷,不过十数回合,那几名雍军亲卫已经被她迫退,其中更有两人中枪,难以再战,虽然这些人都是精兵猛士,可是在石玉锦千锤百炼的银枪面前却是相形见挫。

飞骑营将士见状都是高声喝彩,李麟一皱眉正欲上前,耳边却传来一个少女银铃一般的笑声,心中一动,凝目瞧去,却见是那个和石玉锦一起前来的布衣少女,正在大声喝彩,满面仰慕地瞧着石玉锦在两军阵前耀武扬威。方才李麟只留意到了石玉锦,对这少女视若未见,但是此刻他却觉得脑海一片空白,眼中只有那少女艳绝人寰的仙姿。

正在这时,那少女怀中的婴儿大声哭叫起来,少女熟稔地拍着婴孩的襁褓,脆声道:“宝儿肚子饿了,快些击退他们吧。”

石玉锦一皱眉,厉声道:“留下几个人护着梅儿,诸君随我来。”说罢举枪冲上,在她身后,飞骑营将士呼喝相随,初时还有些阵形散乱,可是不到百步之远,便已经如同一人,千人结阵,奔腾如雷。

见敌军士气如虹,李麟收回早已魂飞天外的思绪,泄愤似的大吼一声,举槊率军迎战,不知怎么,他心中恼怒非常,对于淮西楚军极富盛名的两位少年将军他早已神往,陆云是他旧识,石玉锦乃是石观之子,陆云更是娶了石观之女,两人应是郎舅至亲,而去年九月,石玉锦护着陆灿之女陆梅逃出寿春的事情也是人尽皆知,这样想来,这少女定是陆梅,他们两人既是亲戚,又有诸般恩义,想来定会亲上加亲,只是这样一想,心中便生出恼怒。至于陆梅怀中的婴孩,想来应该无关紧要,李麟早已自动将他略去。

两军尚未交接,却见飞骑营急折向左,李麟一怔之间,飞骑营已经冲入雍军左翼,石玉锦领军冲阵,将雍军搅得大乱,李麟上阵未久,哪里是石玉锦对手,更何况如今的石玉锦更是少了几分冲动,多了几分冷静,左冲右突,不到片刻已经占了上风,李麟却是当机立断,立刻下令撤军,自行压阵,向钟离方向退去。飞骑营虽然取胜,但是毕竟力弱,所以石玉锦也没有领军追击。雍军退后,飞骑营将士簇拥着石玉锦欢呼雀跃,庆贺他们敬服的少将军重返军中,又领着他们战胜雍军前锋,洗雪了连战连败的屈辱。

石玉锦却是神色紧张,策马上前迎上陆梅,接过她手中的婴孩,探视一番,才放心下来。陆梅埋怨道:“大嫂,恩公说让你好好调养,一年之内最好不要上阵厮杀,你却是不肯听从,若是再病了可怎么办。”

石玉锦赧然一笑,道:“是,我知道错了,下次不敢了。”

这时候飞骑营中诸将都上前道:“少将军,不若留在军中不要走了吧,干脆我们帮你夺回淮西军权,免得还要受那蔡群贼子的窝囊气。”

石玉锦黯然道:“如今玉锦已经是朝廷钦犯,岂能再领军作战,这次我不过是路过这里,马上就要带着梅儿去南闽,想来不能再与诸君并肩作战了。”

众人听了都是垂头丧气,可是却也知道石玉锦所说才是正理,若真得那样做,岂不是犯上作乱,可是飞骑营若是这样下去,必是覆灭之局,他们又十分痛恨南楚朝廷屈杀陆灿,其中便有人道:“与其在这里白白送死,不若我们护着少将军去南闽吧。”此言一出,多有响应,就是石玉锦也觉得去南闽的一路上必然是艰险重重,若有些得心应手的亲卫保护,却是好上许多。想到飞骑营乃是陆氏嫡系,如今必是饱受排挤为难,与其让他们在淮西送死,倒不如弃了军籍,从今后海阔天空。石玉锦性如烈火,对南楚朝廷早已恨之入骨,更没有了捍卫社稷的心志,便道:“愿意去的就跟我走吧,我们分批南下,免得惊动那奸相心腹。若是不愿去的,就去淮东投奔杨参军,也不要在这里送死了。”

当下仅剩的四千飞骑营将士商议之后,有些仍然顾念淮南危局,大概两千五百多人决定转道淮东,再不受蔡群节制,还有一千多人已经心灰意冷,便商定分散南下,到南闽随侍陆氏一门。石玉锦形迹不甚掩饰,早已惊动了淮西军各部,可是众人都顾念陆灿、石观恩情,石玉锦又是他们同胞故旧,都是暗暗相助,更有些石观昔日的亲军心腹,也已经无心战事,便也弃了军籍,随着石玉锦去了南闽。等到蔡群有所察觉的时候,淮西军中精英已经去了十之二三。石玉锦这般举动,却是不曾顾及大局,只是以她的性子,没有起兵报仇,已经是难得非常了。只是淮西军实力大损,蔡群又是庸碌之辈,雍军在淮西势如破竹,全无阻碍,不到一年,淮西已经落入雍军之手。这般情形却不是陆灿生前可以料及的,若是石观不死,淮西局势断然不会糜烂至此,就是石玉锦弃军而走,也不会有这许多人相随而去的。

李麟自然不知道接下来会发生什么,只是垂头丧气地返回钟离,心中恼恨不已,岂料刚到城下,便见城门大开,一个青衣少年随众而出相迎,李麟一见这人,不由大笑道:“霍大哥,你怎么来了?”跳下马飞奔迎上,那少年也是疾步走出人群,两人把臂相视,都是欢喜非常。

李麟将军务交给副将处置,自己拉着霍琮向城内走去,一边走一边问道:“霍大哥不是跟着皇兄在楚州坐镇么,怎么会来钟离看我,皇兄怎肯放走你这个左膀右臂?”

霍琮笑道:“我不过是跟在太子殿下身边整理一些文书罢了,哪里谈得上什么臂膀,今日是太子殿下听说郡王爷领军上阵,心中不安,命我押送一批粮草到钟离,顺便来看看你,还嘱咐你小心在意,不可轻乎生死。”

李麟笑道:“皇兄总是当我没有长大,替我向皇兄致谢,对了,柔蓝还好么,这边兵荒马乱的,可别让她四处乱走,若是有什么闪失,只怕我皇兄要心痛死了。”

霍琮目光一闪,自从去年十月,长乐公主领着柔蓝和慎儿到徐州探视江哲病情,初时柔蓝还乖乖待在徐州,后来江哲病情好转,柔蓝便呆不住了,常常寻个理由跑到楚州去见太子李骏,这件事情众人心知肚明,都知道昭华郡主迟早会嫁入皇室作太子妃,只有李麟总是硬撑着不愿松口,不肯承认李骏与柔蓝的两情相悦。难得他今日的语气中全无嫉妒之意,莫非是发生了什么变故。想到此处霍琮便故意询问李麟近日的战况,李麟毕竟直率,没多久就被套出了话风,更是因为知道霍琮消息灵通,出言问道:“霍大哥,你有没有听说过陆小姐的事情,她可有了婚配么?”

霍琮暗中差点笑破了肚皮,知道李麟误会了石玉锦和陆梅的关系,这也难怪,南楚朝廷向来习惯掩耳盗铃,有意无意之间,就将石玉锦和石绣当成了两个人,而在雍军看来,不论石玉锦是男是女,最重要的却是她的能征善战,自然也不会刻意传扬此事,而李麟虽然身份尊贵,却不过是寻常将领,他既然全没想到那方面去,自然也不会有人告诉他石玉锦的真正身份。

不过纵然如此,霍琮也不看好李麟的心思,纵然南楚灭亡,陆氏也不会甘心投降,最多是不闻不问,隐在民间罢了,绝对不会生出攀附权贵的心思,李麟若想追求陆梅,那更是难于登天,不过想来想去,总比李麟一颗心始终系在柔蓝身上好些,便忍着笑道:“郡王爷,你大概不知道吧,那位石玉锦石少将军乃是陆云陆少将军的结发妻子,那个婴孩就是石少将军两月前所生的儿子,乳名宝儿,尚未取名,不过石少将军毕竟是武将,所以那孩儿便由陆小姐照看。”

李麟心中只觉狂喜,此刻他全然没有想到被个女子打败的屈辱,只想着陆梅与石玉锦并非情侣,自己便有了机会,也顾不上问霍琮如何知道得这般详细,只是拉着他结结巴巴地道:“霍大哥,能不能帮我想想法子,我,我很想娶陆梅为妻。”

霍琮不怀好意地上下打量了李麟片刻,看得李麟心中发毛,良久,霍琮才笑道:“这件事情,我倒是会替你想法子,不过只怕艰难得很,你是堂堂大雍郡王,陆梅小姐却是南楚大将军之后,国仇家恨挡在其中,你若没有破釜沉舟的勇气,只怕是没有什么希望的。”

李麟连忙道:“霍大哥放心,若是皇伯父和父王拦阻,最多我不要这个爵位,若是陆家的人不肯,我情愿死在他们面前,也要求得他们谅解。”

霍琮肃容道:“你可是一片诚心要娶陆小姐为妻?”

李麟指天誓日道:“若有二心,就让李麟死在刀剑之下,尸骨无存。”

霍琮心道,此事若成,不仅免去李麟和太子殿下的相争,也可以保证陆氏将来的平安,先生定是欢喜的,就是皇上和齐王也不会反对,只不过若想得到陆氏许婚,只怕是十分艰难,想了许久,霍琮狠狠心道:“郡王爷放心,这件事情我一定想法子帮你,不过你也得想清楚,只怕没有十年八载的水磨功夫,你是别想成功的。”

李麟道:“精诚所至,金石为开,本王绝不会放弃的。”心中却暗自想道,这么长时间,可要留心有人捷足先登,回去我便求父王想法子,还有霍大哥虽然答应了,却还不够,还得去求姑夫才行。此刻的李麟自然想象不出来,他的追妻之路,会是何等的艰苦卓绝。

\chapter{第四十六章 相报甚时休}

十一年,郡王承命为先锋,王甚勇武,每自为前驱,耀武军前,人不敢正眼视之。

十三年春,三军承帝命渡江,荆迟部、裴云部,将会师建业,南楚国主惊惧,率宫妃禁卫奔当涂,禁军闻之大乱,烧杀掳掠,建业官民皆苦,乃开城门请降,郡王为荆部先锋,军仅五千,或劝其待主将至,郡王不许,乃悉众入城,先遣军士护宗庙,自率军号令城内,有乱军为害,皆杀之。建业乃平,王亦名噪天下。

以郡王功显,令独自领军,王乃席卷江南,破豫章、宜春、庐陵、鄱阳、临川诸郡,皆有大功,军中皆许为后起之秀。郡王性端严,军令严苛,杀伐决断,楚人惊惧,然颇爱豪杰忠义之士,不忍伤之,纵有冒犯,唯槛送建业耳,时,太子骏镇建业,见而皆笑赦之。

十四年,天下稍定,太宗欲遣重臣抚南闽,闽中多蛮荒之地,道路艰绝,人皆不欲,郡王自请镇八闽,意甚诚,愿为南海藩障,太宗嘉许之,任其南闽节度使,许建牙,开府仪同三司。

郡王抚闽九年,修商道,浚江河,劝农桑,慑豪强,闽人皆服膺。

二十二年,聘故楚大将军陆灿女为王妃,太宗遣使赐婚,特旨许用亲王仪仗。

翌年,太宗诏郡王还朝,民皆扶老携幼,望尘相送,几三十里。

——《雍史·嘉郡王列传》

霍琮来到钟离,除了奉太子之命来看望李麟之外,还有一个缘故就是为了石玉锦和陆梅,原本董缺奉江哲之命救下两人,江哲准备等到荆迟攻之时,遣人将她们接到徐州去的。想不到荆迟还未尽得淮西之地,江哲就得到董缺的消息,石玉锦生子之后,修养了不到两个月,就不愿再逗留了,从董缺那里得知外面的情势之后,便要将陆梅和爱子送到汀洲,然后再北返寻找陆云的下落。董缺本就是以游方道士的身份相救两女的,自然也不好阻止石玉锦这般行事,只能迅速将消息传到徐州。霍琮这次就是奉命前来,若是石玉锦和雍军发生什么冲突,也好从中周旋。如今李麟对陆梅一见心许,他自然不用再操心了,交割了粮草之后,又暗暗和荆迟透了些端倪,嘱咐了李麟一些言语,第二天一早便启程往徐州去了。

因为急于返回徐州,所以霍琮只带了四个虎贲侍卫就上路了,这四人都是在定海之时保护他的旧人,相处数年,彼此十分知心,知道他心中焦急,一路上快马加鞭,不曾停息,直到正午时分,阳光刺目,人马都疲惫了,这时,霍琮见到路边有一座荒废的庙宇,便提鞭道:“快午时了,就在前面休息一下吧。”四名侍卫同声应诺。

这里本是过路旅人常常休息的地方,只是这几年雍楚对峙淮西,所以才变得残破,但是仍然可以遮风避雨。五人到了庙前,翻身下马,将马系在庙前,一人取了廊下木桶,到庙后林中清溪提水,另外三人伺候马匹,在阶下准备午饭。霍琮见几人都忙着,便自己在庙外散步起来,想要松弛一下筋骨。见到侍卫提水出来,又听见树林中传来潺潺水声,隐约仿佛,如同琴音淙淙,不由生出寻幽探胜之心,向几个侍卫招呼了一声,就向林后走去。一个侍卫起身想要跟来保护,却被霍琮阻住。如今江淮局势和去年不同,自从陆灿死后,淮南楚军龟缩不出,更别说派遣斥候深入雍境了,所以霍琮也没有遇刺的担忧,更何况霍琮也会些武技,若是寻常南楚斥候,倒也不会被人随便杀了,所以那侍卫一犹豫,也就没有跟来。

霍琮走了几十丈远,便看到林中一溪清泉,泉水清澈见底,水中尚有游鱼,心中生出闲适之意,便坐在溪边石上,临水观鱼,不亦乐乎。

正在霍琮倚在石上,任由透过绿茵的温暖阳光照在身上,昏昏欲睡的时候,耳边却传来一个讥讽的声音道:“霍公子如今已经是青云直上,想来已经不记得杀父之仇,灭国之恨了。”

霍琮只觉得浑身一震,他紧闭双唇,忍住呼救的冲动,不仅仅是因为抵在他背后的尖锐利刃,还因为那人的言语。

身后那人见状笑道:“霍公子果然聪明颖悟,想当初锦绣盟主霍纪城死于敌手,就连名头也被人夺去之时,却想不到自己的爱子竟会有今日吧。”

霍琮目光闪过寒芒,冷冷道:“你胡说些什么,霍某不明白你的意思。”话音未落,只觉身后利刃已经移开,有一人坐到他身侧青石上,从容道:“不知道霍公子还记得我厉鸣么,当初可是我送公子和霍夫人一起到长安的,这些年来,公子相貌竟是没有什么变化,只是眉心那颗红痣仍然如故,当初便有相士说这是‘草里藏珠’,主聪明多智,遇难呈祥,如今看来,那相士当真是铁口神算,谁会想到大雍、南楚两国都要擒拿的钦犯霍纪城的亲子,如今竟是大雍重臣江哲的弟子,更是深得太子李骏器重,将来必定是位极人臣,富贵双全。不过也当真是有其师必有其徒,令师叛楚投雍,霍公子却是认贼作父,这倒也是青出于蓝。”

霍琮面如死灰,也不望那人一眼,只是盯着眼前的溪水沉默不语,他本不是这样轻易就会被人慑服的,只是这人说穿他多年心事,这才让他变成这般模样。

那人冷冷道:“盟主昔年决意复国,为此不惜舍身,只是人都有私心,所以和夫人成亲之时,便秘而不宣,在公子出世之后,更是将家人送到了长安,这却是盟主一番苦心,长安虽然是雍都,但是反而比起寻常地方更加安全,又没有兵燹之祸,只要夫人和公子身份不泄露,就可长久安居。虽然世人都以为盟主是死在隆盛元年东川庆王之变时候,可是你我都清楚,自从武威二十四年之后,夫人便失去了盟主的音讯。只是我却不是锦绣盟中人,夫人也没有法子和盟中盟主亲信联络,所以始终不知道那用盟主之名,纵横天下的到底是谁罢了。武威二十五年年初,夫人病殁,公子在夫人葬后便突然出走,我还曾暗中寻访过,只是想不到公子竟然进了雍王府。如今想来,公子当时应是想探知盟主下落,盟主若是已遭不测,那么最可能的凶手就是雍人,只不过不知道是雍王李贽还是太子李安下的手,你投入雍王府也是没错的,只是富贵逼人来,荣华乱心志,如今公子早已忘了父母之仇了吧?”

霍琮紧咬牙关,不知何时鲜血已经溢出嘴角,那人见了冷冷一笑,道:“厉某没有出息,后来流落到南楚,跟随韦首座左右,凤仪门虽然是落毛的凤凰,但是仍然是百足之虫,死而不僵,却也让我知道了许多秘密。韦首座这些年来苦心思索,早已断定锦绣盟从武威二十四年,便已经落入雍帝李贽掌握之中,那江哲性子,最爱藏着掖着,真正掌管此事的除了江哲之外,不会有别人,这样看来,盟主死在谁人手里,不问可知。据闻江哲对公子爱重非常,公子难道真的一点都猜不出来谁是杀父仇人么?”

霍琮眼中似乎要冒出火来,恶狠狠地盯着那人,那人却仿佛浑不在意,从怀中取出一个玉瓶放到霍琮面前,道:“这瓶中是首座向毒王买来的秘药,寻常人若是吃了没有妨碍,若是重病受伤的人吃了,便会越来越虚弱,只需要数月时间,就可以令服药之人无声无息地死去,公子是江哲爱徒,只要将此物下在饮食汤药中,就可以报了国仇家恨。公子不必担心,那厮虽然是岐黄圣手,但是用毒之道,高深莫测,申如晦在毒药上面的本事天下无双,纵然是医圣亲临,也不能发觉此药,更何况这药严格说来并非剧毒,乃是一种强身健体的补药,只不过不适用于病人罢了。”

见霍琮仍不言语,那人却知霍琮非是不动心,又道:“公子若是不肯动手,厉鸣丑话说在前头,半年之内,那人若没有死去,我便将公子身世泄漏出去,只是不知到了那时,那江哲可会心慈手软.就连他少年知交,亲如骨肉的爱徒和他为敌,他都不肯放过,更何况是你一个无依无靠的孺子,他纵然不舍得杀你,只怕你也从此青云路断,再不能得到雍廷的信任,到了那时,只怕公子生不如死,倒不如舍命一搏为好。若是公子肯杀了江哲,实不相瞒,厉某早已心存死志,也不愿苟活于世,必会到九泉下去向霍盟主和韦首座报知这个好消息,绝不会留在世上,令公子如芒在背,耿耿于怀。公子若是不放心,可以到寿春城内平安客栈来见我,想必到时候寿春已经被大雍攻破了吧,若是我死在公子面前,想来公子就会放心了吧?只是公子也别想事情未成就杀人灭口,我早已将书信留在心腹人手上,若是没有我的信物,明年此时,他就会拆开书信,按照我的遗命,将公子身世传遍天下,到时候公子只怕会后悔莫及。若是公子杀了江哲,我自会将信物和那人身份相告,公子就可永绝后患,岂不是一件美事?”

霍琮怔怔望着玉瓶,不知什么时候,身后传来侍卫的声音道:“公子,已经可以用饭了。”

霍琮下意识地将玉瓶藏入袖中,抬起头来,那厉鸣早已不知去向,木然道:“这就过去,等我一下。”然后走到溪边,也不伸手掬水,却径自将头扎入水中,清冷的溪水寒意尤重,过了片刻,霍琮才抬起头来,起身回头笑道:“这溪水凉得紧。”水线如珠,从他发上面上淌下,却丝毫不给人狼狈之感,反令人觉得他洒脱率直。那侍卫随他数年,知道霍琮偶然会有这般不拘形迹的举动,却也没有看出霍琮心中波澜,凑趣笑道:“这溪水本就是冷的,现在又是暮春,难免会有凉意,公子还是擦干水迹吧,要不然受了风寒可就糟了。”

霍琮微微一笑,用袍袖拭去水痕,谈笑自若地随着那侍卫走到林外庙前,只见庙前阶下行军炉灶中已经是热气腾腾,浓汤就着烙饼,倒也是一顿丰盛的佳肴。霍琮丝毫不露声色地和几个侍卫说笑用饭,全无人知道霍琮此刻已经是食不知味。用过午饭后,休息了半个时辰,五人再度上路,一路上无话,第四日清晨,五人便到了徐州城前。赶了一夜的路,身上衣衫几乎已经被露水浸透,急欲入城换衣,眼看着晨光中屹立的徐州城,不用商量,五人都多加了一鞭,快马向城门奔去。还未到城门,却惊见城前旌旗招展,霍琮心中疑惑,策马停在路边,凝神瞧去,明黄的龙凤旗帜,衣甲鲜明的龙骧禁军,富丽堂皇的公主仪仗,都明示了正在出城的车队的身份,未几,霍琮便看到长乐公主的金辂。

霍琮心中奇怪,长乐公主是因为江哲病重而到徐州的,算起来江哲应该还没有痊愈,怎么公主就要回去了,避在路边发怔,霍琮却忘记了可以上前相问,那林间溪边的一番谈话给他的打击之重,绝非表面的平静从容可以遮盖的。

大雍公主按照礼制本应使用翟车,唯有宁国长乐公主特旨许用金辂,这本是雍帝荣宠之意,可是霍琮心思数转,已经想通今日之事,他去钟离之前,便从太子李骏那里得知有御史进谏,弹劾长乐公主久离雍都之事,想来定是皇上下旨诏回公主,再望见金辂,心中已是蒙了一层阴影。这时,霍琮又看到长乐公主銮驾之侧,柔蓝和慎儿各骑骏马相随,但是慎儿穿着行路便服,柔蓝却穿着一件淡黄春衫,全不似要赶路的模样,只是依依不舍地透过珠帘高挑的窗子和长乐公主低头说话,便暗暗猜测长乐公主定是将柔蓝留在徐州了。

这时候,长乐公主和柔蓝都看到了在路边的霍琮,停住銮驾,长乐公主柔声道:“琮儿回来了,你若再晚回来一些时候,就不能向本宫辞行了。”

霍琮这才上前见礼,有些惆怅地问道:“师母这是要回京么?”

长乐公主轻轻一叹,秀丽的容颜上露出黯然之色,道:“母后微恙,下旨诏本宫回京,我将蓝儿留下照料她爹爹,只是她还年幼,多半不能得心应手,你若在随云身边,可要多担待一些,随云虽然已经好转了许多,可是我始终放心不下。”

这时候,江慎隔着金辂在另一边探出身子,急切地道:“霍哥哥,你可要跟爹爹说,不是我不想把《诗经》抄十遍,可是皇上舅舅让我一起回去的,说是外祖母很想念我,师父也要我回去练功,所以我才走的,最多等爹爹回京之后,我再把抄好的诗经交给他。”

柔蓝原本已经泫然若泣,听到江慎言语,却破涕而笑道:“慎儿,你不是想请人照着你的笔迹抄书啊,爹爹的眼力可是很厉害的,瞒不过的。”

江慎闻言立刻愣住了,一双清澈明晰的黑眸滴溜溜转个不停,似乎在考虑姐姐所说的是真是假。

却听长乐公主笑道:“是啊,慎儿,你姐姐从前可是吃过亏的,原本只是抄五遍《论语》,结果又多抄了十遍。”

江慎张大了嘴巴,愣在哪里,却忘了自己还在马上,差点跌了下来,幸好他武功已经初成,手忙脚乱地控住马缰。霍琮也是“噗哧”一声笑了出来,这几日的愁苦烦闷几乎是一扫而空,只有柔蓝满面通红,越发娇嗔不依。

这小小的插曲却是冲淡了离别的愁绪,直到长乐公主銮驾消失在视线当中的时候,霍琮仍然是面带笑容,直到柔蓝在他耳边嘀咕道:“皇上舅舅也真是的,不就是有人上折子弹劾么,就忙着将娘亲诏回京去,我若是爹爹,干脆就一起回去了,免得平白无故地呕心沥血。”

霍琮心中一颤,原本的欢乐沉寂下去,淡淡道:“蓝儿不可出言不逊,这话若是传了出去,只怕引起麻烦,皇上对先生怎会有什么疑心,多半是为了堵那些谏官的口舌罢了。”

柔蓝闻言不忿地道:“爹爹也这样说,可我就是不服气,若给我知道是谁弹劾爹爹,定要拔了他的胡子去。”

霍琮笑道:“好了,不要闹了,我要去见先生了,你若不想回去,我可不等你了。”

柔蓝眼珠一转,道:“霍哥哥,你给我求个情,爹爹不许我再去楚州,还说让我好好学些女红中馈,我可不喜欢那些麻烦的事情,爹爹最疼你了,你若说话爹爹必会答应的。”

霍琮心中更是刺痛,勉强道:“好吧,我去向先生提一下,不过先生若是不答应,我可也没有法子。”

两人策马走向江哲养病的凝碧园,耳中听见街道两侧嘈杂的声响,不知怎么,霍琮的心思渐渐沉静下来,不复方才的凄苦沉沦,往事一幕幕涌上心头。他知道那人话中有许多不实之处,爹爹并非是复国志士,而且将自己和娘亲送到长安隐居也不全是为了母子两人的安全。虽然那时候他还年幼,但是却记得很多事情,尤其是娘亲常常向自己倾诉心中苦恨,或者是以为自己听不懂吧,否则娘亲那样贤惠温柔的女子,绝不会说夫婿的不是。可是那人却有一点没有说错,爹爹的确死在先生手中,而自己的确是忘记了国仇家恨。

他从未将自己当成蜀人,在他出生之后,蜀国早已经亡了,他的童年是在长安度过的,后来又在寒园之中长成,国仇他从来不曾念及,唯有家恨,他却是一刻不曾忘记。当初冲撞了雍王府车驾,他是存心的,想要用这个法子混入雍王府,那时他的愿望不过是想要得知父亲的生死,然后去告诉已经香消玉陨的娘亲一声。谁知因缘际会,他投入了江哲门下,这也是他心结之始。江哲的器重和信任,让他得以知道了许多隐秘,更是从蛛丝马迹中猜到了父亲的死因,可是江哲的教诲爱护,却让他领略到从来没有得到的父爱,在他心中,早已将江哲当成了至亲之人,可是偏偏是这人害死了他的生身父亲。

最终他决定不去面对这个事实,只要自己没有得到真凭实据,就可以不去想江哲便是自己的杀父仇人。到后来,他最怕的就是身份泄漏。一旦江哲知道了自己的身世,以江哲的性情,必会将真相说明,他不怕江哲将他驱逐出寒园,不怕江哲让他陷入求生不得、求死不能的窘境,甚至也不怕江哲杀了他,他怕的却是恩仇之间不知要如何抉择,只怕到了那时,他除了自尽而死之外,再无别的路可走。

可是自己竭力掩盖的隐秘终于被人揭破了,自己终究是不能自欺欺人,终于到了凝碧园,霍琮下了马,跟着柔蓝一步步走向江哲的居处,只觉足下仿佛踏在棉花上,全无支撑,目光落在虚掩的门扉上,霍琮突然觉得前所未有的冷静,原来当真面对的时候,并没有想象中那样可怕。门内传来江哲淡漠的声音道:“琮儿回来了么,进来吧,蓝儿,昨日的那碗汤我很喜欢,你去告诉厨下,今日晚膳还要那道汤。”

微微苦笑,听着柔蓝远去的足音,鼓起勇气,霍琮推门走了进去,目光一闪,便顿时凝住,在他意中,江哲还应是月前那般郁郁寡欢的模样,孰料放眼望去,江哲坐在椅上,只穿着中衣,身披宽袍,正端着香气四溢的香茗欣赏书案上的一幅字帖,神色闲适自若,全无一分愁容。而小顺子则坐在棋坪前面,手中拿着一本古旧的册子,正在那里打棋谱,不时的拈起棋子放落在棋盘上。主仆两人这般悠闲自得,仿佛数月前的阴云消逝无踪了一般。

见到霍琮进来,小顺子眼皮也不曾抬一下,江哲却抬头笑道:“琮儿遇见你师母了吧,其实她也是过分操心了,我如今已经好了许多,纵然她不在我身边,也不会有什么问题了,倒是回京好些,也免得那些腐乳多嘴多舌。”

见江哲神色祥和,霍琮只觉心中一宽,下意识地将心中愁苦抛到一边,道:“先生这般高兴,可是有什么喜事么?”

我笑道:“哪里有什么喜事,四路大军一起兴兵,只有淮西这边顺利非常,巴郡那里原本余缅已经有意投降了他,却有一个人送去了陆灿的一柄佩剑,那余缅已经指天立誓不会投降了,只怕想要攻下巴郡,得费些功夫了。”

见江哲说到陆灿,已无戚容,霍琮心中一动,试探地问道:“先生已经不再为大将军的事情难过了么?”

小顺子闻言抬起头,眼中露出不满之色。霍琮低下头去,也觉自己不该刺及先生心中隐痛。这时耳边却传来江哲淡雅平和的声音道:“唉,此事我其实早有准备,那些日子不过是一时懵懂住了,逝者已矣,纵然难过又能如何呢?我和陆灿纵然情谊再厚,也抵不过忠义二字,若是陆灿将我杀了,多半也会痛楚难当,只是事过境迁,他却也还要领军上阵杀敌的。我既不后悔当日所作所为,何必还要郁结心中,徒令亲痛仇快罢了,想来他虽然杀身成仁,却也不会喜欢看到我那般难过吧。有些事情终究是要面对的,何谓对错,何谓忠孝,只要此心能安,又何需在乎世俗之见。”

霍琮听到江哲最后的两句话,只觉如同醍醐灌顶一般,心中顿时豁然开朗,生机也再度出现在面上,沉默片刻,笑道:“先生能够想通就好了,难怪师母肯奉诏返京,却是因为先生已经没事了,弟子此来也有好消息禀报,先生若是听了,只怕会更开心一些。”

我饶有兴趣地道:“你这样快就回来,我便知道那件事情定是已经解决了,说说你的好消息吧。”

霍琮便将李麟钟情陆梅之事仔细道来,我听得眉飞色舞,不由拊掌大笑道:“这倒是有其父必有其子,当初齐王殿下为了嘉平公主,却是惹出了多少笑话,费了多少心思,才娶到佳人,只怕将来李麟这小子费的心思要超过其父十倍,才能如愿以偿,不过这件事情却也要极力促成为好。不过说起来这些孩子也都大了,蓝儿去年也及笈了,也应该为她择个佳婿,虽然还想多留她几年,却也不能误了她的姻缘。”

霍琮心中已经有了决定,上前拜倒道:“先生,弟子有件事情想要拜托顺叔,还请先生允许。”

眉梢轻扬,我的目光在霍琮身上停留了片刻,温和地道:“你自己去求他吧,若是小顺子答应,我这边自然没有问题。”

霍琮再拜叩首,起身走到小顺子身边,目光炯炯,却是垂手不言,小顺子放下棋谱,淡淡道:“走吧。”说着向门外走去,霍琮低头跟在他身后,虽然是背对着江哲,却能感受到他的目光一直跟在自己身后,直到房门在身后关上,那炽热的目光才被厚厚的木门阻住。

两人走到园中,小顺子负手站在一池碧水之前,漠然道:“你有什么事情?”

霍琮淡淡道:“弟子想求顺叔杀一个人。”

小顺子微微一怔,道:“你想杀什么人?”

霍琮取出怀中玉瓶,把玩了片刻,放在地上,退了一步道:“弟子想杀一个叫做厉鸣的人,想来应该能够在寿春的平安客栈找到他,若有顺叔出手,想必是万无一失,弟子才能放心。”

小顺子却不问厉鸣是谁,冷冷道:“你不担心只杀他一人没有用处么?”

霍琮笑道:“凤仪门已经烟消云散,辰堂也是尽毁在仙霞岭上,想来厉鸣也没有什么心腹人了,他所言多半是恐吓,我却是不信的,再说就是流言传了出去,却也没有什么关系,我本也不在意那些荣华富贵,少些牵绊,却也少些责任,不会像先生这样,始终不能脱身。”

小顺子回过头,目中满是寒意,却又隐隐有些期望,问道:“你已经决定了么?”

霍琮点头道:“是的,有些事情终究是要面对的,既然我的心已经告诉我应该如何抉择,我就不会再有为难,便是认贼作父又如何,便是忘了杀父之仇又如何,霍琮只知道,在寒园之内的生涯终生难忘,先生、师母、顺叔、蓝儿和慎儿就是我的亲人。”

小顺子眼中闪过一丝喜悦,却迅速敛去,肃容道:“这件事情我会处理的,去陪他下盘棋吧,昨日又输了给我,很是不高兴呢,若说让棋,还是你做的天衣无缝,这一点我却是万万比不上你的。”

霍琮微笑道:“弟子遵命,还请顺叔多多费心。”说罢,霍琮转身向江哲的居室走去。

在他身后,小顺子从袖中取出一张绵纸,上面皆是蝇头小楷,写道:“携陆灿佩剑阻余缅顺义者,名厉鸣,凤仪门辰堂所属,韦膺心腹,明鉴司奉命追查,其人于钟离至宿州道上,密会霍琮,所言不详,请先生留意。”

小顺子微微一笑,手指轻振,那张绵纸瞬间化为灰烬。

看到霍琮再度走入房间,我放下手中字帖,他既然再度走了进来,那么一切事情都已经不必问了,放下心中大石,望向霍琮的目光满是喜悦宠溺,想起一桩早已盘算过许久的美事,我微笑道:“琮儿,有一件事情我想了很久,蓝儿是我掌上明珠,我总是不舍得将她嫁出去,可是毕竟男大当婚,女大当嫁,我不能误她终身,你是我的弟子,也如我的家人一般,我有意将蓝儿许配给你,不知你意下如何?”

说完之后,我热切地看着霍琮,若是他答应下来,我就不用将蓝儿嫁出去了,原本以为霍琮应该欣喜若狂地答应才是,岂料霍琮愣了片刻,语气古怪地问道:“先生,你问过蓝儿的意思没有?”

这是什么意思,我皱紧了眉头,道:“还没问过,不过你们两人青梅竹马,你又是这样的人品才华,想来蓝儿不会拒绝才是。”

霍琮有些哭笑不得,却不敢挑明,委婉地道:“先生,蓝儿和太子殿下、嘉郡王都是一起长大的,先生莫非没有考虑过他们么?”

我笑道:“麟儿就不说了,一来他年纪比蓝儿还小一岁,再说这孩子若和蓝儿一起,多半会吵得翻了天,更何况如今他已经有了意中人,这父子俩的个性像得很,我是不指望他移情别恋的,至于太子,就更不用说了,那是万万不行的,这次长乐到徐州,便说过皇后已经准备为太子选妃了,蓝儿和他怎有可能?再说就是太子有意,我也不能答应,就是你娶了蓝儿,将来也不许你娶妾纳婢,需得一心一意对着蓝儿才行。”

霍琮暗自庆幸自己将李麟拉上做了陪衬,若非如此,只怕自己还不会知道先生的心意呢。犹豫了一下道:“先生,太子殿下选妃,必定是从名门淑嫒中选取良配的,蓝儿也是郡主身份,似乎也在膺选之列。”

我不在意地道:“这无妨,我已经写好了折子,你若同意婚事,我就上书说明此事,想来皇上也会给我这个面子的,蓝儿素来也得太后和皇后的宠爱,应该没有问题的。对了,你的意思到底如何?莫非你觉得蓝儿有什么不配你的地方么?”

霍琮差点叫苦连天,此刻他恨不得自己方才被小顺子解决掉,也免得要面对这样的难题,姑且不论自己是否有胆子和太子殿下争夺爱侣,问题是蓝儿和太子分明是钟情已深,自己如何能够横刀夺爱。想了一想,还是暂且拖延一下,他可是知道江哲的性子,若是弄得不好,说不定会立刻将柔蓝许婚给自己,这件事情若是传了出去,就很难有挽回的余地了。所以霍琮想了又想,婉转地道:“先生,若是这事现在定了下来,只怕蓝儿羞恼,不敢再留在徐州了,不如等到战事稍平,先生再告诉她吧。只要蓝儿愿意,霍琮情愿娶她为妻。”

我全没留心霍琮话中玄机,只是想着也应约束一下柔蓝,不要再和太子过分接近,免得未来的太子妃嫉妒,也免得琮儿烦恼。因为从未想过我的爱女会去和别人争夺丈夫,所以柔蓝和太子之间的任何可能,早已被我抛诸脑后,完全不知道自己拆散了一对小鸳鸯,我拿起写好的奏折,道:“明日就把折子送上去,免得太子选妃的事情还要牵涉蓝儿,就和军报一起吧,也好快一些,免得长乐还要多费唇舌。”

霍琮更是苦恼,心道,我可没有办法偷走奏折,是传信给太子,让他上书向皇上求助呢,还是传信给慎儿,让他想法子中道截住折子呢?”

望了江哲一眼,霍琮只恨自己为什么要放弃报仇,否则也不会面对这样的窘境吧。

寿春,平安客栈,孤灯零落,夜雨凄凄,凄风苦雨中传来更漏之声,越发的估计难眠,厉鸣披衣而起,将桌上的灯火挑亮一些,然后将冷酒倒了一盏,缓缓饮下,一双满是血丝的眼睛越发迷蒙了几分。

正在他想再倒一杯酒的时候,温暖的房间之内突然无端阴冷了起来,竟似有滴水成冰的模样,厉鸣身子一颤,却仿佛没有察觉任何异样一般,继续倾尽壶底,却也只得半盏浊酒。端起酒盏,他也不急着饮下酒液,淡淡道:“阁下可否等我说几句话再动手?”

身后传来一个冰冷的声音道:“我不急,有什么话你可以慢慢说,天明之前的时间都是你的,只要你不想着挣扎求生,我就不会动手。”

厉鸣转过身来,看到一个相貌洁若冰雪的青年神色漠然,负手站在门前,虽然只是青衣装束,但是傲然之姿却令人不敢忽略他的光彩,不由笑道:“原来是邪影李爷亲自前来解决在下,厉某深感荣幸,不如让在下再要壶酒来,春夜当垆,也是人生快事,只是不知道在下微末之躯,可有这个荣幸?”

小顺子目光中多了几分柔和,淡淡道:“你有这个资格,来人,拿酒来。”随着他一声令下,房门悄然洞开,两个伙计拿着火炉、木炭、大铜壶和一坛上好的美酒进来,将这些摆在窗下,施礼之后便退了出去。

厉鸣挽起袖子便开始煮酒,只是见他粗手笨脚的模样,当真是令人汗颜,小顺子看得郁闷,冷冷道:“还是我来吧,这般美酒落在你手中,多半是焚琴煮鹤。”说罢熟练地开始加上一些木炭。

厉鸣见状笑道:“若是知道李爷肯纡尊降贵,我就是原本会煮酒,此刻也定是不会了。”

小顺子冷冷道:“你倒是好胆量?不过看在你马上就要奔赴黄泉路的份上,我也就不和你计较了。”

厉鸣自得地道:“天下间能够让邪影煮酒之人,除了江侯爷之外又有几人,只凭这难得的荣耀,在下的胆量也会大起来的。。”

小顺子熟练地控制着火候,观看壶中酒色,口中却道:“若是寻常人物,我定不会给你废话的机会,不过你这人倒也有趣。据我所知,你先为霍纪城侍从,后为韦膺腹心,霍纪城死后,你仍旧赡养他的妻儿,直到霍夫人过世,霍琮失踪之后,你才离开长安,可谓仁至义尽。韦膺死后,你又秉承他的遗命,先去巴郡呈剑,后至淮西胁迫霍琮,意图谋害我家公子,你可知道,这两件事哪一件都可以让你粉身碎骨,可是你却有胆量做了。霍纪城、韦膺都不是什么人杰,对你也是利用多过恩义,为何你还要不顾生死,对他们忠心耿耿呢?”言罢,他倒出一盏已经温热的美酒递给厉鸣。

厉鸣接过酒盏一饮而尽,道:“厉某乃是蜀中厉家的外系子弟,生来愚笨,父母早亡,族中就是寻常的外姓弟子也敢欺凌我,别人瞧我不起,只有霍师兄将我留在身边照应,虽然多半是为了指使我做些琐事,可是平日却也指点我的武功,对我也算不薄,后来他叛门而出,建立锦绣盟,我想在厉家也没有什么意思,就随他去了。不过我武艺低微,他也看不上眼,就只让我作个随从,不过没多久他就结识了夫人,夫人也是大家闺秀,只是因为战祸才被迫避难乡下,霍师兄说夫人像极了他弃婚出走的未婚妻子,所以就强行娶了夫人为妻。那时候锦绣盟也越来越艰难,夫人刚刚生下公子,身子也很不好,霍师兄就让我诈死,然后带着我将夫人和公子送到长安隐居,从那以后我便留在长安照看夫人和公子。当年霍师兄和太子李安合作的时候,还曾经暗中来见过夫人,可是后来却突然没有了音讯,虽然锦绣盟仍然纵横江湖,我和夫人却都知道他已经死了。没过多久,夫人就一病不起,其实从到长安那一日,夫人就一直病着,她过世之后,我带着小公子安葬了夫人,本来想将小公子带回蜀中去,谁知道他竟会突然不见了,后来我就没有再找他,霍琮聪明得很,我想他一定是已经想好了该做什么。”

小顺子又倒了一杯酒,这次却是自己饮了,道:“霍纪城生性凉薄,他不过将你当成仆役,又不惧你背叛他,才以妻子相托,若是他需要的时候,必会毫不犹豫地将你牺牲,你能做到这般地步,当真是仁至义尽了。”

厉鸣也斟了一杯酒,喝下之后,面上多了几分潮红,又道:“我没有什么本事,从前霍师兄说什么我就做什么,霍师兄死后,我一个人江湖飘零,很是艰难,后来沦为盗匪,可是我心不狠手不辣,经常吃亏,不是平白放过了肥羊,就是被别人黑吃黑,幸好当初在霍师兄督促下,我的功夫倒也说得过去,才能挣扎着活了下来。后来有一次我被人暗算,被首座救了起来,他见我人还老实,就让我跟在他身边。若论武功本领,辰堂中胜过我的人很多,可是首座却将我当成心腹,很多事情都让我去办,就是有些什么差错,首座也往往掩盖过去,首座御下极严,若是别人出了差错,多半是要重重责罚的,可是对我总是网开一面,这般恩情我终生难忘。这次他要去南闽,便跟我说,他不会活着回来了,临行托我两件事,一件事就是将大将军留下的佩剑和书信送到余将军手中,首座说,这件事最重要,让我一定要做到,如果这件事办完了,就让我找到霍公子,逼他刺杀江侯爷,我原本很担心连累霍公子,可是霍师兄的恩我报了,首座的恩还没有报,就只好答应了,当日胁迫霍公子的话语就是首座让我背下来的,果然很管用。”

小顺子眼中闪过利芒,道:“你可知道此事一旦被我发觉,不仅霍琮必死,就是你也不能逃过我的追杀,我家公子何等身份,岂容你等阴谋暗害?”

厉鸣眼中闪过黯然之色,道:“首座说这件事情有六成把握,如今既然是李爷到这里来,那么下毒之事定是失败了,不过首座说过,就是霍公子失手了,江侯也未必会杀了他,首座说江侯爷虽然狠毒,可是有时候又会有些妇人之仁,否则两国征战,害死敌方大将这种事情,还顾什么师徒情谊。首座也说过,不论成功失败,我都是不能活了,所以我若不愿意,他也不为难我。可是我想来想去,还是觉得不能辜负首座的信任,所以就答应了下来,不知道霍公子现在是死了还是活着。”

小顺子目光闪烁良久,道:“那毒药的确厉害,不过也瞒不过公子的眼睛,不过霍琮没有死,公子没有杀他。说起来,我倒真是佩服韦膺的计策,挑动霍琮刺杀公子,若是成功了,自然最好不过,若是不成功,令公子师徒相残,他也是达到目的了。”

厉鸣愕然道:“你也知道首座是这样想的么?当日我告诉首座,江侯身边的少年竟是霍盟主亲子之时,首座苦思良久才想出这个法子,他说江侯心脉最弱,当年曾经险死还生,这次见到江侯祭奠大将军之时,首座便看出他的心脉再度受到重创,七情伤人,自古如此。所以首座并不指望霍公子可以得手,但是只要江哲得知自己的心爱弟子竟会刺杀于他,必然加重病情,纵然不会伤及江侯性命,至少也可夺了他十年寿元。首座还说,这计策最好等他死后再用,江侯心思细密,只要首座在世一日,就不会放松对辰堂的监视,可是首座死后就不同了,人死如灯灭,谁会提防一个死人呢,所以让我办完巴郡那件事情之后再动手。”

小顺子目中闪过悲悯之色,也有一丝敬佩之意,道:“韦膺果然厉害,死后犹有遗策,公子想来也不会想到韦膺心中仇恨,竟是死也不能释怀吧。”

厉鸣闻言大笑,面上的质朴神情淡化了几分,却是多了些慷慨之意,又倒了一杯酒饮下,道:“能够得到邪影赞誉,想来首座也会死得其所了,我也不劳阁下动手,霍琮身世,我并没有告诉别人,他既然活着,你就告诉他一声,无论如何,当初我也受过霍师兄的恩惠,总会替他留下一线生机的,否则就是江侯爷不想杀他,霍盟主的仇人也绝不会放过他的。”说到最后几句话,声音渐渐低沉下去,面色开始变得青黑。

小顺子目光一寒,走到他身边,把脉探视,心知这人是在见到自己之后便服下了毒药的,不过是直到此刻才毒发身亡,方才他纵情饮酒,应是为了促使毒药快些发作。这种毒药他也知道一二,发作之时颇为苦痛,但是外表却不见征兆,等到被人发觉之时,已经无法可救,不由叹道:“离天明之时还有不短时间,你何必这样急着去死呢?”

厉鸣艰难地道:“我不过是个寻常人,我怕死,也怕有人折辱,所以很早就向首座要了自尽的毒药,见到李爷亲自来寿春,心中很是害怕,所以才提前服下了毒药,若是早知道李爷这般和气,就会等到天明再死了。”

小顺子急急问道:“你可知道陆风在何处,我家公子知道他在韦膺手中。”

厉鸣眼中露出释然之色,勉力道:“是要问这件事情么?首座让他住在毒龙泽,可是首座死后我去寻他的时候,他已经不见了,不过应该没有死。”最后几个字已经是几不可闻,眼中的神采更是渐渐黯淡下去。。

小顺子知道再也问不出什么了,叹道:“劝君更进一杯酒,此去泉台多故人,你也算是个英雄,好生去吧。”不愿看厉鸣再挣扎下去,一指点断了他的心脉,厉鸣的呼吸终于停止了,青黑扭曲的面容上仍然带着一丝微笑。

小顺子心道,这人虽然才能平庸,但却是心直意诚,怪不得能够得到韦膺信任,又以身后相托。想到此处,心中也有怜悯之意,若是他知道韦膺计策失败,只怕临死也会自责不已吧,自己为了斩草除根而来,为了探听是否还有人知道霍琮身世,所以没有告诉厉鸣真相,虽然是白来了一趟,却能让他安心死去,倒也不错。

要办的事情已经办完,小顺子此刻想来,却仍觉侥幸,韦膺遗策,当真是狠毒非常,若非霍琮自己想通了,只怕江哲当真会被迫面对师徒相残的惨剧,若是从前倒也无妨,偏偏是江哲心脉再受重创之时,当真是趁人病,取人命,这等雪上加霜的手段,若真的得逞,公子恐怕不死也要去掉半条命,折损十年寿元都是韦膺少算了吧。长叹一声,小顺子走出客房,见两个店伙计仍在廊下候命,淡淡道:“你们将此人妥善安葬了吧。”说罢身形便没入雨中,转瞬不见,那两人面面相觑,心中都怀疑见到的是否鬼魅。

丝毫没有停留,小顺子连夜赶回徐州,无论江哲身边有多少高手保护,他若不在身边,总是放心不下。奔行之间,突然想起六年前随公子前往拜谒魔宗之时,京无极曾对自己说过,欲成大道,需先放下,若是自己不能放下心中牵挂,终究只是井底之蛙,心中虽有不服,但是想到京无极浩瀚如海,不可揣测的修为,比起当年道左相逢之时不知精进了多少倍,想来就是放弃了世俗之争的缘故。身形轻展,便如轻尘随风,瞬间掠过百丈荒郊,小顺子微微一笑,若是没有那人,将一切放下,倒也没有什么,只是现在自己却是万万不舍的。

数百里道路,在小顺子来说不过是寻常,天色未明,他已经到了凝碧园外,目光一凝,却见门口许多侍卫在那里探头探脑,有人满面苦涩,有人焦虑非常,不由心中一惊,莫非自己只去了一日,便有什么古怪的事情发生了么?

心中满是疑惑,但是确信空气中没有悲哀和痛悔的意味,小顺子略略放下担忧,走到门口,向几个侍卫冷冷问道:“怎么回事,你们都跑了出来,若是让刺客混了进去,你们是不是不想活了?”

众人都是只觉眼前一花,便看到小顺子负手站在门前,一个职位较高的虎贲侍卫连忙凑到小顺子身前慌忙禀道:“李爷你可回来了,霍公子吩咐下来,若是李爷一回来,便要请你去劝劝侯爷。”

小顺子微微一愣,疾步走入凝碧园,只见园中侍卫都被逐了出去,心中不由十分烦恼,霍琮做事一向很是得体,今次却是怎么回事,走到江哲居处前面,目光便是一凝,只见在门外跪着两人,一人明黄袍服,正是太子李骏,另一人黄衫翠袖,却正是柔蓝。小顺子心中立刻明白过来,怪不得霍琮将人都赶了出去,这种情形若是给人看到,不仅太子颜面全无,就是公子也难免会有麻烦。

走到两人身后,有些无奈地道:“太子殿下、柔蓝,这是怎么回事,这要是传了出去,岂不是有失体统。”

两人听见小顺子声音,都如同听见纶音一般,柔蓝第一个要站起身来,大概是跪得久了,膝盖一软,差点跌倒在地,被李骏扶住,两人转过身来,柔蓝一看到小顺子便是泪如泉涌,哇的一声大哭起来,继而扑到小顺子怀中,哽咽道:“顺叔叔,你一向最疼蓝儿的,你去跟爹爹求情好不好,蓝儿不要嫁给霍哥哥。”

小顺子这才想起前几日江哲将柔蓝许配给霍琮的事情,只是霍琮不是暂时稳住了公子么,怎么这两人现在就知道了,见他脸上露出疑惑之色,李骏尴尬地道:“是我的错,我接到霍琮的书信,一时按耐不住,就从楚州连夜赶来,想求姑夫将蓝儿许配给我,姑夫断然拒绝,还让我立刻回楚州去,我,我一时想不开,就在姑夫门前跪着,结果惊动了蓝儿,蓝儿也来相求,姑夫却是不肯答应。”李骏在小顺子面前一向是不敢摆太子殿下的架子的,也不知是否早想到有今日之事。

小顺子有些犹豫,柔蓝和李骏两人有情,他自然是知道的,可是却没有看的十分重要,再说柔蓝和霍琮、李麟也颇为亲近,所以知道江哲的意思之后也并未相劝,在他看来,公子之命自然不可违抗,更何况霍琮和柔蓝订婚,倒是更妥当一些。想不到竟会掀起轩然大波,不说别的,李骏擅离职守,就是一大罪责,更何况让未来的天子跪了这许久,这也说不过去。想了一想,他也不理李骏,放开柔蓝,淡淡问道:“蓝儿,公子爱你如同掌上明珠,他将你许配给霍琮也是一片苦心,若为太子妃,你便要将来和别的女子争宠,若是嫁给霍琮,他绝不敢有纳妾之心,再说你和琮儿也是自幼一起长大,他的为人品性你应该清楚的很,这样的好男儿若是错过了,便再没有这样的机会了。”

柔蓝见小顺子也这样说,不由泣道:“顺叔叔,我知道霍哥哥很好,可是我一向都将他当成亲哥哥看待的,我一直都喜欢骏哥哥的,若是我真的答应爹爹,嫁给霍哥哥,岂不是对不起他么。”

李骏也急急道:“顺叔,李骏可以对天起誓,绝不会辜负蓝儿,若我负她,天诛地灭。”

小顺子冷冷道:“太子殿下,你将来是要做皇帝的,按照礼制,不论是你自己怎么想,四妃九嫔的位子上都要摆个人的,我家蓝儿,如珠似玉,一向娇宠,岂能去和别的女子争宠。”

李骏愣了半晌,道:“顺叔,我不敢说将来只有蓝儿一个,您说的对,不论我对蓝儿如何真心诚意,一朝登基为帝,必然会有妃嫔侍妾,这是礼法,也是规矩。可是李骏情愿立誓,今生今世,绝不会有别的女子夺去我的心,更不会让别的女子生下子嗣,日后的事情我不敢说,可是父皇如今春秋正盛,我这个太子怎么也可以再做二三十年的,在我即位之前,绝不会再娶妃妾。”

若是李骏信誓旦旦地说自己绝不会再纳妃嫔,不仅小顺子不信,就是柔蓝也会生疑,可是他这样说来,两人却都相信了他的诚意。

柔蓝虽然依旧满面泪痕,但是仍然忍不住露出笑容,便如出水芙蓉一般娇艳清丽,李骏不由看得呆了,直到柔蓝一脸羞红地避开他的目光,他才清醒过来,又企求地看向小顺子,他知道,若是没有此人从中转圜,只怕不等自己父皇设法,柔蓝便要嫁给霍琮了。

小顺子叹了口气,道:“这件事情便交给我吧,我可以说服公子,只要蓝小姐不愿意,就不会迫她成婚,但是太子殿下也不要急着求婚,柔蓝年纪还小,过两年再论婚姻也不迟。太子殿下身负重责,还是快些回楚州去吧,今日之事若是传扬出去,只怕你和柔蓝小姐的婚事就更没有希望了。”

李骏心中一寒,立刻想起了自己擅离职守的罪责,虽然楚州那里暂时应该无事,可若是万一有变,父皇必定怪罪下来,窥伺太子妃之位的人不在少数,若是柔蓝担上“祸水”之名,这婚事果然是没有指望了他虽然一时情令智昏,可是毕竟不是蠢人,望了紧闭的房门一眼,咬咬牙道:“孤这就回楚州去,不过霍琮这些日子本在孤身边行走,若是没有他参赞,孤总觉得不安心,就让他和孤一起回楚州吧。”

小顺子和柔蓝都是一愣,继而明白过来,若是霍琮留在柔蓝身边,只怕李骏是绝对不能放心的了,而且霍琮本来已经跟在李骏身边效命,李骏这样说话也是情理之中,霍琮就是想不去都不行。

小顺子和柔蓝踌躇未决,房门却开了,霍琮满面苦涩地走了出来,道:“先生吩咐,为人臣子应以国事为重,令霍琮跟随殿下左右,即刻动身。”李骏大喜,上前拉着霍琮的手道:“霍兄放心,若非霍兄传信,孤只怕已经终身遗憾,孤绝无恶意,只是需要仰赖霍兄大才,还请霍兄不吝助我。”

霍琮又是暗暗苦笑,心道,我这是何苦来由,本来是想助你成就好事,却将自己也陷了进去,你若不是这般急躁,说不定过不了多久就可以水到渠成,也免得生出这许多是非了。

将外面的事情一一安排妥当,小顺子这才抽身去见江哲,进得室内,只见江哲神色恼恨地坐在榻上,黑白棋子拂乱一地,几本书册翻落在地上,显然是遭到了池鱼之殃,忍不住露出笑意,道:“公子此番错点鸳鸯谱,惹起这许多麻烦,可是已经有了解决的法子么?”

我气恼地道:“最可恨的就是霍琮那小子,他若不愿娶蓝儿也就罢了,直接跟我说明白,不就没事了么,却非要传信给李骏,惹出这么多事来,当真可恨至极,这次就让他跟李骏去淮东,我倒要看看,李骏这小子怎么对付自己的情敌。”

小顺子失笑道:“琮儿不说也是情有可原,若是公子前几日就知道柔蓝已经和太子两情相悦,只怕立刻迫着他们两人拜堂都有可能,只不过他大概没有想到太子这般沉不住气。公子,其实太子也是真心诚意,蓝儿对他也是一往情深,你何必定要相阻呢?”

我摇头道:“先不说李骏的身份,我虽然不愿蓝儿嫁入皇室,但还另有一个原因,命相之学虽然虚无飘渺,却也不是没有道理,李骏这孩子聪明颖悟,又有仁厚之心,本是极好的,可却偏偏少了几分福气,蓝儿这孩子我素来钟爱,实在不忍她将来受苦。”

小顺子见江哲神色坚决,知道这一次很难改变江哲的心意,便道:“那我方才答应太子的那件事,公子可不会让我失信吧?”

我笑道:“那我怎敢,若是让邪影失了信诺,只怕我这苦头就吃不尽了,也罢,柔蓝的婚事先缓一缓也可以,不过这世上能够配得上蓝儿的少年本就不多,除了琮儿之外,我还真没有看中哪个,若是蓝儿不想嫁琮儿,我可以不逼她,不过她若想嫁给别人,也得我中意才行,只是李骏却是不行。”

小顺子无奈地摇摇头,江哲若是执拗起来,他也是没有法子的,能够让他做出些让步已经是不易了,无意中想起江哲已经上过请求赐婚的圣旨,不由问道:“公子,那你的奏折已经递上了去,这怎么办?”

我已经有些疲倦了,淡淡道:“这有什么要紧,若是皇上下旨赐婚,那可就不是我们说话不算了,李骏若想娶蓝儿,自会解决此事,不用我们操心,再说有那道奏章在,皇上也不能随便将蓝儿立为太子妃,这不是很好么?”

说到最后几个字,语声已经是极为低微,小顺子见江哲气息渐沉,竟是昏昏欲睡的模样,想来太子殿下在外面跪着,他心中也十分不好受吧,轻轻一笑,将江哲身上的裘被盖好,轻手轻脚地将散落的棋子和书本收起,然后便坐在椅上调息起来,一路急奔,他倒也有些倦意呢?

\chapter{第四十七章 离鸾别凤}

十六年,雍军据江淮之地,欲南渡,朝廷恐惧,屈膝求和,以金宝女乐赂齐王显,急切未得,以柳姬色艺冠绝江南,令甲士劫取,舆送雍营。

——《南朝楚史·柳姬传》

大雍隆盛十二年,扬州城外,瓜州渡口,两岸皆是大军云集,旌旗遮天蔽日。雍军再度兵临长江,这一次大雍的主将仍是裴云,只不过尚有大雍江南行辕的副帅太子李骏督军,令人深悉雍军渡江南征的决心。

寒风萧瑟,阴冷刺骨,彤云密布,霍琮掀开帐门向外看了一眼天色,寒风扑面,令霍琮精神一振,眉宇间却多了一丝烦恼,补给的粮草和御寒冬衣昨日就应该到了,眼看今冬的第一场大雪就要下了,雪落之后,必定寒意大增,若没有足够的御寒衣物,将士们可要受苦了。叹了一口气,他放下帐帘,觉得周身有些寒意,便走到帐内一角,从床边黄杨木箱上面拿了一只杯子,然后从帐内中间的铜火炉上面煨着的酒壶中倒了一杯酒。等到酒液变得温热之后,才缓缓喝了一口,幽深的双目中多了几分懈怠。拿着酒杯回到书案前,提笔将剩下的公文处理完毕,等到他将整理好的文书放到一边的时候,杯中酒已经涓滴不胜。

正在这时,帐帘被掀开,寒风卷着飞雪扑入,却是一个身穿明黄戎装的少年大踏步走了进来,大氅之上满是积雪,却正是太子李骏,李骏笑道:“还是你知道偷懒,孤和裴将军到江边观阵,可是冻得半死呢?”

霍琮连忙站起身,上前帮李骏解去大氅,又取杯倒了酒呈上,辩解道:“殿下这可是随便冤枉人了,臣若不是忙着整理文书,也定会陪着殿下去观阵的,不知道楚军的虚实如何?”

李骏喝了一杯酒,觉得身子暖和了许多,笑道:“急切之间也看不出什么,不过裴将军可是很想快些开战呢,五年前他在瓜州战败,至今仍然当作奇耻大辱,更何况后来南楚军在淮东发难,泗州失守,差点连楚州也不保,却都是兵力不足的缘故,接下来两三年,王叔又不许他攻泗州,这些年隐忍不发,早就将裴将军这只猛虎憋惨了,若不是孤拦着,只怕他就要催舟渡江了。”

霍琮笑道:“裴将军只不过想一鼓作气,攻过江去,免得时日拖延久了,反而让杨秀稳住了防线,毕竟长江天险极难逾越。不过齐王殿下有令,让咱们明春再渡长江,想来定是已经有了定策,我军自然只能遵命行事。其实这两年,裴将军步步进逼,夺泗州,渡淮水,破泗州,重夺广陵,再临扬州,饮马长江,还有何人能以从前之事嘲讽他呢?”

李骏深以为然地点点头,目光无意中落到书案上,却看到一封书信,落款却是江哲,脸色立刻阴沉下来,叹了口气,道:“姑夫又有信来了么?”

霍琮淡淡道:“是啊,先生来信说今冬扬州应该没有战事,让臣去合肥见他。”

霍琮话音方落,李骏已经捏碎了手中酒杯,恶狠狠地看向霍琮,道:“你准备去合肥么?”

霍琮心道,我若真的想去,只怕都走不出大营,只能苦笑道:“殿下,臣的心意,殿下又不是不知道,若我对蓝儿真有求凰之意,只怕此刻早就和蓝儿成婚了。”

李骏闻言愣住,脸上露出一丝歉意,继而又变得愁眉苦脸,在他心目中,早将柔蓝当成了自己未来的太子妃,父皇和母后也都早已许可,本以为迟早可以两心如一,白首偕老,不料两年前突生大变,姻缘路上凭添波折,他已经是苦苦相求,无奈江哲就是不肯许婚,反而几次有意将霍琮招回身边,好让霍琮和柔蓝完婚,若非柔蓝坚决不肯,自己又扣住霍琮不放,只怕自己已经情天抱恨了。虽然他暗中写信给母后求助,可是母后回信说,父皇已经暂时压下了请婚的奏折,只不过若不得得到江哲同意,就是父皇也不好擅自赐婚的,这可怎么办呢?

见李骏愁眉苦脸,霍琮心中也不好受,这两年战事进展十分顺利,西线秦勇攻下巴郡、夔州,长孙冀将军也已经攻下了竟陵和随州,淮西荆迟部更是已经攻到了历阳,就连江南行辕也已经在月前移到了合肥,这本是令人心情愉快的事情,可是只要想到自己却在太子身边提心吊胆地效力,时刻都要提防太子想起自己乃是情敌身份,就越发后悔当初自作聪明地报信给李骏,若非如此,想来先生也不会任由自己跟在李骏身边受这些尴尬吧。

正在帐内气氛越发沉闷的时候,有军士在外禀报,说是有人求见霍琮,霍琮虽然不知是何人求见,但是一来心中奇怪,二来也正想避开一下,便和李骏说了一声,任由他在那里烦恼,自己走到旁边的军帐,令人将求见之人带来。来人是一个三旬年纪的男子,相貌平平,却是隐隐威仪,令人不敢小觑。霍琮一见到他便大惊起身,上前施礼道:“白义师兄怎会来此,莫非是先生有什么谕令么?”

白义微微苦笑道:“这两年我们已经很少接到先生的谕令了,这次来见你也是为了一件私事,想要求你帮忙。”

霍琮心中越发疑惑,这些师兄的本事他是知道的,而且八骏之间彼此同气联枝,还有什么事情需要自己相助呢,转念一想,已经猜到定是和先生有关,说起来自己在先生面前应该比八骏占些优势,想通这一点,他恭恭敬敬地道:“师兄请说,小弟必然尽心竭力。”

白义犹豫了一下,才道:“现在大雍已经尽占江北之地,南楚朝廷便如日落西山,所以有意求和,为了讨好雍军主帅,除了金银珠宝之外,又送了些美人女乐,希望能够换取齐王殿下暂缓攻势,允许和谈。”

霍琮闻言,不由笑道:“这不是病急乱投医么,谁不知道齐王殿下自从和嘉平公主成婚之后,早已经不再流连声色犬马了。”

白义苦笑道:“有些事情很难令人相信的,更何况齐王殿下领军在外已经五六年了,也难怪他们这样想,不过寻常美人也就罢了,为了博得齐王欢心,尚维钧强行将秦淮两大花魁送到了合肥,这却有些过分了。这两人一人叫灵雨,乃是凤仪门幸存之人,一人叫柳如梦,却是四弟逾轮的心上人,如今先生就在合肥齐王殿下身侧,我是想请师弟去向先生说项,请他向齐王进言,放过柳姑娘。”

霍琮有些奇怪,道:“这样的事情若是先生知道,自然会尽力的,为何师兄却要托我进言呢?”

白义苦笑摇头,只能将逾轮离开秘营之事略略说来,霍琮听后凝神想了许久,道:“师兄放心,我接到先生书信,正准备去合肥呢,这件事情在下一定尽力相助,逾轮师兄现在何处,可知道此事么?”

白义叹道:“正因为他已经知道此事,更已经赶向合肥去了,我才这般担心,逾轮不知何故,对先生似有怀恨之意,我担心他不会去求先生,可能会用武力救人,可是雍营高手如云,又有千军万马,我担心就是先生不为难他,他也逃不过一死,再说柳姑娘才貌天下少有,若是有什么闪失,就是逾轮得以生还,只怕也会心碎而死,所以才求师弟去向先生求情,若没有先生援手,只怕他们,唉!”

霍琮点头道:“逾轮师兄虽然已经离开秘营,毕竟仍是我们的同门,怎能不尽力相助,而且据师兄所说,先生对他一向优容,这次说不定也是一个转机,不过凤仪门怎么还有余孽存活,莫非先生不想斩尽杀绝么?”

白义笑道:“凤仪门已经烟消云散,剩下的余孽只要没有大成就的就不必过问了,那灵雨姑娘虽然是入室弟子,但是一来生性平和,并无野心,二来却是有人看中了她,所以我们也不敢去为难她,还要设法照顾一二呢。”

霍琮听得奇怪,道:“能够令师兄屈尊照应,想必那贵人身份必然不同寻常,怎么却任由灵雨姑娘流落风尘呢?”

白义闻言低声道:“这件事情为难得很,看中灵雨姑娘的是秋四公子,原本他是想把人接走的,可是偏偏灵雨姑娘是纪霞的弟子,四公子不敢擅专,需要魔宗许可才行,据说魔宗没说答应也没说不答应,只是让四公子闭关三年,所以灵雨姑娘现在还在建业。不过也难怪四公子中意她,这位姑娘温柔贤淑,又是精通音律,想来和四公子定是知音相遇,彼此情投意合吧。只是魔宗若不点头,四公子却也别想将她娶回去,不过虽然如此,我们也不敢怠慢了她,倒还担心魔宗干脆派人取她性命呢。这样我们可没有办法向四公子交待。”

霍琮听得不由长叹,道:“世间偏有许多风雨,拆散鸳鸯无数,不过这位灵雨姑娘既然是四公子的意中人,想来先生必然不会慢待,倒是柳姑娘的事情也不知道先生是否知道。”

白义犹豫了一下,道:“有些事情师弟你不清楚,柳姑娘品貌性情都似先生一位故人,为了不愿先生伤心,她的事情我们是不敢向先生禀报的,要不然现在也不必去求情了。”他没有说出另外一种担心,八骏对于江哲昔年与柳飘香的情事都是知道一些的,甚至大半都曾见过这位在秦淮河上光芒四射的名妓,虽然江哲和长乐公主相敬如宾,但若是江哲因柳如梦神似故人而移情在她身上,那可是大大的麻烦,姑且不论长乐公主这边,逾轮又情何以堪呢?

霍琮听得模糊,他虽然深得江哲喜爱信任,但是江哲昔年情事自然不会告诉他知道,如今隐隐猜知江哲当年也有伤情之事,原本模糊的想法渐渐明晰起来,送走了白义之后,他回到帐中,不由扼腕道:“这可是难得的好机会,若不趁机解决太子殿下和蓝儿的婚事,我恐怕非得和太子殿下抢心上人了。”

合肥内外,大军云集,原本的淮西重镇,如今已经成了大雍江南行辕的大营,四个月之前荆迟攻下合肥,一月之前,李显将行辕移到此处,大雍已经尽得江北之地,只待李显一声令下,就可渡江南下,不过目前似乎李显还没有在隆冬作战的打算。除了严防南楚军的反攻之外,便是在合肥休整士卒,每隔三日五日,便要召宴军中将士,合肥城内歌舞升平,倒似是雍军有意划江而止一般。当南楚求和使者来到合肥城外的时候,就感觉到了这样的气氛,只觉求和成功的希望凭白添了几分。

这次前来求和的使者便是尚维钧尚承业,非是尚承业胆量够大,只因此事牵连极广,为了取得和议,尚维钧已经准备答应任何苛刻的条件,只要换取雍军不渡长江的承诺,雍军如今挟必胜之威,若要他们同意和议,必然要付出惨重的代价,这些事情不足为人道,自然只能派尚承业来了。

到了城外,已经是日暮黄昏,按照齐王李显之命,南楚使者今夜就在城外扎营,又遣了军士在外宿卫,明日上午才会召见南楚使者。虽然觉得李显无礼,但是此刻尚承业也不敢计较,只能吩咐安顿下来,这次他所带的贡品礼物就有三十余辆马车,安置起来也是费了半天时间,等到一切安排妥当之后,已经是酉时末了。尚承业尚不放心,又到被选为女乐掌班的柳如梦、灵雨帐中巡视一番,见两人神色冷漠,但是气色还好,这才放心下来,又劝慰了几句,见两女都是恍若未闻,也只能摇摇头回去休息了。

见到尚承业走了,柳如梦眼中闪过一丝恨意,又担忧地对灵雨说道:“妹妹,你是会些武功的,不如趁机逃了吧,若是进了合肥,就再也没有机会了,我虽然不大清楚江湖事,也知道妹妹从前所属的门派在大雍乃是钦犯身份。”

灵雨叹道:“我怎能让姐姐独自去面对雍人,更何况灵雨纵然想逃,又能逃到哪里去呢,姐姐不必说了。”

柳如梦见灵雨神色黯然,纤纤素手却在抚摸着那块雕成古琴模样的玉佩,不由叹道:“世间偏多薄幸男儿,妹妹何需日日牵挂那无情之人,多半是个纨绔子弟,偶然间留香月影罢了。”

灵雨淡淡道:“小妹和那位四公子不过是音律知交,却也谈不上什么无情薄幸,小妹只是惋惜没有机会从他学琴罢了。”

见到灵雨楚楚可人、淡雅清灵的风姿,柳如梦笑道:“如此佳人,我见尤怜,何况那些鲁男子,我便不信那位四公子见到妹妹才貌,会不动心?不知是出了什么纰漏,才会鸳梦难温吧。”言罢却动了兴致,放声唱道:“珊瑚叶上鸳鸯鸟,凤凰巢里雏鹓儿。巢倾枝折凤归去,条枯叶落狂风吹。一朝零落无人问,万古摧残君讵知。(注1)”

她本是江南歌舞第一的名妓,唱支曲子正是最容易不过的事情,原本她是有心调笑灵雨,岂料只唱了两句,便觉悲从心起,想起那一去无踪的宋逾,当真动了深情,唱到最后两句,已经是悲切难言,令人闻之泪落。

灵雨自从当日被柳如梦接去之后,两人琴歌相合已经是寻常之事,见柳如梦歌中已经是悲难自抑,担心她伤心过甚,便取来古琴,轻抚一曲《猗兰操》,琴音平和,不过片刻,柳如梦便已经止住悲声。灵雨心中也是惆怅难言,琴声一变,却是弹起了《离鸾操》,漫声唱道:

“妾本书香子,爱清商、朱弦弹绝,玉笙吹遍。不学国风关雎乱,闲来幽兰白雪。总不涉、闺情春怨。无端陌上狂风急,要珠鞍、迎入梨花院。清泪洒,意踌躇。

夕阳红处是金屋,泣孤芳、生在秋江,晓寒漠漠。勾弦拨珠话风雨,道是华堂遣愁。回首望、音尘绝矣。我有平生离鸾操,颇哀而不愠微而婉。聊一奏,更三叹。(注2)”

若单论歌喉,灵雨自然不如柳如梦,可是也是一时之选,这一曲更是自伤身世,情真意切。

两女自以琴歌抒怀,却听得营中众人如痴如醉,便是营地外面宿卫的雍军将士,虽然多半是些只知杀伐征战的豪勇战士,却也不由心醉,浑忘却身在何地。

而在南楚使者大营之外,幽深夜色之中,一个身影紧握双拳,痴痴地听着夜风中缥缈的琴歌,良久,他低声道:“一朝零落无人问,万古摧残君讵知。如梦,是我辜负你的情意,今次除非是我死在这里,否则定要将你带走。”声音未息,他的身影已经如同魅影一般前掠,江南第一杀手的绝技展现无疑,不过片刻之间,已经绕过重重防线,接近了柳如梦和灵雨居住的营帐,透过帘幕可以隐隐看到灯火明灭。那人伏下身形,听了片刻,在帐外低声唤道:“如梦!”拼着他的灵敏听觉,可以听到帐内两人都是一声低呼,一个熟悉的动人声音道:“宋逾,是你么?”

宋逾心中一暖,闪身进了帐内,只见灯光之下,身着素衣的柳如梦正凝神瞧向自己,两年不见,虽然柳如梦风华更胜昔日,可是在宋逾看来,却觉得她眉梢眼角多了几许轻愁倦意,强自抑制的深情瞬间迸发出来,全没留意到帐内另外一人何种形貌,他上前一把将日思夜想的佳人揽入怀中,当他感觉到柳如梦反手将他抱住的时候,原本深刻心中的影子渐渐淡去,这一刻他心中只有柳如梦一人。不知过了多久,宋逾清醒过来,低声道:“梦儿,跟我走,我绝不会让你被人当成礼物送到雍营。”

柳如梦拭去面上清泪,回头道:“灵雨妹妹,和我们一起走吧。”

灵雨面上也露出喜色,道:“恭喜姐姐和宋先生今日团圆,小妹从前不走,是因为没有把握带着姐姐一起走,既然如今有宋先生相助,自然是要一起走的。”

柳如梦大喜,对宋逾道:“灵雨妹妹也会轻功,应该不会妨碍你吧?”

逾轮微微苦笑,心道,你既然已经答应了,我难道还能反对么,他不知灵雨和秋玉飞之事,却知道她的出身,想来应该武功不会太差劲,便点头道:“你们收拾一下,等到三更我们便一起走。”

两女都知道情况紧急,只是收拾了一下首饰细软,灵雨又将古琴带在身上,这却是无法让她放弃的。三人熄了灯火,苦苦等到三更时分,逾轮到帐外探察了一回,便带着两人潜出营帐。营内乃是南楚禁军守卫,守卫松懈,逾轮本就是杀手,纵然带着柳如梦,仍然游刃有余,灵雨虽然武功生疏,可是凤仪门轻功名动天下,不多时三人就已经到了营地边缘。逾轮折扇轻指,然后身形疾闪,将两个被扇中毒针射杀的军士扶住,将他们摆成僵立模样,回身便欲带了柳如梦出去。刚刚握住柳如梦素手,便觉一缕剑气从后袭来,逾轮几乎是本能的向前扑去,耳中传来柳如梦的惊呼,逾轮也顾不上惊动营中楚军,狂奔疾驰,想要抛开身后威胁,可是那缕剑气如附骨之蛆一般在他后心吞吐,逾轮心中生出不能逃脱的颓丧之感。

就在这时,身后传来剑刃相接的铮鸣之声,那剑气蓦然一滞,逾轮趁机转过身来,只见灵雨手执一柄软剑正在和一个身着南楚禁军服色的男子交手,那人剑势便如星河影动,浩瀚如海,实在是绝顶的剑术,而灵雨素衣雪剑,剑光闪烁绽放,便如寒梅立雪,华光溢彩,正是凤仪门嫡传的绝世剑法。

逾轮一声冷笑,手中折扇一指,一缕乌光射向那男子要害,他看准了灵雨剑势,这枚暗器觑准了那男子身形移动的位置,本是万无一失,但就在暗器飞出的一瞬,逾轮却神色大变,灵雨身形突然出现在暗器的轨迹上,出乎逾轮的预料,自己的暗器竟然向灵雨背心袭去,眼看这素来温柔婉约,从不与人相争的女子就要香消玉陨,逾轮不由一声惊呼。

灵雨仍不知身后危机,她虽然不喜武功,可是若是练得太差,也难以应付纪霞,再加上她天资聪颖,倒也有几分成就,只不过缺少和人交手的经验,也没有交锋厮杀的勇气。这一次被迫送到雍营,她也心中惊惧,便寻出原本纪霞赐给她的软剑带在身上,除了柳如梦之外,别人都不知道。方才见到突然有人出现追杀逾轮,危在旦夕,灵雨眼力不足,看不出那人并无杀意,又见柳如梦神色惊惶,这才鼓起勇气拔出腰间软剑冲出拦阻,什么也不敢去想,剑光电闪,连绵不绝,为了救人心中全无杂念,摒去惧意,却是意与剑合,得心应手,竟然拦住那人追袭。但是交手三四招之后,心知宋逾必然已经脱险,又见那人剑势如山,灵雨心中生出怯意,剑势立刻变得散乱,便索性向一边闪退,不敢再和那人交战,孰料逾轮料错她的修为胆量,以暗器助阵,却将灵雨陷入死亡之境。

就在逾轮惊叫不忍目睹之时,那禁军军士长剑剑势一转,已经掠过灵雨身形,将那枚乌光击落,这样一来,不免露出了破绽,灵雨原本正欲退走,见状心意一动,她知道这人武功剑术极为高强,担忧宋逾不是他的对手,又不知那人正在救她,便狠起心肠,一剑向那人左肩刺去,她手中软剑可以切金断玉,这一剑又是如同电闪,竟是轻轻刺入肩甲缝隙,鲜血溢出,灵雨顿时骇得手足发软,这一剑再也不能刺下去,只见那人如同冷电的眼光落在她身上,灵雨一声惊叫,也不敢拔剑,闪身疾退,已经避到柳如梦身后。

这种种变化发生在电闪雷鸣之间,直到此刻,柳如梦才明白过来,看到落在地上的暗器,以及跌落在地上的染血软剑,以及灵雨苍白的面色,她虽然不知道灵雨方才之险,却也猜出一二,更是感激她舍命相救宋逾,连忙将她搂入怀中,低声安慰。

那军士苦笑着看了一下染血的肩头,他便是看出灵雨毫无厮杀经验,所以一时不忍出手相救,岂料却被她刺伤,幸好灵雨不敢杀人,这一剑只是皮肉之伤。虽然受了伤,那人心中却并无恨意,一来他出手拦阻已经是心有愧意,二来也是看出灵雨心地善良,乃是从未手染血腥的善良女子,这一剑实在是不得已而为之,轻轻一叹,他将那柄软剑拔下丢到一边,随手扯了一块战袍裹住肩伤,然后取下掩住面容的头盔,道:“宋兄,你还是离开吧。”

逾轮目光落到那人面上,露出难以掩饰的惊容,神色千变万化,对周围闻讯聚集的南楚军士视若不见,良久才道:“当日义薄云天的吴越第一剑,曾为了大将军出生入死,乔园劫囚,仙霞拒敌寇的丁铭丁大侠,为什么如今成了尚维钧的走狗?”

丁铭面上露出一丝惭色,黯然道:“宋公子,丁某非是趋炎附势之人,只是国事艰难,江南危殆,若能和议成功,我南楚千万黎民才有安身立命之地,为着大局着想,丁某只能接受杨参军之托,一路护送使团北上。柳姑娘、灵雨姑娘乃是贡单上有名之人,若是任她们脱逃,必然惹怒大雍,和议便没有任何希望,公子也是心存大义之人,当知利害得失,勿要为了私情湮没大义。”

宋逾环视四周,冷笑道:“和议,哼,大雍席卷天下不过是时间的问题,既无实力,何谈议和,再说,纵然是国家兴亡,匹夫有责,莫非朝中文武大臣,二三十万带甲壮士没有本事捍卫社稷,却要将这重责压到两个女子身上么?纵然你们想做勾践卧薪尝胆,还要看别人愿不愿意做吴王呢,我宋逾不过是个杀手刺客,当初害死大将军我也有份,跟我说什么大义社稷,当真是对牛弹琴,你若定要阻我,我纵然无功而退,也会夜夜窥伺,将你们这些人一一杀死,若是聪明的,就让我们三人离去,否则,哼!”随着他冰冷刺心的话语,一缕漂浮不定的杀气瞬间溢满天地。

众人都听出宋逾话语中凛冽的杀机,都有身处三九冰雪天中也似的感觉,几个胆小的军士已经是面色青白。原本已经在侍卫保护下出帐察看的尚承业只被宋逾那双冰寒刺骨的眼睛望了一眼,顿觉心胆俱寒,再也生不出上前叙旧的胆量,只觉面前这人陌生得很,不像是从前的好友知交,模模糊糊地想起当初欧元宁曾对自己说过这人乃是杀手身份,莫非这才是此人真面目么?

丁铭武功本已极高,感觉却又不同,只觉如海浪一般狂涌的杀气却是变化莫测,飘拂不定,倏忽来去,若有若无,令人生出难以捉摸的无力感觉,便肃容道:“无情公子果然名不虚传,想来从前不过是韬光养晦罢了,就让丁某领教一下公子的杀人绝技。”他本来心有惭意,但是听到宋逾自承与陆灿之死有关,不由生出怒意,想到这人从前为尚承业幕宾,心中已经是信了几分,也不由生出杀意,凌人剑气冲天而起,和宋逾散发出来的杀气撞击在一起,数丈空间内顿时狂风骇浪,迫得那些围伺在侧军士连连后退,柳如梦却是神色怔忡,愣在那里不晓得后退,流溢的剑气劲风呼啸而过,柳如梦一绺青丝削落在地,灵雨醒悟过来,连忙拉着她后退几步,那些军士都怔怔望着对峙的两人,全没有想起可以将两女先挟持住。

剑光一闪,便如星河动摇,逾轮的身影几乎是转瞬之间便被剑浪淹没,丁铭将被迫护送尚承业的仇恨和悲愤全部发泄在逾轮身上,每一剑都是万分凶险,若是逾轮一招失守,便会在流虹飞电一般的剑光下粉身碎骨,只不过这一次逾轮也是全无保留,折扇开阖挥洒,风流雅致,身如柳絮,随风起舞,形如鬼魅,在滔天剑海中若隐若现,丁铭剑势略缓,他便发起致命的攻击,每一次都令丁铭有险死还生之感。两人身形越来越快,劲风激荡中,满地飞沙走石,两人的身形仿佛交缠在一起,可是一个如同天神临凡,任意挥洒手中电芒,一个如同九幽魔神,随手使出追魂夺命的杀招,彼此又是泾渭分明。

丁铭一边厮杀,一边心惊,此人武艺比起两年前简直不可同日而语,自己几乎难以辩明他招式的来去踪迹。他却不知这两年逾轮的心境因为柳如梦之故不再消沉寂寥,生机再燃,潜心修练之下大有进境。练武之人,若有名师指点,初时的成就主要是看根骨天赋,但是到了后期却要看品性智慧,逾轮本是聪明颖悟之人,又历经种种情仇磨难,两年前更因为陆灿之事,心灵遭遇强烈的冲击,令他有了突飞猛进的契机。

只不过逾轮虽然大有进境,毕竟不如丁铭根基深厚,两人苦战百招之后,丁铭渐渐稳住了局面,剑势变得越发灵动流畅,逾轮却是渐渐守多攻少,别人虽然看不出来,他自己却是知道自己很难取胜了。

柳如梦双目神采尽失,虽然眼前正在进行着一场关乎她命运的激斗,可是她却全没有看在眼里,只是想着宋逾自承有份害死陆灿的言语。她不是寻常女子,并非不知亡国恨的商女,自从大将军被诬下狱之后,她便深恨尚维钧误国之举,更是数次相劝宋逾,希望他能向尚承业进言,挽回此事,虽然知道希望不大,却也不愿袖手旁观。虽然知道宋逾和尚承业交好,可是在她心目中却从未想过宋逾会加害国之栋梁,就是宋逾在陆灿被赐死那日失魂落魄地返回住处,柳如梦也只道他伤心,全没有想到陆灿之死会和宋逾有什么关系。爱之深,责之切,故而柳如梦才会这般伤心欲绝。

这时,丁铭突然厉喝一声,剑光电闪,接连刺了五剑,每一剑都生生刺在逾轮折扇扇骨之上,声音清越如铮鸣,连绵不绝,逾轮竭尽全力闪避反击,但是却不能避开那凌厉堂皇的剑势,到了第五剑,逾轮手中的折扇脱手而飞,踉跄后退,丁铭手中长剑丝毫不曾放缓,刺向逾轮心口,逾轮自知今次真得无法逃生,冰寒幽深的双眼透出绝望灰心的神色,神色平静地看着那长剑没入自己的身体。

与此同时,唯一看清局势的灵雨惨叫道:“不!”声音凄切惊恐,丁铭心中一颤,想起了当日宋逾给自己等人陆灿的确切消息,让他们可以见到陆灿一面,虽然未能救回大将军,可是此情不能不酬。而且激战许久,丁铭心中悲愤稍减,也能比较理智的思索,在他看来宋逾还未有影响大局的能力身份,纵然他说了些不该说的言语,也不过是推波助澜,但是若非尚维钧存心如此,也不会最终自毁长城,更何况见宋逾言辞,颇有悔恨之意。心思电转,丁铭手中长剑一偏,避开了要害,虽然如此,顿时鲜血滚滚涌出,染红了逾轮半身。丁铭却也不好过,他原本被灵雨刺了一剑,虽然不甚重,可是激战许久,伤口迸裂,此刻也是血透衣衫,只是他全神贯注地迎战,直到此刻才有所发觉。

场中战势寻常人根本无法看清,只觉突然之间正在激战的两人身形凝住,然后便看到丁铭的长剑刺入宋逾的右胸,只是两人身上却都是一般的鲜血浸透,几乎看不出谁胜谁负。

逾轮目光淡凝,仿佛那利剑不是刺入自己身上,缓缓伸出左手,握住剑刃,鲜血瞬时从手掌和剑锋之间淌落,汇入地上的血河之中,他冷冷道:“丁大侠从南闽生还之后,却是改变了许多,不是已经被大雍的恩惠收买了吧,才对和议这般用心?”

丁铭眼中闪过狂怒,继而变得冰冷,道:“不错,丁某为了身上毒伤,亲赴南闽越氏求医,幸蒙大雍靖海公夫人越青烟援手,得以逃过死劫,可是丁某之心天日可表,姜夫人大度宽容,并未留难于我,也不曾收买丁某叛国求荣,此事不论你信不信,丁某都无愧于心。”

逾轮冷冷一笑,正欲再言,耳边响起一个动人悦耳的声音道:“逾郎可是一心求死么?所以才这般激怒丁大侠?”

逾轮浑身一震,缓缓松开左手,身子已经有些站立不稳,目光艰难地望向一旁,只见柳如梦不知何时已经站在血泊旁边,一双流波明泉也似的眸子正望着自己。

突然之间,丁铭闪电一般地拔出长剑,顺势点了逾轮几处穴道,止血上药,等到逾轮从急剧的痛苦中清醒过来的时候,只见自己已经倚在柳如梦怀中,柳如梦跪在地上,一身衣裙已经被鲜血浸透,却那般温柔坚强地抱着自己,四目相对,两人都是痴了,再也记不得身在何处。

不知过了多久,听到丁铭黯然的声音道:“宋公子、柳姑娘,两位有些什么言语,还是快些说吧,只怕现在我们这里的纷乱已经惊动了外面的雍军,若是他们询问起来,尚大人便不好交待。”

逾轮这才清醒过来,他知道方才的激战绝对会惊动外面的雍军,看到尚承业青白的脸色,知道他随时可能下令杀了自己灭口,自己的时间已经不多了,他艰难地伸手握住柳如梦的素手,道:“梦儿,对不住,我真的没有办法救你了,与其看着你被人凌辱,我宁愿先走一步。”

柳如梦略带苍白的玉颜上,两行清泪滑落,便如明珠玉碎,她柔声道:“逾郎,我想了很久,大将军的事情怪不得你,要怪只能怪定下千古奇冤的昏君奸相,你纵然有些过错,可是如今你已经后悔了,是不是?”

众人听得奇怪,都不明白为何这对一见便是情深意重的爱侣,为何会在诀别之时说起不相干的话。逾轮却是明白柳如梦的性子,答道:“是,我从前说了许多对大将军不利的话,虽然有些别的缘由,可是在我心里,总觉得他迟早会变成王莽,我不信世间会有那般赤胆忠心的臣子,可是大将军临终之前,我有幸在他身侧,才知道他的胸怀便如光风霁月,任何猜疑和污蔑都不能玷污他的为人,梦儿,若有重来一次的机会,我便是自己死了,也不会说半句不该说的话。”

柳如梦露出微笑,只是那微笑便如将要消逝的晚霞,纵然美丽,却是转眼就要湮没,她轻声道:“那就好了,我一直再想,若是逾郎不曾后悔,那么我就只好亲手杀了你,然后再和你一起上路,若是我所爱之人心中没有忠孝节义,那么我就是有眼无珠,自然该和你一起死的。”

听到柳如梦斩钉截铁的话语,已经是泪如雨下的灵雨惊叫道:“不,姐姐,你不能死。”

尚承业心中大惊,上前几步,却觉得想不出什么话语相劝。丁铭却是心中一紧,上前一步,已经决定若是柳如梦想要自尽,定要拦阻下来。

只有逾轮平静依旧,似乎全没有想过柳如梦是生是死有什么不同,只因他了解柳如梦,知道这个女子不论作出什么决定,都不会没有原因,若是她真得决定一死,那么对她来说,定是已经没有更好的选择,更何况他听出柳如梦的话中之意,至少柳如梦现在已经没有了自尽之意。

别人的反应柳如梦似乎都没有放在眼里,只是深深地望着逾轮苍白的面容,珠泪滚落在他面上,发上,昔日横波目,今成流泪泉。直到周围的楚军开始有了骚动,似乎是外面的雍军发觉里面有了异状,她才抬起头,看向满面狼狈的尚承业,淡淡道:“尚大人,妾身知道逾郎所为,必然惹怒了大人,他伤重如此,又在重围之中,大人若要杀他,正是情理之事,可是妾身却有不情之请,希望大人肯放过逾郎,待他伤愈之后放他离去,若是大人不许,妾身虽然微贱,却只有一死而已。”

众人都是脸色一变,若是柳如梦一死,已经递上去的贡单就成了南楚不恭的铁证,那么只怕求和之事立刻告吹,尚承业尤其心惊,虽然听了宋逾方才之言,他早已忘却昔日交情,恨不得立刻杀了此人,只是此刻却也只能按耐下来,道:“柳姑娘放心,宋逾是我旧交,我怎会害他,只要他不再妨碍和议,本官保证他可以平安返回江南。”

柳如梦只是淡淡一笑,却看向丁铭,道:“丁大侠为人,妾身一向敬重,纵然是今日之事,也有不得已处,若是丁大侠肯承诺保证逾郎的平安,妾身承诺绝不会自寻短见。”

丁铭闻言深深钦服,道:“柳姑娘言重,宋兄乃是性情中人,在下不得已重伤了他,已经是心存愧疚,绝不会容许别人伤害于他。”

柳如梦这才放下心来,露出心满意足的笑容,一双明眸焕发出耀眼的光彩,轻轻让逾轮平躺在地上,便要起身,逾轮目中俱是悲愤,挣扎着握住她的素手不放,顾不得伤口再度溢出鲜血来,厉声道:“梦儿,我的生死何需你顾惜,你肯忍辱偷生,难道我就不能一死相报卿的深情么?”

柳如梦双目透出无限深情,缓缓地,坚定地将手抽出,轻声道:“逾郎,莫非你以为一死便足以相酬知己么,妾身不过是个风尘女子,本就是路柳墙花,纵然沦落天涯,又有什么要紧,只要逾郎能够好好活在世上,妾身就会很开心了。更何况你又何必担忧,如梦虽然姿色平平,所幸还会些歌舞声艺,未必不能得到贵人宠幸,纵然没有这个福分,也有法子平安度过余生,或者将来会把逾郎忘了也不一定呢。”

说罢,她站起身子,一步一步走向原来的营帐,无双风华,纤弱高贵,这一刻再没人记得这女子原本是江南第一名妓,天上的仙子的风姿想来也不过如此。

浑不知身外的一切,柳如梦眼中便只有那熟悉的营帐,快到了,快到了,三步,两步,一步,当她终于走进营帐,随着帘幕的垂落,她的双腿一软,再也不能支撑下去,踉跄跌倒,却落入紧紧跟来的灵雨怀中,灵雨惊骇欲绝地望着她霜雪一般的苍白容颜,此刻的柳如梦,气息微弱,竟是立刻就要死去一般的模样,灵雨连忙点了她几处穴道,催动她的生机,柳如梦才悠悠醒转过来,灵雨泣道:“姐姐,你又何必如此,纵然你说出这般伤人的话语,莫非他就会相信么?”

柳如梦低低呻吟一声,醒转过来,面上露出凄凉的笑容,低声道:“我与逾郎,虽然两情相许,却是生前不曾同枕席,死也不能同墓而眠,但是如梦却觉得,纵然是百年偕老朝朝暮暮,也不如这片刻相知,我知道他不会相信,可是只要他心中存着我会好好活着的期望,他就不会赴死,妹妹,逾郎他从来都漠视生死,我早就很担忧他会舍我而去,如今我只盼他能够好好活着,便是我受尽屈辱又有什么要紧,或许,或许等到我鸡皮鹤发之后,还有机会活着见到他。”

灵雨抱紧柳如梦那纤弱冰寒的娇躯,似乎能够感觉到她生命的流逝,低声道:“姐姐,灵雨原本很害怕,我很怕雍人将我当成师父她们的同党,如果他们杀了我,我会很遗憾,因为我再也没有机会练成绝世的琴艺,如果他们不让我再有机会弹琴,我也会生不如死,若是他们真的,真的欺辱我,灵雨只怕再也不能活下去,可是现在灵雨发誓,我一定要活下去,不论遭遇到什么,我都要护着姐姐,一定要让姐姐有机会再见到他。”

这时候早已经陷入昏迷的柳如梦,却是听不到灵雨的誓言,只是她那苍白的面容上始终带着笑容,却是令人觉她早已心碎肠断。

————————————

注1:卢照邻《行路难》节选

注2:刘克庄《贺新郎·席上闻歌有感》改

\chapter{第四十八章 倾城一舞世所稀}

显颇爱声色,闻柳姬之名而喜,召入银安殿,略略数语,乃令起舞,乐师惧王威,曲调不成,王欲斩之,姬曰:“妾舞不需管弦。”乃作无声舞,将士皆醉。

——《南朝楚史·柳姬传》

望着柳如梦消失的背影,逾轮心中悲愤交加,气急攻心,却是又昏迷了过去,等他醒来的时候已经躺在一个空荡荡的营帐之中,耳边传来两个争辩的声音,却是尚维钧和丁铭。

只听见尚承业气恼地道:“丁兄,在外面要人的可是大雍的嘉郡王,齐王李显的亲生儿子,若是得罪了他,和议别想有任何希望。”

丁铭冷冷道:“在下承诺了柳姑娘,保护宋逾的性命,雍人声言要将在他们宿卫下惊扰南楚使团的贼子千刀万剐,如果他落入雍人之手,岂不是有死无生,大人只需对雍人说是内部纷争,想来他们也不能进来搜查。”

尚承业似乎犹豫了一下,良久才道:“好吧,就这样吧,对了,宋逾也是我的故交,虽然如今他不顾大局,颇为可恨,可是也是情字害人,这样吧,我那里还有些上好的补药,我一会儿令人送过来,丁兄看看若有可用的,就给他用上吧,他若早点好了,也好让他快些离去。”

丁铭似乎很满意,道:“大人顾及旧情,在下没有异议,只是在下对于医道只是略知一二,还要向大人请教。”

尚承业道:“我还要去向嘉郡王解释此事,副使向大人深通医理,丁兄可以向他请教就是。”

帐内的宋逾露出淡淡的冷笑,他和尚承业交往数年,自然知道他的品性为人,或者数年前他不过是个浑浑噩噩的世家子弟,如今却已经历练成了心狠手辣的显贵,这其中自己或许还有许多功劳呢。丁铭纵然才智过人,但是应付这些最擅虚情假意的世家子弟,仍然是太天真了。

果然等到丁铭的脚步声远去之后,不多时逾轮便听到一阵急促的脚步声传来,他勉强支起身子,定定看向帐门,那些人走到帐前,掀帘而入的果然真是尚承业。

尚承业一走进营帐,便看到一双冰寒淡漠的眼睛,不由心中一寒,虽然知道这人伤势极重,没有可能出手危及自己,可是还是不敢上前,有些尴尬地道:“宋兄弟,不是为兄不顾旧日情谊,只是大雍嘉郡王巡营到此,发觉营中事端,不知是哪个多嘴,告诉了嘉郡王闯营之人还活着,那嘉郡王年少高傲,很是气恼让你闯入了雍军宿卫的营地,所以定要本官将你交出,实在不是我想违背对柳姑娘的承诺。”

逾轮心中生出疑念,自己得到消息几乎是马不停蹄赶到合肥,一路上并没有和任何兄弟通过消息,应该不会有人知道自己陷在此处,那嘉郡王怎会定要索取自己,转念一想,或者自己是多想了,那嘉郡王虽然年少,但是这两年来也是名动江淮,都说是少年气盛,这般要求想来或许并没有什么特殊用意。心思一转,若是自己去到雍营,便可以求见先生,若是向他苦求,或者他会念在过去情分救下如梦。原本逾轮因为怀恨江哲,宁可赴死也不曾想过要向江哲求恳,可是眼见着柳如梦心碎模样,他从前的执念再也不能坚持下去。想通这一点,他并未作出什么反抗举动,只是淡淡看了尚承业一眼,便闭上了眼睛。

尚承业心中生出气恼,看向宋逾的目光又冷了几分,这人原本是自己的知交,自己有些什么疑难总愿和他商量,这人往往只是旁敲侧击轻描淡写说些言语,看似平常,却可以令自己想通许多问题,而对自己的决定他素来不甚关心,令自己全无被人控制的感觉,这是和面对父亲那些幕僚全然不同的感觉。可是原本想要倚为臂膀的心腹却在两年前突然消失,当时为了提防他说些不该说的话,父亲还曾派人暗中寻找过他,可是却全无所获,想不到这次他却突然出现在营中,还一副和自己割袍断义的模样。想到这人竟然会替陆灿说话,尚承业心一狠,冷冷道:“将他送到外面交给嘉郡王的亲卫,记得,不要将消息透漏出去。”

两个尚氏的护卫上前将逾轮挟起,因他伤势极重,倒也没有过分粗暴,饶是如此,逾轮已经是冷汗涔涔,只被挟持着走了十几步,便已经陷入半昏迷的状态。不知道过了多久,他再度醒了过来,只觉齿颊流芳,身上仿佛凭空添了许多力量,唯一移动,虽然仍然疼痛难忍,但是伤口处一片清凉,正是从前用过的秘营特制的伤药。心中一宽,逾轮知道自己安全了,抬目望去,只见自己躺在一间雅洁的卧房之内,勉强支起身子,正欲出声询问,房门无声无息地开了,一个相貌俊雅,服色却略嫌微黑的青袍男子端着药碗走了进来。

逾轮顿时愣住了,直到那人微笑着走到床边,将药碗递到他面前,他才狠狠扯住那人袍袖,放声大哭起来,就仿佛受尽了委屈的孩童,却突然见到了至亲一般。那人轻叹一声,伸手轻拍他的脊背,手中药碗却纹丝不动,一滴药汁也没有溢出。

不知哭了多久,逾轮才止住哭声,哽咽道:“二哥,你怎会来的?”却原来这人正是八骏排行第二的盗骊,如今海无涯已经不怎么管事,海骊已经是海氏实际的主事人,可以说日理万机,想不到却会来到合肥。八骏之中,盗骊无情果敢,杀伐决断更胜众人,逾轮从前和他最是亲近,也最尊敬这个师兄。当初他执意离开秘营的时候,盗骊正随船出海,不在中原,当时若是盗骊出面相劝,逾轮却也未必能够那般绝决,这几年他也是刻意避免和盗骊通消息,便是怕他劝自己重返秘营,想不到却在最落魄的时候,遇到了最尊敬的兄长,这才再也忍不住心中悲痛,痛哭一场。

盗骊长叹道:“逾轮,你的性子也太绝决了,这件事情本可以有别的解决方法的,何必要轻抛性命呢?白义已经通知了我们六个人,如今八骏之中只有你还飘零江湖,却让我们如何放心得下,这件事情我们已经商量过了,你还是得去向先生谢罪,这些年你太伤他的心了。”

逾轮沉默了下来,虽然在他进入雍营之前便已经有了准备,可是想到柳如梦十分神似当年的柳飘香,心中生出不安的感觉。见他沉默,盗骊淡淡道:“你不必担心,我们都会助你一臂之力,如今南楚使臣已经进了城了,你昏迷了很长时间,等到先生见过柳姑娘之后,你再去相求,先把药喝了,否则到时候你连向先生求恳的力量都没有了。”

逾轮接过药碗,默默喝下苦涩的汤药,心中也是一般的苦涩难言。他自然不知道,就在不远处的一间书房之内,霍琮惬意地品味着香茗,李麟则是一副看笑话的模样,大概是忍受不了霍琮的逍遥神情,终于忍不住嘲弄地道:“霍大哥,你真的确定有法子说服姑夫么?那个宋逾可是差点死在里面呢,若不是你让我去要人,只怕你的大计就没有成功的希望了。”说罢便拿起茶杯喝了起来。

霍琮淡淡瞥了他一眼,道:“这也没有法子,事前难以掌握他的行踪,只能守株待兔。郡王爷尽管幸灾乐祸就是了,被先生派去南闽护持陆氏一门的可是渠黄师兄,他和逾轮师兄也是手足情深,若是他巧妙安排一下,只怕还没有等到郡王爷去向陆小姐求婚,陆小姐就已经出阁了。”

“噗!咳咳!”李麟将口中茶水呛了出来,狠狠看了霍琮一眼,道:“行了,本王听命行事就是了,反正我也不愿意柔蓝嫁给你,你这人心机太深沉,就连姑夫也敢算计,还是我皇兄更适合柔蓝,不过你确定父王会那样做,莫非你还能威胁他不成?”

霍琮笑道:“我一个小小的六品文书,怎敢去威胁堂堂的齐王殿下,只不过齐王性情狂放,虽然这些年来韬光养晦,但是本性却是不改的,更何况王爷为了折辱南楚使臣,必然故意为难,那位柳姑娘外柔内刚,又遭遇这样的惨痛离别,想来定会出言相抗,纵然这种情况没有发生,我也敢肯定先生必然会将柳姑娘截下,纵然过程不同,结果却不会有什么变化,你还是想想自己要办的事情吧。”

李麟喃喃道:“你确定我不会被灵雨姑娘的情郎宰了?”

霍琮目中闪过笑意,道:“应该不会吧,如果你被宰了,我会想法子替你报仇的。”

李麟恨恨地顿足骂道:“若是事情不能成功,就是皇兄不怪罪你,本王也会好好报答你的。”说罢转身走了出去。

霍琮叹道:“若是真的失败,只怕也等不到你来教训我了,能不能过了先生那一关,都很难说啊!”

正如盗骊所言,如今南楚使团已经进了合肥城,齐王的帅府便设在合肥城中的南楚国主的行宫之内,座行宫本是武帝时候所建的,气势恢弘,富丽堂皇。尚承业战战兢兢地走上银安殿,也顾不得感叹本来是国主的行宫却成了大雍亲王的帅府,也分不出精力去留意两侧叉手而立,杀气凌人的雍军将领,走到殿中深深施礼,直到传来“平身”的命令,才敢抬头向上望去。

只见御阶王座之上坐着一个俊朗威严的中年男子,身着金色软甲,外罩赤色锦袍,这男子英姿俊拔,雍容威仪,虽然已经是四十五岁年纪,但是相貌气度依旧可以令天下男子汗颜。只是他面带笑容,神色平和,却令尚承业生出陌生的感觉。当年齐王出使南楚的时候,尚承业也曾见过他,只是当时的齐王便如出鞘的利剑一般危险耀眼,如今重见,却觉得这男子昔年啸傲苍穹的霸气已经变得深沉内敛,只有双目中偶然流转的睥睨天下的精光,才会令人察觉这人其实比从前更加可怕。也只有如此风采,才配得上统率大雍精兵,北灭汉土,南征楚国,立下无数显赫功业的齐王殿下

而在齐王左侧的椅上,坐着一个青袍绶带的儒雅男子,虽然是灰发霜鬓,却是神采奕奕,淡凝从容的气度,便在银安殿气势汹汹的众多武将猛士之中,也丝毫不显得逊色。虽然阔别多年,容颜已经有了许多改变,但是尚承业还是立刻猜出这人正是大雍江南行辕的第二号人物,今年已经重新被雍帝晋爵国侯的江哲,他更隐隐觉得,这人望向自己的目光淡漠非常,仿佛自己在他心中毫无分量。

而在齐王右侧椅上坐的却是一个虬髯大将,威势如山,双目射出暴烈的寒芒,正是攻下淮西,一路所向披靡,直抵合肥的荆迟。他目中满是鄙夷戾色,似乎随时都可能起身杀人一般。

不过令尚承业更为注意的却是在江哲身后立着的两人,一人青衣垂首,虽然是谦卑的奴仆模样,但是尚承业却不敢流露出轻视之意,甚至不敢多看那人一眼,邪影李顺之名天下皆闻,若无此人,只怕江哲也不可能活到今日,更不能成就他赫赫威名。另外一人却是一个十七八岁的美丽少女,容光潋滟,端丽秀雅,那少女正低头在江哲耳边说些什么,江哲微微点头,神色间满是纵容宠溺。看到这一情景,尚承业心中一动,按照他事前得到的情报,据说江哲之女昭华郡主江柔蓝这两年一直在军中,此女不仅深受大雍皇室的喜爱,更是未来的太子妃最可能的人选,若非大雍太子李骏正在江淮督战,只怕此女已经被立为太子妃了。眼前这少女不仅姿容端丽,更是仪态万千,又能以女子之身出现在银安殿上,想来必然是昭华郡主无疑。

强自抑制心中的胡思乱想,尚承业在阶下再拜道:“下官奉我南楚国主之命,拜上大雍江南行辕元帅齐王殿下,我主诚意求和,愿割土纳贡,永为大雍藩属……”

刚说到此处,李显已经不耐烦地道:“本王承帝命讨伐不臣,贵使想要求和也应去长安面见陛下,这些话对本王说也没有什么用处,若是不见你,愧对你远道盛情,既然已经见了面,你先下去休息吧,和议之事以后再说。”

尚承业原也没有指望用言辞说服齐王,但是李显却连说话的机会也不给他,不由暗自忧愁,只得道:“王爷乃是大雍帝胄,南征主帅,若王爷肯体念江南百姓深受兵燹之苦,进言贵国陛下,息干戈,止杀伐,共成和议,令两国百姓免受刀兵之苦,则皇天厚土,社稷黎民,皆感王爷恩德。”说到此处,见李显神色颇不耐烦,全无动心之意,心知此人不喜虚言,想起这人从前好色的声名,一狠心,也顾不得颜面,继续道:“为了表示我主诚意,外臣此来,携有诸般贡品,礼单昨日已呈上王爷,请王爷体念我主至诚,笑纳礼物,允许和谈。”

李显闻言笑道:“早这么说不就完了,却这么多废话。”此言一出,荆迟不由大笑起来,笑得是前仰后合,有他带头,阶下众将也不由哄笑起来,尚承业脸色却变得如同猪肝一般。这时原本含笑看戏的江哲按耐不住了,纵然是故意折辱使臣,这样也有失体统,发出一声警告地轻咳,他虽然是文官,但是在军中颇有威仪,只是冷冷环视众人一眼,笑声立刻停住,荆迟更是几不可察地缩了缩脖子,不敢再作声。江哲又瞪了李显一眼,淡淡道:“贵使见谅,这和议之事,事关重大,齐王殿下虽然是主帅,但是也不能擅自作主,等到禀明陛下之后,不论事成与否,总会给阁下一个回复。”

虽然出言替尚承业解了围,但实际上我可是很讨厌这个尚承业,虽然是我设计通过他说服尚维钧加害陆灿,可是这并不代表我会欣赏他,虽然很想直接将他拖下去千刀万剐的,可是既然已经准备今冬休战,用和议来敷衍一段时间倒也不错,免得杨秀、容渊这些人不安分,再说将来他父子自有恶贯满盈之日,却也不用我担心,嘉郡王李麟可是早已磨刀霍霍,准备等到攻下南楚之后,将尚氏一门斩尽杀绝,想要讨好那位至今仍然不知道李麟钟情于她的陆梅陆小姐。

说起来倒也有趣,我将关于陆梅出走建业之后的经历记录下来给李麟看,这一向冷酷无情的小子居然读得抹了半天眼泪。其实这也难怪,若非是听董缺所说,我也不敢相信一个弱质纤纤的小女孩,会有那样的勇气和毅力,带着石玉锦逃到荒村,更别说石玉锦因为动了胎气难产,长达七八个月的时间,都是这个小女孩忙里忙外照顾嫂子和侄儿,虽然得了董缺许多帮助,可是这已经是十分难得的了。我把这些告诉李麟知道,便是希望这小子不是仅仅被陆梅的艳光迷住,也不是因为放弃柔蓝而另寻寄托,我希望他真正爱上陆梅,这才能对得起泉下的陆灿。陆梅外柔内刚,温柔贤惠,若真的嫁给李麟,是这小子不知几世修来的福分呢。

轻叹了一口气,别人的事情都好办,怎么我自己的女儿却这么麻烦,我身子好了以后,本想干脆回京的,可是皇上偏偏让我做了江南行辕的监军,所谓监军,却是连自由行动的机会都没有了,这也罢了,反正军务我也不需担心,留在行辕之内尸位素餐也就罢了。唯一令我头疼的便是柔蓝的婚事,虽然皇上没有明示,可是这两年来劝我的人不少,虽然都未明说,却是意思很明显,希望我同意这桩天作之合的姻缘。

但是我当真不愿蓝儿嫁给李骏,为了提防太后、皇后趁着我不在的机会立了蓝儿为太子妃,我索性将她留在身边,没让她回京,更不让她和李骏见面,希望时间能够冲淡她对李骏的情意。可是这丫头也真倔强,平日里在我面前温顺乖巧,贴心服侍,为我分忧解劳,这几个月甚至可以替我处理一些寻常文书了,绝对看不出什么异样,可是只要我提到她和琮儿的婚事,她便沉默不语,绝不答应。两年没让她和李骏见面,书信未通,可是只要有人无意中提起李骏,便会立刻见到她竖起耳朵,若是听到李骏那里什么好消息,一整天就都是神采奕奕的,偏偏我还没有法子,难道我还能阻止她探听淮东那边的军报么?这般深情,让我见了越发疼惜,唉,若是这丫头效仿寻常女子一哭二闹三上吊,我早就迫她和琮儿成婚了,父母之命,难道她敢违抗么?可是她偏偏一直逆来顺受,除了不肯松口许婚之外,就是最乖顺懂事的女儿也不过如此,叫我怎么狠得下心迫她?

我正在胡思乱想,恍惚间听见李显的声音道:“这柳如梦据闻是江南第一花魁,本王倒也想见识一下她的色艺,传本王令谕,让她上殿献艺。”

我皱皱眉,贡品的礼单似乎柔蓝忘记拿给我看了,回头低声问道:“蓝儿,这柳如梦是谁,也是南楚送来的礼物么?”

柔蓝目光闪动,低声答道:“爹爹,南楚国主送过来许多金银珠宝,还有歌舞女乐,这柳如梦据说是江南第一名妓,歌舞色艺天下闻名,想来是南楚国主量珠而聘,送来这里的吧。”

我冷笑道:“南楚已经沦落如此,岂有不亡的道理。”口中说着,我却皱紧了眉,这柳如梦三字我应该见过,只是没有留下深刻的印象。心思电转,突然想起陈稹呈上来的关于逾轮的情报,里面似乎提到他为一个风尘名妓做琴师,那名妓姓名就是柳如梦。这件事情我并未留心,若非是我过目不忘的本领,却也想不起来,不过这两年逾轮已经离开了建业,想来和这女子已经没有什么纠葛了吧?不知怎地,心中生出不妥的感觉,我淡淡道:“南楚的贡品礼单怎么昨夜没有送来给我?”

柔蓝心中一惊,答道:“爹爹这两年来都不喜欢过问这些琐事,所以蓝儿也没有留意,只是将礼单归档了,既然爹爹要看,蓝儿这就让人取来。”

我摇头道:“算了,等到回去再说吧,以后不可疏忽大意,总要先禀明我之后再处置。”

柔蓝轻吐香舌,道:“是,爹爹。”

这时,我突然觉察身边的空气仿佛凝固了一般,就连正在和我说话的柔蓝,还有素来人前喜怒不形于色的小顺子,两人的眼睛都径自望着殿门方向。

我觉得奇怪,正过身子向殿门望去,只觉脑中轰然,瞬间忘记了一切,目光再也不能移动半分。

银安殿门口,一个头上罩着银纱的女子凝眸伫立,虽然只是静静站着,但是那绝代的风华已经展现无疑,隐隐间,似乎传来微不可闻的一声轻叹,那女子向前走来,步履宛似仙子凌波,行动间环佩叮咚,仿佛仙乐相随,走到阶前,水袖低垂,交臂胸前,曲跪于地,精致的孔雀翎长裙在她四周散开,众人望去只见她青丝如墨,皓腕如雪,心中生出渴望一见花容的执念。

李显青年时本是声色犬马之人,见识过的歌舞女乐不计其数,苦苦思索,却觉得鲜有人能比此女风华,眼中闪过异色,忆起昔日放纵,不由兴起,大笑道:“免礼平身,抬起头来,让本王看看你的容貌。”

那女子闻言起身,然后抬起螓首,银色头纱轻轻滑落,露出秀雅如玉的面容和一双令人心醉的秋水明眸。李显只觉这女子眉宇间带着不屈之意,虽是顾盼生姿,却更有绝世独立的意味,心中生出玩味,目中寒光暴射,银安殿中顿时被他刻意放出的霸气杀机笼罩起来,这样的气势,如今只有在李显挥斥方遒,杀伐决断之时才会展现出来,就是殿中的将领侍卫也都有些战栗不安,那女子初时柳眉微蹙,似有示弱之意,但是当她无意中瞥见李显趣味盎然的眼神之后,心中涌起怒火,娇躯中仿佛生出无穷力量,静静立在殿中,纵然是狂风骇浪,却也吹不折柔弱翠柳。

李显越发兴起,拊掌道:“好个柳如梦,果然名不虚传,来人,传乐师上来,本王要看看你冠绝天下的舞姿。”

柳如梦闻言裣衽为礼,淡淡道:“妾身遵命。”

这时,那些南楚精挑细选的女乐走上殿来,这些乐师都是些秀丽女子,虽然不如柳如梦风华姿容,却也是十分美丽,只是这些女子走入殿来,却是个个战战兢兢,原来李显并未收敛威势,这些女子都不敢正视于他,就连手中的乐器都似乎生疏了许多,乐声断续不成曲调。在一旁的尚承业急得直冒冷汗,忍不住低声申斥,一个弹筝的女子越发慌乱惊恐,手一抖,已经弄断了一根筝弦,顿时吓得跪伏在地,不敢抬头。

李显见状面上露出怒意,指着那弹筝女子道:“贱婢无礼,坏了本王观舞兴致。”

殿中将士见李显震怒,只是心中虽有怜香惜玉之意,却不敢多言。有些胆大的已经目视江哲和荆迟,这殿中也只有他们两人有资格出言劝解,不料荆迟懒洋洋地坐在那里,不知道神飞何处,而江哲却是目光凝注在柳如梦身上,神色有些如痴如醉,更是没有说情的闲心。

眼看这女子就要遭受重责,柳如梦本是侠骨柔肠之人,见状高声道:“王爷威仪如山,令妾等见而惊惧,亦是无奈之事,何必怪罪无辜弱女,王爷若是想看妾身舞艺,妾身能作无声之舞,便无管弦也无妨碍。”

李显闻言大笑道:“好个柳如梦,这般放肆无礼,本王理应加罪,但你既然敢出此狂言,本王也想看看你的无声之舞,若是跳得不好,可要两罪并罚,你可要想清楚了。”

柳如梦微微一笑,轻移莲步,走到大殿中央,长袖挥洒,便开始翩翩起舞,虽然没有曲乐,可是她飞旋的舞姿仿佛蕴藏着天然的韵律,环佩叮咚,连绵而悦耳的金玉之声听在众人耳中渐渐变成了舞曲的旋律。凌波飞渡似的娇姿,繁杂多变的独特舞步,狂放而纵情的一舞扣人心弦。

柳如梦纵情飞舞着,这一刻她的心中仿佛响起了数年来伴着她起舞的动人箫声,何需管弦舞乐,那韵律就在她心中,再也没有可能和他相见,再也不能跟随自己的心意起舞,从今后自己便是笼中丝雀,再也没有自由幸福可言。心中悲愤化入舞姿,殿中众人纵是不识风情的莽夫,也能够感受到柳如梦无声之舞中的洋溢的哀痛凄怆。待到柳如梦一舞终了,殿中已经满是唏嘘之声,柳如梦低首裣衽,广袖下垂,盈盈拜倒,不愿令人发觉她目中盈盈水气。

李显长叹一声,就是以他的坚毅心志,也险些泪落,原本早已决定将这次南楚送来的女乐赏赐军中将领,此刻也不由心动,不由道:“卿的舞艺果然天下无双,不愧江南第一之名,本王府中尚缺一位教授歌舞的教习,不知道卿可愿从命?”

柳如梦眼中闪过冷漠之色,淡淡道:“妾身本是身充下陈而来,生死不能自主,王爷何需动问。”

李显原本心中并无恶意,自从和嘉平公主林碧成亲之后,他已经失去拈花惹草的兴趣,此刻不过是怜惜柳如梦才艺,有心庇佑于她,更已准备让林碧做主,为这女子寻个归宿,但是柳如梦的回答却是这般冰冷,反而令李显越发好奇,道:“听卿的话音,若是自由之身,莫非还不愿随本王回府么?卿不必矫饰,直言无妨,本王这点度量还是有的。”

柳如梦本是心中怀恨,此刻闻言也不论是真是假,一字一句道:“妾身本是楚人,岂能屈身相事仇雠。”

一言激起千层浪,本来殿上众人多半爱慕她的才艺品貌,想不到她说出这般悖逆之言,对于一个被当作礼物的女子来说,这般勇气世间少有,不论是气恼还是钦佩,众人的目光都集中在柳如梦身上,只是不知李显如何处置。

李显却并未恼怒,他初时故意放纵,本是有意戏弄尚承业,对于这些被当作贡品送来的歌舞女乐,他也没有什么特别的想法,对柳如梦诸般相试,不过是一时兴起,见柳如梦这般言语,反觉正合她的气质品貌,本想一笑赦之,目光一转,无意中见到江哲双目迷离,似乎神魂颠倒的模样,不由一愣。

他可是知道的,江哲素来对女色并无多少兴趣,如今这般失态当真古怪,莫非他竟然对这女子动情了么,此刻李显可全没想到这人乃是自己的妹夫,反而生出捉弄之意,故意变色道:“岂有此理,本王对你这贱婢以礼相待,你竟敢出此大逆不道之言,来人,将此女押下去重责百鞭,而后将其送入军中为苦役。”

此言一出,不仅那些女乐个个胆寒,吓得魂不附体,就是那些大雍将领也是心中不忍,只有尚承业心恨柳如梦胡言乱语,唯恐破坏和议反觉心中快意,毫无出面求情之意,看在众人眼中,越发觉得齿冷。

两个侍卫走上殿来,上前欲要将柳如梦拖下去行刑,柳如梦也不哀告求饶,只是淡淡瞧了李显一眼,美目中满是鄙夷,也不待那两个侍卫拖曳,便自行向下走去,仿佛即将面对的不是无边苦痛一般。

柔蓝见状大惊,心道虽然那柳如梦果然和齐王舅舅冲突起来,可是爹爹怎么没有出言相救,看来只有自己出面救下这位可敬的柳姑娘了,正待她想要鼓起勇气求情,却见江哲目中突然清明起来,朗声道:“且慢,王爷,此女虽然冒犯殿下,但请殿下怜她才艺,不要重责于她,也免得他人嘲笑我大雍没有容人之量。”

李显大喜,心想莫非自己竟然寻到了这人难得的软肋,试探地道:“莫非随云怜惜此女色艺,呵呵,这也是此女之福,既然如此,本王就将她送给你为侍妾如何?”

我闻言一愣,连忙道:“这怎么使得。”

李显故意作色道:“随云既然无心,那本王也不多事,快将柳如梦带下去行刑。”

我心中一痛,纵然察觉了李显眼中暗藏的笑意玄机,也不由道:“王爷手下留情,既然已经将此女送与本侯,若要责罚,也该是哲亲自施为。”

李显闻言心中狂笑,却不敢流露出来,只听江哲自称本侯,就知道他已经是十分恼怒,但是他的目的已经达到,大笑道:“好,将柳如梦送到监军住处,好生照顾,不得有失。”

我只觉得面上羞红,浑身上下都不自在,众人的目光好似可以灼穿我的身躯一般,说起来我虽是驸马身份,可是纵然如此,有几个侍妾也是情理中事,只是我不爱女色,纵然皇上赏赐美女,也都淡然拒绝,今日却不得已接受了柳如梦,当真是一世英名付诸东流。气恼之下不由拂袖而起,也不顾什么礼仪,气冲冲地走出银安殿,也不回住处,更不寻车马,便安步当车走出行宫,到了街上,见到街上来来往往的行人,这才舒了一口气。这时候,柔蓝在我身后低声问道:“爹爹,你不是真的想把那位柳姑娘收入房中吧?纵然娘亲不管,女儿也觉得不妥呢。”

我闻言差点被自己的口水呛住,这丫头哪壶不开提哪壶,但是目光落到她面上,却见她目中满是不安烦恼,心中一软,心道,柔蓝自幼便和长乐亲近,母女情深,不啻亲生,她为此忧心也是情理之中。目光一转,又发觉街上行人都在偷偷望来,柔蓝衣饰华贵,容色美丽,未免过于显眼,便叹道:“傻丫头,好了,我和你顺叔到外面散散心,你先回去吧,琮儿这两天应该回来了,这次我可是特意用了军令相召,想来也无人能拦阻,你替他安排一下住处,还有,好好安排一下那位柳姑娘,不要为难她。”虽然有些难堪,可是担心柔蓝为了替她娘亲出气,而欺辱了柳如梦,还是多说了一句话。说完便转头就走,也不敢去看柔蓝的神色,所以我自然不知道柔蓝眼中满是崇敬之色,正在暗暗祝祷道:“霍哥哥果然神机妙算,老天保佑他的计策能够成功,让爹爹越糊涂越好,可别识破了机关。”

此刻我脑中果然是一团混乱,不知道走了多久,突然小顺子拉住我道:“公子,你身子不好,不要过分劳累,不如寻个清净所作休息一会儿。”

我停住脚步,这才察觉已经是额头见汗,今日阳光略有暖意,只是寒风吹拂,若是我再这般胡乱走动,只怕会受了风寒,苦涩的一笑,看到前面有座酒楼,便径自走去,也不理会上前招呼的伙计,走到二楼,看一间厢房帘拢高卷,知道无人,便走了进去,小顺子吩咐了几句,便放下了帘子,我心知暗中保护我的虎贲卫很快就会将楼上客人请出,说话也不需小心,跌坐在椅上,感受着厢房内的暖意,我再度陷入沉思。

已经十八年了,飘香玉碎珠沉已经整整十八年了,手抚指上玉环,忆起佳人的音容笑貌,心中痛楚非常,自从为她复仇之后,我便将昔日深情黄土深埋,纵然见到玉环想起她的时候,也强迫自己只去想些欢乐的事情,再和长乐成婚之后,一来是她的如海深情化解了我心中苦痛,二来也是不愿令长乐猜疑,所以更是将关于飘香的一切深藏于心,时间久了,我几乎也以为自己早已忘记了飘香。直到今日,我才知道,原来我心中的伤痛从来都未痊愈。若非为了这个缘故,我又怎会放纵逾轮,任凭他脱离秘营,只因逾轮的伤痛与我正是同病相怜,只要想到世上还有一人和我一样心中有着飘香的影子,我便不会觉得孤独,所以只要逾轮不会坏了我的大事,我便不愿取他性命。

忆起柳如梦神似飘香的气质风采,不由魂断神伤,她一个弱女子,对着杀人盈野的齐王,在那纵是当世豪杰也不由屈膝的威势下竟敢奋起反抗,这般傲骨,令我想起昔日飘香怒斥韩王的事迹,想必当时的飘香也是这样的凛然无惧吧?

慢慢回忆着关于飘香的点点滴滴,就连惊闻飘香惨死的不堪回忆也再度涌上心头,任凭伤痛肆虐心头,不知想了多久,突然吐出一口黑血,心中却是一清,只觉萦绕心头多年的积郁尽皆化去,挥手推开满面惶急过来探视的小顺子,我抬头笑道:“不要紧,这是心伤发作了,吐血之后就没有妨碍了,到底是怎么回事,是谁在搞鬼?”

小顺子放下心事,只觉江哲神情轻松之极,眉宇间更是多了一种洒脱的神采,恍惚之间,竟觉得仿佛回到了建业初见之时,那时候地江哲便是这般神情,只觉心中感慨万千,鼻子一酸,差点落下泪来,连忙别过脸去,过了许久,才回过头道:“离开行宫之时,我已经传下谕令,查问此事。别人不知,陈稹和八骏多半都见过柳夫人,柳如梦神似夫人,此事他们不曾上报,想来也是怕引起公子伤心,此事倒也情有可原,但是如今柳如梦被送到雍营,他们却仍不禀明此事,令公子促不及防见到此女,此事绝不能容,请公子下令惩戒,以儆效尤。”

我摇头道:“罢了,初时不说,也是他们的心意,再说我记得逾轮和此女有些瓜葛,如今想来也应是此女神似柳飘香的缘故,他们瞒过此事也是用心良苦,至于今日之事,虽然应该责罚,可是毕竟解了我多年心结,却也不要过分怪罪他们,只是查清楚也就罢了,不过我总觉得有些不对,却是想不出来,罢了,我们先回去吧。”

小顺子犹豫了一下,道:“公子,那位柳姑娘如何安排?”

我闻言一怔,目光落到小顺子面上,见他神色似有隐忧,微微一笑,招手示意他过来,等他神色茫然地走到我面前,我伸指轻弹,小顺子立刻捂住了额头,露出无辜之色,虽然明知我这一个暴栗对他来说还不如蚊子咬他一口,而且若非他甘心情愿,我更是没有可能得手,但是仍然忍不住有些得意地笑骂道:“混蛋,你当我是什么人,我岂会这般放不下,更不会做出李代桃僵之事,若是做出那种事情,不仅对不起长乐深情,更是对不起飘香。这女子也是可敬可怜,过几日问问她的心意再做决定吧,飘香已经不幸,我不愿她也红颜薄命。”

说罢我起身走出厢房,果然见到虎贲卫已经在外面宿卫,径自走出酒楼,上了不知何时准备好的马车,径自回府,全然没有留意到小顺子一路上眼神忽忽而迷惑,忽而闪烁,最后变得清明如寒冰。

小顺子侧过脸去,唇边露出一丝微笑,面上更是露出了然的神色,虽然觉得自己应该提醒江哲一下,但是心思数转,瞥了一眼仍在皱眉思索的江哲,终于还是什么都没有说。

\chapter{第四十九章 天长地久}

王爱其色艺,欲以金屋纳之,姬拒之曰:“妾为楚人,不事仇雠。”王闻之而怒,欲加罪,楚国侯哲正从军行,婉言劝王,王遂改颜,将姬赠于哲,哲乃赐金赦之。柳姬离雍营,乃效鸿飞冥冥,或言从良人去矣!

嗟乎,当社稷危亡之时,余每见儒冠降敌,壮士卸甲,不及柳姬多矣,乃为之志,以彰其行。

——《南朝楚史柳姬传》

缓步走下车马,进了府门,我便径自走入书房,柔蓝正在书案后替我整理公文,写出节略供我快速浏览,见我进来便乖巧地起身相搀,等到我坐下之后,又亲自端上香茗,我端起香气四溢的茶水,不由满意地看了她一眼,有女如此孝顺,当真是老怀堪慰。

随手将柔蓝已经放在面前的南楚礼单拿起,打开看了起来,只看到第二行我就已经“噗”的一声将口中茶水全部喷了出去,不由指着礼单大声道:“这是怎么回事,灵雨姑娘也是贡品之一,秘营怎么一点消息都没有,快派人去查,人送到哪里去了,若是有什么三长两短,难道我要拿命去赔么?”

柔蓝露出茫然之色,道:“爹爹,灵雨姑娘是谁啊?”

我这才想起柔蓝并不知道江南的事情,不由急得起身在书房之内转来转去,我离开银安殿已经三个多时辰了,按照事先的安排,这个灵雨一定是赏赐给了哪个将领,这个女子我可是答应过替秋玉飞照料的,若是出了意外,我还有什么颜面去见朋友。这时候小顺子已经出去传令了,等他走进书房,我已经冷静下来,淡淡道:“让呼延寿去荆迟那里看看,既然柳如梦在我这里,齐王殿下多半会将灵雨给了荆迟,如果人果然在那里,就让呼延寿直接要过来,想来荆迟不会驳我的面子的。柔蓝,你查一下明鉴司有没有相关的情报呈上,灵雨既是凤仪门弟子,又是玉飞意中人,这样的身份,明鉴司那里定有记载,若是她名列贡单之上,此时必定已经传遍江南,明鉴司理应呈上节略才对,可是我记得这几日并未看到类似的文书,若果真没有,明鉴司便是失职了。”

柔蓝口中答应,走到书案上开始翻阅明鉴司呈上的文书,低头翻阅了一会儿,忍不住向父亲偷眼望去,却见小顺子神色古怪地望着自己,心中一颤,连忙避开他的目光继续寻找起来。她自然知道是找不到的,只因她早已将明鉴司送来的呈文藏起来了,江哲本就不甚留心这方面的细务,所以被她瞒过。

房中一时变得十分寂静,除了柔蓝翻动书页的哗哗之声,再也没有别的声响。我坐回椅上,凝神想着心事,秘营出了什么变故,这样的事情怎会没有消息,联想到柳如梦之事,虽然肯定秘营决不会背叛于我,但也是疑虑重重。正在我陷入沉思的时候,却有侍卫前来禀报,说是秋玉飞在外求见。

我眉头紧锁,怎么秋玉飞这个时候来了,他不是已经闭关了么,这两年消息难通,按理说他跟本就不应该知道灵雨之事,难不成魔宗提前让他出关了么,让小顺子代我前去迎客,我心中也是忐忑不安,希望灵雨姑娘无事,不过若是李显真的将她赐给了荆迟,倒是无妨,荆迟虽然粗莽,却不是好色之辈,若是灵雨姑娘不愿,他必然不会强迫。

正在我暗暗安慰自己的时候,小顺子已经引着秋玉飞走入书房,我起身迎接,目光落到秋玉飞面上,便是一震,只见他星目无光,容颜苍白,竟似是受了重伤的模样,微微皱眉,径自上前伸指搭在他脉门,良久,我叹了口气,抬起头道:“玉飞,你怎会伤得如此之重,而且似是没有好好调养,若是再晚来几日,只怕要多养上几年了。”说罢,我转头道:“小顺子去拿药箱和金针来,蓝儿回避一下,告诉呼延统领,不论人在何处,都要接过来,便说江某多谢了,翌日必定亲自登门谢罪。”

小顺子心知江哲这般含糊其辞,是不想秋玉飞心中焦虑加重伤势,柔蓝乖巧得很,自然也不会多言,两人走出门去,等到房门掩上,遮去江哲目光,小顺子目光一寒,灼灼望向柔蓝,却不言语,柔蓝心中一颤,悄无声息地跪在地上,面露哀求之色,小顺子犹豫片刻,终于轻轻摇头,径自走去,柔蓝心知小顺子已经答应不过问此事,面上露出明艳的笑容,站起身来,匆匆走回自己的房间,还要将秋玉飞已经到达的消息传出去,好让霍哥哥决定下一步应该怎么做。

等到小顺子取了药箱和金针回来,我让秋玉飞到书房内间的软榻上盘膝而坐,让他宽了衣裳,先用金针通畅了他的气血,又让他服下我秘制的治疗内伤的药物,剩下的就要靠他自己慢慢休养调息了,先天高手一旦受伤,想要痊愈也是极难的。

医治完毕,秋玉飞穿好衣服,起身拜谢道:“多谢随云援手相救。”

我愕然道:“玉飞何时变得这般生分,你我相交多年,在下又略通医术,岂有不出手的道理。”

秋玉飞黯然道:“我是谢随云你相救灵雨,我入城之时已经听见传言,南楚使臣送上的女乐皆被齐王殿下赏赐给将士,其中最出众的两人,分别是赐给随云和平北将军荆迟的,随云一向不爱女色,若非是为了救下灵雨,怎会接受这样的赏赐呢?”

我不觉汗颜,心道,还是等到接回灵雨之后再道歉吧,为了岔开话题,我笑着问道:“玉飞,这天下若论武功之高,你已经在十指之数,是什么人能将你伤成这个样子?”其实我很怀疑是魔宗伤了他,所以很想问个清楚。

秋玉飞似乎明白我的疑虑,摇头淡淡道:“不是师尊。”我松了一口气,正要再问,不料秋玉飞又黯然道:“是大师兄重伤我的。”

我差点一口气喘不上来,好不容易平静下来,不问可知,秋玉飞定是知道了灵雨之事,私自逃出来的,魔宗一向言出如山,必定大怒,派段凌霄擒回玉飞自是情理之事,其中细节却也不必再问,只是不知是何人传递消息给他的,便问道:“玉飞闭关两年,不问外事,就连在下的书信也是石沉大海,不知却是何人将消息送到了玉飞手上?”

秋玉飞目中闪过疑惑,问道:“莫非不是你遣赤骥给我传信的么?我闻信私自出关,中途却被大师兄截住,为了脱身,只能硬受了大师兄一掌,幸好大师兄手下留情,要不然只怕我已经死在路上了。”

我闻言不由问道:“莫非段大公子也到合肥了么?”

秋玉飞有些尴尬地道:“恐怕要给你惹麻烦了,大师兄奉了师尊谕令,是绝不会放手的,恐怕很快他就会到合肥了。”

我心中疑云重重,秘营众人在搞什么鬼,灵雨的事情不告诉我,却费了那么大力气告诉玉飞,还让原本已经退出秘营的赤骥也牵扯了进去,正欲仔细想想其中蹊跷之处,呼延寿匆匆走了进来,禀道:“侯爷,灵雨姑娘果然是在荆将军那里,不过末将去后却得知嘉郡王将人要走了,末将去见嘉郡王,郡王不肯放人。”

我只觉脑子里面轰得一声,也顾不得去看秋玉飞瞬间变得冷森酷厉的面容,怒道:“李麟怎么回事,他小小年纪,莫非也对女色有了兴趣么?”

呼延寿低头道:“侯爷,末将也婉言问过,听嘉郡王的亲卫说过,郡王得知那女子是凤仪门余孽,心中怀恨,郡王说若非是凤仪门谋逆犯上,也不会害了他的生母,所以要杀人泄愤。”

我还没反应过来,秋玉飞已经冷冷道:“随云,这是怎么回事?”

我只得赧然道:“玉飞尚请恕我失察之罪,我方才才知道灵雨姑娘竟然也在贡单之上,所以令呼延统领去要人。”

秋玉飞闻言身子轻颤,淡淡地望了我一眼,眼中满是怀疑,我也知道这话他不易相信,若是赤骥可以传信给他,我又怎会不知道,正欲向他解释,秋玉飞已经拂袖而出,神色冷厉,似乎颇为平静,推门而出,可是当他身形消失在门外之后,那厚重的木门竟然就在我的眼前迸开,我愣愣地望着那一顿巴掌大小的木头碎片,不由心中一寒,脑子里面更是一团混乱,一向以来,我已经习惯了身边事情尽在掌握的感觉,今日的种种变化都脱出了控制,真让我有着无所适从的感觉。

不知呆了多久,我站起身来,高声道:“小顺子,立刻跟我去李麟那里,希望还有挽回的余地,李麟怎会如此胡闹呢?”一边暗悔自己可能忽视了李麟心中的阴影,一边企盼着灵雨安然无恙,如果秋玉飞和李麟冲突起来,那可是天大的祸事,转念一想,就算是灵雨没有事情,段凌霄若是追了来,又该如何处置,心中千头万绪,只觉得头大如斗。

小顺子也不作声,只是下令备了车马,护着江哲扬尘而去,更是带上了府中六七成的侍卫,毕竟接下来的事情可能极为棘手。

秋玉飞离开江哲府上,心中一阵茫然,方才一时激愤,令他拂袖而去,到了外面冷风一吹,他便冷静下来,仔细想来,怎么也不觉得江哲会做什么手脚,虽然他也知道江哲对于他和灵雨之事不以为然,可是若是他有心加害灵雨,却也不必等到今日,其中不知出了什么变故,自己还是先去嘉郡王那里救灵雨要紧。可是四下环顾,却是不识路径,怎知道嘉郡王李麟的府邸在何处,想要回去问江哲,又觉得颜面无存,再说无论如何,江哲和嘉郡王乃是亲眷,总胜过自己这个外人吧,狠狠顿足,决定寻个军士问路,反正现在合肥城内到处倒是雍军军士。

刚要举步,身后一个身着虎贲卫服色的军士疾步赶来,口中喊道:“四公子稍待,属下奉侯爷之命前来替四公子领路。”

秋玉飞一愣,目光落到那人面上,记起方才就在江哲府上见过那人,心中一暖,口中却冷冷道:“江哲怎么说?”

那侍卫施礼道:“侯爷吩咐,让在下领四公子去见嘉郡王,侯爷说嘉郡王虽然年少,却是气度过人,应不会真的伤害灵雨姑娘,还请四公子不要过于心焦,谨慎行事。等到四公子救了人之后,侯爷自会向四公子解释其中误会。”

秋玉飞闻言心中略宽,道:“你前面带路吧。”那侍卫似是十分精明能干,引着秋玉飞穿街过巷,过了不到两拄香时间,已经到了一处禁卫森严的府邸,秋玉飞正要问那侍卫是否此地,便听见风中传来熟悉的清丽琴声,正是他指点过灵雨的那曲《猗兰操》,此曲之意本是自伤际遇,孤芳自赏,可是如今秋玉飞听来,却觉得那如泣如诉的琴音中隐隐有着思慕之意,他本是音律大家,心念一转,已经知道自己与灵雨之间,非是自己一厢情愿,若非灵雨对自己也有倾慕之情,便不会在弹奏此曲之时这般情意绵绵了,琴为心声,所以令这原本凄怆的曲调中也多了些柔情蜜意。秋玉飞听得痴了,竟是忘记了一切,呆呆立在寒风之中,只恨自己走得匆忙,竟连琴也没有带上,否则定要立刻弹奏一曲,告诉灵雨自己两年来是如何的苦苦相思。

琴音渐渐消沉下去,秋玉飞身影一闪,已经跃上高高的围墙,他的身影如虚如幻,掠过重重楼阁,府邸之内守卫并不森严,几乎毫无窒碍。就在这时,琴音再起,这一次的琴曲却是《离鸾操》,哀而不愠微而婉,琴音如同流水,却将抚琴之人的哀愁凄苦尽情倾诉,秋玉飞只觉得自己仿佛在顷刻之间便知晓了一个弱女子颠沛流离的所有往事,秋玉飞只觉腹中气血翻涌,一口鲜血涌上咽喉,却被他强行咽了下去,他本是知音人,故而这琴中无限悲苦也最能伤他。当他顺着琴音终于在重重楼阁之中寻到灵雨所在的花厅之时,琴声中却突露变徵之音,其中更有绝决之意,秋玉飞心中大惊,凌空飞渡,一抹雪影足不沾地扑向那花厅,全不理会四面响起的惊呼声和此起彼伏的警哨声,一脚踢碎了花厅大门,向内望去,只见阔别两年的灵雨正端坐抚琴,琴边的香炉之中余烟袅袅,三支清香已经燃尽,而在灵雨面前,一个黑衣少年手执利剑,正指在灵雨咽喉处。而灵雨神色平和淡漠,对那利剑视而不见,似乎已经漠视生死。可是秋玉飞却能从方才的变徵琴音知道,灵雨心中也有一腔悲愤不平。

秋玉飞突然闯入,惊动了厅内众人,琴声嘎然而止,灵雨满面惊喜,眼中神色变幻莫测,似是担忧,又似宽慰。

秋玉飞目光闪动,只见花厅之内除了灵雨和那少年之外,还有两个中年侍卫,皆是气度沉凝,双目神光隐隐,此刻他们已经拦在自己面前,威势如山,其中一人怒道:“阁下何人,为何擅闯嘉郡王府邸?”

秋玉飞冰冷的目光穿过两人,径自落到那黑衣少年身上,冷冷道:“李麟,便是你要杀害我秋玉飞的未婚妻室么?”

李麟脸上露出古怪的神色,目光闪烁地道:“秋叔叔何出此言,此女乃是凤仪门余孽,本王欲要杀她雪恨,为我生母报仇,魔宗与凤仪门乃是宿仇,她怎会是叔叔的妻室。”

秋玉飞怒道:“我与她的事情无需嘉郡王过问,秋某只问你,肯不肯让我将她带走?”

李麟冷笑道:“本王言出如山,纵然是四公子你也不能改变本王心意,你看见那香炉没有,方才本王和灵雨姑娘约定,许她临死前再抚瑶琴,香尽就是她人头落地之时,如今香已燃尽,人还尚存,本王已经是失信之人,四公子还是速离此地的好,看在魔宗和我姑夫的份上,我不追击阁下闯入我府邸的罪责就是。”

秋玉飞心中冰寒,他和这少年王爷过去曾在江哲府中见过,知道他杀伐决断,更胜齐王当年,他若定要加害灵雨,纵然自己舍命相护,也终究会有无能为力的一日,不由生出杀机,一字一句问道:“灵雨不过是无辜弱女,你为何咄咄逼人,定要她性命,莫非你堂堂的大雍郡王,便是这般恃强凌弱么?”

李麟眼中露出刻骨仇恨,道:“本王原本是父王嫡子,堂堂的齐王世子,若非母妃陷入凤仪门,犯下谋逆大罪,以致宗谱除名,本王怎会失去世子之位,本王与凤仪门誓不两立,这次南来,本欲将凤仪门斩尽杀绝,如今那些恶毒妇人已经恶贯满盈,只可惜却不是本王下的手,如今灵雨姑娘落入我手中,这是她的不幸,也是苍天给本王一个报仇的机会,我不杀她,岂非辜负了天意。”

秋玉飞心中杀机越发浓厚,望着李麟冷笑道:“好,好,你要杀她,我便杀你。”

话音未息,也不见他如何动作,身形已经掠过两个侍卫拦阻,诡异地出现在李麟身前,一脚将他踢飞出去,“砰”的一声,李麟的身躯撞在了墙壁上,烟尘四起。秋玉飞心中虽然杀意极盛,可是想到李麟的身份,终究是没有痛下杀手,饶是如此,李麟只觉眼前发黑,口中一甜,一口鲜血已经吐了出来,四肢百骸更是剧痛无比,跌在地上爬不起来。他心中大骂道:“该死的霍琮,你不是说我身上的软甲可以卸去五成内力,不会让我重伤么?又说秋玉飞见到灵雨姑娘无事,不会痛下杀手,怎么本王却连一脚都没有撑住?”

这时,那两个羞愤交加的侍卫已经纵身过来,不过看在灵雨和李麟眼中,只觉秋玉飞身影一闪,这两个侍卫已经再度被逼退,不过秋玉飞却也没有继续向李麟出手,而是退到了灵雨身边,那两个侍卫护在李麟身前,面上满是惊怒之色,却不知秋玉飞虽然表面一无损伤,但是却已经气血翻涌,若是这两人此刻出手,定可将秋玉飞重伤。

秋玉飞的目光在那两个侍卫身上凝住,这两人一人使得是百步神拳,一人使得是鹰爪拳,都已经可以勉强列入绝顶高手的品级,若和欧元宁相比,至少也有他六七成的水准,而自己却因为内伤未愈,只有平日五成的功力,方才占了上风,不过是靠着身法灵巧,若是真想取这两人性命,却多半会被他们反噬重伤,这样的两个侍卫,纵然以李麟郡王的身份,也未免过分奢侈了。

这时,李麟已经能够站起来了,他拭去嘴角血痕,高声道:“列血杀阵,若要放走一人,你们便给本王抹了脖子吧。”

花厅之外传来惊天动地的应诺声,然后传来兵刃撞击声,弓箭上弦声,而在这其中,秋玉飞更是听见许多或者沉凝如山,或者轻灵如风的脚步声,这些人的身手都是一流以上的水准,其中更有两人,武功更是胜过厅内的两个侍卫,这样的排场,就是齐王殿下也不过如此,秋玉飞心中突然生出莫名的感觉,莫非自己已经落入了一个陷阱么,可是有什么人会这般费心对付自己呢?就是大雍皇室想要对魔宗下手,也不会选在江南未定的今日。只是此刻秋玉飞却也顾不上去想这些,他只是转头望向灵雨,眼中尽是歉疚,他已经知道,凭着自己的力量,已经没有可能救走她了,伸手握住灵雨的素手,灵雨抬头向他望来,清灵如水的明眸尽是感激之意,四目相对,目光纠缠在一起,再也难以分开。

良久,秋玉飞长叹道:“嘉郡王,你当真是用心良苦,想必定是设伏以待,只是不知秋某与你有何等深仇大恨,让你如此费心设下这个圈套?”

李麟目中闪过一缕寒芒,淡淡道:“本王身边禁卫如云,一向如此,秋叔叔言重了。本王一向对四公子十分敬重,就是不看在魔宗份上,也要顾及姑夫大人和四公子的交情,只要留下此女,任凭本王处置,今日之事,本王便当作没有发生过。”

秋玉飞眼中闪过悲色,淡淡道:“灵雨乃是秋某未婚妻室,如果嘉郡王定要加害,那么就将秋某一起算上吧。”

李麟闻言,心知秋玉飞已经隐隐屈服,但是按照事先和霍琮商量过的宗旨,自己却不能轻轻放过,故意在眉宇间露出一丝杀气,傲然道:“四公子言重了,不论是皇上还是我父王,对魔宗都是敬重有加,四公子更是姑夫大人的至交,李麟纵然胆子再大也不敢得罪四公子,只是此女乃是凤仪门余孽,就是魔宗也容不得此女入门,否则四公子怎会被迫闭关,想来四公子今日来此,也没有得到魔宗的许可。纵然本王宽恕此女,莫非四公子还能和魔宗作对么,大雍一统天下,乃是迟早之事,魔宗的手段在下虽然只是耳闻,却也知道不同寻常,天下之大,也无四公子容身之地,还是放弃此女,返回向魔宗负荆请罪,才是正道。”

秋玉飞只觉心中一震,这少年王爷字字句句都深入人心,令他也难以辩驳,但是目光落到灵雨苍白的面容上,却是再也不能移开,纵然粉身碎骨,也难以割舍这样的知音,抬头毅然道:“既然如此,就让在下领教一下嘉郡王的血杀阵,如果秋某能够带走灵雨,此事可否到此为止?”

李麟叹道:“本王不才,却也知道凭四公子现在的实力,纵然护住这女子,也必将重伤难愈,死期不久,秋叔叔何必要为个女子这般牺牲?”他言辞之中信心十足,灵雨虽然不甚了然双方实力的深浅,也已经相信了他的说法,再度抬头望向秋玉飞,只见他神色凝重,显然李麟这番话并无虚假,心中一寒,知道这渺茫的一线生机终于断绝,正欲将手抽还,却见秋玉飞淡然坚定地道:“请问郡王爷,如果秋某带着灵雨闯出血杀阵,此事可否到此为止?”

灵雨闻言顿时愣住,她多年流落风尘,见惯了负心自私之人,心门早已深锁,埋首琴艺,却也有不愿跻身世俗之意,这些年来,只有柳如梦凭着两年来的点点滴滴,得到她的信任敬重,而秋玉飞虽然是她心中思慕之人,可是却也并不十分信任他,更何况在这种生死关头,纵然秋玉飞被迫舍弃自己,她也不觉得有什么意外,可是秋玉飞却终究不曾舍弃她,不知不觉间,两行清泪滚滚而下,低声道:“这又何苦呢,四公子本是前程似锦,何必为了灵雨甘犯众怒,忤逆尊长。”

秋玉飞心中一沉,低头望去,只见灵雨雾水迷蒙的双眼中满是绝决之意,然后便觉握在手中的玉手突然变得柔若无骨,轻而易举地脱出秋玉飞掌握,眼前一花,原本坐在琴凳上的灵雨,已经反纵而起,婀娜的娇躯便如游鱼一般在空气中滑动折转,秋玉飞心中闪现一个早已淡忘的名字,不由惊叫道:“陨玉搏杀术,灵雨不可鲁莽。”说罢展开双臂,径自向灵雨扑去,却是要将她制住,陨玉搏杀术虽然是近身搏斗术中最可怕的一种,但是却也有许多局限,一旦施展出来,多半是玉石俱焚的下场,灵雨非是心狠手辣之人,一旦施展出来,只怕反而更加危险。可是灵雨的动作仿佛游鱼一般浑若天成,娇躯更是仿佛变成无骨灵蛇,当秋玉飞将要把她凌空抱住之时,她却如同鱼儿游水一般,蓦然在空中转过身来,秋玉飞虽然也及时变招,却只能撕下她一幅裙袂,只是一线之差,灵雨已经撞碎花厅的窗子,冲了出去。

秋玉飞再也顾不得伤势,深吸一口真气,身形便如羽箭一般追出了窗子,灵雨本是有心求死,所以纵身而出之后便没有再催力,只是随着余势向地上落下,可是她身形尚未落地,便已落入一人怀抱,然后她便觉得两边的景物都变得模糊,寒风迎面扑来,让她几乎不能睁开眼睛。她没有挣扎,因为她不需回头已经感受到熟悉而又陌生的气息,耳边传来羽箭凌空呼啸的声音,可是她心中却没有了一丝恐惧,只是尽量提气轻身,一动也不敢动,生怕自己的任何动作会影响秋玉飞。

秋玉飞丝毫没有悔意,空明如镜的心湖中映出了那些足以洞金裂石的羽箭的轨迹和力道,共有三十六支利箭织成天罗地网向两人袭来,更是将全部逃生之路全部封锁,纵然是他未受伤之前也不敢保证可以全身而退,更何况如今重伤未愈,又带着一个女子,可是他尽量用身躯将灵雨全部遮住,也不顾伤势的加重,心中只有一个念头,便是定要将灵雨救出此地。

他心中明白,灵雨非是想要脱逃,便是最笨的人,也知道那种情况下冲出去多半是死路一条,灵雨又是兰心慧质的女子,怎会不明白,她不过是不想连累自己,自己一个男子,却不能庇佑心爱的女子,便是活着又有什么意义。

利箭擦过他的衣襟发际,秋玉飞尽了全力冲出了第一轮箭雨,几乎已经是筋疲力尽,可是耳中却传来弓弦响生,第二轮箭雨在他最虚弱的时候袭来,秋玉飞强运真气,挥袖拂去,却是一阵头晕目眩,知道自己旧伤发作,正在他已经绝望之时,耳中传来一个冰玉交击也似的声音道:“统统住手。”,与此同时另一个威严的声音淡淡道:“玉飞住手。”

这两句话都不甚响亮,可是却偏偏直入人心,每一个人都生出说话之人就在自己身边的错觉,而秋玉飞几乎在听到这两个声音的同时,便放弃了一切反抗,便如断线风筝一般向下坠落,而那些向他射来的箭矢几乎是就在他身边被某种力量折断震飞,断矢碎羽零落一地。秋玉飞也顾不得一切,落在地上便放开灵雨,自行盘膝坐下,运气疗伤,但是原本行气如珠的经脉如今却是若断若续,额头上不由渗出汗珠来。

灵雨有身在梦中的感觉,前一刻还在生死边缘挣扎,可是突然之间那些箭矢全部被折断反弹,而自己和秋玉飞也坠落在地,甚至在这时候,秋玉飞仍然小心翼翼地将自己护住,然后他便在雪后仍有潮湿感觉的石径上坐下调息,灵雨只能焦急地跪在他旁边。而就在这时,园中却突然多了两个人,而灵雨几乎没有看清这两人是如何到了自己身边的,其中一人是个灰衣男子,国字脸方正威严,只是淡淡望了灵雨一眼,灵雨便觉气血翻涌,差点摊倒在地,却觉一缕冰寒的真气凌空渡入体内,顿觉神清气爽,气息平和下来。抬头望去,却见另一个容颜如冰雪也似的清秀青年对自己微微一笑。她自然不知道这两人已经借着自己暗中拼了一个回合,只是担忧地看着秋玉飞,就连一个灰发霜鬓的男子在众多侍卫护卫下走了进来,低声传令,挥退园中所有设伏的侍卫的情景都没有留意,只是忧心忡忡地望着秋玉飞额上的冷汗,却连拭去他额上的汗珠都不能够。

见此情状,那灰衣男子眼中闪过忧色,目光落在了已经被江哲召到身边低声训斥的李麟身上,眼中闪过寒芒,原本正在低头做忏悔状的李麟只觉如山威势扑面而来,不由抬头望去,只见一双隐隐似有火焰的幽深黑眸满是杀机地望着自己,胸中如受重击,一瞬之间呼吸似乎都被截断一般,若非他骨子里面的桀骜支撑着他强自和那人对视,只怕已经屈膝在地了。这时候小顺子身形微动,已经挡住了那男子的视线,李麟只觉双膝一软,身上压力骤失的轻松感觉让他差点软倒在地。幸好旁边的呼延寿扶住了他,只不过李麟怎么看都觉得呼延寿的眼神不善,手中的力气也未必太大了些,李麟为时已晚地想起呼延寿的夫人,澄侯苏青的出身,差点委屈地仰天长啸,却不曾发觉,那原本埋藏心中多年的仇恨渐渐淡去,再也不留一丝痕迹。

且说那男子和小顺子四目对视,两人之间的数尺距离仿佛变成了密闭的空气,劲风气流横冲直撞,无数次试探交锋闪避,若非是这两人有志一同,各以内力约束两人之间的暗战,只怕早已经是雷破天惊,到时候只怕园内再无人可以停留,更别说让秋玉飞调息疗伤了。所以不过片刻,两人便都颇有默契地住了手。

这时候李麟低着头走了过来,手中捧着一个八角形锦盒,那男子目光一闪,已经看到锦盒上面的篆字“小还丹”,这竟是少林寺百年只能练就一炉的灵丹,若论天下治疗内伤的药物,无出其右,纵然是医圣桑臣所炼制的药物,也有所不及,尤其是这种情况下,最是秋玉飞所需的灵药。那男子望向李麟的目光柔和了许多,“小还丹”的珍贵自不待言,纵然是李麟的身份,也应该很难拥有,不过他一时也无心去想李麟如何得到此药,伸手接了过来,塞到秋玉飞口中,然后将手掌按在秋玉飞背心,渡气助他疗伤。

李麟心中一宽,知道自己求和之意已经被这位魔宗首徒段凌霄所接受,总算暂时不必担心了,等到过些日子再慢慢解释吧,想到自己所做出的牺牲,差点要落下泪来,不过想到从慈真大师那里偷了一粒小还丹的江慎,李麟又忍不住扯扯嘴角,不知道那小子在受什么责罚呢?(浮云寺之内,江慎正蓬头垢面地在禅房里面抄写着厚厚的经文,不时地对天哀嚎道:“啊——,为什么师父也像爹爹一样罚我抄书啊?”)

此刻的秋玉飞神色已经变得平和,入口即化的小还丹化成一股暖流流入四肢百骸,而背心渡入的同源真气如同甘露一般滋润着他几乎已经枯竭的丹田。他心中已安,邪影和大师兄同时出手,便已再无危险,以大师兄的性情,纵然要杀了自己,也不会为难一个无辜女子,而邪影更是高傲之人,更不会趁人之危,更何况他本就不相信这次的事情会是江哲的意思,放下一切心事,万念皆空,很快就进入物我两忘的境界。

等到秋玉飞收功而起的时候,第一眼便看到灵雨尽是泪水冰霜的狼狈容颜,今日整个下午,天空中都是彤云密布,寒风更是越来越紧,他浑忘了一切,伸手将灵雨揽入怀中,却觉得触手一片冰冷,灵雨周身上下早已被寒风吹透,只是她却不肯回到屋子里面去,若非她的内力已经有了小成,只怕她早已撑不住了。直到灵雨含羞推开秋玉飞之后,他才发觉大师兄段凌霄正和小顺子四目相对,虽然没有任何动作言语,可是在秋玉飞看来,这两人之间已经是一羽不能加,轻尘不能落,即使是衣衫的飘动,眼神的闪烁,都可能是激战爆发的开始。

秋玉飞翻身而起,拉着灵雨拜倒,恭谨地道:“多谢大师兄救命之恩,玉飞自知罪不可恕,还求大师兄不要为难灵雨。”

段凌霄闻言微微皱眉,就在这一瞬间,小顺子已经出手攻去,他的招式诡异狠辣,段凌霄的反击也是凌厉非常,只见人影轻轻闪动,合而又分,除了秋玉飞之外,别人就是连发生了什么都看不清楚,更别说看出谁胜谁负。

而刚从屋子里面踱步出来的我可不管谁胜谁败,方才狠狠地训了李麟一顿之后,我就一直在想如何处理这个局面,此时主意已定,微笑着走到两人身边,道:“段兄,你和小顺子已经较量完了,玉飞可还在那里跪着呢,你这个做师兄的也得说句话才是。”

段凌霄冷冷瞥了我一眼,才看向秋玉飞,冷冷道:“若非看在你险死还生,我便将你力毙当场,为了一个女子,竟然违逆师尊谕令,哼!”那一声冷哼仿佛冷箭一般穿透了灵雨的心,只觉再也无力支撑娇躯,眼前一黑,便向下栽倒,却被秋玉飞扶住。

秋玉飞叩首泣道:“大师兄请不要怪罪灵雨,一切罪责全部由玉飞承当。”

段凌霄眼中闪过寒光,举手欲要拍下,却怎舍得下手,但是看到秋玉飞倔强的模样,心中又是怒气丛生。目光落到灵雨身上,却又迅速移开,他到合肥已经两日了,本是守株待兔等待秋玉飞,所以他几乎是和秋玉飞同时到了此地,一切都看在眼里,心里也对这女子生出敬意,虽然怒火未息,却不愿再为难于她。

这时候我总算松了口气,看来段凌霄并非无情,先上前先对秋玉飞道:“玉飞,这却是你的不对了,你违背魔宗之命逃到合肥,又在李麟这个娃娃的计算下受了重伤,岂不是丢尽了魔宗的脸面,大公子重责于你,也是爱之深,则之切,你理应谢罪才是,怎能还与大公子顶嘴。”偷眼看去,段凌霄的面色果然和缓了许多,只不过李麟的脸上已经是黄连模样了,我也不去管他,径自对着段凌霄一揖道:“不过大公子也有不对之处,玉飞乃是阁下师弟,魔宗无心理会俗务,大公子长兄为父,玉飞的婚姻大事,大公子理应关心才是,窈窕淑女,君子好逑,灵雨姑娘乃是秀外慧中的好女子,又和玉飞志趣相投,乃是天作之合,大公子理应成全才是。更何况灵雨姑娘如今已经无依无靠,若是大公子坚决阻止他们的婚事,灵雨姑娘不免流落天涯,若有什么闪失,不仅玉飞心碎神伤,就是魔宗的面子也有损伤。江某也知道大公子难以做主,不过若是将她留在玉飞身边,大公子应还是可以说服魔宗的,过得三年两载,如果魔宗和大公子觉得此女确是玉飞良配,不妨成全他们,若是仍然不许,也可有个妥善处置,也免得贻笑天下,断不可因为身份地位这些小事便拆散鸳鸯,致令有情人鸾凤漂泊。”

段凌霄心思数转,却也觉得有理,无论如何此女与玉飞之间的事情已经难以遮盖,若是任凭此女流落江湖,若是有个归宿还好,若是不幸被人纳为姬妾,传扬出去,岂不是令玉飞蒙羞么,不若按照江哲的法子好些。转念一想,却不由失笑,江哲虽然表面是为了魔宗的颜面,可是在他看来,那灵雨品貌都属上乘,若是在玉飞身边数年,不仅两人情意更深,就是师尊也会软化的,虽然看穿了江哲心意,可是毕竟他也已经心软,终于长叹道:“既然江侯这样说,段某便担些干系,带他们两个回去向师尊请罪。”

秋玉飞闻言大喜,连连叩首,道:“多谢大师兄恩典。”

灵雨心中一阵迷茫,糊里糊涂地随着秋玉飞拜谢之后,看到段凌霄也微微露出笑容,才知道自己终于挣脱了一生的悲凉,可是心中刚有些欢喜,便想起同病相怜的柳如梦来,又想到从闲言碎语中得知柳如梦如今就在江哲府上,生出求恳之意,转念一想,师父从前经常大骂江哲阴险歹毒,心狠手辣,和凤仪门之间更是不共戴天之仇,他如今替自己陈词,想必是看在秋玉飞身上,若是自己忤逆了他,他随便说几句话,就可以让自己重新沦入苦海,心中生出怯意。

在秋玉飞搀扶之下,她艰难地站了起来,看着江哲与那位段大公子相携而去,突然生出无穷的勇气,挣开秋玉飞的手臂,扑跪在地,高声道:“江侯爷,小女子有事相求!”

我本来正在和段凌霄叙旧,邀请他到我府上流连几日,却听到身后传来灵雨坚定中带着恐惧的话语,不由愣住了,对于此女我其实并不十分关心,只不过她没有什么威胁,玉飞又钟情于她,爱屋及乌罢了,可是她突然作出这般举动,却令我生出异样的感觉,停住脚步,淡淡道:“什么事情?”

灵雨不知怎么,突然觉得四周一片寂静,那灰发霜鬓的男子虽然是背对着自己,可是自己仿佛能够感觉到他刺透人心的目光停驻在自己身上,这一刻,她生出无穷的恐惧,只觉得这文弱之人突然变得可怕至极,纵然是秋玉飞和段凌霄有心救她,也无能为力。但是她很快就平静下来,想起柳如梦伤心欲绝的模样,她抬起头恭敬地道:“妾身非是不知自量,只是受了如梦姐姐大恩,不能不报,两年前妾身遭遇大变,若非柳如梦援手,妾身已经生不如死,这一次国主求和,强行将如梦姐姐和妾身列入贡单之中,妾身幸得四公子相救,侯爷赦免,得脱大难,可是如梦姐姐却仍身陷苦海,求侯爷网开一面,还如梦姐姐自由之身吧。”

我用崭新的目光望向灵雨,不是所有女子可以在这样的时候还记着同命姐妹,有恩报恩的,心中生出钦佩之意,正想告诉她不必担心柳如梦之事,李麟却在一旁插嘴道:“喂,你也太多事了,我姑夫位高权重,难道会委屈你的如梦姐姐么?”

我微微一怔,李麟胡说八道什么,莫非他以为我真的会贪恋女色么?还未想到如何解释,灵雨已经再拜道:“侯爷自然是位高权重,妾身也知侯爷的诗文天下闻名,若能得到侯爷垂怜,如梦姐姐本不会觉得委屈,只是如梦姐姐已经有了爱侣,誓结同心,生死不渝,昨夜宋公子闯入营中想要救走姐姐,却重伤被擒,还不知生死如何,妾身曾听说侯爷与大雍宁国长乐长公主情比金坚,想必也知道有情人不能成眷属的痛苦,还求侯爷放了如梦姐姐吧!”

宋公子,重伤被擒,我低声问旁边的小顺子道:“这个宋公子不会是逾轮吧?”

小顺子目光一闪,道:“想必就是他了,他在柳如梦身边呆了三年多,若非是他,还会有人这么大胆子闯营救人,不过他也未免太倔强了,若是来向公子相求,怎会落一个重伤被擒的下场。”

我微微皱眉,无意中看见段凌霄眼中神色变幻,不准备让他探知太多隐秘,转头对呼延寿道:“派人去南楚使团,将人要过来。”眼中露出一丝杀机,如果逾轮已经被他们杀了,可别怪我将南楚使团全部葬送在归途上。

灵雨闻言大喜,她本不敢奢望,没想到江哲不等她婉言相求,便下令救人,若是苍天庇佑,如梦姐姐和宋公子还可以破镜重圆,便再度叩首道:“侯爷宽宏大量,妾身代如梦姐姐叩谢侯爷大恩。”

我有些尴尬,望了秋玉飞和段凌霄一眼,心道,总不能说我要救的是个不肖弟子吧,我还要谢谢你通风报信呢。秋玉飞自是已经猜知真相,上前扶起灵雨,眼中隐隐有些笑意和自豪,而段凌霄望向灵雨的目光更是柔和了几分,只有灵雨自己不知道这一席话让她的命运从此有了定数。

李麟在一旁暗暗欣喜,虽然本来已经有了法子让姑夫知道逾轮的事情,可是如今这灵雨说出来却是水到渠成,省了自己多少事情,便故意道:“姑夫是说那个姓宋的刺客么,昨天我巡营的时候发觉使团出了问题,已经将那人要过来了,我看那人颇有胆量,没有为难他,正在让军医替他诊治呢。”

我闻言有些惊喜,却又皱眉道:“他没有和你说起自己的身份么?”

李麟状似不在意地道:“没有啊,我见他一言不发,就没有多问,既然姑夫要这个人,一会儿我就派人从营里送过去。”

我有些恼怒,逾轮也太倔强了,到了这种地步仍然不肯向我屈服,心中一叹,此子心结我素来知道,罢了,看在飘香份上,我也不为难他们了。吩咐李麟一会儿将人送过去,我便请段凌霄、秋玉飞和我同行返回住处,不过怎么看我都觉得这两人有些看笑话的意思。段凌霄倒也罢了,我们之间的过节想必他还没有完全忘记呢,秋玉飞却是未免太忘恩负义了,腹诽着两人,我却是一路微笑着面对他们,总不能让别人看了我的笑话吧。

回到府邸,已经是日暮西山,走下马车,却见门前出来相迎的竟是白义和盗骊,心念一转,白义已是秘营之首,盗骊和逾轮一向亲密,这两人定是为了逾轮来求情的,说不定就是为了逾轮,才没有将柳如梦之事告诉我,多半是以为我会旧情难忘,做出夺人所爱之事,想到此处,不由脸色一寒,也不理会他们,拂袖走入大门,令人将灵雨送到后面安置,便请段凌霄和秋玉飞到花厅叙谈。

不多时白义和盗骊亲自带人送了酒菜过来,礼数甚恭,我也不理会他们,等到酒过三巡之后,花厅之中其乐融融的时候,这两个在一边佐酒的师兄弟终于按耐不住了,盗骊径自上前拜道:“先生,弟子有下情禀告。”

我早知他的心意,却故作不知,淡淡道:“有什么事情,等到酒席散后再说吧。”

盗骊叩首道:“此事十分紧要,请容弟子陈词。”

我瞥了段凌霄一眼,终究是不愿让他看了笑话,欺负弟子也不能被人看见不是,便道:“你说吧,我会斟酌的。”

盗骊眼中掠过喜色,道:“先生,若有一对天作之合的爱侣被人活活拆散,请问先生是否应该成全他们的姻缘。”

我心中暗笑,盗骊什么时候也喜欢这么绕着圈子说话,答道:“有情人终成眷属,若真是天作之合,当然应该成全。”

白义也下拜道:“先生言出如山,弟子一向钦服,这两人如今就在合肥城中,彼此情深意重,只是被人阻挠,以致中道分离,若得先生一言,他们便可白首偕老,弟子叩请先生开恩,饶恕那人从前过失,允许他和那位姑娘缔结鸳盟。”

我轻叹一声,心道,看来逾轮这小子终于服软了,罢了,看在飘香面上,我就成全你们吧,不由失笑,看来今日见到柳如梦之后,我的心肠软了不少,开口道:“既是有情有意,我自然不会拦阻,你让他们过来见我,今日我便给他们订下婚约。”

盗骊和白义两人都是大喜,连忙走出门去,段凌霄笑道:“江侯果然最喜成人之美,将来若是玉飞与灵雨姑娘好事得偕,想来也应先拜谢江侯才是。”

我心情舒畅地道:“段兄言重了,我不过是说了几句话,还需段兄作主,玉飞和灵雨姑娘才有希望可言,成人之美的却是段兄才是。”

这时,门外传来脚步声,我也不理会,径自和段凌霄说话,心想给逾轮一个下马威也好,可是耳中传来开门的声音,却有数人走入,我还没有回头看去,只见段凌霄和秋玉飞的面色同时变得诡异非常,就是刚才被段凌霄拉入席中的小顺子也是一脸的古怪神色,我心中一跳,连忙回过头去,只见跪在地上的一双璧人,非是我想像的逾轮和柳如梦,而是大雍太子李骏和我的爱女江柔蓝。

我颤抖着伸出手去,指着两人道:“你们两人来做什么?”浑不觉自己的声音已经变得尖利非常。

李骏这时候膝行上前,叩首道:“李骏从来视先生如父,当日冒犯先生,今日特来请罪,此次乃是旧事重提,骏倾心柔蓝十余年,刻骨铭心,难以割舍,求先生将柔蓝许配给我,骏立誓绝不辜负蓝儿一片深情。”

我大声道:“万万不行,此事绝不可能。”

这时候,花厅的门又开了,霍琮施施然走了进来,拜倒道:“先生,弟子等方才都听到先生同意太子殿下和昭华郡主的婚事,段大公子和秋四公子便是见证,先生既然说有情人当成眷属,太子殿下和郡主乃是天作之合,人人夸赞,他们两人又是情深意重,两年隔绝,深情不改,还请先生成全他们。”

我望着霍琮,心念电闪,许多想不通的事情突然明朗起来,为什么我事先没有见到贡单和明鉴司的呈文,以至于我在银安殿上当众失态,为什么赤骥去给秋玉飞送信,为什么李麟突然想起了报仇,为什么李麟突然有了那么强的实力,差点困死秋玉飞,显然我是落入了一个大圈套之中,也只有霍琮才有这个本事调动了我身边全部的力量。

啊,原来如此,八骏因为逾轮之事和他同谋,李骏、李麟、柔蓝一向同进退,这次又是李骏和柔蓝的终身大事,自然是鼎立相助,而我身边的虎贲卫以及合肥城内的千军万马,谁能不卖太子殿下的面子,更何况这桩婚事皇上的意思也很明显,虽然没有明下诏旨,可是估计没有人不知道他的心意,所以这些人联手将我的耳目蒙蔽,对了,还有那粒小还丹的出现,多半是慎儿搞得鬼。目光一闪,我盯住了小顺子,这些事情可以瞒过我,却不该瞒过他的,怎么他也不露一些声色,小顺子有些歉疚地回望过来,又看了柔蓝一眼,我立刻明白过来,小顺子素来疼爱柔蓝,柔蓝既是我的女儿,也是他的女儿,若是柔蓝相求,小顺子多半是眼睁眼闭的了。

这时候耳边传来段凌霄的声音道:“原来江侯也是棒打鸳鸯之人呢,方才却还劝我向师尊陈词,在下看的不忍,还请侯爷也莫忘了成人之美才是。”

我心中一震,霍琮这小子,骗我开口许诺不说,还弄了个不敢敷衍的见证在此,哎呀,我真是聪明一世,糊涂一时,魔宗是何等人物,怎会两三年都想不通,多半只是还没有机会下台,若是玉飞闭关日满,多半就会给灵雨一个机会了,若非如此,赤骥怎么可能在魔宗眼皮底下送信进去,段凌霄身在长安伴驾,岂是随便可以脱身的,若无皇上授意,他怎能千里迢迢地捉拿秋玉飞呢。

眼角余光一闪,只见段凌霄神色淡定,而秋玉飞则是神色迷茫,罢了,看来也只有秋玉飞是和我一样蒙在鼓里,看来他一路上的挣扎辛苦,多半是魔宗借此磨练试探玉飞和灵雨姑娘,而且有我江哲亲自替灵雨姑娘说情,纵然秋玉飞娶了灵雨,谁又能说出什么不是。不过霍琮这小子能够看出来我定会支持玉飞与灵雨姑娘的婚事,却也是知我甚深了。

想通了全盘真相,我指着霍琮,想要痛骂却难以出口,这小子倒是青出于蓝,利用了种种情势,将我陷入圈套,人人都有好处,只有我有苦难言,如今我若要反悔,岂不是在段凌霄和秋玉飞面前丢了颜面,再说我今日落得“众叛亲离”的下场,俱是为了李骏和柔蓝的婚事,如果我当真还不答应,只怕这些人从此都要和我离心离德,这种日子可怎么过啊,若想答应,一想到柔蓝的终身幸福可能会是镜花水月,我怎也说不出口。

众人只见江哲脸色初时铁青,继而通红,然后又变得苍白,都生出忧虑,面面相觑,谁也不敢上前催促,这时候只见小顺子长叹一声,起身冷冷道:“柔蓝留下,其他人先出去。”

这种时候,就是段凌霄和秋玉飞,也丝毫没有得罪小顺子的打算,不过片刻,所有的人都出去了,就连小顺子也不例外。

我长叹一声,看向柔蓝,柔蓝走到我面前,跪在我膝下,抬起头望着我道:“爹爹,对不住,蓝蓝和他们一起骗了你。”

我伸手轻抚摸她的秀发,目光落到那明艳照人的娇容上,此刻,柔蓝那双黑亮澄净的明眸满是依恋和歉疚,我叹息道:“蓝儿,你莫非不知道爹爹的苦心么?”

柔蓝眼中有些雾气,道:“蓝儿知道,帝王之家,多的是人心险恶,少的是真心真意,爹爹不希望女儿日后受苦,此恩此德,蓝儿终生不敢稍忘。虽然大家都不说,可是我却知道自己不是爹爹的亲生骨肉,但是这些年来爹爹待蓝儿却比弟弟更好,就是不让蓝儿和骏哥哥一起,也都是为了蓝儿着想,可是我只爱着骏哥哥一个人,若是不能和他在一起,女儿终生都不会快乐。”

我强忍眼中泪水,道:“傻丫头,如果他变了心,或者你们终究不能白首偕老,你也不后悔么?”

柔蓝低声唱道:“春日游,杏花吹满头,陌上谁家少年足风流。妾拟将身嫁与,一生休,纵被无情弃,不能羞。”

我只觉得心神巨震,望着神色中尽是柔情蜜意的柔蓝,仿佛看到了当年的飘香,终忍不住用衣袖掩住面容,泪水长流,耳中听见柔蓝温柔而坚定的声音道:“爹爹,纵然不能白首偕老,纵然骏哥哥日后变心,女儿也绝不会后悔。”

不知过了多久,我才哽咽道:“罢了,女大不中留,你出去告诉他们一声,就说这桩婚事我同意了,记得告诉李骏,若是他有朝一日负了你,可别怪我不放过他。”

耳边传来柔蓝惊喜的呼声,然后便是她喜极而泣的声音,推门而出的声音,甚至在门没有合上之前,我听到外面传来的欢喜至极的呼喊声,我转身向隅,不愿让人看见我泪流满面的模样。过了片刻,有人走到我身边,我也不需抬头,知道这人一定是小顺子,他最知我心,这时候是绝对不会让别人打扰我的。

尽情哭了一通,接过小顺子递来的面巾,拭去泪痕,我问道:“他们都知道了?”

小顺子笑道:“都知道了,估计现在齐王殿下也知道了,给皇上的密报说不定已经送出去了,过不了几日,想必就会有懿旨诏柔蓝回京,然后应该就是选妃大典了,无论如何,内定也要走走过场的,方才太子殿下和我说过,他已经向皇后提过,这次不选侧妃,若是成亲三年之后没有子女,再选侧妃不迟。”

我嘀咕道:“算这小子有些良心,罢了,真是便宜了他。逾轮的事情怎么样,霍琮不会过河拆桥吧?”

小顺子忍笑道:“方才琮公子托我向您辞行,太子殿下那边的军职他也辞了,据说他前些日子跟洛阳白马寺的法真禅师通过书信,要去帮助法真禅师翻译梵文佛经,刚才他已经上路了,他身边的侍卫我让他一起带上了,毕竟现在战事未平,路上还不安全。至于逾轮和柳如梦的事情,就连太子和柔蓝的婚事你都点头了,难道还会为难他们么,虽然柳姑娘知道逾轮本是你的弟子,还有些恼恨,可是念在逾轮为她出生入死,倒也没有一怒绝情,过些时候我想他们会过来拜见你的。”

我叹道:“霍琮这小子也算聪明,搞了这么大一件事情,牵扯了大雍上上下下多少人,他若不去避避风头,可就是得意忘形了,不过小顺子,看着这些孩子,我怎么突然觉得自己老了呢?”

小顺子淡淡道:“老便老了,也没有什么不好,我还不是一样。”

我不满地道:“那怎么一样,我还不到四十五岁,就已经头发灰白了,你只比我小六岁,看起来却像是弱冠年纪,当真是不公平啊。”

小顺子眼光一闪,道:“那有什么要紧,想要青春常驻不容易,想要老些还不是易如反掌,总不会等到公子白了头发,做了祖父、外祖父的时候,还让别人将我当成你的孙儿就是了。”

我不由哈哈大笑,和小顺子说笑总是这般有趣,不知怎么,突然觉得很是疲倦,如今蓝儿有了归宿,我也没有什么牵挂了,至于慎儿那个傻小子,我可不会替他操心,陆家的事情都已经有了安排,就连南楚宗庙香烟的延续,皇上也早就答应了,等到明春,姜海涛便可一举攻破宁海军山和吴越水军,然后沿江而上,直逼建业,然后秦勇、长孙冀、荆迟、裴云、姜海涛五路大军就可以在江南大地上纵横驰骋,天下一统指日可待,我也终于不必再在宦海沉浮下去了。

伸了一个懒腰,站起身来,小顺子扶着我向卧房走去,四周寂静无声,连个人影都看不见,想来是小顺子知道我的心情,所以不许他们在这里惹我心烦吧,回到房中,我望见软榻便觉再也不能支撑,这一日来心情激荡,似乎我所有的精力都已经耗尽了,直到我躺在床上,才想起一件事情,半梦半醒地道:“对了,替我写封折子,柔蓝回京的事情还要晚些日子,等到攻下建业的时候,我还要带她去看看飘香,这是她名份上的娘亲,不能不去祭祭坟的……”

小顺子口中应诺,却再也听不到江哲说话的声音,回头望去,只见江哲已经睡着了,耳中传来均匀的呼吸声,知道他疲惫已极,不由微微一笑,将安息香点燃,轻手轻脚地退了出去。走出门外,却见不知何时,已经下起雪来,纷纷扬扬,鹅毛也似的大雪转瞬间替眼前的河山披上了银装,天地之间一片沉静,再无声息,似乎也知道自己的主人是多么的疲惫,不敢惊扰了他的休息。

——《全书终》
