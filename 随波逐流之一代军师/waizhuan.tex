\part{外传 清梵曲}

作者:加兰

青烟袅袅,竹帘内,暖酒一壶,竹箫一管。

帘外,白衣少年端然跪坐,展开一卷刚从鸽足上解下的丝绢。

‘师尊,三师兄来报,凤仪门主遭少林十八罗汉阵围杀,战死。‘ ‘知道了。‘仿佛漠不关心般,帘内男子随口答应了一声,提起酒壶。

‘你们……退下吧。‘纷繁的脚步中,没有人听到酒液散乱滴落几案的微响,以及……酒壶蓦然碎裂的清脆声音。

默不作声地,男子举手、翻腕,将满杯青碧美酒倾落地面,之后,莹洁如玉的瓷杯划过一道弧线,越窗而出,寂然不见。

她,究竟还是死了。

挥袖震开房门,北风呼号,满目飞雪如梨花漫卷。

惠瑶……惠瑶。

我竟不能为你报仇!

1.少时抚剑独闲游

浓荫匝地,清影摇风。

马蹄不紧不慢地击打着地面,青衣男子悠然自得地环顾山道上野草闲花,忽然轻轻皱眉。

有杀气。

手在鞍上一拍,青衣人化流光飞身跃起,只十余起落,就见前方窄窄山道上一顶小轿款款而行,周围深林中,长草隐隐摇动。青衣人一皱眉,刚要出声,猛然间一支响箭从头顶摇曳而落,数十把雪亮长刀闪动,从四面八方包围了轿子。山贼们还没喊出‘打劫‘两个字,两个轿夫早就吓得抱着头蹲在地上,恨不得挖个坑钻到泥地里去。

白光一闪,轿帘被长刀挑开,山贼们贪婪的目光立刻就变成了炽热。

‘好美的娘们!‘ ‘咱兄弟这可发了……‘简陋的竹轿里,一个十六七岁的少女尽力蜷缩成一团,脸颊早已褪尽血色。山贼们已经乱糟糟地拥了过去,离得最近的一个差不多要搭上少女衣襟,忽然顿了一顿,跟着就是长声惨呼!

这一声呼号未绝,小轿外,青衣人身随刀走,连绵不断的一团刀光绕着轿子每闪烁一下,就有一个山贼倒下。片刻之间,山道上几十个人横七竖八地倒了一地,而第一个人发出的惨叫犹自在林间回荡不绝。那青衣人随手将长刀还鞘,弹了弹长衣下摆,这才从容转身。

‘姑娘,不必害怕……‘ ‘谁叫你多事的!‘少女秋水般的双眸即使嗔怒也自然流转着一分明艳,青衣男子在这炽烈的目光中呆了一呆,眼前一道白影闪过,只见少女已经俏生生立在山道上。

‘我怎么多事了?‘看着这样一张俏脸,让人实在没有办法生出怒气来,青衣男子声音里已经带了笑意。

‘还说你没有多事!人家本来盘算得好好的,他们抢到我肯定要献给寨主,到时候我就坐免费的轿子上山,顺便擒贼擒王。这下好了,人全给你干掉了!你说怎么赔吧!‘这样也算多事啊?青衣男子瞠目结舌,刚要反驳,却瞥见少女嘴角轻抿,眼梢向上微弯,美目之中戏谑之色一闪而过,不知为什么,笑意忽然从心底直漫上来。两人你望着我,我望着你,不由得同时放声大笑。

长笑声中,男子一揖到地,朗声道:‘是在下鲁莽了,这就去把此山贼人诛尽,权当给姑娘赔礼罢!‘说着也不起身,轻轻一个倒纵,已经上了树梢,转眼去得远了。

‘喂!你……你!‘少女要喊已是不及,一顿足,对两个抖成一团的轿夫道:‘你们把轿子抬走吧,本姑娘不坐了。‘飘身上树,急急追去。谁知那男子的轻功也远在她想象之上,任凭她尽力追赶,仍只见那道青影越来越远。少女才赶到半山腰,已经有零星的惨叫和着血腥气被山风吹落了下来。

衣衫猎猎飘动,少女把速度提到了极致,在踏入山寨时额角已是微微见汗。而杀戮显然已经结束,满地血泊中,青衣人正还刀入鞘,听到她的脚步声传来,转过身,向她微微点头。

那是这一对传奇男女最初的相逢。

她知道了他叫京无极,他也知道,她有一个美丽的名字,梵惠瑶。

如此而已。

除此之外,便如风拂水面,悄过无痕。

2.披霜卧萍州

此去经年。

他从一个魔门的普通弟子成了魔门日宗宗主指定的继承人,而中原武林的种种风起云涌,也逐日送到他的案头——于是他知道,她消声匿迹四年后,如一颗新星在中原武林冉冉升起,一剑光寒,所向披靡;于是他知道,她成了凤仪门的少门主,一个原本不起眼的小门派,在她手中光彩日生;于是他知道,她周围总是围绕着无数爱慕者,却誓言终生不嫁;于是他知道,她的武功已被江湖上列入绝顶高手行列,也成了人人敬仰的女侠;他甚至知道,哪一日她身上新添了几道伤痕,哪一日她病倒在客栈里,逆旅凄凉,举目无亲……

有关她的消息来得一月比一月密。他的师父,当时的魔宗宗主曾不经意地指着一份情报笑说:‘无极,这女子将来恐怕会是你最大的对手呢。‘他默默点头,心底却不经意地闪过一个明眸少女娇嗔巧笑的影子。

也只是一闪而已。

直至那一日。

闲游中原。客栈中,扑簌簌信鸽飞来,须臾,一个弟子悄然推开房门,单膝跪落。

‘启禀少宗主,梵惠瑶将与黑风寨少寨主成亲,喜帖已出,广邀四方宾客。‘她要嫁人了?

屈指一算,京无极哑然失笑。从上一次见面起已经过了八年,她此刻当是二十四五芳华,早是该成亲的年纪了。

左右闲来无事,何不去看一看故人?

一袭青衫,一份薄礼,京无极踏入黑风寨的时候,心情异常轻松而愉快。满山张灯结彩,遍地笑语喧哗,身处其中,人的心情也不知不觉好了起来。看来黑风寨相当重视他们的新娘,她应该会过得好吧——京无极这样事不关己地想着。

然而,新娘一步步踏着红毡走来时,京无极却暗暗握紧了腰间长刀。

有蹊跷。

那个掩在红巾下的的女子看不清神色如何,身形端稳,步履凝重,武功已经到了一个与八年前不可同日而语的境界——在喜娘扶持下,她步步如箭在弦上,一触即发,京无极甚至嗅到了鲜血和锋刃混合的味道!

于是,在满堂宾客的震惊中,只有京无极抱着几乎有些期待的心情,看雪亮剑光一闪,婚礼上的新郎、黑风寨少寨主长声惨叫,残躯血肉纷飞。

喜堂上,大红罗衣分飞碎裂如血色蝴蝶。潮水般惊慌后退的宾客中,京无极屹立不动,默默看着梵惠瑶素手执剑当庭而立。剑气寒光之下,一身凛如霜雪的白衣竟似在猎猎燃烧。

那不再是八年前他记忆中娇俏明艳、带一份顽皮的少女,却正是他八年来逐日在心底里勾绘完成的模样。美丽耀眼如浴火丹凤,凛然高洁如斗雪寒梅。

她在血泊中冷冷回望,目光掠过京无极身上,随即毫不停留地转了开去。

若不相识。

下一刻,黑风寨众多高手已纷纷怒喝着冲了上去,刹那间将雪衣女子淹没。

京无极闪在一旁,冷眼看着黑风寨的下属把梵惠瑶团团围住。嗯,今天黑风寨人来得不少,四十八位护法、一百零八子寨的寨主全部到齐了,加上源源不断涌进来的头目们,怕是最少有五六百吧……可惜。京无极看着人群正中爆起的青芒冷冷一笑,这样的武功,也不过是送上去给梵惠瑶杀的份罢了。

只不过,人太多了终究有点麻烦啊,单是四面八方砍下来的刀剑就够人头痛一阵子了。

也不见他如何作势,身形宛如不受力一般轻轻飘起,悄无声息地上了横梁,随即双手向下一扬。一股庞大的气势席卷而出,武功稍差的人都是不由自主地一个踉跄。看着梵惠瑶剑光吞吐,顺势冲了出去,京无极满意地微微一笑,在梁上侧卧了下来,收敛全身气息,凭高下望。

八年不见,那个女孩儿的功夫还真是大大长进了。别的不说,看她在人群中穿插来去,那一身轻功不说惊世骇俗,也可以当得起翩若惊鸿四字。更令人动容的是她的剑法,满天都是青色的剑芒,那快剑似乎已经超越了人体可以达到的极限,轻轻一触就是一道血色彩虹掠过,生命和鲜血一起喷涌。

这是一幅以鲜血泼墨而成的图画,而其中转折腾挪的雪白倩影,则是其中最为鲜亮的一抹颜色。不知为什么,那残酷搏杀中仍带着优雅华贵的身影,竟让他不期然想到一句古诗:千山剑气寂寞雪。

那样深寒入骨的寂寞啊……

京无极的眉头微微皱了起来。不应该这样的,谁都知道冲出去是最好的选择,她为什么还要在这里缠斗?

下方,梵惠瑶是真正陷入了苦斗之中。她的疾风剑法尚未大成,内力渐渐支撑不住快如闪电的剑招,身形稍微一慢,背上奇痛入骨,已经添了两道伤痕。然而她性子强韧无比,反手刷刷两剑,背后的敌人惨呼着倒了下去,跟着长剑回旋,在身前划出一个优美的圆弧,护着自己冲出圈子。

运足功力侧耳倾听,远处一声轻啸,跟着就是吵嚷声大作,一个粗豪的嗓子大吼:‘那老婆子已经逃了,给我把这个拿下!‘梵惠瑶心里一松:师父已经救出来了,该往外冲了……

这口气一松才觉得身上多处奇痛入骨。杀到这时候,她已经不再去数身上的伤了,只觉得身体越来越冷,眼前也像蒙上了一层黑影,若非她毅力惊人,几乎握不住长剑。梵惠瑶一咬牙,勉力提起一口真气,缓缓挺直肩膀。

那一刻,她白衫尽被血染,整个人摇摇欲坠,然而所有人接触到她凌厉异常的眼神,都忍不住倒退了两步。

这样的人,死在这里太可惜了。从上方看着她挺拔的身姿,京无极瞬间下定了决心,传音道:‘我给你开路,冲出去!‘说着手中长刀一振,化为千百碎片,运足内力向下掷去!

京无极此时的武功已经可以问鼎中原第一青年高手,全力出手岂是寻常?见三把碎片依次掷出,喜堂内直到门口已经没有任何人可以站立,京无极大喝一声:‘走!‘

3.随流飘荡、任东西

梵惠瑶呻吟一声,悠悠醒来,只觉得全身上下痛得如撕裂一般。

她躺在黑暗里默默回想:记得在黑风寨大战,自己得人相助,身剑合一冲出重围。当时,她身负重伤在山道上勉力奔驰,只能点了自己几处穴道止血,却是远远不够。然后……然后……

深深吸了口气,内力运转正常,身上也没有什么禁制。包围着她的被褥温软厚实,鼻端更传来淡淡的药香。定了定神,梵惠瑶就听见外面有女孩低言悄语的声音。

‘里面那个姑娘好漂亮啊!少主该不会是喜欢她吧?‘ ‘说不定哦!巴巴地一路抱了回来,还一直握着人家的手,连裹伤的时候都舍不得放!‘ ‘你胡说什么啊!少主那是在用内力为她续命,听说那个姑娘伤得很重呢!——咦,少主!‘ ‘吱呀‘一声,房门推开,一个人缓步进来,向她脸上打量了一番,微笑道:‘你醒了。‘语声低沉浑厚,似曾相识。

‘多谢恩公相救,敢问恩公尊姓大名……‘梵惠瑶挣扎着要起身道谢,被那人一把按住:‘不要乱动。你的伤很重,全身上下大小三十七道伤口,昏迷了整整三天。‘顿了一顿,又道:‘你果然是记不得我了。‘她以前见过这个人?梵惠瑶咬着嘴唇苦苦回忆,却无论如何想不起来。那人却已经走到门口,回头道:‘好好养伤。放心,在这里没人能动得了你。‘灵光瞬间闪现,梵惠瑶失声道:‘是你!你是——你是京无极!‘那个满地鲜血中从容回首的男子,那个三分冷峻三分决然三分傲气还带着一分温柔的男子,那个和她只有一面之缘的男子,那个她初出道见到的第一位青年高手……是他!是他!

就是因为遇到他,自己才知道功夫还远远不足以行走江湖,回师门闭关苦修,更因缘际会得到了《太阴心经》残本;就是这个人,让自己第一次认识到了江湖的残酷;如今,又是他把自己救了回来……梵惠瑶怔怔地望着那个高大的背影,心里一时间百味杂陈。

‘黑风寨里,也是你……‘ ‘不错。‘ ‘为什么?‘ ‘你这样的人,死了太可惜了。‘沉吟了一下,终于还是说:‘我收到的消息,你师父已经被人救回去了。你那一战,黑风寨四十八位护法死了大半,一百零八处分寨主死了四成以上,现在江湖震动,黑风寨正在大举搜寻你的踪迹。‘他慢慢地说:‘你想怎么办?‘怎么办?梵惠瑶唇角勾起一丝冷笑:‘此仇不报,誓不为人!‘那之后的三个月内,京无极安静地看着这个女子绽放出惊人的光芒。

只在他身边养了七天伤,一到可以行动,她就把消息传了出去。三个月之内,这个女子不断地奔走各地,邀约天下群雄会盟,共讨黑山寨。趁着黑山寨势力大损,各路豪强落井下石,在梵惠瑶居中调节下,一度曾经风云显赫的黑山寨反掌间就成了过眼云烟。

他曾经对她说:‘你想的话,我为你剿了这个寨子就是。‘语气云淡风清,一如当年他随口说为她尽诛山贼。

梵惠瑶淡淡一笑:‘不必了。‘她有一句话没有说出口:已经让你救了我一次,这次再依赖你的话,我有什么资格……和你并肩站在一起?

4.情休休

黑风寨土崩瓦解的时候,京无极并不在梵惠瑶身边。他只是寄来了一张短笺:‘月朗风和,如此良夜,不能与君清夜把盏,憾甚。‘他也有自己的事情要做,既然梵惠瑶没有开口,那他也没必要天天陪着她东奔西跑。至于她平日一呼百应的那些‘少年英杰‘、‘知己好友‘——京无极看都不屑看他们一眼。

此后数年,两人各自奔走,只是偶尔相聚。幸好魔宗的情报网极其发达,每月总有那么几次,他们能收到来自对方的传书——‘晚来天欲雪,拥炉把酒,此诚人生一乐也!惜不得与君同之。‘同时寄到的还有一小坛甘芳醇厚的梨花白。

‘会少离多,浮生若此!昨夜窗前寒梅初开,聊寄一枝春。‘信笺上若有暗香浮动,经久不散。

‘暮春三月,江南草长,新绿宛然可喜。然见百姓流离失所,良不忍也。吾当为谋之。‘ 字迹挺拔峻峭,墨色淋漓,想见寄书人心中豪情激荡。

‘游天门寨,山色旷远,烦嚣尽涤。然如此高处亦有人避秦来居,乱世猛于虎,诚可叹也。‘每次的飞鸿来往都仅有只言片语,仿佛兴到随意涂抹,有些事情,两人更是极有默契地绝口不提。然而,却有什么东西在暗暗滋生,轻柔而又细密,密密笼住了这一对男女,不动声色……

然而,划破这张轻柔细密的丝网的,也是一张飞鸽捎来的短笺。

‘群雄逐鹿,生民多艰。妾欲使天下归一,以拯黔黎于水火。‘紧紧握着这张丝帛,京无极长叹。

惠瑶,惠瑶,你这是何苦。一个女子,再强也不可能一统天下。

这样叹息的时候,他正在与杨老生策划下一场战役,而梵惠瑶已经走进了李援军帐,一夕长谈,李援身边多了一个叫做纪霞的女子。

5.今越期頤按白首

相见争如不见。

他们本就是会短离长,碰得巧了两三个月能有一次聚首,碰得不巧一年半载才能相遇。两人都是豁达的性子,并不觉得这样相处有什么不对,也就听之任之。梵惠瑶投身大雍后,为李援四方奔走,争取各地豪强大族的支持,事务繁忙。两人不但再没有见面的机会,就连书信来往也渐渐疏远。

再相见时,两人都惊异于对方身上脱胎换骨一般的气质。京无极变得更加沉稳凌厉,俨然有了一代宗主的风范,梵惠瑶却是日甚一日的高贵脱俗。尽管两人都是刻意收敛,一身布衣坐在乡村小酒馆的草棚里,旁人还是一眼接一眼地偷偷瞟了过来。

‘这乡间野酿竟别有一番风味。‘京无极微笑地倒了一碗浊酒推过去,‘多日不见,梵小姐风采犹胜往昔啊。‘自己又倒了一碗,一饮而尽。

‘宗主说笑了。‘梵惠瑶举起酒碗,轻轻啜了一口,‘乡村虽好,可惜四方兵戈扰攘,流寇众多,也许你我今天离开村子,明天此地就毁于兵火了也说不定。‘ ‘正是所谓兴,百姓苦,亡,百姓苦。‘京无极悠然接口,‘东晋失道,诸侯并起,要在乱世中力挽狂澜,并非一人之力。以你我的武功才华,也只能择一明主,助他统一天下,方可止息干戈,救民水火。杨老生虎踞中原,猛将如云,麾下十数万精兵,为天下甲兵之冠。梵小姐以为如何?‘定定地凝视着她双目,一时间气氛顿时紧张起来。

梵惠瑶微垂双目,淡淡一笑。

‘论兵威自然是杨老生最盛,然而其人,霸王也,徒负勇名,难成大器。大雍李援胸怀百姓,政通令肃,若此人得天下,当能与民休息。‘ ‘这么说,梵小姐是下定决心要支持李援了。‘京无极轻轻叹息。

‘宗主难道会放弃杨老生么?‘梵惠瑶轻笑。

以他们今日的身份,有些事情已经用不着多说。更何况,在这乱世之中,她的凤仪门和大雍之间,他的魔门和杨老生之间,牵扯都是千丝万缕,又岂是说断就能断的。今日相见,也不过徒尽人事罢了。

京无极凝视着那双明净的眸子,涌到口边的话终究咽了下去。虽然明知一旦对敌,数年情分必然斩断,可是男子汉大丈夫,怎么可能开口要她退出?她那样骄傲的性子……不过徒自取辱罢了。

要么就等到在沙场上堂堂正正将她击败再揭示自己的心意,要么就什么都不用说!

‘李援非常人也,用不了两年,就要和杨老生正面对敌。‘ ‘到时候不过是各为其主罢了,宗主又何必放在心上?‘梵惠瑶笑吟吟地倾身,为京无极又满上一碗,自己率先饮尽。

‘说得好,沙场相见,生死无恨。倒是我不够豁达了。‘一口将酒液饮干,京无极只觉得一股火辣辣的酒气直冲胸腹,豪气顿生,随手将碗一掷,一声长啸,头也不回地出门径去。

他没有回头,也不敢回头。所以他并不知道,身后凝视他背影的梵惠瑶,在这乡村小店里怔怔地坐了多久。……

虽说各为其主,生死无恨,可是京无极啊,两次相救、多年往还,我真的能对你出手么?

6.兜墨洗清秋

‘凤仪门主梵惠瑶台鉴:久疏通问,时在念中。今闻华山舍身崖风景绝胜,八月既望,愿与门主置酒高会,共论天下大势。

魔门京无极谨启。‘雪白的笺纸,挺拔的字迹,一如往日由飞鸽捎来的三两行小语。分毫未变。

来人面前,已是万众尊崇、光芒万丈的凤仪门主微微一笑,淡然道:‘烦请回复贵门主,就说梵惠瑶定当准时赴约。‘送走使者,梵惠瑶转身入内,才关上门,就觉得心潮翻滚,两膝几乎不能支撑身体的重量。

他要来了么?

他要与她动手?

这当是不死不休的一战啊……否则,为什么选了舍身崖那么一个凶险的地方。

这么多年了,京无极的武功一直在她之上,单论武功她就是凶多吉少,更何况,她不知道自己能否下得了手。然而,梵惠瑶挺直了身子,无论如何,她一定要胜!她为之呕心沥血的大雍还没有统一中原,无论如何,她绝对不可以输!

听了使者的回报,京无极也只有默然长叹。别后数年他们没有相见,甚至连书信也没有通上一封。杨老生和大雍的冲突终于白热化了,通过各自的情报网,京无极不时关注着那个女子的一举一动,凭她现在的身份地位,应该也有源源不断的消息送到她案头了吧!不知道她看到他的消息,眼神有没有多停留哪怕一个刹那……

这算不算另一种‘天涯若比邻‘呢?京无极有时候曾经苦笑着想。身为魔门宗主,他带领大批弟子冲锋陷阵,用鲜血换取功名和荣耀;而梵惠瑶在收服诸多豪强之后,却凭着她绝世武功和无双智谋刺杀对方的领军大将。在各自的战场上,他们都在书写自己的传奇和历史。

只可惜,无论是个人的勇武还是阴谋刺杀,都不能改变历史的走向,乱世的车轮悠悠地、然而不可阻挡地向前滚动。梵惠瑶当初的眼光果然准确,此消彼长之下,政制清明的大雍实力果然已在杨老生之上,更何况有武林白道的鼎力相助。

梵惠瑶真是个惊人的女子啊,长于草野,却在乱世中绽放出夺目的光芒,那绝世的武功和智谋宛若天授。京无极微微叹息,看来,自己和她终究是无缘了呢。数年征战,魔门和凤仪门已经成了生死对头,双方弟子循环报复,抹不去的鲜血,已经在两人之间划上了不可逾越的鸿沟。

京无极终于颁布了对大雍重臣名将全面刺杀的命令。虽然杨老生在民心上确实不如大雍,可武林中,他们还是有一搏之力!

刺杀换来的还是刺杀!

那一场魔门弟子和凤仪门的对决极其惨烈。短短半年时间,大雍的精英将领就倒下了三成以上,而杨老生和魔门的损失却更加惨重。梵惠瑶每每亲自出手,她变幻莫测的刺杀手法让人眼花缭乱。京无极每每对着谍报苦笑,他也许应该换一个密谍首领了。眼下这个家伙,呈上来的刺杀过程,文辞华美,荡心动魄,每一次都赫然是一篇传奇!

而在过去的半年中,配合着凤仪门的杀手,大雍军队气势如虹,已经将杨老生逼到了退无可退的地步,若不是还有魔门支撑着,只怕就是兵败如山倒。

但是独木难以擎天,凭着魔门,改变不了整个大局。

京无极沉吟着,不能再放任她这样下去了。民心不可为,大势不可为,但是他身为魔门宗主,保全整个魔门的职责,沉甸甸地压在他身上!

即使不能把魔门引向荣华富贵,至少也要让它不失体面地退出中原,保留元气。

梵惠瑶,一别多年,就让我来看看你变得多强了吧!

7.从头细书

那一日,风和日丽,天气好得让京无极和梵惠瑶都只想叹息。这关系中原武林命运的一战消息早早就放了出去,莲花峰上密密麻麻站满了人,若不是今天的正主儿还要决斗,只怕留的空隙只够他们跳崖了。是啊,谁不想看看京无极这一代宗师和武林第一奇女子梵惠瑶的决斗呢?

午时,如约而至的两人凝视着对方,四目交投,尽在不言中。

这注定是他们最后一次相见。今日一战,不是京无极陨落舍身崖下,就是梵惠瑶一代红颜永归尘土,即使两人侥幸都保全性命,也将有一方永远离开这中原武林。

可惜……梵惠瑶苦笑着望了望四周,人头攒动,无论说什么做什么,都在百十双眼睛注视之下。即使离别已成定局,即使此生不再相逢,苍天竟如此残酷,就连让他们清清静静说句话的机会都不给!

他们能谈的——他们能谈的只有江湖风云天下大势而已!

史载,八月既望,凤仪门主梵惠瑶与魔门宗主京无极相约华山舍身崖,两人纵论天下大势,言语投机,宛若知己。移时,无极长叹曰:‘只是相逢恨晚,今日一战,必要你死我亡,我若身亡,你在中原一日,我魔门不入中原一步。‘相逢恨晚……当真是相逢恨晚?

他们相逢并不算晚,只是他眼睁睁地看着一个个机会从指缝间溜走。

如果在她初出道时,就将她纳入羽翼之下……

如果黑风寨救人之后,把她直接带回了魔门……

如果之后长达几年的书信来往,他曾经说出他的心意……

如果……

但是,如果他这么做了,那京无极就不再是京无极,梵惠瑶也不再是梵惠瑶。

他们终究不是红尘中的寻常男女,身负绝顶武功,承担着各自门派的重任,又身在这乱世……也许,因为他的自尊和她的傲气,他们之间,只适合道左相逢倾盖如故,只适合小酌清谈一笑知心。

也许,相濡以沫,真的不如相忘于江湖。

‘我若身亡,你在中原一日,我魔门不入中原一步。‘既然不能得到,那就放手!痛痛快快地、彻底地一刀斩断!

梵惠瑶从容微笑。相见争如不见,你是这个意思吗,如果你败于我手,即使能侥幸留得性命,也将以中原为界,永不相见?

你何其狠心,既然要斩断,那就一点希望也不留给彼此。

可是,败的只怕是我呢。看那人气度沉凝,眼中神光含而不露,只怕已经跻身天道,这一战,自己的胜算并不大啊……

‘君若不幸,惠瑶也是再无知音,我若身死,凤仪门也会退出江湖。‘她款款微笑。

我会记住你的,无关胜负,无论生死。此生此世,除你之外,我不会接受第二个人。但是……如君所愿,此战之后,我们再不相见。

再不相见。

两人同时拔出兵刃,纵身而上。这一战,将是有资格被整个武林铭记的惊世之战!

8.自挥洒

刀法霸道绚烂,剑招优雅华美。京无极和梵惠瑶都是全力出手,既然这是最后一战,那就不要给自己留下一点遗憾!

不留遗憾么?京无极苦笑。他已经参悟天道,原本梵惠瑶是没有一点希望的,但是看着那张如花容颜,却无论如何狠不下心去取她性命。如果他杀了她,此生才真的会留下不可弥补的遗憾吧。

毕竟……毕竟……

好吧,毕竟自己的目的是要带领魔门退出中原,而不是力挫凤仪门,接着争霸中原。所以,他只要等待一个体面退出的机会就好,不是吗?

只是,为什么你总是用这种以命搏命的打法,胜利真的对你如此重要吗,惠瑶?重要到你宁可付出生命来争取这一线可能的机会?

梵惠瑶哪有这样胡思乱想的余地。京无极的刀法骇人之极,招招狠,刀刀快,若非她全神贯注地催发剑招,早就倒在了京无极刀下。饶是如此,两人武功的差距也不是简单可以弥补的,虽然经常在千钧一发之际逼得他回刀自救,梵惠瑶身上还是添了一处又一处伤痕,白衫上如点点梅花绽放。

只是她性子高傲强韧之极,开头还是七分守势,三分进攻,见势不妙,一咬牙,索性以快打快,一步不让地抢攻。刀光剑气纵横交织,梵惠瑶心中一片空明,只觉得从未有如此酣畅淋漓的一战。无论今日是胜是败,能得如此一战,已经不枉平生!

这一战,已是她此生武学的巅峰!纵然埋骨华山,她也了无遗憾!梵惠瑶心中豪情翻涌,长啸声中,剑光如惊虹掣电,飞舞而出,奇招妙式连绵不绝。

一声轻响,刀光剑气蓦地凝滞。

这神来之笔的一剑,竟然刺穿了京无极的胸膛!

没有人想到,甚至梵惠瑶自己也没有想到,所以,当她看见雪亮的长剑插在京无极胸前时,整个人蓦地呆了一呆。

她伤了他!

她竟伤了他!

那一刻,冰冷的剧痛贯穿她的五脏六腑。过往恩义情缘,一切的一切,终究是被她亲手斩断了。

梵惠瑶身子动了一动,本能地想要解释,想要上前搀扶,想要为他止血疗伤。京无极却自行拔出了长剑,反手点了几处穴道止住鲜血,微笑道:‘梵门主好功夫。今日之后,梵门主在世一日,魔门不入中原!‘说完,他头也不回地扬长而去。

梵惠瑶凝视着他黯淡却依旧孤傲的背影,收敛了一切惊慌痛楚的神色,淡然目送他离开。她是凤仪门主,是刚刚战胜了魔门宗主的中原白道领袖,是新诞生的一代宗师,是众望所归的天下第一剑!

所以,她只有在莲花峰上傲然屹立,飘然如仙子,凛然如神祗!

9.意悠悠

那一日后,两人再也没有见面。梵惠瑶只听说,京无极带着魔门中人遁入北汉,随后只身远走大漠,在在塞外风烟中刀法大成。

据说,数年之后他成了北汉国师。

据说,他的刀法已经精进到天人之境。

据说……

只是,他遵守承诺再也没有跨入中原一步,只是在北汉静静看着,看凤仪门由如日中天一直到夺嫡惨败。

‘师尊,三师兄来报,凤仪门主遭少林十八罗汉阵围杀,战死。‘看完整个事件的前因后果,京无极黯然长叹。

他知道自己胸中应该翻腾着恨意,当世之中,应该没有什么能够拦阻他杀了那个叫做江哲的青年,那个设计逼杀凤仪门主的南楚降臣。但是很奇怪,他升不起一丝一毫想要为梵惠瑶复仇的念头。

非关畏惧大雍对北汉的报复,更不是顾忌自己的身份。

夺嫡争储,一如乱世中逐鹿天下,正是他当年和梵惠瑶说过的,各为其主,生死无恨。在那样的争斗中丧生,又哪里来什么仇恨可言。

要逼杀天下有数的高手,就是他亲自出手也不敢期于必胜,况且是那样一个文弱多病的青年。何其难得。

这样的人就应当依靠智慧堂堂正正地击败,又岂是可以用阴谋暗杀轻辱。

直到和江哲道左相逢,看他青衫赤足迤逦而来,在他面前侃侃而谈,洒脱不羁,如此气度风华,当世罕有。京无极暗暗叹息,惠瑶,惠瑶,死在这样的人手里,也不负你一世英名。

多年执念终于放下,他知道,从今日以后,梵惠瑶这个名字在他心里,再也不会激起一丝苦涩心酸。

仰望长空,天高云淡。
