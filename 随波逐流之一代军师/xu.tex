\part{绪章}

“真是好画,烟波浩淼,孤舟寂寞,笔法非凡。”我淡然的点头赞赏,毕竟身份摆在这里,总不能太过失态,对收到的礼物若是表示出欣喜若狂,那就必须得给人办事不是,像我这种身份地位,有些事情举手之劳的可以帮个忙,有些事情么,还是袖手旁观的好,虽然陛下现在还是挺英明的,但是总要想到,他已经七十多岁了,听说明年就要传位给皇太孙了,万一他年老糊涂,对我这样的老臣怀疑起来怎么办,我可是想善始善终得到一个好的结局呢。送礼的中年人刘祯见我这样的神色,眼中闪过一丝忧虑,小心翼翼地说道:“老公爷,小侄的父亲年老糊涂,不该胡乱写书,求公爷念在当年份属同年,又曾同殿为臣的份上,给家父说上几句好话,让他老人家得以安享晚年吧。”

“是么,文举兄写了什么书么,快给我看看,我可是很喜欢文举兄的文笔呢?”我来了兴致,当年我和他的老子刘魁刘文举是一起中的进士,我是状元,他是榜眼,不过说句实话,我可是很佩服他的文章,文字严谨,史据翔实,若非他个性太执拗,说什么也不肯侍奉二主,本朝的史官一职绝对是他的囊中之物,前阵子听说他正在写《南朝楚史》,我是翘首以待啊,可是最近却没了消息。刘祯的脸上露出尴尬的神色,从怀中掏出一个布包递了过来。我打开一看,淡青色的封面上写着“南朝楚史”四个大字,我兴奋的打开读了起来,完全忘了屋里还有外人在。等我一目十行的读完之后,不由苦笑起来,文举兄可真是不给我留情面啊。懒洋洋的放下书本,漫声道:“贤侄,你先回去吧,这件事我得端详端详,你是知道的,老夫已经多年没有过问政事了。”

送走了刘祯,我大声叫道:“小顺子,小顺子。”随着我的呼唤,从门外走进一个青衫老者,看上去大约四十多岁的年纪,相貌清秀,面白无须,这人正是跟了我五十多年的亲信随从李顺,他曾是南楚宫中的宦官,武艺绝顶,据说已至宗师级别,为什么说是据说,当然是因为我不大懂武功上的事情,不过看他明明已经六十出头,看上去却是中年人的模样,应该是真的吧。以前有人不相信李顺这样的高手会对我这么一个手无缚鸡之力的书生忠心耿耿,曾经想收买他,不过下场之惨我就不说了,免得你听得吃不下饭。我苦笑着问道:“刘魁是南楚遗臣,他说些过分的话也没什么,怎么朝中那些大臣那么看重呢?”李顺笑道:“老爷想是忘了,明年皇太孙就要继位了,太子妃是您的长女,这当口谁不想讨好您呢,偏偏刘魁那么执拗,非把您老放到贰臣录里面,就是您不计较,太子妃和皇太孙的体面也得维护。”

“是啊!”我恍然大悟,别看刘魁在《南朝楚史》里面说我是“阴柔诡谲、心机深沉”,可谁知道我是一个对政治不大敏感的人,如果不是小顺子的提点和我的明哲保身,只怕早就覆顶了。想到这里,我淡淡道:“你去跟柔蓝说一声,刘魁是南楚遗臣硕果仅存的了,何必为难他呢,有些事情就是他不说别人也会说的,他给我写的《江随云传》虽然有些尖刻,但是总算还是符合事实的,他写了免得别人乱写,再说,我的事情也连累不到皇太孙身上,叫她不必多事了。”小顺子恭恭敬敬的的退下了。

我则是兴致勃勃的打开《江随云传》重新看了起来,虽然我还没有盖棺论定,但是提前看看也没关系吧。

显德十六年丁卯,国主胜微恙,至秋,病愈,开恩科,江南士子雀跃,从者如流,八月十五日,金榜出,状元者,嘉兴江哲是也,其时随云名尚未显,众相诘问,乃知其人。

江哲,字随云,生于同元四年戊申,其父江暮,字寒秋,寒秋少年家贫,然文雅风流,故世家妻以爱女,寒秋以乱世不可进取,故不肯出仕,终日唯教子读书,显德八年己未,嘉兴瘟疫,其妻病逝,未几,寒秋因细故与妻族绝,扶病携子远游,至江夏,寒秋疾甚,随云为之延医,逢医圣桑臣,桑臣爱随云博闻强记,乃倾囊相授,未几,寒秋渐愈,桑臣赴江北,随云侍奉汤药,滞留江夏,显德十一年壬戌,寒秋病故,有《清远集》十二卷传世,典雅清新,今人颇爱之。

寒秋殁,随云贫而不能葬,时镇远侯陆守江夏,为子求师,随云往见,陆侯见其年幼,故难之,命其为文,随云笔下千言,片刻而成《秋水赋》,其中有“少焉,月出于东山之上,徘徊于斗牛之间。白露横江,水光接天。纵一苇之所如,凌万顷之茫然。浩浩乎如冯虚御风,而不知其所止;飘飘乎如遗世独立,羽化而登仙。”之句,陆侯惊甚,起而谢之,命世子出,拜师求教。

--《南朝楚史·江随云传》
