\part{第二部 萧墙之乱}

\chapter{第一章 凤仪传奇}

放下手上的书卷,我不由惊叹出声,这本册子上面是雍王府所能够收集到的所有关于凤仪门主的情报,编撰之人文笔生动,仿佛就是一本传奇。

凤仪门主,出身不明,四岁被原凤仪门主收养,其时凤仪门不过是一个一些孤苦女子组织起来自保的小门派,武功也不过尔尔,而凤仪门主梵惠瑶乃是天纵之才,竟然凭着一本残破的太阴心经练成了绝世武功,年仅二十岁就在江湖上崭露头角,更难得是,她虽是女子之身,为人却是任侠仗义,不过数年,江湖上就将这个总是身穿白衫,气度高洁而相貌秀丽如仙的女子列入绝顶高手的行列。

虽然如此,凤仪门主的美貌纤弱仍然引动了无数狂蜂浪蝶,这个出色的女子没有强硬的拒绝,也没有四处逢迎,而是明言终生不嫁,把无数爱慕她才色的俊杰变成了知己,当然她也用过雷霆手段,曾经黑山寨的少寨主以梵惠瑶的养母兼恩师为人质,不择手段的逼她下嫁,当时的黑山寨是黑道第一大帮,威势震动天下。而凤仪门主慨然应诺,在婚宴之上,宾客之前,身穿大红喜服的梵惠瑶突然发难,剑气如虹,斩杀了新郎,黑山寨主大怒,命令手下将梵惠瑶当场砍成肉泥,而除去吉服一身素衣的梵惠瑶大开杀戒,她手创的疾风剑法名扬天下,在喜堂之上,千人重围当中,那超越人体极限的快剑肆无忌惮的收割着人命,满天都是青色的剑芒,雪白的倩影在这残酷的搏杀中却带着优雅和华贵,这一役,黑山寨总寨四十八名护法,死了大半,一百零八处分寨寨主死了四成,最后,梵惠瑶身剑合一,冲出了重围,而在此之前,她的养母已经被人趁乱救走,后来看到过梵惠瑶的人都说当时她白衫尽被血染,身上大小伤势三十多处,能够逃生真是侥天之幸,而更令人惊奇的是,梵惠瑶在养伤期间邀约天下群雄会盟,共讨黑山寨,趁着黑山寨势力大损,各路豪强落井下石,在梵惠瑶居中调节下,一度曾经风云显赫的黑山寨成了过眼云烟。

黑山寨覆灭之后,梵惠瑶正式成为凤仪门主,在她的英明领导下,凤仪门很快就成了白道翘楚,而梵惠瑶更是纵横天下,一剑光寒,当时东晋崩溃已经三十年,中原纷乱,梵惠瑶虽然行侠仗义,救济贫民,但是一人之力如何能够挽回滔天风浪,在看尽苍生苦难之后,梵惠瑶立誓要令天下一统,当时人人笑她大言不惭,一个女子,就是再有本事能力,也不可能一统天下。而明知确是如此的梵惠瑶选择了一条最容易也是最艰难的道路,她选择了支持李援,这个中原势力并非最大但是政治清明的诸侯,凭着凤仪门在白道上领袖地位,凭着自己纵横捭阖的才干,凭着她绝世的武功,凤仪门为大雍的立国建立了汗马功劳。

为了大雍,梵惠瑶走遍中原,为李援争取了很多世家豪强的支持,为了大雍,梵惠瑶曾经多次刺杀敌人大将重臣,曾有一次,梵惠瑶在敌军首领陪同妻子到佛寺进香的时候,她一身素衣,赤足高髦,手拈柳枝扮成了观音菩萨,在数百名高手护卫搜查大殿的时候,没有一个人发觉那莲花宝座上宝相庄严的观音竟是一个女子装扮,就在那名敌将入殿下拜之时,她一指击杀了敌将,然后飘然如仙子一般走出大殿,外面的守卫目瞪口呆,眼看着她迤逦而去,素足踏在雪地之上,没有一丝痕迹,也没有一丝雪泥可以沾染她如玉肌肤,数千精兵骇然惊呼‘观音娘娘显圣‘,而让她安然离去。

还有一次,雍王李贽领军和杨老生作战,杨老生麾下有一员猛将温虎,手中大戟,所向披靡,斩将夺旗,悍勇绝伦,人称赛吕布,雍王麾下没有可以匹敌的将领,数万大军被一万敌军死死缠住,梵惠瑶恰好亲自护送粮草到了军中,得知此事之后,她含笑而去,当夜,杨老生的使者突然到了温虎的大营,声言传令,温虎对杨老生十分忠诚,亲自前去迎接使者,谁知那名使者手持军令,高声宣道:‘温虎通敌,罪在不赦,本使者奉旨阵斩之。‘说罢,拔出佩剑,那一剑睥睨天下,傲视群伦,将促不及防的温虎斩于马下。敌军大乱,梵惠瑶趁机飘然离去,第二天雍王趁势进攻,尽歼敌军。

梵惠瑶最惨烈的一战就是和魔门宗主京无极的决斗,魔门扶持杨老生,想要一统中原,凤仪门和魔门成了生死对头,魔门中人手段毒辣,刺杀投毒无所不用其极,而梵惠瑶的凤仪门不免势力差了一些,为了保护大雍的君臣猛将,梵惠瑶说服了少林寺的方丈,建立了完善的防护,她自己则开始清剿魔门的杀手暗探,这是一场史无前例的决战,双方互相刺杀,在短短的半年之中,大雍损失了三成精英将领,但是敌人的损失更加惨重,梵惠瑶的才能显示的淋漓尽致,各种各样的刺杀方式让人眼花缭乱,后来,京无极终于忍受不住惨重的损失,下书约梵惠瑶华山舍身崖决战。

那一日风和日丽,莲花峰上群雄聚集,谁不想看看京无极这一代宗师和武林第一奇女子梵惠瑶的决斗,到了午时,两人如约而至,京无极一身蓝衫,相貌儒雅英俊,梵惠瑶一身雪衣,风华绝代,二人在群雄面前款款相谈,谈论天下大势,话语投机,仿佛知己好友,谁知两人却是生死对头呢?

两人相谈过了半个时辰,京无极长叹一声道:‘只是相逢恨晚,今日一战,必要你死我亡,我若身亡,你在中原一日,我魔门不入中原一步。‘

凤仪门主也是淡淡一笑,道:‘君若不幸,惠瑶也是再无知音,我若身死,凤仪门也会退出江湖。‘

两人这一战可是惊天动地,京无极乃是魔门宗主,刀法绚烂霸道,快如电,疾如风,攻掠如龙,飘逸如神,梵惠瑶的剑法却是优雅华美,似乎不带一点杀气,双方激战之下,京无极的刀法固然令人瞠目结舌,但是梵惠瑶的剑法也是精妙绝伦,只是梵惠瑶毕竟稍逊一筹,苦战之中,受伤无数,若非她以命博命,只怕早就落败了,但是到了千招之后,梵惠瑶却是越战越勇,她的全部才智都被这个强大的对手迫了出来,只见她一声长啸如凤鸣九天,长剑越来越快,青芒如浪,一浪高似一浪,十余招后,梵惠瑶手中长剑化作长虹破空穿浪而来,剑招奇幻瑰丽,美不胜收,一剑刺穿了京无极的胸口,京无极惨败当场,黯然离去。

当时,梵惠瑶临风而立,她一身雪衣,上面点点鲜血似红梅绽放,身材修长,长眉入鬓,凤目湛然,飘然如仙子,凛然如神祗,这一战让她成为天下第一剑,位列宗师,也让她成了白道的精神领袖,声名更在另一位武林宗师少林寺慈真长老之上。京无极则遁身北汉,远去草原,在塞外风烟中刀法大成,数年之后他成了北汉国师,据说他的刀法已经精进到天人之境,只是他遵守承诺再也没有跨入中原一步。

如果没有梵惠瑶,大雍一统中原必然要多花十年时间。在她的鼓励和引导下,很多江湖黑白两道的高手都投入到大雍军中效力,而在征战之中,凤仪门的权力也飞速膨胀。

更难得是,梵惠瑶有惊世绝艳之才,她曾经数次参与军政,都有令人震惊的表现,因此李援曾经让自己的几个儿子拜她为师,虽然梵惠瑶声称自己不收男弟子,但也仍然不时提点指导,令他们受益匪浅。这令梵惠瑶的势力开始介入大雍皇族。

在中原略为平定之后,李援曾向梵惠瑶求婚,但是聪明的梵惠瑶拒绝了,她声言凤仪门主必须终生不嫁,这就维持了她超然的立场,但私下里她派遣凤仪门弟子纪霞贴身服侍李援,不久之后,李援立了续弦窦氏为皇后,而纪霞成了贵妃,在大雍的统治渐渐稳定之后,梵惠瑶返回凤仪门清修,不再过问世事,但是她的潜势力却是越来越大。

梵惠瑶在接掌凤仪门之后,重新建立了制度门规,她规定,凤仪门分为内外堂,内堂分为春江堂、金蕊堂、寒霜堂三堂,春江堂是凤仪门的权力核心,堂中弟子都是可以独当一面的干才,可以调动所有人员,没有固定的权限,金蕊堂执掌刑罚升迁,取秋风萧杀的含义,寒霜堂负责征战讨伐,堂中弟子都是武功惊绝,冷酷无情的杀手型人物。内堂弟子只有立誓终身不嫁,誓死效忠凤仪门的资质超凡的女子才可以加入。外堂则包括凤仪门在各地的分舵成员,还有就是梵惠瑶为了扩大势力而收取的记名弟子。但是也只有女子可以加入。不过凤仪门内部十分严密,一个弟子是外堂弟子还是内堂弟子,很难明了,只有一个女弟子嫁了人,才知道她一定是外堂弟子。

梵惠瑶的手段十分巧妙,她首先凭着和大雍朝廷的亲密关系,收了很多朝臣的女儿为徒,大雍尚武,那些朝臣也喜欢女儿练练武功,凤仪门和皇室关系密切,又都是女子,所以梵惠瑶十分顺利的收到了一批官宦千金,她在其中确实选了一些人才,甚至有些女子崇尚凤仪门的威望,宁愿终身不嫁加入内堂,另外她通过和朝臣的关系,将自己收养的一批姿容才貌不俗的女弟子嫁入了豪门,这些女弟子虽然出身各异,但是在凤仪门主的教导之下都成了品貌超人,文武双全的女子,所以大雍朝臣颇以子侄娶到凤仪门的外堂记名弟子为荣。这样一来,梵惠瑶的凤仪门和大雍朝廷结成了盘根错节的亲密关系,若是梵惠瑶有心,足可以撼动大雍的社稷。

不说别人,雍帝的贵妃纪氏,是梵惠瑶的师妹,屡屡参与军国大事,太子侧妃萧兰,美艳脱俗,虽然不理会军政,但却是太子李安的宠妃,宠爱胜过太子妃,若非太子妃出身名门,又早早生了世子,只怕正室之位难保。齐王妃秦铮,才华过人,本来已经要进入凤仪门内堂,但是老父因为只有一女,苦苦相求凤仪门主,梵惠瑶才拒绝了她的请求,后来又得到齐王倾心,聘为妃子。而且梵惠瑶曾经想把爱徒梁婉许配给雍王李贽做侧妃,但是李贽婉言拒绝,据说是因为李贽和王妃高氏感情很好,李贽常常出征,高氏不仅持家严谨,而且尊重李贽麾下的谋士将领,是李援都几度称赞的好儿媳、贤内助,李贽的另外两个侧妃都是高氏的陪嫁侍女,两人相敬如宾,是大雍的佳话,要不然凤仪门可就一网打尽了。

如果说凤仪门主有什么不如意的就是:

其一,纪贵妃虽然得到雍帝信任,可是宠爱上倒是不如皇后和其他几位贵妃,皇后窦氏贤良淑德,又是太子生母,所以母仪天下,后位稳固,长孙贵妃虽然失去了皇二子和皇四子,但是还有长乐公主,雍帝因为歉疚对长孙贵妃几乎百依百顺,颜贵妃是齐王生母,性情开朗大方,在宫妃之中最受帝宠,纪贵妃论旧情不如皇后和两位贵妃,论容貌年轻,又不如雍帝数次选美选进来的新人,所以虽然得到雍帝信任,宠爱却差了一些,而且至今没有子嗣,也是一件憾事。

其二,太子侧妃萧兰虽然得到太子宠爱,又生了皇孙,但是太子倒是对世子十分宠爱,完全没有偏爱宠妃之子的意思,看来就是李安登了帝位,也只会立正妃之子为储君。

其三,就是齐王虽然大婚,可是风流放荡依旧,对秦铮虽然不错,但是金屋藏娇却是不弱于从前,已经有好几个庶子出生,秦铮虽然不满,却也无可奈何。

凤仪门也不是没有反对势力的,皇三子李康,他出身卑微,母亲原是宫女,封为宜嫔,虽然不受宠爱,但是宜嫔性子柔顺,也不争宠,只是一心抚养爱子,希望等到爱子封王之后,可以出宫到儿子的王府享受天伦之乐,可是一次李援回京时,召集后宫妃妾宴饮,突然遭到魔宗行刺,纪贵妃为了保护李援,竟然把宜嫔推到了刺客的刀前,李援虽然安全无恙,但是宜嫔却香消玉陨,虽然李援加封宜嫔为妃,厚礼安葬,但是李康愤恨不平,要求纪贵妃偿命不成之后,愤然出走,数年之后回来,却练了一身好武功,曾经当众行刺纪贵妃不遂,李援念他丧母之痛,没有怪责,封他庆王,让他到东川镇守,李康虽然遭到贬斥,但是不改行止,在他的领地里面,对凤仪门十分排斥,但是李援的同情加上雍王的暗中相助,让李康在东川坐得十分安稳。

还有李援的外甥姜永,李援的姐姐宁华长公主嫁给了一方诸侯姜无涯,后来双方征战,姜无涯被凤仪门的刺客刺杀,长公主自杀,姜永愤然和大雍作战,却落得一个兵败徐州,最后姜永带着仅剩的一些旧部远走东海,成了有名的海寇,屡屡侵犯海疆,骚扰大雍的商船。李援开始还同情这个外甥,顾念姐姐只有这点血脉,想招降他,后来姜永悍然斩杀李援的使者,李援这才大怒,几次下令讨伐,都因为大海茫茫,没能成功,李永曾经多次袭击凤仪门的商船,凤仪门虽然也想对付他,但是无奈李永是天生的水军统领,凤仪门找不到他的踪迹,而且也不便真的出手杀他,毕竟李援少年时曾经受过姐姐的教导照顾,对长公主十分敬重,后来却杀了姐夫,逼死了姐姐,所以对这个外甥更是愧疚,虽然下令讨伐,却还是要求活捉。

凤仪门和一个皇子,一个皇帝亲外甥之间的仇恨大概是凤仪门主心中最大的隐痛了,可是这并不代表他们可以撼动凤仪门的势力,凤仪门若非投鼠忌器,他们只怕早就丧命了,即使这样,庆王李康如今也只能在东川肆意妄为,而且因为他对凤仪门的排斥,导致雍帝李援在东川另外安插了一支军队,对庆王进行监视和约束,凤仪门曾经设下圈套诱使姜永入伏,若非姜永的属下誓死断后,姜永只怕早就被生擒了,所以这些时日姜永已经销声匿迹了。

看完凤仪门的情报,我心中又是震惊又是激动,这样一个可怕而强大的组合,就是我一定要对付的强敌,这个凤仪门主,确实有惊世绝艳之才,看她的行径,虽然似乎很冒险,但是根据她宗师的身份武功判断,实际上倒是如屡平地,谋定后动,既有才华,又精于谋划,怪不得雍王被她压得喘不过气来,不过另外一种喜悦也从心头涌起,如果将这样的强敌逼如绝境,应该是我一生中最大的骄傲吧。如果说我归顺雍王只为了感激他的恩宠,那么如今我的目标就是铲除凤仪门,如果不是凤仪门教出来的骄纵弟子,飘香又怎会死,梁婉,梁婉,你还不足以抵偿我爱妻的性命啊。

\chapter{第二章 献君三策}

整理好思路,我吩咐那个叫李信的书童去通知一声,等到雍王殿下起床之后,我要见他。谁知道没过片刻,我就看见雍王和石彧匆匆走了进来,而且身上衣着整齐,神色略带倦容,根本就是一夜没睡的样子。我先是愣了一下,然后又恍然大悟,看来雍王等得很着急呢?

请雍王坐下,先随便聊了几句,看雍王已经神色安定下来,我这才道:‘请问殿下,为什么定要登上皇位?‘

雍王一愣,他心里早就将登上皇位当成是自己必须得去做的事情,原因除了认为自己应该得到这样的报偿之外,就是觉得除了自己没有人能够令大雍一统天下,为了大雍社稷自己必须不顾毁誉,但是江哲这样问起来,他却突然觉得难以回答,平日里他和属下都将此事看作理所当然,反而不知该如何向江哲解释。

我微微一笑,这是我早就发现的事情,雍王劝我归顺的时候,完全没有解释过自己的理念,这只能说明雍王自己的心志并不明确,立场没有坚定,万丈高楼平地而起,如果没有这样的理念作为基础,那么雍王的大业终究是水上楼阁。

我继续说道:‘依现在情况来看,大雍基业已经颇为稳固,太子殿下占了嫡长之位,又没有明显的失德,那么文武百官何必定要违逆皇上的心意而支持殿下呢,殿下掌握军权,一呼百应,若是强行夺位,不免遭制物议,说殿下谋逆篡位,殿下固然英明神武,但若为后世子孙留下错误的例子,认为只要有了权力功绩,就可以登基为帝,那么谋反就可以名正言顺的进行,君权遭到置疑,一个稳定的制度的作用胜过一个圣明君主,所以说殿下功绩虽然盖世,但是却不可以成为殿下夺取皇位的理由。‘

李贽若有所思地道:‘这大概就是朝中元老虽然看重我的功绩,却不肯主动支持我继位的理由吧?‘

我点点头道:‘我若是那些元老,看着大雍从无到有,必然不希望因为内乱消减大雍的力量,所以他们不可能支持殿下继位,即使殿下是他们心目中比较好的储君人选,他们也不会断然支持,这时愿意支持殿下的人大多为的是日后的荣华富贵,只有少数人才能看透只有殿下登基,才能保住大雍社稷,那些平常人既不了解殿下继位的重要性,那么殿下应该告诉他们。‘

李贽听了我这番话,问道:‘说句实话,本王只是觉得不能将皇位交给皇兄,他和凤仪门太接近了,但是理由还说不大清楚。‘

我正容道:‘所以臣献给殿下的第一策就是明志策,如今大雍虽然欣欣向荣,但是内里却是隐忧重重,这个隐忧就是凤仪门,大雍之忧,不在四方强敌,而在萧墙之内,凤仪门以仁义为外裳,以权谋为内里,掌控后宫,下制百官,长此以往,凤仪门迟早成了垂帘听政的太上皇,太子殿下身为储君,不知修德,不以恭顺贤孝收敛百官之心,而和凤仪门勾结紧密,以求稳固储位,殿下若是任由太子登基,不仅自身性命难保,自毁国之栋梁,还会让大雍社稷被妇人掌控,若是太子殿下屏除身边佞臣,断绝和凤仪门的来往,就是太子殿下没有一兵一卒可以防身,殿下您也不敢加一指于储君,此是太子失德在先,并非殿下存心谋逆。更何况说句诛心的话,天下非是一家一姓的天下,若是凤仪门真是好选择,那么臣也未必要殿下将其铲除,但是以臣看来,凤仪门弟子高傲骄纵,不知天下疾苦,一心只是争权夺利,臣虽楚人,但是梁婉在南楚多年,臣对其颇有了解,这样一批目光短浅,不知轻重,骄纵自大的女子若是掌握了权力,只怕天下百姓都要为之受累,或者当初凤仪门主确是为国为民,但是如今凤仪门已经蜕化成夺取权力的工具,殿下若不能铲除凤仪门的势力,只怕大雍不仅不能统一天下,还会沦亡在妇人之手。殿下身为大雍皇子,焉能见社稷沦亡,百姓受苦,既然太子殿下昧于权力的诱惑,不能善尽储君的职责,那么殿下取而代之自然是理所当然的事情。‘

听到这里,李贽眉飞色舞地道:‘先生真是说穿了本王的心思,本王也是这样想的,只是从没有这样清晰明了,不错,若非凤仪门的存在,我就是作一个安闲的王爷又有什么不好。‘

我微微一笑,没有去计较李贽话中的虚实,反正那并不重要。

我淡淡道:‘明志一策可以令殿下坚定心志,请容臣先为殿下阐述当前局势,现在殿下之所以觉得四面楚歌,就是因为皇上、太子、齐王、凤仪门之间的密切关系,让殿下无从着手,但是在臣看来,首先,他们并非浑然一体,皇上、太子、齐王并非殿下想象的那样对凤仪门毫无防备,只是因为各自的私心才纵容凤仪门的存在,皇上若是没有心存忌惮,那么纪妃不会没有子嗣,这些年来,皇上后宫颇有爱宠,生了十几个庶皇子公主,说明皇上身体康健,但是纪妃却没有子嗣,我想皇上也不想纪妃有了皇子之后,争夺储位吧。太子虽然宠爱侧妃,可是对世子却十分爱护,俗话说,母以子贵,子以母贵,自古以来因为宠爱妾妃而杀妻灭子的不在少数,若非太子殿下对凤仪门也有忌惮,恐怕世子早就失宠了,还有齐王殿下,殿下虽然娶了妃子,却对她若即若离,我曾见过齐王,从面相上看,齐王虽然秉性风流,但是这样子冷落嫡妃还是有些古怪,所以皇上他们并非对凤仪门十分信任,只是如果没有凤仪门,他们就没有和殿下对抗的本钱了。‘

‘其次,皇上虽然偏爱太子,但是若是太子危害到社稷,皇上就是再偏爱也不会姑息太子,所以这些年来雍王殿下虽然屡遭凶险,还是稳如泰山,因为殿下是大雍擎天栋梁,皇上绝不会任由太子伤害殿下,只要殿下没有触犯皇上的底线,那么殿下的安全就是有保证的,只要殿下除去了太子,就是皇上再生气愤怒,也只能够将帝位传给殿下,所以殿下必须在皇上在位的时候控制全部权力,那么殿下就可以名正言顺的即位。‘

‘最后,太子倚重齐王,齐王支持太子,殿下或许以为他们是不可分割的联盟,但是以臣看来未必没有嫌隙,从殿下收集的情报来看,太子不是一个能够容人的人,齐王个性飞扬跋扈,就是在太子面前也常常有所显露,只是为了对付殿下,太子才笼络齐王,臣从情报中得知,太子曾经因为齐王的战败无功而对齐王冷落多日,只是近日因为殿下的缘故才又开始对齐王示好。齐王殿下心如明镜,怎么会不知道太子的薄情寡义,只是齐王却是不得已,因为殿下自己就是领兵作战的将帅,所以在齐王看来,如果殿下登基,那么他就再没有发挥所长的余地,其实这一点臣要面谏殿下,所谓千金之子,坐不垂堂,殿下已经军功显赫,理应培植将帅,何必要去和属下争夺功劳,更何况,殿下将来是要统治天下的,总不能只是关注军事,若是没有人可以代殿下征讨四方,难道还要殿下去亲征么?‘

说到这里,我看到李贽有些赧然的看向石彧,石彧则是满脸的赞同,看来他也曾经这样进谏过。

顿了一下,看李贽已经露出同意的神色,我继续道:‘臣已经为殿下说明局势,那么殿下请听臣说明第二策--剜心策,当前殿下虽然危急,但是敌方仍有嫌隙,以臣看来,殿下的敌人组成的联盟最大的弱点就在于太子殿下,因为太子殿下不能犯错误,否则皇上必然置疑自己的决定,齐王必定忧虑自己的前途,而凤仪门也失去了对抗殿下的依据,所以只要太子犯错,那么殿下就可以让那个联盟分崩离析,但是太子殿下不是蠢人,身边又有谋士劝谏,想要让太子犯错并不容易,所以我们必须从两方面着手,其一,就是在太子身边安插一个我们的人,这个人必须能够得到太子的信任,让太子对他言听计从,其二,殿下必须让太子占据表面的优势,这样太子才会得意忘形,自毁长城。‘

李贽皱眉道:‘我们示弱倒还有法子,可是怎样在太子身边插入这样一个人呢,太子对这件事情还是很留心的,我们虽然在太子身边有几个人,但是都不能参与机要。‘

我轻笑道:‘臣既然说了出来,自然有法子,只要殿下能够提供一个合适的人选,臣自然能够让太子信任他,甚至百依百顺。这个人必须善于讨好太子,又必须能够替太子解决疑难,总之他必须有能够取代太子的智囊鲁敬忠的地位的才能,在臣的策划下,这个人就会成了太子时刻不能离开的宠臣心腹,而殿下就可以操纵太子,太子若在掌握之中,殿下就可以清宇内,震朝纲,何愁不能继承大统。‘

李贽神色又是震惊,又是迷惑,想了一想道:‘控制太子,谈何容易,不说鲁敬忠不可轻乎,就是凤仪门也不会让我们轻易成功。‘

我笑道:‘臣说控制太子,并非是控制太子的生死,而是控制太子的思想,只要让太子按照我们的计划行动,不管太子本来想什么都与殿下无关,殿下放心,臣已经有了可行的计划,虽然中途难免会有些波折,但只要我们目的达到,就可保殿下安全无忧了。‘

李贽道:‘细节我们以后再说,先生既然有把握,那么李贽就放心了,但是控制太子之后,我们要做些什么?‘

我笑道:‘也不做些什么,只是让太子猜忌齐王,太子这种人,本性狐疑,今日嫉妒殿下的功业,他日也难免嫉妒齐王,臣只是让这样的事情提早发生,只要太子自以为已经压制了殿下,那么自然就会原形毕露,我再安排引诱太子做些嚣张的错事,不用两年,太子就会成为天下人眼中的暴君昏君,太子失德,还有人可以和殿下争储么,到时候凤仪门一定十分为难,到时臣和殿下再仔细商议,总要让凤仪门不能再左右朝政就是。‘

李贽听得眉飞色舞,心想,江哲果然才略过人,我怎么就没有想到敌人最强之处就是弱点,太子本是他们联盟的核心,若是太子出了问题,那么他们的联盟自然就会崩溃,虽然还不知道具体的计划,但是李贽已经是一扫心中愁闷烦忧。他站起身,躬身一礼道:‘听君一席话,胜读十年书,贽多谢先生教诲。‘

我起身还礼道:‘殿下过誉了,还请殿下听臣的第三策--纳贤策,殿下虽然素有贤名,麾下文武多人,但是以臣看来,仍然有些不足之处,殿下既然有志天下,那么就要考虑到如何治理朝政,如今朝中百官和凤仪门多有牵连,若是殿下即位之后,还是任用这些人,那么就不免让凤仪门有死灰复燃的可能。‘

李贽皱眉道:‘我也知道这一点,可是若是骤然更换,只怕朝野动荡,豪门反叛,我大雍顷刻间就要亡国了。‘

我淡淡道:‘殿下将文武百官看的那么重要,却忽略了军心民心,这些年来,殿下攻无不克,战无不胜,屡次为百姓张目,天下谁不知晓,可是大雍建国之初,依赖了不少地方豪强,当时这种做法固然加快了一统的进程,但是如今这些豪强侵占民田,不交纳税收,据在下所知,很多平民失去田地,不得不依附世家豪强为奴,天下人无不恨豪强入骨,可是殿下以强兵为由,允许平民开荒种田,田地名义上归属军队,实际上归百姓所有,所以不少青壮男子都愿意从军,好让家人可以得到田地,这也是朝中豪强倾向太子的一个原因,若是殿下姑息这种局面,终有一日,大雍会陷入诸侯割据的局面,不如趁着现在争储之时,让这些豪强卷入其中,殿下以此为由,清洗天下豪门,任用寒门贤才,重建大雍,只要殿下计划得宜,这些豪门万万没有机会谋叛,虽然这样一来短期内大雍不免削弱,但是只要数年时间,就可以让大雍脱胎换骨,成为真正的第一强国,到时候平南楚,灭北汉,逐北蛮,易如反掌。‘

李贽听得入神,这些弊端他也知道,只是屡次想提出改革,却都被压制,这也是他想得到帝位的一个原因,原本他想即位之后慢慢设法,江哲的这个打算虽然狠辣,但是却可以不伤害大雍的筋骨,毕竟皇位争夺,牵连十几万人的事情不是没有,只要自己做的巧妙,就可以清除大部分豪门,再将自己的人才补充到朝中,十年之内,就可以让大雍再不受豪门控制。

想明白之后,李贽再次起身施礼道:‘前面两策,虽然可以让本王登上帝位,本王只是钦佩,这一策却可以让大雍社稷安康,本王代我大雍皇室、天下百姓,拜谢先生。‘

我起身还礼道:‘殿下肯听从臣的狂言,应该是臣代天下百姓谢过殿下,臣本庶民,多知民间之苦,殿下肯替百姓张目,是万民的大幸。‘

坐下之后,我道:‘铲除豪门只是这一策的一部分,若是没有贤才辅佐,朝堂一空,殿下如何治理天下,所以殿下要广纳贤才,治理天下,如果担心皇上和太子的疑忌,殿下可以向皇上要求领地,到时候殿下在自己的领地之内任用贤才,储备人才,等到殿下登基之后,就可以让他们全面接管政务。‘

李贽道:‘本王一直征战在外,虽然父皇将幽州给我做封地,幽州总管裴济是本王心腹,将领地管理的井井有条,但是培植人才,恐怕非其所长,先生看应该如何处理。‘

我笑道:‘殿下担心若是撤换裴济,伤害属下之心,其实不用过虑,殿下可以启奏陛下,让世子到幽州镇守,然后就可以派石先生辅佐世子,石先生是帅府长史,殿下可以提升裴济的职务,最好把裴济调回殿下身边,然后,石先生就可以为殿下招贤纳士,殿下见了贤才,留在身边还容易遭到猜忌,不如将他们暂时送到幽州,让他们熟悉政务,当然石先生要好好指导,让他们将来可以立刻接手朝政,到时候殿下一声令下,他们就可以入京为官了。‘

李贽强忍心中的激动,虽然对江哲的归顺十分高兴,但是前些日子江哲的试探还是让他不免有些嫌隙的,如今明志、剜心、纳贤三策,却让他觉得前些日子的一切苦痛都得到了回报,若非自己虔诚礼敬贤士,如何能够听到这样的策谋。他尽力平静地道:‘既然如此,我将一切托付先生,子攸便到幽州为本王建立根基,此事事关重大,除了子攸无人可为在下分忧。‘

石彧自然明白自己的责任重大,但是他有些担忧,若是这样一来,将来新君的朝臣几乎都是自己的门生弟子了,那么自己未免权柄过大,他有些忧虑的看了江哲一眼,毕竟不好对雍王明言。

我早有准备道:‘殿下,石先生责任重大,世子无人照管管教,不如殿下再选贤能,负责辅佐教导世子,这样石先生也可以轻松一些,也免得耽误了世子的学业。‘

李贽想了一想道:‘这样吧,世子的舅父高融精明强干,太傅褚平之子褚文远品德端正,才华过人,可以辅佐世子,这样一来,子攸就可以专心纳贤之事。‘

石彧这才放心下来,道:‘子攸必然尽心竭力,请殿下放心,一旦殿下令旨到了幽州,子攸必定星夜来归,京中事务,全部托付随云,还请随云费心。‘

我也郑重道:‘石先生放心,随云既然定下谋略,就一定会办到,否则不仅对不起殿下,更加对不起先生在幽州的苦心孤诣。‘

李贽笑道:‘好了,我们谈了这么久,本王觉得饥肠辘辘了,不如我们先去用饭,然后两位先生好好休息一下,否则累坏了两位,谁给本王出谋划策呢?有了子攸,本王没有后顾之忧,有了随云,本王不必再畏惧那些魑魅魍魉了。‘

我摇头道:‘殿下说得不对,‘看看李贽和石彧惊讶的神色道:‘殿下光风霁月,何曾惧怕那些小人,只是无计扫除污秽罢了,臣不过是有些阴谋诡计,君子不能对付的,臣可以做到罢了。‘

李贽看向我一脸诚挚,心中感动,却是不知道该说些什么才好。嘻嘻,想来他不会猜到我虽然有部分是真话,但是还有一些不过是奉承,李贽若是纯粹的君子,只怕也没有资格登基做皇帝了。

\chapter{第三章 风雨前奏}

悠然的坐在二层小楼的雅致厢房里面,透过窗户看向外面的竹林,仿佛又回到了南楚,现在,我已经是我雍王府的司马,地位重要的很,不过我却还是喜欢带着小顺子微服出游,虽然雍王屡次劝我要小心自己的安全。我现在缺一个很重要的人选,能够在太子身边卧底,可惜雍王提供的人选我都不大满意,这个人必须风流放荡,才能合乎太子的性情,这个人又必须善于逢迎,才能得到太子的宠爱,这个人又必须才华过人,才能够得到太子的赏识,这么一个人真是有些难找,雍王提供给我的人虽然勉强可以,但是我还是希望能够找到一个更加合适的人选。

我坐了没有多久,房门悄然开了,陈稹和寒无计走了进来,这里是我早就安排好的地方,这座在大雍十分有名的酒楼的主人荆舜荆是我的表弟,两年前,我在南楚养病,天机阁已经开始崭露头角的时候,荆舜卿前来投靠我。原因是因为他和舅父发生了争执。

说起来我的母亲出身名门,荆氏在嘉兴是首屈一指的书香门第,可是在母亲过世之后,父亲和他们发生了很大的冲突,因为我们居住的房屋,所有的田产都是外祖父送给父亲的,母亲过世之后,素来和父亲不合的舅父扬言要收回一切,按理说,这些财产都在父亲名下,他们无权收回,可是父亲秉性高傲,在舅父的辱骂欺凌下愤然抛弃所有,带着我远离嘉兴,还明确说明和荆氏一族恩断义绝,所以后来我考中状元之后,荆氏也没有颜面来与我和好。

我这个表弟资质驽钝,不喜欢读书,所以不得舅父的宠爱,而他又和家里的一个侍女情投意合,让这个侍女怀了身孕,舅父得知之后,要把孩子打掉,侍女转卖,还要表弟立刻和未婚妻完婚,其实对于表弟来说,如果能够将那个侍女收为妾室,他们两人已经心满意足,可是舅父坚决不肯让表弟得罪了岳家,结果我这个表弟一怒之下带着那个侍女逃到建业来投靠我。

我对这个表弟印象不错,他虽然不善于读书,可是办事精明,听说早就在打理家中的田产和上下事务,他的未婚妻是南楚富商之女,因为岳父看重他的能力,才定了亲事,而我的舅父看表弟不能取得功名,索性就让他攀了高枝。谁知道表弟却和侍女私通,得罪岳家,故而舅父才勃然大怒。

我既是同情表弟,也是对舅父仍有怀恨,所以安排表弟去求见天机阁主,当然‘天机阁主‘寒无计对表弟十分赏识,资助他行商,为了避免岳家的打压,表弟渡江到了大雍,当时南楚和大雍还维持着表面的和睦,所以表弟没遇到什么阻碍,就在大雍站住了脚跟,表弟的确是商业奇才,不过两年,当初我投入的十万两银子就增长了无数倍,表弟通过在大雍和南楚之间交易货物成了巨富,而他又及时将资金投入到其他行业,成了丝绸业巨子之一,这是因为我替他改进了织机和他聪明能干的缘故,而且一年前,他的岳父找上门来,不仅和他和好,还把他的未婚妻送了过来,其实表弟的未婚妻虽然性子倔强,倒不是不讲道理的人,表弟和岳父的合作,也让他的生意飞速发展,商人都是重视利益的,他们看出了南楚的危机,所以两人准备将部分生意和资金转移到大雍,而表弟就是开路的先锋。

我当初没有想到我这个表弟会如此出色,当初投资的时候说好了天机阁占五成股份,后来表弟宏图大展,提出以五十万两的代价购回股份,当然表弟是做好了我们漫天要价的准备的,可是我当然不会太过分,而且天机阁从来不做让合作者太心痛的事情,所以以一个合理的价格出售了股份,但是按照惯例,保留一成的股份,而表弟也知道天机行会的势力,所以双方欣然达成协议。表弟是一个重情重义的人,虽然不知道我和天机阁的关系,仍然几次送来重礼,感谢我当初的指引。

这次我被雍王俘虏,来到大雍,事先就派陈稹他们到大雍等我,表弟的产业当中有很多我安置的人,他们虽然对秘营的事情没有什么记忆,但是还是记得秘营安置他们的恩德,而且他们的资质毕竟都是比较出色的,所以很多都成了重要的管事人员,再加上天机阁的身份,所以秘营在表弟的产业中可以来去自如。而这个酒楼就是表弟在大雍的产业之一,名叫江南春,卖的都是南楚风味的酒菜,很受大雍权贵的欢迎。我这个表弟还是很不错的,知道我被俘之后,亲自来到长安,希望为我尽力,几次通过关系想求雍王‘高抬贵手‘,只是门路不通,直到我成了雍王府的司马之后,雍王才知道表弟走门路想救我的事情,倒是对表弟十分赏识,所以我这次才能轻而易举的出府到江南春喝酒,毕竟这里不会有人能够联合老板暗算我,雍王又派了几个武功高强的侍卫保护我,要不然雍王才不放心我的安危呢。

看看陈稹和寒无计,我微笑道:‘两位近来好么,江某任性,倒让两位担心了。‘

两人见礼之后,寒无计笑道:‘属下费尽心思,安排了公子交代的诈死计划,可惜功亏一篑,公子还是被雍王感动了,公子可得补偿一下我们的心血啊。‘

陈稹白了他一眼道:‘少胡说八道了,是谁一直说其实公子不用那么危险诈死的,听说公子改了主意又在那里欣喜若狂的。‘

陈稹虽然是玩笑话,我的心里却是一动,看看寒无计,心里暗暗盘算,他也是蜀人,怎么会这么赞同我投靠雍王。我怀疑的目光钉在了寒无计身上,如果此人有问题,那么我的秘营岂不是已经泄漏了出去,但是没有这方面的迹象啊。

寒无计从前毕竟日日钩心斗角,看到我的目光,心里一寒,连忙跪在地上道:‘公子,属下确实倾向大雍,前些日子我们在长安等待公子,属下遇到了一个过去的同僚,他见我处境还不错,就对我说,要我和他们一起支持蜀国太子,重立蜀国。我当时婉言拒绝,可是那人说现在有人组织反抗势力,如果我不答应,那些人找到我头上的时候,绝对不会放过我这种数典忘祖的叛逆,属下知道这些人欺软怕硬,如果公子归顺了雍王,借助雍王的势力,那么这些人反而不敢明目张胆的来找属下了。‘

我微微一叹,蜀国的反抗势力的存在我并不奇怪,可是用这种方式真是太愚蠢了,寒无计从前也算是比较反对大雍的,当初我要诈死,他虽然不说,但是十分积极,现在却为我归顺雍王而大喜过望,这样的变化就是那些反抗势力造成的,一个已经放弃过去,有了自己的生活的人,谁愿意再投入到没有前途的反抗势力中去呢?确定了那些势力兴不起什么太大的风浪,我仍然交代寒无计等人留心自己的安危,虽然暗杀不能改变国家大势,但是个人的命运却是可以改变的,想了一想,我对寒无计说道:‘下次他们再来逼你,你就说自己正在做生意,愿意给他们资助,但是你自己不想参加。‘

寒无计惊讶地道:‘公子为什么这么做?‘

我淡淡一笑道:‘我要你掌控他们的行动,这样一来对我会有些帮助,将来要铲除他们也容易一些。‘

寒无计默然不语,我有些疑惑,正要问他怎么了,小顺子的声音在耳边响起道:‘公子,他是蜀人。‘

我这才想起,他刚才虽然表示对我投靠大雍感到安心,但是并不意味着他愿意看到蜀国的反抗势力失败。

轻轻摇头,我道:‘无计,你的心思我明白,可是你要清楚,这些人大多并非是对蜀国忠心耿耿,而是为了夺回失去的权力罢了,他们用这种方式谋叛,不仅没有成功的可能,还会连累更多的人,甚至他们会伤害更多的人,例如,你若没有自保的能力,他们会怎么对付你,你好好想一想,我不勉强你,这些事情我会交代给别人去做。‘

寒无计跪在地上,叩首道:‘属下谢公子宽宏大量。‘

我看看陈稹,他轻轻点头,我知道他会接手这件事情,而且他会监视寒无计,不让他危及我的大业。

陈稹看寒无计已经平静下来,道:‘不知道公子是否准备告诉雍王殿下秘营的存在。‘

我淡淡一笑,问道:‘你的看法呢?‘

陈稹道:‘属下认为,若是告诉雍王,那么公子将来就少了自保的力量,但若是不告诉雍王,只怕将来雍王会怀疑公子的忠心。‘

我看看小顺子,小顺子冷冷道:‘你说得不错,但是绝对不能将秘营显露在阳光之下,公子之所以能够进退自如,全是因为秘营的存在,而且雍王就是怀疑公子的忠心,我们大不了离开大雍。‘

我想了一想道:‘小顺子太偏激了,这样一来,我们就等于和雍王敌对,这样不好,秘营不可以露面,这样吧,以后我尽量不和陈稹见面,陈稹负责秘营的主持,小顺子负责转达我的指示,以后秘营的任务就是将自己融入到长安下层当中,记住我的话,不能涉入到上层权贵的势力当中,这样一来,就算雍王殿下发觉了秘营的存在,也不会对我有太大的忌惮,毕竟雍王殿下也不会相信我完全没有一点可以依靠的势力,大不了我说秘营是小顺子的手下,我想说得过去的,这么长时间,他们至少也能看去一些小顺子的深浅。‘

小顺子点点头道:‘雍王软禁公子的时候,一直派了一些高手监视我们的,我虽然可以出入,但是若是带了公子,恐怕是不能轻而易举的逃走的。现在雍王派在我们身边的侍卫武功也不错,不过只是保护的意味,因为其中没有可以缠住我的高手,武力弱了不少,只是准备协助我保护公子罢了。‘

我正要吩咐他们一些事情,突然外面传来吵闹声和兵器相交的声音,我眉头微皱,这间江南春酒楼是高级的所在,怎会有人会在这里动手,看了小顺子一眼,他会意的走了出去,不一会儿回来告诉我说,原来是外面有人争斗,小顺子说看来好像是江湖仇杀。我从前也曾见过沙场血战,也曾见过文人舌战,还没见过江湖仇杀呢,不由来了兴趣,招呼小顺子一声,我走出了房门。

江南春虽然名义上是酒楼,实际上却是一个小小的园林,园中到处都是江南山水,花卉、竹林、小桥、流水、假山将园中的空间巧妙的分割成上百个小空间,每个小空间都有样式各异的楼台轩阁,最是闹中取敬,处处楼阁之间都有回廊连接,回廊之外便是繁茂的花木,所以格局十分优雅隐秘,最适合密谈相会。

我所在的这座小楼十分清雅,推开二楼的房门,外面是朱红栏杆围绕的楼台,旁边有通往下层的楼梯,雍王的侍卫都在下层伺候,我站在栏杆前面,向下看去,楼下和另外一处楼阁连接的回廊上站着一个负手而立的老者,他身后站着两个相貌威武的中年人和一个相貌秀美娇艳的少女,而在回廊之外的一处假山之上,站着一个黄衣书生,相貌俊秀,只是带着几分轻浮,手里拿着一支玉箫,而在他对面,站着一个英俊的青年,手中一柄宝剑,两人正在交手,那青年剑法似乎不错,剑光闪动中将那个书生逼得十分狼狈,可是那个书生不时笑骂嘲讽,我看那个青年面红耳赤,简直都要疯了。

我往下看的时候,那个书生正在一边还手一边喊道:‘哎呀,真是要命啊,小生不过是说笑了几句,又不是跟你抢美人,你放心,你的师妹虽然漂亮,小生看惯了天下美女,比她漂亮的可不少呢,不过是调笑几句,又没碰到她一丝头发,干吗这么拼命。‘

那个青年大叫道:‘胡说,胡说,你来投靠,我们好意接纳,你却,你却作出那种无礼的事情,冒犯我师妹。‘说着,剑法更加迅疾。

那个黄衣书生一边抵挡一边信誓旦旦道:‘窈窕淑女,君子好逑,小生不过是思慕佳人,追求了沙小姐几日,可绝对没有非礼行为,再说了,沙小姐是凤仪门弟子,小生就是胆大包天,也不敢得罪她啊,晶晶,晶晶,你替我求求情,我可没有冒犯你。‘

那个相貌秀美的少女玉面微红,狠狠道:‘什么追求,天天缠在我身边,没事就在外面吹箫,还,还偷了我的东西,你乖乖的让我师兄打一顿,然后把东西还来,不然我绝不放过你。‘

那个黄衣书生长叹一声道:‘唉,看来你们是不放过我了,喂,看戏也看够了吧,老弟,你要再不救我,我可就没命了。‘说着这个书生手中的玉箫突然化成千百幻影,那个青年似乎分辨不清,下意识的后退了一步,谁知他忘记了自己身在假山之上,一个踉跄,他连忙稳住下盘,就在这一瞬间,这个书生突然凌空飞起,向我所在的方向冲来,口中还喊道:‘老弟,救命。‘

就在他身形闪动的时候,那个老者后面的一个中年人如同苍鹰一般从他后来扑来,这个书生手一抖,只听见一声剧烈的爆炸声响起,数丈方圆之内立刻青烟滚滚,其中还掺杂着红色的轻烟,那个书生大喊道:‘老弟,别使毒啊,我和他们没有什么大仇。‘所有的人立刻都屏息凝立,等到青烟散去,几个人定睛看去,只见那个书生已经没了影踪,他逃跑的方向的楼台上,一个青衣书生正在那里苦笑,他身后站着的一个清秀仆人则侧过脸去,似乎在偷笑。

那个青年怒冲冲的剑指楼台道:‘那个混蛋呢,快把他交出来,你竟然光天化日下用毒,也一定不是什么好人。‘

被指着的我不由更是苦笑连连,我居然被陷害了,刚才小顺子暗中对我说,这个书生似乎要突围,我还只是抱着好奇的心情想看他如何突围,青烟乍现的时候,小顺子立刻挡在我面前,然后我们就听到他的栽赃嫁祸,那红烟是什么我不知道,但我知道绝对不是毒药,可是对着那些怒目而视的人来说,我可怎么解释呢?

这时,雍王府的几个护卫已经冲上楼来,看我安全无恙,一个护卫走到我身边,低声道:‘大人,发生了什么事情。‘

我轻轻摇头,扬声道:‘几位,刚才那人与在下并无关联,还请几位明察。‘

那个青年高声道:‘狡辩,我们遇见那人的时候,他正向你那里走去,刚才又从你那个方向突围,你们不是同党才怪,快说,风流浪子夏金逸和你什么关系?‘

我微微一笑道:‘在下与那人实在并不相干,还请明鉴。‘

那个老者突然道:‘阁下如此轻视我们的才智么,姓夏的原本是向你那里走去,刚才看到你们之后有几次三番想向你们那个方向突围,若是和你们没有关系,你们为何始终不曾反驳。‘

这时那个护卫在我耳边低语道:‘这几个人是长安关中联的人,那个老者是联主沙青元,关中联汇集长安武士,实际上是朝廷控制江湖人士的所在,沙青元现在是中立身份,但比较偏向齐王,因为他的很多弟子都在齐王军中效力。‘

我的脑海里面突然有了一个模糊的计划,便开口道:‘江联主此言差矣,我等听到外面吵闹,故而出来看看热闹,那人突然攀扯,附近还有数处楼台,在下怎知此人攀折的是我们,联主听信一面之词,未免有失身份。‘

\chapter{第四章 故人重逢}

那老者眉头紧锁,眼前这个青年虽然文弱,但是言辞温和,但是却带着一种隐隐的威慑力量,似乎并不看重自己的身份。他也是精明人,知道能够到江南春的都不是什么寻常人,再看我身后几个护卫,都是气度沉凝,目光森冷的高手,不由道:‘阁下说的也有道理,不知阁下怎么称呼。‘

我微微一笑,示意身边的护卫,那个护卫高声道:‘这位是雍王麾下,天策帅府新任司马江哲江大人。‘

那个老者身子一震,天策帅府的司马,那是雍王麾下数一数二的文官职务,他躬身行礼道:‘草民*元,冒犯司马大人,请大人恕罪。‘

我淡淡道:‘不知者不罪,那个黄衣书生是什么人,竟然陷害本官。‘

老者赧然道:‘此人姓夏,叫夏金逸,江湖匪号风流浪子,曾是崆峒弟子,因为行为放荡被逐出师门,但是因为没有犯过什么大错,所以没有被废除武功,此人日前到在下府上,希望加入关中联,草民见此人虽然有些轻浮,但是也还有心报效大雍,所以将其收下,不料此人色胆包天,不仅调戏小女,还偷了小女的物品,原本也只是派人捉拿罢了,不料今日在此地相遇,又被他用诡计骗了,以至冒犯大人。‘

我若有所思的点点头,道:‘联主请自便吧,此人如此放肆,若是被我捉到,定会送到联主手上,任由联主处置。‘

*元喜道:‘如此多谢大人了。‘

回到房中,看看陈稹和寒无计,我突然轻笑道:‘想不到有人连我也骗了。‘

小顺子问道:‘公子真的要捉他么?‘

我笑道:‘不错,一定要捉住他,不过不要伤害他,我想用这个人,小顺子,你有没有法子捉住他,不让别人知道,这有点难度,不成功也没关系,我会有别的法子的。‘

小顺子笑道:‘公子放心,刚才我为了保护公子,没有动他,不过我在他身上用了追魂香。‘

我看看陈稹,陈稹道:‘公子放心,追踪使用的啮香鼠我们都带了过来,不知道公子准备在哪里见他。‘

我想一想道:‘想法子把他暗中送到这里来,记得不能露了痕迹,我明天过来见他,记得,什么人都不能知道,你们把他点了穴道,装在箱子里带来。‘

陈稹道:‘公子放心,这里我们可以做一半主,绝对不会露了痕迹。‘

在回府的路上,我在心里盘算着计划实施的可能性,越想越觉得可能性很大,坐在马车里,我正在反复盘算,突然,马车突然停下,我的身躯向前撞去,幸好小顺子一把扶住了我,我才没有撞到。这时,车外传来禀报的声音道:‘大人,是一个男子冲撞了车驾,此人从巷子里面突然冲出,惊了马,不过这人已经晕了过去。咦,大人,这人背上有个小孩,胸前还有刀伤。‘

这时远处传来刀剑撞击的声音,不一会儿,有人回禀道:‘大人,有几个人追杀出来,我们抓住了两个,但是逃了一个。‘我沉声道:‘把人带回去,详细查问,结果告诉我知道。‘

‘是。‘车外传来恭恭敬敬的回答。

我轻轻一笑道:‘怪不得世人喜欢荣华富贵,令下禁止,谁不喜欢。‘

小顺子低声道:‘要不要我去看看?‘

我摇头道:‘不必了,应该和我们没有什么关系,让雍王府的人去查吧。‘

第二天早上,昨天保护我的侍卫进来禀报,我们救下的人已经醒了,只是伤得很重,只怕性命不久了,此人自称韩章,除此之外,什么也不肯说。我狐疑的看了小顺子一眼,是那个我认识的韩章么?小顺子出去了,片刻之后返回,淡淡的告诉我,正是我在蜀国的护卫韩章。

我腾的站起来,急匆匆的走到韩章养伤的所在,在一间整洁的厢房里面,韩章躺在床上,面如金纸,我走上前按在他的腕脉上,不久就拿了下来,他,已经接近油尽灯枯了,我轻轻摇头,将一粒药丸塞到他口中,渐渐的,他的面色出现了红润,他睁开了眼睛,看见我,他的眼睛出现了神采。我坐在他身边,冷静地道:‘韩兄,我们见得太晚了,你这些日子以来一定是伤上加伤,又没有好好休息,我已经无能为力,你为什么会到了这个地步,还有什么遗愿,告诉我,看在我们相识一场的份上,我会替你尽力。‘

小顺子示意其他人退出去,站在我身后,冷冷的看着韩章。

韩章开口道:‘江大人,想不到在这里见到你,你已经投靠了大雍么?‘

我微微一笑道:‘南楚继蜀国之后已经惨败,日后虽然还可东山再起,但是也最多只能苟延残喘,不错,我已经投靠了雍王。‘

韩章叹息道:‘也好,也好,大雍强盛,那些人鼠目寸光,没有成功的可能的,大人,我的岳母和妻子都死了,求你看在昔日相识的份上,照顾我的女儿,让她平安长大。‘

我神色一动,道:‘发生了什么事情,告诉我,否则我日后如何向令嫒交待。‘

韩章的目光变得幽远,他说道:‘离开大人之后,韩章没有再种田,我原是青城弟子,练了一身武功,国仇家恨,所以我投入了反抗大雍的地下势力锦绣盟,咳咳,可是镇守蜀中的陆侯爷手段高明,我们屡战屡败,后来,他们疯狂了,开始残暴的杀害蜀国的平民,他们说,凡是不肯反抗南楚和大雍的都是叛逆,最后,他们知道了我曾在南楚军中的事情,所以要处死我,我虽然百般辩解,可是还是没有用,我只有抱着女儿逃走了,我原本想我妻子是田将军的女儿,盟主又是她的表兄,应该不会受害,可是后来我抓住了一个追杀我的人,他告诉我,我的妻子死了,死得很惨,因为盟主原本就是我妻子的未婚夫,可是当年拙荆逃婚出走,嫁了给我,他是存心要杀我的,我的妻子,被他*不遂,杀死了,我的岳母悬梁自尽。大人,你当初劝我回到乡下平日度日,我没有听你的话,才有这个下场。‘

看着韩章凄凉的神色,我淡淡道:‘当初你深夜痛哭,我就知道你不会再独善其身,可是你是蜀人,我没有法子劝你不去复国。在你的立场,你没有错,只是你选错了同伴,放心吧,你的仇人不会有好下场的。‘

韩章的目光变得炽热,他道:‘大人,求求你,照顾我的女儿,不要告诉她这一切,我不想她再被国仇家恨牵绊一生,我希望她平平安安的嫁人生子。‘

我轻声叹道:‘去把他的女儿抱来。‘

小顺子出去一会儿,回来了,抱回来一个一岁多的小女孩,小女孩啊啊的笑着,伸手给父亲,要他抱抱,娇嫩可爱的面庞上,那双乌溜溜的大眼睛清澈如泉水。我看韩章神色激动,可是却无力坐起,便伸手抱过小女孩,忍不住亲亲她的面庞,小女孩突然叫道:‘爹爹。‘小手抓向我的头巾,我喜悦的看着她,道:‘韩兄,你的女儿很聪明,也很可爱。‘

韩章不知道从哪里来得力量,居然坐了起来,在床上拜倒,恳求道:‘大人,我知道太勉强你,求你收留这个孩子,好好照顾她。‘

我一惊,正要拒绝,看着孩子秀美的轮廓,突然说不出口,想起若非飘香身亡,也许我们的孩儿就是这么大了,心里一软,我道:‘我孤身一人,没有妻儿,若是韩兄不嫌弃,这个孩子就做我的义女吧,我必然待她如同亲生,韩兄,这个孩子叫什么名字。‘

韩章感激的泪水流下,他低声道:‘大人,韩章本是孤儿,就是这个姓氏,也是跟着师父取得,大人若是不嫌弃,请将这个孩子当作自己的亲生,不要告诉她身世。‘

我看了看韩章,透过那双悲痛欲绝的眼睛,看到他对女儿的挚爱,和满腔的悔恨。我淡淡道:‘也好,拙荆柳氏,遇难身亡,这个孩子我会告诉她,她是我的亲生女儿,名字,就叫江柔蓝。‘

韩章满怀感激地道:‘多谢大人,柔蓝,柔蓝,大人,锦绣盟主霍纪城手段毒辣,大人一定要小心。‘

说罢,韩章闭上了眼睛,再无声息。这时,柔蓝还在伸着双手,向着自己的父亲要求抱抱。我把她抱在怀里,一滴泪滑落尘埃,战乱当中,有多少这样惨痛的事情再发生啊。这时柔蓝大哭起来,似乎也知道自己的父亲离开了人世。

我召来总管太监常恩,让他安排韩章身后事,顺便替柔蓝找个奶娘和几个能干的侍女伺候。先把柔蓝交给侍女,我决定要去提审两个被抓住的犯人。他们既然追杀韩章,一定和锦绣盟有关,竟然在长安这么猖狂,我怎么能不问个清楚明白。

在雍王府的阴暗的地牢里面,我在典狱的带领下走过青石廊道,两边都是厚重的木门,只有在一人高的位置留有一个小窗口,装着精钢的栅栏。廊道尽头是一间刑房,走下台阶,可以看到两个个子不高但是十分精壮的汉子被牛筋和铁链牢牢的固定在墙上,身上没有伤痕,看来并没有人对他用刑,我满意的点点头,若是胡乱用刑,反而会降低作用,看来雍王府很慎重呢。我看了看,四周摆着几样刑具,虽然不多,但是都是血迹斑斑,使得这件刑房立刻透露出阴森恐惧的气氛。

我看了一看两个汉子,对于用刑,我倒是颇有研究的,当初为了对付梁婉,我曾经查阅过所能找到的一切书籍,总算颇有收获,让我发现,用刑最重要的是摧毁一个人的信心,然后才能予取予求。

看了看房间里的十几个狱吏和一个文书,我笑道:‘把他带过来吧。‘我指向一个汉子,两个狱吏上前,熟练的把人解下下来,然后将他手臂扭到身后用牛筋捆绑起来,他们手法娴熟,让那个汉子毫无反抗之力。那个汉子被他们拖到我面前,一个狱吏揪住他的头发,迫使他抬起头来,让我看清楚他的相貌,这人相貌倒还端正,只是神色间戾气深重。小顺子挥手让他们搬来一张椅子,让我坐在上面,我微笑道:‘你们就是冲了我车驾的贼子么?‘

那个汉子眼光一闪,道:‘大人,草民没有冲撞您的车驾,是您的侍卫强行把小人抓来的。‘

我淡淡道:‘那对父女,是被你们追杀的吧,若非你们,怎会有人冲犯车驾,说吧,你们是什么人,若是不肯说明白,你们别想从这里活着出去,若是乖乖招供,我只把你们送到京兆尹那里问罪。‘

那个汉子又是神色一动,若是到了京兆尹,虽然自己杀伤人命,可是最多判个秋决,到时候未必没有机会逃狱,口中凄声道:‘草民实在是谋财害命,想不到撞到了大人的车驾。‘

我不说信也不说不信,随手取下了发上的一根发簪,这根发簪是我上次下令秘营铲除背叛我的商会的时候,陈稹他们从商会的密室里面得到的宝贝,虽然只是一根发簪,但是这根发簪是用天上落下的玄铁陨石的铁胆制成,锋利无比,就是最坚硬的金刚石也可以一刺而穿,但是发簪太小,对于普通的武林高手来说当然没有什么用处,小顺子虽然可以把钢针当成武器,但是他性子高傲得很,除了双手之外不愿意用别的武器,最后我就留下了这根发簪,说不定什么时候用的上呢,这不,我就可以用这根发簪来作针灸的金针,只是粗了一点点,用来动刑最好不过。

我笑着问道:‘你愿意招了。‘

那个汉子连连点头,我淡淡道:‘没有用刑,我从来不信任何人的招供。‘说罢,我的发簪在这个汉子身上轻轻刺了几下,这个汉子顿时面色大变,脸上一阵青一阵红,身子更是嗦嗦发抖,若没有两个狱吏死死挟住,只怕早就软倒在地上,最可怕的是他却连声音也发不出来,只见他的额头上汗如雨下。他抬起头来,眼中满是哀恳之色,我却是悠然自得的看着他,一派温文儒雅,好像眼前并没有在苦苦挣扎。用刑之道,首在攻心,我若轻轻放过了他,施了一个下马威,这样一来一会儿他若是敢胡乱搪塞,我只要说让他受到更加惨烈的毒刑,必然让他恐惧,而且相信我定可做到。

过了片刻,我见他神智已经渐渐不清,轻轻一挥手,发簪刺入这人的身体,这人的身体渐渐放松下来,口中发出低微的呻吟,却也不能怪他,痛苦解除之后,身躯极度放松,他刚才被压抑住的声音才发了出来。吩咐狱吏端来冷水,仔细的灌入汉子的咽喉。他的神智清醒了,看到我,眼中露出掩藏不住的惊恐。

我微笑道:‘好了,现在你说得话应该有些可信了,请问壮士贵姓大名,祖籍何处,为了什么追杀那对父女。‘

那个汉子道:‘小人邱行,原是蜀人,因为蜀国亡后,蜀中落入南楚之手,陆信暴虐,所以流亡大雍,因为没有积蓄,所以谋财害命,这实在是小人肺腑之言,求大人明鉴。‘

我看看小顺子,淡淡道:‘此人的供词靠得住么?‘

小顺子淡淡道:‘我看是靠不住的。‘

我笑道:‘怎么说呢?我看他老实得很,应该不想再受更惨重的酷刑了。‘

小顺子恭恭敬敬地道:‘公子,这人周身衣服都是大雍所产,看来在大雍已经待了很久,身上有千余两银子的银票,若是肯安分守己,足可以逍遥度日,那对父女身上连十两纹银都没有,怎么会是谋财害命,而且敢在大雍光天化日之下杀人,实在是太嚣张了,若没有靠山,奴才就是死也不信。‘

我笑了,笑容和煦,用一种满意的目光看着那个汉子,说道:‘好啊,他若坦白招供,我还觉得没有意思呢。‘

所有的人包括狱吏都看着那个俊秀儒雅的青年,他温和的笑容却让所有人都心生寒意,心中都生出‘原来他是存心想要用刑来的,他跟本就不想得到口供‘的念头。

然后我手中的发簪已经再次刺入了邱行的身体,邱行的身体开始蜷缩抽搐,这次两个狱吏已经几乎不能控制住他了,我看了一会儿,道:‘来杯茶吧。‘见我开口,原本满怀期望的汉子眼中闪过绝望的光芒,小顺子看了他一眼,脸上表情似乎有些同情,犹豫了一下,开口道:‘公子,殿下新送来贡茶得用不少时间才能泡好。‘这一回满怀希望的邱行直接晕了过去。

我的发簪再次刺入了邱行的身体,邱行被冷水灌醒之后,目光茫然的看向我,我淡淡道:‘没关系,你去把茶具拿来,就在这里煮水泡茶,在你完成之前,我会试试几种新的针法。‘

邱行再也忍耐不住,嚎啕痛苦起来,扑向我的椅子,两个狱吏牢牢拽住他,他大声道:‘大人饶命,小人情愿招供,小人乃是锦绣盟杀手,求大人饶命,小人什么都肯招。‘

我淡淡的看了他一眼,不满地道:‘你什么都肯招,怎么这样没有骨气。‘

邱行涕泪交流道:‘大人饶命,小人愿招,求大人别再用刑了。‘

我百无聊懒的摇摇头,道:‘你们把他带到旁边的房间,让他招供,若有隐瞒搪塞,就把他送回来。来人,把另外一个带来。‘

看着我兴奋的神色,早就吓得魂不附体的另外一个汉子哀声叫道:‘小人尚伟,愿意招供。‘

我摇头道:‘不行,你若不受点刑罚,岂不是太不公平了。‘

这是小顺子的声音传入我的耳朵道:‘公子,有点过火了,你没看到那些狱吏的眼神,快把你当成暴虐的邪魔了。‘

我轻轻一笑,一语双关地道:‘没关系,在等待供词的时候,我可以先试试你的忍耐力,你若听话,最多我少用几针,这样吧,一会儿等他的口供出来,我再问你,如果你能够找到他的疏漏,我就放过你,若是找不到,我可还要对你用刑啊,现在,先请用一下小菜吧。‘说罢,发簪插入尚伟的身体。

两个时辰之后,我心满意足地走出刑房,留下了一大堆目中惊惧敬佩的狱吏和两个只剩半条命的锦绣盟杀手。

\chapter{第五章 玲珑棋子}

拿着供词,我向栖凤轩走去,因为我很想去看看我的义女柔蓝,不知怎么,我总觉得她是飘香泉下有灵,送来给我的女儿,匆匆忙忙的回到栖凤轩,我一眼就看到雍王殿下坐在那里,逗弄着小柔蓝。

我上前行礼道:‘殿下久等了,臣刚才忙于盘问口供,不知道殿下在这里。‘

李贽笑道:‘我听说先生收了一个义女,特来看望,情况怎么样?‘

我笑道:‘殿下,臣发觉了锦绣盟在大雍的势力,已经盘问清楚,虽然过了一夜,不免有些变化,但是想要一网打尽也很容易。‘

李贽有些犹豫的看了看我,我心知肚明地道:‘殿下的意思,臣明白,锦绣盟现在主要在蜀中和南楚肆虐,大雍对他们来说目前还是一个可以休养生息的地方,所以殿下希望暂时保留锦绣盟。‘

李贽苦笑道:‘先生,实不相瞒,锦绣盟的存在本王早就知道,只是暂时没有过问,不过他们现在这样嚣张肆虐,将来若是传出去大雍曾经支持过他们,只怕大雍在东川、蜀中的民心就全完了。‘

我躬身道:‘殿下放心,臣已经有了计策,可以放过锦绣盟部分力量,但是要先把他们在长安的势力全部铲除,这样一来,就是将来他想把大雍卷进去也不可能了。‘

雍王道:‘这样也好,免得长安局势混乱之时被他们借机生事,毕竟他们和大雍也是仇敌,我手上有些情报,再加上你得到的供词,应该足够了,本王这就下令围剿。‘

我摇头道:‘一个小小的锦绣盟,殿下就是铲除了它又有什么功劳,若是殿下放心,请让臣来策划,既可以除去锦绣盟在长安的势力,又可以实现臣的剜心之策。‘

雍王目光一闪,道:‘本王既然已经授予全权,就请先生主持,需要本王支持之处,尽管明言。‘

我微笑致谢,这时雍王看看柔蓝,道:‘先生孤身一人,令嫒年纪幼小,没有母亲照顾总是不妥,王妃这段时间一直伤心世子就要去幽州,膝下空虚,若是先生不嫌弃,不如就让王妃照顾柔蓝,免得先生挂心。‘

我想了一想,说道:‘只是这样臣就不方便去看小女了。‘

雍王笑道:‘没有关系,先生若是想念女儿,就让小顺子到王妃那里接她回来。‘

我想,小顺子出入内宅没有顾忌,这倒是一个好主意,便道:‘那么臣就多谢殿下了,王妃必然能将小女教养成名门淑女,请殿下代臣叩谢王妃。‘

雍王看了我片刻,道:‘先生今年已经二十六岁,为何还是孤家寡人,也应该成家了。‘

雍王的话引起了我心中苦痛,我默然良久才道:‘臣本来已经有了未婚妻室,只是还没有完婚,她就去世了。‘

雍王一愣,道:‘这本王倒不知道,只是娶妻生子乃是孝道大伦,先生也不能总是这样孤苦,若是有心,本王当请王妃为先生找一个贤淑女子,不知先生意下如何。‘

我释然一笑道:‘臣性子本来随意,只是没有披发入山罢了,也不愿辜负了人家好女子,还请殿下不必费心了。‘

雍王摇摇头,叹了口气道:‘这些事情以后再说吧,先生去忙吧,本王相信先生定会给本王一个满意的结果。‘

我施礼道:‘殿下放心,不日殿下就可以在太子身边插入自己的心腹。‘

夏金逸觉得自己从来没有这么害怕过,他性情轻浮,偏偏有时又太冲动,因此得罪了师门长老,被赶出了门墙,想通过关中联进身,却又得罪了江小姐,无奈之下只得向一个师兄求救,他这个师兄性情方正,但是和他关系倒不错,现在在太子府上当侍卫总管,他无奈之下只有求师兄引荐,否则,他既没本事考科举,又没本事上阵杀敌,靠什么求个出身呢。可惜还没来的及和师兄见面,自己就被关中联堵上了,无奈之下自己只得施计逃离,谁知道自己栽赃嫁祸的竟是雍王府的司马,这原本让他十分气馁,但是师兄告诉自己,太子殿下若是知道此事,必然会留下自己,好扫扫雍王府的脸面,自己欣喜若狂之余,不免多喝了几杯,回到客栈却乐极生悲,被人偷袭制伏,那些人不知什么来历,将自己捆得结结实实,又用精钢铁拷锁死自己的双手,堵住自己的嘴巴,放在箱子里抬走了,等到自己觉察不到颠簸的时候,却没有人来放出自己,被捆了这么长时间,夏金逸只觉得四肢麻木,血脉不通,而且最大的痛苦在于他只能弯曲着身子,想伸直一下也办不到,这使他感到无比的痛苦,若是能够伸直身子,他甚至愿意付出任何代价,换句话说,他已经意志崩溃了。

终于耳边传来脚步声,有人打开箱子,那人手里拿着一盏油灯,灯光落到夏金逸的脸上,夏金逸下意识的闭上眼睛,免得因为久处黑暗而被光线伤了眼睛。片刻之后,夏金逸睁开眼睛,看到站在自己面前的是一个相貌清秀俊朗的一个少年,看上去不过十五六岁的年纪,他似乎好奇的看着自己。夏金逸目光中透出恳求和询问的意思。这个少年淡淡道:‘小人赤骥,奉命前来处置夏公子,若是夏公子不能得到小人的认可,便要葬身此地,若是侥幸通过,就可以见到我家主人,夏公子,你若大声喊叫,小人只得立刻杀了你,所以还请公子自重小心。‘说罢这个少年将油灯放在房内的一张桌子上,上前掏出夏金逸口中的丝巾。夏金逸深深的吸了一口气,道:‘求小哥先把我放出来吧,再不伸一伸身子,夏某只怕就要残废了。‘看到了敌人,夏金逸的神智渐渐回复,他已经准备开始和敌人斗智了,虽然对自己的敌人竟然是一个少年而奇怪,但是夏金逸很清楚,江湖上最可怕的就是和尚、女人和小孩,所以他心中全没有轻视的心理。

少年微微一笑,将夏金逸从箱子里提了出来,将他放到地上,这样一来,夏金逸虽然还被牢牢捆住,却已经可以伸展身躯,他口中发出舒服的呻吟,闭上了眼睛,似乎想要好好睡上一觉。

少年一笑,踢了夏金逸一脚,道:‘老兄,你是不是忘了什么事情?你的生死可还在我掌握当中呢?‘

夏金逸睁开眼睛,满脸舒服的表情,道:‘小哥,夏某不过一个江湖浪子,如果贵上不是有用我之处,何必那么费力把我绑来呢?我想小哥若是随便杀了我,说不定还要遭到责罚呢?‘

少年忽然坐在了地上,对这夏金逸说道:‘你说得也不错,可惜我的主人性子高傲,若是废物点心,他是绝不用的,所以你得说服我带你去见主子,若是不让我心服口服,我就是杀了你也没什么,反正你也不是唯一的人选。‘

夏金逸心中一凛,他看这少年虽然年少,但是说起话来十分老道,而且说到杀人似乎没有一点动容,便试探道:‘小哥年纪轻轻,可是杀了很多人么?‘

赤骥笑道:‘不敢相瞒,当初小子为了保住性命,也杀了八九个人,后来给主人效力,男女老少都杀过,最可怜的是有一次我们不得已杀了很多无辜的人,其中还有几个妇孺,说句实话,当初真是不想杀的,可是谁让他们偏偏待在不该呆的地方,只有一次,小子一个人也没杀,可是他们却也没有活命。‘赤骥说的含糊,却是没有一句假话,当初秘营训练的时候,他们常常需要互相对决,若是战败次数过多的,就要被消去记忆送走,他们后来便说这些人都已经死了,失去秘营的记忆,在他们来说,真是生不如死的,至于为天机阁办事,杀人更是难免,只有最后梁婉的那一次,他可是没有杀人,不过既然只有长乐公主和痴傻的梁婉逃过一劫,也算不上仁慈了。

夏金逸听得出赤骥语气中的认真以及丝毫没有炫耀的意味,便知道自己真的遇上了杀人不眨眼的小魔星了,他勉强笑道:‘原来如此,那么夏某远远不如小哥了,夏某虽然在江湖上有个浪子的名号,但是杀人倒是不多的,毕竟武功不高,杀人比较麻烦,不知小哥怎样才肯放过夏某呢?‘

赤骥想了一想道:‘这可难了,我虽然没有钱财,但是想要花用的时候不缺银子,我的武功虽然不高,但是已经足够了,若说荣华富贵么,虽然人人都爱,但是我年纪还小,十年以后再去争夺也不迟么?‘ 说到后来,语气渐渐冷淡,赤骥从怀里掏出一把小匕首,比划比划夏金逸的咽喉,笑道:‘好了,你说吧,你能给我什么好处?‘

夏金逸连忙道:‘小哥不要着急,夏某有主意了,看小哥已经十五六岁了,大户人家的子弟都该成婚了,看小哥气度不凡,就是不是大家出身,也得娶个千娇百媚的大家淑女,要不然岂不是明珠投暗,夏某没有别的本事,说到追求女孩子那是没说的,再说这大江南北的出色美女绝对没有人比我知道的多,小哥若是有意,不妨让夏某为你出谋划策,娶个漂亮的娘子如何。‘

赤骥看了夏金逸半天,噗哧一声笑了,道:‘看你被江大小姐追杀的四处逃命,看不出来你还有这样的本事,好吧,就让我听听你的主意。‘

夏金逸松了一口气,他看的出来赤骥的杀气已经消散了,便笑道:‘小哥,你可不能瞧贬我,要说呢,江大小姐是不错,可是千万不能娶做老婆?我也不过偷了她的肚兜,就到处追杀我。‘

赤骥听得张大了嘴,看着夏金逸道:‘你偷了她的,她的那个。‘

夏金逸笑道:‘那有什么奇怪,老子,不,本公子就是看不顺眼,一个小姑娘,惹得关中联上下的青年男子都跟在屁股后来追求也就罢了,老子这么风流潇洒,这小妞整天听我胡说八道倒是很开心,你要是稍越雷池,她就扳起个晚娘面孔,不就是仗着她是凤仪门弟子么,所以老子索性用了迷香把她弄晕,亲自到她闺房偷了她的肚兜,哈哈,让她追杀老子都不敢说明理由。小兄弟,老子告诉你,凤仪门的女孩子娶不得,平常一个个冰清玉洁,全靠着姿色勾引男人,我就不信,一个女孩子没有一点暗示,那么多男人就死死追求你,欲擒故纵比谁都拿手,老子追求美女的功夫比起她们勾引男人的本事可是差的远呢。最可恨的是,你要是真的得了手,平时对你百依百顺,你若不顺了她的心意,跟你翻脸也是转眼的事情,告诉你,就是娶一个不识字的村姑,也比娶那些凤仪门的女弟子强。‘

赤骥愣愣的看着傥傥而谈的夏金逸,道:‘听你说得这么可怕,你遇到过这样的事情。‘

夏金逸愣了一下,神色突然大变,半晌才道:‘没有,没有,我不过一个江湖浪子,人家凤仪门的女弟子不是嫁给官宦人家,就是嫁给武林世家,哪里可能跟我有什么牵扯。‘

赤骥看向夏金逸尴尬的面色,问道:‘你就不怕我和凤仪门有什么关联么?‘

夏金逸的冷汗立刻流了下来,转眼就恢复正常,笑道:‘哪能呢,凤仪门虽然可能会驱使一些男子,不过小哥这样风度气质,应该不会是迷恋美色的人吧。‘他心里嘀咕,凤仪门怎么也不会把手伸到半大小子的身上吧?

赤骥淡淡一笑道:‘既然你不喜欢凤仪门,干什么要投靠太子,谁不知道太子和凤仪门是一条船上的人,雍王和凤仪门可不合呢?‘

夏金逸苦着脸道:‘小兄弟,俗话说穿衣吃饭,可是人生大事,你说我又不能耕田种地,又不能上阵杀敌,想要做保镖护院偏偏我这性子相貌,人家见了就看不上,若是作强盗飞贼,说句不好听的话,大雍的捕头不大好对付,我的武功又不是很强,只怕过几年就到大牢里面吃闲饭了。至于说投靠雍王么,夏某恐怕是没这个福气的,雍王要得是有本事的人,这个,我恐怕混不进去,太子那里就轻松的多了,其实我本来很想投靠齐王的,听说齐王最喜欢风月场所,说不定我还能得到齐王的赏识呢,可是来了长安才听说齐王虽然喜欢走马章台,可是身边用的人都是经过沙场血战的勇士,我这样的人可不行呢。‘

赤骥想了一想道:‘你说得也没错,武林中人练武喜欢小巧的武技,你们崆峒更是奇门武学为主的门派,你若上了战场只怕成不了普通的将领,再说不是任何人都喜欢军旅的,你性子如此玩世不恭,只怕在军中没几天就被军法从事了。‘

夏金逸赞同地道:‘是啊,我虽然什么本事都没有,但是自知其明还是有的,要是能够在太子府上呆个几年,应该总比流浪江湖的好吧。‘

赤骥看着他,终于轻松的笑了,说道:‘虽然觉得放过你没有什么好处,不过真是不想杀你啊,好了,我想你可以去见我的主人了,提醒你一句,我的主人平日倒是仁慈和气的,可是一旦认真起来,你最好希望死的痛快一些。‘

夏金逸突然笑了,说道:‘多谢小哥提醒,夏某从来都是很识时务的。‘

这时在另一个房间里面的我,终于忍不住笑了出来,透过铜管听到夏金逸的一番话,让我心情很愉快,这时小顺子突然也笑了,我问他道:‘怎么,你也觉得他很有趣。‘

小顺子忍住笑说道:‘公子,奴才说句放肆的话,他很像你,如果不是公子才华横溢,我觉得他实在很像你。‘

我本来有些气恼,可是仔细想想,又忍俊不住的笑了,想一想真是如此,我对夏金逸更加有兴趣,而且更加相信我的计划会成功的。

片刻之后,夏金逸被赤骥押了进来,赤骥解开了他身上的束缚,所以他可以自己走进来,不过他也聪明的没敢反抗,否则只怕他就走不到我的面前了。经过十几个时辰的折磨,他如今不仅饥肠辘辘,而且衣衫凌乱,俊秀的面容也都是灰尘污迹。艰难的走了进来,赤骥轻轻推了他一下,他抬头看见坐在书案后面的青衣人以及站在他身后的俊秀仆人,然后很顺从的跪了下来,低声道:‘草民叩见司马大人。‘

我有些以外的看着他,虽然他曾经远远的见过我一面,不过还能记得我倒让我有些惊喜,我笑道:‘夏公子,你出身崆峒,看你也不会犯什么大错,为什么会被逐出师门呢?‘

夏金逸抬头看看我,很直接地问道:‘大人,不知道小人有什么可以效劳的地方,若是小人可以胜任,大人再盘问不迟,若是不能,小人也不愿随便对什么人都谈及往事。‘

我再度认真的看看他,淡淡道:‘我需要一枚棋子,最好这枚棋子有自己的思想,换句话说,我要的是一颗聪明玲珑的棋子,你,很适合。‘

夏金逸露出灿烂的笑容道:‘那么我可以不用死了吗?‘

我也笑了,道:‘你若足够聪明,不仅不用死,事成之后,我会给你一个退路。‘

我们两个人相视而笑,这时一个幽幽的声音飘进我的耳朵,是小顺子的声音在说道:‘你们两个还真像。‘

我忍不住白了小顺子一眼,虽然传音入秘很好用,但是也不用老是用来欺负我啊。

\chapter{第六章 金牌间谍}

我笑着挥手道:‘赤骥,为夏公子看座,你先下去吧。端些茶点来,想必夏公子已经饿了。‘

赤骥转身出去,不会儿端来了茶点,便退了下去,夏金逸有了坐位,连忙狼吞虎咽起来,不一会儿,他打了一个饱嗝,吃饱了之后,他几乎摊倒在椅子上,看向我道:‘大人请吩咐吧。‘

我淡淡道:‘你可知道我的身份?‘

夏金逸赧然道:‘已经知道您是天策帅府的司马,名姓也听师兄说了,听说您是江哲江大人,就是一首词送了蜀王性命的那一位。‘

我笑道:‘你还忘记说了,我是南楚人,被国主免了官职,如今改弦易辙投了雍王。‘

夏金逸笑道:‘那有什么关系,南楚既然看不起你,我听说雍王还是很重视贤才的,大人投了雍王也没错啊。‘

我淡然道:‘好了,本来应该问问你的身世的,但是想一想也没有什么必要,不过你若有什么特别的仇人或者特别的经历不说出来,将来若是有什么意外可别怪我言之不预。‘

夏金逸想想道:‘草民也没有什么特别的事情需要禀报的,不过草民文不成,武不就,不知道能够为大人做些什么。‘

我淡淡道:‘很简单,你还是去投靠太子,但是我要你成为太子的心腹。‘

夏金逸愕然道:‘大人,我一个小人物,怎么可能接近太子?‘

我没有说话,拿起一张纸,上面写着一些密密麻麻的小字,递给夏金逸,夏金逸看了之后神情变幻莫测,道:‘大人,就这样简单么?‘

我笑道:‘是啊,我不要你刺探机密,也不要你和我们联络,只是让你按照我的计划得到太子的宠信,说句实话,你这个性子,虽然作不成得力手下,但是作个幸臣还是措措有余的,我已经安排好了你能够得到太子殿下宠爱所需要的条件,接下来,就要看你随机应变,你只要把握好一个原则,就是让太子放纵自己,遇到合适的时候,说几句离间的话,但是记住,只能涉及齐王,除此之外,你一切都要听太子的命令,即使让你做什么坏事,你都要去做,即使你听到什么天大的机密,哪怕是他们马上就要加害雍王,你也不用理会,知道么,今日是我们唯一一次的见面,日后就是相见我也不会和你说任何一件关于今夜的事情。‘

夏金逸虽然不明白我的用意,但是他还是点头道:‘明白了,大人让我能够得到太子的宠信,其他的事情都由我自己决定,其实我只要尽量讨好太子,陪着太子吃喝玩乐就可以了。‘

我点点头道:‘不错,锦绣帮的情报你可以告诉令师兄,通过这件事情,至少你可以进入太子府,并且得到太子的赏识,之后就靠你的机灵了,我给你的三张药方,是一种极品春药的配方,,可以助你得到太子的欢心,三张药方效力略有不同,你依次提供给太子,记得,就说你自己改进的,你既然使用迷香,那么对春药应该有些见识吧,太子不敢让太医做这种事情,所以你的宠信应该没有问题。‘

夏金逸赧然道:‘不瞒大人,小人虽然没做过采花的恶行,但是迷香春药确实略知一二,大人提供的药方小人也能看明白,却是是一种上好的药方,既可以增加情趣,又不会伤害身体,只要不过分使用,这种药方就是绝顶的好药。‘

我笑道:‘好,这样一来,我就更放心了,你还有什么问题么,过了今日,就没有机会问了。‘

夏金逸犹豫地问道:‘大人,您这么放心小人不会出卖您么?‘

我淡淡一笑道:‘出卖,你出卖我什么,我给你情报让你立功,给你药方让你得到太子宠爱,我又不要求你什么,至于陪着太子玩乐,太子若是贤明,你怎会有用武之地,离间齐王和太子,就是太子知道我让你这么做又能怎么样,他会完全信任齐王么?所以你若聪明一些就照我的话去做,既可以得到宠信,也不用招惹杀身之祸,否则,小顺子--‘

小顺子随手从怀里拿出一块银子,轻轻松松地将银子在手掌里磨成了粉末。夏金逸吞了一口口水,看向我。

我又道:‘不过我也真的不能完全信你,这样吧,你写个字据,就说你是雍王府派去卧底的人,交给我保存,你若是嘴不严,我就让太子见到你的字据,到时候别说雍王府要杀你,就是太子也不会放过你,就是我这个随从,杀你也是易如反掌,你只要忘记了今夜的事情,拿着我给你的见面礼和药方,那么就可以轻轻松松的讨好太子,得到你想要的荣华富贵,不过记着,你若太没有本事,两个月内还不能得到太子的宠爱,那么对不住,我在换人之前只好先宰了你了。‘

夏金逸下拜道:‘小人绝对不敢辜负大人的托付,大人放心,小人不过是讨好太子殿下罢了,不会有什么内疚的。‘

我笑道:‘好,你这就写字据吧,我等着看你的好戏呢,记着,等到事成之后,你不免受到牵连,不过放心,我会安排你的退路的。‘

夏金逸低头道:‘小人知道了。‘说着果然到桌前写了字据,夏金逸也知道若是不写只怕立刻就被杀了。等他写完,我又道:‘来人。‘

进来的是赤骥,我淡淡道:‘你先回客栈休息,明日你不是约好了你的师兄再到江南春喝酒么,席间你就说你发觉了蜀国余孽锦绣盟的行踪,然后雍王府的侍卫就会奉我的命令去抓你,把你送到关中联,不过我想你的师兄会救你的。‘

夏金逸又是心里一跳,无奈的说道:‘公子安排如此周详,小人一定不会出差错的。‘

我摆摆手道:‘好了,你还得原样回去,我就不送你了。‘夏金逸顿时傻了,张开嘴,想要说什么,但转念一想,又垂头丧气的跟着赤骥出去了。

送走了夏金逸,小顺子淡淡道:‘公子,字据要我收起来么?‘

我微微一笑,随手把字据就着银灯点燃了。小顺子奇怪的看着我,我淡淡道:‘这张字据有什么用,就是给了太子,不是也做实了我们雍王府的罪名。夏金逸若是聪明,就不会出卖我们,若是真的那么蠢笨,我们也不会有什么损失的,不过我想,他不会出卖我们的。‘看着灯光,我又笑了,这样便宜的事情,若是夏金逸反而出卖我们,那他可就是天字第一号傻瓜了。

夏金逸还是被点了穴道放到箱子里,在凌晨时分回到了客栈的房间,在路上夏金逸反复的想着自己应该如何作,他不是恪守信义的人,但是想来想去,对方只是给了自己得到太子宠信的机会和手段,自己名声不好,武功平平,若没有这些,只怕一辈子都不能得到梦寐以求的荣华富贵,就算自己把一切都说了出去又能怎样,现在谁不知道雍王日在中天,自己一个小人物如何能够撼动雍王,想起赤骥的手段,一个小随从都有这样的狠辣,那么他们主子的手段可想而知,自己最好的路就是照着他们的话去做,一定要在两个月内得到太子的宠信才行。

回到自己的床上,等到穴道解开之后,夏金逸坐起身来,心里慢慢盘算着该如何说话行事,直到过了午时,他才施施然的走出客栈,再次来到了江南春,伙计们看到他面色虽然没有什么变化,但是眼神却很古怪,谁让他昨天得罪了关中联,却又和太子府的人吃酒呢?

走进布置清雅的花厅,夏金逸一眼就看见几个锦衣人坐在一起谈笑,他几步上前,对着坐在首席的一个国字脸的中年人一躬到地道:‘大师兄,小弟来迟了一步,还请大师兄见谅。‘

那个中年人名叫张锦雄,今年三十七岁,相貌十分端正,他浑身上下衣着虽然华丽,但是却也平常,只是一双袖筒十分宽大,他是崆峒派第二十七代的大弟子,一身奇门功夫出类拔萃,隐隐已经成了下一代掌门的不二人选,原本他一心苦练武功,没有丝毫杂念,除了奉师门之命外出办事之外从来不轻易下山,但是两年前,凤仪门的使者亲自到崆峒,一夕长谈之后,张锦雄就被派到长安成了四品带刀侍卫。张锦雄性子方正,做事认真,从来不肯逾越分寸,所以很快就得到太子信任,成了太子府邸的侍卫总管。他平日从来不理会什么政争,平日除了负责太子府的防卫之外,就是练功,偶尔和几个亲近的侍卫出去喝几杯,虽然他性子严谨冷淡,但是因为他的慷慨大方和行事公正,所以十分得到下属的尊敬。严格的说,他并不是太子的心腹,因为太子的很多不愿公开的事情都有另外的人手去办,就是副总管邢嵩。

他看到夏金逸,神色上露出一丝笑意,淡淡道:‘你来了,过来坐吧,这几位都是我的属下,将来你进了府里还要请他们多多照顾呢?‘

夏金逸上前行礼道:‘几位大哥,小弟文不成武不就,所幸还有几分伶俐,若是几位大哥不嫌弃,有什么跑腿的事情交给小弟就行了。‘

一个瘦削的中年汉子笑道:‘知道了,早就听张总管说过,你这小子吃喝嫖赌无一不精,就是练武不用心,如果不是看在你讲义气懂进退,张总管根本不会还让你叫他一声师兄。‘

夏金逸脸不红心不跳地道:‘当然是师兄疼我,当初我不学好,被师门逐出,如果不是师兄求情,我的武功早就废了,来,小弟敬师兄一杯,以后还请师兄和几位老兄多多照顾,小弟绝不敢惹是生非。‘

几人都笑着喝了这杯酒,张锦雄自然是满意师弟的言行,而几个侍卫也不介意这个不会对自己造成威胁青年,接下来开始有训练有素的仆人送上精致的小菜,一个侍卫拿起筷子,笑道,南楚名菜果然滋味独特,不过名字也太怪了些,你看这道菜,虽然好吃的很,可是却叫美人肝。夏金逸笑道:‘此菜是以鸭胰白,就是鸭胗,配以鸡脯、冬笋、冬菇,用鸭油爆炒而成的,这个名字可是还有来历呢,听说当初南楚一位知名的才子在当地最有名的酒楼之一秋水楼宴客,谁知酒楼的厨师在配菜的时候少配了一样,若是少了一道菜,岂不是坏了名声,这厨子看来看去,被泡在水中的鸭胰白粉红娇嫩的颜色吸引住了,便配上鸡脯肉用鸭油爆炒,结果客人十分赞赏,问这菜的名字,上菜的伙计见色泽乳白,光润鲜嫩,随口说出‘美人肝‘三个字,结果这菜就有了这个名字,其实南楚还有一道名菜叫做西施舌呢?‘

几个侍卫惊异地道:‘西施舌。‘

夏金逸笑道:‘其实就是海蚌的舌足,据说肥白娇嫩,乃是天下美味,不过只有在海边才容易吃到。‘

一个侍卫笑道:‘夏兄弟果然见闻广博,若有机会应该试一试这西施舌。‘

夏金逸心思一转,便道:‘其实小弟最喜欢蜀中的佳肴,听说长安也有不少擅做川菜酒楼,就像长安都会市里面的红云阁和利人市的西子楼都应该有不错的川菜。‘

一个侍卫嗤笑道:‘夏兄弟还说在长安已经混熟了了呢,这下可露馅了,我就是长安的坐地虎,什么酒楼馆子我不清楚,红云阁的确是川菜酒楼,那里的太白鸭天下一绝,西子楼乃是秦楼楚馆中的佼佼者,我可是清楚,里面的美女都是个个窈窕动人,还有不少南楚的女子,据说是私下里从南楚贩卖过来的,酒菜虽然不错,但也没有什么特色,听说老板也是地地道道的大雍人,怎么会有出色的川菜呢?‘

夏金逸故意惊讶地道:‘咦,你们不知道么,小弟游走天下,那西子楼的何老板我可认得,他是青城派的何铁山,剑法很不错的,嘻嘻,大师兄,你知道小弟后来拜得师父是天都观的道士出身,所以小弟也常常卖些膏丸药散什么的,说来也巧,就认得了老何,那是五六年前的事情了,听说那是他是蜀国那个王爷的总管,想不到如今成了大老板,不过说句实话,咱们江湖中人钱是要赚的,像他这样做这种逼良为娼的生意的人倒还真是少见。‘

包括张锦雄在内的所有侍卫脸色都变了,张锦雄沉声问道:‘你没认错人么?‘

夏金逸笑道:‘怎么可能,师兄你知道我的本事,小弟刚来长安的时候,在关中联安身,有一次出去闲逛,在西子阁门前见过何老板一面,只是那个地方一掷千金,小弟实在是囊中羞涩,所以没有进去。‘

看着张锦雄铁青的脸色,夏金逸心中十分好笑,这话么,七分真,三分假,这何老板他是见过的,蜀中他也是去过的,甚至就连卖药的事情也是真的,可是何老板从来没有买过他的药,他也不知道这位何老板居然是青城派的高手,蜀国王爷的总管,不过是从雍王府的江大人提供给他的资料上得知了一些罢了。

看屋子里面气氛不对,夏金逸不等师兄逼问,立刻招呼再上酒菜,一坛美酒刚刚送上,张锦雄正要继续盘问的时候,突然外面传来一个冷肃的声音道:‘夏金逸,快出来受缚,本座免你一死。‘

夏金逸露出惊慌的神色,看向张锦雄,张锦雄看了他一眼,扬声道:‘什么人在外喧哗,本座四品带刀侍卫,太子驾前侍卫总管张锦雄在此。‘

门外传来惊讶的呼声,接着有人高声道:‘本座四品带刀侍卫,雍王府副总管胡威在此,张大人,本官奉命前来捉拿冲犯天策帅府司马江大人的要犯,风流浪子夏金逸,张大人缘何在此。‘

张锦雄狠狠的瞪了一眼夏金逸,夏金逸面色苍白,连连作揖,张锦雄冷冷道:‘胡大人,请进来说话。‘

门开了,走进来一个相貌威武,神色冷厉的大汉,他一身锦衣官服,对夏金逸看也不看一眼,上前对张锦雄施礼道:‘张大人,本官奉命前来拘捕恶徒,还请行个方便。‘

张锦雄神色平静地道:‘我这个师弟虽然胡闹,但又怎会劳动胡大人至此呢?‘

胡威道:‘张大人有所不知,本官现在奉命护卫帅府司马江哲江大人,昨日大人在此饮酒,恰逢夏金逸和关中联冲突,令师弟居然栽赃嫁祸,害得大人几乎和关中联冲突,大人下令定要将令师弟擒获,送到关中联去,所以还请张大人行个方便。‘

张锦雄心里一沉,他自然知道天策帅府的司马,那是雍王麾下数一数二的重臣,自己一个小小的侍卫总管,可是挡不住的,可是看看师弟苍白的神色,他冷冷道:‘夏师弟既没有犯王法,你们也不是京兆尹,还没有拘捕他的资格,况且若是让你们当着本官的面把他带走,本官还有什么脸面留在太子驾前。‘

胡威也是眉头紧锁,殿下吩咐,司马大人的命令必须遵从,若是自己违背,只怕不免遭到责罚,可是张锦雄说得也有道理,雍王府和太子府的水火不容,人人都清楚,若是张锦雄就这么让自己带走了人,只怕削了太子的面子,这样一来,雍王恐怕也会不满自己的行为,想来想去,还是没有办法。他看了张锦雄一眼。张锦雄虽然性子端正,但是不是蠢人,他也看出了胡威的为难,想了一想道:‘也罢,这样吧,我这个师弟先让我带回去,绝对不会让他离开长安,过几日,我亲自去关中联调节此事,再让他去向江大人赔罪,只是今日绝对不能让你把人带走。‘

胡威想想,这也是唯一的办法,便道:‘既有张大人担保,本官就先放他一马,待我回去禀明司马大人,再作打算。‘说罢施礼告辞,张锦雄也亲自相送,毕竟现在雍王和太子还没有撕破脸,面子上的事情还是要顾全的,而且张锦雄本身也不是失礼的人。

送走了胡威,张锦雄怒视夏金逸,道:‘这下可好,上次你说得含含糊糊,原来你竟然重重得罪了雍王府,这可怎么收场。‘

\chapter{第七章 锦绣前程}

夏金逸连连苦求,跟师兄把事情说个清楚明白,张锦雄冷静下来,道:‘也罢,看来得先摆平关中联才行,可是我虽然是太子的侍卫总管,但是想要摆平关中联还不行,除非太子出面,可是怎么可能让太子管这件小事呢?‘

夏金逸眼巴巴的看着师兄,张锦雄看了看他,又是生气,又是发怒,终于叹了口气道:‘你原本也是一个人才,就是不能出类拔萃,也不会比别人差的太多,就为了一个女人,自暴自弃,唉。‘夏金逸神色变得越发苍白,坐在房间角落不再出声,神情木然。

这时一个侍卫突然道:‘总管,你也别担心,刚才夏兄弟说得事情咱们不妨仔细查一查,如果真的有问题,咱们禀告太子,夏兄弟若是立下这样的大功,总管跟太子殿下求个情,这件事情很容易解决,雍王府那边其实问题不大,只要夏兄弟跟关中联和解,难不成他们还会管闲事么?到时候夏兄弟再去请个罪也就成了。‘

张锦雄神色一动,道:‘我几乎忘记了,好,金逸,看看你有没有这个运气了。‘

夏金逸神色茫然地道:‘怎么了,大师兄,小弟立下了什么功劳?‘

张锦雄笑道:‘走,说来话长,到时候你就明白了。‘

太子殿下对长安内外的控制还是很有力的,不过一夜的时间,李安面前就得到了详细的情报,看看自己的心腹智囊,少傅鲁敬忠,李安问道:‘鲁大人,你看孤应该如何处理这件事情。‘

鲁敬忠捻了捻胡须,慢条斯理地道:‘殿下,情况很清楚,锦绣盟是蜀国的余孽,从前他们活动范围主要在蜀中和南楚,所以咱们大雍对他们的事情也是睁一只眼闭一只眼,如今南楚局势一片混乱,他们趁机加速了在大雍的活动,依属下看来,他们其实成不了什么大气候,只知道暗杀谋刺,逼迫贫民,而且还贩卖人口,本来么,太子也不用急于处理他们,说不定还能用的上他们,毕竟灭了蜀国的是雍王,可是既然现在他们不满足在南楚折腾,那么迟早也会在大雍发难,依臣的意思,马上就是元旦祭天了,太子不是正在争取替皇上告祭太庙么,皇上还没有决定,不如太子安排一下,把他们一网打尽,然后上表向皇上请功,再让纪贵妃在皇上耳边多说几句好话,那么太子心愿必然能够实现。‘

李安听了大喜,道:‘少傅果然是孤的智囊,这件事情一定要办的利索,不能让雍王知道,占了便宜,若是孤可以替父皇告祭太庙,那么谁还敢说孤的储位不稳。‘

鲁敬忠笑道:‘那么臣就先恭喜殿下了。‘

李安大笑道:‘李贽啊李贽,你平了南楚,进位天策元帅又怎么样,孤只要坐稳储位,你终究只是一个臣子。‘说得最后,李安已经是咬牙切齿了。

鲁敬忠看看失态的太子殿下,眼中闪过一丝轻蔑。

在鲁敬忠的精心布置下,三天之后锦绣盟在长安的分舵被扫平了,全部人员几乎一网成擒,一些下级官员被牵连,这是一次成功的围剿。经过审问,得知锦绣盟为了从混乱的南楚进一步得到利益,实现复国的目的,短期倒没有对付大雍的计划,但是鲁敬忠授意重刑逼供,很快就得到了锦绣盟想要在新年之际发动,谋刺大雍皇室子弟和文武重臣的供词,当然为了达到目的,鲁敬忠精心的替锦绣盟准备了一份详细的行刺计划,一时之间,铲除锦绣盟,挫败蜀国余孽的谋反的事情沸沸扬扬,为新年之前的盛况增加了一些血色的光彩,先不论各家势力的反应,我趁机让寒无计吧曾经威逼过他的那个旧识和另外一个锦绣盟总盟派来的一个使者收留了起来,据我所知,这个使者是锦绣盟主霍纪城的心腹,通过他,我可以和锦绣盟搭上关系,根据我收集到的情报,霍纪城本是蜀国重臣之子,生性冷酷无情,颇富智谋,只是性子有些傲慢,心胸狭窄,缺少军略才能,要不然锦绣盟不会在大雍欣欣向荣,在南楚却举步唯艰。本来锦绣盟的事情我并不想过问,可是柔蓝已经是我的女儿,我可不想将来要她去拼命报仇,还是我来吧。

接到太子的表章,李援十分高兴,虽然坚持立长子为储君,但是李安的平庸还是让他有些不满,这次李安行动迅速果决,铲除了一个毒瘤,李援放下表章,对丞相韦观说道:‘韦爱卿,看来太子还是有才干的,只是以前没有什么表现的机会罢了。‘

韦观躬身道:‘陛下说得是。‘

李援笑道:‘爱卿,有很多朝臣上表请求让太子这次跟着朕一起告祭太庙,卿认为如何?‘

韦观正容道:‘太子立为储君已经有多年,上承天命,为社稷宗祀所系,理应随天子同祭天地,臣意也应该如此,如今雍王受封天策元帅,恩宠已极,陛下也应该稳固太子的地位,免得发生变乱。‘

李援点点头道:‘卿的意见很中肯,这样吧,这次就让太子替朕到太庙告祭,然后在文华殿接受百官朝拜,这就替朕拟旨。‘

雍王李贽得知这道旨意之后,面沉如水,径自走向王府西侧的寒园,江哲嫌栖凤轩太华丽,在王府外府转了一圈,挑中了这处最偏远的园子,打理了一番住了进去,走近园子,李贽看到隐在暗处的侍卫,满意的点点头,他特意吩咐加强了这里的守卫。走进园门,李贽看到原本荒草离离,花木杂乱的园子已经整理的颇为雅致,满意的点点头。这处园子原本也是客院,可是因为位置偏远,除了照顾园子的两三个仆人侍女和巡夜的护卫之外没有人注意这里,所以渐渐的园林荒废了,这次江哲转了一圈,看到这里之后,常总管原本十分羞愧,还要重责照看这里的仆人,却被江哲婉言劝止了,江哲喜欢这里的清净,便住了下来。他性子和顺,只要没有过多的人来回走动就很满足了。李贽很细心,特意让总管询问李顺江哲的喜好,好让江哲住的舒舒服服。所以李贽虽然心里焦急,但是看到雅致的环境和训练有素的仆人还是微微的笑了。

走进江哲的起居之处,李贽看到江哲正在棋台前看着棋局,江哲性子闲散,这间颇为宽敞的起居室是前厅后卧的格局,中间隔着一扇屏风,隐隐约约可以看见后面雅洁的床榻和帐幕,前面的小厅不算太大,但是棋台、琴台、软榻、书架、书案无一不全,十分舒适清雅,带着浓厚的书卷气。

看到李贽进来,和我对弈的小顺子站了起来躬身行礼,我却在那里绞尽脑汁的想着下一步棋应该怎么走,唉,当初是我教他下棋,可是我现在和他下棋是输多赢少了。这时我听到小顺子说道:‘公子,殿下来了。‘

我抬起头,看到李贽愁容满面,惊讶地道:‘殿下,发生了什么事情?‘

李贽坐在我对面,苦涩地道:‘父皇下诏,今年让太子替他告祭太庙,这样一来,太子储位稳固,先生,你说本王该怎么办呢,唉,先生,锦绣盟在长安的势力被太子铲除,我们若是早些行动就好了。‘

我笑道:‘殿下怎么忘了,我们原本不就是要示弱么,如今太子储位稳固,正好让他得意,欲取之,先与之,这不正好么,殿下不必忧虑,新年之后,就派石先生护着世子到幽州去,这样皇上和其他人都会以为殿下为后路打算,这样太子就会放手逼迫殿下了,而皇上就会体念殿下的功劳,对殿下开始回护,这样一来,殿下安全无虞,而太子的忍耐力就会下降,甚至对皇上生出怨恨。‘

李贽也是聪明人,立刻明白过来,道:‘原来这些早在先生掌握之中,本王倒是过于焦急了,先生,你说的人已经安排妥当了么?‘

我淡淡道:‘殿下请记得,我们从没有安排过什么人?殿下是臣,太子是君,臣子怎么会在太子身边安插密探呢?‘

李贽会意的笑了,道:‘先生,新年父皇必然大宴群臣,先生已经是天策帅府的司马,四品官员,已经可以面君,而且父皇提过想见见你,不知道先生意下如何?‘

我原本不感兴趣,但是想到趁机见见太子和群臣,对于我来说十分有用,便点头道:‘臣也想见见朝中的文武俊杰。‘

在城中一处民宅的密室之内,两个面色阴沉的中年人一坐一卧,神情悒郁,坐着的中年人看着昏暗的油灯,突然道:‘弓老大,你那个朋友真的靠得住?‘

躺着的中年人笑道:‘刘头儿,你放心,姓寒的以前是军中的密谍,杀人如麻的刺客,虽然不知道他怎么发了财,可是你看,没有他通风报信,没有这件密室,我们早就进了大雍的天牢了。‘

坐着的中年人叹了口气道:‘话虽如此,可是你看,咱们现在朝不保夕,他却是富家翁,他这样热心,我总是放心不下。‘

弓老大正要反驳,密室的门开了,一个清朗的声音传来道:‘刘统领,你的疑心还是真不小,寒某如今富甲一方,若不是念在同是蜀国的臣民,谁还会管你们的闲事,你们可知道,若是给主上知道,寒某就是性命无虞,只怕也要脱一层皮啊。‘

刘统领连忙站起身道:‘是小弟失言,抱歉,不知道寒兄弟的长上是哪一位?‘

寒无计笑道:‘这也不是什么秘密,寒某如今是天机阁总管,我们阁主没有别的爱好,就是喜欢金银财宝,所以不管什么地方,不管什么人,只要有钱可赚,就有我们的影子。‘

刘统领眼睛一亮道:‘原来是天机阁,谁不知道天机阁在南楚的潜势力,只怕南楚的大商人十有三四都是天机行会的成员,想不到寒兄在天机阁地位如此尊贵,真是佩服、佩服。‘

寒无计淡淡道:‘也没有什么,说句实话,这天机阁里面迷雾重重,我虽是总管,其实只是一个出面办事的人,真正的大权并不在我手里,不管金钱上面的事情,小弟倒是可以做几分主,其实小弟有心和贵盟做笔生意。‘

刘统领神色一动,道:‘寒兄请讲,只要对我们锦绣盟有好处,小弟回去一定极力促成。‘

寒无计神色有些诡秘,道:‘贵盟想要造反,恐怕急需武器粮饷,若是小弟可以帮忙,你们怎么说?‘

刘统领大惊道:‘什么,你真的可以帮忙,若是如此,我们盟主必然重重相谢,若是我们成就大事,将来必有寒兄的好处。‘

寒无计笑道:‘你也知道,我们天机阁在南楚的势力,最近南楚那些大臣已经立王三子赵陇为国主,明年年初就要即位,现在南楚百废待兴,而雍军肆虐将近半年,又劫掠建业,说句不好听的话,国库都要被搬空了,军械物资更是损失惨重,无力补充,可是南楚毕竟是鱼米之乡,粮食今年产量还是很不错的,现在是南楚缺钱、而大雍虽然战胜,战利品也丰富,可是大雍今年有些干旱,所以缺粮,你们若有胆量,走通了门路,从大雍盗卖军械马匹,然后到南楚换取粮食棉布,卖回大雍,不仅可以满足你们自己的需要,还可以大赚一笔。‘

刘统领皱眉道:‘这恐怕不大容易,现在我们刚刚在大雍受了损失,只怕没有这个能力。‘

寒武纪笑道:‘谁不知道太子的作为是给雍王看的,这笔交易是十倍、二十倍的利润,你们只要派人去向太子输诚,就说情愿效忠太子,求太子网开一面放过你们,只要太子不追究,谁还会盯着你们不放,现在户部是太子的天下,大雍军方的后勤可以说被太子控制,只要太子首肯,这桩生意容易得很,等到过些时日,太子在户部动动手脚,不是就补上了么,到时候上百万两的雪花花的银子进了太子自己的口袋,他有什么不满意的。‘

刘统领皱眉道:‘大雍都是他的,他还会重视这点儿银子。‘

寒无计嗤笑道:‘谁不知道,现在太子上有皇上看着,下有雍王虎视眈眈,你别看他身份尊贵,这享受恐怕还不比我们这些商人,而且,难道他就不想自己畜养一些死士谋士,他用钱的地方多着呢,说句实话,在南楚,我们有的是法子,但是在大雍,就得看你们得了,别瞒我,这次虽然牵连了一些官员,可是太子没有下狠手,你们真正的靠山安然无恙。‘

刘统领狠狠的点点头道:‘你等着,我回去和盟主商量,虽然刘某地位不高,可是盟主对我很信任,不过,我怎么找你。‘

寒无计道:‘联络方式我会给你,我们主上只要发财,不管什么国家大事,你们谋逆也好,复国也好,只要不伤害我们的利益,什么都好说。‘

刘统领道:‘寒兄放心,我们也不是蠢人,金银财宝哪个不爱,更何况这条路子走通,对我们的好处更大。‘

两人相视而笑,笑声压抑而诡秘。

在雍王府的寒园之内,我倚在软榻之上,看着手里的文卷,因为觉得王府的机密书房太拘束,所以近来我每天只到那里待上半天,然后就在寒园之内筹划计策,小顺子看我想得出神,突然道:‘公子,你让天机阁介入,这样好么?‘

我听到他的问话,淡淡道:‘没法子,这件事情将来是肯定要出问题的,若是雍王的人去做,不说瞒不过太子的耳目,惹祸上身,你说雍王能够允许盗卖军械物资么?‘

小顺子忧虑地道:‘公子安排天机阁联络南楚商人,锦绣盟联络大雍的太子,然后走私粮食军械,这样将来天机阁只怕就不能出面了。而且公子和天机阁之间的关系怕也瞒不过雍王,表少爷会不会受到牵连。‘

我轻笑道:‘你怕什么,名义上,天机阁会在出事之前将自己所占的股份全部转卖,这一点我已经让他们安排,将这些股份分别让秘营的人接收,在他们和我的约定期满之前,收益仍然归我,期满之后,这些产业就是他们自己的了,这样也实现了当初我对他们的诺言。反正天机阁本来就是赚钱的工具罢了,这次之后,我所有的产业,扣去分配给秘营弟子的部分,也能有百万身家,天机阁也就不用存在了。‘

小顺子笑道:‘还是公子高明,只是走私粮食军械,只怕瞒不过雍王。‘

我淡淡道:‘等到雍王发现,我会让他隐忍,若没有这个把柄,我们凭什么废掉太子呢?‘

坐起身来,我推开窗子,看向漆黑的天空,冷冷道:‘我江哲用计,凭的就是人心险恶,太子若是没有私心,一心为国,我这个计谋自然行不通的,小顺子,你记着,人若覆顶,不是水不能载身,而是自己心术不正,若是太子真的贤德,有一国之君的气度,我的计策根本没有用,若是他因此失去宝座,不是我心狠,是他没有做天子的福气和雅量。‘

\chapter{第八章 新春华宴}

大雍武威二十四年甲戌,帝颁诏令,令太子安代陛下告祭太庙,受百官朝拜于文华殿,雍王恐惧,同年二月,雍王上书,求就藩幽州,帝不许,令以世子代之,继而,雍王告病免朝,帝许之。

--《雍史·太宗本纪》

南楚同泰元年甲戌,镇远侯陆率百官拥王三子陇为国主,改元同泰,遥尊炀王为太上,奉尚妃为太后,垂帘听政,国事委于陆侯,新主登基,下诏晋封信为镇远公,遣使大雍,纳贡称臣。

--《南朝楚史·楚愍王传》

在一片歌功颂德和莺歌燕舞当中,新年元旦到了,这一天可真忙碌,先是大朝,百官先到太极殿向雍帝李援朝拜,然后再到东宫文华殿向太子朝拜,太子虽然在皇城有自己的府邸,但是象征着储君权威的东宫一直没能入住,直到今年因为各方面的支持,李安才正式入主东宫,坐稳了储位。当雍王作为百官之首到东宫朝拜太子的时候,行了二跪六叩大礼的时候,在天下人的心目中,李安已经是名副其实的储君了。看着一向让自己自惭形秽的雍王李贽在面前叩拜,李安心中涌起滔天的喜悦。

之后,李安又完成了代天子告祭太庙的大典,这一刻,李安完全沉醉在天下臣服的喜悦当中。

比较起来,雍王李贽的神情不免是有些冷淡的,君臣名分既定,也难怪他如此,没有人想到,李贽此时,只能赞叹江哲的计策,他可以看得出来,李安已经飘飘然了,完全压倒自己的喜悦让他好几次都几乎出了差错,那么只要计划得宜,自己就可以让太子万劫不复,欲取先予,说来容易,但是能够设计这样大胆的计策,真是胆量过人啊,到现在为止,李贽也不知道江哲的具体打算,甚至弄不清楚江哲的用意。只觉得江哲的计划似乎环环相扣的罗网,而李安,就是逐渐陷落罗网的那只蝴蝶。

告祭太庙之后,李援传旨设宴甘露殿,大宴群臣,我随着雍王入席,雍王自然要忙着和群臣交杯换盏的,石彧和我坐在角落里面,他低声为我指引朝中的重要人物。

石彧低声道:‘文官首席的那位就是丞相中书令韦观,他是皇上的臂助,当年皇上和雍王都在外征战,朝中由太子监国,但实际上的政务全靠他一力主持,为人心机深沉,十分懂得事君之道,所以多年来身在中枢,荣宠不衰,不过这几年他年纪也大了,朝中争储又很混乱,所以他明哲保身,不怎么发表意见,但是据我们所知,他是比较倾向太子的,因为毕竟和太子共事多年,但这人不会真正介入纷争,如果一旦我们成功了,他也不会有什么反对意见,殿下的意思,稳住他就可以了,但是不可以以他为援。他下面第五席的那个官员是侍中郑瑕,此人忠直敢谏,当年庆王刺杀纪贵妃,很多人上表要求诛杀庆王,以惩起逆伦刺母之罪,此人当面直谏,说庆王殿下刺杀贵妃虽然有些不妥,但是也是为生母报仇,不论此仇该不该报,也没有为此治罪的道理,若是有罪,也不是逆伦,因为纪贵妃并非嫡母,而皇上也对庆王有歉意,这才把庆王打发到外面就藩。这人将来恐怕要跟我们作对的,但是殿下说若是能够以大义说服他,那么此人就是难得的名臣。‘

我看看韦观,相貌平平,星霜两鬓,难得的是神态雍容,果然有统率百官的气度。那郑瑕却是方面大耳,目若寒星,不过三十多岁的年纪,一举一动却带着隐隐的威严气势。只看这两人,就把南楚那些官员都比了下去,大雍成为中原霸主,理所当然。

石彧又道:‘太子身边的那一位,就是太子少傅鲁敬忠,此人虽然相貌平平,但是文章典制十分精通,所以才作了少傅,但是这人外貌忠厚,心实奸诈,是太子手下的第一谋士,我们吃了不少他的亏。‘

我看向鲁敬忠,这人目前是我最大的敌人,看去相貌果然平凡,只是肤色有些过于白皙,那双眼睛总是半张半阖,似乎有些睡不醒的模样,我正在打量他,鲁敬忠似乎有些察觉,双目一张,寒芒如电,向我望来,我连忙低下头去,感觉到冷厉的目光从我身上闪过。

石彧却是回以微笑,鲁敬忠看是石彧,似乎放下心来,遥遥举杯相敬,石彧微微一笑,也举起了酒杯。两人都是一饮而尽。

等到鲁敬忠的目光移开,我才轻声道:‘此人果然不凡,多谢石兄为我解围。‘

石彧淡淡道:‘我和他可算老对手了,所以他不会注意你的。看,那位向陛下敬酒的是魏国公程殊。此人曾经救过皇上的性命,军略上倒也平常,却是一员福将,每战若是胜利必然大胜,若是战败也总是能够全军而退,而且个性轻财重义,爱交朋友,大雍的骄兵悍将最敬重的或者是雍王,但是最亲近的人就是程殊,他若想办什么事情,不用兵部的文书,只要一封书信,只怕没有人不买账,他对殿下倒是很看重,从前就多方维护,对太子不大买帐,但是他人缘好,皇上又宠信,所以太子拿他没有办法。这人对皇上也是一片忠心,让他帮助殿下恐怕不成,但是若殿下登上皇位,他必然是乐观其成。‘

我看向那位神态慵懒,举止有些粗鲁,但是周身上下洋溢着亲和力的将军,虽然已经到了知天命的年纪,但是须发乌黑,神情之间没有一丝倦怠,见他敬酒,李援笑着举杯,君臣之间其乐融融,果然不是平常人物。

石彧又道:‘我大雍军方现在实际上有四大派系,雍王殿下麾下的四十五万大军是力量最雄厚的,很多现在军方的名将都在殿下麾下,不过现在基本上都在外面镇守,所以你没有看到。除此之外,齐王二十万、庆王十万,这些军队虽然没有雍王的兵将精良,但也是精锐,另外一大派系就是秦程系,抚远大将军秦彝和魏国公程殊共同掌握着十五万禁军,二十万边军,换句话说,他们是皇上最信任的将领,是皇上压制诸位皇子的护身符。现在三位皇子,庆王没有能力争储,齐王和殿下水火不容,有了秦程两人的三十五万大军,皇上就可以稳如泰山。‘

我看看武将之首的秦大将军,相貌斯文俊秀,须发灰白,好似文人儒士,但是只见他精神矍铄,谈笑风生,就知道他虎老雄风在,难怪是雍帝最倚重的大将。

这时石彧说道:‘随云,你看,那人虽然声名不现,可是你得记住,他是中书侍郎秦无期,此人平日只是尽忠职守罢了,可是在中书省竟然呆了九年,陛下的诏书十有六七都是他的手笔,而且你记着,齐王妃秦铮就是他的长女。‘

我心中一凛,看向那个斯文的儒生,淡淡道:‘莫非此人和凤仪门有关。‘

石彧笑道:‘随云果然精明,据说此人青年时曾经受过凤仪门主的大恩,所以一直感恩图报。‘

我将此人记在心中,然后淡淡道:‘要见的人都见到了,总算不虚此行,石兄,等一会儿宴席散后,我要先走一步,明后几天,我可要好好休息,你呢?‘

石彧神情诡秘地道:‘你恐怕休息不成啊,从初二开始就有好戏呢?‘

我微微一愣,看向石彧。他笑道:‘皇上今年兴致好,午宴之后,下令在朱雀门外演武较技,凡是大雍四品以上官员或者世家子弟,凡是未满三十岁的青年,皆可报名参加演武,若是取胜,陛下要重重封赏,听说较技分为三种,第一种是赛马,第二种比试箭法,第三种乃是比试拳脚刀剑,若是任意一种取得魁首,就可以光宗耀祖,这样的盛况你怎能不去看看。‘

我笑道:‘原来还有这样的好事,我可真的要去看看,可惜我不擅骑射武技,没有参赛的可能了?‘

石彧笑道:‘这件事情早就传开了,看来你是两耳不闻窗外事啊,这次朝中大小官员,还有长安百姓早就开了赌局,报名的名册早就天下皆知,程将军,就是魏国公还亲自坐庄呢,他老人家可是最公道的。当然,长安几大赌场也都开了赌局。‘

我苦笑道:‘虽然可以压注,可是我对武技骑射都不精通,而且对那些上场较技的人也很陌生。‘

石彧笑道:‘你怕什么,若论对这些参赛之人的了解,只怕雍王府若是认了第二,没有敢认第一,包你不赔就是。这次有三个人是热门人选呢?一个是韦相四子韦膺,现在是吏部郎中,据说今年就可以升为吏部侍郎了,他虽然是文官,可是他擅长马术,韦相家中又有一匹汗血宝马,所以赛马夺魁的可能最大;一个是抚远大将军次子秦青,他是大雍的虎威将军,骑射传自家学,在大雍青年将领中首屈一指,最后一位是礼部尚书夏侯阑之子夏侯沅峰,此人有长安第一美男子之称,有潘安宋玉之美,武功很强,现在是御前二品带刀侍卫,大内副总管,是皇上最宠爱的侍卫,据说此人武功高深莫测,是大内青年侍卫中的第一高手,师承不详。‘

我淡淡道:‘大雍俊杰果然不少。‘

石彧见我有些不悦,莫名其妙的住了口,转念一想,知道我必是想起了南楚文恬武嬉,但他知道不可说破,只是转了话题,又给我介绍一些其他的官员。

正在我们窃窃私语的时候,旁边传来低声的警告,我抬头望去齐王殿下正和一个年轻官员一起走了过来,我和石彧连忙站起,李显走到我二人面前道:‘群臣欢宴,现在都在相互敬酒,怎么两位却在这里密谈啊?‘

石彧从容道:‘殿下,江司马初来大雍,对朝中的事情还不清楚,所以臣为他简单介绍一下,而且我们官卑职小,怎敢放肆,韦大人,这位是江哲江随云,天策帅府新任司马,随云,韦膺韦大人是大雍二十一年辛未科的状元,现任吏部郎中。‘

我从容见礼,只见韦膺大约二十五六的年纪,和我相仿,此人长得相貌清秀,容色雅逸,举止之间,自有一股出尘脱俗之气,虽然是富贵人家的子弟,又是年少显贵,但是却丝毫不带一丝傲慢,让人一见便心生好感。

韦膺原本是听齐王说江哲江随云已经归顺大雍,如今也在甘露殿上,所以一时好奇,请齐王引见,他三年前状元及第,自然是欣喜若狂,可是常常听人说,若论文章锦绣,还数江南人物,而其中之最就是南楚显德十六年(大雍武威十七年)丁卯科状元江哲,此人文才风流,冠绝南楚,一首《月下感怀》天下闻名,一曲《破阵子》迫死蜀王,早被誉为南楚第一才子,只是攻蜀之后似乎卧病不起,外面才渐渐少见他的诗词,韦膺曾经将能够收集到的诗词文章抄录下来,每每爱不释手,今日一见江哲,韦膺顿觉名不虚传,这个比自己大上一两岁的青年虽然有些清瘦,相貌也不如自己这般俊秀,但那种从容自若、温和中带着冷漠的气质,让韦膺生出惺惺相惜的情感。

韦膺上前施礼道:‘久闻江兄才情冠绝当代,今日一见幸何如之,后进韦膺,见过先生。‘

我神情微动,想不到这位丞相公子真的如同外貌一般谦逊,便再次还礼道:‘苟活之人,不敢当韦大人之礼,大人既是大雍状元,才学也自然不凡,若有机缘,哲当向大人请教。‘

韦膺喜道:‘若江兄肯赐教益,韦膺感激不尽,后日有暇,膺当登门拜访。‘

我们这里你一句我一句的相互谦让,李显可听得不耐烦,他原本想文人相轻,若是韦膺过来,不免会讽刺江哲几句,不料两人竞一见如故,这可不好,他心思灵敏,立刻叫道:‘秦青,你过来一下。‘

一个青年将军应声走了过来,我仔细看去,这位青年将军相貌和秦彝有些相似,只是肩宽腿长,身材俊伟,不像其父一般儒将风范,他和齐王似乎很熟悉,笑道:‘殿下找我什么事?‘

齐王指着我道:‘这位就是逼死蜀王的江哲江随云,你不是说想见识见识么?‘

秦青看了我一眼,眼中闪过一丝讥讽,突然高声道:‘昔日江大人一曲破阵子迫死蜀王,想是没有想到今日自己也会屈膝投降吧?‘他的声音很响亮,让甘露殿突然沉默了下来,所有的目光都集中过来。

石彧和韦膺的脸色都变了,但是众目睽睽之下,也无法出言相助,我却神色从容,朗声道:‘蜀王失国丧邦,以死相殉也是应当,哲虽做歌相送,也是蜀王知耻,才成此佳话,南楚国主为陛下爱婿,亲切当如父子,我未听过有父亲责罚,儿子自裁的。况且身为臣子,屡进忠言却遭到贬斥,还会为君王家邦殉死的自古至今只有屈原一人,不说南楚国主尚在,就是国主遇难,哲若以身相殉,则哲于青史上流芳万古,却让后人视我主如楚怀王,乃以君上之辱,而彰臣节,非我所为也,况且若我主为怀王,将军岂不是视陛下为秦惠王,秦二世而亡,我不知将军希望大雍传承几代呢?‘

我这一番话,听得秦青面色铁青,韦膺满面惊叹,石彧低头暗笑,李显眼中却是又嫉妒又羡慕的神色。我们这边僵住了,却有人大声鼓掌叫好。

众人应声望去,却见李援正在鼓掌叫好,顿时都放下心来,雍王正在皇上身边,笑道:‘秦青,你吃亏了吧,父皇,这位就是南楚第一才子江哲,江哲,还不过来拜见陛下。‘

我从容上前行礼,不卑不亢,李援笑道:‘好,朕早就听说你的才名,你能够弃暗投明,朕甚是喜欢,听雍王说,你身体不好,总是卧病在床,若非如此,朕还想让你到中书省做个舍人,代朕草诏呢。‘

我淡淡道:‘臣幼时体质便十分羸弱,昔日从军又染了疾病,虽然病愈,但是病根尤在,雍王殿下念臣体弱,留在身边奉养,这是殿下的恩德,也是臣的荣幸。‘

李援更是高兴,道:‘好,这也是一段佳话,你不可因为秦将军之言而气馁,好好的做事,我大雍绝不会亏待四方的贤士。‘

我再次拜谢。李援挥手让我退下,雍王也跟着告辞。雍王拉着我走到秦青身边,道:‘秦将军,江司马,你们都是青年俊杰,不可互生嫌隙,就让本王作主,你们两人和解吧。‘

秦青原本面红耳赤,见雍王相劝,便趁机下台,向我道歉,我也还礼如仪。

这时有人在我身后笑道:‘好啊,总算让我见到秦兄服软了。‘

我们转头看去,却是一个身穿锦衣的俊美少年,这个少年不过二十二、三岁的年纪,相貌无比清秀俊雅,直如宋玉潘安,更兼身材修长,宛如临风玉树,整个人看起来,倒好像是一尊玉人雕像般精致。这人未语先笑,道:‘好个南楚才子,真让我夏侯沅峰佩服。‘说罢上前深施一礼,我不卑不亢的还了一礼,微笑不语。

这时几乎全场的目光都集中了过来,雍王、齐王、韦膺、秦青、夏侯沅峰都是足以吸引所有人目光的人物,如今这样站在一起,顷刻间仿佛甘露殿上所有的光芒都集中在这里。让那些大臣瞩目的是,站在这些大雍俊杰身边的江哲,既没有显赫的身份,也没有出众的相貌,更没有逼人的气势风范,却是奇迹一般的在他们心底留下了深深的痕迹,那是一种仿佛青山绿水一般的存在,不论其他人光芒如何强烈,也掩饰不住那林间清泉一般的从容淡雅。

\chapter{第九章 演武较技}

南楚同泰元年甲戌元月,哲以雍王属臣,列身大雍朝堂,雍帝召宴群臣,初二,帝令青年才俊较艺于朱雀门外,帝择其优者封赏,实为长乐公主择婿也,其中虽多英杰,公主唯沉默以对,赛终,帝问公主心属,公主泣曰,儿夫健在,焉能再嫁。帝初时大怒,继而黯然。长孙贵妃忧虑,多方抚慰,公主默然,后贵主暗问宫婢,宫婢禀告,公主观战于楼上,对他人皆不留意,唯见雍王司马而喜,贵妃乃悟。

--《南朝楚史·江随云传》

一双黑溜溜的大眼睛小心翼翼的盯着那个‘高大‘的身影,没有动,应该是睡着了吧,趴在地上,两只白嫩的小手交叉向前,借助膝盖的力量,飞快的向前移动,近了,更近了,小手一把抓向目标,绝对是快如闪电,谁知道有人动作更快,眼前一晃,自己的目标被人夺走了,‘啊‘小小婴儿哭得惊天动地,接着一双手手忙脚乱地把小女娃儿抱了起来,又是威胁又是劝哄,小女娃儿却一点面子也不给,直到另外一只手把那个软木雕刻而成的,外用锦缎蓄棉包裹的大头娃娃放到小女娃儿面前,小女娃儿才破涕而笑,一把抱向几乎和自己一样大的娃娃,咿咿呀呀的表示欢喜。

擦了一把头上的冷汗,小顺子道:‘公子,你也不用总是欺负小姐吧,若是王妃知道了,一定要责怪你不够稳重。‘

我下意识的缩了缩脑袋,昨天我不过是故意拿着玩具引柔蓝追我,好锻炼她的反应能力,就被王妃叫去,隔着帘子训了一顿,今天若是让王妃知道我弄哭了柔蓝,岂不是更惨,连忙看看,那个小耳报神在不在,不在,我满意的点点头,世子李骏因为马上就要代雍王镇守幽州,所以今天被雍王叫去了,这可是我提醒雍王的结果,要不然,这小子总站在旁边监视我,昨天就是他向王妃告状。

不过还有一个障碍,我看看小顺子道:‘小顺子,你还是去看看演武较技吧,看看他们武功怎么样,谁最可能获胜,这件事情可是关系很大。‘

小顺子淡淡道:‘殿下不是已经派人去了么?‘

我被他噎住了,连忙道:‘我不是信任你么?‘

小顺子意味深长地道:‘公子不是想着欺负柔蓝小姐吧?‘

我连忙摇头道:‘没有,没有,柔蓝是我的心肝宝贝,我怎么会欺负她呢?‘

小顺子一笑,道:‘那奴才就去看看,公子,你可得记住,若是王妃生了气,恐怕你又有好日子过了。‘

看着小顺子的背影,我一脸的狞笑,走向玩的不亦乐乎的柔蓝,口中说道:‘小蓝儿,爹爹来陪你玩儿了。‘

小女娃还不知道自己身处险境,抬起头,扔下娃娃,张开双手要我抱抱,我一愣,一股暖流从心底涌起,不由把她抱了起来,亲亲苹果一样的小脸蛋,她咿咿呀呀了半天,叫道:‘爹爹。‘我忍不住满心喜悦,抱起她转了几个圈子,银铃般的笑声想起,这可是柔蓝最喜欢的游戏啊。

偷得平生半日闲,我心情舒畅地走进了雍王的书房,雍王果然还在那里看公文,神情虽然平淡,但是隐隐带着不悦。

我上前行礼道:‘殿下,不知道现在外面情况如何?‘

李贽抬起头,看到我,神情松弛下来,道:‘随云,你说谁做长乐的驸马会好一些呢?‘

我想了一想道:‘据臣所知,韦膺、秦青、夏侯沅峰为其中翘楚,臣来长安不久,不知道他们谁更合适一些。‘

那日回到雍王府之后,李贽告诉我这次演武较技是有目的的,原来李援一心想弥补长乐公主,所以想为她择婿,可是现在南楚国主赵嘉还在长安,李援不便公开择婿,所以便借演武之名,让长孙贵妃和长乐公主看看大雍的少年俊杰,好在其中为其挑一个相貌人品都说得过去的女婿,这个消息现在十分隐秘,除了后宫几位娘娘之外无人知道真相,李贽却是从他的王妃高氏那里得知的,这几年长孙贵妃膝下空虚,高氏素来贤孝,李贽又因为提出离间之计,使得长乐远嫁,故而常常让高氏进宫去探望贵妃,这些年两人早就情同母女,所以长孙贵妃才问高氏的意见。

我没有对雍王提及,从我知道这次演武的目的开始,就十分的恼怒,不是因为大雍毫不顾忌国主赵嘉的存在,因为自始至终,长乐公主就没有对国主动过真情,甚至我怀疑当初长乐公主流产也是有原因的,可是虽然我同情长乐公主的遭遇,但是并不赞同她这样的行径,无论如何,国主仍然在世,她就是想改嫁,也不能这样着急啊,至少得等到和国主之间没有了名份之后,再去改嫁。事实上,我一直十分气恼,若非是柔蓝的存在,抚慰了我的心灵,恐怕我早已勃然大怒。平静下来之后,我又觉得,算了,长乐公主是天之骄女,我又何必把她想得太美好呢,或许是当初她大婚之时盛妆之下的珠泪,和行宫觐见时她的温婉可人让我对她产生了同情和好感吧。

现在雍王问及,我尽量用客观的语气来评述这件事情的影响。

看了一眼雍王的神色,我道:‘皇上对公主的宠爱,在有心人眼里就是一道桥梁,若是公主所适非人,不仅现在对殿下不利,而且将来也不免伤了公主之心,这样一来,只怕殿下永远难以得到皇上和贵妃娘娘的谅解,最好的可能当然是公主嫁给殿下属意的人,其次就是嫁给中立一方的人,臣虽然不大清楚这些人实际上的倾向,但是秦彝大将军中立是肯定的,如果公主嫁给秦青,恐怕是最好的选择。‘

李贽面露喜色,但是转而又道:‘你不计较秦青对你的折辱,秉公而论,本王很是钦佩,可是秦青怕是没有可能,当初他和长乐青梅竹马,若非长乐远嫁,只怕他早就成了驸马了,可是我让王妃问贵妃娘娘的意思,贵妃娘娘说,长乐当初远嫁之时,秦青曾经向长乐要求私奔,可是长乐拒绝了,长乐当初对他说道‘本宫乃皇室贵女,又受百姓恩养,岂能不顾江山社稷和国事大局,我若私逃,不仅有损皇家声誉,伤了父皇母妃之心,纵然父皇遣其他宗女远嫁,也不免失去诚意,令南楚离心,两国联姻失败,怕是遗祸无穷,长乐虽然弱女,不敢为此不忠不义不贤不孝之事‘。其实这件事父皇和贵妃娘娘都知道,但他们顾念秦将军的脸面,再说也是怜惜长乐,所以没有治秦青的罪,如今秦青也想贵妃娘娘表示了想和长乐重归于好,可是长乐却是坚决不肯,所以才通过盛典选婿,此事外人还不知道呢?不过秦青恐怕是白忙一场了。‘

虽然不明白长乐公主为何拒绝秦青,但是目前的结果是不得不考虑大人,想了一想,道:‘夏侯沅峰才貌过人,只怕不能对公主体贴入微,而且其父又是*羽韦观大人虽然倾向太子,但是还不至于公然而为,韦膺人品不凡,公主若能得此良配,当是幸福可期。‘

雍王叹息道:‘本王也是如此认为,可是传言太子力保夏侯,他还说动皇后,说夏侯才貌双全,又不涉入朝争,能够好好照顾公主,又说夏侯对公主一见钟情,必然不会因以前的事情而致夫妻反目,韦膺乃是人中之龙,将来仕途显赫已是必然,若是嫌弃公主,不免好事成了祸事,皇后也为他说动,似乎有意夏侯,而皇上也宠爱夏侯沅峰,似乎颇有许可的意思。‘

我神色沉重地道:‘莫非太子有意得到公主的力量,公主受宠,天下皆知,若是夏侯借公主势力,只怕不可遏制此人。‘

李贽苦笑道:‘我也曾想派人加入,但是一来我麾下猛将如云,但是这般文武全才,相貌秀雅的人物却太难找,即使有几个,又都出身不高,何况我若派人前来,恐怕首当其冲的就是秦青,不论是否成功,都会得罪秦青,再说--‘李贽欲言又止,我接着说道:‘再说让人以为殿下贪图公主的势力,没有兄妹之情。‘

李贽连连苦笑,看向我道:‘我虽不想妨碍长乐的幸福,可是她若嫁给了夏侯沅峰,实在是对我不利,你说本王该怎么办呢?‘

我低头回想了一下,道:‘殿下不必忧急,不论皇上和皇后娘娘怎么想,作主的还是公主本人,贵妃娘娘的看法也会影响公主,殿下不如请王妃劝劝贵妃娘娘,我想贵妃娘娘恐怕也不会放心公主嫁给夏侯,毕竟他年纪太轻,不够稳重,公主又是饱经忧患,需要一个体贴温柔,稳重端方的人照顾。‘

李贽大喜道:‘不错,凭心而论,就是不论其他,我也不放心长乐下嫁夏侯沅峰,他年纪太轻了,也太不稳重。‘

初四,朱雀门外,演武正是到了关键的时候,昨日的预赛完毕,今日正是争夺魁首的日子,在门前宽阔的场地上,正是龙争虎斗,而西侧的演武楼上,皇上,皇后、长孙贵妃陪着长乐公主正在观看演武,其他的娘娘坐在后面,这些后宫的娘娘们难得可以出来,所以一个个,兴致勃勃。

这时正是赛马的最后一场,参赛者中夺魁呼声最高的就是韦膺和夏侯沅峰,韦膺的汗血宝马和夏侯沅峰的大宛良驹都是好马,夏侯沅峰的马虽然稍微不如,但是他骑术胜过韦膺,所以胜负也在五五之数。红旗一展,两人都是一马当先冲了出去,将其他的马匹远远甩落在后面,到了跑道尽头,两人折头转回,夏侯凭借精良的马术胜了一筹,但是韦膺也不差,再加上汗血宝马的威力,还是赶了上来,在最后的冲刺阶段,两人皆是全力而为,最后还是韦膺取胜,成为第一项赛事的魁首。

贵妃娘娘喜道:‘韦郎中果然文武全才,臣妾还是觉得他更适合贞儿。‘

皇后却道:‘其实沅峰这孩子也不错,如果不是马差了一些,恐怕还会超过韦膺呢?而且他三场都要参加,就是都取了第二,也是不容易。‘

李援也点头道:‘夏侯果然是少年英杰,不过韦膺人品端重,文武双全,也是不错的人选。‘

长孙贵妃有些忧心,她看看长乐公主,却见公主殊无喜色,只是默默的望着演武场上。

这时,颜贵妃突然道:‘皇上,太子殿下和雍王殿下都来了。‘

长孙贵妃向外看去,只见太子李安和雍王李贽都是一身便装,观武楼下面有专门的席位,准备给他们,前两日他们都没有亲自到场,今日又都不约而同的来了。长乐公主听到雍王来了,不由望去,果然在雍王身边,她见到了那个人,仍然是青衣素服,文采风流,他坐在二哥身边,言笑宴宴,而他身后站着的那个俊秀阴柔的青衣少年,似乎感觉到她的目光,冷冷的望了过来,那冰冷的目光让长乐公主心中一寒,她仿佛曾经见过这样一双冰冷的眼睛,见过这样气质的人物,这时那个少年上前替他倒茶,虽然是楼上楼下,但是距离不是很远,长乐公主清晰的看到那双白皙中有些苍白的手,长乐公主的心都要跳了出来,是他,是他,她再次看向那记忆中的俊雅容貌,莫非就是他么,那逼疯梁婉,迫死十数密探,却放过自己的神秘人。若真的是他,那么长乐公主就不会奇怪为什么他会放过自己,她还记得那唯一一次的相见,还记得他送到宫里来的诗文,她隐隐约约的觉得,这个儒雅风流的青年早就明白了自己的心事和苦衷,并不会因为离开南楚而怪责自己。露出无比温柔欣喜的笑容,长乐公主却突然悲伤起来,他和她,不可能有未来的,低下头,她几乎想立刻离开这个地方。这时,皇上李援却喜道:‘好箭法。‘

第二场射箭,秦青百步穿杨,箭箭射中红心,夏侯沅峰也是毫不示弱,最后两人并列第一,秦青十分不服气,若是真的上阵杀敌,情况不会这般,但是皇上既然已经如此评定,他也只得无奈接受。

李贽微微摇头,他久在军中,知道这样射靶容易,但若是骑射,就没有这样轻易了,但是这是演武,不是军中大比,自然无可奈何,他对江哲说道:‘若是比试射箭,还是应该考验骑射才行,在我军中,斥候回报军情,需以弓箭,五百步外,骑马飞射,必须将带着情报的响箭射到中军大营外面的箭靶上,这样的射箭比试,未免无用。‘

我咋舌不已,怪不得雍王兵精,天下皆知。

皇上和皇后看看沉默的长乐公主,有些忧心,皇后低声问道:‘长乐,哀家看夏侯那孩子真是不错,你不中意么?‘

长孙贵妃连忙道:‘贞儿,若是你看不中他,韦膺、秦青和其他少年才俊,不论你看中哪个,你父皇都不会拦阻。‘

长乐公主仍然沉默,李援笑道:‘还有一场比试呢,或许长乐会有中意的人选。‘但是他的笑容有些勉强。想必是看出了长乐公主沉默中的反对。

这时,下面的武场上,夏侯沅峰和一个黑衣青年对面而立,这个黑衣青年面庞棱角分明,沉静淡漠。身形和夏侯沅峰相仿,不像夏侯那样身姿如同临风玉树,他周身上下透着骠悍的气息,仿佛浑身蕴含着爆炸般的力量,一举一动又如同一只黑豹般优雅。

我看着那个黑衣青年,心中满是赞佩,问道:‘殿下,此人是谁?‘

李贽道:‘他叫裴云,曾是齐王麾下的先锋勇将,是少林的俗家弟子,据说此人武功卓绝,数年前,他两个哥哥都战死沙场,他的父亲中书侍郎裴敬上书父皇,要求将他调回京中,父皇体恤裴家只有这一脉香烟,所以特旨诏回,现在是禁军北营统领,此人忠勇,深受父皇和秦大将军的宠爱,只是性情有些古怪,不喜欢和人交往,若非如此,恐怕也会是父皇看中的驸马人选,他这次参赛,据说是因为夏侯沅峰,因为此人素有禁军第一高手之称,他和夏侯沅峰谁是长安第一青年高手,争议颇多,平日限于身份,不能比武,这次是趁机比武来了。‘

我看看小顺子,问道:‘你看了他们前面的比武,觉得谁比较可能夺魁。‘

小顺子淡淡道:‘裴云是少林高手,我看他修习的可能是七十二绝技中的无敌金刚力,而且已经有了七成火候,再过十年,夏侯沅峰必然不是他的对手。‘

李贽闻言道:‘那么现在他不如夏侯沅峰么?‘

小顺子道:‘启禀殿下,夏侯沅峰此人的武功路数,走的是阴柔路数,所以进境极快,但是到了后期不免多受挫折,若没有过人的才智毅力,只怕难以登峰造极,所以现在他的武功强过裴云,但取胜也不容易,因为比武交手,还要看各种因素,裴云既然是沙场骁将,那么冷静和果决就超过常人,所以这次胜败应是四六之数,裴云还是有机会的。‘

这时场上两人相互施礼,开始交手,夏侯沅峰用的是剑,裴云用的是刀,我虽然不懂武功,却也觉得夏侯沅峰手中之剑轻灵逸动,满场都是雪光飞舞,而裴云的刀法却是端凝稳重,守得严密非常,招式之间更是森严高古,一派大家气象。

小顺子看的很认真,眼神十分炽热,我忍不住问道:‘怎么样?现在谁占优势。‘

小顺子答道:‘裴云使得是六合刀法,是少林嫡传,和外面流传的大不相同,已经到了炉火纯青的地步,夏侯沅峰的剑法乃是越女剑法,相传从春秋时流传下来的,博大精深,裴云虽然守得很稳,但是若是不能反击,也没有什么用处,我看夏侯沅峰的内力也很精纯,恐怕是不会后继无力的。‘

这时,裴云突然一声轻叱,刀法一变,刀法变得凌厉凶狠,可是仍然隐隐带着慈悲意味,这种矛盾让人看的若有所思。小顺子惊喜地道:‘这是少林秘传的修罗刀法,以修罗手段,实现慈悲心肠,果然不凡。‘一时间场上剑影刀光,绚丽辉煌,一种强烈的血腥意味却涌现出来。这时夏侯沅峰身形一纵,跳了起来,接着凌空翻转,一剑劈下,裴云的长刀上举,硬生生的接了一剑,夏侯沅峰虽然是居高临下,却没有占到丝毫便宜,他再度身形弹起,只见他身形矫健如苍鹰,沉浮在裴云的刀风刃海当中,搏杀如苍鹰博兔,往来如如春燕穿梭,看的观战之人都是大声喝彩。裴云被迫得左右招架,手忙脚乱,这时,夏侯沅峰久战不下,似乎极为愤怒,突然身剑合一,如同闪电一般刺向裴云,裴云手中长刀横挡,这一剑是夏侯沅峰全力而为,裴云却是有些仓促,一声脆响,却是长刀折断,夏侯沅峰从裴云身边掠过,但他的身形却诡异的折转,回身一剑,剑如流光电影,直刺裴云心窝,裴云手中只有一柄断刀,场中上下一阵惊呼,裴云面色沉凝,抛下断刀,两手迎上,只听一阵金铁之声,两人身形分开,夏侯沅峰一脸阴冷,使得俊美绝伦的面容有些失色,而裴云衣袖如蝴蝶纷飞,双手小臂之下肤色隐隐带着金色,却是毫发无伤。

这时观武楼上响起鸣锣,不一会儿,有内宦下来传旨,说道陛下有令,裴云兵器折断,当作败论,二卿都是朝中俊杰,不可生死相搏。夏侯沅峰虽然取胜,但是神色间隐隐不快,上前领旨谢恩。裴云却是神色淡淡,领旨之后便退了下去。

大雍的这次演武盛会就这样结束了,夏侯沅峰以两场第一,一场第二,成为其中魁首,而韦膺和秦青也各有一场第一,也是可以满足的,但是出乎我和雍王的意料,皇上没有宣布择婿的结果,甚至连该有的赏赐都没有颁布。这,到底是怎么回事?

\chapter{第十章 心有所属}

演武结束之后,雍帝李援笑着问道:‘长乐,你看夏侯沅峰如何?‘

长乐公主淡淡道:‘不错。‘

李援喜道:‘若是以他为皇儿驸马,长乐意下如何?‘

长乐公主淡淡道:‘其人虽好,奈何儿心如止水。‘

李援又道:‘既然此人你不中意,那么这么多文武俊杰,长乐你可有中意之人。‘

长乐公主突然落泪,上前下拜道:‘父皇,儿臣虽然得归父皇膝下,但是仍是南楚王后,国主还在生,儿纵无廉耻,焉有别夫改嫁的道理。‘

李援大怒,道:‘朕一心为你择取佳婿,你却如此固执。‘气冲冲的站起,正要训斥,却见长乐公主跪伏于地,珠泪滚滚,虽然玉容不似初回时那般憔悴,但是仍然是全无青春少妇应有的光彩,李援颓然坐下,良久才道:‘是朕不该迫你,皇儿,你放心,朕绝不再为难你。‘

当此事传到我耳中的时候,不知怎么,我心里有些高兴,长乐公主仍然是我印象中那样贤淑知礼,无论她对国主如何,但是还是尽到了责任,就算日后她真的再嫁,我也不会瞧她不起了。

这件事情并没有这样平息,虽然李援暂时放弃了让长乐公主再嫁的打算,但是其他人并没有放弃,窦皇后和颜贵妃、纪贵妃都来相劝,长乐公主既不能赶走她们,又不愿改变心意,这一天,雍王妃高氏进宫,闻及此事,便劝长孙贵妃让公主到雍王府小住几日,等到十五再回宫。

长孙贵妃没有立刻答应,她犹豫的看了高氏一眼,有些事情还只有她知道,长乐公主的事情就是别人不过问,她也要过问的,那日回宫,她问身边的宫女,可有注意公主对什么人较为留意,出乎她的意料,宫女绿娥回禀道:‘公主总是冷冷淡淡,不过雍王来的时候,奴婢看见公主看着雍王身边的男子,而且笑得很开心,可是转眼又跟平常一样了。‘长孙贵妃是知道那人是谁的,江哲江随云,自己若是到翠鸾殿,常常看了女儿拿着一本诗卷,里面全是江哲的诗词,其中有一部分是女儿的笔迹,另外一些都是一个陌生人的笔迹,自己曾经问过,却是在南楚时江哲送进宫里来的,原来,女儿心仪之人竟是那个南楚降臣么,可是自己曾经盘问过服侍女儿的侍女,都说女儿在南楚恪守妇道宫规,从来不曾有悖礼教,那些诗词也是梁婉向江哲索取之后送进宫的,自己只道女儿喜爱那人的诗文,如今看来恐怕女儿早就心有所属,只是从前碍于身份,没有表现出来,当然也有可能是女儿原本没有这个心思,如今提及择婿之事,才有了这个想法。若是让女儿到雍王府去,说不定可以让女儿和那人相见。

可是长孙贵妃皱紧了眉头,若是大雍人,就是职位再低微,只要人品好,女儿喜欢,她都不在意,可是那人是南楚降臣,就是女儿愿意,那人也未必答应,毕竟女儿曾是南楚王后,转念一想,长孙贵妃心道,不管如何,女儿去了雍王府,定然能够散散心,至于她心意如何,我也好探究一下,主意打定,长孙贵妃便道:‘长乐去你那里玩玩也好,绿娥,你一向谨慎,也跟着公主去,公主若有什么事情,也好让你回来禀告。‘她打定主意,让绿娥暗中注意长乐的举止行动,好看看女儿心意究竟如何。

长乐公主也很开心暂时离宫出游,到了雍王府,王妃陪着公主到花园游玩,王府的花园从湖泊那里分成内园和外园,中间用花木、甬道等间隔开来,并没有十分明确的界限,但是内外却是分明,今日天空晴朗,在内宅花园里面的凉亭中,王妃命人摆上果品,让奴婢奶娘带着世子李骏和两个庶出的女儿以及柔蓝一起,在亭子外面嬉戏,自己带着两个侧妃陪着公主在亭中观看,不远处就是湖泊,此时天气晴朗,湖水清澈,宛如碧玉一般明净,几个孩子嬉笑打闹,十分天真有趣,长乐公主看了一会儿,觉得心情十分愉快,笑道:‘王嫂,我记得我走的时候,王兄还没有儿女呢,想不到现在已经有了一儿三女。‘

王妃笑道:‘公主猜错了,你王兄子嗣艰难,除了骏儿,就只有两个女儿,那个最小的,叫柔蓝,是江哲江司马的女儿。‘

长乐公主手一颤,用冷淡的声音道:‘噢,江司马已经成婚了么?‘

王妃没有察觉公主的不安,说道:‘这是江司马的义女,很可爱呢,听王爷说司马独身一人,担心他没有办法照顾女儿,所以送到后宅来让我照顾,我跟王爷说,江司马已经二十六七岁,也该娶个夫人,可是王爷说江司马不愿意,好象是因为从前的未婚妻子不幸身亡的缘故,唉,这般痴情的男儿真是少见。‘

长乐公主心里又是难过,却又隐隐欢喜,转念一想,自己和此人断无可能,虽然从这人的诗文看来洒脱风流,但是怎么看来也不是离经叛道的人,若要此人作出臣纳君妻的事情,恐怕是绝无可能的。想到这里长乐公主更是悲伤,这个自己默默爱恋的男子,却是和自己没有丝毫缘分,想起当日看了他的诗文,心中倾慕他的才华,那日梁婉引他来觐见自己,自己更是对他钟情,可是君臣有别,自己从不敢露出丝毫心思,后来他被贬斥,自己暗暗欢喜,以为不必担心南楚亡国之后他被大雍判罪,想不到他还是被王兄俘虏带回大雍,自己一路为之忧心,担心他不肯投降,被王兄处死,如今他已经成了大雍的臣子,自己又担心他被二王兄连累,可是不论自己心思如何,终究是没有可能和他结合,甚至不能表露自己对他的情意。想到这里,长乐公主勉强笑道:‘王嫂,把柔蓝带过来,让我瞧瞧。‘

王妃令人带过柔蓝,长乐公主看看这个小女孩儿,越看越是喜欢,不由把她抱在怀里,柔蓝还没有学会走路,刚才一直在树下的毡毯之上嬉戏,看到秀丽清雅的长乐公主,她好奇的伸出手去抓公主的发髻,一下子弄乱了长乐公主的青丝,长乐公主却没有恼怒,反而笑了起来,继续逗弄着可爱的小女孩儿。她的欢笑让王妃十分喜悦,而站在一边的绿娥却是明白了公主的心思。

正当众人喜乐融融的时候,隔着明净的湖面,远处传来隐隐约约的乐声,那声音非丝非竹,却是动人心弦,这是南楚流行的曲子,每年之时,正是结伴赏梅的时候,总是能够听到这首曲子,这首曲子就叫《寒梅》虽然只是一曲曲调简单欢快的小调,而且吹奏之人也没有什么技巧,但是听来却是让人觉得碧空如洗,寒梅绽放,心中一片开阔。长乐公主听得入神,片刻,曲声终止,她喃喃道:‘是江司马么,他在想念南楚么?‘

王妃心中一动,看了看公主,道:‘是江司马在吹曲,不知道是什么乐器呢?不过听来总觉得声音很是高古。妹妹今日赶得巧,应该是江司马在临波亭赏景。这位江司马闲暇的时候,不时到湖边赏景,就是在客院看书下棋,很是惬意,可不像其他幕僚谋士那么忙碌。‘这时,远处走来一个青衣少年,不过弱冠年纪,相貌清秀,只是带着一丝阴柔,那些侍女都认得他,没有拦阻,那个少年走到亭前,恭恭敬敬地道:‘王妃,我家公子让奴才来接柔蓝小姐。‘

王妃正要答允,看了一眼公主,突然道:‘江先生也太客气了,他久在王府,不必那么拘束,今日公主在此,她很喜欢柔蓝,舍不得放手,若是不见外,就让江先生过来吧,王爷马上也要过来,不碍事的。‘

小顺子一愣,看了看王妃和公主,眼中闪过一丝疑虑,但仍然道:‘奴才遵命。‘

这时,雍王李贽远远走来,看到小顺子,笑着问道:‘怎么,又来接柔蓝,你主子可是一有空闲就来哄女儿啊。‘

小顺子道:‘启禀殿下,王妃说,公主喜欢小姐,让公子不要见外,过来一次。‘

李贽一愣,但他相信王妃必然有自己的打算,便道:‘说得也是,去请你家公子过来吧。‘

小顺子更是惊异,他的目光迅速转了一圈,却没有看到什么异常,这时,他的目光落到公主身上,只见公主抱着柔蓝,喜笑颜开,心中不由一动,但转念一想,又觉得自己胡思乱想。但他不再犹豫,匆匆忙忙的赶回临波亭。

我正在临波亭和苟廉一起饮酒,见到小顺子,笑道:‘柔蓝呢,怎么没有抱过来,苟兄还想看看我的乖女儿呢?‘

小顺子道:‘今日长乐公主到王府散心,很喜欢小姐,不肯放手呢,王妃说,公子也不是什么外人,若是公子愿意,不妨过去,王爷也在那里。‘

我皱皱眉道:‘这样不大好,算了,改天再去吧。‘

苟廉听了,却道:‘随云,王妃既然已经这样说了,你还是去一趟吧,否则王妃会怪罪你的。‘

我一想,也是,如果王妃没说也就罢了,若是说了我若不去真是有些不好,看看小顺子,他也在点头。便对苟廉说道:‘那我去了,苟兄多饮几杯吧。‘苟廉笑着摆手道:‘你快去吧,一会儿董兄来了,我会向他替你解释的。‘

李贽坐下来,看着长乐公主,笑道:‘长乐,你出来散心是对的,宫里面很沉闷吧,若是喜欢以后常来走走。‘

这时柔蓝突然挣扎起来,似乎急着要去玩耍,长乐公主微微一笑,将她递给侍女,让侍女把她抱回去,笑道:‘其实宫里也不沉闷,我见了几个我走后才出生的弟妹,都很可爱,只是宫里规矩太严,不像外面这样轻松,王兄,听说骏儿就要去幽州了,这么小的孩子就离开父母,王兄也太狠心了。‘

李贽笑道:‘这也是没有办法,骏儿是雍王世子,有他的责任要尽,长乐,就不要为他可怜了,咱们皇家的人,有几个能够自主的呢。‘

长乐公主目光有些黯淡,正要说话,远处走来一个青年,他一身月白儒衫,那种逍遥自在的神情,让人见了便觉得欣喜快乐,而跟在他身后的青衣少年仿佛他的影子一般,明明在阳光之下,却令人视而不见。众人的目光集中在这一主一仆身上,仿佛也感到了他们心中的愉悦。

走到近前,我上前施礼道:‘臣参见殿下、王妃娘娘。‘

李贽笑道:‘今日闲来无事,随云也不要拘泥,一同来坐下吧。‘

我的目光掠过公主,笑道:‘请问,臣该称王后还是公主殿下呢?‘

长乐公主欠身道:‘江大人,本宫知道对南楚不起,还请大人见谅。‘

我原本对她就没有什么怨恨,见她这般,便也投桃报李道:‘殿下不必如此,不论殿下是昔日的王后还是今日的公主,总是臣的君上,臣只有必恭必敬,那有怨责的道理。‘

长乐公主见我说来十分诚挚,心中一喜,破颜而笑,这一笑宛如春花绽放,立时添了几分容光。

李贽见了,也是心中一动,莫非王妃的意思是--,正当他胡思乱想的时候,我已经施礼道:‘今日殿下和王妃款待公主,臣不好打扰,这就告辞了,还请殿下见谅。‘说罢,也不等他们答应,示意小顺子抱了柔蓝,便要转身离去。

李贽刚要挽留,却看到一个宫女正在注视着这里,便把话咽了回去,望着江哲的背影,长乐公主心中又是欢喜又是担忧,今日终于得知他不怪责自己,虽然喜悦,但是想到从今之后,自己深锁深宫,再没机会相见,又是十分悲哀,他说的不错,自己和他总是君臣,断没有可能的。正在悲伤,却想到自己仍然是有夫之妇,如何能够对其他男子钟情,便强颜欢笑,免得他人看出破绽。只是雍王和王妃都是心细之人,哪里看不出其中端倪。王妃倒还罢了,李贽却是陷入沉思,按照他的了解,只怕江哲是绝不会同意这桩婚事的,而且恐怕没有人会赞成,怪不得长乐公主始终不曾透露一字,想必就连江哲自己也不知道公主钟情于他吧。别说别人,就是自己也不会同意,若是此事传了出去,只怕太子他们定会为难,若是让他们在父皇面前挑唆,到时候江哲只怕性命难保。若是自己登基之后,赵嘉也过世了,是否有可能呢?李贽越想越是头疼,臣纳君妻,那是犯上,虽然江哲已经归顺大雍,但若让他娶王后为妻,除非江哲全然不顾声名,这恐怕是不可能的。

他想的这么多,王妃倒是另有看法,她心想,若是能够将公主许配给江哲,那么江哲便是自己人了,她知道自己的丈夫很重视江哲,曾经绞尽脑汁的想折服他,最后江哲如何归顺的她不大清楚,但是她知道自己的丈夫为此曾经夜难安寝,若是能够促成此事,那么自己的丈夫多了一个臂助,公主也终身有托,她凭着女性的敏感察觉,那个现在恭恭敬敬的在自己丈夫面前称臣的青年,实际上却有着超脱俗世的气质,若是不紧紧把握住,终有一日会让他飞走,而那样,可能会让自己的丈夫再度寝食不安。

我丝毫没有察觉所发生的事情,抱着柔蓝,我对小顺子说道:‘你说我是不是该娶个妻子照顾柔蓝。‘

小顺子淡淡道:‘公子若想娶妻,倒是好事,可是若是娶个不中意的妻子怎么办,你若有看中的人,当然好,若是没有,还是不要勉强吧,柔蓝小姐也不是没有人照顾。‘

我笑道:‘世间哪里还有飘香那样的女子,我想娶个普通的贤淑女子也没什么,不过你说得有道理,若是言语无味,真是痛苦,罢了,罢了。‘

小顺子突然道:‘公子觉得公主怎么样?‘

我一愣,笑道:‘你胡说什么啊,公主殿下身份尊贵,又曾经是国母,我怎会对她又非分之想,若传了出去,岂不是笑话,现在好几位驸马人选在那里摆着,只怕国主还没有回到南楚,公主的驸马人选就定了呢。唉。‘我叹了一口气道:‘其实那几人,我最看好韦膺,他必然能够让公主幸福的。‘

小顺子撇撇嘴,没有说话,他懒得和这个对自己身边的小事十分糊涂的主子说话了,不过他面色沉重的想道,一定要留意这件事,公主对公子有了情意,这件事可大可小,若是有人因此嫉恨公子,就会危及公子的安危,而且若是和公主接近多了,恐怕会有麻烦,想到这里,对当初答应公子放过公主的事情不由后悔起来,他知道女子通常会有一种超乎理智的知觉,自己当初和公主曾经十分接近,如果她看穿当日自己就是劫持她的人,只怕公子会有危险啊。唉,当初怎么没有想到还会再见到公主,真是太疏忽了。

\chapter{第十一章 动之以利}

新春时节,长安城内一片莺歌燕舞,表面的平静下却有暗流涌动。

户部侍郎崔央从自家的马车走下来,厌烦的看着门前熙熙攘攘的人群,他的姐姐乃是太子李安的正妃,虽然不如侧妃受宠,但是太子世子是姐姐所生,所以夫妻之间还是相敬如宾,原本雍王气焰嚣张的时候,自己虽然是太子的小舅子,可是门前却冷冷清清,甚至有人为了讨好雍王和自己为难,这次太子储位稳固,今年来拜年的挤破了门槛,不理会那些趋炎附势的小人,他昂然走进大门。

到了书房,管家递上一叠拜贴,崔央随手拿来一一过目,毕竟不论他怎么鄙视那些人,但是权势是需要人来支撑的,没有这些墙头草,太子凭什么治理天下,自己又凭什么提高自己的地位呢?翻了一下,崔央突然被一张帖子吸引住了,那是一张精致的名帖,上面的名字自己从来没有听过,叫纪城,是一个东川的商人,原本崔央没有心情见一个普通的商人的,但是名贴附着一张礼单,上面赫然是万两白银和一对白璧,这是很重的礼了,就是看在这份礼物上,他也不能不见见这个纪城。吩咐管家请纪城进来,崔央坐在书案之后,一边喝茶一边琢磨着这个人有什么要求,礼下于人,必有所求,自己这份礼物能不能安然收下呢?

过了片刻,在管家的引导下,一个三十岁左右的男子走了进来,这人相貌颇为俊秀,五官也很端正,只是一双眼睛有些狭长,鼻子有些鹰勾,未免破坏了他的形象,但这人气度卓然,站在书案前,负手而立,一见便是久居人上之人。崔央心中一震,知道此人决非普通商人,崔央能够做到户部侍郎,自然不是常人,他淡淡道:‘阁下请坐,本官官职虽然不高,但是有些事情还是可以办到的,阁下这般重礼,不知有什么事情,若是与国法无碍,本官自然会考虑的。‘

这个男子笑道:‘草民此来,自然是有求大人,草民有一桩生意,想和太子殿下合作,可是殿下何等身份,我们这些草民不能接近,大人是太子贵戚,故而前来相求,若是大人觉得草民的生意值得一做,还请大人向太子殿下转达草民的诚意。‘

崔央皱紧了眉头,冷冷道:‘太子乃是储君,普天之下,莫非王土,何必还要与尔等商人勾连,若是此事,本官无能为力。‘

那男子冷笑道:‘若是太子殿下不想发财,那么殿下在长安郊外那几个庄子做什么用的,长安利人市的金玉楼,长安最大的赌场的后台老板是谁,是谁违背法令,在外面私下开采金矿呢?‘

崔央听得差点心都跳出来,这人怎么把太子的家底打听的清清楚楚,自己一直替太子打理这些生意,若是事情传了出去,太子最多不过受几句训斥,自己恐怕就得丢官弃职了。他眼中闪过一丝杀气,心想,需得将此人擒下,问明他的身世来历,然后斩草除根,于是,崔央故意和颜悦色地道:‘其实此事也未必不能商量,请坐,上茶,咱们慢慢商量。‘

管家连忙上茶,将那男子请到旁边坐下,自己到外面守门去了。

崔央等这男子坐下之后,又问道:‘不知阁下有什么生意想要和殿下合作,若是本官听了觉得可以,才好向太子禀告。‘

这男子悠闲地道:‘这生意说来也不大,今年大雍干旱,粮食歉收,现在市面上米面的价格是往年的三倍,而丝绸之类的江南特产更是有价无市,如今南楚和大雍之间仇深似海,双方之间的贸易全部中断,草民在南楚有些门路,可以提供粮食、丝绸、茶叶种种特产,不知大人意下如何?‘

崔央皱皱眉,心想,这的确是不错的生意,可是有些麻烦,太子殿下虽然有些进项,可是支出也大,收买官员,安插密探,豢养刺客杀手哪一样不花钱,就是每年需要给凤仪门的供奉就不是一个小数字,看看这个男子,崔央眼中的杀气淡了,他欲言又止,总不能说我们没钱吧。

这个男子十分善于察言观色,继续道:‘其实我们也不用殿下真的出钱,若是有殿下和大人的照顾,我们的生意才能顺利,您也知道,这走私货物,没有殿下为我们撑腰,我们迟早会失手的。‘

崔央点点头,说道:‘若是如此,倒还容易,可是你们准备怎么分成呢?‘

这个男子笑道:‘小的愿意孝敬殿下三成利润。‘

崔央皱皱眉,他是知道这桩生意其中的暴利的,若是只有三成,未免太可惜了,可是自己一方不能提供资金,要得过多也不好开口。

那人仿佛看穿了他的心事,神秘地道:‘其实,草民还有一个主意,若是大人胆子够大,小人愿意奉上六成利润。‘

崔央身子一震,道:‘什么主意?你说来听听。‘

那人笑道:‘大人执掌户部,大雍百多万军队的粮饷军械全在户部管辖,现在南楚最缺乏的就是军械,若是殿下肯用库存的军械交换南楚的货物,既不费殿下分毫,而且还能换得更多的货物,这收益可是能翻一番的,等到殿下得到金银之后,再在大雍定制一批军械补上缺口,这其中的差价大人应该了解,只一趟生意五十万两绝没有问题。‘

崔央听到这里,拍案而起,怒道:‘岂有此理,你竟然鼓动本官资敌,你是不是南楚的探子,竟然到了本官这里胡言乱语。‘

那人好整以暇,笑道:‘大人说错了,草民不是南楚的探子,草民霍纪城,忝居锦绣盟主。‘说罢,他手一摆,一道白光从他手中射出,贴着崔央的脖颈飞过,穿透了崔央身后的书架,射入了墙壁,崔央吓得魂不附体,书房的门砰的一声被踢开了,管家站在门口,眼中闪着寒芒,手里多了一把匕首。

霍纪城一笑,身形扑向门口,那管家只觉得眼前一花,手中的匕首已经落到霍纪城手中,崔央再看去,霍纪城已经坐回了座位,笑眯眯的看向崔央。崔央已经镇定下来,他看了霍纪城一眼,心道,此人若没有把握,怎会以身涉险,不说别的,若是惹恼了他,自己这条命就保不住了,他擦了一把冷汗,道:‘霍盟主,请坐,请坐,想必盟主此来不是兴师问罪的吧,太子殿下铲除锦绣盟在长安的分舵,也是为了国事,霍盟主意图复兴蜀国,你我乃是敌对,这也是无可奈何,若是霍盟主想要报复,下官却不能够苟同。‘

霍纪城淡淡道:‘这话说得不错,对你们来说,我锦绣盟是叛逆,长安之事也是无可奈何,不过那些都是小事,他们也算是为国尽忠了,不过俗话说的好,天下没有永远的敌人,我们现在处处受到限制排挤,长此以往,只怕不仅复国无望,就连性命也保不住了,若是殿下肯和我们合作,我们也没有必要定要复国,能作个富家翁也是不错的,殿下虽然已经储位稳固,但是毕竟还有雍王虎视眈眈,需要用银子的地方多得很呢,说句不好听的话,我们是唯一可以和太子殿下合作的人选,现在最盼着太子殿下出错的就是雍王,若是别人替殿下效力,若是落到雍王手上,只怕会把太子全盘供出,我们锦绣盟和雍王仇深似海,亡国之恨永远难忘,绝不会倒向雍王,我们也知道若非雍王定下计谋,大雍和蜀国未必会交战,所以我们和太子之间没有什么深仇大恨,若是能够相助太子铲除雍王,那么我们也算是尽了最大的努力为我王报仇。而且,就算有人揭露了锦绣盟和太子之间的合作,你认为会有人相信么?谁会相信太子和我们这些逆党合作,而且太子刚刚把我们锦绣盟长安分舵荡平了。‘

崔央越听越觉得有礼,虽然觉得此人凉薄,对自己兄弟的死伤毫无心痛,但也觉得此人说的不错,他犹豫了一下,问道:‘贵盟兄弟还有一些在天牢之中,不知阁下有何打算。‘

霍纪城微微一笑道:‘若是殿下觉得没有关系,那么放了也好,若是觉得有碍,就快些处死他们,免得让人怀疑太子和锦绣盟之间的关系。‘

崔央一阵心寒,此人真是心狠手辣。他闭上了眼睛,半天才道:‘此事本官不能作主,这样吧,待我禀明太子之后再说吧,阁下明天再来听回话。‘

霍纪城微微一笑,道:‘这是应该的,应该的,不过崔大人,您在城里养的小妾已经有了身孕,怎么还不接回去,莫非是夫人嫉妒么?‘

崔央手一抖,刚刚拿起的茶杯差点掉落桌上,他看向霍纪城俊逸的面容,仿佛看到了魔鬼一般。

霍纪城恭恭敬敬的施了一礼道:‘草民这就告辞了,若是大人不着急,如夫人临盆之前,还是不要挪动的好。‘

听着霍纪城淡淡的威胁,崔央有气无力的摆手道:‘霍盟主请放心,本官不会使诈的,不论如何,本官不会安排陷阱陷害盟主的。‘

霍纪城走出崔府,呼吸了一口冰冷的空气,觉得心神畅快,这次他确实是冒了很大的险,但是这太吸引人了,蹈海之利,安能不取,至于人命算什么,只要自己活着,锦绣盟就不会灭亡,等到自己得到所需要的粮饷军械,在得到百万金钱,到时候就可以树起蜀王世子旗号,复国立业,若是有朝一日,自己保得世子登上大宝,自己就是当之无愧的摄政王,到时候自己的光辉荣耀谁人能敌。

一阵冷风吹来,霍纪城滚烫的脑子冷静了下来,他心想,我得再去和天机阁的人见见面才行。若是他们那边没有成功,这生意还是作不成的。

在大街小巷里面转了几圈,确定身后没有人跟踪,霍纪城悄悄的进入了一家民宅,老迈的屋主看见他来了,也不作声,带着他走到卧房里面,在一堵墙上下拍了几下,墙壁悄然移开,霍纪城一挥袍袖,走了进去。在他身后,墙壁无声无息地合上了。

在昏暗的灯光下,寒无计悠然自得地坐在椅子上,看到霍纪城进来,他站起来拱手道:‘霍盟主,您过来了,不知道谈得怎么样。‘

霍纪城微微一笑,道:‘还不清楚,就看能不能说服他们的主子了,寒兄,你们那边情况怎样?‘

寒无计笑道:‘我已经接到飞鸽传书,那边已经同意,而且答应长期合作,他们现在急需这些东西,而且他们国库几乎被雍王搬空了,以后也希望通过咱们从大雍得到金银物资。我们天机阁的信誉还是有的,若是这边走通,咱们就可以联手发财了。‘

霍纪城到了一杯茶,一饮而尽,道:‘虽然我们锦绣盟在大雍势力不小,可是这次太子殿下雷霆一怒,今后我们不免步步艰难,若是不能买通太子,那么生意还是作不成,不过你放心,我们和很多官员都有私交,他们大部分都是太子一党的,俗话说物以类聚、人以群分,我看太子也不是什么好东西,这些勾心斗角的事情难不住我的。‘

寒无计施礼道:‘那就请盟主多多费心了,我们的势力还不能深入大雍,所以全靠盟主费心,不过南楚方面请盟主放心。‘

霍纪城眼中突然闪过一丝利芒,道:‘天机阁在南楚的势力我们是知道的,若是贵阁主愿意,我们可以有进一步的合作的。‘

寒无计微微一笑,他是知道的,陆侯镇守蜀中,将锦绣盟的势力打击的四分五裂,德亲王掌控大局,锦绣盟在南楚也始终不能形成气候,反而是在大雍,因为他们目前的目标没有指向大雍,所以才能在大雍的纵容下发展势力,不过现在德亲王已死,看来霍纪城又想向南楚发展。寒无计委婉的道:‘盟主此心,我们是明白的,可是现在我们做这个生意,若是盟主过于急进,不免伤害生意,其实盟主不必着急,反正这生意也就做上几年,等到那时候,盟主兵精粮足,随便盟主怎么动作都行。‘

霍纪城看了看寒无计,神色震动地道:‘寒兄果然足智多谋,不知道可否为霍某引荐阁主,商议一下合作的事情。‘

寒无计傲然道:‘我们阁主将此事全部交给寒某负责,盟主不必舍近求远。‘然后脸色变得神秘,继续道:‘而且,阁主从不见外人,就是寒某也只是见了阁主一次,还没有见到阁主的真面目,不过阁主在任何属下身边都有暗探,盟主可要小心。‘

霍纪城神色有些不豫,但转念一想,天机阁本就是神秘组织,也难怪如此,便温言道:‘寒兄也是蜀国遗臣,为复国大业,理应尽心尽力,还请总管多费心,在阁主面前多讲几句好话。‘他起了笼络之心,立刻言语温和,诚意十足,寒无计心中一凛,心道这人果然喜怒无常,自己可要小心,不要落了破绽,他故意神色黯然,良久才道:‘寒某也是蜀国之人,怎不想复国,只是阁主不喜欢介入国家大事,所以对我们约束极严,这此若非利润丰厚,这桩生意阁主也不会赞同的。‘

霍纪城心想不能急迫,道:‘总管好意,霍某多谢了,我还要安排下面的事情,这就告辞了,事成之后,再来和寒兄商量细节。‘

寒无计连忙起身道:‘此事紧迫,我就不留盟主了,寒某送盟主。‘

霍纪城道:‘不必相送,以免惹人注意。‘寒无计仍然恭恭敬敬的送到房门口,为了掩人耳目,没有送出门外。

他走之后,那个年老的屋主突然站直了身子,从脸上取下面具,露出一张清秀的少年脸庞。寒无计笑道:‘盗骊,你要记得,人若贪心,自促其死,这个霍纪城的确心狠手辣,若是拉拢起人来倒是一片热诚,若非知道他的所作所为,只怕不免上当,为人行事,若是口是心非,就是瞒的了一人一时,也瞒不了一生一世。‘

少年恭恭敬敬地道:‘盗骊受教。‘

寒无计笑道:‘天下熙熙,皆为利来,天下攘攘,皆为利往,公子用计,咱们真是想都想不到,虽然不知道公子用意何在,但是总不会是好事的。‘

此刻的太子府里面也在密谈,崔央等到霍纪城一走,立刻出门往太子府第赶去,他是太子的小舅子,又是太子的心腹,丝毫没有阻碍地进入了太子的起居之处,一走进华丽的的大厅,就看见太子穿着便服,正在那里看歌舞,崔央不由多看了几眼,这一看,崔央的眼睛就转不开了,原来这些舞姬都穿着荷叶罗裙,上身用荷花瓣掩饰酥胸,露出肌光如雪,舞姿翩翩,婀娜动人,扣人心弦,明明衣着暴露,但是曲乐乐而不淫,那些女子动作毫不扭捏,竞没有淫秽的意味,让人看来身心舒畅。

李安见崔央被迷住了,笑骂道:‘崔央,你干什么来了,还站在那里干什么?‘说着一挥手,那些舞姬退了出去。这时崔央才看到,鲁敬忠也坐在一旁,而太子身边还站着一个相貌俊美的侍卫。

崔央定下心神道:‘殿下,臣有机密的事情禀报。‘

太子眉头一皱,还没有说话,那个侍卫已经说道:‘殿下,属下还有一支舞曲,想和乐师舞姬们商量一下,不如属下先去,等到完成了也好让殿下赏玩。‘

太子笑道:‘你去吧,本王等着你的新曲子。让你师兄,把外面守好了,不许任何人擅闯。‘那个侍卫退了出去。李安看了看崔央,道:‘有什么事情?说吧。‘

\chapter{第十二章 阴谋陷害}

崔央将经过详详细细的说了一遍,李安大怒道:‘这些叛逆,好大的胆子。少傅,你说该怎么办?‘

鲁敬忠眯缝着眼睛想了一会儿,问道:‘崔大人,你说这人并不要求救回自己的属下?‘

崔央恭恭敬敬地道:‘是的,霍纪城不仅不急于救人,好像反而想我们杀了他们似的,免得将来有人怀疑彼此之间的关系。‘

鲁敬忠冷笑道:‘此人果真是心肠狠辣,不过这也说明了他正是霍纪城,霍纪城的事情我知道一些,从那些锦绣盟的弟子的口供里面得知此人出身将门,是四川厉家的旁系弟子,当初厉家和唐门争夺川中的控制权,结果厉家惨败,唐门衰弱,大雍攻击蜀国的时候,他们也无力再为蜀国尽力,要不然就是我们攻下了蜀国,恐怕也会有不少将领死于暗杀,唐门的暗器,厉家的大搜魂手都不是什么光明正大的功夫,最适合暗杀行刺,后来大雍占领东川,而蜀中大半归于南楚,唐门投靠我大雍,厉家投靠南楚,偏偏这霍纪城古怪,反而破门而出,创立了锦绣盟,声言要恢复蜀国,这些年还真让他作出了一些事情,可惜此人气量狭窄,镇守蜀中的陆侯又是帅才,所以连连失败,不过这人虽然无能,却有一样长处,他心狠手辣,当断则断,所以虽然锦绣盟屡次惨败,仍然保住了元气,近来南楚局势混乱,他想必占了不少便宜,却偏偏在太子手上吃了大亏,也难得他能够想出这个主意,以臣看来,他倒是诚心诚意的,不管将来如何,这桩生意倒是值得一做的,他有句话说得不错,将来就是他宣扬出去,谁又能相信太子殿下会和他们勾结,若是太子想做这生意,就立刻将天牢里面的锦绣盟逆党全部处死,然后再和霍纪城合作,如今锦绣盟就是再强大,他敢和我们大雍为难么,就是他真的兴兵造反,也是雍王和庆王的事情,正好消减他们的兵力,再说,殿下这几年几乎是入不敷出,这桩生意值得,就是将来有了意外,殿下只要说锦绣盟勾结一些官员所为,然后砍几个人头也就是了。‘

李安沉思了片刻,看看崔央道:‘这事牵连太大,若是一旦泄漏出去,户部恐怕就会翻天,崔卿岂不要担上责任,不成,不成。‘

崔央铁青的脸色才松懈下来,感激地看着李安,但是李安虽然这么说,但是神色上却是十分不舍。

鲁敬忠淡淡一笑道:‘户部尚书梁谨潜虽然是殿下心腹,可是此人却心怀异志,虽然他对殿下必恭必敬,但是却暗里记录了一本殿下数年来从户部挪用公款的账本,这个,殿下恐怕还不知道吧?‘

李安身子一震,急急问道:‘你说得可是真的?怎么知道的?‘

鲁敬忠得意地道:‘殿下,俗话说,老夫少妻最是不安,梁谨潜一生谨慎,可惜年将五十却娶了一个二十多岁的小妾,不免有些难以应付,这女子风流冶荡,就和梁大人的小舅子私通上了,可惜两人行事不秘,被梁大人捉奸在床,一怒之下,把这个女子杖杀,但是小舅子就只能赶出去不许上门罢了,谁知那小妾恋奸情热,竟然告诉了情夫梁大人手中那本私账的事情,那小子怀恨在心,恰好他和我相识,便到我府上告密,我今日原本就是为了告诉殿下此事,可是殿下正在欣赏歌舞,所以还没有来得及禀告。‘

李安面沉如水,问道:‘你可有证据?‘

鲁敬忠站起身,上前奉上一张纸,李安拿过来一看,上面果然是自己到户部挪用钱粮的帐目,什么时候用的,什么时候还得一清二楚。李安平安而起,怒道:‘好个老奴才,本殿下定要取了他的狗命。‘

鲁敬忠笑道:‘这是那小子偷偷抄了一些作为证据,原本臣想摆平这个梁谨潜容易得很,只要殿下有心,咱们就可以夺回账薄,杀人灭口,可是臣觉得太便宜了他,如今正是一个好机会,咱们让崔大人办这件事情,但是暗中作些手脚,若是生意顺利,那样最好,咱们事后再和他算帐,若是不幸出了事情,咱们就让他顶罪,到时候殿下只要安排的好,保管他说不出实情,然后崔大人理所当然晋升尚书,这户部才是殿下的金库呢?‘

李安听到这里,大笑起来,说道:‘好主意,鲁爱卿,你果然是孤的智囊。‘眼中闪过凶光,李安冷冷道:‘不过给我把他看严了,不能让他察觉孤的意图,也不能让他把账薄给了别人。‘

鲁敬忠正容道:‘殿下放心,臣办事您还不放心么?‘

李安突然想起来一件事情,问道:‘那个告密的人呢?‘

鲁敬忠淡淡道:‘这人留着总是祸患,臣大胆,已经先处置了。‘

李安满意的点点头道:‘不错,这人知道了本王的隐私,焉能让他活在世上。‘

崔央听到让自己接任户部尚书,原本喜形于色,可是听到两人说着陷害杀人的事情,却丝毫没有一丝情绪波动,也不免心里冰寒,心道,我可别在这里听了,若是知道了什么隐秘,将来再被杀人灭口可就不值得了。想到这里,连忙说道:‘殿下,鲁大人,时间太晚,臣要告退了,若是殿下同意此事,臣这就回去等霍纪城前来。‘

鲁敬忠心想,下面还有重要的事情商谈,他走了也好,便说道:‘殿下,崔大人回去也好,免得锦绣盟中人去问讯,殿下不如先让崔大人同意此事,具体事宜,明日臣去跟崔大人商量。‘一边说着,一边跟太子使了一个眼色。

李安一见便知道鲁敬忠还有私隐的话要说,不便让崔央知道,他笑着说道:‘好了,崔央你先回去吧,等到商议之后,鲁大人会去和你详说,不过此事本王原则上已经同意,你也好好想想该如何着手。‘

崔央领命退下。李安看看鲁敬忠,笑道:‘还有什么事情,说吧,还瞒着崔央,很重要么?‘

鲁敬忠捻着胡须,冷笑道:‘殿下,虽说户部是您的天下,可是这军方的势力大半还在雍王手里,也是因为这个,我们才不能自己做这个生意,让锦绣盟去跑腿,可是若是军方留意到此事,就是锦绣盟再厉害,他还能斗得过雍王么?‘

李安皱皱眉,问道:‘难不成这桩生意就不能做了吗?‘

鲁敬忠摇头道:‘这怎么成呢?臣有个主意,若是能够让雍王无暇顾及殿下的事情,殿下不就稳如泰山了吗,虽然说如果出事,咱们有替死鬼,可是不免损失金银。‘

李安听到这里眼睛一亮,道:‘你有什么法子可以让雍王自顾不暇,若能够如此,就是孤不作这桩生意,也是心满意足。

鲁敬忠笑道:‘这也是一件巧事,殿下想让夏侯沅峰做长乐公主的驸马,但是公主始终不同意,所以臣就请兰妃娘娘转托纪贵妃问问公主的心事,昨日臣来府中,兰妃娘娘转告贵妃娘娘的原话,说是长乐公主似乎不想改嫁,臣原想公主若是不想改嫁倒也罢了,反正谁也沾不到便宜,所以臣也就没有放在心上,兰妃娘娘顺便还说了几件事情,第一,长乐公主平日没有什么喜好,就是喜欢读诗文,而且最喜欢南楚第一才子江哲的诗文,平日手不释卷,第二,昨日,长乐公主到雍王府散心,回来的时候心情很好。‘

李安皱紧了眉头道:‘你是说长乐公主和那个南楚降臣有私情,胡说,我这个妹妹我是知道的,最是贤淑,绝不会与人有私情的。‘

鲁敬忠笑道:‘臣也知道这是连影都捕不到的事情,可是只要添油加醋一下,自然会有人相信的。‘

李安神色一动,道:‘你是说--‘

鲁敬忠笑道:‘自然是秦青秦将军,秦将军对公主一片深情,至今不变,可是公主这次回国却对将军冷若冰霜,全不念青梅竹马之情,秦将军十分气恼,因此才对南楚降臣十分傲慢,还在甘露殿当众凌辱江哲,听说雍王对这个江哲十分器重,这已经重重得罪了雍王,这就是一个机会,咱们派人在秦将军耳边吹几句风,就说公主在南楚和江哲有了私情,所以才不理会秦将军--‘

刚说到这里,李安怒道:‘住口,我皇妹当年为了大雍,远嫁南楚,如今好不容易回来,别说她没有私情,就是有了私情,也不能容你玷辱公主的声名。‘

鲁敬忠凛如寒蝉,连忙跪下谢罪,直到李安平静下来,才道:‘殿下放心,臣就是天大的胆子,也不敢伤害公主的请誉,此事不会泄漏出去的,秦青就是知道此事,他难道敢声扬出去,秦将军就是再鲁莽,也不能这么做,若是臣料得不错,秦将军必然找机会质问江哲,咱们派了杀手暗中跟随,此事既是捕风捉影,那江哲必然能够解释清楚,等到秦将军满意而去,咱们就杀了江哲,此计对咱们可是好处不小,第一,不管那江哲有没有才干,现在杀了他,就能让雍王痛心,第二,雍王必然怀疑秦青杀了江哲,这样一来,就是秦彝大将军能够解释清楚,雍王也必然心生芥蒂,这样一来,雍王忙着和秦大将军讨公道,哪里还能顾得上咱们。‘

李安面色阴晴不定,犹豫不决,鲁敬忠又道:‘此事关联之人,都是位高权重,谁会把闲话外传,再说,说句当诛的话,公主虽然是有功之人,但是毕竟是南楚王后,南楚覆亡之日,公主就是亡国之后,昔年西施有功于越,仍被越王后投湖,公主不过受几句闲言碎语,又有什么要紧,再说,公主和长孙贵妃倾向雍王,殿下也不是不知道,要不然何必想让夏侯沅峰做公主的驸马呢?‘

李安还是默然不语。

鲁敬忠热泪盈眶道:‘殿下圣明,最多殿下登基之后,多多抚慰公主就是,雍王不除,殿下难安。‘

李安想来想去,终于长叹道:‘你要小心,不可传扬出去,若是传到父皇和皇妹的耳中,孤绝不饶你。‘

鲁敬忠连忙磕头道:‘殿下放下,下官绝不会让流言传出去。‘

李安犹豫了一下道:‘可是此人想必时刻都在雍王府,刺客如何混得进去?‘

鲁敬忠笑道:‘殿下放心,过了十五,就是雍王召宴,要送世子远行,按照规矩,这是必然要宴请百官的,殿下放心,臣一定让秦青背上这个黑锅,至于那个江哲,只能怪他命不好,谁让他投靠了雍王呢?‘

李安微微点头,看看事情已经谈完,高声叫道:‘邢嵩,邢嵩。‘

厅门被推开,一个相貌阴沉的中年人走了进来,单膝跪倒行礼。

李安淡淡问道:‘那个夏金逸去了哪里,有没有和什么人传递消息?‘

邢嵩恭敬地道:‘启禀殿下,夏金逸先是和乐师舞姬谈了一会儿曲子,然后--‘说到这里,李安眉头一皱,眼中带了杀气。

邢嵩接着说道:‘然后此人到了后园和王妃身边的侍女绣春私会。‘

李安先是心中一宽,这个夏金逸立下大功,又看在他师兄的面子上,就留他做了侍卫,原本只当是养个闲人,不料这人幽默风趣,能说会道,更擅长歌舞风月,自己没几日就觉得实在喜欢夏金逸,可是他身边不能随便留人,方才崔央前来秉事,如此神秘,若是夏金逸是奸细,必然要想尽办法探听,在外面守门的就是张锦雄,夏金逸是不难找个借口的留下的。

想不到夏金逸一点探听的心思都没有,反而忙着和侍女私通,若是他是探子,那他可就是天下最蠢的探子了,稍微称职一点的也不敢作出这种事,若是重惩,私通侍女是可以杖毙的,他满意的心想,以后可以把他留在身边了,他倒是个好奴才,忠心有趣,比他那个师兄强多了。至于私通侍女,虽然李安也有些不满,可是这也不算什么大事,王妃身边的那个绣春,相貌虽然不错,但是并不出众,李安从没留意过的一个女孩子,前些日子,王妃还跟自己说想把身边的侍女放出去嫁人呢。

此时太子府邸的后园之内,夏金逸搂着一个相貌秀丽的侍女正在甜言蜜语,他兴致勃勃地讲着自己漂流四方的经历,把个从未踏出府门一步的小姑娘听得入了迷,夏金逸一边说着,一边开始动手动脚,他是情场老手,不会鲁莽惊吓了小姑娘,他温柔的吻着绣春的粉颈,既然轻咬她通红的耳垂,一双手也开始不老实,在绣春的娇躯上游移,不用多久,轻而易举地就让未经人事的小姑娘意乱情迷,夏金逸一看得手,一把抱起小姑娘颤抖的娇躯,躲到了假山之后,正在夏金逸宽衣解带,眼看就要得手的时候,突然有人冷叱一声。夏金逸吓得一个冷颤,满腔欲火立刻消退,连忙整理衣裳,半天,外面没有动静,夏金逸伸出头去,却看见自己的顶头上司,副总管邢嵩站在外面,负手而立,月光下一脸的寒霜。这时候绣春也清醒过来,匆忙的整理衣裳,低着头走出假山,扑通跪倒在地上,满面羞惭,哭泣不止。夏金逸也连忙跪倒旁边,苦苦哀求道:‘总管大人,求您饶了小的一次。‘

邢嵩冷冷道道:‘你这小子竟敢私通王妃的侍女,还不跟我去见殿下。‘

夏金逸吓得面色苍白,道:‘小的求大人开恩,小人不过一个浪子,生死算得了什么,绣春年轻,求总管饶了她这场罪过,小的以后绝不敢再来勾引她了。‘

邢嵩微笑道:‘你这小子,起来吧,以后不可再犯,回去吧,若让我再看见,我就剥了你的皮。‘

夏金逸听了大喜,连连叩谢,直到邢嵩的身影消失,他才发觉已经出了一身的冷汗。

崔央回到府邸,先胡乱吃了一些东西,然后就在灯光下发呆,他知道自己和太子是一条线上的蚂蚱,若是太子倒台,可是自己却越来越觉得跟着太子心惊胆战,倒是雍王,崔央想起从前往雍王军中送粮饷,雍王召见自己的事情,当时的雍王身穿轻假,外罩锦袍,办起事情来雷厉风行,私下交谈却是随和亲切。令人觉得如沐春风。太子虽然是储君,又是自己的姐夫,可是却是傲慢骄矜,每每让自己背生冷汗,总有如履薄冰的感觉。想到这里,崔央几乎想要叛离了太子,可是转念一想,太子妃是自己的亲姐姐,太子世子是自己的外甥,名利欲望终于胜过了良知和恐惧,崔央站起身,心想,自己没有回头路可走了。看看天色,这一来一回,再加上沉思良久,天光居然已经放亮了,崔央推开窗子,昨夜不知何时下了一场轻雪,窗外雪光明艳,崔央走出房门,深深的呼吸了一口冰冷的空气,这时,管家前来禀告道:‘启禀大人,昨日那位客人又来求见。‘崔央微微一笑,道:‘请客人到书房来见我,时光还早,想必客人也没有用饭,你送两份早餐到书房来。‘

外面传来朗朗的笑声道:‘草民又来打扰大人了。‘

崔央抬头看去,却见霍纪城一身灰衣,风度翩翩,当风而立,崔央几步上前,拱手道:‘霍兄,不,纪兄,请到书房叙话。‘

霍纪城一看崔央神色有些倦怠,但是却掩饰不住喜色,便知好事成了,便也还礼道:‘好,那就叨扰了。‘

说罢,两人相视而笑,好似多年旧交。笑声中,霍纪城心思飞得很远,若是能够得到足够的粮饷军械,那么趁着大雍和南楚交战,复国兴邦,指日可待。而天机阁是必须得依靠的,否则自己在南楚寸步难行,若是等到交往深了,自己想法子把天机阁并吞,将天机行会千万财产收入囊中,就是复国不成,自己也可以作个富家翁。崔央想得简单的多,若是生意成功,自己不仅囊中丰厚,还可以进一步得到太子的信任器重,前途似锦啊。

\chapter{第十三章 寒园来客}

南楚同泰元年甲戌元月十六日,雍王以世子将远行,依例召宴群臣,哲未与会,宴中,虎威将军秦青私下求见,以谣言责哲,哲以礼劝之,青惭而退。

--《南朝楚史·江随云传》

我舒服的伸了一个懒腰,这几日我接到消息,太子已经上钩了,这样我就可以暂时放一放这些麻烦的事情了。其实今天雍王府上下忙的要死,谁让世子就要代雍王就藩了呢,按照惯例,雍王殿下要召宴群臣,这种事情我可不感兴趣,所以就跟殿下告了假,准备在寒园里面好好看看雍王送我的几部绝版古书,殿下很谅解我不喜欢热闹的心情,因为今日不少王公贵族都会来赴宴,这样一来,雍王外府恐怕会太杂乱,毕竟他们中有很多人都有资格在王府里面逛逛的,只要不接近几处禁地,那么就没有什么关系,为了防止有人打扰我,殿下特意派了人替我守门,不许他人擅入,其实殿下的意思,我不妨到后宅躲一躲,可是瓜田李下的,我就没有答应,反正寒园外面有人把守,我怕什么呢?

小顺子最知道我的脾气,早上一起来就把门窗打开,放去夜来的浊气,然后点燃一炉清香,而我换了一身宽松的长袍,喝着小顺子为我泡的香茶,真是神仙一般的日子。看了一会儿,无意中抬头,看见小顺子正在拿着一把银刀雕刻着一块白玉,这是他最近养成的习惯,自从上次我逼着他雕刻了一个木头娃娃给柔蓝之后,他忽然喜欢起雕刻来,没事就拿着一把刀刻来刻去,我曾经问他为什么突然喜欢起这些东西,他神秘的对我说,他发觉这是一个练功的好法子,前阵子他总觉得武功似乎没有什么进境,谁知道为了刻好木头人,结果发现他的招式多了几分流畅和自然,这下子他可就找到了练功的新方法,我虽然不明白刻木人和武功有什么关系,可是触类旁通的道理我还是懂得的,看着小顺子从生硬杂乱到流畅连绵的刀痕,也感觉到他似乎有些进步,至少他现在雕刻的东西已经栩栩如生了,所以索性买了一堆普通的玉块给他,让他雕刻着玩一玩儿。这不,我书案上的书镇就是他前几天雕刻的。

看看他,我突然笑了,说道:‘小顺子,虽然你喜欢雕刻,可是也不用天天练啊,今天殿下宴客,在前面有杂耍曲乐,你去散散心吧。‘

小顺子淡淡道:‘今天外面人太多,我不放心你一个人。‘

我笑道:‘你也太小心了,这是雍王府,我不过一个小小的降臣,谁会来刺杀我呢?好了,去玩玩吧,别忘了,你才二十岁,别成天像个小老头,那我可就罪过了。‘

小顺子瞪了我一眼,可是他毕竟年纪还轻,那些杂耍什么的对他也很有吸引力,可是总是有些不放心我,我笑道:‘这样吧,你把胡威叫进来,让他在这里替你护卫,这样你放心了吧?‘

小顺子看看书案上的茶杯,道:‘可是总得有人伺候茶水。‘

我无可奈何地道:‘小顺子,别忘了是我教你泡茶的,好了,去玩吧,今天你不许跟着我,花灯要连放三天的,昨天晚上你保护我也就算了,今天你自己出去好好逛一逛,不许成天闷在府里,我又不出去,不会有危险的。‘

小顺子终于点点头道:‘好,我这就出去叫人,公子你放心看书,我会安排好的。‘

我看着他的背影,欣慰的笑了,就是吗,一个刚刚二十岁的小孩子,干什么这么老成,就应该开心玩乐才对,虽然我二十岁的时候,因为这个臭小子偷了我的盘缠而不得不去考了状元,可是他可没必要一定要委屈自己吗。

送走了小顺子,我继续沉迷于书中,胡威进来叩见的时候,见我没有反应,他跟着有一段时间了,直到我有时候一看起书来就什么都忘了,便悄悄退了出去,没有打扰我。此刻的我还不知道,我一生中最接近死亡的时刻马上就要来到了。

秦青一边应付着身边的同僚,一边想着心事,今日他是代替父亲前来赴宴的,不过他可不愿和那些老狐狸聊天,所以匆匆向雍王道贺之后便跑到外边的彩棚里面,看着高台上正在表演的杂耍,可是他全然没有看进去,满脑子都是长乐公主和江哲的影子。

当年他约公主私奔,却被严词拒绝,当时年少鲁莽的他口不择言,指责公主忘情负义,贪图南楚王后的尊荣,公主含泪而去,却依然高傲的背影让他痛悔万分,可惜却没有机会说出抱歉二字,然后他就被父亲重责之后丢进了军营,他是一刀一枪的杀出了这个四品虎威将军的,可惜没有给他机会到他日日怀恨的南楚作战,公主就回来了,知道此事,他既是高兴又是难过,他最希望的就是领军攻破南楚,然后亲自跪在公主面前请罪,可是现在没有这个机会了。

公主回来之后,他曾经求母亲入宫代为转达自己的心意,可是却是当头一盆冷水,公主竟然对他再无一丝情意。他心痛如死,抱着最后一丝希望,他参加演武,却和那个小白脸拼了一个平手,虽然知道这并不代表自己不如夏侯沅峰,可是秦青知道自己完全失去了和公主和好的希望,而比武之后,自己就被父亲关进了祠堂罚跪,就是因为自己和那个南楚降臣之间发生的纠纷,秦青深恨南楚,迁怒之下,就连南楚的人也恨上了,江哲此人,浪得虚名,屈膝投降还振振有辞,自己讽刺他几句算什么,可是父亲竟然动了家法重责,现在秦青还记得父亲铁青着脸训斥自己的情形。

‘畜生,我不怪你昔日胡为,也不怪你无端迁怒,可是你竟然当众侮辱贤士,这样下去,我秦家还有什么前程可言,你可知道这是何等的大错。江哲此人非是庸才,他为德亲王参赞,南楚轻取蜀中,他一曲长歌,送了蜀王性命,他一道表章,令我大雍有志之士心惊肉跳,此人乃国士也,你竟然因为他是降臣而轻辱之,你可知道若此人心胸稍微狭窄一些,将来你的性命就会送在他的手上。‘

自己虽然不服,可是对着暴怒的父亲仍然只得低头认错,一直到了昨天,父亲才放自己出来,叹着气道:‘小奴才,明日雍王召宴,你替我前去祝贺世子就藩,记着,一定要找个机会去见江司马,向他赔罪,若是此人记恨你,恐怕终究是大祸,我打听过,雍王殿下将此人看作心腹,就是齐王殿下也对他很器重,两位殿下都不是平庸之人,可见这人的厉害,你若不能求得他的谅解,将来你的弟妹恐怕都要被你连累。‘

所以自己满怀怨愤的来到了雍王府,原想胡乱认个错也就算了,可是就在刚才,自己得知了一个几乎让自己气晕了的消息。长乐公主居然和那个寡廉鲜耻的降臣有私情。

得知这个消息是个巧合,见到雍王之后,自己代表父亲表示祝贺之意,自己虽然别扭但还是提出向江哲致歉的事情,雍王欣然答应,不过却说江司马素来体弱,恐怕得等到巳时才能见客,让自己先去散散心,自己无奈答允,一边腹诽着那个没用的书生一边在雍王府里面观赏风景,可是没走多久,就发现两个太监在一片松林后面窃窃私语,自己原本没有打算偷听,可是无意中听到的一句让自己立刻呆住了。

却是一个太监向同伴得意洋洋的宣扬,说是长乐公主到王府的时候,和江司马私下相会,却原来两人在南楚就有私情,若非自己奉命服侍江司马,恐怕还不知道这样天大的事情呢?还在吹嘘说,江司马给了自己千两白银,还说若是自己肯守口如瓶,等到将来他成了驸马,要这个太监去做总管。

秦青听到这里气得昏头转向,半晌才清醒过来,想去查问的时候,那两个太监已经不在了。秦青呆在那里,想来想去,若是公主嫁给了韦膺或者夏侯沅峰,自己虽然难过可也服气,若是公主真的和那个文弱书生有了私情,自己可是绝不甘心,想来想去,公主自幼贤淑温柔,定是那个降臣勾引公主,若非是雍王有话在先,只怕他就跑去责问江哲了,所以接下来的时间,不管是看杂耍还是干什么别的,秦青都是心不在焉,到了巳时,秦青看看那些中下级的官员基本上已经都来了,便找了一个侍卫带路去见江哲,那些侍卫早就得到雍王的吩咐,所以便带着秦青走向寒园。

秦青虽然是满腹怒火,可是他毕竟是将门虎子,一路走来也是心生好奇,这江哲既然是天策帅府司马,长史石彧又要赴幽州辅佐世子,那么在雍王府这人就是一人之下万人之上的人物,可是这越走越冷清,好像是极为偏远的客院。不由自主的,秦青问引路的侍卫道:‘怎么江司马住在这等偏僻的地方?‘

那个侍卫笑道:‘秦将军有所不知,江司马喜欢清净,所以特地拣了寒园居住,没有事情,就连园门也很少出呢?‘

秦青心中既然有了猜忌,不免胡思乱想,这人住得偏僻,莫非竟是想和公主私会方便么?

到了寒园,秦青便发现这里果然戒备森严,光自己看到的就有十几个侍卫,引领自己的侍卫向门前的侍卫说明情况,那个侍卫进去之后,一会儿便出来道:‘司马有请秦将军。‘

秦青走进寒园,却见里面果然也是清幽冷落,看来这江哲确实喜欢清净,他一眼就看到胡威站在一间雅轩门外,胡威是雍王的亲信属下,秦青是很清楚的,看来雍王对江哲果然是十分重视,说不定江哲和公主的事情就有雍王撑腰呢,秦青心中怒火更加炽热。

我正在看书看的兴起,突然胡威进来禀报说秦青秦将军前来求见,我一愣,这人对我当众无礼,今日来见我做什么?想要不见,又想起若非是重要事情,怎么雍王会安排他过来见我,只得放下书册,也懒得更衣,反正也不是公事,就一会儿的时间,我也犯不着麻烦。

一会儿,秦青走了进来,一进来便愣愣的看着我,我心里奇怪,挥手让胡威出去,问道:‘将军此来有何要事,请恕下官衣着随便,居室之中随意惯了,将军请坐。‘

秦青默默的坐下,看着对面那个青年,一身宽松舒适的青袍,长发没有束起,只是用发簪挽了一下,神色悠闲平静,秦青有种强烈的感觉,面前的这个青年根本不是俗世之人,他真的和公主有私情么?秦青心想。

我看这位俊伟的将军一直沉默不言,不由有些烦闷,便冷冷道:‘将军到底有什么事情,若是无事,请恕哲体弱,不便久坐。‘说完,我端起茶碗,品了一口这绝顶的蒙山茶,这可是贡品中的极品,就是雍王殿下也只有几两罢了,分了一半给我,是我的最爱,平日只有这样悠闲的日子我才会泡上一杯。谁知我刚刚喝了一口,就听秦青冷冷道:‘你真的和长乐公主有染么?‘

‘噗哧。‘我口中的茶水全部喷了出来,我愣愣的看着秦青,有些结巴地问道:‘秦将军,你说什么?‘

秦青冷冷的看着我,道:‘我问你是否和长乐公主有私情。‘

我下意识的发挥自己的长处,不错,胡威离得很远,应该不会听到,为什么会有这样的问题呢?我看向秦青,问道:‘秦将军,恕我直言,您和公主可有什么关系?‘

秦青一听,脸涨的通红,道:‘没有?‘

我觉得身上的寒毛都树了起来,知道这人动了杀机,可是想来想去,我总不能让胡威进来,这种风言风语若是传了出去,只怕雍王都保不住我。我镇静地道:‘既然将军和公主并无关系,追问公主私情,这就有些不妥当了,不过将军既然问了,我若不答,未免有些显得心虚,只是此事可一而不可再,还望将军动问之前多用用脑子。‘

我看看秦青的脸色,觉得还有余地,便接着道:‘哲本是南楚降臣,将军鄙弃于我也无可厚非,但是哲平生唯一的好处就是洁身自爱,除了亡妻之外,再没有和别的女子有过私情。将军若是斥责江哲屈膝投降,哲无论如何生气都得听着,只有这等污言秽语,对我来说虽然是过耳烟云,却也不能容你胡说。‘

秦青脸色变了又变,冷冷道:‘你敢发誓么?‘

我鄙夷地一笑,淡淡道:‘将军,江哲此身,上可对苍天神明,下可对黎民苍生,发誓这种事情我是不作的,不过我不妨直言,哲与公主见面相谈只有两次,一次是在南楚,我奉命觐见,一次是日前,邂逅于雍王府,公主乃金枝玉叶,又曾是南楚王后,与哲有君臣之分,秦将军若以此等事看作私情,那天下就没有清清白白的人了。‘

秦青冷静下来,他听得出来我虽然言语凌厉,却是没有一句虚言,想到自己听了谣言前来责问,碰了一个头破血流,还如何遵照父命向江哲致歉,只得一抱拳道:‘是我错了,这是我在王府听两个太监说的,请司马大人见谅。‘

我心中一寒,立刻扬声道:‘胡威。‘

胡威立刻推门而进,我冷冷道:‘有人胡言乱语,触怒了秦将军,你立刻前去把他们带来见我,秦将军,这两人什么模样,在哪里遇见的。‘

秦青原本想不说,但是看到江哲眼中的冰寒,竟然心中一凛,便说了那两人的年纪相貌。胡威听了想了一想,道:‘大人,这两个人属下知道,他们是宫里派过来的公公。请问大人,把人带到这里么?‘

我想了一想道:‘今日殿下设宴,不可惊动客人,你将他们两人抓了,监押起来,等候殿下处置。‘

胡威走后,我看看秦青,淡淡道:‘秦将军,听我奉劝一句,令尊之所以荣宠至今,靠的不是权势凌人,听说抚远大将军为人沉默寡言,平生言出必行,行而必果,最令人敬佩的是,大将军处事公正果决,若无过犯,就是小卒也不轻慢,若有过犯,就是皇室宗亲也不迁就,将军可以想想这些日子以来的行为,可有值得夸耀之处,非我交浅言深,实不忍见大将军后人凋零。‘

秦青原本应该气愤的,但是却觉得江哲所说竟与父亲日常所说意思仿佛,竟然不敢辩驳,想起多日以来被怒火和妒火冲晕了脑子,越想越是羞愧。他本是将门后人,又受严父谆谆教导,虽然一时糊涂,但是终究不是天性,想来想去,竟然觉得心中空明,恭恭敬敬的下拜道:‘多谢先生教诲,青向日得罪先生,请先生原谅。‘

我倒是一惊,想不到这人如此知过能改,不由将他搀了起来,说道:‘将军如此大礼,下官受不起,若有冒犯之处,还请将军见谅。‘

秦青坦然道:‘先生,本来秦青想多听听先生的教诲,只是奉命而来祝贺,马上就要开宴,青不得不出去向雍王殿下道贺,日后若有机缘,还请先生赐教。‘

意外的化干戈为玉帛,我不由心喜,便亲手送他出了寒园。见他走远之后,突然听到有人呵斥道:‘什么人擅闯寒园,还不束手就缚。‘

\chapter{第十四章 坏人姻缘}

南楚同泰元年元月十六日,大雍禁军统领裴云误入寒园,哲喜其豪爽,留之饮,密谈良久,未几裴乃毁婚另娶,时人皆笑之负义薄情,后乃知其明智果决,然哲坏人姻缘,实为智者不齿。

--《南朝楚史·江随云传》

我转头看去,却是一个灰衣青年,仪容不凡,面容沉静,正被两个侍卫拦住,他眼中有些迷惑的神情,似乎不明白为什么这个偏远的地方会有这么严密的守卫。那两个侍卫都是佩刀出鞘,形势紧张,一触而发,虽然这两个侍卫并不看在他眼里,但是他可不会相当真正的刺客,所以并没有反抗。我看过去的时候,他正在沉声道:‘两位兄弟,在下禁军统领裴云,这次到王府赴宴,只因不喜欢吵闹,所以四处走走,并非有意擅闯,请恕在下不知道这里乃是禁地。‘

两个侍卫相视一眼,都是将信将疑,若说此人气度,倒也真的像个将军,可是只见他周身上下流露出来得气息,不仅彪悍非常,而且一举一动,更是带着高手风范,若是此人真有歹意,那么自己还有什么脸面去向殿下复命呢?

我已经认出了裴云,想不到雍王拉拢人真是厉害,裴云一个禁军北营统领,不过是四品武将,虽然拱卫京赍职责重要,但是也没有资格参加雍王府的盛宴,像他这种身份顶多被允许送上一份贺礼,恐怕连入席的资格都没有,现在他赫然出现在王府,恐怕是雍王殿下特意下了帖子吧。想到这里,我微微一笑,让我助殿下一臂之力吧,与其让他在侧厅赴宴,连雍王都看不到,还不如把他留在这里好些。想到这里,我高声道:‘不可无礼,这位是裴云裴将军吧,下官是天策帅府司马江哲。‘

那两名侍卫见我发话,便行礼退下,裴云走过来施了一礼道:‘多谢江大人为末将解围。‘他看向我的目光从容冷淡,这倒是新奇,自我入雍以来,凡是见我的官员,眼中不是好奇就是评估,或者还有鄙夷,这人却被我看成一个普通之人,不免让我对他更有些好奇,于是,我笑道:‘将军想必是不喜欢前面的吵闹,所以到后面走走,下官也是如此,这才在寒园居住,相逢也是有缘,将军到园中坐坐如何?‘

裴云有些犹豫的道:‘殿下的宴席马上就要开始,只怕末将不便留此。‘

我淡淡一笑道:‘将军就是不参加也算不上失礼,那外面的席位也没有什么意思,这样吧,将军如果愿意,随云正要用饭,就请将军留在这里小酌,殿下那里,自有随云担待。‘

裴云心里一动,接到雍王殿下的请帖虽然是荣宠,但是跻身那些官员当中却很不舒服,何况自己无论如何都是只能在外厅赴宴的,真是没有什么意思,如果不是雍王的帖子,自己只要送上一份贺礼就可以了,眼前这人总比那些官员爽朗多了,他的住处如此戒备森严,恐怕雍王对他万分器重,那么自己应邀就不会失礼于雍王,比较之下,留在寒园倒是一个好主意。

我看裴云神色便知道他已经心许,便朗声道:‘去个人,禀报殿下一声,就说裴将军我留下了。‘

一个侍卫躬身应是。我上前拉着裴云的手臂道:‘裴将军快请进,哲对将军的武功深为敬佩呢?‘

裴云有些腼腆地被我拉到花厅,这时候已经将近午时,两个仆人送上酒菜,习惯性的让他们退下,我拿起筷子指着饭桌道:‘裴将军,哲是南人,所以殿下特意专门寻了一个南楚的厨子做菜,请尝尝,看习不习惯。‘

裴云看着满桌的小碟,里面都是色香味俱全的清淡小菜,只尝了几口便赞不绝口,他虽然是无肉不欢的人,但是这几样菜都是南楚名菜,他还是吃得十分开心,我见他喜欢,又倒了一杯酒给他道:‘这是我南楚名酒桃花露,是用每年秋天南楚最上品的灌蜜蟠桃所酿,原本我是没有想到能在这里喝到此酒,这还是我一位故人特意从南楚带来的,昨日才送进来。‘

裴云喝了一口,只觉得如饮甘露,美酒醇香,不过他性子刚烈,不喜欢这种软绵绵的酒,不由皱了皱眉头,我看在眼里,轻笑道:‘看来裴将军不喜欢这酒呢?听说大雍边关有一种烧刀子,辛辣无比,将军可喜爱。‘ 裴云顿时喜上眉梢,说道:‘大人这里有烧刀子么?这酒在长安可不多见。‘

我走到花厅角落,那里有一个黄杨柜子,我从下层取出一个小酒坛,这个小酒坛虽然不大,但是至少也装有十斤酒,我拿来虽然不费什么力可以不容易,裴云连忙过来接过酒坛,提到桌旁,他忍不住看了一眼柜子,里面都是一些小酒坛、食盒之类的东西。

打开泥封,裴云立刻闻到了那让他永远难忘的炽烈酒香,他深深的呼吸了一下,然后急切的把酒倒进我递过来的一只大酒碗,然后很认真地喝了一大口,熟悉的刺喉辛辣让他仿佛回到了边关,他无力的坐在椅子上,再次举起了酒碗,酒液顺喉而下,眼中却是几乎落下泪来,想起当年边关血战,袍泽情深,是多么的快乐逍遥,如今身在京城,虽然荣华富贵,却是知心无人。多想再回边关,可是想到父亲苍老的身影,裴云紧紧闭上了眼睛,强忍心中辛酸苦痛。

我没有想到裴云这样激动,但是很快我就明白了他的心意,看来这位禁军统领最大的心愿就是重新回到沙场啊,可惜,这一点我也没有法子,谁也不能让他抛弃痛丧爱子的父亲,就是他自己,不也是这般为难么?不过看他这般痛苦,我倒突然想到了一些事情,他若是有了子嗣,那么重上战场应该不难啊,便问道:‘裴将军今年贵庚?‘

裴云毕竟是名门弟子,很快就平静下来,抬头道:‘劳大人动问,末将今年二十三岁。‘

我又问道:‘裴将军可成家了么?‘

裴云赧然地摇摇头道:‘家父为我订了一门亲事,但是我却一直不情愿,所以至今未娶。‘

我疑惑的问道:‘这是为何,令尊想必盼孙心切,将军既然孝顺父母,理应早早娶亲才是?‘

裴云看了我一眼,虽然觉得有些交浅言深,但是不知怎么,他对眼前的青年有一种莫名的好感,不排除是那坛烧刀子的缘故,但是他还是觉得情绪十分放松,而且那些事情闷在心里很久,也想找个人说说,便开口道:‘大人有所不知,我练的武功在没有小成之前是不宜娶妻的,不过今年年初,家师就说我已经可以成婚了,不过这还不是主要的原因,最主要的是我的未婚妻身份特殊,师门很不满意。‘

我心中一动,问道:‘请问将军的未婚妻室是谁家的女儿?‘

裴云苦笑道:‘她是工部侍郎薛矩之女,原本两家是通家之好,我和她指腹为婚,从小青梅竹马,也算情意相投,可是我九岁上嵩山学武,十六岁下山之时却得知她竟然拜入凤仪门,师门得知之后,曾经亲自召我回山,戒律院首座慈海师伯亲口对我说,我若是和她成婚,少林虽然不便阻止,但是我从此不能上窥少林神功,他要我好好考虑,所以我至今不愿完婚,几次想要退婚,那边都不同意,岳父说女儿没有失德,若是我无端毁婚,必要到皇上面前评理。家父近来每每催逼,若非我以死相抗,只怕早就被迫完婚了。‘

我暗想,看来少林果然和凤仪门芥蒂极深,殿下的情报没有差错,而且裴云如此轻易说出,看来少林对于和凤仪门反目并不介意。但是我在口头上却问道:‘这下官就不明白了,这婚姻之事父母之命,媒妁之言,将军若要成婚是理所当然的事情,为何贵师门却强行阻挠,这岂不是有悖人伦,莫非将军将门中神功看的如此重要么?‘

裴云低头道:‘末将虽然痴迷武功,却非忘情负义之人,若是她只是平常女子,我就是宁可被师门追回武功也不愿相负,只是七年前我初回长安就去拜见岳父,见了她一面,她变得很厉害,全不像小时候那样纯真无邪,虽然现在她相貌气质都是万中无一,又练了一身好武功,可是我却觉得她总是离我很远,她的笑容虽然甜美,却是再难让我动心,而且,她总是和那些身份仿佛的女子聚在一起,不是出去打猎冶游,就是在长安都市上纵情放肆,虽然我不是那些见不得妻子出色的人,可是我还是希望她能够相夫教子,侍奉双亲。事实上,两年前我从边关回来,原想不再考虑武功的进境,早日成亲,让父母可以含饴弄孙,可是再次见到她,心中不满却是丝毫没有消减,她确实美丽出众,才情过人,可是我要的是一个肯相濡以沫的好妻子,日后成亲,她要替我侍奉父母,而我还想重上沙场,为国效命,可是,她是做不到的。每次相见,她不是谈论天下大事,就是谈论江湖风云,我真的不希望娶一个这样的妻子。‘

我默默的看着裴云,知道他句句都是肺腑之言,对于一个沙场猛将来说,他需要的不是美丽的画图中人,他要的只是一个可以持家的好妻子,凤仪门大概没有想到这一点吧,不是所有男子都喜欢那些容貌绝色、才情绝世却不能善于应付柴米油盐的妻子的。

想到这里,我淡淡一笑,道:‘其实将军过虑了,世间没有不偏爱子女的父母,若是将军和尊亲说明娶妻娶贤的道理,老人家也不会不明白,若是碍于岳家不肯,将军不妨先在外面娶个侧室,等到生子之后,堂上双亲见到孙儿,难道还会生气么?‘

裴云心中一动,到时候自己想用练武的借口拖延完婚只怕就行不通了。我看到他的神色,明白他已经愿意,只是还有碍难,便道:‘将军屡次要求退婚,是对方不肯罢了,想必将军退婚的理由也不够充分,而且也不想得罪岳家,到时候将军不妨说自己无意中在外面和别的女子结下孽缘,又不能弃之不顾,就是对方有再大的背景理由,也不能阻止将军纳妾吧,若是他们一怒退婚,正好合了将军心意,若是坚持要把女儿嫁过来,这夫妻之间的事情,难道外人还能过问,只要将军专宠爱妾,堂上两老又疼爱孙儿,只怕没有多久,尊夫人就会提出‘和离‘。‘

裴云有些不忍地道:‘此计虽然好,只怕太过伤人。‘

我淡淡道:‘虽然伤人一时,但想必将军的未婚妻子追求者很多,将军若是勉强娶了不中意的妻子,将来夫妻失和,上不能孝顺父母,下不便教养子嗣,这才是有违人伦,若是那位薛小姐是贤德淑良的女子,下官这样说,是坏人姻缘,罪在不赦,可是想必薛小姐--‘

我没有再说下去,但看裴云脸色一阵红一阵白,想必我说得不错,凤仪门的弟子有几个不是喜欢抛头露面的,再说大雍风气开放,就是平民家的女子也不会大门不出二门不迈,更别说那些出身显贵的豪门女子了。

又过了片刻,裴云已经神色镇定下来,脸色微红的向我致谢,我笑着道:‘今日将军心事全消,不如多饮几杯。‘裴云举杯相敬,我则是倒了一杯桃花露,烧刀子我可消受不起。

刚才的推心置腹让我们两人开始亲近起来,所以说起话来渐渐不那么拘束,这裴云说起军旅之事津津有味,他曾是齐王麾下勇将,所以他说得很多事情都和齐王有关,虽然齐王不是什么名将,但是他悍勇无畏,而且肯听从幕僚的意见,所以在军中也受到将士敬仰,裴云说起他来也是十分尊敬,看来不可轻视齐王啊,从前他两次攻打襄阳失败,实在是因为襄阳的守备森严,而他的出兵却没有整体的战略目标,我曾听雍王说过,那次出兵是太子殿下一手推动的,想来因为那些事情轻视齐王,还真的有些冤枉他,只要给齐王派几个好的幕僚,齐王足可以独当一面,镇守一方的。

我们谈得正十分投机的时候,我听到前面传来的开宴的曲乐声,虽然隔着重重屋宇,还是依稀可以听见,知道雍王那里已经开宴了,便笑道:‘今日我阻拦你参加殿下的盛宴,不过你也不算吃亏吧,我这儿的酒你一定很满意。‘

裴云笑道:‘多谢江大人的烧刀子,若非太过唐突,我还想将一坛酒都拎走呢。‘

我刚要答话,突然我的耳中传来低低的呻吟,我心中一凛,侧耳细听,又是一声急促的呼吸,伴随着骨折的声音,天啊,有人在狙杀守卫寒园的侍卫,我强自镇定下来,周围的守卫大致的位置我都清楚,听这两个被杀的侍卫的位置应该很近了,其他的侍卫都在百步之外,这样小的声音我是听不见的。看了一眼裴云,他没有发觉这件事情,然后我听到有人推开园门的声音,这个声音我想裴云注意到了,但是他只是略一凝神罢了,我看看他的神情,并没有什么异常,看来他以为是寒园的仆人罢了。我放下酒杯,怎么会有人杀死寒园的侍卫呢,我判断其他方向的侍卫恐怕都已经遇害,否则不会没人注意到有人擅自进入寒园。看看时间,正是前面盛宴正酣的时候,大部分的侍卫都在前面防卫,所以他才这么容易闯进来吧。怎么办,我手无缚鸡之力,看看裴云,他是否靠得住呢,毕竟他曾是齐王的部下。

裴云奇怪的看看江哲,怎么他突然沉默下来,而且神色有些古怪,他不由提聚功力以防万一,可是就在这时,他耳边传来轻微的脚步声,那是踩到园中积雪的声音,裴云心中一凛,他听得出来,来人的轻功极为高明,从这个声音来看,恐怕积雪上只会留下一个小坑罢了,难道是雍王府的高手么,裴云这么想,可是不知怎么他感觉不是,因为那种小心翼翼不像是雍王府的人,他看看身上,没有带兵器,虽然他擅长拳脚,可是有一把兵器还是好的,他立刻低声对江哲说道:‘江大人,外面有人来了,好像不是王府的人,你这里有兵器么?‘

我看了裴云一眼,看来他是可以信任的,我的性命暂时就要靠他了,来得人恐怕是不怀好意的,可是我身边没有什么可以惊动前面的侍卫的东西,那些被杀的侍卫身上倒有铜哨,可我根本不可能取得,不知道裴云能不能挡住外面的人,如果没有人及时赶到,恐怕我的性命就完了。

我没有犹豫,从那个柜子里取出一柄匕首,这是这间屋子里唯一的武器,我头上的那根发簪虽然锋锐无比,可是我不指望裴云可以用它。

裴云皱了皱眉,把匕首塞给我道:‘你留着防身。‘我苦笑着看看这把精致的匕首,这原本是用来切割肉类的小刀,若在高手身上可能可以追魂夺命,可是在我身上有什么用呢?可是我还是收下了,裴云既然是拳脚功夫厉害,给他他也用不上。这时,门外有人轻声道:‘江先生,殿下知道先生不愿到前面赴宴,特遣属下送来御酒。‘

裴云神情一松,尴尬的看了我一眼,似乎觉得自己太敏感,我却拉住他,摇摇头,不可能是雍王的人,殿下是知道我的习惯的,绝不会派个陌生人来送酒,如果他打发小顺子回来倒是正常的,可是一个陌生人,是不可能的。

我淡淡道:‘门外是哪一位,请进来说话。‘

房门悄然打开,走进一个身上穿着侍卫服饰的中年人,相貌平平,让人过眼即忘,我一眼就知道他不是雍王府的人,而且我闻到他身上带着两种气味,一种是厨房的油烟味道,一种是淡淡的血腥气,看着他,我冷冷问道:‘你就是最近来的南楚厨子?‘

那个中年人一愣,裴云也古怪的看着我,我不理会他们的惊疑,又冷冷问道:‘为什么要来杀我?是谁指使你来的?‘

裴云立刻紧紧的盯着那个中年人,眼中满是警戒。

那个中年人的神情突然从平和变得狰狞冷酷,一霎时,那个平庸的侍卫不见了,显现在我们面前的是一个冷血的杀手。裴云上前一步,挡在我身前。

那个中年人突然笑了,他问道:‘江状元怎么会知道我是杀手呢?‘

我的神色变得凄冷,淡淡道:‘我知道你,你是毒手邪心,南楚军中第一杀手,从前听命于德亲王赵珏,现在恐怕已经听命容渊了吧?‘

那个中年人神色变得严肃,他冷冷道:‘怪不得亲王遗命,若是江哲降敌,必然要尽力杀之。‘

听到这里,我再也忍不住,一口鲜血从嘴角涌出,我缓缓的坐了下来,闭上了眼睛。

\chapter{第十五章 黄雀在后}

同日,南楚刺客突然而至,哲侍卫尽丧,哲受箭伤,几伤性命,赖裴将军云相救得免。

--《南朝楚史·江随云传》

裴云大惊,他虽然没有转身,但是却可以感觉到我的气息,出言问道:‘江大人,你怎样了?‘

我从怀中取出一个玉瓶,将里面的药丸服了一粒,胸中翻涌的气血渐渐平静下来,我抬起头,淡淡道:‘我知道你一向替亲王殿下效力,当日若非你奉命在襄阳保护容先生,只怕殿下未必会遇刺身亡。‘

那个中年人微微低头,眼角闪过一丝泪光,冷冷道:‘江大人昔日对殿下情意深厚,不远千里前来相救,可惜殿下福薄,殿下临终,曾私下对我说,江大人若是投了大雍,南楚危矣,要我立誓,若有这样的事情发生,一定要取了大人性命,殿下说,大人会明白他的,刺杀大人,是殿下为了南楚不得已而为之,他请大人原谅一个已死之人。‘

我淡淡道:‘我不会责怪殿下,殿下至死仍对南楚忠心耿耿,我却是没有几分忠心,殿下生前能够容忍随云,已经让随云感激万分,阁下放心,今日我若生还,当日殿下所托,随云不会忘记,若有机缘,必定不负所托。‘

那个中年人神情一愕,继而恢复正常,淡淡道:‘江随云果然气度不凡,此次杀你,也是我自己的主意,国家兴亡,匹夫有责,我不能忍见南楚覆亡,当日殿下每每在我面前叹息,说若是江大人肯全心辅佐南楚,则江山永固,若是大人投了大雍,则南楚覆亡无日,如此南楚内忧外患,若不杀你,日后必然后悔。‘

我看了他一眼,正要继续说话,反正拖延时间也不错,他却似乎看破了我的心思,身影向我扑来,裴云迎上,两人瞬息之间交换了几章,狂猛的劲风杀气迫得我退到了墙角。

看着他们苦战,我的心思却陷入回忆之中,当初从蜀中回到建业,我遭遇大变,养病期间,小顺子早就发觉德亲王派了人监视我,虽然知道一时之间还不会有什么变化,但是不可不防,所以在秘营建立之后,我曾经让小顺子仔细调查过德亲王身边的人,而这个毒手邪心就是德亲王最信任的心腹之一,此人始终隐身暗处,他擅长的就是刺杀,虽然因为德亲王的性情,这个人没有起到太多的作用,可是我早就将他列为有威胁的人物,如今,他在我疏忽的时候出现了,谁会想到他会在亲王死后,在戒备森严的雍王府刺杀我这个普普通通的降臣呢?唉,当日我就知道德亲王的赤胆忠心,想不到他临死仍然留下了对付我的遗命,我虽然能够谅解,可是仍然有些心寒如冰。

我苦笑着看向前方,裴云正面色凝重的和毒手邪心交手,只见他一招一式似乎简单明了,可是却仿佛铜墙铁壁一般阻拦着毒手邪心那如同水银泻地一般无孔不入的杀招,虽然还是一个平手的结局,可是我看裴云神色间的凝重,就知道恐怕是落了下风的。看看房间,只有一个窗户,门口被交战的两人堵得严严实实,拖着疲软的身子,走到窗前,奋力推开窗子,遗憾的看到外面是一丛蔷薇,要说我这个园子,虽然整理过,但是毕竟没到春天,所以杂花杂草还是不少,例如窗子外面的野蔷薇,虽然没有开花,但是花茎上的利刺一点不少,若是我跳了出去,只怕要遍体鳞伤了,打了一个冷颤,决定不到万不得已还是不要跳出去的好。

这时毒手邪心已经有些焦急了,他不是容易混入雍王府的,而且虽然在王府里呆了一些日子,可是这里规矩严谨,他跟本不能接近江哲,平日江哲身边侍卫众多,而且每隔一拄香的时间就有一队巡视的侍卫经过,若是惊动了他们,自己就是三头六臂也是逃不出去的。而且江哲身边的小顺子虽然不知道武功如何,可是他毒手邪心也算是一流高手,看不出深浅代表着什么他清楚的很,难得今日机会来了,小顺子不在寒园,而今日雍王宴客,大批侍卫都在前面,寒园这里的守卫松懈了许多,按照他的观察,半个时辰之后才会有巡视的侍卫经过,所以他大胆的狙杀了所有侍卫,将他们的尸体隐藏起来,这样自己就可以有一段宽裕的时间刺杀江哲,唯一没有料到的是,江哲身边居然有一个少林高手,一套罗汉拳炉火纯青,这套少林防守最严密的拳法竟然硬是挡住了自己。时间不多了,毒手邪心下了狠心,突然一声厉喝,面色变得血红,嘴角渗出鲜血,掌法突然一变,功力倍增,掌法更是多了几分诡异,‘嘭‘的一声,两人手掌相交,裴云面色一白,退了一步,还未来得及还手,毒手邪心已经如影随形,再次扑上。

‘嘭、嘭、嘭‘,接连三次对击,裴云被毒手邪心逼退了三步,已经快要碰到桌子了,掌风激荡中,那坛烧刀子酒坛被波及到,霎时间坛碎酒溅,裴云灵机一动,后退一步,一脚把酒坛替到半空,然后出掌拍出劲风,这下满屋都是酒水,滴滴酒水混合了裴云的真气,毒手邪心不得不双手在身前划出万千掌影,挡住了这些‘暗器‘,这时候裴云冲到我身边,一把将我扛到肩上,合身向窗子冲去,碎裂的木片打得我脸上生疼,裴云脚上的皮靴毫不犹豫的在干枯的蔷薇花丛上点了一下,然后晕头转向的我发现已经身在园中。

身后一声怒喝,毒手邪心已经冲了出来,只见毒手邪心的身形如同闪电一般快捷,向我扑杀,裴云紧紧的护着我,虽然形势更加险恶,毒手邪心的武功本就擅长四面出击,让裴云的防守捉襟见肘,但是地势开阔也有好处,裴云护着我东躲西藏,总算暂时保住了我的小命,可是这样下去是不行的,不说别的,我刚刚病发,此刻手足酥软,这样躲来躲去,我已经气喘吁吁了,只怕再过个十招八招,我就要瘫倒了。

裴云也看了危险,心道只有拼命了,他的面色突然变得庄严肃穆,肤色隐隐带着金色,他不再闪避,抛下我向毒手邪心扑去,毒手邪心一见裴云的宝相庄严,惊道:‘无敌金刚力。‘不敢怠慢,两人身形相交,猝然分开,裴云仿佛没有感觉一般又扑了回去,毒手邪心面色有些苍白,这还是第一次和裴云比拼内力失利呢。他却不知裴云也不好受,他的无敌金刚力只练到七成火候,这次他这样不顾性命的全力使用,若是超出一拄香的时间,只怕他就会受到严重的内伤,就是性命无碍,日后也不能再精进了。虽然冒险,可是他还是义无反顾,不是为了江哲对他的厚遇,也不是为了讨好雍王,更不是为了保护南楚降臣给大雍带来的好处,他心中全然没有立功的念头,此刻他心中唯一的念头就是师父收自己为徒时候的训诫--保护善良无辜。他从不觉得江哲投降大雍有什么失节之处。

我虽然不懂武功,可也知道超越常情必然会有后患,裴云突然武功激增肯定不能持久,看看毒手邪心被他缠住,我撒腿就往寒园门口跑去,那里应该有侍卫的遗体,只要找到他们身上的铜哨,我就可以求援了,那些铜哨精工制作,就是我吹起来,也能让全府听到。

毒手邪心几次想要追杀我,都被裴云挡住,他杀机更炽,面色再次变得血红,功力再增,这一次他忍不住呕出一口鲜血,奋起一掌将裴云击退,正要向我扑杀,裴云已经拼死挡住,一时之间,他也有些犹豫,天魔解体大法若是使用到三次以上,自己就会七窍流血,虽然功力可以增加到三倍,但是事后恐怕要休养数年,想了想,自己功力武技都在这个年青人之上,再有十招就可以杀了他,到时候自己就是再去追杀江哲也来得及。

我赶到寒园门口,在草丛里面找到侍卫的遗体,可是我心中立刻一片冰冷,那些铜哨被扔在尸体身边,却被都已经毁坏了,毒手邪心果然行事周密。我茫然的看向四周,怎么办,怎么办,我恐怕根本就逃不掉了。咬咬牙,我打量一下四周,哪里可以藏身呢,不是我想临阵脱逃,我若走了,裴云还可以脱身,我若不走,裴云只有战死一条路了。突然,我想起在居室里面藏有一些防身的毒药,我连忙又向园中走去,磕磕绊绊的跑向居室。

正在交手的两人见我又回来了,毒手邪心松了口气,心道只要他还在,我就可以专心的和这人交手了,他这一放缓,裴云轻松了很多,可是他心里却是十分焦急,为什么江哲又回来了。

两人心中都有疑问,又拼了几招,裴云已经有些支撑不住,心道,没想到我没有死在沙场,却死在这雍王府的寒园之中,刺客的手上。虽然如此,但他心志坚毅,仍然不肯松懈。毒手邪心也不着急,再过片刻,自己就可以达成任务了。这时,我拿了一个精钢圆筒匆匆忙忙走了出来,看向两人苦苦相斗之处,大声道:‘裴将军放心,我这毒药虽然厉害,可是不会立刻致命,我会给你解药的。‘

说着,我向着两人按动机关,从圆筒中弹出一粒红色的弹丸,在两人头上爆裂,粉红色的烟雾立刻将两人笼罩在其中,毒手邪心大惊,他是知道江哲精通医术的,那么有些毒药防身也是正常的,他连忙飞身想退,却被奋起余勇的裴云狠狠缠住,他只得屏住呼吸,谁知那些烟雾一接触到他的肌肤,就觉得四肢麻木,裴云虽然也有同感,但他所练的武功是正宗佛门神功,所以多忍了十几息的时间,因此一章击中了毒手邪心的小腹,毒手邪心的身形一震,倒在地上,但是却也被掌风推出了烟雾的范围。而裴云也身躯摇摇欲坠,跌倒在地。

我大喜过望,连忙跑了过去,从一个翠玉瓶子里面倒出解药塞到裴云口中,片刻,他坐了起来,声音嘶哑地道:‘毒已经解了,大人放心,云这就护着大人到安全之处去。‘

我搀起裴云,感激地道:‘多谢将军相救,咱们快点离开,若是还有刺客就糟了。‘

裴云也是这样想,若还有刺客,他是无力保护我了,我们两人走向园门,两人互相搀扶,都是筋疲力尽,刚刚踏出园门,我就惊觉远处的杀气,耳中听见弓弦轻响的时候,一支白羽箭已经如同流光飞逝一般没入了我的心口,我愣愣的看着胸前的羽箭和立刻渗出的鲜血,想不到我的生命竟会这般失去,奇怪的,我心中没有丝毫的恐惧,也没有什么仇恨,我不怪那将我杀死的人,人生在世,弱肉强食,他自然会有他的理由。看向羽箭飞来的方向,那隐在暗处的手持弓箭的刺客也正在冷冷的看着我,他一身蓝衫,面上蒙着雪白的丝巾,一双清澈如春水的眼睛带着一丝遗憾的看着我,我能够觉察到身上生命的流失。耳边传来裴云的惊呼,但是我已经没有精力去多想了,临死之前,我心中泛起飘香的倩影,然后是柔蓝小小的身影,最后想起的则是那个一直跟在我身边的清秀少年的身影,眼前的视线已经不清了,朦胧中我看见小顺子满面惊骇欲绝向我飞扑过来,真是很遗憾啊,我还没有机会托付他照看柔蓝呢,不过我想他会知道的,带着淡淡的微笑和遗憾,我终于闭上了眼睛,意识向无尽的深渊沉入,沉入。

所以我没有听到那声凄厉的充满绝望的悲鸣。

雍王府的大殿上,李贽笑着向诸位贵宾敬酒,他眼睛扫过众人,秦青在开席之后不久就告辞了,李贽已经知道他到寒园似乎和江哲发生了一些纠葛,但是看他神情,应该已经前嫌尽逝,虽然还有些芥蒂,但应该不要紧了,噢,夏侯阑、夏侯沅峰父子都来赴宴了,夏侯沅峰职位较低,在偏殿赴宴,此人可不能小看,能够得到父皇宠爱数年不衰,可是不容易,若非此人已经投靠太子,还想染指皇妹,只怕自己也想招揽他呢,文武全才,不愧是大雍军中第一青年高手,从他战败裴云之后就已经稳占魁首之位了。裴云没有来,自己最看好的其实就是裴云,虽然他是齐王的旧部,可是这人也是少林的俗家高手,而且对凤仪门有些不满,应该可以招揽的,虽然没来有些可惜,可是还有机会的。

李贽的眼光掠过,看到仆人装束的小顺子站在殿角,一双冰冷的眼睛看着殿上的百官,这个小顺子只忠于江哲,虽然不知道他的武功如何,可是应该不弱于裴云和夏侯沅峰,他可是一个得力的属下,看他主动要求在这里观察那些可能是敌人的宾客,就知道他的心机了,若非他这般忠心,李贽还想过将来将他安排在宫里呢。

这时,李贽看到一个侍卫匆匆忙忙走了进来,在负责宴席安排的苟廉身边低声说了几句什么,苟廉眉头一皱,吩咐了几句,然后苟廉便走到小顺子身边,说了几句话,小顺子脸色一变,悄然退了出去,苟廉正在向自己走来,可是这时几个朝中显贵也围了上来,李贽一时脱身不得,等到终于找到时机的苟廉接近自己,他低声道:‘殿下,事情不妙,保护随云的副总管胡侍卫和两个属下被人狙杀在内府,旁边还有两个太监的尸体,我已经派人去保护江司马。‘

李贽大惊,连忙道:‘本王要去看看。‘苟廉道:‘现在殿下恐怕不能脱身。‘

就在这时,从寒园的方向传来了清晰的悲鸣,那悲鸣中充满了一种绝望的哀痛,充满了失去至亲的悲痛和仇恨,那声音尖细凄厉,虽然这般遥远,仍然刺得人耳中疼痛难忍。李贽手中的酒杯落地,摔得粉碎,他心中充满了不祥的预感,这个方向,这个声音,他知道只有一种情况的发生才会如此。猛然站起,李贽怒喝道:‘众人听令,守住王府上下,不论贵贱,不得擅自出入行动,随本王来。‘说罢,李贽一抖锦袍,向寒园奔去。他心中的焦虑胜过当日江哲严辞相拒的时候,他一边走一边默默向上天祝祷,若是能够保佑江哲无事,本王情愿折去寿元相代。

紧赶慢赶来到寒园,只见寒园已经被先派来的侍卫亲兵护住,李贽冲进园门,立刻愣住了,只见园中地上处处是殷红的鲜血和血战后的痕迹,除了自己派来的侍卫之外再没有江哲主仆的身影,在居室门前,几个侍卫凛如寒蝉的站着。李贽恍恍忽忽地走到门前,却见软榻之上,江哲面色苍白祥和地躺在那里,心口插着一支折断的羽箭,而小顺子正跪在软榻之前,紧紧的握着江哲的右手。

李贽只觉得心口剧痛,几乎就要晕倒,半句话也说不出来。

\chapter{第十六章 生死关头}

哲性命垂危,王以玄参救之,遂一丝魂系,齐王李显、长乐公主皆送药相救。月半时日,随云日夕徘徊生死,王终日衣不解带,食宿皆在寒园,闻者皆叹服。

--《南朝楚史·江随云传》

他勉强开口道:‘随云怎么样了?‘

小顺子回过头来,清秀的面容此刻异常狰狞,满眼的血红更是令人见而生畏,道:‘公子不知为何仍有一丝呼吸,奴才以真力为公子续命,方才侍卫已经去请太医了。‘

李贽心略为一宽,连忙道:‘去王妃那里取父皇去年赏给我的千年玄参,上好的人参也拿一些来,先煎一些参汤为江先生吊一吊性命,若是御医觉得可以,就把玄参也煎了。‘

小顺子眼中流露出感激,但是却没有心力分神说话,不到片刻,几个侍卫几乎是挟持着两三个御医赶来,几个御医在路上已经得知伤情,进屋来顾不上向李贽见礼,立刻到了软榻前,替江哲处理伤势,他们忙忙碌碌,取箭,处理伤口,一盆盆的血水端了出去,煎好的参汤也及时送来,一碗参汤灌了下去,果然江哲气息渐渐粗了一些,但若非小顺子以内力相助,只怕仍是随时会命丧黄泉。

几个御医商量了一下,走上前来对李贽说道:‘殿下,那株玄参药力过强,请殿下分三次煎药,每隔四个时辰服一次,然后也不能间断,可以用上好的人参吊命,这样至少半个月内这位大人性命无虞,这位大人也是命大,他的心脏偏了一分,所以这一箭虽然伤了心脉,但是总算没有当时毙命,可是接下来我们真的无能为力了。‘

李贽黯然跌坐在椅子上,摆手道:‘立刻去办。‘有人领命下去。李贽冥思苦想了一会儿,突然问道:‘谁知道医圣桑先生身在何处?‘

众人面面相觑,医圣行踪缥缈,如神龙见首不见尾,如何得知,李贽绝望地道:‘若能找到医圣,还有一线生计,立刻派人去找。‘

小顺子突然喊道:‘殿下,公子是医圣弟子,也颇精医术,能不能让公子清醒一会儿,让他先开个方子,维持住性命再说。‘

李贽惊喜交加,道:‘真的,随云竟是医圣弟子?‘

小顺子点头道:‘公子少年时曾经在医圣门下学医,只是时间不长,但是公子医术的确出众。‘

李贽看看几个御医,他们商量了一下,道:‘殿下,我们可以用一副猛药,让江大人苏醒片刻,只是这样以来,恐怕会加重江大人的伤势。‘

李贽断然道:‘医圣行踪不定,若不能维持住江先生的性命到一个月,只怕难以等到医圣前来,你们先准备好药物,等我吩咐,这几日你们辛苦一下,不可离开此地片刻,若是江先生有个三长两短,我必然要你们抵命的。‘

几个御医唯唯称是。

这时董志匆匆赶来,他上前道:‘殿下,现在前面十分杂乱,那些客人很不安,子攸说,请殿下传令,该如何处置。‘

李贽皱皱眉,走出房门,他不想打扰江哲的医治。走到门外,却一眼看到另外一间房间门口站着侍卫,他看了一眼,苟廉立刻上前禀道:‘殿下,侍卫们赶到的时候,禁军统领裴云也在寒园,因为小顺子只说好好安顿他,所以我派了侍卫把他软禁在那里,也已经安排御医替他医治,据说他浑身是伤,恐怕是他保护了随云。‘

李贽惊疑地道:‘裴云怎么会在寒园。‘说着转身走了进去。

这个房间是小顺子的住处,布置的很是冷肃,裴云坐在一张椅子上,上衣已经脱下,满是青紫的掌痕,一个御医正在替他上药。两人见到李贽进来,一起下拜见礼。

李贽摆摆手道:‘你们继续。‘不多时,御医收起医箱,告退出去。

李贽看向十分不安的裴云,叹了一口气,问道:‘裴将军如何会在寒园呢?‘

裴云心道,莫非江大人派去通知雍王自己留在寒园的侍卫也被杀了么,他没有多问,只是将经过详详细细的说了一遍,原来江哲中箭之时,裴云也是震惊得几乎不能反应,正要搀扶,就听到小顺子一声悲鸣,身影如幻如电,转眼间就到了两人身边,小顺子总算跟着随云多年,对外伤医术也知道一些,他知道不能轻易拔箭,便只能点了几处穴道止血,然后渡过真气替随云续命,他看了裴云一眼,神色冷厉,裴云只看他的身法就知道这个少年的武功远在自己之上,连忙简明扼要地说明情况,小顺子抱起随云走进寒园,却看到毒手邪心不知道什么时候已经逃走,地上只留下了片片血迹。

没有多久侍卫们赶来,小顺子让他们立刻去请御医,然后只吩咐他们照看一下裴云就进了房间,裴云自然知道自己暂时被软禁了,但他光明磊落,自然不会畏惧。

李贽听了裴云的话,站起身深施一礼道:‘裴将军,你今日舍身相救江司马,不论他是生是死,本王都感同身受,只是如今情况不明,还请你在王府暂住几日,而且将军伤重如此,也不便回去让令尊担忧,不知道将军可有师门长辈在此,有他们相助,将军的伤势也比较容易医治。‘

裴云连忙道:‘殿下言重了,裴云愿意遵从殿下的命令,末将有两位师叔就在长安城外浮云寺潜修,殿下可以派人前去,两位师叔对裴云十分关爱,必然会立刻前来。‘

李贽点点头,他身边的侍卫大多是军中选拔而出,就是有一些武功极高的,也是外功强过内功,如今更是缺少这样的内家高手,有了两位少林高僧,自己就可以放心江哲的安全了。

这时,苟廉匆匆走了进来道:‘殿下,现在已经查明,守卫寒园的所有侍卫都已经被害,其中一名死在路上,看来是要到前面去的,另外,除了胡威等人之外,还少了那名南楚来的厨子,其余人等都各在其位,相互之间都有人证,基本上可以断定没有人参与此事。‘

李贽冷冷道:‘那些宾客呢?‘

苟廉看了一眼李贽的神色道:‘当时殿下已经开宴,所以几乎所有宾客都在厅中,但有几人有些异常,据仆役所说,这几人事发之时,都不在席上,他们恐怕要殿下亲自征询。‘说着递过一张名单。

李贽接过,上面写着五个名字,分别是魏国公程殊、靖江王郡主李寒幽、虎威将军秦青、禁军统领裴云、大内副总管夏侯沅峰。李贽面色阴沉。

苟廉又道:‘我们在园门外发现了一张强弓和一袋白羽箭,看来是刺客丢弃在那里的。‘

裴云突然插话道:‘殿下,云曾经看到过刺客一眼,这人身材比云略矮,穿的是蓝色长袍,丝巾蒙面,其他的请恕裴云没有看清楚。‘

李贽只觉得心中一动,淡淡道:‘苟廉,我记得秦青穿的就是蓝衣。‘

苟廉道:‘殿下不可妄断,秦将军出身名门怎会作刺客之事。‘一边说一边望了裴云一眼。裴云识趣地道:‘云伤势不轻,请问殿下可否暂时告退。‘

李贽道:‘寒园之内还有几间客房,都已经收拾整洁,请裴将军自行选一间,将军的两位师叔来了,也请在寒园暂住,本王还有要事,请将军好好休息。‘说罢李贽走出了房门。

苟廉连忙跟上。李贽冷冷道:‘若是裴云所见无差,秦青的嫌疑最大?‘

苟廉道:‘也不可这样说,秦青虽然涉嫌,可是‘

走出房门,李贽看向苟廉,冷冷道:‘夏侯沅峰身材和秦青仿佛,而且箭法一样高明,未必不是他所为,还有靖江王郡主,李寒幽虽是宗室女子,却是凤仪门弟子,有传闻说她是凤仪门主座下第九位亲传弟子,凤仪门主就是刺杀的高手,李寒幽若是穿了男装,也可能会是裴将军看到的人。‘

李贽跺足道:‘不论是谁,我绝不放过此人,稍后你再去好好问问裴云,一定要问清所有细节,本王先去见见这几个人,你先去让子攸撤宴,就说本王司马遇刺,无心饮宴,你立刻派人去城外军营,让司马雄带近卫军千人入城,接管雍王府防务,答应军务由董志暂管。‘

苟廉犹疑地道:‘殿下,魏国公恐怕不便强留,还有私自调动军队入城,恐怕会遭到弹劾。‘

李贽冷冷道:‘魏国公不用强留,我不信他会作出这种事情,调动军队一事你不用担心,本王这就入宫向父皇禀告,哼,长安皇城之中,刺客如此嚣张,京兆尹该当治罪。‘

苟廉连忙道:‘殿下深思熟虑,臣这就去办。‘

过了一个时辰,第一服玄参汤药服下,江哲气息粗壮起来,已经不需要小顺子时时刻刻渡气续命,小顺子立刻默默运功,恢复功力,此刻他已经完全冷静下来,在救回江哲之前,他绝不会再冲动的。没有多久,少林达摩堂两位长老慈苦、慈远急急赶来,再看过师侄的伤势之后,方才放心下来,雍王李贽亲自拜托两人代为守卫寒园,两人初时有些犹豫,但是裴云红着脸偷偷地在慈远耳边说了一句话,慈远便欣然答应,虽然不知道裴云是如何说服两位长老的,李贽仍然感激地向三人致谢,然后匆匆出府,飞马赶向皇宫。

事情发生的时候,李援正在后宫长孙娘娘处弈棋,长乐公主在旁边观战,三人共享天伦之乐,正是其乐融融,虽然雍王府出事的消息已经在皇城内流传,但是还没有传到李援的耳朵里面,正在李援苦思冥想的时候,突然宫外一阵喧哗,李援恼怒地问道:‘怎么回事,何人在外喧哗?‘还没有派人出去看看,李贽已经冲了进来,只见他神色狂怒,衣着凌乱,他一冲到李援身边,突然跪倒大哭起来。

李援大惊,这个儿子一向坚韧,自从十岁之后再未见过他流泪,为何今日如此,他顾不得恼怒,连忙起身搀扶道:‘贽儿,发生了什么事情,慢慢说,父皇替你作主。‘

李贽不肯起身,泣道:‘父皇,儿臣今日召宴,为骏儿送行,可是有人趁机闯入府中,杀了二十一名侍卫、两名太监,还重伤了臣帅府的江司马,如今江司马重伤垂危,眼看性命不保,父皇,孩儿如此隐忍,仍然招致大祸,这让孩儿如何还能在长安居住,或是父皇首肯,孩子就要离开长安,到幽州就藩了。‘

李援听得怒火上升,怒道:‘来人,立刻传京兆尹和禁军大统领进宫,他是怎么办得事,竟然让人在雍王府行刺。‘

李贽心中冷笑,知道父皇根本不想追究责任,毕竟很有可能是太子所为,自己就宽宽他的心吧,便道:‘父皇息怒,儿臣认为行刺之人乃是绝顶高手,所以京兆尹恐怕也是无能为力的,只是儿臣实在担心府上的安全,求父皇允许儿臣调动一千近卫充实雍王府宿卫,还有几名宾客涉嫌刺杀,请父皇允许儿臣调查此事。‘

李援冷静下来,道:‘好,一千近卫不算多,你要好好安排,不可让他们触犯军规法令,至于涉嫌宾客,你可以自己处置,不过三品以上的官员或者皇亲国戚你若是要处死,需要得到朕的旨意。江司马伤势如何,他是南楚状元,若是这样死了,恐怕有人会借机造谣,说我大雍无力保护降臣,到时候谁还愿意投降。‘

李贽惨然道:‘江司马心口中箭,若非心脏偏了一些,只怕就要立刻丧命,现在生死还未可预料,儿臣已经用父皇赏赐的玄参替他吊命,另外派人去寻找医圣桑先生,若是找不到人,只怕江司马性命不保。‘

李援叹了一口气道:‘朕这就传旨,令天下各州府寻找桑先生,你放心吧。‘

李贽磕头谢恩,道:‘儿臣府中之事纷乱,需得回去处置。‘

李援点点头道:‘你去吧。‘

李贽起身,刚要离开,长乐公主站起身道:‘父皇,儿臣送送二哥。‘

李援只是摆摆手,表示同意。李贽看去,长乐公主面色苍白,神色之间十分不安。两人走到宫外,长乐公主低声问道:‘二哥,江司马性命果然危急么?‘

李贽叹道:‘若是用玄参吊命,可以保得半月平安,可是为了让他暂时清醒,替自己开方,恐怕只能维持十日。‘

长乐公主面色惨白,低声道:‘十日,桑先生行踪不定,恐怕是到不了的。‘她突然拉住李贽道:‘王兄,我这里也有父皇赏赐的玄参一株,半株我得留给母妃,她身子不好,我需得小心,另外半株我拿给你,还有父皇前些日子赏给我的一副熊胆,我用冰块冷藏,还没有用,王兄一起带去。‘

李贽大喜,玄参、熊胆都是可遇而不可求的,只有父皇那里偶然会有贡品,想不到父皇赏赐给皇妹这些珍贵的药物,他深施一礼道:‘本王代江司马多谢皇妹救命之恩。‘

长乐公主拉着李贽向翠鸾殿走去,一边走一边道:‘皇兄,若是江司马能有机会清醒,你代我向他说一句谢谢,他明白的。‘李贽虽然不明白长乐公主的意思,但是意外得到贵重的药物让他欣喜若狂,也顾不上多想了。

回到王府,李贽一刻不停地去看江哲的伤势,走进寒园江哲的住处,只见小顺子坐在江哲身边,专心的留意江哲的伤势,李贽上前看了一眼,旁边留下来伺候的御医上前低声道:‘方才江大人曾经几乎断气,幸好这位顺公公救了回来,不过已经不用一直渡气了。‘

李贽低声道:‘本王带了半株玄参和一副熊胆回来,你有没有把握多延几日。‘

这位御医喜道:‘若是如此,小医敢保证,至少可以多延十日。‘

李贽欣然点头道:‘本王将药给你,你们一定要尽心竭力,若能救回江司马,本王重重有赏。‘

那个御医连连谢恩,小顺子仿佛没有听见一般,仍然看着江哲,他心中无限后悔,后悔自己不该离开江哲身边,他心中满是杀气,恨不得将仇人千刀万剐。

接下来的日子仿佛是噩梦一般,江哲几次濒危,御医们只能勉强吊住他的性命,随云遇刺二十七日之后,李贽终于狠下心让御医用猛药救醒了江哲。江哲睁开眼睛的时候,看到的是小顺子和李贽毫无血色的面容,小顺子飞快地道:‘公子,你性命危急,若等不到医圣救命,只怕难以生还,如今公子可有什么法子拖延几日,现在王府之中玄参还剩三两,还有齐王殿下送来的一副熊胆,公子怎么办。‘

江哲听明白了情势,低声道:‘去拿我的金针,记得我教你的行刑针法么?‘

小顺子拼命点头,道:‘我记得,记得很清楚。‘

江哲艰难地道:‘在我书房里面有本手抄的针法,那原本是我自创的夺魂金针,共有十三套针法,前面十二套都是用刑的法子,最后面一套是能够迫出人身的全部潜能,救人于逼死之境的法子,这样用刑之时可以让人苦痛而不死,你武功越高,越难免出生入死,我愿本想把最后面的针法教给你,若有急难,好救你性命,所以里面用针的方法我都零散的教过你,这套针法可以将我的生命潜能全部逼出,至少可以保我九日性命,只是用了之后,就没有别的法子了,既然还有玄参、熊胆,我说一张药方,你用针之后,替我服下,可以多延几日。‘

听江哲说完了药方,见御医已经记录下来,小顺子泪流满面,江哲总是时时刻刻替自己着想,他却离开江哲,让他身负重伤,江哲伸出手擦去他的眼泪,低声道:‘不可伤心,我若不幸身死,你将我的计划全部禀告殿下,让殿下作主,免得功亏一篑,你也不要替我报仇,带着柔蓝回南楚隐居,记着,带我的骨灰回去和夫人同葬。‘

小顺子见江哲已经神情涣散,突然叫道:‘公子,你一定要醒过来,你记不记得,害死夫人的凶手仍然逍遥法外,小姐年纪幼小,你若一死,我只能拼死去替你报仇,可是只怕九死一生,若是我死了,谁来照顾孤苦无依的小姐,公子,不成的,没有你,我真的不知道该怎么报仇,你为了夫人小姐也要活下去。‘

江哲神情一凝,微微点头,然后又昏了过去。

小顺子胡乱擦了一下眼泪,看江哲暂时不会有事,匆忙的去取书册。接下来,针灸用药,小顺子能够感觉到江哲的肌肤再颤动,这套针法还不够完善,所以受针之人还有苦痛之感,等到灌下江哲新开的药方,小顺子见江哲已经气息均匀,这才放下心来,突然想起一件事情,小顺子眼中露出凶光,看向御医,方才他们主仆所言都是机密,若给外人知道,恐怕不免生事。

李贽虽然一直在琢磨江哲主仆的对话,但是始终没有头绪,不明白江哲为何从不说起夫人遇害之事以及仇家之事,但是他心思深沉,知道不可多问,如今见小顺子眼露杀气,怎不知他的心思,便道:‘小顺子放心,这位贾太医也是本王信得过的人,他不会出去胡说的。‘

小顺子看了李贽一眼,这些日子李贽全力相救,他也是感同身受,不能不卖雍王的面子,便冷冷道:‘太医,若是你说出去一字半句,休怪我不留情面。‘

说罢手指虽然一点书案,坚硬的红松木桌面立刻留下了一个一寸深的指孔,贾太医身上一阵哆嗦,连忙道:‘小医自会守口如瓶。‘

接下来的时光更加难熬,江哲始终气如游丝,小顺子每日在他身边伺候,神色冰冷,仿佛一切都与他无关。而雍王等人也是愁容满面,这一日,御医来禀报,只怕江哲性命就在今夜,李贽颓然坐下,一句话也说不出来,世子李骏已经去就藩了,石彧也随之而去,若是江哲过世,李贽心生寒意,自己该如何是好。就在他心中惶惶的时候,突然苟廉惊喜交加的跑了进来喊道:‘殿下,殿下,桑先生来了。‘

李贽大喜,刚要站起,却觉得手足发软,竟然站不起来了。

\chapter{第十七章 幕后风波}

医圣桑臣其时采药深山,出山之日,见雍帝皇榜,乃知随云濒死,三日之间,疾驰千里,奔赴长安,至雍王府,随云命悬一丝,医圣妙手回春,哲乃得生,然自此体愈弱。

--《南朝楚史·江随云传》

桑臣一到王府便直奔寒园,带着几个御医和小顺子打下手,闭门不出,只是不时吩咐下来各种事情。

李贽等在门外,心中焦虑无比,裴云伤势已经全然好了,听说桑臣到了雍王府,也赶来站在门外等候,他对江哲十分感激,自己按照他的法子和师门商量之后,两位师叔虽然没有说话,但是已经默许,虽然有悖佛门慈悲为怀的理念,可是也顾不得了,裴云是他们精心培养的护法弟子,断然不能和凤仪门有所瓜葛。为此,他们特意将身上仅有的两粒小还丹分了一粒给江哲服用。

众人在外面等了整整一天,直到第二天晚上,才看见桑臣等人满面疲惫的走了出来。医圣已经是年将七旬,虽然年迈,须发皆白,但是仍然身体健朗,这两日御医们几乎还要轮流上阵,他却是始终没有走出房门一步。

走出房门,桑臣一眼看到李贽,上前施礼道:‘老朽多谢殿下费劲苦心,若非有殿下用贵重药物续命,只怕随云等不到老夫相救了。‘

李贽终于松了一口气,软软地坐倒在侍卫们搬过来的椅子上面,疲倦地道:‘桑先生,是本王要多谢你救回了江先生啊。‘

桑臣微微一笑道:‘我和随云,情同祖孙,我救他也是理所当然,不过虽然他现在已经平安,但是接下来的调养还有费很大心思,老朽只得叨扰殿下了。‘

李贽连忙站起道:‘自然,就是桑先生不说,本王也要请先生暂留王府的,不论有什么需要,请先生告诉本王,一定不会让先生失望。‘

桑臣点点头道:‘老朽也累了,请为老朽准备住处,明日我再来为随云诊治,老朽就住在寒园吧,可以随时照顾随云的身体。‘

李贽连连答应,他早已令人在寒园为桑臣准备住处。

这时王妃派人过来劝李贽回后府休息,这些日子以来,李贽几乎吃住都在寒园,根本没有回去,这下他终于安下心了,这才回到住处,王妃带着两个侧妃和侍女们伺候着李贽沐浴更衣,好好吃了一顿美餐之后,李贽终于心无牵挂的躺在床上睡着了。

一觉醒来,已经是日上三竿,李贽起身,两个侍女过来帮助李贽整理衣衫,李贽笑道:‘王妃呢?‘

王妃从外面走了进来,笑道:‘殿下大喜,方才寒园派人来禀报,江司马已经醒了。‘

李贽大喜道:‘医圣果然名不虚传,一夜之后,随云就醒了。‘

王妃忍着笑道:‘殿下,已经过了两天了,您这一睡怎么也叫不醒,桑先生过来看过,说您是前些日子太过劳心劳力,只要睡醒了就好了。‘

李贽苦笑道:‘怪不得本王饥肠辘辘,快拿些吃的来,本王要去寒园看望江先生。‘

王妃拉着李贽到外间用餐,一边走一边道:‘殿下一会儿带着柔蓝去吧,这些日子不敢告诉她江司马的事情,她已经哭闹了好几回了。‘

李贽点点头道:‘也好,你也陪我去一趟,然后进宫去告诉长乐一声。‘

王妃诧异的看着李贽,道:‘殿下不是说此事不妥么?‘

李贽苦笑道:‘长乐一知道江司马重伤,立刻就把父皇赏赐给她的玄参送了一半给本王,前两日还派人来问,是否需要另外半株,若是医圣还未到,我恐怕真的要去借玄参了。看来长乐对江司马确实用情极深,我就算不能成全她的心意,也不愿她终日担忧。‘

王妃点点头道:‘也好,这样吧,我带着柔蓝一起进宫,就说带给长孙贵妃看看,这个孩子虽然年幼,但是聪慧可爱,长乐也很喜欢她。‘

李贽点点头,王妃又道:‘殿下,您虽然放了秦将军他们,可是这些日子以来一直派人监视,而且明目张胆,昨日秦夫人亲自来拜访,对妾身说绝非秦青所为,不过大将军还是把秦青关了起来细细盘问。‘李贽冷冷道:‘这些事情等我和随云商议过后再说吧,若是他所为,本王绝不会饶了他。‘

王妃犹豫地道:‘殿下,您现在和太子他们势同水火,如果再得罪了秦大将军,妾身实在放心不下,而且夏侯氏深得皇上宠幸,靖江王郡主这次又是奉父命前来,无论得罪了哪一个,都是很麻烦的。‘

李贽顿了一下,淡淡道:‘没什么,有些事情迟早要解决的,只是秦青若真是作出这种事情,真是有辱门楣。‘

王妃小心翼翼地道:‘我看这孩子不会这样做的,大将军家教严谨,这孩子秉性善良,虽然有些鲁莽,但是这种暗箭伤人的事情他是做不出来的。‘

李贽犹豫了一下,没有说什么,他已经亲自问过秦青,秦青毫不隐瞒当日的情景,他可以肯定有人挑拨秦青,可是他不能肯定秦青没有落入圈套,毕竟当日的事情除了江哲之外没有第二个目击者还活着,就是秦青说得是真话,也不排除他趁着有人行刺而落井下石的可能,但是事关重大,李贽不愿告诉王妃,只是淡淡道:‘本王会秉公而断的。‘

无论如何雍王府总算暂时平静下来,这些日子以来王府上下人仰马翻,如今总算风平浪静了,当然暗中的波涛汹涌就不是普通人所能了解的了。

太子府邸,鲁敬忠怔怔地看着手中的情报,突然震怒的撕碎了纸张,过了片刻,他起身走出房间,走到殿门处,却听见里面歌舞正酣,鲁敬忠摇摇头,悄无声息的走了进来,只见太子李安坐在上首,懒洋洋的看着那些舞女优美的舞姿,见鲁敬忠进来,而且神色不好,李安一挥手,这些舞女乐师都退了下去,他问道:‘少傅为何如此忧虑,不就是江哲性命保住了么,我们的目的原本就不是他的性命啊。‘

鲁敬忠忧心忡忡地道:‘殿下,这些日子臣仔细阅读情报,发觉咱们犯了大错。‘

李安一愣,坐直身躯道:‘少傅何出此言。‘

鲁敬忠道:‘殿下,从前我认为江哲虽然是国士,但是说句实话,我们不缺文武之才,所以他虽然军略国策上都有不凡之处,臣也不甚在意,只要殿下登基之后,他若愿意效命,用之不晚,反正殿下还不是皇上,用不着急于招纳才俊,引得皇上疑心,臣还暗笑雍王不知检点,就是江哲才华再高,不入中枢,又有什么作用,原本臣建议刺杀江哲,不过是想挑起雍王和秦大将军的冲突罢了。不料当日情况诡异,竟有另外一批刺客抢了先,虽然他功败垂成,还是我们安排的人得了手,之后雍王的反应殿下也看到了,臣这才想到莫非我们踩到了雍王的要害。‘

李安点点头道:‘你说得有道理,这些年来我们没少难为老二,何曾见他这样强硬,先到父皇面前哭诉,又调近卫军入城,连父皇赏赐的玄参他都不珍惜,还向长乐公主借了玄参,连老六都上杆子巴结,想到这里,孤就生气,怎么六弟这样糊涂,他还有没有把孤看在眼里。‘

鲁敬忠道:‘正是这些让臣起了疑心,仔细阅读旧日情报之后,才知道臣失职了,这个江哲,从前臣只是以为他是栋梁之才,可是臣仔细推敲,此人竟是一个善于阴谋诡划的奇才。‘

李安扬眉,示意鲁敬忠讲下去,鲁敬忠道:‘此人以文才扬名,他在南楚德亲王幕中参赞,臣初时没有觉得奇怪,只道他不过是附骥罢了,而且我们所得情报,虽然知道德亲王很信任他,但是并没有看到他献了什么计策,虽然知道他军务处理的不错,可也觉得无关紧要,这些日子,臣收集了德亲王全部战报,发觉只有在江哲在其军中的时候,德亲王的战术才变得诡异阴狠。还有江哲一曲送了蜀王性命,虽然传为美谈,但是人人都以为是德亲王的命令,江哲不过捉刀而已,但是如今想来,未必不是江哲自己的意思。只是此后江哲卧病数年,所以人人都没有留意,若非臣遍阅南楚情报,只怕也不能发现这一点。‘

李安笑道:‘少傅是否过于忧虑了,这都是没有证据的事情?‘

鲁敬忠道:‘确实没有证据,可是殿下,齐王为什么想要跟雍王争夺此人,这次又巴结地送去贵重药物,殿下不是说梁婉曾经禀告过殿下,说雍王和齐王都曾经让她注意江哲,只是咱们以为雍王赏识江哲的才华,殿下知道,雍王是爱才如癖的,至于齐王总是胡闹,所以殿下也没有放在心上,现在看来,雍王和齐王只怕都知道此人的厉害,只有太子和臣把江哲看成一个才华横溢的文人罢了,所谓一叶障目,不见泰山,雍王是用他的器重,让我们相信江哲是屈原贾宜一般的名士,却让我们忽略了此人实在是良平一流的谋士啊。‘

李安道:‘少傅细心,孤是知道的,可是也未免太过虑了,此人投靠雍王以来,并没有什么建树可言啊。‘

鲁敬忠眼中闪过一丝警惕道:‘这正是臣担忧的,所谓善战者无赫赫之功,另外若是臣猜得不错,此人用计应该是阴狠严密,不拘一格的,只怕我们会中了圈套,所以臣原本希望他死掉的,可是没想到这样的重伤还让他逃出了生天。‘

李安宽慰道:‘少傅才智过人,就是那人再有本事也不是少傅的对手啊,大不了我们再派一次杀手。‘

鲁敬忠眼中先是闪过一丝得意,却摇头道:‘这就是另外一件被我们忽略的事情,他身边有一个暗藏的高手,名叫李顺,据说是南楚宫中一个宦官,在建业城破的时候托庇于江哲,也被雍王一起带了回来,我事后查阅关于此人的情报,发觉十分稀少,因为此人几乎终日和江哲形影不离,江哲深居简出,这人也是如此,殿下知道,雍王府上下如同铁桶一般,很难渗透的,尤其是江哲身边更是侍卫众多,我们的探子根本就没有留心到此人,据夏侯所说,那个李顺武功已经到了不着皮相的境界,除非是他那种级别的高手,或者是擅长品鉴的人物,很难看出他的深浅,我们的探子这一点实在是差了一些,又没有特意留心,这才忽略了这个人,据夏侯说,此人武功必然远在他之上,我们若是再派杀手,只怕行不通了。‘

李安神色不安地道:‘少傅,那你说该怎么办。‘

鲁敬忠道:‘所谓兵来将挡,殿下也不必过于忧心,只是我们多加小心,一旦殿下登基,就不用忧虑了,若是情势紧急,大不了我们派些厉害的杀手去,李顺的武功再高,还能高过那个人么?‘

李安点点头道:‘少傅说得是。那么我们的生意要不要缓一缓?‘

鲁敬忠道:‘这倒不必,夏侯说,那个刺客十有八九是南楚派来的,我怎么也不相信一个南楚降臣会和锦绣盟有什么勾结,而且我也不信他在难处有什么势力可言,现在雍王焦头烂额,我们正好趁机做几比打得,等到雍王有所察觉,我们已经不干了,倒是殿下,臣还是劝殿下疏远夏金逸,他是和江哲起了冲突才进府的,我担心他有异心。‘

李安不耐烦地道:‘少傅,你知道的,夏金逸虽然和江哲起了冲突,却是因为关中联而起的,而且就是孤是江哲,也会像他那么做的,再说本王派人监视夏金逸,他除了和绣春卿卿我我,就是忙着排练歌舞,这次江哲重伤,雍王府一片混乱,他若是奸细,不是特别关心就应该装作漠不关心,可是你也知道,他虽然好奇却没有一丝同情,还嘲讽雍王府的人,除此之外就是把绣春弄到手了,他若是雍王府的奸细,这些日子还不忙着收集情报,再说,这种只会声色犬马的人,老二恐怕看不上的,你放心,本王不会让他知道什么机密的,这小子也不是这块料。‘

鲁敬忠皱皱眉,不再劝谏,他总不能说殿下这些日子被夏金逸引诱纵情声色,已经引起某些人的不满吧,这种事情劝也劝不来的。

李安摆手道:‘好了,少傅加强对雍王府的监视就是了,不用过虑。‘鲁敬忠只得唯唯称是。

李安这时神情一变,道:‘只是有一事我十分不安,齐王事先没有警告你我江哲之事,如今又是巴结讨好,你说齐王是不是有了异心。‘

鲁敬忠道:‘殿下,天下谁没有私心呢,臣认为齐王也只是喜欢贤才罢了,这一点私心殿下应该不用介意的。‘

李安有些不满的看了鲁敬忠一眼道:‘既然你这样说了,孤也就算了,不过你要好好留意齐王,孤可不想众叛亲离。‘

鲁敬忠神色不变地道:‘臣一定注意齐王的举动,若是殿下担心,不妨问问兰妃娘娘,她和齐王妃是同门,一定会知道一些的。‘

李安冷冷道:‘孤已经问过萧氏了,她说齐王妃告诉她,说是齐王不过是因为顾念江哲曾经治好过他的毒伤,这件事情我是知道的,所以没有放在心上,今日听了你的话,老六的话必然不尽不实,还是你替孤留意此事吧,孤绝对不允许另外一个雍王出现。‘

鲁敬忠恭恭敬敬地道:‘臣遵命。‘

在长安一处宅院之中,夏侯沅峰正站在园中,赏玩着初开的梅花,如今已是二月末了,几株早梅含苞待放,这时一个青衣小厮从后面匆匆走来,看到初春的阳光下有着如同梅花一般俊雅容貌的少主人,他神情呆了一下,然后高声道:‘公子,客人想要见您。‘

夏侯沅峰微微一笑,道:‘这就好了。‘说罢剪下一枝梅花,插到瓶中,就这样捧着瓶子向客房走去。走进客房,他将花瓶放到桌子上,淡淡的梅花香气立刻盈满了房间,他对着床上的那位中年人淡淡说道:‘毒手邪心,你的伤势已经好了么?‘

毒手邪心冷冷的看着这个当日救出自己的俊美少年,森然道:‘我的伤势已经好了,你有什么条件可以说了吧。‘

夏侯沅峰微笑道:‘有一件事情要告诉你,很不幸,江哲江随云已经逃出生天了。‘

毒手邪心心中一紧,但他冷然道:‘那么你这个黄雀也没有占到便宜啊。‘

夏侯沅峰淡淡道:‘按理说,你是南楚间谍,我应该杀了你的,可是我实在是不愿杀你,毕竟你我的目标并不冲突,如果你愿意,我可以再给你一次机会行刺,你可愿意。‘

毒手邪心冷笑道:‘你当我是白痴么,不说现在江哲身边必然防卫严密,现在顺公公也不会离开他左右。‘

夏侯沅峰眼睛一亮道:‘你对李顺很熟悉么?‘

毒手邪心看破了他的心思,淡淡道:‘不算熟悉,不过我曾经监视过江哲一段时间,知道他经常出入江哲的府邸,只是没有料到他武功如此高强罢了,这次若非他不在,我恐怕就上门送死了。‘

夏侯沅峰淡淡道:‘这些日子京城的盘查已经松懈了很多,你如果愿意,我可以送你出城。‘

毒手邪心冷然道:‘你到底有什么目的?‘

夏侯沅峰笑道:‘我的目的很简单,我要你引开雍王府的注意力,在他们追捕你的过程中,我希望你能让他们相信,和你合谋杀人的乃是秦青,当然我会尽量助你逃回南楚,若是不行,还请你自行了断。秦青是什么人你清楚,我想雍王和秦大将军发生争执,对你们南楚也是有好处的。‘

毒手邪心知是一个九死一生的任务,淡淡道:‘也好,最多我的性命还了给你就是,只是你也必须做一件事情,你必须在两年之内杀了江哲。‘

夏侯沅峰微笑,举手立誓道:‘夏侯沅峰立誓必定在两年之内杀死江哲,若违此誓,天诛地灭。‘

毒手邪心淡淡道:‘我虽然不信你的誓言,可是我相信你必须杀死江哲,否则除非你终生不让他知道射那一箭的是谁,否则,你一定会死在他手上。‘

夏侯沅峰微微一笑,当日他看见小顺子匪夷所思的轻功,便知道太子殿下得罪了一个十分可怕的敌人,所以临时起意,把抢先刺杀江哲的刺客救走了,希望他能够达成自己的目的,不过射那一箭的人不是自己呢,若非让毒手邪心这样认为,他怎会乖乖的听从自己的安排,想到那个杀手黄雀,夏侯沅峰恶意的想道,不知道那人清不清楚也有人用弓箭瞄准了他呢?世间的事情真是无巧不成书,谁会想到有三波杀手同时到了寒园呢?

\chapter{第十八章 死里逃生}

随云稍愈,桑臣辞别长安,临行王以千金相赠,先生推辞,哲劝曰,金帛非为酬功,仅略助行资,且天下贫病者众,先生善救之。

--《南朝楚史·江随云传》

死里逃生是什么感觉,这大概只有经历过的人才能体会到,所以当我睁开眼睛的时候,虽然感觉到浑身麻木,心口更是剧痛难忍,仍然忍不住露出淡淡的笑容,艰难的动动四肢,又是一阵疼痛,更加确认自己还活在世上,不是到了阴曹地府,我呻吟出声,嘶哑的声音刚刚从唇边溢出,已经有人过来挑起了床上的锦帐,我仔细看去,是一个四十多岁的御医,我勉强露出一丝微笑,他惊喜的回头叫道:‘江大人已经醒了,快去通知桑先生。‘接着耳边传来沉稳的脚步声,然后我就看到了那张熟悉的面孔。

虽然多年不见,可是桑先生的相貌没有太多的变化,须发灰白,相貌清瘦,眼神总带着那种专注和无情,是的,无情,桑先生在天下人的口中是妙手回春的神医,经常不辞辛苦为贫苦之人医病,可是他的心却是冰冷无情的人,这些我当年就知道了。病人在他眼里只是面孔模糊的男女,他医治病人的时候固然是专心致志,对于病人的病情变化、心情波动都了如指掌,可是事后病人若是痊愈,那么在他来说就是陌路之人,若是病人不幸逝世,他也断不会有一分伤心难过。或许,在桑先生眼里只有病人和健康人这种分法,对于他来说,病人只是用来验证医术的工具罢了,若说有谁例外,那么大概就是我了。

记得当年初次相见,父亲求他医病,他只看了一眼就说父亲病根入骨,就是医治也不过数年性命,父亲原本有些失望,甚至有不再医治的打算,是我对父亲说道:‘数年对于常人来说虽然短暂,但是对咱们父子却是至关重要,儿尚年幼,若无父亲照料,不免颠沛流离,父亲若是就此不起,又如何能够放心孩儿将来生计,不如父子相依为命,多捱数年,若是父亲苦痛缠身,儿自然不敢强求父亲,可是只要孩儿细心照料,父亲应该没有多大苦楚的。‘

父亲原本只是一时灰心,见我言辞恳切,便再度求医,桑先生听了我的说话,只是淡淡道:‘这孩子倒也通达。‘说罢就留下替父亲诊治,而且羁留数月,教我医术,我曾听他说过,他没有什么行医济世的志向,行医只是他的谋生手段罢了,虽然他说得如此冷漠无情,可我偏偏喜欢他这般率直,而且桑先生眼中见不得病人,不过幸好他平日沉默寡言,若是给人知道世人心目中的杏林医圣这般心思,只怕要大惊失色了。

看到桑先生,我不由流出泪来,这是我在这世上唯一的长辈了。桑先生明白我的心思,走过来替我诊脉,淡淡道:‘随云,你的伤势已经没有大碍,这些日子,你服了不少贵重的补药,虽然救了你的性命,可是不免有些元气太盛,这些日子,你先慢慢调养,等你伤势痊愈之后我再为你细细医治调养。‘

我用目光询问地看着桑先生,他微微一笑道:‘你是问雍王殿下和那个一直替你用真气续命的小子么,雍王这些日子以来太过疲倦,我已经让他回去休息,据说殿下一觉睡下,现在还没有醒呢,你那个随从确实忠心不二,不过我见他内力消耗太甚,又不肯去休息,所以用了一剂药,让他乖乖的去休息了,等到他醒来之后,正是破而后立的好时机,我会监督他好好闭关练功,你这孩子先天不足,练武不会有太大的成就,他倒是练武的奇才,虽然说他的武功确实是精妙非常,但是能够练到这个程度还是他天资过人,我那几手武功还没有传人,不如教了给他,你是我半个弟子,他对你忠心耿耿,教给他也是一样。‘

我不由大喜,桑先生的武功深浅虽然我不知道,但是从他的语气可以看出应该很不凡的。转念一想,我想起雍王多日来一定是十分劳累,否则怎会一睡不醒,不由有些焦急,连忙握住桑先生的手,在他手心里写了一个‘王‘字。桑先生微微一笑,道:‘难得殿下对你这般亲厚,我已经去看过他了,你放心吧。‘

我这才松懈下来,这时一个侍女端过一碗药来,小心的服侍我服下。服下药之后,我觉得又有了困意,便又昏昏睡去,就这样一连数日,我便是在睡眠和服药之中度过,直到七天之后,我才不用喝那种加料了的汤药,终于可以清醒的躺在床上了。

我摸摸有些僵硬的双腿,很想下床走动一下,可是却被桑先生阻止了,小顺子原想来服侍我的,谁知道却被桑先生一句‘天下武功高手多得是,你还想你家公子受这样的伤么?‘就挡住了,现在正在闭关苦练,好像桑先生的内功心法虽然和小顺子大相径庭,可是桑先生在武技上的研究可不是小顺子可以比的,所以我耳边也清净了不少,至少没有人闹着跟我请罪了。雍王和王妃带着柔蓝看过我一次,之后就被桑先生禁止来看我,说要我好好修养,免得为外面的事情烦心。从那以后,雍王除了每日派人来问候之外,就没有再过来了,听桑先生说,好像雍王正在亲自整饬王府防卫,因为从前的防卫对于江湖高手不免有些漏洞太多。静养虽然有益我的身体,可是也未免太郁闷了,就连我最爱的书本也不让我碰一下,桑先生只拿了几本清净无为的道家经书给我看,说是让我平静心情,不过倒也颇见成效,要不然怎么我心情平静了许多,经历了生死,觉得很多事情都看得淡了,就是飘香的影子也不会总在心上徘徊不去,就是想起来,也多半是那充满幸福的甜蜜,而非肝肠寸断的苦痛。

又过了五六天,桑先生终于允许我下床走动了,小顺子也已经回到我身边,在他的搀扶下,我轻轻的走了病愈之后的第一步,脚步感觉比棉花还软,在房间里走了不到一圈,我就有些气喘吁吁了,如今已经是三月中旬,园子里面几株碧桃已经开花了,春风虽然还有些寒意,但是已经不那么刺骨了,小顺子让人将园子里面的一座凉亭三面用蜀锦围住,挡住了春风,又在亭子里面铺了厚厚的波斯毯,放上软榻桌椅,我舒舒服服的坐在软榻上,披着雍王殿下去年冬天赏赐的银狐裘,桑先生坐在椅子上,双目微阖,小顺子却在一旁煮茶,不多时两杯热茶送了过来,我一饮而尽,只觉的四肢百骸都是一阵舒畅。

桑先生也是一饮而尽,微笑道:‘殿下送来的茶果然不错,随云,雍王待你如国士,看来你是不肯随我隐居的了。‘

我一愣,问道:‘先生为何这样说,莫非是不喜欢哲效力雍王么?‘

桑先生淡淡一笑道:‘这些世间俗事,我也懒得理会,雍王又不是什么昏庸之辈,你辅佐他也没有什么不好,只是为你身体着想,我倒想让你辞官归隐。‘

我淡然道:‘可是我的身体从今以后不能劳累了么?‘

桑先生摇头道:‘不只如此,随云,你伤势虽重,但是只要细心调养,数年之后就可恢复如常,这几年只要仔细一些,也没有什么大碍,只是心病难医,你的心脉被七情所伤,若是不能够平心静气,潜修养病,只怕十年之后就会病入膏肓,若我所料不差,你必然是在身心俱疲的时候经历了大喜大悲之事,因而伤了心经,这些年来又是悲伤未止,所以才养成宿疾,你虽然医术不错,只是良医难以自医,这才导致今日。‘

小顺子听得面色苍白,他一言不发的望着桑臣,桑臣微微摇头,叹息了一声,不再说话。

我方从生死关头走出,却又听到这样的消息,但是奇怪的是,我心中却没有丝毫难过,淡淡一笑道:‘这也是哲命中注定,就是哲从前有心归隐,如今受了雍王殿下这样的恩情,若不能报答,岂非终身难安,再说,若是心绪不宁,深山苦修又有什么用处,弟子不敢相瞒,我身负杀妻血仇,此恨不雪,死不瞑目,如何能够潜修。不过十年时间已经足够,弟子自信可以报仇雪恨,辅佐雍王成就大业,到时候湖海漂泊,至生死于度外,视富贵如浮云,岂不快哉,人生至此,死又何恨?‘

小顺子先是脸色苍白,听到后来却是神色转为平静。桑臣看了他一眼,问道:‘你也由得你主子胡来么?‘

小顺子恭恭敬敬地道:‘公子喜欢如此,奴才只有依着他,最多奴才相随泉下,想必不会令公子寂寞的。‘

他这般说法,我却也不感动,经历生死之后,我许多想法都有了不同,小顺子就是为我殉死,我也只觉得多了一个泉下友伴,而且凭我的手段,让他活下去又有什么困难的,所以只是淡淡的看了他一眼,微微一笑,表示知道他的苦心,他这么说不过是为了让我努力多活几年罢了。

桑臣微微苦笑道:‘罢了,这也由你,不过我这段时间会替你好好调养一下,以后就要看你自己的了。‘

我疑惑地问道:‘怎么先生还要远游么,先生年纪这样大了,又何必还要四海为家呢?‘

桑臣淡淡道:‘我年纪大了,不愿介入俗世的纷争,这里波涛汹涌,我可不比你们年轻人,禁不起风浪了。不过我年纪确实不轻了,这次我准备回故乡隐居,你如果日后有事情,可以到东海蓬莱寻我。‘

我点点头道:‘先生说得是,若是局势平稳下来之后,我也想去看看海外仙山的风光。‘

桑先生犹豫了片刻,又道:‘随云,你的仇人可是凤仪门么?‘

我身子一震,淡淡道:‘先生怎么会这样说?凤仪门执武林牛耳,乃是白道精神领袖,又是大雍元勋,我怎会和她们为敌呢?‘

桑臣淡淡道:‘你不用担心,我和凤仪门没有什么关系,凤仪门主成名之时,我已经是不惑之年,虽然她几次想请我作客卿,我没有答应,这次她们找上门来求医,我看了一看,就知道那个梁婉是中了断恩草配制的毒药,断恩草无药可救,而这世上有这种药物的,只有你我二人,所以我知道必然是你所为,可是你从来不会作出没有道理的事情,所以我并没有告诉她们,只说好好照顾,梁婉还可以恢复如常,只是过去的记忆是回不来了。‘

我有些放下心来,问道:‘先生不会怪我如此辣手么?‘

桑臣淡淡一笑道:‘我从不过问世间俗事,只是这断恩草未免太毒辣,以后不要用了。‘

我又问道:‘先生如何看待凤仪门呢?‘

桑臣深深的看了我一眼,道:‘凤仪门主是个可怜人,可恨之人必有可怜之处,你要做什么无可厚非,但不可伤了自己的身体。若有仇恨,你只要记着仇人是谁,至于那段仇恨还是忘记的好,焚心销骨,不记得才是对自己的善待。‘

我宽心的轻施一礼道:‘多谢先生教诲。‘这个世间唯一能够让我屈从的人已经摆明了不会过问我的事情,那么我就真的没有什么顾忌了。虽然不知道桑先生和凤仪门主有什么样的过往,但是那已经是无关紧要的事情了。

桑臣叹了一口气,这个孩子自己一见便觉得性情相投,虽然年龄如同祖孙,自己也真的将他看作孙儿,可是他也知道自己是不能扭转他的心意的,凤仪门主梵惠瑶曾与他数次把酒相谈,那个女子,是光芒万丈的存在,虽然自己已经习惯独自一人的生活,但是也曾经对她动过心,还将自己收藏的太阴心经的残本送了给她,若没有自己给她的那一份,那么相信她不会有今日的成就,至少也会慢上十年,他从来没有后悔过,因为武功对他来说并不重要,可是为了这个迟早会和凤仪门主对上的孩子,他将所有武功都传了给李顺,想必这样李顺就能够更好的保护江哲吧。

看了看江哲,桑臣淡淡道:‘我这就要走了,你好好保重。‘

我连忙道:‘今日太仓卒了,还是多留几日,我也好送先生一程。‘

桑臣微微一笑道:‘不用了,你身子不好,送我做什么。‘

这时远处传来轻快的声音道:‘怎么,谁要走了。‘我抬头看去,却是雍王李贽带着司马雄走了过来。便说道:‘殿下,桑先生这就要走了。‘

李贽忙道:‘先生怎可如此匆忙,上次救了本王一命还没有来得及报答,这次又救了江司马性命,若不多留几日,只怕都要说本王招待不周。‘

桑臣淡淡道:‘多留无益,随云已经没有大碍,老朽尚有事情待办,所以只得告辞了。‘

李贽见桑臣言辞坚决,知道不可勉强,便令人取来价值千金的金珠,道:‘本王不敢强留,请先生收下这些金珠,不敢言谢,只是相助盘缠罢了。‘

桑臣淡淡道:‘随云是我故旧,若非殿下不惜名贵药物,只怕早已丧命,桑某感激不尽,怎敢还收金银。‘

这下雍王可不答应,连连恳求桑先生收下,我知道桑先生的脾气,不愿他们弄僵了,便劝道:‘先生,这些金银还是收着吧,若是传出去说是雍王殿下连路费都不给,只怕也不好听,而且先生常常为贫病之人医治,他们无钱买药,也常常害先生解囊相助,殿下这些金银,先生就当是替他们收下的吧。‘

我这些话正中要害,桑先生虽然心冷如冰,但是见到患病之人却是总要医治的,当然免不了自己掏腰包,所以总是囊空如洗,幸好他救过的人成千上万,到处都有人接待他,不过那些人恐怕不知道桑先生根本记不得他们是谁吧。

桑臣觉得江哲说得也有道理,便收下金银,告辞而去。雍王亲自相送,只有我被禁止跟去,只能眼睁睁看着桑先生走出寒园,唉,人世上我只有这一个长辈了,相聚不过几日就要分离,离愁别绪不免涌上心头。黯然销魂者,唯别而已。

小顺子代我相送之后很快就回来了,他神色有些犹豫地道:‘公子,要不要查查桑先生和凤仪门主的交往,这些事情好像没有见过情报。‘

我淡淡道:‘不必了,桑先生得为人我很清楚,他既然说了不管就是不管。我们若是杀了凤仪门主,桑先生不会见怪的,只要我们不要对凤仪门主使用过分的手段就行了,再说凤仪门主是什么人物,就是我们毁了整个凤仪门,也未必能够伤害到她。‘

小顺子神色一变,问道:‘若是凤仪门主将来逃走,那么岂不是后患无穷,总要有法子困住她才行。‘

我看看小顺子,微微一笑道:‘这个倒是有法子,不过得看你的了,如果你能够接下凤仪门主百招而不败,那么我就有胜算了。‘

小顺子神色坚毅地道:‘公子放心,我会办到的。‘我微笑点头。又道:‘殿下怎么还没过来,他今日过来一定是有事情和我商议。‘

小顺子脸色顿时变得十分古怪,半晌才道道:‘刚才殿下收到一些消息,所以没有过来,想必一会儿就过来的。‘他的话音刚落,我已经看到了李贽的身影,而我却没有来得及追问小顺子为何神情不安。

\chapter{第十九章 亭中秘议}

李贽走了过来,神色间有些不安,我知道他是觉得这么快就来打扰我未免有些苛刻,毕竟我现在重伤初愈,还不便处理事务,不过想一想我重伤昏迷一个半月,这些天又被桑先生禁制会客,雍王殿下一定有很多事情要和我商讨。所以我在小顺子的搀扶下站起身来,朗声道:‘殿下请到亭中来吧,请恕哲不便远迎。‘

李贽连忙上前道:‘随云不要出来,外面风大。‘说着几步走进亭子,笑道:‘这法子不错,又可以品茶赏花,又不用担心吹到寒风。你快坐下吧。‘

刚才的动作已经让我出了一头的冷汗了,也不客气,我坐了下来,淡淡道:‘殿下,请先喝杯茶再说吧。‘

李贽欲言又止,接过了小顺子递过来的香茶,喝过之后,才道:‘随云,虽然有些难为你,可是你说本王如今该怎么办呢?‘

我微微一笑道:‘殿下做的很好啊,趁机充实了宿卫,得到了陛下的同情,秦青、夏侯沅峰、靖江王郡主都有嫌疑,这段日子只怕会给殿下不少方便,秦大将军恐怕也会因此偏向殿下几分,毕竟秦青嫌疑最重,殿下还有什么不满意的呢?‘

李贽赧然道:‘本王也是情急,可是如今该如何收场才好,还有,不知道刺客是谁,随云可有想法?‘

我淡淡道:‘射我一箭的人我虽然看到了,可是距离太远,不知道他的身份,不过殿下,不管是谁,不可因为随云乱了方寸,殿下还是找个机会和秦大将军和好才是。‘

李贽皱眉道:‘你说得不错,可是若是秦青所为,本王岂可容他活在世上,不查个水落石出,本王绝不善罢甘休。随云,你先前曾说有些布置,不知道现在情况如何。‘

我看了一眼小顺子,这些日子桑臣在我身边的时候,小顺子偷偷出去见过陈稹一次,昨日已经跟我报告过,这短短不到两个月的时间,太子已经和锦绣盟做了两次交易,我们也赚得不少,当然也没有忘记把太子如何走私,调用了哪些亲信一一记录下来,这些昨日已经到了我的手里,不过目前我还没有动用这步棋的意思,总要等到太子积重难返的时候才动手的。

想了一想,我淡淡道:‘这些事情有碍王爷清誉,殿下还是当作不知道的好。‘

李贽一愣,犹豫地道:‘随云,不是本王矫情,你行事可不要太过分,若是有害社稷黎民,本王宁可不要这个储位。‘

我微微一笑道:‘殿下放心,臣不会去做那些事情的。‘有一句话我可没有说出来,若是太子要去做,关我什么事情呢?可没有人会要我替太子承担他的罪名吧,我最多不过是教唆,可是那滔天罪行,可是太子自己犯下的。

李贽释然道:‘那就好,随云,这是这些日子以来的情报,你慢慢看一下。‘

我拿起雍王递过来的节略情报,仔细研究了起来。

这些日子太子方面偃旗息鼓,毕竟他是雍王的死敌,这时候就是落井下石也不能太明目张胆的,毕竟还有皇上在,这次皇上也很猜疑太子,所以不免有些警告,太子因而隐忍,整日躲在府邸里面声色犬马,不亦乐乎,而替太子操办这些的就是太子新近宠爱的侍卫夏金逸,夏金逸虽然放荡,但是这些方面可真的是天才,将府内的歌女舞姬调教的十分出色,唱得是销魂曲,舞的是天魔舞,把个太子迷得神魂颠倒,每日里都在脂粉陷阱里面沉醉,若非内有侧妃萧兰,外有少傅鲁敬忠,只怕太子府上下已经一片混乱了。看到这里,我不由微微一笑,见微知着,太子还没有登基就开始享乐,看来我的目的已经达到了,不过太子这个样子也是一种迷惑吧,陈稹送来的情报早就说过这些日子太子忙着走私军械呢,连着两批军械让太子食骸知味,又忙着走私第三批,锦绣盟也十分满足,不过好事不长久,虽然利益动人心,可是只要我们动动手脚,就能让他们内讧起来。走私的事情太子想必都交给了鲁敬忠他们去办,这样想来,太子的心机也不寻常,就是一旦犯了事情,哪怕追究到鲁敬忠身上,也不过是太子御下不严,甚至可以上表请罪,说自己耽于女色,被下属蒙骗,不过这样的心机再深也不过是个枭雄罢了,想要为万民之君主,焉能推诿罪责。

那几个涉嫌刺杀我的人也都有相关的情报,秦青被秦大将军关在府中不许外出,据说秦大将军动了家法,秦青被打得下不了床,夏侯沅峰一切如常,这一点有些不正常,他的住处很严密,雍王又不便没有证据就去惊动这个皇上最宠爱的侍卫,所以不知道他暗里作些什么,靖江王郡主李寒幽是奉父命进京谒见皇上,靖江王没有子嗣,派女儿上京祝贺也是允许的,不过李寒幽凤仪门弟子的身份未免有些碍眼,而且靖江郡主这些日子总是在皇后身边,据说是因为靖江王有意为女儿选择佳婿,所以托付皇后作主。据说,皇后有意把李寒幽许配给秦青。看到这里我心中一寒,若是让凤仪门勾搭上了秦彝,这可是太可怕了,看来不论这次是否秦青行刺于我,都不能处置他了。仔细的回想那日的刺客,回忆他一切形迹,我豁然开朗,不管这人是谁,这人绝非秦青,想起当日的情形,我终于了然,有人挑拨离间,然后行刺于我,嫁祸秦青,可巧被毒手邪心抢了先,挑拨离间的人不用想也知道是谁,终于锁定了对手,我看向鲁敬忠的名字,这个人真是可怕啊,恐怕他们这次的目的只是让雍王和秦彝结仇,选中我大概是因为我是雍王新收的亲信,我这次算不算是无辜受难呢,不过现在雍王这样情急,看来我就是想韬光养晦也不成了,这人心机真的十分厉害,不过这人缺点也很明显,若是我用计策,绝不会这样赤裸裸的挑拨,要让对手自动跳进圈套才好。现在秦彝和秦青只怕更恨那挑拨离间之人,不过说也奇怪,虽然是为了挑拨秦青,那个我和公主有私情的谣言也太虚假了。

这些情报并非特别详细,想必原始的情报资料都在雍王的书房里面,毕竟雍王不愿我多费脑筋,大部分都只有几行字而已,我也不去多想,那些情报都是关于我遇刺之事的调查,已经发生过的事情,我并不急于知道,目前我只要知道现在的局势就可以了。

放下情报,我淡淡道:‘殿下,这件事臣的心里已经有数,请殿下不用挂心,这几日殿下不妨多到宫中走动,殿下放松一些,臣才有可乘之机,刺杀臣的人不是为了私仇,所以臣也不会报复,只要殿下取胜,臣的仇自然也报了。‘

李贽喜道:‘随云已经明白了么,那么是谁刺杀你的,本王绝不轻饶。‘

我淡淡一笑道:‘殿下,臣没有证据,现在也不好说是谁,可是总不会放过他们就是,殿下暂时不要松口,拖延大将军一段时间,若是此时解决,恐怕大将军也不能阻拦靖江王郡主和秦青将军的婚事,若是秦青有嫌疑在身,那么大将军就可以婉言谢绝了。‘

李贽放下心中大石,喜道:‘不是秦青就好。‘

我连忙道:‘殿下不可声张,甚至可以暂时表示对大将军的敌视,这样才能令敌人疏于防范,大将军不会因此怨怪殿下的。‘

李贽点点头道:‘正事谈完了,还是两件事情本王得跟你说一声,一件就是这次父皇允许我充实宿卫,我已经让司马雄到军中精挑细选,本王从前总是想勇士应该在沙场扬威,所以府中近卫没有特意选拔,以至这次害得随云你受了重伤,这次本王命令在军中举行大比,选出千人作为王府护卫,本王已将千名近卫中的八百人分为乾坤坎离震艮巽兑八卫,负责王府防务,从今之后,雍王府以军法治下,如有懈怠职守者斩立决,另外二百人是其中佼佼者,我已经安排了五十人为随云你的亲卫,他们很多都是内功不错的江湖子弟,绝对可以挡住一流高手的刺杀,这些人的赏罚由你亲自作主,不必向本王请示。‘

我心中一暖,雍军中的大比十分隆重,分为三场比赛,骑射、骑战、步战,必须要三场全胜,才能成为胜利者,这种大比,若要取得军中第一勇士的称号,是要血战无数次才能成功的,雍王使用大比挑选护卫,也就是说中选者都是千里挑一的勇士,而且拨给我的护卫竟然大半都是江湖出身,看来雍王这次是下了血本了。

我连忙向雍王道谢,这是小顺子脸色变了又变,终于忍不住上前下拜道:‘奴才方才出言不逊,得罪殿下,还请殿下恕罪。‘

我心中一惊,虽然早就看他不对头,不料他竟然触犯了殿下,连忙道:‘小顺子,怎么回事,你如何得罪了殿下。‘

小顺子赧然道:‘方才殿下得到回报,说刺杀公子的毒手邪心终于露了踪迹,而且已经几次杀出重围,我很想去杀了那人替公子报仇,但是又担心公子的安危,忍不住讥讽殿下说,‘若非裴将军恰好在寒园做客,只怕公子早就性命不保,毒手邪心一个小小的刺客,在雍王府来去自如,现在又在外面嚣张,真是令大雍勇士颜面扫地。‘‘

我听得出了一身冷汗,连忙起身道:‘小顺子无知,冒犯殿下,还请殿下恕罪。‘

李贽挥手让我坐下,苦笑道:‘是本王对不住你们主仆,这次本王精选的侍卫绝对可以保证随云的安全,而且,本王已经下了令旨,从今之后,每三个月就要进行一次大比,排名最后的十人要和新选进来的侍卫比武,若是不能完胜,就得斥退。随云你的亲卫虽然赏罚由你决定,可是人选更换不能由你决定,小顺子,你武功如此高强,远远超出本王的估计,以后随云的亲卫中,你若觉得谁不能胜任,就可以将他斥退,不要依着随云的性子,他这人有时心慈手软,就是觉得不称职也不愿说出来。‘

小顺子连连点头,他前几日曾经建议我把秘营的人手招一些进来,但是秘营的长处在于隐秘,若是真刀真枪,恐怕还不是这些军中高手的对手呢,而且秘营不适合曝光的,所以我没有答应,此刻见我安全有了保证,小顺子喜笑颜开地道:‘奴才斗胆,请让奴才亲自去选拔公子的亲卫。‘

李贽点头道:‘好吧,这些人将来大半是要由你来调动的,你去选拔也好,司马雄这几日正在分配,你不妨去看看吧。‘

小顺子连忙点头,他举目向我请示,我知道他忙着确保我的安全之后就要去追杀毒手邪心,也就不拦着他,反而说道:‘你快些去办这件事,这么长时间毒手邪心踪影全无,如今我的命保住了,他倒出现了,我也想你去好好问问他。‘

小顺子连忙点头称是,匆匆走了下去,看来他已经急不可待了。

看着小顺子的背影,李贽感叹道:‘好个忠心耿耿的奴才,随云,你真是好福气。‘

我笑道:‘这是在殿下面前,他给殿下面子才这样听话,平日没大没小的时候多着呢?‘

小顺子的身影消失之后,李贽端容道:‘随云,我知道你有些事情瞒着我,我不愿追查,是信你不会害我,但我若是不问,却是不能和你推心置腹。‘

我虽然知道雍王不会对我的事情一无所知,但是事到临头,还是有些不安,心道,他应该不是要和我算帐吧。心里惴惴不安地偷眼看去,李贽已经接着说道:‘本王这次说穿这件事情,不是为了别的,你若有信任的心腹,不妨让他们跟在你身边,若是再有今次这样的事情发生,只怕随云就没有这样的好运气了,我知道你从前身边似乎有几个侍从武功不错,可是在你身边都没有见到,我不怪你对我隐瞒,你若不是处处小心,怎能在乱世自保,只是你的安危要紧,你也不要为了瞒着我,让他们不能在你身边保护你。‘

我有些惭愧的低下头,若是雍王不说此事,我们君臣之间自然相安无事,可是我没有料到雍王竟会拼着让我心生隔阂,也要我更好的保护自己,心中的感激让我几乎落下泪来,想到德亲王至死不忘对我的猜忌,虽然仍然敬佩他的忠义,却也不由阵阵心寒,雍王这般待我,我若是不能让他登上九五之位,如何能够安心,最多我多出些力气,也不要惧怕功高震主,等我事成之后,远遁江湖就是。

不过虽然心照不宣,我可没有多说什么,只是道:‘殿下训示,臣谨尊就是。‘

见我领会了他的意思,李贽欣然道:‘这次你受伤惨重,长乐公主和齐王都对你十分关爱,皇妹将父皇赏赐给她的玄参和熊胆都送了给你,齐王也送了熊胆给你,若没有这些药物,只怕本王也保不住你的性命,对了柔蓝前两天从宫里回来,长乐很喜欢她,还要她常去做伴呢?‘

我神情有些迷惑,问道:‘齐王的心思,臣倒知道一二,长乐公主为何对臣如此厚爱,怎么柔蓝又去了宫里。‘

李贽瞟了我一眼,道:‘皇妹是喜欢你的诗文呢,柔蓝么,是王妃带她进宫的,你这次受伤,皇妹赐了药给你,你既然保住了性命就该谢恩的,王妃见你伤重,索性带着柔蓝去谢恩,柔蓝是你的义女,替你谢恩也是符合礼数的。‘

我迷茫的点点头,为什么公主会这样厚待我呢,我的诗词真的那么好么?

李贽看了我一眼,笑了一笑,又道:‘对了,长乐还有话带给你,说是‘谢谢你‘,也不知道是什么意思。‘

我心中一凛,谢谢我,莫非她知道了当日我放过她的事情,不可能,若是那样,梁婉的事情不就人尽皆知了么。安慰了一下自己,我道:‘臣也不大明白公主殿下的深意。‘

李贽见我神色有些疲倦,道:‘好了,本王也不耽误你的休息,不要累着了。‘我是真的有些累了,便目送李贽离去,两个侍女过来搀扶,她们都是这些日子服侍我的,所以虽然我不喜欢侍女服侍,也没有赶她们走。躺倒床上,我渐渐沉入梦乡,不过睡得不太安稳,那刺客流光电影的一箭让我至今心中惴惴不安,总是不能安睡,我作了一个梦,小顺子把毒手邪心捉到我面前,让我亲手杀了,然后那日那个刺客突然出现了,还是那样一双清澈明晰的眼睛,还是那双白皙的素手,张弓搭箭,然后我就惊醒了,在黑暗中,我淡淡道:‘我知道你是谁了,没有可能是别人,李寒幽,哼。‘

\chapter{第二十章 千里追杀}

随云伤未愈,齐王亲往探之,其时太子、雍王泾渭分明,时人多不解,后乃知齐王善自保也。

--《南朝楚史·江随云传》

第二天早上,我正在半梦半醒当中,隐隐约约感觉有人替我盖好被子,动作很是生疏,我猛然惊醒,说句实话,这次九死一生之后,我对身边的事情不像从前那样无所谓了,眼睛略微睁开一点,然后我就看到齐王李显神色怔忡地坐在我身边,小顺子则虎视眈眈地望着他。我心中一动,听雍王说,在我在生死关头挣扎的时候,齐王知道我需要熊胆续命,不顾嫌疑送了自己手头的一服熊胆过来,长乐公主送我玄参和熊胆已经是我意料之外的事情,齐王如此更是令我吃惊,这将会触怒太子的,他为何如此做呢。可是我没有睁开眼睛,我能说什么呢,我早就做了选择,就是不跟随雍王,难道我还会跟从齐王么,既然是注定不会有君臣之分,那么何必还要多惹一丝牵挂。

齐王叹了一口气,转身走了出去,在门口站住,小顺子关上门跟了出去,我竖起耳朵,听见齐王淡淡道:‘我知道你不是普通人,记得告诉你的主子,他既然是淡薄名利的人,又何必在这里搅和,凤仪门岂是好惹的,就是他们不出手,太子身边难道没有高手么,随云一个文弱书生,这次侥幸保住了一条命,下次呢,劝劝他,不要再留在长安了。‘

我听见小顺子冷淡的声音道:‘王爷教训的是,奴才自会转告公子。‘

过了一会儿,小顺子推门走了进来,神色间满是冰冷的杀气,我奇怪的问道道:‘怎么了?你很讨厌齐王么?‘

小顺子怒道:‘谁要他来猫哭耗子,难道他以为我们还不知道他扮了什么角色么?‘

我挑挑眉表示疑惑,小顺子冷静下来道:‘这些日子以来雍王殿下和我都忙着公子的事情,王府中的盘查由司马将军负责,司马将军查到刺客丢下的弓箭乃是军中使用,但是想要带弓箭进来谈何容易,十六日殿下宴客的时候对客人的盘查是很严密的,当日绝对没有人能够带入弓箭来,如果是府中有内奸,那么弓箭就可能是事先藏好的,可是经过司马将军调查,也没有发现任何端倪,后来我们才想起十五日齐王妃曾经来拜访王妃,齐王妃的车驾我们肯定不能仔细盘查,所以司马将军怀疑那些弓箭是齐王妃带了进来,然后供给刺客使用的。‘

我淡淡道:‘这件事情是没有办法确证的,唯今之计,只有重新规划王府防卫才是,从前殿下虽然屡遭凶险,可是那时候凤仪门还没有正式支持太子殿下,所以雍王府的防卫还是可以的,如今对上凤仪门这种级别的杀手自然是有些不足了。‘

小顺子冷冷道:‘公子,已经可以确定刺杀公子的就是凤仪门了么?‘

我看着他眼中的火光,只怕我说了‘是‘之后,他就要出去杀人了,可是我只能摇头道:‘我只说杀手的水准应该不比凤仪门差,可是没有说是凤仪门的人做的,那天我看到了刺客一眼,若是再见到应该可以认出来,只是我可以肯定不是秦青做的就是了。‘

小顺子皱眉道:‘除此之外只有夏侯沅峰和李寒幽了,裴将军在公子身边,魏国公如此身份,总不至于是他吧。‘

我淡淡道:‘夏侯沅峰说自己是出去方便,李寒幽则说不喜欢前面喧闹,所以祝贺完毕就到后面去见王妃,这两个人都有可能,可是我们也不能排除还有其他人混入的可能,我们都知道有本事直接到寒园行刺的,一定是当日的客人或者是王府中的内奸,可是这不能作为证据,所以虽然他们两人嫌疑重大,王爷却不能将他们拘捕。‘我总不能说是李寒幽啊,毕竟我连她的面都没见过,没有证据的猜测还是不说的好。

小顺子冷冷道:‘王爷不能做的事情,我可以,只要公子允许,我这就去杀了他们。‘

我笑道:‘胡闹,我们岂能不讲道理,若是他们做的,日后还是会和我们为难,你还怕没有机会对付他们么,好了,还是去追杀毒手邪心吧,无论如何,总不能白白放走了他,留下后患。‘

小顺子淡淡道:‘公子放心,我已经将寒园的防务重新安排了,公子从前不喜欢他们离得太近,这次可不能由着公子的心意了。‘

我尴尬地道:‘这个,我不赶他们就是了。‘

小顺子见我如此,才道:‘等到我回来,随便你怎么样,我若不在,公子身边不可无人伺候。‘

我连连点头,这次我遇刺,小顺子十分愧疚,总觉得没有保护好我,但他不是自怨自艾的人,所以从今之后他是绝对不会任由我胡来了,我虽然喜欢自由自在,可是想想还是性命要紧,从前他们还不知道我的重要性,我已经几乎丧命,今后恐怕我的身边更是步步危机,哪里还敢随意而为呢,反正只是身边多了一些护卫罢了,我只当看不到他们就是了。

小顺子匆匆忙忙得走了,我知道他要去追杀毒手邪心,据说是因为毒手邪心又逃过了几次追杀,再不赶去,只怕就要让他逃回南楚了,而若不能亲手杀了那日参与刺杀的刺客,小顺子是无论如何也不能原谅自己的。

我舒舒服服的躺了下去,现在我最重要的任务就是养好身体,想起桑先生的警告,我还不想只活十年呢,荒废许久的养生气功也要重新练起,人生如此丰富多彩,我若早早就死去,岂非可惜,这次死里逃生,我倒觉得很多事情都看得淡了,就连想起飘香,心中也不再苦痛,反而只记得她的美好和曾经有过的快乐了。

离开雍王府,齐王李显神色漠然的返回自己的府邸,刚刚走到只有他可以进去的金谷园门边,就看到秦铮带着几个侍女等在那里,金谷园是李显自己的居处,若无许可,任何人不得擅入,就是王妃秦铮也不能进入,所以她等在门口。

看到秦铮,李显露出讽刺的笑容道:‘哎呀,王妃身怀六甲,怎么站在这里,本王可是担待不起,不知道王妃有什么训诫。‘

秦铮身子晃了一晃,道:‘殿下,妾身不明白殿下为何这般对待妾身的一片好心,您和太子是一条船上的人,可是您前些日子又是送药又是打探,已经让太子不快,如今又前去探望,岂不是不把太子放在眼里,妾身都是为了殿下着想,殿下为何--‘

‘住口。‘李显神色变得冷酷绝情,他冷冷道:‘王妃,你作了什么事情,还要我说么,凤仪门怎么突然想起了刺杀江哲,那弓箭是怎么带进去的,你当本王是傻子么,江哲对本王曾有救命之恩,虽然本王没有那个福分,可以让他为我所用,可是谁让你越俎代庖,请出师门来杀他的。‘

秦铮神色慌乱,李显虽然从前喜怒无常,可是从没有像今日这样暴怒的,她不由辩解道:‘不是妾身的意思,妾身只是说了殿下和雍王都看重江哲,命令是内堂传下来的,妾身也是奉命行事。‘这一说完,秦铮脸色变得苍白,她才发现,就在刚才,自己承认了自己监视李显的事实,而且还承认了自己参与刺杀天策帅府司马的事实。

李显冷冷的看了秦铮一眼,淡淡道:‘若非你是我的妻子,我何必替你苦心补救,铮儿,你真愚蠢,不知道什么人才对你真好,罢了,你去吧,好好休息,这段时间你不要出去乱走,在家里好好养胎吧。‘

说罢李显转身走进金谷园,看着他冷傲的背影,秦铮想要跟上,但是那扇黑漆的大门关上了,秦铮只觉得自己的心越来越冷,不知怎么只觉得头晕眼花,软软地倒在侍女的怀中。

月沉沉,星隐隐,站在江岸之上,一身黑袍的毒手邪心看着对岸迷蒙月色中的苇丛,心中一阵喜悦,这里是位于蕲州府永宁县和田家镇的广济县城郊外,这个小渡口虽然默默无闻,却是他最熟悉的地方,这里地处三县偏僻边际,滚滚长江流过这里的两岸,山峦叠障,草木葱茏,江面又很狭窄,是南楚和大雍的秘谍最常使用的渡口,他望着江岸绝壁上,四个‘楚江锁钥‘的大字,心中却没有丝毫松懈,这里虽然只为夜行人所知是偷渡最佳的地点,可是现在追捕自己的不是军中秘谍就是大雍武林高手,这里必然是他们设伏之处,虽然只要渡过那一衣带水的长江,自己就可以平安,可是这谈何容易。

一路行来,他处处如履薄冰,雍王的令谕传遍大雍境内各处关卡,他虽然化装潜形,仍然数次露了形迹,幸好他的武功过人,潜踪匿形又是他的长处,才侥幸逃脱,最可恨的是,大雍境内那些不受官府约束的江湖中人也将目标对准了他,一个原因是因为雍王在他们心目中崇高的地位,另一个原因却是令他啼笑皆非,大雍第一青年高手夏侯沅峰之所以被认可,就是因为他在演武中胜了禁军统领裴云,但是很多人都认为如果继续打下去,两人还是胜负未定,所以那些江湖中的青年高手在知道他和裴云‘两败俱伤‘之后,都认为如果胜了他,就有资格挑战夏侯沅峰,所以这些门路极多的青年高手开始纠缠上来,在他冲破几次围追堵截之后,这下年轻人觉得丢了面子,竟然传出若是毒手邪心活着回到南楚,那么大雍江湖高手将颜面扫地的传言,这样一来,他才真是四面楚歌,虽然侥幸到了这里,但是恐怕前面就有人在等待自己吧。他微微一笑,将身上衣服整束好,昂然向江边走去。

离江边不到百步,只听弓弦铮然,一支银箭如同闪电一般掠过他的面颊,消失在夜色当中,毒手邪心立足站住,缓缓回头,只见夜色之下,一个白衣青年拿着银弓得意的看着自己,而在他身边一个红衣女子嫣然巧笑。毒手邪心神色一变,淡淡道:‘好,原来是银弓浪子端木秋,火罗刹乔焰儿,当日若非你们行刺,亲王怎会受伤,今日若能杀了你们,也不枉在下大雍一行。‘

这时身后有人轻笑道:‘哎呀,端木,你们的丰功伟绩还有人记着呢,可惜啊,当日你们若是得手就更好了。‘

随着那阵清朗的笑声,从江心飘过一叶小舟,上面站着一个相貌清瘦的青年道士,他的双臂比常人略长,配合他清奇的相貌,显得有些仙风道骨,但一双眼睛灵动活泼,可见是个性情开朗之人。

端木秋听了他的话,不由微微苦笑,道:‘苦竹子,你总是狗嘴吐不出象牙来。‘

这时从左右两方各自出来了几个人,左面是两个秀丽如仙的少女,一个雅丽如仙,一个却是天真烂漫,右面则是三个相貌一模一样的青年,那两个少女都是腰间配着长剑,那三个青年却是一个使刀,一个拿枪,还有一个手上缠着一条软鞭。

毒手邪心微微一笑道:‘果然也只有你们追了上来,说来也奇怪,雍王府的人不着急,太子、齐王和凤仪门倒是着急得很。‘

这一句话仿佛利剑一般穿透人心,除了那小舟之上的青年,在场众人都是面上变色,毒手邪心随手摘下腰间革囊,慢慢的喝了一口囊中美酒,道:‘大雍什么都好,只是这里的酒可不如南楚的美酒,唉,也只有忍忍了,过江之后,再去酒肆品尝美酒吧。今日,你们是谁先来,是齐王麾下的银弓罗刹,还是太子府上的中洲三义,还是凤仪门的三姑娘、七姑娘,难不成你们的事情还要浪里游龙苦竹道长先出手么。‘

那三个青年上前一步,使剑的青年冷冷道:‘毒手邪心,你在雍王府横行也就罢了,偏偏牵累太子被别人怀疑,殿下有令,要把你送到雍王府待罪,你若束手就擒也就罢了,若是不然,可别怪我们手下无情。‘

毒手邪心淡淡一笑道:‘也好,就让我先领教三位的联手吧,听说三位一母同胞,心意相通,前日交手匆忙,也没有来得及领教,今日有暇,三位请。‘

三个青年举步上前,虽然同时踏步,步距却稍有不同,那使剑的青年一马当先,另外两人却是在他身后半步,三人这般走来,那参差而又和谐的韵律让人心中无端郁闷,走了数步,三人身影一晃,已经将毒手邪心围在当中,剑影刀光,配合蛟龙一般的长鞭,将毒手邪心围在当中,这三人心意相通,联手起来天衣无缝,毒手邪心早就领教过他们的厉害,为了对付这三人,他早就想好了法子,就在这三人的阵势将合未合的时候,他的已全力扑向使鞭的年轻人,他知道这三人中以这个使鞭的小子总是心智过人,聪明之人往往容易怯懦,果然,那个青年不由自主的后退了半步,若是阵势已成,他两位兄弟自会趁机夹攻,可是如今那两人却是差了一线,就这一线之差,已是生死之别,当毒手邪心的身影掠过那使鞭的青年的时候,他的咽喉血光一闪,只是毒手邪心也不好过,背上被一刀一剑交叉划过,血光崩现,两人却没有追赶,望着自己亲兄弟的尸体,他们突然呆住了。这时两个女子拦住了毒手邪心,却正是火罗刹和凤仪门的七姑娘,火罗刹虽是女子,却是英姿飒爽,那个七姑娘却是一脸稚气,相貌甜美,而两位姑娘都是手段狠辣之辈,两柄长剑一柄狠厉无情,另一柄虽然华丽,但是剑法中杀气只有更盛,毒手邪心只接了几招就已经有些汗流浃背了,这时端木秋张弓搭箭,一道银光射向毒手邪心,他虽然侧身闪过,但是乔焰儿趁机几剑迫得他更加陷入险境,而那位七姑娘剑法越发绚丽,两位姑娘都是越战越勇,这时,中洲三义剩下的两位已经从后面扑来,四人将毒手邪心围在当中,毒手邪心处境越发艰难。

一旁按剑不语的素雅仙子,凤仪门的三姑娘神色却是有些不安,她高声道:‘七妹小心,这人诡计多端,防他使诈。‘

四个青年男女同时警惕,他们可是知道毒手邪心不仅武功高强,而且十分诡诈,否则怎能逃到这里,可就在这时,毒手邪心突然一声大笑道:‘迟了,哈哈。‘众人只觉得头晕目眩,竟然都软倒在地上。

三姑娘心中奇怪自己什么时候着了暗算,突然感觉到江风徐徐,再看向苦竹子,她叹息道:‘想不到名闻大江南北的苦竹子竟是南楚密谍,真是出乎意料。‘

苦竹子微微一笑,收起手中的一个银筒,双臂用力,几下子停到岸边,他微笑道:‘天机阁的‘暗香浮动‘果然是好东西,迷香无色无味,在风中扩散,却是药力不减,可惜这一筒迷香只能用上一次,千两黄金真是太昂贵了。‘

毒手邪心笑道:‘若非我求亲王殿下买了一筒,你想用还没有呢,你们可知苦竹子老弟乃是我南楚名门之后,虽然自幼出家,可是仍然念念不忘终于南楚,胜过那些见利忘义之辈,尔等受难,也不要怪他。‘说罢走到中州三义的两兄弟面前,捡起长剑,比划了一下,就要刺下。

这时风中传来一个清冷阴柔的声音道:‘你若杀了他们,岂不可惜了接下来这场好戏没人观赏。‘

毒手邪心心中一凛,抬头望去,只见不远处站着一个青衣少年,不过弱冠年纪,相貌清秀,只是却带着几分阴柔,在昏暗的月光下负手而立,神色皎然如冰雪。

\chapter{第二十一章 江边血战}

南楚同泰元年三月十九日,哲近侍李顺千里追杀,斩刺客于江渡,天下皆知,闻者慑服,后数年,未敢有效聂荆者。

--《南朝楚史·江随云传》

毒手邪心神色一变,冷冷道:‘李顺,我还道你在主子身边服侍,想不到你还有胆子追来。‘

小顺子微微一笑,道:‘黑爷,我们虽然素未蒙面,但是我知道德亲王身边有你这么个人,你也知道公子身边有我的存在,你刺杀公子,就是我的死敌,就是我不如你,也要来替你送行的,更何况,你恐怕是不如我的。‘

毒手邪心心中一凛,他的姓名已经多年不用,就是德亲王也不知道,想不到竟被小顺子说破,但他神色上一点不漏痕迹,淡淡道:‘李顺,你也算是南楚的臣子,常年待在君侧,受恩深重,为什么背叛家邦,难道荣华富贵真的对你如此重要么,就是有了些许富贵,也是轮不到你的,你也曾经从军出征,也曾经陪王伴驾,难道不知道忠义的道理么?‘

他这样一说,就是倒在地上的众人看向小顺子的目光也变得鄙夷。

小顺子却是不卑不亢,淡淡道:‘奴才出身微贱,又是刑余之人,说句难听的话,在宫里面,就是猫狗,也比我们尊贵些,黑爷,您不过是个杀手,不也将奴才瞧扁了么。‘说到这里,小顺子神色变得庄严,眼中更是放出光芒,他一字一句道:‘这世间只有一个人,从来没有看不起我,他将我看成人,不是一个奴才,宫中初次相见,公子乃是南楚新贵,我不过是一个微末奴才,他却那般看重我,数年相处,若是稍有虚伪,我早就看穿了,可是公子始终如一,待我如父如兄,教我读书明礼,待我如骨肉腹心,这一生一世,只有公子值得我效忠,南楚待我没有什么恩德,黑爷以大义相责,我就问上一句,公子对南楚可谓无愧于心,可是南楚对得起公子么?‘

毒手邪心默然,他怎不知江哲的功劳,可是最后却被免官致仕,自己去行刺他,无论如何也说不过去。

小顺子却没有继续逼问,反而冷冷道:‘我知道黑爷是奉了亲王遗命,所谓各为其主,公子不恨亲王无情,可是却不能让你生还南楚,所以对不住,今日我要你命丧大雍。‘

这时,身子不能动弹的乔焰儿怒道:‘好大的口气,不知道天高地厚。‘

这句话一出口,就连毒手邪心也神情诡异地看着她,现在的局势明明是小顺子是站在这些青年人一方的,如果小顺子不能取胜,只怕任何一个人都会被杀,怎么乔焰儿反而这样说话。其实乔焰儿话一出口就觉得自己说错了,只是她生性好强,自己莫名其妙的中了暗算,小顺子这样突如其来,救了自己等人,反而让她心生不满。见到众人目光落到自己身上,她不由嗔道:‘怎么,人家说说不行么?‘

所有的人目光都移开,免得笑出声来,小顺子神情却是依旧冰冷,他对乔焰儿等人也没有什么好感,反正都是公子的敌人,若是可能将他们全部杀了倒好,若非碍于自己这次出面必然会人尽皆知,故而不能落井下石,只怕他还会亲手杀了这些人呢。

看了看苦竹子,小顺子目光变得有些柔和,他开口道:‘苦竹子,今日原本也该将你处死,可是我家公子有些话要人带回去,既然你身份已经暴露,这件事情就交给你吧。‘

苦竹子没有嘲笑,他从小顺子一出现就开始寻找他的破绽,只是小顺子虽然就那么简简单单的站着,浑身上下却丝毫看不出破绽。

看看天色,小顺子叹息道:‘雾失楼台,月迷津渡,好一派迷人风光,只可惜黑爷你再也看不到了。‘说罢,他的身形如虚如幻一般向毒手邪心扑去,毒手邪心也知生死就在这一战之中,挺身迎上,身形如同飞鹰展翅,两人身形一相交,只见掌影交错,却没有丝毫声息,原来两人的掌法都是极为灵巧诡秘,十几招相互攻击,都是攻敌之必救,一触即转,竟没有真的碰上,两人斗得凶猛,就在丈许空间之内翻翻滚滚,令人看的眼花缭乱,虽然听不到声息,但是从两人交手之处溢出的掌风杀气却是越来越重,这样打了百招左右,两人的身形突然停了下来,相对而立,小顺子神情冷淡,毒手邪心却是面色铁青,胸衣被撕破,露出几处类似爪痕的伤口,一见就知他已经落了下风,两人虽然静立不语,但是两人之间的张力却仿佛弓弦一般越拉越紧,终于毒手邪心忍耐不住,一声厉叫,面色数变,顿时七窍流血,形容可怖。

三姑娘远远看见,惊叫道:‘这是天魔解体大法的第三变,功力增加到十倍,阁下当心。‘

小顺子却是冷冷一笑道:‘天魔解体大法虽然激增功力,可是后患无穷,不到两个月使用两次,看来就是你回到南楚,也是性命不久了。‘

毒手邪心冷冷道:‘你的主子虽然才智无双,但是若没有你的保驾,只怕也是苍鹰折翼,这次虽然不能杀了他,取了你的性命,也是断了他的臂助,日后行刺起来容易多了。‘

小顺子面色变得铁青,想不到毒手邪心仍然打着刺杀公子的鬼主意,眼中杀机更加浓厚,这时毒手邪心已经扑了上来,这次局势大大不同,小顺子似乎被打得没有还手之力,只能凭着诡异的身法自保,众人看了片刻,都闭上眼睛,只因这两人身影变幻,竟让他们生出头晕目眩的感觉。又过了片刻,小顺子突然深吸一口真气,登时身轻如羽,随着毒手邪心的掌风飘然后退,蓦地升高,然后反扑过来,毒手邪心促不及防,连忙二度出掌拦击,却不料小顺子的身形竟然凭空折转,落到了他的背后,一只苍白的手掌按在他的后心,毒手邪心只觉得一股阴柔冰冷的真气涌入自己的身体,他用尽内力抵挡,那真气却变得炽烈如火,涌入他的经脉,毒手邪心不由一声惨叫,身形踉踉跄跄的向前扑去,跌倒在地,就在这时,苦竹子从小舟之上顺风袭来,小顺子原本已经是真力用尽,谁知他却仿佛神助一般,身形诡异的折转迎上,苦竹子虽然水上功夫天下第一,可是这掌法内力差得还远,这次若非是想用他隔绝毒手邪心水路逃生的可能,也不会有机会被邀请前来参加围攻毒手邪心。小顺子只是三招两式已经把苦竹子击退,苦竹子退到江边,却是进退两难,若是退走则要眼看着毒手邪心丧命,若是进攻,却又不是对手。

这时,毒手邪心已经有了力气,他勉强站了起来,苦笑道:‘顺公公果然武功高强,江哲何幸,得到这样高手为奴。‘

小顺子淡淡一笑道:‘应该说李顺何幸,能得公子厚爱,跟随身侧,如今阁下已经命在旦夕,不知道可有什么遗言相告。‘

毒手邪心自然知道自己心脉已断,不过是凭着精纯的功力苟延残喘罢了,他心中没有一丝恐惧,笑道:‘我知道顺公公想问什么,不就是谁救了我的性命么,在下直言相告,那人就是秦青,他就是射杀江哲的凶手。‘

小顺子冷冷道:‘你没有别的人选可以嫁祸了么?‘

毒手邪心心中一跳,但仍然道:‘我本楚人,何必为大雍张目,所以一字不假,就是秦青。‘

小顺子淡淡道:‘本该用刑罚迫你说出实话,但是你如今命在顷刻,罢了,你就好生去吧,九泉之下见了亲王,请代我家公子问安。‘说罢轻施一礼。毒手邪心心中一松懈,已经软倒在地,这时小顺子突然问道:‘裴云和夏侯沅峰谁的武功更高些?‘毒手邪心不察,答道:‘夏侯--‘突然醒觉,改口道:‘夏侯沅峰未曾交手,不知深浅。‘

小顺子淡淡看了他一眼,道:‘苦竹子,代我家公子转告容先生、陆公爷,从前公子虽然无负南楚,但是念及旧情,仍然心有愧疚,如今公子九死一生,与南楚再无情分可言,今后沙场相见,也是陌路之人。‘说罢他的身形一闪,转瞬就到了数丈之外,片刻之间就消失在夜色当中。

苦竹子神情一松,上前探察,毒手邪心已经死亡,再无一丝气息,面上带着疲倦的微笑,仿佛放下了千斤重担一般,他抱起毒手邪心的尸身,看看地上瘫软的敌人,知道自己若是杀了他们,必然是大大得罪了李顺,便微微叹息了一下,上船取桨,飘然而去。他的小舟刚刚隐入对岸的芦花丛中,功力最深的凤仪门三姑娘已经可以行动,她站了起来,将门中秘制的迷香解药给众人服下,虽然药不对症,但是也起了作用,没过多久,众人就都可以起身了。

七姑娘惊叹道:‘三姐,想不到世间还有这样的年轻高手,就是大姐和九妹也不容易胜过他吧。‘

端木秋等人虽然面色惭愧,却也都点头称是。

三姑娘面上露出悲天悯人的神色道:‘你们只知道他武功高强,却不知此人付出代价的惨重,听他们的交谈,这人乃是太监出身,那么天下只有一种武功可以让他如此厉害,便是失传已久的葵花宝典,只是不知他是为了练这种武功才自残身体的,还是做了太监之后才练了这种武功,唉,这种武功虽然精妙高深,可是练了之后性情不免变得阴狠残忍,有这种人在江湖上存在,只怕终究是一大祸患。‘

乔焰儿方才虽然出言不逊,可是毕竟是感激小顺子救命之恩的,此时开口反驳道:‘明姐姐太过虑了,这人既然是为主子报仇而来,那么他就是南楚第一才子江哲的仆人,妾身虽然与江大人素未蒙面,可是也知道他雅量高致,才华过人,他的仆人怎会危害天下呢?‘

三姑娘叹息道:‘就是如此,妾身才心中不安,这人虽然可怕,不过是一个武夫,那江哲乃是国士无双,两人相辅相成,只怕大雍朝野不安,这次回去定要向师尊禀明,若是将来不可收拾,恐怕只有她老人家才能挽回局势了。‘

众人听了都觉得有理,凤仪门领袖群伦,果然是见识深远。

众人互道珍重,各自离去不提。这一战虽然没有流传到民间,但是朝野多有知者,毒手邪心本就是南楚有数的高手,这次更加是在雍王府内刺杀‘得手‘,而且又千里转战,逃出大雍,小顺子一举克敌,顿时成了各方瞩目的人物,若非他的出身尴尬,只怕已有资格挑战大雍第一青年高手的宝座了。但是此刻的他还没有这个认识,他心想,果然是夏侯沅峰嫌疑重些,可是只怕公子不会许我出手杀他,若是就这样便宜了他,岂不贻笑天下。不如我先去杀了他,只要没人看见,谁知道是我出手的呢?所以小顺子也不和雍王府的人联络,日夜兼程向长安赶去,不过数日,他就已经回到了长安,略略改装之后,挑了一个晚上,他直接找到夏侯沅峰府邸,知道今日夏侯沅峰应该是没有差事,所以他准备直接到内宅刺杀。谁知刚刚接近夏侯府,一个身影就拦住了他,他正要出手,那人将帽子掀起,露出一张略带稚气的脸庞,那人正是赤骥,秘营八骏之首,小顺子脸色一沉,就要不理不睬的过去。

赤骥连忙道:‘属下是奉了公子谕令,在此等候李爷,公子说,李爷不可莽撞,先回去见他再说。‘

小顺子神色冰冷,一言不发,赤骥只得接着道:‘公子说,若是李爷现在不回去,以后就不要回去了。‘

小顺子握紧了双拳,他自然知道江哲是绝不会随便这么说的,看来自己是真的必须回去来,狠狠的看了夏侯府的方向一眼,他转身离去。

赤骥连忙拉下帽子,身影很快的消失在夜色中。

匆匆赶回雍王府,小顺子也不梳洗,直接赶到寒园,见新选的护卫将这里围得水泄不通,他略略有些放心,走进江哲的居室,只见他躺在软榻之上,仪态悠闲,正在那里朗朗颂读诗经,而多日不见的柔蓝倚在他身边,似乎听得入迷。

小顺子只觉得心情一下子轻松下来,罢了,就是现在不杀夏侯沅峰,难道公子还会让他好过不成,上前深施一礼,他说道:‘奴才回来了,向公子请罪,奴才以后都不敢妄为了。‘

我放下书卷,看向风尘仆仆的小顺子,道:‘你辛苦了,先坐吧,你可知道我为何会知道你去夏侯沅峰府上?‘

小顺子疑惑地道:‘奴才也正在猜疑,怎么公子知道我的行踪呢,那些目击之人就是听了我的话,也未必会来得及传出去啊。‘

我微微苦笑道:‘昨日,夏侯沅峰亲自来拜访,向我请罪,说是那日他确实到了寒园,只是下手行刺的不是他,他不过是带走了毒手邪心,因为那射我一箭的人身份尊贵,他不敢出面拦阻,带着毒手邪心不过是想得知一些内情,不过毒手邪心什么也不肯说,还趁机逃走了。‘

小顺子愣住了,半晌才道:‘那岂不是只剩李寒幽了。‘

我淡淡一笑道:‘我本来就猜疑那行刺之人眼若春水,素手纤纤,怕是一个女子,没想到夏侯沅峰居然也自承在场,想必当日来行刺的只怕有三个人,毒手邪心是为了德亲王遗命而来,最不用多虑,夏侯沅峰和太子最亲近,这种事情想必太子也不愿麻烦凤仪门,只怕夏侯沅峰才是太子派来的,不过却赶上凤仪门对我动了杀机,齐王妃先藏弓箭,李寒幽亲自出手,所以当日夏侯沅峰就没有出手。我想,如果夏侯沅峰真是那射箭之人,只怕他早就杀了毒手邪心灭口了。只不过,为什么凤仪门会想杀我呢,莫非是那件事露了痕迹。‘

小顺子神色数变,道:‘公子,凤仪门盯上了您,这下我们可得加倍小心。‘

我淡淡摇头道:‘不妨事,这次他们行刺不成,若是凤仪门主真是传说中那么高傲,那么她们就不会再次行刺,若是不能通过别的途径对付我,她们的名声未免有损,毕竟现在我若死了,只怕人人都知道是凤仪门干的了,我想我的安危暂时可以无忧,不过要提防她们其他的手段,现在我重伤在身,正可以避过她们剑锋所指,倒是你名声突显,要当心一些。‘

小顺子点点头道:‘公子说得是,不过奴才会小心的。‘

我伸了一个懒腰道:‘你说得也有道理,我累了,你送柔蓝回去吧。‘

小顺子连忙道:‘公子,我胡乱妄为,你还没有惩罚我呢?‘

我懒洋洋地道:‘好啊,惩罚你,对了,我很想吃桂花糕,就罚你买一盒上好的桂花糕,要我以前爱吃的那种。‘我已经半睡半醒,完全没有意识到我在说什么。

小顺子愣住了,桂花糕,南楚建业最富盛名的小吃,这里怎么吃得到,就是自己回去建业买了过来,那也不新鲜了。

怔怔地走出门外,这时五十名护卫的队长周武走了过来,见他这样神色奇怪,问道:‘李爷,怎么了,可是大人有什么吩咐么?‘

小顺子苦恼地道:‘怎么样才能买到桂花糕?‘周武愣住了,喃喃道:‘桂花糕。‘小顺子却已经抱着柔蓝走远了。

\chapter{第二十二章 南楚使节}

南楚同泰元年四月,国主陇遣使大雍,纳贡称臣,宛转求和,以重金求赎。

--《南朝楚史·楚炀王传》

我半夜睡得正香,突然被人推醒,等我恼怒地睁开眼睛,却看见小顺子喜津津地捧着一笼热腾腾的桂花糕献宝,我惊讶之余问他从哪里弄到的,毕竟这可是南楚最有名的糕饼店‘桂香坊‘的拿手绝活啊。拿了一块咬了一口,香甜酥软,入口即化,我满足的问道:‘从哪里买的?以后可要常去光顾呢。‘

小顺子脸色一变,一脸的神色惨淡,我奇怪的问道:‘怎么了?‘

小顺子犹豫了半天才说出实情,原来他想了半天,最后决定随便找个大雍的美食代替,谁知道一出门就听说南楚的使节已经到了长安,他连夜到驿馆探听,原本想看看有没有不利于我的事情,谁知使团带了桂香坊的两个师傅过来,正好做了两笼最出名的桂花糕,准备送到被软禁的国主赵嘉和长乐公主那里,或许他们是想讨好长乐公主,以求谈和成功,但是却便宜了小顺子,他用了偷天换日的手法,把其中刚做好的一笼桂花糕偷了出来。

我差点昏了过去,不知道丢了桂花糕的南楚使团会不会报官,转念一想,还是赶快消灭证据吧,狼吞虎咽地和小顺子平分了一笼桂花糕,这时,天色已经渐渐亮了,小顺子便溜走了。我刚想多睡一会儿,小顺子又来禀报道:‘公子,南楚正使陆灿求见。‘

我心中一动,这个我曾经的学生为何来求见我呢,他不是应该对我不屑一顾么,毕竟我已经是南楚的叛逆了。我疑惑的向小顺子求教,小顺子哭笑不得地道:‘公子,如今你是雍王殿下的亲信,这谈和之事,殿下至少可以做四分主,若想从殿下这里着手,公子你不就是最好的人选,虽然都是战败求和,但是能够多得一分好处,对南楚也是有利的呀。‘

我坐起身来,接过小顺子递过来的外衣,一边着衣一边想该如何解决,本来我想着‘相见争如不见‘,并不准备接见陆灿的,可是他若是为了谈和之事四处游说,那么自己若不给他机会就未免有些过分,无论如何自己曾是南楚臣子,现在又是雍王属下,若是自己婉辞,那么在外人看来就会以为雍王殿下无心和议,这件事情可大可小,我就不能随意处置了。走动了几步,觉得今天身体不错,会客应该没有问题,我便说道:‘请陆将军到花厅见我,现在天色还早,叫人将早饭送到花厅,多准备一些,就说我请陆将军用饭。殿下应该已经知道了,你派人去问问殿下的意思,要不要接见南楚的使者,议和的事情我不大清楚,苟廉应该比较明白,若是殿下不便前来,就请苟兄前来作陪,也好探探南楚的底线。小顺子,陆灿是一个人来的么?‘

小顺子答道:‘公子,陆将军带了一个青年,那人相貌不俗,应该是才智过人之辈。‘

我微微一笑道:‘也好,陆灿毕竟年轻,若是他独自前来,我倒怀疑他不过是私自来见我,既有人相陪,那就是公事为主了,好了,去请他们进来吧。‘

陆灿静静的立在雍王府门前,二十二岁的他正是血气方刚的年纪,但是多年军旅生涯让他比同龄者显得成熟,他的相貌粗豪,有些不似江南人物,但是只见他双目中神光隐隐,气质豪勇中带着儒雅,就知道这个少年将军乃是文武双全的奇才。站在他身后半步的是一个二十六七岁的青年,他方巾儒冠,清秀文雅,举止之间,别有一种风仪,令人生出乐于接近的感觉。

这个青年望着神色淡然的陆灿,心中波涛汹涌。他叫杨秀,原是蜀国人,蜀国灭亡的时候,他还游学在外,在南楚占领蜀中的时候,他返回故乡,蜀中在陆侯治理下十分平静,虽然有锦绣盟肆虐,但是他们也没有掀起什么大风浪,杨秀在家中过得日子十分平静,两年半前,他的一个堂兄因为参与了刺杀陆侯的行动被判罪,杨秀也被牵连下狱,负责审理的正是陆侯独子陆灿,这个少年将军办起事情来明快果决,而且合乎情理,杨秀很快就被无罪释放,而且陆灿见他气度才华都有过人之处,亲自上门请他做自己的参军。杨秀不是迂腐的人,他没有在蜀国取得过功名,为南楚效力也不算是失节,跟从陆灿之后,他越发觉得这个青年将军的过人之处,陆灿年纪虽轻,但战阵运筹,兵法谋略都是超人一等,雍王突袭南楚的时候,陆侯带兵回援,东川庆王趁机兵压蜀中边境,陆灿带兵迎敌,两军数次交锋,陆灿苦练的精兵竟然挫败了大雍的雄兵,迫使庆王退兵,保证了南楚不会两面受敌。虽然因为建业失陷,陆灿的功绩没有被公开,但是南楚军中已经隐隐将陆灿当成了德亲王赵珏的继承人。更让杨秀叹服的是,陆灿虽然出身武将世家,也不会写诗作文,但是对于经史颇有独到的见解,每每谈论起史上将帅胜败之道,如数家珍,就是自己有的时候也不得不佩服陆灿的见识广博。

前些日子,杨秀忍不住问陆灿,是谁能够把陆灿这样的武将子弟教得精通文史,陆灿却是沉默不语,不料昨日刚刚到大雍,递上国书,今日陆灿就带着自己来拜会那个久闻其名的江哲。杨秀虽然知道江哲这个人,但是并没有把他看得很重,不过是一个投降了大雍的南楚才子,若不是前些日子的刺杀一事沸沸扬扬,让他留了心,他还不会注意到江哲的存在呢。知道昨日他才知道原来江哲竟然就是陆灿的恩师。他到现在还记得昨日夜里,银灯下,陆灿的面孔隐藏在阴影中,淡淡说道:‘我自幼顽劣,每日里不是爬墙上树,就是耍枪弄棒,再不然就是去和那些街上的青皮打架,父亲不愿看我这样不学无术,就请了西席来教我,我仗着拳头硬,打跑了好几个西席,江先生就是第四个西席,我原本想给他一个下马威,可是他一来就对我说,他也不过是混碗饭吃,反正我若是打跑了他,我父亲还要请新的来,我若是肯和他妥协,他就让我们两个都好过。‘

说到这里,陆灿面上露出淡淡的笑容,接着说道:‘江先生说,只要我每天上午在书房里面呆着,下午随便我去干什么,他不会给我留过多的功课,而且还会帮我瞒哄父亲。我当时答应了,可是没几天我就后悔了,每天上午我闷在书房里,就看着江先生看书看的津津有味,也不理会我,可是若是反悔未免太丢面子,后来我只好求江先生想个法子让我消磨时间。江先生便说,既然这样,不妨给我讲讲书,我虽然觉得无聊,可是总比一个人闷着强,可是没想到江先生真是才华绝世,他不会让我被那些四书五经,也不会要我写诗作文,他说我是世家子弟,又不用去参加科举,学那些没有用处,他先是给我讲论语,一本别人说来枯燥无味的论语,被他讲得妙趣横生,然后他就给我讲史书,他也不给我讲原文,只是把那些史实像故事一样讲给我听,还夹杂了很多他自己的见解和一些野史上的事情,从那以后,我每天上午都在听他讲故事,后来他看我更喜欢用兵打仗,又给我讲兵法、战例,我也不知道他怎么知道那么多事情,他明明不比我大几岁,可惜我那时候太贪玩,不明白先生的教诲是多么珍贵,直到后来我领兵作战,才知道先生教给我的东西有多重要,可惜却已经没有机会再向先生请教了。杨秀,我说这些是要你明白我的恩师是一个什么样的人,如今他已经归顺大雍,日后难免沙场相见,你富于计谋,将来是要做他的对手的,我一个人必然不行,你要把握机会好好了解他,若不了解自己的敌人,那么就没有必胜的把握。‘

杨秀越想越是心情彭湃,他很想看看这个自己十分尊敬的少年将军那样敬重尊敬的恩师是一个什么样的人物。所以等的时间越长,他就越担心江哲不肯接见他们。

幸好过了一段时间,一个青年侍卫过来行礼道:‘陆将军,司马大人在寒园接见将军,大人重伤初愈,不便出迎,特遣呼延寿前来迎接。‘

陆灿看了青年侍卫一眼,只见这人相貌质朴,但是双目寒光四射,一双手掌又宽又大,指节突出,虬筋纠结,必然是修炼外功之人,而他行动之间却是点尘不惊,可见火候已经到了炉火纯青的地步,再看这人周身上下杀气隐隐,身姿挺拔,这一定是久在军中的勇士,雍王让这样的人做恩师的侍卫,可见他对恩师的看重。心里想着,陆灿微笑道:‘麻烦呼延侍卫带路。‘

两人跟着呼延寿走了半天,才到了一处幽静深远的园林,看到园门上的匾额,陆灿知道自己终于可见见到江哲了。呼延寿和园门前守卫的四名同僚打了一个招呼,引着两人走进寒园。一走进寒园,陆灿就觉得心中大震,虽然没有看到,可是他隐隐能够觉察到园中所有关键位置都有人藏伏,虽然见不到人,但是只凭着那种凝厚的杀气,就知道这里的侍卫都至少和呼延寿水准不相上下。看来雍王对恩师的器重是无以伦比的。

两人被请进花厅,第一眼看见的就是坐在那里的江哲和站在江哲身后的小顺子。

杨秀大胆的看去,就在桌旁坐着一个相貌消瘦苍白的青年,他穿着一件淡青的长袍,头发只用一根发簪和一条雪白的丝巾束住。他就那样闲散的坐着,神色平和,若非是见他形容憔悴,绝不会想到他刚从死亡线上挣扎回来不久。杨秀心里叹服,他原以为江哲既然刚才南楚刺客手中逃生,那么对于陆灿不免会冷淡非常,他不知刺杀江哲成功的另有其人,真相早已经被人隐藏起来,就是雍王对外也是说江哲是被南楚刺客刺杀成重伤的,毕竟没有人愿意将大雍的内部纷争暴露在外面。所以江哲并没有对南楚虽然灰心失望,但是并没有十分痛恨。

我看了陆灿一眼,他比起上次见面更显得沉稳,想必是独当一面之后成熟了许多吧,我站起身,笑道:‘小侯爷,多日不见,你越发雄壮了。‘

陆灿一看到我就愣住了,听到我说话才醒觉过来,连忙上前下拜道:‘弟子拜见恩师。‘语气竟然有了哽咽,我知道他是见我形容如此而伤心,就是我自己在铜镜之中看了自己都觉得有些脱形,这也是无可奈何之事,我能够保住性命已经是万幸了,哪里还敢奢求呢,反正最多一年半载我就能恢复健康。

我抬起手道:‘小侯爷快起来,不,你如今也已经是南楚的大将了,我该叫你陆将军,哲不过曾经做过将军几日的西席,怎敢当师徒的称呼。‘

陆灿心情已经平静下来,淡淡道:‘弟子当年顽劣,不知道恩师教诲的重要,如今已然是追悔莫及了,还请恩师不必推诿,弟子不会凭着师徒名分求恩师做非常之事。‘

我微微苦笑道:‘你性子还是这样直率,罢了,我也不想和你争辩,起来吧,我还没有用餐,你陪我一下吧,这位是?‘我看向杨秀。

陆灿站起身道:‘这是弟子麾下的参军杨秀。‘

杨秀上前行礼道:‘久闻江大人声名远扬,下官拜见。‘

我想要上前搀扶,但是只觉的心口一痛,只得皱皱眉道:‘请恕下官不便还礼。杨参军也请入席。‘

杨秀只见江哲额上竟然有了冷汗,连忙道:‘大人身体不便,不需多礼。‘

我们三人坐下,小顺子亲自端了三碗粥上来,我笑道:‘这些粥都是精心做的药膳,里面加了滋补的药物,两位不妨尝尝。‘

陆灿站起身接过小顺子递过来的碗,他可是知道的,前些日子这个李顺在长江渡口击杀毒手邪心,毒手邪心在投靠德亲王隐姓埋名之前就是南楚有数的高手,这次更是在雍王府里行刺‘成功‘,更是转战千里,逃出大雍,声名扶摇直上,不料就在月夜长江岸边,被这个少年所杀,一夜之间,李顺之名传遍天下,所以陆灿不敢怠慢。

林秀也是同样站起接过粥碗,他不由看了江哲一眼,这个瘦弱的青年有什么奇特之处,竟然让这等高手甘心为奴,做着下人的事情呢?

我见他们这般拘束,不由一笑,道:‘这次听说陆灿你是南楚正使,想必已经有了全盘的打算,不知道我能帮上什么忙?‘

陆灿神色有些赧然,但很快就恢复平常,恭恭敬敬地道:‘南楚虽然战败,但是如今新君已立,上下齐心,兵马齐备,所以这次虽然称臣求和,但是希望大雍不要过分索取金帛,并且希望能够赎回太上国主和文武百官,只是此事虚得大雍军方首肯才有可能,雍王殿下更是其中最重要的人物,所以弟子虚得知道殿下的意思,‘

我淡淡道:‘谈和之事自有朝中大臣主持,雍王殿下的心意又有谁敢揣测,再说陛下又没有为难南楚的意思,你倒是过虑了,这些事情我也不大理会,你这可是找错了门路了。‘

陆灿知道江哲这样说只是托词,正要继续劝说,这时门外传来一个爽朗的声音道:‘怎么说找错了门路呢,若非陆将军先来求见你,本王是断不会让南楚轻松自在的。‘

说着,李贽带着苟廉走了进来。陆灿和杨秀都起来施礼。李贽笑道:‘陆将军,本王曾经跟令尊陆公有过一面之缘,早听说陆公膝下有虎子,今日一见果然名不虚传,我那三弟写信来说陆将军用兵如神,他可是佩服得很。‘

陆灿沉稳地道:‘小将不过是假父亲余威,雍王殿下才是天下用兵大家,萤火之光怎敢与皓月争辉。‘

李贽坐了下来,沉着地道:‘两国修好,本王也知道势在必行,但是贵国擅自称帝,不顾臣属的身份,我大雍兴兵讨伐,乃是大义所在,虽然贵国损失惨重,但是理应割地赔款,至于赎回俘虏之事,本王并无意见,只是贵国想付出多少赎金呢?‘

陆灿正容道:‘南楚虽然也有理亏之处,但是贵客齐王先兴兵犯境也是事实,殿下攻占建业,掳走我国君臣,更加夺走金帛无数,如今我国上下一心,若是贵国还想欺凌,我们虽然国小力弱,也要反抗到底,南楚大雍虽是君臣,也是姻亲,贵国久有侵占之意,如今我们虽然屈膝求和,但是也不能容许贵国予取予求,我国新君已经登基,先国主已是平民之身,若是贵国想要留下就请便,先国主与贵国长乐公主乃是夫妻,女婿依附岳父而活,也是理所当然。‘

李贽目光一亮,笑道:‘说得好,果然是年少英杰,南楚奇才何其多也。本王佩服。‘然后意味深长地道:‘事情也是可以商量的,本王虽然不能作主,但是也不会为难陆将军。‘

\chapter{第二十三章 魂归故里}

和议既成,炀王得免,五月,随使臣返南楚,方入楚境,遇刺身亡,归葬建业。王在位四年,疏于朝政,亲小人,远贤臣,至令社稷危亡,身亦深陷囹圄,南楚积弱难返,皆王之罪也。

--《南朝楚史·楚炀王传》

陆灿大喜,他知道只要雍王不为难南楚,那么其他的人或者用贿赂,或者用利益,总是比较容易摆平的,连忙向雍王道谢,不过陆灿神色没有什么变化,他知道雍王必然是要提些条件的,所以他诚恳地道:‘殿下宽宏大量,灿代南楚上下拜谢殿下,若是有什么吩咐,还请直言,灿纵然为难,也要勉力为之。‘

雍王却是一笑置之,他从南楚的府库里面得到的足够他数年军用,所以并不贪求,而且在他看来南楚百姓迟早会是大雍的臣属,所以他也不会在这个时候雪上加霜,若是引起南楚百姓的刻骨仇恨,对于日后安抚江南可是不利的,至于是否割地赔款,那是朝廷的事情,他早就知道父皇的底线就是南楚赔款五千万两白银,分十年还清,这样一来,南楚在十年之内是别想大规模扩充军备了。但是若是不提要求,不免有些引人疑窦,甚至还会让南楚君臣不安,担心自己什么时候发难,看了一眼江哲,他用目光询问。

我收到了雍王的暗示,心中一动,淡淡道:‘雍王殿下很是仰慕南楚的文章风流,听下官说起崇文殿之事十分羡慕,若是陆将军能够作主,将崇文殿收藏书籍的副本送来一份给殿下,当然,若是能够加上一批名家真迹,那么就更好了。若是贵使能够达成殿下的心愿,那么殿下可以保证不会索取南楚一寸国土。‘

陆灿一愣,他是武将,对于这些书本并非十分看重,雍王的要求对他来说并不过分,用些书本字画换来雍王的退让,让南楚不会因为议和损失惨重,那么还是值得的,只是崇文殿乃是先王敕建,若是这样做,不免有人会弹劾自己。想到这里,不免有些犹豫。

我看出他的心意,淡淡道:‘太上国主目前还在大雍,将军若是拿不定主意,可以去问问王上。‘

陆灿立刻醒悟过来,这么好的一个挡箭牌不用,自己还是太没有经验了,于是陆灿欣然道:‘殿下所请,本使代王上同意,等到本使回到南楚之后,立刻派人送来。‘

李贽正要答应,我却道:‘若是这样时间耽搁太久,还是请使节传书回去,若是能够在谈判之前将书籍送到,雍王殿下必有所报。‘

陆灿看了杨秀,露出询问之色,杨秀乃是文士,他凛然的看了江哲一眼,江哲索要的书籍乃是南楚文化之菁华,此人目光之深远果然非同寻常,想当初成都陷落的时候,大雍和南楚都抢着争夺户部的典籍,这些都是治理国家的基础啊,江哲这次索要的虽然不是南楚的户口图籍,但是那些书籍的价值是更加珍贵的,江山总有改朝换代的时候,户口图籍总是能够盘查清楚的,只有那璀璨的文化是源远流长恒久不变的宝物,但是他也知道这个条件对于南楚来说不是难事,至少比割地什么的好多了,在心中常常的叹息了一声,他轻轻点头。陆灿当机立断地道:‘司马大人放心,陆某立刻传书回去。‘

送走了南楚的使节,我已经有些疲惫,李贽便告辞离去。在路上,李贽若有所思地道:‘随云果然精明,若能够得到南楚的典籍,对我大雍果然是很有裨益,毕竟我大雍擅长开疆扩土,但是治理国家却是得靠文治,马上得来的天下,不能再马上治理,随云,真国士也。‘苟廉冷冷道:‘殿下,那个南楚使节不卑不亢,而且文武双全,又是南楚青年将领的领袖,此人不除,只怕日后必是后患无穷。‘

李贽淡淡一笑道:‘天下的俊杰多得是,本王若是见一个杀一个,只怕就要杀得手软了,南楚积弱,独木岂能擎天,没有明君,就是武将再能征善战又能如何,苟廉,替我告诉韦相爷,一定要把尚维钧送回去,怪不得当日随云让我善待尚维钧,看来他早就想到今日了,若是尚维钧回到南楚,陆信必然不会再大权独握,加以时日,身为外戚的尚维钧就能权倾朝野,到时候权臣在内,我倒要看看几个才俊之士能够掀起什么大风浪。‘

苟廉心中一寒,他虽然已对江哲倾慕非常,但是还是没有见过江哲用计的手段,如今听李贽道来,江哲这样深谋远虑,真是令他彻底倾服,不过,他看看雍王,殿下能够这般机敏,看穿江哲的用心,并加以利用,如此君臣,当真会让敌人心胆俱裂,怪不得殿下为了江哲费了那样的苦心,现在看来,一切都是值得的。想到这里,心中久久藏着的一丝妒念终于烟消云散。他一边欣然领命一边道:‘殿下不妨多给南楚一些好处,只当看在江先生面上,这样江先生就是嘴上不说,心里也是欢喜的。‘

李贽叹息道:‘是啊,就是如今他对南楚心灰意冷,还是有些情分,若是不然,他何必答应接见南楚使节,他这般情厚,只怕日后攻打南楚,他是不会出力了。‘

苟廉笑道:‘殿下放心,区区一个南楚,若是都攻打不下,岂不是让人笑话我们这些属下么,殿下麾下文武齐备,还担心什么呢?不过殿下,长孙将军和荆将军都有信来,他们说军中无事,问可不可以跟随殿下左右,他们对前些日子的事情都是心有余悸,而且殿下身边也需要多几个护驾的大将。‘

李贽想了一想道:‘你说得也不错,现在军心我相信不会有什么变化,也应该把他们招回身边,这些日子没有他们,本王总是觉得有些心有余而力不足,你替我传令,让他们进京,他们都是我帅府管辖,没有人能说出什么来。‘

接下来的谈判虽然繁琐,但是倒没有什么阻碍,南楚方面自然是急切的希望谈判成功,而大雍方面也没有人想拖延,雍王既不插手,太子也懒得过问,齐王这段时间似乎心情不好,几乎连门都不出,所以谈判在丞相韦观的主导下进行的十分顺利,南楚使节团其他成员自然是十分高兴,但是陆灿心中却隐隐不安,总觉得不该如此顺畅,虽然事情顺利,但是因为需要商讨的细节太多了,还是拖了将近一个月才达成协议。

按照和议结果,南楚继承称臣大雍,年年上贡,岁岁来朝,这次战败,南楚必须付出赔款六千万两白银,分十二年付给,另外两国协议互市,将近一年的战争和封锁,两国都需要通商,不过南楚的货物进入大雍的税收增加了半成。大雍俘虏的南楚王室成员和文武百官均可赎回,各有身价不等,不过要留下人质,人质的人选,最后入选的是赵嘉的长子赵僖,乃是雍女所生,另外一个人质是赵嘉的亲弟,简亲王赵耘。至于赵嘉身边的雍女宠姬,大半都要求留在故乡,陆灿也不计较,她们将来的生计自有大雍料理,他恨不得所有的雍女都不回去呢,不过还有两位雍女妃子要求跟着赵嘉去南楚,她们所有的青春都留在了南楚,所以宁可回到充满敌意的南楚,也不愿离开儿女。

四月二十日,南楚答应送给雍王的几百车书籍进入了大雍地界,雍王派去的两员大将接收了书籍,然后亲自押送到长安,这两人一个叫长孙冀,金弓长孙,弓箭无双,乃是军中第一射手,他出身贫寒,自幼从军,在军中练就了一身绝艺,他的箭术是在战场上练出来的,五百步外,取人性命如同探囊取物,他形影不离的金弓乃是雍王亲赐,使用特制的翎箭,可以在千步之外射杀大将,另一个叫荆迟,此人性情有些鲁莽,但是斩将夺旗却是无人可比,乃是雍王麾下第一勇将,押送书籍这等小事怎用得上他们,雍王调他们入京的目的明眼人一看就知道。

四月二十五日,南楚使节护送着太上国主和赎回来的文武百官踏上了回家的路程,大雍十分礼遇,太子李安替皇上郊送三十里,而长乐公主也在长亭之上斟酒送行赵嘉,不过雍帝李援有旨意,说长乐公主离家多年,要多留她住些日子。可是人人都知道雍帝根本不会放长乐公主回去南楚,因为谈判之中,南楚使节曾经提及,南楚国主赵陇愿意尊奉嫡母为太后,却被韦观婉拒了,但是这件事情他们暂时也不愿理会,毕竟迎回南楚君臣才是此行最大的目的。

送行之日,还有一个人也很引人注意,就是跟着雍王出城相送的天策帅府属官,司马江哲,虽然他重伤初愈,走起路来几乎摇摇欲坠,但是却没有人敢轻视他,人人都知道,他的一个随身侍从,李顺,做了什么事情,更何况雍王对他的爱重天下皆知。

我上前对着曾经的王上最后一次行了君臣大礼,赵嘉的目光是茫然的,他甚至已经记不起我到底是谁,当他在内侍的低声指点下说着冠冕堂皇的话,祝贺我得到大雍重用,希望我忘记从前嫌隙,为两国和好而尽力的时候,我心中一片淡然,这个人,从来都不是我想尊奉的主君,这次相送实在是为了善始善终罢了,毕竟,我怀疑他能否活着回到南楚,若非陛下想早些除去这个女婿,何必急着结束谈判呢?

看着南楚使节远处的队伍,我神色疲倦的想返回马车,却发觉有两个人正在注视着我,一个是长乐公主,多日不久,她神情很平和,但是比起当日观看演武的时候,显得有些憔悴,另一个人却是一个身穿月白宫装的女郎,她大概二十出头的年纪,相貌绝美,容色清华高贵,身材修长,体态优美,她就站在长乐公主身边,公主容色本也是秀丽清雅的,已经是堪称绝色,但是却被这女子逼人的艳光抢走了全部光芒。引起我的注意的不是这女子的美丽,而是她那双明澈冰寒的眼眸,那是一双我做梦都会梦到的眼睛,她,就是几乎杀死我的刺客。

我低声道:‘她就是李寒幽吧?‘

雍王已经走到我身边,低声道:‘就是她,皇后让她照顾长乐,所以一起跟来了。‘

小顺子一听那女子就是李寒幽,眼中顿时闪过耀眼的寒光,他定定的看向李寒幽,要将这个女子的一切都记在心里。

上了马车,我若有所思的想到:‘这样一个女子,高傲而美丽,正是豪门子弟梦寐以求的伴侣,秦青真的能够拒绝她么?‘

五月七日,消息传来,南楚太上国主,在渡江之后不久被人刺杀身亡,刺客用得乃是蜀中厉家的武功,留下一行血字‘锦绣河山,是我家邦,国破家亡,今日偿还。‘赵嘉身死之后,宠姬数人,皆自尽殉死。

我放下情报,轻轻一叹,皇上想必将刺杀赵嘉的事情交给了太子执行,他们果然有些本事,让锦绣盟主霍纪城刺杀赵嘉,撇清了刺客和大雍之间的关系,这般轻易得手,想必那些宠姬是内应吧,霍纪城名利双收,大雍也是心满意足,只是可惜了陆灿,他身为使节,又担负着护送的重任,可是却让赵嘉遇刺,只怕短期之内他是没有办法翻身了,不管是何人的主意,这人都是一个心机深沉狠辣之辈。只是不知道是太子还是凤仪门设计的刺杀方案,我在纸上写下两个名字,鲁敬忠、李寒幽,应该是他们两个人中的一个吧,虽然没见过李寒幽用计谋的方式,可是只见她行刺我的时候那种果断狠绝,就知道她不是一个平常女子啊。

我正在想着这件事情,李贽走进了书房,他神色阴沉地道:‘太子好手段,父皇今日重重赏赐,这趟行刺不仅天衣无缝,而且撇清了大雍的嫌疑,我只是奇怪,锦绣盟怎么会成了太子的人,虽然太子说只是暗中透了消息给锦绣盟,然后提供了一些方便,可我不信锦绣盟真的这样好利用,我一定要好好查查锦绣盟和太子的关系。‘

我心中有了明悟,太子要对锦绣盟下手了,想必他准备收手了,反正通过互市,他自然有本事得到巨大的收益,不用再冒险走私了,我看了小顺子一眼,使了一个眼色,小顺子的传音入秘在我耳边响起道:‘公子是要我告诉陈稹安排天机阁脱身么?‘

我微微点头,小顺子轻悄的退了出去,李贽迷惑地道:‘随云,怎么了,你有什么事情瞒着我么?‘

我恭敬地道:‘殿下,臣有件事情想禀明殿下,锦绣盟和太子之间确有勾结。‘说罢,我便将太子等人走私军械的事情简单说了一下,雍王皱着眉听了半天,突然拍案而起道:‘岂有此理,竟然把主意动到了军资上面,随云,你如何知道的这样清楚?你可是从中做了手脚?‘

我笑道:‘这可是臣的秘密了,不过臣的手上已经有了完整的证据。虽然是臣设的圈套,可是臣只是提供了一个机会,可不是臣让太子去做的。‘

李贽颓然坐倒,半晌才道:‘你说得是,若非太子愿意,谁能够强迫他呢?好吧,我听你的,太子既然如此行事,也怪不得本王不顾兄弟之情,军械物资何等重要,他竟然作出这种事情。只是你认为可以一举成功么,我总觉得不大可能。‘

我答道:‘殿下不必费心,这件事情自然是不成的,可是水滴石穿,请殿下相信臣的判断,这件事情若是爆发,殿下只要秉公而断即可,不必过于威逼太子,这样臣才好进行下一步。‘

李贽笑道:‘你总是这样遮遮掩掩的?‘

我淡淡道:‘臣擅长的乃是阴谋诡计,若是说了出来,不免让殿下忧心,还是让臣来策划吧,若是殿下放心,臣想调动人手去做一些事情。‘

李贽道:‘这些你不用问我,我府中上下随便你吩咐哪个,没有人敢违令不尊的。‘我轻轻颔首,表示谢意。这时,李贽看到我放在书案上的情报,有些犹豫地道:‘随云,有些事情你知道这是必然的。‘

我淡淡点头,神色一派清冷,缓缓道:‘臣知道,王上自己没有尽到君王的责任,早在建业陷落的时候,王上就该自尽谢罪了,只是王上虽然糊涂,却也不是一个坏人,这些年来他虽然没有什么建树,可也没有做过伤天害理的事情,他只是不应该去做国主罢了,王上如今魂归南楚,也该死而无憾了。‘

说罢,我起身走出了书房,呼吸了一口新鲜的空气,然后随手摘下一片竹叶,吹奏了起来,那颤抖着的古朴乐声,低徊凄切,如泣如诉,让人闻之断肠,一曲吹罢,我神色渐渐平静下来,我不是早就和南楚再无情意了么,再说那个昏君,我何必为他伤心呢。心中这样想着,我却还是有些哀伤,那人,毕竟是我曾经的王上,而且,他是南楚的国君,就这样死在大雍的密谋之下,让我如何不伤情呢。突然,我有些后悔当日逼死蜀王的事情,无论如何他是蜀国的国主,就这样死在我的笔锋之下,也难怪蜀人如此恨我呢。

李贽一直站在远处,这时才走了过来,淡淡道:‘南楚的书籍已经送到了,你不去看看么,正好也见见我的亲信爱将。‘

我轻施一礼道:‘敢不从命。‘

\chapter{第二十四章 布局猎杀}

武威二十四年五月十二日,王妃亲弟,户部侍郎崔央横死于和平坊,事乃发。

--《雍史·戾王列传》

还没有走进大厅,我就听到了一个如同雷声轰鸣的声音在那里兴高采烈的说道:‘司马,你不知道,老子这次可是走了运,那坛烧刀子可是六十年的,你想不到那乡村小店里面会有这么好的酒,所以老子都没有舍得喝,特意运了回来,怎么样,你若是请我去吃一顿好的,我就请你喝酒。‘

然后传来一个沉稳的声音道:‘老荆,别这么大呼小叫,殿下一会儿就要过来了,恐怕又要怪你不守规矩。‘

那个响雷一般的声音不耐烦地道:‘老子知道了,殿下才不会怪罪我呢,这次老子带了好东西来。‘

然后我听见司马雄笑着问道:‘你能有什么好东西,不就是那坛好酒么?‘

那个声音得意地道:‘你小子绝对猜不到,我带着这样东西殿下一定喜欢。‘

李贽微微一笑,轻轻咳嗽了一声,举步走进大厅,我也跟在后面走了进去,一进大厅,就看见已经肃手而立的两个戎装男子站在一侧。李贽走上主位,这两个人上前拜倒见礼,只看他们浑身上下流露出的尊重和敬意,就知道这两人乃是李贽的亲信将领。

我仔细打量着两人,其中一个长眉凤目,面白无须,相貌俊伟却不失清秀,身材将近八尺,却是猿臂蜂腰,丝毫不显得身材巨大,另外一个身材也有八尺,豹头环眼,相貌粗豪,身影魁梧,却如一座小山一般。两人恭恭敬敬的行礼问安之后,李贽一指我道:‘这位江司马是本王的左右手,你们好生见过,以后待他如同本王一般,不可失礼。‘

两人转身走过来向我行礼,我欠身还礼,微笑道:‘殿下言重了,哲与两位将军都是殿下的臣属,不敢当两位大礼。‘

见礼之后,我走到雍王下首的位子坐下,两人又是肃手而立,等待雍王发话。

李贽笑道:‘都坐下吧,这里不是军营,不用那么多礼,长孙,你们一路上可平安么。‘

那个长眉凤目的将军站起身来道:‘启禀殿下,一路上都很顺利,只是车马太多,不免走得慢些。这是南楚使者送来的礼单。‘说着递上一本折子。李贽翻看了一下,随手递给我道:‘这些书画什么的,本王没有什么研究,你看看吧。‘

我随手翻了一下,淡淡道:‘真正的极品不多,不过还算不错,这些本也不是臣留意的,倒是那些书籍,虽然南楚必然会留下一些紧要的经典,但是我想应该不会缺的太多,怎么也能有十之八九,改天请殿下将目录送到寒园,我仔细查一下,看看有没有少了什么珍贵的书籍。‘

李贽点点头道:‘我已经上书父皇,要求整理这些书籍,父皇已经下旨给翰林院让他们来办,太傅褚平总理此事,褚太傅为人严谨博学,必然会处理好这些书籍,这是泽被后世的大事,他不会懈怠的。‘

我笑道:‘我也信得过褚太傅,不过有些书籍当日我只是匆匆过目,还请殿下允许我借阅几册。‘

李贽微微一笑,道:‘这些你自己作主吧,倒是荆迟,刚才本王在外面就听见你大呼小叫,还是给本王带了东西,是什么啊?‘

荆迟连忙站起道:‘殿下,臣带得这样东西殿下一定喜欢。‘ 说罢从怀中掏出一本图册递了上去。

李贽打开一看,突然神色一震,竟然一页页一直翻了下去,直到看完才惊叹道:‘好全的一本山川地理图,荆迟,你是从哪里得到的,是谁画的。‘

我心生好奇,伸过手去,李贽把册子递给我,我打开一看,上面却都是精工绘制的地图,画的是各处关碍险要,山川水流,画的十分精细,我曾经见过南楚的军用地图,可是也很少见到这样精细的地图。

这时荆迟得意地道:‘末将奉命防备荆襄方面的楚军,各处关卡都得巡视,前些日子抓到了一个青年书生,从他身上搜出了这些图册,原本想把这人当作探子杀了,可是宣参军问过之后,说这人不是探子,而是徐衡的后人徐钧,还是一个难得的人才,所以把他强行留在军中,这人可胆子真大,好不容易拣条生路,居然不肯任官,坚持要走,后来老子火了,说他要是再闹,我就把他当探子宰了,他才老实了,这次本来想把他带来的,可是宣参军说让我先请示殿下一下,这是宣参军的书信。‘说着又递过一封书信。

李贽展开书信看过之后,看了我一眼道:‘随云以为如何?‘

我笑道:‘这人果然是人才,不过现在战乱纷呈,若是留在民间不免遭难,殿下不如把他送到子攸先生那里,反正我看这里还没有幽州的地图,让他专心测绘一下也不错。‘

李贽一笑,道:‘好,本王待会儿就写书信给常青,宣参军名叫宣松,其人虽然沉默寡言,但是精通军务,为人轻财重义,你记得前蜀国狂生杨灿么?‘

我想了一想道:‘臣知道此人,他曾经作为蜀国使者到殿下大营。‘

李贽没有问我怎么知道,只是说道:‘这人倒是一个硬骨头,蜀国灭亡之后,他居然投水自尽,留下遗书说田横有八百壮士殉死,堂堂蜀国怎能没有殉主之人,他死后妻儿几乎冻饿而死,后来就遵照他的遗言写了一封信给宣松,宣松曾经和杨灿有过几句谈话,说过愿意替他尽力的话,最后常青居然就真的派人送了自己全部积蓄给杨家,本王听了也十分敬重于他,那时他刚刚投靠本王不久,本王见他重诺守信就让他做了一名参军,荆迟为人鲁莽,所以就派了宣松给他做参军,看来这个宣松果然值得重用,可惜如今要靠他管理军务,不能调他来长安了。‘

我笑道:‘军务是紧要的,而且荆将军如今到长安护卫殿下,军务若没有值得信任的人托付,殿下也不能放心的。倒是这个徐钧,他既然是徐衡之子,应该是精于地理之人,殿下可要好好重用。‘

这时荆迟赧然问道:‘那个,这个徐衡是什么人,怎么宣参军说起来的时候好像末将理应认得似的。‘

我微微一笑,知道这个将军人如其形,是个粗人,淡淡道:‘这人是前朝有名的地理家,平生喜欢畅游四海,写了很多游记,读书人都喜欢看他写得游记,不出门就可以知道天下风土人情,就是将军也应该看看,知道的多了,就是行军作战也有好处的。‘

荆迟立刻露出为难之色,道:‘末将虽然识得几个字,可是那种文绉绉的书本可是看不懂的,而且事情多得很,哪有时间看书呢?‘

李贽突然神色肃然道:‘荆迟,你就是这样不求上进,你虽然作战勇敢,但是那只能作个将领,你要想将来独当一面,还得多读书,现在你来了长安,本王暂时也不会用你做什么,你就乖乖的多读一些书吧,这是军令。‘

想要诉苦的荆迟立刻住了口,满面的悔恨之色,我不由一笑,道:‘殿下,这些日子我恐怕要劳动两位将军做事,不如就把这件事交给我吧,臣保证让殿下满意。‘

李贽道:‘这倒是好事,荆迟,还不快上前拜师。‘

看着雍王威严的神色,荆迟不得不上前见礼,只是神色间满是苦恼。我和雍王相视一笑,这荆迟性子桀骜,不好管束,我若对他发号施令,他必然不会乖乖听话,如今我用这个法子就可以名正言顺的使唤他,他若不听话,我只要罚他多抄几页书,就能让他俯首听命。

看了长孙冀一眼,他神色淡然,只是目光中有了然之色,看来他十分精明,必然是个好帮手,我的计划应该可以顺利实行了。我由衷地露出一丝喜悦。

五月十二日,长安明德门外,天色将晚,城门眼看就要关了,一个商人装束的中年男子走了进来,虽然是初夏时节,可这个男子却是戴着斗笠,面目在斗笠阴影掩饰下看不清楚,守门的兵卒疑惑的看了这个男子一眼,却没有拦阻,又不是什么紧要时候,没有必要严加盘查。这个男子似乎很熟悉长安的街巷,东拐西转,大约花了小半个时辰的时间,走到了长安西南角的和平坊,这里居住的是最下等的贫民,与其他的贫民居住地里坊不同,这里一到了晚上,除了游手好闲的地痞之外几乎看不到人影,小巷两侧都是贫民的住所,不时的从一些门缝里面传出笑声和吵闹声,那是聚众赌博的地下小赌场和一些暗娼的住处,这里,在黑暗的笼罩下也有着一种畸形的繁荣.

这个男子穿过黑暗的小巷,两边阴暗的灯火将他的身影拖得很长,前面那座荒废已久的大杂院就是他的目的地,轻轻的推开院门,他走了进去,正房内灯火通明,这个男子刚刚走上台阶,从房子旁边的阴暗角落闪出两个人,一个人借着前面的灯笼看了看那个男子摘下斗笠之后的容貌,便悄然退下了。

走进房间,这个男子一眼就看到崔央坐在昏暗的灯光下,他上前施礼道:‘崔大人,别来一向可好?‘

崔央还礼道:‘尚称安泰,霍盟主如今名动天下,当真可喜可贺。‘

这个男子倨傲地一笑,淡淡道:‘这次是你我双方最后一次交易,希望我们善始善终,这是提货的地点。‘说着拿出一个蜡丸,崔央微微一笑,递过一个盒子,说道:‘里面是你们的尾款,今日之后,你我双方互不相关,不过殿下说,若是霍盟主愿意,我们可以保持联系。‘

霍纪城打开木盒,看到里面的金珠,笑道:‘还是太子殿下明理,这些金珠比较安全,否则若是贵方止付了银票,我岂不是白辛苦一场,崔大人,每隔半个月我会派人来见大人,若有什么事情,请大人告诉信使就行了。‘说罢霍纪城转身出去。崔央冷冷一笑,心道:‘殿下已经着手铲除锦绣盟,希望你能够活过今夜再说。‘

没多久一个黑衣人进来禀道:‘大人,我们刚想动手,就发现有人接应霍纪城,只得暂时住手。‘

崔央眉头一皱,道:‘是什么人,你看清了么?‘

黑衣人道:‘不知道是什么人,都是贫民装束,可是霍纪城还没出来,他们就抢占了一些重要的地势,您知道,我们必须等到霍纪城进来之后才能布局,没想到他带了人手来,明明他是一个人进城的。‘

崔央叹息道:‘罢了,我们先回去吧,禀明太子,另行处置,反正我们没有出手,那么就还有机会诱他入伏。‘

就在这时,外面突然传来短促的惨呼声,黑衣人神色一凛,低声道:‘有人偷袭,大人小心。‘ 说罢就要出门,这时房门无声无息的开了,一个黑衣蒙面人走了进来,那人身材不高,一双眼睛如冰似雪。

黑衣人拦住崔央,冷冷道:‘你是什么人,竟敢袭击我们,你可知道我们的身份?‘

那人看了他一眼,身影一闪,黑衣人即时反击,两人在这狭小的空间斗了几招,黑衣人只觉束手束脚,那人却是挥洒自如,不过数招,那人一掌拍在黑衣人的胸口,黑衣人惨呼一声道:‘大搜魂手。‘,声音还没有消散,身形已经跌落,其实那黑衣人的武功并非十分差劲,只是这种狭窄的房间让他施展不开,而他面对的敌手若是在这种狭窄的空间出手,恐怕就是三大宗师也不及他。那人静静的走到黑衣人面前,轻轻撕去他面上的黑巾,将他的相貌看的清清楚楚,然后看了崔央一眼。崔央惨叫一声缩到墙角,颤颤巍巍地道:‘壮士,饶我性命,下官必有重谢,下官是太子内弟,壮士若有需要……‘话还没有说完,那人已经拂袖而去,崔央正在庆幸死里逃生,却只觉得心口剧痛,黑暗向自己笼罩过来,到底是怎么回事,崔央朦朦胧胧的想道。

那人走到门外,几十个贫民装束的汉子默然站立,地上躺着二三十个黑衣蒙面人,那人也不作声,只是一摆手,身影便隐入夜色当中。

霍纪城满怀欣喜的走在路上,他想着是否到长安有名的花楼过一夜,一边想入非非,一边低头疾走,毕竟自己还在人家的地头,走着走着,霍纪城突然站住了脚步,他看到前面站着一个灰衣蒙面人,负手而立,高大修长的身躯带着浓浓的杀伐气息,而两旁黑暗的小巷里也隐隐透着杀气。霍纪城没有回头,他感觉得到后面也站了一个人。想也不想,霍纪城的身躯已经凌空而起,向昏暗的民宅扑去,就在他身形纵起的时候,一声弓弦轻响,霍纪城身形一沉,翎箭擦着他的头皮飞过,霍纪城已经落在一家民宅的屋顶,他一个翻滚向侧面逃去,耳边风声响起,而几个黑衣人已经包抄追来,霍纪城只觉得强劲的掌风拍向自己的后心,他转身出掌,那人似乎一声闷哼,但是霍纪城也不得不身形一慢,其他几个黑衣人的刀剑已经接近了他的身体,双方都没有作声,就在黑暗之中展开厮杀,霍纪城只觉得这些人个个武功不错,尤其是那个和自己对掌的人,武功更是出色,他用余光看到,街上站着一个青衣人,看不到容貌,手里拿着一张硬弓,但见身形修长,气度不凡,就知道这人必是领袖人物,大概是不屑于围攻,所以这人没有出手,霍纪城心中暗暗庆幸,眼睛四处查看,希望找到突围的可能。可是这几人将所有逃生的道路都挡住了,霍纪城一边苦战一边想着计策。

就在霍纪城岌岌可危的时候,突然从阴暗出闪出一个矮小的身影,他抛出两个火红的弹丸,顿时两声霹雳巨响,然后红烟滚滚,霍纪城一见机会来了,立刻向早已看好的方向冲去,这时四周已经有了人声,那几个蒙面人一见不妙,也趁着红烟悄然退走。

霍纪城慌不择路,逃了半天,突然前面闪出一个身影,那人挥手示意,霍纪城认出那人的相貌,心中一喜,连忙跟了上去,那人轻功出众,带着霍纪城东拐西拐,没有多久就到了一处宅院的后门,那人推开后门,回头示意,霍纪城连忙跟了进去,那是一间隐秘的民宅,走进内室,霍纪城疲倦的坐在椅子上,感激地道:‘寒兄,若非你相救,只怕我早就丧命了。‘

那人惋惜地道:‘霍盟主,你太不小心了,太子想要杀人灭口你还想不到么,若非我在外面接应,只怕你早凶多吉少,幸好我让属下准备了烟雾弹,否则我也没有法子救你。‘

霍纪城神色黯然道:‘我没有料到他们这么快就过河拆桥。而且我本以为至少可以逃离,太子也不能明目张胆的围杀我,想不到他的人武功那样高强,皇室果然高手如云。‘

那人叹息道:‘你好好休息一个时辰,我带你出城,长安城墙有几处守卫不严,你轻功出众,可以出去的,只怕明日一早就要有人到处盘查,你今夜如果不走,只怕就来不及了。‘

霍纪城面上露出凶狠的神色,冷冷道:‘多谢寒兄,我不会让太子好过的。我不是那么好欺负的人。‘

在三更时分,霍纪城从一处守卫不严的旧城墙,借助飞爪出了长安,而同时,雍王府的寒园之内,换回了仆人装束的小顺子恭恭敬敬的对我说道:‘公子,猎杀行动已经成功。‘

\chapter{第二十五章 进退两难}

王惑于爱宠萧氏,欲以己罪归于崔央,为鲁少傅谏止,然鲁、萧从此生隙。

--《雍史·戾王列传》

我放下书本,问道:‘荆迟和长孙冀可用心么?‘

小顺子点头道:‘公子放心,两位将军都是恪守军令的人,而且他们武功都很出色,霍纪城是蜀中有名的高手,但是被他们围上也差点丧命,赤骥施放烟雾弹救了霍纪城,自己差点被捉住,如果不是他们事先得到命令,不许泄露形迹,只怕霍纪城根本逃不掉。‘

我淡淡道:‘雍王的爱将,岂是寻常,你上次说大搜魂手你有五成火候,不会被人看破吧?‘

小顺子笑道:‘公子放心,在南楚的时候我曾经和厉家的人交过手,大搜魂手虽然厉害,但是我自信偷学的不错,再说霍纪城乃是厉家破门而出的弟子,他的大搜魂手有些不纯有什么奇怪,只是我不明白,反正崔央也要伏杀霍纪城,为什么公子这样麻烦,要亲自插手呢?‘

我摇头道:‘若是任由太子伏杀霍纪城,他未必有本事掏出来,你也看了他们的埋伏,如果不是你们以雷霆之力一举破敌,哪有那么容易,若是太子伏杀成功,那么就是我们揭穿这件事情也没有什么大用,而且我们若是去救霍纪城,不是寒无计他们露了形迹,就是霍纪城对我们生疑,所以我才这样安排,现在霍纪城逃走了,以他的个性,若是自己吃了亏,宁可吃亏下去也要报复的,只有这样,才能把事情闹大,而我让你杀崔央是为了剪除太子的羽翼,若是事情揭穿,太子原本心目中的替死鬼应该是现任户部尚书,而崔央则可以接任尚书之值,如今‘霍纪城‘杀了崔央,我倒要看看太子是舍弃崔央,还是舍弃那个尚书,只要看看太子的处理方式我就知道现在太子最依赖的是谁了。‘

小顺子问道:‘那么我们下一步该干些什么?‘

我眼珠一转道:‘必须引开太子的注意力,这样吧,我去拜见秦彝秦将军,前段时间因为我遇刺的缘故牵连到秦青,我总要去道个歉的。‘

小顺子不满地道:‘秦青也有嫌疑,他们不来解释已经很过分了。‘

我摇摇头道:‘他们也只能如此,否则这件事情会越闹越大,若是他们来了,你说殿下是相信还是不相信呢,你不见其他几个人不也没有过来解释么,这种事情解释是没有什么用的,就像现在我们已经知道了凶手是谁,不也是只能忍了么,而且这件事情对秦青伤害更大,他本是无辜之人,可是几方面推波助澜,只要知道我并非伤于南楚刺客的人,不是大多都怀疑是秦青不忿于被迫向我道歉而杀我泄愤么。‘

小顺子犹豫地道:‘是否要先去禀告殿下。‘

我笑道:‘那是自然,你以为我一个小小司马,够资格求见大将军么?对了,裴云的情况怎么样?‘

小顺子答道:‘公子放心,裴将军不仅伤势已经全好了,而且内力大有进境,他这次生死相搏,已经突破了界限,有望突破第七层的界限,少林方面也很高兴,因为裴将军已经在日前纳了妾室,那个女子出身书香门第,温柔娴雅,而且族中和少林关系密切,虽然还没有公开,但是这门亲事已经得到裴将军双亲的认可,只要等到这个女子怀了身孕,裴家就会向薛家提出退亲。‘

我讥讽地笑道:‘看来裴将军的父母也等不及了,所以才情愿坏了两家情谊。‘

小顺子忍俊不禁地道:‘秘营传来的情报,说是裴将军的父母其实也觉得未婚的儿媳妇有些太活跃,不能很好的相夫教子,而且裴家如今只有一条血脉,他们恨不得裴将军多娶几房妻子好开枝散叶,原本他们就想让裴将军成亲之后,赶快多娶几个妾室,所以这次裴将军一提出纳妾的事情,他们就同意了,只是碍着亲家的面子,才隐瞒起来,等到妾室有了孩子,就可以名正言顺的接回来,到时候薛小姐就是不情愿也不行了。‘

我也忍不住笑了,说道:‘原来还是裴将军太古板了,如今岂不是两全其美,对了他原本是齐王的属下,这次救了我的性命,齐王有没有难为他?‘

小顺子冷静地道:‘齐王不仅没有为难他,而且还支持殿下的决定,晋了裴将军的官职,太子曾经有过一些小动作,想趁机把自己的人安排到禁军北营,不过皇帝很不满意,亲自嘉勉了裴将军,太子这才罢手。‘

我点点头道:‘好了,殿下现在已经休息了,你明日一早就去问问殿下有没有安排,一定要在明天,要不然就有些迟了。‘

第二天一早,雍王李贽莅临大将军秦彝的府邸的消息很快就传遍了京城。

我和雍王是在辰时末到达秦彝的府邸的,秦宅占地足有十亩方圆,在臣子中已经算的上很大了,不过只是格局广阔,外观倒是很朴实的,就连门前的台阶级数、下马石、狮子都是普普通通,看上去只像普通官宦人家。雍王殿下刚在门口下车,早已经见到雍王信使的秦彝早已经带着家人在门口迎接了,虽然秦大将军地位超然,但是无论如何雍王乃是皇子,该有的礼数是一样不能少的。我跟在雍王身后,偷眼看去,只见秦彝身后除了秦青之外,还有一个高大的青年和四五个十几岁左右的少年,那个青年也穿着武将服饰,相貌和秦青有八分相似,只是显得憨厚一些。除此之外两旁站立的都是一些家将仆人,个个都是气度沉凝,杀气隐藏,看来都是千军万马中血战余生的勇士。

秦彝疾步上前,屈身下拜道:‘臣秦彝叩见殿下千岁。‘

这时秦青和那几个青年少年也都上前拜见,那个青年也有官职,我听他自称秦勇,立刻想起这人的身份,他是秦彝的族侄,他的父亲原是秦彝的族弟,不幸战死沙场,秦彝便将他一家人接到府上,他的祖父母过世都是秦彝安葬,现在是秦彝手下的一名副将,据说此人虽然外貌朴实,却是胸藏锦绣,军法战略都是一流的水准,只是此人对秦彝忠心不二,又是事母至孝,除了有几年在边关历练之外,多年来始终跟随在秦彝身边,是秦彝的左膀右臂,对其的宠爱更在秦青之上。

李贽挽着秦彝的手,两人并肩走进府去,我看了秦青一眼,笑道:‘上次秦将军到雍王府,哲曾经请过将军喝茶,今日哲随殿下来访,将军也该接待我才是。‘

秦青看着我的神情有些古怪,见我说了话才走过来道:‘江司马请。‘

走了几步,他低声问道:‘江司马不是怀疑我秦青是刺杀你的刺客么?‘

我低低笑道:‘秦将军这可是冤枉我了,之前我昏迷了将近两个月,后来又在养伤,哪里有精力怀疑什么人呢,再说将军光明磊落,就是想杀江某,大概也会举剑来杀,这刺杀暗算之事岂是将军所为。‘

秦青眼中闪过欣慰的神色,容颜也不再冰冷,低声道:‘唉,可是害苦我了,爹爹把我关了一个多月,差点没有用刑逼供了。如果不是三哥求情,只怕我现在怕都怕不起来。‘

我扬眉表示不解,秦青指指跟在秦彝身后不远处的秦勇道:‘那就是我三哥,我的远房堂兄,幸好他说得话爹爹听得进去,要不然我可就惨了。‘

我笑道:‘原来如此,不过现在将军还相信那些谣言么?‘

秦青连忙示意我禁声,低声道:‘可别说了,我刚跟爹爹说了那件事情,就被爹爹打了鞭子,爹爹说,公主殿下是什么样的人你还不清楚么,若是殿下肯与人有私情,又……‘秦青说到这里突然住了声,面上露出尴尬的神色,我知道他说露了不该说的事情,便把话题岔开道:‘对了,听说皇后有意把靖江王郡主许配给你,你真是好福气,我见郡主天香国色,正是将军的良配。‘

秦青面上露出古怪的神色,似乎是倾慕,又似乎是惋惜,良久才道:‘郡主确是天人,岂是小子可以匹配。‘

我心中一沉,秦青果然迷惑于李寒幽的美色了,这也难怪,他一个世家子弟,不像裴云那样一般希望妻子勤俭持家,更希望娶到一个出色的妻子,李寒幽既有国色,又是气质不凡,既是凤仪门弟子,才华见识必然也是过人的,正是秦青梦寐以求的妻子人选,想必是秦彝不许,难怪他这种神色,只是这样一来,若给凤仪门拉拢到了秦青,那么秦家的中立就不能保证了。心念一转,我看向秦彝和秦勇,只要让他们明白联姻的害处,那么就行了,我可不信到了生死关头,秦青还会恋恋不舍一个女子,这种基于才貌而生的感情,来势虽然汹汹,但是消散的也会很快的,只要他们没有机会接近,那么很快秦青就会忘怀李寒幽的,可惜公主不肯嫁给秦青,否则……,想到这里,我突然觉得有些心中烦闷,大概是我的伤势没有全好的缘故。

在雍王踏入秦彝府邸的大门的时候,太子早已经得到了崔央身死的消息,最麻烦的是,发现崔央等人的尸体的不是太子的人,而是京兆尹,一个堂堂的户部侍郎,太子姻亲,死在贫民聚集,龙蛇混杂的地方,这就已经让太子万分头痛了,他以为必是霍纪城发觉麻烦,奋力反噬,因而杀死了崔央,在震惊于霍纪城手中力量的时候,如何处理这个残局,就让太子万分头痛了,最后一批走私的货物还没有到手,这个损失已经是很惨重,又让霍纪城逃走,若是这人胡作非为起来,李安想起来就是一阵心寒,不由后悔自己斩尽杀绝的手段,唉,他横了鲁敬忠一眼,若非他说不可留下后患,或许就不会有如今的麻烦了。

鲁敬忠是知道太子迁怒的毛病的,也不当一回事,开口说道:‘殿下,事情虽然发生变化,却也不用烦恼,我们虽然损失了最后一批货物,但总的来说还是不要紧的,而且现在也未必就损失了,锦绣盟扣着这笔货物能怎么样,除了殿下,若是有人能接下这么庞大数量的货物,难道殿下还不会察觉么,到时候臣自有法子,挽回大半损失,目前最关键的是崔央和户部尚书梁谨潜您要保住哪个?‘

李安一皱眉头,道:‘当然是崔--‘ 刚说到这里,李安顿住了,原本的打算是出了事情让梁谨潜抵罪,崔央接任尚书,可是如今,崔央已死,若是还这样做,自己岂不是无人可用,户部尚书不是谁都能做的,资历、品级、能力都要够得上资格,而且户部是他的势力范围,若是用了一个不贴心的人,自己办起事来就得束手束脚。可是梁谨潜私自记录自己的帐目,已经是有了二心,若是就这么放过他心有不甘,最紧要的是,崔央和自己关系密切,他若出了事情,他人势必会将目光放到自己身上,岂不是惹祸上身。

李安正在这里犹豫,夏金逸进来禀报道:‘殿下,兰妃娘娘求见。‘

李安对夏金逸已经颇为信任,尤其是得知副总管邢嵩昨夜身死之后,夏金逸临危受命去了和平坊,将几个奉命前去协助的王府死士的尸身毁去容貌,所有和太子府有关的证物全部毁掉,目击的证人更是该清理的清理,该收买的收买,手段十分厉害,京兆尹虽然心知肚明太子和这事情的关联,可是证据全部毁掉,他又不是蠢人,只能装聋作哑。正是因为如此,李安才决定重用夏金逸,这人虽然没有高超的武功,也没有什么气节,但是既善于逢迎谄媚,又是十分能干,在李安心中,夏金逸已经是接替邢嵩的不二人选了,否则,他跟本没有资格在自己商议事情的时候进来禀报。

他听说是兰妃萧兰来了,连忙道:‘让她进来,刚才孤找她,她也不知去了哪里?‘

片刻,萧兰走了进来,她今年二十六岁,姿色艳丽,品貌出众,做了多年的太子侧妃,养移气,居移体,在清丽雅洁的气质上更添了几分雍容高贵。她走进房内,向李安施礼之后,又向鲁敬忠问好,鲁敬忠早已站起,待萧兰坐下之后,也上前见礼。

李安不耐烦地道:‘天天见面,就别麻烦了。‘ 说着将事情始末说了一遍,然后又问道:‘兰儿,事情已经如此,你想必已经知道了,你说该怎么办才好?‘

萧兰微微一笑道:‘殿下,臣妾若是说出来您可别怪我?‘

李安道:‘你说的都是为了孤着想,就是有些不妥,孤不怪你就是。‘

萧兰淡淡道:‘虽说崔大人是王妃的兄弟,可是他如今身死,就是他原先是殿下的左辅右相,如今也成了弃卒废子,殿下虽然不喜欢梁尚书,可是万万不能自剪羽翼,如今之计,只能把一切事情推到崔大人身上,先笼络住梁尚书,臣妾自会请师门姐妹将梁尚书控制住,等到事情平息,殿下有了信任尚书的人选之后,再了结这人不迟,虽然目前让太子妃受些委屈,可是有殿下庇护,谁能难为她呢。‘

李安听到连连点头,道:‘你说得很有道理,只是这件事情若是牵连到崔央,孤只怕也脱不了干系。‘

萧兰眼中闪过一丝寒光,道:‘所以殿下需得狠心,趁着事情还没有爆发,就说您因为崔央死的蹊跷,因而勘察户部帐目,发现崔大人做了手脚,这样一来,您大义灭亲,谁还能把事情扯到您头上。‘

李安听得眉飞色舞,立刻就要答应,却见鲁敬忠神色不安,心道,莫非他有别的看法,便问道:‘少傅,你认为兰妃的意见如何?‘

鲁敬忠看了萧兰一眼,心道,这女子心肠真是狠毒,这种一石双鸟的计策也想得出来,只是自己却不便当面揭穿,便淡淡道:‘崔央虽然不算什么,可是太子妃殿下是您的结发妻子,又是崔央的亲姐姐,世子与崔大人是舅甥至亲,殿下您若大义灭亲--‘

他没有继续说下去,可是李安已经明白了他的意思,若是自己想要大义灭亲,那么崔氏恐怕必须下堂求去,那么若是有人推波助澜,太子妃这个位子只怕已经有了新主人了,父皇定然会因此不满,认为自己不念结发之情。想到这里,他面色一寒,心道,幸好鲁敬忠提醒了我。

萧兰十分聪明,见太子神色不对,便道:‘我说了殿下不可怪罪臣妾的。‘

太子勉强笑道:‘孤不会怪你,只是这个法子只怕不行。‘

萧兰笑道:‘这有何难,我虽然没有别的主意,但是一会儿我的师妹,靖江王郡主李寒幽要来看我,她也是殿下的堂妹,我早就听说这个师妹十分聪明,殿下不妨问问她,她是我的师妹,难道还会向着别人么?‘

这时,夏金逸叩门而进,禀报道:‘殿下,王妃娘娘派侍女来报,说是靖江王郡主已经到了,就在娘娘房中。‘

李安大喜,道:‘快,去派人请她过来,就说孤有急事寻她。‘

\chapter{第二十六章 靖江郡主}

武威二十四年五月十三日,太宗拜会大将军秦,王闻之,携靖江王郡主与会。

--《雍史·戾王列传》

夏金逸站在门外,无聊的看着远处,唉,为什么我要做太子的贴身侍卫呢,虽然从今天开始,自己已经成了可以和师兄比肩的人物,可是他可是很理智的,自己武功不行,心机也不够深,虽然有些小聪明,可是不会有什么大出息,若是权力太高,能力和地位不符,自己是要栽跟头的,总算他平日待人和善,结交了一些狐朋狗友,要不然想调动人手都会遭到白眼吧,在太子身边几个月,他虽然是如鱼得水,可是他心里总是隐隐的恐惧着一个人,前些日子听说那人受了重伤,奄奄一息,他曾经生出希望那人死去的念头,这样就没有人会盯着自己了,可是就在当夜,出去寻花问柳的他在酒壶里面发现了一枚银戒,上面写着一个‘江‘字,他当时就吓出了一身冷汗,立刻求老天保佑那人长命百岁,至少他不像一个过河拆桥的人。

如今时光匆匆,自己成了太子的亲信,那人也已经脱离险境,直到如今,自己再也没有得到任何他送来的信息,就好像他们从来没有见过面一样,这样子的间谍倒是容易做,只要做自己就行了。可是现在的我是真的自己么,夏金逸微微苦笑,仿佛又回到了少年,那时候,自己是一个孝顺父母,尊重师长,众人赞誉的一个善良少年,突然打了一个激灵,算了,往事如烟,何必再要去想那些不愉快的事情,他不由想起绣春约自己今夜相见的事情,只怕自己会没有时间吧,绣春是个好女子,只可惜身在皇家,身不由己,一个侍女的终身,是不能由她自己作主的,而且现在崔大人出了事情,若是牵连到太子妃,不行,自己应该去给太子妃透个消息,毕竟她是绣春的主子,而且还答应过让绣春自由的。

想到这里,夏金逸心想,等到那位郡主到来之后,肯定至少半个时辰自己不会有什么事情,不妨偷偷的跑一趟吧。不过郡主从王妃那里过来,王妃应该已经知道这件事情了吧?

就在夏金逸胡思乱想的时候,他看到远处走来一个雪衣女子,那绝世的风华,那艳丽的容貌,让人一件心中顿时生出爱慕和自惭形秽的感觉。可是夏金逸却完全没有这种感觉,他浑身突然变得冰凉僵硬,胸中却像有烈焰燃烧,那是一种身在地狱的感觉,他几乎不能思想,如同牵线木偶一般行礼如议,他听见自己的声音说道:‘郡主,殿下和兰妃娘娘、鲁少傅已经在里面等候郡主了。‘

然后他甚至热切的亲手为郡主开门,目光更是带着无比的敬仰,那是一个好色风流却不下流的男子见到绝世美人时候的表现,直到李寒幽走进房间,夏金逸才艰难的说道:‘我有些腹痛,你们先盯着。‘然后他不顾同僚善意的讥讽匆匆向住处走去,好不容易走回那间肃静独立的小屋子,推开房门,他看到一个窈窕的身影坐在床上,是绣春,想必是王妃派她过来的,夏金逸突然扑了上去,两个人的身形纠缠在一起,跌倒在床上,然后帷帐垂落,他的粗暴让绣春发出惊叫,没过多久,他粗粗的喘气和她痛苦的呻吟混合在了一起。

过了一阵子,得到满足的夏金逸松开了手,摊倒在床上,绣春恼怒的支起身子,却惊讶的看到这个平日嬉笑怒骂的男子面上都是泪水,他的面孔抽搐着,狰狞可怖,可是绣春却看得出来,这个男子正处于绝望的悲痛当中,她不顾身子的疲乏,将他抱住,这个男子身子一颤,然后也伸出手将她牢牢抱住,过了许久,夏金逸将她推开,跳下床,已经恢复平静的他梳洗之后,淡淡道:‘崔大人身死之事,太子妃若是知道了,你千万要劝她克制,现在太子殿下正在商议如何处置呢,你让太子妃留心暗算,兰妃娘娘在里面半天了。‘

绣春默默的看着这个给了自己突然的刺激的男子,开口问道:‘金逸,发生了什么事情,告诉我。‘

夏金逸笑道:‘我能有什么事情,殿下正要用我做事呢,你不要胡说。‘说罢,转身走了出去,绣春看着他的背影不由一阵辛酸,她第一次知道这个性子轻浮,油嘴滑舌的家伙也竟然有那么深的痛苦。

走出房间的夏金逸又是一个风流倜傥的俊美青年,甚至看不出一丝他刚才失常的痕迹,他赶回太子秘议之处,却见一个侍卫匆匆忙忙地走来,见到他便喊道:‘夏老弟,你去通禀一声,出了大事情,雍王到了秦大将军府,已经快两个时辰了,还没有出来。‘

夏金逸心中一动,问道:‘雍王是自己去的么,你知道用的是什么理由么,我总不能糊里糊涂的禀报吧。‘

那个侍卫道:‘雍王带着很多护卫,还带了司马雄、荆迟、长孙冀三员大将,和江哲江司马,我们原本以为雍王是去找茬的,谁不知道秦青也在行刺江哲这件事情上插了一脚,原本想等雍王离开之后再来回禀,反正想必他也不会待得时间太长,可是没想到这么长时间没出来,我们在秦府的内线听说他们谈得很高兴,所以我才回来禀报,只怕是有些迟了,夏老弟替我多美言几句。‘

夏金逸笑道:‘你放心,我什么时候为难过你们?‘说着夏金逸再次叩门求见。这次他推门进去的时候,看见太子李安神情有些怔忡,而鲁敬忠和兰妃都沉着脸,只有李寒幽仍然是那样神态优雅。李安不耐烦地道:‘什么事情,不见孤正在商议事情么?‘

夏金逸连忙避重就轻的将事情说了一遍,李安一听到雍王去了秦府,立刻脸色一沉,挥手斥退夏金逸,冷冷道:‘他倒是活跃起来了,看来这阵子父皇的偏袒让他忘了自己的身份了,鲁少傅,你献计离间雍王和秦家,如今他们倒联合起来了,你说该怎么办?‘

鲁敬忠想了一想道:‘这样的发展当时虽然没有想到,可是也不难对付,既然雍王和秦家没有生出嫌隙,那么我们就造出嫌隙来,若是殿下现在陪着郡主去一趟秦府会怎么样?‘

李安心中一动,想起李寒幽和秦青的婚事,虽然还没有得到秦彝的同意,但是父皇和母后都是满意的,如果此事一成,就是秦家想偏向雍王,雍王怕也不会相信他们了,自己可不能让他们走得更近,罢了既然那件事情已经决定,我就先去一趟秦府了,想明白之后,李安站起身道:‘郡主是否肯随本王一行?‘

李寒幽脸上飘过一朵红云,低声道:‘寒幽遵命。‘

李安立刻招呼夏金逸安排车马,他带着鲁敬忠坐一辆车,李寒幽有自己的车子,在路上,李安沉声道:‘这个李寒幽果然是聪明绝顶,竟然想出两全其美的法子,就说崔央发觉有人盗卖军械,故而私下探查,不幸被那些贪官发现,因而惨死,这样一来,崔央声名无瑕,王妃和孤都不用担心被牵连,然后在户部随便找几个替死鬼,就说户部尚书失察,然后太子再担保让他戴罪立功,这样一来,两个人都保住了,日后再徐徐处置,这个主意很是不错,为何少傅和兰儿都不高兴呢?‘

鲁敬忠苦笑道:‘殿下,这个主意虽然是两全其美,但是实际上支持的是为臣,崔央的声名保住了,那么太子妃和世子的地位稳固如山,那么兰妃娘娘自然不会高兴,她请了同门的师妹过来,本来是想助自己一臂之力的,没想到郡主却支持敬忠,所以娘娘才会气恼,臣之所以不快,却是因为这李寒幽心智过人,她表面上调和,却是让我和兰妃娘娘心生嫌隙,我想郡主一定会跟娘娘说,我是太子心腹,不能和我作对,她们同门姐妹,很快就可以达成谅解,到时候臣就是众矢之的,郡主如此心机,怎不让臣担忧,殿下,凤仪门可结之以援,不可受其控制,若非李寒幽此举是凤仪门主之命,臣倒要阻止她和秦青的婚事了。‘

李安皱皱眉,道:‘可是如今若不如此,怎能打压老二的气焰,户部的事情马上就要发作,若是老二趁机发难,只怕户部就不再是我的天下了。‘

鲁敬忠叹息道:‘臣也正是因此为难,殿下这几日就要揭发户部不法情事,殿下掌管户部,出了这种事情,虽然可以解释的过去,但是皇上心里不免有些恼怒,所以如今殿下得依赖她们打压雍王,等到风平浪静之后,才来想办法吧,其实拉拢到秦家也有好处,只可惜又让凤仪门占了便宜。‘

李安犹豫地道:‘李寒幽也是皇族,总不至于过分偏向师门的。‘他的声音有些充满了不自信。

鲁敬忠苦笑道:‘殿下说得是。‘面上却现出意味深长的古怪神色。只是一心想去破坏雍王拉拢秦家的太子却没有留意。

今日秦彝可是荣宠备至,正在他和雍王在后园欢宴的时候,家人来报,太子殿下驾到。秦彝微微苦笑,想不到自己一向洁身自好,却成了两位皇子争斗的导火线,不论他如何想,也只能率众前去迎接。

李安走下车驾的时候,看见秦彝和雍王匆匆走来,两人上前下拜道:‘臣李贽、秦彝叩见太子殿下。‘

李安伸手虚扶道:‘二弟和大将军不要多礼,今日孤来此却是陪着郡主前来拜会大将军和秦夫人的,想不到二弟也在这里。寒幽,来拜见大将军。‘

随着李安的声音,从另一辆华车走出一个身穿雪衣罗裳的绝丽女子,她走到秦彝面前,飘飘下拜道:‘寒幽拜见大将军,家父多次提及当年和将军并肩作战的事情,前些日子,寒幽代父亲送来的微薄礼物,却被大将军婉拒,想是将军恼怒寒幽拜会来迟,实在是寒幽近日一直在宫中陪伴皇后娘娘,还请大将军恕罪。‘

秦彝神色淡然,微笑道:‘臣和王爷确是袍泽情深,只是皇命在身,王爷镇守在外,秦某在京中伴驾,故而多年未见,郡主心意,秦某心领,前些日子拒绝郡主的礼物并没有什么理由,只是除了皇上赏赐之外,秦某是从不接受他人礼物的,郡主多心了。‘

当下众人来到了后园,秦彝已经让人重新换上酒菜,李安坐在首席,抬目望去,这秦府的后园与众不同,没有什么奇花异草,亭台楼阁,却是把诺大的一块空地平整之后,铺上青石板,四周种上树木,成了一个小校场,场地上摆着兵器架、石锁之类的东西,而在校场一角,更摆着几面战鼓,如今春光明媚,秦彝就在校场外面的大树下摆上酒席,让家将武士在校场上比武助兴,方才正是最热闹的时候,雍王麾下的侍卫和秦府的家将都下场比武,胜的人赏酒一爵,败得人也不会收到责罚,都是军旅出身,没有那么多心机,雍王和秦彝也不会因此生出争斗之心。

可惜李安的到来让这里的气氛不免有些冷淡,秦彝让家将散去,又让人请来秦夫人相陪郡主,总算这里人人都是惯了官场的人,倒也风平浪静。

这其中有几个人,都忙着在闲谈之时打量对方的动态,鲁敬忠一边附和着太子,一边有意无意的注意着雍王司马江哲,这人始终悠闲的和秦青、秦勇谈着什么,雍王麾下的三位将军也在旁边跟着讨论,鲁敬忠竖起耳朵听去,却是什么兵法战策,山川地理之类,这些他并不擅长,而秦夫人正和李寒幽谈笑,李寒幽落落大方,很得秦夫人好感,原本秦青一直在听江哲他们谈话,但是没过多久,他就明显神思不属,目光屡屡落到李寒幽身上。而太子、雍王、秦彝正在谈得热烈,秦青渐渐开始有些放开胆量,开始和李寒幽谈天,秦夫人似乎乐见其成,不时的替他们穿针引线。

李寒幽虽然表面上专心讨好秦夫人,应付秦青,但她双目的余光却始终落在江哲和站在他身后的小顺子身上,她早已经得到了师门的情报,这个看上去形容有些瘦弱憔悴的青年在南楚的作为的情报她已经看过了,谁会知道这个以文才著称的青年,用得计策是那样狠毒,平定蜀中,离间大雍,若非德亲王已死,这人只怕会给大雍带来更大的损失,可惜凤仪门直到雍王将他俘虏回大雍之后,才注意到他,详查之下,才发觉这人乃是旷世奇才,为了剪除雍王羽翼,门主亲自下令让自己刺杀此人,可惜自己竟然失败了。

至于那个李顺,李寒幽心中顿时生成无力的感觉,论年纪,自己比他还要大一些,论出身,自己的恩师乃是三大宗师之一,可是这个少年的武功竟然超过了自己,根据自己得到的情报,这个少年武功远在自己之上,自己门中除了门主之外,恐怕只有六七个长辈可以胜过他,最令自己不平的是,这么一个武功高强的少年,竟然甘心做那手无缚鸡之力书生的奴才,你看他此刻乖顺听话,完全是一副训练有素的奴才形相,真让人怒其不争,这种高手若是为我所用,李寒幽叹了口气,这人偏偏是个残疾之身,凤仪门的‘神凤心法‘全无用处。

秦青见李寒幽叹气,不由问道:‘郡主为何叹息?‘

李寒幽心中一动,道:‘妾身也听父王说起过一些军旅中事,可惜父王不许我参与,秦将军和诸位几乎都是沙场血战余生的名将,不知道可否给妾身讲一讲战场上的事情呢?‘

秦青笑道:‘郡主是凤仪门弟子,可惜却是宗室,不然想上战场也没有什么难处,末将虽然也曾经沙场血战,可惜这些事情若是说出来,未免有些煞风景。‘

李寒幽见秦夫人面上有些不豫之色,连忙道:‘我可不是想听那些杀伐之事,只是听说大漠烽烟如画,蜀中风光绮丽,南楚更是风月无边,不知道这些地方风光比起大雍来,哪里风光更动人呢?‘

李寒幽的声音虽然不高,但是人人却都听得很清楚,都不由思想了起来,这些人大都见识广博,李寒幽说得这些地方他们没全到过,倒也去过大半,但是若说哪里风光最盛,这却难道了他们,就是心中觉得某处最好,空口说来也觉得没有证据。

李安虽然不知道李寒幽目的何在,但本着同仇敌忾之心,说道:‘这倒是一个好题目,我们今日闲来相聚,尽谈论些军政大事,未免有些沉闷,不如就说说自己的见闻,倒也不错,不如我们就以此为酒令,每人说出一个风景胜地,却需有前人诗词为证,若是说不上来的,就罚饮酒三杯。‘

\chapter{第二十七章 指点江山}

会中,郡主笑问天下风光,王附议,乃行令。令未起,齐王已至,三王欢聚,席间其乐融融,当其时也,浑忘萧墙之乱将至也。

--《雍史·戾王列传》

‘哈哈,好主意,这可不能把我拉下呀。‘太子刚刚说完,远处传来一个爽朗的声音,众人看去,却是齐王李显悠然走来,他身后却是韦膺和夏侯沅峰,太子和雍王神色都是一变,秦彝却是苦笑连连,他万万想不到,今日他的府邸这般热闹,给夫人使了一个眼色,秦彝站起身道:‘今日是吹了什么风,齐王殿下也来到寒舍,秦某真是受宠若惊。‘

李显拦住秦彝施礼,笑道:‘说来也巧,大将军可能不知道,我和夏侯原想一起出去游玩,谁知路上遇到韦大人,听大人说起今日大哥和二哥都到了大将军府上,我就想,这样的热闹我怎能不凑呢。‘

李贽和太子都放下了心,他们知道李显平日就是没事也要找事的,今日这样热闹,他不来倒是奇怪呢。

众人重新落座,三方面倒是泾渭分明,这时秦夫人已经告辞离去了,所以太子、齐王、李寒幽、鲁敬忠坐在一处,雍王、江哲、司马雄等人坐在一起,而韦膺、夏侯沅峰和秦彝、秦青坐在一起,秦勇已经托词离去了,这里聚集了这么多贵人,他们的属下侍卫定然是很多的,秦勇这是去打理了。

李安命人取了几坛子烈酒上来,又取了大酒觞来,这种酒觞一杯就能装下四两酒,若是喝了三盏,就是酒量不错的人也不免醺醺然。他笑道:‘今日酒令严似军令,不知让谁来掌令呢?‘

荆迟连忙站起来道:‘末将不通文字,还是我来掌令吧。‘

李贽笑道:‘胡说,这掌令之人需得熟读诗书,你怎能掌令。‘

李显眼珠一转道:‘我们人人都要行令,大将军是武将世家,家中若是寻个武技高强的家将到处都是,若是寻个熟读诗书的人只怕难了,既然是郡主提议,不如让郡主掌令吧。‘

李寒幽嗔怒道:‘妾身一个弱女子,岂能掌令,谁不知道你们行令的规矩,那掌令之人是要陪酒的,不论行令之人胜负,都要陪饮一杯,你是怕寒幽不醉死么?‘

李显摊手道:‘这样啊,不如我们替郡主找个副掌令,只用喝酒就好。‘

众人面面相觑,谁有人酒量不错,但是做李寒幽的副掌令,未免有些尴尬。

这时,李显突然道:‘这样吧,你来吧。‘说着指向一人。

众人看去,李显指得却是江哲身后肃手而立的小顺子,虽然小顺子只是一个仆人身份,但是在场的人谁不知道这人乃是绝顶高手,大概也只有江哲这种人敢把他当成奴才使唤,否则就是太子、雍王也会把他奉为上宾。

李寒幽心中大喜,她原本只是想借机探一下江哲的虚实,若是能够得到他的好感就更好了,想不到突如其来的齐王这般配合,把小顺子放到了明处,自己就可以趁机施展手段拉拢这两人,至少也要减轻他们的敌意。若非齐王名分上不占优势,李寒幽还真想建议门主,支持齐王比起支持太子那个蠢人容易多了。

小顺子原本的全部注意力都集中在江哲身上,至于其他的人在他眼里则是分成‘对公子有威胁的人‘和‘对公子没有威胁的人‘两类,李寒幽则是有威胁的一类,想到就是这个女子差点杀死了公子,他很早就想一掌杀了她,若非江哲低声对他说道:‘不用着急,来日方长。‘他早就忍耐不住了。

现在听到齐王的建议,小顺子神色一变,眉宇间立刻带了冰寒刺骨的杀气,那双眼睛更是射出冰冷的寒光,令众人都不由提高了警惕,这时江哲悠然道:‘这也是一个好主意,只是小顺子酒量不高,替郡主挡酒也是十分辛苦,若是郡主肯重重赏赐,那么就是他不动心,臣也会动心的。‘

李贽神色一松,道:‘这倒也是,不过既然是大哥提议行令,六弟推荐小顺子襄助,那么两位也不应该吝啬吧。‘

李寒幽露出纯洁无瑕的笑容,道:‘妾身来得匆忙,若是不嫌弃,就把这个做为赏赐吧。‘说着从腰间取出一双薄如蝉翼的手套。众人看得奇怪,不由互相询问,这时候夏侯沅峰笑道:‘郡主果然厚赐,这一定是天山冰蚕结丝织成的手套,刀枪不如,百毒不侵,正是擅长掌法之人最喜欢的武器。‘

李寒幽看向小顺子,只要他神色微动,自己就算达成目的,谁知小顺子只是淡淡看了一眼,说了一句‘谢郡主赏赐。‘神色丝毫没有变化。

李寒幽心中一叹,若是小顺子见此欣喜,那么就说明他的境界还不能摆脱外物的诱惑,那么自己就知道他的深浅,而且他若是越依赖这副手套,那么他的武功就越难进步,可惜,只见他这般冷淡,就知道他不是已经知道这个道理,就是已经过了依赖外物的境界,他既无名师教诲,那么就说明他的武功已经到达了那个境界。

齐王笑道:‘本王身上可没有带什么好东西,这样吧,本王府里有一套《梦华录》,是本王无意中得到的古版,上面是一些失传已久的乐府诗词,本王这件礼物可珍贵么?‘

小顺子神色有些改变,他服侍江哲多年,曾经听过江哲说过这本书,而且似乎还很遗憾没有看到过,不由露出喜色,道:‘谢齐王殿下赏赐。‘

李寒幽等人一愣,心道莫非这人不喜欢武功反而喜欢书本,接着便看到江哲面上露出一丝隐隐约约的喜色,李寒幽心中又喜又忧,看来这个小顺子的唯一弱点就是江哲了,只是这样一来这个高手就不可能为自己所用了,毕竟以雍王对江哲的重视,若是江哲肯归顺了自己,自己大概也不敢用他。罢了,看来只有用雷霆手段了。李寒幽眼中闪过一丝绝决。

太子李安连呼倒霉,心道还我赏赐东西,李寒幽和李显的礼物都是很贵重别致的,若是自己赏了金银珠宝之类,未免有些俗气,他正在犹豫不定,一直站在他身后的夏金逸突然附耳说了一句话,李安顿时眉开眼笑,道:‘本王的赏赐你也不可推辞,金逸,你就绿珠和翠莺明日送过去,这两个女子乃是本王心爱的舞姬,你可要尽情享用。‘

这句话一说出来,空气中仿佛带了阵阵的寒意,虽然没有人明言,但是小顺子的身份大家却是心照不宣的,若他是一个平常之人,这种略带嘲弄的赠与,他也只能忍了,但是小顺子却是一个绝顶高手,若是他一怒出手,那么这里恐怕没人可以脱了干系,不仅太子和齐王留在身边的几个亲信护卫提高了警觉,就连雍王,秦彝和雍王麾下的几个将军也都小心翼翼的注意着小顺子的举动。

却见小顺子不怒反笑,身影一闪,已经站在了太子面前,太子大惊,而李寒幽、齐王、秦彝都同时发动,却在丈许外站住了,只因小顺子明明站得很远,却是第一个到了太子面前,而且太子也没有收到伤害,只因站在太子身后的夏金逸已经挡在太子身前,若是小顺子出手,必然不能一举杀了太子,这样一来,他们自然不会贸然出手。

李贽也站起身来,看了江哲一眼,道:‘李顺,你要作什么?‘

所有的目光都落到江哲身上,这时候大概只有他能喝止小顺子了。

我无奈地看了一眼神色焦急的雍王等人,我开口道:‘臣代李顺谢谢殿下赏赐,殿下必然是觉得他平日劳役繁重,这才送了两个侍女替他分忧吧。‘

李安此时真是有些后悔,夏金逸原本让自己送两个出色侍女,可是自己一时兴起,居然送了两个舞姬,而且语气中暗含讥讽,却惹祸上身,这人虽然离自己还有数步之远,但是李安只觉的从他身上传来丝丝的寒气,一听到江哲开口,连忙道:‘是啊,你武功高强,总是作些下人的工作,本王觉得说不过去。‘

小顺子突然露出淡淡的笑容,施了一礼,十分恭敬地道:‘多谢殿下赏赐。‘

众人都松了一口气,李寒幽心道,还好,这个小顺子还有其他的弱点,她却不知道方才我和小顺子都已经察觉到她的试探和瞩目,偏偏齐王的礼物让小顺子流露出最大的弱点,就是我,所以我故意露出喜色,其实那本书虽然不错,但是也不至于让我连喜色都不能掩饰,我的意思是让人从我这里着手,我有小顺子和雍王的保护,应该不会有问题,可是小顺子很快就发觉了,所以借着太子的讥讽,他故意大怒,似乎忍不住要出手,这样一来就会让人以为他的修养不够,就不会特意针对我了,我知道他的心意,但也只能任由他这般做,毕竟在他心里,我的安全比什么都重要。

等到小顺子退回我身后,李寒幽笑道:‘我们这酒令应该开始了。‘其实众人已是全无兴致了,可是既然已经约定了,自然就要进行下去,而且也都存了比较的意味,所以这次气氛有些紧张的酒令就开始了。在酒令开始之前,韦膺含含糊糊的说了一句话,很多人都没有留意,我却听得清清楚楚,他说道:‘今日真是精彩呢,这些人凑到一起的钩心斗角比什么戏文都好看。‘我不由心中苦笑,什么时候我也成了别人眼里的好戏了,从前我可是一直是看戏的人啊。

这时,李寒幽笑道:‘这个酒令的规矩不难,就是先说一个地名,然后便需要说上几句诗词,若是说的贴切,本令就认可,若是说得不好,那就罚酒三杯,咱们也不能学人家击骨传花,就由我这个令主指定次序吧,不论名位还是辅议先后,都以太子殿下为先,就请殿下先来吧。‘

李安已经心情平定下来,他贵为太子,诗词就算不精通,读也读过几首,便开言道:‘长安--早夏宜春景,和光起禁城。祝融将御节,炎帝启朱明。日送残花晚,风过御苑清。郊原浮麦气,池沼发荷英。树影临山动,禽飞入汉轻。幸逢尧禹化,全胜谷中情。‘众人拍手称好,我也是其中之一,但心中却想,此人喜爱的诗文少了几分天子气,看来果然是没有九五之命。李安饮了一杯,李寒幽也略略沾唇,而小顺子却也得尽饮一杯。

李寒幽笑道:‘太子之后,当是雍王殿下。‘

李贽道:‘幽州--塞草连天暮,边风动地秋。无因随远道,结束佩吴钩。‘说罢自己饮了一杯。

我心中明白,雍王殿下引用的诗句全篇乃是‘黄阁开帷幄,丹墀侍冕旒。位高汤左相,权总汉诸侯。不改周南化,仍分赵北忧。双旌过易水,千骑入幽州。塞草连天暮,边风动地秋。无因随远道,结束佩吴钩。‘这分明是向太子表示自己只想作个一路诸侯,虽然太子肯定不信,但是却让别人挑不出毛病来。

下一个轮到齐王,李显微微一笑,道:‘晋祠--步屐深林晓,春池赏不稀。文章千古事,社稷一戎衣。野日荒荒白,悲风稍稍飞。无由睹雄略,寥落壮心违。‘

我把玩着酒杯,心道:‘原来齐王心心念念的都是平定北汉,想来只有和北汉悍勇的骑兵交锋,才是他心中所想,这人倒是有自知之明,知道自己没有帝王之份,便一心一意想做一个大将军,可惜他陷入皇位之争,只怕终究是空怀壮志可。‘我看向齐王,眼色中满是惋惜,却见李显也向我望来,神色间带着难言的疲惫。

秦彝淡淡道:‘洛阳--步登北邙阪,遥望洛阳山。洛阳何寂寞,宫室尽烧焚。垣墙皆顿擗,荆棘上参天。不见旧耆老,但睹新少年。侧足无行径,荒畴不复田。游子久不归,不识陌与阡。中野何萧条,千里无人烟。念我平常居,气结不能言。‘

别人听了也还罢了,只道是秦彝怀念故土,他们都知道秦彝是洛阳人,李贽却是听得入神,忍不住道:‘洛阳果然已经如此荒芜么?‘

秦彝也不作声,只是默默饮了一杯酒,李贽叹息道:‘洛阳乃百战之地,多年兵祸连绵,致令民生凋敝,我当进言,请父皇重修洛阳才是。‘

李安听了不满,心道,何用你多嘴,我难道不知道进谏父皇么,若非你和我争夺帝位,我早就用心处理政务了。心中这样想,面上却不露神色。

接下来按照官职身份,却是轮到夏侯沅峰,他微笑道:‘西湖--月冷寒泉凝不流,棹歌何处泛归舟。白苹红蓼西风里,一色湖光万顷秋。‘

旁人都道夏侯选的诗文优雅,我却是淡淡一笑,这人心机深沉,机巧灵变,就连吟诗也不忘遮掩性情。若非那日他上门承认救走毒手邪心一事,我怕也看不穿此人面目呢,也会只当他是个风流公子呢。

接下来,鲁敬忠道:‘长沙--三年谪宦此栖迟,万古惟留楚客悲。秋草独寻人去后,寒林空见日斜时。汉文有道恩犹薄,湘水无情吊岂知。寂寂江山摇落处,怜君何事到天涯。‘他念得抑扬顿挫,目光却斜到我身上,除了不通诗文的荆迟、司马雄之外,人人都露出尴尬的神色,谁都知道鲁敬忠是在讥讽我,指我纵然才高八斗,也没有明主赏识,自然在他心里雍王是不可能成为皇帝的,而且贾宜因梁王胜坠马之死而自伤为傅无状,哭泣而死,鲁敬忠词意歹毒,竟是诅咒我这个楚客也会失去辅佐的雍王,我便是另外一个贾宜,贾宜三十三岁而死,看来鲁敬忠也不会让我活过那个岁数呢。

雍王眼中闪过一丝深恶痛绝的寒光,他倒不是恼恨鲁敬忠诅咒自己,既然身为敌人,别说是诅咒,就是挥刀杀向自己也无可厚非,但是鲁敬忠诅咒江哲早亡却让他心中怒火汹涌,因为江哲自从遇刺之后,身体十分羸弱,他经常担心我会病故,所以特别气愤鲁敬忠的行为。他正要发作,我却已经笑道:‘鲁少傅说得好,哲也十分欣赏贾宜,若是有机会去长沙,定要去瞻仰他的故居呢?这一杯江某也相陪少傅。‘说罢,我饮下了杯中酒液,苍白的面容上顿时泛起血色,小顺子定定的看了鲁敬忠一眼,眼中闪过一丝杀气。

鲁敬忠心中略略有些后悔,自己不该这般无礼,但是自从此人进了雍王府,他总觉得自己用计不再一帆风顺,心中久已郁闷,此番忍不住讥讽江哲,一半是泄愤,另一半却是因为他颇通医术,见江哲体弱气虚,倒希望将他气死呢。

韦膺见气氛不好,便开口道:‘也该轮到我了,终南--终南阴岭秀。积雪浮云端。林表明霁色,城中增暮寒。‘他说完便饮了一杯,这么一打岔,气氛有些好转。我心想,这韦膺果然是丞相家教,不愧是韦相之子,这首诗秀雅清新,只可惜终究是不脱富贵荣华,终南捷径,天下皆知啊。

接下来该轮到几个将军了,他们除了长孙冀之外都是面有难色,就在这时,突然有人匆匆走进,是秦府的家将,他看了一眼座上众人,面有难色,走到秦彝面前低声耳语了几句。秦彝身躯一震,挥手斥退了家将。就在这时,几个不同势力的人几乎同时闯进校场,却是太子、雍王、齐王各自的侍卫,我听得清清楚楚,他们说的都是一件事情,就在方才,有人袭击了军部在渭水的军械库,烧毁了那里的所有军用补给,而且留下了标记,那标记是一匹南楚的小寒绢,素白如雪的寒绢之上,用鲜血写着‘锦绣盟‘三个大字。

一时间,太子、雍王和齐王都要起身告辞,李寒幽故作不知这个变故,起身道:‘别人要走也可以,总的等江大人行过酒令才行,江大人南楚才子,怎能这样就走。‘

我心知她设了圈套,我若是说喜欢南楚,她就会诬陷我不忘故国,我若是喜欢大雍,她又会讽刺我不念旧情,这我早就想明白了,所以听到她的指名,我只是淡淡道:‘善鼓云和瑟,常闻帝子灵。冯夷空自舞,楚客不堪听。苦调凄金石,清音入杳冥。苍梧来怨慕,白芷动劳罄。流水传萧浦,悲风过洞庭。曲终人不见,江上数峰青。--哲曾闻洞庭君山湘妃祠,常有人听见夜半琴瑟,每思一见而不可得,今日以此作为酒令,不知可否。‘

李寒幽柳眉轻颦,江哲所选诗词,鬼气森森,却又意犹未尽,不可揣测,只得嗔怒道:‘江大人说得好。‘慢慢饮了少许酒液,虽然李寒幽每人只陪酒少许,但是秦府的烈酒醇厚无比,此刻她已经是面带红霞,更显得美丽绝伦,她这般轻颦浅嗔,更是美不胜收,就连急匆匆要去料理麻烦的太子、心中忧虑的雍王也不由失魂落魄。秦青更是愣在那里,眼中只剩下那个绝丽的倩影。

\chapter{第二十八章 姻缘成双}

武威二十四年五月,帝赐封靖江王郡主为公主,赐婚抚远大将军秦彝子秦青,或曰,皆王之力也。

——《雍史·戾王列传》

在回去的路上,雍王沉着脸道:“随云,你放心,日后我定然为你杀了鲁敬忠。”

我淡淡一笑道:“殿下为何恼怒,理应高兴才是,鲁敬忠长于攻讦,疏于自保,他为太子出谋划策,虽然是一步三策,但是三策难成一策,这不是一件好事么,再说,覆巢之下,焉有完卵,此人臣还不放在心上,臣关心的是李寒幽,此女心智真是过人,一举一动都能牵动人心,这次胜利的是她呢,秦青只怕逃不出她的手心,就是殿下,不也是几乎动心了么。”

李贽不由赧然道:“随云说笑了。”

我神色不变,道:“此女既是凤仪门高弟,又是宗室贵女,难得的是心机深沉而不外露,形容举止不带骄矜,秦青移情于她也是情理中事,我想若非秦大将军托词拒绝,只怕此事早就成了,殿下可要当心她,她若是嫁给秦青,地位越发崇高,只怕将来阻碍殿下大业者,就是此女。”

李贽忧心地道:“若是秦青真的娶了她,那么只怕有些不妥,虽然秦大将军公正严明,但是若是婚事真成了,那么……”李贽没有说下去,但未尽之意我已经了然,挥手道:“殿下放心,秦青虽然是大将军长子,但是却不能有效地影响大将军,父子之情虽然可以潜移默化,但是大将军为了家族着想,定然是不会和殿下为难的,而且秦勇乃是大将军亲信,此人若是能够拉拢过来,殿下就可以无忧了。”

李贽道:“秦勇对大将军忠诚不二,如何能够拉拢过来呢?”

我笑道:“这件事交给臣吧,现在殿下若是加以招揽,反而会让太子和大将军不满,臣有法子让他不知不觉的替殿下效力。”

李贽点头道:“这件事情交给你了,还有一件事,这锦绣盟如此嚣张,父皇必会派将领前去剿平,本王想推荐长孙冀,他精明能干,武功高强,正是首选,你觉得怎样?”

我说道:“殿下,长孙将军若是去办此事,就是找到了什么关联太子的证据,只怕也会受到怀疑,若是臣所料不差,太子也会推荐齐王去办这件事情,毕竟太子殿下麾下没有能征善战的将军。”

李贽道:“若是这样,岂不是得不到任何证据了。”

我笑道:“其实太子和殿下都太着急了,锦绣盟偷袭军资,这只是小小的叛乱,若非太子做贼心虚,怎会急着派人前去围剿,殿下原也不用主动招揽此事,现在太子所作所为还没有揭穿,若是殿下揭穿此事,不免让皇上怀疑殿下的动机。太子推荐齐王去办这件事情是欲盖弥彰,崔央之死已经惊动朝野,户部的事情正在将发未发之际,若是太子推荐齐王,殿下不妨说这等小事用不着大将,就让秦青去吧,若是秦青立了功,皇上赐婚也是理所当然了。想必皇上也会觉得中立的秦家比较合适吧。我想这个人选皇上不会拒绝的。”

李贽眼中神光一闪,道:“而且还可以离间秦家和凤仪门,若是凤仪门执意阻挠,这桩婚事自然是绝对不成的。”

我笑道:“不论是什么结果,对殿下都没有好处,说句实话,臣真没想到锦绣盟有这样手段,霍纪城虽然个性乖戾,但是却是果决之人,怪不得锦绣盟至今不能彻底剿平。”

李贽叹息道:“随云,本王只觉得这世间之事无不在你掌中,幸好你终究保了本王,要不然本王真是寝食不安。”

我的面上却露出萧瑟之色,李贽惊道:“随云怎么了,莫非本王说错了什么。”

我微微一笑,抛却心中怅然,心想我虽然心有顽疾,但是只要我安心静养,未必不能长命百岁,虽然这钩心斗角只能令我损折生命,但是数年之内我必然能够保雍王登上皇位,到时候天下之大,难道我还找不到可以休养的地方么,“湖水绿盈盈,昔人自兹去。时闻棹歌声,扁舟不知处。”我低声吟诵道。李贽笑道:“这是谁的诗,怎么这般逍遥。”

我随口道:“这是臣从前在书上看到的,也不知道是谁写的了,殿下,想必皇上很快就会召见殿下了,殿下还是快回去更衣准备吧,臣答应教导荆将军,就不要让他跟您去了。”

李贽笑道:“好啊,本王要看看随云怎么把这个顽劣弟子教成材。”

我也笑道:“若是我让他自己作诗一首,不知道殿下给臣什么赏赐?”

李贽想了想道:“本王一时也想不出来,金银珠宝你不喜欢,图书典籍你也都看过了,若是随云想要什么,不妨说出来,本王都可以拿来做奖赏。”

我恭谨地道:“殿下言重了,只是有一件事,上次殿下因为臣遇刺的事情大发雷霆之怒,因为毒手邪心是江南春介绍来的,殿下虽然没有查封江南春,但是却令京兆尹严查,这些日子以来,江南春一日也不得安宁,臣请殿下开恩,饶恕了臣的表弟。”

李贽立刻想起了这件事情,当日他愤怒欲狂,若非管休提醒荆舜卿是江哲表弟,只怕就要把江南春查抄了,但是受此牵累,江南春这段时间可是度日如年,后来江哲保住了性命,李贽却又将这件事情忘记了。这时江哲提起,李贽不由有些尴尬,连忙道:“本王是忙得忘了,其实我已经查过,令表弟并未涉入刺杀此事,本王这就派人去通知京兆尹一声。这算不上什么赏赐,这样吧,你若能教会荆迟作诗,本王就将这个赏你。”说着从腰间解下一块翠玉佩,在手中把玩。

我知道这块玉佩是御赐之物,正是玉中上品,何止千金,这也算是极其贵重的赌注了。便笑道:“臣也知道了一段时间,原本臣也想表弟受点教训,让他知道‘谨慎’二字,可是前两天我那位显德的弟媳来哭诉,她一个弱女子背井离乡已经是可怜非常,如今又抛头露面前来求恳,我总不能不给她面子。”

见我微笑,李贽道:“若是随云你输了,可要拿什么做赌注呢?”

我想了一想,道:“臣受殿下赏赐极多,若是拿不出什么特别的东西,倒显得没有诚意了,这样吧,若是臣输了,愿意将此物作为赌注。” 说着我指指腰间的玉带。

李贽疑惑的看过来,这玉带虽然看起来好看,可是不过是衣带上缀了一些羊脂玉带板,虽然华美,但是既没有精工雕刻,也不是上品美玉,怎值得拿来做赌注呢。但他当然不会计较,便道:“这样也好,就赌你的玉带吧。”

我微微一笑,现在还没有必要告诉李贽,这条玉带是我近日设计的,前两天刚刚才由小顺子取回来。里面设计了机关,可以连续三次射出淬了剧毒的毒针,这是为了保护自己所准备的,若是再有刺客到了我的身边,我还有反击的机会,这条玉带何等珍贵,再说,还是“天机阁特制”呢。

说话之间,我们已经回到了雍王府,刚走到门口,就看到远处飞马奔来一个御前侍卫,他手中拿着敕令,见到殿下就下马拜倒,说道:“殿下,陛下召您入宫。”

李贽连忙道:“待本王更衣之后立刻进宫。”

那个侍卫站起后退,说道:“遵命。”

我这个闲人回到寒园,跟着我的除了小顺子还有一个不情不愿的荆迟,他好不容易从那令人头疼的酒令中逃生,恨不得找个地方练几趟拳脚,活动一下筋骨,想不到却被我召进了寒园。

荆迟郁闷的望着江哲的背影,再一次在心中叫起苦来,这个文弱书生居然得到殿下的令旨管束自己,他平日就不喜欢这些写诗做赋的文人,雍王殿下麾下虽然有不少文臣,但是大多都是擅长军略的谋士,这些人荆迟倒是敬鬼神而远之的,可是这个书生也没见他出谋划策过,整天躲在寒园,多走几步路就气喘吁吁的好像要断气一样,可是雍王对他真的很信任啊,让自己等人听从他的命令。昨夜江哲派他去做那件莫名其妙的事情,荆迟到现在还是有些糊里糊涂的呢。

跟进寒园,虽然跟着雍王出去,但是保护江哲的侍卫也都去了,他们熟练的迅速占据了要害地点,寒园很快就成了固若金汤的堡垒,荆迟撇了撇嘴,心道,这人真是一朝被蛇咬,千年怕井绳。这时荆迟突然觉得身上生出寒意,抬头一看,却看见一双冷若冰雪的眼睛,正是小顺子瞧见了他的神情,用目光警告他。荆迟心中一凛,他可是很佩服小顺子的武功的,连忙低下头来。

回到居室,我召进荆迟,拿起一本孙子兵法,放到书案上,淡淡道:“你若能将这本书背了下来,我就放你出去。”荆迟目瞪口呆的看着那本薄薄的书册,脸上泛起苦涩的笑容。

我笑道:“我身子不好,今日就不能教你了,从明天起,我每天教你半个时辰,剩下的时间,你要抄写我教过你的内容,我知道你认得几个字的,对了,你就在旁边的房间背诵抄写吧,小顺子会监督你的。好了你去把行李搬到寒园来。”

荆迟大骇,正要拒绝,却看到那个文弱的书生眼中透出坚决的神光,不由自主地道:“是,大人。”

荆迟离开之后,小顺子不满的问道:“公子何必还要为这种粗人耗费心思,不如多多休养才是。”

我淡淡道:“看此人面相性格,最是忠心直率,果决勇敢,若是能够学些军法,殿下多一员忠心不二的将领,这人最是重情义,日后我也可得到一个保障,何乐而不为呢,你吩咐下去,秘营要派人接近秦勇,最好能够得到秦勇的信任,骅骝、绿耳都是隐组中精选出来的,让陈稹选一个去做这件事情。锦绣盟会把事情闹得很大的,我们也该收手了,其他的人如何安排,你们斟酌,但是秘营至少要留下一半人供我使用,另外一半人安排他们隐入民间,日后我们还用得着他们。”

小顺子默默点头道:“我今夜就去办,我们早有准备,不会很麻烦的,公子,李安和鲁敬忠太无礼了,若非公子有大事在身,我早就取了他们的性命了。”

我淡淡道:“日后他们的性命,必然让你去取。”

当夜我得知经过御前争论,果然秦青被任命剿灭锦绣盟,太子虽然极力支持齐王出马,但是雍王说得有理,一个小小的锦绣盟,当然用不着一个王爷去剿灭,而秦青这个人选十分符合皇帝的心意。雍王回来之后告诉我,他已经发现太子和齐王都私自派了人出去,至于凤仪门是否派了人,雍王暂时还不知道,不过想也知道他们不会闲着的。

接下来我就忙着调教那个笨徒弟,虽然他一看见书本就昏昏欲睡,不过总不会比当年的陆灿顽劣,所以我也就一边讲书,一边讲一些实例引导他的兴趣,荆迟虽然开始听得很无聊,可是很快就被人引起了兴趣,他虽然不懂军法,可是常年在军中作战,很快他就根据自己的经历向我问难,就这样一问一答,教学相长,过了两三天,荆迟已经是听得兴致勃勃了,每天一大早就在外面眼巴巴的等我起床,下午我逼他抄书,他居然也坚持了下来,虽然看到那些墨迹淋漓的文稿让我哭笑不得,可是我总算在半个月里粗粗的讲了一遍孙子兵法,不过可能是性情的缘故,用间这一篇他始终不大明白,我也不强求,孙子兵法博大精深,就是我也不敢说全部精通,何况是此人呢,想起昔日给陆灿讲孙子兵法的时候,他问一知十,十分聪明,只是他有些固执,大概是出身世家的缘故,我说起九变那一章的时候,他总是有些疑惑,而且他虽然军略不凡,可惜对于人心险恶了解的太少了。

这一日,我命荆迟默写全书,看他挥汗如雨的书写,我再次后悔让他在我的书案上书写,罢了,眼不见心不烦,我闭着眼睛躺在软榻上,渐渐的进入了梦乡。突然远处传来急促的脚步声,我猛地睁开眼睛,只见李贽浑身怒火的走了进来。

我微微一笑,道:“小顺子,给殿下上茶,让殿下消消怒气,有什么大不了的事情,让殿下这样子呢?”

李贽一见江哲,不知怎么胸中怒火渐渐消退,再看到愣在那里的荆迟,脸上还有墨迹,书案上也是一团混乱,不由噗哧一声笑了出来。坐了下面,接过小顺子递上的香茶,喝过之后最后一丝怒气也不见了。荆迟连忙起身告辞,我笑道:“你可不许偷懒,到隔壁去写,若是偷看书本,别怪我罚你抄书抄到白头。”

荆迟连忙信誓旦旦的发誓不会偷看,他可不会忘记上次我考他的时候,他偷看书本,被我罚他整整抄了十遍孙子兵法的情景,那晚他可是一晚没睡觉啊。

荆迟出去之后,我淡淡道:“发生了什么事情?”

李贽冷冷道:“秦青果然有才干,锦绣盟贪功,多次骚扰,被秦青故意放纵,然后一网成擒,虽然逃出了一些余孽,但是已经成不了大气候了,可惜霍纪城还是逃了,这虽然有些美中不足,但是也还说得过去。被俘的锦绣盟叛逆都招认了和户部官员勾结走私的经过,秦青倒是正直公正的,全部取了口供,连同人证送到了皇上面前,不过这些只是旁证,因为没有人可以指证太子和崔央,毕竟霍纪城逃走了,所以太子在父皇面前推的一清二楚,说是户部官员上下其手,而崔央奉他之命查实走私传闻,却被锦绣盟暗杀,父皇最后只让他闭门自省,户部尚书梁谨潜原本要被免职,但是被韦相劝止,让梁谨潜暂时戴罪立功,这些都还罢了,父皇接着下旨赐婚秦青和李寒幽。”

我问道:“皇上偏袒太子已非一日,殿下不必担心,如今只怕天下有志之士都已经知道了太子的真面目,这已经是达到目的了。不过李寒幽和秦青的婚事怎么这么快,大将军没有反对么?”

李贽叹息道:“李寒幽原来私下跟着秦青去平叛,又从锦绣盟刺客手上救了秦青的性命,两人同进同出,就是大将军想要阻止也不成了,宗室郡主的清誉岂可玷辱,父皇还特意封李寒幽为靖江公主,秦青如今是驸马都尉了。”

我叹了口气道:“这件事情殿下不是早有准备么,为何如此愤怒。”

李贽沉默片刻,道:“父皇下旨,将长乐公主赐婚韦膺。”

我手一抖,手中折扇落在尘埃,过了片刻,我俯身捡起折扇道:“这也不是什么大事,皇上想为公主择婿,殿下早有所闻,韦膺人品才华都是非同寻常,更难得是性子平和中正,殿下不也看好他们的婚事么?”

李贽苦涩地道:“随云,你真的不知道皇妹的心意么,她为什么平日带着你的诗词朝夕不离,她将千年难遇的玄参给你续命是为了什么,王妃带着柔蓝去看她,她爱如己出又是为什么,你真的这般不解风情么,你可知道皇妹听说父皇赐婚,居然自剪长发,要出家为尼,引得父皇震怒,父皇问她,只要她开口,不管什么人都可以召为驸马,可是长乐一言不发,如今被父皇软禁在后宫,随云,你若有心,本王拼了一切,带着你入宫求婚,你,可愿意么?”

\chapter{第二十九章 残月暗影}

南楚同泰元年五月,雍帝以长乐公主许韦膺,膺贤良以闻,然主断发明誓,不肯屈从,帝暴怒,良久乃息,密语公主道:“儿若有心,无论贤愚贵贱,朕即许之。”公主唯默然。

——《南朝楚史·江随云传》

残月如钩,夜空沉谧安宁,站在御阶之上,长乐公主一身素衣,默默看着幽淡的月光,这些日子她原本已经渐渐丰盈的容貌上带着哀伤,如墨青丝剪断了一半,如今只到肩下,已经有宫女替她修整过,只是太短了,因而不能挽髻,只是用一条丝巾系住,夜风清凉,长乐公主衣着单薄,已是彻骨之寒,她却始终不肯回房。这样的夜色,他可也中宵难寐。良久,她举起素手,手中是一把折扇,上面写着一首绝句。

“冷于陂水淡于秋,远陌初穷见渡头。赖是丹青无画处,画成应遣一生愁。”她低低吟诵,这是前些日子雍王妃带给她的,这是那人手书的折扇,雍王妃知道她喜欢江哲的诗词,便讨来送给了她,江哲或者不在意一把诗扇,可是她自从得到这把折扇便片刻不离,再也不愿割舍。

这时,绿娥走来,拿着一领披风,恳求道:“殿下,您这般自苦,奴婢看了都不忍心,若是陛下和娘娘知道,定然要责备奴婢不好好伺候的。”

长乐公主微微一笑,接过披风道:“本宫哪有那么娇贵,只是喜欢这夜色,好了,你先去休息吧,本宫稍后就会去睡了。”

绿娥见公主神色还好,便壮着胆子问道:“公主,奴婢不明白,夏侯大人文武双全,又是俊美非常,您不中意也就罢了,毕竟人人都说夏侯公子风流倜傥,公主性子温和善良,若是不喜欢也没有什么奇怪,但是韦大人不仅品貌过人,而且洁身自好,从无风流韵事,公主却连他也不中意,真叫奴婢费解。”

长乐公主淡淡道:“你懂得什么,好了,去吧。”

绿娥心中一凛,只见公主秀雅的面容上带着若有若无的冷笑,那种皇室特有的威严让绿娥不敢再多说,蹑手蹑脚的退了下去。

长乐公主微微叹息了一下,觉得夜冷阶凉,绣履冰寒,正要转身回寝殿休息,夜风中传来若有若无的叹息之声,长乐公主黛眉微蹙,道:“什么人在那里窥探本宫?”

翠影一闪,一个身穿绿色宫装的女子站在长乐公主面前。长乐公主微微一怔,道:“原来是寒幽妹妹,怎么这样夜半三更来翠鸾殿拜访本宫。”

李寒幽飘飘下拜道:“妾身今日与秦将军订婚,可是想起姐姐深宫寂寞,不由心中不安,所以特意前来,果然姐姐还没有安寝,不知道姐姐能否请寒幽进去说话。”

长乐公主淡淡道:“妹妹这些日子常常前来相陪,长乐感激不尽,只是今夜夜深人静,不便叙谈,妹妹还是回去休息吧。”

李寒幽柳眉微蹙,转念一想,心道,听说前两日雍王妃进宫,莫不是长乐公主知道了我在秦府为难江哲的事情,长乐公主喜欢江哲的诗词,无人不知,而且听说长乐公主心仪的正是那人,现在看来,果然有些可能,否则上次见面还是亲亲热热,怎么如今却冷若冰霜,若是如此,自己必要问出真情才行,否则事情岂不是失去控制。

想到这里,李寒幽笑道:“久闻殿下喜欢南楚才子的诗文,前些日子寒幽在秦府有幸见到了江状元吟诗,虽然不是状元做的诗,但是有句话说得好,所谓诗以言志,姐姐不想知道其中详情么?”

长乐公主心中一动,前两日雍王妃入宫,无意中说起诗会之时,只是当时母妃在侧,自己也没有机会细问,便说道:“愿闻其详。”

李寒幽便略过众人钩心斗角不提,只是说了众人选得诗文,最后念道“曲终人不见,江上数峰青”的时候,突然长乐公主微微一笑,李寒幽心知有异,故作不知,继续道:“江大人选的诗是极好的,这最后两句最是意味深长,公主觉得江大人眼光如何。”

长乐公主笑道:“江大人选此诗多半是敷衍了事,我看他的诗词,或者幽远典雅,或者气势磅礴,或者情深意重,或者平和恬淡,却偏偏没有这种鬼气森森的作品。”说着不由看向手中诗扇。

李寒幽心中一动,道:“姐姐的扇子,可肯借给小妹看看么?”

长乐公主看了李寒幽一眼,道:“事无不可对人言,这把扇子是我从二王嫂手上抢来的,江大人这几年来很少有诗词流传,手迹更是少见呢。”说着递过扇子。

李寒幽轻轻念了一遍上面的诗文,只觉得淡淡的忧伤扑面而来,诗词清雅隽永,不由道:“江大人诗才果然天下无双。”

长乐公主取回诗扇,笑道:“江大人的诗词四海皆有流传,若是妹妹喜欢,不妨寻找一下。”

李寒幽见长乐公主面上有淡淡的喜色,突然问道:“殿下为何严拒赐婚,莫非是已经有了意中人么?”

长乐公主却神色不变,她淡淡道:“妹妹可知父皇为何急于替我择婿?”

李寒幽裣衽道:“陛下爱重公主,天下皆知。”

长乐公主淡淡道:“天家骨肉,亲情淡薄,父皇虽然宠爱我,却不是因此急于为我择婿,当日父皇遣我远嫁,心中常耿耿于怀,因而我若不能嫁个如意郎君,父皇总觉得对我不起。”

李寒幽眼中神色一变,道:“姐姐可怨恨陛下么?”

长乐公主摇头道:“我自始至终心中全无怨恨,长乐自幼喜好诗文,性子更是柔顺,不似我大雍女子那般坚强果决,若非父皇母妃爱宠,长乐只怕全无地位可言,所以父皇命我远嫁,我虽伤悲,却不埋怨,再说,本宫受天下百姓奉养,怎能不为大雍牺牲,故而我虽远嫁敌国他乡,注定今生夫妻不能白头,更是亲手杀了我未出世的孩儿,长乐心中却从来不曾怨过大雍,怨过父皇,如今父皇要我嫁人,自然是希望我能够幸福,可是本宫历经风霜,早已对情爱二字心灰意冷,只愿父皇母妃膝下尽孝,日后不论那位兄长登基,我一个孀妇弱女,想必也不会遭到猜忌,到时候长乐相伴青灯古佛,此生足矣。”

李寒幽叹息道:“殿下,您莫非还是念着南楚国主么,他不过是庸碌之人,您何必为他守节。”一边说着自己都不相信的话,李寒幽一边用悲切的目光看着长乐公主。

长乐公主淡淡道:“没有这回事,本宫只是心灰意冷,并非每个人都想仿效文君的。”

李寒幽道:“文君本是淑女,只是遇上了才华横溢的司马相如,这才情不自禁,若是江大人有心求凰,不知道公主意下如何。”

长乐公主深深的看着李寒幽,眼中多了一丝冰寒,李寒幽笑道:“皇上已经说过,若是公主愿意,不论什么人都可以做驸马。江大人才华盖世,若是公主心许,小妹愿意替殿下转告皇上。”

长乐公主目光更加冰寒,冷冷道:“李寒幽,本宫在大雍深宫多年,又在南楚身为王后,虽然是深居简出,但是你以为本宫真的一点心机也没有么?”

李寒幽神色大变,裣衽下拜道:“殿下息怒,小妹实在是一片赤诚,不忍见公主愁锁双眉,若是有冒犯之处,寒幽情愿领罪。”

长乐公主神色更加幽冷,缓缓道:“江大人品性高洁,若非二王兄这般人物,寻常人绝难劝他归降,虽然如此,他也不是俗人可以轻辱的,本宫爱他才华,敬他人品,岂可任你等曲解本宫心意。我知道,如今朝中之事,错综复杂,只是长乐本已是局外之人,你为何执意将本宫牵扯进去,靖江公主,本宫虽然不喜欢争斗,可是你们若是再苦苦相逼,本宫只得禀明父皇,即刻出家为尼,到时候也算是遂了你们的心愿。”

接着,长乐公主冷冷道:“本宫累了,靖江公主请回吧,更深露重,你可要小心在意,若是再有这般流言,本宫可是要请父皇和母后娘娘作主的。”

李寒幽目送着长乐公主的背影,双目闪出一丝懊悔,她万万没有料到,平日温柔可人的长乐公主竟然也有如此凛然不可侵犯的一面,自己若是再搅和下去,若是激怒了公主,愤而出家,只怕没有人能够说服陛下平息雷霆之怒了,如今,只能暂时放手了。

匆匆回到住处,只见纪贵妃神色凝重的在等她,她连忙上前道:“师叔,您怎么来了,可是有什么重要的事情么?”

纪贵妃道:“门主传来令旨,霍纪城已经被你大师姐缀上,必然不能逃生,可是如今朝野议论纷纷,必须谨慎处理。”

李寒幽喜道:“大师姐武功绝世,必然能够手到擒来。”却又柳眉微蹙道,“可是,太子之事如何可以挽救,防人之口,甚于防川,门主总是这样教导弟子,不知道师叔可有什么法子。”

纪贵妃道:“你也不用过于忧虑,你斩草除根做得彻底,现在没有人有证据可以证明太子涉入此事,如今梁尚书已经是笼中之鸟,只要过些时日,再将他杀了灭口,就可以了,掌控户部之事虽然紧要,可是咱们不能触动太子的警觉,所以不能明着来,只怕想法子安排进去几个咱们的人就行了,这尚书一职至关重要,还是由太子去折腾吧,倒是和锦绣盟合作的一方可能也知道一些事情,你可有眉目。”

李寒幽苦笑道:“怎么说呢,和锦绣盟合作的是南楚的天机阁,可是他们的势力是在暗处,如今已是踪影全无,就连他们掌控的商行也已经全部脱手,这个天机阁真是神秘莫测,我们在南楚的势力又不够强大,实在是鞭长莫及。”

纪贵妃淡淡道:“这件事情你记着就行了,门主说,若是不能不让人说话,那么就让他们转移视线,既然太子出了这种事情,我们就得让别人也出点事情,越打越好,这样谁还会记得太子的事,就是记得,只要我们扶保太子登基,谁还敢提这件事情,当务之急是不能失去皇上的宠信,长乐的事情,你还是不要插手了。”

李寒幽道:“弟子遵命,我已经有了主意,请师叔放心就是。”

纪贵妃幽幽道:“你是门主心爱的关门弟子,我怎会不放心呢,你好好做事吧。你大师姐虽然得到门主真传,但是门主毕竟还没有选定继承人,你若是功劳够大,我必然在门主面前替你进言。”

此言一出,李寒幽眼中闪过一丝狂喜,但是她很快就恢复平静,裣衽道:“多谢师叔美意,寒幽对大师姐十分敬重,不敢有这等妄想。”

纪贵妃微微一笑,道:“好了,你去办你的事情吧。”

看着李寒幽的背影,纪贵妃淡淡一笑,久在这名利场中,她早已知道不论什么人,若说能够轻易抛却名利的诱惑,都是一句假话,名利权势,富贵荣华,岂是可以轻易放弃的,就是不爱金钱名利,那种手握权柄,一言既出,天下俯首的威仪,却是更加令人心醉神迷,这世间有几人能够抵制这种诱惑呢。

霍纪城伏在草丛当中,屏住呼吸,不敢稍动,他此刻心中似烈火焚烧,不停的诅咒着属下无能,没有好好执行自己的计划,轻而易举的就被秦青大军剿灭,最可恨得是凤仪门,她们已经追了自己一天一夜,如果不是自己擅长隐匿行踪,只怕早就丧命在那个女子手上了。

突然,霍纪城看见黯淡的月光下,突然出现了一个身影,那是一个青衣女子,衣衫朴素,正是大雍平民女子最爱穿的样式,霍纪城本是蜀人,对于锦绣布料颇有鉴赏能力,一眼就看出这女子身上衣衫并非是有名的裁缝所制,倒像是擅长织布裁衣的女子自己所作,若是往常看到,霍纪城只会以为这个女子不过是个乡下村姑,可是此时此地,却让霍纪城心中生出寒意。这时,那个女子点燃了火折,火光明灭,映射出一张平平无奇的女子面孔,这个女子相貌仅是中人之姿,可是眉宇间那种冷淡平静的神情,却让她的形象立刻多了几分莫测高深。霍纪城心中一跳,他已经想起凤仪门的传闻,据说凤仪门主因为自己相貌绝美,所以所收的弟子都要有相当的姿色,其中只有一人例外,这人就是她的首座弟子闻紫烟,闻紫烟是凤仪门主青年时候收得弟子,不仅相貌平凡,而且资质也非上乘,可是这个女子的坚忍毅力当真令人倾服,居然得到了凤仪门主的真传,今日也不过三十岁,据说已经有了凤仪门主七八成的武功,当年凤仪门辅佐李援东征西讨,闻紫烟就是凤仪门主最得力的助手,可谓转战天下,满手血腥,直到大雍平定中原之后,这个女子才归隐凤仪门,轻易不外出,据说凤仪门的弟子武功倒有大半是她替师父传授的。

霍纪城擅长一种心法,可以将心跳呼吸变得十分低微,此刻他仿佛是无生命的石头一般,他感觉到这个女子正在聚精会神的聆听四周的风吹草动,所以一口大气也不敢喘。

良久这个女子似乎有些失望,挥手灭了火折,身形消失在夜色当中,又过了半个时辰,霍纪城才轻轻的移动了一下身子,活动早已麻木的四肢,他运气调息了片刻,看看天上的星光,判断一下方向,天机阁曾经有消息传来,如果他能够到达前面三十里之外的一处农舍,那么就可以把他送出大雍国境,霍纪城觉得精力恢复了许多,再次开始了行程。

暗夜行路,本是十分艰难之事,霍纪城又是惊弓之鸟,一路上瞻前顾后,终于在黎明时分到了那座小农舍,这座农舍十分偏僻,四周渺无人烟,霍纪城暗中监视了半天,没有发觉有埋伏,这才上前拍门,门开了,两个十五六岁的少年看到霍纪城都是露出喜色,霍纪城走进农舍,边看到寒无计的身影。

寒无计看到霍纪城,叹息道:“霍盟主,你为何如此固执,第一次你偷袭成功,我便劝你赶快撤走,你却始终不肯,如今锦绣盟遭到重创,你该如何是好。”

霍纪城赧然道:“都是我那些手下撺掇我,这才落了圈套,不过没关系,锦绣盟还有小半人手在外,我只要找到他们,不过三年五载,又可以东山再起,只是那批货物,就要拜托你替我出手了。”

寒无计笑道:“何必要替你出手,我以五成价格买下这批货物,你带着银票离开,岂不是胜过两手空空。”

霍纪城惊喜地道:“寒兄是说真的?”

寒无计道:“我怎敢欺瞒,这些货物我们慢慢出手就是,总不会亏本的,霍兄你却要重整锦绣盟,无钱怎么能行。”

霍纪城长揖到地道:“多谢寒兄厚谊,霍某若有翻身之日,必然不会亏待寒兄。”

寒无计笑道:“不敢,多个朋友多条路,我也是秉承阁主的钧旨。好了,霍兄,你先沐浴更衣,我已经准备了酒菜,你饱餐之后,换上我给你准备的衣服,易容改装,全套的身份文件我已经准备好了,你就可以大摇大摆的离去了。”

霍纪城担忧地道:“可是凤仪门的闻紫烟紧追不舍,她如何肯罢手。”

寒无计笑道:“盟主放心,我已经准备了一具尸体,可以代替盟主,尸体就在隔壁房间,等盟主走了,我就火焚农舍,做成盟主遇害的假相。”

霍纪城心中一动,道:“先让我看看替身可否相像。”

寒无计手一指一扇小门,霍纪城走进去一看,里面一张木床之上果然有一具尸体,身材和自己十分相似,他这次放心,看来天机阁没有准备落井下石。

换过衣服,狼吞虎咽的吃饱了肚子,霍纪城又喝了一杯茶,只觉得浑身上下酸疼难忍,想来是一夜奔波太辛苦了,恨不得睡上一觉再说,可是追兵在后,霍纪城只得道:“看来我得走了,这里太不安全了。”

寒无计微微一笑道:“对不住,霍盟主,你哪里都走不成了。”

霍纪城大惊,就要跳起,却觉得双腿酥软,居然动弹不得,他惊骇的望着寒无计,道:“你们也要出卖了。”

寒无计冷冷道:“同为蜀人,我们虽然没有致力复国,可也不会残害自己人,你担任锦绣盟主,害死了多少不愿同流的蜀人,你的罪行,罄竹难书。”

霍纪城怒道:“这些和你有什么相干,你可是从我这里得了不少好处的。”

寒无计淡淡道:“是的,我们却是仰赖你不少,可是今日你穷途末路,我们却不愿被你连累,你知道我们天机阁的一些事情,而且,阁主早就下了谕令,要抢在凤仪门之前杀你灭口,绝不能让凤仪门知道你身后还有我们主使。”

霍纪城心中一凛,想起这些日子自己虽然春风得意,可是所作所为当真是大半由他建议,莫非自己竟然做了别人的棋子么?他自视甚高,想到这里,不由愤怒如狂,目眦欲裂。

寒无计微微一笑,道:“霍盟主,九泉之下,见到韩章夫妻别忘了向他们请罪,还有我家公子托我转告,柔蓝小姐一切都好。”霍纪城眼中闪过一丝明悟,道:“你们是替韩章报仇的。”

寒无计不再多说,从袖中滑出一把匕首,轻轻一挥,割断了霍纪城的咽喉。当这个肆意横行的男子生命终结的时候,他的眼中仍然是带着不平和切齿的愤怒。

寒无计拿出一个玉瓶,将里面的粉末倒在霍纪城的身上,一阵令人心惊胆寒的滋滋声音之后,霍纪城的尸身化成了一摊清水,只留下衣衫鞋袜和一些零碎物品,寒无计淡淡道:“山子、渠黄,你们将东西收好,我们也该走了。”

当寒无计带着两个少年清除所有线索,一把火烧了农舍离去之后不久,闻紫烟也赶到了此处,她早就发现了有人掩饰霍纪城的行踪,造了不少假痕迹将她引入歧途,可是终于被她抽丝拨茧发觉了霍纪城真正的行踪,可惜却来迟一步,只找到一具烧焦的尸体。她眼中射出莫名的寒光,自此检查了那句尸体,因为来的及时,所以尸体大部分还没有烧毁,只是面容早已经烧焦了。闻紫烟冷冷一笑,这具尸体只看手足就知道非是练武之人,霍纪城想要金蝉脱壳还要看自己答不答应。

\chapter{第三十章 杀人灭口}

雍王府寒园之中,我披着锦袍坐在凉亭当中,园中春花已谢,树木郁郁葱葱,精致秀雅,我今日清早起来赏玩朝阳,小顺子担心我受寒,还是坚持我披了锦袍,我看着初升的旭日和满天的朝霞,心中却只有一个念头,一个人而已。

小顺子见我缄默,四下的侍卫也已经被打发走,便走近我身旁,淡淡道:‘公子还是为了雍王殿下所说之事烦心么?‘

我轻轻一叹,道:‘小顺子,你说,长乐公主真的对我有意么,为什么我从没有感觉。‘

小顺子轻笑道:‘公子你从未和年青女子接近,每日不是看书就是赏玩风景,你和夫人之间也是夫人先主动你,公主殿下性情端庄贞静,从来没有表白心意,也难怪公子不知,我看公主对您有意是肯定的,否则就不会日日把玩公子的诗词,南楚之事我想公主已经知道了一些实情,可是也没有别人知道,再说若非公主的半枝玄参,公子也早就性命不保,若说公主对你没有情意,我是不相信的,不过公主大概也和您有同样的心思,所以才从来不肯表白。公子你对公主不是也颇有不同么,这些小顺子也都点点滴滴看在心里。只是你们两个都碍于君臣名份,所以才不肯互表情衷吧。‘

我淡淡的看了小顺子一眼道:‘你是责我为声名所累,不肯接受公主的情意么?‘

小顺子默默不语,显然是默认了,我叹息道:‘我江哲岂是爱惜声名之人,只是有些事绝对不可以做,我上次回答秦青的责问没有一句假话,我和公主名分有别,可是我并非因为这个原因拒绝这桩婚事,若是我真的情有独衷,那么没有什么可以阻止我,可是你应该知道,公主没有说过一句要嫁给我的话,这说明公主就算对我有意,可是她绝对不愿违背礼法,既然如此我怎能顺着雍王的意思求婚,这样一来便坏了公主声名,虽然碍于皇室威严,可能没有敢明言,可是笔墨无情,我不想公主青史之上留下污名。再说,我和公主仅有数面之缘,怎知公主是不是真心爱我这个人。‘

小顺子低声道:‘公子说得是,是奴才误会了。‘

我淡淡道:‘这些还是我从私情来说,若是从公事上来说,我一个南楚降臣,凭什么求娶公主,恐怕就是雍帝当面答应,转眼就派人赐死来了,雍帝虽然是任凭公主选择,可是他心中恐怕只想公主嫁给大雍的俊杰吧。而且我若此时作出这种事情,只怕连累雍王,我岂是以私害公之人,再说,我的身体你还不清楚么,若是有什么不幸,你让公主情何以堪。‘

小顺子没有作声,半晌才道:‘奴才只是希望公子不会终身孤独。‘

我微微一笑道:‘等保着雍王登基,报了杀妻之仇,我就把一切都放下来,到时候我若身体好转,就娶一个贤淑女子为妻,你说好不好。‘

小顺子笑道:‘那当然是好的,奴才等着您娶主母,然后添个小主人呢。‘

我松了一口气,倒在椅子上道:‘雍王这几日应该也想通了,所以不会来逼我,对了,外面的情况如何?‘

小顺子神色古怪地道:‘公子是想听好消息还是坏消息?‘

我苦笑道:‘先听坏消息吧。‘

小顺子道:‘这个坏消息就是京城出了一件大事,如今人人都去看热闹,却没有留意太子的举动了。‘

我眉头微蹙道:‘是什么大事,让朝野都转移了注意力呢?‘

小顺子道:‘这件事情原本是件江湖事,公子知道关中联吧?‘

我道:‘记得,联主沙青元,其女沙芷菁乃是凤仪门弟子,是长安最大的帮派。‘

小顺子道:‘说起来,公子和他们有过一面之缘,前些日子沙芷菁到咸阳探望外祖母,却被人给杀了,据说,据说死得很凄惨,凤仪门的女弟子就是武功再差,也练过一种叫做‘陨玉搏杀术‘的武功,这是凤仪门主亲自所创的武功,全是根据女性身体柔韧的特点所创的,近身搏杀无所不用其极,这是那些女弟子在遇到强敌而又逃生无路的时候所使用的,就是不能致敌人于死地,也能同归于尽,再不济也可自杀,现在想来当初还真是可惜,梁婉因为不敢轻举妄动,恐怕伤害到公主,所以没有施展这门武功,总之沙芷菁一死,关中联和凤仪门都是全力缉凶,而凶手连连造成惨案,咸阳一带这十几天又死了不少闺中女子,而凶手已经露了形迹。‘

我说道:‘既然如此,凭着凤仪门和关中联的势力,应该很快就将这个人抓住处死吧。‘

小顺子摇头道:‘原本凤仪门因为主力未到,这个人在咸阳一带肆虐无忌,如今凤仪门人手到了,这人却已经逃之夭夭。‘

我皱眉道:‘这件事又怎会引起朝中群臣的注意呢?‘

小顺子苦笑道:‘死去的女子都有被采补的迹象,所以江湖中人怀疑魔宗的人重入中原,当年魔门宗主京无极拜走大漠的时候,魔门弟子也随之而去,就是没有离去的也都隐姓埋名。魔门其中一支‘怜香派‘就是最擅长采补的,若是魔宗重现,说明京无极可能会重入中原,如今他已经是北汉国师,他的复出可能象征着北汉即将大举进攻,若是如此,朝中文武怎能不关心此事,所以现在没有人还记得锦绣盟的事情了。‘

我下意识的摇着折扇,问道:‘你的看法如何?‘

小顺子道:‘我不认为魔门弟子留在中原有什么奇怪,若是没有我才觉得奇怪呢,而且魔门的人行踪隐秘,这些年虽然不时传出有他们的行踪,可是都是捕风捉影,所以我觉得凤仪门有可能借题发挥,引开众人注意力。‘

我冷冷一笑道:‘魔门的势力已经依附了北汉,京无极要想和梵惠瑶比个胜负,想要凭着武技是没有什么意义的,恐怕这天下一统才是他们胜负的关键。太子一出事情,魔门就出现了,还真是会赶时间。既然如此,我就凑凑热闹,小顺子,你知道现在户部尚书梁谨潜在做什么。‘

小顺子道:‘他现在戴罪立功,但是雍王殿下的情报,太子正在安排接收他的势力,梁谨潜已经被软禁了。‘

我微微一笑道:‘霍纪城的事情寒无计办妥了么?‘

小顺子笑道:‘这正是我要告诉公子的好消息,霍纪城已经消失,可是留下一具假尸体,欲盖弥彰,如今凤仪门和太子到处追杀他,可惜却不见他影踪,秘营已经送来了霍纪城的信物。‘

我站起身道:‘那么你去做一件事情,去杀了梁谨潜,不用动手,用鸩杀,这样一来,你说大家会怎么想?‘

小顺子神色古怪地道:‘自然是太子杀人灭口了,公子此计真是歹毒。‘

我笑道:‘这正是我的打算,我还有事情交给你,霍纪城不会死的,他虽然身死,可是他却会活在他人心中。这也是我报答他让我得了百万金银吧,你可不能辱了他的声名啊。‘

小顺子忍着笑道:‘公子放心,我定要让霍纪城成为太子的梦魇。‘

我嘱咐道:‘小心些,你若是被揭穿身份,我可就糟了。‘

小顺子正色道:‘放心,打不过就跑,我绝对不会让他们逮到的。‘

我还是有些担心,不过想到我会安排荆迟他们接应,这才放下心来。正要再嘱咐几句,就是小顺子笑我罗嗦也认了,却听到远处的脚步声,只听声音就知道是雍王来了,他应该是来致歉的,我总要给他一个台阶的。挥手让小顺子退下,我等着雍王前来。可是雍王面上却带着一种难言的哀伤,我心中一动,问道:‘殿下为何这样难过?‘李贽苦涩地道:‘今日皇妹执意离宫,到无尘庵清修,父皇和长孙贵妃劝阻不住只得应允,只是不许她剃度出家。‘我心中一震,再也说不出一句话来,看向雍王热切的眼神,我淡淡道:‘殿下,姻缘不可强求,公主一心求佛,或许那才是她可以平安喜乐之处吧。‘

李贽微微叹息了一下,道:‘不说了,只要皇妹不剃度,将来总有转圜余地的,下一步我们该怎么办呢,秦青和靖江公主的婚事,太让我失望了。‘

我笑道:‘殿下不用忧虑,唯今之际,还请殿下多多优礼秦家,否则他们投入了太子一方,才是不妙,我想秦大将军不会这么不智的,秦家还有几位小公子的。‘

李贽眼睛一亮,没有说话,我知道这些事情他比我知道该如何作。这时,我看见荆迟偷偷摸摸的身影,想必是昨夜溜出去的吧,谁让寒园把他拘束坏了,我原想当作看不到他,转念一想,道:‘荆将军,还不过来拜见殿下。‘

荆迟住了脚步,走了过来,规规矩矩的拜见殿下,我笑道:‘殿下想让你作诗一首,你意下如何。‘

荆迟张大了嘴,不知道该如何是好。雍王笑道:‘听说你学会作诗,本王很感兴趣,这样吧,本王出个题目,对了,你刚才要去做什么?‘

荆迟尴尬地道:‘末将昨天晚上出去赌钱,现在回来想去睡觉。‘

李贽瞪了他一眼,道:‘这样吧,你就以睡觉为题吧。‘

荆迟想了半天,说了一句道:‘佛爷睡得好。‘

李贽噗哧一声笑了,道:‘这倒是有趣,看来你是去看过大化寺的那尊卧佛了。‘

荆迟连忙说道:‘是的,昨天末将和长孙将军去了大化寺,因为时间太晚,就没有回来。‘

我笑道:‘好了,不用解释了,接着作诗吧,你若是作出诗来,我就饶了你,否作我让你抄一天的兵书。‘

荆迟连忙道:‘有了,一睡百事了。我欲效他睡。‘念到这里,怎么也想不出最后一句。

李贽笑道:‘这第二句虽然有些像打油诗,勉强还可以,最后一句是什么,荆迟,你若作不出来,江先生可就输了。‘

荆迟脑子立刻晕了,心想若是江先生输了,只怕我今天是别想补眠了,想来想去却是想不出来,只记得满头大汗。李贽微微一笑道:‘想不出来就算了,你这个将军,平定天下还可以,作诗恐怕不成的。‘

这时荆迟灵机一动,想起江哲每次给自己讲书,其中经常提到靖胡尘,扫狼烟的语句。便说道:‘狼烟无人扫。‘

我和李贽都愣住了,其实我并没有想过要赢雍王,没想到荆迟居然真的写了一首诗出来。

李贽念道:‘佛爷睡得好,一睡百事了,我欲效他睡,狼烟无人扫。好好,这最后一句,点石成金,又显英雄本色,本王输的心服口服。‘说罢解下玉佩递给我道:‘随云能够让荆迟半个多月学会写诗,李贽可是服气了。‘

我接过玉佩,微微苦笑,道:‘荆迟,这块玉佩是殿下输给我的,我就借花献佛送给你了,若是你作不出诗,输的可是我啊。‘

荆迟恭恭敬敬的接过玉佩,道:‘谢谢先生赏赐。‘

我笑着摇摇头,这让我说什么好呢,想不到这个粗鲁的将军,真的让我刮目相看啊,原本想故意输给雍王,将这条防身玉带送给雍王,看来这次是不行了。

永宁坊,户部尚书梁谨潜望着孤灯,心中满是凄惶,他是宦海沉浮多年的老狐狸,如何看不穿阴晴冷暖,自从户部走私案揭发,他就明白了前因后果,什么崔央奉命稽查,根本是奉了太子之命走私,而自己事先被排除在外,事后虽然没有免职,可是只见太子只是忙着接收自己的势力,就知道自己的未来如何了,他真的很不甘心,很想拿着私自记载的帐册去告发太子,但是一想到人家君臣父子之间情谊深厚,就已经心灰意冷,更可怕的是,他想来想去想不到为什么太子会想放弃自己的时候,无意中想到了自己的妻弟多日不见,心中一动,查看自己私自记载的帐薄,其中自己做下的暗记已经全无影踪,当此之时,他真是如同寒冬腊月一桶冷水泼在身上,身处寒窟,想到自己身死之后,妻室儿女都难以幸免,他真想立刻逃走,可是普天之下,莫非王土,自己又能逃到哪里去呢?还没有想出办法,凤仪门的刺客已经出现在自己身边,这是一个素衣女子,相貌秀丽,可是周身上下带着森然的杀气,望着这个女子抱着自己心爱的幼子,他屈服了,按照她的命令将手上的所有权力交付,如今他已经是无用之人,被太子殿下软禁在家中,想必过些日子,事情平息之后,自己不是顶上走私军械的罪名明正典刑,劝架抄斩,就是削职为民,然后死在路上吧。他真的可以死,这一生他荣华富贵、金钱美色都已经享用过,可是自己一死事小,自己的家人又该怎么办呢?不过半个多月,他已经白发如霜,原本保养良好的容貌也变得苍老憔悴。

他正在苦思冥想,突然书房之门轻轻地被推开了,一个黑衣人走了进来,梁谨潜一眼看到,却没有似乎惊讶,冷冷道:‘你是来取我性命的么,老夫已经等候多时了,其实那位姑娘一直在后宅,让她杀我不是更方便么?‘

那个黑衣人关上门,说道:‘你若一死,还要连累家人,你不想反抗么?‘

梁谨潜心中一动,这个声音阴柔动听,不像是普通人,他抬起头,看向那人的面孔,那人黑巾蒙面,只露出一双冰寒刺骨的眼睛。

他缓缓道:‘老夫何尝不知,可是如今深陷罗网,无力挣扎。‘

那人轻轻摘下面纱,露出一张清秀如冰雪的面容,他微微笑道:‘死有轻于鸿毛,也有重于泰山,你若死在王法之下,不仅连累家人,而且只会让奸人得利,你如果肯自尽而死,我可以保证你的家人可以安度余生,他年你的子孙中有争气的,也可得到功名。‘

梁谨潜眼中一亮,自尽,若是自己自尽而死,或许那些人就不会为难自己的家人,可是,这又如何可以得到保证呢,他真的不敢相信太子殿下的信誉。他良久才道:‘你是太子殿下的人,我若自尽,真的可以让太子放过我的家人。‘语气充满了怀疑。

那人轻轻一笑,道:‘太子的承诺不可保证,可是雍王殿下的承诺你信不信。‘

梁谨潜大惊道:‘你是雍王殿下的人。‘

那人淡淡道:‘雍王殿下知道你为太子做了不少事情,可是如今太子已经准备舍弃你了,你的家人子女更是会成为陪葬,你若肯自尽,你的家人雍王殿下会安排他们去幽州定居,殿下一言九鼎,绝不会欺瞒你的。‘

梁谨潜心思百转,终于道:‘雍王殿下的诚意,我信得过,如果老夫早些跟随殿下,也不会有今日的结果。‘说罢取出一本墨迹尤新的册子道:‘老夫曾经记录了一本太子殿下从户部挪用银钱的账本,可是已经被拿走了,这是我这几天凭着记忆写下来的,希望对雍王殿下有用。‘

那人接过册子,正色道:‘殿下会感谢你的用心,这是鹤顶红,你绝对不会有痛苦的,我知道你希望和家人诀别,可是我不能冒险,所以委屈你了,你若有什么遗言,可以写下来。‘

梁谨潜微微一笑,拿起笔写了一封短信,也不封好,就这样递给了那人,然后笑道:‘我朝大臣犯了死罪,皇上也常常赐以鹤顶红,雍王殿下果然是心计过人,请转告殿下,臣相信他的承诺。‘说罢一饮而尽,顷刻之间,七窍流血而死。

那人打开一看,上面写着端端正正的两行字,

‘勿贪钱财而败名,勿爱权势而陨身。

梁谨潜绝笔

武威二十四年甲戌六月初二‘

