\part{第三部 夺嫡风云}

\chapter{第一章 暗波汹涌}

大雍武威二十五年乙亥,自户部事发后,朝野无声,平静以待风雨。太宗托病免朝,终日不出。

——《雍史·太宗本纪》

南楚同泰二年乙亥,哲渐病愈,其时朝野虽安,然夺嫡之事蓄势待发,哲为雍王主事,唯以隐忍为要。

——《南朝楚史·江随云传》

春光融融,和风徐徐,寒园之内,已经是绿树成荫了,自从去年的户部风波,尚书梁谨潜被突然鸩杀之后,局势突然莫名其妙的平稳了下来,雍帝李援连下诏旨,将户部大小官员尽皆去职的去职,降级的降级,罚俸的罚俸,户部清洗之后,新任的户部尚书是三原韩德,他是在户部多年的官吏,只是没有科举,又没有背景,多年来一直不得志,这次户部清查,只有他那里帐目最清楚,所以李援将他越级提升,韩德此人,不偏不倚,心中只有一个皇上,太子也不敢轻慢他,太子虽然又将不少人手插了进去,可是户部已经不像原来那样如臂使指了。

去年五月,咸阳出现魔宗弟子的消息闹得天下皆惊,最后那个淫贼被凤仪门抓住,那人自称是不服当年宗主被逐,故而到中原兴风作浪,凤仪门将此人杀死之后,亲自派人送了骨灰到北汉,魔门宗主京无极十分冷淡,既未发难,也未致歉,此事也就不了了之。

之后,大雍的政局突然出现了前所未有的沉静,太子每日只是按部就班的理政,雍王除了不放手军事之外,平日只是在王府中潜心读书,既不交结朝臣,也不招揽贤士,唯一的动作就是经常将一些落第书生、贫寒士子送到幽州任官,李援允许幽州自行选官,所以并不干涉,这些人都并非什么旷世奇才,所以太子方面也不愿因此翻脸。两方面都是韬光养晦,所以大雍局势出现了前所未有的安宁平静,可是有心人却都知道,这不过是暴风雨前的压抑罢了,太子和雍王已经是不死无休的局面了。

姑且不论外面的风风雨雨,寒园之内,正有一番奇景呈现,在凉亭当中,雍王悠闲的看着棋盘,小顺子坐在对面,神色平静的放下了棋子,示意雍王该轮到他了,而在凉亭之外,一个白衣书生正在草坪之上,四肢着地,扮成坐骑,而在他身上,一个穿着红衣的小女孩正用娇嫩的声音喊着“驾、驾,爹爹快跑。”

这一年来的安心静养,我已经全然恢复,虽然还是显得文弱单薄,但是容光焕发,已经不是那种随时都会断气的苍白模样了。不过当了一拄香时间的“马”,也已经是气喘吁吁了,只得告饶道:“蓝蓝,爹爹已经不行了,你也不想累坏爹爹,没人给你念骏哥哥的信吧。”

柔蓝乌溜溜的眼睛转了一会儿,终于点点头,从我身上滑了下来,奶声奶气的说道:“爹爹,我要去看公主娘娘。”

我笑道:“今天不行,过几天如果王妃去看公主,我请她带你去好不好?”

柔蓝撅着小嘴道:“公主娘娘都说蓝蓝可以经常去看她呢?”

我微微苦笑,这可不是我们说了算的,自从公主在无尘庵清修之后,她和韦膺的婚事也就拖了下来,皇上没有取消赐婚,可也没有逼迫公主完婚,只苦了韦膺,又不敢娶妻,又不敢要求大婚。我和公主的流言也传了几日,可是毕竟我和公主都不见面,所以在雍王的打压下,又没有太子的推波助澜,流言很快就烟消云散了,毕竟没有人想把不参与宫中纷争的长乐公主逼了出来,再加上不想惹怒李援,所以这些流言很快就被人淡忘了。

其实我想雍帝可能也听到一些风声,可是我和长乐既然没有私情,也没有见面,他总不能因为长乐可能对我有情而处罚我吧,所以这一年来,我还是过得很滋润的,只不过,我经常会想起长乐公主,一幕一幕的回想仅有的两次见面,后来雍王妃常常去看公主,而柔蓝也常常被王妃带去,这一点倒没有引起什么是非,谁不知道雍王妃将柔蓝视若己出,谁不知道世子李骏在幽州,每个月必定派使者进京向雍王述职,而使者每次必定带来一些小女孩的玩具和一封书信,所以柔蓝在大雍宫廷的出现已经成了理所当然的事情。长乐公主喜欢柔蓝,大家只当她膝下空虚,所以喜欢小女孩儿罢了,虽然也有人想到“爱屋及乌”的可能,但是谁也不敢把这件“子虚乌有”的事情搬上台面。而且为了见见柔蓝,长乐公主一年倒有半年住在宫里,毕竟雍王妃进宫拜见皇后贵妃是件平常的事情,她若是总到无尘庵去看公主,这倒会令人担心公主是否和雍王走得太近。因此,就连长孙贵妃也对柔蓝十分疼爱,有时还会把柔蓝留在宫里几天。柔蓝也见过雍帝李援,李援也很喜欢这个精灵淘气的小丫头,这样一来,更没有人敢多嘴多舌了。

虽然这一年来我也没有和公主见面,甚至也不曾想办法问过她是否真的对我倾心,可是总是忍不住将新作的诗词通过雍王妃送给她,她也没有回音,只是经常给柔蓝一些玉佩护身符之类的赏赐。听雍王妃说,这一年来,公主气色大好,不仅常常欢笑,而且在雍帝和长孙贵妃面前也是神色开朗,两人见她这样,反倒觉得不必急于迫她出嫁,让她郁闷不快。如果说还有什么让她不乐的,大概就是韦膺的柔情攻势吧,说起来韦膺对公主倒也是诚心诚意,虽然因为公主拒婚而失意,但是每每送上一些小礼物,或者是孤版书籍,或者是上好的笔墨纸砚,来讨好佳人,这种细水长流的柔情攻势让皇上和长孙贵妃都十分感动和支持,虽然长乐公主并无动心,可是韦膺彬彬有礼,从不咄咄逼人,总是礼数周全,公主又是性子温柔的人,不愿恶言恶语的拒绝,只能冷淡疏离一些罢了,但是对于韦膺和公主的婚事,皇上和长孙贵妃都是乐见其成的,所以长乐公主就不免时常和韦膺“偶遇”了。前些日子,我想既然韦膺痴心追求,我不妨冷淡一些,若是公主能够匹配佳偶,我也可以放下心事了,因此一个多月没有让柔蓝进宫,谁知雍王妃很快就对我说,这段时间公主情绪不佳,又去了无尘庵小住,这种情况,我若还不明白公主的心意,那么我恐怕就是世上最大的白痴了,因而再也不禁止柔蓝进宫,虽然两人从不相见,可是奇特的,总是能够感觉到心中温馨阵阵,虽然咫尺天涯,可是却觉得并无隔绝。

不管怎么说,终于让柔蓝下去了,说来好笑,柔蓝还不认字呢,世子李骏就一封一封的书信送来,当小女孩捧着书信一个字都不认得,苦恼的扯着我教她认字,我只能哈哈大笑了,就是我想教她写字念书,想看懂这封信也得等两年,无奈何之下,只得给她念信,其实内容也没有什么,不过是今天去了什么地方,看到了什么好玩的东西,只是这个李骏倒是很会说话,每次柔蓝听了都闹着要去幽州玩,幸好她不会吵闹太久。柔蓝虽然还小,可是已经有了羞涩之心,绝对不肯让别人看到信的内容,只让我替她念,所以我才能威胁她放我一马,更决定晚点教她认字,否则没有了这个杀手锏我可怎么办呢。

看我终于起身了,领着柔蓝向凉亭走来,李贽笑道:“随云,你来了,好了这一局就这么算了吧。”

我看看棋盘上,李贽的棋子已经七零八落,笑道:“人都说善奕者善战,若是沙场作战,小顺子是必输无疑,可若是下棋,殿下也只能甘拜下风了。”

小顺子面无表情的收起棋盘和棋子,完全没有意思附和,只是嘲弄的看了我一眼,我不由摸摸鼻子,实话说,我和他下棋,现在这小子可以让我三子了。

坐下来端起茶杯,小顺子已经将柔蓝交给王妃的侍女送回去了,觉得浑身上下有些酸痛,一杯热茶下肚,我觉得精神一震,不由舒适地呻吟了一声。

李贽笑道:“昨日秦青申斥了禁军北营统领裴云,说他帷薄不修。”

我微微一笑道:“这是李寒幽的主意吧,如今秦青可是唯妻命是从啊。”

这一年来最风光大概就是秦青和李寒幽了,半年前她已经和秦青完婚了,完婚之后不久,秦青就升任禁军大统领,虽然实际上禁军大统领一直都是个虚职,禁军实际上是由抚远大将军秦彝掌管的,可是秦青乃是秦彝长子,比起别人来当然不同,虽然秦彝仍然没有将权力下放,但是现在秦青还是可以调动部分禁军的。如今秦青已经是大雍颇富盛名的青年将领了,而靖江公主李寒幽本身已经是公主之尊,又是凤仪门弟子,虽然她的出嫁让她不再可能是凤仪门内堂弟子,但是她在凤仪门的崇高地位还是很明显的,这样一对夫妻,自然是万人瞩目了,更难得是,他们又是恩爱非常,更让大雍朝野艳羡非常。

李贽冷笑道:“裴云前些日子正式将爱妾迎娶入门,他的正室夫人却得到一纸休书,这也难怪李寒幽大怒,裴夫人薛秋雪乃是凤仪门弟子,据说和李寒幽情同姐妹。”

我端起茶杯,淡然道:“这也只能怪那个女子愚蠢,裴云摆明了不想娶她,当日裴云上薛家请罪的时候说得很清楚,他已经有了外室,并且已经怀孕,如果薛家愿意退婚,情愿付出代价,那薛小姐却执意要嫁入裴家,这也罢了,若这个女子肯守本分,裴云本是善良之人,天长日久,未必不能接受她,可是她的手段还不到家,手段过于急进,反而让裴云敬而远之,现在还作出加害妾室和初生婴儿的事情,若非发现及时,这就是两条人命,若非碍于凤仪门,只怕裴云早就一剑杀了她了,不过秦青责备裴云也是有道理的,无论如何,这也确实算的上是帷薄不修。”

李贽说道:“这样一来,凤仪门自然不肯罢休,虽然碍于人伦不能直接插手,可是她们指责裴云不应该冷落结发妻子,已经和少林争吵了好几次。”

我笑道:“虽然她们说得不错,可是少林根本就默许了裴云这种行为,裴云是他们精挑细选的弟子,他们是绝对不愿意裴云和凤仪门有什么关联的。”

李贽点头道:“话虽如此,可是少林毕竟不会和凤仪门翻脸,凤仪门虽然也不能公开找裴云麻烦,但是李寒幽还是可以通过秦青来为难裴云,你说该怎么办呢,裴云是你好不容易在禁军扎下的钉子,可不能随便放弃。”

我摇头道:“殿下过誉了,我不过是引了一条路,能够让裴云衷心效忠殿下,都是殿下自己的本事,自古良臣择主,如果不是殿下仁义贤明,裴云怎会甘心效命,这次殿下也得出手相助,必然可以令少林寺真正支持殿下,从前虽然少林有意和殿下合作对付凤仪门,但是碍于皇上和太子,始终只能暗中支持,这次凤仪门太过嚣张,只怕会惹怒了少林,这正是殿下的机会。”

李贽叹息道:“随云,本王对你佩服万分,一年前你的作为,让朝野有识之士看清了太子的一些面目,现在他们即使没有决定支持我,也都转为中立,从前很多人都认为太子是储君,又无失德,所以就算觉得本王贤明,也总是若即若离,如今本王虽然遵照你的吩咐没有随便招揽人才,但是本王却能感觉到他们更加愿意亲近雍王府,不过一年多,你就让本王扭转了局势,本王不知该如何感谢才好。”

我淡淡道:“这也是殿下肯接纳我的意见,我让殿下不要行动,韬光养晦,殿下欣然接受,这一年来,殿下没有异动,这样太子就不能以殿下功高震主的理由攻击殿下,他的种种为难,反而越发让人同情殿下,而石彧在幽州奉殿下之命选官,人人却都以为殿下是为了封地着想,如今殿下麾下文武齐备,已经可以开始大展宏图,臣可以保证,今年之内太子就会失去储位。”

李贽疑惑地道:“虽然太子失去了部分人心,但是毕竟还没有被废的可能,这一年来他也很谨慎,你如何能够确定可以废去他的储位呢?”

我神秘的一笑,道:“殿下这些年来一直致力于在太子的势力中插入人手,从前因为太子谨慎小心,鲁敬忠和凤仪门的力量,始终难以如愿,可是这一年来,太子因为户部之事失去了人心,又因为杀人灭口的行径失去了属下的信赖,而鲁敬忠和凤仪门也是面和心不和,殿下不是已经成功的打入了太子势力的中坚么,虽然还没有接触到核心,可是太子殿下的一些行动还是瞒不过您的,您真的不知道,太子殿下都在干什么?”

李贽尴尬的一笑道:“这我倒是知道一点,听说太子不知道怎么回事,迷上了青楼,好几次包下大雍有名的艳妓秘密金屋藏娇,直到后来父皇知道了风声,他才收敛了,最近他已经没有做这种风流勾当了,倒是总是到后宫陪着父皇皇后,孝顺他们。”

我冷冷一笑道:“那是因为他改了消遣方式,他迷上了皇上新纳的一个妃子。”

李贽一惊,道:“这怎么可能,这是乱伦的事情,若是父皇知道,岂不是要重责于他,恐怕废了他的储位也是可能的。”说到这里,李贽顿住了,半晌才道:“以宫闱之事废储君,恐怕不是那么容易的,毕竟后宫不能干涉国本。”

我意味深长地道:“太子殿下若是有些本事,皇上或者不会废了他的储位,可是皇上本就已经对太子失去了信任,如今对皇上来说,太子恐怕更大的作用是压制殿下你,这件事情发作,就是皇上无心,恐怕也会对太子施以重惩,不管皇上是否有意废除太子的储位,态度总是要表示一下的,这样一来太子心中自然充满忧虑犹疑,父子相疑,这就是臣要的结果。太子殿下心中有愧,就是保住储位恐怕也会日夜担忧皇上是否会秋后算帐,到时候必然会乱了方寸,这样一来,他越是想要弥补,只怕越引起皇上的不满,别说宫闱之事不重要,自古以来天子父子之间,亲情从来不厚,父子相残却是屡见不新,到时候恐怕太子猜忌皇上的心情比猜忌殿下还要多些呢。”

李贽道:“可是纪贵妃等人必然百般相助,恐怕还是没有什么作用。”

我淡淡道:“她们若是明哲保身,臣才担心呢,她们做的越多,破绽也越多,殿下难道不想让她们原形毕露么。”

李贽陷入沉思,面上露出一丝喜色,道:“随云真好计策,其恶不彰,本王焉能无罪加诛。”

\chapter{第二章 淫威肆虐}

武威二十四年,王因户部事受责,帝密令闭门思过,王性暴戾,多有不端事。

——《雍史·戾王列传》

初夏的午后,阳光已经很强烈,在树荫下站着两个侍卫,神色严肃的注视着四周,执行着自己保护皇室的责任,在他们身后不远,一处秀雅的小宫殿里面,门口的几个宫女和太监正在那里低声谈笑。这里是皇上新近宠爱的淳嫔的住处,她今天才十九岁,相貌艳丽,一身媚骨,丽质天生,十分得到雍帝宠爱,不过雍帝毕竟年纪已老,皇后和几位贵妃娘娘都不愿他纵情声色,因此这里李援并不常来。现在是午后,他们也没有什么工作,所以才能这样悠闲,可是若是仔细看去,这些人眼中都带着淡淡的恐惧和忧虑,还不时的回头望向宫殿。

宫殿深处,重重帷帐的后面,一张宽大的红木软榻的上面,一男一女正在抵死缠绵,娇吟声和粗重的喘气声回荡在宫殿当中,终于,在一阵歇斯底里的发泄后,两人停了下来,那个女子紧紧抱着男子赤裸健壮的身体,死也不肯松手,两人相拥了片刻,那个女子终于松开了手,懒洋洋的道:“殿下,您该起身了。”

那个男子留恋的抚弄了片刻女子那雪白娇嫩的肌肤,终于依依不舍的站了起来,走到偏殿,那里已经准备好了浴汤,沐浴更衣之后,那个男子走回寝殿,只见他身上穿着杏黄龙纹的皇子服饰,这是只有太子才可以穿着的颜色,这充满*的寝殿竟是乱伦的所在。

李安迷恋的看着这个女子,其实论起美色,这个女子虽然美貌,但未必就胜过他的侧妃萧兰和其它他临幸过的女子,想当初,他娶到萧兰的时候也曾经这样疯狂,身为男子,能够让一个风华高贵、清丽如仙子的女子在自己身下婉转娇吟、欲仙欲死,那是何种的意气风发,可是后来,渐渐的他有些厌倦萧兰总是谆谆教诲的面孔,开始暗中猎取美人,可惜当时他最忌惮的雍王的压力压得他喘不过气,为了得到父皇和那些道貌岸然的老臣的支持,他不得不谨慎小心,所以轻易不敢放肆,就是家中宴饮也不敢轻狂妄为。

直到他代皇上告祭太庙之后,储位稳固,他才不由放松了许多,开始豢养舞姬歌女,恰好他得到了一个贴心的侍卫夏金逸,这人虽然武功平平,却是擅长各种风流勾当,将府中的舞姬歌女调弄的色艺双全,让自己在温柔乡中沉醉不已。尤其是自从去年户部事发之后,他虽然没有受到父皇责罚,可是他也能够感觉的父皇对自己有些冷淡,想起来也真令人气愤,好不容易出了魔宗弟子进入中原的事情,引开了别人的主意,他就连受害的是凤仪门弟子也顾不得了,可是没有几天,那个梁谨潜却被鸩杀了,这个梁谨潜该死,他迟早不会放过他,可是绝对不该是这个时候,不仅皇上震怒,把他叫去训斥了一顿,不由分说的把杀人灭口的罪名加在他身上,就连鲁敬忠和萧兰也都埋怨他,好一阵子他都郁闷不安,最后还是夏金逸有法子,召集了舞姬侍女,在密室之中召开了无遮大会,就是纣王的酒池肉林也不过如此,原本他应该谨言慎行,可是这样胡作非为,却让他心情从郁闷狂怒中平静了下来,渐渐的,他发觉好像只有通过那种方式才能平复自己的心情,反正他自认做的神不知鬼不觉,再说父皇就是知道了,也不会为了这种事情和自己发怒,他又何尝不是三宫六院快乐逍遥。

开始的时候还只是在府中淫乐,后来却觉得没有趣味,这些女子不是曲意奉承就是强颜欢笑,让他索然无味,不由想起曾经的一次放纵,那个南楚名妓艳光四射,舞姿炽烈,可是却不肯和自己共度春宵,自己一怒之下用强了事,那一次的滋味他至今难忘,想来大雍的名妓也未必逊色,可是自己身为储君怎好走马章台,想到这里就不由羡慕齐王李显,后来他把心思跟夏金逸说了。夏金逸却是聪明,他自己或者派人伪装,将大雍有名的名妓接到一处庄园养起来,然后李安伪装成平常人去挑逗她们,有时候很容易上手,有时候却要苦苦追求,但总是让李安享受到不一样的风情,后来,李安厌倦了这种平常的花样,开始玩弄各种各样的女子,这个庄子也就成了有进无出的死地,不知多少青春少女的香魂埋葬在黄土之下。夏金逸更是提供了一种极品的春药给他,服用之后不仅可以连御数女,而且起床之后还是精神百倍,所以李安更加放肆胡为。

可惜他还没有玩腻,就被萧兰阻止了,萧兰神色阴森,对着他冷冰冰地道:“殿下若想登基为皇,怎能做这种授人以柄的事情,不说别人知道,就是我师父知道,必然也会震怒,到时候若是师父不再支持殿下,只怕殿下后悔都来不及。这次臣妾替您善后,日后再有此事,只怕臣妾也帮不了殿下了。”

李安虽然有些恼怒,可他还是知道这次是自己过分了,接下来的日子只得闷在府里,可是他总是坐立不安,只觉得府中的侍妾宫女都是索然无味,直到有一次雍帝家宴,他看到了在妃嫔最末端的位置站着一个艳丽无双的女子,那一刻,他只觉得浑身的血液都炽热了,那是一个明丽的少女,她的微笑仿佛春花绽放,而当她婀娜多姿的上前献舞的时候,李安终于再也压抑不住渴求的欲望,这个女子乃是北地人,擅长胡旋舞,当她赤着双足,站在不过一丈方圆的圆毯上,飞速旋转的时候,那变化多样的舞姿动态和腾踏跳跃旋转的高难度技巧,让李安心中更是痒痒的,当看到父皇上前扶起舞罢躬身行礼的妾妃,看到青春焕发的淳嫔和已经显得老迈的父皇,李安不由惋惜的叹了一口气。

虽然爱慕,可是李安毕竟还是没有昏了头,这个女子虽然只是下等妃嫔,可仍然是自己的庶母,这乱伦之事在历代宫闱中虽然屡见不鲜,可毕竟不是什么光明正大的事情,再说他还只是太子,可没有这个胆子。可什么事情越是隐忍,引诱力就越强,李安一连多日辗转反侧,脑子里都是那个飞旋的迷人舞姿。

他贴身的侍卫,府中的副统领夏金逸见他茶饭不思,百般劝解也无效,便问他为何这样忧愁,李安对这个贴身侍卫兼副总管已经是十分信任,不仅聪明能干,更是守口如瓶,自己的私事从无外泄,夏金逸功劳非浅。李安终于还是说了自己的心事,这种事情,他就是再信任鲁敬忠,也不愿去和他商量。

夏金逸开始为难地道:“殿下,属下的性命和荣华富贵都是殿下所赐的,就是为了殿下粉身碎骨也不该畏难,可是这种事情是不同的,若是事发,就是属下想替殿下顶罪也不可能啊。”

李安也是心灰意冷,悒郁成疾,居然病倒了,这下可吓坏了夏金逸,最后忍不住道:“殿下,你在宫中势力眼线不少,皇后又是您的亲生母亲,纪贵妃娘娘更是支持你,淳嫔虽然得宠,不过是个下等妃嫔,你只要以势相逼,以权势相诱,这个女子也不是什么三贞九烈的人物,再说,您是将来的皇上,等到您登基之后,她的生死荣辱大半都在您手上,不说别的,若是皇上万岁之后,淳嫔若没有子嗣,就得出家为尼,到时候青灯古佛,清冷寂寞,她青春年华,如何忍受得住,皇上春秋已高,恐怕没有什么机会让淳嫔怀孕了,若是她从了您,说不定还能生个一字半女,到时候就可以有了依靠,就是没有,以后有殿下照拂,也可以安度余生。”

李安听得眉飞色舞,只觉得神清气爽,立时拿定了主意,便和夏金逸商量好了计策,先是请旨要求协助皇上看折子,恰好李援也已经消了气,便允许他在东宫处理政务,而且李援也有些倦怠政务,便派了大臣辅佐李安处理政务,而李安便借机在午后去探望母后,然后便趁机去勾引淳嫔,他是太子之尊,在后宫权势极大,再加上金银开路,很快就顺利的接近了淳嫔,淳嫔初时也是婉言拒绝,后来却架不住太子的热切追求,再加上夏金逸有意无意的威胁利诱,淳嫔终于投入了太子的怀抱,这种禁忌的热恋有效地让太子忘却了外面的闲花野草,每日总是在东宫处理政务,只有在午后的一个时辰在淳嫔那里度过,皇上不知,反而觉得太子最近勤于政务,因此十分高兴,浑不知太子的逆伦丑事。

当李安依依不舍的离开了淳嫔的宫殿,在外面把风的夏金逸和几个侍卫已经迎了上来,簇拥着太子回去东宫,李安却没有注意到,夏金逸的神情有些不安。

夏金逸心中有些不安,这一年来,他用了浑身解数讨好太子,甚至做了很多从前不敢想不敢作的事情,那一个个青春少女,大半是他安排送到太子身边,而各种善后灭口的事情也是他亲力而为,这些事情他不敢对任何人说,可是他告诉自己,若想报仇雪恨,让那个绚丽的身影沉沦在地狱,他就只有一条路,那就是按照那个人所说,让太子放纵肆虐,他相信自己做到了,可是他已经双手血腥,罪孽深重,恐怕九泉之下也无颜拜见爹娘了。更让他不安的是,他始终没有机会和那人见一次面,他是知道的,那个人深居王府,轻易不出寒园半步,身边侍卫更是如狼似虎,自己根本就没有机会传递消息给他,而且,他也不敢,在太子身边这么久,他是深深知道太子少傅鲁敬忠和太子侧妃萧兰的厉害的,他不敢贸然和那人联系,只能心中期望自己所作所为能够帮助那人,让自己终究有一日能够得偿夙愿。可是目前的危机可怎么办呢,昨天绣春偷偷来告诉他,听见太子侧妃萧兰和王妃崔氏说些什么,虽然没有听清楚,可是绣春听到了夏金逸的名字。夏金逸可是心里有鬼的,上次萧兰下令将山庄守卫和那些女子全部处死,然后全部毁尸灭迹,若非自己被太子带走,只怕也难逃厄运,可是他总是忘不了萧兰那看着自己的目光,冷酷而无情,这次自己会不会有这样的好运气呢。

回到东宫,有些疲倦的李安看着折子直打瞌睡,终于忍不住伏案小憩,夏金逸替太子盖上披风,悄悄的退到门外,却是侧耳细听,等待太子的召唤。这时候一个侍卫蹑手蹑脚的走了过来,低声道:“副总管,王妃派人来传令,说是有事要您去办。”

夏金逸皱眉道:“我正在伺候殿下,你是知道的,殿下是一刻也离不开我的。”

那个侍卫苦笑道:“副总管大人,我怎么敢和王妃说这些,您还是回去一趟吧。”

夏金逸想了一想,问道:“可是王妃亲自召见你传令的。”

那个侍卫道:“大人放心,我亲自听王妃说的,她有些事情要你去办。”

夏金逸略略放心,又问道:“我师兄在不在府上,有什么事情不能让他去办。”

那个侍卫低声道:“您是知道的,总管大人性子严正,有些事情必然是不愿意去做的,说句实话,听王妃的侍女说,好像是王妃的外甥在外面犯了事,需要有人去疏通一下,您是知道的,这种事情您若不去,谁还能去办,王妃也不希望这件事情众人皆知。”

夏金逸这才放下心,点头道:“好吧,你们好好伺候殿下,我去去就回。”

在回府的路上,夏金逸却是总觉得心中不安,想起昨日绣春告诉他的事情,总觉得其中有些不妥,在临进府的时候,他吩咐一个手下道:“你不要进去了,就在外面等我,王妃吩咐事情,用不了多长时间,半个时辰之后我如果不出来,你就立刻进宫请见殿下,就说我求殿下救命。”

那个属下连连点头道:“属下明白,副总管小心一些。”

夏金逸微微苦笑,心道:“我如今满身罪孽,人皆可杀,若非心愿未了,就是死了又有什么打紧,可是现在我却不能死,若不见她沉沦苦海,我决不罢休。”想到这里,他仰头挺胸走进太子府,不管伸头缩头,都是一刀,事到临头,总不能退缩,再说王妃相召,焉能推辞。

进得府来,只见往来的侍卫宫女眼中都带着一丝同情怜悯,夏金逸便知道这次不好,他虽然得到太子宠信,为人却是豪爽大方,从不抢夺别人的功劳,也不欺凌弱小,不论是侍卫宫女,只要面子上和他过得去,他就十分周旋,这一年来太子喜怒无常,若没有他求情,只怕府中很多人都会受到太子责罚,所以虽然他这个实际上的弄臣人缘却是很好。虽然现在不敢明言,却道暗中示意,有几个要好的侍卫还示意他快走。夏金逸却知道是万万逃不得的,只得走到了后面的花厅,这里是王妃接见外臣的所在。夏金逸一走进花厅,就看见萧兰坐在上首,神色森然,而客位上坐着一个艳色绝伦的女子,正是靖江公主李寒幽。夏金逸眼中闪过一丝不可觉察的寒光,上前拜倒道:“属下夏金逸叩见兰妃娘娘、公主殿下。”

李寒幽淡淡一笑,看了萧兰一眼,道:“师姐,这人就是那个胆大妄为的奴才,挑唆太子不行正道的幸臣。”

萧兰冷冷道:“正是此人,别看他相貌堂堂,却是一个金玉其外,败絮其中的奴才,谄媚主上,罪大恶极,师妹你今日难得来看我,就让师妹看看我的手段。夏金逸,你知罪么?”

夏金逸抬起头,神色淡然,心中却是汹涌不安,萧兰眼中杀气纵横,看来是决定杀了自己的,可是为什么她也在,难道她还能认得自己么,不可能,不说那时自己形容还未长成,如今她如此尊贵,怎会记得当日被她狠狠伤害的少年呢。他举目看向李寒幽,李寒幽似乎为他的沉静感到吃惊,也看向他,四目相对,李寒幽眼中丝毫没有别样的意味,夏金逸放下了心,想来自己如今气质全然大变,她必然不会想到自己曾是她的旧识了。

李寒幽看向这个男子,明明是那样卑微的身份,又是人品低下,却是神情淡然。气度从容,英俊的相貌也让他颇为引动女子的春心,可是这人却是一个人品低下的弄臣小人,真是可惜了,她微微摇头,看向萧兰。

萧兰见夏金逸不答话,更是恼怒,又问道:“你不答话,是不是轻视于我,我问你,夏金逸,你可知罪么?”她的怒气如此炽烈,让夏金逸觉得胸口仿佛被她身上涌出的杀气重击了一下,不由自主的俯身道:“小人不知犯了何罪,请娘娘明示。”

\chapter{第三章 花言巧语}

武威二十五年五月,靖江公主、王妃萧氏以侍卫夏某谄媚惑主,欲杀之,为王所阻。自此,王与王妃、公主嫌隙益深。

--《雍史·戾王列传》

萧兰柳眉倒竖,神色冷若冰霜,冷冷道:“好,你既然还敢狡辩,那本宫就和你说个明白,这一年来你都做了什么,还要本宫一一道来么,身为臣属,不知道劝谏主上,只知道谄上媚权,调唆太子做下这等没有礼法的事情,你难道不该死么,为臣不忠,为人不义,你既然是这等不忠不义之人,若是还有半点天良人性,就该横剑自刎,难道还要本宫动手么?”

夏金逸神色从容地道:“属下不过是个江湖浪子,既没有满腹诗书,也没有绝世武功,所擅长的不过是些雕虫小技,太子殿下救了属下的性命,属下无以为报,只能尽力让殿下开心一些,如果这也算的上不忠,属下也无话可说,说到不义,属下倒是认得,但是属下一心只是效忠太子殿下,忠义不能两全也只好罢了,再说普天之下,莫非王土,率土之滨,莫非王臣,就是太子殿下有些什么过分的事情,又有什么关系,若非如此,您又何必杀人灭口,而不是大义灭亲呢?”

萧兰顿时语塞,这时李寒幽冷笑道:“好个厉嘴的奴才,太子殿下是君,你是臣,殿下可以犯错,可是你不能,你妨害了殿下的大业,本宫也懒得和你评理,师姐,也不必和这个奴才多嘴,还是快些请太子妃殿下传下谕令吧,这外面的事情自然是太子殿下作主,这府中之事还是得太子妃作主。”

萧兰立刻省悟,高声道:“快去向姐姐禀告,就说夏金逸这个迷惑主子的奴才已经就缚,请姐姐吩咐。”

夏金逸冷冷一笑,心道,这兰妃娘娘倒是心机深沉,这借刀杀人可是做的不错,但他心中却毫无恐惧,死亡对他来说早就是一件求之不得的事情了。

李寒幽微微蹙眉,她原本只当这个夏金逸不过是个趋炎附势的小人罢了,这种人一旦面临生死关头,往往奴颜婢膝,毫无气节可言,可是如今这个青年只是微微冷笑,既不求饶也不哀告,这让李寒幽心中十分不安,是他有什么自保的法子,还是他本性如此,若是这样,他作出这些伤天害理的事情只怕是别有用心的了。

太子妃崔氏的寝殿中,此刻绣春正跪在地上苦苦哀求,崔氏无奈地道:“本宫也知道这夏金逸是你的情郎,又常常替我在殿下面前美言,怎会没有感激之心,可是兰妃说的有理,太子殿下是我们的夫君,也是我们的依靠,若是太子有了意外,我们可如何是好,夏金逸调唆殿下在外面风流,事情如果传出去,只怕要惹恼皇上,本宫也是不得已。”

绣春哭泣道:“娘娘,婢子不是说兰妃的坏话,这些年来,兰妃娘娘何曾把娘娘看在眼里,有什么事情她问过娘娘的意见,她一道令旨胜过娘娘千言万语,怎么如今想起让娘娘下令处置人了,再说,金逸就是百般不好,他对太子殿下忠心耿耿,对娘娘礼敬有加,这些日子以来,娘娘还没有感觉么,不论什么事情,他总是替娘娘说好话,去年舅爷的事情,不是他通风报信,娘娘还蒙在鼓里呢,若不是娘娘在殿下面前哭诉哀求,只怕舅爷死了还要落个罪名,人死百事皆了,可让您的家人怎么办呢,还会连累到您和小世子。就看金逸这片心意,您也该帮帮他。”

崔氏长叹一声道:“是啊,这个人确实对本宫礼敬,这一年来,太子身边的这些嫔妃想要见太子一面是千难万难,只有本宫十分方便,本宫送去的补汤点心,太子都有回书,而且每个月总有几日在本宫这里留宿,我知道夏金逸用了不少心思。”

绣春神情大振,道:“娘娘,婢子说句不该说的话,太子殿下这一年来待您虽然没有特别好,可是也没有冷落您,从前来多少次,现在也是多少次,殿下就是再风流,与娘娘又有什么害处,倒是您这次若是下令杀了夏金逸,等到殿下回来,必然大怒,到时候那一位只说是娘娘的意思,只怕日后太子再也不来娘娘这里了,到时候占便宜的是谁,那位觊觎您的位子也不是一天两天的事情了,您不为自己着想也要为世子着想,别说是现在,就是将来太子殿下登基之后,若没有这么一个心腹人在太子身边,娘娘您可怎么对付那些狐媚子呢?”

崔氏越听越是心寒,道:“绣春,你说得对,本宫几乎被那贱人骗了,你立刻去传我的令旨,就说夏侍卫是太子的心腹人,本宫不便处置,先将他拘押起来,等到太子回来再交付太子处置。”绣春大喜,连忙亲自去传令。

听到绣春的回复,萧兰秀美的面容上现出怒色,她怒斥道:“好你一个贱婢,可是你搬弄是非,让姐姐改了主意,早听说你和这奴才有私情,如今看来果然是的,罢了,本宫也不求人,今日一定要将你们这对奸夫淫妇杖杀在此。”

绣春面上现出恐惧之色,她本是担心夏金逸的安危,这次亲自来传令,不料萧兰居然要连她一起处置,吓得不敢出声,但她虽然羞愧,却是神色倔强,不肯哀告求饶。夏金逸却冷冷道:“属下和绣春的事情,太子殿下和太子妃殿下都早已知道,只是娘娘喜欢绣春侍奉,殿下也喜欢属下服侍,所以没有急着成婚,这奸夫淫妇四个字,属下可不敢当。”

李寒幽面色突然一变,冷冷道:“还和他们罗嗦什么,师姐,他们在拖延时间。”

萧兰立刻站起身道:“来人,用刑。给我把这对狗男女活活打死。”

李寒幽冷冷道:“那个丫头,有自己的主子,她没有廉耻,也该她的主子教训。”

萧兰道:“听见了没有,把绣春送回去,就说让姐姐管教一下这个贱婢。还不动刑,你们等什么。”

两个侍卫走了过来,手中拿着红漆刑杖,另外一个宫女则拖着绣春向外就走,绣春一边挣扎一边哭喊道:“夏郎,夏郎。”但那几个宫女力量极大,很快绣春的声音就听不到了。两个侍卫走到跪着的夏金逸身边,其中一个人低声道:“娘娘在上面看着,请恕属下不能手下留情。”说着一记刑杖已经重重的打在了夏金逸的肩背上,夏金逸只觉得背上一阵剧痛,知道这些人是要快刀斩乱麻,几杖就可以让自己脊骨折断,但他平日虽然好像墙头草,可是此时面对那个刻苦痛恨的仇敌,竟然是绝不肯求饶的。他闭上了眼睛,也不说话,咬紧了牙关等待接下来的痛苦。

谁知下一杖迟迟不见临身,他睁开眼睛,只见一个大汉怒目圆睁,紧紧的抓住了刑杖,他惊叫道:“师兄。”原来那人正是他的师兄张锦雄,此刻他浑身上下威严可怖,眼中满是杀气。

萧兰面色一沉道:“张总管,你要做什么,竟敢对本宫无礼。”

张锦雄冷冷道:“萧兰,你也不必用身份压我,名份上你是主子,我是总管,可是我张锦雄乃是崆峒掌门弟子,你萧兰则是凤仪门高弟,当初凤仪门使者到崆峒结盟,我奉师命前来供你们驱策,可是我这个师弟碍着了你什么,你们竟然要杖杀他,难道,你们真的不将我张锦雄放在心上么,还是以为我会坐看他被你们辱杀。”

萧兰大怒,正要说话,李寒幽已经冷冷道:“张大侠,本宫说句公道话,先不说这人是你们崆峒的不肖弟子,如今他在殿下身边都做了什么,你难道一点风声也没有听到,我们杀他也是为民除害,你是未来的崆峒掌门,理应洁身自爱,怎能庇护恶人。”

张锦雄冷冷道:“靖江公主,你别把我当成傻子,金逸就是有错,也罪不致死,你们有本事还是去劝劝殿下的好,我这个师弟虽然不成材,可是他不是什么坏人,就是他为虎作伥,你们不去杀虎,却和我的师弟为难,也真是好盘算。”

萧兰再也忍耐不住,突然飘身扑上,她手中无剑,长袖便像龙蛇一般盘卷,身形到了张锦雄面前,已经是龙起大海,劲风向张锦雄扫去。张锦雄不敢怠慢,一拳迎上,这一拳意在拳先,似实还虚,正是只有崆峒嫡派传人才能修习的神门拳,拳袖相交,萧兰被迫得后退了一步,她心中一凛,平日她自恃师门心法独特,自己的内力不弱,想不到这位崆峒掌门弟子内力如此雄厚,她心中既然有了忌惮,飞身退下,这时李寒幽已经拔出长剑递了过去,她接过长剑,举起平指,转瞬之间,已经是神色庄重,意态悠闲,张锦雄心道,凤仪门弟子果然名不虚传,一柄长剑使得凌厉狠辣,她的轻功又好,转眼间满屋都是剑光。张锦雄的一双铁掌却也毫不示弱,崆峒的武功本就走得奇门,两人都是攻敌之必救,以攻代守,转眼间就交手几十个回合,萧兰虽然剑法轻功出色,但她毕竟只是一个女子,又是常年养尊处优,怎及张锦雄武功精纯,搏斗经验丰富,渐渐的落了下风。

李寒幽在一旁微微蹙眉,若是换了一个人,或者她就给了张锦雄面子,可是这个夏金逸出乎她的意料,做得是趋炎附势的事情,但居然性子倔强,不肯认罪不说,竟连一丝悔意恐惧也无,若是今日放过了他,他必然怀恨在心,这一年来,太子本来已经对萧兰冷淡了许多,若是再有此人煽风点火,只怕影响到本门对太子的影响力。想到这里,她神色一寒,淡淡道:“张大侠,张总管,看来你是定要庇护这恶徒了,也罢,就让寒幽想您请教。”说罢,飘身向前,向张锦雄后心拍去,张锦雄正被萧兰缠着,李寒幽武功又在他之上,眼看就要被李寒幽击伤,夏金逸突然疯了一般跃起来向李寒幽扑去,李寒幽眼中寒光一闪,一掌劈下,夏金逸的身子宛如断线风筝一般跌落,李寒幽见夏金逸虽然嘴角溢血,神色凄厉,但是双目神光还在,便身形一落,就要补上一章,夏金逸冷冷一笑,抬起袖口,一道银光一闪,李寒幽心中一凛,已经想起崆峒弟子都有几种擅长的暗器用来防身,连忙柳腰轻折,避过一旁,那道银光没入墙壁,不见影踪。李寒幽冷笑道:“看你还有什么法宝防身。”说着再次上前,夏金逸又是抬手一甩,李寒幽这次玉手轻伸,露出银色的护腕,将那枚银光挡住,然后捻住落下的暗器,仔细看去,却是一种五寸长的三棱双锋针,是打磨的雪亮的精钢制成,这种暗器若是中了一支,必然是血流不止,李寒幽冷冷道:“好,本宫就让你自食其果。”说罢手指一弹,那支双锋针向夏金逸射去,其势迅快无比,夏金逸眼看躲避不过,目射怨毒之色,看向李寒幽,那种刻骨的仇恨让李寒幽也不由心中一寒。就在那只双锋针就要射入夏金逸的心口的时候。外面传来怒喝声道:“住手。”

一听到这个声音,不仅李寒幽神色一变,就连萧兰和张锦雄也不约而同住了手,这时,厅门被一脚踢开,李安怒冲冲的走了进来。李寒幽正在庆幸自己已经杀了夏金逸,却见夏金逸已经连滚带爬地向李安扑去,跪在他面前放声大哭道:“殿下,快救属下的性命吧,兰妃娘娘和公主殿下要杀了属下。”

李寒幽一愣,怎么这人还没有死。太子急忙问道:“你没有事情吧,孤一听说就赶了回来,总算十分及时。”

只见夏金逸解开外衣,里面竟然穿着一面护心镜,如今已被双锋针击裂,夏金逸哭诉道:“属下几乎见不到殿下了。”

李安勃然大怒,道:“李寒幽,孤的家事还用不到你插手,你,你走吧。”

李寒幽叹息道:“殿下,你既然不肯接纳忠言,妾身还有什么话说,只是此人实在是留不得的,还请殿下三思。”李安不为所动,冷冷道:“孤知道了,你去吧。”

李寒幽裣衽为礼,又叹息了一声,出门而去。萧兰神色有些紧张,上前吞吞吐吐地道:“殿下,臣妾只是……”还没有说完,一个内侍从外面进来,进门就道:“兰主子,太子妃传话……”话未说完,就看到太子铁青的面庞,他吓得跪了下去。李安冷冷道:“太子妃让你说什么?”

那个太监颤抖地道:“娘娘说,‘既然兰妃你如此胆大妄为,瞒着殿下处置殿下心爱的侍卫,又将本宫的侍女捆了回来,本宫这就上书皇后娘娘,这个太子妃你来做好了。‘”听到这里,李安再也忍耐不住,一挥手,桌子上的茶水被他扫到地上,一片狼藉,李安大怒道:“萧兰,你好,擅自处置孤的心腹不说,还要逼迫太子妃让位,孤明日就上书父皇,将你休弃,孤配不起你这凤仪门高弟。”

萧兰大惊,连忙上前裣衽道:“殿下息怒,是臣妾的不是,求殿下看在臣妾是为了殿下着想,饶过臣妾吧。”

李安虽然愤怒非常,但是想起凤仪门对自己的重要性,自己若是逐出萧兰,只怕这太子之位马上就要不保,不由踌躇起来,这时夏金逸道:“殿下,都是属下不好,得罪了兰妃娘娘,太子妃也是因为此事和娘娘生气,若是殿下允许,让臣给兰妃娘娘赔个不是,娘娘定会饶了属下的。”

李安看看萧兰,萧兰也知道这是一个台阶,连忙道:“本宫不怪罪你了,从今之后你要谨言慎行。”

夏金逸连忙称是,李安满意地道:“这就好了,兰妃,你去太子妃那里赔礼,若是惹怒了她,父皇母后那里都不会答应的。”萧兰已经是十分懊悔,不应该落人话柄,连忙道:“臣妾一定立刻就去,请殿下放心。”

李安满意的点点头,道:“也好,夏金逸,还不和孤回去。”

夏金逸连忙跟着太子离开,临行之时给了师兄一个感激的眼色。等到他走远了,张锦雄才神色冰冷的道:“属下告辞了。”萧兰连忙道:“张总管,都是本宫不好,请你不要放在心上,免得伤害两家情谊。”

张锦雄淡淡道:“娘娘是君,锦雄是臣,怎敢将此事放在心上,我这位师弟身世可怜,或者有些不当的行为,可是他本性善良,还请娘娘网开一面。”

萧兰微微苦笑道:“你真的不知道他都做了什么吗?”

张锦雄冷冷道:“这也正是锦雄想问娘娘的,这样的主上,凤仪门真的认为值得扶保吧,锦雄会将此事回禀师门,请娘娘扪心自问,那些事情,真的怪金逸么?”

萧兰神色凝重,没有答话,看着张锦雄远去的背影,她低声道:“这次真是失策,我可要好好补救,否则师父怪罪下来,我可怎么办呢?”

\chapter{第四章 如烟往事}

离开了花厅,李安松了口气,看了一眼夏金逸,如果不是这个属下他实在不愿舍弃,他也不愿和萧兰、李寒幽翻脸,再说,这两人不顾自己的颜面,也着实可恨,若是真的让她们杀了夏金逸,自己岂不是成了连属下也无法庇佑的无能之辈,看来鲁敬忠说得不错,凤仪门一定要好好防范,否则只怕自己终有一日成了人家手中的木偶,一个傀儡皇帝。想到这里,他温和地道:“金逸,你去召鲁少傅过来,孤有些事情要问他,今天晚上就让你师兄守卫,你不妨出去散散心,也是压压惊。”

夏金逸感激涕零地道:“多谢殿下厚爱,属下情愿服侍殿下。”

李安笑道:“放心吧,今晚我不会有时间了,你这一年来几乎寸步不离,想必也是很劳累了,今日之事,孤也没有什么法子补偿你,就放你一天假,出去好好散散心,多带几个属下,免得有人趁机暗算。”

夏金逸连忙拜谢道:“多谢殿下恩典,属下这就去请鲁少傅。”

李安摆摆手道:“你去吧,有些事情孤也无可奈何,你也不要挂在心上了。”夏金逸眼色一动,低声道:“属下身份卑微,生死事小,可是殿下的尊荣却被人踩在脚下,是可忍,孰不可忍。”李安神色微微一变:“罢了,不要多说了,孤先去太子妃那里看看她,你去请鲁少傅吧。”夏金逸恭恭敬敬的退了下去,低垂的目光中满是得意之色。

坐在很久没有进入的书房里,李安静静的看着坐在对面的鲁敬忠,良久,他才淡淡道:“你也要劝谏本王杀了夏金逸么?”

鲁敬忠恭恭敬敬地道:“夏金逸生死臣并不关心,只是凤仪门若因此事和殿下离心,这就得不偿失,若是殿下舍得,臣自然是希望殿下不要因此得罪凤仪门的。”

李安恼怒地道:“凤仪门也太不把孤放在眼里,夏金逸不过是个幸臣,既不能伤害孤王的大业,也没有和她们争夺权势的本钱,她们也太嚣张了。”

鲁敬忠笑道:“这也是迁怒罢了,殿下你作的一些事情在臣来说只是风流韵事,可是在她们来说未免难以容忍,可是又不能责怪殿下,只好找夏侍卫出气了,殿下如今已经保住了面子,接下来就该好好安抚她们一下,现在局势对我们并非十分有利,殿下不可自毁长城啊。”

李安点点头道:“少傅说得有礼,你说当日究竟是谁杀了梁谨潜,害得孤有口难辩?”

鲁敬忠皱眉道:“说起这件事臣也想过,想来想去,除了雍王,还有两个人嫌疑最大。”

李安感兴趣地道:“我上次问你,你说雍王嫌疑最大,只因杀了梁谨潜,得益最大的就是雍王,可是如今你又说多了两个人,这个人是谁呢?”

鲁敬忠淡淡道:“齐王李显、庆王李康都有可能。”

李安一愣道:“庆王虽然和凤仪门有仇,可是对孤倒是恭恭敬敬的,怎会作出这种事情,还有齐王,他和孤是一条船上的人,怎会如此。”

鲁敬忠冷笑道:“说庆王有嫌疑,是臣查出近年来庆王在京城安插了不少人手,他本是天家骨肉,却因为凤仪门的人而远谪东川,虽然益州富足,可是那里比得上长安繁华锦绣,再说,杀母之仇不共戴天,如今凤仪门保着殿下,他自然就要和殿下作对,当初凤仪门偏向雍王的时候,他不也处处和雍王为难么。臣近日捕获了庆王的探子,严刑拷问之下,得知当年梁谨潜被鸩杀的时候,庆王手下的第一高手叶天秀就在京城,若不是为了浑水摸鱼,他怎会让这个保镖离开身边。”

李安神色一动,冷冷道:“若真的是他,你认为该如何处置,要不要我在父皇面前说几句话,处置了他?”

鲁敬忠摇头道:“殿下不可,庆王没有继承大统的可能,所以殿下理应引以为援,何况将来殿下还要靠庆王制衡凤仪门呢,怎能对付他,再说也没有真凭实据证明是庆王所为,只是这人殿下也应该小心才是,这些日子,庆王的人在长安越发放肆了。”

李安点点头道:“那么少傅怎么又会想到齐王呢?”

鲁敬忠道:“齐王殿下本来是殿下的左膀右臂,可是近年来,殿下不免对他有些冷淡,其实这也难怪殿下,齐王虽然总是跟雍王殿下为敌,可是从来也不肯做过分的事情,手下总是留一分情面,殿下怀疑齐王也是理所当然,这一年来,齐王几次要求到边关镇守,都被您拒绝了,在齐王看来,殿下是故意阻挠他立功,而在殿下看来,齐王却是想避开和雍王针锋相对的场面,其实臣觉得殿下和齐王都没有错,齐王虽然口中不说,但是对雍王确实有些忌惮,而殿下不许他出征,也是不愿他威名更盛,殿下也防着齐王呢,毕竟两虎相争,必有一伤,兰妃娘娘就是这样劝您的吧?”

李安赧然道:“我也觉得兰妃说得不错,而且齐王也太嚣张了些,本王总觉得他有些不敬。”

鲁敬忠捋着胡子道:“这个么,殿下做的也不算错,只是若能好好劝慰一下齐王就更好了,毕竟齐王可是您擎天保驾的大将,您总不好开罪了他,若没有齐王的大军,只怕雍王早就谋反了。”

李安深以为然,道:“你说得是,过几天我请六弟过来,好好劝劝他,让他安心留在京城,将来还怕没有仗打么。”

鲁敬忠意味深长地道:“其实还有一个人,殿下也该想想法子拉拢。”

李安看向鲁敬忠,鲁敬忠笑道:“夏侯沅峰。”

李安失笑道:“夏侯早就是本王的人了,你又不是不知道,他们父子也早就投靠了本王?”

鲁敬忠冷笑道:“殿下现在手上大部分力量都是凤仪门的,凤仪门的人听得是门主梵惠瑶的命令,今日凤仪门主支持您,她们就帮您,明日凤仪门主支持了齐王或者庆王,她们也就会改弦易辙,殿下这一年来暗中招揽了不少亡命,不就是为了建立自己的武力么,夏侯沅峰武功高强,又得皇上宠爱,殿下若能让他真心相从,那么他就是殿下手上的利刃了,如今禁卫军北营统领裴云已经是雍王的人了,虽然他对齐王还是那么尊重,可是他对殿下可没有什么好感,夏侯沅峰曾经击败过裴云,殿下不把他收到麾下,可就太可惜了,只要您礼贤下士,把夏侯沅峰拖上我们这只船,到时候可是多了一个武功高强心机深沉的好手啊,而且还不需要通过凤仪门就可以指挥他。可是殿下却对他若即若离,若是放过了此人,真是太可惜了。”

李安有些不安,他也不便说自己心中有些排斥夏侯沅峰,只因为这人总是十分神秘,无法看透。他说道:“你说,我该怎么拉拢他呢?”

鲁敬忠目光下垂,道:“听说殿下最近得了一柄软剑,削金断玉,十分珍贵,夏侯沅峰最喜欢软剑,据说曾经派人专门到各地搜求。”

李安笑道:“我当是什么宝物,原来不过是一把软剑,这把软剑虽然珍贵,可是对本王来说不过是件玩物罢了,明天我就让人送过去。”

鲁敬忠行礼道:“殿下从谏如流,臣感恩不尽。”

李安笑道:“好了,这一年来,孤也忍得够了,你也该想个法子让孤出了这口恶气再说。”

鲁敬忠笑道:“这有何难,如今事过境迁,正是我们反击的好时候,如果殿下觉得没有妨碍的,就从裴云着手。”

李安皱皱眉道:“一个小小的禁卫统领,能起什么作用,父皇对他也很欣赏,我看还是换个人吧。”

鲁敬忠道:“选中裴云,一则他现在和雍王走得很近,对他下手,也是杀一儆百,其二,这人让凤仪门丢了面子,我们可以通过凤仪门对他下手,这样一来凤仪门和少林接下深仇,殿下就可以更好的将凤仪门控制住,而且,齐王殿下对裴云也很赏识,正好借此警告齐王一下,到时候如果齐王为他求情,殿下就可以卖个人情给齐王,反正殿下只是想去了他的官职,至于他的性命倒也并非紧要。”

李安点点头道:“那么我们从何着手呢?”

鲁敬忠微微一笑,凑近李安耳旁,低声说了几句话,李安喜笑颜开,道:“你告诉夏侯,如果事成,孤定然重重有赏,绝不会亏待他的。”两人相视而笑,笑声中带着不尽的残忍意味。

月夜良宵,佳人在侧,夏金逸却是愁容满面,躺在软绵绵的牙床之上,他愣愣的望着房顶,今日他带着几个侍卫到了这家有名的青楼,和众人宴饮之后,他醺醺大醉的扶着一个绝色名妓进了绣房,但是进房之后他却清醒了过来,一番云雨之后,那个名妓柔顺的在他身边依偎着,可是夏金逸却心中空荡荡的,在他来说,他更想在太子府里抱着绣春好好地睡上一觉,不过他也知道太子既然有话,他还是出来的好,只是今日的生死惊魂让他仍然心有余悸,此刻他更加迫切的想见见江哲,否则他不知道接下来该作些什么。正在他胡思乱想的时候,突然有人轻轻叩动门扉。

夏金逸一惊,回头看看那个妓女已经熟睡,却还是不放心,轻轻的点了她的穴道,然后走到门口,自己站在门后,轻轻拉开了房门,只见一个青衣小婢低头端着一壶茶走了进来。那个小婢看了一眼帷帐低垂的床榻,将热茶放到桌子上,然后似乎便要转身出去,眼睛余光却看到夏金逸冷冷的看着她,她似乎受了惊,捂住了心口。

夏金逸歉意的一笑,让开了门口。那个小婢裣衽为礼,拿着茶盘走到门口,就要出去,夏金逸正要让开,那个小婢突然从袖中拿出一筒袖箭指向夏金逸,夏金逸身子一震,他知道那是三十步内可以轻易穿透轻甲的袖箭,如今两人距离不过三步,自己就是想躲避也逼不开的。但是这个小婢既然没有出手,说明还有转圜的余地。夏金逸从容的看向这个小婢,她已经抬起头,微笑着看向他。

夏金逸却是一愣,原来这个小婢竟是他认得的一个人,江哲的随从之一,赤骥,赤骥的相貌本来清秀俊雅,身材又不高,扮作侍女居然十分神似,夏金逸松了口气,低声道:“赤骥小哥,你吓死我了。”然后又激动地道:“怎么,大人要见我么?”

赤骥笑道:“公子就在隔壁等候,请夏公子过去。”

夏金逸看看身上,这般模样,怎么见人,可是若是清洗之后,明日不免引起那个妓女怀疑。想了一想,拿起长袍,披在身上,跟着赤骥出了房门,迅速跨进旁边的一间厢房。进去之后,只见江哲一身青色丝袍,坐在椅子上,意态悠闲的看着桌子上一副棋盘,而在他旁边,一个青衣秀雅少年侍立着相陪下棋。

夏金逸一见到那两人,便上前拜倒道:“夏金逸叩见大人金安。”

我站起身来,上前伸手相搀道:“夏公子不用多礼,江某担当不起。”

夏金逸恭谨的站起身来,仿佛奴仆属下一般恭顺,我心中不由一喜,原本我还想他可能会不愿听从我的命令,所以准备了威胁逼迫的法子,想不到他如此识相,看来我倒不用强迫了。

示意他坐下之后,我笑道:“这一年多来,夏公子深得太子殿下宠爱,想不到还记得故人。”

夏金逸站起身道:“上次别过大人之后,金逸日夕渴望再见之期,这一年来,金逸竭力周旋,只希望能够对公子有所帮助,如果大人能够实现金逸一个愿望,那么金逸情愿粉身碎骨,以报大人恩情。”

我若有所思的看向夏金逸,这就有了答案,从前我可是强行迫他效力的,这一年来,他荣宠备至,却依然不忘旧约,我本来有些奇怪,可是听他这番话我才心里有谱,若非心有所求,怎能如此。

我也不急迫,缓缓道:“请夏公子详细道来,若有所求,江某定然会仔细考虑。”

夏金逸下拜叩首道:“若是大人能助金逸让那靖江公主身败名裂,身死囹圄,不论大人有何吩咐,金逸无不听从。”

我微微一愣,道:“夏金逸,你本是江湖浪子,李寒幽却是宗室郡主,如今更是公主之尊,论起江湖地位,更是凤仪门高弟,怎会与你有仇。”

夏金逸眼中闪过怨毒之色,惨然道:“什么宗室郡主,公主之尊,李寒幽不过是个假充凤凰的山鸡,虽然羽毛绚烂,却是心肠歹毒,忘恩负义,背情负盟之人。”

我心中一震,道:“你详细说来,若是真情,江某必然为你作主。”

夏金逸神色变得酷厉非常,他缓缓道:“夏某原名夏全,家中三代一脉单传,虽然血脉单薄,但是一家人其乐融融,家乡偏远,当年中原征战也没有波及到寒乡,所以一家人共享天伦之乐,因为担忧血脉断绝,所以在金逸五岁的那年,家父母收养了一个女孩,相等我一成年之后就让我们完婚,这个女孩的父母也是同乡人,只是家境贫寒,又连续生了六七个女儿,无力抚养,所以我家就多了一个童养媳,我那时候年纪幼小,只当是多了一个妹妹,这个女孩却是相貌秀丽,非同寻常,更是聪明过人,先父母十分疼爱,让她和我一起读书,她过目成诵,一目十行,我也自愧不如,十二岁那年,我因缘际会,跟着一位崆峒道长去学武,父母也知道如今是乱世,我若学点武功可以防身,所以很高兴,当时她只有七岁,还拉着我要我常常回家看她。”

“深山学武,不知岁月甲子,等我刚刚有所成就终于得到师父许可回家探亲,那一年我十六岁,她十一岁,虽然年幼,可是也已经知道人事,那一次,因为我母亲多病,为了冲喜,我和她在父母主持下完了婚,虽然因为我还要练武,她年纪还小,没有圆房,可是我们已经名分上成了夫妻,婚后不久,我就再度回到崆峒,可是我们虽然年幼,却也是许下白首盟约。谁料不到两个月,我就接到族中的书信,说我父母亡故,我浑浑噩噩的赶回家中,问过族人才知道,就在我走后不久,有一天有些佩剑女子路过敝村,据说是因为走错了路,家父忝为族长,因此热情款待,谁料她们见了我的妻子,说她资质无双,就要把她带走,我父母自然不肯,可是她们说动了我的妻子,我不知道她们说了什么,可是最后我的妻子心甘情愿地跟着她们走了,只留下她们强行留下的几百两银子,说是替我妻子赎身。我母亲因此忧愤而死,没有多久我父亲竟然也发病死了。我验了父亲的伤势,竟是被人用阴手伤了经脉,是谁下的手还用说么?我也想报仇,可是我不是蠢人,问过那些女子的装束,我就知道了她们的身份,除了凤仪门,哪里还有那么多使剑的女子,可是崆峒却和凤仪门有着盟约,我就是练武练得再好,又能怎么样,我跟本就报不了仇。所以我心灰意冷,从此消沉下去,不到半年就被逐出师门。在江湖上漂流多年。”说到这里夏金逸已经是泪流满面。

我神情凝重地道:“你是说李寒幽就是你的妻子,你可有证据么?”

夏金逸抬头道:“不会错的,她虽然气质大变,可是我绝不会认错,她就是我的妻子乔翠云,虽然她如今风华高贵,可是我和她从小一起长大,她的相貌还留着过去的痕迹,她的一些小动作我也不会认错,若是大人不信,小人还知道她腰间有一枚红痔。”

我真是惊呆了,想不到李寒幽竟然不是宗室出身,那么她怎么会成为靖江郡主的呢?

\chapter{第五章 安排金饵}

大雍武威二十四年,辛未后三年,霍某以一己之力,搅乱大雍江湖,血流成河,其中得力者多不为人所知,仅有一人以霍某义子霍离之名闻于大雍。其时,距霍某刺楚王未及一载也。

——《蜀史·纪城列传》

想了片刻,我心中释然,无论如何,现在李寒幽已经是这样的身份,不论凤仪门和靖江王有什么勾结,这些都是过去的事情,还是看看这个情报有什么帮助吧,可惜夏金逸的证词分量不够,否则定然可以让皇上褫夺李寒幽的公主身份,淆乱皇族血统,其罪非轻,不过没关系,这个消息只要秦大将军信了就行,只是不能轻易走露,得等到适当的时机再拆穿李寒幽的真正身份。

不过为什么李寒幽没有认出夏金逸呢,按理说李寒幽的相貌变化应该大过夏金逸的,我将疑问提出。

夏金逸低着头,两滴眼泪跌落尘埃,说道:“李寒幽自幼就是天生丽质,相貌改变并不多,而且寒幽这个名字本来是她自己起的,当年我们一起读书,她嫌自己的名字土气,便自己取了这个名字,只是怕我父母责怪,所以这件事情只有我和她知道,所以我一听到这个名字,心中就有些怀疑,只是不敢想会有这样的事情发生罢了,所以一见之后,小人就敢肯定她的身份。至于她没有认出小人的,是因为十六岁之前,小人性情木讷,肤色微黑,身材粗壮,与现在截然相反,现在能有这副相貌,是小人跟着第二个师父的时候,他用秘药改变了小人的肤色,又不准小人再练习外家功夫,改练内家心法,不过能有今日的相貌,小人也是没有想到的。”

我听了忍不住笑道:“令师梦道人怎么对弟子的相貌很重视么?”

夏金逸没有追问江哲怎么会知道他的第二位恩师的身份,事实上,如果江哲不知道他才会觉得奇怪呢。他回答道:“这个,家师说他的弟子可以武功不好,可是一定要风流倜傥才行。当年小人已经放弃了复仇的希望,也不愿意辛苦学武,所以反而很高兴跟着他老人家学习那些雕虫小技。”

我深深的看了夏金逸一眼,没有说话,或许他的师父另有深意吧,不过这个我要详细调查之后才能肯定。言归正传,我沉声说道:“雍王殿下和太子已是你死我活的局面,凤仪门既然党附太子,自然也在铲除之列,你且放心,不论你有无可能活到那一日,李寒幽绝对不会有好下场的,这一年来,我不想你泄漏身份,所以从不和你相见,今日也只有片刻时间,你的事情我知道得很多,将来大功告成,我必然不会薄待你,现在有一件事情要你去办,这件事情十分危险,你可能也会有生命危险,本来我是不准备让你去做的,可是也只有你能够做的神不知鬼不觉,你可愿意冒险。”

夏金逸神色从容,道:“小人早已将生死置之度外,太子暴虐,小人深知,若是有朝一日,他登基称帝,只怕天下百姓都会受苦,我虽然不是什么仁义之士,可是若能尽一份绵薄之力,帮助雍王殿下夺嫡,小人死也甘心。”

我深深的看了他一眼,递给他一条翠绿丝帕。夏金逸接过一看,神色大变,却没有说话。我将安排详细的说了一遍,夏金逸面上神色又是恐惧又是佩服,问道:“大人怎知道此事,小人相信做事严密,绝无外人得知。”

我但笑不语,想来也不用告诉他小顺子偷入禁宫收了两个弟子吧,虽然那两个可怜的孩子武功还浅,可是手脚却灵便,再加上心思灵巧,居然探到了这个天大的秘密。

夏金逸见我不说,只得珍而重之的收好丝帕,说道:“小人只能尽力而为。”

我见他答应,便拿出一个瓷瓶,道:“这里面有两颗药丸,到了那日,你先服下那颗裹着绿色腊衣的药丸,那是一颗护心丹,想来那日恐怕你会是被迁怒的人,但是奉命杀你的人不可能用兵器,随随便便在皇上面前溅血杀人是不敬之罪,若是用拳掌,我敢说可以让你保住性命,然后你再偷着服下那颗黑色腊衣里面的药丸,就可以生机断绝,浑似死人,这样我自有法子把你救走。从今之后虽然不能露面,可是我想到了今日你也应该不希望再在混浊的官场混下去了吧,若是你还是想要一个前程,等到日后我必然不会亏待你。”

夏金逸眼中闪过一丝感激的神色,道:“多谢大人顾及小人的性命,小人若能报得大仇,什么荣华富贵都不希罕,只是小人希望能够亲自看到李寒幽遭到报应。”

我淡淡一笑道:“这有何难,事成之后,你脱身出来,我会安排你隐藏起来,等到日后你自然可以得偿夙愿。不过事情也未必到了这一步,如果太子不肯上钩,或者你没有生命之险,你就继续服侍太子好了,记得不论如何,都要忠心耿耿,不可流露出势利的意味。若是你还能留在太子身边,今后你还是自行决定如何行事,只是记得,如果有机会,不妨挑拨一下太子和鲁敬忠的关系。”

夏金逸犹疑地道:“如今太子对凤仪门和齐王心中都有嫌隙,正是对鲁少傅十分倚重的时候,恐怕不大容易挑拨他们君臣的关系。”

我笑道:“也没有什么难得,大凡有才华的人不免恃才傲物,鲁敬忠心思阴险,太子又是心胸狭窄的人,你只要多多夸赞几次鲁少傅计谋过人,太子心中就会有了嫉恨。”

夏金逸半信半疑地道:“小人明白,必定奉命行事。”

谈完了事情,夏金逸悄然离开了,我心中明白他并不十分相信我的判断,不过他也不会阳奉阴违,毕竟我的离间法子对他没有什么损害,夸奖鲁敬忠几句对他有什么损失呢。

小顺子看看我的神色道:“夜深了,公子是就在这里休息一夜,还是现在回去?”

我疲倦地道:“现在回去吧,我不喜欢这个地方,满屋的脂粉腻香,令人闻了就觉得不舒服。”

小顺子拿过披风,我披上之后,接过纱笠,走出了房间,穿过侧门,外面黑暗的小巷子里面停着一辆外形普通的马车,小顺子扶着我进到车里,自己也跟着进来,放下车帘,然后车子起动了,我知道周围有我的近卫保护,带队的人是荆迟,这一年来他几乎除了在军营就在我身边,每次我出门他都要抢着跟随,也不知道是不是被我罚抄书抄糊涂了。

马车左拐右转了半天,夜深人静,街上几乎没有人,所以马车的速度渐渐快了起来,我挑开窗帘,看见两侧街道树木飞快的倒退,两边各有六名侍卫骑马紧跟,我知道荆迟必然在后面压阵,虽然对长安街道并不熟悉,可也知道这里已经离我密会夏金逸的地方很远,所以他们才放心飞车赶路,今日的事情,跟来的是我的近卫中最受宠信的几人,不过他们也不知道我为什么到那个所在,事实上我为了防止有人发觉我出现在那里,特意安排了和另外一个人相见,当然那人是有足够理由和我密会的。如果太子的人发现了那个人的影踪,想必会十分头疼吧,那人就是这一年来行踪不定,声名远播的“霍纪城”。

一年前,我命人杀了霍纪城灭口,却又伪造出他仍然活在世上的假相,这一年来“霍纪城”只做了两件事情,可是却让凤仪门伤透了脑筋。

第一件事情,是凤仪门利用锦绣盟余孽设下了一个圈套,只等他自投罗网,可是霍纪城虽然如他们预料的一般入了圈套,可是却是将计就计,将参与其事的凤仪门弟子和她们请来的高手一网打尽,至于用的什么计策就无人知道了,因为所有人都只剩了一个石灰腌制过的头颅,挂在路边示众。而从此锦绣盟剩下的精锐就销声匿迹,声不见人,死不见尸。直到两个月后第二件事情发生。

那是一件很蹊跷的事情,洛阳乃是大邑,城内黑白两道自然是错综复杂,两大世家罗家和丁家表面上和和气气,都是尊奉凤仪门旗号的名门正派,暗地里却是争夺的不可开交,另外还有一些在两家门缝里面讨食的小帮派,两大世家不愿两败俱伤,便通过这些小帮派争斗,谁知洛阳城里突然风云震动,一个小帮派的势力突然飞速膨胀起来,将那些小帮派吞并了不少,这下两大世家可不能坐视,他们这一联手打压,谁知道那个小帮派居然立刻投靠了罗家,这下丁家担心罗家势力大增对自己不利,不免要暗中作些手脚,可是没等他们动手,罗家的几个重要人物都遇刺身亡,这样一来,罗家自然也不肯善罢甘休,丁家又只道罗家借机扩张势力,双方连番血战,而那个小帮派的二头顶也被丁家收买过去,洛阳城顿时血雨腥风,百业不宁,直到凤仪门的三姑娘“慈心观音”凤非非、七姑娘“芙蓉剑”谢晓彤到了洛阳,她们从中排解,大家坐下来详谈之后才发觉有人从中挑拨,那个小帮派就成了众矢之的,当两大世家联合攻破这个小帮派的总舵的时候,却发现帮主被人刺杀在卧室之内,仔细盘问之后,发现只少了一个叫做霍离的少年,帮众只知道这个少年是帮主新收的侍卫,也是从他来到了帮中之后,这个小帮派才开始大肆发展起来的,而且有人怀疑这个少年正是帮主的军师,只是他年纪轻轻,难以令人相信这个事实。

若是事情就这样结束,虽然令人满腹疑窦,但是也只能就这样算了,最多不过追查那个少年的来历,可是问题是在那个帮主的来往书信中发现了一封密信,却是霍纪城写给他的,信上只是简单写了几句不着边际的话,只是最后说遣义子霍离前来相助。这封信令众人面面相觑,谁会想到一个漏网之鱼会有这么狠辣的手段呢?

自此之后,凤仪门令出如山,四处缉拿霍纪城,可是虽然官府和凤仪门都严令缉拿,可是霍纪城又是全无消息。可是经此一事,霍纪城对中原武林来说已经成了仅次于魔宗的祸害,最可怕的是,他将锦绣盟重新改组之后,锦绣盟也是若隐若现,虽然在凤仪门和大雍朝廷的追缉下还是会有一些人落网,可是这些人若是不幸落网,不是拼个同归于尽就是自戕当场,就是能够活捉一两个,可是这些人大多都十分茫然,既不知道自己再做些什么,也不知道如何和别人联系,他们都是按照从某些地方得到书面指令行事的,可是到了这里就再也查不下去。可是从已经得到的情报,可以看出锦绣盟已经成了一个神秘可怕的组织。所以凤仪门主的大弟子闻紫烟再次出现江湖,负责追杀锦绣盟中人,凤仪门传令江湖,凡是锦绣盟中人,杀之无赦。从那以后,霍纪城虽然行踪偶有出现,可是总是很快就影踪全无,而“血手罗刹”闻紫烟所到之处却是血流成河,只因霍纪城心机深沉,总是留下一些和各地武林魁首“勾结”的线索,而宁可杀错,不肯放过的闻紫烟就成了刽子手,到了后来,大雍江湖已经是听到霍纪城的名字就谈虎变色。直到各大门派纷纷传书凤仪门主,婉转劝说,凤仪门主才招回了闻紫烟,这件持续了半年多的事情才渐渐落幕。如果知道“霍纪城”到了长安的消息,不知道会因起怎样的恐慌呢?

我得意的一笑,谁知道这个霍纪城是我一手策划的呢?当初我觉得霍纪城这个身份可以利用,才让寒无计冒着险去灭口,然后让小顺子配合陈稹、寒无计将凤仪门前来诱捕霍纪城的高手一锅端了,这些虽然靠着小顺子武功高强,可是秘营那些已经成长的少年才是主要的武力,凭着接近一流的武功和我调教出来的军阵,再加上刺杀暗算,将这些各自为政的高手一网打尽,而且因为霍纪城以前太谨慎,造成大部分锦绣盟中人对他的体貌特征不十分熟悉,凭着他留下的令牌,陈稹接收了锦绣盟,将一些生性善良被迫加入锦绣盟的人全部遣散,留下一些生性凶残的盟众,然后使用雷霆手段把他们彻底收服,给他们指令让他们潜伏在大雍各地,其实这些任务都是一些莫须有的任务,他们为了完成这些任务,必须收敛凶性,隐藏在市井当中,既不敢作恶也不敢潜逃,因为陈稹在他们身上下了我提供的剧毒,为了每月一次的解药,他们绝不敢逃走,就这样把这些凶人分别“软禁”起来,而且还可以利用他们的武力。

然后我就开始了第二步计划,洛阳城的罗家和丁家虽然面和心不和,可是他们都是凤仪门的帮凶和支持者,盗骊奉命自称霍离混进了一个小帮派,凭着我的教导和陈稹寒无计的指挥,顺利的挑起了他们的纷争,不仅留下了霍纪城在暗中伺机待动的印象,而且成功的削弱了洛阳城两大世家。前些日子,我得到雍王殿下的消息,现在的洛阳将军是雍王的人,已经成功的掌握了洛阳的控制权,不过我可没有告诉雍王霍纪城的真相,否则我这个雍王司马却是叛逆组织锦绣盟的幕后人,这成什么话。而且接下来引着血手罗刹四处大杀特杀,虽然死的都是江湖中人或者各地世家豪霸,但是雍王也不免会觉得过分。不过这场杀戮我和凤仪门倒是各有所获,我成功的消减了凤仪门的势力,也让凤仪门渐渐从一个清高的形象蜕变成了血腥的象征,让他们想起凤仪门就是靠着刺杀和血腥起家的,不过凤仪门也成功的将现在江湖上渐渐涌现的反对势力血洗了一遍,如果不是凤仪门主这样配合,我的目的也不大可能这么实现,雍王曾经对我说过担心江湖高手损失太大,唯恐伤及军方战力,毕竟军中许多高手都是从江湖中来的。我趁机让雍王示意军方开始趁机招揽高手,并声明若是加入军方,那么就不许那些江湖人前来骚扰,结果不少江湖中人为了躲避风浪而从军,这件事情得到了秦大将军和齐王的支持,谁不想趁机增强自己的武力,结果似乎谁都没有占到便宜,但也谁都没有吃亏,若说可怜的,大概就是那些无端涉入纷争的人么,不过他们不是江湖亡命就是地方上的豪霸,他们死得多些,对平民百姓也不是什么坏事,所以我也就把同情心丢到脑后了。

若是霍纪城进京的消息传了出去,不知道太子会不会心惊肉跳呢?

我正在盘算着即将进行的计划,突然马车前面传来在前面开路的周武的呵斥声,然后就是一声惊呼,接着马车突然停住,毫无准备的我身子向前冲去,眼看就要撞到车门上,幸好小顺子手疾眼快,一把将我拽住,我平息了一下心中惊惶,看看小顺子,说道:“怎么回事?”

\chapter{第六章 东海来客}

南楚同泰二年,哲于长安夜行,路遇庆王近卫叶天秀,东海侯姜永麾下勇将方远新。

——《南朝楚史·江随云传》

小顺子挑起了车帘,只见保护我的十二名侍卫已经手握刀柄,将马车护住,而在前面开道的周武周侍卫正在指着冲撞车驾的两人说道:“你们是什么人,竟敢拦阻我等车驾。”

我从车帘缝里望去,只见在车驾前面站着两个男子,一个穿着灰衣,相貌俊秀,身佩长剑,另一个穿着黑衣,虽然相貌也不错,可是肤色呈现古铜色,一双手正握着周武的马缰,我一眼看见他手心满是淡淡的伤痕,心中一动。目光一转,已经看到那个灰衣男子怀中抱着一个六七岁的小男孩,衣衫褴褛,神色虽然激动,但是倒没有多少恐惧。

这时,只听见周武厉声道:“如今夜深人静,我等虽然纵马飞奔,也很难伤到人,这个孩子虽然出现的突然,但我自信可以及时住马,你们何必多管闲事。”

那个黑衣男子怒道:“不论何时,怎可在城中骑马飞奔,若无我力止奔马,只怕这个可怜的孩子已经伤在马蹄之下。”

周武正要争辩,这时候荆迟从后来绕了过来,瞪了周武一眼,冷冷道:“深夜飞驰,没想到街上还会有人,这是我们的不是,荆某代我这位兄弟道歉,两位既然有胆子管闲事,想必也是好汉子,敢不敢跟我们走一趟。”

那两个男子相视一眼,都看出了对方眼中的犹豫,这一行人簇拥的马车虽然十分朴素,但是只见制作精良就知道不是寻常人家用的起的,而且这些护卫虽然穿着便衣,可是却都气势不凡,只见他们坐在马上的姿势就知道他们乃是军人出身,而且个个武功不凡,这样一队侍卫,不是公侯之家是绝对没有的,他们身份都有碍难之处,两人交换了心意,那个灰衣人淡然道:“既然你们已经道歉,也就罢了,我们还有事情,就不打扰了。”

说着两人就要离去,荆迟朗声一笑,一挥手,八个侍卫从左右纵马冲上,很快就将这两人围在当中,那两人脸色大变,灰衣人眉头紧皱,黑衣人却是面露杀机,这时荆迟道:“荆某在长安也有多日,一看两位就是外乡人,这里是天子脚下,帝都之中,就是外地杀人越货的大盗到了这里也得循规蹈矩,没有几个敢在夜间行走的,毕竟若是遇到巡夜的禁军不免麻烦,两位这么大胆,想必是武艺高强,高来高去不成问题的了。”

灰衣人冷冷道:“怎么,长安没有夜禁,我们黑夜行走是我们的事情,就因为我们管了闲事,你就要借题发挥么,可是想把我们送官么?”

荆迟笑道:“这倒不是,只是请两位到我们那里做客,若是两位都是清白之人,荆某不仅向两位致歉,还要和两位交个朋友,以后在长安若有什么碍难,只要荆某帮得上忙,绝无二话。”

那个灰衣人手卧剑柄,神色凝重,那个黑衣人也将手放到腰间,眼看就要出手,可是他们看这些侍卫个个虎视眈眈,而且荆迟又是虎目含威,冲天的杀气已经将两人笼罩在其中,不由心中十分不安,就是能够冲出重围,只怕也是形迹全露,正在犹豫的时候。这时候车帘一挑,一个青年探身出来,他披着黑色披风,掩住了衣着,相貌十分文弱清秀,他就那么在杀气满盈,箭在弦上的时候显身出来,微笑道:“荆将军,住手。”

两人心中一动,都望了荆迟一眼,眼中闪过了然之色,望向我的目光却是带着疑惑,我更加觉得自己的判断没错,便笑道:“下官雍王麾下,天策帅府司马江哲,方才属下多有得罪,江某代他们向两位致歉。”说着,我拱手行礼。

那两人也不约而同躬身还礼,那个灰衣人眼中闪着莫名的光芒,道:“原来是江大人,在下早有所闻,冲犯车驾之罪,还请见谅。”

那个黑衣人神色又惊又喜,却不说话,我看了他一眼笑道:“叶兄,方兄在长安可要小心,殿下对两位的主上并无恶意,可是若是方兄行踪泄漏,我家殿下也不便手下留情,长安虽好,却难久居,还是请快些离去吧。”

我刚说了一句“方兄”,那两人同时身子一震,全身功力已经凝聚,就要出手,但我接下来的话却让他们松了口气。那位黑衣人犹豫了一下,躬身下拜道:“江大人,方某入京也是情非得以,不知道大人可否借一步说话。”

我倒是一愣,看穿这两人的身份本是偶然,那叶天秀本是庆王属下,也曾经多次秘密入京,我见过他的画影图形,认得他本是应该,那个姓方的却是我猜出来的,这人肤色特殊,显然是常年在阳光下曝晒而成,再见他手上有常年收帆被绳子划出的痕迹,再根据和叶天秀交好的因素,我才猜到他的身份。本来想说几句好话,表达善意之后就让他们离开,免得多了一些不可控制的变素,想不到这个方远新竟然要和我叙谈,这事如果传了出去,姜永毕竟还是叛逆,虽然雍帝根本不想为难他,但是对我终究不大好,但见他目光中充满了恳求之意,我心一软,道:“方兄请到车上一叙。”

方远新看了叶天秀一眼,低声道:“你先回去吧。”

叶天秀也低声问道:“他是雍王亲信,你要考虑清楚。”

方远新苦笑道:“少主性命要紧,这也顾不得了,雍王总不会趁人之危吧。”

方远新踏上了马车,叶天秀忧虑的看了我一眼,行礼告辞,就要带着那个孩子离开。

我扬声道:“且慢。”

叶天秀心中一凛,回身道:“大人有何吩咐?”

我笑道:“叶兄在长安只是过客,这个孩子还是交给江某处置吧。”

叶天秀心中一宽,道:“那就拜托江大人了。”说罢迅速的隐入夜色当中。一个侍卫策马上前,一弯腰将那个孩子提起放在马上,那个孩子倔强的挣扎了一下,充满敌意的目光望向那个侍卫,那个侍卫哈哈一笑,拍了拍他的脑袋。

方远新刚踏进车厢,就看见一个相貌清雅阴柔的少年坐在那里,冷冷的看了他一眼,那冰冷的目光让方远新觉得全身似乎被一桶冰水浇个透心凉,他立刻知道了此人的身份,“邪影”李顺,这个武功邪异惊人,却甘心屈身为仆的绝顶高手。

我见方远新如坐针毡的表情,给了小顺子一个眼色,他周身的杀气立刻收敛不见,方远新只觉得松了一口气,心道,邪影果然不同寻常,我见他已经平静下来,这才道:“不知道方兄想和江某说些什么呢?”

方远新神情黯然道:“江大人既然知道在下的身份,就该知道在下的主上是谁?”

我微微一笑道:“江某自然是知道的,只是方将军既然知道如今贵上仍然是大雍的钦犯,为何却要和江某详谈,若是此事泄露出去,只怕江某就是想要放手也不可能了。”

方远新道:“方某正是见江大人颇有回护之意,才敢和大人商量。”

我回想起他刚才和叶天秀交换的低语,心中一动,笑着问道:“请问可是有什么事情需要在下效劳么?”

方远新道:“不敢相瞒大人,我主上年近不惑,只有一点骨血,不料前些日子少主出海,被海中一种名叫“胭脂玉”的海蛇所伤,生命垂危,虽然我主麾下也有名医,可是却都束手无策,只能眼看着少主日日受毒伤折磨,虽然性命勉强保住,却是求生不能,求死不能。主上也曾经派出手下四处寻找名医,可是人人都说无能为力,最后主上只希望能够找到医圣桑先生,可是桑先生自从在长安神龙一现之后就再无踪影,方某奉命到长安找寻线索,也是没有得到任何消息,可是却得知江大人曾经从桑先生学医,据说医术精深,方某求大人施展回春之手,救救我家少主,不仅方某因此感激涕零,就是我家主上,也不会忘记大人大恩。”

我皱皱眉道:“方将军,先不说你我双方的立场,乃是敌对,也不说在下是否能够救治姜少主,在下自从遇刺之后,体弱非常,若没有雍王殿下和我这位从仆的精心照料,只怕早已身死,若是千里迢迢奔赴东海,只怕人还没有到,就已经奄奄一息了,再说如今雍王正用我参赞,我是一刻也离不开的。”

方远新知道江哲没有说一句假话,先不论他主上的身份,毕竟只要姜永肯归降大雍,必然能够得到雍帝重用,可是只看江哲虽然神色还好,可是种种气虚体弱的迹象一样不少,若是千里奔波,只怕真是到不了东海就病倒了,可是无论如何少主也不能到长安来啊。他心中盘算了半天,还是觉得为难,原本他是想想个法子将江哲劫走,可是一打听才知道这个江哲乃是雍王极其看重的人,若是明目张胆和雍王作对,就是主公也是不愿意的,再说今日一见,果然江哲身边防卫严密,自己是没有可能将江哲劫出长安的。

我留神看着方远新的脸色,初时有些苦恼,然后带了一丝杀气,最后却是绝望,哪里还不知道他的心思,可是我是无论如何不能在这个时候离开长安的,若非桑先生已经说过不再行医,而且桑先生的隐居之处乃是秘密,不能告诉外人,我早就引荐他去见桑先生了,唯今之际,只有让他的少主到长安来,只是人一到了长安,只怕是没有机会离开了,这一点恐怕会让姜永很为难吧。

想了片刻,小顺子突然提醒我道:“公子,已经快到朱雀门了。”

方远新一听,面如死灰,他知道已经不得不离开了,他黯然道:“方某回去之后会向主上说明此事,事关重大,方某是无法作主的。”

我心中一动,道:“方兄何必急着走呢,你既然肯和江某相谈,那么为什么不见见殿下呢,殿下心胸宽广,性情仁厚,或许能想个法子帮助令少主,至少江某可以保证,如果方兄想要离开,殿下是不会阻止的。”

方远新精神一震,他也知道就是江哲肯替少主医治,也需要得到雍王的许可,想到主上待自己恩深似海,自己就是冒些生命危险又能如何。下定决心,方远新道:“那么就拜托江大人代为引见了。”

我神色郑重地道:“方将军放心,江某保证方大人可以安全离开长安。”

方远新正要回答,小顺子突然神色一动,冷冷道:“公子,有人跟踪。”

我问道:“几个人,从什么时候开始的。”

小顺子道:“这几个是在我们遇见方将军的时候缀上的,本来一直离车驾很远,方才突然接近了许多,噢,我明白了,前面有巡逻的禁军过来了。”

我心中一动,问道:“那支禁军是谁的手下。”

小顺子掀开帘子,看了一下,低声道:“大人,秦将军率领禁军巡查,很快就会碰上咱们。”

我冷笑道:“小顺子,你说秦青会不会搜查我的车驾?”

小顺子皱眉道:“雍王府的车驾,他应该不会检查吧。”

我微微一笑道:“按照法令,他有权力检查夜行的车驾,当然若是论我的身份,是可以不用查的,可是他真要搜查,我也不便当场阻止,想必本来那些人是跟着叶兄和方兄的,谁知碰上了我这条大鱼,这人倒也果决,想用这个法子诬陷我一个通敌谋反。”

小顺子蹙眉道:“公子不便拒绝搜查,又不能出手伤害禁军,这可如何是好。”

我笑道:“先让荆迟去对付吧,我若急急出面反而不好,秦青真是可惜了。”

这时那队禁军已经到了眼前,为首一人英姿飒爽,正是秦青,他策马上前高声道:“荆将军,怎么是您亲自护送,车驾里面是哪一位?”

荆迟沉声道:“原来是秦统领,末将奉命保护江司马,重责在身,不便见礼,还请秦将军见谅。”

秦青笑道:“说哪里话,秦青虽然官职略高,可是将军乃是沙场勇将,谁不知道雍王殿下麾下第一勇将,最擅长斩将夺旗的就是荆将军,秦青末学后进,不敢受将军大礼,如今夜深,不知道可否让秦某见见江司马,秦青身负保护皇城安全的重责,不敢懈怠,还请几位见谅。”

荆迟皱眉道:“虽然是检查行踪可疑之人是理所当然,可是这乃是雍王府车驾,车中又是司马大人,秦将军为何定要检查,夜风寒冷,司马大人近日身子不好,恐怕受了风寒,实在不便相见。”

秦青神色一变,回头低声问身边的一个亲卫道:“江司马不好惹,为何公主定要我检查他的车驾,若是雍王动怒,告知父亲,我恐怕会受责备的。”

那个亲卫低声道:“驸马放心,我们的人看见叛逆在他的车上,我们也不是要为难江司马,这样大将军是一定不会同意的,可是那人若是进了雍王府,只怕祸患无穷,只要驸马将那人带走说是盘查,江司马理亏,必定不敢拦阻,到时候只要驸马不说,想必江司马也不会主动把灭门的大罪往身上揽吧。”

秦青有些犹豫,可是想想妻子一向智谋胜过自己,应该不会错吧,便扬声道:“只是例行公事,不会时间很长,应该不会伤害江司马的身体的。”说着策马上前就要掀动车帘。

两名侍卫同时拦阻住道路,他们可是知道车上现在有一个人不能曝光的。秦青剑眉一扬道:“怎么,你们要阻止本统领执行公务么?”

荆迟冷笑道:“若是让你搜查了车驾,过了明日岂不是朝野都知道您秦将军本事大,居然搜了雍王府的车驾,到时候没面子的可是荆某。”

秦青微怒道:“若是雍王在此,末将自然是要退避三舍的,可是如今只是江司马在车上,那么末将就有搜查的权力,若是你们心中没有鬼,何妨让我看上一看呢?”说着一挥手,那队禁军将车驾围住,秦青冷冷的看着荆迟,只要他再说一个不字,就要上前强行搜查。

方远新心中一凛,手再次按住了腰间,他本是叛逆之身,若是落在禁军手中只怕是有死无生,因此生出了拼命之心,他心中不由暗暗责备自己,不该冒险和江哲在车上密谈,自己就是一死也还罢了,若是连累了这个可能是唯一可以救治自己的少主青年,那么自己就是万死也难辞其纠。

我微微摇头,轻轻的按住了他的手,若是这样的事情也不能处理,我还配作雍王的首席军师么,看了小顺子一眼,从腰间解下一块金牌递给他,虽然有很多法子,可是这一种却是最简单直接的,为了安安这位方将军的心,还是仗势欺人一次吧,可惜秦青太固执了,换了一个人,绝不敢要求搜查雍王府的车驾的,铁面无私可不是谁都能办到的,只能说秦青太幼稚了。

小顺子接过金牌,挑帘而出,不到片刻,我淡淡笑了,这块金牌还真是管用啊。不愧是雍王郑重其事借给我使用的好东西。

\chapter{第七章 举重若轻}

大雍武威二十五年,有御史弹劾禁卫军北营统领裴云,帷薄不修,有违孝道,人皆知其冤,不敢辩也,唯太宗曲意护之。

——《雍史·太宗本纪》

就在秦青想要强行搜查的,突然车帘挑动,一个青衣少年站了出来,站在车辕上,负手而立,神色冷傲如冰雪,在淡淡的月光下显得遗世而独立,不似世间凡人。而最令人心寒的就是,他那双冰澈晶莹的眼睛,就那么冷淡的望着自己,秦青突然感到这人根本就将自己这些人看成了没有生命的物品,可以轻易损毁,却没有丝毫内疚之心。

他镇定了一下,出言道:“李兄时刻不离江司马左右,真是赤胆忠心,末将没有恶意,只要让我看上一眼车内就可以。”

小顺子冷冷一笑,道:“江司马对大将军和秦将军都是十分敬重的,想不到今日来落公子面子的竟是秦将军。”

秦青心中一寒,他可是在自己家中亲眼看到过这个少年气焰凌人,若非江哲一句话,只怕没有人敢说他不会一掌杀了太子李安,一年来,长安朝野都已经知道有这么一个少年高手,邪影李顺,武功邪,心性邪,出手无情,这样一个人却是只对一个人忠心耿耿,甘心作他的影子,这个外号也不知道是谁叫出来的,可是却十分形象,他站在江哲身后的时候真的只像一个影子,谁也不会想到这样一个高手会去做那些奴仆才会做的事情,而且毫无怨言,可是当他动怒杀人的时候却是恐怖无情的,数月前,有人趁着雍王外出游春而伏击行刺,这也罢了,谁知那日江哲身子较好,竟然和雍王一起出游,险遭波及,就是这个李顺一怒之下,将前来行刺的十几名刺客尽皆杀死,据事后去清理的人所说,那些尸体没有一具留了全尸,死状之惨,更让那些见惯死人的禁军和仵作回去之后做了好几日的恶梦。

可是秦青又想道,若是自己这样轻轻放手,怎么向寒幽交待呢,便壮着胆子道:“末将也是职责所在,还请李兄见谅。”说罢策马上前,心想李顺总不能当街杀害朝廷将领吧。

却见小顺子冷冷一笑,眼中透出浓浓的杀机,一只右手便要举起,秦青所带的禁卫军同时惊呼,刀剑出鞘,而雍王府的亲卫也随即拔出白刃,一时之间,朱雀门前杀气纵横,形势一触即发。

谁知李顺只是高高举起右手,手中乃是一面金牌。秦青抬眼望去,已经看到那面金牌上面的独特花纹和九条金龙盘绕中的“如朕亲临”四个大字。秦青一声惊呼,他可是知道的,这面金牌是皇上赏赐给雍王殿下的,许他代天巡狩,所过之处,一切军政大事皆可过问,当今世上只有这么一面,只是雍王为人谨慎,而且又是威名远扬,所过之处不需金牌就可以任意行事,所以很少有人真的见过这面金牌。想不到雍王竟然将金牌交给了江哲,雍王对那个南楚降臣如此宠信,将自己的身家性命一般的御赐金牌都借给他使用,秦青不禁有些嫉妒,但是无论如何,现在最重要的事情不是想这些。他连忙一声招呼,带着所有禁军下马拜倒,口称万岁。

小顺子淡淡一笑,收起金牌道:“秦将军尽忠职守,司马大人本应敬重,只是此事非同寻常,若是今日让将军搜了车驾,只怕日后雍王府再不得安宁了,秦将军,雍王殿下乃是当今皇子,又是圣上御封的天策元帅,绝不会作出什么伤害大雍国体的事情,秦将军今后行事,还要慎重,不要平白做了人家的手中之剑。”

秦青只得唯唯称是,心中恼怒非常,正要敷衍两句,远处一队武士飞马赶来,秦青看去,那些人都是雍王府宿卫的服色,为首一人长眉凤目,相貌俊伟,气度不凡,令人一见便生出亲近之心,只看他身上跨着的金弓和马鞍前面特制的箭囊,便知道此人正是金弓长孙冀。他飞马到了近前,先对秦青施了一礼,然后朗声道:“殿下久等不见司马大人回府,特派末将前来相迎。”

荆迟嘟囔道:“还不是有人挡道。”小顺子冷冷的看了他一眼,荆迟立刻噤声,这一年来,我罚他抄书背书,通常都是让小顺子监督,到了现在,小顺子一个眼色,就可以让他噤若寒蝉了。

当下,我们礼数周到的送走了秦青,小顺子仔细的打量了一下那个暗中向秦青进言的近卫,将他的相貌记得清清楚楚。然后我终于回到了雍王府,一到大厅,就听见雍王怒冲冲地道:“随云,出了事情了,你看——”看到方远新,他神色一变,王者威仪顿时笼罩了整个大厅,令人心中生出不敢反抗的念头。

方远新不知怎么,竟然上前拜倒在地,直到膝盖落地,才醒悟过来,心道,我这是怎么了?

我已经躬身行礼道:“殿下,这位是姜永姜侯爷的麾下大将,方远新方将军。”

雍王愣了一下,大笑着上前搀扶起方远新,说道:“久闻大名,方将军擅长水战,天下闻名,听说数年前方将军在东海连番血战,将侵扰海疆的海寇扫平的扫平,收服的收服,有很多海上从商和商人和靠海吃饭的渔民都为方将军立了长生牌位,海疆清平,方将军功劳非浅,虽然如今贵上仍然割据海外,可是都是炎黄一脉,本王也为姜侯爷的功绩佩服万分。”

方远新只觉的心中暖洋洋的,想不到雍王对自己这些人的事情如今赞誉有加,他开口道:“殿下过誉了,主上虽然孤悬海外,但是心向中原,虽然仍然对大雍朝廷心存怨望,可是每每提起殿下战功辉煌,仍然是十分欢喜。”

雍王叹道:“想当初,我和表兄也是童年玩伴,情同手足,可是造化弄人,如今已成杀父之仇,本王每次想起来都十分心伤,若是有可能,还请将军劝劝表兄,就算是为了后人,也不应该久居海外,表兄想必十分想念中原山川秀丽吧,若是表兄肯回中原,贽情愿向表兄谢罪,任凭表兄是杀是打。”

方远新眼神有些黯淡,道:“殿下深情厚谊,末将必定向主上转达,可是殿下应该知道,主上最恨的不是殿下,虽然是殿下率军击破老侯爷的大军,可是这也是老侯爷野心太大,不肯接受大雍封赐的爵位的结果,可是若是老侯爷死在战阵之上,主上虽然悲痛,也不会定要报仇雪恨,可是老侯爷却是被那毒妇梵惠瑶刺杀,这种屈辱主上终生不忘,此仇不报,主上是死也不肯瞑目的。”

雍王又是一声叹息,道:“方将军先坐下来说话,这些事情以后再说吧,事情总有解决的一天的,但不知方将军这次莅临寒舍,有什么需要本王帮忙的,只要不干涉社稷大事,贽绝不推辞。”

方远新连忙又将求医一事说了出来,目光中又是恳求又是担忧,他自然知道这样一来自己主上的把柄就被雍王握住了,可是无论如何少主的一丝生机也不能这样错过啊。

果不其然,听了方远新的话之后,雍王李贽的神色有些犹豫苦恼,他刚刚坐下来不久,就又站了起来,负手在大厅里转了几圈,看看方远新,又看看早已经坐在一旁,打着呵欠昏昏欲睡的江哲,终于道:“方将军,本王也不瞒你,若不是江先生身体如此差劲,本王无论如何也要拜托他去一趟东海,可是可是自从他不幸遇刺之后,虽然将养了一年多,仍然是体弱气虚,除非是一路上缓缓而行,稍有差池就要休息几日,我才能放心他远行,可是这样以来,没有个一年半载,只怕他到不了东海,这样一来拖延日久,先不说本王实在不能少了他,这日子一长,这件事情必然传扬出去,到时候又该如何是好,你也知道,其他人不是聋子和瞎子,到时候会发生什么事情,本王也无法预测,可是江先生是肯定到不了东海了。”

方远新心中一片冰凉,他知道雍王一句谎言也没有,难道只能把少主送到长安来么?

雍王同情的看了他一眼,又接着说道:“唯今之际,本王倒有两个法子,一个是本王暗中向父皇禀明此事,父皇或者会默许这个孩子到长安治病,可是这样以来,姜侯爷必须得作一些让步,或者就是表兄想法子把侄儿送到长安,瞒过他人耳目,到时候若是一切顺利,侄儿就可以自由回去东海,可是我不妨直言,如今长安各方势力错综复杂,本王不敢保证能够始终消息不会外泄。”

方远新想了半天,道:“末将会尽快通知主上,请他决定,如果有什么消息,还希望殿下能够不吝相助。”

雍王笑道:“我和贵主上乃是骨肉至亲,怎会相害,只要侄儿来了长安,本王绝不会撒手不管的。夜已经深了,本来我该留你的,可是你也知道如今本王事事都得避嫌,我会派人送你出去的。”

方远新下拜道:“多谢殿下,不论事成与否,末将和主上都会感谢殿下的这番心意。”

李贽叹息道:“这也是时机不巧,有些事情我不说你也知道一些,本王实在是不能让江司马远行的。”

方远新心道,如今你们兄弟争夺皇位争得你死我活,江哲又是你这般看重的心腹,也难怪你不肯放行,更何况这个江哲身体也太差了,我们这里说着话,他都快要昏倒的样子。

就在方远新要告辞的时候,我出声道:“方将军等一等。”说着我从刚刚溜出去一趟的小顺子手中接过两个玉盒,懒洋洋的道:“胭脂玉这种海蛇我只是听说过,所以必须看过伤势才能医治,可是我也不能让方将军这样空手而归,这里有两种药物,一种可以救治大部分常见的毒药,效果很好,至少可以不让令少主毒气攻心,另一种药物每日一粒可以让人沉眠昏睡,却不会因此伤害人的身体,这样就可以让令少主不必每日苦痛难耐。”

方远新听了大喜过望,道:“末将代我家少主多谢江先生慈悲。”他想到能够暂时减轻少主的病痛,已经是难能之喜,故而千恩万谢的接过药盒。

我笑道:“这种药物原本是我自己使用的,只因我伤愈之处,伤口疼痛搔痒,难以入眠,所以特意配了这种药物,没想到效果十分好,只是配制起来十分麻烦,而且这种药方不能外泄,要不然我就写一张药方给你了。”

方远新离开之后,我深深的叹了一口气,问道:“殿下,可是发生了什么大事么?”

李贽这才想起自己原本要说的事情,苦笑道:“今日晚上,父皇受到一份谏章,弹劾裴云帷薄不修,有失孝道。”

我微微一愣,问道:“殿下,裴云宠爱妾室,疏远嫡妻,令她意图伤害妾室和幼子,这可以说是帷薄不修,可是有失孝道,怎么说的上呢?”

李贽苦笑道:“怎么说呢,那个蔡御史也真是胆大,他指责说裴云冷落父母为他订婚的妻子,致令父母伤心担忧,所以这是不孝,毕竟自从这件事发生之后,裴云的父亲因此气怒,病卧在床。而且,那个御史还隐晦的说,薛小姐至今仍是完璧,可见裴云有失人伦。”

我愕然道:“御史理应留意国家大事,怎么人家闺房中事,他也管起来了?”

李贽冷笑道:“对他们来说,为虎作伥胜过为国分忧,不说他了,你说这事该怎么办,总不能让裴云的父亲上书说自己是支持裴云纳妾,冷落嫡妻,闹得家宅不宁的,这样一来,裴云可真是不孝了,自古以来,只有儿子替父亲顶罪的,可没有父亲替儿子顶罪的。”

我也有些苦恼,怎也想不到竟然有人会这样做文章,还扣了一顶不孝的大帽子,可是一时也想不到什么法子,历朝历代都是以孝治天下的,裴云若是担了一个不孝的声名,只怕从今之后仕途艰难,从眼前来说,只怕铁桶一般的禁军北营就要易手了。

小顺子突然冷冷道:“皇上未必这么看?”

我和雍王都抬眼望去,小顺子却不说话了。我和雍王很快都醒悟过来,皇上对凤仪门是有戒心的,若是知道裴云不愿和凤仪门弟子联姻,只怕心中不会责怪。转念一想,我奇怪地道:“这一点太子他们也未必不清楚,为什么他们要做徒劳无功的事情呢?”

小顺子微微一笑,道:“殿下和公子当局者迷,若是这种事情传出去,只怕无脸见人的是薛小姐,一个女子被人嫌弃如此,再加上声名败坏,只怕只有一死了之,到时候工部侍郎薛矩必然上书攻讦裴将军,不论如何,裴将军也不能说行止无亏,薛矩又是工部重臣,精通兵器制造改良,天下谁不知道薛矩研制的‘神臂弓’乃是守城利器呢,到时候薛大人拼了担上教女不严的罪名,一定可以把裴将军拖下水,就是陛下再偏袒,也只得让裴将军暂时停职,只怕等到裴将军复职的时候,禁军北营已经不受控制了,而且裴将军乃是新近归顺殿下的军方新锐将领,殿下无力相护,而且又让薛矩成了殿下的敌人,这可是一举三得了。”

李贽听得心中一寒,敬佩地道:“小顺子你果然看得透彻,本王却没想到,只怕明日这道表章传遍朝野,薛小姐就是不想自杀也得自杀了,你说如今可怎么办那,裴云乃是名将之姿,本王实在舍不得让他受污。”

我明白其中的关节之后,叹息道:“这条计策果然狠辣,不过也不是没有法子解决,最好的法子就是裴将军的妾室若是身死,那么薛小姐杀害人命,裴云所为就算不上过分了,可惜这是行不通的,那位如夫人余毒已清,这一点很多人都知道,另一个法子就是要从薛小姐身上着手,若是她肯上书请罪,说自己内疚神明,情愿出家清修,以赎罪孽,那么别人也就不能再怪责裴云。”

李贽苦笑道:“若是她肯倒是好的,可是她恐怕不肯服软的,凤仪门弟子个个心高气傲,恐怕死也不肯认罪服输。”

我微微一笑道:“一个青春少女,怎会想死呢,只怕她如今万分懊悔嫁给裴将军吧,问题是她若不肯上书认罪,只怕就要‘自杀’了,生命可贵,她又怎会不珍惜呢,若是给她机会,改名换姓,远走天涯,嫁夫生子,她不会不愿意的。只是这件事情交给谁去办,有些碍难,若是办得不好,只怕弄巧成拙。”

李贽想了想,眼中一亮,道:“我有了法子了,魏国公程殊素来交好群臣,也是可以和薛矩说的上话的,而且此老鬼主意最多,心肠又好,薛矩一定不会对他戒备排斥,而且魏国公性子诙谐,朝中很多重臣的子弟都把他当成叔伯长辈,薛小姐也曾经是其中之一,就是现在见到魏国公也是十分亲热,他去说项一定成功。事不宜迟,本王这就去求魏国公,他素来提携后进,绝不会看着裴云收到不实的责难的。”

当夜李贽亲自到了魏国公府,一番促膝长谈之后,程殊飞马赶到薛府,进了薛府之后,正是早朝刚过的时候,此时的薛小姐刚刚得知奏章的事情,正在万念俱灰的时候,正要举剑自刎,程殊一声大喝,闯进房中,将她的长剑打落,若是别人,薛小姐或者会恼羞成怒,可是看到从前在自己小时候就常常让自己当马骑的程伯伯,她终于忍不住跪在地上大哭起来。

\chapter{第八章 宗师莅临}

程殊怜惜地道:“傻孩子,你也是一个苦命的孩子,受了那些人的蒙骗,告诉程伯伯,你今后有什么打算?”

薛小姐茫然道:“程伯伯,我也不知道该怎么办,从前我是凤仪门的弟子,家世又是不错,所以追求我的男子不知有多少,可是我心中只有一个裴云,不是因为他是我的未婚夫,而是我喜欢他这个人,他到少林学武,我总想配得上他,我不想他只当我是一个平常女子,我希望他能够以我为荣,所以我才拜在凤仪门中,如今我勉强也可称得上文武双全,相貌也是称得上绝色,我原以为他会视我如珍宝,可是他却对我越来越冷淡,最后竟然娶了别人,爹爹原本劝我不要纠缠下去,可是我不甘心,我这般辛苦都是为了他,他却把我视若破履,所以几个姐妹一怂恿我就强行嫁给了他。可是没有用,他对我从来都是客客气气的,晚上却从来都在那个女人身边,我好恨,好恨,可是我不愿意示弱,只能眼睁睁看着他们在一起,后来那个孩子出生了,我从没看见过他那样欢喜,还有公公婆婆,也都只顾着那对母子,这些我都忍了,只求他能看我一眼,可是他来了,却是和我商量仳离之事,我终于忍不住,想要杀了那破坏我幸福的孩子,可是却失败了,他是绝对不会原谅我了。”

看着痛哭出声的薛小姐,程殊心知若非她如今已经崩溃,是绝对不会将自己的心事说给自己这个外人,他心中又是怜悯又是惋惜,不由道:“孩子,别怪伯伯说你,你千错万错不该去凤仪门,凤仪门教出来的弟子确实是高贵典雅,就是作皇后妃嫔也够格,可是裴云只是一个平常人,就像伯伯,当年伯伯和你伯母成婚不到三个月,就去从军,你伯母独自一个人侍奉二老整整十二年,还是我当了将军之后才将他们接到长安,那时候我的儿子已经是个半大小子了,可是伯伯才第一次见到他,后来我又跟着陛下东征西讨,哪里还顾得上父母儿女,都是你伯母辛苦持家,所以人家笑话我老程惧内,可是谁知道我是内疚于心,这一生我亏待她太多,换了你,若是裴云出征,只怕你会跟了去,虽然凭着你的武功才智,至少不会成为累赘,可是裴云要得却是一个能替他在家孝顺双亲,抚养子女的妻子,孩子,你太出色了,所以裴云才不肯娶你。”

薛小姐愣了半天,道:“他不是因为师门的缘故么?”

程殊苦笑道:“你若这么想,我也不怪你,可是裴云不是这种人,这不也是你喜欢他的地方么?”

薛小姐苦涩地道:“如今说什么都迟了,侄女已经无脸见人,还请伯伯不要阻我。”

程殊冷笑道:“你这孩子怎么糊涂了,天大的事情也有个解决的法子,你若是肯重新开始,凭着你的才貌,哪里还找不到归宿,这天下这么大,你若是听了伯伯的话,到个没有人认识你的地方去,改名换姓,不是胜过寻死么?”

薛小姐痴痴的望着窗外,神情迷离,程殊见她如此,知道正是紧要关头,自己却不可相劝,这时候最好有一个知心人劝劝她,可是这个人却难找得很。

突然窗外传来一声歉疚的叹息,薛小姐神色一动,扑上前拉开窗子,却是一个黑衣男子,相貌英俊,周身上下洋溢着淡淡的杀气,只是神情黯淡,剑眉深蹙。

薛小姐啊了一声,泪水滚滚而下,程殊微微摇头,转身走出了房间,那个黑衣男子跃进了窗子。薛小姐狠狠地道:“你来做什么,是来看我笑话的么,如今人人直到我薛秋雪残忍狠毒,都说你应该休了我,你得意了吧。”

那人正是裴云,他沉声道:“秋雪,我从未想这样伤害你,可是事情到了今天这一步,我也没有料到,我原想你若肯退了亲事,一定能找个如意郎君,没想到会有今日。”

薛秋雪想起从前往事,不由悲从心起,道:“你真的只想找一个平凡女子为妻,也不愿意娶我么?”

裴云黯然道:“秋雪,你真的很出色,文武两途都有不小的成就,我曾见你谈论诗文,很多都是我没有听过的,还有你对朝政军务都有涉猎,若是娶了你我会多一个贤内助,可是秋雪,我真的对这些不感兴趣,从军报国是我的夙愿,可是我并不想和人钩心斗角,在外面已经是如此,回到家里我只想平平淡淡的过日子,我希望我的妻子会做几道家常小菜,可以缝几件衣服给我,可以跟我说些家中琐事,这样就够了,我并不需要一个满腹心机的妻子。可是秋雪,你如此耀眼,是我配不上你。”

薛秋雪苦涩地道:“你说得对,原本是你配不上我,配不上我……”一连说了几遍,说到后来已是声嘶力竭。裴云上前一步,却又停住了脚步,他终究不肯冒犯这个从前的未婚妻,他是真的希望这个女子能有一个好的将来,若要如此,就要让她对自己死心,此刻的温柔对她来说已经太迟了。

薛秋雪良久终于冷静下来道:“谢谢你,告诉我实情,不是我不好,只是你不需要我这样的妻子。你放心,我不会连累你的,长安这个伤心地我不会待下去的。”

裴云默认,片刻之后道:“我有一位师弟在南海行商,他和我乃是生死之交,你若肯前去,他必然会好好照顾你。”

薛秋雪默然,就在裴云以为她不会接受的时候,薛秋雪淡淡道:“谢谢你,我听说南海风光奇特,还有夷人往来,很早就想去看看了。”

裴云的事情就这样大事化小了,虽然多名御史和很多朝臣纷纷上表弹劾,但是薛小姐的谢罪书一呈上来,这些弹劾就没有了力量,而薛小姐也消失了,虽然薛家对外声称薛小姐已经削发出家,可是却没有知道她在何处落发。这个可悲可怜的女子就这样消失在人们的视线当中,没有人知道薛小姐早已在程国公的家将护送下离开了长安,离开了这令她心伤万分的苦痛之地。

可是事情的结果也不像我想像的那样如意,裴云还是受到牵连,虽然没有降职罚俸,可是皇上指派了夏侯沅峰兼任禁军北营的副统领,这样一来,本来铁板一块的北营还是被硬生生的插入了一个钉子。偏偏夏侯沅峰风度翩翩,长袖善舞,又是皇上宠臣,所以很快就站住了脚,幸好裴云素来深得军心,还不至于被架空,总算夏侯沅峰也不敢过于急进,局面陷入了僵持阶段。

坐在凉亭之中,享受着习习的晚风,淡淡的草木清香扑鼻而来,我口中含着一片刚刚摘下来的竹叶,专心的吹着一首简单的曲子,那没有什么技巧,却是委婉动听的乐声随着夜风流淌在寒园之中,一曲终了,小顺子的身影出现在远处,不知道从什么时候开始,每当我心情烦闷或者忧虑的时候,我就抛却一切,坐在这里吹着竹叶笛,这总是能让我心情平静下来,我从没忘记桑先生的诊断,既然不能远离尘嚣,那么只好用这种方式洗涤自己的心灵了。事实上,寒园中的侍卫都知道在我吹叶笛的时候是绝对不能打扰我的,就连小顺子也不会在这个时候来打扰我。曾经有一个本来颇受我看重的侍卫只因犯了这条规矩,被我逐出了寒园,自从那以后就再也没有人敢触犯我的逆鳞了。

接着小顺子递过来的香茗,我笑道:“裴将军虽然受到些挫折,但总算不至于影响他今后的前途,其实我们也不算失败,反正我们看重的是裴云这个人而非那一营禁军,明天下帖子邀请裴将军来寒园做客,邀请殿下也来作陪。”

小顺子淡淡道:“殿下已经邀了裴将军明日来府上,既然公子也想见他,我去告诉殿下将宴席开在寒园吧。”

我摇头道:“既然殿下已经相邀了,我就不去了,有些事情还是让殿下自己去处理吧,对了,少林怎么样?”

小顺子皱眉道:“名门大派果然沉得住气,现在还没有动静。”

我微微一笑道:“若不是这般谦抑隐忍,你以为少林凭什么经久不衰,百多年来,多少帮派昙花一现,就是少了这份气度,有时候世事就是如此,仰而求怎如俯而就,若非俗事牵绊,我焉能在红尘久住,小顺子,你的武功本来是极好的,只是我见你出手太过狠辣,少了几分隐忍,总觉得不妥,所谓刚不可久,柔不可守,奇不能胜正,用兵打仗不能一味用奇,我想武功也是如此,你好好想想。”

小顺子若有所思,正在这时,一个平和的声音说道:“江檀越果然是灵性天成,这个道理老衲乃是四十岁之后方才明白的。”

我心中一震,这个声音柔和清远,仿佛近在耳边,可是我自认六识过人,分明百丈之内绝无这样一个人,我看向小顺子,小顺子却是想得入神,显然早已忘记了保护我。我转念一想,突然笑了,道:“慈真长老莅临寒园,真是蓬荜生辉,请恕哲不便远迎,请大师到园中相见。”

眼前仿佛一花,一个穿着灰色僧衣的中年僧人出现在园门口,缓缓走来,我极目望去,只见这个中年僧人相貌清秀,面如满月,眉心一点胭脂红痔嫣然如同丹朱,怎么瞧去也觉得这位僧人只是一个寻常和尚,可是我却隐隐觉得这位大师缓缓行走的步伐,一举一动浑然天成,好像和这天地乃是一体一般。小顺子这是也抬头看去,眼中神光闪烁,他虽然知道这人身份,但是天下之人在他看来都是可有可无之人,所以他反而起了杀意,这样一个人若是要伤害公子,自己可得有能力阻止才行。

他杀意一起,只觉得四周强大的压力向他逼来,他心中一惊,看向公子,却发觉江哲神色没有变化,便知道这种压力只针对自己,便全力抵抗,但是那种压力越来越强,他只觉得隐隐似有人在自己耳边念诵佛经,要自己忍受屈服,可是他心志本是十分坚定,反而死撑着不肯后退,那种压力越来越强,小顺子只觉的周身上下几乎动弹不得,突然他心中一动,收了一些抗力,果然那种压力减弱了一些,他冷冷一笑,突然周身上下杀气冲天,那种杀气冰寒刺骨迅速蔓延在寒园之内,奇异的景象出现了,明明是夏日黄昏,可是寒园从园心凉亭到园门之间,一半是秋风萧杀,一边是春意融融,两种气势相争,那萧杀之气虽然越来越弱,可是那种誓死无归的气魄却是越来越强,就连那种融融的气息也渐渐带了些肃杀之气。

我虽然身在亭中,没有亲身感觉到那种水火不容的气氛,可是只见百丈方圆之内树叶无风自落,然后狂乱的旋转飘荡的样子便知道有异。后来更是见到小顺子脸色越来越不好,想也知道谁落在下风,眼珠一转,随手拿起一只茶杯用力向地上摔去,果然如同我想的一样,这小小的惊吓,让正在较劲的两人颇有默契的开始收功,不过片刻,就已经恢复正常。那个僧人也不见怎么迈步,百丈距离仿佛一步之遥,一抬腿就走到了亭边,他微笑道:“李施主的武功另辟蹊径,走得乃是‘无情’的路子,老衲原本想以梵音点化,不料李施主已经是心如金石之坚,不受外力所动,若是李施主潜心苦修,达到‘忘情’的境界,必然是一代宗师的身份了。”

小顺子上前施礼道:“大师过誉了,小人并没有成为宗师的野心,只要能够保护我家公子一生平安也就够了。”

慈真若有所思的看了小顺子一眼,只见他双目之中神光凛然,那是一种坚定而不可动摇的决心,他心中不由慨叹上天安排巧妙,这人若是毫无拘束,只怕是为所欲为,纵横天下,到后来不免造下滔天杀孽,为害之深,胜过魔宗百倍,可是上天有好生之德,竟安排了一个人可以约束他,指正他,他看向方才掷杯示警,令自己两人罢手言和的江哲,这个青年虽然双目神光黯淡,可是那双眼睛却带着透彻世情的觉悟。他向江哲轻施一礼道:“老衲慈真,见过江先生。”

我有些慌了手脚,连忙还礼道:“大师乃是宗师身份,哲焉敢受此大礼,还请不要如此,大师请坐。”

慈真微微一笑道:“日后檀越自然知道老衲这一礼您是当得的。”

我恭恭敬敬地道:“大师此来,哲受宠若惊,不知道有什么事情指教。”

慈真淡淡道:“老衲此来原本是想见见雍王殿下,可是久闻檀越才智惊人,故而先来拜望。今日一见,小檀越心脉暗伤,只怕长久滞留红尘,有伤寿元,小檀越既是精通医理,为何不为自己考虑。”

我微微一笑道:“哲也是凡夫俗子,雍王殿下待哲恩重如山,殿下的宽宏大量,也让哲感佩于心,若是哲此刻抛却凡尘,实在是内疚神明,故而不敢为之,还请大师不要告诉殿下此事,免得他心中忧虑。”

慈真微微一叹,道:“江檀越此心天人共鉴,老衲自然守口如瓶,檀越对我少林敬重,老衲虽不会仿效世人斤斤计较恩怨,但是也有投桃报李之心,老衲有几句内功心法,也没有什么别的作用,只是能够强身健体,调养心脉,檀越虽然没有练过武功,但是这几句心法只是呼吸吐纳的法子,想必不会费力,希望能够对江檀越有所帮助。”

我喜道:“多谢大师厚赐,桑先生曾说天下武功,只有少林寺的心法最是清净无为,涵养身心,哲若是能够多活几年,都是大师所赐。”

慈真微笑道:“江檀越辅佐贤王,功在社稷百姓,这几句心法算得什么。”说着将几句心法说了出来,又仔细的解释给我。小顺子在一旁,面有喜色,他原本最担忧我的身体,如今见有了转机,自然是大喜过望,看向慈真的目光也多了几分柔和。

过了一会儿,远处传来了脚步声,正是雍王李贽带着管休、苟廉、长孙冀、荆迟、司马雄等人匆匆赶来,众人到了亭前,都是恭恭敬敬的行礼如仪。慈真虽然是宗师身份,却丝毫没有倨傲的表现,也是微笑还礼。

李贽上前神色激动地道:“自此上次拜见大师之后,已经有数年时光,大师容颜依旧,倒是李贽,苦于政争,苍老了许多。”

慈真沉静地道:“殿下,老衲此来,乃是转达敝寺上下的心意,若是殿下有所命令,敝寺上下无不从命。”

李贽一愣,神色间反而有了犹疑,他原本只希望少林寺有限度的支持,就可以了,想不到竟然得到了少林寺的全力支持,这是怎么回事。

\chapter{第九章 失德惊天}

大雍武威二十五年六月,天子下诏,告祭黄帝,立祭坛于桥山,诏太子于长安陪祭,未料太子其间行止有亏,帝惊怒,幽禁太子。

——《雍史·戾王列传》

慈真见状淡淡一笑,道:“殿下勿虑,少林如此也是迫不得已,太子殿下所作所为,虽然尚未昭然于天下,可是却瞒不过天下百姓,更何况凤仪门近来倒行逆施,已经引起黑白两道的不安,少林忝为白道第一大派,不能眼见这等情形发生,殿下素来优容敝寺,又是勤政爱民,敝寺虽然不能涉入政争,可是凤仪门乃是江湖门派,敝寺还可以有些作为。”

我和雍王心中都是一宽,原来是少林看不过去凤仪门的嚣张了,新仇旧恨一起算了,不过我心想,因为“霍纪城”一人,引起江湖大乱,凤仪门借机横扫武林,这件事情可不能泄漏出去,至少不能人人皆知,否则我只怕也成了祸乱江湖的罪魁祸首了。

这时慈真又说道:“老衲这次前来还有一件事情,近日陛下有意祭黄帝陵,老衲师兄慈休奉命前来主持其中一项仪式,师兄虽然佛法高深,可是不谙武技,故而老衲特意保护他前来。”

我和李贽都暗暗点头,这件事情我们是知道的,慈休大师原是先朝名臣,国破家亡之后投身佛门,如今已经是佛门中数一数二的高僧,他佛法精深,精通梵语,多年来翻译了千卷以上的梵文经典,乃是弘扬佛法的第一功臣,这人离寺,果然值得慈真亲自护送,要知道慈真虽然是一代宗师,可是论起在佛门的地位,并不比慈休大师尊贵。想到这里,我不免有些歉意,这次的祭奠黄帝陵的大典只怕是难以善终了。

大雍立国以来,多次举行过祭祀黄帝陵的大典,这次却有些不同寻常,天子自然是要亲自前往桥山祭陵的,可是同时还要在长安设立祭坛,同时祭祀,翼求大雍国运昌隆,这陪祭之人自然只有储君可以担任了,所以从六月开始,陛下诏令太子入东宫斋戒,他自己则在斋宫斋戒,六月十四日,天子才会起驾到黄帝陵,六月十五日举行大典,奉诏伴驾的有雍王,齐王和一干文武重臣,而丞相韦观和侍中郑瑕则奉命在京协助太子祭天。

斋戒可不是什么等闲的事情,要不吃荤、不饮酒、不听音乐、不近妃嫔、不吊丧、不理刑事,更要平心静气,不能烦躁不安,可是太子李安如何能够忍耐得住,饮食只是清汤寡水,全无滋味可言,这已经让他食不下咽,不能处理政务倒还罢了,他本就厌烦这些琐事,可是不能听音乐看歌舞,已经让他郁闷不乐,更难忍受的是他是每日不可独宿的,不近女色让他烦躁苦恼,却还要苦苦忍受半个月,更要在侍中郑瑕的监管之下恪守各种禁令,若非此事重大,他早就不肯忍耐了,心里正想着日后如果自己登基,再举行祭祀绝对不能这么麻烦的时候,送午膳的内宦已经到了,将那些青菜萝卜之类的菜蔬放到桌子上,再端出一碗糙米饭,然后是一壶茶,李安再次诅咒了一次老天,然后拿起了筷子,草草的用了膳,然后他开始喝茶,茶一入口,他心中就是一阵愉悦。

早在他入东宫斋戒的时候,就考虑到粗茶淡饭未免太苦了,早就命人将送来的粗茶偷偷换上参茶,这是夏金逸出的主意,若没有这参茶,只怕他早就因为饮食不如意而形容憔悴了,可惜,若是能有一壶酒就好了,喝了一杯,他觉得精神好了许多,便将参茶放到一边,准备下午读经的时候再喝。

来撤膳的小太监手脚轻快,很快就完成了工作,然后郑瑕亲自送来他下午该诵读的经文,李安不耐烦的看了一眼经匣,便先去午睡了,可是多日以来养精蓄锐,让李安更加想念那些爱宠,翻来覆去就是睡不着,不由想起淳嫔,多日不见,不知道她情况如何,越想越是心中痒痒。忍不住坐起身来,心道不如到外面走走,免得这样辗转反侧。

走出寝殿,只见东宫侍卫环伺,而侍中郑瑕却不见影踪,代替他的是一个礼部官员,他随意问道:“郑大人呢?”那个官员诚惶诚恐地道:“殿下,韦相派人请郑大人去商量祭奠的事情,要等到未时末才能回来。”

李安一喜,若是郑瑕在此,他不敢放肆,可是郑瑕不在,那么自己在宫院里面散散步就没有关系了,抬头一看,自己的亲信侍卫夏金逸正在旁边侍立。他低声道:“金逸,孤想玩玩投壶,你去悄悄的拿来,不可让别人看见。”

夏金逸听了左顾右盼片刻,道:“殿下稍候,属下这就去拿。”不过片刻,夏金逸果然拿了投壶过来,这是李安心爱的东西,一直放在东宫,常常在看折子烦闷的时候用来消遣,这个银壶乃是广口大腹、颈部细长的形状,内装一些豆子,却是为增加难度而设,如用力过猛,投入的矢会反弹出来,那些用来投壶的箭矢都是精雕细刻,美伦美央。夏金逸递过箭矢,笑道:“殿下还请手下留情,属下上次就输惨了。”

李安笑道:“若论这投壶,你们可都不如我。”说着投出箭矢,果然一箭中的,他得意的一笑,可是接连赢了几局之后,却又觉得意味索然,往日夏金逸总是恰到好处的让李安输上几局,这样一来,李安总是能够反败为胜,自然是十分开心,今日夏金逸却是连连失误,让李安赢得十分容易,他不免没了兴致,不由怒道:“金逸,你是在敷衍孤么?”

夏金逸连忙道:“殿下,属下怎敢敷衍您,实在是属下心中有事。”

李安疑惑地问道:“有什么事情让你如此心事重重?”

夏金逸道:“今日属下收到一件信物,原本应该呈给太子,可是如今正是太子斋戒之时,故而不敢呈上。”

李安笑道:“我当是什么事情,东西拿来吧。”

夏金逸不敢拒绝,连忙从怀中掏出一个织锦香囊呈上。李安接过,只见这香囊十分精美,上面绣着并蒂莲花,他心中一动,将香囊打开,里面除了香包之外,却是一条薄如蝉翼的翠绿丝帕,他将丝帕展开,只见那丝帕上绣着一对红羽白首的交颈鸳鸯,下面还有一行小诗,“天阶遥望隔云烟,相思几重残月天。今宵红豆重有约,玉露金风到枕边。”李安只觉得心中一荡,这丝帕情意缠绵,莫非是淳嫔托人送来。

正在他遐思逸想的时候,夏金逸已经说道:“殿下,来送此物的乃是淳娘娘身边的亲信小太监,可是殿下如今正在斋戒,此物未免不妥,故而不敢呈上,可是若是扣了下来,又是对殿下不忠,因此属下十分为难。”

李安笑道:“你有功无罪,好了,你下去吧,本王也该念经了。”夏金逸连忙收起投壶,退了下去。

下午的时光,李安表面上看着经书,心中却在盘算,淳嫔一定是邀我今夜私会,可是我如今不能近女色,这可是万万不行的,可是一想起淳嫔那娇艳美丽的容貌,因为长期练习舞蹈而来的迷人体态,他就心中痒痒,再说上次和萧妃争执之后,他已经没有进宫和淳嫔私会了,现在他在东宫斋戒已经有十二天,早就已经孤枕难眠,一想到淳嫔今夜会等候自己前去相会,不由心猿意马,浮想联翩。

到了夜里,躺在床榻之上,李安越想越是睡不着,终于站起身披了一件衣裳,看见在外面伺候的小太监已经熟睡,他轻轻走到殿外,看见几个侍卫正在守夜,他到了偏殿看见夏金逸正在和衣而睡,这是侍卫们在东宫伺候的规矩,他上前轻轻推了夏金逸一下,夏金逸立刻惊醒,他还没有资格在宫中佩刀佩剑,手向腰间抚去,李安知道他腰间藏着暗器,连忙低声道:“是我。”

夏金逸立刻清醒过来,连忙起身下拜,正要问安,李安已经挥手阻止,他低声道:“你陪我去看看淳嫔,别惊动了外人。”

夏金逸大惊道:“殿下,万万不可,这事如果传扬出去,只怕皇上震怒。”

李安笑道:“没事,不会有人知道的,我们快去快回,不会有什么妨碍的。”夏金逸苦苦劝解,可是李安却恼怒地道:“平日你对孤百依百顺,怎么今日这么执拗,还不起来,和孤一同前去。”

夏金逸眼中闪过一丝绝决,道:“属下遵命,只是殿下这样出去不免有些不妥,不如换了衣服。”李安心想有理,便换上一件侍卫的衣服,带着夏金逸两个人偷偷向淳嫔的住处潜去,虽然宫中侍卫不少,可是夏金逸最是擅长偷鸡摸狗,带着太子居然没有碰到多少人,一次碰上了巡夜的禁军,也被夏金逸拿着东宫的侍卫腰牌,用花言巧语敷衍过去。

到了淳嫔的住处,李安迫不及待的推开殿门,那殿门果然没有关上,李安向内走去,却是不见人影,他只道淳嫔遣走了宫女太监,匆匆走入寝殿,只见一盏银灯放在桌上,锦榻之上,淳嫔只穿着薄纱睡衣,睡得正香甜,两截藕臂露在锦被之外,越发诱人,而她的心腹宫女却没有相陪,可见必然是淳嫔相候良久,忍不住睡去了,李安心中越发觉得愧疚,而被淳嫔勾起的欲望也更加按耐不住,胡乱脱了衣服,向榻上扑去。

淳嫔原本正在熟睡,突然觉得有人压了上来,她半梦半醒的也无从抗拒,过了一会儿,她从激情中醒来,发觉身上有人正在肆虐,原本就要惊呼,可是那熟悉的感觉让她没有喊出来,借着昏暗的灯光,她看清了男子的身份,心中不由一震,怎么太子会在斋戒期间前来和自己私会,可是不过片刻,太子的疯狂就让她沉迷其中,再也顾不得盘问了。

他们在抵死缠绵,夏金逸却是心中一片惊惶,他暗暗的查看了一下,所有的太监宫女都睡得很沉,显然是被人轻轻点了睡穴,看来这里是一个已经设好的陷阱了。而太子就是落入这个陷阱中的麋鹿,自己就是帮助收紧绳索的帮凶。可是他转念一想,太子如此行径,又有什么值得同情呢,自己还是赶快服下药物,免得惨死才是真的。

他连忙拿出江哲给他的药丸,先服下绿色腊衣里面的药丸,一种沁人心脾的淡淡香气让他心旷神怡,然后又把黑色腊衣的药丸藏好,可不要不小心失去了。他站在寝殿之外默默的等候着,却不知等候的是太子出来还是此事揭穿时候的*。

就在太子进入淳嫔寝宫不久,在斋宫守戒的李援睡得正安稳,他年纪已老,多日斋戒只当是清心寡欲的休养罢了,突然,半梦半醒中,他看到窗纸上一片红彤彤的,不由披衣起身,高声问道:“高厚、冷川外面发生了什么事情?”

一个四十多岁的杏衣太监匆匆进来,禀道:“陛下,是东宫走水,现在侍卫们正在救火,冷总管在外面护驾呢。”

李援心中一惊,今天已经是十二日,怎会在祭典之前发生这种事情,真是大大的不吉利,想起是东宫走水,他心中泛起不像的预感,问道:“太子殿下呢?快去把他接过来,不可让他出了差错。”

高厚有些神色不安,偷眼望去,却是不敢说话,李援微怒,问道:“怎么了,可是太子受了伤?”

高厚不得不说道:“殿下在东宫斋戒,是由郑侍中负责的,可是今夜东宫走水,郑侍中派人去救太子,却发现太子不在寝宫。”

李援只觉得一盆凉水从头上直泼而下,心中一片寒冷,他缓缓问道:“太子去了哪里?”

高厚冷汗淋淋地道:“奴婢也不知道,不过刚才郑侍中派人查问,说是,有两个东宫侍卫去了含香殿。”说到这里,已经是战战兢兢了。

李援呆若木鸡,道:“含香殿,淳嫔,哼,冷川,你跟朕去一趟含香殿。”

身影一闪,一个身穿御前侍卫总管服色的中年人走了进来,这个中年人相貌平平,却是气度雍容,双目开合之间寒光四射,他是雍帝的亲信侍卫,一身武功登峰造极,最受李援信任,如今更是大内侍卫的总管,备受帝宠。他淡淡道:“陛下不要过于烦恼,以免伤了身体。”

李援冷冷道:“好了,快些去含香殿,吩咐夏侯,将东宫所有侍卫太监宫女全部监禁起来,不得有误。”

李援带着冷川、高厚和几个侍卫太监,匆匆赶到含香殿的时候,这里还是波澜不惊。全然不知东宫那边出了问题。李援使个眼色,一个侍卫上前,一脚踢开了殿门,正在前面守卫的夏金逸打了一个激灵,抬头看去,只见月色之下,雍帝李援怒气冲冲的盯着自己,他心中反而平静下来,转身呼喊道:“皇上驾到。”

李援眼中闪过凶光,也不用他吩咐,冷川身形一闪,一掌重重的打在了夏金逸的背心,夏金逸只觉得自己腾云驾雾一般飞起,身形种种的撞击在墙上,狂猛的内力顷刻间涌入自己的经脉当中,夏金逸眼前一黑便昏死过去。

李援看也不看那个被杀的侍卫一眼,闯进寝殿,只见自己的长子脸色惨白,锦榻之上,淳嫔身无寸缕,正吓得六神无主。李援只觉得五内俱焚,头晕眼花,一个踉跄就要跌倒,却被高厚和几个太监扶住。李援也不说话,怒道:“冷川,还不给我把这个逆子杀了。”

冷川目光一闪,却不敢奉旨,默然不动。李援怒道:“怎么,你连朕的话也不听了么?”

冷川淡淡道:“陛下,太子乃是储君,就是有罪,也得明诏天下,焉能如此处置。”

李援原本只是气急攻心,冷川这一句话让他冷静下来,这时候李安已经清醒过来,扑上前连连叩首道:“父皇饶命,父皇饶命。”

李援嫌恶的看了他一眼,一脚踢出,将李安踢飞到一边,道:“高厚,你将这个逆子送到‘锦安殿’软禁起来,不许任何人探望,还有,将这含香殿上下全部给朕处死,淳嫔,淳嫔,朕不想再见到她。”说罢,李援转身出去。冷川连忙跟上。

高厚却奉旨留下,他到殿外一声招呼,一干侍卫虎狼也似的冲进含香殿,不过片刻,含香殿的太监宫女都已经被勒死,他们大多都刚刚从睡梦中醒来,还不知道发生了什么事情就已经丧命了。而夏金逸则在李援等人进入寝殿的时候醒了过来,他艰难的拿出黑色腊衣的药丸,里面是一颗气味古怪的药丸,夏金逸心道,我是死是活全看你了,服下药丸之后,夏金逸只觉得四肢麻木,周身上下无法动弹,眼睛也无力睁开,只是偏偏还有一丝感觉。不多时,李援走了,那些侍卫开始奉旨灭口,到了他的时候,一个侍卫探探他的鼻息,说道:“这人已经死了,其实不用看的,冷总管手上焉能有活口存在。”

\chapter{第十章 心狠手辣}

这些侍卫走后,自有人将这些尸体送到西宫里面的化人场,这些人的尸体可没有下葬的风光,只能塞到炉子里面火化了事,这些事情自有那些粗使太监去做,也无人顾及,因此也就没有人注意到在火化之前,少了一具尸体,就算有人注意到,也不会自寻没趣。

六月十三日,东宫走水,太子被禁的消息已经传的沸沸扬扬,太子少傅鲁敬忠和靖江公主李寒幽、太子侧妃萧兰也不顾什么嫌隙,聚在一起商量对策,可是却是束手无策,太子作出这等事情,无论如何是不能立刻让皇上消气的。三人愁苦之时,突然有人笑道:“怎么,遇到难题了么?”

三人抬头看去,只见门口站着一个布衣女子,虽然相貌平平,但是那一种凌人的气势却是让人不可小看。萧兰和李寒幽大喜,起身道:“大师姐,是您来了。”

闻紫烟笑道:“不仅是我来了,师父他老人家也已经到了,就在栖霞庵清修呢。”

萧兰和李寒幽又是欢喜又是担忧,她们战战兢兢的看着闻紫烟,萧兰鼓起勇气道:“我们办事不利,门主若是怪罪下来,还请师姐为我们美言几句。”

闻紫烟微微一笑,道:“好了,师父她并没有生气,你们先去见她吧,有什么事情让师父作主,也免得你们这样烦恼。鲁少傅,你也去吧,师父说想见见你。”

李寒幽等人大喜,匆匆换了便衣,飞马出城,一路上也顾不得引人注目,直到了长安东郊外的一座庵堂,才住马缓行,鲁敬忠马术不精,落在后面,李寒幽和萧兰也顾不上他,将马匹一丢,便走进栖霞庵,这座栖霞庵有数亩方圆,乃是凤仪门的产业,每次凤仪门主进京,都是在这里居住,两人一边往里走,就发现平日照料这里的女尼已经踪影不见,通向门主居处的林荫小道上两旁侍立着无数青衣女子,都是身佩长剑,面寒如霜。两人到了门主居住的梧桐轩门前,只见门前左右各站着四个女子,都穿着雪色罗衫,虽然没有钗环锦饰,可是衣衫也都十分华美,两人连忙施礼,这四个女子容貌虽然不过三十多岁,却都是四十岁以上的年纪,她们都是凤仪门主的亲信,当年曾经陪着梵惠瑶转战天下的侍女,因此地位十分尊崇。

两人走进轩内,梧桐轩内陈设十分清雅,地上铺着雪白的毡毯,四周墙壁上都垂着淡青帷幕,一道珠帘从中将房间分为两半,帘内隐隐约约放着一张胡床,一个身穿雪衣的女子侧倚在胡床之上,珠帘隔绝,因而看不到她的神情容貌。

萧兰和李寒幽在帘前跪倒,齐声道:“弟子叩见师尊,我等无能,还请门主责罚。”

那个女子开口道:“这也怪不得你们,你们也已经是尽力了。”那声音如珠玉一般圆润,却又如寒泉一般清冽,虽然看不到神情相貌,可是这女子一开口,淡淡的威仪就笼罩在雅室之内,萧兰和李寒幽却是不敢懈怠,两人交换了一个眼色,萧兰开口道:“师尊,都是弟子无能,太子殿下和淳嫔私通,弟子已经知道,并想方设法想要太子断绝和那个女子的往来,可是太子殿下十分恼怒,不肯听从,还为此和我们生出嫌隙,弟子不得已只得另寻蹊径,没想到竟在这时出了问题。”

那个女子长叹一声,道:“太子殿下不肯听从,为何不让纪霞设法杀了淳嫔?”语气温柔中带着冷肃。萧兰吓得冷汗直流,说不出话来,李寒幽连忙道:“此事已在筹划,我们万万想不到太子会在斋戒期间去和淳嫔私会,原本是想等到祭典之后再动手的。”

那个女子淡淡道:“也罢,事已至此,追究也已经是没有意义,寒幽可知道如今形势如何?”

李寒幽膝行一步,恭谨地道:“皇上已将与此事有关之人全部赐死,淳嫔也已经投缳自尽,太子幽禁宫中,皇上还没有进一步的处分,另外,陛下今日诏丞相韦观、侍中郑瑕、抚远大将军秦彝、魏国公程殊进宫商议,只怕日内处分就要下来,弟子已经拜托驸马向公公求恳,求他替太子求情,但是据驸马说,公公不置可否。”

那个女子叹息道:“这件事情不同寻常,无论什么人求情,皇上也不会消怒,唯今之际,只要暂时保住太子的储位就还有转圜的余地,否则可就是平白的让雍王得逞了。本座方才已经传下令旨,发动全部力量,压制意图倾覆太子储位的势力,只有雍王那里,必须要本座亲力而为才行。”

李寒幽疑惑地道:“门主,雍王觊觎太子储位已非一日,如何肯在这个时候隐忍呢?”

那个女子淡淡道:“若是平时,他自然不肯,可是这次他却不得不从,锦绣盟的事情,就是他最大的致命伤。”

李寒幽一愣,道:“门主,锦绣盟的事情和雍王有什么相关么?”

那女子冷冷道:“寒幽你还是太年轻了,我且问你,若是太子和锦绣盟勾连走私,真的能瞒过雍王的耳目么,这大雍天下,军方势力倒有半数在雍王掌握之中,若不是他有意纵容,太子岂能如愿以偿?”

李寒幽道:“可是当时江哲重伤,雍王为此忧心如焚,哪里还有精力管这些事情呢?”

那女子笑道:“寒幽,你可知‘量小非君子,无毒不丈夫’的道理,若是雍王真的会为了一个江哲就忘了天下,那他也不配做本座的对手了,再说,锦绣盟本来在南楚是千夫所指的叛逆,怎么有本事和南楚做起了生意,那天机阁虽然神秘莫测,可是它是南楚的势力却不会有错,若非是雍王,谁能让原本受到大雍军方支持的锦绣盟和南楚势力媾和,本座想来,那天机阁就算不是雍王的属下也是和雍王有着莫大的关联。那江哲在南楚虽然地位不高,可是此人用计神鬼莫测,我当初让你刺杀此人,原是防范于未然,可惜却是功亏一篑。”

李寒幽谨慎的问道:“若是锦绣盟为雍王所使,那么门主为什么却四处追缉霍纪城呢?”

那女子叹了一口气道:“若是真要追缉那霍纪城,不如去盯着雍王府那,寒幽,你可知道近年来江湖上有很多人不愿意屈从我凤仪门的权威,可是我凤仪门乃是白道领袖,又不能随便镇压,若没有这个借口,我怎能找机会把那些野心勃勃的帮派一一铲除。他们想要让霍纪城兴风作浪,在外面败坏太子的声誉,本座却是利用了这个机会铲除异己,再说太子的名声和我们有什么相关,他名声差些,就更离不开我们的支持了。只是这次太子太过分了,授人以柄,我们若不出手,只怕他这储君的位子就不保了。”

李寒幽眼中一亮,道:“门主,若是我们趁此机会和雍王商量,若是他肯乖乖听话,我们就让他登基,也免得扶持这个扶不起来的阿斗。”

那女子怒道:“糊涂,若是雍王肯这般听话,我当初何必要选择太子作为傀儡。”

李寒幽吓得拜伏于地,不敢出声。

过了片刻,那个女子语气淡然地道:“好了,兰儿,你先回去安抚太子妃和上下人等,就说本座定会保住太子的储位。”

萧兰神色犹疑,却是不敢多问,再拜道:“弟子遵命。”悄然退出。又过了片刻,那个女子语气淡然地道:“罢了,寒幽你说得也不无道理,太子如此失德,我们辅佐他也不免落人话柄,等我见过雍王之后再说吧。不过你还不可以出去胡说,这件事情事关重大,不可传扬出去。”

李寒幽这才松了口气,道:“弟子鲁莽,请门主恕罪。”

那个女子叹息了一下,道:“寒幽,你可知道那诱惑太子的夏金逸是何人?”

李寒幽惊道:“弟子只知道他是崆峒弃徒,一个无行浪子,门主为何问起他呢?他不是已经死了么?”

那个女子沉默片刻,道:“他虽然死了,但是有一件事情还是得让你知道,他的本名乃是夏全。”

李寒幽喃喃的念了几遍这个名字,目光从迷惑变得恐惧,她面如死灰地道:“师尊,他怎会活着,您不是答应过弟子不会留下后患么?”

那个女子冷冷道:“你是在责问本座么?”

李寒幽猛醒,连忙下拜道:“弟子不敢,只是一时情急,求门主宽恕。”

那个女子幽幽一叹,道:“孩子,当日靖江王妃求我去寻找她和王爷所生的爱女下落,当年王妃待产之时,正值贼兵犯境,王妃失落郡主,痛断肝肠,可惜我后来仔细查访,这个女婴早就死在乱军之中,本来这件事情也就算了,可是那日惠秋路过你家,见你资质过人,不忍你良质美玉被弃民间,将你带了回来,当时并没有用你冒充郡主的意思,所以只是杀了你的公婆,免得他们四处宣扬此事,毕竟你已经是人家的媳妇了,不料我一见你,就发觉你和靖江王妃品貌相似,这才动了李代桃僵的心思。原本只是想这个出身对你有利,如今果然是起了作用,可是当初我派人去斩草除根的时候却出了问题,你那个夫婿返回崆峒之后已将此事禀明师门,虽然他们没有证据知道你被凤仪门带走,可是也已经有了怀疑,这样一来杀人灭口就不免露了形迹,所以我虽然答应你,却不能办到。原本想等到他下山之后,想个法子让他死于非命。可是他却很快就被逐出师门,我猜想必然是崆峒掌门不愿和凤仪门为敌的缘故,因此就更不愿杀了他,否则他一条贱命死活没有关系,却做实了凤仪门杀人灭口的事实,后来我安排监视夏全的人回报,这人不堪上进,不会有什么危险,我想你已是皇室中人,怎会有机会和他相见,所以也就没有再留意他,想不到你们竟在太子府上见面了。”

李寒幽神思不属地道:“师尊,你说,他是不是认出了我?”

那个女子微笑道:“无论他是否认得你,如今已经死在了冷川掌下,尸骨成灰,你还怕甚么,不过不知道他有没有把事情告诉别人,你知道他有什么亲近的人么?”

李寒幽想了一想道:“只有两个人可能知道,一个是王妃的侍女绣春,一个是他的师兄张锦雄。”

那个女子冷笑道:“那么应该如何作,用不着我说了,是么?”

李寒幽犹豫地道:“张锦雄乃是崆峒掌门弟子,只怕是杀不得的。”

那个女子想了一想,道:“他就先留着,崆峒现在谅也不敢和本座为难,只是要严密监视,不可让他将这个消息流传出去,你的身份,如今已经是至关重要的了,绝对不能泄漏给人。”

李寒幽咬牙切齿地道:“师尊放心,此事关系弟子一生荣辱,弟子绝对不容许有人破坏我的努力。”

那个女子淡淡道:“好了,你去吧,鲁敬忠也来了,就让我见见这个少傅大人吧。”

离开栖霞庵,李寒幽看着正随着闻紫烟走进庵堂的鲁敬忠,银牙一咬,转身向京城奔去,她心里只有一个念头,绝对不能让那人毁了自己的心血,朦朦胧胧中,她仿佛回到童年,自己明明是天生丽质,聪慧过人,却不敢显露,只因为她常常听见公公说着“女子无才就是德”,要不是夏全替自己遮掩,自己只怕没有机会读那几年书,因为自己是女子,私塾的先生也没有教自己什么经史,只是教会自己认字之后就让自己随便翻看藏书,这是因为自己聪明伶俐,讨他欢喜,可是看来看去,她更加向往外面的世界,读到那些风景名胜的诗句,她就想去看看和这个荒僻村子不一样的动人风光,读到那些描写荣华富贵景象的诗句,她又想去品尝一下那样的滋味,越读她就越怨恨自己的处境,可是她知道自己一个弱女子,是没有可能离开这种地方的,无力自保的她只能沦为奴婢娼妓,所以,她满怀委屈的嫁给了夏全,那个宽厚听话,却没有一丝让她心动的少年。

可是突然,她的机会来了,那些佩剑女子一个个神采飞扬,她们有着不一样的人生,所以她极力接近她们,虽然不知道会有什么结果,可是她不愿意放弃唯一的机会,很快,那些女子就注意到了自己,惊叹着道:“如此良才美质,怎可荒废在山林。”然后就要强行带走自己,可是公公婆婆自然不愿意,那些女子毫不手软,丢下了银子就将她带走了,在路上,她听见她们低低说着,已经除了后患,她明白这些人的意思,可是却没有丝毫同情,凡是阻碍她得到幸福的人都该死。然后就是梦幻一般的生活,她成了凤仪门主的关门弟子,靖江王爷的郡主,她抓紧一切时间充实自己,她绝对不容许再度失去这样的生活,终于,她蜕变成美丽的凤凰,这是她应得的报偿,绝对不容许任何人破坏。

没有走正门,她施展轻功进入到了太子府邸的内室,太子妃崔氏正在佛堂诵经祝祷,那个侍女绣春果然在佛堂外面守候。李寒幽看四下无人,上前轻轻点了绣春穴道将她带到花园中偏僻之处,解开她的穴道,冷冷问道:“夏金逸有没有跟你说过和本宫有关的事情?”

绣春面无血色,呐呐道:“婢子不明白公主的意思?”

李寒幽冷冷问道:“你有没有和你提起过我?”

绣春一边摇头,眼中闪过疑惑的神色,李寒幽心中稍安,摸摸剑柄道:“夏金逸已经身死,你既然是他的相好,就该殉情而死。”

绣春眼中闪过惊惶,连连叩首道:“公主饶命,公主饶命。”

李寒幽冷冷道:“怎么,你不想为他殉情,看来也是个水性杨花的女子。”

绣春哭泣道:“公主饶命,奴婢已经怀了身孕,不敢寻死,若是公主定要奴婢去死,也求公主让奴婢生下了孩儿再死,金逸只有这一点骨血,他家数代单传,求公主让绣春苟活几日,若是侥幸生了男孩儿,绣春死了也可瞑目九泉。”

李寒幽手一抖,想起当日夏母在自己和夏全成婚之时,温和地道:“孩子,夏家数代单传,如今就要靠你开枝散叶了。”心中一软,就要罢手,可是转念一想,自己有今日荣耀岂是容易,为了学习礼仪,自己日夜练习,直到无论何时都不会改变仪态,学习武功,攻读经史,十年寒窗,才成了今日的靖江公主,这个女子虽然什么都不知道,可是今日自己这般盘问,就已经露了形迹,想到这里,狠狠心肠,弹指点了绣春的死穴。绣春正在哭泣,促不及防,就这样无声无息的死去,面上的凄惶之色仍然清晰可见。

李寒幽上前将绣春抱起,她早就知道这个女子的住处,这本是她从监视太子妃的记录中知道的,李寒幽将绣春放回她自己的卧室,伪装成自缢身死的模样,也不敢再看这个女子死灰一样的面庞,转身离去。还有一个张锦雄,李寒幽心里想,他也有可能知道我的身份,绝对不能让他泄露给别人知道,现在不能杀他,可是也不能让他跟被人通消息,对了,就说夏金逸涉嫌诱惑太子,张锦雄身为师兄也有嫌疑,命他待在府中,不许出去。一边想着,李寒幽露出得意的神色。

\chapter{第十一章 魔宗之秘}

鲁敬忠坐下之后,神色更加从容,微笑道:“门主可能知道,我们魔门传承分为三支。”

帘中人开口道:“不错,据本座所知,魔门分为烈日、寒月、隐星三支宗门,如今的魔门宗主乃是日宗所出,而鲁大人你却是月宗元老,日宗弟子,武功超群,月宗门人却是擅长谋划,只有隐星已经多年不见传承。”

鲁敬忠正容道:“门主果然知之甚详,我们魔宗自古以来流传四句话,所谓‘乾坤乱,烈日现,寒月辅,隐星守’。门主可知道其中之意。”

那个女子早已经端坐在胡床上,听到这几句话,站起身来,在帘内缓步而行,淡淡道:“想必是说,若是天下大乱,日宗弟子就要出来造反起事,而你月宗弟子是辅佐日宗的军师,不过这‘隐星守’是什么意思,是说守护日宗么,不对啊,日宗武功高强,何必人守护,还是说星宗隐逸不出,也不对啊,你们的星宗只是听过名字,从未见过传人,本座已经糊涂了,还是请鲁先生直言相告吧。”

鲁敬忠敬佩地道:“门主已经猜得八九不离十了,不过其中稍有差池,我魔门宗旨,就是为了天下百姓,天地不仁,以万物为刍狗,圣人不仁以苍生为刍狗,我魔门就是为了挑战权威而生,故而每当朝政败坏,我魔门必然要出现,让这乱世越乱越好,将那些权贵豪门一扫而平,日宗弟子自然是先锋大将,我月宗弟子就是辅佐的军师,我们通常各自辅佐不同的主君,这样一来,可以让他们互相残杀,这留下来的胜利者面对满目疮痍,自然只能让民众休养生息,这也是祖师爷而星宗么,则是魔门最神秘的一宗,他们的事情就连我们也不知道,故而无法向门主解释。不过目前局势出了意外,当初,日宗弟子京无极登上魔宗宗主之位,全力支持杨老生,遭到惨败,而我们月宗却依旧各自为政,所以元气还在,如今京无极远走北汉,还要继续和大雍为难,就是为了消耗大雍的实力,可是人谁没有私心杂念,我们这些留在大雍的月宗弟子实在舍不得现在的权势富贵,也不愿看日宗压在我们头上,我们情愿和门主共享富贵,辅佐太子登基,到时候岂不是双方如意。”

那个女子沉思片刻,道:“你说得有礼,有了你的存在,太子虽然对我们忌惮,可是也就敢放手让我们施为,你我双方虽然对立,可是却是有好处的,也罢,我们不会揭穿你们的身份,今日之事,就当作从未发生。”

鲁敬忠正色道:“不过目前门主想必有心抛弃太子了吧?”

那个女子沉默片刻,淡然道:“本座不愿相瞒,太子胡作非为,我们若要支持他,只怕名声受损,你们魔门可以为所欲为,我们却不能如此。”

鲁敬忠笑道:“常言说锦上添花不如雪中送炭,说句不当的话,现在雍王用不着您呢。”

那个女子叹息道:“总得试一试,无论如何,雍王乃是明君之姿,若是能够礼遇本门,那么本座放弃的也是心甘情愿。”

鲁敬忠淡淡一笑,道:“我们却是辅佐定了太子的,若是门主也下定决心辅佐太子,在下倒有一个法子,可以保住太子。”

那个女子冷笑道:“还有什么,不过是诋毁有人暗害太子么?”

鲁敬忠毫不脸红,道:“正是如此,我已经在皇上派来调查的侍卫中安插了人,他们会说,太子当日所喝的参茶当中被人混入春药,太子因此乱了神智,而淳嫔因为担心自己日后凄凉,从前时时勾引太子,并买通了太子身边的侍卫送来情书绣帕,所以太子乱神之后,就去了含香殿,这样一来,皇上就会去查谁下得春药,反而不会过多怪责太子。”

那个女子冷笑道:“你想把事情推到雍王身上,只怕没有这么容易。”

鲁敬忠冷笑道:“不论皇上怀疑是谁,暂时就不会废了太子,时间长了,自然就会淡忘此事,再说,皇上如今年事已高,只要拖上一年半载,我看就够了。”

那个女子沉默片刻,道:“本座若有决定,会通知你的,你先尽力而为吧。”

鲁敬忠起身告辞,说道:“门主不必多想,雍王雄才大略,岂容有人掣肘,门主怜惜天下苍生,希望能够借用新君之手,匡扶社稷,可是在人家看来,却是谋夺他们李氏江山。”

凤仪门主微微一叹,没有说话。

鲁敬忠走后,闻紫烟上前道:“师尊,你可相信他们么,魔宗之人都是心思奸诈之徒。”

那个女子冷冷道:“他们虽然奸诈,可是也有作用,让他们多担些恶名有什么不好,等到事成之后,就说是他们调唆太子,将他们全部杀了,也是名正言顺,到时候谁还能和我们争夺天下,你这些师妹,一个个骄纵任性,成事不足,败事有余,这次本座亲来坐镇,我倒要看看谁还能翻了天去。”

闻紫烟真心诚意地道:“门主神威,必然马到成功。”

那个女子淡淡道:“也不能大意,在雍王身上,我们失手多次,这次可不能坏事了,等我见过他之后,他若再不识时务,就休怪本座无情了。紫烟,本座并非看重权势,只是我真的不放心将天下交给他人,不论一家一姓,乃至一个朝代,无不是其兴也勃焉,其亡也速焉,我只望凤仪门可以代代暗中控制朝政,可以让百姓安康,不再受离乱之苦,你本是我心爱弟子,可惜少了几分谋略,不然我必将门主之位传给你,让你继承我的大业。”

闻紫烟肃然道:“师尊,不论您将门主之位传给何人,弟子都会遵从师尊之命,监视她们的行为,若有违背师尊的训示,弟子必定取她性命。”

凤仪门主满意的点点头,道:“我尚未决定,不过无论如何你都是地位超然的监察使,本门这些年苦心栽培的武力也都交给你管理,你要好好做事,先完成这大业的第一步才是真的。”

闻紫烟欣然道:“弟子谨尊教诲。”

当夏金逸从昏迷中醒来的时候,他真的满怀感激,真的活下来了,江哲没有杀人灭口,自己真的死里逃生了,呻吟一声,他坐了起来,看到旁边的椅子上放着清水和方巾,他跳下床,惊奇的发觉身上已经没有异样,难道那些药那么好使么,他迅速的洗过脸,换上旁边准备好的一件单衣,然后看看门,无法决定是否要自己出去,无论如何,现在自己身份尴尬,卧底是不能做了,自己已经是个“死人”,最方便的处理已经是杀了自己,不过他们既然费力救了自己,应该不会杀人灭口吧,正在胡思乱想,自己见过两次的赤骥已经走了进来,看到夏金逸正在呆呆的坐着,目光闪过一丝惊诧,开口道:“夏兄真是好底子,受了重伤,又有毒药挞伐,居然还是生龙活虎。”

夏金逸反应过来,道:“怎么,不是大人的药物的作用么?”

赤骥看了他一眼,道:“这个公子没有说过,公子说,最近局势不稳,让夏公子在这里住一段时间,等到大局稳定之后,再来和公子相见。夏金逸坦然道:“全凭吩咐,不知道我可否自由行动?”

赤骥道:“这个院子公子可以随便走动,但是不要离开,等到局势稳定之后,公子就可以自行决定行止,不知道您有什么喜好,赤骥会替您准备,免得您闲居无聊。”

夏金逸笑道:“这种悠闲生活,我可是求之不得,若是没有妨碍,请替我拿一些曲谱和一管洞箫过来吧。”

赤骥道:“这些院子里面都有,旁边的书房里面有各种书籍可以阅读,这个庄子远在郊外,无人打扰,只要公子不出去,安全定可无虞。”

夏金逸淡淡道:“我已经是一个死人,谁还会留意我,请小兄弟转告大人,我夏金逸情愿俯首听命,绝无二心。”

赤骥神色庄重地道:“公子也有话传下,必然不会亏待夏公子的。”

夏金逸微微一笑,他历经人生巨变,早已经看透了一切,只要心愿得偿,死也无憾,更不会计较什么报偿了。

而在此时,雍王府已经是风云突起,太子突然出了事情,雍王自然也要召集属下商议的,事关重大,就在花厅之中,管休、董志和苟廉,这雍王属下的三杰全部到齐,司马雄去了近卫军镇守,荆迟和长孙冀也都在座,其他的幕僚和雍王亲信的将领也都分列左右,就连几乎从来没有参与过议事的江哲也破例出席,坐在雍王下首悠闲的喝着茶。

众人无不喜气洋洋,这几年来被太子压制,雍王又是一味隐忍,虽然他们也知道不得不如此,可是还是难免郁闷,如今太子被禁,若是能够推波助澜废了太子,岂不是大功告成,所以他们商量的都是如何火上加油,我在一旁笑眯眯的听着,完全不发表意见,李贽几次用目示意,我都装作看不见,现在不让他们发泄发泄,不是自找麻烦么。

李贽虽然也觉得这是一个好机会,可是他总是觉得有些不对,觉得若是这样做会出问题,所以更加希望江哲说出自己的看法,大家争论了许久,都是谈论如何着手弹劾太子,正说得热闹的时候,突然外面传来怒喝声道:“什么人?”

众人一惊,怎么会有人闯进议事厅呢,长孙冀和荆迟交换了一个眼色,荆迟走到厅门,推门出去,只见一个布衣女子身佩长剑,站在不远处,神色淡然,彷佛这是她自己的地盘一样悠闲,虽然被侍卫团团围住,却丝毫没有惧色。荆迟看到这个女子,吃了一惊,上前行礼道:“原来是闻仙子驾到,不知道有什么事情让仙子突然闯进雍王府呢?”

那个女子冷冷的看了荆迟一眼,道:“门主在后面和王妃叙谈,若是殿下有意,门主请殿下后面相见。”

荆迟愣了一下,回头看去,这时候厅中众人都已经听见了闻紫烟的声音,面面相觑,李贽神色肃然,走出厅门道:“本王这就前去拜见门主。”看了一眼江哲,目光中闪过一丝犹豫。

我淡淡道:“请容臣随行,能够一见凤仪门主,幸何如之。”

小顺子这时已经出现在不远处,虎视眈眈的望着闻紫烟,闻紫烟也毫不示弱的看向他,四目相对,却都是寒光四射。

我向雍王行礼道:“殿下,请让小顺子随行伺候,另外,荆迟速到寒园请慈真大师前往会见门主。”

闻紫烟眼神中闪过一丝莫名的寒意,她知道慈真大师到了长安,却不知慈真居然住进了雍王府,这也难怪,慈真大师的行踪岂是平常人可以监视的。

在王府内眷常常游乐的凉亭之内,一个面覆轻纱的雪衣女子负手而立,抬眼望去,不远处正是水光潋滟的小湖,雍王妃高氏带着两个侧妃,恭恭敬敬的侍立一旁,不远处的大树下雍王的两个女儿和江柔蓝正在嬉戏,雍王妃原想把孩子送走,却被那女子阻止,她也不敢违逆,她可是知道这个女子的来历的,就是自己的丈夫来了,也要以晚辈的礼节拜见的。

雍王的两个女儿毕竟是皇室中人,也觉得情况有些异样,不免有些拘束,倒是柔蓝素来受宠,又没有那么多拘束,反而十分快乐的跑来跑去追着蹴鞠用的圆球,踢蹴鞠本来是要比谁踢得花样好看,只是柔蓝年纪小,因此没有法子踢起来,只能踢着球跑来跑去。

雪衣女子看的有趣,笑着问道:“这个小女孩是谁的女儿?”目光落到高氏身上,雍王妃裣衽道:“启禀门主,这个孩子乃是府中司马江哲义女,王爷吩咐臣妾代为照顾。”

雪衣女子目光闪动,道:“好个聪明灵秀的小女孩儿,真是难能可贵。”

雍王妃笑道:“门主说的是,宫中几位贵主也都很喜欢这个孩子,她年纪虽小,却是天真懂事,解人烦忧。不过就是淘气了一些,常常抓着她爹爹当马骑。”说到这里不由忍俊不住,微微一笑。

雪衣女子也是淡淡一笑,她长眉入鬓,原本带些杀气,可是一笑之下,眉目之间多了几丝柔和,一双透彻世情,如同璀璨双星的眼睛也露出了一丝柔和的气息。

然后她的目光便落到远处,那里雍王李贽正向这里走来,在他身后一个青衣男子正在缓缓而行,若非李贽刻意放慢脚步,只怕那个男子早就跟不上了,虽然如此,那人仍然是额头见汗,在他身后,一个青衣少年迤逦而行,虽然距离还远,可是以雪衣女子的武功,自然是看的清清楚楚,数年不久,雍王李贽神情多了几分冷静,少了几分霸气,可是那种由内而外的英风豪气却是丝毫不减,而那个青年男子,相貌斯文秀气,但是那种优雅从容的气度却让他纵在千万人当中也不会黯然失色,最后就是那个青衣少年,虽然穿着仆人的装束,可是那冰寒的双眸,一举一动之间隐隐的风华气度却是非同反响,雪衣女子轻轻一叹,若非雍王如此雄才大略,支持他真是一个好决定,今日若是雍王肯退让一步,那么自己也不妨改弦易辙。

不久,雍王已经到了近前,上前施礼道:“贽拜见门主,多年不见,门主可安好?”

雪衣女子素手虚扶,道:“雍王殿下安好,本座偶来京城,想起昔日沙场相互扶持的情分,特来探望。”

雍王恭敬地道:“门主盛情,贽感激不尽,门主可见过父皇了么,这些年父皇总是惦念着门主,总是说若无门主援手,就没有我大雍的今日。”

雪衣女子淡淡一笑,看向江哲道:“这位就是江司马,本座早有耳闻,今日一见,果然气度不凡。”

我上前施礼道:“晚生拜见门主,今日得见门主风仪,当真是三生有幸。”一边说,我一边打量着凤仪门主,虽然相貌用轻纱隐藏,可是那种睥睨天下的风姿却是遮掩不住,那双灿如明星的眼睛,清净宛如秋日寒江,全无一丝可以分辨的情绪,却又隐隐透着慈悲之意。

凤仪门主看向小顺子,道:“这位就是邪影李顺了,听说你武功不错。”

小顺子冷冷道:“奴婢只是一个下人,不敢当门主赞誉。”

凤仪门主意味深长地道:“你这样的下人,只怕世间也没有几个人用的起。”

说罢凤仪门主淡淡一笑,又说道:“雍王、江司马,这个小女孩儿本座很喜欢,若是你们不嫌弃,就把她送给我作弟子吧。”说罢,她指向柔蓝。我和雍王立时都愣住了。

真痛苦啊,我这两天忙于加班,都没时间写文,真希望在存稿发完之前可以不再加班。

\chapter{第十二章 最终决裂}

悲情通告,已经连续加班数日,明天后天也要加班,因此实在是没有时间写文,这篇文章发完,周末两天暂停,周一我会发文,希望到时候我已经有时间可以写作了。

大雍武德二十五年,六月,帝以太子失德,命太宗代祭于长安。

——《雍史·太宗本纪》

雍王李贽心中思如潮涌,他怎不知道凤仪门主这是向他示好,也是最后一次向他摊牌,虽然他很清楚如果得到凤仪门主的支持,自己的储位便是十拿九稳,可是想来想去,他都不能甘心作一个儿皇帝,若是这次妥协,必然要让凤仪门渗入到自己的势力,到时候自己就很难励志改革了,若是凤仪门主提出收他的女儿为徒,他自然可以当面拒绝,可是凤仪门主却是要收柔蓝为徒,虽然凤仪门主已经是他们的首要敌人,可是不能否认的是,梵惠瑶仍是三大宗师之一,而且很可能是居于首席位置,这样一个人要收柔蓝为徒,这是柔蓝的荣幸,若是自己断然拒绝,江哲会怎么想,想到这里,他抬目向江哲望去。

我的心里也正在翻江倒海,让柔蓝拜她为师,想也休想,我和柔蓝的生身父母都希望她一生活得快快乐乐,我只希望能够让她衣食无忧,嫁一个如意郎君,白头到老,甚至我都不准备让柔蓝嫁到富贵人家,免得那些三妻四妾,自命风流的豪门子弟耽误了她,怎会让她去学什么剑,将来让小顺子教柔蓝一些轻身功夫防身就行了,当然如果她真的喜欢习武我也认了,可是绝对不会让她拜到女暴君门下,可是凤仪门主明显是向雍王殿下求和,如果我断然拒绝,雍王会不会不满呢。

我和雍王四目相对目光中都是忧虑,可是却罕见的没有达成共识,我心中苦笑,凤仪门主果然出手不凡,简简单单的一句话就让我们进退失矩,君臣离心了。

这时,我的身边突然想起小顺子的声音道:“不可……”话还没有说完就中断了,我抬头看去,凤仪门主双目含着淡淡的嘲笑,而眼睛的余光更是看见小顺子满头大汗,神色羞怒。心里知道必然是被凤仪门主隔绝了小顺子的传言,但我素来知道小顺子对于察言观色和随机应变实在是在我之上,灵智一开,我已经想通了雍王的为难之处,便扬声道:“门主厚爱,哲本应代小女谢恩,但是小女自幼孤苦,我们父女相依为命,实在舍不得分开,更何况小女性情顽劣,不堪学剑,哲只望她一生平安康泰,不愿她出类拔萃。”

果然我的话一说完,就听到雍王送了口气的声音。

凤仪门主眼中闪过淡淡的阴蠡,说道:“本座看江司马的诗词别具一格,想不到为人也是这样迂腐,不喜欢看见女子出人头地,是么?”

我恭谨地道:“门主误会了,哲并无此意,只是为人上者,所耗心力必然百倍于人,哲只愿儿女都是资质平庸,不求显达于诸侯,只求承欢于膝下,不求功高盖世,只求耕读传家,国家有难之时,当尽力挽救,国家平安之时,当为社稷之顺民。”

凤仪门主眼中闪过嘲讽,道:“若是人人如此,还有何人能够匡扶社稷,江司马可是过于独善其身了。”

我微微一笑道:“所谓时势造英雄,天下有大志有野心的人数不胜数,可是若是没有平凡的黎民百姓,谁又能掌握乾坤,若是人人都想去做豪杰,那么岂不是天下大乱,我虽然不幸,身处乱世,不得已深陷缧绁,可是绝不会赞同我的儿女也如我一般呕心沥血。”

凤仪门主沉默片刻,道:“道不同不相为谋,雍王殿下,不知道你意如何?”

这人可是人人都知道她话中之意,雍王淡淡一笑,道:“小王也觉得柔蓝不适合学武,若是门主能够见到太子殿下,请代小王问候,就说小王必定上本相保,还请太子殿下平心静气,好好养息。”

凤仪门主微微长叹,我们都是心中一乱,只觉她这声叹息充满了慈悲和惋惜的意味。但是我和李贽却都不为所动,凤仪门主见状,冷然道:“殿下,太子乃是你的长兄,如今他身陷缧绁,不知道殿下是要落井下石呢,还是静观其变?”

她这一问,雍王又是心中苦笑,虽然他和太子已经是不死无休的格局,可是此事如何可以当众说出,言出如风,无论如何,太子是他的君,是他的长兄,私下里自然可以将太子当成死敌,可是当着这么多人的面,若是自己说了出来,只怕是没多久就会传到父皇耳朵里面,就是王府中没有人吃里爬外,凤仪门主也不会守秘的,可是若是自己说是静观其变,那么无论如何,自己这次就不能大张旗鼓的发动对太子的抨击。正在他左右为难的时候,凤仪门主又是淡淡一笑,道:“太子因户部案和锦绣盟案失爱于陛下,不知道雍王殿下如何看法,这件事情,想必雍王殿下清楚的很。”

李贽眉一扬,虽然对这两件事情他不甚明了,可是他很清楚这是谁策动的,他也没有指望这些事情一直隐秘下去,可是若是凤仪门主没有证据的说话,可就怪不得他无礼了。他淡淡道:“这两件事情,天下人有谁不清楚呢,只是碍于淫威,不敢明说罢了。”

凤仪门主冷冷一笑,笑声中带着一丝嘲讽,她缓缓道:“若说证据,本座自然是没有什么拿的出手的,不过殿下应该明白,这件事情若是传扬出去,只怕证据就有了。”

李贽一皱眉,他自然知道若是李援起了疑心,细细查下去,虽然查不出实际的证据,可是一些旁证还是有可能得到的,这样一来对自己便是大大不利,可是就这样俯首,他又不甘心,心中的怒火越来越猛烈,他的眼光仿佛利剑一般看向凤仪门主。

我这时却是胸有成竹地道:“门主放心,我家殿下只是不愿表功,事实上,殿下已经准备上本保奏,多年兄弟之情,数年君臣之义,雍王殿下乃是信人,若不仁至义尽,是绝对不会擅动干戈的。”

凤仪门主听了江哲这一番绵里藏针的话,却不在意,笑道:“那么本座就代太子谢谢雍王殿下了,时间不早了,本座还要去看看几位故人,这就告辞,若有机缘,自然会再相见。”说罢她的目光落到远处,那里不知何时多了一个布衣僧人,她用目光微微致意,也不见如何行动,身形便如轻烟一般,转眼消失不见。这时,我们在场的人才真的松了口气。

李贽苦笑道:“本王突然觉得压力倍增,凤仪门主亲自出马,这次可没有什么希望了。”

我淡淡道:“殿下放心,这次本也不是就要立刻达到目的。”然后看向小顺子,关切地问道:“你没事吧?”

小顺子眼神有些羞怒,道:“我不是她的对手?”

我闻言笑道:“你胡说什么呢,你才多大,和人家宗师级别的高手比什么,再说慈真大师都说你前途无量,一时失手用得着那么难过么?”

小顺子脸色缓和了许多,默默不语,我见他已经恢复正常,这才放下心来,这时慈真大师已经消失不见,奇人就是奇人。李贽含笑看了我一眼,道:“好了,随云,你也别再掖着藏着,有什么打算快说吧。”

我正要答话,这时远处总管常恩匆匆跑来,道:“殿下,宫中有旨意传下。”

这下我们也顾不上说话,先簇拥着雍王到了前厅,红衣使者拿着黄绫诏旨,高声道:“朕命雍王贽代太子持长安陪祭,钦此。”

雍王心中一阵狂喜,却是不露声色,上前接过诏旨,谢恩之后,问道:“请问钦差,本王可否入宫谢恩。”

那个宦官尖声道:“陛下已经提前起驾黄陵,命殿下和韦相、郑侍中商议祭典之事,不过据咱家所知,虽然时间有些仓卒,可是斋戒还是不能免得,陛下已经下旨让殿下即刻到斋宫,奴婢想,郑侍中很快就要到了。”

他还没有说完,已经有人通报道:“殿下,郑侍中奉旨前来,请殿下随他入宫斋戒。”

李贽沉声道:“请郑侍中稍候,本王更衣之后便随他入宫。”送走传旨的钦差,李贽有些忧虑地道:“随云,你说会不会有诈。”

我目光一闪道:“殿下,虽然按理说没有什么问题,可是殿下孤身入宫,臣等无法放心,小顺子武功还不错,让他陪殿下一同进宫,想来郑侍中也不会说什么?”

小顺子脸色一边,脱口道:“公子,你的安危……”

我手中折扇一收,淡淡道:“请殿下传令,到殿下回府为止,府中大小事情,由哲主持。”

李贽立刻道:“金牌在你手上,就是本王亲到,谁敢不听你的命令,你可以立刻斩之,小顺子,这次本王要借重你了,放心,慈真大师就在府上,一定会保护随云的安全。”

小顺子看了我一眼,道:“李顺遵命,请殿下和公子放心,就是凤仪门主亲自出手,小顺子也会舍命保护殿下平安。”

我见众人面色严肃,轻笑道:“大家不用这么担忧,这才我们又不是有什么悖逆之举,只是为了防止有人狗急跳墙罢了,而且凤仪门主既然来了,就不会在这个时候放手施为,毕竟,这大雍还有皇上和宗室在。”

大家这才略略放心,当下雍王到前面去见郑侍中,郑瑕果然没有对小顺子的随行表示什么惊异,雍王如此慎重也是理所当然,很快就请雍王入了斋宫,斋戒沐浴,指点礼仪,雍王是一刻也不得闲暇。他这里繁忙,却让太子一系的人心焦如焚。谁都知道,太子和雍王乃是死对头,此消彼长,去年年初,太子代圣上告祭太庙,自此之后,雍王便偃旗息鼓,甚至忙着在幽州巩固势力,如今雍王取代太子陪祭,那么象征这什么不言自明。太子一系的人自然是议论纷纷,而其中的中坚力量自也不肯放手。

可是李援毕竟是一代霸主,那里不会想到这个问题,这次离京,他将在京禁军交给秦青,李寒幽是太子一系的人,自然不会让雍王动手害了太子,而秦青虽然年轻鲁莽,可是秦大将军可不含糊,留下了自己的亲信副将秦勇监督秦青,这样一来,太子也别想趁机加害雍王,再说,韦相和郑侍中乃是文臣的领袖,有他们坐镇,自然是万无一失。为了安全,郑侍中亲自管理雍王斋戒的斋宫,而太子被软禁的锦安殿则由韦观提议,派其子韦膺看护,韦膺如今虽然已是吏部侍郎,又是皇上心目中的佳婿,又是立场中立,有他守护太子,既不用担心有人暗害太子,也不用担忧太子和外面私通消息,而侍中郑瑕的铁面无私人人都知道,这样一来,等于是太子和雍王双双被软禁起来,反而是齐王比较自由,随驾到桥山祭拜,不用陷入这场政治风暴。

在这种情况下,双方的布置就很重要,既不能惊动了雍帝留下的镇守长安的文臣武将,又需要维持局势,不能让自己的主君覆顶,所以太子府和雍王府联合要求长安戒严,韦观也只能同意,而在这之后,秦青迅速将有嫌疑的不明身份的人士拘押的拘押,赶出长安的赶出长安,而雍王府也不示弱,雍王属下三杰,管休负责雍王府内部事宜,苟廉负责和韦观等人协调,而董志则带着荆迟返回驻扎在长安城外的近卫军,全军备战,司马雄则带着雍王府宿卫随时听候吩咐。而指挥这一切的江哲江随云则寸步不离寒园,而慈真大师则寸步不离他左右,裴云虽然失去了禁军北营的绝对控制,可是毕竟还是控制着大部分力量,有他坐镇,夏侯沅峰就不能随意调动这部分禁军,只能尽量调用大内侍卫,这样一来,双方势力犬牙交错,谁也不敢先动手,更何况人人都知道,凤仪门主已经到了长安。

不过在风浪之中,有一个人却是悠闲自在,那就是我了,我虽然每日留在寒园之中,小心翼翼不敢外出,可是却没有做什么大事,每天的情报我翻阅一遍就归档,各种应变措施也让他们自己去计划,我只负责下几个命令。说也奇怪,我这样可以说是不负责任的行为却有效的让众人心平气和起来,看来是我平日给他们的印象太好,让他们不自觉的相信我了。

其实本来也用不着着急,对我来说,这次唯一的目标就是可以看看太子的势力,我很清楚,这次不是一劳永逸的机会,雍帝若是真的对太子完全失望,早就废了他了,而不是将他拘禁起来了事,这次雍帝是想试探一下雍王,如果这次我们心急火燎的想铲除太子,必然让雍帝认为殿下心肠狠毒,若是毫无准备,又会让雍帝觉得我们过于矫饰,所以我这般外紧内松,既震慑太子势力,让他们不敢趁机生变,也可以让雍帝明白殿下没有谋逆之心,再说,太子储位已经是岌岌可危,我们若是火上加油,只怕反而引起雍帝的同情怜悯,我们只要不偏不倚,那么凤仪门上蹿下跳为太子张目的做法就一览无遗,什么恩情也不能一辈子压人,这次凤仪门主可以靠着过去的恩情说服雍帝恢复太子的尊荣,那么下次那,再说,太子已经失去人心,虽然势力庞大,却已经是纸老虎了,所以这次的事情我的目标只是平安度过,下一步,就可以着手策划真正的夺嫡大计了。

可是就在我悠闲自得的时候,却得到了一个令我意想不到的消息,说起来只是一件小事,可是却让我有些追悔莫及,今日太子妃安排了亲信侍女绣春的丧事,而绣春是自缢身亡,据说死前已经有了数月身孕。这个消息让我十分遗憾,原本我对于夏金逸的私事并不关心,可是这个女子竟然殉情而死,我不杀伯仁,伯仁因我而亡,叹息了一下,决定传个消息给夏金逸,让他知道一下有个女子深爱他至此,只是可惜了那个没有出世的孩子。

而同一时刻,大内斋宫之内,李贽专心致志的诵着经文,坐在屋角默默练功的小顺子睁开眼睛,眼中闪过一丝钦佩之色,虽然他跟随江哲投靠了雍王,可是一直以来,他都对雍王存有敌意,一个原因是当日雍王曾经想要鸩杀江哲,另一个原因却是因为江哲为了替雍王效力,不仅险些遭到刺杀身死,而且还要强行撑着病体为他谋划。所以尽管很感激雍王对江哲的爱重,小顺子仍然是不大愿意理会雍王。可是今日小顺子却是真的敬佩这个皇子。

小顺子不是白痴,他知道自己的能力和地位,做江哲的奴仆是他心甘情愿,可是这并不代表他不了解自己的身价,扪心自问,自己若是雍王,肯定会忍不住招揽这样一个高手,就算不指望自己全力效忠,得到自己好感也是物有所值,他也想过这次和雍王独处斋宫,雍王可能会用一些手段来招揽自己,可是出乎他的意料,自始至终,雍王只是专心致志的学习礼仪,埋头诵经,虽然对自己客客气气,却没有丝毫收买之意。小顺子在雍王府多年,不止一次看到过雍王待人的手段,平心而论,若是雍王对他用上,他也难以视若平常,可是雍王却没有对他说过一句额外的话。

小顺子明白,这并不是雍王看不起自己,而是,在雍王心中,自己是一个恪守忠义的人,这种尊重,才让小顺子真的接受雍王作为江哲的主君。

对于李贽,并非没有想过收买小顺子,毕竟这样一个武功高手,实在值得留在身边,可是雍王并非是一个定要将天下俊杰掌握在手中的人,在他看来,小顺子忠于江哲,那么只要自己抓住江哲,就不用担心小顺子的问题,而且,这样一个雅量高致的人,他又怎会用收买来屈辱他呢。此时的李贽,绝对没有想到,会因为这个缘故让小顺子终于消除了对他的敌意。

\chapter{第十三章 隐星宗主}

夏金逸瞪大了眼睛,寒声道:“你再说一遍。”

赤骥同情的看了他一眼,说道:“绣春姑娘已经自缢身亡,而且已经身怀有孕,我家公子特意派我来通知你。”

夏金逸愣愣的看着自己的双手,不再说话,赤骥退了出去,就在他的脚步刚刚跨出门口的时候,他听见了呜咽的哭声,那是一种痛断肝肠的哭声,赤骥心中一酸,连忙加快了脚步。

夏金逸浑浑噩噩的坐在房间的地上,心中再也没有出现李寒幽的身影,他只是回想着和绣春结识之后发生的一切,从一开始的轻薄玩弄,到后来,这个娇弱的女子已经走进了自己的心灵,多少次两人相拥而眠,一起憧憬着美好的将来,他甚至想,自己过几年囊中丰厚,可以带着绣春远走他乡,故乡是伤心处,是不能回了,可是天下还有很多地方可以让他们安身的。直到,那一天,自己看见了李寒幽,那个吞噬自己的生命和梦想的女子,那个改变了自己的命运,却已经将自己完全忘记的女子,从那一刻起,他的生命就已经终结,他每日只是想着如何讨好那个残暴的太子,如何想方设法的报复李寒幽,所以他心甘情愿的冒着生命之险,完成了江哲交给自己的任务,只因他知道,自己的力量是多么微不足道,对于一个凤仪门弟子,一个皇室公主,一个将军夫人,自己的生死在她来说只是蝼蚁一般,那么想要报复,就只有推倒她所依靠的大树,所以雍王和江哲成了他唯一的选择,可是就是在那段痛苦的日子,他身边也总是有那个倩影,安慰他,鼓励他,让他心中还有一线光明,可是他没有顾及她,在自己接受那个九死一生的任务之后,为了保守秘密,他甚至没有和她道别,他甚至以为,如果自己诈死,那么这个温柔的少女就会忘记自己,就会有属于她自己的幸福人生,可是没有想到,她居然殉情自缢,而且带着自己的孩子走了,多么残忍的决定啊,她为什么要这样绝决,这是报应么,这是他帮助太子残害那么多无辜少女的报应么?

越想越是苦痛,夏金逸只觉得五脏如焚,头晕目眩,很快就昏迷了过去,半梦半醒之中,他仿佛和绣春回到了家乡,男耕女织,过起了悠闲自己的生活,隐隐约约的,好像自己的父母还活着,正抱着自己的儿子笑得合不拢嘴。朦朦胧胧中,夏金逸下意识的运起了师父传授的内功,那是一种没有什么作用,却能让人精神振作,睡眠更好的内功,多年来,夏金逸每日都不间断,虽然没有什么别的好处,可是自己的内力虽然没有增加,可是越来越圆润,而近一年来为了不再梦见李寒幽的倩影,夏金逸可是练的异常努力,今日他痛苦万分,忍不住练了起来,可是练着练着,夏金逸只觉得从丹田升起一股炽热的暖流,夏金逸略一犹豫,那股暖流已经流入四肢百骸,夏金逸只觉得全身经脉好像被烈火焚烧一样,可是奇异的,心中的苦痛居然减轻了几分,心中一动,他继续运功,果然从丹田涌出阵阵暖流,他存心承受最大的苦痛,反而更加认真的运功,那种仿佛撕裂他浑身的痛苦让他心中有些安慰。不知何时,他已经沉迷于其中。

若是有人在这个时候进来,就会看到一桩奇景,一个男子周身真气隐隐,却如烈火焚烧,神色痛苦中带着安详。也是夏金逸运气好,中午来送饭的赤骥看见门扉紧闭,以为他因为伤心而不愿出来,所以只是在外面喊了一声,将饭菜放到桌子上,没有想到进寝室看他,否则夏金逸必然有死无生。

到了半夜子时,夏金逸只觉得从丹田涌出一股清凉的真气,流遍全身,真气所过之处,四肢渐渐复苏,等到真气运行一个周天之后,夏金逸只觉得精神一震,心中的悲伤内疚竟然不再让他痛苦的想要死去了。他坐起身来,只觉得身上一股酸臭,仔细看去,竟是漆黑一片,连忙跑到院子里,提了井水冲洗干净,沐浴之后,他伸出双手,只觉得肌肤白皙得近乎透明,润泽而富有弹性。他不由大惊,不知发生了什么事情,正在这时,身后有人叹息道:“逸儿,你终于突破了七情关了。”

夏金逸回头一看,皎洁的月色下,一个黄衣道士正在微笑而立,那个道士不知多少年纪,相貌秀美,肤若婴儿,但是须发皆白,却又彷佛百岁年纪,夏金逸一声低呼,这人正是自己第二位恩师,天都道士梦道人,他上前拜倒,本来想痛哭一场,却觉得无泪可流,不由心中更加奇怪。

梦道人上前将他搀起,道:“逸儿,有些事情今日你已经可以知道了,为师非是平常人,乃是当今魔门星宗宗主。”

夏金逸微微一愣,他曾听师父说过魔门三宗的事情,到了外面才知道这些事情很少有人知道,也曾经怀疑过恩师可能是魔门中人,可是想到自始至终只有恩师对自己最好,便抛却一边,今日听到恩师亲口承认自己的身份,夏金逸心中反而放下了一块大石。他笑道:“不论师父是什么身份,金逸都不在乎,可是金逸有很多事情都不明白,还请师父告诉逸儿。”

梦道人拉着夏金逸,在院子里的石凳上坐了,微笑道:“好徒儿,为师果然没有选错传人。听我慢慢讲来。我从前说过魔门三宗之事,乾坤乱,烈日现,寒月辅,隐星守,说得正是三宗各有分工,我魔门首代宗主出身寒微,他恨透了那些豪门贵族,认为一个国家之所以衰败,都是因为那些吸食百姓膏血的皇室豪门腐败不堪,他曾经说过,若是君王贤明,百姓不过少受一些苦楚,若是君王昏庸,百姓则是雪上加霜,所以他创立魔宗,为的就是铲平这不平乱世,祖师认为,若是百姓困苦,就要有人揭竿而起,另创新天地,而新朝又能让百姓有百年安康,所以他不希望王朝衰败的的时候,还要让让百姓苦苦忍受,所以他创立三宗,日宗就是揭竿而起的大将,月宗就是促使那些豪门自相残杀的军师,跳起战乱,颠覆朝纲,促使新的局面出现,可是这样一来,若是新朝根基稳固之后,我日月两宗的门人只怕剩不下几个了,战乱纷呈,也难怪如此,可是这样以来,我魔门如何可以维系命脉,所以祖师他智深如海,另外创下了星宗,星宗的宗旨就是隐遁于世,如天上繁星,虽然常见而不相识。而且我们星宗担负着魔门传承的大任,世世代代守护着本门密藏,等到天下乱相呈现,我们就要从那些身份低贱却是心有大志的少年中间选择一些传授他们日宗的武功和月宗的兵法谋略,所以虽然魔门常常被黑白两道和朝廷围歼,却总是死灰复燃,正是我们的功劳。可惜的是,祖师爷也想不出更好的法子让黎民得到安宁。,只能用战乱来涤清世间的污浊,创造新的太平。”

夏金逸眼中闪过疑惑的神色,问道:“师父,那样一来,星宗岂不是成了坐山观虎斗么,挑起天下变乱,本身却置身事外,那岂不是太过分了。”

梦道人苦涩地一笑,道:“傻孩子,你以为星宗的传人很容易找么,星宗代代一脉相传,每位宗主接下上代宗主的衣钵之后,就要寻找可传衣钵的弟子,而上代宗主就要回到我们星宗守护的密藏那里潜心修炼,星宗秘传心法,叫做‘九死神功’,练了这种心法,心脉最是强韧,只要不砍下头颅,那么就绝不会死去,而且这种心法可以让我们活到一百二十岁以上,可是到如今星宗十七代传承,却有两次险些中断。”

夏金逸想了一想,问道:“莫不是,星宗传人有什么特别的要求很难达到。”

梦道人苦笑道:“星宗传人第一项要求是无亲无故,六亲断绝,这一点还罢了,不难找寻;第二项要求是终身不婚不嗣,这一点就已经有些为难了;第三项要求是需在三十岁前饱经风霜,看透生死。这三项要求已经让可以选择的人选寥寥无几,更何况我们星宗还要求传人至少要有中人以上的资质才行。”

夏金逸想了一想,道:“这些条件,弟子确实勉强可以达到,可是弟子相信,若是仅有这些条件,那么也没有什么困难的。”

梦道人深深的看了他一眼,道:“这是因为星宗的宗旨所限,本门弟子,既不能享受荣华富贵,需要四处流浪增长见闻,终生漂泊无家,又不能显露武功,即使遇到生命之险,也只能逃避不能还手,这样一来,虽然身为星宗宗主,却终生默默无闻,这种枷锁岂是一个身负绝世武功的人可以忍受的,所以本门的规矩,三十岁之前若是通过考验,就可以成为记名弟子,从那之后直到六十岁之前可以自由放荡,但是不能修习上乘武功,反正九死神功可以保住性命,若是不幸身亡,只能说明此人性情不能隐忍,不配作星宗传人,六十岁之后,我们才认为可以辨明此人心性,正式收为弟子。”

夏金逸深思地道:“这样说来,弟子并非唯一的候选人?”

梦道人歉然道:“是的,在你之前我已经选择了两个人,可是目前看来你的希望最大,如今你诈死隐身,又是历经惨变,看破情劫,如今你已经突破九死神功的第三重‘七情关’,如果你能够在今后三十年内恪守星宗律令,那么我相信你会成为我的传人。”

夏金逸自从突破七情关之后,只觉得神思敏捷,心中情感渐渐淡漠,也不劝慰恩师,反而追问道:“若是我们几个人都达到要求,那么恩师如何抉择?”

梦道人傲然道:“我魔门强者为尊,若是都通过了,那么自然就要看你们在自相残杀之后谁能活下来了。”

夏金逸淡淡一笑,又问道:“既然如此,我已经取得预选资格,师尊也该教我一些小玩意儿,好让我保住性命要紧。”

梦道人不以为忤,从怀中掏出一本小册子,上面写着一些蝇头小楷,梦道人道:“这些东西都是一些雕虫小技,你学会这些自保应该没有关系,可是你也要明白,如果你不甘寂寞,靠这些东西就可以名扬天下,到时候你就失去继任宗主的资格,不过按照本宗规矩,如果你甘心放弃成为宗主的机会,那么星宗不会收回你的武功,只要你终生不提星宗二字,那么就可以安渡余生。”

夏金逸冷冷一笑,道:“您老真的信任我们这些候选之人么,恐怕是另有控制手段。”

梦道人目光一闪,露出一丝笑意,从怀中掏出一颗红色药丸,道:“这是我魔门祖师在苗疆蛊毒的基础上所研制的真情蛊,只要你服下此药,然后立誓除非成为星宗宗主,否则终生不能提及星宗之事,再经我施以手法,那么就可以了。”

夏金逸接过蛊丸,漠然道:“此药可是师父一声令下,我就会毒发身亡。”

梦道人摇头道:“并非如此,只要你不再提及星宗之事,那么你的生死为师也管不了,而且真情蛊还有一桩好处,就是可以让人延缓衰老,不受其他蛊毒所害,所以为师直到今年八十三岁,需要向你们解释本宗隐秘的时候才解去此蛊。”

夏金逸相信恩师所说没有一字虚假,面色渐渐和缓,他拿起蛊丸,又问道:“恩师,是否徒儿身上所发生的事情你都清楚。”

梦道人微微一叹,道:“为师知道十之八九,当年为师在崆峒山挂单,见你虽然忠厚老实,可是面相却是一生凄苦,所以才留下来观察,你回到崆峒,向师门禀告家中之事,我就已经替你查过,凤仪门派人前来杀你灭口之前,我就想法子让崆峒掌门知道此事,所以他才因为不敢得罪凤仪门,将你逐出师门。你拜我为师,我不教你其他武功,反而将你变成今日的浪子,第一是为了让凤仪门对你放心,第二则是因为你若想要成为本门宗主,若非放荡不羁,自娱自乐,怎能熬过那漫长的岁月。后来你下山之后,我虽然没有跟着你,但是我却事先用重金收买了一个梁上君子,让他跟踪你数年,所以这次你在长安出事,我才会匆匆赶来,唯一可惜的是绣春,我原想你既然已经有了牵绊,我也不再冀望于你,只要你带着那个小姑娘和你未出世的孩子平安离开,我也就和你再无缘分,可是谁知道,这个小姑娘竟然被人杀了。”

夏金逸脸色一变,沉声道:“师父,你说什么,江大人不是说绣春是自缢的么?”

梦道人怜惜的看了他一眼,道:“我到王府的时候晚了一些,绣春姑娘尸体尚温,她是被人点了死穴,虽然隐秘,可是还瞒不过我的眼睛。”

夏金逸嘶声道:“是谁,是谁杀了绣春,她不过是个弱女子,既无威胁,也无价值,谁会杀她。”

梦道人淡淡道:“我去的晚了,没有看见凶手,不过你还猜不到么?”

夏金逸只觉得心如刀绞,侧过脸去不再说话。梦道人叹息道:“这件事情我若是不告诉你,你很有可能成为我的传人,可是我不想你终生遗憾,孩子,今后你好自为之。”

夏金逸看看窗外的曙光,却觉得欲哭无泪,他淡淡道:“师父,我可以做到什么程度,才算有资格参与竞争宗主之位。”

梦道人深深的看了他一眼,道:“我相信你失去资格的时候就会明白,一个小人物可以借助别人的光彩,可是如果当很多人都看到你自己的光彩的时候,你就不用去了,三十年后,就在我们师徒当年居住的寺观里面,我希望你能准时赴约。现在,你该服药了。”

夏金逸看着蛊丸,低声道:“曾经沧海难为水,除却巫山不是云,除了仇恨,这世间我还有什么放不下的呢。”说罢他服下蛊丸,不知是心理作用还是真的,他只觉得药丸一沾唇就自动滚入腹中。

梦道人欣慰的看了他一眼,道:“希望我们师徒有缘再会。这房子里的人我已经点了他们的穴道,现在他们也快醒了,为师走了。”

黄影一闪,梦道人影踪不见,夏金逸俊秀的面容上露出笑容,那是一种令人见了反而觉得辛酸的微笑。

没过多久,神色有些不安的赤骥出现了,他昨夜被点了穴道,梦道人手法高明,他不仅毫无所觉,而且谁的很好,可是他是秘营出身,总觉得不该睡得这样沉,所以一起来就过来查看夏金逸的情形,进来一看,只觉得夏金逸肤色有些变化,但是见到夏金逸神情茫茫,似乎十分苦痛,所以也不好多问,只是试探着问道:“夏公子昨夜没有休息么?”

夏金逸淡淡一笑,道:“所爱身死,金逸无法安眠。”

赤骥了然的神色闪过,道:“夏公子还是节哀顺便,失去挚爱,虽然痛苦,可是绣春姑娘泉下有知,也会希望夏公子过得快乐一些。”

夏金逸微微一愣,道:“怎么,小兄弟你年纪轻轻,也知道失去爱人的痛苦么?”

赤骥微微一叹,道:“我家公子有一首词,从来不曾流传在外,若是夏公子有兴趣,我可以唱给你听。”

夏金逸感兴趣地道:“是什么词,我替你伴奏。”

赤骥眼中闪过忧伤,道:“是一首沁园春。”夏金逸取了洞箫,心神一凝,吹了起来,赤骥随着乐声,低唱道:

“瞬息浮生,薄命如斯,低徊怎忘。记碧波月冷,翠袖燕舞;雕阑曲处,银汉暗渡。情好难留,花残莫顾,赢得更深哭一场。病中久,纵相思百转,倩影谁描。

夜阑卧听苦雨。料短发朝来定有霜。唯碧落茫茫,尘缘断矣;蝶影翩翩,触绪还伤。欲思卿颜,不堪赤血,梦里几度昨日香。真无奈,倩声声叶笛,谱出回肠。”

夏金逸一边听着一边吹曲,可是到了后来,曲声开始断断续续,却是越发百转愁肠,一曲终了,夏金逸只觉得那原本似乎消失的心痛竟然再次出现,终于泪落如雨。

\chapter{第十四章 长安血夜}

代祭礼成,与祭者皆言太宗端谨。

六月十五夜,长安乱起,人言有谋逆事,太宗披甲持剑,威震京赍。

——《雍史·太宗本纪》

六月十四日,夜深人静,在长安一处隐秘的府邸,一间密室之中,一男一女正在秘密商议,那男子身穿黑色夜行衣,披着黑色的披风,面貌全部隐藏在纱笠之下,那女子相貌平平,却是满身剑气,正是闻紫烟本人。两人对着昏黄的灯光,沉默良久,那个男子终于开口道:“请禀告门主,这次我们不能动手,现在只论京中的力量我们和雍王不过是五五之数,而齐王的军马只有他或者他的兵符才能调用,这次不可能参与夺嫡,再说,太子殿下还是有机会的,我们若是急急动手,反而中了圈套。”

闻紫烟叹息道:“门主也是这么想的,可是我总觉得若是不趁机杀了几个眼中钉,真的不甘心。”

那个男子冷冷道:“我们可以去杀谁呢,雍王身边有邪影李顺,除非门主亲自出手,谁能一举得手,江哲身边有慈真大师,其他的人就是杀了又有什么用,凭白造成他们报复的借口,难道去杀在无尘庵清修的长乐公主么?”

闻紫烟微微一笑道:“公主我们自然是不敢杀的,不过叶天秀怎么样,他现在身在长安,我们不若趁机杀了他,斩断庆王羽翼。”

那个男子若有所思地道:“这个主意也不错,只是叶天秀毕竟是名正言顺的留在京城的,庆王侍卫总管的身份可不寻常,我们杀他也得暗中下手,要不就得借刀杀人。”

闻紫烟神色冷然地道:“杀一个叶天秀易如反掌,若非不想激怒庆王,我早就动手了,如今我们趁着局势混乱将他杀了,庆王就是想兴师问罪也找不到人。”

那个男子淡淡一笑道:“咱们还是不要动手了,就让夏侯去吧,他也是魔门月宗弟子,你别看他表面上似乎武功花样太多,但是可不是那么简单的人。”

闻紫烟笑道:“好,就按你说的办,师父常说你才是她的得力助手,果然名不虚传。”

那个男子淡淡道:“就是得力助手又如何,还不是只能听命于人。”

闻紫烟正色道:“你放心,事成之后,你定会满意门主的安排。”

那个男子默然,片刻才道:“我要走了,时间不早了。”

闻紫烟轻轻点头道:“路上小心。”

那个男子出了密室,身形轻捷如飞鸿,转眼就消失在夜色当中。而一场血腥的杀戮也即将展开。

六月十五日,雍王李贽代替太子在长安陪祭,当李贽恭谨而完美的完成了祭典之后,就是最挑剔的大儒也只能赞叹不已,而雍王也借着这一场祭典的形势重新回到了大雍朝廷的权力中心,这一点让很多人痛恨不已,也有人欢欣鼓舞。叶天秀就是其中一个,身为庆王的侍卫总管,他对庆王和凤仪门的仇恨一清二楚,而他也明白,庆王根本就没有任何可能取得胜利,唯一的办法就是借助强权,可是直到今日,叶天秀才心甘情愿的承认只有雍王才是配作帝王的人。

叶天秀依依不舍的看了雍王远去的车驾,终于决定回去住处,近日来,姜侯爷已经有信给庆王殿下,小侯爷的毒伤已经暂时得到控制,所以侯爷更希望能够尽快将小侯爷送到长安,可是现在长安局势如此盘根错节,姜侯爷的势力难以保证爱子的安全,所以转托庆王,可是庆王也有碍难之处,在长安,庆王的势力是不稳固的,虽然凤仪门不能明着对付庆王的人,但是不是因为无能为力,而是因为她们不愿给庆王口实,若是小侯爷到了长安,被她们发觉蛛丝马迹,就可以光明正大的将自己这些人一网打尽,到时候不仅庆王殿下的苦心经营化成泡影,姜小侯爷也会陷身长安。

回到庆王在长安的秘密据点,已经是天将黄昏,叶天秀吩咐属下们小心守夜,便到书房回信给庆王,说明自己的意见。写完之后还不到一更天,叶天秀心中烦闷,难以安眠,就在书房中翻阅起近期的情报来。

而就在这时,一个神秘人站在不远处的街巷里,漠然的看着这里,他身上穿着一件灰黑的长衫,腰间略略束紧,身材修长,有如玉树临风,虽然面上罩着青纱,看不见容貌如何,只是那露在面罩外面的眉眼已经是秀雅非常,他看看天色,突然飞身扑进那所安静的宅院,他飘飞的身影有如轻鸿飞燕,转眼就已经跃过院墙,他的身形高高向院中落下,就已经惊动了叶天秀的属下,他们一边发出暗号向上禀告,一边向那人围去,那人也不惊慌,只是信步向内走去,几个庆王侍卫按耐不住,向前阻拦,却只见一道青光如同电闪一般攸然出现,立时鲜血横流,那几个侍卫俱是被一剑刺穿了咽喉。

这时叶天秀已经匆匆赶来,他大声喝道:“你是什么人,竟敢夜闯民宅?”

那人轻轻一叹,道:“在下也是奉命而来,叶兄见。”说罢已是扑向众人,那些侍卫都是武功高强,擅长技击的高手,不约而同的出手抵挡,可是那人轻功十分卓绝,只见他身影飞腾,剑光更是有如流光一般,处处在灰黑的身影中闪耀,时而破空击出,时而横闪刺目,所到之处,剑出见血。叶天秀怒喝一声,拔剑扑上,那人却是不和他交手,只是四处追杀那些侍卫,叶天秀更是惊怒,一声长喝道:“你们速退。”

这些侍卫都是训练有素的人,立刻四面八方散去,而叶天秀也趁机挡住了那人,两人的剑法都是十分高明,叶天秀的剑法辛辣,凶狠,快捷,可是其中又透着沉稳,而那蒙面人的剑法却是轻灵快捷,而又变幻莫测,配合着他神幻莫测的轻功身法,更是难以抵挡,两人顷刻间就斗了七八十招,精妙的剑招精彩纷呈,剑气汹涌,两人都像是*中的小舟一样凶险万分。

那些庆王侍卫知道若是自己出手反而添乱,又不愿惊动官兵,因此只能围住场地,准备好暗器,心道都想,若是两人分开之时,就要向那蒙面人招呼。

两人斗到酣处,那个蒙面人突然一声轻喝,人剑齐飞向叶天秀飞去,这一剑奇快无比,叶天秀沉着非常,横剑拦阻,两剑相交,各自飘飞,叶天秀发觉那人身躯似乎一颤,不由心中一喜,知道那人功力比自己要弱一些,身形闪过一个弧形,向那人后心一剑刺去,他算准了那个方位那个蒙面人不及转身,而那个蒙面人果然真气不继,身形一滞,叶天秀这一剑就向他的右侧半身刺去,眼看就要得手,谁知那人反手一剑,剑光如同电闪雷鸣,总算叶天秀心思细密,留了一分力,也只来得及躲开要害,他一声痛呼,按住伤口,喝道:“各自突围。”说罢不顾伤痛,向外闯去。

那个蒙面人本想追赶,不知怎么突然住了脚步,转身扑向那些拼命向自己杀了,好为叶天秀阻截敌人的侍卫,他这次却是凌空飞斩,身影如同飞隼,剑光如同暴雨,不过十几招,就把留下来断后的几个侍卫杀的干干净净。最后,那个蒙面人看着满地血腥,轻轻一叹,从怀中掏出一块雪白的丝帕,将剑上血痕擦去,然后将那柄长剑插入伪装成腰带的剑鞘,那柄利剑,竟是一柄软剑。

这时,大宅中突然火光四起,蒙面人微微皱眉,立刻便知道是庆王的属下自己烧了宅子,免得留下什么证据,他也不恼怒,只是在惊动四方之前隐入到了黑夜之中。

可是,这一场血战却只是这一夜噩梦的开始,就在巡夜的禁军赶到火场,将火扑灭不久,长安城就出了两件大事,一件是郑瑕遇刺,另一件则是长安都会市事变。

郑瑕遇刺是在二更初,完成祭典的收尾工作之后,郑瑕夜行回府,他虽然是文官出身,可是大雍崇尚武勇,他也不喜欢坐轿,只是骑马缓缓而行,两边的随从左右相护,不时的用目四处瞧看,郑瑕一向以刚正耿直,直言敢谏闻名天下,因此上虽然廉洁清正,品性光明,仍然结下了不少仇家,因此身边颇有几个武功出色的护卫,有的是受过郑瑕的大恩,感恩图报,有的是敬重郑瑕的人品,所以倾心相投,还有一些干脆是雍帝派给他的侍卫。李援虽然有些事情不免糊涂,可是却非难纳谏言的昏君,对于郑瑕,他十分尊重,所以在一次郑瑕遇刺之后,李援就下旨派了四名御前侍卫做郑瑕的护卫,后来又赏给郑瑕的另外四名江湖人出身的护卫三等御前侍卫的虚衔,李援对郑瑕之荣宠冠于百官之上,郑瑕也因此对李援更加赤胆忠心。

就在郑侍中和守门的侍卫打过招呼,刚刚走出朱雀门不久,一个黑影匍匐在道路一边的屋顶上,此时,郑瑕的护卫因为这里禁军众多,所以稍微松懈了一下,谁知就在这个时刻,那个黑影突然急射而出,一剑刺向郑瑕。这一剑快如流星闪电,原本郑瑕是绝对没有生机的,但是说来也是侥幸,这个黑影凌空刺杀的时候,恰好郑瑕想起,迎接圣驾还京的仪式虽然已经安排好,可是按照礼仪应该去向雍王请示一下,毕竟皇上指派雍王代祭,那么就等于让雍王坐纛一样,虽然这段时间雍王等于是被软禁在斋宫,可是礼节上却不能轻乎,郑瑕原本就是最重视这些礼数的,所以他从马上俯下身子低声吩咐一个侍卫,让他今夜先去送一封帖子到雍王府,说明今夜不能去拜见的原因。就在他俯身的一刹那,那个刺客已经飞身刺来,两相凑巧,郑瑕只觉得一阵剧痛,那一剑已经刺穿了他的肩背。

而就在刺客飞身而出的时候,明亮的月色已经将他的身影显露无疑,那些侍卫虽然没有能够阻拦这一剑,可是亡羊补牢却做的不错,郑瑕俯身跟他说话的那个侍卫,一把将郑瑕扯下马来,而另外几个侍卫也都拔出刀剑,向那个刺客围去,可是那个刺客不同寻常,颇得一击不中,飘然远引的真谛,在这些侍卫合围之前,已经冲出重围,消失的无影无踪。

郑瑕忍着剧痛道:“立刻派人去通知雍王殿下、韦相和禁军统领秦青。”说罢就已经昏迷过去。这些护卫连忙将郑瑕送到不远处的太医院救治,而郑瑕遇刺的消息也立刻就传到了长安各大势力的耳中。

就在各方势力心中猜疑的时候,六月十五日,令长安天翻地覆的大事件发生了。

长安最繁华的两处集市,分别是都会市(东市)和利人市(西市),而毗邻东市的平康坊更是不夜之地,按照惯例,两市的宵禁比别处要晚两个时辰,而平康坊更是不夜禁的好所在,所以三更时分这里正是灯火通明、春意盎然的不夜天,就在子夜时分,火光四起,东市之内各处商家群起救火,可是混乱之中,却有人一边呼喝着“蜀人誓死不降大雍”一边杀人劫货,东市没有坊门,所以市中民众纷纷外逃,一时之间,一片混乱,同时,离东市最近的春明门也开始起火,有人在城内外呼喊要杀的长安血流成河。大雍立国以来,长安一直是歌舞升平,一时之间东市的官员措手不及,只能无可奈何的派人去向秦青禀报。

若非秦青已经因为郑瑕遇刺的事件而惊动,只怕还要拖延,但他带了秦勇早已经出了门,一看到东市方向火起,秦青和秦勇都是究竟战场的将领,立刻传令所有禁军全部出动,秦青派出禁军各自保护长安重要的衙门和府邸,然后下令紧闭城门,秦青亲自带着一支禁军将东市团团包围,这一切只花了大半个时辰,秦勇则负责大街小巷的盘查,禁军四处高声传达军令,宣布长安进入戒严状态,所有居民必须待在家中不许出门,如有违反军令者杀无赦,这样的手段果然有效,等到秦青和秦勇在东市会合的时候,整个长安只有这里还没有平静下来,只因东市之内外来的商贾武士最多,里面火势虽然已经平息,可是却互相残杀起来,秦青和秦勇虽然也想派禁军进去镇压,可是这里乃是长安繁华之地,若是禁军镇压不免玉石皆焚,两人一时之间也拿不定主意,如今长安可以作主的人只剩下雍王和丞相韦观,韦观乃是文官,两人只得派人向雍王请示。

火起之前,雍王李贽正在和我商议这些天发生的事情,李贽神情愉快地道:“随云,如今本王可以说已经得到了大雍的军心和民心,你认为如何?”

我恭谨地道:“殿下这次长安陪祭,令天下得见殿下风采,虽然皇上仍然有心庇护太子,可是如今谁不知道太子失德,故而臣恳请殿下,这次不要急急逼迫,反而殿下还要顺着凤仪门主的意思上本保奏,若是殿下真的攻讦太子,只怕天下人都以为殿下不顾念兄弟之情,而且皇上急急灭口,显然是只想给太子一个教训就罢了,若是殿下逼得太紧,害得皇上无法下台,只怕还会迁怒殿下。”

李贽皱眉道:“你说得是,只是你也知道,如今凤仪门主已经亲自出马,只怕从今之后太子就不会有什么失误,拖上几年,只怕本王就没有机会了。”

我笑道:“殿下放心,如今凤仪门已经是孤注一掷,她们势力再大,也抵不过天下的民心,太子殿下也不是任凭摆布的木偶,他的本性难改,什么事都可能作出来的,当然我们也不能就这样等着,臣的计划已经有了,只是齐王太碍事了,齐王虽然性情粗暴,心计也浅些,可是有些事情别人还没有发觉,齐王就已经凭着天赋机敏而察觉,所以殿下当务之急就是把齐王殿下遣离长安。”

李贽想了一想道:“这倒不难,进来北汉有些异动,我正可以推荐齐王到边关巡视。”

我连忙道:“殿下不妨自请巡视边关。”

李贽一愣,然后便是恍然大悟,道:“你是说欲擒故纵?”

我拊掌道:“正是如此,殿下若是回到军中,便如蛟龙入海,那些人怎肯放殿下前去,到时候有这个资格的除了齐王没有别人,齐王一走,殿下就可以安心和太子一战,等到事成之后,只要一纸军令,还怕齐王不乖乖的自缚还京么?”

李贽点头道:“好,我等到父皇回来,就说明此事,等到六弟一走,我就可以放心了,现在太子方面的领军大将只有六弟,若是他走了,我就可以安枕无忧。”

我摇头道:“那也未必,靖江驸马也掌握君权,精通兵法。”

李贽含笑道:“随云,你别可告诉我你没有在秦家做什么手脚?”

我微微一笑,想起骅骝,秘营精英,我曾经的亲卫,如今不正是在秦勇的身边么?

就在我和雍王谈笑的时候,有侍卫回报,说是郑瑕遇刺,雍王和我正在忧心忡忡,没多久,府中的侍卫又来禀报看到了火光,这是今夜的第二处火光,位置似乎是东市,我和雍王面面相觑,我飞快的动着脑筋,怎么会有这么多的事情碰在一起发生呢,若说是巧合,那也太过分了吧。

\chapter{第十五章 王者神威}

时,东市之内鱼龙混杂,秦将军青告于太宗,太宗奋起,携宿卫百人,亲临东市,于市门高呼道:“奸细作乱,凡我子民,静立莫动。”当是时也,太宗金甲锦衣,见者拜服,乱乃定。

——《雍史·太宗本纪》

雍王派人出去打探,没有多久就有回报,李贽听了之后倒是松了口气道:“早年我在兵部的时候,曾经考虑到如果发生变乱该如何处理,因此曾经给禁军训练过该如何处理这样的事情,现在看来,秦青果然还是将门虎子,处理的十分妥当,如今不过是一处城门着火,变乱也集中在东市附近,只要处理得当,倒也不会酿成大乱。”

我一边在心里庆幸表弟荆舜卿的江南春在利人市,一边担忧接下来必然会有的大搜查,要知道夏金逸还在长安呢。听了雍王的话,不由赞叹道:“殿下深谋远虑,精通军务,臣万分钦服,只是这东市发生暴乱一事十分蹊跷,臣实在有些不明白。”

李贽深深的看了我一眼道:“随云你毕竟少经军旅,以本王看来,是我大雍疏忽了,这几年,争储之事越演越烈,浑忘了天下还未平定。”

我恍然大悟,拊掌道:“定是北汉的密谍,南楚柔弱,而且现在百废俱兴,那些人虽然自称蜀人,可是蜀人在庆王治理下颇为安定,锦绣盟又已覆灭,所以只有北汉才有可能,殿下方才说边关有警,只怕正是因为北汉有心犯境,这次先派人挑起长安动乱,这也是一举两得,既可以跳起民怨,抵消皇上告祭黄帝陵的影响,又可以让大雍各方势力彼此猜疑,方才臣还在怀疑郑瑕遇刺是否是因为太子迁怒,若不是东宫失火,郑瑕禀告皇上太子不在东宫,只怕太子也不会被软禁,如今看来可能也是北汉所为。”

雍王摇头道:“北汉民风彪悍,若是派人劫杀还有可能,若是刺杀大将也有可能,可是刺杀一个清正廉洁的文官,这样的事情他们作不出来。”

我摆弄着手中的折扇,皱眉道:“今夜发生了三件大事,庆王在长安的秘密据点被人捣毁,郑侍中朱雀门前遇刺,如今又是东市变乱,东市变乱很有可能是北汉密谍所为,唉,我也是疏忽了他们,没想到他们敢如此嚣张,如今看来正是他们举兵进犯的前兆,庆王,庆王,这倒有可能,长安之中若说谁和庆王有仇,只怕是凤仪门嫌疑最大,不过这件事情也罢了,就是猜错了也不是什么大事,只是是谁刺杀郑侍中呢?说句实话,郑侍中乃是皇上忠臣,素受陛下信赖,如今他亲自参与此次东宫之事,他素来刚正不阿,对太子只怕已经是心生不满,有这样一个人在皇上身边,对殿下只有好处,莫非,莫非……”我不再说话,接下来的猜测太骇人听闻了,就是我也不敢多想。

雍王也心中一动,可却没有说什么,只是道:“随云,当日凤仪门主用柔蓝相试,我们断然拒绝,只怕从今之后我们日夜都要小心凤仪门的刺客了。”

我冷冷道:“殿下不想为人掣肘,臣也素来不喜受人限制,凤仪门早和殿下水火不容,如今从少林派的反应看来,凤仪门众叛亲离之日已经不远,若是殿下和凤仪门媾和,反而失去了难得的人心和机会。”

雍王傲然一笑道:“本王虽然知道凤仪门可以让我轻而易举登上皇位,可是世间之事往往是不能贪图捷径的,本王立志一统天下,靖肃宇内,焉能受人胁持,凤仪门主虽然用心良苦,可惜本王不是受教之人。”

我施礼道:“殿下志向远大,臣敬服,希望臣能够看到天下太平的一天。”

雍王肃然道:“随云你对本王襄助良多,日后本王还要和你共商国事,你定然可以亲眼看到四海升平之日。”

我微微一笑,虽然得到了少林寺的心法,这几日练来,果然有点进步,可是若是这样劳心劳力,不知道我还能活上几年。

看看被火光映红的天空,我有些不安地道:“殿下,公主殿下没有随陛下去桥山,今日可是在无尘庵么?”

雍王看了我一眼,见我神情有些恍惚,轻轻摇头道:“你放心,无尘庵那里定有人去保护的,皇妹身份贵重,乃是父皇爱女,又是深受大雍百姓敬爱,所以不会有人敢松懈的,随云可是不放心么。”

我面上一红,道:“不论是否有人去保护公主殿下,殿下您也应该派人去看看的。”

雍王淡淡一笑,道:“这个应该不用我操心了,想必王妃已经派人去了。”

这时,一个侍女进来禀道:“王妃命奴婢禀报殿下,派去探望公主的侍卫回报,夏侯总管已经带人护住了无尘庵,现在情况混乱,王妃还给公主殿下送了一封信,劝公主明日回宫,公主已经答应了,还说让娘娘带着柔蓝小姐进宫去看她。”

雍王挥手让侍女退下,我这才放心下来,问道:“那么现在应该是谁护卫雍王府,殿下可留意了么?”

雍王失笑道:“若是本王要等你提醒,只怕早就迟了,现在在外面的正是裴云,你放心吧,绝没有人能趁机加害本王,再说,你不是早就让王府宿卫小心戒备了么?”

我赧然一笑,刚才私下里让司马雄出去传令戒备,想不到也没有瞒过雍王的眼睛。

正在我们继续研究今夜事变的时候,秦青的使者已经进了雍王府的大门了。

听了使者的禀报,雍王面色一沉,道:“这些密谍也太可恨了,东市乃是长安重地,这次可是损失惨重,如今恐怕是他们从中浑水摸鱼,东市的商家,哪个没有保镖护卫,这样发展下去,只怕东市就成了废墟了,这可不行,本王得立刻前往处置。”

我连忙拦阻道:“殿下,如今东市已经是一片混乱,殿下若是前去,平息了争端还好,若是无用,只怕会有人把这件事情的责任算到殿下身上,如今一动不如一静,还请殿下三思。”

可是这次一向对我言听计从的雍王却摇头道:“随云,本王乃是大雍亲王,三军统帅,这等时候,正是我为朝廷和百姓尽力的时候,怎能斤斤计较个人得失,东市之乱早一刻平息,损失就要少一些,长孙冀、司马雄,你们点上一百亲卫,随我前往东市,府中诸事,随云你要小心,慈真大师和小顺子至少要有一个在你身边才好。”

我还想劝阻,可是举目望去,李贽神采奕奕,气势迫人,竟然说不出话来,只得低头道:“臣遵命,请殿下放心,小顺子我会派他到后宅保护王妃和几位郡主,有慈真大师和外面的裴将军在,殿下不用担心府中的事情。”

李贽淡淡一笑,喝道:“取本王的金甲来,本王倒要看看,什么人敢搅乱我大雍的皇都。”

门外的侍卫齐声高喝,不多时已经有雍王的亲卫拿来了金甲,雍王也不避人,脱下便衣外袍,穿上金甲,外面披上蜀锦战袍,举步向外走去,龙行虎步,矫健非常,那些侍卫都是跟着雍王千军万马中杀出来的,见到雍王这般气势,就似从前开战之前一般,不约而同的下拜道:“雍王殿下千岁,千千岁。”

我分明的感觉到那一种沙场血战的强凝气氛,不由被那冲天而起的杀气豪情所动,也高声道:“预祝殿下马到成功,臣在府中设宴相候,待殿下归来庆功。”

雍王大笑道:“众将士,可听到司马大人要为我等设宴庆功呢,我们还不快去快回,也好畅饮通宵。”

那些侍卫都已经结束停当,大开了中门,簇拥着雍王上了战马,霎时马如龙,人如虎,冲出府门,顷刻不见,只留下御道之上尘烟四起和渐渐低微的马蹄声。

我目送着雍王的背影消失,心中思绪万千,虽然雍王没有接受我的意见,可是我却没有丝毫恼怒,这样的人,才配作万乘之君,才配作我江哲的主君。

这时,拱卫雍王府这一带的禁军统领裴云策马过来,对我说道:“大人,雍王殿下不愧是一代名将,只见殿下的近卫骑兵,就知道殿下治军严谨,将士用命,可惜裴云没有机会在殿下麾下作战。”

我淡淡一笑,道:“总会有机会的,近日来北汉有些异动,边关有些风险,殿下准备向皇上请旨巡视边关,你若是愿意可以向殿下请求随军。”

裴云眼睛一亮,思索起可行性来,不过这个消息,给裴云身后的禁军听了,却是各有所思。

我心中暗笑,用这个方式传出消息,不怕太子一方不连忙筹划如何阻止雍王回到军营。突然之间,我想起一件事情,这次北汉策动长安事变,虽然声势浩大,可是实际作用并没有想象中的大,除非,他们另有打算,若是我策划这件事情,应该如何盘算呢,心中千回百转,突然想起一件事情,心中大惊,连忙道:“裴将军,我需借助你一臂之力。”

裴云一惊,道:“请大人吩咐。”

我招手道:“裴将军,你跟我来一趟。”说完也不顾他是否跟来,便急匆匆的赶回寒园,心里盘算,时间应该会来得及,不由庆幸我想到了那件事情,就是我杞人忧天,也好过后悔莫及。

李贽来到东市的东门,如今秦青正在那里指挥禁军,秦青已经是等得十分心焦,一看到雍王来到,策马上前高声道:“殿下,如今里面已经是一片混乱,末将几次下令若是他们不肯停手,就要强行镇压,可是他们都不肯听从,请问殿下,是否准许末将动武。”

李贽冷冷道:“东市乃是长安菁华所在,几乎大雍的所有大商家都在东市设有店铺,若是玉石俱焚,只怕有伤大雍的经济命脉,还是本王来处理吧,秦青,你将禁军指挥之权暂时交给我如何?”

秦青抹了一把头上的冷汗,道:“末将遵命。”说罢迅速传下将令,禁军都是大喜,他们对雍王的声威早有所闻,很多人还曾经见过雍王上阵杀敌的英姿,在将领的带领下,万余禁军同声高呼道:“谨尊雍王殿下将令,殿下千岁千千岁。”

东市之内正在混战的人们听到禁军们的高呼,很多人都不由放慢了手脚,这时临近东门的人群中发出惊呼,只见一个身披金甲,外罩红色蜀锦战袍的雍容男子神色温和,高坐战马之上,出现在禁军之前,身旁两员战将,一个黑衣黑甲,笔直口方,相貌端正,一个长眉凤目,面白无须,身穿青色战甲,那黑衣将军手中乃是精钢打造的马槊,只看上一眼也知道重量不低于二十斤,腰间则佩着横刀,一见便知是一员勇将,而青衣将领手中乃是丈二银枪,背着一把金弓,马上挂着四个箭囊。两员大将和左右虎赍皆是杀气隐隐,气度沉凝,更显得金甲将军气度从容冷静。

这东市之人大都是走南闯北之人,对大雍的名将豪门如数家珍,一见之下,便知道是什么人到了。手中的刀剑更是用不上力气,心中惴惴不安,唯恐雍王殿下下令镇压。

雍王用目一瞧,已看出这些人气势已弱,便高声道:“现在奸细作乱,挑拨离间,尔等皆是我大雍子民,焉能助纣为虐,若是心无反意,便需坐倒在地,双手抱膝,司马雄,你给本王数上十声,十声之后,若还有站立者给我全部射杀,本王当年纵横天下,攻城略地,焉能被这小小东市所困阻。”

李贽说话之时用了内力,这些人都听得清清楚楚。

这时司马雄已经高声道:“众军随我高呼,雍王殿下有令,不是奸细,坐倒在地,双手抱膝,十声之后,站立者定杀不赦。”

不过片刻,军令已然传下,只听见雷鸣一般的喊声,将雍王军令高声重复三遍,东市之内人人听得清楚。这时司马雄将手中马槊指向高空,高声道:“一。”众军也同声附和,声音惊动天地。司马雄以马槊指天为记,到后来,那些禁军只要看见司马雄的动作,便同声高数。

十声还未数罢,那些在东门口拥挤的人群已经纷纷坐倒,这时有人尖声高呼道:“他们都是骗人的,我们混战不过为了自保,可是他们为了掩盖此事,必然要将我们当成叛逆。”

他的声音一响,人群中已经有人惊惶失措,眼看局势就要难以控制。雍王冷冷一笑,长声道:“长孙,给我杀了那些造谣生事的奸细。”

长孙冀早在雍王出声之前,就已经准备好了弓箭,如今听到雍王令旨,抬手一箭,箭影仿佛流光一般,射入人群,将一个汉子钉在地上,这一手立刻震慑了全场,那些人开始用惊惶的目光看向全副武装的军队。

李贽高声道:“此人胡言乱语,意图煽动,本王若是将你们当成叛逆,早已下令围剿,如今本王体念你们都是受人蒙蔽,只要服从军令,本王绝不追究。”

说罢,李贽策马前行,司马雄、长孙冀和百名近卫虎视眈眈的簇拥着雍王,一行人所到之处,李贽不断高声宣布赦令,大雍百姓对雍王都是崇敬非常,都很听话的坐倒在地,并且将大路让开,李贽沿着东市的大道缓缓前行,长孙冀手中弓箭紧握,若是有人出言挑拨便是一箭,他箭术绝伦,目光敏锐,竟然没有错杀一人。

李贽面上带着淡淡的微笑,但是他的目光却是冰寒中透着威严,他只是用目环视众人,那些还在满怀犹疑的人也不由自主地松开兵器,在雍王等人身后,被阻挡在外面的禁军井然有序的进入东市,将那些已经坐倒在地的各种身份的武士兵器收缴,然后监视他们回到自己的住处,不许外出。东市很大,李贽沿着市内的纵横的四条主道一一巡视,所到之处,就是有人想要趁机作乱,可是奇异的是,雍王明明手无寸铁,可是他的目光只要一扫过来,就人人心惊胆战,仿佛都忘记了他身边的护驾将军的厉害。一路行来,虽有几处有人悍然不服,可是长孙冀和神箭和近卫们的横刀,让他们很快就被当场斩杀,而雍王的凛凛神威,也让他们意图挑起事端的行动化成泡影。

直到天将黎明,东市终于被禁军全部控制,幸好很多地位举足轻重的商贾都闭门不出,只让手下守住商铺,这才没有造成不可挽回的损失。李贽终于松了口气,他不是不可以下狠心镇压东市变乱,可是想到后果就不敢动手了,如今总算局势已经控制住,接下来只要好好盘查这些人,定然可以查出北汉的密谍来。

李贽对秦青道:“秦将军,如今局势已经控制住了,本王将军权交还,剩下的事情你好好处理吧,若有不能决断之处,可以到王府见我,还有,去向韦相禀报一声,本王这就要回去更衣,如今大局已定,本王还要进宫向母后和诸位娘娘通报一声。”

秦青万分佩服地道:“今日得见殿下威严,末将拜服,请殿下放心,末将一定会将事情处理妥当。”

李贽微微一笑,就要告辞离去,这时候,一队禁军押着几个绳索捆绑的汉子走了过来,李贽住马,看了一眼,问道:“这些人都要好好看押,一定要仔细审问。”

秦青正要答话,那些大汉突然嘶声道:“李贽,纳命来。”说着同时振臂,绳索寸寸断裂,那几个剽悍的汉子和那一小队禁军同时向李贽扑来。

\chapter{第十六章 错综复杂}

乱初平,有苏定峦者,凌空刺杀,幸宗师慈真禅师隐在侧,太宗无恙,苏定峦,北汉三品将军,性暴烈,斩将夺旗,攻无不克,常为大军先行,号“先锋将军”是也。

——《雍史·太宗本纪》

这时只听弓弦响起,声如珠落玉盘,长孙冀施展开连珠神射,几个冲在前面的没有衣甲的大汉首当其冲,被利箭射穿血肉之躯,却原来长孙冀心细如发,他发觉那些禁军的步伐有些混乱,这是不应该发生在训练有素的禁军中的现象,故而及时发箭阻挡那些刺客。而这一耽搁,李贽的近卫已经将那些刺客挡住。

就在李贽微笑着看着已经占了优势的近卫的时候,突然路旁一座商铺突然有人破门而出,身如闪电,势若雷霆,手中步槊向李贽刺去。

这时司马雄正在前面督战,不及赶回,长孙冀张弓搭箭,连射三箭阻拦,不料那人手中短剑挥动,长孙冀那可以断金裂石的长箭竟然被硬生生反弹而回,长孙冀大惊之下来不及闪身,只得用弓身拨打箭支。那反弹而回的箭支居然中蓄强力,长孙冀连人带马向后退了三步,金弓弓弦更是已经断裂。一时之间,长孙冀竟然无力救护雍王。

这时雍王身边只有四个近卫,他们同时以身躯挡住那人的来势,可是那人的身躯居然诡秘的绕了一个弧形,向雍王刺去,李贽虽然也是沙场骁将,可是那人锋芒所指,竟然让李贽觉得无力闪避,心中一叹,难道我壮志未酬就要死在此处么,不由闭上了双眼。

就在千钧一发之时,一声宛如天籁的佛号传来。

“阿弥托佛。”声如九天惊雷,然后李贽便觉得身上一松,那逼人的剑气已经消失无踪,连忙睁开眼睛一看,只见自己的马前,慈真大师双手合十,正在念诵佛号,而两丈之外,一个身高九尺的大汉满面怒火的看着慈真大师,手中拿着一柄精钢打造步槊,李贽一眼看去,就是抽了一口冷气,这柄步槊竟然是紫黑色的,李贽久经沙场,知道只有人血才能将兵器染成这个颜色,如此身材,如此武功,如此杀性,李贽立刻就知道了这人的身份。他朗声道:“原来是北汉先锋将军苏定峦驾到,不知道本王何幸,竟然让将军亲来行刺。”

雍王的亲卫还好,那些禁军有很多都曾经和北汉做过战,对这位先锋将军早闻其名,却是没有见过,不由都用好奇和凶狠的目光望去。

北汉军素以勇猛凶悍闻名天下,或者在训练精良上不如大雍军队,但是若论个人战力却在大雍展示之上,凡是大雍军士对北汉出名的将领战士都是耳熟能详。北汉军方领袖乃是威远将军龙庭飞,此人出身名门,精通军略,虽然只有三十岁,但是屡次将大雍军队击败,唯一能在他面前败而不溃的至今只有雍王李贽一人,就是齐王李显也曾经惨败在他手上。若非大雍兵多将广,只怕不仅不能出关攻击北汉,还会被龙庭飞给攻破关隘呢。除了龙庭飞之外,北汉还有四位将军名震天下。

飞虎将军石英擅长长途奔袭,一举克敌,磐石将军段无敌擅长守城,铜墙铁壁,鬼面将军谭忌,擅长行军布阵,而先锋将军苏定峦则擅长阵前斩将,他乃是魔宗宗主宗无极的二弟子,武功虽然没有能够登峰造极,却是难得的沙场骁将,想不到此人竟然出现在长安行刺雍王,真是令人难以置信。

他们这里惊疑,却不知苏定峦也是心中叫苦,刺杀雍王是非小可,就是事成,只怕也只能是玉石俱焚,这种事情宗无极怎会让他这个阵前斩将夺旗的猛将来做,他原本是因为这两年边关无事,闲的无聊,特意扮成商人到大雍游玩,顺便也想探探军情,在长安已经流连了一个多月。

谁知道北汉秘谍系统竟然在此时下手跳起东市事变,意图扰乱大雍皇都,为半月之后的大举进犯作准备,而苏定峦也接到宗无极的命令,让他相机行事,刺杀雍军统帅李贽,苏定峦在长安已有多日,很清楚若是今次事变,雍王李贽定然要到东市镇压,果然被他等到了雍王,凭着他一身绝世武功,原有七成胜算,他只想一举杀了李贽,然后趁着局势混乱之际逃走,北汉秘谍早已为他准备了撤退的后路,不料事与愿违,竟被慈真大师阻拦,苏定峦越想越是恼怒,也顾不得慈真大师具有与宗主同等地位的宗师身份,手中步槊指向慈真,怒喝道:“你这秃驴,不在寺里修行,屡次坏我魔宗大事,真是可恨可恶。”

他虽然骂得难听,慈真大师却不恼怒,只是淡淡道:“老衲乃是大雍子民,雍王殿下军功卓著,乃是大雍军神,更是朝中擎天之柱,焉能坐视你等刺杀殿下,若是苏施主放下屠刀,老衲愿为施主求情,请殿下饶了你的性命。”

苏定峦四下瞧看,只见雍王亲卫和禁军已经将这里围得水泄不通,眼前又有一个宗师级别的高手,心知这次绝难逃生,但是他心志坚强,冷冷道:“好,就让你们看看老子的厉害。”

说罢步槊闪动,直向慈真大师扑去,慈真大师神情不变,眼中却闪过一丝赞许,左手一晃,右手握拳猛击出去,却正是少林拳法中最基本的一着“冲天炮”。但是慈真大师使来却是威猛绝伦,让人一见便觉不可抵挡。

苏定峦心中一紧,但他心性凶悍,毫无畏惧的一槊刺出,拳槊相交,慈真大师丝毫未动,苏定峦却是被迫退了一步,但他眼中凶光一闪,步槊矫如游龙,再次扑上。

两人过招不到数合,只见慈真大师一掌击中苏定峦胸膛,苏定峦被击飞数丈,只见他嘴角溢血,步槊脱手,而胸口更是凹陷下去,眼看着就要活不成了。慈真大师一抖袍袖,长宣佛号,退到雍王马后,不再作声。

一个雍王的侍卫小心翼翼的走上前去,用钢刀去碰了碰苏定峦的身躯,见他纹丝不动,便俯身下去探他的鼻息。谁知苏定峦却在此时眼睛一睁,劈手夺过钢刀,用力斩去,那个侍卫临危不乱,一个铁板桥向后仰身倒去,钢刀险险的划过他的身躯,苏定峦横刀下劈,那个侍卫已经翻滚闪开,而就在同时,慈真大师在远处一指轻弹,一声脆响,那百炼钢刀竟被从中击断。

那个侍卫跳起身来,心有余悸的退到一边,这时,长孙冀拿着刚刚讨过来的一张强弓,张弓搭箭,指向苏定峦,喝道:“苏将军,你若再擅动,休怪长孙冀箭下无情。”

苏定峦眼中闪过萧瑟的神色,大笑道:“苏某何许人也,北汉先锋将军,这些年来,你们大雍死在本将军手上的将军和勇士不计其数,今日苏某行刺失败,却断然没有束手就擒的道理。慈真大师,你和家师也是同等身份之人,总不会为难晚辈,定要苏某被俘吧?”

说罢,苏定峦看向慈真大师,他心知就是他想要自杀,若是慈真大师出手阻拦,自己可当真是求死不得。慈真大师微微一叹,道:“老衲是为了大雍社稷,援手雍王殿下,苏施主若非在老衲面前伤害人命,老衲也不愿多管红尘俗事。”

苏定峦见慈真大师已经表示不会为难自己,更是得意的笑道:“李贽,你今日幸逃大难,若非慈真大师在此,你早就死掉了,可惜我事先不知道慈真大师到了长安,否则老子倒是愿意在沙场上多杀你们几个大将。”

苏定峦的话虽然凶狠,可是大雍军士最是敬佩勇士,见他虽然奄奄一息,却仍然如此豪气冲云,却也都目露欣赏之色,虽然如今就是让他们亲手杀了苏定峦,他们也不会有丝毫心软,可是却也绝对不愿折辱于他。所以都看向雍王,担心他发怒。

雍王却是长笑一声道:“苏将军失手却是本王侥幸,将军放心,本王答应你,不仅不迫你投降,还会将你的尸体送回北汉,让你的国主将你当作英雄好好安葬。”

他说话之时尽显英雄本色,神色更是顾盼雄飞,令众人皆是心中折服。

苏定峦惨然一笑,摇摇晃晃的站起身来,一张口,却是鲜血泉涌,他也不在意,只是行走几步,俯身去拿步槊,人人都当他要自尽,谁知他的身躯还没有站起,竟然用力一甩,那步槊快如流星,向李贽射去,众人不由惊呼,李贽却是似乎早有所料,在马上一侧身,避开了步槊。众亲卫勃然大怒,一个个刀出鞘,箭上弦,只待雍王令旨,就要将苏定峦千刀万剐。

苏定峦却是毫不畏惧,直起身躯,坦然道:“苏某大好男儿,只能死在勇士刀下,怎可自尽身亡,若是殿下肯亲手杀了苏某,才是苏某荣幸,定峦将步槊送给殿下,为什么你却避开呢?”

雍王微微一愣,笑道:“魔宗弟子,果然是厉害,本王也很喜欢你的脾气,可是你行刺本王事小,杀害我大雍无辜百姓事大,苏将军手上染满了我大雍子民的鲜血,请恕本王不能容情,众将士,谁为苏将军送行。”

司马雄提马上前道:“殿下,此獠意图刺杀殿下,罪大恶极,末将保护殿下不周,失职之罪难逃,请准许末将杀之。”

雍王微微颔首,司马雄策马上前,居高临下看向苏定峦,苏定峦抬头望去,目中竟无一丝恐惧。司马雄也是心中佩服,就在苏定峦抬头的瞬间,司马雄横刀斩下,众人只觉的眼前流光一闪,苏定峦已是头颅落地,鲜血四射,人头飞起,口中仍然呼道:“好快意!”

司马雄却是神色不变,自行回马缴令。李贽高声道:“此人虽然凶残成性,却是豪气干云,本王已经许他身还故里,你等可有异议。”

众军齐声道:“谨尊殿下令旨。”

雍王见事情已经平息,这才带着亲卫和慈真大师回转王府。

一路上,雍王奇怪的问道:“大师,您不是在寒园潜修么,怎会前来相救本王?”

慈真的骑术只是平平,虽然凭着他的身手,不会有什么危险,可是还是要小心翼翼的驾驭着马匹,他答道:“殿下,老衲是受了江先生所托,方才江檀越匆匆前来,说殿下到东市处理事变,他说想来想去,若是只想凭着扰乱长安来打击大雍,未免有些问题,所以担心有人是想把殿下诱出去,加以刺杀,所以老衲也赶到东市,暗中保护殿下,想不到江先生真是神机妙算,居然一语中的,也是殿下仁德感天,才有这样的奇士襄助。”

李贽也是惊叹不已,转念一想道:“这样一来,随云身边岂不是无人保护,若是有人趁机刺杀可怎么办呢?”

慈真大师笑道:“殿下放心,裴云正在江先生身边,而且还有五十亲卫,就是老衲亲自出手,一时半刻也难以刺杀成功,邪影李顺就在府中,若是发生意外,也来得及赶来,殿下勿忧。”

李贽这才松了一口气,但是眉心却有些紧锁,从前他没有和太子势成水火之前,凤仪门也推荐过护卫给他,不过他不喜欢女子在军中,所以留用的都是男子,但是王妃和内眷的安全还是有凤仪门保护的,今日一看,一旦发生事变,王妃身边没有得力的保镖就是有些碍难。

这时,慈真大师突然道:“殿下,老衲俗家有一对远房侄孙女,今年只有十九岁,拜在峨嵋门下学剑,今年已经艺成下山,两个丫头虽然剑术和品性都不错,可是却淘气的很,老衲闻之王妃贤德无双,若是能够得到娘娘言传身教数年,真是这两个孩子的福气。”

李贽心中一喜,连忙道:“多谢大师,李贽谢过。”

慈真大师微笑道:“殿下言重,这是老衲求殿下相帮,怎敢受殿下谢礼。”

李贽有客气了几句,两人心照不宣,谁也没有说穿这两个女子乃是为保护雍王家眷而来,而且这两个少女出身峨嵋,也是峨嵋向雍王示好之意。

回到寒园,看到江哲安然无恙,李贽终于松了一口气,送走了慈真大师和裴云,李贽这才对江哲说道:“幸好你请慈真大师相救,否则本王恐怕真要丧命了。”

我赧然道:“也是臣思虑不周,所幸亡羊补牢,犹未晚也。”

李贽苦笑道:“其实这次也不错,虽然这次本王险些遇害,可是杀了北汉的‘先锋将军’也是足可以补偿了。”

我叹气道:“虽然话是这么说,可是这件事情闹得如此之大,庆王定会因为属下被杀戮而恼怒,若是派人来追查凶手,只怕这混乱的局势会更加混乱,郑侍中遇刺,东市事变,虽然殿下镇压变乱有功,可是只怕会有人趁机说是殿下取代太子陪祭,上天才会降下灾难,而且这件事情也会掩盖太子秽乱后宫,对天地神灵不敬的罪行。”

李贽听得心中一寒,道:“难道这样颠倒黑白的事情也会有人相信么?”

我看了雍王一眼,道:“不是会不会让人相信,而是有人愿意相信,陛下恐怕会给太子一次机会,殿下威震皇都,可是陛下听了不免觉得殿下声威太高,为了压制殿下,也会原谅太子一次。”

李贽苦笑道:“想不到本王苦心为了社稷,却因此遭到猜忌,唉,可是今日之事,本王焉能袖手旁观?”

我微微一笑,施礼道:“殿下,这次您是作对了,皇上对您猜忌,可是天下人谁不敬仰殿下的德行,此事传扬出去,对殿下只有好处,何况皇上若是借机饶了太子,也会对太子已经是失去信心,太子更会因此事而心中惴惴不安,这样父子君臣之间相疑甚深,太子失去皇上恩宠和储位只在朝夕之间,只要遣走齐王,殿下就可以放手而为了,如今殿下已经是万事俱备,只欠东风,还请殿下传令给石大人,让他准备回朝之事。”

李贽面上露出喜色,转瞬消失,道:“写信可以,不过本王还是想看看父皇这次会如何处置此事。唉。希望父皇秉公而断,否则我这个做儿臣的也未免太寒心了。”

我没有答话,雍王恐怕是注定要失望的。看看已经明亮的天色,我有些疲倦了,就请雍王也回去休息。回到房间,小顺子已经回来了,满面的不悦之色,我问道:“怎么了,这样难看的脸色。”

小顺子抱怨道:“公子,你让我去保护王妃也就算了,可是怎能你让慈真大师去救殿下,怎么不告诉我一声。”

我苦笑道:“我总不能把你叫回来吧,不用担心,慈真大师已经和雍王有了安排,下次你就不用离开我身边了。不过今天你得去办一件事情,这几天长安风声一定不好,你先让夏金逸出城躲躲,免得被人发现,毕竟他在长安也不是个无名无姓的人。”

小顺子脸色有些古怪地道:“这个我早就想到了,不过赤骥传来话说,他们那里去了一位不速之客。”

我惊奇地道:“不速之客,那里是他们精心布置的密窟,怎会有外人来到?”

小顺子脸色更加古怪,道:“那人是叶天秀,庆王殿下的侍卫,你也见过的。”

这下我可真的呆住了,怎么会有这么巧的事情呢?

\chapter{第十七章 各有心思}

这几章很多人都有些各种各样的意见,可是坦白说,这都是我亲笔写的,也是我自己的思路,这也是没有办法的事情,我不可能总是在那里说江哲是多么阴险厉害,若是不将环境铺垫好,怎么写出那场血腥的夺嫡之战呢,所以大家耐心看下去,很快就要进入高潮阶段了。不过遗憾的是,我这周还是加了大半周的班,所以写作进度不够理想,所以我决定从现在开始,暂时改为一周发表五章,周末就不发文了,毕竟我已经进入了工作的高峰时期,不过相信我没有滥竽充数,不管什么文章,都不可能一直激荡人心的,总要有缓冲和铺垫的。

————————————————————————

却原来昨夜东市事变,长安城内全部戒严,叶天秀虽然侥幸逃生,可是却实在无力移动,最后便随便选了一间民宅,心想哪怕是用强将屋子里的主人给制住,只有自己能够休息一晚,将伤势调理一下,明日应该能够勉力逃走。可是世上就有这样巧的事情,这间宅子正是夏金逸的住处。

叶天秀一进院子,就被夏金逸听得清清楚楚,不过他知道自己不方便处理,便去叫醒了赤骥,而赤骥过去的时候,叶天秀已经昏迷不醒,待赤骥替他包扎好伤势,内外用药之后,叶天秀才醒了过来,他请赤骥替他到雍王府求救,这是因为他知道自己如此伤势,是绝对不可能生出长安了,而唯一可以保住性命的方法就是得到雍王府的援手,雍王殿下因为太子已然和凤仪门势成水火,看在庆王面上,或者会救自己一命。

若是别处,赤骥恐怕会为难,可是这人提到雍王府,赤骥心就放下了一半,他将消息送到雍王府的时候,小顺子听了也是一愣,他可是知道今夜庆王侍卫在京中被人屠杀的事情的,想不到叶天秀这样命大,不过叶天秀出现在夏金逸的藏身处,这该如何处理他就不能擅自作主了。

我沉吟了片刻,庆王和凤仪门为敌,那么就是自己这一方的盟友,而且多个朋友总比多个敌人好,叶天秀自然是要救的,可是夏金逸就不能住在那里了,如今的局势,如果夏金逸露了行踪,可不是好事,等到叶天秀离开之后,恐怕会有人来追查这个地方,所以必须让夏金逸离开,可是让他到哪里去呢,今日开始,长安必定是风声鹤唳,只怕难以藏身。思来想去,我道:“你亲自去一趟,让夏金逸想个法子改头换面,离开长安一段时间,现在的局势,我也无能为力,他应该能够明白。”

小顺子淡淡道:“公子,这人留着总是一个祸患,不如杀人灭口吧?”

我摇头道:“不行,我从未做过亏心之事,此人助我良多,不顾性命,我若是这样做,未免令人齿冷,你好好劝他,反正他在长安也没有什么作用,不如离开的好。”

小顺子点点头道:“那么我就亲自去一趟,我想赤骥不会让叶天秀见到什么不该见到的事情的。”

李顺带了雍王府的马车,向那藏身之处驶去,今日长安果然是一片萧条,街上到处都是禁军,不过雍王府的牌子很够用,没有人敢拦阻。车中,李顺心中暗想,若是夏金逸不肯答应,自己就是拼着公子责怪,也要将他杀了灭口。

没过多久,车子到了位于偏僻民巷的宅子,李顺命令随行的仆人在外面等候,自己独自进去,走进院子,李顺的眼睛突然闪过寒光,瞳孔因为杀气而有些缩小,因为他看到了一个有些熟悉,但又陌生的青年,那个青年相貌俊秀,肤色白皙透明,而更加独特的是那种冷淡的气质,他虽然站在那里,欣赏着院中那池荷花,可是在他眼中,李顺却看不到一丝喜悦,也看不到任何悲伤,仿佛他这个人就是没有情绪的存在。可是那种熟悉感又从哪里来呢?他仔细的打量着那个青年,终于闪过一丝惊诧和恍然,这个人,竟然就是那个夏金逸,这是怎么回事,为什么赤骥没有告诉自己夏金逸有了这样的变化。想到这里,他狠狠的瞪了一眼从旁边的房间出来迎接的赤骥。

赤骥却是有些莫名其妙,虽然夏金逸这几日变化极大,但是赤骥日日和他接近,反而觉不出来,对于夏金逸气质上的变化,赤骥只当是他悲伤而致,故而没有禀报给小顺子知道。他虽然心中奇怪,但是不敢多问,上前道:“这位夜爷,您就是雍王府的官爷吧,叶公子已经在房里等您了。”

李顺淡淡道:“你先下去,我和夏公子有话要说。”

赤骥神色有些不安,默默退下,夏金逸却是好像刚刚看到小顺子一样,亲热的走了过来,笑着说道:“原来是您亲自来了,大人最近可好?”

小顺子默默的看着夏金逸,他能够感觉到这人的确是真心高兴看到自己,可是古怪的,他也能够深刻感叹到这个人,根本就是一丝情绪波动也没有。突然,他一掌击向夏金逸,夏金逸神色似乎有些惊慌,可是却是飞快的举掌相迎,手掌相交,小顺子只觉的夏金逸的真气似阴柔,又似阳刚,十分古怪,一声巨响之后,小顺子纹丝不动,夏金逸却是后退了两步,白皙俊秀的面容上闪过一丝红晕。

小顺子没有继续出手,夏金逸却也没有惊慌之色,肃手而立,却是微微一笑。

小顺子淡淡道:“你发生了什么事情?”

夏金逸眼光一闪,微笑道:“也没有什么,只是觉得自己像是换了一个人,从前的事情都不放在心上了,”

小顺子冷冷道:“公子命我转告你,如今长安城十分危险,若是你愿意,可以暂时到外面避一避,如果你愿意,我可以代公子作主,放你自由离去。”

夏金逸眼中杀机一闪,道:“不,若不看到李寒幽收到惩罚,夏某绝不离去。”

小顺子眉头一皱,道:“凤仪门之事不是一天两天就可以解决的,你不方便留在京城。”

夏金逸默然,片刻之后才道:“你不是也觉得我有很多改变么,现在他们还会认得出我么?”

小顺子想了一想,道:“乍看之下可能不会,可是你在太子府呆了许久,很多人都有可能辨认出你。”

夏金逸神色恭谨地道:“请李爷向大人转达夏某心意,夏某情愿替大人效力,改变容貌并不困难,夏某相信不会随便被人认出。”

小顺子心中一动,夏金逸不知道发生了什么事情,武功突飞猛进,此人聪明伶俐,若是留在公子身边,倒也不错,易容术虽然不能彻底改变一个人的特征,但是夏金逸的气质发生了很大变化,只要深居简出,应该可以瞒过他人的眼睛。而且他若胡闹起来,不肯离开,自己纵然是杀了他,也不是一招两招的事情,若是给叶天秀听到一些事情,也是后患,不如将他带回雍王府,若是公子说可以留用,就留他在寒园,若是公子说不行,自己就杀了他。想到这里,他心中一宽道:“你跟我回去雍王府见公子吧。”

夏金逸不是不明白小顺子心中暗藏的杀机,可是他也相信自己能够得偿宿愿,便恭恭敬敬地道:“草民谨遵官爷谕令。”

小顺子无奈地一笑,走向叶天秀养伤的厢房,在病榻之上,叶天秀神色惨白,大半个身子都用白布缠绕包裹着,看到小顺子,他勉强坐起身来,苦笑道:“原来是李兄亲来,天秀感激不尽。”

小顺子肃然道:“昨夜闻叶兄遇袭,殿下和我家公子都是十分担心,想不到叶兄逢凶化吉,大难不死,定有后福,但不知叶兄可知道昨夜是何人出手么?”

叶天秀苦笑道:“来人蒙面出手,剑术高强,叶某自愧不如,但却不知那人身份。”

小顺子目光一闪,又问道:“可知道那人是男是女,用的是什么剑法?”

叶天秀早已将那日情形回想了千遍万遍,此刻他毫不犹豫地道:“那人是个男子,虽然他眉目秀雅,可是叶某和他苦战良久,那人绝非女子,否则我也不用猜是谁做的了,他的剑法也很出众,精妙高深,有些像越女剑法。”

李顺眉梢一动,道:“你是怀疑夏侯沅峰么,他练得不就是越女剑法么。”

叶天秀摇头道:“我也想过可能是他,可是我曾经见过夏侯大人的剑法,觉得没有这个蒙面人凶狠凌厉,而且越女剑法虽然博大精深,可是并非一脉单传,江湖上有很多流派,凭着这一点实在不能确认是否夏侯大人。”

李顺也不去多想,这件事情总有水落日出的时候,何必急于一时,便笑道:“叶侍卫,还是先到王府吧,您的伤势也要重新处理一下,这些事情以后再说吧。”叶天秀欣然点头。

这一天虽然长安局势渐渐平定,可是私下里却是暗波汹涌,一大早,李寒幽就进宫拜见纪贵妃。两人在纪贵妃居处对坐品茗。李寒幽明显的神思不属,纪贵妃却是神色淡然。两人说了半天闲话,李寒幽终于忍不住了,问道:“师叔,这次恩师前来接管权力本是无可厚非,可是昨夜长安乱成这个样子,寒幽却是什么都不知道,您说,是不是师父对寒幽有了不满?”

纪贵妃淡淡一笑道:“你过虑了,这些年你做的很好,若是门主觉得你有错,是绝不会轻轻放过你的,只是这些事情不适合你去做,你虽然是内堂弟子出身,可是如今嫁给了秦青,名义上就成了外堂弟子,这些事情是不适合你们做的,对凤仪门来说,你们维持今日的荣耀地位,远比你们做那些事情更重要。”

李寒幽叹息道:“当日门主安排我下嫁秦青,说句心里话,我是不愿意的,师叔,我真的很想成为师父的衣钵传人,可是……”

她没有再说下去,纪贵妃却很清楚她的未尽之意,凤仪门主的权威不容反抗,而且,富贵荣华逼人来,又有几人能够狠心拒绝。手中团扇轻摇,纪贵妃雍容地道:“其实你不用太担心,虽然下任门主你是不能了,可是门主的意思很清楚,未来的凤仪门并不是门主一人作主,紫烟修为最高,又对师姐忠心耿耿,凤仪门这些年精心培养的武力大半都在她掌握之中,只是凶残之名太盛,所以是没有什么希望继承门主之位,你二师姐萧兰和五师姐秦铮,都已经嫁人,已经失去继承资格,三师姐凤非非在江湖上虽然有些名望,但是却不能驾驭群雄,也只能处在辅佐地位,你四师姐梁婉如今已经是神智不清,你七师姐又是性子轻率,更不能担当大任,只有你六师姐凌羽和八师姐燕无双一个清丽出尘,一个艳冠群芳,武功也不错,最符合门主的要求,不过你也不用担心,按照现在的情形,紫烟这监察之位是跑不了的,我们这些身在朝廷中的弟子自然是一派,非非、羽儿、晓彤、无双也是一派,谁也别想独断专行,只要你够本事,让兰儿和铮儿对你惟命是从,还怕不能和她们分庭抗礼么。”

李寒幽越听越是欢喜,道:“多谢师叔指教,还希望师叔多多提点。”

纪贵妃笑道:“你是冰雪聪明的人,还糊涂什么,只要你不要露出不满之色,师姐是不会放弃你的,这次的事情不是我们安排的,我们自然可以理直气壮的说话。”

李寒幽有些忧虑地道:“可是弟子听说是大师姐策划了刺杀郑暇,若是传扬出去可怎么办?”

纪贵妃冷笑道:“你怕什么,别说不是你干的,就是你亲自出手也不用怕,这次为什么门主同意月宗的人去屠杀庆王的人,不就是用来掩饰我们刺杀郑暇的行动么,若是庆王的人死了,只怕人人都会怀疑我们,可是就是怀疑也没有关系,谁不知道我们和庆王之间的恩怨,只要我们没有直接去杀了庆王,皇上是不会责怪我们的,何况又没有证据,谁会想到我们要杀的是郑暇呢?”

李寒幽叹息道:“门主真是难以揣度,现在弟子也不明白为什么去杀郑暇。”

纪贵妃叹息道:“唉,师姐也是不得已,郑暇为人严刚,这次皇上回来就是有心放了太子,这郑暇也必然像上次召见一样,直言批评太子失德,偏偏皇上又对他十分敬重,若是让他在皇上面前多进谏几回,只怕太子的储位是保不住了,为了我们的目的,也只好牺牲郑大人了,只是可惜没有成功,不过他这次是别想动摇太子的地位了。”

李寒幽笑道:“只有月宗最蠢被我们当成了挡箭牌。”

“谁蠢还不一定呢。”鲁敬忠笑着轻摇折扇,缓缓说道。而坐在对面的礼部尚书夏侯阑说道:“师弟,你也不要太过轻敌,凤仪门主手段厉害,你又不是不知道,当年我们日月二宗不少人死在她手里。”

鲁敬忠神色一肃道:“师兄,我知道这女人的厉害,可是如今她也不可能把我们铲除,太子殿下虽然不算精明,可是提防凤仪门他还是知道的,而且他和凤仪门心中嫌隙已经很深,我自信可以和凤仪门主分庭抗礼。”

夏侯阑微微一叹道:“师弟,我们月宗自从二十年前会盟之后,如今已经是人才凋零,可经不起折损了。”

鲁敬忠冷冷道:“师兄是月宗元老,自然爱惜羽毛,可是我鲁敬忠却是在三十年前得到恩师传授,虽然现在我也不知道恩师在月宗是什么身份,可是如今的一切都是我自己双手得来的,我绝对不容许被人夺走。”

夏侯阑苦笑道:“这件事情我也不大清楚,可是听先师讲,我们月宗传承了十七代,中间多次发生典籍散失的情形,但是也总是香烟不断,先师曾说,魔宗必然另外有专门负责传承的分支,甚至先师怀疑那些人就是只闻其名,却连我们自己也不明白详情的星宗弟子,先师这一支十分侥幸,连传数代而不断绝,有些事情他也曾经深为不解,可是先师有一件事情却说的很明白,历代月宗弟子,多以阴谋为体,不得善终,所以我极力阻止沅峰涉入魔宗事务,可是你却总是不肯放过他,这次又让他去杀庆王侍卫,你真得要和我作对到底么?”说到后来,夏侯阑的眼中闪过一丝杀机。

鲁敬忠却坦然道:“师兄,你可以大隐于朝,可是侄儿青春年少,如此人品才智,你怎么忍心让他碌碌无为,再说,自古以来,若是智勇之士,鲜有安逸偷生之辈,我既然有这般才华,这世间就应该有我的地位,若非是野心和傲气,月宗怎会传承不断,明知道每次会盟之后,二三十年之内相互残杀,最后不过一两个能够得到富贵权势,可是可曾有人放弃过,谁不想辅佐明主一统天下,画影凌烟,而且还可以成为月宗宗主,凭借宗主符令,就可以得到星宗接引,往窥‘阴符经’真本,可惜这近千年以来,只有第十三代有位祖师晋为宗主。”

夏侯阑神往地道:“而且那位宗主神秘消失之后过了半年又回来了,心满意足地含笑而逝,可惜终究不肯说他看到了什么。”

鲁敬忠眼中闪过狂热,道:“我若生不能一窥阴符经,宁愿一死。”

夏侯阑淡淡道:“不错,我也曾经这么想,祖师爷当年智深如海,只将七层所学传下,就有了今日的月宗,我愿曾经愿意付出一切代价,想看一看祖师爷的遗作。可惜如今我心灰意冷,只想平平安安的渡过一生,所以你还是不要再打沅峰的主意了。”

鲁敬忠眼中闪过一丝讥讽,道:“师兄真的以为是我一人的主意么,侄儿聪明过人,你又曾经传了所学给他,他也是气盛少年,怎肯俯首于人,师兄,你若是当初不教他读书学剑也还罢了,今日已经迟了。”

夏侯阑神色一变,良久才道:“不错,你说的不错,确实迟了。”

\chapter{第十八章 雍帝回銮}

高祖归,于太宗着意嘉勉,太宗自请巡边,帝未许之。

——《雍史·太宗本纪》

六月十六日,未时末,长乐公主在禁军和御前侍卫的保护下返回皇宫,她坐在公车之中,秀丽的面庞上带着淡淡的担忧,就在方才,夏侯沅峰通过绿娥求见,她原想拒绝,可是转念一想,夏侯沅峰从前虽然有求凰之意,可是自从自己拒绝之后就没有前来纠缠,现在想起来,夏侯沅峰倒比那个温文尔雅的韦膺识趣一些,便许他觐见。

夏侯沅峰此来也没有说什么特别的事情,只是委婉的说道:“近来臣得到消息,有人想极力促成殿下和韦大人的婚事,从前陛下赐婚,殿下虽然拒绝,可是陛下始终没有撤回旨意,所以有人想迫使公主履行婚约,因为这一年多来,殿下和雍王府走得很近,虽然殿下不愿介入纷争,可是在有些人眼中,殿下还是支持雍王的,所以有人想让公主迅速完婚,这样一来,韦家的立场本是中立的,公主乃是德言容功出类拔萃之人,绝不会让夫家为难,那些人也是想釜底抽薪,谁不知道殿下和雍王府交好,而且皇上对公主恩宠非常,他们也不想让公主影响了皇上的观感,何况现在太子的储位岌岌可危,正是他们不敢轻忽的时候,所以殿下的婚姻,他们看的很重,可是他们也不敢用强,恐怕会用些手段,公主千万小心在意,韦大人虽然人品端重,可是他对公主一片痴心,恐怕会受人利用。”

长乐公主透过车窗上的轻纱帷帐,向外看去,长安街上一片肃然,禁军密布,车马不行,她心中不由十分怅然,想起当年建业危急之时,自己被大雍密谍救出王宫,也是在车中看到原本繁华德街道上倒是都是慌乱的人群,如今车外剑拔弩张的气氛,和那时比起来其实也没有什么不同吧。

六月十八日,雍帝李援返回长安,这次李援明显的心情不好,即使在百官跪迎的时候也是一脸的铁青,在他回来之前,对着接驾的雍王勉强称赞了几句,便匆匆回宫,然后便立刻召了韦观、李贽和秦青进宫。而随驾的抚远大将军秦彝、魏国公程殊和齐王李显却都奉旨回府休息了。

当着三人的面,李援愤怒的摔碎了茶杯,道:“你们真是好本事,短短的几天,朕的长安就成了这个样子,郑侍中遇刺,东市事变,长安火起,好,你们说,朕该如何处置你们。”

三人连忙跪下请罪,韦观诚惶诚恐地道:“臣奉命主管政务,都是臣失职,才有这样的事情发生,还请陛下重重治罪。”秦青则是满面羞愧地道:“臣有负圣恩,没能维护皇都安宁,郑侍中遇刺在先,东市火起在后,若非雍王殿下亲临东市主持大局,恐怕事态还会扩大,请陛下免了臣的官职吧。”李贽也歉疚地道:“都是儿臣失察,数日前,儿臣已经得到边关不靖的军报,可是没有看在眼里,如今已经查明,乃是北汉密谍趁机作乱,儿臣乃是父皇亲封的天策元帅,罪责难辞。”

李援看着争先恐后请罪的三人,却是觉得十分疲倦,他跌坐在龙椅之上,心道,若非你们争权夺势,怎会让长安如同不设防的集市一般,任由敌国间谍出入。可是李援很清楚这种情况实在是自己一手造成,自己立长子为储君,虽然是制度的缘故,可是自己并不是没有私心的,李贽的精明强干让他总是心中有些嫉妒,所以总是想压着他,可是李援又深知,自己的子嗣之中只有这个儿子能够青出于蓝,可是因为种种情势,自己还是决定支持李安。难道,我错了么,李援想起自己在黄陵得到八百里加急的奏章之后,愤怒的想要杀人,却不知道可以怪罪谁。

韦观乃是文官,怪罪无用,秦青虽然有亏职守,可是想一想,如今的长安也不是他可以作主的,再说自己不就是因为秦青比较容易使用才让他当禁军统领的么。还有雍王李贽,自己又能怪他什么,这几年来,他几乎日日身处凶险之中,不得已韬光养晦,这次事发之时,他也刚从斋宫出来,而且若没有他不顾生死力挽狂澜,只怕这长安不是成了废墟,就是成了屠场,而且还险些遇刺,理应嘉勉,可是如果自己嘉奖他,那么太子又怎么办,真得要废他么,李援心中虽然对太子十分失望,可是还是不愿轻易废黜太子,他心中很清楚,这样的事情写在史书上,是要让自己脸面抹黑的,更何况冠冕堂皇的借口还是要有的,可是目前太子的罪行却如何能够让外人得知。

想到这里,他疲倦地挥挥手道:“罢了,韦观罚俸一年,秦青官降一级,仍然暂代统领之职,以期戴罪立功,雍王有陪祭之功在前,又有平乱之功在后,本应重赏,只是如今你已经封无可封,朕就赐你黄金三千两吧。”

李贽叩首道:“儿臣叩谢父皇赏赐,只是儿臣不缺金银,这次长安事变,平民百姓多有无辜受害者,愿父皇将这些赏赐用作救济,则儿臣感同身受。”

李援深深的看了李贽一眼,心中又是欢喜又是忧虑,笑道:“贽儿你果然不愧贤王之称,好了,朕准了,你遇刺受惊,回去要多多休息。”

李贽连忙道:“父皇,从这次的事情和边关军报来看,只怕北汉蠢蠢欲动,若是父皇允许,儿臣想到边关巡视一下。”

李援目光一闪,道:“这件事情朕再想想,你先下去作些准备吧。”

李贽心中一喜,来之前,江哲曾经说过,若是皇上立刻同意,那么殿下恐怕是没有机会光明正大的登上储位了,虽然说龙腾深渊,虎啸山林,自由自在,可是那就意味着皇上根本无心立您为储君,否则绝不会让您在这个时候远离朝政中心,若是那样一来,臣恐怕殿下您只能用武力夺取皇位了,那绝非殿下和臣所期望的。若是皇上坚持留您下来,那么殿下还有五成机会被皇上立为储君,因为还有五成可能是皇上对您猜忌已深,绝不愿您回到军中。但若是皇上犹豫不定,那么恭喜殿下,皇上已经对太子失望,只要殿下处理得当,那么取得储位并不困难。

李贽对江哲最佩服的一点,就是他能够一眼看穿他人的心思,不过却不包括他身边的人,例如小顺子,例如柔蓝,这大概就是可察秋毫之末,却不见泰山的道理吧。满怀欣喜却不敢宣于言表的李贽,兴匆匆的告退回府了。

李贽回去雍王府自然是满心欢喜,韦观回府也没有人敢责备他,只有秦青,满心惴惴不安,不知道父亲会如何惩罚自己。想来想去,还是先去找秦勇,让他陪自己去见父亲,也好让父亲对自己轻罚一些。想到这里,离开皇宫的秦青也不回自己的驸马府,也不去拜见父亲,而是先去秦勇的家里。秦勇虽然是被秦彝收养在府里,可是早在十年前,秦勇就搬出了秦府,据说是因为他的母亲不大适应大将军府的威严,秦青在成亲之前就经常去秦勇家,其实两家隔得并不远,秦母出身贫寒,虽然上了几岁年纪,但是身体健康,还是喜欢种菜养鸡,秦勇又雇了几个仆妇照顾母亲,所以母子两人都是十分惬意,秦青就最喜欢去吃秦母做的小菜,总觉得比起家里的名厨做的还好,可是他成亲之后,却是渐渐的远离了这些生活。

一边回想,一边策马而行,没有多久,秦青就到了秦勇的住处。跳下马,他用力敲门,门内传来一个充满朝气的声音道:“来了,大哥回来了么?”秦青一愣,这是怎么回事,难道勇哥搬家了么。还没等他想清楚,门已经开了,一个十六七岁的俊秀少年探出头来,看见秦青就是一愣,问道:“这位官爷,您找谁啊?”

秦青犹豫地问道:“秦勇在么,我是他的堂弟。”

那个少年眼睛一亮道:“干娘总是说起将军呢,还说您最喜欢她的菜。”说罢转过头去喊道:“干娘,干娘,秦青秦将军来了。”

门里面传来笑语声道:“什么秦将军,在这里他也是你堂哥,华儿,还不让青儿进来。”

那个少年嘻嘻笑着,把门拉开,秦青满面糊涂的牵马进去,将坐骑系在院中的大槐树上,对着站在台阶上笑容满面的苍老妇人道:“婶娘,这些日子没有来看您,您老身体可好?”

老妇人道:“好着呢,就是你勇哥,总是忙得不着家,幸好还有华儿陪我。”

秦青疑惑地问道:“这位小兄弟是您的义子?”

老妇人笑道:“他叫刘华,原本是江南人,自小无父无母,在外流浪,前几年跟了一个大商人做了几年工,也算是读了些书,长了些见识,后来流浪到长安,却不幸生了病,幸好你勇哥有一天发现他病倒在路边,就把他拣了回来,我看这孩子聪明懂事,索性就收了这个干儿子,他也没有别的好处,就是知疼知热,勤劳肯干,现在在一家绸缎庄当伙计,已经升了领班了,不像你勇哥,就知道在军营里面厮混,现在也没有给我找个儿媳妇,让我抱抱孙子。”

秦青这才明白过来,看向刘华,只见这个少年眉清目秀,眉弯如月,眼明如星,嘴角含笑,令人见之便觉得可亲可爱,不由心生好感,便笑道:“既然是婶娘的义子,你也叫我一声四哥吧,我们这一辈,勇哥排行老大,我是老四。”

刘华乖巧地道:“小弟给四哥见礼,四哥是来找义兄的么,方才大将军已经把义兄叫去了。”

秦青心里一慌,问道:“你看勇哥神色怎么样,有没有担心我爹爹责罚。”

刘华差点笑出声来,忙道:“勇哥没什么异常,就说今天晚上可能不回来,让我和干娘不用等他。”

秦青心里嘀咕,当然不用等他,看来今天晚上跪祠堂的时候有人陪我了。想到这里,他也不敢再耽搁时间,便道:“婶娘,你们忙吧,我也得回去给父亲请安了。”

老妇人笑道:“这也是的,你们兄弟都一个样,今天勇儿也是正要去见大将军,就被大将军派来的人召去了。”

秦青听得更是心慌,连忙匆匆告别,上马就向大将军府驰去,他可没有看见,送自己出门的那个少年刘华,眼中露出了一丝古怪好笑的神色。

秦青满心都是忧虑,又想快些到家,免得父亲火气更大,又害怕见到父亲之后,不容分说就是一顿棍棒下来,让自己进祠堂跪着。就这样犹犹豫豫地回到家中,一进门,就有家将禀告,老爷有令,公子一回来就到书房见他。

秦青心中就是一凛,父亲的书房可是他最恐惧的地方,每次自己若是犯了错,第一件事情就是被叫到书房,可是现在也不能溜走了,只得故作镇静地来到书房门前。当秦青终于鼓起勇气推门进去的时候,却是一愣,秦彝一身便装,正在和秦勇指着地图说着什么,见到秦青进来,只是淡淡的看了他一眼,就继续和秦勇说话,秦青仔细听去,却是父亲正在和秦勇商议,如何重新布置长安防卫,免得今次的事情再次发生。秦青不由一阵惭愧,也不敢插话,只听父亲和秦勇商量着如何布防,从前禁军的主要职责是维护皇城,对于长安城内的治安主要是由京兆尹负责的,所以这次发生事故,禁军有些措手不及,虽然也有禁军的实质上的统领秦彝不在的缘故,可是随机应变还是有些不足,所以秦彝重新规划了禁军的布防以及训练的方案。

等到两人商量的差不多了,秦彝才拿起茶杯,喝了一口,淡淡道:“青儿,你有什么要对为父说得么?”

秦青心里一跳,连忙道:“父亲,都是青儿无能,还请父亲责罚。”

秦彝微微一笑,道:“如今你是靖江驸马,我也管不了你了,这次的事情我不怪你,你年纪尚轻,声威不足,这次能够处理成这个样子,也是勉强合格了,我要问你的是,前些日子,你为什么拦阻雍王府江司马的车驾,这些日子,我一直等你来向我说明这件事情,可是你却一直没有来。”

秦青先是一愣,然后恍然道:“原来是这件事情,父亲不提,我几乎忘了,说起来我现在还是有些气恼,当日明明是有叛逆藏在车上,可是江哲用金牌迫我不能搜查,如果不是寒幽说不应该多事,我还想密奏陛下呢……”

话刚说到这里,秦彝已是满面怒火,手指轻颤,几乎拿不住茶杯,良久才道:“我倒不知你有这样的才智,好,好,我真是有个好儿子。”

这下秦青可吓坏了,他对父亲的畏惧由来已久,连忙跪倒在地,颤声道:“父亲息怒。”但是神色迷茫,显然不知道自己错在哪里。

秦彝心中一阵悲凉,这世上至亲莫过父子,他何尝不希望自己的儿子出类拔萃,领袖人伦,可是秦青却是如此愚顽,总是看不清事实,这样的资质,作个军官也就罢了,可是他却是跻身朝廷的中心,如今有自己照顾,还可以平安无事,将来若是自己去了,还有谁能够照顾他,就是靖江公主李寒幽为了夫妻之情指点于他,也恐怕只能沦为棋子,早知今日,自己当初就不会同意把他调回京师。他强忍怒气道:“你这逆子,雍王府是你惹得起的么,别说江司马车上的人未必就是叛逆,可是就是真有其事,也轮不到你来插手。”

秦青嗫嚅地道:“可是那是真的,父亲不是说行事主管禁军要光明正大,不畏权贵么?”

秦彝怒道:“我要你光明正大,不畏权贵,是要你不要为虎作伥,保护无辜,却不是让你去和雍王为难的,如今谁不知道雍王功高盖世,却得太子忌惮,他们之间乃是兄弟閲墙,我们作臣子的只能袖手旁观,自古以来争夺储位没有什么善恶可辨,只要他们不伤害平民无辜,要你这个小子多什么事。你要替靖江公主的闺中密友抱不平,为难裴云也就罢了,虽然裴云没做错什么,可是却不该公然和雍王府为难,别说当日车中可能有不便让你见到的人,就是没有,若是他们让你乖乖搜了车驾,岂不是雍王府颜面无存,到时候就是雍王再宽宏大量,也不能饶恕你的无礼。”

秦青也不是笨人,听到这里,满面通红,不知道该说些什么,秦彝叹了口气,道:“何况有些事情并非如同表面看上去那样简单,你以为那人是叛逆,可是却忘了他和皇上乃是血缘之亲,你若报了上去,却是让皇上管是不管,这些事情你怎能随便插手,罢了,我也不多说你,去祠堂好好反省一下,妇人之言,怎能百依百顺,哼。”

这时,门外有人禀报道:“秦大哥,皇上传下旨意了。”

秦彝微微一愣,道:“什么旨意?”

那人推门进来,却是魏国公程殊,他肃容道:“皇上下诏,太子前些日子养病宫中,如今病愈,可回府邸继续休养,暂时不用到东宫主政,雍王这次功劳卓著,本应重赏,但允其所请,将赏赐用以赈济受害百姓,还有,齐王明日出京,代天子巡视边关,提防北汉进攻。”

秦彝品味良久,道:“陛下今次决断可真是耐人寻味啊。”

\chapter{第十九章 公主密谏}

六月十九日,高祖下诏,王得以归家,然免王主政之权,王恐惧不安。

——《雍史·戾王列传》

李援的诏旨如此迅速,自然是人人惊异,但是也只道他早就有了成算,谁知道此诏的拟定却是一夜之间的事情,那日雍王等人走后,李援心中烦恼,从前他若是有了疑难之事不能决断,便常常和自己重臣商议,可是今日之事却是不同,韦观一向中立,必然不会多说什么,秦彝、程殊都是军人,他们平日对于政务都是不愿插手的,郑瑕,唉,郑瑕为人刚直,凡事总是秉公持正,可惜如今身负重伤,不能参赞,想来想去,只有纪贵妃可以商议,可是李援却不愿去找她,若是从前,李援属意太子继位,自然纪贵妃的献策是有用处的,可是如今他对太子十分失望,可是凤仪门的态度却很明确,凤仪门主据说已经亲自到了长安,虽然没有来见自己,可是只看她的作为,就知道她仍然是支持太子的,这样一来,纪贵妃的态度也就定了,此刻李援只希望能有一个不存私心杂念的人可以和自己商量一下这件事情。想来想去,李援十分烦恼,想起后宫之中,人人和朝政有着牵涉,唯有长孙贵妃无欲无求,不如到她那里去消磨一下时间吧,看看天色,他也不令人先去通知,便走向长孙贵妃居住的长春宫。

走进长春宫,长春宫的总管太监连忙过来叩见,并说娘娘和公主正在宫内的花园里面散心,李援走向花园,还没有走进园门,便听到一阵轻快的笑声,不由心中郁闷稍减,走进去一看,却是长孙贵妃坐在凉亭之内,长乐公主穿着胡服,正在和两个宫女在空地上陪着柔蓝蹴鞠,柔蓝虽然年纪小小,却是十分灵活,追着球到处跑,再加上众人相让,居然踢得不错,只看她天真烂漫,就令人心中苦恼尽消。

这时太监高声道:“陛下驾到。”

众人听了,连忙过来见驾,李援笑着道:“朕过来看看,你们不用拘礼。”说着上前抱起小脸红扑扑的柔蓝,问道:“小柔蓝踢得很好么,今天怎么有空进宫啊,每次都得你长乐姑姑亲自邀请,才肯进宫呢?”

柔蓝忽闪着大眼睛,奶生奶气地道:“皇上爷爷,蓝蓝也想来看公主娘娘和皇上爷爷,可是他们都说如果蓝蓝总是来看公主娘娘,有人会生娘娘的气,蓝蓝就不敢来了。”

李援心中不由一怒,他自然知道柔蓝的意思,有人是担心长乐公主和雍王府太亲近了,他面色的变化却让长乐公主吓了一跳,连忙过来道:“父皇,柔蓝不懂事,您别见怪。”

李援叹了一口气,挥手斥退服侍的宫女太监,长乐公主连忙让绿娥也将柔蓝抱了下去,而冷川也知道他们有私事要谈,便也退到远处,李援淡淡道:“长乐,真是苦了你了,你这些兄长不成器也就罢了,却还要牵连到你。”

长乐公主连忙笑道:“父皇,也不过是二皇兄他们过虑了,其实也没有什么人为此迁怒儿臣。”长孙贵妃也说道:“是啊皇上,贞儿是你最宠爱的女儿,谁敢和她为难。”

李援叹了一口气,道:“唉,朕对太子十分失望,可是这废立之事岂是可以轻易决定的,如今朝中上下这些大臣,不是希望保住太子,好在储君面前邀功,就是想拥立雍王为储君,朕也是十分难办。”

长孙贵妃眼中闪过一丝忧色,她虽然素来不参与军政,可是也知道如今情势,按她的本心来说,不论何人继位,和她关系都不是很大,虽然因为雍王妃高氏的缘故,她不免对雍王有些好感,可是还不足以让她支持雍王,如今皇上却对自己说及此事,自己若是说了什么不该说的话,只怕今天说了,明日就给人知道,从今之后自己可是要难以安宁了。因此,她只能不着边际地道:“皇上也不用忧虑,这些臣子心思各异也是理所当然的事情,这立储之事还是得您乾纲独断。”

李援听了虽然觉得有些空泛,却也觉得舒心,忍不住道:“话虽如此,朕也是进退两难,太子虽然不好,可是毕竟做了多年的储君,雍王虽然好,可是却是野心太大,朕深觉立国不易,很担心他急功近利,毁了家山社稷。”

长孙贵妃欲言又止,虽然十分欣慰李援如此信任自己,可是后妃干政,毕竟是后患无穷的事情。

李援也知她为难,他原本也不指望长孙贵妃给他什么意见,只是想发发牢骚罢了,所以也不多问,之事将自己烦恼之事说了出来,图个心中痛快罢了。谁知说着说着,却见长乐公主若有所思,便好奇地问道:“长乐,你可是有什么看法么?”

长乐公主稍一犹豫,便开口道:“父皇,儿臣虽然不懂得军国大事,却觉得,不论是父皇心里打算如何,都应该将事态平定再说,不论您如何决定,都可以日后慢慢安排,现在这样悬在半空,不仅是太子忧虑,二皇兄苦恼,就是文武大臣也不免惴惴不安,担心看错了风向。”

李援心中一动,心道,长乐说得很有道理,我这样迟迟不作决定,太子固然是担心储位不保,心生怨望,就是雍王也不免心存期望,到头来若是不合心意,双方都不会满足,若是自己现在暂时将他们安抚下去,主意拿定之后,再慢慢安排,岂不是两全其美,想到这里,他高兴的站了起来,道:“长乐说得不错,好了,朕要去拟旨,你们随意吧。”说着李援立刻回到御书房,下了诏旨,也不容群臣劝谏,雷厉风行的颁下了圣旨。

这道旨意一下倒是皆大欢喜,太子固然是欢欣鼓舞,叩见父皇谢恩之时,感激涕零,几乎是指天誓日的向李援保证必然会洗心革面。齐王也是心中欢喜,这一两年来他几乎是被拘在京中,平日除了走马章台就是弄鹰调犬,早就恨不得回到边关打上几仗,现在有了机会自然是很高兴的,所以几乎是诏旨一下,齐王就连跟太子说一声也顾不上就匆匆出京了,这自然是让太子恨得牙痒痒的。

除此之外,按理说,本来颇有机会促使太子废黜,而自己登上储君之位的雍王应该是希望落空,不免烦恼了,事实上,这几天雍王却是一派雍容气度,第一个去给太子贺喜的是他,当然理由是贺太子病愈,然后又亲自送齐王去了边关,去探望郑侍中的伤势,倒是天天忙得很,虽然他面上一片平静,可是从他脸上看不出一丝欢容,因此人人猜他确实有些不满气恼,不过也都交相称赞雍王气度宽宏,心胸宽阔,浑不知李贽若非是在外面装个样子,只怕已经喜上眉梢了。

接到李援的圣旨之后,李贽原本是心中郁闷的,觉得父皇太偏爱太子,谁知进了寒园,江哲却向他贺喜,李贽烦恼地道:“随云,现在摆明了父皇的偏心,你还庆贺什么。”

江哲笑道:“殿下这是当局者迷,如今皇上对太子已经是很失望了,若是皇上秘密的训诫太子一番,说明皇上还是对太子有所期望的,可是据臣所知,皇上并没有什么训斥,俗话说,爱之深,则之切,现在皇上竟然一点也不责备太子,这正是皇上已经不愿浪费什么时间了,依臣之见,如今殿下离储位只有一步之遥了。”

李贽苦恼地道:“就算是一步之遥,也是咫尺天涯,现在凤仪门主进京,太子势力大增,就是立刻刺杀了我们也是可能的,再说有她督导,太子必然谨言慎行,这次父皇没有废黜太子,那么就是还有余地,若是拖下去,恐怕对我不利,再说,废黜太子需要有罪状,太子若是不犯错,那么就是父皇想要废黜他,也是不可能的了。”

我笑道:“如今太子恐怕不是这么想的,这次皇上虽然放了太子,可是不许他在东宫理政,疏远之心已经有了端倪,太子如今恐怕是心中狐疑,很怀疑皇上会对他动手,为了自保,恐怕太子就会越陷越深,现在殿下只要传流言出去,说皇上这次不废太子,不过是因为太子的后台势力罢了,然后我们就以掌握的太子一党的罪状发起进攻,也不攻击太子,只说那些人有负皇上和太子的恩泽,以殿下的声威,必然是手到擒来,我们这样做,表面上不会损及太子自身的安危,因此太子不会想到是我们故意而为,反而会以为我们是奉了皇上的密令,所以殿下最近找个机会和皇上密谈几次,不要让别人知道实情,这样太子更会怀疑皇上已经下决心立殿下为储君,所以才安排殿下剪除太子羽翼,这计策就是打草惊蛇,只要太子心中惊疑,那么就会盲目妄动,自然会出错,到时候就可以水到渠成的废黜太子了。”

李贽听到心服口服,道:“随云可谓是看透人心,不错,谁会想到我们这样大张旗鼓的剪除太子羽翼,其目的却不是为了打击太子的势力呢。”

我站起身道:“殿下,如今已经到了最紧要的关头,殿下应该诏子攸先生回长安主持大局,臣虽然有些谋略,可是很多事情只有石大人才能处理的妥当,石大人乃是相辅之才,若是他不回来,就太可惜了。”

李贽动容道:“随云说得不错,如今确实需要子攸回来,现在幽州的局势已经很稳定,子攸也招揽了大批可用之才,他在幽州也没有更大的作用了,还不如回来的好,子攸处事周密谨慎,这个时候本王也确实需要他来主持大局。”说罢,心中暗道,江哲果然是心胸宽阔,子攸回来之后,自己虽然还要倚赖他出谋划策,可是不免会更加信任重用石彧一些,毕竟石彧是自己心目中的丞相,文官之首,可是他却丝毫没有忌惮。

他却不知我本就不在意什么权势富贵,再说我身体不好,很多细节的事情都是管休、董志和苟廉安排的,就是石彧回来,对我也没有什么影响,再说,石彧回来对我还有一个十分重大的好处呢。

商量妥当之后,我送雍王殿下出去,还没有走多远,李贽就看见一个身穿侍卫服色的青年走了过来,他相貌俊美,气质淡漠,李贽一见便觉得这人不凡,可是奇怪的,李贽觉得这人自己似乎曾经见过,可是却想不起来曾经见过这样一个气质独特的青年侍卫。

他脚步一缓,我就察觉到了,却没有作声,雍王殿下曾经见过夏金逸几次,这次正好试一试夏金逸的易容是否成功。说到易容,我也曾经被野史中的传说骗了,说是有人可以改变容貌,让熟人也认不出来,可是这是不可能的事情,首先是相貌的改变有很多局限,天生人的相貌,不论丑俊,总是能够给人一些和谐的感觉,若是妄自改变,反而容易让人觉得有些突兀,而且想要易容,本身的特征也很重要,若是你的相貌身材有些特别之处,就是易容也难以掩人耳目,就是相貌改变的成功了,还有行动举止和言谈上面的改变,很多人只要看了背影,听了声音,就可以认出自己的亲人朋友,所以要想让熟人都难以认出,真的是难度很大。

不过这一次,我却是很相信自己的手段,虽然我对易容只知道一些前人的心得,并没有亲自着手试过,可是夏金逸对易容倒是有些手段的,我只要指导他如何做就行了。经过仔细研究之后,我首先让他在相貌上作些小小的改变,不过是改变一下梳理头发的样式,眉梢眼角稍微改动一下,配合他改变的气质,很容易就让他像是变了一个人一样,然后我又花了一些时间,教他改变一些动作,说话的时候改变节奏和音调,他学得很快,现在果然表现不错,雍王就没有立刻认出他来,再加上“夏金逸”已经死在皇上迁怒之下,所以只要他深居简出一段时间,自然不会有人认得他了,再说过上一两年,也不会再有人追究这件事情了。

见到雍王神色犹疑,我笑道:“殿下可是见到生人了么,他叫董缺,是臣新收的侍卫,虽然不是军中出身,不过殿下放心,此人忠心无虞。”

雍王恍然道:“原来是你新收的侍卫,怪不得本王虽然觉得面熟,却是想不起来他叫什么。”

夏金逸,如今的董缺,上前给雍王见礼,礼数一丝不苟,神色却十分漠然,李贽也没有留心,只是笑道:“随云难得收一个属下,想必是个人才,你要好好上进,也不枉江司马的看重。”

董缺恭谨地道:“属下谨尊殿下教诲。”

看着雍王离去,我微笑道:“这下你可放心了,留在雍王府里可以安全无恙,夏,不,董缺,对于那个人你是很了解的,你说他现在最想作些什么?”

董缺神色漠然,但是却十分恭顺地道:“那个人性子是忍耐不住的,十天半月还可以忍住不出去,但是绝对忍不了一个月,他现在最喜欢的就是和有夫之妇私通,只有这样才能满足他寻求刺激的意愿,其实淳嫔虽然美丽,比起他府中的侍妾也未必超过多少,只是妻不如妾,妾不如婢,婢不如偷,偷不如偷不着,所以他才那般沉迷。”

我仔细想了一想,露出一丝带些诡异的笑容道:“你在王府很久,不知道东宫官员和太子的亲信中谁的妻妾最美丽呢?”

董缺神色一动,想了一想道:“翰林学士劭彦之妻霍氏美丽绝伦,半年前太子曾经在佛寺见过她一面,十分动心,可是没有多久他就遇见了淳嫔,劭彦是近年来投靠太子的新锐,为人颇有才华,太子对他也颇为看重。”

我详细的问道:“霍氏人品如何。”

董缺毫不犹豫地道:“太子曾经派我查过,霍氏出身名门,乃是淑女,夫妻和睦,十分恩爱。”

我轻轻叹息了一下,道:“可惜了,这样我就不便出手了。”

董缺微微蹙眉道:“何必可惜一个女子,又不是什么重要的人。”

我淡淡一笑道:“我从不轻易强迫一个人,就是要人去死,也要他死得心甘情愿。”

这时,小顺子的身影出现了,他神色古怪地道:“公子,不知道是不是老天爷帮你,方才吏部奉了圣命将原先东宫的官员黜退,而翰林学士劭彦则擢升东宫侍读。”说着递给我一张名单,上面是新任的东宫官员,我果然看到了劭彦的名字,不由笑道:“这也真是巧极了,我让殿下递了一份密折给陛下,说太子失德东宫官员难辞其纠,应该汰换,原是为了在东宫多安排几个自己人,没想到太子后台果然挺硬,还是安排将自己的亲信安插了进去,只是不知道这个劭彦是不是太子自己选的。”

小顺子微微一笑道:“公子真是一语中的,这是太子昨夜给纪贵妃的名单,我让人抄了一份。”我接过那张绵纸,上面有一些人名,拜在第一位的就是劭彦。

我不由叹息道:“自作孽,不可活,我还没有动手,他就自己忍不住了。”

董缺冷冷道:“现在他未必有这个心思,只是想必看到劭彦便下意识的将他留在身边罢了。”

我看了一眼董缺,笑道:“东宫侍读不是一个普通官职,按照礼法,霍氏已经有了封诰,是要去觐见太子妃的,你说,太子只要无意中见过霍氏几次,他忍得住么?”

董缺默然,半晌才道:“不能。”

\chapter{第二十章 恶孽重重}

武威二十五年七月,太宗履参贪渎事,因而去职下狱者多人,大半乃王亲信也,又,太宗数次觐见雍帝,皆秘而不宣,王乃生疑,与帝嫌隙更深。

——《雍史·戾王列传》

太子李安愤怒地将桌案上的公文扫到地上,又是雍王搞得鬼,这些日子以来不知道雍王发什么疯,居然连连上书参奏官员的不法情事,原本这不关李安的事情,可是雍王这次却是针对李安一系的官员,不仅准备的罪证十分齐全,而且手段如同雷霆,往往一个官员上午还在办公,下午却被一道表章参奏进了天牢,如今满朝文武凛如寒蝉,都担心被牵连进去,毕竟为官者有几个是清廉守正的,甚至有些官员已经偷偷的去向雍王示好,毕竟雍王针对的主要是太子的亲信属下。

鲁敬忠微微皱眉道:“殿下,雍王攻击您是理所当然的事情,如今对他来说,是前所未有的好机会,皇上对您生出嫌隙,他若不趁机进取,也就不是雍王了,但是臣担心的是,从前殿下之所以总是能够压制雍王,主要是因为皇上的支持,如今若是皇上起意废黜殿下,那么殿下失去储位就是朝夕之间的事情了。”

“不错,如今皇上很可能已经改变心意了。”一个悦耳的声音传来,可是李安和鲁敬忠同时皱了皱眉。

房门推开了,走进来的是两个美若仙子的女子,前面的是李寒幽,后面的却是萧兰。

李安恼怒地道道:“孤的书房倒成了不设防的所在了,侍卫呢?”

李寒幽笑道:“殿下勿忧,只不过他们看见兰师姐,因此不敢阻拦罢了。”

李安更是恼火,心道,从前张锦雄做侍卫总管的时候,何曾让人这样子闯进我的书房,因此说道:“靖江,张总管你也应该把他放出来了,这么长时间将他软禁起来干什么?”

李寒幽心中一跳,道:“殿下,您这次出事,夏金逸难辞其纠,张锦雄乃是夏金逸的师兄,家师担心他也有所牵涉,为了稳妥起见只得暂时将他软禁,过段时间,若是他没有什么问题,我们自然会放了他的。”

李安更是不悦,虽然出于推卸责任的目的,他也将自己所犯之错退到夏金逸的身上,可是夏金逸毕竟已经死了,他才会这样做,张锦雄却不同,不仅一向克尽职守,而且李安根本就不相信夏金逸有什么恶意,所以对张锦雄也是爱屋及乌。他刚要说话,鲁敬忠却是轻轻的踢了他一脚,李安立刻醒悟到现在不是争执这些事情的时候。只得按耐怒气道:“不知道你们怎么知道父皇改变了心意呢?”

李寒幽轻轻一叹,坐了下来,道:“这件事情虽然没有明证,可是已经有了蛛丝马迹,殿下可知道,皇上这次赦免殿下并非因为有人保奏,家师原本打算亲自面见陛下,为殿下求情,可是却还没有来得及,殿下就被赦免了。”

李安心中一喜,心道,这样也好,免得我还要承你们的人情。可是鲁敬忠却是眉头一皱,道:“皇上这样很不寻常,说句不当说的话,殿下这次所犯之罪,实在是重大,皇上就算想原谅殿下,也应该是过一段时间消气之后,而且还得有陛下重视的人保奏才行,那时候皇上赦免殿下才是真心实意,现在我们还没有发动,皇上就赦免殿下,果然是有些问题,这是我疏忽了,还请公主明示。”

李寒幽冷笑道:“我从宫中得到消息,皇上在作出决定之前是和长乐公主一起商议的。”

李安大惊道:“怎么会,长乐从来是不参与政事的。”

李寒幽叹了口气道:“我们也这样想,所以虽然我们很希望能够迫使她成婚,但那不过是为了让她和雍王疏远一些,想不到她竟会在这关键时刻给了我们重重一击,虽然没有得到她和皇上说了些什么的情报,可是从目前的情形来看,皇上已经有意废黜殿下,只是缺少一些借口,而且殿下为储君多年,身边不免有些羽翼,皇上几次和雍王密谈,我们的人都没有办法接近,恐怕,皇上真的改变了心意了。”

李安只觉得一盆冰水从头浇下,冰寒彻骨。他从未如此惶恐,他可是很清楚自己是凭着什么才能到了今天的地位,没有皇上的庇佑,自己拿什么去和雍王争,从没有如此后悔勾引淳嫔,李安懊恼的想到,自己是发了什么疯才会去激怒父皇。

鲁敬忠看了一眼李寒幽嘴角的冷笑,心道,你们想趁机要挟殿下,可是还得过了我这关才行,便说道:“殿下不用过于忧虑,现在皇上虽然已经动摇了,可是还没有做下最后的决定,所以殿下还是有机会可以挽回的,凤仪门主她老人家可是和雍王不睦的,若是让雍王当了储君,只怕悔不当初的就是另有他人了。”

李安听得有些糊涂,李寒幽却是立刻把握了鲁敬忠的威胁,鲁敬忠分明是说,如果太子失去储位,那么自己凤仪门也是损失惨重,还是不要趁机要挟的好。她心里虽然恼怒,却也知道这是实情,如今凤仪门和太子已经是一条船上的人了,因此她淡淡一笑,道:“殿下,唯今之际,只有殿下早日登基。”

李安吓得跳了起来,抬眼看去,只见李寒幽、萧兰和鲁敬忠都是一派淡然的神情,他先是想严词拒绝,可是转念一想,如今自己的储位危如累卵,竟然一句反驳的话也说不出来。

李寒幽和萧兰交换了一个眼色,站起身道:“殿下虽然是恪守孝道,可是如今皇上圣聪被小人蒙蔽,若是不幸让雍王登基为帝,那么必然穷兵黩武,大雍从此不得安宁,殿下若是能够下了决心,我们必要拥殿下登基,皇上年事已高,不如好好安养,殿下以为如何?”

李安语气软弱地道:“可是如今我们势力太弱,六弟去了边关,禁军也难以控制,这可怎么办呢?”

李寒幽微微一笑道:“这一点门主已经有了安排,只要殿下首肯,我们凤仪门便要冒险行事,殿下放心,我们必然会小心谨慎,一举功成。”

李安终于吞吞吐吐地道:“你们有什么计划?”

李寒幽得意的一笑道:“殿下放心,我们已经有了妥善的计划,只需要数月时间就可以让殿下继位,不过殿下这些日子可以谨言慎行,以免触怒皇上,若是皇上废黜了殿下,只怕我们也只能黯然收手了。”

李安脸色一红,道:“本王必定谨慎,可是还得小心行事,最好等到齐王回来再说。”

李寒幽淡淡一笑道:“殿下放心,这件事情我们早有准备,齐王殿下最迟十月也能够回来,到时候正是我们发动的最好时机,现在我们也要趁这段时间布局,太子殿下也想把雍王的势力一网打尽吧?”

这时鲁敬忠淡淡道:“凤仪门如此重视殿下的大业,却不知道对殿下有什么要求。”

李寒幽微微一笑,道:“还是鲁少傅明白事理,其实我们要求也不高,若是殿下继位之后,肯立我兰师姐为后,那么我凤仪门必定全心全意为殿下效力。”

李安面有难色地道:“崔氏从无失德,又为我抚育世子,我怎能无故将她贬斥。”

李寒幽道:“殿下,从前您不肯废了崔氏,是因为皇上的缘故,现在皇上已经不再支持您了,您若不肯答应我们的条件,我们又何必冒这样声名尽毁的风险呢,再说我们只是要您立兰师姐为后,可没有说让您一定贬斥崔氏,作个贵妃也还是可以的。”

看着李寒幽嘴角的冷笑,李安心中明白,若是自己答应了这个条件,那么崔氏和世子是绝对没有希望活下去了,他怎能忍心如此。李寒幽看他犹豫,也不强迫,道:“殿下好好想想吧,这件事情并不急迫,您和鲁少傅可以慢慢商量。”说罢,她站起身来告辞道:“臣妾还有事情要做,请殿下仔细想想,事情决定之后,可以告诉我兰师姐。”

看着李寒幽和萧兰的背影,李安气得摔碎了茶杯,气冲冲地道:“少傅,你说,他们哪里把我看在眼里,我若是答应了这个条件,只怕就成了她们手中的傀儡了。”

鲁敬忠神色冷厉,半晌才道:“殿下不必担心,这件事情还有转圜余地,她们漫天开价,殿下也可以落地还钱。”

李安犹豫了一下道:“少傅,如今孤是自身难保,不若就牺牲了崔氏和世子吧,若是雍王继位,她们母子只怕只有一死,咱们和凤仪门商量一下,我可以废黜太子妃,然后将世子封个王位。”

鲁敬忠心中冷笑,太子果然是无情之人,这样就要抛弃妻儿,可是他面上却没有流露出鄙薄的意思,淡淡道:“殿下虽然现在可以牺牲太子妃和世子,博得凤仪门的全力相助,可是若是日后兰妃立了皇后,她的儿子做了太子,那么只要殿下您不合她们的心意,她们就可以杀了殿下,立兰妃之子为帝,到时候殿下只怕会后悔莫及。”

李安苦笑道:“可是我若不答应,只怕现在她们就要弃我而去,我岂不是成了雍王的阶下之囚么?”

鲁敬忠阴险的笑道:“殿下过虑了,现在就是殿下想放弃储位,凤仪门也不愿意呢,雍王摆明了不想和她们合作,如果没有殿下,她们就不能名正言顺的跻身朝堂,所以只要殿下强硬一些,凤仪门绝对不敢和您翻脸的,不如您拒绝此事,就说可以封兰妃为贵妃,而且暂时不立储君,若是兰妃之子才能卓著,可以立他为储,先拖延下去,等到殿下登基之后,也就由不得他们了,毕竟若没有殿下的支持,那些朝中重臣可不会支持凤仪门的谋反呢?”

李安这才眉开眼笑道:“还是你说得不错,那么孤就这样和兰妃说。”

鲁敬忠恭谨地道:“殿下也要去安慰一下太子妃才是,这件事情若是给太子妃知道,只怕会很担心的。”

李安点头道:“你放心吧,对了,东宫的官吏已经都来觐见了吧?”

鲁敬忠笑道:“已经来了,殿下虽然暂时不能理政,可是这些官员还是应该选好,这样也好不致让人看出皇上已经对您生出不满了。”

李安点点头道:“这些事情你处理吧,我去看看太子妃和世子,这段时间,可是让她受了惊吓了。”

李安刚走到后宅,只见一些三两成群的少妇从里面出来,这些女子身边都有侍女仆妇相陪,见到李安,便都行礼叩见,一个王妃身边的侍女上前道:“殿下,这是东宫新选的官员的内眷前来拜见王妃。”

李安点点头道:“原来如此。”也不多说就要去见王妃,可是没走几步,就看见一个身穿诰命服色的少妇容颜秀美,仪态万千,李安不由心中痴了,这个女子他可是认得的,只是当时他迷恋了上淳嫔,所以没有对她动手,这次东宫选官的时候,他看见劭彦的名字便不由自主的画上了,虽然当时未必有明确的想法,可是也有将他们夫妇笼络到身边才好下手的想法,想不到这么快就见到了霍氏,半年多不久,她出脱的更是秀丽,尤其是那种温柔如水一般的气质,让人见了又爱又怜。李安故作镇定的看着这些命妇离去,眼中闪过一丝光芒,若是夏金逸在此,必然立刻知道他的心意,可是现在自然没有人能够帮他安排了。

匆匆的安慰了王妃几句,李安回到书房,此刻这里只剩下他一个人了,他想着自己不应该在这个时候动手,现在正是紧要关头,可是再想想,其实如何安排夺位根本用不到他出力,凤仪门和鲁敬忠会全盘计划的,自己只要照做就行了,这种事情没有人比他们更擅长,那么自己是不是可以散散心,也好弥补一下这些日子的心惊胆战呢。

反复挣扎了很久,李安终于还是忍不住,这些日子又是斋戒,又是软禁,他已经很是苦闷了,回到王府中虽有歌女舞姬,可是他却没有丝毫的兴趣,这一年来的放纵,早就让他对于那些俯首可得的女子失去了兴趣。李安心中揣测着是否会造成麻烦,过了许久,他想起从前淳嫔不也是不愿意,可是自己威胁利诱之后不就屈服了,只要自己许下给他的丈夫加官进爵,害怕这个女子不屈服么,再说,一个官员的妻子,就算父皇知道了也不至于太发怒吧。

第二天,便收到一张太子妃的帖子,邀请她入府一会,霍氏倒也不觉的奇怪,昨日到太子府上觐见太子妃,就觉得太子妃心情不是很好,据说是因为除了太子的事情之外,她的一个心爱侍女死了,太子妃对霍氏十分亲热,而且很赞赏霍氏送给太子妃的绣品,所以霍氏倒也不觉得奇怪,何况自己的丈夫是东宫侍读,自己若是能够得到太子妃宠爱,那么对于丈夫总是没有坏处的。所以霍氏欣然而往。

在几名宫女的引领下,霍氏被带进了一间雅致的楼阁,她心中有些奇怪,怎么这里不像是王妃的寝宫,虽然雅致,却少了几分气势,一走进花厅,霍氏顿时吓得惊叫出来,这是一间十分华丽的私室,地上铺着厚厚的毯子,四周都是华丽的陈设,房间一角摆着一张宽大的床榻,四周罩着粉红色的纱帐,而四周墙壁上却都悬挂着精美*,霍氏只是看了一眼便不敢再看,她心中充满了恐惧,正要退出去,却见房门处站着一个一个男子,霍氏认出了那人正是太子殿下,心中一凛,对于太子的事情,虽然还没有沸沸扬扬,可是她还是知道一些风声的,她强忍着恐惧道:“臣妾误闯此地,还请殿下见谅。”

李安暧昧的一笑道:“是我派人请你来的,怎会不见谅呢。”

霍氏大惊道:“殿下,这于礼不合。”说着就要向外走去。却被李安一把抱住,李安练过武功,轻而易举的将她拦腰抱住,霍氏还要挣扎,李安突然恶狠狠地道:“你信不信我立刻派人去杀了你的丈夫。”霍氏手一软,眼中闪过惊惧悲哀的神色。李安已经冷冷道:“你若是顺从孤,那么你的丈夫就能够青云直上。”

霍氏心神已经失守,李安趁机将她拖到床榻之上,粉红罗帐垂下,从里面传来了低低的哭泣声。

第二天午后,当霍氏上了轿子返回家中的时候,一双眼睛趁着霍氏出入的短暂时刻将她打量的清清楚楚,眼中闪过一丝无情的光芒。

不久之后,董缺已经回到了寒园,将掩盖身份的伪装卸下之后,冷冷道:“太子已经得手了。”

我轻轻摇动着折扇,道:“可以肯定么?”

董缺露出一丝笑容道:“这种事情没人比我更加清楚,这个女子绝对是被太子尽情蹂躏过了。”

我笑道:“这点我自然是相信你的判断的,你说霍氏会怎么样。”

董缺露出一丝同情的神色道:“按照太子的习惯,暂时是不会厌倦的,所以霍氏就要想要自杀也不可能,我看到她的神色,欲哭无泪,但是却没有死志,我想她暂时不会寻短见的,而且这个霍氏恐怕不是威武不能屈的女子。”

我淡淡道:“你说她会告诉丈夫么?”

董缺摇头道:“这种事情,短期之内她是不会说的,而且劭翰林是个有些古板的读书人,很难原谅这件事情,我想,她不会这么愚蠢的。”

我微微一叹,道:“其实我是可以告诉这个女子小心的。”

董缺冷冷道:“公子,这种慈悲心可是没有用的,就是你提醒了他们夫妻,他们也只会当你构陷太子,还会打草惊蛇。”

我苦笑道:“这道理我也清楚,所以我冷眼旁观这场悲剧。董缺,我现在真的觉得从前给你的任务太残酷了。”

董缺默然良久道:“这是我自愿的。做出这种事情的是太子,和我们有什么关系?”

\chapter{第二十一章 局势突变}

我微蹙双眉,看着眼前的战报,这是雍王的情报网传回来的消息,正式的军报还要等几日才能到达。

“七月十六日,齐王巡边至镇州,北汉军叩关,齐王领军出战,初战告捷,七月二十一日,飞虎将军石英兵至,齐王坚守不出,待石英兵退,王出关击敌,遭鬼面将军谭忌伏击,败退。七月二十六日,石英叩关,王示弱于先,诱使敌军一部攻入城池,聚歼之。八月三日,两军战于城关,凤仪门凌羽伪装成敌将侍卫,暴起刺杀谭忌,谭忌重伤,北汉败退。八月十四日,证实北汉已经收兵,齐王上书报捷。”

我放下情报,忧心忡忡地道:“想不到齐王殿下如此之快的就稳定了边关,看来很快他就会回来了。”

雍王和昨日刚回到长安的石彧石子攸对视了一眼,石彧说道:“殿下可以上折要求齐王暂时不可回京,随云为何这样忧虑。”

我叹息道:“齐王这样快就平定了边关局势,凤仪门用了很多心思啊,军中刺杀大将,是何等危险的事情,如今两国敌对,不似从前中原争霸,只要将领一死,兵士多半投降,两军将士皆有效死之心,大将受伤,必然是拼死攻击,凤仪门这个弟子纵然能够逃生,恐怕也是九死一生,凤仪门已经是迫切的需要齐王回来参与兵变了。”

石彧蹙眉道:“随云是说,如果殿下阻止齐王回来,她们会铤而走险。”

我苦笑道:“若是她们铤而走险也就罢了,问题是怕她们怀疑皇上目前根本就没有下定决心废黜太子,有一件事情我们双方都清楚,齐王虽然是太子殿下的支持者,可是如果不是皇上心意如此,齐王是不会铁了心支持太子的,齐王,从某种意义上说,更加是一个忠臣,这也是这次皇上去黄陵带着齐王护驾的一个原因。虽然没有齐王对我们更方便,可是如果我们得到了皇上的支持,那么齐王就不会给我们造成太大的麻烦,所以如果我们坚决阻止齐王回京,凤仪门主是绝对不会想不到这件事情的。”

雍王眉头深锁道:“本王预计,数日之内,齐王就会轻骑回京,若是我们不能阻拦,那么京中局势就会不可收拾,这样一来只怕军方会损失惨重。”

我又拿起一张纸道:“还有一件事情,叶天秀通过殿下的渠道,给庆王殿下报了平安,可是庆王殿下大怒,已经派了一些属下前来京城。”

雍王叹了口气道:“真是麻烦,庆王总是这样冲动,如果当初不是因为那样冲动,怎会被贬斥到东川。”

我淡淡一笑道:“以臣看来,庆王倒是聪明人,若是留在京中,凤仪门必然处处为难,还不如远走高飞,为一方诸侯镇守的好。”

雍王和石彧相视一眼,露出一丝尴尬和歉意,我心中一动,道:“可是有什么哲不了解的地方。”

雍王看了一眼石彧,石彧苦笑道:“有一件事情,殿下和我早有怀疑,庆王的武功有些近似魔宗的路数。”

我心中一震,道:“难道是北汉魔宗。”

雍王答道:“事实上,魔门并非是一个整体,据说京无极离开中原的时候,有很多魔门弟子脱离了魔宗滞留在中原,他们擅长隐匿,而且我们也不想过于逼迫魔宗,免得激怒京无极。”

我淡淡一笑道:“这也是皇上和殿下不敢信任庆王的缘故吧。”

雍王苦笑道:“正是如此,本王不敢确定他为了复仇可以做到什么地步。”

我疑惑地道:“若是如此,东川可是要地,皇上和殿下不担心么?”

雍王微微一笑道:“庆王若是不想谋反,在东川自然是可以为所欲为,可是若是有了反意……”

雍王含笑不语,我也识趣的不再多问,看来庆王身边有人监视控制,这大概是皇室内部也很少有人知道的秘密了,可是问题还是存在的,我问道:“殿下可否劝阻庆王来京呢?”

雍王想了一想,道:“本王写封书信,就让苟廉兼程拦阻,他定有法子说服庆王不要进京。”

我叹息道:“可惜齐王没有那么好打发。”

过了数日,果然在太子和一些大臣的建议下,齐王奉诏回京述职,这已经是意料中的事情,所以雍王也没有阻止,不过雍王殿下已经下了决心即使引起凤仪门主的怀疑也要阻止齐王进京,我心中已经在盘算一场刺杀,这样一来,可以让齐王暂时不能介入皇储之争,即使有些嫌疑也说不得了,总比让齐王的大军和雍王的军队开战好的多。

正在我和雍王、石彧商量如何安排刺杀的时候,一名侍卫却进来禀报道:“殿下,齐王遣来密使求见殿下。”我们听了都是一愣,齐王怎会派遣密使来见,无论如何,齐王的使者不能不见,雍王就在书房接见,石彧和我左右侍立。

不过片刻,一个骁勇的齐王亲卫走了进来,见礼之后,递上一封书信,雍王看后神色一动,将书信递给了我,我一看,却是齐王邀请雍王殿下在八月二日,在距离长安百里之遥的一处佛寺密会。雍王淡淡道:“请转告齐王,本王必定与会。”

信使走后,石彧犹疑地道:“齐王殿下的举动太不寻常了,殿下真的要去么?”

雍王道:“若有机会让六弟改变立场,本王冒些险也是值得的。”

我却一摇折扇道:“殿下,齐王性子不是知难而退的人,恐怕他不会改变立场,不过这倒是一个好机会,若是太子的人知道殿下和齐王私会,那么他们就不敢完全信任齐王了,那么至少可以减轻我们的压力。”

雍王犹豫了一下,道:“这离间之策用是用得,可是我担心六弟会怀恨本王。”

我笑道:“齐王本来就是和殿下作对的,就是多些恨意也没有什么,倒是太子和齐王本就有了嫌隙,这样一来,正是损人利己的好法子。”

雍王意动道:“可是要秘密将消息传出去给太子知道。”

我淡淡一笑,道:“凭着凤仪门的本事,只要殿下故意装作十分谨慎小心,是一定会有人监视的,到时候我们正可以让她们远远看见,因为不能得知事情,到时候自然是往坏处想了。”

雍王淡淡一笑道:“若是能够让六弟置身事外,那么就什么都值得了,六弟乃是将才啊。”

九月二日,黄昏,雍王轻车简从的离开了长安,随行的除了百多名先后出城会合的侍卫之外,还有我和小顺子,我坚持随行实在是有些好奇齐王的用意,而且临机应变也需要我的决断,至于小顺子,要是没有他保护,我怎么能放心这样的远行呢,这种情况下,除了凤仪门主亲自出手,我相信可以确保安全了。

齐王指定的约会地点是一个十分荒凉的破庙,已经没有人主持,我们到的时候已经是天明时分,齐王的近卫已经将这里打理的干干净净,四周戒备森严,却是人人便装,除了停在庙旁边的一辆马车之外,毫无引人注意之处。

雍王的近卫赶到之后迅速布下防线,双方带着敌意的对峙,恰好形成一种力量的平衡,将这里围得水泄不通,我看了小顺子一眼,他会意的站到可以将小庙全部收到眼底的位置,这样一来,可以不让有人侵入到可以见到庙中情景的位置。我则随着雍王走进小庙。已经打扫的纤尘不染的庙堂之内,破旧的佛像之前,一个锦衣男子负手而立,傲然仰首,注视着佛像。我停住了脚步,现在的齐王和我从前所见的又不相同,四年之前,南楚的第一次见面,他是霸气凌云的大雍亲王,浑身杀气,令人退避三舍,第二次见面,他身中毒伤,奄奄一息,可是却显露出他豪爽的一面,第三次雍都相见,他是一片热诚,若非有些感动,我怎会准备借他脱身。此后的日子,我在雍王府和太子一方斗得风起云生,齐王虽然是太子一党,可却是备受压抑,不能回到军中,纵然是嚣张霸道,也难脱几分失意,从前的霸气渐渐消退,今日一见,或许是边关大战的洗礼,已至而立之年的齐王殿下也有了一种含蓄雍容的霸气,有些酷似当年的雍王。

我在这里胡思乱想,雍王已经上前道:“六弟,我来了,不知道你有什么想对我说的?”

李显缓缓转过身来,面上露出淡淡的微笑道:“二哥如今已经是箭在弦上不得不发了,是么?”

李贽神情一凛,没有说话。

李显背过身去,道:“这九五之位谁不想要,如今大哥自己作孽,现在又是疑神疑鬼,看来这皇位迟早是二哥你的了。”

李贽缓缓道:“若是你肯真心相从,我待你还如从前一般。”

“从前一般?”李显哈哈大笑道:“从前我少年时候仰慕于你,进入军旅,若无二哥教导,只怕我没有几日,可是我总是想摆脱二哥的阴影,所以我没有紧跟在二哥后来,而是成了今日的齐王,可是二哥,我或许放荡,或许愚蠢,可是我不是朝秦暮楚之人,既然我扶保了太子,那么就是死也不会背叛。”

李贽压抑不住怒气道:“若是大哥阴谋叛乱,你也要跟着他胡作非为么?”

李显神色一愕,转而恍然道:“原来如此,二哥你是想迫使大哥叛乱,这样你才可以名正言顺的成为储君。”

李贽叹了一口气道:“六弟你一向聪明过人,我是很佩服的,可是你为什么不肯稍做掩饰呢,如今箭在弦上,只怕我不能让你进京了。”

齐王却是淡淡一笑道:“二哥放心,我不是蠢人,如今的局势我很清楚,你要做什么,我不会阻止,也不会告诉大哥,但是除非大哥真要犯上作乱,否则我是不会背叛他的,若是二哥不信,外面的马车里有一个人,二哥见了他就会相信我不会将今日之事说出去的。”

雍王神色一动,看了我一眼,始终沉默不语的我退了出去,走到马车前面,掀开车帘,只见车内一坐一卧乃是两个人,坐着的是一个五十多岁的中年人,神色恭谨,而躺在那里的是一个相貌清秀的少年,肤色微黑,虽然在昏睡之中,却是神色不安。中年人低声道:“这是我家少主姜海涛。”我呆了一下,笑道:“在下江哲。”

那个中年人欣喜地道:“您就是江大人,方将军带回您的药,我家少主伤情已经稳定多了。”

我宽慰道:“阁下放心,现在雍王殿下还在里面等在下回报,请阁下稍侯。”

回到庙中,我敬佩的看了一眼齐王,走进雍王身边低声道:“是姜侯爷之子。”雍王神色大变,惊讶的看着李显,李显神色冷傲,却是一言不发。雍王神色变得柔和,道:“你可知这件事情你既然已经插了手,那么就有了把柄在我手中,这件事情传出去我还罢了,太子和凤仪门可是不会放过你的。”

李显冷淡地道:“我不管他们怎么想,这个孩子叫我一声表叔,我若束手旁观,也未免太无情了,不知道二哥有没有这个胆子接手这件事情。”

雍王突然轻施一礼道:“六弟你的侠义之心本王自愧不如,你放心,既然这个孩子已经到了长安,那么我定然会尽力而为。”

李显转过身去道:“好了,你带走他吧,等到他毒伤痊愈之后,你若不方便将他送回去,就来告诉我。”

李贽深深的看了一眼齐王,道:“你真的不肯改变主意么?你可知一旦大局已定,你我就是生死相见的结局。”

李显微微一笑,笑中满是嘲讽,冷冷道:“多谢二哥美意,就是我投了你,你当真信得过我么?”

雍王一滞,说不出话来,他是很想说信得过齐王,可是想到齐王多年来和自己屡次作对,想到齐王妃秦铮,终于软弱地道:“我相信六弟会有法子表示自己的诚意。”

李显侧过身去,低声道:“铮儿虽然有不好之处,可是她总是我的妻子,我孩儿的母亲,李显不才,也不能杀妻以求富贵。”

李贽深深叹了一口气,道:“那么六弟你好自为之。”说罢转身走出了庙堂。我看了齐王一眼,行礼道:“从前哲只道殿下豪爽,今日才知您敢作敢为,还请殿下今后小心行事,太子昏庸,凤仪门野心勃勃,殿下何必为她们殉葬。”

李显看了我一眼,淡淡道:“随云之才天下无双,若是我当日狠心杀却,只怕就不会今日的下场。”

听到这里,我心中有些凄然,只听这句话就知道这个高傲的亲王已经放弃了掌控命运的机会,情愿灭顶在这场血腥的夺嫡之战。可是我却无能为力,到了这种时候,无论齐王是怎样的人,雍王和我都不可能放手了。若非是今日相见已经可以离间齐王和太子,我是绝对不会让齐王返回长安的。

告辞离开,上了马车,雍王已经是面如寒霜,马车启动,他没有说话,良久,才淡淡道:“齐王可惜了。”

我心知李贽已经动了杀机,可是也听得出他语气中的深深遗憾,这是前几日我们商量刺杀的时候所没有的,便说道:“殿下放心,齐王看来是不会随着太子谋反了,至少殿下不用担心齐王的大军会发难了。”

雍王摇头道:“不怕一万,只怕万一,若是不能确实的控制住老六,本王是绝对不能放心的,随云可有什么什么法子。”

我心中转了千百个念头,终于遗憾的摇头道:“除非杀了齐王,臣也没有办法可以控制住他。”

雍王轻轻一叹,不再说话,我这才又道:“除非是让齐王暂时生场重病,没有齐王亲自指挥的军队就如同没有首领的群狼。”

雍王神色一动,道:“先看一看,不过要做好准备,总不能临阵磨枪。”

我微微一笑,道:“就不知道凤仪门会怎么想了?”

雍王的车驾远去之后,齐王的近卫首领走了进来,禀报道:“殿下,我们也该走了,若是这件事情给太子知道,恐怕太子是要生疑的。”

李显点头道:“这也顾不得了,我已经尽了臣子和兄弟的情义,若是太子生疑,我也是无可奈何。”

那个近卫突然道:“殿下,属下不敢置疑殿下的决定,可是那个人真的值得您如此忠心么?”

齐王面色一寒,道:“这不是你该说的话。”

那个近卫神色惶恐,但是倔强的眼神却是丝毫没有改变,李显看了他一眼,叹息道:“太子本性显露,我也是十分失望,可是如今我已经是骑虎难下,纵然是他无情,我不能无义,无论如何,从前没有太子提携,我绝对没有今天的成就。”

就在齐王的车驾出发之后不久,从不远处的一座小土山之上,站起一个布衣女子,虽然是荆钗布裙,却是明艳不可方物,她望着齐王的身影,冷冷一笑,可是虽然是如此寒冷的微笑,在她那张如同初升朝阳一般灿烂耀眼的花容之上,却是显得那样动人。

\chapter{第二十二章 风仪之谋}

栖霞庵中,当那个明艳无双的女子将自己所见向凤仪门主禀明的时候,凤仪门主淡淡道:“齐王难以控制,这也是本座很早就知道的,若非他没有继位的可能,本座也不会放任他任性妄为,只是如今,他竟然在这个时候和雍王密会,无双,你说齐王会不会在这个时候投向雍王。”

燕无双犹豫了一下道:“以弟子之见,齐王应该不会完全投向雍王,没有一个背叛者能够得到真正的重视和信任,齐王就是此刻投降雍王,也不过能够在雍王得胜之后保住性命罢了,若是齐王扶保太子登基,那么日后就是一人之下,万人之上,这个道理齐王不会不明白的,师尊,要不要让铮师姐问清楚齐王殿下的心意。”

凤仪门主摇头道:“不可打草惊蛇,唉,秦铮真是我亲传弟子中最没用的一个,不仅无力约束齐王,更是将自己的心也丢了,当初我教导你们的最重要的一点,就是不能动了真情,若论聪明机智,才略野心,我们女子不比他们男子差到那里去,唯一的缺点便是我们太容易为了那些廉价的情感而迷失自我。”

燕无双道:“师尊过虑了,秦师姐虽然无力约束齐王,这也是因为齐王个性独特,身份尊贵,若是师尊下了决心,秦师姐必然能够遵令行事,控制住齐王。”

凤仪门主神色欣然道:“无双果然是聪明过人,这次羽儿行刺鬼面将军成功,你又探得如此重要情报,本座心中十分欣慰,你们要好好做事,让本座看看你们的努力。”

燕无双裣衽道:“弟子必定不负师尊厚望。”

犹豫了一下,燕无双又道:“师尊,这件事情要禀报太子殿下么?”

凤仪门主冷笑道:“禀报他做什么,让他对齐王也心生怀疑,如今太子殿下只怕是已经草木皆惊,就不要打击他了,何况,若是我们握住了这个把柄,等到日后太子登基之后,我们也可以更好的控制齐王,想来他也会知道如果太子知道他曾经有心叛变的事情,会对他作出什么的。”

燕无双崇敬地道:“门主谋略深远,弟子钦服,不过弟子有一件事情不明白,长乐公主与朝政并无关系,门主为何要执意逼迫她呢,若是因此引起雍帝不满,只怕是得不偿失。”

凤仪门主微微一叹道:“这件事情你日后会明白的,可是有一个原因你要知道,长乐公主的心上人是谁,那人虽然才智无双,可是这样的人都是心灵脆弱之人,我们都知道他曾经因为攻蜀之事而心力交瘁,休养数年,本座派人去南楚查过,证实那时他的确情况危急,有名医的诊断,说他心经受伤,濒死多日。上次雍王府本座特意留心,他却是心脉受伤极重。这一年多来,他和长乐公主暗通款曲,必然是已经有了极深的感情,若是这个时候,长乐公主别嫁,以他的身体,必然会因此卧床不起,甚至危及生命也是可能的,这样对我们会有多少帮助你应该很清楚。”

燕无双眼中闪过惊叹道:“此人一入雍王幕府,我们便处处不顺,如今又不能再次刺杀,若是能够这样铲除了此人,真是值得冒险。”

凤仪门主淡淡笑道:“其实这对长乐公主也不错,韦膺人品相貌都很不错,能够嫁到这样的佳婿,对她来说已经是很好的结局了,何必还要眷恋一个病弱短命之人呢。”

燕无双忧心地道:“听闻长乐公主外柔内刚,不知道门主如何施为?”

凤仪门主微微一笑,没有回答。

这时,门外传来清越的声音道:“启禀门主,齐王殿下已经入京,太子率百官郊迎。”

在隆重的郊迎仪式之后,李显被太子邀请同车进宫觐见皇上,这个邀请一出口,李显就是心中冷笑,他不是白痴,太子从前对自己虽然无可奈何,可是除非有了用自己之处,他才会这样礼遇,看来二哥说得不错,太子已经是迫不及待了,要不要说明雍王应该是虚张声势呢?想了一想,李显决定,如果太子诚心询问自己,那么自己便丝毫不会隐瞒。如果他只想利用自己的力量,那么,自己就一言不发吧,只要步迫使自己起兵谋反,那么也就唯唯听命吧。

太子车驾之上,李安犹豫片刻,道:“六弟,你也知道现在的情况,雍王步步紧逼,父皇暗昧不明,我的储位已经是岌岌可危,六弟,你一向是我的人,如果我失去储位,就是雍王看在父皇的面子上不加罪你,你也别再想带兵出征,到时候不是被软禁起来,也会被免去职务,到那时你恐怕悔之晚矣。”

李显神色一黯,他又何尝不知道这个道理,可是无论如何自己总不能起兵谋反,那样岂不是辜负了父皇的信任。

李安又道:“六弟,我也不多说什么,你应该明白如今你已经无路可退,若是我能够登基,必然封你为辅政亲王,到时候你就是一人之下,六弟,你意下如何。”

李显冷冷道:“那么大哥又把凤仪门放在什么地方,她们全力相助是为了什么,大哥应该心知肚明。”

李安面上露出尴尬之色,道:“她们自然是有些要求,不过我可以处理的,六弟,你我乃是兄弟,将来我们联手,总有法子限制凤仪门的。”

李显轻轻叹了一口气道:“臣弟知道了,殿下可以放心,只要父皇心意不变,臣弟绝对不容许有人伤害到殿下。”

李安皱皱眉,这并不是他希望的结果,他更希望李显能够提出助他谋反,可是这种事情是不能由他开口的,一旦说出口,必然后患无穷,犹豫片刻,看到李显冷淡的神情,他终于不愿意再冒险,现在,他已经不能肆意妄为了。

齐王被雍帝召见之后,走出皇宫的时候,看见一辆华丽的马车等在那里,他犹豫了一下,旁边他的亲信近卫低声道:“殿下,王妃亲来迎接,您若是不见,未免有些过分,还是敷衍一下吧。”

李显想了一想,走到马车前,车内一个侍女挑起了车帘,露出齐王妃如花笑黡,李显神色带着几分嘲弄和油滑,笑道:“原来是铮儿亲自来迎接本王凯旋,真是荣幸之至。”

秦铮面上一红道:“王爷总是这么没有正经。”李显一笑,纵身上了马车,车帘垂下,掩盖住了车厢内的笑语*。

李安却是沉着脸回到了府邸,将谈话结果告知鲁敬忠之后,只是匆匆说道:“这件事情就交给你处理了。”便回了内宅。片刻之后,一张太子妃邀请东宫侍读劭夫人霍氏的帖子送了出去,过了半个时辰,容颜惨淡的霍氏走进了太子府邸中专供太子淫乐的密室。在太子发泄情绪的狂暴中,流淌着无辜女子的血泪。

就在当日午后,凤仪门主进宫与皇后娘娘相会,不久之后,窦皇后派遣女官赵尚宫前去传诏长乐公主。

长乐公主秀眉微蹙,看着面前传旨的赵尚宫,皇后娘娘传懿旨让自己前去觐见,这不是什么好兆头。而且还是尚宫亲来,按照大雍内宫的制度,除了皇后和贵妃身边的首席女官为尚宫之外,其余各殿的首席女官皆为尚仪,这些女官大半都是年纪较长的宫女,就像自己的翠鸾殿的周尚仪,乃是母妃从前的亲信侍女,今年已经三十岁了,不论是尚宫尚仪,都是地位极高,这种传懿旨的事情,是用不着这位后宫女官之首的赵尚宫亲自来作的,而且赵尚宫嘴很紧,只说皇后娘娘有请公主,什么事情却是不肯言明。长乐虽然心中忧虑,但是转念一想,兵来将挡,水来土掩,自己是深得帝宠的公主,就是皇后也不能将自己如何的,因此,她的面上露出坚毅之色,微笑道:“请赵尚宫前面带路。”

赵尚宫引领着长乐公主东转西绕到了一间畅轩,里面陈设精美,棋坪瑶琴不一而足,皇后娘娘窦氏正和一个雪衣蒙面女子下棋,见长乐公主走了进来,便一推棋坪道:“罢了,本宫认输了。长乐过来,拜见凤仪门主。”

长乐公主心中一跳,上前拜倒道:“长乐叩见母后,参见门主。”

那个雪衣女子一双清澈冰寒的明眸透出淡淡的不明情绪,上前将长乐公主搀起,笑道:“上次见你,还是一个小娃娃,如今已经是婷婷玉立了。”

皇后叹息道:“只是这个孩子命苦,从前被她父皇遣嫁南楚,如今又是孀居在家。”

雪衣女子笑道:“长乐端庄娴雅,怎会长久独居,听闻皇上已经为你择婿,不久之后定然可以琴瑟和谐,相敬如宾。”

不容长乐公主说话,窦氏已经笑道:“她父皇给她选的驸马乃是韦相之子,虽然没有说明迎娶时间,可是这件事情总不好拖的太久。长乐,你说是么?”

长乐公主虽然早有准备,仍然是心中一寒,紧紧的握住了常年不离的折扇,似乎那人正在旁边支持自己,微笑道:“母后多虑了,长乐如今有佛祖相伴,正是心如止水,还请娘娘不用费心,这婚事长乐已经和父皇推辞过了。”

皇后有些犹豫,看了凤仪门主一眼,凤仪门主嘉勉道:“长乐说得不错,咱们女子也未必要有夫婿相陪,皇后娘娘也是怜惜你青春年少,你还是要好好考虑一下,你这把扇子倒也雅致,可否给本座看看。”

长乐心中一紧,却只得将折扇递过去道:“请门主赏鉴。”

凤仪门主接过折扇,看了看上面的诗文,轻轻念道:“冷于陂水淡于秋,远陌初穷见渡头。赖是丹青无画处,画成应遣一生愁。好诗,不愧是南楚第一才子。”说罢,用充满寒意的目光望向长乐公主,道:“公主是真得不愿意成婚么?”长乐公主只觉得呼吸急促,仿佛有泰山一般的压力扑面而来,她虽然素来柔弱,但是性子却是外柔内刚,凤仪门主又碍于她的身份,只是用了气势相凌,所以她居然能够忍耐得住。凤仪门主那清冷的声音在她耳边响起道:“公主,韦膺也是皇上为你苦心挑选的夫婿,你若是顺应天意人心,不仅自己一生幸福美满,也免得你的父皇母妃为你担忧。”长乐公主只觉的心神恍惚,那种强大的压力几乎要逼得她开口答应了,可是她的脑海中很快就浮现出那个苍白文弱的青衣书生的形象,目光落到折扇之上,她颤抖着声音道:“多谢门主关爱,长乐如今并无再嫁之心,韦膺随好,却非良人。”

凤仪门主长眉轻扬,轻轻摇动折扇道:“公主如此拒绝皇上和皇后的美意,想必是其意已坚,本座也不便相劝。”说着突然素手用力,那柄精工制作的折扇竟然化成齑粉。

长乐公主一声惊叫,美目之中泪影涟涟。凤仪门主歉疚地道:“本座一时失手,毁了你的折扇,这样吧,本座赔偿一把好的给你。”

长乐公主只觉的心中有一团火焰在燃烧,怒冲冲道:“不必了,不过是一把折扇,门主不必自责。”虽然是这样说着,可是她的明眸之中投射出刻骨铭心的恨意,就是凤仪门主也觉得心中一寒。

这时,窦皇后开口道:“长乐你身子不好,见你面色苍白,想必也累了,还是早些回去休息吧。”

长乐强忍着心中悲愤,告退如仪,只是脚步有些踉跄,刚才站在远处的绿娥对这一切却是没有丝毫察觉,只是觉得公主神情不好,连忙搀扶她返回寝宫,刚走了不久,突然远处传来一个惊喜的声音道:“殿下,怎么你也在这里?”

长乐疲倦的抬眼望去,却是韦膺和一个小太监站在那里。若是从前长乐定然会借故离开,可是现在她却是几乎不能思考,有些怔忡地问道:“韦大人怎会在此?”

韦膺容色隐隐带着欣喜道:“臣已经进了中书省,在皇上身边侍奉,方才皇上得知凤仪门主驾到,特意派臣前来禀报娘娘,请门主多留一会儿,皇上想请门主晚膳。”

长乐听到凤仪门主四个字只觉得心中怒火燃烧,正要离去,却只觉得头晕目眩,娇躯软倒在地。绿娥惊叫一声,她力气不大,虽然勉强搀住了公主,却是力不从心,这次前来觐见皇后,长乐公主本来就没有多带宫女,这里又不知因为什么缘故,竟然没有宫女内宦,唯一一个小太监又是年纪幼小,根本不可能搀扶公主,无奈之下,绿娥只得抬目向韦膺望去,虽然韦膺乃是男子,但他毕竟是公主的“未婚夫”,虽然绿娥知道公主另有所爱,可是总不能让公主这样昏迷倒地吧。

韦膺略一迟疑,急步上前伸手相搀,道:“附近可有房间,让公主在那里休息一下,也好召太医来诊脉。”

绿娥喜道:“多谢韦大人提醒,这里是御花园西侧,旁边是端妃娘娘的寝宫,拜托大人相助将公主送到那里。”

韦膺将公主抱起,道:“那么就请绿娥姑娘带路。”

绿娥对那个小太监道:“你快些去禀报长孙贵妃,就说公主忽然晕倒了,请娘娘到端妃娘娘寝宫来接公主。”

小太监连连答应,转身跑开了。韦膺抱着长乐公主跟在绿娥的后面,绿娥虽然匆匆走着,却始终留意身后,只见韦膺眼中闪过又怜又爱的神色,也不由心生同情,心想若是公主因此改变心意,倒也不错。

没走多远,绿娥可能是走的太匆忙,不小心一跤跌倒,不由捂住脚踝痛呼起来,韦膺焦急地道:“绿娥姑娘,你怎么了?”

绿娥苦笑道:“韦大人,奴婢怕是走不动了。”

韦膺高呼道:“可有人在附近么?”

绿娥也喊了两声,可是最后绿娥只能无奈地道:“韦大人,劳烦你顺着这条路向前不远,就是端妃娘娘的住处。”韦膺犹豫地道:“后宫之中,我多有不便。”

绿娥急道:“这都什么时候了,您若还要顾虑,只怕公主病情加重。再说,您和公主尚有婚约,应该无妨的。”

韦膺只得道:“绿姑娘请在这里稍等,我这就让人来援救姑娘。”说罢,继续沿着小路向前,不一会儿,韦膺有些糊涂了,前面竟然出现两条道路,自己该走哪一条呢,想了一想,他沿着左手那条小路向前走去。又过了片刻,前面出现一间宫殿。他欣喜得走上前,敲开宫门,却只有一个老太监出来迎接,他惊慌地道:“这位大人怎会到此。”

韦膺苦笑道:“我是韦膺,长乐公主突然昏倒,我想送她到端妃娘娘宫中,没想到却走错了路。”

那个太监诚惶诚恐地道:“这里久已无人居住,请韦大人先送公主进来休息,老奴这就去叫人。”

韦膺只得道:“烦劳你了,麻烦你去找人过来照料公主。”

那个老太监离开之后,寂静的宫殿之内只有韦膺和长乐公主两人还在,看着躺在床榻之上,容颜苍白的丽人,韦膺心中波澜顿起,他本是名门之子,又是天资聪颖,得人敬重,可是长乐公主却是固执的拒绝了他,想到这里,他心中不由生出怒气,可是目光一落到长乐公主身上,却是变的温柔和煦,虽然遗憾,可是长乐公主却是让他心中敬佩的女子。

紧闭的殿门让寝殿之内光线幽暗,不由令人生出暧昧的感觉。韦膺只觉得心绪加快,寝殿一角,香炉之内焚烧的香料气味越来越浓厚,韦膺心中越来越觉得按耐不住,看向长乐公主的目光也多了几分晦暗不明。

\chapter{第二十三章 孰不可忍}

终于,韦膺走向长乐公主,刚刚走到公主身边,突然外边传来急促的脚步声,韦膺一惊,连忙退到一边,这时,殿门被大力推开,长孙贵妃带着十几个宫女内宦闯了进来,看到殿中情景,长孙贵妃眼中闪过怒色,她也不说话,只是一挥手,一个太监走上前将殿角的香炉盖上,几个宫女走到榻前,将长乐公主扶起,然后一顶宫中使用的软轿抬了进来,宫女们将公主扶到轿中,迅速抬走。韦膺一脸的迷糊,上前道:“娘娘终于来了,那位小公公已经禀告娘娘公主昏倒的事情了么,娘娘可是看到绿娥姑娘才会想到臣可能走错了路途。”

长孙贵妃露出疑惑的神色道:“本宫得到通报,说是长乐遇到危险,因此急急赶来,想不到却是韦大人不顾嫌疑,和长乐独处殿中,正要责问于你,你如此说是什么意思。”

韦膺坦然将事情讲了一遍,长孙贵妃面色数变,终于道:“原来如此,韦大人也是一片好心,只是长乐乃是孀居,多有不便,大人理应避嫌才是,周尚仪,你去把绿娥带回翠鸾殿,韦大人还有旨意在身,还是快去办事吧。”说罢长孙贵妃就要转身离去。韦膺连忙道:“不知道臣是否可以前去问安?”

长孙贵妃略一犹豫,可是想起哪炉宫中秘制有催情作用的熏香,终于冷冷道:“不必了,大人是外臣,理应避嫌。”

望着远去的长孙贵妃,韦膺只觉得浑身一片冰冷,他知道,他已经失去了梦寐以求的佳人。

回到翠鸾殿,招了太医前来诊脉,说是公主急怒攻心,再加上身子虚弱,才会晕倒,长孙贵妃虽然有些奇怪,毕竟这一年多来,长乐公主身子还是不错的,但是总算没有大碍就好,可是她心中却将窦皇后恨透了,自己好好的女儿,被她召去之后竟然成了这副模样,怎不叫她心痛难忍。可是这口气却是出不得的,人家是皇后,太子又是她的亲生儿子,自己有什么法子呢。越想越是恼怒,这时,看到绿娥被周尚仪带了回来,她大怒道:“绿娥,本宫如此信任你,让你亲自照顾公主起居,虽然因为你年纪轻,没有让你做尚仪,可是本宫自问待你不薄,你为何恩将仇报,构陷公主。”

绿娥连叫冤枉,争辩道:“奴婢并没有此心,娘娘明察,实在是情况危急,韦大人也是皇上认可的驸马,奴婢实在是没有构陷公主的意思。”

长孙贵妃怒道:“你还敢狡辩,不论韦膺是何等身份,你跟着公主这么长时间,还不知道公主的心意,若是今日本宫晚去一步,只怕长乐名节受损,就是心不甘情不愿,也只能嫁给韦膺,无论本宫和皇上如何心意,总是要长乐自己许可才行,你这贱婢,肆意妄为,若是损了长乐名节,就是你死上一千次,也难辞其咎,周尚仪,给我将这贱婢带下去重重的打。”

几个太监将哭喊的绿娥拖了下去,周尚仪下去执刑,长孙贵妃疲倦的坐下,看看身边的田尚宫,道:“绿娥这丫头本宫素来宠爱,特意遣来伺候贞儿,想不到今日如此糊涂,本宫想明日就将她撵走,你说呢?”

田尚宫神色一动,低声道:“娘娘,绿娥跟着娘娘多年,又伺候公主这么长时间,公主的心事她总是能够知道一些的,如果撵了出去,只怕胡言乱语,有损公主清誉,今日之事,娘娘带去的都是老成厚道的宫女太监,是断断不会出去胡说的,如今除了绿娥只有韦大人知道,奴婢想韦大人不会敢胡说,若是有流言蜚语,就是皇上也饶不了他,倒是绿娥,是绝对不能让她出去乱说的。”

长孙贵妃虽然心性慈和,可是深宫多年,又是贵妃之尊,哪里不明白田尚宫所说有理,心下一狠,心道,为了长乐的名节,本宫也顾不得你是无辜还是有心了。她没有说话,只是轻轻看了田尚宫一眼,田尚宫会意,出去对着正在监刑的周尚仪使了一个眼色,周尚仪心领神会,不过片刻,外面惨叫之声猝然停止。周尚仪回来禀报道:“启禀娘娘,绿娥受刑不过,已经身死。”

长孙贵妃叹息道:“将她好好安葬,对外就说是急病身亡,对她的家人也要好好抚恤。”

田尚宫又道:“娘娘,这次报信有功的那个小太监小六子,也应该将他调到娘娘身边服侍,免得他走漏风声。”

长孙贵妃神情一震,道:“这个孩子,亏得他了,若非他看见此事前来禀报,只怕,唉,长乐性子贞烈,若是醒来之后,恐怕宁可一死,以雪耻辱,也不会甘心下嫁的。你去安排吧,这个孩子既然如此聪慧忠心,就让他留在长乐这边,让他留心,不能让这些吃里爬外的奴才害了长乐。”

田尚宫笑道:“奴婢这就去办,娘娘放心。”

这时,一个宫女出来道:“娘娘,殿下醒了。”

长孙贵妃连忙走进寝殿,只见长乐公主容颜惨淡,一看到她便泪如雨下,长孙贵妃心痛的上前,将长乐公主揽在怀中,道:“贞儿,你可是受了什么委屈,快说给娘知道,若是有人对你无礼,娘就是拼了性命,也要替你报仇。”

长乐公主苦了很久,这才止住哭声,将事情说了一遍,长孙贵妃越听越是气怒,她知道那把扇子乃是女儿寄托相思之物,如今被人毁去,怪不得她悲愤晕倒,可是凤仪门主就是皇上也不能将她怎样,想来想去,长孙贵妃打定主意道:“贞儿放心,你二哥和她们誓不两立,你总有报仇雪恨的一天,不过是把扇子,我让雍王妃再送一把给你。”

长乐公主泣道:“母妃,还是不要多事,江——他身体不好,若是听了此事不免气恼伤身,孩儿担心的很,这件事情还是不要让他知道的好。”

长孙贵妃苦笑道:“你这孩子,总是为了别人着想,好,娘就不去告诉她们,不过你父皇那里我可得去说一声,总不能这样委屈了你,就是不为你报仇,也不能让你父皇再来迫你下嫁。”

长乐公主抽噎道:“全凭母妃作主。”

离开了翠鸾殿,长孙贵妃从愤怒中清醒过来,无论皇上如何宠爱长乐,毕竟窦皇后和凤仪门主都是他们母女得罪不起的,若是自己想要去找回公道,只怕只是能让皇上为难罢了,越想越是悲伤,长孙贵妃心想,至少也要让皇上知道这件事情,她知道皇上这个时间应该在御书房处理政务,就匆匆赶去,得到允许之后,长孙贵妃踏入了御书房,可是一眼看到皇上身边坐着纪贵妃,长孙贵妃心中就是一寒。

李援看到长孙贵妃,笑道:“哎呀,今日爱妃怎么也来了,正好,一会儿朕要请凤仪门主晚膳,爱妃也一同去吧,你和门主也是旧识,也正好叙叙旧。”

长孙贵妃满腔愤怒化成冰霜,她知道李援是绝对不可能替自己作主了,只得强颜欢笑道:“臣妾是来禀报皇上,长乐突然病倒,臣妾想将长乐送到无尘庵暂时休养几日。”

李援大惊道:“朕前几日见长乐还是容光焕发,怎么今日竟会生病了,宣了太医没有?”

长孙贵妃正要说话,纪贵妃却开口道:“皇上,长乐身子一向柔弱,依臣妾之见,不如早为长乐完婚,也好冲冲喜。”

李援听了微微点头道:“爱妃说得有理,长孙,你意下如何,长乐的婚事已经拖了很久,若是能冲冲喜也是好的。”

长孙贵妃口气冷淡地道:“皇上心意是好的,可是长乐性子固执,这桩婚事她一直不肯,只怕皇上这道旨意一下,长乐就会一病不起了,皇上若是想为长乐着想,还是让她出口调养吧。”

李援不是迟钝的人,一看长孙贵妃敢怒不敢言的神情,再一联想这几日皇后和纪贵妃总在自己耳边撺掇长乐的婚事,心中了然,长乐他素来宠爱,当初长乐远嫁南楚,却是无怨无尤,令李援至今心有愧意,如今自然是不愿再强迫她,想到这里,他不由心中生出恼怒,便道:“爱妃,你这就送长乐去暗中休养吧,传我的旨意,让柔蓝去陪陪长乐,长乐素来喜爱那个孩子,也好宽宽她的心。”

长孙贵妃大喜道:“多谢陛下,臣妾这就去送长乐出宫休养。”说罢转身出了御书房,纪贵妃面色却是有些不豫。李援看来他一眼,淡淡道:“长乐这孩子为大雍牺牲良多,朕只想她后半生过得如意,以后这桩婚事就不用提了,还是让她自己作主吧,我想长乐不会做出不合礼法的事情的。”

长乐公主虽然不希望江哲知道今日之事,可惜事与愿违,我已经得知了详细经过。说起来,在凤仪门势力极强的后宫,有几个小太监敢去打扰凤仪门布下的局,小六子,原名柳杰,他就是小顺子收的记名弟子之一。

当初我想在皇宫之中安插几个人,可是这件事情说起来容易,做起来就难了,现在后宫的势力打扮都掌握在太子和凤仪门的手里,若是这个密探泄了身份,那么不仅我要被治罪,雍王殿下也脱不了干系,在我江哲为难之事和小顺子商议之后,过了一个多月,小顺子告诉了他已经办完了这件事情,他的法子也很简单,就是潜到皇宫的外围,在几处偏远宫殿找了几个资质尚可的小太监,小顺子本就是这样的出身,自然知道他们的苦楚,所以凭着自己的身份和武功很快就得到了他们的崇拜和认可,然后教给他们一些武功,这样一来,他们就成了小顺子的记名弟子,会了武功,再加上小顺子时不时的点拨,他们就如同被雕琢过的璞玉一般大放光彩,很快,就能够办事了,这个法子虽然不是很好,有些后患,可是无可奈何之下,我也只能认同了,而在我得知皇上曾经和长乐商议过太子之事后,特意让小顺子安排他们小心留意公主的安危,所以他们才能够在千钧一发之际请出长孙贵妃,救了公主。也因为这个缘故,我当晚就知道此事,虽然有些事情,小六子不可能目睹,可是却也能猜测出来一部分。

听闻此事,我只觉得心口剧痛,吐血不止,吓得小顺子连忙召来医士,直到半夜,我的病情才稳定下来,躺在床上,昏昏沉沉地想起当日飘香惨死之事,心中悲痛难忍,凤仪门啊凤仪门,当日你们害死我的飘香,今日又要加害公主,我若是不能铲除你们,死不瞑目。

第二天醒过来,看见小顺子自责的神情,他是在责怪自己不该将这件事情告诉我吧,其实我迟早会知道的。过了一会儿,雍王和石彧走了进来,满脸关切之色,李贽焦急地问道:“随云,你怎么会突然发病。”

我看着雍王的神色,他是这般焦急,让我心中莫名感动,可是那是我心中最深的伤口,也是我的逆鳞,这件事情,我是绝对不愿讲出来的,只得微笑道:“让殿下忧心了,哲不过是旧病复发罢了,只要休息几日就会好的,不知道现在外面情形如何?”

李贽忧虑地道:“随云不如好好休息,现在也没有什么急事。”

我苦笑道:“恐怕是要耽搁几日了,小侯爷的毒伤我虽然诊治过,可是现在却无力为他针灸,小顺子,你用我教你的针法先替小侯爷针灸一次,这样可以暂缓毒性,我昨日开的方子让他连服七日,然后我再亲自替他驱毒,这几日,太子和凤仪门应该忙着和齐王商议兵变的事情,殿下可要好好监视他们的行动,臣虽然旧病复发,可是应该不会有大碍,还请殿下放心,每日按时将情报送来,臣这段时间若是一松懈,只怕局势就会无法控制,那样就愧对了殿下待我的恩情了。”

李贽无奈之下,只得道:“随云你要量力而行,子攸,你好好和随云商议,多替他分担一些重担,他的身子可不能有损啊。”

石彧点头道:“殿下放心,臣必定会鼎力协助随云行事。”

在我养病这几日,情报如同流水一样传来,自从齐王回京以来,太子的势力可是全力以赴,齐王的军队开始暗中调动,看来齐王已经完全投入了太子一党了,虽然觉得有些意外,可是很明显的,凤仪门还是准备兵变的,所以我们也就没有放弃计划。

齐王的异动是瞒不过雍王和秦大将军的,但是却也无法阻止,因为齐王在长安附近的军队是用各种冠冕堂皇的名义来运动的,而且还看不出他们的目标,所以雍王和秦大将军的军队都开始提高了戒备,长安附近,风雨欲来。

寒园之内,身体渐渐好转的我在替姜海涛针灸之后又是几乎累得病倒,这次雍王可是不许我再耗费心力了,我几次争执之后也只能无奈地好好修养了,反正现在雍都附近的军力布置雍王一清二楚,我倒也能够安心休养,反正若有急事,雍王也得来问我的。

这一日,我正在房内看着前几日搜集到的孤本,董缺进来道:“公子,姜小侯爷前来求见。”

我放下书卷,道:“怎么,他已经可以下床了么,果然是底子好,想不到这么快就痊愈的差不多了。对了,那件事情怎么样?”

董缺神色带了一些讥讽道:“恐怕是隐瞒不住了,他可能是最近心情不好,这几次霍氏回去都是形容憔悴,东宫侍读劭彦劭大人已经起了疑心。”

我淡淡一笑道:“既然如此,就让这件事情结束吧,记着,最好是弄得沸沸扬扬。”

董缺躬身行礼道:“属下明白,还有一件事情,李爷方才从外面回来,又很匆忙地走了,说齐王似乎被控制住了。”

我听了一愣,转而笑道:“怪不得这些日子齐王的手下这么活跃,却是没有齐王一贯的狠辣老练的作风,原来是有人挟天子以令诸侯,罢了,这样也好,到时候雷霆扫穴之时可以容易一些,等小顺子回来,让他来见我,跟他说我不会这么容易死掉的,有什么事情还是跟我说一声,最多我让他去处理,自己不费心思就是了。”

说到这里,我不由苦笑,现在雍王、小顺子上下联手,我几乎看不到外面的情报了,虽然他们是为我好,可是我怎能放心呢?

董缺躬身答应,转身出去,片刻之后,姜海涛走了进来,虽然是毒伤初愈,可是他的面庞上已经有了健康的血色,步伐仍然轻浮,却已经十分轻快。进来之后,他躬身施礼道:“海涛多谢江大人救命之恩,连累大人旧疾复发,海涛真是十分不安,因此特来问候。”

我指了一指椅子道:“按理,小侯爷是殿下的血亲,哲不应该受你的大礼,可是江某总算为你耗费了心力,受你的大礼也不算过分,小侯爷请坐,不知道有什么事情想和江某商量?”

\chapter{第二十四章 万事具备}

武威二十五年九月十四,帝下诏秋狩,变将起。

——《雍史·高祖本纪》

姜海涛用崇敬的目光望着江哲,他可是很清楚这个人的分量,这些日子以来,他一直在客院养病,可是雍王妃常常来看望他,不免和他说了一些事情,而姜海涛最好奇的就是这个病弱的几乎随时都会没命的青年。明明自己都要自身难保的样子,却是救了自己的性命,而且听说表叔雍王对他可以说是言听计从,所以姜海涛就用当面致谢的理由进了寒园。一进寒园,姜海涛便知道雍王果然对这位江大人十分重视,寒园守卫的森严,恐怕还要胜过雍王身边的守卫。

我微笑着看向这个少年,年纪不大,脸上带着稚气,一双明晰的眼睛让人可以立刻看穿他的心事,这样一个明快的少年,令我不禁生出好感,可是疑惑也同时产生,身为东海侯独子,怎么可能会有这样一双眼睛。想了一想,我技巧地问道:“小侯爷乃是将门虎子,想来一定是深通水战,今日前来不是要来借阅我收藏的《海洋图志》的吧?”

我这个问题却是问得巧妙,《海洋图志》对于寻常人来说只是一本深奥难懂的破书,但是对于擅长水战和造船的姜氏来说,却是万金难求,这本书原本已经散失民间,但是在我状元及第之前,却无意中得到了半部残本,对我来说,这种珍贵的孤本乃是万金难求,在我入翰林院之后,从翰林院浩如烟海的典籍之中,收集到了部分残篇,凭着自己的学识和博览群书的基础,我将这本书修缮完整,我将此书献给南楚朝廷的时候,却是无人重视,只是作为孤本送入了崇文殿。原本这本书也就如同被黄土掩埋的珠玉,再也没有出头之日。可是大雍再议和之时,雍王曾经要求南楚献上典籍,这本《海洋图志》也就因此重新回到我的手里,对于这本书我有很深的感情,所以将它留了下来,不知怎么,雍王手中有一本《海洋图志》的孤本的消息不胫而走,而雍王将这本书赏赐给我的消息也被人得知,大雍有识之士还是不少的,曾有不少人希望一见,却都因为我一向不见外客,无从着手,今日我用这个问题盘问姜海涛却是其意甚深,若是那位败走东海,因此自称东海侯的姜永姜侯爷真如情报中所说那样精通水战,目光深远,那么这本《海洋图志》焉能不被他提及,若是姜小侯爷知道此书,那么说明姜永对这个爱子十分看重亲近,那么这个少年流露出的表象便是虚假的,若是他一点也不知道,除非是此子不堪造就,否则就是姜侯爷存心放纵爱子,可是我看这个少年纯真无邪,资质上佳,宛若浑金璞玉,这两个原因恐怕都让我难以相信。

姜海涛站起身兴奋地道:“海洋图志,我听爹爹说过多次,爹爹还叹气说,可惜不能亲眼目睹。你真的肯借给我看么?”他的神情振奋激昂,这是一个少年看到心爱之物的神情,却是越发显得胸无城府。我心中好奇,这位姜小侯爷是个怎样的人呢,便道:“董缺,你去把我那本《海洋图志》拿来。”

片刻之后,那本我重新抄录编撰的《海洋图志》拿来了,我递给姜海涛之后,笑道:“不过我不能白白借给你看,你每看完一篇之后,我要问你一些问题,你若答的好,我就允许你继续看下去,若是答不出来,就不许你再看了。”

姜海涛神色自若地道:“海涛虽然年幼,可是常年跟随父亲,有些事情虽然不甚了了,可是也能略知一二,只要江大人不要问得太难,海涛自信可以答出来。”

我微微一笑道:“我自然不会故意为难你。”说着示意董缺将书拿出,放到书案上,姜海涛知道这种珍贵的书籍,自己是不能亲自翻阅的,便兴冲冲地搬了椅子坐到书案前,董缺站在一旁替他翻页。

他看完一篇之后,我寻了几个问题问他,他果然是对答如流,有些问题虽然答得浅薄,可是以他的年纪来说已经是很突出了尤其令我惊讶的是,在我修缮这本书的时候,涉及到很多缺失的内容,虽然我补充上了从其他海事典籍中整理出来的内容,可是还是有很多不确定的地方,在那样的地方,我都在旁注中标明了从何处得到这种见解,还有其他几种见解和我最后的判断,这些地方我故意问他,他都有自己独到的见解,有些还明显比我的论断要正确一些。接下来几天,我和他每日交相问难,其乐无穷。

最后,我除了得到姜海涛乃是天生的水军统帅之外,还得到一个结论,他是一个除了水之外什么都不大关心的直性子人,若是驾船出海或者是水上作战,他绝对是一个好将领,可是其他的事情,还是不用指望他了,想来姜永定然是又是欣慰又是苦恼吧。微微一笑,我写了一封短柬让董缺收藏起来,董缺慢吞吞的收了起来,这些日子,我给了他好几张短柬,不过董缺倒是聪明人,一张也没有看,也没有问我到底要做什么。

这一天我正在花园中赏菊,雍王李贽来到我面前,沉声道:“随云,现在局势已经是一触即发了。”

我淡淡一笑道:“殿下请讲。”

李贽道:“父皇宣布,后日前往猎宫举行秋狩,在京皇族都要参加,齐王上书告病,但是父皇却要他抱病同行。”

我若有所思地道:“看来皇上也是很小心的,不知道皇上为什么举行秋狩呢?”

李贽叹了口气道:“这几日发生了很多琐碎事情,真是一言难尽,本王原本以为不需要劳烦随云,可是现在看来,只得辛苦你了。”

我正容道:“殿下对哲厚爱如此,若是哲不能在关键时候为殿下效力,岂不是辜负大恩,请殿下直言就是。”

李贽叹了一口气,给我讲了这几天的情形。

自从九月初三我病倒之后,在我养病期间,齐王初时只是小动作不断,但是雍王乃是军略大家,没有多久就发现,齐王的军队唯一的目的就是准备伏击。

今日李援下诏举行秋狩,这次随行禁军两万人,由秦大将军秦彝总领,其中隶属于禁军东营的共有一万人,以秦青为正统领,两位副统领黄厦、孙定分统五千人,禁军南营五千人,由统领杨谦、副统领呼延久率领,禁军北营五千人,正统领裴云、副统领夏侯沅峰都会随行。太子、雍王、齐王都奉诏同行,除此之外,窦皇后、纪贵妃、长孙贵妃、颜贵妃、长乐公主李贞、靖江公主李寒幽都会随驾,在京中留守的是丞相韦观和伤势好转的侍中郑瑕,负责京中安全的是禁军西营统领谭义,另外大臣随驾不计其数,其中值得我注意的是魏国公程殊、齐王妃秦铮的父亲中书侍郎秦无期、新入中书省不久的韦膺和太子少傅鲁敬忠。

这还不算,皇上下诏这次雍王和齐王都只能带百名近卫,秋狩期间,一切以军令行事,抚远大将军秦彝就是统帅,看来皇上已经知道如今的紧张局势了。

齐王上书推辞随行不果之后,齐王的军队就停止了行动,但是雍王判断,这些军队只要一夜之间就可以急行百里,可以在回京之路上伏击皇上的圣驾,而且齐王调军的理由都很充分。当然雍王也做了准备,可以随时阻击齐王的军队,只是这样一来,必然会酿成大战。

但是令雍王和属下将领幕僚不解的是,为什么齐王会同意随驾,这样一来,绝对没有人可以指挥齐王的军队进攻圣驾的。

我看着手中的情报,也不由皱紧了眉,有这样的结果我是能够想到的。就在前日,雍王送了一封密信给秦大将军,信上告知李寒幽身世可疑,虽然没有显示证据,可是李寒幽确实是自幼失散,后来被凤仪门送回靖江王府的,这样一来,至少也会让秦大将军生疑,有些事情,宁可信其有,不可信其无,效果我已经知道了,那封信一到大将军府,程殊就被请了过去,然后秦勇也被召去,虽然不知道他们商议了什么,可是秦勇已经赶赴秦大将军军中坐镇,事实上,秋狩期间,秦彝所掌握的军队就在秦勇的控制之下。我原本就不指望他们相信,只是让他们戒备罢了,这样已经超出了我的预计。

另外,就在昨天,东宫侍读劭翰林的妻子霍氏悬梁自尽,然后一夜之间,太子*臣妻,令其羞愧自尽的消息传遍了全城,虽然只是街谈巷议,可是和太子从前所为一对照,倒是人人都很相信,虽然皇上可能还不知道,可是秋狩之后,那是绝对瞒不过了,所以太子若是不能在秋狩期间逼宫,那么恐怕被废的命运已经难以改变。

我叹了一口气,太子虽然被我逼反了,可是为什么凤仪门的布置这么古怪呢?

我原本以为凤仪门会安排齐王的军队突然闯入皇上行宫,毕竟两万禁军太子和凤仪门至少可以控制一部分,里应外合突然袭击,我应对的计划是让秦大将军“及时”发现异常,然后设下圈套,那些齐王的军队一旦到了,有秦大将军和雍王出面,无论齐王如何,我方都可以控制住局势,然后在各派高手的配合下,一举铲除凤仪门。可是现在却不是这样,最近的齐王军队也在秋狩地点两百里之外,而最近的秦大将军的军队在百里之内,雍王军队则也是两百里之外,那么,我绝对不相信凭着凤仪门主的门下弟子就可以谋反成功,而且凤仪门主根本还在栖霞庵,没有准备同往秋狩。在我预料中,凤仪门主应该会随驾的,可是现在却是全然不同,我真的有些不知所措了,局势会怎样发展呢,凤仪门主果然是非同反响啊。齐王的军队不比雍王的军队多,如果两军交战,又没有齐王在军中,那么是绝无可能成功的,现在禁军有秦大将军掌控,叛乱是不可能的,那么凤仪门凭着什么造反呢?

对于实际上的军务,我可是不如雍王和那些将领的,反复商议之后,仍然得不到太子可以逼宫成功的可能,可是若是没有成功的可能,他们是绝对不会进行的,最后,我们只得商议好,由长孙冀带着雍王的军队随时出击,阻击齐王的军队,荆迟、司马雄随行护驾,石彧等人在京中主持大局,慈真大师指派了五十名各派高手担任雍王近卫,并且坦言是几大门派的共同意思,而他自己则监视凤仪门主,事实上,像他们这等级数的高手,彼此之间就是隔着几里路,也能察觉到对方的存在,所以,我们是不担心他会跟丢凤仪门主的,而小顺子和董缺都随我一同参加秋狩,虽然我病势未曾痊愈,可是今次事关重大,我如何能够不去。

虽然现在只能静观其变,可是我还是让小顺子传出我的命令,秘营全部运动起来,一定要可以随时应对各种变化,这个我倒不担心,他们都是随机应变的好手,而且我还把雍王府的令牌给了他们,他们可以随时得到支持,我紧握双拳,一定要相信自己,就算是局势突然有了变化,我也可以力挽狂澜,更何况现在还看不到雍王和我的布局有什么欠缺呢。

栖霞庵中,凤仪门主站在月光之下负手而立,在她身后,两侧站立着她的亲信弟子,闻紫烟、萧兰、凤非非、谢晓彤、燕无双、李寒幽,除了梁婉已经疯癫,凌羽负伤不在,秦铮难以脱身之外,所有人都到齐了,而在这些弟子的身后,站立着一共百名的女剑手,都是衣衫如雪,面色冰寒,她们就是凤仪门主亲自培养出来的凤仪门的中坚力量,这些女子都是自幼被凤仪门收养,她们所练习的太阴真经少了一部分关键,所以她们个个无情无欲,心冷如冰,在她们眼中,只有忠诚和杀戮。

良久,凤仪门主淡淡道:“秋狩期间,就是我们发动之时,此事务要成功,否则我凤仪门就要万劫不复。”

闻紫烟寒声道:“师尊放心,一切已经安排妥当,若是我们还不成功,那就是天命如此。”

梵惠瑶冷冷道:“我从来不信什么天命,紫烟,你记着,我虽然不能亲自前往,可是你们务要精诚合作,寒幽,晓彤,皇上那边的事情由你们负责,秦铮到时候会听从你的命令,萧兰、非非,你们要负责配合太子清剿所有反抗势力,紫烟、无双,你们要负责围歼雍王,本座还要对付那个多管闲事的慈真,就不能去支持你们了。”

众人单膝点地道:“弟子遵命。”

梵惠瑶也不让她们起身,又道:“还有一个人会配合你们,他是本座秘密所收的记名弟子。”

随着她的语声,一个男子从房内走了出来,闻紫烟等人目光落到他没有遮住的面容,都露出了惊讶的神色。

梵惠瑶淡淡道:“他乃是凤仪门的护法,这次,你们要多多听从他的意见。”

闻紫烟等人突然明白了很多从前不明白的事情,却都没有表现出来,只是恭谨的应声。

凤仪门主看看迷茫的夜色,道:“纵然是雍王他们如何猜想,也不会想到本座的布局,哼,他们想迫使太子谋反,难道本座不知道么,只有太子和鲁敬忠才会相信李援确实准备废黜太子,却不知道,本座认识李援多年,对他的个性很了解,他虽然已经有了这个心意,却还没有下决心,不过这样也好,李援若是动摇,必然会对我们不利,再说,太子谋反成功,也是后患无穷,以后更要依赖本门。你们听着,事成之后,我凤仪门就是大雍的幕后主宰,所以你们必要尽心竭力。”

闻紫烟等人眼中都涌起强烈的野心,作为女子,她们即将完成无人能及的事业,还有什么比这个更加让她们自豪和骄傲的呢?

齐王府内,重重帘幕之后,李显神色慵懒的躺在软榻之上,神色一片冷漠,秦铮神色有些不安的走过来,端来一碗参汤,道:“王爷,请用参汤,明日就要起程秋狩,您还是早些安歇吧。”

李显看着秦铮,嘲讽地笑道:“好啊,齐王妃,你很厉害,一碗药就让我手无缚鸡之力,看来你对师门可是忠心不二啊,却忘记了什么是三从四德。”

秦铮落泪道:“王爷,妾身实在是为了你好,从前妾身虽然是奉命接近殿下,可是妾身对王爷却真的是一片真情,可是我是不能反抗师尊的,而且她们说得不错,若是太子登基,王爷可以位极人臣,妾身和孩儿也可以安然无恙,若是雍王继位,不仅妾身和孩儿性命难保,就是王爷你也是迟早会被雍王所害,若不是为了王爷,妾身宁死也不愿伤害王爷。”

李显苦涩地一笑道:“我是不是也是口是心非呢,虽然责骂你,可是我竟然也希望你能成功,否则,真的是要一家人共上黄泉路了。”

秦铮激动地道:“不会的,不会的,我们一定会成功的,师父绝不会失败的。”

李显叹了口气,心道,真的会这么容易么,他想起那张清瘦文弱的面庞。

今夕何夕,不知道有多少人中宵难寐啊。

\chapter{第二十五章 顿失先机}

武威二十五年九月二十,帝至猎宫,至夜,太子安叛,雍王危殆。

——《雍史·高祖本纪》

南楚同泰二年九月二十,贼矫诏命雍王觐见,为哲识破,哲临危受命,指挥若定,雍王得以突围。

——《南朝楚史·江随云传》

我是昏昏沉沉的在马车里面睡到了猎宫的,猎宫是大雍皇室每年秋狩所使用的行宫,位于骊山脚下,有大小几十处宫院,禁军在三面扎营,将行宫护在当中,皇上自然是在行宫的正殿晓霜殿驻驾,皇后、纪贵妃、颜贵妃分别居住在附近的几处宫院,长孙贵妃则和长乐公主住在东侧含香苑,含香苑遍地菊花,李援有意让近日郁郁寡欢的长乐公主疏解一下愁绪。太子住在东侧的玉麟殿,而雍王住在西侧的雅宁轩,齐王住在西侧的宣华苑,我可是知道现在自己是经不住奔波的,所以特意服了药,一路上沉沉睡去,直到安顿下来之后,我才清醒过来。

小顺子告诉我,皇上已经下旨,今日旅途疲劳,各位殿下和大臣都不用去问安,明日会猎之时再去朝拜即可。我问道:“太子和凤仪门是否有情报传来?”

小顺子道:“还没有,除了秦大将军带着秦青将军亲自布防之外,并没有任何异常。”

我接过小顺子递过来的布防图,秦大将军不愧是名将,布防无懈可击,保护皇上居处的是秦青带领的三千东营禁军,保护猎宫东侧宫殿的是南营禁军杨统领,保护西侧的是北营禁军统领裴云,负责大内侍卫的是侍卫总管冷川,而从西侧进入中宫必须通过的月华门,以及从东侧进入中宫的钟萃门,都被保护中宫的禁军和大内侍卫严密控制,想要兵变恐怕是不可能的。

不过,我淡淡苦笑了一下,秦大将军对自己的儿子还是有些偏心的,这种安排,虽然将秦青置于控制之下,但也有让秦青在有事之时立功的打算。

夜深之时,我和雍王一边品茗一边讨论着局势,我有些不安,可是雍王倒是十分沉稳,对他来说,不知道经历了多少风险,早已不会因此而担忧苦恼了。一更天才过,突然司马雄进来禀报道:“殿下,韦大人前来传旨。”

雍王和我都是一愣,韦膺来了,转念一想,这也难怪,这次皇上秋狩,只带了韦膺替他拟旨,其余文官都没有带来,再说近年来韦膺十分得宠,日日在君王身侧,不知道有多少诏旨是韦膺的手笔,雍王不比寻常,若是皇上有旨意,自然应该是韦膺来的。我陪着雍王走进正殿,只见韦膺紫衣绶带,风度翩翩,气度闲雅,看到雍王,他笑道:“殿下,臣奉陛下口谕,前来传旨,请殿下跪接。”

雍王看了我一眼,俯身拜下,我也跟在后面跪下,而荆迟和司马雄虽然也跪下,却是虎视眈眈的望着韦膺,今日的局势,是谁也不敢懈怠的。

韦膺似乎对这种紧张的局势毫无所觉,道:“皇上口谕,宣雍王李贽前往晓霜殿见驾。”

李贽口称遵旨,起身之后却笑道:“韦大人,不知道父皇有什么吩咐,今日早些时候不是说不用我们去问安了么?”

韦膺道:“皇上本来很疲倦,可是小睡之后却是精神好多了,皇后娘娘和几位贵主都在伴驾,共同品茗闲话,方才皇上起意,所以诏几位殿下和长乐公主前去参加家宴。臣已经去太子和长乐公主那里传过了旨意,这就要去请齐王了。”

雍王略略放心,道:“韦大人请去传旨吧,本王这就去觐见父皇。”

韦膺传旨已毕,行礼之后告退而去。雍王笑着对我说道:“韦膺有相辅之才,将来可以重用。”

我正要附和,可是心中却无端生出一种寒意,韦膺的表现堪称完美,可是为什么我却觉得有些不妥,下意识的,我全力侧耳倾听,这时,韦膺已经走到了雅宁轩门外,这时,我听到他松了一口气的声音,然后听到了低微的轻笑,那是一种志得意满的笑声。

我突然想到了很多事情,一向中立的韦家一直风平浪静,而凤仪门全力拉拢秦家,虽然可能是因为秦家掌握兵权,可是对韦家总不该一点动作也没有啊。再想到,太子东宫事发,韦膺奉命监护太子,郑侍中御前会议上态度明确的指责太子,随后朱雀门前遇刺。长安血夜,袭击庆王侍卫的蒙面人和刺杀郑侍中的刺客都是男子,韦膺应该是武功不错的,这是小顺子曾经无意中说过的。越想,我越觉得已经身陷罗网当中,如果韦膺甚至韦家和凤仪门已经有了勾结会怎么样。

我断然道:“小顺子,你去看看外边可有埋伏,记着,不可露了形迹。”

雍王等人都是脸色大变,小顺子面色一寒,身形隐入夜色当中。片刻之后,小顺子回来了,脸色有些苍白,他冷冷道:“月华门有东营的禁军埋伏,四下都有凤仪门弟子隐藏,我看到了闻紫烟,不过不敢接近。”

雍王面色急剧变化,片刻才道:“韦膺和凤仪门有勾结。”

这短短的时间之内,我已经想明白了很多事情,神情变得冷淡从容,轻轻摇动折扇,我淡淡道:“这是我的失算,韦膺的身份可以让猎宫中很多人相信他的话就是皇上的旨意,另外,我已经想到了凤仪门的计划,她们用齐王的军队引开我们的视线,而她们真正用来叛乱的乃是禁军。”

李贽剑眉一扬道:“禁军怎会被她们所用。”

我苦笑一下,道:“殿下和臣都有一个错误的想法,如果不能获得禁军的控制权,那么就不可能驱使他们叛乱,而能够得到控制权的只有秦大将军和秦青,现在我们可以确信秦青无法完全控制禁军,所以就疏忽了一点,能够控制禁军的还有一个人,就是皇上。”

司马雄和荆迟都是一声惊呼,我不理会他们,继续说道:“李寒幽身为公主,又是秦家的儿媳,如果她拿着皇上的旨意,说是奉命指挥禁军,诸位说会怎么样。”

众人都是心中一寒,我继续说道:“李寒幽在禁军中已经颇有影响,秦青这两年来虽然实际上不能掌管全部禁军,可是至少东营禁军还是他直接管辖的,李寒幽乃是公主身份,那些禁军又是秦家嫡系,那么李寒幽收买个几千人又算什么,再加上韦膺是随驾拟旨的大臣,太子又是储君,只要控制了晓霜殿,皇上的旨意传不出来,那么殿下就是孤立无援,如今殿下的军队在百里之外,只能是任人宰割了。”

司马雄和荆迟等人都是十分震惊,但是雍王却是神色冷静地道:“随云既然已经想通了凤仪门的布局,想必已经有了应对的法子。”

我叹了一口气道:“殿下果然深知为臣之心,他们这个法子唯一的破绽就是不能引起我们的怀疑,所以他们不敢提前铲除裴将军,现在殿下唯一的生机就是在此了,这也是他们矫诏招殿下去晓霜殿的原因,他们想在月华门伏击,一举杀死殿下,到时候裴将军也只能俯首听命,毕竟裴将军还有身家性命。现在托殿下洪福,臣得以看破他们的布局,那么就有生机,请殿下按照臣的安排行事。”

李贽淡淡道:“随云,本王相信你有法子,今日本王的性命就交给你,你下令吧。”

我躬身一礼道:“都是臣这些日子昧于心伤,这才没有发现敌人的诡计,殿下不怪罪臣,已经是万千之幸,多谢殿下仍然相信臣的判断。”

李贽还礼道:“请随云不必多虑,也是本王这些日子刻意不让你知道外界情形,才有今日之变,请随云下令,本王定会谨尊将令。”

我直起身子,道:“那么臣就越俎代庖了,现在殿下必须突围出去,而在突围之前,殿下必须会合裴将军,臣相信裴将军现在还安然无恙,凤仪门主行事,必然不会打草惊蛇,裴将军武功高强,又得军心,若是用强,只怕会引起殿下怀疑,所以现在小顺子立刻去见裴云,让他和殿下会合,一起冲出猎宫,裴将军身边一定有凤仪门的刺客隐藏,小顺子必须去保护裴将军,否则殿下就没有机会突围了。现在矫诏应该还没有传遍全军,所以殿下突围应该没有问题,不过在和裴将军会合之前,凤仪门的围杀就要靠殿下的近卫和各大门派派来的高手支撑了。至于会合地点,我想要由殿下决定。”

雍王指着布防图道:“现在只能从西南方向突围了,小顺子,告诉裴将军,在这里会合,看到这边火起,就是我们行动之时。”

小顺子点点头,身形再次消失。

我又道:“殿下突围之后,立刻把这件东西送到最近的秦军统领秦勇手中,这原本是臣以防万一准备的,想不到派上了用场,有这件东西,至少秦勇不会向殿下进攻。”

这时,司马雄进来道:“殿下,我们都已经准备好了,只是……”司马雄看向我,欲言又止。我淡淡一笑道:“殿下,这次臣就不能相陪殿下突围了。”

雍王大惊,一把握住我的手道:“随云,你在胡说什么,你手无缚鸡之力,若是留下来必然遭害,岂能不走。”

我苦笑道:“殿下,随云体弱,这次殿下突围,必然是快马加鞭,臣若是随行,只怕会死在路上。”

李贽摇头道:“你放心,本王用马车载你,再说,跟着本王突围还有生机,若是留下来,只怕是必死无疑,凤仪门绝对不会放过你的。”

我淡淡一笑,走进雍王,低声说了一句话,雍王一愣,面上泛起深思,我不等他想明白,就道:“殿下不可再耽搁时间,我让董缺保护我留下来,殿下若是能够杀出重围,就算臣落入敌手,也有一线生机,殿下,如今殿下和齐王的军队都是远水不能救近火,秦大将军的军队已经成了关键,请相信臣可以尽量为殿下争取到大将军的支援,大将军久经沙场,也不会甘心被制。”

这时司马雄走近来道:“殿下,约定的时间就要到了,请殿下速速决断。”

我肃然道:“司马将军,殿下安危寄予你手,哲重托于你。”

司马雄施礼道:“末将就是粉身碎骨也要保护殿下杀出重围。”

我又看向荆迟道:“荆迟,你是殿下身边大将,这次你身担重任,不可懈怠。”

荆迟苦涩的笑道:“若是我不尽力,最多先生罚我多抄几本书。”

他们虽然听我说有自保之道,可是谁都知道那是不可确定的事情,他们突围,还有三分生机,我留下来却是生机渺茫,可是他们自问无法携带我突围,心中的愧疚更让他们充满了愤怒和杀机。

李贽看向董缺,这个沉默的青年,沉声道:“董缺,你若能保护随云和本王重逢,本王必定重重有赏,就是你从前有些什么不好之处,本王也绝不加罪。”

董缺神色不变,只是轻轻施了一礼,我却是微微苦笑,看来雍王还是对董缺的身份起了疑心啊。

李贽大步走出殿门,扫视了全副武装的众人一眼道:“都是本王连累你们,现在太子谋逆,意图杀害本王,诸位随本王突围,乃是九死一生,贽无以为报,唯有当天立誓,若是本王幸免于难,诸位都是本王患难之交,必有重赏,若是有胆怯者,可以留下投降,本王绝不怪罪。”

众人都知道不能大声,都是沉声喝道:“太子无道,圣聪蒙蔽,殿下身系大雍社稷,臣等肝脑涂地,万死不辞。”

李贽一挥手,在司马雄和荆迟保护下当先上马,疾驰而去,这雅宁轩只留下我和董缺二人,我看看董缺,笑道:“你怕不怕?”

董缺淡淡道:“公子都不怕,董缺又有什么可怕的,不知道公子如何安排。”

这时,远处传来震耳欲聋的喊杀声,而我也丢下一个火把,点燃了司马雄等人收集的可燃之物,火光中,我苍白的面容带了几分血色。

在雅宁轩之外,闻紫烟和燕无双带着五十名凤仪门剑手,正在监视雅宁轩,韦膺则已经到了月华门,拿着“圣谕”指挥禁军准备伏击雍王一行,凭着韦家的声望和皇上的手谕,那些禁军虽然心中疑虑,可是却也不敢违背命令,毕竟对他们来说,皇上才是他们效忠的对象,纵然如此,最接近雅宁轩的地方,韦膺还是安排了凤仪门可以完全控制的部分禁军,以便减少雍王逃脱的可能。

就在他们有些心焦的时候,突然,雅宁轩大门敞开,雍王身穿金甲,手执马槊,高声道:“太子谋反,意图杀害我李贽,本王乃是天策元帅,焉能被小人所害,凡我大雍子民,不可受奸人挑唆。”言罢,在司马雄、荆迟左右护持下,率领百骑冲杀而去,这猎宫本就是秋狩所使用,所以宫中御道皆可纵马,闻紫烟一愣之下,眼看这些人就从眼前冲了出去。

闻紫烟反映极快,心道,他们的方向正是月华门,想必是想去向皇上申诉,我们不妨在后面阻截他们的后路即可。便一声轻啸,四下皆闻,带着禁军从后面合围而去。

月华门设伏的韦膺,听到雍王的大喝和闻紫烟的轻啸之后,心中一凛,立刻下令准备弓箭,自己却带着一千禁军迎了上来,毕竟,他要防范雍王从别的方向突围,雍王精通兵法,他可不认为雍王会走向这条明显的死路。

月光之下,一道黑箭和身穿青色衣甲的禁军迎头相遇,荆迟一声大喝,手中马槊闪动,将那些未曾骑马的禁军扫荡开来,司马雄的马槊也不等闲,鲜血四溅,雍王大喝道:“本王李贽,谁敢拦我。”手中的佩刀闪动,斩杀了一个禁军,那些禁军若是对敌自然是前仆后继,毫不畏惧,可是面对心中仰慕已久的大雍军神,战意低落,只是瞬息之间,雍王指挥的锋矢阵已经冲破了禁军的封锁,站在远处指挥的韦膺一皱眉,他可是不便出手的,因为他要维护钦差的身份。这时,闻紫烟身影显现,快如闪电,几个纵越已经逼近雍王侧面,然后身剑合一,向雍王疾刺而去。

这时雍王一声号令,明明已经接近月华门的军阵迅速的转身向西南方向突围而去,若是有高明的将领指挥,或许还可事先设下防线,可是在场的韦膺和闻紫烟都不是精通军阵的将领,事先也没有料到雍王会发觉阴谋迅速突围,所以一愕之下,已经看到雍王再次突破后方禁军的薄弱防线。

闻紫烟高声道:“反贼是想和裴云会合,不能放过他,追。”

这时,雅宁轩突然火起,火势蔓延的极快,烟尘蔽目,雍王的锋矢阵就从雅宁轩的边缘冲过,直仆猎宫西南方向的角门。就在雍王刚刚越过雅宁轩的时候,一道剑光从地上电射而出,直扑雍王,一个雍王亲卫从马上跃起,手中的长刀劈下,剑光刀光一触而灭,那个亲卫从半空中坠落,鲜血洒落,而那道剑光却也不能再进一步,雍王已经冲过了雅宁轩的范围。

剑光一黯,一个素衣劲装的女子飞速退走,避开了那些冲过来的雍王亲卫接连劈下的长刀。

闻紫烟心中一凛,燕无双刺杀失败,这时候若是动用那些凤仪门剑手,虽然可以缠住雍王,可是必然损失惨重,她可舍不得,何况雍王想和裴云会合,只怕是没有希望,到时候进退维谷,才是凤仪门剑手发威的好时机,所以她没有发动那些剑手,而是任凭雍王杀向西南。

\chapter{第二十六章 猎宫突围}

时,太宗佯攻向东,转而西南,幸得裴将军云接应,方突围而出,然叛军追袭百里,太宗数次险遭合围,幸得众将并义士拼死保护,方脱险地。

——《雍史·太宗本纪》

就在雍王突围之前的一刻,禁军北营统领裴云遇到了前所未有的危机,就在自己的袍泽兄弟的重重包围之下,他只能孤身面对敌人。

轻轻叹了口气,裴云随手撕下一片战袍,裹住肩上的伤势,而在他对面,一个玉树临风、容颜如玉的青年——夏侯沅峰,正含笑而立,在他身旁,是四个雪衣女剑手,其中一个女剑手的剑锋上还带着殷红的鲜血。

在六人外面,夏侯沅峰一系的禁军将六人团团围住,而更外面则是忠于裴云的禁军,此刻双方正在对峙,夏侯沅峰不敢过于逼迫裴云,否则外面的禁军大怒之下,可能会让他们骨肉化泥,而裴云也不敢让自己的属下进攻,否则恐怕还没有攻入重围,就会让裴云先丧命了。

夏侯沅峰笑道:“裴将军何必这样固执,您原本是齐王麾下,齐王殿下又是太子殿下的支持者,雍王对您的一些小恩惠怎如齐王殿下当日的厚爱,若是将军悬崖勒马,下官保证,太子殿下和齐王殿下绝对不会为难将军。”

裴云冷笑道:“本将军乃是大雍将领,不受乱命,我不相信皇上会下诏处死雍王,所以你夏侯沅峰还是不用徒费唇舌,谁不知道你和凤仪门都是太子一党,太子要谋逆作乱,怕是因为恶名昭彰,担心皇上废黜吧。”

两个女剑手突然剑如电闪,交叉划过,裴云身形一闪,夺过剑刃,另外两个女剑手恰好发动,裴云手中的佩刀化作铜墙铁壁,五人落下,四个女剑手仍然是将裴云围在当中。夏侯沅峰再次扑上,耀眼的剑光绮丽无比,四名女剑手也再次发动,裴云的武功本就和夏侯沅峰在伯仲之间,一时有些应接不暇,这时,外面的禁军同声高喝,夏侯沅峰露出苦笑,只得放慢了攻势,裴云这次勉强支持得住。

就在这时,一个青色身影瞬息间穿越重围,夏侯沅峰只觉得背心仿佛被鹰狼盯住,连忙侧身退下,却仍然被掌风扫中脊背,一时之间无力反击,而青色身影已经闯入了凤仪门女剑手的剑阵中心,裴云只觉得一道阴柔的掌风将自己送出了剑阵,这时,四个女剑手同声轻喝,剑光如雪,肆无忌惮的向那青衣人扑去。青衣人身形闪动,一双空手将那四个女剑手狠辣绮丽的进攻压制住,斗了不到十招,青衣人,身形闪动,令人目不暇接,然后传来四声惨叫,四个女剑手都是被青衣人击中要害,倒地身亡,可是她们疯狂的进攻,也在那个青衣人的身上留下了痕迹,他的青衫已经是下摆碎裂。

小顺子皱皱眉,看看倒在地上的女剑手,这些剑手疯狂而狠辣,她们若是数人联手,威力更胜过李寒幽等人,看来,这才是凤仪门的杀手锏啊,他的目光落到夏侯沅峰身上,杀气凝聚。

夏侯沅峰心中一寒,此时他已经恢复了内力,连忙道:“退。”说罢向外冲去。

小顺子刚刚抬起手掌,裴云已经喊道:“李爷,现在不是时候,还是救援殿下要紧。”

小顺子皱皱眉,没有说话,裴云也不拦阻,夏侯沅峰控制的禁军和大内侍卫都是精兵高手,没有必要在这里动手,若是被缠住,只怕就来不及救援李贽了,裴云可是很清楚,如果不是那边已经对雍王动手,夏侯沅峰是不会对自己出手的,更何况小顺子前来,不是为了求援,还会有什么缘故。

小顺子匆匆对裴云说道:“韦膺是太子一党,假传圣谕,殿下要突围,要你接应。”

裴云立刻下令出发,他对猎宫的地势很清楚,又有小顺子引路,没多久,就看到了冲天而起的火光,也听到了杀伐之声。裴云看看身后,在事发之前,东营黄统领送来的新的布防图,将倾向自己的四千禁军分散在猎宫西侧,而夏侯沅峰那一千禁军却是集中在一起,当时裴云虽然心中疑惑,也没想到会有这样的变化,所以尽管下了召集令,仍然只有两千多人来得及跟上自己去保护雍王,这点人马,能够保护雍王突围么,裴云忧心忡忡的想着。

当裴云和小顺子赶到西南角门的时候,正是雍王被外面的禁军阻挡的时候。这个方向的禁军乃是秦大将军嫡系,东营禁军副统领孙定的辖区,孙定和凤仪门并无勾连,可是就在他布防之前,曾经有秦青将军的近卫拿着秦青的兵符前来颁下严命,今夜不论发生何等变故,这个方向不许一兵一卒出去。所以虽然他们对雍王高喊太子谋反的事情心中将信将疑,可是雍王既没有皇命,也没有秦大将军或者秦将军的手令,所以他们万万不敢让雍王过去。双方争执之下,成了不死无休的僵局。雍王不过百余随从,就是再厉害,也难以通过禁军布下的防线。就在激战最酣的时候,雍王身后,追击的闻紫烟已经可以看到身影了。

裴云也顾不上什么敌众我寡,高呼道:“殿下,末将裴云前来护驾。”

雍王冷峻的面容上闪过一丝如释重负的神情,若没有裴云的禁军,只怕是很难冲出宫门的。他高声道:“裴云,给本王杀开一条血路。”

裴云高声领命,手一挥,众多禁军将雍王和近卫护在当中,向宫门冲去。

闻紫烟一看到裴云就知道不好,身影连闪,向雍王的方向扑去,她武功高强,当年又屡次赴过战场,所以避开了禁军的拦阻,很快就接近了雍王,这时,一道青影凌空扑来,闻紫烟一剑刺出,那个青影赤手像剑上抓去,闻紫烟大怒,这人也未免太过瞧自己不起,真力贯注在剑上,这时却听见一声脆响,她那把可以切金断玉的宝剑居然从中折断,闻紫烟一愣,那人已经一掌拍向自己的胸口。闻紫烟毕竟是心如铁石,已经用短剑刺去,这一剑乃是两败俱伤的招式,那人果然略略一滞,两人就在乱军之中交战起来。这时闻紫烟已经看清了那人面容,那人正是“邪影”李顺。闻紫烟精神一震,若是杀了此人,那么雍王身边就没有可以依赖的高手了,所以她稳住心神,全力和李顺交手。这时,一个白衣女剑手抛过一柄长剑,闻紫烟顺手接过,然后凤仪门名震天下的疾风剑法终于全部展开,那超越人体极限的快剑掩盖住了冲天的火光和交战双方兵刃上的血光,而李顺的身影却是诡异非常,在剑光之中若隐若现。这场厮杀,若是平日,自会有人惊叹折服,可是此刻,双方却都无暇顾及了。

这时候,在荆迟、司马雄和裴云的冲锋之下,猎宫外面的禁军已经支持不住,杨统领虽然也是一员猛将,可是面对着大雍将领中若论武勇战略,皆可排在前三十位的三位将军,终于还是露出了破绽,战阵露出了一处薄弱的所在,雍王等人都是久经沙场的宿将,一眼看穿,荆迟高声大喝,马槊横扫,将阻拦去路的一命禁军将领斩杀,禁军更加是气势大弱,雍王趁机下令猛冲,千余铁骑就这样冲出了猎宫,在茫茫夜色中消失了身影。就在这短短几拄香的时间,裴云麾下倒有将近半数的勇士倒在了猎宫围墙之内。这时,几个凤仪门弟子已经逼了过来,闻紫烟狠下心要将李顺留下。

小顺子心明如镜,自己的武功虽然高强,可是在这些剑手的猛攻之下恐怕是得不偿失,而且那些禁军已经渐渐围拢过来,自己再不走就来不及了。想到这里,他的身形突然诡异的滞留在空中,几个女剑手所料未及,剑势不由露了破绽,小顺子已经向闻紫烟扑去,闻紫烟凝神静气,一剑刺出,这一剑势若雷霆,小顺子右手一扬,却是食中二指之间夹着一枚发簪,剑锋在划过小顺子右肋的同时,那枚发簪也划过闻紫烟的脸颊,闻紫烟只觉得一缕寒气扑面而来,下意识的侧过螓首,因此才避过了失目之祸,而小顺子已经趁机越过了她的身侧,将一名禁军踢下马去,策马追赶雍王去了。

闻紫烟眼中满是怒气,道:“给我死死咬住他们,追杀百里也要杀了雍王。”说罢接过旁人递过来的马缰,一马当先追向雍王等人。韦膺站在高处微微皱眉,在这猎宫之中,他们可以完全控制的禁军不过五千人罢了,其他的禁军只能让他们协防而已,控制晓霜殿和追杀雍王都只能牌亲信的禁军前去,这样一来,人手就很紧张了。他想了一想,还是派了两千人跟着闻紫烟去追杀雍王,剩下的三千人应该足够控制晓霜殿和其他的禁军了。

真可惜啊,韦膺看着远去的滚滚烟尘,不知道雍王是如何发现了陷阱的,竟给他逃出了猎宫,若不是齐王那里出了岔子,齐王妃虽然偷到了齐王的兵符,可是调动齐王的军队换防可以,想让他们攻击雍王的军队或者围攻猎宫,他们都是坚决不肯答应,声言除非见到齐王的手令,否则不能从命,看来还要在齐王身上下些功夫,现在雍王已经脱身,如果让他和近卫军会合,那么若没有齐王的大军支援,自己这一方有败无胜啊。韦膺一边想,一边眉头深锁,发动宫变之前以为一切都想到了,可是还是没有料到雍王居然这样快就看穿了伪诏,这是怎么一回事呢?

天色拂晓,从清冷的雾气中传来清脆的马蹄声。这一夜闻紫烟带着禁军穷追不舍,纵然雍王精通军略,也是无可奈何。雍王本想不顾追兵,全力行军就是,可是这次雍王突围十分仓促,甚至还有两人一骑的情况,而追兵却先从其他禁军处征用了马匹,平均每人带了两匹马,可以随时换马,这样一来,速度就比雍王等人快多了。无奈之下,雍王辗转迂回,调动追兵,一路上连番设伏偷袭,想要将追兵歼灭。可是那率领追兵的两人,黄统领黄厦乃是沙场宿将,闻紫烟乃是多年在战场上出没的刺客,两人一个小心谨慎,精通兵法,一个武功高强,乃是最出色的斥候,一路之上居然没有让雍王占到什么便宜。

巧妇难为无米之炊,不论兵力还是速度,雍王都不占优势,战术上面的优势又被强悍的武力抵消,雍王从未如此狼狈。直到天明时分的一次伏击,雍王麾下所有的高手联手在一个狭小的谷口设伏,才让追兵遭受了较大的损失,跑了一夜的雍王终于可以暂时松了一口气了。

雍王很清楚,这些负责追杀自己的禁军,绝对是凤仪门可以如臂使指的力量,他们人数虽然不多,可是比较起来,自己的兵力更弱,除非能够会合自己的近卫军,否则自己就要遭遇平生最危险的境地了。目下当务之急,就是和自己的军队会合,雍王还是不敢相信秦勇会帮助自己,现在凤仪门可能已经拿到兵符,到时候秦勇恐怕只会听从矫诏行事,所以和长孙冀、董志率领的军队会合就成了雍王最大的目标。

这时,远处一骑飞驰而来,雍王等人都心中一凛,虽然只有一骑,可是若是闻紫烟或者那些凤仪门女剑手亲任斥候,那么自己的行踪就会立刻泄漏,如果不能休息一下,这样下去只怕会被拖垮的。人近了,一个目力绝佳的侍卫高声道:“殿下,是李顺李爷。”众人这才放下心来。

小顺子原本就落在后面,马术又不如雍王这些人精良,所以索性隐在暗处,换了一身禁军的衣甲,跟在闻紫烟等人的身后,虽然也只有一匹马,可是他一直用轻身术尽量减少马的负担,这才跟上了闻紫烟率领的禁军。他原本想趁机刺杀,可是闻紫烟率领的那支禁军也是大雍的精兵,那里是那么容易混入的,虽然几次趁着他们住马判断雍王等人的去向或者给马饮水的时候发起袭击,可是收效都不大,最后一次还被闻紫烟带人围住了,幸好小顺子机灵得很,事先预备了退路,这才得以脱身,看这样子不会起什么作用,小顺子便一心追上雍王,凭着高明的追踪之术和几分运气,终于被他先追上了雍王。远远的看见雍王的金甲,小顺子大喜,应该可以见到公子了,他可是十分担心江哲在乱军之中受害呢。

可是小顺子离雍王等人越近,心中就越来越不安,面色也是越来越冷厉,来到雍王面前,他劈头问道:“公子怎么不在这里?”

若是别人这样问,雍王就是有意说明,也要震怒的,毕竟君臣之别,上下尊卑之分是不能含糊的,可是小顺子这样厉声喝问,就是包括雍王在内也无人发怒,谁不知道此人心中只有一个主子,他奉了江哲之命,去向裴云传令,才能够让众人突围成功,此刻他身上皆是血迹,想起他平日纤尘不染的形象,更是让人无法对他生气。雍王坦然道:“随云留在了猎宫。”

小顺子一听之下,神色大变,杀气冲天而起,眼中寒光乍现,恶狠狠的盯着雍王,众人下意识的将雍王护住,这时荆迟上前道:“李爷,是江先生自己的决定。”小顺子看了他一眼,目光变得有些柔和,毕竟这个荆迟常年出入寒园,自己多次监督他抄书,还算是熟稔。

李贽见他已经心气渐平,策马上前低声说了一句话,小顺子眼中闪过异色,继而躬身施礼道:“奴才冒犯殿下,请殿下恕罪。”

李贽笑道:“你能够谅解就好,本王也是觉得若是带随云同行,只怕九死一生,这样却还多了几分生机,你若是担忧,不妨赶回猎宫,凭你的武功,应该可以保护随云周全。”

小顺子却是神色凛然,淡淡道:“不,奴才请命,亲自去见秦勇。”

李贽惊道:“这是为何,你不担心随云的安危么?”

小顺子冷冷道:“我家公子若是有了意外,奴才就是粉身碎骨也要将仇人满门杀死,可是如今公子生死未明,若是公子的计策失败,现在就已经落入敌手,只怕是有死无生,我就是赶去也没有用处,若是公子在生,那么奴才孤身一人也不能将公子从重围之中救出,既然如此,我便只有尽力而为,让公子早脱险境。如今殿下孤军在此,后面的追兵半个时辰之内就会赶到,殿下的大军和齐王大军恐怕都无法赶来,双方互相监视,没有一方可以脱离战场,那么秦大将军的军队就是殿下唯一的生机,可是殿下不能指挥秦军,若是太子一方得到兵符圣旨,秦军还会成为殿下的敌人,唯今之际,只有让秦军支持殿下,殿下才能在此战中获胜,这样也才能救出我家公子,这件事情只有李顺可以去做,秦勇身边我家公子曾有安排,此事只有我清楚,公子虽然没说,我却知道他的意思。”

李贽神色大振道:“原来如此,你们主仆都是智勇无双之士,那么本王重托于你。”说着将一个小锦囊递给小顺子,小顺子接过来,也不察看,淡淡道:“殿下小心,秦军就算是能够来帮助殿下,也不是短时间可以来的,殿下虽然兵法战略过人,可是闻紫烟兵强马壮,武功高强,殿下必然是万分艰辛,可是只要殿下拖延两日,奴才保证,可以让秦军赶来救护殿下,若是奴才失败,那也没有什么好说,奴才主仆陪着殿下一死就是。”

李贽道:“本王素来知道你的本事,你尽力而为就是,两日之约,本王还是有几分把握的。”

小顺子轻轻一礼,转身策马而去。李贽看着他的背影,高声道:“再休息一会儿,我们继续赶路,如果能和长孙将军会合,至少可以安全无虞。就是不能会合大军,也不能和闻紫烟正面交锋。”

众人同声应是,各自抓紧时间休息去了,李贽望望初升的太阳,道:“随云,本王可要指望你了。”

在雍王突围之时,猎宫之中已经是全部惊动,李援正在和皇后贵妃们一起用膳,他皱眉道:“冷川,外面发生了什么事情?”

冷川应声走出殿门,然后就看到了远处的火光,他心中一凛,猎宫之中起火绝对不同寻常,更何况还有隐隐传来的厮杀声,他连忙返回殿中,禀报道:“陛下,好像发生了变乱,陛下,要不要召见秦大将军。”

这时,外面传来沉重的脚步声,有人沉声道:“秦彝、程殊求见陛下。”

李援连忙道:“进来。”

随着他的喊声,秦彝和程殊匆匆走了进来。李援劈头问道:“秦卿,发生了什么事情?”

秦彝神色沉重地道:“陛下,韦膺可在?”

李援愣了一下,道:“今夜朕不用他拟旨,让他下去休息了。”

秦彝神色大变道:“方才韦膺前来传旨,说皇上召见臣和魏国公,可是臣刚到这里,就发觉西宫那边起了变乱。”

李援怒道:“这是怎么回事,朕没有让韦膺传旨,秦爱卿你可看到圣旨。”

秦彝苦笑道:“他说是陛下口谕,诏臣询问猎宫布防。”

程殊急急道:“殿下,恐怕是有人要造反,应该快些召集禁军和侍卫护驾。”

秦彝脸色一变,今日负责守护晓霜殿的应该是秦青,为什么自己进来的时候却没看到,他也顾不上请示李援,冲出殿门,高声道:“秦青,秦青,快给我滚过来。”

可是秦彝发觉除了守卫的大内侍卫应声望来之外,四周禁卫都是一声不吭,手握刀柄,秦青更是影踪不见。秦彝的心渐渐沉了下去,他从未像今日这样痛恨自己的姑息,自己怎么会没有想到,这营禁军在秦青统领之下已有很长时间了,那么凤仪门可能已经插手进去,这个无能的逆子。

李援也已经走出殿门,高声道:“还不快去召来秦青将军。”

这时,远处传来银铃一般的笑声,远处走来了一群女子,为首的正是李寒幽,她的身边是齐王妃秦铮和另外一个俏丽少女,三人都穿着月白劲装,身佩长剑,在她们身后,三十六名雪衣女剑手分成四列,她们周身都洋溢着冰冷杀机,而且步伐矫健,行动之间彼此呼应,杀气更是成倍的增长。

李寒幽走到阶下,裣衽一礼道:“陛下,臣妾奉太子之命,讨伐叛逆雍王,殿下担心陛下安危,特遣臣妾前来保护陛下。”

李援面色阴冷,他冷冷道:“你们以为可以做到么?”冷川走到他的身边,李援冷冷道:“你们是不可能控制所有禁军的,只要朕登高一呼,那些禁军便会倒戈。”

李寒幽冷笑道:“陛下说得不错,大将军治军严谨,我们确实没有办法控制整个禁军,甚至现在,我们也只是能够控制这五千禁军罢了,还要派出两千禁军追杀雍王,不过这就足够了,只要皇上出不了晓霜殿,那么臣妾就可以控制整个禁军。”

李援面色大变,道:“你们盗走了朕的金牌。”

李寒幽笑道:“陛下果然英明,能够完全控制禁军的只有秦大将军本人和陛下您的金牌,现在秦大将军身在此处,陛下的金牌在我们手上,陛下您已经无能为力,等我们将叛逆一网成擒,到时候,太子殿下自会来向陛下请罪。”

李援身躯有些颤抖,无比的愤怒让他几乎站立不住,他冷冷道:“是谁偷了朕的金牌。”

这时从殿内走出了皇后和三位贵妃,皇后面如寒霜,纪贵妃微微浅笑,长孙贵妃浑身颤抖,而颜贵妃惊惧交加。李援的目光落到纪贵妃身上,不可能的,他从来对纪贵妃防范很严,那么是谁呢,长孙贵妃绝对不会做这种事情,她没有理由这样做,颜贵妃温柔怯懦,更加不会这样做,那么只有一个人,他的目光落到了皇后身上。

\chapter{第二十七章 血溅行宫}

皇后窦氏眼中闪过一丝愧疚,但是转而变成得意和骄傲。李援冷冷道:“梓童,你本是皇后尊荣,却为什么要做这种事情?”

窦氏苦笑一声,道:“皇后尊荣?哼,臣妾只知道若是我儿不能继位,那么臣妾和他只有死路一条,如今陛下你意图废黜太子,改立雍王,又将臣妾和太子置于何地。”说到后来,窦氏渐渐有些声嘶力竭,语气也越来越激烈。

李援一愣,怒道:“朕什么时候要废黜太子了,你是听谁挑唆。”

窦氏眼中闪过愧色,避开了李援的目光,纪贵妃却轻轻一笑,道:“陛下,您的心意动摇,朝中上下人尽皆知,再说,太子做了一件错事,担心您的责罚,所以不得不说服皇后如此行事。”

李援目光一寒,望向秦彝,秦彝尴尬的道:“陛下,臣也是听到流言,说是太子*东宫属臣的妻室,造成人命,不过臣不便提起,这原本是谏官的职权。”

李援大怒道:“好个畜生,刚刚让他修心养性,却作出这种无耻之事,自古以来,君不君,臣不臣,朕定要……”说道这里李援沉默了,他看向皇后。窦氏面色苍白地道:“安儿对我哭诉,若是此事传入皇上的耳朵,只怕储位不保,臣妾只有这么一个儿子,无论如何也不能眼看着他走上绝路。”

李援惨然一笑道:“好好,多年夫妻,原来你只是惦记着那个逆子,也罢,也罢。”他的神色渐渐冰冷道:“纪霞、李寒幽,若是朕出了晓霜殿,只怕你们的计策也不会成功了。”

纪贵妃嫣然一笑道:“臣妾知道皇上这里有侍卫百人,可是臣妾相信绝不会让一个人脱身出去。”

李援冷冷一笑,高声道:“给我将这些叛逆全部杀了。”随着李援的语声,在偏殿隐身的侍卫们冲了出来,这原本是李援体恤他们,没有轮值的侍卫都在偏殿休息,所以虽然外面的侍卫已经被凤仪门清除干净,但是仍然有一支生力军存在。

李援一声令下,秦彝和程殊都挡在雍王和长孙贵妃前面,将两人护住,而冷川则扑向纪贵妃,纪贵妃甩去宫衣,露出一身黑色劲装,两人交战在一起,那些身穿黄衣的侍卫也和那些凤仪门女剑手交战起来,顷刻间,晓霜殿前成了修罗屠场。

颜贵妃惊恐的看着这场景,这时候秦铮已经扑了过来,高声道:“母妃,快和皇后娘娘一起避到殿中。”

颜贵妃虽然平素软弱,可是此刻她犹豫了一下,却叫道:“皇上,臣妾实在不知道这件事情。”说罢向李援扑去。

秦铮一愣,原本伸手要拦,却终于没有伸出手去,李援眉头一皱,看向满面惶急的颜贵妃,他知道这个妃子平日最是温顺柔弱,确实不可能参与谋逆之事。便叹了一口气,任凭颜贵妃扑到自己怀中,秦彝和程殊原本已经准备出手,可是颜贵妃身份贵重,两人都没有敢出手,这一犹豫,颜贵妃已经扑到李援怀中,李援将她交给长孙贵妃,两位贵妃相互扶持,都是惊骇的看着阶下。

秦铮一跺脚,已经扑上去和纪霞联手对付冷川,纪霞多年来担负着保护雍帝的责任,和冷川更是常常合作,所以对冷川的武功十分了解,而秦铮虽然很少出手,可是她天资聪颖,剑法高强,两人将冷川困住,虽然不能取胜,可是冷川也别想突破她们的联手。

这时候,下面的那些侍卫的情况就要不利多了。他们虽然都是武功高强,又多半出身军旅,擅长联手作战,可是那些凤仪门的女弟子的剑阵却是狠辣歹毒,配合严密,她们互相支援,剑法狠辣,将那些侍卫分割开来,没有多少时候,地上已经到处都是尸体和鲜血。

李援心中焦虑,想不到凤仪门的剑阵如此厉害。这可怎么办才好。

李寒幽一边指挥若定,一边自己也震惊这些女剑手的武功,可惜将来自己不能掌控她们,那样一来,自己岂不是始终为人作嫁,她一边盘算着如果夺得这些女剑手的控制权,一边留意场中各人的动向,只见冷川虽然仍然占着上风,可是已经无力脱身,而自己带来的三十六名女剑手布成的天罡剑阵,正在迅速的吞噬着生命,看来想要尽快解决,只有去面对皇上了。她带着谢晓彤向李援走去,面若寒霜。

秦彝和程殊都是面显忧色,若是沙场征战,他们自己无所畏惧,可是这种江湖厮杀,他们就没有把握对付李寒幽和那个凤仪门女弟子了。大雍和别国不同,武功高手多在军中效力,反而皇宫之中的高手不免少了一些,平日还看不出来,因为凤仪门负担了很大一部分的防卫工作,所以一旦凤仪门倒戈,雍帝身边的防卫力量立刻大大削弱。当然,凤仪门剑法高明,这也是如今凤仪门稳占上风的缘故之一。

这时,谢晓彤突然拉住她的衣袖,低声道:“外面有人喧哗,师妹,得去看一看,现在可不能让人知道咱们在逼宫。”

李寒幽眉头一皱,道:“你留在这里,我去看看。”

说罢飞也似的出了宫门,外面正是她们可以控制的三千禁军,将晓霜殿和四周围得水泄不通,这时,只见宫门处,一个宫装女子厉声道:“本宫乃皇室公主,要去向父皇请安,谁敢拦我道路。”却正是长乐公主,带着几个宫女和一个小太监。

李寒幽眼睛一亮,控制公主在手,不怕李援不妥协吧,她走近长乐公主,冷冷一笑道:“公主殿下怎么到了这里,路上没有人阻拦么。”

长乐公主望向她,眼中满含莫名的情绪,冷冷道“本宫见宫中震动,担忧父皇和母妃,故而前来问安,一路上虽然有人拦阻,可是谁敢真的为难本宫,李寒幽,你为什么在这里?父皇和母妃可还平安。”

李寒幽看向长乐公主,只见她平日清冷的容颜突然平添了几分皇室的威仪,怪不得无人敢拦阻,毕竟外面那些禁军只是受了自己的蒙骗罢了,怪不得竟然让长乐公主来到晓霜殿外,不过这样也好,李援宠爱长乐公主,恐怕可以迫使李援屈服,若是李援真要拼个鱼死网破,只怕将来不好收场。于是,李寒幽冷冷道:“雍王叛乱,靖江特来护驾,公主殿下请。”长乐公主眼中闪过冷厉的光芒,淡淡道:“好,本宫正要去见父皇。”

说罢长乐公主举步向内走去,她身边的几个宫女连忙跟上,那些禁军正要拦阻,李寒幽却一摆手,心道:“这些人进来正好,难道还要他们出去胡说八道么?”

长乐公主走进宫门,边看到遍地血腥,她的娇躯摇摇欲坠,这时长孙贵妃在高处已经看到她,惊呼道:“贞儿。”就要走下,却被李援挡住。李援看看站在长乐公主身边的李寒幽,怒道:“李寒幽,你也是宗室,朕又赐封你为公主,想不到,你却如此忘恩负义。”他这句话,秦彝、程殊和长乐公主都是脸色剧变,可是李寒幽羞恼之下,没有留神,只是笑道:“陛下,若是您肯退让一步,臣妾万死不敢冒犯,否则——”她看向长乐公主,这时候长乐公主已经恢复正常,她看也不看李寒幽,高声道:“父皇,儿臣有事启奏,请父皇暂息雷霆之怒。”

李援心中一动,再看看如今局势对自己不利,便长叹道:“也好,长乐,就听听你要说些什么?都给朕退下。”

李寒幽心中一喜,反正她也不怕李援逃出生天,便也一挥手,那些女剑手飞速退到李寒幽身后,那些幸存的侍卫则退到阶前,护住了李援等人,只剩下窦皇后孤零零的站在一边。

长乐公主看了一眼李寒幽,冷冷道:“总不能在大庭广众谈论这些事情,靖江若是没有意见,我们不妨进殿中商谈。”

李寒幽只要事情容易解决,便乐得大度,笑道:“正该如此。”

李援、秦彝等人心中都是一喜,这样一来,他们就可以凭借房屋布防,不由都对长乐公主刮目相看。当下,凤仪门剑手将晓霜殿围住,李援等人小心翼翼的进了晓霜殿,那些侍卫控制住各方出入口,李援坐在龙椅之上,秦彝和程殊分立左右,纪贵妃和李寒幽站在对面,双方对峙,气氛沉闷,都不知该如何开口。这时,长乐公主站起,先对李援施了一礼,方道:“靖江公主,不论你们如何狡辩,如今总是在围攻父皇,这是犯上作乱之举,不论是太子还是二皇兄,对于这种事情恐怕都不能容忍,而且,你们的目的不过是要暂时让父皇在晓霜殿休息,若是用强,迫得父皇不能接受,对你们也没有什么好处,你若肯平心静气和父皇谈上一谈,商议几个条件,不剩过现在这样打打杀杀么,再说,二皇兄如今已经突围出去,你们的要务可不是在这里纠缠。”

李寒幽神色一变,长乐公主所说她自然明白,可是她所要求的,李援岂肯答应,她看了纪贵妃一眼,眼中透出询问之色。纪贵妃笑道:“长乐果然是明理之人,我们要求也不多,请皇上和秦大将军交出兵符,让我们可以调动秦大将军的军队,事成之后,太子自然是要来向皇上请罪的。”

李援等人面上露出怒色,正要拒绝,长乐公主已经道:“此事事关重大,不可贸然决定,不如几位先到外面等一下,容我们商量一下。”

李寒幽想了一想,道:“一拄香时间,可够么?”

她的要求很是苛刻,可是长乐公主却立刻道:“时间足够了,请几位先到外面等上一等,容本宫劝解父皇。颜贵妃,您不想问问六皇兄的情况么?”

颜贵妃正是六神无主的时候,听到长乐公主的话,便道:“铮儿,显儿在哪里,本宫不信他会作出这种无君无父的事情。”

秦铮为难的看了李寒幽一眼,李寒幽淡淡道:“你去和娘娘说明一下。”说罢转身走出殿门。纪贵妃也笑着招呼窦皇后和颜贵妃到偏殿相谈,当下殿中只剩下长乐公主和李援等人。

李援见人走了,才疑惑地问道:“长乐,你在搞什么鬼?”

长乐公主微微一笑道:“父皇,现在局势险恶,但是二皇兄已经逃了出去,勤王救驾也是指日可成,若是父皇出了意外,却怎么拨乱反正,所以父皇不妨暂时隐忍,想必他们捉到二皇兄之前,是不敢对父皇动手的,父皇也可暂时保全一部分力量,免得到时候他们狗急跳墙,伤害了父皇母妃。”

李援叹息道:“朕何尝不知道这个道理,可是她们的要求太苛刻,若是将兵符给了他们,别说你二皇兄没有了生机,就是朕,也成了人家的囊中之物。”

长乐公主道:“父皇,这一点不必担心,他们要兵符圣旨,就给他们,可是指挥秦大将军军队的乃是大将军心腹,难道就没有私下的信物么,到时候再加上父皇一道密旨,不就成了。”

秦彝神色一动,道:“皇上,这倒可行,秦勇是我族侄,对皇室忠心耿耿,请皇上写一道密旨,盖上私章,他是认识的,再加上我的信物,定然可以调他前来勤王。”

李援神色一喜,道:“好,长乐真是心思细密。”可是看了一下,身边却没有纸笔。长乐却从怀中取出一方白色绫帕,道:“父皇,只要你盖上私章即可,稍后自然有人写上旨意。”

李援神色犹豫,他此刻心中实在有些不敢相信任何人,长乐公主见状连忙道:“父皇,儿臣也是没有法子,若是父皇您写了旨意,这封密旨绝对送不出去,父皇,您也知道,儿臣和太子素来有些嫌隙,难道还会替他们出力么?”

李援又看了长乐一眼,终于摘下手上的扳指,在绫帕之上盖了私章。长乐公主连忙将绫帕接了过来,又看向秦彝,秦彝却是毫不犹豫,将一块玉佩递给长乐公主,这块玉佩十分普通,长乐公主不由有些疑惑,秦彝道:“这是勇儿送给我的寿礼,他一定认得。”长乐公主这才放下心来,道:“大将军,秦青将军恐怕已经被凤仪门所拘禁,待会儿不妨要求他们将秦将军送来。”秦彝神色一黯,没有说话。

这时,李寒幽高声道:“时间到了,本宫进来了。”这次进来,李寒幽满面寒霜,看来是一定要个结果了。长乐公主不卑不亢地道:“靖江,父皇已经同意你们的要求,可是我们也有条件。”

李寒幽神色一动,道:“只要合情合理,我们都可以商量。”

长乐公主笑道:“这些条件并不苛刻,第一,若是没有二皇兄亲来,或者见到二皇兄的首级,你们不许再来骚扰父皇。”

李寒幽干脆地道:“这一点没有问题,叛逆不除,我们自然不会来打扰陛下。”

长乐公主淡淡道:“第二个条件,秦青将军恐怕已经被你们所制,将他送来应该没有问题吧?”

李寒幽冷冷一笑,心道,秦青已经没有用处了,便道:“这一点也没有问题,稍后本宫就将人送来。”她虽然没有流露什么表情,可是这殿中谁不是察言观色的高手,立刻看穿了她的心思,更是多了几分厌恶。

长乐公主微微一笑,道:“这第三个条件却是为了本宫提的,本宫和母妃的侍女都在含香苑中,现在猎宫之中一片混乱,本宫想让那几个侍女也到晓霜殿来,不知道可否允许呢?”

李寒幽心想,就是你不提,我也不能让你回去含香苑,点头道:“这是当然,本宫这就派人将她们接来。”

长乐公主却道:“且慢,请带他同去,本宫离开含香苑的时候,曾经有话,除非本宫命令,否则不许她们擅离含香苑半步,让这个奴才回去传我的命令,也免得多生是非。”

李寒幽原本要拒绝,可是听到最后一句,却也觉得有理,有些事情,宁为人知,莫为人见,若是弄得人尽皆知,就是将来灭口也是麻烦。她看了纪贵妃一眼,见她轻轻点头,便道:“也好,就是这样吧。”

长乐公主微笑道:“那么就请靖江你去办吧,若是没有问题,等到秦青将军和本宫的侍女来到之后,父皇就会将兵符给你。”

李寒幽目中光芒一闪,道:“若是本宫履行了承诺,皇上却又反悔,那该如何,本宫可没有那么多时间和你们纠缠。”

李援冷冷一哼,长乐公主却冷然道:“若是如此,本宫就将性命给你。”

李寒幽得意的一笑,道:“好,君子一言,快马加鞭。”说罢抬起右手,长乐公主淡淡一笑,走上前来,举起纤纤素手,两人击掌为誓,四目相对,两人目中都闪过一丝寒芒。

长乐公主又是淡淡一笑,拿出一块玉佩,玉佩外面裹着一条雪白绫帕,长乐公主将玉佩递给小六子,道:“你去告诉周尚仪,让她带着咱们的人都到这里来。”

李寒幽用目瞧去,这条绫帕大半露在外面,并无文字墨迹,便没有上前查看,毕竟她也不想过于得罪皇室,无论如何,将来凤仪门都是要通过大雍皇室来控制政局的。

小六子接过玉佩和绫帕,恭恭敬敬的告退,李寒幽做了一个手势,谢晓彤带了两个凤仪门女剑手跟了上去。

长乐公主吁了一口气,终于完成了那人托付的事情,她含笑看向李寒幽,道:“大概还得等上片刻,靖江可要喝杯茶么?”

\chapter{第二十八章 含香惊魂}

这时,猎宫之内已经是渐渐平定下来,雍王突围而出,虽然给凤仪门造成了很大的麻烦,可是也减弱了猎宫之中的反抗力量,韦膺凭着执掌禁军的令牌,很快就控制住了局势,剩下的禁军,除了控制晓霜殿的三千禁军是凤仪门完全控制的之外,保护玉麟殿的禁军已经换上了夏侯沅峰的一千禁军,其余的禁军皆被打乱编制,派到各处控制猎宫,所有的随驾大臣都被软禁起来,就是其中有倾向凤仪门的也不例外,当然这些大臣若没有必要,也不想真的涉入叛乱,有碍声名。

韦膺带着禁军四处巡视,他要确认没有残余的反抗力量。原本文雅俊秀的面容上带着淡淡的杀气,全然没有了从前从容自若的风度。

此刻他的心中焦虑非常,可是奇怪的是,脑海中却想起从前的事情来,韦家和凤仪门的关系从来不为人知,谁会想到韦夫人竟然和凤仪门主乃是金兰姐妹,韦膺出生之后不久,就被凤仪门主看中,秘密的传授给他武功,而韦膺也不负凤仪门主所望,成了一个文武双全的俊杰之才,由于韦家一直以中立自许,所以没有人知道韦膺乃是凤仪门主唯一的男性记名弟子。

随着韦膺长大,他和凤仪门渐渐疏远,毕竟身为丞相之子,又是人人称誉的年少英才,他的前途不可限量,若是和凤仪门关系密切,反而会造成皇室的疑忌和排挤,因此他几乎从来不显示自己的武功,一心一意的要做相阁之才,可是就在他春风得意的时候,意想不到的打击来了。

不管是太子的计划还是凤仪门的假公济私,他成了皇上选中的驸马,长乐公主的未婚夫婿,坦白说,他对长乐公主并没有什么情意,毕竟对于外表谦抑,内心高傲的他来说,长乐公主并非他梦寐以求的妻子,可是娶到公主对他意味着什么,他却是很清楚的,所以他欣然接受了皇上的安排,可是打击随之而来,长乐公主宁可出家也不肯下嫁,这让一向顺风顺水的韦膺心中涌起前所未有的愤怒。也就在这一年,他开始和凤仪门接近,只是在这个过程中,他总是蒙面而行,除了凤仪门主之外,没有人知道这位深得帝宠的韦大人,竟然成了凤仪门主亲自封赐的护法。

初时韦膺还是不想谋反的,甚至几次故意延宕了凤仪门主的决定,对他来说,十年之后执掌相位是很容易的事情,没有必要这样冒着身家之险。可是,当凤仪门主提出那个计划的时候,他还是没有办法拒绝,得到长乐公主已经是他晋身皇室的唯一途径,所以他任凭凤仪门主主导了那场闹剧,甚至事前,他凭着温文儒雅的外表气度和温柔甜蜜的言辞,暗中取得了绿娥的芳心。因为每次长乐公主故意避开他的时候,绿娥都不免奉命来敷衍推辞,韦膺趁机骗取了少女的一片真心。而那一天,满心期望能够陪着公主嫁到韦家的绿娥果然处处装着糊涂,若非是长乐公主的亲生母妃赶到,想必长乐公主已经被迫嫁给他了,可是那一天,韦膺知道,自己再也没有任何机会。眼看着青云之路被拦腰斩断,他终于下定了决心,只要扶保太子登基,那么凭着自己的功劳,要想迎娶公主就绝对没有问题。

可是世事总是不如人意,雍王不知如何拆穿了他天衣无缝的骗局,竟然冒险突围成功,这让他心中充满了恐慌,虽然李寒幽已经去逼取兵符,好调动秦军追捕雍王,可是万一失败那,韦膺从未像现在这样觉得苦恼和忧虑,所以在清除反抗势力之时,他前所未有的辣手无情,这一路行来,已经是十多名官员因为反抗而被他斩杀,鲜血,染满了猎宫禁苑。

令韦膺恼怒的是,李寒幽和萧兰等人商议之后,也不知会韦膺,就将太子少傅鲁敬忠软禁在玉麟殿内,原因是凤仪门众女都觉得鲁敬忠将来必是敌手,与其让鲁敬忠从中搅局,损害了凤仪门的利益,不如趁机将他杀了,幸好韦膺及时赶到,可是木已成舟,既然已经得罪了鲁敬忠,总不能再得罪了李寒幽和萧兰,无奈之下韦膺只得同意将鲁敬忠暂时软禁起来。可是对于凤仪门众女不顾大局,大事未成就先斩断臂膀的行为,韦膺却是深恶痛绝。

一边巡视,一边想着如何控制大局,韦膺走到含香苑的时候,突然心中一动,对他来说,不论谋反成功得到什么利益,都不如长乐公主的下嫁重要,走到这里,他突然想到,现在长乐公主一定是为了外面发生的事情而心中惴惴不安,自己若是趁机前去安慰,或可得到公主放心,想到这里,他便向含香苑走去,守门的禁军并非凤仪门和太子一系,可是看到韦膺,却都不敢阻拦,毕竟他们不是傻子,这猎宫之中发生事故还是知道的,可是皇上和秦大将军踪影不见,这些禁军也不敢妄自行动,毕竟这是皇室的内乱,若是他们站错了位置,可是要丧命的,而韦膺在他们眼中就是皇上的使者,毕竟掌控禁军的金牌就在他手中。走进含香苑,韦膺只觉得一阵萧瑟之意,满园的菊花透着萧杀的气息。他走到公主寝殿阶前,高声道:“臣韦膺求见公主殿下。”

殿内一片静寂,良久,一个三十多岁,相貌端庄秀丽的宫女走了出来,道:“翠鸾殿尚仪周氏见过韦大人,公主殿下已经去了晓霜殿,不在这里。”

韦膺一愣,道:“猎宫中现在一片混乱,怎么周尚仪会让公主去了晓霜殿?”

周尚仪裣衽道:“奴婢怎敢阻拦公主的行动,公主担心皇上和贵妃娘娘的安危,这才去了晓霜殿。”

韦膺面上露出失望的神色,突然之间,他发觉周尚仪神色有些慌乱,脑中千丝万绪,雍王突围,可是江哲却没有随行,至少没有人看到,自己搜遍雍王住处火焚之后的废墟,却不见尸体,那么江哲有可能还在宫中,自己四处巡视,也有搜查此人的打算,只是还不确定此人是否真的留下,才没有大举搜查,毕竟现在凤仪门的优势实际上只是镜花水月,只要有人登高一呼,只怕那些禁军就会控制不住,想起传言,长乐公主和那江哲颇有私情,若是此言当真,那么江哲有可能就在含香苑中,想到这里,韦膺露出冷笑道:“既然这样,就让本官搜一搜含香苑,现在宫中叛逆还未铲除干净,若是惊吓了公主,本官担当不起。”

周尚仪大惊,她可是知道这含香苑是搜不得的,就在夜中火光初起之时,长乐公主的寝殿突然来了不速之客,周尚仪虽然没有见过,却是知道这个人的,江哲江司马,南楚才子,雍王心腹,也是长乐公主的意中人。扶持他的是一个相貌清秀,气质冰寒的青年,周尚仪曾经听说过江哲身边有一个南楚宦官出身的仆人,可是这人怎么看上去也不像。这两人来的隐秘,竟是直接闯入了公主的寝殿,当时只有周尚仪相陪。然后那个文弱憔悴的青年让自己和他的仆人到外面守着,他和公主秘密谈了很久,然后长乐公主便带着几个宫女和那个小太监小六子去了晓霜殿,临行嘱咐周尚仪好好照顾江司马,还不能让别人发现。可是如今韦膺要搜查含香苑,那可怎么办,公主可是说过了,韦膺是叛逆一党。她的神色变化俱被韦膺看在眼里,他心中又喜又妒,若是捉到江哲,那么等于是将雍王的一切机密掌握在手中。他正要进殿搜查,却想起“邪影”李顺来,若是邪影在江哲身边,那么自己等于是自投罗网,韦膺并没有得到闻紫烟的回报,还不知道小顺子已经突围出去,邪影忠于江哲,这是人尽皆知的事情,不知多少人为此扼腕呢,韦膺可没有胆子去面对那种高手,狠狠心,韦膺下令道:“去召集禁军将这里围住,再去兰妃娘娘那里调几个剑手过来。” 原本为了避嫌,他是没有留凤仪门的剑手在身边的,可是现在,若是没有那些凶悍的剑手,他可不放心就这么闯进去。

含香苑,公主的寝宫之内,我坐在软榻之上,心中计算着胜负的可能,只是情况错综复杂,实在是难以计算,虽然不知道为什么禁军会倒戈,倒是我估计最可能的就是凤仪门发动了在后宫的力量,窃取符令,然后再隔绝皇上和外界的联系,这样凤仪门在局部就占据了优势,然后就可以使用矫诏发动皇上的全部力量围剿雍王,谁会想到,在这个皇上势力最大的地方会出现这种事情,这也是我几次取胜之后低估了凤仪门在后宫的力量的结果,可是目前不是考虑这个的时候,长乐公主是我唯一能够扭转乾坤的途径,否则我就是尽了全力,最多也是一个两败俱伤的结局,那是大雍承受不起的。

而且拿到皇上的密旨和秦大将军的信物之后,最重要的就是如何将这些安全的送出去,这个人选我虽然已经选定,可是却是没有把握的,若是一旦失败,那就是万劫不复,不行,我的眼中闪过一丝无情的光芒,若是这人有不妥,我必须立刻杀了他,绝对不能让他有机会说出去,到时候只好让董缺去了,可是董缺并不安全,他很可能半路上就被凤仪门的人截杀下来。

正在我苦思冥想的时候,窗棂一响,董缺飘然进来,低声道:“公子,事情已经办好,他一会儿就到。”

我沉声道:“他可靠么?”

董缺道:“公子放心,我师兄东宫事变之后,被李寒幽软禁起来,直到日前,才被太子放了出来,师兄对凤仪门和太子已经是心灰意冷,所以我一以大义相责,他就同意了。”

我心中一宽,道:“他认出你了么?”

董缺苦笑道:“看来我的改变真的很大,师兄虽然有些疑惑,可是没有认出我来,若非我拿了雍王金牌,他还不会相信我呢。”

我微微一笑道:“那就好,还有你不要介意,一会儿我会在你师兄身上加上禁制,这也是不得已的事,这是雍王殿下唯一逆转局势的可能,我不能掉以轻心。”

董缺点点头道:“师兄会明白的,而且我清楚的很,雍王只要逃了出去,就是暂时势弱,过些时候也能够力挽狂澜,只是损失大些,师兄为了师门着想,也会同意公子的安排。”

我正要说下去,突然耳边传来脚步声和周尚仪焦急的声音道:“韦大人,你不能搜查公主的寝宫,这太无礼了。”

我心中一声哀鸣,怎么韦膺会到了这里,难道真的是我气数已尽。连忙打量一下寝宫,我一直想着如何对付凤仪门,却忘了找一个隐身的所在。董缺微微苦笑,上前将我扯住,轻轻一指床榻,我用莫名其妙的眼光看着他。他轻轻上前,在床榻上错落有致的拍了几掌,然后床板无声无息的滑开,露出下面的暗格,里面勉强可以容纳一个人,我瞪大了眼睛,这里怎么会有暗格。董缺也不理会我的疑惑,一把将我提了起来,在我身上点了几下,我只觉得神智渐渐模糊,隐隐约约的好像被塞进暗格里面,然后眼前就是一片黑暗。

韦膺令人将含香苑的宫女太监全部赶到一间偏殿里面,自己带人搜查了起来,接到他的指令,萧兰派了凤非非过来,太子那里一片平静,自然是用不到那么多人手的,两人将其余房间搜查了一遍。却是没有发现,最后两人的目光都集中在公主的寝宫上。韦膺犹豫了一下,若是真的搜查公主的寝宫,不论是否能够搜出人来,只怕长乐公主都会对自己心生怨恨,可是转念一想,若是搜出人来,或者可以迫使公主屈服,因此,韦膺对凤非非道:“这里是公主寝宫,我不便搜查,还请三姑娘代劳。”

凤非非微微一笑,秀美的面容上带了飘逸柔和的笑容,轻轻理了一理鬓角,她柔声道:“若是能够捉到江哲,师尊一定是非常高兴。”她只道韦膺害怕邪影李顺,心中有些鄙夷,便提剑走进寝宫。

含香苑本来就是给贵妃或者公主所住的宫殿,地位稍低的妃嫔和宗室都没有资格住进来,一走进寝宫,只觉修饰华美,清雅高贵,凤非非淡淡一笑,虽然名义上也是公主,寒幽师妹所住的地方可是比这里差远了,她细细的搜索了一遍,却是没有丝毫发现,机关暗器她虽然并非十分精通,可是这宫中没有暗道密室却是可以确定的,最后她的目光集中到了床榻之上。这张床榻乃是沉香木所制,精美非常,香气优雅,凤非非走近床榻,仔细检查了半天,这整张床榻浑然一体,是不可能有机关的,不过凤非非有些羡慕的看了这张床榻一眼,这才走出宫去。

看到韦膺,她微微摇头,韦膺懊恼的皱皱眉,凭白无故的再次得罪长乐公主,真是得不偿失。正在这时,谢晓彤和两个凤仪门女剑手带着一个小太监走了过来,一看到凤非非,谢晓彤便兴奋地道:“三姐,我们那边快成功了。”说着飞快的将晓霜殿那边的事情说了一遍,她言词伶俐,说得很清楚。

凤非非眼中闪过一丝喜色,道:“想不到长乐公主却是如此识趣,可是我们这边却搜了含香苑,不知道事情会不会因此生变。”说着有些忧虑和恼怒的看了韦膺一眼。韦膺微微一笑,凤非非这些人无论如何都是女子,虽然够狠毒,可是却不够果决,也难怪凤仪门主不让她们负责此事。可是他也不想得罪她们,便淡淡道:“只要警告一下,你们还怕这些下人敢多说什么,只要过了这几天,就算他们说了出去又有什么关系,长乐公主又不会回到含香苑,这件事情暂时她不会知道的。”

谢晓彤点点头,道:“你快去办事吧。”她这句话是对着小六子说的,小六子满面惊慌的点着头,飞快的跑去见周尚仪,这些宫女太监飞快的收拾着东西,贵妃娘娘和长乐公主都有不少随身之物,收拾起来一时半会儿不会完,韦膺和凤非非也懒得看下去,和谢晓彤交待了一声便离开了,韦膺等人离开之后,禁军也撤了下去,这时,菊花丛中一个身影悄悄站起,他身上披着一件薄薄的丝绸披风,上面的颜色和花丛颜色十分相近,那些禁军和韦膺都没有留心,毕竟他们的目的是寻找一个手无缚鸡之力的文弱书生。

他轻轻进了公主寝宫,这时,小六子和周尚仪已经等在那里,小六子一见他,低声道:“公子何在?”

董缺指了指床榻,周尚仪心里一宽,这张床榻乃是宫中密制,内有暗格,可是这件事情不是所有人都知道的,含香苑一年也使用不了几次,所以更没有人知道了,而长孙贵妃就是知道的一个,她当成玩笑将给了长乐公主听,昨夜江哲避难到此,他自己没有想到,长乐公主却想到若是有人搜查该怎么办,所以将这个所在告诉了董缺,反而是江哲心中都是如何逆转局势,反而没有注意这件事情。

周尚仪放心的点点头,现在还不是把江哲放出来的时候,小六子把公主交给他的绫帕和玉佩交给董缺,简单的说了一说情形,然后便和周尚仪收拾了公主的衣服首饰,匆匆离开了寝宫,没有多久,他们就跟着谢晓彤离开了含香苑,含香苑的苑门也被他们锁上了,这里就成了不被人注意的地方。

然后董缺才将点了穴道,气息微弱的江哲从暗格中抱了出来,只见他面容苍白,董缺连忙解开他的穴道,心道:“他可别有问题,这种手法是最轻的了。”

当我从昏迷中醒来的时候,就看到董缺焦急的面孔,我摇了摇沉重的脑袋,低声道:“人已经走了么?”

董缺道:“公子放心,韦膺已经走了,这是公主送来的。”说着将绫帕和玉佩递给我。

我展开绫帕,看到上面的印章,微微一笑,吩咐董缺拿来笔墨,迅速写了几行字“太子谋反,着秦勇听命雍王,猎宫救驾,其余矫诏兵符,不必奉行。”

放下笔,我微笑道:“只要把这两件东西送到秦勇手中,就不用担心了,对了,你的师兄能不能成为去传旨的使者?”

董缺正要答话,却听到外面又传来脚步声。两人心中都是一震,难道韦膺又回来了么?

\chapter{第二十九章 明暗信使}

脚步声停留在门前,过了一会儿,一个悦耳的声音道:“夏侯沅峰请见。”

我心中一震,看了一眼董缺,一把匕首正轻悄悄的落在他的右手,心中一叹,若是小顺子在,夏侯沅峰自然是可以轻而易举的拿住,可是若是董缺,恐怕就不行了,据小顺子估计,董缺的武功只是二流而已,虽然比从前高强了许多,可是若是动了兵器,只怕还是不行,让他留下来保护我除了其他人不合适之外,还有一个原因是他擅长很多鸡鸣狗盗的本事,这才是我倚重他的地方,反正若是真刀实枪的交手,就是小顺子在也没有用。所以索性用了董缺,可是现在可就为难了。

我使了一个眼色,道:“夏侯大人请进。”

门开了,夏侯沅峰一身黄色的侍卫服色,走进来之后,他躬身一礼道:“自从上次蒙大人开恩饶过性命之后,夏侯无时无刻不再惦念大人。”

我冷冷道:“夏侯大人言重了,上次蒙大人相告行刺哲的真凶,这是大人的好意,江某怎会恩将仇报,如今大人只手掌控江某生死,不知道旧日之事还有什么好提的呢?”

夏侯沅峰露出笑容,更加显得丰神如玉,他道:“韦膺等人虽然才智也还不错,可是这种事情不免有些欠缺,若是夏侯主持搜查,一定要派人多监视上半天,提防有人躲在暗处,或者回来这里。”

董缺眉头一皱,他也明白这个道理,可是时间紧迫,他又担心点了江哲的穴道时间太久会有害处。

夏侯沅峰见状神色更是柔和,目光落到书案上面的绫帕密旨上,他淡淡道:“请问江大人,不知道和公主如何商议,其实如今雍王虽然暂时脱险,但是闻紫烟正在追杀,若是没有援军,雍王迟早必然身陷罗网,下官也很想知道江大人如何力挽狂澜,才不负雍王首席智囊的身份啊?”

我神色渐渐从容,事情若是真的到了紧急时候,我从来都是越发冷静,拣了一张椅子坐下,我微笑道:“夏侯大人乃是太子心腹,为何不带了侍卫禁军过来将江某抓了,这可是大功一件。”

夏侯沅峰笑道:“如今太子仰仗凤仪门,就连凤仪门将鲁少傅软禁起来,太子也不敢过问,我就是立了大功也没有什么用处,更何况,邪影还在生,若是我将你献给太子,只怕没有几日,这条性命就会送掉。”

我心中疑惑,这也不是他放过我的理由,时间紧迫,我也不愿和他纠缠,便道:“小顺子虽然武功高强,却不过是一个人,夏侯大人将来是官高爵显,还怕他做什么?却不知夏侯大人希望江某替你做些什么?”

夏侯沅峰眉宇间闪过一丝喜色,道:“我的要求很简单,若是江大人肯割爱,将邪影送给我为奴,今日夏侯一定拼了性命保全大人。”

我只觉得脑子里轰的一声,差点失去了理智,幸好董缺及时的推了我一下,我忍着怒气道:“小顺子和我虽然名为主仆,却是情同骨肉,夏侯大人这个要求也太过分了。”

夏侯沅峰微微一笑道:“邪影对江大人视若父兄,忠诚不二,夏侯十分羡慕,想来若是江大人在我手下,邪影也会听命于我。”

我冷冷道:“夏侯大人,你太得意了,可是你却不该自己来的。”

夏侯沅峰看了一眼董缺,摇头道:“他不是我的对手,如果不是知道李顺护着雍王逃了出去,我也不敢独自来捉你,江大人放心,我绝不会将你交给太子和凤仪门,江大人才智过人,夏侯也很想恭聆教益。”

就在这时,董缺突然出手,一缕寒光向夏侯沅峰刺去,夏侯沅峰却是不慌不忙,出剑相迎。两人战在一起,身影在寝殿之内交错,剑光如同流星闪电,两人都是不想惊动他人,所以都很克制,没过多久,董缺已经渐渐不敌,他的长处本就不在武功上,对上夏侯这种武功高过他很多的人更是没有胜算。

又过了几招,夏侯沅峰已经一剑刺穿了董缺的大腿,董缺跌倒在地的一刻,就在这时,夏侯沅峰眼睛的余光看见江哲手中多了一把短剑,正在刺向心口,心中一急,连忙飞身扑向江哲,对他来说,江哲可是死不得的。就在这千钧一发之时,他突然看到从江哲腰间射出一簇寒芒,夏侯沅峰心中一惊,正要避开,却是人在半空,无法相避,而且那簇寒芒不仅快逾流光,而且角度十分刁钻,虽然夏侯沅峰极力避开,却仍然有小半射中了他的身躯。夏侯沅峰下意识的一掌击出,江哲向后跌倒。而夏侯沅峰只觉得浑身酸软无力,不由跌落在地上。这时董缺惊惶的扑了过来,俯身去看江哲的情形。

我悠悠醒来,看见董缺惊惶的神色,低声道:“我没有事情了,人抓住了么?”

董缺笑道:“公子的暗器果然厉害,夏侯沅峰中了之后立刻就不能动了。”

我这才松了一口气,方才我一直在想如何摆脱困境,因为我明白董缺不是夏侯沅峰的对手,唯一的可乘之机就是夏侯沅峰独自前来,我不是蠢人,小顺子的武功才智都是当世罕有,这样一个人才,屈居在我之下,不知道有多少人为他不平,也不知有多少人想招揽他,不过是碍着雍王罢了,夏侯沅峰野心不小,居然想打他的主意,不过也正因为这个缘故,他才不能将我交到凤仪门手上,既然他是独自前来,那么只要制住了他,我就安全了,可是这也是最难的事情,我一个手无缚鸡之力的文弱书生有什么法子制住一个绝顶高手呢?

幸好总算是被我想到了法子,他既然有所求,那么他就不能让我自尽,所以我在董缺落败之时,举剑自尽,在他来说,这符合我这个雍王的首席谋士的身份,宁死不辱,所以他飞身来救,就是他用其他方式打落我的短剑,也定会赶过来制住我的,而我就趁这个机会,将腰间玉带中暗藏的毒针射了出去,那些毒针原本上面淬着见血封喉的剧毒,可是前些日子,我换上了刚刚配制好的一种麻药,能够让人在呼吸之间软倒,只是时效很短。当然我还是遇到了想象中的危险,夏侯沅峰反击的一掌击中了我,幸好那时候他已经几乎力道全失,我这才保住了性命。

站起身来,看向神色有些狰狞的夏侯沅峰,我还是有些不放心,我那枚心爱的玄铁之精制成的发簪已经给了小顺子,他平日不用兵器,可是为了他的安全,昨夜突围之时,我将发簪给了他,那对他来说是比什么都厉害的兵器了。所以我摘下现在那根三分金七分精铁的发簪,尖锐的发簪刺入夏侯沅峰的几处隐穴,我这下可以确保他不能反击了,现在控制局势的已经是我了。

过了一会儿,夏侯沅峰开始能够活动了,可是他能够感觉到自己浑身的力量全部失去了,苦笑一下,道:“想不到江大人也有这等手段。”

我谦逊地道:“这实在是只能靠着出其不意才能得逞的小人伎俩。”

夏侯沅峰神色从容,仿佛现在成了阶下之囚的是我一样,他笑道:“不知道江大人要如何处置在下,若是下官突然失踪,只怕有人不会善罢甘休呢?”

我淡淡的看了他一眼,道:“你放心,我将你杀死之后藏在暗格之中,这样你就不用担心有人找到你的尸体了,说不定还会以为你私下逃了呢?雍王脱走,有些人心中可会很惧怕的。”听到我的话,董缺立刻又去打开了床上的暗格。我道:“董缺,别见血,免得血腥气太重,引起了别人注意。”董缺笑道:“属下遵命。”说罢,一指缓缓点向夏侯沅峰的死穴。

夏侯沅峰明明知道这两人存心吓唬自己,否则江哲何必只是禁制了自己的武功呢,可是恐惧还是从心中升起,那个董缺神色冷酷无情,一见就是杀人不眨眼的人物。这时江哲又道:“我可没有亲手杀过人,所以还是你动手吧。”这下,夏侯沅峰可是忍不住了,他是知道的,这些谋士大多都是君子远庖厨的奉行者,若是真的这样死了,可就太不值了,冷汗涔涔而下,他惊叫道:“江大人饶命,下官情愿投降。”可是江哲却没有出声,只是淡淡笑着,董缺的手指越来越近,终于一指点在夏侯沅峰的死穴之上,夏侯沅峰只觉得心胆俱寒,正要开口大叫,董缺已经伸手捂住了他的嘴,夏侯沅峰只觉得一阵头晕目眩,片刻,才清醒过来,却原来董缺指上只用了两分力,因此没有杀死夏侯沅峰,可是夏侯沅峰却是吓得面色惨白,他从未这样接近死亡过。

我坐下来,看着转瞬之间就恢复正常的夏侯沅峰,不由有些叹服,这人是个人才,心机深沉,随机应变,能屈能伸,可惜却是太子一党,有些惋惜的看向他,现在不是我发慈悲的时候,若是有了丝毫闪失,那么雍王可真是万劫不复了。

夏侯沅峰看到江哲冷淡中带着惋惜的眼神,心中一寒,方才虽然吓得他半死,可是他能够感觉得到江哲不过是相出出气罢了,可是现在,那种眼神,看来自己是非得死去了,连忙叫道:“江大人,就是不念在下当日向您透露刺客的一片好意,也请大人体念沅峰对公主的一片忠心。”

我愿本已经要下达诛杀令了,听他这样一说,我不由一愣,夏侯沅峰连忙道:“是下官向公主殿下禀明凤仪门有谋算公主之意的,公主当日宫中遇险,虽然不是下官相救,可是若非公主事先有了准备,怎会如此侥幸。”

听到这里,我心中一软,当日公主确实通过雍王妃告知我凤仪门的谋算,可是我和雍王殿下都以为凤仪门会通过威逼利诱的手段,可是没有想到他们竟然用了那样卑鄙的手段,若非我和小顺子事先安排了人,公主恐怕难免落入圈套,可是我还是得感谢夏侯沅峰的好意的,再次看向夏侯沅峰,我叹息道:“夏侯大人,你确实对公主有功,可是你也知道如今情形,你用什么可以说服我,让我觉得放了你是件值得的事情。”

夏侯沅峰开动脑筋,想着可以活命的法子,没有多久,他的目光落到书案上面,那方绫帕密旨,眼睛一亮,道:“除了在下,没有人可以更方便的将这些东西送出去,那是皇上的密旨吧,我想公主殿下兰心慧质,是绝不会做无用的事情的。”

我淡淡道:“你很聪明,可是这件事并非是非你不可。”

夏侯沅峰笑道:“皇上的旨意和秦大将军的兵符虽然已经到手,可是想要调动大军,必须有人去传旨,我不知道雍王殿下在太子身边的密探是谁,可是太子只会让心腹之人去传旨,凤仪门是不便出面的,如今太子的心腹不多,而鲁少傅就是其中之最,我是鲁少傅的师侄,除了我,还有谁更适合这项工作。

我听了眉头一皱,不错,张锦雄虽然可以要求前去,可是却是不如夏侯沅峰这样名正言顺,可是我可以信任他么?用疑惑的目光看向夏侯沅峰,这时候,外面传来几声鸟叫,董缺神色一动,看向我道:“公子?”

我心知是张锦雄到了,轻轻点头示意。

董缺走了出去,月光之下,一个相貌豪勇的大汉站在那里,看见董缺,他神色一宽,低声道:“我只有片刻时间,方才太子殿下和鲁少傅商议,要派夏侯大人前去传旨,张某随行保护,我托言出来寻找夏侯大人,才能来到这里。”

董缺心中一动,低声道:“请张总管稍侯,现在夏侯沅峰已经被我家公子所制,大人请到偏殿说话。”

张锦雄一愣,他可是知道夏侯沅峰的武功的,若是两人交手,他纵然不至于落败,要想取胜也很难,想不到夏侯沅峰竟被制住,不由对那位江哲江随云更加心仪。

两人进了偏殿,董缺走回公主寝殿,在江哲耳边低低的说了几句话。

我听了之后心中十分惊讶,不知道这是否老天爷的眷顾,想了一想,我从腰间玉带里面的暗格里拿出几颗药丸,看了半天,选定了其中一颗,看向夏侯沅峰道:“你将这颗药丸服了下去,我便相信你真心弃暗投明。你应该知道我是医圣传人,这种毒药不是没有解药,可是没有十天半月,解药是配不好的,你若是想要荣华富贵,太子可以给你,雍王也可以给你,但是你若想要性命,那么只有一条路可走。”

夏侯沅峰犹豫了一下,可是他本是果决之人,更何况如果不吃这粒毒药,那么根本就不可能走出含香苑,因此立刻接过药丸服了下去。我见他服下,又道:“还有一件事情,你和凤仪门既然共事太子,那么你可认得梁婉。”夏侯沅峰一愣,道:“下官认得,不过据说梁姑娘已经被毁去神智,虽然凤仪门讳莫如深,可是我听鲁少傅说过。”

我淡淡一笑,道:“当日用药物毒疯梁婉的就是在下。”

夏侯沅峰的眼睛瞪大了,不可置信地望着我道:“不可能,难道那时候你就已经投靠了雍王么?”

我一愣,立刻明白他的意思,便笑道:“此事与雍王殿下无关,梁婉是我杀妻仇人,我对付她不过是为了报仇。”

夏侯沅峰心中一寒,望向江哲,此刻他真的相信江哲有举手投足之间就可以杀死自己的本事,但是他却反而坦然起来,道:“不知道在下还有什么可以效力之处。”

我却有些疑惑起来,道:“夏侯大人为何这样说,看来倒是比江某更关心此事。”

夏侯沅峰笑道:“如今我既然已经受了大人控制,那么就是上了雍王殿下的船了,既然如此,我自然希望这船越稳越好,最好让我多立些功劳,也免得将来没机会加官进爵。”

我宽心的一笑,夏侯沅峰若是想加官进爵,我还放心一些呢。我挥手让董缺拿过那块绫帕,郑重地递给夏侯沅峰,夏侯沅峰也是神情郑重的接过,我深施一礼道:“这是圣上密旨,你一定要交给秦勇将军,让秦勇将军前来救驾勤王。”夏侯沅峰施礼道:“大人放心,夏侯必定不负所托,雍王殿下那边,还请大人多多美言。”

董缺送走夏侯沅峰之后,回来道:“公子,他真的离开了。”

我对董缺道:“去请张总管过来。”

看着张锦雄的背影,我终于松了口气,如果夏侯沅峰不会背叛,那么就更加安全,如果夏侯沅峰心口不一,那么他必然不会想到我还有其他的信使,这样我才能够放心张锦雄的安全。而且我原本打算下在张锦雄身上的禁制也取消了,我既然要他担任暗使,就要表示出对他的信任,对于名门正派出身的弟子,这一点更会让他们尽心竭力,在已经有了夏侯沅峰作为明使的情况下,暗使也不需要严加控制了。何况张锦雄毕竟是更值得信任的,不论是他的人品,还是他的师门,现在崆峒派也已经和凤仪门离心了,前不久,崆峒的重要人物就暗中和少林联络过,表示了合作之意。就算夏侯沅峰马上带人来捉我,我也不用担心了,只要能够召来秦勇,我的安危又有什么要紧,而且我相信,秦大将军的玉佩比皇上的密旨更能让秦勇相信,更何况,还有我事先的准备呢。感觉到浑身的精力都已经散尽,我躺倒在床榻上,心想,下一步我还可以作些什么呢?反正我最好留在这里,这样夏侯沅峰才会认为我信任他,就算他背叛了,也不会怀疑还有别的信使。

\chapter{第三十章 搬兵勤王}

秦青满面木然的坐在房内,方才他被李寒幽送到晓霜殿之后,父亲一解开他的穴道,就是一记耳光,秦青却是什么也说不出口,他能够说什么呢,父亲多次告诫自己不可让李寒幽接触禁军,可是自己却没有做到,还轻而易举的让人夺去了兵权,如果没有他手下的禁军,那么,凤仪门是无论如何也不可能发动政变的,秦彝见他面如死灰,更是气不打一处来,恶狠狠的痛加责打,幸好魏国公阻止了父亲,他还记得魏国公劝慰父亲的的话。

“老秦,你也不要再发火,贤侄毕竟是年轻无知,那李寒幽又是公主,贤侄不免没有戒心,这也要怪你,平日不好好教导,再说,指婚的是皇上,你如此痛责,若是皇上知道不免难堪。”

就这样,父亲将自己关在这厢房之中便不再过问,可是秦青心中之痛却是越来越剧烈,他仔仔细细的想着和李寒幽一起度过的时光,一点点一滴滴,那是说不尽的柔情万种,那个美丽耀眼的女子,让自己完全沉醉,他忘记了沙场血战的艰辛,忘记了袍泽手足的深情厚谊,只要李寒幽一个幽怨的眼神,他就忍不住去做任何事情。可是李寒幽呢,她从来对自己都是一片虚情假意,若非如此,为什么她甚至没有问过自己是否愿意和她一起谋反,她根本就不想策动自己造反,或许是因为她认为自己是绝不可能背叛家族的,不是么,很早之前,她不就抱怨过这一点么。秦青不知道,如果李寒幽真的问自己是否愿意和她一起谋反,他是否会答应,可是她从来都没有问过,就像方才将自己送回给父亲时候一样,她的眼神中满是冷淡,仿佛自己是没有生命的物体一般。难以遏制的怨恨从心中涌起,秦青低低的咆哮一声,握紧了拳头。紧咬的牙关渗出鲜血来。

含香苑中我却是陷入了困境。这里已经被所有人遗忘,除了禁军偶尔会过来巡视,但是他们并不细心,甚至有些草率,看来凤仪门的控制力并不强,而且公主殿下事先准备了一些食物,足够我和董缺食用,所以原本我可以安然待在含香苑等待结局。可是我却发病了,想一想这也没有什么奇怪,本来我到猎宫之时就已经是在病中,昨夜和今日又是这样折腾,换了别人自然没有关系,可是我却是支撑不住了,大概是觉得自己已经做了一切可以做的事情了,精神松懈下来之后,我便一病不起。

可是昨夜匆忙来到含香苑,虽然可以避开禁军控制的宫门,却是没有办法带上一大堆药物的,名医也没有法子不用药物治病的,所以我只能服了几粒自己配制的药丸然后就昏睡过去。等我醒来之时,看见董缺坐在一边,神色不安,我低声道:“董缺,夏侯沅峰已经出发了么?”

董缺镇静地道:“是的,我师兄随行护卫,一直没有人到含香苑来抓我们,所以公子的计策已经成功了。”

我叹息道:“我不是让你躲到别处去么?”

董缺淡淡道:“我若任你被人捉了,只怕将来李爷第一个找我算帐。”

我苦笑道:“小顺子不是这么不讲理的人吧?”

董缺笑道:“若是你们再次见面,公子还是担心怎么解释吧,您让他去救裴将军,又没有告诉他你会留下,我想李爷知道之后一定会气死的。”

我心里一抖,小顺子生气的模样不想也罢,不过,不知道现在他在做什么,但是急急冲回来不是他会做的事情,毕竟若是雍王失败,那我可真的是天下虽大,无处可逃了。

董缺犹豫了一下道:“公子,现在你病情沉重,就是秦勇能够赶来救驾,也至少还需要将近一天的时间,而且没有数日时间,恐怕无法平乱,你的病若是拖下去,恐怕——”

我知道他的担心,可是现在又有什么办法,现在不是在雍王府,我现在可是在保命啊。觉得一阵头晕目眩,我又向床榻上软倒下去。董缺担忧地道:“公子,这样是不行的,若是再拖几天,只怕你的性命就不保了。”

我无奈的笑了一下,再也没有精力说话,就这么昏迷了过去。

日正中天,秦勇走出大帐,舒展了一下筋骨,这次大将军将军权交付给自己,自己可不能有丝毫懈怠,也不知这次秋狩情况如何,雍王殿下和太子殿下之间已经是势同水火,如果不是这个缘故,皇上也不会下旨让伯父在猎宫百里之外驻扎军队了。

秦勇看看天色,正要回去大帐,突然有军士来报,有一个叫李顺的人前来求见。秦勇一惊,李顺他可是知道的,可是雍王司马的亲信为何会来求见自己,要知道自己这支军队是只能听从皇上的命令的。犹豫了一下,他道:“请他到大帐相见。”秦勇心想,自己只要召集所有近卫,就是那人前来是想行刺,自己应该也能够逃得性命,只要自己准备下弓箭手,就是杀了他也是可能的。

当李顺走进大帐的时候,秦勇便是心中一寒,只见这个平日衣着雅洁的青年此刻身上全是干涸的血迹,面沉如水,双目开阖之间,闪出残忍冷酷的光芒。秦勇强颜笑道:“李爷请坐,不知道李爷不在猎宫服侍江大人,为何到我营中求见,还是这番狼狈模样。”

小顺子冷冷看了看两旁的近卫,道:“我今日不是为了刺杀而来,如果秦将军肯和在下私下谈谈,那么最好不过,否则,只怕我会多有得罪。”

两旁的近卫大怒,一起拔出刀剑,只待秦勇将令,秦勇却是知道李顺的厉害,若是惹恼了他,只怕他立刻出手杀了自己也是可能的,就是自己逃了性命,自己这些近卫也会死伤惨重,更何况,这人的身后还有雍王司马江哲,还有雍王,自己是万万得罪不起的,更何况只见他形容如此狼狈,就知道发生了大事情。因此秦勇挥手道:“你们退下。”

那些近卫迅速的退了下去,秦勇站起身来,走到李顺身前,问道:“请李爷实言相告,猎宫发生了什么事情?”

小顺子看了他一眼,道:“太子谋反,雍王已经突围,特遣我来请将军前去救驾。”

秦勇深吸了一口气,道:“这怎么可能,禁军都在伯父控制之下。”

李顺将经过情形讲了一遍,他虽然有很多事情都是猜测的,可是根据那情形,秦勇已经知道事态紧急。他跌坐在椅子上,禁军出了事情,又是凤仪门主导的叛乱,想也知道秦青一定出了问题,可是这是真的么,自己不能凭着一面之词就调兵前往,若是想要谋反的是雍王,那么这一调兵可能就会落入圈套。

他的犹豫李顺看在眼里,他眼中闪过一丝冰寒,冷冷道:“秦将军还在考虑什么,雍王殿下只要你前去救驾,又没有要你去救他,现在殿下虽然危急,可是你若是救了圣驾,雍王殿下也就可以脱险。而且秦大将军和秦青将军都在猎宫之中,恐怕他们也是危在旦夕。”

秦勇犹豫了一下道:“没有皇上的旨意和大将军的兵符,末将若是私自调兵,是要犯死罪的。”

小顺子嗤笑道:“死罪?现在皇上和大将军都落在敌手,若是秦将军还要抱残守缺,只怕后悔莫及。”

秦勇坚定地道:“我会派人前去查探,请恕末将不能立刻发兵。”

小顺子低下头,眼中闪过一线杀机,可是他深知若是用强,引起了秦勇的反感,更是不能及时救援猎宫,可是现在每过一刻,公子便多一分危险。良久,小顺子从怀中掏出一个锦囊,递给秦勇,叹息道:“秦将军请看看里面的东西。”

秦勇接过锦囊,打开一看,脸色突然变得苍白,里面是一根银质发簪和一块普普通通的翠玉佩。他颤抖着问道:“你,你怎会有这两样东西,这是家母的发簪和家母送给义弟刘华的佩玉。你是要威胁本将军么?”

小顺子有些疲倦地道:“这种手段我们是从来不喜欢用的,可是如今却是不得不用,刘华真名叫骅骝,乃是我家公子的属下。”

秦勇身躯一震,恶狠狠地道:“刘华,他是你们的细作,想不到雍王竟会关心我这样一个小人物。”

小顺子淡淡道:“秦将军过谦了,大将军对你的重视尤在秦青之上,秦青倾向太子,和凤仪门过从甚密,我家公子担心大将军抛弃一贯中立的立场,所以才安排了人在将军身边,将军乃是大将军亲信,若是秦家有什么动向,将军为了不让令堂担心,不免漏些口风,公子不想惊动大将军,所以在您的身边安插了人,而且公子很看好你,他说你的才干胜过秦青,这也是他让骅骝到你那里去的原因,骅骝乃是公子身边八骏之一,若非紧要的人,公子是不会让他去监视的。

秦勇眼中多了几分阴郁,他冷冷道:“你是在说,我和家母那样爱护的少年,却是一个骗子和细作。”

小顺子叹了口气道:“并非如此,事实上,这次临行之前,我去见骅骝,他求我无论如何不要伤害令堂,他说,你的事情,他自知无能为力,可是令堂待他如同亲生,他情愿接受任何惩罚,换取我们不对令堂为难。所以他拿来这两样东西,只是为了让我们不去惊动令堂。”

秦勇心中有些轻松,虽然李顺所说没有什么证据,可是他就是觉得这人根本就不屑于说谎。有些放心的将锦囊收好,他不会认为李顺这样说就代表自己的母亲不会受到威胁,可是至少他可以确信,李顺不是随便杀人的人,而李顺的主人江哲和雍王也不是这样的人。可是若是自己拒绝出兵呢?

小顺子看到了秦勇忧心忡忡的神色,他冷冷道:“我知道让你出兵有些为难,可是至少如果猎宫有人前来传旨要你做什么,你不可遵命。”

秦勇皱了一下眉道:“若是皇上的圣旨和大将军的兵符,你也要我拒绝么。”

李顺冷冷道:“若是这一点你都不肯,那么我也没有什么好说了。”

秦勇抬头,看见李顺眼中清晰的杀机,无奈地道:“我会先派人去向伯父请安,如果一切正常,就是你如何逼迫,我也不会出兵。”

李顺神情变得十分冷淡,他早就知道秦勇不是可以轻易威胁的人,如今只能尽量得到最好的结果了,能够让秦勇不会轻易遵从猎宫传来的命令,那么他的目的就已经基本达到了,而且若是秦勇派人去了猎宫,那么很快就会发现情况的异常,这样虽然晚了一日,还是有机会救出公子的,现在只希望公子和雍王都能够平平安安的活着了。

他看看天色,淡淡道:“若是明日此时,你还不出兵,我也只能得罪了。”

秦勇冷冷道:“我知道阁下武功高强,可是谋逆之事我是绝不会做的,若是我的人没有发现异常,就是阁下动用武力,我也不会就范,我这里大军数万,若是阁下发难,就是秦某不免身死,阁下也要陪葬的。”

小顺子冷冷一笑道:“给我准备住处和食物,我已经很累了。”

秦勇无奈的高声道:“来人。”几个亲卫进了大帐。秦勇厉声道:“给他准备一个单人的营帐,按照他的吩咐行事,记着,若无本将军许可,不许他走出营帐一步。”

小顺子淡淡一笑,站起身向外走去,一边走一边道:“只有一天一夜的时间,秦将军还是快些派人吧。”

秦勇叹了口气道:“我会立刻派人去猎宫向大将军问安的。”

九月二十一日黄昏时分,如今猎宫已经被凤仪门全部控制,虽然晓霜殿仍然在皇帝控制之下,可是人人都知道,只要凤仪门发起攻击,皇帝也不能幸免。可是凤仪门也有自己的难处,若是皇帝身死,雍王就可以以大义名份勤王讨逆,所以必须保住李援的性命,好完成禅让的大礼,因此凤仪门不敢过于强逼。而李援却陷于空前的弱势之中,他这次来猎宫,所带的侍卫虽然不少,可是和凤仪门比起来并不占优势。在凤仪门苦心经营的禁军控制下,李援等人和外界的联系全部断绝,而那些仍然忠于皇帝的禁军,他们的将领已经被韦膺矫诏召集到一起,全部软禁起来,没有将领指挥的禁军不敢擅自作为,因此明明手握大雍无上皇权的李援,却没有办法将自己的意旨传递出宫墙。李援纵然可以派侍卫强行出去传令,可是凤仪门强攻之下,就算李援保住性命,那么长孙贵妃和颜贵妃以及长乐公主也不能逃生,这样一来,在晓霜殿形成了双方力量的平衡,在外界情形没有变化之前,晓霜殿这里是无论如何不会有动静的了。

宣华苑中,齐王躺在软榻之上,神色淡淡,秦铮走进来,挥手让自己的两个亲信侍女退下,为了李显的安全,她没有同意让凤仪门弟子来监视李显,而是让两个自己一手调教出来的侍女照顾监视李显。她解下佩剑,坐在椅子上,眼神中充满迷惑,良久,她见李显不肯开口想问,只得苦笑道:“王爷不想知道母妃娘娘的情况么?”

李显眼中闪过一丝寒光,道:“母妃恪守妇道,绝不会背叛父皇的。”

秦铮微微苦笑道:“正如王爷所说,母妃丝毫没有犹豫便选择了皇上,妾身不明白,对一个母亲来说,儿子不是最重要的么?难道你的生死荣辱,母妃都不会放在心上。”

李显淡淡一笑道:“对于一个妻子来说,难道还有比忠于丈夫更重要的事情么?父皇是母妃的丈夫,也是大雍的君主,母妃怎会背叛他呢?”

秦铮反驳道:“可是皇后娘娘不是背叛了皇上么,还有,为什么女子一定要忠于丈夫,男子却可以三妻四妾,风流快活。”

李显看向秦铮控诉的眼光,不由一笑,想起从前初见之时,这个女子也是这样喜欢争辩,但是那一缕柔情立刻消失了,他也不愿争辩这些事情,岔开话道:“太子殿下心情如何,现在二哥突出重围,恐怕太子已经十分苦恼了吧?”

秦铮神色一整,道:“闻师姐带着几千人追杀雍王,他们就是本事再大,也逃不出去,倒是你可怎么办呢,等到太子登基之后,若是想起今日你不肯出力之事,只怕你这个亲王位子也坐不稳了。”

李显淡淡的看了他一眼道:“可是李寒幽他们让你来作说客的,你不是拿了我的兵符,怎么调不动军队么?”

秦铮神色有些尴尬,半晌才道:“调兵遣将自然是可以的,可是你的几个亲信爱将都说除非你亲自到了军中,他们才肯围歼雍王的军队,你知道雍王正在想法子和他的军队会合,若是你肯亲笔写一封书信,若是雍王真的和他的部下会合,如果没有你的相助,那么胜负还在两可之间。王爷,如今你已经和我们在一条船上了,难道你还是不肯顺从么?”

李显神色一动,片刻才道:“让我见见太子,如果我们谈的妥当,这封手谕我就写给你,你应该清楚,我和那些属下之间都有暗语,你们是仿造不了我的书信的。”

秦铮露出一丝喜色道:“若是王爷肯顺应天命,妾身无有不从。”李显淡淡一笑,神色间更是多了几分嘲讽。

\chapter{第三十一章 齐王手段}

一乘软轿抬着齐王向太子居住玉麟殿缓缓行去,李显如今身上被药物所困,虽然勉强可以行动,可是根本无法走动这么远,玉麟殿在猎宫东侧,齐王所居住的宣华苑却在西侧,两者之间有数里之遥,自然只能乘轿前往,抬轿的四个武士乃是齐王亲信的侍卫,就是齐王妃也不能随便使唤他们,秦铮带着两个侍女前面引路。

一行人到了玉麟殿,这里防守很是严密,萧兰闻听齐王到了,亲自出来迎接,她也是一身劲装,见到被秦铮扶下轿来的李显,她上前施礼道:“六叔此来,殿下一定万分欣喜。”

李显冷淡地道:“李显如今不过是贵门阶下之囚,哪敢当你的大礼。”

萧兰面上露出一丝尴尬,却立刻笑道:“六叔,这事是我们不对,还请六叔见谅,殿下在里面等着呢。”

李显走进殿内,只见李安正在殿中负手而立,太子少傅鲁敬忠侍立一旁,虽然鲁敬忠已经被凤仪门软禁起来,可是如今事态紧急,在太子的要求下,凤仪门不得不又将他放了出来,只是不许他离开玉麟殿罢了。或者是因为这个缘故,再加上太子也没有尽心相护,所以他的神色有些冷淡憔悴。

一见到李显,李安便亲切的上前握着李显的手道:“六弟,你可来了,这次你可定要帮帮为兄,你是知道的,现在为兄已经是船到江心,不能回头了,不是登基为帝,就是圈禁赐死。弟妹可也是叛逆了,你若不肯尽心,到时候若是为兄不幸失败,你也脱不了干系。”

李显神色从容道:“小弟知道如今局势,可是太子不是已经矫诏去招秦家军了么?”

李安微微一愣,赧然道:“秦家军毕竟不是我的嫡系,若是发生了什么意外,实在难以控制。若是六弟你的军队来了,为兄的帝位才能稳如泰山。”

李显似笑非笑的道:“既然如此,就请太子解了我身上之毒,让我去军中坐镇如何。”

这句话一出口,李安立刻说不出话来,他看向萧兰,神色有些为难,这时候鲁敬忠道:“齐王殿下千金之躯,现在雍王还在逃,殿下若是轻身涉险,若有个三长两短岂不让太子担忧,还是在这里好一些,只要殿下一纸书信,让殿下的军队急行军赶来行宫即可,不知殿下可肯替太子效力。”

李显冷冷道:“谁不知道雍王的军队虎视眈眈,若是我的军队调动,只怕会惊动他们,少傅不担心弄巧成拙么?”

鲁敬忠笑道:“雍王近卫军冥顽不灵,虽然太子已经派人追杀雍王,秦家军也很快就会前去围剿,可是若是不幸让雍王和自己的军队会合,不免让战况更加复杂,所以太子才会希望殿下派军队将雍王所部歼灭。殿下所部和雍王军队兵力相近,精锐程度也不相上下,相信殿下定能旗开得胜。就是殿下暂时不能取胜,太子已经命令秦军擒杀雍王之后,带着雍王首级前去助殿下所部平叛。到时候,殿下就是勤王的最大功臣,太子必定重重赏赐。”

李显深深的看了鲁敬忠一眼,心道这人真是狠毒,竟是让自己去歼灭雍王的近卫军,到时候自己就是侥幸成功也是伤亡惨重,而雍王和自己的军队大部分都在边关,镇守京畿的秦军只忠于皇室,只要掌握父皇,就可以保证李安登上帝位。可是李显没有说破这人的狠毒心机,只是冷冷道:“好吧,本王可要写书调动军队,可是太子殿下却要答应臣弟几个条件。”

李安大喜道:“六弟尽管讲来。”

李显神色冷冷道:“第一,不论我们如何争夺皇位,可是祸不及妻儿,二哥的生死我不管,可是二嫂和侄儿不许你下毒手。”

李安微微皱眉道:“斩草不除根,六弟也太心软了,若是老二胜了,我们的妻儿也只有死路一条。”

李显默然不语,鲁敬忠使了一个眼色,李安只得勉强道:“就依你。”

李显微微一笑道:“第二个条件,皇兄你若是继位,不能因为凤仪门的功劳废黜皇嫂和世子。”

李安爽快地道:“这个没有问题,孤也是这样想的。”

李显淡淡道:“第三个条件,我知道从前太子殿下对臣弟颇有不满之处,还请殿下不要秋后算帐。”

李安尴尬地道:“怎会呢,六弟你襄助孤取得皇位,孤定然不会恩将仇报。”

李显点头道:“还有一个小条件,现在我被药物所困,就连下床也是艰难,先解了我的毒再说。”

李安看了一眼萧兰,萧兰犹豫片刻道:“臣妾只有可以暂时让王爷行动自如的解药,若想恢复武功,恐怕得等到师尊到了之后才行。”

李安看向齐王,担心他因此反目,谁知李显只是淡淡道:“本王不过是躺在床上闷了,原也不急着恢复武功。”

萧兰神色一松,取出一颗药丸递给了李显,李显接过药丸服下,过了片刻,觉得体力渐渐恢复,便走到书案前,一挥而就,写了一封书信,便转身离去了。

走在御道之上,李显神情轻松自在,好像再没有什么心事,他也不再坐轿,只是安步当车向宣华苑走去。秦铮见他高兴,心中也很愉快,便陪着他慢慢走去。

因而无人注意到抬着软轿的几个武士放慢了速度,而且改道接近了含香苑,这里已经是十分冷落,看守的禁卫也不多,四人选了一个隐秘之处,将轿子藏了起来,便纵身进了含香苑。进去之后,一人在外面放哨,三人进了含香苑,轻轻的四处探察了一下,最后探察公主寝宫的那人打了一个手势,另外两人立刻飞身过去,其中一人轻轻推开殿门,然后立刻闪开。

殿内董缺听到声响,浑身一震,回头一看,殿中已经闪进两个武士。董缺心道糟糕,难道夏侯沅峰还是告密了么,但是若是如此,又怎会只来了两个武士。他不敢出声,也顾不上昏迷不醒的江哲,拔剑向两个武士扑去。

那两个武士都是一流高手,同时拔刀还击,双方都是默不作声,交手数招,董缺方才受的伤渐渐渗出血来,渐渐不支,而另一个武士也闪身进来,避开三人交战之处,到了床边,低头查看江哲的相貌,过了片刻,他抬头做了一个手势。那两个武士都是神色一振,刀法更加凌厉。另一个武士低声道:“我们是齐王属下,不论你在雍王府何等身份,应该知道齐王殿下对江大人从无恶意,现在你们身在险地,不如暂时托庇殿下如何?”

董缺神色一动,剑法更加散乱,那两个武士见状停手不攻,只是提防董缺出手。董缺也住了手,看向床边,那个武士虽然说的和气,可是只见他手按刀柄站在江哲身边,董缺就知道没有反抗的余地了,可是他深知自己的身份是见不得光的,和齐王见面有害无益,想了片刻,他突然转身冲出寝殿,那几个武士都是一愣,料不到他弃主逃走。两个武士追出门的时候,轻功高明的董缺早已无影无踪。三人一商量,无论董缺怎样,也不会去告密的,反正江哲已经到手,还是快些回去的好。

他们将昏迷不醒的江哲挟持到外面,将他藏到轿子里,然后若无其事的抬着轿子返回宣华苑,一路上都无人留意他们的行动。

回到宣华苑,他们按照齐王的吩咐将江哲藏到偏殿当中,然后一个武士前去向齐王回禀。秦铮早已经回去晓霜殿了,所以房内只有齐王妃的两个侍女,也因为齐王的合作态度而不敢违命,被齐王赶到了外间。

这个武士低声禀报之后,李显微微皱眉,他虽然被困住,可是消息却还是很灵通的,不仅秦铮不是的告诉他一些消息,他在禁军中也有几个亲信,自然知道雍王突围、长乐公主斡旋和韦膺搜查含香苑的事情。所以在他的判断中,江哲很可能藏在含香苑,所以他才会借着去见太子的机会让手下去含香苑搜查。可是现在情况太诡异了,江哲的护卫怎会逃走,于情于理都有问题。

他正在思索,那么武士低声道:“王爷,江大人气息奄奄,若是不救治,只怕会有生命危险。”

李显一震,道:“让太医去给江哲诊治,记着,小心行事,别走漏了风声让王妃知道。”

李显这次名义上是卧病前来,所以特意带了一个太医来,现在就在偏殿,正好用上了。那个太医此刻心中十分苦恼,他并不是太子一党,如今深陷这样的困境,如果情况一变,自己可能就会成了叛党,但是他可不敢违背齐王的命令。进到偏殿之后,看到江哲他就是身躯一震,当年江哲遇刺,他也是前去诊治的御医之一,自然认得雍王的亲信幕僚。眼前的情景让他糊涂起来,齐王明明是太子一党,怎会私藏雍王的幕僚。但是他知道这种事情自己还是装聋作哑的好。上前一诊脉,他的眉头就紧锁起来,道:“这位大人原本就在病中,有没有好好修养,如今心脉衰弱至极,若是不好好救治,只怕熬不过今夜,我开一个方子,用参汁下药,好好修养,还是可以治的。”那几个武士大喜,道:“乔太医,你要好好医治,若是此人有了三长两短,王爷绝对不会放过你的。”乔太医连连答应,他这次带着的药物十分齐全,果然连着几服药下去,江哲的面色渐渐红润,气息也渐渐粗壮,神色也十分安宁。乔太医这才擦着汗道:“总算没事了,不过大人的身子太弱了,需要好好调养才是。”两个武士面面相觑,他们也听说过雍王的这个亲信幕僚身子极弱,而且自己的主子对他也是推崇备至,想不到竟是这样一个好像随时随刻都会死亡的文弱书生。

一夜无事,将近天明的时候,江哲终于睁开了眼睛。他们连忙去禀报齐王李显。

感觉到有人在叫自己的名字,我艰难地睁开眼睛,事实上,上次晕倒的时候,我都很怀疑是否还能醒来,此刻虽然浑身无力,但是我还是感谢了老天爷一番,低声喊道:“董缺,董缺。”

耳边传来声音道:“随云,你醒了。”

我心中一震,这个声音很熟悉,可是绝对不是董缺,偏头看去,却看见齐王匆匆走了进来,我下意识的看了一下四周,苦笑道:“原来哲已成阶下之囚,却不知怎会在王爷这里?”

李显苦笑了一声,坐到床前的椅子上,道:“今日一见,恍如隔世,想来随云已经运筹帷幄,二哥已经稳操胜券了。”

我艰难地想坐起来,只是四肢无力,无法如愿,齐王连忙上前搀扶,我才坐了起来,问道:“现在是什么时候了?”

李显淡淡道:“今日已经是九月二十二日,马上就到辰时了。”

我松了一口气,看来勤王之兵很快就会到来了,希望雍王还是平安无事,神色从容地道:“不知我怎会在此,我身边的侍卫呢?”

李显笑道:“昨日我派人去含香苑,果然找到了你,现在除了本王和几个心腹之外无人知道你在这里,你的那个护卫倒也奇怪,见你落入我的属下手中,竟然逃走了。”

我松了一口气,董缺若是和齐王见了面,凭着齐王过人的直觉,只怕会有身份泄露的危险。

李显有些痛惜地道:“随云,你为了二哥呕心沥血,若非本王的人即时将你接来,你恐怕已经丧命,真是何苦来呢?你当日若是跟了本王,何至于此。”

我淡淡一笑,道:“哲受雍王殿下大恩,此时若不尽力,岂不辜负了雍王大恩。”

李显面上露出不豫之色道:“本王自信若是你肯归顺于我,本王待你绝不逊于二哥。”

我不由想起当日我步步紧逼,雍王却终于手下留情,放我生路的情景,犹如还在昨日一般,片刻,我道:“殿下秉性直爽,天资过人,哲也是十分倾慕,可惜殿下当初一步走错,以至今日进退两难,不过从前之事,说也无益,不知道殿下此次可有行止差错么?”

李显苦笑道:“昨夜我答应她们的要求,写了一封手令给我的部下。”

我微微一愣道:“殿下应该知道,如今殿下所部已经用不上什么力气了?”

李显叹了一口气道:“我实际上的命令是让他们按兵不动,他们什么也不会做,至于大哥和二哥谁能取胜,就看他们自己了。”

我恭敬地道:“殿下悬崖勒马,臣十分佩服。”

李显有些惆怅地道:“事后不管是谁取胜,本王的命运恐怕都没有什么不同了,若是二哥取胜,本王想求大人一件事情。”

我神色凝重地道:“殿下救臣性命,若是小臣能够做到的,就是肝脑涂地,也在所不辞,请殿下示下。”

李显叹了一口气道:“我知道胜者为王,若是太子获胜,二哥一家定然也会遭殃,虽然昨日太子答应了会放过雍王府的眷属,可是我是明白他这个人的,就算一时碍着我的面子放了,也会另想办法斩尽杀绝。同样的,若是二哥胜了,大哥的家人也不会有什么好下场。可是都是骨肉至亲,我实在不能撒手不管,希望随云向二哥进言,放过太子妃和世子,将他们废为庶人就好,二哥一向宽宏大量,或许还可答应。若是二哥肯答应这个条件,我愿将手上兵权拱手让出。”

我沉默了半天,道:“殿下就不为自己和王妃、世子着想么?”

李显的面色大变,很久没有说话,半晌才道:“我知道这是不可能的,铮儿参与了叛变,若是二哥取胜,不论是国法还是家法,铮儿都不能幸免,就是我和铮儿的儿子也会受到牵连,或者父皇会顾念我没有参与叛变饶了我的性命,可是妻儿皆死,我还有什么面目安享富贵呢?”

我看了李显一眼,知道他说得不错,齐王妃和齐王世子都不能脱罪的。可是不便这样说,只得道:“现在胜负还未可期,殿下不必过虑。”

李显苦笑道:“本王可不敢奢望,只见随云你如此气定神闲,就知道太子的胜算不大。”

送走了齐王,我心中思虑万千,今日之前,我心心念念都是怎样增加雍王的筹码,别的什么都顾不上了,现在局势如何发展已经与我无关,若是雍王败亡,那么我自然没有什么好说,只有以身相殉,可是若是雍王取胜,后事又该如何处理呢,到时候雍王肯定是要问我意见的,我的一念之间,就会涉及到千万人生死,不可不慎。

在我本心,太子自然是该死,凤仪门更是绝不能继续存在,而韦膺险些坏了雍王大业,不论韦膺之事韦观是否知道,都是要受到株连的,可是韦观却是丞相之尊,门生无数,如何处理才妥当,不伤害国本。还有最关键的,就是齐王。虽然多年来,因为齐王的缘故,使得太子气焰嚣张,雍王上下对齐王可能怨恨极深。可是不容置疑的,齐王性情光明磊落,重情重义,又是难得的帅才,若是将其处死或者贬斥,都是大雍的损失。可是齐王个性激烈,又是心狠手辣的人物,齐王妃会成为他和雍王的死结,若是轻易放过齐王,那么日后可能后患无穷,真是进退两难了。想了一会儿,我突然笑了,这些事情雍王和石彧自会处理的妥妥当当,我何必费心呢?想到这里,我渐渐放松下来,昏昏睡去,一切今明两天应该就会有一个结果吧。

\chapter{第三十二章 邪影罗刹}

第三十二章邪影罗刹

人困马乏,已经连续转战一昼夜,千余人只剩下半数存活,还是个个带伤,李贽苦笑着摇头,想不到自己在拥有了千军万马之后还会尝到这样的苦头。闻紫烟率领的两千禁军和裴云率领的千余禁军乃是大雍最精锐的部队之一,个个骁勇善战。闻紫烟即在兵力上占了优势,行军速度又快过雍王,再加上闻紫烟的麾下除了两千禁军之外还有五十名凤仪门女剑手,这些女剑手都是武功高强悍不畏死的死士,她们虽是女子,可是各个精通剑术,擅长弓马,虽然不擅长正面进攻,可是她们配合禁军勇士在外围用弓箭射杀,而两军接近之后,她们又可以凭借精湛的剑术和骑术刺杀雍王麾下的高手和将领,这些女剑手本来就人手一柄宝剑,可以轻易刺穿大雍将士的甲胄,所以她们造成了雍王很大的损失。而李贽的手下或者是只擅长沙场厮杀,或者是只擅长武林技击,比起这些在战场上神出鬼没的女剑手就逊色多了。若非是李贽凭借出色的指挥抗衡,只怕早就被闻紫烟给围杀了。

李贽回头看看远处的烟尘,再次叹息,凤仪门主可真是非同反响,她训练出来的这支女子军队真是绝世无双的,就是北方蛮族的弓骑兵也未必如此厉害,自己一向自负擅长练兵,可是却没有想过训练这样一支轻骑兵。当然,这样训练的代价未免太高,但是却绝对可以成为一支神鬼俱惊的铁骑。而闻紫烟,这个让李贽最头疼的女罗刹,更是让李贽赞叹不已,虽然在阻截雍王突围的时候,闻紫烟表现的差强人意。可能是因为闻紫烟虽然负责训练这些女剑手,但是将这些女剑手训练成军的却不是闻紫烟吧。不过李贽不得不佩服闻紫烟的能力,从最开始的手忙脚乱到现在的指挥若定,如果闻紫烟早些领军作战,可能会成为有数的名将吧。

李贽不由想到,凤仪门主真的选错了道路,如果当初她不是致力于掌握朝政和后宫,那么凭着闻紫烟和这些女剑手,大雍可能会有一支震惊天下的娘子军吧。虽然那样的道路必然坎坷曲折,却会是一条更加光明的道路。

和部下分吃了剩下不多的干粮,李贽再次上马,高声道:“再赶一程,如果能够越过苦云岭,那么我们就可以阻截叛军的追击,我们就可以和援军会合。虽然是这样说着,李贽心中却很担忧,闻紫烟率军迂回阻截,迫使李贽不能向自己的亲卫军方向转移。如果再这样下去,李贽心想,自己的人头可能就会成为献给太子的礼物了。在他身边的裴云眼中闪过愤恨的神色,精心练出来的军队却被凤仪门的女剑手杀得人仰马翻,虽然是有兵力不足的因素,可还是让他丢尽了面子。

众人奔驰了一段时间,前面已经看见了一个险峻的小山岭,众人都提高了警惕,昨日他们曾经到了这里,可惜却被闻紫烟拦住,最后不得已折转突围,这一次他们用尽了各种方法掩盖形迹,分兵诱敌,这才重新到了这里,只要过了这里,那么接下来的七十里路都是丘陵古道,只要留下死士埋伏断后,那么就可以保证雍王回到亲卫军的保护之下。那些追兵再厉害也不能在数万大军中加害雍王。

看向前面山岭,李贽一挥手,两个轻身功夫最好的高手下马,如同猿猴一般飞身上了山岭,他们的身形刚刚从众人眼中消失,一声大叫传来,李贽等人立刻握紧了兵器,山岭之上出现了一个骑着骏马的青衣女子,虽然相貌平平,可是那种傲视天下的气魄却让这个女子在众人眼中形象鲜明起来。凤仪门主首徒果然不是凡品。

闻紫烟提马上前,在她身后四十多名白衣女子策马上前停在她左右两翼。闻紫烟高声道:“李贽,本座早就料到你会回来这里,所以不论你如何分兵相诱,本座仍然提前赶到这里,如今你已至必死之境,还不下马受缚,或者太子殿下仁德,还会饶你性命。”

李贽长叹一声道:“闻姑娘不去领兵作战,真是万分可惜,本王佩服,可是想要本王性命,还要凭你的本事,李安叛上作乱,无父无君,你们凤仪门唆使太子叛变,也是不赦之罪,想要本王人头,你自己来取吧。”

闻紫烟放声长笑,一挥手,从她两侧涌出无数的骑士,居高临下,直冲而下,李贽心知地利为李寒幽所占,若是自己现在急于逃走,只能是被闻紫烟衔尾追击,若是自己死命抵挡,更会损失惨重,可是却有一线生机,若是能够挡住一波攻击,那么还可以寻机会脱身。

因此李贽拔出佩剑前指,高声道:“宁死不退,杀!”喊罢,一马当先,向前冲去,左右近卫见状都是心中一热,抢着上前掩护雍王。两支劲旅撞击在一起,狭路相逢,血肉横飞。雍王凭着高超的指挥,终于艰难的挡住了第一波攻击。这时,裴云已经发觉山岭上的闻紫烟带着凤仪门女剑手,从右侧较为险峻处冲下,显然是要攻击雍王侧翼。裴云心一横,高声道:“兄弟们,随我断后,殿下快走。”

在裴云的一声令下,千余禁军中有三百多名事先已经得到过裴云指示的禁军同时爆发出强大的战力,死死的挡住了叛军,李贽微微一愣,就看到裴云一马当先冲向了闻紫烟。他痛惜地喊道:“走!”虽然事先没有计划过,可是雍王久经沙场,自然知道这是唯一的机会,当即烈士断腕,离开了战场。人人都知道,若是雍王不能活着和近卫军相见,那么大家都是死路一条,所以剩下的禁军和雍王一些侍卫也不迟疑,护着雍王撤离。

闻紫烟和裴云交战数合,裴云乃是少林高手,又是沙场骁将,此刻他又是悍不畏死,所以竟然阻住了闻紫烟的攻势,而他身边的亲卫和各大门派送到雍王身边的一些武林高手也留下了,他们虽然不擅长沙场征战,可是凭着血气之勇居然挡住了凤仪门女剑手的利剑和铁蹄。

闻紫烟剑光如虹,那如雪的剑刃终于寻机刺入了裴云的身躯,裴云见身边亲卫高手已经接近溃散,也就不再闪避,而是反手一刀劈向闻紫烟,少林青年高手的拼死反击岂是易与,闻紫烟躲避不及,虽然她青衣之内穿着软甲,仍然是被这一刀砍伤了右臂。但是裴云也被围过来的凤仪门女剑手刺了几剑,坠落马下。闻紫烟虽然看见裴云还没死去,但是为了追杀雍王也顾不上了,一声长啸,带着军队向雍王的残部杀去。

这番追杀不同寻常,闻紫烟不顾一切策马狂奔,雍王无论如何也无法摆脱追兵,跑了二十多里,马匹的速度渐渐放慢,李贽心一横,举起佩剑就要向马臀刺下。这时前面烟尘滚滚,似有大队人马杀来,李贽不由心灰意冷,一时之间竟然不知所措,可是他毕竟一代人杰,眼看着前后两方可能会同时赶到,索性住了战马。想起这两天的厮杀奔波,自己已经是狼狈不堪,大雍的军神岂能死得如此狼狈。便将佩剑的平面当成镜子,整理仪容,整理衣甲。而左右禁军和护卫也是一片灰心,都是握紧兵刃,准备迎接最后一刻的到来。

后面闻紫烟的追兵渐渐接近,这时候,李贽也看清前面来的军队为首之人俊美无双,正是夏侯沅峰,而他身边的军士看衣甲似乎是秦彝的部下。李贽心想,莫非江哲的计划失败,太子已经控制了秦彝的军队么,此刻死亡在即,李贽反而心如止水,看看左右,司马雄和荆迟都已经是遍体鳞伤,众侍卫也是形容惨淡,衣甲破碎,不由笑道:“大丈夫生于乱世,当带三尺剑立不世之功;今所志未遂,奈何死乎!只可惜连累了诸位。”众人泣道:“能随殿下共赴黄泉,虽死犹荣。”

这时,夏侯沅峰所带的军队突然向两侧延伸,形成了一个两臂怀抱的军阵,李贽等人一愣,这个样子,不像是要冲散自己的残军,倒像是要将自己等人包围起来,莫非他们是想生擒么。还没有等到李贽想清楚,夏侯沅峰的军队已经从中间一分为二,从李贽残军两翼越过,迎向闻紫烟的追兵。一方是兵强马壮,一方是强弩之末,一触之下,高下立见,闻紫烟的军队被夏侯沅峰率领的五千多人包围起来。

“夏侯沅峰!”从重重包围之中,传来闻紫烟尖利而愤怒的喊声。

李贽神色一振,虽然不明白怎会发生这种情况,但是他立刻明白,这一刻,他已经稳操胜券。

这时,那些赶来的援军留下来的一些将士来到雍王马前,一个豪勇的将领在马上行了军礼,高声道:“秦将军奉了陛下密旨,派出大军四处寻找救援殿下,末将张雄,随夏侯统领一路,幸遇殿下,救援来迟,还请殿下恕罪。”

李贽喜道:“将军不用多礼,这是怎么回事,你慢慢讲来。”

那个将军恭谨地道:“昨天夜里,夏侯统领到了我军大营,传达陛下密旨和大将军军令,言道太子谋反,雍王殿下被叛军追击,秦勇将军代传军令,大军分为八路,寻找殿下行踪。请殿下准许末将发出信号,通知各路人马,殿下所在位置。”

李贽想了一想,道:“你们用军中传信方式,通知各路将士,到平远镇会合即可。”

平远镇距离猎宫十五里,正适合设立勤王军的大帐,那个将领眼中闪过敬佩之色,自去派遣信使,使用烟花烽火等方式将雍王军令传下。

李贽举目望去,只见闻紫烟虽然被困,可是更加悍勇,围攻的将士死伤无数,不由心中痛惜万分。便对那个将领道:“附近可有友军?”

那个将领也有些忧虑的看着战场,闻声道:“殿下,秦将军所率领的中军应该就在二十里外。”

李贽大喜道:“速招秦将军前来,歼灭叛军之后,合兵共赴平远镇。”那个将领连忙传下令去。另外一组烟花信号升上天空。

过了小半个时辰,闻紫烟率军突围数次,都被将士舍生忘死地挡住,夏侯沅峰在雍王麾下高手和军中勇士地协助下,十分艰难的挡住了凤仪门的锋芒。此刻他们才真的领略到了凤仪门的厉害,从前他们虽然对凤仪门的淫威十分忌惮,可是实际上却对这些女子心存轻视。可是闻紫烟那绚丽万方而狠辣绝情的剑法让他们时刻都在生死边缘徘徊。

李贽心中虽然忧虑,可是另外一件事却让他心中十分喜悦,他派去救助断后将士的属下发现裴云仍然活着,虽然伤势很重,可是少林心法确是十分神妙,居然保住了他的性命。

又过了一阵子,眼看凤仪门虽然也是死伤惨重,可是闻紫烟却是即将突围成功的时候,远处烟尘滚滚,秦勇亲自率领的援军到了。这时候闻紫烟终于一马当先冲出了重围。

夏侯沅峰无奈地摇摇头,在苦苦的交战了半天之后,他终于是无力支撑了,为了不死在闻紫烟剑下,他还是退让了,这个实在是出乎他的意料。夏侯沅峰虽然喜欢两面讨好,但是他更加擅长明辨形势,从雍王突围之际,他就知道局势的变化已经不受凤仪门的控制,所以在江哲的威逼下,加上对凤仪门的失望,所以他很快就决定投靠雍王。他自嘲地想,虽然雍王比较难伺候,必须用实际的功劳换取官职和信任,可是至少比仰人鼻息好一些。既然想要投靠雍王,那么如何尽快立下大功就是当前要务。而且,老天爷保佑的是,居然是他第一个找到了雍王,功高莫过救驾,夏侯沅峰自然是喜出望外,而歼灭闻紫烟本来似乎是老天爷赏赐的功劳,可是闻紫烟和凤仪门女剑手的强悍却让他碰了一个大钉子。

而此时,夏侯沅峰终于也留意到了远处的援军,可是在他来说,让这些援军去围剿闻紫烟,虽然功劳被别人抢走,可是自己消耗了凤仪门的实力的功劳雍王已经看在眼里,所以他也就不计较了,只是在凤仪门冲破重围的时候下令合围,毕竟将剩下的禁军一网打尽也算是不小的功劳。

从那些援军中,一匹黑色的乌骓马脱离军阵,迎向闻紫烟,冲天的杀气从马上的戎装青年身上涌出。闻紫烟看到那支援军,收住了战马,她闭上了眼睛,片刻再度睁开,原本那已经从希望中坠落到绝望深渊的眼神已经变得平淡无波。那些白衣早已经被鲜血染红的凤仪门女剑手一个个默无声息的整理兵器,前两日不过折损数人的凤仪门女剑手在方才的苦战中已经损失了大半,她们的弓箭早已经损失殆尽,外面所穿着劲装已经破碎褴褛,露出里面所穿的黑色软甲,可以切金断玉的宝剑也已经刃钝锋黯。可是她们面上却是没有丝毫惊惧软弱。

闻紫烟挥手让那些女剑手莫要擅动,自己提马上前,迎上那人。就在着片刻之前,那些反叛的禁军已经全无斗志,夏侯沅峰策马到了雍王身边,正要报告,只见李贽的目光凝视着前方。在那里,百步之内没有一兵一卒,只有闻紫烟和李顺正在对峙。夏侯沅峰微微一笑,没有说话,也不把那些雍王亲卫戒备的目光放在心上。此刻,战场上除了那些将死的禁军的呻吟声和失去主人的战马地嘶鸣声,再没有别的声息,几乎所有的人都静止不动,注视着那对同样有着可怕名头的绝世高手。邪影李顺和血手罗刹闻紫烟。

所有的人心中都有同样的想法,对于这样一个值得敬佩的敌人,就让她死于和旗鼓相当的对手的决斗中吧。人人都知道,若是闻紫烟落败,那么这些凤仪门女剑手也就没有了反抗力量,可是若是李顺落败,若是雍王一方无人能够挽回面子,只怕就是闻紫烟身死,也会重重打击雍王一方的士气。

这时,闻紫烟微微一笑,翻身下马,她爱怜的拍拍马颈,将它驱走,看向李顺。李顺的衣衫早就在突围之时破碎不堪,所以身上穿的是一身戎装,只是没有披甲。他的目光落到闻紫烟身上,露出一丝敬佩和更深的怨恨。看到闻紫烟的行动,他也飘身下马,驱走坐骑,两人就在瑟瑟秋风中对立而望,激荡的杀气冲天而起。

就在众人被两人的杀气所震撼的时候,两人已经由静化动,身影纠缠在一起,雪亮的剑光纵横飞舞,而李顺手指捻着一根玉簪,随着他变化万千的招式,发出刺耳的破空之声。两人越战越勇,旁观之人已经看不清他们的身影,这一场惊人的厮杀没有持续多长时间,闻紫烟早已是筋疲力尽的人,所以她毫不顾惜内力和体力,要在最短的时间取得战果,而李顺本是心性高傲,再说通过和闻紫烟的交手,也可以对他将来可能会对上的凤仪门其他高手有所评估,所以他没有采用避敌锋芒的战术。两人全力交手之下,不过就是数十回合就已经分出胜负。闻紫烟的娇躯如同断线风筝一般坠落,虽然身上有一些小伤痕,李顺却是神采飞扬,通过和闻紫烟的全力交手,他有自信可以应付凤仪门主以外的任何凤仪门高手。

这时,闻紫烟缓缓坐了起来,鲜血从她身下流淌,她却是彷佛不知不觉,摇摇晃晃地站了起来,她的目光缓缓环视了一周,最后落到李顺身上,低声说了一句什么,然后举剑高声道:“胜者为王,败者为寇,李顺,我在九泉之下等着你。”说罢便横剑自绝,一代不让须眉的巾帼女剑客,就此黄土深埋。

这时,秦勇一挥手,千余弓箭手引弓待发,指向剩下的女剑手,秦勇遥遥向雍王施礼,等待他的命令。

那些女剑手面面相觑,虽然她们因为艰苦的训练和有问题的心法变得几乎没有正常人的情绪,可是如此情形,还是让她们心中明白绝无生还指望,死亡的阴影清晰地笼罩在她们身上。所以她们尊重和服从的闻紫烟就成了她们效仿的对象,她们互望一眼,同时举剑自尽,随着这些女子的身躯坠落马下,凤仪门的崩溃开始了。

\chapter{第三十三章 惊天逆转}

九月二十二日傍晚,离猎宫十五里的平远镇已经成了勤王军的大营了,现在猎宫对外的道路已经全部切断,雍王坐在临时的帅帐中感慨万分,他万万想不到会这样子摆脱困境,在来这里的路上他已经问过了秦勇、夏侯沅峰等人前后原委,虽然有些事情他们也不清楚,可是李贽还是明白了大部分经过。

当初留下江哲,李贽其实并没有抱太大希望,想不到竟然真的被江哲力挽狂澜,当听到夏侯沅峰带着矫诏和父皇的密旨去秦勇大营传旨的时候,李贽已经觉得不可思议,更令他震惊的是,当秦勇对着矫诏兵符和密旨还有些迟疑的时候,护送监视夏侯沅峰的太子侍卫总管张锦雄居然拿出了秦大将军的信物。为了调动秦勇的兵马,江哲居然用了三重保险,而且调动的人竟然包括太子心腹之人,这件事情不仅让秦勇和夏侯沅峰越想越是心寒,就连李贽也觉得江哲确有神鬼莫测之机。

尤其是单独召见夏侯沅峰的时候,夏侯沅峰丝毫没有隐瞒,将自己如何猜测江哲藏身含香苑,如何因为私心独自去捉拿江哲,如何被江哲冒死所制,听得李贽不禁钦佩惊叹,同时他对夏侯沅峰也多了几分好感,此人虽然有些阴险和摇摆不定,但是目光倒也深远,若是用得好,倒是一个臂助。所以他对夏侯沅峰颇加抚慰。

对张锦雄,李贽也嘉奖了他的功绩,并且明言不再追究崆峒从前的过错,不过张锦雄倒是没有留下来,他声称背叛太子只是因为看不过去太子和凤仪门的行径,却不愿参与对故主的攻击,所以自请离去。李贽对他这种忠义性格颇为赞赏,亲笔写了一道手令给张锦雄,允许他自由离去。

秦勇也亲来谢罪,说因为自己迟疑犹豫,以至于救援来迟,请雍王治罪,可是李贽倒没有怪罪秦勇。在雍王心目中,秦彝麾下军队既然只忠于朝廷,若是秦勇轻而易举地倾向自己,没有得到上命就来救援,虽然此刻他会觉得感激,但是却会担心将来遇到类似情况的时候,秦勇会因为判断错误而铸成大错。这样看来,秦勇虽然过于谨慎小心,却是拱卫京畿的好人选,所以李贽不仅温言劝慰,还亲解佩剑赏赐给秦勇,让秦勇感激涕零。

经过紧张的军议之后,李贽下令夏侯沅峰带着一些高手先潜入猎宫,增援晓霜殿,然后大军犁庭扫穴,里应外合,一举荡平叛逆。而跟着夏侯沅峰潜入猎宫的就有小顺子,这可是让夏侯沅峰伤透了脑筋。当初夏侯沅峰到秦营传旨的时候,为了防止消息外露,他是先当着众将官之面传了矫诏之后,又私下向秦勇出示了密旨的。所以差点被闻讯赶来的小顺子一掌打死。这还罢了,当小顺子逼问出含香苑之事之后,听说他打了江哲一掌,就一直冷着脸,看向他的目光总是充满杀机。如此种种,怎不让夏侯沅峰抹一把冷汗,此刻他实在是不敢相信自己从前居然动过想把小顺子收到麾下的念头。

夜色刚黑,夏侯沅峰带着十几个武功高强的夜行人接近了猎宫一角,这里十分接近太子居住的玉麟殿,负。守卫这里的乃是夏侯沅峰所控制的侍卫和禁军,所以在夏侯沅峰显身之后,这些人很容易的进了猎宫,然后夏侯沅峰命令他们拿来了一些禁军服饰,众人换上之后,随着夏侯沅峰向晓霜殿掩去。可是路过含香苑之时,果然不出夏侯沅峰所料,小顺子坚持要先去见江哲。夏侯沅峰早有准备,他觉得含香苑离晓霜殿很近,而且现在就去晓霜殿很容易惊动凤仪门,不如等到李贽大军到后再行动。所以他表示赞同。

众人进了含香苑,其他人先到偏殿休息,而小顺子和夏侯沅峰则去了公主寝宫。小顺子自然是第一个闯进了公主寝宫,可是当夏侯沅峰跟进去的时候,却看见小顺子怒冲冲地看向自己,夏侯沅峰四下一看,却看不见江哲主仆,不由吓了一跳,但他转念一想,笑道:“或许江大人避到别处去了,李爷不要过于忧心。”

小顺子觉的他说得有道理,神情渐渐平缓下来,可是这时他却听到有人轻轻向这边接近,他心中一动,过了片刻,有人轻轻推开了门。小顺子看到了董缺,心中不由一喜,问道:“公子何在,他可安好么?”

董缺有些心虚的低下头不敢说话。这下小顺子和夏侯沅峰心中都是一颤,夏侯沅峰可是清楚,若是得不到解药,什么荣华富贵也没有用处,连忙抢着问道:“江大人发生了什么事情?”

董缺无奈地道:“公子被齐王殿下的人劫走了。”这个消息如同晴天霹雳一般震撼了两人,小顺子和夏侯沅峰都是心思敏捷,同时问道:“没有落到太子和凤仪门手中吧?”董缺连忙答道:“没有,我监视了很长时间,公子还在齐王那里,太子那边肯定不知道。”两人的心安定下来。夏侯沅峰怀疑的看了董缺一眼,心想他为什么不随侍在侧,小顺子却是知道原因的,他冷冷道:“我想先去宣华苑一趟。”

夏侯沅峰阻拦道:“齐王殿下和太子殿下不是一条心,如果雍王大军到了,那么齐王绝对不敢伤害江大人,若是李爷现在去了,反而可能会让齐王用江大人要挟李爷。”

小顺子也知道这一点,可是江哲在齐王手中,他很担心最后齐王用江哲做人质要挟雍王,所以他没有说话,可是坚定的目光却显示出不肯妥协的心意。

夏侯沅峰一阵头疼,这时董缺低声道:“公子在夏侯大人走后病情加重,属下无能为力,齐王定会为公子医治,若是李爷现在赶去,若是一个不好,还会害了公子,还是等到雍王到了之后,大局已定,我想齐王不会不识趣的。”

小顺子神色渐渐冷静下来,可是看向夏侯沅峰的目光却越发冰寒,夏侯沅峰哪里不明白他的想法,他是在怪罪自己加重了江哲的病情,可是这个他就只能苦笑了。倒是过了一会儿,小顺子自己叹了一口气,冰冻三尺,非一日之寒,江哲的病情沉重,并不是因为夏侯沅峰的一掌啊。想通了之后,他默默看向窗外,等待发动的时机。公子,希望你能够安然无恙。

在黎明时分,按照预定的计划,夏侯沅峰等人悄然掩向晓霜殿,这里是猎宫防守最严密的地方,在接近晓霜殿的时候,夏侯沅峰让其他人躲藏好,自己一马当先走向宫门,守门的禁卫军同时提高了警惕,几个手势之后,夏侯沅峰已经隐约看见了凤仪门女剑手的白衣。他故意趾高气昂地道:“我乃大内副总管夏侯沅峰,奉太子之命,前来求见陛下,还不速速通传。”几个禁军不敢怠慢,他们知道夏侯沅峰乃是太子一党,无论如何,太子总是将来的皇帝,他们自然不敢得罪。没有多久,李寒幽从里面走了出来,她的神色有些烦恼,这么长时间还是没有雍王落网的消息,她自然十分不安。见到夏侯沅峰,她冷冷道:“夏侯大人,你不是去传旨了么,怎么深夜到此,有什么事情?”

夏侯沅峰神秘地一笑道:“这就要问你们了,闻姑娘虽然武功不错,可是却还是没有捉到雍王,反而是我运气好,如今雍王的人头已经给太子验过了,太子让我前来跟陛下禀报此事。”

李寒幽疑惑地道:“你说得若是真的,为什么太子没有亲自前来禀报皇上,反而让你前来,人头在哪里,我可还没有看到?”

夏侯沅峰左右看了一下,低声道:“公主,这你就糊涂了,这件事情是怎么回事,咱们都是心知肚明。如果太子现在拿着雍王的人头来见陛下,陛下一怒之下有些不妥当的举动,这传出去不好。下官虽然职位卑微,可是常年伴驾,皇上的性情倒还熟悉。少傅大人让我前来先跟陛下透个底,等到陛下生气过了,冷静下来,太子再亲自来觐见。公主不要声张,现在这件事情还没有外人知道呢,就连擒杀雍王的军队我都没有让他们过来,要等到陛下同意禅让之后,才会向天下宣布雍王的死讯。”

李寒幽一皱眉道:“怎么,你见到了鲁大人么?”

夏侯沅峰冷笑道:“公主,这可不是下官责怪你,无论如何我们都是一条船上的人,飞鸟未尽,公主就想藏起良弓了么,下官不想和贵门为难,所以没有放出鲁大人,不过我想和鲁少傅说上几句话,就是萧王妃也不能无理阻挠的。”

李寒幽凤目闪过一丝寒芒,虽然很讨厌鲁敬忠,但是不得不说,这是一个十分稳妥的办法,可是看看夏侯沅峰身后两个低着头恭恭敬敬地站着的侍卫,李寒幽道:“你可以进去,这两个侍卫不行。”

夏侯沅峰脸色一变道:“这不行,说句不客气的话,现在雍王死了,我们都在太子的船上,可是谁知道你们准备什么时候卸磨杀驴,我这两个侍卫乃是心腹亲卫,武功不在下官之下,若没有他们保护,我可不敢进晓霜殿。”

李寒幽误以为那两个侍卫乃是月宗的死士,所以才会藏头露尾,当然这也是夏侯沅峰故意误导她的结果,因此讽刺地道:“你倒是谨慎,罢了,本宫不过是小心一些,难道还会作出什么负义之事么?你进去吧,话可要说在前头,你要是想搞鬼,我可不会放过你,现在我燕师姐、谢师姐都在里面,你们三个人翻不出什么大浪。对了,韦膺呢,他还在巡视么?”

夏侯沅峰笑道:“管他做什么,堂堂的丞相之子,非要和我们这些人争夺功劳,平日里道貌岸然,我可是瞧他不上。”

李寒幽皱眉道:“你也太心胸狭窄了,不就是韦膺瞒过了你的眼睛么,你们今后都要同朝为官,最好不要闹翻了。”

夏侯沅峰冷冷一笑,随口道:“遵命。”,李寒幽见他皮笑肉不笑,那张俊美绝伦的脸上带着讥诮的神色,只当他是嫉妒了,也不再多话,道:“好了,你进去吧,皇上他们都在晓霜殿的正殿里面的暖阁中休息,正殿是不许我们进入的,你自己去请见吧。”

进了晓霜殿,夏侯沅峰这才松了口气,他的目光环绕了一圈,守卫十分森严,西偏殿的门口,一个艳冠群芳的绝色女子手按佩剑,正在那里向自己望来,夏侯沅峰知道那个女子就是燕无双,因为不擅长骑术,因为没有随闻紫烟去追杀雍王,而是来了这里帮助李寒幽。他微微一笑,向那燕无双颔首示意。燕无双微微蹙眉,返回了偏殿。夏侯沅峰这才走到正殿门口,叩门道:“臣夏侯沅峰,求见陛下。”

晓霜殿的正殿除了前面的金殿之外,后面还有六座暖阁,如今雍帝李援等人都在里面休息,只有那些侍卫和秦彝、程殊轮流在金殿守卫,这是为了防止叛逆进攻。听到夏侯沅峰的声音,负责守夜的程殊皱皱眉,若是秦彝,可能会板着脸让夏侯沅峰天亮以后再来。可是程殊性子最是机变,现在无端得罪夏侯沅峰也没有什么意义,便走到殿门口,让守门的侍卫开门。

门一开,程殊就看到夏侯沅峰和他身后两个低着头的侍卫,他正要说只让夏侯沅峰一人进来,一个侍卫轻轻抬起头来,程殊一愣,立刻醒悟过来,冷冷道:“进来吧,你若是想凭着两个手下搞鬼,我老程可不会轻饶你这辜负皇恩的逆臣。”

夏侯沅峰三人进去正殿,殿门再度关上。程殊想要说话,可是他毕竟久经风雨,便道:“皇上还在休息,现在也没有什么君臣礼可讲,你们跟我进去吧。”说罢,领着夏侯沅峰三人向后面走去,那些侍卫虽然有些奇怪,却也没有多问。

皇上所休息的暖阁外面戒备森严,长孙贵妃、颜贵妃和长乐公主歇在另外一处暖阁,程殊带着夏侯沅峰三人一走到暖阁门前,那些侍卫都是用警惕的目光看着夏侯沅峰,他们都知道夏侯沅峰乃是太子一系的人。这时,暖阁门开了,冷川走了出来,目光中带着敌意,他冷冷道:“皇上问发生了什么事情。”

站在夏侯沅峰后面右侧的那个侍卫抬起头,摘下帽子,道:“雍王府李顺奉殿下之命,前来向陛下问安。”

冷川眼睛一亮,道:“你就是邪影李顺?”这个相貌清秀的青年眼神寒若冰雪,而且声音阴柔,正符合邪影李顺的形象,不过他还是用疑问的目光望向程殊以及从旁边暖阁赶来的秦彝。两人的目光打量了李顺片刻,都是肯定的点头。

冷川进去片刻之后,出来道:“陛下召见李顺、夏侯沅峰,大将军和魏国公也请进去。”

李顺和夏侯沅峰走了进去,雍帝李援这两天来心中焦虑,更加显得苍老,为了防止意外,他乃是和衣而睡,此刻他坐在龙床上,目光希翼地道:“贽儿如今情形怎样。”

李顺在南楚宫中多年,自然知道礼节,上前跪下道:“奴才李顺,奉雍王殿下之命,前来问安,如今雍王殿下已经和秦勇将军会合,追杀殿下的叛军已经全部伏诛。殿下大营设在平安镇,今日就会兵发猎宫,只是殿下担心陛下安危,特遣奴才前来禀报。”

李援龙颜大喜,满天乌云终于开始散去,不由跳下床来,在地上转了几圈,道:“夏侯,你是怎么回事?怎么会带着雍王的信使过来。”

夏侯沅峰可不会笨得说实话,道:“陛下,臣和太子交好,只是因为太子乃是储君,绝没有背叛的意思,所以太子派臣带着伪诏去调动秦勇将军麾下的兵马的时候,臣随身也带着雍王司马江哲江大人所给的陛下密旨,秦勇将军忠心不二,立刻发兵救援雍王殿下。见到殿下之后,臣又奉命潜回猎宫作内应。”他这番话虽然不尽不实,可是他既然立下大功,自然也无人揭穿他。

李援笑着点点头,一直以来他都在烦恼政变之事,此刻大事已定,他不由想到雍王司马江哲怎会通过长乐向自己索取印信,莫非从前宫中流传长乐和那人有情是真的不成,可是不说那人乃是南楚降臣,和长乐身份有碍,而且听说此人身体极弱。虽然江哲这次立下大功,赐婚也无不可,可是此人体弱多病,如何能给长乐带来幸福,罢了,还是用别的法子嘉奖于他,谅他也不敢违背礼法,向朕求娶公主为妻。

心中计议已定,李援吩咐去唤醒众人,都到大殿静候雍王的军队,夏侯沅峰更是和冷川商议之后,找到了一个可以让外面的几个高手潜进来的薄弱之处,小顺子和冷川两人一起出手,制住了十几个禁军,接引进了和他们一起进来的援军。虽然这样很冒险,可是雍王即将发动,就顾不得这些了。将这些人暂时藏到殿后,夏侯沅峰连忙去应付已经起了疑心的凤仪门中人。

纪贵妃面色如霜,站在殿门之前坚持要进去,她在深宫多年,早就习惯了勾心斗角,一听到燕无双说起此事,她虽然也觉得合情合理,可是她很怀疑闻紫烟会让夏侯沅峰拣着便宜,抱着宁可杀错,不可放过的心态,她便来到正殿察看。秦彝和程殊挡在殿门之前,不许她进去,虽然两人说是皇上不想见她,可是纪霞却是铁了心要见到李援和夏侯沅峰。对她来说,如果夏侯沅峰说得是实情,那么自己这样做最多是得罪了夏侯沅峰,而魔宗的人她还不放在眼里,若是有诈,那么自己可能就会挽回大局。所以她的言词越来越激烈,李寒幽、燕无双、谢晓彤也都被她召来。虽然三人未必赞同纪霞的看法,可是同仇敌忾之心,让她们至少不反对纪霞的决定。

就在这时,猎宫之外号角长鸣,雍王的军队到了,一个时辰之间,借着夜色,大军偃旗息鼓,悄无声息的潜往猎宫,因为猎宫之中死忠于凤仪门的人不多,所以凤仪门不得已放弃了外围的巡逻,而雍王又让所有军队,人衔枚,马摘铃,马蹄用厚布包裹,就在黎明时分到了猎宫之外。宫内还无人发觉。雍王等到第一线阳光射出云层之时,才下令响起号角,大举进攻。猎宫之中的禁军本就疑虑重重,毫无斗志,雍王带来的军队却是目的明确,顷刻之间就攻入了猎宫之中。

\chapter{第三十四章 晓霜鏖战}

巨变发生,凤仪门如今的弟子多半都是凤仪门主在这十多年调教出来的,当年随着凤仪门主出生入死的那些弟子大半都已经死在战场上,或者仍在门中隐修,这次政变因为凤仪门主的决定,她们并没有参加。李寒幽这些人,武功才智虽然都不错,却是没有受过太多的挫折,一时之间都是手足无措。眼睁睁的看着猎宫的防线被撕破。可是纪霞却不同,她曾经跟着李援转战天下,立刻就明白了现在的处境,也不和李寒幽等人商量,就一声长啸,如同凤鸣九天,这是凤仪门召集弟子的信号。李寒幽也立刻明白了纪贵妃的意思,如今勤王兵到,到么凤仪门所发动的政变已经到达,那么唯一的生路就是挟持皇上突围。所以她高声道:“攻进去,一定要抓住皇上。”

听到她的喊声,秦彝和程殊同时退后一步,李寒幽正要闯进殿中,但是一缕阴柔的掌风迎面而来,李寒幽正要抵挡,心中一动,翻身退出,那人随后走出殿门,虽然穿着侍卫服饰,可是相貌清秀,一双眼睛寒如冰雪,正是已经突围离去的邪影李顺。

李寒幽心中一震,不由后退了几步,看见了一些穿着夜行衣的人跟在李顺后面走了出来,个个神完气足,步履矫健。李寒幽心一横,现在什么都顾不上了,她高声道:“两位师姐,众位姐妹,我们一起上。”燕无双和谢晓彤同时按剑上前,那些凤仪门女剑手也齐齐拔剑逼上,眼看大战就要开始。小顺子冷冷道:“你们也想和闻紫烟泉下相会么?”这一句话充满了杀气,如同三九寒冬一般肃杀,说话的时间也恰到好处,李寒幽等人虽然也隐隐猜到闻紫烟可能不幸,可是这个消息还是让她们心中一惊,不由手上一缓。就在这瞬间,那些黑衣人已经稳稳守住了殿门。李寒幽目光一闪,心中懊恼,现在已经没有了速战速决的机会,只得提剑上前,杀向殿门。这时,晓霜殿宫墙外已经听到了厮杀的声音,而晓霜殿殿门前已经打得如火如荼。虽然凤仪门女剑手的战力强大,可是殿门狭小,剑阵施展不开,更何况对面还有李顺这样一个高手,一时之间虽然占据了上风,却是不能攻进殿门。这时候另外一处偏殿的殿门推开,秦铮搀扶着面色惊慌的窦皇后走了出来。

秦铮听到外面的喊杀声,只觉的如坠冰窟,她想起了毫无自保之力的齐王还在宣华苑,想起了政变失败之后的下场,一时之间忘记了如何动作。

这时,外面传来清啸声,纪霞一皱眉道:“秦铮,还不去接应她们。”秦铮这才如梦初醒,带着一些女剑手冲向宫门。

就在雍王开始攻击猎宫的时候,玉麟殿也是一片混乱,李安魂飞魄散,抓着萧兰问道:“爱妃,快救孤一命。”萧兰心中也很慌乱,这时候他们听见了纪霞的啸声。萧兰无计可施之下,便拽住李安向晓霜殿冲去,这时候,雍王的军队还没有冲进来。但是等到他们到了晓霜殿的时候,秦勇亲自指挥的一支铁骑已经和守卫这里的禁军厮杀起来。萧兰心中一慌,便要冲进晓霜殿。可是秦勇深知里面的内应压力已经很大,若是让萧兰进去只有坏处,所以下令用弓箭和人墙将他们死死挡住。李安只见前面血肉横飞,身边羽箭纷飞,吓得魂不附体,大喊道:“我投降,我投降。”这时候他已经顾不上什么身份了,就差没有跪倒求饶了。跟着萧兰她们一起过来的还有太子身边的侍卫,他们或者贪生怕死,或者早就对太子不满,此刻一见太子如此窝囊,都再无丝毫战意。有的高喊着投降退到一边,有的抛下一切向外溜走。没过多久,太子身边就只剩下凤仪门的人了。而秦铮虽然已经出了宫门,却被挡住,无法接应萧兰等人进去。

这时候,四周开始渐渐沉寂下来,进入猎宫的大军奉了雍王命令,因为宫中有很多被软禁的朝中官员,各处若是没有反抗,就牢牢围住,此刻除了晓霜殿之外已经没有强力的抵抗了。

萧兰扯着太子奋力拼杀,可是周围的禁军却越来越多,那些女剑手虽然厉害,可是她们都只带了一柄长剑,那些擅长沙场厮杀的大雍将士,用长槊远远攻击,她们陷身军阵当中,只能自保罢了。此刻萧兰从没有这样后悔,若是不带着李安,她早已经闯进了晓霜殿了。

当猎宫初步平定之后,得到战报的雍王赶到晓霜殿的时候,正看见萧兰和凤非非一左一右护着太子,她们身边都是大雍将士和凤仪门女剑手的尸体,两人已经是花容惨淡,眼看就要丧命。李贽看到李安蜷缩在地上,全无一丝皇家仪态,便是一皱眉,幸好那些将士都没有向李安下手,看起来除了身上沾染的鲜血之外,倒是没有什么伤口。李贽高声道:“凤仪门叛逆听了,若是束手就缚,还可有一线生机,若要顽抗,别怪本王无情。”

凤非非抬头看去,属于自己一方的禁军已经马上就要支撑不住,而这时,在宫门处苦战等着接应自己的师姐妹也已经支持不住,若是不趁现在冲进晓霜殿,那些如狼似虎的军士已经开始冲进晓霜殿去了,心中一狠,提起李安将他当成兵器在前面挥舞,她心想既然那些军士不敢攻击李安,那么自己不如用他阻上一阻。果然,她这一手让那些将士不敢向她攻击,不得不被她逼开,转瞬之间晓霜殿外凤仪门仅剩的两个女子就冲到了宫门前。

事关太子性命,秦勇可不敢作主,虽然太子叛乱,可是要杀要剐也是皇家之事,还轮不到秦勇作主,所以他的目光看向雍王,等他下令。

李贽心中怒火熊熊,凤非非的作为让他恨得咬牙切齿,虽然对太子,他也是十分痛恨和鄙视,可是无论如何,那是他的兄长,本来想下令将三人乱箭射死的他终于改了主意,这三个人就是进了晓霜殿也起不到什么作用,怎么也不能让自己的兄长在这种情况下死去,皇子自该有皇子的死法。所以他没有作声,任凭那三个人冲进了晓霜殿宫门。

萧兰三人虽然进了晓霜殿,可是随着她们身后,秦勇也已经指挥着麾下将士冲进了晓霜殿,这时,在纪霞、谢晓彤、李寒幽三人和二十多个凤仪门女剑手的攻击之一,虽然有小顺子等高手死命拦阻,可还是被迫退入了正殿之中。

李援在秦彝、冷川等人保护下坐在龙椅上,长孙贵妃、颜贵妃和长乐公主都避在宝座之后,被侍卫护着,当凤仪门众人冲进正殿之后,小顺子等人都不再恋战,迅速退到宝座之前,摆开了坚守的阵势。而在李寒幽等人身后,那原本已经被双方争斗破坏的稀烂的几扇殿门也被冲进晓霜殿的将士彻底撞碎。李寒幽等人围住了李援等人,而她们外面则是投鼠忌器的雍军将士,若是引起混战,虽然凤仪门众人必定被擒杀,可是若是李援、两位贵妃和公主不小心受到一点儿损伤,这里的每一个人都吃罪不起。一时之间,大殿之内一片寂静,每一个人都不敢大声喘气,殿内气氛十分沉闷。

这时雍王排众而入,他冰冷的目光在凤仪门和李安身上掠过,对着李援施礼道:“父皇,儿臣救驾来迟,往父皇恕罪。”

李援欣慰地道:“贽儿你安然无恙,秦将军,你尊奉朕的密旨前来勤王,朕心甚慰。好了,你们不用管朕,给朕将这些叛逆全部杀了。”

李贽苦笑,李援这样说,他可不能这么干,连忙道:“父皇不用担心,现在这些叛逆已经陷入罗网,请父皇保重身体,等到儿臣将她们擒拿之后,交给父皇处置。”

李寒幽冷冷道:“雍王殿下也不要太得意,虽然我们落败,可是皇上和太子还在这里,若是殿下想趁机弑父杀兄,那自然是可以下令进攻,到时候正好铲除了障碍,顺理成章的继承皇位,若是不然,还是和我们好好谈谈吧,也好保住你的父兄。”

李贽也知道需得如此,可是他很厌烦李寒幽的作为,目光在凤仪门众人身上转了一圈,最后还是落到纪贵妃身上。他微微一笑道:“不知道贵妃娘娘有什么意见,若是太过苛刻,只怕就是父皇和本王答应,这些将士也不会答应,叛上逆伦大罪可是诛灭九族之罪,若是本王太过放纵,引起朝野清议,只怕会贻笑天下。”

纪贵妃眼神从迷蒙变得阴森,她冷冷道:“若是要诛九族,皇上和雍王你不也是罪责难逃,现在说什么清议都是废话,只要殿下放开一条生路,我们自然不会伤害皇上。”

雍王目光一闪道:“本王若是现在让开一条出路,你们真的肯就这样走么?”

纪贵妃一滞,若是这样出去,若是雍王反悔,自己这些人岂不是自陷死地,什么千金一诺,她可是丝毫不信雍王不会落井下石。这时候李寒幽突然冷冷道:“这有何难,若是殿下放开大路,再让长乐公主做人质,不就是两全其美了么?”说罢,充满杀机的目光看向长乐公主,她也是冰雪聪明,李援所说的密诏和夏侯沅峰的背叛自然是秦勇率军前来平叛的原因,可是这密诏是如何落到夏侯沅峰的手上的呢?想来想去,只有长乐公主派人出过晓霜殿,眼看荣华富贵成了泡影,李寒幽已经将长乐公主恨透了。更何况,虽然李寒幽也有公主的身份,可是和真正金枝玉叶的长乐公主比较起来,虽然她自负才貌双全,可是心中却总是有些忌惮和嫉妒,所以她才会提出以长乐公主为质。她虽然是私心自用,可是凤仪门众人听了却都觉得是个好主意,李援对长乐公主的宠爱人所共知,果然是最好的人质人选。

李援和李贽却都大怒,他们都因为南楚和亲之事对长乐心存愧疚,怎忍心让她做人质,所以异口同声地道:“不行。”这句话以说出口,殿中局势陡然紧张起来,可是李援和李贽父子四目相对,却都觉得父子两人的心从未像这一刻这样接近。可是李贽看着那些凤仪门弟子面上露出的不肯妥协的神色却是头疼起来,不由心道,我让人去找江哲,怎么还没有找到,若是随云在此,或者会有什么好法子解决现在的事情吧?

雍王在入宫之前就已经安排心腹去寻找江哲,江哲不畏生死,留在险地,运筹帷幄,逆转了大势,此刻李贽对江哲的感激已经到了极至,所以下令若是找到江哲立刻要来禀报,可是直到现在却没有消息,雍王早已在担心江哲的安危了。

从雍王攻入猎宫的一刻起,我就被四个大汉死死的盯着,这几个齐王身边最亲信的侍卫都很担心雍王会趁乱派人来伤害齐王,所以早就劝齐王暂避一下,可是却被齐王轻描淡写的拒绝了,他们无奈之下只有死死盯着我。

这四个侍卫可是知道江哲在雍王心目中的份量的,心想万不得已就用此人做人质,只要等到齐王殿下见到皇上之后,殿下没有参与叛乱,到时候皇上就是再怎么责罚殿下,也不会伤害殿下的性命的。

过了一阵子,外面的喧嚣声渐渐沉寂下来,又过了片刻,有人重重的敲门,一个在宣华苑伺候的太监战战兢兢地前去开门。门一开,一队军士将这个太监推到一边,迅速将宣华苑上上下下全部控制起来。一个青年将领大踏步走向正殿。齐王正负手而立,站在窗前,向外望去,那边正是晓霜殿的方向。

这个青年将领行了一个军礼,虽然齐王也有叛逆的嫌疑,可是和太子不同,齐王在军中的威望也是很高,他的勇猛和直爽很得人心,而他虽然风流好色,又有喜新厌旧的恶名,但却没有抢夺人妻妾的行径,而且他府中姬妾虽多,可是却从来不会用严刑家法约束,凡是姬妾侍婢只要自己愿意,都可以要求出府嫁人,齐王不仅不会为难,反而会送上一份丰厚的嫁妆。

齐王定下这个规矩的起因也是一段佳话,当初齐王府上有一个别人送来的舞姬,相貌十分秀丽,不过齐王宠幸了几次之后就没了兴趣,偏巧这个舞姬青梅竹马的恋人进了齐王府做侍卫,两人旧情重燃有了私情,却被另一个侍卫发现,这个侍卫原想逼迫这个舞姬和他私通,不料这个舞姬坚持不肯,因此一怒之下向齐王密告。齐王果然召来两人问罪,问明实情之后,下令将那个侍卫拖下去打了几十杖,当时人人都道齐王会将这一对恋人杖杀,却没料到齐王将那侍卫责打了一顿之后就将那个舞姬嫁给他为妻,然后又将这个侍卫推荐到下面做武官,反而是那个告密的侍卫被齐王赶出了王府,然后齐王就订下了这个规矩。有幕僚劝谏他说,这样未免有失尊严,谁知齐王笑道:“本王喜新厌旧谁人不知,这些女子在我王府之中独守空闺岂不可怜,不如将她们嫁了出去,也免得耗费本王的钱粮。”

虽然很多持重的文臣因此对齐王多有诟病,可是军中勇士倒是因此对齐王更加爱戴,因为齐王常常召集军中勇士参加宴饮,宴中总是让身边的姬妾舞姬前来歌舞行酒,不乏有被那些美女看中下嫁的例子。

所以这个将领虽然奉命来收押齐王,但是倒没有什么太深的敌意。他高声道:“末将田隆奉雍王殿下之命,前来保护齐王殿下,雍王殿下有命,请殿下不要外出,以免为乱军所乘。”

齐王转过身来,他的面色苍白,可是神色却很安然,他淡淡道:“晓霜殿情况如何?”

那个将领一愣道:“末将不知。”这时他的副手走进来在他身边低声道:“在偏殿之中有几个齐王的侍卫不肯缴械。”

田隆看了齐王一眼,低声道:“这个还要我来教你怎么做么?”

副将为难地道:“他们挟持了一个人,说是雍王司马江哲江大人。”

田隆一惊,他能够被派来监押齐王,自然也是很得信任,所以他自然知道江哲的重要,雍王还特意吩咐众将,若是发现江哲,一定要好好保护。警惕的看了一眼齐王,田隆道:“殿下,能否请殿下下令让属下不得抵抗。”

李显微微一笑道:“本王想去晓霜殿,不知道将军能否作主?”

田隆一脸为难,他可没有这个权力允许齐王去晓霜殿,可是江哲又被齐王属下挟持,这可怎么办呢?这时外面传来一个温和的声音道:“殿下,何必如此呢?”虽然明显中气不足,可是声音十分坚定。田隆和副将向外望去,只见一个青衣书生在两个齐王侍卫搀扶下缓缓走来,另外两个侍卫执刀相护,那书生手中拿着一块金牌,却是“如朕亲临”的金牌,本来现在这块金牌未必有用,可是金牌右下脚却有一行小字,写着“钦赐雍王李”,说明这块金牌乃是皇上赐给雍王的,所以无人敢阻拦。

田隆立刻知道这个书生果然是雍王司马江哲,连忙上前见礼。

我挥手让那两位将领退到一边,道:“殿下,如今大局已定,不可挽回,您又何必去晓霜殿呢?”

李显淡淡道:“就是因为大局已定,我才要去看看,你应该明白,我的王妃在那里。”

我摇摇头,有的时候齐王真的很是固执,想了一想,终于道:“下官要去晓霜殿,如果殿下不嫌弃,就和下官一起去吧。”

李显面色一变道:“你的身体什么状况,难道自己不清楚么,这个时候去逞什么能?”

我微微一笑,道:“今日是我日思夜想,想要见到的一天,怎能在这里苦苦忍耐,请殿下将轿子借给我一用。”

李显神色变了又变,道:“好吧,本王答应你。”

田隆两人惊叫道:“殿下、大人,这个?”

我举起金牌道:“雍王面前,自有下官承担罪责,与你们无关。”两人这才默然不语。

就在这时,远处的旷野之上,一个白色的淡淡身影仿佛流星一般迅捷,那方向直指猎宫,秋风吹过,一方白色的丝绢滑落在地,露出绢帕上面殷红的血迹。

\chapter{第三十五章 情深似海}

今天修改了前面一些章节的细节部分,或者是原来忘记写了,或者是觉得有bug,所以发新文完了,如果有时间的话,可以重新看一下

第三部二十四章至三十四章,虽然修改的不是特别多,可是我觉得能够解决很多读者心中的疑惑。

—————————————————————————————

晓霜殿之中,谈判正陷入僵持之时,站在龙椅一侧保护雍帝的小顺子略一皱眉,毫无征兆地飞身而起,凤仪门众人只道他要偷袭,几乎是同时上前一步,就要发起进攻,而保护雍帝的侍卫和武林高手都在心中抱怨小顺子鲁莽急躁,只得略略后退,缩小了保护圈,眼看混战就要爆发,谁知小顺子却向龙椅之后那扇上面绘着山河地理图的锦绣屏风扑去,屏风后面是通向暖阁的宫门,秦彝早就令人将那扇宫门锁上,再加上大军早已经将晓霜殿重重包围,所以也无人留心那里的动静。可是就在小顺子向那里扑去的时候,一道耀眼的剑光闪过,锦绣屏风被剑气撕裂,一个青色身影电射而来,正被小顺子截住,两人凌空交手,仿佛苍鹰夜隼,盘旋往复,那青衣人不过数招就已经身形迟缓,被小顺子一掌击中,只听那人一声闷哼,从半空中坠落,这时,纪贵妃目光一闪,纤足飞踢,一柄落在地上的单刀被她踢到了那人身下,那人在空中一个翻身,右足点在单刀之上,借力飞起,轻飘飘的落在凤仪门剑阵之侧,青衣人目光阴冷的看向小顺子,冷冷道:“想不到我韦膺一番苦心,竟被你这阉奴破坏。”

却原来韦膺发觉雍王进攻猎宫的时候,丝毫没有犹豫就直接赶来晓霜殿,可是到了之后,他发现李寒幽等人正在强攻正殿。韦膺心思灵敏,知道自己就是加入也没有什么用处,于是绕到正殿后面。原本为了防止有人从后面刺杀,正殿后面的处处都有机关,将出入口全部封闭起来。若是旁人绝没有办法在一时半刻之间进去。可是韦膺出身丞相之家,自己又是高官,他曾经在工部任职,曾经私下偷阅过皇家各处宫殿的建筑图,而且他对宫室营造本就颇有经验。所以不过花了两拄香时间就进入了宫中。等他用身上削铁如泥的宝剑轻轻破坏了宫门,躲在屏风后面最接近李援龙椅的位置的时候,却又苦恼地发现,凤仪门还没有冲进正殿,李援身边有冷川和几个武功不错的侍卫保护,他若是出手,绝对没有办法一举成功,只得暂时隐忍下来。直到方才因为雍王等人到来,而凤仪门几乎所有幸存的人都被困在殿中,因此冷川等人全神贯注地提防着这些凤仪门弟子铤而走险的时候,他才觉得找到了好机会,准备一举擒下雍帝。谁知他杀机才动,就被小顺子发现,而且抢先出手,将他逼了出来。

韦膺受业于凤仪门主,对于刺杀本是颇为擅长,当初他就曾经在朱雀门前刺杀过侍中郑瑕,可是他毕竟不是身经百战的绝顶刺客,行动之际不免露了一丝微弱的杀气,被武功高强,感觉灵敏的小顺子察觉。此刻,他秀雅的脸庞上满是狰狞之色,若是挟持了李援,无论他提出什么条件,李贽也不得不屈从,他们就可以安然脱身了,想不到大好的机会却被小顺子破坏无遗。

见此情景,雍王等人都是又惊又喜,若是李援被挟持,那么只要凤仪门提出的要求不是太过分,他们都不得不接受,否则雍王难免给人留下借刀杀人的话柄,这一点在如今,雍王拥有大义名份之后,是绝对不能容忍的,因此看向小顺子的目光都是感激的神色。

小顺子却对众人感激的目光视若不见,心中只在想着猎宫已经平定,那么公子怎么还没有消息。正在盘算的时候,外面传来嘈杂的声音。一个将领匆匆跑了进来禀道:“启奏陛下、雍王殿下,齐王殿下和天策帅府司马江哲江大人求见。”

李贽和小顺子都是大喜,李贽也顾不上齐王怎会出现,道:“快宣他们进来。”话音一落,才想起父皇也在,连忙向上面一揖,表示谢罪。此刻李援却也十分欢喜,虽然他对江哲和长乐公主之事不表赞同,可是正是江哲的计策,才召来了勤王之军,刚才他又被小顺子所救,所以他也没有不满雍王的行为,反而高兴地道:“正是,快宣他们进来。”

没有多久,齐王脚步沉重的走了进来,两个侍卫搀着江哲跟在他身后。虽然直到晓霜殿前才下轿,总共走了不到百步路程,江哲的面色已经是苍白如纸。雍王一见心中大痛,不过两三日不见,江哲却已经是病骨支离,两鬓竟然星霜斑斑。李贽连忙上前伸手相搀,眼中含泪道:“随云,都是本王害你如此,你,你——”语不成声,竟然再也说不下去。

我自然知道雍王为何这样伤情,事实上昨日我在铜镜之中看见自己的容貌,也是大吃一惊,现在我可是相信了一夜白头之说了,不过幸好,我不过是添了几缕白发罢了。倒是小顺子一见我如此憔悴,立刻面色铁青,再也顾不得什么皇上和凤仪门,飞身扑到我身边替我诊脉。这一年多来,他已经开始学习医术,虽然还不能独自开方,可是诊脉和针灸倒是已经有了几分火候,这可能和他内力高强、心思细密有关。我可不敢看他越来越皱紧的眉头,向前望去,只见长乐公主面上露出惊骇之色,望着我的目光满是痛惜关切,若非是她生性端庄贞静,再加上长孙贵妃轻轻扯住了她的衣袖,只怕已经是忍不住要走下御阶了。我露出温和的笑容,劝慰的看了长乐公主一眼。向上施礼道:“臣江哲叩见陛下。”

这时神色有些茫然的齐王才在身边侍卫的提醒下上前施了一礼道:“儿臣叩见父皇。”

李援看了齐王一眼,目光落到秦铮身上,微微皱眉,这时颜贵妃神色惊惶地看向李援,李援叹了一口气道:“显儿,今日之事真相未明,你先退到一旁,若是你没有谋逆之举,想必你二哥也不会责怪你。”

李贽看了李显一眼,眼中闪过复杂的神色,道:“六弟先到一旁休息,待我平乱之后再和你慢慢叙谈。”一边说着话,李贽一边做了一个手势,一个伶俐的侍卫连忙去搬了一把椅子过来,放到我身边,我用请示的目光向上望去,雍帝点点头,示意我尽管坐下。我又施了一礼,这才坐下,擦擦头上的冷汗,笑道:“臣体弱多病,让皇上见笑了,殿下也不用担心,臣幸得齐王殿下延医救治,性命已经无碍。”

李贽心中一动,看向李显的目光多了几分柔和,李显却是目光呆愣,只是看向秦铮,秦铮却是低着头,看不清神色如何,只是不时有几滴晶莹的水珠坠落地上。

李贽神色雍容地道:“父皇,这些事情我们慢慢再说,还是先将这些叛逆擒住才是,韦膺,李寒幽,你们犯上作乱,罪在不赦,若是束手就擒,父皇念在你们年轻无知的份上,或者还可法外施恩,否则你们都有亲朋好友,难道不怕族诛之祸么?”

听了雍王的喊话,我微微一笑,目光一转,看到了站在雍帝身边,一脸忠心耿耿的夏侯沅峰和站在长乐公主之侧,虽然手拿佩剑,却是神色木然的秦青,不由想起我初入大雍朝廷参加的那场盛宴,这三人被并称青年俊杰,可是历经大浪淘沙,却成了今日情状。

韦膺在江哲一进来就心中烦恼,他比凤仪门那些眼高于顶的女子更加看重江哲的才智,所以在宁愿得罪长乐公主也要搜查含香苑,不知怎么这人一进来,他心中就生出不祥的预感,为了摆脱这种感觉,他冷冷道:“雍王殿下何必说的冠冕堂皇,殿下想夺取皇位已非止一日,谁不知道这位江司马就是殿下的智囊军师,太子殿下本是储君之尊,如果不是雍王你咄咄逼人,太子何必行此不得已之事。昔日汉武帝一代明主,只因存了废立之心,以至太子在忠臣辅佐下不得不谋反,虽然太子最后身死,可是武帝却作思*与归来望思之台以怀念太子。今日我等虽然落败,可是殿下难道不是也想趁机夺取皇权么,只怕今日之后,皇上就会被你软禁宫中,若不杀了我们,恐怕殿下会担心难以堵塞天下悠悠众口吧?”

我见韦膺言辞犀利,雍帝和其中众人面上都带了犹疑之色,便扬声道:“韦大人此言真是颠倒黑白,太子殿下虽然是储君之尊,却是失德败行,朝野谁不知晓,雍王殿下功高盖世,虽然因为长幼有序,不能继承大统,可是殿下从无嫉恨之心,反而是殚精竭虑,为大雍社稷呕心沥血,原指望太子殿下宽厚仁德,善待功臣手足,我家殿下也就情愿屈身为臣。可是太子殿下只知妒贤忌能,屡屡加害雍王殿下,更是贪淫酒色,为所欲为,君子耻以为伍,小人逢迎鼻息,如今更是犯上作乱,全无君臣父子情分,更是矫诏相召,意图加害我家殿下。若非殿下仁德感天,众位将军侠士舍生忘死,早已经身死猎宫。如今殿下奉陛下密旨,率大军前来勤王,此是顺天应人之事,尔等叛臣,不思悔改,反而意图离间陛下父子,真是万死难赎其罪。”

韦膺怒道:“江司马,你虽然是雍王宠臣,可是官职卑微,这大殿之上哪有你说话的地方,想当初,你是南楚状元,翰林学士,南楚两代国主以及德亲王赵珏待你皆有深恩,可是你枉读圣贤之书,为了苟全性命,投降奸王,为他出谋划策,设下无数诡谋,太子性情忠厚,误入你彀中,以至今日身败名裂,像你这种不忠不义的贰臣贼子,还敢人前出言,我等举义旗,清君侧,虽然落败,却也不是你这种小人可以诬蔑凌辱的。”

我面上露出讥诮之色,挥手阻止了雍王想要出口的怒喝,道:“韦大人,当初江某受南楚君恩,却投降大雍,这贰臣之称我认了。可是自古道,君不正,臣投外国,所谓良禽择木而栖,良臣择主而侍,江某在南楚也有微薄功劳,也曾上书直谏,可惜主上不纳忠言,将我贬斥为民,在我归附大雍之后,南楚又遣刺客来袭,说起来,是南楚弃我在先。雍王殿下不嫌弃江哲无能之人,解衣推食,哲纵是铁石心肠,又怎能弃之不顾。哲入殿下幕中,常年卧病,不能为殿下分忧解劳,可是殿下却从无嫌弃之心。雍王殿下有伯乐心肠,礼待天下贤士,江某不过是马骨一般,王仍以重礼优待,所以江某甘心这贰臣之名,死而不悔。可是这贼子二字,江某却是愧不敢当。韦大人,令尊身为丞相,领袖群伦,韦大人你少年中举,一日三迁,晋升之速,天下罕见,未至而立之年,已经身在中枢,相阁之位迟早是大人囊中之物,可是大人不念君恩深重,勾结叛逆,挑唆太子不顾君臣之别,父子之情,犯上作乱,这贼子二字,除了韦大人你,还有何人可以承当。”

我的声音刚落,殿中响起喝彩之声,魏国公程殊高声道:“江大人,你说得真是痛快,老程是个粗人,早就想痛骂这小贼一顿,只是俚语粗俗,不敢君前失仪罢了,韦膺,你这贼子背弃皇恩,早该千刀万剐,才是不配在这大殿之上说话呢。”

韦膺面色一时铁青,一时潮红,他心中后悔不该忘记江哲此人言辞如刀,当年此人在蜀中一曲新词,迫使蜀王自裁,在大雍新春华宴之上,更是将秦青的攻讦化为乌有,自己怎会如此糊涂,和他在口舌上争起高低来了。

他深吸了一口气,正想绕过这个话头继续谈判,突然谢晓彤的娇躯开始摇晃,然后是秦铮、李寒幽等人,一个个凤仪门弟子开始摇晃、软倒,只有萧兰和风非非虽然神色惊慌,却没有软倒,韦膺大惊,他知道若是凤仪门这些帮手出了问题,那么自己绝对没有挟持皇上的能力,没有了投鼠忌器的顾虑,自己这些人马上就会死无葬身之地。

虽然殿上之人除了大雍君臣就是军令森严的将士和功力精深的武林高手,因此无人慌乱惊叫,可是眼中都流露出莫名其妙的神色,一些不够深沉的人脸上也露出惊容,面面相觑。我却疲惫地道:“殿下,大事已成,可以动手了。”

雍王看了我一眼,眼中满是震惊,可是也顾不上问我,挥手就要下令将所有叛逆全部擒拿。

李显自从站到一边之后,他的目光就一直盯着秦铮,秦铮却是始终不肯抬头,两人浑然忘了周围的一切,直到秦铮也软倒在地,李显才惊叫一声,就要举步上前,却被身边的心腹侍卫拉住了,那个侍卫低声道:“殿下不可授人以柄。”李显不得已收住了脚步。

就在雍王挥手下令,在冷川率领下,十几个侍卫向韦膺三人扑去的时候,突然间一声巨响,泥沙碎木和金色绿色的琉璃瓦片纷纷而下,大殿顶上已经穿了一个大洞,白影闪动,直堕而下,伴随着一声如同凤鸣九天一般的轻啸,直向雍帝李援扑去。众人大多都被那啸声震得心旌动摇,只觉得周身无力,全无阻止之力。只有冷川和小顺子同时一声怒叱,飞身拦截,两人都是身影如电,全力出击,谁知那白衣人衣袖一拂,冷川和小顺子都被那激荡的劲风震得踉跄后退。不过冷川和小顺子都是跻身绝顶高手的人物,那人虽然一举将两人逼退,却仍然是速度缓了一缓,就在这瞬息之间,雍帝身边的侍卫和武林高手都各自施展绝技拦阻,可是一声龙吟,那人手中多了一柄长剑,只听见十几声脆响,那些护卫雍帝的高手都被那人刺中,更有一人被那白衣人一剑斩去了首级,鲜血四溅,九级御阶,成了血腥屠场。那人瞬息之间,已经到了雍王面前。长孙贵妃和颜贵妃早就吓得不能动弹,可是就在那人出剑斩杀侍卫的时候,两人不知哪里来得勇气,一起向李援扑去,长孙贵妃离得近些,扑到李援身上,将他要害挡住,颜贵妃虽然慢了一些,可是她张开双手,挡在李援和长孙贵妃之前,那人似是微微一愣,长剑指在颜贵妃胸口,却没有刺下去。这时,长乐公主和李显同时惊叫道:“父皇、母妃!”

众人这才反应过来,向那人望去,那人身形婀娜,一身雪衣,青丝如墨,一条雪白的丝巾掩住了大半面庞,那人长剑虽然只是指着颜贵妃,可是众人却都觉得只要她一剑刺下,皇上和两位贵妃都别想保全性命,都是一口大气都不敢喘。

就在这时,殿内突然响起了剧烈的咳嗽声,那雪衣人虽然威势如旧,可是不知怎么人人都觉得她的杀机似乎少了几分,不由心中一宽,应声望去,想看看是谁想出这个法子摆脱刚才的僵局,一看之人,不由都是一声惊呼。只见江哲用一块雪白的绢帕捂住嘴,咳嗽不止,转瞬间,那块绢帕已经渗出了殷红的血迹。却是江哲被那刺客啸声中蕴含的内力所伤,正在咳血不止。

小顺子目中闪过冷电一般的寒芒,面上的严霜更加凝重,他飞身回到江哲身边,取出一粒黄色蜡丸,剥去腊衣,露出雪白的龙眼一般大的药丸,顿时满殿都洋溢着沁人心脾的药香。小顺子将药丸塞到江哲口中,过了片刻,江哲神色渐渐平和,也不再咳血,他用丝帕想擦去唇边的鲜血,可是那块绢帕已经是被鲜血浸透,竟然无法再用。

这时,站在御阶之上的长乐公主缓缓向下走来,她若想走下御阶,必然要经过那雪衣女子的身旁,所以李援和长孙贵妃同时惊叫道:“贞儿,不要胡来。”

可是长乐公主却是仿佛没有听到一样,缓缓的走过那雪衣女子身边,两日来的忧虑和难以入眠,让长乐公主的花容带了几分憔悴,可是此刻她那失魂落魄的神情却是那样惹人爱怜。她慢慢走到江哲身边,单膝跪下,拿起手中丝帕就要替江哲擦拭血迹,可是一拿起来,才发觉那块丝帕已经被她在焦急中扯坏了。她眼睛微眨,晶莹的泪珠坠落在月白的凤裙上,她眼中一亮,用力撕扯裙袂,裂帛之声在殿中清晰可闻。终于,她撕裂一块月白的锦缎,然后轻轻的替江哲擦去面上的血迹。然后,长乐公主低下螓首,伏在江哲膝上,轻声哭泣起来,一时之间,大殿之内鸦雀无声,只听见长乐公主强自压抑的啜泣之声。

我服下那粒桑先生千叮咛万嘱咐托付给小顺子的“九转护心丹”,知道自己的小命终于再次保住了,可是长乐公主的举动却让我完全的呆愣住了,一直以来,我对长乐公主都是怜惜多于爱慕,可是这一刻,我真真切切的感觉到长乐公主对我的一片痴心,不由心中生出万缕柔情。我也顾不得什么君臣礼法,男女之别,伸出手去轻轻抚摸她的秀发,从没有像现在这样,我清楚的知道,这个女子,已经占据了我心中一个重要的位置。

殿中众人都是深吸了一口冷气,长乐公主拒绝了雍帝所选驸马之后,不是没有人猜测过她可能有了意中之人,凤仪门和太子也曾经散布流言,不是没有人听说过江哲和长乐公主彼此情钟的流言。可是这两人,一个深居简出,一个贞静自守,几乎是没有任何见面,所以众人大多只当作传言罢了。可是眼前的情景却让他们第一次相信了那个传言,可是奇异的,人人都没有觉得这两人违背了礼法,反而心中生出强烈的同情和怜惜。

这时,那个雪衣女子收起长剑,缓缓转过身来,白色的面纱之上,那一双璀璨如寒星的眼睛轻轻一转,殿中人人都觉得那女子正望着自己,那冰寒刺骨的目光仿佛一记重锤敲击在心上,都不由后退了几步。

李贽深吸了一口气,道:“凤仪门主芳驾至此,本王不胜荣幸,但不知门主有何指教。”

\chapter{第三十六章 以退为进}

凤仪门主的目光落到了江哲身上,眼中闪过一丝莫名的神采,她用清冷的声音道:“雍王殿下,事已至此,不论我凤仪门本意何为,对于大雍来说,已经是叛逆仇敌,殿下就是想将凤仪门斩尽杀绝,也无人可以阻拦,本座至此,只是想提醒殿下一件事情,虽然殿下如今依然占据优势,可是只要有本座在此,那么殿下就要顾虑一下自身的安危。

皇上和本座乃是患难之交,所以本座可以不对他下毒手,可是两位贵妃、长乐公主、雍王你、齐王李显,还有这些忠臣勇将,若是本座愿意,你们一个也别想逃出晓霜殿去。虽然本座这些弟子也会因此丧身猎宫,可是我凤仪门还有一些隐藏的力量,不会因此一蹶不振。说起来这原本是本座的一点私心,我那些师妹和侍女都是身经百战、劫后余生之人,我不愿她们再涉险境。另外,本座也有些看轻了雍王殿下,以为凭着韦膺、寒幽等人就可以顺利夺宫,留下这份力量也可以应对魔门可能的挑衅。

殿下,你应该清楚的很,不论你我双方谁胜谁负,北汉魔门都不会放过这个机会,否则殿下和齐王也不会都事先传下密令,令大军严守关隘,防止北汉的突袭。现在若是殿下不肯网开一面,那么本座也只能大开杀戒,不过本座一定会放过殿下的性命,然后回去率领本门残余,在大雍境内掀起动乱,到时候,内部不稳,北汉军趁机入寇,大雍社稷内忧外患之际,殿下虽然活着,却恐怕会后悔莫及,只恨未死吧。”

她的声音虽然平淡清冷,可是殿中众人听了却都是心中冰寒。李援这时候已经扶起长孙贵妃,缓缓站起道:“梵门主,不要意气用事,门主和大雍乃是休戚相关,若是大雍社稷危亡,门主也有不测之祸,虽然这次贵门弟子犯下大错,但是凡事都可商量,还请门主息怒。”

他这样一说,殿上众人都是哗然,不论凤仪门主武功如何高强,始终都是叛逆,李援身为君王,怎能如此软弱。李贽一皱眉,看了父皇一眼,道:“父皇所说也是本王所想,门主为人光风霁月,这叛逆之事或者不是门主主使,只要门主痛下决心,将这些叛逆交给本王处置,然后门主若是愿意,大雍皇室愿意为门主修建宫室,以供门主清修。”

李贽虽然说是赞同李援的决定,可是人人都听得出来,李贽是要凤仪门主亲手杀了参与谋逆之人,然后自愿被软禁起来,到时候凤仪门被清洗之后,就只能成为皇室的附庸,而凤仪门主虽然参与谋反,可是若是能够将她控制起来,凭着她宗师的身份,倒也可以震慑北汉的魔宗。这也是李贽无可奈何之举,负责牵制凤仪门主的慈真长老影踪不见,而凤仪门主却来到猎宫,联想到凤仪门主三大宗师之首的身份,那么慈真长老恐怕已经遭遇不幸,这样一来,凤仪门主若是再背离大雍,那么大雍就失去了可以和北汉魔宗抗衡的人选,所以李贽虽然对凤仪门深恶痛绝,却也不得不提出妥协。

梵惠瑶眼中闪过一丝得意的神色,正要开口说话,却只听见一声脆响,举目望去,却见江哲神色清冷,长乐公主已经站起,站在他身侧,秀眉微蹙,望向江哲的眼中充满担忧,而在江哲脚下,一块晶莹透明的玉玦四分五裂,显然是江哲将身上所佩玉玦掷碎在大殿之上。

李贽神色一惊,这两年来,他若是见到一些竹扇、砚墨及风雅玩好之物必然令人收买,送给江哲赏玩,这块玉玦就是年前送给江哲的,若论起材质,虽然珍贵,却也平常,难得的是刀工精美,背面更刻了一幅鸿门宴的图画,虽然只有寥寥几笔,却是气韵生动,形神兼备。江哲对这块玉玦十分心爱,所以一直戴在身上。今日却将玉玦掷碎,看来是愤怒非常。

可是还没等李贽作出反应,江哲已经微笑道:“门主今日身履险地,哲窃为门主不值,所谓千金之子,坐不垂堂,门主何必为这些叛逆张目,慈真大师乃是宗师身份,虽然可能比门主稍逊一筹,可是门主想要轻易脱身,也是不可能之事。江某略通医理,虽然门主用药物维系一线生机,可是若是想保住性命,还是不要轻举妄动为好。否则哲之微命可以双手奉上,但是门主却也别想活着离开猎宫。小顺子,如今这殿上,皇上乃是九五之尊,雍王殿下、齐王殿下都是大雍社稷重臣,如果我要你不必顾忌我的生死,你有没有把握保住至少一个人呢?”

小顺子冷冷道:“公子放心,奴才虽然无能,也绝不会让凤仪门主为所欲为。”

我的笑容更是欢畅,继续道:“门主,普天之下,莫非王土,率土之滨,莫非王臣,无论如何,凤仪门还在大雍境内,乃是大雍子民,今日皇上和两位殿下只要有一人生还,凤仪门和贵门的盟友也别想留下一个余孽。到时候不止门主一世声名毁于此地,就是大雍朝廷也必然损失惨重。无论如何,大雍立国,门主有大功于焉,若是大雍社稷危亡,凤仪门犯上谋逆,危及国家神器,只能留下千古骂名,为后世所不齿,就是门主又有何颜面对天下人呢?”

凤仪门主面沉如水,似乎对江哲所言丝毫没有动心,可是李贽的眼睛却是一亮,若是凤仪门主已经身负重伤,那么自然是斩尽杀绝的好,想来江哲投玦于地,是在催促自己不可犹豫迟疑,促使自己下定决心吧。他的目光一闪,已经暗中打了几个手势,殿中众人迅速组成三个军阵,将雍帝李援、雍王李贽、齐王李显护在当中。虽然众人为了顾虑激怒凤仪门主,没有轻举妄动,可是人人都下定决心,一定要在凤仪门主发动之时,保护好这三人。就是保护齐王李显的侍卫和将士也都下了狠心,宁愿用生命换取李显存活的可能,谁人不知,现在除了雍王之外,李显也是有能力接掌皇位的人选。

凤仪门主心中一叹,看向江哲的目光更是带了几分杀气,这时,小顺子和冷川同时向凤仪门主跨进一步,若是凤仪门主发动,那么这两人就是阻挡凤仪门主的主力。

这时候,我见压住凤仪门主的气势的目的已经达到,若是再强迫下去,让凤仪门主铤而走险,那么结果就未免有些凄惨,便道:“门主,如今虽然我方可以斩尽杀绝,可是顾念门主的功劳,雍王殿下还是希望能够和门主达成协议,现在贵门弟子大多身中迷毒,若是混战一起,她们必然首先死在刀剑之下,若是门主肯退让一步,那么化干戈为玉帛也不是不可行的,就是这些涉入谋逆的贵门弟子,江某也可以作主放过她们。”

凤仪门主冷冷一笑道:“江司马果然好算计,不知雍王殿下也是这样的意思么?”

李贽高声道:“江司马所言就是本王的决定。”他心中有些疑惑,江哲所言含糊不清,可是似乎并不想凤仪门主交出参与叛逆的弟子,这个条件岂不是更优厚,但是他素来相信江哲,所以没有阻止。

凤仪门主轻轻一叹,她却是明白,江哲的用意不在于那些凤仪门弟子,而是在于自己,若是自己果然留下这些亲信弟子,那么日后还如何统领凤仪门,必然是众叛亲离。再加上江哲点出了自己身负重伤的事实,那么雍王就会不惜代价围杀自己,到时候凤仪门自然是损失惨重,自己也别想生离此地。可是若是如此,江哲大概心痛围杀自己所要付出的代价,所以才会先挑明自己无法尽杀重要的人物,然后又点出自己身负重伤的事实,再暗示自己,有小顺子这样的高手存在,自己是绝对没有可能生离猎宫的,这样一来,所谓的退让一步,既然答应放过凤仪门剩下的这些弟子,就只有是自己自尽以谢天下了。

凤仪门主心中思虑万千,若是她身上无伤,自然是来去自如,那么江哲的这个目的就只是笑话了。可是慈真大师佛门神功天下无双,她是拼着重伤才将慈真大师击败的,虽然老和尚已经迫于承诺,回去养伤,短时间内不会来阻碍自己,可是为了赶到猎宫挽回大局,她的内伤已经十分沉重,如果不是服下那粒救命的丹药,此刻凤仪门主恐怕已经不能出手了。可是即使有药力相助,若是再经一番苦战,自己只有一个结果,就是气散功消,心脉尽断。而有了小顺子这样的高手存在,自己无论如何也不能同时杀了雍帝父子三人,到头来,不仅自己命丧九泉,就是自己的这些弟子也是一个都不能逃生。

微微苦笑,凤仪门主心想,想不到自己一世英雄,却被这个手无缚鸡之力的文弱书生逼杀于此。本以为倚仗宗师的声威,可以迫使雍王屈服,想不到江哲竟然看穿她的伤势,是啊,自己怎会忘记,这个江哲的医道师承何人呢?方才邪影李顺给江哲服下的九转护心丹不就是明证么?而且,自己若不是服下了二十年前那人亲手所赠的九转护心丹,只怕现在也没有法子站在这里了。

虽然面覆白纱的凤仪门主神色如何,旁人看不出来,可是只见她沉默不语,就知道江哲所言非是虚假,有些心思灵敏的人也想到江哲用意,可是逼杀凤仪门主,这可能么,所有的人都自动摒弃了这个想法,所以仍然在猜测江哲的用意所在。

良久,凤仪门主轻轻叹息了一声道:“退让一步,也不是不可,若是雍王殿下现在肯答应放走我这些门人,并且七日之内不下令追杀,那么本座就可以答应这个条件。”

我看了李贽一眼,他神色有些迷惑,却是仍然轻轻点头,而李援原本就不想激怒凤仪门主,自然也是没有出言拒绝,我目光一闪道:“这个条件雍王殿下并无异议,不过太子李安还有韦膺都不是贵门弟子,可不能算在其中。”

凤仪门主淡淡道:“李安乃是皇家之人,本座不会去管,韦膺乃是本座记名弟子,必须离开。”

我只要留下李安已经心满意足,便道:“既然如此,我们也无话可说,不过这里不适合休息,门主还要在此监督皇上和殿下七日,总不能这样耗着,若是门主允许,我们为门主准备清静之地,供门主休息如何。”

凤仪门主忽然心中一动,道:“这也无妨,不过本座需得留下人质在旁,否则若是你们背信,本座岂不是找不到人来杀了。”

我早有准备,坦然道:“皇上乃是九五之尊,雍王殿下还要掌控大局,两位贵妃娘娘和公主殿下都是饱受惊吓,怎忍让她们继续担惊受怕,诸位将军还要约束兵马,朝中大臣就是愿意为质,只怕门主也是信不过的。如果门主不嫌弃,齐王殿下和江某都可以作为人质,如果皇上和雍王殿下有背信之行,门主可以取我二人性命为偿。”

凤仪门主淡淡一笑道:“江司马倒是会选人,也好,本座同意就是,不过我也要说个清楚,如果皇上和雍王殿下在七日之内想要离开猎宫,可别怪本座不顾承诺。”

李贽看了李援一眼,出声道:“门主既然这样说,本王和父皇七日之内也不会离开猎宫,以示诚信。”

这时人人都觉得江哲果然才智过人,虽然不知道他和凤仪门主到底达成了什么协议,可是至少可以暂时稳住凤仪门主,七日之内,足够众人做好妥善安排,到时候凤仪门主就是再度发难,也未必会比现在损失更大,而且若能妥善解决,倒也不失上策,毕竟现在人人都担心凤仪门主大开杀戒,至于那些叛逆,总可以慢慢处置的。

而且江哲所选的两个人质也是十分巧妙,他自己愿意做人质,自是心存忠义,而齐王做人质也是将功赎罪的机会,想来也不会拒绝。而雍王绝不会忍心牺牲江哲,李援也绝不会忍心牺牲齐王,这样一来,既可以让凤仪门主安心,也不会引起担当人质之人的不满。所以即使最后不能将那些叛逆治罪,对于已经可以将凤仪门的势力全部清除的大雍朝廷,已经是所得胜过所失了。

就在人人松了一口气的时候,李寒幽突然高声道:“师尊,师尊,就是李贞、江哲和夏侯沅峰坏了我们的大事,师尊可不能放过他们。”

凤仪门主瞧了一眼李寒幽,眼中闪过一丝失望,道:“寒幽,不要说了,江司马,想来我这些弟子中毒都是你的杰作,却不知你是如何下毒,解药何在?”

我早有准备坦然道:“晚生早就担忧,如果雍王殿下带兵前来勤王,若是贵门挟持陛下等人,我们投鼠忌器,不敢进攻,该如何是好。为了顺利救出皇上,所以晚生请长乐公主派遣心腹从前日开始,将晓霜殿的香炉中燃烧的香料换成了南疆出产的逍遥香,这种香料气味沁人心脾,人若闻了神清气爽,说起来也是侥幸,凤仪门弟子大半是常年生活在富贵豪门,对于燃香这等雅事是不会阻止的。可是这种逍遥香若是连续闻上十二个时辰,再吸入另外一种南疆特产的乌头草,就会令人四肢酥软。晚生不顾病体坚持赶来晓霜殿,就是为了带来乌头草粉末精制的药膏,再让小顺子用内力催发乌头草药物。由于诸位被江某身上的药香混淆,所以没有留意到乌头草的气味,而且公主也早就将解药混入酒中给皇上和诸位大人服下,所以才会只有贵门弟子中毒。”

凤仪门主淡淡一笑道:“江司马不愧是医圣弟子,精于混毒之术,本座佩服。”然后她就看到江哲眼中一闪而过的得意,她心中泛起欣慰之色,看来江哲也是一个人,不免会有骄傲的情绪,那么对于她接下来的举措是很有好处的。于是,她越发和气的道:“既然如此,还请江司马送上解药,让我这些弟子早早离去。”

我看了一眼雍王,用目请示,李贽点头道:“随云,将解药交给门主,不过门主还请贵门弟子暂时交出武器,否则本王可是不敢放心的。”

凤仪门主眼中闪过一丝寒芒,道:“这是当然,若是雍王殿下不放心,可以先请皇上暂时避开。”

李贽大喜,道:“既然门主如此大量,那么本王就承情了。长乐,快和两位娘娘陪父皇到偏殿休息。”

长乐公主略一踌躇,看了江哲一眼,道:“长乐遵命。”说罢,向御阶之上走去,她是要去搀扶李援。谁知刚刚走了一半路程,突然一道寒光电射而起,一声娇叱传来道:“李贞,受死。”本来瘫倒在地的李寒幽竟然飞身而起,一剑刺向长乐公主的胸口,这一下却是出人意料,谁会想到中毒到地的李寒幽竟然会暴骑发难。众人的注意力原本都在凤仪门主身上,谁会留心一个中毒的女子,更何况也无人想到李寒幽会在凤仪门主同意妥协的情况下出手。这时,冷川和小顺子都在数丈之外,虽然两人同时惊喝一声猛扑上前,可是却根本来不及阻止。其他护卫雍帝的高手死的死,伤的伤,就是没有受伤的人也没有留心到李寒幽,竟然没有一个来得及救援。而凤仪门方面,韦膺和萧兰、风非非在凤仪门主出现之后就退回去护着凤仪门众人,更是无法阻止,而且长乐公主是让他们热望成灰的罪魁祸首之一,他们更是不会想到救援长乐公主。唯一有能力救援长乐公主的只有凤仪门主,可是凤仪门主刚要出手,只觉得胸中一阵气血翻涌,为了不露出破绽,无奈之下只得作出一片淡然的神色,冷眼相看,此刻她心中在盘算如何不让长乐公主的死亡影响了双方的约定。

\chapter{第三十七章 以血赎愆}

眼看长乐公主就要香消玉陨,这时一个身影竟然奇迹般地挡在了长乐公主身前,霎时间利剑入胸,鲜血四溅。那人撕心裂肺的惨叫一声道:“李寒幽!”

李寒幽在利剑刺入那人胸口之时,原本十分欣喜,可是看清楚那人面容之后,不由目瞪口呆,再听到那人饱含怨毒的叫声,李寒幽慌乱地摇摇头,手中的剑柄仿佛如同烫手的烙铁一样,她松了手就要退去,可是在那双血红的眼睛注视下,她竟然觉得双腿酥软无力,就在这时,那人已经拔出了身上的佩剑,挥剑斩来。若是从前,这人武功剑法不如李寒幽甚远,李寒幽自可以轻松的避开。可是如今李寒幽正是心慌意乱的时候,无论如何,这个人她是万万不能亲手杀死的,所以就在李寒幽神智恍惚的时候,那锋锐的剑芒划过了李寒幽面颊,留下了一道深可见骨的狰狞伤口。李寒幽这才清醒过来,迅速退后几步,免去了头颅被人斩开的命运,可是面上的剧痛和容貌被毁的担忧让她惨叫一声,捂住了面孔坐倒在地。

这个变化让所有人包括凤仪门主都震惊了,突然一人高叫道:“青儿。”正是抚远大将军秦彝,他只觉得脑子轰的一声,一片混乱,眼中只有胸口中剑的爱子。他大步上前就要搀扶秦青,可是有人动作更快,长乐公主悲叫一声道:“青哥哥,你不能死!”已经扶住了秦青,可是她力弱手软,虽然勉强搀扶住了秦青,可是却几乎自己也被带倒,幸亏这时候秦彝已经过来抱住了秦青。两人扶着秦青,让他缓缓躺倒御阶之上。

原来挡住李寒幽那一剑的人正是秦青。秦青和他人不同,自始至终他的目光就停留在李寒幽身上,一时痛恨,一时却又想起从前恩爱之情,所以李寒幽的异常举动只有他留意到了。他早就心存死志,而且他知道李寒幽剑术在自己之上,若是用兵器阻拦恐怕难以成功,所以心一横就用身躯挡在长乐公主前面,凭着一腔死志,他竟然超越了人体的极限速度,成功的用血肉之躯挡住了这死亡之剑。

利剑入胸,秦青郁结在心的仇恨怒火,终于完全爆发出来,所以也顾不上两人武功的差距,就是一剑斩去,这一剑他本没有得手的奢望,可是却成功的毁去了李寒幽的容貌。

李寒幽本是贫家出身,素来不喜欢那些熏人的香料,虽然为了维护皇室郡主的仪态,从来没有表现出来,可是总是尽量离香炉远一些,而这种逍遥香虽然香气清幽,不知怎么李寒幽就是不喜欢这种气味,可是若是不许燃香,李寒幽又担心被人知道丢了面子,所以她就刻意到外边巡视或者作些什么别的,所以虽然她也中了毒,可是毒性却是最轻。暗中服下一些不是很对症的解除迷香之毒解药之后,居然很快就恢复了功力,可是这时候凤仪门主已经到了,正和李贽谈判。她担心凤仪门主抛下了她们,为了有反抗的能力,所以她没有起身。

可是越听,李寒幽心中越是气恼,长乐公主传递密旨在前,下毒在后,害得她心心念念的荣华富贵付诸东流,若是不杀长乐公主,她此恨难消。可是她出言提醒凤仪门主之后,却被凤仪门主置若罔闻,她本是心高气傲的人,一时之间,怒火冲昏了头脑,竟然趁着长乐公主经过之时,出手刺杀,这一剑她是志在必得的,可是却被秦青挡了这一剑。

无论她如何心如铁石,秦青都是她的丈夫,纵然她心中对秦青并无一丝真情,可是名份攸关,亲手杀夫的罪名她是绝对不想承担的,事实上,她原本想等到大事成后,用权势胁迫秦青重新接纳自己,毕竟秦青也算是一个驸马的好人选。就是秦青不识抬举,要杀秦青,也不会是她亲自出手,自然有人动手的。

杀夫的罪恶感和震惊加上混乱的神智,李寒幽居然忘记了躲闪,这才被秦青斩伤了。

这一番变故,使得气氛更加紧张,所有的人都握紧了兵器,混战眼看就要爆发。

凤仪门主这时觉得气血已经平复,冷冷道:“李贽,你是想本座大开杀戒么?”

李贽身子一抖,一字一句,咬牙切齿地道:“谁都不许妄动。”在李贽的命令下,即将爆发的血腥厮杀才被强行压制下来。可是大殿内气氛已经是令人一口大气也不敢喘了。

我怔怔地看着秦青,张开右手,右手已经是一片鲜血淋漓,方才长乐公主遇刺之时,我只能眼睁睁的看着,浑然忘记了一切,清醒过来,才发觉右手的指甲已经将手心刺破了。我奋力站起来,急促地道:“小顺子,扶我过去。”

小顺子面色铁青的走了过来,将我扶到秦青身边,这时候秦青已经昏迷过去,我跪坐在地,伸手放到了秦青的腕脉上,半晌,我抬起头,看见泪水盈盈的长乐公主的眼睛,以及秦彝满怀期望的目光,无奈的轻轻摇头道:“秦将军被这一剑刺伤了心肺,已经无力回天,若是大将军许可,下官可以用金针刺穴之术,让秦将军可以清醒一段时间。”

秦彝只觉得生命仿佛离自己而去,他愣了片刻,道:“拜托大人施针。”

我叹了一口气,接过小顺子递过来的那根玄铁之英的发簪,下了几针,过了一会儿,秦青咳嗽了几声,睁开了眼睛。秦彝颤抖的手抚摸着秦青的脸庞,老泪纵横道:“青儿,都是为父不好,从前忙着征战,没有好好教导你,让你被人欺骗玩弄,如今又——又——”他已经无法再说下去了。

秦青的眼中没有了怨恨,而是一片清明,他平静地道:“父亲,都是孩儿贪恋美色,以至害得皇上和父亲几乎陷入绝境,如今孩儿已知昨日之非,今日以死赎罪,请父亲不要为孩儿难过。”他说话十分清晰,面上更是一片潮红,人人都知道他此刻已是回光返照。秦彝更是悲痛难忍,却不知道该说些什么。

秦青的目光落到长乐公主身上,笑道:“殿下,秦青与殿下本是青梅竹马,可是秦青驽钝,不能够理会公主为国为民牺牲的苦心,反而出言苛责,也怪不得公主对秦青失望。”

长乐公主柔声道:“青哥哥,过去的事情不用说了,你还是本宫从前的青哥哥,长乐虽然怨过你,可是你不念旧恨,救我性命,长乐不知道该如何谢你才是,青哥哥,若有什么未了之事,尽管告诉长乐就是。”

秦青目光有些黯淡,他说道:“殿下,秦青无能失职,贻祸家门,求公主念在家父从来一片忠心的份上,求皇上和雍王殿下不要因为秦青怪罪秦家。”

长乐公主掩面道:“青哥哥放心,本宫一定会向父皇和皇兄求情。”

这时候李援答言道:“秦青,你救了朕的爱女,而且若非你秦家勤王有功,也不能这样快就平定了叛乱,朕对秦家只有奖赏,怎会怪罪,你不用担心。”

秦青的目光又落到雍王李贽身上,李贽正容道:“秦将军,本王在此立誓,绝不会无故加罪于秦家,秦勇将军救了本王性命,老将军一片赤胆忠心,你又救了皇妹,本王心中万分感激,绝不会恩将仇报。”

秦青这才放心下来,伸手握住江哲的手,轻声道:“江兄,我秦青从前瞧你不起,可是今日对你已是心服口服,公主殿下际遇堪怜,你不可负她,不要因为名份礼法踌躇不前。”说到后来,已经是十分低微,除了我,恐怕没有几人能够听到。”

我心中一酸,虽然从来都知道李寒幽的真面目,可是我从来没有想过提醒秦青,我是眼看着秦青一步步越陷越深的,歉疚地道:“秦将军,你放心,我对公主一片真心,绝不会辜负她,只要江哲在生一日,就不会让秦家遭遇劫难。”

秦青听了我低声的许诺,面上露出欣慰的笑容,又看了一眼父亲,道:“父亲,孩儿拜别了。”话音刚落,秦青就已经合上了眼睛,气息渐弱,转瞬之间,已经身赴黄泉。

秦彝悲叫道:“青儿!”那悲痛的叫声混合着李寒幽凄惨的叫声,传得很远很远。

眼中怒火熊熊,李贽冷冷道:“门主,李寒幽在这时杀人,若是放过她,也未免太说不过去了。请门主将李寒幽交给本王处置。”

凤仪门主沉默了一会儿,道:“她杀的是秦青,长乐公主既然无恙,你就不能留难她,不过日后如何追杀,是你们的事情。”

李贽有些犹豫,若是如此放过了李寒幽,也太对不起秦家了,这次秦青虽然犯错在前,可是救驾的也是秦家。这时候,抱着儿子尸体的秦彝突然沉声道:“殿下,不用顾及老臣,陛下安危要紧,先放了李寒幽吧,日后报仇,来日方长。”他的语声充满了沉痛和悲凉。

李贽犹豫的看了一眼江哲,江哲眼中闪过冰冷的寒芒,沉声道:“殿下,请不要辜负大将军的心意。”李贽叹息了一声,不再说话。

凤仪门主遥遥一指点出,李寒幽扑倒在地,已然晕了过去,萧兰过来从小顺子手中接过解药,给凤仪门中人一一服下。不多时,这些人便都可以行动了,不过为了以防万一,他们的兵器都被收去。凤仪门主冷冷道:“你们先到本座事先安排的地点会合,那里本座已经留下了手令,你们照着行事就是了,若是本座不能再执掌凤仪门,由凌羽出任门主之职,韦膺担任门中客卿,纪霞出任执法长老,纪师妹,转告凌羽,你们三人要同心协力,不可互相争权夺势。”

凤仪门众人都肃然行礼道:“谨遵门主谕令。”然后纪霞首先向外走去。

韦膺、凤非非跟在纪霞身后,两个凤仪门女剑手挟着李寒幽跟了上去。萧兰正要跟上,一直瘫倒在地上的李安突然连滚带爬的一把扯住萧兰道:“爱妃,带孤一起走吧。”

萧兰略一犹豫,抬头看向凤仪门主,凤仪门主冷冷摇头,萧兰低下头看向李安,如今的李安更加是全无一丝皇室气度,萧兰心中生出厌恶,足上用力,一脚把李安踢飞,轻轻松松的脱身出来,向殿外走去。李安则顿时痛得鼻涕眼泪一起流下。李贽一皱眉,一挥手,几个侍卫上前将李安拖到一边,免得他再丢人现眼。

这时秦铮已经低头向殿门走去,她不能不走,身为叛逆,她若是不走,只有死路一条,可是她心中却是顾虑重重,因为没有齐王的手令,不可能调动齐王大军发起对雍王大军的攻击,所以她配合同门迫使齐王写了亲笔手令,为了确保万无一失,最后去送手令的乃是她的父亲秦无期。可是现在很显然齐王并非真心相从,否则江哲不会躲在齐王那里,那么那封手令一定是有问题的,恐怕自己的父亲也已经被齐王的手下软禁了,如果自己现在赶到齐王军中,虽然不可能指挥他们挽回大局,可是救出自己的父亲还是很有希望的。夫妻恩情已经薄如白纸,爱子在京城也不可能救出,那么自己便只能指望救出父亲了,这样的时刻,秦铮更不愿意失去这世上仅存的亲人了。

走了几步,秦铮下意识的转头望去,看见那重重刀剑之后,齐王李显负手而立,他神色平和,定定的望着秦铮,他的眼睛里面充满了欣慰和欢欣。秦铮心中一震,知道李显是在高兴她能逃生,想到因为自己的作为,害得齐王今后前途渺茫,再想到在长安齐王府中的娇儿,她停住了脚步。齐王见状,突然侧过脸去,不再看向秦铮,可是秦铮却看见他的身躯在颤抖,他分明是不想自己因为担心丈夫而留下。

秦铮心中一片茫然,想起自幼读过的女则,里面说过出嫁从夫,这原本是她十分不屑的一句话,可是如今她才真的明白这句话的真谛,夫妻之间如果不能同心同德,那么便只有痛苦纷争,想到皇后娘娘和纪贵妃如今的凄惶,想到长孙贵妃和颜贵妃不顾生死挡在李援身前,想到那死于李寒幽剑下的秦青。秦铮终于停住了脚步,她的目光痴痴的落到李显身上,虽然这人带给自己很多苦楚,可是若非自己始终不肯和师门断绝往来,怎会如此,即使在自己给他带来这样的苦难之后,这人也没有和自己划清界限,得夫如此,夫复何憾,这一刻,秦铮真的后悔没有一心一意的侍奉丈夫。

这时,谢晓彤回头叫她道:“师姐,快一些。”凤仪门主也一皱眉,道:“铮儿,你还在犹豫什么?”

秦铮心中拿定了主意,她回身拜倒在地道:“师尊,请恕弟子不能听从你的命令了。”凤仪门主冷冷道:“铮儿,你一向糊涂,为师都不怪你,如今难道你还心存奢望,指望齐王殿下救你性命么?”

秦铮也不理会凤仪门主,高声道:“秦铮身为大雍王妃,不知道忠心为国,反而犯上谋逆;秦铮身为人子,不能劝谏父亲忠义之道,害得父亲为了我这个女儿作出不当之举;秦铮身为人妻,不知恪守妇道,相夫教子,有悖人伦;秦铮身为人母,不知以身作则,善养娇儿,致令孩儿受我连累。父皇,二皇兄,王爷素来忠于朝廷,虽然太子和罪妇百般威逼胁迫,也没能调动王爷一兵一卒,请父皇、二皇兄和诸位将军明鉴。秦铮做下这等不忠不孝,不仁不义之事,有何面目苟活人世,请父皇饶恕了王爷吧。”

李显听到这里,大叫道:“铮儿,你不可做傻事。”就要上来拦阻,可是两人之间隔着很多军士侍卫,李显内力又没有恢复,他只来得及走出几步,只见秦铮手中不知什么时候多了一支金簪,尖锐的发簪指向咽喉,她嫣然一笑,那笑容是如此灿烂,那是嫁给李显之后,再也没有过的美丽笑颜,然后金光一闪,金簪划破咽喉,鲜血迸流,秦铮已经自尽身亡。李显只来得及冲过去将秦铮的娇躯抱在怀里,他慌张地用手去挡住流淌出来的鲜血,可是血如泉涌,却哪里拦得住。他悲声呼道:“铮儿,铮儿,你不能死,都是我对你不起,我不该任由她们主宰你的人生。”可是秦铮却是再也没有气息。李显的目光落到凤仪门主身上,充满了无限的悔恨和愤怒。旁边有人在对他说什么,可是他却听不见,抱起了妻子,再也不看任何人,他踉踉跄跄地向外走去,想要去拦阻的人见到他衣襟上的鲜血和那双充满绝望悲愤的眼睛,都默默退后了。雍王李贽轻轻一叹,一挥手,几个亲信跟了上去。

当齐王的背影消失之后,李贽淡淡地道:“凤仪门主,你是否满意了,我父子兄弟之间被你挑拨离间,以至于此,如今贵门弟子已经离开,请门主暂时到挽秋居暂住,七日之内,本王绝不会派人去追杀贵门弟子,可是门主也要恪守信诺,不得离开秋挽居一步。”

凤仪门主淡淡道:“就是没有本门参与,难道雍王能够放弃皇位么,如今皇位你已是唾手可得,太子谋逆,再无登基为皇的资格,齐王也有嫌疑,从今之后你可以任意将他杀死或者软禁,至于你的父皇,不知道你是否要逼他退位。”

李贽冷冷道:“门主也不用多费心了,这是我皇家之事,若是门主还不放心,最多本王去做门主的人质。”

凤仪门主看了一眼满殿怨恨的目光,心中一阵怅然,慢慢道:“本座有承诺在先,江司马和齐王殿下做人质就可以了,不过我也要说清楚,如果殿下派人追杀我的弟子,那是绝对瞒不过本座的,七日之内,若有一人离开猎宫,本座都不会善罢甘休。”

李贽没有反驳,他的目光落到江哲身上,江哲的目光是那样的冰寒和坚决,那是充满了仇恨和死亡的目光,他坚定的点点头,李贽心中一动,莫非江哲已经有了办法可以达成将凤仪门全部摧毁的目标,因此他淡然道:“本王答应这个条件,门主请。”

\chapter{第三十八章 此恨绵绵}

这一章不知道会不会显得急躁

——————————————————————

我皱着眉头放下一粒黑子,一边拿起那一碗气味扑鼻的黑色汤药,一口气灌了下去,放下药碗,我对凤仪门主笑道:“门主若是不介意,在下颇通医术,愿意为门主诊治一下。”

凤仪门主面纱之上那双清澈明晰的眼睛透出一丝寒光,淡淡道:“不敢有劳,不过是区区七日,本座还能支撑。”一边说,一边放下一颗白子。

我无奈的一笑,想来凤仪门主是不信任我吧,担心我在药物中下毒,若是平常,凤仪门主无伤之时,区区毒药自然伤不了她,可是现在就难说了,凤仪门主当真是小心谨慎。

看了一眼棋盘,我的一条大龙已经被凤仪门主杀得七零八落,真是丢人啊,如果不是我另有目的,我又何必缠着凤仪门主下棋呢,不过凤仪门主大概也是不想我们怀疑她会一走了之,这才答应我的要求,一起在花厅下棋吧,否则不论是养伤,还是练功,都比对着我这个让她大业成空的仇人强得多吧?

又看了一眼棋盘,弃子认输之后,我拿起放在一边的笔,斟酌一番,又重新写了一个药方,递给董缺道:“这个方子我又加了两味药,两个时辰以后送过来,还有,你去看看小顺子是否已经出关,如果出关了就让他过来见我。”如果小顺子过来下棋,可比我强多了。谁让齐王一直闷在房间里面呢,否则何必我抱病陪着凤仪门主呢?

董缺接过药方,恭敬地退了下去。凤仪门主默默的看向珠帘之外,也没有什么兴趣拾拣棋子。中庭梧桐叶黄,西风渐冷,这一个秋季真是萧瑟啊。过了片刻,凤仪门主柳眉轻蹙,她听到了一个人正在缓缓走来,那人的步伐轻缓中带着奇特的韵律,仿佛和周围的环境融于一体,似落叶无声,似水过无痕,这个人的武功已经进入了先天境界,梵惠瑶一声轻叹,记得自己有这样的成就是在三十五岁的时候吧。

过了片刻,小顺子跳起门帘走了进来,三日不见,他的气质又有了改变,如果说从前的他仿若匣剑帷灯,虽然平时隐晦,但是一到关键时刻,例如站在凤仪门主面前的时候,就再也不能掩饰住那种凌人的气势和锋芒。可是如今,他的气质变得温文如玉,多了几分圆润平和,就是对着凤仪门主,也是那样从容闲雅。我虽然不明白这其中的奥妙,可是也猜到多日来的压力逼迫和这几日的苦心潜修,小顺子的武功已经达到了更高的境界。倒了一杯酒,我端着酒杯道:“小顺子,恭喜你百尺竿头,更进一步。”

小顺子上前双手接过酒杯道:“多谢公子,奴才能有寸进,应该多谢梵门主。”言罢,他从从容容地给凤仪门主施了一个礼。凤仪门主眼中闪过一丝遗憾的神色,道:“李少兄武功进境之速,真是令本座敬佩。可惜以李兄之才,竟然屈居僮仆之列,岂不可惜。江大人也未免过于委屈李少兄了。”

我和小顺子都是淡淡一笑,四目相对,他人怎知我们之间的渊源,我们之间又是普通的主仆关系可以形容的,再说,小顺子屈就仆从之列,就可以对他人的招揽推得一干二净,旁人既不能真的将他当成仆人对待,而这个仆从身份又可以让小顺子行事之时无所顾忌,不用顾虑什么身份道义,这才是我们一直主仆相称的最重要的缘故啊。

睁开眼睛,李显觉得宿醉之后的头疼袭来,这几天,他几乎都是醉醺醺的入睡,然后带着头疼醒来的。起来之后,他果然又看到旁边的桌子上放着一碗醒酒汤,他将醒酒汤一口气喝了下去,酸酸涩涩的味道让他不禁皱起了眉头。这几天,他奉命做凤仪门主的人质,倒也用不着做什么,只需要呆在挽秋居就可以了,所以他索性用醇酒麻醉自己。这虽然有秦铮之死带给他的打击的缘故,可是李显明白,那并不是真正的原因,无论如何,李显对秦铮之死是早有准备的。一旦政变失败,皇家容不得一个背叛谋逆的王妃,秦铮的死虽然是她自己所选择的,就是今次她逃了出去,也不过是苟延残喘罢了。令李显如此痛苦的是如今的他所面临的困难处境,雍王还没有说过如何处置他,可是李显明白,最好的结局也不过是收了自己的兵权,让自己作一个闲散的宗室。如果不能再上战场,李显真得不知道该如何度过以后的人生了。

沐浴更衣之后,焕然一新的李显走出房门,既然命运已经如此,那么他也不想让人看自己的笑话。刚走到院子里面,李显就听到花厅之中传出棋子落到棋坪的声音。心中一动,他向花厅走去。挑开珠帘走了进去,一眼就看到,在西窗之下,江哲正在和凤仪门主下棋,不过只看他神色悠然,而他旁边的小顺子神色严肃,捻着棋子苦思冥想,就知道真正下棋的是谁了。在他进来的时候,凤仪门主和小顺子都是头也不抬,只有江哲转过头来,微微一笑,然后江哲站了起来,将小顺子按到椅子上,走了过来,施了一礼道:“殿下,精神可好些了么?”

李显叹了口气道:“你又何必明知故问,对了,这几天外面的事情我都没有理会,父皇可有什么旨意下来么?”

我看了看李显憔悴的面容,道:“据臣所知,皇上已经下旨废黜了太子殿下的储位,太子叛逆之罪要交由三省议处,不过据臣推测,会是圈禁或者赐死。太子东宫臣属均要交部议处,最轻也会削去官职,永不录用。萧妃宗谱除名,所生皇孙贬为庶人。太子妃贬为韩国夫人,太子世子贬为安国郡王,遣去封地,不得圣旨,不得擅离封地,其余妃嫔所生庶子交由韩国夫人抚养,虽然仍然列名宗谱,可是一切封号赏赐都被褫夺。至于殿下的罪责要等到回京之后议处,不过齐王妃虽然自尽,但是罪名仍然难免,皇上已经下旨宗谱除名,齐王妃所生世子不会受到牵连,只是不能继承王爷的王位了。”

李显叹了一口气道:“二哥仁德,也算是手下留情了,你可以转告他,我不会抓着兵权不放的。”

我劝慰道:“殿下,您和雍王殿下不妨好好谈谈,或许会有殿下意想不到的收获也不一定。”

李显苦涩地道:“随云,你不用劝我,我也不会恋栈兵权,想必只要今后我谨慎行事,二哥也不会过于为难我的,对了,鲁敬忠如何处置,二哥对他恐怕是深恶痛绝了吧?”

我淡淡一笑道:“雍王殿下已经派了夏侯沅峰去赐死鲁敬忠了,应该就是现在吧,前两天事情太多,殿下忙不过来。”

这时候,我听见凤仪门主说道:“成王败寇,不过如此罢了,李显,你问这些也没有什么用处,若是想多活几年,还是早些去向雍王表表忠心吧。”

李显没有说话,但是神色间却多了几分讥诮,想必委曲求全,屈膝求饶这样的事情,是这位高傲的王爷一辈子也做不出来的。

玉麟殿的一间偏殿内,鲁敬忠站在窗前,静静的看向窗外,他自知自己资质不高,所以在练武上面从来没有多费心思,所以凤仪门将他软禁之时,他虽然恼怒也没有反抗。反正凤仪门想要控制朝政,没有自己是不可能办到的,太子身边的原有势力除了他鲁敬忠之外是没有人能够理清的,所以李寒幽等人的得意妄为,他从来没有看在眼里,反正夺宫需要的是武力,他也犯不上插手。可是有些事情不是这些心比天高的女人可以办的,不说别的,为了迫使齐王发兵,她们不就不得不将自己从软禁的厢房里面放出来么,虽然还是不许自己走出玉麟殿,但是等到需要和雍帝谈判的时候,她们就不得不让自己出面了,这些事情李寒幽那些人是办不成的。就是韦膺,虽然才具过人,可是要谈到那些微妙的朝政,还是不如自己远甚。

可是雍王成功的扳回了局面,当听到猎宫四面的厮杀声起,鲁敬忠真的心寒如冰,他是很清楚的,谋士不论如何才智过人,对着那些刀枪剑戟都是没有用处的。太子的失败,就意味着自己的失败,覆巢之下,焉有完卵。这几日他被雍王下令软禁在玉麟殿偏殿,也曾想过是否有求生的可能,可惜他虽不是情愿为太子殉死,却没有投靠雍王的进身之阶。雍王身边相辅之才有石彧,文有三杰等谋士,武有长孙、荆迟等大将,更有精通谋略如奇才江哲者,那里有自己的容身之处,更何况自己从前为太子出谋划策,屡次逼得雍王险些遭难,雍王绝对不会生出招纳之心,只怕这几日只是将自己软禁,没有处置,不是忙得忘了,就是不想让自己死的痛快吧。

这时,外面传来一片脚步声,整齐有力,想必是一队训练有素的军士,那些人分立在门侧,然后其中一人推门走了进来。鲁敬忠回头望去,只见夏侯沅峰一身青衣,皎如临风玉树,手中端着一个托盘,上面放着一个翠玉瓶。夏侯沅峰一走进房间,后面的军士就合上了房门。夏侯沅峰将玉瓶放到房中央的桌子上,淡淡道:“鲁大人,下官奉命前来送行。”

鲁敬忠心中一颤,莫名的倦怠从心头涌起,他走到桌前,拿起玉瓶,在手中把玩片刻,道:“夏侯,我月宗弟子互相残杀也是常情,只是我始终不明白你为什么背叛太子,要知道如果不是你传出了令秦勇勤王的密旨,这次雍王必定身死,到时候你的地位只有比现在更高,看在我们乃是叔侄一场的份上,你就说个明白吧。”

夏侯沅峰沉默了片刻,道:“师叔不是知道了么,我中了江司马的毒,所以被迫投降。”

鲁敬忠笑道:“你不要瞒我,你的为人我清楚得很,你是宁可用毒刑逼供求得解药,也不会舍近求远的。”

夏侯沅峰愣了一下,笑道:“师叔果然了解沅峰,那么小侄也就不瞒师叔了,其一么,江司马当时病势沉重,我若严刑迫供,只怕还没迫出解药,他就身死了,而且此人外柔内刚,若是寻常小事,或者可以相迫,若是这等大事,就是以生死相迫也是没有用处的。”

鲁敬忠神色不动,因为他知道这不是夏侯沅峰投降的真正目的。

果然夏侯沅峰又道:“还有一个原因,那就是小侄从来不当自己是月宗的人,月宗的宗旨就是在乱世之中辅佐明君,一统天下,就是同门之间为了争夺宗主之位,得到一窥‘阴符经’真本的机会也是互相残杀,可是我夏侯沅峰胸无大志,什么阴符经在我眼中根本全无分量,辅佐明主一统天下自有别人去做,我只想手掌大权,享受荣华富贵罢了,根本不想成为什么月宗宗主。所以对于我来说,投一个明君才是捷径,太子殿下昏庸无能,若他当了皇帝,不说大雍前途渺茫,就是凤仪门那些女人也比我们更容易控制太子,我夏侯沅峰就是想做佞臣都还怕作不成呢?

雍王殿下就不同了,虽然雍王殿下贤明练达,不免难伺候一些,不能敷衍了事,若是没有真本事,不下死力气办事,终究是逃不过殿下的眼睛的,可是凭我的才能,还怕得不到殿下的赏识么?虽然殿下麾下人才济济,可是君子多,小人少,不论什么明君圣主都是需要我这种小人的,有些事情明君不能做,贤臣不能做,可是我可以做。只要我忠于雍王,定有飞黄腾达的一天。比起那虚无缥缈的阴符经,师叔不觉得侄儿的选择才最实际么?只是投靠也要选时机的,这次我救驾有功,日后必能得到雍王重用,还有什么机会比这次更适合呢?”

鲁敬忠的面色初时一片愤怒,后来渐渐变得失望,最后来却是变得平静,他苦笑道:“原来如此,是我没有看穿你的心意,罢了,罢了,这是你自己的选择,你父亲可知道么?”

夏侯沅峰淡淡一笑道:“知子莫若父,何况父亲从无牵涉叛乱,所以师叔不用为他担心。”

鲁敬忠打开玉瓶的塞子,似乎想起了什么,道:“贤侄既然已经决定跟随雍王,我还要提醒你一句,江哲其人,心思诡谲,布局深远,此人若是有心害你,你是必定会输的,不若趁着如今雍王还没有登基,江哲又重病在身,将他害死,否则你终究得被江哲压着一头,而且为叔早就怀疑雍王手中可能有一支暗处的力量,那力量多半掌握在江哲手中,邪影李顺,人中俊杰,此人多半就是那支力量的领袖,否则实在难以解释以他的武功才智,还要屈居仆从之列的理由。”

夏侯沅峰神色渐冷,道:“师叔果然心狠,临死还要害我,沅峰虽然糊涂,也知道江哲此人只可为友,不可为敌,而且我看此人虽然心机深沉,却不是喜欢劳心劳力的个性,否则也不会担任司马这么长时间,雍王府上的事情却很少过问,石彧一回到长安,立刻重新掌管雍王府政务大权,若是江哲乃是争权之人,岂能如此。而且若是此人真的恋栈权势,当年在南楚,德亲王对他重用之时,凭借此人本事,就不会大隐于朝了。更何况,他若真的如此贪恋权势,雍王也迟早容不得他,何必我和他为难呢?”

鲁敬忠微微苦笑道:“你不信忠言,将来后悔晚矣,罢了,罢了。”话语中充满了惋惜和一丝丝几乎不可察觉的怨恨,鲁敬忠神色泰然地将瓶中毒药一饮而尽。

看着鲁敬忠的尸体,夏侯沅峰伸手替他合上了那圆睁的双眼,淡淡道:“师叔,你何必死前还要挑拨离间,以至于死不瞑目呢?”

七日时光匆匆而过,这一天早上,凤仪门主运气一遍,觉得内力已经恢复了七层,不由大喜,当日她答应留下,就是抱了养好伤势,然后凭着一身武功冲出猎宫的打算,如今虽然没有合适的药物调养,可是七成武功足够她使用了。推开房门,凤仪门主深深的呼吸了一口秋日新鲜的空气,仔细探察一下,她准备第一个杀死江哲,然后就是齐王,之后若有能力,就去看看是否能够杀死雍王,反而是坏她大事颇多的长乐公主,她心中全无杀意,一个女子能够作出那样的事情,凤仪门主心中倒是颇为敬佩,故而因此反而不愿加害。虽然据说太子李安还活着,可是带一个活人太辛苦了,若是即使赶回长安,将萧兰所生的皇孙控制在手中,到时候也未必不能重整旗鼓,控制大雍江山。

可是一探之下,凤仪门主心中一动,那江哲和齐王居然都不在挽秋居之内,凤仪门主柳眉紧锁,再用心探察,只觉周围数里之内居然只有两个人在外面相候,只听那两人的步伐声音,凤仪门主就知道这两人身份。她冷冷道:“慈真大师,邪影李顺,你们不必等了,本座已经在此相候,看来江哲倒是聪明,知道本座乃是用得缓兵之计,不过就凭你们两个,难道就留得住本座么?”院门无风自开,一个灰衣僧人双手合十,眉心一点红痣嫣然欲滴,在他身侧,李顺一身青衣,微微含笑。

凤仪门主冷冷一笑,手握剑柄道:“凭你慈真,本座的手下败将,前几日受得伤这么快就好了么,邪影,你虽然已经晋入先天之境,若是公平决斗,接本座百招还是不成问题的,可是真的生死相搏,凭着本座的剑术和经验,你是必死无疑。

小顺子淡淡一笑道:“门主,在厮杀之前,我要先替我家公子传几句话。”

凤仪门主心中一动,道:“本座倒要听听他的神机妙算。”

小顺子不理会她的讥讽,道:“我家公子说,门主虽然取胜,可是杀人一万,自损三千,慈真大师和门主同列三大宗师,那么门主所受之伤必然惨重,慈真大师不来,或者是已经死在门主剑下,或者是重伤远遁。无论那一种,凭着当时门主的状况,必然会以死相拼,陛下和三位皇子、一位公主和数位军中重臣都在殿中,若是折损过多,只怕大雍难以应对接下来的战争,而且也不是公子愿意接受的。所以公子才用门主也不希望同归于尽的私心和门主达成协议,公子算准了门主会接受七日之约,以为缓兵之计,可是门主却忘记了一件事情,慈真大师不论生死,都不会放任门主贻祸天下,果然,五日之前,少林寺十八罗汉已经到了猎宫,而慈真大师也在两日之前到来,不过公子早就请雍王殿下派军士远远迎接,所以直到今日,他们才来到挽秋居。”

凤仪门主眼中闪过冰冷的寒光,嘲讽道:“人数虽众,可是群狼难抵猛虎,他们人数虽多,也是没有用处的。”

小顺子淡淡一笑,道:“我家公子也知道这一点,他说一千精兵胜过万余乌合之众,所以他立下这七日之约还有别的用意,请问凤仪门主,门主所服的救命丹药可是九转护心丹。”

凤仪门主傲然道:“正是医圣亲制,若无此丹,本座恐怕也不能奔波数百里,赶来猎宫。你家公子如果不是服了此药,只怕早就死在晓霜殿上了。”

小顺子眼中闪过一丝杀机,道:“正是九转护心丹,可是门主似乎忘记了一件事情,就是桑先生曾说,此丹只有在九死一生之时方可服用,而且还要在服丹之后数日多加调养。”

凤仪门主一愣,心中生出不妙之感,当初桑臣果然说过这话,可是自己一来没有想过自己会有需要服丹之日,二来,也自信自己所练内功的神奇,只要保住性命,就可以自疗内伤,没有将这句话看得很重。

小顺子讥诮的一笑道:“门主果然没有将桑先生嘱咐放在心上,桑先生当日将此药托付给我的时候,曾说,九转护心丹乃是使用天材地宝,各种名贵药物炼制,可以激发人体潜能,维系生命,若是内伤发作,心力衰竭,奄奄一息将死之时,服下此药,就可以将全部的精血激发出来,可是有一利就有一弊,潜能激发,虽然可以起死回生,却是十分耗费服用之人的生命之力,所以性命保住之后,就要服用各种大补药物来弥补,桑先生是因为我家公子心伤太重,用平常法子无法治愈,所以才留下此药,等到了万一之时,用此药激发公子潜能,达到破而后立的效果,这个法子虽然十分凶险,可是若是成功,公子虽然不能完全恢复健康,却是可以不用担心会随时丧命了。前些日子,门主可见我家公子每日里几乎以药物为食,就是为了把握良机,医治顽疾。当时,公子曾经提出要替门主疗伤,可惜,门主也如公子所想一般拒绝了。”

凤仪门主声音有些嘶哑地道:“本座岂敢服用江司马的良药,医圣亲传弟子,下毒之术天下无双,本门主还不敢尝试。”

小顺子傲然道:“这也在我家公子意中,当日公子将如何毒倒贵门弟子的手段说出,就是为了让门主生出戒惧之心,所以门主才不敢随便用药,否则就是门主不论对桑先生的话信了几分,也都会宁可信其有,不可信其无,延请名医调治身体的。我家公子这七日之约,就是为了让门主没有机会服药养伤。当然若是门主真的敢用药,我家公子说,他也只好冒险下毒了。”

凤仪门主额上冷汗涔涔,她从未想到,江哲的心机居然到了这种地步,这缓兵之计竟是平白便宜了他。

小顺子又道:“我家公子冒险留在挽秋居七日,每日邀请门主下棋品茗。门主为了迷惑我家公子,造成门主会遵守承诺,自尽谢罪的假相,必然不会拒绝。所以门主也就无暇留意自己的变化,而且内力的恢复,也会让门主再加倍消耗生命的同时,产生一切尽在掌握之中的错觉,不会留意到生命力的衰竭。”

凤仪门主下意识的看看双手,那从前晶莹美丽的素手,果然失去了光泽,她只道是伤势的牵累,想不到竟是生命消失的迹象。

这时候小顺子又补上重重一击道:“公子说门主素来自负,只会防着别人暗算,不会想到时间就是公子最大的本钱,如今慈真大师内力已经恢复五成,而且绝对没有隐患,在下也有一拼之力,而门主如今的内力实际上是您的生命和精血,所以公子相信,我们可以将门主留在此地。原本若是慈真大师不来,公子只少尽出高手和门主周旋,可是慈真大师和少林高僧的到来,让公子手上的人力更加充沛。不过公子说,他不会武功,就不留在这里等死了,现在猎宫中所有重要之人都已经隐藏起来,门主无论如何厉害,也不可能立刻找到他们,公子说,门主远赴黄泉,他就不亲自送行了。”

凤仪门主突然高声大笑,良久,才止住笑声道:“好,好,本座一生转战天下,到头来竟为这样一个文弱书生计算,好,就让本座看看,是否可以多取几条人命。”

慈真大师和小顺子同时上前一步,三人之间的空气仿佛凝固,一阵秋风吹过,漫天黄叶向三人扑去,可是还没有接近三人身旁,就被无形的真气推开了。

此刻,在猎宫一处可以遥遥望见挽秋居的小楼中,江哲和雍王李贽站在窗前,看着挽秋居的方向。这时,突然挽秋居中响起了震耳欲聋的声响,初时是真气激荡如雷的声音,然后是剑气撕破长空的声音,然后是房屋崩塌,飞砂走石的声音,再然后,那声音越来越刺耳,虽然离得很远,可是李贽和江哲的面上都露出一丝被苦痛,那些声音入耳犹如雷鸣,几乎要刺破耳鼓,幸好江哲早有准备,将两团棉花塞到耳中,李贽也照做不误。

过了一段时间,十八条灰色身影飞纵入已经成了废墟的挽秋居,挽秋居方圆百丈之内烟尘滚滚,看不见他们如何交战,可是江哲和李贽站得高远,还是看见了那雪亮如同银虹的剑光。终于,那烟尘中传来一声长笑,那笑声原本应该是悦耳动人,可是如今却充满了愤怒和不舍。然后“蓬”的一声,烟尘之中冒起耀眼的猛烈火光,这一大蓬烈火,冒起之时,势如闪电,所占的面积,几乎有一丈方圆。炎势乃是呈圆柱形,中心之处颜色发青,再外面是白色的火焰,临到最外面,则呈耀眼欲花的红色。

我心中一宽,听这笑声乃是女子所发,其中充满英雄末路的悲哀和壮志成空的怨恨,想来我的计划已经成功了。心神一泄,我坐倒在椅子上,觉得手足发软,成功的逼杀凤仪门主,这大概是我一生中最大的冒险吧。

\chapter{第三十九章 余波未歇}

先发一章,等到第三部完成一起上传其余的章节,因为看到有读者发书评表示等候,所以先发了这一章。

——————————————————

勉强站起身子,我回到窗前看向挽秋居,过了片刻,十几个身影从烟尘中缓缓走了出来,我用尽目力仔细看去,走在最前面的灰衣僧人只看步伐身姿,就知道定是慈真大师,他身后的一行僧人,个个龙行虎步,步履矫健,虽然只有十二人,却是丝毫不显得颓废。半晌,我没有看到小顺子,心中一紧,按在窗框上面的双手不由越抓越紧。又过了片刻,滚滚烟尘终于被秋风散尽,我才看见一个青衣人负手站在废墟之中,一身青衣上鲜血点点,嫣然如桃花,杂布如星罗棋布,在他面前,大火熊熊燃烧,渐渐蔓延到残破的屋舍和周围的草木上。这时候救火的禁军已经过去了。突然青衣人的身形突然消失了踪影,我连忙揉了揉眼睛,他的身影已经在另一处显现,不过瞬息之间,我还没眨上几次眼睛,他已经出现在楼下,这时候,慈真大师和那些少林和尚还在里许之外。

这时,李贽几乎是手舞足蹈地走了过来,兴冲冲地道:“随云,真亏了你,不仅逼杀了凤仪门主,还没有造成不可挽回的损失,本王真是无话可说,无话可说。”

终于放下了心,我转过头笑道:“这都是慈真大师和诸位少林高僧不顾生死,才令凤仪门主伏诛,臣不过是拖了几日时间罢了,而且若非殿下信任臣,当日在晓霜殿上任凭哲自作主张,臣的计策也行不通的。如今凤仪门主已经身死,凤仪门已经再没有什么翻天之力,臣恭喜殿下消除了心中大患。殿下,还请亲自去迎接慈真大师,以表谢意,今后殿下稳定江湖,还要靠少林寺呢,而且对付北汉魔宗也要有慈真大师这样的高手挂帅。”

李贽摩拳擦掌,满心喜悦地道:“随云放心,本王这就去迎接大师,不过,随云,你不去见见大师么?”

我苦笑道:“臣可是快撑不住了,若是殿下体恤,还是让臣好好休息一下吧?”

雍王担忧的看了我一眼,见我不过神色有些疲倦,这才宽心地道:“随云,你可要好好休息,接下来本王还要将凤仪门的党羽一网打尽,继而重整朝纲,其中千头万绪,还要多多仰仗随云呢!”

我微微一笑,没有答话,接下来的事情还多得很,重整朝纲不是那么容易的,皇上尚在,凤仪门虽然已经失去了擎天柱,可是多年来的经营和盘根错节的势力不是那么容易对付的,不过这些我就不用亲自参与了,想来石彧定然是早有准备的,而且锦上添花的人永远是比雪中送炭的人多的。

看着雍王兴冲冲的背影,我却是轻轻一叹,泪水潸然而下,自从我入雍以来,每每徘徊生死,殚精竭虑,呕心沥血,为的不就是今日么,如今雍王继位已经是毋庸置疑的了,太子失去储位,身犯谋逆大醉,想来就是不死也要圈禁终生,为虎作伥的凤仪门已经失去了昔日的光彩,剩下的残兵败将我也早有了对付她们的计划。可以说,我的大仇已经报了,那么这我原本就不留恋的荣华富贵还有什么用处呢,恩仇了了,我也该抽身了。心中泛起一缕柔情,我想起了长乐公主和柔蓝。

这时有人推动房门,我没有回头,会这样自行进入的,除了小顺子不会有别人的。果然身后响起小顺子阴柔却有些嘶哑的声音道:“公子,我幸而不辱使命,凤仪门主已经催动三味真火*身亡。”

我淡淡道:“你身上的伤势可严重么,凤仪门主虽然死了,可是我还有事情需要你去办。”

小顺子笑道:“公子放心,这点伤势不算什么,慈真大师几乎接过了凤仪门主大部分的攻势,所以我只要好好调息一下就可以了,而且我这次和凤仪门主交手收获颇多,受这点伤绝对是值得的。公子要我去办的事情,是不是追杀凤仪门的余孽呢?”

我转身过来,肃然道:“那日晓霜殿我虽然给了解药,可是却也做了一些手脚,那些中毒之人一月之内,身体会散发出一种特殊的气息,只有南疆的一种野鸟可以嗅到,我曾经令人训练了几只这种禽鸟,所以我要你去调动秘营,使用这种禽鸟掌握凤仪门余孽的动向,不要惊动她们,如今她们为了隐秘行踪,使用的一定是轻易不被人所知的密舵,我只要这些密舵的详细情况,不过,有一件事情必须办到,我要李寒幽,这是我答应董缺的事情。”

小顺子担心的看了我一眼,道:“公子,董缺终究不便久留在公子身旁,不知道公子准备对他如何安排。”

我叹了一口气道:“董缺心中的仇恨只有比我更深,父母之仇不共戴天,太子妃的那个侍女,死的时候已经怀了身孕,若非得到雍王在太子身边的密谍传来的情报,我还真不知道这个女子是被谋杀的呢,唉,也是我低估了李寒幽的疯狂,想不到她会对一个小小的侍女这样残忍,你不是曾见董缺夜里祭奠妻儿么,这样的深仇大恨,别说董缺不肯善罢甘休,就是我也不能放过李寒幽,若非是我思虑不周,或者绣春姑娘不会身死,董缺也不会孤苦伶仃,所以我要你将李寒幽带给董缺,随便他如何处置。”

小顺子想了一想道:“只是若想生擒李寒幽,不免会惊动了凤仪门余孽,只怕会坏了公子的大事。”

我微微一笑道:“那些事情不过是我为了雍王殿下将来做了一些打算,成与不成无碍大局,不过若是平白毁坏了那些好用的棋子也未免可惜,这件事情我们不能去做,可是锦绣盟却是可以做的么。而且,若想凤仪门成功的走上我安排的道路,总是要给些蜜饵的,何况她们这些人心中只有利益得失,若是做的妥当不仅不用动手,还可以留一条控制凤仪门的长线。”我见小顺子若有所悟,低声给他讲了如何作法,他一边听一边点头,还不时补充一些看法。

最后我们两人商议已定,才回到住处,一回到那座小宫院,我就看见董缺神思不属的看着远房的天空,便笑道:“董缺,你可是急着想去追杀李寒幽么?”

原本以为董缺会一时失神脱口而出,谁知他却迅速的清醒过来,恭敬地道:“公子当日面许为董缺复仇,必然不会失言,董缺一切仰仗公子。”

我赞许的看了董缺一眼,道:“这件事情,我已经有了安排,不过旬日之间,必然让你见到李寒幽,而且我会尽量给你一个完整无缺的李寒幽,任凭你如何处置,不过此事一了,你也得离开长安了,不知道你有什么打算,如果想为官,我会替你安排,不过你暂时不便留京,若是再过五六年,回来就无妨了,如果不想为官,我会给你一笔金银,足够你作个富家翁了,不知道你有什么打算?”

董缺想了一想道:“小人原本就是一个浪子,就是大仇得报,也没有什么去处,如果公子不弃,小人想跟在公子身边伺候,公子虽然有李爷在身边,虽然公子身边的事情,李爷是断断不能交给别人的,可是外面有些琐碎的事情总不能都让李爷去做,小人自知没有什么大本事,可是总还能作个外务总管的,不知道公子可否收纳。”

我心中一动,说起我身边的人,小顺子傲然不群,又是时刻不离我左右的,所以没有实际的职务,陈稹实际上负责秘营的管理,寒无计掌管天机阁的生意,八骏虽然都是不错的人才,可是一来基本上都已经独当一面,而且我也不想埋没了他们,将来不论在何处有了家园,都是要有一个外务总管负责家居的琐事的,这董缺倒是一个不错的人选,何况虽然知道此人身上有些诡秘之处,可是若论诡秘,只怕我和小顺子都在他之上,这样看来,董缺倒是值得收纳。虽然心许,我却笑道:“可是你也知道,你若是我的外务总管,不免经常见到一些眼利心明的人,你不担心被人识破身份么?”

董缺却是一笑道:“公子不是说五六年以后就无妨了么。”

我一愣,不由笑了,道:“也好,既然你有意相随,也是我们有缘,日后宾主相待,也不枉一场相识。”

董缺又行了一个礼,从前他虽然礼数无缺,却是臣属之礼,如今他行的乃是从仆之礼,我上前将他搀起,虽然不知道他为什么定要留在我身边,不过只要无害于我,我也不想放过这样的得力属下。

突然,小顺子眉梢一动,轻声道:“慈真大师来了。”

我心中有些疑惑,如今慈真大师应该已经去休息了,晚上雍王要宴请各派高手呢,慈真大师怎会突然来此。片刻有侍卫进来禀报道:“大人,慈真大师请见。”

我对小顺子和董缺挥挥手,两人会意,小顺子陪着我亲自出去迎接,董缺则躲到内室,虽然慈真大师从前没有见过董缺,可是凭他的眼力,不难看出董缺易容过,虽然如此大事已定,有些事情还是不能泄漏的。

慈真大师已经换过了衣服,虽然伤势不轻,面色苍白,可是他的神色还是那样平淡。我疾步上前,施礼道:“本当前往多谢大师鼎力相助,可是哲身体羸弱,未能前去,反而劳动大师亲来,还请大师勿怪。”

慈真大师抬眼望去,此时江哲已经将近而立之年,只是面白无须,再加上相貌清秀,虽然一向体弱多病,又是劳心劳力,如今已是两鬓星霜,可是却更加显得飘逸风流,气度更是雍容优雅,一双眼睛仍是深邃幽冷,神光淡然,只是比起上次见面更多了几分神采。无论如何看去,都只会觉得这个青年不过是一个品性高洁的书生罢了,谁会知道此人乃是心思狠毒周密的谋士呢?

慈真大师心中一叹,凤仪门主武功比他略为高强,像他们这种级数的高手,或者可以击败,但是想要杀死就不容易了,即使自己和京无极联手,凤仪门主不敌之下,也可以飘然远遁,可是就是这个文弱书生,通过丝丝入扣的连环毒计,逼得凤仪门主陷入必死之局,终于让那一代巾帼,绝世红粉,葬身在皇家猎宫之中。这已经让慈真大师心中凛凛,方才又从弟子口中得知了许多详情,就是这个青年在危急关头,以身涉险,力挽狂澜,平叛救驾,细察他行事风格,其人用计阴柔诡变,无孔不入,令人心中陡生寒意。

对江哲了解越深,慈真大师就越担忧,昔日凤仪门主也是惊才绝艳,若非一念之差,怎会贻害天下,此人才智更在凤仪门主之上,如今眼看雍王显然就是大雍的下任君主了,此人乃是雍王心腹重臣,更是手中握有重权,若是一念之差,不免生灵涂炭,血流成河。

正因为有着这样的心思,慈真大师才会私下来见江哲,双方见礼入座之后,慈真大师念了一声佛号,道:“江檀越智谋通神,凤仪门主被迫*身亡,老衲虽然略尽绵薄,但若无江檀越的谋划,凤仪门主终究还是会鸿飞冥冥,只是檀越用计过于狠毒,檀越如今身为殿下重臣,身边又有李少兄这样的高手随侍,一念之差,就会有千万无辜受害,今后还请檀越上体天心,与人余地,老衲多言相劝,还请檀越勿怪。”

我心中原本觉得这位高僧未免有些多事,可是见慈真大师看向我的目光十分凝重严肃,便肃然道:“天道轮回,报应不爽,晚生心中时刻铭记,今后若有行止差池,不到之处,还请大师提醒江某。”

慈真大师心中一跳,心道,莫非此人竟然趁机想让我不能独善其身么,若是我时时刻刻关心他的行止,或有劝谏,岂不是欠下了此人的情面,可是仔细看去,只见江哲神色之间一片诚挚,不由想道,罢了,若是此人当真是大奸大恶,终有泄漏的一日,何况雍王殿下圣明烛照,我又何必杞人忧天。慈真大师一想通此事,便不再多说,只是闲话几句,就起身告辞。临行之时,他深深的看了一眼内室,他隐隐约约的觉得室内有人,可是那人呼吸平缓细微,显然是内功精深,而且颇有独到之处,这人隐遁不出,或者有些碍难,无论如何,慈真大师心中终是隐忧重重。

慈真大师走后,小顺子铁青着脸道:“这老和尚竟然敢训斥公子,真是岂有此理,公子可要给他一点教训么?”

我淡淡一笑道:“清者自清,浊者自浊,大师有慈悲心肠,这是他的好处,而且这件事也给我们提了醒,这世间之事哪有终究能够隐秘不泄的,这些年来,我为了复仇,做了许多残忍之事,我虽不后悔,可是难免会有人仇恨于我,只是这次凤仪门之事,就不知道要牵连多少人,招惹世间怨恨,又让众人忌惮,看来我已经渗出险地,这样一来,我们商议好的事情就要快些办了。好了,我还要想想如何安排,你就不要过问了,还是去办李寒幽的事情吧,这件事情不了结,我总是放心不下。”

小顺子默默听着,神色渐渐和缓下来,道:“公子说得是,这老和尚虽然无礼,可是他送给公子的心法也颇有些用处,这几日公子练了,果然身子有些好转,只为这件事情,我就不会与他为难。”

十月四日,圣驾回銮,我坐在随军的马车里面,神色悠闲,雍帝回銮之后,就要掀起狂风巨浪,这也是无法避免之事,即使李援想敷衍了事,雍王殿下也断不会同意。虽然这次救驾的是秦家,按理说大局应该还在李援控制之下,可是有些微妙的原因却让这种理所当然的情势出了变化。首先,秦青之死虽然是李寒幽所为,可是如果不是当初李援的指婚,也不会有今日,秦勇虽然救了圣驾,可是人人都知道传出密诏的乃是雍王的属下,这样一来,雍王既有拨乱反正的大功,又是当之无愧的储君人选,再加上他素来的声威,已经显然盖过了李援的权威,这件事情又是雍王冒的风险最多,所以这之后的处置是万万不能绕过雍王的。不过雍王对京中事务早有安排,这倒不用我操心了。

早在猎宫救驾之前,雍王就派了心腹侍卫到京中送信给石彧,石彧在得到消息之后周密安排,将敬重大臣全部监控起来,虽然负责京师军政的韦观和郑瑕都不是寻常人,可是雍王多年的经营岂是寻常,再加上这几年雍王广为布间,早就暗中控制了大半中低级官员,虽然不能控制朝政,可是这种监控却是轻而易举,再说石彧本就是在长安经营多年,所以猎宫和长安之间的消息传递被石彧封锁的滴水不漏,猎宫那面生死相见,长安却是一片平静。凤仪门众弟子脱身之后,不是没有想过传递消息,可是她们不敢回长安送死,所以派来的都是些普通的弟子信使,都被石彧擒的擒,杀的杀。

等到凤仪门主身死之后,雍王派了人回京向石彧说明情况,石彧更是不敢掉以轻心,而且凤仪门在朝中多有同党,韦观更是满朝门生故旧,若是在皇上和雍王回京之前出了变故,恐怕大雍社稷的根基都会动摇。所以石彧果断的去找侍中郑瑕,郑瑕一向是刚正不阿,虽然韦观资历官职都在其上,可是郑瑕却是雍帝的主心骨。郑瑕在看到皇上的密令和雍王的手书之后,又仔细查问之后,才相信了石彧所说。他行事十分果断,立刻和石彧联手将韦观软禁在府中,然后轻而易举的控制了长安的局势,有郑瑕出面,朝中文臣都是凛然遵命,而那些武将虽然分属不同派系,但是有郑瑕和石彧出面,就意味着皇上和雍王的令旨,谁敢违抗,齐王的麾下,一来是处于劣势,另外齐王也从没有下达什么命令,所以他们都默许了一切的发生,所有人都在等待雍帝回銮之后的大变,山雨欲来啊。

第四十章    恩深怨消

大雍武威二十五年十月九日,帝以太子谋逆不孝,下旨赐死,以王爵之礼葬之,未许入皇陵,谥“戾”。

——《雍史•戾王列传》

十月五日,雍帝在路上的时候,长安已经平定下来,由于郑瑕和石彧商量之后,都决定继续隐瞒消息,所以长安之内虽然人心惶惶,可是却仍然不知道猎宫发生的大变。十月六日,郑瑕带着几个侍卫先赶来见驾,就在郑瑕和雍帝密谈之时,早已经得到报告的我胸有成竹,虽然不知道他们谈些什么,不过想来郑瑕不是糊涂之人吧。

再说郑瑕进了雍帝的寝帐,见到雍帝安然无恙,这才放下心来,行过大礼之后,李援连忙将事情经过说了一遍,他对郑瑕信任非常,将自己所知全部详详细细的告诉了郑瑕。郑瑕听过之后也是瞠目结舌,可是他素来善于决断,镇定下来问道:“陛下,您可有什么打算?”

李援苦恼地道:“朕也是十分头疼,太子和雍王都是朕的儿子,朕自然不希望他们手足相残。可是雍王这次险些丧命,朕也险些遇害,若是不严加追究,无论如何都说不过去;可是太子有今日,朕也有不当之处,而且皇后曾经自缢,虽然被宫人救下,可是已经奄奄一息,多年夫妻,朕实在不忍心;还有齐王,这个孩子素重情义,这是他的长处,也是他的短处,如今他牵连其中,不论如何处置,只能说轻了重了,却断不能说处置错了,他的性子又是那样执拗,朕担心雍王一怒之下,要求将他圈禁或者废为庶人,这样岂不是令朕为难;还有韦相,听你说他在京中安之如素,看来真是不知道谋反的事情,可是谋逆大罪,如果不株连,也实在不象话,郑卿,你为朕想想,这该如何是好?”

郑瑕神色肃然道:“陛下,如今以臣看来,这些事情怎样处置都不重要,重要的是陛下如何和雍王父子相安。”

李援心一震,他毕竟做了多年的皇帝,这些心思他也隐隐约约想过,可是郑瑕说得如此直白,他还是有些措手不及,不由怒视郑瑕。

郑瑕毫不畏惧地道:“陛下待臣恩重如山,若非是为了陛下和大雍的江山社稷,臣也不会说这些非礼之言,若是陛下肯听臣详述,就是杀了臣,臣也甘之如饴。”

李援犹豫了一下,道:“郑卿说吧,朕知道你的忠心的。”

郑瑕凛然道:“陛下,如今雍王继承大统已经是大势所趋,太子谋反,理应废黜,雍王功高盖世,又是年纪最长,这次无论皇上如何打算,这储位已经是雍王囊中之物了。从前皇上为了维护太子,对雍王殿下多有打压,雍王心中难免没有怨恨。如今就是雍王想趁机夺了皇位,也没有几个人会坚决反对,对臣等而言,效忠雍王殿下和效忠陛下,已经没有什么区别,可是这样一来,皇上的地位就十分尴尬了。如果陛下亲自处置太子等人,难免会有什么地方惹雍王不满,若是雍王心中怀恨,就是现在陛下保住了太子和齐王,等到陛下万岁之后,谁知道日后雍王会如何做呢?若是将这件事情交给雍王处置,那么陛下再婉言表示一下自己的意见,雍王必然不会不顾念陛下的心情,到时候陛下既可以达到心愿,也可以和雍王殿下父子之间隔阂尽消。”

李援低头想了半天,起身向着郑瑕施了一礼,郑瑕大惊,连忙避开道:“陛下这是做什么,臣担当不起。”

李援欣慰地道:“郑卿良言苦口,都是为了我李氏着想,若是日后朕和雍王父子相安,太子和齐王能够得到保全,都是卿的功劳。”

郑瑕连忙连连谢罪,李援笑道:“朕和郑卿君臣多年,也不用如此俗套,何况朕虽然看错了一些人,可是却没有看错郑卿,朕知道卿直言相谏,都是为了朕着想。不过有些事情还得你替朕拿主意,你说接下来朕该怎么办呢?”

郑瑕道:“陛下,您是否定要保住太子呢?”

李援有些犹豫地道:“太子虽然不肖,可是毕竟是朕的骨血,朕实在有些舍不得。”

郑瑕又问道:“那么齐王殿下呢?”

李援正色道:“显儿虽然有些过于重视情义,不足为皇,可是朕实在很爱惜这个儿子,朕是万万不能让贽儿伤害他的。”

郑瑕正色道:“既然如此,陛下就不应该庇护太子,否则就是害了齐王?”

李援惊讶地道:“这怎么说呢?”

郑瑕道:“陛下,齐王若论文治武功不如雍王,若论嫡庶长幼,也不如雍王,所以如果没有太子的存在,那么齐王可以为将,也可以为臣,可是若是太子尚在,那么无论如何,太子终究是嫡长子,齐王和太子联手就有谋反的可能,所以若是皇上庇护太子,雍王殿下若是勉强答应,就终究会疑心齐王,到时候有心人从中离间,迟早齐王都会因此死在雍王手里。到时候,陛下想要保全两个儿子,却是一个都保不住。若是舍弃了太子,那么齐王殿下就不可能危及雍王的皇位,到时候就容易君臣相安了。”

李援沉默半晌道:“郑卿说的是,既然如此,朕也顾不得那个逆子了。”

郑瑕又道:“这还是从私情上来讲,若是从国法来说,太子逼宫谋反,又引诱皇后殿下失德,这是无父无君的不孝之罪,追杀手足兄弟, 这是不悌之罪,不孝不悌,如何能够饶恕。陛下的基业是要流传千秋万世的,若不为后世留一个警惕,人人效法这等行径,岂不是要让天家骨肉自相残杀么?”

李援听到这里,悚然动容道:“郑卿此言,真是天下至理,好,朕决心已下,赐死太子,皇后本应赐死,念在多年夫妻恩情,废为庶人,就让她自生自灭吧。齐王的事情,我就交给雍王处置吧。”

郑瑕肃然道:“皇上圣明,这样一来,既可警惕后世,也可以让雍王心服口服,而且齐王的事情,雍王也就不好过分处置了。”

李援心中清明,继续道:“太子家眷的处置已经决定了,以后就作为规矩吧。还有一件事情,回京之后,我要晋封长孙氏为后,郑卿意下如何?”

郑瑕先是一愣,立刻醒悟过来,道:“陛下圣明,正该如此。”君臣相视而笑,彼此心照不宣。

郑瑕心中明白,立长孙贵妃为后的确是一个好主意,现在很明显的,李援还要在皇位上坐一段时间,后宫不可无主,而且将来雍王继位之后,也要有一位母后来孝顺的,如今窦氏被废黜,雍王生母又早已亡故,纪贵妃身为叛逆,那么只有长孙贵妃和颜贵妃有资格晋升皇后,可是齐王也牵涉到叛乱中,颜贵妃自然也失去了立后的资格,而长孙贵妃身份尊贵,长乐公主这次又立下大功,身为长乐公主的生母,那么长孙贵妃封后就是顺理成章的事情了。而且最妙的是,长孙贵妃没有皇子存活,不会影响到雍王的储位,所以正可以母仪天下。李援能够想到这一点,看来已经是为雍王登基铺路了,而且对雍王再无忌惮了。作为臣子,郑瑕自然是心中欣然,不过这种事情只可意会,不可言传的,君臣二人自然只有相视而笑了。

过了片刻,雍帝有些犹豫地道:“郑瑕,长乐公主钟情江哲的事情,你看怎么办呢?”

郑瑕谨慎地问道:“不知道皇上和雍王的意思如何?”

李援不满地道:“贽儿曾经私下来见朕,希望朕为长乐公主和江哲赐婚,可是朕看那江哲心机深沉,体弱多病,实在不是长乐的良配,所以已经拒绝了,可是江哲立下这样大功,朕如果执意不许,未免有些冷了他的心。”

郑瑕想了一想道:“这件事情,臣看怎样都无所谓,一方面,江哲曾是南楚臣子,公主曾为南楚王后,陛下拒绝赐婚,也是符合礼法的,另一方面,如今江哲乃是大雍臣子,又立下平叛大功,公主乃是陛下爱女,身份尊贵,这功臣尚主,也无可厚非,只看陛下的意思了。”

李援想了一想道:“若是那江哲身子好一些,朕就成全了长乐也无不可,可是现在朕实在不放心,先放一放吧。”

郑瑕见夜已经深了,李援也有些神色疲倦,就道:“陛下,事情已经商量妥当,不如陛下先就寝吧。”

李援笑道:“朕已经想通了,以后什么军政大事都交给雍王吧,朕要好好过上几年舒心的日子,卿先别走,替朕拟旨之后,再去休息吧。”

十月七日,李援回京,连下三道旨意,其一是赐死太子,加谥号戾王,皇后废为庶人。其二是立雍王为监国太子,一切军政大事悉由雍王决断。其三就是立长孙贵妃为后,则日正式举行立后大典,另外以长乐公主传诏有功,赏赐食邑万户,加封号宁国,敕建宁国长乐公主府赐给公主。

皇上的雷厉风行震惊了不少人,朝野或者以为是雍王趁机挟持了皇上,或者以为李援是受了惊吓,无心再理会朝政,却不知道这件事情的最大功臣乃是侍中郑瑕。

雍王主管朝政之后,开始了后来被称为“戾王大逆案”大肆清洗,以牵涉太子谋反之罪被下狱的达官显贵数以万计,被牵连的人更是数不胜数,一时之间朝野惊恐不安,只有少数有心人才会发现雍王的清洗实际上控制的很好,被牵连的朝臣多半是出身世家豪强,这些世家在大雍崛起的时候虽然立下了功劳,如今却是争霸一方,兼并土地,甚至私养甲兵,隐隐有割据之实。这次雍王借着谋逆大案,运用手上的军队,将这些世家豪强几乎全部摧毁。他的手法刚柔兼备,对于那些世家的中坚分子经常是当作叛逆剿灭或者下狱,毕竟这些世家都不免和凤仪门、韦观有些关联,而对于世家旁系的子弟和那些依附世家生存的平民却是不会轻易加罪,托从前锦绣盟和凤仪门的福,这些豪门世家很多本就早已经被杀得支离破碎了,再借着大逆案的名义,让各大世家凛如寒蝉,不敢出头,更是方便雍王各个击破,一月之间,大雍朝堂已经焕然一新,石彧带来的幽州官员和那些真正肯做事的中低级官员很快就让大雍的中枢恢复了正常的运转,鲜血洗清了大雍朝堂上的蒙尘。

而在这其中,有一种官员是被最先清洗的,那就是家中妻女和凤仪门有关联的官员,这些官员最轻的惩罚也是贬斥降级,稍微严重一点的就是免官去职,甚至直接上法场也是可能的。很多凤仪门弟子原本都是千金小姐,入凤仪门倒有大半是为了提高身份,所以多半都是立刻和凤仪门划清界限,这样的女子若是能够得到父兄和夫家的庇佑,倒还是可以安然度日,虽然不乏有抛妻弃女的事情发生,但是总算大半还能重新做人。可是若是那种贫寒人家出生,因为进入凤仪门而得以嫁给朝中显贵或者豪门世家子弟的女子,命运就要凄惨的多了,不是被夫家休离就是被打入冷宫。可是在屠刀霍霍的时候,这些女子的凄苦哀怨也被血腥的清洗掩盖住了。

雍王也并非总是这样辣手无情的,有些官员从前党附太子或者出身韦观门下,只要没有明显的谋反证据,自身再有不错的才能,那么也不会被清洗,而在雍王的清洗过程中最不会受到牵连的就是军方。雍王下了诏令,军方将士为国血战,都有汗马功劳,所以不许在军队进行清洗,就是发现了有些将领和凤仪门确实关系密切,只要肯写一份详细的悔过书,就可以得到赦免。所以雍王的铁血清洗,不仅没有危及大雍的根基,反而加强了军队的实力,因为很多世家子弟和江湖中人都通过从军来避免被牵连到大逆案中去,危机过后,大雍的军方力量倒是更加强大了。

十月九日,郑瑕带着鸩酒、白绫和一把短剑到了太子被囚禁的锦安殿,这是太子第二次被软禁在此,上一次,李安虽然也是担惊受怕,可是既有韦膺暗中照应,又有凤仪门和鲁敬忠等人在外奔走,总算是心中有底,这一次李安却是再无倚靠,缩在殿中,茶饭不进,已经是只剩一口气了。

郑瑕正要进去,突然看见远处一行人走来,只看他们的灯笼就知道是雍王府的人,走近之后,郑瑕一眼就看到了为首之人正是江哲,他身后侍立之人正是邪影李顺,而周围更是侍卫环立,守备森严。

江哲上前深施一礼道:“下官奉雍王殿下之命,前来为太子送行,请侍中大人允许。”

郑瑕一皱眉道:“这有违礼数,可有皇上的旨意?”

江哲眼中闪过一丝炽热的杀气,低声道:“侍中大人,下官不妨直言,我这次前来雍王殿下并不知道,是我使用了殿下的金牌,骗过了禁军进来的,这一次我是定要见到太子,如果侍中大人不允许,那么江哲只有硬闯了。”

郑瑕听得一愣,他仔细看去,只见江哲眉宇之间竟是宁为玉碎的神情,郑瑕虽然恪守礼法,可却不是固执不化之人,心想此人辅佐雍王,对太子步步进逼,莫非竟然是因为他和太子之间有些仇怨么,此人心思深沉狠毒,若是我执意不许,他怀恨在心,必然生出大祸,若是加害于我也就罢了,万一此人故意挑拨皇上和雍王的父子之情,那可就是我的罪过了。想到这里,他说道:“既然是雍王殿下的命令,本官也可以从权,江司马就和本官一起进去吧。”

江哲露出一丝狂喜,挥手让侍卫们留在外边,只带了小顺子跟着郑瑕进去,郑瑕身边原本带着两个勇武有力的太监,原本是为了防止太子不肯自尽,让他们动手帮忙的,如今看这样情势,为了不让这两个太监见到不该见到的事情,郑瑕挥手让他们留在外面。

三人进了锦安殿,看到瑟缩在床榻之上的李安,郑瑕不由轻轻叹息,江哲却是面寒如冰。

郑瑕宣旨之后,小顺子端着方才接过来的托盘走了过来,上前摆着鸩酒、白绫和短剑。李安只是一边惨叫一边后退,果然是不肯自杀。

走到近前,我低声道:“太子殿下,请问殿下可记得南楚的柳飘香么?”

李安眼中一片迷茫,过了很久才道:“记得,孤曾经临幸过她,不是早就让梁婉送回去了么?江大人,求你跟二弟求求情,只要饶了孤的性命,孤情愿终生圈禁,或者出家为僧。”

我胸中一阵血气翻涌,想不到当日梁婉还是骗了我,原来害死飘香的真正凶手竟然就是她自己,而这个李安虽然是罪魁祸首,却不是杀人凶手,不过我却仍然越想越恨,若不是他荒淫,若不是梁婉为了保护他的身份秘密,飘香怎会被害。想到这里,我转头看了小顺子一眼,道:“太子殿下不肯上路,你就帮帮他的忙吧。”

小顺子看了郑瑕一眼,随手拿起鸩酒,上前执住李安,轻轻松松的将鸩酒给他灌了下去。李安很快就断了气,面色一片青紫,带着不甘心和悔恨,却不知他在悔恨些什么。

我只觉得心中一片空落落的,大仇得报,我反而有些茫然了,这时候郑侍中意味深长地道:“江大人,往事已矣,来者可追,你可要把持得住。”

我看了郑瑕一眼,上前施礼道:“郑大人放心,哲虽然有些私心,可是却从来没有挑唆过雍王殿下不顾兄弟之情,只是如今太子恶贯满盈,哲若是不能前来看着仇人上路,实在是不能甘心。”

郑瑕虽然只听见片言只语,却也能猜出几分真相,可是他知道如今木已成舟,自己也无需多事,只要警告这个青年不要为了私仇有害大局一下也就罢了。

三人正要离去,突然外面传来嘈杂的人声,走到殿外,只见雍王匆匆而来,看到郑瑕和江哲之后,雍王神色一宽,道:“郑大人,本王派江司马前来为太子送行,也是略尽兄弟之情罢了,还请郑大人不要见怪。”

郑瑕不由有些好笑,但也不揭穿,只是道:“这也是人情,臣怎会怪责,陛下正在等臣回报,殿下请便。”

等到郑瑕走后,雍王过来狠狠的瞪了江哲一眼,道:“你真是胆大包天,竟敢假冒我的谕令,回去再和你算帐。”然后又低声道:“随云,你既有这样的心事,为什么不和本王明言,你这人真是,唉。”

我心中一片温暖,连忙侧过头去,免得被人看见将要溢出的泪水,也低声道:“臣不敢以私心害公义,殿下对臣的爱护,臣感激涕零,以后万万不敢再瞒着殿下了。”

雍王叹了一口气道:“走吧,若非是夏侯见到你深夜进宫,本王还不知道你如此妄为呢,幸好郑大人没有怪罪你。”

我又施了一礼表示歉意,这才跟着雍王殿下出宫了。一路之上,我心中满是感激之情,雍王殿下的大恩,我终究是报答不完啊。

——————————————

请等待第四十一章 春梦无痕

\chapter{第四十一章 春梦无痕}

在雍王忙着清洗的时候,朝野上下人心惶惶的时候,却有一支神秘的力量没有停止行动,十月十二日晚上,在一处僻静的乡下农庄里面,一些黑影悄悄的掩向农庄,再将农庄包围之后,一个黑衣蒙面人低声吩咐了几句,另外一个面目阴冷的中年人带着两个少年走向农庄大门,高声道:“有远客来访,主人还不出来迎接么?”

农庄的门轻轻开了,一男一女走了出来,那个男子看他的面目赫然竟是逃出猎宫之后踪影全无的韦膺,他虽然改了农夫装扮,可是仍然掩饰不住他的气度风华,而那个女子也是一身村姑装束,但是看相貌却是秀丽清雅,气度如同月中仙姬一般绝俗飘逸。韦膺神色阴冷地道:“你们是什么人,怎么会找上这里?”

中年人平和地道:“你们可真是难找啊,我们跟踪了你们数日,才终于将你们围在这里。”

韦膺一皱眉,这些日子以来他们早就发觉有人窥伺,可是他们不敢公然发难,这才想尽力避开那些神秘人的监视,可是没有想到他们还是找上门了,他们是谁,若是雍王的人,只怕早就出动大军来捉拿他们了。一边想着,他一边问道:“阁下应该知道,你们能够跟踪我们,不过是仗着我们不敢声张,可是这里是穷乡僻壤,若是我们反戈一击,你们可就得不偿失了,还是快些说出来意的好。”

那个中年人眉一挑道:“虽然阁下等人武功高强,可是也不见得胜过强弓硬弩,至于我们的身份,也不算什么荣耀的门派,我们是锦绣盟中人,在下姓霍,现在担任锦绣盟护法一职,我身边这两位乃是我家盟主的心腹弟子,这一位你可能听说过,他叫霍离。”他说到强弓硬弩的时候,韦膺和那女子都听见弩机的轻响,从声音判断,至少已经有三十多把硬弩将农庄前面包围住了,虽然农庄后面没有弩弓的声响,可是却能够隐隐听见呼吸之声,看来来人果然是有备而来,自己一方纵然能够胜出,也会惊动外人,得不偿失。

那个女子黛眉一蹙,她仔细看去,那个中年人虽然相貌平平,可是神情气度却是不凡,而他身边两个少年都是人中俊杰,那个叫霍离的少年气质沉稳,相貌俊朗,而另一个少年也是相貌清雅,眉宇间带着几分淡淡的促侠气息。这个霍离他自然听说了,这个少年凭着一己之力,在洛阳掀起了滔天巨浪,那另外一个少年和他身份仿佛,看来这锦绣盟似乎是人才济济。可是她记得曾听师父说过,锦绣盟可能和雍王有些秘密的关联。所以这女子突然道:“早听说贵盟和雍王达成盟约,怎么今日是奉命来捉我们的么?”

那个中年人冷冷一笑道:“我们锦绣盟不敢说和雍王没有打过交道,可是盟约还谈不上,当初我们和太子殿下联手走私,可惜李安过河拆桥,还要为难我们霍盟主,所以我们才将情报透露给了雍王,虽然没有能够把李安的储君位子废了,可是也让他多了些麻烦,这世上只有我门对不起人,可没有人可以对不起我们。不过我们可不是雍王的附庸,我们锦绣盟和什么人都可以合作,可是只有一件事,我们不会忘记,我们锦绣盟是为了反抗大雍而建立的,凡是能够让大雍头疼的事情,我们都会去做。所以贵门这次失手惨败,已经和大雍成了生死之敌,我家盟主派在下带了礼物过来,送给诸位。”

说着他一挥手,从黑暗中闪身出来一个黑衣少年,神色冰冷,他手上端着一个锦盒,将锦盒呈上给那中年人。那中年人将锦盒打开。韦膺和那个女子一眼看去都是一惊,之间里面乃是一叠厚厚的银票,而且都是南楚最富盛名的金陵钱庄的银票。

中年人淡淡道:“这里是二十万两银票,我家盟主说,如今你们败给雍王,必定要和大雍为难,可是若是在大雍境内,你们就是势力再大也不能和军方对抗,所以只有远走高飞,北汉是魔宗的地盘,你们是去不成的,想来化外之地也不是你们的目标,那么只有南楚才是你们东山再起的好去处。可是你们这次惨败,只怕缺少盘缠,我们知道贵门虽然日进斗金,可是消耗也大,如今贵门的生意也大都留在大雍,恐怕也没有法子继续掌握,所以特让本护法带了这些银票来,希望你们能够在南楚重整旗鼓,盟主说,只要是大雍的敌人,都是我们的盟友,凌仙子,你可愿和我们结盟。”

那个女子正是凤仪门主指定的下任门主,凌羽,她看向银票,冷冷道:“你们虽然舌灿莲花,可是本仙子有些不明白的地方,就为了一个共同的敌人,你们就舍得二十万两银子么。”

那个中年人诡秘的一笑,道:“我们盟主从来不作赔本的事情,若是你们肯答应我们一个条件,不仅二十万两银子是你们的,我们还会将在南楚的一部分产业让渡给你们。”

韦膺和凌羽都是神色一动,二十万两银子会坐吃山空,可是产业却可以维持凤仪门的开销。可是这个条件会是什么呢?韦膺走上近前,道:“阁下不妨说说条件,如果我们觉得合理,也未必不可。”

中年人笑道:“说句实话,凤仪门已经身败名裂,你们在明处的产业自然会被大雍朝廷充公,可是你们还有一些产业却是暗处的,如今你们不便控制,不如给了本盟,双方利益交换,谁也不吃亏。”

中年人见凌羽和韦膺都有些意动,又拿出一个锦盒,打开之后,里面是一些契约文书,他接着说道:“这里面有南楚十四处产业的契约文书,总值四十万两。你们若肯拿相当的产业来换,那么我们之间的盟约就已经定下,我们锦绣盟在南楚是寸步难行,因为过去盟主青年气盛,不免在南楚肆虐太过,可是想要颠覆大雍,南楚却是不得不重视的力量,只要你们尽快的帮助南楚强大起来,到时候不仅你们可以报仇雪恨,我们也可以得偿夙愿。”

凌羽和韦膺两人交换了一个眼色,韦膺上前接过第二个锦盒,将其中的文书查验之后,对凌羽轻轻点头,凌羽神色一喜,道:“本门确实有一些暗地里的生意,虽然不值四十万两,可是也值三十万两,不过这样一来,你们可是大大受了损失,我可不信你们情愿吃亏,若是有什么其他要求,不妨明言,只要不大过分,我们都可以商量。”

中年人眼睛一亮,道:“其实我们也是无可奈何,现在南楚的那些生意虽然不错,可是在南楚只要涉及到锦绣盟三字,那就是破家之祸,所以这些产业虽然丰厚,对我们却没有什么更大的帮助,反而是在大雍,因为大雍的朝廷对我们锦绣盟并非是深恶痛绝,所以我们大有可为,这样交换,对我们没有什么太大的损失。不过若是仙子和韦大人同意,我们确实有一个小小的要求,这是本盟一位客卿的私人要求,他想要贵门……”说道最后,中年人放低了声音,只有近在咫尺的韦膺可以听见。

韦膺一皱眉,走回凌羽身边,低声说了一句,凌羽下意识的就要拒绝,可是韦膺又低声说了几句话。凌羽神色有些犹豫,过了片刻,她默默转身回去。韦膺微微一笑,对中年人说问道:“这个要求似乎有些古怪,她一个人,值得三十万两银子么?”

中年人低声道:“韦公子,说句实话,这是本盟客卿和她的私人恩怨,本盟这位客卿立下了天大的功劳,这是他唯一的要求,我们盟主也同意了,其实我们损失也不大,那些金银也都是些不义之财,本盟最希望的是,和贵门结为盟友,将来你们在南楚,我们在大雍,联手对付大雍朝廷,为了这个目标,这些金银算什么。至于我们要得这个人么,不过是个额外的要求罢了。说句不客气的话,从前她是宗室,身份尊贵,自然对贵门十分重要,可是如今她只是一个容貌尽毁的废人,若论武功,你们比她强的人多得是,若论才智,你们也用不着她,等到到了南楚,她唯一有用的大雍宗室身份恐怕是只有害处,没有益处,她对你们已经是全无价值了,而本盟却可以用她的性命,换来一位客卿的忠心,这可是好买卖,不过要说此人么,别说三十万两,就是一两银子也不值得。可是若能够换来贵门的合作,别说是三十万两,就是再多三十万两,也是值得的。”

韦膺叹息道:“贵盟有你这样的人才,怪不得从前凤仪门总是奈何你们不得,这些日子,我们消息闭塞,不知道情况如何,你可有什么消息么?”

中年人眼珠一转,道:“韦公子是担心令尊吧,公子放心,听说雍王对令尊还是手下留情的,只是将令尊暂时软禁起来,不过令尊如今心灰意冷,几次求死不成,如今已是卧病在床。”

韦膺叹了一口气道:“都是我害了父亲,不知道贵盟可否帮个忙,让家父不要这样痛苦。”

中年人眼中一寒,他已经听出了韦膺的意思,这种情况下,想要救出韦观是不可能的,韦观乃是丞相,天下皆知,又没有什么绝世的武功,想要逃过追缉是不可能的,韦膺这个要求竟然是想让锦绣盟杀了自己的父亲。

韦膺见他神色大变,低声道:“这不是我心狠,家父对大雍朝廷是忠心耿耿,所谓知子莫若父,将来不论我做些什么,只要没有了亲情的遮蔽,家父都会一眼看穿,这对我实在不利,而且家父一片忠心,若是自尽身亡,朝廷念在往昔家父的功劳,必然不会牵连族人,这也是韦膺一边苦心,还请阁下成全。”

中年人犹豫了片刻道:“这件事情在下还要禀明盟主,若是可行,盟主就会下手,若是不可行,我们也暂时无法和贵门取得联系,只要令尊没有死,公子就会知道这件事情的结果了。”

韦膺满意的点点头道:“还有一件事情,凤仪门主身死猎宫之事,虽然有些风声,可是却不知是真是假,贵盟可有消息。”

中年人道:“这件事情我们盟主亲自出马查探,应该有七成可能是真的,因为少林寺的十八罗汉去了一趟猎宫,只有十二人回来,慈真大师一回来就闭关养伤,恐怕凤仪门主身死乃是真的,不过大雍朝廷却不愿宣扬。”

韦膺道:“那是当然,北汉魔宗宗主和门主曾有誓约,若是门主身死,京无极就再不受誓约约束,所以朝廷讳莫如深也是可以理解的,若是贵盟将此事宣扬出去,北汉魔宗必定蠢蠢欲动,到时候岂不是有利于我们。”

中年人皱了一下眉道:“这件事情事关重大,在下不能决定,不过若是这样一来,魔宗入侵,不免影响我们的势力,所以我们盟主只怕不会同意的。”

韦膺笑道:“天下没有不透风的墙,这件事情迟早会传扬出去的,若是贵盟策划的好,当可以趁机谋取利益。”

中年人有些意动,却没有说话,韦膺知道点到即止才是上策,便没有继续劝说。

不多时,一个青衣妇人从农庄走出,虽然只看相貌也知道那妇人绝不年轻了,可是相貌却仍然是艳丽华贵。她身后跟着两个剑手,两人用担架抬着一个昏迷过去的女子,那个女子的脸上包着厚厚的白布条,看不到相貌。

中年人眼中闪过一丝喜色,他转身打了一个手势,从黑暗中闪身出来一个黑衣人,他的相貌全部隐藏在面纱之后,走到担架前面,毫不怜惜的掀开那受伤女子衣衫,仔细验看了那女子腰间一颗红痣,然后点头退下。只见他身法诡秘,内力深厚,就知道此人身份定然不凡。中年人满意的一挥手,他身边两个少年接过担架,将那女子抬了下去。

中年人将两个锦盒递给韦膺,道:“盟约既成,这些东西还请笑纳,不过我们最好留些联络方式,等你们在南楚立稳脚跟,我们也好交换情报。总有一天,大雍内忧外患,会有覆亡的一天的。”

那青衣妇人眼中闪过一丝杀气,道:“这日子不会太久的,这次大雍内乱,北汉肯定会趁火打劫,等我们控制了南楚朝局,两面夹攻,一定会让大雍君臣寝食难安的。”

中年人大喜道:“若是如此,我们锦绣盟一定会趁机发动民变,我们里应外合,管叫大雍亡国。”

双方又谈了一些联络的暗号,那中年人心满意足的离去了,凤仪门众人都可以隐隐看见黑暗中不知多少黑衣人互相掩护着退走,看见他们手中的弩弓,所有人都倒吸了一口冷气,如果刚才大打出手,那么只怕自己这些人早就死伤惨重了,敢在大雍神出鬼没的锦绣盟果然非同反响啊。

这时,神色憔悴的萧兰从农庄中走出,走到青衣妇人身边道:“师叔,那人虽然看不见相貌,可是我看他举止,有几分像一个人,可是那人早已死去,所以我不敢肯定。”

青衣妇人,从前的纪贵妃道:“没关系,你说说看,我相信你的眼力。”

萧兰郑重地道:“那人像极了太子身边的侍卫夏金逸,不过他早就死在淳嫔一事之上了。”

纪霞想了片刻,拊掌道:“说不定就是此人,想不到锦绣盟如此狠毒,怪不得他们想要李寒幽,李寒幽的真正身世我听门主说过,这就对了,看来锦绣盟和我们果然是真心合作,好了,准备一下,我们即刻离开,早日出了大雍地界,我们才能安全无虞。”

众人都是齐声答应,她们对李寒幽的真正身世都不大清楚,但是纪霞既然这样说,那就是十拿九稳的了,也就不忙着追问,只要锦绣盟确实真心合作,那么至少不会立刻被大雍朝廷发现她们的行踪,这才是最重要的。

当李寒幽被冷水泼醒的时候,她立刻下意识的想去那身边的佩剑,可是却是摸了个空,她睁开眼睛,惊觉自己竟然是躺在冰冷的地上,而在自己面前,一个黑衣人背对着自己负手而立,在他身边,两个少年正在看着自己,其中一人手上拿着一个空盆,显然是他泼醒了自己。

李寒幽努力回想,只想起自己临睡之前,乃是喝了纪霞亲自送过来的伤药,然后就不省人事,怒火燃烧而起,她冷冷道:“可是她们出卖了我?”

那个黑衣人冷冷道:“正是,我们用二十万两银票和四十万两的产业和贵门交换,贵门付出的代价就是三十万两银子的产业和你。”

李寒幽心中寒冷如冰,这些日子以来,她的精神早已接近崩溃,日夜的逃亡,加上面伤,和失去权势的打击,早就让她万分痛苦,如今凤仪门将她抛弃,她更是意冷心灰,被出卖背叛的怨恨虽然仍然焚烧着她的心灵,可是却再也没有活下去的勇气。她有气无力地道:“好,好,你们杀了我吧,反正我李寒幽也已经是没有什么活路了。”

那个黑衣人转过身来,微笑道:“不,我不会杀你,那对你太仁慈了。”

李寒幽只听见刺耳地惊叫声响起在耳边,她下意识地想去捂住耳朵,可是接着她就发现,这高声尖叫的就是自己。她颤抖着指向那黑衣人道:“夏金逸,你还活着,你怎么没有死?”

董缺微微一笑,他特意去掉了易容,还刻意做了和从前一样的装束,所以李寒幽一眼将他认出,毫不稀奇。他开口道:“不错,我应该早就死了,可是我不甘心,所以又从黄泉之下回来了,乔翠云,你当日害死我的父母,杀了绣春和我没有出世的孩儿,可想到会有今日么?”

李寒幽慢慢的向后缩去,心中充满了恐慌,那唯一可以充做门面的宗室身份,在眼前这个男子面前却是一文不值,她下意识地狡辩道:“我不是乔翠云,我是李寒幽,靖江王的爱女,我——”

董缺开始大笑,笑声中充满了讽刺和仇恨,半晌,他才说道:“你放心,我不杀你,那太便宜你了,山鸡也想冒充凤凰,天下没有这样的好事,乔翠云,你太蠢了。你可知我要如何处置你么?”

李寒幽心中一冷,若是这人要杀自己,她并不害怕,可是他说不杀自己,李寒幽却是从心底生出寒意,她自然知道对于一个女子,最为惨痛的事情是什么。她突然一掌拍向自己的天灵,想要自尽,可是谁知手掌一抬起,却是无力的垂落,她才惊恐地发现,自己的内力已经消失的无影无踪,然后耳边传来了董缺的笑声。

董缺一字一句地道:“乔翠云,你放心,我知道你在想什么,可是如今你容貌被毁,就是我想将你卖入青楼,只怕也没有人愿意买你。不过你可知道,有些深山老林中的人家,因为外面的女子不肯嫁入山中,所以经常三四十岁还没有妻子,你虽然相貌毁了,可是你的身体还是足够让他们满足的,我已经为你选了一户人家,那是一对兄弟,他们已经快四十岁了,可是还娶不到妻子,所以他们情愿用多年积攒下来的金银买一个女子作他们的妻子,只要能够生儿育女,对他们说来就已经心满意足了。我已经派人告诉他们,我手上有一个女子,因为不守妇道,被夫家休了,还被毁了容貌,可是她的身材可是十分动人,而且身体健康,就是生上十个八个孩子也没有什么问题,我想低价卖给他们。他们已经表示很愿意接收你。”

李寒幽面上露出恐怖的神色,董缺继续道:“不过为了保护他们的生命安全,我不能让你完好无缺的嫁给他们,所以我已经废去了你的武功,这样一来,你就无法反抗他们,而且内力消散之后,也可以让你顺利的怀孕生子。不过你知道的东西很多,想要害死两个猎户也是轻而易举的,所以我准备金针将你的手筋脚筋挑断五分,这样一来,你虽然还可以勉强行走,也能够拿起一些轻巧的东西,可是因为我会告诉他们,你曾经意图杀夫,所以他们会严密的防着你,你绝不会再有杀夫的机会的。不过还有一件事情,你学会了那么多东西,若是教给你的子女,也是后患无穷,所以我会点残你的哑穴,不能说话,在人人都不认字的深山中,你还有什么法子教他们呢?反正对于那对兄弟来说,只是想要一个女人罢了,他们不会介意你是个面容丑陋的残废的,而且,说句真心话,你的身子足够他们享受的了。”

李寒幽开始崩溃,她仿佛看到了地狱的火焰,她一边喊叫一边后退,想要避开董缺,可是董缺不理会她,反而继续道:“我不担心你会疯狂,女人的忍耐力是很强的,而且那对兄弟也不会虐待你,对于他们来说,你是值得珍惜的财产,虽然他们身强力壮,不免会索取无度,可是凭着你练过武功的身体,是绝对可以承受的,好了他们已经等得很急了。我这就动手,你不要害怕。”

董缺上前按住李寒幽的娇躯,盯着她的眼睛道:“你在深山中苦熬岁月的时候,不妨想想从前的荣华富贵,虽然对你来说只是一场梦而已,梦醒之后,你不是什么宗室郡主,更不是什么公主殿下,甚至也不是什么名门女侠,可惜梦终究是梦,一场春梦了无痕,你不过还是乔翠云罢了,只是没有了爱护你的公婆和丈夫罢了。”

李寒幽,不,乔翠云,银牙一咬,就要咬舌自尽,可是董缺已经制住了她的穴道,低声道:“你想咬舌自尽,没有那么容易,本来我是想拔去你的牙齿的,可是那也未免太难看了,所以我特意学了一种刺穴的方法,可以让你两颊的肌肉无法强行用力,这样一来,你就是想要咬舌自尽,也不能达到目的,最多是流些血罢了,我不信你有勇气可以多次尝试,而且你的两个丈夫会有一个总是陪着你,你别想自尽成功,而且你的死志若是过于坚决,为了不想损失这样珍贵的财产,连我都想不出他们会做出什么,是会将你堵着嘴捆绑起来,还是别的什么?”

李寒幽再也忍受不住,头一歪,昏迷了过去,这一次,董缺没有强迫她清醒,因为他知道若是再这样下去,只怕李寒幽会疯狂的,可是只要让她昏迷下去,等她醒来之后,就不会因此疯狂了,这也是人自我保护的方式。他看着昏迷的李寒幽,眼中充满了炽热的火焰,道:“乔翠云,当你再次醒来的时候,应该已经身在深山了,你练过武功,意志坚强,那会让你不会轻易疯狂,清醒的承受你的报应,还有什么比这个惩罚更合适呢?”

站在旁边的两个少年对视一眼,眼中满是惊恐的神情,他们都知道董缺和李寒幽只见的恩怨,可是董缺这样的报复方式,还是让他们心中有些忌惮,不过却也不会阻止就是,李寒幽曾经刺杀过公子,这件事情,他们早就心知肚明,想到公子当日九死一生的情形,无论李寒幽遭遇怎样的惩罚,他们都不会心软的。

当董缺走出密室的时候,看到陈稹正在等候自己,董缺上前施礼道:“多谢陈总管相助,董缺感激不尽。”

陈稹微微一笑,递给他一张绵纸,道:“上面是我们这次跟着凤仪门的行踪查出来的暗舵,这是公子要得情报,烦你呈上去,还有,请禀告公子,一切已经安排就绪,只要公子一声令下,就可以行动了。”

董缺拿过那张纸,道:“我回去之后立刻禀告公子,陈总管放心就是。那些凤仪门的秘密产业,还请转告寒总管,让他快些接收,也免得这次天机阁损失太大。”陈稹笑道:“寒兄早已经去办了,他的性子可是急得很呢?”两人相视一笑,拱手告辞了。

————————————

请等待最后一章,明天发表

\chapter{第四十二章 清风明月}

我拿着董缺呈上来的绵纸细细查看,一边看,一边将其中的一部分记录在另外一张纸上。董缺已经将经过跟我说了一遍,虽然董缺的报复手段有些残忍,可是比较起来,我的手段怕是更加狠毒的,所以我也没有责怪董缺,不说他和李寒幽之间仇深似海,我一向认为每一个人都应该为自己所做的事情负责,而且我也认为与其等老天去报应,不如自己动手,否则我有何必定要和一国太子为难呢?

等到我将可以交由雍王处置的凤仪门密舵整理出来之后,便让人去请雍王殿下,在雍王没有来之前的短暂空隙,我对小顺子说道:“你说,韦大人的事情,应该怎样处置?”

小顺子想了一想道:“我看韦膺如此心狠手辣,就是留下韦大人也没有什么用处,而且韦大人父子情深,怕也不能尽心尽力地对付韦膺,不如就杀了韦大人,也好让锦绣盟得到凤仪门的信任,不知道公子以为如何。”

我想了一想道:“韦观虽然没有参与谋反,可是他身为文官之首,治家不严,理该惩处,而且我想韦膺之事他也不是完全不知道,只不过没有想到韦膺会参与谋反罢了,对于太子继位,他还是乐观其成的。韦膺倒是聪明,若是韦观活着,那么自然是要对韦膺大义灭亲的,韦观若是死了,其父的学生故旧,很多人不免对韦膺会有些香火之情的,不过韦膺也太自作聪明了,所谓人走茶凉,那点香火之情无碍大局,顶多是这次凤仪门退出大雍的时候有点用处罢了。而且雍王殿下刀锋所指,谁敢徇私情呢?这样吧,让韦观自尽好了,也不用多事,只要让监视他的侍卫放松一些,再说上几句风言风语,还怕韦观不能自杀成功么?”

这时,小顺子突然使了一个眼色,我知道是雍王殿下到了,便也不在多说,起身出去迎接。远处,雍王在石彧和夏侯沅峰的陪伴下走来,只见雍王殿下神色,就知道他心情定然很好。我上前施了一礼道:“劳烦殿下前来,臣之死罪。”一边请罪,我一边看向夏侯沅峰,什么时候雍王对他这样信赖了?

雍王也看到江哲犹疑的目光,他也有些懊恼,后悔自己不该带着夏侯沅峰前来,可是此人这些日子以来倒是十分得力,在此人相助下,对宫中朝中太子势力的清剿进行的十分顺利,而且最难得的是,此人十分贴心,前两日,江哲私下入宫,若非夏侯沅峰传来消息,他也来不及去打圆场,所以近来,他渐渐将夏侯沅峰列入了心腹之中,为了这一点,石彧等人都有不满之心,难道江哲也是因此不满啊。雍王尴尬的笑了一笑道:“随云,这几日你养病养的如何,本王可还有要事和你商议呢?”

我请雍王等人落座之后,将那张整理过后的单子呈上给雍王道:“殿下,这里是臣查出来的凤仪门的密舵,请殿下把握时机将这些密舵控制住,不过最好不要立刻动手,免得引起凤仪门对属下的暗探的怀疑。”

雍王接过单子看了半晌,叹息道:“随云,你手下的密谍好像比父皇和本王手上的更厉害呢,这些密舵本王只知道十之三四,还是这几天才发觉的。”

我听出了雍王的言下之意,竟然是想打我手下的密谍的主意,可是天机阁和锦绣盟现在都不适合交给雍王,毕竟若是被人察觉出来雍王和这两个组织的关系,那么就没有用处了,为了打消雍王的念头,也为了岔开话题,我坐了下来,轻摇折扇道:“孙子兵法有云,用间有五:有因间,有内间,有反间,有死间,有生间。若论用间之学,殿下本是十分擅长的,曾听董先生言道,殿下用兵行军,每到一处必定召来当地土人,亲自问讯,可谓善用因间。当日大雍在南楚安插了梁婉,可谓死间,通过梁婉,大雍广为收买威慑南楚官员,可谓内间。殿下于初创近卫军的时候,就在军中设置斥候营,专司负责侦察军情敌情,可谓生间。至于反间,殿下昔日在蜀中不就是用了反间之计,才迫得德亲王急攻雒城的么?”

李贽有些尴尬地道:“本王用间的本事怎比得上随云呢?”他看了夏侯一眼,有些事情还是不要让他知道的好,便没有说下去,事实上他对江哲用间的本事佩服的五体投地,若非是江哲的安排,太子怎会失德如此,若非是江哲的安排,如何能够调动夏侯沅峰这些人为自己所用,才成功的逆转了局势。自古名将不过是擅长使用指挥自己的力量,而江哲却是擅长运用敌人的力量为自己做事,这种神乎其神的用间之术李贽自然是学不到的。

我笑道:“殿下用间的缺憾之处,就是只知针对敌人,所以殿下对太子身边的事情查的很清楚,可是对中立的韦大人、秦大将军那边的事情就不甚了了,所以才会在猎宫之变中失了先机。不说别的,殿下如今也该知道,臣有些私事一直没有禀告殿下,可是殿下一直没有多疑,虽然这是殿下用人不疑的好处。可是今后殿下就要成为大雍的君主,这天下的人才都会来投靠殿下,难道殿下个个都要用人不疑,疑人不用么?所以臣建议殿下在禁中另设一司,在朝野广设耳目,专司监察百官臣民,才能保证君权稳固,社稷长安。”

石彧皱眉道:“这样一来,岂不是使朝野上下人心惶惶,而且若是这样一来,掌握监察之权的人不免权力过大。”一边说,他一边用隐晦的怀疑目光看着我,显然是怀疑我想掌握这个机构。

我淡淡一笑,道:“这个就要看殿下如何安排了,只要殿下将监察之权和处置之权分开,这个机构就不会权倾天下,而至于会不会人心惶惶,道目以路,就要看殿下怎样行事,只要殿下不以监察所得情报擅定人罪,那么又怎会人心惶惶呢,只要无关大局,或者并非叛逆行为,殿下看了那些情报也不过是一笑了之,可是事关重大,那么就可以未雨绸缪了。”

李贽听得很认真,眼中不时闪过摄人的光彩,等到我说完之后,他开口道:“本王也早有意在禁中设立监察司,随云可愿掌管之。”

我微微一笑,道:“殿下,哲虽然颇擅用间,可是这等事情需要一个细心人去做,臣素来粗枝大叶,怎能担任这样的重担,而且臣近来大病初愈,也想好好调养身体,这等劳心劳力之事,臣恐怕做不来的。”

石彧和夏侯沅峰眼中都闪过一丝惊诧,他们原本以为江哲是想自己掌控监察之权,不料他却推辞了。石彧心中有些愧疚,心道,我本就该知道,江哲乃是品性高洁之人,从来没有争权夺利的心思。夏侯沅峰却是目放炽热的光芒,这监察司简直就是为他设立的,他自信可以胜任这种黑暗中的职务,而且,这个职务必然是官职低微,权利极大,若是旁人担任,不免会让雍王生出大权旁落的忧虑,这恐怕也是江哲坚决推辞的原因吧,可是自己本是太子一方的人,如今雍王手下控制军政大权的属下基本上都对自己心存戒备,若是自己担任这个职务,雍王自然可以放心,因为自己只有雍王一个靠山罢了,为了维护雍王的统治,自己必然是殚精竭虑,不敢轻忽,也不敢生出背叛之心。

这时,李贽也想到了这一点,他忍不住向夏侯沅峰看去,夏侯沅峰反应很快,立刻流露出赤胆忠心的神色,李贽轻轻点头,没有说话。

这番互动,我都看在眼里,不由心中一喜,其实即使我不说,雍王迟早也会想到建立一个对内监察的机构,我主动提出来,又不肯担任这个机构的负责人,雍王必然对我更加信任,而雍王也定然会想到夏侯沅峰是一个好人选,因为他只有忠于雍王才有荣华富贵可言。而夏侯沅峰对我来说,也是一个好人选,他虽然反复无常,心思阴毒,可是也是一个识时务的人,他知道我的厉害,除非是雍王对我生出了杀意,他是绝对不会来和我为难的。

过了片刻,雍王醒悟过来,道:“随云,再过几天,就是本王的立储大典,你乃是首功,可要来观礼啊。”

我自然是欣然答允,又问道:“殿下,您立储之后就可以正式监国了,您可有什么安排么?”

雍王道:“本王已经禀明父皇,原中书令韦观因为涉嫌谋逆,已经不能担任丞相之职,父皇想要侍中郑瑕升任中书令,本王已经同意,父皇也同意子攸担任尚书右仆射。”

我拊掌道:“殿下果然圣明,子攸先生虽然是相辅之才,可是若是现在进中书省,毕竟资历还浅,而且现在朝野上下人心不稳,郑侍中德高望重,接掌中书令就可以镇住局势。而尚书左仆射也是相辅之一,而且现在的尚书令本是一个懦弱之人,子攸担任尚书左仆射就可以在尚书省放手而为,尚书省直接管辖六部,殿下正可以趁机重整六部,等到过几年,子攸先生就可以进中书省了,不过这样一来,侍中一职由何人担任呢,这个职务需得一个严刚敢谏的人担任。”

李贽笑道:“随云果然明白其中深意,郑侍中出掌中书令,正是众望所归,新任侍中,本王已经有了打算,已经决定由魏国公程殊担任。”

我愣了一下,道:“魏国公?”脑子里泛起魏国公程殊那种总是有些神态慵懒的模样。

石彧笑道:“正是,魏国公虽然平日有些玩世不恭,可是为人却是忠直的,他作侍中,虽然是以武官转任文职,可是一来殿下也不想他老人家再上战场,另外也好让皇上放心。”

我想了一想,果然魏国公果然是最适合的人选,想当初,凤仪门权势熏天的时候,此老也是敢仗义执言的一人,而且他和皇上关系密切,也是一个很好的中间人,可以避免郑瑕和石彧只见发生冲突。

雍王说得兴起,又道:“另外,大将军已经决定辞去官职,本王已经任命秦勇将军担任禁军统领,这样一来,父皇和本王都可以安心了。”

我微微一笑,看来郑瑕、秦彝和程殊这些纯臣才是真正的常青树啊。

这时候李贽突然看了我一眼,有些不安地道:“不过有一件事情,倒是本王愧对你了,本王曾向父皇请求为你和长乐赐婚,可是父皇担忧你体弱多病,不肯许婚,不过你放心,只要过一两年,你身体好转,本王一定会再次请求父皇赐婚的。而且你也不用担心,父皇已经下旨加封长乐为宁国长乐公主,又为她建造府邸,看来父皇是不会逼着长乐另外嫁人的了,你们年纪还轻,再等一两年,定可以琴瑟和谐的。”

我心中暗笑,早知道雍王会寻时间说出这件事情,好劝慰我不可灰心,于是我作出怅然若失的神情,接下来的谈论中,我似乎神思不属,说话开始有些混乱,最后雍王只好告辞离去。等雍王走后,我立刻找来小顺子,对他说,计划可以开始了。

为了迅速地稳定局势,策立太子的大典是在十月二十五日举行的,当日,雍王司马江哲因为受了风寒卧病在床,没有能够参与大典。所以当一辆普普通通的马车出城的时候,没有任何人想到我就在马车里面。早就换上了普通的青衫,我一边玩弄着手上的折扇,一边想着是否会留下什么破绽。在雍王登上储位的时候离开是我早就做下的决定,一来是恩仇了了,留在雍王殿下身边已经没有什么用处了,军政人才雍王殿下身边多得很,另外么,就是为了殿下着想,我在猎宫之变中锋芒毕露,不仅让慈真大师这些人心生忌惮,就连郑瑕、秦彝等人也不免心生寒意,我若继续留在雍王身边,他们必定时刻担心雍王殿下用了我“阴毒”的计策,与其让他们因此怀疑雍王光明磊落的用心,我还不如离开的好,雍王这样的身份,没必要留下一个宠信阴毒诡谋之士的阴影。所以我早就决定离开了,而雍帝不肯许婚,也是我离开的动力之一。

所以我便趁着雍王府上下忙着雍王的册立大典的时候,先让小顺子接来柔蓝,然后趁着守备松懈的时候,在陈稹、寒无计等人的接应下离开了长安,一路之上,我已经安排了重重假相,绝对可以顺利的消失在人海之中。

轻轻的抚摸着柔蓝熟睡的小脸,我叹了一口气,唯一的遗憾就是和公主有缘无份,如今的宁国长乐公主身份尊贵,怎忍心让她和我流浪四海呢,何况可能我是不会再返回大雍的了。

出了明德门,我想起当日身系缧绁,被雍王俘虏之后,押进长安的景象,不由微微一笑,虽然只有两年,可是长安在我心中留下的印象却是这样深刻,想来,我不论到了哪里,都会想着长安的风光吧。不知走了多久,突然,小顺子的声音飘进来道:“公子,有贵人来送行了!”

我一愣,我离开长安,除了我的属下之外,是无人知道的,怎会有人相送,掀开车帘,我的目光立刻凝固了,就在前面路边的长亭内,一个素衣女子立在长亭之中,云鬓高耸,身披翠绿色的大氅,她身边站着一个三十多岁的秀雅女子,还有一个十几岁年纪的清秀少年。我惊呼出声,那三人竟是长乐公主,还有她身边的周尚仪和小六子。这是怎么回事?我连忙在小顺子的搀扶下跳下车来,走进长亭之内,急迫地问道:“殿下,你怎会前来相送?”

长乐公主幽怨地道:“若非小顺子相告,你是否就要从此远去,也不顾本宫一片深情。”

我尴尬地道:“殿下,从此以后,哲就是一个草民了,公主却是身份尊贵,不说其他,只是这宁国两字的封号,就可以让公主一世荣耀了,我——”

长乐公主伸出纤手,捂住了我的嘴,嫣然道:“本宫不管什么荣华富贵,本宫只知道对于大雍已经再无亏欠,父皇母后身体康健,而且皇兄也会恪尽孝道,若是你不嫌弃,我情愿随你离去,从此平淡度日,做一对民间的恩爱夫妻。”

我再也压抑不住心中的狂喜,我不是没有想过可以带着长乐远走高飞,可是皇上的加封却让我退却了。宁国长乐公主的封号,并不是随便得到的,凡是皇室的女儿都可能得到公主的封号,可是这封号都是只有两个字的,而宁国两字封号是因为长乐公主立下平叛大功才加封的,历代以来得到这样的封号的公主不过寥寥数人。所以我放弃了,没有想到长乐公主情愿抛弃这样的殊荣,随我远走高飞。

上前一步,握住长乐公主的素手,我说道:“殿下,承蒙你青睐,哲感激不尽,虽然哲不过一介草民,定会让公主得到幸福的。”

长乐公主玉颜之上一片嫣红,她低声道:“我若是不相信你,又何必恬颜相从呢?不过你也不要叫我公主殿下了,我名李贞,以后你就叫我贞儿吧。”

我心中只觉得柔情万缕,低声道:“贞儿,我定然不会负你。”

长乐公主想到多年来的苦恋,终有今日,不由眼圈一红,扑进了我的怀中,我一手紧紧的抱住长乐公主的娇躯,眼睛却是感激地向小顺子看去,若不是他自作主张,只怕我真的要孤身终老了。小顺子微微一笑,没有说话。

我搀扶着长乐公主上了马车,周尚仪和小六子则上了后面的马车,他们都对长乐公主忠心耿耿,而且也不想因为丢了公主受责罚,所以就同行了。

马车再次出发了,我握着长乐公主的纤手,只觉得满心欢喜,老天,终究是眷顾我的,让我在失去飘香之后,得到这样的知心爱侣。至于皇上和雍王会有什么反应,我早就顾不上了,反正我也不大算再回去那钩心斗角的官场了。

正式册立为太子的雍王回到王府,很快就得到了江哲失踪的消息,匆匆赶到寒园,只见所有的侍卫都被迷药制住,园内所有雍王赏赐的玩赏之物都被封存起来,一介不取,所有文书信件都列出目录,有的注明收藏何处,有的注明已经焚毁。而在书案之上,留有江哲的一封书信。李贽打开之后,只见上面写着一首七绝小诗。

“腰佩黄金已退藏,个中消息也寻常。世人欲识寒园客,只是江南读书郎。”

李贽叹了一口气,坐倒在椅子上,道:“难道孤还是不能让你心服口服么?”

这时候夏侯沅峰开口道:“殿下,有一个消息或许会让殿下开心的?”

李贽扬眉表示疑惑,夏侯沅峰含笑道:“刚才臣得到江大人出走的消息,就让人去探听了一下,好像长乐公主今天出宫去了,而且公主只带了周尚仪和一个小太监随行和一些侍卫,可是这些负责保护的侍卫已经回宫请罪了,因为他们被人制住了了,好不容易才脱身回来报告的。公主殿下也失踪了。”

李贽眼睛一亮,道:“你是说长乐跟随云私奔了。”

夏侯沅峰恭敬地道:“臣不敢妄断,不过殿下,若是公主一直没有消息,应该就是跟着江大人一起走了。”

李贽大笑道:“好,好,长乐总算是有魄力,只要随云成了孤的妹婿,孤就放心了,迟早他会回来的。”皱了皱眉,又道:“不过父皇那边恐怕会发怒的,我得快进宫劝解一下。”

这时候石彧匆匆忙忙的走了进来,道:“殿下,边关军报,龙庭飞率军出明水关,攻入镇州,军情紧急。”

李贽剑眉一挑,道:“果然来了,立刻传旨,本王要亲自迎战。”

石彧断然道:“殿下,这不行,从前您是带兵的亲王,自然可以领兵作战,如今你是国之储君,又负有监国之责,如今国内局势还未平定,殿下必须在京中掌控大局,否则就是因小失大,而且殿下也不能再以身涉险了,殿下的身份已经不同了。”

李贽眉头紧锁,身份的变化让他有些不适应,一时之间陷入了苦恼之中,除了自己还有谁能领兵作战呢,大雍多得是将才,可是要选一个能够抵挡龙庭飞的帅才,就不是那么容易了。看到旁边桌案上江哲的留书,他苦笑道:“随云,你怎么在这个时候离孤而去呢?”

这时候,夏侯沅峰突然道:“殿下,信后面好像还有字。”

李贽一愣,上前拿起信笺,果然背面还有一行小字,写道“北汉必然趁机兴兵犯境,可为帅者,唯有齐王李显,殿下诚心相请,齐王殿下必定俯首听命。”

李贽拿着书信,愣了半晌,神色变化万千,良久没有说话,这时候,一个侍卫奔来道:“殿下,诸位大人已经在大殿等候殿下前去议事了。”

李贽清醒过来,微微一笑,道:“孤这就去了,传孤的谕令,这寒园从此以后封闭起来,不许任何人擅入,园中的仆人都留下来,好好打理这里的一切,不可懈怠。”说罢,李贽一甩袍袖,向外走去,还有大事等待他去处理啊。

这时候,已经是深秋时分,明月在天,清风满园,李贽走在寒园之中,心中却满是一派豪情,北汉,南楚,等着吧,我大雍铁骑很快就会来了。

——————————————————

第三部至此告一段落,江哲也归隐江湖了,结尾或许有些仓促,可是我已经尽力了,毕竟我不大擅长写感情戏,希望大家能够满意,不过这本书并不会就此终结。接下来我会休息两周时间,然后开始更新第四部北汉烽烟,希望大家在我停笔期间,不要吝于发表书评,不妨提出一些建议,我会常常上来加精看书评,如果觉得那位读者的建议合理,我会采纳的,纸短话长,我也不知道该说些什么,只能是多谢读者的支持和爱护了。

——随波逐流于2005-5-22日

