\section{第27章 鸾鹄飞残桐竹冷(下)}

王安石已经不记得自己第一次走进这座位于天下至中的宫城时到底是怀着什么心情。但他知道自己现在心情中,有着一半是是愤懑,剩下的一半却是如释重负。

在这座有大大小小殿阁楼宇的城池,王安石有着难以割舍的回忆,十年来心血的结晶,都已经全部颁行下去,收获的成果也是令人难以想象的,如今官军战力飞涨,也是因为靠着推行新法,而是朝廷有了足够的银钱。

这一座座宫殿楼阁见证了王安石的成功,但在这一次入宫之后,三五年内,他是不会再回来的。

王安石向着崇政殿走过去,一路上的内侍和班直都躬身退避。宰相的权威,让他们不敢有所怠慢,但这些人基本上都知道,今天是王安石结束宰相生涯的日子。

“王介甫是今天入宫陛辞吧?想不到他终于还是要走了。”

章俞难得上京一趟,没想到一进京城,就听到了这个惊人的消息。章俞对满不在乎,但他也知道王安石的离开不是那么简单。

‘兔死狗烹,鸟尽弓藏。’章惇只会将这些悖逆不道的话藏在肚子里,就算父子至亲也不敢说出来。但朝臣们泰半都知道,天子会放王安石离开,是因为他不再需要王安石这名太过强势的宰相,“少了介甫相公,还想推行新法也只有依靠天子了。”

“政事堂中不还有吕吉甫吗?”章俞随口问着。他的气色依然极好,坐在吊着水壶的火炉前,正拿着两块包装精美的小龙团,在比较着该用上哪一块。

“吕吉甫可是一点也不靠谱。”

章惇无意去跟吕惠卿争抢什么,他有自己的位置。只是要想让他去跟吕吉甫低头,向依附王安石一般依附吕惠卿,现在已经是枢密副使的章惇,怎么也不可能去那么做。

只是吕惠卿潜藏的野心,章惇看得很清楚。他多半是想取代王安石在天子和朝堂上的地位。但他跟王安石比起来差得实在太远,无论从品行还是人望上,都无法做到服众,更没办法将新党臣子都聚合起来,如臂使指的让他们为着朝廷做事。

章惇叹了一口气,王安石一走,对许多人来说,是散开了天空的阴云,是消失了头顶的巨石,是挡在身前的障壁崩离瓦解——吕惠卿多半就是这么想的,想必他现在就在家中暗喜于心。但也有可能消失的是船底的压舱石,稍大一点风浪就能让少了王安石来镇压场面的新党整个倾覆。

水开了,咕嘟咕嘟的响着,章俞随性一笑,将选好的茶团掰开来放进茶碾,慢慢的亲手磨练起来。

赵顼已经将江宁府的一座官宅,赐了王安石。

王安石病后初愈的脸色,让他之前告病的辞章添了一分现实的证明。

十年之前,王安石也是坐在这里,想赵顼介绍着富国强兵的方略。十年后,则变成了山岗的,时间在君臣二人的脸上留下深深的刻印,王安石没有了当初的意气风发,赵顼也褪去了少年时的稚嫩。

当初两人订立的目标,还远远没有达到。但为实现目标而使用的手段,则一条条的化为现实中的法律,在世间广为流传。

但他们现在讨论的并不是新法的问题,而是韩冈的去留。

“广西初定未久,交州更是百废待兴,没有韩冈在交州盯着,朕如何能放得下心来。”

赵顼其实希望韩冈能在外多磨练几年……最好是十年。也不一定是在广西、交州,其他地方也可以,只要等到他三十五六再回京师,在翰林或是三司,又或是群牧司做上几年,然后到了四十岁之后再进政事堂。

而在这期间,韩冈是没有机会返回京师。像韩冈这样的重臣,回到京师后,不可能就几个月就离开,而多是一年半载。以韩冈的才干,再立下点功劳,又该怎么安排?

王安石知道赵顼的想法,但他对此并不会反对。韩冈若是升任宰执的速度,也跟之前升官发财的初衷相违背。那就实在是太危险了。看似是快了,但对日后发展不利,稳一点慢一点才是好事。

但以他女婿的才能功绩,只要是在京城中立下些功劳,转眼就能跨进两府之中,谁还能当着他,就算是天子出手,也不可能将韩冈压得太久。他功劳太大,能力更是出众,一旦给他一个机会,就立刻能创造出奇迹。。

王安石心中想着,口中却将自己的打算说出来:“韩冈曾经给臣写的家信中,提到过襄汉漕渠。”

“襄汉漕渠?”赵顼并不是万事通,对于百年前失败的运河开凿工程,当然不可能会有多了解。甚至是连听都没有听说过。

王安石并不意外赵顼的‘无知’,如果没人去灌输常识给他,皇帝也不过是圈在高墙深垒之后的可怜人。王安石将自己了解到一些关于襄汉漕渠的事情,向赵顼作了说明。

“……如果漕渠完工通航,便能通湘潭之漕。荆湖两路和蜀中的出产也可走汉水直达京师。”

“能通湘潭之漕?”赵顼只听了这一句,眼神一下就变得专注起来。他当然知道这意味着什么,荆襄、蜀中乃至于江西的大宗货物,可以不去汴河绕个弯子,而是能由汉水北上,直接抵达京师。

想想一年六百万的粮纲,年年都要弄得沿途州县鸡飞狗跳。如果其中能有三分之一转由襄汉漕渠北上,那么汴河上的水运也能清闲上一点。整个京城的安稳与否,都与汴河挂上钩,如果能有另外一条路,分流一部分,汴河水运也就能变得轻松起来的。

“此事是否可行?”赵顼的心中还有着疑问,毕竟之前已经有过两次失败,都是水渠挖通了,却没有足够深的水。

“韩冈是如此说的。当不会有假。”王安石笑了笑,“以他的脾性,不是确认了有所把握,轻易不会发话。”韩冈的话已经成了金字招牌,许多方面,他说出来的话,比王安石这位宰相还要管用。

赵顼眉头皱了起来,他也选择相信韩冈的话,毕竟之前还有着一桩桩先例在,韩冈绝非是信口开河之辈。

那既然是如此,到底要不要将韩冈从外面调回来?还是直接将他调到京西去?。

……………………

这几天,转运使的行辕内外都是冷得如同冰点,

往常对待下人总是很和气的小韩相公——或是叫韩龙图,韩运使,转运相公——都是冷着一张脸,阴阴的,如同雨季的天空,见不到一丝阳光。

没人知道他怎么突然之间变得如此阴冷,但所有人都知道该如何趋吉避凶。尽管韩冈的心情不好的时候,并不会发泄一般的跟人过不去,也不会将自己的坏心情,转移到下人们的心上。

只是周边的人,还是会在经过韩冈身边时,尽量踮起脚尖来走路,争取不要打扰像是在想着事情,也像是在发呆的韩冈。

韩冈能够控制自己的情绪,但不代表着他的心情能一下好起来。张载的去世,给他的震动很大。

不再是重病不起,而是‘卒’。

早在韩冈还在熙河路的,他就已经知道张载身患绝症。同时也清楚的知道以张载的病情,不可能一直拖延下去,也做好了最后的心理准备,但他没想到这个时间这么快就到来。

如果当初张载接受了自己的提议就好了,肺痨要是能在山清水秀的地方将养起来,好歹也有个三五年,七八年。

只可惜张载不答应,韩冈当时也没办法强逼着他退隐山林。只好看着张载一天天的衰弱下去。

张载去世,说道悲伤韩冈的心中其实并不多,拖延了十年的宿疾,什么时候爆发都有可能,快也好,慢也好,迟早有那么一天,韩冈只是感到物是人非的无奈。

现在该怎么办?韩冈有些发呆。

交州的情况越来越好,大片的荒野重新开始变成有产出的耕地。就是诸多疍民,也在官府的引导下,开始了在陆上的生活。七十二家蛮部也驱赶着交趾人,挥舞着锄头开始在自己的土地上开垦着。将种植在这一片土地上的,有粮食,也有甘蔗。

糖业作坊虽然还没有主管过来,更没有开张,但地皮和房屋都已经准备好了。相对于糖业,交州的盐业越发的兴盛起来,晒出来的食盐每天都能有上百石。官府从中得到的利润,也不算少了,甚至是正常的夏秋两税都不能比。

到了这时候,交州之事已了,想要看到成果,则不是一两年之内就能看得到的。韩冈无意在这里留到成果出现的时候。

应该回京城去了,想到家里的娇妻美妾,还有六个聚少离多的子女,韩冈的心中满满的是无奈和歉意。自己的确不是一个合格的儿子,也不是一个合格的丈夫和父亲。如果有可能,他希望能在北方一个宅院同住下来。

事随人愿,到了腊月的时候,王安石去相位,韩冈转调京西转运使的讣闻和捷报就传到了正在桂州的他的手中。

