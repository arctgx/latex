\section{第45章 从容行酒御万众(七)}

缓缓压上的敌军,给三千汉军士兵带来的压力如同当曰初遇,甚至胜过很多。那散布在雪原上的攻城器具,比起跟在周围骑兵,给人更多的压力。

甲坚兵利是汉人最大的依仗,霹雳砲砸开了多少西贼城池,八牛弩更射下了北虏第一大将。而不论是西夏还是北辽,他们都打造不出类似的武器。只能凭着马快来回逃窜。远远超过四方蛮夷的武器优势,是很多将士自信心的最大依仗。

之前的试探,已经决定了黑汗军无法在正面交战中胜过宋军。很多士兵都已经感到胜利就在眼前。可现在黑汗人的攻城车一出,却让他们都感到了近在眼前的危险。

早间吃饭时的轻松不翼而飞,现在营地中的气氛压抑得可怕。只有少部分经历过官军尚未崛起的时代,只能拿姓命与敌相拼的老兵。在他们的,全军方才能够稳得住阵脚。他们现在都是都头、队正一级的武官。官位虽不高,却如父如兄,是最底层的士卒所依赖的对象。这些老兵才是一军之中真正的脊梁。只要他们还稳得住,这支军队依然是一支可以一战,让人放心的强兵。

有着为数众多的百战老兵,王舜臣都没有为士气担心上半点。而且孰优孰劣很快就能见分晓了。

走在最前的黑汗军骑兵,都快要抵达神臂弓和霹雳砲的有效射程了。

但他们并没有停住脚步,更向着营垒攻上来。看起来打算让更多的砲车和床子弩进入最佳的射击范围中。

“先打个招呼。”

王舜臣的命令立刻就转化为现实。

靠着寨门东侧的霹雳砲,将抛杆拉了下来,装上了石弹。木槌重重的击中了卡住抛杆的机关,崩开的木楔让抛杆高高的翘起。人头大小的圆形石弹划着完美的抛物线,飞向营垒外的敌军。石弹落下的位置,正是一台砲车。

战场上,数千人瞪大了眼睛,看着这一枚石弹抛起又落下。甚至有人屏住了呼吸,紧张到难以自已。

让许多人叹息出声。砲石仅仅隔了三尺,没有打中砲车。不过却正中砲车侧面的一名骑兵的胸口。胸甲立刻就凹陷下去了,甚至连后面的肋骨都一并打断。骑兵口喷鲜血,顿时便从马背上掉了下来,一只脚卡在马镫中,被惊吓到的战马拖着一阵乱跳。

这并不遗憾。

王舜臣明白,第一发就如此接近目标,接下来甚至不用做任何调整,只要多抛掷两次,必然能击中那门行砲车。

运气吧。

好些黑汗军士兵在想。

不然第一击又如何会离投石车如此之近?做过砲手的士兵有好些人,他们都不觉得投石车能够如此精准。要想说起准确,只有同时上来的弩炮才有资格。

可是很快,他们就知道自己错了。

马秦人营地中投石车,接二连三的发射,目标都是方才差点被击中的投石车,令人感到无比意外的是,他们的射击都没有偏离正确的方向,终于在第四击上,飞来的石弹重重的撞上了支撑的木架,让沉重的抛杆落了下来,成为了一堆碎木。

王舜臣冷哼了一声,但后方的士兵却没有他那么矜持,大声的欢呼着,之前压抑的气氛转瞬便消失了。

随着这一次开火,剩下的霹雳砲也开始了初次的射击。毫无例外的,它们抛出的炮弹不是近失,就是命中。六门霹雳砲仅仅两轮射击,就让数量更多的攻城器械,一下就垮了两架。这都是形制最大的行砲车,也理所当然的成为了最好的目标。

这样的结果,在黑汗人的眼中,犹如就在这样的天气下,被人泼了一盆冷水。本来还打算依仗攻城器械上的优势,争取一举攻下末蛮城。但现在还没有开始攻城,便一下就被石头给砸烂了。

军心的动摇显而易见,连威力最强大数量也最多的投石车都无法赢过对方,还能有什么可以依仗?

王舜臣正微笑着看着几架霹雳砲给对面的行砲车点名。六门霹雳砲连环发射,黑汗人的砲车在呼啸而来的石弹中,完全来不及移动躲闪,更不要说回击。一轮射击总有一颗石弹能够命中敌人,这样战斗下去,很快对面就不剩什么行砲车了。

霹雳砲下,每一门石砲都有近二十人服侍着。指挥他们的砲队队正,正大呼小叫,有条不紊的指挥着士兵们选取石弹、调节配重,然后给砲车上弹,将石弹投射出去。

霹雳砲的威力不在于远和重,而在于准确。砸不中敌人,射程再远,砲石再中,也只是能吓唬人。只能用数量来抵冲质量上的差距,然后祈祷运气。

抛杆臂长、配重,砲车的位置和高度,以及砲石的重量,经过计算之后,便得出射程。通过调整配重或是不同分量的砲石,也很容易就能改变投射的距离。虽然说不会精确到箭靶那个等级,但落点和目标基本上都差之不远,大部分皆能在十几步之内。

具体的数字和算式掌握在每一个掌管霹雳砲发射的队正手中。而且他们的手里都有一份射表,当他们指挥砲队操作霹雳砲时,大部分时候,都不需要临时计算,只要将砲车的数据对照射表,试射几砲后,进行一下必要的调整,就能准确的发砲。

这些砲队的队正都经过严格的培训,其中任何一个都能够读写书算。名为队正,实际上拿着的却是都头甚至副指挥使的俸禄,有的甚至还是流外官。再升几级便能晋入正从九品的流内品官行列。

这些砲队队正的许多训练,都是在王舜臣的麾下,或是赵隆那边进行。李信被调去河北后则极少有类似的培训——西军中可以做的,河北军却不方便。

比起制造规模更大的砲车,准确的使用要重要百倍。想要准确的使用,炮手们的素质就要放在第一位。素质有靠天分的,也有刻意训练的,更有学习和训练参半,然后培训出来的。

在这其中,自然是最有一项最省时间,也能早就更多的人才。就算敌人手中有了同样的武器,但在使用上,没有射表的辅助,就发挥不到一半的实力,不足为惧。

王舜臣的霹雳砲,盯住对方的砲车。一辆,两辆,然后三辆、四辆,很快推过来的行砲车便在宋军的攻势下损失了小半。但这时候,他们开始反击了。

在雪上艰难的行动,终于到了射程之内,黑汗人迫不及待的开始了他们的报复。

行砲车的发射速度比不上霹雳砲,但另一样攻城器却让王舜臣大开眼界,类似于床子弩,发射的是石弹。发射的速度甚至比得上训练有素的霹雳砲队。威力虽不如霹雳砲和八牛弩,但也已经十分惊人。弹射而出的石弹重重的撞击在栅栏上,咔的一声响,硬是撞断了木栅,陷进了后方的雪堆中。

如果多齐射几次,现有的防线很快就能被打破一个缺口。不过他们的距离实在太近了,已经近到了神臂弓的发射范围。

王舜臣一声令下,数百张弩弓同时射击,腾起的箭矢犹如云翳,一击之下,便让一架石弹弩车失去了所有的砲手。而下一击,只在几次呼吸之后,便又开始了。如同雨水冲刷过地面,另一架石弹弩车下哀鸿遍野,除了拿着两张大盾做掩护的士兵,其他砲手只要稍稍离开防护,就被密如雨点的箭矢射成了刺猬。在砲车后活动,不可能穿上碍事的铠甲,遇上神臂弓射出的箭矢,一两支就彻底失去了战斗力。

砲石和箭矢的交换还在继续。有着充分的防御措施,交换比的优势远远的偏向了宋军的一方。

官军的神臂弓都是集中在极短时间内爆发。用最少的消耗达成最大的效果。不是身经百战的精锐,也做不到这一点。

欢呼声和射击声都在宋军一方交替响起。

王舜臣等待着对面将领的反应。

他能不能派出更多的精锐?会不会再进攻。在他们被消耗光之后,黑汗军中还能剩下多少可以操作这些器械的士兵?他能不能损失得起这几万杂兵。

现在的形势远远不利于黑汗人。

古拉姆近卫是用来押阵的。就算是对面的主帅,也没权力把他们当成是消耗品。伊克塔的情况也类似。加起来能有一万的古拉姆和伊克塔或许可以保全,但其他士兵呢,在这样等级的战斗后,又能剩下多少。这样的损失,就算放在禁军中,也是伤筋动骨。

在自己选择的战场上与敌人作战,而不是反过来。

用适合自己的战法作战,而不是反过来。

王舜臣对致人而不致于人的理解就这两条。而那位稳重的黑汗统帅,正配合他做到了。

自用砲车、弩车攻城失败之后,又是三天。

一夜狂风过后,天气骤然转寒。

王舜臣几乎是个冻醒的。走出营帐,望着澄清的夜空,他不禁打了个哆嗦。

如果是他,这时候肯定要走了。

但有个问题,黑汗国的主君有没有足够的器量,原谅一个劳而无功的将领?
