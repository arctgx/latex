\section{第二章 天危欲倾何敬恭(四)}

章敦走后许久,韩冈犹在书房中。

同样一件事。分别从积极和消极两个角度看过去,得到的结论必然会大相径庭。

在韩冈的看来,幼主总比成年的皇帝好利用。需要担心的仅仅是十年后的未来,而不是迫在眉睫的危险。

随着对自然进一步的加深认识,气学与皇权的冲突在所难免。赵顼当初已经给气学添了太多阻碍,那还是望远镜刚刚出现,世人才将镜筒对准天际的时候。到了现在,气学快将天人感应之说给掀下台面了,已是图穷匕见。

当年旧党反对新法,指斥王安石为了变法,是‘天变不足畏,人心不足恤,祖宗不足法’。王安石对于这样的指控,也只是曲言辩说,没敢直接将天人感应给否定掉。而气学,可是要将这一切都给戳破。

追求真相的气学与皇权是天然的死敌。纵使来自皇权的反扑,不会像西方教会竖起火刑台那般激烈,但毁禁书籍、禁绝传习,将气学门徒难入官场,甚至治罪流放,一名有足够见识的皇帝,肯定会这么去做的。韩冈之前用子嗣的安危来钳制赵顼,这样危险的手段,不可能一直有效。

幸好赵顼早早的发病,不然真得拖到他死为止,才能放眼天空。也幸好让向太后接了手,否则以高太皇太后的脾气,自家说不定要到岭南去看星星了。

时间虽然变了,事情也变了,但如今一切的关键还是在向太后身上。

宰辅们害怕赵煦亲政,而向太后也会担心赵煦日后会将罪名推到她的身上。尤其是现在的向太后,与年幼的赵煦之间有着化解不开的心结。如果从这个角度切入进入,向太后是有可能被说动。可韩冈相信,向太后在被说动之前,必然会征求自己的意见。

“官人。可还醒着。”王旖在外轻声叫门。

章敦走后,韩冈久无声息,又未点灯,让人看了不由的担心起来。

“进来吧。”韩冈在内应声。

王旖随即盈盈走进了书房来。

韩冈之前先回家来,就让王旖带着长子、次子去王安石那边探视,并让两个小子一并跟着去,没事就住上一阵。这是他回来前,答应王安石的。

“什么时候回来的?”韩冈问着她。

“才到家。”王旖抬手点起了灯。

灯火亮起,火光映得屋中透亮。王旖转过身,关切的问道,“官人,出了什么事?”

“没什么。”韩冈摇摇头,“还是官家的事。”

“官家?章子厚过来还是想要废立天子。他们就那么怕皇帝长大?”

见妻子一切门清,韩冈惊异的扬了扬双眉,“是从岳父那边听来的?”

“爹爹那边没说什么,娘那里倒说了些。”

“那就是了。”韩冈咧开嘴笑道,“生年不满百,常怀百岁忧。”

“是在说章子厚?”

“是为夫自己说自己。”韩冈长身而起,“明天要入宫请天子、太后听政,得将衣服准备好。”

请罪归请罪,纵然衙门里的事情是不做了,但大行皇帝的丧仪还是得参加。

说起来天子的丧礼也是很繁琐的一通礼仪,而且不是一天两天。只是服丧,天子以日易月,也要二十七天才除服。而在梓宫入山陵,神主正式祔太庙之前,更要好几个月的时间。

“早就备好了。还能等官人你来问?”王旖横了韩冈一眼,然后唤了人进来,让她去取韩冈明天去宫中要换的衣物。很快,一叠衣服就被抱了过来,后面还跟着云娘、素心和周南。

所谓君臣如父子,天子丧期中,群臣也得一同服丧。平常的紫金鱼袋自然不能穿,得穿丧服。

不同的品级,穿戴的丧服等级也不同。韩冈的寄禄官是礼部侍郎,从三品。不过决定服色的不是寄禄官,而是散官阶。韩冈在这里是从二品的光禄大夫,属二品以上文武官,布斜巾、四脚、头冠、大袖、襕衫、裙、裤、腰绖、竹杖、绢衬服,一整套给配全了。

拿起四尺竹杖,青玉色的竹竿打磨的连一点毛糙的地方都看不到,光泽圆润,用来抽不好好读书的小子倒是趁手的紧。不过想到自己才三十岁就要拄着拐杖上去,而那些年纪老大却官位不及二品的官员,却只能空手站着,韩冈也不禁觉得,这样的形式主义是在让人烦。

不过要这么打扮的,好象还包括了亲王、皇子。皇子现在还没有,但二大王、三大王倒是有的。二大王那边,好像病突然间就好了。好好看一看这位大王又想闹个什么了。

妻妾们一起帮韩冈整理着丧服,除了竹杖只有一根以外,其他都有好几套来替换。

“到禫除还有二十多天。这些衣物都是官里送来的,做得匆匆忙忙,最是容易绽线。下午的时候,奴奴跟南娘姐姐、素心姐姐一起重新缝了一遍。”云娘仰着脸,请功一般的对韩冈说着。

“宫里面要做的也不过二三十人的丧服,怎么就不用心一点。”

周南则是抱怨着宫里面的手艺。与韩冈一个等级的文武官就那么多,他们丧服都是由宫里面帮忙裁剪缝制。二品以下的文武官,可都是发了布料让他们回去自己做。不过他们的丧服也简单,没那么多鸡零狗碎的配件。

“宫里面有的忙。她们自己还要给自己裁衣服,少不了事。”

王旖轻声说着。提起一件素麻的衣服打开来,却是她自己的裙装。王旖查了针脚和布料,然后小心的叠起来。外命妇同样要入宫吊祭天子,布裙、衫、帔、帕头,首绖,也是零零碎碎的一整套。

除了素色的麻衣孝服之外,韩冈还有浅色的公服,色泽比平常所穿的公服要浅淡得多,

这浅色公服名为惨服,是除服后改穿的官服,按照礼制,过了丧期,脱下丧服之后,还不能立刻穿上正色的官袍,得先穿惨服过渡才行。

韩冈这边的惨服自是淡紫色。朝廷直接给了布料,让官员们回家自己裁剪。如果是授五品服的官员,则便是将朱色换成浅红,绿袍、青袍,也都是更换成浅绿和淡青色。

家里面的织补班手脚一向快,不过韩冈的衣物,全都是王旖她们亲自来缝的。

望着房中的娇妻美妾,悉心的为自己整理着服装,韩冈的烦恼都沉淀了下去。

就是烦心,也没必要日夜。

……………………

房中素白一片。

床铺被褥是素色的,帐帘是素色的,茶壶杯盏也素色的,就连蜡烛也全是白。

在素白一片的厢房中,向太后一身素白的孝服,静静的坐在桌前。

厚厚一摞奏章放在桌上,很长时间都没有拿起来过。摊在面前的一本奏章,也不见翻动和批阅。

拿在手中的朱笔已经干了,许久不见动上一下。

但周围服侍她的宫人,没人敢打扰她。

向太后头很疼,头疼欲裂。

丈夫的死,本应让所有人都解脱了,包括他自己。但现在这种情况,缠绕在身周的负累,却是又加重了一重、两重、三重。

明明她一点都没做错,为什么现在她要担惊受怕?

明明她主持国政时,尽自己所能的做到尽善尽美,只想着等儿子成人之后,能对丈夫说一句不负所托,却为什么要担心起日后被人唾骂的危险?甚至亲族都有可能难以保全。

这明明都不是她的错!为什么现在还要为那个孽子苦心积虑?

犯下了弑父之罪,纵然是意外,但终究是他害死了先帝。

本来念着年幼无知,因一片纯孝犯下的大错,其情可悯。

前日在殿上,并不是韩冈说服了她,而是让她觉得这个选择更好一点,但现在却又不能那么看了。

蔡确说的,其实有道理啊。

“太后……”

“太后。”

“太后!”

身边的小黄门越提越高的声音终于惊动了向太后,“怎么了?”

小黄门颤着声,“禀太后。石都知回来了。”

“让他进来吧。这些奏章都撤了,明儿再说。”向太后吩咐着。

几名内侍将几堆奏章搬了出去,石得一则随即进来。

待石得一叩拜问安之后,向太后问着他:“保慈宫那边怎么样了?”

“禀太后,太皇太后一切都安好!也已经准备好”

“没有其他异动?”

“……”石得一一阵沉默,然后慢慢的摇着头,“没看出来。”

“吾那位二叔呢?”

“病已经大好了,不疯不傻,说话也清楚了。只是在哭,一直都念着先帝。”

向太后冷笑着:“病好得还真是时候,这病气还真是体贴。”

石得一汗流浃背,他面前的太后,明明白白的带了杀意了。

“三叔和蜀国怎么样了?”

“三大王自回京后,一直在读书,至于大长公主那边,则一直在抄经,是用舌血。”

“也不知道学一学。”向太后哼了一声,又盯着石得一,“这几日,不要让京城里出乱子,警醒一点。”

