\section{第28章 大梁软红骤雨狂(二)}

韩冈说是沐浴更衣,其实也是想在见王安石前,与王韶互相之间通个气。

王安石如今正得圣眷,换作普通的官员,当听到他的召唤时,只要不是与其党派有别,都会忙不迭地跑去听候差使。甚至不用招呼,只是为了能在王安石面前说上一句话,每天在王府门前能站上一排人。若是一些心机略重的,更是想着用满面风尘到王安石面前,换声‘幸苦’。

而韩冈不因当朝宰相的看重,改变自己的行事步调,这是纯正的士大夫的脾性。王安石会怎么想,王韶并不知道,但至少他是很欣赏。

只是宰相家人,王韶也不便轻忽视之,随便丢在一旁。他看了王安石派来请人的家丁一眼,正想找个借口进驿馆中。四十多岁的仆役,脸上看不出半点不快的神色。并没有宰相家仆人傲气凌人的脾性,心思通透的躬身道:“请官人自便,小人就在这里等候。”

王韶暗赞了一声,点点头,便也转身进了驿馆中。

驿丞正要领着韩冈去他的房间。由于是上次的老熟人,加之方才的一幕,城南驿的驿丞对韩冈点头哈腰,恭谨非常。驿丞一叠声的催促着馆中的驿卒,让他们挑住一间上房给韩冈。又让人立刻准备洗浴之物,为韩冈准备上。

韩冈温和谦退的笑着,并不因为驿丞的礼敬有加,而变得狂妄起来。虽然王安石的家丁正在门外等候,但他仍旧是不慌不忙,一点也没有心浮气躁。他的这副宠辱不惊的作派让驿丞加倍恭敬起来,腰低了两寸,笑容也多了三分。

韩冈并不怕王安石会因为苦等而生气,他到京城的具体时间,连他自己都确认不了,何况王安石?门外的王家家丁,摆明就是计算过韩冈的行程,一直等在驿馆外的。眼下这个时间,王安石应该还在中书衙门里,就算不下马就去王府,也还是要在门房或是偏厅中等着。

韩冈一边听着驿丞的奉承,一边望着大厅的入口,很快,王韶果然走了进来。一别一个多月,再相见时竟然却是在京城,世事难测,这也是一个现成的例子。

时间短暂,王韶和韩冈见礼过后,也不多余的废话。驿丞带着两人一起往里走,远远的在前面领头,其他人也识趣的远远落在后面,总计才七八个人,就分成了三拨前后走着。

雕栏画栋的长廊,通向韩冈前次入住的院落,不过今次驿丞没有在那间院子前停步,而是向后绕去。

王韶神情郑重的问着拖后半步的韩冈,“玉昆。韩相公上书要调你去延州,你的想法到底如何?”

王韶问得直接,韩冈便摇摇头,正色回覆:“河湟功成在即,下官何苦去延州受牵连。”

听出了韩冈的言下之意,王韶微一扬眉,故意反诘道:“朝中鼎力支持,陕西河东同心协力,横山一役未必不能成功。”

“即便成功又如何?河湟是下官心血所在,而横山却是少见亲近。舍近求远,舍此而就彼,智者不为也!”

两个选择摆在面前,韩冈挑选起来却没有半点犹豫。他在河湟已经扎下了根基,那里是他的根据地,从瞎药开始,诸多蕃部,都要听着他的号令,一句话就能让他们奔走起来。而在秦州,上至郭逵,下至小吏,他都能说得上话。让他去几乎可算是敌占区的延州,一切从头开始,韩冈没那么傻。

而且河湟之地直接连通河西走廊,日后攻下兰州,还可以直往西域。虽然在眼下,还没有听说天子要拓土西域的打算,而在韩冈的记忆中,他前世也没有听说过北宋有远征西域的事迹。但韩冈自信有他在,承汉唐之遗风,重开西路,绝不是梦想。只要把根留在河湟,功劳可以说是源源不断。

这样的情况下,他去韩绛手底下做什么?横山的蛋糕早就被瓜分光了,在韩绛帐下,就算把分派给他的任务做到百分之两百,也只能分润一点残羹剩饭。不比在河湟,作为王韶和高遵裕的副手,同时也作为各项政令最重要的执行者,他受功的顺位始终排在前五。

尽管韩绛是首相,而王韶仅仅是个缘边安抚使,要辅佐的对象地位天差地远,可韩冈一直都是宁为鸡口,不为牛后。

“……这样我就放心了。”听到了韩冈的表态,王韶点了点头,默默地走了两步,踏着长廊地板的声音有些空洞。神情慢慢变得严肃了起来,声调微沉:“玉昆,你还是去延州一趟比较好!”

韩冈闻言便是一楞神,转过头看着王韶,见他的神色不似在试探。他心知必有枝节横生,皱眉问道:“究竟出了何事?”

王韶轻声叹了口气,“韩子华前日重又上书,要调玉昆你去延州。”

“第二本?!!”韩冈顿时失声惊道。声音传到了前面,领路的驿丞顿时加快了两步,以示自己无心。

王韶点头,望着前面:“第二本。”

韩冈顿时默然,王韶也不知再说话。两人跟着驿丞绕过前廊,穿过一堵院墙,一座面积广大的园林顿时出现韩冈的面前。

淡泊的腊梅香在园中浮荡,十几重小院落在假山、水池还有花木之间前后错落的布置着。这里城南驿最好的客房,没有一点地位根本住不进来。韩冈地位虽然不够,但他身后有人,驿丞也不会傻到秉公依律,安排他住进普通的房间里去。

在冻结的水池边走过,沿着蜿蜒的石板路,从近百株腊梅中穿行,最后在略显偏辟的一间小院前停下,驿丞指着这间院落,“这件院落虽然偏僻了一点,却是清净得很,不知韩官人意下如何?”他又指了指近处的另一座小院,“那边是王官人的院子,正好就做个邻居,无事时也好走动。”

韩冈哪还有什么挑的,他本也不看重这些,爽快的点头同意。

见韩冈首肯,驿丞便带着他们进院参观。韩冈这边就算加上李信,也只有三人的规模,住进至少能容纳二十人的小院,实在是宽敞过了头,也过于浪费。这里不愧是京城,最简单的布置也是让秦州的酒楼望尘莫及。

韩冈很是满意,谢过驿丞,驿丞回礼后,说了声请韩官人少待,很快就把洗浴之物送来,便快步离开。

李小六抱着行李去内间安顿,而韩冈和王韶在正厅中坐下,望着攀爬在院墙上的丛丛枯藤,他终于有些讽刺的笑出了声,“……韩丞相的看重,真是让下官受宠若惊啊!”

他虽然对官场的认识还不深,也清楚这样的征辟并不正常。韩绛再看中他都不至于连上两本奏章。除非有人从中作梗,需要多次上书,否则无人反对的情况下,何须多费笔墨……

想到这里,韩冈突然扭头,看着王韶。王韶猜出了韩冈的想法,则摇了摇头。

韩冈苦笑起来:“事有反常必为妖,这就更是要拒绝了。”

“拒绝韩子华的征辟要有分寸才行。实在推却不过,应下也无妨,莫要惹得天子和两相不快。”韩冈在前面表现出了忠诚不渝的姿态,加上他一贯的表现,王韶如今早已视他为亲近子侄,说得话都是为韩冈着想,“古渭寨……不,通远军总有你的位置,玉昆你也不必怕会我有什么芥蒂”

“通远军……”韩冈先是一愣,转而就恍然大悟,起身对王韶道:“恭喜安抚!”

王韶也笑着回礼,“要到年后中书才会发文,升古渭寨为通远军。我将会兼任通远军知军……辛苦了几年,也终于能见到回报了。”

“日后的回报当是会更多,辟土服远,封侯亦是等闲。”

王韶笑容平淡,但眼神中有着浓浓的喜色,“不说这些了。李信现在住在我那里,这时候去了三班院,大概要到晚间才能回来。他试射殿廷的时候也快到了,大概会赶在腊月廿三祭灶前,也就是没几天了。”说着他站起身,“好了,不耽搁玉昆你了,我也回去换身衣服,等会儿跟你一起去见王相公。”

……………………

换上了正式的公服,韩冈终于和王韶一起从驿馆中出来。从他进去,到再出来时,已经有半个时辰。而王家的家丁依然心平气和的在门口守候着,并无一声怨言。周围的官吏看到后,都少不得赞一声王安石治家有方。而韩冈也暗赞着,上前道了声辛苦。

而韩冈方才进去时风尘满面,灰头土脸的,疲惫不堪的神情看起来稍显狼狈破落。但他自驿馆一进一出,更衣沐浴之后,整个人就完全变了。顾盼之间,目光如电。神采焕发又不显张扬,文翰中带着英武之气,是个人物难得的少年郎君。

看到韩冈此时的形象,众人暗暗喝彩,如此人物,的确当得起王丞相的看重。

驿丞已经殷勤为王韶、韩冈安排下了马匹,谢了一句,韩冈就翻身上马,跟着王家家丁,一起向王安石府行去。

