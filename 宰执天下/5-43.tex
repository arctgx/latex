\section{第六章 千军齐发如奔洪(中)}

【昨天第二更】

攻下了两道关口,穿越了横山,出现在苗授面前的不是严阵以待的敌军,而是两条通往灵州的道路。

一个是向东北方向直接攻过去,翻过黛黛岭之后,便能抵达韦州,再往前就是流向灵州的灵州川,正好能与环庆军会合。这是预定中的计划,也是环庆路副总管高遵裕的命令。

而另一条路,则是继续顺着葫芦河向西北绕上一点路,先行抵达黄河。就在黄河河畔边,有一座鸣沙城。

“听说鸣沙城是西贼的粮仓,囤积了数十万石粮草。”

“不可能有那么多,西贼这些年穷得都要将裤子押进质库了。这两年关西都是丰收,粮价却依然上涨,有西贼遣人回易关中粮食的功劳。”

“能有个三五万石,也够全军一个月食用了。至少不用全数依靠后面的民夫运粮上来。他们有几个会拼命卖力?”

“高总管可是下令要我们去韦州跟他会合的。”

“当真去跟环庆军会合之后,还会有我们立功的机会吗?”

下面的将校低声私语的声音不知不觉间因为争吵而大了起来,传到了苗授的耳朵里。

苗授用力揉着眉心,头疼欲裂,却无心呵斥几句。他手底下的这几位将校其实说得并没有错,所以才会成他头疼的主因。

依照高遵裕之前的吩咐,要泾原军尽早与其会合。不过泾原军的粮草问题更是一个麻烦。

到了高遵裕那里之后,泾原路的民夫却很难推着粮车追过来。口粮得要靠环庆路帮忙解决,可苗授并不认为在高遵裕手下,就能及时的得到粮草上的补给。高遵裕变不出粮食来,而且环庆路出动的兵力,比泾原路要多得多,哪里还能有多余的粮食给苗授?

但高遵裕的命令也不便置之不理。以苗授对高遵裕的了解,太后的亲叔叔可不是宽容大量的人,绝不会原谅自己的冒犯。

到底该走哪条路,苗授现在处在两难境地中,相比之下,打仗反倒是简单了。

下面的将校议论了半天,却不见主帅有任何反应,只顾着揉着脑袋。终于有人忍不住催促道:“苗帅,你发个话吧,到底是走黛黛岭还是去鸣沙城?这样犹豫着不是事啊!”

所有人都静了下来,一对对眼睛盯住苗授,等着他的决定。

“总管!”苗授的儿子苗履用官职称呼着父亲,“吾等十数载枕戈待旦,只为如今一战。还请总管示下。”

麾下将校和士卒们希望建功立业的心思都摆在脸上,儿子眼中的野心,更是瞒不过苗授。一声总管,乃是在提醒苗授,他是跟高遵裕平起平坐的一路统帅,只要能有足够的战功,即便不理会高遵裕,也无关紧要。

“……去鸣沙城!”苗授盘算了半天,一咬牙下了决心,“西贼多半会防着我们与环庆路会合,往黛黛岭去当少不了还要打一仗。攻向鸣沙城正好打他们个措手不及。”

苗授的解释谁都听得出来是借口,但下面的将校们也只需要一个借口。

人人抖擞精神,高声喊着末将遵令,便开始争夺先锋将的位置。

随着粮道的延伸,粮草的补给会越来越慢,军队只能停下来等补给。而一旦有了足够的粮食,就不用等待后方将粮草送上来,进兵的速度自然能更快上几分。

而速度代表了什么?——是功劳,封妻荫子的功劳!

只要能在眼下将高总管搪塞过去,没人愿意放过立功受赏的机会。党项人避而不战,现在泾原军上下都充满自行,可不需要靠拢主力来壮胆。

麾下众将为争夺先锋开始争吵,苗授眉宇间的忧色却没有任何变化。身为主帅,考虑的不仅仅是功劳,还有迫在眉睫的危机。

在战前,有一点所有人都忽视了。陕西一路,从来没有在战争时成功维持过一条超过三百里的补给线。而兵力在十万人以上的粮草转运,也同样没有任何经验。

当这两点结合在一起,同时路程延长到一千里,兵力增加到三十万,尽管事先计划得再好,筹备得再充分,后方有着足够的粮食,又有着充裕的人力,但运送粮草这个行动的本身,却无论如何都难以维持到最后——苗授对此十分悲观。

秦凤转运司,永兴军路转运司,以及各路大军的随军转运,都缺乏将粮草及时送到前线大军手中的能力。不过是刚刚抵达西夏境内而已,苗授已经用亲身体会感受到了这一点。每天运抵他手中的粮草数量,随着大军前进的脚步,不断的在减少。尤其是在翻越横山之后,昨天比起三天前到账的粮草数目,少了整整三分之一。

这是无可奈何的一件事。

为了保证粮秣的安全,最靠近前线的一处粮草囤积地,离战前的国境足足有两百里之遥。而其他几处囤粮点,则离前线更远。即便苗授已经下令以磨脐隘为兵站,命后方将粮草尽快运抵,但能比得上韩冈、沈括那般能力的官员本就是凤毛麟角,就是王厚都少有人能比得上,对于苗授的要求,能完成七八分已经很难得了。

在这样的情况下,必须要用最快的速度攻下灵州。

灵州是兴庆府的门户,西贼再是诱敌深入,也不可能放弃灵州。灵州城中必然有大量存粮,足够全军食用。

对于高遵裕,苗授也只能先说声抱歉,他身荷数万将士的性命,还有天子的嘱托,先保证自身的粮草供给才是第一位,至于命令,得向后推一推。

……………………

与此同时,刚刚攻下夏州的种谔也在为粮草而头疼不已。

党项人根本就没有坚守夏州的打算,但他们有充足的时间去将城中所有的粮食全部处理掉。在官军攻下夏州城后,用了两天的时间,才找出了三千石,正好够全军两天吃的份量。

之前种谔领着鄜延军一来一回,不仅仅耽搁了时间,消耗了士气,还让种谔现在只能依靠后方运送粮草上来,已经被挖过一遍的地里,掘不出第二遍的粮食。

眼下比之前攻入银夏的时候,种谔麾下多了三万京营禁军。兵力虽然增加了一半,可全军的战斗力却是不增反减。更加不幸的,是粮食的消耗跟兵力增加的幅度相同。

加上骑兵的坐骑,整整多了三万五千张嘴,而战马的食量几近常人的十倍,刚刚走到夏州,种谔就已经不得不停下来等待后方的粮草运上来了。可京营禁军的几位将领,却一个劲的来催促种谔加快速度——之前被朝廷强令撤军,打击的不仅仅是鄜延军的士气,同时也让京营禁军的气焰变得嚣张起来,甚至在种谔面前也很是不逊。

立于夏州城头,种谔无心观赏夏州内外难得一见的风物,头顶着烈日,右手无意识中的一下一下的捶着墙头雉堞,汗流浃背亦不知所觉。

夏州是银夏的核心重镇,但一心想将宋军诱过瀚海的党项人放弃得很干脆。城中守军只有两千多人,而且还都不是精锐。种谔就是通过俘虏和飞船侦察到了这一点,才硬是不顾京营的力争,而将攻城的任务交给自己的人。

功劳就是功劳,斩关夺城不会因城中守军多寡而有太大的波动,攻克夏州的功劳并不比斩首千八百稍差。不过这么一来,京营禁军就更难带了。

“太尉,刘归仁他们闹着要出兵,怎么办?”

声音从身后响起,敢在种谔沉思的时候过来打扰,也就种家的几个子弟。

“想在太阳底下走路,尽管去,本帅不会拦着。还会顺便帮他们往京城家里捎封信,把以身殉国的赏赐送上。”

说句难听话,种谔最想做的就是将京营禁军派去北面的沙漠里面去,让他们自生自灭。对于眼前的这一场战争,不要浪费宝贵的军粮,是他们唯一能发挥价值的地方。

种朴咳嗽了一声,脚都没动一下。

种谔转回身来,脸上阴云密布的表情,与头顶热辣辣的烈日有着鲜明的对比。

“传本帅军令!”种谔一提声,十步开外的亲兵忙跑了过来。就听鄜延路主帅冷声传令:“营中禁喧哗。营中喧闹者,杖六十。扇惑人心者,立斩不赦。若不自重,就莫怪本帅的刀子不留情。”

亲兵应了之后,见种谔没有别的吩咐,就立刻下城去传令。

种谔转过来又对儿子:“把第四将的骑兵带去,查一查瀚海绿洲里面的水源。看党项人现在的架势,应该没有下狠心毁了才是。”

种朴一愣,立刻又恭声道:“末将谨遵太尉钧令。”接了将令,他又问道:“大人,党项人眼下千方百计的就想将我们诱到灵州城下,他们就有那么大的把握?”

“不然他们能怎么办?”种谔冷笑着,“毁了瀚海中的水源地?就算让他们侥幸赢了官军,日后怎么能跟银夏交通往来?”

“但粮草怎么办?光有水,瀚海也过不去。”

“那得看看李转运的本事了。”种谔冷哼一声,“若是他做不到,只好请天子公断了。”

