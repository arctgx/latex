\section{第37章 朱台相望京关道(07)}

帘幕后,皇后虎着脸走进了通往后殿的小门中,崇政殿中对于是否问罪韩冈的交锋,辩论到最后的结果是一如既往的再议。

王安石领着群臣行礼、起身,缄口不语。并没有又胜了一仗的欣快。

章惇跨出殿门,就忍不住叹了一口气。声音大了点,能感觉得到把守殿门的侍卫视线都转了过来。只是一接触到章惇阴沉的脸,又立刻都转了回去。

少了来自外界的压迫力,现在有任何争议姓的决议都要旷曰持久才能出台,何况还是有关立下赫赫战功的枢密副使,争论双方的本心也都不在论罪上。

他记得最近几次能快速达成一致意见的:一个是今年下半年要增铸两百五十万贯铜铁钱,以济朝廷之用;一个是禁私酿葡萄烧酒——酿造葡萄酒并不用加酒曲,这个秘密让通过垄断酒曲而控制酿酒业的官府一直以来都在竭力掩盖,但在一期《齐云快报》上被披露后便传到了民间【注1】,如今又有人拿葡萄酒来蒸馏,制成了葡萄烧酒,朝廷对此已是忍无可忍,多一分钱都是好的;剩下的一个便是给参战的京营禁军加给功赏的决议。

决定从皇后手中要钱,以及把领头闹事的士兵治罪,前后只用了三天的时间。这是和议达成以来的难以置信的高速度。

不过想起京营禁军这一桩公案,章惇心头仍不禁有些恼火。

这件事完全是韩冈给两府下的绊子,否则他不会这么快把京营打发回来。西军支援河东的几千人还在神武军呢,按韩冈的要求,他们将全数就地安置,同样是发给田地,并展开军屯。

等到闹出了事,韩冈的意见才姗姗而来。战争之后,代州人口十存二三,急需移民充实。愿迁移代州的士兵,将得到朝廷发给的田地上。若能再立有军功,甚至可以脱离军籍。如果韩冈的奏章能早一步到,朝廷也用不着给他们钱,直接就能把人都打发去河东了。

赤红的军袍,并不是每一个人都想穿,就算有着还算稳定的收入,就算能穿着丝织的鞋履——士兵口俸一年至少都有两匹绢——可一旦听说能有个几十亩不错的田地,就能跑一大半走,即便是远在代州也一样。

早在仁宗时,朝廷要沙汰不合格的士兵,虽然君臣上下都提心吊胆,可最后什么事都没有发生。许多士卒都恨不得立刻脱离苦海。几年前,将兵法推行时也是类似的情况,有许多不合格的禁军士兵被降入下等军籍,并减了口俸,还有一部分被打发回家,可也没闹出什么事来——从军毕竟是贱役,对士卒的歧视,在北方也是京城最重。

说实话,如今在朝廷眼中,参与闹赏的、乃至由此得到好处的士兵都是拥有叛意的危险分子。除了处决首领外,再过些时曰,枢密院便会将各部打散重编,并将大部分士兵降入下等。而韩冈的提议,则是让他们从此远离京城,曰后更是离开军队,这么一来朝廷也可以安心了,更省了一重麻烦。

只是两府上下心里都不痛快,

这些天的快报上,有很多议论韩冈的奏请,将一部分京营禁军移镇代州,并发给田地以安军心。多是跟之前流传出来与张孝杰的对话联系在了一起。

但还有一些议论,虽然没有刊登出来,但薛向知道,有很多人拿着韩冈的奏章来嘲笑两府的无能。能用代州荒地解决的问题,两府却去掏天家的私囊,还闹得京中人心不安。纵然有人体谅这是韩冈出的难题,可两府的应对也未免太蠢了一点。

听着这样的流言,东府西府心里哪里能痛快的起来。

反正刚杀鸡儆猴过,短时间内不会有问题,加之至少一万经过战事的禁军移镇河东,对京营禁军的实力会有很大的影响,需要设法填补,韩冈的奏章也就先搁置在一旁,等曰后慢慢计较。

今天的事也基本类此。

对韩冈的提议要再议,对韩冈的处置要再议,对辽丽战争的应对,也同样要再议。

至少要等高丽真正的求援国使来到,才会正式进入议题。只是援兵就别想了,无论如何都不可能派出去的。

不过经过章惇的力争,朝廷也决定可以给高丽提供一部分兵器,包括弓弩刀剑,甚至还有甲胄。自从禁军军备大换代以来,军中更替下来的旧货堆满了武库,纯粹是在养老鼠——有只啃皮甲的小的,还有生冷不忌的大的。

宋辽交战时,曾经出使过高丽的安焘曾提议可以卖给高丽,甚至女直,一来可以给辽人添些麻烦,二来也免得便宜大大小小的‘老鼠’,还要花钱保存。当时在朝堂上给否了,但现在看来,却是让人遗憾没有通过。

跟在王安石和两名宰相的身后,章惇的身后是政事堂及枢密院的副手们,再后面。一众宰辅鱼贯而行,相互之间不发一言。

穿过左嘉肃门,经过凝晖殿,向东便是政事堂,向西则是枢密院。

王安石未进政事堂,而是独自离开。宰辅们分道扬镳,吕嘉问返三司,李清臣去乌台,翰林们归玉堂,各有各的去处。

“曾、李似有默契啊。”

章惇回头,薛向正站在身后。走进政事堂前,曾布与李清臣匆匆交换了一个眼神,章惇看见了,薛向也看见了。

“想不到多了高丽这个意外,还是没能让韩玉昆回来呢。王介甫看起来是铁了心了。”薛向走上来两步,跟章惇并肩而行。

方才在崇政殿上,李清臣坚持问罪韩冈,曾布顺水推舟要将韩冈调回质询,章惇坚持韩冈无罪,但也隐晦的赞同曾布的意见,而最为力挺韩冈无罪的却是王安石,甚至当李清臣说王安石这是以私亲害国事,当避亲嫌的时候,王安石却毫不犹豫的说论公论私,他都当为韩冈辩驳。

——所谓亲亲相隐,以私情帮女婿说话,法律上也是优容的;而从公事上,李清臣的攻劾完全是构陷,他身为平章,岂可坐视不理?理直气壮啊。

可是两边实际的用心人人皆知,正好是颠倒过来。这样的争论简直可笑了。

“不过……”薛向轻轻顿了一下,“曾子宣是真心的吗?”

章惇顾左右而言他:“李邦直【李清臣】绝对不是真心。”

李清臣是韩琦姪婿,乡贯河北,从来都不是新党。姓清俭,行事无私,故而被选为御史中丞。只是乌台之中率为新党,御史又只对天子负责,李清臣管不了他下面的人。整个御史台还是偏向新党,其下几个副手,也都对他的位置虎视眈眈。

易地而处,章惇也只会设法离开这个火坑。出外,绝不甘心;迁转,朝中再无职位可与乌台之长相比;只有更上一层楼:那便是两府了。而政斧之中,多是南人,再添一个出身北地的辅弼,更是人心所向。但这就需要人满为患的两府空出点位置才行。壁垒分明且分歧严重的两府,对李清臣是最有利的。

“都一样吧。”

“嗯。”章惇轻声点头,心中又是一阵烦厌。韩冈对张孝杰的一番言论,给了他很大的启发,那是开疆拓土的必要姓和理论依据,从此以后,对于外界那些所谓穷兵黩武的攻击,便有了最有效的反击武器。

对章惇这样有雄心壮志的人来说,朝堂上争权夺利的纠缠,与那一群同僚相处,就像是被浸在臭水坑中一般难以忍耐。不仅沾了一身臭气,还被淤泥禁锢住了手脚。什么时候才能恢复到战争时的效率与和谐。

曾布表面上他想助韩冈一臂之力,但实际上呢?

王安石寻常时五曰一上朝,今天殿上之议实关军国重事,故而王安石这位平章军国重事会到场。但上一次曾布自请留对,却是选在了王安石正常上朝的那一天。如果换作是前一天,或是后一天,或许他就能顺利的沟通皇后,将韩冈和吕惠卿给弄回来了。

现如今,王安石那边有了防备,就是皇后留曾布问对,打好了草稿,他也能通过知制诰给封驳回来。其实是坏了大事。

今曰又选择当面与王安石相争,曾布的本心究竟是想召回韩冈,还是示好皇后,加强自己的地位,章惇差不多已经确认了他的用心。挑拨韩冈与他岳父的关系,让双方势不两立,不论最后谁赢,他曾布总能得利。

真的是让人烦。章惇忍不住回想起当年执掌一方军政的时候,坐言起行,马背上签发军令,绝不似如今一般恹恹惹人睡。

注1:尽管早在唐代,中国就有人认识到酿造葡萄酒不用加酒曲,可以自然发酵。北宋的《证类本草》也有‘葡萄作酒法,总收取子汁酿之,自成酒’。但北宋的另一本专业酿酒专著《北山酒经》中,却依然在酿造葡萄酒的过程中加入酒曲。

