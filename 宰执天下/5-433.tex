\section{第37章 朱台相望京关道(04)}

“竟然有三千斩首?!”

“不过是附贼的黑山党项。之前置胜州的时候,就已经杀了不知多少,现在只是斩草除根罢了。”强渊明觉得这个数字没什么大不了的,之前见多了,黑山党项现在连个报仇的都不存在了,“现在倒霉的是要掏钱的三司衙门。还有就是不知辽人会作何想。”

只因为韩冈在奏章中还说了一句,‘错用舆图,迷途失道,误越疆界,幸无大碍’。折家的兵马追杀黑山党项追到辽境去了。 . .

强渊明摇摇头:“辽人可不会觉得是误会。”

“当然不是误会!”

这话说的,多轻巧啊!

蔡京看了之后直磨牙,韩冈这是骗鬼啊。

“三千啊,三千!”蔡京站起来在公厅中来回打转。

今天三千斩首,明天就八千斩首。一个斩首就是一桩功劳,一桩功劳就是一份赏赐。之前两府、三司几乎都被功赏赶上了避债台,现在又是几千几千的斩首过来,这是要让王安石做周赧王吗?

所谓同仇敌忾。国库的钱帛少了,对在京的百官、群吏都是噩耗。他们的吃穿用度都是要靠国库的。纵使日常的俸禄不会增加,但节庆时的加赐可就会被克扣惨了。这样的河东,这样的韩冈,在朝堂中,如何不会被视为一个麻烦制造者? . .

不提政事堂,只想想枢密院,章子厚为了京营禁军闹赏一事已经很恼火了。再想想三司,为了新添的封赏,上上下下多少人伤透脑筋。韩冈这么做,却是把所有人都得罪了。就是士林里面,也会对韩冈的行为不以为然。

要是三五百,蔡京倒是相信了,出动的将帅都到了折克行这一级,怎么能没有几百人头祭旗?但眼下是足足三千斩首,绝不会是什么‘错用舆图,迷途失道,误越疆界’,若真的如韩冈所说,折克行当真该去死了早该死了,哪里能活到现在。

肯定是追杀过去的,然后杀光了人再拿着脑袋回来,所以叫‘幸无大碍’。

“那些黑山贼在辽国败退后,绝大部分肯定早就逃到了辽国境内了,谁还敢在府州多留。麟府军能拿到那么多斩首,肯定是故意越境……不对,”蔡京的脚步忽然停了,脸上露出来惊恐的神色,声音变得发颤:“黑山党项原本就不剩多少,辽人入寇,敢于背反的也没有几个部落。三千斩首哪边来的?谅折克行也不敢拿老弱妇孺来充数!”

“……辽人?!”强渊明一下跳了起来,“他疯了吗?!”

韩冈究竟在想什么?强渊明完全不明白。他难道不想回京城了吗?北方不稳,他肯定回不来了。如果只是黑山党项那还好说,说不定耶律乙辛还会觉得多一事不如少一事。可要是杀到自家人头上,耶律乙辛怎么都不可能再忍的。

“会不会是韩冈没管住下面的人?”强渊明想着理由,“之前府州和制置、转运二使司的奏报中不都有说过吗,黑山党项趁之前府州兵力空虚的时候,在河外造了不少孽。虽然不会有代州、忻州那般严重,但以折家的骄横不可能不报复的。”

“绝不可能!”蔡京一口否定,“没韩冈准许,折克行一兵一卒都不敢调动!”

蔡京绝不相信折克行敢自行其是。且不说折家一向谨守法度,就是以韩冈如今在河东的声望,即便折克行换了种谔的性子,也不会没得到韩冈的允许,就出兵边境。

“但韩冈是肯定想要回来的。他放纵折家,岂不是南辕北辙?!”强渊明摇头,只要边境上辽人闹起来,韩冈可就必须在河东多留上两三年了。

蔡京也想不明白,韩冈为什么能那么笃定,辽人一定不会与他为难?

……………………

“这如何还要想什么?这自然是韩冈有把握!”邢恕在蔡确府中放声道,“他有把握,辽贼不会为此报复,甚至边境行猎都不会有。这样妄启边衅的罪名才不好安”

若是恶狠狠的责罚了韩冈之后,辽国却没事一般一声不吭。到时候一句‘乃复坏汝万里之长城’,两府诸公还要做人吗?

蔡确眉头皱得死紧,问道:“他哪里来的把握?”

“邢恕不知……但只有确定辽人会做缩头乌龟,韩冈才敢下令折克行去越境杀人。他必然有把握!”

宰辅之中,对辽国最了解的不一定是韩冈,可最熟悉河东北方的则肯定是韩冈。对于雁门附近的人文地理,现在开封城中,就算是三尺孩童都能说上两句,可胜州的位置则太远了,国境对面,究竟有多少敌人则没几人能说明白。如果一定要说有,现任河东制置使韩冈必然是其中之一。

邢恕很确定,韩冈有把握完成这一切。

但这完全是废话。

不过蔡确也不能说邢恕说得不对。韩冈历年来给他带来了太多‘惊喜’,猜度韩冈会怎么做,很难;但猜测结果却很容易,把赌注押在韩冈身上就行了。若蔡确不是宰相,他完全可以这么做。

可是他是宰相啊,不想清楚韩冈的心思和手段,他怎么敢下注。

蔡确的手指用力抚着眉心,刚刚解决了京营禁军的问题,正是头疼的时候。

增给的赏钱发下去后,闹事的禁军偃旗息鼓,朝廷随即秋后算账,将领头闹事的二十四人,一起押赴刑场,生剐五人,腰斩十三,剩下的六个也都判了斩立决,一个都没放过。还有近二十名的大小将官,也以治军无方为由,被左迁、罢职甚至是追毁出身以来文字罢职仅仅是丢了差遣,而追回出身以来文字,直接是削了官籍。这就是朝廷一贯的对付军中骚乱的手段。定众心,诛首脑。

“难道韩冈和辽人达成了什么密约不成?不然他怎么有把握?”他突然抬头问道。

“或许真的有密约。”邢恕猛点头,“耶律乙辛的斡鲁朵本是黑山党项的牧场,他肯定是不想黑山党项重回辽境。也许韩冈正是看到了这一点……黑山党项正是给辽人卖了。”

蔡确却紧抿起嘴,眉心处的纹路更加深了。虽然话是他说的,可他还是觉得不可能。与敌私定密约那是什么罪名?!韩冈为了什么把自己名声都压上去。一旦给查出来,莫说回不了开封,就是气学也完了。

何况这一回斩首的人数也不对,黑山党项的数目在当年安置的时候早就点算清楚了,能拿到三千斩首这个数目,除非所有部族都叛乱了但这可能吗?砍了辽人充数,还要耶律乙辛帮他的忙遮掩,要是把这个猜测说出来,蔡确他能成为今年京城中最大的笑话。

不管可不可能……肯定会有人会相信韩冈的话。

而且不是少数,而是绝大多数。

京城的百姓,开封的士林,都在为河东的成就而欢欣鼓舞。

他们并不清楚三千斩首的严重性,只知道叛逆得到了应有的下场。至于犯界,难道走路还不带走错的?

士民间的欢呼一时还影响不到两府,可终究是个祸害。

“韩冈的目的,终究还是想要被召回。”蔡确慢慢的说着。

邢恕点头附和,“没错。自是如此。”

韩冈之前曾在奏章中,不经意的提到了要清理一下胜州,虽然只是简单的一句话,可掀起的风浪却不是一句话就能代表。

主要的矛盾还是在韩冈暧昧不清的想法:“他的手段,两府中有哪个想得到?他的目的,又有哪个想不到!”

韩冈的这一手是逼两府将他召回。

邢恕道:“但杀敌一千,自损八百。韩冈这一回麻烦大了。”

“万一辽国没反应怎么办?”

正如邢恕所言,既然韩冈敢于将手中的链子松开,放了折克行出去,那么他肯定有把握让辽国不会出头。说韩冈妄开边衅,但辽国到时候一点反应都没有,那不就成了一个笑话?

“辽国肯定不会有任何动作。”邢恕笑了:“难道不能栽到他身上?”

欲加之罪,何患无辞。

韩冈跟张孝杰私下里达成了协议。黑山党项原是耶律乙辛属地的旧主,他们死光了,最高兴的只会是耶律乙辛。

说是就是,说不是就不是。有分教:贼咬一口,入骨三分。何况说话的还不是贼?

“韩冈远在天边,京城内外的议论还由得他来主张?”

要是辽国有动静,那就是折克行妄开边衅,韩冈管束不严。要是辽国没有动静,那就是韩冈与辽人达成协议。怎么都能把罪名加在他头上。

两府也只需要一个理由。

……………………

从京城的流言蜚语中,开始传出了不和谐的声音。

欢呼韩冈胜利的百姓数量有增无减,但士林中,持有另一个观点的则越来越多。

这是舆论的争夺。

但要掀起舆论上的声势,就必须要有特点,要能吸引得住人。

就像后世的新闻原则,狗咬人不是新闻,人咬狗才是。如果内容不能耸人听闻,那么标题就必须耸人听闻。

所以韩冈勾结辽国的传言一时间便甚嚣尘上。

只是在韩冈的又一封奏议抵达京城后,这些流言随即不知去向,消失得无影无踪。

韩冈的新奏章上没有更进一步的内容,只有帮忙解决士兵封赏的手段。给有功之臣分配土地,然后让他们离开军队回去屯垦。

这一篇文章,几乎让人完全忘了韩冈身上还有罪名没有洗干净。

无论如何,京畿一带的军队和军属加起来少说也有三四十万,韩冈的提议事关他们的未来。这远比他跟谁谁谁勾结,要重要百倍。

当朝廷正在讨论韩冈的建议的时候,又一封奏章送抵京城不能叫奏章了,而是一封国书

辽军犯界,高丽求援。

