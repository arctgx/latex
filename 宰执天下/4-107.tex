\section{第18章 青云为履难知足(13)}

【迟了两个小时,真的很不好意思。】

韩冈奉召入宫。但他跟着前面引路的宦官,却发现领着他前往的去处,并不是在崇政殿,而是通向武英殿。

“天子是在武英殿?”韩冈在后面问着。

虽然引路时说话并不合规矩,但韩冈的问题,领头的小黄们却不敢不回答,“回直学,官家和相公们现在都在殿中。”

‘武英殿。’

韩冈点点头,觉得那个地方正合适。武英殿的偏殿,如今是放置天下九州舆图、沙盘的要地。也是当今的大宋天子寻常闲暇时,最喜欢逗留休息的地方。要议论的是兵事,当然少不了要说起如何用兵,他和章惇商议出来的用兵方略,参考地图来叙述也是最为合适的。

在殿外通名,等到殿中出声传唤,韩冈跨步走进武英殿。

十几道目光一齐看着门口,一个高大的身影先是挡住了从殿门出照进来的阳光,继而人影晃动,久违的身形踏入宰辅罗列的殿中。

吴充的眼睛一下眯了起来。

不过半年多的时间,韩冈的气质就有了很大的变化。虽然还是肩张背挺的高大身材,最多也不过是整个人变得瘦削了点,肤色也晒得黑了些。但领军破敌带来的威势,让他如同换了一个人一般。眼神、举止都是沉稳如山岳,让人完全忽略了他过于年轻的容貌。

吴充甚至有种感觉,穿着三品服章进殿来、在殿中向着天子拜倒行礼的韩冈,已经不再是靠着别出心裁的功劳屡获晋升的新进,而已经是可以与他们共论国事的重臣了——一个才二十五岁的重臣!

“韩卿快快请起。”看到每每为自己分忧解难的臣子,赵顼心中很是欢喜,“邕州之战,非卿家不可有此大功。”

“臣愧不敢当。此乃是陛下圣德庇佑,也是将士用命,更有苏缄力遏交贼兵锋两月,让交贼疲不能兴,最后为官军所破。”韩冈叹了一声,“未能救下邕州,臣有愧于陛下。”

“卿家一路南下,行程之速,世所共知。邕州城破,运数而已,非是卿家之责,卿家何须愧疚。”赵顼宽慰了两句。话头一转,便向韩冈问起他最关心的问题,“交趾妄兴大军,入我国中。荼毒三州,致使生民罹难。不知以卿家看来,可否攻入交趾,破国惩贼?”

“交趾可破!”韩冈猛然抬头,双目灼灼的看着赵顼,给了一个明确地回答:“也必须破!”

“为何?”赵顼为之惊讶。

韩冈沉声说着:“想必陛下已然得知,在臣返京之前,奉陛下圣谕驱兵破左州、忠州二蛮贼,由此驱动左右江三十六峒蛮部攻入交趾境内。”

赵顼笑着点了点头,这个消息他已经收到了,而且因为急脚递的速度比韩冈进京要快得多的缘故,他了解到了战果比韩冈还要多。

韩冈的做法完全依从着他的口谕,现在成功了,当然就是他的功劳,“此事朕已得知。左州、忠州既然不肯顺服,韩卿杀得正好!非如此,三十六峒蛮部哪里可知我皇宋的雷霆手段!”

吴充暗自冷笑了一下,他现在已经知道韩冈想说什么了。

韩冈叹了一口气:“在臣北返之前,三十六峒已破敌数十部,解救出来的中国子民已有两千余。仅仅是边境之地,就有两千百姓被掳掠为奴。那升龙府处又该有多少?皆是皇宋臣民,岂可让其久沦贼手?”

“陛下。”王韶首先出来支持韩冈,“韩冈此言说得正是,交贼不可不除,中国子民亦不可不救。”

冯京反问道:“丰州岂能留于西贼之手?”

“丰州自当平,但奈何被掳往交趾的数万生民。”吕惠卿长声一叹。

王珪则道:“邕州还有两千俘虏,可用来交换。”

“交趾敢于冒犯天威,自当予以剿平,只是西军兵力一时不能抽调,可以少待时日。”蔡挺说着。

吴充则紧跟一句:“西军不可动,但河北兵可用。有两万官军,辅以蛮部,当可平灭交趾。”

只听着两府宰执们的几句对话,韩冈也就明了了,当下的两府加起来虽不不过六人,但各自的心思则是没有一个相同。

除了夺回丰州是共同的心意,其他的问题则全然有别。是否放弃攻打交趾;对交趾是惩罚还是灭国;攻打交趾的时机;甚至为此调动的兵马的数量和来源,全都不一样。

就如吴充,他也说要攻取交趾。但他的提议,让韩冈不能不怀疑他的目的。调动河北兵?而且才两万!身为枢密使不可能不知道河北军的情况,从河北调兵南下,他是打算平交趾,还是想看朝廷大军的失败?

韩冈还没有走进宫城的时候,就已经预料到了今天殿上的形势。在王安石无法到场的情况下,自己在殿中宰辅们中肯定是得不到多少支持,除了王韶之外,就算是吕惠卿多半都是心怀鬼胎。

赵顼已经是听够了两府中的争执,眼神中带着些许怒意。但他不好发作,只能改而问着韩冈:“不知韩卿对此有何方略?”

韩冈点头,“臣确有一份方略,只是需要沙盘来解说。”

赵顼一听,立刻就有了兴趣,忙带着群臣起身,从正殿转往偏殿。

吴充半眯着眼睛,心里揣摩着韩冈到底会如何来说服天子。不过他不可能成功,只要丰州的问题还在,关西的兵就动不了,韩冈即便想出兵,也只能用河北的军队。

跟随者天子脚步的冯京在心中冷笑着,眼下丰州的问题更重要,而交趾得放在第二。章惇、韩冈想要立功劳,就必须在调兵的问题上进行退让。作为宰相,冯京比世人都要清楚邕州大战的内情,他绝不相信,韩冈还能再复制出在邕州创造出来的‘奇迹’。

来到偏殿之中,里面还是如同旧时一般,放满了沙盘。现如今,放在正中位置的几副沙盘,一幅九州地形图,另一幅是关西地形,而剩下的两幅都与广西有关。

赵顼站在其中一幅八尺见方的沙盘上首,而宰辅们则按照班列立于两侧。君臣们的视线一起汇聚到韩冈的身上,等待着他说出自己的方略来。

下首位置的韩冈看了沙盘两眼,没有开口说正事,而是惊讶的抬头问着:“此乃何物?”

“此是两广地形沙盘,也是宫中有关两广的舆图沙盘中最精细的一幅。”

韩冈眉头皱得很紧,“这沙盘看着精细,但未免偏差得太远。琼崖不过一海岛,划得也太大了,雷州、琼州以西到交趾之间的珠母海又怎么这么小?还有邕州、钦州、廉州之间的道路,全部都错了,左右江三十六峒羁縻州的位置也都错了大半。”

韩冈在殿中看到的两广沙盘,虽然不如他说得这么夸张,但也的确是有些问题。并不是说雕工不好。宫廷之中,绝不会缺上等的工匠。整幅沙盘雕琢的很精细,山峦河流清晰可辨,连一座座城池,都用坚硬的木头细细的雕刻出来。但这副沙盘与韩冈来自后世的记忆差别有些大,连广东、广西,再到交趾的这一条千里海岸线也变了形。

而旁边还有一幅广西地形沙盘,是新近制作而成,结合许多地图、记录来打造。不过在韩冈眼中,也只能说大体的东西南北没有错,河流山脉的位置没有颠倒,除此以外的细节到处都是错漏。不过韩冈的本意并不在指出错误。

“这两幅沙盘是依从广南两路的舆图打造,如何会有多少错处?”

“敢问相公,从广州到琼州的水程多少?从琼州到钦州又是多少?从雷州至琼崖,有多远水路。左右江汇合之后为郁水,为何不从最近的钦州廉州入海,而直入千里之外的广州?又为何交趾军能分两路进兵?”

韩冈一系列的质问,让冯京脸色阴沉了下来。他哪里能看不出,韩冈这是在先声夺人。表现出一副行家里手的态度,只要在天子心中建立起这幅形象,就能将自己的观点灌输给天子。

韩冈正是这个打算。

所以他一开口,就是直接指出了沙盘中的错误,这一套招数虽说老套,但使用起来却很有效果。在他提出异议后,赵顼立刻招来了主管沙盘制作的官员,熟悉的面孔很快出现在韩冈眼前。

……………………

两名内侍抬着个装满了蜜蜡的木箱,走进了武英殿的偏殿之中。

在殿外的时候,两名小黄门还哼哧哼哧的喘着粗气,但一进殿中,就立刻变得连大气也不敢出,轻手轻脚的将还冒着热气的蜜蜡放了下来。

“送来了?”就在大殿中央的一人回头,看见了热腾腾的蜜蜡,立刻吩咐着身边的人:“快抬上来。”

当今的大宋天子,宰相、参政、枢密使,还有如今最年轻的一路转运使此时都在偏殿之中。可位于殿堂中心位置的并不是他们,而是方才出声的穿着青袍的官员,还有他的几名得力部下。

虽然身穿官服,但双手满是厚厚的老茧,短短的手指指节粗大,黝黑的脸上也都是皱纹,看着就是一副工匠模样。他正指挥着手底下的匠人,在一幅六尺见方的台桌上,用混着木屑的蜜蜡,堆砌雕琢起一幅地形沙盘过来。

“雷州三面环海,直拖向南,是个半岛,南北略长,东西则并不宽。琼崖岛形如掌心,中央是山,沿海则为缓地。朱崖最南,而琼州在北端。”

韩冈就站在沙盘旁边,不时出言指点着,告诉工匠们哪边是山,哪边是海。洪亮的声音中充满自信,地理专家的气派摆得十足,就算他的叙述有错,也没人能指正的出来。

“想不到韩卿连广东的地理都了如指掌。”赵顼在旁说着。

韩冈躬了躬腰,微笑道:“既然要对交趾用兵,两广、海中和交趾的地理,臣不敢不悉心探问。不过在广西缺乏名家,无法制作合用的沙盘。多亏有了田将作,才能将臣了解到的地理地形拟制出来。”

