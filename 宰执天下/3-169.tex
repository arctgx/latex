\section{第44章 岂惧足履霜(中)}

政事堂位于皇城西南角,与西面的枢密院相对而置。故而一为东府,一为西府。

论起建筑并无多少出奇的地方,既不如宫中诸殿的宏伟,也不如禁中楼阁的秀美,甚至都远不远比不上皇城之外,飞桥如虹、五楼勾连的樊楼。

但这座有二十余座楼阁组成的建筑群,就是大宋不可或缺的中枢。天子不过一人而已,勤政纵如祖龙,一天下来也不过批阅数担尺牍。而每天呈送到中书门下的公文,又何啻千万?!没有群臣襄助,天子根本治理不了幅员万里的国家。

从参知政事的公厅望出去,窗外的梧桐光秃秃的,不见一片绿叶。梧桐之后,就是一堵院墙,多年未有整修。墙面上的石灰早掉光了,透出了内里砖石的斑驳。与其说有着古意,还不如说是残破。

这座院子的景致,甚至不及中书都检正所在的公厅,那座院落中尚有几支腊梅,此时当是已经临风绽放。

但高处的风景就是不一样。

吕惠卿尚记得在乡里时,他往往喜欢登上乡中的后山。对人说性喜山水,但吕惠卿真正喜欢的,还是站在高处向下俯视的畅快。立于山岩之上,村落人居,城池河流,尽收眼底。

如今他已经站在参政之位上,俯视天下群臣、亿万生民。张起清凉伞,这样的畅快即便金榜题名也是难以比拟。就不知坐在宰相之位上,又会是一种什么样的感觉。

收回视线,来此干谒的官员犹在絮絮叨叨,可说了一大通废话,却半点也不见说到正题上。问着他任官当地的风土民情,又是张口结舌,驴头不对马嘴。

吕惠卿心中大感不耐,此辈何堪使唤?说了句点汤,便下了逐客令。

点汤送客,吕惠卿起身将其送到厅门前——过往宰相迎客送客,都只是从交椅上站起来就足够了,而执政也只须多送两步。但到了富弼为相之时,却都是殷勤的送到门前。富弼此举,在士林中大受好评,之后便沿袭下来,如今已经成了定例。

今天按照定数需要接见的官员,这是最后一位

吕惠卿坐回来,看着衙中小吏上来将杯盏给撤去,看看时间,已经是黄昏,暮鼓很快就要敲响。今日并非他值日,吕惠卿准备收拾一下就回家去。今晚在家里,还有些官员、士子要见。在家中接见的客人,可不像方才的那一位,是依照制度被安排上来干谒宰执的官员,而是吕惠卿真正有心招揽驱用的。

正亲自收拾着要带回去的文案,就见自己的弟弟吕升卿走了进来。

今天是吕升卿侍奉天子经筵的日子,吕惠卿一见到他,便当头问道:“今天经筵上,天子可说了什么?”

“……倒也没什么大不了的。”吕升卿试图将问题糊弄过去。

吕惠卿了然一笑,必然是又被天子给问住,没有及时回答,靠了沈季长帮忙。见着弟弟脸上的尴尬,吕惠卿暗叹了一口气。缺乏捷才那还真是没有办法,并不是答不出,而是一时想不及。

吕升卿干笑了两声,转头看着外面,“方才出去的那矮个儿的京官可是来干谒的?怎么见他骂着出了院去。”

“是吗?”吕惠卿随即提起笔,在桌上名单的最后斜斜一划,将一人的姓名给勾去。怨望,不论是天子还是宰执,他们都不希望看到与这两个字沾边的官员。

见到了吕惠卿笔杆的动作,吕升卿犹豫着,“不须如此吧……”

“此辈庸碌短浅,何堪驱使?空食俸禄,尚不及乡里一俗吏。”吕惠卿丝毫瞧不起这一干庸人。

吕升卿也不会为此与兄长争辩,坐了下来:“外面现在正热闹着,方才就见着后妃去大相国寺祈福回来。太皇太后的病情,看来当真有些不妙。”

“几天前天子招了智缘入宫,开了几剂汤药,到现在也不见有什么效用。不过太皇太后自有神佛庇佑,倒不必太过担心。”吕惠卿心口如一,他自己当真是一点也不担心。若没了太皇太后,宫中便是又少一掣肘,反而是桩喜事。

“不过太皇太后已然年近花甲,身子骨的确是一日弱过一日。说不得过几年,内宫之主要换成保慈宫了。”

“此事勿要多言,自随它去。”

即便换成脾气倔强的高太后主持后宫,吕惠卿也无所畏惧。如今的这位皇帝为人纯孝,不过在祖母和生母之间,却是与太皇太后更为亲近。太皇太后加皇太后都没有动摇到天子坚持变法的心意,若只剩高太后一人,如何还能做到?除非天子寿数不及其母,接位的新帝又是年幼,否则完全可以高枕无忧。

见吕惠卿不想提及太皇太后的事,吕升卿便道:“对了,方才在讲筵上,天子还提到了韩冈的《浮力追源》,问着我有没有听过。不过是刚刚出炉的新论,这几日竟然一下子就传播开,连天子都听说了。”

“韩冈在京中已经颇有些名气,他的新论传扬快一点很正常。”吕惠卿问着弟弟,“你是怎么答的?”

吕升卿咳嗽了一声,道:“似有几分道理在。沈季长则说,韩冈与经义大道无涉,只是在说着寻常事。”

“天子的反应呢?”

“什么都没再说了,应该不是很放在心上”吕升卿道,“若天子当真对此事很在意,何不将韩冈招进宫去询问?”

吕惠卿摇了摇头,“是韩冈并没有申请入宫奏对,而不是天子无意。天子的确打算招韩冈入宫详询,但今日被冯京抢先撺掇了两句,反而让天子打消了主意。”

“怎么?!冯当世竟然没有说韩冈的不是?!”吕升卿惊讶的说道。

“他敢再说韩冈什么?不见杨绘的前车之鉴?”吕惠卿冷哼着,“现如今提起杨绘,京城里面都是把他当笑话,这辈子都不一定有脸再入朝为官。何况韩玉昆说得的确有几分道理在,不涉经义,却是合着自然之道。沈季长说的话,天子肯定没听进去。”

吕升卿的疑惑还没有得到解释,“但冯京为什么撺掇天子招韩冈入宫询问?”

“铁船哪有那么好造的?虽说韩冈将道理公诸于众,自有一番成算,但他的成算,却不一定能压得住悠悠众口。要造出铁船,不是那么容易。可有哪家的工匠有此经验?又有哪家的工匠能打造出如同船板大小的铁板?铁船下水后,生锈了怎么办?太沉重了无法行驶又该怎么办?而且一艘铁船又要花多少钱?比之木舟又如何?”

一句句质疑说出口,吕惠卿喝了口茶水,润了润喉咙,“《浮力追源》中也只说了金铁之物浮于水上的道理,可没说能让铁不生锈,也没说过铁船可以在水上飞速而行,更没说过铁船价廉。如果仅仅是能浮水的榔槺笨重之物,单是无用二字,韩冈一番辛苦都将白费。”

吕升卿皱着眉,他的兄长说了这么多,可他还是没想透这跟天子不召见韩冈有什么关系,冯京又是有着什么图谋。

吕惠卿看了弟弟满脸的疑惑不解,叹气之后继续解释,“现在韩冈只是拿出了浮力之论,没有明说能造出铁船,也就是一切未定。即便他失了手,也不过是多个笑话而已。但如果在君前开了口,说了铁船之事。一旦不能成功,那又会是什么罪名?”说着,他冷然一笑,“天子不纳冯京之言,当已是看透了他的为人了……明示忠朴,暗怀诡诈!”

“那大哥你究竟打算怎么做?”

“当然全力支持,若铁船当真有用,水战上倒能用得着。”

做过判军器监的吕惠卿最为清楚,打造铁船这等大事,不是简简单单就能成功的,他并不认为韩冈在冶铁和打造,能胜过浸淫几十年的工匠。即便自己全力支持,不让军器监中设置障碍,没个一年半载,很难见到成果。

可话说回来,若是当真看到铁船在汴河上跑,肯定会轰动整个开封城。

家里的瓷碗浮在水上,没人会注意。铜盆、铁锅都能在水上漂着,也没人仔细想过到底。韩冈的设想别出心裁,造出的铁船即便没有多少实际的用途,也能证明他对格物致知四个字的创见乃是符合大道,推广起气学来,当能事半功倍。

只是……以韩冈为人才智,当真有这么简单吗?

尊师重道四个字,韩冈早已是坐实了。雪地里站着程家门口一个多时辰。为了推重张载,而跟做宰相的岳父翻脸。如今又放弃了在中书中的优差,而硬是抢下了军器监,就是为了推广横渠气学。说起韩冈在尊师这方面的品行,人人都要竖起大拇指。

可吕惠卿总觉得有哪里不对劲。看着章惇这些日子并没有多提及此事,想必他的心中也有所疑惑。

如果将期望全然放在铁船之上,实在太不符合韩冈行事周密面面俱到的一贯作风。但要说韩冈别有计划,却又想不出来。

他究竟是打的什么盘算?吕惠卿百思难解。

