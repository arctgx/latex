\section{第一章 坐忘渭水岸(下)}

虽然韩冈安排了许多,却也不过是布局而已。他现在绝大多数时间还是坐在家中静心读书,准备到八月初的时候,启程前往秦州。

相对于儿子的两耳不闻窗外事,一心只读圣贤书。韩千六就忙碌了许多。他如今是早出晚归,麦田现在虽然已经收割完毕,但同样重要的棉田却快到了收获的时节。

今年扩种的二十顷棉田即将成熟,而棉花专用的纺机和织机,也在河州打得热火朝天的时候送到了陇西,现在十几个高明的木匠正在加班加点的仿制中。今年眼见着就能出布,秦州几大商号的东家现在不是准备着亲自到场,就是已经派了族中最亲信的兄弟子侄前来查看。

两千亩棉田,足以收获数万斤皮棉,全数织成棉布来贩卖,纯利润同样得以万来计数。种植棉花的利润如此之高,没有谁不垂涎三尺?这二十顷地,王、高、韩三家都有份,还有秦州的几家豪族,几家一分,就是数千贯的收入。

人人都知道,单是巩州就还有上千顷荒地没有分配。而熙州洮水的干流和支流河谷,其中肥沃而无人开垦的河谷地,更是足有五六千顷之多。想想棉布现在的利润,只要能将两州的荒田开垦出其中十分之一,并种上棉花,那就是上百万贯的获利。而以整个大宋的富贵人家对布料的需求,区区几十万匹的数量,最多也只会让贩卖的价格打个九折而已。

自然……那只是美好的前景。不过看到眼下收获在望的千亩棉田,又有哪人能忍耐得住?

高遵裕就先一步出动,到城外看了一眼棉田之后,就没有半点犹豫的亲自跑来找韩冈父子。

大宋社会商业发达,官员们当然也脱不了被世间的风气所影响。曾公亮、冯京、郭逵,都是有名的精于货殖之术,陕西、河北的边境守将,更是不会浪费优越的地理条件。官员借用官船运送私贩的货物十分常见,苏轼就曾经被栽了一个利用官船贩运私盐的罪名,就是因为查不胜查,最后不了了之。

自从陇西开始设立榷场,以王韶、高遵裕和韩冈为后台的三家商行,就垄断了榷场中的大半民间交易,三家都是因此赚足了钱钞。现在高遵裕跑来商量赚钱的买卖,当然不是什么让人羞愧的事。

高遵裕进门后,行过礼,便惯熟的大剌剌的坐下,直接对韩冈道:“本不该打扰玉昆,不过这事还得劳动你拿个章程出来。”

“我那表弟也是高家的女婿,总管这么说那就是见外了。”

韩冈看了父亲一眼,韩千六便连忙点头,“三哥说的是,总管太见外了。”

冯从义娶得是高家的远支。韩冈跟高遵裕定下来的亲事,不是官场上的媾和,而是为了维系韩、高两家在巩州的利益。高家是皇亲国戚,不论到了什么地方那都是跟强龙一般。而靠着韩冈,韩家在巩州更是已经成为了地头蛇。利润最大的蕃货转卖,蕃人们都要看着韩冈的面子。

高遵裕走了,高家和王家的商号也许还能吃得开,但控制权就不会像现在这么稳当。而韩冈离开,在巩州还有韩千六看着,又有陇西疗养院为蕃部贵人们治疗伤病,人脉关系不断被加强,怎么看都不会丢了主控之权。

而且韩冈在水面下的影响力,高遵裕隐隐约约也知道一点。广锐军对韩冈感恩戴德,说不定招招手就能出来一群死士。但想要拿此事出来攻击韩冈,却是捕风捉影,不可能找到实证。前日韩冈将广锐军送上的贺礼,转捐给正在建设中的县学,说是划清关系也无不可——真实的内情不是外人可以了解。

不过现在重要的还是棉田一事。

韩冈父子两人的表态,让高遵裕满意的点头,“这群饿狗,前两年求着他们来陇西,没一个肯来的。现在看到棉田有出息了,却涌过来摘桃子。官府的地,不能就这么轻易的给人,玉昆你说该如何是好?”

韩冈暗自冷笑,他都是锁厅的人了,身上的差遣早就卸掉。而高遵裕如今掌控熙河全局,真要不给人分派荒地,只是他一句话的事而已。

不过朝廷对于迁移到边地种田的人家,一直都是持着鼓励的态度,也有正式的公文。为了充实边地人口,甚至还下令南方各路,如果有当流三千里的罪犯,那就都发配到熙河路来。高遵裕如果阻止秦州的豪门进场分一杯羹,转头就会被捅到京中去。事情闹大了,太后的面子也别想压下去。

所以高遵裕来找韩冈,就是希望在不给人抓到把柄的情况下,堵上外人分大饼的道路。要韩冈为此出个主意。

但韩冈他可是要把熙河的棉花产业给做大做强,恨不得外人来得越多越好,不可能支持高遵裕意欲独吞的行为,“棉田不是这么好开垦的。别看家父种得容易,棉田势头长得好。其实论起田垄之事,能比得上家父的不多。先放人进来,亏上几家再说。”

“真的有那么难种?……他们学着来总会吧?”

“当然也不能让他们轻易的得了官中的土地去。天子想看到的是熙河人畜兴旺的样子,因而才会同意在路中屯田。分田都要有户口入籍。总管若是下令,新来人户分到的土地撂荒超过三分之一,就立刻予以没收,应该没人能说不是……这是逼着秦州的那些人不能分占太多。”

‘这还差不多。’高遵裕点点头,“这个主意好。”只是他又愁起来,“但我们几家怎么办?”

不许撂荒,那高家、韩家又能分到多少土地?只要他们一离开,就不能再借用厢军来代为种地,到时候土地肯定要撂荒不少。

“不用分地,可以租种官田嘛。能扩大官田的数量,天子也会乐见。为租种的官田借用一下厢军,就根本算不得什么。人手足就多租点,人手少就少租点。能将定例的税赋交上去,租多租少谁会管?只要不拖欠租税,就算会遭人眼红,但又有谁敢虎口夺食?”

韩冈一直都担心高家因为高遵裕不能在熙河久留,照顾不到太多的产业,会渐渐减少对巩州的关注。虽然韩冈不能改变朝廷的条令,至少还可以钻个空子。只要能稳定的租种官田,如此一来,高家肯定是要在巩州扎根了。若是高家的利益能稳定在巩州,那么韩家产业的安全也当能得到保证了。

高遵裕从韩冈这边得了建议,虽不是很满意,可也算勉强过得去了。心情好了一点,喝了一口酸梅凉汤,漫不经意的问着:“近日听说王介甫托了子纯来向玉昆你提亲,可有此事?”

王安石请王韶代为提亲一事,韩冈没有跟父母说,高遵裕突然间问起来,韩千六一听就吓了一大跳。“三哥!这事怎么没听你说?!真的假的?!”

“确有其事。”韩冈点着头,心如电转。暗道肯定不是王韶、王厚那边传出来的,两人口风紧,知道的人又少。倒是京城那边,瞒不住事,多半是谁说走了嘴,传到了高遵裕这边。

“那可要恭喜玉昆你了。”高遵裕笑着拱了拱手,就是不知笑容中有几分真情实意。

韩冈连忙摇着头:“不过那是处道来探口风,学士并没有正式出面。”他转对韩千六道,“既然如此,这事孩儿就觉得没必要多提。”

“怎么,宰相家的女婿都不放在眼里?”高遵裕半眯起眼睛,似是吃惊的问着。

“倒也不是,只是想着过上一阵再决定。下官的心思现在都现在举试上,一时无心于此。婚姻大事,还是等到中了进士后再说。”

“……是不想事后遭人议论?”

“也有这番考量在。”韩冈点了点头,宰相家的准女婿若是考中了进士,肯定会有好事者将卷子弄出来。文人多相忌,是不能指望从他们嘴里听到好话的。

“还是玉昆你想得周全。”

高遵裕倒也是想把韩冈招做高家的女婿。但他也清楚,韩冈绝不可能答应。跟宰相家结亲,能借用岳家积累下来的人脉,只要才能不差,日后高官显宦是轻而易举。但与皇亲联姻,对有心宰执的官员来说,却是他们上升时的阻力,而绝不是助力。能靠着冯从义与韩冈拉上关系,已经很难得了。

没了正事,说了两句闲话,高遵裕就想告辞,总不能太耽搁韩冈的功课。而这时高府那边却来了人,急急忙忙的跟高遵裕禀报,说是他的小妾七娘明珠已经要生了。

韩冈在旁听了,连忙起身拱手致意,“韩冈恭喜总管。”

“等生下来再说。”高遵裕虽然有儿有女,孙子都快有了,但多子多福,儿女多了从不是坏事。他哈哈笑着,“过两日,可就是要恭喜玉昆你了。”

不过高遵裕的好心情没有保持多久,到了第二天清早,高遵裕就急匆匆地遣人来找韩冈。高家的总管火烧火燎,对韩冈道,“韩官人,七娘子难产,总管要小人来讨个救命的方儿!”

韩冈怔了一怔,‘这是病急乱投医啊!’

