\section{第20章 土中骨石千载迷(五)}

听到韩冈从安阳掘出了商人占卜甲骨的消息,蔡京整整愣了有半刻钟之久。

当他回过神来的时候,发现整个御史台的气氛都变了样,御史们一个个仿佛有人欠钱不还的阴沉着脸,默不吭声,连带着让胥吏们也都屏声静气,蹑手蹑脚走路如同做贼。

政事堂的一个书办捧着一沓子公文奉命来御史台,甫进门就被森森阴气激了一个机灵,连脚步都迈不开了。

“这是咋的了,”他挪着步子凑近了门房后的司阍,压低声线问道:“又是被谁招了惹了,怎么连树上的乌鸦都不叫了?”

“谁他娘的知道。”司阍离得远,现在也是一头雾水,不敢往里面去问,却不忘提醒经常一起喝酒的朋友,“小心一点,别犯到刀口上。”

书办干咽了口唾沫,心中发慌。不知道是现在送了文书,还是过一阵子再来的好。一时便在门前进退两难起来。

蔡京没去注意门前的那点小插曲,他只顾看身边的张商英。领头打击气学一脉的张御史面色灰败,神经质的用牙齿咬着下唇,出了血都没察觉。

这般阴郁的气氛,似乎是在台中传说里,当年王安石为推行新法清洗政事堂时才有的情况。

‘消息传得还真快。’蔡京心里想着,才多点功夫,御史台中似乎每个人都听说了韩冈开始反击的消息。

半个时辰前,还有几个新晋御史正摩拳擦掌的准备拿私习天文的罪名,借着千里镜禁令这个东风,向几个被子弟连累的侍制以上的高官开刀,现在就不见人言语了。

蔡京往西厅张望了一下,也是一片沉寂。

可怜何正臣,当年曾经上表弹劾时任京西转运使的韩冈,但被不得不安抚韩冈的皇帝赶出了京城。半个月前刚刚被调回御史台,本想着报仇雪恨呢,但韩冈的这一下子,满腔心愿又成了泡影了。

新学刚刚藉由千里镜禁令对韩冈展开反击,孰料当即便被韩冈反手一剑。接下来,新学免不了要陷入了困局之中。

‘只是这么做未免太离谱了吧!’

蔡京暗地里抱怨着,心中却是五味杂陈。他知道,包括自己在内,任何人事先都没可能想象得到,韩冈竟然可以用上这种手段来。如同天外飞来一剑,一举将《字说》的根基给斩断。

一直以来,蔡京都不认为自己会比韩冈和韩琦这样的人差到哪里,只要有机会,他肯定能做得更好。但今天看来,支持这样的自信所需要的能力,终究还是比韩冈缺了一点。

道统、兵法、医术什么的,蔡京没兴趣跟韩冈比,自家在这方面有缺陷,蔡京本人也是清楚的。而在其他方面,比如诗词文章……或许还要包括书法,他都比韩冈要强,但蔡京也没兴趣去比。

这些都是末节,身在庙堂之中,要比就该比做官。蔡京相信自己迟早能赶上韩冈,最多也就一二十年而已。眼下都已经做到了监察御史,蔡京确信自己迟早能够在两府之中得到一把交椅。

只是当今天韩冈抛出了殷墟遗物,给蔡京的感觉,就完全不一样了。

光是为了道统之争,从韩冈回京后便挑起战火,气学、新学两家一番拆招应招,这两个月已经是撕破了脸皮。本以为借用对千里镜的禁令能一举将气学压倒,孰料韩冈辣手无情,让人措手不及。

别人看到的是殷商时所用的文字,让刚刚写出《字说》的王安石有苦难言,可蔡京看到的则是韩冈手段和心计,以及能狠下心来的决断。

从天子到朝臣——或许里面还要囊括进韩冈的岳父——这一次在韩冈手上折戟沉沙的不在少数。

或许从请求编纂《药典》的时候开始,韩冈他就已经在做准备了。之前以生物分类学的名义对螟蛉之子、腐草化萤的否定,而带出的对《诗经》和《礼记》注疏的攻击,完全是试探用的斥候,抛出来的棋子。他真正的目的和手段,如同剥丝抽茧,在新学一脉开始反击之后,才一步一步的表露出来。

 所谓相州龙骨,韩冈也定然是早就攥在手中,就如种痘法一般,到了合适的时候才抛出来。就像埋伏在山谷两侧的军队,耐着性子,等待敌人走入陷阱,而后一击致命。

——如果不是这样,而当真是在搜集药材的过程中,来自相州的甲骨落到韩冈面前,那他的气运未免就太骇人了。若是韩冈当真集气运于一身,那蔡京还真得远避为宜。

片刻之后,御史中丞李定尚未回来,作为台副的侍御史知杂事也没有回来。七八名侍御史、御史和御史里行则是济济一堂,难得在一起共议时事。但坐在一起之后,几人不是眼观鼻鼻观心的正襟危坐,就是你看我我看你,反正是一句话也不开口,做起了佛像。

终于有一个愣头青的新晋御史:“什么殷墟甲骨,定然是韩冈伪造!”

蔡京摇头。以韩冈的头脑,怎么可能会做出这种蠢事?一旦他当真这么做了,被拆穿后,气学可就完了。

但有了一人起头,便开始有更多人说话了。另一位御史则道:“假应该是假不了,但韩冈使人发掘殷墟,这条罪名他可洗不脱,可依盗墓律深究。”

这分明又是一个说蠢话的,虽然他否决前面一个更加愚蠢的说法,但他的话也只是好了那么一点而已。

蔡京暗暗摇头,左右看看,张商英和何正臣的脸色依然如同冻结了一般,完全没有松弛下来的迹象。

‘倒还没糊涂。’蔡京想着。

韩冈会去做摸金校尉和发丘中郎将?他早把自己从浑水里摘出来,洗得干干净净,清清白白。

韩冈派去的人收购的是药材,相州百姓将龙骨从地里挖出来的也是当做药材用的。几天之前,除了韩冈之外,没人知道龙骨。不知者不罪,而韩冈他是保护了殷人遗物,要不然还不知有多少商人的占卜甲骨会落人肚子里去。

想将罪名安到他身上,先想想弹章得怎么上才能说服天子?不然韩冈一个反扑,运气不好的人就又要出外去监酒税了。

“弹章上的罪名真的能这么写吗?”有人质疑道,“相州百姓有人会出来作证吗?”

用重利引诱,或是严刑拷打,或许能弄来几人,正常是不可能的。但这么做的结果,依然动不了韩冈。

韩冈的声望在民间有多高,出去走走就知道了。当真以为他的牛痘,是白白拿出来的?天下多少人感激他,不仅仅是普通百姓,就是皇宫之中,除了三两人之外,感激他可是满宫城都是。

当然这样的人望,对人臣来说,并不是好事,蔡京觉得韩冈迟早会毁在这上面。但眼下离那个时候还早得很,在韩冈作法自毙之前,与其为敌的一干人等,得倒霉吃亏。

韩冈被御史们盯上不是一天两天了,但每次失败受苦的都不是他。御史台中想跟韩冈过不去的御史,少说也有一半,但张商英上奏禁千里镜,连个韩字都敢没沾。

一群人议论了半天,到了放衙的时候,还是没有个结果。最后的决定是挪个地方再议,张商英掏钱请客,愿意去捧场的有四五人。

一名名让朝臣们闻风丧胆的铁面御史从小厅中鱼贯而出,张商英要出厅门的时候停了一下,回头问拖在后面的蔡京道:“元长来不来?”

蔡京拱手一礼:“承蒙天觉厚爱,设宴相邀。不过家里方才遣人来通报,说家里有些事要蔡京尽快回去处理。今天的宴席蔡京就不去了。”

张商英点了点头,“那就请元长代商英问候元度一声。”

蔡京行了一礼,以示回应。表字元度的蔡卞是新学的中坚,蔡京可是要早点回去与他的这个堂弟好生议论一番。

蔡卞在国子监中,到了回家的时候,崇政殿和政事堂中的消息还没有传到他耳中。听到蔡京的转述,蔡卞的脸色阴晴不定了好一阵,最后抬头,眼神冷硬:“这算什么事,有什么好在意的,不去理他就是了。”

“视而不见可不是良策,有的是人提醒。你可知道,王禹玉今天可是将韩冈送到了院中。”看到蔡卞脸色又阴郁了几分,蔡京又道:“‘字者,始于一二,而生生至于无穷。如母之字子,故谓之字。’这是介甫相公的原话。‘秦烧《诗》《书》,杀学士,而于是时始变古而为隶。’这也是介甫相公所说。韩冈说字有源流,当追溯上古,本意是相通的。而殷墟之文便是货真价实的古文,更接近于源流,这一关怎么绕过去?”

蔡卞咬着牙,过了好一阵:“殷墟甲骨文字到底作何解,有三馆和国子监在这里,论得到他人说话!”

“这正落入韩冈彀中,如此一来,新学一脉可就更是众矢之的,没人会放过这个机会的。”蔡京摇头,知道蔡卞也没一个合用的招式。

丢下一块鲜肉,引来一群饿狗,新学如今最大的问题就是敌人太多了。

王安石借助天子想要一道德同风俗的心理,将新学打造成官学,在儒林中也是开罪了绝大多数的儒者。

他做字说,说是要重光先王之道。韩冈现在将先王之道从地里面掘出来了,就算新学一脉想视而不见,天下群儒也会群起而攻之。可不独是气学!

苏轼不是喜欢杜撰,这下有了新目标了。司马光重史,殷墟中的金石甲骨,想来他也不会放过。洛阳二程纵然跟气学正闹着,但在推翻新学上,两家可是有着共同的心愿。司马光和苏轼也必然少不了同样的心愿。

殷墟遗物是武器,也是工具。当今儒者一向是以己为主,连六经都能利用,何况殷人之物?拿着甲骨文,一个字一个字的挑着《字说》里面的错,谁会管他是不是正解?

接下来数年甚至十数年,儒林中的局面,蔡京可以想象得到。将会是一片乱象,诸多学派先将新学从官学上拉下来,然后互相之间再一通乱打。

别想有一个安生了!

