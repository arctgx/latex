\section{第18章 青云为履难知足(18)}

【一觉睡了十二小时,写得也迟了。不过该补的不会少,十二点还有一更。】

韩冈从宫中回到王安石府上的时候,远远地就看见了吴安持送了几人出了门来,竟然还在接待着宾客。

虽说是到了晚间,但来王安石府上吊丧的客人依然络绎不绝。车马拥堵在相府的大门口,比起平日里来拜访王安石的官员塞门的情况,甚至还要严重。

只是其中有几个是真心来悼念王雱,又有多少是因为王安石的身份,也根本不需要多想。

嫌着麻烦,韩冈没有走拥挤不堪的正门,而是转了个方向,从侧门进了相府。

因为王家的亲友都来上门的缘故,此时的相府之中,比平时热闹的十倍。不仅是一家老小都到场,也带来了大批的随从仆役,虽然都因为身在宰相府邸中,不敢有所放肆,但人数一多,怎么看都有些乱。

韩冈从侧门进来,门后就是偌大的用来停放车马的庭院。只是过来祭拜一下就走的官员,他们的车马都停在外面的街上,而要逗留一段时间的,则是将车马都停在了偏门内的庭院中。

院中被车马占了大半,还有更多的仆役,只是比起白天的时候少了些。

车夫、马夫们许多都是席地而坐,闲极无聊的聊着天。不过他们交谈时,还知道尽量压低声音。就是听到了句笑话,在笑起来的同时就连忙捂住自己的嘴,不敢笑出声来。

只是乱糟糟的样子还是显得缺乏秩序。唯有一队陌生的元随,站在院中一角。人数还不少,几乎是执政数量,只是还不到宰相的规格,皆是静静的不言不语。他们所在的那一个角落,与院中的其他地方有着截然不同的对比。

“是两府中的哪一位?”韩冈问着迎上来的王家家丁。

“回姑爷的话,是吕参政来了。”

韩冈暗道自己是糊涂了,自己是当真糊涂了。王雱的丧报一出,吕惠卿的确是应当到场的。虽然他现在与王安石疏远了,但以吕惠卿与王安石旧日的关系,这第一天就必须来的。

吕惠卿一贯的治家严谨,在朝堂内外也算是有名的。治家如治军,也难怪他门下仆役的气象与他人家中截然不同。

“相公现下就在书房中,跟吕参政说着话,”家丁讨好的又问着,“姑爷要不要去书房一趟?”

韩冈摇摇头。王安石正在接待吕惠卿,他去凑哪门子热闹。

低头看看自己身上紫袍犀带,韩冈道:“先得去换身衣服。”

韩冈一句吩咐,王家家丁连忙小跑着进去,帮韩冈去取素服。

韩冈脱下了觐见天子的公服,换回了素服,就直接往外厅的灵堂过去。唱经的声音充斥在耳间,和尚道士被来百八十人,就在外面的灵棚中招魂忏经,而智缘、愿成等京中赫赫有名的紫衣大师则被招待进了内厅。

不过除了做道场的僧道外,外间的人的确是少了。关系略远一点、不需要守灵的亲戚回去了许多。关系疏远、却没有离开的,则基本上都是抱着另外一番心思。

韩冈的父母并不需要为儿子的大舅子守灵。往灵堂走的时候,韩冈顺便找人来问了一问,他的父母果然是见了天色将晚,就先告辞回家去了。

但韩冈走不得,他需要为王雱守灵。

灵堂中烟雾缭绕,缕缕香烟绕着一条条垂下来的白布,渐渐散在空中。

王雱的儿子还守在灵堂内,王旁在旁往火盆中添着纸。王安石兄弟家的王旉、王旊、王斻、王防、王旗等子侄也都在;王家的孙辈,还有韩冈和吴安持的儿子,也是同样一起在旁陪着。

韩冈进来时,灵堂中的人都站了起来。各自上来行礼,王旁疲惫的抬起了眼:“玉昆,回来了?”

韩冈告了声罪,“耽搁一些时间,这时候才回来。”转身先给王雱上了香,添了纸。

可能是王安石兄弟几个用尽了王家的气运,王雱的堂兄弟们都算不上出色。不过在为王雱守灵时,倒是诚心实意,就是在灵堂中久了,各自都有些疲色,

韩冈看看自己的儿子,韩钟、韩钲,两个小子现在还精神得很。就是年纪太小,到了累的时候,也熬不了夜。不过他们也不需要守上一夜,没甚关系。

就是王雱的长子王栴,才六七岁的小孩子在乌烟瘴气的灵堂中跪了一日,进来一名前来吊祭的宾客,还要叩拜还礼。中途只有短暂的时间用来吃饭、方便,脸色已经很不好了,再守上一夜,保不准要出事。

王雱就留下这一个儿子,又是王安石的冢孙,一直以来身体不好。如果王雱还在的话,肯定不会让儿子这般吃苦,但躺在灵堂中的王雱不可能再起来说话。

“栴哥儿可能快吃不住了,等人少的时候就让他去歇一歇吧。”韩冈拉过来王旁。

王旁看了侄儿一眼,一张小脸的确是泛着病态的青白,一点血色都没有了。但让他放弃守灵去休息,这可是不孝。一旦传出去,小孩子不会受到责难,但过世的王雱可要会被人说教子无方,贻害自身。

“难道仲元不知经权二字。没外人的时候,还不能歇吗?当真要栴哥儿跪昏过去啊!”韩冈对守孝要守到形销骨立才叫孝子的世间认识,完全无法认同,看着王旁犹犹豫豫的不肯动,“算了,去找医生来。”

“医生……”王旁愣了一下。

韩冈没理会王旁,让人出去传医生。守灵守到重病,就没人能说不孝,反而要夸至孝。以王栴的身子骨,就算平常找医生来,也是照样也要拿出纸笔开药方。

王旁这时候反应过来,叹了一口气,也不拦着了。只要不会坏了亡兄的名声,他也想侄儿能好好的休息一下。

府中就有医生候着,过来看了看王栴的情况,就忙让人将他抱了送进去。王栴被母亲耳提面命,本来就是咬着牙在守着。现在提在心中的一口气一松,却当真昏了过去,倒闹得里面乱作了一团。

外面倒是没有乱,王安石的侄孙王朴暂时代替了王栴做了守灵的孝子,叩谢来致奠仪的宾客。韩冈则是继续做他的迎宾,替了吴安持下来。

“这半日辛苦正仲了。”

“算不上辛苦,下面就要多劳玉昆了。”

吴安持与韩冈说了几句话,就进府去歇脚。从粮料院的属官换作声名煊赫的龙图阁直学士,出入王安石府上的宾客则立刻就多了一分恭谨,与韩冈互相致礼时郑重无比,一丝不苟。而在门外排队等候入灵堂拜祭的官员们,也一下将说话的声音降了下来,只有灵棚中僧道的呗诵之声还在继续着。

身为一路转运使,还能压在韩冈头上的文臣最多也就三五十了。而且二十五岁的年纪到底代表着什么,对于朝臣们不言而喻。相对于他立下的累累功绩,还有现在的地位,宰相之婿这个头衔只是个附属品。不像吴安持,除去了枢密之子、宰相之婿两个身份之后,就不剩下什么了,一个太子中允而已。

站在门口,迎来送往了一个多时辰后,将及深夜,拥挤在门前的车马终于不剩多少了。这时身后的门中突然喧闹了起来,韩冈回头,由王安国、王安上还有王旁一起相送,吕惠卿从里面走了出来。相府侧门所在的巷道中,一队车马鱼贯而出,转向正门这里迎过来。

在武英殿中相别不久的吕惠卿,紧抿着嘴,满脸的沉重。走下台阶,与王安国等人殷殷告别,提起过世的王雱时,又摇头悲叹不已,似是对王雱的死,王家的悲伤而感同身受。

转过来,吕惠卿对上韩冈:“想不到元泽走得竟会这么早,玉昆上京时当没有想到元泽的病情会一至于此吧?”

“能见到元泽最后一面,总不枉这一路兼程而行。”

“俊士归天地,惜乎哉,痛乎哉!”吕惠卿长声一叹,翻身上马。

送着吕惠卿远去,轮到王旉来代替韩冈迎客。韩冈还没吃饭,肚子饿得正慌,只是一进内厅,就从内帷中,出来一名婢女,叫着他:“韩姑爷。”

在婢女身后,站在帷幕中的是个四十多岁的妇人。韩冈当然认识,是王安国的妻子曾氏。曾氏是曾巩妹妹,南丰曾家的女儿。王、曾两家是来往几代的姻亲。而曾巩的弟弟曾布因为背叛新党的缘故,与王安石又反目成仇。这份关系错综复杂,不过今天的丧事,内外皆是由王安国夫妇主持

韩冈走过去,就听曾氏说道,“旖姐儿有些累了,老身让她进房里歇着去了,韩姑爷还是去看一看。”

韩冈听了之后,忙忙的谢过曾氏,就往王旖的房间去。

推门进房,王旖正躺在床上。闭着眼睛,像是睡着了,边上只有两个贴身使女站着。但韩冈一走进来,她就一下睁开了眼睛。

“官人?!”王旖见到是丈夫,就立刻挣扎着要起来。

“先睡着吧。”韩冈坐到床沿上,按着王旖纤弱的肩膀让她躺下去,手向外挥了挥,示意婢女出去。

俯下身子,理着妻子耳边散乱的发丝,柔声道,“别累着自己。”

王旖用脸颊感受着韩冈手上的温度,抓着衣角,静静的过了好一阵,轻声问道,“官人能在京里多久?”

“广西的事不能耽搁太久,最多只能有十天。”韩冈说着,就感觉到抓着自己的衣襟的手一下就收紧了。

