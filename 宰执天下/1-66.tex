\section{第30章 臣戍边关觅封侯(一)}

韩冈并不知道因为自己区区一个从九品的官身,已经惊动到天子和宰相头上。他现在一边读书,一边安安心心的等荐章被批准的消息从开封过来。届时他就要启程去京中流内铨缴三代家状——所谓家状,也就是包括祖宗三代的姓名、年甲、以及有无过往罪行的个人简历,其上还要有乡邻作保,证明身份确凿——如此一来,就能领到一份告身,这就是他身为官员的凭证。

自家的房内,韩冈伏在案前运笔疾飞,一行行蝇头小楷出现在雪白的纸面上,转眼便是一页。这是他在抄写过去那一位曾经抄写过的《谷梁传》。虽然现在可以买得起自己想要的书籍,但韩冈深信一句话,好记性不如烂笔头,再怎么读书背诵,也比不过亲手写上一遍记得更牢,书架上的所有经书典籍,他都打算重新抄写一遍。

谷梁传是春秋三传之一,与左传、公羊传都是孔子所著《春秋》一书的注释。春秋是鲁国的史书,为孔子所删改修订,后来成为儒家经典——孔子这番作为,称为‘笔削春秋’。为其注释的传,据说有九种,但流传下来的,便只有左氏、公羊、谷梁三传。

不论春秋还是三传,都是经部中的重要典籍,韩冈的前身早在张载门下就已通读过。如今韩冈拿后世的眼光来比较,觉得这三传里,左传更像是历史书,用丰富的历史资料将《春秋》中的简短记录进行扩展注释;而公羊、谷梁更接近于政治书,并不关心书内记载的历史,而是通过阐述《春秋》中的微言大义,来体会孔子笔削春秋所要表达出来的用心和儒学理念。

左传姑且不论,公羊和谷梁两传提起先圣的微言大义,总少不了一条华夷之辨。而韩冈的老师张载,向学生解说《春秋》时,提得最多的也是隐藏在书中字里行间的华夷之辨。春秋时,周室衰弱,四夷兴起,南方的楚国本是蛮夷,却自称为王。

后齐桓公在管仲的匡助下,尊崇周室,九合诸侯,压制四夷,即所谓的尊王攘夷。此一事,最为孔子所看重,所以他说,‘微管仲,吾其被发左衽矣’——没有管仲,我就要被迫学着夷人的模样,披散头发,穿起左衽的衣服,意指泱泱华夏被夷人所毁。

在孔子千年之后,胡人安禄山毁了大唐盛世,五代又有胡人轮流坐庄,眼下西北二虏猖獗,中原不振,所以宋儒一说起春秋,就要提到华夷之分,尊王攘夷,至于其他方面,却是泛泛而谈了。

‘民族主义看来并不局限于时代。’韩冈边抄边想,受到的伤害越重,激起的反弹也越大,尤其是汉族这个自尊心和自豪感都极强的民族,更是如此。

虽然此时对民族之分还没有一个清晰的认识,但单是提倡华夷之辨已经足以在汉人与夷人之间划出一条深深的鸿沟,唐代那般海纳百川的情况绝不可能出现在宋代。韩冈本就是从民族主义思潮强烈的时代来到北宋,对宋儒对隋唐外族策略的反省,当然有着很深的感触。

思绪如潮,韩冈一不留神,将一个字抄错了。白纸上,别字分外显眼,就算有后世的橡皮也擦不干净,但雌黄可以。韩冈的手边就有一块雌黄,拿起来在别字上一涂,墨迹就被雌黄留下的颜色所掩盖。雄黄是端午时泡酒用的,而雌黄却是古代的橡皮和修正液。信口雌黄这个成语,便来自雌黄的用途。

放下雌黄,重新拿起笔,房门这时被轻轻的敲响。韩冈又把笔放下,道:“进来!”

韩云娘应声推门。一身新制的襦裙,剪裁得更为贴身,一条黄丝绣花的腹围勒在腰间,俗称的‘腰间黄’衬得腰肢纤纤。一件花菱褙子罩在襦裙之外,遮住了胸前微微隆起的动人曲线。比起三个月前,韩冈刚病愈的时候,又长高了些许的小丫头更多添了几分颜色。她步履娴雅的走进房中,先道了个万福:“三公子……”

韩云娘的新称呼,韩冈听着扎耳朵,打断道:“早跟云娘你说了,不要这么喊我。不就是当个官嘛?过去怎么叫的,现在还是怎么叫。”

韩云娘低着头怯生生的说道:“那样会被人说我……奴奴没有规矩。”

韩冈眉头皱了起来,真不知是那个混蛋教了她这些无聊的东西。韩云娘本来就是个温良贤淑的性子,小小年纪就已经有了贤妻良母的范儿,只是谈吐举止比不上大户人家出来的女子。

但跟在韩阿李身边长大,没有学着满口老娘,已经是老天保佑了。韩冈对此并不是很在意,不管怎么说他都是来自千年后等级制度已经宽松许多的时代,对言辞上的一点不合礼节并不是很在乎。

“在家里,又不是有外人,讲究这么多作甚?性情贵在自然,刻意学着别人家的范儿,丢了本来模样,反为不美。”韩冈一伸手,很熟练的把她纤巧的身子揽在怀里。让人迷醉的温香软玉紧紧贴着身体,晶莹如玉的小耳朵就在自己嘴边,韩冈一时兴起,忍不住张口咬了一下。

小丫头浑身一颤,仿佛过了电一般,如羊脂玉般娇嫩细滑的脸蛋蹭的变得通红,扭过身子瞪着韩冈,嗓音细细的嗔怪道:“三哥哥!”

略有凹陷的眼窝中,一对泛着棕色的剪水双瞳清澈纯净,还能看见自己的倒影。看似嗔怒的圆瞪着的眼睛,却隐约有三分羞意,七分柔情。小丫头这样的反应,韩冈百看不厌。他双手收紧,贴在在韩云娘耳边柔声道:“你现在这样子,三哥哥才是最喜欢的。”

偎依在熟悉的怀里,嗅着熟悉的气息。小丫头的一颗惶惶不安的心,开始轻缓的跳动起来。自从韩冈被举荐入官的消息传入耳中,她高兴之余,也有些失落。身份的差距越来越大,心中总是担惊受怕,生怕三哥哥什么时候讨厌了自己。她只是一个无父无母的孤女,又没有个兄弟姐妹可以扶持,今生能依靠的良人只有韩冈。

感觉到怀里的少女心情平复下来,韩冈轻轻的放开了手,再不放自己恐怕就忍不住了。只是他知道,小丫头的心结不会那么容易解开。更好的安慰方法不是没有,但韩云娘太小,至少要再过两年。韩冈暗叹一声,这也是做官带来的副作用。

副作用虽有,但做官是件好事。免徭役,减税赋,这些都是跟着官身而来。而做官的好处却不仅仅这一些。正如《儒林外史》中所写,范进一旦中举,便成了岳父胡屠夫口中的‘天上星宿’,自此田宅有了,钱财有了,奴婢也有了。

在北宋也是一样,每逢进士放榜,多少富贵人家守在皇榜下,准备找新晋进士为女婿,即是所谓的榜下捉婿。可这女婿也不是好捉的,如今赠给进士女婿的嫁妆底价已经涨到千贯,而且还有继续上涨的势头——这是前几天王厚找他聊天时,当作笑话随口提起的。

韩冈虽然不是进士,但他的行情却也是一样的好。被推荐为官的消息已经传扬开来,一个才十八岁的名门弟子,又得多人推荐,前途实是无可限量。上门赠钱赠物的不说,提亲更是为数众多,所以王厚才拿着榜下捉婿来打趣。

韩云娘碍于身份,做不得韩冈正妻。小丫头自己也清楚这一点,从没有奢求过什么。韩冈自问也没有这个必要去挑战世俗,但心中对韩云娘不免多了几分愧疚和怜惜。不过换个角度想,小丫头有自己和父母给他撑腰,日后就算明媒正娶个性格不好的大家闺秀进来,也不能把她怎么样。

其实因为一个官身而战战兢兢的不止韩云娘一个人,韩千六也是有些不适应身份的变化,对挤上门来的生客,很是头疼。反倒是韩阿李,对待人接物的规矩心中都有个谱,不论来客身份高低,她都能暗地里帮着韩千六做得妥妥贴贴。

而韩冈本人,在成了秦州城中一颗冉冉升起的官场新星之后,则是表现出一副更加诚惶诚恐的样子。送上门的礼物,该推的推,该辞的辞,一件贵重点财物都没有收取,只收了些笔墨纸砚,以尽人情,至于提亲的,也让父母给推辞掉。

在他看来,有了官身,能做的事就多了,根本不需要在这个节骨眼上见钱眼开。多少人在盯着自己,一点差错都会影响到自己的评价。何况如今来奉承韩冈的,多是些想投机的寒门,一干豪门大族都还在观望中。

州中的传言都说韩冈杀性太重,几次出手,折在他手上的人命,都有几百条,算上末星部,一千往上跑。而他日前捉了陈缉,斩了过山风,送了近三十个首级去衙门,彻底绝了陈举家的后,更是印证了这番谣言。根基深厚的大家族很少喜欢招这样的女婿。

对于此事,韩冈倒是一点不在意,大丈夫何愁无妻。何况三十岁没娶浑家的措大多了去了,他身体的年纪才十八岁,精神年龄倒是年长一些,却更不会把婚姻之事看得太重。身体实在憋不住,也不是没地方可去。现在的当务之急,是尽快辅佐王韶完成收复河湟地区,从九品的幕职官,韩冈可没兴趣做太久。

ps:韩三望空大喊:我回来了!

今天第三更,俺也望空大喊,你们的红票呢?!

