\section{第31章 停云静听曲中意(11)}

瀚海难渡,灵州川路也是极之难行。为了阻截宋军的后援,有不少辽人骑兵游荡在沙漠之中。拖延援军,破坏粮道,清扫信使,竭力阻止更多宋军进入兴灵。

种谔从灵州城下转移到兴庆府外,与党项人会合,正是为了避免进退两难的窘境。同时也是在躲避吕惠卿的军令,但就算再怎么耽搁,来自宣抚使的命令送到种谔手上,也不会超过十天,留给种谔的时间已经不多了。种谔相信吕惠卿会拼命的追回自己,从时间上算,他最多还有三五日的功夫。

经过一番休整,当战鼓的节奏重又激越起来,种建中就再一次领军上阵。两刻钟后,当他被替换回来的时候,腿上又添了一处新伤。

两翼的战局跟之前没有两样。仁多家和叶家的具装甲骑并没有再次出战,重骑兵的冲击一天也不过两三次,只能用在最关键的时候。

中军的战局虽也是依然在僵持中,不过已经没有之前的混乱,一部骑兵下马改作弓弩手,再辅以轻骑兵护翼左右,打得有声有色。种建中方才一去一回,伤亡的情况也小了许多。

腿上多了一重绷带的种建中回到种谔的身边,失血的脸色苍白,双瞳却更形幽深。

他望着战场中央渐渐占据了上风的局面,不禁喟然一叹,西军中最为精锐的骑兵在骑战中连辽人的头下军都赢不了,终究还是只能用步卒与辽军交手,完全可以说是一种悲哀。

不过在以步兵为核心的战斗中,种谔和他麾下大军的战力也终于彻底的发挥了出来。根本不需要种谔一条条的去下命令,下面每一个百人都的都头都知道自己该做什么,能做什么。绝大多数跟随种谔北上的都头一级的小军官,皆是跟了他十几年、几十年的旧部,甚至有不少还是清涧城时代就从军跟随种世衡种老令公的老人。种谔只需要通过旗帜和鼓号节奏的变化,就能如臂使指的驱动数以千计的士兵。

突进、后退、射击,十几个都组成的军阵如同一只巨兽,单个部分的行动各不相同,但整体却在一步步的向着前方缓慢却又坚定的移动着。箭雨不断的狙击着阵前辽军,军阵的每一个变化都将神臂弓的杀伤力释放到最大。

不过中军和左右两军之间,脱节得很严重,使得种谔不得不留上一手,以备万一,没办法使出全力。种建中暗自庆幸,对面的辽人也是以不同部族组成的军队。相互之间的配合也是生疏得紧,要不然上午的时候可就难看了。

一队明显是精锐的骑兵瞅准了神臂弓发射的间隙,终于冲到了军阵之前,立刻就被排在后方的神臂弓手给乱箭射了回去。

射击节奏的变换不定和前后呼应,是箭阵对敌时必须要掌握的技巧,否则很容易为敌军所乘。种建中看到弓箭手们的精彩表现,忍不住叫了一声好。

骑兵在步战时的战斗力要远远强于骑战,这是极为讽刺的现实。观战的种建中在欣喜之余,也不免心生感慨。但马军能全数装备上战马只是这两年的事,之前西军中任何一个马军指挥,能拥有坐骑的骑兵基本上不会超过一半。日常从来都是依照步军来训练。

其实如果手上有这样的一万大军,种建中不需要党项人的配合,也有充分的自信将对面的辽军彻底击败。之前也早就直接把灵州给攻下来了。可种建中也知道,就算是三万辽国最为精锐的宫分军,也不会与一万已经完成列阵的官军对战。他们只会绕过去,分散开来劫掠乡镇。而官军的列阵以待又能坚持多久?迟早会被来去如风的敌人给拖死。

河北的广袤原野,就是骑兵们纵横奔驰的乐园。如果战场转移到河北,种建中没有自信逼迫辽人能停下来决战。他同样还没有自信,在会战中指挥数以万计的大军。河北军习惯于大规模的战争,一个平戎万全阵就要有十几万兵马组成,绝不是陕西这边,习以为常的是数千人的战斗。纵然伐夏之役的三十万兵力,也是给陕西缘边的地形分割得七零八落。

只不过将战斗局限在陕西,局限在当下,种建中却有着充分的把握。

歇息了好一阵,种建中此时已经回过气来,伤口处也不再是麻木的感觉,终于有了一点痛感了。

种建中双手握了握拳头,已经恢复气力。再拔出鞍后的马刀,血迹斑斑,却砍得卷了刃口。命人换上了一把长槊,他驱动战马前进了两步,更接近了种谔一点。

种谔瞥了侄儿一眼,回头又望向战阵,只有一句随风传来:“小心一点。”

冷心冷面的种谔说出来的话声也是冷的,只是其中也掺杂着浓浓的关心。

“末将明白。”

种建中向种谔行了一礼,离开种谔到了阵前,举起长槊将一队骑兵聚到身边。步兵的战阵两侧必须要有他们保护。在必要时,也要追击和迟滞溃退的敌人。回头望着旗帜,等待着他叔父的号令。

不过种建中还没有等到种谔的命令,就看见一名来自后方的骑兵直奔大纛之下。

“大帅!青铜峡那边传信过来了。”一名亲兵小跑着到了种谔身边,递上来了一颗从斥候手上收到的蜡丸。种谔回头看了看送信来的斥候,运力捏开了蜡丸,将里面两寸宽五寸长的丝帛展了开来。

展开帛书,匆匆看了一遍,种谔的脸色一下变得极为怪异。疑惑、愤怒、安心以及沉思的神色,走马灯一般的在种谔脸上掠过。

种建中吃惊非小,完全猜不出种谔到底在信上看到了,想说些什么。但下一刻,种谔便高高举起手中的短笺,用尽了全身的气力高声吼道:“援军就要到了。鸣沙城的赵隆带着八千兵马就要赶过来了!!”

话声刚落,欢呼声就在种谔身边爆然响起,然后便如同巨石落水,激起的波浪一圈圈的扩散开去,一瞬间传遍了整个战场。

鸣沙城的赵隆如今也是关西声名显赫的大将,种建中都不如他。不比王舜臣差到哪里。银夏军中人人都知道,青铜峡南端的鸣沙城是抵挡辽军和党项人的第一线,其中的兵马皆是精锐,每一个指挥都是从泾原、秦凤两路精挑细选出来的。

而地位更高一点的军官更清楚,鸣沙城现有六个指挥两千出头的马军,龙骑兵——也就是有马步人——同样有两个指挥八百人,这些都是实数,丢下没有战马的步兵后,进兵速度不会慢到哪里。种谔说是援军八千,照常例打个四折,三千马军正好对得上。当他们赶到,这一战的胜负必将就此决定!

此时联军士气大盛,而辽军听到欢呼声后攻势陡然一落。虽然不知道赵隆赶来的消息,但辽人也不会看不出来大宋这边必然在战场之外出现了一些好事。

本来就已是勉强维系平衡的战局逐渐偏向一侧,宋军的欢呼声开始压制辽人的士气。原本仗着心中的一股悍勇死战到底的辽军,变得束手束脚起来。

‘赵隆要来了?他怎么会来?!’种建中的心中却满是疑惑。在周围欢呼狂叫的人们中显得格格不入。

赵隆不是一热血就能昏头的人,没有来自本路经略司的将令,他根本不会动,也不能动。

自家五叔只带了自己北上,却把十七、廿三留了下来镇守溥乐城,甚至连咫尺之间的耀德城都没让他们驻守,就是因为种朴、种师中不是银夏路兵马司辖下的部将,而他种彝叔才是。

究竟是谁给赵隆下的命令?是泾原路经略安抚使熊本,还是……

种建中思念所及,一下被惊得望着南方。

吕惠卿!!

其余几位将领也都想到了种建中想到的问题。甚至有人觉得这件事本身就不对劲,若当真有援兵来,有必要说出来吗?三千生力军突然出现在战场上,就不是胜利那么简单了,而是歼灭,是数以千计的斩首。种五太尉既然公开宣扬,要么是缓不济急,要么就可能是兵不厌诈。

可随即种谔的行动,就让那几个多心多思的将领没时间想东想西了。

种谔没有理会任何人,他已经驭马靠近了大纛。

赵隆到底何时能来?是不是真的能来,都是一个疑问。只是种谔完全没有拖延等待的意思。他一手便拔起了大纛,巨大的旗杆便张扬的斜挑在空中。

种谔放声大吼:“让鸣沙城的兵去打扫残敌吧,我们只要干翻对面的辽军!”

种五太尉的豪言一呼百应,种建中也抛下了无谓的疑问,挺起长槊扬声大喝。

种谔举着大纛,驱动战马开始前进。身前身后的千军万马如影随形,两翼同时响应。仁多和叶家的具装甲骑再一次冲出阵列,领头向敌军攻去。

这是种谔自开战之后最为猛烈一击,在这一瞬间产生了巨大的效果,纵然前方人山人海,也没有人能挡下种谔前进的步伐。

随着种谔的冲击,正面的辽军在接战的那一刻溃败下去。苦战了一日之久,当第一支来自于一个奚族小部落的队伍脱离阵线,辽军就彻底崩溃了。

之前的鏖战仿佛是一场梦,眼下四散而逃的敌人却是眼前再真切不过的事实。

种建中早已冲击到了阵前,他一马当先,不管事后如何来自朝廷的究竟是甘霖还是雷暴,他现在只追求一个胜利!
