\section{第13章 晨奎错落天日近(14)}

是者,则也。

所谓国是,国政之则,天子、宰相、诸大夫共定。

国之有是,如政之有刑。触刑者如律,犯国是者,自然是逐离朝堂。

熙宁十年,新党用‘国是’二字,干掉多少反对派?

元丰之后,新党又是用‘国是’二字,让多少反对者噤口不言?

不用提点,向太后自己就能数出许多。

当韩冈为什么说辽国撕毁盟约,接连入寇是国是之故,这其中道理却是让向太后不明白。

正想发问,吕嘉问便跳了出来,“夫家自为政,人自为俗,先王之所必诛;变风、变雅,诗人之所刺也。朝廷惟一好恶,定国是,澄清朝堂,国势大兴。南交亡而西域定,西夏灭而北辽败;”吕嘉问瞟了韩冈一眼,“若先帝昔年未定国是,承祖宗之旧法,从之富韩之谬言,含辱忍垢、不言兵事,韩参政岂能站在这里?”

吕嘉问的一番话如暴风骤雨,噼里啪啦的砸向韩冈。

但他说的也的确有理,没有王安石的新法,没有先帝熙宗皇帝的提拔,韩冈哪里有出头的机会?有其才,却不得其时的人物,历史上太多太多。

“的确如此。”韩冈不可能昧着良心否认,对着太后道,“先帝与王平章当年所定国是,便是新法种种,总而言之,不过是维新图强四个字。今日臣能立足垂拱殿上,实赖于此。”

太后平静的等待着,韩冈之后肯定有转折,吕嘉问也知道,抢先一步,“既然……”

“但一时之法,当一时之用!”韩冈声量陡然提高,截断了吕嘉问的话,“祖宗之法,国初之时,祖宗持之以平诸国、定天下。至仁宗时,便已难以应付变局,之后抱残守缺,至先帝登基,已几近病入膏肓,如此方有变法之事。先帝登基时,外饰太平,内则倾颓,兵不堪战,财不足用,西贼猖獗,北虏虎视。先帝见及于此,擢贤能,用新法,不数年便兵精粮足,进而平交趾,灭西夏。虎贲三千,就能抵定西域。精兵数万,便可遏阻北虏,诚乃新法之功。可如今国是已不合于时,是到了该变一变的时候了。”

“何谓不合于时?!”吕嘉问立刻反驳,“韩参政这么说是因为国势昌盛?!因为四夷畏服?!因为国计丰裕?!因为百姓安居?!”

一连串的排比,让韩冈的言辞变得薄弱无比。

“不谋万世者,不足谋一时;不谋全局者,不足谋一域。眼下国势虽盛,却隐患重重,若不及时加以弥补,日后难免熙宁之危。请问吕三司,祖宗之法残民乎?”

吕嘉问深深的盯了韩冈一眼,却露出一丝微笑:“祖宗之法承之五代。只是因为国事初定,方抱残守缺,承袭下来。而自太祖至今,幸得诸圣勤于政事,又得上天庇佑,方得保平安。昔年平章进于先帝疏中亦言,‘赖非夷狄昌炽之时,又无尧、汤水旱之变,故天下无事,过于百年。虽曰人事,亦天助也。盖累圣相继,仰畏天,俯畏人,宽仁恭俭,忠恕诚悫,此其所以获天助也。’”

韩冈双眉一挑,想不到吕嘉问将王安石吹响变法号角的《本朝百年无事札子》,背得这般滚瓜烂熟。

按王安石在折子中的说法,祖宗之法早就该丢到垃圾堆里,一开始就是有错,皇宋能安享太平百年,是诸帝勤政爱民,引得上天相助,之所以要变法,是老天爷的帮助越来越少,不能再期盼其帮忙了。

这是标准的黑白分明,直接否定祖宗之法的效果,将之归功于开国以来历代天子克勤克俭、敬天畏人。也是因为正是两党相争的时候,当然不可能去肯定对方坚持的宗旨,只会一棒垩子打到死,自然不会有辩证法存在的余地。

但更让人意料不到的,是吕嘉问敢于直接攻击祖宗之法残民。真是奋不顾身。现在能驳回自己的言辞,转头来,除了少数几人外,绝大多数御史都不可能坐视。

韩冈气定神闲:“熙宁十载,天灾频频。自改元元丰,风调雨顺直至今日。偶有灾异,不过一路而已。数年前,割让国土与辽,数年后,却能让北虏无功而返,前后相异,岂是无因?”

什么四夷畏服、国势昌盛、百姓安居、国计丰裕,这是老天帮忙!

而且不管是什么原因,割地一事,的确发生在王安石为相的时候,这是他洗不脱的。韩冈没有明着拿此事指责王安石,但国势不济,不能助天子免于耻辱,要么是王安石本人的问题,要么就是老天不给面子。那么现在情况好了,还不是老天的功劳?

不过殿上争辩,绝不是给人讲道理。

“仁宗时无尧、汤水旱之变,又为何困于二虏?”吕嘉问反问。

韩冈正要开口,却不提防蒲宗孟抢先道:“参政先立功于西方,后平蛮于南方,却都是在熙宁时。”

真是好助攻。原来韩冈看蒲宗孟不顺眼,今天倒是要刮目相看了。

韩冈接上道:“韩冈虽有微劳,不敢居功。西北二虏之势,岂是南方小国与吐蕃诸部能比。何况不论是平定西羌,还是剿灭交趾,都是靠了屯田加上诸路输送,才能够支持大军出征。到了灭西夏时,已是用上了两年丰稔后的举国之力。辽国国势十倍于西夏,没有十年之积,谈何攻辽?”

吕嘉问反问:“北虏大军业已叩关,难道还要看一看仓库,才去决定打还是不打?”

“参政!吕卿家!”向太后终于忍不住喝止了双方的争吵。

韩冈、吕嘉问都住了口,齐齐谢罪。

韩冈松了一口气,想要正正经经的把话说完,不先吵一下,让太后来弹压,根本就做不到。肯定是说几句,就会有人出来反驳。

向太后对韩冈道:“还请参政说一说,辽人屡屡入寇为何是国是之故?”

“方今国是,是变法图强,是富国强兵,是为了日后能够不再困于四夷,收垩复汉家故土。可辽人畏于中国日渐势强,忧惧日后难以抗拒天兵,便想方设法将战事提前。或暗助西夏,或主动南侵,或引诱官军北上,只要其中有一条成功了,伐辽的时间就会推迟许多。”

“现在还不是伐辽的时机?”向太后问道。

“臣已经累番上书,陈述此事。且皇宋之患,不在外而在内,当务之急,不是伐辽,而是安民。”

“国中将有乱?”太后心中一惊。

“陛下。”吕嘉问立刻放声道,“韩冈这是造危言耸听之辞,欲以祸乱圣心。”

今天就数吕嘉问最是积极,其他人如章敦、曾孝宽的话,似乎是让他一人给说了。

而向太后明显不喜欢吕嘉问这样的积极,语气不快:“吕卿,且听了韩参政说了再议论。”

吕嘉问瞥了韩冈一眼,低头再次谢罪,然后退入班中。

现在太后还没有明显拉偏架的意思,若是冲得太前而惹怒了太后,反而会坏了事。

没有了干扰,韩冈继续说道:“老聃有云:治大国若烹小鲜。在臣看来,治国却如同给人医病。医者与人疗伤治病,必先及危及性命的重症,然后才是头疼脑热的小病。如一卒伍战阵上受伤,一伤在手指,一伤在腰肋要害,那么军医肯定会先去治腰肋之伤,然后再去包扎手指。治国亦如此理,必须要先分清主次,解决最为危急的症结。

先帝践祚之初,国计乏用,兵不堪战,盗贼横行,此亟待诊治之重症。故而先帝以青苗、免役诸法济国用,以将兵、军器练军卒,以保甲法安国中。而如今国势已盛,却尚未能轻取辽国,人口虽众,兼并却日益增多。臣观此患,远过于北虏。没有足够的土地,没有足够的粮食,怎么养活亿万生民?须知三代以降,中国或有不绝若线之时,却未曾为蛮夷所灭,只有因内乱而亡的例子。”

今日大宋国中的主要矛盾,是日益繁衍的人口与增长缓慢的口粮之间的矛盾。

韩冈到底想说什么,王安石、吕嘉问都清楚。

关于人口膨胀,以及与口粮、土地之间的矛盾,韩冈早前曾经说过很多。以他的身份,他的这番论断,在当时的确被主流所重视,甚至为新党所喜,在朝堂层面上,很多人都把这番话当做了对外扩张的借口。但现在听韩冈的一番陈词,日后多半就会是气学与新学争锋的工具。

所以吕嘉问又忍不住出来驳斥:“空口白话,毫无实证。皇宋万里疆域,无人处极多,岂有土地不足之患?”

“皇宋万里疆域,山丘多少,坡地多少,沙漠又是多少?苦寒、瘴疠之地,又是多少?大宋土地虽广,能豢养生民、适宜耕种的土地,也不过十之二三。近三年来国中户口,因有隐户逃丁,故而变化不大。但京垩城中出生的幼子,每年都要比前一年多上一成。”

“田籍户簿之中,不计非丁妇孺。不知此语,有何凭据?”

“有保赤局簿册为证。为避税赋,隐户逃丁不知凡几,如河畔蚊虫,捕不胜捕,查不胜查。而为了保幼子平安,十文一剂的牛痘却没人敢省,而且多少富贵人家和寺观,都会出钱买药施赠,平民百姓家的子女往往一文不用便能在保赤局种痘,故而无人逃避。论起数目是否可信,保赤局的记录远胜于籍簿。”

向太后连连点头,“参政之言有理!这等道理吾还是能想明白!”

太后如此说话,就不方便出来驳斥韩冈,更不方便胡搅蛮缠。王安石、吕嘉问都保持了沉默,跟之前的曾孝宽和章敦一样。

“那么,去保赤局种痘的幼子到底有多少?”太后好奇地问道。

“回太后,去岁开封府界,种痘数量是十二万三千九百余,比之前一年的十一万,增加了十分之一。而京师军民百万,十二万三千的新生幼子,也占了人口总数的一成还多。如果年年保持这种速度,是要七年,京垩城人口就会翻上一番。”

“不是十年?”太后纳闷的问道。

“不,每年都是在已经增加过的前一年的基础上再增加,所以只要七年。”

吕嘉问却笑了起来,“试问世上生民怎么会光生不死?只计生,不计死,世间早就人满为患了。韩参政以算学闻名,怎么连这么简单的算术都做错了?”

“每年京师过世之人都不会少,可再多,能有十二万三千吗?开封百万军民,八个人中就有一个死了?”

堵了吕嘉问一句,韩冈继续说道,“如果国中一开始就有男女老幼共五千万口,七年之后,就是一万万,十四年后就是两万万,二十一年后,是四万万。”他就在殿上扳着手指数着,“即是这二十一年中,一开始的五千万都死光,二十一年内出生的三万万五千万人死了其中的一半,那也有一万万又七千五百万。何况,根本是不可能死光的。”

千分之一百的自然增长率,这当然是不可能的。这样的计算方法有太大的问题,可有了确切的数字,这么算起来却是让人心中不寒而栗。

“如此多张嘴,请问如何让他们安居乐业?”
