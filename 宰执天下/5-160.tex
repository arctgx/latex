\section{第17章 往来城府志不移(七)}

数句寒暄作为开场白,赵顼便道:“韩卿,不知对太常寺、厚生司和太医局”

“臣尚未就任,太常寺、厚生司、太医局三处此前亦从未与任,不敢妄言。”

赵顼笑道:“韩卿无须自谦。张载以明古礼而著称于朝,卿家师承张载,判太常寺可谓适任。厚生司是缘韩卿之言所创,而医事上,更是卿家所长。如何不能言?可直言无讳。”

韩冈沉吟了一下,道:“太常一职,周曰宗伯、秦曰奉常,先王以之掌礼乐医卜。得列九卿之首,便是因其掌礼乐,以明纲常。不过如今悖于古礼者不胜枚举,先师曾有言,兴己之善,观人屯志,群而思无邪,怨而止礼义!入可事亲,出可事君,但言君父,举其重者也!”

在儒学中,礼为纲纪之本,而乐有教化人心的作用。礼乐不分家,故而太常寺依制当统管礼乐之事。同时诗乐也一样不分家,孔子编修的诗经,跟礼乐脱不开关系。故而在朝廷的各项典礼上,歌者所唱的多是模仿诗经的四言诗。

赵顼怡然颔首:“韩卿此论甚佳,过几日当下太常礼院共议。”

当面提起太常礼院,自然是提醒韩冈判太常寺的职权范围在哪里,不过紧接着赵顼又道:“稍待时日,再与韩卿论此事。”

韩冈闻言略感讶异,前一句没什么,单纯的提醒而已,但追加的一句‘稍待时日’,却有些怪异,似乎有深意。难道过些日子,他就有资格谈论礼乐之事了?

韩冈隐隐听说赵顼有意改易官制,将如今叠床架屋的官制正本清源。前两年,赵顼下诏校勘《唐六典》时,曾经有了些许风声。可因为不见后续的动作,又是撞上平夏之役,便没了声息,也无人在意,只当是普通的典籍修订而已。但赵顼方才补上的这一句,似乎是有了点意思……不过也可能是自己想多了。

既然天子已经点明不要侵犯太常礼院的职权,韩冈也不再多提太常寺,“至于厚生司,虽为臣所主张,然其职掌久已有之,太医局、翰林医官院皆曾掌其事……”

赵顼笑道:“过去可没有保赤局。”

赵顼很看重专责种痘的保赤局。韩冈的名号因牛痘传遍天下,但赵顼的名声何尝不如此?毕竟韩冈也是赵顼的臣子。

同时尽管种痘的价格极为低廉,由于地方的不同,一剂都在三五十文上下,最贵的也不会超过一陌。也就是七十八文,一斗米的价格。而且还有许多富户和寺观为了阴德之事,一口气包下几十份、几百份甚至上千份痘苗,散于普通人家的幼儿,与自己要种痘的子嗣同时施种,使得天下间几乎没有种不起痘的儿童,但举国上下,每年需要种痘的幼子何啻百万,使得朝廷一年也有几万贯的结余。

这等有名有利,而且对自己和儿女都有好处的好事,赵顼很想韩冈多拿出几桩出来,反正过去从没有因为医事而晋升两府的例子。而且韩冈就算能让肺痨、风疾、消渴症之类的重症都可以免疫,要想普惠天下,也要数年乃至十年之功,那时候让他入西府,也不犯什么忌讳了。

不过韩冈的回话完全涉及新的药方,“臣在河东,保赤局由于专责种痘,事务最是繁忙,名号时常传在耳边。但厚生司却极少听人提起。臣当初请陛下设立厚生司的本意,应该是在主持防犯疫症上。痘疮只是疫症的一种。‘疾医掌养万民之疾病,四时皆有疠疾’。如痢疾、伤寒、时瘟、蛊毒水肿,皆是伤民百万的疠疾,厚生司不当袖手旁观。应参与其中,以保生民,使大灾之年不至有大疫,让陛下圣德庇佑万民无伤。”

“自当如此。”韩冈这是想让厚生司能起更多作用,不要像现在这样,只有保赤局最忙,其他几个分司则是清闲得不像话,赵顼对此当然不会反对,“厚生司中事,卿家可放手施为。”

“至于太医局。”韩冈想了想,又道,“太医局有教养学生、试选医官之职,不过良医靠的是多习多练,并非是读书受教可得,如今在局中因为圭臬的《本草》等书亦因编目不明,不得历练,教训不出良医。但臣听闻如今的太医局,只有十几位翰林医官最得看重,日日邀约不断,而太医局生得到邀请上门问诊的则寥寥可数。”

“韩卿或许不知,太医局生须往在京诸军和诸学问诊,医治的学生和卒伍不在少数。”

“依然是太少。若是登门问诊,一日不过三五人,而坐馆,一日三五十人不在话下。太医局生上中下三等共百人之数,而在京诸军和诸学学生,无论如何都不可能每天都有三五千的伤病。”

赵顼差不多听明白了:“韩卿的意思是设医馆,让太医局生坐诊?”

韩冈拱手一礼:“臣请陛下设医院收治京城病患,并设药局以供在京百万仕宦军民。太医局生由此可以磨练医术,而一众翰林医官,也可以隔上数日在医馆或药局中轮班坐诊,见过的疑难杂症越多,也能让他们的医术百尺竿头更进一步。”

听着韩冈侃侃而谈,赵顼暗自庆幸,幸好将韩冈调到了不算重要的职位上。还没有上任,就有了这么多想法,等到上任之后,肯定还不知会有多少。若是放在紧要的职位上,当亦是要大刀阔斧一番。

赵顼忽然有些想笑,王安石不论在哪一任上,都想有所成就,他的这位女婿也是一般的性格和为人。好像不兴作一番,就不能安心受领俸禄一样。

不过两人都是能做事的臣子,放在哪里都能有所成就,不是纸上谈兵之流,这是赵顼能安心任用王安石和韩冈的主因。

眼下将韩冈放在太常寺,又让他兼管厚生司、太医局,赵顼就是想借助他的才干,又不用担心他之后的功劳太过惹眼,以至于不好安排。

不过赵顼还以为韩冈这一次会有与牛痘差不多的新医方献上,没想到听了半天,还是建医馆,任医官之类的寻常建议,让他不免有些失望。见韩冈没有新的说辞,便点了点头:“韩卿金玉之言,当书札条陈,以供朕细细观之。”

赵顼明显想结束今天的话题,韩冈心领神会,躬身一礼:“臣谨遵圣谕。”

正待告退,赵顼却又想起了什么:“方才韩卿言及《本草》编目不明,可有什么说法?”

世称的《本草》,就是《神农本草经》的简称。即名为经,其在医药界的地位,大略就跟儒门六经差不多。若不是韩冈的身份特殊,他敢说出这种话来,不是招人骂,就是惹人笑。

但韩冈的确对编纂医典药典有些想法。他本就打算以提举厚生司及太医局的名义召集人手,编一本医典或是本草纲目出来。前面的话题中就设法留了个引子,本以为还要再提上两三次才能见功效,不意赵顼已经主动提及此事了。

韩冈站定下来,胸有成竹的侃侃而谈:“陛下明鉴。典籍之要在于编目分类,医药之事,亦不能例外。所谓分其类属,明其源流也。《神农本草经》,不分动物、植物和矿物,仅以上中下三品并玉石、草木、禽兽来区分,附会天地人之意。却是满屋铜钱,连根索子都没有。”

动物、植物这两个词汇出现得很早,定义也与后世无甚分别,并非韩冈所创见。赵顼倒也不会听不明白,轻轻点头,示意韩冈继续下去。

只听韩冈继续道:“先师于《正蒙》有言,‘动物本诸天,以呼吸为聚散之渐;植物本诸地,以阴阳升降为聚散之渐。’两物截然不同。如丹砂、雄黄等矿物,差别更远。其下万物,也同样千差万别,不可混淆为一类。须以纲目区分之,以便医药之用。”

对生物分类的初步,韩冈早在《桂窗丛谈》中便有所阐述。《桂窗丛谈》中,韩冈将生物别做一编,统一加以叙述。昆虫,鱼类,还有虾蟹为首的甲壳类,在编目的时候就有所区分,江豚、海豚就不属于鱼类而是水兽之类的常识,都有所阐明。但言辞明确的要像编订书目一样给动物植物分纲目,韩冈还是第一次说出来。

苏颂编写过《本草图经》,是近年来药典中的最新著作。但他也免不了受到《神农本草经》的影响,在编目上,依然上中下三品分类,并无任何规则可言。

分类学是生物学的基础,就像代数是数学的基础一样,就像训诂文字音韵这样的小学是儒学的基础一样。可惜韩冈没有林奈的本事。最多也只能做到提出基本原则,门纲目科属种,他的知识范围只在最上一级的门,再往下就是一片片的空白,具体的内容得让人去填空。而且在生物学、矿物学尚未确立的情况下,只能先寄身于医药学中来安身。

不过分类学有个特点,学名之后要加上命名者的名字。这个特点,韩冈肯定是要继承下来。诱之以名,诱之以利,永远都是最简单有效的办法。

