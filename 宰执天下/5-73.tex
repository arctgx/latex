\section{第十章 却惭横刀问戎昭(五)}

【晚上酒席喝多了。将下午写好的一章发出来,晚上的一更没办法了。】

让自己兄弟去送折可适,从城门口转回来的种建中的脸色,让每一位迎面过来的巡城甲骑都心惊胆战的闪到路边,给他让出道来。

等种建中目不斜视的打马而过,又是一个个扭头回望,窃窃私语的议论着,在鄜延军占据夏州城中,究竟是谁惹到了这位衙内?

种建中胸中有一团火在燃烧。折可适临别时的几句话,让他心头充溢着莫名而来的怒意。

并不是种建中想要这么做,对于坐视徐禧的愚行,他一直是持反对意见的。但种谔的决定不容动摇,而且鄜延军中属于种家一系的将校,基本上都对徐禧,以及跟着徐禧一起闹腾的京营禁军厌烦透顶。

——他们要找死,就让他们去好了。

抱有这样的想法,才是夏州城中将领们的主流。

如果说之前折可适钓鱼论是事实的话,也是徐禧等人自愿跳下水,而不是种谔将他们穿到鱼钩上的。

而且说钓鱼也过分了点,没人能将手握密诏、身后又有执政支持的徐禧当成鱼饵。只不过是冷眼看着他带着一众想立功的京营禁军去寻死,不加理会罢了。自家的叔父也只是想从其中求取好处。

徐禧已经说服两府。宰相王珪称病。吕惠卿即将拜相。

一条条传言从京城传来,使得徐禧一时间拥有了独断之权。种建中更明白,这些传闻,也是让他叔父下定决心,推了最后一把。

种谔打算坐视徐禧自取败亡,他不打算与徐禧一起送死,但他绝不会白白浪费这个机会。当西贼出手的时候,也是他们将弱点暴露出来的时候。

如果一切能按照种谔的计划,这一次的伐夏之役,依然有着翻盘的机会,甚至更进一步攻下兴灵也不是不可能。

但种建中没有半点信心。他可不敢确定,自家叔父反败为胜的计划肯定能够实现。

之前叔父料错了天子,原本能打西夏一个出其不意,让其国中部族分崩离析的大好良机,却给莫名其妙的理由废掉了。

嵬名氏和梁氏以及几大部族,由此利用侥幸得来的一线生机,整合内部,凝聚人心,同时加快坚壁清野的速度。灵州之败岂止是因为失察之故?若是当时不收兵,径直攻向灵州,不用打就有人献城了,就连粮草也能就地征收。

当时五叔没想到皇帝会不顾军心强令收兵,现在万一再一次判断错误,种子正的名声,可是已经损失不起了。

而且当消息传到韩冈那里,以他的经验和眼力,不可能看不住自家叔父的私心,到时候,能有几成把握让韩冈不站出来说话?

韩冈立身之正,在军中是有名的。无论是之前反对速攻兴灵,还是之后反对逼迫自家叔父撤军,都证明他从不看人情面,只会就事论事。

在州衙门前下马后的种建中,脚步又沉重了许多。同为张载弟子,交情又颇深,他实在不愿看到种家与韩冈反目。

回到衙后的偏厅中,种朴正埋首在地图上,拿着根新近流传开来的炭笔点点划划。

种建中进门后,向他打了个招呼:“十七哥。”

种朴从地图上抬起头,回望了一眼,“送了折七回来了?”

“嗯。”种建中意兴阑珊,没什么心情说话,在角落里找了个位子坐了下来。

见到种建中的模样,种朴丢下地图和纸笔,走过来:“是不是折七说了什么?”

“嗯,说五叔这一回是钓鱼呢。”

“挺会打比方的嘛……”种朴笑了一声,在种建中身边坐下,“这不是折克行会说的话。”

“我知道这不是折七替他老子传话。”种建中沉着脸,叹息道,“但既然他都能看得出来,当也瞒不过其他明眼人。”

种朴盯住种建中看了好一阵,最后一声叹,“我说十九你啊,书读得多,那是好事。可心思也跟着多了,这就不是好事了。想得太多,就容易瞻前顾后,多谋无断。”他敲了敲座椅扶手,“既然已经成了定局,现在就该尽力将事情做漂亮了,而不是在这里叹气啊!”

“曲珍和高永能哪一个都不会甘心跟着徐禧一条路走到黑……”

进驻盐州的官军,大部分是京营——几名来自开封的将领一直都想立功,但始终没有机会,所以这一次闹腾得最凶便是他们——但还有一小部分是西军,以补充缺口。徐禧点人时,刻意排除了种家的势力,大概是不想让种家一系的将领立功。但曲端和高永能两人又不是傻子,徐禧认为这是对他们的奖赏,可在曲端高永能那边,恐怕都想哭的心都有了。

种建中问种朴:“五叔的计划当真能成吗?”

种朴的眼瞳中只有坚毅:“事情能不能办成,是做出来的,不是计算出来的。与其在这里想东想西,还不如想想怎么才能将事情做好。”

种建中出去了。

种朴又回到摆放地图的桌边。桌上的这份地图,有西夏、有横山、有辽人的西京道,连河东一部分都包括在内。

辽人的动向事关天下大局,摆开的架势似乎是准备从河北开刀,但实际上,往西边来也不是可能。对于辽人,不能不将他们的威胁考虑进来。

但种谔,他知道种建中有个名字没说出来——韩冈。

韩冈出任河东路经略使,这个任命意味着什么,只消回去查一查就知道了。看看能在危急之时出守边疆的都是什么人?

范仲淹、韩琦、庞籍……挂着宰执的名头,出典要郡的例子不胜枚举。韩在战时被派来镇守河东的臣子,加一个参知政事、或是枢密副使衔都在情理之中,郭逵正是如此。韩冈能在此时出镇河东,即便他受限于年资进不了两府,但基本上已经可以算是宰执一级的人物了。

种朴最担心的就是韩冈。

之前灵州之败已经证明了韩冈有着不下于郭逵的战略眼光。眼下自家父亲想要成事,就不能让人惊扰了徐禧的美梦。

但韩冈一旦得知此事,就绝对会这么做。种朴相信以韩冈的为人和品性,不会坐视数万甲士为敌所乘。

自家父亲对徐禧的态度是坐视,不论徐禧有什么动作,只要他还没有出事,就必须让他一切照常。可要是韩冈插足进来,情况就难说了。若是不能顺利的归罪徐禧,种家可就危险了。

……………………

“甘凉一时间是夺不回来了。”

当宋人以重兵进驻凉州的消息传来,兴庆府攻城,重又陷入了阴云之中。就是因为灵州之役而信心十足的仁多保忠也不由一阵哀叹。

从眼下传来的消息中看,秦凤、熙河两路的宋军已经将重心放在了甘凉之地上。

王中正甚至还派兵在葫芦河河口修寨。一旦葫芦河河口成为宋军的控制区,黄河岸边的应理城【今宁夏中卫】也将保不住多久。当应理城成为宋军的据点,通向甘凉的道路便就全给宋军封死了,应理上游的黄河河段,再也不属于大夏。

王中正的用心不难理解。弦高犒师的故事,梁乙埋也曾听说过。因为弦高的缘故,秦军偷袭郑国不成,不得不撤军,为了回国后有个交待,同时也是因为贼不空手,就将路上的滑国给灭了。

王中正眼下转向河西的甘凉一线,便是为了能有个交代。而他这么做的结果,就是彻底的断绝了西夏短时间内收复甘凉的可能,除非能下定决定放弃银夏。

但这个决心是没有人敢下的。

甘凉虽然重要,但毕竟不是大白高国的命脉所在。丢了甘凉还好说,但失了银夏,粮赋财税都要减半。。同时只剩兴灵一地,那样的大白高国只有灭亡一途。就算侥幸赢了灵州之役,大夏也只剩苟延残喘的气力。

银夏之地,能生财济国用的惟有盐州,青白池盐是不逊于宋国解盐的上等精盐,价格又便宜,最多时,青白池盐占到了陕西食盐用量的三成之多。多少年来,横山深处的小道上,来来往往的尽是私盐贩子。

但盐只能生财,粮食才是一切。而银夏之地的粮食主产地只在无定河两岸,更确切点就是银州至夏州的那一段。

如果宋人毁掉了盐州、石州,不过是一时没钱,盐田还是在那里,终归能恢复。但若是宋人夺了银州、夏州,少了银夏的存粮供给,又没了横山蕃部的支援,以兴灵为起点的粮草转运,甚至无法支持国中大军抵达横山脚下。

必须要赢!

就算使尽手段,也要使动辽国正式动起刀兵。不论如何都得想方设法,将盘踞在银州、夏州的宋人给赶出来。

梁乙埋下定了决心。

七月末的兴庆府已经渐渐由酷暑转为秋凉,迎面而来的风中也少了几分夏天时的燥热。

梁乙埋不,忽然停下脚步,弯腰从地上拾起了一片半黄半绿、形似手掌的落叶。

“相公?”亲兵队正疑惑的问着。

梁乙埋小心的将叶子收进袖中,抬头注视着宫中依然浓绿的一株株梧桐树,意有所指:“秋天到了。”

