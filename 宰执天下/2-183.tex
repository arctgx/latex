\section{第31章 战鼓将擂缘败至(一)}

‘肯定要退兵了。’

这一点毋庸置疑,韩冈向左右各瞟了一眼,视线在帐中转了一圈,在场每一个官员的脸上明明白白的都写着退兵两个字。

必须退兵了,罗兀城的现状,已经比鸡肋都不如。抚宁堡的问题还可以解决,如果是之前的局势能继续拖下去,西贼那里多半会先一步溃退。但庆州的叛乱却完完全全是个死结,不是将之简单的扑灭就能了事的。

当庆州广锐军的举起叛旗,罗兀城的命运已经注定。这不仅仅是一支几千人的骑兵部队叛乱的问题,更重要的是,广锐军为什么会叛乱?!是因为军饷、将领,还是由于畏惧战争?有广锐军为先导,其他陕西缘边各军会不会也跟着叛乱?

一旦这点疑问在天子心中扎下根来,韩绛和种谔的恢宏计划,还有现在几万人在罗兀城的血汗,都将成为了无用功。就算在叛乱之初就将之消灭在萌芽状态,也是一个结果。

何况,以现在环庆路的实力,究竟能不能将叛军消灭,这也是一个问题——已经很严重的问题。

环庆、鄜延两路的精锐,不是在罗兀,就是在绥德,要么就是在罗兀和绥德之间的某个地方。整个关西的战略重心现在就在这沿着无定河拉出的一条弯弯曲曲长约六七十里的线路上。而环庆路,在张玉和姚兕被调来援助罗兀的时候,当是不会再有能阻止广锐军的实力来。

韩冈也不由暗叹着,广锐军当真本事,几万将士拼杀了许久,好不容易才挽回出来的局面,在他们举起叛旗的那一刻,就已经彻底破局。

‘张玉不知道会怎么想?如果有他坐镇庆州,这场兵变不一定能闹得起来!’韩冈又看向张玉。老将花白的浓眉下,一对看起来很和气的眼睛半眯着,看不出有什么不对。

话说回来,广锐军叛乱的原因虽然没有明说,但韩冈也能猜想得到。从战马被韩绛夺去给蕃人,到在军中深受尊敬的吴逵被下狱,也许还有最近被逼着要出兵牵制西贼,每一条,都是火上浇油,让原本就不算恭顺的广锐军终于变得彻底疯狂。

安静得落针可闻的主帐中,只有沉重的呼吸声。每个人仿佛都把沉默是金当作了座右铭。但他们的心理都在转着同样的念头,“还是早点退兵吧!”

现在的关键是怎么才能顺利的离开。要放弃罗兀城,必须先得到朝中的准许,否则失土的罪名,没人能承担得起。韩绛和种谔,就算肯承认失败,也绝不会在天子没有点头的情况下,主动下令撤离。而以鄜延和东京之间的金牌急脚递的速度,罗兀城中的大军,想等到撤退的命令,至少还要六七天的时间。

可城外还有党项人,现在他们的攻势稍减,但不代表梁乙埋会在得到庆州叛乱的消息后,依然采取现在的消极态势。以党项人在关中的耳目,梁乙埋收到这个喜信,也只是数日间的事。如果不能在这之前离开,再想走,难度就要大上十倍。把鄜延、环庆两路的精锐一举荡清,这个诱惑,没人会认为梁乙埋能忍得住。

在场的可都是聪明人,想通这么简单的道理并不困难。但不是每人都能想出顺利退兵的主意,你瞥一眼我,我瞟一眼你,皆希望别人先出头。只是谁也不肯先开口,在得到朝中允许之前,在得到宣抚司准许之前,先行提出放弃罗兀,必然会成为众矢之的。

韩冈自然也想走,罗兀城已经成了一艘撞上冰山的海船,随时都有倾覆的危险,他可没有与之偕亡的想法。

一场大戏在近处看的确有趣,但把自己的小命也搭进去,韩冈却敬谢不敏。因为韩绛对缘故,韩冈自抵达绥德种谔麾下之后,从不干预军事,但眼下的情况,却是给了他一个机会。

‘韩相公啊韩相公,你千不该,万不该,就不该让我等到这个机会啊!’在危局之前,韩冈私心中却是有些兴奋。

张玉和高永能已经等了一阵,见没有人说话,对视一眼,就要宣布散会。究竟后续该如何处理,他们也不能立刻做出决断。而且谋不决于众人,现在只是通报消息而已,一些必要的应对还要由他们两人私下里来商议。

韩冈这时站了出来,拱手行礼,阻止了高永能宣布散会:“张总管,高监押,韩冈有一事想说。”

这是韩冈第一次在军议上插话,帐中众人纷纷侧目,心道难道他要做第一个?

张玉一皱眉,想要阻止韩冈。而高永能却先了他一步,“韩冈,你有什么话要说?”

韩冈朗声道:“今日还请大军照常出城邀战。不论接下来是走是留,我们现在面临的问题,都不是能让西贼知道的。”带着一点挑衅味道的眼神,在众人脸上一划而过,他用重音强调着:“必须要一切如常!”

韩冈的口气稍显强硬,不顾尊卑之别,但因韩冈的话而沉思起来的张玉和高永能却没有为此而恼火。他的话就像当头棒喝,一下提醒了两人。

这两天的出城邀战,由于西贼不算配合,都是应付故事一般,两边派兵打上一回。以兵法来说,守城最忌闷守,围城也忌讳闷围,为士气之故而已。两边又都不肯放弃,而在等待时机,所以才会如此滑稽的场面。韩冈看史书上,经常有一围经年的战事,究其因,也是因为这个缘故。而现在,机会是给党项人等到了,但却绝不能让他们知道。

高永能当即转头对张玉道:“下官现在就领兵出城邀战,还请总管坐镇城中!”

张玉点了点头,又厉声对帐中官员下令道:“今日之事,要严加保密,否则便有全城尽墨之忧!”

“下官谨遵命。”“末将遵命。”众官纷纷恭声应是,事关自家性命,容不得他们不小心。

‘西贼到底在等什么?!’韩冈不认为梁乙埋能事先猜得到庆州会有兵变,而他派兵阻断罗兀后路的行动,成效又不显著。而要拼毅力,也不是党项人能拼得起的。这样的情况下,他还在等什么?

在战鼓声中,回到疗养院之后,他还是在想这个问题。

经过了韩冈悉心的管理,疗养院内外之事已经井井有条。依照他和郭逵在秦凤推行的军中医工方案,这些天韩冈在高永能和张玉的支持下,罗兀城中的每一个百人都,都派了一个头脑聪敏伶俐的士兵来疗养院里实习,并学习基本的战场急救。所以现在韩冈反到是稍显轻松起来,只要发派命令,有时间想些事情。

但张玉却找了过来,呶呶嘴,把正在向韩冈汇报公事的护工队正赶了出去。直接问道:“玉昆,今次之事你怎么看?”

张玉想征求一下韩冈有何高见,而韩冈却指了指外间躺满了病房中的伤兵们,“是该问他们怎么办?……总得把他们送回去!”

“玉昆?”张玉微微一愣,不知道韩冈为何如此说。

“前日种帅从罗兀回军,就是以护送伤兵的名义。不论是从情理上说,还是道理上,伤兵先行离开罗兀,并不会引起城中军心慌乱,也不用担心被秋后算帐。当然……”韩冈又加了一句,“为了避嫌,我可以最后再走。但须得先把他们送出去。好不容易救回来了,总不能看着他们被丢下等死。”

敌前撤退,难上加难,纯用骑兵,撤回绥德不难。但加上城中的步兵,就很麻烦了。如果再有行动不便的伤病,那就是难上加难。正常的情况下,他们肯定要被抛下。韩冈要救人,他在鄜延军中费心费力才留下的人脉,不能就这么浪费掉。而且这些天跟伤兵们朝夕相处,也不忍心眼睁睁的看着他们被抛弃。

张玉不意韩冈有如此仁心,不过又想想,若不是韩冈有此心境,如何能在军中医疗之事上自出机杼,而且自来到罗兀后,韩冈的辛苦他也看在眼里。

“玉昆果然仁义。”张玉由衷的赞了韩冈一句。坐下来又长叹起:“其实,本也不会变得如此仓皇。如果没有广锐军叛乱,这次完全可以彻底解决西贼的问题。让党项人不能再越横山一步。”

得到了横山,就是得到了银夏,有了银州夏州,就可以跟占据了兴灵——也就是后世的宁夏银川——的党项人隔着瀚海对峙。前线北移到横山对面数百里的地方,环庆和鄜延两路自此便可以安心的休养生息。

张玉跟西夏人打了几十年,当然想在致仕前为毕生的心愿做个了断。可如今功败垂成,而且因为是叛乱的缘故,为防重蹈覆辙,至少数年之内,大宋都只能稳守疆界,以稳定内部为上。张玉当然失望!

“广锐军兵变,岂是他们自己愿意的?根子在谁身上,总管当比韩冈要清楚。”韩冈言辞锋锐,“不过事已至此,再后悔也无济于事。罗兀城保不住了,但为了能安然离开,城外的敌军却还是要设法处理一下的。”

