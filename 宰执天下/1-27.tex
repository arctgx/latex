\section{第14章 辘辘尘道犯胡兵(上)}

从秦州往陇城县的官道长三十里,宽四丈,顺着藉水修筑,厚厚的黄土夯筑得坚硬如石,是秦州向东连接凤翔府,直通关中的主要通道。如此宽阔的道路,足以容下八匹马或是四辆车齐头并行,也能容纳每年从关中腹地向秦州运来的三十万石粮秣通行。但现在,韩冈和他的辎重车队却都是站在官道旁的泥地上,等待这条官道重新开放。

一对对旗牌官,各自举着旗号、官牌赞导喝道,后面则跟着数百名戴盔披甲的骑兵迤逦而行。骑兵分前后两部,护持着中间的一支三百多人、服色参杂的队伍。

这一整条队列从头到尾有近一里长,人数大约七八百。只看其中带甲骑兵的数目,少说也有一个指挥的兵力。秦州虽是前线,但骑兵始终不多——或者说,整个大宋的骑兵数量都是少得可怜——秦州连着蕃兵、汉军一起算上,也不过五千上下。而现下在韩冈面前鱼贯而过的队伍,就占了其中的十分之一。

“是李相公回来了!”

“是经略李相公!”

不是一路经略的身份,如何能以数百名骑兵为护卫?的确是李师中回来了。

秦凤路的经略相公为了就近调配输送给笼竿城和甘谷城的军需物资,他在陇城县上——也就是韩冈去甘谷城这条路的第一站——整整待了半个月之久,直到此时,方才回镇治所。

李师中位高权重权势,其人出行自是闲人远避。虽不像天子出巡要沿途人家摆起香案、山呼叩拜,但远趋避道,却是少不了的。

‘要是他能早几天从陇城县回来就好了。’韩冈心中不无遗憾的想着。

李师中的的性格为人,州中多有传言,那是拢着权力不肯放手的性子,同时为人刻薄,近于酷吏。德贤坊军器库之案如是落到他手上,铁定给他办成株连数十家上百家的大案,成纪县连句嘴都别想插上。陈举也肯定逃不过这一劫。而陈举垮台,韩冈现在就应该已经回到藉水对面的家中,让小萝莉为自己暖被窝了。

‘回来得实在太晚了!’

“好威风……”看着李师中的队列,王舜臣则是另外一种心情。

“这不是当然的?!秦凤经略相公啊,天下文官武官数以万计,但在他之上的也没多少。如果入朝,再升一步便是一任宰执了。”

虽然如此回复,但站在路边,韩冈看着浩浩荡荡的护卫着李师中的骑兵队伍,心中照样有种说不清道不明的味道。半是羡慕,半是渴望。羡慕他的权势,渴望的也是李师中现在拥有的权势。

能做秦凤路经略使,在大宋文官序列内,说起来应该能排进最前面的三五十人之列了。大宋的地方行政区划,从下到上是镇(乡)、县(羁縻州)、州(府军监)、路(京)这四级,其中路是最大的区划单位。

路有转运使路和经略安抚使路的区别,转运使路整个大宋才分了十五路,而后才加到十八路,经略安抚使路多一点,也没超过二十五。而不论是转运使路还是经略安抚使路,其序列都是北方排在南方之前。而如今西北多战事,关西四路以及河东一路尤为重要,李师中的地位,在天下二十多个经略安抚使中,其实是排在前五的。

看着身着紫袍的李师中气势轩昂的骑在一匹高俊的枣红色河西良马上,在众军的护持下从眼前穿行而过。韩冈神思突然间有些恍惚,究竟是在什么时候,汉人的文吏虚弱得连马背也爬不上去了呢?

在前世,韩冈总是以为文官乘轿,武官骑马是古代的惯例。但在这个时代,连文官也多是骑马,少有坐轿乘肩舆的。以人为畜,名声上殊不好听。就算是宰相,除非是年老腿脚不便,得到天子特旨赐以肩舆,否则也一样是骑着马入宫。

——这还是修文偃武的宋代!而且还是北方的优良养马地皆尽丧失,战马数量不足的宋代!而明清,不缺地,不缺马,文官们却都是以人为畜,不坐轿子就走不了路。

这该叫做一代不如一代吧!

班超手上只有三十六人,却也是敢在敌国杀人放火。王玄策据说单人匹马就带领附庸国的军队击败了一个印度古国。

虽然宋朝的尚武之风远不如汉唐,但书生至少还是能骑马,也能拉弓——韩冈自己的箭术就不错,他在张载门下游学时,也有过几次在初春与同学一起射柳【注1】的经验,而真宗朝的状元陈尧咨更是以箭术闻名天下,还留下了一段熟能生巧的典故来——但到了明清,多少读书人好像只能拿扇子,玩兔子了。

李师中的队列已经走远,只看着一条尘龙滚滚西去。被逼到路边的民伕们纷纷把骡车赶上官道,王舜臣来到韩冈身边,“韩秀才,该走了!”

韩冈回神过来,对王舜臣歉然一笑。

他再回头,望着滚滚的尘尾。这就是一名经略使的权势。论才智,他不认为自己会输人,论刻苦,不论是他还是前身,都是能一心苦读的人物,论眼光、论学识,韩冈更是自信。只要有机会,不论是去参加科举,还是得人荐举,他如何不能在北宋混出头来?

虽是无缘无故的来到这个时代,但韩冈怎甘心浑浑噩噩的过上一辈子?不论叫野心也好,雄心也好,他的眼界如今放得很高!

总有一天,他会站在比李师中还要高的地方。

总有一天……

……………………

韩冈带队重新上路,不过两个时辰,一行人便赶到了陇城县中。照着惯例,他们被安排着在县城外的一座旧军营中歇了下来。王舜臣虽然跟韩冈带的辎重队不是一家,董超又与营门守卫咬了半天耳朵,想堵着不让王舜臣入内。但王舜臣拿着吴衍开出来的关文令扎——但更有用的还是他的那根马鞭——也大摇大摆的一起入了营。

此时还未交申时,但冬天天色黑的早,日头已然西垂,半幅天穹都泛着血红。

安排着吃了饭,四十多人便占了两间营房,一边二十人挤在两张大通铺上。韩冈用着看管民伕的名义,把薛廿八和董超两个分开来各安顿在一间房中,他自己和王舜臣则分睡在两座营房外间的军官专用厢房内。

“记住了,这是军营,不是惠民桥后的私窠子【注2】,没得让你们进进出出!入夜后无令不得出房,要是给洒家捉到,老大军棍伺候,别以为洒家不敢打断你们这些猴崽子的腿!”

王舜臣板着脸站在营房中,他威风凛凛的教训着一众民伕,三十多人老老实实的站成两排低头听教。按理说辎重队的领队是韩冈,而王舜臣不过是顺路同行的外人,就算教训,也该韩冈出头。可韩冈就在旁边站着看着,而董超和薛廿八被逼着跟民伕们站在一起,只冷着脸,什么都没说。

韩冈瞧着两人的神色,有一半好似因为王舜臣背在身后的双手正用力捏着他的那柄马鞭,但更多的应该是想着后面把场子找回来,而在忍着一时之气。

王舜臣的条令并不是他私编出来。夜间私出军帐、营房,按照军法都是要打军棍。莫说到帐外透透气,就是想方便,也是要先得命令;没得命令,那就直接解在裤裆里。

韩冈对此军规倒是了解不深,但能帮着困住薛董二人,自不会有二话——如果薛廿八和董超敢犯军条,他绝对会乘机废掉两人的腿——何况这条令也不是用来约束他。先去检查了一下车辆,还有牲畜的食水,让值守的民伕好生的看管。而后韩冈又去了军营外。

附近的百姓都是惯会做生意的,靠山吃山,靠水吃水。靠着军营,那就做着里面过往军队的买卖。为了多谢王舜臣相助,韩冈在外面买了酒肉回来,吃饭聊天顺便拉拉关系——也多亏韩千六在临出发时,塞了一贯多一点的大小钱给他,不然也没钱做这些。

王舜臣的房间就在营房中隔出来的厢房中,这也是为了让军官和士兵不至于离得太远,也能监视到士兵们的进出。韩冈拎着酒肉过来,他也是高兴。不多说二话,两人在桌边坐下,便吃喝起来。

酒过三巡,韩冈抹了抹嘴上的油腥,正容向王舜臣谢道:“今日之事,真是多谢王军将了。”

韩冈真的很感激王舜臣,若不是有他在,今夜说不得自己就要先下手为强了,否则明天到了山道上,保不住会出什么幺蛾子来。吴节判做事也是妥当,让他直接出头他是绝对不干,可请他调一个可信的军官,他找来的王舜臣却不仅仅是可信,而且可靠。

注1:射柳,中国古代传统的春季游戏活动。不论汉人和胡人,到了春天柳树发芽,都有在校场上插柳枝,比赛射术的传统。君子六艺,礼、乐、射、御、书、数。射居第三。

注2:私窠子,就是私娼妓院,与教坊司官妓相对。

ps:不知什么时候才能站到新书榜的第一位,俺会努力的,各位兄弟要多多支持啊。今天第三更。

