\section{第38章 何与君王分重轻(17)}

向皇后牵着赵佣的手,跟在抬着赵顼的肩舆之后,从侧门进入前殿。

她从崇政殿赶过来,并没有耽误了经筵开启的时间。

虽然不知道丈夫为什么突然之间要重开经筵,可向皇后至少知道,官家绝不会是突然想读书了。

赵顼被扶上御座,向皇后也在一侧屏风后坐下。御座的另一侧,赵佣也落座,坐得端端正正。

王安石、韩冈、程颢,连同三馆成员,分左右立于殿下。

看到韩冈与王安石隔着殿中央分列东西,再看看下面的其他臣僚,向皇后脸色一沉,这果然是围剿。

回头怒视了丈夫一眼,怎么就有这么深的成见。一看到韩冈,就如临大敌。要不是当初有韩冈挺身而出,现在坐在集英殿中的,就是那个装疯卖傻的赵颢了。

向皇后满腹怨言,群臣这时候已经礼毕,在宋用臣的主持下,王、韩、程三人又谢恩落座。

经筵上,侍讲并不赐座,王安石当年初入经筵,曾经上表要求天子确立侍讲官坐而论道的资格,不过赵顼同意之后,他再上经筵,却多还是站着。

有此故事,之后的其他侍讲上经筵,同样都是站着为天子开讲,久了,赵顼也不再赐座。今天的集英殿上,则是又破例了。

韩冈大大方方的坐下来,等着皇帝的开场戏。

宋用臣又站上前台,手上拿着一卷绫纸,照着念道:“夫儒者,通天地人之理,明古今治乱之源……”

韩冈乍听,感觉上就颇像是聆听圣旨的味道。仁宗说过的话,鼓励文治,只是不如真宗的劝学诗流行。

他用余光瞅了瞅御座上用来固定天子身体的靠垫,赵顼口不能言,长篇大论也只能用手指写出来,倒是辛苦他了。

宋用臣絮絮念着:“……道术为百家裂,圣教为俗学弊……”

韩冈眼皮跳了一下,对面一下投过来十几道的目光。差不多都是要看他的笑话。

听到两句,在列的哪能还不明白天子想说什么?赵顼这是避开了直接议论韩冈昨天的课程,改而在经术上做文章。而且还是主张‘一道德’,不然就不会有‘道术为百家裂’一句了。

赵顼手脚不便,用指尖蘸着墨水所写的开场白很短,不过十几句话。抑扬顿挫的念过一通之后,宋用臣就代天子点起了王安石,“王卿作《三经新义》,训释经义,发明圣人作经大旨。布教化于九州,卿之功也。”

王安石连忙起身,颤声道:“臣有陛下,方得一展羽翼。”

“韩卿。十年间,外定四夷,内抚万姓。生民幼子多赖卿家得全。善莫大焉。”

赵顼这不能是称赞,韩冈能站在这里,不是因为他有出将入相的才干。几句话只擦了格物致知的边。韩冈却浑若不觉,也起身行礼:“臣得陛下简拔于草莽,不敢不用心于王事。”

“程卿之正,朕早已知之。论事不论人,程卿之后再无一御史有此德量。”

程颢同样起身拜谢。他看着若无其事,不过下面的吕大临脸色不好看。天子对新学可谓是一往情深。这不是拉偏架了,提都不提两家学问,根本不让韩冈和程颢有发挥的余地。

“三位卿家各有胜擅,故朕礼聘入资善堂中讲学。只是三位卿家在道理上各持一端。太子年幼,无所适从。‘惟精惟一’,道不纯,则心难正。士庶心不正,一家之祸。卿大夫心不正,朝堂州邑之祸。天子心不正,天下之祸。不知诸卿可有良策以教朕?”

这是谁弄出的问题?韩冈倒想问问赵官家,把自己和王安石、程颢一并招入资善堂,究竟是集英殿上的哪一个?!

程颢眉头也稍稍皱了一下,天子的话听起来就是要以新学教太子,无论是韩冈还是他程颢,都必须向新学低头。

王安石、韩冈、程颢在教书育人上的观点大都类似。三家都是义理一派,只是各自的理,或者说道,不一样罢了。但孟子的修齐治平,却是三家共同的依归。现在根本没必要这么做。

“陛下。‘片言可以折狱者’,子路一人也。正所谓兼听则明,偏听则暗。尧清问下民,故有苗之恶得以上闻。舜明四目,达四聪,故共、鲧、驩兜不能蔽也。治政如是,治学亦如是。当博学之,方能审问之,明辨之。”

进攻就是最好的防守,宋用臣话音刚落,韩冈就再一次站起了身。迫不及待,选择直接开战。不能顺着皇帝心意,要不然仗就难打了,“子曰:‘好古,敏以求之’。陛下循圣人之教,追崇唐虞之三代,不为不善。然时过境迁,礼法亦应时制宜。殷因于夏礼所损益,可知也;周因于殷礼损益,可知也。陛下欲追三代,不可不损益之。三代之治亦多有难行于世者,今当付之公论。”

蔡卞前面受了气,正等着韩冈,立刻反驳,“先王之道,仁也。先王之术,礼也。《周官新义》,明先王仁礼之本意。煌煌之作,烛照百世,何须再议与群氓?”他挑了一下眉,“卞敢问枢密,何者为应时制宜?”

“以先王之法考之,又以实验之。验之得实,又合先王之法,人情所顺,可为‘宜’也。非此,则悖于时。不说读周官要应时制宜,就是论语亦须如此。论语曰:君薨,百官总己听于冢宰三年。今曰可行否?”韩冈反问。

古时天子驾崩,新君要守制三年,这三年里,百官悉听命于宰相。这是孔夫子所说。

于今当然是不可行的。这不是出权臣的问题了,而是被篡位的危险了。如今天子服丧,皆以曰为月。哪里会将国政交托给大臣?

“自是不可。”不等蔡卞组织好言辞,韩冈就自问自答,“三代所行良政,于今已不可行。三代之国,国小而民寡,事不繁,讼不多,君王可垂拱而治。皇甫谧《帝王世纪》有载,禹之时,天下人口一千三百五十万。成王时,天下人口一千三百七十万。又裂土分疆,甸服不过五百里,五百里外封侯,千里之外,就得抚之绥之。广南鸟兽居,江左蛮夷地。冀北有狄,雍西有羌。王命难离黄河南北。可见国之小,民之寡。于今四百军州,疆域万里,人口以万万计,岂是三代时可比?小国寡民可以清静无为治之,而今疆土人口远过之,又如何不当应时制宜?”

“应时制宜,相时所变者,用也。其体当如一。”王安石以体用论回应韩冈,体,是本质,用,是表象,不论时代是否变了,根本和本质的东西是不会变的。他又转身面对赵顼:“臣奉陛下之命,作三经新义,一道德,变风俗,十余年来,小有成果。然如今风俗虽稍变,道德尤未一。臣虽老迈,不敢辞其责。但各家之说,亦有可取之处。诚不可弃,当择其善者而用之。”

赵顼的心意,王安石明白了。并不是要压制韩冈,这并不是聪明的做法,而是将他纳入体系之中。在重释经典的无穷多的争议中,将他的精力消耗殆尽,不再为患。

王安石方才确定了韩冈态度,不再有何犹豫,先配合把韩冈弄过来编书。《三经新义》不可更动,但五经之中还有《易》和《春秋》未解,慢慢跟他争好了。

新法难以撼动,新学又在国子监中成为钦定的教科书,想要改变这一切,根本不可能。王安石也不会像变法之初时那般,有不合己意的论调立刻加以攻击,要除之而后快。十几年的时间沉淀,已经给了他足够的自信。就是总能别出机杼的女婿,王安石也有信心让他心力耗尽。毕竟在五经之中,《易》和《春秋》是公认的麻烦。

“敢问平章,何者为善?”韩冈转身面对王安石,“孔子曰:尊德姓而道问学。治事当诚于实,论学、治学亦当以实验之。如若不实,不可称善。”

“枢密之实,可是道理之实?”蔡卞斗志满满,又率先反问,“枢密旧年曾经讲过以‘旁艺近大道’,如今再看,却将旁艺作大道。”

韩冈所倡导的学术,很难被经义所约束,实际上也完全跟经义挂不了钩。蔡卞毫不客气的指出了这一点,还把韩冈当年学业尚未有成时的话,当面丢了出来。这也不算是秘密,当年知道的人就不少,现在也早传开了。

“傅说,版筑之徒。为殷高相,国大治。其何以治国?技近乎道也。触类而旁通,举一而反三,于版筑间,治国之术已明。”

蔡卞冷笑了一声:“看来枢密觉得不需要读书了?”

“皋、夔、稷、契之时又有何书可读?”韩冈看了对面王安石一眼,王安石脸色黑了三分,韩冈是戳他的软肋。

当年王安石初入政事堂,与同列宰辅争论变法,曾‘公辈坐不读书耳’,当时同为参政的赵抃反驳道:‘君言失矣,皋、夔、稷、契之时,有何书可读’。堵得王安石一时没话说——尧舜和他们的臣子所在的时代,当然是不会有儒门经典,也就三坟、五典、八索、九丘而已。

不过韩冈紧接着又对蔡卞道,“圣人之所以为圣,就是因为圣人留下了《诗》、《书》、《礼》、《易》、《春秋》,使后人有书可读,贯通之后可明道理。自此世人有了通衢大道可走,不必辛辛苦苦从头自悟。只是当有了经典之后,却让世人少了应用。读书人姓情、智识、阅历迥然有异。对经典的理解也各不相同,这就是传注多歧的缘故。若想明辨其对错是非,就只能再以实验之。‘诚者,天之道也。诚之者,人之道也’。不能惑于传注,惟诚于实。”

两句孟子的话,也正是韩冈拿来做幌子的依仗。

“民胞物与,何如墨翟之言,不知父母所亲何在?”说话的是排在后面的陆佃,也是王安石的弟子,同在馆阁中,韩冈方才没有注意到他。

张载的爱必兼爱被说成是墨家,已经不是第一次了。但每次反驳起来都很费口舌,“大君,宗子也,大臣,家相也;‘长其长;幼其幼’。由近而远。有亲疏之别,上下之序,礼也。墨家兼爱,视父母路人如一,悖于常姓,非礼也。”

“‘乾称父,坤称母。大君者,吾父母宗子’,枢密亦天子呼?”
