\section{第28章 官近青云与天通(13)}

城南驿的驿丞周至都快要疯了,听见外面的传进来的一片喧嚣,他甚至觉得跟二大王一样发发疯其实也不错。

司马光到了,但王安石却还在驿馆中。

住在馆后上院中的王相公虽然前几天得了赐第,但还在整修中,一时间还没搬过去。现在司马光比预计的早到了两日,该怎么安置这位太子太师?

让他们住得门对门吗?

虽然是大冬天,但周至的头上身上依然是汗水直流。

即便司马光形同贬斥的在洛阳住了十余年,在朝堂上也是不受欢迎,但他的身份决不是区区一个驿丞能开罪得起的。可若是安排不好,让王相公觉得心头不痛快,那就更加危险了。

自家可不是进士,能得到现在这件官袍可不容易。做了三十年吏员才交了鸿运得了官身。天下百万胥吏,一年才十几二十人能从吏职升官,而且绝大多数还是给显贵们的亲信占了。周至不指望自己还能走第二次狗屎运。

若是现在发了疯,多半也就是提前致仕,说不定还能得个恩典。周至正在考虑是脱衣服裸奔,还是去茅厕里打个滚,派出去找顶头上司的人终于回来了。

“怎么样了?管勾可动身了?”周至一把将人给扯住,火烧火燎的仿佛当真火上房顶了。

“三叔,三叔。”被当胸扯定的驿卒在周至手中挣扎着,“侄儿去了赵管勾府上,但看门的军汉就说了,赵管勾有事出去了,不在家中。”

周至咬牙切齿,那名宗室出身的城南驿管勾官分明是知道要出事了,才避而不见,推说有事外出了。根本就是山里的兔子,听到点风声,觉得有危险,就登时往洞里钻。若是今天的事办得不妥当,办事不利的罪名当即就会被他推到自己的头上来。

“三叔。”被周至塞进城南驿做驿卒的侄子好不容易才从他的叔叔手中挣脱开来,又大着胆子催促着,“司马宫师可是已经在外面了。”

“难道我不知道!?”周至顿时暴怒,要不是知道司马光已经到了门外,他这么急做什么。

在房内绕了两个圈,一名亲信驿卒也进来了,通报说司马宫师的车马已经进了外院,然后眼巴巴的等着周至的吩咐。

“急什么,王相公还在里面呢。”周至站定了,咬着牙转头吩咐侄儿,“小七,快去通知王相公!”

“这个……”周家的小七犹豫起来,这样好吗?

见侄儿竟然还耽搁时间,周至兜头就是一耳光,“还不快去!胆子大了啊,连话都不听了!”

周至的侄儿捂着脸,也不敢回嘴,赶急赶忙的就往后面去通知王安石了。

周至连推带踹的将侄子赶去通知王安石,自己则整整衣冠,向外面走去。不管怎么说,让王相公自己来处理吧。至于最后会是什么结果,他认命好了。

外厅已是人头涌涌,不论京官选人,城南驿中百十名大小官员全都出来了。矜持一点的就在外间的大厅里找张座位坐下来,轻浮些的就站到院中去了。

谁不知道王安石和司马光是死对头,今天司马光上京,正好撞上王安石还在城南驿。两人十几年前就割席断交,王安石的《答司马谏议书》更是遍传天下,眼下撞个正着,还不得天雷普降地火丛生。

“可是司马宫师到了?”周至一踏出大门,立刻就换上了一副讨人好的笑脸。拿块擦桌布,换身短衫,就是活脱脱的跑堂小二。

司马光已经从车上下来了,不过他没兴趣跟这姗姗来迟的驿丞多话。抬头打量着沧海桑田一般变化巨大的城南驿,让自己带着上京的儿子司马康去处理一应事务。

司马光上京,身边没有带太多的伴当。就是一辆车,六匹马,出去属于驿馆的车夫,连司马光本人在一起也只有八人。但他惹起的动静,却跟带着一家老小的执政回京时还要大上三分。

周至与司马康办理入住和交接驿车驿马的手续,一点也不感到委屈。他巴不得司马光和王安石根本就不理会自己,当个屁放掉那就是最好了。

但司马光的儿子显然没有太多与驿站打交道的经验,尤其是城南驿作为天下驿途的终点和起点,手续要比路途上的驿站繁琐得多。弄了半天,还没有结束,而司马光已经有些不耐烦的看了过来。

周至脑袋都发白了,照规矩是不该让司马光这一级的重臣在外面等,而是先将人迎进厅中坐下来再说。但他今天竟然失措到给忘了。

惶恐之中,身后终于起了一阵骚动,周至紧绷的身子也一下放松了下来,王相公终究还是愿意出来见一见司马光。

王安石已经出现在了门口。在儿子王旁的陪伴下,他大步从外厅走了出来。

“君实。”他高声打着招呼,甚至是有几分惊喜,“久违了。”

“介甫,好久不见了。”司马光站定了,看着渐走渐近的老朋友,先行拱手一礼,“向来可好?”脸上的神色却是淡淡的,不见喜愠。

王安石立刻回礼,一揖到底,“君实,多年未曾谋面,。”

司马光古井不波的脸上,终于有了点笑容,“劳介甫挂念了。”

见王安石对司马光似乎毫无芥蒂,两人也并没有一见面便火花四射,周至也稍稍恢复了一点挥洒自如的跑堂本事,陪着笑脸将两人往厅中请:“相公,宫师,外面天冷,请先里面安坐。”

“说得也是。”王安石点点头,请司马光往里走,“君实天寒远来不易,还是先到里面暖和一下。”

周至忙忙请着两尊大佛和他们儿子进了厅,安排了一个清净且生了旺火的小厅,热酒热茶伺候。待两对父子安坐,方告辞出来。

‘总算是安生了。’

周至点头哈腰的悄步出厅,此时他背后已经汗水淋漓,冷冰冰贴着脊梁骨,寒意透骨。可终究还是过了这一关,心中也轻松了许多。

正准备办完剩下的手续,给司马光安排一个上院,但他的侄儿带着守门的驿卒又跑了过来,“三叔,韩资政来了。”

“韩资政?!”周至失声叫了起来,又立刻捂住嘴。他都想撞墙了,王相公的女婿来凑什么热闹,还来得这么快。

话音未落,韩冈已经笑吟吟的驿馆外进来:“可司马十二丈到了?”

“是……是……”周至的舌头发涩,指着内厅,“就在里面。王相公和司马宫师现在就在里面说话。”

韩冈已经将端明殿和龙图阁两个贴职全都辞掉了,现在就是单纯的资政殿学士兼翰林学士。本来韩冈甚至准备只留着端明殿一职去兼职玉堂——端明殿学士本就是给资历深的翰林学士的加衔。这样去迎接萧禧,礼节上正好合适。

但上面不答应,韩冈想想觉得谦让得太过分了也不合适,未免会让人往王莽的方面去想,还是该是什么就是什么比较好。以萧禧的身份,在礼仪上重视一点也没什么。

不过确切点说,这一位大宋君臣的老朋友已经不能叫萧禧了。因为辽章宗——也就是刚刚夭折的幼帝耶律延禧——的缘故,国书上是避讳改名为萧海里。说起来,韩冈觉得章宗这个庙号用在一个夭折的小儿身上实在是不合适。当然,耶律乙辛也不会在乎这一点。

韩冈是在来城南驿的路上听说了司马光的消息。他今天在放衙后来驿馆,明面上是帮王安石准备搬家的。不过实际上,他是想来跟王安石议论一下政事,主要就是想说一说司马光。

从御史们最近的兴奋,以及向皇后表现出来的能力来看,还是尽快排除旧党的干扰比较好。成事不足,败事有余,在外任州县中占了很大一块比例的旧党成员,很可能会给朝局带来不可知的变化。

虽然私心里,韩冈并不是没有早一点看一看旧党赤帜,名传千古的历史大家的想法——他当年任职京西,曾经拜访司马光而不果,这个心愿就一直留到了现在——但司马光比预计的早了两天进城,倒让他有些措手不及。王安石如果按计划在明天乔迁至赐第,根本就不会跟司马光在城南驿打上照面。

不过既然知道了王安石和司马光已经在里面坐着了,韩冈整了整衣冠,方迈步进厅。

周至眼睁睁的看着韩冈走进厅中,心道多担心也没用了,以里面三位的身份,应该不会吵起来吧。他并没有将司马康和王旁算进来,那两位衙内只是插花而已。

“韩资政来了,韩资政来了。”周至的侄儿又跑了过来。

周至瞪了他一眼,“韩资政刚进去了。”

“是许州的韩资政啊!”周至侄儿慌得一脸是汗,“他和吕枢密一起来的。”

韩维是知许州,也就是昨日才进京,不过是他是照常例诣阙进京,而且还是住在了兄弟韩缜的家里,根本就没来城南驿。

王安石、吕公著、司马光,再加一个韩维,在仁宗时,情谊甚笃,平日里多聚会于相国寺外僧坊中,人称嘉佑四友。

但王安石主持变法后,吕公著、司马光和韩维全都跟他翻了脸,现在三对一,不对,有个韩资政,那以一当二也不在话下。

