\section{第38章 何与君王分重轻(27)}

夜色下的天街比起京城中的其他去处,显得更为黯淡,也缺乏足够的人气。

宽达两百步的街道已经跟广场没有两样。不过天街中央,有占了近一半宽度的御道,这是天子出行时所走的道路,堆着厚厚的黄土。御道两侧,还有河渠,河渠之外,才是人们正常行走的道路。

真正说起来,供给章惇和他五十名元随行走的道路,不过三十余步。

就在御道对面,同样规模的队伍正在前进,与章惇一行齐头并进。

看了看灯笼的数量,章惇知道,那边应该是张璪。

知枢密院事和参知政事,都是五十名元随。只有宰相、枢密使才能有七十名元随跟随左右。而想要更多元随,要么做到宰相兼节度使的‘使相’,要么就是卸任的宰相得赐节度使,或是别的原因得到节度使的官职,才能达到百名元随。

而在章惇的前方,隔了半里,快要抵达宣德门下的那一队人马,灯笼的数量比河对岸少了近一半。可章惇知道,那不会是别人,而是王安石。

参知政事有五十名元随,而宰相视加衔与否,决定元随的数量是否有百人,至于平章军国重事,过去没有先例,但皇后特地下了恩赏,王安石拥有一百二十名元随,前后随行鼓吹、喝道。

一百二十名元随,比起天子出宫,动辄万人的盛况当然远远无法相比。可比起其他臣子却又是远远超过了。

要不是看着这一队人马所出来的路口,是王安石上朝的必经之路,章惇也猜不到前面是王安石。一百二十人的确太多了,临时都召集不起来,赶着入宫的王安石就这么只能带着四分之一的人手出来。

这远远比不上章惇能以军法治家,今天回来后,就让下面的元随随时等候吩咐,轮班值守。一等中使离开,就换上坐骑,直接奔向皇城而去。

自然,这个速度也跟章惇的元随,多是随他征战的亲兵所组成的缘故。换作是别人做同样的事,也难有这样的速度。

“不知能不能赶得急了。”章惇远眺宣德门,矮而厚重的城墙,也只有在月光下才能看到其中的意义。

前来传诏的使节没有多说什么,就是章惇让家人拦着,又封了一大笔好处,但到了最后,也还是没有得到任何更为精确的消息。

“子厚!”身后传来薛向的声音。他带着他的队伍,汇入到章惇队列中,而薛向本人,也挤到了章惇身侧,“在看什么?”

“看老鼠。”章惇左顾右盼。御街两边的街巷中,到处都能看到人影。

这些全都是来打探消息的。

天子第二次发病——也可能是第三次——有点常识的官员都知道,天子原本就跟快烧到底的蜡烛一样的生命,已经到了尽头,灯芯和烛油都在火中了,也许就在下一刻。既然如此,天子的病情也就没有什么好隐瞒的,只用了半个下午,就传得到处都是。

自然,其中肯定会有人打探更进一步的消息。而消息的来源,只有御街之上。

说起来,这也算是京城的一道风景。

每当皇城之内成为动荡之地的时候,都有许多老鼠感受到了洪水将至的信号,一起跑来打探消息,以便能跳上船,不至于沉溺于之后灾难。

开宝寺铁塔的黝黝暗影正嵌在东北面的天空之中。也许再过片刻,全京城的钟声都要开始响起。

宣德门渐渐近了,薛向忽然回头,看了几眼,对章惇道:“韩子华也来了。”

“子华相公府离得不算远,还以为他早就进宣德门了,想不到却是最后才姗姗来迟。”

“他是首相嘛。”薛向又道:“前面是王介甫,对面是张邃明,后面还有个韩子华。再加你我,人是都到齐了。就不知道,到了福宁殿,会是什么事。”

“多不过是拜太子。”

“多半是。”薛向点头,在他看来,也不会是其他事了,“不过今夜宿卫宫中的是蔡持正和曾子宣,有他们两个在,若当真是天子大行,说不定直接就封了皇城,明天早朝时把太子推上来了。”

“所以不是留了韩玉昆嘛。”章惇笑道,“韩玉昆现在都不缺什么了,正经是有东西大家分。”

蔡确、曾布的为人品姓大家都是知道的。遇上帝位传承时,他们的想法也是不难猜的。今天决定宿卫顺序时之所以没跟他们争,只是觉得天子不至于就在当夜出事,只是没料到竟然当真出事了。

不过既然早已成了定策元勋的韩冈在宫中,章惇就不需要担心。不说韩冈的为人,就是凭他的头脑,都不会让蔡确、曾布独占好处,自己却一并受到其余宰辅的敌视。

天子可能已经晏驾,但章惇和薛向却是口气轻松。

对于天子大行,他们已经没有太多的感慨了。在皇后垂帘听政半载之后,国势大涨,百姓安定,皇帝存在与否,都无法影响到天下局势。

而皇帝的死,在大多数人的心目中,最多也只是叹一句‘终于走了’。

说是君父,可当真能当父亲看吗?怎么可能能做到如丧考妣?

在梓宫前嚎上两声就已经很给面子了。最多也只能学西晋羊志,对着殷贵妃的坟茔自哭亡妻【注1】。

站在宣德门下,章惇觉得,他现在要考虑的,是接下来自己的定位了。

……………………王安石感觉将这辈子还剩下的吃惊都用到了这一桩案子上。

一张纸条还在宰辅们手中传递。而宰辅们脸上的表情,也随着纸条到手,而变得冰冷起来。

天子没有继续昏睡,更没有就此远离尘嚣,他一清醒就开始在沙盘上写字,一点也看不出他刚刚从昏迷中走了出来。

这是好事,可是并不值得王安石为他高兴。

毕竟以赵顼现在的表现,已经不能算是一个合格的皇帝了。

天子醒过来是桩喜事,但第一句写下来的话,就是‘皇后害我’。

这基本上就是一个误解,可是这也确定了天子对皇后的成见已经根深蒂固。

如此一来,这就让王安石,必须在天子和皇后之间做出一个选择。

不仅仅是王安石,所有的臣子都必须做一个选择,究竟是支持皇后,还是站在天子这一边。

“内禅。”蔡确当先表明自己的意见:“官家的病情现在很明显,以他现在的情况,也只能做太上皇了”

“太子才六岁,可以即位吗?”张璪反问。

曾布回复道:“又不是让太子主政,依然是皇后垂帘。如章献明肃待仁宗。”

“本朝并无太上皇例,这第一次怎么做?”薛向出言质问。

“先让太子登基,其他事,什么时候都可以去做。”

“这像什么,哪有这样的做法!”

宰辅们聚在一起议论着,韩冈虽在其中,却不曾开口。正如很多人走知道的,他不需要再卖气力,相反地,可以坐着看别人辛辛苦苦去寻找机会杨戬探头进来,小声道:“平章,列位相公,官家醒了!”

争议戛然而止。

天子虽然在他们心目中离死不远,甚至已经死亡,可实际上赵顼醒来,宰辅们还是多想听一听赵顼还有什么话要说。

‘是、朕、不、是’。

赵顼就在沙盘上,写下了让人心惊肉跳的文字。

这算什么?想要重新做人不成?

韩冈瞥眼看着赵顼,这时候示弱,向着皇后道歉,其中有多少是属于心机,又有多少是对失去一切的恐惧。

韩冈无动于衷,赵顼的问题不是一份罪己诏就能解决。最大的问题是朝臣们对其不再有信心。

之前就算是经历过一次中风,赵顼依然能保证朝政的顺利运行。这是赵顼加上宰辅的功劳,但现在赵顼已经不能让宰辅们再对他的吩咐言听计从。每一名宰辅,在接到他的诏书后,第一个念头是想要确定真伪,以及皇后对此的态度。

赵顼对此还没有清醒的认识,但朝臣们已经知道该怎么做了。

曾布上前半步,“陛下御体违和多曰,今又疾作,为防万一,还请陛下下诏传位于太子。”

‘我、无、事。’

皇帝在臣子面前,少有用我做自称,只有寻常面对家人,才会用一用。现在这是真的急怒在心,什么都不管不顾了。

可正是这般模样,让群臣对其再无一丝畏惧。

蔡确就在床榻边跪了下来,“陛下,当以天下为重啊!”

有蔡确领头,其他宰辅们过去叩头,“陛下,请以天下为重!”

赵顼闭着眼睛不理会。

王安石早就躲到了外殿内,不想与赵顼打照面,向皇后更是没再出面,而是守在偏殿中。

韩冈阻止了想要进一步规劝赵顼的蔡确:“当真想气坏天子?”

直接给赵顼灌下了医生掺了罂粟的汤药,赵顼根本无力反抗,很快又再次睡去。

‘到底怎么办?’好几位宰辅心中都缠绕着这个问题。怎么办才能顺利的让天子退位,并将还是小孩子的太子顶上来。

他们得到了答案,一切都要先说服皇后主动出面。

“殿下?”

“你们找官家,不要找吾。”向皇后避之唯恐不及。

“殿下,可知何事为重?!”

皇后是小君,但小君亦是君。现如今垂帘听政,行事就必须将天下放在最前面。

“殿下,当以国事为重!”

