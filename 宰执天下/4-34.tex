\section{第七章 都中久居何日去(二)}

六月正是一年中最热的时候,行走在没有树荫遮挡的街巷中,汗水流出来,转眼就能给晒干掉。一杯水翻在地上,转过脸来就没了踪影。

但就在连蝉声都变得稀稀落落的时候,忽然有两个一胖一瘦的士子小跑着穿过内城西面的郑门。

“快点!快点!”瘦削的士子跑在前面,穿过门洞后,还对身后喊着。

稍胖一点的儒生跑得呼哧带喘,身上的衣襟都给汗水湿透了,连回话的气力都没有,但他的脚步一点不停,低着头,只往前冲着。

看两人身上的穿戴,都不是富裕,雇不起车马也正常。但身为士子,不方规矩步的徐步前行,这样大呼小叫的穿街过巷,按说应该引得人人侧目才是。但沿路的商铺行人,最多的也只是抬头看上几眼,便毫不在意的收回了目光。在这条临近吴起庙的街道上,这样小跑着招摇过市的士子早已是不足为奇,惹不来路边上惊讶的目光。

“又是两个迟到的。”一名开着书画铺子的掌柜摇着手上折扇。

隔壁同样是书画铺子的掌柜也在幸灾乐祸:“迟了这么久,看来连门都别想挤进去了。”

“今天是横渠先生亲自出来讲学,那一天不是几百人早早的就来守着,拖到现在才到,肯定是没地方站了。”这条街上全是卖字画的商铺,正摇头笑着的第三人,也同样是书画铺子的掌柜。

这些天来,他们店里的书画没卖出去多少,但附带的笔墨纸张却是突然间畅销了起来。对给他们带来生意的源头,几个掌柜当然都是心里有数,也是暗自感激在心中。

一胖一瘦的两名士子气喘吁吁的冲进吴起庙中,也不看正殿的神像,直接转去西院。这样行为,连庙祝对此也都习以为常,没有出手拦着他们。

一走进西院,一个虽然苍老但依然清晰的声音便传入两人耳中:“蒙何以有亨?以九二之亨行蒙者之时中,此所以蒙得亨也。蒙无遽亨之理,以九二循循行时中之亨也。”

听见张载解说易经中的‘蒙亨,以亨行时中也’这一段,两人跌足失声。东京城中的士子,现在都知道张载聚毕生所学的著作是以‘正蒙’二字为题,而正蒙之名的来源,就是出自蒙卦。这么重要的讲学,竟然没有听到全文,两人都是后悔不迭。

“怎么都开始了……”

“都是你出门前硬是要换身衣服。”

“你若是起早一点,就是换两身衣服都不会迟到。”

两名年轻的士子一边小声的抱怨着对方耽搁了时间,一边轻手轻脚的打算往西厅里挤进去。可是走到门前,才发现厅中早已站满了学生,别说落脚,连个插针的地方都没有。只是这一百多人都在全神贯注的聆听横渠先生授业,安静得连声咳嗽都没有,让两人直到走到门前才惊觉。

两人面面相觑,谁能想到只是出门时耽搁了片刻,就连落脚的地方都没了。想离开,但听着里面传出来的讲课声,又是心痒难耐、难以舍弃。也没做太多犹豫,两人就站在门外,竖着耳朵旁听起来。

熙宁二年的时候,张载入京任职,那时就是受赵顼看重的臣子。只是因为不附和新法,加上其弟张戬做御史时弹劾王安石,才辞了官位,退居关中著书授徒。如今重回东京,前日受命入宫觐见天子,因为应对得当,当场就又擢了史馆修撰,负责编修日历。

所谓日历,是史官对国家、宫廷大事和天子言行的记录,按日记载,依照年月编订集合,是日后编纂国史的主要的依据。张载得此馆职,比起之前的集贤校理又高了一层。

不过如今东京城中的士子,都不用官名来称呼张载,绝大多数都是恭称一声横渠先生。

张载在崇文院中的工作很清闲,编修日历不仅仅是他一个人的工作。得以有闲暇继续授徒,就在开封府学讲学,京城士子对此趋之若鹜。

当年张载在相国寺设虎皮椅讲易,被他的两个表侄给驳倒了,第二天就回转关中。但现在经过了这么些年的钻研,张载对儒学经典早已经融会贯通。换到如今,已然自成体系的气学理论,想要将之驳倒,决不是那么容易的事了。

再加上张载的弟子韩冈,以实物为凭证,为格物致知四个字创下了偌大的名头。任何一家学派想要与气学争锋,就必须从飞船的顶上越过去——这个难度可想而知——而想绕道而行,避而不谈,也瞒不过明眼人,免不了会被人视为心虚。

既然没人有这个把握,当然就不见有人跳出来打擂台。所以这些日子张载和几个得意门生,借了郑门附近的吴起庙中的场地讲学,便是顺顺当当没有半点干扰。

而韩冈这边,也尽量抽空去聆听教诲。恭恭敬敬的跪坐在讲堂中,老老实实的记着笔记。有了声名远布的韩玉昆这个姿态,同在一个课堂中的士子们,当然就更加对张载的传授认真起来。

只是张载所在的崇文院是清要之所,而韩冈的军器监却是紧要之地。事情多而杂,千头万绪且互相关联。一个工坊出了问题,处理不好,就会连带着数个相关工坊一起出乱子。

不过以韩冈的能力,如果仅仅是处置日常事务,差不多也就一两个时辰的问题。当初吕惠卿身兼多职,照样做得轻松愉快。在治政上,经验逐步累积的韩冈并不会输他多少,可问题是现今军器监一是要设立新厂区,另一个还要保证板甲的顺利打造,加上韩冈还有各项发明要实验、要推广,也只能隔三差五的去一次张载的课堂。

另外最近,韩冈要负责军备的任务因为局势变动,一下又重了许多。种谔任了鄜延路兵马副总管,又开始调集西军中精兵强将,这件事所代表的一切,大大加重韩冈的负担。

种谔是军中最好战的一派的代表,他返回鄜延路,吴充曾出言阻拦过,但没能成功。王安石回来之后,东西二府的宰执们又重新生活在他的阴影之下,就像是参天巨树下的草木,受不到多少阳光雨露的滋润。

王安石究竟是什么心思?许多人都在揣测着。

依照熙宁三年的例子,如果当真要攻取横山,肯定会让宰执级的高官去主持此事。指挥全军的大权,绝不会留在武将的手中。

如今两府宰执中,王安石不可能出外,那么领军的人选到底会是谁?这个问题,在大大小小的酒店、茶馆中都有人讨论着。皇城脚下的百姓,就算事不关己,也喜欢拿着朝堂上的变化来当做下酒的小菜。

虽然攻取横山的战争根本还没有得到最终的确认,但为了主帅的人选,坊间多了许多猜测,也让酒家、茶舍多了许多收入,甚至私下里,都有人为此设了赌局。

“韩相公如何?”

有人提着当今次相的名字,却顿时引起一阵哄堂大笑。

“熙宁三年他若是硬气一点,罗兀城不会丢,横山也早就夺下来了。他在西军中可没有留下好名声,有几个赤佬还会听他的话?到时难道要用刀子来立威不成?……换作官家也不能放心啊!”

“冯相公?”

“更不可能。”有人又嘲笑起来,“当日不就是他在天子面前一力反对种谔去鄜延路吗?”

“吕参政?”

听到这个名字,有人沉吟,有人点头,但还是反对者更多一点:“吕参政倒是有些希望,但他毕竟没有领过兵啊!军中没人服他,官家也一样会担心。”

的确,从没有统领大军的经验,是吕惠卿的致命伤。万一指挥失措,少不了就是一场大败,马谡、赵括的例子就在前面。

吕惠卿被否了,枢密使吴充的名字也没人提了。虽然是管着大宋的百万大军,但他同样也没有统领大军的经验,加上他又是种谔就任鄜延路的反对者。任谁都知道,天子肯定不会点了他去。

只有两位副使,不论王韶和蔡挺,得到的认同最多。两人都是经验丰富的主帅,尤其是王韶,“其开疆拓土之功,真宗皇帝以来数他第一,不选王副枢去,还能选谁?”

“蔡副枢也不比王副枢差。他镇守泾原路多年,党项人有几个在他们面前逃过好去的?”

“眼下是要攻,不是要守。蔡副枢善守不善攻,要攻横山,换了王副枢才差不多。”

“还是蔡副枢资望更高一点,王副枢就要差一点。以种五的脾性,可是那么好使唤的?”

领军出征的究竟是王韶,还是蔡挺,一时争论不下。拜这争论所赐,东京百万军民差不多都知道关中又要打仗了。

这一现状,不知有多少人感叹过。皇城就是个筛子,再是如何的机密军情,转头来都能给泄露出去,根本都没有半点秘密可言。不过在大宋君臣看来,如果让西夏人紧张起来,也不是坏事。

