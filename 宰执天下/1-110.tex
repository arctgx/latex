\section{第42章 诡谋暗计何曾伤(三)}

【今天第一更,红票,收藏。】

流内铨的衙门,就位于宫城内,这是因为流内铨本就是中书门下的下属机构,自然不能离着政事堂太远。自从日前过来递过家状后,韩冈天天来流内铨报道,熟门熟路。从右掖门查验了身份后进入宫城。正面的文德门过去,就是每月举行朔望大朝会的文德殿。而韩冈要去的地方,则是要再往西,处于大宋的政治军事中枢——别称政事堂的中书门下和枢密院的合称也正巧就是中枢。

流内铨衙门前有凉亭一座,号为阙亭,但这个阙不是宫阙,而是官阙。亭子也并不让人歇脚,是为张榜所用。就在亭中,并排着挂了一圈水牌,有十几块之多。上面贴满了近日在流内铨登记过、尚未注人的官阙单子,以示公正之意。

这等自撇清的做法,究其因,还是因为如今官场上是僧多粥少,主管低品武臣的三班院中总有三五百个闲官,而统管选人的流内铨之下,同样有着三五百人。天下官阙不过一万多,而文武官员加起来超过两万。一个好官阙,总是引来多少闲官争抢。有多少人自入官以来,一直没能等到个好差遣,更是心中不耐。

可韩冈完全不需要等,从张守约、王韶,到天子赵顼和王安石。都为他的差遣尽了自己的一份心力,即便参加铨选,也只是照规矩要走个过场——这是昨日,接待他的一位小吏所言,还说是因为主考的刘令丞不便在考前见面,所以让他转告。不过韩冈一向谨慎,并没有因为一句陌生人的话而放松心情,知己知彼百战不殆,是他一贯的行事准则。昨日他便特意从程颢和张戬那里问了不少消息,也清楚了铨选的大致内容。

武官姑且不论,文官铨选大致分为两种。一种是选人改官,从地方幕职改为京官。另一种是新进选人注官,是新进官员进入官场的考试。

如果是选人改官,照例要判案四道。成绩合格者,方能改为京官。这是为了测试被考者的政务处理能力。因为由选人转为京官后,便可以出任知县、通判甚至知军知州这样的亲民官。亲民官集行政、民政、司法甚至军事于一体,是国家政权的支柱,必须要检验一下他们署理公事之才是否能胜任这一关系重大的职务。

相对而言,初出官选人的铨选难度就低了很多,如果是有出身,如进士科或是制举,就没有铨选,直接授职。剩下需要参加铨选的,大部分都是荫补官。集中在这个档次的荫补官,虽然他们的官品不高,但身后都有着一个或几个高品的父兄亲族,为难他们,等于是找不自在,所以考试的难度很低。

韩冈从程颢和张戬打听来的消息就这么多,但具体的考试科目他们却没提,只说让他按照参加明经科考试来复习就行了——韩冈不通诗赋,这一事几天来已经被他们看透了。

在守在流内铨门房中的一众闲官们又羡又妒的眼光中,韩冈被一名小吏领进了衙门。不过他没有被带进主厅,而转了几转,到了一间偏厅中。

厅内只有两名身穿青袍的文官。韩冈猜测,其中一个应是昨天传话给自己的刘令丞,另一人跟他平齐坐着,应是同一级别的官员,难道他是流内铨的主官?

走进厅中同时,韩冈心中隐隐觉得有些不对。他昨夜听张戬说过,初出官选人的铨叙都是要由一名两制官来监考,也就是翰林学士或是中书舍人。而以两制官的阶级,都是司马光、王珪那个等级的人物,有哪个没有一身朱袍穿,腰间没有金鱼袋?更何况怎么才他一个人来,应该是一批人一起考试才对!

“刘令丞,程令丞,秦州待铨选人韩冈带到。”吏人禀报了一声便退了出去。

证实了两人的身份,韩冈更加疑惑了。流内铨的主官是判流内铨事,而张戬昨日也说了,判流内铨的秘阁校理陈襄是正人,让他无需担心其他。但没有想到,那位陈校理并不在,而是两位令丞在候着他。

韩冈上前行了礼,低首垂眼的退后一步,等着两位流内铨令丞的发话。只是在他低下头的那一刻,两名流内铨令丞互相之间交换了一个眼神,脸上都多了一点忧色。

“韩冈?”刘易声音低沉。

“正是在下!”

“哪里人氏?”

“本贯密州胶西【今山东胶县】。出身秦州成纪。”

确认身份的对话,说了几句便结束了,单纯的走过场而已。放下手上的家状,刘易换上一副笑脸,“韩兄来京也有多日了,怕是等不及了吧?”

“不敢!”

“没什么敢不敢的!外面的一众官人天天骂,也不照样没事吗?”刘易哈哈的说笑了两句,不知为何笑声中有些发干,又道:“既然韩兄有天子特旨,这铨选也就走个过场而已。毕竟朝廷本有条贯在,无出身者必须考上一次,我等也不好违背。不过韩兄既然能得三人齐荐,又得王大参青眼,还让官家下了特旨,这才学自然是极好的。铨选连那些不成材的荫补衙内都能过关,韩兄自不必说了。”

“令丞过奖了,韩冈愧不敢当。”

“哪的话,是韩兄太自谦了!”刘易哈哈又笑起。

韩冈陪着一起轻轻笑了几声,但在他看来,此次铨选的迷雾却是越来越多了。这刘令丞是官场上的老油子,要看破他的心思,不是件简单的事。韩冈看着刘易,总觉得在他笑容中有着一点隐藏得很好的忧虑和困扰,这让韩冈怎么想也想不通。很快就很干脆的便放弃了。猜一个人怎么想,还不如看着他怎么做。从行动推断出目的和立场,可比察言观色准确得多。

“程兄,你怎么说?”刘易笑完,问着身边的人。

“是不是该开始了?”

“嗯,是该开始了!”

按唐朝的规矩,新官释褐,要经过四道审查,即所谓的‘身言书判’——相貌、谈吐、书法,以及判事的能力。而到了此时,虽然四项基本原则还是要讲,但检查起来就没有唐时那般严谨。

相貌没说的,在唐朝也许还讲究个五官端正,不能长得歪瓜劣枣。但到了此时,却已经不再追求长相,而是指的身体健康,无残疾。如果是进士,甚至这一条也可以含糊过去,瞎只眼睛,脖子有个瘤子,都能当官。

谈吐之类更不用说,完全是主观判断,如今不会有铨试官拿这一条来卡人脖子。太得罪人不提,说不定还会被投诉。

书法则是做官的基本条件,字都写不好做什么文官?改去做武官得了。武职好过关,只要亲笔写的家状上错字不要超过三个,计算钱谷五题对三题,武官中的书算科便算合格,可以成为一名合格的后勤武官。如果还能骑骑马,射射箭,水平不差的话,两项合一还能评个优等。

而判,就是指断案写判词,依律对州县呈上来待处断有疑议的案牍公文作出合理判词,考验官员是否能称职的处理公务,也即是是否能‘通晓事情,谙练法律,明辨是非,发摘隐伏’。到了宋代这里,同样要考。不过不仅仅局限于判案,另外还要加写诗赋一首或是试墨义十道——这两项可以自由选择。

刘易和程禹受了上命,要给韩冈添点堵。让官家知道,王安石请他下特旨抬举的秦州布衣,究竟有多无能!使得天子在群臣面前丢了多大的脸?

但两人都明白,跟韩冈过不去并不是代表可以在结论上大肆作假。比如韩冈是一个五官端正身体康健的小白脸,就不能说他颜陋貌寝,兼之缺胳膊少腿,并不适任为官。明明口齿伶俐,堪比苏张,便不能说他本是昌徒,又为非类,虽无雄才,却有艾气。明明写了一笔好字,就不能说他目不识丁。

这样太容易揭穿,韩冈的名字毕竟通了天,若是有什么情弊,韩冈自诉上去,两方对质,倒霉的只会是作伪的一方。但把他的缺点扩大,长处不提,改动一下评语判词,也照样能让韩冈吃足苦头,这样也才能显出孔门弟子一字褒贬的手段。

只是初与韩冈见面,刘易和程禹就知道事情不好办了。

韩冈相貌外表没话说,任谁也挑不出毛病,只往面前一站,俊杰才士的气质展露无遗。

程禹和刘易又问了韩冈几个问题,无论是经术上的,还是史书上的,他都是胸有成竹的一条条、一款款,极有条理的回答出来,谈吐温文尔雅,平和淡定,看不出半点紧张,配合上他本身的气质,更不可能睁着眼睛瞎说他粗鄙不文。

至于书法,看着家状上的字就知道是刻苦练过,铁划银钩,端正的就像刻出来的一般。程禹肚子里计较,这韩冈,莫不是崇文院那边抄书的出身?一笔的三馆楷书,未免太标准了一点。

这样的一个年轻人,言谈举止各个方面都有着大家风范,完全不似家状上所写的三代农家出身。刘易看着他,都想帮自家女儿招来当夫婿了。

