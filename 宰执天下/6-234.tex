\section{第31章 风火披拂覆坟典(三)}

“二郎呢?!”

一声暴喝,上车之后便开始打盹的王珏顿时被惊醒。

睁开惺忪的双眼,抬起头来,就看见一个老头子正在车厢入口出训斥一名小厮。

“不是让你跟着二郎的吗?怎么一转眼就把人丢了?!”

老头火冒三丈,把那小厮呵斥得只抹眼泪。

王珏坐起身,左右望望,车厢中本来裹着毯子睡在床铺上的官员,现在一个个都醒了,坐起身望着吵闹声传过来的方向。

在车中的十几人,基本上都是**品的小官,还有几个吏员,尽管能坐进官车,却享受不到单独的包厢。如果是携带家眷还有一丝希望能弄个小间,可惜在列的都是单身上路。

但在这里,几乎都是有品级、有俸禄、衣着青绿的官人,岂有一个老苍头在他们面前任意呵斥小子的道理。

只是所有人都跟王珏一样,在一旁冷眼旁观。

这个老苍头敢这么做,要么是没有眼色,要么就是心中无惧。

能在官宦门第做仆役,不长眼的都呆不长。敢当着十几名官员的面大呼小叫,怕是也没将他们放在眼里。

王珏看那老苍头和小厮身上的穿着,至少是议政重臣那一级。

“……议政……”

零碎的话声传入王珏耳中,车厢中看出这一点的不只是王珏一位。

“二郎!”老苍头一声大叫。

“二郎来了。”小厮也惊喜的叫起来,如释重负。

王珏探头看了过去。

出现在车门处的是个眉清目秀的少年公子,只有十四五的样子。

他走进车厢,登时吸引了所有人的视线。

人长得白净高挑,穿着倒是很朴素,衣服上没有刺绣之类的,身上也没有什么装饰,只在腰间系了一块玉。

但衣料是棉布,而且应该是贵价的陇西细布,一匹当在八贯以上。王珏曾经咬着牙为浑家买了一次,用掉了他一个月的俸钱。

在这公子进来之后,老苍头和小厮也进来了,且在他们身后,还有两个挎刀的护卫,一高一矮,却都是一脸精悍。除了那老苍头之外,其他三人身后都背了一个造型奇特的大号双肩背囊,看色泽是牛皮所制,而那小厮手中,还拎了一个方方正正的藤条箱。

老苍头跟在那公子后面絮絮叨叨,“出来的时候,老爷和夫人可是千叮咛万嘱咐,不要乱跑。”

……………………

韩钲一脸的无奈。

身后的是府中的老都管,也是王旖乳母的丈夫,在府中的身份不同于普通的仆人。他们这些哥儿、姐儿见到了,也得礼数周到。

韩钲被他看着长大,也来往过江宁,道上路熟,大事小事都能照应。故而出来时,韩钲就被叮嘱,必须听话,不得乱跑。

老家伙有了金牌在手,韩钲也只能听着。

“二郎,方才到底去了哪里了?”老都管絮絮叨叨了好些句,终于问了韩钲刚才的去向。

韩钲找到了自己的床铺,是上铺。隔着通道的正对面,是一个圆脸的中年官员。下面的两张床铺,幸运的都没有人。

韩钲看了那官员一眼,回头道:“我方才去后面的车厢看了一看。那里臭气熏天的,你们也别挤到后面去了,就在这里休息吧。”

出来的仓促,专列没有,专属的车厢没有,连包厢都没有,只能跟其他小官挤这种上下两层铺位的车厢。

韩钲觉得自己父亲完全是故意的,否则只要一句话,弄一节车厢又有什么麻烦?现在却是按照自己的品级,去让人拿了一张车票,和四张仆人的车票。

除了自己能睡在这里,跟着自己的四个人,只能去各家仆人混居的车厢里去。那个车厢,韩钲也看了。床板钉在板壁上,上下三层,只能勉强坐起身,就这样,还有很多人只能坐在地上,甚至躺在床底下。

而且最大的问题,是很多官员将贩运的货物让仆人随身携带,占去了大半车厢。

苏轼当年就被人首告借用官船贩卖私盐,不管苏轼有没有做过这件事,官员借用官船、官车贩运货物的行为一直都是屡禁不止。列车对上车的货物都要征收印花税,普通旅客上车都要搜包,以防有人逃税。但官车不会搜检,所以官员们的走私行径依然肆无忌惮。

韩钲只瞧了一眼,就立刻决定让跟着自己的仆人都到官车车厢来。也不知里面带了什么货,仆婢车厢中一股子汗臭和香气混合的异味,差点就将他给熏昏掉。

要是自己身边的人也被熏染上这种怪味,韩钲简直难以想象自己到泗州后怎么度日,南下江宁可离不开他们。

韩钲站在床铺前,眉头又皱了起来,其实他的这个床铺,也不咋样。要比仆人那边好一点,但好的也有限。跟家里、跟别业,都差了不知多远。从小到大,他还没有睡过这样的床。

一节车厢中,一条两尺宽的通道连接前后,通道左右都是床铺。床铺上下两层,左右相对,躺在床上,呼吸相闻。

只是站在床前,一想到自己睡下来之后,头顶隔着一层板壁就是别人的脚,浓浓的嫌恶感便从心里咕嘟嘟的泛了起来。

更别说这张床榻不知多少人睡过,又沾了多少脏东西,想想都觉得恶心。万一染了病怎么办?

韩钲在家中锦衣玉食,父母持家虽不喜奢侈,家中器物、陈设无一金玉之物,但宰相家的生活品质,亦是当世最上品,宰相家的嫡生公子怎么可能习惯得了旅途中的寒酸?

但韩钲没有将心里的想法宣之于口。

他出来时,被母亲吩咐‘注意饮食,不得惹是生非,尽快抵达外公家’这么几条,还不如跟着他的管家、仆人受到耳提面命多。而父亲则说了,出门在外不比家中,凡事要多忍耐,不要挑剔。

从小听多了父亲筚路蓝缕的故事,又知道自己跌兄长在横渠书院怎么生活,韩钲不想回去被说是娇生惯养,不成大器。

不过他虽不说,下面还是有贴心人。

“二郎你先等一下,待小的先来收拾。”

……………………

小厮说着,手脚上更是麻利。

原本铺在床上的被褥给一把掀开,丢在了地上。只见那小厮从身后的背囊中拿出一个铜瓶,拧开盖子,手一翻,带着淡香的药粉便从瓶中洒到了床板上。

味道很熟悉,王珏想了一下,好象是和剂局成药坊卖的驱虫药粉。

方才那小厮带着哭腔回话时他没听清楚,现在听来,有点淡淡的关西口音,确切的说,是因为京腔有些别扭,所以才让几代开封人的王珏听出了其中一点关西腔调来。不过那老苍头却是标准的江南腔调,似乎是江西那边的。

那小厮细细的撒了一层药粉,才从背囊中拿出一条细麻布的床单,整整齐齐的铺好,又拿出一条毛毡,准备铺上去。

“太热了。”那位公子哥儿皱着眉头。

立刻就听见那老苍头在后面道:“夫人吩咐过,出门在外,宁可热着,不能冻着。”

公子不说话了,小厮也老老实实的将毛毡给铺上,然后又铺上了一层棉布被单。

麻布被单、毛毡、棉布被单。最后上面又是一层套了白布被套的薄被,这是身上盖的。但这还不是全部,让人睡下去的,是一条用绸缎缝起的睡袋。

好一通布置,不过是睡上一觉罢了。就让王珏都感觉身上发痒起来,好像自己床铺上的这层被褥上面爬满了跳蚤和臭虫。

这位贵公子站在车厢中,直等到下人将床铺整理了一遍,磨蹭了半天才肯坐下来。这番动作,落在王珏眼中,更加确认之前的判断。肯定就是一个养尊处优的公子哥儿,家中也必然是上等门第。

几名仆人身后的背囊,与神机营的牛皮双肩背囊相同款式。

自从禁军武备由皮甲全数改为铁甲,大量的牛皮就闲置了起来,这两年,闲置下来的牛皮很大一部分就被制作成背囊、内甲,还是有绳索。最为世人所知,正是那双肩牛皮背囊。比起包袱皮能装更多东西,也更适合走远路。但禁军之中,也只有需要出战的边军和神机营有装备。市面上仿造的不少,可真品牛皮双肩背囊,一直都是有价无市。

但相对于议政重臣的身份,区区军用双肩牛皮背包,可就算不了什么了。还有那睡袋,其实也是军用之物,不过军中多是皮毛所制。

待韩钲坐下来,王珏立刻凑了上去,下床后先行了礼,问道:“不知小哥是哪家的衙内?”

十四五岁的样子,就能坐上官车,这自然就是衙内。因为有臭味,就把仆人们都弄进官员的车厢,这也是衙内的脾气。

在朱门子弟眼中,没出身、没靠山的小官也就是个锦衣吏。过来后连个拱手见礼都没有,也不足为奇。

只不过一个门宦家的衙内,怎么会弄不到一个包厢,跑到这个下等官吏才会乘坐的车厢来?这就让王珏想不通了。
