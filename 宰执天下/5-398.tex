\section{第35章 势颓何来回天力(下)}

“不行了。”

张孝杰在下面走了一圈,回来后就冲萧十三摇头。

“那些人完全排不上用场了。”

无论是伤亡惨重的部族军,还是宫分、皮室这样得以保全的精锐,只要长了眼睛,就能看得出来他们对接下来的战局都已经是不抱希望了。

纵然代州城中的兵力,汇聚了从大小王庄撤回的兵马,以及之前放在五台山北麓防守山口的一部分军队,还包括原本的留守队伍,兵力甚至接近了三万。可那股子丧气颓唐的样子,像足了一群落荒而逃的败犬。就算不通兵法,也能看得出来他们已经完全成了一堆派不上用场的废人。

尤其是部族军,人人都喊着的要回去。早一点回家去享受劫掠来的财货,回到家中去.舔舐伤口。

之前没有回代州,直接返回雁门关的那几个部族,萧十三都调不出来。甚至有两家直接一路跑回朔州去了。说是要协助抵御从神武县过来的宋军,但到底是真是假,根本都不用去想。军令如山,哪有自行其是的道理?

在有对比的情况下,什么样的劝说或是命令都是没用的。凭什么不听军令的人能回后方享受,而听话的却要吃亏送死?太过强烈的反差,已经将士气打压到了最低点。

惨败没什么,破釜沉舟、背水一战都是有先例的,但现在有退路,且就在身后。这样的情况下,怎么逼着下面的人拼死作战?怕是一出城就要往后面跑了。

而且最严重的一点,就是在萧十三这位主帅身上,垂头弓腰的样子,让张孝杰完全看不到斗志。

在张孝杰看来,萧十三之所以会留在代州,完全是因为畏惧耶律乙辛,而不是因为他的战略布局需要他留在这里。

“城外宋人正在准备攻城,就在城下营地中打造的攻城器械,云梯和霹雳炮的架子都能在城头上直接看到。数量还不多,但韩冈能在这么短的时间内修出一条轨道来,几百架霹雳炮也不会花费他太多的时间。”

辽宋双方最大的差距就在能工巧匠上。既然宋军能在极短的时间内造出轨道,一日一夜运兵数万,那么足以供下代州城的攻城器械,又需要多少时间?一直以来,张孝杰最为畏惧的不是宋人的兵马,而是这种完全超乎想象的生产能力,一年之内甲兵数十万,那是大辽无论如何都追赶不上的恐怖实力。

“一旦霹雳炮在城外架起来,飞石如雨,代州还能守吗?”

萧十三默然不语,整个人纹丝不动,仿佛张孝杰的话并没有传入他的耳中。

可沉下去的脸,拧起的眉,使得张孝杰看得出来,萧十三是听到了。只是他虽无斗志,却还怀着一线侥幸的心理。

“现在代州、雁门的兵马都已经是惊弓之鸟,繁峙县那边难道还能例外不成?”张孝杰直接揭开了萧十三心中的那点侥幸。

在这边坚守住代州,然后趁宋军攻城时无暇分心,东侧飞狐陉入口处繁峙县的万余人马正好可以出宋人之背,彻底将战局扭转回来。

只是那完全是做梦!

“繁峙县粮草远比不上代州丰足。他们也等不了多久。而且即使要出动,一个是退回南京道,一个是出来与宋人决一死战,只要看看现在城中的样子,就知道那边会怎么选择了。”张孝杰关注着萧十三脸上表情的细微变化,见其依旧毫无反应,词锋便更尖锐了一分,“代州这里有枢密你在都安抚不了,繁峙县那边靠涅鲁衮能压得下来吗?!”

萧十三终于有了动作,抬起头,愤怒的眼神投射了过来。

张孝杰夷然不惧,肃容质问:“敢问枢密,代州虽重,可比得上朔州?”

萧十三瞪着张孝杰,而张孝杰毫不畏惧的与其对峙着,许久之后,萧十三低下了头,“不如。”

“正是。朔州乃是西京门户之地,非是忻代可比。失了朔州,不知萧枢密可有安寝之地?”

萧十三又低下头去,久久,化为一叹。

相比起代州,朔州的地位自然更加重要。其与西京大同府共处群山之中,相间一片坦途。一旦有失,大同府前便再无屏障。

朔州州城鄯阳县现在被折克行统领宋军围攻,鄯阳一破,接下来折克行必然就要攻打离之不远的马邑县。而从雁门关北上,正对着的便就是马邑县!

给宋人夺了朔州城,宋军就能直抵雁门关北。到时候,想走就来不及了。还都留在代州的近三万兵马,可就是瓮中之鳖,釜底游鱼。

而且整个西京道最为繁华富庶的地界也就在大同、朔州这一片。萧十三麾下最为得力的皮室军的驻地,就在大同和朔州之间的应州。

围绕着水脉丰沛的桑干河,应州有西京道最好的田地和草场,养育着数以万计的生民和牲畜。

萧十三自从镇守西京道后,将此处视为自家的根本。尤其是应州,他下得功夫最深。举族上下,连根基都扎进了桑干河畔的土地中。

在他的心里,是绝不愿意看到朔州、应州和大同府变成代州、忻州那般惨状。

手颤抖了起来,但犹豫的眼神,却终于稳定了下来。

在败退到代州的第四天,萧十三领军撤出了代州城。

宋军出营列阵,远远的监视着。而宋人的骑兵则在外围,与保护大军撤离的五千多皮室、宫分对峙着。在宋人的骑兵之中,甚至还有许多是阻卜人的装束,耀武扬威般的在他们的面前纵马狂奔。

阻卜人的嘲弄让强烈的屈辱感蔓延胸臆,骄悍的契丹战士几曾受到这些贱民的羞辱?可抬起头,望着不远处倚靠大营列阵的宋人,很多人又低下了头来,随着大队走进群山之中的关隘。

身后的城池中欢呼声震耳欲聋,随风传入了耳中。

萧十三紧紧咬住牙关。

我们绝对会回来的!

绝对!

……………………

“辽贼看起来还是依依不舍呢。”

辽军走得拖泥带水,比韩冈预计的多留了两天。

原本韩冈以为主力到了代州城下,城中的辽军就该打好包裹往回走了,没想到萧十三竟然还敢拖延。

不过韩冈也算松了一口气。他的攻城准备都做好了,要是萧十三再不走,到时候,究竟是挥军攻城,还是再等等,他也拿不定主意。

就算是现在,来自京畿的主力已经经过了战火的磨练,可韩冈还是不愿意指挥京营禁军去主动进攻,尤其还是攻城。辽军虽不善守,可京营禁军也不擅攻,能少一些伤亡便少一些。

辽军让出了代州城,韩冈也并没有趁隙攻击的想法。辽军虽是败兵,但也是哀兵,心中都堵着一口气,万一压迫过甚让他们绝地反击,满营骄兵的这边可能会输得很狼狈。

代州城终于回到手中,可韩冈并不往城内去,指挥各部堵着城门扎营。然后分派人手入城清理。一片狼藉的城中,不是短时间内就能清理干净的。

占据了代州城后,韩冈和他的幕僚并没有太多的兴奋,抬头便能看见北面近在咫尺的关隘,不夺雁门,便是未尽全功。而且还要腾出手来去处理繁峙县的敌军。

只有堵上了雁门山和飞狐陉上的通道,代州才能算得上是真正获得了安全。

“去神武县的人还走了吗?萧十三可已经回去了。”韩冈问道。

章楶道:“早就走了。折侯老将,定不会有失。”

从雁门关北上是马邑县,由此向东北去是大同府,而朔州州城则是在马邑县的正西面,更靠近古长城。

就算萧十三已经领军北上救援朔州,折克行也有足够的时间撤离。撤到山上,现在的这一支辽军,应该不会有心思去追击。即便敢于追击,也赢不了士气正旺、精锐则犹有过之的麟府军。

“辽贼肝胆已寒,繁峙县料无大战。枢密,接下来该如何?只是收回雁门、西陉吗?”

韩冈摇摇头:“来而不往非礼也。只要朝廷没有严令休兵,夺回雁门后,当然是要继续北进。”

韩冈话里话外全然忘了之前的休兵诏书。不仅是他根本不提,就是下面的幕僚,也全当没有这回事。

除非朝廷下诏走马换将,否则只要韩冈还在置制使的位置上,他都敢把不合人意的诏令当成耳旁风。

收复了国土,赶走了强盗。接下来,理所当然是反击。

难道朔州不好吗?

难道大同不好吗?

若一时还做不到,至少把掳走的百姓多救一点回来吧。

韩冈可从来没想过仅仅是收复了代州就偃旗息鼓,至少要辽人一个刻骨铭心的教训。

而不是现在这样,入寇的辽贼虽然吃苦受累,也受了不小的打击,但损失不多,收获不少,几年后多半就是好了疮疤忘了疼,到时候说不定又开始找麻烦了。

“白玉已经到忻口寨了吗?”他问道。

这边夺回了雁门、西陉,就该这群饥渴已久的饿虎上阵了。
不过韩冈也算松了一口气。他的攻城准备都做好了,要是萧十三再不走,到时候,究竟是挥军攻城,还是再等等,他也拿不定主意。
