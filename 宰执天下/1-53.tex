\section{第25章 欲收士心捕寇仇(下)}

“大人!”王厚快步的走进王韶的公厅中,“陈举、刘显已然束手就擒。除了陈举的两个儿子,两贼的党羽、亲族也被一网打尽!王舜臣现在正押着他们往州狱中去了。”

“知道了!”王韶淡淡的应了一声。他坐在桌前,头也不抬。注意力依然放在手上的一份公文上。

王厚一脸兴奋,并没注意到父亲的不对劲,“没想到捉拿陈举这么容易。大人只提个头,多少人抢着去做,连李经略也没意见。”

“……因为陈举原本是只刺猬,现在却是头肥羊!”

王厚笑嘻嘻的点头说着,“大人说的是!几十万贯的身家,就算放在东京城中,也是一等一的富户了。只是陈举原先势强,又没几人知道他的家财多少,就算有人垂涎其产业,还要防着被他反咬一口,得不偿失。可现在就没这么多麻烦了,陈举要杀玉昆,却是把自己的脑袋放到了斩首台上。”

在大宋,财可通鬼神。如果陈举的几十万贯家资运用得宜,又没有耽误时机,那今年被远窜偏僻小郡的官吏名单中,说不定要加上王韶一个。可现在,陈举的丰厚身家,却成了人人都想咬上一口的肥肉。

“韩玉昆被陈举害得不得不去服衙前役,连父母也得远遁凤翔去避风头。若他知道陈举垮台,不知会多感激大人!”

“谁知道呢!”王韶叹了一句,将手中的公文丢在了桌上。

王厚终于发现王韶神色不对了。他探过头去,只看了一眼公文上的文字,当即便惊叫了起来:“张守约要荐举韩玉昆?!”

“以三班借职管勾路中各处伤病事宜。”王韶神色淡然的补充道。闭起眼,靠上交椅的靠背,秦凤经略司机宜深深感叹着:“想不到韩三秀才不但文韬武略皆有所长,连治病救人的本事也都有所涉猎……范文正【范仲淹】倒是说过‘不为良相,便为良医’。张子厚是范文正的私淑弟子,多少也懂点岐黄之术,记得他还给蔡经略开过方子。不成想他教出来的韩冈竟也是学了个十足十,才几天工夫,就从张希参【张守约字】那里挣了个三班借职下来……”

抵达甘谷城连十天都没有,韩冈就能让张守约荐其为从九品的三班借职。这完全出乎王韶的意料。

三班借职,是武臣品官中最低一等的官阶,而管勾路中各处伤病营事宜则是韩冈要负责的职事。前一个是本官,代表着韩冈的官身阶级,同时决定了俸禄【工资】级别,故而亦称为寄禄官。后一个是差遣,决定了韩冈要做的工作。

这种官职和差遣分离的做法,也为后世所继承。比如有一人担任着市卫生局长,正处级干部,那么按宋代的说法,卫生局长是差遣,正处就是本官。当然,宋代的官制更为复杂。

宋代的差遣与品级无关,知县、知州都是差遣,却不是固定品级。担任同一等级差遣的官员,他们的品级高的能有三四品,低的可能只有七八品。比如王韶,秦凤路经略司机宜文字只是他的差遣,是他的职事,没有品级,只有他的本官——太子中允——才确定了品级:正八品的朝官,这是能参加朝会的最低的品级【注1】。

尽管张守约为韩冈荐举的官身,仅是从九品的三班借职,但终究已是有品官身。整个大宋朝,有品级的文官武官加起来也不会超过四万人。如王舜臣,才一个正名军将,离三班借职,尚有五级。王君万,指挥四百精锐骑兵的指挥使,也不过一个殿侍,离三班借职还有三级。

王舜臣在裴峡谷亲手斩获十一个贼人,如果背后没人的话,勉强能升个两级;而王君万于南谷一战中领军冲阵,计算功劳后,也最多跟得了官后的韩冈平起平坐。说实话,韩冈由布衣得荐举而任官,算是一步登天。

虽然对韩冈可说是崇拜,但王厚却不希望韩冈因张守约推荐而得官,这份人情当留给自家做,以用来结好韩冈。他怏怏不乐道:“张守约只是一个路分都监,他的荐举,不一定能成。”

张守约作为路分都监,当然有荐举权,但路中经略司也有反对的权力。不仅如此,张守约的荐举还要上报到三班院,由专门负责低品武臣审查的三班院来评判韩冈是否够资格入朝为官。

“向宝多半会反对!”王厚很确定的说道。

“不要小瞧向宝!”王韶冷笑:“只是他现在的确是进退维谷。若是赞成,还能落个宽宏大量的名声,如果他反对……盯着他都钤辖位置的,不知有多少!张希参怕是也有份!”

“难道张守约是故意做给向宝看的?”

“多半就是。”

王厚还是聪明的,眨了眨眼睛,顿时明白他老子的意思。向宝是路钤辖,而张守约是路分都监,两人分别是秦凤路武将中的第二和第三号人物。向宝如果去职,留下的位子,要么是朝中另派,要么便是由张守约直升。张守约刚刚在甘谷城立下了功劳,中枢的相公们不会看不到这一点。张守约现在怕是满心思都是将向宝从秦凤赶走,好取而代之。

“张守约真会抓时机!”

“这机会是韩冈送给他的。”

“大人!”不知是多少次向王韶推荐韩冈不果,王厚不弃不馁,又开始谈论韩冈,“韩玉昆才智手段皆远过常人,如果不及早将之招揽,日后必然追悔莫及!”

“此事为父当然知道。”王韶不知是看到甘谷城的公文抄件后第几次叹气。

从韩冈能让自己一向心高气傲的次子如此敬佩,其才不问可知。不过,王韶对韩冈真正的了解,还是回到秦州城后。当日韩冈北去甘谷,而王韶先发了马递加急传信秦州,第二天又跟甘谷城的捷报信使一起返回。

裴峡谷中的一战,究竟是突发事件,还是不妙的征兆,这一点谁也不能确认,李师中和王韶都不会冒半点风险。而等王韶加急赶回秦州城,与李师中一起安排下人手调查裴峡谷后,再去收集关于韩冈的信息,如此一来,军器库一案便浮出水面。

以王韶的眼力和老道,当然不会被表面的文章所蒙蔽。穿过书写在文牍上的迷雾,韩冈自从离家入城后的一番作为,王韶已是了若指掌。身处绝境之中,竟然能在一夜之间,连杀三人,以至于翻盘获胜,逼死仇家。除此之外,两个原本是陈举一方的库兵,也不知韩冈是如何向他们称述利害,加以说服,让他们死心塌地的抛弃陈举,在案发之后,毫不动摇的站在韩冈这一边。

“杀伐果断,临阵勇决,又有苏张之辩。这韩三,论性子论勇武论才智,当不输旧年治蜀的张乖崖!”这是当日,王韶了解到了军器库一案的内情后,对王厚所说的一番话。

张乖崖,是太宗、真宗朝的名臣,乖崖是自号,本名是张咏。张乖崖以剑术闻名于世,据传言他少年游学时曾误入黑店。当店家要谋害他的时候,他拔剑斩尽店主一家老小,又放火烧屋,弄出了个无头的灭门公案来。

而他为崇阳令,崇阳县看管钱库的库吏偷了库中一枚钱币,张乖崖意欲杖责,而为库吏所诟骂。张乖崖不说二话,直接批了判词‘一日一钱,千日一千,绳锯木断,水滴石穿’,便亲手一剑将其斩杀,那是绝对是豪侠的性子,即便放在侠客遍地的两汉,也不输人多少。而韩冈杀人不眨眼的脾气,与张乖崖比起来,也相差仿佛。

“如果此子能考个进士出身,说不定日后又是一位名臣。”这是王韶现在说的,只看韩冈病愈后,短短两个月间的一番作为,他的确有这份能耐。

韩冈如此人才,王韶当然想收归门下。但儿子王厚不争气,被韩冈诳得五体投地。如果这种情况下把韩冈招来,那就不是门客就能安抚得下,少说也要个官身才够。驴子还没开始拉磨,就给他吃饱草料,如此蠢事,王韶不愿去做。

只在伏羌北门匆匆一会,韩冈过于锋锐的眉眼,已经给王韶留下了深刻的印象。相由心生,韩冈装出再多的谦恭平和,也掩饰不住心中的狂傲。所以王韶打算先磨一磨韩秀才的脾气和傲气,让他不敢奢求太高,再清理掉害过他的仇家,让他别无后顾之忧。这一打一拉,想来韩冈也该俯首帖耳。如果日后他办事得力,便荐举他为官,如果是言过其实的废物的话,也可以赶走了事。

王韶的盘算很精巧,剧本写得也很好,但他忘了韩冈虽算不上大牌,却也没有照着剧本演出的义务。王韶更没料到,韩冈还有着自己编写剧本的能力。

谁能想得到呢?韩冈到甘谷城不过数日,就能作出张守约可以名正言顺荐举他的功绩?!

“置锥于囊,如何不脱颖而出?”王韶叹着自家的天真,对王厚道,“二哥儿,明日你随我去甘谷!”

注1:王韶正八品的品级看似很低,但北宋官制中,高品官员其实数量很少,低品官员也能任高官,许多时候,正六品就能担任宰相。再举个例子,比如县令俗称的七品芝麻官,但在北宋,知县一职基本上都是从八品的京官,到了正七品,知州都能担任了。关于北宋官制,俺会在后文中慢慢解说。

ps:韩冈得张守约推荐为官,王韶这下坐不住了。你争我夺,石头都能卖出宝石的价来。

今天第二更,红票收藏。

