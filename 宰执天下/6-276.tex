\section{第38章 天孙渐隐近黄昏(下)}

与韩冈一番深谈,从相府出来,冯从义带着从人一路向南。

车轮辘辘,经过州桥离开内城,然后继续向南行去。

当街边穿着宽袍大袖的年轻人多起来的时候,就知道国子监就在前面了。

一道斑驳的白色围墙,上覆青瓦,这就是大宋的最高学府。里面有超过两千名学子求学其中。

其中大部分,这辈子都无望金榜题名的一天。不过依然一个个趾高气昂,自觉可以慢公卿、傲王侯,就在街边小店中指点江山。

冯从义没有在里面上过一天学,但他手下的人与国子监生多有往来。这些年来,冯从义看过不知多少密报,秘密评价过不知多少士子。国子监中,真正可以入他之眼的杰出人才,一只手就数完了。

幸好不用看见这座国子监太久了。

接近南薰门,人流越发的汹涌,路上的车辆越来越多。一名巡卒,站在大街中央,看见有人违反行路规则,

出了外城不远,前面的道路两分,一向前,一向左,皆是宽达百步。

继续向前,是开封车站。京城前往天南地北的旅客,如今大多从此处出发。

左侧是前往青城行宫的道路,那里也是祭天的圜丘所在。

在其附近,是国子监新址。如今上千名大工小工正干得热火朝天,到了明年就能入住了。

到那时候,国子监的旧址上,将会修起一座大体育场,专门用来进行各项赛事。不论是蹴鞠,还是赛马,又或是射箭、相扑,甚至观兵,都可以在这片场地上进行。

当初国子监外迁,通过得很顺利。但在原址作何改建,则争论了很久,期间还几经反复。最后才变成了大体育场。

韩冈一直都鼓励全民强身健体,士人更要文武皆能。上古士人为诸侯臣,四方皆敌,入则需临民,出则需治军,文武不能偏废。诗词歌赋,仅是六艺之一,却在隋唐之后,因以诗赋取士,而变得凌迫所有学问。

韩冈最是想改变这股风气,让士林之中,在邀风赏月之余,也知道金戈铁马的好处。

但在国子监旧址上修建大体育场的建议,据冯从义所知,却不是韩冈的意见。而是蹴鞠、赛马两大总社,分头说服了诸多议政重臣,又通过报纸操纵舆论,最后在朝会上顺利通过了。

整个过程中,两位宰相都没有干涉太多。韩冈对此甚至抱着喜闻乐见的态度。

待两年后,便有一座能坐下三万多观众的巨型建筑矗立在开封城南。也难怪蹴鞠、赛马两家死对头会在此通力合作,仅仅是三万张门票,就足以让他们把杀父之仇都放下了。

不过,冯从义一想到当大体育场中坐满了三万多观众,一旦有人在其中闹事,引起了慌乱,那可不是三五条人命就能收场的。

不论是在京师,还是在陇西,冯从义都亲眼见识过,赛场旁的观众头脑热起来,会变成什么样的混乱场面。

他希望大体育场的四周,能多修几条离开的道路,再将观众席一段段的分割开来,各段不能相通。即使发生了混乱,也只局限在其中某一段,而不会蔓延全场。

但这些顾虑,除了在审定大体育场设计图的时候,他提了一下,在这之后,冯从义就没再对外说了。商人讲究和气生财,一张乌鸦嘴总不会受人喜欢。而且,在他之前,韩冈就提过相同的意见,表兄弟俩的意见恰巧相合,自然就没必要再多说。

不过他的那位表兄,什么事都能未雨绸缪,甚至看起来被动的事,实际上已经做了多少埋伏,真要细想起来,在叹服之外,依然还是叹服。

在刚刚越过青城行宫,离车站还有一段距离的地方,冯从义的马车转向了另一条路。随着前行,路上的车马渐渐稀少起来,路边的行人和商铺却不见减少,这是冯从义外室所居的信乐坊。

开封外城外的厢坊数量并不少,居民也多,商铺同样多。除了夜中不能出入京城,与居住在城门内没有区别。而且现在,随着外廓城和七座堡垒的建立,外城的城门就像内城的城门一样,都开始常年开启,不再禁人夜中出入。所以冯从义就干脆在南薰门外又买了间院子,顺便养了一个外室。

车速慢了下来,冯从义的外室就在前面,隔着车窗,他发现那间惹人恼的磨坊已经不见了。

磨坊与冯从义的外室小院相隔有数十步,中间隔了两户人家。冯从义要买下磨坊,并不是因为太过吵闹,而是打算逐渐蚕食这片位置绝佳的坊市,可不仅仅是为了养一两个外室这么简单。可惜磨坊的东家就是不肯卖。

冯从义知道他的表兄爱惜羽毛,所以也没仗韩冈的势强买强卖,只是让人传了一句话。

到家下车,冯从义的外室迎了上来。

曾经闻名京师的歌伎出身,相貌身段都是极为出色。看见冯从义出外多日终于回来,还没说话眼圈先红了。

对这种手段,冯从义也算见识多了,搂着安抚了两句,让下人搬下礼物,让外室伺候着梳洗更衣,闲下来后,漫不经意的问道:“磨坊搬走了?”

女人贴在冯从义怀里娇声道:“两个月前搬走了,临走时还问老爷什么时候回来,又说请老爷多关照。还是老爷厉害。”

“用他儿子的前途换的。”

虽然就是个开磨坊的,在京城外还有三十几亩田,算是个小地主。但养个儿子,在家读书,都希望儿子能够金榜题名。

让人查清了这一切,冯从义在离开京师前,跟磨坊主只说了一句,‘金陵书院,嵩阳书院,令郎可以任选其一。’

冯从义给开出的条件,包括了天下间最有名的三家书院中的两家。能进这两家书院,高中进士的几率立刻就高出了两成,那一位磨坊主就是再倔强,也不愿意为了一点意气,而罔顾自己儿子的前途。

但三大书院中剩下的那一座,冯从义却没拿了出来做价码。唯有横渠书院,是韩冈所看重,里面都是气学种子。冯从义虽然能插手书院中的人事安排,可他也不会为了区区一间房,就随意荐人进入书院读书。他再糊涂,也没有拆自己墙角的道理。

一点小事,换了外室曲意奉承。推杯换盏,被翻红浪,冯从义一夜睡到日上三竿。

醒来梳洗,吃了早饭,刚刚准备出门办事,却见贴身伴当带了一人进来,

冯从义小吃一惊,“钟哥儿,你怎么来了?”

韩钟笑嘻嘻的道,“家里闲着无事,便出来逛逛。”

以冯从义的阅历,如何看不出韩钟是说谎,“坐吧……有什么事?”

韩钟坐下来:“其实也没什么,侄儿只是有件事挂在心上……秀州倭人坊的几家丝厂厂主,昨日爹爹与四叔是怎么说的?”

乍听到韩钟的问题,冯从义有几分惊讶,之前在江南时,他这侄儿对这件事也没关心太多,想了想,说道:“可观其自败。”

“就这些?”韩钟有些不满意,“依爹爹的脾气,应该不会容忍他们得意太久的。”

冯从义皱起眉,深深的盯着韩钟,“……钟哥,是不是有人向你打听了什么?你可要知道轻重。”

“四叔放心,不是别人问。是侄儿昨天问爹爹,爹爹让我自己找答案。”韩钟涎着脸笑道,“可惜侄儿太笨,左思右想想不通,这就过来求四叔你帮帮忙了。”

冯从义安心了。韩钟若是撒谎,回头见了韩冈立刻就能戳穿。笑道,“你爹这件事做得好,你爹娘,把你们这些小子保护得太好了。想当年,你爹十五岁就出门求学,你是十五岁就出门游玩,说是行万里路胜读万卷书,可以增长见识,现在路走了不少,至于见识涨没涨,当然要考一下。”

“就是这么说啊,所以来求四叔解惑。”

冯从义摇头,“这个忙叔叔可帮不了。你爹既然没说,四叔又怎么能说?”

“四叔,侄儿已经不是小孩子了,也知道该为家里分忧了。但不经历,不领会,一直都懵懵懂懂,就不知该如何做才对。润州、秀州的事,侄儿想不明白,希望能多得一点指点。”韩钟眼神坚定,看着冯从义。

冯从义笑了起来,“既然钟哥儿你这么说了,那四叔也不好在推脱。不过你爹既然考你,直接告诉你就是舞弊了,这可不好。”

“那四叔说怎么办?”

“你就说说,如果你在你爹的位置上,你会怎么做?”

韩钟不假思索,“当然是严查各家工厂。”

“天下工厂工坊众多,查不胜查。家家皆有靠山,你若是强行干涉私家的产业,你爹在士林中的名声可就臭了。”

“孟子有云,虽千万人吾往矣。依爹爹的性子,也不会怕。”

“你爹是打算推广工厂,吸纳无田的人口。工厂新起,弊端必多。若是有人借机攻击工厂,坏了你爹的大计又如何?没有了工厂,再过十年,就有数以百万计的百姓将无田可耕,无食可吃。”

韩钟皱眉说道:“可以移民他乡。”

“移民,又能有多少活下来?三万五万,官府照顾得了,一路上九成九能活下来,十万八万,那就有些勉强了,要是五十万,八十万,不是饿死在路上,就是揭竿而起。”

“但倭人之苦,爹爹不可能不管。”

“倭人非是华夏贵胄,化外野人而已。何况他们在辽人手中,本就是朝不保夕,随时随地都可能性命不保,将他们招到中国来,尽管苦一点,可大部分还是能活下去。一日两顿,每日鸡鸣起床下地,你觉得苦,天下农夫都不觉得苦。每天都要跟开水打交道,中国人觉得苦,但异邦野人却觉得比过去的生活都要好多了。”冯从义很认真的指点着侄儿,“钟哥,你爹的书要细读,要抓住主要矛盾,另外,要学会用全局的眼光看问题。”

韩钟眨着眼,深思起来。

“好了。你只要知道一件事,你爹的决定,事关天下亿万百姓,绝不是轻率而为。”冯从义打断了韩钟的思考,开始往外赶人,“这件事剩下的,就等你哥哥姐姐和表妹成婚之后再想吧。现在,你可没这份闲工夫。”
