\section{第25章 阡陌纵横期膏粱(二)}

“俞龙珂一起进京?!”

韩冈、王厚都吃了一惊。俞龙珂在古渭大捷中的功劳已经酬奖过了,而瞎药、张香儿进京,是因为渭源之战的功绩。今次渭源之役俞龙珂什么都没做,从头到尾都是在打酱油,他也能随之进京,肯定有人在背后使力:

“是谁推荐他的?!”两人异口同声的问道,而两人心中,此时已经隐隐有了答案。

王韶将公文甩手丢给韩冈,“俞龙珂的名字是在秦州添上去的。”

果然一如所料。既然是在秦州做出来的事,下手的究竟是谁,当是一目了然。

“好个郭仲通!”王韶拍着桌案,恨声叫着郭逵。他不怒毫无功绩的俞龙珂能进京——即便俞龙珂一点功劳都没有,只要他能去京城表示顺服,王韶能用十八人抬着肩舆送他去东京——但郭逵插手缘边安抚司内事,却是他难以容忍的。

韩冈向厅外望了望,无论是亲兵还是被派来服侍的蕃女,都识趣的在外面站得很远。

“郭仲通未免太小心眼了。他今次卖好俞龙珂,不就是要往安抚司钉个钉子进来?!”王厚抱怨着。从他和他老子的角度,肯定是对郭逵乱插手的做法怨恨极大。

而韩冈至少还能保持冷静:“郭太尉想要揽下并吞河湟之功,就不能眼睁睁地看着缘边安抚司将古渭的蕃人全数收归帐下。但有李师中、窦舜卿和向宝三人的结果在前,想来郭太尉也不愿登时翻脸。就算他想把缘边安抚司拿在手中,安抚有三战的功绩在身,天子至少不会偏听偏信郭逵一人。”

郭逵只要战功,在渭源之战后就开始压制缘边安抚司的好战之心,同时也变相警告过了王韶等人。虽然这段时间以来,缘边安抚司的确老老实实的在种田。但换作韩冈是郭逵,也不会相信缘边安抚司会就此一直老实下去,所以郭逵要挖个墙角,给王韶等人足够多的压制,分化他手上的战力。

“凡事分阴阳,有坏的一面,也有好的一面。如果俞龙珂不走,瞎药肯定也不敢离开青渭。”瞎药在自己手底下老实听命,韩冈自然明了他对其兄俞龙珂的敌意和顾忌,“蕃人只有上京面圣后,才能证明他已经归顺朝廷。窝在老巢里的木征,就算接受了河州刺史一职,谁也不会以为他会做大宋的忠臣。今次俞龙珂、瞎药还有张香儿三人会同入京,正证明了安抚三年来的辛苦没有白费。从这方面想,郭逵其实也是做了一件好事。”

韩冈不主张跟郭逵撕破脸,这对他并没有好处,也不利于日后在河湟展开的战事。

而且他说得也在理,同时郭逵此事做得又是冠冕堂皇,并不是直接干预缘边安抚司内政,仅仅是钉个钉子下来。王韶心中纵然不满,却也不好把郭逵对俞龙珂的推荐给压下。

王韶想了半天,自问还是有能力把俞龙珂给镇住的。最后便往交椅背上一靠,放松了下来的伸了个懒腰:“古渭寨近,秦州城远,就看看俞龙珂有几个胆子。”他抬头,又笑了笑,问韩冈道:“不过以力服人,不如以德服人。玉昆,你有什么想法?”

王韶想要把俞龙珂抓在手里,恩威并施是必要的手段。王韶自问能压制俞龙珂,但要施恩可就是要跟郭逵正面相争了。

如果今早韩冈听说此事,他也许还会感到有些头疼,要费上一番心思去想办法,但现在胃里直泛着的黄芪味道,让他有了主意:“以下官之见,授人以鱼,不若授人以渔。”

王韶愣了愣神,很快就明白过来,摇摇头:“……俞龙珂家的渔网可不小,一天八匹马啊!”

青唐部的几口盐井就算王韶看了都要眼馋,一天出产至少值八匹马,近一百贯的收入。算起来一年就是三万五千贯,这是青唐部能在古渭附近立足的根本。

要知道,秦凤路的私盐有三成是从青唐部的盐井中流出来的——说起私盐泛滥,也只能怪如今的朝廷太过贪婪,盐价订立太高的缘故。平均一斤二三十文,而且口味还差。而私盐一斤只卖七八文,同时出自西夏青白盐池的私盐质量在天下间数一数二,只是青白盐多是行销关中河东,至于已近陇右的秦凤路,则是靠着来自河湟的私盐。

不过韩冈今天说的并不是盐:“古渭物产丰富,盐、牲畜不必说,就是药材也不少。方才桌上能拿出黄芪烩肉,可见张香儿手上究竟有多少药物可用。”

古渭即是千年后的甘肃陇西,韩冈记得在那个时代,此地药材出产丰富,很有些名气。不过韩冈也是方才吃了以金钱为名的滋补特产后,才回忆起曾经的一个出身陇西、家里作着药材生意的朋友跟他吹嘘过的故事。若不是有此一事,他早就忘得干干净净了。

得韩冈提醒,王厚有着恍然大悟的感觉:“玉昆是要以药材引人?亏你想得出!”他猛点着头,“说得也是,首阳山中黄芪倒是挺多的。甘草、柴胡也不少。在古渭,药材是纳芝临占部的出产得最多。手上就一口盐井,张香儿几个小妾身上的金银珠宝、绫罗绸缎,光靠卖盐哪能买得起?”

“黄芪益气补中,补肺健脾,实卫敛汗,可补元阳,充腠理,治劳伤,长肌肉。”王韶背着黄芪的功效。范仲淹都说过,不为良相,便为良医。此时的士大夫,懂一些医术的有很多,王韶也不例外,“不过要是有止血的伤药就更好了,玉昆你前些天,从疗养院回来好像是这么说过吧?”

韩冈点了点头,他的确是说过。论止血的中药,韩冈前世只知道一个三七。当年他应酬的酒喝得多了,胃有些问题,云南白药和三七药粉吞了不少。但如今,他在本草和医经中还没有找到三七的名字,大概还在大理的深山里长着,现在营中的伤药多是白及、艾草、血余为主,论效果当不如三七。

“蜀地以药、锦闻名于世,大的药商动辄数十万贯的身家,天下成药近三成出自于蜀中,而张香儿也因为涉足药材,能给妻妾用上金银丝帛。青唐部在青渭占地最广,已经没落的纳芝临占部,靠着仅存的三条谷地还有附近的首阳山还能大发横财,如果青唐部把山里的药材都翻出来,俞龙珂还能再跟郭逵一条路走到黑吗?”

“首阳山里青麻也多,就是白白长在那里,如果收割出来,绳索、布匹也可以不假外求。”王厚平日里为了制作沙盘,在古渭跑来跑去,地理已是一清二楚,“桑树不宜在此处栽种,桑麻两物也只剩麻可种了。”

青麻就是大.麻,不过这里长的大.麻,只能编麻绳,做不到让人醉生梦死。韩冈现在一直想要找到麻沸散的配方。如果有了合格的麻醉药,外科手术的安全性便能提高一大截,而伤员的存活率也会随之上升。当然,普通用来织造的青麻也不错,这可是贵重的战略物资。

“还有棉花。”韩冈补充道:“等过两年,就有大批的棉花可种了。今次让商人去找种子,也有棉籽这一条。”

“吉贝布是不错。有了棉花,可就有了吉贝布。这种布匹,可比绝大多数药材金贵得多……啊”王厚像是突然想起,“说到药材,其实另外还有一味夜明砂。鸟鼠山里山洞中可多的是。”

如果王厚不提后一句,韩冈还想不起来夜明砂究竟是何物,但听说是鸟鼠山中山洞出产,答案也就显而易见了,“什么夜明砂,就是蝙蝠粪!此物说是能明目,可蝙蝠双目皆盲。瞎子的粪便能治眼睛?”

王厚奇道,“蝙蝠是瞎子?没一对锐眼,怎么可能在山洞里不撞墙?”

“靠耳朵。可以去捉几只蝙蝠,分作两队。一队蒙住眼,一队用蜡堵了耳朵,看看是哪一队会撞墙。”

见韩冈说得煞有其事,王韶都吃了一惊:“玉昆,这事你该不是做过吧?!”

“家师曾言,凡事须重实证,随意臆测却是要不得的。”韩冈笑说了两句,又转回原先的话题,“古渭山中药材遍地,种类繁多,这是明摆的事。不过药材从山上挖,都是要看运气,有一波没一波。最好是能自种,像蜀地,就是药农药田最多。虽然这不是几年内能见功的,但俞龙珂和瞎药都是有头脑的聪明人,不会看不出其中的好处。青唐部和纳芝临占部都是半农半牧,也会种田,土地又不缺,种植药材可是一本万利,比种粮的收入要多得多。”

韩冈充满自信的说着:“如果古渭是个要用钱粮填进去的无底坑,终究还是会有人要反对拓边一事。但如果古渭有了特产,引来足够的移民,每年财税收入超过十万贯,朝廷便不可能再轻言放弃。”

“俞龙珂不是蠢人,郭逵能给他的,我们能给他更多。”韩冈放声豪言,“那只老狐狸当知道该怎么做。”

官位也是虚的,而韩冈带给青唐部的利益却是能延续下去。

王韶被说服了,韩冈过去的功绩,也让他对韩冈的才华有着绝对的信任,“这要多劳玉昆你了。”

“请安抚放心,韩冈必不负所托。”

韩冈抱拳,低下头去,嘴角露出了一丝得意的微笑。蕃部的主导权,王韶想要,郭逵也想要,但在韩冈看来,不如拿在自己的手上更好一点……

