\section{第九章 拄剑握槊意未销(七)}

韩冈眼看着天子的脸色越来越难看,心中只觉得好笑。

王珪此前迎合圣意,借用天子的权威压得一众执政不是沉默不言,就是曲意迎合,说起来没有哪个是心甘情愿的。

即便是元绛,难道这只老狐狸当真愿意事事顺从王珪不成?执政不是宰相家的走马狗!

若是今日灵州已然克复,没人会跟王珪为敌,只能选择暂避锋芒。但现在既然已经败了,哪里还会愿意给王珪翻身的机会?哪个不想趁势捡个大便宜?

看看吕公著、吕惠卿和元绛,谁人不是眼中带着熬夜的血丝,眼袋浮凸,下眼睑带着青黑色。肯定是在拒绝了圣谕之后,招了亲信幕僚一直商议到上朝前。

不过王珪也有优势,他独自奉召入宫,也就意味着他跟赵顼有了两个时辰的商议时间。看模样、听说话,他们两位是不甘心就此认输,还想赌上一把来回本——标准的赌徒心理。

韩冈听着上面的宰执们争来争去,自己则是老老实实的待着,他是列席会议,不是出席,绝对不会主动发言。反正他看赵顼的样子,应该快忍不住了。

赵顼的视线的确不时的扫过韩冈的身上。他一边听着执政们各持己见的议论,一边关注着韩冈。只见韩冈始终低眉顺眼的垂首坐着,半点也看不到插话建言的打算。

因为韩冈此前一直都反对攻击兴灵,也说了一些让人不痛快的话,如今他的乌鸦嘴一一应证,赵顼只觉得面上无光,看见韩冈在面前就感觉不舒服。

其实赵顼并不想招韩冈上殿议事,可如今的局势,与当初郭逵和韩冈的预言别无二致。现在郭逵坐镇河北,只有一个韩冈在朝中,赵顼却不能不征询他的意见,亦不敢不征询——不过一时之气,总比不上国事重要。

赵顼瞅了韩冈半天,韩冈却垂着眼皮,身形自坐下后似乎就没动弹过,让赵顼想打个眼色都没办法。

王珪和吕公著越争越激烈,而吕惠卿和元绛多多少少又偏帮吕公著,赵顼见磨蹭不下去了,只能开口:“韩卿。如今西北战局,不知你有何看法。”

韩冈眨了下眼睛,腰背又直了一点,从方才的木雕状态终于变回活人。

殿中君臣的视线齐集韩冈身上。韩冈站了起来,持笏向赵顼拱手道:“以臣愚见,灵州之败,首先在于孤军深入,十万军汇聚城下,而友军则远在千里之外,加之粮道绵长,一败便不可收拾。兵法有云,未虑胜、先虑败。胜而不骄、败而不乱,方可谓之用兵如法。灵州之败,乃是不合兵法正道的缘故。”

赵顼沉下脸,反驳道:“用兵当以奇正相辅,岂不闻李愬雪夜入蔡州?”

“臣斗胆敢问陛下,遍观青史,用奇兵为胜者,除此之外又有几桩?用正兵为胜者,则又有多少?”韩冈毫不客气的将赵顼的话堵回去,“奇者,异也。异者,非常也。力不如人、势不如人,为求一胜,于非常之时,行非常之事,故而曰奇。且用奇兵者,败者良多,胜者极少,亦是世人之所以目之为奇的缘故。以六路官军三十万人马,稳扎稳打便可得胜,何须自蹈险地?非非常之时,却行非常之事,胜则不能加功,败则不可收拾,灵州之败一至于此,此乃本因。”

赵顼眼中怒意蕴藉,但却不再跟韩冈辩论,那太有失体统。

听着韩冈的发言,看着天子的神情,吕公著眼神中带起笑意。韩冈这分明是在发泄之前的怨气。终究太过年轻气盛了,天子要的是解决问题的方略,不是清算战败的责任谁属。

不过这样也好,有韩冈发难,只要敲敲边鼓就可以了,免得自家一把年纪还要冲锋陷阵。吕公著想着。韩冈的话传出去,正好让王珪消受了,而天子日后算账,也是落在韩冈身上,与自家无关。

吕惠卿却深悉韩冈为人,心中疑云大起,眯眼抿嘴,等着韩冈的后续。

韩冈歇了口气,又道:“灵州之败,其次在于将帅失察,西贼避而不战,一路引诱官军至灵州城下,当知其必有奸谋,又在黄河之滨,如何能糊涂到让西贼成功的决堤放水?经此一败,环庆、泾原损兵折将,数年之内难以再用。”

韩冈话声刚停,吕公著便跟上去道:“自横山至灵州,路程几近千里,西贼一路追击,逃得生天者不知会有几人。臣请陛下三思,实是不能再动刀兵了。”

赵顼虎着脸不说话,王珪看了看天子,就要砌词反驳,韩冈却是抢先一步,“诚如枢密所言。两路败军自灵州一路逃回,身后必有铁鹞子追击,路途迢迢,能生还者恐怕仅有半数。”

他停了一下,飞快的瞄了神色木然的赵顼一眼,“但相对于三十五万官军来说,这依然仅仅是小挫罢了。需要休养生息的只是环庆泾原二路,王师主力犹存,不知吕枢密何来不能再动刀兵之语?”

韩冈的表态出人意料,赵顼双眼亮了起来,而四名宰执,也是神色各异。

吕公著不意韩冈竟然反手一刀,沉下脸,声音亦是危险的低沉:“两路精锐尽丧,”

“打个比方。如果从一条狗身上取下一斤肉来,肯定是没命了,但如果是从大象身上取下一斤肉,却绝不会致命。灵州之败,纵是全军覆没,丧师也不过十数万人,此役官军三十余万,六路齐发,如今不过三分之一不到,丁口数千万的大宋还能承受得起!而西夏在灵州一战中收到的损失,他们却承受不起!”

“西贼避而不战,有何损失?”吕公著拿韩冈的话来驳斥。

“怎么可能没有损失?”韩冈笑道,“官军深入兴灵,西夏国力损耗只会在官军之上。放水、拆屋、砍树、焚田,灵州城外的一切全都毁了。银夏,河西、天都山,莫不如此。除了兴庆府和西夏北方的荒原,西夏国中其余人丁富集的膏腴之地不是毁于官军,就是毁于其自手。相对于官军仅止于兵将的伤亡,西夏的损失已经远远超过了此数。”

“西贼大军犹存!”吕公著厉声道。

“此辈不足虑。中国胜于西北二虏者,不在军力,而在国力。丁口、税赋、物产,皆是远远过之。两国相争,若是争夺边地,那是军力之争。如若是灭国之战,那比拼的则是国力。此《孙子》之中,食敌一钟,当吾二十钟的本意所在。”

赵顼、吕惠卿都为韩冈的话沉思起来,元绛盯着韩冈,不知在想些什么。王珪则是在看眼神越发严厉的吕公著,嘴角含笑,韩冈至少不是站在吕公著那一边。

韩冈朗声说道:“春秋吴越相争,越国军力远不及吴国.越王勾践卧薪尝胆,十年生聚、十年教训,女子十七不嫁,父母有罪焉。此乃厚植国力。献美人,诱夫差修宫室,消耗的是吴国国力,以煮熟的稻种诓骗吴国耕种,同样是在削弱吴国国力,最后一举灭吴,岂止是因为夫差帅吴兵北上会盟、国内空虚之故?”

“韩卿言之有理。”赵顼第一个点头。国力论乃是投其所好,明大宋必胜二虏之因,听得他心中欣喜难耐。

“自熙宁四年攻略横山始,西夏接连败绩丧师失地,国势日蹙——其损兵折将之处,远过于灵州。”韩冈顺口又戳了吕公著一下,他实在不喜欢这个喜谋私利,却又装得正直无私的老家伙,“之前又岁献马驼三万与辽,其国力不及十年前的一半。如今灵州城下的胜绩,不过是回光返照而已。开战旬月,可曾见过铁鹞子出阵与官军正面交锋——不敢御敌于国门之外,西贼虚怯可见一斑。自元昊叛立后,直至熙宁之前,官军可曾有过一次攻入西夏境内?”

“上兵伐谋,须知西贼奸狡。”吕公著火气上来了,与韩冈针锋相对,当初他可是为了废新法,动摇赵旭的意志,敢说韩琦有心清君侧:“从继迁至元昊无不是狡猾之辈,三川口、好水川哪一战不是西贼施狡计而得胜,灵州之败更是最新的例证。高遵裕、苗授皆为一时名将,西贼决堤却都没有发现。”

“敢问枢密,若官军再至灵州城下,西夏还有河堤可掘?官军岂会再给他们这个机会?!没有了狡计,区区西贼如何能抗拒天兵!”韩冈笑了一下,“狡计乃是力不能敌时的无奈之举,人言狐性多狡,但狐狸安可与虎豹相争?虎豹在山,又何须狡计。”

“韩卿国力之说,对朕深有启发。”赵顼不想听两人再吵,他只想听一听如何挽回西北战局的方法,“不知韩卿对眼下局势有何方略,尽请直言。”

“官军旧年曾一举灭亡交趾,收复汉唐故地。不过西夏不是交趾,疆域是其五倍,军力是其十倍。想一举攻取西夏,以臣观之,直如登天。但一步步的蚕食,十数年内西夏必亡。这也是为什么横山易取,灵州难得的缘故。将西贼逼入官军预定的战场,则官军必胜。如果是深入西贼预先划定的战场,则官军危矣。”

这是韩冈一直以来的见解,至今未变。

“如今除泾原、环庆两路之外,其余四路都未有大的伤损。如果稳扎稳打,假以时日,足以将西贼碾碎。纵然间或有小挫,只要胜势在我,西贼便无法扭转最终覆灭的结局。此乃战胜于庙堂之法。”

韩冈话声刚落,吕惠卿就差点要笑出声,但很快又感慨起来。

说来说去,韩冈其实就又绕回了他这几个月来一直主张的对夏战略,缓进、蚕食。哪里是帮赵顼和王珪说话,分明是在炫耀自己的先明之见。

赵顼和王珪也全然明白了。不管他们多么想得到灵州,到最后也只能转回来,按照韩冈的计划来行事。

究竟打算怎么做?

所有人的视线都投向了赵顼。

