\section{第16章 山入四荒更郁苍(上)}

蒲宗孟所说的那些话,韩冈听到不过不止一次。

他的打算,韩冈更是知道的一清二楚。

不过他的想法,还是太早了一点。

蒲宗孟急的不是皇帝,而是他自己。

清凉伞的诱惑力让蒲宗孟变得不顾一切。

但两府就那么大,可不是随随便便都能进来的。

两府的新近人选,尤其是参知政事的人选,韩冈其实属意沈括。

以沈括在工程技术上的才干,在他担任参知政事之后,韩冈就能把轨道修筑这方面的事务都丢给他去处理,免得自己劳心劳神。

随着铁路在九州大地上的不断延伸,韩冈越来越感觉到相应的技术储备实在很欠缺,车厢底盘、铁轨、车轮等零部件的制造,以及组织上的欠缺,还有包括钢铁冶炼、建筑规划等一系列的问题,使得现在正在修筑的铁路,尚不能达到韩冈已经降低了许多的预期。

不过让有轨马车奔驰在河东、山西之间,让轨道沟通京城和泗州,让洛阳和开封之间的旅行距离缩短到两天,倒也是足够了。

洛阳和开封之间的铁路经过两年的铺设,只要中间经过的几条河上的大桥修好,今年年内就能开通。只是其中最长的大桥总长度超过三十丈,单跨跨度十丈,技术难度有些高,能不能按期完工,韩冈并没有把握,完成的质量如何,韩冈更没有把握。

幸好开封到泗州的京泗铁路,速度就快了许多,施工难度也小。现在也已经到了最后阶段,很快就能够完工。因为贯通了黄河水,高出两岸地面的汴水,相当于一条分水岭,从京城到泗州,并没有其他水源汇入,这使得与汴水平行而筑的京泗铁路,并不需要考虑桥梁架设问题。

而从太原南下关中河中府的并蒲铁路,需要跨越的河流更多,路线也更长,河东的人力财力还不足以支持大规模的修筑,现在处于缓慢施工阶段。等到京洛铁路和京泗铁路开通之后,才会加快速度修筑下去。

至于那些从这几支干线铁路延伸出去的支线,在主线还没有表现出足够的盈利能力的现在,当地世家大族一时间还不会去争夺支线的修筑权力。但只要大动脉打通,相信没人还回犹豫。而且据已经通车的几个地方回报,当地许多大族,已开始邀请工匠进行路线勘探,确定修筑的范围。

之前的轨道修筑,不论是最早的方城轨道,还是现在的并代、并蒲、京泗、京洛等几条新修铁路,都是以修筑官道的名义来强行征收沿线的私人土地,并用陕西沿边的荒地来进行交换,发行授田证给失地的百姓。

不过这些授田证,基本上都是落到了当地大族手中。早在动工之前,他们就收购了将要被征用的大部分田地,然后拿着授田证,来与雍秦商会的成员再进行交易。

在韩冈看来,阻挡公共事业、阻碍社会发展,不论是谁来做这个拦路石,那就该砸个粉碎。不过在铁锤下来之前,先拿好处来引诱是必须的。而朝堂中的消息总是那些世家大族先得到,所以最后给出的好处也都落到了他们手中。不过那些土地的原主,如果他们不先卖出去,朝廷给予的补偿,大多会被官吏给干没,卖给当地大族,对他们来说反而是件好事了。

只是雍秦商会也没吃亏,同样是授田证,在不同人手中,得到的荒地自然是不一样的。有业已开垦多年的屯田堡下的千亩良田,也有荒芜不毛的荒山野岭。韩冈没去占这个便宜,可雍秦商会中的大部分人都从中得到了丰厚的回报。

朝廷也不吃亏,就算损失多少官产,只要轨道开通,经济和军事都会上一个台阶。这样的收获,只付出一些官田和荒地,绝对获利丰厚的一笔投资。

这是多方共赢的好事。

这也是韩冈没有因为大兴营造而得到太多骂名的原因——尽管他兴修工役,征发各州各县人力无数,但有发言权的各方都吃到不少好处了,不论新党旧党,都摩拳擦掌的等待着下一次的大餐。

所以轨道的铺设不会就这么停步。

韩冈望着挂在他座位背后的舆图,上面的红色线条,正代表着朝廷所拥有的几条铁路轨道。

一条条铁路,如同血脉一般在北方大地上延伸,当这些血脉交织在一起,就是这个国家彻底进入一个新时代的那一天。

不过在这之前,必须先把蒸汽机发明出来才行。

蒸汽机的原理和功用,韩冈已通过各种途径散布了出去,在他的提议下,朝廷给出的悬赏,也是极为丰厚。

世间为此进行研究的人,成千上万。

据韩冈所知,很多地方,已经有些眉目了。

最基本的蒸汽机,或许近几年就能看见。

二十年后,也许就能有装在火车上的蒸汽机了。

……韩冈想了想,也许三十年、四十年也说不定,但肯定会出现。

然后九州大地上,都能看见冒着浓烟的钢铁机器在轨道上奔驰。

“相公!”

韩冈闻声回头,看见了宗泽。

宗泽是状元出身,出外一任之后回京,本应该在崇文院任职。但他回京后并没有去三馆秘阁,却改任了兵礼房检正公事。

这是宗泽主动向韩冈要求的,他想要一个能够处理实务的职位,所以韩冈就给了他几个职位进行选择,而宗泽选了中书门下。

宗泽向韩冈行礼,“相公,下官奉命将杨总管带来了。”

宗泽的背后,是北海水师都总管杨从先。

杨从先是回京述职,今天是去枢密院汇报。韩冈有事要问他,去枢密院门口拦他过来,宗泽去比较方便。

杨从先在韩冈面前更是紧张,“相公……末将杨从先拜见相公。”

“好了,不用多礼。”

韩冈没有与杨从先多寒暄,后面要见的人太多,他只想了解一下最近水军新型船只的现状。

几年前,当水师成立之后,在韩冈的指示下,全国各大船场都在设计有别于民船的战船,以配合火炮的使用。

在过去,民船、军船不分,只要换一下船上的装备,普通的商船就成了战舰。但现在设计出来的战船,在速度和坚固性上大做文章。

“禀相公,去岁九月十三,登州船场已经将第一艘巡洋舰送抵末将处,从那一天开始,每月在港都不及五日,日日操练,不敢有所懈怠。”

新型战船,按照韩冈的习惯,分为巡洋舰和战列舰两个类型。巡洋舰速度快、不过火炮少,船型也比较修长,装载量也不算差,就是人手少,用来巡海、搜检走私商船绰绰有余。

“按照相公的要求,巡洋舰是能追上,能逃掉。现在这第一艘巡洋舰,得太后赐名伏波之后,便去了泉州一趟。泉州大小三十六港,船只千万,没有比这艘船更快的了。”

“风帆呢?”

“从大食人手中学来的三角帆,泉州早就有了那样的船,不过不适用。明州船场改了一艘用三角帆,又加挂前帆的船,在换上棉帆布之后,的确快了一些,也更灵活了,不过要用的人手多,而且绳索太多,水手不习惯。”

韩冈点头,这些他都知道,杨从先没敢胡说。

提供军用的棉布的重量,接近市面上普通棉布的一倍,而帆布的重量更是普通军用棉布衣料的一倍,想要缝起这样的布料,所用的针看起来都跟钉子差不多了。

一直以来,中国的船帆都是硬帆,中间有支撑物,升上去吃力,降下来一松手就够。也可以轻松调节帆的面积,应对不同等级的海风。换成没有支撑物的软帆,升帆简单,此外船帆转动起来也容易,能够让船只更为灵活。

“战列舰如何?”

“犬子看过了,明州船场已调用最好的工匠在打造,都是当年打造两艘神舟的工匠。末将也看过了图样和模型。只要有五条战列舰,相公要末将攻下那座港口,末将就能攻下!”

“这么有把握?”韩冈笑道。

“战列舰一艘船,一面就有三十门火炮,五艘,便是一百五十门。船就不说了,天底下哪座港口能挡得住这么多门炮?”

巡洋舰只有十几门炮,而战列舰则是上下两层火炮,总共六十门轻重榴弹炮,而船头还有一门长管炮,至于船尾,由于船型是前尖后宽,船尾抬高,所以分上下三层,总共八门火炮。若有追敌,能让其吃上大亏。

杨从先语气激昂起来,“战列舰一开炮,两三里地不能近人。又足够结实,外壁整整三层厚板,砲石难伤。从龙骨到船肋,都是坚实无比。什么船都能轻松撞开,就如车轮碾鸡蛋。直接开进港口,炮门一开,赶走守军,船中士兵上岸,哪座港口夺不下来?”

杨从先不是多有才干的水军将领,可惜的是,大宋百万军,水师将领,也只有他能用一用。

对杨从先,韩冈大加褒奖,“听你说,船入列后,训练就没有懈怠过,就该这么做。不说训练,只说船。这船是新船,得一边造,一边改,一边用,。你们用的时候,有什么不合意的地方都挑出来,到造下一条船时改好。就这么修三五艘船,才能定型。”

“相公说的是。”

韩冈笑了起来,“等水师有了十余艘战列舰,二十余艘巡洋舰,这东海、南洋,都将是你们的猎场。”

