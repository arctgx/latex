\section{第36章 可能与世作津梁(一)}

已经是仲春,但出城踏青的热潮方兴未艾。

唐州城外的名胜,到处都是出来踏青的游人访客。

几处私家园林,只要主人家并不住在里面,也都向游人敞开了大门。这是一年一度的好时节,一季下来的收入,往往能将一年的维持费用给赚回来。

韩冈骑在马上,眺望着远近,路边游人如织,有不少人模仿着东京城的风俗,无分男女老少,在头上簪上一朵鲜花,在街道上招摇而行。

观花吟诗的酸丁为数甚多,但更多的还是有些闲暇和闲钱的百姓。还算是太平年景,就是底层做些小买卖的市民,也都有闲心出来游逛一番。一个个拖家带口的,望着湖光水色,脸上都带着满足的笑容。

韩冈从关西来,参与的是军事;在开封时,则遇上了几十年不遇的大灾;接着又去广西攻打交趾,他这些年来,任官天南地北,却几乎没怎么见到如今出现在他眼前的这幅太平盛世的画卷。

看着前路行人渐多,韩冈随行的伴当就想将旗牌给打起来,驱赶前面的人群。韩冈则是将他斥退了下去,摇摇头,“大家都开心的时候,何必吆喝几嗓子,扰人兴致。”

王旖和周南透过车窗上的竹帘,看到韩冈训斥家人的这一幕,相视而笑:“官人心情终于好了。”

“都是那个吕与叔。”周南抱怨了一句。

“好了,这几天你跟云娘就没少骂他。”王旖笑道,“官人心情好了就行了。”

韩冈现在的心情的确不错。

虽然因为种种缘由,坏了心情,韩冈还是打算在离开洛阳前,去独乐园拜访一下司马光,谁料到司马光去了嵩阳书院,半个月之内都不会回来。这就没办法了,韩冈不可能因为司马光一人而在洛阳久留,随即整理好行装,携全家启程南下。

因为得知司马光去了嵩阳书院,在路上,韩冈也在计算着道学的支持者。

司马光去嵩阳书院,当然是为了讲学。同在一堂讲学,司马光和二程的关系自然也不会差。而富弼、文彦博、以及住在洛阳的一干老臣,二程凭着当世大儒的身份,也都能悠游的穿梭于他们的行列之中。

二程在洛阳授业,有人引荐、有人相助,由于旧党元老来往频繁,相对于关学,位置得天独厚,除了开封府,其他地方都比不上。

如果韩冈当初没有将张载举荐入东京,恐怕关学在失去了核心之后,只要程颐一入关中,转眼就会败落了。毕竟当初对张载一力支持的蓝田吕氏,现在似乎已经偏向二程那一方了——如果只看吕大临,甚至可以将似乎二字也去掉。

韩冈已经写信给苏昞和范育,以及身在陕州的游师雄,更重要的是,他也没将自己的师母和小师弟忘掉,没多说别的,只是将吕大临起草的行状的片段寄了过去。他的记性虽说达不到过目不忘的境界,但‘尽弃其学而学焉’几个字,却是记忆深刻。同时在犹豫了一阵后,又给吕大钧和吕大忠写了信,向他们对此事表明了自己的态度。

韩冈也不在乎被人批评是背后论人短长,以他的身份地位,加上吕大临犯的错,无人能用这个罪名批评他。而韩冈之所以会这么做,是为了向张载的几位重要弟子展示自己的立场,自己并不是程门弟子,受教于程颢是事实,但依然是气学一脉。他不想让自己之前对程颢程颐两位的敬重,当成是投入程门的标志。

不论回话如何,韩冈有信心将除蓝田吕氏以外的几位张门弟子,都拉到自己这边来。吕大临所做的行状,只要公布开来,都会让所有的气学一脉感到愤怒。加上韩冈这位地位最高的弟子态度十分明确,就不用担心有人顾忌他的立场。

但这只是见招拆招的应对,如果不能解决气学核心缺失的问题,再多的计算都是无用功。

韩冈对此已经有了觉悟,他本来也有成为气学学派核心的打算。经过这几天来对计划的不断推演,也算是有了足够的把握。

唯一担心的就是到底能不能来得及,程颐不久便会入关中讲学,目标自然是关学弟子。如今的这个时代,道统之争近乎于你死我活,但门户之见的程度并不深。在气学的墙角被彻底撬光之前,韩冈就必须表现出气学衣钵传人的实力——不是靠官位、而是靠学术。

‘时不我待啊。’

韩冈很明白时间的紧迫,而他的信心依然充足,在都转运使的任上,不论政事还是学术,他都打算将自己的地位彻底确立。

道边的建筑越发的多了起来,道上的行人也多了,离着唐州城就剩二十里。

韩冈望着前方,前天抵达方城垭口时,沈括派出来的人已经在那里候着了。穿过方城山,进入唐州地界后,这一个个驿馆铺递的过来,都能看到沈括的人。唐州城就在眼前,“沈存中也该出来了。”

……………………

沈括的确出来了,论地位、论关系、论恩德,他都不能不出来迎接韩冈。

带着满城的官吏,还有城中耆老,沈括出城十里相迎。连同唐州教坊司中的妓女都带出来,用着远比洛阳要盛大百倍的场面,迎接都转运使韩冈的到来。

沈括从京城贬谪而出,由高位一落而下。加上又是毁了名声,从心情上,当然是十分失落的。不过上天也没有就此抛弃他,一个绝佳的机会落在他的面前。

韩冈为了保证打通襄汉漕运,而请动了天子,将他的贬谪之地定在了唐州,而不是更远的南方。幸运的得到这个机会的沈括,就像是抓到了救命稻草一般。很是勤力的在襄汉漕渠上挥洒着自己的才智和汗水。而在筹备襄汉漕运工程的同时,他也出色的尽了一名知州的责任。

在盛大的欢迎场面上,韩冈与沈括见了面。看着沈括凹陷下去的双颊和凸出来的颧骨,韩冈不禁有些感慨:“存中清减了不少,只看信上,哪里知道有着这般辛苦。”

“还好,还好。”沈括连声说着,转而却又笑了起来,“若不是有这番辛苦,也不敢来见玉昆你。”

寒暄了两句,沈括便将他手下的属僚一个个都引见过来,韩冈一一见过礼,接着又与当地的父老说了些惯例的废话。

说起来,以韩冈的感觉,唐州的当地人中,真正欢迎他的只有那些个在旁努力做着壁花的官妓——多半是因为韩冈年少位高,外形看着又不错而已——其他人则是看着谦卑,但实际上都只是在应付故事。

韩冈估摸着,这或许是因为自己是重启襄汉漕运的倡议者,由此在唐州兴起大役的缘故。这世上有远见的不多,被触犯到一点利益就立刻跳起来的人却是不少。还没见面就被人讨厌了,韩冈也只能摇头感叹。

迎客的一遍流程走完,韩冈便上马往唐州城过去,沈括则紧随在后。

一路上看着道路两面的田地,韩冈和沈括脸上都有掩不住的喜色。

唐州这里有水稻、有小麦。小麦经过了一个冬天的蛰伏,到了三月的时候,已经长得郁郁匆匆,水稻长势亦是喜人,沈括指着满眼的绿意炫耀似的展示给韩冈:“今年的收成不会差,当是个好年景。”

韩冈笑着点头:“若能丰收,今冬兴工可就省心省力了。”

沈括答道:“京西这几年收成都不差,府库充盈,无论是入冬后的工役,还是眼下动用厢军铺设”

从熙宁五年起,大宋各地年年灾异,基本上各路都轮上了,唯有京西一路没什么大的灾害,正如沈括所说,年年收成都不差。

时间已经是傍晚了,正是寻常人家一日两餐的晚饭时候。一道道炊烟腾上天空。韩冈望着远山近水,发现炊烟的数量并不算多,“怎么人家都聚在官道附近,远一点的地方似乎就没有了。都说京西诸军州户口远比京畿少,想不到还当真如此。”

“不过户口再少也比熙河路要好,唐州好歹也有八万户,熙河一路的汉人户口,如今当也没有超过五万才是。”沈括信心十足的对韩冈说道,“襄汉漕运打通之后,沿线州县的户口会渐渐多起来的。就像现如今的淮南西路,开国前连番遭劫,白骨露於野、千里无鸡鸣,但开国后,依靠汴水,如今一路之地,各州各县都是富庶无比。”

襄汉漕渠之事,关系到他是否能将功赎罪的关键,容不得沈括不放在心上。甚至说牵肠挂肚都可以。对于漕运开通后的好处,也是如数家珍。

一边说话,一边前进,很快就得抵达了唐州城。一行人入城之后,直往州衙而来,这便是给韩冈的接风宴。说奢侈也不算奢侈,但菜肴和酒水也绝不便宜。如果沈括抵达唐州后,都是如此使用公使钱,到了年终查账,他少不了会有些麻烦。

韩冈并没有提醒,沈括做了那么多年官员,政治智慧欠缺,但头脑不差,此事不会不知,既然如此铺陈,想必会有积极的解决办法。

