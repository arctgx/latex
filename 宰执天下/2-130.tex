\section{第26章 西山齐云古今长(下)}

【等下还有一更。】

韩冈陪同着高遵裕站在校场点将台上,看着下面的球员在活动着身体,做着热身。蹴鞠盛行于世,这一点韩冈早已知道。就连在家里,素心、云娘闲下来时,也会带着招儿踢两脚,因为没有球门,所以唤作‘白打’。

不过在亲眼看到之前,韩冈很难相信这世上已经有了专用的足球鞋,专业的球队——唤作齐云社或圆社——连足球也是跟后世式样相差不大的内外双层。上好的足球,外面用十二块成型的硝制牛皮缝成,针脚内隐,不露于外。内胆则是用牛膀胱,可以向内充气,也被称为气毬,其重量也被规定为十二两。

尽管足球制作要求甚高,但在韩冈现在所处的这个时代,所有的集体运动中,还是以蹴鞠比赛最为简便,流行最广。因为前世留下来的恶劣印象的缘故,韩冈对足球并不感冒。只是由于如今世人对蹴鞠的爱好,才让他打算利用这项运动。

吐蕃人其实更善于马球,但古渭寨可没那么多马匹可以浪费,故而韩冈前天便很干脆的定下计划,以蹴鞠运动加强汉番之间交流活动。就在当天午后,他便通过冯从义找来几个大商家,说了几句,当即就一起凑了三十贯钱作彩头。

而疗养院这边,朱中则奉命让外科和内科各自拼凑了一支球队。虽然备选的都是五劳七伤的伤病员,但上百人中,找几个快要出院、能跑能跳的也很容易。不过依照韩冈的指示,这一队中间都是一半汉人,一半蕃人。

另外,韩冈更直接把现行的比赛规则全都改了,给出的理由是复古,私下里则对高遵裕和王厚说,规则、技巧若是太繁复了,参赛的吐蕃人怕是来不及学,有失共同参赛的本意,故而越简单越好。现在就是二十个人争一个球,往对面的球门里踢就是了。除了不许用手触球,不许故意殴打对方球员,就没有其他的规则束缚了。

大约是未时刚过的时候,点将台上,今天有空的官员终于都到齐了,这一方面因为韩冈的面子,另一方面也有高遵裕亲自过来捧场的缘故。

韩冈没兴趣出来多费口舌,事先也没安排什么垫场表演。打了个手势,一声尖利的笛响传遍校场内外,比赛随即开始。

红队一方,有着近七尺高的鲁平最为惹眼,高大的身躯通常会显得笨拙,但鲁平的动作却是令人难以想象的灵活,以他这样的身材,竟能轻而易举地把球抢走,并绕开冲过来抢球的对手。抬起一脚,皮球便直奔褐队球门而去。

虽然那一脚并没有进球,但还是引起了开场以来第一阵欢呼。王厚捂住一边的耳朵,在震耳欲聋的噪声中问着韩冈:“玉昆,你觉得哪队会赢?”

韩冈摇了摇头,凑近了道:“说不准,得往下看了才知道。不过红队的盘口比较高,因为有个在秦州齐云社做了三年球头的。”

王厚已经很熟悉韩冈的说话方式:“怎么听玉昆你的口气好像并不看好红队?”

“规则变了,踢法也该跟着变。可惜的是,有些人的习惯已经根深蒂固了。”韩冈微微带着冷笑,像是期待着可以幸灾乐祸的恶劣笑容。

球场上,鲁平把足球从脚后跟挑起,十二两重的皮球如同被吸在身上一般,顺势滚过腰背,越过他的头顶,一直落到了他的脚前。这一精彩的表演,在观众席中又掀起一阵欢腾。可是当鲁平正要再炫耀一下自己的球技的时候,却被一个褐队的球员从旁猛然撞倒,让另外一名队友硬是把球抢了去。

韩冈的声音随即响起:“其实论起技巧,褐队要远逊红队。那个剃光头发的鲁平,在秦州城中踢球的人中,也是小有名气的……不过一人之力如何当得了十人之力。何况他习惯的都是隔着球网的踢法,遇上今次的规程,肯定是要吃亏的。”

“球怎么能这么踢!?”陪在高遵裕的中年清客,尖声叫了起来。他的姓氏很特别,复姓第五,单名一个丰字。正事一点不会,但诗词歌赋、吹拉弹唱、踢球把戏却是行家里手。

韩冈露出很惊讶的神色:“第五兄此话何意,为何不能这么踢?”

“人步拐、退步踏,人步肩、退步背,这些可都是禁招!”第五丰指手画脚,他说出的这几句,便是如今通行的蹴鞠比赛的规则,也就是不许绊人、撞人、踩踏。

韩冈当然都知道,事先他找过人来问过,但他却没兴趣去让人遵守,他笑道:“第五兄此言差矣。上场的又不是待字闺中的女儿家,何必有那么多讲究?都是刀枪上取火的厮杀汉,皮糙肉厚,撞上一下,打个滚就起来了,哪需要那么多规矩。”

就是抱着这样的想法,比赛规则被韩冈放宽了许多,只要不是故意伤人便放过去,但也因此,冲突起来的几率便随之增大。

“都见血了!”第五丰突然指着球场,气急败坏的说着。

此时,再一次拿到球的鲁平被人一脚铲翻在地,可能是被缝合起来的伤口裂开了,鲜血顿时浸透了裹着头的细麻绷带。木笛声急促的响了起来,穿着黑衣的裁判中断了比赛,而从比赛开始前就守候在旁的医工则跑上前来,检查鲁平的伤势。

“见血才好!”韩冈却是不以为意的笑着,“蹴鞠本就是练兵之法,若是隔网而踢,反而失了本意。也会让蕃人小瞧了去。论起正面冲杀,汉儿当不输蕃人,何必斤斤于一干陈规旧矩,让人不得踢个痛快。傅寨主,你说是不是?”

傅勍干咳了一声,不敢搭话。倒是王舜臣性格爽快,更不怕高遵裕的清客敢拿他如何,“三哥说的一点也没错。左不能,右不能,蔫蔫的像个新妇,哪比得上现在踢得痛快……就该死命的踹,死命的撞!三哥不是说了吗,这也是唐朝时候的做法。”

第五丰冷笑了起来,王舜臣的话正是他要等的:“不闻唐时有此说,只曾见王右丞【王维】的‘蹴鞠屡过飞鸟上,秋千竞出垂杨里’。”

王维的这句‘蹴鞠屡过飞鸟上’,虽然有着夸张的成分在,但也只有把球往几丈高的球网上踢去,才能使用这样夸张的修辞,先有本,才有变。如果只是分队对着敌方的球门踢,当是不至于用夸张的词语去形容球踢得有多高。

前面随口说的瞎话,被人翻出典故戳穿,韩冈却也不脸红,哈哈笑了两声,满不介意的说道:“大概是我记错了,也许是汉晋时候的事了。”

第五丰气结,一时说不出话来。以韩冈的身份若是不要脸起来,就算他是高遵裕的清客,也只能徒唤奈何。人家明摆着要耍赖,他指出来只会自找不痛快。做人清客的最是会看人眼色。第五丰很明白,在高遵裕眼里,他连韩冈的一根脚趾头都比不上。

韩冈根本都没把第五丰放在心上。他只要两队球队能面对面的拼斗,不要像如今,你一脚我一脚往球场正中、高悬在上的球网里踢,也没有激烈的争斗,娘娘腔一般的让人不耐。所以只是拿着复古当借口,他哪里还真会去考古不成?王安石变法,也是举着复古的旗号,却又是哪里‘古’了?

越激烈的运动,其实喜欢的人会越多,要不然相扑也不会从京城热到边疆,一场相扑比赛,随随便便就能招来几千观众。而京城桑家瓦子中最大的象棚,里面的女相扑,哪天不是满场,连天子都忍不住让人进宫来表演。

韩冈其实也是很闲,所以才会在读书之余,把蹴鞠拿出来打发一下时间。当然,他不喜欢做无用功,就算消磨时间,也是要带回点好处。

若是换作前几个月,先是一场围绕渭源堡的战事,接着便是主持屯田——当时不仅是韩冈在忙碌,其他文武官员也都跟他一样忙得没有一刻得闲——哪会像现在这样,一场疗养院中的内部球赛,就引得所有官吏一齐出动。

高遵裕并不知道韩冈的本心仅仅是为了打发时间,昨日听过韩冈的一番说辞,还以为他准备当个正经事来做。平心而论,在高遵裕看来,这场比赛踢得不像样子,技巧上的差距跟京中的高手比起来实在天差地远。

但现在这样的比赛,却更是让人热血沸腾,连一开始都纳闷着蹴鞠比赛怎么变成了相扑的观众们,都开始狂吼乱叫起来。

一个精彩的冲撞抢断,让对手在地上滚得老远,总能博来一阵鼓掌欢呼。而当一名球员倚着猛烈的气势,在球场中横冲直撞,连续撞开几名敌人的拦截,把球踢进对方球门。这时候,喝彩声几乎能把天都撞破。不论普通的百姓和士兵,还是点将台上的官员,无不放下了平日里的拘束,纵情狂呼。

