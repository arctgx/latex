\section{第21章 欲寻佳木归圣众(三)}

将编纂成功不过三十年的《武经总要》进行修订,这是韩冈为了《本草纲目》日后的修订工作,而事先打下的埋伏。既然韩冈打算将《本草纲目》,扩充成一部自然百科全书,那么注定不可能圆圆满满。

但当年主编者曾公亮的儿子就在西府之中,为两府和睦着想,韩冈不打算在曾孝宽的心中留下芥蒂。

现如今,自从韩冈当初的五年国是出台之后,施政有了一个稍微具体一点的目标。政府之中,也就能相应的协调政策。两府同心协力,以增强国力为共同的目标。几年来,朝廷中虽偶有风波,但还是以和平安定的局面居多。都快要赶上仁宗前期,那段太平无事的日子。

如今的和睦局面难能可贵,韩冈没有打算再闹什么政治斗争来。而且再有两日便是廷推,两府的位置人人想要,为了重申与章惇、曾孝宽的默契,韩冈不免要多费些手段。以其过去的行事作风,今天的一番商讨,相信章惇和曾孝宽也不会怀疑他的诚意。

而且军器、将作两监,与枢密院中千丝万缕,大事小事都脱不开干系。将作监辖下还有三千兵,军器监下面的兵数量更多。韩冈虽然在两监之中有着绝对的影响力,但但凡有大举措,也会先知会西府。当着章惇、曾孝宽的面,吩咐赵子几、王居卿,同样是为了避免两人心生芥蒂。

韩冈的一番话后,尽管章、曾两位枢密使脸上的表情上没有太多的变化,可已经能感觉到气氛比他们进来时,要好了很多。

交情也好、信任也好,人际关系是要用心去维护,若是疏忽大意,很有可能在某个时候,就会出现让人意料不到的变乱。

市井中,时有贤相拿着金瓜骨朵上打昏君下打奸臣的段子。但作为一国宰辅,拿起武器赤膊上阵,其实已经很丢人了。那一次的宫变已经过去了很久,韩冈这些年,还是经常在反省,如果他当初能够多与蔡确、薛向沟通一下,或许变乱能够消弭于无形。

“玉昆,熊本今早送来的消息收到了?”章惇喝了一口凉汤,问韩冈。

“收到了。罗氏的兵力看起来不弱。”韩冈点点头,说道。

前几天是石门关一战的捷报,之后连着几日的战报都是在说官军在五尺道上高歌猛进,但今天早上的一封军情,却是在说作为全军前锋的水西蛮,已经杀到了大理边界,与大理国的军队打了一场,然后是摧枯拉朽一般的大捷。

“黄裳在夔州路手太软了。”章惇道。

韩冈苦笑道:“没办法,他们投降得太快。”

水西的罗氏鬼国过去一直叛服不定,但自朝廷定下了平定西南的战略之后,从熊本开始,十几年一直压着西南夷打。黄裳去了西南之后,下手更加狠厉,以夷制夷的手段也越来越圆熟,最后在枢密院决定消灭罗氏鬼国之前,他们先一步就降了,质子也送了,族长也亲自到京中觐见,也不好再下狠手了。

章惇道:“只盼着大理国能多消耗一点了。”

依靠战功出身的两位宰辅,一个比一个黑心肠。尽管政见、派系都有不合的地方,但对外的立场还是一致的,非我族类,死得多一点比较好。

曾孝宽道:“熊本这里十分顺利,再有两场大一点的会战,多半就能拿下大理城了。李信、李宪那边,希望也能顺利一些。要是年底前,一边占了洱海,一边占了滇池,那是最好不过。”

“时近八月,天气渐渐转凉,疾疫也少了。官军攻入大理境内之后,正好是秋时。不仅仅粮食不用操心,连气候也最适宜行军作战。再有两三月,或许就是高氏二贼授首的时候了。”

韩冈道:“就怕他们太急。”

广西的偏师,以李宪为主,由李信领军。两人带了一队神机营,又带了六门新式轻便火炮,过了方城山后便顺水而下。当年两人都参加过南征之役,现在统领广西溪洞蛮兵,以及当地的一部禁军,再有一个指挥的神机营配合,莫说牵制大理军,直接攻下高氏老巢的善阐府,饮马滇池畔,也不是什么让人惊讶的事。

但战争这件事,从来没有说百分百的胜算。兔子急了都能蹬鹰,何况人呢?

多路进军有多路进军的好处,如果挤在一条路上,兵力就会受到运力的限制;但分成几路出发,又会因为距离上的差距,无法设立一个指挥中枢来统括全军;可若是因此而让各路自行其是,齐头并进,却又有可能因为相互争功急进,最后造成当年伐夏之役初期的那一场惨败。

“苏子元在邕州多年,有他主持粮秣事,玉昆你还需要担心?而且李信、李宪都是打老了仗,前车之鉴不会清楚。”章惇笑道:“玉昆你在他们出发前,应当也没有少耳提面命才是。”

稍稍议论了一下西南的战事,章惇、曾孝宽便告辞离开。

送了章惇、曾孝宽出门,韩冈正准备处理一下手中的公事,太后那边又派人来请。

韩冈看了一下堆在桌上的公文,揉了揉额头,然后便应诏入内觐见太后。

抵达内东门小殿的时候,正好看见沈括从殿中出来。

看到韩冈,沈括先是吓了一跳,然后才知道行礼。

沈博毅、沈清直,沈括的两个儿子,一个是上舍及第的进士,另一个则是在横渠书院学习多年之后,于上一科考中了进士,位列三甲。现在两人都在外做官,刚中进士的沈清直还是县尉,而沈博毅,已经是乌程知县了。

在韩冈出手相助前,沈括的两个儿子都给他家的悍妇给赶出了家门。沈括发妻的娘家势力太弱,不然也能帮沈博毅、沈清直撑撑腰。可惜,他们没有一个能与张刍一较高下的外家。

现在两个儿子在韩冈的护庇下,先后中了进士,韩冈于沈括的恩德,可谓恩泽两代,他在韩冈面前也越发的谦恭。

见得多了,韩冈一下便发现沈括有些不对劲,神思不属,失魂落魄。

韩冈皱了皱眉,随着廷推一日近过一日,沈括也是越来越紧张,患得患失的表情,甚至连藏也藏不住。但再怎么样,也不能行诸于外,沉稳的二字评语,对宰执来说必不可少。

“存中,出了何事?”韩冈问道。

“没有什么大事,只是又有些弹章。”

沈括故作轻松的说着,只是笑容难看得很。

“又是那些,都几次了,该习惯了才是。放心,放心。”韩冈笑着安慰了两句。

沈括这两年,虽然在有了晋身两府的前景后,越发得清贞廉洁起来,做事也是鞠躬尽瘁。不论是御史,还是地方上的监司官,想要在他的账簿上找麻烦,都无功而返。但只要是在朝堂上做官,就没有不出错的时候。

欲加之罪,何患无辞?李南公附和新党,不嫁同产妹这等私人家事,便被旧党御史拿出来敲打——虽然这件事在如今的确挺严重——御史找不到李南公贪赃枉法事,便去刻意翻他家里的老底。沈括比李南公问题更大,一个是过去没节操的事做得太浑了,旧账一次次的被人翻,另外一个,就是治家无方,连家都不齐,还如何治国平天下?

每一次沈括想要晋身两府,都会被一堆弹章砸到头上,这一次,他晋身的希望大增,头上的炮火也更加猛烈。

沈括故而也苦笑得更厉害。韩冈要他放心,但如何能放心。

之前韩冈也劝过,说是‘不招人嫉是庸才,存中你既然得太后看重,自是不免议论。朝廷设御史,也是催人勤谨。章疏中所论过犯,有则改之,无则加勉,不必忧虑重重。’但要是这么简单就好了。

韩冈也只是劝慰,既然是沈括本人犯下的事,他自己当然要为这些事负责。

太后正在殿中相候,匆匆两句话,韩冈别过沈括,便来到殿前。

通传之后,被招入殿内,正好看见一个朱红色的背影没入后门之中。

天子的常服是朱色,瘦削的背影韩冈更不会看错。

知道韩冈看见了小皇帝,待韩冈坐下,向太后便解释了一下,“官家有些累了,先让他回去休息。”

太后虽是如此说,却也不知赵煦究竟是主动离开,还是被太后请出去的。这小皇帝年纪越来越大了,自幼聪慧,却因一个意外,长年累月之下,性格不免扭曲。但韩冈也没太在意,小皇帝想要亲政还早得很,而太后的身体也十分康健。

“相公方才进来时,看见沈括了?”向太后问道。

“是,臣看到了。”

“以相公来看,沈括当真适合入两府?吾这几日收到的弹劾上百封,全是说得他不是。如果是为了酬奖沈括修铁路轨道的功劳,也不必给他一张清凉伞,金银什么的朝廷也不会吝啬。”

“陛下明鉴。功高易赏,即便不赏赐,沈括也不敢有怨言。但日后沈括要负责铁路轨道的一应事务,他若没有足够的权柄,便难以使动地方上来配合。”

“但还有说他不能治家,继室逐子而不能制。沈张氏看起来也不是悍妒的样子。若沈括日后以宰辅之尊,还要受辱家中,岂不是朝廷之耻?”

“房玄龄亦有悍妻,但唐太宗用其为相,虽有贞观之治。沈括虽私德有亏,家中不靖,可其才足以治国。”韩冈起身,“沈括人才难得,臣愿陛下尽用其才。”
