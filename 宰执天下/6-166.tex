\section{第15章 经济四方属真宰(下)}

“官家今年就十一了。”

韩冈今日押班,退朝后,从文德殿转去垂拱殿的路上,忽然就听到同行的蒲宗孟小声说道。

韩冈抬起眼,等待他的下文。

“一转眼就这么大了。”

蒲宗孟感慨着。

“小孩子,长大也是一转眼。我家的几个小子,之前连走路都不稳,一转眼,都能出去游学了。”

还有六年。

天子大婚一般是在十七岁,之后就可以亲政了,最迟也不该超过二十岁。不过有章献皇后和仁宗的例子在前,向太后一直执政到她去世都可以——章献明肃刘皇后便是在真宗驾崩后一直垂帘听政,等到她过世,仁宗亲政时,都已经二十四了。

“若说游学,天下哪有比得上京师的?相公可真放得下心!”蒲宗孟悠悠说道。

儿子可以丢到外面去,小皇帝就必须约束在宫中,只要赵煦不能听政,当然放得下心。

“哪里能放得下心?”韩冈停了一下,叹道,“出门半日就开始担心了。但都那么大了,总不能留在家里读死书。读万卷书,行万里路,不多走走,多看看,也难成才。而且去的也是横渠书院,就跟自家一样。”

“原来相公家的大公子去的是横渠书院!”蒲宗孟惊讶道。

韩冈将儿子送去横渠书院读书,而不是安排进国子监,京城中有几个不知道的?亏得蒲宗孟能装得出一幅才听到的惊讶模样。

气学宗师韩冈做了宰相,士人们都在猜测他什么时候将国子监中的教科书给改掉、将礼部试中的科目给换下。士林中,为此而开赌的不计其数。

韩冈即使再大度,也没人相信他会不在乎儿子拿不到一个进士头衔。既然他把儿子都送去了横渠书院,那他改变科目是迟早的事,至少在他儿子能够参加进士科考试之前,肯定会改掉。

“横渠书院是先师明诚先生和韩冈的心血所寄,若是犬子不去,那还有谁会去?”

“相公真是一片苦心啊。”蒲宗孟长叹道。

话题从皇帝身上,给韩冈强行扭转到了他出外的长子身上。蒲宗孟知道韩冈不想多谈这个话题,不过他不信韩冈不介意。

深得太后看重的大臣,在赵煦的朝堂中肯定找不到位置。若是现在的皇帝亲政,肯定会打着绍述熙丰之法的名义,趁机将韩冈的党羽清洗出朝堂。

现在还有几年,甚至十几年,但说不准什么时候就变了天。

小皇帝一天天长大,韩冈怎么可能放心得下?今天不想提,明天也得提。只要等着就行了。

不过蒲宗孟不想等。

有些事,等一下,就彻底错过了。

事不过三,亲自提出廷推之法的韩冈,绝不会允许下一次的廷推再没有结果。

“读万卷书、行万里路,此二句出自于《九域》,想不到相公也看此杂书。”

《九域游记》虽是佚名,可有几个不知道这是他写的?韩冈淡淡瞥了蒲宗孟一眼。

这位老资格的翰林学士承旨,在玉堂中的时间差不多可以算的上是开国以来前三名,现在虽然在笑着,脸色却有些发白,有些紧张。

一张清凉伞,竟然如此挂怀?

韩冈知道若是自己把心中的想法给说出来,立刻就能成为满朝文武憎恨的对象。不是每个人都像寇准、韩琦还有他韩冈这样,进入官场不久,便被视为宰相之备,之后一路顺风顺水。绝大多数朝臣,能够拿到清凉伞的几率近乎于零。就是蒲宗孟这等已经熬老了资历,距离两府只有一步之遥的臣子,也对横拦在两府与朝臣之间的那巨大的鸿沟,望而兴叹。

“闲来无事。我不善诗文,一下就少了多少文集打发时间,总不能天天读经。”

蒲宗孟哈哈笑了两声,道:“相公说的是,读史读经是打发时间,看话本也一样是。以《九域》为肇端,才几年功夫,市井中话本之类的杂书越来越多了,还有杂剧,也多有所谓剧本在流传。”

“哦,是吗?”韩冈饶有兴致的问道。

“宗孟岂敢胡言乱语?现在就有《莺莺传》改的杂剧本子,前日在玉堂,宗孟听说乌台有人上表,说是诲淫诲盗。或许……”蒲宗孟顿了一顿,压低声线道:“或许日后的剧本就不只是诲淫诲盗了。”

不是或许,是已经有了。

“是《许止传》?”韩冈直接挑明了。

当今天子,乃是弑父弑君之人。这让十一岁的小皇帝,在天下士民的心目中,绝不是一个合格的皇帝。他们可以用感慨的口气说这是宿世冤孽,但绝不代表他们会否认对小皇帝弑亲弑君有罪的判定。

甚至成为帝师,都已经不是朝臣和大儒们目标。程颢都回了洛阳——新学依然盘踞在朝堂上,而气学则挤占了剩下的所有空间,不想成为帝师,又没有办法在京城士林中站稳脚跟,他也只能回去。

天下士民都觉得这个皇帝不合适,为大庆殿中的那个位置而动心的人自然就不会是一个两个。

有人能够想到用话本来传播目标,自然也会有人用杂剧来达到目的。

京城的各大瓦子中,上演杂剧的舞台没有一日停歇。在九域游记出现之前,就已经有抨击时事的新出剧本,逆王赵颢在市井中的名声,便是一出出杂剧给毁掉的。在《九域游记》出现之后,越来越多的剧本开始从目连救母之类的神鬼故事中脱离出来,开始贴近现实,影射现实。或许现在就是杂剧历史上的第一个高峰。

《许止传》主要内容就是许止弑君,另外还参杂了另外的一些传奇故事,由此敷衍成篇。许止的结局也不是历史上的逃亡国外愧疚而死,而是改成了许止自尽,临死前自诉的那一场,算是很催泪。如果用后世的话说,是现实主义悲剧中的杰作。

不过这部杰作,不必多有见识,看过了就知道是直指御座上的小皇帝。只是幕后黑手,还是扑朔迷离。

“正是《许止传》。”蒲宗孟见韩冈不再绕弯子,精神顿时一振,“这一部,必是有心人所著。”

“传正意为何人?”

“宗孟看《九域》,其中有林冲断案一节。其中有一句说得最为合意:谁得利最多,谁嫌疑最大。”

濮王一系,自英宗后便成了宗室中最为尊贵的支系,后继者当然有可能从他们中挑选出来。但更名正言顺的继承者,虽是濮王系,却不是出自濮王府。

“嗯?”韩冈不说话,只用鼻音表示询问之意。

蒲宗孟咬咬牙,低声道:“三大王的儿子最多,不是他,还有谁?”

韩冈笑了。

终于说出口了。

回头再看蒲宗孟,却像是刚刚经历过一场大劫,额头上皆是冷汗,脸色亦是苍白。

“三大王重病已多日了。”

已经开了头,蒲宗孟完全不怕了,直言道:“英宗皇帝被选为皇太子时,濮安懿王早已不在人世。”

这其实就是濮王一系为何能出一个皇帝的原因。能开枝散叶,可保皇祚不绝——英宗家中排行十三,而英宗的亲兄弟,有二十一个之多——同时,生父不在人世。否则新帝以继子登基,置生父于何地?

赵覠这两年身体欠佳,从年前到现在,所有的朝会都缺席了。按太医局方面的回报,赵頵已时日无多。

赵頵喜好医术,还组织人手编订医书,近两年沉湎于生物分类学中,完全不理世事,宗室中有贤王之名。但他最大的问题,是喜欢自己给自己开药方,太医给他开过的方子,都要自己过目,很多时候,都会添减一二。日常饮食,包括养生的饮子,都出自己心,

缺乏经验、只抱着医书的医者,比点着的火药包还危险。

高太后所诞四子之中,除了甫出生、尚未赐名便告夭折的那位皇子,以幼子赵頵的体质最弱,比他的两位兄长都要差,总是爱在日常用药上折腾,在韩冈看来,其实就是自杀。

但他的儿子多,而且是很多,赵顼只有一个儿子;赵颢有三个,皆贬做了庶人,至于赵頵,时至今日已多达八人,如果他恢复健康的话,这个数量还会继续增长。八个儿子中,就只夭折了一个。

这也是多亏了医学的进步,因为牛痘的出现,以及护理学的进步,皇室婴幼儿的夭折率一下降到了不到十分之一,普通百姓也降低了许多,也许在后世,千分之一百的幼儿死亡率绝对是骇人听闻的惨剧,可在此时,已经可以被世人视为奇迹。而赵頵夭折的那个儿子,是他的长子,正是病夭在牛痘出现之前。

天子年幼,又无幼弟,依照血缘关系的远近,赵頵的儿子中,排在后面几位的都有可能成为帝位的继承人。可以说,如果赵頵不是重病缠身,他必然会被视为这段时间以来,在市井中散布谣言攻击天子的幕后黑手。不过从另一个角度来说,如果赵頵近期病死,皇帝的继承人基本上就确定了。

