\section{第36章 沧浪歌罢濯尘缨(31)}

“四郎,通济坊到了。”

时隔多月,再一次看见通济坊的门额时,车夫叫醒了车厢内闭目养神的冯从义。

冯从义眨了眨眼睛,坐直了身子,就在车厢中整理起自己衣服来。

位于通济坊中的宅院不是三衙、枢密院和在京百司的秘密基地,也不是冯从义手底下的顺丰行新近搬来。而是雍秦两地商人集资共建的雍商会馆的所在地。

片刻之后,冯从义的马车停在了一间宅院外,从车上下来,面前的雍秦会馆就跟普通的地方会所差不多。完全看不到一掷千金的底蕴。

冯从义悄然抵京,但还是惊动不少人,而这些人,眼下泰半守在雍秦会馆中。。

说起来冯从义其实并不想在夏天出门,不过韩冈的生辰将至,还是三十岁的整生日,在情在理都要来一趟京师。

此外还有连韩冈也看重的飞钱业务。即将开张的飞钱商号,并没有打着顺丰行的名号,而是起名做平安号。平实到朴素的名字,很难让人想得到,这是当世大儒韩冈亲自给起的。

平安号将会在京城开设分号。让将钱钞存在平安号中,然后拿着凭据,到关西再取出。

京城的商人也可以将钱钞存在开封分号中,然后拿着凭证,到关西直接购买当地的特产,不必到了关西再兑换一遍。

顺丰行依然是平安号的唯一出资人。只有声名卓著的韩冈做后盾的顺丰行,才能够得到雍秦商会内部的信任。也只有韩家才能调动起拥有足够实力和数量的汉蕃精锐来完成两地钱钞运送的工作。西军的子弟,吐蕃人中的强手,凭韩冈的面子,冯从义的人缘,一句话就能调来百八十个来看家护院。

不过现在雍秦会馆中,最关心的并不是开业在即的平安号,而是韩冈在河东的去留。

刚刚被迎了进来,还没坐定,冯从义就给包围了起来。

“冯世兄,枢密能回京吗?”

“冯四哥,枢密可还说了些什么?”

“冯兄弟,王平章这一回可是连翁婿情面不讲了。”

人声嘈杂,闹得冯从义头晕眼花。

“诸位担心什么?”冯从义双手向下压了两下,示意他们安静,“想想太子,想想官家。皇后怎么可能会答应将枢密留在京城之外”

政事堂下面的检正中书五房公事,以及再下一级的吏、户、礼、刑、工各房的检正官,只要把这几个位置抓在手中,中书的权柄就不会落到其他宰辅的手里。这里还没提已经给瓜分完毕的台谏官,几乎都是在京宰辅们的应声虫。

所以他们都不想韩冈和吕惠卿现在就回京。只要再迟一点,就可以只剩宜春苑、玉津园、琼林苑和瑞圣园这几座皇家园林的管勾官,留给韩冈和吕惠群两人了。

吕惠卿多半很急,雍秦商会的成员也很急,不过冯从义的情绪稳定。

冯从义可以确信,皇后肯定是希望自己的表兄回去的,之所以无法成功,是因为宰辅们联手干扰。除非是赵匡胤、赵光义在位,否则就是之后的真宗、仁宗,遇上眼下的局面,也只能暂时收手。但他再了解韩冈的性格不过,岂会因人成事?

无论在京宰辅们怎么样折腾都改变不了韩冈在天下人心中的地位。只要太子身体有些不适,只要天子的病情稍有反复,就没人敢拦着皇后将韩冈召回。

从一开始,冯从义就知道,韩冈想要宣讲气学肯定会阻力重重。随着他一心想要将他的‘气学’发扬光大,挡在前面的拦路石将会越来越大。现在的局面,明眼人早就看出来了。皇帝都做了好几次堵路石,但都给韩冈一脚踢飞了,现在只凭几名宰辅,又能当住多久?

天子不能视事,皇后又缺乏经验。少了上面的束缚,外在的威胁又不复存在,在京的宰辅们要做的,自然是肆无忌惮的划分势力范围,抢夺重要的职位。

“就算家表兄和吕枢密不打算争权夺利,照样会被提防着。何况怎么可能不争?这其实跟做买卖一样啊。”冯从义说得肆无忌惮。

不论到哪里开商号,地头蛇哪有会主动让出地盘的?如果不能亮出后台把人镇住,就要争斗一番了。有放火烧屋的粗手段,也有收买衙役上门找茬的细手段。往往都会粗细搭配着来。

“文诚先生刚刚去世没几日,程家夫子就赶着过潼关。王平章甚至把枢密这个女婿当贼防着。大家都看在眼里。”

“冯世兄,有什么要我们做的?”

“家表兄做事,何曾因人成事。他想要回来就能回来,说起来家表兄的三十岁整生日就要到了,肯定要赶回来跟我那表嫂和侄儿侄女团聚。”

反倒是地位更高一点的吕枢密,他回京的可能要低上许多。

跟新党相比,韩冈手上的势力可谓是微不足道。但只要有王安石在,吕惠卿就控制不了新党,新党也不需要第二根主心骨。而韩冈,支持他的力量,却要比吕惠卿来得大。

“眼下的问题,对家兄来说,只能算是道小门槛。真要回来,可是当轴诸公能当得住的?”

没人会质疑。现在大家都还记得,当初天子想要把韩冈留在京西,韩冈直接就把牛痘拿出来了,逼得皇帝把他召回京城。

“要是再有个牛痘就好了。”

“哪里有那么容易,上一次的牛痘,可是用了整整十年才得到一个好结果。”

用了十年的时间才得到的收获,哪里可能说拿出来就拿出来。但冯从义相信韩冈肯定能够有办法让朝廷不得不将他召回京城。只要回了京,不管是用什么名义,韩冈都有办法留下来——就是吕惠卿也肯定有办法。这两位枢密使现在的问题仅仅是不能回京。

冯从义之所以不为他的表兄担心,是因为韩冈的手中有实力和规模都冠绝中国的雍秦商会。

他的目光扫过了厅中,雍秦商会所代表的,就是厅中之人背后的庞大势力。

跟随韩冈的脚步将势力扩展到天下四方的这个商业团体,行事却十分低调。京城只有棉行这样的关西行会,以及顺丰行为首的商号,几乎没人听说过雍秦商会这个名字。可是实际上,其向心力远比京城两大总社那种松散的联盟要强的多,筹备了许久的飞钱业务即将开始运行,各家的联系将会越来越紧密。

因为关西属于铁钱和铜钱通用的区域,与关东币制不同,从商业上便与崤山以东有着很深的隔阂。而气学扎根于此,随着时间的推移,从学术到商业,已经与中原分道扬镳。有了雍秦商会的支持,天下各路的蒙学中还是以千字文、兔园册为蒙书。而关西早已变成了三字经、算术和自然。只有论语是共通的。

王安石能够将他的新学捧成官学,让三经新义成为钦定的标准。那么当韩冈当政之后呢?以他对气学的重视,会继续留着新学在国子监中一统江山不成?

没有人会怀疑韩冈日后能不能成为一国辅弼,那只是迟早的问题。一旦气学成为官学,那么自束发受教以来,便进入气学门墙的关西子弟,便有了绝对的优势。

比诗赋,关西永远赢不了人文荟萃的南方。比经义,关西士子也比不过中原、河北的文人。在过去,西夏尚为中国之患的时候,多少关西士人去精研兵法,打算依靠军功跻身官场。张载都是其中之一——要不是范仲淹让张载回去读书,如今世上也不会有气学的存在。

但如果有更好的进路,谁还会去冒风险去投军?

冯从义知道,关西士林中,已经很多人已经赌在了他的表兄身上。

只要韩冈有需要,自然会有人愿意帮忙。若是韩冈点火,自会有人去扇风。若是韩冈要放水,自有人去掘堤。

有此为凭,加上皇后的看重,韩冈回京只是时日问题。

从会馆回到在京置办的宅邸,冯从义很安心的让下人去整理礼物,收拾好后便去韩家的府上拜访。

“四郎,这是枢密昨日的奏本。”冯从义随身的家丁悄无声息的进来,递上了一片纸页,然后躬身退下。

冯从义拿起来只扫了一眼,脸色顿时变了。

与外国使者的对话可都是要记录并上报的。韩冈与张孝杰的对谈走得正常的驿传,经过十余天后,终于抵达了京城,

天下奏章,基本上都要经过通进银台司——其‘掌受三省、枢密院、六曹及寺、监、百司奏牍,文武近臣表疏,及章奏房所领天下章奏案牍,具事目进而颁布于中外’。通进银台司下属的通进、银台两司的吏员几乎都有这样的本事——一眼看到几千字奏章中的重点,并牢牢记在心中。

比如哪路旱,哪路涝,或是朝野内外哪个的官员被弹劾,又或是那位的官员受到了荐举,这些都是有价值的情报,记下来后都可以拿出来换钱的。有的是人拿钱来买。

地位越高的官员,他们奏章中蕴藏的价值就越高,韩冈的奏折当然是属于价值最高的一部分。

之前的奏章,韩冈举荐了一批幕僚。不过还留下了很多空缺让在京的候补官员争抢。具体的官缺,是很多人想要的。

韩冈最近的这一封奏章,并不是让人关心的官阙问题,不过这封奏章传出来的信息更加让人不由得悚然一惊。

大宋万里疆域,竟然快要不敷使用了。人口日多,而田地不增,长此以往,的确免不了韩冈所说的那一个结局。

牛痘,是救人,还是杀人?

耸人听闻的题目,转眼就在冯从义的脑中闪过。如果以此为题,这一期的报纸,肯定会卖得很疯。

“这是投石问路……不对,是兴风作浪。”

韩冈投进水里的石头太大了,已经不是问路的路数了。

竟然当着辽国宰相的面说出这番话,看来无论宋辽,哪一国的朝堂都要乱一点或许才符合他表哥的心意。

韩冈的一番话,使得两国未来的国策都要受到影响,甚至点明了日后大宋将会大举扩张。

指点江山都到了这一步,究竟是召他回来,还是不召?
