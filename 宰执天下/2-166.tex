\section{第30章 肘腋萧墙暮色凉(一)}

【第二更,求红票,收藏。】

在京城盘桓了数日,在年节前即将祭灶的日子,韩冈才刚刚离京就任。对于盼望他及早上任的种谔、种建中等人来说,这并不是什么好消息。虽然他们根本不可能知道韩冈何时离开东京城,但东面始终没有消息过来,让种建中还有种朴都等得有些不耐烦了。

“喂,十九,韩冈到底什么时候能到?”种朴问着沙盘边的种建中。连日围着沙盘推演战局,让他的头都痛了,但他的堂弟却是乐此不疲,一遍遍地重复,丝毫不嫌厌烦。

“该不会不来了吧?”种朴又追加了一句,他坐在火盆边的交椅上,两脚翘上另一张交椅,舒舒服服的仰靠着。顺便一把捞起几块放在一边几案上的莲花糕,一股脑的全都塞进了嘴里,用茶冲下满嘴的食物,等着堂弟的回答。

种建中低头看着沙盘,专心致志。以无定河为中心,从绥德到罗兀再到山后的银州,全都事无巨细的描绘了出来。在这份精细比例的地形图上,有着最新的军事部署。不论是大宋的情报,还是西夏的情报,竟然都出现在沙盘上面。即便延州城白虎节堂中的那幅更为巨大的沙盘上,也没有如此精准并即时的军情。

这不是朝廷派出的谍报所能做到的,而是种家细作的功劳。从种世衡开始,种家三代镇守边地,西军将门世家手上所掌控的人力资源,在这幅沙盘上淋漓尽致的表现了出来。

种建中对着沙盘沉思良久,只分出一部分心思随口应付种朴:“韩相公前后两次至书朝廷,点名要韩玉昆来延州。就算天子也要卖宰相的脸面,韩玉昆尚是选人,当不至于会拒绝,也拒绝不了。”

种朴也算清楚堂弟分心二用的本事,“那也该到了。前些天韩相公去京兆府,不是说当日韩冈正好从那里经过,还见到了你的那位姓游的师兄,叫游师景的那个!”

“是游景叔,讳师雄的!”种建中很不高兴的抬起头,都见过几次面了,种朴竟然还没记得姓名,“前几天游景叔来信,对韩玉昆深为赞许。说以其之才,当能对战事有所助益。”

其实游师雄给种建中的信中,依然老调重弹的说北进罗兀太过冒险,要小心为上,还说韩冈跟他是一样的看法。不过种建中并没有说出来,不出差错的话,韩冈很快就要到延州上任,没必要让他还没到的时候,就在鄜延军中得罪人。

“说是有所助益倒是没错。”厅中并不止种朴、种师道两兄弟,还有最近跟着担任种谔副将的叔祖折继世,一起来到绥德的折可适——被郭逵赞为‘将种’的麟府折家新生代.

折可适对两名好友说着:“今次攻打罗兀,事发突然,出其不意,当不至有太大的伤亡。韩冈未至,暂时也不会有何影响。但到了一两个月后,西贼点集兵马,南下反扑的时候,军中如果再没有把疗养院建起来,军心怕是要大挫。”

折可适跟年龄相当的种师道、种朴打得火热,说话也少顾忌,“秦凤因为有了韩玉昆,每一个百人都,皆有一名医工来拯救危急。此事军中都已经传遍了,其余各路军中,多少人都在盼着何时能推广秦凤的德政。韩冈来不来,对军心士气的影响可是大得很。”

“这叫不患寡而患不均。”种师道半开玩笑的说着,“如果都没有倒也罢了,现在就秦凤一家有着疗养院,士卒得病都能得到安治。看看别人,想想自己,谁也不会甘心啊!”

折可适笑道:“圣人说得当真有道理。”

军中医疗,从种谔开始,到下面的种建中、种朴都看得很重,只要不是空读兵法、从未领军的赵括马谡之辈,一个完备而有效的军中医疗制度,能给战事带来多少好处,再糊涂的将领都能体会得到。

“当年先祖父守清涧城,逢上士卒有恙,都会遣几位叔伯还有家严中的一人,去专管他们的饮食汤药,所以能得人死力。”种建中对折可适解说着种世衡的丰功伟绩,“韩冈做的其实就是先祖当年所为,不过规模更大上一些,也显得更为正式一点。”

“此事俺也听说过,尊祖的确善抚士卒。”折可适点着头,表示自己听过,“韩冈能跟尊祖做得差不多,已经是难能可贵了。何况他还有一个药王弟子的名头在,有他在军中守着,那些愚夫愚妇,也能安心上阵助阵。”

“不过韩相公好像有些不喜欢韩玉昆。”种朴不像种建中,他在外面就一个大大咧咧、除了战争,其他是都不放在心上的衙内。但种朴察言观色的本事,其实远在他粗豪的外表给人的印象之上,“前几天韩相公来绥德,听到韩冈的名字脸色就有些不痛快了……”

“韩玉昆讨不讨韩相公喜欢,那是他的事,我们只求他能把他的分内事做好就行!”

一个洪亮得能震动屋瓦的声音传进厅来。种朴等人纷纷起身,向着大踏步跨进厅中的绥德主帅行礼。

种谔大步走到沙盘边,望着用蜜蜡雕出的重重山峦,上面密密麻麻的小洞,都是一次次推演留下来的的痕迹。即将领军北征的大将笑了,为自己子侄的勤力而高兴。

他回转身,一手指着横山的层峦叠嶂,高声喝问:“自好水川之后,至夺绥德为止,我大宋在此处可有分毫进取?”

几人微一犹豫,便同时摇头:“没有!”

“可有攻夺一座西贼重镇?!”

更为响亮的回答齐声响起:“没有!”

种谔的笑容更为自负,放声道:“所以说……这三十年来,我们将是第一支重返横山深处的皇宋官军!”

“三十年了……我们已经隐忍了三十年了!”

自从三十年前,韩琦主导的北进攻势,因为任福惨败于好水川而宣告终止。范仲淹倡导的堡垒防御,便成了对夏战略中不可撼动的圭臬。陕西、河东两地的战局,便一直都是西夏攻,大宋守。偶尔的反击,也不过是战术性的攻势,往往一攻即退,再无长力可言。

这三十年来,为了守卫绵延数千里的防线,每年投进去的各项开支,吞吃掉了全国总军费的四成;林林总总的徭役、兵役,也几乎耗尽了陕西的民力。但即便困厄如此,朝中诸公还是反对任何进取之策。

三年前,种谔得到天子的密旨,费尽心力,引得西夏绥德守将嵬名山来投。而这个功劳,在枢密院被定性为贪求边功、无端生事,因为将其降罪夺职,连居中联络天子的高遵裕也受了牵连,一同被降职。要不是郭逵坚持,连绥德城都会被文彦博给还回去。

在枢密院的诸公眼中,年年巨额的军费支出,加上捱打后,还要腆着脸送给西夏人几十万岁币,都比不上天子绕过枢密院,直接命令地方武将的危险。种谔时常在想,是不是这不要脸的事做久了,就会成为习惯。

范文正当初因为大宋军力不振,所以才选择了保守的战略,到了如今却成了不能触动的规矩,任何想振作一番的将帅,都会遭到枢密院的打击

岂不知事过境迁,时势更易,如今的局面已经不是当年元昊崛起时可比。三川口、好水川和定川寨三次惨败耗尽的西军精锐,如今经过了三十年的时间,也已经逐渐恢复了过来。该到了反击的时候了。

“幸好圣天子在位,又有韩相公的全力支持,我们才有放手施为的机会!”在种谔的心中,他才是横山战略的主帅,而韩绛的作用则仅仅是坐镇后方。“今年夏时,西贼虽在罗兀筑了一座寨堡。却不过是个不及百步的寨子,最多也只能做一做烽火台。由此可见他们的对罗兀并没有重视起来。而我们这一边,虽非雪夜潜出兵,但攻其不意,必定是出乎于西贼意料之外。”

忽略了作为闲杂人等的折可适,种谔愤愤不平的对着种建中、种朴说道:“你们的祖父,在军中辛苦了一辈子,世人皆将他与狄青齐名并称。无论是范文正【范仲淹】,还是欧阳永叔【欧阳修】,都是把你们的祖父与狄青并排写在奏疏上。但如此功绩、如此才能,却连横班都没入过!

好不容易设计离间了李元昊和他手下的大将野利旺荣、野利遇乞两兄弟,让李元昊将两人冤杀,却还给庞藉给抹去了功劳。你们的大伯去京中评理,又给强押了出来。——当时有人说这是冒功。但他们也不想想,若非真有其事,你们大伯吃了熊心豹子胆,跑去京城跟一位宰相过不去?”

“但今次不同了,有韩相公全力支持,又有早早的报予天子,没人能吞没我们的功赏。”种谔紧紧握拳,“整顿兵马,兵发罗兀,要将这百多年来的恩恩怨怨,亲手结束在我这手上!”

