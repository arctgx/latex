\section{第九章 长戈如林起纷纷(四)}

【第一更,求红票,收藏】

星火如海。

银河黯淡。

在韩冈等人眼前,数以千百计的火炬所组成的海洋正在沸腾。一条光焰自星火之海中分出,那是集结了数百名精锐骑兵的队伍,就如沿着河道逆流而上的潮水,争先恐后,向着韩冈这小小的队伍直扑而来。

千百人的呐喊同时暴起,与仍未停息的号角声一起穿梭在山谷之间,直往云霄传去。谷地两侧的山壁将声浪一重重的放大,最后汇成的巨大轰鸣,与奔流而来的蹄声汇合,就像突然卷高的潮水,要把韩冈等人彻底埋葬。

相对迎面而来的滚滚洪流的喧嚣,从古渭寨出来的队伍静得可怕。自他们点起火把走进谷地,到现在也不过才一刻钟的时间,想不到青唐部就已经点齐了兵马,就在他们面前,掀起了如惊涛骇浪一般的声势。一股难以名状的恐惧在队伍中弥漫开来,紧紧攥住了在场众人的心脏。

——至少韩冈除外。

“无聊的把戏。”对青唐部的行动,韩冈嗤之以鼻。他拍着马鞍哈哈大笑,嗓门提得更高,“无聊的把戏!为了预备了这套猴戏,俞龙珂和他的人怕是这两天都没能睡好觉!”

王舜臣第一个反应过来,郁郁沉雷一般的蹄声中,他跟着韩冈放声大笑,他的声音压倒了千军万马,“三哥说得没错!青唐部的这些蕃贼肯定练了不止一夜!”

“以如此大礼来迎接我等,俞龙珂当真懂得接客之道。”韩冈的音量沉下去,带着讽刺,直透人心。

“俞龙珂那老货,肯定是心虚了!”王舜臣毫不客气戳着青唐部的老底。

一番对答,队伍中的紧张气氛终于一扫而空。

一群青唐骑兵终于冲到了韩冈面前,火粉散落,光流围绕着十二人的队伍旋转,马蹄声碎乱如雨,鼓点一般杂乱的响着。他们转着圈,口中不住呼喝,尽情的对这支小队施加着的更大压力。

韩冈高居马上,腰背挺得笔直,微微仰起脖子,不屑的瞥着这群装模作样的青唐骑兵。王舜臣则紧紧的钉在他身后,左手搭着弓袋中的战弓,右手反背身后,他的箭囊就挂在马鞍后。在两人周围,由十来把火炬组成的小小圆阵纹丝不动,就如同矗立在江心的一座礁石,任由风吹日晒,狂涛怒浪,依然千百年也毫不动摇。

这是数百与十二之间的对峙,人数上的绝对劣势,却不影响韩冈一众的坚定。

韩冈拍马上前,独立在众军之间。深吸一口气,他放声大吼:“本官乃皇宋秦凤路经略安抚总管司勾当公事韩冈是也。今奉命来见贵部的俞族长,有要事相商,尔等还不快快给本官带路!”

韩冈的声音在夜风远远的传出,对面的骑兵顿时一阵骚动。他们也没想到今夜过来的,既不是董裕的部众,也不是被攻打的七家部落,却竟然是大宋的官人。

青唐骑兵中稍稍乱了一阵,一个骑手也拍马出阵,他与韩冈隔着三丈在喊,“莫要诓人,你说你是个官人,可有什么凭证?”

韩冈哈哈大笑,放纵的笑声是在嘲笑眼前的蕃将不懂看人:“吾乃是朝廷命官,岂是闲杂人等可以伪装得来。莫要多说他话,去通知俞族长,本官的身份自有俞族长来评判。”

骑手狠狠盯着韩冈两眼,转身穿出了包围圈。韩冈在千百人的环绕下静静的等着俞龙珂的答复。大约两刻钟后,包围圈又被打开,蕃将转回,却是带了了俞龙珂肯定的答复。

青唐部是过着定居生活的吐蕃部落,除了粮食,出产以盐为主。因为据有几口出息丰厚的盐井,青唐城虽不大,但俞龙珂的居所之内,却到处用着精美的丝绸和瓷器作为点缀和装饰,处处透着暴发户的气息。

在青唐城门口,韩冈的十名随从被拦住了,只放了王舜臣过来。而到了俞龙珂居所的主厅外,王舜臣也被拦在了外面。王舜臣作势欲怒,却被韩冈阻住,命他在门外安心等着。

韩冈踏步上前,一左一右站在厅门口的两名守兵夹过来要搜他的身。韩冈的眼神顿时锐利起来,如刀锋一般将两人瞪住,然后向内提声问道:“敢问俞族长,这可是青唐部的待客之道?”

停了一下,一个低沉浑厚的声音从厅内传出:“不得无礼。快请韩官人进来。”

跨入厅中,韩冈终于见到了俞龙珂。

青唐部的族长如今是四十上下的年纪,相貌古拙,高挺的鼻梁在脸上拉出了深深的阴影,一根粗大的发辫盘在头上,油腻腻的反射着火光。俞龙珂见着韩冈入厅,却还是稳坐不动。而在大厅两边,十几名青唐部的首酋们分作两排,也是个个安坐如山。

“不知韩官人连夜来访我,到底是为了何事?”俞龙珂也不请韩冈坐下,就这么直接问道。

“我是来向俞族长求援兵的。”韩冈开门见山的回答,毫不掩饰自己的目的。说完他弯腰行礼,神情也是诚恳无比,起身后又重复强调了一遍,“我是向俞族长求援兵来的。”

换作是别人来青唐部做说客,不用说,肯定是拿着上国官员的谱,先威吓一番。但韩冈不一样,他首先肯定的是俞龙珂的智商,不会把蕃人都当成容易欺骗的蠢货,第二点他清楚他是来求人的,第三点,韩冈多年的经验告诉他,自曝其短的实话其实也很有用。

俞龙珂愣住了,难以置信得几乎要揉起眼睛,什么时候宋国的官人会向蕃人弯腰了?他狠狠地搓了搓胡须,平复住有些混乱的心情,韩冈这卑躬屈膝的姿态,让他分外感到痛快:“想不到你们也有求人的时候?!”

俞龙珂的话让周围的首酋们一阵哄笑,而韩冈神色不为所动。

“如果只是为己,古渭寨并不需要青唐部的一兵一卒!”韩冈的声音冷了下去,前面低声下气过了,现在就是要让他们清醒一点了,“想必俞族长也清楚,以古渭寨的高墙深垒,即便只有一千人,凭着董裕也是打不下来的……而且他敢打吗?董裕有这个胆量吗?他敢不顾俞族长的脸面兵犯青渭,可他不敢向古渭射出一箭!”

一个年轻的首酋嘴角翘起,冷笑的问着韩冈:“那官人何必来求救兵?”

韩冈只对俞龙珂说话:“韩冈今次漏夜至青唐见族长,只是为了亲附我大宋的七家蕃部来求救。”

“为那七家蕃部?”俞龙珂的脑筋一时没转过弯来,追问道:“是为张香儿他们求救兵?”

“当然。”韩冈点头道,“古渭寨的守军如今自保有余,却无力向外救援。我家王机宜念在七部一向恭顺的份上,不忍他们受董裕所欺,所以遣本官来向俞族长讨个人情,求个援军。”

“官人是来诳人的吧,什么时候宋国会在乎我们吐蕃人的性命了?”另一个老首酋毫不客气的说着。

“既然张香儿等人向朝廷献了户籍田册,便是我大宋子民。既然是为了自家人,就算来求出兵,让在座的诸位首酋嘲笑,本官也是在所不惜。朝廷的脸面不在韩冈腰背上,不能保护自家子民才会丢脸。”

韩冈的话让所有人都沉默下去。俞龙珂看着他面前这位站得如山岳一般沉稳的年轻人,心中微生感触。他与宋国的官员打过不少交道,还是第一次听到这个说法。

看了韩冈半天,俞龙珂又说道:“如今董裕已经过了渭源,前锋已在百里之外。官人现在才来求援,怕是已经迟了。”

“能救多少就是多少,韩冈也只求心安罢了。至于董裕,等刘昌祚帅师回镇,自会一报还一报。”韩冈说了两句,眼神突然锐利起来,抬头直盯着俞龙珂的双眼,“敢问俞族长,你以为能卖人情给我皇宋,给王机宜的机会还能有几次?!”

韩冈两句话说得毫不客气,人群中一阵骚动,俞龙珂脸色沉了下去,冷哼了一声,没有作答。

韩冈则步步紧逼,他直上前一步:“俞族长,日日在悬崖上走,总有跌下去的那一天。以青唐部的实力甚至远远不及木征、董毡之辈,身在虎狼群中,总得选一边站。自二十年前,古渭寨建起来的时候,俞族长就该有这个觉悟了。”

“难道就只能卖给你们宋人不成?”一个首酋冷着脸反问道。

韩冈笑了起来,雪白的牙齿在火光中闪闪发亮:“既然要卖,为何不卖给出价最好的。试问木征、董毡之辈,又或是西夏党项,他们能出什么样的价钱,可比得上我皇宋的一根寒毛?何况今次也不是让俞族长你立刻就跟西夏、董毡、木征他们划清界限,只是结个善缘,对付董裕而已,又有什么好犹豫的?族长该不会真的以为董裕能带着五万大军吧?董裕最多也不过万人的乌合之众,猝不及防下,又何能当青唐部的一击之力?”

韩冈把话说到这一步,该说的都说尽了,只等着俞龙珂的回复。

可青唐部的族长沉默了许久,到了最后,还是摇了摇头。

俞龙珂的反应让韩冈生疑,他有什么理由不肯点头?韩冈想着。当把所有的可能性全部排除,剩下的结论无论多么不可思议,都是正确的答案。而现在,韩冈将俞龙珂拒绝的原因一个个都排除,而最后剩下的一条,即是俞龙珂拒绝的原因,也应是董裕胆敢侵犯青渭的答案:

“可是因为令弟?!”

俞龙珂不为所动,只是嘴角难以察觉的抽动了一下,但周围长老们的脸色终于却完全变了。

跳跃的火光中,韩冈没看出俞龙珂的情绪波动,但长老们的神色变化却尽落在他的眼底,“可是因为令弟?!”

虽然依旧是疑问,但语调却是完全的肯定。

