\section{第38章 何与君王分重轻(20)}

“蛮夷所居,四荒八极,皆是不毛之地。枢密意欲以举国之兵强取之?”吕大临冷笑着诘问道。

韩冈的话中之意,是在太明显了,不会有人听不明白。什么华夷之辨,人禽之分,就是给他攻打他国找借口。

“普天之下,莫非王土,率土之滨,莫非王臣。四荒八极亦是王土,蛮戎夷狄同为王臣。使禽兽复为人,天子之任。”

如果在十几年前,用这样的话稍一撩拨,赵顼说不定会立刻热血沸腾,拔剑长啸,一舒胸臆。但现在,靠坐在御榻上的皇帝,连眼皮都没眨一下。

韩冈也不意外,这一位早就不好糊弄了,更休提一个瘫子,还能有什么样的雄心壮志,又继续道,“交州算不算蛮荒之地?去岁仅是税赋就有四十万贯石,从交州运出的粮食更是几近百万石。加上广南两路的出产,保证了江南粮价的稳定。若是在元丰之前,一旦纲粮开运,就是丰年粮价也是会涨上一成两成。”

“国虽大,好战必亡。须知兵凶战危,国之大事,在戎与祀,不可不慎。”要找论据,翻翻史书就不知有多少,吕大临立刻就反驳回去,“汉武攻匈奴,文景所积耗尽。攻月氏,天下户口减半。汉兵一能当戎兵五,以汉之强亦如此,况于今之大宋?”他指着殿门外,质问韩冈,“殿外的禁军,可能以一敌五?!”

“北地草原,可牧不可耕,汉人得之无用。以举国之力,得无用之地,汉武不得辞其咎。然南方国弱民寡,攻取易也。若说西域,如今也开西域。王舜臣新近占了高昌,大食天马运到京中也有数百匹了,户口减了多少吗?”

用数字说话,是韩冈最擅长的。比起那些拿着经史传注中的文字来说话更能说服人心。

何况最近大宋刚刚击败了辽国,无论朝野,心气正高。韩冈的话传出去,肯定能得到大部分人的拥护。宋人不是不好战,过去反战,是因为总是败,连输之后,当然厌战。现如今,连年大捷,又没影响到京中生活,又有几人会厌战?

“言不正则名不顺。南方海外,尽是大宋藩国,连年朝贡,恭顺无比。出师可有名?何况天下苦兵事久矣,自元昊叛,三十余年间无一年不与战。如今幸得四夷皆安,正是休养生息,让百姓安享太平的时候。枢密欲兴无名之兵,岂不贻笑北虏?”

‘卧榻之侧,岂容他人鼾睡。’吕大临等着韩冈说出这一句。太祖皇帝,玉斧划界,将大渡河外都给了大理。更远的南海诸国,哪里算得上是卧榻?要不然就是拿辽国攻高丽做例子,那样的自比蛮夷禽兽的话,更是有说道的。小小的陷阱,等着韩冈自己跳下来。

不过韩冈却没选择那两种说法:“不谋全局者,不足谋一域;不谋万世者,不足谋一时。开元、天宝,亦可谓太平矣,可安史祸源早已潜伏其中。如今虽太平,致乱之源却早已扎根了。”

吕大临没想到韩冈更糊涂,精神一震,大声道:“安史之乱,其祸源正是明皇好大喜功!连年用兵,置大兵于外,不知虚外守中的道理!安禄山若无三分天下兵权,如何敢叛?!”

韩冈没理会吕大临,转身面对赵顼,欠了欠身,“臣昨曰做西席,出了三道题。其中一道有关赌棋的,不知陛下和殿下是否听说了。”

赵顼没反应,向皇后在屏风后道:“吾听说了,也命人算了。枢密出的题好。”

韩冈点点头,他前面添了一个殿下,就是怕赵顼不肯搭腔,“臣曾与人赌象棋,若臣输了,便以百贯偿之。若臣侥幸赢了,只要对方赔些麦子就可以了。第一格放一粒麦子,第二格两粒,第三格四粒,此后都是一格翻一倍,直至第六十四格后。”韩冈扭头看看吕大临,“修撰可知最后要赔的是多少?”

并不是所有人都知道韩冈昨天给赵佣出的题目,就是知道的,也不是所有人都算了一遍。但吕大临是知道的,算过了,他已经明白韩冈想说什么了。抿着嘴,不开口。

“一千八百四十四兆又六千七百万亿余粒,合约三万亿石……足够大宋百姓吃上五六千年。”韩冈笑了笑,“这当然是游戏而已,最后也没赌成。但在数算上,却是很有意义的。任何大于一的数字,就算比二还小,只要互乘,不要几十次,就会变成让人瞠目结舌的大数。”

他停了一下,继续说道,“天下户口,于今是一千六百余万户,丁口三千三百余万。至于老弱妇孺都加进来,人口肯定是超过一亿了。即以宗室论。太祖太宗兄弟三人,开国至今现在不过两甲子,一百二十余年,名登玉牒的宗室已经有多少了?具体数目臣并不知晓,不过宗室所居,先有南宅、北宅,又有西宅,继而是睦亲、广亲二宅,又有睦亲北宅,广亲北宅,除此之外,上清宫、芳林苑也都安置了宗室。”

这些不是单纯的一间宅邸,而都是一个坊,就像后世的社区一样。多少家聚居在一起。韩冈手上有宗室的数目,不过他更命比啊,有时候,还是收敛一点比较好。用房子作证据,已经足够了。

“这还是未计入南京、西京和燕京的宗室的情况。”韩冈稍停了一下,又补充道。

“枢密不知道,吾是知道的。”屏风后的向皇后突然,“每年授官的宗室,熙宁八年是四百多,九年五百多,十年就快七百了。这还是少的。枢密的种痘法之后,现在哪年授官不多个八九百?就只有天家单薄。”

向皇后叹着气,从前几年开始,赵顼就因为子嗣单薄一直在叹气,看到每年添了那么多宗室就烦心,白天黑夜都在念叨。向皇后听得多了,不知不觉也就记下了。

只有做皇帝的总是子嗣单薄,其他宗室比下耗子还厉害。这一点是皇帝的心病。

宗室五岁授官,登名入玉牒中,每年的八九百至少都满五岁。想也知道,零到四岁的宗室幼子,加起来也有四五千了。而且这还只是五服之内的宗室,五服之外,那些宗室不再赐名、授官,现在因为世系尚少,还不算多,但曰后只会越来越多,超过能授官的宗室。

韩冈转身直视王安石:“平章当年修宗子法,袒免亲以下不再授官、赐名,究其因,也是宗室过多,国用难以支撑。”

‘祖宗亲尽,亦须祧迁,况于贤辈?’当年一干宗室,求王安石看在祖宗的份上能高抬贵手,王安石则是毫不客气的发作了一通。

天子七庙,供奉的是四亲(父、祖、曾祖、高祖)庙、二祧(高祖的父和祖父)庙和始祖庙,十几年前曾经有过太祖赵匡胤的高祖父僖祖赵朓是否该移出太庙的争议。王安石可是发了狠,祖宗的高祖都要被迁出宗庙,何况出了五服的宗室?

王安石沉着脸,并不答腔。但韩冈也不在意,“宗室如此,世人何异?人口多了,当然是好事,但有时也会变成坏事。福兮祸之所依。遇上天灾,民乏口粮,更多的人口,就是更大的灾祸。天下人口每年增长一成,七年翻倍。增加百分之五,一百人中多添五口,十五年翻倍。可以想象一下,人口多到极致,天下的田亩养活不了那么多人,那么结果又会如何?”

“大宋现在人口一亿,十五年后,还会有多少?更别说还有多少逃户隐户,都没有计入进来。那些都是少了几分税,就能勉强苟活下来,虽然干犯法令,却也其情可悯。但他们家里面又多了几张嘴后,还有多少能活吗?兼并之家,田宅万亩,而贫者无立锥之地。天下户口,客户据其三分。天下户口,客户据其三成。越是富户,越是囤积粮食。熙宁七年八年,并不是天下的粮食不够吃,就是河北民间,早在粮食吃尽前就有流民在道,何也?富者有三年之积,无惧灾异。而贫户,连隔夜粮都难有有恒产者有恒心,家无产业,也就心无顾忌,稍有动乱,此辈便是最大的祸源!”

兼并的坏处,任谁都明白。驳也难驳。韩冈是堂堂正论,换了谁上来,都没办法驳斥。

如今有保赤局,有厚生司,人口也是增加得更多。卫生防疫制度的确立,让国家人口增加的速度更快了许多。

如果不从韩冈的指引,内守自足,百年之后,当土地再也承受不了人口上的压力,还能再安享太平吗?——那大宋国是要完了啊。

“危言耸听。”蔡卞前面被韩冈堵了许久,终于等到了机会,“天下荒地甚多,尚未开辟的不知凡几。何况上有圣君,下有贤臣,安民有术,何忧致乱?”

“编修所言甚是,天下荒地的确不少。寒家在巩州有田三百顷,都是新开辟出来没几年的。在巩州,人人有田,最少都有一顷,一个都头以上更没有一个在十顷以下。但这些田是哪里来的,是从蕃人手中夺来了!敢问编修,君家在福建,有几顷地,可比得上巩州一名守门官?”

“三代不论,汉四百年,唐三百载,其亡可是人多地少?西汉东汉皆有两百年,唐至安史前亦有百四十年,其致乱,可是人多地少?”

“因为人多地少有个很简单的解决办法……溺婴!生子不欲举,辄溺之水中。江南人烟稠密,尤其是九分山水的福建,由于田地稀少,更是溺婴成风。编修家在福建,此事有也无?也就是近年来,有了交州和两广粮食运抵泉州,压下了粮价。编修宁可坐视国中的人伦惨剧,陷君于不仁,却要保护那些蛮夷。多年向学,却不知学到哪里去了?墨翟之徒也只是一视同仁,没有说重禽兽而轻国人的。”韩冈重重一哼,一甩袖子,“名为儒,实为蠹,君辈也!”

道统之争,争的是意识形态,争的是人心向背,争的是国家发展的纲领,儒家跟佛家不同,从孔子开始,就是一门注重现实的学派。修身、齐家、治国、平天下,跟诸子百家一样,以治国理民为核心。

王安石的新学,主张的是复三代之治,佯为复古,实则变祖宗之法。通过多种途径改变旧有的分配。

而韩冈的主张,则是明华夷之辨,扬夏贬夷,为扩张而寻求理论依据。不服教化,那就是禽兽,人杀禽兽,天经地义。但光是理论是不够的,在现实中,必须要有推动力。这个推动力,就是紧迫姓,从人口着手,逼迫朝廷采取扩张的国策。

当整个国家,循着韩冈的思路,循着气学的理论,来发展来扩张,这道统谁属还要多想吗?

