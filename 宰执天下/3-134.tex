\section{第35章 甘霖润万事(下)}

【对不住各位,同事有喜,出去赴宴了。】

次日,便是同天节,大宋天子赵顼的生日。

昨日一场暴雨下过,尽管今天雨停了,仅是天阴着而已,但大宋君臣就没有了之前数月的焦躁,典礼上的气氛也是千真万确的喜气洋洋。

紫宸殿前,一队宫廷乐班奏着韶乐,京中数以千计的文武官员皆齐聚在此。

王安石作为宰相,领着百官,上殿奉酒。

文资重臣一班,而后枢密使、宣徽使等武职重臣又是一班。

亲王为首的宗室也都到齐,韩冈亲眼见到了天子的二弟,当年与己争夺周南的雍王赵颢,不过离着太远,只看见了衣服,没看见长相。

还与辽国、西夏的使臣打了个照面,也没什么特别之处,只是服饰装束不同而已。

也许这个时代的汉人觉得契丹和党项人的装束怪异到了极点,甚至从骨子里面将之鄙视,但韩冈过去是见惯了奇装异服,了不以为异。

另外还有大理、交趾、三佛齐一干小国的使臣,也在恭贺大宋天子寿诞的行列中。而且韩冈还意外的看见了当年被砍掉了一只胳膊的瞎吴叱,木征的这位弟弟他现在是熙州刺史,又被赐了姓赵,在朝堂上站得位次比韩冈要高得多。

韩冈的位置靠着后面,与一众小臣站在一起,举着金杯,一觥酒恭祝天子千万岁寿。

等到一切结束,已经是午后。从天还没亮,就聚集到宣德门外应卯,到此时,京中的几千名大小官员,在皇城中站了差不多有四个时辰。

韩冈随众出了皇城,站在紫宸殿前几个时辰,变得酸麻起来的腿脚终于活动开了。虽然他没看到周围有伸懒腰的官员,但看着周围人的脸色,也一个个如释重负。

站上几个时辰,就为了向天子敬一杯酒,这等仪式乃是国之重典,不能轻忽视之,但轮到个人头上,对皇帝忠心到甘愿来吃这等无谓之苦的还是不多。

所谓的圣节,对于臣子们来说,也就是例行故事罢了。

想当年南朝宋孝武帝,因为最为宠爱的殷淑妃病殁,带着一众大臣来祭拜,并宣称:“如有哭淑妃哀者,不吝重赏。”

众臣中,有一名为羊志的,哭声最哀,得了许多赏赐。事后有人问羊志:“君哪得如此急泪?”

羊志则道:“我自哭亡妻尔。”

对于来庆贺当今天子生辰的官员们来说,差不多也就这么一回事。

数千人在宣德门前各自散去,回去后,还要派家里的下人去领取今天参加典礼的赏赐。

王安石这边还有着正经事,韩冈也没什么事找他。昨天将该说的都说了,治河的策略是否要改为束水攻沙,不是在小屋子里就能商议定的,王安石那边肯定还要找来朝中的一干水利专家来进行商议和确认。

打发了下人去领赐物,韩冈自己先去了开封府中,与自己的同僚,也就是开封府界同提点刘漾打了个招呼,就准备动身回白马县。

这些天来,陆续抵达白马县的河北流民,差不多已经有十万了,而韩冈此前已经责成与白马同属旧滑州的胙城、韦城两县,划出位置适合的空旷地带,作为兴建流民营的场所。而此前,白马县还有三座新建流民营已经开工建造,现在差不多要完工了。

这三座新营地,能为韩冈缓下一个月左右的时间,这段时间内,以白马流民营为蓝本的流民营地将会在滑州三县一座座建起,以迎接五月开始的河北流民大潮。

从开封府出来,韩冈领着几名家人、随从,往城北而去。一切都跟他来时差不多,就是多了一辆马车。里面都是吴氏托韩冈带给女儿的东西,有药材、有补品、还有衣服,大包小包装了整整一车。

渐渐的抵达开封东北的陈桥门,从这座城门出去,一路直通黄袍加身的陈桥驿,再继续往北,就是旧滑州的地界。

随着接近城门,前面行人车马也渐渐多了起来,韩冈一行慢慢的随着人流向城外去。他抬头看了看天色,同天节大典耽搁的时间太久,今天说不定当真要在陈桥驿睡下了。。

“韩提点!韩提点!”

几声高亢急促的叫喊,忽然远远的从身后传来。

韩冈一扯缰绳,停下马,回头望过去,却是久未谋面的童贯骑着马一路追了过来。

韩冈立刻下马,心知肚明童贯所来为何,天子实在太沉不住气了,不过这样也好,省得自己来回跑。

童贯冲到近前,附近的行人看着他身上的窄袖紫袍就纷纷,滚身下马,先喘了一阵,回过气来后,“奉天子口谕,诏权发遣提点开封府界诸县镇公事韩冈即刻入宫觐见!”

……………………

一班宗妇退了出去,赵顼长舒了一口气。

来贺寿的臣子已经可以回去休息,但他还要接受宗妇的拜贺。对赵顼来说,这等母难之日,也是同样的繁琐和无趣。除了终于下雨之外,他没有任何欢庆的心情。任何节庆一旦与大礼仪式挂上钩,基本上他这个皇帝就成了坐在御座上的木偶,还不如宫外的一个小民自在。

今天赵顼坐在紫宸殿的御榻上,看着下面的臣子舞拜于庭,然后就是一片声的‘同天节,臣等不胜欢抃,谨上千万岁寿。’要不然就是‘伏惟皇帝陛下吉辰,礼备乐和,臣等不胜大庆,谨上千万岁寿。’

而后,自己就再让内臣宣一句,“得公等寿酒,与公等同喜。”

一批批臣子上来贺酒,将同样的对话不断重复着,而赵顼也拿着金杯,重复着举起、放下,根本都不沾口。

现在终于可以歇一歇了。赵顼松了松腰,就听着殿外通名,宰相王安石在外求见。

宣了王安石进殿,赵顼就问道:“不知王卿有何急务需禀?”

王安石没有浪费时间,直截了当的就将韩冈束水攻沙的治河方略向赵顼说了一通。

赵顼听着先愣了一阵,醒过神来,就立刻遣了在殿上听候使唤的小黄门去找韩冈入宫觐见。

当日在延和殿中,赵顼听着韩冈说起近日已有雨兆,当时高兴了好几天,后来又一直不见雨落,便又当成了臣子宽慰自己的言辞。但昨日在福宁宫中见着暴雨如注,方知韩冈所言的确其来有自,并非宽慰之语。在兴奋于天降甘霖化解旱情,以及赞赏韩冈言必有据的同时,也对欺骗自己的郑侠,也讨厌到了极点。

经此一事,对韩冈的为人有了更深一步的了解,赵顼就盼着韩冈能在府界提点的位置上,能再给了他一个惊喜。只是赵顼没想到,这惊喜来得如此之快。

大宋君臣苦于黄河久矣。如今的治河之策,如同墙衅敷土,屋漏补瓦,一年一年的没有个尽头。每到夏秋时节,黄河水涨,京畿之地就紧张起来,一夕三惊的情况时常有之。

而韩冈束水攻沙的方略却别出心裁,一举从根本上解决了黄河河槽逐年上涨的问题。尽管韩冈自言乃是治标之法,但赵顼琢磨了一番,这一套方略却当真是一劳永逸的做法。

如果真能如韩冈所言,那日后到了夏秋洪水暴涨,赵顼也不用再担心得要沿河州县将水势逐日上报。

韩冈很快就到了,从陈桥门往宫中来,路程并不远。

一见韩冈进殿,赵顼就立刻问起了治河之事。

韩冈详详细细的与赵顼说了一遍,最后又道:“此套方略,世人恐难信服,而最简单的办法就是以实验之。”

赵顼立刻道:“此事不须验,这番道理人尽皆知。”

此乃常理,住在黄河边上怎么可能有人不知道。

而且赵顼对于韩冈的信赖度不一样了。琼林宴上的落体实验,雪橇车的大规模运粮,还有最近的观露而知雨,赵顼对韩冈的信任,尤其是有关格物之说上,朝中已无人能比。

但赵顼还有几个问题:“黄河水急,洪水一来,内堤不知能不能保住?”

“所以内堤外堤都要整修。内堤束水攻沙,而外堤则是防洪。”韩冈登时回道,“一开始的时候,河上洪水一至,内堤必定会有垮塌之处。不过当河水开始向下深切,那时候,内堤就逐渐变得安全起来。……不过越到下游,地势越是平缓,束水攻沙的效果也会越来越小,不过从洛阳到大名的这一段,如果施行起来,当能有所成效。”

韩冈虽然说着束水攻沙的不利一面,但他的话已经足以打动天子。

赵顼的确很想让黄河从此不再为害,但整条河流也分了轻重缓急。京畿一带是重中之重,如果能保证京畿——也就是韩冈所说的洛阳到大名的一段——的安危,下游的堤防其实就可以暂且放上一放。

“不知如今是否可以立刻施行。”赵顼很心急,“正好河北流民有数十万之多,可以以工代赈,让其上堤修造。”

“这个时候是不可能了。”韩刚摇摇头:“如今已经是四月,算是进入了夏秋涨水的季节。即便是旱灾,黄河水量也比冬天时要大了许多。当要等着秋汛过后,方可实施。不过现在就可以开始加固外堤,并调查河中水情,以确定黄河各段内堤的宽窄。”

