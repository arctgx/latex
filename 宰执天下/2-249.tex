\section{第36章 万众袭远似火焚(五)}

听着五更的钟声,韩冈睁开了眼睛,坐起身,偌大的床榻只有一人睡着。

素心、周南如今都有孕在身,韩冈不得不孤枕独眠。虽然已经有几个月了,但他还是觉得身边床榻,实在太空旷了。

云娘的年纪在这个时代也算到了时候,不过韩冈还是想正经操办一下。虽说有点对不起没有任何仪式就入了韩家门的素心和周南,但云娘是从小到了家中,真的要收入房里,最好正式一点才说得过去。

韩冈突又失笑,这事还是得再一等等,至少等攻下了河州,凯旋回师再说。

“三哥哥,起床了?”

听到房内的声音,云娘推门进来。披着厚实的丝棉夹袄,端着铜制的水盆。

韩冈就着盆中的热水开始梳洗。而浅褐色的一对眸子则在旁关切的看着他。见着韩冈的眼睛中布满血丝,云娘嗔怪的说着:“三哥哥,怎么又熬夜了?”

“多看了会儿书而已,算不上熬夜,”

“要没有熬夜,早起来操练了。”云娘嘟起嘴,“三哥哥,娘娘都说过,再忙也要睡好,不然会伤了神。”

韩冈呵呵笑了起来,小小年纪,教训人来当真是有模有样。捏了捏小丫头高挺的鼻子,“越来越像管家婆了。”

韩云娘捂着被捏痛的鼻尖,瓮声瓮气的嗔道:“三哥哥!”她努力的睁大着双眼瞪着韩冈,只是用力睁圆眼睛的模样只觉可爱,却半点不吓人。

韩冈微微一笑,小丫头生气的时候才有趣,让他心情轻松了不少。整了整衣服,拍拍她的头,出屋向父母问安去了。

正屋中,严素心和周南都在这里。一边的桌上,堆满了大包小包的杂物。

“这是什么?”韩冈问着。

“是为官人你置办的出行行装。”素心、周南异口同声。

知道韩冈明天就要出发去熙州,她们便为韩冈做好了准备。两女都知道韩冈喜洁,光是换洗的衣服就包了一个大包裹。鞋袜也准备了许多备用的。连马鞍下的垫子,都有两个备份。

“再减减吧,实在太多了。”韩冈哭笑不得,过去也不是没有出门过,从没这么多准备,看来还是因为两女闲得太厉害。怀了孕后,什么事都不让她们做了,闲下来的时间,就是剩下飞针走线的活计。

眼下才四个月左右,素心和周南都还没有显怀,只是看着腰身比过去稍显丰腴了一点。而且因为都是头胎,稳婆和医官没少叮嘱不能滋补得太厉害,否则到时候很有可能会难产。这个时代可没有剖腹产,一旦难产,很多时候就是一尸两命。

重新整理着包裹,周南顺口对韩冈说着:“前面高总管家的明珠姐姐也有了身孕,比我们还早两个月。听她说去拜的南空庵能保胎,赶明儿,奴奴也想跟素心姐姐一起去拜上一拜。”

“信佛祖,不如信医生,木像土偶能管什么用?”

“不要乱说话!”韩阿李信佛,听到儿子如此说,立刻怒道:“佛祖会怪罪的!”

“娘,孩儿只拜先圣。佛祖、道尊可管不到孩儿头上。”韩冈半开玩笑的回了一句。又对周南和严素心道,“如果拜一拜就能安心也是好事,就是出门时要小心,不要磕着碰着。”

在家里吃过早饭,韩冈抵达衙门时,被征发起来的前广锐军将校们都已经到了。

今次韩冈在巩州签书征调的民伕有三千人,而刘源这一队是第一个到的。

韩冈满意的坐下来,说了声:“辛苦了。”

“辛苦二字当不起,即是机宜相招,我等自当从命。”刘源躬下腰,“不论是押送粮草,还是守备城池,我等都会为机宜做得妥当。”

“刘源!”

“……还请机宜吩咐。”

“你们跟着本官就行了,其他事不用多理会。”

虽说刘源这正正好两百五十名保丁,名义上是占了乡兵名额的弓箭手而已,可精锐不让选锋。全是能征惯战的将校组成的队伍,就算京中也难有一见。有他们跟在身边,多余的护卫其实就不需要了。至于押送粮草和守备城池之类的事,那要看情况了,一般来说,韩冈不会把他们放出去的。

刘源听出了韩冈的话语中,准备将他们引为心腹的用意,便聪明的顺势而为,“机宜,那我们下面要去哪里?”

“先是熙州。”韩冈顿了一下,“如果顺利的话,接下来就是香子城。”

……………………

王韶已经抵达了熙州。

而在计划中,景思立和韩冈也将一同前往熙州。

韩冈正犹豫着,到底要不要让秦凤军轻装。将太过沉重的一干物资,用雪橇车给先运到熙州。

自从年后,经略司利用改装后的雪橇车,将大批的粮秣军资运往狄道城。普通的马车改装成雪橇车很是容易,熙河经略司手下又从不缺马车。从上元之后到二月前半的一个月时间中,纯以出动的车次论,已经接近了一千。

只是眼下积雪消融,雪橇车的用武之地已经快要到头了;而平坦的抹邦山、竹牛岭这一条南线大道也会在接下来的两个月中,失去承载交通运输的大半能力。

再过半月,雪水尽化,到时黄土路面容易翻浆。车马驶过,路面上就是一道道车辙、蹄印,里面全是泥浆。甚至有的地方,看上去是一汪很浅的积水,但踩上去,才会发现其实是个深达数尺的深坑,人都能陷下去。但这是平地上才会有的情况。换作是山路,就因为路面下的山石,并不会有太多翻浆的恶劣路况。只是有一利,必有一弊。过鸟鼠山的时候车子都必须轻载,就是独轮车也是一样,要不然就得用马驴来驮送。

韩冈在考虑该如何安排今次秦凤军的行军计划,如果不能安排得好的话,就要挤占运输粮秣的时间。

“玉昆你想得太多了。先把人给派去狄道,该怎么运输粮草,事后再想也可以。”王厚笑道,“你随军转运使还没上任,就把事情考虑的这么多,若是出了什么意外,可就算是为他人做嫁衣裳。”

韩冈满不在意的说着,“只要安抚率领的大军中能做到令行静止,能够攻城拔寨就已经足够了。”

只是转头过来,从京中传来的一份紧急诏令,用着最快的速度递到了韩冈、王厚手中。

需要转发熙河的公文,韩冈作为经略司机宜,有权先行拆看。只是看到诏令上的一个名字,韩冈差点要失声叫起:“沈括?!”

“怎么,玉昆你认识他?”

王厚有些奇怪,这分明是个没什么名气的一个人物,只是新党中人而已。据王厚所知,沈括就算在新党之中,都算不上中坚成员。

“只是章子厚的信里听说过他。”

韩冈并不是在说谎,不过其实是他先询问章惇的。仗着自己的几项发明和独树一帜的学术观点,向章惇提起了沈括。而在章惇的回信中,都很奇怪为什么韩冈会知道沈括擅长算学、水利和工器,但仍详细的将沈括的事向韩冈说明。

看过了章惇的介绍,韩冈现在有理由怀疑,是不是他给章惇的私信,促使了王安石将沈括派来熙河。

沈括这个名字,在此时不过是淹没在大宋朝数以万计文臣的名讳中的、微不足道的一个,论起名声,立功于西北,争风于京中的韩冈都比他响亮得多。但在千载之后,除了王安石、司马光以及苏氏兄弟,如今声名煊赫的宰辅名臣们,没有一个的名气能比得上沈括沈存中。

从发来的诏书上看,沈括现在的本官是太子中允,跟韩冈平级。听闻他有四十岁了,却跟只有他一般年纪的韩冈同一职位,说起来,真的让人为之心酸。可是进士出身,内外数任,在四十上下升任朝官,这才是官场中的正常情况。王韶成为正八品的朝官时,也就在四年前,他三十八岁的时候。

三十五岁入居政事堂,韩琦是个特例,二十一岁就晋升朝官,韩冈也是一个特例。韩冈不知道沈括会不会因此而嫉妒自己,若是要打起擂台来,经略司就要有麻烦了。

“沈括曾有修造海堤河堰,又精于算学,想必在钱粮转运之事上,能有所长才。朝廷派他来做随军转运使,当不会有太大的祸害。”韩冈说不清他是要说服王厚,还是要说服自己。千年前后的距离,对于人的性格谁也说不准。

但事情不会等人的,从熙河经略司的行军速度上看,也许等沈括抵达陇西城的时候,大宋官军就已经将河州城给夺占下来了。那时候,沈括的神色肯定会很精彩。

暗笑了几声,韩冈重又将自己的注意力放在了眼前。他并不会刻意与沈括为难,说起来对他还有几分尊重,但对于自己的工作,他不会向任何人作出无谓的退让。

“明天小弟和景思立携军启程,关于沈括被派来秦凤路的事,全就要靠着处道你了。”

“不过是接待人而已,玉昆你大可放心,还是想想接下来要做什么吧?”

