\section{第24章 兵戈虽收战未宁(七)}

“燕逢辰就是这般说的吗?”

王韶撇向一侧的嘴角传出了讥讽的声音,好像韩冈说了什么可笑的话,而高遵裕也露出了仿佛要嘲弄谁的笑容。

“燕副总管便是如此说的。”

韩冈点了点头。他自秦州回来后,便直奔正厅,向王韶和高遵裕汇报他在秦州州衙中的经过,自然不会忘了把燕达说的话转述出来。

王韶嘿嘿的冷笑了两声,转头对高遵裕道:“郭仲通果然还是不喜我等插足兵事,只想让我们去种田。”

高遵裕则同样回以冷笑:“郭逵若不是贪着开疆拓土之功,何苦违了文枢密的意思在秦州守着。现在看到河湟一个胜仗接着一个胜仗,他哪还能坐得住?”

当日燕达向韩冈传递的,其实是郭逵的心思……也许说警告更合适一点。缘边安抚司最好把精力放在屯田和市易上,不要老想着瞒着监司挑起战事,如若不然,作为秦凤经略安抚使,他郭逵可不会再坐视下去。

这种事,郭逵不可能当面明说,所以他的心意才由燕达透过韩冈传达给王韶和高遵裕。韩冈对此很清楚,故而一字不拉的说给两位顶头上司听,但他看王、高二人的模样,可是完全没有把郭逵的警告放在心上。

“郭仲通就没说其他什么了?”王韶冷笑了一阵,又继续追问起韩冈。

韩冈这次则是摇头,“郭太尉只是问了渭源堡一战详情,还有伤亡情况,并没有再说别的了。”

对韩冈的回答,王韶也不意外。郭逵让燕达转述的是他自己的私心,有燕达提过也就够了,哪里还有自己赤膊上阵的道理。

王韶端起热茶,用碗盖拂去茶汤上的泡沫。古渭荒僻,连王韶手上都没有几饼好茶。现在喝的茶,都是平常卖给吐蕃人的茶砖,只能算是有点茶味道的水而已。但在西北边境吃了几年苦后,王韶对这样的粗茶却已是喝得有滋有味,不像高遵裕,宁可喝清水,也不喝用茶梗、老叶压成的茶砖。

啜了两口,王韶抬头问着韩冈:“玉昆,你对郭仲通和燕逢辰两人说的这些觉得如何?”

“……郭太尉私心太重,但眼下暂且顺了他的意,也于我无损。”

韩冈看得出来,王韶和高遵裕是绝对不会同意让郭逵来摘果子的。就算他们肯分郭逵一杯羹,也只会是冷饭残羹。军功没人会嫌多,开疆拓土也好,擎天保驾也好,一旦在战场上立下足够的功绩,就可以轻而易举地遗泽数代子孙。

想想踏平南唐的主帅姓什么?看看如今的太皇太后又姓什么?

再想想在澶州推着真宗皇帝过黄河的殿帅姓什么?再看看如今的皇太后又姓什么?

曹、高两家,从开国时到现在,已经一百年了,却始终是名门望族中的一员,甚至还能与天家联姻。而那些国初时煊赫的文官豪门,到了如今早就没有踪影。

开拓熙河、拓边河湟的功劳,如果能成功,当是平灭北汉之后第一功。除非有人能讨灭西夏,否则在西北不会有更大的功劳了。王韶正想着靠这份功劳给他和他的子孙后代争一个世袭不移的铁饭碗,怎么可能会甘愿让给他人?

前面李窦向三人明抢,王韶费尽手段,在高遵裕、韩冈的帮助下,将三人一股脑的全都逐走。现在郭逵过来争夺最后的领军之权,王韶当然不会甘心让出去。

但韩冈不看好王韶的指挥能力,文官用兵——连带韩冈他自己——不经过一番历练,很难有所成就。在今次的战场上,无论是王韶还是韩冈犯的错实在太多,若不是禹臧花麻那边也同样出了问题,胜负尤为可知——不,韩冈并不认为今次和禹臧部分出了胜负。两边的损失相当,禹臧花麻又是顺顺利利的撤走了。怎么看都不能算是官军这一边的胜利。

“现在禹臧花麻已经回老家舔伤口去了,木征看起来只要我们不去攻打武胜军,他也不会有什么动作,至少在半年内不会有大战。如今正是把缘边安抚司的根基打好的时候。等费上半年左右的时间,把根基稳定了,也就不用担心郭太尉还有什么手段。”他看看王韶、高遵裕,“现下有郭太尉顶着枢密院,我们这边要轻松许多。若是把郭太尉得罪狠了,情况会就比当初李、窦、向三人皆在秦州时,要严重得多。而且毫无必要”

韩冈话中的意思就是先把郭逵糊弄过去,等着半年后,看看事情会不会有转机。郭逵的地位身份太高,跟他硬拼不是个好主意,能拖一阵就是一阵。

而反过来说,也许这半年中,王、高二人的想法可能会发生转变也说不定。韩冈希望由郭逵领军,这样才能保证有最大几率夺取最后的胜利。

“等到明年年初,也到了安抚回京诣阙的时候。”王韶在秦州已经快有三年,以他现在的职位,回京面圣是分内之事——边臣一任,总得要回京一趟,“如果安抚届时能推动朝廷在古渭设军,给缘边安抚司正式的治兵理民之权,郭太尉那时再想插手河湟战局,难度就要大上许多。”

高遵裕笑道:“要想让古渭升军,从建言、到批复,就是正好如玉昆你方才所言,至少要等半年时间。”

“接下去的半年,就算想开战,也调不来钱粮,只能先歇上一歇——鄜延那里吃得太狠了。”韩冈说道。

“因为韩子华还没有死心。”王韶冷笑着,驻扎在京兆府附近的陕西禁军并不放在他眼里。“虽然梁乙埋抢先一步修起了罗兀城,但延州那里应该不会就此罢休。”

与西夏争夺横山,是已经经由天子批准的国家级战略。如今虽然计划受阻,可王韶并不认为韩绛和种谔会轻而易举地认输,这也是高遵裕和韩冈等人的共识。

又说了一些公事上的话,辞过了王、高二人,韩冈便要回他的公厅。只是他跨出院门,却见王舜臣就等在门外。

见到韩冈,王舜臣便立刻唤道:“三哥!”

韩冈脚步停了下来,问道:“怎么,是来找我喝酒的?”

“有一半是。”王舜臣笑嘻嘻的答道。

“另一半什么?”

王舜臣从怀里掏出一封信,“这是十九哥托人带来的信,跟着十七哥给俺的信一起来的……”

“十九、十七……”韩冈微微一愣,旋即醒悟,笑着把信接过来:“原来是种彝叔的信啊。”

……………………

延州。陕西宣抚司衙门。

种建中抬头望着天空。铅色的云翳遮蔽了天际,灰沉沉的,给了人一股子千斤巨石压着心口的感觉。

虽然身处宣抚司的主院中,可抬头只能看到一方不大的天空,让种建中都感到莫名的压抑。另一面,就在主院的另一侧,商讨军机要事的白虎节堂中,他的五叔正在跟韩绛一起商议着最新的军情。周围来往的军官再经过时,都是轻手轻脚,这种被压迫着的气氛也让种建中觉得很不痛快。

“彝叔……”身后有人叫着种建中的名字,种建中回头定睛一看,却是他的老熟人折可适。

种建中朝白虎节堂紧闭的大门呶呶嘴,“是来等令叔祖的吗?”

折可适点了点头,也问道:“彝叔也是来等令叔的吧?”

“是啊!……里面正在商讨该怎么把无定河上的那根钉子给拔掉。不知什么时候,才能讨论出个结果来。”

“肯定是要打的。但具体到什么时候,动用多少人,都还听说,这些都要打听清楚。”折可适曾被郭逵称为将种,论起军中名声,比种建中可要高出许多。

折家是蕃人出身,在河东路的麟州、府州势力广大。种建中曾经听折可适吹嘘过,折家的谱系可以一直追溯到北魏孝文帝,是帝王之后。折可适便是孝文帝的三十三世还是三十四世孙。

虽然从魏孝文帝到此时,不过六百年不到的时间就传了三十多带,但拉虎皮做大旗的事,大唐李家做过,如今的赵官家也做过,折家所作所为也不出奇——不是每个人都有狄青那样不认狄仁杰为祖的洒脱。

折家世袭府州。从唐末到今日,已经两百多年,论起家门渊源,折家足以傲视大宋国中的任何一个将门世家,唯一让折家人觉得不痛快的,就是他们仍旧被视为蕃官。

作为两名微不足道的随从,种建中、折可适他们还不够资格进入白虎节堂中去讨论军情。现在两人就在韩绛的主院中,更是要谨言慎行才对。

种建中出手转移话题,问道:“听说折九你今次在金汤城立了大功了?”

“远远比不上彝叔你上次提过的韩玉昆。”折可适摇着头,“秦凤的战报你也看了,韩玉昆在其中可是出了不少力。还有传言说,连那个西夏来使,也是他亲手斩杀的。”

种建中惊讶道:“不是说是他手下的一个蕃部族酋所为?。”

折可适则反问着:“自铁壁相公后,你见过这般不给党项人面子的蕃部族长吗?”

