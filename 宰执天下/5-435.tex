\section{第37章 朱台相望京关道(06)}

“辽国此时绝不可能攻打高丽。”

“这必是谎报误传。刚刚结束的那场战争,无论宋辽都是元气大伤,哪里还有可能再动刀兵?”

“辽人虽凶蛮,却也不是傻子。耶律乙辛更是狡诈狠愎。怎么会不顾国力耗竭,而强取高丽?即便要出兵,也该选择在冬日。至少能让战马好生的休养一番才是。”

“此外从辽人入寇高丽,再到派出求援使节渡海而来,就算高丽君臣能当机立断,当日便发信,可从边境传信到开城需要时间,渡海西来同样需要时间,加起来好歹也要十天半个月。可从登州的奏报中,从辽人开始渡江,到使节抵达,这个时间甚至不及五日!”

“这番道理稍作思量,便能想得通透。”

故而就在两天前,王安石、韩绛、蔡确、张璪、曾布,还包括章敦、薛向,一众宰辅都信誓旦旦,于殿庭上严辞驳斥了所谓高丽使节带来的紧急军情。

最关键的,还是因为没有国书。那位‘高丽国使’声称是海上遇上风暴给弄丢了。这简直是笑话了。没有国书为凭,怎么证明身份?那一干诈称国使来骗取回赐,让大宋君臣不胜其烦的回鹘商人,他们好歹也会伪造一份国书出来,才敢叩关东来。整件事疑团重重,就是登州知州,也都没敢把话说死,发来的奏报也仅仅是说其是自称国使。

可是出乎所有人意料,仅仅两天之后,从河北传来了消息,辽军已经度过了鸭绿江,正大举南下。

与此同时,由于一开始就有的疑虑,登州知州暗中遣人对‘高丽国使’的随从进行了盘问,只用了两天便查出其竟是一名高丽行商。

这一事实,把登州知州给吓得魂飞胆丧。来自登州的第二份奏报,满篇都是请罪和自辩的文字。不过也说明了究竟是怎么回事。

这名高丽商人平日往来于东京道和高丽之间,很巧合的发现多达数万的辽军正向鸭绿江畔集结,准备过江南下。所以他一边通知国内,一边就渡海至登州,伪称国使,向大宋求援,而且为了能打动朝廷,还故意说尚在鸭绿江北岸的辽军已经渡江了。

崇政殿上,气氛凝重仿佛冻结成冰。

任谁都以为辽国会消停一阵,孰料转头就去攻打高丽。这件事倒也罢了,但宰辅们的脸丢大了。前一日的报纸上已经刊载了,宰辅们的话也登了上去。心胸再宽广,也禁不住才两天便被事实打了脸,而且是公诸于众。不止一人恙怒于心,只是不便发作。

御史中丞李清臣在列,三司使吕嘉问同样在殿中,翰林院中的学士们,中书门下的舍人们,包括苏颂,皆在崇政殿中。确切的说,是在朝的两制以上官——所有的重臣都被紧急召入崇政殿,共商高丽之事。

但在公布了这两封紧急奏报后,殿上便再无一人吭声。看着被啪啪打肿了脸,说不出话来的两府诸公,没人敢多话。

“不意竟是弦高一流的人物。忠信之国,故有忠信之民。”蔡确呵呵干笑,打破了沉默。只是这话他自己都不信。

两面称臣的藩国,哪有半点节操可言?忠信二字,更是不用提的。保不准那个商人的背后,就是高丽国中哪家有势力的王公,后顾无忧,正好可以搏一个封妻荫子。

“此人事后自当封赏。只是辽人为何去攻打高丽?”

刚刚经过一场得不偿失的战争,耶律乙辛至少要几年的时间来休养生息,整合人心,哪里有闲空去攻打高丽国?难道他们不怕大宋抄他们的后路?

就算现在事实就在眼前,也还是有很多人觉得匪夷所思。之前宰辅们的判断在道理上并没有错,有问题的是有悖常理的辽国。

“当是辽人有恃无恐。”李清臣挺直了腰,板起的棺材脸全然不见被欧阳修、韩维同声称赞的文采风流,“有韩冈为之说,不欲国家生乱,就必须开疆拓土。”

韩冈的话究竟是什么意思?至今也没人能想明白。乍听起来仿佛是恐吓。可是辽军攻高丽的消息确认后,却像是鼓动辽国对周边邻国下手的味道。

终于说出来了。

早就有传言说,韩冈通过萧十三与耶律乙辛达成了默契,让折克行可以肆无忌惮的追杀黑山党项。蔡确知道,这一流言正是从御史台中传出来的。

苏颂的视线飞快的扫过几名同僚,最后定在王安石的脸上。

女婿被御史中丞弹劾,而且是私通敌国,以这个罪名阻其入京,王安石事前可曾想过。

可惜王安石的脸一如既往的黑,外人看不出他内心深处的变化。但向皇后的心情却显而易见的变得极坏,“李清臣你是想说韩枢密通辽?!韩枢密指挥大军,杀得辽人成千上万,这叫通辽,那要杀多少贼人才是不通辽?你给吾找一个来!”

尖利的声音几乎能将屋瓦给震下来,可李清臣夷然不惧,“殿下,私通外国,并非有害于中国。但问题是在于一个‘私’字上。与辽人往来,岂能不予朝廷知晓?且大宋与高丽通好,乃是天子定下的方略,韩冈如今却唆使辽人攻打高丽,有违圣意。”

没人会说韩冈有异心。谁那么说可就是太蠢了。以言辞打动敌国,这是纵横家的本事。张仪通六国吗?苏秦通秦人吗?韩冈杀了辽贼成千上万,谁敢说他通辽卖国?构陷功臣哪能这般肆无忌惮,御史台中也没人会这般蠢。

现在李清臣指责韩冈,也只是说他越权了。就像范仲淹当年私下里给元昊去信劝降,又烧了西夏国书,无论是否初心如何,结果如何,事情从一开始就是错的。

这件事传出去,或许民间会觉得这是韩冈的本事,能祸水东引,堪比苏张。可真要以朝纲追究,这便是无法容忍的过错。

如此一来,韩冈通辽的罪名这一回可就坐实了。唆使也好,暗示也好,让敌国去攻打盟国,怎么都不可能轻轻放过。那不再是韩冈回不回来的问题了,一旦坐实了罪名,枢密副使便做不得了。便是看在他历历战功的份上,也少不了一个出知大名府,或出知河南府。让韩冈在外路的京府坐上几年的冷板凳再说。

好几些人正幸灾乐祸的瞅着王安石。既然这位王平章坚持把自家女婿堵在京外,也就没什么人会为韩冈叫屈。日后韩冈回来要报复只会先跟他岳父过不去,不会有太多精力来顾及他人。

苏颂冷冷的说道:“只恨寇准、富弼无此罪,以至朝廷一年要给付五十万银绢。”

李清臣哼了一声:“此言差矣!与辽交通,恰如与虎谋皮。就像好人家的子弟被诱进赌场,一开始总会赢,可一旦沉迷进去,便再无翻身之日,直至赔光家业。不说唇亡齿寒,当高丽得知辽人侵攻是为皇宋辅臣指使,日后岂会再亲附中国?韩冈此举,实是得不偿失。”

苏颂一皱眉头,正要说话。张璪已站出来缓和气氛,拉着李清臣问道:“不知以中丞之意,韩冈当作如何处置?”

李清臣当即回复:“有范文正故事在,何须劳心多想!”

一直都没开口的章敦神色阴郁,李清臣话里话外都把韩冈当成了罪臣。

但要说韩冈通辽,可从头到尾都没有确实的证据。谁抓到他的把柄了?只是从韩冈敢放纵折克行一事上推断出来了,因为高丽被入侵又好像得到了验证。

但这毕竟不是证据,定不了罪。难道要让辽人作证不成?或者说,从韩冈身边的人下手,让他们出来作证?

韩冈面会张孝杰的时候,章楶全程在场,章敦可以确信,韩冈绝没有向张孝杰提到高丽半个字。而以韩冈的聪明,也不会私下里让人带信留下字据。虽然还有其他办法避过章楶等人的耳目又不留下字据,但如此麻烦的手段只为了让辽国攻高丽,至于吗?

韩冈也完全没必要把自己的名声赌在辽人的信用上。除非构陷,把韩冈身边人抓过来,否则绝无可能入韩冈以罪。但这样做的话,且不说能不能成功,可是要跟韩冈结下死仇了。

还是那句话,至于吗?最终又不能把韩冈怎么样,连损兵折将的韩琦、韩绛都能做到宰相,韩冈不过是祸水东引,是为国着想,并非叛国,就算定罪也不知天下会有多少人为他叫屈。

章敦再也耐不住性子,一步跨到殿中,清朗的声音震动着大殿:“高丽国使虽假,但辽人攻高丽却是千真万确。章敦有一事不明,敢问各位,如今的当务之急,究竟是要问韩冈之罪,还是救援高丽?”

崇政殿上又一次回复寂静,但没有多少时候,曾布站了出来。

“殿中诸臣皆不习兵法,无从议论。章枢密亦只于南方立功。以臣之见,欲明辽事,当问吕、韩,臣请殿下召韩冈、吕惠卿回京。”曾布冷然一笑,“既然要问罪韩冈,也得让其自辩才是……”
