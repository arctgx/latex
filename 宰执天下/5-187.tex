\section{第21章 飞逐驰马人所共(上)}

政事堂中,没人敢于耽搁天子的诏令。

请王安石出山的诏书,已经交由快马南下。皇六子均国公赵佣晋封郡王一事也确定了下来。延安郡王这一封号,将会在冬至南郊之后,成为赵佣的新头衔。而供皇子读书的资善堂,也将在明年年节之后,正式开启。

至于侍讲资善堂的人选,韩冈自然是呼声最高的人选之一,王珪、蔡确这些宰辅,在议论中也同样有着很大的机会一同侍讲——真宗时的宰相王钦若曾侍讲资善堂,有此故事,宰执出掌资善堂顺理成章。

只是因为天子将任命王安石主管殷墟发掘,立场明显的站在了新学的一边,国子监中的学官也都摩拳擦掌,想抓住这个机会,将下一任的皇帝给拉到新学这边来。反正给皇子讲课,跟官职高低无关,而且也没人规定,资善堂中给皇子上课的老师的人数。

对于世间纷纷扰扰的议论,韩冈却是全然没有放在心上,至少表面上看上去如此。就是《本草纲目》的编修工作,他也一点不着急——反正司马光奉诏编修《资治通鉴》,已经拿了十二三年的朝廷拨款,也没有见有个成果出来。尽管司马光肯定是想要将他的那部流传千古的名作完成得尽善尽美,但有他在前,韩冈完全不用将自己逼得太紧。

所以到了休沐的这一天,韩冈便悠悠闲闲的带着一家老小去城外看比赛,不是蹴鞠,而是赛马。赛马场离城有近十里地,一去一回,再看个几场比赛,一天时间都可以打发了。

其实这也是如今许多东京市民,乃至士大夫们打发闲暇时光的新去处。

秋末冬初的时节,秋收秋赋全都结束了,县中也好、百姓也好,闲了下来。尽管秋收秋种,与东京城内城外的百万军民关系不大,但比赛的赛程,依然是要配合农时而来。东京蹴鞠联赛,以及赛马联赛,也就在这个时候重新点燃战火。

赛马场离城近十里,原本还不在大路上,离得还挺远,是一片引水不便的平台地。这样的台地,做农田实在是不适合,又不是依山傍水,也没人拿去建别墅。不过修成赛马场却是再合适不过。

若是在官道边,就是离了东京城有二十里三十里,一样是人烟辐辏。也就这种不在官道上的地皮,才能让主管联赛的东京赛马总社给买下来。

“本来义哥儿是准备起名叫做驰逐联赛、驰逐总社,但华阴侯则反对说还是叫赛马干脆利落,一听就知道做什么的。又不是孔夫子写书,字喻褒贬,越隐晦越好。本来就想要东京城中无论士庶都能来这里,起的名字太晦涩,引来一群没钱的村措大就不好办了。义哥儿后来写信给我抱怨说,以后改蹴鞠叫踢毬好了,这样也是干脆明白。”

韩冈笑着,坐在车中,向王旖说着赛马总社组建时的趣闻。言辞间,倒是不掩对那个干脆爽快的华阴侯的欣赏。

王旖则蹙着她那一对线条优美的秀眉,她还是刚刚从韩冈这边知道赛马总社的背景。且不说对铜臭味太重的对话觉得不舒服,华阴侯的身份更是让她感到不自在:“华阴侯不是太祖一脉吗,怎么拉了他进来。官人,你的身份不一样,可不能跟宗室走得近!”

王旖满脸的忧心,这跟齐云总社不同。

主管京中蹴鞠联赛的齐云总社,虽然也有不少宗室、皇亲、世家、重臣、豪商参与其中。但由于最早的发起人都是商人,之后掺合进来的派系又太多。以至于去年更替新会首时,甚至不得不找拿幅屏风遮着,让一众大小东家到屏风后投黄豆黑豆来选,而后又安置了二十多个副会首来平息众怨——在东京城中,都是当笑话来说的——这样人多嘴杂的反而就不用担心。

可赛马联赛,一上来就是宗室,如今华阴侯还在里面占着会首的位置,这可是遗人把柄。

 “不用那么担心。先看看是为了什么走得近?”韩冈在车厢里冷笑着,“飞鹰走马才是宗室的本分。越是败家的子弟,越是一名好宗室。”

华阴侯赵世将出身太祖一脉,秦康惠王的嫡孙。不用说跟武人打交道,就是交接文人、题诗唱和那都是犯忌讳。但跑马走狗就不同了,便是天子也能优容,甚至巴不得他们那么做。御史台也不会瞎了眼睛,去找这么老实做人的宗室的麻烦。

当然,赵世将作为一名宗室子弟,是不会出来见韩冈这名重臣的,韩冈也不会见他。

“宗室之中,一个个花钱厉害,却没本事去做营生。朝廷每年花出去的钱粮,六成半用在军中,两成半是官吏的俸禄。剩下一成,则是养着几千宗室。但入不敷出的人还是多。在岳父立宗室法之后,许多人连官俸都没了,”韩冈瞥了妻子一眼,“就只剩个宗室的名头。前几天,大宗正寺里面闹腾的事,你没听说?广州蕃坊的一个大食蕃商竟然娶了宗女,秦悼王那一脉的,是贪着两千贯的聘礼嫁了出去。要不是那蕃商暴卒,市舶司出面要析断遗产,这件事还不至于会爆出来。”

韩冈掀开窗帘,让车窗外冰寒的空气冲散车厢中浓浓的檀香烟味,“华阴侯只是站在外面的门面,太祖一系多少人靠他接济。要不是看在这一点份上,天子也不会这么容易就对赛马场点头。赛马场这么一大片地皮中,里面可是有七成是官产。从开封府手中买来时,奏章都是从天子手上走了一遭……不过如今赛马场一个月能上缴给府中一千余贯,已经赶得上京城蹴鞠联赛的五分之一,比桑家瓦子、朱家桥瓦子都多得多,天子无论如何都不会让人撤了这赛马联赛。”

何况日后还有诱使豪门贵胄从西域求购上等良马的好处。汉武帝从大宛夺回来的数千汗血宝马,一千多年下来,血脉早就断得干干净净,肩高四尺半的战马在军中都能算是顶级货色了。现在出现在赛马场上的基本上都是河西马对河北马,或是青唐马对契丹马,很是可怜。

更好的汗血宝马,眼下就只有天子的那一匹由王舜臣献上的浮光,自然不可能下场比赛。但赵顼如今放养在御苑中,听说是爱如珍宝的浮光,已经让不少参与赛马联赛的豪门动起了心思。

韩冈正与王旖说着话,车厢外突然传来了几声敲击声,然后一个谦卑讨好的声音透进来:“学士,夫人,前面转过去就是赛马街,再有半刻钟就要到低头了。”

“还挺快的。”韩冈只觉得才出了城门没多久,想不到大半程路这么快就走完了。

载着家眷的马车一般快不起来,跟驼了人的驴子差不多,几乎是行走的速度。在京城的街道上,经常能看到一辆马车旁边,跟着十几名徒步前进的仆役。

“这是今年夏收后,招人重修了道路的缘故。可是费了不少的神。”

“记得是何矩你的提议?”韩冈说着就掀开车帘,先看到了一张讨好的笑脸,而圆圆的笑脸上的一对眼珠,看着车窗下,不敢乘机偷窥车中。不过车内王旖早早的就将帷帽带上,用垂下来的薄纱遮住了面容。

何矩听到韩冈的话,脸上喜色更甚,他是顺丰行在京城信任的大掌事。他事先得了韩家的通知,早早的便在京城西门口候着,一出城门就迎了上来。胖大的身子,就骑着匹老马在前面领路。

他一心就想在韩冈面前讨个好,眼下听到韩冈的赞许,顿时心花怒放,却还是竭力谦虚的说着:“小人只是提了一句而已,比不得做事的。”

韩冈笑了笑,不置可否。视线越过何矩,在通向赛马场的赛马街两边,有着两排店铺。全都敞着门,里面人满为患。

韩家的车队转进了这条街中之后,前进的速度也慢了下来——路上的车马和行人竟然比之前官道上还多了许多。

韩家一家老小,总共坐了四辆车,外面还有十几骑做护卫。韩冈不想宣扬身份,自己就坐在车厢中,外面的护卫,也没有打起他的招牌。

而且从规模上看,韩家的车队走在路上,也不是很显眼。那些排场大一点的重臣,仆役往往百十数,包括吃朝廷俸禄的元随,老远就举着肃静避道的牌子在前面。就是富贵一点的人家,家里的女眷出去上香,往往也是仆婢男女几十号人一起出动。

在赛马街上,带着家里人出来,全家出动来看比赛,看起来不独韩冈一人。就在前面,还有举牌喝道的。韩冈示意了领路的家人,不要去跟人争道,在后面慢慢的跟着,也不急着上去。

但就在韩冈收起窗帘的时候,向后面一瞥眼,就看着行在侧后方的车窗上,探出了两个好奇的小脑袋,张望着街上的行人和店铺。

身为官宦子弟,出生后常年闷在家里,极少能出来走动,也难怪这般好奇。韩冈也不对一向严厉的王旖说,笑着放下窗帘,在车中端端正正的坐好,等着到地头。

