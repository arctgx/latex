\section{第18章 弃财从义何需名(下)}

【有些事所以迟了点,幸好还是赶在了十二点前。求红票,收藏】

向舅父、表弟问明了一切,心中盘算得定,当天午后,韩冈便亲笔写了诉状,又亲自递交进府衙之中。看着接过诉状的衙役为他身上的官服吓得慌慌张张地跑进府衙内,韩冈笑了笑,转身回去等消息了。

李译已经年过花甲,在凤翔府知府的位置上也做了三年的时间。而从考上进士时算起,到现在以从四品谏议大夫的本官知凤翔府,他沉浮宦海有三十年了。三十年的时间,消磨了他年轻时的雄心壮志,也消磨去了他的精力。

最近李译身体有些不适,不想理事,将府中的事务都推给下面的属官,而推不掉的则交给养在家里的清客们,自己则可落得清闲。虽然他这么在想,但事情总会推到身上。

“谏议。”李译的一名亲信清客叫着李译的官名,走进书房中,“现有试衔知莱州录事参军、管勾秦凤缘边安抚司机宜等事、韩冈一人,携表弟冯从义,舅父李忠,表兄李信,共诉冯从义之兄冯从礼等三人,恳请根究……”

“韩冈?”

李译念着这个陌生而又耳熟的名字,打断了清客的话。虽然近来他身体有恙,无心管事,但韩冈的名字还是听说过的,前日招待王中正,这个名字,在宴席上就听了好几次。

“他一个好好的官人递什么诉状,有事不能上门说?”李译听着心里就有了点火气,也有些疑惑,伸手要过韩冈亲笔写就的诉状,前后用眼一扫,面色便阴沉了下去,“递诉状还把官身写在上面,这算什么,要仗着官职让本府去判冯家有罪?!”

清客见着李译动怒,便忙提议道:“谏议,要不要先晾上两天,韩冈有官在身,待不了多久。”

李译又看了诉状几眼,摇着头:“这个案子没法拖,控告的罪名实在太重了——竟然是弑母!可能韩冈是故意这么写,逼本官明天就开审。”他抬手将诉状丢到一边,咂了一下嘴,神色不渝,“这个灌园小儿,把凤翔当成秦州了。”

“这里是凤翔!不是秦州!”陈通判此时在拍着桌子,怒容满面:“韩玉昆是不是在秦州做得久了,性子怎的如此跋扈。这是明着欺上门啊,大府那里心中能痛快得了?私下里说说,我这边直接就帮他把事情给办了。拿弑母这么大的罪名能吓唬得了谁?反把事情给弄糟了!”

他对着站在面前的慕容武瞪眼道:“韩玉昆这么做是要惹众怒的,现在让本官怎么帮他?”

慕容武心中也在埋怨韩冈,太过年轻气盛,也不先打个招呼就把诉状递了上去,刘节推那里可能要笑得合不拢嘴了。

刘节推现在在冯氏三兄弟面前冷笑着:“尔等何须再忧心,韩冈这是自找苦吃。以为扳倒李师中那三个就能在凤翔府横行了?他这份诉状一递上来,凤翔府里想给他好看的,现在可不只本官一个。”

刘节推得意的用手指敲着桌面,哒哒哒哒的声响,却是按着《好事近》的节拍,“韩冈名气够大,但终不过一个入官才半年的小子,这场面上规矩,当是要好好给他指点一番。”

…………………………

因为韩冈以自己的官员身份,向凤翔府衙递上诉状,为他的四姨喊冤。且在诉状中,又指出冯李氏暴毙之事甚为可疑。故而知府李译不得不亲自来审此案,并拉了府里的通判和节推二人过来,一同参审。

毕竟如果诉状中言皆为实据的话,绝对是凤翔府近年来稳稳排在第一位的重案,让李译不能不慎重。单是杀母一条,冯家三子不管是哪个涉案,最后的结果都少不了被千刀万剐——此乃十恶不赦的重罪。

刑部、御史台、大理寺这三家与刑名有关的三法司同审一案,俗称为三堂会审。而今天一案,是知府、通判和节度推官同审,也可以说是小三堂了。

原告、被告都被带到了堂上。一众衙役手持上红下黑的水火棍,分东西站定。正中央,冯家四兄弟,还有李忠、李信父子都老老实实的站着,两边互相交换着带着恨意的眼神,而韩冈有个官身,得了张杌子大模大样的坐下。

很快,陪审的陈通判和刘节推也都到了。陈通判看了站起来行礼的韩冈一眼,摇了摇头,暗暗叹了口气。在他看来韩冈的做法是在犯了大忌,摆出这副蛮横的模样,穿着官袍坐在堂上,而且亲自写诉状递诉状,这等于是明着以他的身份来干扰断案,看到他这么做的凤翔官员,几乎都起了同仇敌忾的心理。

刘节推则是在冷笑着,也不跟韩冈见礼。走到李信身边:“李信,你打伤了冯家十几人,现在却大模大样的站在堂上。不知为国杀贼,却来殴伤良民,你可知愧!”

韩冈立刻在旁为李信辩解起来,“冯从礼三兄弟殴伤舍舅,致使其卧病不起。舍表兄子报父仇,乃是孝行;事后自首,甘受国法,也是敢作敢当。而冯家三兄弟所作所为,却是与舍表兄差得甚远。还请节推明察。”

“韩抚勾……不,现在应该是韩机宜了。”刘节推说起韩冈的官名时,充满了讽刺,拿人钱财、与人消灾,刘节推在凤翔的口碑还算不错,昨日钱拿到手,现在就不顾形象的跟韩冈顶起牛来,“机宜方才说了这么多,怕还是为了争夺冯家家产吧!”

“节推误会了。”韩冈虽然语气谦和,但话中却绝不退让,“以弟讼兄,有违纲常之道。若舍表弟是为了财帛之物,而要上递诉状,韩冈第一个不会饶他。不过舍表弟是为母正名申冤,此是纯孝之事,在下哪有坐视不理的道理?”

韩冈无意替冯从义争夺家产,这等事费时费力,还不一定能成功。幸好冯从义也会看人脸色,没让他费心去想推脱之词。

表弟如此知情识趣,韩冈很是满意,前面因为二姨家的两个混小子而对姨母家的儿子歧视起来的看法,也改变了少许。恰巧他现在身边缺个能办事、懂货殖的人手,他这表弟自幼锦衣玉食,却在被赶出家门后,还能活得顺顺当当,看起来就是个不错的人选——若是冯从义成了富家翁,驱动他反而难了。

不过为了让冯从义归心,又要安慰吃了亏的舅舅,更重要的是,他回去后还要跟老娘交差,韩冈现在就不得不卖些力气,费点口舌。

他指着冯从礼三兄弟厉声道:“先姨母故后,在下表弟冯从义便被赶出家门,其中最为得利的便是此三人。且这三人为了能掩人耳目,又诡言先姨母并非正妻,买通族中,使先姨母受辱于九泉之下。就算这官司要打上个十年二十年,韩冈和舍表弟也要为先姨母申冤!”

韩冈的话掷地有声,正气凛然,李忠、李信还有冯从义连连点头,冯从礼三兄弟脸色发白,嘴唇动着,像是要反驳。可听到这番话的一众官吏,眼神却顿时就变了。

韩冈只说要为他姨母洗雪冤情,宁可把官司打个二三十年,而不是直说要讨个公道——这番话本身就有问题。他都穿公服上堂了,看上去就是要逼着尽快结案的模样,怎么会又说二三十年的话来?

不过联想到冯从义前面所说的不要家产,众人的眼睛一下都亮了起来。都是官场中打过多少滚的,韩冈话中的隐义,很快就都想了个通透。

再看韩冈时,他们的心境就跟方才截然不同。眼前的这位身穿绿袍的韩机宜哪里是不通人情、只知耍横的秦州蛮子,分明是个大吉大利、仗义疏财的送财童子。

韩冈视线扫过厅中的官吏们一对对灼灼发亮的眼睛,以及还没有反应过来的冯家兄弟,李氏父子,心中冷笑连连。

这就是他的本意,官司不是要赢,只是要人倾家荡产。反正这些家资,自家表弟都不要了,干脆全都送人。

在凤翔官场留个好人缘,让舅舅表哥舒一下心头怨,在老娘面前好也交差。而冯从义那边,他虽然说着不想要家产,但看到三个哥哥能分享万贯家财,心里肯定是堵得慌,而韩冈能把他们都变成同样穷光蛋,冯从义也是乐意——子曰:不患寡,而患不均。

至于这个盘算能不能成功,韩冈根本都不会去担心。

贪官污吏是什么德性,他最清楚不过。鹌鹑嗉里寻豌豆,鹭鸶腿上劈精肉,蚊子腹内刳脂油,这是毫不夸张的说法。一桩案子,不把原告被告吃个干净,他们是不会放人的。所以百姓畏惧诉讼,怕进衙门,原因就在这里。

而韩冈既然把话放在这边了,明摆着要把冯家的家产送上去,接下来该怎么做,在场的官吏们当然不会不知——尤其是衙门中的胥吏,他们要拖延案件的审判,五花八门的手段可是应有尽有。

现在就看冯家有多少钱来买通打点。如果韩冈硬是要求官司得胜,还会有人说他是倚权势欺人,但要将案子拖个十年二十年,断不出个结果来,却是轻而易举,而且经手的官吏必然乐意——其实以谋杀至亲这个罪名,最多三五年,就足以让冯家成为穷光蛋。

到时官司的胜负与否,韩冈无论现在和未来都不会在意……他看着厅中一群眼底都闪起幽幽绿光的豺狼虎豹,还有正从堂后蹒跚而出的知府李译,低下头去咧嘴冷笑。

