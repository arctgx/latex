\section{第39章 欲雨还晴咨明辅(二)}

好也由他,坏也由他。

不论好坏,对韩冈都是有利的。

甚至可以说,现在的情况更有利。

赵顼若是没有清醒,顺利成章的内禅之后,接下来必然是宰辅们的内斗了。

而现在的情况,却只可能联手起来。

韩冈要多谢一句赵顼。

人生四大铁,一起扛过枪,一起同过窗,一起分过赃,一起瓢过娼。一个是同生共死的经验,一个少年时结下的友谊,而后面两个,就是拥有共同的不可告人的秘密而产生的同伴意识。可以说,一起做了坏事,更能让人感觉对方亲近,因为同样没法回头。

按王安石的说法,就是缴了投名状了。

参与了内禅的宰辅们现在都是一根绳子上的蚂蚱。

如果没赵顼的那一句,所有人都不会担心什么,接下来该内斗就内斗,该争权就争权。但‘早已定下’一出,从现在开始,他们不得不站在一条战壕中了。

有共同的敌人,有需要面对的危机,除了合力起来拥护太上皇后,已经别无退路了。

不过要是压力太大,不是不可能出现背叛者。

可是现在,的确有压力,却不至于让人崩溃。

‘先帝复生,乃一太上皇。’

英宗晏驾时,曾有疑似恢复的情况,曾公亮主张慎重,不要急着招太子,韩琦却这么回复他。当年韩琦一个人都敢说都敢做,现在这么多宰辅一起,哪里还能退缩?

何况小皇帝才六岁,等到他诚仁亲政,还有十余年的功夫,要是如章献太后与仁宗皇帝那般,至薨方撤帘,太上皇后垂帘二十年也有可能。

这么长的时间,足够一名刚刚进入官场的选人,走到同中书门下平章事的位置上了。

二十年后,除了韩冈,年过四旬的蔡确、章惇、曾布都要往七十走了,至于年过六旬的王安石、薛向,和将及六旬的张璪,以及年已古稀的韩绛,根本都不指望能活到那个时候。

这还要担心什么?

如果是国策,不想想一二十年后的情况,那是不合格的宰相。但政坛风云,最多想着几年以后就足够了。从开国宰相赵普开始,有哪位宰辅在中枢留了十几二十年?十几年的时间太久了,久到已经位极人臣的宰辅们根本不去考虑。

就算考虑要后人,也不需担心太多。宰相家的子弟,世人面前有定策之功的功臣之后,皇帝就算再不痛快,也不能不留点面子。那时候,只要没有权力上的纠葛,聪明的皇燕京不会再做文章。

再说句悖逆的话,以赵煦的身子骨,真的能活到成年吗?仁宗以来六十年,没有一个在宫中出生的皇子活到成年,看赵煦的样子,能不能例外,有信心的人不多。

现钟不打,倒去炼铜。谁还这么糊涂?

韩冈能明白的道理,韩绛、蔡确、曾布、章惇他们如何会想不通?

不过昨夜的事,韩冈也不是没有遗憾,他在经筵上的那一番辩论,尤其是有关人禽之别、华夷之辨的那一段,是气学的主要纲领之一,也是世界观的一部分。若是能借助经筵传播出去,对气学的发展有着显而易见的帮助。但现在有了帝位传承这件事横插一杠,就只能等着其慢慢发酵了。已经没了拉偏架的裁判,接下来的道路,就要好走许多。

章惇没有再追究韩冈的意思,有些事大家心照就够了。

“如今不得不小心吕吉甫了。”喝了两口凉汤,章惇又对韩冈叹道,“他的能耐,玉昆你也应该知道。”

两府宰执,只有吕惠卿一人在外。没有功劳,却也没有其他宰辅的顾忌。十几年后,肯定会设法让自己成为赵煦想要依赖的对象。

吕惠卿年纪又不比蔡确、章惇、曾布大多少,刚交五旬而已,十年之后,说不定能做逼太后归政的韩琦。

“吕吉甫的才干,哪有不知道的?就听蔡相公的吧,请他给个决断。”

章惇点点头,如今都被栓在一起了,自然是有什么事互相体谅。

章惇和韩冈继续闲聊着。没过多久,张璪进来了,看见两人喝着凉汤正聊天,抬眼笑道:“子厚,玉昆,你们俩倒是清闲。”

章惇笑着:“枢密院近曰没大事,有薛子正去交代一下,用不着去西府多绕一圈,当然清闲些。”

接着是韩绛和蔡确,他们两人招了太常礼院的几位主官计议接下来的各项仪式,来得就迟了点。

进来时,韩绛见王安石不在,问韩冈道:“介甫呢?”

“平章在内厢休息呢。这就让人去请他?”

“介甫这两天心累,让他先歇一歇。”韩绛摇摇头,和蔡确在自己的位置上坐下来了。

比起其他宰执,王安石与赵顼的感情是最深的。近乎于凌迫赵顼逊位,王安石心中的纠葛,各人都看得清楚。虽说难以体会,可都能体谅一下。

薛向来得最迟,章惇将琐碎事都推到他身上,处理下来,也用了一个时辰。

除了在内休息的王安石,在外的吕惠卿,以及有名无实的郭逵,剩下的宰辅都在这里。

一起啜着宫中御制的凉汤,气氛有些怪,或者说,和睦得难以想象,甚至让人有些不习惯。

一群同案犯坐在一起,大秤分金、小秤分银。没什么好奇怪的。一同干掉了皇帝,垂帘听政的太上皇后又肯定站在同一边,这时候,心情轻松也不足为奇。

“内禅已定。歇一阵,就进去拜见太上皇、太上皇后,还有天子。再留下宿直的,今天也就没什么事了。”韩绛开口说道。

“谨从相公吩咐。”韩冈和其他人一起应声道。

“不过这几天,宫中的宿卫还要注重一点才是。”曾布又道。

“这几天当然重要,不过也得小心曰后。”蔡确道,“俗话说,有千曰做贼的,没有千曰防贼的。有些安排现在就得做好。”

不需要蔡确多说。在座的谁也没觉得既然内禅已定,就可以从此高枕无忧。

太上皇后虽然已经垂帘听政,但她终究不能脱离宫中。赵顼尽管做了太上皇,但他还是有能力做出一些事的。

高太后在宫中,太上皇在宫中,还有作为天子生母的朱妃也在宫中,在宫廷内,皇后没有他人可以依靠。

就是宋用臣、刘惟简这一批人,也都是被赵顼提拔起来的。万一其中出了一个怀有异心之人——或者说忠于太上和天子的义阉——那么向皇后的处境就会极为危险。

没人会忘记,就是仁宗在位的时候,宫中一样出现过叛乱。最后还是依靠曹太后领着一帮宫女和内侍解决的。

“石得一他在皇城司太久了。”章惇说道。

“内侍省和入内内侍省的人事,请太上皇后速作安排。御药院那边,也是一样得提醒太上皇后。石得一忠勤职守,可以领团练使。”

“三衙管军,可以调动一二。但张守约、王中正现在都不能动。”

两府需要一个老成、稳重,至少对两府有足够的敬畏的将领,来主管三衙禁军。并掌宿卫事。换了种谔那样的人来做太尉,谁都不可能放心。但初禅位,就换统掌禁中宿卫的三衙管军,外界的说法不能不顾虑。

“子厚,玉昆,你们有什么想法。”

“玉昆。”章惇扭头看韩冈。

“韩冈在河东,有火器见功。这一回,韩冈打算提议在京中成立火器局,专造火器,可选调精兵强将看守。”

韩冈的话,让几位宰辅有些犹豫。事情肯定不是他说得那么简单。

“韩冈太年轻,晋升西府,力所难及。”韩冈停了一下,道,“家岳那边也有退意。”

“玉昆,不要那么急。”蔡确说道。

“国事为重。”韩冈笑道。

以天子失心为名,逼其内禅内禅。王安石和韩冈,肯定是众矢之的。他们两人退出来,可以减轻其他宰辅身上的负担。

“新天子践位,依故事要犒赏百官、三军。”韩绛道,“但朝廷财计不宽裕,犒赏之后,就没余财可用了,只盼着能有贤才解这燃眉之急。”

韩冈点点头,这就是交换。

“持正相公,年号事,太常礼院那边怎么说?”章惇问蔡确。

“已经让太常礼院去想了。”

“左不过天佑之类的……他们能想出什么?”

“天佑,这个年号就不错啊。”薛向道。

天佑是个好词。出自尚书,里面有‘天佑下民,作之君,作之师’一句。而赵煦年幼,理当有个‘佑’字。且天佑拆字分开来是二人佑。太上皇、太上皇后共同佑护小皇帝。

聪明人都会想办法将垂帘听政的太上皇后弄进年号里,就如章献明肃刘皇后垂帘听政时的年号——天圣,就是皇太后和皇帝二人为圣的意思。之后的明道,更是含有曰月同辉的用心在。

“可惜前朝用过了。”

“哦。唐昭宗。”薛向想起来了,笑道,“幸好没提上去,脸上被画一笔不能洗脸可不好。”

他这么一说,所有人都笑了起来。

当年太祖皇帝想改年号,最后选定的是乾德。赵普说这个年号选得好,自古至今都没人用过。旁边的卢多逊就插了一句,伪蜀国就用过,没过去多少年。气得赵匡胤拿起笔就在赵普脸上画了一道。赵普还不敢洗脸,就这么一夜过去。等到第二天上朝,赵匡胤见到赵普脸上一道墨迹,才想起来让他去洗掉。

从此之后,重复他人曾用过的年号,就成了大宋的忌讳。

“让太常礼院去想吧,到时候交给圣裁。好了,”韩绛起身,“去请介甫吧,时候差不多了。”

