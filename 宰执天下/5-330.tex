\section{第32章 金城可在汉图中(九)}

田腴深深的吸了一口气,空气带着丝缕寒意透入体内,稍稍平复心中激荡的情绪。

瞅瞅厅中众官,脸上或多或少都有着一抹难以掩饰的激动。

同样的话,从不同的人嘴里说出来,给人的感觉是完全不同的。

看着韩冈线条明晰硬朗、不怒自威的侧脸,田腴暗自喟叹,也只有如此人物方能放此豪言。

即便如王.克臣一般地位高峻的边帅,说要将入寇的辽军全都留下,也只会惹来嘲笑。可换作是当初在河东,拿着数万人头妆点自己的战功,其中包括数以千计的皮室军首级在内的韩冈,又有几人会不相信?

“拿笔墨来。”韩冈下令让人准备好文房四宝,又招呼陈丰过来,“公满,你来写。让王.克臣坚守太原城二十天。二十天后,援军必至!”

陈丰应声展纸提笔,而几名官员闻言惊喜:“枢副,二十天后援军就能到?!”

“差不多就在二十天上下。”韩冈点点头。他将话说得如此肯定,这让一众官吏更加安心。

“诚伯。”韩冈又叫起田腴,“你来写给汾州,榆次县的文书。”

田腴点头应诺。

看着两人坐下来提笔草书,韩冈放松下来,对其他人道:“河东关山险阻,易阻截,难进退。这一回北虏深入河东乃是自寻死路。”

众官纷纷附和:“有枢副坐镇,就是耶律乙辛亲来,也一样只有丢盔弃甲的份。”

“辽贼深入汉土,都是命悬一线的。先不说有杀胡林旧事在前,澶渊之时,若不是真宗心念苍生,顿兵澶州城下的北虏能回去的不会超过一半。”

直接批评皇帝不适合,可谁都听得出韩冈是在抱怨真宗皇帝太软弱了,这下就没人敢附和了。

韩冈笑了一笑:“这几年来,北虏与我官军对垒连战皆北,国力、军器都远逊于皇宋,且耶律乙辛秉国名不正言不顺,其后方不稳,又选在春来发兵,,这是回光返照,后力难继”

“有枢副的这番话,人心可安啊。”知军笑道,“若传到北虏的耳朵里,说不定吓得他们就向北逃回老家了。”

韩冈拿着拿着杀胡林和辽太宗做例子,很快就会被传出去。当越来越多的官员拿这番话来激励和鼓动军民士气,不用多久会落进辽人的耳朵里。在这个节骨眼上,不是没有可能将辽军给吓走。

“凡事还是要往坏处准备。我倒是觉得会将北虏给吸引过来。北虏连破雁门、石岭,气焰正是嚣张之时。他们要走,一个是抢得心满意足,另一个就是被打得丢盔弃甲后逃窜!”韩冈一扫厅中,“如果再死一个辽太宗,这一战后,辽人当从此不敢再南顾。”

就在韩冈继续鼓动人心的时候,陈丰很快就完成了任务,将一份草稿恭敬的递到了他的面前。

字还不错,有欧体的神韵,而内容简洁明了,没什么文辞华饰,把事情也说明白了,照着念,就算是不通文墨的也能明白。这可比韩冈预计的要好。很有不少官员为了表现自己的文才,硬是在公文中弄个四六骈俪之类的赋文来,却连该说的公事都说不清楚,尤其是以自负文采的进士为多。

韩冈点了点头,陈丰做官做了十几年,看来并不是白做的。将文书稍稍修改了几处,韩冈便盖印画押。装入信封后用火漆封口后,他瞅瞅知军。威胜军知军心领神会,“下官这就去安排,现在就走。”

该吩咐的都吩咐了,韩冈也没什么话还要多说,威胜军知军带着属官起身告辞。韩冈也不留人,将他们送到了厅门口,又让田腴和陈丰将他们送到了驿馆外。

离开了驿馆的的大门,一行人中私下里就窃窃私语起来,多是赞着陈丰的好运气。

“福灵心至吧。平常也不见他如此精明厉害。“

“多半是……军守,接下来该怎么做?”

“韩枢副说什么,我们就做什么。”知军回道。“不过是些粮饷、兵员,早早筹备好对大家都有好处。”

但有人犹有顾虑,“……但京营能赢得了辽人吗?”

“别忘了,还有河外的兵马。枢副若要调兵,折家敢耽搁片刻?再迟些,西军就上来了。”

“如何?”待两名幕僚回来后,韩冈就问道。

“看起来都对枢副有信心。”田腴回道。

“的确!”陈丰立刻接话,“来时个个忧形于色,但走的时候,却都是脸上带笑。全都是有枢副坐镇河东的缘故!”

“主要还是辽人没能在河北占到便宜的缘故。”韩冈摇摇头。河北那里是硬桥硬马的真打,明眼人都能看得出辽军真正的实力。

田腴点头道,“连河北军都赢不了,何论西军?而且辽军入太原,以运气居多,但打仗是不能只靠运气的。只要西军还在,朝野的信心就还在。”

“好了,威胜军可以暂时放一边去,现在的关键还是太原。”韩冈抿了抿嘴,“外无必救之军,内无必守之城。只要知道外面有援军会来,那么太原定能守得住。”

“但二十天是不是太紧了。”田腴方才就想问了,“若是二十天后援军不至呢?”

韩冈哈哈一笑:“辽人可能围攻太原一个月吗?要耶律乙辛当真如此做,这送上门的大礼,我可是却之不恭了。”

田腴皱着眉:“但也没必要就定下二十天。”

韩冈摇头,“不得不如此。”

虽说以太原的城防,即便是被赵光义毁坏后另修、防御力远不如唐时晋阳的新城,可也不是辽人用上十天半个月就能攻破的,但韩冈真是怕了。

这个时代,不仅属猪的多,就是猪的官员为数也不少。应对兵事时,什么昏招都能出。有面对区区百余贼寇,献了牛酒请其高抬贵手的,也有十几兵卒喧哗闹事,就吓得带着一家老小离城逃窜的——这后一位还是名门子弟,乃是当今首相的长兄,最让韩冈心中不舒服的地方,就是这位宰相长兄的姓名居然与他同音。

不给王.克臣等太原文武官以信心,以及足够低的目标,保不准他们就能将太原府丢给辽人。

“只是枢副,光靠开封的援军妥当吗?”田腴追在后面问道。

“还有河外的兵马!”

“麟府军也要调回来了?”田腴知道,在收到石岭关破的消息之前,韩冈并没有打算让麟府军也挤到太原来,而是另有任命,但现在局势已经变了。

“王.克臣此时必然已经遣人从河外调兵了,救命稻草都要抓,何况精锐冠绝河东的麟府军?我这也算是顺势而为了。”

河外的情况一直还好,但胜州方向也有消息说遭到了辽军的进攻,以麟府军为首的河外兵被牵制,短时间内很难调回来。支援河北的兵马一时也回不来,而且韩冈更希望他们能发挥更大用处。

但河东军的主力,大半是在边境上。剩下的,则又大多集中在太原。当王.克臣将太原兵马派往河北之后,河东的兵力便极度空虚。如果把丢在石岭关和赤塘关中的军队算进来,河东禁军还能不能凑出个两万三万来,那都是一个未知数。这样的情况下,韩冈不可能再留着河外军与辽人牵制队伍。

韩冈进了厅,借着方才田腴所用的笔墨纸张,匆匆写了几行大字。

‘存地失人,人地皆失。存人失地,人地皆得。’

陈丰在旁抻着脖子看了,却看不明白韩冈的意思。

韩冈依样签名画押盖印后收入信封,点了一名跟在身边数年的亲信,“借驿马,将这封信送去府州。”

“这是给折家的?似乎说不通。”田腴问道。

“我是在说辽人。”韩冈笑着,视线转到陈丰身上。陈丰却愣着,没有半点闻言即答的敏锐。

韩冈暗暗一叹,就听田腴道,“田腴明白了。歼灭了入寇的辽军,失去的土地都能拿回来,还能多饶几分。若是只想着收复失土,却不愿与辽寇硬碰硬,后患将无穷无尽。”

韩冈点点头:“基本上就是这个道理。”

为了达成目的,麟府军仅仅是调派能动用的机动兵力并不够,韩冈需要的是无所保留的付出。只是将家底都拿出来的折家和麟府军,甚至有可能连麟府丰三州核心之地都受到威胁和劫掠。这就得看折家能不能顾全大局了。

韩冈将折家视为自己在河东军中的助力,不过若是折家不能在大是大非上作出让人满意的决定,过去的交情自是雨打风吹去。

重新落座,韩冈突然猛不丁的问道:“公满,之前可是受了你家门客的指点?”

方才韩冈确认了陈丰的见识,只能说差强人意,之前那仅有一次的敏锐反应和判断,甚至并不是福灵心至的运气,从直觉上韩冈认为那只是转述。现在以陈丰的反应看来并不是错觉。

给陈丰指点的人绝不简单。从陈丰接到急报,到出门赶来驿馆,最多也就能换身衣服。在这么短的时间里,没有任何进一步的详情,便独自判断出辽人的动向,眼光和见识比今天韩冈看到的官员都要强。

而那一位当也不是官员,只可能是陈丰家中的人,门客或是亲属,否则刚才就应该有人会给他使个绊子——向陈丰投去满载着嫉妒的眼神,韩冈方才已经看到了好几个。

陈丰被韩冈盯得脸上血色尽褪,身子都僵住了。藏在心中的秘密,竟然一下子就曝了光。

“公满?”韩冈温和的声音在陈丰耳中,却像柄冰冷的刀子在背上划过。

韩冈眼毒,又是年少得志,在陈丰看来,那是分外容不得底下人欺瞒的性格。若是想蒙混过关,恶了韩冈,这辈子在官场上就没指望了。而说出实话,虽不能指望再受重用,但好歹还有份人情,何况那位也是自家人:

“不敢欺瞒枢副,那是下官侄婿。他游学天下,近日正好到了河东。前几日正与下官议论和河东局势,曾与下官议论过若是不幸让北虏攻破了太原城,可能会有什么局面,又该如何应对。”

韩冈抬了抬眉毛,有些惊讶。

读万卷书不如行万里路,士大夫的见识和人脉有很多都是从游学开始的。不过游学不往京城去却往河东走,这个倒是稀奇。

听口气,还并不是陈丰门客,而是一个年轻士子,更是在石岭关陷落前就开始议论辽军入太原的局势,这就更难得了——真不知该说是乌鸦嘴,还是有先见之明。之前自家却是猜错了。

“可将他一并召来,你那位侄婿,我倒想见见。”韩冈听了听外面的更鼓,都已过了午夜,又笑道,“今晚是用不着睡了,干脆再多见几个人。”

“枢副……这事有些不巧。”陈丰变得吞吞吐吐起来,“下官的那位侄婿前两天就走了,说是去京城。”

韩冈已经准备点人去陈丰家请客,没想到已经走了,只是想来陈丰也不敢说谎,倒是让他微感遗憾:“……还真是不巧。是准备进国子监求学吗?”

“只是游学。国子监不易入,两浙乡贡更是难得一中,所以下官那侄婿是准备先游学数载,再预乡荐。”

“两浙一解的确不易中。”韩冈想起了自己当年参加进士科,是从锁厅试上得到了上京赶考的资格,若非如此,便要从十倍于乡贡名额的同乡士子中突围才行。陕西解试都是十里挑一,更别说独木桥一般的江南了,“磨刀不误砍柴工,多用上几年时间游学四方不算耽搁时间。”

“枢副可要将人追回。”田腴问道。

“算了,就不兴师动众了。”韩冈摇摇头。又不是韩信,他也不准备做萧何,且陈丰也让他失望,没什么心情。何况这一追多半要追到京城,直接写封信回去让人留意就好了,“令侄婿的名讳是……”

“姓宗名泽,表字汝霖。”

