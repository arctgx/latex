\section{第25章 闲来居乡里(八)}

通往崇政殿的走廊上,吕惠卿与判太常礼院的常秩迎面碰上。随口问了几句,便各自拱手别过。

太常礼院的主官地位不高,难得有机会去崇政殿奏事,今天是为了三皇子赵俊的生母宋婕妤的金册而来。

四月初一,宋才人为天子诞下皇三子,赐名为俊。近日,宋才人因此而晋为婕妤。名位高了一级,自然要以金册册封。

天子有后,乃是大宋的喜事,群臣皆上贺表。但吕惠卿还记得四月初群臣朝贺的时候,在上的天子笑的开怀。,而在下面的雍王赵颢则笑得极为勉强。

天子既然一已经有了亲骨肉,做兄弟的不论之前有什么心思,现在都可以收一收了。

不过两个月前,天子笑得开心,但现在,应该就没有什么笑容了。

吕惠卿脚步沉重,已经六月末了,天气依然酷热难耐。走在宫廷中,虽然没有蝉鸣让人心浮气躁,但迎面吹来的穿堂风都是热哄哄的。

天上的一点云翳都不见,热辣辣的阳光毫无遮挡的直晒到地面上,从殿阁顶上的琉璃瓦反射下了的阳光,眩得两眼发花。

前几天王安石领着众宰辅去东郊祈雨可以说是白费了功夫。

今年气候干旱。尤其是京东京西还有河北,都接连上报旱情。

中原一带,今冬就没怎么下雪,幸好春天的几场透雨让地里的庄稼不至于绝收。不过夏收之后,雨水又没了,两个月滴雨未见,莫说陂塘湖泊干得底朝天,就是汴河水都低得只有一尺余。

为了此事,上上下下都已经紧张了起来。唯一可以庆幸的,就是夏粮早就收入仓中,至少不会担心今年中原、河北会有太重的饥荒。

前日天子接连下诏,‘凡河上诸水硙、碾、碓有妨灌溉民田者,以违制论,官司纵容亦如之’,为了灌溉田地,一点水都不能再浪费了,连水力驱动的石磨。碾子和水碓都不给使用。否则就是违制——违逆圣旨,这个罪名可足够重了——而且官员若是纵容不理,亦是同罪。

同时为了让汴河保持通航,汴口两月内开放了八次,涌进来的黄河水不仅让汴河水位恢复到六尺定深,同时涌进来的泥沙,也顺便将河口到东京的这一段河床又抬高了半尺。汴河中行驶的纲船竟比两岸的屋顶高,这屋上行船的情况越发的变得严重。

汴河还是小事,只要加高堤坝,保持通航,就不会有太大问题。最让人的头疼的,就是旱灾之后的灾情。自来旱蝗并发,夏季大旱,下半年多半会有蝗灾。就算不是今年,明年也会有。到时候,饥荒恐怕就难免了,就不知常平仓能不能有所准备。

吕惠卿越发的觉得从里到外都是让人烦躁。

京东京西好办,因为靠着京城,常平仓的情况由中书一手掌握,三五日就是遣人去检查一次。为了能保证京城粮食的稳定供给,没有人敢疏忽大意。但河北东西二路,就很棘手了,旧党盘踞的河北,青苗法本来就推行不利,今夏旱情,河北的告急奏疏又是来得最勤快的,王安石都已经在考虑着是不是要派得力之人去两路进行察访,以防其中有人借此生事。

正思忖着,吕惠卿脚步一停,已经到了崇政殿的殿门前,让阁门官入内禀报了,就在门前等着通传。

赵顼此时正看着河北东路转运判官汪辅之的奏章,听到吕惠卿受招而来。命其入殿后,便拿着这份奏章对他问道:“吕卿,汪辅之的这份奏章,但言文彦博至大名之后,只知邀客饮宴,公事从无一顾,不知你说该如何处置?”

在赵顼身边久了,虽然天子只是拿着奏章来询问,吕惠卿还是听出了他话语中的倾向。明白了赵顼的心意,他就知道该如何回答。拱手回道:“回陛下的话,以臣之愚见。元老重臣,不当以琐事拘之。若以汪辅之奏疏为是,恐有失陛下优待前朝元老之本意。”

吕惠卿的回答,赵顼很是满意。不以政见有别而籍故倾轧,能秉公直论,这才是纯臣。

“正是此理,汪辅之不知朕意,掇拾元老细故,不可留于原任。”亲提朱笔,在奏章上几笔写下判语:“以司空旧德,故烦卧护北门,细务不必劳心。辅之小臣,敢尔无礼,将别有处置。”

转过来,吕惠卿却又帮着汪辅之说起话来,“不过汪辅之也是忠于国事,虽不明陛下之苦心,也不便责之过甚。”

“自是如此,着中书将其择地迁转便是。”

优待元老归优待元老,赵顼知道从道理和法规上,汪辅之做得并没有错。要是严加惩罚,日后谁还敢监督那些老家伙?将汪辅之调离而不是贬官,也能让元老重臣们明白,国事不是由着他们乱来的。天子可以优抚他们,但他们也得自重才是。

将汪辅之的奏章放下来,赵顼问着吕惠卿:“吕卿,祈雨之事可定下了?”

赵顼所问,正是吕惠卿近日来此的目的:“前日辅臣祈雨,至今雨水未至。以故制,当遣辅臣于东郊筑坛,再行祈雨。”

“不须朕亲自来?”

今年春时,雨水不定,田间小麦急需灌溉。所以在三月三,赵顼亲自至后苑华景亭粉坛祈雨。而从第二天的三月初四傍晚开始,便连着两天下了一场透雨。赵顼有了此番成功,也对自己信心大增,今次也想大展一番身手。

可吕惠卿从来不信天人感应一说,不过是董仲舒弄出来骗皇帝的招数。虽然在《尚书》和《春秋》中,也有提及,但正经儒门中人都知道,这不过是一件用来震慑天子不可胡作非为的工具而已。当然,有需要时,也是用来攻击占据高位的政敌的好武器。可有几个会当真相信的?

天子如今要亲自祈雨,一次撞上大运,不代表两次三次还能撞上,还是悠着一点为好:“伏旱虽重,幸而不在农时。若是秋来待耕时节还未有雨,那时陛下再祷于上天不迟。”

吕惠卿如此回覆,赵顼想了想也便做罢。夏天的田里虽然还有些作物,但毕竟不如作为主粮的小麦那般重要。现在去祈雨的确有些不合适,如果到了入秋后还是没有雨,再去不迟。

此事放到一边,先等着下面宰辅们求雨的结果,赵顼顺道问起另一桩事:“经义局的情况如何?”

“已经渐有所成,十月之前,当有回报。”吕惠卿胸有成竹的回道,《诗》《书》《礼》三经新义,其实早在经义局成立前就已经编写了大半,现在只是在修改而已,但话不能照实回覆,“这也是陛下重视此事的结果。如余中等新科及第的进士,被陛下置入经义局后,都不敢怠慢,为此而竭心尽力。”

今科进士中以状元余中为首的前六名,都给赵顼调进了经义局中,想要借用他们的文才,同时也是有着让其学习的用意在。

“他们都已经从乡中回来了?”赵顼惊讶的问着。进士参加过琼林苑之后,基本上都要衣锦还乡。家乡离得越远,回京越迟。而据赵顼所知,余中等六人中,可是有福建人在内。

“余中、邵刚、练亨甫都已经到了。”

“他们倒是勤勉。”赵顼点头赞了两句,任凭哪位天子,都会喜欢看到用心于国事的臣僚。“……那韩冈可曾有消息?”

吕惠卿摇了摇头,“尚无。”

赵顼微感失望,但又问道:“韩冈的差遣,不知中书可有什么想法?”

韩冈的本官品级跟章惇同列,只比自己稍低,这样怎么安排。吕惠卿不想为此头疼,推说道:“韩冈品阶太高,而资望不足,实在难以决定,还是等其入朝后再议不迟。”

………………

韩冈对自己的差事并不关心,也没有赶着回朝的想法。每天还是读书习字为主,有时还学着写些诗词,不过远远比不上家学渊源的王旖,而闲时还带着父母妻儿,到了城外的庄上修养了半个月。比起陇西城中的宾客盈门,还是在自家的庄子上,过得轻松自在。

这一休息,就一直到了八月初。算起来在家中已经差不多待了有三个月。外面的暑热渐渐的消散,阳光也不再如之前的两个月那般炽烈。

冯从义那边有了好消息,经过一番友好而坦诚的交流,蕃物行会终于在七月底成立。行首总共有六家,韩、王、高三家的代言人,占了其中的半壁江山。有了行会,团结起来的力量也便容易在京城打开局面,等八月中下旬棉花开始收获,整个行会都会绕着此事而开始运作。

终于到了离乡的时候,韩冈带着四名伴当先行返京,等到任职的地方定下来,再将王旖她们接过去不迟。

辞别了父母,辞别了双目含泪的妻妾,别过酣睡中的儿女,韩冈翻身上马,一行五人,离开了陇西。

