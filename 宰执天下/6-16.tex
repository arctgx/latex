\section{第三章 岂得圣手扶炎宋(上)}

屋外气朗天清。

抬头望着日出前灿烂的群星,韩冈眨着酸涩的双眼,明明困倦得很,却偏偏没有半点睡意。

昨夜夜不能寐,勉强躺了下去,都没能睡好。翻来覆去的,连带着王旖也是整夜不得安寝。现在韩冈起来了,王旖才重又沉沉的睡过去。

韩冈很清楚是什么原因。

王厚一两天之内,便要就任枢府和皇城。人脉深厚,功绩卓著,而且还得圣眷,不管哪一个位置,他都能轻松上手。

一旦等王厚这根钉子扎下来,某些人实现他们那些小心思的机会就少得可怜了。

而且以现在的情势,时间拖得稍长,局面就会稳定下来。

人都是很容易产生惰性的,也很容易变成习惯。

也就是现在,上皇刚刚驾崩,故而人心动荡。过两个月之后,人心思定,再想闹出些事来,要多付出十倍的代价——那是根本不可能的。

他们的机会,其实也就在这不到一个月的丧期之中。而眼下已经就要走到了尽头。

韩冈仰头看天,郁郁难安。

最后的几天,心神不定也是在所难免。

这不是两年前的冬至夜,事发突然,一切都要在短时间内作出决定,片言决生死。眼下这种漫长的等待,反而是最难熬,也最不合韩冈一贯的脾性。

早知道二大王刚刚‘病愈’的时候,就说动向皇后,将他弄出京城去。只是不想让小皇帝在世人眼中继续失德,才忍了下来。

那时候不忍就好了。

“官人。”

王旖推门而出,轻步走到韩冈的身边。

“怎么不多睡一会儿?”韩冈回头看着妻子,“今天又没事了。”

他刚才起床的时候,王旖还睡得正沉呢,现在却已经起来了。

王旖轻嗔道:“官人要上朝,奴家还能睡吗?”

看着薄怒含嗔的妻子,韩冈稍稍轻松了起来,心口不再那么压抑,一下放松了许多,咧嘴笑道:“贤妻持家辛苦了。”

王旖又瞪了丈夫一眼,却又不安的问起来:“真的没有事?”

“没事,没事。”

韩冈之前已经提醒过向太后,皇城也因为处在丧期,上下管束得极为严格。

另外,韩冈安排在外面的耳目,也是一晚上都没有回报说有异状。

这几日宵禁虽然严厉,可还是有漏洞钻。毕竟此时的开封不是唐时的长安。唐时长安,城有城墙,坊有坊墙,入夜后将里坊大门一关,长安城内就是一座皇城加上一百一十座寨子。

而开封府中里坊数不下长安,可每一座里坊,外面的坊墙都没了。弄得与后世一样,一个个破墙开店,除了皇城左近的一圈里坊,大多数里坊,临着大街都是一排门面房。大街小巷,内外畅通,怎么防也防不住。能守的,也就几条大街的街口。

韩冈安排了人手藏身在离御街不远的院落中——那是顺丰行在京城中的产业。如果有宰辅入宫的迹象,几十人、上百人的大部队打着灯笼直趋入宫,与十几人的巡城甲骑完全不同,无论如何,只要长着眼睛都不会错过。

不与宰辅联络,宫里面再怎么折腾都是笑话。没有宰辅配合,谁会犯傻去跟名声都臭了的二大王结交?还要去联系深宫中的太皇太后。

看看上朝的时间将近,韩冈梳洗更衣,吃了点早饭,便上马出门,前往皇城。

这是大祥祭典的次日。

依然还是在丧期之中,也是丧期内的仪式之一,在京的全体朝官都要参加这一日的朝会。

韩冈出门后,很快便转上御街。

快到上朝的时候了,御街上人头涌涌,一队队的都往北面的皇城赶过去。

看到了章敦一行,不过中间隔了挺远,中间还有几位低品的朝官,在御街上不方便追上去,韩冈也就随着人流逐步前进。

快近皇城的时候,王安石和他的亲随们也从另一条路上过来,不过离得也远了。

一路过来,韩冈看到了十几支侍制以上重臣的队伍,还有一堆皇亲国戚,韩冈认识其中几个,都是在赛马和蹴鞠两大总社中常常抛头露面的。

整整三个月,京中不得赌赛,估计都憋得慌了。赛马总社的会首淮阴侯赵世将脸色就难看得很,小小的县侯周围围着一圈王公,都在长吁短叹。赵顼的丧事,影响的不仅仅是日常娱乐,还有他们的日常生计。一年中四分之一的事件被耽搁,三分之一收入泡了汤。

他们的这副可怜模样,前几天就已经很明显了。唯一看起来没有影响的,就是二大王了,他在两家总社中没有半点产业,赚钱也好,亏本也好。都与他无关。只是赵颢周围空无一人,似乎被孤立了。这本在情理之中,也没人会同情他。但今天韩冈却没看到二大王,只有一人的空白圈子,理应十分显眼才是。

不过韩冈很快就没时间多考虑了,宰辅们正陆续抵达皇城。

除了王安石和章敦,韩冈还见到了郭逵,隔着近十丈,遥遥的打了个招呼——彼此之间官员很多,接近起来都不方便。

后面上来的张璪近前来打了个招呼,对行了礼,聊了两句闲话。

待张璪再去与他人打招呼,周围的文武官,便纷纷上前,向韩冈问安行礼。

就算韩冈一时间受到了挫折,但谁都知道,不会太久,向太后便会给他补偿回去。难道还真的让他只管着现在还不存在的图书馆?

李信和王厚,也在人群内,他们同样要上朝,先后过来与韩冈聊了两句。

首相韩绛姗姗来迟,骑着与昭陵六骏中的名马同名的飒露紫,直抵宣德门前。

他前方的官员,如同海水分开,全都给他让出了道。宰相可以骑马直入皇城,就是在宫门前也不用下马。

不过他在王安石面前,还是从马上下来,行礼打招呼,寒暄起来。待会门开后,两人都会骑马入宫——虽不掌实权,但王安石从官职上论,依然是宰相一阶。

剩下的宰辅,曾布、薛向正在宫中。而蔡确,自矜身份的次相,一向是到得最迟,总是卡在时限上抵达。朝臣们早就习惯了。

一切就跟平日里没有什么两样。

看着门前广场上纷乱却隐然有序的文武官们,韩冈觉得自己的确是担心太多了。人到得应该是蛮全,宫里面怎么还能有事?

礼炮声响,伴着晨钟之音,宣德门的侧门缓缓开启。

宰相们领头,一众大臣鱼贯而入。

石得一守在皇城城门内侧,督促着新近的士兵。

看到宰辅,他的态度一如平日,恭恭敬敬的向包括韩冈在内的宰辅们低头。

近千文武官云集在大庆殿前。

曾布和薛向来得很早了,一东一西的对面站着。两人昨夜宿直宫中,理所当然的要比任何人都早一点。

只是……蔡确在哪边?

韩冈的心情突然间有些焦躁,蔡确虽说是习惯了迟到,但现在也应该到了。

作为两位宰相之一,蔡确与韩绛要率领群臣入殿,少了一个可就是笑话了。

幸而蔡确没让韩冈等朝臣担心太久,很快便从后匆匆而来,站进了班列中。而与他近乎是在同时而来,还有二大王赵颢和中书舍人苏轼。应该都是从宣德门那边过来的。

蔡确、赵颢和苏轼先后入列,赵颢刚刚站定,韩绛与蔡确便率群臣列队徐步走进了大庆殿。

大庆殿中,一如往日一般阴暗。阳光穿不透高大的殿宇,而现在也还是清晨,更没有阳光来照明。

朝臣们按照昨日的排列,在殿中依序站定,开始等待太后与天子的出现。

等待的时间过去很慢。但实际上,也就是半刻钟多一点而已。

净鞭响过,宋用臣尖着嗓子提醒着一众朝臣的仪态,而一阵轻微的环佩响,太后和幼主从后门步入前殿,往御座上走去。

朝臣们照例低头,等待皇太后和天子入座。只有韩冈瞟着上面,遍体生寒,如坠冰窟。

进入屏风后的不是向太后,而是老迈的太皇太后。

而坐上御榻的,身形虽的确是幼童,但比天生就有不足之症的赵煦,那个小儿还真是大了一圈。韩冈还认识他,那是赵颢的长子孝骞!!

他们真的做了!

他们真的成功了!

纵然一直在考虑这个可能,但突然间变成了现实,这还是让韩冈觉得匪夷所思。

怎么做到的?

不!

现在该考虑的是要怎么去面对。

“尔等是谁?!”

“太后何在?!”

“天子何在?!!”

韩冈第一时间怒吼了出来。

敢在皇太后与天子出场的时候,盯着上面辨认的,也只有韩冈一人。

就像当年的吕端,在真宗即位的情况下,看见披头散发的皇帝,叩拜之前,还要去拨开头发认个清楚,担心跪错了人。

但王安石也不遑多让,看清了坐在御榻上的人,也愤怒的从班列中一步踏出去,颤声怒喝,“上面的是谁?!”

朝臣们一时间糊里糊涂,一齐抬头往上看去。本来照常是在韩绛、蔡确的引领下叩拜圣安,但现在韩冈突然大叫,王安石也同样的怒吼,是小皇帝给二大王篡了位?

赵颢瞪大了眼睛,兴奋得盯着韩冈,身子都在颤着!

没看到他跪拜下去,的确是个遗憾,但看见韩冈绝望中的怒吼,却让他有着数倍于之前的快感,浑身酥麻直欲登仙。

正是这个感觉啊!

不枉自己昨夜随寥寥数骑夜入皇城,等的就是这一客!

更不枉自己装疯卖傻也要活下来,盼着的正是这一天!!

韩冈!

你完了!

你完了!!

