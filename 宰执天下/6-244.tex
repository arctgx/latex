\section{第33章 为日觅月议乾坤(中)}

一条条由琥珀缀连而成的长链,组成了一道蜜色的珠帘。

帘幕深垂,将厅室一分为二。

外间别无他人,只有淡淡的乳香弥散在室中,几位高鼻深目的胡姬沉默的守在门前。

而帘幕之后,则是另外一幅景象。

房间中,没有高过三尺的器物。六七人或坐或卧,靠坐在软榻上,地上铺着来自西域的厚重毛毡,占去了房中大半地面。每个人的手中都拿着一只波斯名匠手制的金杯,杯中殷红如血,那是最为上等的葡萄酒。手边一只金盘,盛着椰枣、葡萄干之类的零食。

房内完全是模仿了大食的风格来装饰,也许与真正的大食风格还有些差别,但足够糊弄大宋的子民了。

这是大赛马场中专属于冠军马会的休息室,也只有马会成员才能踏足内间。

对于室中众人来说,门外千万观众的呼声已无法让他们的血液沸腾。

冠军的头衔没人会拒绝,这攸关他们的脸面。但他们来大赛马场,与其说是看比赛,还不如说是大赛马场给了他们一个相互交流的场所。

“天子要纳后了?”

赵世将手一抖,金杯中的葡萄酒泼洒了出来。红色的酒浆顿时染红了地毡。

地毡上的殷红仿佛鲜血,赵世恩看得心里都滴血。

这样一丈宽两丈长的巨型羊毛地毡,只能由船走海路运来,其价堪比等重的黄金。一路上风高浪急,都被小心的呵护着。但这一杯酒之后,清洗不净,就只能值白银的价了。

可赵世将都没在意,房中的其他几人也都连看都没看一眼,一齐在问,“太后打算给天子筹办婚事?”

赵世恩是赵世将的叔伯兄弟,更是现任的舒国公。作为秦康惠王这一脉的嫡长,他有着比赵世将更高的爵位。

但京城人都知道,无事称呼赵世恩舒国公,他肯定要发火。赵世恩想要的是楚国公,秦康惠王德芳的奉祀嫡脉,连个大国国公都没有,当然憋屈。可谁能去跟王安石争?

而在赵世将面前,赵世恩也摆不了谱。

第一个挂下脸来参与到赌马中的宗室,赵世将刚开始时没少受人白眼。赵世恩也能仗着身份,将赵世将冷嘲热讽一番。

只是随着赛马总社的地位越来越高,影响力越来越大,赵世将的身家越来越丰厚,他资助过的太祖后裔越来越多,赵世恩已经连摆谱的资格都没有了。

即使他费劲了周折跻身冠军马会之中,可一位新人如何在首任会长面前妄自尊大?

见几位冠军马会的成员一起发问,赵世恩忙道,“只是听到这么说还不知道是真是假,按年纪也差不多了。”

“太后会答应吗?”

“本就是保慈宫那边传出来的。”

“这怎么可能?!”赵世将难以想象太后会主动提天子大婚的事,本以为会由下面的臣子千请万催,小皇帝才能再近女色,但他转眼便恍然大悟,“哦,我知道了,这肯定是说给东边那位听的。韩相公、章相公自不会点头,既然如此,太后当然会做大方一点。”

“我觉得也是。”赵世恩配合着点头。

“好像不是这么一回事儿。”一个胖大的男子从外间掀帘入内,“老会长,国公爷,你们这回可是都猜错了。”

赵世将看见他就一皱眉,向家的姻亲,姓陈名薮,最擅吃喝,人称老饕,与现任会首关系亲密,可不得赵世将所喜。

“陈老饕,你去马棚可去得够久的。”

“顺便与人多说了几句。”

陈薮大模大样的在赵世将身边坐下,拿了颗椰枣丢进嘴里,一幅等着人来问的表情。

赵世将偏偏不问,“说起来你的那匹摸不着,怎么想起来起这么怪的名字。”

“好名字都给抢光了。要是超光、乌云还留着,我会起这名字?!”陈薮愤愤然的抱怨了两句,语气一转,“不过这名字也不差。我那是黑驹,全身黑,晚上去马棚,不打灯别想摸得着。”

坐在角落里的一人发话道,“陈老饕,你方才说的话是什么意思?”

“还能是什么意思?政事堂那边已经说了,要给天子选一个德言容功皆备的勋贵之后。”

“什么时候的事?”赵世将追问。

“昨天太后召见了韩相公,今天早间两府就坐一起说话,之后韩相公又进宫入禀太后。给官家找个皇后主持中馈,看来是板上钉钉了。”

赵世恩也迷糊起来:“难道太后当真想让天子亲政?”

陈薮摇头:“肯定不会,太后不做章献,两府肯定要闹翻天。”

赵世恩再问:“难道要让皇帝等到二十四岁不成?”

又有一人道:“只怕会更长。有仁宗的时候,真宗皇帝已经四十多了,有当今的时候,熙宗皇帝才多少岁?”

赵世恩道:“这是说官家要到三十多岁之后才能亲政?”

章献皇后比真宗还年长一岁,享寿六十有六,所以才让仁宗只等到了二十四。而当今的向太后虽同样比先帝年长,但当今天子赵煦出生的时候,先帝熙宗离三十还差一点。而且向太后的身体情况一直很好,让天子等到三十岁,当真不是问题。

那人摇头:“那也不一定。以当今的身子骨,可不一定能千秋万岁。”

“少说两句,这话也能乱说的?!”赵世将呵斥了一声。

“我知道,不是看这边没外人吗。”

冠军马会中说出来的犯忌讳的话,不知有多少。谁也不担心会泄露出去。光是以他们聚众结社一事,只要抓上其中一个必然会带起所有人,哪一个都不是简单角色,案子放到御案上,太后都要头疼不已。

当然犯忌也分三六九等,说皇帝活不长,可就是最重的一级。说话的被提醒了,想想心里也发毛,嘴硬了一句,却也不敢再提。

赵世恩道:“不是有说法,太后和几位相公都希望天子早日留下后嗣吗?”

“让天子做太上皇?这是谣传!”

陈薮冷笑:“可不一定是谣传,如今章韩二位可不比霍光稍差。至于太后,难道还不如……”

“真的不要命了!?”赵世将怒道:“说天子倒罢了,太后和相公可是能乱说的?”

室中稍稍沉寂了一下,片刻之后,一个声音才响起,“不知会选哪家的女儿。”

“不是说四德兼备的勋贵之后吗?”

“勋贵也分三六九等,至少不会是向高二家的。”

“也许会是向家的亲戚。这两日可以看看我们的那位新会长是什么反应。”

“如果选了向家的亲戚,还是打算给天子亲政。如若不是,太后上仙之前,天子是没机会了。”

“都少说几句吧。”赵世将沉声,打断了厅中的议论,“这一次水太浑,当真给选上了皇后也难说是件好事。”

“可真要找勋贵,脱不了是两家总社中人。”

“过两日,我会遣人去韩相公那边打听一下。”赵世将道,“会长那里也要问问,早点定下来,免得乱了人心。”

“会长回去了。”站在窗口的一人回过头来,“大概是听到消息了,回去见向宁海了。”

“不论是宁海军节度使,还是保平军节度使,都不是糊涂人,选后之事,向家可不一定会乱搀和。”

……………………

“太后是这么说的?”向宗回手一抖,差点没丢了手上的茶盏,“要为官家选后?!”

向清节点头道:“姑母对儿子浑家说了,官家也到年纪了。还说九叔人面熟,正好多打听一下,哪家的女儿更好一点。”

“什么人面熟,都是一群赌徒。”向宗回冷哼一声,又皱起眉,“这未免也太早了吧。”

“也不算早了。”向清节说道,“官家只比儿子小五岁,转年就要十五了。”

“两位相公那边怎么说。”向宗回问。

向清节摇头:“儿子不知道。”

“你都没去打听?!”

“儿子听了就过来禀报爹爹了。”

“你呀,怎么就不多动动脑筋?”向宗回恨铁不成钢,“还不赶快找人去政事堂问一问!”

向清节不服气,“姑姑既然觉得是时候了,又关两位相公什么事。”

“蠢材!两位相公若不点头,这件事根本就成不了。”

“儿子知道了。”向清节应诺,却没立刻走,“不过爹爹,姑姑既然让儿子回来传话,是不是有打算让家里选一人出去待选?”

向宗回瞪着儿子,“本朝何曾有一家两皇后的?我和你叔父都是节度使,就是家里再出一皇后,还能做使相不成?要是太后当真这么做了,怕不就有人想起东西汉了。一门二后是祸不是福,你那几个妹妹也都没这个命!”

“本朝不也有曹、高旧事吗?”向清节嘟嘟囔囔,“我向家不行,难道四姑母、五姑母家的几个表妹还不行吗?”

“别胡说,这件事还是听你姑母分派!”向宗回忽然抬起头,望着府邸前院,又哼了一声,“多半是你九叔过来了,听到消息可真快!”
