\section{第30章 随阳雁飞各西东(11)}

崇政殿中,宰辅俱在。来自辽使的消息,永远都比地方事务拥有更加优先的地位。 

听了韩冈的汇报,宰辅们神情都放松了一点。之前韩冈和萧禧之间的僵局,并不是他们想看到的。对于现在的两府八位来说,稳定国内国外的局势才是第一要务——即便是最不怕战争的章敦也不例外,新任的知枢密院事需要时间去掌握他手上的权力。 

强硬对待辽人的贪欲,这当然是必须的。可宋辽两国关系恶化,边境冲突扩大为战争,那么更是一个糟糕透顶的结果。 

折干现在递出来的密信,可以让他们把心收回肚子里去了。 

向皇后有着几分不解,问韩冈道:“折干此是何意?” 

“折干请密谈,不过是为了说萧禧不方便说的话罢了。正使有恙,副使代为行事,也在情理之中。” 

韩冈的回答直接忽视了折干交付密信时所用的渠道,殿上也没人有疑问——谁让萧禧称病了? 

在三天前韩冈丢下狠话之后,次日理应上殿觐见的萧禧直接就称病了。 

一旦以正旦使的身份上殿,那萧禧肩负的秘密使命便无法完成。按韩冈的说法,正旦宴后就会直接请他上路了。若还想改口,大宋朝廷这边甚至可以直接对萧禧关上大门——只要持强硬态度的韩冈还负责对辽事务,他就不会有机会翻身,而使团中的每一个人都知道,皇帝重病、太子年幼的大宋,绝不可能撤换韩冈,即便这会导致与辽国为敌。而耶律乙辛启用萧禧的目的是想从大宋这里得到更多,一旦萧禧做不到,必然是要换人。 

但若是萧禧拿出了第二份国书,就是证明了现在边境上的冲突是辽国早有预谋。辽国毕竟也是自称中国、北朝的大国,表面上也要讲究脸面,不能像西夏那般,今天拿了钱,明天又翻脸来攻——就是流氓,有了一定名气后,也会开始讲究身份和格调——而且萧禧若是在韩冈的逼迫下拿出第二份国书,在他个人而言,等于是输了一场,谈判的主动权将会落入宋人手中,能反败为胜的几率着实不大。 

韩冈敢于在都亭驿中翻脸,那是因为他背后有着大宋皇后和两府诸公的支持——皇宫就在两里之外。而萧禧则绝对承担不起辽宋破盟的后果,他不可能确定耶律乙辛到底能支持他到哪一步。现在的僵局如果不能打破,萧禧就必须要为他自己的独断负责。耶律乙辛有可能会支持萧禧的决定,但更有可能连生吃了萧禧的心都有。 

面对两难境地,萧禧选择了先称病,留个应变的时间。这当然是件丢脸的事,不过这也至少算是一个合乎规则的理由。对萧禧本人来说,脸面很重要,为大辽挣得实利更为重要。离新年还有一段时间,这段时间中说不定就会有转机。 

而折干的出现,更是萧禧不愿意仅仅是被动等待,而是打算同时以换人来改变被动局面的手段——在韩冈和萧禧闹僵了之后,必须要有人出来缓颊。 

甚至有可能他或许还有不想让折干站干岸看笑话的想法在。大宋派去辽国的使臣,文臣为正,武臣、内侍为副,全权在正使,但副使往往负有监察正使的任务。辽国的情形也差不多。要倒霉一起倒霉,这样的想法很正常。 

这些可能,老道的宰辅们早就分析过了,心中都有数。萧禧作出的应对,既然在预料之中,自然也让他们安心。 

向皇后虽然想不到那么深,但她能抓住关键,问着韩冈,“不是学士打算怎么谈?” 

韩冈早有定见:“如果折干准备说的是疆界之事,那就不需要回应——宋辽之间自去岁划界之后,便无疆界之争,此事不须谈!若是想说岁币,如果愿意减少,那当然可以谈。但如果有什么痴想妄想,那同样是没有谈判的必要!至于其他要求,估计也不会有了!” 

这番话话掷地有声,强硬得像一块钢板一样。听起来就很解气,只是这根本就不是谈判的路数。 

韩绛嘴角翘了一下,蔡确低头看着袖口,章敦眯了眯眼似笑非笑,而张璪则跟对面的薛向交换了一个眼神,只有闭目养神的王安石没动静,但他也知道韩冈是什么想法。 

一旦韩冈坐下来同萧禧开始为土地和岁币谈判,进入了大辽林牙的节奏,那么撒泼耍赖的招数,萧禧就会一套套的玩下来——六年前,萧禧可是厚着脸皮硬是赖在大宋境内,皇帝都拿他没有办法。 

对韩冈本人来说,不论结果如何,不论他在谈判时有何等主张,只要他参与到谈判中,丧权辱国的罪名都会有人肆无忌惮的往他身上栽。谣言这种东西,本来就不需要任何证据。贼咬一口入骨三分,狗屎沾上身,洗得再干净都会有臭味。 

所以最好的办法就是直接不理会辽人的讹诈,更不承认有什么疆界纷争。谈都不去谈,自然就没有谣言存在的余地。对韩冈,乃至对这一班满是新党的政府成员,都是必然的选择。 

“这是不是太过强硬了?纵然不能让辽人逞其所欲,但话还是可以好好说的。”向皇后有些担心。 

王安石帮韩冈出言解释:“耶律乙辛遣使来,就是想逞其所欲,用以安抚国中。话说得和气也好,强硬也好,辽人都只会看结果。既然不能逞其所欲,那就只会是一个结果。” 

“东西还是要给的,否则这件事就没完没了了。”韩冈更正道,“不过不是给辽人,而是耶律乙辛。他想要的东西,大宋可以给他!” 

大宋视辽国为大敌,若有可能,绝不会放弃削弱辽国的机会。但这并不意味着一定要敌视耶律乙辛。在一些特殊的情况下,耶律乙辛也是可以和大宋有着共同的利益。但想要把握这一点,就要看怎么去运作了。 

王安石很明显的皱起眉头,不是因为韩冈的否定,而是韩冈的想法让他觉得心中不快——有些话不用说明,也能明白。 

“耶律乙辛想要什么?” 

“耶律乙辛,奸雄也。”蔡确出班说道,“窃国权奸,他最想要的东西自然只有一个。”他回头看了看韩冈,“或许韩冈便是此意。” 

向皇后恍然大悟:“是要大宋支持耶律乙辛篡位?!” 

“万万不可!”韩冈抢在所有人插话之前当先一口否认。这个污名他可不能担。 

这是大是大非的原则性问题。可以承认现实,但绝不能明着说要支持耶律乙辛篡位。不论哪个臣子随意说出这样的话,肯定会让人怀疑起他有没有最基本的忠义之心。韩冈绝不会糊涂到开这个口,就算这么做对大宋再有利,韩冈也不会站在赞成者的位置上,更不会主动提出来。 

蔡确这话说的,还在记恨自己支持设立陕西宣抚司?那么章敦应该排在更前面吧——因为心急的缘故,可是章敦先动议设立宣抚司的。 

不过蔡确究竟是什么心思,韩冈现在没空多想,他厉声道:“相公此言谬矣!使臣叛君,如何教训臣下以忠,如何教训万民以孝!相公宰衡天下,如何能说出此等缪言?!” 

向皇后面色赧然,方才的话其实是她捅破的,蔡确并没有明说。她有些庆幸眼前至少还隔了一层屏风,轻轻咳嗽了一下。 

向皇后问道:“那学士究竟是什么想法?” 

“耶律乙辛乃是弑君权奸,眼下是以强权来控制国中。为了安抚人心,他需要银绢来赏赐臣下,或是用胜利来加强自己的声望。这就是他遣萧禧为使的缘故,也是他出兵占据兴灵、黑山河间地的原因。但眼下大宋国势昌盛,若是贸然开战,辽人绝难获胜,一旦失败,耶律乙辛便有覆灭之危。所以他更想要的是必然是财帛之物。以臣之见,不如投其所好。” 

也就是婊子不是不能做,而且为大宋的利益必须去做。但公娼是做不得的,会坏了名声,只能去做半掩门。 

蔡确笑了一下,却不说话。可章敦忍不住开口了:“这还不如用银绢去支持辽国国中的忠臣。如果他们能起事,耶律乙辛自是无暇南顾!” 

“可如今耶律乙辛势大,支持契丹朝中正臣拨乱反正,与其相争,的确对大宋最为有利。不过辽国国中,与耶律乙辛为敌之辈,究竟是何人,根本就弄不清。万一找错了人,那就是授人以柄,会让耶律乙辛更为猖狂。” 

“但从国库中调拨银绢,不论是给辽人,还是给耶律乙辛,结果不都是一样?甚至更糟——添支岁币有仁宗朝的前例故事,贿赂辽国权臣可是从来没有先例的。” 

章敦言辞犀利,不过他倒不是打算反驳韩冈,而是在做配合。他知道韩冈肯定已经有了答案,只需要铺好路,将韩冈的计划引出来。 

对章敦的默契,韩冈送去一个感谢的眼神,轻轻点头,然后胸有成竹的笑道。“但如果是不需要动用朝廷一分一文,甚至不需要诏令呢?……朝廷什么都不要做,只要能够默认就够了!” 

