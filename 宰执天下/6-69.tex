\section{第八章 朔吹号寒欲争锋(九)}

王安石没有为韩冈言辞所打动,

“找不到书桌?北虏入寇可是在十余天后?”

王安石的心情看起来很糟,韩冈猜测自己昨晚是不是把他给气到了,不过更有可能的是王安石想藉此警告新党中人,不要奢想能够平平安安的换边站。

连女婿都能拉破脸来训斥,王安石的态度很快就会在朝堂中传开。那些想换船的新党中人,在作出决定之前都要想一想,会不会成为杀鸡儆猴的对象。

既然岳父大人有着这样的想法,韩冈当然要配合。

“知贡举的人选,也不是急在半个时辰之内。”

韩冈说话的时候,并不似王安石那般冷硬。但两人之间的紧张气氛,仿佛暴风雨前的天空,黑沉沉的压着人心。

夫妻间要吵架,筷子位置摆得不正都能成为导火索。新党、韩党当真要撕破脸,议事先后顺序当然也可以作为理由。

“陛下!”章敦抢出班列,“日本之事,虽非紧要,可事关北虏,也不能轻忽视之,韩冈之议当可尽快施行。”

章敦出面打了圆场,向太后立刻松了口气,“当然可以……”

一听太后同意,章敦又接口道,“修造海舶,事在军器监。加强水师,增兵耽罗岛,事在枢密院。至于京师城墙增筑,砖石交由各地转运司,而如何增筑城墙,并修筑炮台,陛下可选一内侍提举,与开封府、火器局共议。”

王安石、韩冈翁婿对骂的场面虽然有趣,但当真闹大了,就会将太后给扯出来。

这个惯会拉偏架的裁判,章敦如何敢让她出场?

原本会让朝堂争议半日的开封府整修城墙一事——包括从各地州县征收砖石的提议——在没有任何反对声中,轻易的得到了通过。

不过整件事也不算太出格。韩冈对日本局面所提出来的三条意见,其实前两条早就得到了通过,现在不过是重复强调而已。至于第三条,本质上还是整修开封城墙。这种事,在政治上,是不会有错的。至于来自各地的砖石,朝廷只要给足钱,百姓自然会乐意。

韩冈躬身向太后行礼,心中却在想:日后可以让下面的人提议,自己再在朝堂上助阵。没有到了参知政事,还要自己再冲锋陷阵的道理。贵为执政,下面总该有几个马前卒才对。

日本之事暂时告一段落,在更新的情报传来之前,朝廷对此作出的决议就是一如既往,顺便再将开封城墙给修一修。

乍听起来,两府里面的成员都是糊涂蛋,而作为提议者的韩冈更是糊涂得可以。不过从太后到诸宰辅,没人对这个决定还有心思多考虑,下一个议题

,是迫在眉睫的元佑元年礼部试考官人选问题。王安石方才的愤怒,也可以说是为了接下来的争议来热场。

因为一场宫变,使得原定的锁院之期被延误。加之许多官员被牵扯进蔡确大逆案中——不论他们是否当真是逆党,只要有嫌疑,朝廷就不可能安排他们为国取士——使得之前由蔡确主持定下的考官人选全都作了废。

之后又因为韩冈提议以侍制以上官推举宰辅,所以考官名单一直悬而未决——朝廷要选择考官,至少得以一个两制官为知贡举。这在当时一心想要垄断入选名单的新党中,肯定不会答应宝贵的票数被分薄。就算为此拖延上一点时间,也有先帝大行、宰相谋叛之类的理由,没必要担心会为此惹来士林的非议。

但时至今日,知贡举的人选已经不能再拖下去了。

“诸位卿家,知贡举的人选,不知可有何提议?”向太后环顾殿中,问着下方的臣子们。

知贡举的人选其实很好定,一切循例就可以了。

又不是初次创新,这是自唐时开始,就延续了数百年的考试,有的是先例可以供后人参考。

依照近年来的惯例,基本上都是由现任的翰林学士权知贡举,然后在三馆或知制诰、御史中,选两三人出来权同知贡举。再从国子监的教授、博士,以及前一科排在一甲二甲的进士中,选出一干人,作为初考官、覆考官,还有参详官、封弥官、编排官等等。

而三衙也会挑选一名将领,率宫中禁卫护卫贡院,同时开封府也会派出府中兵将,共同封锁贡院内外。从考官进入贡院开始锁院,直到考试完毕,位于开宝寺附近的贡院,都是天底下禁卫最为森严的位置之一,不会比皇城稍差。

不过一二十人的考官中,最为重要的还是作为考官之首的权知贡举。

除了身份地位需要是玉堂华选——至少是堪比两制——此人还必须是文学出众。

如韩冈一般的官员,就算人望很高,但文学水平上若是过于拙劣,便不可能被任命为考官,所以能被任命为知贡举,便是一份极为难得的荣耀。

不过如果没有太后亲口所说的半月之后再行推举,根本就不会有现在的纠结。

由于考官人选定下之后,就需要立刻进入贡院锁院,若是在朝中的翰林学士、或是地位相当的重臣里面挑选一位出来担任知贡举,那么十余日后的廷推,至少会少上一票,说不定还会再少上一位候选人。

朝堂中资格的担任枢密副使的就那么几位,试问李定他愿不愿意放弃投票的机会,去做一任权知贡举?

在经历了前一次的推举后,恐怕所有人都明白了,以参加投票的人数,任何一票的分量都是重中之重。少了一票,很有可能就会导致之前所有的计划化作了一场空。

尽管能够跳出来的背叛者,之前应该都跳出来了,接下来投票的重臣,在前一次推举时,都已经表示,但在李承之这样的铁杆新党都转投韩冈的时候,谁能保证没有下一位李承之、王居卿?

少了一票不仅仅是一票,是人心。万一再多一人转投气学,使得新党的候选人不能占据前三,那么下一位枢密副使很可能就是从韩冈的支持者中推举而出。

韩冈成为参知政事后,他这一边的确也已经少了一票。可韩冈就任参知政事所带来的影响,却远比一票要重得多。重到会让王安石担心新党之中,会出现更多的王居卿、李承之。

“臣举蒲宗孟。”章敦立刻说道,“蒲宗孟久在禁林,正堪为知贡举。”

韩冈也猜新党会选择蒲宗孟。

翰林学士之中,排位第一的翰林学士承旨曾孝宽算是最适合的——如果他有进士资格的话。可惜的是,他与吕嘉问一样,都是荫补出身,并非进士,当然做不得知贡举。以曾孝宽的资格,做到参知政事没有任何问题,但缺少进士出身的他,却升不到宰相。

除了曾孝宽和吕嘉问之外,当日与韩冈相争的三人中,李定是进士。御史中丞虽不如翰林学士名正言顺,却也勉强够资格了。只是他肯定是要参加枢密副使的选举,在参知政事的位置给韩冈占去之后,剩下枢密副使这个位置,成了他唯一的选择。

而蒲宗孟对新党来说,并不可靠。性好奢靡的他,常常为人所诟病,若能选择他知贡举,倒是免了他投向自己。

但蒲宗孟是李定的支持者,蒲宗孟在贡院中消息不通,少了他这一票,对李定不啻一个巨大的打击。从前一次推举可以看出,每名候选人多不过六七票,少了一票,就是第三名和第四名的区别。

“陛下。”就听张璪出班说道,“半月之后有廷推一事,蒲宗孟若知贡举,将不得与会,此事不可不虑。”

“朝中有可堪知贡举,又不得参加廷推的大臣吗?”向太后立刻就问道。

“自是没有。”张璪道,“故而以臣之间,知贡举者,可先行决定推举何人,将章疏进于宫中。”

“那外任的侍制以上官,是不是也要去信,让他们先行决定,上表推举,存于宫中?”苏颂立刻出班质问,“同为侍从官,岂能厚此薄彼?!”

那是韩冈的推举本身有问题!

张璪在肚子里大叫,但他不敢说出口。

在京的重臣公推宰辅,这样的推荐制度并不合理。就是韩冈自己提出来后,都知道迟早会被修改。

但反对是不可能的。当韩冈提出他的建议之后,任何一名——包括王安石和宰辅们——想要阻止这项提议通过,完全可以想象得到,侍制重臣们都站在韩冈的一边,不论党派亲疏,都不会答应有人阻止他们获得更多的权力。

即便现在只能眼巴巴的看着同列在京城中享受着决定宰辅归属的权力,但他们宁可自己现在没有投票权,也不会同意宰辅们代替他们将推举给废除。现在没有,可回京后就有了。若是给废除了,日后找谁哭去?

当侍制以上的重臣有了推举之权,便是有了制衡宰辅的权力。日后要给谁加上侍制的贴职,意义将不会像过去那么简单,而是事关宰辅归属的重要角色。有着这样权力的角色,至少能让宰辅在见面时,多一点笑容,而不是居高临下的一瞥。

张璪无法辩驳,向太后自是支持苏颂:“正如苏卿所言,既然不能登殿当面推举,那么就不能算上他的一票。否则外任的那些侍从、学士就不好说了。”她又问着章敦,“章卿,还觉得蒲宗孟合适?”
