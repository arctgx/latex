\section{第13章 晨奎错落天日近(十)}

韩绛将手上的奏章一丢,面沉如水:“吕惠卿玩得好戏法!”

“唯恐天下不乱啊!”张璪也是愤然恨声。

谁都知道吕惠卿已经将赌注压在宋辽开战上,以他的为人,不可能坐等辽人因岁币之事来攻,而会想方设法尽快加速战争的开始,否则时间拖得太久,朝中也会生变。

可推测与事实之间,终究给人的感觉是不一样的。

之前不论吕惠卿怎么倡议开战,挟士论以制朝廷,韩绛和张璪总有一丝犹豫,不肯参与到新气两派的争锋中。现在看见吕惠卿终于做出事来,心里不免开始后悔,没有早一步将吕惠卿压制住。若是能够早一步做出反应,至少能够将他的爪牙调离边境。

也没人怀疑事情是否是雄州误报。腊月廿九事发,刘舜卿用了三天的时间去查证,估计也派人去辽国境内打探了消息。定然是确凿无疑才会上报。

现在三名皮室军死在了官军手上,辽人那里又会是什么反应?

“这件事看看吕吉甫怎么说吧!”韩冈说道。

“玉昆?”

韩冈的口气太过平和,张璪分不清韩冈是含怒挟愤,还是单纯的要听吕惠卿的解释。

“诛杀辽军三人,若是当真是刘绍能大胆妄为,其身后多半是有吕吉甫在指使。”韩冈的发言,比起韩绛和张璪都要保守一点,“现在还是得瞧一瞧吕吉甫接下来有什么打算。”

韩冈也很想看看吕惠卿会怎么做,是破罐子破摔,还是设法补救。

吕惠卿刚刚给自己写了信。而刘舜卿拖了三天的时间才发出急报,有可能让吕惠卿在发信之后才收到消息。

不过刘绍能下手之后,必定会立刻遣人通知吕惠卿,这么想的话,在发信前收到消息的可能性也不小。

韩冈猜不透究竟是哪一种情况,而事情业已发生,也不用急着去处理吕惠卿了,他觉得还是等等看再说。

在他看来,仅仅是三条人命,已经做了皇帝的耶律乙辛还不至于压不住下面的异动,拖上一两个月到了春天就不适合再出兵了,没有了岁币这一因素,宋辽两国的战争,不会那么快就打响。也就让韩冈有足够的时间,去探一探原委,了解一下吕惠卿到底是在怎么想。

“这样也好。”张璪点头,朝廷现在不表态,之后就有了挽回的余地,免得先行定性,之后结论相反,就不方便改变朝廷立场了。

从这方面来说,韩冈已经是个称职的官僚。

张璪没料到韩冈会如同变了个性格,倒是白白担心了一场。

韩绛将来自雄州的急报看了一遍,也改换了口气,心平气和的说道,“文字写得不错。”

“是刘舜卿亲笔。刘希元日常多读书,晓吏事,谨文法,不是普通的武夫。”

韩冈推荐的人基本上都是有实务之材,以实干作为衡量的工具,他对人物的评价自不会太离谱,一直以来都有所印证。

韩绛点头,赞道:“也难怪玉昆你信重于他。”

“不过是人尽其才……雄州那边怎么办?”

吕惠卿归吕惠卿,雄州归雄州。吕惠卿的反应,韩冈想要看看,但雄州的问题却是更需要优先解决。

“还是要先把刘绍能召回来。”韩绛道,“玉昆,你看呢?”

韩冈点头:“理所当然。”

“怎么处置他?妄启边衅?”张璪问道。

韩绛道:“先招回京师询问详情,然后在京师里面给他安排一个好一点的位置养起来。”

从地方回京,就是平调也能算是升迁,也免得世人误会朝廷怕事,故意打压功臣。即便还有议论,至少也有分说的余地。而调回刘绍能,也算是给辽人一个交代。

宋辽两国正常时期,若边境上有些龃龉,多半都会采用这样的手段来化解矛盾。也就是到了熙宗皇帝针对性的开始变法,而耶律乙辛掌握辽国大政之后,才变了一个样子。

“就按相公说的办。”韩冈说道。

政事堂中宰辅分工,韩冈在军事上分担的责任更重一点。即便是如对辽事务,只要事关军事,大部分还是交由韩冈先做决定,然后韩绛、张璪再发表意见。不过人事安排,只要韩绛发话,韩冈基本上都是会尊重的。

“刘舜卿怎么办?”张璪问道。

刘舜卿是朝中公认的名将。否则也轮不到他去守雄州。不过他也可算是韩冈的人,能够名满朝野,就是因为在韩冈麾下所立下的功劳。

朝廷调走吕惠卿的人,却让刘舜卿继续留任,韩冈身上免不了会有些闲言碎语。

“让刘舜卿继续镇守雄州。”韩冈的态度十分坚定,“皮室军不同于南京道的兵马,即使是一介小卒,说不定都能牵扯到朝中的高官显宦。现在辽国那边必然想要报复,雄州若贸然换将,等于是给了他们一个机会。”

“辽人终究还是会来。”张璪道。直到此时,他依然不清楚岁币之事已经无法干扰辽国决策。

“所以要看吕吉甫怎么做了。”韩冈冷然。

在已知岁币无法引动耶律乙辛之后,吕惠卿如果想要引辽人南下,肯定会去干涉雄州防务。

究竟是悔改还是没有悔改,只看他接下来会怎么做就够了。

张璪又问:“如果辽人当真来攻,当如何处置?”

“若辽人敢于来犯,当然是坚决予以回击。若辽人举国而来,就在河北、河东设立宣抚司,以御辽寇。”

“这岂不是让吕惠卿如愿以偿?”

韩冈胸有成竹:“章子厚久在枢府,为帅时功绩显赫,若北虏来攻,宣抚河北河东,统括两路兵马,此一职非其莫属。”

韩绛闻言不禁摇头。他当然清楚韩冈怎么都不会让吕惠卿统领河北大军,可韩冈和章敦之前关系已经疏远下来,没想到他还是会支持章敦去河北。但是有王安石在,章敦会不会答应下来?与王安石和吕惠卿彻底决裂的决心,章敦可有没有?

只是看了韩冈的表情后,章敦心中有了一丝明悟,若韩冈与章敦还没有达成默契,他是绝对不会贸然主张让章敦宣抚两路的。

王安石当年支持吕惠卿,让曾布叛离新党,如今又是因为支持吕惠卿,让章敦也起了异心,众叛亲离,新党的天下还能支撑几日?

韩绛和张璪暗自嗟呀不已,却没有反对韩冈的建议。之前可以坐看王安石和韩冈翁婿两人打擂台,可如今图穷匕见,早没了让两人台下看戏的余裕。在这件事上,他们要么支持韩冈,要么支持王安石,可不论支持谁,都意味着不久之后政事堂中要多上一名新同僚。

韩绛早就与吕惠卿在政事堂中搭档过,张璪也不是不知道吕惠卿的为人,相对于刚刚担任参知政事就开始推行手实法、同时将韩绛挤兑得没处立足的吕惠卿,还是章敦稍微强那么一点。

他们不指望章敦能如韩冈一般——韩冈意在气学,除了一干有关气学发展的职位,其他方面权柄他都无意去争夺——但只要比吕惠卿强就好了。

“那吕吉甫呢?”张璪问道。

韩冈推荐章敦做了主帅,张璪很想知道,他打算如何处置吕惠卿?

“可为章子厚副手,分司转运之职。”

韩绛、张璪尽皆哑然,韩冈这是要吕惠卿自己辞职吗?

政事堂中的三位宰辅用了很短的时间就决定了接下来的对辽方略。

如果不是因为牵涉到了国中政治倾轧,这个决定其实应该更早做出才对。

现在他们还是打算等待辽军先来攻击,然后再做出应对。如果等辽人在坚城之下碰得头破血流,王师再顺势北上,当可轻取幽燕故地。

这是对宋人来说是最好的结果。

政事堂对辽战略的决定暂时不会公开,会等到辽军确实攻击边关之后,才会去报请太后,公开朝廷的任命。

韩冈需要时间去观察吕惠卿的行动,而韩绛和张璪两人,也不希望在辽人还没有开始以举国之力来袭时,就设立宣抚司,推动章敦上位。

但章敦身为当事人,不可能听不到政事堂中的议事声。

枢密院几乎是在同时得到了雄州奏报的副本。可对于如何应对辽军入寇的讨论,他们都只能局限在军事范畴。甚至如刘绍能的任用,也因为审官西院归属于政事堂,而无力直接干涉,只能通过与政事堂宰辅们共议来安排。

章敦当然想握有更重的权柄,所以在听说政事堂议定的结果之后,心境也难免一阵起伏。韩冈之前只是隐晦的提起,那样没落到实处、甚至没有明确的许诺,怎么也比不上现在几乎确定的事实。

如果他在宣抚使的任上能够成功抵御辽军,回来后就肯定升任宰相。不,其实引用韩绛的先例,同时兼任两路宣抚,统领北地两路禁军,不给章敦一个宰相头衔根本说不过去。

同中书门下平章事的头衔,只要接任河北河东宣抚一职,定然立刻就会戴到头上,这可是货真价实的宰相。

今年来的第一次,章敦迫切的期待起辽军的到来。
