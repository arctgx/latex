\section{第33章 枕惯蹄声梦不惊(三)}

太谷城的南北两门处,焰上云天。太谷县外以繁华著称河东的南北集市,全都陷入了火海之中,天空中都因此泛起了一片血光。

而烟火不及的黑暗里,正有数以千计的辽国战士潜伏。

萧十三并不在营地中或是附近,在开战之前,他便已经悄无声息的带着一小队亲兵潜至了城墙近处。他要亲眼见证他麾下的儿郎攻上太谷县城的城头,直至将韩冈擒拿到他面前的一幕幕。

但战局的发展让他出乎预料,宋人并没有为南北两处先行开始的进攻吸引去所有的注意力,当出战的其余各部刚刚潜近城下,便立刻受到了城上的攻击。

“城里的南蛮子发现了?!”

“箭声好急!”

随行萧十三的将校和幕僚中有了一丝慌乱,任谁看到原本黑暗如墨的城下,转眼间便被火光照亮,都会感到一阵心悸。那数千精锐岂不是都成了跑到太阳底下的老鼠,马前的兔子——自寻死路?

“南人有句话叫做骤雨不终朝,长不了。”萧十三不见慌乱,气定神闲的风范让许多人羞愧不已。

但半刻钟后,弦声依然。

一刻钟后,仍不见停歇。

两刻钟后,弦声终于稍稍缓了一点,但也变得更加拥有节奏。甚至诡异起来,长长的一段停顿之后,又在刹那间突然变得激昂猛烈。

萧十三变了脸色,只要熟悉战阵的,都能从其中听到陷阱的痕迹。

“城里的宋军到底有多少人?!”

“谁派的拦子马?!”

不止一人狂怒出声,那样密度的箭矢,绝不是三五千人就能做到的。在这样的箭雨下,出击的儿郎究竟伤亡了多少,很多人甚至不敢去想。

“宋人肯定是将平民都征发了上来。”

“不可能,平民百姓哪有这么快就进入角色的?要真是这样,以南朝的人口,谁还敢招惹他们?!”

兵贵精不贵多,民兵乡勇很多都是上了战阵就会腿软,再多也排不上大用场!但只听那急促却稳定的弦声,又怎么可能是平民百姓能做到的。

萧十三双手紧紧握着拳头,眼中尤闪着坚定地光芒。

他手上还有一支真正的精锐,也是他打算用来破城的依仗。大约两千人的精兵,主力身穿重甲,手持硬弩,与城头对射,前面还有人拿橹盾来抵挡破甲重弩。等到了城下,随行的未着甲的勇士就可以援梯而上,用最快的速度夺占一截城墙。

在萧十三最初的计划中,当第一批攻城的队伍找出了宋军城防上的弱点,第二批攻城部队将会立刻投入进去,不给宋人以喘息的机会。

他相信自己的人,他也确信自己的判断。宋军密集如狂风骤雨的射击不可能持久,士兵们的体力迟早会消耗殆尽,只要在合适的时机投入一批生力军,便能攻下太谷,擒下韩冈!

夜色下,一切都给模糊了。

两丈三丈的城墙是永远不会看不见的,但在城头上,却分不清城下的士兵谁更加精锐。

只要有人配合,宋人很难分辨得出主攻的方向,直到精兵登城的最后一刻,才会明白过来。

人,已经出发了。萧十三静静地等待着。

可能是发现了新一批的攻城军来了,城上箭矢的弦鸣再次急促起来。

萧十三一直用千里镜追逐着他最信任的儿郎,纵然在夜色下只是一片蠕动的黑影,但也看得清楚他们正在稳定的向着城墙接近。

‘箭矢并不总是管用!’萧十三欣喜的想着,也更加热烈的盼望着。

只是突然间咚的一声响,像是远处有重物坠落,随风而来,并不响亮,却使得萧十三一阵心慌。

那是床子弩的射击声!

至少是双弓合一的床子弩。

代州和几处军寨的床子弩的试射,萧十三和他手下的将领们都见识过了。就算没有旧年丧生在澶州的萧达凛,宋军至宝的威力也让他们心惊肉跳。

那三十多架大小床子弩和形如长枪的铁箭,都被各部给瓜分。但也让萧十三等人牢牢记住了床子弩发射的声音。

咚的一声接着一声响,在萧十三胸腔中带起了一记记重音。

为何宋军最可怕的利器会在这时候投入使用?!

这完全不用多想。

‘回不来了。’

萧十三哀叹着。

……………………

如同石头落水的床弩弦鸣,时不时便响起一声。每一次响起,很多时候,就是一名或几名辽国的精兵被串成了肉串。

韩冈之前让人匆匆修了敌台,只是用木料草毡和泥土搭起的简易建筑。本是打算将上弦机放在马面上敌台中,但经过计算,要保证足够的射击速度,给神臂弓上弦的人数就必须相当于弓手的两倍,这样一来,敌台的空间就显得太小了。

不过对于床子弩来说,敌台上的空间并不算小,足够放下最大号的八牛弩——床子弩的大小本就是要配合城墙的厚度——依靠城中的工匠,用木条加木轮,用了一天的功夫,就打造出了可以在轨道上旋转射击的床子弩底座。虽然不能用得长久,可支撑三五天也足够了。

而这样一来,床子弩便能轻易攻击到城墙近处的敌军,而不仅仅是瞄准远处的敌人。这也让太谷城的防御力又上了一个台阶。

太谷知县谄笑着赞不绝口:“枢密来太谷不过数日,便把县城打造成金城汤池一般。有此城池,就是辽贼再来十万又能如何?”

韩冈摇摇头,“比之统万城还差一些。”

韩冈旧年快离开陕西时,曾去参观过赫连勃勃所修建的统万城。那座矗立在无定河的古城,虽然被太宗皇帝毁弃,但依然可以从残迹中看到那统御万邦的万千气象。

统万城之所以号称坚不可摧,为数百年来无定河流域的第一名城,让赵光义将其视为另一座太原而干脆了当的毁弃,不仅仅是因为修建时曾以铁锥刺墙来检验城墙硬度——铁锥入墙一寸则杀工匠,入墙不及一寸则杀锥墙的士兵——而是因为城墙的结构安排合理。

统万城城墙的马面密集且向外突出于墙体甚多,攻至城下的敌军,攻城墙则同时受到两面马面的投射,攻马面,则临近的马面则可以支援。这样的布置,就算城墙不高不hou,也能使得城防固若金汤,何况统万城墙光是残基都有五丈,最高处甚至有**丈,厚也有七八丈,比之东京城也丝毫不逊色。

太谷县城的布置,韩冈有几处是学习了统万城,也曾想过太谷县的城防若是能如统万城十分之一的水平就好了。

只是这个时代大多数人见识有限,好些个官员都不知道韩冈所说的统万城究竟是哪里的哪一座。不过并不妨碍他们带着谄媚的笑容,来奉承韩冈的‘谦虚’。

韩冈听得腻味了,挥挥手让他们住了口。

今夜可以算是过去了。

夜战无功,想必萧十三不会再蠢到将脑袋继续往墙上撞。但韩冈不会这么说出口,谁也不能保证会不会有意外。甚至必须更加小心才对。

“传令下去。令四壁各部加强防备,决不可有丝毫松懈。天亮前人心思睡,是危险的时候,不要辛苦了一夜,最后功亏一篑!”

韩冈的吩咐让所有人悚然而惊,连忙应声答诺。几名将领也分头赶去城池四壁,去检视和督促守军不要懈怠下来。

但韩冈的口气又稍稍缓和了下来,环顾左右:“不过要是今夜无法登城,那辽贼到了明天,也就很难再组织起第二次进攻。”

幕僚们人人点头,下面的几个参与过军议的文官武官们也同样点头。

之前制置使司之中,已经对战局发展的各种可能性进行过了推演。如果不能顺利的攻下太谷城,且败得干脆利落,那么辽人决不会蠢到再把脑袋往石头上撞,肯定会清醒过来,明白自己的攻城水平到底有多差劲!——绝不是依靠时运得了代州、石岭等军城要塞就可以自大起来的。

接下来,辽人要么退回北方,要么干脆再南下去攻击北上的援军。如果萧十三信心不堕,又能控制得住他手下的兵将,多半会是后者——接连攻城不克,他肯定是不甘心的。

只要辽人能消灭了北上的援军,让河东的宋军失去了野战的能力,就是韩冈也绝无回天之力了。只能放任辽军肆掠河东。

不过韩冈之前一系列的命令,并让章楶去统领全军,都是为了让他们在与辽人的交战中保全自己。在这样的情况下,如果能拖住疲惫不堪的辽军,结果只会更好。

当然,退回去整备攻城器具,以期卷土重来,这同样是个选择。但那样的话,至少要十天以上的时间。这样的话,可就是让韩冈称心如意了。

就韩冈的角度来说,他最希望辽军能在太谷县中多留一阵,这样全歼萧十三所部当不是幻想。虽然河东一路丧师辱国,吃了大亏,不过大宋的核心力量并没有受到多少伤害,甚至还没有动员起真正的力量。

一旦辽国西京道的主力被全数歼灭,如果有朝廷的支持,韩冈他甚至敢直接去进攻大同,直逼奉圣州【张家口】。那可是击垮辽国的千载良机。就不知道能不能得到这个机会了。

