\section{第19章 波澜因风起(下)}

金榜贴出已经过去了三日,而再过五天就是新科进士带花游街的日子。

但东京士林中,对韩冈、叶涛两人的质疑却是一天比一天更为激烈。士林中的舆论,直接针对韩冈和叶涛的身份,来抨击王安石在抡才大典上徇私舞弊。

就算叶涛文章写得再好,只要想找茬,照样还是能找到不少拿来当靶子的地方。文人心思坏起来,本就是没有底的,几千人围观一篇文章,轻而易举就能戳得漏洞处处。何况文章好坏,主观上的评价占了很大的分量。若是带了成见来看叶涛的文章,也不可能给出太高的评价。

这些天来,心高气傲的叶涛又急又气,每一次被人挑衅,都会被气得七窍生烟。

“致远兄你又何须如此?你我的名次都是天子亲笔提上来的,即便是御史,也不敢乱弹劾!”

清风楼楼上,韩冈帮着自己和叶涛倒着酒,顺便出言安抚着。

“可是……”

叶涛本来还是因为韩冈比他还高上一位,心中多有不快。但现在外界的压力越来越大,对韩冈便有了同病相怜的亲近感。今天便来找韩冈诉苦。

‘可是什么……不就是没人围在你周围,原本的同伴全站到了对立的一面去了吗?’

只是韩冈没有半点同伴意识,他心情安稳的很,即便不停的有疯狗在耳边乱吠,也不可能要咬上来。偏偏有人在耳边长吁短叹,让他不胜其烦。难道不知道两个倒霉蛋坐在一起,只会让自己感觉更悲惨吗?

原本跟着叶涛走在一起的朋友,全都在在礼部试上被黜落。如果叶涛没有收到攻击,他们应该会很有风度的祝福叶涛,并把叶涛当作日后的靠山和助力,而更加恭敬的结交。

但现在。他们早就忘了叶涛是在礼部试后才与王安国的女儿定下亲事,一齐跟着士林舆论攻击起叶涛来。嫉妒之心,就是让人变得失去了理智,原本交情不错的朋友,这下彻底翻了脸。

叶涛来自浙江龙泉,跟他亲近的也基本色都是浙江士子。说起来,对于他们十几人,除叶涛外都没有通过礼部试,这一点韩冈都是很惊讶的。

要知道,今科籍贯福建的进士有四十一人,占到了进士总数的十分之一,仅仅少于有国子监在的开封府,接下来就是浙江,只是稍逊而已。

浙江路的贡生,则只有两百余人,差不多是贡生总数的二十分之一。浙江贡生中进士的比率,比全国平均录取率高出一倍,这样还近乎全军覆没。既缺乏人品,又没有能力,叶涛挑选朋友的眼光,的确让人叹息。

“玉昆你倒是安心。”

叶涛灌了口闷酒,睁着布满血丝的双眼,很不高兴的发现韩冈还是那等风清云淡的安定。

“谤人者甚忙,受谤者甚闲。流言蜚语只要不去在意,便会感觉很轻松。”

金榜题名,进士及第。

前一事韩冈梦寐以求,后一事他却从来没有幻想过。能做个同进士已经很难能可贵了,想登堂入室,来个及第,谈何容易。

出乎意料的成绩如同天降馅饼,尽管免不了要带来一身麻烦,可韩冈想了一想之后,就完全看开了。现在他根本就不在意,既然已经有了进士资格,加上他还是朝官,日后官途已经没有制度上的阻碍。

这样难道还不够吗?

叶涛就是既要名声好,又要名次好,太贪的结果当然就是睡不好觉,吃不好饭。韩冈所求甚少,所谓无欲则刚,闲杂人等的看法何必在意。

尽管眼下闹得厉害,但风头一阵就过去了。更别说,韩冈和叶涛的名次还是天子钦定,难道要赵顼自己承认选错了人?说韩冈、叶涛这两位二十多岁的青年人,年轻才薄,不堪为进士?

韩冈、叶涛并不是今科进士中岁数最小的,不过也是年轻到足以惹起他人嫉妒的年纪。

今年的探花郎,刚刚十九岁。而二十二岁的韩冈,论年纪,从小里排还是能进前十。就算是王安石,王韶这一干人杰,中进士的时候,都是二十岁以后了,没有说是十几岁就能跨马游街——司马光早一点,是正好二十岁。

三十老明经,五十少进士,这些科举场上流传的俗话,凝聚了无数四五十岁才得中进士的儒生们斑斑血泪,不是胡乱说出来的。所以有人对此嫉妒无比,让韩冈和叶涛,连杯水酒都喝不清净。

韩冈和叶涛坐在清风楼上风光最好的一桌,这也是韩冈定下的。若是坐在阴暗的角落中,就算能避开他人的耳目,也显得自己太过弱势了。

而座位风光好,也代表了被人看到的几率要高得多。先是楼梯蹬蹬一阵响,然后一群士人上了楼来。一见韩冈,立刻有人提起现在传得沸沸扬扬做的事来:

领头的士子也上来了,对着韩冈道:“原来九进士和十进士,今日二位进士来清风楼上,是为了借酒浇愁吗?”

“比起贤辈的饯行酒,当是稍胜一筹。”叶涛忒着眼,连站都没有站起来,口舌丝毫不饶人。

“不知贤辈有何指教?”韩冈却站起来,欠了欠身子。看似有节有礼,但高大的身材可以让他居高临下的向下瞥着人。而且还引用了叶涛对他们的称呼,讽刺意味自然都听得明白。

这些都是不着边际的甲乙丙丁,看起来就知道不会是多出色的人物。想来打落水狗,也得先看看自己手中有没有趁手的打狗棍。

已经中了进士的在这个时候都不会冒出来。千金之子,坐不垂堂,宝贝到手了,别人手中的也不过亮上一点,本质都是一样的东西,哪个会为此去闹?

而官员们更是都知道韩冈和叶涛的排位在呈与天子前,分别是第五等和第三等,是天子亲自拔擢起来的。指责王安石徇私,授意考官,然后拉倒天子面前做评判?打天子的脸很好玩吗?被天子打脸更不好玩啊!

所以就让落榜的穷酸们来闹好了,自己站干岸看着。同在清风楼上,有好几张桌子坐了新科进士和南省出来的官员,都在一边看热闹,没有过来解围的意思。

“韩官人的大作我等都拜读了,当真是让人叹为观止。”打头的一人出来说道。这句话听起来像是夸奖,但实际上还是讽刺。

韩冈呵呵笑了两声,不以为忤:“韩冈的确是短于文字,一榜进士已是喜出望外,侧身一甲之列,却是从来也没想过。礼部试和殿试之上,也是靠着见多识广而已,并不是说文采有多出众。”

韩冈的姿态足够低,却是一块滚刀肉。批评他的文学水平不够,他根本就不在乎,一口承认下来。

“韩冈在殿试多言关西河湟之事,也只是因为对那里内外诸事最为熟悉而已。既然天子要我等‘以所见言之毋隐’,韩冈也自当以所见所闻报于圣上。不知贤辈于此事上有何指教?!”

要是批评韩冈在策问中说的那一条条一款款,说句难听话,就是班门弄斧,没人有这个自信。如果闹到了天子面前,皇帝是相信韩冈这个出自陕西、参与收服了河湟的专家呢,还是相信与陕西、熙河八杆子打不着的外人?

他的策问,文采虽是不彰,但字句之中却是滴水不漏,想找漏洞都难。在殿试上写就的文章是事先预备好的,是他和王韶共同点心血。两人都是官场中人,怎么正确而圆滑的撰写奏章和公文,不让政敌找出错来,他们都是经常练习,不敢懈怠。这一篇经过仔细推敲过的文字,说得又是只有自家最为了解的事情,一点破绽都没有。就像一颗涂满油的珍珠,局外人想找茬,手沾上去就能滑开。

而且韩冈后两句更是说得十分清楚,他的排名是天子的决定。质疑天子的决定,到没有什么关系,说不定还能博一个直名。但韩冈已经说得很清楚了,天子提拔他,并不是喜怒爱憎而定。要想反对,自己掂量一下后果吧。

“天子青眼,不过是看在韩冈能直言而已,并不是韩冈文采高人一等。听说状元郎最近上书,说要将自己的功名让给其落第的兄长。韩冈虽不才,可此事上不敢后人,若有贤者能有鸿篇巨著,一述西北边事的来龙去脉,韩冈让了这位置也是心甘情愿。”

韩冈笑意吟吟,话里话外却是明明白白的反击,既然不服,那你就也写一本出来好了。

这个姿态强硬至极,让每一个士子都出离了愤怒。

韩冈其实是最让人嫉妒的。

今科的进士已经授官。除了本有官身的进士以外,其他的绝大部分都是授予了选人中的最低一级——从九品的判司簿尉。只有前六名,状元余中为大理评事,榜眼朱服为淮南节度判官,第三名榜眼邵刚为集庆军节度判官,第四名叶唐懿为处州军事推官,第五名叶杕为秀州司户参军,第六名练亨甫为睦州司法参军。

状元余中是直接升为大理评事,进入京官序列,这是应有之理。而其下朱服、邵刚等人虽然比其他进士多走上了两三个台阶,但依然还是选人,必须于选海中浮沉数载。

可韩冈已经从无出身朝官的国子监博士,转成了有出身的太常博士。

他的晋升速度。一辈子爬不出选海,或是越不过京官朝官那条分界线的官员,是根本不能与之相比的。而拿现在朝中的侍制以上的重臣来比较,韩冈从入官开始,到走到从七品太常博士这一阶级,也至少快了十年到十五年的时间。

众人正待要开口围攻韩冈,楼梯又是一阵响,一人上了楼来。看服色、外貌,是宫中的宦官。他上了楼来,立刻尖着嗓子叫道:“韩冈何在?”

韩冈甩开一群儒生,上前两步:“韩冈在此。”

“韩博士可是让小人好找。”那宦官抱怨了一句,立刻又道:“官家有旨,招韩冈即刻进宫,勿得拖延!”

