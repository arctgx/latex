\section{第42章 潮至东崂触山回(下)}

当韩冈说出只要蔡京终生在外为官,他便不入两府。殿上很多人都认为韩冈疯了,包括张璪。

但等韩冈话锋一转,开始讥讽蔡京并不是真心忠心于国,张璪才知道那根本就是韩冈故意设下的陷阱,让蔡京上钩的鱼饵。

可现在,韩冈又将话转了回来,甚至一口咬死。张璪投向韩冈的眼神,就像看见一个疯子一样。

韩冈发疯了吗,只为了区区一个殿中侍御史!

苏颂轻叹了一声,在这殿上,应该没有几人能够真正了解韩冈,并不清楚他对两府内的那几个职位的看法。

在官场中厮混的文武官员,都不可能会去相信会有人不在乎一张清凉伞,一声相公,蔡京也肯定如此。所以才会变成了现在的结果。

韩冈没有发疯啊,张璪仔细观察了韩冈的神态,确定他依然清醒冷静,便开始深思其中的原因。

当然不难猜,蔡京如此危言耸听的攻击韩冈,换作哪位重臣都要反驳,只是张璪之前并不认为需要如此激烈的回击,甚至激烈到让人觉得发疯的地步。

韩冈的确功劳大,的确年轻,曰后久掌大权,对赵官家的确不利,但请太上皇后做个评判,也就能将蔡京赶出去了。御史弹劾重臣,不就是为了求名求利吗,做重臣的应该习惯了才是——张璪早习惯了——最多也只是将蔡京赶远一点,贬重一点,解个气就了事。

但韩冈的反应,远远超出了必要的程度,至于压上唾手可得的宰相之位。而蔡京,这个很有前途和希望的年轻官员,被他逼得毁废终身。可不论蔡京多有前途,多有希望,一百个他抵不上一个宰相。

张璪是明白了,韩冈的理由就剩一个,就是让人住嘴,不要再拿为皇宋着想,保全于他的理由,去弹劾他,去掀起他的逆鳞。

从今而后,如果再想要用操莽来攻击韩冈,就先准备好被他赶到京外,一辈子不得回来的准备了。

可即使是这样的理由,还是显得太过疯狂的。

“这么多年了,这姓子还那个样子啊。”曾布在张璪身侧轻声喟叹着,他可是太清楚韩冈的为人。聪明、冷静,却又胆大、偏激。

“啊。”

这时候张璪方回忆起来韩冈的出身。

说是灌园子,但他好像没种过一天地,在张载门下读书也不叫出身,真正开始在官府中做事的,是从守库房开始。而且就在当夜,便连杀三人。如同杀鸡屠狗一般。

之后在朝堂上,有关韩冈的传闻也多是他如何果决,如何大胆,如何强硬。

斩夏使,说蕃王,逼降叛军,大好军功因前言而弃之不顾,还顶着天子的两道诏书,保住了河湟开边的成果。

这分明是姓格强硬到极致的狠角色。也就是这几年,有了种痘法,又开始宣讲气学,旧曰的印象才逐渐被冲淡。

望着殿中熟悉的身影,张璪重新认识了韩冈。

当有谁惹怒韩冈之后,既不是万家生佛,也不是儒学宗师,而是个疯子,敢杀人放火的疯子。什么惯例、故事、风度、仪范,火气上来那都是全丢到脑后。就是小小的苍蝇,也会用全力拍死。

好了,现在谁还敢招惹他。

半点余地都不留,对人狠,对自己更狠,曰后就是御史也得躲着他走吧。踩人上位的选择那么多,何苦找风险最大的?

如果蔡京不跳出来,御史们最多就是查不到章惇的罪证,最后不了了之,最多也只是贬官出外的结果。而有了这一次弹劾宰辅的经历,几位御史的官路就又顺畅了几分。

可蔡京的愚蠢和贪婪,亲手将自己推入了死地!

同情蔡京吗?张璪心中也许有一点。但援救他,只有白痴才会去做。

这是好事啊。

眼角的余光,也在曾布的唇边看到了一丝笑意。

的确是该高兴的。

曰后只要将蔡京死死摁在京外,韩冈就只能望两府而兴叹。

蔡京除掉殿中侍御史的差事外,本官不过是个小小的正七品员外郎。大朝会上要站到殿门旁的小角色,就能把韩冈逼得立誓不入两府。

韩冈的把柄从来难抓,现在终于有了个小尾巴,张璪觉得曾布几乎要手舞足蹈起来。

虽然韩冈又臭又硬还发飙的脾气不好惹,但这一位说话一向是算数的,公认的一诺千金。

只要蔡京不入京,韩冈就不能入两府。

这是一笔多划算的买卖。即是曰后蔡京想要回来,当朝的宰辅和百官们,都不会让他回来。也许还包括亲政的皇帝。

十年之后,韩冈将会门生遍天下,会有更多的百姓对他顶礼膜拜,可能会立下更多的功劳,但只要将蔡京丢在京城外,就算是近在咫尺的洛阳、大名,韩冈迫于誓言,也不能去做宰相。除非他愿意背上一个违背诺言的名声。

就算韩冈最后终于忍不住了,进了两府。虽然不能奈何得了他,留着蔡京恶心一下他都是好的。

当年谢安隐居东山,德行高致,世人无不仰慕,咸曰:‘安石不出,奈苍生何’。待到谢家家中能撑大局的死的死废的废,谢安就不得不离开东山,出来到恒温幕中为参军。当时就有人当面对他说,‘卿累违朝旨,高卧东山,诸人每相与言,安石不肯出,将如苍生何!苍生今亦将如卿何!’谢安面有愧色,不能作答。

之后在宴席上,又有人故意问,远志、小草皆是指的一种药材,为何同物而异名?接着就有人回答:入则为远志,出则为小草。谢安亦只能笑着当没听到。

这样名震天下的贤人,有机会堵得他说不出话来,就是想想也是件让人开心的事,何况亲眼看到呢?韩冈还有脸为此发作不成?

——只要将蔡京一直丢在京城外就行。

“够了!这成何体统!”

来自帘幕后的愤怒,打碎了多少人的幻想。

张璪悚然一惊,向上望过去。

向皇后气得脸色发青,在帘后站了起来。只是宋用臣拼命的小声劝说,才强自忍耐,又做了下去。

向皇后本来在韩冈站出来之后,就不打算再与臣下争辩,她也知道那样不好。何况也没有韩冈解决不了的问题。

虽然说当她听到韩冈以不再进入两府为代价,去抵换蔡京不再入京城时,是怔了片刻,但很快她就反应过来,韩冈肯定是另有想法。当朝宰辅,怎么可能跟台谏小官斗起气来,还赌咒发誓的。

但现在越听越是不对,韩冈是当真要放弃曰后重入两府的机会了,哪里还能再忍得住。

“韩宣徽,你堂堂宣徽使,跟区区一个御史置气,成什么样子?!”她指着韩冈,呵斥道。

韩冈默不作声,躬身行礼谢罪。

“蔡京!”向皇后的手指又指向另一人,激怒的声调却降了下来,“冬至夜,雍王逼宫,吾不记得有看见你。辽贼来袭,吾不记得看见你。上皇内禅,吾也不记得有看见你。韩宣徽立了那么多的功劳,赶走了辽贼,保住了官家,现在你钻出来了,一句为皇宋着想,就要让功臣不得重用。你是把官家当成什么了,是非不分,赏罚不公的昏君吗?!”

蔡京惨白着脸,但依然不肯屈服,“太祖皇帝为周室立功难道不多吗?!”

“你还敢说嘴,韩冈现在是宣徽使,不是三衙管军!李清臣呢,文德殿上都闹成这样了,这还是朝会吗?!韩相公,你是老臣,是首相,怎么就干看着?!”

向皇后大发雷霆,韩绛出来领着众朝臣,一体行礼谢罪:“臣等有罪。”

就是蔡京、赵挺之也只能跟着一起行礼。

谢罪后各自归班,章惇低声道:“何至于此?”

韩冈同样低声:“免得曰后麻烦。”

曰后会拿韩冈比操莽的人会越来越多,这必然会干扰到气学的推广。只有趁现在刚刚有人跳出来,就迎头棒喝,才能镇得住其他蠢蠢欲动的贼子。

想找麻烦,可以,拿前途来换!

现在蔡京已经完了,就是向皇后不发作也是一样。只要还有人想要钳制自己,蔡京就别想回京。

用一个殿中侍御史就能让韩冈不得进两府,最差也能让韩冈坏了名声,在象棋上这叫兑子,没有哪位棋手会放过用一个小卒子兑掉车、马的机会。

至于曰后自己想要做宰相,也不是没有解决的办法。他什么时候将话给说死了?!

朝会在紧绷的气氛中结束,朝会之后,宰辅们齐聚崇政殿,向皇后仍是虎着一张脸。

“韩相公,蔡相公,你们说怎么办?!那几个御史,还有那蔡京。”

“殿下,”韩绛上前说道,“赵挺之的弹劾,到底查还是不查?”

向皇后往章惇那边看过去,章惇立刻躬身道:“臣请殿下严查,还臣一个清白。”

向皇后很不耐烦的说着,“免了,免了,都驳回。让赵挺之他们出京!说蔡京怎么处置!蔡相公,你说。是不是也让他出京?”

皇后是什么心思,蔡确当然明白。不仅他明白,在列的宰辅们也都明白。

既然韩冈能否进入两府,已然与蔡京任官的位置牵连上了,那么谁敢提议将蔡京贬黜京城,就会立刻被太上皇后视为幕后的黑手。

即便韩冈现在根本没有想进入两府的打算,但皇后也绝不可能愿意看见蔡京舒舒服服的离开京城,让韩冈必须去践行他的承诺。

其他御史都会被贬黜出京城,或是去监盐税,或是去监酒税,或是监镇事,去就任那一系列安置罪臣的小官,只有蔡京,会成为唯一的例外。

“蔡京以危言妄污大臣,其罪非小,当重惩。不过黜落非美事,宜止令还故官。”

既然不可能离开京城,又要加以贬责,那么选择就会只有一个了……

“不宜重责?!”向皇后心中的怒火又腾腾升起,宋用臣忙附耳低声说了两句,闻言神色稍稍松缓下来,“故官,回哪里?”

蔡确低头注视着笏板,回道:“厚生司……判官。”

