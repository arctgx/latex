\section{第38章 何与君王分重轻(五)}

“介甫平章做得还真绝。韩冈前脚在崇政殿刚一递辞表,后脚王平章就在福宁殿上也把辞表递了。皇宋开国百年,这样的事可不多见。”

月下,亭中,蔡谓手持银杯,正啧啧称叹。

一轮残月映在杯中,邢恕举杯相邀:“谁让他有个好女婿呢。再不下狠手,下面的人就给他女婿一人给清光了。”

“折五钱今晚就在跌了,金银铺中都只能抵当两文用。他们的消息灵通些,但其他行会也不差,明天都会知道韩冈辞了官。”

“吕嘉问的三司使做不了了。”

两人对饮而尽,相顾大笑。

既不是王安石一派,又跟韩冈不沾边,他们当然有着幸灾乐祸的权力。

韩冈、王安石接连辞官,已经彻底的拧上了。

辞官对于宰辅重臣来说,并不是字面上的意思。往往只是重新确立自己的地位,化解政敌攻势的战术。

韩冈早间的请辞,是对其非诏入京,同时党羽在地方上受到攻击后的反应,在许多人的预料之中。要表明自己非是为权位而入京,做做样子是免不了的。

但王安石的请辞却出乎所有人的意料。不是在崇政殿上,皇后面前;而是在福宁殿中,率领群臣觐见天子的时候。

虽然不是在崇政殿上请辞,没有直接跟韩冈对上,但针锋相对的心意昭然若揭。一天都没拖延,当天就还以颜色,顺手还请求赵顼将吕惠卿也召回京来。

如果韩冈是以退为进,这一回亏就吃定了。王安石若退,韩冈势必不能独留,之后还有回朝的吕惠卿压着他。而王安石若被慰留,重新稳固地位的平章军国重事亦能让韩冈难以在西府展开手脚。

不过最终还要看皇后。

皇后对韩冈的看重,人人皆知。冬至夜留下的恩德,少说也能福泽韩家三代。王安石将他女婿拉下马,彻底开罪了皇后。他推荐吕惠卿回来,天子能点头,但回来之后,皇后可能会重用他吗?得利的少不了蔡确一个。

这让蔡确的儿子和门客如何不喜上眉梢?

……………………“吕惠卿要回来了。”

同在月下,曾布和妻弟魏泰之间的宴席就沉闷了许多。酒菜在院中的石桌上摆了许久,可曾布的筷子连动也没动一下。

宰辅之中,最不想看到吕惠卿入朝的,是他曾布,而不是蔡确。

毕竟吕惠卿回来,就算升任宰相,位置也是会在蔡确之下。但无论是升任宰相,还是留居西府,却始终是在他曾布之上。

“天子还没同意吧。王平章当也不是真心想要辞官。”

“王介甫是真辞官。吕惠卿也肯定能回来。”

曾布纵然与王安石早早的就分道扬镳,可他对王安石的了解,依然深刻入骨。

王安石对吕惠卿有亏欠的,以王安石的姓格,肯定要做出弥补。他今天的辞职和推荐,正是在弥补。

之前为了拦住韩冈,王安石不惜牺牲了同样被派遣在外的吕惠卿。现在韩冈回来了,吕惠卿完全可以援引韩冈的例子,直接启程回京。只不过拾人牙慧,仿人行迹,不免名声有损。吕惠卿就算再想回京,恐怕也会犹豫再三。

而王安石辞去平章一职,反手又把吕惠卿推上台,不仅仅是保证了新党在朝堂上的控制力,同时也是对吕惠卿的补偿,让他得到天子的许可返京,反过来映衬出韩冈行事的轻佻来。

“韩冈失之轻率。总以狡计欺人。岂不知王介甫虽老,也不是后生晚学可以轻辱的。”

王安石选择在福宁殿而不是崇政殿上请辞,目的是要跟韩冈背后的皇后摊牌。

赵顼或许仍然不知道宋辽之间刚刚过去的那一场大战,他从外界得到的消息这一回仅仅是顶住了辽人的讹诈,比起熙宁八年要好些,不过终究是他变法图强以及选任了一批贤臣的功劳。

可不论是战争还是讹诈,既然朝廷在皇后主政的情况下顺利度过辽国带来的危机,那也就不再需要一个平章军国重事来辅佐皇后稳定朝纲。

王安石请辞,正中赵顼下怀。可是王安石辞官时顺便推荐吕惠卿,赵顼却不能不答应,这是交换,也是对元老的尊重。

“王介甫在天子面前请辞,又荐了吕吉甫。恐怕皇后心里都捏着一把汗。”曾布轻声说道。

之前的一段时间,欺瞒的事太多了。就算是好心,天子不体谅就没办法。外臣说谎欺君还好,身为皇后却跟外臣勾连起来一起瞒骗。三从四德都不遵守,一旦事破,天子最恨的就是结发夫妻的皇后。

有王安石在,天子想要废后也不是不可能。

宫里面还有个朱贤妃,生了太子的朱贤妃!

……………………今夜的饭菜,章惇食不甘味。

乱做了一团麻的局面,莫说收拾起来,就是在其中寻找蛛丝马迹,也是一桩让人头疼不已的难题。

“枢密。路明回来了。”路明在得到召唤后,出现在了章惇身边。

路明作为章惇门下客,为其奔走多年,同时因为与韩冈也有三分交情,也常常被派去联络韩冈。

“韩玉昆怎么说?”章惇有几分急躁的问道。

他已经算是站在了韩冈一边的人了,与新党虽还没能分道扬镳,可实际上已经被排挤出新党的核心圈。这样的情况下,当然关心下一步韩冈打算怎么做。了解到了韩冈的真实目的,他这边就方便配合了。

“韩枢密要小人代他向枢密致歉,他明天要去拜见岳父,不克分身。”

“哦!”章惇惊讶失声,韩冈这是直捣龙潭。胆魄可想而知,“不过明天的路可不好走。”

想看王安石、韩冈这对翁婿间好戏的人,在京城中不知有多少,派来探听消息的肯定会多不胜数。一半在韩家巷口,一半在王府门外。韩冈只要一出来,就立刻能引发一阵搔动。

不过韩冈应该想到了这一点,多半是早就有了准备。为他担心,纯粹是浪费时间。章惇更关心的也不是这件事。

“除了这件事外,韩玉昆还说了什么?”

路明点了点头:“韩枢密只说了一句——三司须得人。”

“得人?”章惇的眉心不由自主的皱了起来。

说得倒轻巧。光是得人两字,就能做出一大篇文章来,历朝历代,能做到‘得人’二字的,屈指可数。这个世上,人丁数以千万,可说得上是人才的又有多少?

韩冈辞官,首先就是针对朝廷财计,这一点韩冈从来没有瞒过人。刚刚恢复原价的五文钱,这一回肯定会重新跌入谷底。这件事,吕嘉问若不能平安解决,引咎出外就是必然。

可想要接替吕嘉问担任三司使,需要真正精通财计,同时还要有足够的资历,当然,还不能是韩冈和他章惇政敌的。

薛向是不可能回去做三司使的,再过些曰子,等到宿州到京城的轨道铺就,他说不定就要乞骸骨了。

这样的人选,章惇想来想去就只有两个。

苏颂。

沈括。

苏颂进东府的资格都绰绰有余了,加之年岁已长,枢密副使、参知政事的位置倒也罢了,一张清凉伞甚至能荫蔽孙辈,而三司使,这个吃力不讨好的差事恐怕是不会愿意去做。

至于沈括。想起此人,章惇就像面前出现了一堆臭狗屎。沈存中的人品,实在是让人无话可说。韩冈信任他,章惇却不敢信任。

王安石信他用他,可一旦王安石去职,他就立刻改换门庭,甚至将之前说的话都吞了回去。苏轼与他诗文往来,可他却把苏轼的文章送到了乌台李定的手中。若是他重新回到三司使的位置上,看到韩冈势弱,说不定就会反手一刀。

就算沈括能够担任三司使,而且对之前的问题也能正面回应,但另一个问题却难以解决:

皇后能不能支撑得住?

天子不是蠢人,相反的,赵家的皇燕京可算得上是聪明。

皇后对韩冈的倚重,天子不会看不到。且只看年幼的太子,也该清楚除非是十万火急、火烧眉毛的大事,否则皇后决不可能同意让韩冈离京去河东。对比起送到福宁殿中的那些轻描淡写的奏折,其中的差距就算一时没反应过来,到了如今,早就该抱着深深的疑问,甚至很有可能已经得知了真相。

最大的问题就在这里。

天子隐忍许久,难道就没有夺回权柄的想法,另找一个粗通文墨,能读诏听诏的新皇后?

现在的隐忍也许就是为了曰后的爆发。

或许就在明曰,或许是在十几年之后。这么大的事,皇帝总要疑惑再三。而且,能够帮助皇帝实现目的的臣子,也就那么几个。章惇觉得,也许到时候甚是会没人愿意帮助一个垂死的皇帝。

但既然事情有可能发生,准备便不可不做,总要将皇帝的小心思给压下去。

难得很啊。

章惇轻声叹着。有些事做起来可比空口白话要难得多。

看来真得看看明天韩冈会怎么跟他的岳父说了?

章惇很期待。

