\section{第11章 安得良策援南土(六)}

晨雾蔼蔼,内城城门刚刚打开不久,守门的士卒正靠着巨大的包铁木门在打着哈欠,三名骑兵就擦着他的身子,如旋风一般冲出朱雀门。

暴烈的蹄声在门洞中回荡,差点被撞倒的士兵扶着大开的城门,惊魂甫定的冲着三名骑手的背影骂骂咧咧,只是声音不敢放大。仅仅一瞥之间,内侍和班直的服饰都从眼前划过,在内城城门已经守了七八年的他,当然知道这是带着紧急诏令出外的使臣。

‘该不会又出大事了吧?’他望着远去的背影,想着。

在韩冈酸涩的双眼中,艳红色的晨曦分外刺眼。为了出兵一事,他竟然在宫城中待了整整一个晚上。天子可以找借口辍朝一日,等睡起来后再招群臣入宫议事,而章惇的翰林之任,也有好几人可以代替。但他韩冈可就没这么好运气了,只能在初起的晨光中,往衙署行去。

眨眨眼的功夫,当韩冈将视线从东方的霞光中重新转回到正南的时候,方才刚带着诏令一路向南的骑手,已经穿过了朱雀门,越过了州桥,只剩下小小的背影。

“希望玉昆你的表兄接令后的动作能快上一点。”章惇从信使身上收回视线,回头看着韩冈,“如果我们抵达潭州的时候,他还没有将援军整顿好,那问题可就大了。”

“家表兄的为人、行事,难道学士不知?”韩冈笑了一笑。如今被点上统领荆南军南下的将领是他的表兄李信,这算是交换条件了,“整顿兵马要一定的时间,但不过兵力不算多,调集起来应该不会太慢。就算我们是兼程南下的话,最多也只能在京中再待两天的时间,两天后就得启程了。关键还是能带出来的兵力够不够。”

天子已经批准调出了两千兵作为南下的援军,同时两千人也是章惇用来改造广西军的核心兵力。当初他也是这样带着一队关西精锐抵达荆南,在短短的时间里,就让荆南军成为翻山越岭、涉水逐寇的南国精兵。只不过这两千兵只是军籍簿上的数字,其中被吃掉的空额,当然并没有排除。

“但玉昆你放心,愚兄选定的几个指挥,都是荆湖南路的精锐,至少能保证人数在一千五百上下。”章惇对韩冈作着保证。

“一千五百人?那就是七成半喽。”韩冈点了点头,“也算得上是精锐了。”

区分当今大宋官军的精锐与否其实很简单,就看被吃掉多少空饷就足够了。空额越多,理所当然的战力就越差。潭州的驻军同样是分属不同的军额,其中的精锐,空饷比例大约在两成半到三成左右,而剩下的数字,则是五六成的样子。章惇、韩冈在军中时日不短,对此中情弊也是了如指掌。

一起向前行了几步,御廊两边的早点摊子这时候早都摆出来了,人气倒是旺得很。章惇和韩冈看了看,嫌人太多,收了在这里吃早饭的心思。

“要不是靠着这三五个精锐的指挥,哪里奈何得了沅州的田元猛,更别说梅山的苏方,飞山的杨光潜了。当初刘仲武和令表兄从关西带出来的兵,只要还留在荆南的,泰半尽在其中,算得上是西军的一支了。”章惇笑道,“只是不能与真正的关西劲旅相提并论。”

“已经很不错了。”韩冈叹了一口气,“西军中,只有极少数的精兵才能达到九成,更高的就只有各路的选锋军。剩下的,基本上都是七成八成的样子。”他说着又笑了起来,“学士可知西军中有哪一支是没有空额的?”

章惇想了一想,反问道:“可是蕃军?”

韩冈呵呵冷笑了两声,点头道:“正是!善财难舍嘛,上上下下对蕃军查得都是最严的。”

章惇长叹一口气,“如果军中纲纪,都能如对蕃军一般森严就好了。”

“将兵法也没能做到,此事谈何容易。”虽然很可笑,但这就是现实。就算王安石、蔡挺推广将兵法,也没有办法改变的现实,“这两年的裁军练兵,的确挤出了许多空额,不过没再深入下去。而将兵法也仅是加强训练,并统一号令。想要跟蕃军一般严查,天下的军汉可都要得罪了。”

将兵法改动的只是指挥一级以上的编制,对于指挥以下——相当于后世营级以下——的编制,完全没有做出多少调整。只要如今军制的根本没有改变,吃空饷、喝兵血的弊政就不会消失。

在韩冈看来,不从头打散编制,另立新军。官军要想保持一定水准上的战力,只有依靠战争来磨砺了。如果少了战争,虽然记不得是多少年后,河北军面对南下金兵的表现,韩冈可是如雷贯耳的。

两人一时都有些沉郁,沉默走了一阵。

转过西十字大街,街上的行人渐渐多了起来。虽然不是城东的鬼市子那般,里面的商贩都是三更起五更收,不过西十字大街再往西去,也同样有个早市,不比鬼市子差到哪里。

章惇回家,就是再往西去。而韩冈要去军器监,则是要往北。

正要分道扬镳,章惇忽然问道:“好不容易在军器监中有所成就,现在就要去广西任官,不知玉昆担不担心监中之事?”

韩冈呵呵的又笑了起来。世间能做到萧规曹随的人毕竟不多,接任者少不了跟前任过不去。吕惠卿接曾布司农寺的任,就直接发文要下面的属僚上书‘未尽之事’。而韩冈接吕惠卿的任,也一样不客气,嘴里说着萧规曹随,但转眼就用一项项发明,将军器监彻底的揽在了手中。但接手军器监的下任,想找他韩冈的错处,也不是那么容易的:

“军器监的账目干干净净,我没在里面探过一次手。里面的各项规条,又多是依从吕吉甫,难以更易。再说了,如今板甲局的产量是一天四百五十具。一个月就是一万三,光是要维持现有产量就够人忙活一阵了。而想要在军器发明上胜过我,更是不可能。别的我韩冈不敢自傲,唯独在军器这一处,是绝不会输人的。”

“玉昆,解释得太多了。”章惇看了韩冈一眼,眼中带着笑意。

“……嗯。”韩冈默然片刻,忽的自嘲一笑,“因为不甘心啊。”

韩冈的确有些不甘心,倒不是因为要去广西任官,而是邕州之事出乎意料之外。如果没有邕州的事,他还想在军器监中再做上一年,至少将钢铁冶炼方面的工艺,尽量再推进一步。不论谁接手军器监,对冶炼方面的投入绝不会比在他的手上时更多。

可惜交趾南侵,韩冈自然需要南下。相比起还可以找到一定水平的合用人选代管的军器监,韩冈在军中医疗上面的作用——尤其是给军中将士的信心——则难以替代。

“军器监可能会由曾令绰【曾孝宽】回来兼管,玉昆你就放心好了……”

韩冈点点头,也算放心了一点,至少曾孝宽不会蠢到想要在军器监。

“京中之事都要放下了。此次你我去广西,可是要还南疆一个太平!”章惇顿了一下,“另外,则还有一句。”

“什么?”

“卧榻之侧,岂容他人鼾睡!”

……………………

‘卧榻之侧,岂容他人鼾睡。’

回到军器监中,韩冈向下属们通报了要调任广西的消息。引起了一阵混乱之后,他就让人快点去做交接的准备。下面的监丞、令史,都赶着去忙了,韩冈则是坐在公厅中,想着章惇的话。

章惇出任桂州知州,韩冈任广西转运副使,新党两位干将全都出外,而且是出任州郡和漕司,并非是临时性质的差遣。从人事的角度上来说,新党似乎是吃了大亏。而且一旦平南行营建立,在领军的主帅压制下,章惇、韩冈能立功的机会都不会太多。

但话说回来,只要章惇、韩冈表现得足够好,在平南行营正式建立之前,就有了足够了战果。天子和朝堂也只能围绕他们两人来组建南征的行营。

听章惇的口气,他便是这么想的。

章惇由翰林学士出任经略使,如果当真能做到,那么自此以后,便有了进入枢府的资格。

而反过来看自己。说起职位,转运副使其实要比转运使低上两级到三级——这是为了凸显主官的权威,不至于因纷争而坏政事——也就是中州知州的资序。这一项任命对韩冈来说算不上是高升,如果再将京城、地方的问题算进来,他其实是吃了大亏。可挣功劳的职位,就像是沿海管盐的肥缺一样,都是多少人宁可高职低配,也要抢到手。只要这一次能顺利的平定交趾,靠着预定中的随军转运一职,少不了会转成正任的转运使。

从京中监司的主官,升任一路漕司,绝对是大跨步的前进。尽管冯京等人的推荐别有一番心思,韩冈还是要‘感谢’他们给了他这个机会。

至于日后想将他阻于京城之外,倒还没有那么容易。

