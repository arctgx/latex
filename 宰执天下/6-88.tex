\section{第十章 千秋邈矣变新腔(十)}

“什么准备?”黄裳坐直了一点。

“此非小人可以知晓,等明日博士可以自去问一问我家参政。”

黄裳稍稍有了些精神,韩冈并没有忘记自己,也没有放弃的想法。

这样对他来说,已经足够了。

……………………

到底要怎么安排黄裳,韩冈已经有了腹案。

以黄裳的情况,想要改变在世人心目中的形象,只有建立功勋,尤其是去他人不愿去的地方建立功勋。

而现在能安排黄裳这个太常博士的地方,在韩冈看来,一个是去与辽国争锋的海外耽罗岛,一个则是去大小战事从无一日而绝的夔州路。

巴蜀之地,川峡四路,益、梓、利、夔向来并称。而在这四路中,甚至可以说天下诸路中,以夔州路【重庆附近为主】——也就是益州路【成都周边为核心】和梓州路【南充、资阳一带】以东,利州路【阆中、汉中一带】以南,两湖以西,这片由群山组成的区域——最为穷困,甚至比两广、也就是广东广西还要穷困。

而且在夔州路上,是土官们的天下。土官们的庄园里面,蓄养着大批的农奴,他们的性命只取决于主人的心情。以经济制度来说,这里甚至还是属于奴隶制的阶段。

这些土官是大大小小的部族之长,也是当地的土皇帝,在其领内往往恣意妄为,甚至过路的商客都会遭其毒手。

不过这都是常见到没有人会为此而惊讶,名义上管制这一片土地的衙门也不会处理。只会等其作法自毙——其抢掠商客的行为,只会让商人绝足领内,最后连盐都买不到。

夔州路各州县的官员们,只会去注意哪家夷族又在开始向周围扩张了。对于这种想要强行改变土地和人口归属的行为,朝廷一直抱以极高的警惕,甚至称其为獠贼。局面每每会出兵攻打,以维护当地局势的稳定。

而自从朝廷分出了一部分注意力在西南夷身上之后,如果有哪家夷族闹得太过分,不等周围的部族向附近的州县求援,附近的州县多半就会开始行动,或是召集众部共讨,或是从周边州县,调兵过来。

夔州路的范围不小,但朝廷实际掌控的区域却不大,核心区域只在后世的重庆一带,其羁縻区域,却可以向南延伸到后世的贵州。

就是在韩冈眼中,夔州路的绝大多数地方,现阶段都没有什么价值。只要能够维系朝廷表面上的统治就够了。但其中一些关键性的战略要点,就必须占据下来。

而且以夔州路幅员之广,不是没有可供移民安置的地方。尽管如今对夔州路的评价,就是瘴疠多,蛮夷多,但不论是以后世的眼光来看,还是以现在的眼光来看,夔州路上适宜耕种的土地还是很多。

大宋和韩冈,现在都需要土地和战争,其中夔州路南面的羁縻区,正是得到土地和战争的好地方。而羁縻区再向西南,便是大理国的所在地。

韩冈准备将朝廷军事上的主要精力,转移一部分到夔州路上去。那些土官能够如此跋扈,还是仗着地利,朝廷大军进剿不利的缘故。

就在夔州路上,还有着一支习惯于山地作战的军队,尽管是校阅厢兵序列,韩冈还是相信他们的表现不会太差。加上一部分西军,也将会南下——主要是曾经虽王中正南下蜀地,平定茂州之乱的军队——两边合力,足以将不曾恭顺的当地土官狠狠的整治一番。

如果黄裳去了夔州路,借助自家的面子,自能先行收复一部分西军将校。有了这些本钱,在夔州上,黄裳还是能够有所成就。

当然,与夔州路紧邻的梓州路,穷困与混乱的程度都差不多。但梓州路上,还有一个熊本在,跟这个熊本争夺位置,黄裳的资格实在太低了一点。

当今几位著称于世的帅臣中,熊本便是在平定渝州獠贼之乱一举成名。

其名望虽不能与韩冈、章敦、郭逵相提并论,但也是震慑一方的名帅。一旦四边有事,尤其是西南方向上,都会征求他的意见。

韩冈用烛台照着面前的地图。

夔州路的朝廷控制区,最南端是南平军【今南川】,而南平军稍东北一点即是黔州。黔州的州治在彭水,但黔州名下,还有地域远大于黔州的诸多羁縻州,就像韩冈当年在邕州,下面也是一片羁縻州。

黔州的羁縻州远在南平军以南,包括后世贵州的部分。而穿过这些羁縻州之后,便是大理。

太祖皇帝当年玉斧一划,以大渡河为界,宋兵自此不过大渡河。说是如此说,但大渡河南还是有许多部族听从朝廷的吩咐。那些夷族族酋,皆以得到朝廷封赠为荣,尤其是朝廷赐予的冠带衣裳,最为族酋们看重。

在西南这个战略方向上,韩冈的最终目标便是大理。想要实现这个目标,难度并不低。不过相对于北方边界,可能会维持不断的一段时间的和平,一个合适的练兵与拓张之地,是大宋不可或缺的需要。

收起了并不精确的地图,现在在韩冈面前,还有更重要的事情等待他完成。

有第二次廷推,有进士科殿试的考题,还有制科御试的题目。

原本制科都是在中书门下考试,出题的都是宰辅,之后才改去崇文院,由此有了阁试。

这其实就是像太祖皇帝赵匡胤在礼部试之后,硬生生的加了一个殿试。阁试的出现,来自于天子对中书门下的不信任,加上削弱宰相权柄的想法。

在蹇周辅等阁试考官被驱逐之后,制科十科就需要进行改变。但在加以改变之前,不久之后的进士科殿试和制科御试才是迫在眉睫的问题。

考试的问题很让人心烦。

不仅仅是韩冈在头疼,那些贡生更是心乱得厉害。

在礼部试的结果出来之前,许多对自己有那么一点信心的贡生,都开始为即将到来的殿试做准备。

殿试的题目,来自于天子。但以向太后的水平,当然不可能给三四百名通过礼部试的贡生出题。依照过去的惯例,当天子不打算为殿试出题的时候,进呈考题供天子选择的,是政事堂的宰相参政们。

只是自从熙宗皇帝登基以来,将出题的权力牢牢掌握在自己的手中。尽管是殿试,但来自全国各地、不同阶层的考生们,也是皇帝除了官僚体系之外,了解京城之外各地治政水平的唯一途径。只是为了了解新法的推行情况,熙宗皇帝都不会让宰相来剥夺他了解地方内情的机会。

但向太后决没有先帝的水平,尽管在她的治下,对外战争的表现远比先帝要亮眼,不过论起才学,她的确远远不能同她的丈夫相比。

气学的韩冈和新学的王安石,到底谁才能占上风,而太后又会偏向那一边,这个问题困扰了太多的人。

加之制科阁试上几位考官因为开罪了韩冈,被发配去了西京,这样的结果,会不会影响到殿试考官们的评判标准,是谁也说不清楚的。

礼部试中有没有韩冈的人?到底有几个?又是哪几位?

很多人开始去在五千参加礼部试的士子中,寻找来自于关西的那批贡生。希望在其中找到与韩冈有关系的人,以他为突破口,得到有关殿试的蛛丝马迹。

韩冈自然知道那些等待礼部试的结果的贡生们,正在慌张个什么。但他对此毫不关心,连看乐子的想法都没有。

在制科的御试之前,首先是进士科殿试。

由于被推荐人都进入了御试,韩绛、张璪都不可能与韩冈争夺出题的权力。而且在殿试的考题中,政事堂进呈以供御览的题目,韩冈也有参与其中的权力。

在黄裳被黜落之后,韩冈对谁能通过御试没有任何兴趣。而能够通过礼部试的气学弟子,数量也极为可怜。

韩冈现在所想的是怎么才能在考题中体现气学的观点。

韩冈所主张的气学,其核心是格物致知四个字。对个人能力,讲究经世济用。重实际,轻言谈,这是气学的偏向。相对于经术诗赋,更看重官员对具体政务的处理能力。

怎么才能在题目中反映出考生们经世济用的水平?不改变考题结构的话,想到达到这个目标很难。而想要临时改变,就要面临考生能不能适应的问题。

不过在临考之前,对考试的内容做大改变,并不是不可能的事。礼部试不可能,但殿试却有先例。

熙宁三年的殿试,考官们给考生发下了一部韵书,这本是为了殿试时的测试诗赋做准备,但赵顼上来之后,便发下了一道策问为题。

省试定去留,殿试定高下。能进殿试,基本上就是进士了,这时候改变一下考题,不会引起太多的愤怒。

有旧事在前,韩冈对此便有一些把握。

但到底怎么去做才更为合适,就是一件需要多费思量的难题。

灯罩中的烛花闪了又闪,韩冈终于提起了笔。
