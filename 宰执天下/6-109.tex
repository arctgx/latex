\section{第11章 飞雷喧野传声教(六)}

徐良走进厂房,喧嚣声顿时就缠绕了上来。

这里只是给铸好的炮管进行后期处理的地方,检查炮管上是否有裂痕、沙眼、毛刺之类的缺陷,然后将合格的炮管打磨光滑。

但上百人围着十几根炮管一起动手,滋滋沙沙的杂音充斥耳间。这种声音,听着的时候就让人心里发毛,浑身都不自在。

厂房中做工的匠师、小工,耳朵里都塞着棉纸团,要说话时就得放大嗓门,如同吵架。

没什么人注意到徐良进来,这位火器局的同提举,从匠人一下成了官人的幸运儿,也悄无声息的走到了厂房的一角。

几名小工正围着一根粗长的炮管,用刮刀、磨石细细打磨着。

这支青铜炮管远比同一厂房中正在处理的其他炮管要粗长得多,从口径到外径,再到长度,皆远胜不远处的另一根城防炮的炮管。

不过小工们的打磨,对炮管外壁的重视却远过于内壁。仿佛磨镜一般,将青铜的炮管磨得光可鉴人。

徐良走到近前,便清晰地从炮管上看见了自己扭曲了的影像。

炮管看起来打磨好了,炮架也早已经准备好,只等明天的安装,两天后进行试射。试射成功,便可交付出去。

这已经是第三门了,赶在新年之前,还要再造好第四门同样型号的重炮,交付给神机军。

眼看着第三门炮并没有耽搁时间,对于按时交付第四门炮,徐良现在终于可以向上拍胸脯保证了。

不过,这终究不是能送上战场的火炮。

时间过得很快,徐良从铸币局调来火器局已经半年了,但他原本所负有的使命,却始终未能完成。

两个月前的最后一次测试中,炮膛再次炸膛。想要将百斤重弹发射出去,填入炮膛内的火药,根本不是现在所造出的炮管能承受得起的。

一次次加厚炮管的管壁,可只要考虑到运输和安装的问题,炮管都会嫌太薄。只有不管不顾的加大重量,才能承受住火药的爆炸。

但现在的火炮已经超过了万斤,再重下去,还怎么用?什么炮车能撑得住?

“还是这里暖和。”

一人搓着手,哆嗦着走了进来。

徐良回头看了一下,是火器局中几名作头之一的臧寅。

这是监丞臧樟的儿子,与他管着斩马刀局的兄长不同,是个碎嘴爱说话的,有时候也不是太注意尊卑,不过从小被他父亲用鞭子抽出来的一身铸造本事,在火器局中如鱼得水。

臧寅看见徐良,隔着老远就大声的打了一个招呼,“提举,来得这么早?”

徐良点了点头,没太理会。

臧寅早就习惯了,走到徐良身边,搓手跺脚,“还是这里暖和……提举何必天天过来看这玩意儿?都到了这一步,还能有什么变化?又不用装弹,装药量还减半,只要听个响就行了,还怕炸膛吗。”

徐良声音低沉,“太后、天子面前要用的。”

“也就是要好看。”臧寅一口道破,然后拍着胸脯说,“提举放心,等明天肯定跟镜子一般亮,拖去皇城里,包管闪得辽国国使不敢正眼看。”

徐良嗯了一声,臧寅插科打诨一番,让他的心情倒是好了一点。

辽人的正旦使的确很快就要到了。

两家快报上这两天有关辽国的报道多了起来。有关高丽和日本的消息,也同样的多。上个月,扬子江口的秀州来了一队日本僧侣,说是来向大宋求援的,到底要不要让他们入京,不经朝堂上在议论,报纸上也刊载了此事。

另外报纸上还说,辽国幼主近日身体有恙,恐有不测。继宣宗、章宗之后,再过半年,辽国的帝王世系表上就不知道又要添上什么宗了。

提及此事,臧寅嘿的一声冷笑,肆无忌惮的说道:“其实哪边还不都一样,都是几世的冤孽。”

徐良脸色木然,看着前面,当自己没听到。

辽国的太师算什么冤孽?那分明是权臣想篡位,觉得皇帝碍手碍脚就杀了了事。徐良读书不多,只能识得几百字,但瓦子里说三分的多了去了,司马懿一家怎么做的,好歹也知道一点。

这边倒的确是夙世冤孽,否则才六岁的孩童,连魂魄都不全,怎么能杀了自己的父皇?

弑君的权臣和弑父的皇帝,都是大逆不道弑父的罪行也许更重一点,意外误杀要低一级,却也不会下于臣子弑君。

像这样的误杀,世间公认是前世的冤孽。至于是什么冤孽,那就众说纷纭了,不过没人敢放在公开场合议论。

但有一件事可以确认,现在这位皇帝还能在位,正是王平章、韩参政这样深受先帝重恩的重臣,拼却一死也要报先帝恩德的结果。

“不管辽国皇帝怎么改,这一回打下了高丽、日本,辽国派来的使者还不知怎么得意呢。”徐良不答腔,臧寅一个人说得口沫横飞,“不过他们也得意不了,我们这边,才一个安西都护府,就杀败了黑汗的十万大军,斩首成千上万。”

“嗯。”徐良难得又有了一次反应。

王舜臣在西域的胜利,京城内外,无论官民,都觉得与有荣焉。

辽国刚刚打下了高丽和日本,收获了上千万的人口,拓张了数千里的土地,抢到的金银财货不计其数,正是志得意满的时候。

日本和高丽的上层听说都被杀光了,而高丽的农民,则成为了契丹贵族的部众。至于日本,尽管没有太多的资源,但终究比高丽要大上许多,户口也是高丽的数倍。

可大宋这边,虽说一整年的时间都在休养生息,但西域那边,还是与黑汗人打了一仗,功业决不输给辽人想那高丽和日本,加起来也比不上西域幅员广大。

整个秋天,京城内外都在为西域的胜利而欢呼鼓舞。

黑汗是西方大国,带甲数十万,在得知疏勒要地被攻取之后,一等雪化,便立刻举兵东来。

十多万大军兵分两路,一从北路直取北廷、高昌,一走南路,试图收复疏勒。另外还有一部分来自于黑汗北方突厥各部的援军。

王舜臣手中经过补充之后,也只有六千多官军。而天山南北,听从他号令的蕃军加起来也不过三万余人。唯一算得上是幸运的,就是由于黑汗国中,征发了大量的部族兵,让安西都护府早一步得到了情报,也让王舜臣得以针对性的作出布置。

在留下两千汉军镇守疏勒之后,王舜臣便举兵北上。没有退守仰吉八里、北廷甚至高昌,而是直取黑汗军北线东征的必经之路伊丽河谷。这里本是黑汗国的领土,对王舜臣的长途奔袭,黑汗军全无所备,数万大军还没有踏出国境,便惨败在安西都护王舜臣亲领的汉番联军手中。这一场千里奔袭的大捷,阵斩了黑汗国又一名自称狮子王的大汗,斩首总计超过万人。

而留守疏勒的两千汉军,加上了一万多来自北方的高昌兵,则采取坚壁清野的战术,迁走了所有居民,拿走了所有存粮,毁去了疏勒地区除疏勒城之外所有的城镇,然后退守经过加固增修的疏勒城中,硬是顶了一个月之久。

不论黑汗人动用了云梯,还是挖掘地道,都没有攻下疏勒城,反而为城头上无穷无尽的箭矢射得伤亡惨重。而黑汗人辛辛苦苦造好的投石车,却对抗不了床子弩射出的铁枪。

就这样一直拖到了黑汗人的后勤支持不住,而主力惨败的消息传来为止。王舜臣在伊丽河谷击败了黑汗军主力,斩杀了黑汗大汗,随即放纵蕃军抄掠伊丽河谷,自己则领三千余汉军返身南下,还想在疏勒城下来一次奔袭,可惜大军行进的速度终究赶不过信使日以继夜的奔驰,当他赶到疏勒城下,黑汗军已经撤走了六天了。

经此一役,黑汗国短时间内已经没有了再次东侵的能力。其国中本就分裂,东黑汗如今元气大伤,接下来必须要小心应付西黑汗的进攻,没空收复失土。

报纸上没有说这么详细,像那种屠杀、劫掠、坚壁清野之类不宜宣扬的手段,都没有登出来,但身在与军事关联紧密的衙门中,徐良就算不想听,消息都会往耳朵里钻。

但西域的胜利,实在太过遥远了,想要吓到辽人,根本毫无作用。

徐良很清楚,不论西域那边有多少战果,朝廷的目光始终放在北方。当辽国使臣南来的时候,朝廷上下所想看到的,就是他对大宋望而生畏的表情。否则就不会费尽心思的催促军器监早一步将这种外面光鲜、只能吓唬人的东西给造出来。只不过,现在想吓人都不一定能吓得了人。

在京师混迹久了,徐良可不觉得这火炮只是样子货的消息能瞒住多少人,尤其这边有不少还是什么都敢说的大嘴巴,向旁边的臧寅瞥了一眼,徐良摇了摇头,上面的那些相公们,实在是太久不食人间烟火了。
