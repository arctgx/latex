\section{第二章 天危欲倾何敬恭(十)}

“西军裁撤?”蔡确扬了扬眉,这是不是图穷匕见?很平静的问:“难道玉昆还有什么新想法?”

韩冈摇头:“之前朝廷已经议定了,韩冈也参与其中,怎么可能还有什么想法?哪些军队该裁撤,哪些不该裁撤,自有两府主持。不过之前王厚入京,倒是提起关西有些传言,致使军中人心不定。只希望朝廷及早公布此事,以安军心。尤其是被裁撤的各个指挥,指挥使和都头们的安置办法,必须尽快公开,以防生变。”

去年大战,宋辽两国都伤了元气,大宋这边躲进窝里养伤,辽国就是将目标转移到东面去找补回来,表面上的和平局面,看起来能再撑上几年。可西夏既去,辽国便是唯一的大敌,不可能不加强防备。

随着战略重心的东移,西北军费要削减,连军队数量也会削减。王厚这一次上京,对韩冈将这件事提过几次,希望韩冈能够让朝廷克制一点,不要伤了西军的元气。

对于西军,如旧日一般投入绝对不可能,没有敌人了,鸟尽弓藏是应有之理,但马放南山也是不可能的。河北军拢总才几十年功夫就变得那副烂样,这个前车之鉴,让朝廷上下都引以为戒。纵然西北暂时不会有外患,可辽国的威胁还在,维持一支能征惯战的西军,是朝廷的共识。

朝廷的处置办法,是给退下来的士兵分地。旧有的寨堡,如果周围有可耕之地,便就地安置,如果没有,或是数量不足,便迁移到其他地方。新辟之田,三年内不收税赋,旧时军屯田地,则是免去一年税赋。这是省钱省力的好办法,最关键的还是省钱。

朝廷的目标是将西军的数量压缩到二十万人。包括禁军、蕃军和极小部分的厢军。一般来说,是尽量保留上位军额的禁军,遣散的重点放在下位军额的营头上。能赚钱的厢军留下,不能赚钱的则清理掉。

相较而言,以战斗力而论,一般都是上位军额的禁军更强一点。尽管真要比起军额高下,没有谁能比得了京师的上四军,可若是打起来,上四军又有几人敢在西军的将校面前吹嘘?不过在西军中内部比较,情况则大体如此。许多下位军额的士兵,都是从上位军额的营头中被刷下来的淘汰者,以老弱居多,参加过的战事也不多,更不会成为主力。

但最终到底要留下谁,则还是要讨论过历史传承和过往战功,挑选参加过几次大战,战功卓著的指挥。蕃军中,也要留下一部分有历史有战绩的,用来钳制蕃人,以夷制夷。

不过以过去几次裁军的经验,裁军最后的问题都不在士兵身上,而是军官。有品级的军官好说,朝廷直接养起来,相比起十余万大军的消耗来,几百名将校的俸禄,连零头都算不上。但冇没品级的呢?从都头到指挥使,这一等级的军官,是军中的骨干,却又因为未入流品,而不受重视。西军兵力拦腰砍去,若朝廷弃之不顾,多少军官都会没了身冇份。

朝廷自不会那么做,底层军官都有着极大的危险性。蛇无头不行,五代的故事人人熟知,指挥使、都头这个等级的军官,与下面的卒伍更近,比高层的将领更容易带着士兵起来造反。必须好生的安排他们,免得心生怨怼,而且还得将他们与手下的士兵给分开,另外择地安置。

“哦,当真有此事?那可不能等,得赶紧公诸于众。”蔡确点头,等着韩冈的下文,他可不信韩冈就这么点事。

果不其然,他就听韩冈说道。

“另外还有另一件事,韩冈想要得相公应允。”

“就是朝廷既然是以过往战绩来衡量各军高下,那么留下来的各个指挥必然是军功赫赫的队伍。”

“自然。”

“所以韩冈就想,是不是在这些指挥的营房冇中,专门辟出一间房间,陈列过去缴获敌军旗帜、兵器,再将过往战绩列于墙上,以及所受到的封赏,用以激励士卒,也让其忠于王事。”

蔡确迷糊了一点,“这是为何?”

“有人方有国,有国便有史,有史方能聚人心。李唐追尊李耳,如元昊这等蛮夷,立国时也得攀个好祖宗。”

而本朝的真宗皇帝,则弄出个曾转世投胎为轩辕黄帝的圣祖赵玄朗,都是一个道理。不过这一条不好拿来当例子。

看了一下蔡确的反应,韩冈继续说道,“一支有着传承和功绩的军队,要远远超过没有底蕴的营头。而一名新人入行伍,在老营头和新营头待上同样的时间,出来后也绝对不一样。大胜之后,夸功耀武又是为何?一为奖誉,一为激励。奖誉者,有功之辈;激励者,便是后人了。如果在营中陈列过往功绩,新兵入伍,让指挥使亲自领着他们讲授军史,又岂能不受激励?”

“有教无类?”

韩冈在蔡确疑惑中点头笑道:“这也算是教化了。”

韩冈一直都在鼓吹着教化,有教无类都做到了赤佬的头上。

可回想一下去年京营为了赏钱到底变了一副什么模样。这激励之功不假,但也是要看人的,效果最多也就一时,真想要赤佬们卖命,还是得用犒赏,而不是所谓的教化。

“此事玉昆说得有理,可以考虑一下。不过还要与西府商量着办。玉昆当是已经与子厚说过了吧?”

“还没有,也是才有了点想法。”

“是吗?……不过这个想法的确不错。”

韩冈谢了一声,又微带苦涩的笑了起来:“这么做,究竟有几分作用也说不准,但总不能眼睁睁的看着西军变成河北军的那副模样。西南夷若是有变,还有用得到他们的地方。”

真的是这个原因?蔡确懒得猜了。韩冈虽然郑重其事,但说起来也不过是些琐碎小事,与国家大事无关。让蔡确白白期待了,不过答应他也没什么关系。

上一回西南夷起事,便是王中正率西军给平定的,以后若是蜀地生乱,朝廷会动用的兵马当然还会是熟门熟路的西军。若是日后北方有变,西军纵然离得远,也照样有派上用场的地方。

“西军此番要裁撤近四成。日常开支会减少四分之一,”下位军额的禁军和厢军,粮饷数量是不能与上位军额相比的,“加上省去了战时开支,今年政事堂手上的盈余,恐怕是过去三五年加起来都赶不上。”

“要用的地方也不少,”蔡确很快的接上去,“还要多谢玉昆你,光是邮政局一项,就是上百万贯要花出去啊。”

“也只是一时开销大,而且还是分几年投入。等邮递所遍布乡里,天下邮件递送皆从此中来,日后肯是赚得更多。”

几年总投入才百万贯,在韩冈看来,一点都不多,甚至觉得太少了。在渠道上投入的资金越多,整个网络成型的就越快,邮政业务的影响力就会越大,作为倡议者的韩冈从中自然能得益不少。他可是盼着邮政能越早成型越好。

随着天下邮政系统的铺开,京畿、江南、荆湖、河北、河东、关中、陇西以及蜀中,天下各大区域的州县都投入了大量人力去建设。并不是重整户籍,仅冇仅确定门户,将邮政驿传从官用军用,转为民用,所以开支真要计较起来,也并算不大,区区百万贯,只是一时的投入而已。

开封的建设速度最快,从外围的诸多畿县到京冇城内外的街巷,全都纳入了邮政体系。而下面的乡镇,到了开春也就能够成型。那时候,就已经可以开始赚钱了。订阅的图书、期刊,与邮件递送,都是能够赚钱的营生。在只能通过信件来传递消息的时代,拥有一套邮政驿传的网络,不可能不赚钱。能够贴补朝廷驿站体系的亏空。

“照玉昆你说的意思,那两家报社应该没少赚吧?”蔡确抿了口冷了一点下来的茶水,笑问道。

“听说的确是多卖了一些。”韩冈想了想,“不过两家报社的根子都是在联赛上,京冇城里面的比赛,外县也看不到,县中的比赛,报上又不会登。想来多也多不了多少。”

“聚沙成塔,集腋成裘。开封府这么大,若是下面的每个村子都能买上几份报纸,合起来就数以万计了。一份赚上几文钱,虽然不多,一期数十贯,一年下来也够吓人了。”蔡确啧啧叹着。

邮政局的业务情况,哪里能瞒得过他这位宰相。既然韩冈之前提议时,信誓旦旦的说帮人送信能赚钱,自然早就在他那边挂上号了。

仅仅是多了外围县镇,可是以开封府辖下的县镇数量,两大报社的订阅量说顿时上了一个台阶都显得保守,而是向上翻了个跟头。核心影响力也从京冇城周围,散布到整个开封府中。从这个势头看,下面乡村里的富户,对订阅报纸也会十分踊跃。

