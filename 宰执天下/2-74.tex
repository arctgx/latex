\section{第17章 家事可断百事轻(中)}

【祝各位书友情人节快乐。求红票,收藏。】

秦凤路缘边安抚司的设立,以及王韶、高遵裕所得到的新差遣,让秦州官场上的风向更加偏往开拓河湟一边。天子和朝廷用着再明显不过的态度表示了对王韶的支持,即便再没有眼色的官员,也知道现在不是跟王韶他们过不去的时候。

除了大获全胜的王韶、高遵裕,竭心尽力的韩冈理所当然也是一个赢家。管勾缘边安抚司机宜等事,比起勾当公事肯定是高上一级,而且可以名正言顺的参与开边事务,而不是盯着勾当公事的职衔,做着不该属于自己的事情。

另外依照高遵裕的说法,如果拓边河湟进展顺利,将河州等地收归朝廷,古渭不但可以升军改州,连以古渭为核心,在秦凤路以西再设立一个经略安抚使路都是有可能的。

秦凤、鄜延、环庆、泾原这边境四路,地盘都不大,这是为了方便对路中军队进行指挥调度,敌军来袭时,也能及时作出应对。而在这四路中,秦凤路的辖区是最大的一个,秦州城距离渭源堡已经超过三百里,再向西扩张,就很难对边境军情作出适时恰当的处置,必然要将之分割。

若是高遵裕所言成真,那么等新路设立,韩冈若那时还在河湟之地,王韶在秦凤担任过的职位,也逃不出他的手掌心。

韩冈身份的变化,使得尚未定亲的他更加炙手可热。每日韩家的门槛几乎踏破,都是上门来做媒。不过当王韶亲自登门后,这些事也就无影无踪了——在忙碌了许久之后,王韶终于想起了韩冈的终身大事。他算是媒人,将他原配的侄女许给了韩冈,这件亲事一成,韩冈跟王家就成了姻亲。

在婚姻大事之中,韩冈是当事人,但纳彩,征期等婚前礼节之事,完全由王韶这个媒人负责,韩冈一切不问。连他未来的夫人唤作什么名字都不清楚,现在也只知道在杨家排二十六,来往书信都只说二十六娘——按照礼制,女方的闺名向不外露,只有小名和排行让人称呼。也只有问名之后,交换了婚贴,才会知道到底叫什么。

个人的事,韩冈很快就放到一边。他现在白天跟着王韶一起做着安抚司的筹备工作,有些忙碌,不过回到家中,有严肃心曲意奉承,夜里则过得舒心畅意。

打仗拼得是兵钱粮三项。钱粮一事,王韶在担任缘边安抚使之后,手上少不了会有专门的拨款,而不是像过去那样,事事都要跟经略司打饥荒。剩下的兵,在王韶接下来统领的辖区中,有着五六千汉军,而他能动用的蕃军更是一倍有余。只是指挥兵卒的将领,却让人颇费思量。

王韶和韩冈都是文官,指挥经验虽然各自或多或少都有一点,但他们不可能直接领军上阵。而高遵裕虽为武职,但实际上也是不可能提弓跨刀出阵。他们需要一个能上阵杀敌的古渭寨主,能代替得了刚刚升任秦凤兵马都监的刘昌祚。

“刘昌祚在古渭节制得当,让士卒能效死命,他这一走,古渭寨的事就让人头疼了,”王韶还没有搬离机宜官厅,镇日都在做着最后的筹备工作,他对高遵裕和韩冈说道:“秦凤路中,能在资历和能力这两项上与他相提并论的,屈指可数。”

韩冈附和着王韶的想法,“有能力的就那么几个,哪个都掉不过来。这边赵隆、王舜臣能力不差,就是年轻一点,担任寨主也不够资格。这事的确不好办!”

刘昌祚作为西路都巡检,镇守在古渭寨,有着不短的时间。如今他跳过排在他前面的一众秦凤路的将领,接任张守约的兵马都监一职,他接下来的镇守地,不会是古渭寨,而将是甘谷城。少了刘昌祚这名悍将,古渭寨驻军的战斗力免不了要大打折扣。

高遵裕则从文案中抬起头来,道:“刘昌祚才能虽不差,可关西这么大,本路找不到能替代他的,外路难道没有。鄜延正要谋取横山就不说了,环庆的苗授、刘舜卿,泾原的姚兕、姚麟,哪个也不输他。”

韩冈总觉得高遵裕对刘昌祚好像有些反感,不知这是不是自己的错觉。但从情理上说,高遵裕的确有不喜刘昌祚的理由。作为西路都巡检兼古渭寨主,刘昌祚早前对河湟开边之事支持得太少,除了攻打托硕部时,他暗中帮着王韶来回联络各家蕃部,让王韶一战得胜,但高遵裕来秦州之后,他却完全没有亲附的意思。

而且如果没有刘昌祚的话,以高遵裕的閣门通事舍人的本官,接手都监一职是绰绰有余,就是担任钤辖都是够资格的。钤辖,但刘昌祚占去了兵马都监一职,让高遵裕看不顺眼也不足为奇。

说起来,依照编制,一经略安抚使路,应有都总管、副都总管各一人,钤辖二人,都监四人,但这是全路的高级将领数目。秦凤路共有五州一军,治所位于秦州内的钤辖和都监,如今都只有一个编制。高遵裕想要一个能名正言顺在秦州领军的差遣,也就两个位置可以争。

韩冈不知高遵裕是不是因为没能从刘昌祚嘴里将都监的肥肉抢下来,但他对接任古渭寨主的人选,也有自己的想法:“从外路调人总不如自己身边熟悉的。不知傅勍此人如何?”

“傅勍?”王厚登时叫起:“那个醉鬼?”

王韶和高遵裕也不禁摇头,虽然傅勍在前面对付窦舜卿时曾经帮了个大忙,但他酗酒的毛病不改,谁也不敢用他。

“安抚使司安在古渭,傅勍只是带兵而已。他早年曾与刘昌祚并称,只是好酒误事,才久不迁调。现在有两位安抚在旁盯着,谅他也喝不出事来。傅勍在秦凤年久,人头熟,故事也熟,未必没有用处。而且他认真办事自然最好,但如果不理事,其他人也就有机会多历练一下了。”韩冈向外瞥了一眼,若是傅勍天天醉酒,王舜臣、赵隆他们就有机会趁势而起,多了许多历练的机会。

“傅勍还是小使臣吧?”高遵裕想了想又说道。

“以傅勍现在的官职,担任古渭寨寨主的确有些勉强,但他的资历足够了,加个权字就可以,权知古渭寨。”韩冈力挺着傅勍,他看得出来高遵裕已经动了心。

王韶他对傅勍实在不看好,不过韩冈的说法也有几分道理,心中犹豫着,一时难以决断:“这事再考虑一下,不用急着下决断,还有一点时间。”

王韶既然不想就此决定,韩冈自是不便再说,换过话题,他问道:“既然缘边安抚司已经设立,屯田市易的事就不能再拖了。不知给缘边安抚司的钱粮什么时候能给拨下来?”

“十天半个月内就该有消息了。”王韶屈指算了一下,“六月夏税已经在收,便民贷款的利息也在收着,转运司手上有钱,不至于拖延时间。”

高遵裕丢下手中帐册,靠上交椅的椅背:“韩子华在京兆日日观兵,又提拔种谔掌事。眼见着最近就要继续向横山深处攻去,天子和政事堂的心思接下来也许就都要放在鄜延那一边了。”

“横山再紧要也不能夺占河湟的钱粮,天子都在看着,转运司当是不敢拖欠我们的帐。”韩冈说着,“不过两百份度牒到现在都还砸在手里,我们得给招募来屯田的弓箭手发耕牛、种粮,这些度牒不换成钱粮,根本排不上用场。”

王厚被韩冈一句话点心头火起,这些废纸还是他带回来的。他发作道:“真想把这些破纸抵给质库去,换回的钱钞说不定还比卖的多上一点!”

王韶、高遵裕摇头失笑,做和尚的把自己的度牒压给质库,这事时有发生,可哪家质库也不可能一下吃下三百份度牒,就是让几家质库联手吞下都不会干,三百份这个数量会让他们把本都亏光的。但韩冈眼睛一亮,王厚的气话提醒了他,“不知能不能先用度牒在州里做抵押,换个五六万贯,等有了钱了再赎回来。”

“州里怎么可能同意?”高遵裕道。

“请中书下堂扎如何?反正秦州的常平仓里钱粮充足,便民贷款也只散出去一半,用度牒做抵押暂借一部分,再加上转运司拨下来的数,足以撑过今年了。就算州中不同意,也可以在转运司作抵押。只要走王相公的路,十一二天之内应该就能又回复,应该能赶在郭太尉之前。”

王韶、高遵裕沉吟起来,而这时,一人自外匆匆走进院中,韩冈看过去,却是张守约身边的人。那人在门外通名后进来,对韩冈道:“钤辖请韩机宜过去一趟,说是凤翔府那边出事了。”

韩冈一听,脸色瞬变,肯定是李信出了事。他连忙跟王韶、高遵裕告了罪,几句话说明了情况,跟着来人去见张守约。

“玉昆,你家表兄在凤翔出了事。”甫一见面,张守约就开门见山的说道。

“究竟出了何事?!”韩冈阴声问着。

张守约回头看了下身边的一个军汉,那人上前一步,对韩冈道:“李二哥被关进凤翔府大狱里去了。”

