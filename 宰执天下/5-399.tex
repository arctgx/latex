\section{第36章 沧浪歌罢濯尘缨(一)}

福宁殿中细语声声。

细密的窗格中嵌着一片片半尺见方的透明玻璃。

阳光透过玻璃窗,毫无遮挡的照在了斜倚着躺椅的皇帝的身上。

赵顼闭着眼睛,感受着阳光晒在身上的温暖。自从寝殿的门窗都开始把糊窗的绢纸换成玻璃后,赵顼最喜欢的就是在阳光下晒着太阳。

就在他的身边,如今宫中当红的小黄门杨戬,正为皇帝念着一封封送过来的奏章。

“江淮六路发运司言:黄河今春水枯,汴水宿州北水深不及四尺,纲船多为搁浅,滩涂之上已达千余,仓中百万纲粮难以发运。故申达中书,今岁两浙、江东、江西三路夏税,请转自襄汉发运上京。”

这是一份很紧急的奏章。

百万石纲粮堆在宿州,而如今离夏税却只剩一个月。黄河涨水一般也得到夏汛开始的五月。待五月夏税开始上京,必然还是会被堵在宿州。如此一来,江淮六路征收上来的夏税,数以百万计的钱、绢、布匹——粮食倒不会多,南方种麦子的很少——也就无法及时运上京城。亏空的财计可等不起,靠着大江的三路,分流出去是必然。

不过皇帝看起来没什么兴趣,闭着的眼睑完全没有睁开。

杨戬仔细观察着赵顼的反应,见他不睁眼,便把这一份奏章放下。他读也只读贴黄的部分,除非赵顼想听详细内容,不然一封奏章听过去也就几句话的功夫。

“知相州满中行言:‘林虑县前修合涧河水,以济民用,三年修得渠道仅十四里。今孟儿等村凿井取水,深两百七十尺及泉,水可自流出井。渠道无所用,徒耗民力,乞罢之。’圣人和两府已准其请。”

这是一封来自河北的奏章,不过只是政事,而且是一桩提不上台面的小事。

所有呈于天子的奏报,都是经过仔细挑选。就像现在,河北边境虽然因为休兵议和,大规模的战事已经停止,可并不代表河北已然平靖,但皇后和两府还是尽可能的挑拣出那些述说地方政事的奏章,避免提及军事,防止天子从中窥破之前的谎言。故而这等鸡毛蒜皮的小事,也一并送到了福宁殿来。

杨戬也不知赵顼对此是不是不耐烦,反正皇帝的眼皮没有动静,便又放到一边。

“凉州知州、甘凉路经略安抚使游师雄言:本路钤辖王舜臣,已于二月十五离开伊州[今哈密],帅师七千进兵高昌[吐鲁番]。其中汉军一千三百,蕃兵五千五,马驼总计两万。”

天子重病,太子年幼,朝廷在军事上必然要变得保守起来。杨戬的见识尽管还不足以让他有一个明确的认识,但多多少少也能感觉到一点。

现在王舜臣还在西域开疆拓土,只是之前留下来的惯性罢了。这是中国重新掌控西域的关键一战,王舜臣要是败了,在皇太子成人之前,对西域多半不会有新的行动了。

不过皇帝的眼皮依然阖着。杨戬心中狐疑,他从小生长在宫中,虽没接触过健康时的皇帝,可这位天子的性格还是听说过许多。无论是哪里的军事,都不可能不多问一句。

尤其是现在的这一份奏章,说的可是在直追汉唐的丰功伟绩,只是天子还是没反应。

是不是睡着了?

杨戬低头看着天子,赵顼突然睁开了眼,惊了杨戬一跳。

“官家,是不是累了?”他立刻问道。

赵顼眨了一下眼,否定。

“继续念吗?”

两下。

杨戬又拿了奏章,不过只又念了两三份,便被打断了。

皇后和两府宰臣结束了崇政殿的每日公事,过来探视赵顼的病情。

这样的探视已经成了例行公事,问候过天子,确定了赵顼的身体没有恶化也没有改善,宰辅们便在王安石的率领下行礼告退。

“平章,请留步。”

赵顼眨着眼睛,通过杨戬,出言留下了王安石。

王安石脚步停了,余光瞥了眼走在最后的韩绛、蔡确,然后立定弓腰:“臣遵旨。”

不比之前与同僚共同拜见天子,王安石君前独对时,得到一张小园凳,离赵顼也更近了许多。

虽然赵顼现在的情况肯定不能与未发病之前相比,但比起王安石过去见识过的瘫痪的中风患者,无论从气色还是从身体状况上来说,都是要强出许多。

王安石知道,这其中不仅仅是皇帝得到的照顾无微不至,也有自家女婿的功劳。

勤翻身,勤擦洗,然后让人帮助活动肢体,以防四肢萎缩。这是韩冈留下来的医嘱。不施针药,却比贵重的药物都管用。

理所当然,韩冈给赵顼的医嘱也流传了出去,成了世间照顾瘫痪病人的标准。医疗护理已经成了医学方面的一个大课题,甚至在如今的太医局都有专门设立一门护理科的想法。

排开脑中的胡思乱想,打叠起精神,王安石等着赵顼的发话。

皇帝只能通过眨眼来传话,故而问题都很简短,一字、两字、三字而已。当王安石听到杨戬翻着韵书,念出“平章辛苦”四个字,就楞了一下。

“不敢。为君分忧,岂能称苦?”王安石等着赵顼的下文。

“河……东……”

王安石迟疑了一下,然后熟极而流的念着:“河东有韩冈镇守,辽军不得其门而入,陛下无需挂念河东。河北战局平稳,也是多亏了从河东派去的两万兵马。如今西北大胜,河东稳定,而河北有名将强兵,辽军也无从南下。这一仗,当不会再有反复。”

奏报给赵顼的军情,大半是假,小半是真。弄到现在,很多时候不得不为了圆谎而撒更大的谎。

王安石并不是有洁癖的人,当年主持变法时,欺上瞒下的事也没少做。可是现在,对着重病的皇帝公然撒谎,还是忍不住老脸微红。幸好面黑,看不出来。

“令……婿……为……国……”

“韩冈在河东,不过镇抚而已,比不得吕惠卿、郭逵的功劳。”

王安石斟词酌句,生怕一不小心就会说漏口。

这位皇帝本来就是极聪明的一个人,一句疏忽,说不定就能让他想通一切,找到真相。要不是性格问题,必然会是一名留名青史的明君。

只是可惜得很,当年割让河东北界的土地,被士林宣扬成割地七百里,纵然灭交趾、灭西夏、开拓河湟,武功远胜先代,可在天下人的心目中,依然被辽人压上一头。而这一回辽军入寇却被迫乞和,功劳却会被算在皇后头上。

话说回来,商纣不仅有扛鼎之力,也是绝顶聪明,只可惜没用到正道上。辩足以饰非,材足以拒谏。故而众叛亲离,身死国灭。太过聪明的人,很难成为一个好皇帝,即便是唐太宗,到了晚年也差点英名尽丧——幸好死得早,而不是像其曾孙明皇一般活到了七十多。

现在的这一位,心思用在臣子身上太多了,却忘了真正应该去关注的对象。即便没有发病,再过些年说不定会变得让人不敢相信是原来的皇帝。

王安石正在捉摸着赵顼到底想说什么。

方才进来时,他身边的小黄门可是正在给皇帝念着奏章上的贴黄。

在过去,所谓的贴黄,只是在用白纸书写的奏疏、札子背后,大臣如意有未尽,以黄纸摘要另写,附于正文之后。不过这段时间以来,为了方便起见,变成了对全篇的总结归纳。

臣子进奏的章疏往往字多语繁,在后面贴上一张小黄纸片,作为内容的简介,对帮助卧病的赵顼了解朝政,作用远大于几名心腹。

原本一天只能听上十数本,而如今却是一下能将所有的上百份奏章都听上一遍。

这一点的改变是从陕西宣抚司开始。看着虽不是什么大变动,可能只是体贴皇后和皇帝而已,但以王安石对自己的学生和助手的了解,吕惠卿的想法绝不会是那么简单。

西府有一半在外面,还没回来就开始勾心斗角,王安石也只想叹气,不过他并没漏听杨戬转述的话,

“云……中……空……虚……”

西京大同府?皇帝的心思怎么跳到了那里:“辽人在西京犹有余力,大同府的守备也难以攻破。”

“西……军……”

西军和河东军合力?!

是的,王安石记得很清楚,皇帝到现在也不知道河东军除了麟府一部以外,代州、太原两部早就已经完了。韩冈现在苦苦支撑的依靠,还是京畿的京营禁军。

随着宋辽两军在代州城附近对峙日久,针对韩冈的保守,议论也越来越多。有人认为韩冈是尸位素餐,认为他的打算是将入寇的辽军‘礼送出境’,等辽贼自己离开,也有更多人觉得能把辽军逼得退往代州,就算是大成功了——辽军是主动撤回,这一个看法已经得到了所有人的认同。所以代州的形势,不论韩冈的奏表中怎么说,在京城之内,都认为辽贼之所以坚守代州,只是为了收缩兵力,并非无力进攻。一旦形势有变,积蓄的力量随时可以爆发出来。不要逼得辽贼拼命,以免坏了大好局面,这是朝廷中的共识。

以己之长,攻敌之短,这本就是兵法正道。韩冈以置制使坐镇河东,逼得耶律乙辛选择和谈,有他很大一部分功劳,不管怎么说,他麾下的大军也是反击入辽境,并占了一块战略要地下来。

但要说夺取大同,那根本不可能。

王安石组织着话语,想要打消赵顼的念头,但他随即就呆住了。

“复……幽……云……可……封……王……”

而半日之后,京师沸腾了起来,河东在代州城外大破辽军的捷报终于传来。

