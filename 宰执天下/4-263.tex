\section{第43章 庙堂垂衣天宇泰(十)}

京西转运司将成立卫生防疫局,先行在襄州推广种痘免疫法。在韩冈和黄庸议定之后,这个消息甚至就在当天便传扬开了。

成功劝说韩冈在襄州推广种痘之术,黄庸的名声因此一下就涨到了顶点,而敝帚自珍的韩冈,却也没人敢说他坏话。

在种痘之术出现后,韩冈身上的神秘色彩越加的浓厚,身为药王孙思邈孙真人的嫡传弟子,摆摆架子在世人眼中也是理所当然,哪家仙人不是如此?仙人的弟子好歹也能算半个。在许多人看来,没让黄知州三顾茅庐,已经是韩冈纡尊降贵了。

在无数人的期盼下,卫生防疫局很快就成立了。打出来的金字招牌,是在伏龙山中闯下了偌大名头的李德新李神医。而在卫生防疫局中奔走的椽属,无一例外都有在城中挂出牌子行医。

有了转运使和知州的联手推动,两个一向不对盘的衙门,表现出了难得一见的高效率,以及让人惊叹的娴熟配合。仅仅用了两天的时间,从人员到地点,以及编制、预算,全都一起搞定。

就在这两天中,韩冈本人已经签发几份公文,移文路中各州,通知他们卫生防疫局成立了,并知会成立的原因,更早的时候,他还给关西的苏昞写过信,提起种痘法。而在这之前韩冈也已经将种痘法和自己准备在京西做的事写在奏章里,连同讲述应对灾异的《肘后备要》,一并送进了京城。请求天子能给卫生防疫局一个正式的编制,并赠给李德新一官半职——殁于国事的铁面相公李士彬的儿子,其实本来就该有个荫补在身——以便于他管理卫生防疫局。

种痘的地点已经公诸于世,就在离着漕司衙门后门口不远出的观音院,包括种痘要遵守的流程和规矩,用榜文张贴在衙门外的八字墙以及四座城门处。

观音院是间不大的小庙,由于位置不错的缘故,香火一直很旺。不过观音院的地皮属于官产,加上庙中屋舍不少,就给卫生防疫局给征用了。连同主持在内,六名肥头大耳的和尚,还有十几个沙弥、两个火工道人,全都给赶到了另外几座寺庙里去挂单。

转运使和知州同时关注的要事,加之又是城中百姓人人期盼,一群秃驴当然连摇头都不敢,就在州衙命令下达的当天,就匆匆忙忙的收拾好金银细软搬到了相熟的寺庙中去。这般乖巧识趣,黄庸都跟韩冈说,有机会还是该还他们一间庙——不过这就不干韩冈的事了。

鞭炮声响遍了城中,空气中弥漫着一股子浓浓的硫磺味,韩冈和黄庸身着公服,带着全套仪仗,为卫生防疫局的开张捧场。路中、州中的官员几乎是一个不落的全数到场,许多人暗自庆幸,要不是从昨天开始就过来排队,正还不知要拖到什么时候。

城中百姓这些天早就为种痘之术而兴奋,甚至疯狂,鞭炮声刚停,看着韩冈和黄庸被引进去喝茶,身影刚刚消失,外面等候已久的人们拥作一团,向着大门挤过去。

“排队!排队!”安排在衙门里的小吏很尽责的留下来维持着秩序,不让正排着队的人们乱作一团。

房间中,听着外面的喧哗,韩冈和黄庸相识一笑,襄州百姓的热情,他们都已经感受到了。

笑声过后,黄庸就严肃起来,“关键还是痘苗的数量,到底够不够使用,襄州有近四十万人,亟需种痘的小儿,少说也有七八万,甚至十万。”

黄庸问的是关键,但韩冈早就安排好了,“常伯放心,这些事早就考虑过。不会耽搁到正事上。”

伏龙山远比襄州城要封闭,乡下的村子访客从来都不多,短时间内不用担心外来的干扰,可以放心的收回种痘之后生出的痘浆采集回来,作为疫苗重新利用。而且疫苗数量是增值的,相比起从广西带回来的那点痘苗,经过了伏龙山中的试种之后,掌握在韩冈手中的疫苗翻了十几倍。收集在一支支鹅毛管中的痘苗,足够襄州城中初行种痘的使用。

但现在在襄州城中,作为内陆即将兴起的一个交通枢纽,来往于城中的商旅行人已经到了让人咋舌的地步。城中的犯罪率和乞丐数量在短短半年内都上升了一个数量级,据说州衙中对此很有些怨言。尤其眼下的深秋初冬,是一年中商贸活动最繁忙的时候,州衙上下都忙得不可开交,黄庸前两天能抽出空来拜访漕司衙门,还是因为事关重大的缘故。

车水马龙的街道,熙熙攘攘的行人,出门后看到这一幕,就没人敢打起从种痘后的小儿那里收集痘浆的主意。谁也不敢保证,收回的是牛痘而不是人痘的痘浆。只要出了一点差错,极有可能就是一场爆发性的疫情,不但韩冈有麻烦,对种痘法也是一个十分严重的打击。

现在新成立的京西路卫生防疫局,主要是使用事前储存的干苗,另外则是利用养在慈幼局中的一干孤儿,以及官吏家中能够确实掌握的幼子,用他们来保证疫苗的供给。韩冈的几个子女,也在其中出了一份力。

“等到襄州城中处理完毕,卫生防疫局里面的医工,就会分出一部分到下面的县中去。”韩冈说着自己的计划,“不能让他们白白的干吃俸禄,总得让他们有些事做。”

黄庸连连点头,韩冈的想法于他不谋而合,先内后外,先从简单点的着手。

韩冈和黄庸正在后面说着闲话,而文三则是挤在人群中,鞋跟都被人踩落了下来,他连一声抱怨都没有。文三本人是不需要种痘的,但他的一对儿女可少不了,还要靠人帮忙呢。

‘痘苗一剂三十五文,一天只收治五百人,先紧着十岁以下、年满周岁的小儿。周岁以下,身体太小,不一定能撑得住,年纪大了,暂时不用着急。这是没病的预防生病,已经得了病,那就没办法了,生过痘疮的也不需要再多此一举。’

文三牢牢记着在八字墙下,被派出来向满城百姓宣讲种痘规矩的小吏所说的每一句话。一个字都不敢忘掉。

一大清早他就到了观音院,在卫生防疫局的匾额下,已经排出了长长地人龙。观音院的正门紧闭,两侧的小门打开了,挂号的走一扇门,种痘的则走另一扇门。

文三先排队挂号,挂号时要登记姓名、年龄和家庭住址,并确定种痘的时间,然后在规定的时间来种痘。文家家中有仆有婢,但这件事关系到儿女的安危,不亲自来办,文三哪里能放心得下?昨夜就没怎么睡,听到外面的鸡在叫,早早的就梳洗了一番出了门。

挂一个号就要先将钱缴了,而排一次队,就只能挂两个号,幸好自家只有一子一女,否则子女多些,还不知要排上几次队。

看着快要到自己了,文三往怀里掏了掏,出门时带的钱还好好的在原处。没有因为人挤人、人挨人的给挤丢了。钱掉了是小事,重新排队那可真是要人老命了。

从门中走出个小吏,拿了个用铜皮卷起的像是漏斗或是号角的东西,冲着后面的一群人放开胆量治愈,“后面的人听好了。十月廿七的五百号,你们的钱,由城西连大官人代付了!”

先冒出来的是个圆滚滚的肚皮,然后一个胖子就仰着脖子从门里出来。都是做买卖的,虽然小本经营的文家,没有连家垄断襄州半城绸缎布匹买卖的豪奢,但连家胖子还是认识的。

迎面的人群中就是一阵恭维和讨好声响起,连胖子出钱买好百姓,文三倒是沾了一点小便宜

“真亏连胖子想得出来。”排在文三前面的一个老头子啧着嘴,“不过几十贯的事,讨得百姓开心了,韩龙图和黄知州听说了也不可能不欢喜,难怪生意能越做越大。”

排在更前面的一个中年人回头笑道:“有连胖子起头,多半日后种痘,人人都可以免费了。”

“向庙里捐五百斤香油点长明灯,当真不如捐钱助人种痘,让五百人种痘,积下的阴德可比让和尚吃香油吃得油光满面要强得多。”

“这的确是阴德。种痘不过三十五文一次,就是给人打小工,三十五文也是转眼就能筹齐。即便是一千人,也不过三十五足贯,实在太便宜了。”

“相比起产生的好处,只付出这么一点代价,的确是连家赚了,还有个好名声做附带,怎么看都是个一本万利的买卖,要是能多来几次就好了。”

文三左右的人议论纷纷,的确是连家赚大了,日后若有人跟他过不去,半个襄州城都能站到他的背后为他撑腰。

当文三正在为连绸缎计算着未来的好处,后面的韩冈和黄庸,他们的议论也差不多结束了。

起身送了黄庸回来,韩冈找个舒服的位置坐下来,正准备好好想想接下来的行事,就见一名伴当匆匆而来,“龙图,唐州那里派人来了。”

