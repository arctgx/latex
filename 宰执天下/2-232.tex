\section{第34章 山云迢递若有闻(13)}

雪片纷纷洋洋的从天空中洒落,轻吐着白色的雾气,韩冈搓了搓手,抬头看着天空。

阴沉沉的天穹,是一望无垠的素寡的浅灰,死寂、空旷。只有一片片白色雪花覆盖起来的山野,给了暗色调的天地,增添了一些亮色。

这是这个冬天的第四场雪,在韩冈如今每天记录的日记中,他经历的前三场雪,都是细小的雪珠,下了半日便停下来。只有今天从晨起时便下起的雪,才算是第一场真正意义上对农情有用的降水。

这一场在十月初降下的大雪,对于农耕工作算是个不错的兆头。如果这场雪是个先导,后面的两三个月,继续有雪降下,明年的收获应当不会太差。

前两天,从陇西县传来的一个还算是不错的好消息,在他的父亲韩千六和一众主管屯田事务的官吏主持下,眼下的战事并没有影响到通远军今年的开垦和播种。但比起去年,只增加了一成的田亩数量,对于刚刚开始一年的开荒屯田的工作来说,不得不说是一个挫折。

但韩冈现在没空在意这些,从秦州来的第一批民伕,总计两千八百多人,已经在今天的晚些时候抵达了渭源堡。

又是在冬天接待民伕。去年在罗兀时,韩冈已经积累了不少管理经验,眼下他的手下又有不少能力出色的吏员,而在渭源的随军医院院中,还有十几个精于治疗冻伤和外伤的人才。这让韩冈处理起会让一般官员叫苦不迭的工作来,如同吃饭喝水一般轻松简单——有分教‘天子垂衣裳而天下治’,韩冈现在也是抱着胳膊就能把几千民伕都安顿好。

多了三千民伕,营寨之中,一下变得熙熙攘攘。这些从秦州各县被征发起来的壮丁,行走在寒冬腊月的风雪中,大部分人都已经被冻得瑟瑟发抖,走路时尚不觉得,可眼下一停下脚步,顿时都变得脸青唇白。如果这样受冻挨饿的情况持续下去,必然是接踵而至的一场传遍营中的大病。

还好韩冈早有准备。一切都事先有所规划,有条不紊的进行。

在营寨门口内侧的地方,他排出了一溜三十口大锅。锅下火焰正旺,而锅中水花翻腾。雾腾腾的热气向着周围散发着阵阵肉香。前几日的一场大战,韩冈手上多了不少伤马、死马,足有五六百匹。这些都是上好的精肉,在冬日又不易腐烂,不但让渭源堡的士兵能日日开荤,还连带着可让来到渭源的民伕们也享受不少。

韩冈从渭源堡中挑选出来的兵丁,向着这些从数百里外的家乡被征发而来的民伕,递上了一碗碗暖身用的热汤,还有一块块同样热气腾腾的炊饼。在提供给民伕们的饮食上,韩冈没有打上半点折扣。

奔波一日,他们都已经疲累不堪,几百里路连续赶下来,就算是铁人都开始吃不消了,人人肚饿身疲。他们在路上盯着风雪行进,只盼着到地头后,有口热水喝,发下的干粮能填饱肚子。孰料现在一进营中,便得到了远超他们想象的待遇。端起碗,闻着汤中的马肉香,掌心处传来暖透心头的热量,一阵发自肺腑的感激声,便从民伕们的队伍中传了出来。

喝过热汤,吃完炊饼,几千人便按照各自不同的队列,被引导到安排给他们的营地中。

民伕们的营地安置在营垒中一处背风的地方,临时搭起的屋舍却并不缺少遮风挡雪的作用。虽然因为韩冈手中的柴草和煤炭不足,没法给他们生火取暖,但营中有足够多的提供给战马的干草料。厚厚的一层铺在通铺上,又是多人聚在一屋中,并不会太过寒冷。

如果在千年之后,韩冈的这番布置可算得上是虐待,没有哪家军队或是工厂,会如此对待士兵和工人。但在如今这个时代,已经是韩冈在条件允许的范围内,尽可能做出的优待了。

而民伕们显然也很满意。从韩冈私下里让人打听来的消息,这群民伕两天前经过陇西县城时所得到的待遇,与韩冈现在给他们的有天壤之别——这番回报,让韩冈对昨天才被逼着经过渭源、前往临洮的蔡曚,又多了不屑和厌憎。

如何安排民伕们的饮食和住宿,稳定他们的情绪,让他们不至于因为长途行军和水土不服,引起大规模的减员,还能保持水准以上的士气和足够的精力,完成他们亟需完成的工作。这一项看似简单的任务,其实并不比行军打仗容易一星半点,能做好的官员,至少都能得到一个能吏的评价。

而韩冈表现出来的水平,比起能吏可更上一筹。他的人望,使得民伕中人心安定,准备充分的饮食和住宿,让民伕们的精神面貌。而且暗中宣扬官军最近的战绩,化解民伕们心中的隐忧。本来在韩冈的计划中,还有一场足球比赛,给民伕的行军生活增添一点娱乐活动,只是因为今天的大雪而终止。

看看蔡曚主持陇西城的接待工作,在民伕心中了留下的恶名,再看看他韩冈在渭源堡准备的一切。如果拿蔡曚的治事手段与自己相比,韩冈都觉得这是一个莫大的侮辱了。

巡视过民伕的营地,收来一片感激声后,韩冈转到了随军医院之中。

尽管是临时性质的治疗场所,而且因为没有伤病调养的空间,并没有冠上疗养院之名,但这处营地,依然是渭源堡中位置和条件最为优良的一处。

在前几天的大战中,守城时靠着强弓硬弩和霹雳砲等军国利器,韩冈麾下没有多少伤亡,而广锐军将校们出去追击时,伤亡也不算大。只是换了王君万带队追捕余众,随行的蕃人们伤了不少。现在这些伤兵都在医院中被医治着,汉蕃两边加起来也有百十个之多,只是重伤员只有三分之一,其他的多是些皮肉轻伤,只是伤到了腿脚,不便行动而已。

这些个伤病精力充沛,躺在床上是闲极无聊,没事都是要找出事来。当韩冈进来的时候,他们这些伤员们正赌得热火朝天,呼幺喝六的不仅仅是汉人——两颗牛角骰子,就那么六个面,即便是蕃人也能数得清上面的点数。

几十个人围着一张桌子,被堵得严严实实的人群中,还能听到叮叮当当的骰子滚动的声音。蕃人和汉人,头挨着头,肩并着肩,紧张着盯着碗中不住翻滚的骰子。很有几个腿上绑着石膏绷带的,因为被挡在人群之外,还单脚蹦着,向里面张望。

这一个战地医院的院长施俞本,是当初跟着韩冈从秦州去甘谷城中三十民伕中的一人。与现在被调去了延州主持疗养院的朱中一样,都是靠了韩冈而改变了一生。

陪着韩冈走进来,见着伤兵们聚赌,施俞本脸色变得很是难看。用力咳嗽了一声,外围的几个伤兵闻声懒洋洋的回头,可一见到。“韩……韩机宜!”

这一声叫唤,如同捅了马蜂窝,一阵鸡飞狗跳。

韩冈看了看他们,一个个被吓得跪在地上,连同吐蕃蕃人都不例外。摇了摇头,笑叹了一声,“还不躺回去,好生养病!”

一众如蒙大赦,连忙上床躺着,桌上的钱钞都不要了。

韩冈对着脸色犹然铁青的施俞本笑道:“看起来不用担心他们的伤了。”

施俞本唯唯诺诺,领着韩冈进了内室。

韩冈来此并不是为了探视伤兵,而是来找住在院中的瞎吴叱。

今次一战,渭源堡斩获的蕃人首级数超过一千。虽然韩冈能确定,其中必然有不少当是从住在附近的部落中弄来的假货——因为最近两天已经有哨探回报,渭源堡附近三十里,有好几个小部落被灭了满门——但打个折扣,也有七八百是真货。

首领一死一擒,主要的战力又损失大半。从木征手上分出的两支部族,他们在河湟之地,可以说已经被除名了。王韶在临洮城都没有这么大的功劳,可韩冈作为随军转运,却能独占此功,不是没有人眼热,但他们也嫉妒不来。又不是韩冈从他们手上抢的,而是瞎吴叱和结吴延征自己送上门来。

瞎吴叱受伤不轻,被截了肢后,短时间内下不了床。而韩冈看他的模样,苍白的脸色如初,也没有起床的意思,兵败的打击对他的影响很大。

依照王韶的命令,韩冈需要说服瞎吴叱来对抗木征在武胜军的影响力。这个任务倒是容易得很。瞎吴叱在被俘之后,摆在面前只有两条路,一个是被斩首示众,一个则是在大宋做官领俸。

但前两天第一次见手术后的瞎吴叱的时候,他很快就昏睡了过去,韩冈等了两天,听到他已经有了足够的精力,才又来见他。

有了瞎吴叱,就可以对抗木征对武胜军的蕃部们的命令。吐蕃人敬重松赞干布的血脉,如今正听命木征,向禹臧花麻供给粮草。但如果两个赞普家系的向他们传达截然相反的命令,那他们的选择只会是对自己有利的一方。

——在宋人帐下享受与青唐部一样的丰裕生活,还是跟着木征,继续与宋人日夜交战,该如何选择,并不是一个难题。

韩冈第二次来见木征的弟弟,口气依然严厉,“瞎吴叱,何去何从,该有个决断了!”

瞎吴叱闭上了眼睛。过了一阵,他挣扎着坐起身,向韩冈低下头了,“机宜有命,小人哪敢不从……小人愿降。”

