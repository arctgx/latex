\section{第30章 狂潮渐起何可施(中)}

【好了,今年的事告一段落。今天拼命补更。这是第一更】

朝堂上一团乱象。

本以为将冯京弄下去之后,升任宰相的当是吕惠卿,最差也该是王珪这个老牌的参政——王珪虽然不亲附新党,但他不会违逆天子的心意,只要天子还要推行新法,他就不会加以反对,王珪做宰相,也是新党勉强可以接受的选择——这也是为了维持新法的稳定。

吕惠卿和章惇都以为天子也会如此想来,可谁知道竟然是吴充,而接替吴充位置的,更是吕公著。

吕惠卿不死心,毕竟现在的御史台还是偏向新党,如果有机会,得到将吴充或是吕公著钉死的铁证,也不是不能翻盘。但谁也说不准一台之长的邓润甫还能控制得住御史台多久?

毕竟宰相和枢密使都成了旧党中人,如果他们打算往御史台中掺沙子——这几乎是必然的,王安石也好,吕夷简、韩琦也好,大权在握的宰相没有不这么做的——那么事情就会变得很麻烦。

御史台中的监察御史们,纵使是弹劾重臣,行动亦各自独立,许多御史除了在礼节上对顶头上司表示尊敬外,根本就不理会御史中丞的号令——这也是天子所期待看到的,没人会希望拥有弹劾审讯之权的御史台,变成一个用同一腔调发声的权力机构——邓润甫对他下属的约束力其实很小,又等于无。

章惇也是纳闷,为何皇帝会任用旧党?不是想不出原因,而是可能性太多了,不知哪一条才对。所以章惇干脆就不想了。反正他现在只想暂时偃旗息鼓一阵,看看天子的葫芦里到底卖的什么药。

青苗、免役、市易诸多新法给国库带来的收益,与每年的夏秋二税比起来,已经是不小的比例了。如果想要废除新法,想想会给朝廷的财计造生多大的窟窿。

如果吴充敢在此事动手,章惇乐得看他自食其果,多余的钱是变不出来的,几年来接连不断的大灾让国库如同泻.了肚子,根本没攒下多少积蓄。巧妇难为无米之炊,再有本事的宰相也应付不了千万一级的窟窿,而吴充,在章惇看来也不过只能算得上中人之材罢了。

章惇不像吕惠卿那般不甘心,他的资历浅薄但地位稳固,枢密副使至少能做上两三年,也需要坐上两三年,并不指望能往上跳,而吕惠卿可就是想着能宰衡天下,并不甘心在吴充之下做事。

章惇从鼻子里冷哼了一声,也不知是朝着谁人而发。不过他很快就低下头去,看着手上的厚厚几份纸页。

这是他几个儿子的功课,虽然已经请了西席,家中的门客也有有才学的,但章惇都是习惯于隔上十天半个月就检查一遍。

章家是福建大族,又是书香门第,宰相出过、状元也出过,各房之间的竞争心理很强,要不然章惇也不会因为不愿意屈居自己的侄儿章衡之下,便放弃了嘉佑二年所考上的进士之位。

眼下自己已经是执政高官,而比自家还要年长十岁的状元族侄,现在则远远不及自己。只是他眼下算是赢了半步,但自己的儿子若不能考上进士,该丢脸的还是会丢脸,就算自己能做到宰相,在族中还是会被人暗地里耻笑。心高气傲的章惇哪里肯接受这样的耻辱。

对于儿子们的功课,章惇一向有着十二分的耐心,但也有着同样程度的苛刻。点头、摇头,接着又是摇头、点头,手上的笔,也不停的在原卷上加以批改。

只是这样挑剔中带着一分满意的神色,再看到新的一份卷子之后,立刻就化为数九寒天中冰结的黄河,在厚重的冰层之下,有着直欲爆发出来的滚滚激流。

“将章持给我找来!”章惇厉声对书房外喝着,立刻就有人应承了一声,脚步声转眼就跑得远了。

章惇的第二个儿子章持很快就被找了过来,来到门外的时候,脸色已经变得如同一张上好的澄心堂纸,脚步欲抬又止,就是不敢踏进去一步。

“还不给我进来!”章惇在房中一声断喝。章持不敢违抗,只得缩头弓腰,迈着细碎的步子挪进房间。

章持这样的,章惇看得更怒。一等儿子进房间,章惇便拿着那份卷子在他面前抖着,“这是你做的功课?!易、书两经的题目没一条答对的,这些天你到底在玩些什么?”

“孩儿……孩儿……”章持吞吞吐吐,舌头打结。

章惇脸冷着,“别以为我不知道你在玩什么。从你娘那里要钱去买水晶镜,要了几次,买了几块?!你是眼睛不好吗?”

章持坦白:“孩儿是在做显微镜。”

“显微镜?”章惇顾名思义,很快就想明白了实际的用处,“与韩玉昆的放大镜有什么区别?”

“比放大镜要强出百倍!就是将凹透镜和凸透镜交叠起来,用两节纸筒或是铜皮做的圆筒包住,做成的便是显微镜。就是细小如蚁虫,其身上的须腿眼口,都能看得一清二楚。就是两种透镜都要经过特别的挑选,不是随意取用便能派得上用场。”

章惇脸色好看了些,口气也不那么严厉,“这显微镜是谁的发明?你又是从何得知?”他不觉得这是韩冈的发明,否则在广西共事的两年,韩冈肯定会跟他说。

章持摇了摇头:“也不知是从谁那里流传出来的,反正现在京城中不少人都在自己造。孩儿是从国子监里学来的。”他偷眼看了章惇一下,低头道:“孩儿也是想着格物致知,所以才会去造这显微镜。”

听说章持不过是随波逐流,章惇已经缓和了一点的脸色,立刻又冷了起来。

“韩玉昆之才是天授,看得简简单单的东西,在他之前就没人归纳得出来。这样的学问,越是浅显处越是见真功。就像是介甫相公的诗文,看着平易,却没人能学得来。想要沉在里面钻研,等你六经皆通之后再说!”

被疾风暴雨的一番训斥,章持方才煞白着一张脸,从父亲的书房中退了出来。

“爹爹近日心情不好,哥哥你正好给撞上了。”章援不知是从哪里冒出来的,向着书房还没有关上的大门张望了一下,嘻嘻笑的说着。

“你也别想逃!”章持发狠着,“少不了三哥儿你一份。别以为你前两天偷偷去了城东车马行的蹴鞠场我不知道!今年的总决赛好看吧?!”

“好歹也能混过去。”章援不以为意,他的成绩比章持要好上一点。

但随即书房中又是一声吼,看到三儿子最近的功课,章惇的火气又上来了。

章援笑不出来了,而章持的脸上则多了有难同当的欣慰笑容,“兄弟,‘子其勉之’……”

这时候,一名家丁从外匆匆而来,进了书房,就听见里面传出章惇惊喜的声音,“玉昆终于抵京了?!”

“好险。”章援如蒙大赦,丢下不甘心的章持一溜烟的先跑了。

……………………

原本属于王安石的相府,已经被开封府所收回。王旁跟着王安石回了江宁,任职当地的粮料院。而借住在相府中的韩冈一家,自是不能跟着去江宁,也就搬了出来。

尽管又添了几个孩儿,服侍他们的婆子、使女和乳母也随之多雇佣了许多,不过王旖这位主母带着一家老小从相府中搬出来之后,还是住回了原来的院子。

家中的人口多了,旧时的宅院就显得过于狭窄,也不符合龙图阁学士的身份。不过韩冈的职位早就公诸于众。既然一家之主很快就要任职京西,全家便都可以跟过去,也就不在乎宅邸的狭小了。暂时挤一挤,倒还都能忍耐,也显得家中热闹些。

不过这一天的韩龙图府,却是热闹得近乎于喧闹,但周围的邻居只是派人打听了一下,也就明白了到底为什么会这样吵闹。

这个家的主人终于回来了。

早在昨日快入夜的时候,韩冈提前派回来的家丁终于将消息带回了家中。为了迎接韩冈的归来,韩府之中从上到下都是显得手忙脚乱。

王旖倒不愧是大妇的作派,管了几年的家,一个一个的吩咐着,从里到外的安排得一一当当。当一切消停下来,都快到了中午时分。

王旖歇了下来,忙了一个上午,也是累了。严素心去了后院的小灶,准备亲自为丈夫做几道拿手的小菜。而周南则看着几个孩子。

云娘已经做了母亲,但她的心性还是如同少女一般,心急的要去门外守着:“三哥哥也该到了吧?”

王旖则拉着她,“前面都已经派了人去城门口守着了,官人要是到了,就会先赶回来通报的。”

“可是……”云娘还是心急。

“云丫头,莫要让人看笑话。安心坐下来等着,官人不会耽搁的……”

话声未落,外院就一片声喊了起来:“回……回来了。龙图,回来了!”

