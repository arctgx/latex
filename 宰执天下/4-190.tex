\section{第31章 九重自是进退地(四)}

正旦大朝会上的一整套流程当然不会有任何新意,吴充作为唯一的宰相,统领整套仪式。

先是朝臣拜舞于庭,而后外国的使臣上殿——只不过比往年少了两家,多了一家。从去年开始就没了交趾,今年又以扰边为由,拒绝了西夏派出的使臣,多出来的是高丽,也是从去年开始,派遣使臣入贡。

紧接着就是颁大赦诏,冬至日的郊祀大赦之后,因为改元元丰的缘故,天子又颁布了一道赦令。

到了最后,便是天子赐宴,基本上也没得吃,群臣奉酒为天子、太皇太后和太后祝寿。总计大约四个时辰的样子,今年的例行公事宣告结束,群臣中没有差事在身的就可以回家了,但皇帝和一干宰辅还有得忙。

韩冈在正旦大朝会上,按部就班的做着他的龙套角色,不起眼、不醒目,也不犯错。也就是他所立足的位置,引来诸多羡慕的目光。

另外还有契丹的使臣,往他这边看得多了一点。韩冈也听说当今的大辽天子很喜欢坐着飞船上天去游览,不知道是不是这个缘故。

回到家中,王旖这位北海郡君已经按品大妆,去宫中觐见太皇太后了。

王安石的女儿在宫中不受待见,无论是曹太皇还是高太后,都是不喜欢王安石,更别说一干宗室,更是将王安石恨到了骨头里去。当初吴氏就没少受过气,王旖也不可能像普通重臣妻室那样会被留下来说上两句话,更没机会与宫中的嫔妃攀上交情。

韩冈原本是这么想的,可没想到王旖到了快晚上的时候才回来。

“被皇后和朱婕妤给留下来了。”王旖跟韩冈说着,“家里都是六个孩儿平平安安的,皇后想问问是怎么照料的。还有寻常在家吃的喝的,还问了官人爱喝的汤药的方子。”

“你是怎么回的话?”韩冈问道。

“就把家里的情况说了,也没有什么好隐瞒的。家里的汤药饮子也寻常,都是市面上常见的,皇后和婕妤听着追问几句也就算了。”

韩冈摇头失笑,为了儿子连根稻草就不放过。

王旖进去换衣服。他对正在已经让下人将晚餐布置好的严素心笑道:“方子是方子。同样的方子,做出来的汤药饮子不一样。同样的菜谱,做出的菜口味也不同。这可是素心你的功劳。”

严素心回头横了韩冈一眼,“奴家哪里能比得上宫里的御厨。”

韩冈哈哈笑道:“御厨可没有如此千娇百媚的。”

严素心一跺脚,不理韩冈就出去了,。

玩笑是这么开,不过向皇后和朱婕妤想问什么,韩冈也知道。赵顼的底子不行,锻炼的时候又少,为了能多些儿子还日夜操劳,两边的情况根本就没办法比的。

韩冈悠闲的家里度日,陪着妻妾和子女,享受着难得的亲情。但看到正旦大朝会之后,韩冈依然老老实实的排队等候召见,外面的议论就渐渐多了起来。

许多人都猜测着韩冈是不是失了圣眷,又有许多幸灾乐祸的。韩冈在高层的人脉并不深,不像许多世家子弟,只要往殿上一站,前后左右都是攀上亲戚,有的甚至还能攀到端坐在御榻上的那一位。他与高太后的亲戚关系隔得不知多远,眼下没了王安石照看,有些人就想看着他这个灌园子怎么倒台。

但韩冈倒是什么都不在乎,这一切,都与闲居在家的他无关。看眼下的样子,天子是不着急让他去京西上任。不过也的确不急,要想征调民夫开挖运河,一般只能选在冬日农闲的时候,要不然就是灾年,否则误了农时,哪边都不讨好。韩冈就算是现在上任,也是来不及了。

时间一天天的过去,韩冈照样陪着家里的妻儿,偶尔出去拜访王韶、章惇一干亲朋好友。

苏颂去年作为贺生辰的使臣,去了辽国一趟,也是年底前才赶回来。有消息说天子有意任命他为开封知府。韩冈也去他府上拜会,说起苏缄之事,都是唏嘘感叹一番。

不管怎么说,韩冈与苏子元成了亲家,与苏颂也成了平辈的亲戚,而且两人的兴趣爱好也相同,对于天文、机械、算学上的爱好,都有许多共同语言,也算是忘年交。

苏颂也曾有问起韩冈打算如何整饬襄汉漕渠,毕竟韩冈曾经提到的多级水闸在京中也传扬开来。韩冈则反问道:“不知子容兄可还记得《禹贡》中的‘盖河漩涡,如一壶然’?”

“黄河壶口?”苏颂皱眉想了一想,他也是走南闯北多年,见识远胜普通的官员,头脑更是敏锐,当即恍然:“旱地行船!是轨道!”

“正是!”

“玉昆你不说用多级水闸,能提高水势的?”苏颂疑惑了起来。

“那也要有水才成。”韩冈摊摊手,“方城垭口的那一段渠道只有从方城山上下来的溪水,而且还不稳定,根本就排不上用场。只有向下掘深渠道至六七丈,引活水来,多级水闸方有施展之地。”

“五六丈啊……”苏颂听着就要摇起了头

“想要将开凿到如此深度,非是穷数年之功不可为。但换个想法,当中无法行船的一段,只要用轨道中转一下,其实就没那么困难了。”

“但运力?”

“通过襄汉漕渠的运力,能有一百万贯就够了。而利国监中的轨道,每年转运出来的生铁、矿石,又是多少?”韩冈笑着,他早有定计,“先把轨道修起来,如果顺利的话,当年就能有一百万石粮纲入京。至于渠道,轨道运粮归轨道运粮,渠道开挖归渠道开挖,两不相悖嘛。”

使用轨道,并不比运河花费更多。轨道不能持久,但运河也要年年清淤,汴河清淤动辄数万人,花销其实也省不到哪里去。韩冈也是为了满足急功近利的天子,先给他看到成效。就算运河开凿失败,还有轨道可以拿来抵数。

除了跟亲朋好友往来,另外的时间,韩冈也会见一见上门来拜会的访客,每天递到韩冈门房的名帖往往数十近百,韩冈也只能从其中挑着来会面。

尽管有人幸灾乐祸他失了圣眷,但韩冈并不是靠着天子的宠信爬上来的官员。圣眷就算再衰落,他也还是二十多岁的龙图阁学士。何况他还是京西都转运使,这个职位明显的就是天子为了让他能够施展自己的才华才交给他的。

地位在这里,总有人会拜上来,何况张载在京城的门生甚多,又有谁愿意放过韩冈这个难得机会?要知道,做过他幕僚的现在都已经得官,没有一个例外!

而这段时间中,朝堂上则是很平静。后王安石时代,朝堂上第一轮交锋在年前告一段落,而第二轮,则各自还在酝酿筹备之中。只要上元还未过去,就还是在年节里,暂时还没有人出来让天子过不痛快这个新年。

时间忽忽而过,转眼就是上元节了,又是到了天下同欢的放灯之夜、

韩冈一家受邀与王家一同观灯,两家是通家之好,也没有什么顾忌的。

宰执官们在天街之上,都有属于他们用来观灯的帷帐,从帷帐中可以仰望宣德门上的皇帝。

今年天子没有将宰执请上宣德门城楼,大概是不想好端端的上元夜变成两派攻击的时间。

难得得空,但王韶不喜欢人挤人的灯会,韩冈也是一样。他带着全家借了王韶的帷帐,看过了各衙门、贵戚、商行所修的灯山,又与王珪、吕公著、吕惠卿、章惇他们打了个招呼,就与王韶一起去了王家。

两家仆佣带着自家的小主人出去观灯不提,而韩冈却在王韶的邀请下摆开了棋盘,王厚在旁边看着。在上元节下棋,倒是难得的雅兴。

韩冈围棋水平不高,王韶也只能算是勉强,一胜一负的下了两盘,到了两更天的时候,出去观灯的两家人陆陆续续的全都回来了,但就不见王家的十三郎。

王韶的夫人刘氏急了,派人出去寻找,而过了片刻,一名家丁脸色白得跟纸一样被人领进了房中。进了门就跪下,膝行到了王韶面前,便砰砰的磕起头来。

“王全,你不是带着十三出去的吗?”王韶的棋子拿在手中问道,眼睛还盯着棋盘。

“小人罪该万死,小人罪该万死。”王全抬手给自己两个嘴巴,“小人把十三哥儿给丢了!”

这下子房中没几人能坐得住了,王厚一下跳了起来,一脚将王全踢飞,怒吼道:“你怎么办的事!”

王廓转头就对王韶道,“要速去开封府通报!”

“须得请开封府密访,动静太大,或有不测。”韩冈也沉声说着,官府追逼得急了,杀人灭口的事不是没有。

王韶敲着棋子,放了一颗在棋盘上,却是又紧了韩冈一步。抬头看看韩冈,“玉昆,该你了。”

“爹爹!”王厚急叫道,哪还能这般悠闲。

“无妨。他子当遂访,若吾十三,必能自归。……玉昆!”王韶抬头,不快的催促道:“你还下不下了?”

韩冈看看棋盘,又看看王韶,来回两次后,终于摇摇头,坐了下来,“枢密既然这么说了,韩冈怎敢不奉陪。”

