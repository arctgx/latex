\section{第21章 山外望山待时至(四)}

消息是去了渭源堡视察巡视的苗授传回来的。虽然是新官上任,但苗授做事比纳芝临占部的张香儿要靠谱得多,传回来的消息也更为准确。

星罗结部对韩冈来说并不陌生,渭源附近的大族。在董裕死后其势力大涨,如今只比青唐部略略逊色。

其族长康遵星罗结声名也同样响亮。他能在古渭之战中率部全身而退,虽然有着董裕这条大鱼吸引仇恨的因素在,但他一见中军遇伏便拔足狂奔,毫不拖延片刻的决断,也是很让人佩服的。而且董裕死后,他留下的部族势力,大部分是给木征收拢,而剩下的,则是归入了星罗结部的旗下。

但康遵也因为他参与了古渭之战,跟七部结下了死仇,无法倒戈向大宋一方。不过他同样并没有投靠木征,而是在古渭战事结束之后,保持了事实上的中立。不得不说,这是个很聪明的选择——只要避过一阵风头,再表现出一点恭顺,过个一年半载,王韶也就会向他伸出手来了。

以上是韩冈早前对康遵星罗结的猜测,前日听到他投向了木征,韩冈还惊讶了一番,不意康遵如此不智。现在听说星罗结部全面倒向木征的主事者,不是康遵而是新近上台的别羌星罗结,心中终于释然了。

只是他又立刻诧异起来,与七部结下仇怨的康遵已死,为何别羌还要改变在大宋和木征之间保持中立的策略?这对星罗结部又有什么好处?

在河湟地区,星罗结部算是大族,随时都能调动起两三千人。但在大宋面前,也不过是只蚂蚁而已。而且别羌又不像木征、董裕那般还有着吐蕃王家的血统,能对吐蕃众部有着足够的影响力。如果惹怒了大宋,起兵来攻打,没有一家会出力去帮助他——就是木征也不会。

“木征到底许了他什么好处?”韩冈百思难解,“别羌当真以为他只要不攻打城寨,我们就会允许他在白石山招兵买马?还是说他以为木征会去救他。”

“这就不知道了,不过木征是吐蕃赞普嫡系子孙,羌人一向畏服贵种,该不会是别羌忠于……”王厚的声音在王韶、高遵裕和韩冈的尖锐目光下越来越小,最后终于说不下去了。

‘忠?’

韩冈在王厚的尴尬中暗自冷笑。吐蕃人畏服贵种,指的是最底层的愚民。但凡能做到一族之长的,岂有一个善与之辈,又有哪个会被空洞的忠字迷惑住?他们最多也只会对延续数百年的吐蕃王家血统略表敬意,却绝不会为董毡、木征等唃厮罗的后代尽忠全节。木征、董毡若是没有他们手下的部族和军队,又有谁会去理会他们。

“别羌星罗结是如何盘算的,没必要去多想。”王韶不耐烦的说道,“关键还是决定究竟该如何处置别羌。”

对王韶的意见,高遵裕表示同意,“不论是剿还是抚,都比干看着他四处招兵买马跟朝廷作对要好,再拖下去,说不定又是一个董裕。”

“自来都是先礼后兵。先让人去做个说客,如果不成的话,再动刀兵不迟。”王厚从尴尬中恢复过来,说着自己的看法。

“如果是要进剿,以星罗结部的实力,出动的兵力不能少于三千。而以三千人计,出兵一个月,军费少说也要五万贯,粮草三万石,骡马千头,箭矢二十万,另外还需要动员同样数目的民伕……”韩冈掰着手指,给王韶、高遵裕算着开战的消耗。打仗最重要的就是钱粮充足,没钱没粮,就不要想着动刀兵。

而古渭缺的就是钱粮,“玉昆你觉得是要招抚喽?”高遵裕语气不快,他并不喜欢招抚,与一颗颗血淋淋的首级比起来,招抚得来的军功实在微不足道,“别羌可不是俞龙珂和瞎药,这等愚顽之辈,不杀一儆百,只会让人小看了官军。”

韩冈点头道:“钤辖说的是。别羌星罗结自接掌族长之位后,大肆招兵买马,四处散布谣言,并无一丝恭顺之心。观其行,正是个要顽抗到底的愚顽之辈。”

韩冈两头说话,高遵裕听着不耐:“玉昆你到底是何意,究竟是要进剿还是招抚?”

“如果钱粮问题能解决,当是以进剿为上。”

‘这不是废话嘛……’王、高两人暗骂道。

“不如让青唐部出兵……”王厚又提议道,但这次只说了半句就自觉失言,停了口。

王韶、高遵裕和韩冈一起摇着头。在座的四人都很清楚,名为归顺朝廷的青唐部究竟是个什么样的情况。无论俞龙珂还是瞎药,都是拿到了朝廷的封赏后,就回去做他们的土皇帝了,哪还会理会缘边安抚司的命令。

前次董裕举兵来攻青渭,俞龙珂迫于形势,同时也是为了与自己的弟弟争胜,所以他才会被韩冈说服出了兵。而瞎药是为了自己的野心,设陷阱阴死了董裕。要他们守着老家,反击来袭的敌军,他们会做得很卖力。但为了宋人出兵攻打有木征在背后支持的部族,他们可不会那么蠢。

让青唐二酋收下封赏好说,没人会跟钱做对,但要想让他们真正的归顺,听从朝廷号令,就像张香儿那样,王韶一句话就能让他点起族中军队,绝不敢稍作拖延,却是难上加难。

“要想青唐部彻底归顺,必须要让他们见识到官军的实力。现在贸然求助,不但事机难成,还会助长其骄横之心。”王韶今次完全没有借用蕃人之力的想法。真正听话的纳芝临占等七部现在只剩一个部族,总体实力下降了一多半。而有能力解决的青唐部,又不够听话。现在去求人,根本绝不会被理会。

“唉。”韩冈先叹了口气,“只恨两人都是狡诈多智,行事自有底限,不会为了与兄弟相争而失了分寸。不然就可以利用一下了。”

高遵裕动了动嘴唇,便没说话,韩冈把他想说的提议先一步给堵上了。

王韶低头想了一阵,最后也跟着叹了口气,“等苗授之从渭源回来,再做商议。”他苦笑着,“想不到缘边安抚司坐拥四千精兵,竟然拿一个小小蕃部没辙,真是可叹啊!”

韩冈看得出来,只要能解决钱粮人力的问题,王韶也想打上一仗。战斗力是打出来的,组织力是磨合出来的,不通过小规模的战斗来逐步积累经验,等到大战之时,可就要等着吃亏。这种最基本的认识,在座的几人中都很明白。

高遵裕、王厚一齐叹气,这才叫一文钱难道英雄汉。任凭你心比天高,囊中空空,就是没有底气。

不过对于开战的钱粮一事,韩冈还是有办法的,但这个主意有犯律条,他不想从自己的嘴里说出来。反正只要是做官的,迟早都能想到,他也没必要多嘴,暂且等着就是了。

韩冈打定了主意,低头喝茶。

接下来的两天,韩冈忙得脚不沾地。虽然屯田之事被王韶交给了高遵裕,但来此屯田的移民的驻地,高遵裕却要韩冈来安排,比起高家门下的清客,还是韩冈这个官人更能镇得住场面。

古渭左近,二十年来,已经吸引了近两千户来此屯田的汉民。除了有四成围着古渭寨居住,剩下一千两百多户组成了大大小小八个村落,都是位于东面的渭水边,以古渭为屏障,抵御西侧的来敌。不过新抵达古渭的移民,他们居所就必须安排在古渭西侧,筑成军堡的式样,来组成护卫城寨的防线。

要成为古渭寨的屏障,新移民们当然都不愿意。他们最希望的是在城边上找块好地住下来,要不然就是住到东面去,那样才安全。

能抛下一切,到古渭寨来寻个出路的,无不是敢赌敢拼敢冒风险的汉子。高遵裕派来管理这些屯田移民的清客镇压不下这些彪悍的关西汉子。不得不请了韩冈出马,虽然对他们来说,青色的官服要并不比士子的襕衫多了多少威慑力。但韩冈在秦州是威名赫赫,在一群吵吵嚷嚷的移民面前,把名一报,顿时就没人敢多啰嗦半句了。

为了整顿移民们的秩序,按照籍贯、亲缘分派到城外几个已经选定的筑堡地点,花了韩冈整整两天的时间。而在这两天里,王韶还要他跟王厚一起,先把出兵的计划定出来,而不用管钱粮的问题。

韩冈与王厚分工合作,整理着出兵的方案。能编纂出《武经总要》的宋代,军事方面早已正规化和公文化了,出兵开战,也不是将帅们拍拍脑袋,说句话就行的。钱粮军资、行军路线、驻地营垒这些最基本的东西不提,军情信报,口令密码,都要提前准备好

——王厚为了准备机密密码,在王韶的一份破旧诗集中,好不容易才挑出了一篇没有重复字样的五言律诗。五言律总计四十个字,其中每一个字都代表着一种情报,遇敌、被困、获胜、败阵,等等等等,必须事前确定。到了战时,最机密的信报就要用这些密码来传递。

就这么忙忙碌碌的到了第三天,新任秦州西路都巡检苗授终于回来了。

