\section{第一章 纵谈犹说旧升平(八)}

【下一更还是在下午。】

在一记记忽轻忽重冲击,将体内的快乐推到最高的时候,素心的喘息是从喉间挤出来丝丝缕缕的呻吟;一直被娇宠着的云娘,是不胜挞伐的低声抽泣;而王旖则总是紧紧咬着下唇、一声不发,但她会用力抓挠着让她飞上云端的罪魁祸首,狠狠的一点也不留情,不过在一次看到韩冈背后的伤口之后,便再没有将指甲留长。

只有在韩冈面前柔顺无比,但心中却是藏了一团火的旧日花魁,会毫不在意的哼吟出自己的欢愉,甚至一些韩冈要趁着王旖意乱情迷的当儿才能哄骗得她做出来的姿势,连素心、云娘也只是被动接受的姿势,周南她也能主动为之。

跪坐在韩冈身上,周南如同骑着烈马上下起伏着,一声声娇.吟如泣如诉。常年练舞锻炼出来的气力,让她能将这个动作持续不短的时间,而不是像王旖她们一般,几下十几下之后,就瘫软在韩冈的怀里。但之前已经来了好几次,周南其实也是快不行了,不过身后还有墨文襄助,尚能勉力的支撑一阵。

墨文穿着薄薄小衣,跪在宽大的架子床上,少女纤细的娇躯没有多少遮掩。她从后面扶着周南,帮着周南一起动作——贴身的婢女几乎都要担着这份任务。不过两三年的时光,跟在周南身边的这位少女,已经从还带着几分青涩的小女孩儿,变成了只差一步就要熟透的果实。似乎只要轻轻一捏,就会淌出芳香馥郁的果汁来。

周南已经气喘得不成调子,但抓着那两团雪腻的大手,却是仍毫不留情的揉捏着。韩冈的肌肤是暴晒过后的古铜色,常年锻炼的身躯,如同钢铸铁浇一般。而周南是粉白似玉,如山头的新雪,如新织的素绸。筋骨如钢似铁的大手没入胸前软玉之中,黑与白之间是惊心动魄的对比。

一声高亢的吟唱之后,周南软瘫如泥俯在了韩冈的身上,腻滑如羊脂美玉的肌肤,正一阵阵不由自主的轻颤,散发着高温,滚烫得将白皙细嫩仿佛最上等湖丝的皮肤都熨得通红。

墨文只觉得自己的掌心都被灼伤了,这股热流从掌心传到心底,又从心底传到了那个羞人的地方,春水潺潺湿透了亵衣。那股春潮之后的甜甜腻腻的气味,弥漫在垂下了幕帘的狭小的空间中,直往鼻子里钻进来,让她不由得夹.紧了双腿。

周南也只剩下喘息的气力,但韩冈的手指指尖却仍在背后慢慢划着。春潮之后,敏感至极的肌肤被指尖划过,她忍不住颤抖着。杵在身子里的那个东西依然火烫,熨得小腹又热了起来。自己都一次次的攀上巅峰,身子已经软得没有了气力,还是没能让丈夫的第二次缴械出来,凑在韩冈耳边低声告饶,“官人,让奴奴歇一歇吧。”

声音即娇且媚,还带着一丝沙哑,荡人心魄。韩冈不再玩了,用力拍了拍如同最为细嫩的豆腐一般的饱满臀股,却又爱不释手的揉捏起来,不过没忘叫着正春意涌动的小丫鬟:“墨文,给你姐姐端碗饮子来。”

日常滋补用的药汤,就在外间用小炉子炖着。韩冈在喝着,而几名妻妾也同样在喝着。这等在战乱时会被丢到一边的奢侈的养生之法,在如今的太平时节中,却是普遍而又普通,官宦人家无不如此。

墨文颤声应了,披着一件背子就掀帘下床。只是她浑身都软绵绵的,连走出去的动作有些不自然。

周南目光追着她娇小的背影,低声唤着:“官人。”

“嗯?”

“墨文都十六了。”

“这事不急。”韩冈轻轻一笑,“为夫今天可是要将你给喂饱。”

周南的身子又热了起来,轻咬银牙,声音婉转如歌,“官人要奴奴,奴奴就拼将性命服侍……”

一夜的欢愉没有影响到韩冈日常作息,他还是在日出前的晨曦中起身。

以房事来调剂身心和旦旦而伐的涸泽而渔,完全是两回事,韩冈有着足够的自控能力,家中的绝色纵然让他贪恋,但也不会如同吸毒般的沉迷。不过昨晚是周南的生日,未免用力多了一点。回头看看房中,被折腾了半宿的周南尚在海棠春睡之中,也不知何时能起。

外朝不厘务者谓之常参,他们日日都要上殿,在天子并不出现的垂拱殿上,由当值的宰相领着向着空空的御榻朝拜。而韩冈管着军器监,就不需要去每天去宫中站班,只参加起居以上的朝会。在家中悠闲的吃过早饭,直接去往军器监。

“周全拜见舍人!”

韩冈到了衙门之后,处理了一些日常的公务,便将如今大名鼎鼎的周全,叫道了面前。

作为飞上天空的第一人,他不仅在市井的说书人口中,有了一个‘飞天周铁钩’的匪号,还被赵顼赐了一个武官的身份,以奖励他敢为人先的胆量。

至于韩冈,是靠着献上板甲和飞船减了两年磨勘。这个奖励对普通按部就班熬资历的官员倒是很有用,但对像韩冈这样,从来都没有做满一任、以三级跳的动作在官路行走的人来说,其实是有等于无。

倒并不是朝廷不重视发明创造,只是韩冈他走的是文官路线,如今离侍制又只有一步之遥。想靠板甲和飞船的发明来挣功升级已经远远不够了,只有板甲局成功的大批量出产板甲,给禁军换装之后,让天子满意,那才是他加官晋爵的阶石。

周全的相貌粗豪,一看就是猛将的模样,失去的一只手又是为国而伤,所以在面圣的时候,这副卖相对了赵顼的眼,原本预订的恩赏是从九品的三班借职,但赵顼发出的口谕,却变成了正九品右班殿直。

官阶高了两阶之后,让韩冈在军器监中安排周全的工作也方便了许多。当以飞船为名的新作坊,从城外搬回到之后,周全就成了在军器监中任职一名官员。等到韩冈顺顺利利的将两位暗中使坏的官员送去了广南,使得他在军器监中的声威,一时无人敢于反对他的命令。周全不但管着飞船作,也便兼管起了板甲局和飞船作中的保卫工作。

“新飞船的情况怎么样了?”韩冈问着。

“回舍人的话。只是载人的飞船,天天都在金明池那里试飞。可是要想将油炉子也一起搬上去,飞船上的气囊差不多还要再大上一倍。可这样一来,油炉子又显得不够用了。”毛茸茸的胡子脸上显出几分急躁。韩冈吩咐下来的话,周全他一直催着下面人去动脑筋,但一个多月了,却还没有结果。

韩冈呵呵的轻笑了两声:“这事不用着急,悬赏出去让人想办法就是了,要个好一点、能生旺火的炉子。”

韩冈已经让军器监中的工匠们习惯了悬赏,比起空泛让人发明一个有用的武器,直接指出需要在哪一项上有个合用的发明更为有效。给出一个明确的问题,让军器监中的能工巧匠们去思考答案,得到让人满意的回报的几率要大得多。

“小人明白。”

“监中今天的情况怎么样?”韩冈又问道。

“原本担心被监中沙汰的工匠这下都安心下来了,也没几人说不愿去东京城外。”

韩冈点点头。周全他性格精细,为人又善机变,与外表完全不同,要不然韩冈也不会让他出面演那场戏。而如今也让他在监中做着包打听的工作。

周全有些犹疑:“舍人,如果当真迁去汴口边上,会不会让那些水力磨坊里的人闹出事来。”

韩冈心中的一番盘算也不瞒着周全,笑道:“动人饭碗,肯定要得罪人。但帮人保住饭碗,也同样能卖好人。我是判军器监,安抚工匠是份内事。水力磨坊的事,不需要我来操心。”

昨日才上殿向天子禀明的提议,韩冈此前已经让周全在监中露出口风,来安抚因为新型锻锤推广之后,而变得惶惶不安的工匠们。机器代替人力,手工业者失业是必然,但韩冈无意去做军器监工匠们眼中的恶人,自己管着的这个地方,他需要留下一个好名声。

这几日也有几人来向他确认关于军器监中作坊迁往城外的消息,不过韩冈没有给他们一个明确的答复——在赵顼点头或是摇头之间,他也没有权力给出一个答复。

但现在周全的回报,已经证明了他安插进军器监中的亲信,已经将他的心意没有扭曲的传了出去。想必接下来的几日,几代人都在一直居住在京城的匠人们,会想方设法的打听到他韩冈昨日在殿上的发言。

“你现在回去后,想必找你打探消息不会太少。该怎么做,想必不需要我再多说,只要能安定住监中人心就够了。让他们明白,只要我还做着判军器监,就绝不会抛弃任何一个人的。”

周全一拱手:“小人明白。”

“不是‘小人’。”韩冈笑着摇摇手指,“是‘下官’!记住了,是‘下官’,不要再说错了……”

韩冈对自称的纠正,让周全眼中满是感激,一挺胸,拱手壮声:“下官明白!”

