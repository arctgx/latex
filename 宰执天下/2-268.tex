\section{第40章 败敌逐远山林深(上)}

苗授还未出发,就被人知会,他想要的东西已经被人抢先一步拿了过去。

这当头一闷棍,外人也许看不出,但苗授本人却当真是被敲得晕头转向,脑门嗡嗡的响了一阵才停了下来。

冷了一下场,苗授这才哈哈的笑了起来,“韩玉昆果然不简单,有他镇守后路,怎么也不用担心了。”

他哈哈对着王舜臣说着,“这样下去,日后关西用兵,若是没有韩玉昆来主持转运、镇守后阵,恐怕没有哪家士卒愿意上阵了。”

王舜臣陪着他也笑道:“三哥能文能武,本就不是他人能及。”

这时候,珂诺堡大捷的消息已经遍传军中,又传到了香子城内,数千士卒们的欢呼声猝然而起,仿佛炸雷一般惊得天都要塌下来。

万众欢庆着胜利,苗授、王舜臣则是面对面的绽着笑脸,赞着韩冈的功绩。

到最后,王舜臣的脸笑酸了,转头盯到了依然站在一边的石勇身上。

“石勇,你还愣着这里做什么?还不快点将这个消息传到王经略的手上去,耽误了事你担待得起吗?”王舜臣说了两句,跳下马,将自己坐骑的缰绳递到石勇手中,“快点去吧,王经略那里有重赏等着呢!”

石勇听着眉花眼笑,又朝王舜臣磕了一个头,与苗授、王舜臣别过,跳上刚刚得到的骏马,向着河州城的方向飞驰而去。

目送信使远去,王舜臣又转回过来,挑起眉问着苗授:“都监,下面该怎么办?”

“既然后事无忧,我自然是要回河州去。经略手上多上一份力量,也就更容易胜过木征。”

没有了立功的机会,苗授片刻也不想在后方久待,最后一战很快就要开始,如果能在合适的时间赶到战场上,很有可能成为决定战局的最后一枚棋子。

最后的机会苗授不会放弃,他拍拍王舜臣的肩膀,“此地还要靠王都巡你了!越是最后关头,越是要谨慎,切勿有失啊!”

“都监放心,末将必然不负所托。”王舜臣拱手行礼,低垂下去的视线钉着地面:‘恐怕王经略已经不需要你这三千人了。’

……………………

一片欢呼声之后,宋军离开了大营。在毫无阻碍的情况下,于昨日的战场处摆开了阵势。

虽然并不能确认宋军因何而欢呼,但木征一方的首领们隐隐约约的都能猜测得到,究竟是哪里出了问题。

偷袭香子城、珂诺堡的行动必然失败了,不然宋人不会这般兴奋。莫名其妙的欢呼起来,肯定是让他们后顾无忧的大捷。木征古铜色的一张脸,全都失去了血色,脚步都有些踉踉跄跄起来。

“我们的兵力还在宋人之上!”一个年轻人奋力的大叫着。

但木征的视线从簇拥在身边的将领们身上扫过,超过一半的部族族长都垂头避开了他的眼神。

宋人的欢呼,其实就是宣告木征躬身离开河州的信号。前面的屡次败阵已经让诸多蕃部失去了信心,而为了抵抗宋人的侵略而付出的沉重代价,也同样让他们忍耐到了极致,唯一支撑他们坚持下去的,就是还有着一个反败为胜的机会。

但眼下,木征苦心积虑作出的计划,却还是遭到了可耻的失败。现在连一丝胜利的机会都看不到了,除了少部分木征的死忠亲信,其他人都是为自家的部族打起了小算盘。

他们都听说了宋人的皇帝赐了王韶许多空名宣札,填了名字就能得官。就是用来赐予归附宋国的蕃人的。如果实在没有办法,投靠宋人也是一桩美事,好歹能落下点赏赐。

‘树倒猢狲散吗……’唃厮罗的长孙心中透着悲凉,最后的计策都没有用处,真的到了穷途末路的时候了。

‘不!’木征双眼重又恢复了神采。能盘踞河州几十年的他,可会是那等一次挫折之后,就一蹶不振的废物?

他还有南方,就在露骨山的对面,还有臣服于他的诸多蕃部,还有属于他的一片土地。宋人翻不过露骨山,他们也不敢翻越那座几千丈高的山峦。与高耸如云,山头积雪常年不化的露骨山比起来,河州周围的无数山岭,就只能算是丘陵而已。

只要他堵上通往岷州的路,就不虞宋人能够拿他怎样?完全可以坚持到宋人在河湟支撑不下的那一刻。

元昊的祖父当年几次三番的被宋人追着跟一条狗一般,最后还不是翻了盘过来,到了他孙子辈手上,更是建立了横跨数千里的西夏国。河州暂时让给宋人又如何,只要他还活着,只要能保住最后一块地,终究还有将河州夺回来的时候。

木征绝不会就此死心!

至于眼前的战事,既然难以挽回,那就还是为他们各部留一点元气,也好日后重新将之收拢做好准备。

……………………

而在战场的另一边,王韶和高遵裕还不知道木征已经有了退意。仍在脸上摆出了欣喜的模样,跟着他们麾下的将校们一般。

只是他们的眉宇之间,依然透着隐隐忧虑。若是在平日里,普通一点的会战中,他们对韩冈都有着充分的信心,过去的一次次成功,都证明了韩冈的能力。但今次的局势实在太过关键,过往的战绩所建立起来的信心,也难以维系他们心情上的稳定。

不管怎么说,以珂诺堡与河州之间的距离,木征短时间内不可能收到消息。就算珂诺堡有所闪失,在午后之前,对木征来说,肯定依然是夜袭大军已经失败的结果。

与昨日同样响彻天地的战鼓声又被敲响,严整的军阵开始向着敌军聚集的方向前进。

踩着鼓点,数以千计的宋军将士做好了与敌血战的准备。可是今天的情况与昨日不同,在他们的兵锋之前,吐蕃骑兵纷纷溃败,甚至还没有等到宋军冲到面前,就已经四散逃离。

木征汇聚起来的数万大军,转眼间就风流云散。数以万计的宋军将士,愣愣的站在战场之上,一时竟忘了追击下去。

‘就这么胜了?’

王韶和高遵裕疑惑的对视着,他们都不能肯定,难道光凭着模糊不清的欢呼声,就已经让吐蕃人失去了战斗下去的勇气?难道吐蕃人已经风声鹤唳到了这般地步?那昨天跟他们血战的又是谁?

“……不对,木征是要逃!”王韶脸色猛的大变,“拿不下木征,河湟就安稳不下来!”

“要追!”高遵裕也警醒过来,不拿下木征,怎么都不能算是一个胜利。

可望着一队队仓皇离开战场的吐蕃骑兵,王韶和高遵裕的惊怒也无济于事。并不是昨夜在珂诺堡下的情况,同样歇息了一夜的四条腿的骑兵要跑路,就不是两条腿的步兵能追击得上的。眼下手上仅有区区两千骑兵,怎么也不敢让他们深入敌阵、追摄上去。

万胜的高呼响彻云霄,但王韶和高遵裕的心中却是没有半点胜利的欣喜。

王韶紧紧的咬着牙。虽是穷寇,却必须是穷追到底!

深呼吸过后,王韶平复了心情,他转过来对高遵裕道:“既然木征已经放弃了河州,这里还会忠心于他的部族应该不多了。正好空名宣札还留有许多,也该散一散了。”

在开战前,一开始天子就下发了两百道宣札,后来又追加了一百道,加上便宜行事的权力,理论上王韶可以将任意数量的小使臣的官职授出。而实际上,比三百道空名宣札的数量多个一两百,天子也不会放在心上。

虽说给蕃人的官位,从来都是听着好听,看着好看,没有什么实际的东西。木征被授予河州刺史,照样被打得老巢河州都丢了。从元昊的父祖开始,大宋就册封他们为夏国国王,定了君臣尊卑。但两国之间,还不是照样厮杀了几十年。

但是王韶要用来收复蕃人,宋国的官位便是最好的工具。

“今次一战斩获还是太少。这一片吐蕃蕃部,不知会有多少愿意臣服。”高遵裕说着自己的忧虑,“万一他们只是名面上臣服,实际上还是听着木征的话,那又该如何是好?”

王韶咬着牙,“木征还是要追下去。必须要追击到底,木征不除,河州绝然安定不了!”

“看木征的离开的方向,那是去南面啊!”

高遵裕和王韶都是老于兵事,对河州的地理有过深入的了解。木征今日的去向,代表了他日后的战略。

木征如果想要借助他叔父的力量,投靠眼下的吐蕃赞普——董毡,那必然是要向西越过离水,去湟水边的青唐王城。如果他想依靠党项人,则应当去北面,接近兰州的地方。

但现在他向南走,那里没有他可以借助的势力,却有着一片属于他的土地。

就在群山之南,林密山深之处。

“露骨山!……”高遵裕的声音中有着一丝畏惧,“那道山我们绕不过去!”

“那就翻过去!”

