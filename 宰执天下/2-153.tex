\section{第29章 顿尘回首望天阙(五)}

【大封推,求各位书友的红票推荐。】

喝酒唱曲,闹了一通后,章惇便请各人落座,周南怯生生的坐在韩冈侧后,做足了少女新嫁的模样。

今次的宴会是分席,包厢内摆出了五个席位。除了现在的三人,加上入宫的王韶,当是还有一名客人未至。

章惇留意着韩冈的视线,就向他解释道:“待会儿还有位贵客要来。如今正在开封府用事,是个极有趣的朋友。他对玉昆你钦慕已久,听说玉昆要赴宴,便一口应承下来。”

‘应该是为了王韶才对。’韩冈笑了笑,问道:“不知检正的朋友究竟是开封府中的哪一位?”

“不急,来了便知晓。”章惇故作神秘的不肯明言,让韩冈去猜。

周南却是早就知道,她附在韩冈耳边低声说道,“是管干右厢公事的蔡确……”

周南贴得很近,高挺的酥胸正压在韩冈手臂上,绵软中带着弹性的触感从接触的地方传来,温热的呼吸呵着耳朵,韩冈心头就有些发痒。

虽然惊讶不是章惇那个名气响亮的朋友,但开封府管干右厢公事这职位,已经不低了,普通一点的京官都坐不上这个位置。

东京城周围五十里,整个大开封府更是相当于一路的地界。包括十七个县二十多个镇子。依照惯例,东京城内事务,归于府衙,城外则是由两个附廓的赤县——开封县、祥符县处置。就像秦州州治成纪县,城内归州衙管,城外则是成纪县的管理范围,所以住在城外下龙湾村的韩冈当初作为衙前去州城,就得去县衙报道。

不过东京城实在太大,周围五十里,府衙不可能一力统管,因此便把城中分为十个厢,依东西划归左厢和右厢两都厢统管,各自分厢坊管理民政。其实这跟后世的区划没有区别,就是将一个大城市分为郊县和市区两部分。

而蔡确管干右厢公事,他的身份,实际上就是相当于后世的一个首都区长。一般来说,都是资历深厚、成绩斐然的知县或是通判来担任。虽然比不上章惇的位置,却也不是韩冈能望其项背。

韩冈思忖起章惇邀请蔡确的用意,在王韶本来确定与会的情况下,章惇不可能随随便便请来一些无聊的闲人。路明身份虽卑,但他两边说得上话,上席也没问题。而蔡确不知有何特殊的地方,让章惇特意请了他来。

路明在旁边看到了周南的耳报,不用想也知道她到底说了什么,就调笑道:“周小娘子才喝了交杯酒,就偏向玉昆,当真是宜家宜室。”

周南脸红了,推开韩冈重新坐好。这时门外就传来一声长笑:“子厚,蔡确可是来迟了?”

“尚未开席,持正来得正是时候!”章惇闻声便长身而起,大步过去,把人迎进来。向着韩冈介绍道:“这位就是现今的管干右厢公事的蔡持正。原是邠州司理参军,新进由韩相公从陕西荐到韩大府处。与我即是同乡,亦是同年。”

被介绍给韩冈的蔡确,年纪大约在三十出头,跟章惇相仿佛。身量颀长,仪貌秀伟。气度非凡,并不输给章惇。章惇说其是同乡同年,当也是福建出身的进士。

章惇两次中进士,一次是嘉佑二年,一次是嘉佑四年——仁宗朝的科举时间,前期是四年一次,后期则是两年一次,间隔三年的定规,还是今次的熙宁三年这一科才开始——说起章惇的同年,嘉佑二年中进士的王韶也能算,不过以章惇的脾气,他只会承认嘉佑四年的进士才是他的同年。

韩冈看蔡确的服饰,本官的品级应该不算高,比起章惇还差了一些,但十年前中进士入官,现在就已经是开封管干右厢公事,论进速,已经是快得让人惊讶了。

韩冈上前与其见礼,自报姓名。蔡确回礼后,便拉起韩冈的手,亲热的笑着道:“在下蔡确,尚在关西时,便久闻玉昆之名。与游景叔共事时,也多有提及玉昆你。渴慕久矣,今日终于得见!”

韩冈闻言谦虚了两句,问道:“不知管干是否就是‘儒苑昔推唐吏部,将坛今拜汉淮阴’的蔡持正?”

“不过是席上的敷衍之作,”蔡确见韩冈竟然听说过自己的作品,神色间略显自得,“不意玉昆竟然有所听闻,有辱清听。”

“今次韩冈进京,过京兆府时,在席中正听得人将此一篇传唱不已,闻者皆赞,韩冈望尘莫及。”

韩冈其实并没听说这两句诗,是周南方才在耳边悄声说给他听的。‘汉淮阴’说得当是韩信无疑,‘唐吏部’虽然所指宽泛,但前面有个‘儒苑’,说起来唐代能跟吏部扯上关系的儒学大家,也只有追赠吏部尚书的韩愈了——韩吏部。

文韩愈、武韩信,这两句诗看意思,就是在吹捧韩绛文武兼备。也难怪如今的首相听着喜欢,把写了诗的蔡确荐到正任开封知府的韩维处。

蔡确与韩冈见礼后,仍是亲热的拉着手说话,但他的视线则是不经意的在包厢中转了一下,

章惇当即笑道:“只可惜王子纯将要赴宴的时候,被天子传入宫中,不克前来……今日饮宴的也就我们四人。”

蔡确听到王韶被召入宫中,脸上不由闪过一丝混着失望的羡慕,但立刻就隐了去。坐下来喝酒吃菜,欣赏着歌舞,跟章惇、韩冈说笑起来。

蔡确很善于与人交流,没过多久,就跟韩冈混得没有半点初次见面的隔阂。只是他一口标准的官话让韩冈有些吃惊。

韩冈本人在关西生活,说话不免带上秦腔,王安石、王韶皆是江西人,说话带南音。章惇是福建人,福建腔调都参杂在官话里。可蔡确也是福建人,却没有半点福建口音。

当韩冈问起,蔡确便解释道:“寒家自迁居陈州已经近三十年,乡音也是早改。”

“原来如此!”韩冈点着头。

轻柔的琴声为四人的闲谈做着伴奏,而陪酒的官妓也说些有趣的轶事,宴席上的气氛显得很轻松。除了韩冈身边只有周南,章惇三人身边都有着两名官妓作陪,尤其是蔡确身侧的两位,打扮起来姿色都不比周南稍差,不过周南胜在年少,不施脂粉已是清丽无双。蔡确觉得有些奇怪,便多看了周南和韩冈两眼。

章惇见了,便指着周南:“一刀惊退了高密侯的周小娘子,不知持正可曾听说?她的那柄匕首就是玉昆送的。”

“难怪!”蔡确恍然,拍案而笑,“虽然蔡确来京不过旬日,但周小娘子的威名已是如雷贯耳。以匕定情,名传京中,想不到竟然是玉昆送的。”

用‘威名’来形容周南,蔡确说话的确有促狭。他转过来又对韩冈笑道:“化芍药为刺蘼,不意玉昆竟是园圃中的圣手。”

刺蘼就是蔷薇,蔡确还是在调侃周南一把匕首吓退了诸多狂蜂浪蝶。不过说起园圃,那就牵连到韩冈的出身上了。蔡确当是无心,但章惇和路明还是担心的看向韩冈。而韩冈则不以为意,侧脸看了周南一眼,笑道:“圣手不敢当,非是己力,只是幸逢佳人垂青罢了。”

“玉昆真是惜花之至。得如此佳人倾心,当不能轻负!”

蔡确能听说周南的名字和事迹,当然不会没人跟他说,雍王如今正看上了周南。但他没有在席上表现出半点对天子亲弟的顾忌,而是直截了当的表示对韩冈的支持。

韩冈举杯感谢蔡确的善意,不论是真是假,他能当众说出来,已经让韩冈感觉章惇的确会选人。

章惇也道:“美人垂青,正如伯乐看重。玉昆得王子纯荐举,功绩累累,也是不负那一份荐书……”

“说得正是!”蔡确道,“说起来,子厚亦是不负王相公看重,事事用心,中书之事井井有条,得到的赞许甚多。”

“彼以国士待我,我当以国士报之。持正难道不是想一报韩相公的恩泽吗?”

蔡确笑着点头,“自当如此!”

章惇再次举杯:“不过持正声名鹊起,还是先是得自薛师正的荐举,这些年来也不负其所荐。”

蔡确被勾起回忆,一口喝下满杯的酒,叹道:“前些年在邠州得罪了小人,若无薛师正相助,怕是要去官夺职了。”

薛师正就是薛向,当朝首屈一指的财政专家,遍历地方,治事亦能恩泽百姓。他连陕西转运使都做过,离统括天下财计,号为计相的三司使也只有一步之遥。但因为不是进士出身,加之擅长的又是钱粮之类让士大夫鄙薄的行当,所以一向被人鄙视——最重要的,是薛向升官太快,位置太高,让许多进士出身的官员看不顺眼罢了。

而他现在担任六路发运使,主持均输法,统管大宋命脉的纲运,是王安石重要的盟友,也就惹来一大批言官坚持不懈的弹劾他。不过薛向理财方面的能力实在是太过出色,朝中找不到能替代他的官员,所有的弹劾都如石沉大海,毫无音讯。

“每年六百万石的粮纲,六路发运使的位置可不好坐,朝中现在也只有薛师正能坐得稳。”

章惇说着,韩冈则是眉头微皱,他总觉得章惇现在好像是刻意在引导话题。他望过去,章惇则是回了他一个平和的微笑。

韩冈眼神收紧:‘这章子厚,到底想要说什么?!’

