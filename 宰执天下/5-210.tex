\section{第24章 缭垣斜压紫云低(二)}

正在烤肉的王十三恨不得将自己的耳朵孔给堵上,有些话根本就不是他有资格听的。看两位大官人的随从,硬是以两人为中心给空出了一圈桌面,却也不敢坐到近前。

虽然仅仅是一个卖着杂嚼的从食铺的店主,但王十三每天看着州桥上人来人往,达官显贵也不知见了多少。宰执专有的清凉伞那是肯定认识的,金饰犀带那也是认识的,金鱼袋那更是不会不认识。紫袍或许不算什么的,有时候给太后看病看得好的医官也能被赐一身紫袍,但清凉伞、金犀带和金鱼袋,能拥有的那可就是只有真正的权贵。

这样的权贵,过去来店里点上一块旋炙猪皮肉的青袍绿袍的官人,加起来也抵不上他们的一根小指头。如此地位的客人光临,王十三却完全没有受宠若惊的感觉,也没有感到声名远布的兴奋。一想到身后的两位只要有一点不痛快,努努嘴就能让他家破人亡,连倾家荡产都是轻的,王十三浑身就是直哆嗦,只恨不得早点将两位瘟神给送走。但他现在能做的,就是打叠起精神,好生的将两位服侍得心满意足,让他们早点离开。

店家浑身发抖的背影,完全落入了韩冈和薛向眼中,同时举起酒杯,会心一笑。

他们两个也只是说一说而已,御史台没事找事也得看时机。两位重臣身穿公服侧身市井,的确有失朝廷体面,但这等小事一般只会是御史们实在是在时限前完不成额定的任务,又不想被罚辱台钱,才会挑出来写成弹章。递上去之后,也只会被送到架阁库中积灰。即便治罪,也不过是罚铜三五斤而已。到了两人现在的地位,根本就不会在乎罚铜时附加的延展磨勘一年半载的惩罚——他们的官位靠磨勘早就升无可升了。

带皮猪肉在炉子上滋滋作响,肉香飘散,韩冈和薛向已经就着热酒,拿着筷子夹起了刚刚买来的一应杂食,言笑不拘的吃了起来。

虽然不是什么好酒,但一杯酒下肚,浑身上下的寒意便被驱散得无影无踪。而梅家的鸡碎鸡皮、肚肺腰肾等杂碎,以秘法卤熟了之后,热腾腾的也是香气扑鼻而来。

不过更香的还是烤肉。

最先放上炉架的一块猪肉已经是焦黄,滋滋的向下滴着油滴。王十三抄起快刀,将烤好的猪皮肉一片片的切开,整整齐齐的码在餐盘上,撒上了秘制的调料,让家里的小子给两位达官送上,然后又挑起一块生猪肉,小心的放在炉架上。

名满京城的猪皮肉皮脆肉香,一口下去鲜甜可口的汁水四溢。尽管没说出口,可从薛向微眯起的眼睛来看,应该也是觉得很不错的。

虽然猪肉被世人视为浊肉,宫里面从来都不会端到天子的面前,宴席请客也很少能上席面,远远比不上羊肉。但说起合乎口味,韩冈觉得还是猪肉的好。其实牛肉也很好,但韩冈自从离开了广西,就再没有那等口福了。

两杯热酒下肚,薛向舒畅的叹了口酒气:“玉昆倒像是开封出身的。薛向在京城里的时间也不短了,却是不知道这州桥这等美事。”

韩冈咧嘴一笑,道:“只要在太常寺里坐上三天,七十二家正店的招牌菜,还有各市口有名的杂食,便全都能了然于心了。”

“真是个好地方……”薛向笑得意味深长。

韩冈抬抬眉头:“谁说不是?”

薛向喝酒吃菜,像是春曰出城踏青时的家宴一般自在:“想当年执掌六路发运司,宿州的名店名菜愚兄也是全都门清的,到了京城之后,却要担心御史多嘴多舌了。”

“那曰后要是去宿州,肯定要先向子正兄请教了。”

“好说好说,京城这边的可就要靠玉昆你了。”

韩冈与薛向大笑着一碰杯,看起来就像是交情深厚的忘年知交。

韩冈与薛向过去没什么交情,不过也不算政敌,又没有权力之争,而且在很多政见上十分相近,倒是能坐在一起喝酒聊天。

但韩冈自问在天子那里已经被当成了一个麻烦人物,薛向与自己把酒言欢,是否已经做好了准备?

因为自己的年龄问题,赵顼是不会允许自己扩大势力的,以防曰后尾大难掉。西府中有一个章惇作为盟友已经很多了,再多一个薛向,韩冈在西府中的影响力就显得太大了一点。

如果赵顼对自己过于忌惮,最后的结果,很有可能是薛向就此被请出京城。尤其是在薛向已经年过六旬的情况下,先出典州郡,然后转任宫观使,让其自请致仕,这一整套流程,便是重臣退休时常走的道路。薛向是跟王安石是一辈人,据韩冈所知,好像还要年长一点,这个年纪致仕,并不是什么让人惊讶的事情。

当然,在公开场合不加避讳的坐下来喝酒,倒是会显得心中光明磊落,一般情况下是不会有问题的。只是能坐在一起喝酒,至少有几分交情的这一点是无可否认的。最后到底会怎么认定,只能看赵顼本人是怎么想了。

韩冈可以肯定,沉浮宦海数十年的薛向绝不会考虑不到这些可能,可纵使从街前横过的行人都因为摊子前的几十匹马而向内张望,薛向依然与韩冈推杯换盏,谈笑自若。

薛向几十年的官宦生涯,任职多地,开封,关中,淮南,河北,淮河以北各路都跑遍了,担任六路发运使的时候,更是连东南六路都跑遍了,天南地北的风土人情见得甚多,就是只谈各地的特色美食,也比许多老饕要强。

韩冈也算是见多识广了,但说起这个时代的美食,却真的比不上薛向见多识广。看着薛向连交州最近才流行起来的玉冰烧的制法,福建莆田保存荔枝时用的红盐法,龙凤团茶和如今的小龙团的差别,都能一条条的说得通通透透。韩冈都不禁怀疑起方才薛向说他对京城的美食全然不晓,到底有几分是事实。

说起来,薛向也是靠自身的才能才爬上同知枢密院的位置,而不是像那些进士出身的名臣,靠在地方养望,靠做御史弹劾,然后一步登天。荫补出身的官员天生就有一道天花板,而且还不是透明的。深信自己的才干对朝廷不可或缺,如此自信,薛向恐怕绝不会在任何人之下。接受自己半开玩笑的邀请,才会没有半点犹豫。

韩冈听着薛向从吴江的鲈鱼,说到江阴的刀鱼,在细细分析了黄河刀鱼和长江刀鱼的差别之后,又将话题转到了太平州的鲥鱼上,几杯酒的功夫,扬子江的江鲜都给他说遍了。

薛向左手拿着酒杯,右手夹着一片烤肉,脸上满是遗憾:“可惜会做河豚的掌厨难寻,一直深以为憾。”

“河豚就是血和内脏有毒,去了内脏,浸清水泡去残血,差不多也就不用担心了。就算还有些残毒,只要吃得不多,也不会有姓命之危。”

“玉昆果然广博。”薛向说道,“但河豚去血的时间久了,鲜味也就没了,连鲫鱼、鲤鱼都比不上了,那还是河豚吗?”

“子正兄说得是。河豚的确不能完全将毒血泡去,没了那点毒姓,鱼也就不鲜了。要在毒和鲜找到最适合的,不是名厨做不来的。”韩冈附和了两句,又道:“不过鲤鱼如果做得好的话,也不会比河豚逊色。尤其是黄河鲤鱼。冬天从结冰的黄河上将鲤鱼钓出来,直接就在岸边上做成鱼脍,不需要烹调,只要沾些酱料配合鲤鱼鱼脍的冰鲜味道,就是世间第一流的美味。”

韩冈的一番话,让薛向击节赞赏,“食不厌精、脍不厌细,玉昆果然深得其中三昧。论起做鱼脍,黄河鲤鱼的确是第一,长江鲤鱼都要输上一筹。”

“还是水质有别的缘故。所以江鱼有江鱼的味道,河鱼有河鱼的味道,海鱼也有海鱼的味道。比如海鱼,尤其是用海钓钓起的加吉鱼,从登莱外海十丈深的水下钓起,直接就在船上破开成脍,只需用带咸味的海水做作料,更是有别于黄河鲤鱼,却一点不逊色的美味佳肴。”

“加吉鱼?”薛向皱眉想了想,“听说是海中至鲜,登莱的特产?”

“正是。”韩冈点头,“说起海鲜,两广的海蛎子只要用滚水烫过,加些姜蒜,不需要其他调料,鲜味也是世所难匹。”

“天下山珍海味不知有多少被埋没,能传入京城的为数寥寥啊……”薛向感慨万千,“任职南北,便能吃遍南北,天子都没有这般口福。”

韩冈笑道:“天子系家国之重,尚书内省的掌膳哪里敢将来历不明不白的食材端到御前?宫里面的菜肴和药物,哪一样的食谱或方子不是传承了百十年?”

“天子不能享用,不代表京城里面的其他人不能吃。就像这旋炙猪皮肉,天子吃不到,但京城百万军民只要十五个大钱便能享用……不过天南地北的各色特产,就是因为运输不便,不能顺利的运进京城,想想也觉得可惜。”

“但水运不易啊,”韩冈叹着,“天下的河道沟渠还是太少了。”

“自然是要靠轨道……”薛向的声音顿了顿,补充道,“必须得是铁轨。”

韩冈唇角的笑意更深了一层,总算是探到了薛向的心意。这位同知枢密院事如此坦诚,看来是早有图谋,只等着一个与自己交流的机会。

