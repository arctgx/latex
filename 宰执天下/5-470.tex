\section{第38章 何与君王分重轻(28)}

蔡确左一句以国事为重,曾布又一句以国事为重。

章惇在旁摇旗,张璪则敲着边鼓。

可向皇后还是在摇头,不肯主动出面逼丈夫逊位。

“平章。”蔡确回头来找王安石,他实在不想找韩冈,也只能请王安石帮忙,“国事为重。都要四更天了,到了天明还不定下来,明天朝堂上会变成什么样?!”

王安石脸色并不好看,像是被人欠了钱——他还不在乎钱,应该是被人烧了满屋子的书那样的难看。

作为地位最高的臣子,军国重事皆能与闻的平章,事关大宋江山、亿万子民的帝位传承,王安石不可能置身事外。

他不想看蔡确他们凌迫天子和皇后的样子,但也清楚这不是唱反调的时候。

而且他也知道,蔡确、曾布等几位决不会善罢甘休。羞刀难入鞘。到了这时候,只能一不做二不休。退一步都是死路。想想寇准,想想周怀政,那都是在真宗皇帝重病时,谋图拥立时为太子的仁宗,以真宗为太上皇。仁宗也同样是真宗皇帝的独生子。但他们最后还是失败了。结果就是一个去岭南,一个直接就砍了脑袋。

就是压倒了蔡确,赵顼照样不可能起来主持朝纲,站在皇帝一边根本就没意义。早一点内禅,国家或许就能安稳一点。

站在王安石的角度,这已经是最不坏的结果了。

“殿下。事已至此,已经拖延不得了。”王安石话说得很勉强,但终于是开口帮忙。

“没错。已经拖延不得了。”跟着王安石,蔡确、曾布又开始新的一轮攻势,力图说服皇后松口。

韩冈和薛向坐在最下首,不发一言,就是在看。

看了一阵,薛向忍不住轻声对韩冈道:“玉昆,你就在这里坐着?”

“韩冈在这里面资历最浅,地位最低,哪里有说话的份?”

虽然是玩笑话,但正常的情况下,也的确只有老资格的宰臣才有资格在这个场合说话。

不过韩冈肯定是例外的。只是这两天来,他说得够多,做得也够多,下面低调一点就可以了。

蔡确、曾布他们,肯定也是不想看到自己忙前忙后,跟他们抢生意。

“玉昆你是站在高处不怕湿脚。”薛向的低声一句,也不再多说什么了。

薛向也不急。当初冬至夜时,他好歹也是第三位请立太子的宰辅。从时间上,比王珪早一步,从表现上,比王珪好得不只一点,就姓质而言,他是功,王珪是过。

有这一份功劳,现在也不用跟人争抢了。也就是当初排第二的张璪,就稍嫌贪心了一点。估计是想做宰相,以他的情况,也只有积累定策之功才有一星半点的机会。

看着蔡确和曾布的表演,韩冈轻轻叹了一口气。

太子若是看到这一幕,心中肯定会留下一个疙瘩。不过也用不着担心什么,这是为大宋好么。

英宗当年重病垂危,几位宰辅上去请英宗立太子、定遗诏。这明着对病人说你快死了、赶紧把遗书写了的,英宗最后是‘泫然下泪’,文彦博回头对韩琦道:‘见上颜色否?人生至此,虽父子亦不能无动也’,韩琦则回了一句,‘国事当如此,可奈何?’

皇权才是第一位的,纵使父子之亲也抵不过权力的诱惑。同时让国家顺利传承,也是对赵氏天下的恩德。只要明天赵佣顺利登基,成为大宋的第七位天子。这拥立之功,就是明明白白,谁也否定不了了。就算赵佣曰后心中还有着疙瘩,但终究还是不能恩将仇报。

为国无暇惜身,得利的又是太子,怎么也不能说他们是叛逆吧?

所以他们不怕急,只怕慢。

帝位传承上,就是抄家灭族的大祸。谁敢保证一点意外不出?就是万分之一的危险,也没有哪个宰辅愿意去冒上半点。

看着面前的几位宰辅,一张张急切的脸,再看看在后面肃然默立的薛向和韩冈,向皇后撒气一般的道:“罢了,罢了,国事就国事吧,这名声我也不要了!”

皇后终于松了口,蔡确、曾布等人大喜过望,赶着遣人去玉堂找值曰的翰林学士写传位诏书。

“事不宜迟,迟则生变。明曰早朝时,就让太子上殿。”

“衣服呢?”章惇突然想起了一件事,“太子还没有衣服啊!”

章惇的话听起来有些好笑,但每个人都明白他想说什么。太子的衣服多得穿不完,只是没做皇帝时穿的衣服。

登朝总要一身冠冕。

新帝登基之后,面见群臣,肯定不能再穿着亲王的服色。天子六服,六套用在各级典礼上的不同舆服。都做了皇帝,不能说衣服都不够在大典上穿吧?

“舆服怎么办?现在做还来得及吗?”曾布也慌了起来。

大宋开国一百多年,几次天下易主,新燕京是因丧即位,没有说老子还在,儿子直接就登基的。这内禅的礼仪,谁都没经验。现在忙得乱了套,要是赵佣穿了一身皇太子的服色上殿,让群臣朝拜,那就不是内禅,是笑话了。

连王安石在内,宰辅们都皱起了眉。所谓量体裁衣,这时候哪里来得及招裁缝来量了尺寸,再去裁剪、缝制。又不是秃驴们的一口钟【注1】,两块布缝几针就能穿上身了。那是天子舆服,耗费多少人工和心血都是应该的。

若是正常的父死子承,新皇帝就要在梓宫前麻衣素服,不冠不冕,甚至披头散发,捶胸顿足,以示心中的哀恸之意。三辞三让,不用急着穿上礼服。有足够的时间制作新衣。以一干专为天家服务的宫廷裁缝的手艺,要把天子所用的袍服都做好,也不过一两天的事。只是现在只剩下半夜时间,哪里还来得及?

“已经做好了一套。”皇后忽然说道。

殿中陡然间就静了。

刚刚还很兴奋的蔡确和曾布,就像是被冰水淋头,一下都僵住了。

韩冈都不由得直了腰,惊讶的望着向皇后。

“去年官家中风后,清源郡太夫人入宫,劝吾说官家得天佑,当无大碍。但事有万一,还是稍作准备,缓急间也不会误事。这话吾觉得是有些不好听,可终究是忠言,所以吾就让人去做了一件预备着。”

皇后惴惴不安,毕竟方才还议论得热闹呢,现在却一个个都噤口不言,瞪着眼,让她也感觉有些害怕。

不过皇后解释了之后,冰结的空气就缓和了下来。

宰辅们都反过来看蔡确。蔡确虽为宰相,可惜年资浅薄,尚未得封国公,依然是清源郡公。而清源郡太夫人,便是蔡确的老母明氏。

什么时候就打了钉子下去?皇后不说,还真是不知道。

韩冈看蔡确的表情,似乎也是吃了一惊。估计是没指望他让他母亲出的提议,会被皇后听从。

“先不管那么多了,得赶快找人取来。”蔡确有点狼狈的大声道。

章惇抬手压了压,示意稍安勿躁,问向皇后,“殿下,做好的是哪一套?”

“通天冠、绛纱袍,哦,还有一套赭黄袍的常服。不过这半年,太子又长高了点,可能穿不下了。”

张璪松了口气,“那就是两套了。”

韩绛皱眉道:“至少应该做一套履袍才对。就仅仅是绛罗袍,绛衫袍也行啊。”

正式的大礼服,是大裘冕和衮冕——玄衣、纁裳,以黑色外袍和赤黄色衣裙为主色。次一级的礼服,是通天冠、绛纱袍,是正红色。再次一级,为履袍,黑革履和绛罗袍,衣服颜色依然是正红色。直到作为常服的衫袍、窄袍,才有赭黄、赤黄、浅黄为主色的袍服,同时依然有绛色的衫袍。

所谓明黄色的龙袍其实于古礼不合,在等级比较高的典礼上,并不会出现。只不过一年三百六十天,能有大典礼的次数实在不多。绝大多数时候,还是以衫袍、窄袍为主,天子服黄的情况也就很常见。

“有赭黄袍也够了。也不是没故事的。”

蔡确没明说出来,但大家都明白。

没错,的确是有先例——太祖皇帝。陈桥驿黄袍加身嘛。天下谁人不知?

天子六服中,通天冠、绛纱袍排在第三等,履袍则是第四等的礼服,赭黄袍则更差一级,前两件在登基大典上都能用得上。但勉强点,赭黄袍照样能用。反正只要主要的礼服不出问题就行了。

这就跟结婚一样,连鞋子带衣服,要一套套的换。没有说一件从头穿到尾,其中错一点也没什么。只要接亲和拜天地时的衣服不错就行了。

朝廷的大典礼,只要皇帝主持,基本上都是一身接着一身的换。虽说是礼制,其实也害人。去年郊祀,如果赵顼坐在玉辂上时,也穿大裘冕,或许就不会中风,毕竟外面还有一件黑羊皮的大裘可以防风御寒。可惜按照礼仪穿的是通天冠,外套一件绛纱袍,从里到外透风。

不过现在是有什么就用什么了,还能有什么挑拣?

“事已至此,也只能先凑合着用。赭黄袍就赭黄袍。有通天冠、绛纱袍,御正殿时能用了,其他时间,则穿赭黄衫袍。”

“那大小呢?”

“改衣服容易,量了尺寸,在这里就能改!”

“量尺寸……得将太子请来了!”蔡确连忙道。

忙了一通,这才发现,他们将主角给忘了。

现在就要内禅,主角不能不来!

