\section{第14章 辘辘尘道犯胡兵(中)}

“也就是洒家,换个别人也不会这般卖力。”王舜臣从嘴里扒出根鸡骨头,看了两眼,又丢回嘴里嘎嘣嘎嘣的嚼起来,“日他鸟的。洒家看陈举不顺已经很久了,韩秀才你让他吃了个大亏,洒家看着煞是痛快。军器库一案,有没有人告诉秀才你,陈举为了赶在经略相公回来之前结案花了多少钱吗?”

韩冈点了点头,“八千多贯!”顿了一顿,又强调道:“铜钱!”

北宋铜钱不足,铜价又贵,而且多产于东南。万里迢迢运送到陕西、蜀中十分不便,所以许多时候,两地都是通用铁钱。铁钱的价值远远小于铜钱,官价有时是一比二,更黑一点的则是十比十二,但在民间,多是三四枚铁钱才能换一枚等大的铜钱。

“八千贯铜钱!”王舜臣摇头叹着,“陈举那厮,单是收买州中官员就用了八千多贯铜钱,补充军器库亏空又费了万多贯,还有安顿黄大瘤的家眷又是一大笔。韩秀才你在德贤坊射出的三箭,让陈举不是出血,而是大块大块的割肉啊……”

韩冈苦笑着点了点头,这也是为什么陈举将他视为死敌的缘故,而他也因此绝不会奢望能与陈举达成谅解和妥协。不过陈举一次过拿出了两三万贯钱钞,将自己的家底摊在了阳光下,连王舜臣都知道得一清二楚,秦州这么多官员,韩冈不信没人会对此动心。只不过他们近期内很难有动作,韩冈也等不及陈举在秦州被人连根铲除的那一天。

不想再提陈举之事,韩冈转而问道:“不知军将是哪里人氏?”

王舜臣回得爽快:“洒家是延州人。世代都是吃兵粮的,不比你们读书人光彩。”

韩冈奇道:“既然军将出身延州,不在当地投军,怎么到秦凤来的?”

王舜臣沉默下去,神色在跳动的火光中变幻不定,最后猛然仰脖灌下一口酒,将酒气化作憾然一叹:“若不是犯了事,洒家现在应该在绥德城啊……”

绥德……

韩冈还记得陕北有句俗话叫做‘米脂的婆姨绥德的汉,清涧的石板瓦窑的炭’。可在此时,瓦窑堡此时尚未修筑,米脂在西夏人手中,青涧城被宋人控制。而绥德,一直都是党项人的控制区,直到三年前西军名将种谔用计逼降了当地的守将嵬名山,方才占据了绥德。

位于无定河边,横山深处的绥德城,是控制无定河流域以及附近百里横山蕃部的核心所在。种鄂夺占绥德就如将一枚钉子钉进了横山,让宋军的控制区向着西夏的腹地拓展了一大步。

“若不是犯了事,洒家何必避到秦州来?若有五郎照拂,过两年也该升做殿侍,等再立些功劳,升做三班何在话下【注1】?……洒家的老子曾在种老太尉帐下行走,守过青涧寨,筑过细腰城,倒是洒家生得晚,没能得见老太尉的威仪。”王舜臣说起他父亲曾经跟随过的种老太尉,在面上闪过的憧憬和仰慕的神色,在他身上实是难得一见。

“军将说的种老太尉可是种公世衡?”

“这天底下哪还有第二个种太尉?!如今打下绥德的五郎也当不起太尉二字。”

韩冈至此方是恍然:‘原来是鄜延种家的人,难怪气魄如此。’

王舜臣说的老种太尉,就是十几年前去世的关西名将种世衡。也是如今鄜延将门种家的前任家主。种世衡是真宗朝著名隐士种放的侄子——既然是著名,那所谓的隐居其实也便不过是做做样子,终南捷径这句成语不仅是韩冈,此时的人们也都耳熟能详,在终南山做隐士只可能是为了做官——不过当其时,世称隐君的种放深得真宗皇帝的宠信,名位颇高。

等种放去世之后,由于其无子,便由种世衡这个侄儿受了恩荫,入了军中。种世衡在关西为将数十载,战功卓著,范仲淹向朝中推荐陕西将官时,将种世衡列在第二位,而第一位便是狄青。欧阳修也曾上书说,‘臣伏见兵兴以来,所得边将,惟狄青、种世衡二人’,都是把种世衡和狄青狄武襄视作同一等级的将领。

只是种世衡的官运远不如最后当上了枢密使的狄青。他名声虽响,可名位却不甚高。虽是关西人称种老太尉,但终其身也不过一个正七品的东染院使,离横班这等高阶将领还有七八级,离真正的太尉之衔更是十万八千里。称横班是太尉,那是世间的习俗,就像将民间将经略使称为经略相公。杨文广能称太尉,因为他曾为秦凤路兵马副都总管,而种世衡无论从品级还是差遣上都是远远不够资格。

韩冈前身是士人,对名位高低而带来的不同称呼有着天然的敏锐,在他的记忆里,从没有以太尉之名来称呼种世衡,一声世衡公已经是很恭敬了。但现在是跟崇拜种世衡到五体投地的王舜臣说话,称呼一声‘太尉’也是理所当然。

“后来老种太尉故了,大郎去京中告御状又犯了事,洒家的老爹就跟着五郎,不过前两年病死了。洒家是自小跟着五郎的儿子十七哥儿,只是今年年初酒后恶了个鸟官的衙内,逼得洒家在延州站不住脚,不得不到秦州避避风头。吴节判曾在延州监酒税,跟五郎交好,洒家便投到了他门下。”

韩冈并不清楚种家内部的排行,但王舜臣既然说种五郎现在正驻守在绥德城,那定然是种世衡诸子中,最为有名的种谔。王舜臣与种家因缘不浅,若能拉好关系,日后也多一条出路。至少韩冈可以确定,直到北宋末年,种家在关西依然是武臣名门之一——因为有留名千古的种师道。

韩冈为王舜臣将酒斟满:“令尊既久随老种太尉,功绩当不在少数,难道没能给军将留下个荫补?”

王舜臣又一口将酒灌下,愤愤道:“鸟荫补,轮也轮不到指使的儿子头上,洒家的爹又是死在床上的,哪有那个命!”

一个指挥使,如果是禁军中的上四军——天武、捧日、龙卫、神卫——指挥使,好歹一个从八品的大使臣。但若是驻泊禁军的指挥使,恐怕连品级都不会有。但要想荫子为官,上四军指挥使都不够资格,请先升到从六品!当然,还有另外一条路,那就是战死在沙场上,作为抚恤,朝廷也会录用一两个儿子。王舜臣的老子两样都没有,当然荫补不了。

韩冈笑着劝道:“算了,以军将之才,入官也是迟早的事。”

王舜臣哼了一声,“你们措大就是会说好听的。一点实诚都没有。”

韩冈笑了笑,丝毫不以为忤。只是他心中有些奇怪,种世衡死在二十四年前的仁宗庆历五年【西元1045】,王舜臣说他那时还没出生。难道他现在才二十出头?韩冈有些吃惊的看着王舜臣的侧脸,那一张毛茸茸的大胡子脸,横看竖看也有三四十了!

王舜臣低头摇着酒水,突然叹道:“还是找个好根脚有用。秀才你跟着横渠先生,怎么着都能考个进士,不比俺们厮杀汉,拼死拼活也不定能混到一个官身。”

“说是弟子,韩某投到先生门下也不过区区两年,难得先生教诲。”韩冈也叹着:“真要说起根脚,韩某不过是灌园出身。若非如此,怎么会被陈举、黄大瘤之辈所欺?”

王舜臣抓了抓头,“管他时日短长,学了一天也是学。不是有说法叫朝什么死的……”

韩冈笑道:“可是‘朝闻道,夕死可矣。’”

“对!对!就是这句。十九哥说过几次洒家都没能记住。”王舜臣今天不知叹了多少次,“当年老尚书的文章连真宗皇帝看着都喜欢,到了老太尉时,便弱了许多,现在传到第四代,也就七郎家的十九哥算是有文有武。洒家跟着的十七哥在文事上还差一点。”

老尚书说的是隐君种放,他死后追封的官位是工部尚书。他算是第一代,种世衡第二代,如今关西军中有名的三种——种诂、种谔、种诊,也就是王舜臣方才说的大郎、五郎还有个没提及的种二郎,是第三代;而现在王舜臣说的十七哥和十九哥则是第四代。但种师道是第几代?也许是第五代吧,韩冈猜测着,若是能打听到这位日后的名将的下落,有机会自当多亲近亲近。

“不知军将说的十九哥大名为何?若是上承隐君之才,日后一个进士当是探囊取物。”韩冈问道。

“咦,秀才你不认识吗?十九哥正是投在横渠先生门下,与秀才你应是同学的!”王舜臣因酒水而变得有些恍惚的眼神突然锐利起来,“韩秀才你既然也是横渠先生的弟子,应该不会不认识罢?!”

韩冈的呼吸有那么一瞬间停滞,这王舜臣真是不简单,心思细密得与外表完全相反。一番话弯弯绕绕,竟然是在探他的底子……幸好他还是继承了前主的记忆,而那一个韩冈的的确确正是横渠先生张载的弟子。

“也是在先生门下吗?种……种……”韩冈轻轻念着,一个陌生的名字从幽深的记忆中跳出水面,他眼睛一亮,“种建中!军将说的十九哥可是种建中种彝叔?!”

注1:军将、殿侍和三班都是指得宋代武臣的阶级,相当于现代的军衔。这些军衔都是属于没有品级的低阶武官。从高到低为:三班借职,三班差使,殿侍,大将,正名军将,守阙军将。王舜臣现在的阶级为正名军将。

ps:一个历史名人终于露头了,虽然要等他正式出场还有一阵子。各位可以猜一猜这位究竟是谁?提醒一句,现在的名字不是他日后的名字。其实百度一下就能知道。

今天第一更,照例求红票和收藏。

