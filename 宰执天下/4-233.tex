\section{第38章 岂与群蚁争毫芒(六)}

已经是一年之中最热的一段日子,雨水虽然不少,但挂在天上的艳阳,依然是炽烈得能将地面晒得裂开来。

船舱两边的窗户敞开着,两岸的堤坝、草木清晰入眼,就是感觉不到一星半点来自于水面上的凉风。

随行的伴当给韩冈打着扇,但照样还是热,如同蒸笼一般。

韩冈已经怀念起京城的日子,到了夏天,半年前存放在冰窑里的冰块,就能拿出来用了。可惜襄州冬季无冰,否则以韩冈的身份,在船舱里放上十几桶冰块来降温还是不难的。

离唐州已经不远了,堤岸上的行人也多了起来,最多半日功夫就就能抵达。

从五月开始动工,方城垭口的轨道已经快要修成了。韩冈在家休养了没几日,就又要离开襄州,前往唐州。完工在即,他这位主事者在情在理都得去一趟,总不能就此袖手不理。

不过配套的设施还没有修好。轨道两端连接的都是运河,要将货物从水上转到陆上,再从陆上转到水上,两处的港口要有一年转运两百万石到三百万石的能力,眼下只能运送筑路原材料的码头运力当然是远远不够。

在沈括主持堰坝、船闸等工程完工前,方城垭口轨道至少要撑上四五年的样子,码头上的建设自然不能偷工减料。仓库、栈桥什么的,都得修好。想要投入使用,大约得等到九月底的样子。

“船已经多起来了。”方兴在韩冈身后说着,转运使的勾当官透过船舱敞开的窗户,望向水面。极目一望,汉水之上,大小船只已是数以百计,“当是听说襄汉漕运快要开通了。”

“商人若是耳目不灵,又怎么做生意。”韩冈轻笑了一声。

他的表弟可是早在李信任职荆南的时候,就将顺丰行的招牌挂到了襄州来。虽然一直以来,摆在外面的只有个小门面,但当韩冈将有意重启襄汉漕运的想法在信上说过之后,冯从义预计到了漕运畅通之后的情况,便立刻在襄州城外的汉水边买了十几顷地,准备修建库房。

不过这一片地离着后来确定要扩建的码头位置稍远,虽然有些让人遗憾,可不招忌讳也算是个好处。且只要将轨道一建,也不会比码头边的库房差到哪里。

另外一件让韩冈很满意的就是冯从义不仅仅是一家赚钱,还拉了一批陇西豪商过来一起置地。秦凤、熙河的都有,甚至还有几家有钱且有见识的蕃部,要借着襄汉漕运这个东风,将势力在荆楚之地扩张开来。陇右、京城、广西还有荆楚,随着韩冈的步伐,一个名为雍商的团体,也正在逐渐形成之中,并逐步扩张着势力。

“方城山挡了襄汉漕运百多年,人人望之兴叹。如今能有畅通无阻的一天,全都是龙图的功劳……”韩冈正想着雍商集团未来的发展,方兴则在一旁将马屁拍得兴致高昂。

“好了。”韩冈摇摇头打断他的奉承,“还不到庆功的时候,等到冬天,第一船粮食运抵京城,才算是初见成效。”

方兴躬身受教,韩冈指着外面的民船,“漕运开通是九月底。而到十一月中旬,京畿的河道就要上冻了。三十天到四十天的时间,能运送多少纲粮入京,就决定了这一次能收到多少功劳。襄州眼下的纲粮有一百一十万石,这是定例要送到扬州走汴河的。但等到秋天完税之后,还将有六十万石入库。旧年也是同样要运去扬州,等明年开春汴河漕运重启之后,一并送入京中。今年就不一样,有了襄汉漕运。能通过襄汉漕运从这六十万石里面送多少入京,就看你的本事了。”

韩冈是将今年漕运发送的工作交给了方兴全盘处置,方兴明白机会难得:“龙图放心,下官一定竭尽全力。”

韩冈点点头,“这件事交给你我也就放心了。办得好的话,明年我荐你入襄汉发运司,也便是顺理成章。”

方兴用力的点头,脸上带着兴奋和期盼。像他这样的没有一个进士出身的官员,想要从选人转官已经千难万难,再想从京官晋身朝官,那就更难了。别说是韩冈的幕僚,就是宰相的幕僚,都少有机会能转官。已经三十多岁快四十岁的的人了,一造青云的机会就在眼前,哪里可能会放过?

知道方兴不可能会懈怠,韩冈也就不会费口舌。辅佐他韩冈开通襄汉漕渠的这份功劳,足够方兴晋身朝官行列了。为了主持襄汉漕运,朝廷肯定要成立一个新的发运司衙门,方兴虽然远不够资格担任发运使——沈括肯定够资格,就不知道他愿意还是不愿意——但新晋的朝官充任发运判官,只要加个权发遣的前缀,也能勉强够得上。

韩冈望着舷窗外的粼粼水光,“襄汉漕运一通,荆湖、京西的户口自当日渐增多,两广藉此也能与中原联系得更加紧密。看着只是条补充汴河水运的漕渠而已,但实际上,却事关天下的百年大计,不得不慎重。”

韩冈眼光之长远,早已将方兴慑服,他很郑重的再次行礼:“下官明白,一定会慎重小心,尽心尽力。”

在船舱中没有热多久,韩冈所乘的官船便到了唐州城外的码头上,事前得到通报,沈括已经出城来迎接,正站在栈桥上。

“劳烦存中兄久候。”

韩冈下了船便上前行礼,一起一拜,却对沈括脸上的新伤视而不见。想来沈括也是希望所有人都不去关注他家后院葡萄架子的事。

“玉昆,襄汉漕渠这么大的事,你可是放得开手!”沈括笑着抱怨,“在这栈桥上等着你到没什么,隔几日就要帮你跑一次方城县,可是马都跑瘦了。”

“能者多劳。”韩冈笑笑,又疑惑的问道,“方城山中的轨道,按部就班而已,又无大事,存中兄怎么数日一去方城?”

“山洪难道不是大事?”沈括反问。

“难道是坏了堤坝,还是毁了道路?”韩冈随口问着。其实看到沈括脸上的表情就清楚了,要是当真发生这等情况,沈括不会这般轻松。

“七天前,唐州暴雨下了两日,方城山山洪直泄而下。方城垭口中的溪水暴涨,差点就淹上了堤坝,幸好两座木桥修得坚固,在水中纹丝不动。雨停后的几日,桥下洪水滔滔,而桥上照样是车轮滚滚,一点也没有受到影响。这是李诫的功劳啊……”

韩冈听着沈括的介绍,满意的点着头。唐州紧邻襄州,暴雨山洪的消息早就收到了。当时韩冈还提着一份心,现在看来,还是多虑了。

经受住考验的当然不是当初韩冈与李诫说定的石拱桥。从山里采石,再运来修起,就算只是数丈跨度的小桥,以此时的工程技术水平,也要一年半载的时间。

轨道对于襄汉漕运来说,本来就是暂时性的替代品,最终还是要修建水道,让船只可以从襄阳直抵京城。在韩冈的计划中,也只是让轨道从矿山和码头进入道路交通的范畴,同时尽快打通襄汉漕运。

所以最后还是决定使用木质结构的桥梁。李诫带着几名大工匠绞尽脑汁的去设计,最后造出来的木桥,虽然是拱桥的形制,但桥面的坡度足够平缓,比起汴河上常见的高拱如虹的虹桥,更适合有轨马车的通行。

两座新建的木拱桥通过了洪水的考验,而这段时间每日都有大量的原材料从桥上通过,最重几乎达到三万斤的有轨马车,木质的桥梁也承受了下来。日后改运纲粮,也就没什么好担心的了。

韩冈听沈括说着前些日子的山洪,一起往城中去。

进了城,韩冈突然想起了什么:“对了,襄州的港口正要扩建,漕司也需派人去配合州中。只是韩冈身边人手不足,不知存中兄可有何推荐。”不待沈括提名,韩冈跟着道,“存中兄家学渊源,想必博毅的治事之材也是极好的。”

沈括的脸上有些尴尬,他的长子博毅,前些日子被张氏找了个借口赶出家门,不得已安排在府外居住,时不时的还送些钱过去。但这件事给张氏知道后就是不依不饶,当着儿女的面大骂沈括。

韩冈眼下指名长子博毅作为他的幕僚去襄州,肯定是知道此事后,帮他一个忙。就在马上向韩冈行了一礼,却不再多说什么。

韩冈平平淡淡的点了点头,也不提这件事了。他虽然身在襄州,但耳目还留在唐州,总不会对闹得这么大的事情毫不知情。

自家的私事,闹得远近皆知,沈括免不了有些尴尬。静静的陪着韩冈走了一段路,才忽然指着前面一排楼阁——那是唐州城中的驿馆,“新近就任信阳军的范尧夫刚刚到了唐州,正在驿馆之中,不知玉昆你见与不见?”

“存中兄是说笑吧,去信阳军怎么可能会走到了唐州来?难道范尧夫迷路了不成?”韩冈虽是这般说,但也明白沈括就是说笑,也不会拿着毫不相干的范纯仁来开玩笑。

可范纯仁要想上任,从颖昌府【今许昌】直接南下就行了,经过蔡州就是信阳军,这一条路几乎就是向南的直路,有必要走唐邓,多绕个几百……不对,韩冈摇摇头,这一千里都有了。

韩冈想不明白,到底有什么事情,值得范纯仁绕这么远的路?

