\section{第31章 风火披拂覆坟典(一)}

春日的阳光熏得人昏昏欲睡。

向太后在半睡半醒间猛然惊醒,发现自己仍在金明池中的水心殿内。

“什么时辰了?”

王中正侧脸看了一下一旁的座钟,钟面上的短针斜斜向上,而长针平直的指着左手的方向。

“回太后的话,已经巳正三刻了。刚才章相公派了人来问,得知太后还在休息就回去了。”

“章惇派了人来催了?!”

向太后一下子全醒了。

方才她半睡半醒的时候,隐约感觉听到整点的报时声,却不想起来时已经都十点多了。

她偏过头,看了一下放置在殿内的巨型摆钟,果然,钟面上指针的位置的确已经快要接近黑色楷体的午初和子初了。

两年前,苏颂发明了时钟。随着朝廷的重视——尤其是对其中利润的重视——以及制造的难度较低,时钟已经在大宋境内普及开来。

起居需要时钟,比赛需要时钟,上课需要时钟,人们的日常生活都离不开对时间的把握。也许大部分人,对此的需求还不迫切,但这个经过改进后,不仅能够显示时间,还能够按时按点的以钟鸣报时的机器,实在是太合乎富贵人家对奢侈品的需要。

时钟一开始的设计,是按照时辰的初、正,将钟面分成二十四个格子。但韩冈说分得太细,很容易看不清楚,白天晚上也没人会弄错,所以还是改成了时针半天一圈。子正、午正放在最上面,卯正、酉正则在最下方。

之后韩冈又说很多人不识字,而且钟面的同一刻度上,还得标上早晚两个时间的文字,不如数字直观,所以又改成了数字,早晚各十二点。在钟面上,草码数字绕了一圈。韩冈的权威,压制了所有的反对声,现在市面上对外发售的时钟,大部分都是数字钟面。

不过皇宫中的时钟,钟面上还是以地支为标识。宫中有那么多能工巧匠,螺蛳壳上都能刻个道场出来,何况那么大的钟面。

可是不管怎么样,数字总比拗口的天干地支要容易说容易记,即使这是所有人从小到大都在使用的计时法,一旦习惯了数字计时之后,就立刻会觉得子丑寅卯十分别扭。

向太后也觉得一二三四更方便一点,在她的寝宫中,大部分的座钟还是数字钟面,只有最大的两三座,宫中的能工巧匠将之做成了十二时辰的模板。

“时候差不多了,也别让人多等。”

向太后说着,抬起了手,贴身的宫女轻手轻脚的将她给搀扶起来。

一群人拾级而上,来到水心殿的三楼。

推开门扉,走到殿外,华盖就在向太后身后打起。

凭栏而望,春日的金明池便展现太后的眼前。

水面波光粼粼,闪射着和煦的阳光,一条金光闪闪的楼船从水面滑过,切开了闪烁的湖水,留下两条越荡越开的水线。

“官家还在御舟上?”太后问道。

王中正看着船上的天子旗号,点头道,“官家就在船上。”

水心殿位于金明池的正中央,通过一座拱桥与岸上相连。

水面上,龙舟已是蓄势待发,另一侧,标旗已经树在了水中央。

看到了黄罗伞在水心殿上张起,原本只是蓄势待发的鼓声,陡然间激昂起来,高亢的直入湖水深处。

金明池争标的鼓声,从城外的池水上空,传到了皇城之中。

“终于是开始了。”韩冈看了一下房中的时钟,比起预定的时间要迟了快三十分钟,“章子厚在那边多半是等得急了。”

厅中另一边,苏颂听了便笑道:“本来不是该玉昆你去的吗?都推到了章惇身上。”

“章子厚是劳碌命,事情当然由他去做,我就轻松点了,只是集贤相嘛,大事小事都插手,不是乱了规矩。”

作为首相,章惇负责太后出巡的一切事物,韩冈就留守在皇城,处理章惇丢下来的大事小事——不是郊祀、明堂这样的大典,有一个宰相带着大部分臣僚,陪太后天子游赏就够了,没必要所有宰辅都上阵。

“何况年年都是一个调子,次次都如此,都不嫌腻味。”

韩冈带着些嫌恶的说着,金明池的游乐活动,实在没有新义,看了几次韩冈就厌烦了。这次有机会可以推脱,就让章惇去了。

“去年开始可就有车船了。”

“等明轮船早出来再说吧,用人力只能在湖水上逞逞威风,只有用上蒸汽,才能入海。”

车船,就是用脚蹬踩来驱动船只前进,再进一步就是明轮船了。这的确是个进步,但上船的蒸汽机还没有着落,韩冈还是没多大兴趣。

“也快了。不要急。”苏颂安抚道。

为了蒸汽机,宋辽两国的争相悬赏,的确有了效果。

前年年终的时候,可以用来抽水的蒸汽机终于被制造出来,在韩冈的大力推动下,蒸汽抽水机被大量制造。大批的抽水机,已经用在了大宋的矿山中,尤其是煤矿,可以就地使用煤炭资源。更有许多抽水机被大户买去,用在了自家的深井中。

因为这一个进步,朝廷践行承诺,拿出了一个八品的武职,同时以购买专利的名义,从国库中支出了整整一万贯。

而辽国那边,耶律乙辛也封了第一个节度使,而且是有头下军州的节度使。他所得到的蒸汽机,通过细作的回报,还不能装在船上,但用来抽水,应该是没问题了。

两相比较,的确辽国更大方一点。所以河北有点水平的工匠,现在都被集中到了京师。

倒是没人管那一堆打算偷渡到辽国那边的士子。

很多士人都以为耶律乙辛大事悬赏工匠是为了千金市骨,可惜他们错了。到了辽国那边,耶律乙辛只让他们去教授蒙学,完全没有他们臆想中的朝为田舍郎、暮登天子堂的风光。

边境线上,两年前每天都有人被抓到,去年耶律乙辛封出第一个节度使,又是一泼狂潮,到了今年,辽国那边的消息传回来情况才好一点。这两年的时间,逃过去多少人不清楚,但发配去了西域和云南的士人,超过了八百。

“希望我们这边手脚能再快一点,让耶律乙辛多拿出些肥肉来,也能多勾走些蠢货。”

韩冈笑着说道,心中还是有些浮躁。

他对良性的竞争一直抱着鼓励的态度,但他对于大宋工匠的不争气,倒是觉得不耐烦了。这都多少年了?

“就算辽国先有了船用蒸汽机,数量太少也是没用。”苏颂说道。

“这倒是。”韩冈点头。

大宋这边的蒸汽机已经大批的投入使用,但辽国那里,据信报还只是耶律乙辛手中的大玩具,并没有像大宋这边大规模制造,更没有投入使用。也是因为成本太高,用不起的缘故,也有可能等待性价比更好的新型蒸汽机出来。

不能投入实际使用,蒸汽机就只是了大而无当的铁家伙。不能大规模生产,蒸汽机也只是精致的玩具,更无法实现从农业国家向工业国家的转变。

大宋这边的蒸汽机制造,已经是工厂化、规模化了,培养出了大批的工人和工程师,一旦有了新型的蒸汽机,立刻就能转产。

即使辽国那边先一步发明了新型蒸汽机,发明了能够使用蒸汽驱动的船只,宋辽两国之间的差距,也只会越拉越大。这是国力和意识上的差距。

“比起蒸汽机,我更在意什么时候有人能过来拿了天文钟的悬赏。”

“一天误差一分钟?这可不容易。”

“但这条路肯定是要走的。”苏颂看了韩冈一眼,“就像蒸汽机一样。”

韩冈笑着点头,“的确如此。”

时间的精准化对工业的发展有着难以估量的意义。

精确的齿轮是时钟制造的关键,各式模具不知制造了多少。但铸造完工之后,还需要匠师进行加工调整。而且也不是所有的齿轮都能调整到完美的精度上来。只有百贯以上的贵价货,才能得到最好的零件,以及最用心的调整。

普通七八贯、十几贯的座钟,一天下来,差上十分钟八分钟都十分正常。必须每天中午通过日晷来进行纠正。

而官营时钟厂所出品的高档货,一天超过五分钟那肯定是出故障了,两三天不用校对时间都没太大问题。

不过皇城中的时钟,误差也差不多是这个水平,最好的也只能达到三分钟的误差。看起来已经是到了现有设计的极限。在韩冈看来,只能通过采用更新的设计,才能更进一步减小误差水平。

苏颂对韩冈的看法也表示同意。而且误差在三分钟的时钟,想用在他所喜欢的天文事业上,还远远不够。所以在苏颂的倡议下,朝廷正在悬赏能够提供更为精确地天文钟的设计。

“有了座钟,现在出一炉铁,不需要大匠一直站在炉子前面盯着了。工厂中制作生产计划表也能更精细了。有了更好的天文钟,就能够更好的编订星表。上次还是玉昆你说的,只要星表编订成功,行星运动的定律,就可以从中进行归纳总结,最后化为代数方程。”

苏颂卸任已经两年,这两年里,他并没有离开京城。

朝廷先是以宫观使相留,之后又依王安石旧例,改任平章军国重事,让苏颂继续留在朝中。

不过苏颂担任平章军国重事之后,并没有干涉政务,依然是做自己喜欢的事。机械、天文上的发展,都是他在背后推动着。

前些日子还说动太后,更改太宗皇帝开始便颁布天下的禁止私习天文的诏令,凡以天象妄说休咎,一律流放万里,但私下研习天文星象已不再入罪。

这是一生中最幸福的时候,苏颂越来越这么觉得了。

官位已经到了人臣之极,而个人的兴趣爱好,又与国是相合。

兼济天下和独善其身时要做的事,现在都正在做着,还有比这更幸福的事吗?
