\section{第34章 为慕升平拟休兵(11)}

忻州的春风醉人,从五台山中流淌出来的淙淙溪水,滋润着山下新近被翻耕出来的田土。

遭受过辽军的肆虐的村庄,时隔多日终于有了炊烟。阡陌纵横,第一批被安置下来的代州百姓,住进了草草修葺的房屋,同时也将春小麦的种子撒进了地里。

站在田头的农民们,望着这一片失去了主人后分配给他们的土地,眼神中有隐藏不住的忧虑,有远离家乡的迷茫,但更多的还是对未来的期盼。

虽然说到底在这不合时节的播种中,有多少种子能发出芽,并顺利抽穗长成,许多人心中都没有底。可是看到这些日子来,官府井井有条的安排,不折不扣达成的承诺,加上毫不悭吝的赈济,对未来并不再是全然的绝望。

陈丰的脸上有了淡淡的笑容。并不是为了重新开垦出来的土地,而是为了人们脸上的希望。这段时间以来,他不知看到了多少张绝望的面容,也不知看到了多少麻木到失去了表情的脸庞,来自于北方的侵略者,在这片土地上造成了无数的悲剧,这些许希望,比珍珠还要珍贵。也是对他的工作最好的证明——相对而言,那些词不达意的奉承和讨好,倒是让他听得生厌了。

‘河内凶,则移其民于河东,移其粟于河内。河东凶亦然。’这是春秋时代就知道的遇到灾荒时该如何做的手段。韩冈的做法一脉相承,但实际上要做到做好,可没那么容易。

韩冈不是宣抚使,按道理是不能干预地方民政,所以安置流民的工作按道理是不便插手的。聪明的人也不会插手,当出了问题的时候,也就不用担负责任。但韩冈还是派了陈丰等人巡视各处安置了代州流民的村庄,尽量配合甚至主导安置工作。

这是很蠢的做法,也并不合乎官场上推诿责任的习惯——看忻州知州贺子房听说制置使司打算插手流民安置时兴高采烈的样子就知道,这么做的到底有多稀有——但当这一回看到流民们彻底麻木的脸上多了一丝希望的时候,陈丰终于觉得这个决定或许并不是那么的愚蠢。

不过陈丰心情轻松一阵后,又重新沉重起来。他可没有多余的时间耗费在这里。因为他刚刚收到从前线传来的公函,让他和在太原的田腴准备好营地,来安置南下的阻卜部众。

陈丰十多天前才从南面才被调上来,不过之前也是一边处理粮草转运,一边帮着安顿太原府的难民。太原府的差事刚刚告一段落,便被调来接手忻州,这一下子就更是忙得脚不着地了。

陈丰现在算是明白了,做了枢密副使的幕僚,的确是如同踏上了天梯。但想要再往上攀上去,就要付出过去担任地方官员时几倍甚至十几倍的努力。

由此得来的回报,与其说是韩冈的恩赐,还不如说是自己努力的结果。韩冈所给的,只是一个机会。

并不是说陈丰他不感激韩冈,但在韩冈门下,时常会感到难以适应他的行动风格。还有韩冈对工作投入的态度,也根本不像是一名重臣,这也让他难以理解。但付出就能得到回报,已经让陈丰觉得很是满足了。

骑着马在田头绕了一圈之后,勉励过负责代州流民安置的几个官员,再听了了从流民中推举出来的乡老们的汇报,陈丰便又匆匆赶着回忻州城。

这一路上他已经尽可能加快速度,但回到忻州城时,依然到了次日午间,从前一天中午到第二天中午,六十多里的路程,竟然走了整整一个白天——春天时翻浆的道路,耽搁了他太多的时间。

冬天时,黄土夯筑而成的官道被冻得结实如钢。道路下的冻土融化,硬实板结的路面变成了烂泥塘,马车开过去,就是两条深深的车辙,然后车辙便被泥浆水给填满。有的路面表面上只是一小滩积水,但积水之下,就是一个深可没人的巨坑。这样的道路不比冬天的河上冰面更安全。

这个其实是困扰支援忻口寨的整条补给线的最大问题,大量马车因此而毁损,同时粮草也因此受到大量污染而损耗。为了整修道路,整个河东路投入了大量的人力,甚至不比参与到粮草转运中的民夫少多少。沙砾、卵石等修路的物资被消耗一空。但整条补给线,依然是如同一根被抽紧的细绳,随时有绷断的可能。

而在数以十万计的民夫被调动的情况下,河东路还要进行春耕补种的工作,还要保证没有受到辽贼侵略的军州能够正常生产、收获。陈丰都有些庆幸了,幸好韩冈不是能兼理民政的宣抚使,否则以他的性格,肯定是所有的责任都担负在身上。那样的情况下,他们这些幕僚恐怕逼得上吊的心都会有。

还好眼前只有这些事。

不过当陈丰赶回忻州城设在北口仓边的临时衙门时,连喘气都还没停下来,就又发现自己多了一桩差事。

“白都监在定襄那边要粮?”

“是的。白老都监说,他马上要入山剿匪,要好携带又能立刻吃的干糒,不要米、麦。还有酱、醋,也要好携带的干货。”

干糒就是干粮,米面做熟之后打制成型并晒干,可以边走边吃。而便携式的酱料和醋,则是用布匹浸透了之后晒干,然后剪成一片片,吃的时候,将碎布丢下去就行了。说起来简单,但要做起来就很费事了。不过陈丰这边早就有了准备,只要直接调拨过去就可以了,可是这些干粮是忻口寨那边需要的。

“白都监遣了谁来传信?”陈丰问道。

“是白小衙内,正在州衙那边。”

“白小衙内?是白昭信?”

“对。”一人嗤笑道,“就是前些天,拿了百多个山贼的首级来忻州耀武扬威的白小衙内。这一回又转到定襄县去了。”

白玉是陕西宿将,在熙宁初年就已经为都巡检,只是运气不好,几次大战都擦肩而过,没有立下多少功劳,最后被派来镇守河中府。

直到这一次辽军入寇,一下攻入了太原,作为长安东北门户、位于汾河谷地的南面出口的河中府的驻泊禁军,便成了第一支北上救援河东的陕西援军。

只是白玉这一回的运气依然不好,韩冈一直将他们留在了后方,甚至当大军北上忻口寨的时候,远比京营、河东两军更为精锐的西军,还是落到了一个剿除太原匪患的差事。

一提起白玉,这边的官吏就少有恭敬了。

跟在陈丰身边的一名亲信吏员冷笑了起来,“定襄那边都靠五台山了,山上只有和尚,哪来的贼?”

另一名小吏挑了挑眉毛,眯着眼笑,“贼秃也是贼啊……不过那是淫贼!”

“白都监若是能在五台大发神威,斩了大头斩小头,也算是还了文殊菩萨道场一个青天白日了。”

唐武宗灭佛,会昌法难。文殊菩萨这座道场受到了极大的打击,北魏时大寺三百六,兰若无数,到了唐武宗后,就只剩下位于深山中的三五座破庙。不过会昌之后,佛法重兴,到了此时,又是漫山遍野的寺庙和僧侣了。

韩冈不喜浮屠,有一个说法是因为他祖上是韩昌黎【韩愈】的缘故。另外,也就是佛道不两立,出自道门的孙思邈孙真人的私淑弟子当然不会与僧人多有瓜葛。什么药师王菩萨,那是贼秃往自己脸上贴金。为了迎合韩冈,他下面的官吏,都不会明着说自己多信佛家,甚至其中的大多数会有事没事讽刺几句。反正现在的和尚不守清规的居多,不愁没话说。

十几名官吏在下面吃吃而笑,陈丰的眉头则越拧越紧,忽的暴怒起来,“说够了没有!?闲得没事吗?给阻卜蛮子准备的营地位置选好了?粮草都准备好了?水源都确保了?药物和医工都安排妥当了?我之前的吩咐都措办完成了?!”他恶狠狠的扫过了一群愣住了的官吏,轰的一拍桌子:“什么都没做,还站在这里嚼什么蛆?!要是到了晚上事情还没准备出个眉目,仔细你们的皮!”

一群人卷堂大散,只剩下几个陈丰的心腹在他身边,你看看我,我瞅瞅你,不知陈丰为什么发这么大的火。

一个胆大点的低声试探,“管勾,你可是枢密的心腹人,那白玉左不过一个赤佬,连枢密都不待见他。何必为了他发这么大的火?”

“……这一位资历老、官品高,在陕西人脉又深厚,枢密能将他一丢了事,但他可是我开罪得起的?”陈丰摇着头,叹着气,“我可不想开罪这位老将。谁知道他能不能请出韩枢密的亲朋好友来为他讨个公道?何况我可没有进士资格,朝廷虽说右文左武,也不能让我这个明经骑到一名七品宿将的头上。”

陈丰说罢,瞥了几名亲信一眼,森然道:“白老都监的要求,必须不折不扣的完成。谁敢在其中作祟,就别怪我不讲人情。”
