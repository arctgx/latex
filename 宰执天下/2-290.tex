\section{第48章 一揖而别独骑归(下)}

“宰相家……”韩冈闻言一愣,向来脑筋转得快的他,竟然一下没反应过来。他有些愣愣的问着:“是王相公?”

王厚点了点头,“正是王介甫王相公。”

得到证实,韩冈心中顿时如怒海烈风,一片惊涛骇浪。想不到不过几年的功夫,他竟然让一国宰相、千古名臣都看上了自己。

但韩冈也只是心头一阵激荡,却没有受宠若惊的感觉,心中反而涌起一丝不快。不是对王厚、王韶的,而是对王安石。

在他看来。王安石这个做法有些不地道。要是真的看好自己,早就该请人做媒了,章惇就是现成的人选。到了现在他已经是七品朝官,在他头上的文臣,也就两三百人而已,这未免就有些势利了。而且还托王韶做媒,这不是逼着王韶不能再与自家结亲吗?

“处道,你是不是在王相公那里说了些什么?”韩冈突然问道。

王厚脸上的笑容一下僵住,干咳了两声,旁顾左右。韩冈摇头叹气,看他的样子,多半是不小心说漏嘴了,要不然就是没能在王安石面前糊弄过去。

虽然因为韩冈的前任聘妻,也即是王厚的表妹已经亡于时疫,两家暂时没有了姻亲联系。但这个消息两边都没有向外散播,韩冈甚至为此还告诫过自己的父母。外界都以为韩冈和王韶还是有着亲戚关系,所以这一年多来,韩冈身边也是清净得很,并没有人上门来做媒。

可王安石今次转托王厚带信,让王韶带他向韩冈提亲。要说他不事先打听一下韩冈有无婚配,那是不可能的。而王厚正好就在眼前,抓过来一文,就把底给露了。

王厚被韩冈弄得有些尴尬,不快的问着:“愚兄是来问玉昆你的想法的。”

“父母之命、媒妁之言,这件事不应该是来问小弟吧?”韩冈又将皮球踢回去,“不知学士是什么想法?”

“家严不是让愚兄来问玉昆你吗?”王厚同样是一句反问,皮球踢来踢去,就是不肯明说王韶的态度。

不过对韩冈来说已经足够了,王厚的反问,让他的推测得到了确认。

王韶要是高兴王安石这般拦腰一刀,他就直接上门来找韩千六了,帮宰相做媒人,也是与王安石拉近关系的途径。现在却是让儿子来探查韩冈的态度,多半就是不愿接受,只是不便推辞。

王厚没有去盯着韩冈的表情。他了解韩冈,别人是口不对心,而韩冈却是脸不对心。他的神色变化,向来跟心中想法无关。心有山川之险,胸有城府之严,这两句形容韩冈正合适。要不是知道韩冈为人还算正直,不是阴险之辈,这样的人物肯定是要躲着走的。

王厚只是在等韩冈的回答。

“如果是结亲,相公家的女儿的确是个上佳的选择,”韩冈微微一顿,“但我可不想落到沈存中的下场。”

提起沈括,王厚便忍俊不禁,扑哧一笑:“以玉昆的手段,就算娶了公主,也不至于家里的葡萄架子会倒。”

韩冈也是莞尔一笑。葡萄架子的笑话,还是他对王厚说的。

虽然战事已经结束,王中正与王厚一起押送木征去京城后,就只有王厚一人返回。而蔡曚、吕大防等人也早早的离开。但担任随军转运的沈括,到现在还留在熙河,在经略司任机宜文字一职。韩冈也跟沈括来往频繁,在学术上都互有见证,不禁有些相见恨晚的感觉。不过随着对沈括的接触,他家中的情况韩冈也有所了解。

偶尔去衙门时看见沈括脸上的遮掩不住的指爪淤痕,韩冈不禁感叹,难怪沈括在历史上会有那么大的名声——娶对了人的缘故。但娶妻在德,能让丈夫变成哲学家的妻室,韩冈可不想要。

知道了韩冈的心意,王厚心情便放松下来。说起来,他也想跟韩冈能成为姻亲,但要怪就得怪他家已经没有更合适的人选了,要是选错了人,反倒是亲家成仇家,

“看来玉昆是要推掉了。不过宰相家的家教也是不错的,王小娘子应该不至于像沈存中的浑家那般凶悍。”

“……这是处道你的想法?!”

“原本是想着跟玉昆你做姻亲的。只叹现在族中戚里都没有合适的人选,不过日后你我有了儿女再做亲家也不算迟。至于现在,愚兄觉得玉昆你还是先做了宰相的女婿。想想富彦国、冯当世,日后玉昆也是多半能当个宰相的。”

‘原来如此。’韩冈总算是全明白了过来。王韶不想韩冈跟王安石结亲,说不定已经存了跟新党疏远的心思,但王厚却另有想法——儿子跟老子想法不一,也是常事。

不过不管王韶父子怎么想,婚姻是韩冈自己的事,是韩家而不是王家的事,做主的还是他自己:“此事且等小弟中了进士后,若是连个进士出身都没有,小弟岂有脸面迎宰相家的女儿入门。”

一句话将宰相家的提亲拖延下去,又过了几日,终于到了王韶启程离开熙河的时候了。

为了给王韶送行,由高遵裕领头,熙河路中的官员基本上都到了。

因为王韶的离开,太后亲叔暂时还会留在熙河。他将会暂时以兵马副总管的身份来代管熙河内外军事。不过武将是不可能在经略使的位置上久居,他很快就会让贤。除非高遵裕能升到郭逵的那个地位——一任执政之后,地方上的官职都有资格担任——不然他也只有偶尔才能品尝一下经略使的味道。

高遵裕之后,苗授、韩冈领着一路的上百官吏相送。出城列队的骑兵,轻轻松松的就超过了千匹之多,已经远非旧时可比。

如今的陇西马市,每天市马已经超过了十匹。这可是夏天!一般来说马市真正开张的时候,都要等到农历七月之后,也就是秋高马肥的时节。去年七月末到十月中旬的不到三个月的时间里,平均每天都有四五十匹良马在陇西马市中交割,其中能充作战马的至少有三分之一。

城池、官员、将领、士兵,这几年,王韶所创建的成果,就在这里。

驻马于渭水之滨,回头望着熙河的山山水水,新任的观文殿学士眯起了双眼。没有喜怒哀乐的表情,眼神又深深敛起,让人看不出他的心中究竟在想些什么。

‘应该是有些舍不得吧……’

韩冈心中想着。也许王韶下一次回来,说不定就是熙河路遍地烽火的时候,为了救火而被调回。

王韶的成事,带动了天下边臣的野心。

章惇收复荆蛮的行动还没开始,西南夷那边就又要动手了。朝中遣了一名朝官去了梓州、夔州两路担任察访使,目的就是这两路不服王化的蛮夷。中书户房检正公事,虽然还不是核心,但作为新党中坚力量的熊本,他被派去西南,可见新党因为王韶的成功,而再难按奈下建功立业的迫切了。

王韶为大宋拓土两千里,真宗以来,边功以此为首。但也不是没有后患,不少人都在担心,自此以后,大宋的周边将会永无宁日。看到了熙河经略司的成功,意图仿效的官员不知凡几。目标荆湖山蛮的章惇就是个最好的例子,前往西南、查访梓州路、夔州路的熊本也是个例子。镇守河湟、横山的边臣,都有可能为了功勋而挑起战争。甚至南面的大理、交趾,也都有机会成为下一个热点。

好战必危,如果朝廷不能早早的加以制止,迟早要在边臣的好大喜功上吃一个大亏的。虽然没有预言的能力,从记忆中也搜寻不到有关的历史,但韩冈完全可以从眼下的局面中,推断出最后的结果。

不过王韶离开了,熙河的盛宴也暂时结束了。秦凤转运司辖下诸军州的仓囤中,已经没有多少存粮,就算来接任的经略使有何雄心壮志,也得先等到填肚子的东西能准备好才行。

高遵裕领众将王韶送出了十里之外,韩冈亲自将王韶又送出十里。举荐于草莽之中,数年相知之情,他也当多送上一程。

回头已经望不到陇西城池,王韶拨马而回,“远送千里,终须一别。玉昆你到这里就停步吧……”

韩冈洒然一笑,也不惺惺做小儿女态,拱手回应:“半年之后,韩冈将至京城拜见学士。”

王韶放声大笑:“就等着玉昆你来。”

目送着王韶的队伍远去,滚滚尘烟渐渐飘散。

韩冈掉马回身,向着穿行在山峦之中的渭水上游望去。重镇陇西,已经隐没在群山深处。炙烤得火热的天地之中,一时只有韩冈和他身边的亲兵。

马鞭一甩,一声呵斥,韩冈胯下的战马带着他疾驰而出。奔马如龙,包顺送来的龙驹很快就将他的亲兵远远的抛到了身后,

的的的马蹄声中,韩冈单人独骑,向着陇西奔驰而去。

‘六二之卷——河湟开边’完。

请期待下一卷,‘六三之卷——开封风云’

ps:下一卷开始,本书进入深度阅读模式,也就是要登录之后才能阅读。这本书点击一向很高,但收藏始终悲剧,希望还没注册的各位书友能用上一分钟的时间,注册登陆一下并不麻烦。而收藏之后,无论是阅读新章,还是书评,都很方便。

