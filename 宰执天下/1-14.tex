\section{第八章 破釜沉舟自专横(下)}

“爹爹,娘娘,还是让孩儿去罢。爹爹你去了县里又能如何?认识的人中又有几个官绅?总不会有人为了菜蔬,就跟陈举、黄大瘤放对罢?……没得求人的门路,河湾上的那块地迟早还要卖出去的!”

“三哥儿你去就能成?”

“爹爹,娘娘,真当孩儿在外两年游学是闲逛不成?!”韩冈站起身,抬手指着东方:“孩儿师从横渠先生,同窗学友多有官宦子弟,甚至还有一些有官位的弃了职来聆听子厚先生教诲。李癞子纵然是县里黄大瘤的姻亲,两人在陈押司面前又说得上话,可陈举本人也不过是个吏户,黄陈之辈又并无官身,孩儿哪会怕他们!”

“可那陈押司在县中说一不二,甚至连知县都得让他三分。恶了他,整个秦州都没一处地方可待。”韩千六愁眉依然不解,陈举的名声实在太大,那是连县尹也不敢轻易得罪的主儿。在他看来,儿子是初生牛犊,日后前途自然不小,可真对上陈举,也只有被吃得份。

“那又如何?!陈举在成纪县衙二十余载,再往上父子传承三代近百年,县衙中的公人都是对他唯命是从,说是在县衙内一手遮天是不错,更别提他在军中还有奥援。但成纪县衙拐弯过去便是州衙,莫说小小一个押司,就算是成纪知县在秦州城中又能排上第几把交椅?真闹得家中破产,以孩儿士子身份,径自去州衙门前敲鼓,经略相公还能打孩儿板子不成?!”

韩冈心中已经有了定计,接着对父母道:“李癞子即做了初一,也莫怪我做十五。大哥二哥战死沙场,孩儿又重病刚愈,现在李癞子明着欺我,这正是喊冤的时候。……李癞子想让我家家破人亡,若不能让他自食其果,我也枉为人子了!”

韩千六、韩阿李低头去考虑韩冈的说辞。韩冈有人在背后扯着他的衣裳。回头一看,却见是韩云娘用着两支白如葱管的纤指,捻起韩冈的一片衣角,轻轻的扯着。小丫头的瓜子小脸仰起,宝石般的黑眸眨巴眨巴的看着韩冈,看起来像只可怜兮兮的小狗,有些怯生生的,让韩冈心中怜意大起。其实不必她提醒,韩冈自己都会提出来,这么好的一个女孩儿,他可舍不得有半点损伤。

“爹爹,娘娘,孩儿还有件事要说!”韩氏夫妇闻声抬头,韩刚起身跪下来对他们正色道:“云娘这些日子来辛辛苦苦照料孩儿,苦活累活也都做了,也亏得她小小年纪能耐住这般辛苦。知恩当图报。孩儿也不能负了她。”

韩云娘年纪还小了一点,真正要收房大约还要再过两三年。不过韩冈也怕他去了秦州城后,会出什么意外。对于此时的人们,除了发妻外,其余的侍婢妾侍都不过是个值钱的物件,说卖也就卖了。韩冈可不想去城里走了一遭后,自家的田保住了,但回到家中却发现小丫头已经给卖掉了。

“三哥儿,娘也知道你再担心什么!”韩阿李一眼看透了韩冈和韩云娘两人心中的隐忧,精明厉害得不像一个农妇,“云娘在家里待了也有四五年了,平常都是小心勤快。这么多年,云娘早就是韩家的女儿了。卖儿卖女那是畜生都不作的事,三哥儿你也别多担心。云娘,为娘的会给你好好的留着,断不会舍了,韩家就算卖地卖房都不会卖女儿的!”

韩阿李的一番话掷地有声,让韩冈喜出望外,而韩云娘更是感动得哭了个雨带梨花,“娘……”

韩阿李将小丫头轻轻抱在怀里,抬手轻轻抚着她的头发,“傻孩子,哭甚么!娘不说难道你自个儿就不清楚吗?……”

……………………

第二天。

韩冈双眉照旧锋利秀挺,神情依然从容不迫。仍旧是一袭青布襕衫,将一个装满书的小包裹背在身后,在摆渡处辞别依依不舍的父母和小丫头,独自登船渡河。

韩千六本想送着韩冈一直到城中,但还是给韩冈劝阻了。而把调韩千六应差役的县中行文送到韩家,又一边剔着牙哼着小曲,远远的跟着韩家人一直到渡口边的李癞子,看到是韩冈跳上船,而不是韩千六去支应差役,却是大吃一惊,脸色数变。渡口附近看见韩冈上船的村民们,没去将军庙的诧异莫名,去了将军庙的则是不出意料的神情:

“怎么是韩家的三秀才去了城里?难道是他去服衙前?!”

“怎么可能,他可是读书人啊。”

“莫不是去告状?……那不是正落到黄大瘤手上吗?”

“成纪县衙在秦州城的衙门里能排第几?韩三秀才可是有大才的人,州衙也是想去就去。黄大瘤能堵着州衙的门?”

“我看韩家三哥不简单,这两年在外游学,回来后说话做人都不一样了。李癞子把他得罪狠了,肯定有苦头吃。”

“可不仅仅是苦头啊……”

藉水泱泱,韩冈坐在船头听着哗啦哗啦的流水声,心底甚至还有些风萧萧兮易水寒的悲壮暗中滋长。可回头一想,就算入城后,离家也不过四里多地,这算是哪门子的荆轲?但临别前,小丫头哭得红肿的双眼,让韩冈心中波澜横生,而父母的殷殷嘱咐,也是让他心情微沉。

毕竟韩冈拥有的只有自信,而陈举和黄大瘤有的却是实实在在的势力。三名至亲忧心五内,也是理所当然。只是韩冈坐在船板上,伸手入河,眯着眼感受着初冬的寒水冰彻入骨,却并不把黄大瘤和李癞子放在心头。真正能碍着他的,是黄大瘤身后的陈举。

作为黄河支流的支流,藉水并不宽阔,而在少雨的秋后,低落的河水也十分平静。坐在渡船上,也不过小半刻,便结束了行程。下了船,回头望望。还能看见站在对岸渡头上的家人正隔河而望。举起右手用力挥了一挥,韩冈转回身,毫不犹豫地向着五里外的秦州城走去。

作为大宋西北边陲的战略要地,一路重心,从地理位置上也是占据着沟通东西南北的河谷要道。秦州城中南来北往的各族商人为数众多。跟李将军庙一样,秦州城也是二十多年前韩琦韩相公知秦州时主持扩建。当其时,东西城外的草市【注4】兴盛,倚城而居的民家几近万户。

秦州的富庶名传西北,而城外的市场民家又全然不设防,每每遭到西夏人的攻击,有鉴于此,韩琦便招揽民夫扩建城墙,耗时数月,将城市东西两侧的民家店铺一起包入城中,城民感其恩德,故号为韩公城。

也因此,秦州城是东西宽南北窄,是长方形的结构。而从南北两面来看,城墙是两段新墙夹着一堵旧墙。

随着那段半新半旧、高达三丈半的城墙在视野中越来越大,韩冈行走的官道两边也越发的热闹起来。难以计数的商贩拥堵在官道周围,将四丈多宽的官道占去了半边还多。

道路两边的行商有挑担子的,也有背背篓的,更多的则是赶着大群的牲畜,驼马用来载货,羊群则直接是拿来卖。这些行商如果要入城,都要照规矩缴纳两厘也就是百分之二的过税,到了城内贩货时,还要缴纳百分之三的驻税。商人赚钱也不容易,自是能省一分就是一分,几乎都是聚在城外做着生意,形成了一个规模庞大的草市。

韩冈一路走来,四周叫卖声不绝于耳,道路两边的茶肆酒铺也是鳞次栉比。在草市内做着生意的不仅仅是汉人,还有许多蕃族商人由于身份所碍进不了城,便在草市边缘摆起了地摊。

如果在草市内逛一逛,说不定能掏到不少有趣的东西。只是韩冈无心驻足游逛。走到秦州南门外,忠于职守的城门守兵正一个个搜检打算入城人们。每一个被检查到的人,都要他们自己拍拍身子,示意自己并没有夹带货物,耽搁上半日才能进城。

绵长的队伍慢慢前进,直轮到韩冈。站在门洞下,城门守兵只上下看了韩冈几眼,连包裹都不动,只一挥手,就放着韩冈进了城去。

“怎么连查都不查一下,就放他过去了?”一个十几岁的小兵奇怪的问着。

“那是个读书人啊!搜检全身,不是有辱斯文?”城门卫为自己辩解道。

韩冈虽然没有表露身份,眉眼又稍显锐利,但当他负手而立,一缕清风卷动他的衣角,几乎是随身而来的文翰之气,却是遮掩不住,岂是西贼奸细能有的气度。

穿过阴暗的门洞,眼前豁然开朗。大小道路纵横如阡陌,店铺宅院以千百计。行人络绎不绝,虽远比不上后世的城市,但与韩冈记忆中的京兆府比起来,却也不遑多让。唯一有别于京兆的,便是街巷之中,有铁骑巡道,城墙之上,有弓手护持。只要看到他们,就能明白秦州还是一座防卫森严的要塞,再如何繁盛的商业活动也是冲不去蕴藉城中的肃杀之气。

商业繁荣,军威肃重,这便是西北雄城——秦州!

注1:民间自发形成的市场叫草市。北宋商业发达,各地草市墟市为数众多。有许多草市最后还被升格为镇,当地衙门在其中收取的商税往往还在城池之上。

PS:北宋的秦州就是如今的天水,天水市区秦州区得名便因此而来。不知本书的书友里有没有来自天水的朋友。

求红票,收藏。

