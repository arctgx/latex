\section{第12章 恶客临门不待邀(下)}

梁乙逋被一名小校引进了中军大营的核心地带。

环目四顾,大辽国相的儿子、太后的侄儿,看得心中寒气直冒。回头看了看随行的仁多瀚,他也是阴沉着脸,不住的扫视着营中远近。

宋军攻下乌池堡才两个时辰而已,但现在已经是刁斗森严,营地内都布置得如铁桶一般。巡逻的小队在营内来往有序,不露一丝破绽。

但梁乙逋知道,就算宋军营地内部乱作一团,已经火烧房梁的大白髙国也决然不敢放手一战。

在收到黑山威福军司遣人送来的急报之后,西夏军上下完全失去了战意。黑山的部族大半投向了辽国,那一片河间地,已经沦入契丹人手中。

在这里与种谔厮杀一场之后,大白髙国还能有多少兵力剩下?又要多长时间才能休整好,重新上阵?眼下耽搁一日,入寇的契丹铁骑就离兴庆府近上一分,已经没有时间可以耽搁,也没有兵力可以损失。

换马不换人的急脚传信,的确是要比大队的骑兵快上两三倍。可从大夏的北疆赶来南疆,也要比辽人直趋兴庆府的路程远上一倍还多。从时间上算,辽军当已通过了顺化渡,再有两天,便能抵达兴灵之地的北面关口右厢朝顺军司,也就是克夷门。

一旦辽人突破了克夷门,接下来,就是兴灵之间的肥沃平原了。

还在盐州城中的大军,必须星夜赶回兴灵,这样才有可能挽救危局。但有种谔在身后,临阵撤退难度极高。而且任谁没指望能将国中生乱的消息瞒过心明眼亮的种谔。

“辽人南下的消息已经从河东传过来了。既然秉常派你们来,他们应该不是好心来帮你们助战的吧?”

见到了种諤,这位宋国名将开门见山的一句话,彻底证明了这一点。事实也好,诈术也好,都证明了种谔已经了解了内情。

梁乙逋遍体生寒,种谔的话中竟然只提那个被囚禁的国王!

关键的语句,仁多瀚也不会听不出来其中的用意。他更正道:“太尉有所不知,鄙国之君如今病重不能理事,现由太后垂帘。”

“我如何不知道。但你们似乎是忘了我大宋兴兵是为了何事?”种谔咧开嘴,一口白牙在油灯下反射着火光,森森发寒,“解救被囚禁的夏国国王!”

梁乙逋强压下心中的羞恼。种谔眼下的确是强硬到底,但他肯于接见自己,就代表了还有商量的余地。现在不过是讨价还价的手段。完全不知道种谔的想法,这一点就很危险,而一旦把握到了心思,就好办多了。

他的父亲和姑母,之前派人来示好,明白说出要退出盐州,而不是直接领军就走,就是因为需要大宋的帮助,否则西夏只有灭亡一途。

梁乙逋没有时间可以耽搁,他深深的向种谔行了一礼,“上国西征究竟为何事,乙逋不敢相争。乙逋出来时已受命。只要大宋愿援助鄙国对抗契丹,鄙国愿意割让河南之地,以为大宋天子寿。”

从盐州回兴灵,浩浩瀚海,没有粮食是走不过去的。而且仓促回军,也很难赢过南侵的辽军。必须争取到宋人的支持。割让土地也无所谓,再不有所行动,也就没有土地了。

种谔的眼睛越来越深沉,梁乙逋的坦诚让他终于确认了西夏面临的处境。耶律乙辛当真在背后捅了一刀子。

对嘛,这才是契丹人该有的手段。

“我是武夫。不太习惯与人讨价还价。”种谔摇摇头,“而且我受命是与你们西夏作战。除此以外,别的事我都没有兴趣”

“种太尉,何必做这等令亲者痛、仇者快的事,最后得意的只会是辽人。别忘了,盐州城还有十万大军!哀兵十万,太尉可能挡得下?!”梁乙逋厉声髙喝。

“别说十万了,嵬名家和梁家还能确实掌握的兵力,这才是实数。”种谔自在的靠上他的熊皮交椅,“西夏已经完了,只要我在这里拖上两天,兴庆府也就成了辽国的囊中之物”

梁乙逋脸色一变,转头就看了仁多瀚一眼。

“然后大宋就跟契丹做了邻居?!”仁多瀚正正的与种谔对视着,他没有表现出对种谔发言的动摇,“有什么条件,还请太尉直说吧。”

条件……种谔眯起了眼。西夏既然灭亡在即,愿意同舟共济的还有几人?何不顺水推舟一把?

一名年轻英俊的将校这时从门外进来,也不多看梁乙逋和仁多瀚一眼,凑到种谔的耳边,匆匆说了两句。

梁乙逋不知道这个年轻将校的几句耳语究竟是什么,但他知道绝对不是什么好消息。

年轻将校退了下去,种谔撇着嘴笑了起来:“识时务的人这世上从来都不缺。叶家人来了,是为了请降。还有一个叫李清的也派人来了,也是为了请降。梁乙逋,你的十万大军还剩多少?”

梁乙逋的脸色一路惨白下去,种谔在他面前直说此事,不论真假,都是不安好心。他咬着牙:“太尉是想挑动我军内乱?太尉可别忘了,六国征战不休,最后却让强秦得了天下去。”

“这话说得好。”种谔拍了拍手,“现如今大宋国势昌盛,辽国不思援助西夏,反而出兵并吞,灭亡可谓是指日可待了。”

梁乙逋一口气堵在胸口,甚至都说不出话来。

种谔转头看着仁多瀚,“仁多瀚,你出来前,仁多零丁跟你说了什么?”

梁乙逋猛然间瞪大眼睛,仁多瀚也是一呆,半天后才干笑,“太尉说得什么话……”

“仁多瀚。”种谔落下了脸,“你要想清楚,你的每一句话,都会影响到你仁多家的前途……你在跟我说句没有!”

仁多瀚看了看怒容满面的梁乙逋,无可奈何的叹了一声,“太尉说得对,的确是有的。”

“好贼子!”梁乙逋指着仁多瀚的手颤抖着,急怒攻心,一口血就喷了出来。

种谔都没理他,他对着仁多瀚:“仁多瀚,想必你也知道这世上有个东西叫做投名状。想改换门庭,投名状是必须要交的。”

仁多瀚百般无奈,苦笑道:“种太尉,何苦如此!”

“仁多瀚……”种谔冷了脸,“我是不是在跟你讨价还价。你可以推脱不干,叶家和李清的人马上就要来了。”

仁多瀚踌躇了半天,看看种諤,终于颓然一叹:“……小人明白了,这就遣人回去通报。”

梁乙逋强忍着头中的晕眩,厉声叫道:“种太尉,你难道不知道什么叫唇亡齿寒?”

“大宋跟辽国做了多少年邻居,西北这边再贴个门也没什么。”

种谔可不会去管什么唇亡齿寒,这是笑话。从他的角度来说,若能灭了西夏,纵然他之前有什么罪责,都能洗得干净。而从他的平生夙愿上,灭亡西夏也是最让人痛快的选择。

梁乙逋放不放都无所谓。叶家和仁多家一旦举旗,绝大多数的部族都会跟他们站在一起。孤家寡人的嵬名家和梁家,只有覆灭的份。

梁乙逋和仁多瀚被带了出去,分在不同的地点安置下来。

种朴进来了,“大人,叶家和李清的信使都安排下来了,什么时候见他们。”

方才进来通报叶家和李清派人来的就是种朴。他这一次随军西行,虽然伤势未愈,但帮种谔处理一下机密文书还是没问题的。

种谔想了想:“先晾一下,让他们跟仁多瀚照个面。”

“儿子知道了。”种朴点了点头,却又问:“大人为什么不帮一下西贼。让他们跟辽人狗咬狗去,也是好的。”

“事情有那么简单?要是仁多零丁和叶孛麻跟着北上,投了契丹人,将嵬名家给灭了怎么办?那不是给契丹人送兵送将吗?!”种谔怒瞪了种朴一眼,“何况军粮呢,陕西还有多余的粮食给党项人吗?辽人占了兴庆府,便不愁吃喝。官军要帮西贼,就要为他们提供粮草,你以为梁乙逋为什么过来,为什么梁氏兄妹要派人来说让出盐州,直接走不好吗?我难道还能追到瀚海里去!?这是要让官军隔着瀚海给他们送粮!要不是算到这一点,辽人怎么可能会南下兴灵,占了黑山差不多也就心满意足了。”

种谔喘了口气,阴郁的声音压得极沉:“一场大战,连今年的新粮都耗得差不多。民夫调动得那么多,明年再是风调雨顺,也肯定是歉收的局面。得到两年后才能缓过气来。到时候,兴灵之地,就成了铁桶一般了。”

种师中跟着种朴一起进来的,笑着缓和道:“不管怎么说,西夏都是完了。五叔,这可真是太好了!”

种谔脸色一变:“好?哪里好了?我看不出有哪里好!?廿三,你说哪里好?”

种谔尖利的语气,将种师中都吓住了。缩着膀子都不敢搭腔。

“这一次是捡了契丹人的便宜!”种谔面目狰狞,一跃而起,“从我开始占下绥德城,到如今也有十二年了。整整一纪,每每都是占尽优势,却被人扯了后腿。要是什么都听我的,西夏早就完了!若没那些措大,我早就踏平兴灵了!”

他怒视着噤若寒蝉的子侄,神色转又缓了下来。颓然的坐下,喃喃念着:“要是没有那些措大……该不知有多好。”

