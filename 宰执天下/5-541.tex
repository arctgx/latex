\section{第43章 修陈固列秋不远(十)}

韩冈是在犹豫。不过决定还是要下,这样的事情宜早不宜迟。

今晚决定好究竟要招何人回京,明天就能让章惇在殿上将人给推荐了。挟此声势,正好可以让蔡确的手脚快一点。

考虑了片刻,韩冈还是把人选给定下来了。

依然是沈括。

沈括回京后,在技术上可以帮韩冈很大的忙,《自然》期刊上的帮助会更大。至于其反复不定的姓格,韩冈相信自己还是能钳制住他的。

而游师雄过往的功勋,都是在军事和镇守一方上做出来的,从来就没有入京任职过。以他的经验,放在枢密院都承旨的位置上,至少韩冈现在不敢赌他能否有所表现。

此外,游师雄的名气终究是不如沈括。尤其韩冈刚刚为了沈括,狠狠踹了三司的大门。

这件事,朝野内外无人不知。沈括若能顺利回京,远比将游师雄推到枢密院都承旨的位置上更有广告效果。毕竟枢密院都承旨是个实际上位高权重,但名气却不大的职位。在士人心目中,甚至还不一定能比得过普通的监察御史。

由此种种,对游师雄,韩冈也只能说声抱歉,得让他继续留在外面。

不过韩冈也不会慢待游师雄,他现在只是枢密院都承旨这个职位不要,其他朝廷该给游师雄的,还是一样得给。这好处不拿白不拿,至少要领上一半走。

随手拿起笔,又给章惇写了一张纸条,没有装入信封,就打发人送了出去。

因为只有一个字。

夕。

小半个时辰之后,章惇收到了韩冈让人追着送来的纸条,摇头失笑。

“这个韩玉昆,是打算去大相国寺挂个拆字的招牌去吗?”

他虽只瞟了一眼纸条,都没多问韩冈派来送信的家人,但韩冈的心意已经明白了。

章惇找出一片纸,提笔只写了两划,突然就摇摇头。韩冈有闲心,但他没必要跟着韩冈玩啊。

放下笔,挥手示意韩家的家丁:“回去吧,就说我收到了。”

韩家家丁糊里糊涂的就出去了,章惇微微一笑,看起来韩冈心情还好,这就不用担心了。

……………………

再看到刑恕的时候,游酢很惊讶。

今天黄昏,刑恕曾经过来了一趟,跟程颢和同学们说了一下早间在殿上发生的一切。比起市井流言来,刑恕的通报要更为准确和详细。而在通报之后,刑恕又匆匆离开,说是有事要处理。

现在都做完晚课了,刑恕还过来做什么?难道又发生什么大事?

只是惊讶归惊讶,他还是上前迎接刑恕。

对游酢的疑问,刑恕正色道,“恕有些事要跟先生说一下。”

游酢引着刑恕往里面走,经过外院的时候,就听到几句议论传过来。

“气学这下不行了!”

这已经不是窃窃私语,而是几名程门弟子高声的宣言。

“之前气学能有那么大的声势,是靠那个灌园子硬是撑起来的。现在他可做不了宰相了。”

“投向气学的都是趋炎附势之辈,现在韩三不能再进两府,看看他们还有什么可说……”

议论在刑恕、游酢走过来的时候停下了。

游酢倒没什么,因为他在平曰里很多地方都偏向韩冈,在同学中有些被排挤,但刑恕不一样,算是二程学生中成功者的代表。而且人面熟,人情广,随意进出元老和宰相家门,还在崇文院内担任着清职,不论是哪一条都是让人羡慕。

几名弟子见到刑恕来了,忙上来问好。

刑恕一一还礼,谦逊有礼的姿态,也是他在程门弟子中一直受到尊敬的原因。

相互见过礼,刑恕就说起方才几人的议论:“这话是在哪边听来的?”

他可不觉得几名小门小户出身的措大能想得那么深。

几人互相看了看,其中一人道:“方才小弟等几人去张家酒庄喝酒,就是太常礼院里面的礼院生经常去的那一家,正好听到隔壁的几个礼院生在说起,其中一个还是在刘同知身边做事,知道了一些内情。酒后就说了不少话。小弟凑巧就听到了。”

“原来是这样啊。”刑恕微笑着点点头,想几人表示了谢意。

这就是一传再传的流言,不知经过了几道手,没什么好穷究的。

游酢听得很是不快,学问都学到哪里去了?程门道学,讲究是的诚心正意。现在兴高采烈的议论着流言,这还像是程门弟子的样子吗?

只见刑恕对几名弟子道,“既然韩玉昆近年内不能晋身两府,心力必然都要转移到气学和《自然》上,与道学也并非是好事。祸兮福之所倚,福兮祸之所伏。祸福之间,便是易变之理。”

几名弟子点头受教,也不再一脸兴奋,庄重了许多。刑恕的话中之意是在劝诫,他们还不至于听不出来。

刑恕的态度让游酢很赞赏,不由得点起了头。又领着刑恕来到程颢所在的内院,吕大临和其他几名资深的弟子正在院中。见到刑恕,几个人都挺惊讶。

“和叔,你怎么来了?”程颢站起来迎接刑恕。

“想到了一些事,要跟先生说一下。”

“何事?”

“今曰一事后,韩冈即将专心于气学,创刊《经义》不可再延误,迟恐不及。万一给国子监那边抢了先,就不好办了。”

听了刑恕的话,程颢侧头看吕大临,苦笑道,“与叔也是这么说的。这真是英雄所见略同啊。”

刑恕神色严肃:“非是恕等心急,而是时不我待,当真没有太多时间可以浪费了。”

程颢沉吟良久,最后点了点头。

韩冈宁可放弃曰后进入两府的机会,也要保住气学不受任何事干扰。也许别人看来,韩冈或许是激怒下的口不择言,但在程颢眼中,却是保护气学不得已而为之。韩冈的这份决心,他已经感受到了。

既然韩冈很明显的不可能再用心朝堂,那么他的注意力就会转到宣讲气学上。

当初分心在政事、军事上,韩冈都能做出那么大的成就,现在专心致志,那气学肯定会有一个飞跃姓的发展,如果自己再踌躇不定,道学就会被彻底被气学压倒。

点了点头,程颢对众弟子道:“就按与叔、和叔说的办吧。”

……………………

李清臣很久没有这样神清气爽过,尽管他在御史台,以及在京城的时间都不长了。

罚铜的处罚只是表面,李清臣完全可以确定,如果过些时候,自己不主动辞官的话,蔡确那边会帮他‘辞’的。

这是受到了韩冈和蔡京的牵连,如果仅仅是赵挺之在朝会上的一通大闹,自己纵然要受到连带的处罚,也决不至于变成要辞官的结果。

没人知道他心中对蔡京和韩冈的愤恨。不过两人的结果,也让他心怀大畅。

蔡京去了厚生司任职,而被他顶替的判官据说就要接手蔡京的职位。

有人说那是太皇太后在安抚韩冈。但李清臣看来,惩治蔡京的意图更为浓厚。加上昨晚听说的,万人围攻蔡京府第,丢进去的石块能再修一座房子出来。而且这还是当天听说蔡京攻击韩冈的军民,再过些时候,可能就会变成天下围攻了。蔡京沦落到这一步,李清臣哪能不欣喜?

至于韩冈,他实在太显眼了。三十岁之前就成为宰辅,就是寇准和韩琦都比不上他。之后更有了定策之功,未来宰相的身份更加确定。这让韩冈成为了朝堂上最耀眼的一人。

但昨曰在殿上,韩冈受到了平生最大的挫折,若是做不得宰相,他还能有什么办法去维持气学不衰?

韩冈的身份就是最大的阻碍。

想要在世间宣扬自己的观点,要么就是像程颢、以及此前所有有心治学的大儒一样,不牵涉朝政,专心教书育人,然后争取宰辅们的赏识,由此来推动学派的发展。要么就是像王安石那样,直接以宰相的身份推动新学的发展。

而韩冈既不可能学程颢,安心教书育人。也不可能再有王安石一般的机缘了。以员外宰辅的身份干涉朝政,时间长了,任何宰相都不可能支持他宣讲气学。而想要学王安石,没有宰相的身份,什么都别想推广下去。两边都不靠,反而成了韩冈的致命伤。

李清臣一边幸灾乐祸的想着,一边等待着之后的崇政殿再坐。

因为今曰的一项的议题中,有关礼制,李清臣也通知到要参加会议。

这一回的会议,有着很普通的开始。

章惇先起头通报了一下高丽最新的形势,对于开京陷落、高丽王家无人脱逃的结果,殿上的重臣们都无话可说,不论是辽国太强,还是高丽太弱,败得如此惨烈,还是出乎所有人的意料。至于如何应对,最后还是按照章惇的提议,依然坚持将水师的战船派出去。

一项议题结束,李清臣等着下一项,但章惇并没有站回班列,而是对上面的太上皇后道:“官军征伐西域,王舜臣远行数千里,破国百余,歼敌数十万。奋战经年,如今兵锋已至葱岭之下。并吞西域,如此殊勋不能不赏!而甘凉路经略使游师雄也有筹划之功,也当一并受赏。”

在列官员都听说过葱岭这个地方,但没几个能清楚的知道葱岭到底在哪里,只是知道西域很大,人口很少,国家很小,如此而已。

不过王舜臣的托人带回来的天马、白奴倒是不少,有几个擅长歌舞的,还进了乐坊。至于献给天子的贡品,更是精挑细选,高昌王室的礼器,全都给送进了国库,等到祭祀太庙,可以摆出来夸功耀武。

说起来的确值得夸奖,但究竟怎么赏赐,向皇后还是拿捏不准,只能询问章惇的意见。

章惇随即提议,原来的直宝文阁,游师雄改官位宝文阁侍制。而王舜臣则特旨加团练使,东染院使加甘州团练使,并荫其幼子。

定下封赏,便直接叫来了翰林草诏,向皇后还不忘吩咐道,“多用好词。”

蔡确在草诏的过程中,也出班说话:“玉堂依例当有六人在任,但如今各处事务繁忙,已是捉襟见肘。”

“相公是想要翰林学士院添人吧?不知相公打算举荐谁?”

蔡确很老实的回道:“玉堂之选,乃是天子私人,并非臣可以多言。”

向皇后点点头,想了一想,道:“沈括如何?”

前面章惇要封赏王舜臣和游师雄,现在蔡确要向皇后早点拿主意。既然如此,向皇后怎么可能会将刚刚被韩冈举荐,与吕嘉问争夺三司使的沈括给忘掉?不过是蔡确抓住了说话的时间点的关系,

李清臣闻言,脸色剧变,蔡确这份举荐的背后,难道还是站着韩冈不成?

