\section{第三章 时移机转关百虑(十)}

“是,是,娘子所言甚是!”韩冈笑着点头。

只要认为自己没有错,王安石就是死活不肯低头的脾气。他这个做女婿的也是吃过苦头的。

王旖不高兴了:“是只要说一遍就够了!”

“是,娘子。”韩冈毕恭毕敬。

有时候韩冈觉得自家浑家的脾气,真的有几分像他的岳父大人。

韩冈还记得王安石旧年有一桩公案,因为鹩哥还是鹧鸪……反正是只鸟而起的杀人案——具体什么鸟,韩冈记不清了。

两个朋友,其中一人带了只鸟,另一人看着喜欢,想要过来。前者不肯给,而后者直接抢了过去。到此为止,还仅仅是朋友间的龃龉,但当前者拔出刀子将后者刺死之后,事情就闹大了。

开封府断案,当市杀人,没话说这是死罪。而正好担任纠察在京刑狱的王安石进行复核时,则认为,既然抢了鸟,那就是白日劫盗——杀强盗不当论死!

两边相持不下,最后这桩案子交付大理寺和审刑院公议,定了是杀人罪。既然结果与王安石的判决相反,照规矩,王安石当为此受罚。不过仁宗皇帝和了稀泥,赦了王安石的罪。但王安石可好,梗着脖子说我无罪,连叩谢皇恩都不干。一下惹动了御史台,弹章连番而上。可王安石根本就不在乎,最后又是仁宗皇帝出来和稀泥,几句算了就当没这回事了。

至于在包拯手下做群牧判官时,包拯劝酒怎么也不给面子的小事,就不用提了,例子实在太多。

顶牛顶到不给皇帝面子,王安石的倔强可见一斑。韩冈当年在王安石面前硬着脖子说横山必败,有功劳别算我一份。与王安石当年相比还差一点点。

不过自家的浑家,跟黑脸的岳父,就是同样倔强,感觉也是不一样的。王旖生气时瞪眼抿嘴,很有几分可爱。有时候,韩冈甚至要故意逗一下。

王旖则是气呼呼的瞪着韩冈,她也觉得自己的丈夫许多时候惫懒起来,还真是让人恨不得咬他一口才解气。

幸好送热水的婢女进来了,王旖去净房先洗了脚,然后才回来用热水泡着。木盆的热水中放了个药包进去,专门用来泡脚的,属于香料一类,能在活血的同时,给双脚一并熏香。

王旖大家闺秀出身,甚至没有走过远路,在宫城里面走了一天,脚上就起了好几个水泡。不过泡在水里就舒服多了,心情也好了不少。

白生生的小脚泡在热水中,王旖探头看着韩冈手上的书,“官人现在看的是哪一篇?”

韩冈将手上书举了一举:“小苏的六国论【注1】。”

“苏家父子三人同论六国,各自见解不同,不知官人觉得哪篇写得好?”

“都看过?”韩冈抬头,反问道,“娘子觉得哪一篇比较好?”

王旖轻快地回答:“苏老泉的弊在赂秦。爹爹和大哥都觉得他写得好。官人呢?”

韩冈摇头道:“老苏一篇文章借古讽今,道理其实说得牵强,可拿当今之世做对比,让人心生感触不奇怪。当年正好是朝廷拿岁币岁赐贿赂辽夏的时候,出来的时间可巧得很。”

“那大苏小苏呢?”王旖兴致很高的问着,丈夫与她谈论文学的时候很少,今天可是难得的机会。

韩冈沉吟了一下,道:“苏子瞻的《六国论》,与其说他论的六国,还不如说他论的养士,偏题了。他说秦兴乃是养士之功,六国能在强秦的压迫下维持多年,也是靠了养士,当秦一统天下,不再养士,士人生怨,所以亡了。这是从张元、吴昊身上引发出来的议论。张元、吴昊都是不第士子,投靠党项,乱我中国。朝廷如今厚待士人,特奏名一科,就是为了不让不第士子心生怨意、投降敌方。这也是为什么唐时行科举,唐太宗会说,天下英雄尽入吾彀中。”

王旖坐直了身子,皱眉回忆道:“爹爹倒是觉得这一篇写得很不好,看了就丢了。”

“岳父不是写过论孟尝君的一篇史论吗?经天纬地的方可称为士,而孟尝君身边的鸡鸣狗盗之辈并非士。因为孟尝君身边尽是鸡鸣狗盗之辈,所以才士不至。”韩冈笑道,“岳父要是看得惯苏子瞻这篇文章,反而怪了!”

他其实很佩服王安石,《读孟尝君传》才一百个字不到,道理却说得通透,比那些连篇累牍的文章强得多。而孟尝君本人的行事作风,也的确只是类似于黑社会头目的人物,并无雄才大略,王安石给他鸡鸣狗盗之雄的评价确实深刻入骨。

王旖当然读过他父亲的著作,想了想,也觉得丈夫说得有几分道理。

“至于小苏的这一篇。”韩冈继续说道,“则是从地理战略的角度来说,是要山东六国保住韩、魏这个屏障。韩、魏位在中原,地处天下之中,当韩、魏不保,其余四国就只能被各个击破。”

“官人觉得谁说得对?”王旖兴趣盎然的问道。

韩冈打了个太极拳:“史论本就是借古喻今,他们想说的从来都不是六国,问他们说的道理对还是错,根本没有意义。先圣编写诗经,跟现下文人自纂诗集,可会是一般的道理?”

“就事论事呢?”王旖却不放过,追问着,“哪一个说得对?”

韩冈想了一想:“就事论事的讲,六国之亡是内因外因的集合,不仅仅是一种原因。三苏的六国论得合起来看才是,赂秦是一条;小苏的韩魏不保也是一条;至于苏子瞻说的秦能养士故而兼并六国,不能养士,故而覆亡,同样是一条。”

王旖捂着嘴笑了起来:“官人的这种说法可是狡猾得很,这个也对,那个也对,说出来就是谁也不得罪。”

“但他们加起来也不全面,这个有不足,那个也有不足,说出来可是谁都得罪了。”韩冈笑了笑,“其实都没有说到点子上。”

王旖眨着眼睛:“那奴家就洗耳恭听官人的高论。”

韩冈呵呵笑了起来,“为夫可没有高论,有的只有先圣之言,‘足食足兵,民信之矣’。兵精、粮足、国人信服,秦人做到这三点,自然能兼并六国。”

“就这个?”王旖疑惑道,“该不会是搪塞奴家吧?”

“圣人的话岂会有错!有一干人战国策读得多了,以为纵横术无所不能,一张嘴就能‘致君尧舜上’,却没有心思认认真真耐下性子去做实事。殊不知,治政之要,就是在于兵精粮足,军民信服,有了这三条,便能纵横于世。不过呢……”韩冈叹了口气,“知易行难,要想做到可是难得很。”

王旖想了一阵之后,还是点了点头,认同了韩冈的说法。不管怎么说,韩冈治政用兵的手腕都是一流的,出将入相对他来说,不是夸奖,只是恰当的评价。他说的话,天子听不听是一回事,但肯定是重视的。

“秦人有关中之利,又得巴蜀之地,辟沟渠,开阡陌,北有郑国渠,南有都江堰,加之民风尚简,兵粮之丰,远过于山东。而商鞅立木、不韦悬金,都是为了民信。军功赏爵,首级易功,秦人百年间用之不移,自然是‘民信之矣’。”

“……那足兵呢?”

“操干戈者为兵——拿着兵器的人才叫兵。所以足兵的话,就要精良的兵器以及敢战的士卒皆备。军功赏爵之制一出,秦人好战如饥似渴,六国远有不如。至于兵器……”

韩冈从放在桌下的一个盒子里摸出几个黝黑的箭簇来,“为夫摆在书房里的这些青铜箭簇看过吧?”

王旖点头。这些青铜箭簇、还有几支秦戈、秦剑,都是韩冈书房中的装饰。现在书房泡了水,里面的东西都拿出来了,贵重点的堆在正房中,不值钱的就放在院子里。

“这些青铜箭头,可能是殉葬之物,也可能是出自秦国军库的遗址,听说当年被掘出来的时候,数以万计。不过都给人拿去熔了造铜器,剩下的不及百一。”韩冈捏着几枚箭头给王旖看,“这些箭头,在土中埋藏千年,但你可以看看,形制如一,没有丝毫的差别。”

王旖仔细的看了半天,除了锈斑的位置有所参差以外,这几个箭簇的大小、外形当真是一模一样,“过去从都没注意呢……还是官人眼睛好。”

“什么眼睛好,这就是格物致知,从小处就能知道秦人打造的兵器有多精良。为夫是判过军器监的,这些青铜箭头有多难得是再清楚不过了。军器监出产的箭簇比不上秦人……为夫藏的秦剑上,还有相邦吕不韦造的字样。物勒工名,军器有问题,能追到宰相头上。六国输得一点都不冤。”

韩冈将箭簇叮叮当当的丢在了圆桌上,“士兵更好战一点,人才更丰富一点,政治更清明一点,兵器更精良一点,农事工业更出色一点,多少方面的优势集合起来,对于东方六国,有了压倒性的优势。而山东六国人心不一,想让他们齐心合力共抗强秦,是缘木求鱼——兄弟间还能争产争得你死我活,最后让奸猾胥吏们占了便宜,放到六国,还不都是一样的情况?”

“商鞅变法,只在耕战二策。其人虽为法家,但其治政之本,却与先圣相合。所以岳父如此推崇商君,不是没有来由的。秦国国力之强,在商君之后,就远在六国之上,只要在台上的不是昏君,兼并六国是迟早的事。三苏的文章,以古讽今做的不错,但论六国,就论的太偏驳了。”

注1:苏轼苏辙两人的六国论写作时间不确定,姑且当做元丰之前所写。

