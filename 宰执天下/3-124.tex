\section{第33章 道远难襄理(中)}

王旁骑在马上,穿梭在东京城汹涌的人流中。

市面上的情况比往年要差一点,但想及大灾之年,而绫罗绸缎依然大卖特卖,还是显得过于奢靡了。

由于吕惠卿的手段,魏继宗已经下了开封府询问,因而曾布几次在天子面前说不能与吕惠卿共事。此举太过于失态,他排斥一同奉旨根究市易司弊病的同僚,而且还是与其在争夺权位上的唯一对手,如此行事就不免让天子有所联想。曾布之前对市易务的指摘,以及对吕嘉问的弹劾,是否可信就值得商榷了。

至少以王旁看来,他父亲这一边已经暂时稳定了形势。而韩冈托他传的话,王旁回来后也跟父兄提过了,很干脆的要钱要粮,同时也直说以白马县的条件,最多也只能安置住十万流民。

是扩大韩冈职权范围,还是将处置流民的工作收归开封府,将这个选择交给父兄来处理,王旁随即离府外出。韩冈另外还托付了他一件事,要他查看一下东京城内外的流民情况。

京畿本来就受灾,当然不会没有流民。最近一段时间,河北南下的流民被挡在白马县中。从每天过河的数量来看,韩冈之前的一番布置,至少在五月份之前,从河北抵达京师的流民都能安置下来。

不过河北今年的收成可以说是完蛋了,一过五月,新粮补充不上,河北流离失所的灾民数目将会有个爆发式的增长——这个词汇是韩冈说出来的,王旁觉得很是形象——魏平真和方兴都推测,南下的流民数量将会是现在的三倍到五倍。

出了城南的西侧偏门戴楼门——这是俗称,门洞顶上的门额刻着的是安上门——大约一里多地,在蔡河边上,搭起了一座座粥棚。有官府出面设立的,也有一干富户所建的。长长一列,差不多排出有半里地。

在粥场外,人头涌涌的场面很是拥挤。而灾民们衣衫褴褛的样子,看着让人心中恻然。但粥棚前流民的数量,远远小于王旁的预计。他沿着蔡河一路看过来,现今设在城南的几个粥场周围,差不多有两千多人的样子。如果其他几面都是这般数目,最多也不过万人左右。比起白马县的流民人数,根本算不了什么,而日常东京内外的乞丐也差不多有数千人。

而且开封城外流民如此惨状,乃是开封、祥符二赤县的知县不作为的缘故——开封府直管城中,城外归于县治——开封终究还是富庶之地,各县又都备有仓场,赈济本地灾民还是绰绰有余。如果他们能有韩冈一半用心,这一干流民早就处置完毕了。

王旁不屑的撇着嘴,换作是自己来处理这些流民,也不会出现眼下的场面。

抬头看看天色,王旁调转马身,返身回城。今晚在家中住上一夜,明天就要赶回白马县去。虽然很是忙碌,但王旁觉得这样的生活,比起郁闷在家中要好得太多了。

逐渐近了城门,王旁不经意间看见一名身着绿袍的官员站在门洞中的耳室前,对着一名军汉不知在说些什么。

王旁眼睛尖,一眼之间就看清了那人的相貌,到了城门前返身下了马,走过去拱手问道:“可是介夫兄?”

那人三十上下,已进入中年,相貌朴实,矮小黑瘦。他抬眼看着王旁,抬手回礼:“原来是仲元啊,郑侠有礼了。”

面对宰相之子,郑侠的态度平平淡淡,毫无热情,并不像与故旧见面的模样。

但王旁和郑侠的确有旧。王旁本来并不是擅长与人结交的性格,可安上门的监门官郑侠郑介夫,是他老相识,见了面理所当然要打个招呼。

当年王安石在江宁府时,郑侠随着监江宁酒税的父亲也就在江宁读书,便拜在开门授徒的王安石门下,算是王门弟子。只是郑侠的政治倾向,却与王安石完全不同。

两年前,王安石曾想大用郑侠,将其从光州司法参军调入京中,只是一见面,郑侠就满口的要王安石尽废新法,所以就被安排了一个监门官的差事。

到了去年,王安石要编订《三经新义》,估摸着郑侠这名学生经过了一年的时间,想法应该变了,就准备招他进经义局中编纂新义,但郑侠再一次向王安石提出要废新法。王安石也只能无可奈何的放弃了。

可不管怎么说,王安石对郑侠这名学生还是挺看重的。监门官的职位虽然不高,终究还是在京城中,可见他还是有着任用郑侠的想法。

郑侠的固执,王安石能够优容,毕竟不同于与旧党元老,争执中参杂了太多的私人利益。对于理念上的坚持,在年轻的官员中尤其多,不比沉浮宦海多年的老吏,人都磨砺得圆滑了。而御史台中尽用年轻资浅的官员为御史,也就是因为这个道理。

王旁知道父亲的想法,所以见到郑侠也并不疏离。

寒暄了几句,郑侠神色一凛,突然问着王旁:“仲元从城外来,不知蔡河边的流民有没有看到?”

王旁点点头:“看到了。”

“不知以安上门外的流民之众,仲元可有什么想法?”郑侠冷然问道。

“此岂为多?”王旁摇摇头,“若开封、祥符二县措置得力,不过数千人而已,早就该安置下来了。若论流民人众,还是白马县那边多一点。”

“白马县的流民很多?”郑侠神色一动,立刻追问道。

“是啊,已经有五六万了。小弟这一段时间都在白马县中……”

王旁话说到一半突然停了,他本想说说自己在安置流民上的功劳,但这么若是这么说就显得是太过自吹自擂了,做人应该谦虚一点。

而郑侠眼神忽而转利,沉下了脸。

……………………

白马县中的流民越来越多,人数之众,已经远远超过县中弓手、衙役的管理能力。冉觉几天来已是叫苦不迭,求着韩冈早一点出手。

对于这样的情况,此事最常用的手段就是籍民为兵。将流民中武艺精强的那一部分给收编下来,花钱给养着。不然一旦流民举事,作为中坚的力量,全是这等人。不得不说,这是个好主意好办法,能用钱粮来解决的问题就不是问题,总比出了事动起刀兵要强。只是韩冈现在还没有这份权力。

知县与知州同为亲民官,除了级别不同以外,最大的区别,就是知州有兵权——如秦州知州会兼着经略安抚使那样,基本上都会兼着一个武职——而知县没有。知州知府可以直接籍民为兵,但知县就没有资格。

所以韩冈现在就想着,究竟是将流民编组成临时的保甲,将其中精壮组织起联防队;还是再等上一两天,等王旁那边将话传到,有诏令为凭,来籍民为兵。

不过第二天一早,东京城的方向便来了带着诏书的天使,奉召而来的是天子身边的侍臣蓝元震,让韩冈不需要再多想。

“……以右正言兼集贤校理、知白马县事韩冈,权发遣提点开封府界诸县镇公事,措置畿内流民……开封府界提点司并徙往白马县……”

白马县衙之中,蓝元震抑扬顿挫的念着诏令。韩冈闻言却是一愣,一时不敢相信自己的耳朵。

提点开封府界诸县镇公事,这是王安石当年曾经担任过的职位。从职权范围上来看,相当于外路的转运使兼提点刑狱使,只是管不到东京城中。但开封府界,除了东京城,其余诸县、诸镇刑狱、盗贼、兵民、仓场、库务、沟洫、河道等事,皆由府界提点来主持。权限要远远大过一个滑州知州。

韩冈在白马县辛苦了数月,一桩桩未雨绸缪的事项做下来,在流民当真开始大举南下之后,他的这一番布置,不但证明了自己的能力。也为他争取更多的职权铺平了道路。

只不过,权力不是这么容易能到手的。

主持安抚流民之事,肯定要有一个名目。恢复滑州那是绝不可能,才不过一年的时间,就复归原状。朝令夕改,等于是在当初同意这一项行政区划改变的朝堂诸公脸上拍拍打打,而且也会让原属滑州的三县百姓同声反对。

所以韩冈原本以为朝廷最多给一个临时的差遣,如察访使、巡抚使、管勾府界灾伤赈济安抚事之类的官职。在此之前,无论是太宗、真宗、仁宗,还是今时,都有类似的任命。有先例,有故事,只要天子和宰相都相信他韩冈的才能,要得到这个位置,并不算困难。

但韩冈决然没有想到,天子竟然让他来做府界提点。只看以他从七品的品阶,还要加上权发遣的前缀,便可知这个职位至少相当于上州知州的等级。虽然还够不上望州或是次府的那一级,但也是实打实的知州资序了。

开封府中并无通判,知府以下,就是两判官两推官,而韩冈监察京城之外诸县镇公事,其权位仅次于知府,尤在推官、判官之上。而且天子甚至下旨将治所移到白马县,等于就是给了韩冈便宜行事的权力,让他措置流民时,不至受到开封知府的干扰。

得到的远比想像的要多,多到让韩冈犹豫着该不该接旨的地步。

看着韩冈挺着腰,久久没有动作,蓝元震心叫糟了,以为韩冈要辞了这份诏令。忙着催促着,“韩正言,如今天下遭逢灾异,流民遍道,官家夙夜忧叹,两宫亦是不安,但忧生民安抚不及而致乱。正言之才,天下闻名,官家遂以重任付与正言。还请正言勿要推辞,速速接旨,无负天子之望!”

韩冈回过神来,一声叹道:“为人臣者,君忧臣劳,君辱臣死。今诸路逢灾,天子、两宫寝食不安,韩冈何敢置身于外,而不鞠躬尽瘁以报?此诏韩冈不敢推辞,韩冈遵旨……”

