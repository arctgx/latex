\section{第12章 锋芒早现意已彰(三)}

一场夜雨之后,东京冇城迎来了黎明。

雨后的空气澄澈而又透明,缭绕城中、经久不散的烟气在此刻也终于消失不见。

地面上变得晶莹透澈起来。屋顶上也闪烁着莹润的光泽,长长的冰凌从屋檐倒挂了下来,高高低低,错落有致。

韩府后园的一小片梅林,已是玉树琼花。远看一树皆白,宛如珊瑚,近看则是晶莹剔透,单薄冰层下,嫩色的枝芽透着薄薄的红晕。

而日常起居的后院正厅前的两株桂树,也都换了一身新装。

“是树介!”身后略显兴冇奋的是韩云娘的声音,“三哥哥,是树介吧?”

“嗯,是树介。”

韩冈应声,只见韩云娘兴冇奋的冲到桂树下,仰起了头。做了母亲多年,性子还如小孩子一般。

正值腊月,天寒地冻,昨夜细细的雨滴落到地面、屋顶和树枝上,便立刻冻结起来。

此时人称之为树介,觉得枝条上的冰层仿佛甲胄。另外也有称其为树稼的,因为看起来像是庄稼一样。

这是冬天里难得一见的气候,冬雨一年总有几次十几次,但能在枝头凝结成冰、又如此恰到好处的就不多了。

“爹爹,这是不是雾凇?”

金娘也一同过来了,看到院中的美景,也惊喜的一声。

韩钟、韩钲兄弟俩也在,但也只有他们两兄弟。仅是卯时初,家里小一点的孩子这时候都还在睡,得再过一会儿才会起来。

韩冈摸着女儿的小脑袋,“雾后凝冰,方是雾凇。雨后,就只能说是雨凇吧。爹爹在京冇城多年,还没见过雾凇。”

“雾凇一词的词义,在《字林》中说得很明白,寒气结冰如珠,见日光乃消,齐鲁谓之雾凇。今日是雨后,的确不是雾凇。”

父亲是天下数得着的大家,王旖有一肚子的书,掉起书袋时,韩冈只能逼退三舍。

“王学究高才。”

韩冈半开玩笑的说着,换来的是王旖的一记白眼。

王旖最烦韩冈的就是他总是喜欢在儿女面前乱说话。

韩冈讨了个没趣,低头笑着对女儿道,“树介、树稼,仅指枝上结冰如壳,雾凇当然算,冻雨凝冰也能算,民间不分那么清楚。”

金娘睁着黑白分明的大眼睛,点着头,“孩儿知道了。”

韩冈抱起女儿,见王旖双眼中满是血丝:“怎么眼睛红着,又熬夜看书了?”

“前日官人不是带回来那几卷书,昨晚闲来无事,多看了一阵。”

“别在灯下看太久,伤眼睛。”韩冈笑笑,“早知道就不把那几卷《资治通鉴》带回来了。”

“奴家还盼着官人早点催司马君实早点将书写好,”

自从韩冈推荐了几位馆阁官去洛阳,也不知是不是他们想早些会京,又或者是想让同样负有编修典籍任务的韩冈感到羞愧冇,资治通鉴的编写速度陡然加快,两个月前将

“那怎么行。要是《资治通鉴》写好了,为夫可就偷不了懒了。慢慢写,最好等《本草纲目》编好了,他那边才交上来,那时为夫也有空读一读了。”

韩冈态度正经无比,王旖却抿起嘴瞪着他,开玩笑说得跟真的一样,家里的孩子还不懂事,当真了该怎么办?

“官人带回来的书,自己还没读过?!”

“哪有那个空?”韩冈摇头,他每日公务繁忙,有一点空都要写书,没有闲来看书的时间,“司马光那是写给天子、太后看的。以史为鉴,可以知兴替嘛,为夫这个做臣子读来着做什么?”

王旖哼了一声:“满口的歪理。”

韩冈摇摇头。他尊重司马光的成果,也确认这部书可以流传千载,甚至不会吝惜赞美之词——司马光的这部著作绝对当得起他日后所获得的声名。可韩冈现在还是认为自然科学比历史更重要一点,将自己记忆中那些知识整理好并传播出去,是眼下的当务之急。

“不在其位,不谋其政。为夫是参知政事,看自己该看的,做自己该做的。”

王旖道:“官人其实也该写点。听说司马君实每天都记日记,还有一部纪闻专记听来的流言蜚语。还不知那部纪闻中,怎么编排爹爹和官人。”

“司马光有话没地方说,为夫有话可以直接在朝堂上说,他只能诉之于笔端,而为夫却可以宣之于口,情况不同,怎么能做同样的事?何况圣谟国政,当在朝堂,不当在纸笔之上。”

笔记也好,回忆录也罢,基本上都是给自己涂脂抹粉,然后将过错推给其他人。

这样的笔记韩冈是看得多了。尤其是那些文笔老辣,名声卓著的文臣,几条明面上事不干己的见闻,就可以把自己犯下的过错洗得干干净净,顺道将洗下来的脏水泼到已经不能自辩的老对头身上。

韩冈有哪个空闲与人在文字上勾心斗角?韩冈的《桂窗丛谈》,以及最近在增补的《肘后备要》,从头到尾都没有涉及朝事。

尤其是《肘后备要》,韩冈这一段的业余时间,大半都丢在了这上面。世上人人都希望可以日日风调雨顺,年年五谷丰登,但万一遇到了灾害,总得有个应对的方略,韩冈的《肘后备要》便是一本相应的参考书。

他人的笔记,正篇之后就是补,补之后又有续,续篇不会重复前面的内容。而韩冈的《肘后备要》,如今是第三版。每一版都会在继承前文的基础上,修改之前错误的部分,然后添加一部分内容。

现如今韩冈正在编纂的《肘后备要》第三版,已不再局限于灾害时的应对,而是分作三篇,有日常疾病的急救和廉价药方,有灾害时的逃生与救助,现在又加上了救荒本草一篇,教导人们在灾荒时怎么寻找可以果腹的食物和清洁的饮水。

这已经不是韩冈一人的手笔了,韩冈公器私用,本草纲目编修局中有不少人被他拉了进来。韩冈希望,日后遇到灾害,不论是当地的官员,还是百姓,都能因为曾经读过这部书,或是手边有这部书作为参考,由此渡过难关。他确信,若哪天有一群人被冰雪困在大图书馆中,《肘后备要》肯定是最后被烧来取暖的,这样就够了。

劝诫无用,王旖也只能暗暗叹息,她明白丈夫是无心、甚至是不屑于此,并且有绝对的把握,不惧任何中伤。可文人用心酷毒之处,那比背后的暗箭还要防不胜防。何况作为妻子,王旖也分外不能容忍有人给丈夫的名声上抹黑,想到有小人背地里中伤韩冈,心口就是一阵堵得慌。

一阵晨风吹来,寒气逼人,韩冈不禁打了个哆嗦。

王旖见状,忙让端茶递水的下人,奉上加了胡椒的热汤,劝道:“官人,还是先去换了衣服再来。”

韩冈点了点头。

他刚刚锻炼过,只穿了一身薄衫。汗水湿透了衣服,肩上头上雾气蒸腾。方才不觉得,可经一阵寒风,顿时就感到冷了。没有抗生素的时代,感冒都是件危险的事,韩冈可不打算拿自家的性命冒险。

放了女儿下来,喝了两口热汤,他说道:“你们先去前面,为夫换了衣服就来。”

“爹爹呢,今天不上朝吧?”冇

被韩冈放下来后,金娘扯着韩冈的衣袖,仰头问道。

韩冈将茶盅递给下人,笑道:“今天爹爹休沐,都在家,不出门。”

“能陪金娘下棋吗?”

旁边的韩钟立刻就急了:“爹爹该教我们射箭了!”

才八岁的韩家嫡长子,拉着哥哥韩钲,冲着姐姐嚷嚷:“爹爹上次答应我们的。”

“爹爹也答应过金娘的。”

没几个宰辅会在待漏院中等着皇城开门,大多是踩着鼓点抵达,韩冈平常也随大流。可再怎么迟,也不会迟过家里的孩子起床的时间,在韩冈要上朝的日子里,孩子们往往只有夜里才能见到父亲。自就任参知政事后,韩冈与家人相处的时间就少了许多。

相对于严厉的王旖,对家里孩子的一向宽和的韩冈更被儿女喜欢,韩冈当然很高兴着一点,不过吵闹起来,也让韩冈感到头疼不已。

“好好,爹爹上午教大哥、二哥射箭,下午跟金娘下棋。”

韩冈和了几句稀泥,三个儿女终于不闹了,不过最后还是王旖重重的一哼管用。

让王旖带着孩子先去外间,韩冈回屋后的浴室沐浴更衣。

不论是否休沐,他早上会都在家中的小校场中锻炼一番,风雨无阻,仿佛武夫一般。寻常士人如此做,不免惹人嘲讽。可到了韩冈这个地位,就是名人的轶事了。甚至还因为他在医道上的声望,让很多士人都开始了晨间的练习,弓术是不用说了,夫子所称六艺之一,早上起来练五禽戏的文官,京冇城中不在少数。

洗了个澡,换了一身干爽的衣服,韩冈走出起居的院子,往前面走去。

院中早撒了化雪药,踩起来沙沙作响。不要等日头出来,地上的薄冰就已经在融化了。

化雪药是制盐产生的苦卤晾干后得到的产物。

如今京冇城中,遇到冰雪天,穷人家跟街上一样撒点煤渣来防滑。而皇宫和高门显宦,就用化雪药。原本根本没什么大用的苦卤,能废物利用起来,也是来自于韩冈的提议。

如果按照来源和韩冈记忆中的名字,其实应该叫做化雪盐才是,可如今盐价极贵,名字传出去,人们就只盯着盐了。以盐洒地,免不了会给人以一种奢侈无度的错觉。化雪药这个名字就好一点了,不明白的多问一句,就知道这不过是本要被抛弃的废物,是韩冈向陶侃学习,将之利用上了。

说起来,这还是王居卿的提议,在枝节问题上,韩冈的心思也没那么细腻。

地上的冰开始化了,可枝头上的冰层依然晶莹。

此景虽令人赏心悦目,不过景致的背后,却是明明白白的灾害。

天寒地冻,偏又遇上冻雨,京府内外必有不少人家受灾,而那些无家可归之人,不知会冻毙多少。

韩冈心中记挂着,也不知朝廷怎么应对。

……………………

京冇城冻雨,树木砖石上皆凝水成冰。

皇城中撒了几十包化雪药,终于是将主要道路给清出来了。

不过枝头上的冰层在日出后都没有消失。

政事堂上,向太后问起此事有何预兆,张璪便兴高采烈的向太后称贺,说是祥瑞来了。

“此名树稼,以禾为名,明年当会是个好年景。”

向太后因此欣喜不已。

回过头来,到了政事堂中,韩绛就埋怨张璪,“树稼兆丰年,这是哪里的说法,出自何处?”

韩绛有几分生气,这种哄太后高兴的话,根本就不该宰辅来说。

“淮左乡里,遇上雾凇必如此说,与瑞雪兆丰年类同。”

“绛不信邃明未读过《旧唐书》。”

“那也不过是俗谚,而张璪所知,亦是俗谚。与之权衡,还是不说那种牵强附会的无稽之谈比较好。”

所谓俗谚,出自《旧唐书》让皇帝李宪的列传。李宪看到树稼便说“树为稼,达官怕,必有大臣当之。’不久之后,李宪的确死了。死后被李隆基追赠为让皇帝。

玄宗李隆基是睿宗第三子,而让皇帝李宪是嫡长子,可玄宗在灭韦氏一役中功劳最著,而且兵权在手,李宪也只能将太冇子之位让给李隆基,自己做了一个太平王爷。

不管史书中记载的是不是事实,将树稼说成是祥瑞,以讨太后欢心,两府宰执之中,只有张璪可以毫无心理障碍的做出来。

韩绛脸皮不够厚,在朝中多年,也早看透了所谓祥瑞,对此嗤之以鼻,所以让张璪讨了个好。

韩绛又感到遗憾起来,可惜韩冈不在,否则他不会给张璪糊弄太后的机会。

与张璪之间的小小争执很快就过去了,可韩绛的胃却加倍的抽疼起来,盯着手中的急报半刻,他找来一名堂官,“去韩参政冇府上,跟他说,今天休息不了了。”

那名堂官不明所以,一脸茫然。

韩绛一声断喝,“去请韩参政速来政事堂!”

