\section{第31章 停云静听曲中意(28)}

刚刚送走了准备跟韩冈一起北上的家丁,周南正愤愤不平:“难道朝廷就没有别人了吗?!为什么总是官人吃苦受累!”

教坊司的前任花中魁首,随着年纪渐长,越发的成熟诱人,纵然是盛怒之中,依然是风情万种,如同一朵怒放的玫瑰,让人移不开目光。但她心尖上唯一的那一人,却连一声再见都没有,便赶着出京,这让周南出离了愤怒。

丈夫总是临危受命,哪里危险就被派去哪里。出生入死的经历,朝廷中哪个文臣能比得上?

“每次都是这样。官人刚刚让地方安定一点,朝廷就立刻过河拆桥,将人调回京城架起来。但一乱起来,却又想到了官人。这不是明摆着欺负老实人吗?!”

韩冈当然不是老实人,他的妻妾们都清楚。但韩冈一派为国无暇谋身的的作风,在连和平时出使辽国都视为畏途的文臣中,的确是十分罕见。每每临危受命的情形,也让人觉得这是朝廷欺人太甚。

“京中可用的统帅之才除了官人就只有章子厚。可要是章子厚走了,谁执掌枢密院?薛向连进士都不是,官人又是新手,想要理顺手上的事需要的时间不会少,西府之中离不得章子厚。何况章子厚只在南方有经验,官人可是久镇河东。”王旖的解释带着无奈,却又有几分骄傲。

“是啊,朝廷缺人。平时还好,一遇大事,真正能派得上用场的也只有官人在内的三五人!”严素心同样为丈夫骄傲,但笑容却是无比的沉重,凝聚在眼角眉梢的忧色浓得化不开。

虽然只是少了一人,但这座院子却一下就变得空空荡荡的一般,弄得她的心也是空落落的。看看最得丈夫宠爱,依然是小孩子心性的云娘,也没了笑脸,静了许多。

只要丈夫在家,就算是不声不响的坐在书房里面看书,她们也是安心的。可一旦韩冈外出,就像房子少了主梁。

悔教夫婿觅封侯。不知为什么,王旖的脑中浮起了这句诗,她很早就后悔了。就算挣回一个郡公,挣回一个国公又能如何?终比不得在家教着儿女读书识字的时候。

“可要平平安安的回来啊……”她远眺着天空,低声念着。

……………………

“出事了!出事了!出大事了!!”

一个手短脚短身形也短的五短汉子几乎是滚着冲进了八仙楼。

楼外的开宝寺铁塔上的风铃,随着风声清脆作响。而楼中则是一片人声:“打听到了?!”

只要生活在京城中,就少不了有一双好耳朵,哪个不知道今天肯定有坏消息入京了,市井中的气氛都明显不对了。一个四十多岁的中年人声音低低的,“可河北那边败了?”

“不是河北。”那个身材五短的汉子声音抖得厉害,两只眼睛睁得老大,凸起的眼珠子仿佛就要掉出来,“不是河北,是河东!河东丢了!!”

他用着介乎于尖叫和惨叫的声音高喊着。

酒楼中的一群人都跳了起来,“怎么可能?!”

那可是河东啊,有险关,有名将,前两年还把契丹人打得跟狗一样,砍了一堆脑袋,让辽国的尚父吃了个哑巴亏,哪里会这般容易就失陷,事前还连个风声都没有。

中年人指着五短汉子的鼻子:“孔二,别乱说话啊!河东怎么可能会丢?小心给抓到衙门里治罪!”

“呸,俺可是圣人子孙,什么时候乱说话过!”孔二气得往地上吐了口痰,“李家哥哥,你老贵人多忘事,忘了俺那在皇城里做事的表兄了?当真是河东丢了!”

李姓中年人默然。他是知道,孔二这个常在一起喝酒的街坊,的确是有个在皇城里当差的表兄。

“这下可不妙了。”坐在店内深处的一个儒生打扮的老头儿扯着花白的胡须,头摇了起来:“其实五代时,从河东来的贼人可比从河北来得多。后唐的庄宗皇帝【李存勖】、后晋的高祖皇帝【石敬瑭】,还有后汉的高祖【刘知远】,哪个不是河东节度使出身?就是北汉,也是抵抗天兵到了最后才被灭掉。辽人夺了河东,可比夺了河北更危险。”

老头儿的话让每个人都变得脸色苍白。

“张先生,可别自己吓自己。河东失陷是真是假还说不准呢。”孔二听了这话就又一下鼓起了眼,但那李姓中年却当没看到,“退一万步讲,就是河东当真失陷,朝廷里面也不是没能人。”

“河东失陷这事多半是真。”老头儿又说话,“你们怎么不想想,开战这么些天了,辽狗竟然还被堵在边界上。要不是他们用的是声东击西的计策,手脚怎么可能会这么慢?真宗的时候辽狗可是转眼就打到黄河边了。就算有神臂弓斩马刀,但架不住辽人有快马,见到坚城、军阵就绕路走,如水银泻地,如何阻挡得了?现在打了这么些天,辽军也没多走一步,肯定是佯攻。”

这张先生在八仙楼周围的几个坊中有些小名气,一群人对他的见识都很佩服。听他这么一说,还残存的一点侥幸之心,全都化为乌有。

当真是河东丢了!

“不过。”张老头儿话锋一转,“现如今的朝堂里面,也的确有人能挽回河东的局面。”

“是韩学士吧?”并不是人人都知道,韩冈就是从河东卸任下来的前任安抚,可遇到外寇入侵,人人都会盼着精通兵法的韩冈出来领军,但韩冈还有另外一重身份在,“只是皇后愿意放人吗?那可是关系到太子的安危啊。”

“那就不清楚了。过几天就会知道了。”

“不用过几天。”孔二摇头,“早上就在崇政殿里面,皇后已经派了韩学士回河东救急!”

“韩学士又回去做了河东安抚使?!”一名酒客惊喜道。

“不是安抚使,是什么制置使?而且皇后刚刚拜了韩学士做枢密副使。是枢密副使兼制置使。”

老头儿皱了皱眉,这个没听过的职位,估计跟宣抚使差不多。

“既然韩学士出掌河东兵马,援救河东,这几天说不定就要点将了。”李姓中年道:“前几日东门外校阅后,驻扎在白马的一个将听说就过河了。”

“这哪儿跟哪儿啊。”

“反正俺听说上四军这一回说不定也要出动。”

“胡说八道。捧日、天武、龙卫、神卫那一部是可以随便乱动的?!除非天子亲征,否则上四军怎么可能会出动?”

孔二则是狠狠的冲着地上啐了一口浓痰,也不顾身边的小二皱眉苦脸看着脏兮兮的地面,“都是那个什么吕枢密害的,这一回,河东还有河北要是出了事,肯定要拿他的脑袋开刀。”

“河北别担心,有郭太尉在呢。方才甄老不是说辽人打得是声东击西的主意。河北那边怎么都打不大。只要保住河东,辽人也就败了!”

“说的没错!这两年郭太尉就没离开过河北,河北给他打造得跟要塞一般。但河东又岂是这么好守的。”

“可不是有韩学士吗?”

“要是韩学士一直都在河东,谁会担心啊!给辽狗一对翅膀,也会被飞船给打下来!可现在他都离开两年了,辽人也早熟悉河东的现状”

酒楼中一时议论纷纷。韩冈到底能不能保住河东,一时间众说纷纭,直至日暮。

夕阳的余晖从西面照了下来,一群骑手就在楼前的大街上扬鞭而去。

骑手人数近百,大街上是人人侧目。毕竟只要生活在京城之中,不会不清楚,出行时能拉起这等规模的队伍,朝廷上也就那么几人。

“是金枪班。”

班直护卫在皇城外的驻扎地就在八仙楼附近,在八仙楼喝酒的人们,当然不会不熟悉班直们的衣帽服饰,那可是与普通士兵截然不同。

“怎么这么多?”班直是天家护卫,他们护送的岂会是寻常人,何况人数还不少,竟有四五十骑。

除了班直之外,还有二三十名身穿朱衣的元随,就是没看见清凉伞。不过从元随的数目上看,肯定是执政一级的高官显宦。

“这是往北去呢,莫不是韩学士?”一人猜测到。

“哪可能那么快!?好歹也要准备个几天功夫。哪能说走就走的。”

可也就在这时,二楼的雅座一片哗啦啦的椅子响,从头顶的天花板传了下来。紧接着楼上的窗户一扇扇的被推开,一连串的叫声在二楼响起:“是韩三学士!”

“是韩玉昆!”

能在楼上雅座喝酒的不是富贵人家,就是官宦,见识自比市井中人要多,一眼就认出了名望日隆的韩冈来。

“阿弥陀佛。想不到韩学士将国事看得这么重。”

“这一下河东算是能让人放下一半的心了。”

楼下的酒客随之轰动,就像在人群中点起了火,一下变沸腾了。只是又是一句话,让沸腾的气氛冷了下来。

“放心什么?堂堂枢密副使也走得这么急。河东的局势,肯定是糟透了。”那个老头儿冷冷的说道。

军情如火,当然要快。但快到韩冈这般,却让人们不得不为之动容。连楼中的惯常见的嘈杂,此时却化为了寂静。

“阿弥陀佛,佛祖在上,惟愿韩枢副能旗开得胜。”李姓中年口宣佛号,为韩冈祈福。

得他提醒,其余人众也纷纷为韩冈向诸天神佛祈求胜利。

暮色苍苍,马蹄声声,韩冈就在京城军民的希冀和担忧中,驰离了东京城,赶赴山岭重重的北方战场。

