\section{第九章 纵行潼关道(中)}

“章惇做得好!章惇做得好!”

崇政殿中,赵顼难得的放弃了天子的矜持,大声为前线的捷报而叫好。

吕惠卿拱手道,“章惇以才智论,犹在王韶之上。如今的胜利也仅是开局而已,大捷当在后面。”

“前日听说章惇所用非人,致使多名使节被杀。今日看来,他还是有所准备的。”

“不名其罪而以刀兵相临,朝廷何以服远人?所以章惇遣人为使。若荆蛮当即归顺,那是当然最好。如其不肯顺服,天兵征讨便是名正言顺。蛮贼杀了朝廷使节,正是自寻死路!”

赵顼连连点头,嘴角含笑,再一次称赞着:“章惇做得好。”

章惇以察访使的名义,前往荆湖两路,经制南江事。那是还是在秋时。等章惇理清了刚刚接手的一番杂事,开始时准备进兵的时候,已经是十月初冬了。

一开始,章惇没有立刻攻击,而是先派去了李资、明夷中、愿成等一干僧俗为使,去说服辰州的山蛮蛮酋田元猛。但他派出去的几人实在不成器,据说他们在蛮部之中,恣意妄为,甚至淫辱妇女,最后忍耐不住的蛮人将使节全部杀死,只留了一个愿成和尚回来报信。

这个消息被荆湖走马承受传回来的那几日,赵顼都是阴沉着脸,人见人畏,连带着宫中的宦官宫女,走起路来都要掂着脚。

不过今天终于有了点好消息。

半个月前,官军与辰州山蛮大战于武山。这一战官军出兵四千。而蛮贼十余部,各据险要,总计有万人之多。

虽然兵法有云十则围之,五则攻之,倍则战之,章惇以不到一半的兵力攻打几座位于险要地势上的寨子,看起来是个很疯狂的举动,但笑到最后的,却是章惇。

那一战,统领前军的李信当先出阵。他身披重甲,手持坚盾,身后跟着两名各背一捆投枪的小校。带着三百名从西军调来的弩弓手,就这么一直冲到了寨墙下四十余步的地方。

山蛮居高临下,一时箭落如雨。不过蛮人所用弓弩皆是绵软不堪,远不能跟大宋军中所用的强弓硬弩相比。沐浴在这样的箭雨之中,只要拥有重甲,根本是无所畏惧,而宋军的神臂弓也是轻而易举的就将他们压制。而李信,更是连续投出掷矛,转眼之间便击杀了数名在寨墙上指挥着军队回射的蛮部大将。

与此同时,章惇的亲信爱将刘仲武,领着两百跳荡,悄无声息的攀上山崖,从后方直接杀入贼军主寨。前后交击,蛮酋田元猛仓皇出逃,落于悬崖者无数。

这一战,总计攻破六寨,俘获百人,斩首三百余。对于山中部族来说,这样的损失,没几家能承受得起。

在章惇的奏章中,也充满了他对刘仲武和李信的赞赏。

刘仲武自从三年前得官之后,因为向宝的倒台,一直很悲剧在者达堡中数星星。幸好是于章家有恩,本身亦有才能,故而被章惇举荐。而李信本是韩冈所荐,前日还在笼竿城七矛杀七将,立下了赫赫威名。

李信前日上京时,赵顼也见识过了他的武艺。七支四尺铁矛,几乎是在一眨眼之间就飞到了五十步外,将一字排开的七具铁甲都扎了个对穿,完美的展现了他是怎么在笼竿城下,于千军万马之中,一举击杀敌军数将的壮举。

精妙绝伦的箭术,赵顼见识过不少。同样是关西新一代的出色将领,王舜臣的连珠箭术曾让赵顼叹为观止。但能与他相媲美的,在赵顼的记忆中,还是能找到几个人。可李信的掷矛之术,却是第一次见识到。

“李信亲冒矢石,临阵勇决。今次一胜,当以其功为首。特赠其父一官,本人则转两官,赏赐亦加倍。望其能勤谨如初,在荆湖早立新功。”这是方才赵顼口述给中书舍人的原话,让中书舍人依此来起草诏令。对李信这样的偏裨小将,竟然动用了单独的诏令,可见赵顼对他的看重。

“李信是韩冈的表兄,其父乃是韩冈之母的亲兄。”赵顼这时候心情很好,半开着玩笑,“前日朕也曾听李信亲口所说,他的掷矛之术乃是家传,就不知道韩冈他懂不懂?”

吕惠卿道:“韩冈是否懂得掷矛之术,臣是不知。不过韩冈当也是武艺过人。他在包约部中,曾经亲手斩杀西贼使者,逼得包约不得不降顺。虽然此事归功于包约,但实际为谁所杀,熙河尽人皆知。”

吕惠卿说的,赵顼早就知道,“韩冈一向以国事为重,往往推功于他人。包约部中如是,罗兀城中如是,咸阳城下亦如是。此子大有古人之风,在朝中难得一见。”

赵顼对韩冈的激赏不已,以吕惠卿之智,很容易便能明了其中缘由。一方面是韩冈本人的确功绩累累,另一方面也有天子始终想见而不得见后,在心中对韩冈的美化。

哪个隐士被征起前,不是让天子引颈而望?只是见到后,失望的不少……当然,吕惠卿也清楚,如果让天子见到韩冈,应该不会失望——韩冈本人的能力,可是远在名望之上。

现在赵顼的心情很好,吕惠卿瞅准时机,“若朝中人人如韩冈这般不爱权威,以争功诿过为耻。国事岂会如此艰难。正如那华州,地震之后已有数月之久,但陕州【今三门峡市】知州却上本,如今犹有流民在道。”

吕惠卿只是天章阁侍讲,兼同修起居注,照常理并没有议论此等朝事的资格。但他身为天子近臣,随意发上几句议论,谁也不能说他不是。

赵顼也没在意吕惠卿捞过界的行为,“眼下已经是深冬,华州之事的确不可拖延了,郭源明也的确不能胜任。依吕卿你的意思该如何处置?”

本来王安石是想让吕大防去知华州的。但赵顼觉得吕大防此人难得,便将他留在了朝中,放到了审官西院上。但现在看来,这个处置的确是错了。要是吕大防这位能臣在华州,不至于到了腊月还有华州流民走上了潼关道。

吕惠卿则道:“还是先自朝中派遣使臣前往察访,流民在道的事究竟是真是假,还有人数多寡。如果百十人,陕州在那就是危言耸听了。至于是奖惩之事,还是等救完了华州百姓,再论其余。”

赵顼默默的点了点头,吕惠卿的意见,才是公忠体国的做法。先救人,其余等赈济结束了再说。不像有些大臣,一心放在政争上。前些日子,以地震山崩为借口,请天子将王安石罢相的奏文,如雪片一般的拥往了崇政殿。反而说着如何赈济、救灾的奏文,却是寥寥可数。

赵顼多读史书,拿着灾异作为武器,用来攻击政敌的故事,他在史书上见过不少。当时就想着日后对此要警惕,科事情落到自己的头上,想法就不一样了。空穴来风、未必无因,上天的警示也许是真的,赵顼一这么想,就越发的感到心惊肉跳。幸好事情没变得那么糟。

吕惠卿冷眼看着赵顼的神色变换。

他所侍奉的这位天子,说聪明也聪明,做了近六年的天子,政事上一概门清,许多事都瞒他不过,连带着在京中的耳目消息也越发的敏锐,不再是熙宁初年时的稚嫩可比。

但赵顼最大问题便是心志不坚,极易受到外事干扰。华州地震山崩,让反对新法的一干旧党重臣群起而攻,拿着市易法为突破口,声言这是上天对天子不行德政的警示。那段时间,这位皇帝都有了废除市易法的想法。要不是王安石和他们新党中人这些拼命坚持,国事必然大坏。

吕惠卿对天子一向没多少敬畏。离着皇帝越远,才会越把皇帝当成神。换作是他们这些能天天见到皇帝的,就知道,所谓的天子,不过是个普通人。只是因缘巧合,或是前世修福,才坐到现在这个位置上。

前些日子看到赵顼心烦意乱的模样,吕惠卿私下里没有少冷笑,真是如此忧心国事,干脆下罪己诏好了。

人在天子面前,转着这等悖逆无道的念头,吕惠卿的心中有种难以言喻的快感。不过对于快感的沉迷也只是一瞬间,一呼一吸的时间中,理智就已重新占据了吕惠卿的脑海。

他已经按照王安石的命令,将华州的察访权控制住,对此旧党当无可施为。只要附近各州的救援粮一起到了,华州可保无恙。也不用担心有人会对此借题发挥了。

吕惠卿现在关注的焦点,不是在外,而是在萧墙之内。

在前段时间,竭力挽救市易法的那两个月里,身为王安石副手的曾布,却是动作很少,上书时也是将几桩新法连在一起说,并没有将市易法挑出来单独。

曾布的这个态度,天子和王安石都忽视了过去。但吕惠卿一直在盯着曾布,不会让其蒙蔽。

看曾布的反应,应是对于市易法不以为然而已,就不知吕嘉问知道后,他会如何想?

