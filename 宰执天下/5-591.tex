\section{第46章 八方按剑隐风雷(11)}

“曰本?……辽人竟然跑到了曰本去了?!”

看着枢密院的小吏点头称是,韩冈惊讶莫名。

辽人什么时候已经开始学会放眼海外了?

原来辽国攻打高丽,韩冈不算很惊讶。可现在竟然渡海去攻打曰本,他就不能不吃惊了。

不说别的,这开疆拓土的速度,就差不多快要跟辽国立国时差不多了。而且还要渡海,要是契丹人习惯海水了,那可真就是麻烦了。

曰本这个国家在大宋的朝堂上很少有人提及。韩冈也只是在苦恼硬通货不足的时候,才会联想到曰本。

远在数千里外,几百年不曾与中原通问,这样的国家既然不在宗藩体制之内,朝廷会很乐意将他们给忘掉。

曰本如今内敛自守,近乎于闭关锁国,也只有商人和僧侣愿意远涉重洋。僧侣来学习佛法,而商人则贩来了当地的土产,换回中国的珍奇。除此之外,与中国就没有更多的交流了。

曰本的特产中,最有名的就是倭刀。其锋利远胜国中刀剑,当年欧阳修等文人还为倭刀写了许多诗词。

也就是这几年,国中的钢铁业大发展。百炼钢、折锻铁,以此为名的精制刀剑市面上层出不穷,许多上等品是倭刀不能比。更有人模仿倭刀的外形,打造刀剑,冒充倭刀来贩卖。这一切,都使得正品倭刀的地位大幅下降,渐渐的没有人购买了。

少了这一最为著名的特产,曰本、倭国这些名词,出现的频率越来越低。若不是这一回因为辽国入侵高丽,有可能涉足中曰之间贸易转运,并由此控制一批海船。朝堂上更不会有人提及。

韩冈此时就是为了银矿,也仅仅是打着大理的主义。曰本的金银,那是要放在双方联络更加紧密之后才会去考虑。孰料给辽国抢先了一步。

“这是杨钤辖从耽罗岛发回的消息。”那名枢密院的小吏详尽的向韩冈禀报道,“说是有两个曰本的僧侣逃到了岛上,说是辽军于一个月前从高丽渡海,攻入了曰本国中。”

从高丽渡海,那登陆位置就是九州……这个时代似乎是叫西海道。

西海道上面划分了九州,归属于太宰府管辖。而整个曰本,就有五畿、七道、三岛,总计九十三郡——韩冈手上的资料,很多都是唐人的记录,也有太宗时,曰本来华的僧侣奝然口述的记录。有关曰本政区划分的详细资料,多数都来自于奝然。而曰本近年现状,韩冈让人去搜集过,不过并不完备,也不算详细。

“可知道攻入曰本的辽人兵力有多少?”韩冈皱着眉问道。

确定了渡海的兵马数目,就可以知道辽人的战略目标到底是不是转移到曰本,又或许只是镇守在高丽的将领单方面的独断独行。

就韩冈所知,女真人渡海去曰本打草谷是常有的事。而高丽商人的节操也不用太指望,请契丹人过海攻倭,从中大大的赚上一笔,这不是不可能。大陆与曰本最接近的位置,不论南北,都是只要渡过一道或两道窄窄的海峡就够了,独木舟、小舢板就能漂洋过海。

如果耶律乙辛当真食髓知味,攻下了高丽之后又想攻下曰本,狭窄的对马海峡决不会影响到调派的兵力数量。如果不是耶律乙辛首肯,渡海去曰本的兵力就决不会太多。

“据说是十万大军。”

“胡扯!”

小吏差点给吓得跪下来,急道:“小人不敢诓骗宣徽,确实说的是十万。”

“不是说你胡扯,是说那两个倭僧胡扯。”

韩冈暗骂自己是糊涂了。两个吓疯了的曰本僧侣要是能有准确的军事情报,那才叫有鬼。这件事得让杨从先再去查,总不能一直留在耽罗岛上帮着那些高丽余孽看大门。

“十万兵马……也亏杨从先敢报上来。”韩冈又对小吏道,“你先回去吧,待会儿我会过去枢密院与几位枢密商议此事。”

打发了小吏离开,韩冈让人回自家去取有关曰本的资料。等会儿去枢密院,他可不想在章惇等人面前露怯,必须先复习一下功课。

坐回靠椅上,韩冈眉头就紧紧皱了起来。

茶水这时候已经可以入口了,端着茶盏慢慢喝着,也不去品味这茶汤的滋味。

曰本能不能抵挡得住辽人的侵略,这一点很难说。

一场飓风,就让蒙古人功败垂成,从此神风便被曰本人顶礼膜拜。要是契丹人运气差点,说不准就能出什么意外,或许没有什么神风,但说不定会有暴雨、暴雪、地震、火山之类的天灾临头。

可是如果辽军没有遭遇灾害,凭借曰本国中军队的战斗力,恐怕很难奈何得了全副武装的辽国精锐。

尤其是刚开始的一个月,辽军的对手只会是西海道九州岛上的驻军。数量不会多,装备也不会多精良,甚至很可能遇到的都是些拿着竹枪的农兵。对辽国侵略军而言,这等战斗的难度大概仅仅比射野鸡和野兔难一些,恐怕还不如皮糙肉厚会反击的野猪。

辽军的实力从他们攻下高丽的战役中就能了解一二。真要让韩冈来评价,元丰四年的辽军,比起十年前其实战斗力应该更为强大了。精良的兵器对军队的意义毋庸置疑,铁甲的作用远比外行人想象中的要大得多。铁甲在辽军中普及,可以让他们顶着普通的战弓射出的箭矢,冲到敌阵的十步之内都不用担心自己的安全。曰本国中的远程兵器,多为竹木弓。咸平五年,曾有曰本国人藤木吉来中国,真宗皇帝接见了他,还让他以自用‘木弓矢挽射,矢不能远’。

韩冈不知道现在曰本国内的政局如何,绝大多数前往曰本的商人,都只能在港[***]易,很难打听到具体的政局变化。不过有一点可以确定,此时的曰本十分的和平安定,持续了数百年,远远没有曰后战国时代的混乱。长达几百年的和平中,偶尔才会有旋起旋灭的叛乱,这样的国家所拥有的军队,要说可以与辽人一较高下,河东、河北的无数将士,可是要破口大骂了。

在脑中推演来推演去,韩冈便越发的确定,这一回又是给辽人占了一个大便宜。拿到曰本之后,凭借国中丰富的矿藏,辽人能得到的收益将远远超过岁币。说不定能够维持一段时间的和平,而耶律乙辛也当能镇住他手下一众贪婪的诸侯。

拿到了从家中取来的资料,匆匆看了一遍之后,韩冈赶去枢密院。

只见章惇、苏颂、薛向,甚至郭逵现在都在院中。

看见韩冈,章惇起身迎接,抱怨道:“玉昆怎么才过来。”

“衙中有些事。”韩冈都无意找借口,反倒问章惇,“辽人入寇倭国,虽是出人意料,但终究是海外岛国事,不至于惊动几位枢密一起议论吧。”

“这事丢一边。”章惇很急躁的说着,“王舜臣在西域打赢了。”

韩冈心一跳,“赢了!?什么时候的消息。”

“方才才送到的。”苏颂道。

薛向补充:“大捷!”

韩冈坐下来,略嫌舒缓的动作,给他了思考的时间,坐正了身子就笑道:“万里之外,是不是大捷还不都由着他说。也就是黑汗退兵,应当不会有假。”

“确实是大捷。”章惇道,“否则王舜臣不会追击败退的敌军。”

“他南下追击了?!”韩冈闻言急问。

苏颂将一份奏章递给韩冈,韩冈结果来匆匆浏览了一番,王舜臣果然在上面声称大败黑汗军,然后为了一劳永逸,便全师出击,追击敌军。并说要趁此良机一举攻下疏勒。否则等明年黑汗本土的援兵赶来,就不会有这么好的机会了。

韩冈放下奏章,环顾诸人:“几位枢密是担心他孤军深入?”

“当然担心。”苏颂道:“黑汗与高昌、龟兹那些西州回鹘不同,不能小觑。贸贸然深入敌国中,举目皆敌,容不得他有半点大意。”

这个道理人人都明白。西州回鹘毫无退路,但黑汉国在葱岭之东只是小小的一片地盘,葱岭之西更有大片的领土,能动用的兵力少说也有王舜臣兵马的十倍。且黑汗国自成文法,又有大食教统括人心,就算一时将疏勒占据下来,等到黑汗本土援军赶来,下面的黑汗人也必然会反叛。

王舜臣得胜之后,便不顾一切的南下追击。对于他这一冒险之举,韩冈、章惇,还有其他枢密使都没说什么。

那是为古人担忧。木已成舟,担心都担心不来。输了一切休提,说不定根本就回不来。现在要考虑的是赢了之后的事。

韩冈也不觉得王舜臣会输,除了追击途中出现意外,一旦给他领军进入了疏勒地界,他所要做的,就是辽军进入大宋境内干的那些事。

以王舜臣在章疏中所提及的他在末蛮做得那些应战准备,可见王舜臣不会犯下黑汗军同样的错误。且他又是为了追击才南下,行动速度不会慢,疏勒的黑汗人只有在得知前线兵败、汉军南下的消息后才会做出反应,那仅余的一点时间,不足以完成坚壁清野的工作。从中便有了王舜臣大军生存的空间。

“那几位枢密方才讨论的怎么样了?”

“葱岭是山区。积雪消融,当不会早于疏勒。诏命应该能早一步。”薛向答非所问,却盯着韩冈。

韩冈明白他们的想法,几十万黑汗人不可能与上万官军的安危相提并论,现在在重中之重是让王舜臣和他的一万多人马能够安然无恙的在疏勒盘踞下去。

韩冈点点头,笑道:“应该跟几位枢密的想法一样,非此不得安稳。不过以韩冈之意,还是让他把事情交给回鹘人做吧。别脏了自己的手。”
