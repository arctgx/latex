\section{第14章 霜蹄追风尝随骠(16)}

近两个月的时间,黑山河间地终于差不多都清理干净了。

大规模的动用西京道的皮室军等精锐扫荡黑山下的河间地,那些以为投降大辽,放开通往兴庆府的道路就能自保的蠢货,全都被驱逐了。

本来萧十三还打算拿西阻卜的地盘交换,但没一家愿意,也只能下杀手了。这是敬酒不吃吃罚酒的典型,草场差点又如何,总比灭族要好吧。

听着传递来的报告,萧十三算是松了一口气。东边对此催促得很急。耶律乙辛应该是正急着将他的斡鲁朵给确定下来,迁移人口是件很麻烦的事,逐水草而行跟种地类似,耽搁一时,说不定就会耽搁一年。

作为亲信,若是不能及时完成恩主的吩咐,那个后果是萧十三绝不愿意承担的。

“幸好是解决了。”

正经事完成之后,剩下的就是脸面上的问题。丰州被宋人抢先一步占据,虽然萧十三心中不忿,但毕竟相比起黑山河间地来,不算什么大地方。过去曾经属于宋人,后来又被西夏占据,如今又被宋人拿回去,对大辽来说完全是事不关己。只是想到被捡个便宜,让人不痛快而已。

但堂堂大辽北院枢密使的心情不痛快,难道就这么算了不成?他当然是不甘心的。

“文美今天有没有消息传回来?”萧十三问着整理一切往来公函的幕僚。表字文美的耶律世良究竟能不能安然抵达目的地,同时完成预定的计划,萧十三对此很是关切。

“耶律团练还没有,他就只有前两天遣人送回来的一切平安的回复。不过,萧、高两位巡检报平安的文书今天都有传回,他们都已经走了一半的路程,再有三日就能抵达旧丰州了。”

萧十三点点头:“算起来,文美差不多也该到旧丰州了。希望他一切平安。”

“肯定的。宋人正缺人口充实丰州,而且招降纳叛,那可是大功劳,而且韩冈是药王弟子,仁心仁术,不可能一杀数万来解决问题。”

南下的黑山党项差不多有四五万之众,这个数目就算打个折扣也远比丰州的户口为多,只要宋人接收了他们,那么接下来,旧丰州是否安定,可就着落在这群黑山党项身上了。

宋人要想占据旧丰州,手上正缺乏户口和兵员。而且对于归附的逃人,宋人一向十分宽厚。对于边臣,能收服更多的降人,也是一桩功绩。

而且手上有了黑山的土著,日后宋辽开战,便可以驱动他们上阵。以复仇、回家为名,这可是能得到上万死士的。萧十三相信年纪轻轻便身居高位的韩冈,肯定跟他的皇帝一样,都打着收复燕云的念头。既然如此,如何能放过这数万人丁?

至于仁心仁术就不必去幻想了,能做到高官显宦,不论是在北朝,还是在南朝,心肝纵然不是黑的,也不会红得太鲜艳。

耶律世良等人仅为三部人马,人数一千出头。人多了很容易露出破绽,一千人分作三部,不但安全,也有足够的实力在宋境内部引发骚乱。到时候纵然韩冈及时稳定局面,但杀戮可是免不了的,在东京的政敌如何会放过这个机会,韩冈才智虽为一时之选,但身在南朝,又怎么可能不束手束脚?

不过还要加一层保险。萧十三想着。正好黑山河间地的问题已经解决,可以将人调回来了。

“去查查罗汉奴他们什么时候到,让他们到东胜州来驻扎,还想要拖多久?”

屯兵东胜,紧贴旧丰州,逼韩冈将主力放在边境上,无暇分心他处。这样一来,丰州后方空虚,而且为了保证前线军粮不至匮乏,必然进行大批的囤积,不可能再大笔的将粮草支给黑山党项。如此一来,黑山党项反乱的几率就更大了几分。

“呃,耶律总管派来的人就在外面候着,说是粮草不够,马力不济,想回大同暂歇。”

萧十三顿时瞪起了眼睛,怒道:“别给我睁眼说瞎话,黑山河间地有多富,我一清二楚。什么粮草不够,马力不济?几倍的亏空都能填满了!把人给我赶回去,让罗汉奴立刻将他的兵给我带来东胜!迟了一步,莫怪我军法无情。”

……………………

“杀得是不是多了一点?”

黄裳看着下面报上来的数字,只觉得轻飘飘的几张纸片上的小字,完全是鲜红的。

才十天的功夫,河东、麟府两军,就已经有七八千的斩首。而与此同时,被收留的归附蕃人,则不到三千。

一开始的十几个部族,不论是在麟府军的防线处,还是在河东军那里,只要对征调修城有所推搪,甚至一句话应答不对,就被斩杀殆尽。

本来黄裳还觉得杀人立威不是坏事,但看到这个数字,就觉得做得过头了。

“没什么大不了的。”韩冈完全不以为意,笑道:“勉仲,你是延平人啊,贵乡的风俗难道忘了?”

“……什么风俗?”黄裳皱眉想了片刻,却想不出他家乡里的风俗跟这个话题有什么关系。

“就是溺婴啊,建州可是有名。”

黄裳闻言立刻叫起屈来,“龙图,鄙乡延平在南剑。溺婴之风,乃是建州恶俗。建州溺婴的确惨不忍闻,泰半人家因为养不起,只留三子一女,或两子一女,甚至有的到了为防分家,只留一个儿子的地步。但闽地十里不同风,南剑与建州可是差得远了。”

黄裳噼噼啪啪的叫了一通冤。韩冈歉然笑了,“是我误会了,勉仲勿怪。”

不过他心里可是在摇头。建安、瓯宁、剑浦同在一条建阳溪边,只隔了数十里,有河水沟通往来,哪来的十里不同风?一座龙焙监,可是管着建阳溪边所有的茶场,上下联系可是紧密得很。

“也只是打个比方。”韩冈说道,“溺女婴的事不用说了,天下各地难免。若是家中贫寒,不能养活;或是怕后生下的儿子争产,闹得毁了家业,连男婴都会抛到水里。自家的儿女养不活丢到了河里都不心疼,杀些外族的流民又算得了什么?既然养不活,与其等他们闹得州中生乱,还不如先行解决。”

黄裳无奈的苦笑起来。韩冈在关西生长,与西夏又有血仇,肯定是从来不觉得杀些党项有什么大不了的。

韩冈笑了笑,也沉默了下去。这算是有点强辩了,其实他也觉得杀得有些过头,让修城的人手变得太少。但他并不打算为此下令,朝令夕改对他的声望没有好处。

而且还是有变通的办法,在放过大批归附的黑山党项的基础上,斩首再多个一万两万也不是不可能,但韩冈很疑惑,怎么到现在还没有动静。放在西军那边,早就是蜂拥而上,争抢起来了。

冒功、谎报,可不是王舜臣一个人的喜好。过去在陕西的时候,韩冈见得多了。难道河东这边,当真这么纯洁?

……………………

杀得太多了。

李宪也在嘬着牙花。实在多过头了。收留下来的三千归附蕃人,麟府军和河东军各占一半,但近八千斩首,却是河东军占了三分之二以上。

从两边的差别来看,折克行那里,比起多少斩首,对节省民力一事更为在意一点。作为府州知州,以及折家家主,加上之前也有了功劳,这么做是必然的选择。不过两边的数据如此之大,还是很让人头疼。

但李宪也没办法,下面的将校可是对斩首功心热得很,要不是李宪这两天三番几次的强调要留些人下来充门面,早就下手杀光了,连一千多归附蕃人都不会留下来。

为了不把斩首功都吓跑,子河汊大营那边甚至还不惜人手,将尸首都拖回来。免得给那些南下的部族看见了,都不敢再过来。

下面的部将绞尽脑汁都想多杀一点,恨不得所有的黑山党项换成斩首,李宪有没有一言九鼎的能耐。寻常的时候,能让他们听命,遇上功劳在前,根本阻止不了。

斩首多了,并不一定是好事。没有相应的伤亡,就代表不是经过战斗得到的收获,而是纯粹的屠杀。或许有人打着趁机将过去吃的空额给销账,编造出伤亡,顺便还能得到大批的抚恤。可这些事哪里是那么好遮掩的?河东人多眼杂,不可能将所有人的耳目都瞒过去,最终肯定会报上给朝廷。

而且韩冈本身都不会将这些斩首的真相隐瞒天子,李宪也不会。不过以天子的为人,只要能见功,当不会在意将领们的一些私心。可是一旦做得太过分了,尤其是将大量归附蕃人屠杀,让天子在心中记上一笔,怎么都是得不偿失的一件事。

思来想去,李宪还是没想到一个解决的

“经制,折府州遣人送信来了。”

“什么?”李宪闻言愣了。

有韩冈在,折克行绕过经略司私下里联络自己其实挺犯忌讳,就算以韩冈的为人不会在意,但李宪不觉得自己需要冒这个忌讳。

李宪考虑了片刻,方道:“……让他进来。”

