\section{第38章 何与君王分重轻(15)}

送走了前来传达口谕的中使,韩家重又平静下来。

不过王旖发现,回到后院书房的丈夫脸上并无喜色,皱起的眉心处还参杂着疑惑。

王旖很少见丈夫露出这样的表情,总是自信满满的韩冈难得有皱眉的时候。

韩冈坐回躺椅上,王旖在身后为他捏着肩,轻声问,“官人,怎么了?是有哪里不对吗?”

“不用担心,没事的。”韩冈敷衍了一句,扭了扭肩膀,道:“还是捶着吧,你手上没力气。”

“就知道使唤人。”王旖啪的用力拍了一下,倒也依言有节奏的捶了起来,又问:“到底是怎么了?”

“没事。”韩冈闭起眼睛,舒服的享受着,眉宇间稍稍放松了一点,“今天去你五婶婶那里,。”

“其实是娘找我去的。在家里不方便说。是为了大嫂的事。”

“大嫂的事?”韩冈想了一下,就猜到了:“……岳父岳母打算让大嫂再蘸?”

王旖的大嫂就是王雱的遗孀萧氏。在王雱去世后,三年孝期满,依然心思坚定的要为亡夫守节。

士林舆论中,主动守节的孀妇都会受到尊重。在感情上,王安石夫妇也觉得很欣慰,而且他们也不希望长孙没了母亲。不过王安石还是觉得,大儿媳正值年轻韶华的时候,总不能误了人家下半辈子。

自汉以来,历朝历代都有奖励矢志守节的烈妇,但改嫁的妇人也不会受到歧视,视若平常而已。此时也是如此。甚至有的士大夫还鼓励或是乐见改嫁。比如范仲淹,其寡母就是带着他改嫁朱氏。在范仲淹中进士前,更是一直姓朱,名为朱说。所以他在为家族设立的义庄中规定,孀妇再嫁,义庄是要给钱做嫁妆,而鳏夫再娶,则一分钱没有。

不过寡妇改嫁事一般是娘家人来主持,将女儿拉回家,然后再寻一门亲事。但王安石贵为平章,天子之下一人,萧家的地位不知差了多远,王安石不发话,他们又哪里敢自作主张?

现在王安石看得开,吴氏也表示支持,还避开萧氏,拉了王旖去王安礼那里商量,看这个意思就是要全家动员说服萧氏。

“爹和娘的意思,萧家那边离得太远,还是在京城好些。就当是嫁女儿了。”

“嗯,这是好事。”韩冈点点头,表示支持,其余的他也不便多说。

王旖的手慢了下来,声音沉了,“过得也真快,一晃都四五年,栴哥也都十一了。”

“白驹过隙啊。”韩冈有着同样的感慨。

王雱的容貌,韩冈已经记不太清了,不过当年在学术上争执,在新法上携手,共同应对天灾[***],那一幕幕尤在眼前。那时候,他自己还不过是个刚入朝的小官,王雱的官位更低一点,可都是意气风发的时候。自己刚刚考中了进士,文资武功皆备,正欲大展其才,而王雱则是成为了人人都羡慕的经筵官,能利用给天子讲学的机会,来维系新法。

“这一回为夫也算是做到经筵官了,不过终是比元泽迟了好些年。”

经筵。

韩冈的话提醒了王旖,让她想起方才的事,“官人,方才到底怎么了?是不是不想上经筵?”

“不是,是太快了。”韩冈也没再兜圈子了,“官家的姓格是轻燥,可也不该反应这么快才是。经筵可是那么容易开的?才给太子上过课啊。”

王旖悲恸伤怀的情绪一下就消散不少,“可现在不就是开了吗?”

“但总觉得哪里不对劲……哎,终究是不省心。”

韩冈本没打算这么快再见皇帝。

在他的计划中,给太子上过两三次课后,就应能在京城里面掀起一股研究气学的风潮,与王、程两家开始正面交手。

韩冈说气学惟诚于实,只用事实说话。学术高下和道统归属姑且不论,现在他就正是打算用事实证明他更适合做帝师。

在这个以儒学为根基构筑了意识形态的时代,一切自然科学都是社会科学。当诸子百家说起寓言,当后世学者以政治姓和社会姓的目光去诠释经典,世人也都习惯了从自然万物中寻找微言大义的成分。

韩冈丢了三道题出来,有引人研究气学的用意,也有讽谏天子的成分,当然,培养赵佣对数学、物理的爱好,同样是重点。

反正是仁者见仁,智者见智。

韩冈如此自信,他所依仗的,就是除了气学,其他学派都无法对一干自然现象和实验结果作出合理的解释,而这些现象或结果,当韩冈拿起来作为武器之后,便成了无法绕过的话题。

当道统相争时,最激烈和直接的手段无法使用,最后的结局将只可能遵循韩冈所了解的历史那样发展。

这是一场必胜的战役,胜利仅仅是时间的问题。但天子的经筵,打乱了他预定的计划。

“可能上经筵终究是好事……”

韩冈叹了一口气:“天子开经筵,什么时候说过只有为夫一人?”

王旖的手停了,犹疑道:“难道说……”

“或许岳父和伯淳先生也会被请过来。”韩冈说道,眉头又皱了起来。

舌辩群儒,而且还不是普通的儒者,而是王安石与程颢这样留名千古的饱学鸿儒。想要赢过他们,难度肯定不小。

不过如果是在公平的情况,他还是有胜利的自信,可若是主持人在议题上有所倾向,气学的特点得不到发挥,却有大败亏输的可能。

“一场比赛,裁判的倾向是关键。”不论在是蹴鞠赛场上维持比赛秩序,还是赛马时判断抵达终点的先后顺序,又或是学术交锋,胜负谁属最后还是掌握在裁判的手中。

“不至于吧?”

当今的这位赵官家拉偏架的时候还少了吗?哨子跟木炭刻的一样,里外都是黑的。韩冈摇摇头。只是另一边还有王安石,总不能说得太过份。

“有备无患。”他说道,“凡事可以往好处去想,但必须要做好最坏的准备。”

王旖为韩冈担心起来,“要不要派人去打听一下?”

“用不着。还是稳重一点为好,左右明天就能知道了。”韩冈回首笑道:“娘子,你的手也可以再重一点,可别停。”

……………………听说了韩冈在太子课上到底做了什么,章惇第一个直觉就是下战帖。

玉昆到底想说些什么?

章惇一直都在很努力地去了解气学。

他一直觉得,只有真正有所了解,才能决定自己立场。

在章惇看来,现在的气学已经完全不是圣人之学了,而是韩氏之学。可偏偏韩冈能东拉西扯,让人看不出破绽来。

本于实,诚于实。

这话说得不错,而且永远不会错。一切以事实为重,所以韩冈可以光明正大的宣称他的学问是属于气学,跟前人截然不同。

可谁能说从事实中归纳出来的结论一定会合乎圣人之学呢?

如今的儒门,对圣贤经典的态度,基本上都是随意裁用。觉得合用的就留下来,不合用的就说是杜撰、附会。但韩冈的态度则更偏激,甚至放弃了对儒家经典的解释。他很少阐述自己对经典的诠释,而是选择从实际着眼。

韩冈说‘诚于实’,可没说要诚于《诗经》、《尚书》、《论语》、《春秋》、《礼记》。作为一派宗师,都少不了为经典写一些传注。可韩冈什么时候给五经写过传注?

相反的,还通过指出经传中有关自然的错误,如螟蛉有子,腐草化萤等事,打破儒门经典的光环,设法降低其对气学的干扰。这比王安石直接攻击《春秋》三传为后人附会,张载说《易》传十篇只有四篇为真,还要更狠一点。

就是心太大了,想想就该知道,不会是那么容易的事。

会剑走偏锋的原因,就是不能以煌煌之兵临堂堂之阵。章惇身为枢密使,又曾为一方方面大帅,哪里看不出来。玉昆之学不为不善,可惜对圣人之教却不甚看重。一步错,步步错。

但章惇不打算反对气学,或是新学,学派之争离得他很远,都当上了宰辅,有几个会被卷进去的?站一边看着就好,没必要将自己给牵扯进去。

只是韩冈似乎不这么想。还在给太子上课时,提到自己的名字,还有那个沈括。

章惇有些后悔,早知道当年就老老实实去下那盘棋了,输了大不了浑赖。

章惇其实不通算学,可他精明厉害,韩冈既然敢拿百贯赌金去赌,肯定是胸有成竹,而且不是一般的情况。章惇了解韩冈的为人和姓格,绝不会上当。至于沈括,在数算上的才气,或许韩冈都比不上,韩冈给出的题目,他说不定直接就算出来了。

再等等看,肯定会有变化,章惇心想着。而他很快就得到了新的消息,新的变化。

“哎呀呀。”章惇听到消息,就忍不住叫了一声,还带着点幸灾乐祸的口气,“这是石渠阁?还是白虎观?”

