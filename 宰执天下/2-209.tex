\section{第32章 吴钩终用笑冯唐(16)}

【不好意思,迟了一点。】

“为何不可?!”

“设立秦凤转运司,分明是意在河湟。横山大败,环庆兵变。试问关中先因进筑罗兀困厄在前,后有环庆兵惊扰于后,如何还有余力再谋划河湟。”

文彦博直接把话挑明了,现在他落在下风,容不得他耍弄再云山雾绕的说话技巧。

“文卿误会了,秦凤转运司的确能有助于河湟之事,但秦凤、泾原的缘边寨堡,受益得却更多。何况即便秦凤转运司设立,等到能有助于河湟,也还要一段不短的时间。朕不想生民受累,不会急于求成。”

“上有所好,下必效焉,看到朝廷下令设立秦凤转运司,有哪人能体会到官家不愿生民受累的苦心?如何不会自以为是的来迎合上意?秦凤转运司一旦设立,秦州缘边必然战事不绝!”

文彦博一点也不委婉的把赵顼的话顶了回去,毫不理会赵顼的辩解。

其实以文彦博的想法,并不是打算如此挑明了顶撞天子,尽管他是元老重臣,并不用担心这点小事能把他怎么样。但与天子过不去,等于是在刀尖上走路,一次两次无所谓,但迟早有一天就会栽上去,终非好事。只是眼下的局面,被远在陕西、刚刚卸任的韩绛坏了预定的计划,让文彦博变得无从选择。

就算现在,文彦博还是会疑惑,韩绛的奏章怎么跟过去完全不一样了。

韩绛从来都不是行事谨严的人,写的奏章也从不是一条条纲目罗列。面面俱到、不厌其烦的叙述方式,分明是循吏书写公文的手法,在文学高选的朝臣们的奏文中,几乎无人使用——奏疏和下发的公文在文体上本也是两回事。

而且韩绛在已经被确认卸职调任的时候,照常理该是上谢表进行谢罪,同时感谢天子的宽容和恩德,而不是上书来为自己收拾残局,这本是郭逵的工作,也不符合韩绛的性格。

文彦博忽然警醒过来,韩绛是不是换了幕僚了?连同让王中正到叛军那里送死,洗脱跟自己的干系,这分明是军中将帅处置想杀又不方便杀的部属的行事手法,韩绛过去没带过几次兵,怎么可能用得这般纯熟?

赵顼隐隐有了一点脾气,文彦博实在太不给他面子了:“秦凤缘边安抚司,无论将帅谋士,皆是一时之选。此前连番大捷,功勋不在横山之下。就算开启战端,当也是会有捷报传回。”

“横山那里何尝不是连番大捷,但还不是无功而返?”

“种谔、张玉没有败!罗兀城那里是大捷!”

赵顼强调着,他在罗兀城已经看到大宋军队的强势。可以说自赵顼登基以来,宋军在战场上几乎没吃过亏。只要不是主帅犯浑,最差也能自保。如果摊上一个有才能的将帅,比如种谔、比如王韶,又如张玉、高永能,还有燕达,只要他们出手,那结果就是大捷。

捷报如此轻易,哪能不让一直想着讨灭西贼、收复燕云的赵顼,急着想看到一个阶段性的成果。但拥有如此强军,最后却不能如愿以偿,赵顼哪能不后悔派错了人?

“原本是不需要撤离罗兀的!”他再一次强调着。

“撤守罗兀,势在必行。自古从未有国中内乱,大将能建功于外者。”接下来的话,文彦博没有明说,但锐利的目光就是在质问。难道这不是陛下的旨意?

“朕在京中,西事不明。若是韩绛有郭逵的胆略,朕的旨意,他完全可以推掉。朕可是给了他便宜行事之权!如何能让一个郎中夺了权柄?!”赵顼对韩绛有着几分怨恨,但更多的还是赵瞻,何必如此卖力。

朕让你传诏,让你体量军事,有让你插手军务吗?

赵顼全然忘了当日官军将叛军围困在咸阳城的军情传来前,自己连续数夜难以入眠的日子;还有消息传来后,他终于酣然入睡的那一夜。

在无法确定罗兀城能否抵挡梁乙埋大军,再加上吴逵的叛乱,赵顼和两府都只可能选择撤军。谁能保证后面不会有第二个吴逵。但撤了下来后,再看一眼收获,对这个决定后悔的,决不止赵顼和韩绛。而因后悔而迁怒到赵瞻头上的,则绝对有赵顼一个。

赵顼的话中,显而易见的对赵瞻很不客气,文彦博知道不能助长这样的想法,他当即质问道:“赵瞻忠于职守,恪守君命,臣不知他有何错?是错在将叛军围堵在咸阳?还是宣读了放弃罗兀城的诏书?!”

对于文彦博的强硬,赵顼有一肚子驳斥之词。但皇帝的身份,让他不便于臣下出言争执,那样做有失体统。只是反驳的话堵在嘴边说不出来,赵顼都感觉憋得难受。早知道把王安石一起叫来,或者口才出众的曾布、章惇也行。

君臣两人一对一的时候,吃亏的往往是天子。而且就算被臣子喷了满脸口水,还必须要虚心接受,否则就是拒谏的罪名。自真宗之后的几个天子,在惯出了脾气的文臣们面前,没一个能强势得起来。

让天子无话可说,这才体现出了元老重臣的本事,轻轻松松就扳回了局面。只是文彦博还要趁胜追击,让赵顼放弃设立秦凤路的想法。

“赵瞻行事谨严稳重,对君命兢兢业业。哪如种谔,一次侥幸功成,便自以为功,日后都想着侥幸行事,期望能一步登天。如今的大挫,种谔岂无罪责?”

“种谔有功无过!”

赵顼很坚定的要保种谔。三军易得、一将难求。种谔、张玉还有高永能这样的帅才,赵顼保护还来不及,哪能将他们治罪,“今次之事,罪名不在他们身上。”

横山攻略功败垂成,实在不关种谔的事,即便河东军被伏击,使得罗兀防线被撕破一个大口子,但靠着种谔和他麾下众将的努力,使得罗兀城依旧安稳。要不是庆州兵变,局面绝不至于如此。

“种谔之过或可商榷,但韩绛用人不当的罪名,却是他洗不脱的。”

“王文谅已经死了……战死!”

在王文谅已经战死的情况下,其实逼反广锐军的罪名,已经栽不到任何人头上。不论王文谅犯了多少错,不论是不是他逼反了吴逵,因为他忠心耿耿,忠心到为国赴死的地步,单是‘忠勇’二字,韩绛信用王文谅就不能算有错。

现在在赵顼的心目中,横山攻略的失败,除了吴逵祸国,就是赵瞻坏事,韩绛只是担着一点微不足道的罪责而已。而且韩绛的处置早已决定,文彦博现在重又提及此事,不知是在转着什么想法。

通过一些有关联的人、事,从侧面慢慢造势,声势起时便单刀直入,这是文彦博常用的手段。刚刚过去的一番对话,就是文彦博手段的明证,只是被韩绛的奏章给堵住了。但现在赵顼看文彦博说话,分明又是故伎重拾。

“无人有过,人人有功,可战事自败。臣不知区区一个吴逵,能不能但得下这些罪名?军心不稳,岂是可以等闲视之?……臣请陛下休兵止戈,且还陕西百姓数年清净!”文彦博一下跪倒,言辞恳切的求着赵顼。

赵顼连忙让这位老臣起身。见文彦博反对得如此激烈,看起来很有可能会以请郡相要挟,赵顼一时无法作出决断。他需要一个元老重臣坐镇朝堂。再说契丹人最近插手了宋夏两国之间,赵顼知道他需要一个知兵强势的枢密使,而不是一个没有经历过战阵的执政。

为了维护朝堂内的势力平衡,赵顼不得不选择文彦博。就算秦凤转运司能短时间内设立,但对于河湟的帮助还要登到六月夏收之后。既然如此,此事过两个月再提也不迟。

赵顼还是纳闷。

文彦博到底是为什么如此反对设立秦凤转运司,是在怕河湟那里立功不成?但也不该这么急,无论哪一项要出成果,肯定还要耽搁时日的。

设立秦凤转运司,首要划分的就是钱粮。将陕西转运使路一分为二,对河湟之事,好处甚多。但要将区划、收支等一系列权责划分清楚,就跟兄弟分家一样麻烦。

还有在郭逵改任长安后,谁去担任下一任兼任秦凤经略使的秦州知州?这也是要需要考虑到问题——不管怎么说,都必须是支持河湟开边的人选,而且野心不大,没有与王韶争权夺利的想法。。

加上新的秦凤转运使?又该分派给谁人?——赵顼准备先进行考察,要到最后有结果,还是要稍等几个月的时间。

想到这些,赵顼都不由得头疼起来,人事上的选择从来都是困扰着所有文武官员的问题。相对而言,还是量功记赏的工作轻松。给参加了战斗的诸多将校的封赏,现在已经初步定了下来。在最终败阵的情况下,赵顼仍是尽量给他们最多的回报。等到王中正和赵瞻回返,在参考了他们的报告之后,便能定下最终的结果。

不管怎么说,赵顼一直都很大方的。

