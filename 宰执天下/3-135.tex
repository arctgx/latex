\section{第36章 望河异论希(一)}

在京城又多留了一日,不过次日晨起,韩冈赶着城门刚刚开启就往回赶,入夜之前,就抵达白马县。

回到县中,韩冈不急着去后院见妻妾,而是拉着王旁、方兴和游醇问着这段时间白马县中的情况。

白马县并没有什么问题,韩冈这段时间尽管不在,但他留下的幕僚团队依着既定的方针处理府界提点司的事务,而县衙的一干属僚也都密切配合,加之陆续调来白马的提点司吏员,尽管流民渐多,却并没有出什么大篓子。尤其是侯敂在接手了县务之后,诸多事宜处理得很得体,让几位幕僚赞赏不已。

王旁赞了侯敂两句,又担心起来:“侯县丞做得很好,就不知新知县到任之后,他会怎么样做?”

“白马知县暂时不会除人。”韩冈为此已经跟王安石提过了,正好白马县的职位安排是属于堂除范围——也就是归于政事堂管辖,而不是审官东院,“这两个月都会由侯敂继续代管,省得在此事上面分心。”

魏平真这时从厅外进来,“胙城县的终于有回音了,说是已经将地界画好,只等提点司安排人手过来修造流民营。”

韩冈面色微沉:“怎么胙城现在才有回复?韦城县六天前就已经将事情办好了!”

他是在白马县接任之后,就在公文上盖了府界提点的大印,让人送往韦城、胙城两县,让他们在官地中,给流民营划出位置。韩冈还在东京城的时候,就收到消息说韦城县有了回音,本以为胙城县也不敢拖延,没想到到现在才有回复。

魏平真喟叹道:“胙城县之前始终都没有消息。前两日在下派人去催,胙城知县阎簿也是一再拖延,一直在叫苦,就是不肯给个准信。”

“哦,是吗?”韩冈笑了笑,“现在倒是爽快了!”

方兴冷笑道:“谁叫四月初九下了雨,今天看样子又要有一场雨……”

一场雨后,王安石重新坐稳相位,那等观望风色之辈,当然知道该如何取舍。

“此辈小人只会见风使舵!”王旁愤愤不平。

韩冈笑笑,他在京中的一段时间,韩冈将开封府二十余县的档案图籍都看了一遍,虽然仅是大略看过,但心中好歹有了点数:“若是交友往来,倒要看一下小人君子。可这治政上,还是得看理民的手段。阎簿这两年的考绩,都要在韦城知县吴椿之上,即便不论税赋,胙城县户口的增加比例也比韦城要髙。去岁夏日的一场时疫,吴椿报了四百三,而胙城则是一百三十六人。”

“也有可能是作假。”王旁不服气的说着,“希合上意的事情可从来不少。上面喜欢看到河清海晏,下面就会有小人附和……”

王旁反驳了两句,就突然停了口。这么一骂,差不多就要骂到自己老子头上了。

方兴笑道:“其实也有那等故意夸大灾情,而让朝廷派下钱粮赈济的官吏。他们的官声和口碑往往都要过人一等。”

的确也有这样的官员:不清查田地,不清查隐户,遇到一点小灾就立刻向朝廷报灾,要求免税免赋,并开仓赈济,自诩为视民如伤。这等人,在治下百姓眼中当然是好官,而他们的口碑也能在士林中传扬,得到举荐的几率反而要大过老老实实做事的官员。

“其实这也是奸!”魏平真叹道。

游醇却摇头:“百姓宽得一分就是一分。更何况报灾也不会年年都报,路中监司也会派人下来察访。”

“‘夫诚信者,君子所以事君上、怀下人也。’欺君难道不是罪?”方兴反问着:“若天下州县皆如此,朝廷如何治事?”

“不说这些事了,扯得都没有边际。”韩冈拍了拍手,打断了眼看就要开始的争执,“只要韦城、胙城两县愿意配合,我这里也就阿弥陀佛、谢天谢地了。”

韩冈这么一说,在座的几人都笑了起来。在座的哪个不知道韩冈的厉害?

阎簿、吴椿其实该庆幸自己的配合,真要拮抗到底,韩冈的手段能让他们后悔一辈子。

杨绘去了鄂州;诸家现在连庄子都不敢出;三十七名粮商已经绞死了五人,流放远州的有十九名;现在的郑侠,眼见着也要编管远恶军州;再往前,向宝、窦舜卿皆在京中修养,几年都没派到差事。韩冈下手之后,有几人能安安生生的继续过活的?

笑了一阵,又说起了正经事。

王旁道:“三座新创流民营,水井、沟渠、引水道等诸事都已完备,石灰也都铺洒过一遍。修筑这几座流民营的六千民夫,依照提点的吩咐,都已经率先在营中住了下来。”

方兴也道:“在下也已经与白马各乡乡老约定好了,流民营出产的粪肥他们都会包下来。”

虽然是腌臜了一点,但出售粪肥的确是此时的一门大生意,而掌握这么生意的粪行在各地州县中的势力,都能排在诸多行会的前十位,甚至粪车每日进出所缴纳的城门税,也是任何一座城市的一宗大项收入。大户人家靠着出售此物,对家计也不无小补。而提点司也不会放过这门填补亏空的买卖,按照韩冈的吩咐,将行会撇在一边,自己直接与消费者对接。而流民们生活在营中,一切都是受着赈济,在这方面也不会站出来说要分肥。

方兴笑着:“有着几十万流民在手,单是粪肥一项,一年都能有十万贯的出息。”

韩冈苦笑摇了摇头:“流民怎么可能全都留在京畿?都要逐渐转移到外地去的。而且,最近可能要整修洛阳到大名的一段河防。流民都会派上用场。其中三分之二的精壮,都要离开京畿之地。”

“河防要得了十万流民精壮?!”游醇惊讶的问道,“之前不是说只要两三万民夫进行修补吗?”

“事情有变,今年对大堤会有个大的整修。而到了秋冬,就要起大役了。”

这就是韩冈提出的束水攻沙的方略所带来的结果。处置流民最好的办法就是让他们有工作养活自己和家人,以工代赈一直都是这个时代安置流民时,最为常用同时也是行之有效的手段。

尽管束水攻沙的方略可以说是韩冈在听说了浚川杷之后才想起来的,但在他事前的规划中,整修河防一开始就被列为一个大项。

兴建工役,可不只是开封一府的任务,这是整条黄河流域的大事。西京洛阳到北京大名的黄河曲折上千里,其中京畿一段只是一小部分而已。只要能让天子下诏,募流民兴建工役,将流民礼送出境,他肩头上的压力立刻减去一半。

韩冈详细解释了一番后,笑道:“等到流民移往西京去筑堤,提点司这里就能轻松一点了。”

不管怎么说,这付担子,他都没想过要从头到尾将之全数挑起来。

今天河北旱、京畿京东【山东、淮北】旱,两淮旱,两浙旱,但京西却没有什么旱情,差不多能肯定是丰收。两个月前,还听着熊耳山、方城山一带,因为连绵春雨,加上山上雪化,导致了桃花汛爆发。暴涨的白河、堵水【唐河】差点破堤,淹了邓州南阳和唐州的泌阳。

看着京西的好年景,韩冈一直都在想着该如何将负担让京西也带着分担一下。如果能让旧党顺便转移一下注意力,那就更好了。若是将全部的精力放在抨击治河之策上,韩冈处置流民起来,耳边也能清净一点。

不过,那也只是附赠品,有也好,没有也无所谓。

韩冈精通水利,在座的无人惊讶,如今的官员少有不习水利的。对于河防,王旁、游醇都能说出个道道来。既然韩冈治政出类拔萃,他在水利上的见识当然只会更高。

魏平真等人静声思考韩冈方略中的道理,方兴则试探的问着韩冈:“提点献束水攻沙一策,不知是否可以提举其役?”

“你说呢?”韩冈笑着反问。

方兴脸色一黯,叹了口气,“可惜。”

韩冈倒不觉得可惜,他并不指望自己能提举河防工役。黄河之重,有如泰山,要坐上河防工役的提举——从此次修整河防的规模上,应该会冠以‘都大提举’的前缀——他的地位、资历都还不够高。而且还要协调沿途州县,从诸路调集物资、力役,都必须有着能与一路监司主官分庭抗礼的资格,甚至要更高一级,这样才能保证顺利整项工役顺利而无所阻碍。

王安石的手底下,只有吕惠卿勉强够资格,而章惇和韩冈都差得远。要知道熙宁初年时,赵顼都有让司马光出任都提举的想法。虽然被吕公著否决,但从其中也可以证明只有司马光一级的声望或地位,才有资格就任这个职位。

当然,还有一个变通的办法——就是任用宦官。

不过这就不干的韩冈的事了,他现在最重要的任务只在眼下。

