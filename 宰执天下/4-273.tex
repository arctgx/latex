\section{第44章 本无全缺又何惭(上)}

韩冈没有去干涉厚生司的人事任命。

政事堂的几位宰执肯定都不会喜欢有人插手他们的职权范围,而韩冈本人更不可能去担任这个职位。

论级别,厚生司不会比军器监更高;论地位,中书五房,更为重要一点。韩冈在去广西之前,就担任了判军器监,推掉了中书五房检正公事的职位,以他现在从四品的官阶,加上龙图阁学士的贴职,不可能再去管理中书门下的下属机构。

这事就让赵顼和王珪他们费心去好了,韩冈如今正在风尖浪口上,还是少说一句比较好。

赵顼瞥了韩冈一眼:“朝堂上通医术的倒不少,明医理的却不多,韩卿名位已高,不便就任此职。可想找出个适任的,倒是难了。”

韩冈捧笏躬身:“臣也只是侥幸能有一愚之得,不敢说明了医理。不过执掌厚生司,陛下择一长于政事的老成官员即可,至于其下防疫厚生之事,自有专才负责。陛下选择群牧司中官吏,当也并不是看他们会不会养马。”

赵顼微微一笑:“韩卿所言在理。”转过来对三名宰执道,“不知诸卿有何推荐?”

元绛道:“厚生司若以依军器监例,当以两制以上侍从官判司事。权厚生司,当也得侍制以上。”

韩冈用眼角余光瞥了这位比王安石资格还老的新执政,虽然有自己挡在在前面,但厚生司的差事做得好了,名望功绩都不会少,而且做起来似乎也不难,眼下只要派人下去传授种痘之术就可以,是块难得的肥肉,想要的人自然不少。但如果抢了,得罪的人更多。这是老狐狸一个,只定下框框,不去得罪任何人。

吕惠卿想要这个职位,但亲附他的官员品级都不高,侍制这条底线就全给挡了。至于其他人选,吕惠卿不觉得自己有必要为了他们,而得罪更多的人,只能暂且保持沉默。

“还请陛下示下。”三旨相公请天子来发话。

赵顼想了一想:“就安焘吧。他刚从高丽出使回来不久,应该还没有新差遣。”

“陛下。”王珪为难的说道,“安焘两天前已经定了判将作监。”偷眼瞧了赵顼一下,“不过还没就任。”

“改判厚生司事。”赵顼爽快的敲定了安焘的新职位,对韩冈道:“安焘是个难得的老成之人,算是正合适。”

吕惠卿眼神闪过一丝惊疑,怎么看起来赵顼对韩冈的话这么放在心上。

王珪这时已经捧笏弯腰,“臣遵旨。”

“陛下。”等王珪直起腰杆,元绛就跟着道:“厚生司新立,事务繁芜,官吏未定,光是一判本司事远远不够,本司判官也当先行定下,以便能早日恩泽百姓。”

“说得也是。”赵顼点头道:“就从中书五房中挑一人来担纲好了。”

元绛闻言神色一僵,只是瞬息之后就又变了回去。韩冈全都看在眼里,嘴角微不可察的向上翘起了一点点,元绛很明显是看上了厚生司判官的差事。

这个位置需要的资序、阶级都不高,相对而言权位却很重,博取功劳的机会并不比判厚生司事的主官要差到哪里,是个升官的捷径。自然比判厚生司事更受欢迎。可赵顼却不动声色的就将条件划了下来,想必知道元绛打算推荐的人选并不在中书五房之中。

‘不愧是做了十二年的天子了。’韩冈想着。赵顼有了十二年的皇帝经验,已经是对朝堂上的明争暗斗有了精巧的控制力,对臣子的小算盘,许多时候也能看得一清二楚。

吕惠卿本也想说话,提议一个他十分看好的年轻人,没想到天子的一句话,就给挡了回去。想了想,中书五房中的各房检正公事,有两个是他的人,但只有一个位置,权衡了利弊,很干脆的放弃了自己的机会。

等到元绛和吕惠卿将视线投过来,王珪才说话道:“权检正中书礼房蔡京,学识优良,才具过人,当能适任。”

韩冈听到这个名字,心中立刻风起云涌,几十年后的奸相,现在终于走上舞台了?不知道他什么时候转得京官,竟然已经进入了中书五房中。按不知是谁人的说法,大奸大恶之辈,必然是有大智大勇。熙宁三年的那一科进士,蔡京算是爬得很快了。

“蔡京?”赵顼回忆了一下,倒也没费什么力气就想起来了,前些天才见过面,“他的字不错。”

呈到赵顼面前的奏章贴黄以及公文上,经常能看见蔡京的手笔,赵顼也知道中枢之中有个善书法的。至于蔡京本人,尽管仅仅见过两面,但赵顼对那位英俊不凡的青年官员还是很有些印象。

他问着王珪和两名参知政事:“蔡京此人如何?”

元绛眉头微皱,道:“才学或如王相公所说,可他仅仅是京官,而且是前日刚刚转官。若依军器监例,判官也当由朝官担任,或资深京官。”

赵顼打断道:“事有从权,既然才具不差,当然是个好人选。四位卿家,你们的意下如何。觉得呢?”

天子都这么说了,吕惠卿和元绛哪里还能有意见。韩冈也没有意见,只是他是做为旁观者,觉得赵顼似乎是越来越独断独行了。再看看王珪,三旨相公和独断独行的天子是相辅相成的,一个巴掌可拍不响。

定下了厚生司的人事安排,其内部的事务,韩冈不插手,也不便插手。又听赵顼和三位宰执讨论过厚生司衙门的位置和内部官员人数定额,将大部分事给敲定,韩冈便从崇政殿中告退出来。

接触到殿外清冷的空气,韩冈总算是放松了下来。也算是解决了一桩心事,过两天厚生司的衙门就能开始运作了,多半还会跟蔡京打几次交道。

回到驿站的后花园中属于自己的小院,里面没有什么人气。方兴李诫多半是访友去了,但同样进宫,为太皇太后等人,展示如何种痘的李德新,比韩冈还要早一点回来。

见到韩冈,李德新连忙行礼问候。

“算是见过了太皇太后了?”韩冈坐下来,笑着问道。

李德新点点头:“小人进宫后,太皇太后、皇太后、皇后都来了,看小人给一名小宫女种痘。”

“哦?”韩冈拖长的声调,证明他对这个话题似乎有些兴趣,“太皇太后她们怎么说?”

李德兴道:“只是夸奖龙图的时候为多,小人则沾了龙图的光。”

也就是得到了赏赐了,太皇太后赏赐给人的份量从来都不少,而且太皇太后给了赏赐,其他如高太后、向皇后,都会应个景。

“还说什么时候开始给六皇子种痘了?”韩冈继续问道。

“没有。”李德新摇着头。今天他去宫中,仅是展示种痘的技巧,并没有提及给不给六皇子均国公种痘,他正色起来:“不过太皇太后也问了襄州种痘的情况,尤其是夭折了几名小儿都详详细细的问了一遍。”

韩冈也变得郑重起来,追问:“你怎么说的。”

“小人按龙图的吩咐,一切照实说。种痘之后,死了两名小儿的事都说了。”

种牛痘算是针对天花的特效疗法,但并不代表种了牛痘之后,就会百分百安全。吃饭能噎死,喝水能呛死,这门技术肯定也会有意外。总有一个两个运气不好的,或许是对牛痘过敏产生的并发症,或许干脆是无关的病症,正好撞上了。

在李德新上京前,他领导的队伍,已经在襄州给近万名小儿种过痘了,有两名意外身亡,比例已经很低了。

“没说其他的了?比如说种痘一定能成功?”

李德新摇摇头:“不敢多言。”

没哪个医生会大包大揽,说自己对治好患者的能有百分百的把握。肯定是事先说些没把握的话,出了事也好开脱。若是成功了,也一样能分功。李德新是吃过大亏的人,什么话该说,什么话不当说,有着鲜血的教训。韩冈也正是欣赏李德新的这一点,才让他负责种痘之事。

天花是重症,得了天花的小儿少说有一半撑不过来,不过牛痘出现后,没得的至少能通牛痘不再受天花所苦。但种痘多多少少还有那么一点风险,宫中便不敢拿六皇子来冒险。

韩冈倒不是太在乎。

按韩冈的说法,牛痘本就比人痘要安全,先在是种痘之后,从人身上取牛痘痘浆出来做疫苗,这样的轮回已经反复多次,若是还死人——虽然比例极小——总会让人在种痘之时胆战心惊。牛痘都如此,人痘的危险性就可见一斑。

这一事实,让天子或后妃在责怪自己之前,可以好好想一想,更是韩冈的一个借口。

不过韩冈也不会靠这点小事推卸责任。将宫中之事放在了一边,韩冈笑道:“窦舜卿如今正住在京城,说不定还担心你记着旧年恩怨……”

李德新摇了摇头,“过去的事就算了……这是龙图当初教训小人的话,小人一直谨记在心。”

