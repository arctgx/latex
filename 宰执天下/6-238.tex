\section{第31章 风火披拂覆坟典(七)}

四天又十二小时。

韩冈从放在角落的摆钟上收回自己的目光。

到现在为止,铁场那边还没有坏消息传来,这就意味着正在进行试验的那一台蒸汽机,距离自己定下的第一目标,已经越来越近了。

自两天前,这一台进过改进后的新式蒸汽机突破了旧有的长时运作记录之后,韩冈就开始关注开封铁场那边的实验。

不过他也没想到,这一次的实验竟然会这么顺利,一下子就把旧日的记录甩得那么远。

韩冈还没有想到,自己竟然还会有着这种迫不及待的心情。

空无他人的公厅中,他自嘲的笑了起来。

毕竟他已经等待太久太久了。

为了实用化的蒸汽机,悬赏仅仅是一个方面,朝廷每年对蒸汽机这个项目的拨款,远远多过区区官职和赏金。

朝廷,或者说韩冈,付出了那么大的代价,需要的不仅仅是一个瓦特式的蒸汽机,而是一个成功的研发体系,还有机械研发上的成功经验。因为在一个成功的蒸汽机之后,还有齿轮箱,还有更进一步的改进,还有更多亟待解决的问题,

铁场那边的一个个研究小组,无一不是久经考验的能工巧匠,都有着自己独到的一面,谁成功都不奇怪。就算现在有一个小组先人一步,也不代表其他小组的研发水平是一无可取。

韩冈可没说过,蒸汽机只能有一种形态。新造出来的蒸汽机,也不可能不加改进。有了蒸汽机,还需要有传动装置,将动力传输到各式各样的生产机器上。

更多的问题,需要更多的研究者去解决问题。

“相公。”宗泽快步进屋。

“怎么?!是铁场那边有什么消息了。”

韩冈立刻问道,竟然带了点紧张。

“不是。”宗泽摇头,“是安阳殷墟的事。”

想了一下,韩冈问道:“……是上次相州说的盗掘不止、伪造成风的那一桩?”

“便是此事。”

得到了宗泽肯定的答复,韩冈便道:“这件事简单,让韩师朴把他家里的人管好就行了。”

表字师朴的韩忠彦是枢密院都承旨兼群牧使。但更重要的一个身冇份,是韩琦的长子。

韩琦四守乡郡,在他死后,堂弟、儿子都做过相州知州,安阳知县,近二十年来,更是一直都是韩琦家的门客出身。

如果说曲阜是衍圣公家的地盘,那相州也可以说是韩家的地盘。安阳那边的大事小事,无一不是跟韩家有着牵扯不清的关系,什么事少不了韩家。

殷墟肇事当然是韩冈点的火,没有他,只有几百年后,才有人能亲眼见识甲骨文的存在。但眼下安阳那边乌烟瘴气,就不是韩冈的错了。

依靠发掘以甲骨文为首的殷墟遗物,安阳的许多人家都发了大财。

这些年殷墟散佚的甲骨无数,被伪造出来的甲骨冇更多,相州那边已经形成了产业链,包括青铜器皿在内,每年问世的商代器物,至少有八成是假货。相州越来越多的人投入到古董伪造这个很有前途的职业中来,甚至秦代的青花、周朝的汝窑、还有商代的钢刀铁剑,因为总有傻瓜受骗,也都纷纷出现在市面上。

但巨大的利益也必然带来持续不断的纷争。大大小小的黑社会团体,自然就运应而生。

据韩冈所知,在安阳,每天都有人死于非命,只不过韩家势大,将这些乱子都给压了下去。

半年前安阳重修城防,在城壕边挖出了三十七人的尸骨,而且都是还没有腐烂的新鲜尸体。

如果是病死的还有话说,但相州、安阳县两方共同验看,所有人都不是病死,而是被杀。所以这件案子第一时间就给报上了河北提刑使司,御史台和大理寺都派了人去安阳查案。

可是死者的身冇份最后也没查清楚,只知道不是当地人。就这么一拖半年,相州和安阳的官员为此大受牵累,最轻的都是罚俸,上下两位亲民官都换了新任。

所以就有当地的官员上表奏明,自辩说这不是他们的错,而是殷墟出土之后,巨大的利益败坏了当地民风,字里行间不仅将罪责归咎于安阳韩家,还把韩冈也牵扯了进去。

韩冈可不觉得这是自己的问题,明明是安阳韩家治家不严的错。

匆匆将宗泽带来的文件看了一遍,韩冈便道,“汝霖你这两天再去跟韩师朴说一下,如果他那边不能解决现在的问题,政事堂这边会派一个好一点的知县去安阳!”

“宗泽知道了。”

宗泽点头,他知道,韩冈现在很不耐烦,不想再迁就韩家。

“汝霖。”宗泽正准备出去,韩冈又叫住了他,“韩师朴那边会怎么想?”

“……韩群牧必怒。”宗泽直言不讳的说道。

就像韩冈不认为这是自己的问题一样,韩忠彦也不觉得这是自家人的错。明明是韩冈为了道统之争,把乡里弄得乌烟瘴气。

而且韩冈也的确从殷墟中捞取了巨大的好处。

不仅仅是当年王安石兴冲冲拿出来想要一统异论的《字说》,被韩冈用甲骨文给当头砸了回去。

直到如今,气学一脉都备受其利。

到了如今,除了天干地支、一二三四,日月山河等简单易明的单字,绝大多数文字依然是一团迷雾,各家各派,对同一个字都有属于自己的解释。既然谁都拿着甲骨文为自己的理论张目,那么也意味着谁都不可能取得对甲骨文的诠释权。

就算安阳一带的土地已经千疮百孔,盗掘的风气甚至蔓延到了陕西,可这也让儒家各门无法形成合力,来攻讦韩冈的气学。

韩冈并不在意韩忠彦的愤怒,“看在织机的面子上,他不会多说什么的。”

自从雍秦商会将水力织机和缫丝机的技术公诸于世之后,各地都出现了大量的工厂。相州也产丝绢,韩忠彦可是背地里入了股。

更何况,韩冈让宗泽先去找韩忠彦,依然是让他推荐一个韩家门人来做这个安阳知县。

除了韩冈的话不怎么客气之外,安阳韩家的利益还是得到了最大限度的保留。

“只可惜了殷墟,不知毁损了多少。”

“是啊。”韩冈叹着气,没有半点诚意。

殷墟甲骨的确很重要,对中冇国的历史有着无可估量的意义,但在社会发展这个终极目标上,韩冈并不介意将其当成牺牲品。

“对了,汝霖。”在宗泽再准备告辞离开的时候,韩冈突然又道,“你出去后,让人再去铁场看看,有什么消息尽快报过来。”

“知道了。”宗泽点头应诺,想想又问,“相公很在意那边的成绩?”

“当然,一直都在盼着呢。”

韩冈丝毫不遮掩自己的迫不及待。

这可是开启工业革冇命最重要的一步,用什么样的褒誉,都不会让他觉得评价太高。

“但这也只是第一步,甚至不能用在铁路上。”

“能先走出第一步,已经是难能可贵了。哪家的孩子一出生就长大成人的?可汉家之兴,由此而始。”

韩冈无意求全责备。

先有了实用化的蒸汽机,下一步才会去考虑如何更好的利用这一动力,以及蒸汽机本身的改进设计。

除此之外,最重要的还有冇规模化的制造。

仅仅是为了进行合格的工业化生产,韩冈甚至连番下文,将度量衡标准化精确化。

尽管没有公制的度量衡,依照旧式的尺寸、重量的标准,也一样能够造出合用的机器来。如果是要测量零件的尺寸,游标卡尺暂时是足够使用了。

英国人使用英尺英寸,没有影响工业革冇命的爆发。暂时使用现有的度量衡,自然也不会太过影响工业化的进程。

在蒸汽机的发明过程中,韩冈甚至还考虑过发电机、电动机。在《自然》中,他也撰写过论文,阐释了电的定义,从闪电,到冬夜里脱衣服时闪烁的电火花。

怎么产生电力?是切割磁场发电,还是先利用电池。

能够铺设海底电缆的工业能力,这个时代还不具备,漆包线和硅钢片组成的电磁铁,当然更不可能。

不过不要发电机,只要有有效的电池,和无线电发报的能力,韩冈甚至可以直接挥军去攻打辽国。

可惜这还仅仅是梦想,而且是遥不可及的梦想。

但有了蒸汽动力,修建跨河大桥将会简单许多。修筑大桥时,不可缺少的钢铁零件,将会更加简单的大批制造出来。

洛水、淮水,最终目标就是黄河——或许几十年内,修筑黄河大桥的可能性微乎其微,不过比跨越长江冇的可能性还多一点。

不管怎么说,一切美好的未来,蒸汽机都是第一步——开创未来的第一步。

不过这也只有韩冈这么看,韩冈面前的宗泽却并不觉得蒸汽机的地位有韩冈说得那般重要。尽管他觉得的确很重要,可也没有重要到事关天下兴亡的地步。

“汝霖,你可知何为革冇命?”韩冈突然问道。

宗泽心颤了一下,不动声色的回答:“所谓革冇命,天命鼎革也。自汤武革冇命始,至太祖定鼎为止,改朝换代,即为革冇命。”

“非也。”韩冈摇头,“此乃一家一姓的鼎革,非是天命之变。”

“那依相公之说,何为天命?”

“是天下谁属:华夏,还是夷狄。夷狄之有君,不如诸夏之亡。圣人之言即是其意。”

宗泽皱起眉,汤武革冇命可是出自易传,孔子亲笔。当然,张载曾经说十翼之中,只有彖象四篇是孔子亲笔,韩冈作为其弟子,将汤武革冇命所在的《彖》都赶出孔子文集也不足为奇。

当世大儒,想怎么说就怎么说。当今宰相,他说出来的话,不知有多少人会主动为他做论证。

韩冈看得出宗泽内心中的想法,以状元郎的才华,当然难以苟同自己恣意取用圣人之言的说法。

换做平时,韩冈根本就不会说出这番话,但他今天兴致高涨,一时间也难以自抑,

“真正的革冇命是什么?是华夷相替!三皇五帝,渺茫难寻,暂且不论。周以代商,分封诸国,东至海、西至漠,北至燕代、南至苍梧,八百年间,诸夷星散。幽王时,犬戎能破国都,至秦兴,犬戎何在?周兴夷亡,华夏气运鼎盛,天命自此归于华夏,此方可谓之革冇命!”

韩冈停了一下,见宗泽默然静听,继续道,

“秦后至今一千三百年,五胡之乱,华夏不绝如缕,但李唐之兴,犹能横扫胡虏。惟国朝开国,便始终未能如汉唐一般扫平北虏南蛮。北虏势压中冇国,以天子之尊甚至得与虏酋约为兄弟,华夏之衰可见一斑。”

“大宋文治远胜契丹,武功也力压四夷。”宗泽低声辩驳。

“那是最近十几年,仍不足以扭转乾坤。”韩冈就是这一变化最主要的推动者之一,他又足够的资格去否定宗泽,“大宋在变法,难道辽人没有?但蒸汽机一出,工业大兴。枪炮之前,武力高下,不足为论。火枪一造千万,小童持之亦能胜壮勇。契丹铁骑可千里离合,旧日为官军之困,但轨道一出,机车为用,步卒亦能一日千里。自此之后,人多势众者胜。”

“华夏再兴,蛮夷之亡无日矣!”
