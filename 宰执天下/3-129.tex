\section{第34章 雨泽何日及(四)}

郑侠上流民图,惹得天子震怒,韩冈入对,而王安石留殿不出。

山雨欲来,狂风将作。此等很有可能改变政局的重要消息,不用半个时辰就在皇城内传开了。现在多少双眼睛在望着延和殿,等着天子最新的判决。

早一步知会了韩冈的王雱和吕惠卿已经回转政事堂,守在中书检正的公厅里等消息,吕嘉问、曾孝宽等新党核心都得到了通报,如同火燎了尾巴的兔子一般往政事堂这边急急忙忙的赶过来。

几人一会面,吕嘉问和曾孝宽在王雱口中证实了传言,原本还带着一丝万一的希冀,现在都化了惶惶。

韩冈在白马县中的一番用心事实俱在,而京城流民现在也得到了安置,郑侠的攻击其实并无依据,也就是流民图麻烦。但许多时候,政争的胜负与否并不是看事实的,而是看需要——天子的需要,朝廷的需要,天下万民的需要。

如今大旱遍及天下诸路,持续时间说七个月可以,说连着旱了两年也没问题。如今民情汹汹,需要一个出气口,很难说天子不会趁这个机会,将王安石踢出来当替罪羊。

罪名就是现成的,权奸当国,蒙蔽圣君,钳塞悠悠众口,使下情不得达上,只是纲纪紊乱,天下大灾。幸而有小臣郑侠拼了性命,绘下了流民图,将流民们的惨状呈到御案上。否则,还不知天子会被权奸欺瞒多久……

多好的借口!多好的理由!

要不是担心着这一点,方才在阁门处见韩冈的时候,王雱和吕惠卿何必急得要吐血。

远的不论,庆历新政是怎么败的?

不是范仲淹、富弼改革官制,被士论大肆攻击,而是他们最大的支持者宰相杜衍,他的女婿苏舜钦出了问题。苏舜钦在崇文馆中为官,卖了馆中的废旧字纸,而后拿着钱招妓宴客,饮酒作诗。虽然卖官中旧纸是惯例,但从未成文。这一下就给范仲淹的政敌吕夷简抓到了把柄,与会的青年才子全都被逐出了朝堂,连带着杜衍亦得罪,使得范仲淹主持的新政一下被断了根基,也不得不出外。一桩不算很大的小事,让声势浩大的庆历新政转眼间灰飞烟灭,‘一网打尽’的成语也由此而来。

但凡政争,几乎都是从小事开始,或是由一个小臣出面,先挑起战火,然后一波接着一波的弹劾、抨击,最后将对手连根拔掉。而眼下的情况,就很明显是这一条路数。市易务是开头,又利用了现在旱灾,经过几个月的酝酿,尽管中间新党的反击解决了一批出头的粮商,但眼下久旱不雨的局面让王安石再也压不住阵脚,很可能就因为一个监门官的弹劾,让天子彻底抛弃新党。

吕嘉问此时急得都快要哭出来了,他为了投效王安石,可是叛出了家门。当年曾拿着叔祖吕公弼的奏章草稿来给王安石看,被骂作家贼。本想着藉此飞黄腾达,可如今怕是要落到远州安置的结果。王安石若倒台,他这个市易务提举必然首当其冲,根本不可能逃过去。

让京城行商闻风丧胆的市易司提举,这时在厅内厅外的前后转着。前前后后不知转了多少圈,再一次踏出厅门的时候,眼前忽然出现了一片紫色,一个修长笔直的身影站定在身前。他抬起头来,竟然是参知政事冯京!

冯京沉着脸,狠狠盯了吕嘉问一眼。吕嘉问脑中还是糊涂,先是下意识的退到一边,然后才反应过来要向冯京行礼。

而冯京则踏前一步,向着厅中瞥了一眼,一句话都没说,怒哼了一声就从门前穿过去,径直走了。

只是厅内厅外的几个人都知道,冯京现在恐怕肚子里笑开了花。好端端的参政,不再他自个儿的厅中坐着,跑到中书检正的公厅来过路做什么?他是特意来看风色的!

盯着冯京的背影,吕嘉问恨得牙痒痒。王雱、曾孝宽也是冷着脸。

众人之中,只有吕惠卿心气最为平和,自始至终没有表现出半点惶急不安来,“望之,不要心急。有相公和韩玉昆在,必不致有大变。”

吕嘉问摇着头,就是韩冈在才让人急啊!

从关系上说,除了王雱、王旁两兄弟以外,韩冈是最亲近王安石的一人。可韩冈在新党中,却又是对新法最为疏离的一位,将他算作新党,其实都很勉强。不论从出身来历,还是从背景学派,他都跟王安石没有直接关系。

对于新党,韩冈的态度一直若即若离,有时帮忙,有时捣乱,虽然他的能力、地位、才智,都为人所认同,但就算是天子,也不会将其视为王安石一脉。

说句难听话,今日之事,韩冈他也根本不需要站在王安石这一边一起死,他只要将身上的冤屈洗脱就够了。以天子对韩冈的看重,罪名压不到他头上。

吕嘉问怎会相信韩冈会站在新党这一边?

……………………

延和殿上,旁听了韩冈的奏对,王安石惊讶不已。

不论是辩称流民众多是新法行之未久的缘故,还是向天子解释为何五年新政,百姓仍多流离,都可以看得出来,韩冈是彻底站到了新党这一边,全力支持起新法来。

而赵顼将韩冈的一番话仔细想过,叹道:“然世间有贫富,三代之法已难行于世,难道就只能看着一场灾异之后,百姓流离失所?……不知韩卿可有甚良策?”

韩冈当然没有。后世都没办法解决的事,他哪有招数。总不能说什么均贫富,王小波说的话,韩冈哪能在赵顼面前提,劫富济贫更不能当做手段。但天子的问题不能不回答:“扶危济困,常平是也。”

赵顼摇了摇头:“常平仓只能救急,不能常保百姓安居乐业。”

“天之道,损有余而补不足。”韩冈拖了老子来做帮手,“朝廷之税赋,纵不能多取之于富民,而用之于贫者,也当均之如一。”

“方田均税?”

尽管因为市易法在京城闹得沸反盈天,使得来自于开封城外针对新法的反对声显得相形见绌,但同在熙宁五年开始推行的方田均税法,同样受到极大的阻力。

乡中隐田,以富户为多,要清查田地,士大夫们当然一力反对。同时重新划定田地等级,使之税赋均平的工作,则是富户担任的保甲之长来主导,使得富民可以从中取奸,也因此给了反对者们足够的借口。

而韩冈现在却支持方田均税法,他点头:“不仅如此。免役法,便民贷,无不是秉持此意——施政以公,使百姓安稳。”

韩冈已经摆明车马的站在新党这一方。既然他已经接受了府界提点一职,就不可避免的就会成为旧党们的攻击目标。对此已经有了心理准备的韩冈,当然不能再做个逍遥派。

但站队也要讲个时机,去年娶王安石女儿时他不站队,因为那是新党势力大兴的时候,去了也只会被目为趋炎附势,而眼下正是新党危局之时,现在旗帜鲜明的站出来,可比前两年好处更多——锦上添花,哪及得上雪中送炭。

得到韩冈的回答,赵顼不再发问,再问就是惯常听到的空话了,“京畿流民之事可就要靠韩卿了。”

韩冈躬身一礼:“此乃陛下所以用臣之缘由。”

“多劳卿家。”赵顼点了点头,忽而又叹道:“现在就盼着天降甘霖了。”

虽然说了这么多,但终究还是仅仅是对流民的应对,并没有触及到核心的问题。

如今的旱灾如何解决?

想着几个月来滴雨未下的河北和京畿,赵顼还是难以释怀。这场天灾是不是因人祸而起?要不然郑侠为什么敢拿性命做赌注?

王安石欲言又止,瞥了女婿一眼,没有开口。而韩冈犹豫了一下,眼神重新坚定。

政坛这趟浑水,既然踏进来了,就别想着身子还能干干净净。漩涡卷过,可不管你是正人君子,还是卑劣小人。既然郑侠已经确定是敌人,还对自己下了手,韩冈就不会因为对方的道德品质而留手半分。

“说起雨水,陛下诚心动天,这几日京中层云渐多,或许不日将有雨至。”韩冈说着。可惜这个时代没有湿度计,否则可以藉此来推断一下降雨的概率。但最近两天空气变得湿润起来的情况还是很明显的,今天早晨他出门前,更是特别留意了一番,“昨日晨起,臣于院中树上有见露水凝集。而今晨臣在驿馆之中,亦有见之……”

郑侠的一番赌咒发誓,说十日不雨乞斩于宣德门外,韩冈则是轻轻巧巧的摆出了事实,他不会将话说绝,也没有说谎,更没有出言攻击郑侠,但足以引导赵顼去往他希望的方向去想。

赵顼就顺着韩冈的话头想过去。所谓‘山云蒸,柱础润’,看到树上、石上都有了露水,怎么想都是快要下雨的征兆。而韩冈能看到露水,想必守在城门处的郑侠应该也能看到。既然他敢在奏章中说十日不雨愿受刑于宣德门外,必然有所依仗,多半也是因为看到与韩冈一样的地方。

已现之兆,不禀明君上,反而用来在君前一博,赵顼对郑侠的感觉顿时大坏。可一想到说不定很快就要下雨,比什么祥瑞都要让他高兴,连着点头:“韩卿说得有理,明日朕也要留意一下。”

殿门忽而打开,方才出去的小黄门捧着一个卷轴进来,赵顼笑道:“好了,就让朕看看韩卿你的一番心血。”

