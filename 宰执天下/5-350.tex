\section{第33章 枕惯蹄声梦不惊(七)}

“什么叫做谋国之臣?!这就是啊!” 

向皇后在崇政殿上,正拿着洛阳传来的消息,将触了霉头的李清臣骂了个狗血淋头。 

因为都是旧党,且与司马光交好的缘故,一直以来向皇后并不喜欢文彦博和富弼。但当她这两天听说了两位老相公此时正在洛阳兴高采烈的办着牡丹花会的时候,对他们的看法一下就转为了正面。 

文彦博、富弼私心虽重,却也知道轻重。但有些人却宁可看着国家生变,也要呈上一番意气。这一干人等死不足惜,坏了前线的大局后,就会得意的站出来宣扬自己的先见之明:‘看,我早就说了吧!’若是前方胜了,他们也照样有能耐一进谗言。 

真宗朝的王钦若不正是这样的人?纵然向皇后不想冒犯真宗,但王钦若的人品,就是在她的丈夫的口中,也是一个彻头彻尾的小人、奸臣。 

战前一个劲的添乱,要真宗弃国南逃,等到寇准好不容易才挣下了一个维系了七八十年太平时光的澶渊之盟,可等到战后,王钦若一句孤注一掷,就让真宗就此将寇准贬斥出朝。其后王钦若之子无后,不得不过继,被世人说是现世报。 

河东鏖兵,此时京畿作为后方,最重要的就是稳定。一旦局势动荡,前线好不容易才稳定下来的局面就有崩溃瓦解的可能。 

为了安定人心,韩绛都派了家人去看球赛。其他人也都尽量表现得一如常日,就是要维系京师的稳定。 

“为什么正经事不见这般勤快?!息兵止戈,重修旧好,这是太常礼院是该说的?!李清臣,你是怎么管你衙中的人?!” 

李清臣低着头:“这是臣的疏忽。” 

“疏忽?”向皇后的声调一下就提高了八度,“在京百司,就数太常礼院要求和的奏议最多。你倒好,就是疏忽两个字!怎么其他人不疏忽?!” 

皇后在殿上大发雷霆,宰辅们也是相顾无言,看着判太常寺的李清臣被训得面红耳赤。当着众宰辅的面被训斥,李清臣除了自请出外,没有别的路可以走了。 

“辽军兵围太谷城。章楶军行迟缓,恐救之不及?”向皇后随手拿出一封御案上的奏章,冷哼着丢到一边,“笑话!韩资政都在奏章中说了多次,兵无常势,水无常形。疾时当疾,缓时须缓。如今太谷兵势,急则易为辽贼所乘,正是要徐如林啊!” 

这是今日清晨才送抵京中,由随军走马送来的密报。正好印证了太谷被困的紧急军情。 

作为从宫中或是班直、禁军挑选出来的耳目,比起领军的将帅或地方的官员,他们的奏报总是更加受到天子信任。但皇后倒好,直接就给丢了。 

没有人比皇后殿下更加坚定!在这一次辽军入寇中,向皇后的表现竟比躺在床上的那一位还要有气概。 

在过去,官军出战,只要战事稍有不利,皇帝便会茶饭不思,甚至几日几夜的合不上眼。但向皇后却一直坚信韩冈能挽回河东的败局,根本看不到有半点脆弱的样子,而且打起主和派的手段极重,在半月之中,已经有二十七名地位高低不同的官员因为主张求和,或是劝皇后巡幸金陵、鄂州甚至蜀中,被贬出京城的。 

皇后在这件事上完全不遵循应有的规则。正常情况下,天子一开始只会用留中等比较平和的手段来表露自己的倾向。除非事情变得严重起来,否则绝不会走极端。 

但皇后一看到请和的奏章,却连留中都不干,直接下诏将这些人贬官出外,甚至还把一个缴还词头,不肯为她草诏的知制诰给送去了荆南任知县了。而对于所有明里暗里指责两府导致了如今乱局的奏章,也一概驳了回去! 

纵然有越来越多的人,认为河东的局势堪忧。纵然辽军南下的消息已经传来。可向皇后还是照样对韩冈抱着信心。 

放下奏章,向皇后隔着帘幕瞪着下面的李清臣,“吾前日也说过,要一切如常。六哥该上学就去上学,吾该去亲蚕就去亲蚕。河东、河北,从将帅到士卒都在拼命,可你们倒好,尽在添乱。是想做刘康义吗?!” 

‘不是刘康义,是刘义康。彭城王刘义康。’张璪肚子里咕哝着,却不敢出声,唯恐皇后转移目标。 

看起来皇后是知道檀道济的,知道究竟是谁让那一位被谗言冤杀的刘宋名将,在临刑前喊出了‘毁汝万里长城’的怨愤之言。不过那当也是另有人跟皇后提起的,囫囵吞枣的记下了个故事、人名——还记错了。 

但皇后并不管那么多,她冷眼看着宰辅们:“国事危难,前线从将帅到士卒无不用命。谁敢在这时候主张议和,即是资敌!” 

…………………… 

走出了阴暗的地窖,司马光就不由得眯起了眼睛。 

春日午后的阳光很是和煦,但泛白的天光落在司马光的眼中,却还是不由得一阵头晕目眩。

“君实?!” 

司马光站定了脚,冲一脸担心的老仆摇了摇头,示意自己没事。 

“不要让富德先久等了。”说着便往前院去。 

与富绍庭在庭中互相致礼,司马光便将韩国公富弼的儿子请入厅中坐下。 

待下人奉上茶汤,司马光便寒暄道:“韩公日来可好?” 

“劳宫师挂心,家严身体尚算康健。” 

司马光似乎不喜欢有人在他面前提到太子太师这个头衔。富绍庭话出口后,看到微皱起的眉头才反应过来。 

司马光眉头皱了一下便放了下来,又道:“前日韩公生辰,光未能登门道贺,还望恕罪。” 

“宫师哪里的话,送来的贺寿诗,以及那两部书稿,家严看了很是欢喜。”富绍庭仍用着之前让司马光心中不喜的称呼,若临时改了称呼,反而就会显得过于刻意了,“尤其是《稽古录》的书稿,家严是赞不绝口:言简意赅,可备讲筵。” 

司马光点点头,带着点苦涩的笑道:“那些是旧年的书稿,最近抽空整理了一下,能得韩公一言,也算是不枉一番辛苦。” 

富绍庭端起茶盏,垂下眼帘,掩去脸上略显尴尬的表情。 

自从在京中落败归乡,又钻进地洞里修书的司马光连着多月也不出门。现在看看,比之前瘦了不少,干枯得像根劈柴。世人见他如此,本以为是准备寄情于修书,谁想到还是打算战斗到底。 

一部《稽古录》是对《资治通鉴》的补充。《资治通鉴》是从周威烈王二十三年为开篇。而《稽古录》则是从伏羲说到周威烈王二十三年,取名自《尚书》开头的‘曰若稽古’一句。富弼对这本书的确很看得上眼。 

但司马光的另一部得到的评价就不一样了。名为《潜虚》,完完全全是跟气学打擂台的一本书。气学说太虚即气,而司马光则说‘万物皆祖于虚,生于气’,气自虚空中生来。其针锋相对之意极重。富弼对这一本书的评价很低,直接就批评司马光到现在都没抓住根子。 

气学在韩冈手中已变成了以实为本、以实为证的学问,以可以眼观的事实来证明气学要义的正确。就算司马光的《潜虚》这部书,看起来是想将易学的义理、象数两派合二为一,有着很大的气魄,也的确似乎走出了一条新路,但如果不能以实相攻,而仅止于空对空,最后的结果不过是落进故纸堆给人忘掉。 

富绍庭在司马光面前当然就不能这么说,但他只称赞《稽古录》,而不提《潜虚》,司马光也明白了富弼的看法。 

司马光暗自轻叹,等富绍庭放下茶盏,他又说道:“至于德先今日的来意,光已知晓。此为国事,光岂敢辞?!请上覆韩公,司马光知道了。花会之时,司马光必至。” 

“宫师若能出面,洛阳人心可安。”富绍庭点头。 

他的父亲年纪与文彦博相当,却远比不上那一位太师精神。刚刚过了生日,给闹腾得很不舒服,寿宴后连着多日抱恙卧床。但一见河东危倾,洛阳也随之陷入了混乱,便强撑起病躯联络文彦博,一起来安定人心。能做到这一步,也算是对得起朝廷给的那些荣宠恩遇了。 

“韩公和潞公乃是国之重鼎,值此北虏入寇,天下板荡,非二公不得安定人心。司马光世受国恩,得韩公相召,自当一附骥尾。” 

富绍庭更多了一份喜色,扬眉正想说些什么,却看到了司马光家最得信用的老仆来到了小厅门外。 

“君实。潞国公府上的六衙内来了,正在门外。” 

司马光和富绍庭同时站了起来,文及甫此来不用多说,当然是跟富绍庭一个打算,都是来请司马光的。 

不过当两人迎了文及甫进来,还没重新落座,就有一名司马光家的仆役气喘吁吁的跑进来,面色惶急,似乎有急事禀报,只是看见了厅中的两位客人就犹豫了起来。 

“出了什么事?”司马光大大方方的说着。他一向自诩光明磊落,凡事无不可对人言。 

那仆役喘了几口气,就叫了起来:“学士,大齤事不好了。韩枢密被困太谷,河东的辽贼南下了!” 

司马光倏然起身,脸色变得更加苍白。 

两句话分开来都没错,但顺序在消息的传递中颠倒了个儿,意义便完全不同。听起来,河东和韩冈都已经是危在旦夕了。 

回顾脸色同样大变的富绍庭和文及甫,司马光沉声道:“德先,文翰,好久没去天王院花园子了,不知可否与光同往?” 

两人互望了一眼,一齐点头:“……自当同去!”

