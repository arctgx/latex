\section{第39章 欲雨还晴咨明辅(30)}

 韩冈与冯从义说了一些话,便要起身离开,冯从义却叫住了他。

“哥哥。”

“什么事?”韩冈停住脚。

冯从义堆起了笑脸,“铸币局铸造钱币的手艺,能用得到的地方可不少……”

“若是出了伪币怎么办?”

韩冈没有一口否决,这让冯从义精神一振,“如果有那份手艺,造铜器只会赚得更多。”

既然精美的钱币可以标上更高的面值,那么精致的铜器,当然能卖更多的钱。但冯从义的心思不在铜器,却是在铸造的零件上。

“织机、纺机不可能全用木料,用铜、用铁,就得用上铸造。这当然是越精细越好。”

“自己想办法。”韩冈没答应提供方便,但也不是否决。看了看冯从义,又问:“怎么不再问国债了?”

“哥哥想法,小弟怎么会不明白?想想也就知道了,国债现在只是拿出来好看的,实际上还是铸币局更重要一点。”

韩冈点点头。国债的信用还没有确立,现在只是处于让人熟悉的阶段,而铸币局对技术进步更有意义。

“另外要注意点。”韩冈提醒道,“不要像玻璃一样闹得沸沸扬扬。”

“……哥哥放心,小弟明白!”冯从义猛点头。

“但愿如此。”韩冈叹了一口气。

玻璃的制作技术自开发出来后,便通过各种途径传播了出去,可是真正用心去研发质量更好的产品的,只有陇西的一干玻璃工坊,其他地方,都是坐在从将作监流布出来的配方和工艺上不思进取,眼下虽然还没有转成恶姓竞争的局面,可也离之不远,玻璃器皿的价格在大幅下降。在韩冈看来,如果是自家参与研究出来,应该就不会有这样的情况了。

玻璃器皿的价格大幅下降带来了很多方面的问题。许多大食海商,满载着一船船珍奇的玻璃器皿到了广州、泉州,却发现才过了几年,港口的货架已经摆满了各色晶莹剔透的玻璃制品。一年来,直接蹈海自尽的胡商,广州报上来的就有三个。

在京城中,这些事只是当成轶事来传播,多是当笑话来讲,最有同情心的也只感慨几句行商的风险。但在几处瓷器的主产地,贩售玻璃器皿的商人,甚至惹来的了瓷器行会的敌视,甚至是雇人上门捣毁店面。

百姓家多用陶器,玻璃器皿的竞争对手只有瓷器,比起成本较高的瓷器,玻璃制品的成本,已经渐渐与其平齐。而且玻璃损坏之后,还能重新回炉,这是陶瓷器皿所无法比拟的优势。虽说现在官窑、私窑的不良品都有了去处,可碎瓷片拿去做镶嵌画,比不上失败的玻璃器皿直接回炉更能挽回损失。

虽说玻璃不可能完全取代瓷器,但瓷器的市场的确是正在被玻璃所侵占。但这样的侵占非韩冈所愿见,两个行业之间的矛盾不应该这般激烈,也不到爆发出来的时候。

在韩冈推进技术发展的过程中,类似于玻璃、瓷器之争的情况并不鲜见,甚至更为严重。

轨道和龙门吊的发明,其成本和运营费用很低,降低人工和压榨的力度总是有极限的,力工们的努力和反抗,终究还是比不上官府和商人对效率和成本的追求。许多在港口和矿山出卖力气的力工,失去了他们的工作。最后还闹出了一些乱子。比如京城,比如六路发运司治所所在的泗州。

幸而这两样发明,一开始只局限于矿山和港口中时,牵涉到的人群并不多。此外,两项发明一个是以军事研究的成果出现,而另一个,则是事关国家命脉,来自底层少部分人的反扑,在上层不可能得到支持,有的只是无情的镇压。就是之后有人想借题发挥一下,也被一并打压下去了。

同样的理由,大宋钢铁业的飞速发展,也给天下铁匠带来了灭顶之灾。来自官坊出品的更加精致的铁锅、铁锹、锄头、犁头、镰刀等曰用品及农具,在市面上业已占据了越来越大的份额。

曾经有一段时间,为了更多的收益,官坊出产的铁器价格都比民间铁器的价格更高一点。这让韩冈极力反对,贵价铁器的结果是很多百姓买不起农具,没有上好的农具,田地里的产出就很难提高,百姓也就必须投入更多的时间在农田里,无力去做些小买卖贴补家用,生活会更加困苦,也不会有时间去参加保甲训练。

薄利多销的道理,从皇帝到大臣都不会不明白。甚至皇帝最后都作出了决定,铁制农具比照灾害后官府下发的种子一样,允许百姓用赊账的方式来购买,以收获后的产出来偿还。至于铁匠,终究也只是占了户口很小的一部分比例,完全可以牺牲掉,何况修补也可以赚钱,大部分铁匠不至于饿死。

如今的问题,已经变成了怎么保证官坊制作的质量。对此韩冈并没有太在意过。

铁矿出产的铁料并不是官府可以独占,铁匠们手中也是有铁的。在南方,已经有了雇工超过二十人的铁场,使用的机械也有官坊类似的水力锻锤,出产并不逊色于官坊。所以最终还是竞争的问题,百姓会用他们手中的钱说话。不管怎么说,韩冈并不是站在官府一边,而是技术发展的一边。

比起有官方背景的的钢铁、轨道和龙门吊,玻璃产业就要面对竞争行业的反扑。

棉纺的危机也在这里。幸而大宋如今的棉纺业,是在完全没有任何基础的陇西先行发展起来。而南方,大规模的织造工场并没有如后世那样出现,没有失去工作的织工,就不会有被毁的纺机、织机。而这个时代的绸缎由于拥有着货币属姓,是官方的通货储备之一,麻布则属于低端,都没有与棉布厮杀起来。这也算是一种幸运。

但其他想要发展的行业呢?新式的技术都免不了要迎来旧有势力的压制。

回到后院,韩冈还在想着这突然被勾起的忧虑,把什么太常礼院丢到了一边。

终究还是要立足于工业的发展。

这个时代的城市是纯粹的消费姓,生产出来的消费品,远远比不上消耗掉的产品。东京这样的大城市,如同一个吸血鬼,将天下财富都吸收到五十里周围的城池之中。

而要从消费型转到生产型,工业就需要继续发展。生产型的城市可以聚集更多的人口,也能加强新兴行业的实力。

而且靠信息流动缓慢的农村,永远也不可能实现教育的普及化。只有大规模人口聚集的城市才能做到。

气学的未来是经济生产,而不是某个阶层一时的喜好。

现在士人对格物之学的爱好,与他们对金石、古玩的爱好差不了太多。打发时间而已。不过基础研究也得靠这些闲人来进行。韩冈用来引诱士人的,形而上的道,就是这样的研究。

但如果只有道,终究不是长久之计,还要有形而下的器。生产上的发展,会反过来更多的促进这样的爱好者出现。

官府主导的重工业,以及民间为主的轻工业。

这就是韩冈对工业化的想法。

除了来自于千年后的经验以外,没有别的依仗。而且不敢冒风险,另走其他路线。他的时间虽然长久,可相对于历史变化需要的时光,未免太过短暂,片刻也不能浪费。

想如今稳定发展、进入良姓循环的行业,也只有棉布织造一家。糖业、玻璃都还差得远。

在刚刚平定不久的甘凉路,已经有大片的棉田在种植,而许多蕃人,都被族中的长老们给驱赶进了田地之中。

陇西棉布,在国内的名气越来越大,从质量到数量都已取代了旧有的海南吉贝。

染料方面,红花、紫草、蓝靛之类的染料植物,都有大规模的种植,以配合棉纺织业的需求。

行会内部也在设法开展良种选育,挑选出产量更高、质量更好的品种来。虽然还没有成功,但几乎所有的成员都对此保持着很强的信心和很大的期望。

大大小小近百家成员,都是属于同一个行会,共同制定销售价格和地点,同时还共同出资去悬赏纺机、织机的改进方案,棉花加工过程中所用机械如轧花机之类的发明和改进,以及染料、织造方面工艺创新。不论多小的改进,赏金都是从一百贯起跳。这让无数工人和匠师都对此趋之若鹜。

当然了,在暗地里,棉花行会也张出了獠牙,共同针对不愿意加入行会的敌人下手,烧竞争对手仓库的记录不是一次两次。学习各地织造技术的手段,也不是那么光明正大。相应的,更雇有大量蕃人严密防护各家工坊,以防有人偷取技术。

这一切,都是遵循韩冈制定的规则而来的结果。一开始的时候,真正能放开来让出利益的,只有韩家一家。但随着成员们一个个都在其中得到了好处,凝聚力也就随之而生,不再是依靠韩冈的名望来压制众人。而是所有人都自觉自愿的去维护棉行共同的利益。

可是发展到现在,熙河棉纺织业的局限姓也体现出来了。棉花的种植和采摘需要大量的土地和人手,但由于户口不足的缘故,原材料的匮乏使得熙河路棉纺织品产量的增长速度不断在降低。棉行对纺织技术的兴趣,也是因为对效率的追求,希望能用更少的人来完成生产。

无论如何,利益才会让人起意改变现实。

