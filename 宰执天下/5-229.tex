\section{第26章 当潮立马夜弯弓(中)}

韩冈是面对着天子开的口,但所有人都知道,他到底针对的是谁。

所谓至亲,当然不是太后,也不是皇后,更不会是唯一的皇子。两座药王祠,一在北,一在西,离京城皆有千里之遥。两位亲王一人分一座,一去一回差不多也要一个月,至少在天子内禅之前,是别指望能赶回来。若是中间再有个什么波折,说不定要在药王祠中留到天子龙驭宾天的那一天。

赵颢的脸抽搐了一下,眼皮直跳。韩冈完全是撕破脸皮了,竟然想将他和老三一并赶出京城。

他瞄了一眼韩冈。这灌园小儿脸色平静得仿佛只是提了一句奇闻异事,就像寻常聊天时不经意间提起的一般。

临到大齤事有静气,这样的人才比旁边流汗的王相公要强得多。赵颢也不由暗暗心折。但韩冈的想法绝不可能那么简单,绝不可能仅仅是为了内禅的顺利。

赵颢又瞄了瞄他的母亲,只见她一双眉毛高高吊起,脸色铁青,正死死的瞪着韩冈。赵颢打了个寒颤,以他对母亲的了解,心头的怒气当已是到了极点。

上一次亲眼看到母亲这般怒气冲天的时候,还是她得知京城中正流行有关自己的唱本。再前一次,是太皇太后劝说母亲不要将父皇管得太死,让他能去接近其他嫔妃。

寝宫中的气氛就像张开的弓弦,绷得越来越紧。越来越多的内侍和宫女都尽量缩到墙根边,努力使自己不至于成为被迁怒的目标。

而看到太后气得发昏的模样,贵为王珪也不由得缩了缩脖子,脖颈子上的寒毛全都竖着。那可是发起火来,连身为姨母和姑姑的曹太皇都压不住的主。

王珪方才还想既然前面比韩冈迟了一步才一同请立太子,那么现在就该将功补过,将事情做得圆满了。可当他看到高太后怒视着韩冈的双眼里,都染上一层血丝,他发现自己的一张嘴怎么也张不开。

张璪盯着眼前的稿纸,尽力想将心神给收拢住。可寝殿内犹如山雨欲来,如芒在背。但手上的笔越来越慢,最后已是字不成句,不得不暗暗一叹,干脆将起草诏书的笔给停了。前面是韩冈不肯干,这一回是自己的思路给乱了。

他很佩服韩冈的狠决。出手之后,就不再给自己任何回转的余地。毫不留情的凌逼太后和雍王,根本不在意自家也一并断了后路。

可是,韩冈办了一件蠢事,难以挽回的大蠢事!

没人会认为韩冈说的是真话,河北和陕西的两座药王祠灵不灵应也不是人们所关心的,他的目的是一目了然。

以韩冈的身份当然可以拿着药王祠编个有灵应的故事,然后将他想打发的人打发出去。但他不该在太后面前说出来。即便是可以说出来,也不该用方才的那种语气。

以太后之尊,臣子可以动之以情,可以晓之以理,但不能就这么公然的丢下一句极为明显的谎话,近乎于强逼的将她的两个儿子赶出京城。难道不要照顾太后的面子?

而且更重要的一点,天子要保儿子平安登基,平安成人,难道太后就不想要保住儿子的性命?!

表面上看,韩冈不过只是想在内禅的过程不受干扰,能让延安郡王安安稳稳的即位。可事实上,雍王、嘉王如果都留在京城中,太后还能保住他们。可一旦出了京,从开封往河北、陕西的一路上,出点什么事都不会让人意外!

太后会想不到吗?看她现在的愤怒就知道了。

高太后等着韩冈半天,也不见他有半点悔意。那从容冷静的神态,不断的在挑动高太后的神经,终于让她是出离愤怒了。她没想到韩冈竟然敢有这等提议,竟然要将两个儿子都赶出京城。

“韩冈!”她猛地站起身,一把甩开想搀扶她的陈衍,上前两步,直指着看着就心头生厌的措大的鼻子:“你这外臣不思忠心报国,却离间天家兄弟骨肉,究竟是何居心?!”

“臣不敢。”韩冈只微微垂下眼,身子却纹丝不动。并不加以解释,更不承认自己有错。

年近五旬的太后更是恼火,尖声道:“你还有什么不敢的!?”

“还请太后息怒。”薛向想上来打圆场,“晋时庾衮事兄,疫盛不避。如今……”

“别说那么多场面话!”高太后一声断喝,惊得薛向倒退了一步,“韩冈打得什么主意,你们还想瞒着老身?”她回头又指着赵顼,颤声说着:“看你用的好臣子!!”

太后雷霆之怒,床边的嫔妃们一个个噤若寒蝉,就是向皇后也在积威之下,呐呐不敢开口。但她们都知道事情的关键该着落在谁身上。

韩冈既然说了药王祠灵验,聪明的亲王这时候就该知道怎么做了。

至少要自请出外,决不能当做没听到。不论韩冈之言真伪与否,该装的样子就不能少。

可赵颢垂眼看着身前的地面,不过片刻时间,他就已经汗流浃背。几次欲开口,却完全发不出声来。

赵颢知道自己在情理上,应该立刻自请出京,去韩冈说的什么耀州、祁州。只要他这么做了,立刻就能扭转他在世人心目中的坏名声。日后接手帝位,朝堂上的反对声也能少许多。

为了皇位,仅仅是跑跑腿而已,这样的交换是大赚特赚。就是刳臂割股、尝粪吮痈,也不是不能做的。反正他的算计是着落在侄儿区区五岁的年纪上,而并不在乎现在皇兄内禅于谁。

韩冈如今撕破脸皮,反倒是一件好事,能让即将成为太皇太后的娘亲,彻底站在自己这边。

可谁能保证自己就能顺顺利利抵达千里之外的,又有谁能保证自己事后能顺顺利利的返回京城?路上风风雨雨,说不定就染上疾疫。说不定就失足落水。说不定就水土不服。要死人,太容易了。就算没这些事,安安稳稳的到了地头。当皇兄顺利内禅,至多当其病死之后,就能被召回来。可万一皇兄在临死前下一份密旨呢?一杯鸩酒就足够了。

有太祖太宗的亲弟秦悼王在前,有太祖的两个儿子燕懿王和秦康惠王在前,有太宗长子楚王元佐在前,赵颢决然不敢破釜成舟。只要翻一翻史书,就能知道,皇帝的宝座分明是血色的,决不是光明正大的明黄。

一旦出京,性命就不是自己的了。

赵颢怎么敢开口要求出京?他盼望着母亲的愤怒,能让皇兄退缩。

赵顼的确退缩了。在高太后发了一大通火之后,所有人都只能等待天子的裁决,而赵顼眨起眼,传出来的却是:

娘。

息。

怒。

“息怒?大哥儿,你说怎么办?”高太后质问道。

向皇后在被褥下紧紧攥着赵顼手腕的手,无法遏制的颤抖起来。

官家都已经妥协了!已经退让了!新法准备废了,旧党也要重新启用了!都已经做到了这一步,只要求两位皇弟出外一阵,为他们的皇兄祈福,竟然还不愿意!难道赵仲针就不是她十月怀胎生下来的亲儿子,只有赵仲糺【注1】才是吗?!

她是多么希望她的夫婿能稍稍强硬一点,能让太后答应下来,但赵顼让她失望了。

下平十一尤——留。

向皇后眼前顿时一黑,只觉得天都塌了。

天子既然当着太后和宰相执政的面做了决定,几乎就不可能再改变。尤其是赵顼只能用眨眼来传话,想反口,不知要费多少精力。

‘你这是要将我们母子逼死不成?!’向皇后紧紧咬着下唇,等着赵顼,却不敢将话宣之于口。

高太后终于是重新坐了下来,胸口上下起伏的喘着气,时不时的瞪一下韩冈,脸色还是难看,显是余怒未消。

在母亲的身边端茶递水,劝着她稍息心头之怒,赵颢一边也在偷眼观察着韩冈。

明明图谋已经落了空,但赵颢在韩冈的脸上,找不到胆怯,找不到慌张,找不到一星半点投注落空的恐慌。依然是宁宁定定的站着。如果从气度和城府上来看,他远比王珪更有资格成为宰相。可惜他是敌人,是必须要铲除的对象。

尽管应该可以放宽心了,赵颢也不断的跟自己说韩冈的图谋根本绕不过他的母亲,但雍王殿下却还是神经质的想要从韩冈的脸上找到失败服输的痕迹。越是找不到,心就越是没底,完全没有感到一丝一毫胜利的喜悦。

赵颢依然沦陷在不安中,赵顼在稍事休息之后,又开始让王珪传话。

招。

宰。

执。

招宰执?

今夜留守宫掖,宿直宫城的两名宰执——王珪和薛向可都在这里。

王珪小心的询问:“陛下,可是要将两府里所有的宰执都召入宫?”

赵顼眨了两下眼睛。

但所有的人都没敢动弹,甚至连传话的王珪都犹豫了。

毕竟半夜招宰执入宫,这就等于是在说天子即将驾崩,甚至是已经驾崩。

这不是边关烽烟连绵的时候,不会有哪位宰执为了安定京城人心,硬是拖到白天才入宫。以赵顼的病情之重,他们一听到消息便会立刻动身。

赵顼突然发病的今夜,不知有多少双眼睛在黑暗中盯着皇城城门,或是宰执们的府邸。只要宫里面派去几位宰执府邸的内侍一亮相,不等天亮,皇帝大行的流言便会传遍京城。

“官家的病才好了这么一点,就累了半夜。是不是先歇一歇,等明天白天,群臣入宫后再说?”向皇后也开口劝阻。有半夜的时间作为缓冲,至少在太后和赵颢离开后,她还能有机会劝一劝她的夫君,看看是不是能够将之前的决定给改回来。

可赵顼却不肯等待:

速。

去。

注1:赵顼原名仲针,赵颢原名仲糺。
