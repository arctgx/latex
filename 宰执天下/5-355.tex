\section{第33章 枕惯蹄声梦不惊(12)}

政事堂中盘绕的疑云,大概就跟方才福宁殿中一样多。

基本上无人怀疑韩冈奏表中的真实姓,唯一的问题便是奏报发来的形式。

韩冈在奏报中说得极为保守,甚至连捷报的‘捷’字都没提,但战果无论从何种角度都是大捷。战果也好,河东战局也好,朝堂也好,皆是一场无可置疑的胜利。

但韩冈为什么不这么做?

要真是露布飞捷,他们只会比皇后更早一步,而不是等到通进银台司将奏表送进福宁殿后方才得人走报。而了解到了具体内容,更是等到了皇后转发过来的现在。

“这是大喜之事啊!”

张璪哈哈两声笑,对于因为河东危局而备受朝野攻击的两府来说,这一封奏报等于是久旱逢甘霖,让他们可以为之解脱。但只有两名检正公事奉承讨好的赔笑。韩绛、曾布神色皆是严肃无比,蔡确也拧着眉头。

只是最大的问题,谁都想不通。

曾布嘴角抽搐了一下,他也并不怀疑韩冈是否在奏报中说谎,又因为害怕事后被拆穿,而不敢公诸于世。

韩冈真要是作假,绝不会这么半调子。

都是敢作敢为的人,越是没底气,就越是要做出理直气壮、胸有成竹的样子来。这个道理,曾布明白,章惇明白,更是胆大包天的韩冈当然更明白。

而且有多少辽兵会为了没好处的战争而拼命?辽军已经失去了锐气。纵然明知会有更好的方略,但萧十三也没办法强迫他手下的兵将做到拼死攻城。此番评论,韩冈早早的就在奏章上说明,同时也得到了宰辅们的认同。因为契丹贵胄的秉姓,是为世人所共知。

当初富弼是怎么说服辽不要兴兵南犯而接受增加岁币的条件的?正是因为富弼指出了辽宋通好,维持岁币换和平的澶渊之盟,则‘人主专其利,而臣下无获’;一旦两国交兵,则‘利归臣下,而人主任其祸’。

即使是辽兴宗,也没有否认富弼的说辞中,指称辽军南下只是为了劫掠财货这一段。因为这是事实。也正是靠着这一事实,富弼才达成了维系澶渊之盟的初衷。让‘契丹主大悟,首肯者久之’。以劫掠为目标的军队,一旦赃物满橐,在有退路的情况下,又岂会放下一切拼死一战?

甚至韩冈在奏表上说,将会领军继续北上,追在辽军的身后,打算夺回代州。可见他根本不怕辽人会反咬一口。

正如澶渊之盟时,名将杨延昭杨六郎所指出的:‘敌顿澶州,去境北千里许,人马罢乏,虽众易败。凡所剽掠,悉在马上。原饬诸军扼要路掩杀,其兵歼,则幽、易数州可袭取也’——辽军深入河北,人困马乏,兵马虽众亦可轻取,且其剽掠所得财物,皆在自家马上驮着,毫无斗志可言。只要能让各军扼守要路,将之掩杀歼灭,幽州、易州失土将唾手可得。

这一观点跟韩冈如出一辙,同时河东的地理远比河北对步兵为主的宋军更有利。韩冈敢于放手追击撤退的辽军,正是看清了这一点。

对于韩冈在河东的决断,在京师的一众宰辅就算有异议也没有办法奈何得了他,只能听之任之。唯一可以由他们来裁定的,仅仅是知太原府,同时兼任河东经略和兵马都总管都的王.克臣。

由于太原最终被守住,不论这到底是谁的功劳,王.克臣都能分润一份,韩冈很自然地在奏章中为王.克臣说了几句好话。

“纵有过,稳守太原亦已相赎。”韩绛啧啧嘴,却似乎有些不满,“韩玉昆好大方!”

“毕竟王子难守住了太原城。”曾布说道。

“太原府丢了多少县城?被毁了多少村镇?百姓又死伤多少?何况他还是河东经略,雁门、代州、石岭接连陷落,他岂能无咎?”张璪难得的表现出自己的倾向,言辞也比往常激烈许多。

蔡确看了看张璪,又不动声色的瞧了瞧韩绛,什么时候张璪跟韩绛站到了一条线上?

通判留光宇是韩绛的侄女婿,又是韩冈的同年,如果韩冈能赶走王.克臣,太原知府的位置即便由由韩冈本人代掌,而实际上的政务也是交托给留光宇这位通判一大部分。如此一来,到了战后,他的侄女婿一下跳上七八步都有可能。

可惜韩冈没有这个想法,看起来只想维持河东稳定,将得罪人的差事都留给两府。而蔡确也没打算去为了韩绛的侄婿去得罪王.克臣这位国戚。不过,如果韩绛肯欠下些人情的话……那就两说了。

“原本的晋阳城还好说,现如今的太原也不是那么好守的。”蔡确委婉的表达了自己反对的意见,“岂能以深罪责之。”

‘又在卖菜了。’曾布心中冷嘲,不仅是蔡确的那点小心思,还有蔡确的话:‘反过来还差不多。’

赵光义毁晋阳,‘尽焚其庐舍,民老幼趋城门不及,焚死者甚众’,城高四丈,城周一万五千余步——差不多四十二里——的旧太原城,被彻底毁坏。而新修的太原城城周则不过十里出头。

旧太原跨汾河而建,分东、西、中三城,所谓‘都城左汾右晋,潜丘在中’,西城中更有仓城、新城和大明城——即为战国遗存的古晋阳——三个子城。

这样的城池的确坚固,可相对于极度匮乏的兵力,过于庞大的城池反而是个累赘,根本守不过来,总不能去依靠刚刚征发来的民兵?若太原犹在原址,旧城依然保留,辽军说不定就敢打太原城的主意了。那样的情况下,可就有的韩冈头疼。

“持正相公多虑了。王子难是国戚,身在八议之列,依例当减罪,岂会深责?”张璪说道。

关闭<广告>

“八议是定罪上才用到的。现在是给王.克臣定罪呢?还是在商议他是否适合继续留任太原?”曾布突然开口发问,“如果王.克臣手中能多上两万兵马,结果自会不同。所以石岭关破,他的责任本不算太大。雁门、代州的主要责任也不在他身上。”他明确的反对,“何况韩玉昆也希望他能留任。试问万一撤换了王.克臣而导致战事不利,那么制置使司会不会归咎于东府?”

蔡确敛起眼神,转瞬又对韩绛道:“子宣言之有理。子华兄,既然如此,不如交予圣裁吧。”

韩绛沉默了一下,点了点头:“也罢,就交予皇后圣裁。”

“邃明?”蔡确再问张璪。

张璪随即点头:“如此也好。”

让皇后选择如何处置王.克臣的决议,就这么定了下来。

不算太大的事情——比如如何处置王.克臣——如果两府中有分歧的话,就会交予皇后来决定。甚至会故意在一些小事上表现出分歧的样子来。只是大事上的决定,两府却会尽量做到同一个立场。

这是逐渐为两府所默认的规则。

异论相搅的帝王心术,宰辅们哪个不熟悉?他们可都不想看到皇后把这一套招数练到当今天子那般炉火纯青的水平上。

不过韩绛、张璪脸色都有些阴沉。并不是因为他们的打算被曾布阻止,而是因为曾布让他们清醒了一点,不要去挑战皇后对韩冈的信任。

皇后学不会异论相搅的确是好事,但变成对韩冈言听计从的情况,可就让几位宰执心中不舒服了。

蔡确轻叹了一声,问几位同僚,“韩玉昆发来的奏表到底该如何处置?皇后殿下肯定要问,好歹得有个章程出来。”

绕了一个大圈子,终究还是逃不掉这个难题。

韩绛一番思前想后,最后道:“就照例发在邸报上吧。”

他回望蔡确,蔡确微一沉吟,点头道:“也好。”

“子宣、邃明,你们看呢?”他又问着曾布和张璪的意见。

两位参知政事点了点头,投下了赞成票。

来自边关的军情,如果是有利于官军的,通常都会如此安排。如果是大获全胜,飞捷入京的话,更是要入告太庙,或是在文德殿上君臣共贺。既然韩冈本人没发捷报,那就当成是普通的有点战果的军情,在邸报上向下通报便是。

先等等看吧。

政事堂中不止一人这么想着。

……………………

“辽贼退得还真是干脆。”蔡京笑了笑,右手轻抬,示意来通报消息的耳目退了下去。

乌台中的御史们,可谓是整个京城中,对小道消息最敏感,同时也消息最灵通的官员。其中身为侍御史的蔡京得到河东制置使司的奏报内容,连同宰辅们的决定,也不过隔了一个时辰而已。

从通进银台司和东西两府中传出来的消息很是夸张:辽贼在太谷围城一夜,贼众多有伤亡。又因为援军随之北上,便闻风而退。又用计火焚城外南北二市,没于火海的贼众数以千计,在烈焰中尽为飞灰。

同属台院、又问蔡京知交的强渊明摇着头:“都不知那一份奏报到底掺了多少水,要打上个几折!”

“若是别人说来,也就一成两成可信。以韩三的姓子,对半折吧。”蔡京边想边说,“烧死烧伤以千计,多半是假,事后斩首则肯定为真。韩三领兵多年,倒还不至于犯这样的错。”

上报的战绩通常都是尽可能往天上吹,一方面是说着好听,另一方面也是助长声威。反正实际上的功绩,基本上还是以斩首来评判——守住城池是功劳,逐走辽军也是功劳,但功劳大小,端得还是看缴获——吹得再厉害也没用。斩首是不可能作假的,缴获的军器也做不了假。至于击败的敌军数目,十倍八倍都是可以吹的。

其实蔡京对奏报中掺了多少水并不关心,他在意的是两府对整件事的处理意见。

赢终究是赢,可终究还是因为揣摩不透韩冈的想法,这让两府不敢将战果以捷报的形式向外公布。要是两府真的敢做,还不如立刻大张旗鼓的去宣传,将韩冈架在火上来烤。

“不过我倒觉得那样反而会让人以为宰辅们沉不住气,韩冈都不当回事,他们却拿着当宝,一个对比就显得有失宰相的身份而不够稳重了。”强渊明说道。

“……或许都有吧?”

蔡京眉峰深锁,强渊明的话一下点醒了他。韩冈的随意之举甚至让其余宰辅都为之气短心怯,不敢有所异动。

到底是从什么时候起,韩冈的一举一动,连宰辅都要仔细去揣摩了?

等他挟大胜而归,那朝堂将会是什么样的局面?

……………………

“想法?”韩冈放下望远镜,回头对留光宇道,“没有!不过就算没有捷报,士卒的赏赐不会少,真正没落到好处的只是我而已。”

“那枢密你何须如此?!”

韩冈笑道:“等收复了代州,再向天子报捷不迟!”

留光宇很疑惑道:“枢密何必如此自清?”

韩冈笑了一笑:“我来河东所奉王命为何?”

“收复河东。”留光宇没有二话。

“那么差遣呢?是太谷知县吗?”

“……下官明白了。”留光宇向韩冈行了一礼,赶快去做他的事了。

目送了留光宇离开,韩冈则在营中冷笑着,重新举起千里镜,对准了远处的寨墙。

‘大家都别闲着。’
