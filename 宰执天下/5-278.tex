\section{第30章 随阳雁飞各西东(九)}

不出意料,号角声中,第一批被派上来的是党项人。 

当先出阵的党项军大约三千多人,手上的兵器是全的,带甲的士兵也占了一半。基本上都是明晃晃的板甲。也不知党项人平日里保养起这些甲胄,擦了多少羊尾油,亮得在城上的种朴都觉得刺眼。 

种朴咧着嘴,低低骂了一句,全都是高遵裕送去的好处。灵州城下一败,泾原、环庆两军加起来五六万人马,陆陆续续逃回来的有一多半,可还带着甲胄的就没几个了——逃跑的时候只会嫌盔甲重——全都送了人,还附带了神臂弓、斩马刀和不少能造霹雳炮的工匠。等到辽人占了兴灵,通过两国和议,俘虏换回来不少,但工匠一个都没有弄回来。 

盐州之战的时候,若不是有了那么一批兵甲,党项人也不会那么容易就攻破盐州城。且在盐州城下,他的父亲已经将高遵裕送出去的礼物拿回了大半,想不到还有这么多留在外面没有收回。 

眼下党项人都能有一半装备上甲胄,辽人自不会缺。有了甲胄护身,神臂弓的有效范围顿时减半。 

种朴阴着脸想着,不过立刻又叹起自己的鸿运来。幸好之前在城下守着辽人的一千步卒,因为辽人追自己追得太紧。为防误伤,不得不放近了才射。否则纵有事前这一番布置,也一样翻不了盘。种朴最后返身与辽人追兵厮杀时看得很清楚,他们身上分明穿了军器监造的胸甲。 

在出战的党项军的背后,还有第二批同样有三四千人的党项军队——这一批人中,装备甲胄的比例看起来就少了许多——再后面,又有着为数更多的辽军压阵。北、东、西三面足够万人之众。只看气势,就远远超过不知为何而战的党项人。之前在城下的损失,对他们来说不过是一点皮肉伤,伤的只是颜面。 

党项人没有骑马,而是步行。三千余人的攻城队伍以北面为主,从两里外的军营出发,扬起了一片烟尘。而在烟尘中,两艘飞船冉冉升上了天空。 

“都派飞船上来了?!” 

种朴的副将李清踏上了城头,眯着眼睛望着天空。 

从西夏国唯一的汉人大将,到一城城主的副手,落差不可谓不大。在对降人一向宽厚的大宋,这是个很少见的例外。不过这主要还是李清本人的身份作祟。 

如果李清像党项和吐蕃的豪酋们有自己的部族,那他至少能得到一个刺史的官称。可惜他是汉人,麾下都是汉军。降顺后,便立刻被解除了对军队的控制,然后便给安排到了不掌实权的位置上。不过在这过程中表现得极为恭顺,因为他本身的能力又得到了赵禼的看重,几经辗转才又被派到前线上来领兵。 

“飞船倒没什么,反正溥乐城就这么点大。” 

溥乐城并不大,城墙周长一千一百步,也就是三里出头。而且是标准的军城,没有什么闲杂人等,有的只是官兵们各自的家室。三千军汉城墙上一站,都能勾肩搭背起来,还不用担心背后有人坏事。 

自然,守城不可能尽把人往城墙上堆,五六百人就足够了,保险一点也不过八九百。种朴将两千七百多名步卒,按指挥分作六部,轮番交替上城。只要保证每一刻都有两个指挥在城头上就足够了。 

种朴说起话来有些闷闷的。他的脸上缠着绷带,从左颊斜斜的挂到右边的耳朵上,紧紧地缠了好几圈,说话都不方便。不过有两个大嗓门的亲兵正拿着个两头没底的薄铁桶在他旁边。帮他喊话,城上城下都能听见他的号令。 

抬起手,传令的亲兵便靠了过来,种朴吩咐道:“把我们的飞船也放出去。” 

传令兵拿起薄铁桶,跑到内侧的边缘,扶着女墙冲城墙根大吼着将种朴的号令传了下去。 

空气中烟味立刻就重了起来。 

存在城中的燃料不少,可现在是冬天,如果不是辽人放了飞船,需要维持城中士气,种朴也不想浪费燃料在飞船上。两丈多高的城墙加上两层高的敌楼用来监察敌情,高度已经绰绰有余。 

“那些云梯才是麻烦。”种朴指着越来越近的党项人,人群中那一架架三丈长的长梯十分显眼。 

李清点了点头,“能造这么长的云梯,就能造霹雳炮。也不知辽人的营地里藏了多少工匠。” 

“还有木料!” 

种朴自问他在这一年来没有少下功夫。为了修复溥乐城,周边能用得上的木料全都给砍伐一空。加上还有寻常使用的柴草,也要对周边的草木植被的大肆砍伐,溥乐城附近,根本见不到半点绿色。但党项人的队伍中还是有云梯存在。 

这让种朴很惊讶,“莫不会是从耀德城运来的吧?” 

长梯的数目并不算多,种朴粗粗一数只有十几二十具。要打造能攀上高约两丈半的城墙的长梯,必须要有一流的木工手艺和上等的木料。即便不说木料,单是这个等级的木匠,就跟能开两石弓的猛将一样,已是为数寥寥。就是大宋这边也不多见,城外辽人手上自然更少。 

在大宋军中,一般是干脆了当的直接打造云梯车,长长的梯子下面有了车体支撑,便不会因为长度过长而容易折断。只要有了图样,几个普通的木匠配合,轻轻松松就能做出来。比起单纯的三丈长梯,工艺上要简单得多。 

可就是这十几具长梯已经让种朴觉得怵目惊心了。这代表对方营中至少有一个大匠级的木工。也代表接下来的霹雳炮绝不会少。 

高遵裕到底送了多少东西给外人啊!比真宗皇帝还大方。 

冲在最前面的党项士兵已经到了两百步以内,如果不计八牛弩,再向前三十步就到了神臂弓的有效范围了。城上的数百将士虽然神色依然轻松,不过双手已经握紧了神臂弓的弩身。 

种朴冲亲兵招了招手,“放近了再射!”神臂弓的最大威力,还是在面对面、脸贴脸的时候。 

城外的党项人随着越发接近城墙,速度也开始提升,从徐步到疾步,此时已经开始奔跑了。就在这一批的后面两百步,人数还多上一点的第二批党项军,也开始接近。 

蚁附攻城。 

而且是节奏感把握得很不错的蚁附攻城。 

对于为辽人卖命,被强占了家园的党项人自然不是心甘情愿。不过看到溥乐城上稀稀拉拉的守兵,他们倒还是鼓起了一点勇气来。 

呐喊声冲天而起,三千多党项士兵冲向了城墙。他们的装备也看得越来越分明,除了抬着云梯的士兵,每一人都随身带着一个看起来很有分量的包裹。 

城头上的种朴和李清,只是顺带着瞥了一眼,就确定了党项人的打算。这是准备垒土成山,在进行云梯攻城的同时,也准备用人数上的优势硬吃溥乐城中的守军。而这些土包,也可以垫在云梯下。以防攀登时折断。 

只是这样的攻势,在最为擅长守御的宋军眼里,连笑话都算不上。 

最前面的党项士兵下到只有一层被冻结的底水的城壕中了。指挥使们回头看看敌楼,种朴摇摇头,城上便毫无动静。 

党项士兵开始向城墙脚下投掷携带的土包。种朴说了句‘再等等’,城上依然没有动静。 

当党项士兵拉开长弓,射击城头,以掩护云梯越过城壕,种朴还是不许官军反击,城上还是没有动静。 

直到已经超过一半的党项士兵越过城壕,云梯陆续架上城头,一群最为勇猛的党项战士开始攀爬云梯,种朴终于挥下了手。 

压抑已久的将校们一阵欢呼,只听得城上一通鼓响,檑木、拍杆、灰瓶、滚油随即倾泻而下,砸断了云梯,浇伤了人体,然后便是数百张神臂弓齐射。 

每一名神臂弓手的脚边,都有七八张已经张好的神臂弓,拿起弩弓,放上箭矢,一个呼吸间就能射上一箭。 

鼓响不过十声,城上就已经射下来数千支箭矢。在最短的时间,杀伤最多的敌人,这样的一击之下,便能让敌军完全失去战斗意志。 

不到十丈的距离下,神臂弓的威力充分发挥了出来,纵使军器监打造的板甲和头盔也抵御不了这么近距离的射击,仅能化解一些致命伤,让中箭的士兵只伤不死。但满地的惨叫声,同样能让人失去战斗意志。在这第一批攻城军近乎崩溃的时候,突然打开的城门给了他们最后一击。 

仅仅据守城墙不叫守城,以城墙为凭,不断出城反击,才是真正的稳守之道。养精蓄锐的千名战士手持战斧从城门中杀出,冲进城下混乱的敌群中,挥斧肆意砍杀,带起一片血浪。城上的神臂弓也顺势外延,用弓矢阻断敌军的退路。 

三千党项士兵冰消瓦解,完全失去了战斗能力,如同羔羊一般被宋军屠戮。高高在上的飞船将这一幕传给下方的辽军将领,在号角声的催促下,拖在后面的第二阵立刻加速赶来,而两队各五百多人的辽人骑兵也随之出动。 

敌楼上,种朴好整以暇的比了个手势,“霹雳炮,给我挡住后面的贼人。” 

城内的霹雳炮开火了,一包包石子飞上天空,划着抛物线,又重重的落到了地面上。包着石子的绳索在重力下根根崩断,从灵州川中搜集来的鹅卵石四散飞溅,一下阻断了第二波攻势。让他们躲避着飞石的同时,只能坐视城下的友军在宋人的重斧下哭喊哀号。 

李清略带冷淡的看着这一幕,并没有感染上敌楼中其他将校们的兴奋和狂热。 

这里根本不需要他。种朴只需要下命令,剩下的就交给参谋们。在他们的安排下,甚至连霹雳炮的配重都事先调试好了,只需要等待城上的号令。之前种朴敢于领军出外偷营,而不是让他这个副将领队,也是仗着参谋们对城中军队的控制力。 

或许种朴从来都没有信任过他这个降人。或许自己来到最前线是个错误的选择。 


