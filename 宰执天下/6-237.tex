\section{第31章 风火披拂覆坟典(六)}

“竟然说韩相公的儿子是冒充宰相衙内,多少日子都没听过这么好笑的笑话了。”

“不会吧,那个官儿肯定要倒大霉了。”

“岂止是倒大霉。是诬告反坐啊!”

“诈称官身,这是能大辟的罪名,轻的也得去西域住一辈子了。”

“太重了,又不是故意。”

“不是故意又怎么样?谁让他开罪了韩衙内?”

“韩相公家的二衙内好端端的包厢不坐,偏偏去坐小官儿的车厢。受小人之辱,也是自取。”

“韩相公治家严,韩家二衙内就算有个宰相爹,却也只是个京官罢了。京官做什么车?”

“当朝的两个相公治家都严。章相公的两个儿子中了进士,全都到外地做县尉了,没一个留京的。”

许嵩从议论的人群边走过,喉咙干干的有些发痒。

用力的干咳了几声,冲着地上吐了口痰出来,痰中带黑。

许嵩拿鞋底蹭了蹭地上的痰迹,在水泥铺砌的地面上拖出了一条深色的痕迹。

正在说话的人中,有一两个看了许嵩一眼,但立刻冷淡的将视线扭开,仿佛没看到他一样。

许嵩也同样都没多撇他们一眼,继续向前走。

全都是些闲人,上工的汽笛响了有半日了,他们还在这里拿着报纸端着茶盏聊天。

开封铁场的高炉昼夜不息,时时刻刻都有工人在工厂中忙碌着。负责管理的匠师也都是分日夜两班,一刻不歇,包括许嵩在内,几乎所有的军器监、将作监派驻于此的官员,都是忙得脚打脑后跟。只是并不包括坐在这间院落中,上午最忙的时候,能懒洋洋的坐在树荫下享受凉风的人们。

全都是通过不同门路进来的闲人,或多或少都有些背.景,或豪门远亲,或显贵门客,只是还不足以得到荫补,无法入流,无法任官,来此拿一份干俸。

铁场每年产铁两百万石,给朝廷带来收益数以百万,几十名闲人还是养得起的。只要他们不贪心的想要去插手进入铁场的实务,上面的那些大人物都不会计较这点多余的支出。

不过一旦忍不住想要从中弄到更多的好处的话,从王居卿,到两府中的相公们,那就全都变成了吃人的老虎。

上一个蠢货从铁场中弄了几千石铁出来,一下就被抓到了把柄,然后连流放都没有,直接就被太后下旨赐死,与他勾结的内部人员,被斩了七个,流放了九户,总计一百零三口。这还是娶了宗女的。换作是其他人,怕是连白绫都讨不到,只有铁场出来的精钢利斧相送。

所以现在着一干闲人一个个都学乖了,只管拿钱,不管做事。

铁场中做实务的官吏们,也生怕被误会是内外勾结,绝不敢与其有半点接触。两边是井水不犯河水,路上遇到了,就会跟许嵩现在一样,谁都当做没看到对方。

从铁场中央偏北一点的公厅出来,许嵩先上了马车。

开封铁场是从冶炼到制造的庞大机构,占地面积也巨大无比,纵横皆在三里以上,高炉在一端,而码头在另一端。转过一个方向,军器监的制造工坊也占据了四分之一的面积。工场中间甚至得用铁轨来运送材料。

而许嵩正要过去的试验场,也是在铁场的边缘。

马车走得不慢,渐渐的,耳边开始充斥了各种各样的噪音。

与军器监制造工坊中,那些车床、磨床、铣床发出的声音完全不同,更多的是低沉的轰鸣。

许嵩在试验场的靠后一点的位置上下了车,前面十几间小型的厂房,各自独立,甚至有围墙相隔。

几乎每一座厂房里面,都是一阵阵如同低咳的轰鸣。

那是蒸汽机运转的声音。

如果仅仅是字面上的蒸汽机,其实早已发明了,甚至已经投入了实际使用。

方才许嵩过来的地方,蒸汽机已经在轰轰的运转着。

早在半年前,在一个用青砖和水泥砌成的平台上,一具用钢铁铸造而成的怪兽,就开始将深井中的水不断抽取上来,一直提取到七八丈的高处。

许嵩只要回头,就能立刻看见一个顶端暗红的高塔。

不同于同样耸立的高炉,那是一座水塔。是以钢筋水泥修起了支架,然后再用红砖在支架顶端修了一个两丈径圆,一丈高的蓄水池。洁净的深井水,正是被蒸汽机送进这个顶端封起的蓄水池中,然后再利用高低差,让水流流进工场中每一个需要水的地方。

但这样的蒸汽机是远远不足以承担更重的作用的。

每天能够正常运行的时间不超过四个时辰,仅仅是因为只要半个时辰就能见水塔充满水,不需要持续到运作,这才让这种最简陋的蒸汽机有了用武之地。

不仅仅是在工场中,已经有四五处煤矿开始采用同类的蒸汽机,用来抽水。更有一批土地众多的大户来考察过,是不是可以用来灌溉农田,可惜没能推销的出去。

有了《自然》长年累月的进行普及,谁都知道,蒸汽机的作用绝不仅仅是用来给水井抽水,而一台合格的蒸汽机,绝不应该才做上一个、半个时辰,就开始要检修。

最短正常运转时间,是能带动列车以中速跑完三千里,也就是至少能够五天连续运转。达成这个目标之后,就是在持续运转的一个月之内,维修次数不能超过四次。最终目标,则是以日常检测、按月维修、年度大修的维护标准,能够运行五年、十年的机器——这样才符合钢铁的强硬,这样才可以将骡马远远地甩到后头。

只有达到这一标准的蒸汽机,才有了最广泛的使用价值。

但只要达成了第一步的目标,就会以此为原型,进行小规模的制造。在实际的使用中,进行改进,以期达到第二、第三步的目标。

蒸汽机驱动的重锤,能达到现在水力重锤十几倍的力道。

甚至按照韩相公的液体压强理论,有了蒸汽机驱动之后,可以造出上千石、上万石压力的水压机来,用来锻造各种零件。

能够抽水的蒸汽机,尽管经常出故障,也不需要太多的齿轮结构来传动,但已经可以拿来做一些基础实验。看到一块钢胚在重锤下一锤成型,变成一个合格的头盔,许嵩当时兴奋的连汗毛都竖起来了。

毫无疑问,也正如那位高高在上的相公所说,有了真正可以推广使用的蒸汽机之后,现在的工厂、乃至这个世界都会完全改变。

许嵩甚至都已经设想过,如何使用那种能把骨头都压成粉的水压机。先铸造出的一根铁柱,然后利用车床,在中心处钻出炮膛来。再用水压机处理炮管,可以将炮管压紧,减少炸膛的风险。比现在铁模铸炮法更好,也更简单。

为了达到这一目标,十三个小组,同时在进行试验。

你追我赶,就在许嵩眼前的这些厂房里。

没有哪个工匠能够独立完成蒸汽机的制造,仅仅是原材料,就不可能不经过控制了钢铁产销的官府。

只要那位匠师能够展示出合理的设计,并拿出一定水平的实物来,政事堂都会为其敞开钱袋,给人给地给钱。

但如果进展不利,就会劝说其与其他小组合并,若是发现滥竽充数,甚至会直接淘汰。

十三个小组就是这样不断组建、不断合并、不断淘汰而成。

而他们,经过了几年努力,也越来越接近最后的终点。

许嵩走进了其中一间厂房。

三五丈见方,一丈多高的厂房内,热浪滚滚。钢铁的零件堆得整整齐齐,煤堆,水桶也都在角落,正中央,只有一台机器正在不断怒吼。而高高矮矮七八人,有坐有站,还有用铲子不断向炉膛里填进煤炭,但每一个人都目不转睛的盯着这台机器,每一个人,都是满脸黑灰。

“多长时间了?”

许嵩连招呼都没有打,进来后直接就问道。

“一天……”其中一人看了看放置在一角的座钟,“带十一个小时三十八分钟。今天早上刚通过汽笛对了时间。”

他穿着官袍,但同样是满脸灰黑。

“夜里面没断?!”

许嵩提高起来的声调,让人知道他对这个数据不是无动于衷。

“没有。”

那人简洁的回道。

许嵩相信自己的副手,何况这边还有从军器监、将作监、盐铁司出来的官吏,监视着所有正在进行试验的研发小组,更何况,竞争对手们也都在看着,谁也收买不了这么多人。

“这已经是第三好的成绩了。”

许嵩压抑着自己的兴奋。

“我们可以做到最好!”

除了铲煤的工人之外,站在最前面的一人回头道。

除了个头偏矮,他与其他人没有任何区别。

气学讲究日渐日新,一次成功,只是修好了一级的台阶,对目标的追求是永无止境的。不要轻易满足。小富即安,这是大部分人的特点,但对于研究者来说,绝对不可如此。

韩冈的话是所有人的圭臬,许嵩也同样如此。

但这第一步同样是重要的。

再有四个,他们就能成为第一。再有三天半,他们就将获得成功!
