\section{第30章 回首云途路不遥(上)}

皇太妃的提议!

向太后的话,让一阵诡异的静默来到大殿中。

好几位重臣低头盯着自己手中的笏板,仿佛要看出花来。

韩冈、苏颂和章惇,都是早早就猜到太妃多半会不甘寂寞。

章惇扭动了一下笏板,斜斜的指了一下御阶之上,韩冈轻轻摇头,根本就没必要自己出马。

朱太妃明摆着就是有着不轨之心,而太后这么说,多半就是要借臣子的力量来压制那个不安分的朱太妃。

眼下的朝堂上决不会有人就这么选择支持朱太妃。

在场的虽都是既得利益者,但想要更进一步的绝不在少数,不过要让他们短时间就做出选择,可没有那么容易。要压重宝在皇帝和太妃身上,所要做出的取舍和决断,可不是这么一瞬间就能做出来的。

如果有个人站出来横拦一刀,这个决定就更难做了。

李承之正这么想着,他就看见新上任没几天的开封知府王居卿站了出来,“天子先天元气便弱,之前为人所诱,更是伐根伤本。如今保养还来不及,哪还有火上添油的道理?太妃所言大缪。”

好了,王居卿没给人思考的机会就站出来,将朱太妃的想法给砸了回去,短时间内,还敢为之做仗马之鸣的,恐怕一个都没有了。

几乎所有人的目光都是先汇聚在王居卿身上,然后又挪到了韩冈那里,

王居卿好大胆子!李承之都小吃一惊。

王居卿什么时候变成了现在这种敢出头的脾气,为韩冈做起了马前卒。站在韩冈的角度,来打压朱太妃和她想要帮助的小皇帝。

李承之转念一想,也随即迈步出班,“寻常人家娶妻大率在十四五,若是读书,二十前后也是平常。官家原本身体变弱,如今更是伤了根本,如果,岂不是在点着了的树木上再浇油,还能再烧多久?”

翰林院的不倒翁蒲宗孟跟着出班:“天子选后,调理阴阳,乃天下至重。当由太后与群臣共议,岂有太妃说话的余地!”

这是更鲜明的表态了,随着韩冈一系的纷纷出场,一些重臣暗藏的小心思一个接一个被摁了下来,想为将来讨好天子,现在就得受大罪,这有何苦?

“太妃所言的确不当。”

“且待天子成人后再议不迟。”

议政重臣们纷纷支持韩冈一系的意见。

但在蜂拥群起的嘈杂声中,只有宰辅们没有表态发言。

“苏相公,韩相公,章枢密,还有其他几位参政、枢密,你们怎么看?”向太后也发现了这一点。

曾孝宽道:“事关国政,太妃不当议论。”

苏颂想了想,道:“官家才因女色致病,太妃太心急了。”

章惇则道:“此事不妥。”

被太后点名的三位宰辅,只有韩冈还没发言。向太后试探的问道:“韩相公?”

韩冈之前岁没说话,其实他的态度早就有下面的自己人表明了。同为一系,各自所持的立场都是清楚明白的,尤其是韩冈这位首脑的。遇上这种事,如果韩冈没有先出面来定调,那么只可能是他打算继续维持过去的立场。既然韩冈立场确定,下面的人要做的就是帮他说出来,而不是让他自己打头阵。

现在太后一问再问,韩冈终于是站出来,“大婚与否,端要看天子御体是否安好,若一切安好,便可大婚。若是根基未固,贸然让天子大婚,事有万一,谁能担待得起?依臣之见,此事不能贸然决定,提前、推后皆有不便,还是再等等看为是。”

再等等看,也就是继续拖下去。

韩冈并没有一口就将时间给推后到二十多,也没有将之定在十四五或是十六七,更不会答应现在就给天子准备婚事的打算。

将时间确定下来是最蠢的做法,什么定不定,往后拖就是了。满朝文武,到底是什么人会去在意赵煦什么时候成婚?只有想要看到朝堂动荡的那一部分人,这样他们才有机会浑水摸鱼。

所以韩冈不论是怎么确定时间,都是把自己的手脚束缚起来的蠢事。只有把大婚时间与赵煦的身体状况联系起来,那就一点问题都没有了。

究竟是何年何月,还不是韩冈这个医道泰斗和他手下的一众医官说了算?就算是赵煦日后变得身强体健,能夜御十女,也照样是外亢内弱,本质尤虚,需要静养个十年八载。

“诸卿说得有理,就按照韩相公所说,等天子身体好了,再操办大婚之事不迟。”

向太后飞快的做出了决定。

小小的太妃,就算有一个做皇帝的亲生儿子,朝臣们照样可以不加理会。

确定了朝臣们不会添乱,向太后也能理直气壮的将那位太妃给打发掉了。

因为韩冈之后又说了,“至于太妃,臣不记得上先帝诏书上有太妃权同听政一条。”

朝中事,太妃无权与议,即是那是她亲身儿子的婚事。

……………………

“太妃得为天下着想是好事,但也要为官家多想想。”

“官家这一次大伤元气,不好生调养身子骨,却匆匆大婚,日后怎么千秋万岁?”

教训了朱太妃一通,向太后挥挥手,让她下去了。

朱太妃脸色铁青,从王中正身旁出门,摔得珠帘一阵劈啪作响。

王中正在后面摇头,太后的性子还是太软了,竟然容得太妃如此放肆。

尽管从先帝时起,向太后就与朱太妃不对付,可现在都没先帝撑腰了,太后更是得到了几乎所有重臣的拥戴,朱太妃竟然还敢时不时的冒犯一下太后,于今更是敢插手国家大事,这不能不说是给太后的性子惯坏的。

别说是换作权势犹如吕、武的章献皇后,就是曹、高二后,都是没哪位嫔妃敢在她们眼前炸毛的。

仁宗时,宫中兵变,慈圣曹后能指挥宫女、内侍拿着弓刀跟乱兵对阵,而高太皇,能顶着姑姑兼姨母的慈圣,能压着做皇帝的丈夫,这更是威风了。

向太后手中的权力绝不比垂帘听政过的刘、曹二后稍差,要是从大宋的国力上来看,更是远在其上。至少章献明肃和慈圣光献两位皇后,她们所说的话,不能让西域蕃人俯首帖耳,也不能让大理国君瑟瑟发抖。

从民间的声望上来看,向太后更是远超刘、曹二后,大宋国事昌盛,国计渐丰,在朝廷的三令五申下,各地的苛捐杂税也少了一点——尽管减少的比例不多,也足以让太后和宰相们得到天下百姓们的拥戴。

可太后就是过于善心了,多少该死的却不判其死,只用了一个流放打发了事,朱太妃就抓着向太后的这个性子,又觉得自己儿子已经坐上皇位,就是向太后也得顾忌向家的未来,所以才敢猖狂如此。

当年以章献刘后对仁宗生母章懿皇后李宸妃的嫉恨,还不是照样要用皇后之仪将她发送,将尸身浸在水银中,那时候,章懿皇后可还没被追认为皇后呢。而日后,仁宗在被人揭破了他并非章献所生,而是章懿皇后之子,并收到谗言说章懿皇后是被章献所害,也是开棺确认了章献对章懿皇后的厚遇,方才不再怀疑。如果章献把章懿皇后只当做普通嫔妃来发送,那么刘家的结果,也就不问可知了。

当年这一场宫闱秘闻,如今早传遍了天下,朱太妃肯定听了不知多少次了。仗着自己的肚皮生下了当今皇帝,朱太妃自然有恃无恐。

幸好朝臣们当头给了她一棒子,让她不要干预朝政,否则还会继续嚣张下去。

当年富弼只能对英宗说‘伊尹之事,臣能为之’,而韩冈可是当真杀过宰相,一手平定了宫廷之乱。章惇、苏颂,他们也都是参加过平乱的功臣。

若是当面遇到沉下脸的宰辅,恐怕朱太妃连囫囵话都说不全。

现在重臣们同声叱问,朱太妃总是有十个胆子,怕也是不敢再乱来了。

“王卿。”

太后的声音打断了王中正的胡思乱想。

“臣在。”

“这几日你就守在宫中。”

王中正暗暗笑了,太后也不糊涂,“臣遵旨。”

……………………

廷议之后。

天子因亲近女色而致重病的消息很快就传开了。

十二岁的小孩子竟然开荤开到昏倒,这终究是一桩吸引人的有趣话题。

韩冈在政事堂中翻着报纸,并没有看到相关的报道。谁也不会蠢到在报纸上公然泄露天家阴私。但这件事,已经通过酒店、茶肆的口耳相传,传遍了京城内外的每一个角落。

宗泽在外面通了名,然后走了进来。

出外担任了一趟体量学政等事,用了半年时间,到淮南东西两路绕了一圈。现在回来,已经是同中书五房检正公事,也就是中书门下的二号管家。

宗泽手上拿着一份墨迹尤新的公文向韩冈汇报,“福宁宫宫中出来的内侍、宫人总计五十八人,都已启程离京。得受天子宠信的三位宫人,也分别安排到了瑶华宫和洞真宫暂住,等确认了是否有喜脉才会决定行止。”

瑶华宫和洞真宫分别是当年的仁宗废后郭氏和尚、杨二美人出宫后所居,说起来绝不吉利,不过三名宫人也差不多跟尚、杨二美人的情况一样,如果没有怀上龙种,这辈子就要念经度日了。

随着赵煦身边的近侍纷纷被拿问,一些细节也呈现了出来。

根据后来审问的详情,赵煦比他名义上的曾祖父还要高杆一点。

韩冈听说之后,除了摇头叹气,还真做不出其他反应了。

本钱不足,勉强去做大生意本就是错,要是再想着三个篮子分别装鸡蛋,鸡飞蛋打是没得跑的。
