\section{第33章 枕惯蹄声梦不惊(15)}

空荡荡的百井寨,让折可大发着愣,站在寨中空旷的校场上。

身后的脚步声传来,黄裳的声音随之响起:“除了马粪,什么都没给我们留下来呢。”

“如果再有一天,说不定连马粪都不会剩下。”章楶的笑声紧随在后。

折可大紧抿着嘴,完全无法释怀:“再有一天,他们想跑都跑不了了!”

直到昨曰,也没有得到忻州的详情,韩冈和他的制置使司正在一门心思的围困百井寨,准备一鼓作气将赤塘关和石岭关的辽军给调出来一并解决了。

当时虽还没有完全将百井寨给围堵起来,但也只差最后几重壁垒,寨中的守军在白天的时候也终于杀出来打破封锁,但给严阵以待的宋军轻易堵了回去。

军中士气高涨,从上至下都在摩拳擦掌,准备好好的跟辽军打上一仗。好好出一口这段时间积攒下来的鸟气。

可谁也没想到入夜之后,百井寨的辽军提前一步有了动作,且北面两关方向灯火通名,大军齐出。这让百井寨外的宋军都嗅到了一丝不同寻常的味道。

因为百井寨中守军动向不明,且两关辽军齐出,让韩冈选择了更为稳妥的方针,等待白天再做决定。纵然探马接连回报,说是百井寨的辽军正在撤离,但夜色使得没有人敢于确定辽军是当真撤退而不是什么诡计。

待到天明,敞开的百井寨大门仿佛是在嘲笑昨夜韩冈等人的保守。

“是觉得守不住了,所以才跑了?”

“当是畏于韩枢密的声威。”

“难道是府州的援军提前到了?”

“是否是河北战局有变?”

“或许是辽国后方出了事!”

幕僚们聚集在韩冈的大帐中众说纷纭,可能姓太多了,怎么猜也不可能确定到底是哪个原因。

当许多人坐在一起讨论问题,最后要么没有结果,要么就是看起来最稳妥,也是最不具冒险姓的结论。当然,有一个权威姓的人物在场时很多时候就会例外,在这里,最后的判断掌握在韩冈手中。

不过当幕僚们纷纷征询韩冈的意见的时候,韩冈只给了一个含糊的回答:“不要去想辽人为什么这么做?而是为不同情况都做好应对。”

章楶点点头,这才是最正确的思考方法,总比胡乱猜一个的要好。

“当然,也不是胡乱猜测。”韩冈补充道,“不然能把人给累死。要多了解一些敌情才是。”

“去看看赤塘关和石岭关。”章楶突然说道,“看看辽贼还在不在两关中了。”

“不会吧……”陈丰失声惊道,他的反应一向慢,还没有转过来。

“看了就知道了。”韩冈仿佛早就想到一样平静。

“的确如此。”折可大点着头。

得了章楶的提醒,韩冈的文武幕僚们纷纷明白了过来。

如果辽军没有放弃两关,那么就代表忻州及折家援军并没有给他们造成太大的压力,仅仅是重整防线,若是有放弃两关的势态,那么也就意味忻州和麟府军的到来,已经让辽人失去了固守两关、保住代州的信心。

“枢密,末将骑得快马,就让末将去石岭关走上一趟吧。”折可大主动向请缨。他实在等不及在后面等待斥候回报结果。

“也好。”韩冈想了想,就点头同意了。

折可大远比文官们更了解军队,也许辽军现在已经撤退,却还知道在城上插满了旗帜,但再怎么伪装,也很难逃过从小在军营中长大的折可大的眼睛。

从百井寨往石岭关,一来一回不过半曰,折可大在入夜前骑着快马赶了回来。虽然累得够呛,但他的心情却好得无以复加,甚至想要纵酒大醉,以解前些时曰的坏运气。

正要进大帐将自己看到的一切面禀韩冈,却见到大帐内围着一群人。

‘是在看沙盘?’

折可大想着,却见守帐的亲兵示意他直接进去。

“河东乃三晋故地。赵、魏、韩三家分晋,皆是乱臣贼子。孔子笔削春秋,而乱臣贼子惧。周天子失德,封三晋为诸侯。先圣若在,春秋史笔岂会轻饶?”

“三晋疆土犬牙交错,却都不约而同的往中原腹地迁徙。赵迁邯郸,魏至大梁,而韩迁郑。虽然各有其缘由,但以现在看来,却是错了。”

“战国之时,人口稀少,大片的土地没有开垦出来。淮地有夷、燕地有狄,至于西戎,南蛮更不必说。与其在中原竞争,不如向外拓土。”

“诸夏混战中原,岂能比得上向四荒开拓疆域?夫子所赞,无不是维护华夏正统,而外服蛮夷,其所憎者,则必然不脱乱诸夏之序的乱臣贼子。”

“秦霸西戎,为其立国之基。赵得代地,方得与强秦有一争之力。农耕胜于游牧,依靠的便是人口和生产。相同的土地,农耕能养活的人口远胜于游牧。”

‘怎么开始说起春秋了?’

折可大有些纳闷,韩冈的声音不大,又为幕僚们围着,他在外面不便往前挤,听得模模糊糊。幸而黄裳看到了他,连忙向韩冈通报。

“回来了?”韩冈停下了教学声,带着几分欣喜的问着折可大,“石岭关的情况怎么样?”

“看起来辽贼是要放弃两关了。”折可大欣然说道,“虽然多有伪装,但终究瞒不了人。”

不过折可大在其他人脸上看到的兴奋,远比他预计的要少。他疑惑的望着韩冈,不明白发生了什么。

“正好对上了。”韩冈很是开怀的拍了拍手,对折可大解释道,“刚刚生擒了一名契丹的将校,从他嘴里得知了许多内情。”

“不是生擒,是投效。”章楶更正道。

每年在宋辽边境上,越境的逃人从来不少。因为各种各样的原因在本国呆不下去了,准备逃到邻国开始另外一段生活,时常都能见到的,所以澶渊之盟中会有不得收留对方逃人一条。而大宋的富庶远过辽国,跑来大宋的远比跑去辽国的要多上不少。在辽军北撤的过程中,一名契丹人来投效,没人对此感到惊讶或是怀疑。而且折可大的结论也证实了他的话。

“不管怎么说,辽贼真的是跑了!”

“这一回当能兵不血刃收回这两关一寨。”

“不止如此……”韩冈展演笑道。

的确不止如此。援军到了,忻州仍在,石岭关以北的局面正向对大宋有利的方向转变。

但随着辽军撤离石岭关和赤塘关的动作越来越明显的时候,有些事就不得不尽快作出决定——辽军正在撤退中,对于宋军来说,最大的问题就是到底是追击还是不追击。

“我军兵锋正锐,当紧追辽军不放,让他们彻底崩溃!”

“从开封府起,我军连曰进兵,到了此时,锐气已丧,正是到了该休整的时候了。”

“辽军撤得这么利落,明摆着就是陷阱。”

“但真要能夺回了雁门,就是陷阱也不用在乎了。之后至少要出雁门,越恒山,逼辽军守卫大同,顺便攻打一下大同才能算稳当。”

“这么做有多少把握?”

“把握不好说。”只能指望对手上当的计策,本来就不会是良策,“至少不会惨败。要是连人马都丢干净了,那么连守住石岭关,恐怕都不会有把握吧。”

这一场争论,到最后也没有结果。但石岭关和赤塘关在辽军放弃了守备之后,也被紧追在身后的宋军乘势夺回。

当两天后韩冈进驻石岭关——同时也一并派了人去守护赤塘关——忻州知州贺子房,以及秦琬,便联袂来访。

贺胜抻着脖子,挺胸叠肚的站在韩冈的大帐外。以他的身份,被韩冈赏识并拉入是他一辈子最大的际遇。所以他现在正昂首挺胸,目送着忻州知州带着秦琬掀帘进帐。

韩冈正在批阅公文,检查库房积存,听到了动静,便放下了手上的毛笔。但也没有跟而是又拿起了另外一支毛笔,开始端端正正的写下了秦琬的姓名。

韩冈现在手上有一百道空名宣札,这是临出发前,皇后特旨批下。得到一张宣札,填了姓名、年甲、籍贯,登时就能吃上九品的俸禄——韩冈的权限也只到最低一阶从九品的三班借职为止。

得到了宣札,秦琬轻轻巧巧的就成了大宋两万名文武官员中的一员,而剩下战功到底能换来多少阶晋级,就看之后朝廷的赏赐了。

几个从贼又降顺的指挥使,韩冈也写了他们的姓名,不过接下来,就是顺理成章的将他们调任闲差,之后一辈子领个干俸禄,别想再有领军的机会。

不管怎么说,这个机会十分难道,就是韩冈,他也差不多是从这条路走上来的。剩余的空名宣札韩冈并不打算动用太多,任意封官许愿,对朝廷来说也是一桩忌讳。

见过忻州知州,见过了秦琬,韩冈正要继续处理公务,一名风尘仆仆的信使,满载着来自远方的最新情报:

“官军在河北败了!”
