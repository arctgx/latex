\section{第46章 了无旧客伴清谈(一)}

司马光认为天下财富自有定数,薛向的观点可比司马光的观点进步多了。

大宋内部和平百年,边患真要细论起来只能算是癣癞之疾,但人口已经快要达到土地允许的极限,而田地的增长也快要到了极限,工商业至今还仅仅是补充。这样的社会,其每年生产出来的财富基本上就是一条略微向上的直线,而且绝大部分的增长还都被同样增加的人口所抵消,甚至由于人口增长的幅度更大,人均收入都在隐隐的下降之中。

尽管此时工商业发达,但从朝廷税赋的构成上来说,依然是彻头彻尾的农业社会。所谓资本主义的萌芽,也就仅仅是萌芽而已。

王安石隐约看到了这一点,可他由于本身的局限性,所创诸多新法,除了农田水利法以外,其他有关财计的政策,便民贷、均输法、市易法、免役法、方田均税法,从本质上说,无一不是对社会财富的再分配。从士绅阶层手中,将他们过往攥在手中的收入收归国有。对于国民经济的发展,并没有太大的帮助。

而司马光和王安石所争的,就是这份收入,是应该给国家多一点,还是留给士绅阶层多一点。

至于升斗小民、愚民黔首、百姓、庶民,也就是处在社会底层的人们,在新旧两党的交锋中,从来都是拿出来的幌子而已。

无论是变法前、还是变法后,他们的收入并没有多少区别。免役法,让五等户也要交免行钱,不比过去,做衙前做到倾家荡产的,都是三等户以上的富户。但便民贷,则让底层的自耕农少了一份盘剥,多了一分保住土地的希望——地方大户可以将欠债的自耕农的土地收来抵债,而地方官一般是不敢大规模这么做,闹出乱子,他们少不了被弹劾被治罪——一出一入,差不多就抵消了。

相对于朝廷的政策而言而言,还是雨水多寡对于百姓们的生活水平影响还要更大一点。

元丰元年是赵顼即位以来难得的丰年,由于税赋的数额大体上是固定的,朝廷的财政收入没有太大的变化,相对的,百姓们留在手上的钱粮自然要比前些年多了一些。

为了弥补熙宁后期的连年灾害对各地常平仓的消耗,今年各路都是敞开收粮,同时也就保证了粮食出售价格的稳定,没有出现丰年谷贱伤农的情况。

可若是遇到灾年,则还是少不了朝廷的赈济,不论是变法前还是变法后,平民百姓都没有能靠自己的积蓄度过难关的能力。

司马光和王安石都着眼在财富的分配上,而薛向却能想得到如何增加财富——并不仅仅局限在农业上——这是分蛋糕和做蛋糕的区别。

物流的畅通,自然能带来商业的兴盛,并必然会促进工业的发展,这是韩冈最想看到的变化。但并不是所有朝臣都喜欢薛向的说法,农为国本、商兴害农的思想,在士人心目中根深蒂固。

而且并不是完全的没有根据。前几年冬天极冷,太湖冻结。在太湖湖中岛上上种柑橘的果农,因为运粮的船只被冰层阻挡无法上岛而被饿死,成了朝臣攻击商业害农的最新的武器——在此时士人的眼中,所谓的农,只包括五谷和蔬菜。至于种植水果,那是商业生产的一部分,与耕战二字并不搭界。

韩冈能看到王珪和元绛的眉头都皱了一下,但他们都没有出来驳斥的意思。因为赵顼现在正在点头微笑。

天子并不是很清楚薛向的一番话中隐含的见识——恐怕薛向自己都没有清醒的认识——也就没有韩冈的惊讶,但他对薛向的回答很是满意。商业兴盛,自然财税大增,至于会不会妨害农事,这件事等真的出现了再考虑也不迟。

“方城轨道开通,运送行旅,转运民间的商货,不及月余,便入库两万贯。不过轨道兴修之初,本为渠道修成前暂用,如今轨道转运不输水运,这渠道是否该继续开凿,倒想问一问薛卿你的看法。”

薛向一瞥几名宰执和韩冈,看他们面上漠然的神色,心中就有了底。以他们的身份,以及韩冈在此事上的发言权,如果愿意作出决定,方城渠道的事轮不到天子来征询自己。

略作思忖,薛向便道:“以臣之见,轨道易修易用,何须浪费公帑?纵有损坏,最多数日便可修复,比起疏浚河渠动用的人工,俭省甚多。”

宰执们没一个愿意下定论,甚至韩冈都因为种种原因缄口不言,但薛向不同,他一向勇于任事,也不得不勇于任事。

仅仅是个荫补官员的薛向,只因少一进士及第,在朝堂上被人视为另类。他的处境,不比当年的狄青强到哪里。

当年狄青屡遭韩琦欺压,他倚之为臂助的将领,因为韩琦想杀鸡儆猴,随便找个了过错就被杀了。狄青为部将求情时,说他屡立功勋,为国杀敌,是好男儿,韩琦则说‘东华门外以状元唱出者乃好儿,此岂得为好儿。’之后为枢密使时,又遭文臣群起而攻之,只得悲愤的说,‘韩枢密功业官职与我一般,我少一进士及第耳’。

薛向自是知道,韩冈不肯就渠道和轨道之间的取舍下一定论,这是他的机会。作为荫补起家的官员,不比进士出身的官员更拼命、更努力,表现出自己的不可替代,想要朝堂上站稳脚跟,永远也不可能。

薛向掌管的是汴河水运,正是这方面的专家,他既然说漕渠不如轨道,也就让赵顼拿定了主意:“修造方城渠道的差事,就不要放在襄汉发运司中了。”

王珪这时又上前一步:“臣领旨。”

解决了一桩事,赵顼又问起韩冈:“韩卿,京西修了轨道,河北也修造轨道,不知陕西能不能也修上一条到两条。”

“若是京兆府周围,直至出潼关,有渭水和黄河水运,若是想要往缘边各路转运,则山势起伏,轨道难修,尚不及冬日于冰雪上以雪橇车输送粮秣。”韩冈转了一下,“不过可以先行勘察地理,寻找合适的路线。”

赵顼点了点头,收起了在陕西修造轨道的心思。

“河北轨道开始修造,陕西缘边各路的筹备……”

赵顼可能是想要提及对夏战争的话题,不过话声到了这里一下就顿住了,崇政殿中,统掌军事的枢密使不在,枢密副使也不在,只有武将身份的同签书郭逵一人。

眼下的情况当真是个笑话了,枢密院中三位执政,现在两位被御史逼得避位,方才讨论轨道之事时浑没在意,现在将讨伐西夏的战争一提上台面,没有枢密使应答,他脸色就变得难看起来。

天子的视线在殿中转了一圈,定在了韩冈的脸上:“韩卿……西北之事,你有何看法?”

被点了名,有所准备的韩冈随即朗声道:“西夏国势已衰,加之母子失和,内乱近在眼前。但秉常为辽主之婿,当年丰州之战,有皮室军助阵,由此观之,契丹当有唇亡齿寒之心。故而西北之事,不在党项,而在契丹。于河北修筑轨道,瞒不过契丹耳目,不过只要西北不开战,契丹君臣当还下不了破釜沉舟的决心。”

韩冈的话,基本上就是之前朝堂上已经讨论过的结论。在轨道表现出出色的运输能力之后,天子和宰辅们都有了一个共识,暂时并不对西夏开战,等到河北御敌的准备完成,倒时候,再挥兵攻打西夏。省得打到一半,被契丹人陈兵白沟之外,逼得前方退军。

赵顼尽管急着想要将西夏剿灭,然后北收幽燕云中,做他的‘唐太宗’,天可汗。但仅仅修造轨道的一两年的时间,他还是有些耐性的。

毕竟契丹的威胁性太大,赵顼一直从心底里,甚至是骨髓里对其感到畏惧。就算是丰州的胜利吹得神乎其神,仿佛河东军一举大败西夏和辽国联军,可实际上参战的皮室军,也不过是区区数百人而已,而辽国动员起来的总兵力,百万以上不成问题,实实在在的控弦百万。

韩冈以眼角余光瞅见赵顼在点头,便继续说了下去:“只是朝廷御敌,当做万全之虑。以臣之愚见,河东一路,西制党项,北当契丹,东更能援河北,当以精兵驻守其间,以防不虞。”

赵顼一听,立刻就看向郭逵,之前郭逵在河东待了好几年,丰州就是他领军收回。

郭逵眉峰骤起:“陛下进取之心,即便辽国亦不会不知。如今朝廷于陕西缘边诸路整军备战。如今契丹上下都绷紧了弦。河东如果骤然增兵,恐怕他们会有所误会。虽然官军不惧辽人,但无妄之灾,自是能避免就避免,而且攻夏的机会一旦失去,就再也弥补不回来了。”

赵顼的视线又移了回去。他想知道韩冈有什么解决的办法。

“更戍法。”韩冈就只有三个字。

