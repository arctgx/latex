\section{第20章 冥冥鬼神有也无(八)}

韩冈在桂州待了七天。

先用了一天的时间,与章惇就眼下的局势和他们能做的应对,一起商议过。接下来的六天中的大部分时间,韩冈就是去整顿和了解广西转运司的现状。

作为漕司主官,韩冈可以将所有的事务都交给副手去做——依制,转运使一年至少有半年要巡回地方,只有副使才会常年待在治所处理事务——但监察之事却必须做到位,账簿、库房,都要清点明白。

等一切都检查完毕,韩冈便毫不耽搁的启程南下。

眼下的当务之急,就是抢先一步将开战的全部准备一切都安排好,而不是等着军队和物资到齐之后再做考量。而在这些准备中,邕州这个出发地,就是最大的关键。

韩冈远比章惇要熟悉邕州,两人一番商议之后。最后的决定就是由韩冈先下邕州,整顿军备、粮秣、道路、寨防等一应事务。待到燕达领军抵达广西,休整几日,章惇便会与这五千西军精锐一齐南下。

尽管从桂州到邕州可以乘船直达,只是要稍稍绕个弯子,但韩冈带着南下的两名幕僚都摇头拒绝再上船只,而是说既然,还是骑马更快一点,不要耽搁时间。

跟着韩冈南下的是李复和陈震,马竺和周毖则是要先熟悉一下转运司中的工作,过一阵子则和章惇一起行动。

另外韩冈早在刚刚抵达桂州的时候,就先行下了一道命令,让左右江三十六峒的洞主们,即刻赶往邕州听候号令。

在交趾国中抢了几个月,左右江三十六峒蛮部也差不多都到了极限,各自回乡休养生息,同时也在整理着他们的收获。在韩冈抵达桂州时,章惇就提醒过韩冈,三十六峒蛮部送回来的汉人有八千,那么他们抢到手的交趾人又该有多少?必须对此有所应对。

不同于前一次,此次韩冈传令左右江,人人悚然听命,没有一人敢于慢待。等十日之后,他终于抵达邕州的时候,三十六峒的洞主们也几乎全都到齐了。

韩冈也不拖延,一到邕州,梳洗过后,先见过忙得瘦脱了形的苏子元和李信,当天就将分住在城中的七十多位洞主全都找了过来。

这几十位大小洞主,有的是仇家,有的则是戚里,一见面就吵吵闹闹的,见了亲戚朋友问候几句,见到了仇人虽不敢在堂上捋起袖子就开打,但也少不了骂上几句,闹得州衙大堂如同水烧开了一般。

只是听到几声锣鼓响,几名卫士髙喝肃静,韩冈在苏子元和李信的陪伴下从后门走上大堂。一个个正吵闹不休的蛮部首领,被韩冈温文和煦的眼神一扫而过,菜市场一般喧闹的大堂就顿时静得针落可闻。

‘不意龙学积威一至于此。’李复、陈震在后面感叹着韩冈的威势。

他们当然知道韩冈在邕州做下了何等的功业,也在桂州看到了城中军士、漕司僚属对韩冈的恭敬。但现在亲眼看到一个个脸上刺青、耳上带环、蓬头垢面,形如妖魔鬼怪的蛮部首领,在韩冈出现之后,连大气也不敢喘,比起儿孙进了祠堂后还老实的样子,这时他们才更进一步切身体会到韩冈在广西立下的赫赫声威。

走进大堂,韩冈当先坐了下来,又请了苏子元和李信左右坐下。洞主们一个个屏气息声,先是大礼跪拜,站起来之后,便俯首帖耳的等待韩冈的发落。

韩冈左右看了看堂上七十多名洞主,开门见山的说道:“这几个月,尔等能遵奉本官号令,清扫交趾北疆,救出我汉家子民八千余人,本官对此很是欢喜。”

听到韩冈定下了基调,洞主们一个个都松了口气,神色也放松了一点。其中一名洞主操着一口流利的广西腔官话,点头哈腰:“相公的号令,小人自是要尽全心全力,不敢打半分折扣。”

“想必在这其中得到的交趾人口不少吧?”韩冈知道,对这些洞主们说话,与其绕来绕去,不如直截了当的询问,“听说这几个月,有几家可是一口气得到了过千丁口。”

年轻的广西转运使的话似乎有着深意,被他说话时扫过来的视线压着,洞主们又都惶惑不安起来。

韩冈也不跟他们打哑谜,“你们就没有想过,万一有一日他们起了歹心,联手反乱,到时候,你们峒里会要遭多大的罪?”

这番话,没人会误认韩冈是在关心他们家里日后会不会有麻烦,在场的没有那么蠢的人。但到底要怎么做,一时没人开口询问,都不想听到韩冈说出他们不想听的话来。

只是沉默不能一直维持下去,过了片刻,方才说话的那名洞主又先出头道:“那小人回去后就将他们都砍了,省得日后麻烦!”

一众洞主的脸色都难看起来,但韩冈却摇了摇头,安了他们的心,“妄兴杀戮有伤天和,不当如此。且此辈仆从,都是各位辛苦招来,本官也不会一句话就让你们全都杀了。”

‘那到底该怎么做?’众人你看看我,我看看你。那名洞主想了一阵,也想不出个眉目,勾着头哈着腰,小心谨慎的问道:“不知相公想要小人怎么做,还请明示。”

韩冈抿了口茶水,并不说话。

李信咳嗽了一声,不耐烦的开口,“只要不让他们能再犯事不就行了,好生去想想该怎么做!”

那名洞主眨了眨眼睛,点头道:“小人明白了。回去后就将他们的大脚趾都砍了去,只要能种地做活就行。”他抬头堆出一副笑脸,“秦汉隋唐之时,南面的都是中国的交州,没有交趾的说法。砍了他们的脚趾,日后就只有交州,没有交趾!”

韩冈略感惊讶仔细看了这位知情识趣的洞主。在诸多蛮人打扮的洞主之中,只有他装束像个汉人,或者说根本就是一个汉人。但上一次召集洞主时,并没有见到了他出现。

“小人龙礼合,现下是冻州洞主。少年时在江上讨生活,也曾读过几年书,知道何为忠义。后来因缘际会,被老洞主招做了女婿,三个月前才接了位子。”

“原来如此。”韩冈点了点头,“看来冻州的老洞主挑了个好女婿。”

龙礼合砰砰的磕了几个头,“多谢相公夸赞。”

站起来后便喜笑颜开,他的浑家虽然是老洞主的独生女,但老洞主还有几个关系隔了一层的堂兄弟和堂侄在。而韩冈一句话,就让他彻底坐稳了冻州洞主的位置。

等到韩冈的视线又从一个个洞主的脸上扫过,所有人都跪了下来,纷纷叩头应承,等回去后,就将峒中所有交趾奴隶中的男丁的大脚趾都给截去。

这件事给定下来,韩冈也没什么好吩咐的了,又说了些勉励的话,便摆下酒宴,招待这一干洞主。

等到夜深,席终人散,韩冈与两名幕僚回到房中。李复就问道:“龙学为何一定要对那一干交趾人施以肉刑?”

韩冈并不解释,反问着另一人:“不知子孝对此有何看法?”

陈震道:“龙学是不想让这三十六峒势力过大吧?”

“嗯,确有此意。”韩冈笑了一笑,“邕州此次元气大伤,户口损失数万,二十年内都不恢复不了。若是三十六峒蛮部多了一批兵源,日后又会是个麻烦。”

“龙学当还有以此为先例的打算。”陈震接着又道:“即有如此先例,日后攻入交趾之后,便可以将其国人尽数施以剕刑。”

韩冈笑着点点头:“正如子孝所言,日后攻入交趾境内,但凡捉到的男丁,我虽不会杀他们,但也得设法让他们不能再为恶。”

“龙学是说笑吧?”李复脸色大变,“此番有损龙学声望。”

“履中可还记得我曾经说过的,当初是如何对待交趾俘虏?早就做过了。”韩冈不以为意,有多少人会关心交趾人的脚是否有十根趾头,“一个脚趾换一条命,愿意交换的不在少数。”

“你们可曾想过,攻下升龙府之后,该怎么做才能保证南疆的长治久安?是设州置县,还是交还给交趾宗室,又或是将其土地分割给当地大族?”

“交州地处海外,即使是设州置县,也无法顺利招来多少汉人来此安家落户,至少得穷尽三五十年之功。”陈震摇摇头,如何安置交趾,他们几个早就讨论过了,“此法不当取。”

“交还交趾宗室更不可能。安南郡王之罪,当诛九族,哪里能再让李姓之人封王?!”李复也道,“只有分割土地。分封给当地的大族、甚至还有交趾的官宦,让他们去争夺厮杀,可保日后南疆平安。”

韩冈则是另一个想法。

“现在无法设州置县,不代表以后不能。交趾立国百多年,民心已经疏离中国,不论怎么分割国土,最后也只会让他们重新确立秩序。既然是如此,还不如先交给这些蛮部,日后设州置县,阻力就会小一点。”韩冈看看还有些没想通的两名幕僚,笑问道:“立有文法和未立文法,两家不同的藩属,你们觉得哪一家比较好?”

