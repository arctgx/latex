\section{第30章 回首云途路不遥(下)}

州桥这里的夜间摊上,难得有一天的安静。

这本是东京城的夜晚最热闹的地方之一,是进出内城的交通要道,车马行人,如同水脉,川流不息。就在州桥边的路边摊点,永远都是行人驻足最多的地方。

并非是年节,也并非是暴雨、冰雹,但今天的州桥夜市却是一模一样的安静。

经过此处的行人,连呼吸的声音都轻了下来。

只因为停在夜市旁的车马队伍。

宰相和枢密使,帝国文武两班的首脑。当他们坐在这里,哪里还有人敢于随便靠近?旁边一圈腰挎长刀的元随,更是如狼似虎,就是经过的行人,几十双眼睛虎视眈眈,哪个不是加快了脚步,或是干脆绕了过去。

额头上的汗珠星星点点,摊主双手都被烧烤占了,也没敢出声让自己的婆娘来帮忙擦一擦,缩头缩脑的在袖子上蹭了一下,又赶急赶忙去给正滋滋冒着油的猪皮肉刷上一层秘制调料。

“好些年没吃过州桥这里的旋炙猪皮肉了。”

距离摊主不远的一张桌子旁,韩冈丝毫不顾仪态拿夹起一块外脆里嫩的猪皮肉放进嘴里。

咽下去后,看着没动筷子的章惇,韩冈挑了挑眉:“怎么?看不上眼?”

章惇一脸挑剔。

食不厌精、脍不厌细的枢密使,自不喜这等粗俗的民间食物。在韩冈面前,他也没必要故作豪放。

不过韩冈既然说了,面子得给,稍稍的尝了一口,他就皱眉,“孜然倒是不少。”

他没想到韩冈请他吃饭会在这个地方。

宰辅不可私会,这条规矩已经不能再束缚当今的两府,但光明正大的在街市上聊天吃饭,正面挑衅的做法,章惇觉得并不合适。

“打通了西域、南海、大理,香料和香辛料的价格全都降了。”

孜然、胡椒、八角、豆蔻、丁香、没药,大宋原本要进口或是偏远之地才能生产的珍贵调料,如今已经

“多到可以做暴发户?”

章惇把筷子举了举,孜然和胡椒粉扑簌簌的往下掉。

“不,不是香料太多,是宰相和枢密使太少了。”

章惇笑了一声,又夹起了一块来看了看,“火候不差。”

尽管这么说,章惇却没再动筷子放嘴里。

推销不出去,韩冈故作叹息,“以后看来不能找福建人出来吃烤肉了。”

“烤羊肉可以,烤牛肉也行,这猪肉就算了。”

看起来,偏近西北的饮食,对福建人来说,的确没有太多的吸引力。当然,没有改良过的猪种,味道也的确不如后世,加上养殖不得法,也难怪一直贱过牛羊,不为人喜。

“合口的烤牛肉可不容易吃到。”

“嗯,一年也不定有一回。”

朝廷禁屠耕牛,就是牛病死、老死,也要先报官之后,才能分解发卖。若是牛受伤了,不得不宰杀,同样是要先报官,待衙门派人确认之后,才能宰杀。一般来说,市面上的牛肉,还是以病死老死的为主。

富户如果当真想吃新鲜的上好牛肉,有的是变通的办法。可韩冈、章惇贵为宰辅,为了口腹之欲触发律法,这未免太蠢了,所以两家都是不沾牛肉,日常以羊肉、猪肉为主,鱼类、禽类辅之。

真要说起对牛肉大快朵颐的日子,还是两个人还在广西的时候,那边杀牛就跟杀猪一样普遍,新鲜的小牛肉都是想吃就吃。

见韩冈和章惇都停下筷子在说话,摊主汗水流得更多,将新烤好的肉装盘,借着上菜的机会,来到桌边。

摊主一阵点头哈腰,小心翼翼的问:“两位相公,小人秘制的旋炙猪皮肉可是哪里不合口味。”

韩冈哈哈笑着指了指对面的章惇,“合我的口味得紧,只是不合这位章枢密的口味罢了。”

摊主顺着韩冈的手指看向章惇的时候,满是油汗的一张黑脸,几乎要哭出来,

“别听他胡说,只是没胃口。这里不要你服侍,去烤些给外面的人吃。”

摊主连连点头,忙不迭的答应下来。

回头就在心中感叹,两位相公真是菩萨脾气。包括以前来过的坏了事的薛相公,这三位来店里吃过饭的相公,个个都是和和气气,比来每个月过来收税的税吏还好说话。

“等等。”

摊主刚准备捋起袖子,好生再亮一亮手艺,把两位相公手底下的人都喂饱喂好,就听见背后有人叫。

忙转身回来,见是韩冈叫住了他。

叫了摊主到身前,韩冈问道:“上次我来这里的时候,烤肉的不是你,人呢?”

“小人阿爹……那个先父,两年前就去世了。临去前,在开宝寺那里开了分店,给了小人的弟弟,这里就给了小人。也多亏了相公,相公上次来过之后,家里的生意就好了几倍,天天客满。小人的先父在家里把相公的长生牌位供上了。”

“可惜。”韩冈叹息道,“你手艺还不错,但比你爹还差一线。”

摊主连忙低头:“是小人学艺不精。”

“算了,先去烤肉,那么多张嘴等着你呢。”

摊主离开,韩冈转头对章惇笑道,“开宝寺边卖烤肉,真定家的那群小和尚口福不浅。”

章惇冷哼道:“哪家的贼秃缺了吃喝?多了一个烤肉,也只是换换口味。”

开宝寺的主持大师真定和尚,御赐紫衣,在僧录司中列名,是京师中数得着的名僧。只可惜韩冈、章惇皆看透了那些和尚到底是什么货色,吃喝嫖赌的水平,小时候还荒唐过一阵的章惇都赶不上其中的平均水准。

摊主离开之后,韩冈环顾四周,几年前,他就是跟薛向一起,在这里吃过烧烤。好像很久之前的事了,想起来,就让人有种莫名的怀念。

他感叹道:“如果薛师正还在就好了。”

“说得好像死了一样。”章惇嗤笑道,“薛向不还活着?”

韩冈回手指了指皇城,“在那里已经死了。”

以大逆之罪被发配岭南,这辈子不可能再翻身。说他死了,正是因为他的政治生命,在蔡确被杀的那一刻,已经死了。

章惇笑容消失了,的确,成为罪囚远流岭南,薛向其实已经死了,有无死讯,不过是一条消息罢了。

“没有了薛师正,汴河纲运就给弄得一团糟。每年损失超过一成,费用增加三成,养肥了多少条饿狗?”

“还不是薛向害的,他做了叛逆,害了多少人才。”

薛向掌控六路发运司的时候,大刀阔斧任免官员,他所提拔的基本上都是人才。但他一倒台,这些人全都受了他的牵累,而没被重用的咸鱼翻了身,可惜不是贪官污吏就是废物。

等韩冈厘清朝局,腾出手来准备整顿纲运的时候,薛向立下的规矩和制度,已经败坏的不成样了,所以韩冈修京泗铁路另起炉灶时才那般容易。

韩冈轻叹,“人生一世间,如白驹过隙,一切转眼即逝。”

“过得是快。”章惇追忆着过去,“记得当初我被贬出京,玉昆你一大清早就来送我。还在汴河边吃了两个炊饼,那时候,就有白糖馅炊饼了。”

“因为那时候交趾的种植园已经开始产白糖。”

“那时候拿下交州才两三年。”章惇道,“如今天下产糖,交州居其半。运出交州的稻米,每年也是数以百万石。”

“天下人口日繁,未来的大宋,需要更多的交州。”韩冈试探着章惇。

“未来?”章惇回望韩冈,“玉昆你觉得这样的局面还能维持多久?”

韩冈沉默了片刻,抬眼问道:“不知子厚兄你怎么看天子?”

章惇摇摇头,“是我先问玉昆你的。”见韩冈苦笑起来,他又道,“我先答也没什么。小儿罢了,最多有几分聪慧,可惜性子差了。”

“先帝的性子其实也不算好。”

韩冈还记得熙宁八年的时候,赵顼被辽国佯作南下的恐吓,吓得逼谈判的臣子割让国土。甚至拿臣子的家眷作威胁。

韩维还在谈判桌上保护国家利益,而做皇帝赵顼却从后面拆台,逼着韩维早点把土地割出去好结束谈判。从那时开始,韩冈就对赵顼失望透顶。

“但先帝会用人,能用人,知道什么样的人不能用。如今的皇帝,全无分寸。等他亲政,也许一切都会付之流水。”章惇抬起眼,盯着韩冈,“玉昆,我这可是掏心掏肺的说给你听了。”

“多谢子厚兄能坦诚相告。”韩冈拱了拱手。

章惇没回礼,一双眸子仍是盯着韩冈,“玉昆,你还没回答我的问题呢。”

“有希望,不容易。”韩冈回答得很简洁。

“重臣议政都出来,还是不容易?”

“权臣不能做,不可做,只有集合众力一途了。”韩冈无奈的笑道。

如果是自己的东西,当然要牢牢抓在手里,可惜不是自己的,而且又难以独力抢上手,那么也只有拉上一帮人上来瓜分了。

“但这也只是第一步。”韩冈继续道。

“那第二步呢?”章惇问道。

几年前韩冈就跟章惇说过他的想法,当时章惇决定与韩冈分割开来,但现在,局势易变,两人又重新坐到了一起。
