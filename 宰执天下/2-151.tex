\section{第29章 顿尘回首望天阙(三)}

【第一更,求红票,收藏。】

韩冈赶在午时前回到居所。王韶他这边正好送人回来,看到韩冈,便是有些惊讶:“玉昆,怎么回来得如此之早?”

“已经是午时了!”韩冈看了看天色,提醒王韶道。

“午时算什么?去中书等差使,不到申时哪可能能回来?!”王韶可能是有着一段不愿去回想的苦难过去,两句话里就透出了他对中书门下的旧怨。“大概是玉昆你得人看重吧,王介甫和冯当世你见到了哪一个?”

“都没见!”韩冈摇起了头,这两位她也不想见,“只跟章子厚说了两句话。他那边帮我们把事都办妥了,王相公也不会特地再招下官去。至于冯大参,他当是不喜见到下官。”

王韶听出了韩冈对冯京好像有些看法,但他没心思追问这些小事,而是问着更为重要的一件事:“还是要去延州?”

韩冈摇了摇头,“怎么都推脱不了了……”

“……苦了玉昆你了。”王韶的叹气声中满是无可奈何。

虽说韩冈在王安石面前,给自己的推辞找了诸多借口,王韶也明白韩冈至少有大半的理由,是因为他更为看重河湟之事。对韩冈的忠诚,王韶深为感动。韩冈可是放弃了在宰相面前卖好的机会——而且是两名!

王韶另外还担心韩冈离开后,会给新成立的通远军带来什么变数。这一年来,河湟开边能如此顺利,连番大捷,韩冈的功劳绝对是占到了很大的一块分量。韩绛强行把韩冈调走,这是明摆仗势欺人,王韶就算已经认命了,也免不了一肚子的火气。

“……关于调任延州之事,章子厚已经说了,这仅是暂调而已,不会在鄜延久任,不久还是会回本职。”

“章惇没这本事,王介甫也不可能虎口夺食!韩绛若不答应,天子也挪不动你。”王韶摇头不信,但他又想了一想,却是恍然大悟,“是韩绛在延州留不久!”

得王韶提醒,韩冈只慢了一点,也便明白了为什么章惇能说得那么肯定:“不论横山得失与否,韩相公都不会在延州久留,长则一年,少则半载,就会回京——从没有宰相长久在外领军的道理,就算天子不担心,言官也会找机会说话。届时韩相公一走,下官就可以回古渭……不,是回通远军了!”

韩冈和王韶正在说话,这时李小六从韩冈的小院跑过来。韩冈向王韶告了罪,过去问李小六,却道是路明前来拜访,并带了章惇的请帖而来。

关于章惇要在樊楼摆宴的邀请,前面在中书的时候,韩冈已经听章惇亲口说了。不过路明带着章惇请帖亲来,显得更为郑重。

韩冈转身要向王韶告辞去见客,不过王韶却道:“是当日与玉昆你一起上京的路明?……前次二哥进京,也跟他见过面,得了许多指点。也该见他一见,谢上一谢。”

路明很快被领了进来。王韶端坐着,韩冈则起身相迎,“明德兄,别来无恙?”

路明当然无恙,境况甚至比当初要强上十倍。

才一年不见,他的气象大不同于从前。原来一身的穷酸措大气消失无踪,现在是红光满面,如面团一般发起来的一张脸,把皱纹都冲淡了许多,竟变成一个略显富态的官人模样。

路明在两人面前拜倒行礼:“有劳韩官人挂心,在下这年来一切安好。”他又看向王韶,问着韩冈,“韩官人,这位是否就是大破西羌、威震边陲、名震天下、引得天子垂顾的王子纯王安抚?”

路明会说话,马屁拍得也好听。王韶自昨夜听到噩耗时起,就变得木然的一张脸,终于松懈了下来,微不可察的笑了一笑。他今次上京升了正七品的左司谏,不过安抚使比司谏听起来还要高一些,路明便是往高里喊去。

倒是韩冈,一直以来他在官运上,跟王韶相比算是比较背时的。尽管韩冈自入官后一年三迁,其进速已经足以让人目瞪口呆,可比起他的功绩,仍是不免要使人叹一声朝廷刻薄。韩冈今次上京,预定之中是要进宫面圣,依例必然是要特旨迁官,为了能让天子亲自加官,以收买人心,所以在渭源之役的封赏名单上,也就没有他的份。但现在韩冈因故见不了天子了,他这一场辛苦,却什么都没换到。

对于自己的运气,韩冈也没了什么想法,只盼着皇帝能记得他在这方面吃了亏就行了。

路明与王韶见过礼,寒暄了两句,从袖中掏出两份请帖来。看写在信封上的收信之人,不仅有着韩冈的名字,而且还没忘了王韶——章惇是准备将韩冈和王韶两人一起请到。

“路明受章检正所托,带了这两份请帖来。今日入夜后,在樊楼之上,已经备下了一席水酒,恳请安抚和韩官人勿要推辞。”

王韶和韩冈同住驿馆中,如今是炙手可热,多少人正愁找不到跟王韶拉上关系的途径。章惇既然要摆宴,他的请帖没有只发给韩冈,而不给王韶的道理。

王韶将请帖展开了看了一看,里面的文字当然不会像路明说的那么没有一点文采,王韶看了之后都不免默默点头,难怪能两次考上进士。当下就在韩冈这里拿了纸笔,随手写了回覆,让路明待会儿带了回去。他准备去一趟,与章惇多多拉近关系。

韩冈也写了回书,正式的谢过了章惇的邀请。今日章惇办席,他和王韶算是主宾,而路明提不上筷子,照规矩多半会再找个朋友来。

听说章惇跟开封府的推官自少相交,情谊匪浅,如果有章惇能把这位推官请来。韩冈倒是很期待。

……………………

周南的闺房中,没有金玉之类的俗物,只有少少的几件素雅的装饰。

横阔只有一两丈的房间中,有着一床、一桌,一张古旧的梳妆台,还有一个只容两人并坐,中间架着矮几的短榻。一张古琴横放于榻前,沉黑色的附足棋墩连着两只棋盒则堆在短榻一角。方枕边有着一卷柳屯田的诗集,而一张烟锁重楼的画卷,则是挂在素白的墙壁上。虽然落款的李公麟非是当世名家,但出自今科进士的尽心手笔,也正证明了周南的魅力。

静谧的房间,碎檀木阴阴的燃烧着,浅淡的香烟,从狮耳螭纹的兽头绿釉香炉中徐徐腾升而起。若有若无的檀香味,让人的心神全都变得平和了起来。

周南对着镜子,用墨笔轻轻描着眉线。原本就是不描而翠的纤秀双眉,被墨笔划过,便把更加惹人心动的线条,用笔画勾勒出来。

周南瞧着镜子里面的自己描好的双眉,左望望右看看。作为东京城中屈指可数的花魁行首,若是不能把自己最好的一面表现出来,那就实在太丢脸了。

一名四十余岁、风韵犹存的半老徐娘推门走了进来,圆润丰满的身材甚是惹人注目。只不过也许是为了遮去皱纹,脸上的脂粉便用得多了些。红红白白的,像极了用掺了丹砂的石灰抹过的墙壁。这一位,家中排行第一,人称许大娘。二十年前是教坊司中有名花中魁首,现今则成了教坊司的教习,管着周南和其他十几名官妓。她的这个身份,如果是在民间青楼,也就是老鸨了。

周南从镜中看到许大娘进来,便站起身,冷冰冰的唤了声:“娘。”

周南的冷淡让许大娘微微变色,但很快她又挤出笑容:“今天秦二官人可能会来,南姐儿你就留在家里,哪里都不要去了。”

周南仿佛没有听到,丝毫也不加理会,重又对着镜子坐了下来。今夜的妆容才做到一半,当然不能半途而废,她还想着在情郎面前做到最好。

拈起一片来自杭州的胭脂饼,浅浅的在掌心抹了一层,白玉一般细腻的掌心因胭脂而染上了晕红,这样的红,就是等待情郎的妙龄少女脸上才会拥有的颜色。一点也没有许大娘脸上用来刷墙的红色那么粗俗。

只是看着满手的红,周南想了一想,又把胭脂都收下来,手很快也擦干净了。当今世人,喜欢浓妆的甚多。多有将胭脂粉如抹墙一样厚厚的擦上脸颊,虽然不比唐时宫女,太过浓烈的装束弄得洗脸后,盆中都是鲜红一片,周南不喜这样的妆容。她一直都是淡妆,甚至素面朝天的时候都有。只不过今天还是要花一点妆,不能让人以为她是个没有受过正确教导的土包子。

也不理睬正瞪着自己的许大娘,周南信手抽开梳妆盒上的一个小抽屉,里面的放着一件只有掌心大小的龙凤磁盒,封在上面的金漆纸证明了这是出自于官造的器物。揭开磁盒上的封条,打开盖子,一股丁香混着藿香的味道散了出来,里面盛满了丹红色的口脂。

探出嫩如葱管的手指,周南轻轻抹起一层脂膏,涂在了嘴唇上。

