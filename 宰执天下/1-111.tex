\section{第42章 诡谋暗计何曾伤(四)}

【今天第二更,求红票,收藏】

“怎么办?”程禹头疼了,低声问着刘易。韩冈的前三项完全挑不出毛病,他们一年要审查考核新进官员数以百计,但能如韩冈这般出色的,也不过一个巴掌就能数得出来。差不多能与那些不用铨叙的进士媲美了。

“你糊涂了?!秦州三家齐推,天子亲下特旨,你还敢把他当成普通的从九品选人看?!过三关是肯定的,过不了才奇怪。”刘易眉毛扬了一扬,阴阴笑道,“但别忘了,还有‘判’啊!”

程禹总觉得事情正往他们不想看到的方向滑去,韩冈表现出来的才气实在不低:“……万一他还能通过呢?”

刘易冷笑着,他才不信才十九岁的韩冈能有天纵之才,普通才子即便只是背背经书,学学诗赋,等到有一点水准,也早过了二十岁了:“真有那本事,他早去考进士和明经了。弄个正经出身,不比他人推荐要强?有出身升官有多快,天下有谁不知?”他摇摇头,把藏在心底里的一点忧虑压下去,对程禹的担心过度不以为然的冷笑了一声,“别傻了,把题出难一点,专挑冷门的词条,谅他也做不出来。”

程禹沉吟着点点头,刘易说得是没错。他提声问道:“韩冈,你身言书三项皆过了,接下来便要试判。可还有别的话要说?”

韩冈摇摇头,微笑着轻快的说了声,“没有!”

他现在心中很轻松,至今为止的三关测试,对严阵以待的韩冈来说确实很轻松。没想到所谓的铨试真的这么简单。不过随便的谈了几句,就说他身言书三项都过了。不但比不上前世打过交道的那些挑剔苛刻的客户,也比不上应聘面试上的考官,也就跟他上的那所二流大学毕业辩论的程度差不多,现在想想,那些教授还真是好说话。

而刘、程二位也是一般的好说话,想到自己方才还误会了他们,韩冈心里还真有些过意不去。即便方才总觉得两人神色不对,也应该是自己太多心了的缘故。自家就是这个毛病,凡事总会想得太多。

“那好!”程禹觉得韩冈脸上善意的微笑有些扎眼,说话的速度便促了一些:“判试分为墨义诗赋和断案两项。照规矩先考墨义、诗赋。这两部,韩冈你可自选。你选哪一部?”

所谓的墨义,就是在九经挑出一些片段做为题目,然后要求考生写出这些句子的大义。而答案,基本上是出自各经流传在世间的权威注疏。韩冈的诗赋是不成的,而出自九经的经义,他的水平还算不错。故而他毫不犹豫:“墨义!”

“选定了?”刘易再问一句,“选定便不能再改了。”

“选定了!”

韩冈的回答斩钉截铁,心中突然却又忐忑不安起来。已经是铨试的最后一项,过了这一关,就正式成为一名从九品选人了。第一次在这个时代参加考试,还是关系到是否能拿到差遣的考试,若是失败,可就要等下一次。流内铨的‘次’,是轮次的意思。以如今在流内铨外守阙的选人数目,轮上一次,少说要一年。韩冈虽然有自信,但心底也免不了要打着小鼓。

借个准备试题的名义,程禹和刘易留下韩冈,从偏厅里走了出来。

“下面怎么办?”程禹问着刘易。

刘易将早已准备好的考卷从袖子里掏出来一展:“你看这几题怎么样?”

程禹接过来仔细看过。说来惭愧,几题一看,他都有些发懵了。除了《易》《礼记》《尚书》的文字特别,不会错认,其他应是出自《春秋》三传的几题,进士出身的他竟然连具体出处都把不准。而且这些题目,他现在一点都做不出来。他瞧了一眼刘易,自家是考诗赋论出来的进士,而刘易则是明经九经科出身,他出的题目,自己做不出来也不奇怪,就不知能不能难得住韩冈。

刘易得意洋洋的自夸着:“《左传》一道,《礼记》一道,《书》两道,《谷梁》和《易》各三题。这十道墨义,我可是挑着最生僻的句子摘录,谅韩冈也做不出来。”

“一题兼经的都没有?”程禹低声阴笑:“做得好,做得好!”

明经诸科,并不是像科举那样,是同一个科目,统一的考题,而是分为九经、五经、开元礼、三史、三传、三礼、学究诸科,连考试内容,考试科目都不一样。但在这些科目中,《论语》是必须要学要考的,所以称为兼经。以韩冈的年纪,《论语》必然已经精通,还是不要冒险得好。

“万一过了怎么办!”程禹笑声一顿,又抓着头苦恼起来,“新进选人注官的铨试实在太容易了。十题九不中才算不中格,万一给韩冈撞个大运……”

“若只对个两三题,也是一样啊。照样可以给官家看看,看王韶他们荐的是什么样的‘才子’?!让天子下特旨的究竟是什么样的大才?而且……”刘易压低声音,眯起的眼睛显得更为阴险:“别忘了,还有最后一道判事没考。”

“妙!”程禹醒悟过来,顿时抚掌大笑。

偏厅中,韩冈静静的等着,没有半点不耐烦的神色。前面面试的宽松,韩冈本不再为最后一项而头疼,但刘易和程禹久去不回,却让他的心又提了起来。该不会又有什么变数吧?

这时两人走了进来,刘易示意韩冈做到偏厅一角的一张桌案后,递过来一份试题,“韩冈,这十条经文,须写出正文大义,不可有悖逆之言,更不要犯了杂讳。如十题九不中,便得再次守选,即便你有天子特旨,也不能违例。”

‘十题九不中才会被打回去?!’韩冈惊得下巴都要掉了,一百分的卷子只要考到二十分就算合格?!

不对!铨试的规则既然这么宽松,难度定然不低,戒骄戒躁啊,韩冈!

他在心中提醒着自己,站起来恭恭敬敬的接过考题,道了声“韩冈明白!”就坐下来紧张的翻看考题。

“这……这……”韩冈只看了一眼,便轮到眼珠子要掉下来了。他指着考卷,张口结舌的转头瞧着刘易。

刘易跟程禹交换了一个眼色,得意洋洋。他凑上前,故意嘘寒问暖一般关心的问着:“怎么,题目有什么问题,是不是太难了?!”

韩冈忙摇头,怎么可能难?!他回头再看一眼试卷,没错,他没有看错!

第一题是‘大夫执则致,致则名;此其不名,何也?’

第二题是‘六五,贲于丘园,束帛戋戋:吝,终吉。’

第三题是‘尔惟践修厥猷,旧有令闻,恪慎克孝,肃恭神人。’

一直到第十题——‘为尊者讳,敌不讳败,为亲者讳,败不讳敌。’

整整十题墨义中,没有一题不是出自九经。韩冈的前身,对此下了多少年的功夫。而他本人,自来到这个世界后,手不释卷,一部部又重新抄写过。到如今,倒背如流是吹嘘,但用滚瓜烂熟来形容,却一点也不过分。而且甚至有几题所摘录的经文,还是他这几天刚刚跟程颢讨论过的,想不到连运气也在他这里。

韩冈从头到尾,从上到下,翻过来覆过去的看了五六遍,终于确定不是出题人的陷阱。他心中暗自感叹,完全没想到,所谓的铨试就是这么个考法!十道试题全数出自于九经不说,连要求的答案也标明不得超过注疏的范围。

‘这是公务员考试啊,你给我初中毕业考试试卷做什么?!’

韩冈暗自揣度,自家能如此顺利,多半是因为他仅仅是一名从九品选人。若是高品的京朝官,保不住会有哪个看河湟开边战略不顺眼的官员横插一杠,表现一下不畏君上的气节的同时,还可以坏了王韶的好事。但自己的品级实在太低,为难他根本没有任何好处。武松打老虎挣回一个都头,打老鼠能挣回什么?打苍蝇又能挣回什么?

韩琦当年一封弹章,把两名宰相两名执政都一脚踢出了政事堂,这才叫本事!而把门一关,将一个从九品的选人踢回老家,这算什么?!本事?刚直?屁都不是!

所以现实就是这么回事,没点利益,谁会无缘无故与人为难?而且这人身后还有天子背书?

韩冈越想越觉得事实当是如此,他感激的抬头看着刘易和程禹,发现他们正微笑着看着自己。韩冈还以微笑,当真是好人啊!

当即提起笔,韩冈先抄考题,再写答案,三下五除二,转眼间,十条试题的答案跃然纸上。行行蝇头小楷,排得整整齐齐。检查过是否有犯杂讳的地方,发现没有问题,他便添上姓名,站起身,将墨迹淋漓的卷子交给两位笑容已经变得勉强的两名流内铨令丞。

“怎么办?”偏厅旁的另一间房中,程禹脸色难看的问着。

刘易默不作声,阴着脸,拿着笔批改韩冈的卷子。一个圈,两个圈,三个圈,到最后一直连圈了十个圈。放下笔,他呆呆的说着:“十题皆对,无一条错……他干嘛不去考明经?!”

“所以我问你怎么办啊?!”程禹的声音第一次大过刘易,完全气急败坏。

刘易狠狠抬起头,反问着:“这题你来做,你做得出?”

“…………怎么办?”程禹的声音这回小了许多,他是靠诗赋论出身的进士,又不是明经。何况他自入官后,哪还有年轻时熬夜苦读的劲头,当年的才气能剩下三四成就不错了。他又横了刘易一眼,这位老明经怕也是如此,过去的学问全丢下了,才把自己认为难的题目拿出来给韩冈做。

“还有断案!”刘易咬牙发狠,“把登州阿云的那桩案子找给他!”

