\section{第18章 青云为履难知足(12)}

晨钟响起的时候,夜色刚刚化开。

吕惠卿望着自己的身前,只有两人——冯京和王珪,本应站在最前的王安石今天又没有上朝。

政事堂中的首相,已经有四五天都在假中。是天子特意降诏,以王雱重病,特给王安石假,令其在家中抚视。连着几日的常朝皆是由冯京在文德殿中押班。

吕惠卿也听说昨夜宫中连夜发诏之事。擢王雱为天章阁直学士,从天子的心意上是冲喜,可怎么看都像是追赠,王雱那边也是辞而不受。

王雱的病情已经拖了好些日子了,从太医局传出来的闲言碎语中,吕惠卿本来估摸着差不多也就在这几天了。不过就在方才,吕惠卿听说韩冈昨天已经到了京中,但他没有在群臣中看到韩冈,论理是不应该的,除非有什么大事让他请假。

‘……看来下午的时候,要换身衣服去相府了。’吕惠卿这么想着。

冯京和王珪肯定也能想到,但神色中不见有何异状。吕惠卿的视线扫去西班。吴充那是不必想,再怎么说都是亲家,若有事必然要遣人知会,他的儿子吴安持也肯定要去王安石的府上。班列中气氛有些诡异,想必听到消息的人,都会明白发生了什么。

作为王安石一直以来的亲信,吕惠卿很清楚王雱在王安石心中的地位,也清楚王雱对新法、新学的坚持,是王安石一直坚持将新法推行下来的重要原因。而王雱所处的位置,更是许多时候能说服天子的关键,不能轻动——否则他早就应该放外任去了,也不会现在还是朝官最低一级的太子中允。

王雱这一去,可谓是内外皆失。不过影响的并不是新法,而是……

净鞭声猝然响起,吕惠卿连忙收起心中的想法,将注意力集中起来。

今天的常朝,天子也是照例不坐。冯京带着文武百官向文德殿中空无一人的御榻行过礼,百官们便各自散去,而一干重臣则是往崇政殿行去。

崇政殿后殿中,赵顼已经等着很久了,低头看着刚刚送来的一份急报,沉思不语。

王雱做了几年的崇政殿说书,之后又是直讲,一直都是天子近臣,时常跟随在身边,也是赵顼很欣赏的年轻臣子,想不到就这么去了。

赵顼叹了一口气,人的寿数真是说不准。王雱一时英杰,才学过人,就只有三十三岁,再往前,一些名动天下的才子,如杨亿、苏易简,也都三四十而已。

说来自己也快三十了,身体一向不算好,赵顼抬头看着殿顶承尘上斑驳的红漆,也不知还能在这座殿中坐上几年。而且做皇帝从来命都不长,前数几代,赵家都没有出过一个过六十的天子,赵顼也不指望自己当真能千万岁寿。

更大的问题还是子嗣。王雱听说还有个儿子。自己这边,儿子、女儿则是生一个死一个,加起来都九个了,就只留了一子一女下来。而且这唯一的儿子自出生后身体就没好过,前两天还犯过一次惊厥,不知能不能养得大。

赵顼咬着牙,难道要像仁宗皇帝最后从宗室中另找一人作为养子?

说起来他能成为皇帝还是靠着这份幸运,可一旦仁宗皇帝境遇落到自己身上,就让赵顼感到难以忍受了。自己父亲当初是怎么做的‘孝子’,赵顼都看在眼里。听说仁宗皇帝到了晚年的时候,时常在宫中哭泣,都是靠了太皇太后来劝慰。一想到自己会变成那幅模样,赵顼就感到不寒而栗。

但要说宫中阴气太盛,对寿数、子嗣不利,那也不对。宫城内寿数长的,赵顼也不是没见过。真宗皇帝的沈贵妃现在还留居宫中,已经八十岁了,身体仍可称得上康健。逢年过节,太皇太后和太后都要过去拜望。要是说起宫女、宦官,在附属宫掖的几座寺庵,甚至有年过百岁、服侍过太宗皇帝的人瑞。当真活不长、养不大的,也就他们这些天家的子嗣了。

“官家,两府已经到了殿中。”当值的石得一悄步走过来提醒着。

赵顼的头上下动了动,示意自己听到了,只是依然愣愣的做着,没有动弹。

等了片刻,石得一忍不住又催促了一下,“官家!”

赵顼身子一震,回过神来,“啊,知道了。”

天子终于起身,让石得一松了一口气。忙在前领路,向着重臣罗列的前殿过去。

坐上御榻,群臣叩拜之后,赵顼赐了宰辅们的座位。

没人提起不在班列中的王安石,更没人提起已经王雱。赵顼看了一眼吕惠卿,连他都没有多提上一句。对于这间大殿上讨论的国家大事来说,病死了区区一个天章阁侍制,只是一桩微不足道的小事而已。

“郭逵自太原上书,但言河东兵马已经准备就绪,只待朝廷之命,便可出兵收复丰州。”

“丰州沦于贼手半载有余,州中生民涂炭,望官军如赤子望父母,不可再拖延须臾。”

“交趾之事也不能置而不论,当从西军中拈选精锐,南下攻敌。”

“西军不可轻动。为茂州事,已在熙河调兵数千,熙河路的守军不能再少。眼下将及秋高马肥之时,缘边诸路旧年都要防秋,现在更要提防西夏铤而走险,哪里还能调兵。”

“交趾在广西烧掠三州,杀戮以十万计,又掠我中国子民数万入国中,岂可视而不见?”

“契丹国中不稳,自顾不暇。可从河北调集精兵强将南下。”

“契丹在河北耳目众多,路中异动,必惹其疑窦,兵力不能调动太多。”

“荆南军能以千五破十万。河北精兵又远胜荆南。即便为防万一,有两万已是足矣。”

“虽云十万,疲军而已。若以官军入交趾,将是交贼以逸待劳,皆是兵少恐不足用。”

群臣们的争论,赵顼都没有插上话。就像过去的一个月一样,怎么都达不成一个共同的意见。日子一点点的拖过去,留给赵顼已经没有多少时间了。

收复丰州箭在弦上,已是不得不发。且正如吴充、蔡挺所言,在这样的形势下,关西诸路的兵力不可能轻易调动。唯一压力不大的熙河路,能动用的驻军又被调去了镇压茂州叛乱。攻打兰州都说了好几年了,明明很容易的一件事,都因为各种各样的事情给耽搁了。在罗兀城陷落之后,党项人派在兰州的驻军又增加四千,禹臧花麻那边恐怕也不敢轻举妄动。

但交趾也不能置而不论,通过章惇、韩冈发来的战报、以及前些日子与苏子元的对话,赵顼对邕州之战的前前后后,已经了解得差不多了。且夷族灭国四个字诱惑着赵顼,执其君长问罪于前,是赵顼在登基之后日夜盼望的荣耀。必须在这几天决定从何处调兵,并在这个月内发兵,否则时间上就来不及了。

幸好韩冈刚刚入京了,还是招韩冈入宫询问,因为王雱之事,他现在应该正在王安石家里。

从韩冈身上想起了王雱,赵顼问道,“王雱昨夜病亡,此事该如何处置?”

赵顼突然发言,让殿中冷了一下场。纵然是宰相之子、天子近臣,也不够资格让宰辅们议论,朝中自有制度。天子若要是加恩,直接下手诏就行了。

冯京作为宰相,率先开口:“王雱官至太子中允、天章阁侍制,依制当由太常礼院处分。可待其遗表奏上,循故事而行。”

冯京既然如此说话,吕惠卿就不好不发言:“王雱明经术,通国事,惜壮年而丧,朝廷当优加抚恤。”

对此没人反对,反正连赠谥都不够资格,就算再有旧怨,也没必要在这时候添堵。赵顼看了下方诸臣一眼:“赠左谏议大夫,官其幼子,余事交由太常礼院处分。”

……………………

王雱还是没能多熬过一夜,在快四更的时候咽下最后一口气,撒手人寰。

人走了,剩下的就是礼仪。

一切在一个多月前就开始准备了。一个时辰不到,灵堂就设好了,家中、门前的灯笼都换成了白色,白帐子也在相府内外挂了起来。

站在大门外的迎客是王旁,而韩冈则换了素白头巾,没带冠的站在灵堂内,在烟熏火燎中眯着眼睛,迎接进门来祭奠王雱的亲朋好友。站在韩冈对面,则是连襟吴安持。

对韩冈来说,王雱是亲戚,更是朋友,能送王雱最后一程,韩冈很庆幸自己没有在路上耽搁时间。只是说起悲伤,其实不多。但他真心为王雱感到难过,不论两人的目标是不是相抵触,但壮志未酬身先死,总是让人遗憾无比。

一道帐幕拦着灵堂内外,女眷都在里面。王雱的独子则跪在帐外,往火盆里丢着纸,烟火从火盆中腾腾升起。王安国、王安上家的子孙在旁边陪着。吴安持的儿子,韩冈的两个儿子也一起跪着。

王旖是已出嫁的女儿,以五服算是大功【注1】,要为兄弟服丧,穿着熟麻布做的丧服就在里面陪着她的母亲。韩冈的子女,不论是否王旖亲生,都算是王安石的外孙,也是王雱的外甥,同样是穿着孝衣,不过是用比王旖略细一些的熟麻缝制。

韩冈有些担心望着里面,王旖有孕在身,在送了王雱的同时就哭成了泪人,伤心过度动了胎气可就不好了。

“玉昆,不用担心,里面有人会看顾着。”是站在对面的吴安持在说话。

的确,王安国、王安上家的女眷不会犯糊涂,而韩阿李也在里面待着,当不至于有事。韩冈点点头,向连襟表示感谢。因为吴充的缘故,加上吴安持之前常年在外任职,韩冈与他来往不多,不过毕竟是亲戚,老死不相往来那就没必要了,更没必要成为仇敌。

这是外面一阵喧闹,是宫中派来的中使到了门外。

王安石穿着一身麻衣,被人搀扶着,拄着拐杖蹒跚的走了出来。悲伤在脸上刻画出深深的纹路,须发又白了一片,在朝堂上面对多少强敌险阻都不会弯下的腰背,这时候是佝偻着,一日之间仿佛又老了十岁。

听着中使传达的天子恩典,王雱从太子中允一举成为左谏议大夫,还有许多远超普通臣子的赏赐,可这样的恩典没人喜欢。

王安石麻木的依例谢恩之后,接下来,另外一名中使走过来,不是向王安石,而是向韩冈,“皇帝口谕,招龙图阁直学士、广西转运使韩冈即刻入宫陛见。”

“……臣遵旨。”站起来,韩冈转身过来向王安石告辞。

王安石点点头:“国事要紧,玉昆,你去吧。”

向着王安石行了一礼,进灵堂又拜了一下,韩冈去了丧仪,换了紫袍犀带,上马往皇城疾驰而去。

注1:五服是为亲戚服丧的五个等级。由重至轻,分别为斩衰、齐衰、大功、小功和缌麻。另外还有更轻的袒免,为五服之外的亲戚或朋友服丧。不同等级,服丧的时间不同,所穿丧服的等级也不同,粗麻、细麻、生麻、熟麻,乃至是否收边都有定制。。

