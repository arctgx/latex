\section{第47章 百战功成朝天阙(上)}

【补昨天的第二更。】

自收到王韶得胜、木征降伏的消息,熙河经略司上下又等了半月之久,王韶终于回来了。

提前两天,王韶他们的行程就通过快马来回传递。确定了回来的日子,狄道城中的官员,也都准备好了出城迎接王师凯旋的工作。连这些日子杜门不出的韩冈和王中正,也都要一起去迎接归师,只有吕大防和蔡曚留守城中,而沈括都动了。

这一日的凌晨,天色还是黑沉沉的,统领狄道城的全班人马便一起出动。来到城外,一直向南,直至狄道城南面的要塞南关堡。出城相迎,也就是‘郊迎’,是迎接归师的礼节,离城越远,礼数也就越重。南关堡据狄道二十里,这个距离也就比天子巡游回来,百官郊迎的距离稍差。

一众官员立于南关堡的城头上,眼望着远方。而憔悴了许多的蔡曚,则是死死盯着韩冈,眼中都冒着火光。感觉到来自身侧的视线,眼角的余光瞥到了蔡曚脸上的神情,这等充满憎恨、但又全然奈何不得自己的眼神,韩冈丝毫不会顾忌,只会在心头感到一阵痛快。

前面在出城前,吕大防和蔡曚来送行的时候,韩冈可是按照礼节,上前与两人见礼问候。

吕大防回了礼,但蔡曚却板着脸根本不加理睬。

韩冈那时是一笑转身,他把自己的礼数尽到也就够了,蔡曚怎么样,他可不在乎。而旁边的吕大防脸色却更为沉郁。

可以死,可以败,可以办些蠢事,可以坏了国事,但绝不能在应有礼仪上失态。既然身在朝中,就不能学山野中那等疏狂的名士,可以放言‘吾辈岂为礼法所拘’。对于拥有官身的士大夫们来说,这不仅仅是丢人的问题,更是直接会让人质疑起他们士人的身份。

‘居上不宽,为礼不敬,临丧不哀,吾何以观之哉?’——不能具礼,如何称得上是士大夫?圣人都会看不过眼。

蔡曚此举实在是有失身份,韩冈离开前,还看见站在一边的吕大防微微的摇着头,看起来也是觉得蔡曚太过失态了。

时间慢慢的过去。星月仍在挂在天空上的凌晨,他们就从狄道城出发。日上三竿的时候,抵达了南关堡。现在日后已经升到了天顶,五月下旬的太阳,火辣辣的晒着路边的蝉虫直叫唤,更是晒得城头上等候归师的众官们汗流浃背,身上的公服,都被汗水浸成了深色。

但没有人提议要去道边树荫下,或是城门门洞中避上一避,这个时候的态度是最重要的。身为官员,应当知道什么时候必须吃点苦头。而且对于一直跟着王韶的熙河经略司中的官员们,迎接为他们带来一个个胜利的统帅的时候,根本不会在意一点暑热。而不属于经略司的几人,也不至于蠢到这个时候,就连蔡曚也是一样。

不过蝉虫鸣叫还是嘈得让人心急起来。城上城下,许多人引颈而望,远处一点点尘土飞扬,就引起他们的一阵骚动。

“来了!来了!经略回来了!”

两匹奔马飞驰而回,就在城门下仰首对着城头上的官员们大声喊着。

一大清早,一队探马就被派了出去,现在终于回来报信。而远处的山头上红旗招展,这是一开始就约定好的信号。就在这时候,一彪人马从摇晃着红旗的山坡脚下转了出来,带起的尘烟一下刺入人们的眼帘。

“来了!”

韩冈用力一拍墙头,立刻转身下城。跳上放在门洞中的战马,纵马而出,领先一步赶去迎接。有着韩冈领头,众官一愣之下,也立刻纷纷上马跟着飞驰而去。

迎着凯旋而归的王师,韩冈还有出迎的官员们,终于见到了久别多日的王韶。

一路主帅领军远征,前后不过两个月不到的时间。但对于急盼他们安然归来的熙河经略司众官来说,已经恍如天人之隔,数十载的光阴。

原本就是十分瘦削的王韶,现在变得又黑又瘦,不过气质却更为沉凝。还是那对沉重如山岳的眼神,并不犀利,但传出来的压迫感,却已经足够摄人。

百战功成的名帅,数年之中,为大宋开疆拓土两千里,其名留青史已是定局,任何时候都是一个让人敬仰的存在。

王韶现在带在身边的兵力就只有一千出头,战死、病殁还有各种意外,一路损失了近千人。但与此同时,他还收复了洮州的诸多蕃部,现下跟在王韶身后的,还有上百位,都是蕃部中的重要人物,以质子的身份跟着王韶回来。

从先一步传回来的消息中,人们知道,王韶已经留了高遵裕驻守洮州,而部将赵隆、傅勍同样留下协防。以千人守卫一州之地,说起来的确有些危险。不过高遵裕是当今太后的亲叔,天子的舅公。在并不了解大宋朝规的吐蕃人心中,还是与天子的亲缘关系更能震慑他们。

韩冈立于王韶马前,领着一众官员躬身行礼,“吾等恭迎经略凯旋归来!”

王韶生受了他们一礼,然后下马扶起韩冈:“这些日子多劳玉昆……也辛苦各位了!”

“不及经略远征之艰险。”

“都一样,都一样啊!”王韶哈哈的大笑了几声。韩冈在狄道城的抗旨不尊、还有伪传诏令的行为,他到了岷州后,都已经听说了,这番作为不是等闲官员敢做的。胆量之大,行事之危险,也跟他翻越露骨山,远征洮州的行动,差不了多少了。

恭喜过王韶的赫赫战功,韩冈见到了木征。他就跟在王韶的身后,身上的衣袍还是簇新的,看起来并没有吃苦的样子。

对于吐蕃王家的赞普血脉,河州曾经的统治者,王韶对木征还是依礼相待,吃穿用度皆是尽可能的丰裕。不过物质上的款待,应该抵消不了精神上的失意。但韩冈从木征的脸上,看不到半点穷途末路的败将模样。

韩冈上前与木征见礼,木征抬眼看着他,“可是当日驻守珂诺堡的韩官人?”

字正腔圆的官话,韩冈并不惊讶,但木征平和的态度倒是让他暗地里啧啧称奇:“正是韩冈。”

“久闻韩官人的大名了,经略相公已是当世英雄,又有官人辅佐,木征败得不冤。”

木征竟然很平静的跟杀了他弟弟的韩冈提起前日的惨败,又十分圆滑的恭维着王韶和韩冈。看他的模样,仿佛已经超脱得道,心中没有半点遗憾。

这就是河湟之地的规则。弱肉强食,愿赌服输。既然打不过,那就干脆投靠你。这对于吐蕃人来说,实在没什么好纠结的。

韩冈不知道是该称赞他洒脱,还是说他是看得开、识时务。也许木征已经知道了,他的用处不仅仅是在收复河湟蕃人,他的降伏更是大涨天朝脸面的一桩事,当他到东京城中走一遭后,只要冲着赵顼磕上几个头,官位必然远在韩冈之上。

就像木征的弟弟瞎吴叱,曾经的熙州之主,现在仍是熙州刺史,正五品的武将。如果木征效顺,去了京中一趟之后,只会更在瞎吴叱之上。他现在的河州刺史一职,很可能会更升上一级。

真不知看到木征后,他的弟弟瞎吴叱会是什么模样。

只可惜断了一条胳膊的瞎吴叱并不在这里。当官军开始攻打河州,瞎吴叱就被发遣到了陇西城去。投降大宋的蕃人首领,与逆贼继续暗通款曲,这是有西夏为先例的,熙河经略司中的官员怎么都不会冒险。

韩冈与王韶汇合之后,便全师向北,继续往狄道城前进。

天色将晚的时候,终于回到了狄道城中。吕大防和沈括此时早安排下了酒宴,让凯旋的将士纵酒狂歌。当韩冈打听起蔡曚的时候,才知道已经告了病,在住处卧床不起了。

这是何苦呢?

韩冈摇着头。若不是一开始就抱着敌意,但凡稳重一点,如何会落到这种丢人现眼的结果。就像吕大防,王韶还要感谢他这些天来,在狄道城中的作用。

王韶在酒宴上先喝了两杯,领头庆祝之后,让着参战的将领们自己快活,便起身进了内厅,同时也不忘把韩冈招了进去。

喝两口醒酒汤,王韶沉吟了一阵。对韩冈道:“今次一战,河湟抵定。接下来的几年不是打仗,而是要稳定熙河路各州的统治。不过这几年来,我从一介选人晋为封疆大吏,全是靠了河湟之功。如今不知有多少对眼睛盯上了我的位子了,怕是到京城诣阙之后,我就要离开熙河了,除非能北上攻取西夏,不然我怕是不会再有回河湟的机会……”

韩冈沉默着,这些事他都清楚。不用王韶说,他早就考虑过了。河湟开边,在外人看来是一帆顺水,但其中的艰难困苦只有实际参与了开边之事的王韶、韩冈他们自己知道。不清楚其中内情的人们,怕是只会将王韶的成功看成是幸运,而认为自己也能做到。换了有着这等想法的人来治理河湟,怕是会有大乱子。

“不知届时玉昆你何去何从?”王韶问着。

