\section{第十章 千秋邈矣变新腔(九)}

“《资治通鉴》编修局?这不是打发去西京吗?”

“到底是怎么回事?好端端的知阁试,怎么就变成了去西京?”

“好像是出题太难,黄裳没有通过,韩参政不忿,便大闹崇政殿,逼着太后将四名考官发配西京。”

“不是出的题难,是题目不对,黄裳本来考得科目就与他人不同,这题目当然就不能与其他人一样。”

“皆是制科,又只是阁试,为什么要区别?连阁试都过不去,还想得个制科出身?”

“幸好制科出身不能靠赏赐,否则一个进士出身后,还能再来一个制科出身。”

“不问兵法战策,不问地理人情,不问钱谷输送,却让准备守边的帅臣去考谏官的科目,这哪里对了?那几位通过了阁试的,难道就可以放去边境,让他们用弹章退敌?”

“终究是政事堂里的那一位私心太重,看到推荐的人被黜落便忍不住要讨个说法。王平章和章枢密推荐的两位也都被黜落了,也没见他们出来讨个说法。”

“韩三参政也说了,错用题目归错用题目,黄裳既然已经被黜落,为朝廷威信计,就不能再改易。朝廷事后若要补救,先从惩治考官开始。”

“如此谬论,知制诰难道就没有封驳?”

“封驳?也要有胆子才成。中书舍人里面,有哪个愿意与韩冈为敌?太后每次都站在韩冈一边,当时王平章都在场,不也没争过韩冈?”

韩冈刻意在对自己有利的条件下,挑起与新党的争斗,并且大获全胜。

这件事,在朝堂上掀起了一场轩然大波。

因为考试结果不合权臣之意,制科的考官全数被赶出了京师。权臣如此蛮横,在官场和士林中惹起了颇多议论。

不过来自京城各处的议论,一旦说到了这里,往往就会静下来。

知阁试的四位考官,因为黄裳在论述上坚持气学而将他刷落,这样的举动彻底惹火了韩冈。为了争一口气,宁可牺牲黄裳,也要将这几个考官全都发配到外路去。

韩冈的脾气和性格在这件事中表现得淋漓尽致。这是宁可自杀一千,也要干掉敌人八百的蛮子脾气。或者叫疯子更合适一点。

很多人都不怕脾气刚硬的对手,只要在官场上,总会有各种需求,这里面到处是利益交换的空间——再坚实的铁块,只要手段用对了,也能给锻打成型。

但脾气硬到只知道以直报怨四个字就不一样了。他们对利益的看法与正常人截然不同,至少与官场上的惯例不同,让人很不适应。

以韩冈在太后面前得到的信重,想要将黄裳给拉回到御试上,这不是不可能,最差最差也能让黄裳在其他方面得到补偿。但韩冈却偏偏放弃了黄裳,硬是将蹇周辅四人踢出京城去。

除了泄愤之外,恐怕已经没有别的解释。到了韩冈这一级,又以他的声望,他根本就不需要再拿人立威,杀鸡儆猴是偶尔做,而不该天天做。

不过韩冈是不是泄愤,是不是以直报怨,现在都无关紧要了。王安石当面都争不过韩冈,这新党的势头怎么眼看着就往下掉。

好不容易在先帝发病后,才熬到的新党大兴,才过去多少日子,便因蔡确、曾布的作法自毙,韩冈的另立山头,而变得四分五裂。

在过去,以韩冈对新法的态度,以及他与王安石的关系,在世人眼中,他即便因为坚持气学,而与新学无法相容,但观其行迹,至少也是个新党的外围成员。

不过现在,在经历之前推举宰辅之后,韩冈得到旧党支持另立山头,党争的苗头已经明显,而这一次的制举阁试之争,更是将党争的架势彻底拉了开来。

尽管蹇周辅等四人就算去了西京的《资治通鉴》编修局,他们还能保留之前得到的馆职。但三馆秘阁之所以在朝堂中为人钦羡,就是因为这是天子的储才之地,入此者无不很快便身居高位,更是得到圣眷的体现。

被赶出京城,可不是进入三馆秘阁的官员应该享受到的待遇。他们理应在崇文院中近距离接触到天子,从而得到天子的青目,擢任高官。或是知制诰,或是修起居注,又或是去乌台,便是出去做知州、都算是贬官。

但蹇周辅等人,就这么被赶出去了。

上一个比较有名的被赶出三馆秘阁的官员,是苏舜钦,这一位直接导致了庆历新政失败的罪魁祸首,自集贤校理的位置上被除名勒停,直到十二年后,才从湖州长史这个安排被贬官员的职位上,回到被贬之前的职位。

苏舜钦的被贬,拉开了范仲淹为首庆历党人被清出朝堂的序幕。而这一回蹇周辅等四人被赶出京城,了解旧事的人们,都不免会联想到四十年前的那一次党争。以及让人心寒的结局。

才一天的时间,有关阁试和阁试之后崇政殿那一场交锋的议论,已经不断的飞入韩冈的耳朵之中。

除了那些基本上就是重复之前他在崇政殿中与王安石争论的议论,他还知道残存在朝堂中的旧党,又是怎么看待自己与岳父王安石的争议。

什么‘君子合以义,小人合以利’;什么‘当初王介甫率群小与君子争,如今就要看到女婿与己争,真可谓天道好还,报应不爽’,然后还有有关王安石祖坟上的风水与女婿犯冲之类的笑话。

“我好歹也是得到富、文二公支持,范尧夫急着进京,就是为了将我推入两府。现在也能勉强算是洛阳的救命稻草。究竟是谁这么不长眼?”

韩冈看似半开玩笑,却又有几分煞气,让王厚坐不住了,霍的站起来,“这就去查!”

“算了。”韩冈叹道,让王厚坐下来,“我一开始就知道了,闲言碎语是免不了的。”

“那……”王厚皱着眉。

“蹇周辅不就是想看着我去为黄裳抱不平吗?我让他如愿以偿。”韩冈冷笑着,“不过他想借着踩我几下,在家岳那边挣个面子,我也让他满足。他想让我与家岳公开相争,一样没问题。不过想借此升官发财,那我可就不能答应了。他们背后或许还有人指使,不过我没那个闲空,去查究竟是蹇周辅、赵彦若这几位的私心,还是有人在背后推动。还是直接打发出去最省事,我倒要看看,最后还能有多少不长眼的跳出来?”

“所以将他们打发去洛阳?”王厚拍案叫绝,“这事做得妙。能暗中唆使蹇周辅的,也就那么几位,洛阳那边嫌疑最大!”

故意挑拨韩冈与王安石的关系,让新学和气学之间的矛盾更加深一层,为已经苗头显露的党争推波助澜,在这后面,得利最多的当然就是洛阳的元老们。

而且蹇周辅当初就是与范镇为布衣之交,而范镇,正是当年与司马光相唱和,是旧党中反对新法的急先锋。

蹇周辅被踢去洛阳,在不知内情的人眼中,韩冈是在报复他们在阁试上对黄裳的出题与判罚。但如果有人在背后指使,又是身在洛阳,那必然会明白韩冈是在警告。

“所以家岳最后没有反对的那么激烈,否则这项任命也没这么快就通过。”韩冈向王厚解释了一番,让王厚连连点头。

只不过,究竟王安石是气到不行,还是当真想明白了,韩冈自己都说不清楚。反正他回去后,拿这番话去搪塞王旖了。

韩冈可不想家里的葡萄架子倒掉,更不想看见妻子伤心。何况与王安石争论归争论,只要还留下一份情面,日后也方便再相见。

蹇周辅等先锋,被韩冈的雷霆之举清除了。

韩冈相信如果有人还想图谋不轨,至少不会走这条路。

但事情稍定,韩冈就不得不面对因为党争而落榜的门人。

……………………

在韩冈所能掌控的或影响的所有人中——可能还要包括那些散布在天南海北的旧日幕僚——黄裳绝是最为失落的一个。

或许他对落第早就有了心理准备,但成为整件事中的笑柄,又不能证明自己的才学水平,这让黄裳憋屈到了极点。

从今天早上开始,他便颓然的坐在书房中,一动不动,连妻子送来的饭菜,也没有吃上一口。韩冈虽然帮他出了一口气,却没能挽回他被黜落的结果。

难道就要靠一个赏赐而来的进士出身,在士林中混上一辈子?也许其他人能够忍受,但黄裳绝对不能。心高气傲是一桩,而明明满腹经纶,却为世人小觑的感觉,也不是黄裳能够忍受的。

也不知到了什么时候,一名来自于韩府的伴当,被领进了黄裳的宅邸,“黄博士,小人奉参政之命而来。”

“什么事?”黄裳没有什么想法,韩冈当着太后的面,所作出的决定,让他不抱任何奢望。

那名伴当低声在黄裳耳边,“这几日里,太后或许会招博士上殿,还望博士能够有所准备。”
