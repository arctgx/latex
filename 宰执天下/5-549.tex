\section{第44章 秀色须待十年培(五)}

 一进铸造的厂房,立刻就热了起来。

今天本是阴天,因为前曰下了两场雨,更是凉快了许多。

但厂房中热浪滚滚,让人感觉又仿佛进入了炎夏。

“真够热的。”章惇转头对韩冈说道,才进来,头上就已经见汗了。

“浇铸的地方,哪有不热的。”

章惇看着脚下黑乎乎的地面,有几分嫌恶的说道:“也就是玉昆你,愿意往这里跑。”

上次试射松木炮,可是在外面。这一回就给换到了室内,幸好只是看看,还不是试射,否则那声势,还不得把棚顶给掀了。

韩冈笑道:“有些事,还是亲眼看见比较安心。”

真正浇铸的时候,韩冈也是不会过来参观的。正经的工作时间,他这个宣徽相公贸贸然跑过去,让下面的工匠怎么做事?也就是浇铸成型,他才会过来。

上一次松木炮试射,蔡京和章惇被韩冈先后请来做个见证。这一回火炮铸成,而且是一次两门,章惇心急,就先过来了。而蔡确和郭逵,还是等着实验的时候,让他们来观看。

一行人被方兴引导着,来到新铸成火炮的地方。

两门青铜火炮并排放着。

方兴特意将两门炮放在一起,一眼看过去,几乎是一模一样,完全看不出哪里有差别了。

两门炮都是用青铜铸造。铜料并不便宜,但火炮有个好处,坏了之后,可以融掉重铸。里面的材料不会浪费多少。

只是不仅仅是铜钱需要合适的配比,青铜火炮中的铜配比也同样需要一个精确有效的数字。现在的这两门炮,也只是让人看看的。知道这是个有前途的武器。之后还要实验不同外形的火炮,前面装弹的前膛炮,后面装弹的后膛炮,都要进行设计和制作。而火炮所用的材料,其中铜、锡、铅的不同配比,也同要试验。韩冈虽然带着章惇来炫耀,可是只是听了韩冈的解释,章惇反而更加不懂了。就是当初造神臂弓也没有那么多麻烦的。

不过一对眼睛,章惇他还是带了来的。

这两门青铜炮,比起前一次章惇看到的松木炮,完全变了个样子。质地不同了,尺寸也小了许多。幽黯深沉的金属色泽,看起来就是有着莫大的威力,所以色泽深沉内敛。

火炮的大小,远远比不上霹雳砲和八牛弩,之前还是松木炮的时候,已经显得很小了,现在则更加显得小。加上炮身两边被韩冈起名做炮耳、用来支撑的短杆,也大不了多少。若是其威力能与两者相媲美,霹雳砲和八牛弩……不对,是行砲车和床子弩都要被淘汰掉。

“这就是火炮?”

之前看到松木炮的时候,章惇就这么问韩冈。现在终于看到成品,他再一次向韩冈发问。

“当然,哪里还会有假?”韩冈笑着回道。

方兴跟着解释道:“以火药驱动,将炮弹送去敌军那边,所以名为火炮。”

听着方兴的介绍,章惇突然凑近了,眯着眼睛看炮身的表面。

“怎么了?”韩冈问道。

“看看有没有沙眼。”章惇笑着抬起头。

沙眼是很难出现的,铸钟最怕的就是沙眼。一旦有一个稍大一点的,整只钟的音色立刻就会被毁掉。在怎么预防沙眼这个问题上,但在的这群铸钟匠是最出色的专家。

四寸的口径,其实还不如一个男子的拳头大。章惇用拳头在两门火炮的炮口处比了一下,两门炮的炮膛内径看起来完全一样。

“不用比了,肯定是一样的。如果不一样,直接就打回去了。”韩冈笑着说道,“炮弹是要放在炮膛里面射出来,炮膛的口径与炮弹必须配合上,否则就发挥不了最大的威力。现在火器局所监造的每一门火炮,是同样的型号,每一门都必须是一样的。炮身的长度、径围、重量,以及炮膛内部的直径、深度,都必须一样。”

“能不能做到啊?”章惇上下打量着火炮,还不忘跟韩冈说话。

“隔壁制造的炮弹,还有药包,全都是严格按照尺寸来制造,若是这边制造的火炮尺寸走样,要么炮弹放不进去,要么就是放进去了却嫌太宽敞。那还有什么用?想想一张弩,万一弩机和弩身配不上,箭矢装不进箭槽,那样的弩弓还能派什么用场?”

军器监成立伊始,便在精工细作及量产两件事上寻找平衡,至少在吕惠卿和韩冈主张监中事时所里下的规条,将尺寸工艺放在了极端重要的位置上。而工匠们也适应了对工艺的要求。

“子厚兄,你觉得这火炮怎么样?”

章惇没有立刻回话,而是说道:“先试一试吧。”

“就等着子厚兄你发话了。”韩冈笑着,让方兴去解决问题。

工匠们很快就架好了的滑轮组,将其中的一门火炮,吊装到了厂房内的轨道小车上。由两匹马牵着,人群跟在后面,同样是慢慢地走。

正是韩冈的计划。是先从轻型的步兵炮开始,这样也好铸造。能到有一点眉目,再抽调人手将重型的城防炮给设计出来。守住城池也能更安心一点。

口径四寸,重量在千斤左右,可以装在双轮的炮架上,炮架要安在轮子上,轮子要足够宽,如此才能适应野地的穿行。但炮架又要适量的重,这样才能压得住火炮的反作用力,但又不能台太重,这样才能用最少的马匹一起拖着随军走。

现在这种青铜炮由于韩冈担心会炸膛,所以命工匠们将第一第二门火炮,着意加厚制作。等到展示过后,真正的研发开始,那时候就要试着如何减少重量。甚至最后要将熟铁炮给研究出来。

怎么在一系列需求中,找到合适的平衡点,这就是得让设计者们去费尽心思的考虑的问题。韩冈估计这差不多要一年的时间。找出最合适的尺寸,这样才方便定型。等到定型之后,才是去设计量产方案。

如果能有镗床就好了。韩冈在工坊中不止一次这么想。一把刀具直接将实心的炮管给镗出炮管来,顺便再把火门给弄好。这样做出来的火炮,内部一定光滑如镜,必然是最为合适射击的炮管之一。

但现实中没可能那么简单。光是稳定的刀具就是绝大的难题,还有精确有力的转动刀身,这也不是这个时代能够给出答案的。

退上一步的结果,就是只能设法进行精密的铸造,剩下还有许多问题,所能变通的便是将误差许可的范围稍稍扩大一点,但还是要尽可能的精细,这样才能不负汉人精工之名。

火炮很快就给架设好了,就在军器监中的一片空场中。火药和炮弹都拿来了,瞄准的标靶也竖在了炮口之前,隔了有百步的距离。接下来就是如何去发射了。

“开始吧。”章惇等得急了,他希望还能遇到一个真正的惊喜。

工匠们立刻就忙碌了起来,装填药包,装填炮弹,接上引线,最后一切就绪,方兴冲着章惇、韩冈行了一礼,道:“还请枢密、宣徽来主持。”

章惇看看韩冈,韩冈比了个手势,请章惇自便。

章惇也不再谦让,拿起一支小火炬,小心的将引线点燃。

引线滋滋作声,转瞬就没入了炮膛内,还没等章惇闪回躲避的位置上,火炮就轰然巨响,将章惇都给惊得脚尖一颠,整个人一下高了两寸多。

回头一看,就看见火炮的炮口正被一阵白色的青烟所笼罩。没看到炮弹出膛的状况,枢密使感到很有些失望,再看韩冈,却发现他的神色有异。

“怎么了?”章惇很焦急的问着。

“中了靶子,却飞出去了。”方兴帮韩冈指了出来。

章惇仔细去看,果然木质的靶子上端有着一个小小的缺口。

“炮弹呢,飞出去了多远?”他顺着炮口的方向望过去,脸色陡然间也变了。那边再便宜点就是皇城的城墙!

“至少三百六十步。”韩冈的声音不复往曰的沉静。

他现在感到后怕不已,方才的这一炮弹,实在放得太过轻率,竟然差点就往皇城那边奔了。幸好这边的地势不方便瞄准皇城城墙,正好偏过去。而工匠们也不敢将火炮对准皇城。

“那不就是一里多地了?!”章惇再看了一下,脸色更显难看。没砸到皇城是幸运的,但砸到了人怎么办?

派人去看看吧。韩冈对章惇说了一句,然后点了亲信去火炮炮口的指向去寻找。

“怎么样?玉昆你觉得如何?”章惇之前紧张了一阵,现在放开了。只要不命中皇城,其他罪名都好说。

“终究是火药不行。”韩冈摇摇头。

韩冈清楚的看见那一支清理炮膛的拖把,带出了多少火药药渣。

看起来火炮火药最大的问题是原材料的提纯。

硝石的来源是茅厕墙上和地上的土,还有牲畜粪便堆积处的附近。那些地方才会产硝石。除此之外,还有其他种类的硝,名字类似,只是不能用来做火药。

硝与硝是不一样的。这个时代的命名法,完全没有任何条理可言。一个药材就有十几种名字。制作火药所需的是火硝,韩冈还记得是硝酸钾。而另一种叫做芒硝的,却是一点排不上用场。

只是元素的分离和提纯到现在还没有办法进行。韩冈粗浅的学术功底,不足以让他完成这样的任务。

他有时候还想过用光谱分析来确认新元素。煤气灯,三棱镜,显微镜,不论哪一项,军器监和将作监的技术储备都能实现他的要求。但有问题的是韩冈本人,他只知道光谱分析这个名词,其他一窍不通,具体的细节都没有,现在完全排不上用场。

真的是很难!
