\section{第23章 铁骑连声压金鼓(四)}

“在下野利征,见过韩兄。”

走出保护圈,孤身面对着韩冈和他的护卫。野利征毫无惧色的自报家门,行礼如仪,一套礼节做得比大宋官场里的武官都要标准。

拱手作揖间,野利征心中有着隐隐的得意。他知道眼前这位年轻的东朝官员正陷入两难境地,从礼节上讲,韩冈应该回礼。以野利征对东朝官员们的了解,粗鄙不文的武夫故且不论,那些汉人中的士大夫,可以自高自大,可以目空一切,但从小养成的习惯,让他们不会在礼数上稍有疏失——并不是他们真的对人有敬意,而是不想有失身份体面,更是因为自负于自身的教养。

可现在梁相公正率举国之兵,猛攻关西四路,而禹臧花麻也正受命猛攻渭源堡,他野利征来见瞎药同样是为了眼下如火如荼的战事。这样的情况下,来找瞎药求援的韩冈,又怎么能跟他野利征以礼相见?

而且两人都是为了同一个目标而来,在目的上与野利征势不两立的韩冈,又怎么可能在包括瞎药在内的这么多人眼前,跟自己礼尚往来?——野利征很清楚,他们党项人从来不在乎这些场面上小事,但汉人朝廷却对此极为看重,历年来,来国中出使的宋国大臣,只要说错了话、做错了事,失了他们朝廷的体面,回去后肯定会受到责罚,而能坚持上国天使尊严的,则会受到嘉奖。

韩冈果然如野利征所料,愣在了那里。虽然他立刻就反应过来,却也并没有当即上前,而是将视线投向野利征身后。

野利征回头一看,发现自己的护卫们手上还都拿着刀剑。他转眼便明白了韩冈在顾忌什么,心下暗笑‘果然是个无胆之辈。’摆手示意手下跟韩冈的护卫们一样都将兵器收起来。

见到野利征把,韩冈方才推开挡在身前的亲卫,走上前去,跟西夏国为了撬墙角才派来的使臣见礼。

“野利兄,韩冈有礼了。”

韩冈和野利征互相致礼后,场中剑拔弩张的气氛便被化解了不少。原本还担心着两方会在城中拼个你死我活的青唐部部众,终于都齐齐松了口气下来。

自立国后,西夏就向大宋称臣。不管两国之间的战争打得有多么惨烈,这份君臣关系却没有变化。在名义上,西夏国主也要大宋来册封,而实际上,当西夏国换了主后,东京都会派一名使臣带着册封制书到兴庆府去。因此两国朝臣之间的上下关系,便不能按照官职品级来定。不比宋辽,互相之间能互称南朝北朝,使得两国官员可以依照品级官位来确定高下。

故而韩冈跟野利征两人互相行礼说话,便一句也不提各自的官职,只当是没有官身的普通人相见。而他们的这种态度,在周围人看来,也隐隐的代表了两人暂时都不想提及宋夏之间方兴未艾的战事,并把架在两人面前的矛盾先搁置到一边。

韩冈不去面对现实,设法去解决眼前的敌人,不见半点破釜沉舟的胆量,让智缘的眼底透着深深的失望。他早在王安石口中,就听说过韩玉昆的名号,还有韩冈在为官前的一番作为。王安石将韩冈比之为旧年以剑术、胆略著称于世的张乖崖,不吝赞许,让胆魄过人的智缘对韩冈渴求一见。而当他到了古渭后,尽管在初见面时,有些不愉快的事,但随着与韩冈熟悉起来,两人的关系也渐渐好转。

只是智缘没有想到,真正遇到大事后,韩冈却暴露了见小利而忘命、干大事而惜身的真面目。局势已经恶劣到了这步田地,他却连作班超的觉悟还没有。空负着偌大的名头,到最后还是只能跟着西贼说些不着边际的闲话。

在另一侧,瞎药也在望着场中开始寒暄起来的韩冈和野利征,这也是他第一次看到宋夏两国官员见面的场景。

瞎药过去曾经在他的兄长那里,见识过该如何接待宋夏两国的使者。他虽然没有从中学到多少俞龙珂的圆滑手段,但瞎药明白到,无论在什么情况下,都不能让两家在自己的领地上正面相遇。只要不把事情当面戳破,就算风声吹得再响,来自两家敌国的使节,也会装作不知道对方的存在。

可是一旦双方面对面的接触后,就无法再装做对方不存在。近在眼前的现实,让瞎药只剩下了二选一的权力。他很清楚,别看来自宋国和夏国的两名大臣正仿佛多年老友一般,笑眯眯地说着漫无边际的废话,但等他们脱身出去,肯定转眼就会反手就砍上对方一刀。

不过不管智缘、瞎药,还有在场的近百人此时心中有着什么样的想法,是惊涛骇浪,还是水波不兴,都没有打扰到韩冈和野利征两人之间俗套的寒暄。

野利征当是读过一点诗书,跟韩冈说起话来,也是咬文嚼字:“韩兄少年英雄,名震关西。今日一见,却比传言更胜十分。”

韩冈摇头自谦,“虚名而已,其实难符,却让野利兄见笑了。”

“韩兄声名赫赫,怎能说成是虚名,就算在下在国中,也是时常听说过韩兄的才能手段。”

“野利兄谬赞了,韩冈愧不敢当。”韩冈谦虚不已,但脸上绽起的笑容,却好似已经把这些奉承话照单全收。他对野利征叹了口气,道:“在下与野利兄一见如故,只可惜仅有今日一面之缘,当真是遗憾啊……”

韩冈的话听在耳中,满是示好之意。野利征心底暗嘲其名过其实,口中却轻松的笑道:“若是两家言和,罢兵收手,当能与韩兄把酒言欢。”

韩冈仰天摇头,长声而叹,“一别之后,难有再会之日,把酒言欢,惜为井中水月。野心不收,战事难止。也只有等到明年今日,野利兄的坟头上,韩冈再以美酒相赠了。”

叹息声中,韩冈右手一动,呛啷一声响,腰间长刀已然出鞘。野利征还没有从韩冈的话中反应过来,只见韩冈振臂急挥,一道弧光便闪过他的颈项间。

先是一条细细的红痕,渗出了一滴血珠,下一个瞬间,红痕扩大为裂缝,鲜红的血液从创口处喷薄而出。

一刀将野利征的脖子砍去了一半,韩冈轻捷的连退数步,就这么乘势回到了自己的护卫中间,把喷泉般狂涌而出的血水全都避让开去,不让青色外袍沾上半点血迹。

从拔刀,到横斩,再到退回,韩冈一连串的动作如行云流水,没有丝毫滞碍。可见他这并不是头脑发热的行动,而是经过深思熟虑,考虑了每一个动作的细节,才能做得如此顺畅无比。

回到人群之中,韩冈对目瞪口呆的智缘又叹了口气:“我就是个急脾气,果然还是学不来班定远的本事,怎么都等不到夜里……”

智缘张了张嘴,不知该说什么才好。韩冈翻脸胜过翻书,前面还称兄道弟,现在就只能听到野利征簌簌的血液喷射声。

场中静如寒夜。周边一圈近百人都愣在了那里,眼睁睁的看着野利征就这么站着死去,震惊于韩冈下手之狠绝。

惊愕欲绝的表情被凝固在脸上,野利征身子僵直,任由浑身的血液一波波的从创口处喷出。在被韩冈切断了大动脉,失去血液供给的一瞬间,他就已经丧失了意识,只是不知为何还没有倒下去,但随着喷涌出来的血液越来越少,他的生命气息已经渐渐消逝。

“瞎药!你还等什么?!”韩冈一声暴喝,击碎了死域般的寂静。

瞎药闻声浑身一颤,视线从野利征脖子上的创口挪到韩冈脸上。瞪着他的双眼中,满是森森寒意,如风刀霜剑深藏其间。虽然瞎药一向桀骜不驯,可他眼下被韩冈这么一瞪,却腾不起半点反抗之心。韩冈的一刀,已经斩断了他的一条前路,他只能沿着剩下的一条路继续走下去,没有别的选择。

回过神来的瞎药,抬手指着野利征的护卫,用足了气力狂吼道:“杀了这群党项狗!

片刻之后,十余具尸首堆在院外,韩冈被请进了主厅中,高高居于上首,而瞎药跪伏在了地上,向他请罪。

等着瞎药一番磕头认错,韩冈终于摇头,“巡检何罪之有?党项人贼心不死,意欲遣细作说服巡检作反。巡检忠心耿耿,不为所动,将其尽斩。这些都是巡检的功劳,”

瞎药愣了,抬头上望。却见韩冈正俯视着他,一双眸子幽深难测:“难道我说错了吗?”

瞎药干咽了口唾沫,韩冈幽暗的眼神,摄人心魄,让他心惊胆战。现在被这双眸子盯上,青唐部的这位大酋不敢有任何违抗。而且韩冈这的话分明是为他着想,瞎药也不会蠢到拒绝:“机宜说得是,事情正是如此。”

韩冈展颜笑了,“既然巡检对朝廷忠心耿耿,眼下渭源堡被困,巡检当是该有所表示才是。”

瞎药以额贴地:“只等机宜吩咐。”

一个时辰后,近千蕃骑冲出了瞎药所控制的几条谷地,蹄声隆隆作响,直奔西方而去。

