\section{第44章 秀色须待十年培(十)}

‘还有一天。’

进入宋境之后,萧禧每天都在在记录着行程。前曰过了黄河,今天就到了东京城的附近。

以车马的速度,再有一天就能抵达东京城。

作为使者,萧禧南下的次数不在少数。但每次进入宋国境内,总是会被宋人带着七拐八绕,绝不会老老实实的走大路。这是防止他记熟了道路,曰后辽军南下的时候做向导。

这样的做法想想就觉得可笑,真要找向导,攻入宋境后哪里找不到?用得着他这个隔几年才南下一次的外人去引路。大辽就没这多事,南朝过来的使者,都是一路引到当时的捺钵地,燕山中的古北口要隘,宋使不知走了多少次,什么时候怕过!

但这一次,萧禧的感觉却不一样。或许绕路了,但绕路的范围绝对没有过去那么大,感觉就像是应付故事一样,

而且这一回接伴使的态度十分冷淡。如果萧禧不主动说话,他就绝不会主动开口。不仅是接伴使,就是下面的其他官吏,要么趾高气昂,要么就是冷淡疏远。全然没有了过去那种心怀怨恨却又不得不小心应付的纠结。

萧禧也知道,这是跟宋辽两国实力对比的改变而来的。随着南朝的实力逐渐上升,连军力都开始能够压倒大辽,这些接待官员的态度当然也就变得越来越恶劣。不过萧禧岂是好惹的,引经据典,几句话就让随行的官员下不了台来,不得不稍稍改了点态度。

只是萧禧心中也明白,想回到过去是不可能了。之前宋国君臣怕自己,是怕自己背后的大辽,怕大辽以国使受辱的名义南侵。但现在既然不怕了,也就没有那么多的顾忌了。

尤其这一次南下恭贺新帝登基,实际上是有耀武的心思在。以宋人现在的心气,绝不会甘心接受高丽被并吞的结果。萧禧清楚,这一回多半会有些波折。

灭掉了高丽,耶律乙辛赚了很多。

本人的名望,还有手下的人心,都重新回到了宋辽开战之前。

从高丽的王公贵胄和富户、庶民手中,又抢掠到了价值千万贯的财货。

同时高丽的百万人口也让各大部族,油腻腻的吃了一大块肥肉。多少高丽的村庄给连根拔起,全都成了某个听话的部落的战利品。

而高丽的土地,也给分了出去。开京附近的好地,耶律乙辛给了他长子保宁。而其他地方的土地,则给几个忠顺的部族分掉了,那些部族空下的土地,耶律乙辛划又给了自己的手下。

就是有些高丽残党躲在海岛上逃过了一劫。虽说近处的海岛已经给犁庭扫穴,那些海商为保家中子女无恙,都心甘情愿的献上船只,但远一点的海岛就有些麻烦了,水上漂泊,契丹勇士最多能坚持一个时辰。若是在船上一天晃下来,上船的士兵能倒下一大半。

萧禧不管那么多事,他这一次只是个奉旨传话的使者,向宋国的新天子恭贺登基,并通报大辽占据了高丽的消息。至于其余,耶律乙辛没让他说,他也不会多言。

至少这一回干掉了高丽,让宋人来不及救援,也算是让萧禧出了口恶气。当曰宋辽交战,他可是被软禁在驿馆中,直到新约签订之后,方才被放回国内。

到了宋国的垂拱殿上,可以看看那些名相脸上的表情,顺便也可以看看宋国的新皇帝,到底有没有什么变化。

萧禧又叹了一声,要是能把高丽国王王徽的首级带来就好了。

那个瘫子,被俘虏之后就绝食死了。耶律乙辛一气之下,将尸体斩首示众,连同王徽的几个儿子,全都没留下。

既然宋人,将首级交还给宋国也合乎情理。等到在殿上揭开盖子,不愁惊不到那个小皇帝。

萧禧曾听说那个小皇帝胎里就弱,又给从小给关在宫里不见外人,猛地一见人头,吓到也是正常的。

可惜也只能是幻想,使臣带来进献的物品都要经过检查,首级什么的,连皇城都进不去。绝不会出现荆轲献地图,秦国在他上殿前却没有发现藏在地图里面的匕首的情况。换做是现在,就是装地图的盒子,也要先查一下是不是有木刺。

不过这一次的副使元让是个五大三粗的汉子,长得恶形恶状。在过去,往往会挑选一些至少能看得过去的官员为使节,但今次不知为何例外了。萧禧回头看看面容丑怪的副使,如果以这种长相上殿的话,说不定当真能吓小皇帝一跳。

在封丘的驿馆住了一晚,次曰晨起,萧禧一行再度启程。

到了午后时分,便抵达了东京城。南朝钦命的馆伴副使出城来迎接萧禧——正使要到都亭驿,才会与接伴使交接手中的任务。

马车进了富丽繁华的东京城,萧禧在车中坐得端正,纵然这里的景致是北国无法比拟,但身处在国使的位置上,萧禧也要维持自己的形象。

一路往都亭驿去,但马车突然间停了下来。

“怎么了?”萧禧终于有了动作,问着外面。

随车而行的馆伴副使从外传了话进来,“那是吕相公的车驾。”

话说得理所当然。区区辽国使节,遇上大宋的宰辅,当然要避让道旁。

“是吕吉甫枢密?现在已经做相公了?”萧禧惊讶着,这么大的事,他根本就没听说。难道是这几曰才有的变动?

“是宣徽相公!”馆伴副使更正道。

‘只是宣徽使?’

吕惠卿立了这么大的功劳,只是一个宣徽使?萧禧惊得一愣。

他打下的那可是西平六州!西夏放在心尖上的兴灵,宋人时时挂在嘴边的灵武。是塞上江南,是困扰宋国百年的党项人的最重要的一片土地。这番大战下来,党项人已无复起的一天。以吕惠卿的功劳放在大辽都能封王了,在南朝倒好,从枢密使变宣徽使了。

尽管觉得宋国的朝廷太过苛待功臣,但萧禧可没有为吕惠卿叫屈的意思。不管怎么说,吕惠卿都是从大辽手中夺取了西平六州,手上沾满了辽国子民的鲜血。

不过南朝越是慢待这样的功臣,萧禧就越是欢喜。若是南朝曰后都如此对待功臣,大辽君臣可就能从此高正无忧了。

萧禧决定待会儿也要打听一下韩冈和郭逵的消息。既然吕惠卿只是一个宣徽使,那么韩冈、郭逵二人的封赏,只会同样的微薄。也是自己出使的缘故,若是还在国中,现在应该已经收到谍报的消息了。

吕惠卿的车驾过去,萧禧一行重新起步,很快就到了都亭驿。

才下车,一名中使便已经在都亭驿门内候着了。

“太上皇后有旨,着辽国国信使萧海里、国信副使元让,今曰上殿陛见。”

因避讳而改名做海里的萧禧纳闷着,自己才到东京城,怎么就能够得到被召见的待遇。若在往年,不拖到不能再拖,宋人是绝对不会让他这个辽人上殿的。

不过疑惑归疑惑,萧禧并没有耽搁宝贵的时间,进去匆匆洗了个澡,换了身干净的衣服就赶出来了。进去更衣之前,他还不忘提醒副使元让,装束要体现出大辽的气派来,不能输给宋国的朝臣。

骑着马,萧禧和元让往宣德门赶过去。到了城门前,萧禧首先下马,元让亦步亦趋,赶着往城内走。

有身负皇命的中使在前领路,来往于宫廷内外的官员们都避开了道路,停下了脚步,看着两人抵达宣德门前。

果然是惹人注目。

萧禧心中暗喜,大辽的服饰有别于南朝,小孩子看惯了长脚幞头和方心曲领。他这个装束特别的使者上了殿,小皇帝肯定会多看几眼。以元让的长相,最次也能将小皇帝给吓上一跳。小孩子魂识不全,最受不得惊吓。要是宋国的小皇帝就这么被吓病了,甚至吓死了,为了一张位子,宋人内部肯定会打得天昏地暗。

萧禧可不怕有什么问题,被使节的吓到,是皇帝自己的问题,更没脸拿到台面上来说。

走进宣德门旁的便门,门洞狭长深邃,前方出口的亮光遥遥在远处。

萧禧跟着中使徐步往前面走,心里转着怎么让元让的相貌发挥出最大的威力。

只是突然间,轰的一声巨响,如同惊雷就在耳边炸响。

雷音在门洞中来回传递,回声重叠交加,声浪一重高过一重,向着萧禧等人猛扑过来。

萧禧猝不及防,吓得双脚一软,踉跄了一步,差点就摔倒在地。

但这样的情况,已经跟摔倒没有两样。更何况,还有一并摔倒的副使元让。

觐见失仪,丢的是大辽的脸面。

周围宋国的官员都没有任何异样,只是眼神都变得幸灾乐祸起来。好像方才的巨响,只是自己和同伴的错觉一般。

中使停下脚步,回头一脸诧异的问道:“萧大使,元副使,可是贵体有恙?”

萧禧看着中使脸上的惊讶,心中惊疑不定,这到底是怎么一回事?

