\section{第33章 为日觅月议乾坤(九)}

‘怎么办?’

当听到韩冈问题,向太后一时间头脑空空。

还没有人如此直接的问过她这方面的问题。

随着官家的长大,每一个人在说话时都更加小心,怕引来不必要的误会。

只是向太后不会自欺欺人,她知道,每个人都希望知道他的想法。包括她的儿子,包括刚刚被抬下去的朱氏,包括她身边的宫女、内侍,也包括站在眼前的一众宰辅,就是宰相,也不曾例外。

归政的时间,是等到官家大婚之后,还是依照很多人的希望,将权位一直控制到死为止。

两种选择,向太后过去都考虑过,但她始终没有一个明确的答案。

有时她想过,干脆等到天子大婚之后便撤帘,将亡夫交托的天下还给儿子,这样日子也可以轻松一点,还能留下一个不恋权位的好名声。

可有的时候,她又觉得那孩子实在不成器,明明聪明过人,却总办蠢事,自己真要撤帘归政,万一败坏了如今君臣相得的大好局面,可就辜负了将国事相托的先帝。

现在,官家的亲娘刚刚闹得宰辅离心,就连一贯冷静从容的韩冈都怒不可遏。有这样的生母,自身又缺乏自制力,如果就这么让他亲政,近十年的心血,难道要付之一炬?

两种想法一直在心中回旋不去,让她难以作出决定。

维持着得过且过的心思,向太后今天突然发现,如今就连韩冈都开始担心自己撤帘后会变成什么样的局面。

这可是与青史中任何一位贤相都毫不逊色的名臣,无论遇上什么风浪都可以倚之为干城不论是在先帝重病垂危的那一夜,还是在奸佞篡逆的那一天,韩冈都以他的冷静和勇敢将一切敌人扫平现在他却担心天子亲政后会败坏国事。

这都要失望到什么样的地步,才会这么做?

难道那孩子,当真已经不可救药了吗?

向太后不知怎么回答,她只能沉默着,沉默的等着臣子们给她一个可行的提议。

等不来向太后的回答,韩冈终于再次开口,却不是提议,“元丰四年,朝廷两税税入不到八千万贯石匹两,粮价因北虏入寇而激增。而元佑八年的朝廷两税税入,仅只钱绢两项便超过九千万,粮秣盈仓。一年新增八百万人口,米价反而一直维持稳定,此乃陛下之功。军事上,大理覆灭后,除北方契丹,西方黑汗,大宋周边再无一千乘之国,这同样是陛下之功。”

“是相公们的功劳。”向太后摇头,这不是她的功劳,而是韩冈等宰辅的功劳,她岂会贪人之功为己有。

韩冈欠身一礼:“是陛下能信用于臣等,君臣相得,和衷共济,方有了如今的局面。”

回想起这十年来,勤民听政、旰衣宵食的每个日夜,向太后油然点头,“的确如此。”

“但宫墙中人不知如今局势来之不易,亦不知陛下劳心劳力之苦,只知道以己身之尊,理当受天下供奉。多,不念其德;少,则怨声载道。稍有不遂意,便说天下皆为天子所有,取用亿万亦不为多。太妃如此想,天子又何能例外?若陛下就此撤帘,放任天子亲政,试问国事将如何?”

向太后默然良久,问道:“相公觉得该如何做才好?”

韩冈强硬的摇头,今天必须要向太后自己做出决断,“非是臣觉得当如何,而是陛下想要如何。”

向太后心中一阵委屈,韩冈实在是太咄咄逼人了。扭过头去,她不想作答。

等来了又一次的沉默,韩冈放声道,“陛下,吾辈出仕,为天下,非为君也;为万民,非为一姓也。”

熊本心中一凛,难道韩冈打算上表劝进?转眼望过去,张璪、曾孝宽等几位都是悚然动容。但转念一想,他又立刻否定了这个猜测,韩冈头脑坏了才会去劝太后做则天皇帝,这对气学一点好处都没有。

“若国势不可救,天子不可谏,臣退隐归家,独善其身不难也。但陛下身在宫中,可能独守其身?”

熊本松了口气,韩冈不是劝进,不过拿孟子的‘达则兼济天下,穷则独善其身’,继续要挟太后。

向太后怒上心头,“难道相公当真要吾一直守着这权同听政不成?”

韩冈拜倒于殿上:“太妃如此,天子如此,臣不敢以愚忠而乱天下、害万民。臣恳请陛下,为大宋、为天下,再操劳几年。待天子年岁稍长,明了人情是非,再还政不迟。”

这是宰辅们第一次公然声称要太后继续垂帘,而且是出自最惜羽毛的韩冈。

向太后眼圈红了,“相公……”

而就在韩冈领头下,宰辅们或先或后一个个拜倒,“臣等请陛下继续垂帘。”

章惇首相,最后一个表态,“天子年幼,德性尚薄,难承大任,臣请陛下勉为其难,继续听政,以待天子厚养其德。”

宰辅们先后表态,向太后终于意动了,但还是有几分犹疑,这毕竟是要夺取自己儿子的权柄,不免损害自己好不容易培养起来的名声,“先让吾考虑几日,官家还有一阵才大婚。”

韩冈先瞥了章惇一眼,道,“陛下,吕惠卿今日至京师,明日上殿,必以陛下撤帘、归政天子为事由,以期留于京中。即使臣等能等,吕惠卿也不会容陛下等到后日。”

……………………

“越来越热闹了。”

下车后的吕惠卿言辞淡淡,将心中的惊讶给掩盖了过去。

离着外城城门还有两里,却已经是人头涌涌。即使往站外望去,也是一片鳞次栉比。

在往昔,城外的虽有繁华不下城中的厢坊,但也只是局限在城市东西两侧有水运经过的地方。南薰门外,除了每隔几年天子率百官去京城郊祀,一般情况下,猪走得比人多。

可东京车站建成之后,才几年功夫,吕惠卿过去几十年积累的印象全都做了废。

而一起下车的吕家家眷,却无法掩饰自己的难以置信。几个出生在京城的仆婢,更是目瞪口呆。

京师的变化已经远远超过他们的想象。

吕惠卿从平民所用的站台一直打量到身后的候车棚,以及站台侧后方的一排商铺,轻哼了一声,“点石成金的好手段。”

京城的范围已经扩张到了数里外的外廓城,繁华的厢坊,并不限于外城以内,以及汴水两岸。

如今外廓城诸厢坊中,最为繁华的去处便是东京车站附近。即便是远在京师,吕惠卿也知道东京车站附近的地价房价涨到了什么样的价格。

原本只有一座破砖屋的穷夫妻,只因手上有了一张地契,转过年来,摇身一变,成了年入百贯的殷实人家。

原本家中不过三分地,只能靠半年种菜半年做工来糊口的老鳏夫,车站建成后不过三年,便有妻有妾有儿有女,只因为他把地改成了仓库,租出去旱涝保收。

这些还只是运气好,蹭到了好处的当地居民。还有好些消息灵通,又敢于下赌注的显贵们,更是在铁路站点刚刚确定的时候,便秘密购地,最后一个个发家致富。

车站带来的繁荣只是一小部分,更大的手笔还属外廓城。

还在长安的时候,吕惠卿曾听属僚说,修了几座棱堡倒算了,还特地用柳树、栅栏还有低矮的胸墙括起一条外廓城,劳而无功,空耗钱粮,到底是为了什么。那几座以重兵把守,又是重炮云集的堡垒,足以将任何来敌消灭在外廓城外,而有铁路穿过的边墙,却根本毫无阻拦的作用。

吕惠卿在京师有产业。所以他很清楚,只花了半个月的时间修起的外廓城,直接让城郭之内的地价凭空涨了一倍,地段好的产业,价格直追外城,吕惠卿也受益不少,京师地主,哪个不谢韩相公?

一排小店,就在车站之内。食肆、酒铺占了一半,还有一半,是买京师特产的小店。甚至有两家卖的是鸡零狗碎的小饰品,虽不值几个钱,看起来却很适合带回去送人。这些东西不起眼,可架不住车站中人如流水。

吕惠卿不是那种不食烟火的文官,工商都比寻常人要精通,粗粗一算,就大吃一惊,这样的一间小铺子,一个月下来,少说五六百贯的收入。

大部分店铺外面还摆着报纸。都是些小报,认识五百字,便能通读。最适合拿上车打发时间。

“真是大不一样了。”

吕惠卿第三次发出感慨,却是针对市井中越来越多的报刊书籍。过去只有措大才会随身带着书,现如今,却是许多出行的旅客都拿着份报纸,还能包着点东西。

天下各家,也就韩冈一个还想着有教无类。执政多年,识字人口渐多,纸张和印刷的成本大幅下降,贩夫走卒亦能读书看报。

尽管依然对韩冈不服气,但挫败感还是不免从心中滋生。

近十年来,周边点点滴滴的变化,泰半源自于韩冈。

不过这不代表他吕惠卿要认输投降,韩冈想要做圣人的念头,就是他最大的弱点。

无欲则刚,既有所求,哪能不束手束脚?
