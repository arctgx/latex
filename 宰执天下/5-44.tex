\section{第六章 千军齐发如奔洪(下)}

‘怎么还没回来?’

李宪在帐中来回踱着步子,脚步落下又重又快,尽是心浮气躁。

自从出兵以来,河东军还没有打过一仗,作为钉子挡在河东、鄜延两路中间的左厢神勇军司,也被种谔连根拔掉了。

两年多前,葭芦川一役,种谔和王舜臣已经让左厢神勇军司大伤元气,而这一回,当年曾经大败河东军,让种谔进筑罗兀之役功败垂成的罪魁祸首已不复存在。

西落的斜阳依然炽热,虽有帐篷阻隔阳光,但帐篷之中则热得跟蒸笼一般。

李宪和河东军一路过来,最大的敌人是头上的烈日,仅有的伤员基本上都是蛇虫造成的意外。

但现在李宪已经很难再继续前进了。离开出发地三百里后,后方的粮草供给只剩开始时的三分之一。

幸好在开战前为了提防辽人,又少带出来一万多人马。否则能不能走出三百里都是两说。

李宪眼下唯一拿得出手的战绩就是参与了交趾的灭国之战,章惇进了两府;燕达晋身三衙管军;韩冈要不是年龄问题,宰辅是少不了的,但眼下的龙图阁学士也不差了;李信在河北的定州路做钤辖,参与过这场战争的领军者,一个个都飞黄腾达。但那份功劳吃到现在,也差不多都吃空了。

李宪本有立功受赏的想法。可粮草的匮乏让他完全放弃了建功立业的打算,只求能安安稳稳的追上种谔的鄜延军。

昨天收到了种谔攻下夏州的消息,就算西贼坚壁清野,夏州城中也该有点粮草。李宪已经派人去联络种谔,鄜延路是主力,情况应该比河东这边要好一点——粮秣转运的线路好歹不用渡黄河。

兵无粮不行。手上缺乏粮草,一旦遇到西贼的铁鹞子骚扰,在毫无险阻的荒野上,全军崩溃都有可能。

李宪叹了一口气。

当年李宪在河湟、广西,看着韩冈提举军中转运时举重若轻。远出崇山峻岭之外,周围敌军环伺,数万大军的人吃马嚼一点却都不当一回事。现在才走多远,竟然就要饿肚子了。

如果后方的粮草还不能送上来,他就打算驱动麾下兵将强行军,一天百里,用两天时间赶到夏州。

从地理上说,河东路的兵马想要打到灵州城,最简单的方法就是向西横穿沙漠,走过去就是兴灵了。不过要在沙漠里走上六百里,纵然不饿死,也会渴死。尤其是头顶的太阳,不仅能让头盔热得能煎蛋,也能让脚底板在滚烫的沙子中烤熟。

“观察,回来了!回来了!”

一名李宪的亲信小校,突然跑了来,在帐外大呼小叫。

“訾虎回来了?!”

李宪闻声一下停住脚步,忙将人招进帐来。惊喜和轻松,他心中兼而有之。派去督促粮草的将校自然不可能是空着手回来,好歹也有万石粮秣,赶到夏州应该没问题了。

小校声音小了点:“……观察,是折可适回来了。”

李宪脸板了起来,在马扎上坐下,沉声道:“命他进来。”

进来通报的小校脸色更苦,嗓门又低了两分,“回观察,折可适遇袭受伤,是被抬回来的。”

“遇袭受伤?”李宪眼眉剔起,全身的汗毛一下都竖起来了。

折可适是他派去地斤泽堵党项人退路的,这只是个顺带的命令,以防万一而已。

地斤泽就那么大,当年能藏下迁贼麾下百余残兵,却藏不下数万大军,两个指挥的骑兵足以防止任何意外了,这本就是个跑腿的差事。

李宪自认为已经考虑得很周全了,哪里想到这么简单的任务,折可适竟然受伤而回。

“跟折可适去的人呢?地斤泽里到底出了什么事?”李宪尽量放缓了声音,这时候,万万不能乱了阵脚。

“观察。听说折可适出事了?”

“观察,是不是西贼派人来偷袭了?”

几名得到消息的将领都匆匆赶来。

“慌什么!”李宪呵斥了一声,“等问清楚来龙去脉再说!”

李宪这些日子也利用各种手段,在军中立下了几分声威,河东军的将校不敢再多话,静下来等着进一步的消息。片刻之后,李宪的副将高永能,就领着折可适出行的副手,一起到了帐中。

折可适的副手同样姓折,是折家的子弟——折可适所带去的两个骑兵指挥,其中一个就是折家的精锐。

“人是清醒的。就是胳膊和大腿上被划了两下,只是皮肉伤,没伤到脏腑。”高永能已经去随军疗养院转了一圈,看过了折可适和他麾下骑兵的伤势,“他的肩甲上,留着铁锏的记号,被敲得反折过来。还有胸甲背甲,上面都有好几处箭痕。幸好来得及着甲,否则肯定回不来了。”

“全军伤亡如何?”李宪紧跟着问道。

高永能低头答话:“折了七十多人,回来的有一半带着轻重伤。”

李宪的眉头皱得更厉害。阵亡了一成,加上受了重伤的也为数不少,这两个骑兵指挥一时间都失去了战斗力。而能让八百骑兵吃了这么大一个亏,对手的规模不会小——当然也不会太大,否则折可适也就回不来了。

“到底是在哪里遇敌的?贼人有多少?打得是什么旗号?速给本帅细细道来。”

“是在受命出发的第三天,离地斤泽快四十里的地方。当时由于快到地头了,天色又是将晚,都想着早一步赶到地斤泽。却没想到突然就遇上了贼军。幸好是放在外面的探马先期撞上,让我等有换马着甲的时间,否则就情况就不会向现在这样了。不过贼军有两千骑,折承制见敌众我寡,加之贼人又是养精蓄锐,利于久战。便身先士卒,率我等反冲敌阵,一番鏖战之下,贼军远遁,而官军也折损不小,折承制都受了伤,只能退了回来。”

折家的这个军官说得前后条理分明,但显然就有人不相信,“这不可能,地斤泽才多大,囤积不下一千兵马!”

“若只是一千铁鹞子,官军八百甲骑,绝不会连主将都是受伤。”折家军官反驳道。

“将种不是疏忽了嘛……”有人嘲笑道。

“你!”曾经被郭逵称赞为将种的折可适,显然在折家很受看重。折家的这位军官登时就义愤填膺,眼睛瞪了两下,却又转成了冷笑,“我家承制再是疏忽,好歹还能挣下换马着甲的时间,可不会在葭芦川连盔缨都丢了!”

熙宁四年,鄜延路进筑罗兀城,河东路派出去配合筑堡,希望将防线向北推进百里,并将河东、鄜延两路联系起来的行动,却因河东军在葭芦川被伏击而宣告破灭,最后此役以失败告终,便是肇因于此。

如今在帐中的一众将校,倒有一多半经历过当年的惨败。丢盔弃甲的经历,至今还铭刻在心。听着折家人的讽刺,一个个脸色就难看起来。

“党项人藏兵的地点不只是地斤泽。”高永能出言缓和,“地斤泽左近,绿洲也有三五处,不是绿洲的沙中草场、灌木,则数目更多。”他顿了一顿,“挤一挤的话,两千人马没问题。”

高永能发话,帐中众将校都不敢再议论,只能等着主帅李宪的训示。

“兵多兵少其实无关紧要,关键的是,沙漠中的确有贼军。”李宪笑了一下,“想来不会有人认为这一支贼军是学着李继迁在沙漠中躲避官军,等待日后复兴西夏的吧?”

几名将校附和的笑了几声,就听高永能道:“这当是西贼用以乱我粮道的奇兵。”

李宪点点头:“当也不会有其他作用。”

没有哪位将帅会一门心思的在城头上等着敌军过来决战。即便西夏的太后、宰相和一应重臣,都将反败为胜的希望放在了灵州,但用来威胁宋军后路的奇兵却绝不会少。以正合,以奇胜,这才是兵法正道。

以沙漠中水草的数量,党项人能藏在其间的兵马很有限。但就像之前高永能所说的,几个绿

洲加起来,也差不多能有两千骑。用对了地方,在千军万马的战场上足以扭转战局,放在后方骚扰粮道,也能让十万大军的主帅难以安寝。

“骑兵来去如风,想拦住他们可不容易。”一名中年的将校提醒道。

“所以我们去安庆泽【今乌审旗】!”

安庆泽正处在沙漠之南,夏州之北,从名字看就知道是一水草丰茂的地方。

不用李宪多解释,众将都能明白去安庆泽道理。

长途奔袭和长时间的骚扰对战马脚力的消耗都很大,都需要水草优良的地方落脚,否则也就出战一次两次,接下来就没用了。在荒漠之上,适合骑兵的落脚地也就那么几处,安庆泽是其中最大的一处。守住安庆泽,再设法用粪尿或是毒药毁弃其他几处,这一支铁鹞子也坚持不了多久。

李宪环顾众将,“如果沙漠中的西贼南下,骚扰我官军粮道,我堂堂河东王师,就在安庆泽堵住他!”

