\section{第29章 顿尘回首望天阙(13)}

【对不住了,今天就一章。】

关西大战在即,而京城中却被争风吃醋的绯闻闹得沸反盈天。韩冈、周南还有自己弟弟之间的纠葛,赵顼本是当作趣闻轶事在听。但这两天事情闹得越来越大,士大夫中甚至开始有了指责赵颢的声音——要知道,赵颢无事去逛教坊司,之前都是被朝官们视而不见的。

这让赵顼心中有些烦闷。因为他很清楚,再过不久,御史台就要蹦出来说话了,然后朝堂就是一片乱,各派借机攻击政敌——要引经据典的将毫无关系的两件事拉扯在一起,正是文人的特长。而赵顼真正在意的横山战事,反而没人去在意了。

从御桌桌面堆得老高的文山顶上,赵顼拿过一本奏章。先看了看姓名,是中书章惇的文字。王安石手下的得力之人,赵顼想着,这人该是能说些正事。可他展开了只看了两眼,脸上怒容顿起,甩手就把章惇的奏章丢飞了出去。忙得今日轮值而随侍在殿上的王中正,蹑手蹑脚的跑过去把奏章捡回来。

“乱来!”赵顼很少发火骂人,现在的语气已经够重了。

章惇竟是奏请他下旨将周南赐给韩冈,以息众论。‘还嫌不够乱吗?!’赵顼也不笨,一旦他照着章惇的话来做,可就是变成他亲自出面,证实韩冈和赵颢的争风吃醋是确有其事。

姑且不论这样做,必然会让朝臣对赵颢群起而攻,根本做不到息事宁人。那章惇他可是中书五房检正公事,正事不理,反而在这等事上做文章,政事堂中的公事有这么清闲吗?

但气了一阵,赵顼忽然觉得有哪里不对,他平素里见章惇的时候虽少,其少年时的无行之举也听说过,但不论从他自己的观察,还是他人的评论中,章惇都绝不是如此愚蠢之辈。

赵顼冲王中正伸手示意,让他把章惇的奏章再拿过来。重新从头到尾细细一读,顿时恍然。章惇写得实在有些隐晦,但分明是撺掇着赵顼,趁着如今的大好时机,把他的两个弟弟请出宫中。

章惇的提议,让赵顼心中五味杂陈。他对自己的弟妹还是很有感情的。他出生时,父亲赵曙也不过是个郡王家的第十三个儿子,不能继承亲王的封爵,而继承皇位更是遥不可及。因而他赵顼也只是普通的宗室子弟,一母同胞的几个兄妹,一起读书、游戏,与普通的平民没有两样。直到赵顼过了十岁之后,才开始渐渐有传言说,仁宗皇帝要立他的爹爹为皇储,从那时起,他才被人看重起来。

如今赵顼由偏远宗室成为了天子,情况已不同于以往。幼年时的情谊仍在,兄弟姊妹之间关系还是不差。可是作为皇帝,赵顼对自己皇位的看重,也是天然存在。

赵顼到现在还没有儿子,而两个弟弟就住在宫中。从好处想,万一他有个三长两短,空出来的位子立马就有人能填补上,不会坏了国事。但往坏处想呢?未必没有人惦记他现在统御亿兆万民的权柄。

两个弟弟,就在背后紧紧逼着,让赵顼有时候都觉得背心发凉。尤其是最得母亲疼爱的二弟,赵顼更是心中暗带了几分提防。

当初章辟光上书说,两名皇弟已经成年,理应建邸出宫。当这番话传入宫中后,四弟赵覠当即就请求离宫,但二弟赵颢却没有说过半句。而接下来就是母亲大怒,逼着他将章辟光贬到偏远小郡去做官。

二弟的心思,赵顼隐隐的有些察觉。赵颢处在现在的位置上,离九五尊位只有一步之遥,有这个心思也不足为奇。

但赵顼现在拿着章惇的奏疏,想了又想,终于还是叹了口气,再次丢到了一边去。再怎么说都是自己的亲弟弟,赵顼还是不想做得太过分。

“官家!”王中正叫了赵顼一声,“陈衍求见。”

赵顼放弃了拿取新的奏章,道:“……让他进来。”

高太后身边的亲信内侍陈衍闻声便进了殿中。

等陈衍行过礼后,赵顼便问道:“太后有何吩咐?”

“太后请官家不要太过操劳,保重御体。另外,若是官家有闲,还请至保慈宫一行。”

陈衍的转述,当不是自己母亲的原话,天下重孝,母亲对儿子也用不着说请,再生疏也是一样。

赵顼的确是与他的生母有些疏离,反倒是跟他的名义上的祖母感情不差。当初过继来的英宗皇帝为了追赠生父濮王,而跟要维护仁宗地位的曹太皇针锋相对,朝堂上分裂成两派互相攻击,几乎闹到要废立天子的地步。那时就是时任颖王的赵顼到曹太皇面前晨昏定省,弥合两边的关系。

而赵顼登基后,曾经有一次身穿金甲,跑到曹太皇那里,问自己穿这套甲胄好不好。只看他去问太皇太后,而不是到自己母亲那里去展示,就可见赵顼心中的亲疏关系。

不过一点疏离感,并没有影响到赵顼对母亲的孝心。随即放下手上国事,由陈衍、王中正一起陪同,前往高太后所居的保慈宫。

不同于赵顼理事的崇政殿的老旧,去年刚刚修起的保慈宫,无论外墙内壁,上瓦下梁,皆是簇新光鲜。赵顼自登基以来,只为曹太皇、高太后两人分别修造了庆寿宫和保慈宫,而自奉甚简,并没有整修自己所使用的宫室。

进了殿中,赵顼就看见他的二弟赵颢,陪在自己的母亲身边。兄弟两人相貌有五六分相似,都可算是俊秀。就是赵顼稍显瘦弱,而赵颢则是身体强健了的一点。而两兄弟在轮廓和五官上,也都能看到高太后的影子。

对高太后行过礼,赵顼起身问道:“娘娘,唤臣过来,可有甚事?”

对儿子,高太后没必要绕着圈子说话,就是算儿子是皇帝也一样。“听说最近外面有些传言涉及天家,是不是有此事?”

赵顼有些不快的瞥了赵颢一眼,‘已经告了状了吗?’

随即点了点头,“是有此事。不过是市井谣言而已,日久自散。”

高太后不让儿子这么容易脱身:“听说已有人。王安石多用新进,祸乱朝纲。想那韩冈考中进士才三几年,才做官没多久,仅仅是个选人,便沉溺女色之中,还闹得京城内外乱起。”

高太后说得几乎没一句对,赵顼也知道,宫中的传言要有三分准头就了不得了。但她对韩冈的不满却清楚明白的传递出来。

赵顼对韩冈本就觉得有些亏欠,又看重他的才能,却是要保着他:“韩冈实有大功于国,周南节烈也甚得人敬,如今并非二人之过,难以论罪。士论也尽数偏向两人,若是将之惩办,反而会伤了二哥的名声。”

“那就任由外面传言败坏二哥的名声?!”

‘亲王而已,在乎什么名声?换作别人,自污还来不及。’赵顼腹诽不已。但他知道自己母亲的脾气。硬起来的时候,连亲手将她抚养长大的太皇太后都不搭理。自己若是不能让其满意,可是有得头疼。光是为了坚持新法,就已经闹得母子不快,现在再驳了她的面子,日后肯定会更麻烦。

赵顼又看了看自己的弟弟,“不知二哥想要如何处置韩冈、周南?”

赵颢低头:“全凭大哥处断!”

叹了口气,赵顼眼神冷了下来。真要全凭他的处断,偏偏到到这边来告御状,难道他的崇政殿会不见客。“既然娘娘要保住二哥的名声,臣便下旨将周南赐予韩冈。安抚下士论,好还二哥清白。二哥,你的看法如何?”

“……”赵颢沉默了一阵,无奈的点了点头。赵顼的处理结果不能让人满意,但赵颢的名声是第一重要的。就算不甘心,也只能相信赵顼的处理结果。

而半日后,赵顼的处断传到了庆寿宫中。大宋朝的太皇太后听了后,却让人摸不着头脑的念了一句佛:“阿弥陀佛,也该出去了。”

……………………

口舌之过,算不得大罪,最重也不过杖二十。而且甘穆也不是污蔑宗亲,说的都是实话。正常时候,一般人都会一笑了之,不过遇到现在的情况,却会让人不由得注意了起来。

眼下,苏轼在明知前面错判了周南申状的情况下,并没有穷治甘穆之罪,好用这等手段来表示自己并不是畏惧雍王的权势。仅仅是斥责了两句,便将甘穆放了出来。

从周南口中听说了此事,苏轼能秉公直断,不受他事干扰,倒让韩冈更正了一点对他的初步印象。但士林中对苏轼的评判越来越严苛。韩冈也终于知道什么叫文人相轻。苏轼虽然名声广布,但得罪的人可真是不少!

“韩官人!韩官人!”一叠声的叫喊和奔跑声由远及近,一名驿卒气喘吁吁的冲进韩冈的小院,“天使……宫中派天使来了,说是传天子口谕,要官人你快点去接旨!”

“果然来了!”韩冈微笑。

而周南则紧张得攥紧了拳头,问着韩冈,“官人,他们真的是为了脱籍而来?”

“放心!”韩冈大步出门,向前院走去。

