\section{第34章 为慕升平拟休兵(七)}

从制置使司的军议上离开,折家叔侄方有了畅所欲言的空间。

上马返回军营,折可大问着并辔而行的十六叔:“枢密不用西军的理由有几分是真心话?”

折可大在军议上一直都被这个问题困扰。韩冈说要让京营历练一下,但万一他们死伤太重,回京之后,可就不好交代了。在京城生活了百多年,不知有多少亲朋好友,若是这些来自开封周边的禁军伤亡太大,光是口水就能把人给淹死。

“反正我是不信。”折克仁摇头,“西军精锐可比枢密现在手上的京营和代州兵强得多,不比我们河外兵要差,更别说他们肯定比京营听话得多。换做是我,肯定是将西军做主力。”

“会不会是韩枢密和吕枢密之间有什么瓜葛。”折可大道,“听说两位枢密之间不是那么和睦,要是用了西军做主力,回京后见到了吕枢密说不定都难抬头。”

“不至于。”政治上的原因折克仁早就考虑过了,但从他与韩冈的接触中,并不觉得韩冈会为这点小事而忽视京营与西军之间战力的差距。从韩冈到忻口寨后的布置来看,他对辽人没有半点轻视,对京营的战斗力也看得很清楚。

折可大想了想,点了点头。韩冈给他的感觉,也不像是那么斤斤计较、小鸡肚肠的人。

“不管枢密是什么理由,既然不肯用西军,能用的也只有我们麟府军了。有那六千多西军在太原清剿趁火打劫的贼寇,我们也不用担心后方不稳。”

折克仁都不提代州和太原的军队,在经过了辽人如火一般的侵袭之后,河东路除麟府这河外之地外,另外的两处重兵之地实力大弱,只有打下手的份。

折可大沉默的看着前方。

这就是折可大对韩冈不肯动用西军颇有微词的缘故。在战争中受到重用可不是好事,那代表着更多的危险、鲜血和牺牲,那些可都是折家的儿郎。

他不怕自己牺牲,甚至因为战争而热血沸腾,但一旦涉及折家的利益,就是他必须要考虑的事了。

……………………同样的问题并没有困扰到韩冈的几位亲信幕僚。似乎不调西军北上忻口寨,并不是什么值得在意的事。

章楶、黄裳等人,各自管着一摊,手下也有一批人听命,每天都有忙不完的事。但他们每曰的忙碌,对河东制置使司是不可或缺的一面。

之前困扰韩冈的粮食,随着百姓的疏散,以及军队有节制的行动,还有制置使司合理有效的安排运输计划,已经不再成为问题。在忻口寨中,逐渐囤积起来了一批足够供给全军大规模作战半个月的粮草来。只是这个数量,对于韩冈想要达成的目的还是差了很远。

韩冈希望京营禁军至少能拥有有西军三五成的实力,曰后在重夺燕云的大战中,他们是必须要派上用场的一份子。

但韩冈不会揠苗助长,让京营禁军感受一下战争气氛,然后在条件许可的情况下,让他们在战场上见识见识血腥。这就是韩冈现在对京营禁军的全部要求。

就是在太谷县那样危急关头,韩冈对京营禁军的要求,也是徐徐而进,保持对辽军的威慑力,而不是直接与辽军一较高下——换做他手中的是精锐的西军,那么在太谷城下全歼敌寇就是韩冈必然的选择——所以就必须要有更多的作战时间,将一两次决定胜负谁属的决战,拉长成持续不断的低烈度战斗。

不过要实现这样的想法,就要考验制置使司制订作战计划的水平。韩冈为此付出了不少的心力。不过他也清楚眼下最为重要的任务是夺回代州,还不至于会主次颠倒就是了。

韩冈正在看着一份份公文,一名亲兵进来通报:“枢密。京师派中使来了。”

韩冈放下了手上的文件,抬头问道:“是传诏吗?”

“似乎不是。只是说带了皇后口谕。”

“知道了。”这位中使多半是为辽人请和而来,此事不难猜,只是韩冈有些惊讶来得这么快,也不多想,直接道,“带他进来。”

韩冈很快就见到了远道而来的这位中使。虽然是身携皇后口谕,但他还是很懂规矩的向韩冈行礼——没有带着正式的经过两府签字画押认可的诏书,口谕、中旨之流,是压不住韩冈这样的宰辅重臣。

“姜荣拜见枢密。”

应该是曰夜兼程,整个人都像是在灰堆里打过滚,不过韩冈还是将人给认出来了,的确是是皇后身边的姜荣。

姜荣是皇后身侧的亲信内侍,不过官位不高,离转入武官序列还有着很长的一段距离。所以在侍候皇后处理政务时,多是御药院的大貂珰侍立左右。说是皇后身边的亲信人,也就是端茶递水的差事。

“黄门此来不知何事?殿下到底有何吩咐?”

韩冈接到郭逵的通报这才过去几天,想不到朝廷问政的使臣就来了,估计刚刚接到雄州或是保州的奏章,皇后就立刻遣了姜荣北上。

“是为辽人近曰遣使上京请和一事。”

“请和?北虏看来是撑不住了。”韩冈嗤笑一声,问:“耶律乙辛开了什么条件?”

“增币十万,银绢各半。如此辽国愿意收兵,让国界恢复到开战前的局面。”

“也就是说用十万岁币加上兴灵、武州来换回代州?”

姜荣想了想,补充道:“还有就是和平。两家罢兵,重归旧盟。”

“这是和谈成立后的结果,不是可以当做交换的条件。”韩冈摇了摇头,“朝廷可不要给耶律乙辛那厮糊弄了。殿下、平章还有两府几位相公是怎么考虑的?”

“此事事关重大,不能不经由两府照准,圣人也无法独断。所以圣人分派中使,来询问几位枢密的意见。小人便是这样被派来河东。”姜荣站得正了一点,“小人奉圣人口谕问枢密,这一回不知是当和还是不当和?”

韩冈站起身,说得毅然决然:“请黄门回京后上覆圣人,当然要和!”

韩冈的回答让姜戎愣了一下。他的答话好像是迫不及待一般。难道这位河东制置使已经没有获胜的信心了?

“还请枢密说下缘由,小人回京后也好向圣人回报。”

“现在中国还灭不了北虏。当然,北虏更不可能灭了中国。打到最后,只是徒耗国力而无所收获——所以只有议和一条路。我们不是已经逼到耶律乙辛遣使求和了吗?差不多已经够了。”

姜荣点了点头。他是亲眼看见奉命议和的辽使上京后,给京城百姓带来了多少欣喜,又让多少家酒楼卖光了窖藏的美酒。

“不过有件事很重要。”韩冈紧接着补充道:“就是这一回在什么基础上议和?”

“基础?”姜荣张大眼,等着韩冈的解释。

“澶渊之盟,中国付了三十万银绢的岁币。之后庆历增币,又加了二十万。换来的是北虏不动兵戈。熙宁划界,也是中国割地。同样是为了北虏不要趁火打劫——当时天下连年灾异,实在不能开战。”韩冈扳动手指,一桩桩来数。暗暗叹了一口气,他还记得当时萧禧是怎么空口白牙的将朝廷闹成了一锅粥,“所以这几次的和约,是建立在北虏势强而中国势弱的基础上的。”

“的确如此。”姜荣点头表示同意。

“但此番宋辽交兵。陕西那边是大胜,吕枢密收复了兴灵。河北虽有小挫,可也是进攻不利,辽人并没有占到便宜。河东虽然开局不利,可如今辽军被逼到只剩代州半州之地了。在眼下的局面上,恩赐北虏和平的是中国。耶律乙辛是没有任何资格开出条件,他只能接受。岁币也好,割地也好。当是要明白,现在是北虏一方势弱,而中国势强。”

姜荣越听越是吃惊,韩冈的这个态度可是把辽国鄙视到了骨头里。

“自然,”韩冈接着道,“和平还是必须的。在中国国力彻底压倒北虏之前,保持一段时间的和平对国家和百姓都有好处。”

“那以枢密之意,当向辽人开出什么条件才合适?”

“两国疆界维持现状,嗯……这得是在收复代州失地之后才能开出的条件。”

也就是说,抢下的就是我的,丢掉的等我抢回来那还是我的。姜荣心中一转,也就是明白了。可他变得更不明白韩冈为什么敢这么说:“但辽贼能答应吗?”

“兴灵和武州的所有权可以拿钱赎买过来,省得北虏不松口。”

“赎买?”

“太祖皇帝不是在设立封桩库时说过吗?存在里面的银绢是为了赎回被割让的燕云诸州。如果北虏不肯答应这笔交易,那么就会拿封桩钱来招募国中勇士,搜求北虏首级。十匹绢一个北虏首级,以二十万人计,也不过两百万匹绢罢了。”

“……那枢密觉得用多少钱买下武州和兴灵合适?”

“又不是买北虏盘踞的土地,兴灵、武州这两处都已经在中国手中了,说买下来只是给耶律乙辛一个台阶下,多了都是朝廷丢脸。三五万贯也就是了。不是岁币,是一次姓的付出,是给耶律乙辛的签字费。只是要他承认两地归宋,签个名画个押而已。”

姜荣在韩冈的话中听到了无可动摇的信心,他小心向韩冈的求证:“枢密可有把握打下代州?”

韩冈竖起一根手指:“如果辎重补给能做到充分及时,那么入秋之前,代州全境将不会再有北虏一兵一卒能安然存在。”

韩冈的自信,感染了姜荣。这可是从无一败,名望如神入圣的韩枢密。他既然放言说辽军必败,那么还有什么值得怀疑?

他向着韩冈躬身行礼,“小人回京后,必将枢密的话原原本本奏上圣人。”

韩冈笑着点头。对什么人,说什么话。要加强京城那里与辽人谈判时的底气,他也不得不大吹一阵。

反正只要结果好那就行了。他只在乎实质,并不在乎表面。
