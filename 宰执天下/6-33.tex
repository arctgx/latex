\section{第五章 冥冥冬云幸开霁(12)}

‘蔡京?!’

听到这个名字,王厚的心脏就猛的一跳。

如雷贯耳啊。

王厚眼神陡然转利,盯着蔡京上下打量了起来。

年纪不轻了,看起来气色倒好,也不知是不是人逢喜事精神爽的缘故。从回望过来的眼神中,看得出来是行事果决之辈。

区区一个台官,就将宰辅逼得不得不赌咒发誓。纵然许多寄希望于韩冈的关西士子,对他恨之入骨,但也不能不承认,蔡京的确有能耐,做到了文彦博、王安石都做不到的事。

韩冈如此失态,王厚都从来没听过,更没见过。反过来的情况,倒是知道不少。

不过蔡京本人也算是毁了,在韩冈的全力反扑之下,任谁也不可能全身而退。

只是韩冈不能升任宰相,仅仅换来了蔡京就此沉沦,这依然是桩亏本买卖。

如果有机会,能砍掉束缚在韩冈身上的枷锁,王厚无论如何都不会放过。

蔡京背后的两名伴当,一左一右紧紧夹着蔡渭,正紧张的望着王厚。

王厚明白他们的心情。抓了宰相家的衙内,又是叛党的余孽,蔡京不知给他们许了多少空头愿。可人尚在手中,还没有交上去,貌似抢功的敌人就过来了。还领着十几名如狼似虎的班直禁卫。

但蔡京本人,双眼向左一瞥,向右一瞥,然后又回望了过来,不见一点畏惧。

王厚的牙立刻就咬了起来。

韩冈得势,对所有西军系统出身的将领都是一个好消息。对王厚更是天大的喜讯。自家的儿子还是韩冈家的女婿呢。岳父做了宰相,女婿当然水涨船高。

能将蔡京干掉,韩冈身上就再无束缚。

过去还要担心什么新莽,现如今两度扶危定难之功,哪个还能说上半句?

可现在不是地方!

王厚开始痛恨起京城的人烟稠密来。街上的行人人数虽不如往日,但数量依然不少,很多都在望着这一边,这么多双眼睛盯着,根本找不到下手的机会。

要是早来一步,在巷子里将蔡京堵上,他敢立刻就下手。

他往这边来,本就是为了找蔡京。

不管蔡确家有没有人投奔蔡京,他跟蔡确之间确实有着亲戚关系。只要一刀砍死了事,人死了,怎么栽赃都没问题。

这个节骨眼上,谁还敢为蔡确亲族叫屈?!

但众目睽睽之下,王厚纵有满腔杀意,也不方便在光天化日之下动手。

也许是看出了王厚心中的犹豫,蔡京嘴角多了一抹笑容。

“不知将军何人?”

“……德安王厚。赵颢、蔡确谋反,王厚奉诏讨贼,正是为蔡确党羽而来。”

听到王厚的自我介绍,蔡京脸色瞬息间变了一变,但王厚再定睛看过去时,却只能看见嘴角微扬的笑脸,之前的变化仿佛是一场错觉。

“蔡京见过上阁。”

蔡京冲着王厚行了半礼,他现在的官位虽在王厚之下,又郁郁不得志,但正牌子的进士,用不着对武官太过谦恭。

能一听王厚的姓名,便知道他还未就任的正式身份。对朝廷人事,蔡京显然还是十分的了解,也不知是从哪里得到的消息。

“蔡确父子狼子野心,竟然不顾朝廷深恩,悍然谋反,京与蔡确纵有缌麻之亲,也不敢与其同流合污。今日蔡确事败,就见此贼逃窜,故而将之绑了,过来投官。”

听到蔡京的对话,蔡渭猛地挣扎起来,但又为蔡京的伴当牢牢按住。

压着蔡渭的只有两人,可能是蔡京家仅存的家仆了。

看见区区三人的队伍,王厚杀心又起,将三人带着一起走,只要找到机会,怎么都能料理了这三人……不,四人。王厚可不会让蔡渭事后多嘴多舌。

王厚眯起眼笑着道,“能捉到叛贼蔡渭,自是大功一件。蔡京你带着蔡渭跟本将走一趟吧。若之后确认有功,朝廷自不会吝啬。”

拿着蔡渭、蔡京的首级,从太后手中换来的功劳,足以堵上随行的一众班直的嘴,填饱他们的肚子。

但王厚的笑容和言辞,在蔡京的眼中,明显的没有任何吸引人的地方。

“如此甚好。”蔡京点头,“蔡京正要将此贼械送皇城,惟恐贼党夺人。有上阁护卫,那是最好。”

蔡京的态度让王厚看得心头大怒,真把他当成护卫了?

去宣德门一路都是通衢大道,御街上更是人来人往,想下手当然不成。

不过……王厚看看左右,又丧气起来。

都是些还没有用顺手的班直禁卫,换作是在关西的亲兵,不用自己使眼色,就能围上去将几人一起绑了。自己一个命令,更是杀人放火都不在乎,完全不需要多解释。

可这些班直听到自己的命令后,能不能下手?下手后会不会让蔡京和他的仆从逃走一两个?更重要的,他们到底能不能不惊动外人的情况下,将蔡京蔡渭给擒住,弄去没人的地方下手?

“让两匹马给这两位壮士。”

听到王厚的吩咐,班直们先是一愣,然后互相交换了一阵颜色,才有两名最为年轻的班直下了马来。

王厚当真对这些班直越发的没有信心。

不用自己多说,他的亲卫们会主动将马让给蔡家的家丁,并将有坐骑的蔡京挟持住。

蔡京眉头微皱,显然是知道王厚的打算。

而王厚也在苦恼,怎么才能让手下人聪明一点。

“王上阁!”

“上阁!”

突然就听到后面有人叫,王厚闻声回头,只见一队骑兵从身后过来。

队伍中的两名官员,他认识其中一人,另一人就很陌生了,似乎见过,却没什么印象。

两人到了近前,便向王厚行礼,

“黄裳见过上阁。”

“章辟光见过上阁。”

王厚就在马上回了一礼,“奉旨讨逆,无暇礼数,还望勉仲勿怪。”

在上京后,王厚只见过黄裳一面,但韩冈两任河东时的第一助手,给王厚留下了很深的印象,也知道韩冈正在着力提携他。

向黄裳回了礼,他看向另一位官员,问道:“这位是……?”

“这一位是开封府判官……”

黄裳正在向王厚介绍章辟光,双眼却陡然瞪大,嘴也张得老大。

他在王厚的队伍中竟然看见了两个意想不到的人物。

“蔡京!蔡渭!”

章辟光和黄裳同时惊叫。

“正是蔡京。”

蔡京气度沉稳,向两人行礼,“蔡京与王上阁刚刚捉到了这名贼子,正打算送去皇城。”

黄裳和章辟光狐疑的望向王厚。

王厚立刻摇头,蔡京明显是想搅混水,可惜他可不贪这份功劳,“王厚是方才才看见这位蔡官人押着蔡渭出来,究竟有什么内情,王厚是一点不知。”

“哦……”章辟光拉长了声调,“不是跟着上阁一起的?”

蔡京脸色微变,但仍是镇定,高声道:“此贼走投无路,蹿奔到蔡京家中,但蔡京一贯只知忠心事主,便将此贼擒住,要送去见官。”

“谁知是真是假?”章辟光冷笑起来:“在我看来,倒是故作伪饰,护送此贼出城。”

“上阁,你能为此人作证?”

王厚摇头,“初相见,从未相交,如何为其作保?”

“吾乃开封府判章辟光,奉诏讨贼。”章辟光一指蔡京:“一并捉了。大府正在府衙等着呢。是功是罪,等大府审过之后,就知道了。”

下手竟比王厚还要果断干脆。

章辟光身边的几名士兵一下就扑了上去,横拖竖拽,将蔡京给扯下马来。

蔡京本来为了张扬自己的身份,保护自己能够顺利的将蔡渭送官邀赏,还特地穿了一身官袍。顺利的压住了王厚,却没提防章辟光根本就不管周围有多少人做见证。

章辟光眉飞色舞,神采飞扬。本来只是为了讨好韩冈的亲信,顺道送上一程,却不曾想天上掉下了大礼,蔡京、蔡渭两人一并捉了。

管那蔡京是不是叛逆,进了开封府,要什么口供没有?

也不要拷问,只是蔡渭,就绝不会放过蔡京。

到时候,顺水推舟的事,沈括会不干?

就是沈括不干,章辟光也是要干的。

王厚有几分紧张的看了看周围。

章辟光笑了一声,低声道:“怕什么?以蔡京的名声,谁会为他多说一句?”

黄裳扯着王厚的衣袖,“上阁你是不在京城,所以不知。蔡京当初陷害相公,京城百姓哪一个不是恨不得寝皮食肉?若是现在在大街上喊一声蔡京在此,包管有石头砖头砸过来。”

王厚听了,转头再仔细看蔡京。方才没觉得,但现在看他,脸色发青发白,其实还是害怕的。

“竟然给这贼人唬住了。”

他低声骂了一句,要不然何须等到无人处,直接就下手了。

蔡京从马上被揪下来,官帽被踢飞,连身上的官袍都给扯烂了,转眼便被五花大绑。想要大喊,肚子上立刻就挨了重重一脚,什么声音都出不来了。

身为一名叛贼的族亲,又是蔡渭投奔的对象。纵然有反戈一击的功劳,也不一定能得到朝廷的谅解。

蔡京的依仗,就是剩余的宰辅们想要留一个钳制韩冈的工具在朝中。王安石、韩绛之辈,不会看着韩冈就此逃出束缚,能够毫无顾忌的成为宰相、权臣,甚至新莽逆臣。

但这终究还是行险,是迫不得已的举动,终究还是要拿着性命来做赌注。

所以蔡京要将蔡渭光明正大的押送去皇城,如果世人都看见他将蔡渭送去皇城,就算韩冈想要下毒手,就算蔡渭想要反咬一口,宰辅们也会帮他蔡京渡过难关。

但他所没想到的,这条路竟然如此难行。

“一个都别放过。”章辟光高声叫了一声,让手下人将两名仆从也一并捉了,他们的口供正好可以将蔡京钉入死地。

又低声向王厚、黄裳道,“韩相公有经天纬地之才,又有匡济赵氏之功,早该进位宰相。可惜却为此贼所沮。我等要为韩相公分忧解难才是。”

章辟光什么时候投奔了韩冈?还是说看到现在的形势,向韩冈献上投名状?

从章辟光当年能第一个上书谏言,将两位亲王请出宫中,就知道此人善于投机。只是运气不好,撞到了一个护犊子的高太后。

只要没了蔡京,宰相一职对韩冈来说,就是探囊取物,再无半点可以顾忌。章辟光所献上的大礼,可谓是厚重无比。

