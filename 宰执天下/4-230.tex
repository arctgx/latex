\section{第38章 岂与群蚁争毫芒(三)}

韩冈奔波多日后,经过一番梳洗,感觉是神清气爽。天热得厉害,在太阳底下跑了一个多月,怎么也比不上在家里的舒心安适。

“方城山那里不用再看着吗?”周南问着。家里的妻妾都知道在襄汉漕运上,韩冈更关心哪一段,但她这么问,却是有着几分幽怨。

“不用担心。”韩冈对自己安排的人选信心十足,“这路是修一段、用一段。修好了一头一尾,就用马车沿着轨道将材料向中间运过去,运上一程,修上一程,再用上一程。为夫回来时,南北两端加起来修了有二十里了,上下都磨合好了,李诫做事也稳当,完全可以放心了。要不是天气暑热,两个月就能完工了……现在也就还差两个月,完工后去验收就够了,还是正经事要紧。”

“什么正经事?”云娘好奇的问着。

“当然是陪着你们。”韩冈哈哈开着玩笑。

周南和云娘同时皱起鼻子,哼哼着表示不信,都做了母亲了,却还有少女般的娇痴。在另一边,王旖头仰着,一对眸子眨也不眨,专注的看着韩冈脸上的笑容。

严素心看了韩家主母一眼,抿着嘴笑了一下。

王旖是骄傲的,她有着胜人一筹的出身,有着让人敬仰的父亲。不论王安石的评价在世间有了什么样的变化,但从品德、到声望,再到才学,都是大宋百年来首屈一指的人物。胸中怀有天下,以世间苍生为念,十年不到,便改变了这个国家。熙河、荆南、广西、西南、横山,军事上的节节胜利,根源都在王安石的身上,韩冈的功劳都是建筑在新法带来的变化上的,可谓是因人成事。

严素心过去每次听王旖说起王安石,都能看得出王旖对自己父亲的崇拜。韩冈虽然在年轻一辈中已经可以算是当世第一人,王旖也是亲眼看着韩冈于白马县救了几十万的流民,在熙河、在广西,更是军功赫赫,文治武略皆有所长,但在她的心中总是比王安石差了一截。

但最近从她们的丈夫那里稍稍透露出来的目标和愿景,却绝不下于王安石改变了整个国家的手笔。那并不是幻想,而是完全可以实现的现实。这些年来的事实证明,只要韩冈想做的,便一定能成功。

同样是胸怀天下,眼界目标绝不逊色分毫,绝不是庸浅俗吏可与之相提并论。加上在家里又体贴温柔,在外又从不耽于声色,有这样的丈夫,世间女子又哪有不愿倾心相许的?

自从南下京西之后,严素心看着王旖,发现她很明显的对韩冈的态度,在夫妻间的亲昵中,又多了几分崇敬。

韩冈眼下想做的事当然不可能全然瞒着枕边人,将终极目标隐而不露,只是能透露出来的一些事,就已经足够有震撼力了。虽然在学术上有别于王安石,但王旖对韩冈决心承袭张载遗志,并对自己所学加以推广,也并无二话。

襄汉漕运只要将人事安排好了,财务控制住了,完全可以放手。但韩冈现在要做的事,必须他亲自来掌控,能流传千古,同时也能将气学和格物之说发扬光大,这件事怎么可能放在他人手里去完成?

这件事做起来不难,也就是写起来太麻烦。韩冈的行囊里有着厚厚一叠稿纸。要想宣讲气学,就必须立文字,贴合上格物致知四个字,但要完成这项工程,韩冈推算着,至少还有一年半载。

吃过了饭,问过了儿女的功课,韩冈先进了书房中,离开襄州一个多月,公事就不必说了,私事上也有许多要处理。

烛火下,韩冈一封封翻着信。他在汝州、唐州、邓州到处跑,转运司的公文追着他都不方便,更不用说私信了,积了有好几十封。

来自父母的信,韩冈一向先看。二老一切都安好,家里的情况也一切都好,庄上粮食的收成很不错,就是想念儿孙。

冯从义也来了信,说了些顺丰行中的近况,无论陕西还是交州,都是在稳定的发展中。第一批白糖顺利出产,而棉布、菜油、蜂蜜之类特产,规模也在扩大。顺丰行与当地部族的联络十分密切,皮毛、药材之类的自不必说,甚至在叮当作响的铜钱引诱下,湟水和青海畔的部落将宗教上的忌讳丢到一边,都开始捕鱼了。来自于河湟的咸鱼和腌湟鱼卵在关西都很受欢迎,让当地的几个部族有此发家,贯彻了韩冈立足当地、开发特产、以利诱之的方略,在经济上成为一个稳定的附庸。

写给关学同门的信,全都有了回音。游师雄和种建中,对吕大临的行文愤怒异常。而苏昞、范育则是稳重了一点,给韩冈的信中就加以规劝,并说吕大临已经对行状修改过了,不复之前扬程贬张的说辞。

韩冈看着连连冷笑,吕大临在自己面前脾气甚硬,回过头来悔改的倒也不慢。要不是自己的名位已高,说不定吕大临这一手,还能落一个恶意诽谤的罪名——好吧,这个猜测,有点过于阴谋论了,吕大临或许并没有这么想。

不过程颐已经入关中去了,在气学缺乏核心的时候,不必吕大临在行状中做文章,许多弟子都很有可能转投程门。

而种建中的信里并不只是说张载的行状。更多的还是希望得到韩冈的支持——对他叔父种谔攻略西夏的支持。韩冈看了种家十九哥的信,摇头叹着,种谔还真是不消停。

不过以韩冈的看法,对西夏的战略应该是蚕食,而不是鲸吞,若是打算直取兴灵,七百里的瀚海对这个时代的任何一支军队,都是一个灾难。如果主力走兰州,那倒是不用穿越瀚海,而且熙河路这两年积蓄的库存,也能支撑三万到五万的大军出征。

但想来种家也不会同意,鄜延、环庆、泾原路攻打银夏吸引西夏的注意力,而秦凤、熙河的军队乘机夺占兴灵的战略——而且这同样要冒风险,需要翻山越岭的千里跃进,绝不是一次轻松愉快的行程,粮秣的来源大半得放在缴获上。

游师雄的看法与韩冈类似,现任的秦州通判觉得种家最近似乎太活跃了,甚至跟庆州知州起了龃龉,很有可能是准备对西夏开启战事。游师雄担心这一次很可能会因为将帅贪功而冒险激进。

对照种建中的信,游师雄的直觉自然没有错。庆州知州范纯仁责授知信阳军的公函,韩冈已经收到了,范仲淹的次子应该很快就会到京西来了。

除此之外,还有王安石的信。现在王安石已经辞了江宁知府的差事,做了一任类似于后世政协养老的宫观使,什么差事都没有。就住在修于城外谢公墩上的宅子里,离城七里、离山七里,号为半山园。每天不论风雨都跨驴去蒋山【钟山】,天晴上山,雨雪就在山脚下转一转,累了就随便找间小庙或是小店休息,日子过得悠闲自在。

在王安石的信上半点也不提政事。除了问候外,就只是说他最近在撰写《字说》,专注于训诂小学。此外还说了江南的风景好,信里附了好几首描写江南风土的诗词。大概也是看得出来,韩冈几年内没机会回到京城,言下之意是希望他能到南方做几年知州,也能顺便见见外孙和女儿。

王韶和章惇的信则很有趣。王韶在信中尽管说得豁达,但到了最后还是没忍住讽刺了几句章惇费尽心思、却为他人作嫁衣裳的愚行。而章惇的信中,则是隐晦的为自己分辩了一下,说王韶去职,并不关他的事,元绛做了参知政事,正好为他证明了清白。

孰是孰非韩冈是弄不清,但两边跟他都是关系密切,要选择站在哪一边都让人头疼。只能日后设法调解了。

剩下的信,比如王舜臣、赵隆他们的,基本上都是问候而已。倒是不见李信的来信,上一封还是三个月前收到的。

韩冈很快就把给父母和冯从义的回函写好了,打算明天让老大老二和大丫头去给祖父祖母写两句问候的话。又拿起苏昞的来信,正推敲着该如何措词,将自己的想法传达过去,就听见房门被轻轻敲了两声,严素心的声音随即在外面响了起来。

韩冈将信放了下,应了一声,严素心这位美厨娘端着只要韩冈在家便雷打不动的滋补药汤进来。

见到韩冈笔墨纸砚在桌案上铺了一摊,严素心嗔怪着:“回来后也不知先歇一歇,给爹娘的信先回了,其他隔两天写也不算晚。就知道忙,也没见其他人跟官人你一般辛苦。当初借住在王相公府里的时候,宰相都比官人你清闲。”

韩冈自嘲的笑道:“谁让为夫有私心呢,要心思都放在公事上,也就不需要这么辛苦了。”

王安石已经是看得开了,在京城不到十年,已经将他一辈子的心力都耗尽了,无心再谈政事,无萦于外物。但韩冈精神年龄虽与实际有所区别,但他的雄心壮志可不会输给任何人。许多事不必争,但有些事则必须争。

纷争都是官场上的,韩冈的目标甚至比王安石都要高,更不用说那些狗苟蝇营的官员,并不用放在心上。

但在学术上却是两样。比如王学,那是得到官方认可的学派,不去钻研,就别想考上进士。王安石可以高枕无忧,但韩冈则必须去为他的气学去鼓吹,去联络。在程颢已经的抵达关西开始讲学的时候,一刻也耽搁不得。

