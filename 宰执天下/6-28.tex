\section{第五章 冥冥冬云幸开霁(七)}

一声来自远方的爆鸣,模糊地传入耳中。

韩冈敏感的偏了偏头,那是火炮在轰鸣。

不过他立刻又端正了姿态。

崇政殿上,分心并不合适。

尤其是在讨论如何处置参与叛乱的内侍与禁卫,以及如何清算蔡曾薛三人党羽的时候。

“方才在殿上,臣等曾立誓只诛首恶,胁从不问,故而叛党犹豫,误从叛逆的班直也纷纷反正。非如此,臣等亦难见陛下。为朝廷信用计,还是只根究首恶为宜。”

“十恶之罪,不闻可赦!”御史中丞李定比起早间在殿上的时候,正气凛然了许多,“谋反一罪,十恶之首,此罪可赦,何罪不可赦?!”

“李中丞此言乃是正理,今日谋反之罪可赦,他日有人毁损皇陵,是论死还是赦除?”

“误事者入刑,贪渎者远流,朝廷自有律条在,纵重判亦无人敢怨。如今谋反之迹昭彰,却能蒙赦,日后依律定罪如何不招人怨?”

“律令,公信也。誓言,私信也。遵私信而弃公信,这是哪家的道理?”

“臣曾闻兵法有兵不厌诈一说。圣人亦曾云‘要盟,神不听’。诸公殿上立誓,乃是事急而为,如今事定,自当依律而行。”

一名名重臣出来反对遵从宰辅们之前的誓言,对蔡确、赵颢、石得一、宋用臣四名主犯之外的从犯进行赦除,或者宽待。

当庭发誓的是宰执,与李定和其余重臣无关。

在这个节骨眼上,敢于为叛贼说话,就等于招认自己就是叛贼的党羽。至少会戴上一顶同情叛逆的帽子。

除了当庭发誓的韩冈、王安石、韩绛等人,其余在场重臣,无不是要穷追猛打,将所有叛逆绳之于法。包括叛军在内,都要从上到下清洗一遍。

韩绛瞪着韩冈。

就是韩冈弄出来的事,两府宰执议论了将事情定下来,不就了结了?之后谁还敢当庭再驳回来!

也就是韩冈,偏偏将朝中的金紫重臣一起都拉了来,说是要征求他们的意见。章敦就是不愿意,也不能当着那么多人的面反对,其他宰辅都跟他一样,最后崇政殿中,又是二三十人济济一堂。

也不想想,现在为了个人的脸面和信誉,要放从犯一马的,只有诸位宰辅。而其余重臣,却完全没有这份顾忌。

韩冈是首倡之人。正是他让宰辅们开始立誓。可现在他又硬是将对手拉过来。

韩冈这是要在事后扮可怜,让其他人做恶人不成?

韩绛也不免往坏处想。

他区区一个大图书馆馆长倒是没问题,但被他逼着发话的两府其余宰执呢?

就是不说个人信用的问题,就是在面子上也得保住那些叛逆从党的一条性命。

韩绛不怕这些余党再叛乱,处理的手段多得是,关键是要维护自己作为宰相的威信。

就是软罢无能的张璪,也极力反驳的重臣们的论调:“曾布、薛向虽为执政,宫中他们不比宋用臣、石得一能使动禁卫兵马,朝中又不比蔡确能率领群臣,说他们都是叛逆并无错,但说是主犯就未免太高看他们了。至于苏轼、刑恕辈,更是无足轻重,不过是一班希图定策之功的小人罢了。如今首恶已出,但人心不定,未免京中再生动荡,正是需要镇之以静的时候。”

李定立刻反驳:“此等犯官罪行,是轻是重,是主是从,待有司审后方知晓。张参政又是从何得知苏轼、刑恕他们无足轻重?!”

张璪冷笑了一声:“不见中丞方才殿上出来指明蔡确、赵颢之罪。”

重臣们的立论虽正,宰辅们的私心虽重,但有平乱之功在手,就是向太后想将所有叛贼都给送去与蔡确作伴,也很难出来支持李定等人。

韩冈不是知道宰辅们是怎么想自己,但他拉侍制以上的重臣过来,可并不是让他们将自己的誓言推翻。

现在宰辅们有了压制群臣和太后的想法,确认了这一点就够了。

至于之后的事怎么安排,韩冈还是有些想法的。

又是一声炮响传来,距离之前的炮声只有须臾片刻。

韩冈依然不动声色,不过这一回,确认了炮声的就不止他一个了。

“什么声音?出了何事?!”

向太后突兀的打断了臣子们的争论。

冬天不会打雷,而且类似的爆鸣,她每天都能听见。那是每日上朝前都会随着晨钟传遍京师内外的声音,更代表了大宋威慑万邦的最大依仗。

“是火炮!”章敦对炮声同样熟悉,他盯着韩冈,“有人从火器局将火炮拉出来了。”

王安石脸色微变,随即转头问韩冈:“韩冈,你是怎么吩咐王厚和李信的?”

韩冈与郭逵全权负责平叛和捕捉党羽,王安石、韩绛之前让他随郭逵、张守约一并出殿,就等于给了明确的口头授权。

之后的细节怎么安排,就是韩冈与郭逵的事了,没必要向其余宰辅通报。

郭逵镇守宣德门,控制皇城局势,而王厚、李信领兵出宫,这都是韩冈与郭逵商议下来的布置。

王安石等人不会在意这些,他们只要一个结果。

只是没想到,韩冈竟然让将火炮拖了出来。

“臣与郭枢密商议了,逆贼亲属不足为虑,遣一小黄门携十余班直便可成擒。但京营之中,有多少从逆之人尚难知晓,未免其心存侥幸、最后铤而走险,只能大张旗鼓一点。”

韩冈冲着向太后弯了弯腰,

“现在必须得尽快镇住京中民心军心,否则乱事一起,平定虽不难,但京城可就要遭劫了。除了用上声势浩大的火炮,臣一时想不出仅有数百可信兵马,还能怎么做。”

……………………

在街道两侧的围墙中回荡的雷音犹然不绝,炮口的余烟仍袅袅而生。

从炮膛中飞出的弹丸,洞穿了厚达三寸的王府正门,只留下了一个内外通透的大洞。

门后的尖叫声旋即而起,堵在门后的齐王府人众,不知伤到了几个。王府高高的门槛,让里面的血水流不出来。

一名士兵上前,推了一下大门,门扇松动,却没有打开,看起来并没有打中门闩。

王厚皱了一下眉,虽然这时候派人去叫门,多半里面就会立刻开门就擒,但他没有这么做。

“继续!把门给我轰开!”王厚下令道。无视了越来越浓烈的火烟。

炮兵们又开始装药上弹,不再对准大门,而是将炮口瞄准了门框和支撑门框的柱子。

借用齐王府厚重的正门,王厚亲眼见证了火炮的威力。也终于知道为什么方才去军器监取弓弩时,李信非要让自己换一匹马。

为了拿诏书之后才赶上来的李彦用的宫中的御马,高大神骏,都是御龙四直随天子出行时才能骑乘。但一听到火炮发射的爆响,一下就人立而起,乱跳乱蹦,要不是周围有人死命扯住缰绳,李彦能在青石板路上摔断脖子。

而王厚的坐骑,只是晃了晃脑袋,完全无动于衷了。

在他收到的书信中,韩冈曾经多次与他提起过火炮,并宣称会超越过往的所有武器。

以韩冈本人的信用,兼之信中又将火炮原理剖析甚明,王厚自不会不信,只是没有亲眼见到实物,还是有些难以理解。

就是前几日在韩冈府上看到了一具具严格按照比例缩小的模型,又从韩冈那边看到了李信编写出来的,有关火炮训练和运用的操典,有了些许纸上谈兵的水平。

不过李信的兵练得好,王厚只要指着门,让他们瞄准就行了。

李彦皱着眉,完全不知王厚为何如此大动干戈,而且是一次、再次。

“上阁,让小人过去叫门吧,贼子早已胆寒,必然会开门的。”他向王厚请命。

“李彦,你是担心他们的性命?”王厚转头问道,眯着眼微笑。

看见他的笑脸,李彦脸色一白,连忙闭嘴。

自来到齐王府外,除了围困和宣诏,王厚就没有几句要求府内人众束手就擒的喊话。

若是遣人上去多叫两次门,再遣人拿着斧子去砍,保管转眼有人出门投降。

但王厚明白韩冈的心意。

现在什么最关键,安定京城中的人心、军心。

要么是春风化雨,润物无声,消灾弭祸于未发,要么就是风暴雷霆,巨石压顶,将浮起的叛心再压回去。

这就是他从韩冈那边收到的嘱咐。

王厚、李信在出宫前,韩冈便吩咐他们将声势闹得大一点,时间拖得长一点,若是失火了,不要让其蔓延。言外之意最好可以点把火。

虽然韩冈的话乍听起来完全不成道理,可郭逵就在旁边听着,他也没有反对的意思。

就在官场、又多读史书的王厚,当然明白韩冈为什么要这么做。

依照韩冈的吩咐,火炮肯定要上场,甚至里面的火势也可以不用救。

毁了屋舍,伤到人的确有些不妙,但那些都是叛逆之属,不算大事。而与蔡确、赵颢书信往来的不知有多少人,从两人的府中搜检出大批的信件才是大事。

若是穷究下去,可都是要人命的。

王厚好歹也知道,官渡之后曹操做了什么,更知道御史台想要在一封家常信中找出叛逆的证据有多么容易。

若能一把火烧干净,朝廷内外不知有多少人都要念着好。

望着愈演愈烈的火焰,王厚清楚,这是韩冈的目的,至少是其中之一。

能自己点火,倒真是省了大事。

