\section{第11章 飞雷喧野传声教(九)}

前一天的夜里,为了肆无忌惮的六路发运司官吏,韩冈坏了一个晚上的心情。

次日,韩冈在宣德门外见到韩绛,这位宰相的脸色也是难看。

由于要押班,韩绛难得来早了一点,不过这位宰相的身边,只有一个张璪陪着他说话。

两位宰辅并肩站着,却没有什么人奉承,与平常的情况截然不同。

都会看风色呢。

韩冈想着,走上前去,向韩绛行礼问好。

“玉昆。”韩绛看见韩冈,甚至都没回礼,就急着问,“六路发运司昨日的那份奏报看了没有。”

韩绛年纪大了,地位高了,心情倒也不需要多加掩盖,细节上也不需要那么注意。

“是那份奏报?韩冈也看到了。一个月三十多条纲船损坏,两万多石纲粮损失,也亏六路发运司敢报上来。”

韩绛冷笑道:“薛子正不在了,就敢糊弄人了!……”他又冲着张璪道,“蒋之奇到任也有几个月了,可情况越来越糟,他到底去做了什么?”

现任江、淮发运使蒋之奇是张璪提名的,被韩绛质问,张璪也觉得难堪。

蒋之奇在朝中一向以干练之称,不论是水利还是理财,皆有所长。

尽管他因为弹劾举荐他的欧阳修帷幕不修,在朝中一向被视为奸人,可从来没有人质疑过他的才干。正是因为有这份才干,所以蒋之奇才能在朝中屹立不倒。

说起来这位现任的发运使,跟沈括一样,都是顺风倒,还总是会反咬一口。欧阳修在南方郁郁而终,也多亏了蒋之奇揭了欧阳修家的家丑。

从蒋之奇身上想起了沈括,韩冈又顺带想起另一件事——沈括的堂兄弟可是蒋之奇的岳父,这可以叫做不是一家人不进一家门了。不过沈括的堂侄女嫁给蒋之奇是去做继室的,又不像沈括家的河东狮,能把丈夫使唤的滴溜溜乱转。

可是,这一回蒋之奇的表现,远远对不起他所得到的评价。

不过这段时间,韩冈与张璪合作愉快,遂在旁帮他解围,“其实也不能怪蒋颖叔,他是投鼠忌器。薛向留下的规条,他遵从不是,不遵从也不是。”

“不知道该怎么办,就该以公事为重。不然朝廷将他放在六路发运的位置上做什么?!蒋之奇向以理财和漕运著称于朝,怎么会这么糊涂。”韩绛气哼哼的说着。

“发运司中刚刚又换了一批新人,蒋颖叔想要打理好内部,还得一些时间。”

其实从发运司三月时的奏报中,就可以看出了些苗头了。那是汴河解冻之后纲运重启的第一个月,纲船的损耗率就超过了过去几年的同期水平,只是超出不多,没有引起太多的注意。

但从这之后,纲船损失率一个月比一个月更多,就算政事堂中的三位宰辅再迟钝,也很快看出了不对。不过当时以为是薛向曾经提拔的一干重要官员,被调离和贬官所引起的结果,只要调派得力之人去掌管发运司,就能解决这个小问题。蒋之奇就是因为这个原因,所以才会被派去做了发运使。

只可惜政事堂的期望给他辜负了,发运司中的情况并没有变得更好,而是更糟了。

“哦,既然玉昆你这么说,那下面该怎么办?”

“邃明参政怎么说?”韩冈将球丢回去,他是帮张璪说话,可不是在帮蒋之奇。剩下的,该是张璪的事了。

“这要问相公了。”张璪反问回韩绛,“薛向过去在发运司中留下的规条该怎么说?”

“人有过,策无过。薛向的罪过又与他在六路发运司中的举措无关。”韩绛肯定了薛向过去的成绩。

“玉昆?”张璪又回头来问韩冈。

韩冈道:“过去薛向在六路发运司中定下的规条,让朝廷每年都能少损失数十万石的纲粮。既然是善法,当然该留下来。”

“既然如此,就这么告知蒋之奇,让他好生把衙门里面处置一下,明年的纲运必须回复到之前的情况。”“再给蒋之奇一次机会,若他还不能将纲运事安排好了,他还是去监酒税吧。”

“最好能明确一点,明年的纲粮损失率不得高于之前五年的平均水准,否则六路发运司上下一律磨勘加一年,若毁损数量远远超过旧年平均,那么别怪朝廷的刀子斩人了。”

“玉昆的这条好。”韩绛当即说道,

张璪也点头。他现在可不介意将六路发运司都洗一遍。

“那今年怎么办?”

韩冈又问了一句,抬头看看周围,苏颂这时候也到了,不过他见政事堂的三位宰辅围在一起,脸色严肃的说着话,就没有上前来打招呼,而是远远的站到一边。

“对那一帮奸猾贼子,必须严惩不贷。”

韩冈听见了韩绛杀气腾腾的声音。

也难怪韩绛生气。薛向倒台之后,来自南方的纲粮损失率立刻就升上去了,发运司那边是想证明什么,没有薛向就没办法了?

这让当朝宰相的脸往哪里放?

没了老猫,一干鼠辈的确就得意了。可这不也是再说,剩下的猫不会捉老鼠吗?分明是在为叛逆张目。

“贪渎官员不得不严惩。”张璪也附和着。发运司中的官员,可算不上是士大夫。而且他也对不知死活的发运司官吏动了真火。

“那就全部送去西域吧!”韩冈提议。

“全部?”张璪顿时吃了一惊。

汴河之上,与纲运有关的,连同拉纤的厢军在内,也不过数万人。可把数万人都送去西域,这依然不可能。但一味喊打喊杀,对底层官员并不一定有用。想也知道,朝廷怎么可能当真杀那么多官吏,只是调去边疆才是最好的处罚。

“把最后一批运送纲粮上京的所有人,军校士卒也好,民夫也好,都抓起来,问出到底谁是主谋,谁在收购纲粮。收购赃物的贼子,抄斩!其余人犯,让他们在问斩和流放中选一个。”

“如此甚好。”韩绛立刻点头。

“玉昆,犯罪的不能全都流放西域,各地都缺人。”

边疆缺乏户口充实,不论是东南西北,都缺人。

西域、交州就不说了。就是人口最多的河北,像沿海的沧州等地,同样是人烟稀少。

偌大的沧州,沧州城以北,界河以南,南北百五十里,东西百余里的土地上,连一个县城都没有。虽说此处是黄河入海口,多有沼泽,地质又偏盐碱,不怎么适宜耕种,但更不适合耕种的西北照样有很多人在那里生长繁衍,沧州北部渺无人烟,就显得太过浪费了——不能种粮,还能种棉啊。

此时来到城门下的朝官越来越多,而站在门前议论汴水发运事的韩绛、张璪、韩冈三人越发的成为关注的焦点。

韩冈心中有些恶作剧的想法,要是他们误以为东府的三位宰辅在朝堂上来什么大动作,那可就有趣了。

“邃明兄说得是。”韩冈也没耽搁说话,“照韩冈看,重法地也该改改了,有了流放,也不用都问了死罪,也免得三法司的麻烦。”

所谓重法地,就是对盗劫等重案的罪犯,一律往重里判,一般取判罚上限的地区。

而重法地的制度,是仁宗皇帝开始。当时为了补充对抗西夏的军费,税赋提高了许多,各地盗贼蜂拥而起,按欧阳修的说法是‘一伙多过一伙’,故而在京师等地,对犯人论以重法,以遏制犯罪的猖獗。由于重法地制度推行,越来越多的路州被归入重法地的行列,被判死刑的人数也大幅上升,至今快有四十年了。每年冬至,都有数千人被勾决。

也就是这两年,被勾决的人数数量少了。去年是以给太上皇祈福为名,今年便是太后德政了。往年都是在三五千,近两年则是五六百,除非是十恶之罪,或是杀人重罪,其余全都改成了流放,主要是西北,也有岭南。一般就是视情节轻重,而决定路程远近,而且变成了遇赦不得归,只能在流放地一辈子。

“重法地已是名存实亡,当然可以废除,只要不杀人,就都改流放。”

韩绛很爽快的就同意了韩冈的意见,少一点犯人被处决,在治政上,也算是一个亮点。就像监狱狱空,就是祥瑞一般,少杀些人,在儒者的眼中终归是一件好事,而在佛道两家来说,也算是积阴德了。

“早该如此了。”也不知张璪是投桃报李,还是当真这么想,抚掌对韩冈,“说是流放,照样能分到田地种,这样的惩处实在太轻了。也幸好有一条遇赦不得归。”

“的确。”

不论是关西的哪里,只要犯人流放过去后,都会让他们老老实实的种地,除了遇赦不得归一条以外,其他方面都是太过宽松了。

“玉昆,这件事你先提上去如何?……”

号炮声按时响了起来,掩去了韩绛的问题。火垩药在炮膛中爆炸的声音代替了过去的钟声,成为了皇城开门的信号快有一年了,上上下下都已经习惯。

即将入城,韩绛也不再多说话,让元随牵过马来,然后翻身上马——只有宰相可以骑马进入宣德门。

望着韩绛的背影,张璪的眼中流露出淡淡的羡慕,随即又藏了起来。

韩冈收回自己的目光,微微一笑,等待着城门的开启。
