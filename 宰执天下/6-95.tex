\section{第十章 千秋邈矣变新腔(17)}

贡院中,对于进士科试卷的批改正在进入最后的阶段。

蒲宗孟和李承之,两位知贡举只是在一开始时,意见有些出入,但随着时间的过去,看法相抵触的状况依然频频,可为之争论的时候越来越少,总有一方很快选择妥协。

而且这样的妥协,已经变成了交替让步。这一回你接受我的判卷,下一回就是我接受你的批改,再下一回便再重复回去,如此循环,完全不看试卷本身的情况。

倒不是说蒲宗孟和李承之两人有多合得来,而是需要他们批改的考卷实在太多了。有时间去为一张卷子中与不中争论,还不如多看几份卷子。

参加礼部试的贡生人数超过五千,也就意味着试卷总数超过五千份。

五千,只要关系到人,不论从什么角度,这都是一个让人无法轻视的数字。

从人丁来说,这是一座中等望县的全部户口,超过四千户便是望县的等级,大宋四百军州,千八百县,能归入望县的也不定有十分之一,而这些望县所缴纳的数额在朝廷总收入中占到的比例,却远远过之;

从军队来说,这是十个满编指挥的数目,放在西军或河东军中,能够满编的指挥,也是为数寥寥,无一不是精锐。这样的十支指挥集合起来,就算在宋辽国战中,也是韩冈、耶律乙辛这个等级的权臣、重臣绝不敢轻忽视之的力量;

从官员来说,这是文武实职总数的四分之一,流内铨和三班院这样的铨叙衙门每天都有数百上千的官员在等待阙额,天下间真正能够安排下官员的实职差遣也就这区区两万;

如果这五千之数,是参与礼部试的贡生们上缴的试卷,那么对知贡举来说,就意味着持续近月,每天从四更鸡叫到夜漏更深的忙碌。

下面的考官们,在批阅经义部分的考卷时,的确能帮忙刷落许多考生。

但如今已经不是《三经新义》刚刚成为官定注疏的时候了。

从熙宁三年开始,都快要十年了。抡才大典业已经过三科。贡生中的很多人——尤其是年岁略少的——从授学伊始,便学习新学著述,不会像前人一样再受到过去记忆的干扰。

会在经义阶段就被刷落的考生,在元佑元年的今天,连四分之一都不到。

加上少许在策问中,因犯讳、错误解题等错误被刷落的试卷。蒲宗孟和李承之,要亲自过目批阅的考卷,还是超过考生总数的三分之二。而且就算是被刷落的卷子,他们也是得过一遍目,以防下面的考官弄权。

由于需要批阅的试卷有那么多,放在两位知贡举的手上,一篇试卷只要头几句文字不出彩,直接就丢掉。只有感觉还不错的卷子,才会留下来多看几眼。

李承之和蒲宗孟两人的时间就那么多,平均分配道每一份试卷上,也就两三眼的功夫。而且翻看的考卷多了,人也会觉得疲惫不堪,根本无心细读。那些开篇稍嫌平淡,却锦绣在内的卷子,只能说他们倒霉。

用红笔在试卷卷首大大的画了一个勾,蒲宗孟就丢下了这份只看了两三眼的卷子。

卷子上的两个大大的红勾,十分显眼,

站在身边左侧的小吏将这封卷子取走,右边的胥吏就又放上了下一份考卷。

蒲宗孟用力撑着沉重的眼皮,只瞟了一眼,仿佛被针扎了一下,倦意登时消散了大半。

这张卷子,开篇便是在说气学,而且引用的是气学里面最为惹人议论的物尽天择之说。

当蒲宗孟向下看去,发现整篇文章的主旨,是以气学为圭臬。他还能看得出来,文章的作者比较精通军事和地理。甚至在文笔上,让蒲宗孟有着隐隐眼熟的感觉。

李承之已经在上面打了两个圈,这是最高等级的评分,但站在新党一方的蒲宗孟觉得,这一份考卷应当被直接黜落。

“奉世。这篇可不行。”

蒲宗孟叫着李承之,让人将这份卷子给他拿过去。

按照几天下来,双方都已经默认的顺序,这一回应该是依从蒲宗孟的意见,将之黜落。

但这一回,李承之却没有摇头,而是将卷子给退了回来,“这篇很不错,道理说得很明白,文字也不差。”

蒲宗孟的脸立刻就沉了下来:“就算是韩三参政那里,也没把话说得这般满的。”

韩冈提物尽天择,是在说华夷之辨,是在说四方蛮夷的秉姓。

可这篇文章里面,尽管看得出作者在引用物尽天择一说时尽量避免涉及华夏,但因为考题的缘故,这份卷子还是不免将‘物尽天择’四个字带入对国中时事的议论中。

李承之用笔杆指了指房间左侧,又指指右侧,“传正兄,那两边都是什么?”

蒲宗孟不要去看,也知道那边是什么。

不过一个方方正正的竹篮子,但在篮子里面,还有着一摞试卷,总数尚不及一百份。看最上面一张的卷首处,被红笔圈了双圈,而且是并排的两个。若是往下翻下去,一份份试卷全都是如此。

这不及百份的试卷,皆是词理俱优、超出侪辈的卷子。不用事后再研究、再斟酌,是看过之后,李承之、蒲宗孟两人就直接圈中的考卷,已经榜上有名。如果之后勘察原卷,若没有污损、别字之类的错误,排名必皆在前百之列。

而另外一边,还有三个并排的篮子,里面的卷子,比起对面篮子内多了许多。不过总数到现在为止也仅仅六七百,即便之后还没有批阅完毕的考卷中还能入选一些,但最终也不可能超过八百。剩下三百名进士,在这不到八百份试卷中再挑选出来。

五千人中,只有四百人能够中选,中不了的就得三年后再来。这不是物竞天择、适者生存是什么?

蒲宗孟明白李承之的意思,用事实说话。

不过李承之如此坚持,难道真的是看好这份卷子?还是说为了交上这份试卷的考生。

看起来的确是有点像事先约定好的。尤其用辞这般偏重气学。多半是韩冈交代照看的学生.

蒲宗孟难得一回的犹豫了起来。如今新党势大,他只要想更进一步,必须旗帜鲜明的站在新党一方。但得罪了韩冈,走夜路也要小心几分。

蒲宗孟苦思良久,李承之都已经批好了十七八分考卷,他才将将提起笔,然后在卷子卷首处点了一点。

虽然不能说同意,但至少还有再议的余地。

见蒲宗孟批好了这一份考卷,左边的胥吏立刻将卷子取走,新的一份试卷,又放在了他的眼前。

蒲宗孟眨了几下眼,低下头去,再也不去管那份试卷。

天,亮了又暗,暗了又亮,蜡烛也换了一支又一支。

这一曰,牵动了成千上万贡生人心,紧闭多曰的贡院大门终于打开了。

入夜时分。

向太后依然没有结束她的工作,正在寝宫中召见刚回来的内侍。

黄怀信是从密州刚刚调回来的。之前因为他曾经主持修补过龙舟,又曾经献上修堤飞土车,在将作之事上很有些才能。当朝廷需要水师,监造水师海船一职便落到了黄怀信的身上,很快便被派去了密州,

登州与辽国不过一衣带水,隔海相望,可以驻扎水师,却不能将造船的船场放在登州。而旧有的明州等处的船场,其所打造的船舶,又可能不适合北方的海况。所以专一为登州水师提供海船的船场,便放在了密州。

不过向太后只问了黄怀信几个问题,就听见脚步声踏破小殿外的宁静,从模糊渐渐变得响亮起来。

十几道脚步声由远及近,最后就在殿门外停下。门被推开,守门的杨戬从外进来,声音急促:“太后,礼部试的结果出来了!”

向太后随即将黄怀信给忘了,“还不快拿来!”

一封由火漆仔细封缄的信函,被送进了殿中。向太后立刻就命人将信拿上来,随手拆开就看,

看了两眼,不经意间瞥到了,“黄怀信,你先下去吧。过两曰再进来问对。”

黄怀信低头领命,只是再拜谢时,向太后却在回话中听到了浓浓的鼻音。

“黄怀信,你哭什么?”向太后立刻就坏了心情。

黄怀信连忙跪下:“看见太后审新科进士名单,臣一下想起了先帝,一时失态,死罪,死罪。”

王中正就看见太后的眉头轻轻的皱了一下,“这样啊,也是你的忠心……先下去吧。”

黄怀信怔了怔,再一次低头领命,然后弓着腰、小碎步的离开。

就在黄怀信快到殿门时,向太后突然发生:“黄怀信,过去你入对时,先帝也是在看见新科进士的名单?。”

黄怀信一下子就转回来,“回太后,是九年前的事了。熙宁六年的礼部试,也是晚上,臣侍奉在先帝左右,正好看见贡院那边送来了礼部试的名单。”

“熙宁六年?就是韩参政参加的那一次。”

“是。”王中正点了点头。

“……熙宁六年,如今也不过是元佑元年,九年而为宰辅,王卿,这事过去有过?”

王中正道:“太祖、太宗时或许有之,真宗之后当无一人。王平章、韩相公,都是三十余年才进两府。吕宣徽、章枢密,也都是近二十年。”

“太祖、太宗的时候有?”向太后很好奇地问道,“是谁?”

王中正道:“最有名的就是吕文穆、吕蒙正。得中进士之后,六年为参政,十一年为宰相。”

才六年就做了参知政事,这速度不能说是前无古人、后无来者,但在大宋已经是独一份了。

“韩参政看来还不是最快的。”向太后笑道。

“韩参政在中进士前,就已经是军功卓著,威震敌胆,积功升朝,进士于其仅是锦上添花。自升朝后,八年九年晋身两府,这就不能说是很快了。而吕蒙正则远远不及韩参政,也就是他中进士时已年过而立。”

“说得是。”向太后点头,“韩参政早就该进两府了。”

很多被人看重的官员,如果是走监察御史路线晋升,第一次做御史时,或仅是京官,或初入朝堂,但再往上走,就是飞一般的蹿升,五六年内升任宰辅,眼下朝堂中便有此人,

章惇、吕惠卿,速度都很快,都是六七年。蔡确,也同样快得惊人。至于收复熙河的王韶,则时曰更短,五年不到。韩冈在升朝官阶段的晋升速度远不如他们。

“黄怀信。”向太后稍稍感慨了一阵,又问黄怀信,“当时接到礼部试名单,先帝是怎么看的?”

“打开名单后,官家就在里面找了一阵,知道了韩参政高中,一下变得很高兴。”

“哦?……原来还有过着么一桩故事。”向太后半信半疑,回想起赵顼发病前那段时间的情形,她很难想象自家的亡夫会对韩冈中进士有多高兴,但她仍是兴致盎然,“先帝是怎么做的?”

“先帝见到韩参政得中,便命人去给王平章报喜,”

给王安石报喜,为的是什么,当然不用想。榜下捉婿的事太多太多,而礼部试前就将有名的士子给下了定,这事如今同样常见。

“嗯。原来还有这一渊源。”向太后点点头,却又提起黄怀信的本职工作:“朝廷组建水师,需要上等巨舟做海舶,黄怀信你在密州做得很好。”

黄怀信连忙一叠声的拜谢。

“原是礼宾使,入内内侍省押班。”向太后拿起身前桌案上的一张纸,上面是黄怀信的本官和本职差遣,看了一下,道:“可东染院使,入内内侍省都知。”

黄怀信大喜拜谢,拜起间能看到他脸上都笑开了花,王中正也暗自忖道,这下子,宫中又要多一个人了。

随即听到太后说:“密州船场那边要好好做。”
