\section{第23章 奉天临民思惠养(中)}

“检正,这是你要的许州济养院的文书。”

宗泽从堂吏手中接过卷宗,点点头,让人先下去。

“怎么还在忙着?”

堂吏刚出去,一人溜溜达达就进来了。

宗泽忙得连眼都没抬,“相公给的差事,等他从宫里回来要办好。”

他的案头上,已经摆了几十份公文,皆是有关济养院的文函。

在许州的公文中翻了两下,找到了他所要的数据,端端正正的记录在一张纸上。

那人伸着头,看了看宗泽刚刚丢到公文堆最上面的那一份:“济养院。这不是兵礼房的差事吧?”

宗泽头也不抬,从另一侧又拿起一份公文,边翻着边说,“相公交代下来的,你能跟他说不关我的事?”

这其实并不是兵礼房的差事,所以宗泽也没有让下吏来帮忙。

“汝霖你是宰相之才,韩相公不过是想让你多历练历练。”

宗泽又提起笔,在纸上认真的写着什么,“多谢夸赞,宗泽不敢当。”

韩冈不止一次的夸奖宗泽是宰辅之才,然后十分干脆的将很多要事都丢给了宗泽来处理。

宗泽一方面感念韩冈的知遇之恩,但另一方面,也觉得自己是不是被压榨了。像是磨坊里的驴子,看到前面挂着的青草,然后不停的一圈又一圈的拉着磨。

“京师是不是要抓人了?”那人忽然放低了声音,悄声问着。

“赵伯坚,你是替谁问的?”

宗泽无奈的抬起头,望着眼前的中年官员。

赵令铄,表字伯坚,是太祖五世孙,已经出了五服,除了玉版留名之外,没有别的好处,不过也就可以参加科举。

谁都知道,宗室中人不出五服不可能参加科举,最少也得是开国匡字辈的五世孙才能算是出五服。而太祖一系的令字辈,太宗一系士字辈,魏王系的之字辈,加起来也不过两千人不到。其中没有官职荫补,还认真准备考进士还不到百人。所以到了如今,宗室进士就只出了这么一个。

开国以来的第一人,可想而知,赵令铄在宗室中的地位会有多特别。赵令铄如今官品低微,但他的名字连皇太后都听说过,虽说没指望做宰相、参政、枢密使,可进入议政重臣行列的几率却要比普通进士高得多。

赵令铄本人,因为其宗室的身份,在中书中虽不当重任,但地位也十分特殊,谁都能说个好。宗泽与赵令铄的关系不错,而且前两年赵令铄因为道遇叔祖宗晟不致敬,被告上了宗正寺和中书门下。赵宗晟是太宗曾孙,太祖、太宗系之间,心结很重,总是有着明里暗里的冲突,看见赵令铄这个春风得意的皇家进士自有几分不顺眼,这下抓到把柄,当然不愿放手。幸好宗泽在中间帮了把手,免了官面上的处罚。

不过两人能轻松相谈,还是气味相投之故。

“人多了。”赵令铄抽了张小几子过来坐下来。

宗泽拿他没办法,摇摇头,“不是去洛阳办差吗,怎么这么快就回来了?”

“你当去哪里,洛阳啊。今天去,明天回,能要几天?”

“没再多留两日,洛阳风月难得啊。”

“第一次去多留两日倒也罢了,去的多了,洛阳城里面也没什么意思了,还有什么好玩能比得上京城?等到夏天,再抽几天空去邙山走走。”

有了铁路之后,去洛阳、去应天、去大名府等四京出一趟差,都变得十分简单。

用京洛铁路,走一趟洛阳,一般是八个时辰。

如果是五更发车的早班车的话,就能在城门落锁之前抵达洛阳。第二天办了事,做当天晚上的夜车回京,次日清早便能上工。满打满算也就是两天的时间。如果都是坐夜车往来,甚至一天多一点就够了。当然,洛阳风月不输京华,朝廷虽是差使人,但还是会讲些人情的,一般都会多给两日差。

“真要说到风月,洛阳城里还不如城外的车站,汝霖你怕是不知道,洛阳车站周围现在变得有多好。”

宗泽拍了拍手边的公文堆,“洛阳车站去岁净入十四万三千余贯,我会不知道?”他笑了笑,“东京车站更是三十多万。铁路的维修、人工,只靠车站下面的产业都包下了,运费就是净赚,商税还要另算。”

所有已经建成的铁路车站,在建造时,无一例外的都顺便占下了很大一片地。除了一部分属于车站本体建筑之外,剩下的都修起了屋舍。可以做仓库,客栈,酒店。

京泗铁路通车的这两年,仅仅是车站出租房屋的收入,已经可以将铁路的运营费用给抵过去了。并代铁路,虽然地处河东偏远之地,但车站的额外收入,也保证了整条铁路能够正常的回本。而最早修成的方城山轨道,尽管刚刚完成了新一期的改造,但由于是不亚于汴河的要道,半年的收入足以抵得过当年刚开始修轨道的支出。

“一本万利啊。”赵令铄干笑了两声,又道:“不说这个了,相公要办济养院,肯定不是花钱卖好简单,但大理就这么缺人吗?”

“不是大理,是云南。”宗泽更正道,“相公曾经说过,尽管这些人多是污了汤的老鼠屎,但放到边地,还是要比蛮夷干净一点。”

“但其中也有些可怜人。”

“的确如此。”宗泽点头,“可朝廷给了他们属于自己的产业,难道不是仁政吗?能堂堂正正做人,难道不比比卑躬屈膝强?何况乞丐之中,不乏将他人家的小孩子绑了去,打断腿,毁了容,养大之后,用以乞讨的恶徒。这样的人,岂能容他继续不作而食?!”

宗泽现在手上除了日常的事务之外,主要就是韩冈丢过来的云南路的移民工作。

近年来,但凡刺配流戍的罪犯,要么去岭外,要么去西域,已经形成了制度,哪一路的去哪里,都有规矩。现在多了个云南,想要移民,就要从他人口中夺食。

因为动辄刺配边远,天下间的犯罪是一年比一年少,比起十年前,案件总数整整少了三成。本来人就不够分,哪里还能经得起再一家的抢食。

所以政事堂那边就想,与其三家抢饼,还不如将饼做得更大一点。这主意,免不了就打到了满街的乞丐的身上。

去年岁末,朝廷诏令天下各路州县,设立济养院,用以收容衣食无着的贫民和乞丐,并提供食宿。济养院的制度,名义上是恩泽天下贫民、乞丐,但实际上,就是一个吸纳移民的衙门,要让一干因各种原因不事生产的劳力,为大宋稳定边疆。

饥民乃是祸乱之源,饥荒时,朝廷在流民中选强壮者为兵,便是预防有人作乱。而太平年景,虽不虞有流民于途,但因各种各样的原因而沦为赤贫,衣食无着的百姓,数量依然不少。与其让他们沿街乞食,最后开始作奸犯科,还不如先给他们一条出路。

所以按照预定的计划,将会用几年时间,逐步让天下城中禁乞,只要发现乞丐,全都收入济养院中。其中有劳动能力的,便是送往云南等偏远之地,让他们耕种。暂时没有劳动能力,才会养起来,只要有双手,就不愁没事给他们做。

很多乞丐,都是有手有脚,做些体力活肯定能养活自己,会乞讨,只是懒而已,到了边地,自有劝农官来帮他们改正这个毛病。不怕他们敢闹事,到了人生地不熟的蛮荒之地,汉人必须抱团,不听官府的,就要在蛮夷手中吃苦。而那些因为失地而不得不乞食的流民,则更受地方上欢迎,都是老实人,不会闹出一些幺蛾子的事。

赵令铄沉默了片刻,“所以相公才会选在三月正式推行养济院制?”

“当然。”

每年到了青黄不接的时候,京师的乞丐就会多上几成。设在三月开始推行养济制度,韩冈虽没有明说,但明眼人还是能看得出来,这分明是撒大网捞大鱼。

“乞丐都不留了,那两厅三院那边又有个什么章程?”

所谓两厅三院,就是开封府左右厅和府院、左军巡院、右军巡院,管理着开封府的刑狱诉讼。

“乞丐不论,如果没有正当职业,又找不到三个保人,只要定了罪,不论多小,都会去云南。其实伯坚兄你也不用急着问,过两天章程就会公布的。”

近几年,京师内部对大小过犯管束已是极严。

京师百万军民,市井中不免多有一干破落户,走着偏门吃饭。如果自身不学好,骚扰街邻,或是勾引好人家的子弟学坏,往往就会被告官。一旦罪行确定,登时就会被发遣边地,一辈子都难以再回京师。

这样的案子,隔三差五就有一起,只要在京城中生活,经常能听说这等事情,甚至报纸上都会进行刊载,以警士人。

宗泽家旁边有户官宦人家,主人是兵部员外郎,在枢密院职方司办差。他家的大儿子就被一个泼皮引诱了去赌球,而且还是私人的外围赌球,去年一个冬天就输了两百多贯,然后被报了官,引诱他家儿子的泼皮,给判了去北庭都护府。而那个开私家赌球的,则是杖遣交州——先杖一百,再发配交州。

不过在左军巡院中挨了五十多下,就咽了气,一条草席裹了出去,也没机会出京师——敢从蹴鞠、赛马两大总社手中抢食,自然会被杀鸡儆猴。

但现在还要办得更严,但凡没有正当职业,都在打击行列。没有正当职业,也找不到三位以上的保人,一旦犯事,就得去云南走上一遭。

赵令铄有点发愣,“这下子,京师中的风气可是要大变样了。”

“这正是相公要看到的。”
