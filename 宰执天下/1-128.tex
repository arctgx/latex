\section{第46章 龙泉新硎试锋芒(四)}

【第一更,求红票,收藏】

给低层官吏添支俸禄要花的钱实在很惊人,可并非没有好处。高薪养廉的效果,也许有,也许没有,除了王安石外,吕惠卿他们并不是很在意。但对朝堂政争,却是益处多多,显而易见。

一旦听说王安石要给天下卑官胥吏加发俸禄,反变法派到时会怎么说?

如果韩琦、司马光等人继续反对,好吧,全天下的低层官吏便一股脑儿的都会被他们得罪干净,变法派肯定会兴高采烈、加油添醋的为韩琦、司马光宣扬。

不反对,那陆续增加的巨额支出,就越发的让天子不敢轻易动摇各项以填补亏空为目的新法的施行,王安石的地位由此可以稳固。

当然,韩琦等人还有推波助澜这个选择。王安石说给每名监镇、县尉这样的选人月俸加上一贯,那韩琦可以喊‘加三贯’,文彦博说‘你看他们这么辛苦应该加五贯’,司马光说不定会喊个‘应该加十贯才对’。这等操蛋的做法的确可以让变法派偷不着鸡蚀把米,但那时,天子又会怎样看待搅乱朝纲的反变法派?

对吕惠卿他们来说,这一招实在是妙不可言,因为只有变法,才有足够的财力支持添支俸禄这个政策。而反对变法,就没钱拿来收买人心,只能陷入了进退两难的境地。

话说回来,这事以王安石等人的才智不是想不到,等到财政状况好转,他们说不定就会想到并提出来。可现在王安石的口袋里空空如也,当然只会想着如何挣钱,省钱,而不是花钱,赵顼起用王安石,也是为了弥补财政亏空。

韩冈心中有些小得意,这是英国人在香港做过的事,让后接手的政府有苦说不出,韩冈只是随手拿过来使用。明明白白的阳谋,就算司马光、文彦博他们能看破,也化解不了。

当然,他既然给王安石支了这一招,就等于确定了自己的政治派别。但对韩冈来说,投靠哪一边根本不是问题!他本就没有选边的资格,举主王韶的依靠是王安石,河湟拓边所需要的朝堂支持也只有从变法派这里得来。

即便他是张载的学生,同时又承张戬、程颢之教,但在反变法派里依然找不到自己的位置。就算是张戬,程颢二人,身为可上谏君王,下弹重臣的御史,在反变法派中的地位,也不过是马前卒而已,根本无法与王安石相提并论。

人总是趋利的,韩冈只会选择符合自己利益的一边,即便不看好王安石和变法的结果,但韩冈个人而言,变法派却是如今最好及唯一的选择。

曾布最后还是忍不住,哈哈笑了起来。因为这一招实在太妙了。近日他被吕公著、司马光还有陈升之的各种小手段弄得一肚子恼火,却无处发泄。现在韩冈给他们支了一招,只是坐在这里想一想,就觉得一口怨气终于出了大半。

他笑着对韩冈道:“到底是韩玉昆,这一招确是有才。”

韩冈不接口,笑而不语,有些话说明白就没意思了,含而不露才是正确的应对。

吕惠卿却在盯着韩冈。他觉得韩冈提出的策略,就跟他的眉眼一般锐利……而且老辣。不像是个年轻人。但韩冈没有明说,一切只是他们自己的推演,也有可能韩冈根本没有想那么深,只是不好意思为……

吕惠卿忽而失笑,这个想法的可能性反而更低,洋洋两万言的《伤病营管理暂行条例》可是摆在过他的案头上,心思缜密,面面俱到,这是他当时就给韩冈的评价。现在说他想不到这么深,那就是在说自己没有识人眼光了。

后生可畏啊!吕惠卿感叹着。韩冈今年才十九,就已经如此出色,日后若能考个进士出来,前途不可限量。

为低层官吏添支俸禄,事关重大,牵连到朝堂的方方面面,不是短时间就能决定。即便决定了,也不可能一步到位,而是会逐步增长。放在现在,就仅仅是个可以考虑的提议而已。

但这个在预计中,必然能行之有效的提议,成功的影响了书房中的气氛,让在座的五人,心情都变得很轻松。

王安石拿起茶盏,啜了一口,冷掉的茶水口感发涩,但他喝得很是舒畅。王安石一向想得多,吃饭都是心不在焉,只会吃面前的一盘菜。喝茶往往也是茶杯摆在面前,一天都不会记得要喝。也只是现在心情放松,才会记得要喝水。放下茶盏,他笑问着韩冈,“玉昆见识过人,难得一见。如今中书检正五房之中,也是缺着玉昆这样的人才。不知玉昆是否有心到京中来?”

韩冈心中一惊,想不到表现太好也有问题。他摇摇头,如果自家有一个进士出身,或许可以有另外一个选择,但他是王韶为了河湟开边才推荐的官员,他的去处只有秦州,“相公的夸赞,韩冈愧不敢当。在下才疏学浅,又未有实务经验,中书里的事务不是在下能做得来的。何苦饮水思源,王机宜的恩德,韩冈始终铭记在心,不敢须臾或忘。”

韩冈的回答,王安石心中早已有数,也只是问问而已。韩冈虽然年轻,却是豪侠的性子,王韶对他有恩,他自然不会因为一点好处而背叛。

王安石沉吟了一下,又道:“天子对河湟之事一直放在心上,王子纯的《平戎策》也是天子先看到的。玉昆你自秦州来,对河湟如今局势自然了如指掌,可有意入朝向天子述说一二?”

韩冈出了这么大力,立场坚定的站在自己这边,又谦逊如此,手上一时也没有什么可以奖励他的。王安石便想着让韩冈越次面圣,也好在天子心头留下名字,在崇政殿偏殿的屏风上留下名字。

韩冈心惊肉跳,头摇得更厉害,坚辞道:“无有寸功,如何可以面见天子?下官又不过一个从九品选人,卑官朝觐天子,也不合礼制。此事万万不可!”

开什么玩笑,让他出主意没问题,让他冲杀到前面去,这是让他做炮灰啊!回秦州挣军功才是真的。朝中有王安石支持,李师中、窦舜卿之辈不足为虑,辅佐好王韶,收复河湟边塞,这军功,当是太宗朝收复北汉以来第一功。王韶日后说不定能升到枢密使,而自己也有了青云直上的根基。

“也罢,那就下次好了。”见韩冈辞意甚坚,王安石也便不再坚持,心中则更加看高了韩冈几分。

此事一了,话题便不再局限于朝政,而是很随意的闲聊起来,众人谈笑风生。

看着辞锋往往一针见血,却又不失诙谐的韩冈,吕惠卿突然发现,不经意间,韩冈已经是跟他们几乎平等的在交谈,在说笑,在评论如今朝局。这与一开始打算考验一下韩冈的初衷,完全不一样了。

局面的改变,大概就是从韩冈说的那个笑话开始,而因方才他的计策,而成为定局。但除了自己,好像谁也没有发现这一点。

吕惠卿心中暗暗赞叹,能在潜移默化中引导气氛,确立自己的地位,韩玉昆的心思的确不简单——如果并非刻意,而是自然而然做出来的,那就更不简单——吕惠卿看出来了,但他乐见其成,因为韩冈的才能得到他的认同。

在参与变法事业之前,吕惠卿在士林中得到的评价是‘学有操术,才剧器博’,‘为当今士大夫之高选’,这些话是欧阳修、沈遘、韩绛、曾公亮所说。但到了变法开始之后,由于吕惠卿是变法派王安石之下的第一号干将,直接掌管制置三司条例司,变成了人们口中的奸佞。

评价急转直下,但这么短的时间,吕惠卿却不会有太大的变化。未来也许不知,至少现在,他还是‘才剧器博’的吕吉甫。对于他来说,地位的高低不算什么,豪杰居陋巷,蠹虫据高位,这样的情况太多了。才智、才学,才是他所看重的,这也是真正的士大夫们共同的认识。

书房外厅中的谈笑从紧闭的门缝中传了出来,王旁站在书房外厅的侧门前,心情阴郁,已经忘了自己来此的原因。

昨天尚与自己难较高下的对手,现在已经成了父亲书房中的座上宾,而且和吕惠卿、曾布、章惇这些被父亲赞不绝口的俊杰才士,毫无畏色的谈笑着。

因为身边有着父亲和长兄这样人物,王旁心中一直有着隐隐的自卑,而且父兄来往的友人,无一不是才气纵横,也让王旁自惭形秽。而愿意跟他结交的,却都是因为他父亲的权势而来。

可韩冈不同,他虽然是父亲请来,但昨日却被晾在了偏厅。与他势均力敌的下了两盘棋后,王旁便觉得多了一个能平等相处的朋友。可谁知,韩冈竟然毫不逊色于他过去见过的那些父亲和兄长的朋友,以卑官之身,却能在父亲面前言笑自如……

“二哥!”

王旁闻声猛然一惊,从失落中被惊醒,回头看去却见是自家的妹妹王旖【注1】。

十七岁的王旖,继承了母亲那一边的容貌,修长高挑的身材,又有着江南水乡女子的柔美。只是她的举动却一点不像大家闺秀,让开王旁,凑到门缝前眯着眼就想向里面看去。

王旁连忙拦住她,“二姐儿,别闹!”

“里面的是爹爹这两天常提起的韩玉昆?”王旖眼中闪着好奇的目光,“他真的亲手杀了那么多人?!”

“看不出来……”王旁突然醒觉,“二姐儿你到这里作甚?!”

“还说!”王旖气哼哼的说着,“二哥你不是来叫爹爹他们吃饭的吗?娘娘看你去了就没消息,才让我来找的。”她又望望堵在门前不动的王旁,不高兴冲他哼了一声:“话带到了,我先回后院了,今天的功课还没做完呢。二哥你也让爹爹他们快点去吃饭,别耽搁了。”

王旖说罢,就踏着轻快的步子往后院去了。王旁看着性子太过活泼的妹妹,不禁叹了口气。回过头来,伸手敲响了厅门。

注1:实在查不到王安石的两个女儿究竟叫什么名字,即便是王安石写给女儿的诗作中,也没有透露。也只能自由杜撰了。王安石的子侄辈,都是单名,都带一个方。如王旁和王雱。虽然女儿一般不会模仿兄弟的名字,但以王安石的不拘俗礼的性格,让女儿的名字也从‘方’旁,也是很有可能的。

