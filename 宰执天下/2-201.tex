\section{第32章 吴钩终用笑冯唐(八)}

韩冈看得出来,韩绛明着是询问自己的意见,但实际上是希望自己站出来自荐,说一句‘相公有命,韩冈何敢惜身?’

韩绛这是投桃报李,想把功劳送给自己。只要韩冈肯自荐,现在正微笑的看着他的韩相公,便会顺水推舟将这任务交给他。

可对于这份劝降的功劳,韩冈的兴趣却不大。他现在不缺功劳,加上曾经在王安石家中说出的话,与横山有关的功劳,他都不准备要,也包括这一次。

另外帐中众将都在盯着,自己虎口夺食太容易得罪人了。韩冈可是清清楚楚的看到,方才韩绛招降二字一出口,帐中的这些个将领的眼睛一下就亮了起来。

吴逵在环庆路待了多少年了,打起仗来,左近的鄜延路、泾原路都带兵去过,在列的将领中,有几位跟他没有交情的?去找吴逵拉拉交情,攀攀关系,诓得他举城出降。只要今次能在咸阳建了功,保不准日后就是下一个郭逵。

韩冈也知道这些赤佬有多渴望建功立业,封妻荫子、富贵三代的梦想哪个在军中混迹的将校没有做过?立功的机会就在眼前,韩冈真的无意去与他们争夺。韩冈不喜欢这样,韩绛虽可能是无心之举,却是让他成为了众矢之的。

只是眼下的情况,韩绛的好意不能放着不理,众人也都是明白韩绛的心意。自己若是退却了,传扬出去却免不了会让人小看,以为他不敢贪生怕死,不愿为国效命。

赵瞻好像就是这么想的,他看着韩冈犹豫着,皱起眉来。他是标准的旧党士大夫,处理兵变的手法也是按照故事惯例。完全没有反对韩绛打算招降城中叛军的意思,一个是因为过去处置兵变,基本上都有招降这一项过场,另外一个就是全力攻城,城中百姓必然会有所损伤,如果能少一点伤亡,他也是乐于见到的。

“韩冈,难道你就没什么话要说?”赵瞻诘问着。

“兵变当斩,其家人依律亦当斩。但如果招降就不能这么做了……吴逵定然不可轻饶,但关于其余叛贼的处置,既然是招降,总得有个名目,既能示以朝廷的宽大,但也必须体现律法的森严。”

韩冈就事论事,只当作没听明白韩绛对言下之意。不过他说的也不算错,总不能去跟城里的叛贼说,你们家里人都得死。宽大的条件总要开出来。

赵瞻心头有些火气上来了,在他看来韩冈这是故意为难或是想要推脱,才说这些话。

招降之事本来就是骗,骗叛贼投降了再行处置。投降后被杀的叛军、贼人,多得数不清。当年被郭逵招降的保州叛军,最后有几个活下来的,谁也说不准。尽管不会明着杀,但找个借口处置了,朝堂上都不会放在心上。跟贼人讲信用,那就太蠢了。

只是这些事可以做,不能说,朝廷的面子上要说的过去。韩冈却是把话挑明了,直接询问给叛贼开什么条件,这让他赵瞻怎么回答?

“吴逵是明白人。说能赦他之罪,他也不会相信。”韩冈的话更为直率,他的确是要为难人。他不信赵瞻敢跟他明说把人骗来,再行处置掉。

而且吴逵也绝不是糊涂人,他是西军中有数的出色将领,要不然也不会因为他受了冤屈,而引来麾下几千兵将起兵为他讨个说法。

投降朝廷的结果,他自己最为清楚。韩绛、种谔哪个会饶他?他这次兵变毁了多少人的心血,就算并不是他领头起事,但这怨恨还是照样着落在他身上。

赵瞻一时结舌,他无权做决定,也无权开条件,必须让有便宜处事的权力的韩绛来发话。

看见赵瞻无话可说,韩绛倒是挺乐的。虽然韩冈是在驳他的好意,但能把越俎代庖的赵瞻堵得说不出话来,却让他不去在意韩冈的不知好歹。

招降本意就是讨价还价,条件必须开出来,底限也得把握好,韩绛沉吟了一阵,开口道:“吴逵绝不可饶,但下面的士卒,可以只判流放,还有他们的家属,也可以加以开释。玉昆你觉得呢?”

“全凭相公处置。”

韩冈低下头,他当然有想法,但这不是他能插嘴的事,韩冈可不会在这上面犯浑。不过流放的惩罚,却是他想看到的。

三千叛军不能杀,诛杀首恶就可以了.吴逵虽然可惜,但他得罪的人太多,以他犯下的事也不可能饶了他,但下面的兵若是全处理掉就很可惜了。全都是难得的精锐,不是普通的厢兵可比。而河湟那里缺人手,多了三千户能打的屯田兵,总归是一桩好事,韩冈相信以缘边宣抚司的能力,安抚下他们,应该没有什么难度。

等真的招降后,就向朝廷申请,以王韶的面子,以他韩冈在王安石、韩绛面前的地位,应该能成。现在这么一想,韩冈倒觉得亲自走一趟咸阳城也不是不可以。只是韩绛看到韩冈方才的拒绝之意,却也不让他去了,问着下面,“有谁愿意去咸阳走一遭?”

韩绛这一问,下面的将校们顿时兴奋起来。本以为会给韩冈抢了去,没想到韩三识趣,不跟他们争抢,反而把路铺平了,看着韩冈感激颇深。顿时一个个跳出来,一片声的齐齐在说:“末将愿往!”

俗话说,一个和尚挑水喝,两个和尚抬水喝,三个和尚那就没水喝了。十几人抢着要去招降,在韩绛面前闹得不可开交,闹到最后也没有个结果。最后变成了再议,真是一件讽刺的事。

看着众将失落的表情。韩冈也觉得很是有趣。不过他也无所谓了,现在去也不一定有个回音,刚刚击败了攻城的王文谅,吴逵的人望还在,要让他的手下背叛他,现在还不到时候。打上一阵再说,把叛军的气焰打下去,如果能等到他提出的新型投石车投入战场,那就更好了

军议还在继续,不过从招降却变成了如何攻打咸阳。因为有招降的想法在,韩绛不想看到大的伤亡,着眼点便是如何打击叛军的士气,好在招降时能够顺利一点。

在商议中,韩冈没多说别的,只是隐隐的把他的想法透露了出来,希望能将三千叛军的流放地安排到河湟去。

韩冈说得隐晦,除了韩绛、种谔等人,许多将领还是迷迷糊糊的,可赵瞻却一下听明白了。

赵瞻实在不太喜欢韩冈的性子,帮韩绛解围的几次行为,落到他眼里,就变成了溜须拍马。而且韩冈很不给他面子,前面还给他难看,这就更让赵瞻不喜。现在寻到了韩冈的错处,却是一点也不放过,“韩冈!广锐叛卒祸乱关中,即便招降以贷其死,也当是远窜南荒,如何能将此腹心之疾留于关中!”他稍稍眯起了眼,“听王文谅生前所说,你跟叛贼吴逵曾经同行甚欢,是不是有开释吴贼的心思?”

赵瞻说得过火,在列的将领一时有些骚动。韩绛也心生不快,知道赵瞻是借机发作。

而韩冈则是眼神一凛,抬眼与赵瞻正正相对,毫不客气的反驳着:“广锐叛卒震惊关中,若不将其平定,天子也难以安寝。不过说他们是腹心之疾,却是过了一点。不过是一群进退失据的叛贼而已,虽有吴逵主持,但缘边四路中的任何一路,都能将之扑灭。前面燕总管不也是差点就将其剿灭吗,而后虽因故小有不顺,但也在种总管来援之前,便将其围定在咸阳城中。至于与吴逵同行甚欢,那是因为他当时叛迹未显,下官不便妄自猜疑。”

赵瞻脸色渐渐的阴郁起来,一抹厉色在眉头凝聚。韩冈的一番话,最后一句姑且不论,前面的话分明是在指责他举止失措,强逼罗兀前线回师,坏了横山大局。

韩冈对赵瞻的怒视视若无睹,‘你要跟我过不去,别怪我不给面子。’他从来都是人敬我一尺,我敬人一丈。赵瞻要与他为难,甚至说他跟吴逵有瓜葛,那就别怪他直接一巴掌还回去。

韩冈根本不怕得罪身负王命的赵瞻——因为赵瞻的想法和判断,不一定会跟赵顼一样,尽管他在关西是代表了天子。

关键是赵顼那里会怎么想——赵瞻可是逼着罗兀撤军的主事者。如果让赵顼自己选择,在当时,他必然还是会跟赵瞻一样,选择从前线退军,以保住关中内部的稳定,攘外必先安内,这个选择是必然的,所以韩绛才会点头。但人是喜欢后悔的,就算做出了决策,总是会想当初如果换个选择,也许结果会更好。

如果前线不撤军,稳守住罗兀城,当能一举定下横山,继而给西夏的脖子上拴上一根绞索。而广锐叛军虽然直趋南下长安京兆府,可毕竟在罗兀撤军以前,吴逵和他的三千骑兵就已经被围在了咸阳城中,并不一定需要聚集在前线的精锐回师。也许只要剿灭了他们,关中的局势也就稳定下来了。

赵顼会不会这么想?韩冈完全可以肯定,他对人心的把握还是稍微有些谱。就算赵顼不这么想,韩绛、种谔都会让他去往这个方向想的,尤其是韩冈方才已经提醒了他们。这可是推卸责任的好机会!

一旦这一想法在赵顼心中扎下跟来,后悔的心理,就会让赵瞻成为发泄的目标——尤其罗兀撤军又是赵瞻逼着韩绛做决定。从程序上说,这其实并不合规矩,他也没有收到这个权力,只是借助天子使臣的身份,加之韩绛本身又有些灰心丧意给了他机会而已。

还得意吗?韩冈微带冷笑着与赵瞻一点不让的对峙着,后面有的是苦头让你吃!

