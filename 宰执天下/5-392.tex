\section{第34章 为慕升平拟休兵(24)}

宋军挑衅的行动,萧十三并没有出动主力,仅仅派出了一部分骑兵,阻拦宋军骑兵的接近。

这一天剩下的时间,大小王庄中的辽军全军上下眼睁睁的看着宋人将双方营垒的间距缩短了两里多。

夜幕降临时,宋军留了四千多人马在新修的营地中驻守,然后主力退了回去。

在这半rì的时间里,宋军也不只是单纯的修营地,他们同时还将新营和主营连在了一起。新修营地和主营之间的道路外侧,还有着很大一片由鹿角、壕沟、陷坑和铁蒺藜组成的防御设施。

本来就是土质松软的田地,又有了密集的陷阱藏在其中,就像是变成了一片荆棘林。想要穿过去,必须付出不小的代价,何况全程还是在宋军重弩shè程内。

如此一来,从第二天开始,宋军的主力就不必在主营外列阵,完全可以前进到新营地再出营列阵。

两里路,寻常自然不算什么。可走在敌军虎视眈眈的战场上,和走在有保护的营垒中,差别就大了,体力和时间的消耗都不是一个等级。

两万辽军亲眼看见宋军将他们的营垒前移了两里,接下来会怎么发展,再蠢的人都能猜得到。到了明日,他们肯定会再往前修筑一座新营地。到了后日,宋军一出寨,来自床子弩的铁枪和霹雳炮的飞石就会砸在自家的头上了。

乌鲁回来时已经是精疲力尽。

整个下午,他都带着族中的子弟兵在营地外与宋军的骑兵遥遥对峙,一边还眼睁睁的看着宋军的新营盘从无到有,在眼前修筑起来。

这一趟的差事,时间仅有半天,乌鲁和他麾下的子弟兵都是垂头丧气,仿佛连着在外巡游了半个月。

不仅仅是身体疲累,心也一样疲累。看着敌人耀武扬威,毫无顾忌在眼皮下修筑营地,而几千契丹勇士,就只能在旁干看着,士气跌得一干二净。甚至都想着回去后怎么见人?

回到了自家的营区,已经成了乌鲁智囊的老胡里改迎了上来。白布裹头,弄得脑袋大了一圈。

老胡里改之前在拉动配发下来的神臂弓时出了点意外,被断裂的弓弦在脸上抽了一记。几百斤力道的断弦抽在脸上,就跟被刀剑砍过一般,脸上的伤口深可见骨。幸好大小王庄这里有军医,及时给救治了。现在脑袋都包扎了起来,细麻布带绕了一圈又一圈,倒是避开了出营巡视的差事。

帮忙牵着马,老胡里改问着乌鲁:“宋人把营寨给修起来了?”

“嗯。”乌鲁根本就不想开口,只从鼻子里应了一声。

“就在你们眼皮子底下?”

“嗯。”

“离大小王庄只剩五六里了?”

“嗯!”乌鲁的神sè越来越不耐烦,心中的烦躁让他恨不得拔出刀来砍上一番。

“唉……”老胡里改没再多问了,只是长叹一声,“这一回只能走了吧?”

“……早就该回去了,这一回来南朝一趟,子女金帛收获了多少?本来能舒舒服服过上几年快活rì子,可现在呢……”

要不是畏惧耶律乙辛和萧十三,他们这些部族军早就撤回草原上享受这一次入侵南朝得到的战利品了。可萧十三偏偏不知道见好就收,一直拖在这里,枉死了多少战士,损失了多少战马,好些部族都已经是入不敷出,亏损严重。

“但也要个贱种肯走啊!”乌鲁眼中血红一片,对拦着他的萧十三恨不得将其寝皮食肉。

“的确。”老胡里改点着头,“之前退保代州的方略,多半是依从尚父的吩咐。现在萧枢密不肯走,只怕是畏惧尚父事后怪罪他。”

“那是他的事,凭什么拉我们一起下水!”

“他是主帅嘛,而且也有人愿意听他的话。不过这一回情况不一样,”老胡里改左右看看没有生人,就凑近了乌鲁,“正是要求退军的时候!”

乌鲁神sè一凛:“真的可以?”

“这是理所当然,宋军在眼前修筑营垒,他连出兵都不敢,这仗还怎么打下去?我们留在代州又不是要多看几rì山水。不过……”见乌鲁一个劲的点头,胡里改话锋一转,“萧枢密今夜也有可能不走,而是调人去夜袭宋军的营垒。”

“爱去谁去,我可不会带着儿郎去!”乌鲁立刻愤愤然说道,宋军的营垒要是那么好攻的话,也不会变成现在的局面,“说不定那就是宋人的陷阱。”

“谁说不是呢。萧枢密真的有那个打算,到时候就让他带着他家的兵马去量一量宋人的营寨到底有多坚固。”老胡里改更凑近了一点,声音也更为低弱,“小王庄这边都是同样的心思,大王庄那边少说也有一半……”

营帐前的篝火不时的噼啪作响,夜幕下辽军营地如同子夜一般压抑。

来到前线的各支部队的将领,此时都聚集到了萧十三的中军大帐中。

高高矮矮二三十名将领,眼神冷淡甚至透着敌意。没有任何意义的与宋人周旋,到了今rì的避战,已经将他们最后一丝耐xìng给消磨掉了。之所以还不敢发难,那是考虑到耶律乙辛,而不是萧十三本人仅存的那点声威——至于张孝杰,则被所有人给忽视了。只有耶律乙辛做靠山,萧十三加上他也不会改变什么。

他们并不是要萧十三拿出个应对的方略来。这些日子以来,萧十三的决定给他们带来了太多的损失。

他们只需要萧十三明确下一步做什么,是战还是走?

这决定了他们之后的态度。

萧十三的视线从麾下的一众将领们脸上划过,终于有了决定。

……………………

战鼓敲响了很久,战旗也已经牢牢地扎在了地上。

但对面的敌人却没有任何反应,而是缩在他们的龟壳里死死不肯探出头来。

这么说或许有些不确切,宋军的骑兵还是被辽军的远探拦子马拦在了外围,没人能多靠近辽军的营垒一步。

可这样的作战,却像是宋辽两军颠倒了个个儿,宋军不像宋军,辽军也不像过去的辽军。

失去了进攻意志的契丹人,到底还能剩下什么?

半rì的时间,已经让所有参战的宋军官兵更增添了一份信心。

纵然对辽军的畏惧依然存在,纵然还有很多人觉得避而不战是辽军主帅的骄兵之计。但更多的人已经确信,将辽贼驱逐出大宋国土的rì子已经为时不远了。

收回向北面眺望的目光,韩冈领着几名幕僚,继续巡视着他的营垒。

夜中的宋军大营,一簇簇篝火映着人们的脸庞,轻快的笑语传遍了营地。

“军心可用啊,枢密!”黄裳在耳畔的低语,藏不住内里的兴奋。

就算之前官军一直是高歌猛进,从太原府的太谷县一直杀到了距代州城不到五十里的地方,但只要辽军的主力还存在,就没人敢说这一战是必胜之局。

可辽军这样不利的形势下依然避免开战,这已经足以让黄裳看透了敌方的虚实。

“不到最后一刻,不切实拿到胜利,就不要放下心来。”韩冈的笑容中有着无奈,“军心可用,可不是军队可用。”

巡视过一堆堆篝火旁的营帐,多少士卒都向他高声宣誓,将入寇的辽贼剿灭。可这依然没有让他看高麾下军队的实力一分两分。

不仅是辽军在避免硬碰硬的战斗,其实双方都在避免正面决战。即便是在太谷城下的那一战,韩冈都没有让他麾下的主力——也就是来自京畿的步军——跟辽军正面交锋过。

至今为止的大战小战,基本上都是河东军为主力的战果。这虽使得宋军的士气没有因为连续作战而无法避免的下降,可也让京畿禁军的信心和求战yù望提高到了极点。

而韩冈始终在避免将他们派上与辽军对垒的战场。

这是韩冈小心的一面。

熟练工和新手的差距是很大的。韩冈无意用人命来换取京营禁军的作战经验,所有也只能让人数少得多的河东军成为作战的主力。

“我更希望萧十三能够立刻撤回代州……”韩冈边走边对几个幕僚说道,“吕吉甫那边已经差不多该到了黑山脚下。一旦抄截耶律乙辛的老巢功成,萧十三必然要带兵回救,那时候,就是大同也不是没有指望。”

“可那也要他们的粮草能跟得上才行。”可能是门户之见,黄裳以下的一众幕僚都不想看到吕惠卿的名声更加响亮。

“不用担心。”韩冈自是知道,长距离的行军时,粮草的消耗量当然是个巨大的数字。不过由于为了维系兴灵和边境的联系,西夏在沿途的寨堡都存有数量不小的一批粮草,“辽人若没有糊涂到极点,只会增添储粮,而不会减少。”

韩冈正结束了对军营的巡视,准备回帐时,一名骑兵赶到了近前。黄裳上前跟那名骑兵说了一声,而后黄裳又转回来,低声跟韩冈说了些什么。

韩冈转回来,对幕僚们笑意盈盈,“没什么大事,也就是萧十三跑了……”他轻笑,用着更加坚定和昂扬的语气大声道:“辽贼宵遁!”
