\section{第39章 遥观方城青霞举(三)}

【第三更,下一更在凌晨,请书友明天早上再看。】

韩冈在方城县待了两天,跟汝州知州方静敏坐着新修好的有轨马车在六十里的轨道上跑了一回,便掉头返回襄州。

摇晃的灯光下,韩冈低头翻看着自己的文稿,而王旖坐在他的旁边,也在看着一部草草装订的书稿。她神情专注,嘴角边带着清浅的笑意。

韩冈手中松散的稿纸上全是点画删改的痕迹,这是他用炭笔写在白纸上的初稿,而且还是从左至右的横排书写。十年来,眼睛里都是看的竖排文字,横排写起来甚至都有些不习惯了,不过日后如果是关于数学、物理方面书籍,插进公式后,感觉还是横排比较合适。

不过拿在王旖手中的第二稿,则是韩冈重新用毛笔誊抄过一遍,已改为了竖排。毕竟韩冈现在想做的只是科普而已,通过新书向世人灌输自己的理念。这就决定了他不能在文稿中插进让普通人感到难以理解的公式和方程。

韩冈依稀还记得,后世某位著名的科学家在他那本同样著名的科普书籍中曾经说过,科普书中公式每多上一条,销量就会减少一半,韩冈只盼望着自己的书能传播得越广越好,该留在专业书籍中的,最好还是留在专业书籍中。

就像当初他利用《浮力追源》,作为他执掌军器监后的第一声冲锋号角,自此改变人们对世界的认识,并奠定自己在工器营造上的权威地位,也顺便挖上几个坑,达成一些政治性的目的。

韩冈现在想要做的,也是打算使用刚刚写好的新书,在现在和未来,在政治和学术,在朝廷和民间,在天子、朝臣、士人和百姓们心中,进一步树立自己的地位。

不过要达成这个目的,一本书是不够的,所以这一次,韩冈准备拿出来的并不是一部书,而是两部。

一部是关于地方官府应对灾疫的针对性的手册。这是韩冈一直想写的。远在他撰写军中卫生条例的时候,就有了这个想法。

一个幅员万里的国家没有说哪年没有大灾大疫。但大部分的亲民官——直接治理着一方水土,为天子牧守亿兆元元的官员——却都是进士出身,诗赋经义水平不错,为了撰写策问,也有去学习农事水利,比如《齐民要术》和《水经注》等等,但一旦遇上干旱、洪涝、地震、蝗虫、瘟疫、饥荒这些灾害,到底该如何应对,谁也不可能有地方有机会去系统性学习,而且也没有一本合适的参考书,一般只能靠经验、靠惯例,比如免税放粮什么。

对于普通的灾情,免税放粮勉强还能派上用场,最多也就多死些百姓,多一两个乱葬岗的事。只要灾民不揭竿而起,流民人数不过千、不过万,地方官员倒也不会太在意。

可一旦灾情严重,波及数路,绵延数载,牵涉到数以千万的百姓,道上流民以十万计,那么这点可怜的应对手段,当然也就远远不够了。那时候的灾民,就像已经将堤岸顶出道道裂缝的洪水,随时都能破堤而出。

所以富弼、韩冈能平平安安的安置下几十万流民,才会成为人人传颂的奇迹,因为其他人几乎不可能复制他们的成功,没有那个能力!没有那个手段!没有那个经验!

也便因为如此,所以韩冈才会写下这本书。其直接目的就是为亲民官们所准备的。遇上大灾大疫,到底该怎么安抚百姓,怎么防止灾民中爆发疾疫,乃至在瘟疫爆发后,该如何处理,都可以参考书上的条款。

不过一人计短两人计长,韩冈几年前还是准备自己编订条目,写下一个大纲。再向天子加以申请,集合众人之力来编纂。只是眼下的现状,韩冈只能亲力亲为。一个人闭门造车,粗浅是肯定的,但韩冈还是很有些信心。这信心来自于他本身的声望,也来自于书中的内容。

另一本新书则是文人的惯例。类似于随笔,是被称为笔记小说的形式。

这个时代,文人总会将身边的人和事,以及一些道听途说的传闻,加以记录,最后编纂成册。有的说玄怪,有的说历史,有的记录言谈,有的描写人物,甚至也有记载制度、政事的笔记。更多的笔记则是以上几项的集合,也就是杂记。韩冈的书架上,这样的书就有不少。

《世说新语》算是早期的笔记,唐时的有牛僧孺的《玄怪录》,段成式的《酉阳杂俎》,刘肃的《大唐新语》,五代有孙光宪的《北梦琐言》,而进入宋代后,则为数更多,比如陶谷的《清异录》,钱易的《南部新书》,杨亿的《杨文公谈苑》,欧阳修的《归田录》,这些书多达数百卷,占了整整一面书架。

即使在后世,这样的著作也很受欢迎,甚至流传极广,同时更是极为重要的史料。韩冈前世不研究历史,但他走南闯北,消耗在路上的时间很多,旅途上总得有些打发时间的东西。就像沈括的《梦溪笔谈》,他就曾经翻阅过——虽然现在还没有成书的样子,而韩冈本人也记不清其中的条目了。

韩冈之所以会用笔记小说的形式来撰写科普书籍,一个是笔记小说在士人中容易传播,另一个,则是他来自于后世的记忆有很多零碎的科学常识,基本上很难撰写成某一方面的专著,但作为笔记,体裁却正巧能与韩冈零碎的记忆配合得上,甚至可以说相得益彰。

从王旖的表情上,就能看得出,韩冈的这本还没有命名的笔记,还是很有些吸引人读下去的能力的。

她在旁边翻着,神情专注,连韩冈放下手上的书稿,开始盯着她看,都没有察觉。

科普性质书籍其实很受欢迎,能多了解一点,与人聊天时也有谈资。笔记小说也同样受人欢迎,同样是因为能增广见识。当两者相互结合的时候,自然而然的就有了吸引力。

韩冈的这本以窗外的一株桂树起名作《桂窗丛谈》的笔记中,分为生物、医药、物理、化学、算学、地理等几个大篇目,将一些科学常识记录下来,掺入一部分理论,同时与气学和格物之说联系在一起。

王旖翻着的正是生物一篇,眼睛盯着稿纸上的细密小字,自言自语的:“螟蛉当真不能变成蜾蠃?”

“螟蛉有子,蜾蠃负之。”韩冈将手上的文稿理了一理,笑了一声,“诗三百,先圣只是编修而已,也不是没错的。”

王旖抬起头,带着笑:“官人,你当真挖过土蜂窝?”

蜾蠃俗称土蜂,韩冈点点头:“蚂蚁窝都挖过,何论土蜂?”

“那是小时候的事了,也不是为夫一人,是两位兄长带着出去玩的。听说了蜾蠃,也就是土蜂收螟蛉为义子的故事,就想知道到底是怎么念的七七四十九遍‘像我、像我’,就一口气连挖了几十个土蜂巢。却发现里面可不只是有螟蛉,尺蠖、蟋蟀之类的虫子都有,而且全都是活的,上面还有更细小的虫子一口口啃着。

后来为夫就仔细观察过,发现这些虫子都是被土蜂捉来,先用尾上针扎上一记,然后才丢进窝里。虫子被蜇了一针后,就如喝了华佗的麻沸散,不能动弹,却是鲜活的。土蜂幼虫从卵中孵化出来后,就能有鲜肉吃,不至于因时间而腐坏。”

韩冈说得活灵活现,这些话在他的笔记中也都写了出来,名人轶事也是世人喜欢看的,编一下也不是多麻烦。

“这也是格物致知?想不到官人小时候就能暗合圣人之道了。了不得啊……”

“去其伪,查其真,这就是格物,只是小时候不知道罢了。圣人说的道理,本就是在寻常处,哪有艰深难懂的?只是不易学而已。”韩冈向妻子说道,“其实这也不是为夫第一个发现,梁时的山中宰相【陶弘景】很早就在书中了写明了。”

“官人让人造的水银镜,也是从陶通明【陶弘景字通明】的书上得到的启发吧?”

“不只是他,提到汞能融金化物的有很多书,知道的也有很多人。一方面,汞能融金、融银,鎏金鎏银都是用汞——汞这一字,可拆为工和水。水,是指其常温为液体,而这个工,一个合其音,就是指其可用于工匠之用——其实是一部分得自于道人,一部分得自于匠人。只是匠人知其然而不知其所以然。而道人则是故弄玄虚,用些模棱两可含含糊糊的话语,来招摇撞骗。”

韩冈还是一贯的看不起释老两家,视其为外道。王旖抿嘴一笑,听着韩冈毫不客气的说着道家的不是,“这些道人,一心一意的去炼丹,竟没有一个想到有用于国的。”

“奴家可是感激道士,”王旖的剪水双瞳望着韩冈,“若是没有道士,奴家可是遇不上官人了。”

韩冈老脸一红,他排斥佛道,却把自己虚构的救命恩人给说进去了。不过这也算不上什么,“为夫说的是那些只顾炼丹药的道士,生生浪费了多少好东西。”

