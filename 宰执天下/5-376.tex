\section{第34章 为慕升平拟休兵(八)}

按照预定的计划派出了麾下最为的队伍,但主导着河东战局的萧十三和张孝杰却并没有焦急的等着前方传来的消息,而是在下着棋。

两人都是正襟危坐注,像是两位大国手争夺天下第一的名头。但这边萧十三刚刚缓了一手,那边张孝杰就送了一条大龙过来。表现出来的棋艺,比起夏天在街头纳凉时赌棋玩的平民都差了许多。

萧十三也不看棋盘,随手便落下一子:“两千三百甲骑具装,加上五个配了铁甲的千人队。想必忻口寨的枢密相公派来的那些苍蝇也该收收翅膀了。”

提到韩冈时,萧十三语带讽刺。可是为了解决韩冈派出的兵马,他却是把底牌都掀出来了。

经过了近一个月的休整,萧十三终于可以把他麾下因为之前的鞍马劳顿而失去战斗力的给调出来派上用场。

两千多具装甲骑,和五个千人队总兵力四千余的带甲骑兵,正尊奉他的军令向代州西南四十里的一处官道上的小镇挺近。这六千多兵马,是用来组建代州的外围防线,清剿宋人越来越嚣张的游骑探马。

“但宋人的铁甲更多啊。”张孝杰落子提子,却忽视了另一角处能收获更多的机会,“换在二十年前,这五六千铁甲甚至能让我们直接打到大名府去。现在只能打苍蝇。”

“铁甲现在不值钱了。”萧十三抬起头,“还得多谢那位韩枢密。”

自从板甲和水力锻造出现在这个世界,大宋的钢铁制造业一下上了一个台阶,但辽国在这方面始终没有拉下太远。不论是遣人窃取,还是收买有关军事的一系列发明在宋国普及后,很快就会传入辽境。飞船就是最有名的例子,用大辽天子的让飞船之名传到了极北的荒原上。而铁甲则是更有实用例证。

虽然说由于制造工艺、匠人水平以及管理能力上的差距,使得宋辽两国的生产力依然差距很远,甚至是越来越远。但比起过去,辽国的军事生产能力还是有了几近十倍的提升。如今的辽军正军,最次也会有一幅锻铁的护心镜,而宫分军、皮室军更是大多数人都能拥有一副铁甲。

在过去,一场战争中,几乎看不到十万甲士出战的场面。可如今,难易程度姑且不论,宋辽两国都是能拉得出十万铁甲

耶律乙辛能压服辽国境内无数反对力量,一方面是他在政治上的手腕了得,或打或拉,使得无人能聚合起足够实力的反叛力量。但另一方面,也是更为重要的一个因素,他控制得最为牢固的南京道,是辽国工匠最为集中的地区,出产的大批量铁甲让耶律乙辛控制下的军队拥有更为强劲的战斗力,能轻易碾压任何反对者。

“铁甲虽然好,但也要用对地方。”张孝杰随手落下一子,“而且宋军怎么用他们的神臂弓你也不是没看到。更别说还有破甲弩。”

“破甲弩没什么大不了的,小心点就是了。至于神臂弓,我不是也让人带着了吗?”

破甲弩的威力虽犹强上神臂弓一筹,但对身着铁甲的战士的威胁,也必须在三十步以内,否则即便能甲叶,也伤不到几分皮肉。而神臂弓,倒是让辽军感到头疼。宋人骑兵的实力不值一哂,可当他们不惜损耗让骑兵都带上了上好弦的神臂弓后,马上交锋时宋人的骑兵便开始占了上风。

不过这一回萧十三也大方了,给所有出战的骑兵都配上了神臂弓。反正在代州等地的武库中,神臂弓的数量也是数以千计。虽说一次百人队的出巡就能有十几张弩的损失,虽然让萧十三心疼肉疼,但还是能够承受得起。

张孝杰停了手,大声叹着气:“终究还是不如宋人财大气粗。宋人一砸钱,我们要么跟着一起砸钱,要么就得用命来换。”

“但不这么又能怎么样?”萧十三反问,顺手又落了一子,不过这一着下得漂亮,正好将张孝杰的一步给切断了。“越退缩,宋人的气焰就会越嚣张。只有毫不妥协的顶上去,才能逼韩冈认清大局。”

“韩冈是聪明人。他现在的做法就是步步进逼,多半是想逼到代州城下,不想冒野战的风险。”

从开战伊始,直到现在,快两个月过去了,早就过了布局的阶段,而河东的宋辽两军除了太谷县一战外,依然没有大规模交锋的记录。这其中有萧十三和张孝杰将队伍收得太快的缘故,但要说没有韩冈的控制,怎么也不可能会是现在的局面,足可见韩冈他并没有与大辽决一雌雄的信心。

“不能给宋军攻到代州城下的机会。”萧十三神sè严肃,将手上的棋子丢回旗盒。“一旦宋军能够包围代州,那么河东战事就会转入宋人最擅长的城池攻防战上。在宋军的攻势下,代州城根本支撑不了多久。”

从太原一路退到的代州,已经不能再退了。尚父的命令是必须要执行的,而且是不折不扣的执行。代州必须要守住,这是与宋人谈判的最为关键的筹码。

尽管底下也有人叫嚣着要重现燕京大捷的荣光。但萧十三可不指望韩冈会跟南朝的太宗一样蠢,只顾攻燕京城却不提防古北口等燕山要隘。

从韩冈现在的布置来看,那是比森林里最狡猾的狐狸还要狡猾三分。沉稳得不像是一名三十上下的年轻人。

……………………

韩中信正领着一队骑兵,在代州的原野上慢行。

他不似秦琬在代州军中有根基,也没有过往的战绩撑腰,光靠韩冈家仆的身份,也得不了人心。坐在寨子里面等消息,只会让人心越来越疏离。所以他必须出来。

秦琬和韩中信顺利的进驻了土墱寨,除了三人掉队,一人落马外,并没有受到任何干扰。

那一支具装甲骑也并不是冲着他们来的,而是半路上的一座村寨。距离代州四十里,距离土墱寨也有三十多近四十里。随着这一支队伍的向前进筑,宋军斥候的活动空间又被压缩起来。

不过在韩中信这里,辽军既然还离得远,就没必要缩在寨中。

在代州地界,一座村寨,一个镇子,都有着即高且厚的围墙,这就让宋军在战事不利的时候,可以退守到最近的村镇中,等待援军的到来。

韩中信便是以此来说服了秦琬,让秦琬答应他领着一队骑兵出外巡视。

而他这一出场,便幸运,或者说不幸的遇到了一支人数相当、正准备归队的远探拦子马。

辽军的远探拦子马实力惊人,身高体重,而且还有足够的经验。韩中信仅仅是一瞥之下,就逼得韩中信不拿出神臂弓就别想赢过他们。韩中信都想跪下来感谢韩冈在军械上的慷慨,神臂弓这样的至宝护身,不管与辽人在马术上有多少差距,一箭过去就拉平了。

调转马头,迎面而上,韩中信第一个拿起了提前上了弦的神臂弓,用最快的速度在箭槽中放上了短箭。紧随在韩中信身后的骑兵,取攻上箭的速度丝毫也不比他慢。

举着神臂弓,这一支斥候小队飞速的接近对面的敌人。只是当他们终于看清了敌人手中平端的兵器后,无不大惊失sè。

“神臂弓?”韩中信差点就要从马背上跳起来,迎面而来的辽骑手中,竟然都是一张张神臂弓。

那前段的铁质圆环,那原木sè泽的弓臂,几近三尺的横宽,无一不是神臂弓的特征。

骑着高大的河西马,身上的甲胄也是最为显眼,韩中信立刻就成了箭矢对准的目标。箭矢如雨,转眼就把他扎成了刺猬。

其中一支如毒蛇一般扎进了韩中信腰间,根本就没有反应的时间。不过韩中信的盔甲内外两重,内衬皮铠,除非是箭簇抵着盔甲扣下牙发,否则弩矢依然穿透不了他的甲胄。但这一下也让韩中信惊出了一身冷汗。

不过在此之前,他也将手中的刚刚过的神臂弓随手丢掉,换上了一架新弩。

宋军和辽军不同的一点,就是他们杀起战马来根本毫不犹豫先上阵的宋军几乎是被集体灌输了这个意识。从韩中信开始,所有宋军骑兵的第一箭都是冲着战马去的。而更重要一点,抛下神臂弓时,宋军也远比辽军更爽快。

对辽军的战士们来说,一柄来自大宋军器监的神臂弓,拿回老家后多半能换来一匹毛皮油光水滑的好马,或是十只母羊。他们很难像韩中信和他的下属一样直接松开手指,让弦鸣声尚未停歇的神臂弓下,再顺手抄起鞍后的铁锏,然后借着战马的速度,将铁锏向着最接近自己的那名控制不住受伤后变得疯狂起来的坐骑的辽兵,用力挥过去。只是一瞬间的迟疑,就决定了最后的结果。

沉沉的一声闷响,右手中也承受了一记猛烈的反弹,在飞驰而过的一瞥中,让韩中信清楚的看见了目标凹下去的头盔,以及从眼眶中像燃着的烟花般喷出的右侧眼球。

双方都是不到二十人的小队,交锋前人数相当。可一回合之后,还在马上的骑兵人数就已经变成了十五对十。而当韩中信和他的手下士兵再抽出第二张、第三张神臂弓后,骑在马背上的就只剩下了宋军。
