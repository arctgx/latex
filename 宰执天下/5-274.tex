\section{第30章 随阳雁飞各西东(五)}

‘这蔡确到底在想些什么?!’

当天稍晚的时候,韩冈通过自己的渠道,了解到了蔡确入宫后向皇后说了些什么。

虽然不可能是每句对话都一清二楚,但蔡确话中的大体内容,传出来的三五句也一样概括了。

韩冈真没想到,蔡确不仅仅是跟吕惠卿过不去,之后还顺带给了程颢一棒子。

这是帮自己吗?韩冈可不这么认为。

自家的确有跟王安石、程颢一较高下的打算,但蔡确横插一杠算什么?!

学术之争,自然是争于学术。韩冈文斗武斗都不怕,该用的时候也不会心慈手软。但现在明明是能在学术上堂堂正正击败对手,为什么要用权术来攻击。徒贻人口舌,坏了自己和气学的形象。

想想皇帝前段时间偏帮新学,连禁令都出来了,这在士林中帮了自己多少的忙?多少人觉得新学是理屈词穷,才只能托于天子之威?日后跟新学吵架,都是个能一下翻盘占上风的好理由。

现在王安石和程颢没玩盘外招——韩冈也不认为流言跟程颢有何瓜葛——他怎么能先下手?蔡确等于是拿污水往他韩玉昆身上泼。

也好!韩冈阴沉沉的想着。这样一来,他进言留吕惠卿在陕西也没什么关系了。

蔡确不是要蔡延庆去接手吕惠卿的职位吗?正好,韩冈与蔡延庆也有交情。但并不是说蔡延庆上任,吕惠卿就需要回京。宣抚使司在陕西,吕惠卿做了宣抚使后,本也不方便再插手京兆府的内部事务。

“爹爹。”韩冈家长女清脆的声音从门外响起,随即一个小脑袋探进书房,“娘说要开饭了。”

韩冈一笑起身,抱去女儿去吃饭。

家里的年节气氛已经很浓了。

从巩州乡里送来的年货今天白天的时候进了城,整整五车的各色杂货,吃的用的,全都给备齐了。给孩子们的玩具、衣物,更是整整装了半车,甚至还给还没出生的第九个子女都准备好了小衣服和长命锁。就这么一箱箱的送进了库房。

各色绢绸、棉布则是一匹匹的从库房中搬出来,家里负责缝补裁衣的一班婢女天天挑灯赶工,以便能在过年的前几天发下去。

桃符、门神、烟花、灯笼,也一项项的备齐。不过由于天子重病的关系,今年韩家其实已经是缩减了很多的布置——这一点,普通百姓可以无所谓,但朝廷重臣,则尤为需要注意。

吃过饭,检查子女的学业,韩冈的生活与平时没有什么两样。

次日一早,韩冈便照常先去了都亭驿。

馆伴使的陪客工作一日都不能跳过。

辽国使团上下,这些天来在驿馆中好吃好睡,已经养得元气尽复。

萧禧的气色更好,面色红润,声音洪亮。

大声说,大声笑,前一句跟韩冈说契丹人在草原上如何围猎,后一句又赞韩冈的种痘法让辽国齤保住了多少男丁。

与韩冈相处久了,副使折干也逐渐变得挥洒自如,紧接在后,就开始大谈特谈旧年围剿五国女真的战绩。

萧禧和折干早已明白,宋人越是看重自己,就越是显得他们心虚。而且心虚的原因也找到了。剩下的,就是看看北面是否已经确定要发动了。

韩冈则是同样不拘言笑,说起天南海北的风物和地理,是如数家珍。当他将谈论的话题渐渐引到辽国国内,尤其是配合着折干的话语,渐及女真各部所在地域,萧禧和折干两人脸上的笑意便一点点变得僵硬起来。

“据闻按出虎水【今黑龙江哈尔滨市东南阿什河】多产金,鸭子河【松花江】畔的头鱼宴韩冈亦是闻名久矣。”

“五国部所在之处山林茂密,但传闻再往北,则是多黑土,多沼泽,却是一片平原。”

“听说从按出虎水再往北去数百里,夏至无黑夜,冬至无白昼。一年间昼夜变化,远甚中原。”

“按出虎水入鸭子河,鸭子河又汇入黑水【黑龙江】,之后黑水转为北流,向东北两千里入海。”

“据闻黑水入海口之东不远,有一巨岛,南北上千里,东西则窄得多。其南端与东瀛虾夷岛近邻,虾夷岛再往南,就是倭国了。”

韩冈说得开心,萧禧还能配合着在笑,折干却不说话了。

最后萧禧维持着笑意不变,对韩冈道:“海客传闻,多有荒诞不经之处。正如《山海经》中所言诸多怪兽异人,又有几人亲眼目见?”

“说得也是。”韩冈点头,“传闻总是有着夸大之处。但凡事若总要亲眼目见,有时候却是挽回不了了。”

萧禧眯起眼睛:“内翰似有所指?”

“此事自有所本。”韩冈叹道:“日前一个西域小国劫掠了鄙国的商旅,其国主不信鄙国能为商人出兵。但半年后鄙国官军出现在其国中,这位国主倒是信了,可惜迟了。”

萧禧怎么听怎么像是夜郎自大的故事,看着韩冈,觉得他真会说故事。

韩冈当然也是说故事。就任甘凉路后,王舜臣很快便提兵往西域去,动武的借口,太祖皇帝的卧榻之侧就足够了,何须商旅被劫掠?

上一次甘凉路回报,说是已经稳定了伊州【哈密】周边,镇齤压了好几个西州回鹘的部族,而下一个目标就是高昌,等开春后便动手。由于只是千余官军加上甘凉路的汉蕃联军,消耗并不大,西州回鹘的实力又不强,得到天子批准时,朝中并没有什么反对声。攻下伊州后,在朝野也有了小小的轰动——班超张骞的名气还是很大的。可是等到天子发病,朝廷上几乎是在转瞬间将这件事给忘了。

韩冈一边与萧禧聊着天,一边计算着告辞的时间。但很快,一名亲随匆匆进来,告罪后附耳对韩冈说了几句,说是外面有内侍奉旨传话,招韩冈即刻上殿,不过为防辽人知悉内情,便托韩冈的亲随转告。

韩冈摇摇头,根本不瞒萧禧和折干,起身告辞:“失礼了,朝廷有急事相招,韩冈请先告辞。”

萧禧脸上的笑容不再僵硬,而是变得深沉了,他与折干交换了一个眼色,便道:“内翰请便。”

崇政殿议事。韩冈能身列其中,乃是以翰林学士的身份。执掌内制、为天子私人的翰林学士,只有参与到朝堂大政中,才能撰写让天子满意的诏书,不至于在遣词造句上有所讹误。韩冈虽然不带知制诰,但翰林学士就是翰林学士,也没有明文规定,便钻了这个空子。

不过本质上,还是韩冈在军事上有着足够的发言权,而皇后也给他了充分的信任。所谓依靠翰林学士的身份列席,也只是为了让他立足于崇政殿中时,能显得更为名正言顺。

但韩冈还是尽量谨守本分,皇后不问便不会开口,也不会对军事以外的其他方面多说一句,无论人事、政事,避免干涉到两府的职权。其实也就是备咨询而已,跟他的资政殿学士的贴职相配合。

这些日子以来,基本上已经形成了惯例。可今天的紧急召唤完全出乎预料,在路上,韩冈基本上已经大体上猜到了究竟发生了什么事,只是不知道事件发生的具体地点。

入宫上殿,韩冈匆匆行过礼,顺道将诸宰执的神色收入眼底,便立刻发问,“究竟是出了何事?”

“韦州溥乐城被围。”章敦沉声道,“据报兵力超过两万。”

答案不出所料,地点也在预料之中,“原来是在溥乐城。”

韩冈自然记得,多日前环庆路上报伏击了犯界辽人的地方,正是为韦州外围防线中最为靠前的据点溥乐城。

向皇后心神不安,双眉皱着,问道:“学士,这可是辽人的报复?”

换作是赵顼这么问,韩冈只会反问一句‘是与不是又有何干’?但对眼前的皇后,就不方便这么说了。

“不是,纵然辽人这么说,也只会是借口!”韩冈一口咬定,“辽人本有犯界之意,无论是否有溥乐城的伏击,他们都会南下,否则就不会有萧禧为使。且若不是辽人攻到溥乐城,如何会给城中守军伏击的机会?乃是贼喊捉贼而已。”

“那以学士之意,当如何处置?”

“命环庆路出兵逐寇,稳固韦州,保住溥乐城。至于如何做,赵禼自知。”韩冈停了一下,又补充道:“不过辽人若有心犯界,绝不会局限于区区韦州。泾原、银夏,乃至河东,都要做好防备。尤其是泾原和银夏,加上环庆,此三路与辽人的交界处相隔甚近,同时受到攻击的可能性很大。”

“之前章卿也这么说。还提议设宣抚司,以枢密使吕惠卿为宣抚使,统掌陕西兵事。以此可以震慑辽人,以示决心。”向皇后主动将韩冈上殿前的议论说了出来。

之前进殿时见蔡确脸色不对,韩冈已经有了几分猜度,现在倒是证实了。看看脸色更加难看的蔡确,还有看着脚前,不与自己对视的章敦,韩冈也把握到了许多事。

立陕西宣抚司,留吕惠卿在陕西。是他跟章敦、薛向的约定,也准备回报蔡确的昨日之言。没想到章敦没忍住,倒是先说了。
