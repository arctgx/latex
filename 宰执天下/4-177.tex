\section{第27章 鸾鹄飞残桐竹冷(中)}

【下一更会很迟了,大约在凌晨,朋友们明天早上起来看吧。】

“张载病卒?”

听提举皇城的宋永臣的汇报,赵顼一下放下的手上奏章,神色也变得沉重起来。

前两天还特意赐药与他,还让御医为其医治,这份殊恩基本上都是侍制以上的重臣才有资格享受,想不到还是这么快就病故了。

赵顼是听过张载讲学的。过去张载担任御史时不提,他复官后在崇文馆中任职,赵顼见到他的机会很多。

尽管专门为皇帝讲习经义的经筵官,张载没有做过,赵顼也不便任命,但也曾多次在君臣问对的时候,听过张载说起他关于对易经等儒家经典的诠释。

有许多地方,赵顼觉得他比王安石说得要透彻。而据说是挂在横渠书院院墙上的一篇《钉顽》,只有区区两百余字,赵顼看了之后,却是为之击节。融孔孟要旨为一炉,就算是王安石的三经新义中,也没有说得简明扼要,却又鞭辟入里。

赵顼在福宁殿中黯然兴叹,此人病故,世间又少一名儒。

尽管一干大儒本身很难做到高位,能如王安石一般的官运亨通,可以说是凤毛麟角,就算是韩愈,都可算是仕途畅通了。但他们在官场、士林和民间的声望,却远远超出他们身上的官职。

张载这几年来在士林中声望直线飙升,不过因为关学与新学相抵触的关系,一直无法进入国子监教书育人,可他的的确确是世所公认的名儒。

想想当年被称为真先生的胡瑗,他被范仲淹举荐后,也只是个最低阶从九品的京官,但最后教出多少英杰来?文官不用说,就说武将,连镇守西陲的现任秦凤路兵马副总管苗授,都是他的亲传弟子。

而张载本人绝不逊于当年的胡翼之。

尽管眼下他的学生们,绝大部分地位还不高,但随着时间的过去,其中必然会许多人逐渐崭露头角。而且……在他们之外,还有个例外的韩冈。

能教出韩冈这样的学生,当然不会是普普通通的庸师。就是韩冈这个学生出色得有些过了头。

赵顼叹了口气,挥挥手示意宋用臣退下。照常理来说,接下来当是会有臣子上表,为张载请求封赠。

的确也不出他所料,第二天,在张载病故的消息传播出去后,事情的发展就一如他的猜测那样,很快就有人过来打算帮着张载最后一把。

张载官位不高,连上遗表的资格都没有。但王珪、吕惠卿,以及下面的一干臣僚,总计四五十人,都为他上了请求追赠的奏表,表中对张载多有溢美之词。在赵顼想来,要不是王安石称病,没办法自己拆穿自己,他当也会上表为张载请一个追赠。

赵顼完全没有否决的意思。毕竟张载为士林所敬,在民间声望也高。而且还有一个有名的尊师重道的韩冈。

当年韩冈在有半师之谊的程颢家门前站到积雪没膝,现如今在民间的图书和年画上,都有韩冈程门立雪的绘图,就跟司马光砸缸的事迹,很快就遍传了全国各地。多少人家在教导家中子弟的时候,都拿韩冈做例子,教子弟们该如何尊师重道。

赵顼将一摞子奏表放在这里,用手拍着最上面的奏折,最后吩咐道:“张载官位虽卑,但他于经义儒术上多有创建,又为国作育英才,当厚给赏赐。”

只是荫补和追谥就不可能了,前者身前至少得六品官,后者则要更高——张载虽是一代师表,却也还不够资格入文庙,不能走这一条路——只能赐钱赐物。追赠官职同样是得按照礼制,赵顼看看张载的官职,当是郎中一级。

宋用臣出去了。赵顼又拿起了奏章,崇政殿上静悄悄的,在王安石称病之后,赵顼便很少留人独对。不仅仅是赵顼没有那个心情,也是他不觉得还有必要让人太过于接近自己。

他手上的这份奏章,来自于关西。是种谔送回来的奏报。

种谔之前的功劳也算是煊赫,回到京城后连赵顼都不好安排他的职位,只能让他在外界继续镇守边疆,必须再过几年的时间,再招他回来也不迟。

低头看着种谔的奏章,上面说官军最近死死压着党项人的骑兵。而横山的部族已经近乎所有部族都投向了大宋。尽管他们毕生的盼望依然是钱和利益,身为渴望能从富庶的中原地带,再得到足够的财富。但在大宋的军势下,绝大多数还是觉得命比钱来的要重要。

在横山蕃部投效的过程中,也不是坐等他们派人上门,而是直接有人去接近他们,一家家的去将横山蕃部给说服和压制住。

接受了这个任务的人有许多,毕竟有了官军做靠山,安全性提高了不少,而且有军队在背后,直接说服他们并不算什么难事。只是其中有个人叫做种建中——只从姓氏上就能知道,他依然是种谔的近亲。翻看过去的记录,种建中在种谔幕中多有功勋,如今功劳也最大,种谔上表主要目的就要奖励他。

这种建中似乎也是张载的弟子。赵顼隐隐约约的想了起来。种谔前一次上京入觐,曾经听他说起过。这也不足为奇,关西世家子弟很多都在张载门下学习过,种建中考上了明法课,这一点还比较让人感到惊讶。

赵顼感叹了一声:西军种家英才辈出,与种谔一到戍守边州的几兄弟,种诂、种谊如今都是损伤不得。而种谔的儿子种朴,从熙宁初年的罗兀城之战,便立有殊勋。之后依然跟随种谔,也就在二个月前,他靠着在其父种谔那里得到的细节,接任了王舜臣的罗兀城主一职。

种朴、种建中、王舜臣、折可适、李信、赵隆,军中年轻有为的将领数不胜数,都是点起一支兵马,便能克敌制胜。只要有了他们,未来的几十年,大宋的边疆都可以保持安定,甚至可以让边疆不再是边疆。

作为军中核心的大将,有燕达、种谔等一干人,都是四十岁上下,头脑、经验和精力,都处在巅峰状态上。而张守约这样的宿将老将,也不会输给年轻一辈,用来领军,半点都不用担心会出问题

而领军的主帅也不缺人选。武有郭逵,文有王韶,两人兵法、战功和地位都不缺,随时都能出来统领大军。再年轻点的,也有章惇、甚至韩冈。就是李宪、王中正两人,尽管皆是阉人,但他们也都是功绩累累,在战场上有过出色的表现,绝不是纸上谈兵的赵括马谡之流。

有了他们为将为帅,军器监的几个作坊也在拼了性命制造板甲,再有一两年的准备,便能举兵西向,将江河日下的西夏国给剿灭。

赵顼呆呆的在崇政殿上幻想了半天,终于清醒过来。这些事可以放在一边,更重要的还是该如何安排王安石。

有王安石在总掌朝政,赵顼做起事来总觉得有些束手束脚。虽然有吴充、有冯京,但许多事,王安石的一句话,能抵得上所有宰执的合力。这么多年下来,赵顼觉得是该变上一变了。

大宋天子低头看着王安石的第三封辞章,前面两次他都已经毫不犹豫的给否了,眼下这第三封辞章,很快就又递上来了。

随意的将辞章浏览了一通,文字依然出色,不愧是文坛宗匠。但赵顼不是要跟王安石比较文采,而仅仅是想将王安石的辞章给驳回去。

亲自提起笔,赵顼将王安石的辞章再一次毫不犹豫的打回去,这一次他还是不能答应。若仅仅是三请便允许,对于王安石这样的宰相等于是侮辱,就仿佛是赵顼等不及的要将他赶走一般——尽管赵顼的确觉得王安石离开比较好,但他对辅佐自己富国强兵的宰相依然敬重有加,他不会也不愿去做这样的事。

将笔放下,赵顼吹了吹墨迹未干的纸页,便放在了一边,待会儿就让人送过去。

‘再有个三四次就差不多了吧?’赵顼想着。不过转念一想,是不是该再过来个两次?毕竟王安石不是普通的宰相,是富国强兵的贤相,赵顼与他是君臣相得,得加以优容和褒奖,在每件事上都得如此。

王安石铁了心要辞官,赵顼也有心成全,但王安石一手辅佐自己近十年,让大宋的军队逐渐建立起对契丹和西夏的优势,这份功绩,赵顼一直都记在心中。

换作是十年前,一听说契丹与西夏勾连,整个京城都得乱起来。哪里能想象得到,眼下是契丹为了避免唇亡齿寒要去支援西夏。京城内外对契丹骑兵的恐惧,随着这些年来的一桩桩大捷,已是逐步的烟消云散,已经完全不用再放在心上。

而这一次的灭国之功,让他进了太庙都能昂首挺胸,能毫无愧色的面对太祖和太宗皇帝。

不过这仅仅是开始而已,接下来还有更为光辉灿烂、甚至让太祖太宗都会自愧不如的成就正在等着他。

赵顼直起腰,他还年轻,还有的是时间。

