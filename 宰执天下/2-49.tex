\section{第12章 平生心曲谁为伸(三)}

【第一更。一月的最后一天了,两个月六十万字的承诺圆满达成,俺说话算话,请求奖励。】

站在西班中的首位,瞥眼上望。文彦博就看见赵顼将这份秦州来的紧急军情看了一遍、两遍、三遍,而他的脸色也是一变再变,最后凝固在脸上的事似笑非笑,似怒非怒,很是怪异的神情。

也不知看了几遍后,赵顼将奏报放下来了,往文彦博看过去。文彦博连忙收敛自己的视线,垂眼看着手中的笏板。

赵顼的嘴角绽出了一丝莫测笑意,他将身边的小黄门招过来,低声的说了两句。见小黄门听明白了,便把军情奏报着他传下去。

小黄门手托军报,走下陛阶。文彦博抬起头,他是枢密使,当是能先看到。只是小黄门并没有向他这边转过来,而是走到对面给了首相曾公亮。

曾公亮拿着奏报只看了一眼,表情顿时也变得跟赵顼一样怪异。立刻紧抿起嘴,不知在忍着什么。他抬头看了看文彦博,又瞥了瞥赵顼,低下头又细看了奏报一通。最后神色庄重起来,抿着嘴的将奏报递还给小黄门,跟着赵顼一样,沉默了下去。

小黄门托着奏报,依然没有回头往文彦博那里去,而是走到曾公亮下首的副相陈升之处,将奏报递给了他。

陈升之接过来一目十行,猛的把头低了下去,肩膀微微颤着。过了一阵,他平静下来,也是神色诡异的看了文彦博一眼,将奏报还给小黄门。

文彦博手中笏板一紧,盯着小黄门,下面该轮到他了。赵顼和两位宰相的神色让他觉得很不对劲,现在心急着要看一看这份奏报上到底写了些什么。

可是事情出乎文彦博意料,小黄门依然没有走过来,而是把奏报交给了再下面的王安石。

文彦博呼吸一促,脸顿时就阴了下去。朝中论班次顺位,他这个枢密使,只在两位宰相之下,却在王安石之上。军情奏报不先给自己,而给了曾公亮、陈升之,此事还说得过去,但接下来却传给王安石,而不给他文彦博,这事怎么也不对。

文彦博用眼角瞥了一下赵顼,当是这位年轻的皇帝让小黄门将奏报送下来时说了些什么。

王安石拿到奏报在手,很性急着展开来细看。一看之下,他先是喜色上脸,但很快就被怒意替代。他抬起头狠狠的瞪了赵顼一眼,又转头用力钉了两位宰相一下,抬手把奏报递还小黄门,冷声说道:“把奏报给文枢密!”

文彦博板着脸,心中犹疑不定的接下了奏报。正待要看,那边赵顼因被王安石瞪了,有些尴尬咳嗽了一声。他见着玩不下去,也不等文彦博自己看奏报,便公开了其中的内容:

“方才秦州急报,古渭已定,王师大捷。今次为复日前托硕之仇,董裕统领五万大军来犯。王韶、高遵裕率部坚守于古渭,并遣勾当公事韩冈夜出城寨。韩冈领命一夜奔驰百里,调集蕃部部众。青唐部族长俞龙珂并其弟瞎药奉其命,统领七千部中精锐抄截董裕后路。

五月初七午后时分,于渭水之滨的荒石谷西突袭董裕大军。血战半日,五万贼军皆尽溃散。此役共斩首一千一百余级,没于渭水中者不计其数,贼军主帅董裕、军师结吴叱腊二人并授首,其下大小将佐、族酋授首者百余,被擒者亦有百人。”

崇政殿中只听见赵顼强忍着兴奋的声音在回响。他不怀疑王韶和高遵裕联名发出的这份捷报的真实性,相对于平常听到的击退几万几十万敌军的吹嘘,只有斩首和缴获才是最能体现战果的实绩。

一千一百余级,还附带两个贼军主帅的首级!

这是个多么辉煌的胜利!

连着托硕大捷一起,依靠这两次胜利,赵顼也向天下臣民证明了一直支持着河湟开边策略的他,是多么的英明!

除了提前看到奏报的三人,其余大臣们先是一阵惊讶,五万贼军来攻,竟然给王韶他们赢了,而且还斩首一千一。这可不是个小数目!只从斩首数目上看,王韶最近的两战,已经彻底压倒绥德城此前的战果。但很快,他们又都想起此前文彦博说得几段话。

几十只眼睛齐刷刷看向文彦博,有幸灾乐祸的眼神,有站干岸上看好戏的冷眼,当然也有把同情投向文彦博的视线。

‘怎么会?!怎么可能!?’

文彦博紧紧捏着奏报,脸色涨得血红,手背上的青筋跳了起来。一阵天旋地转,他高大壮硕的身子摇摇晃晃,眼珠子直转着不听使唤,仿佛下一刻就要栽倒。

赵顼急了,气一下文彦博可以,但气死了可就麻烦了,他指着文彦博急叫着:“还不快扶着文卿家!”自己也是哗的一下站了起来。

刚刚递过奏文,就站在文彦博身前的小黄门连忙伸手把他扶住,文武两班的宰臣们也乱了阵脚,一齐涌上前。拍背的拍背,舒胸口的舒胸口,围着文彦博一通忙活。

章惇站在班次最后,看着文彦博身边乱作一团的样子。他心中乐得很,几乎要笑出声来。前些日子,王韶把向宝气得中了风,当着几千人的面昏倒在地。眼下看着文枢密的模样,好像也是要不成了,若是他今次也昏倒在朝堂上。日后若再有人想跟王韶过不去,比如那些御史,怕是都要先把开窍行气的苏合香丸随身带着,才敢披挂上阵了。

可惜文彦博让章惇失望了,殿中唯一的三朝宰辅终于还是平静了下来。毕竟在朝中起起落落几十年,心思城府不是向宝可比。

被御史指着鼻子骂过,被天子当面斥责过,还从宰相的位置上掉下来被赶出京城过,经历了这么多事后,文彦博这个历经三朝的元老重臣,岂是这么容易就被打垮,气倒?

用力推开天下官品最高的一群急救医生,文彦博重新站定,与站在身前关切的看着他的王安石对上眼,从牙缝中迸出话来:“老夫可不是唐介!”

王安石沉默的走回自己的位置,连带着其他宰执,还有重臣们都站回了原位。章惇退了两步,也站回去了。

章惇归班,就见着他上首的吕惠卿正正的双手持着笏板,纹丝不动,他的姿态就跟崇政殿廷对刚刚开始时那样,一点变化都没有。章惇看了吕惠卿一眼,他清楚的记得,方才的那一阵乱,吕吉甫可是连根脚趾都没动弹。

‘养气功夫还真够好的……’章惇冷笑着想着。

等东西两班再次站定,赵顼关切的问着文彦博:“文卿,可有何处不适。”

“臣无事。”文彦博硬邦邦的回答,竭力让自己站稳脚跟。

‘那里无事了!’赵顼看着文彦博还是站不稳的样子,连声说道:“来人,给文卿家一个绣墩坐着!李舜举,你速去御药院把御用的至宝丹、灵宝丹、苏合香丸、如圣饼子、八风散,还有……还有……”

赵顼一口气把他所记得的治疗风邪的成药都报了出来,剩下的一些他记不得了,‘还有’了半天,最后不耐烦的说道:“把该拿的都给拿来!”

李舜举小跑着从殿后小门出去了,一名内侍也奉旨为文彦博端来一个绣墩。

“臣无事。”文彦博坚持说着。他挺直了腰背,连赐坐都不要,就硬是这么站着。他知道自己若是坐下来,露出一点病态,尚留在朝中反变法一派,土崩瓦解虽不至于,却必然大受挫折。

一双虽已浑浊却仍锐利的眼睛狠狠地盯住王安石,‘老夫可不会就这么认输!’

照理说看到捷报后,群臣都会赞礼拜贺,向天子恭贺战事的胜利。赵顼在看到这份捷报时,脑中就在想着文彦博究竟会是用着什么样的表情来向他恭喜。

他对此很期待,但文彦博眼下这副模样,赵顼真的不敢玩了。气死了三朝宰辅重臣,他的名声可就要打着滚的往下跌了。就算他赵顼是天子,也堵不上天下悠悠众口。

等李舜举带着个两个小内侍大包小包的抱着一堆急救风疾用的成药过来,赵顼便一股脑儿的全数赐给了文彦博,最后他对群臣说道,“今日已是无事,各位卿家还是各归本司去。”

本来今天还是有不少议题要讨论的,否则吕惠卿和章惇也不会站在殿中,他们就是为了要与文彦博打嘴仗而来的。但赵顼现在没了心思,他接着又唤来方才的小黄门,对他尊尊嘱咐:“去找张肩舆过来,好生送文卿家归宅。”

再次拜过天子,宰相们领班而出。文彦博紧紧地跟着他们,腿脚上看不出有什么问题。等到走出崇政殿外的廊道,品级从高到低排出的队形终于散开,大臣们各自向文彦博问过身体安适与否,见他似是无事,也就各自散着走了。但不知不觉间,文彦博已经走得慢了些去,落在了后面。

文彦博一步步的向前走着,他身后是两个抱着大堆御赐药物的内侍,而领了赵顼旨意的小黄门则是紧紧的跟在一边。

台阶出现在眼前。文彦博举步走下去,走了两级,他脑中一阵突如其来的晕眩,脚下一软,一个踉跄,就要栽倒下台阶。小黄门连忙冲过去扶着他。但文彦博身高体胖,壮牛一般,他的重量却连着把小黄门都带了下去。正当他们就要滚下台阶的时候,一双坚实的手臂伸了过来,稳稳的将文彦博扶住。

文彦博脑中晕眩稍定,抬起头,却见救了他的,竟然是章惇这个王安石的手下干将。

抓着文彦博的肩膀,章惇柔声说着:“文枢密,要小心脚下啊……”

