\section{第45章 仁声已逐春风至(上)}

敲定了吴衍之事,韩冈从王珪府上告辞出来,没有回府,而是往开封府衙没发生

他打算趁机见苏颂一面,只为他给自己挑了一件好宅子,韩冈就得好生的感谢他。

只是苏颂现在是因为陈世儒弑母案而焦头烂额,而且这件案子还牵连到吕公著的身上,又有蔡确搅浑水。

“难啊。”苏颂摇头叹息,整个人仿佛老了许多。

精巧的家宴,设在开封府的后园中,权知开封府一边给韩冈倒酒,一边大煞风景的说着案情。

“陈世儒当真弑母了?”韩冈还是不敢相信外面的传言,“应该是李氏杀姑吧?”

弑母,千年之后,都是稳稳的死罪。而眼下更是被父母告了不孝,依律可以将儿子论以死罪的大宋!以孝治天下,将三纲五常视为维护社会稳定的铁律的时代。弑母可是十恶不赦的重罪,跟造反一个等级的。相对而言,造反还能博得一些人的同情,而弑母就只有鄙视和厌憎。反倒是媳妇杀婆婆还好说些。

“本来我也是不信的。”苏颂摇着头,向韩冈介绍着案情。

陈世儒是仁宗朝的宰相陈执中的独子,案发前以国子博士的本官,在江南的太湖县做知县。他的妻子,被家仆首告唆使婢女用铁钉锥杀陈世儒之母张氏的李氏,是李中师的女儿。而李中师之妻吕氏则是吕夷简的孙女,吕公著的侄女,李氏自然也就是吕公著外侄孙女。

陈世儒在太湖县任官,而他的妻子李氏,跟绝大多数官宦人家一样,留在京城的家中侍奉舅姑——陈执中早已不在人世,李氏侍奉的是陈世儒的生母张氏,也就是本案的被害人。

被杀的张氏,是陈执中的宠妾,性格暴虐,几十年前在京中很有些名气。使人杖死十三岁的婢女迎儿是她,逼死另外两名婢女的也是她。一个月死了三人,当初这件事闹得很大,御史台接连上书,陈执中也被罢去相位。

而在陈执中死后,仁宗亲自安排张氏入寺庙中修行,直到陈世儒成人,方从庙中迎回。不过没多久她就死了,陈家报的是病死,然后陈世儒丁忧回京。但案发后,经过检查,张氏是被人谋杀。先下毒,后用铁钉锥入心口。

“那出首的仆人,说李氏吩咐下来‘博士一日持丧,当厚饷汝辈’。听起来倒像是唆使,但奴仆欺主的事太多,这话一开始我是不信。张氏待仆婢刻薄寡恩,又有昔年旧事,死于仆婢之手更对得上。但之后审问陈家家人,却发现整件事的确都跟李氏有关。陈家上下的仆婢,都收了赏赐封口,而出首告官的这一位,只是因为赏赐分配不均之故,才忿而站出来自首。”

“原来如此。……这还真让人想不到,吕家怎么出了这么蠢的女儿。”

“她是李家的。”苏颂更正道。

“李中师手段也挺厉害,当年逼得韩国公【富弼】全家缴免行钱。”

“再厉害也是教女无方,而且这件案子不仅仅是李氏,陈世儒也脱不了罪名。”

“应该不可能吧?”韩冈不信,苏颂前面都已经明示他有证据,但从情理上推断,总觉得有几分让人疑惑。

李氏唆使婢女杀姑,有人证,有物证,口供虽然没有,但迟早能审出来。只是陈世儒他被牵扯进来就有些奇怪了。

“陈世儒没必要为了回京而弑母,想回京直接报病请辞就可以了,有的是人等他的位置。而李氏想要让丈夫回京,就没有别的理由,除非能说服陈世儒,否则就只有让他丁忧才可以。”

苏颂微微一笑,笑容还是脱不了苦涩:“那依玉昆你的想法,这个案子的真相该是什么样的?”

“应该有人要将水搅浑,闹到现在,外界对这件案子的称呼已经不是李氏杀姑案,而是陈世儒弑母案。但不管是李氏杀姑案,还是陈世儒弑母案,现在事情闹得这么大,只怕是项庄舞剑意在沛公。”

在韩冈想来,这多半是想通过这件案子将吕公著给拉下来。而吕家也的确派了人来找苏颂关说,且吕家对大理寺很有影响力,几次出手干扰苏颂的审案。不过这件事已经给爆出来了,让吕公著好不狼狈。

此外韩冈还听到一种说法,御史中丞蔡确之父蔡黄裳,在陈执中离开相位,至陈州担任知州的时候,曾经因为一次疏忽犯了错,被陈执中逼得上表辞官。当时蔡确尚未成人,家里的顶梁柱没了收入,一家老小衣食无着,只能流寓陈州。几十年的恨意积攒下来。如今是蔡确在报复。

“真是一汪浑水。”韩冈感慨着,苏颂摊到这件案子运气真的是糟透了。身为权知开封府,想脱身都难,“不过怎么看,陈世儒都不会跟这一桩案子扯上关系。”

苏颂摇摇头:“虽然情理之中的确是这样么错。但玉昆你恐怕想不到,此案当真是与陈世儒有关,从审出来的口供来看,至少他是知情的。”

韩冈脸色变了,知道妻子唆使仆人杀母而不阻止,其实就是弑母,没有别的解释。

这件逆人伦的大案,如果出在地方上,当地的知县少不了会因为教化无功要被迫辞官,知州也得受到责罚,现在还是两位宰相的后人做出来的,谁碰了谁都会觉得棘手,已经不仅仅局限于案子本身的问题了。

“那这件案子还有什么好犹豫的?”韩冈疑惑不解的问道。

“玉昆,你不知道……”苏颂叹了半天的气,最后和着酒意将原因说了出来,“天子要保陈世儒。”

弑母,属于恶逆,排在十恶不赦重罪中的第四位,仅次于谋反、谋大逆、谋叛这三条,直接触犯天子的大不敬之罪则排在第六。

韩冈声音都尖锐了起来:“就这样,天子还要保他?!”

“‘止一子,留以存祭祀何如?’陈世儒是陈恭公【陈执中谥号恭】的独子。他若是死了,陈恭公这一房可就断了香火了。”苏颂苦笑,“虽说天子没明说,应该还有陈家、吕家的体面在。只是吕家一家倒罢了,两门宰相,其中还是独子,天子不想闹得太大。”

陈执中死的时候,英宗还只是汝南郡王府上的十三郎赵宗实——其父赵允让在几个月后病逝方才追封濮王——当今天子甚至没跟他打过照面。

可尽管这样,陈执中的儿子是犯下了恶逆之罪的弑母罪囚,赵顼照样还是想要保这个逆子一条性命——只因为陈世儒是陈执中的独子,更因为陈执中是宰相。

包括韩冈在内,他们这个等级的高官一向是受到优待的。做到了学士、直学士的文臣,晋身两府的宰执,才是真正能与天子共治天下的士大夫。至于下面的官员,那完全是两个阶层。

甚至是武将,比如当年做到枢密使的曹利用,也是明面上将其远斥,私下里让人下阴招,使其自裁罢了。明正典刑的杀,几乎是不可能的。

“前面只当是李氏杀姑,天子说‘此人伦大恶,当穷竟。’但现在变成了陈世儒涉案,天子要放他一马……给蔡确顶回去了。”苏颂抱怨着。

韩冈冷眼问着:“如果这一次没有御史台,这件案子基本上就能定下来了吧?”

苏颂叹了口气,自是默然不语。

陈执中逼得蔡黄裳辞官,蔡确与陈执中有深仇大恨,能毁了陈执中的儿子,灭了陈家的血脉,他是绝不会放过的这么好的机会。

御史大夫向不授人,御史中丞是实际意义上的一台之长,虽然御史们各个桀骜,许多时候不服管束。但蔡确他既然统领御史台,要想引导一下清议风向的,还是很容易的。

天子都压不下这些言官。乌鸦一叫,肯定是要死人了。

不知是谁人问的,御史台的言官为什么总是能升得那么快?有人就回答,因为色黑近紫——都是把他们当乌鸦看了。

“三纲五常,这是天条,子容兄,这一桩案子得尽快审结,否则御史台只会乐得将整件事给扩大下去。”

韩冈神色严肃,语气郑重。要是苏颂这边继续拖延,到时候,御史台的目标就不仅是预定中的吕家、陈家,连苏家也会给牵扯进来。说不定到时候,苏颂的职位也一并保不住——尽管之前御史台已经借用另一桩案子来弹劾苏颂宽纵,但间接的攻击,和直接的指责,两者之间的力度毕竟不一样。

“哪里还要玉昆你来说,愚兄这么多年见得多了?”苏颂自嘲的笑了一下,昨天苏颂心血来潮,算了一卦,最后可是没有好答案,“就是决心难下。现在正好,多谢玉昆襄助。”

“不敢。插一句话,只是口舌之劳而已,喝了子容兄的酒,要还上一句。”韩冈哪里会居功,根本就不干他的事,只是说了一句眼下这桩案子还是早点结束的好罢了。

苏颂呵呵笑了两声,又与韩冈一并喝酒聊天,暂将此事抛出脑后。

