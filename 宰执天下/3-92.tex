\section{第28章 临乱心难齐(七)}

王雱没有耍嘴皮子的多余精力,跑了几个时辰,累得不行,直截了当的将来意说了一遍。

运输粮食来打压粮价,这正合韩冈的心意。

只是在韩冈想来,用来平抑粮价的粮食,只要动用开封府的常平仓应该就够了——而且这还不会影响开封府的粮食安全——在开封城内外,有着富国、永丰、顺城、五丈河以及夷仓等七八个大的储备粮仓库,最小的一个都是存粮超过十万石以上的大仓库。

不过听王雱的意思,他是准备要一下将东京城的奸商们打死,省得到了春天再为此而烦心——一旦动用了常平仓,就等于是告诉人们,政事堂的手上只剩最后一招了,心思蠢蠢而动、准备到春天发难的必然不在少数。

所以王安石和王雱才有了开汴口,破河中之冰,在冬天输送泗州、宿州的存粮上京的想法。所以侯叔献以碓冰船破河冰的建议,才得到了王安石和王雱的看重。也因此,一听说韩冈这边有着更好的办法,王雱便立刻飞奔而来,连夜向韩冈讨教。

只是他们的想法初衷虽好,却一点也不现实。

“用雪橇车大规模的运输绝不可能。”韩冈立刻就否定掉了王雱的幻想,“熙河路从来没有过在冬日大量运输粮草的经验。”

熙河路冬日的交通路线,的确是通过冻结的河道来运输。不仅是渭源堡通往狄道城的道路,就是陇西与渭源的交通,还有狄道与临洮、珂诺等寨堡的交通,同样是通过雪橇马车来联系。

熙河路主要的城池寨堡,基本上都设在河边。借助洮水、渭水等河流冻结后形成的通途大道,冬天的熙河路一样能够顺利交通往来。即便其中有几段河道中间有瀑布,但附近的兵站都在那里设了哨卡,在合适的地点安装了大型的绞盘提升装置,将雪橇马车卸载后,分批吊运上去,其难度并不大,只是偶有损失而已。

但这样的运输方式,主要还是以传递消息为主,加上一些过年时的犒赏,运力并不大。从来没有说用雪橇马车大规模运输的:冬天不比春夏时节,山间的一场暴风雪就能让运输队损失惨重。就是眼下小规模的运输,路途上的损失其实也不少。

而眼下要大规模组织雪橇车来运输的还不是熙河,而是汴河。在过去,根本没有河道冰面上运输的经验,而且又没有足够的准备。仓促行事,临危受命的六路发运司,能做得好这件事吗?

改装车辆其实不难,只要肯动用人力,六路发运司手上用来修补纲船的过千匠户,半个月之内,改装出两三千辆以上的雪橇车都没问题——只要牢牢的钉上形状合适的木条,就可以将马车车厢,甚至一些不大的平底船改造成能在河面上跑的雪橇车。

可是为了保险起见,同时临时改造的雪橇车本身肯定也有问题,其载重撑死也就二十石上下。二十万石粮食,就是一万车次。这样的数目,韩冈不是小瞧人,六路发运司当真是组织不起来。运到肯定是能运到,但动用大批车次的运输过程中,中间的损耗不知会有多少。

而且还有一个问题,“雪橇车,顾名思义是在雪上行驶,不是冰橇。但汴河上那里只有冰,雪橇在上面不好走啊……”

王雱脸色为之一变:“雪橇在冰面上就不行吗?”

“有个几寸厚的雪就够了。”韩冈倒不会骗自己的大舅子,“就是没有雪,短途运输其实问题也不是很大,最多颠簸一点,跑开了就没什么关系了。但问题的关键不在这里。”韩冈神色严肃,沉声说道,“从宿州到东京,路途有五六百里之遥。河中情况懵然不知,埋头向前送,一路颠簸,仓促改装后的雪橇车很难承受得起,中途的损耗可是难以计数!”

听说雪橇车可以在冰面上使用,王雱的神色就缓和了下来,不为韩冈后面的话所动:“依玉昆你看来,用雪橇车比起驱船使碓来捣冰凌如何?”

那还用说,肯定是要强出百倍。韩冈绝不会相信,在千年后都让人头疼的河道破冰问题,在这个时代能够轻易解决。再怎么说,雪橇马车也是经过几年的实际验证,大规模运输难度虽说很大,可比起侯叔献的方案来,还是靠谱得多。

王雱的心意已经完全从问题中透露出来,韩冈沉默一阵,终于开口问道:“……真的要如此行事?”

“箭在弦上。”王雱声音沉甸甸,“所以才来问玉昆你,究竟可不可行。”

“这样的损耗绝然不小,能有一半入京就已经是万幸。”韩冈再一次提醒王雱,“运输成本可要比纲船高出许多。”

“再多也多就跟从关中运粮去熙河路差不多。”王雱不以为意的笑道:“记得当初蔡子政在泾原路时,曾上书言及‘自渭州至熙州运米斗钱四百三十,草围钱六百五十’。”

“蔡副枢说得夸张了一点,斗米四百三,一石运费就要四千三,也就是五贯半【宋代一贯为七百八十文,足贯方为千文】。”韩冈摇头,蔡挺是虚言恐吓而已,拿着最高时的价格来做例子,并不是平均数,不能当真的,“记得熙宁三年、四年、直到五年年初,总计从泾原路运来米麦差不多有三十余万石,草料也有四十五万束。难道光是在运费上,就用了三四百万贯?”

“但运费比粮价要高出数倍那是没跑的。”王雱立刻接话道。

“这倒是。”韩冈点头承认,“三五倍总是有的。说起来如果没有小弟所创的兵站制度。以旧时路中的消耗,要将粮秣运到最前线,三斗能有一斗就不错了。如今至少省了一半。”

王雱倒没在意韩冈的自我吹嘘,“不论路中要花多少,这边都没有关系。就算是以蔡挺所说的运价……只要能在今年冬天将宿州、泗州的囤粮运抵京城,中书可以承受!”

王雱的声音斩钉截铁,韩冈顿时没话可说了。

心中暗自叹着,财大气粗就是好哇!

当一个有足够实力的政府,不惜代价的开始全力运作的时候,损失和阻碍的确不算什么了。只要能达到目的,只是单纯以三五倍的金钱为代价,他们都能够承受下来。就像后世的工业化国家开战,国家机器运作起来,无数物资在战火中化为灰烬,但对于国家来说,在胜利面前,这些消耗和损失,根本算不上多大的问题。

虽然大宋并不是工业化的国家,但掌握在王安石这位宰相手中的资源,要想支撑起这条道路来,并不算很难。既然王雱如此说了,韩冈此前的顾虑全都得到了解脱。既然是不计损失和花费,有没有什么好担心的了。

心思一活,脑筋就开始转动起来,原本的担忧全数化成了动力,“改装过的雪橇车,只要修补整备后,年年都可以使用。今次改装的成本就可以平摊下来。”

“这点花费也不算什么了。”王雱轻松的笑着,他这个妹夫,推三阻四的半天,现在终于肯帮着出主意了,

韩冈点了点头,“既然如此,不知能不能秘密一点的改装呢?”

王雱眉头一动,顿时笑道:“明修栈道,暗渡陈仓吗?”

原本准备使用侯叔献的碓冰船,其实王雱自觉还是失败为多,甚至有死马当活马医的想法。但听说了韩冈有着雪橇车这一冬季运输队利器,他当时就有了暗中阴人一把的想法。现在见韩冈也是如此提议,顿时有了惺惺相惜,英雄所见略同的感觉。

韩冈察言观色,也笑了起来,“看来元泽兄已经胸有成竹了。”

王雱点头:“愚兄会通知六路发运司,让他们全力改造,并且严加守秘。”

“另外雪橇车有个好处,就是可以几辆车厢连在一起。可以节省不少马力。”驳船可是一拖拖上好几船货物在后面,火车也是一个车头带着,在阻力并不大的冰面上行驶,雪橇车也可以多拉上两车。

当然,还少不了能在冰面上使用的重钉马蹄铁。

韩冈当初在熙河就已经将马蹄铁拿了出来,可是由于他的功劳太多,原本敝帚自珍,准备用来博取功名的武器,早就被视如平常,王韶、高遵裕看重归看重,呈递上去后,却也没有帮韩冈换回来多少封赏。

不过现在跟着王雱一说,王雱拍案叫绝。至于分段运输,也就是兵站制度,同样不在话下,韩冈都拿了出来。

一番话说了一夜,两人精神抖擞,将前后事一起细细讨论,将各项步骤逐一敲定。不过他们都没有考虑过其实在没有下雪的道路上,可以用马车来运送粮草。

毕竟打压粮价,与其说是商战,还不如说是心理战。陆上运输的运力多寡,每一位粮商的心中都是有数的,粮商背后的靠山们也是有数的。故而王安石硬是要开河口,因为畅通的汴河,可以彻底的将粮价给打压下去。而韩冈的雪橇车则是一个谜团,没人能猜测得出能运多少粮食进京,这就可以让那些粮商们投鼠忌器,不敢轻举妄动。

