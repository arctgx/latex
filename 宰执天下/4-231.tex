\section{第38章 岂与群蚁争毫芒(四)}

严素心服侍着韩冈将汤药饮子喝光,正收拾了准备回去,却被韩冈硬拉着说些梯己话。

“最近家里可有什么大事?”韩冈拉过严素心,楼主后问道。

家里面的事,除了子女教育,韩冈基本上都放手,交给王旖统管。但作为一家之主,该了解的还是得了解。

“官人不在家,哪里会有什么事。就是招儿前两天来信,说是已经有了身孕。”靠在韩冈宽厚的胸膛出,严素心半闭着眼睛,轻声的说着。

招儿是严素心带着离开陈家的唯一一人,当时不过是个小女孩而已。在陈家彻头彻尾的完蛋之后,招儿跟了严素心一个姓。不过她毕竟是陈家的女儿,严素心就是想要留在身边,韩千六、韩阿李和王旖也不会点头。稍长大了一点就被留在了庄子上,到了去年,长到十三岁,韩冈的老娘就给她挑了个好人家嫁了过去,还送了十几个箱笼做嫁妆,当成女儿出嫁一般。

“年纪还太小,十三四虽说能嫁人,但怀孕生子却是要冒不小的风险。”韩冈摇摇头,只能盼严招儿吉人天相了。

“前几天,奴家看到了表兄,就在衙门里,是奴家姑姑家的儿子。”严素心轻声说着。

韩冈身子一震,脸上的笑容顿时就少了,“家里还有人能联络啊。”

“奴家只跟着官人。奴家最需要他们的时候,是官人出手,又不是他们。”

严素心的父亲是一名进士,不过在秦州为官的时候,开罪了陈举。被一番陷害,便被押去了岭南。

而在其父坏事,被发配道南海后,其母为陈举所凌迫,甚至归乡不得,最后被收进陈家的宅院中。这也因为是严家只是寒门素户,严素心的父亲是鲤鱼跃龙门的幸运儿。要是严素心的父母随便哪一个是官宦人家出身,靠着亲戚早就翻身了。

但进士总归是进士,是被士大夫承认的一份子。韩冈纳士大夫之女为妾,不免遭别人看作是挟恩图报,而且也是一桩忌讳。若是闹起来,她在韩家也就待不住了。

周南是天子赐予韩冈,背后是皇帝,而云娘与韩冈更是亲近,韩父韩母也是她的靠山。就是严素心身后什么都没有,就只有一个儿子。

作为一名小妾,并不是生了儿子就能安心的。变法之初,被反变法派群起而攻的御史李定,他的生母就是生下他后,被赶出了家门。

后来李定被人攻击为不孝,就是说他在生母去世后,没有丁忧守制三年。不过李定则辩称他并不确定生母就是仇氏,仅是隐隐有怀疑,不敢询问父亲,所以在仇氏病逝之后,就以归养老父为名辞了官职,虽然没有报请丁忧,但那两年也的确没有出来做官。

只是反变法派可不管这么多,名不正则言不顺,照样咬着李定不妨,这件事在朝堂上闹得沸沸扬扬,甚至连着有三名知制诰封还了词头,驳了天子对他的任命。当时毫不相干的苏轼,则主动跳了进来,写了首赞美孝子的诗来嘲讽李定。

此外还有一件很有趣的事——与苏轼诗文往来频繁的高僧佛印,正是李定的同母异父的兄长,仇氏是在生下佛印之后,被李定之父李问纳为妾室。

所以李定不为生母守制的这件事,究竟是不是第一个弹劾李定的御史陈荐爆出来,在世人的议论中还当真有些疑问。韩冈曾听王雱提起过李定,据说他一向善待宗族,分财赈赡,以至于家无余财。在王雱口中,其为人不恶,就是对苏轼衔之入骨,就不知道是为了苏轼的那首诗,还是别的原因。

而李定、佛印的生母仇氏生下的不仅仅是两个声名远播的儿子,据说开封教坊司中的名妓蔡奴也是她的骨肉。蔡奴本姓郜,行六,是仇氏自李家被逐出后,再嫁所生。

蔡奴如今在京中艳名高炽,可比周南当年闯下的名头还要大。韩冈自广西回京后,留京不过月余,就听了不少提起了蔡奴。

儒臣、高僧以及名妓,乃是同产兄妹,在遗传上应该受到了母系方面影响,至少从学问上来看,当时如此。

李定、佛印的学问就不用说了,而蔡奴也绝不会如何逊色。但凡能成为名妓,才学在女子中都是顶儿尖的,大家闺秀很多都难以企及,要不然她们也不能与士大夫们相唱和。

就如周南,琴棋书画以及歌舞方面的水准都是一流的,就是作诗作词,在韩家也能排在第二——第一当然不是韩冈,而是深受王安石熏陶的王旖。

尽管职业不同,佛印、李定、蔡奴兄妹三人在各自的领域都能冒出头来,这一点的确很有意思,成为世人的话题也不足为奇。但若是从三人之母仇氏的角度来说,想必她更愿意过着相夫教子、从一而终的生活。

严素心潜藏在心中的忧虑恐怕就有这个因素。加上她又是士人家的女儿,如果身份暴露出来,以其为妾的韩冈,就算不会受到律法上的惩治,也会被世人所责难。到最后,说不定就不被请出韩家家门。

“都这么些年了,难道你还不知道为夫的心?”韩冈也知道,这个时代的女子,只要不是正妻,往往都缺乏安全感,只是他没想到平常在自己面前都是笑语盈盈的严素心,竟然在她的心中,有着这么大的不安,“你们哪一个我韩冈能放下?再说,我可不会让我的儿子,连亲娘都见不到!”

“官人!”韩冈坚定异常的承诺,让美厨娘的声音颤抖着,鼻翼翕动,带着浓重的鼻音。

韩冈搂着严素心,“不过为夫还有条件。”

“什么条件?”严素心仰头问着。

韩冈低下头去,咬着耳朵说了几句。素心的娇颜,瞬息间红到了耳朵上,含羞带嗔,“你去找云娘和南娘去!不管官人你说什么,她们都会点头……”声音又低了下来,“上次离开前,南娘不是服侍过了吗,轻车熟路的,怎么不找她去?”

韩冈探手揉捏着严素心罗裙下修长笔直的双腿,笑道:“好菜要隔着顿来吃才好,这样才有新鲜感。素心你做菜不是这样吗?”

“……就一次!”严素心在狠狠瞪了韩冈一阵后,终于松了口,但立刻就补充道,“但今天不行,该由姐姐陪着。”

“那就明天好了。”韩冈像是很急的样子,一点也不给严素心逃避,又笑道:“其实现在在这里也可以。”

素心掐着韩冈的腰间软.肉,用力拧了一下,赌着气不理韩冈了。过了半晌,她却又低声问道:“官人,当真不要紧?”

“就算被人挑出来,也就是名声坏点罢了。”韩冈哪里会将这等小事放在心上,严素心的担忧落在他的眼里,倒是让人觉得傻的可爱了,周南的事都担待下来,难道严素心的这点小事他还担待不了,“到了为夫这一步,难道还怕坏了名声?就是犯了弃土的大罪,也不会受多重的责罚的。襄州辖下有个光化县,几年前叫做光化军。襄州不算大,长舌的到不少,在襄州多少日子了,你应该听说过曾经知光化军的另一位韩纲吧?”

“是韩子华相公的长兄?”严素心明显听说过与韩冈同音不同字的前光化军知军的‘光辉事迹’,只是不能确定。

“自然是那一位弃城而逃的韩纲。”韩冈语带不屑。

襄州辖下的光化县,熙宁五年之前还叫做是光化军。韩绛、韩维和韩缜这三位出身自灵寿韩家的高官显宦的长兄,前朝参政韩亿八个儿子中的长子,正做过一任知光化军。

但在他的任上,却不幸碰上了贼盗和兵变同时袭来。内外交困下,韩纲不是设法解决眼下的困局,反而是丢下了满城百姓,带着妻子儿女弃城而逃。在这一过程中,韩纲发挥出了超人的行动力,与全家老小一起,从城头上用绳子滑了下去。如果换成是其他背景不深的官员,项上人头肯定是难保了。但轮到韩纲身上,这样的罪名都没给斩了,仅仅是编管英州。

“这些衙内,先坏国事,再坏国法,该举家流放的罪名,一个编管就算是给光化军百姓的交代了。这受管束也就两三年,到了三年一度郊祀之年,便能受到大赦。”韩冈叹了一口气,“想起了这些人,为夫倒想起了一首乐府来。”

“什么乐府?”严素心转了心情,好奇的问道。

“举秀才,不识书。举孝廉,父别居。寒素清白.浊如泥,高第良将怯如鸡。”韩冈音声森然,“我是宁可大哥到五哥都是庸碌守家之辈,一辈子守在乡里,也不愿意他们挂着个贤名,去坏了国事。”

韩冈森然冷冽的语气,让严素心听着心里都觉得有些畏缩,勉强笑道:“二哥向来聪颖,不会丢了官人的脸。”

韩冈很快收起了脸上的寒霜,安抚似的轻拍怀中佳人的背部:“其实大哥也不差。”顿了一顿,“再过几天京西这里又有一位衙内要来了,名气也大得很——就是从年纪辈份上,这么叫他衙内也不太合适了。不知他到底是什么样的脾性,只盼他不要辱没他的父亲。”

