\section{第37章 蒿目黄尘顾世事(中)}

给远在京城的章惇写了一封信去,韩冈便毫不在意的将吕惠卿希望用来展示自己才干的《手实法》抛在脑后。在一天热过一天的元丰元年的初夏,他把自己大部分的精力都放在了襄汉漕运之上。

半个月的时间中,汝州和唐州之间的漕运道路,韩冈来来回回跑了四趟之多。从正在疏浚中的水道,到穿越方城垭口的轨道地基,他都仔仔细细的往返查看了几遍。都转运使如此上心,下面的官吏乃至厢军的官兵当然也不敢轻忽视之,

到了快六月的时候,工匠和材料陆续抵达工程现场。

筑路的工匠分别来自京城和徐州,总共二十多位。其中大工六人,每一位都拥有丰富的经验,在开始修建轨道前,分别都有着常年修造宫舍、桥梁、道路、堤坝的经历,几年来又不停的修建轨道,可以说是国中能拿得出手的最好的一班人马。

筑路的木料主要来自于南方,做轨道的硬木和做枕木的软木,经由汉水、泌水和堵水,运抵方城山下的木作中。皆是从江陵的船场运出来,已经经过了几年的晾晒,切割处理之后,就能立刻使用,不用担心新鲜木头脱水后的干缩。

而方城垭口北面,沿着水路而来的还有一船船的作为轨道路基的矿渣和卵石,当然,还有上万斤的铁料。虽说方城山附近没有冶铁的矿渣,也缺少卵石,但千里迢迢的从京畿将这些沉重的原材料运来,更多的还是想测试一下方城山北麓到蔡河的漕运是否畅通运行。

除去已经准备动工的堰坝,年底之前襄汉漕运便能初步打通。剩下的就是能运多少的问题。同样的一条道路,如果调度指挥出色,单位时间的运输量翻个一两番,甚至上升一个数量级都是有可能的。

当然,韩冈不会指望这个时代的运输调度,能比得上后世的专业人才,但即将为此而设立的发运司,韩冈期望他们至少能有六路发运司和三门白波发运司的平均水准。这事现在来想虽说是早了一点,但早一天练上,便多一天经验。韩冈只想看看实验的结果。

结果当然很完美。事实证明百年前在方城山山南山北的开凿出来的漕运通道至今依然能够使用,指挥调度的官员也是十分出色。不过卸货的地点离着方城山有些远,接近垭口的两段都要再疏浚一番方能使用,而在计划中,更是要通过堰坝提升水位并设置船闸,以连通深凿后的垭口渠道。就是因为此事,当设堰提水的方案敲定之后,暂时用来连接方城山南北的轨道便不得不加长了三十余里。

一旦堰坝提升了水位,之前的疏浚河道就是无用功。而相比起疏浚的工程量和对时间的延误,还是修建轨道更简单一点。不过这样一来,就多了一桩麻烦——修桥。

方城垭口前后也是有河流的,南面的是堵水的支流,北面的则是沙河的支流。如果襄汉漕运的中转点能向上游移动,可以避过这些支流,但眼下的情况,却必须设桥跨过去。。

总共四条溪流,每一条都不宽,平时最宽的一条也就四五丈而已。以此时的桥梁建筑水平来说,可算得上是轻而易举。但每隔几年,方城山一带往往就会有一次雨量偏大引发洪水的年份,百年前漕渠开凿失败,也有沙河堰坝被洪水冲垮的因素在,要怎样避免跨河的桥梁被冲毁,也是一桩难题。

“要跨过这条三里溪,还是设石桥比较好。方城山不缺石头。”李诫在溪边对来巡视的韩冈和沈括说着自己的意见,“木桥要防洪,桥拱必须要抬高,可抬得过高,车马难行,换成是石桥就方便多了,也坚固得多。可以赵州桥样式为范,设敞肩石拱,一大拱挑四小拱,跨过河道的行洪区域,桥拱弯曲,桥面平缓,正好适合轨道通行。”

“赵州桥的样式也记得?”韩冈有几分惊喜。

“石拱桥多半大同小异,”李诫很有自信的说道,将石拱桥如何修造,一条条的说给韩冈和沈括听。

韩冈连声赞许。李诫对他来说算是一个惊喜。他在建筑营造上的才能不仅让韩冈为之激赏,也让沈括赞叹良久。

“野有遗贤……”沈括叹了一句,又觉得这个说法不太合适,“难得人才,竟遗珠于外。”

李诫闻言连忙谦虚:“学生自幼不喜学,唯有工匠之事稍有心得。”

韩冈冷笑一声:“世间只重诗词文章。可经世济用之材,岂是区区章句能衡量得了的。”

李诫又为韩冈的夸奖而自谦了几句,聊了片刻,便向韩冈、沈括辞行,他还有事情要去做。

待李诫走后,沈括私下里问着韩冈道:“玉昆是打算举荐其人?”

“当然。”韩冈点头。

“那是李南公的儿子。”沈括提醒道。

韩冈身为都转运使,不但将转运副使的儿子收为幕僚,甚至还委以重任,这肯定会为人诟病。若是再举荐李诫为官,那就不是诟病的问题了。

“没关系。”韩冈无所谓,此事不会影响到他地位的稳定。

韩冈既然是这个态度,沈括也就不说了。将此事细细想来,韩冈的自信其实也没错,相对于他受到的信任,的确是没关系。

“如今唐州的钱粮也收上来了,路中前半段的需用应该没有问题。”沈括问着,“是否要加快进度?”

夏税征收是一年一度的重头戏,端午之后,唐州用了二十天才收了额定的八成,剩下的两成欠额,看往年的情况,多半要到秋税时才能补得上来。这半个多月来,沈括的时间和精力都是放在收税上。

“快一点、慢一点都无所谓,能在时限内完工就行了。”韩冈说着,沿着溪水河畔走了起来,走了两步,见沈括跟上来,就偏头笑道:“唐州收税难易如何?”

“京西民风彪悍,税赋征收不易。”沈括摇摇头,“要是在两浙,半个月的时间至少能收上来九成。”

“也是两浙富庶,京西的收成本来就不算多,交的税却不比江南少多少,自然要难些……但有个七八成,也够抵上朝廷要的数目了

。”

“说得也是。”沈括点了点头,“能收的都收了,剩下的就算再催逼,也不一定能收得回来,弄得百姓卖儿卖女就不好了。”

“将下面的胥吏管束住,收满足额也不是难事。”韩冈说道。

“如何管得住?”沈括叹了一声,韩冈说得根本是废话,“重禄法也只让他们的手伸得短了一点而已。”

大宋的国策是虚外守中,除了边州,禁军兵力都集中在京畿,外路为数寥寥。而在经济上也是如此,除了边州以外,国内各路的每一个州县在留足了一定的积存后,剩下的税入都要上交朝廷。上缴的税入,额度基本上都是确定的,但这个额度并不是征收的数目。

夏秋完税之日,官府从百姓手上收钱,从来都没人能指望可以百分之百的完成,预定征收的税额远比实际需要要大得多。一般来说,能征收到七八成就能有足够上缴朝廷的数目,以及补完州县一年来消耗的仓储。至于多下来的,也不可能私分掉,照样要运回京城。上交的多,就能为当地守臣换回一个优良的考绩,以及一个干才的评价——可要是在征收的过程中,闹出了乱子来,亲民官们就那就别指望能有好结果了。故而能收个七七八八,剩下的欠账着人去慢慢督促就是了,官员们一般都不会逼得太紧。

但并不是说这个时代的官府治政有多宽松温和,实际情况正相反,只是把住了不让百姓造反的底线而已。在田赋丁税之外,还有折变、支移等名目繁多的附加税,这些钱素无定额,全凭税吏们的一张嘴。使得百姓们最终交到税吏们手上的钱粮,许多时候都能涨个五成六成,甚至一倍、两倍。这些钱,则是可以私分的。交一部分给朝廷做个样子,剩下的大头则是官员、胥吏各自分肥,这早已成了世间通行的规则,也仅有少数官员能够做到清廉二字。

说起来,如果当真按着税额来征收,将苛捐杂税一概罢去,倒不见得会有几人逃税。王安石当年提议变法的时候,在一系列的奏章中都提及了此事,谓此乃致乱之源。因此之后颁行于世的新法,对中等以下的贫民多有倾斜——免役法向五等户征收的免行钱也不算很多——但朝廷收入上的损失和增加的部分,则是让富人承担了去,得到的骂声比以前更多,就是良民为盗的可能性却小了不少,不复仁宗后期,欧阳修在奏章中所说‘一伙多过一伙’的盗贼遍地的情况。

收税的事,韩冈说说也就算了,也就暗叹了一声。时代的风气不是他一个都转运使能扭转得过来的,就是他面前的沈括,虽说不上贪腐,但一般的灰色收入也不会清高的放弃。

但只要襄汉漕运能就此打通,到了那时候,漕运沿线自然会繁华起来,此地百姓们的生活当会比眼下要过得好上一些……如此,足矣。

