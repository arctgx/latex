\section{第16章 山入四荒更郁苍(下)}

“辽国和水师的事,还是放一放再说吧。近几年,朝廷用兵的重心也不会在北面。”

宗泽应声道:“只要朝廷调集精兵强将,西南指日可定。”

韩冈笑了一笑,命人进来收起地图。

方才的一番对谈,宗泽的回答并不是很完美,但那时因为他所处的位置还是太低了一点,看得不远也不够全面是很正常的一件事。

眼光放在未来,宗泽在北事上,应当能够起到更大的作用。

不过相对于用水师控制渤海、黄海一带,现在韩冈更重视南洋运输线的保全。两广的出产,尤其是逐年增多的粮食输出,是稳定国中粮价的关键。而无数南洋特产,也让国内的商业更加繁荣。

南洋运输线上,载重量超过万石的大型海舶已经出现了十几艘,而中小型海船更是成百上千。

这么多海船,每年从两广将当地的粮食、香料、海产等特产运抵扬州,再从扬州将丝绸瓷器等特产返回两广交易给广州的大食商人。一年之内,往来多次,运送两三百万石的粮食轻而易举。

这才是水师现在的重心,在攻打辽人之前,先拿海上的海盗历练一下——竞争对手,自然是越少越好。

韩冈让宗泽坐下来,继续说着之前的话题:“平定西南不是那么容易的事。你也听洛阳的文老相公说了,太祖玉斧划界啊……过河不易。”

黄裳在西南的几年,辟地千里说不上,但他那边的情况的确十分顺利。先期是他探查形势,分辨敌我,做好储备,之后便是曾经有西南作战经验的赵隆领军,直接性的将一干不肯顺服的部族给剿灭掉。

久经烽烟的西军士兵们的手中,有精铁打造的甲胄、武器,还有威力强劲的弓弩,更有虎蹲炮,一个十人小队两门炮,用散弹就能打垮数百敌人。

在有了硫酸之后,出现在天空中的不仅仅是热气球,也有了氢气球。尽管氢气球很危险,硫酸同样危险,可火药照样危险,对军队来说,只要有用,只要效果好,这点危险不算什么。依靠天上的眼睛,敌人的埋伏多少次都化为了泡影。

更重要的是,西南夷中部族极多,相互间仇怨极深,黄裳拉一派打一派,地理人情都顺利的掌握在了手中。

训练更胜一筹,武器更胜一筹,还有诸多带路党,就连最大的敌人——疫病——也减轻了很多,反乱的西南夷当然没办法与官军为敌。

但接下来,官军所要面对的敌人,就不是手下仅有几千一万人的洞主,而是南方大国大理。

半年前,大理段氏密书至京,痛诉权臣凌主,恳求大宋太后为其做主。这个消息传出来,朝中登时就是一片哗然。既然朝廷痛斥耶律乙辛篡位,与之绝交,那么大理国内的篡逆之举,也不可能坐视,尤其是大理国主卑辞告求,让士林和民间都开始呼吁出兵。

大理或许没有西夏那样的辉煌的战绩,但唐时的南诏,却是几次与大唐的军队交锋,也多次取得了胜利。

文彦博拿着太祖玉斧划界的故事来阻挠对大理的征伐,也有很多人担心攻打大理的战争会旷日持久,朝廷无法支撑。这样的逆流,想要压下去,颇费点劲。

宗泽慨然道:“太祖说以大渡河划界,可大理国中权臣欺主,我大宋岂能坐视不管?为藩国拨乱反正,乃中国之任。”

大理国中,段氏世代为王,高氏世代为相。不过近年,段氏衰弱,高氏日强,几次有流言说大理高氏代段,自行称王。

大宋作为华夏正统,当然不能容忍这件事的发生,辽国的耶律乙辛对付不了,区区大理高氏还对付不来?等打下来后,再让段氏献上舆图田籍,这就名正言顺了。

其实朝中对于平灭大理,有着高昂热情的官员也相当的多。

攻下大理有两个最主要的好处,一个是大理的以人口土地和银矿为主的资源,另一个就是军功。

攻打大理的方略早已议定,朝廷不会征发大军,也无意在西南的崇山峻岭之中投入数以万计的大军,而将会采用攻打交趾的方法,选派一万左右的精锐,同时调集可以动用的西南夷参战。

以官军为刀刃,蛮兵为刀身,联手覆灭大理。等到拿下大理之后,就将其中大部分土地分割出去,并迁移一部分西南夷来此安居,朝廷只取走洱海、滇池周围的一部分土地,安置汉人。

灭国的功劳,只看着朝堂最上面的几个人,就知道有多丰厚了。自问有资格参与进来的朝臣,这两个月都在紧锣密鼓的为自己谋划一个能够博取功劳的位置。

“汝霖,你想不想去?”韩冈问道,“参赞军务。”

宗泽心中一跳,面上却不动声色,“宗泽当然立功于外,以报朝廷厚恩,但不是已经人满为患了吗?”

韩冈微微眯起眼睛,“不愿意?”

“不,与其让那些希图幸进之辈败坏国事,不如让下官去。运筹帷幄不好说,但参赞军务,拾遗补缺,下官自问比那等人要强上一筹,于国于军,都是好事。”

韩冈沉默的看了宗泽片刻,哈哈笑了起来,“说得好,就该有这样的气概!”

“不敢。”

“不过那等人也不能全部拒之门外。不坏事,不抢人功劳,这样的人来也无害于事。”

王中正就是最好的例子。

尽管他很多时候都是混功劳的,可是王中正不贪功,不惹事,谨守本分,还能在天子面前帮着说好话,关键时候也能帮把手,这样的人过来分功,让人给的也乐意。

这一回,王中正也有心在南征之事上分上一杯羹。不过并不是他想要再去西南,而是想要给他的儿子运作一个职位。

王中正的年纪不小了,但精神还好,在军中有威望,太后也十分信任他。去年曾经太后有意封他为节度使,不过给王中正给辞了。但谁都知道,只要他致仕,节度使是手到擒来,不必像其他内宦一样,非要病死才能得到追封。

但他的儿子就没有他的幸运了。

地位高的宦官收养养子是惯例,不过为了防止宫里的大貂珰通过这种手段来扩张自己的势力,朝廷也规定收养者年龄、官职,以及被收养者人数的限制。但在宫外收养的义子,就没有那么多规定,那是继承香火用的。

王中正早年在宫中先后收养的两名养子都死得早,之后就没有在宫中收养义子,而他在宫外的养子王祁,以节度使留后的嫡长子的资格得到了荫官,正式授职后,很快就升到了内殿崇班。但也仅此而已。

如果是武官,做到王中正这样的位置,完全可以通过联姻的办法,为儿子再铺上一条道路。但阉宦的儿子想要结亲,找不到多好的人选,岳家的帮助不用指望。

再往后,就得看他自己的本事了,王中正也帮不了他许多。所以王中正前日求到了韩冈这边,希望给他的儿子一个机会。

有王中正的事例在前,想去边疆立功的宦官为数甚多。包括李宪,也包括童贯。

童贯这两年正得宠,不过因为缺乏军功,虽然还在御药院中任职。可他肯定要出外一趟,现在正在上下活动,争取一个去西南的机会。

“下官明白。”

宗泽也知道韩冈最近是如何被骚扰,多少故旧来向他讨要一个名额,好像大理的军功是路边的石头,俯首可拾。

“其实这件事也可笑。”韩冈笑着,“朝廷连主帅都还没选出来,现在就开始争了。当真要打,除了三两人之外,剩下的都得让主帅自择。”

“相公说的是。”

“汝霖,兵出大理,你觉得何人适合为帅?”

“此事非宗泽能言。”

“但说无妨。此事非一人可决,需在朝堂上商议。你姑且一说,我姑且一听,无碍国事。”

韩冈说得坦然,宗泽便不再推脱,“当以熊龙图为主,黄直阁为副。其下武将,也当以曾在西南用事过的将校为宜。”

龙图阁学士熊本,龙图阁直阁黄裳,都是在西南有所成就的文臣。

尤其是熊本,他是西南方面的专家,熙宁以来,朝廷对西南夷的战争,一直都是在他的主导下进行,也是朝中不多的几位能担当主帅的文臣。

宗泽说完,静静等着韩冈的回应。他担心韩冈会提议让王中正为帅。有过领兵西南的经验,王中正也是主帅的人选之一,但那毕竟是阉人。去大理的路并不好走。想要攻下大理,必须经过一段艰苦的行军,没有得人望的将帅统军,走完这一段路之后,仗也不用打了。只是熊本是新党,否则他这个主帅的人选,是不会有任何争议的。

“熊伯通的确合适。”韩冈点头,“黄勉仲都差了他十年平蛮的经验。”

不管两党如何竞争,韩冈都不打算打破底限。如果有两个合适的人选,他会选择贴近自己的,但人选若只有一人,那他就不会因为对方不属于自己派系的成员,而横加干扰。

不过这主要还是现在两党之间还算和睦,否则韩冈会直接阻止这场战争的爆发,根本就不给人以争夺功劳的机会。

攻打大理并非急务,段氏的密书也不过是个由头。维持禁军战斗力、同时实验火器在军中的使用,找出合适的战法——总不能在面对辽人的时候,火炮还是第一次上战场。大宋周边,也只剩大理一家,可以用以练兵。而且,突破太祖皇帝的玉斧划界,这一战背后的意义更大一点。

但开战的好处,也不过仅此而已。

得到韩冈认可,宗泽更加沉稳,“若有熊端明为帅,黄直阁赞辅,此战或不敢说必胜,但必不至大败。”

韩冈沉默了下来,等得宗泽心神不宁,韩冈才再次开口,“要是有熊本在,才能保证不大败,那就已经败了。何至于此?”

宗泽不明所以,韩冈也无意解释。让日后的变化来说明吧,这个时代的人们,是无法明白的。

火器最大的好处,就是对士兵的体力要求不是那么高了,就算因为行军累得拿不动刀、使不动枪,背不起甲胄,拉不开弓弩,但只要他们能够摆好虎蹲炮,装好弹药,然后点燃就够了。

传言说欧洲曾经禁止弩弓,因为十字弓能让农民将骑士射死在泥地里。可就是重弩,因为还要耗费力气去拉开。而一个小孩子,只要拿得稳火枪,也能有机会将万夫莫敌的大将的脑袋打成豆腐渣。

只要稍加训练,就能力克强敌,这让汉人在人力和国力上的优势能够彻底发挥出来,不会像冷兵器的时代,数万蛮夷也能欺上门来。

统领一支久经战火,且装备了火炮的精锐大军,怎么只能保证不大败?!那样又何必开战?

韩冈需要的是胜利。

以己之长,攻敌之短,将敌人拉至自己擅长的领域将之击败。如今火炮已经装备军中,火器局中,每天都有更多的钢铁和青铜被铸造成型。不能再耽搁,必须尽快让大宋禁军适应火药武器,抢先一步进入热武器的时代。

很久以前,韩冈就希望大宋的胜利,是生产和组织的胜利,而不是名将的灵光一闪,更不是依赖于士兵们勇猛无畏。那样的胜利,才是真正的胜利。他为此努力许久,为的可不是一句不会大败。

“大理。”韩冈轻声道。

目标到底能否实现,就用大理来做个试验。

