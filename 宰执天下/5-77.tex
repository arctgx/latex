\section{第十章 却惭横刀问戎昭(九)}

【算是第三更。昨天夜里写着写着就睡着了,今早起来才算补完。】

夜里韩冈吃得是山中的野味。

秋天的兔子、麂子都是肥嫩可口,放养的鸡鸭也是,但雁门寨里的厨师水平不行,大概是平时舍不得做菜放盐的缘故,将盐看得很重。今天来得都是显贵,盐只管往多里放。

当今之世,盐价并不便宜,所以这样的厨子,韩冈着实见过不少,也听严素心说过。这个时代没有厨师的等级认证考试,也难怪一些讲究的大户人家,出门都是带着自家的厨师、厨娘。

吃了两口之后,就连刘舜卿都受不了了,拍了桌子将雁门寨主叫过来。就算他不在乎饭菜,但经略使韩冈就在这里,把饭菜弄个如此难以入口,也是给他这个代州知州丢人。

“淡就多吃菜,咸就多吃饭。”韩冈拦着刘舜卿,他并不是很在乎口腹之欲,时间长了虽然不惯,但一顿两顿吃得差点也没什么好在意的,“吃饱了就行。”

归根到底,还是盐业的问题。河东食盐主要来自解州。尽管河东和山西几乎是一个概念,可后世的山西产盐,但这个时代的河东偏偏不产盐。关键就在解州,也就是后世的运城。此时的解州,在地理上更近于河中,在区划上属于陕西,跟河东的关系,仅仅是解盐的专卖之地。

“盐卖得贵,人吃得少,当然就当成了宝,有机会多放盐的时候,就不会浪费。”韩冈有心改变一下现状,但困扰大宋君臣多年的盐政,相关的既得利益者盘根错节,当年初行盐钞法,被刺杀的官员也不是一个两个,并不是一名经略使就可以解决的问题,现在能做到的也就放过雁门寨主和寨中的厨子一把,“今天的情况,并不是厨子的问题,为此苛责就不必了。”

韩冈既然不在意,刘舜卿当然也愿意做个大方。他本意也不是想用这样的罪名处置自己的部下,雁门寨主也算是他的亲信,只是想先一步发作,防止韩冈先说了重话,让自己留情不得。

不过韩冈也吃不下跟腌肉相媲美的烧肉,下面的士兵或许会吃得很开心,但不缺盐的官员、将领听了韩冈的话之后,都苦了脸,只有雁门寨主一个人感激涕零。方才刘舜卿发火时,他脸都白了。

韩冈直接用茶水泡了饭,一向随身带的炒青茶叶,用来泡饭倒是正正好。茶泡饭一向吃得省事,口味又不错,而且还不嫌油腻。当然,也只有炒制的散茶可以这样用,要不然就是更早的时候,加盐、加香料的茶水,那种放了龙脑的龙团,可是没办法让人配着饭下肚的。

刘舜卿则是放下碗筷,宁可饿肚子也不吃了,对韩冈笑道:“这荒郊野外,想遇到一个好厨子,就跟三月在开封城中想撞上一个头上不带花的一样难,还望。”

三月帽上簪花,是东京人的习俗——最近似乎又向外传播开了——无论男女老少,到了这个时节,都少不了在头上簪一朵花。新科进士少不了戴上一回,天子出游金明池也照样不能免俗。而在河东、陕西这样民风淳朴的地方,就是当做猎奇一般的轶事来谈笑。

不过东京城实际的的情况,也没有刘舜卿说得那么夸张,不带花的比例少,但以京城人口为基数,使得总数并不少。韩冈本人也除了中进士的那一次,之后也从不带花。不过就当笑话听好了,世间的流言本就颇多,不在乎多这一个。

但有的流言就让人无法笑出来了。

半夜里,西陉寨的方向突然有信使叩关意欲夜入寨中。等到韩冈起身,主寨北侧的军营中,已经是一片骚动,辽人来袭的流言随着信使的马蹄声一起传遍了营中。直到雁门寨主将他的亲兵散出去镇压营地,才逐渐平息下来。

但也并不是全然是流言,也有一部分的正确成分。韩冈和刘舜卿的面前,赶来禀报紧急军情的西陉寨小校火烧火燎:“相公、太尉,大约两千辽骑已经进驻大黄平,寨前的车场沟也看到辽人的游骑。寨主,命小人来报与相公和太尉。”

所谓相公和太尉,只是民间对高层文官及武将的称呼,韩冈和刘舜卿都不到那一层。但韩冈并不在意这些,刘舜卿也没空尴尬。

“车场沟就是西陉东谷吧?”韩冈遽然问道。

“回相公的话,正是西陉东谷。”来报信的小校甚至有几分惊异,毕竟能一口报出当地的详细地名,这样的官员并不多。

韩冈扭头又对刘舜卿道:“记得当年与辽人论北疆划界事,当时双方谈判的地点似乎就是在大黄平。”

刘舜卿点头:“正是……经略博闻强记,”

韩冈笑道:“做了河东经略,只是想尽量多了解一点河东。之后了解到的的确不少,但不知道的则更多了。”

几年前割让代北地的谈判就是在雁门关外的,一开始谈判地点本来就定在西陉东谷,也就是车场沟,但负责谈判的吕大忠认为那里是无可争议的大宋领土,所以坚决不同意——边界谈判的地点应该是两国的交界处。光是为了谈判大帐的位置设在哪里,双方就争论很久,好不容易才定了下来,放在大黄平。外交无小事,即便是有着千年的距离,道理依然是相通的。

不过大黄平的地理位置尽管划界前是位于宋辽两国的中线,在划界之后,却已经属于辽人,离西陉寨有十余里。辽军进驻此地,只是他国中的事,只有游骑侵入西陉东谷,才算真正意义上的犯界。

只是辽人一下动用两千骑兵——就算照惯例在军情上打个折扣,也有一千。这已经不是可以忽略不计的数字,要预先做下的筹备可不是张张口就能办好的。可这么大的军事行动,怎么都没有细作事先打听到?韩冈很是有几分疑心——除非只是来前线打个转而已。

“相公、太尉!两千余名辽骑中,有四百到五百骑是配三马的精锐。”小校见韩冈和刘舜卿并不在意,急着想跳脚,“他们不是宫帐就是皮室,绝非等闲辽骑可比!”

韩冈略略有些惊讶,这名小校胆子还真大,说话的态度让人感觉其中少了一份恭敬。

刘舜卿眉头也皱了起来,“宫分军也好,皮室军也好,都是骑兵吧?”

“……是。”

刘舜卿脸一翻,一声暴喝:“既然是骑兵,秦怀信难道还担心他们攻城不成?!你爹什么时候胆子变得那么小了?!”

原来是西陉寨主的儿子。算是解开了韩冈心头一个疑问。

不过这样的恍然,也只是在脑中一划而过,转瞬即逝。正经事还是在西陉寨面对的辽骑上。不过就如刘舜卿所说,其实并不需要太担心。

像雁门寨,主寨在勾注山颠,而南北向下又设了几道营垒,两侧山壁上,也有箭堡,加上烽燧、望台,由此组成了一个南北七八里的寨堡防线。西陉寨的情况,与雁门寨类似,并不仅仅是单纯的一座寨子,以辽人的攻城水准,想要攻下这样的险隘,不付出数倍于守军的代价那是不可能的。

不论是韩冈,还是刘舜卿,都觉得辽人不会蠢到硬碰连绵于河东山中的无尽寨堡。不过刘舜卿考虑的要多一点。

“秦怀信一向武勇。区区一两千辽骑,绝不至于慌乱不堪。当是其子大惊小怪而已。”刘舜卿看看韩冈的神色,又道,“不过事有万一,以末将愚见,当是先派上两个指挥去西陉支援一下为好。”

韩冈点点头,下面的战术问题他并不打算干扰刘舜卿的指挥:“就这么办。”

天亮之后,一名辽人的使节被领到了韩冈面前——依然只是讹诈。

还是前几日,被派到代州城的使节。上一次是以朔州的名义出面,韩冈没有理会他,不过这一回,则是声称带着北院枢密副使萧十三口信,韩冈却不好不见。

成为了萧十三的传信人,信使趾高气昂。昂着脖子,向着韩冈微微一欠身。弯腰的角度,不仔细看,还觉察不出来。

“好胆!”

“无礼之辈!”

几名将领齐声怒喝,韩冈是什么身份,区区一个信使竟然连应有的礼节都欠奉,这哪里是平等相对的两国,分明是上国来藩属宣示的样儿。

韩冈抬手拦住正欲发作的刘舜卿和众将:“大宋乃礼仪之邦,自然重礼。但不能用大宋的标准苛求外国,须知华夏只有一个。”

韩冈话声一落,顿时哄堂大笑,在列的将领立刻就挺胸叠肚,开始用眼角瞧人。

信使涨红了脸,可在传说颇多的韩冈面前,却不敢发作。

先帝耶律洪基死在他的发明治下,辽国国中有人归咎于耶律乙辛,但也有人认为这就是发明之人韩冈的手段,尤其是韩冈又发明种痘法的消息在辽国传开之后,持有后一种想法的就越来越多——其中也有耶律乙辛为了转嫁罪名,暗中推波助澜的因素在——使得韩冈在辽人心目中的形象,也变得有几分神秘和诡异起来。

犹豫再三,信使终于勉勉强强的向韩冈又行了一揖。动作有几分僵硬,惹来了几声嗤笑,让他的脸色更行紫胀。

当他行过礼,正想要将萧十三的吩咐一一宣示,好出一口气,却见韩冈抬手阻止:“贵国不顾盟约提兵犯境,实乃背信弃义。不论萧副枢开出什么样的条件,无论好赖,都是城下之盟。我都不会答应的,你就不必多费口舌了。”

信使怔住了。那有这样的说法?!

韩冈的态度甚至让他的麾下将领震惊,刘舜卿都没想到他竟然这般决绝。但转眼之后,他们却又热血沸腾起来,若是听了萧十三开出的条件,那般也是憋屈,还是这样痛快!

韩冈扯了一下嘴角,化作一抹浅笑:“承天太后和圣宗打到澶州城下后还能回去,太师若是领军入境,还指望能回去吗?不想让太师平安北返的不知凡几。对于太师,我大宋天子其实颇为期待的,期待他能让宋辽两国之间的友谊天长地久。如今太师秉国,两国却起了纷争,那就太让人伤心了。”

韩冈沉稳的嗓音传递在厅中,“请回去告知萧副枢,大宋与大辽乃是兄弟之邦,这份情谊,希望能一直保持下去,如若不然,非是两国之福。还望副枢能够三思……来人,送客!”

