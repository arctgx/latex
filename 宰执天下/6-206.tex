\section{第23章 奉天临民思惠养(下)}

暮色将临,内东门小殿中的会议仍在继续。

政事堂对上一个五年的成就的总结还没有结束,接下来的施政方针也在计议。

熊本拿着笏板,出班奏道:“如河南、京兆、大名、太原等府,为一路之中,皆是户五万乃至十万以上,田地数十近百万亩,足以养军。而云南路初定,汉家户数,仅有一千八百零七户,四千余丁。其耕牧仅足自用,不足以补军需。用兵则仅以自保一城,亦难以克敌制胜。”

“嗯。”向太后应了一声,示意熊本继续说下去。

“依臣向日所计,昆、理二州,至少各有两千户迁入,才能达到税入和日常支出的平衡,若要支应云南一路兵马所需,至少都要达到万户方可。而入滇道路沿途诸县,平均每县也至少需要五百户汉人,才能保证过路车马的日常补给,千户以上方可确保县中安定,不虞乱贼。”

“熊卿。”向太后有些不耐烦,“云南一路,总共要多少户汉人?”

熊本道:“下则至少需要八千户,中则需三万户,上则多多益善。”

“八千户……这数目可不少。”

如果是刚开始执政的时候,向太后多半会说‘八千户,不算多啊,一个军州出二十户,四百军州八千户就满了。’但现在她已经知道,这是一件不可能的事。

“但八千户口,对于稳定云南路,是必不可少。臣闻韩相公昔年随王襄敏开拓河湟,第一桩事不是剿灭蕃人,而是设法在当地屯田,种植棉花。只有汉人能在当地稳定下来生活繁衍,这块土地才能真正属于中国。”

地方财政长期入不敷出,在有心人的推动下,朝廷上很容易通过放弃那块土地的决议。

“这个道理,吾明白。此事就交给黄裳去安排,至于户口迁入,相公,各地的济养院怎么样了。”

韩冈道:“各地济养院都已经上报修成,中书也已颁发了条贯,具体是否有效,则需等到施行之后方可一观后效。”

“这件事有相公主持,吾就放心了。这是好事,事关百姓,得做得妥贴了。”

“必不致使陛下忧心。”

济养院关系到的只是乞丐和流民。对官府来说,只要不死人,怎么安排都可以,太后也只是顺口一提,并没有太重视。

伸了个懒腰,又喝了口茶,向太后问道,“西北、西南的事都说了,下一个是什么?”

“是铁冶之事。”韩冈道:“此事由邓润甫禀于陛下。”

“邓卿,你来说说吧。”

邓润甫依言出班,“钢铁产量今年继续增长。东京铁场,去岁生铁产量总计二百三十万石,由于前一年改进了炼钢法,产钢量也达到了五万五千余石。徐州铁场,生铁八十万石。磁州铁场,生铁三十万石,钢五万石……”

“磁州的钢怎么这么多?”向太后打断了邓润甫的发言。

东京铁场的生铁两百三十万石,钢才五万多石,而磁州的生铁产量三十万,钢也是五万石。这个比例未免相差太过悬殊了。

邓润甫道:“如今刚刚改进的炼钢法,正是磁州铁场的铁工高虎所创,首先实行于磁州,亦名为高氏炼钢法。”

这是与现今通行的动植物命名法相类似,以名利诱人,吸引后来者。《本草纲目》至今未成的缘故,有一半是为了要辨别来自四面八方、越来越多的新发现的动植物。

“高氏……”向太后明显的不太喜欢这个姓氏。

太后的低语从台陛上的那面屏风后传来,在场的朝臣一时无言。有谁不知道向太后的这个心结,但这也太敏感了一点。

韩冈出班道:“高虎此人祖孙三世经营铁冶,本人也是久为铁工,磁州铁场以其为督工三年,钢铁产量年年大幅增长。年前中书有表奏上,表其为官,以酬其功,陛下是许了他的。”

向太后仔细回想了一下,印象中似乎是听过这个件事,“原来如此,吾的确记得。如果这个高氏炼钢法有,铁多自是好事,钢多了那就更好了。”

苏颂、韩冈领着宰辅一同赞过太后的英明,邓润甫继续列举今年的钢铁行业的成果,最后总结道,“……民间铁冶难以计算,官营铁场去年的产量总计五百八十三万石。比上一年,增加了四十二万石,增长九个百分点。”

“仿佛没有去年的增长率高?”向太后一直在认真听着,听到最后一句,立刻发问:“记得去年是百分之十一吧?”

“陛下明察,那是因为去年年初江南东路的太平州【马鞍山】铁场完工,并开始出铁了。”

“这样啊。”太后恍然,道,“没有新铁场出铁,去年还能增加百分之九,当真是难能可贵了。”

章惇看着太后与参知政事之间的对话,突然间觉得有几分怪异的感觉浮上心头。

如果是十年前,邓润甫和太后在朝堂上的这番对话,怕是没几个人听得明白,什么叫做增长了九个百分点?什么叫做没有去年的增长率高?

懂算学的听不懂,不懂算学的更是听不懂。这遣词用句太过特异,即是精通算学,乍听了也不知所以然。就像那些应用题,如果不能理解题目中文字的真实意义,算术再好,也只会得到一个错误的答案。

而这一切的源头,自然是站在对面的韩冈。

这种用词方法,最早来自于《自然》,随着时间的推移,已经在逐渐改变朝廷中人说话的风格。甚至太后都习惯了这样的数字列比,简单又直观。

仅仅从这一件小事上来看,韩冈对世间的影响力是越来越深了。无论朝野内外,仅仅是说话做事的方式,都受到了他的潜移默化。

章惇记得上一次,韩冈还让人依照朝廷的支出画了图来,图纸上只有一块圆形,从圆心引出的条条直线,将这个圆形图案分割成大小不一的扇形。韩冈就利用这个扇形,用不同颜色,表明了财政开支的具体对象。这就像一块烧饼,谁占了多少,那是一目了然。

军队占了最大的一块饼,宗室的补贴,官吏们俸禄,也同样是巨大的支出。冗兵、冗官、冗费之外,其他的开支就少得可怜。即便是功在当、代利在千秋的厚生司,从朝廷手中分到的钱,甚至不能保证对医学的投入,还要依靠医院和保赤局的收入来支撑。

所以当太后看明白了那幅图之后,立刻就加大了对厚生司的支持力度,但对军费的开支,暗地里则颇有微词。

邓润甫总结完毕,韩冈接着出班:“五年前,天下钢铁产量,仅与今日的东京铁场相当,比起五年前,天下的钢铁产量增长了一倍还多。若是五年之后,理当再增加一倍。”

东京铁场的年产量,比千年后的村级钢铁厂还不如。但在世人眼里,这已经是让人瞠目结舌的飞跃,这是几年前天下一年的铁产量,若是放在熙宗皇帝即位前,更是连一半都没达到。钢铁业大发展,自然就是这五年来掌握朝堂的几人的功劳。

“这两年的增长率,都在一成上下,要是再过五年七年,就是又翻了一倍。”向太后问道,“朝廷还用得了这么多铁?之前铸币局还说,今年计划新铸的铁钱还是两百一十万贯,增加的铁料还用得出去吗?”

天下生铁,有很大一部分化为钱币,币制改革的前两年,每年铁钱的产量是四百万贯,几乎占去了官营铁场产铁量半壁江山,这两年,钢铁产量大幅增加,而铁钱因为要保证币值,每年只新铸两百余万贯新钱。

“回陛下。”韩冈道:“铁钱耗用比之前虽少了许多,但熟铁炮经过了大量实验,终于定型。日后火器局铸炮,三寸、四寸口径的火炮,都可以使用铸铁,而不是过去的青铜。铜料可以节省下来许多,但铁料的消耗却大大增多。仅仅是为了满足军中的需要,也需要大量的钢铁。此事,章枢密最为了解。”

“沧州泥姑寨,三女寨近日刚刚重修完成,其中泥姑寨六寸榴弹炮四门,四寸榴弹炮二十二门,三寸子母快炮六门,虎蹲炮三十七门。三女寨六寸、四寸榴弹城防炮与泥姑寨数量相同,子母快炮八门,虎蹲炮三十门。包括大名府在内,河北一路,配备火炮的城池、寨堡,总计七十三处,虎蹲炮不计,三寸及以上火炮数量共计一千一百九十四门。”

“一千两百门了。”太后四舍五入的题目做得飞快,“不少了啊。”

“不,陛下,是太少了。”

“平均到每一座寨堡,还不到二十门。因为有的寨堡火炮多,使得有些州县只有四五门火炮防守城墙。大名府十万户,城中人口十余万,驻兵近两万,为京师北门。如此要地,却只有八十余门轻重火炮,平均一里城墙,只有三门,如何能够防守?”

向太后沉默了下去。

韩冈忽然抬起头,看了眼屏风后那隐约可见的身影,隐隐能感觉到她现在心中的不快。

章惇别的都好,就是总爱瞧不起人,前些日子,在家里见外客的时候,穿了件闲散道袍出来,明显是对人不尊重。此事传出来,士林中多有议论。

他对太后虽然明面上尊重,但这话的语气也仿佛是在教训人。

“陛下明察。”韩冈出面缓和气氛,“河东有雁门天险,而河北全无,若想使辽人不敢犯境半步,便必须用火炮让河北变成金城汤池。”

“嗯。”太后听起来很勉强的应声。

“各地军中,也都需要更多的火炮和火枪。军器监的产量是不是能够再提高一点。六十万禁军都在盼着能够领取新装备,在情在理,都不能让他们一直空等下去。千斤炮,一万门,可就是一千万石了。”

千斤炮,一万门。这把太后都惊住了。

不过在韩冈看来,虽说一万门这个数字稍稍夸张了一点,但海船上,千斤以上的火炮没有二三十门,还填不满一艘新进入役的巡洋舰。而大宋水师,现在只会嫌船少。

 
 “而且第一艘使用钢铁龙骨的海船,已经在江宁船场制造完成。这同样需要大量的钢铁。民间的锅铲刀具,还有各色农具,也都少不了钢铁。”韩冈细细的给向太
后分析,“此外铁路如食铁兽,每铺设一里,耗用的铁料都是以千石来计算。如今申请修筑铁路的州县日渐增多,即使如今的钢铁产量再增加五倍、十倍,也还是会
入不敷出。细细算来,现今钢铁的产量,还远远不足以满足日后的发展。”

向太后稍作沉吟,“现在有多少家要修支线铁路了?”

“四京的每个县都有人申请修路,所有铁路干线所经过的州府,都至少有一个县申请修路。若计算里程,总长度已经数倍于现有的干线铁路。”

一旦支线铁路开建,所需钢铁的数量就是个天文数字。若是钢铁产量不增加的话,修筑铁路的成本将会大幅上涨。这是所有准备修筑铁路的富贵人家的噩梦。若是修路者因此在铁轨上短斤少两,日后更是难免事故频繁,平白给人口实。

有关支线铁路的一干琐碎事,也不用劳烦韩冈,不过想要筹办支线铁路,就必须过韩冈这一关。相对的,铁路本身也影响着韩冈的声望。

一直以来,为了支线铁路而奔走的灵寿韩家,他家里已经定好了路线,整理好了沿途的土地,连枕木、煤渣、石块都准备好了,只等朝廷准许开始兴修轨道。

 
 韩冈之所以一直吊着胃口,一方面希望所有参与者能够沉下心去做好前期的准备工作,比如路线勘探、资本筹集之类的事,而不是一时脑热,另一方面,也是希望
让更多人看到铁路的好处,交换利益时能居于优势。最后一点,也是希望有时间多培养一些人才出来,免得那些只看到钱的外行人将铁路修得一塌糊涂。

因为韩绛的缘故,灵寿韩家是韩冈最有力的支持者,而韩冈也需要灵寿韩家的支持。世家大族的利益远高过普通庶民,灵寿韩家在朝堂中的影响力,比起几十万的平民百姓都要强得多。

包括灵寿韩家在内,宗室、勋旧,等京中豪门,没有哪个不对铁路感兴趣。不仅仅是铁路能赚钱,铁路带动的地产也同样赚钱。开封、洛阳等地车站周围的繁华,多少人都看在眼里。韩冈在朝堂中地位日渐稳固,也是跟他如同财神一般普施恩惠有关。

要是钢铁价格大幅上扬,那还不都要闹起来?韩冈也会损失一干得力的盟友。而且铁路修造的成本降低,对铁路的发展也是有着立竿见影的效果。

“依相公来看,支线铁路是不是该修了?”

“这五年,京泗、京洛、京保等铁路相继通车,代蒲铁路也通车在即,国家财计由此日渐丰裕。”“但铁路轨道,一路仅只一条,多少县城都不能得享其利,朝廷一时无力修造,民间若能代朝廷修成,商贸大兴,朝廷可坐享其利。”

“修路开支不小,若是民间筹款,有几家能修起来的?”

韩冈立刻道:“臣请陛下允许各地成立铁路商社,由合股经营铁路。”

海外行商,有财力直接造船买卖的人很少,能包下一条船来运货的人,数量也不多,大多数是三五人、七八人的货物,共用一条船。

 
 这样有两个办法,一个是买舱。海船上,包下一个舱或几个舱,用来装自己的货,但这样蕴含了巨大的风险,也给了船上的水手们上下其手的机会。船行海上,船
只多多少少都会进些水,有的是从缝隙中渗进来的,也有的是船帮上有了缺口,但每个仓都相互隔离,其中一个进水,而其他的舱室却不一定会进水。万一运气不
好,或是有人做手脚,自己的舱室浸了水,而其他人则安然无恙,那即便船顺利返回,货主也照样要破产。

所以另外有个好办法,就是募股。多人合资包下一条海船,并采购一船的货物,各自按出钱比例拥有相应股份。赚了,按照比例分配,要是亏了,也同样按比例分配,占股越多的那就是亏得越多。

类似于此的股份制很早就出现了,这本就是各家打算去做的事,不必韩冈多费唇舌。但那终究只是合股经营,韩冈暗中所希望看到的不是股份制,而是股份制的下一步——股份的买卖现今也是有的,可还没有发展到设立有关股票买卖的专业交易所的地步。

韩冈曾经设想过,让股票市场提早出现于世——也许应该说是东方的第一家股票交易市场,他并不清楚西方的股票市场到底是什么时候出现的,或许这个时代已经出现了也说不定——然后给官宦豪门再多一个从小民身上攫利的手段。

以韩冈对他周围官宦家族的了解,一旦股市出现在这个时代,立刻就会被掌握在权势者的手中,用来骗取小民手中钱财的工具,没有什么手段能约束得了他们。

不要说股市,只要朝廷允许公开发行股票,这个问题就大了。以修筑某条铁路为名,成立一个铁路商社,上市募集资金,然后修上几里路充下门面,接着就干脆了当的让商社倒闭,募集来的股金自然就落到控制商社的世家大族的手中。

这中间,只要买通了当地的官员,最后再推一个替罪羊上来,那些鳄鱼就能很简单的将所有钱都吞下去,而不用担心任何后果。而同样的事情,他们可以做上一遍、两遍、三遍,乃至十几遍,总有贪心而又缺乏判断力的蠢鱼会上钩。

但股市运用得宜的话,也是一个集合民力,发展工业的机会,同时也是改变民间风气的好办法。

说实话,韩冈也不能确定这个炸弹丢出来,局面会变成什么样。但上千年的淤泥,不弄个炸弹炸一下,不知要到哪一年,淤积才会化开。而且有官办的铁路在这里做标杆,私家铁路变成什么样,都不会太过影响铁路的地位。

“此事相公明日可具条陈奏上,吾当细览。”

“臣遵旨。”韩冈领命,又道:“臣再请陛下召回沈括,事关铁路,此事当由他来主持。”

“嗯。就宣沈括回京。”太后道。

“陛下。”邓润甫突然出班,“臣有一言。”

“邓卿请讲。”向太后道。

 
 “铁路本是御道,其支线交由私家修筑,无前例可循,又无成例可证,不可遽然推行天下。当依青苗、免役诸法旧例,先自一二地开始试行,若无碍,再推广至天
下。”邓润甫道:“以臣之愚见,河北直面敌锋,京师最为富庶,两地一个迫切需要铁路,一个则是不用担心本钱不足,诸路中最为合适。臣请陛下允许河北、京畿
两地开始修筑支线铁路。”

“韩相公,你怎么看?”

邓润甫说的,就是韩冈准备做的。会筑路的人才就那么多,要是摊子铺得太开,如何能保证质量?而且修路中间事情不会少,试行之后也能有个解决的章程。

但他事前与邓润甫没有任何交流,章惇等人都知道铁路是他的地盘,等闲连插话都不会,邓润甫这时候站出来,不只是打算做什么?

韩冈只想了一下,就丢到了一边,道,“臣无异议,此乃老成谋国之举。”

“好吧。就先这么定下,”向太后拍板道,“等沈括上京了再计议章程。”

君臣议事良久,向太后也累了,喝了茶,换了一下姿势,疲惫不堪的问:“户口、钢铁、铁路,还有何事要说?”

韩冈犹豫了一下,一时无法决定,是到此为止,还是再说说其他方面的事。

只见一名内侍,这时候慌慌张张的过来。在太后耳边只说了两句,屏风后啪的一声响,不知是什么东西落了地,前面的小皇帝都跳了起来,

“陛下。”

几名宰辅一起惊道。

“苏相公、韩相公、诸位卿家,太皇太后……”太后斟酌了一下用词,“方才上仙了。”
