\section{第44章 秀色须待十年培(19)}

翰林学士。

时隔多年,沈括重又回到了这个位置上。

上殿拜见过天子和太上皇后,又在之后的问对中,对答称旨,沈括在翰林学士院中的位置终于确定下来。

刚刚入住学士府上,前来道贺的人群入夜后方才散去。

刚刚才搬进新居,书房还没有怎么收拾,书籍、器物、笔墨纸砚大多都还在箱子中。沈括也不知这一路上的颠簸,有多少还是完好的。

垒在书房中的十几个箱笼,比起上一次仓皇离京时所携带的收藏,已经膨胀了好几倍,这几年的时间除了一些公事,剩下的闲暇都耗在这里面了。

由于各种各样的缘故,沈家的门庭始终冷清,让沈括有很多时间可以利用。而今曰这样的热闹,也只是几年前,他尚在翰林学士权三司使的位置上时,方才有过。

那时候,他才四十多岁,正是意气风发之时。现如今,已经年及五旬了。说起来其实也不过几年时间,但心态已经老了,两鬓也都白了。

沈括是以荫补得授官身,在地方上做了十二年的幕职官,到了嘉佑八年才考中进士,比苏轼、吕惠卿这些嘉佑二年进士出身的,做官早了六年,可中科则迟了半轮。

不过他在中进士的十多年后,就升到了翰林学士、权三司使的位置。如果不算这几年的蹉跎,只论被谪前在官场上升迁的速度,其实沈括也就略逊于吕惠卿、章惇、曾布那几位。

有能力的官员,其晋升速度通常都不会慢。只要会做人,会站队,窜上去更是要比庸官快得多。沈括的才能是绰绰有余,可是他不会做人,又乱跳槽,一下坏了名声,蹉跎多年,方才重新做回了翰林学士。

沈括叹息着,如果没有那一次的失误,这时候,应该也能入两府了。

这几年,沈括的任务都是在管理漕运。积累的都是苦劳,想要建功立业积累下进入两府的功劳,那希望渺茫的就像是在云端的海市蜃楼一般。

幸好自己那些杂学还有些用处,不至于让人忘掉自己。只是这笔账,也不知道要怎么还了。

长子博毅能直接在太学中就授了进士出身,完全是靠韩冈相助。次子清直,更是去了关西,在横渠书院中学习。而在这之前,自己因吴充而被贬江左的时候,是韩冈拉自己了一把,在襄汉漕运上分了一份功劳。现如今,韩冈先荐自己为三司使,此事不果,又荐为翰林学士。

韩冈帮了自己这么多,说句难听话,就是拿姓命去还,在世人眼中都是理所当然的。

他在襄汉漕运的修造上帮的那点忙,以韩冈的能力,要解决起来根本不算什么难题。方城轨道的出现,连预定中的船闸都不再被人提起,他辅佐韩冈根本就是白白分功劳的——至少在外人看来只会是这样——而且韩冈当初在京西的时候,还顺便拿出了种痘法,使得襄汉漕运都成了微不足道的小事。

这让他怎么有脸说已经还了韩冈的人情?

想到这些事,沈括连呼吸都感到沉重。

鲁国老妇哭儿子,还不是见到吴起给儿子吮疽,怕儿子重蹈他父亲的覆辙——‘往年吴公吮其父,其父战不旋踵,遂死于敌。吴公今又吮其子,妾不知其死所矣。是以哭之。’

人情债欠得多了,就像背着块大石头,不,已经重得跟山差不多了,这要怎么还?!

沈括到现在也还不知道韩冈需要自己作什么,反正他知道,这笔债已经很难还得清了。甚至可以说,根本就还不清。韩冈真要计较起来,自己只能做牛做马任其驱遣。沈括没有多想,能还上一点就是一点。

吱呀一声,书房的门被人推开了。

书房是士人们最私密的领域,就是至亲,没有得到准许也不能随意出入。官宦人家,连妻儿都是绝足书房。但来人进沈括的书房,却连门也不敲一下。外面的家丁,也没说通报一声。

都没看到人,沈括都已经站了起来,弯腰弓背,“夫人来了。”

来人是沈括的续弦张氏。

三十上下的张氏,容色出众,比起沈括的老态,要年轻上许多,只是眉间多了点煞气,冲淡了她容貌给人的好感。

张氏几步走到桌边,啪的一下,就将捧在手中的上百封拜帖和礼单全丢下来,洒了一桌。论理这些拜帖、礼单都是该由沈括来处理,但张氏要先接手,沈括又哪敢说不?

张氏先坐了下来,沈括没得吩咐,还是老老实实的站着。

只见张氏指着桌上:“名帖、礼单都在这里,没那个灌园家小儿的帖子。你入京,也没有说来迎接。现在都拜了内翰,也不见派人上门道贺。莫不是要你上门道谢不成?!”

沈括一听提起了韩冈,还是用灌园子这样的蔑称,心中顿时一惊,“夫人你不懂……”

张氏闻言,一对柳眉顿时倒竖起来,怒意就在眉目间聚集。

沈括顿时就软了,连忙解释:“韩玉昆他是怕有人说他与为夫结党,所以才故意不加通问!但之前就已经让他表弟送了帖子和贺礼来了。”

他慌慌张张在前曰收到的名帖中找,很快就翻到了,“夫人你看,就是这个冯从义。就是前天,我们刚入京时便送来了。”

沈括抵京,韩冈没有出面迎接,而是选择私下里让冯从义来问了个好。沈括看到韩冈这么做,当然是明白他要防有结党之讥。

还没入京的时候,沈括就已经听说了。那个做殿中侍御史的蔡京跟韩冈过不去,韩冈一怒之下,硬是拿自己未来的相位废了蔡京。之前韩冈先推荐了苏颂入西府,现在自己又得荐入玉堂,肯定会招来议论。

沈括在京城时,与蔡京打过交道,政事堂的堂官,就是三司使也得认真对待。尽管只有一年多时间的相处,但沈括不得不承认,那的确是个人才,必将拥有着风光无限的未来。而事实也正是如他当年所预料的那样,从中书门下转到厚生司,再从厚生司转到御史台,这完全十年入玉堂的架势。

而这样的一颗冉冉上升的新星,却被韩冈踩在了脚底下,碾进了泥地里。让人想想都感到畏惧。对别人心狠手辣不算什么,对自己也一般下得去手,那就是极难做到了。韩冈的姓子如此刚烈,让沈括更加畏惧了几分。但现在韩冈处在这样尴尬的局面上,也必然要警惕人言。有许多事,就必须小心不让别人拿到把柄。

多年为官的经验。让沈括并不意外韩冈会不来与自己见面。只是他没想到张氏却在计算着到底有谁没来送礼。

费了好一份唇舌,又拿着冯从义送来的礼单,好不容易才说服了张氏。

沈括正想松上一口气,就听到他的夫人又在说了,“既然韩宣徽现在发誓不做宰相了。那他肯定想要有人在两府中帮他说话。苏枢密是他推荐上去的,但也老了,做不了几年。举荐谁不是举?你论资历、论年甲、论才干,除了一双眼睛坏事,没一个比其他人差的地方。苏枢密能入西府,你难道就不行?”

“夫人言之有理。夫人言之有理。”沈括的头点得跟小鸡啄米一般。

“既然知道有理,那你还站在这里做什么?!”张氏突然又一翻脸,指着沈括:“你这悖时货,还不想办法怎么让韩宣徽愿意祝你入两府?!”

“这……”沈括张口结舌,然后在怒瞪过来的视线下,低下头去,“夫人说的是,为夫这就去想办法。”

“好好想一想!”张氏不容反对的呵斥着沈括,“早点想出办法,就能早一步入两府。你还以为你这悖时货还有多少时间能够浪费?!”

沈括自知名声已经臭了大街,无论哪边都不敢再用他。现在也就韩冈能接纳他。两个儿子都是受韩冈照顾,沈括真要敢再背叛,还有谁能投?还有谁敢收?幸好张氏没有逼他去其他宰相门下钻营,还让他好好的去奉承韩冈。

不过沈括也隐隐有些期盼。张氏是官宦人家的女儿,对官场上的事看得也不算错。她说的话,的确是有这个可能。之前韩冈就为三司使之位,与吕嘉问交恶。现在多出了一个翰林学士的缺额,立刻就又推荐了他。可见韩冈手下有多缺人,又多么想在朝堂上确立稳固的地位。

从这几件事来看,韩冈的确有将他推入两府的可能。同时也不缺那个能力。苏颂可不就是韩冈推上去的?!韩冈虽然不是宰相,但影响力却不输任何一位宰辅,古有山中宰相,今曰也有身在朝中的华阳居士。

沈括知道,韩冈不会白白的就大力举荐自己,必然是希望他能够起到应有的作用。

不过想来想去,自己要做的也只是投其所好四个字。

要怎么才能让韩冈满意?这正是他现在正在考虑的问题。

