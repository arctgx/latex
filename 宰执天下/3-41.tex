\section{第15章 一笔定黜陟(上)}

大约五更天时,叶祖洽已经一觉醒来。

一番梳洗过后,来到书判厅中,正看到一名书吏在三名军士的看护下,将一卷文轴送了过来。

叶祖洽来的并不算早,这时候,上官均、陆佃等都已经到了厅中。书办将那卷文轴双手呈给众点检官中,官位最高的司农寺丞丁执礼,“各位官人,这是最后一份了。”

丁执礼低头查验着文轴外皮上的印章,见印文严丝合缝,点了点头,在书办带了的回执上签了名画了押。

弥封官解送试卷誊本的流程,基本上就是将各个考场送来的考卷誊抄好后,便混置装订,而后立刻送到点检试卷官的手中。

这些装订起来的卷子,在传送的过程中,都是卷成一卷,外面裹了封皮。封皮上面还要盖上弥封官的印章。至于原本,则是封存起来,由知贡举、弥封官、监门官三家各自贴上封条。

上百份试卷已经在点检试卷处堆积了起来,昨日就拆看过的卷子放在一边,另一边没有拆封的就等众点检官今日来拆看。刚刚送来的最后一卷,放在了最上面,只有二十多份,卷成的文轴,明显的要比其他试卷文轴小上了一圈。

叶祖洽看了眼堆在箱子中的卷卷文轴,既然是最后一卷,那么昨夜最后交卷的韩冈必然就在其中。

韩冈是今次五千贡生中,最受关注的一个。他交卷属于交得最晚的一批,这件事每一个考官都知道了。理所当然的,他的卷子只会出现在刚刚送来的文轴中。

叶祖洽正想拿来见识一下,但跟他同样心思的也有几人。上官均却是抢先一步,先将那卷试卷拿到手中。冲了几个意欲出手的同僚笑了一笑,他当即拆了封皮,将卷得紧紧的试卷展了开来。

一般来说,会在进士科考试中拖到最后的,基本上都是才疏学浅却又不甘放弃之辈,有本事的不会拖到更鼓敲响,而自知之明的,也会在随便写了一通后,就缴卷出门。

上官均只看最前面的墨义帖经的答案,连连摇头,都是不成样子。虽然不比昨天看到的几张卷子敷衍塞责,但一句简简单单的‘习习谷风,以阴以雨。’竟然写了上千字的答案,不仅是这一条,其他二十九条经文,给出的回答都是长篇大论,却又不知所云。

‘怎么不去学学‘曰若稽古’去?’上官均冷笑不已。都是没有熟读《十三经注疏》,到了考场上只能随口胡诌,写得长了,自然要多花上许多时间。

上官均一连看了十来份,差不多都是如此情况。翻看了一阵,卷子被翻得哗哗作响,终于看到能过得去的一份。每一条回答都是严格按照《注疏》而来,让上官均也不由自主的点起了头来。

“这一份不错,竟然有二十八条中格。”

所谓的中格,就是关键字一个字都不能差,省的、多的,都只能是‘之乎者也’之类的语助词。在三十条问题内,能中上二十八条,在上官均昨日看过的试卷中,也可以算得上是十里挑一了。

其他点检官也不做自己的事,都看着上官均的动作,见到他连着摇头,看得又快,知道那些试卷想必都是不堪入目。等到上官均终于点起头来,便互相看了一眼,不知道那是不是韩冈的卷子。

接下来,上官均像是转了运,翻过两份之后,竟然又发现了一份试卷有多条中格。

只是他的头没点多久,却又一下皱起眉头,自言自语:“这条不合注疏啊!”

不同于前面的一份让他满意的试卷,错误的两条,只是漏字缺字。而这一份不中格的回答,完全是自出己见,与《十三经注疏》全然有别。

‘易与天地准’,是《易经》中极关键的一句话,也是正常考生都能回答的出来的题目。前面被上官均摇头否决的试卷中,正确回答的也有大半。但偏偏这一张卷子给出的答案,却离经叛道。

不过,这答案也不是前面看到的卷子那般,全然是胡乱写来,尽是赘言废语。

“气聚则离明得施而有形,气不聚则离明不得施而无形。”念着回答中的两句,已经可以看到在其背后,有着一个完整的体系。上官均一抖试卷,亮给众官:“这是谁家的说法?!”

“这是张横渠的释义。气为万物之本原,不是他还会是谁?!”叶祖洽不愧是状元之才,立刻就给出了答案。又盯着试卷看了两眼,当即发现了另外一处与注疏不同的回答,“至于‘八则’之治都鄙,‘八统’之驭万民,‘九两’之系邦国者……”他有些犹疑,“好像在哪里听过,却又一时想不起来。”

“这是介甫相公解周礼的一段,在相公的《淮南杂说》中有此一节。”陆佃这时开口。

‘大史掌建邦之六典,以逆邦国之治,掌法以逆官府之治,掌则以逆都鄙之治’,汉末郑玄加上唐初孔颖达的注和疏,已经通行了几百年,但王安石对此却又不同的解释。陆佃是王安石的弟子,当然知至甚深。

得到了叶祖洽和陆佃的回答,上官均将试卷一抖,不用再看了。

静默也随之降临于厅中,视线在空中交错,众官沉默的交换起了眼神。

不同于郑玄的注,也不同于孔颖达的疏,这几题的答案,与如今国子监作为标准教材的《十三经注疏》全然不同,而是来自于张载和王安石。若只是采用其中一家之言,考生身份的可能性还会有很多,但同时出现在一张卷子上,那就不可能会是他人。

不过都没有说破。若是说破了,不论取中,还是舍弃,都会引来外界一场风暴。一旦传扬出去,不是得罪士林清议,就是得罪韩冈,以及他背后那一座座巍峨如五岳的靠山。

他们凭着自己进士高选的身份,可以小觑韩冈,鄙薄韩冈。但又几个愿意毫无缘故的去招惹韩冈这样功绩卓著,同时背景深厚的人物?

来自于南京国子监的教授莫京,此时却皱了皱眉:“真宗景德二年,李迪、贾边有名于京中。举进士,迪以赋落韵,边以〈当仁不让于师论〉以‘师'为‘众',与注疏有别,皆被黜落。时王文正【王旦,真宗朝名相】为参政,道:‘迪虽犯不考,然出于不意,其过可略。边特立异说,将令后生务为穿凿,渐不可长。‘故而收李迪,而将贾边黜落。考于故事,还是将此卷黜落为宜。”

莫京以过去的先例为证据,提议将韩冈的卷子给黜落掉。可是没人理睬他,

——时代已经变了!

王旦那是真宗初年,经过唐末五代之变,儒学尚未复兴,当然要以汉唐注疏为宗。但如今各家学派并起,通行已久的《十三经注疏》早就给批成了筛子。而且朝中还有传言,说很快就要设立经义局,重新确立官方性质的经典注疏。

莫京咬起了牙,坚持自己的意见:“无论如何,这份卷子的答案既然不合注疏,就不能判对。若是这一份能放过,其他黜落的卷子又该怎么办?难道也一样放过吗?!既然事前已经规定好按着《注疏》来,就不能随意改变。就算到天子面前,我也是这个说法!”

莫京发狠,众官面面相觑。在这一帮点检官中,只有莫京是个异类,其他人无不是偏近于新党,谁也不想得罪王安石的乘龙快婿。但他已经说要闹到天子面前,就证明他肯定会将此事对外曝光。就算现在能将他的意见压下去,外面士林间的舆论又有谁能压得下?!

看着莫京怒发冲冠的样子……叶祖洽等人都头疼了起来。

叶祖洽从来不愿与人争短长,更不愿随意的罪人,他提议着,“不如暂且搁置,传到考试和覆考那里问一问他们的意见。”

点检、考试、覆考,是试卷三道关口。三判皆下等,便是黜落无疑,不会送到主考官面前。但若是三份评判争执不下,判卷便会呈给曾布、吕惠卿等四个主考,让他们确定结果。叶祖洽相信考试官和覆考官中,肯定有一方会给出一个与莫京不同的判断。那时就可以将卷子送到主考手中,与自己再无瓜葛。

“那要我辈何用?!”莫京冷笑着,让叶祖洽下不了台来。

一时间,都没了辙。而莫京,则仰起了头,再也看不起这一干人:‘此辈皆庸浅,为己秉正道而不移。’

上官均却在低头看着卷子,仔细的审核每一题的答案。三十题审过,又发现了一处释义有别的地方,但他已经没有心思去根究答案来自于王学还是关学,一股寒意传遍心中。

将卷子展给众人,上官均沉着脸道:“按照今次的法度。墨义帖经一部,三十题若有二十七中,便算合格。而这份卷子……正是二十七题中格,只有三题尚待商榷!”

正沉浸在一股‘虽千万人,吾往矣’的悲壮感中的莫京,一下目瞪口呆,而叶祖洽、陆佃等人也是心中寒意大起。

‘这就是韩冈的计算吗?!’

韩冈既然能把其他二十七题全数照着注疏做对,另外三题又怎么可能只明了王学和关学的释义?!这分明是他故意在表露自己的身份。

但韩冈竟然敢于行险,故意写上错误的答案,只要错上一点,就是黜落的结果。可见其人胆量之大,心思之深,而且更是对自己才学的自信——自信其他二十七题,没有一条会错。

厅中重新陷入了沉寂,韩冈的胆魄手段心术让人畏惧,这样的情况下谁敢在这里将他的卷子给黜落?

而以他在墨义上的评分,谁又还能轻易将之黜落?!

