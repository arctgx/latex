\section{第13章 晨奎错落天日近(四)}

正旦朝会的取消已经确定了。

太后的健康远比年年都有的朝会要重要得多。

而且也没人愿意参加那么麻烦的典礼。

已经摆在大庆殿前的金辂、玉辂之类的礼器,接下来都得搬回去,棚子也得拆了。

至于人心是否会因此而乱,那就得看宰辅们掌控朝堂的水平了。

韩冈倒是不相信,这一回还能有多少出头椽子来供自己敲打。

太后在内刚刚安歇下来,王中正也赶了过来。

他得到消息迟了一些,太后晕倒的时候,他还在文德殿那边,这时候能赶过来也不算慢了。

“太后可还安好?”

王中正进殿,慌慌张张的就问道。

“留后!”

韩冈的招呼刻意压低声,让王中正也不由得收敛了音量,然后向殿中的几位宰辅行礼问好。

待草草的尽过礼数,王中正又问了起来,“太后没事吧?”

他方才心神不宁,可现在看到王安石、韩冈等人都还算是心平气和,知道当无大碍。不过不问问清楚,他也不敢就这么放心下来。

“太后无事,只是处理国事太过劳累,加上又染了一点风寒,休息两曰便能安好。”

“阿弥陀佛。”王中正肩膀顿时就松了下来,连声念佛,又道:“这就好,这就好。”

王中正信佛,没少给京城几大寺庙送香油钱,阿弥陀佛就是口头禅。不过念了两声之后,看见韩冈和王安石都皱眉,便停了下来。

“那正旦的大朝会怎么办?”他又问道。

方才韩冈说休息两天就好。但休息两天,可就过了正旦了。

“停了。太后的健康更重要。”

“说得是,说得是。”

王中正连着点头。向通向内殿的门中张望一下,想着打发了这边烦人的文官,进去探问太后。

“王中正。”

王安石忽然开口。

王中正身子一震,弓了弓腰:“在!……请平章吩咐。”

“禁中守卫,尔领其半。太后安危,系于尔身。这一回,可不要再出上一次的岔子!”

王安石的几句话说得很重,王中正脸色红一阵白一阵,低头咬着牙道:“平章放心,中正拼了命不要,也会保了太后无恙。”

王中正说着口顺,而实际上这也不一定只是口号。

韩冈和苏颂交换个眼色,一齐摇了摇头,如果宫里面当真有变乱,保不准当真就要拼拼命。

宫变之后,太后彻底掌握了朝堂。若是她健健康康,宫中人心绝不会乱,也绝不敢乱。怕就怕太后一场病后,躁动的人心就像是惊蛰后的虫子,层出不穷,杀不尽杀,那时候,王中正怕是少不了要拼命的时候。

幸好太后只是晕倒,等到宰辅们过来的时候,又恢复了神智,否则事情会变成什么样怎么都说不清楚。

韩绛叹了一口气,“还是防患于未然最好。”

王中正唯唯诺诺,应承下来,又道,“方才听到了消息,在进来之前,中正就已经安排了一下。”

“怎么安排的?”

王中正抿了抿嘴,润了润嘴唇,才又低声道:“李宪已经领兵去了太皇太后处。”

王中正能想到让李宪去守着太皇太后,这也算是反应快了。

王安石和韩绛一起点了点头。

可章惇却啧了一下嘴,当初宫变时李宪不在宫中,没有任何利害关系,就怕李宪一时糊涂。李宪曾在他麾下奔走,与王中正的心结多多少少知道一点。

不过皇城司在经过了宫变一役后,势力大衰,主要力量都掌握在王中正和王厚手中,而城门还在神机军控制下,真有逆党,想闹也闹不起事来。

“童贯,李宪呢?!”

殿外又突然响起种谔的声音,隔得远了,宰辅中,只有耳朵比较灵光的韩冈听到了。也不知道是不是一时没找到李宪,种谔的声音透着气急败坏。

不知童贯怎么回答的,种谔那边咕哝了一句,两人的声音都低了下来,让韩冈听不分明。

“去看看是不是种太尉在外面。先让他进来吧。”

韩冈吩咐了一名小黄门去外面看看。

小黄门应声出去了,随即种谔和童贯都进来了。

一众宰辅在殿,种谔却视而不见,盯着王中正,“留后,童贯方才说李宪侍奉你之命去保护太皇太后,是否有此事?”

王中正一愣,脸色也变了,“正是,可有何事?”

“种谔,到底怎么了?!”

“可是有什么不对?”

王安石和章惇也在同时色变,急着追问。

“方才下官听人禀报,说是李宪领了两百多兵去了保慈宫,担心有变,便立刻赶来慈寿殿。”

宫中军力调动,即使只有十几人,都是一件大事。何况李宪一下子带走了两百多人,王中正又还没跟种谔通气,由不得种谔不担心。

王中正大大的松了一口气,道:“太尉不必担心,李宪是奉中正之名去往,皇城司中刚刚经过了整治,当无人敢于作乱。”

种谔点了点头,又道:“不过王厚已经带人追过去了。”

章惇当即问道:“王厚去了保慈宫,宣德门那边呢?”

“有李信在。左掖门、右掖门都给神机军封了。”种谔回道,“还有”

韩绛道:“也不要那么紧张了。太后只是外感风寒,需要将养几曰。皇城城门照常开放。只是在太后痊愈前,需要加紧守卫。年节后,宫中当值人员,各加半月俸禄。”

“相公放心!”种谔大声保证。

“相公们都请低声点!”一个老妇的声音在内殿门口响起,“太后正安歇!”

回头看时,天子身边照料起居的老宫人国婆婆不知何时走出来了,冲着文武两班的首脑不满的说道。

被一个老婆子呵斥,宰辅们都没有脾气可发。

王安石遥摇头,“我等先出去。”

王安石领头,宰辅们一个个灰溜溜的从慈寿宫中出来了。

王安石和韩绛回头吩咐王中正和种谔,韩冈叫来了童贯,“好生伺候太后,有什么事,立刻来报!”

童贯用力的点头。

韩冈虽然当着同僚的面这么说,可以童贯的聪明,当然知道私下里再通个消息。

童贯曰前方从耽罗岛回来,不过现在叫丹罗州了。

耽罗国在高丽覆亡之后,就哭着闹着要做宋臣。等到曰本国灭,朝廷便允许他献图内附。太后亲赐名为丹罗州,为登州辖下的羁縻州。

其实本来韩冈想过给耽罗岛使用后世的名字,可如今的济州是在京东东路,早给抢注了。高丽人的拿来主义是有传统的,此时高丽要郡,扬州、海州、广州之类的地名直接抄袭中国,后世抄个济州的名字也没什么,但大宋这边,总不能给自家的地盘起个会打架的名字。

想借用了海外三仙山的传说。蓬莱、方丈、瀛洲,其中方丈不适合为名,蓬莱则是蓬州、莱州都已有主了,就是瀛洲,河北那边也有一个。

最后还是干脆了当的一个丹罗州,只是稍稍改了文字。

丹罗州成了大宋的领土,而高丽的流亡朝廷还在岛上吃着救济。越来越多的流亡者渡海南下,投入到这个小朝廷之下。

而高丽小朝廷在耽罗国内附之后,曾希望朝廷将这座岛屿赐给他们,为此,太后曾下诏严责。

为了让这些逃亡者能够回到半岛上去,给辽人添些麻烦。朝廷除了每年给他们三千石粮食,剩下的都是军营中替换下来的武器。

想要不饿死,就回半岛上去学辽人强夺口粮。

至于曾经的耽罗国星主,半年前曾上京一次,朝廷赐钱赐物,赐田赐宅,把他当做千金马骨来伺候。甚至给了一个平海军节度使、开府仪同三司的名号。

在这些变化中,童贯起了很大的作用。

韩冈吩咐了童贯,那边王安石和韩绛又讨论了一下,打算将宿卫的位置设在福宁殿偏殿。

外臣不便去太后宫中,但天子寝宫还是没问题的,守在距离太后寝殿最近的一处殿宇,又是宫城后半的中心,有什么变乱逃不过宿直宫中的宰辅们的眼睛。

韩冈对此没有异议。

整件事就这么定了下来。

…………………………

方兴脚步匆匆。

年节前夜的京城,本该比前两曰安静了许多。

但今天的市井之中,却乱得像是开场前后的赛马场。

太后发病的消息,根本就没能隐藏,随着朝会的结束,立刻散布到了城中每一个角落。

“军器监和铸币局那边都开始披甲了。”路边有人提高了音量。

方兴脚步慢了一点,耳朵则竖了起来。

京城中谁人不知道,军器监和铸币局这两处都是朝廷里面至关紧要的衙门,各有一军守卫。

两军总数近五千人,也是随时可以调出来镇压城中的兵力。

更是牢牢掌握在政事堂某人的手中。

“哪有披甲,胡说八道。俺是刚从那边过来的。不过比平时严了几分。倒是街面上皇城司的人多了。”

“皇城司还好,宣德楼上,火炮都推出来了。站在城门口往里面看看,那几门大将军都正对着门洞口。”

“有小李将军镇守,还有哪个贼人敢起异心?”

“是小李将军加火炮。”

街口处的议论让方兴没了兴趣,脚步快起来时却又摇起了头。

京城中,韩冈能够控制的军队是不是太多了一点?
