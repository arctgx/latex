\section{第31章 九重自是进退地(六)}

韩冈得到王家十三郎被天子送回的消息的时候,已经是第二天的傍晚。

按照王家派来传话的家丁的说法,天子亲自下旨,让中官送了十三郎回家,而且还有诸多赏赐。王家家丁转述中官的话:是官家和宫里面的圣人和几位娘子所赠——圣人和娘子是宫里的宦官宫女对皇后、嫔妃的称呼——言语间充满了对小主人的自豪。

其实在这之前,早在接到宫中之命的时候,苏颂就已经遣人通知过了王韶家里和韩冈了。知道是宫中的御药院副都知救了王寀,除了感叹王韶第十三个儿子的运气,也就是赞叹他的临机应变之材。

不过到了这时候,韩冈才原原本本的知道了真相。对王寀在危急关头甚至还不忘留个证据,以便用来捉拿贼人归案,之前的赞叹也变成了对王家小十三才智和夙慧的惊叹。

王旖也是满心的惊讶:“知道十三哥打小儿就聪明,没想到竟然聪明到这个地步。”她有些敢,“要是家里面的几个孩儿都能有个七八成就好了。”

韩冈的几个儿女中,并没有出现资质卓异的天才,像白居易一样六个月能识之无,或是今日的王寀的水平,韩冈都不指望的。不过就是韩冈自己,也没脸说他小时候能跟王寀差不多水平。

“王家的小十三可比为夫当年要聪明多了。”韩冈呵呵笑道,“子纯枢密一直都盼着家里能再出一个进士,这下终于不用担心了。十几二十年后,保准就是个进士及第。”

连同王寀被拐一案同时告破的,还有英国公家真阳县主受辱的那个案子,就是同一个团伙作的案。当年这个团伙将真阳县主拐走,强辱之后又将她卖给了一个想要纳妾的富户。而误买下了英国公之女的那个富户,得知真阳县主真实身份,不敢轻辱,而是悄悄地将她送了回家去。

为了女儿名节着想,英国公府并没有将此事宣扬出来,不过京城之中从来都是只有谣言、没有秘密,转眼就流传开来。

这个让宗室脸上无光的这个案子,时隔多年,如今终于告破。从律法上说,被擒获的罪囚基本上可以去为自己找两个和尚来超度忏经了——犯到了宗室头上,即便是遇上了大赦,都别想能逃过一劫。而将他们送进刑场的却是一个五六岁的孩童,这一奇闻,顿时轰动了整个东京城。

“听说已经有好几家准备向王家的十三哥儿提亲了。”过了两天,韩冈就听到刚刚从王家回来的王旖说着才听到的新闻。

“未来的进士要先定下来嘛。”韩冈笑笑。

韩冈他终于发现自己预感王寀不会有事,并不是直觉有多出色,也不是像王韶那样,对自己的儿子深有了解,而是旧时残留下来的一点印象,如同河底淤泥一般从记忆的深处泛起。这个支离破碎的回忆,当日只是让韩冈隐隐约约的有个感觉,而现在才全数呈现在眼前。

韩冈当然是没想到王家的十三哥儿,竟然就是自己曾经耳熟能详的那一个故事的主角。而那一个故事,就发生在自己的眼前。

不过韩冈倒是没有太过惊讶,他见过的历史名人太多了,王安石、司马光、苏轼这些现实存在的人物不说,就是传说中八仙里的何、曹二人,韩冈都是打过照面,说过话的。

何仙姑现在还在荆湖南路的永州给人算命,断人休咎,章惇和李信都找她判过命数,不过她已经成名快五十年了,听说是个干瘪没牙的老婆子,韩冈就没了多少兴趣。不过他从京城到广西,又从广西回京城,来回两趟,总共四次经过永州,闲来无事,韩冈也就在这一次的回程时,抽空见了她一面。

隐了自己身份,穿了个普普通通的儒士襕衫,韩冈去问了一问家人和前程。得到的回答是含含糊糊的江湖术士口吻,前途是光明的,道路是曲折的,如果能恪守正道,当可如履平地云云,说的一番话基本上怎么解释都可以凑得上,但也不能说错,只是要骗韩冈这个老江湖却是不够了。

至于曹国舅,人更是好找。想想现在的太皇太后姓什么?曹太皇的两个弟弟,韩冈都见过,好道的那一个是老大曹佾。前些日子的正旦大朝会,韩冈还见过他。挺富态的一个人,行事很低调,有个开府仪同三司和观察使的虚衔,朝会上站在很前面,修炼内丹的事的确有传闻,只是没听说他见过吕纯阳。

也就是因为曹佾好道,所以韩冈得以跟他在大朝会后聊了几句——如果说眼下朝中有谁跟传说中的神仙关系最密切,不是别人,正是他韩冈。只不过韩冈依然是绝口不认自己药王弟子的身份,让曹佾失望而归。

而后世的故事中,被包拯用虎头铡砍掉的曹家老二,则是老老实实的紧守门户,没听说过有什么劣迹——基本上此时的外戚一个比一个老实,一方面宫里面管束得严,曹太皇、高太后从不为自家人要官要钱,另一方面,士大夫们一个个如狼似虎,外戚、内宦敢有蹦跶的,立马一棒子打死。也不管什么新党旧党,一旦遇上阶级敌人,立刻就会联合起来群起而攻之。

因为儿子在宫中住了一夜,也是被宫里面的人照顾了一夜,王韶特地进宫一趟,向天子表示感谢。

王韶为了王寀的平安归来欣喜无比,还因此设宴。不过哪边都怕奖誉过度,小儿容易夭折,故而并没有大事操办的庆贺,不过也请了韩冈和几个相熟的朋友,好好的吃喝了一顿。

……………………

经过了王寀一事,赵顼因为子嗣艰难的坏心情也因而好了不少。

而朝堂上的平静也没有因为上元节的结束而宣告终止,虽然眼下的和平时光,看起来只是暴风雨前宁静,但对于早就习惯于臣子们拿着弹章互相攻击的当今天子,只会享受眼下的耳根清净,而不会去奢望这样的安稳能够保持很久。

经过一段不算短的时间,赵顼终于想起了韩冈——其实也是韩冈排队排到了点,从阁门的排班顺序上,也该他入觐了。如果可以将韩冈跳过去,那就不是冷遇,而是处罚了。无过而罚,怎么都说不过去。

收到了宫中的传召,确定了两天后进宫入对,韩冈在新开的棉行楼上笑道:“总算能出京了。”

章惇没有笑,他今天请韩冈可不是说的这件事,“听说玉昆你与沈括有些交情?”

韩冈不知道章惇为何突然提到当今的三司使兼翰林学士——其实这几年,沈括的官阶、差遣一直被韩冈压着,尤其一年前平交之战结束,韩冈稳稳的压过沈括一头去,但现在却反是他压了自己一头上来,这就是年龄、资历达标后的结果,一旦开始晋升,根本就没什么阻碍——指名道姓的说话,看起来章惇对沈括有些看法。

“不知沈存中犯了何事?”韩冈问道。

“他给吴相公上书……”

章惇的话才起了头,韩冈就立刻惊讶的打断:“给吴相公上书?!”

“对,就是给吴相公上书。”章惇冷淡的声音着重强调了最后的五个字。

“当真是糊涂!”韩冈立刻说道。

翰林学士也好,三司使也好,都不是中书属吏,怎么给吴充上书,要上书也该给天子才对。按后世的说法,这是犯了组织错误。

“他那里是糊涂,再聪明不过了。”章惇冷笑,“沈括他说两浙役法不利于民,当加以更易。吴相公今天就拿着他的话,开始找免役法的不是了……”狠狠的哼了一声,章惇突然就变得声色俱厉起来,拍着桌子上的碗碟叮当响,“两浙路上实行新法,可是他亲口说的没有问题!”

韩冈皱起眉来,听章惇这么一说,他也觉得沈括还真是不会做人。

韩冈也知道,为了确定农田水利法是否能见成效,当年王安石曾经让沈括去相度兩浙水利、并察访两浙新法推行情况。而后两浙推行农田水利法和免役法,天子问王安石:‘此事必可行否?’王安石是拿着沈括的报告来作为证据,说‘括乃土人,习知其利害,性亦谨密,宜不敢轻举。’

但现在王安石下台,眼见着天子有着偏向旧党的打算,沈括就跳出来说免役法有问题。说句难听话,这就叫做见风使舵,做事未免太不地道了。

“介甫相公曾说沈括是壬人。”章惇板着脸,一点也不客气,“放到今天看来,当真没有错!”

壬人就是奸人,韩冈也知道王安石不待见沈括,虽然并不否认沈括的才能,但前两年当赵顼准备重用沈括的时候,就是在王安石那一关给挡了。沈括升任翰林学士兼权三司使,还是在王安石辞相之后。

“沈存中性子软,也许是当初有问题不敢说。”韩冈设法帮着沈括解释。他与沈括也有几分交情,关系不恶。

章惇不答话,两只眼睛盯着韩冈,嘴唇抿着,动都不动。包厢中一片寂静,过了片刻,韩冈摇头一笑,在官场上,看的只是结果,性格也好、动机也好,从来是没人关心的。

“打算怎么处置沈存中?”韩冈问道。

“自有御史台对付他!”

