\section{第35章 重峦千障望余雪(五)}

“力学原理?!”吕大临听说过韩冈欲以旁艺近大道的宏愿,但一直没有放在心上,追求大道,当行正途,旁门是他所不屑一顾的。

“是很有趣的说法。”张载却有着博采众家的气度,对韩冈的想法也十分支持。

他把一杆毛笔平放在桌面上,“一支笔,如果放在桌上,没人碰它就不会动的……”他手指一推,笔杆就咕噜咕噜的滚出去,“一旦有了推力,笔杆才会动起来。世间万物不受力,都不会动。必须有力加诸于上,才会运动。”

吕大临奇道:“这有什么好说的?天天都能看到。”

“道理的确很浅显。但玉昆又问了一个问题,”张载拿着笔,在吕大临疑惑的目光中,松开手,毛笔啪嗒一声掉在了地上。“为什么笔会往下落,这力是从何而来?”

“下面没有东西托着。”吕大临说了一句,觉得哪里有些不对,“韩玉昆怎么说?”

“玉昆的信中说,大地对万物皆有引力,无处不在,无可阻碍。毛笔落,皆是因为有力向下拉着。”张载翻了翻桌上,把韩冈的信抽了出来,厚如一卷书,展开来有十几页之多,吕大临一看,上面甚至还有图案。韩冈竟然是用图案、数字加文字,一点点说明了自己的观点。

吕大临看了两眼,便皱起眉来,上面的点点画画让他看了头痛,“韩玉昆这不是走火入魔了吧?”

“还是仔细看看为好。……玉昆的信中说要从中格出日升月落之理。”

“怎么可能?!天地大道,岂能与笔杆等同?!”

“日升月落,天道也。但其中必有理可循,未必与笔杆不同。玉昆说要寻出其中道理,也不是不可能。”

听见老师这么说了,吕大临又皱着眉头看起来韩冈的来信。

张载起身支起窗子,一阵寒风吹散了房内的暖意,但也把浑浊的空气给替换。

张载深呼吸一口清凉的空气。他自从辞官回到横渠镇后,创立了期盼已久的书院,亲眼看着门下的学生日渐成才,而自家的学术也逐渐形成体系。

横渠先生盼望着韩冈能够成功,他那位年轻出色的弟子,其格物致知的想法当是来自程颢,但用数算解析自然大道,必是韩冈自出机杼。如果能有所得,当能补全气学学术论述中的许多缺憾。

上承圣教道统,下开万世太平,天地、生民皆入心中。

这便是张载的愿望。

……………………

河湟熙宁四年的腊月,交替在风雪和晴天之中。

前两天的一场暴雪将熙州【武胜军】和巩州【通远军】的联络给中断,压垮了城里城外的上百间屋舍,但到了今天,天上又是晴空万里,白雪皑皑的山头上反射着夺目的阳光。

韩府的大门前,韩云娘呵着手,暖暖的白雾从指缝中散逸出来。韩云娘过了年就虚十六了,完全长开的身子,看着还是偏着纤弱。披着猩红的斗篷,一整条狐皮围脖绕在颈中。扬起的小脸冻得通红,挺翘的鼻尖都是红红的。

地处边城,陇西城中的大户宅院,无不是高墙围起,韩家也不例外,连大门都是高约近丈。一个韩家的仆役,正要在两扇门扉处挂上刻着神荼、郁垒二门神的桃符,掂着脚都够不着位置,只能踩着一张方凳上,挂着桃符,还要回头问着下面在看的韩云娘:

“小云娘子,你看正了没有?”

“偏了一点,再往左来一点。”

再有几天就过年了,韩家现在是巩州排得上前三的头面人家,操办起年事来,也是热闹非凡。要祭祖、要开席,人多嘴杂,场面本有些乱,但有了韩阿李出来指派,倒也没有落下什么笑话。

韩冈无视着外面的喧闹,在书房中,专心致志于书本之上。

昨日雪停后,他就带人在城里城外走了一圈,在联络不上在熙州的王韶的时候,自作主张打开府库,拿出钱粮,招募灾民出来务工。以工代赈,清理城中街巷上的积雪。

韩冈已是通判,他下了命令,自然就有人去处理,并不再需要他亲历亲为。以工代赈的差事,他也是交托了出去,只要每天抽空去看看下面的管理有没有把事情安排好就行了。

不管怎么说,韩冈作为一任亲民官,他并不想看到在他治下,有平民死于冻饿之中。那些鳏寡孤独的无丁户,韩冈也跟王韶通气后,将他们收拢进疗养院,做些不费力气的杂活,也能有口饭吃。

凡事预先安排,将各项事务分派给合适的手下去完成。让普通官员觉得繁琐无比的工作,韩冈做起来是,却是清闲无比。有空坐在家中书房里,安安静静的读着书。

明年就是熙宁五年,地方的解试在八月的时候就要开始了。论时间,他并没有多少可以浪费的——对木征的决战,在开春后正等着他,眼下能坐下来系统的读一读书的时候,也就过年前后的这么一段时间。

到了朝官这个阶段,进士出身的官员,不会再像选人和京官的时候,能一次两级的跃迁。但缺少一个进士及第,升到一定程度,就会撞上一块透明天花板。无出身的官员即便再有才能,在与进士官员交流的过程中,都少不了被冷嘲热讽。最明显的例子就是朝中在财计方面首屈一指的薛向,他几次在陕西这样的要地任职转运使,但王安石提拔他担任六路发运使,主管汴河纲运的时候,便是一摞弹章压上来。至于其他例子,韩冈倒是一时想不出——非进士的文官,再没几个能如薛向一般升上来。

为了日后的顺利发展,韩冈他需要一个进士的身份。军功不足为凭。狄青当年都说过,他于韩琦的差距,不过少一个进士及第罢了。但两人的结局,却是天差地远。

还有八个月就要去考贡生,中间又有一场大战要分去大半时间,对韩冈来说,可谓是时不我待。

不过他拥有的官身,算是个走后门的钥匙。

作为官员,韩冈不能参加军州中的解试,而是要去所在路分转运司的治所,参加专门由官员参加的锁厅试。名义上是防止官员抢夺贫士的贡生名额,可实质上,却是让那些有着荫补官身的世家子弟,能够方便的通过解试。而韩冈就占了这个便宜,而且便宜不止一桩。

如果在一年前,陕西转运司还没有分割的时候,韩冈肯定要去长安京兆府参加锁厅试,与陕西各地的官员竞争。虽说是十中选二、选三的机率,比起福建、江西那样的三四百中挑一个的解试要容易许多,但毕竟不如陕西转运司一分为二的现在——今科预备参加秦凤转运司锁厅试的官员,即便算上韩冈,也不知会有三人还是五人。

如此之低的竞争率,加之秦凤一带低劣的学术水准,想要在他们中间脱颖而出,对韩冈的经义水平来说,当真不是什么难事。而且主持锁厅试的是转运使。在河湟大战前后,为了保证秦凤局面的安定,朝廷不到逼不得已,不会走马换帅,如今的转运使蔡延庆当不至于会给自己下绊子。

只是到了礼部试的时候,就没有那么多便利了,韩冈也必须跟来自于其他地区的数千贡生,争夺区区三百个名额。可对于考中进士,他还是很有几分自信——毕竟这一科很特别。

“官人。”严素心端着热汤推门进来,还没走近,盖碗中的汤水已是香气扑鼻。

韩冈正是读书读得累了,便放下书。视线在盖碗和俏脸上来回转着,盘算着先吃哪一个为好。

熟练的将少女扯着坐在腿上,随手探入怀中,不知是不是自己逐日滋润的缘故,严素心原本略显纤巧、一手可握的胸房,这段时间好像变得丰腴了起来,连手感都不一样了。

只是韩冈稍稍一捏,怀中的娇躯却是猛然一震。连忙松开手,他关切的问着:“素心,怎么了?”

少女细细的叫着痛:“有些疼。”

韩冈有些纳闷,自己都没用多少力。再试探的轻轻握上去,严素心便又是抽着凉气,皱起了修长的轻眉……韩冈忽然间灵光一闪,想到了一个可能。便伸手用力一扯,一轮丰润了许多的酥胸骄傲地挺翘着,在空气中上下轻颤。

“官人!”

严素心一声惊叫,手忙脚乱扯起了被拉开的半边襟口。血一下涌了上来,脸红得跟熟透了的苹果一样,热得发烫。咬着下唇,小拳头捶了韩冈几下,嗔怪的责难着,“这是白天啊……”

虽然暴露了一下便被遮起,着力注意的韩冈还是发现那一处的颜色的确变深了一点。“素心,你这是不是有喜了?”他立刻惊喜的问道。

“有喜?”少女楞然。

见严素心茫然不知,韩冈又换了个问法:“最近你有没有感觉想吐?”

素心摇了摇头:“奴奴没有,但南娘妹妹今天早上还吐了一次,昨天的胃口也不好。”

韩冈拍拍脑门,怎么赶到一起了。他小心的扶着严素心站起身:“得找个能断喜脉的医生来看看了。”

