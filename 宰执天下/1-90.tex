\section{第37章 长安道左逢奇士(下)}

【第三更,求红票,收藏】

“路兄连续数科皆得发解入贡,才学那定是好的。但入京一次,家财可是耗用不小。”

“一箪食,一瓢饮,回也不改其乐。区区阿堵物何足挂齿?”

“若这些税吏也能如路兄这般便好了!”

被韩冈一提,路明一下愤怒起来,“晚生本想着能运点土产进京,好贴补一下盘缠。谁想到突然之间税卡就变得那么严。‘王何必曰利’,这分明就是与民争利啊!”

路明的愤怒,韩冈为之失笑。他上下打量了一下路明,从骨头里透出着穷酸破落。大宋不同明清,考上举子,也不能被称为老爷,除非能得中进士,不然便是一辈子的措大。

路明的坚持,韩冈则难以理解。他一次次重复的去京城考试,还要靠着贩运来支持。这样盲目的行动,最终什么回报都不会有。韩冈对如此无谋的行为实在难以理解。

屡考不中,实在不行可以去考特奏名,那难度比起进士试要低得多。只要考上了,便能补授文学、助教一类的学职,领着朝廷俸禄足以养家糊口。总比要抱着一个不切实际的幻想,要强得多。

别过山羊胡子,韩冈一行终于再次启程,只是三人变成四人,多了个路明出来。

韩冈和刘仲武都是驭马而行,连李小六也有匹马骑着,而路明骑的仅仅是头骡子。虽然原本的那头老骡子已经在税卡上被换了一匹健壮的大青骡,但骡子背着大捆的货物,又加上了路明的重量,走起路来仍是呼哧呼哧的一步三晃。

韩冈看了半天,心中不忍——对象当然不是路明——便说道:“路兄若是不嫌韩冈冒昧,不如就跟在下同行,等到了驿站,也可换乘了马匹,如此行程上也能快上一点。”

路明一听,当即滚下骡子,哭拜在地上:“官人大恩大德,路明粉身难报。父母生我,官人救我,官人就是路明的再生父母!”

韩冈听得寒毛根根倒竖,如此奇人当真难得一见。他赶紧跳下马,将路明扶起来,“使不得,使不得,韩某哪里当得起!”

路明又重重的磕了一个头,方才起身,抬着袖子擦着脸上不知何时挤出来的泪痕。

路明绘声绘色的表演,韩冈心中暗赞。他其实本对这位免解贡生没有什么好感,只是看到一名儒生路遇坎坷,顺手帮上一把,也是情理之事。既然是惠而不费之举,帮一下又无妨。但现在看来,路明当真是个妙人。而且在韩冈想来,他既然是免解举人。自然有过多次前往东京应举的经验。人头熟,道路熟,有他做伴,也可算是个向导。

一行重新上路,往着京兆府赶去。

一路上,路明拉着韩冈谈诗说词,费尽心力的想表现一番。只是这水平基本上是在陕西路贡生们的平均水准之下,韩冈听着有些不耐,但犹装出饶有兴致地样子。

而当韩冈把话题转往军事水利方向的时候,路明又大吹胡吹了一通瞎话,连一边的刘仲武都听得摇头。很快,路明自知肚里无货,便又把话题转回到诗词歌赋。过了一阵,不知怎么的又扯到了历年进士科举时的应试考题上去了。

“晚生第一次入京,还是三十年前的事了。那一科,有参大政的王介甫【王安石】,有做翰林的王禹玉【王珪】。都是跟晚生极好的。晚生尚记得王介甫的那句‘孺子其朋’,好好的一篇文章,就给这四个字毁了。从考场出来时,相熟的几人互相一说,都是叹息王介甫用错了词,连王介甫自己都摇头。最后也没错,一个状元就这么飞掉了。”

胡扯!韩冈半点不信路明会是身临其事。

王安石的‘孺子其朋’,是写在殿试时的考卷上。因为这是周公旦教训周成王的话——小子啊,朋党害政,尤宜禁绝(少子慎其朋党)【注1】——而看考卷的人是仁宗皇帝,他都做了几十年的皇帝了,那可能喜欢才二十多岁的年轻人拿着周公的话把自己当晚辈般教训?虽然不会黜落,但还是从第一降到了第四。

这是殿试的考题,而路明若是能进殿试,就不可能落榜。殿试定高下,省试定去留,能进殿试,进士是当定了,只是要再考一次决定名次高低罢了。路明哪有这个机会,他应该只是跟自已一样,是从别人嘴里听来的。

“晚生最遗憾的还是嘉佑二年那一科。当时是欧阳永叔主考,出的题目是《刑赏忠厚之至论》。孔子国【即孔安国】的注疏,晚生也是背过的,但在考场上一时间没有想起来。‘刑疑附轻,赏疑从重,忠厚之至’,偏偏在下把‘疑’字给漏了。”

‘这哪里叫亏?考官出的题眼都没发现,明明白白的陷阱还踩进去,’韩冈在肚子里面腹诽着。‘疑’这个字是欧阳修故意漏的,出题人就是通过这种手段来测试考生对经典的熟悉程度。但孔安国给《尚书》作的注解记不得,但原文总该背下来吧?‘罪疑唯轻,功疑唯重’不一样都有个‘疑’字!

‘罪疑唯轻,功疑唯重’是出自《尚书•大禹谟》里的一句,后面还有一句‘与其杀不辜,宁失不经’,体现了中国古代司法的仁厚宽和,跟后世通行的疑罪从无道理其实是共通的,就算是他也是滚瓜烂熟。孔安国的注疏不过是化用《尚书》中的文字,最关键的‘疑’字并没有改动,怎么能漏掉?

“真是可惜啊!”路明仰天长叹,有着需要捶胸顿足般的痛苦,“要不然一时之误,晚生便能够跟苏子瞻、曾子固【曾巩】一科出来了。那一科,欧阳永叔任主考,厌于当时太学体的钩章棘句,改崇古风,文章只以浑醇为上。浮薄之风一扫而空,拔擢了多少人才。苏子瞻,苏子由,曾子固,吕吉甫都是一时英杰。”

嘉佑二年的那一科进士,的确称得上是群星荟萃,韩冈也知道。苏氏兄弟不说,单是同为唐宋八大家的曾巩,他一家四兄弟,连同两个妹夫同时中了进士,这是大宋立国百多年里的独一份。除此之外,他的老师张载,他的举主王韶,二程之一的程颢,都是嘉佑二年的进士。另外,据说如今辅佐王安石订立变法条例、被反变法派骂成大奸大恶的吕惠卿,也是在嘉佑二年考中进士。

“嘉佑二年何其多才!”路明说得兴起,他肚子的墨水还不如韩冈,但考试考多了,肚子里难免存着一堆见闻,“当年晚生入京应试,同科举子中,以苏子瞻、苏子由兄弟二人文名最盛,其下曾氏四子及其姻亲二王,不让两人专美御前。福建章子厚、章子平叔侄也是名声远布。还有新近深得王相公所喜的吕吉甫,最后是章子平首冠蓬山。

不过众子之中,唯张子厚【张载】、程伯淳【程颢】得道学三昧,亦有传人在侧。张子厚还设了虎皮椅开讲《易》,文相公都过来捧场。但子厚的两个表侄也来与辩经。一夜之后,子厚就撤坐辍讲,自愧不如二程。”

路明说得口沫横飞,而韩冈的眉头却皱了起来:“先生通晓大道,烂熟经典,只是口舌之辩并非所长。‘吾道自足,不假他求’,天地至道上,先生何曾认输过?”

程颢、程颐的确捣过张载的场子,虽然美其名曰辩经。张载第一次去考进士时,已是三十有八,早已名满关中,弟子环伺,他弟弟张戬都已经考上进士好几年了。当时殿试刚刚结束,张载榜上有名,而琼林苑的闻喜宴还没开始,趁这个空闲,文彦博帮张载设虎皮椅与兴国寺中,宣讲易经要旨。而程颢、程颐与他一夜相谈之后,张载便撤去虎皮椅,向人说,易学之道,吾不如二程,可向他们请教,二程由此在京中名声大振。

可张载并不是认输,他当时便说了‘吾道自足,不假他求’,不论是佛老之道,还是二程传承自周敦颐的道学,张载都不认为是真正的道。他有自己的世界观,自己的‘道’,不会因为在易学上辩论失败而动摇分毫——能当众承认自己的不足,便足以体现出张载的自信。

路明脸上的笑容不变,接口道:“没错,以天地大道论,横渠远比程正夫说得更明白。程颐连进士都没考上,怎么能跟横渠先生相比。”

韩冈为之乍舌。这位免解贡生的舌头真是会转弯,知道自己是张载的弟子,便不再用张子厚来称呼,而是尊称为横渠和横渠先生,变得够快的。

只是他讨好的言辞实在太过恶心,韩冈都被噎住了,干咳了几声,自行转过话题,“路兄多次前往东京,在当地相熟的朋友应是不少才是。”

“说起来,晚生当年也的确在京城结交不少好友。”路明答非所问,“王介甫相公面前,晚生都是能说得上话的。与如今在在秦州做官的王子纯【王韶】也是要好得很。他几次写信请晚生去秦州做事,说要荐晚生为官,信中还说‘明德不出,奈苍生何’。可晚生总是想着考个正经出身,便去信多次推辞。”

韩冈的神色变得古怪起来,抿着嘴,不知该恼还是该笑,这一位当真是极品啊,拉着虎皮做大旗,这是标准的江湖声口,君不见后世的一些骗子公司,总是在办公室里,挂起一些与名人的合影纪念。

不过古代信息不通,一般人的耳目都很闭塞,像路明这样信口胡诌,照样能骗到一群人。而韩冈自己,也是有着深切体会和经验的。只是路明用王韶的名头来给自己垫脚,还是让韩冈好气又复好笑。

可路明并不懂看人脸色,兀自说的兴高采烈。他历经多次科举,关于进士科的话题在肚子里能搜到千八百来,熟悉的各科人物更是多不胜数,说上三天三夜也不带重复。

见到韩冈被路明缠住,刘仲武也松了一口气。再看着韩冈脸上时不时闪过的不耐烦的神情,心中大乐,不由得哼起了小曲儿,‘你韩三也有今天!三十年河东,三十年河西,你让俺吃尽了苦头,风水轮回转,也该轮到你韩三了。’

注1:关于孺子其朋,现代人还有另外几种解释。不过这里只取孔安国的注疏。

