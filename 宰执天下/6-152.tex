\section{第13章 晨奎错落天日近(17)}

“何为国是?”曾孝宽在王安石的书房中问着。

“不就是新法嘛。”吕嘉问没好气的说道。

反对新法,就是反对国是,就是该被赶出朝堂。这是一直以来新党对反对者的态度。

而韩冈在殿上又进行了归纳,内容更加明确,王安石对此没有什么异议,曾孝宽也同样觉得韩冈归纳的没错:“更明确一点,就是依靠新法来富国强兵,进而恢复灵武故土,收复燕云失地。”

“一回事。”章敦说道,他半眯着眼,说话也是有气无力。

新法是施政的手法,富国强兵是施政的目的,而西夏和辽国,便是一前一后两个目标。后三条其实都是以第一条为基础,而旧党所反对的,归根到底还是触动他们利益的新法。

章敦、曾孝宽、李定、吕嘉问,今夜都来到了王安石的府上。宰辅之间,依故事是不得无故串门,而言官之首的御史中丞,更不应该登门造访他监视的对象。不过在御史台几经洗劫之后,朝臣们早就没那么多顾忌了。

新党一脉的核心人物济济一堂,挤在王安石家中不算宽敞的书房外厅中。

在灯下,王安石脸上的皱纹更多也更深了,脸色也不好,仿佛蒙了一层灰,看得出来他最近一段时间着实是心力交瘁。

“的确是一回事。”曾孝宽道,“但韩冈要修改国事,到底是打算修改哪一条?”

听曾孝宽如此问道,房中的重臣们不约而同的皱起了眉。

韩冈一向将自己的真实目的掩藏的极深,他今天在殿上说要把收复燕云的最终目标改一改,的确是就事论事,针对现在双方相持不下的焦点,可实际上没人相信到了垂拱殿上共商国是的时候,他会只针对进行攻击。

章敦想起了在王安石府上初遇韩冈时,韩冈所提出的几条建议;曾孝宽想起了与韩冈同判军器监时,韩冈拿出的板甲和飞船;吕嘉问也想起了让自己失去了进入两府的机会的廷推;而王安石更是想起了自初遇韩冈,直至如今,韩冈身处逆势时所用出的种种手段。

不论他说了些什么,背后总是会藏着更多。他说的的确都是实话,但绝不是全部的真相。就像河上的浮冰,永远都只有十分之一露在外面。

曾孝宽道,“如果韩冈是要更动新法,就反而好办了。新法中不论是哪一条哪一款,都是在天下各路进行了长时间的试行,才最终推行天下。”

其实曾孝宽所说并不客观,即便是便民贷、免役法,其中某些条款也是没有经过验证便开始推行了。不过对大部分新法来说,曾孝宽的话倒是没错。青苗法、免役法的,都是几十年前就有人在呼吁和试行,并非王安石拍着脑袋独创出来的。韩冈过去在地方上并没有推行过有别于新法的法度,若贸然拿出一条两条来,驳斥他很容易。

“新法诸条,不可能轻易更动。没有经过试行,什么法度能推行于天下各州各县?”吕嘉问收起了愤恨,平静的说着,“即使韩冈蓄谋已久,只要一日没有在州县中试行过,就别想推行天下,代替行之有效的法度。”

王安石轻轻的点头,吕嘉问分析得没有问题,即便韩冈想要有所动作,也不会选择从新法入手。而且即使韩冈能改动新法中的某些条款,也并不伤及新法的根本。便民贷、免役法、保甲法等诸多法令加起来才叫做国是,只是改动一点其中的条款,不影响大局,且以韩冈的身份从政事堂直接动手就可以了,没必要这么麻烦。而要将作为国是核心的整套新法加以改变,那样的变动,不是他几句话就能成功的,垂拱殿上的会议,也不可能让他如愿。

“富国强兵……”李定跟了上去,“这一条是先帝拟定国是之初衷,正是有了相公的富国强兵,韩冈才得以进用。他最多也只能说富国须富民,不可能否定强兵。”

王安石和章敦都点头。用排除法,将一个个选项都删去,王安石道:“那么也就剩收复燕云一项了。”

西夏已经被灭了,最后还有可能被韩冈攻击,成为他的目标的,终究还是由熙宗皇帝赵顼和王安石共同定下的北进方略。

“都该预备着,若事涉新法,也好应对。此外……”李定沉声道,“今天殿上韩冈说要改变北进的方略,三天后在垂拱殿上若敢言辞反复,乌台不会坐视不理。”

章敦皱眉道:“北进要分开来说。当年曾与韩玉昆议论过,若要收复燕云故地,最好从不利骑兵使用的云中着手,而燕蓟得放一放,不能从河北进兵,得以守御为主。”

“河东?”吕嘉问咧开嘴,笑道,“两任河东没白去啊!”

章敦看了王安石一眼,对吕嘉问道,“如果当真打算收复失地,从河东出兵的确比河北好。即便败了,也还有雁门关在,不至于丢城失地。河东是能守故能攻,河北是不易守故而不易攻。”

河北有陂塘防线,除了河水上冻的几个月,其他时候还是有着不错的防御力——尽管远远比不上燕山。如果春夏时节,河北稳守边境,而自河东全力北上,辽人就只能在云中大同那块狭窄的盆地中与大宋最精锐的禁军相抗衡,骑兵最擅长的战术完全施展不开。地域狭小的盆地,也约束了辽人向其中投放军力的数量。

而且在太原、代州方向上,聚集兵力也比河北还要容易一点。由太原到关中的轨道已经在修建中,如果加紧进度的话,两三年内就能修筑完毕。到时候西军要北上代州,第一批在十天之内就能赶到雁门关,而且不损战力,这是全骑兵的辽人都很难做到的。而河北方向,即便修成了京城到北界的轨道,河北前线得到的援军也是京营禁军,而不是国中最有战斗力的西军。

王安石和吕惠卿明面上都是想在河北打开局面,要不是抓住了耶律乙辛篡位的机会,当韩冈提出河东之议,肯定当时就败了。不过现在通往关中的轨道还未修成,这就是河北方略最后的机会。

听了章敦的说明,吕嘉问问道:“也就是说,到时候韩冈在垂拱殿上,肯定会说河北出兵不如河东出兵?”

“韩玉昆论兵一向以稳妥为上。”章敦道。

“可惜用兵就不是了。”吕嘉问冷笑,“天下人都知道的。”

李定摇摇头,章敦和吕嘉问越说越偏了,“如何进兵,已非国是,是庙堂运筹!如果韩冈觉得可以在垂拱殿上谈论此事,那就大错特错了。”

攻辽和如何攻辽,两件事的确不是一个性质。韩冈若是东拉西扯,御史中丞的李定肯定也不会坐视。

“资深有所不知,如果韩玉昆打算在全国推广轨道,那便是国是了。”

“推广轨道?”李定不解,茫然道:“为何?”

“韩玉昆所求南洋、西域,远及万里。而北虏只在千里之外。没有轨道连接各路,如何能让官军远行万里?”

相对于辽国,韩冈提议作为目标的偏鄙小邦,一方面更容易扩张拓土,另一方面,也能积蓄国力,给日渐增多的人口一个安置的地方。

但这就有了个问题,太大的国土管辖不易,离中原越远就越难治理,大宋边陲的羁縻州成百上千,想要将之纳入朝廷的直接管理之下,近乎于呓语。

而轨道正是缩短了国内各路的距离,即使只能连接到几个靠近边境的大州,也能给吞并异域和镇压边地叛乱节省大量的时间。

吕嘉问脸色阴沉下来:“这么说来,南洋、西域之类的话,只是幌子喽?!”

“是二而一,一而二,相辅相成。南洋、西域、西南夷,不论哪一处,都不需要动用太多的兵马。朝廷即使在攻打辽国之余,都能腾出手来攻取其中一处。只有轨道这一件事,才能够让朝廷没有余力去攻打辽国。”

“郑国渠?”王安石抬起眼,问道。

章敦点头,“正是。”

曾孝宽、李定和吕嘉问都沉默了下来,如果韩冈如此提议,就不可能轻易击退他。

战国末年,秦国国势如日中天,虎视关东六国,韩国正当虎口,遂遣水工郑国入秦,游说始皇,开凿河渠,连接泾水、洛水,以灌溉关中。虽然这是想让秦国的国力消耗在庞大的工程上,但郑国渠,最后也让秦国取得了更大的优势。

曾经与韩冈交好的章敦,很清楚韩冈有着属于他自己的一套治国方略。一旦他主政,可不仅仅是将新法修修补补,也不会只是去推动修筑轨道。铸币局和国债都只是冰山一角。

今日之事,韩冈隐藏在后面的到底是什么?章敦现在唯一能够确定的,必然是与韩冈的那一套方略有关。可到底是什么,会拿出哪一件来,他与一群人聚在一起,讨论了半日,自觉还是一头雾水。

“无论如何,国是都肯定要改动了。”章敦提醒王安石,“秉国,人之所欲,何人能无动于衷?”

熙宁六年时,因市易法一事,王安石受到了旧党疯狂的反扑,当时先帝赵顼也觉得废了这条非议最多的法令来安定人心比较好,但王安石坚持不退让半步。因为在他看来,这就像是遇上洪水的大堤,即便只有一处涌水的小洞,接下来也会造成大堤整体崩溃。

不能像过去那般倔强,国是虽不是吕嘉问主持的市易法,可韩冈拉拢来的重臣们也不是旧党。

韩冈提议由重臣推举两府宰执的人选,最后的决定权还是在太后手中。几天后的共商国是,最后拍板的依然还是太后。也许新党局势占优的情况下,太后不敢一意孤行,但万一两边相当,她必将支持韩冈。

必须舍弃一点,然后最根本的那一部分才能保存下来。

章敦相信王安石能够了解,只是他不知道王安石会不会退让。

书房中,一时之间静了下来,几人都望着王安石,看他如何回答。
