\section{第12章 锋芒早现意已彰(15)}

现任雄州知州刘舜卿,是河东功勋卓著的名将。地位、声望皆非普通将领可比。

如果当初他没有被调离代州,辽军不可能偷袭雁门得手,更不可能拿下代州,然后一路打到太原府。而刘舜卿曾为韩冈旧属,亦曾得韩冈举荐,如今他受命镇守雄州,刘绍能想做什么,都会被他给盯着。

幸好有一件事,大宋对军中的祖制是大小相制,刘绍能若只听吕惠卿的话,不理睬刘舜卿,刘舜卿也拿他没办法。只能设法寻其错处,然后上表弹劾。

相应的,如果刘舜卿不想理会吕惠卿,吕惠卿也不可能直接将他给撤职查办。同样是只能设法寻其错处,然后上表弹劾。

不过以吕惠卿的地位,他说出来的话份量自是要比刘舜卿重得多,而他在御史台和河北诸监司中的影响力,也不是刘舜卿能比。若真有必要,请走刘舜卿也不是不可能。

“可是……”吕温卿紧皱着眉,欲言又止。

吕惠卿的想法,他这个做弟弟多多少少能够体会一二。

可刘舜卿能在雄州这个大镇出任知州,背后未必没有韩冈为其张目。当年韩冈第一次为河东帅臣,时任代州知州的刘舜卿便因为亲附韩冈,在战后立刻被调离,间接导致了前年辽军得以入寇河东。现如今刘舜卿镇守在比代州更为紧要的位置上,未尝不是韩冈给他的补偿。

“放心。”吕惠卿笑道,“刘舜卿也只是一个人。”

雄州只是河北千里国境的其中一个州而已,最多因为是高阳关路第一道防线,而地位更高一点。可雄州向东还有沧州,向西更是有真定府、定州、保州等军州。河北四路,除去大名府路外,位于缘边区域的可是有高阳关、真定府、定州三路。

知定州的是蔡延庆,定州路的路治所在。这是绝对不可能听他吕惠卿指使的主,以他的资历、地位和性格,也不可能听任何人使唤。不过功名之心,人皆有之。看着当年在堂上连座位都没有的后生小辈身后都能张起清凉伞,而自己一事无成,吕惠卿相信,蔡延庆不会没有想法。

“那南面呢?”吕温卿用着更低的声音问道,“那个灌园子……”

“不必担心。”

吕惠卿的语气很平淡,听起来不是很在意韩冈的样子。

韩冈在河北路上的声望,不会比他在陕西、河东差到哪里。但他在河北禁军中的影响力,完全没有他在西军和河东军中要高。而韩冈的表兄李信,因为那一次北进的惨败,在河北禁军军中的声望也跌倒了谷底。

尽管如此,吕惠卿也不会自大的认为在河北军中,能够跟韩冈直接比拼影响力。韩冈的影响力再低,也是相对于他本人在陕西、河东的水平,其他人如何能比得上?

军中医疗制度的确立,种痘法的推行,以及累累军功和一桩桩发明,让韩冈在军中的声望无人能及。其他文官要恩威并施才能控制的军队,他出来亮个相就足够了。

可县官不如现管,韩冈远在东京,而吕惠卿他现在就在这里。

作为安抚使,吕惠卿手上的军权其实极其有限。想要调动任何一部禁军,都要经过朝廷的同意。即便是调动数百厢军修补大名府河防,事后也必须要向朝廷报备。

所谓安抚,只是安靖地方而已,又非宣布威灵的宣抚使,连经略的名衔都给去掉了,单纯的安抚使,根本没有对外作战的权力。

可是如若敌国侵犯疆界,攻击边寨,指挥守军进行反击,朝廷绝不可能对此进行责难。

刘绍能身在雄州,直面辽境,帐下数千兵马的驻地控扼要冲,一旦边地有警,立刻就能出兵。那时候,朝廷想不打都不成了。

吕温卿却不能不担心,吕惠卿仅仅一句话,如何能让他安心,“但介甫平章如今也压不倒韩冈,太后总是偏帮着他。”

韩冈的资望,在朝野中自无法跟王安石相比,但说起对太后的影响力,王安石就要瞠乎其后。至于其他人,那就差得更远了。

“那是在边境无警的情况下,万一边境有警,他若还想要点颜面,就不可能再反对出兵。”

“啊……说得也是。”吕温卿被说服了。

不是因为吕惠卿的言辞,而是看到兄长的表情,让他相信吕惠卿已经做好了充分的准备。

“耶律乙辛篡位,辽国人心必乱。若大宋坐视不理,数年间,他就能坐稳皇位。如果官军北进讨逆,原本只能隐忍的辽国忠臣,就有了举义的机会。机会不是等来的,是打出来的。”吕惠卿沿着池塘边的小路慢慢走着,边走边说,“如今士林清议,民间议论,皆曰可战。国中兵精粮足,也非旧日可比。不趁辽国人心混乱时进攻,更待何时?”

“的确如兄长所说,外面连卖云吞的小贩,都说要趁着辽国内乱去打上一场了。要是必须得出兵,谅那灌园子也只能附议,不至于像文相公当年一般,连脸皮都能不要。”吕温卿又笑着低声说,“听说他当年被介甫平章硬是派去在横山,明说不要任何功劳,可他还是尽心尽力。像他这般重名,想必不会故意再从中阻挠兵械粮秣的转运。”

吕惠卿轻轻摇头,向前走去,他可不会相信韩冈的品性。

韩冈当年在横山尽心尽力,是因为他有恃无恐。有韩绛那样的主帅,最好的结果也不过是打个平手,占下一小片地来,十万兵马劳而无功。看透了这一点,韩冈又有什么不敢用心做的?放到现在,韩冈如何会给自己这个机会?

王安石是平章军国重事,而韩冈只是一个参知政事。两者之间的差距,绝不仅仅在地位上。这一点,在战争期间,会分外明显的表现出来。

探手拂过池畔的枯枝,吕惠卿道:“河北动刀兵,漕司最为重要,李公择精擅会计之术,可惜不能助我。”

吕温卿明白吕惠卿的心意:“兄长放心,本地的粮秣供给,小弟会尽心尽力。”

“那就好。”吕惠卿了解自己兄弟的才干,也相信他能够做到这一点。

河北路转运判官,已经有足够高的资格去为三军的粮秣操心,也能够适时的派出转运使的干扰。

河北转运使路曾为一路,后分为东西二路,仁宗时再次并为一路,熙宁六年又分为东西两路,等到前岁宋辽百万大军战于河北、河东时,为了方便河北路军略,河北东路、河北西路再一次合并,战后也没有再变动过。

河北帅司、漕司居于一城,自然多有龃龉。吕惠卿与李常从变法开始,就没合得来过。如今坐在一座城中任职,李常还有监察大名府治政的权力,两边当然

李常在第一次廷推时,是韩冈的支持者,之后便被任命为河北转运使。这一年来,也在河北站稳了脚跟,要不是自己设法将兄弟调往漕司任职,许多盘算,现在都没机会去施行。

吕惠卿在池畔的凉亭中坐定下来,远远跟在后面的仆役,纷纷上来,摆好暖炉、香炉,架好防风的帐子,摆上各色果品、糕点和热饮子,又摆好了酒水和温酒的器具,做好这一切,又全都退了下去,留下一个可以让吕氏兄弟继续座谈的空间。

吕温卿亲自给兄长温酒,吕惠卿则按着不知配合什么曲调的拍子,轻轻拍着自己的腿。状似悠闲,一无所虑。

吕惠卿并不担心韩冈会一直压在自己头上,尤其是韩冈在皇权之争中卷得太过深入之后。

向太后能信任王安石,能信任韩冈,能信任韩绛、张璪,能信任章敦、苏颂,但太后绝不会信任当时并不在文德殿上的他吕惠卿。

吕惠卿知道这是自己如今最大的缺憾之处。可相对而言,对皇权的牵涉没那么深,也就少了许多挂碍,真到了难以预测的日后,这就是好处。

而如今只要韩冈还幻想着一石二鸟,一边做他的宗师,维持着好名声,一边还要在朝堂上压倒新党,那么就注定他难以成事。

纵使是天子都做不得快意事,必须得学会放弃。仁宗皇帝连宠爱的嫔妃都保不住,何况臣子?

能舍故能得,韩冈过去做得很好,为了推广气学,连高官厚禄和累累功绩都能舍弃,可如今却不见当年让先帝与宰辅都哑口无言的灵气了。

看到现在的韩冈,吕惠卿也只能说如今的他是在太贪心了,过去的成功冲昏了韩冈的头脑,思虑太幼稚,想得也太简单。

温酒壶中的酒渐渐热了,酒香四溢,吕温卿为吕惠卿和自家满上酒,端起酒盏,他说道:“小弟在这里先预祝兄长能够旗开得胜,收复燕云。”

“燕云。”吕惠卿端起了酒杯,头也摇了起来,“倒是可以去想想。”

他冲呆愣的吕温卿笑了笑,举起酒杯,一饮而尽,“愚兄的打算,一开始就不是燕云。”
