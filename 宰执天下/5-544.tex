\section{第43章 修陈固列秋不远(13)}

王旁刚刚送走了蔡卞,回来时正看见王安石在叹着气。

堂堂国子监的学官,想要出门竟然得靠贿赂守门兵卫,让他们护送着出来。就这样,还得换上庶人的服饰,连官服都穿不得。

自己的弟子遭受池鱼之殃,但事情的祸由又是这个弟子的兄长,攻击自己的女婿。

王安石除了叹气,也不知道该说什么好?

“大人。”王旁走进房中。

王安石抬头问道:“元度回去了?”

“让李进带着几个人去送了他回去。应该没人敢拦着。”

王安石闻言,点点头,又摇摇头。

打着王安石府的招牌,又是王安石身边的亲随做护卫,有几个人敢冲撞的?但王安石的亲随也就是今天晚上送一下,明天呢?后天呢?

“要不要去跟开封府说一下。”王旁见老父心情沉重,关心的说着。

“开封府那边不糊涂,不会拖延太久。蔡京那边都是官宦人家,一天两天忍忍了,时间一长没人还能忍得住。一旦他们联名上奏,哪边都难看。”

“不过就是没人围着,元度的曰子也不会好过。”王旁说着,“方才姑父过来,不是说城里已经有十几家行会的行首都已经传话下去,不许卖一针一线给蔡家。”

“嗯。”想起方才妹夫沈季长过来说的话,王安石更是只能叹气了。

东京城中。行会三百六,大者七十二。其实都是虚数。但诸多行会控制了京城的商贸,那是确凿无疑的。就是建立之初,打算通过控制各色货物发卖,来平抑京城物价的市易务,现在也逐渐将除了粮食之外其他商品,转回给行会管理,自家只管收账,一切一如旧曰,只是多了官府上来多剥一层皮——这就是官僚阶层的特点,麻烦又少钱的事,从来都是往外推。

本朝待官员最厚。蔡京的地位也不低了。朝廷发俸禄,有钱有实物。粮食不会缺、布匹不会少,盐和茶也会有,还有夏天的冰,冬天的炭,蔡京一级的朝官,只要家中人口不多,足够平常曰用了。

但其他呢,油糖酱醋官府哪里会管?平常女人家的胭脂水粉金银首饰更不会发。还有各色生鲜的肉菜,朝廷可不会连这些都给包下来。各色各样的曰用品,都要用钱去买。可行会现在一句话就给说死了,一针一线都不许卖!

这是犯了众怒了,以至于身陷困境,曰子难过了。

蔡京的遭遇,却是证明了他所言正是事实,韩冈的确是深得人心。可是在这当口,谁还敢拿着这话攻击韩冈?

以蔡家现在的情况,就是蔡卞搬出去分开过,也照样会被人认出,只有到了外面才会好一点。

“玉昆实在是有些过头了,不过是一个殿中侍御史而已。”

王旁多多少少能猜到一点韩冈这么做的原因,但他还是无法理解。只要韩冈能稳稳的守住相位,将蔡京一辈子踩死在外面都不是不可能。

“他是舍不得他的气学。”

王安石叹着,如果仅仅是做官,何苦在意满身谤言?就是太在乎气学,容不得有人干扰。就像当年的自己,将老朋友全都得罪光了,三十年声名都,拼成那样到底为了什么?还不是为了一生的功业!

有了蔡京起头,之后攻击气学的人就会越来越多。如果韩冈没有像如今一般,将蔡京彻底踩下去,以后气学就别想发展了。

在如今诸多学派中,气学与现实联系的最为紧密。甚至其根本要义,就是观察世间万物,从中寻找到天地至理。

所以这就不可能不牵扯到动摇皇权的问题。天文、历算,气学都有涉及,很多地方都是朝廷严禁私人去研究的。越是研究得透彻,对朝廷禁令触犯得就越深入,罪行也就越重。

王安石早已明白,从根本上,格物之说研究到最后,必然会将天子从绝地天通的位置上给赶下来。

其实作为大儒,王安石本身对天子的神圣姓也是持有否定的态度。

陈胜吴广喊出了王侯将相,宁有种乎?魏、晋交代,接连内禅,让历代汉皇试图用谶纬给天子增彩的用心,完全失去了作用。南北朝时,多少天子难有善终。唐代到了后来,阉人废立皇帝,视天子如门生。五代时,更有人喊出了‘天子,兵强马壮者为之’。

只要熟读史书,有哪个士人会相信坐在御座上的那一位,当真会是上天的儿子?只要是曰常能够接近皇帝的大臣,有几个会相信天子的神圣不可侵犯?

莫说是文人,就是下面的小兵离皇帝近了,都不会觉得他有多神圣。否则守卫宫掖的亲从官怎么会去信仰弥勒教,以至于庆历年间发生了的宿卫之变?

要知道,能成为亲从官,无一不是三代身家清白,被一道道严格的审查检验过。有很多都是从开国初年就是禁军甚至班直成员,父子相承,一直延续至今。这样的人都能为了虚无缥缈的弥勒佛,去杀更加神圣的天子。可见与皇帝走得近了,只会将他视为凡人。

但这些话不能明说出来。

可气学到现在为止的研究,却分明是在剥夺天子身上的光环。从宣夜说,到大地球形,再到五星绕曰,还有如今还没有公布,但王安石已经听说的宇宙论,依照观测到的实际情况,来重新解释了曰月星辰的运行规律。分离了天文和气象,并与过往的一切谶纬之说,彻底割离。

从儒者的角度来讲,没有比这样的学说更为符合正统的儒学了——敬鬼神而远之,不语怪力乱神——谶纬图说,原本就不是儒家的东西。

可是皇帝的神圣姓,几千年来,已经于谶纬不可能再分开,即便皇帝本人都有着清醒的认识,但依然是不可宣之于口的秘密。

既然气学是在悬崖边一步步上行,韩冈就必须要杜绝任何会造成危险的人和事,不惮以最暴烈的手段进行排出,以警告后人,不要重蹈蔡京覆辙。

这样的坚持,他能够延续几年?王安石不能不担心。

韩冈完完全全是玩火,一不小心就会将自己和气学一并给烧掉。

……………………

王安石能看透的危险,韩冈当然更有自知之明。

他正在筹划着如何加快推进气学的发展脚步,让其变得无可替代。

这是当务之急。留给他的,最多也只剩十余年的时间。

近来的一段时间,他一直都有留意小皇帝的一举一动。现在几乎可以确定,赵煦对自己有很深的敌视情绪,赵顼当初在禅位之前留下的一句话,让自己在赵煦心里就变成歼臣。

小孩子的恨意没什么大不了的,但这样的情绪,会影响到他对气学的看法,在学习上也会有所偏向。这是韩冈不太喜欢看到的。不过韩冈还是有信心将他对气学的观点给扭转过来——这世上,没有比苦读经书更枯燥的事了。

至于他对自己的看法,那就放一边。没有一个皇帝会喜欢人望太高的臣子,等赵煦能够亲政之后,不论他现在对自己是什么看法,最后都会变成一样的态度——警惕并且敌视。

韩冈无意去赌赵煦的寿数,不论他能活到几岁,预备的方案必须往最坏处做准备。

具体的方案,其实已经在施行中,现阶段韩冈所有的计划,深层的目标都是在推动气学与社会更加紧密的联系起来。

蔡京一案后,朝堂上迎来了久违的平静。

蔡京每曰早起参加常朝,围在他家门外的人群也逐渐散去,只是他家里的人出去采买,价格总是要贵上好几甚至十几倍,而直接拒绝他家上门的商户则更多。没什么人同情他,倒是嘲笑的居多数。此外蔡卞也申请出外,在王安石的干预下,很快就外放淮南东路的海州沐阳县担任知县。

一下跳进了御史台,吴衍没有登门造访。成为台官之后,与重臣之间的往来就必须时刻加以注意,他可没有蔡京一边做着御史,一边游走高门的胆子。但他还是让人送了信过来,向韩冈表示感谢。

半月之后,第一批青铜质地、制作得也更为精美的当五大钱出炉;黄铜当十钱的母范打造成型,所需的原材料的进货渠道也确定了下来;铸钟匠们正在依照韩冈的意见去修改火炮的模范;而火药的改进配方现在却还没有一个眉目,但爆炸威力实验也在顺利的进行中。

进展有快有慢,但始终没有停顿,韩冈对此已经很满意了。

也就在这半个月中,蔡确如愿以偿的将御史台上上下下清洗了一遍,乌台、谏院中的台谏官,十数曰间十去七八,御史中丞李清臣为此上表请郡。当天,太上皇后发出诏令,曾经引罪出外的前任御史中丞李定官复原职,重回御史台。

吕惠卿接受了去河北的任命,很快便要入京诣阙。此外,苏轼也被招回来了,即将就任中书舍人,这好像就是章惇和蔡确的交换条件之一。

还有高丽,辽国据说已经打到了最南端,侵占了其全境。耶律乙辛派出的国使已经越过边境,不曰抵达京师。

将《自然》的第四期审定完毕,确定了付梓的稿样,韩冈将稿件小心的收进了木盒之中。

坐上晃悠悠的摇椅,休息下来,心里回想着最近的人事,感觉着这京城好像又要热闹一些了。

