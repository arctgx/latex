\section{第25章 鸟鼠移穴营新巢(中)}

连着几个月,状态都差得可以,对各位书友很是不好意思。编辑长河那边也在跳脚。俺现在在这里顿首谢罪,并保证这个月的更新二十万字,以赎旧过。这是今天的第二更。

云南这个地名乔二苟知道,戍边这个词,乔二苟也明白,两个词合起来的意思,他一样清楚。

若是脸上刺字,那是发配充军,本来乔二苟以为会被这样处置。——充军可不是当兵,是在军营里面做杂役,吃得最少,干得最累,逃得最晚,死得最早,若是充军在边地,多半就等不到刑满释放的一天。

可现在字刺在手背上,又是戍边,这是当官军、吃官粮了吗?虽然这不比在京城做乞丐的舒坦,但好歹能留下条性命,比配军要强。

但乔二苟的美梦很快就被打破了。

改头换面的乞丐们被集合在大营门口,原来载着他们过来的货运大车换成了客运的四轮马车,还有一队比军营中的同袍,看起来更加彪悍的军汉正等着他们。

在大门前等了一阵,那些军汉也没什么动作。乔二苟的手上一阵一阵的刺痛,他心里开始担心伤口会不会烂掉。抱着右手,不想惹事的他蹲在了靠边的位置。

一名军汉来回踱着步子,最后晃了过来,乔二苟忙起来让开,赔笑道:“官人……”

“什么官人?”乔二苟刚开口,那军汉就瞪起眼,“俺哪里像官人了?叫俺十将。等指使过来,你们再喊官人。”

十将是一都中的小军头,比都头低,比队正高,的确不能算是官。

这位十将将一众乞丐看了一圈,阴森森的道,“你们仔细别犯了事,让指使拿鞭子抽你们。一路都听话点,想吃杀威棒,现在就说,免得道路上伤了还要人服侍你。”

乔二苟讨了个没趣,小鸡啄米般的点头退开。刚退回来,旁边就挤过一个人,一张让人厌恶的笑脸,“原来二狗哥也来了,小弟真是瞎了眼,方才都没看到了。”

乔二苟定睛辨认了一下,放松下来,“是李花子啊。”

“现在可不是花子了。”李花子咧开嘴,身上干干净净,但一口烂牙却是污糟的让人恶心,他故作神秘的低声道:“你听说了没?”

“听说什么?”

“李大官人啊。”

“哪个李大官人?”

“还能有哪个李大官人?”

两个人的对话仿佛在打哑谜,但乔二苟听明白了,也知道是谁,城中有名的李大官人,娶了妻,捐了官,妾室成行,儿子一堆。场面上光鲜得很,但他出身是乞丐,营生也是乞丐,是京师中有字号的丐头之一。寻常人说李大官人,可能性多了去,但乞丐中提到李大官人,那么就只有一个。

“他怎么了?”乔二苟张望一下左右,也同样低声,“这一回,哪个头领都没送来,是不是出了事。”

“他啊,”李花子捂着嘴,却没遮住幸灾乐祸的笑容,“前几天过堂,被挖出了旧账。”

“旧账?”乔二苟哎呦一声,“这不是死定了?”

李大官人在乞丐中素来是个名人。一个丐头出身,平素里做买卖,便是拐了好人家的小孩来,女的留在家中淫辱一番,然后远远的卖出去,男的就挑断脚筋,毁了相貌,然后拉出去行乞。父母看见都认不出,后面有人盯着,小孩儿也不敢认。

每天这些孩子都要上缴讨来的钱,讨了再多也吃不饱,到最后没一个能活过五年。李大官人呢,一看到人死了,就丢出去喂狗,最是狠毒不过。而他最狠的一面,是将小孩儿砍了手脚塞进坛子里养起来,十个里面不定能活一个,但活下来一个,一年就能带来上百贯的好处。

手中掌握了这么十几二十个残疾乞丐,每年都是几百贯的收入,再掺和些其他买卖,那就是上千贯了。可为了这上千贯,祸害了的孩子不知有多少。大多数丐头都看不过眼,暗地里咒他生儿子没屁眼。但京师中能买房买马的丐头,就他一个。其他的丐头,有钱归有钱,最多在城外买个小院子。

更是因为有了钱,李大官人手底下的亡命之徒也有好几个,夺田、夺产的事情也没少做,手底下的人命官司堆起来能有一人高。

听到这样的一个人的坏消息,乔二苟半点同情心都没有。他平素里最多也只泼人一身粪水,那等绝子绝孙的阴毒勾当,乔二苟可从来没干过。

“可不是就死定了。”李花子嘬着牙花子,对乔二苟道:“俺听牢里的孔目说,当天这案子就报上去了。太后娘娘大怒,不但定了凌迟,还把李知府叫了过去一阵痛骂。”

“太后都知道了?”乔二苟吃了一惊,这不是捅到天上去了吗?

“这么大的事,怎么能不让太后知道?”

朝廷每年秋决名单,皇帝、太后都是要过目的。而京师里面发了大案,又有谁敢满着太后而不上报?

李花子先向军汉那边张望了一眼,手臂一伸,搂过乔二苟的脖子,将声音压得更低:“你家的刘黑头,这一回,那颗黑头多半也是留不住了。十几家丐头,家全都给抄了,家里的人不分老幼也都给抓起来了,运气好发配云南,运气差就全家死光。就像那位李大官人,手上苦主太多,被判了凌迟。过两天就行刑。”

凌迟!乔二苟浑身一个激灵。

他可是看过凌迟的,前些年有个宗室打算谋反,给抓了起来,有两个想要跟他一起谋反的蠢货,一个被判了腰斩,一个就被判了凌迟。

行刑的那一天,法场那是人山人海,住在京城内的人,怕是有十分之一来看热闹,比大赛马场和大球场人都多。乔二苟也挤过去看了。

一开始的腰斩就已经很惨了,在铡刀上被拦腰斩成两截,只剩半截的人,拖着肠子惨呼了许久才死。乔二苟感觉他叫了足足有半刻钟,跟他一起去的也有说一刻钟,也有说两刻钟,总之感觉很长很长。

可腰斩虽长,却不如凌迟。人犯给绑在柱子上,脚下放了个大瓦盆,里面都是灰。侩子手就提着一柄牛耳尖刀,在那人犯身上一片一片的把皮肉割下来,丢进脚下的灰盆中,血也是流到盆里,一点也没外溅。一千多刀后,柱子上就只剩骨突突的一个红人,皮给割干净了,红的肉、白的筋,还有肚子上的一块黄色肥油,都是血淋淋,可人还活着,还在有气没力的惨嘶着,一直叫到两千多刀后。

这一场戏,乔二苟看了足足两个时辰,看到一个大活人变成了瓦盆中的一堆碎肉,事后他回去,做了整整三天的噩梦,几日没有吃好一顿饭。

想起旧事,李花子的声音听在乔二苟的耳朵里,就变得分外阴森,“他的两个儿子都要陪着一道上路,菜市口上的枭首一刀等着他们。可惜我们看不见了。”

李花子与乔二苟说了一阵话,又悄然离开,看着他转头又找上一人,乔二苟心想,这样的人,难怪能够左右逢源。还有那些被捉走的丐头,乔二苟私下里恨不得他们去死,但表面上,也要为他们唏嘘几分。

不过那个刘黑头,乔二苟在他门下快十年了,对人还是够仗义,拿完份子也会给人留下吃碗汤饼的钱。想到他就要被处死,乔二苟心中一股兔死狐悲的感慨还是免不了。

所在墙角边,望着门前的车马、军汉。

守在门前的这一群军汉。几个坐在马车边,经过乔二苟的仔细打量,都是要走远门的装束。两个军汉在那边不知说了什么笑话,一群人在哈哈大笑。另一头是一对夫妻,看起来才结婚的样子,丈夫是军汉,浑家来送行,手里提这个包袱,拉着手说话。浑家抹眼泪,丈夫直叹气,一对儿难舍难分的模样。

乔二苟明白,这队人马,将会押送他们南下什么云南路。

又等了一阵,军汉们终于有了动作,但他们并没有立刻赶乔二苟等人上车,而是先过来几个人,先给乔二苟右脚上给拴上了绳子,然后又拴上了旁边的叶小三,接着又加了三个人。五人一组,被一根绳子连在了一起。

乔二苟原本笃定的判断,这下又没了把握。心中惶惶不安,这是要上法场吗?他围观过不少次法场,要处死的罪囚,都是全副镣铐枷锁,脑袋跟手绑一起,脚上也套一条两尺长的索子,让犯人只能走不能跑。

旁边就是十几个人拄着长枪,稍外一点,还有人提着神臂弓,尽管人人都是百无聊赖的懒样,但看见周围戒备森严,兵器罗列,乔二苟都不敢乱动一下。身边的叶小三更是吓得差点就要漏尿,眼泪水也是咕嘟嘟的往下滚。

“别怕,到了地头就给你们解开。”过来绑脚的倒是个和气人,对叶小三这个看起来只有十一二岁的后生好声好气的说话,“到了那里有房住,有地种,只要老实肯干,也不会再饿着,日后还有自己的产业。”

“李老实,话挺多啊!”

听到这个声音,正说话的李老实立刻闭了嘴,慌慌张张的站起身,与其他同袍们一起向来人行礼。
