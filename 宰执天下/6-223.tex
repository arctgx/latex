\section{第28章 夜钟初闻已生潮(五)}

又是一年中秋。

同样的月圆之夜,并非阴云密布,也并非冰轮高挂,给大地遍洒清辉,而是一重薄薄的云层,挡在了满月和大地之间。仿佛重新用薄纱糊上了窗棱,透过薄纱的月光,映得一切都得朦胧起来。

没有天灾人祸,也没有战争兵乱,朝廷上没有大的变动,韩冈近来的工作也十分轻松,一家人理所当然的坐在了一起。

除了远在家乡的父母,除了求学异。地的长子,韩冈的家人,现在都在相府的后园中。

池畔的水榭里,韩冈坐在上首,面前摆着精心制作的美酒佳肴,精心装扮过的妻妾们陪在身边,未喝酒就让人迷醉。

孩子们坐在更下面,他们被严令禁止喝酒,不过他们的面前也都按个人的口味,放了果汁或饮子。

上一个中秋之夜,朝堂上有些风波,云南那边战局未定,那时执掌朝政的韩冈,不能说不辛苦。过节时,也是心事重重。

但这新的一年,韩冈却过得很轻松,云南的战事结束了,朝堂上也无大的纷争,一切都上了轨道,所以这日子过得飞快,只觉得上一个中秋刚刚过去不久,八月十五的满月,就又升到了天顶。

或许这是自己开始变老的征兆。

韩冈自嘲的想着。因为小孩子,总会嫌时间过得太慢,只有年纪大了,才会在突然间猛然想起,时间,都去哪儿了?

“大人,母亲。”

韩钲捧着一杯酒,领着弟弟们上来为父母祝寿。

韩钲的说话用词,已经不是小孩子的口吻了。到了明年,他就将要离开家,前往他期待已久的横渠书院读书。

不过此时的韩钲,外表上还是小孩子,脸上的青涩依然没有褪去,拿着变声期的嗓音,努力装出一副大人的模样,反而惹人发笑。

韩冈与妻妾们对视一眼,千言万语化作一个会心的微笑。

结缡十余载,孩子一个个长大成人,这期间,又有多少风风雨雨,多少言语都难以尽述,但不论如何,他们都是一起走了过来。

韩冈还记得当年任职开封府界提点,曾带着全家人在黄河上凿冰钓鱼。那时候,官位虽卑,却比现在要自在得多。

当时,最大的几个孩子,要么刚刚学会走路,要么还在襁褓之中。

而现在,最年长的一对儿女,再一两年就都要结婚了,下面的弟弟们,将会一个接着一个,成婚、生子。

韩钲定下了富家的孙女,这是韩冈示好旧党元老的行为,也是韩冈融合这个依然有着巨大潜力的团体的又一步。

韩冈犹记得他前世曾经看过的一部风靡了几十年的小说,其中有一个丐帮,帮中分作污衣、净衣两派。污衣派下层人多,而上层的四个长老中,却有三个是净衣派。

如果排除掉小说中的童话成分,那么可以肯定,净衣派的势力会稳如泰山,不论出了何等变故,他们都会牢牢控制住丐帮的大部分权力,即使出了变故,换了一批人上台,但很快,新的净衣派就会再次出现在丐帮的高层

因为每一个污衣派长老,他们的儿孙都会是净衣派的成员。绝不可能有了身为长老的长辈,还要饿着肚子去讨饭的。

随着时间的推移,旧势力的颠覆者总会成为自己刚刚打倒的一类人。

有时,这是颠覆者的梦想,有时,这又是颠覆者的悲哀。

韩冈也不例外,他是颠覆者,但绝不会做一个抛头露面的颠覆者,梦想也好,悲哀也好,都必须藏得极深,在表面上,他绝不会与社会风气为敌。

就像子女的婚姻问题,他打定主意要为为每一位子女做好安排,作为封建家长,韩冈已经合格了。

在已经定亲的儿女下面,韩冈已经跟李信和王舜臣定下了儿女之亲,而李信则是与赵隆为子女交换了生辰八字,此外,李信更是与已经去世的老帅张守约成了姻亲。

从韩冈身上牵出的亲缘关系,犹如一张网,将整个朝堂的文武势力笼罩。

但韩冈的势力再大,也不能强迫他人成亲。子女们受到越来越多的关注,也有他治家严谨,子女各个成材的成分在。

所以韩冈绝不会去主张什么自由恋爱。

他肯定要考虑,如果自己提倡这件事,会不会让自己的子女陷入受人嘲骂的风险。

更何况小孩子不定性,自家的女儿倒也罢了,而自家的儿子,作为宰相家的衙内,有了合理的借口后,不知能用这件事祸害多少家的闺女。

婚姻制度的变化,自主婚姻的增多,来自于经济基础的变化,建筑在教育的普及。绝不可能在这个世代就出现。

韩冈可不打算超前个几步,让自己变成疯子,儿女也都要受到连累。为人父母,不得不考虑打这一点。

几巡酒后,韩冈让子女们都先退下下了,与妻妾坐下来继续喝酒。

“官人,昨天你看了大哥的信上吧,知道上面写了什么?”严素心几杯酒后,红晕上面,拿着酒杯,愤愤的对韩冈道:“除了问好,就是说在蒸汽机!”

韩冈摇摇头,“没有光说蒸汽机,还有高压锅。”

这个时代的高压锅,本质上不过是锅炉的副产品。可是,当韩冈第一次听到这个名字的时候,甚至比起看到了锅炉的进步都感到惊讶。而惊讶之后,便是难以言喻的惊喜。

这世上,已经开始有人沿着他打下了地基的道路继续向前迈进,已经并不全然需要韩冈继续手把手地在前面领路。

不论是自然,还是其他科目,很多都是在韩冈的控制之下,一点点取得进步,但高压锅不同,完全出乎了韩冈的意料,而且还给了他巨大的惊喜。

韩冈从来没想到能用石棉作为橡胶的替代品。

也许作为堵塞缝隙的材料,石棉的物理与化学性质,都远比不上橡胶。但这个时代的中国,只可能有香蕉,而不可能有橡胶。

作为替代的石棉,解决了有没有的问题,至于好不好,现在也没有太多的需求。

“官人,王守义今天来了。不过晚上去了六叔那边。”

酒过三巡,王旖对韩冈忽然道。

王守义是王安石家的老家人,姓名在这个时代,也是不鲜见的。

就是让韩冈莫名耳熟,每次听到这个名字,不免带着点恶作剧的想法,比如想着让素心传他一道十三香的手艺。

十三香在千年之后,成分是不保密,但配比算是秘传。但在今日,材料才是难点。香料之贵,堪比金银。也只有钟鸣鼎食之家,才能集齐需要的香料。

现在韩家的所谓十三香,与后世自然大不一样,这香料方子,是严素心多年的心血。顺丰行在南方的分号,每年都要将大量香料运来北方,而生长在西北的孜然等香料,也同样是顺丰行的主要商品之一。能像她一样把香料当成姜葱一般的辛香料来糟蹋,也只有御厨才有这份大方。

不过这王守义,韩冈可没听过他的庖厨手艺,倒是忠心耿耿,让王安石喜欢使唤他。

“哦,岳父、岳母可还安好?”

“爹娘都写了信,还有二兄的,也有一封信让他转呈。就放在官人的书房里。”

韩冈寻常与王安石并没有太多的鸿信往来,尽管通过邮局递送很是方便,可是王安石还是喜欢每隔几个月让人上京来转交。相对于王安石,韩冈的岳母吴氏倒是经常利用邮局与两个女儿通信。

如今的邮局,已经越来越频繁的介入人们的生活,尽管要说什么私密话,很多人还是不太相信朝廷的邮局,不相信官府能够保护他们的阴私。若是能够托亲朋好友来送信,大部分人还是愿意多费一番周折。可是大多数情况下,人们也只能信件,而且邮局的出现,也使得人们更加乐意与人通信,而不是像过去,使得家书有如万金之贵重。

但王安石完全不同。

“不知老相公要跟官人说什么事?”严素心好奇的问着周南。

周南低声道,“反正不会是诗文。”

每一期的《自然》,韩冈都会通过邮局寄给王安石,但王安石写了些得意的诗文却总是会忘了韩冈。

不知是不是韩冈改变了世界,有好几首韩冈印象中的杰作都没有出现,不过也有可能是王安石瞧不起韩冈诗词水平,尽管王安石最喜欢玩集句的游戏,将过去的诗词,东抽一句,西抽一句,拼凑起来合成一首诗,可他从来也没跟韩冈讨论过有关诗词的问题。

“不知会是什么?”韩冈也在想着

……………………

这一个中秋节,向太后心中很烦。

将桂花酒换了,冰凉的绿豆香薷饮喝下去也不减心中的烦躁。

“官家到底是怎么回事?”向太后黑着脸,问着身前的垂手恭立的翰林医官。

钱乙恭声道:“太后放心,官家并无大碍。”

“上次钱乙你也说官家只要平日里小心饮食就够了,身体并无大碍。现在你看看,怎么还是这样?”

钱乙是小儿医的圣手,他的医术,在来到京城之后,不断与其他名医切磋,又在医学院中多方磨练,早已超越了过往所以儿科医生。

但向太后信任他,让他担任天子的贴身御医,最重要的,是他不像其他御医,给人看病时总是不肯给个明确的说法,而是直话直说。

为人父母,最恨的就是给儿女诊病的医师,绕着圈子述说病情,又因为小儿体质与成人的差异,不肯将药开实在了,总是用一些吃不好病,却也绝不会把人吃死方子。

看着儿女从小病拖到大病,从大病拖到绝症,哪个父母不恨?钱乙说话虽直,但至少是毫不欺隐。或许有人反感他的做法,甚至暴怒,但他的做法,得到太后和韩冈的认同,同时也包括更多的病家。
