\section{第30章 随阳雁飞各西东(15)}

一个多时辰的崇政殿议事,将应对辽人的方略大体确定了下来。

基本上都是依照韩冈的提议。西北的防务交给陕西宣抚司来负责,连便宜行事的权力也一并给了吕惠卿。而去拜访耶律乙辛的使团,要等一等溥乐城的消息,十天八天之内没有城破的消息传来,那么就可以出发了——出发前的准备其实也需要这么长的时间。

对河东、河北加强守备的再一次强调,两府也达成了共识,诏令明日就会发出去。相信前后两份诏令接连颁布,边州守将们不会有人敢于懈怠。

商议已毕,各人公廨中还有一堆事要处理,恭送了向皇后,便各自离殿。

韩冈与章敦、薛向走在一起。殿外的廊道上,宰相参政和枢密使们隔得有些远,东府、西府看起来就有点泾渭分明的架势了。

薛向跟章敦说了两句之后,转过来就又跟韩冈道,“宿州那边已经安排得差不多了。这两天,发运司就会将札子递上来。到时候,可就要劳烦玉昆了。”

“此乃韩冈分内事,子正兄尽管放心。”韩冈再一次申明自己的立场,让薛向放心。

薛向点点头,脚步随即快了一点,让出了韩冈和章敦讨论的空间。

见薛向离得稍远,章敦就侧头对韩冈道:“薛子正可是一心想要将这件事做成。”

韩冈笑道:“现在也没别的事值得他上心了……也最好多用些心,东府那边可不会那么容易就松口。”

修筑轨道一事,当然不会由薛向先行提出,而是会先安排一名六路发运司的官员请求修筑宿州至京城的轨道。之后才会引动了薛向出面。韩冈和章敦的支持,更是得在薛向表态之后。

绝大多数有关政策政令的提案,基本上都是这个模式,从地方传到中央,从底层推到高层。除非是一些重要的人事、国策,一般来说,很少会一开始就由宰辅提出来。这样意图性太强,也会少了许多讨价还价的余地。

现在听薛向的口气,差不多就该安排好了。过几日下面报上来,就要在崇政殿中与东府商议。只是这件事,若没有韩冈的全力支持,只凭章敦、薛向,再加上一个不管事的王安石,通过的几率着实不算大。

六路发运司属于中书门下管辖,薛向以枢密副使的身份插手其中,等于是侵犯职权,东府的三位宰执——过几天还要加上一个曾布——将奏报直接丢进废纸堆里那是理所当然。熙宁初年,种谔奉天子密诏招降嵬名山,夺取绥德城,枢密院就因为整件事没有通过院中批准,便一力主张将绥德城还回去,同时还将种谔和居中传递密诏的高遵裕一体治罪,硬是贬去了南方。连天子侵犯职权都容忍不了,何况平起平坐的同僚?

接下来的日子里面,薛向还不知要向东府妥协些什么,丧权辱国的条约必然是要一个接一个签。不过他在两府也没多长时间了,不趁此时挥霍一下手上的权力,日后还真的不会有太多机会了。

薛向的事说说也就罢了,毕竟不是眼下的重心,“关西那边的事,不知玉昆你怎么看?”

“一切还要看吕吉甫,坐在京城谈关西,跟纸上谈兵也没两样。”韩冈摇摇头,“对辽人的挑衅,要坚决回击,但也不能往大里打。其中缓急,都要靠吕吉甫来把握。不是件轻松的活计。尤其……”话说到这里,却猛地一顿。

“尤其夏帅还是种谔。”章敦将韩冈没说出口的话补充完整。

“是啊,离得溥乐城最近的偏偏还是种谔。”韩冈苦笑起来,他说的不是距离,而是关系,“种朴被围溥乐城,不论种谔怎么喊打喊杀,他都占着人情。只希望吕吉甫能赶得及压住他。”

当然,章敦和韩冈就算在担心吕惠卿能不能及时阻止种谔的独走,也不会想到这时候新任的陕西宣抚使正在肚子里面骂娘。

虽然在表面上完全看不出吕惠卿已经气得七窍生烟,他甚至还能好言好语的抚慰银夏路派来报信的士兵,道一声辛苦,然后让其下去休息。

但郑希作为跟随吕惠卿多年的亲信门客,能清晰的感受得到,数尺之外从吕惠卿身上传来的如火如荼的熊熊怒意。

种谔竟然出兵了。

虽然表面上还不能叫做出兵,只是他本人带着一班亲信去盐州观察敌情。亲生儿子正在被敌军围攻,做父亲去救援放在什么地方都能说的过去。何况种谔还没有调动兵马,仅仅是本身去盐州坐镇。吕惠卿前两天还被他给迷惑了,反而对种谔能坐镇盐州而感到安心。

但今天种谔自盐州派来的信使,带来了辽人游骑在盐州外围活动的消息。这可就真正的是图穷匕见了。

盐州是除了韦州城之外,离溥乐城最近的一处要地,屯有重兵。辽人遣斥候盯着盐州那是情理中事,没派人去才会让人惊讶。

可种谔竟然说,为了要提防辽人,他准备调遣银夏二州的本部,暂时驻泊于盐州。同时还请求宣抚司调遣鄜延兵马,从延州和绥德北上,来填补银州、夏州的人员空缺、

这是骗鬼啊!

辽人若真的想去攻城,直接去韦州不好吗?何必多走上一百里往盐州去?论起防御力,盐州可比韦州和溥乐城强得多。种谔的名气也远比赵禼要响亮。契丹人何时蠢到会用牙去咬石头?

要是种谔就在眼前,吕惠卿可不在意将他拎出来教训一整天。

只是以吕惠卿对种谔的了解,基本上种谔在发信的同时,已经先将事情做出来了。而且他绝不会认为种谔会满足于将辽人逼退。

种谔只凭手上的盐州兵马,已经足够他翻手为云、覆手为雨了。没人可以小看种谔,自己的亲家徐禧究竟是怎么成就了种谔的威名,吕惠卿记得很清楚。

丢下种谔发来的公函,吕惠卿阴着脸大口大口口的喝着微凉的茶汤。

辽人动手太快,宣抚司刚刚成立,还没有来得及整备各路资源——要知道,当年韩绛宣抚陕西,整整用了一年多的时间来整合——这就给了种谔上下其手的余地。

郑希劝着他的东家:“枢密宣抚陕西,种谔也是归入宣抚帐下。他的功劳,就是宣抚你的。”

吕惠卿当然知道这个道理。只要将辽人挡回去,就算什么都没做,最大的功劳还是他吕惠卿的。

“倒不全然是为了这件事。”吕惠卿随手从桌案上拿起一封信函来,郑希进来前他刚刚批示过:“仁多零丁和叶孛麻写了信来,说是他们这个冬天肯定过不下去了,要朝廷给钱给粮。熊本不肯担干系,便给转到我这里。”

“青铜峡的那批余孽又来打饥荒了?”郑希嫌恶的眼神扫了那封信一眼,也不接过来看,“怎么,若是朝廷不给他们过冬粮食,是不是就要起兵了?”

“就怕给了他们粮食,正好就可以拿来充军粮了。不用怕围攻城寨却久攻不下,会因此而断粮。”

这等前面表顺服,拿到好处就捅刀子的手段,党项人过去用了一百多年。看到泾原路转送来的这封信,吕惠卿的第一反应就是要再移文泾原路,让熊本加强防备,并准备好支援鸣沙城。

吕惠卿将信丢回到桌上:“韩玉昆在河东一通好杀,其实却是一劳永逸。”

“可是这也大损阴德。真正跟辽人勾结,掩护辽军潜入的黑山部族,也就那么几家,多不过五千——要不然黑山党项各部也不会在南下的半路上死那么多——剩下的近两万人全都是枉死的。”郑希叹着气,“也就他不在乎。”

“杀人多损阴德的事,韩冈他不是不怕,而是不信。”“我也一样不信。若仁多零丁和叶孛麻还不肯老老实实,就让他们去追黑山下的亲戚好了。刀子递到我手中,就别指望我会放下……”

“宣使。”一名属吏匆匆来到庭前,“曲珍来了。”

话刚说到一半的吕惠卿,闻言便立刻站起身,走到厅门处,没有犹豫,抬脚跨了过去,然后径直走下台阶。

已是白身的曲珍,堂堂枢密使、宣抚使的吕惠卿竟是为他降阶相迎,让看的心中暗惊。

须发皆白的老将,在帅府行辕中奔走的一名虞侯引路下,绕过了正院的照壁,只见衣着金紫的吕惠卿端端正正的立于院内,温文浅笑:“曲侯,久违了。”

…………………………

青铜峡中,风沙更烈。

积雪的山头,仿佛被一层黄沙抹过。冰结的黄河贯通了峡谷,旧年的契丹残部隔岸而居。叶孛麻靠东,仁多零丁则住在西面。

比起一年多前,现在的叶孛麻要苍老得多,上万族人生活在只有数十里长的青铜峡中,与其他部族日日相斗,峡谷中万余族人就跟孤魂野鬼一般,拥有的土地不到旧时的十分之一,要牧场没牧场,要田地没田地。

去年刚刚安定下来的时候,宋人那边补助了一点,又派了两个官过来指点怎么种田。但一年下来,收获远远比不上消耗。若不是他还有一点手段,早就压不住族里的年轻人了。

叶孛麻收起刚刚从黄河对岸送来的信,脸色瞬息数变,但最后还是叹着气将信收了起来。仁多零丁既然下了决心,他这边也不能落后。

“团练。”一名契丹人正大马金刀的坐在叶孛麻的帐中,扬起的下巴让人知道他心中的傲气,“信上说了什么?”

“仁多那边也有消息了,仁多零丁已经准备好了。”叶孛麻叹了一声之后,便抖擞起精神,他对这位契丹人道:“既如此,我叶孛麻也不能输人。今日召集各部,明天便出兵!”
