\section{第21章 山外望山待时至(一)}

【大汗,专访的题目怎么变成了讨论熟女了。趁此机会,求红票,求收藏。】

朝会之后,便是崇政殿中天子加上宰执重臣们的议事。而议事结束后,王安石照例被留了下来。不过没有留在崇政殿,君臣两人一起往着武英殿去了。

赵顼最近心情很好,行动如风,神采焕发。陕西连番大捷给他的兴奋还没过去,宫中又紧跟着给了他新的惊喜。虽然向皇后那里自长女延禧早夭后就一直没有消息,但昨日有两名嫔妃却一齐传了喜信。消息传出来,今天朝堂上,便是一片恭贺天子的声音。

子嗣艰难是赵氏天子的通病,从真宗时起,皇子的数量就从没超过三个——真宗一个独苗,仁宗一个都没有——尽管赵顼真正的祖父和曾祖父皆是以多子而著称,生下的儿子都是两位数,但过继给仁宗的英宗也只生了三人。

而赵顼继承皇位后,已经三年多了,好不容易生了两个儿子,却全都夭折,向皇后生的女儿同样夭折,让赵顼对已经有了儿女的二弟甚是羡慕。不过如今宫中又有喜信,赵顼正日盼夜盼几个月后他的子女能安然出世。

而朝堂上,尽管反对变法的声音依然激烈,但随着在御榻上坐得时间越来越长,他已经能对无稽的党争之词做到充耳不闻。再不会因为几个臣子跳出来指着变法一阵乱骂,就坏了一天的心情。

文彦博今天上朝时中气十足,指着免役法骂了一个时辰没停口。不过等到章惇把司马光、吴充前两年对旧时差役法的评价拿出来后,文彦博虽然还在骂,但气焰却被压下去了许多。

虽然赵顼也不喜文彦博对新法事事反对,但凡王安石的主张也没一处赞成。但在司马光、吕公弼、吕公著接连出外的情况下,赵顼却必须留一个反对的声音在朝堂上。

异论相搅,是宋室天子控制朝局的家传法宝。文彦博在朝中一日,反变法的声音虽然低弱,但毕竟还有着主心骨,但若是文彦博再去职,朝堂上的反变法派肯定是树倒猢狲散。只剩变法派一家,赵顼亦难自安。

其实免役法的出台有些仓促,若是依照王安石一开始上报给他的规划,这一法案应该是再经过一年的体量,到明年下半年时机成熟后才开始推行。但为卑官加俸并给胥吏俸禄的计划不知怎么流传了出去,却不得不将之提前。

因为事发仓促,颁布的条令中有不少缺憾,文彦博抓住其中的几点加以攻击,便是闹了一个上午。也就是因为文彦博闹腾得太厉害,赵顼留王安石下来商议军务,却没有把文彦博一起留下。

王安石跟着赵顼,君臣二人一路走到武英殿。摆在偏殿正中的沙盘不再是前些日子的秦州山川,而是以横山为主轴,囊括了鄜延、河东山川地理的沙盘。沙盘之上山峦起伏,无定河和黄河穿山而过,条条支流清晰可辨。

不过当王安石在殿中见到了一名武将,就再没去在意沙盘的事,‘燕达?’

前日在绥德城立下大功的西军将领正在沙盘边跪着。燕达现在已经是鄜延都监,但因为他是郭逵被提拔起来,跟种谔不合,在韩绛面前也不受待见。今次他上京诣阙,也是被韩绛打发出来的。

“平身。”赵顼出声示意燕达和殿中的内侍都站起来。

燕达年纪在四十上下,身材雄伟,挺身而立有渊停岳峙之态。不过容貌丑陋,面如锅底,虬髯蜷曲,略显细小的双眼寒芒隐生,瞪起来仿佛就要吃人,如同古之恶来,让殿中内侍也不敢正眼看他。

不过燕达的性格完全没有半点外表上的暴躁刚戾,相反的,却是以带兵宽厚著称。他前日面圣时,赵顼问他带兵当以何者为先,他的回答是‘爱’。赵顼诧异的问道爱怎么能超过威,燕达则道,‘威非不用,要以爱为先耳。’

这番话让赵顼听了赞赏不已。若天下统军的臣子都这么想这么做,也不会时不时的就有兵变了。李复圭在庆州,恣意威福,苛待众军,连钤辖都监都是想杀就杀。读了多少年的书,连个武夫都比不上,真该让已经被贬到外地的他来听一听。

大宋天子走到沙盘边,王安石跟在后面走上去。燕达见状,躬身退后了两步,不敢居于王安石的身前。

赵顼双手扶着沙盘边框,眼睛盯着无定河,沿着河道从无定河与黄河的交汇处一直向上看去,越过绥德城,停在了横山的北麓。这里插着一面小旗,白色的只有半个巴掌大小,上面写了两个字——罗兀。

“韩绛奏请进筑罗兀,并言其地有十利三胜。据有此地,横山便稳入我手。不知燕达你对韩绛的说法如何看?”

罗兀城的城址与绥德城一样,同样位于无定河畔。不过比起犹在横山南麓的绥德城,罗兀城是一下向北跃进了近六十里,距离西夏东南重镇银州,则只有十里之遥。

这是个很冒险的计划,西夏的反扑将会比绥德筑城时更为激烈,很可能要面对十万以上的敌军——不再是号称,而是实实在在的人数。

可一旦计划成功,大宋便能完全控制横山地区。西夏倚之为屏藩的横山蕃部,以及由祥佑、左厢神勇两大军司共同坚守的东南防线,将彻底崩溃。横山一失,同在无定河畔的银州、夏州将不复西夏所有,而被党项人视为生命的青白盐池,也将落入宋人之手。

西夏国的两个核心地域,一为兴灵,一为银夏。兴庆府和灵州是西夏的中心,位于黄河之畔,处于荒漠之中,有七百里瀚海阻隔,兵力难及。而由银、盐、宥、洪、夏几州合称的银夏地区,就位于横山北麓。银夏诸州向兴庆府提供西夏一半以上的财税,以及超过三成的兵员,失横山,则西夏不保,若能控制银夏,西贼覆亡可期。

立一城而夺西贼半壁江山,赵顼心动了,王安石也同样心动。燕达在天子面前,也是如此说道,“罗兀若能守住,横山必定。横山一定,西贼便不足为虑。我越瀚海攻兴灵,转运劳苦,粮秣难以为继。而铁鹞子、步跋子没了横山蕃人支援,越瀚海来攻,同样会困于粮草。且失了横山,只靠兴灵一带的出产,并不足以供养西贼的十万大军,到时候,党项人也只有向朝廷乞降一条路可走。”

但这一切的前提就是罗兀城能守住。燕达不好在天子面前说韩绛不是,只能用此曲言。王安石轻轻颔首,燕达也算是心思细腻了。

他问道:“光是一个罗兀城不知能不能守住西贼的攻打?罗兀孤悬在外,若是贼军突至,绥德城缓急间却是难以及时救援。”

单一的城寨即便再坚固,也不过是个点,在城池附近必须修造可以相互支援的堡垒,才能构筑起一条稳固的防线。孤城难守,只要稍稍了解军事,就能知道这一点。

宋人自仁宗时起,不惜国力的在宋夏交界处大规模的修造堡垒,连成了两千余里的防线。每一处关键性的战略要地,其周围不论哪个方向,无不是十里、二十里内便是一处寨堡,城寨群互相交通勾连,组成一个完整的防御体系,

比如秦州的甘谷城,其左近,就有吹藏、大甘、陇诺三堡护翼,而最近开始驻守甘谷的秦凤都监刘昌祚,又向朝中申请向北修建尖竿、陇阳二堡。这几座堡垒都是在开始修筑甘谷城时就有了规划的。

“罗兀城虽然孤悬,但只要力保连接绥德的道路不失,西贼必然劳而无功。且其地向东五十里,便是河东地界,若是西贼来攻罗兀,河东便可出兵救援。”燕达停了一下,沉声道:“要稳守罗兀,须得陕西河东同时出力!”

赵顼沉吟良久,方说道:“……你先下去吧!”

燕达叩拜了之后,退出了武英殿。神色坦然,并没有因为天子突然命他退下而慌乱失措。

赵顼双眼直勾勾的盯着沙盘上的黄河东侧的一片山地,缓缓低吟:“河东……”

王安石提声道出了赵顼心中的犹豫:“若如燕达所言,当加授韩绛河东宣抚一职。”

韩绛以执政之身出掌陕西宣抚。临机有自由处断之权,而且朝廷已经赐了他空头宣扎两百道,填上姓名年甲就可以给人封官。这是为了方便他指挥军中,招揽横山蕃部。如果把河东划到他手上,当然得给他同样的权力——至于另外派人宣抚河东,只会添乱,达不到护翼罗兀外围的初衷,赵顼和王安石想都不会去想。

赵顼叹了口气:“不过要想兼任陕西、河东两路宣抚,光是一个执政资格却是不够。”

而且赵顼还担心着韩绛本无军功,素不知兵,为陕西宣抚已经有些怨声,若为遽为两路宣抚,他怕是要杀掉一批河东将领来立威以固权威。桀骜不驯的骄兵悍将当然要严加处置,但赵顼怕闹出乱子来,反会耽误正事。

“那请陛下加韩绛同中书门下平章事,以宰相之尊领河东陕西两路军事,当能如臂使指。”

