\section{第31章 离乡难知处(下)}

侯敂拍了两句马屁,又对韩冈道:“不过这些流民都是赶着要往东京城去,要不要将他们拦下来?这些流民都没有过所,要拦下他们倒也不难。”

此时人们离乡出外,并不是自由通行。和尚道士靠度牒,官员靠驿券,而百姓则是要靠过所。过所,就是路条,路引,相当于后世身份证、介绍信之类的东西。只是一张不大的纸片,但关系到外出行人是否有着合法身份。

而侯敂说得的确没错,流民们不可能拥有过所,他们在离开乡里的时候,绝不肯还记得到县衙去花钱办一张通行证,要扣留下他们在律法上有充足的理由。

但韩冈却不同意:“此事不妥。必须是让流民自愿留下来,否则必落人口实。”他对侯敂笑了笑,“反正今天他们走不出白马县,现在就派人去招募雇工,想必他们也想早一点找到养家糊口的工作。”

韩冈否觉了自己的意见,侯敂的态度依然恭恭敬敬,“下官明白,这就去办。”

韩冈点了点头,腰略略一弯:“劳烦了。”

“不敢,乃是下官份内事。”

侯敂行礼之后退了下去。对他的恭敬,韩冈已经习以为常,现在在外面巡视乡里的冉觉见到自己时,也是一百分的恭谨。不仅仅是官位的问题,更是进士和非进士的差距。换作是进士来做县中的僚属,绝不会像现在的侯敂和冉觉这般老实听教。

世间重文,进士出身的官员一入官就身着绿袍,高出侪辈一头,晋升之速更是远远过之。非进士出身的官员,就算在进士面前有些自傲,也是得靠着才干,但侯敂和冉觉在韩冈面前,却是没有自傲的底气。

侯敂走后,厅中一阵静默,过了片刻,魏平真叹了口气:“终于来了。”

韩冈也跟着轻叹一声:“……是来了。”叹声过后,目光复为锐利,沉声道:“终于到了!”

“正言。”魏平真向韩冈一揖,主动道:“在下去再查一下库中的钱粮,不再看看怎么都不放心。”

韩冈点点头,魏平真老于衙中事物,比自己考虑得更周全。视线转到方兴身上,韩冈的要说的话,方兴心领神会。不待吩咐便说道:“我去帮着侯县丞,也顺便去看一下那群流民。”

“拜托了。”韩冈拱了拱手,起身目送他们各自出门。

回过身来,他对着最后一名幕僚。这名福建士子,虽然年轻,但将白马县学的几十名士子管束得当,当得起出色二字。

“虽然现在正撞上大灾,但学业决不能放下。接下来的几个月,我会尽量抽时间去县学,但剩下的还是要靠节夫你多多费心了。”治下士子的水平也是考绩的一个方面,韩冈可不愿在这方面丢脸,“今年县学推荐举子去考太学外舍的时候,希望他们都能高中入学!”

“在下明白,正言放心。”游醇抱拳,朗声说道。

三名幕僚各有各的事要做,纷纷离开之后,公厅中只剩韩冈一人。手指习惯性的叩着交椅扶手,韩冈陷入沉思。

野渡既然能够通行,那么官渡也肯定要通航了。明天后天,白马渡镇那边就该上报,申请开渡口——也有可能会担心流民的问题,而拖延一阵,自己倒是不能让他们这么做。但不管怎么说,接下来的几个月,必然是最后的难关。就不知道朝廷中,能够给他多少支持——如果能让自己的职权早一点确定下来那就太好了。

在河北走了一趟之后,想必吕惠卿和曾布都不会再抱着什么幻想。而是要全心全意的支持自己的工作。有他们的建言。说服天子就不会那么困难。

昨日曾吕二人从河北匆匆经过白马县返回东京。在比前一次更为简朴的接风宴席上虽然没有多说什么,但只看他们难以掩饰的忧色,河北两路的整体情况肯定是十分不妙,比起韩冈隔着一条黄河看到得更为真切。就不知道他们回到京城后,会怎么跟天子汇报了。是如实,还是曲笔,又或是掩饰。

两人心境的变化,韩冈觉得短时间内,也不用担心他们会闹出什么幺蛾子的事了。争权归争权,但以河北如今的情况,一个不好,说不定整个新党都要完蛋。而旧党上下开始摩拳擦掌的样子,几乎都已经可以预见。外部的压力变大,内部也不得不团结起来。这个时候,肯定先要将眼前的麻烦给解决掉。

他们又能靠谁呢?

如果只看白马县,其实情况还算不错,水也有了,春麦也种下了,蝗虫正在清理中,安置流民的场所更是完备。在白马县的百姓们看来,他们的运气还是很好的,摊上了一个年轻有为的知县。而白马县的情况落在天子和朝堂眼中,也能明白,要想不让流民困扰京城——

——最简单的办法就是找他韩冈。

起身回到后院,韩云娘带着个使女迎了上来。

“三哥哥,回来了。”

韩冈向内张望了一下,奇怪只有云娘一人相迎,“你姐姐呢?”

云娘帮着韩冈换下外出的衣服,“旖姐姐又害喜了,素心姐姐去厨房,说是要炖些补品,南姐姐去照顾金娘和奎官了。”

“怎么又害喜了。”韩冈摇摇头。

王旖自查出有妊后,就害喜得很厉害,这些日子都是吃了一点就吐了出来,着实让人担心。

换了一身家中穿戴的宽袍,韩冈去了王旖房间。

王旖此时刚刚吐过,脸色稍显苍白,头发有些乱,看起来憔悴了许多。严素心正端了一盅炖好的汤在房中,要服侍着王旖喝。听到韩冈近来的动静,两女一起看过来。

“官人!”素心屈了屈膝,作为行礼。

“又忙到这个时候。”王旖用胳膊支起身子,“也要顾一顾身体啊!”

“没事的。”韩冈坐下来,将严素心手上汤盅端来,“凡事预则立,不预则废。现在辛苦一点,后面就能轻松了。”

揭开汤盅,一股带着药味的鸡汤鲜香就散了出来,韩冈向着里面看了看,去了骨头的鸡肉一片一片,散在白粥中,却是看不到一片药材。严素心熬补汤,都是用着小布囊装着切碎的药材,一起放到汤锅里炖,炖好后,将袋子拿出来就行了,不用担心药渣。温温的热气熏着,熟悉味道之后,韩冈还能嗅得出来鸡汤中用的是当归、黄芪还有党参。

韩冈不喜奢侈,而王旖自幼也是朴素惯了的。而这几个月,听说了外面的灾情,又见着韩冈的忙碌,家中的吃穿用度也都更加简朴——当然,棉布棉被则是要另说,自家的出产都是不花钱的——只是王旖怀孕后,她这个孕妇得到的照顾便是最多,吃得也是最好的。

韩冈用汤勺舀起一勺粥,吹了吹,递到王旖的嘴边。有着韩冈来喂,王旖乖乖的一口吃下。一勺一勺,吃进肚里的鸡汤药粥却熨得她心头暖暖的。

吃完之后,王旖拿着丝巾擦了擦嘴,脸有些发红,不敢看韩冈。却问道:“二哥怎么样了?”

韩冈笑道:“仲元越来越有架势了,他照管的事都没有问题,而且有他盯着,下面的人可是一个比一个卖力气。”

听说自家二哥能做事了,而且做得还很不错,王旖喜上眉梢,却又有些担心:“不要让二哥太累着。”

“让他一个太常寺太祝来帮忙,说实在的,有些当不起啊。”

王旖嘟起嘴瞥了韩冈一眼,知道他是开玩笑,嗔道:“只会耍嘴!”

韩冈开怀一笑,帮着王旁,让王旖心情也好了,这是他乐于见到的。从王旁身上就能看出来,人还是忙一些好。

接下来的数以万计、几近十万的流民,也必须要让他们有事可做,决不能仅仅是养在流民营中。就算仅是挖土堆山的空耗气力,也比每天用粥棚养着要强。王旁就是最好的例子,他过去一直守在家中,看着父兄处置国家大事,而自己一事无成,心理才有了问题。现在有的忙了,虽然只是很小的一桩事,但一段时间下来,却是如同换了一个人般。

究竟要怎么安排这些劳动力呢?是重造黄河大堤,还是整修官道?韩冈不由自主的又叩起了手指。

见着韩冈又陷入了自己世界之中,王旖和严素心不约而同都是叹了口气。但隐隐的却有几分骄傲,世间又有哪个女子不希望自家的夫婿是顶天立地的男子,靠着仁心仁术救民于水火的贤者?韩冈不正是在这么做吗?

三天之后,滚滚的黄河浪涛中终于看不到冰凌沉浮。准备了许久,终于到了正式开场的时候,韩冈来到了白马渡。此前通过野渡过河的流民已经多达千人,但此前做了那么多准备,倒也是将他们不费什么力气的安置了下来。

渡头上挂红披彩,以猪羊牛三牲祭过河伯,随着一声嘹亮的吆喝声,白马渡的渡船终于在停滞了四个月后离开了码头,而与此同时,对岸黎阳津也有数艘渡船离岸。

流民们终于来了……

