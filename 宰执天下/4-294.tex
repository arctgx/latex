\section{第48章 辰星惊兆夷王戡(下)}

王珪撺掇着天子早攻西夏,莫等痛失良机,元绛在旁敲着边鼓。

韩冈有趣的发现。他自进殿后,只见王珪和元绛在说话议论,其他几名宰执都没插话。不由自主的揣摩起几个没开口的宰执的态度来。

吕惠卿多半乐见速攻西夏。朝廷要钱要粮,自然是要加强手实法的推行。不过如果士绅们反弹得太厉害,为了维持后方稳定,赵顼也有可能拿吕惠卿开刀。这件事完全说不准,得看形势的变化了。

吕公著最近受了打击,由于陈世儒弑母案的牵连,在朝堂上的份量大跌,连累着西府被东府死死压住,连军事上的议题也给政事堂占据了主动。

而没说到钱粮,新入枢密院的薛向,暂时也不好开口。

枢密院中,唯一在军事上还有着足够发言权的郭逵却是沉默着。

从郭逵的表情上,看起来方才殿上的商议,天子表现出来的态度,并不打算派他去总掌攻夏战事,甚至是并不准备派他去前线——以郭逵的身份地位,去了前线,指挥权自然而然的就会聚拢到他的手上,就是种谔恐怕也抵挡不了。

也许能从他们的态度上下手……

王珪、元绛此时已经一番话说完,赵顼点头开了金口:“辽国内乱当是定局,没有辽国支持,西夏岂能抵挡得了官军?韩卿,你看看西夏该如何攻打?”

韩冈正在考虑用什么说辞说服天子,赵顼就已经下了定论,不再是攻不攻,而是怎么攻了。

韩冈这一下可就有些头疼了。

天子的态度已经出来了,加上西军是自己的基本盘,不便拦着他们立功的机会。想了一想,道:“横山已在官军手中,银夏唾手可得。禹臧花麻久欲投献,兰州也可轻易攻取。但兴庆府道远路长,其势难攻。”

“可是因为瀚海?”赵顼诘问:“从兰州沿着黄河走,不用过瀚海吧?”

“陛下明鉴,自秦凤和熙河出发,可以直逼兴庆府,不用穿越瀚海,唯一的问题是粮秣供给。两路的储备粮秣可以保证驻军的食用,但不足以维持总数可达十万人马的大军远征千里。相对而言,鄜延、环庆两路的情况就要好不少,身后是白渠粮仓,又有京兆府百万石的粮草储备。可偏偏有七百里瀚海阻隔。让步兵来走,最后能出来三分之一就很了不起了,至于骑兵,又怎么攻下灵州和兴庆府?”

韩冈话声一顿,郭逵立刻开口:“陛下,韩冈所言甚是。西夏大国,兵马众多,非交趾、河湟可比。如今势弱,也是百足之虫死而不僵。要想一次攻下兴庆府,的确不易。不如先攻下银夏和兰州,在两地做好准备,然后一举合围。”

韩冈暗自点头。这老家伙应该是明白了自己的意思。既然攻打西夏不可避免,那就分成两个阶段来攻。

先吃掉容易下肚的银夏和兰州,然后歇一歇,稳定住了新辟疆土之后,等待合适的时机,再行攻打兴灵。而这样一来,灭国大功不会集于一人之手,郭逵上阵也就有了可能。同时,有足够稳妥,赵顼也当喜欢这个方案。

郭逵的确是够老辣,配合得很好。只是这老家伙,说西夏就罢了,还要踩一踩交趾、河湟。韩冈瞥了一眼,心中有些火气。

赵顼沉思起来。韩冈和郭逵是殿上最了解兵事的文武大臣,他们的话不能不理会。同时说的也在理,这样也稳妥些。而且与直接攻打兴庆府的计划,在前期并没有任何区别。最后选择哪个方案,可以看打下银夏和兰州的情况而定。

他点了点头:“先由此来筹办。”

回到家中,周南的情况好了一些,睡得也安稳了,韩冈也稍稍放心了下来。

随着辽主的死讯在朝堂上传播开来,接下来的几天,大部分臣子都上书,催促着要与西夏一战,直捣兴庆府。

大宋年年大捷,灭国拓土。直接领导和参与战事的官员,一个个飞黄腾达,早就让人红了眼。区区河湟、交趾,就造就了两个枢密副使,那么西夏呢?

——人人都疯狂了。

一时间,请战声不绝于耳,甚至冲淡了已经近在眼前的过年的气氛。

而在响彻朝堂的的一片速攻声浪中,韩冈依然坚持着自己的想法——稳扎稳打,一口口吃饭才是上佳的选择——让他成了显眼的另类。

不过由于郭逵和韩冈采取的是同样的态度,所以赵顼一时还没有做出最后的决断。也许在天子心中,稳妥点也不差,只要没有进一步的变化。

韩冈这两天忙着公事,贤内助王旖则处理着家里面的大小事务。

入住的宅子破损的地方很多,整修房屋是少不了的。韩家也不缺钱,派人请了十几名工匠来,要好好整修一番,要以一个新面貌迎接新年。

韩冈出入家门时,木料、砖瓦、沙石还有石灰,都堆到了院中。连照壁都有几个工匠围着。说是素白的照壁不合如今京城中的风俗,要好生打理一番。

“正面随你们弄,背面冲家里的,弄得好看些。”韩冈吩咐下去:“去找些官窑的碎瓷片来,各种颜色的都要……汝窑的青瓷要多一点。”

虽然不知道韩冈葫芦里卖得什么药,但他的吩咐,对工匠们来说,不比圣旨差多少。而且官窑的瓷器贵重,但碎瓷片却便宜得很。才两天的时间,材料就都准备好了。十几个工匠在韩冈面前躬身待命,连王旖也出来了,想知道韩冈到底打算做什么。

韩冈要做的自然是镶嵌画。

照壁对外的一面,按照如今的习俗来。但对内的一面,韩冈准备拼出一幅山水画来。他向京中以山水闻名的王诜,邀了一幅画。

韩冈的面子,驸马都尉不能不给。也就一天,王诜就派人送了一幅《烟岚晴晓图》来,说是仓促之间难下笔,旧时习作还请韩龙图不要嫌弃。

韩冈怎么会介意?虽然他并不识画,但王旖看到之后,就抽了一口气,说是这一幅在王诜的作品中,也是顶尖的了,让韩冈不得不又封了一份重礼回去。

将样本依照雕刻的方法打到照壁上,摹写出大概的图样构成,工匠们就用糯米汁拌的黏合剂将合适颜色的碎瓷片贴到照壁上。虽然韩冈的要求不合规矩,但工匠们皆是一丝不苟的完成。而且很是兴奋,这等于是教了他们又一门手艺。

汝窑一片等黄金。‘纵有家产万贯,不如汝瓷一片’,在后世留下传说的汝窑瓷,在韩冈这里就成了马赛克瓷砖。

工匠们分块包干,一起动手。半日的功夫,烟岚晴晓图大体的构成差不多就成型了,虽然细节还要几日功夫去琢磨,但已经能看出韩冈这一番布置的佳妙之处了。

以王诜的佳作为本,或浓或淡的青色组成了远山近水,加上妙至毫巅的留白,都显示出韩冈本人还是有那么几分雅骨。

韩冈站在正堂前,欣赏着依然如同草稿一般的成果。西北之事只要有了定论也就没有自己的事了,闲着无聊,分心打理一下自己家中也不差。

王旖也出来了,盯着照壁仔细看,一幅兴趣盎然的样子。

已经是黄昏了,一阵急促的马蹄声由远至近,在门前戛然而止。

随即前日来传诏的小黄门绕过了照壁,出现在院中,又是来传诏的:“天子有旨,着韩冈即刻进宫。”

小黄门在宣诏之后,收了谢礼,便很痛快的倒出了缘由:“西夏国主秉常被囚,梁氏出面垂帘听政。”

韩冈都怔住了,老天爷这是玩什么把戏,要帮忙也不是这么帮的。西北二虏的国主一死一囚,其国中内乱可期。换作是任何人来选择,也只可能是速攻了。

暗叹一声,韩冈进内屋换了一身公服出来,方心曲领、长脚幞头、金带、鱼袋都戴好,正准备动身,又是一名内侍飞奔而至,却是常来韩家传诏的童贯:“天子口谕,宣韩冈速速进殿。”

韩冈愣了一下才接旨,怎么接连传诏?童贯下一刻就解释了:“龙图,辽主的死因从雄州传来了……是从飞船上摔下来的!”

话声刚落,连同工匠在内,院中的几十对眼睛都看着韩冈。就是童贯也是紧盯着韩冈,在他脸上搜索着表情泄露出来的真相。惊讶、崇拜,各种情绪蕴含在视线中的,不一而足。

“官人……!”王旖都忍不住惊讶。

“别看我,不关我的事。这肯定是耶律乙辛的功劳。”韩冈摇头苦笑,“飞船上天多少年了,死了几个?怎么偏偏契丹国主碰上了!”

韩冈的辩解怎么会有人信?院中不少人都在摇头。不是韩冈发明了飞船,写出了《浮力追源》,哪有从天上掉下来摔成肉饼的大辽皇帝?这因缘巧合放在他人身上也许还真的是凑巧,但放在发明了牛痘,有遇仙传说的韩冈身上,又有谁会信是单纯的巧合?!

韩冈不理会他们,韩信已经牵了马过来了。龙图阁学士出行时的一众元随也都整装待发。天子在崇政殿中等着,岂能让其久候?

韩冈动身离家,走到照壁前,脚步一停。

“对了。”他拿起绘底样的笔刷,在照壁的左下角一笔连勾画了个图样,对着工匠道,“在这里用红瓷片拼出来……就是这五个角的。”

第四卷,六|四之卷——南国金鼓完。

