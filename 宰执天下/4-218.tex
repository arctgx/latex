\section{第35章 愿随新心养新德(上)}

【前一章的标题错了,应该是第三十四章的‘下’而不是‘中’】

韩冈变脸变得极快,方才还带着微笑,为着行状上出色的词句点头称赞,转眼间,就是脸挂的老长,如同冰雪扫过一般。

但吕大临神色上却不见有半点疑惑和纳闷,沉静如水的面对着韩冈充满怒火的视线,“不知玉昆所言何意?”

“与叔你写的一篇好文,怎么还要问小弟?”韩冈像是听到了很好笑的话,呵呵笑了起来。就是他脸上的笑意,却是阴晦如朔日雨夜,看着就让人心中发寒。

吕大临寓居的是一间不大的僧院,院主听说都转运使韩龙图来了院中,便连忙亲自烹了茶汤来侍候。只是当他端着茶小心的走到吕大临的房门前,乍看见房中韩冈冷至冰点以下的笑容,浑身就猛地一抖,往里面小心迈出的步子,立刻就退了回去。离得房间远远的,老和尚的心口还扑通扑通的跳着,吓得三魂七魄都散了一半去。

养移体、居移气,韩冈久居高位,身为高官显宦,又曾经多次领兵,赏罚皆由己意,千万人的性命曾操纵于掌中,曲折远过常人的经历所锻炼而成的威势,寻常人被他冷冷一瞥,也免不了要胆战心惊,更不用说他现在怒极反笑,眼神中都带了几分狰狞。

吕大临却一点动摇都没有,依然冷静如初,回视而来的眼神看不出任何畏缩。不言不语,等着韩冈的下文。

韩冈心头怒意更盛,声音却又更柔和了几分:“‘尽弃其学而学焉’,与叔,你写这句话时,当真手一点都不抖吗?”

行状中的这一句,说得是嘉佑二年,张载在洛阳设虎皮椅讲易。程颢、程颐夜访,经过一番对易理的深谈之后,张载便撤下了虎皮椅,对来听讲的士人们说道,‘今见二程深明《易》道,吾所不及,汝辈可师之。’

这件事,虽然可算是张载打了一次败仗,但写进行状也没什么大不了的。张载返回横渠之后,卧薪尝胆,重研六经,俯仰而有所得,这才真正创立了气学一脉。

但吕大临竟然在行状中说张载弃了自己之前的学问,而就学于二程。这一句其实是将气学说成了道学的一个分支,韩冈如何能忍——这是要挖关学的根啊!

相对于韩冈的激动,吕大临则是平平静静:“玉昆你追随先生时日太短,嘉佑二年的时候,在下已经在先生身边侍奉多年了。相对于之前所学,嘉佑二年之后,先生所见所识,所传授的一切,全都变了。”

吕大临跟随张载的确很早,才十来岁就跟着兄长吕大忠和吕大钧拜在了张载门下,嘉佑二年他才十八岁,但已经跟在张载身边好些年了。

韩冈自然不能跟吕大临比资历。但吕大临身为张载的,难道不知道,他写的这句话一旦公诸于世,气学在道学面前就别想再抬起头来了。

“本以为与叔为,当能彰显先生一世风标,没想到竟然会有‘尽弃其学而学焉’。若是说得是旧年先生为范文正所劝,回乡攻读《中庸》之事,用上此一句,倒也不为过……”韩冈深呼吸了一下,压住心头火,“可与叔你看看先生的三卷《易说》、十篇《正蒙》、十二卷的《经学理窟》,可有几处与道学相同?”

“皆以六经为本。有所同,有所异。”吕大临回得很强硬。

“好个有所同,有所异。”韩冈瞪视了许久,听到这句话,当真是忍不住火气了:“与叔,你写的好投名状啊!”

吕大临的脸也沉下了来,韩冈的话实在太不客气,甚至诛心:“玉昆你还是先扪心自问再说这句话。程门立雪的,不知是谁人?”

“没错,韩冈的确曾就学于伯淳先生门下,自是要持弟子礼。”韩冈声音顿了一下,声音更为冰寒,“但韩冈所学根本,依然出自张门,归于关学一系。格物之说虽有借鉴于道学,但根基则是从先生虚空即气的源头而来。何曾敢说‘尽弃其学而学焉’,几至肆无忌惮!”

韩冈与吕大临的关系并不算好,但总归是份属同窗,而且他跟吕大忠、吕大防和吕大钧交情匪浅,更是当吕大临是自家人一般。由于吕家兄弟跟随张载最久,行状由吕大临撰写,韩冈事后得知也是点头赞同,并没有提出异议。

可谁又能想到,吕大临竟然直接在行状中给关学捅了一刀子,‘尽弃其学而学焉’,这是什么话,张载是他两个表侄的弟子吗?

“韩玉昆你礼敬先生,难道我吕大临会不如你?!”吕大临火气也上来了,“先生的行状,皆出自我之亲眼所见,只是这些年来所看到的都写下来而已,岂会有一字妄言?!”

“那就请苏季明【苏昞】,范巽之【范育】、还有进伯【吕大忠】、和叔【吕大钧】几位来看一看与叔你的大作好了,看看他们会怎么说?”韩冈低头又看了被他丢到桌面上的行状初稿,冷冷一哼,“这篇文章,我韩冈是不会认的!”

说罢,韩冈便拂袖而出。

作为张载如今地位最高,声望最隆的弟子,只要他不认同,这份行状就是废纸。

吕大临脸色泛白,却紧抿着嘴,也不送一下韩冈,直直的站在房中,一动也不动。

在门外守候的伴当听到里面吵起来后,就退得老远,不敢竖着耳朵乱听。终于看见韩冈出来,便连忙跟上。也不敢多说多问,老老实实的跟在面沉如水的韩冈身后。

韩冈心中一团火在烧,当张载病逝,对于气学会有一个挫折和低落期,韩冈已经有心理准备了。但因为自己的关系,韩冈有信心在几年或是十年后,将气学重新推上。但没想着这个低落期,竟然会导致气学核心弟子的背离。

行状乃是盖棺定论,要为尊者讳,为长者讳,即便张载当真曾经‘尽弃其学而学焉’,也不该明明白白的写出来,总得曲笔,或者是干脆不提。何况张载创立的气学,在根本大义上就与二程的道学截然不同,如何是从二程那里学来的。

而且韩冈即便是为了自己的目标,也要保住气学的根基。

韩冈从来没想过,来自于后世的科学理论与儒学能毫无隔阂的融合起来。但如今正流行的对儒家经典的重新诠释,却是给了他一个再好不过的机会。

经过这么多年,张载也免不了受到韩冈带来的科学理论的影响,将有所抵触的观点加以改变或是干脆摒弃,将之融入在自己的学术理论中。

而二程的道学虽说也为了与韩冈经过实证的一些理论相配合,将他们的观点也有所改变,但改变幅度很小,实际上依然完全无法与科学配合得上。

虽说气学、道学都是用儒家经典为原材料编出来的筐子,但由于释义不同,劈出来的篾条也截然不同,用来承载学术的箩筐自然也不会相同。除非二程能将他们以易学为基础的道学理论加以大幅度的修改,否则来自于后世的科学理论,绝不可能塞进他们的筐子中。相对而言,气学就简单多了。

不过吕大临会转投程门,韩冈也对其中的原因知道个大概,这是关学几乎无法修复的缺陷造成的。

关学的世界观,没法脱离思孟学派的观点,其中一部分在挂在横渠书院中的西铭上,说的已经很明白了。

‘乾称父,坤称母’;‘大君者,吾父母宗子’。从西铭的开头,就将天子和天地对应起来,用自然大道来证明人世间父子君臣这三纲五常的合理性,隐隐有让天子神格化的成分在。

可张载在《正蒙》中又有‘虚空即气’的说法,天地与人无碍,观点又类似于唯物主义。

也即是说,关学的世界观,对自然和社会的看法是严重背离的,有着很明显的破绽。如果凡事都实事求是,将自然大道钻研下去,又怎么可能会相信‘大君者,吾父母宗子;其大臣,宗子之家相也。’这样的话?

但道学就没有这个问题。所以程颢、程颐对名为《钉顽》的西铭赞不绝口,但极少谈论正蒙,便是因为这个原因。

就是没有韩冈的掺和,关学的理论也是自相相悖的——以韩冈粗浅的历史常识,也知道关学在后世根本没有流传下来,其缘由想来多半也因如此——而当韩冈插了一脚进来后,分歧则更为明显。

由于科学理论可以实证的关系,关学中世界观已经将西铭中的观点压缩到很小的地步了,这就让对韩冈的理论始终无法信服的吕大临无所适从,他投到二程门下,也是其来有自。

不过韩冈能理解吕大临的改变,但他无法体谅。作为张载的嫡传弟子,还是张戬的女婿,竟然在行状中如此贬低气学,从情理上还是韩冈为了实现目标的需要上,他都无法忍受。

要分裂就分裂好了,看看如今的气学门墙,在他韩冈的支持下,到底能不能将张载留下的衣钵传承下去。

