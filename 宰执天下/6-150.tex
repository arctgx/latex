\section{第13章 晨奎错落天日近(15)}

如何养活亿万生民?

这的确是个很严重的问题。

但从出生到长大成丁,中间不只要经过多少坎坷,即使没了天花,也还有别的病症能够让幼儿活不到成年。绝不可能出现像韩冈说的那样七年翻番的情况。

几十年后的事,至少也是十几年后,新生代成长起来才会造成危机,需要现在就考虑吗?

类似的话,韩冈之前已经说过了,现在再重复,即使有着新的证据,也很容易让人认为是危言耸听。

而且韩冈也清楚,自己的立论和论证所引用的论据,也稍显薄弱了一点。

“西夏覆亡后,陕西少了十万兵马,千万贯石钱粮的转运减了一半,山中寨堡的工役从此,更不用担心贼人入寇而寝食难安,关中百姓终于得到了休养生息。此事相信参政最为清楚。若是北虏灭亡,河北、河东百姓又会如何?”

“七十年来,河北百姓受了多少苦?”韩冈反问。

“澶渊之盟安在?!”吕嘉问同样反驳回去,“太平长久是天下人心所向,惜乎北虏意不在此。”

他抬头面向御座的方向,大声道:“千里燕山,自太行至渤海,横贯东西,仅有十数隘口可通车马。若有燕山为障,只要有良将十余,精兵数万,则北虏铁骑不足惧也,陛下亦可高枕无忧。省军费,节民力,天下可安。”

“若兵败燕山,天下如何可安?昔年太宗皇帝亦有收垩复故土之念,可结果如何?”韩冈转向太后,“陛下,若是能够一战抵定,天下自是从此太平。可臣之所虑,正是北虏实力犹存,攻取不易。若陕西稍定而河北变乱,烽火连年不绝,北方战乱不休,生民岂得安稳?这又岂是天下人心所向?”

大概是不想让太后觉得是党同伐异,绝大多数新党成员都没有出来与韩冈辩论,只让吕嘉问一人冲杀在前。

现在看起来双方是势均力敌的情况,韩冈没有像过去一般,轻易的便取得绝对的优势。

应该是心有顾忌。苏颂猜想着。

韩冈与吕嘉问的辩论毫无意义,即使太后再偏袒韩冈,也绝不可能现在下定决心改变国是。

朝堂上再如何争论,雄州边界上的辽军才是重点。太后除非打算在辽人即将入侵的情况下失去大半个朝廷,否则绝不会在这个时候,让朝局陷入混乱。

以韩冈过去的表现,殿上辩论时,不能压制住吕嘉问,让王安石妥协,肯定是心中有所顾忌,否则不至于此。

可韩冈的顾忌,吕嘉问那边却没有。反而恨不得穷追猛打,让韩冈低头认输。

听到韩冈的驳斥,吕嘉问又道,“何谓攻取不易?耶律乙辛新近篡位,北虏人心混乱,目下正是北进之时。若待到北虏国中安定,那才时是攻取不易。”

韩冈盯着吕嘉问:“总计能代吕宣徽立下军令状否?若两年之内不能收垩复幽燕故土,便从此辞官归乡,子孙终身不得进用?”

韩冈话音刚落,王安石顿时勃然作色,大喝道:“韩冈!国事岂能置气!”

李定也立刻捧笏出班:“韩冈君前妄言,渎乱朝纲!”

“若受人弹劾时,立誓对赌,当然是置气。”韩冈笑了一下,倒是承认了旧事,辩论到了争执不下的时候,就是看谁更能浑赖,不过当年的对手早已不在朝堂,也没什么大不了的,“但军令状一事,但凡交战,比比皆是。便是攻取一寨一堡,都是以阖家性命为状。而赌上皇宋百年国运的大战,只要以区区官职和子孙仕进之途立下军令状,已经是太优容了。”

“即是事关国运,岂可决于片纸?”章敦叹了一口气,“庙堂上运筹帷幄、群策群力,方能决胜疆场。”

“终究还是不敢。”韩冈毫不客气,“自家连半点风险都不敢冒,却要让太后、天子和天下百姓去冒险,让数十万大军去搏命,不知忠心在何处?仁心在何处?”

韩冈让吕嘉问和王安石代吕惠卿立军令状,两人当然不能这么去做。

吕嘉问冷声道:“如果朝廷全力支持、国中无人干扰,收垩复故土,非是难事。军令状也好,赌誓也好,当然都可以立下。但朝中有人沮坏,这让将帅如何立功于外?立下的军令状岂不是催命符?且疆场上的军令状,是欲让武人舍生忘死,但今日参政所言,却分明是欲置人于死地。”

“总计心虚了。要收垩复幽燕故地,需要多少钱粮,多少甲兵,多少精兵,可以先提出来。”韩冈悠然道,“这样也可以看看,到底是真心敢于立誓,还是在找借口来搪塞。若是国力可以满足,当是真心。若是随口一个亿万之数,那可就是欺君了。”他昂首对太后道,“殿中诸位皆熟悉国事,臣也不能妄言。譬如火炮,若索要千百门火炮送至北方,臣推托不能,便是臣欺君。若吕嘉问相代吕惠卿讨要万门火炮,那可就由不得狡辩了。”

王安石道:“不知朝廷欲拜何人为帅?若以吕惠卿为帅,自当让吕惠卿来说。”

韩冈冷笑,分明是在拖时间了。大战在即,怎么可能调吕惠卿回京?

“倡北进之议,也有平章的份。平章不会不知吧!?”

韩冈一点也不给岳父脸面。本来就只是让吕惠卿赚点功劳回京的手段,说道需要多少钱粮、兵马、兵械,具体的细节问题,他们能仔细去谋划就有鬼了。

“辽师已至城下,如何还奢谈北进?御寇才是当务之急!”曾孝宽出来解围,“而且方才韩参政说辽人屡屡南犯是国是之故,若依韩参政所言,到底该如何改才能让辽人不再南侵?”

“欲阻北虏南侵,最重要的还是国势昌盛,让北虏不敢动念。”

吕嘉问反问:“如今国事不盛?”

“民惟邦本,本固邦宁。天下盛衰在庶民,庶民多则国势盛,庶民寡则国势衰。盖国之有民犹仓廪之有粟、府藏之有财也。昔年先帝与平章所定国是在于富国强兵,平章只说民不加赋而国用足,却不论减赋,非是养民之法。”韩冈提声强调,“为国者,莫急于养民,养民之政,在乎去其害民者尔。”

“何为害民者?!”

“臣只举一例,臣家现有八子一女,而官宦富贵之家,有三四子女者为数众多。至庶民,则生而不育者却比比皆是,如福建路上,多有二子一女之后,所生子女皆溺于水中……”

吕嘉问冷笑,“参政欲言幼子生而不养为害民?”

韩冈瞪大眼睛,惊讶道:“妇孺非人,死可不论乎?!”

这种话题是没办法辩论的,不说太后还坐在上面,就是韩冈没将妇孺并称,吕嘉问也不敢明说小孩子可以随便死。

吕嘉问的辩驳只是一个磕绊,韩冈立刻就说了下去,“安民者,只在温饱二字上。不能让百姓与幼子温饱,岂非害民?”

其实福建不养幼子,还有继承家产上的问题,但章敦等福建人虽然明知此事,却不敢提出来——这是新法教化不利的过错。

王安石出来说道:“若能以燕山为屏障,俭省军费,税赋自然可减,百姓也能得到安宁,且幽燕之地,良田千万,正是养民之地。”

“平章应该没有做过买垩卖。”韩冈微笑着对王安石道,“不过道理是相通的。如果一百贯本钱,不知平章是去做赚十贯而且有三成可能赔掉五十贯本钱的生意,还是去做能够赚上五十贯,即是有一成几率赔本,也只赔上三五贯的生意?

前者即辽国,后者如交趾。于今每年从两广输出的粮食,已经接近两百万石。而各色特产,香料、木料,价值在两千万贯,甚至更多,朝廷在其中得到的各色税入能达到百万贯,这是征南之役、收垩复交州后的两广。而五岭之南,还未开垦的土地仍多不胜数。”

韩冈滔滔不绝,“南海周围小国,如不论瘴疠,更是不缺一年三熟的沃土,一如交州。敢问诸位,夺回燕云之后,朝廷付出的代价不说,得到的土地能与南海周围相提并论否?夺占幽燕,朝廷要付出多少伤亡,才能换得一次两次胜利?一万,两万,还是五万,十万。而平灭那一干小邦,又需要朝廷多少人力、物力?”

话说到这里,韩冈的心意已经是昭然若揭。就是将朝廷的战略重心,从北转向南。对北安抚,对南进取。

“参政欲以南海济中国?”蒲宗孟问道。

“正是。”韩冈转头看了看王安石。

他的岳父紧抿着嘴,神色冷淡。

韩冈不以为意,道:“两年前,河北已让北虏无功而返。如今国势更胜,将他们拒之门外是理所当然。但北进燕蓟,现在远远不是时候。与其去北方冒险攻打强敌,还不如去南方拓土,不仅更容易,即便失败,也不会影响国中。不过……”他顿了一下,目光在群臣的脸上转了一圈,“不过此事非是一人倡议,便可定夺。事关天下,当以太后、宰辅与卿大夫共定。”

又来了!

章敦就知道韩冈最后会来这一垩手。

殿上争论,能驳倒对方的本来就不多。

韩冈现在不是要太后下定决心,仅仅是扩大议论的范围,把有资格参加廷推的重臣都拉进来,让所有人一起来决定是否改变国是。对太后来说,下定决心并不难。

而且也不是对国是大变动,并非否定新法,只是暂时不要北进,而是如交州之例,去开拓南方。

有资格与会的朝臣都不介意使用一下自己的权力,以体现他们的存在感。

但王安石绝不会答应。韩冈的一番言辞,也根本不可能说服王安石,即便能驳倒也并无意义。但韩冈攻击由王安石订立的国是,意味着他与王安石彻底决裂,也意味着被国是压制住的旧党,终于看到了压在头上的大山有了土崩瓦解的迹象。

当韩冈开始举起战旗,还敢趟浑水的会有多少,想要从中牟利的又会有多少?

又是兴风作浪!
