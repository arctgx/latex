\section{第29章 顿尘回首望天阙(一)}

【第二更,求红票,收藏】

韩冈跟着那名公吏,行走在楼阁之间的廊道中。擦身而过的官吏,许多人手上捧着一卷卷的公文,都是脚步匆匆,以着近乎小跑的步子,无暇旁顾,仿佛有人拿鞭子在后面赶着他们。

中书省的楼阁还是那副破破烂烂的样子,比起外面的酒楼要差了许多。韩冈前次上京,虽然没有进来参观,但从门前经过去流内铨时,他不禁为宰执们的艰苦朴素而惊叹不已。如果有外人来到这里,应当很难想象,这就是当今世界最为繁荣的一个帝国的行政中枢所在。

天子平常要修宫室,一般都会被朝臣们骂上一通。不过官员们就没必要由此顾虑,天子就算说些酸溜溜的话,谁也不会放在心上。但修好后自己享受不上,也便没人愿意多事。说起来,只有胥吏在会在一个衙门中待上几年、十几年,甚至一辈子,相信他们应该想有更为舒适的工作场所。只不过,不会有人去征求他们的意见。

韩冈被人领着,走了大约有半刻钟。最终抵达的并不是最后面的主殿,而是隔邻的一栋人来人往的偏阁。走到这里,韩冈心中也有了些数。所以当他被带到章惇面前时,并没有感到任何诧异。

“韩冈拜见检正。”

“玉昆,别来无恙。”

章惇如今担任的检正中书五房公事,如果拿后世的职位来比较,应该算是国务院办公厅主任……在吏、礼、户、兵、工、刑六部全都成了摆设的情况下,章惇眼下的职位,应当更为重要一点。只是他的官品还是不够高,依然是绿袍,并没有能像王韶一样被特赐五品服色。不管怎么说,章惇现在所做的工作,只能用位卑权重四个字来形容。

与韩冈相见,章惇表现得很亲热,寒暄了两句便拉着他平坐下。让人送上茶水,斥退了厅中人众,摆出了要长谈的架势。

韩冈看着阁外小院中,忙得恨不得长出四条腿的胥吏们,也不避忌的直言问道:“检正,你这一职事务繁芜,千头万绪,是怎么有闲坐下来喝茶的?”

章惇笑道:“玉昆你是白担心了。不妨事的,我这里的事自有下面的吏员和各房检正官去做!”

韩冈皱起眉头,为章惇担心起来:“万一有人见检正你行事阔达,升起了不轨之心,又该如何是好?”

“以我的手段,自不会让他们有机会作出不轨之举!”章惇对韩冈的担心毫不在意,他抬头很自负的说道:“大凡役人者,授其法而观其成,苟不如法,自有刑律候着!使人可尽其才,吾当为之。底下的琐事,便由他们去做。吾只需做一监察,又何须事必亲躬?当然能有空喝茶闲谈。”

章惇的一番话,让韩冈有会于心。他赞道:“如果在下说检正疏其小节,执于大略,乃是宰相气度,不知算不算奉承?”

章惇闻言,顿时放声大笑,“玉昆之赞,吾当仁不让。宰衡国事,吾之所欲,也是迟早之事!”

章惇丝毫不掩饰他的野心,韩冈也不免要佩服他的自信。宰相一职,开国以来,也不过几十人坐上去过。就算是一榜状元,能做到宰相的,也不多见。乃是人臣的巅峰,不是那么容易爬得上去的。韩冈虽也是有心于此,但现在还做不到章惇这般能放声豪言,这其中,并不仅仅是性格上的差别。

又说笑两句,章惇终于跟韩冈谈起正事。他收起了笑容,正色对韩冈道:“其实今次中书发文招玉昆你上京,主要还是天子想见玉昆你。你在过去立下的那些功劳就不提了,天下间,弱冠之年便有如此功绩的也就玉昆你一人。你的名字,早已让天子记下。现今连韩子华都上表要用你,官家当然想见你一见。”

章惇说到这里,停了一下。看了看韩冈,却见这位年轻人仍是一副从容淡定的微笑,不见任何情绪上的波动。章惇不由得有几分佩服起韩冈宠辱不惊的气度来。换作其他官员,听说天子一直看重自己,赶着要召见,怕都是要涕泪横流、激动不已了。

又喝了口茶,斟酌了一下言辞,章惇方才道:“不过玉昆你昨夜在王相公那里,把话说岔了。对横山的事指手画脚作甚,冷水也不是你该泼的。”

“事关国事,不能欺瞒。”韩冈很坚定的摇了摇头,但又很坦陈的说道,“不过这也是下官不想离开秦州的缘故。”韩冈自知他的一点小心思,毕竟瞒不过明眼人,还是直言为上。

“我知道因为有王子纯【王韶】在,加上你在河湟的心血,所以才不想离开秦州。可你要看看是谁对你说话!王介甫!韩子华!两名宰相都要你去延州,你还推搪什么?!让你去延州,就去好了,把疗养院办起,将伤兵们照管好,其他的事何须你操心?功劳不会少你的,有过不会摊到你身上。你以为天子和王相公对你的看重是句空话吗?即便横山那边,最后结果真如你所说,也不过连带着吃点排头,最多降一官,转眼就会升回来,甚至能超迁一官补偿玉昆你!何必把话说得那么绝?”

章惇近乎推心置腹的一番话,让韩冈有些感动,但他并没有半点后悔,他相信自己的决定和判断——韩绛必败无疑——只要这一点确定,不论王安石现在怎么想,只要最终横山战略宣告失败,那么最后的胜利必然是他韩冈的。

“不如此,不足以证明下官对横山战事的看法!”

章惇深深盯了神色坚毅的韩冈一眼。无奈的摇起头,叹起气来:“现在说什么都迟了。王相公已经发了狠,延州,玉昆你还是要去;功劳则是你自己不要的,日后就不会算给你;还有觐见天子一事,也一起没有了。”

韩冈的脸色这下终于有点变化了。人心当真难以预料,韩冈的确是没想到王安石竟然还会耍小孩脾气。现在王安石硬是要他去延州,加之韩绛的两本奏章还在天子案头上,两名宰相一齐用力,这个任命想推掉都难了。

“其实王相公虽然有些火气,倒也没真的阻拦官家召见玉昆你。昨天三更时,官家还特意遣了内侍到中书来。说是要中书候着,等你入京,就即刻安排你越次入对。”章惇抬眼看了看韩冈,又叹着:“不过当值的冯当世给挡回去了!”

韩冈将提起的心放了下来。夜半传谕,这实在太过了一点。这已经不是受宠若惊的问题了,要是没被冯京挡回去,那自家可就是成了众矢之的。御史台里的台官们,说不定就要盯着他韩玉昆,也好来完成每月的功课了。

可以算是逃过一劫,韩冈倒也有着一点感谢冯京的意思:“冯大参之刚直,着实令人敬佩!”

“刚直?”章惇不屑一笑,不只是针对韩冈的话,更是为了他对冯京的评价,“对上天子的时候,自然人人都会刚直。不过一点小事违了天子之意,难道官家还能降罪他这个执政不成?没后果的事,谁还会怕?平时的冯当世,可不是这副模样。玉昆你也是出自陕西,难道不知他的那个匪号?”

韩冈抿起了嘴,想笑。冯京的那个见不得人的匪号,他转在嘴边,倒也没有刻薄的说出来。金毛鼠相貌出色,但可就人品堪虞。在京兆府任上,贪得城中商家鸡飞狗跳——这也难怪他,商人出身,对钱财的确是看重了点。说起刚直,能让俞龙珂和瞎药都求着要赐姓包的包拯包孝肃可以算,冯京可就远远不够资格。

一名吏员这时在院外叫了一声,等章惇招了手后,就匆匆上厅来,把他手上的一份公文交给章惇,“延州军中急报,还请检正查收。”

章惇接下了,看了眼火漆的完好程度,便点头应了。

吏员匆匆离开,韩冈看着他的背影,心中的疑问越来越高涨,“军情为何不递到枢密院,怎么送到中书来了?”

“因为陕西、河东宣抚司是由韩子华亲领,天下间没有宰相要向枢密院报备的道理。别的都能让,但权位之别,却容不得一点他人沾染。延州的文字都是先发回中书,再由中书依照事宜缓急,决定是呈交天子,还是转给枢密院。”章惇弄开火漆,随手翻了翻,招了远在院中守候的小吏过来:“抄写之后,转交西府。”

见章惇脸色变得沉重起来,韩冈心中有些打鼓,小心翼翼地问道:“究竟出了何事?”

“绥德城中,两万大军已然点集,箭在弦上,随时便会引弓而发。”章惇完全没有对军事情报保密的念头,看到刚刚自关西而来的韩冈,也不觉得有必要向其隐瞒刚刚收到的情报。“对了,玉昆,还没问你为何对横山一事这么不看好?光是罗列出一些困难,应当不至于让你望而生畏!”

