\section{第16章 晚来谁复鸣鞭梢(上)}

四月初八,佛诞日。

每年的这个时候,东京城中,大小百十寺院都要举行浴佛斋会。寺院内外总是人满为患。熙熙攘攘,热闹非凡,进香的,做买卖的,人山人海,拥挤在每一间寺庙中。

而京城诸多寺院里,敇建大相国寺最是热闹。每月五次的万姓烧香,都是数万人云集的大集市。而到了新年、上元,或是今天这样的佛诞日,更是一大清早合都士庶妇女骈集于此,四方携老扶幼交观盛会。人人斋戒茹素,等着浴佛会后求浴佛水来饮漱。

但偶尔也会有些年份,大相国寺虽然依然观者如堵,但却没有人多时都少不了的喧闹。就如今年一般,刚刚结束没多久的琼林宴还在京城百姓们的津津乐道之中,但四月初八佛诞日之后,人们议论的中心,则变成了大相国寺的浴佛之会。

一辆辆装饰华美的车辆,停放在大相国寺正门前的广场上,簇拥着一辆装饰最为华美,形制最为高大,却以布幔为帐,以至于四处漏风的马车。而在马车旁,还有一匹高大健壮却又给人一种轻盈之感的龙驹静静的站立着。淡金sè的皮毛犹如最上等的锦缎一般在阳光下闪闪发亮,同样sè泽的鬃毛被jīng心打理成一缕缕的小辫。这是一匹应在天河边奔驰的骏马,竟然落于人世。

此外又有上千名金甲金盔,手持各sè御器的士兵围着数千亩大小的古刹,绕了一圈又一圈。原本在广场和回廊中盘下铺面做买卖的摊贩,连同他们的货物,全都不见了身影。而大相国寺外的等候斋会举行的数万百姓,则还有许多仍旧跪伏于地,久久的都没有抬起头来。他们拜的不是佛祖,而是当今的天子——赵顼。

竟是天子驾临大相国寺。

自东汉明帝遣使求法,在洛阳建立白马寺后。经过了上千年或激烈或温和的改造和被改造,这个来自于西域的教派,已经彻底的融入了中国。供奉大小佛主菩萨的寺庙遍及天下,朝廷也要设立僧录司,发放度牒,来统管天下僧侣。

但皇帝亲自出宫礼佛还是很罕见的一件事。外人只当是天子为去年刚刚病逝的太皇太后祈求冥福。之前就有天子拿了私房钱让大相国寺的僧侣为太皇太后和皇太后点长明灯一事在京中广为流传。不过赵顼身边亲近人等都知道,天子之所以这么急着来大相国寺上相,却不仅仅是为了太皇太后,更还有皇嗣的问题。

韩缜随着班列慢慢的一步步的走着,陪着皇帝,走进了大雄宝殿。身为两府中的一员,与天子朝夕相对,耐心和稳重都不会缺乏。对天子的心思把握甚深。

前几天,宫里面一个怀孕的嫔妃不意小产,落下来的还是个男胎。这对正困于皇嗣稀少的赵顼来说,不啻又是当头一棒。原本是准备让宰相来上香的活动,却变成了天子御驾亲临。

王珪紧随在赵顼的侧后方。从熙宁三年入政事堂,十年间宰执如走马灯一般打着转,但只有王珪一人屹立不倒。就是去年官军惨败灵州城下时,市井中谣言蜂起,都说王珪在政事堂中的日子已经不长了。但世事无常,往往柳暗花明、峰回路转,最终西夏还是灭国。纵然未尽全功,但也不能算是失败。首倡的王珪,依然稳稳当当的做他的宰相。

大相国寺的正殿平日并不开放,就是正殿前的三门,也是非得天子诏令,才能为此打开。但天子的身份毕竟不同,一道道大门在赵顼面前敞开,

在佛像前,拿着一束线香躬了躬身,便让旁边的内侍将香火插到佛像前的香炉中。

太祖皇帝当年入大相国寺礼佛,曾经问寺内僧人,他见了佛祖到底要拜,还是不需要拜。这个问题很难回答,既不能让天子心头不痛快,也不能让人觉得自己礼佛之心并不虔诚。当时的住持犹豫着不知如何回答方好,幸亏旁边有个小和尚帮忙接口,说现在佛不拜过去佛,很是巧妙的帮住持解了围。自此以后,朝廷对大相国寺的封赏日渐丰厚,而天子入寺不拜的旧例也有此延续了下来,

拜过佛祖,赵顼也没打算就此回宫,找过来一名时常出宫探问的内侍,“药王庙就在附近?”

王珪和一众执政眼皮一跳,怎么要去药王庙?但那名内侍去完全没有多想,老老实实回答:“回官家的话,的确就隔了两条街。”

“嗯,时间还早。”赵顼望着殿外,也不知是对谁说话,“去药王庙走一趟。”

内侍惊得跳起:“陛下。预定中没去药王庙这一项!这时候,往药王庙去,可是要兴师动众。”

“多走几步路也无妨。”赵顼坚持自己的意见,一定要去药王庙中走上一遭。

“哪里是去拜药王,根本是去拜韩冈。”薛向在韩缜身边喃喃自诩。

“韩冈?”韩缜听在耳中,漫不经心的随口问道:“拜他做什么?”

薛向笑道,“不信玉汝兄会猜度不到。韩冈与如今七个儿子一个女儿,到现在都没有夭折一个。这个药王弟子,可是让天子羡慕得不得了。玉汝家的兄弟人数甚众,但排行绝不可能仅止于第八。却也比不上韩冈。”

韩缜微微一笑。韩亿八子,人人显宦,韩绛、韩缜,更是一为宰相,一为参政。但韩缜的兄弟,却也不是仅仅只有八人,照样有几个中途夭折的。王韶也以儿女多著称,还是只有七八个儿子存活下来。与南方的穷人家不同,绝大多数的官宦人家,都不会因为养不活儿女,而将婴儿溺死。养不活的原因除了意外,就是各种各样的疾病。

韩冈的子女年岁尚幼,按说谁也不能保证说他们日后不出事。可是他是孙真人的弟子,传授世人牛痘免疫法。身上的神秘sè彩怎么也抹不干净。就算韩冈本人不肯承认,就算他向所有人讲述牛痘术的原理。但事到临头,为了能保佑皇嗣的安康,为了能让天子多子多福,让他入京,当然是顺理成章。

“看到天子不去调理身体,而是从求神拜佛上入手。韩冈恐怕要气得头疼。”韩缜闲闲说也着,听在耳中,却有几分幸灾乐祸的味道。

但薛向却道:“见庙就拜,本就是寻常事。运气若是不差,天下神佛,总能碰到一个愿多事的。让韩冈出来护翼皇子,说不定就有用。”

韩缜遥遥头:“韩玉昆若在,肯定会直言谏阻,怎么也可不能让自己给人当土偶。”

“也不能怪世人。自从牛痘术大行于世,天下的药王庙,如今香火日渐旺盛,各州各县,都赶着趟的重

修或新建药王庙。这一番sāo动岂是无因?”

说话间,天子已经从后门出了大相国寺的正殿,韩缜、薛向和其他几名宰执,紧紧的追随在他的身后。

护卫天子到大相国寺的班直已经收到天子的要求,行动起来的她们在短短时间中,已经做到了断路、清场,让药王庙中没有可疑的闲杂人等。

随天子走进药王庙。新修的前后殿阁,琉璃瓦灼灼生光,分明是新近重修过的。正殿中的神像,也不是过去的扁鹊,而是一身道袍、留着三缕长须的道人。

天子站在正殿中,抬头望着神台上一坐一立两座神像,韩缜等人则在殿外,对药王庙的里外簇新啧啧称奇。

薛向道:“旧时供奉的药王,中原为神农、扁鹊,河北为邳彤、华佗,南方还多个张仲景,就是陕西,除了孙真人外,还有一个韦慈藏,同是唐人,他也是药王。可现如今,坐在正殿里的基本上都是孙思邈。神农还好说,孙真人抢不去他的位子,但扁鹊、邳彤、华佗、韦慈藏,那就得委屈到偏殿蹲着。而且在孙真人的金身边,还总有一名青年士子侍立——这当然也是有缘由的。”

韩缜撇嘴笑了笑,这座药王庙的神台上,当然也有那年轻的侍者在孙真人的身边站着。

薛向坦率直言:“说实话,孙真人这些年接连得到朝廷加封,从唐太宗封的妙应真人,到慈济妙应真人,再到慈济医灵显圣妙应真人。过两日更是要改封真君——慈济医灵显圣守道妙应真君。名号越来越长,声威一日.比一日高,这其中的功劳,全都是他传说中的私淑弟子给他带来的。”

韩缜垂着眼帘听着薛向的评论,偶尔点两下头附和,“说得正是。”待到薛向话声一顿,他便接口道,“天子这一次来药王庙,多还是做给人看的。让人提议将韩冈招入京中。”

“没错,肯定是这个原因。”薛向点头,“韩冈好兴事,在白马,有束水攻沙之议,在京西,又建江汉漕渠之策,到了河东,大战本都是该结束了,还硬是出兵,从辽人嘴里将胜州抢了回来。不论是让他继续留在太原,还是将他调到南方,多半会弄出什么事来,还不如招他入京。也能分一分他的心!”