\section{第11章 飞雷喧野传声教(五)}

在火器局的新试验场走马观花的看了一圈,韩冈回到京城时已经是黄昏。

尽管试验场位置有些远,但视察的结果让韩冈还是比较满意。

在方兴的主持下,这个试验场虽然简陋,却已经很好的开始运作。虽说有问题的地方的确还很多,不过在韩冈看来,再多的问题也不能掩盖试验场的价值。

这座试验场,可以说是军器设计、生产正规化的第一步。

军器监日后的设计和生产,不能像神臂弓、斩马刀一样,在皇帝面前耍了一套之后,就得到了配发军中的许可证。尽管韩冈也曾是受益者,但他始终认为,比起来自高层的许可,专业的测试是更加不可缺少的部分,这样才能够不断积累和进步。

火炮试验场的建设才刚刚开始,现在只是第一处。在计划中,主要是虎蹲炮的试验地,日后在臼炮和火枪开发出来之后,也可以于此处进行试验。不过等到那时候,运输用的专门轨道就必须提上台面,不能将火药火炮这些机密之物,用马车在官道上装来运去。

而射程更远的野战炮、城防炮,则需要更大的试验场地,否则射程等数据根本无法进行统计。动辄数里的射程,才一里见方的试验场完全约束不了。射角、装药量、炮弹重量与射程之间的换算关系,必须进行大规模的测试才能得到,不可能让炮手凭着个人经验来玩。

只是开封府附近短时间内还没办法找到更大的空地来试验。中原平陆就是有这样那样的问题,绝大多数土地都有了主人,就算有旧军营可以利用,可周围人口太多,还是无法保证实验武器数据的隐秘。短时间内,只能暂时使用这边的试验场,瞄准土山上进行射击。

回到京城,王居卿早一步下马告辞,韩冈准备去一趟政事堂看看情况便回家。

韩绛早一步就回家了。这位宰相一向不到散衙便回家,太后也罢,御史也罢,都睁一只眼闭一只眼,不去管他。张璪比起韩绛也不过小了十岁多一点,却是学不来韩绛的大牌,每天是按时来按时去。不过当韩冈回来的时候,却也正好准备回家,都走到了正院中了。

看到韩冈回来,张璪就停住了脚,跟韩冈寒暄起来:“玉昆,情况怎么样?”

两位参政站在正院中说话,跟在张璪身边的人,跟在韩冈身边的人,呼的一下散到了两三丈开外,以两人为中心,空出了一圈地来。还有政事堂中奔走的官吏,也都远远地绕过了路去。

韩冈左右看了看,摇头对张璪道:“我们这是雄黄吗?”

张璪哈哈笑道:“艾草也差不离。”

“也算他们晓事。”韩冈道:“虎蹲炮的情况还不错。一年千余门不成问题。”

“又便宜又好,就跟玉昆你的板甲一样了。”

“得等到弹药的成本降下来才能当得起价廉物美四个字。”

“好像也不比箭矢更贵……”

“现在差不多,之后还能更便宜一点。”

神臂弓训练耗费的是箭矢,重弩本身也容易损坏,而火炮好歹是铜铁金属,好歹要比弓弩要结实些。火药、弹丸,现在加起来比箭矢要贵一点,但东西一多,成本立刻就会降下来。

张璪多少知道一点,,“尔禄尔俸,民脂民膏’,这军器花费也同样是民脂民膏,价廉物美才为最好。”

“不过如何配合弓弩列阵发射,下面就要等训练了。不好生训练,再好的军器也是废铁,白白浪费了钱。”

“虎蹲炮,张璪也看过。比神臂弓、床子弩都易用,看几眼就会了,还不像床子弩和神臂弓那样费力气。”张璪与韩冈面对面的坐下来,“其实照张璪说,都用不到五个人,两三个人就够了。”

“还是人多些安稳点。行军时,轮流背着火炮不损气力。”

“原来如此。”张璪点点头,突然又道,“对了,之前宫里面几次派人出来请玉昆你。”

“韩冈出外前,应该已经通报过了吧。”韩冈奇怪的问道,“是什么事?”

“杨戬没说。”张璪神情淡淡的说道。既然中使不说,从臣子的角度,也不方便问。他正想再说什么,突然扬了扬眉,“人来了。”

过来的就是杨戬,或许是一直在听着消息,得知韩冈终于回来了,就忙小跑着过来。

“参政终于回来了。”杨戬给张璪、韩冈行过礼,就急不可耐的对韩冈道,“太后有旨,请参政速至内东门小殿。”

韩冈没有立刻领旨,而是先问道,“什么事?”

“太后想了解一下火器局的情况,让小人在这里等着参政。”

按照分管项目,军器监和将作监的管辖权,在韩冈手中。韩绛和张璪都不会插手。

但如果太后想要了解军器监的发展情况,派个中使去不就行了?满皇城的阉人,随便挑几个都比外臣更靠谱。

韩冈能从对面的张璪脸上,读出他想说的话。

一般而论,太后更相信外臣而不是宫人,对朝臣们来说并非是坏事,甚至可以说是好事——只要不是专信某个人。

“臣遵旨。”

‘某个人’领旨之后,又与张璪打了个招呼,随即从侧门往内东门小殿而去。

韩冈一向知道,向太后很看重火炮。

神臂弓、斩马刀,乃至板甲、飞船,都是先帝赵顼看好,并配发军中的军器,只有火炮,是在太后执掌国政后才出现。

而且论起声势和威力来,火炮远远超过其他兵器,非刀枪弓弩所能比。

当初金明池试射,湖水上方,火炮的轰鸣声惊天动地,而摧毁目标的能力和重新发射的速度更是远远超过了床子弩。

事后太后兴奋的心情,从她的话语中就能听得出来。

韩冈去了城外试验场一趟,回来就要问详情,并不让人感到惊奇。

当韩冈来到内东门小殿,都已是皇城快要落锁的时候了,但太后没有半点急躁。

赐了韩冈的座,赐了韩冈的茶,方才道:“参政今日辛苦了。”

“此乃臣份内之事,亦只是半日来回而已。”

宰辅为国之鼎鼐,素不轻动。实地视察是韩冈的习惯,而不是宰辅们的惯例。

不过韩冈,而以火炮的价值,和太后对火炮的看重,也的确当得起一名参知政事,花上一天时间去试验场视察一下。

“不知情况如何?”

“一切顺利。军器监、火器局的奏报并无虚言,若无意外,虎蹲炮年内千门非是难事。”

“如此甚好,火器局上下当重赏。”

“陛下,还请完成之后再赏不迟。是赏是罚,得看结果再说。”

“吾知道了。”屏风后先叹了一口气,然后又振奋起来,“参政,吾前日听曹诵说火器局有人献上了一种能发八寸炮弹的火炮?”

军器监的两位判监,一位是过世的慈圣光献曹后侄儿的曹诵。在京百司的主官,每天都会有两人上殿奏报本司事务,前几天正好是军器监奏报的日子,由曹诵入殿。

“的确,不过仅只是图纸。以火器局现有的水平,想造出来有些难。”

“吾看参政当初所进火炮,不是有一种炮口还要大的吗?”

“臣所进的乃是臼炮,形如石臼,口大而身短。而局中近日所进,却是与野战炮、城防炮一个形制。臣因其炮弹如石榴般圆滑,故而名为榴弹炮。野战炮、城防炮都是榴弹炮的一种。所用榴弹的直径每增加一倍,重量就要增加到原来的八倍,四寸榴弹便有十余斤,而八寸的榴弹更是达到了近百斤。”

现在野战炮的口径是四寸,城防炮则是六寸,再大的话,就是七寸、八寸。

武器当然是越粗越长越有威慑力。一人拿着匕首,一人举着斩马刀,给人的感觉就完全不同。而炮口酒杯大小与海碗大小,也同样不是一个等级。

要是先用普通的四寸炮,将城墙轰的土石横飞,再让十几头牛拖着口径跟缸一般大小的重炮上来,包管城中的守军立刻缴械投降。韩冈也曾经向人描述过这样的画面,但那水缸一样的重炮终究只是臼炮,不需要太远的射程,也不需要太长的炮身,同时对炮管强度的要求也不高,仅仅是口径大而已。

如野战炮、城防炮这类榴弹炮体系的火炮,想要将口径造得更大,其自身重量差不多要有万斤了。

火炮若以万斤为标准,根本就没有前例。不论是青铜也好,黑铁也好,以最常见的铸器——钟鼎来说,千斤是最多的,三五千斤就已经凤毛麟角了,最大的铁鼎才八千余斤,这还是之前为了庆贺新天子登基,并对辽胜利,以那尊殷墟方鼎为蓝本,又因虚荣心而加以放大,方才铸造出来的。

而且炮车的主体部分,也是用模铸法造出来,重量与炮身相当,这样才能承受开炮之后的反冲。用最简单的说法,就是多重的炮身就需要多重的炮车。

听了韩冈的一番解释,太后冷静了下来,“也就是造不出来?”

“很难。而且造出来后,又太过沉重,只能放在城中。不过一旦成功造出,经此磨砺,火器局中一众大匠的技术当又能更上一层楼,”

那样的火炮,现在肯定造不出好货来。可韩冈明白,有些钱尽管肯定是浪费,却也该花。技术储备,不论成功还是失败,都能够积累,而且在研发的过程中,也有可能得到一些惊喜。

“那样的话,还是不要吝啬这份钱为是。”

“臣明白,请陛下放心。”
