\section{第17章 往来城府志不移(十)}

用了一个晚上的时间,韩冈将之前已经写就的草稿修改了一番,以札子的形式进呈给了赵顼,条陈编修药典,并以厚生司的名义设立医馆,医治在京军民两事。

对于韩冈的申请,赵顼那边自不用说,御笔一挥,便批复给了政事堂。而政事堂也没有耽搁半点,十分顺利的就让韩冈拿到了几位宰辅的签名。

而且赵顼为此还特地责成翰林院的医官们,让他们全数听从太医局和厚生司的调遣。只是以韩冈如今在医学界的声望,基本上每一名御医,都没胆子跟他的吩咐顶着来。

钱乙在翰林医官中算是很大牌了,是天下有名的小儿科名医,一部《小儿药证直决》,乃是如今儿科诊疗的圭臬。六皇子的健康问题,都是由他来日常看护。

但韩冈说是要在医馆中设立小儿科专科门诊,让他选派门下弟子参与,同时,还让他在医馆成立之后,每隔十天就抽出一天来参与医馆的工作,对此钱乙都没有二话就答应了下来,而且还毕恭毕敬的递了帖子请求韩冈接见。

且不说医馆少不了他这个小儿科专家,就是自家儿女日后若是生病,也还要靠他来诊治,没有慢待的道理,韩冈很快的就出面接见了他。

一般来说,能学医的都少不了攻读诗书。不为良相便为良医的说法,也让许多在儒学上无法成就的读书人,转投医术的天地。钱乙这个名字虽然跟那一干没名字只有排行的平民百姓一样,但出现在韩冈面前的这位五十上下的儿科权威,完全就是一名温文有礼的儒者。

钱乙是翰林医官,同时在太医局中也有一份差事。作为下属来拜见韩冈,他的表现比起一干卑躬屈膝的官僚,更让韩冈欣赏。

太医局有教学的任务,其下分为九科,相当于内科的大方脉科,儿科的小方脉科,外科的疮肿兼伤折科,专治风疾的风科,连妇科一起包括在内的产科,看名字就知道其治疗手段和范围的口齿及咽喉科、眼科、针灸科,最后,还有使用祝由术、几近于巫术的金镞及书禁科——这是用符水来医治病人的专科,虽然韩冈对这一科嗤之以鼻,但有些病症,比如心病,用符水往往比针灸药石更管用。

就在去年,朝廷还准备成立第十科——免疫科,但给时任判厚生司的安焘顶回去了。在他看来,免疫学是厚生司的禁脔,如何能让太医局插手进来?此事遂不了了之,但韩冈准备在太医局中成立此科。

太医九科中的每一科,都设有一名教授,下面教着一班弟子。钱乙便是小方脉科的教授。疮肿兼伤折科的教授则是韩冈的旧识雷简——他在西北军中多年,最近才调回来。

基本上,太医局就是一座将教学、研究和医疗融为一体的综合性医学机构。从制度上说,已经有了后世的医学院及其附属医院的雏形。韩冈如今要做的,不过是让其规模更大,更加贴近后世而已。让普通百姓也能享受到皇亲贵戚才能得到的医疗,同时也将御医和他们的学生们的医术水平加以磨练。

对于韩冈的设想,钱乙没有半分难色,而且是十分的欢迎:“旧岁钱乙尚在郓州时,官宦寒门无分高下,皆医治如一。可自从入了京,上门问诊率为公侯子弟,不见寒门素户。在乡间,医治的多是疑难杂症,但到了京中,则都是头疼脑热。太医局生更是历练不足,多有学医数载,却不辨脉象的。”

韩冈对此很是赞赏:“博学之,审问之,慎思之,明辨之,笃行之。天下的学问皆是如此,医术何能例外?光读书不能叫做博学,学医的不多练一练手,如何能出师?”

从钱乙开始,韩冈逐一接见过几名在太医局中有教授职位的翰林医官,向他们交待自己的构想。并根据几位专家的意见,对初期的构想加以修订。

医馆的制度,经过韩冈与医官们多日的交流已经确定得差不多了,而与此同时,医馆的位置和设施,韩冈也都一一安排妥当。他与开封府和枢密院两家商议过,就是将京城中原有的四座疗养院加以改建扩建,由此设立面向所有京城百姓的医馆。

将大体的框架搭建出来,剩下的琐事,当然就不需要韩冈来做了。

厚生司中的吴衍是韩冈的老朋友,也是韩冈的恩人。在厚生司创立时,被韩冈推荐给王珪,在司中做判官,算是司中老人。

两年来,他的官位没有变,远比上比不上当初同做判官的蔡京在官场上的顺畅。不过也让他能够顺利的扎根在厚生司中。韩冈想要以厚生司和太医局主导成立医馆,自然就选了吴衍作为助手。

吴衍是官场老人,向韩冈问的第一个问题就是钱,“敢问玉昆,这医馆是由朝廷拨款,还是要靠诊金来维持?教授、医师和医生们的俸禄又怎么算?”

“当然是要靠诊金和药费来维持。至于俸禄,在太医局中挂名,本来就有,不需要另外给付。不过出门诊一次,当计人头,另给贴职钱。”

吴衍又问道:“教授、医师,和太医局生,诊金是不是应当不一样?”

“自是当然。”

韩冈很清楚,光靠善心,任何事业都是不可能长久的。收养弃婴,安置鳏寡,安葬无名死者,从法度上说,官府都有责任。但这种只管花钱的政府福利制度,早已是名存实亡。朝廷的钱粮划拨得本来就少,加上一干贪官污吏,哪里还有实际的效果?真要说起来,寺院都比官府做得好。

朝廷曾有规定,各县每逢夏日,每个月都有两百贯药钱,用来向百姓散发避暑药。但除了很少的一部分官员还能记得这件事,绝大多数不是没有划拨,就是划拨了之后给人贪了去。

恩泽百姓不是不好,但有名无实,朝廷损失钱粮,百姓得不到实惠,最多一两年就会名存实亡,就毫无意义。韩冈是现实主义者,不会一听到福利就高潮迭起,要想善政能让朝廷维持下去,而不被日后各种各样的借口裁撤,就必须有一定程度上的盈利能力,至少要能做到不会亏本。

吴衍是老官僚了,在中下层的官场混迹多年,朝廷的善政到了下面,多半会成为地方官吏渔利的手段,这一点哪有不清楚的?所以直接了当就发问。

得到了韩冈的回答,吴衍心中有了底。不过韩冈还是提醒他,若是急症,还是得以救人为先,收钱得放在后面。

让吴衍主管医馆组建,而医生们也都安排好了人选,等到四处疗养院改建完成,就可以正式接待京城百姓了。

将医馆之事丢给吴衍,剩下的就是医典的编纂了。

说到医药之学,大宋立国以来就十分重视。别的不说,太宗皇帝就是最爱玩毒药的,御药院里不知藏了多少毒药方子。毒与药向来不分家,这医药从太宗皇帝开始,也一直重视有加,士大夫少有不研习医术的。

仁宗的时候,朝廷更是成立了校正医书局,将古代流传下来的各色医书,一一加以校正、修订、出版。

医家最重要的几部医经——《黄帝内经》,《难经》,《神农本草经》的整理和考订,都是校正医书局的功劳。如《伤寒论》、《金匮要略》、《脉经》、《诸病源候论》、《千金要方》、《千金翼方》、《外台秘要》,这一干医书,校正医书局也都一一加以编修。

不过校正医书局是个临时性的机构,要修书时往里面塞人,不修书时就裁撤,当前两年将孙思邈的《千金要方》和《千金翼方》修订完毕后,便没有动静了。

说起来韩冈前些年因为疗养院而声名鹊起的时候,也有让他参与修订孙思邈医书著作的提案,毕竟传言说他是孙思邈的私淑弟子,但韩冈基本上都在外担任实职,不可能回京中任差编书,这个提议也就不了了之。

现如今,赵顼既然将编纂药典的差事交给韩冈,理所当然的,韩冈是必然的主编,而辅佐的助手,就要从校正医书局的旧人中找寻。

林亿、高保衡,都是曾经在校正医书局中做事的官员,虽不是医生,但皆是精通医术,对医书也知之颇深。韩冈第一个就点了两人作为自己的助手。但还有一人,是韩冈想要的,却有些难处。

“其实要不是苏子容名位已高,他其实是最合适的。”韩冈与章敦喝酒时还提起此事,“他前些年在校正医书局参与修订《神农本草经》,又编写《本草图经》,说起医药,韩冈是瞠乎其后。”

韩冈很想借助苏颂在医药上的才干,但苏颂的地位不低,资历又老,官阶上跟韩冈相差无几。调他来做编纂药典的副手,乃是屈尊,朝廷那边也很难通过这项提名。何况韩冈出言批评《本草》,对苏颂也有些不好意思。

章敦倒觉得韩冈的顾虑太多了,“只要苏子容自己愿意,朝廷又岂会拦着?玉昆你何不写信问一问苏子容,相信他也是愿意回京的。”

