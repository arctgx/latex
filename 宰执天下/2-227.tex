\section{第34章 山云迢递若有闻(八)}

当进攻渭源堡营地的蕃军开始撤退的,刘源等人也绕了回来。队伍中少了几人,但他们后面的追兵,似乎比刚开始追击时也稀疏了一些。

看到了眼前的战局,追兵稍稍犹豫,便停止了追击。

刘源回头,阴狠的一笑,也随之放慢了马速,反手就是一箭射去。由他领头,这一队重新骑上战马的骑兵,也纷纷开弓搭箭,将一支支利箭射向身后的敌军。

挑衅失败了,几名愤怒的敌骑,被领头的蕃将给拦住。但刘源他们也射中了几人,其中一名蕃贼不幸被命中要害,捂着被长箭贯穿到眼眶,一头栽下马去。

刘源哈哈大笑,得意的收起了长弓,在士卒们崇慕的视线里,回到了营地中。

奔逃、射敌,飞驰在死亡线上。在待罪半年之后,刘源重又回到了战场。好战的血液在血管中疯狂流淌,听着弓弦鸣响,看着敌人在自己的手上变成一摊死肉,他发现自己还是喜欢这样充满了血腥和刺激的生活。

尽管折了两个兄弟,但能死在沙场上,总归是件好事。在面朝黄土背朝天的日子过了小半年之后,刘源更是进一步的确定了这一点。

“辛苦了。”

王中正随着韩冈下城。看到韩冈好言抚慰归来的将校,也免不了跟着一起道了声辛苦。

平平淡淡的几个字,换作是从前,谁会放在心上?都是盼着朝廷的封赏。但一场大波折之后,刘源以下却对韩冈、王中正的一点善意,心有所感,甚至有些难以自持。

让刘源他们找个地方坐下来休息,又命人赶紧给所有上阵的将士,赶快送上他方才下令烹煮的热腾腾的肉汤。

贼军撤退后要想重新整军再杀奔上来,不是转眼就能做到的,肯定要休息一下,这让他至少多了半个时辰的时间,可以着手加强防备,顺便让守军稍事休息。

韩冈望了一眼一里之外,细小如同虫豸的敌骑,不知这些蕃人能坚持到什么程度。但他韩冈会给他们一个难忘回忆——

保证!

肯定!

…………………………

“想不到宋人还有这样的利器!”瞎吴叱远望着宋军营寨中,那几具高高挑起的霹雳砲,脸色难看已极。

那几架砲车,远远的看过去,就像是有些奇形怪状的望楼。架设在营门附近,让人根本不会放在心上。谁能想到会是宋人使用的防守武器?!刚才没有多加防备,冲上去的骑兵都拥挤在营寨大门附近的那一段,正好受到了最强的攻击。

结吴延征熟视良久,半晌之后,才说道,“……这些当是宋人的行砲车,听说是用来攻城。看起来就很沉,少说也有几百上千斤重,怕是不好转动吧?”

结吴延征的话,让瞎吴叱眼睛随之一亮,一下被点醒,双手一拍,“对啊!这么大的物件,肯定也不好打造。看角度,只是守着营门附近。只要攻打的时候换个位置,就能避过去了!”

瞎吴叱自问有了克制霹雳砲的手段,脸色便好上了不少。

只是方才的一番攻城伤亡不算小,他们本是气势汹汹而来,在奔出了上百里之后,一点。现在兵锋受挫,必须修整一下。

“先歇息片刻!一个时辰后,给我全力破城!”瞎吴叱下着令。

但他也算有些军事经验,明白不能让守城的宋人有休息的时间。随即点起一队人马,让他们绕着渭源堡的营寨打转,向营中抛射箭矢,让守军难以得到休息的时间。

蹄声奔烈,急速绕行的吐蕃骑兵,让霹雳砲难以下手。

“这算什么?又不是对付野外列阵?”

可对于吐蕃人骚扰营中的伎俩,甚至不用韩冈下令,就是王中正都知道该怎么对付。

“把毒烟火球拿出来。”

毒烟火球是记载在《武经总要》中的。以涂了沥青的纸和麻布外壳,内里填充巴豆、砒霜、焰硝、硫磺、草乌等引火和毒物。这毒烟火球在使用前要先戳出洞,然后将之点燃,最后再用投石车投出去。是最简易的化学武器。

不过今天,释放毒烟火球时,并没有使用投石车。暮秋初冬,正刮着北风,风向向南,被滚出营地外的几十个毒烟火球,燃烧着,滋滋冒着黄色的浓烟。将营地南面外侧掩盖在波浪滚滚的毒烟之中。

宋人释放毒烟火球的时机选得恰到好处,让这一队派来骚扰的吐蕃人猝不及防,一下就冲进了烟幕之中。

毒烟火球中里面掺了巴豆和砒霜,毒烟呛人,更呛马。人能主动摒住呼吸,但战马做不到。从烟雾中迅快的一穿而过,骑手最多是咳嗽流泪,但下面的战马却齐齐的打起了喷嚏,团团转着,像是受到了极大的刺激。

接下里,就是箭矢如雨,得意的,将这一队淹没在狂风暴雨之中,只有聊聊半数逃脱。

瞎吴叱勃然做色,这一队骑手,是他最为亲信的一支,方才都没舍得让他们上阵,孰料还是吃了大亏。

看着一个个红着鼻子和眼睛的亲信,瞎吴叱心都在滴血,他等不下去了。一等黄烟消散,他就起身下令。号角声中,千军万马再次奔驰而起,避开营门处,而选取远离霹雳砲的那几段营栅攻打。

眼见蕃贼避实就虚,韩冈却得意的在笑。他只用了六具霹雳砲,就把营门连同附近一长段的营栅给守住了,这是何等的轻松。而且敌军不敢堵在营门外还有个好处……就是可以出击。

在瞎吴叱和结吴延征吃惊的眼神中,渭源堡南面的营门中开。两百多宋军从门中涌出,在营门前结阵,用弩箭射击着远处蜂拥在营栅几处角落的敌骑。

这是一个机会!每一位蕃骑都看到了这一点。

巨大的诱惑,让他们忽略了一切可疑之处。只要冲破这道单薄的敌阵,便能冲进敌营,而不是在守军的箭矢中,用人命拆除营栅。只要冲得快一点,那两具砲车应该排不上用场。

没有坐等瞎吴叱的命令,几名蕃将同时下令,调转马头直奔营寨大门而来,

的确,区区两具霹雳砲并不足以抵挡吐蕃骑兵们的奔驰,而单薄的宋军阵列也阻拦不了他们的冲击。

在勇猛的吐蕃战士面前,宋军纷纷退让开去,可是敞开的营寨大门之后,却并不是一片坦途。正对着大门处,是数架由三条弓臂和粗重的弓弦所组成的战具:

八牛弩!

用着大型绞盘上好了蕴力千钧的弓弦。在弩槽上,三支黑沉沉的铁枪还带着锈迹。长约五尺,粗如儿臂的铁枪却与一尺长短的箭矢同一个性质。而且还是六具,十八支铁枪并排着。

先是三具齐射,接着,又是三具联发。

前九支,后九支,一支支铁枪,在空中化作一道道黑色的雷光,穿透马身,掠过人体,连续洞穿多人,带起一蓬蓬血雨。

这是开战以来最为凄惨的一幕,数十名冲在最前面的吐蕃勇士,不论他们的武艺有多么的高强,不论他们的性格有多么的武勇,在坚硬的铁枪面前,如同纸一般脆弱。

突如其来的打击,让吐蕃人晕头转向,看着血淋淋的一幕,一下失去了战意。

这时候,刘源领着三十名骑兵,带着一群旧日的将校,反冲而出,以猛虎下山之势杀入了敌阵之中。长枪、铁简、骨朵,诸般兵器一齐上阵,在呆滞的敌群中肆意杀戮。

加大了配重的霹雳砲开始向远处投射,连同神臂弓手们一起,将后续的敌骑阻拦在数十步外。逼得他们只能眼睁睁地看着营门前的自家兄弟,在一众疯狂的广锐将校们手中,变成尸体和战绩。

刘源双目皆赤,如同恶鬼一般挥舞着长枪。枪尖刺穿了一名有一名蕃骑的胸膛,从他们的胸口标出的血箭,让刘源更加疯狂。突然眼前一空,敌军再无踪迹。回头再看,冲杀到营门前的上百敌骑,竟然已经给他和他的同袍杀了个干干净净,而骑兵的数量已经扩大到六十多人。

‘不能再打了。’

第二次用号角将前线的部众召回,瞎吴叱和结吴延征对视一眼,对方的脸上都是露出了同样的神情。

渭源堡的守敌的确如他们所料,兵力十分空虚。但他们的战力,却出乎意料的强悍。

当偷袭变成了强攻,而强攻又变成了屡攻不克,再留在渭源堡下,情况只会越来越坏。攻下渭源的机会不是没有,但瞎吴叱和结吴延征

地盘很重要的,但手上兵将更是关键。手上有人,还能抢地盘,而人没了,得到地盘也别想保住。

存地失人,人地皆失,存人失地,人地皆得。

这样的道理,在弱肉强食的河湟地区生活了几十年的瞎吴叱和结吴延征,早已通过切身体会了解得透彻。淡薄的赞普血脉并不能保证他们的地位,只有手上的兵马部众,才是保证手中权力的一切。

瞎吴叱再盯了暮色下的渭源堡一眼。在微光下的深色剪影,如同一只匍匐在渭水源头的巨兽,散发着危险的气息。

趁着夜色,说不定还有一星半点的机会,但他已经无心再赌上一把。

“退兵!”瞎吴叱颓然下令

“想走?!”片刻之后,韩冈却是一声冷喝,“哪有那么容易!”

