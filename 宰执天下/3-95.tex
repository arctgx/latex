\section{第28章 临乱心难齐(十)}

天色将晚,曾布方才回到家中。

书房已经点了灯,将袖袋中的几封文函掏出来,就一下坐到了书桌前。书房应该是日日打扫,但曾布一坐下来,就发现桌上有着薄薄的一层灰。手指一抹就是一道印痕。

曾布顿时脸色就变了,拍着桌子大怒道:“今天是谁当值?连桌案都不知道要擦一下!”

“官人,怎么这么大的火气?”曾布的夫人魏玩正好走了进来。在外界人称魏夫人的她,乃是如今有名的诗人词家。她的作品,纵使是文章如曾巩之辈看了,也都是要赞其文采过人。

曾布对自己的这位夫人是又敬又爱,听得她如此问,顿时就收敛了火气,摇了摇头,挥手示意被他的声音惊得跑进来的婢女出去。

魏玩走到曾布身边,为他到了杯热茶,坐下来轻声问着:“可是朝堂上又有什么事了?”

曾布也不瞒着魏玩,他们夫妇感情也甚好:“还有能什么事,前面王相公用了,要开汴口放水,还要用碓冰船来碎冰开河道。现在又改了旧策,准备用雪橇车来运粮了。”

“难道那个雪橇车会比侯水部的碓冰船更管用?!”魏玩惊讶的问着。碓冰船听着虽不靠谱,可侯叔献再怎么说都是朝中首屈一指的水利专家,难道还会有比他更有用的方案。

“说有用也有用。听说那雪橇本来就是熙河路用来在冬天大雪封道时交通消息所用,乃是韩冈所发明,用了格物学的知识。几年来的确堪用,但熙河路上奏后却不知怎么没人在意,送去了架阁库中,如今才又翻出来。所以吕吉甫密奏天子,准备与侯叔献的碓冰船同时试用。”说到这里,曾布又冷笑一声,“只是说是这么说,实际上还是王元泽连夜跑去了白马县,从韩玉昆那里得到了图样和指点,这才将旧卷宗给翻出来的。现在正准备着明修栈道、暗渡陈仓呢。”

魏玩能诗能文,冰雪聪明,丈夫一说,顿时就明白了王安石是准备明着用碓冰船,暗地里则是用已经得到验证的什么雪橇车,这样多半就能让粮商们猝不及防,使得如今兴风作浪的罪魁祸首将本钱都给陪掉。只是明白归明白,魏玩却是摇着头,很是不屑:“堂堂宰相,用此鬼蜮伎俩,未免小家子气了点!”

“天子已经移居偏殿,日常御膳也减了。但这天还是一日旱过一日。都快腊月了,黄河都给冻透了底,但京畿和河北还是一点雪都没有,两浙那边也没有雨。”曾布摇头叹息,感慨着王安石的策略连妇人都看不过眼,“转眼就要大难临头,王相公如今已经是慌不择路,当然抓到一根稻草就当作救命绳,自然什么招数都给用上了。”

“难道相公觉得王相公用这等招数情有可原?!”

“怎么可能?”曾布摇了摇头,“堂堂宰相,竟然将粮商视为大敌。不能举重若轻的泰山压顶,却要千般算计,想想也真是有失朝廷体面。”

“那官人怎么不劝上一劝?王相公好歹也是于官人有恩呐!”魏玩嗔道,对丈夫的态度有些不满了。

“怎么没劝?!”曾布急着为自己辩解,“但也要他肯听啊!王元泽一力主张,韩玉昆推波助澜,那个吕吉甫又是全力支持,剩下的几个全都是唯唯诺诺,我一个人反对又有什么用?”

魏玩摇着头。她虽是女子,却一向心气极高。就算不在文学上,也是照样看轻天下英豪,自问绝不会输于男儿。王安石父子如今的策略,实在是难以入她的眼界。

‘这样也好。’曾布心中则是冷笑着,王安石父子昏招迭出,吕惠卿却不加以劝谏,这样的人如何会是自己的对手?如果是暗藏祸心,那就更好,那份鬼蜮之心怎么都瞒不过人的,迟早会拆穿。

无论如何,新党第二人的位置,曾布绝不会让给吕惠卿。

眼下的情况是明摆着的,以朝廷如今的开销,新法绝不可能废除。朝廷的收入倍于英宗之时,但开支同样也是加倍。如果新法一切尽废,韩琦、富弼、文彦博这一干元老重臣上台,

可是目前的大灾不能不处理,为了能给天下臣民一个交代,只有让王安石辞相一条路可以走。现在王雱虽然准备要从南方运粮入京来打压粮价,稳定政局。可在曾布看来,此举即便有用,也不过是苟延残喘而已。拖上两三个月,王安石的相位依然还是保不住。

看看韩冈,他给王安石父子出了主意——而且是成功率极高、本有明证的方法——但他却根本不肯站出来参与其中,依然做他的白马知县,明摆就是不看好最后的结果。曾布不喜欢韩冈,但这位才二十二岁,就已经爬到自己三十五岁才走到的位置上的年轻人,其能力和眼光不需要怀疑。

其实从今年上元节时的宣德门之变中,天子赵顼对整件事的处理,其实就能看得出王安石的圣眷已经大不如前。现在拖了一年,差不多已经到极限了。如今的大旱对于相位不再稳固的王介甫来说,乃是百上加斤,不论做什么,其实都没有挽回的余地。

而王安石一旦去职,为了能维护新法的稳定,天子必然要从王安石的几名助手中提拔一人进入政事堂中。

新党如今人数虽众,可真正算得上是核心的,也就四人:吕惠卿、章惇、曾布他自己,另外还要加上一个王雱。如曾孝宽、吕嘉问之辈,离着核心还有一段距离。

王雱作为宰相之子,连侍制还没有做到,完全没有机会。章惇这两年多在荆湖平定蛮夷,准备走的是由边帅至枢密院,再从枢密院至政事堂的那条路,可以说是已经暂时放弃了对新党次席位置的争夺。

真正能与自己一争高下的,就只有吕惠卿一人。

论文采、论才智、论治术,曾布绝不会认为自己会输给吕惠卿。

就是从家世上,南丰曾家也稳稳压着晋江吕家。曾家一门三代出了十九个进士,通过几代联姻,与如今大族世家都能拉上关系。就算是富弼、韩琦这等元老,绕个两层也照样能攀上去。更别说王安石,他的弟弟王安国可是自己的亲姐夫。

可是从一开始,吕惠卿就死死的压在自己的头上。变法之初,不论是商议新法的条款,还是职位的升迁,福建子总比自己要早上一步。

好不容易等到吕惠卿因母丧而丁忧回乡,近三年的时间,曾布便跃居,仅在王安石之下。最多的时候,他身上一口气担了十几个差遣,一时风光无限。

只是等到吕吉甫从福建老家回来,情况又发生了变化。

明明是自己孤身支撑了新法推行中最为艰苦的那一段时光。王安石乃是一国宰相,独掌大略,不暇细务。具体的事务全是他曾子宣来主持。没有自己一番心血操劳,哪还有新法顺利推行的今天?!

吕惠卿倒好,新法出台时他掺上一脚,中间的辛苦全都避过,现在回来却想方设法的要压着自己。天子和王安石,也并不介意将自己手上的权力分给吕惠卿。

而吕惠卿与自家并没有着同僚之谊。原本吕惠卿所定的助役法,自己为了能推行顺利,将之改名为免役法,同时又修订了其中几处不合情理的条贯,整件事全凭公心在做。吕惠卿倒好,竟然给记恨上了,顶了自己中书检正的位置,没几天便将自己定下的几条制度全都给改了。

这样的对手,曾布怎么都不会让他压在自家头上。现在他曾子宣已经是翰林学士,离着只有一步之遥。加之薛向眼下就要去宿州,他身上的职位又要自己来兼管。官位水涨船高,看看吕吉甫,还来不来得及在接下来的几个月时间里追赶上来,只要慢上一步,先行进入政事堂的必然是他曾布。

曾布头靠着交椅的椅背,双眼盯着房梁,忽然又开口道:“薛向过两天就要回去掌管六路发运司了,他的三司使之位虽然还留着,但他在宿州肯定管不了衙门里的事。”

魏玩一听,登时吃了一惊。丈夫的话中之意她哪还能不明白,瞪大眼睛,问道:“官人可是要执掌三司了?”

曾布的头点了点,“预定的是同判三司。薛向不回来,朝中财计之事必然得有人承担。”他回头看看妻子,只见魏玩双眉蹙着,“怎么,不高兴我任此职?”

“官人能受天子和相公看重,当然是好事。”魏玩却是心疼丈夫,另外她对于曾布一忙起来就时常日以继夜的作风,也是有那么一点怨怼,“但三司使一职,妾身素闻最为繁剧,官人的判司农寺难道还要兼着?”

“现在还要暂兼一阵,过些时候就要让贤了。”曾布忽而冷笑:“不过他身上还有军器监和检正中书五房公事两个差遣,怎么都轮不到他头上。”

魏玩自是知道曾布嘴里的‘他’是谁,也知道丈夫对那人的心结。并不多话,悄步走到曾布身后,一双素手熟练的为丈夫揉捏着肩膊。

曾布很欣慰,家有贤妻总是让人能如此舒心。闭着眼睛,头后仰着,在熟悉的体香中,渐渐便沉沉睡去。

