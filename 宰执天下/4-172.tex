\section{第26章 鸿信飞报犹觉迟(四)}

【年底了,事情多,上班时只能抽空写写,所以拖到现在。不过说好三更不会少,晚上还有一更。】

自从前日拜见了韩冈回来,钦州疍民的两位大首领就陷入了苦恼和烦闷之中,都是茶饭不思,夜不能寐。

转运韩相公嘴里说着自愿,但自家只要回个不愿,下场还不知会怎样。但放人去交州,他们更不愿意,这几千部曲可都是家当啊,

但到了最后,还是保守家业的心思占了上风。再怎么说,前一次见面,那位年轻的转运相公并没有喊打喊杀,说话也多是和和气气的,让两人有了点侥幸的心思。

“等小韩相公当真要下狠手时,再服软也不迟。没得看着天上飘起一两朵云,就收网回澳避风浪去。”

“说得也是。三十岁不到就做了转运相公了,在广西也肯定呆不久,拖个一年半载,他肯定就回京去了,到时候,还能想着我们没听他的话。拖!拖下去!”

计议已定,两人也就不再犹豫。

武福、俞亭二人,并不打算将韩冈的命令正面顶回去,但韩冈既然故作大方的说是愿意去就去,不愿去拉倒,以他们的权威,让部族之中无人去应募却也不是难事。

自然,他们不会强逼。只是没有老老实实的将韩冈开出的优厚条件转达下去,而是给出的待遇说是说了,但官府为什么会这么做,在两人的嘴里则变成了官府因为交州病死的人口太多,要从疍民中凑人数,那些好处也多是幌子。

这一番扭曲过的传言,引得疍民们人人惊惧,之前两人唉声叹气、茶饭不思的样子正巧也成了证据。

转天钦州在韩冈的命令下,开始在疍民经常泊船的地点挂出了招募屯丁去交州的旗子,便是一个人也没有来。

俞亭和武福不给韩冈面子,这一事做出来,也是在提醒吊胆,等着韩冈下一步会怎么做。

开始的几天,钦州城中的反应也只是派了胥吏来向疍民们宣讲朝廷的德政和去交州屯田的好处,只是有了两位首领下药在前,自然是毫无效果,无人肯信。

但等到第十天,情况陡然变了。一艘艘战船从东面的海上驶来,高高的船帮,是远洋船只的证明,耸立如城池的战舰在一艘艘如同蛋壳一般脆弱的小舟之前驶过,直往钦州港而去。

总共三十多艘战船,都是两广海上特有的广船,为了抵抗南海强劲的风浪,关键部位都是用铁力木打造,比起福船、沙船都要结实许多。

这些战船进港时,张旗击鼓吹号,声浪遍传海上,惊得两名疍人首领魂飞胆丧。都在心里嘀咕,难道就为了这等小事,出动了大军不成。

只是他俩很快就放心下来。从船上下来的水手,有三成多是疍民,却是广东疍民中的一支——号为卢停。为疍民中唯一一支善于水战的部族。

两边虽然隔了几百上千里,但毕竟同为疍人,很快就混熟了。武福、俞亭也从卢停疍人口中得知,这一支水师不过是移防来此驻泊罢了,并不是转运韩相公招来清剿他们的官军。

两人一开始还松了一口气,觉得自己想得是太多了。武福如释重负的说着,“我就说嘛,我们只是没有听转运相公的话罢了,官府最多派几个衙役来,怎么可能调兵来!”

“对!对!没错!没错。”俞亭用力点头,给自家壮胆,哈哈笑着说道,“若是衙役来提人过堂,我那是半点不惧,正好能坐实我们说的话,下面的儿郎谁还会再怀疑?”

武福也悠悠点着头。若当真如此,到时候,他们威望必能再上一层。只需逃出去一年半载,等那位麻烦的韩运使转调他处,自家就能回来如过去的几十年那般,继续在族中称王称霸下去。

“拿酒来!”放下心来的俞亭大声喊着,“我今儿要和武大哥好生喝上一通。”

“如果要出外躲一阵,可不一定能喝到这么好的酒了。”武福举起酒碗与俞亭用力碰了一下,“今天当一醉方休!”

但出乎两人的意料,问题并不是出在官府。钦州官府根本没把他们当做一回事,并没有派什么衙役来,出手的是跟他们打成一片的卢停部的疍人水手。

也就半个月的时间,韩冈当初对他们说的话,在部族中原原本本的传扬开来。不少疍民水手拍着胸脯说转运相公是个一言九鼎的人,再体恤下人不过,哪里会骗人去交州抵数?!

尽管绝大多数人还是选择相信他们的首领,但已经有人觉得碰一碰运气也不是坏事。再差的生活,也不会比如今在裤腰上拴着脑袋,日日潜入深海更恶劣,万一运气好,当真如同传言一般,那就是能有块土地,安安生生的好好过下半辈子了。

这一下子,俞亭和武福坐不住了。谁能想到韩冈会用上这等釜底抽薪的手段?一旦被证明自己说了谎,人心立马就能散掉,到时候,谁还会听他们两人的。

“不能让他们走!”武福狠狠的叫着,通过船舱上的舷窗,能看到有几艘小船上,站着几个汉子,正向着周围的亲友辞行,“万一去了交州,一两年后得了好处回来,就更拦不住了。”

“照我说,一不做二不休,干脆……”俞亭铁青着脸,用手刀往下挥了一下。

武福心领神会,阴森森的应声道:“就这么办!”他透过舷窗对外狠厉的狞笑着,“今天晚上我就带人就把他们沉到海底去喂鱼虎。”

“不,”俞亭摇头,“现在杀了他们,肯定会被怀疑。等到他们去钦州报了名,肯定还是要回来乘船走,那时再动手,也没人知道他们是死了还是失踪,最后要怀疑也只会怀疑官府头上!”

……………………

赵顼拿着广西转运司发来的奏折,里面的内容让他看了不由自主的点着头,挺满意韩冈的成果。

韩冈也算是勇于有为,为了充实交州的人口,将手伸到了海上的疍民身上。从奏章上看,这其中还是受到了不小的阻力,两名钦州疍民的首领,因为谋害族人,被拘入牢中。

那些疍民已经先一步报名成了屯丁,在衙门中入了籍簿,谋害他们,就是铁打的死罪!这一桩案子应该已经送入了京城,在大理寺和审刑院中走流程,一待判罪确认无误,就是该勾决了。

两个首领为了一己之利,竭力阻止族人去交趾屯田,甚至不惜杀人。这件事曝光出来,倒是帮了官府大忙,一下子就多了七八百户出来,钦州的疍民几乎都走光了。

赵顼也知道,若事情只是如此,肯定会有人说韩冈是生事。万一临近州县的疍民首领兔死狐悲,起兵作乱,到时候罪名全都是韩冈的。不过钦州现在将两名疍民首领在交趾破城的时候,趁火打劫的罪行也都同时给翻了出来,也不会有人蠢到站在他们一边说话。

但话说回来,如果连廉州的疍民也一并去了交州,恐怕合浦南珠日后就少见了。

赵顼笑了笑,将奏章写了两句勉励褒奖的批语放了起来,他对此并不是很在意。采珠之苦,在韩冈的奏报上也说了许多,赵顼并不是为了自己的喜好,而不顾子民痛苦的皇帝。比起珍珠,稳定的南疆边州才是他所期盼见到的宝物。

光是在奏章上,当然看不出韩冈的布置,只能看到他想让天子看到的。

在安南行营还为组建的时候,广东都监杨从先奉旨在广州组建水军,本人也担任安南行营的战棹都监,但安南行营本部在交趾速胜,尚未成军的水师根本都没来得及派上用场。到最后,官军都已经攻破了升龙府,开始兴建海门港,连周围的岛屿都已经借着从交趾人手上夺来的船只,派兵去清剿了一遍,他们才慢悠悠去了富良江口绕了一圈。

韩冈不是经略使——即便是经略使,在没有枢密院的命令下——也无权调动水师作战。他只不过是借用了一下水师船队进港的时间而已。

本来水师组建在广东、钦州是在广西,没有枢密院的命令,水师不得越界。不过当初为了方便起见,杨从先水师的驻泊地是钦州而不是广州,到了安南行营解散,也没有说将这支水师调回广州去,而杨从先本人,也是调任了广西都监。

当然,若不是韩冈在安南行营解散前,在请功名单上将水师加了进去,并密奏天子,说这一干水军虽无甚功劳,但好歹也有点苦劳,而且日后守护交州还用得到他们,不能失了军心,让朝廷因此颁下恩赏,疍民水手也不会帮着他拍胸脯作保。尽管其中的内情,疍人水手什么都不知道,却也并不妨碍他们为韩冈说话。

而知悉内情的杨从先,不论从恩德上,还是从地位上,又或是为自家利益打算,都不会违抗韩冈的命令。而且能顺手帮韩冈一个忙,这份人情日后可是多少钱都换不来的。

俞亭、武福的反应皆在意料之中,最后的结果也让他满意,只是还是死了人让韩冈觉得有些遗憾。

交州的人丁多了,自然就更加稳固。分到了土地的疍民是汉人的身份,他们在上岸后只有依附于官府,是日后用来制衡蛮部的重要一环。

当交州的汉人超过万户,这一个边州也就稳定的掌握在了大宋的手中。等到蛮部的种植园经济发展起来,交州也就彻底回到了中国的版图上。

