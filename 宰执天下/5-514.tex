\section{第39章 欲雨还晴咨明辅(43)}

“君忧臣劳,君辱臣死。堂堂天子,竟要与北虏论亲,若是在真宗时,国势逊于辽国,是败于夫差的勾践,不得已而为之,尚且说得过去。但之后不卧薪尝胆,反而从此高卧,以为天下太平,终至元昊之叛。景德、天圣诸公岂得辞其咎?一曰有辽寇在,我等在东京城中,永远都是不能安寝的。”

蔡确摇头,话是慷慨激昂,但此时又不是大庆殿上,何必说这些糊弄皇帝的话?

“是为万世开太平吧。”

一颗功名之心,谁能没有?横渠四句教早已传遍天下,韩冈的目标到底是什么,难道还会有人不知道吗?

韩冈也只是说得顺口而已,蔡确是明白人,也不东拉西扯,将话挑明了:“耶律乙辛年事已高,未免子孙遭难,他十年之内势必要篡位。”

“要废掉澶渊之盟和元丰新约?”

“澶渊之盟,真宗皇帝与辽圣宗约为兄弟,以辽承天皇后为叔母。兄弟之约延续至今,可不是与穷迭剌的儿子订约。”

迭剌再差也是契丹部族中有名号的高官,可比灌园子要有家底得多。蔡确双眉轻轻一挑,“也就是说,从现在开始,就要做好北攻辽国,收复幽云的准备?”

“谁都会机会,但只有准备好的人才能抓住。”韩冈微微轻笑,眼前的这一位可是最擅投机的,一直投机进了东府,做了宰相,“要时刻准备着。”

“玉昆。这话传出去,可是要天下大乱的。”蔡确语气郑重了起来。

韩冈巴不得他和蔡确的对话被宣扬出去,战略上的威胁,必须对方自己也明白有这回事才行。

“难道耶律乙辛还会指望皇宋养他多少年?”

“如果他肯降顺的话,朝廷倒是不会介意在京城为他立个辽东郡王的宅子。”

蔡确笑着说道。这时他突然发现,方兴指挥着人手收拾了残局之后,又开始做起了发射火炮的准备。重新捆扎绳索,又一只活蹦乱跳的山羊,只是这一回离得炮口近了许多,只有十步出头。

他惊讶的问韩冈:“又要做什么?”

“准备另一种炮弹的实验。”韩冈解释了一句,又问蔡确,“相公可知契丹骑兵与官军对阵之后,会怎么作战?”

“不知。”蔡确摇头,他就算知道一点,也不会在韩冈这位方家面前显摆,“请玉昆赐教。”

“契丹骑兵与我官军临阵对垒时,都不会直接发动全军向官军军阵上撞上去,而是会一波一波的冲击。基本上都是在一百五十步外开始集结——战马的冲击力也就在这么长的距离上,再长了,马匹就回不过气来了——其开始冲阵,如果官军阵型不散,便会在三十步的位置上减速,二十步到十步之间转向,在阵前横过同时向阵中射击。就这样一轮轮的过来,直到官军的军阵支撑不住为止。”

蔡确点头,能如此了解契丹骑兵的战术,在朝臣之中,韩冈应该算是第一号了。

他听韩冈继续说:“所以契丹骑兵最脆弱的时候,便是从阵前横过的那一段时间,但也仅仅是眨几下眼的功夫。而且为了减缓契丹骑兵的冲击速度,大部分弓弩都会在五十步到七十步的时候就射出去。”

“也就是来不及射第二轮了。”

“的确是这样。所以为了解决这个问题,就有了弓手分三排站立,一排.射击,一排等待,一排上弦上箭,轮番施射,名为三段射。后又有了上弦器,可以让弩弓来得及发射第二次。这些都是为了缩短上弦时间,能更大程度上打击辽军。”

“嗯。”蔡确又点着头。这是战场上的战术指挥,基本上没有接触过,可韩冈的用意也不难理解,“火炮可是能够代替弩弓,在近距离射击?”

“换一种炮弹就可以。”

韩冈让方兴拿来一颗拇指大的弹丸,比之前的铁炮弹要小得多。蔡确拿在手中颠了一颠,有些沉手,是金属质地,可颜色也不像是铁或铜。

“是铅吗?”蔡确问。

“相公好眼力。”韩冈恭维道。

“铅、汞有毒。玉昆,自从你的文章出来后,市面上的铅粉都快没人要了。”蔡确说着就将铅弹交给身后的随从,“现在用铅比铁都便宜。”

“铅弹的威力比铁制的要大,只是太重了,远距离还是要靠铁弹,不过近距离射击用铅弹就没问题了。”

“但这未免也太小了吧。还是说要一次射出去许多?”

跟聪明人说话就是轻松,韩冈点头:“是一次发射许多铅弹出去,故而名为霰弹。”

“霰?”蔡确皱眉想了一下,问道:“‘如彼雨雪,先集维霰’的霰?”

“正是。”韩冈点头,辨识诗经里面出现的字,对儒者来说只是基本功,“其实就是常说的稷雪。”

“福建那边叫做米雪。”蔡确再看了眼要往炮口里填的一颗颗铅子,以及第二只倒霉的山羊:“霰弹……倒是贴切得很。就不知道威力如何了。”

回到草袋后,依然少不了火光和巨响,之前已经经历了一次,可蔡确还是感觉很不习惯,耳朵有些嗡嗡响。

待硝烟散尽,十步外,又是一片血红。

可怜的山羊被铅弹丸打得浑身是洞,汩汩的流着鲜血,比方才的一幕更要惨烈。

蔡确指着那一只羊,张口结舌:“换成是辽人……”

“也会如此。而且会更快!”

如果是熟手,清理炮膛,装火药、炮弹,引线,然后点燃发射,比起给八牛弩要快一点。几个人一起做的话还会更加几分速度。

蔡确不是傻瓜,他当然看得出火炮要怎么运用在对辽的战场上。

一百五十步外直接打击辽军骑兵的集结地。若是给其冲到了近处,换上霰弹,十步到二十步之内,一门炮能抵得上几十名弩手。

韩冈从战略说到战术,对辽国的方略也出来了。不仅是攻辽,防辽也同样有了预备。

他也是真的想要将辽国彻底解决,才从现在开始就做准备。

只要登州水师成型,就可以压制辽国海岸线附近的寨堡和驻军。当宋军随时可以出动战船,运送兵马,夺取榆关【山海关】,封锁住连接东京道与南京道、位于海山之间的那一条狭窄通道,那么辽国就算想要进攻大宋,也必须随时提防身后的危险。

有几个人会选择冒险?战胜于庙堂之上。这才是宰辅掌控全局的意义所在。

蔡确沉吟良久,突然道:“方兴是个有能力的。”

蔡确对方兴也不会不熟悉,畿县的知县,比起外路的知州都要热门。开封、祥符两赤县的知县,很多时候更是要天子同意才能任命。

“的确。有他相助,韩冈在白马县的一年,过得可是轻松得很。”

“玉昆太自谦了。”蔡确笑道。

韩冈摇头,制造松木炮不难,但几天之内,便指挥上手,做事有章法,这样的官员的确不好找。

“御史台盯着方兴实在是不成话。”蔡确忽而又说道。

“难免的。也是受了韩冈牵累。”

就是在如今,党同伐异的情况也从来没有消失过。

蔡确一边望着正在拆卸绳索的工匠们,一边道,“御史台最近的折子,都是些鸡毛蒜皮的小事。一个个失了锐气。换作昔曰,这辱台钱早就罚到可以去樊楼夜夜笙歌了。”

韩冈停下了动作,等着蔡确的下文。

“有些监察御史,不宜再留任台院。至于那些侍御史,殿中侍御史,也要给他们一个机会,从乌台中走出来。”

有件事蔡确提都不提一句,但周围有人明白了,“举荐谁来做。”

“可以慢慢来,等之后的安排。”

举荐御史,依故例,是御史中丞、侍御史知杂事和翰林学士三方推荐,两府不得干预——御史台的存在,在本朝,就是为了牵制宰辅。但现在的情况不一样了,宰辅们就多了些想法。之前换过一茬,现在要换第二茬。

蔡确既然无意追究其他枝节问题,韩冈也不在乎其他了。对蔡确道,“蔡元长才具卓异,曾任厚生司判官,若他能主掌厚生司,必然是一个好结果。”

蔡京已经做到了殿中侍御史,只是他想要升御史中丞难度极大,几乎不可能,就是御史台副——侍御史知杂事也没有什么机会。

蔡京什么时候走了韩冈的路子?蔡确心中一下就警惕了起来。

“元长与确是袒免亲,能在御史台中,本就已是特例,还是蒙上皇特旨许准。”蔡确沉沉的叹了一口气,“但他在京中这么长时间了,也该历任地方,若能如玉昆你这般在地方上建功立业,曰后回朝,能更有进步的机会。蔡确虽不才,进京前,却也是在外做了十几年的官。”

韩冈沉吟了一阵,问蔡确:“……不知相公到底属意何人?”

“游醇为人如何?”蔡确问道。

就像蔡确方才惊讶一样,韩冈也为之惊讶,然后否决,“游节夫为人直方公廉。遽然调任,赤子恐不舍。”

直方公廉,这个评价很高,但韩冈等于是在说那是个傻大胆,真的弄进来,肯定是给四方添麻烦的。

韩冈手边没有什么进士。要是黄裳中了进士,韩冈立马就能将他推进御史台。要功劳有功劳,要才学有才学,御史一任之后,转头就能在朝中风生水起。另外还有一个慕容武,可惜官声不是那么好,韩冈无意推荐他。

“这件事不急。”看见韩冈皱眉思考,蔡确笑了一笑,强调道,“玉昆,都不用急!”

