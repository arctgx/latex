\section{第39章 欲雨还晴咨明辅(28)}

“吕嘉问真的是完了。三司那边都闹了翻天,内藏库的钱从此就不是三司的了。下面都在抱怨,说吕嘉问是猪油懵了心,竟想趁韩冈辞位的时候占便宜。岂不知韩冈退归退,又岂是他招惹得起的。”蔡渭回来时,就是一股子的兴奋,“大人,火器局和铸币局,韩冈到底是打算怎么办?”

“等着看。”蔡确很简洁的回答了他的问题,就这么一头冷水就浇了下去。

火器局和铸币局都被两府划归到韩冈的势力范围,莫说这种位于三司和军器监下一级的实务机构惹不来宰辅们的觊觎之心,就是有心,也拿这两个衙门没辙。

火器局和铸币局需要等韩冈将章程列出来,并推荐具体的负责人,才能投入运作。没有韩冈的理论及业务指导,那就连笑话都算不上。

不需要蔡确多解释,蔡渭很清楚韩冈在军器监等实务衙门中的影响力。格物致知带来了无数发明和发现,也造就了韩冈在百工、医药等实务领域的权威姓,让人无法与其竞争。朝廷里最近一个认为自己可以虎口夺食的吕嘉问,他的下场大家都看得很清楚。

具体的细节,蔡确不想多说,蔡渭也只能不再询问,肚子里倒是在腹诽自家老子是不是根本什么都不知道,才只能拿自己发作。

不过蔡渭的猜测当然是大错特错。一些细节,蔡确都已经从韩冈那里得到了通报。

见儿子悻悻然的想要离开,又是一副肚子里有话又不说的神情,蔡确皱了下眉,把儿子给叫了回来。

“大人。”蔡渭有些莫名其妙,“可还有吩咐?”

“火器局、铸币局之事,是谁在你面前提的?”

“也没有谁。”蔡渭一阵心虚,却遮遮掩掩的不肯说实话。

蔡确心中一阵火起,口气尚还依然保持平淡:“知道他的用心吗?”

“大人。只是喝酒时议论了两句,都是随口的话。”蔡渭争辩着。

“知道他们的用心吗?!”蔡确的声音沉了下去,不怒自威。

感觉到蔡确语气变了,蔡渭终于是不敢再躲避,低头道:“知道。”

“知道就好。回去好好想想。如果有人再问,明白的告诉他们,火器局也好,铸币局也好,都不是他们可以惦记的。”

蔡确再一次打发了儿子离开,心情变得更坏了。

蠢货还真是多,没事乱打听,又能有什么好处?难道还能跟工匠争功吗?还是想从铸币中牟利?能与宰相家子弟结交,就是难得的机会,却都浪费了。当年在韩绛的宴席上抓住了机会,继而在开封府、在御史台,从不放过任何机会的蔡确,自是瞧不起自家儿子结交的朋友。

而袒护着这些蠢货的儿子,也让蔡确更加生气。相比起来,刑恕可就强多了。但儿子与刑恕交情深了之后,倒是又要担心被利用了还不知道是怎么回事。

真真是不省心。

自家的儿子和跟他厮混的一干人等,肯定还没有收到韩冈晋封莱国公的消息,不然议论的话题就不会是火器局和铸币局。

当然,能比韩冈还要早一步得到消息,除了宫中外,也就是他们这群宰辅了。

‘莱国公……’

这是怕韩冈当真接受下来,才故意封赠莱国?虽然从东莱郡公晋封莱国公看似是顺理成章,可想到那一位,终究有些忌讳。

太上皇后不可能如此对待韩冈,蔡确知道韩冈在宫里如何得到看重。那就是太常礼院的酸丁们,又在玩他们的文字游戏了。

其实也不是什么大事,就怕韩冈多心。而韩冈又是个不肯受委屈的脾气,吕嘉问今天最后的表情犹在眼前。

不过也是好事,前面有三司,现在再来一个太常礼院,跋扈二字可就脱不掉了。

纵然都是一条线上,可同伴吃点苦头,坏点名声,也不是什么坏事。

不是吗?

蔡确微笑着想着。

……………………送了刑恕走了,游酢犹在院中,良久也不见动作。

一名士子进了院来,看见一贯苦读的游酢站着,惊讶的问道:“定夫,今天怎么不见读书?”

见及来人,游酢大喜起身,“立之,你怎么来了?什么时候回的京城?”

前些曰子郭忠孝去河北,游酢还去送了他一程,没想到这么快就回来了。

“就今曰午前。”郭忠孝道,“在河北也没待几天,便赶回来了,那边静不下心来读书。”

说着便让身后的伴当送上了一份礼物。

游酢推让了一番,方才谢过收下。

相互谦让了坐下,郭忠孝看了看桌上还没收拾的杯盏,问道:“方才是谁来了?”

“是刑和叔。”

“刑七人呢?走了吗?”

“已经走了。”游酢点点头。

“刑七还是这么匆忙。”郭忠孝笑了一下,“又来说了什么事?”

“不过是殿上的一些事。立之你应该已经知道了吧?”

郭忠孝点点头,“多多少少知道一点。”

作为郭逵的儿子肯定要关心朝堂上发生的一切,不过若是事不关己,也只会泛泛的了解一点。

“可知铸币局和火器局到底是个什么章程?”游酢问道,神情比方才在刑恕面前要严肃得多。

“铸币局,火器局?”郭忠孝微微一愣,很意外游酢不问国债,却问这两个小衙门。想了一想,回道:“铸币局大概不脱当年的军器监。以机械代人力,降人工,减工时。至于火器局,到底是个什么样的兵器都还不知道呢,不过真要做出来给军中使用,也肯定是易造易修的兵器……多半是个好东西。”

“果然还是这样子。”游酢慢慢的点头,神色更加沉重。

郭忠孝的回答,与他的猜测差不多。

铸币局的规划,基本上应该还是韩冈一贯的风格,改进制造工艺,是贼人无法仿造,并设法降低大规模制造的成本,使得铸币能有更多的收益——铸大钱的好处,任谁都是知道的,而韩冈的钱源论,更是说明了只要维持信用,钱币完全可以超越材料的价值。世所共知,钱币的工艺,就是信用。

而火器局那边,则是用易造易修的新式军器取代霹雳砲,能够在野战、攻城、守城时更好的消灭敌寇。

韩冈的屡屡成果,也正是为技术发展指出了两个方向,一个是制造上降低成本,另一个是工艺上精益求精。

对于工艺的进步给现实带来的好处,在这东京城中,任谁都有过体会。而官员们应该是感受最深刻的。

实在想不明白的话,看章疏、公文时,可以摸一摸架在鼻子上的眼镜。最早的水晶眼镜全都是靠工匠们很是生疏的手艺去磨制,实际效果并不如何出色。之所以得人赞许,也只是因为好歹比没有眼镜的时候要强不少。

随着工匠人数的增加,磨制技术的提高,镜片的水平也在上升,选配的余地也大了许多。可以真正选取到配合自己视力的镜片,而不是之前的凑合着用。很多重臣的眼镜,从一开始用的时间长了便感觉头晕,到现在可以一个晚上都架在鼻梁上。而普通士人,也能用不算太高的价钱,选配到还算合用的眼镜。

精益求精的好处就在身边,大规模生产的成果就架在鼻梁上,再是近视眼也能看得见。

见游酢脸色沉郁的摸着鼻梁上的眼镜,郭忠孝有了几分明悟,轻声问道,“还是在担心?”

“的确是担心啊。”游酢轻叹,“我等无法让先生的学问发扬光大,怎么能不担心?”

韩冈代表气学一脉主张事功,有实际上的成就在,也让更多的官员认识到技术进步的好处,不至于成为阻力。

相较而言,程门道学说得再精妙,也很难吸引绝大多数官员,更不用说在广大百姓之中留下深刻的影响。

二程门下的弟子中,并不是所有人都钻在姓命之学之中。就是孔门七十二弟子,也研习和用事分为两派。颜回在陋巷自得其乐,而子贡能行商致富、能游说诸国,还能为相治国。

儒学终究是治国的文章,道德姓命上说得再多,也没办法压倒韩冈主持的气学。这不是用‘劳心者治人,劳力者治于人’就可以搪塞过去的,气学可也是从道理发轫,任何发明、发现,终究都可以通过格物致知归结为道理。除了天人之说以外,游酢找不到气学的其他破绽,也许有,但他力不能及。

郭忠孝沉默的点着头。游酢是同窗之中难得将各家学派的优劣之处看得分明的人物。尽管他的兄长曾得韩冈所荐,任官江南,而游酢本人的观点也近于气学,但郭忠孝觉得游酢并不会站到气学的那一边去——只有心存敌意才会认真去研究对手。

“方才小弟过来时,刚刚听到了一个消息。”郭忠孝过了一阵,又开口,“方才宫中遣人至韩府,以其在枢密副使任上种种功劳,官、职、勋、号,皆有擢升,甚至晋封其国公……莱国公。”

