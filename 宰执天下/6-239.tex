\section{第31章 风火披拂覆坟典(八)}

七天十九小时。

比起韩冈给出的最低标准多出了近三天。

而且,经过简单的维修之后,那一台试验机又开始轰鸣运转。

如此完美的实现了第一步的要求。

政事堂毫不犹豫的将悬红已久的酬赏给了那个小组中的三名匠师以及十一位小工。

匠师三人皆得官,而小工也有上百贯的花红。比起辽国的悬赏虽不如,可这也是让千万人羡慕不已的奖赏了。

京师的报纸从第五天开始,便在连篇累牍的报道。万众期待蒸汽机搬上轨道,取代成千上万匹挽马的那一天。

所谓功成名就,不外如是。

“今日乃至金榜题名远不如打铁。唉,明日就让我家那儿子去拿锤子去。”

“拿惯了两钱重的毛锥子,可抡得动十斤铁锤?”

“拿不动铁锤,烧火棍总能拿得起来。”

即使在政事堂中,也不免有人看着眼热,酸溜溜的话一串接着一串。

宗泽瞥了眼过去,倒是安静了片刻,但他一走开,立刻又冒了出来。

宗泽脸色微沉,李诫从陕西回京复命,这要与他去见韩冈,半路上听见这些浑话,若是传到韩冈的耳朵里,堂后上下都要吃挂落。

李诫却是轻笑:“俗话说得好,水火相济,盐梅相成。这蒸汽机一成事,咸酸话就出来了。”

宗泽稍一欠身:“都是些鼠目寸光之辈,让提点见笑了。”

李诫放声笑道:“何谈见笑?无有此一等人,如何见得我辈高明。”

在人人谨言慎行的政事堂中放声大笑,十年中都不见得碰上几次。

性格疏狂,倒是不讨人厌。宗泽心里想着。

他与李诫打得交道不算多,本来一直以为李诫是那种专注于自己喜好的老实人,没想到还有这样的一面。

李诫黑黑瘦瘦,栉风沐雨的生活,在他脸上留下了深深的印记。与他兄长李譓相貌上差得甚远。不过李譓此人品性不佳,上次来拜见韩冈,给宗泽留下了很坏的印象。

“不过朝廷给的奖励的确太多了。”李诫笑罢,敛容又说道,“这才第一步,日后蒸汽机装上车船,那时候又该如何奖励?”

“相公不会吝啬,绝不会比北虏差到哪里去!”

“我的意思是,第一人要奖励,但之后成功的也不代表他们的东西差。我上次入京,去铁场看过,每一组都有自己的一套,丢额任何一个,都是莫大的损失。”

宗泽点头道,“提点所言,正是相公所虑。之前已经让王学士去安抚过了。”

“如此,我就放心了。”

宗泽知道,韩冈并不是只关注在这场漫长的竞赛中首先获胜的那一个小组。不佳这仅仅是第一步,。

要怎么用在列车上,要怎么驱动车轮,这都是需要解决的问题。光靠成功的这个小组,人手根本不够用。而且现在造出来蒸汽机,并不代表能将蒸汽机更好的应用到实际中去。

机械设计,首先是要有数学基础。

宗泽曾经看过这方面的书籍,但再看图纸,还是一样看不懂。君子六艺,宗泽也没脸说自己能够贯通。

当初韩冈从大食得到了一大批种子,同时还有成百上千的书籍。韩冈招揽了一批精通大食文的翻译,又亲自定下了所有的名词,连几何原理这个翻译书名也是韩冈定下。等到成书之后,就作为《自然》中的推荐书籍,传遍了中原。

这样连状元郎都不懂的专业知识,铁场那边就有几十个专家。换作宗泽在韩冈的位置上,也绝对不会为了其中一个小组,而放弃其他同样有才华、可能只是运气不佳的匠师们。

宗泽领着李诫到了韩冈的公厅前,韩冈已经出厅来迎接。

与感动的李诫一番寒暄,韩冈对宗泽道,“汝霖,你今天上经筵吧?准备好了没有。”

“还要什么准备?”宗泽摇摇头,给皇帝开经筵的时候,该说什么不该说什么,他早就知道了。“倒是历法的事,天子会不会问起。”

旧日的奉元历错漏频频,钦天监颇受其累,故而在苏颂主持下,大宋又一次重新制定了历法。

自皇宋开国,历法一向是大问题。尤其是在太宗皇帝禁止民间私研天文之后,天文历法水平陡然降低到连辽国都不如的地步。

开国初,沿用的是后周的《钦天历》,只是将其改名为《应天历》,之后又编修了《乾元历》,可大哉乾元没二十年,就换成了《仪天历》,之后二十年是《崇天历》,又四十年造《明天历》,九年之后再改成了《奉元历》,如今又改成了《元佑历》——这是懒得想名号,直接用年号了,这也是有开皇历、贞元历等先例的。

历法十几二十年一变,最多也没能沿用到五十年,开国一百余年,使用过的历法已经于享国三百年的大唐相媲美,远远超过东汉、西汉。这并不是说钦天监的官吏们有多勤快,也不是大宋的天文历法水平进步的有多快,相反地,是水平太差,从而导致节气始终与历法对不上。

能经过节气、朔望、五星、日月交食这些验证的历法,至今为止,能全数通过考验的一个都没有。

沈括当初荐举卫朴编订奉元历,推演过去的日食、月食也只能达到十中五六,胜过以往,但也不是绝对的优势。

气朔渐差的问题,困扰了皇宋百年,甚至到了域外,都落人笑柄。苏颂出使辽国,因为辽宋历法差了一日所以被询问,他以学霸的身份给予强硬的回击,但事实上,辽人的历法是正确的,大宋这边才是大错特错。

历法对于中原王朝的意义十分重要,只要是称臣,就要接受历法和年号,正朔二字的由来,正是源自于历法。不过如今变来变去的,这个意义已经快要跟笑话差不多了。

十四五岁的小皇帝难得对此事有所反应,“娘娘怎么说?”

宗泽低头,“此事非臣可知。”

赵煦身边的黄恩中是之前太后丧期一案后才调到赵煦身边,终于是熬出头了,低声道:“太后说先用着看,过些年不合用再换好了。”

熙宗之后,历法几年一换。在场的哪个没有经历过?也没人在乎历法今日改,明日改。

只要一年三百六十天不出问题,也没几个人会去在意日月五星运行轨迹与历法不合。

再怎么说,也没哪次的历法会错到指着满月说今天初一,只能看见星星的夜晚,说是八月十五。至于节气之差,一天两天也耽误不了耕种,何况被废除的历法,最多也就差个三五十刻钟。

但小皇帝对此十分不满,“如今万邦来朝,皆用国历。当那些番邦发现历法有错,皇宋脸面往哪里搁。事关朝廷体面,岂容得如此轻忽?”

“官家息怒。官家息怒。”黄恩中立刻安抚赵煦,“苏平章说好,韩相公也说好。所以太后才同意,否怎么可能随随便便就颁布出去?何况那些夷狄,连字都不会写,哪里还会计算。”

赵煦沉默的一语不发,察言观色的基本能力也让黄恩中闭了嘴。

宗泽正想开始自己的课程,却不意赵煦突然又提起一事,“近来朕听说,北虏有焚书坑儒之举,此事可有之?”

宗泽摇头,“多有谣言,实则仅只是驱逐而已。”

辽国弃儒主工,早就随着宇宙大将军炮的威名传到了南方。

那一座巨炮着实惊到了包括宗泽在内的大部分人,上万斤巨物,上车下车都不方便,但一炮便能摧毁哪怕再结识的城墙。相对而言,驱逐儒生的消息被掩盖到万斤火炮的阴影下。

而且辽国驱逐的多是腐儒,或者是空具野心而别无才干的一帮人。这样人被赶走,倒是让宗泽越发的佩服起耶律乙辛的眼光和手段了。

“卿家看辽国此举如何?”赵煦追问道。

宗泽立刻回道:“此买椟还珠,舍本逐末耳。”

“此话何解?”

“如今中国军备精良、火器犀利,夷狄垂涎兵甲之利,不足为奇。但国势日盛,乃是朝廷施以仁政、人心亲附、贤良毕集之故。若无贤人,失其本,得其末,陛下勿须忧虑。”

说起这件事,宗泽就忍不住想起前几天韩冈对他说得那番话。

将天子说成是一家一姓,韩冈对所谓天命嗤之以鼻的态度,十分明显的表达出来。

从那个物尽天择适者生存的座右铭来看,韩冈最重视的便是华夷之辨。

如果从出身上来说,这也的确是不足为奇。来自于关西的韩冈,对外族的敌视是潜藏在血脉之中,其中种种,远不是生长在太平百年的南方所能体会到的。

但宗泽还是为韩冈的态度所震惊。毕竟真正的忠臣,不会直接将皇室裹在身上的虎皮给扒拉下来。

赵煦皱着眉,却仿佛也听说了韩冈对这一次变局的态度:“那依卿之见,如今朝中大贤又是何人?”

“此人陛下岂不知?”宗泽鹰隼一样的双眼盯着赵煦,”正是韩相公。“
