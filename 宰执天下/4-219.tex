\section{第35章 愿随新心养新德(下)}

盛怒之下,韩冈直到回返家中的时候,脸色都有些难看。

拥有两世人生,在两个截然不同的时代浸淫于红尘中,韩冈的城府其实已经修炼得很到家了。

喜怒不形于色只是寻常而已,如果仅仅是政坛上的纷争,无论是占据上风,还是吃了点亏,对他来说,都不是什么大问题,就像流水过石一般,留不下什么痕迹。

但今天的情况就大不一样了。张载是他尊敬的师长,而通过气学为媒介,将后世的科学包装一番后带到这个时代,也是韩冈平生的夙愿。吕大临的这一手,不但侮辱了已经过世的张载,同时对韩冈的计划有着无法预测的影响。

所谓关心则乱,韩冈虽然没乱,但心情的确是糟透了。

回到家中,几个妻妾都看出了韩冈心情有异。王旖第一个上来,眼中满是关切,“官人,可是在宴席上发生什么事?”

“怎么会?”韩冈表情顿时一变,脸上浮现出看不出任何异样的笑容,“为夫一向与人为善,又是在富郑公的寿宴上,更不会有人闹啊……”

王旖看看丈夫的神色,眼中的担忧没有消退,但也不追问了,只是帮着服侍韩冈沐浴更衣。换了身家居的常服,韩冈看起来十分悠闲的坐在书房中,翻看今天呈上来的公文。手上的笔不停,看起来已经全心全意的投入到工作中。

韩冈不想让妻妾担心,同是吕大临完成的又只是草稿而已,并非正式的行状,还是可以修改的,韩冈也不想就此闹起来,闹得大了,与如今失了主心骨的关学并无好处。

只是韩冈无法确定,将张载毕生心血所得的源头,说成是他的两位表侄。这究竟是吕大临一人的独断,还是受到他人的蛊惑。但从情理上来来判断,应该不是出自二程的授意。否则当此篇公诸于世,横渠门下的态度只会跟自己一样,甚至会将怨恨归咎于二程。

无论程颢和程颐,又或是所有的大儒,都必须珍视自己的名声,否则便无人会向他们求学。在世人的看法中,德远比才要重要。在过世的张载的行状中动手脚,由此带来的恶劣影响实在太大,他们都承受不起,也不会愿意承受。

不过韩冈也很清楚,如今的气学一脉,虽然因为张载在京中讲学数载,门徒为数众多,一时间兴盛无比,可门中的核心成员,依然是来自于关中的那些弟子。

如果张载的寿数能多延长几年,在京城来聆听张载讲学的那部分新弟子,将会有许多彻底的投到张学门下。只是在张载已经过世的现在,大部分已经风流云散。而旧弟子们也需要一个新的核心。

从名气上看,吕大钧、苏昞、范育和韩冈这张门四弟子,的确都是合格的人选,但他们各自都有着官身,在外任职的时候居多——要不是由于身份地位的关系,没有多余的时间来推广和教学,韩冈倒想在此事上多下些功夫——而且在韩冈出现之前,蓝田吕氏一直都是张载最大的支持者,也因此,一直跟随在张载身边的吕大临才成了撰写张载行状的不二人选。不过这也是韩冈当时还在广西的缘故,否则他更相信吕大钧或是苏昞。

只是现如今,吕大临转投程门,韩冈会坚持着自己的道路,这条路也与程门道学无法融合,剩下的弟子也会有各自的选择,这样的情况下,关学内部的分裂不可避免。

韩冈反思自己最近的行动,是不是跟程家走得太近了,可程门立雪,席上退避,这些事都是他做出来的,名气已经打出去了。韩冈与程家的关系自然还是紧密深厚,但若是被人归为程门弟子,却也是韩冈所不愿见的。韩冈打算发扬光大的去处,依然是在气学之中。

但程颢与自己有授业之恩,是时所公认的半师之谊,如今张载已然仙去,韩冈被人误认不足为奇。对于程家,韩冈无意否认师生名分,更不打算割席断交。先不说名声问题,他跟程家的关系不恶,为此而反目就未免有些举措失当了。

只是一码事归一码事,韩冈可不打算抛弃关学的未来,将自己的一番辛苦所得付之流水。这个时代需要的是,不是经义大道。

可到底该怎么做,韩冈一时还想不出个简单而行之有效的办法来。

书房外响起了脚步声,严素心亲自端了一盅紫苏饮子过来。韩冈慢慢的喝着滚热的药汤,就听严素心问道:“官人今天可是为了横渠先生之事?”

都是亲近无比的枕边人,他的四名妻妾看来并没有被他粗劣的演技所瞒过。寿宴后,跟吕大临的一番争执,韩冈带在身边的伴当尽管并不知道详情,但并不代表他们会不知道韩冈心情变坏的原因。四女只要问一问跟在韩冈身边的随从就能知道到底是怎么回事了。

“不妨事的。”韩冈向严素心宽慰的笑了笑。

公务上的事情,他尽量不想跟家里面多说,如果是喜事倒也罢了,但一些勾心斗角的对话,传到自己家里,可就是连块清净之地都找不到了,所以韩冈也只会对自家的人说一句‘不妨事’。

可是韩冈虽然是说不妨事,但实际上的变化却出乎他的意料。

程颐准备入关西讲学!

——当隔了几日,韩冈将洛阳城中剩余的致仕老臣们一一拜访过,前去程府中辞行的时候,程颢这么跟他说的时候,韩冈也不免要楞上一下。下手太快了一点吧,张载才走了多久,这么快就开挖墙角了。

只是韩冈听到这个消息,在一瞬间的呆愣之后,甚至不知该说什么好。他没有任何合乎情理的理由来阻止程颐满怀着诚意入关中,也没有手段来阻止。

道统之争,本来就没有什么情面可言。张载不合去世得太早,留下来的遗产,后人若是保不住,也别怪他人来抢了。

“不知正叔先生何时入关中?”韩冈的脸上完全看不出他心中的愤慨,仅仅只是当做寻常询问。

“也就是再过半个月的样子。”程颢向韩冈解释道,“陕州知州和永兴军路转运司,同时来信邀请入关中讲学,已经不好再推脱了。”

韩冈笑了笑,表示自己能够理解,却不再多问。

程颢也知道这么做事犯忌讳,但为了自家的道统,也顾不得那么许多。幸而从韩冈表现出来的态度上看,应该还没有想过这些事,或是说,并不是很在意。

韩冈其实十分在意,但他现在却没办法去计较。潼关的大门并不是由他来掌握,关中更不是他说了算,程颐受邀入洛阳讲学,韩冈既不可能去阻止,也不可能等程颐开讲后,再派人去居中拆台。

韩冈能做的就只有加强己方的实力。

他的基本对策是通过格物致知,将各家学派对于自然的理论给颠覆掉,难度看起来很大,但韩冈知道,只要在人们的心目中树立起自己的绝对权威,要做到这一步就很容易了。

在有意无意之间,他早早的就已经打下了基础,如今的在民间,韩冈的声望能压得住任何一家学派。得到民众的支持,韩冈所倡导的学术便能得到南方的认同。

普通人对权威的信任是盲目的,韩冈要做的只是选择一点突破,只要他突破的这一点,为世人所认同,甚至更进一步,全心全意的相信,那么他对其他事物的议论,也自然成为圭臬,也就可以带动起全局性的变化。

回到家中,书房也收拾的差不多了。韩冈打算将他的治所移到原京西南路转运司所在的襄州。这样可以就近监察。一个多月要般两次家,韩冈的家人并没有抱怨什么,而是在王旖的指派下,有条不紊的将搬家事宜一一处置完毕。

家中的事有贤内助处置,韩冈自然是的轻松了许多,当起了甩手掌柜。

回到书房,他从空搭档的书架上拿下一只小木箱,沉甸甸的看起来不是装了金银财物,就是本身拥有足够的重量,厚实的壁板乃是樟木所制,以防蠹虫。

拿着钥匙,打开了木箱,可以看得到里面的收藏。

韩冈的木箱,只看壁板很厚,沉重的重量皆来自于此。里面没有装着金银珠宝,而是一叠叠装订粗糙的手稿。这些皆是韩冈亲笔所写就,对韩冈本人有着无可估量的价值。不过其中的一部分,对于世界的意义更大。

木箱中的主要纪录,全都是回忆录,但毕竟是韩冈回忆前世,用笔记录下里的记忆,虽然为防被人误读,而加以变化,但也的确值得用个结实的木匣来承载。而另一部分的价值更大,是韩冈这些年来的诸多著作的手稿,其中还没发布的一部分,接下来能派上大用。

再一次点验了一番,合上了盖子,韩冈拍了拍小小的木匣,自己将来的名声就在其中。

