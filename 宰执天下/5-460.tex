\section{第38章 何与君王分重轻(18)}

这两句都是张载亲笔所撰。只要聪明人都会选择了利用张载来攻击韩冈。

张载的西铭中的乾称父、坤称母,跟韩冈讲究以实证之,其中有着无法弥合的缺点。很多人都看到了张载的天人合一与韩冈的格物致知之间的问题。

不过韩冈并不是那种明知有问题,却不知弥补,至少能在表面上说得像那么一回事。

“先师曾有言,《订顽》之作,只为学者而言,是所以订顽。天地更分甚父母?只欲学者心于天道,若语道则不须如是言。”

张载著西铭的本意‘只为学者而言’,将乾坤比喻做父母只是让人比较容易理解,所以名为《订顽》。其实根本就不该分什么天父地母。若直接讲‘天道’就不须这么麻烦。

只是韩冈不这么看,其实换一个角度,可以得到更能说得通的解释,“不过在韩冈看来,则另有一份解释。屋舍,木料、土石、砖瓦所集。江河湖泊,滴水所合。而万物所成,也必是尤小于沙烁水滴的细微之物。细微至无法再分解,谓之原子。原子的类别之分,则是元素。”

韩冈的原子元素论,流传得很广。很早便与张载的虚空即炁,配合起来。炁凝为原子,原子又以元素分,自由组合,拼凑成世间万物。成为支撑气学世界观的主体。

很多实验已经证明元素论的正确,至少殿中对此没有太多认识的,除了班直侍卫和一小部分内侍,就只有年方六岁的太子了。就是皇后,也多多少少了解了一点。更别说殿上的其他人,敌视气学的新学和程门,他们中的任何一个都见识过其中一个或几个实验,其中最强硬的,也不得不承认元素原子论确有那么一份道理。现在在韩冈面前,出来就是被反击的份。

见无人出头应战,韩冈了然一笑,继续道:“厨上烹饪,米饭、菜蔬、猪羊鱼肉,若放置于炉灶上不加理会,最后水汽散尽,皆会化为炭黑。可知稻麦、菜蔬、肉蛋,本原之中,大部是水和碳。人亦如此。组诚仁体的成分,也不过是水、碳等物。”

“饭菜鱼肉,烧焦后蔡卞皆曾见,惟人从不知。枢密何从知晓?”

韩冈眼神冷了下来:“大战之后,受伤截肢,不得不火烤封住创口的伤兵以千百计。上过战场,又有几人没看过?烤到最后,也只会是碳。”“无一物不可在天地间可寻找到。并不比鸟兽虫豸更多。身为天地所成,人、物皆如此,‘故天地之塞,吾其体’‘物,吾与也’。”

成了众矢之的。韩冈瞥了眼沉默的天子,现在的情况,当是他乐见吧。韩冈也是乐见其成。反正这样的经辩其实吵不出个眉目,闹到最后,一拍两散,让赵顼劳而无功就行了。

一切自然科学都可以是社会科学。关键是解释权在自己手上。

吕大临脚尖动了动,有点忍不住想说些什么。

在殿中的官员,只有吕大临对张载的西铭最为了解,也最为通透。韩冈能东拉西扯将气学与他的学说挂上钩,别人看不出破绽,可吕大临就能从中看出问题来。

在吕大临看来,这一次,韩冈为了证明自己的见解,又再次曲解了张载的观点。这是一个极难得的机会,吕大临正想出来指斥,但他立刻发现韩冈正直视着自己。

仅仅只是将眉梢轻轻一挑,吕大临却不由得心虚起来。焉知这不是韩冈的陷阱?万一弄错了,让韩冈趁机在集英殿上再来个丢石块、吊铁锤的实验,这场经筵还怎么持续下去?天子左袒,也是偏在王安石那一边,而不是程颢身上。

现在韩冈很明显的是想要把话题往实验实证上引,若是自己一步踏进陷阱,自家的颜面不要紧,连累到师友可就是罪莫大焉。

吕大临心中默念着,提醒自己,在旁艺上不要跟韩冈争辩。只要被拖进他的节奏,韩冈能立刻逆转取胜。只有经传,才是他的弱点所在。

吕大临针对韩冈准备已久,也自问寻到了伪‘气学’的致命伤。但他每次再见韩冈,都发现准备得不够多。大多数的时候,是韩冈总能用实验来证明,甚至就是他的陷阱。不过有的时候,则因为韩冈太过肆无忌惮,对不合己意的经典直接否定。韩冈的理论最大的问题就是物化,凡事都从实证,眼见为实,须知有些东西是做不到眼见的。

正想说话,韩冈抢先一步,“说到征战,五经之中,以《春秋》所言尤多。”

“明上下之序,分华夷之别。《春秋》是也。”程颢说道,“《春秋》一书,无外乎尊王攘夷,明礼教纲常。征战不能不多。”

在仁宗朝,以泰山先生孙复的《春秋尊王发微》为起点,诠释《春秋》的儒者极多。没什么好奇怪的,北面被辽国逼,西面为西夏欺而已。

所以要尊王攘夷,明华夷之辨。既然武力上不能胜人,就在文治上来个精神胜利法好了。我打不过你,但我可以鄙视你。

世传王安石不喜欢《春秋》,但确切点说主要还是不喜欢《春秋》三传,认为《春秋》自鲁史亡,其义不可考。后人传注,纯粹是‘一时儒者附会以邀厚赏’,‘决非仲尼之笔也’。

故而当王安石的学生陆佃、龚原打算为《春秋》做注,仿效孙复等人,王安石就直接批评说是‘断烂朝报’——这说的是陆佃、龚原所作的注解,可也足见王安石对《春秋》的偏见。就是现在的国子监,课程中也没有春秋一科,国子博士中同样没有春秋博士。

不过对于程颢所言,他也是没有什么好批驳的。孙复对《春秋》本经的重新诠释,在此时儒林,已是士人为研习《春秋》的重要传注,只比《公羊》《谷梁》略逊——仅仅比不了《左传》——不管怎么说,至少尊王攘夷四个字没有多少人会质疑。

“夷狄者,禽兽也!人所共知。可论事,当据于实,本于理,方可谓之正论。韩冈敢问伯淳先生,为什么说夷狄是禽兽?道理何在?又是怎么得出这个结论?”

没有论据和合理的论证,怎么将夷狄和禽兽挂上钩?并不是每家夷人都会跑来打劫中国,也不是每家蛮部都有子蒸其母、兄亡收嫂的习俗。

华夷之辨,是儒家治平的关键。人与禽兽之别,更是世界上每一个哲人都要考虑的问题。

程颢对人禽之别、华夷之辨的观点,是人至中至正,合中庸之道,若有偏,那就是禽兽、夷狄了——‘中之理至矣,独阴不生,独阳不生,偏则为禽兽,为夷狄,中则为人。’

但在经辩上,却不能这么说当年张载在洛阳设虎皮椅讲易,程颢与程颐登门挑战,一战成名。经验十足。他很清楚,经辩胜负的关键是不要在对手之前露出破绽,持论要正,论述要稳,不要求新求变,不求有功但求无过,等待对手犯错。

所以他选择了很大路,在儒者中又无可辩驳的回答:“有礼者为人,无礼者禽兽。”

‘凡人之所以贵于禽兽者,以有礼也。’这是晏子春秋中的话。基本上是儒家的共同认识。

蔡卞则更放恣一点:“‘夫唯禽兽无礼,故父子聚麀’。而父子聚麀,也就无礼如禽兽了……难道枢密不这么认为?!”

“什么叫父子聚麀?”皇后小声的问身边的大貂珰。

刘惟简张口结舌,出了一身白毛汗,将蔡卞恨到了骨头里。被皇后狠盯了两眼,低低的颤声解释了两句。

向皇后乍听,刹时白皙的脸上一片红晕,直烧到了耳朵上,隔着屏风,怒瞪了蔡卞一眼。就是出自圣人的《礼记》,也不当在太子面前说。蔡卞茫然无知,仍是正盯着韩冈。

“那只是表象,且并非人人如此。天下蛮夷千万,都有这般风俗吗?”韩冈瞧瞧赵佣,再看了眼屏风,有小孩子和女人在场,不好多说,“在韩冈看来,自然中之芸芸众生,无论动物,植物,遵循的道理惟有一条。这亦是人禽之分、华夷之辨的大关节。”

牛皮吹破天了。吕大临强忍住没冷哼出声。

韩冈的说法,等于将历代辨析华夷之别、人禽之分的论述全都否定了,就连《礼记》都一口气都贬到了底,口气之大,都让人想到井底的蛤蟆。

王安石心生,也觉得未免韩冈未免说得太过了。芸芸众生,除人之外皆从一理。

大言不惭。这是蔡卞的想法。

“莫不是生老病死?”

“此四事,何人可例外?”

蔡卞恨自己嘴快,进士出身,又进了馆阁,只是被韩冈气糊涂了,一时气话脱口而出。韩冈却没多理会他,只反问了一句,并没追击。

程颢心思一动,神情更加专注起来。韩冈的姓格,他很了解,没有充分的把握和自信,不会将话说得那么满。

“是什么?”宋用臣代天子发问。

“物尽天择,适者生存。”

