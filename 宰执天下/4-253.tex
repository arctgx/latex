\section{第42章 壮心全向笔端含(下)}

【昨天断更的原因众所周知,就不多做解释了,今天会补上。】

虽然是一月才得三次的休沐之日,不过沈括仍在书房中忙碌着。

并不是衙门里的事,沈括治事之材,放在当世数万官员中,也是第一流的,衙门里一成不变的琐事,每天只用一个时辰就解决了。

也不是方城山的事。方城山那边,进度已经进行过半,每天传回来的消息都是在说一切顺利。看这样子,除非出现大的意外,否则六十万石纲粮在十一月的时候,肯定能全数抵京。

按道理是现在就可以去筹划庆功宴要怎么开,但襄州那里却没有动静。不过韩冈这位正主都不放在心上,沈括也不会表现得太过急切。

他今天只是在整理着残篇断简一样的片段,分纲目进行记录。等到致仕之后,有了余暇,再进行更进一步的修订,以便成书传世——一部承载了自己毕生的见闻和经历的笔记。

在往日,沈括能得空整理自己的心血,顺便避开总是一幅坏脾气的续弦,心情肯定是很好。

只是今天不同往日,沈括神情严肃的拿着一封信,从书架上抽下一卷已经装订成册的草稿,刷刷的翻了几下,很快就停在了其中的一页上:

‘方家以磁石磨针锋,则能指南,然常微偏东,不全南也,水浮多荡摇。指爪及碗唇上皆可为之,运转尤速,但坚滑易坠,不若缕悬为最善。其法取新纩中独茧缕,以芥子许蜡,缀于针腰,无风处悬之,则针常指南。其中有磨而指北者。余家指南、北者皆有之。磁石之指南,犹柏之指西,莫可原其理。’

对照手边的信,沈括苦笑了一声,韩冈在信中就磁针指南一事,说得更加通透,绝不向自己,只能说一句‘莫可原其理’。

这是前两天韩冈才从襄州寄来的。本来在前一封信中,两人讨论的是北极星与北极之间的角度差异,沈括也只是在信中随意的提到了司南指向的方向,与实际上的南极北极有着不小的区别。

沈括还在京城时,分管过主管天文的司天监,曾经重新设计浑天仪,并通过浑天仪来观察过北极星,持续了三个月之久。

星象之事向来招犯忌讳,从太宗皇帝开始,就禁止民间私下研究,就是官员也很少会光明正大的去研究。沈括也是跟韩冈相熟之后,才会偶尔在信中提到一句两句,而且半点不沾占星判命。

从韩冈的回信中,沈括发现他对于星占甚至是嗤之以鼻,也秉持着依靠张载才兴盛起来的的宣夜说,反对浑天、盖天的说法。

对星象,韩冈的观点不同于流俗。而对于磁铁、司南等堪舆上的用具,他也同样有着一番独特的见解。竟然说大地本身有磁性,南北向,故而能让磁针指南。虽然也纯属臆测,但仔细想过来,却并不是毫无根据。

司南、司北,沈括家里两种磁针都有。将不同种类的磁针针尖对针尖的放在一起,就会一下吸住,而则是互相排斥,如果将磁针掉个儿,情况就正好相反。正符合韩冈在信中所说的‘同性相斥、异性相吸’的这一句。而将两根磁针,一根磁针一根钢针放在一起,磁针的指向也会产生变动。

所以韩冈说藏在地下的磁铁,引得天下磁石能定方向,也不是没有道理。而且南北磁极毕竟不是真的南北极,所以沈括能观察得出两者之间有偏差。

尽管多有臆测,但毕竟能说得圆。

沈括将信纸折了几折,好生的收了起来。

磁石指南的成因只是很小的一桩事,但韩冈从中体现出来的广博学识,又一次让沈括感到惊讶,甚至想不通,他哪里来的这番见识。格物致知四个字,可搪塞不了所有人。

韩冈说黄河之所以为黄,乃是西北高原水土流失之故,河北海退陆进,这是合乎他的经历,沈括也是有着同样的观点。但嵌在太行山壁中的无数贝壳,证明了沧海桑田的之说,自家是出使辽国时,才亲眼见证过。而韩冈并没有去过太行山,就已经一清二楚,并说此乃百万年、千万年、亿万年逐渐演变而来。哪里来的见识?

而且说着也好笑,唐尧也不过出自三千年前。邵雍修皇极经世书,一元才不过十二万九千年。韩冈张口就是百万千万亿万,邵康节到了他眼前都得避退三舍。

在沈括和韩冈三四天便有一次的信函中,如同太行山贝壳之类的事情说得很多,充分体验了韩冈本人学识上的的渊博。但相对的,自从入秋后,沈括在与韩冈的书信中,能明显的感觉到他对襄汉漕运没有之前说得频繁了。

韩冈不能算是突然间冷了下来,看起来只是像将最后的工作全都交托出去,交给了方兴和李诫来处置。

说起来就像是种地,犁过田、下了种,除草施肥都做了,剩下的自然就是等着开镰收割了。当然,这个比方联系起韩冈的出身就显得有点刻毒了,更恰当一点的比喻,是宰执治事的手段,只管定下目标、安排人手,具体事务让经手人自行掌控。

韩冈有这番气度,沈括多有感慨。不过他也热切的期盼着襄汉漕渠能有所成效。毕竟自家的长子在韩冈那里,李南公的儿子也在韩冈那里。韩冈一旦成功,两人都有好处。

而且沈括和李南公还要另外承韩冈的人情。光是为了两人的儿子,韩冈就担了很大的风险。

李南公的儿子还好说,在营造和机械上是难得的人才,这一次的工役也是立了大功,一句‘内举不避亲、外举不避仇’就能将所有的弹劾挡回去。

但自家的长子就不同了。自己是亲民官,韩冈是监司官,韩冈这位转运使在监察他沈括的同时,却将他的儿子收归门下,这是致人话柄。当日情急,无暇细想,草率的答应了下来,不说欠下的人情越来越大,日后一旦给翻起来,两边都少不了一个罪名,往重里根究,结党之罪都是能栽到头上。

“老爷。”沈括贴身的小厮进了书房,“韩龙图那边派人送信来了。”声音突的压低了一点,“还有大郎的信。”

“哦……快让他进来。”

沈括让人将韩冈派来送信的家丁带进来,是惯常往沈府来送信的。问了韩家上下可否安好,就打发了他下去休息,“明天过来,我这里还有回信让你带回去。”

儿子的信上没有说太多,只是问候和报平安。沈括叹了口气,也是无可奈何,将信藏好收起。

韩冈这一次让人带来的不仅仅是一封信,还有一部多达十卷的书籍,不过仅是手稿而已。韩冈在信上说,是近年来的一些见闻、心得的记录,其中多有疏漏,敬请斧正。

看见韩冈在封面上写下的《桂窗丛谈》四个字,沈括为之一笑,知道他没在标题上费太多的心思。

不过这一部《桂窗丛谈》,单是纲目就很有意思。沈括给自己日后准备撰写的笔记所整理的资料,是分为故事、辩证、乐律、象数、人事、官政等十七门,而桂窗丛谈中则是算学、地理、生物、物理、化学、医药。编目是一本书的大关节,明眼人从目录中就能看得出作者的用心所在。

沈括第一眼落上去,就发现整部书丝毫不涉人事、官政的内容,若在别人看来,定是韩冈做官的时间太短,家中在官场也无底蕴,不像一般的阀阅世家、书香门第,从小就耳濡目染,对官场上的传闻、轶事、典故了如指掌,与其写出来让人笑,干脆就不写。但沈括了解韩冈的性格为人,更清楚韩冈出身的气学如今的现状。

在张载去世之后,气学中衰,开创洛阳道学一脉的程颐已经进关西讲学去了,气学再不站出个力挽狂澜的人物,就要给人斩草除根了。韩冈这本笔记,是去维持气学道统不衰的。

笔记一物,有的是想把儒家三立做到实处,有的则是为了给自己脸上贴金,更有的根本就是用来搅混水,七真三伪编造谣言——这世上,十里不同风,百里不同俗,隔得远了,好端端的事都能传得千奇百怪,若是有心造谣,实在太轻易不过。

韩冈写出这一部《桂窗丛谈》自然不是为了编造谣言,而是宣扬气学——立言罢了。因为他跟程家的关系,加上本身的学问所限,不便在经义上与人比高下,却是想出了别出蹊径,彻底的贯彻他最擅长的格物之说,以实证道的手段。

而韩冈之所以先把手稿给他沈括来看,可能就是看在自己在这一方面上的名气,要自己来捧场鼓吹而已。沈括微微一笑,还韩冈一个人情也是好的。

随手从中抽了一本出来,慢慢的翻着。韩冈为官不及十载,却是比天涯海角还远的地方都去过了,一条条的倒是很有些意思,但翻过两页之后,他却陡然间就坐直了身子,脸上的笑意也收敛了,再也无法将视线挪开去。

