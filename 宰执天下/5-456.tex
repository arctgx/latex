\section{第38章 何与君王分重轻(14)}

赵顼靠坐在一张躺椅上。

让大匠配合身形打造的木质躺椅,赵顼躺在上面就像嵌进去一般,使得瘫软的身子不至于左右歪倒。至少能与他的儿子面对面。

‘六哥,一,一百,多少?’

赵顼在沙盘上画出的字断断续续,但赵佣站在福宁殿里,就是在说韩冈所出的题目。杨戬也不会误会赵顼想说的话。

他向赵佣转述着:“殿下,官家是在问韩枢密出的那一加到一百的题,最后算出来是多少?”

“五千零五十。”

赵佣说话时有些不好意思。

他自下课后用了一个多时辰算了两遍,但答案并不一样。赵佣本来想再算第三遍,宋用臣劝他,韩枢密说了可以向别人询问。

可赵佣只记得,韩先生指的是棋盘上放麦粒和尺子锤子的两题可以问人,一加到一百,是要让他自己算的。

前一道题他问了母后,后一道题,他过来问父皇。最后一题,赵佣还是坚持自己做。因为母后让他听韩先生的话,父皇也让他听韩先生的话。

不过方才半路上,刘惟简追了上来,告诉他还有简单的办法,让他一加一百,二加九十九,一直到五十加五十一。

赵佣很快就想明白了,总共是五十个一百零一,拿着笔算了一下,正好是五千零五十。跟前面第一次一步步加起来的结果相同。

刘惟简并没有告诉他答案,可是如果刘惟简不说,赵佣根本想不到还有这么简单的计算方法。

赵佣不知道这该怎么算,算自己的,还是别人帮忙?小孩子心里有些别扭。

‘对否?’赵顼写到。

“方才奴婢让人去算了。六人里面有四人报的是五千零五十。太子算的应当没错。”杨戬低声回话,顺手抹平了沙盘。

赵顼眨了眨眼,抬手又开始写字。

此时,门外宫人入内禀报,“官家,苏学士到了。”

苏颂是翰林侍读学士,乃是经筵官,为赵顼讲习经史。他在朝中是有名的博学,跟韩冈来往久了,也被视为气学一脉。

苏颂被招进宫来,具体是什么事,他已经提前知道了。

“其理在于重心!”苏颂回答天子的询问。

韩冈在《桂窗丛谈》中说过重心的问题。曾经拿尺子、木板、盒子和捕醉仙[注1]来说明什么是重心。

重心向下的铅锤线没有移出底面,盒子就不会翻倒。木板的重心如果落到了桌面外,就会掉到地上。坐在椅子上,手不用力的情况下,身子不向前倾就站不起来,想要起身,重心必须要移到脚上。

苏颂亲手做过实验。结果的确如此。

重心的原理,完美的解释了大堤为什么要下宽上窄的缘故。而空车空船为什么容易倾覆。相扑往往是个子矮壮的人是赢家。

苏颂用了半个时辰的时间,从头到尾,详详细细的向赵顼、赵佣这对父子作了解释。并且还画了图,又做了几个小实验。

苏颂的教导浅近易懂。旁听的杨戬也明白了其中的道理,有种恍然大悟的感觉,原来道理就是这么简单。

尺子挂了锤子后之所以掉不下去,就是因为这一整套系统——这是苏颂用的生僻词汇——的重心位于桌子下,位于桌面的投影内——这个词同样生僻,但解释了就很容易理解——其实就等于放在桌子上。单一的尺子之所以会掉下去,则是因为重心在桌面外,且受力不平衡的缘故。

‘重心’。

杨戬在心中默念着,也看见天子在沙盘上写着。

现在想想,韩枢密想要说的就是这两个字吧。

想不到小小的尺子和锤子之中就蕴含了这么多的道理。

把握到了重心,看似匪夷所思的事,其实也很平常。关键就是要抓住其中的道理。

伊尹对商汤说‘治大国如烹小鲜’。

毛传曰:‘烹鱼烦则碎,治民烦则散,知烹鱼则知治民’,‘烹小鱼不去肠,不去鳞,不敢挠,恐其糜也。’。《淮南子》《韩非子》也都有提及。

就是唐明皇也曾作注解:‘此喻说也。小鲜,小鱼也,言烹小鲜不可挠,挠则鱼溃,喻理大国者,不可烦,烦则人乱,皆须用道,所以成功尔’

以烹小鱼喻治大国,这是杨戬之前在宫中上学时学到的。而现在韩冈岂不是在用重心之说来比喻治事?找到重心,便能举重若轻。

杨戬自问是明白了韩冈的想法。

上古贤人都喜欢做比喻来规劝帝王,药王弟子难道不正是跟他们一样?可谓是用心良苦啊。

从他这个小小宦官眼中,王、程两位,看来远远比不上他们的后辈晚生。

杨戬的眼中,官家也在沉思着,在苏颂走后许久,他才又开始写字:‘棋盘……’

杨戬会意,让人去寻答案。

结果让杨戬瞠目结舌,聪明的太子歪着脑袋迷糊了起来,而天子,则又是久久不动。

远远超过想象极限的数字。开始时仅仅是一粒两粒麦子,在六十四倍……呃,六十四番之后,就变得庞大的难以相信。

杨戬心中明悟。

这同样是劝诫。

乡里的高利贷中常有倍利,逼死人命无数。杨戬幼年还没进宫时家境贫寒,对此深有体会。

想一想,只要借一文钱,六十四年后,就会变成大宋几百几千年的税入都抵不过的巨额债务。

而正常借几贯钱,也不要六十四年了,三五年就能逼死人命。

这又是在讽喻天子,该抑兼并,减民贷啊。这也是让在宫闱内长大的太子知晓民间疾苦的唯一办法了!

朝有贤良,家国之幸。

在崇敬和激动中,杨戬又看见赵顼在沙盘中划着:

‘明曰,招韩,经筵。’

韩冈的下一堂课,本应是放在两天之后。

并不是赵佣的课程安排不合理,而是王安石、韩冈这样的重臣,让他们天天给太子上课根本不现实。至于程颢,因为另立道统的缘故,他来上课的次数,也被王安石压着,不可能更多——据说其中还有不得皇后认同的缘故。

大部分时间,教导太子的工作都是交给资善堂中的其他教师。从地位最高的王安石,到最下面的小黄门,在资善堂中任职的多达近百人。礼仪、射术,甚至《论语》等经书的背诵抄写都是另外有专人负责。在韩冈回来之前,算学课也是另外有人来上,赵佣这么点大就学会了百以内的加减乘除,皇城中长大的宦官,其中允文允武者所在多有,也足见皇家教学的水平。

不过王安石、韩冈和程颢,终究是天子以诏书聘来的太子师。每天赵佣的课程中,都有一个重点科目,不是王安石,就是程颢,现在则又多了韩冈。

而现在赵顼所说的经筵,并不是给赵佣讲课,却是给身为天子的赵顼所开设。

文臣在经筵上讲读经史,借古喻今,杨戬面现难色,轻声劝道:“官家,可你的身体?”

赵顼闭目不言,只敲了敲手指。杨戬低头躬身:“奴婢知道了。”

……………………韩冈此时早回到了家里。

周南帮着更衣擦脸,云娘端上茶,素心也端来了亲手做的点心。

一切起居都有娇妻美妾服侍,自自在在,清闲无比。

比起十几天前,还在河东辛辛苦苦的曰子,不啻天壤云泥。

到了傍晚的时候,被王安礼的夫人请去说话的王旖才赶回来,见到韩冈在书房里靠在摇椅上自得其乐的看书,不禁笑道:“官人给太子上课可是辛苦了。”

“怎么可能?”韩冈呵呵笑着,“辛苦的该是学生才是。”

王旖闻言脸色一变,连声问:“怎么了?难道又是问那些刁钻古怪的问题?还是让太子去养蚕养蝌蚪做记录?该不会是速算吧。”

在王旖眼中,韩冈是有前科的。

韩冈给儿女出的题目常常比鸡兔同笼还刁钻,河里面两个岛,怎么不重复走完连接岛上和岸边的七座桥,要是不能,又是为什么?大人都做不来,他给小孩子做。

为了培养子女的观察能力,让老大老二去养蚕,自古男耕女织,拜马头娘【注2】该是女子才对。还抓了蝌蚪来,放在价值几十贯的玻璃花瓶中养,本以为是青蛙,却养出了蛤蟆,把满心期待的女儿委屈的大哭一场。

在早一点的时候,刚学了加减法,就开始要求速算心算。从上到下一百题列出来,喝完茶的时间看看能做多少,做完才有奖励。

为了儿女的教育问题,王旖跟他吵了好几次,但韩冈总是振振有词,最后让王旖说不出话来。

“怎么会。”韩冈摇头否认,“这第一次上课,是做给天子看的。怎么可能跟自家儿女一样。今天就是考了一下太子学到哪一步,剩下的就是出了三道题。”

“什么题?”

韩冈抬眼,“还记得当初一加到一百的那道题。”

王旖当然记得。家里的儿女都是死算才得到的结果,连王旖自己也是没想到还有那么简单的计算办法,倒是周南很快就找到了窍门。

“那另外两道呢?”

“另外两道题,不是让太子去做的,而是让他去问的。可比第一道要有意思得多。”韩冈卖着关子。

王旖正想细问,一名家丁匆匆而来,“枢密!外面来了中使,说是奉了天子口谕,要官人明天上经筵。”

注1:捕醉仙,唐宋时的不倒翁。酒席上放在盘中,转动后视其指向来劝酒。

注2:马头娘是蚕神,本为马头人身,但神像多为身披马皮的少女。有少女为马皮所卷化为蚕的传说。

