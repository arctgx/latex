\section{第33章 枕惯蹄声梦不惊(13)}

当入寇的辽军退回了石岭关和百井寨一线,北上的宋军也随之陆陆续续抵达了百井寨外。

从太原往石岭关的一路上,车马军民川流不息,将太原府中积存的粮秣、军械向前线运去,一直抵达距离百井寨五六里处的大营中。

“王子难做得不错啊。”章楶手握着厚厚的帐籍,望着一辆满载着干草的马车停在了营地一角的草料场外。大营中一处处粮囤草垛都是这几曰送来的,“这才几曰功夫,就已经存满了半月的口粮了。”

在一旁并肩而立的黄裳笑说着:“使功不如使过。王天章正是要待罪立功的时候,哪里敢懈怠半点?”

章楶瞥了黄裳一眼,这个口气可真够大的。

身为枢密副使的亲信,天章阁侍制当然不足以让黄裳有多少畏惧。但章楶还知道另一件事,之前王.克臣为了脱罪,可曾遣人带了厚礼走了黄裳的门路。黄裳这般口气,也不知有多少是因为此事的缘故。

“不过人皆有所长,王子难虽然临兵事稍有不足,但转运输送、治政理民皆是其长处,枢密用人用得好,留他一条路,反而平添了一份助力。”

章楶持平而论,并不附和,也不否定。可黄裳没多想,而是点着头,和章楶两人目送着一队又一队的小推车满载着各色物资,进入了大营中。

三天的时间已经为攻打百井寨做好了应战的准备,甚至都有了将石岭、赤塘、百井这两关一寨的辽军一网打尽的计划。

“这一下子,枢密也可以稍稍松一口气了。”

“或许吧。”黄裳叹道。

韩冈本身也承受着很大的压力,人也绷得很紧。虽然在表面上他尽量不让人看出来,但跟随着他的幕僚们都清楚,选择了主动追击辽军的这一条路,韩冈到底冒了多大的风险。

其实当官军已经将辽人逐出了太原府的现在,剩下的问题完全可以交由谈判来解决。以兴灵换回代州,如果再添加上五万十万的岁币,耶律乙辛有很大可能会同意这桩买卖。而现在官军继续北上追击辽军,主要就是韩冈的坚持,不能对强盗姑息养歼。一旦他失败了,别说用兴灵换回代州,辽军甚至可能重返太原。

纵然辽人因为已经抢到了足够的财物而开始变得厌战,可这并不代表宋军就能轻易的得到胜利。任何时候,一点小小的意外,也能让战局瞬间扭转。所以战前的准备,永远都是不足的,再多也不够。

“哦,赏钱也送来了!”当看见两辆马车从南方缓缓驶来,四轮的大车在道路上留下了深深的沟壑,黄裳更是松了一口气。

这是发给曾在太谷县城中坚守的宋军的赏赐,而跟随章楶的援军,虽然他们实际上在太谷一役中起到的作用很大,但既然没有实际的战功,当然也不会有多少奖赏。合理的赏赐能刺激更多的士兵用命,。

这也算是最后一项准备了,剩下的可就是要在战场上检验一下这些准备到底起到了多少用处。

韩冈此时正在前军营地中,距离百井寨只有一里出头的距离。

之前是先期抵达的前锋经过了一番鏖战,逼得寨中守军退守寨墙,这才将这一座前军营地按扎在此处。之后想要出击百井寨,就可以直接在此整军,而不是从五六里外的中军大营。

不过一里的距离,对于拿着望远镜在手的韩冈来说,依然觉得远了一点。他还是想着能更靠近些查看辽军在城头上的布置,以及寨中守军的士气。

只是韩冈刚刚吩咐了人下去做出行的准备,就立刻被拦住了。

陈丰站在帐门前,拦着韩冈不让他离开:“枢密,万万不可!”

“没关系。”韩冈摇头笑道,“我不会穿着太扎眼的盔甲。”

但陈丰依然强硬的拦在他前面:“枢密,辽人手上可是有八牛弩的,就在百井寨中!枢密身负朝野重望,岂能以千金之躯犯此险境?”

“百井寨什么时候有八牛弩了?”

以韩冈的记忆,百井寨中曾经配发的大小合蝉弩倒是有几架,但重型的三弓床子弩——比如八牛弩——百井寨中却没有,同时之前侦查的结果也报告说没有在城墙上发现,不仅是三弓床子弩,就是普通的两弓床弩也没有一架。

“但石岭关和赤塘关都有!”

“要真的已经运到了百井寨,早就搬上城头用上了。何况之前不是讨论过吗,床子弩运去忻州城的可能姓更大一点。”

“只要有一分可能,枢密你就去不得!”

虽然在韩冈不愉快的视线下,陈丰如同在山里遇上老虎一般冷汗直冒,但他还是坚持拦在韩冈的面前,不让韩冈更靠近百井寨的寨墙。

还真是麻烦。换作是十年前,班超都做过的,出生入死的次数不胜枚举,眼下的这点小阵仗算得了什么?哪里还会担心可能会有的床子弩。

不过现在身份也的确不同了。韩冈无奈的摇了摇头,放弃了去查看敌情的想法。

只不过他想去城下也是一时心血来潮,不然韩冈也不会如此干脆的放弃。具体的战术安排,他已经分派了下去,并不打算多干涉,这也就使得韩冈现在很是清闲。上午刚刚巡视过营地,又批完了公文,现在除了对着沙盘,却也没有别的事可做了。

虽然因为石岭、赤塘二关就在左近,百井寨的布置远比不上关西的边境要隘——关西的边防要塞,往往就是一处主寨配合六七副堡,加上遍布附近各处要点高地上的望台,方才交织而成一道坚固严密的纵深防线——不过百井寨的布置也不算差,其依山而立,虎视官道,外有高墙深垒,内则不缺水源粮草——尤其是水源,百井寨之名虽然夸张了点,但地下的确是多水脉,内外也多泉眼。

且在其所处的山峰峰顶,也设立了一座从属的小寨,屯以兵马,以防敌军攀上峰顶居高临下。山峰虽不高,却也不是那么好爬上去的。如果硬攻的话,就得做好惨重伤亡的准备。

陈丰见韩冈没有再想着往前线上跑,也算是松了一口气。平静下来,才发现身上的内裳已经为汗水湿透了。

朝廷鼎臣,威仪自生,方才只是被韩冈瞪了一下,就感觉整个人被冻住了。人都说见一次皇帝,都跟过鬼门关一样。现在只是宰臣面前都这样心惊胆寒了,真到了文德殿上,官家一皱眉,那不是都要吓死?!

轻手轻脚的走到韩冈身后,陈丰将视线投向面前的沙盘。通过对原百井寨中官兵的询问,以及飞船的侦查,百井寨内部的详图都在沙盘上展示了出来,每一间营房、每一眼水井、每一座望楼,甚至连几处安置在山岭上的暗哨的位置都在沙盘上标识了出来。

不过韩冈并不是在看百井寨,而是沙盘上更北面一点的地方。

在韩冈的指挥下,攻到百井寨下的宋军依然是有条不紊的做着准备,仅仅是派出游骑巡狩城下,寻找守备空虚的地方,并不急于大举攻城。

韩冈的目标是围城打援,最希望的就是能用百井寨调来石岭关或是赤塘关中的守军。纵然心知忻州危急,但攻取百井寨的准备依然是不紧不慢。

安置八牛弩的高台,防备石岭、赤塘二关援军的木栅,以飞快的速度在城寨外成型。城寨中的守军竟然没有出城干扰,而仅仅是拿着神臂弓射击。

百井寨的兵力的确不多,但既然有石岭、赤塘两关在后为奥援,实在不该这样等着挨打。

辽军在城池攻守战术知识上的匮乏,让韩冈得以从容布置。越来越重的压力,正随着时间一点点的压向了百井寨的守军。韩冈相信,百井寨中的辽军的神经不可能绷得太久。

习惯姓的屈指敲了敲沙盘的边框,韩冈轻吐出一口气。最多再有两天的布置,就可以发力攻城了,到时候,说不定能一举攻破三关。

韩冈正对着沙盘左右盘算,却感到身后一阵风起,帐内也亮堂了起来。他讶异的转过头,怎么有人不通报就进他的营帐?却见是一脸喜色的折可大,而守门的两名亲兵则慌慌张张的跟了进来,一左一右的夹着他。

“有何事?”示意两名玩忽职守的亲兵退下,韩冈回身问着。

折可大看模样就是兴奋得难以自抑,连请罪都忘了,“枢密,麟府的援军到了!”

“什么?!”却是陈丰沉不住气的叫出声来。

“领军的是折府州?”韩冈安安稳稳的坐下来,不过眉心微皱。

折家的兵马来得太快了,比预计的要早了好几天。

从府州往忻州一路翻山越岭,虽然东西相隔的直线距离也就四五百里,但正常的行军是不可能这么走。通常是先向南到麟州,然后渡河经岚州至太原,最后从太原北上,这算是山路走得少的路线了。麟府军这一回却直接杀到石岭关北,自然不是这一条路,而是更北面一点经过岢岚军的小路。

且大队的兵马跑不了那么快,也很难抄小路,想也知道绝不会是主力,只可能是前锋。也不知有多少人马,别三四百就打发人了,即便肯定是精锐,也未免太少了一些。

折可大这时终于知道正了正仪态,躬身道,“禀枢密,麟府的折克仁所部已经抵达了忻州地界,并已与秦琬部会合,遣来的报信之人正在营外等候。”

“哦?是折十六到了?!”证实了猜测,韩冈却露出了讶色。

折家的情况他很了解,折克仁虽然官位不高,可这两年他正掌握着折家近三分之一的子弟兵,能跑这么快,当也是统领精锐的本部,而不会是临时配给他的人马。只是不知是不是他手上所有的兵马。

“正是家叔。”折可大点头道,“家叔所领七百部众皆是有马步人,所以为先锋。奉了家父之命领兵兼程来援,而家父所率本部兵马则还需几曰功夫。另外来此的也是家叔的身边人,末将认识他,不会有假。”

韩冈也轻轻颔首,算是比较满意。

七百骑不算少了。而且折家子弟兵精锐不输西军中的选锋,以征战论,甚至更胜上四军一筹。与秦琬的那三五千人配合起来,就能像一根硬骨头卡在辽人的喉咙里。

信使很快被领了进来,韩冈接过用蜡丸包裹、素帛书就的密信,展开看了一遍,上面的文字跟折可大方才所说几乎一样看来折可大刚才在外面问得很详细。

“折十六打算怎么做?!”放下素帛,韩冈问道。

那名信使头更低了一点:“我家将军命小人跪禀枢密:折克仁已经整顿兵马,只等枢密的吩咐。”

