\section{第九章 鼙鼓声喧贯中国(五)}

“因为天上出了彗星,这两天来,朝堂上闹得正是厉害。不过所谓天兆吉凶的话,为夫是不信的,所以闲来无事,就有心查一查过去的记录,将天文志翻了一翻。”

韩冈回手指着书架,“只是这么一翻,为夫就发现每隔七十六年左右——有时少个一年半载,有时多个一年半载——就会出现一次彗星。从始皇七年开始,一直到英宗皇帝在位的时候,一次都没有错失过。而往前,其实还有两个记载,‘秦厉共公十年,彗星见’,这是在始皇七年之前两百二十余年,差不多是三个七十六年。再往前,《春秋》中有‘秋七月有星孛入于北斗。’之语。这是在鲁文公十四年,离着始皇七年,差不多有五个七十六年。只可惜中间缺了几段,不知是史家遗漏,还是当时没有出现。”

“当然喽,说不定也有可能那几次彗星造访,鲁地正好是阴天,毕竟就是京东的那么一小片地方。可惜晋之《乘》,楚之《梼杌》都没有流传下来,”

《春秋》是周时诸国国史通名,但流传下来的春秋是鲁国国史,孔子为鲁人,他也只能笔削《春秋》。不过各国国史还有别名,在《孟子》中有载,晋国国史名为《乘》,楚国国史名为《梼杌》,可惜都没有孔子这样的圣贤帮着记录、流传,最后消失在历史之中。

“可其他的时候也有彗星。”

“道理很简单,彗星不只一颗!当然,也不是每次来的都是新客。反正总有一颗彗星会按时而来。而其他的彗星记载,也许有缺漏,如果补全的话,应该也能找出规律来。”

“依官人的说法,如今的彗星就与灾异无关喽?”王旖兴奋的问着。

韩冈点了点头。哈雷彗星的周期,在后世不知道的人可不多。既然心中有数,从史料中找起来当然容易。

“镇星【土星】周天二十八载,岁星【木星】周天十二载。与其说彗星是昭示兵祸的恶兆,还不如说是依时巡天的星辰。如同太白、岁星、镇星这样的行星一般,周天而行。只是有的隔三岔五,有的则是几十年一轮。为夫找出这一颗是最为稳定,记录也最全,正好七十六年一轮回。”韩冈长叹息,感慨着,“并非世人多愚,只是没有去想。只要有心之人将历代所见彗星列出年表一看,就能知道所谓恶兆乃是穿凿附会罢了。所谓格物,就是要格出道理,革除虚妄,多思多想,不可人云亦云,附会俗论。”

韩冈靠在交椅靠背上,十指交叉,双手就放在小腹上。沉沉的语调诉说着道理。晕黄的灯火映在眼中,双瞳却更显幽深,仿佛满藏着智慧。

王旖和严素心看着韩冈,两张俏脸忽然一齐都泛起了晕红。她们的丈夫感慨着世人不思不想、庸庸碌碌的时候,似乎就是在俯视着芸芸众生,看似淡漠,但又有着几分痛心。这样的姿态,让她们的心中都不由得涌起一阵崇拜——她们只是不知道韩冈的立足之地有多高。

严素心定了定神,只觉得两颊烧烫:“可不是还有很多时候,天上来了彗星,天下就有了灾异?”

“许多上天无兆的时候,不照样有灾异吗?祥瑞频出的年代,也不见灾害少过。”韩冈摇着头,“其实都是附会,天下这么大,总能找到对应的灾异。就算明天南边出了乱子,也只是巧合,否则根本无法说明为什么每隔七十六年必有彗星。”

“官人,那爹爹他……”王旖心中阴云尽散,喜笑颜开。

“没用的。”韩冈没等王旖说完,直接摇头,“此事只是为夫的揣测,并无实证,上一次此颗彗星出现是在十年前,治平三年三月己未。想要确认为夫的猜想,则要等到六十六年之后。天子是千万岁寿,我们做臣子的可是很难看到六十六年后的事。怎么可能取信于人?道理的确说得通,可想要作为证据,却是远远不够。”

韩冈他原本希望这一次出现的是后世的哈雷彗星,这样他就可以帮着王安石一把,也顺道给格物之说添砖加瓦。可惜他费了一番周折后才发现,原来哈雷彗星早已经离开了十年。

既然是无法即时证明的推测,韩冈也不会在有争议的风尖浪口之上,将他对彗星的看法拿出来做凭证,这等于是给对手一个攻击他的机会。不过日后他会依着如今士人的习惯,写些笔记,将这猜想写进书里,等待几十年后再来验证。

……………………

彗星一直悬在头顶上,已经有五天了,但人们议论依然不减。

“幸好是凌犯轸宿,要是应在北方可就麻烦多了。”

“再怎么样天子也不会因为天上出现彗星,而令前线撤军。”王雱的声音轻微,透着虚弱。

入冬之后,王雱身体就有些不适。原本他体质就不好,在江宁时就已经是几次卧床,上京的过程中,顶着烈日更是大伤元气。只是入京后,因身负重任,需要辅佐王安石秉政,反而振奋起精神来,看不出有半点病态。但最近这段时间,又开始觉得身体变得沉重,到了天上出了慧星,王雱殚思竭虑,欲设法朝堂议论,但精力不足,终于一头病倒。

韩冈坐在王雱的病榻前,他面前勉强在床上坐起来的大舅子,脸色泛着不健康的青白色,双颊也深深的凹陷了进去,探出被子的双手,干瘦得皮包着骨。看他现在的模样,就算这一次病愈,身体不好生的将养个一年半载,依然恢复不了健康。

“横山今日情势如何?有没有什么消息?”王雱因为医嘱要他多休养,少耗神,王安石这两日为了儿子的身体着想,也就尽量避免跟他谈及政事上面的消息。

“就算有金牌加急,我们也只能知道四天前的回报。”

“玉昆!”王雱不愉的提高了嗓门。

看来自己还不是会说笑话的料,韩冈摇摇头,“并没有正经消息,不过今日白天的联络,种谔已经将六十余架霹雳砲全都运了上去。近百里的山谷狭道,加上党项人占据罗兀城后,又大肆破坏联通南面的道路,就算是将霹雳砲拆散了上运,普通的随军转运,就算再多一倍的时间,也不是这么容易就能做到的。”

“记得管着随军转运的是鄜延经略司的机宜文字游师雄吧?”王雱想了一想,道,“是几年前在广锐军叛乱时立了大功的?”

“游景叔与我份属同窗,同在子厚先生门下,不过他比我入门要早得多,出师也早。”

“横渠门下,文武双全。”王雱靠着背后的靠垫,轻声笑道:“与胡安定【胡瑗】门下相比,倒也不遑多让。”

“情势迫人,也是逼出来的。谁叫我等生在关西。”

王雱笑了一笑,“如果这一次能够如愿以偿,朝堂上的局面就能好上许多。军功才是根本,天子这些年苦心积虑,就是为了对西北二虏战而胜之。可笑富文之辈,空食朝廷俸禄,不能使天子免受二虏之辱。”

“元泽,不要多说这些事了。”韩冈叹了口气,“你这是元气不足,要以养生为上。心神耗用过度,这病怎么能见好?”

“……若父亲能得玉昆你全力匡助,愚兄如何需要日夜忧心?”王雱眼神忽而锐利起来。

“元泽你太看得起小弟了。何况新法当助、可助、须助之处,韩冈何曾袖手旁观过?”韩冈用反问来回答,轻轻避过了王雱的要求。

王雱叹了口气,闭起了眼睛,不再言语。

韩冈从王雱的房中出来,王安石就在书房里等着他。一本书放在面前,就随手哗哗的翻着,显是心浮气躁。

“玉昆,依你之见,现在情况如何?”见到韩冈,他便立刻问道。

“以小婿之见,鄜延路那里若能尽速见功就好了。只要横山见功,一切攻击皆是虚妄。”

王安石摇着头,“我是问大哥儿的病究竟如何?”

韩冈怔了一下,看了王安石一眼,腰背驼着,很是疲累的样子,须发苍苍、脸色皱纹尽显,分外显着苍老。心中不无感慨,毕竟是父子连心:“小婿不通医术,但看元泽他的病,应该还是调养为上,不能劳累过度。”

“是吗?”王安石声音暗哑,用手按着额头,心底隐藏着的痛苦再也遮掩不住。韩冈的话,还有医生的嘱咐,话里话外其实都是在说他长子的病情已经很严重了。

王安石现在表现出来的脆弱,韩冈还是第一次见。虽然是撑起一国大政的宰相,但还是一个有血有肉的人,会为自己的志愿难得支持而感到愤懑,也会为儿子的身体而感到痛苦。

“玉昆。”过了好半天,王安石才又开口,这时候,他已经收拾了心情,心底的脆弱完全看不到了,“军器监中的情况如何?”

“一切如常。板甲、斩马刀、神臂弓这三样都在用最快的速度来生产。如果还要再求快的话,就得将监中工匠分作早中晚三班,昼夜开工,不过给付的工钱要多上一些。另外现在的问题是生铁供给不足,河北要尽快推广焦炭炼铁,徐州附近也要尽快找到石炭矿。还有就是猛火油,有了焦油之后,猛火油作的产量也翻了一番。军器监中,一切安好。”

“也只有玉昆你这边能让人安心了。”王安石点头赞了一句,眼神变得坚定起来,“有强兵,有利刃,有坚甲,横山必取。灭亡西夏,也是指日可待。”

