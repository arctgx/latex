\section{第35章 重峦千障望余雪(三)}

文彦博的话近似于威胁,赵顼心头隐怒。

如果有御史在殿中,少不得会站出来斥责……就像章惇现在做的,“文彦博语胁天子,目无君上!当下有司治罪!”

赵顼没理会章惇的话,冷眼问着文彦博:“文卿对河湟设立经略安抚司又何看法?”

文彦博都不在乎天子的怒气,“臣即是备位宰辅,朝事有何事不可议论?!陛下既然觉得臣无议事之权,臣又如何能立于朝堂?”

他走到大殿正中,屈膝跪倒,直着腰背,一点不让的与赵顼对视着:“臣老悖无用,执掌密院数载,不能使陛下顺天应人,徇祖宗正道,即无补于朝事,又愧对于先帝,无颜再留于朝堂。臣……请出外就郡!”

赵顼皱起眉头,文彦博这是在要挟吗?一点猜疑让他口气变得很不客气:“文卿主管枢府,数年来多有功绩。河湟决战近在眼前,枢府岂能少得了文卿主持。”

赵顼的话,让文彦博心冷了下去,天子的这番话就是在表态,河湟拓边容不得反对,看起来事情是不可能挽回了。他再行叩首:“臣年老力衰,密院事务繁剧,已是不胜其劳,还请陛下另选贤能。”

文彦博坚持请辞,赵顼看不出是真情还是假意,只是在心中盘算着利害关系。

在重用王安石的同时,他一直将最为激烈的反对派文彦博留在朝堂上,就是要维护朝堂上的平衡,但如今有了冯京、吴充这两个跟王安石并不和睦的执政,赵顼觉得,他已经不再需要文彦博留在朝堂上。

作为元老重臣,文彦博的确有普通臣僚比不上的威望,就如河口处镇河的铁柱,在一些突发事件上,能镇压得住人心。可现在,王安石已经能够取代元老重臣在朝局动荡时安定人心的能力。

今年年初,契丹人送信来掺和横山那边的战局。当时赵顼慌乱不已,是王安石给他吃了定心丸。而文彦博虽然对契丹人的要挟不屑一顾,但还趁机让赵顼从横山撤军。

两相对比,赵顼对文彦博的作用也就看淡了,只是依例他还要出言挽留,“文卿是三朝宰执,朕之左右,少不了卿家的辅弼。卿家的请辞,朕是不会答允的!”

文彦博一番闹腾,崇政殿议事也议不下去了,向赵顼叩拜之后,一干重臣都回各自的衙门,而文彦博则是径自出宫,回家写他的请郡奏章去了。

结束了议事,赵顼今天却没有留下王安石,只把参知政事的王珪留了下来。

偌大的崇政殿中,除了几十个如壁画一般的卫士、内侍,就只剩君臣二人相对。

赵顼一直沉着脸,没说话。王珪也不敢先开口,惶惶不安的垂头等着天子发话。

过了不是多久,赵顼打破了沉默,“王珪,你觉得朕不该提拔韩冈吗?”

“诚如陛下先前所言,韩冈有功社稷,不能不赏。不过他年纪尚幼,任官太短。进用太速,恐有后事难终之忧。”王珪一边说着,一边看着赵顼的脸色。见着赵顼的表情突的冷了下来,他心头一紧,立刻把方向调转:“让韩冈处于风尖浪口之上,并非优待功臣之道。以臣愚见,不如依功封赏,以示朝廷之公。而韩冈入京面圣的事,暂且搁置一阵,也防着木秀于林。”

赵顼脸色变得好看了,王珪算是说到了他的心里,处理方法也不错。

升官还是要升的,赏罚不均是朝廷大忌。但暂时不要让韩冈进京来,把他拉到风尖浪口上,对其也的确并不是一件好事。太过年轻的朝官,资历又浅,很容易成为众矢之的。若是韩冈受到太多的攻击,肯定会影响到明年河湟的决战。

韩冈暂时就不见了。选人转官时虽说是必须陛见,可这陛见的时间,赵顼要拖上一阵也没人能说不对。

王珪难得有机会留对,却也不肯放过这么好的时机,进一步的向赵顼建言,“陛下,明年河湟大战在即,届时关西各路精锐将齐集河湟。王韶、高遵裕虽是,但二人如今品位太卑,不足以慑服众将……”

“以王卿之意,那是要设立经略安抚司喽?”

“陛下圣明!”王珪一向擅长揣摩圣意,赵顼前面既然已经表明了态度,他当然不会跟天子拧着来。何况庙堂运筹之功,他也想分上一份:“臣请于河湟之地设经略安抚司,王韶为经略使,高遵裕为兵马副总管,以高官显禄佐其声威!”

……………………

屋外细雪纷飞,隆冬已经降临到河湟。

韩冈坐在一张交椅上,旁边炉火正旺。手上拿着本汉书,慢慢的翻着。手边的银杯中,有着半杯羊乳酪,温热得带着点酸甜的香气。

屋门外突然响起脚步声,王韶推门走了进来。看着韩冈的闲适,便笑道:“玉昆,你好自在。”

韩冈连忙跳起,向王韶行礼。

王韶摆了摆手,示意韩冈坐下,自己坐到韩冈对面,对着火炉烘着手,说道:“文彦博去了河阳。”

“陛下还是放他走了?!”

王韶点了点,“临走时还升了司空和河东节度使。……这已经是使相了。”

北宋的职官表中,并没有宰相这个名号,但许多官职都可指代宰相。同中书门下平章事自不必说,此是政事堂中真宰相才有的职衔。而侍中、司空这些名号,也可说是宰相,只是没有实职。一个宰相的头衔,加上节度使的加衔,便是使相,班列位置犹在宰相之上,但基本上都是元老重臣被清出朝堂后,给的安慰奖。

“文相公没有自请致仕吗?”韩冈问着,前面的范镇、富弼,被赶出朝堂后,可都是陆续告老了。文彦博也都六十多往七十走了,今次被请出庙堂,脾气大点的就该顺便就把告老的折子上了。

王韶摇了摇头,王安石给他的信中可没有写:“韩稚圭【韩琦】没告老,而富彦国【富弼】也是先判了一任汝州之后才求退的。文彦博大概还要再等几年,说不定还能再起复。”

“文相公当真是老而弥坚!”韩冈由衷的感叹着,文彦博在朝堂上与新党斗了几年,也算是劳心劳力了,如今出外后,还打着东山再起的主意,这份韧性,就值得他们这些小辈好好学习。

“玉昆你今次能晋身朝官,也多亏了没有文宽夫的阻挠。”

韩冈笑道:“说得也是。”

经此武胜一战,王韶继续升官,高遵裕继续升官,今次出战的众官、众将,人人得受天霖。而韩冈也终于脱离了有功不赏的厄运,先因功擢为安化军节度判官,然后,以天子特旨转官。因为节度判官是选人的最高一级,一旦转官,就不是京官,而是朝官。

选人和京官在名义上是平级的,只是任官的位置不同而已,所以在转官时,高阶的选人并不会转到低品的京官上去。而是晋上一阶,升到更高一级的京官上去。只是到了最高阶的节度判官这一级,京官中并没有更高一阶的官衔对应,便直接转为正八品的朝官。

——正八品的太子中允,也就是朝官的最末一级。

这是王韶两年前担任秦凤经略司机宜文字时的职衔,现在韩冈都坐上了。

但韩冈如今的职位并不在当初的王韶之下,他如今同样也是经略安抚司的机宜文字——新成立的熙河路——同时又是改名巩州的通远军的通判,也就是留在陇西县,而不是王韶之前推举他的武胜军。

前几天,从京城传来的封赏,与王韶、韩冈他们预计的完全不同。

韩冈曾经以为朝廷对武胜军的处置,是改个名字而已。好一点的情况是维持军一级的建制,差一点的,大概就是改成城或者寨,隶属通远,相当于县的编制。

王韶希望韩冈能主持改编后的边地大城,就是让他能够依靠这个任命而顺利转官。

但实际情况却让人出乎意料,朝廷对武胜军的处置竟是升为州——熙州。而原来的通远军,也升为州——巩州。

“大概是捷报上说得太过了一点。”王韶在拿到诏书后,私下里对韩冈这么说着。

官军在洮水边的实际控制区,其实只有临洮周边的一小块,以南关堡、北关堡为界限,而西面仅仅是攻破了木征打造的营寨,贴着洮水筑下了一座小寨。洮西大部分地区还在木征手中。至于武胜北方,包约正跟禹臧家的军队,互相清理亲附对方的蕃部,打得一团乱。

但在呈给朝廷的捷报中,却把这些用春秋笔法轻轻掩过。

所以新设立的经略安抚司,便是熙河经略安抚司,也就是以改称熙州的武胜军为核心,且把还没夺下来的河州,都算了进来。

这个名字的用意,就是绝不容许失败。一旦河州攻取不下,朝廷的脸面便要丢尽,而熙河经略司也不会有好下场。

“玉昆,这巩州之事可就要靠你了。”王韶说道。

