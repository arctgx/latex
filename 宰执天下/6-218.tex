\section{第27章 更化同风期全盛(下)}

东京辽阳府,一贯多见跑马的汉子唱歌、呼喝,却很少有朗朗书声。

但近几年来,在辽阳皇宫附近的一片修起不久的楼阁中,却不时的传出稚嫩的读书声。

这是契丹王家的宗学,数百名以耶律、萧两家为主的贵胄王孙们在此处读书。

尽管是新设立的学校,但规模和地位,都是前所未有的高。

连皇帝耶律乙辛现在都来到了这里,走进了教室中,拉着一名小学生问着问题。

“七乘八是多少?”

答案脱口而出,“七八五十六。”

“八十七除五,余数是多少?”

小学生想了一下,“余数是二。”

“从析津府到辽阳府一千五百里,骑马每天能走两百五十里,要走几天走完全程?”

小学生拿起纸和笔,飞快的列了一个算式,然后告诉皇帝:“六天。”

耶律乙辛满意的点了点头,笑着道:“答得好。”

旁边听考的祭酒和讲师如释重负,悄悄地擦了把汗,提醒圆满的回答了三个问题的小学生跪下来叩谢天恩,

他们知道耶律乙辛的捺钵回到了辽阳,却没想到皇帝如此重视宗学,不仅仅亲自来到学校中,而且还不顾尊卑,亲自考核学生。幸好问题不难,也幸好被挑到的孩子是个聪明的。

正在他们庆幸的时候,耶律乙辛又选了一名学生,

“大辽有几道?”

“中土有五京道,外有高丽道和日本道。”

“析津府在哪一道?”

“南京道。”

“认识多少字了?”

“三字经和千字文都学了,国文也能读写了。”

“把钢铁,煤炭,火枪,火炮,几个字写出来。”

刷刷刷,十岁出头的小学生在黑板上端端正正写下了几个汉字。

幸好,幸好,成绩优秀的好学生都在这个班上。

宗学中,依照成绩高下,分为外舍、内舍、上舍,这是从南方学来的制度。

现在皇帝所在的,正是上舍生的教室。在耶律乙辛进来的时候,刻意将他引到这里,才没让排在后面的外舍生漏了马脚。

接连考校过了四个学生,无论是数学、地理、自然还是汉学,每一个都回答得很好。

耶律乙辛大喜过望,回头便夸奖祭酒和讲师,“教得好,是用心了。”

皇帝的一句夸奖,代表着从天而降的好处。

被提过问题的学生,一人一匹绸缎,一串银钱,外面还有十只羊。宗学中,没被抽到的学生,也有一串银钱——这是从日本挖来的白银所铸的新钱。不仅在辽国国内备受欢迎,在宋国那边,也成了许多人家储蓄的货币。

而这些学生的老师,则是一个高丽婢,一个倭奴,百贯银钱,百只羊,五匹马,此外还有官职和俸禄的加赏。

至于祭酒,更是被赏了一个夷离堇的称号——虽说如今这个称号已经没有过去那么重要,但依然有着极高的地位。

皇帝亲临宗学,又厚加赏赐,这是向世人昭告他对教育的重视。

过去辽国是没有这类的学校,但现在有了。而且不仅仅有了宗学,耶律乙辛还将国中各族贵胄家的适龄子弟都招致捺钵,将其编为一军,号为神火军。教以枪炮、弓马,日夜操演,又五日一犒,十日一赏,弄得这些年轻人人人归心。

‘谁说契丹人不如汉人聪明?’耶律乙辛心情好得就像是在飞一样。

尽管他知道宗学的祭酒玩了把戏,可是宗学的学生本就不如那些贫寒学子刻苦,能有一部分人达到现在这种水平,可见他们还是用心了。

参观过学生们的弓马表演,更加满意的耶律乙辛召集了宗学中的所有学官,大加褒奖之后,又嘱咐道:“除了考试,公布成绩,还要在不同科目进行各自的考试,比赛高下,排前面的重赏,排后面的重罚,要人人争先,认真学习。要告诉他们,朕会一直看着他们的表现。只要一直学得好,朕的金帐就有他们的立足之地!”

“汉人赛马是绕着圈跑,我们契丹人赛马是直线跑。汉人比赛踢球,我们比赛射箭。汉人在读那些没用的经书,只分出一部分心思学气学,我们就要用全部心思去学气学。至于这个……”他拿起一本厚厚的册子,《科学》二字在封面上分外鲜明,这是耶律乙辛在宗学中发现的,“比绸子差点,比厕筹好些。”

做了皇帝之后,耶律乙辛生活愈加奢靡,学着南朝的皇帝,方便后用绸缎来擦拭。不过他手下有金矿、有银矿、有榷场,更重要的是有这个幅员万里的国家,即使没有了岁币,但他的身家,还是吹气球一般的飞涨。而且铸币带来的好处,也尽归其手,拥有充裕的财富和军力,这让耶律乙辛的地位日渐稳固,也让他的说话一言九鼎,国中无人敢于违逆。

啪的一声重响,他把书册砸在了桌子上,“以后就放在茅厕中!”

收订《科学》的一名教授,面色如土。耶律乙辛的这句话,宣判了他前途的死刑。

回到了城外捺钵的御帐中,耶律乙辛对儿子和张孝杰又说起《科学》的事:“其实也是聪明人,可惜没用对地方。”

学习韩冈的手段,通过《科学》来发展新学,集合众人之智,反过来与韩冈对抗,这当然是反败为胜的好手段。耶律乙辛还听说南朝那边有谣言,说日后科举选派的考官,都可能来自于在《科学》上发表文章的作者。所以《科学》的销量,没几个月就要赶上了《自然》了。

汉人蠢吗?一点也不,但他们偏偏就在这种地方花心思。

哪像大辽这边,已经是齐心合力去学习南朝出色的地方——当然不包括哪些破烂经书。

一名细作从南朝学来了蜂窝煤的做法,回到辽阳后就开始做买卖了。

石炭场的煤粉,混上黄泥,根本就不用钱,先做成的胚子,然后在压制成型,中间用人力和畜力就够了。造出蜂窝煤价格也低得惊人。蜂窝煤的炉灶,结构、外形也都十分简单,成本很低。现在辽阳城中,几乎家家户户都开始用蜂窝煤烧水。

南朝的大臣们要是能将他们勾心斗角的心思分出那么一半用在正事上,大宋会变得更加富庶,而大辽可就要危在旦夕了。幸好这一切都不存在。

韩冈为了移民边疆,只能去抓乞丐。而耶律乙辛要让契丹贵胄放他们手下的汉人奴隶为平民,只要一道诏令,这就是差别。

不是那些贵胄能体谅国事,而是他们不敢,怕耶律乙辛手上的大军,也怕他手中的煤和铁。

“煤和铁,就是一切。”

就连张孝杰道知道,这是《九域游记》中的一句话,而且流传得很广。

在宋国那边,这句话备受指摘,没有人,没有粮,空有煤铁又有什么用?

但他们不想想,若是没有煤和铁,那可就什么都没了。

别人有了煤,有了铁,就有了精钢,就有了各种各样用精钢制成的器物,刀枪、甲胄、火炮、车辆、船只。当敌人拿着精钢的武器过来抢人抢粮,没有煤和铁,怎么抵抗?

宋人新近灭掉的大理国,不就是因为没有钢铁,所以才被灭的吗!刚刚扫平了北地,新编了一批宫卫,不也是靠了煤和铁的支撑!

辽国之中,从耶律乙辛以下,对韩冈的一句‘物尽天择、适者生存’奉若圭臬。排除掉其中腐儒的老生常谈,绝对是千古以来数一数二的至理名言。

恶劣的自然环境,狂风,暴雪,强盗,生活在大草原上,只有去适应,活下来的就是赢家。

“那是,可惜他只是臣子,这是最值得庆幸的一件事。”耶律乙辛捋着胡子放声笑道,“若他是皇帝,我等就只有脱下冠冕,穿上白衣,去开封朝拜了。幸好他不是。”

等到那个小皇帝登基,韩冈再无可能回到朝中。就算他想学自己,可他能有那个胆子吗?能掌握足够的军队和大臣吗?南朝的风气又能容忍吗?

拿出了节度使之位来悬赏,已经有了一个为耶律乙辛铸炮的大匠做了节度使,还给了他头下军州,如今已经是起居八座的贵官了。现在南京道上的汉人,不是自己去学做工匠,就是让自家的儿孙去学。

“可若不是有陛下,即是韩冈不在了,大辽亦有可能为南朝所灭。”张孝杰对耶律乙辛的深谋远虑没口子的称赞,“可如今,等到十年之后,国中的能工巧匠将层出不穷,又何惧南朝的火器犀利?”

“这些还不够。”耶律乙辛很坚定地摇头,“给我昭告天下,若能有谁给朕带来放在船上使用的蒸汽机,朕不惜王爵之赏,列土封疆!不论是大辽,还是南朝,都要把这个悬赏给我传遍!”

在儿子和心腹大臣惊讶的目光中,耶律乙辛有重复了一遍,既然是韩冈说的,那么蒸汽机必有大用的,与其与南朝争来争去,还不如开出一个让人无法拒绝的筹码。

“黄金台朕能筑起,南朝呢?”
