\section{第四章 素意兰心得君怜(下)}

【第三更,求红票,收藏】

既然严素心的身份家世已被查清,韩冈也就可以放心下来,不必担心半夜醒过来时,面前出现一个拿着尖刀的黑色剪影。

只是他这时倒是佩服起陈举的胆量,能把一个仇家的女儿放在家里,不过严素心当时年纪应该还不大,又是女孩子家,估计陈举才有这个胆量。换作是男丁,大概就会给装进麻袋扔藉水里去了。

“三哥,你现在还没吃吧?”韩阿李终于想起儿子大概还饿着肚子,“素心做得一手好菜,也会做汤,也会烹茶,你都可让她试试。”

韩冈点点头:“随便弄些就可以了,快点就成。”

严素心正等着韩冈的发落,听到韩冈让她去准备饭菜,知道他这是答应了。抬起头,泪水还挂在脸上,却已经笑了起来,“素心明白了,官人请少待。”

少女转身去了厨房,韩阿李便急着问儿子:“三哥儿,你看素心如何?”

严素心无论身材还是相貌,都是难得一见的出色。韩冈也不是七老八十、古井不波的年纪,当然免不了要动心。不过在听说了严素心的身份后,他便有些犹豫。

严素心是士人家的女儿,虽然他父亲是因赃罪而丢官去职,被编管琼州。但这是陈举的陷害,如今陈举族灭,他过去陷人于死地的案子,不用说都可以翻案。

把一个流囚的女儿收入房中做妾,不算什么大事,但收一个士大夫的女儿,传扬出去,在士林中却要受到不小的压力。

韩冈盘算着利害得失,却没想到才一转眼的功夫,严素心便端了一碗热腾腾羊肉汤,两块胡饼和一盘子炒豆芽上来。

“这么快?”韩冈微微吃了一惊。

“本就是准备好的,官人回来,只要再炒个菜就够了。官人且垫垫饥,一会儿就入夜了,晚上素心再做些费功夫的。”

严素心把碗筷摆好,看着韩冈拿起筷子,手攥得紧紧,双眼睁得老大,紧张的等着韩冈的评价。

韩冈先喝了一口汤,羊肉的鲜味在嘴中漫开,却没有半点腥膻,也不知炖了多久,羊肉嫩得入口即化。豆芽是掐头去根,炒得晶莹剔透,看着就是美味可口。胡饼即是烧饼,芝麻如今称为胡麻,也是烤得一般金黄香酥。

说起来,的确比过去家里的饭菜要强。但过去做菜的是韩阿李和小丫头,韩冈可不会笨到说过去的菜实在比不上严素心的水准。

“蛮不错的。”韩冈点了点头,很平淡的说着。筷子动得却很快,转眼便吃了个精光。

稍稍把饥肠辘辘的肚子填饱了一点,韩冈接过严素心递上来的擦嘴的手巾,开始期待晚上的饭菜。

把碗碟撤下去,严素心又给韩冈端来一盏消食的茶汤。莹白如玉的一双纤手掀开茶盅,深褐色的乌梅汤在白瓷盏中荡漾:“官人轻慢用。”

韩冈轻抿了一口茶汤,汤水酸甜适口,的确能开胃消食。喝了两口,他问着严素心:“不知严小娘子在乡中还有没有亲族?”

严素心的脸色冷淡下去:“当年爹娘受苦的时候,可没哪位叔伯为素心的爹娘说过半句话。这样的亲族,有不如无。”说着,她眼中又噙起泪花,“官人可是要赶素心走。爹娘都不在了,素心已是无处可去……”

“胡说什么?!安心住下就好!”韩阿李一声断喝,“既然都定了契,你也不想走,哪个会赶你走?三哥……你说呢?”

韩阿李的声音中带着杀气,仿佛韩冈要说个不字,她就会杀去厨房,抄起擀面杖。

严素心双眼红红的,雨带梨花,楚楚可怜。韩冈看了她,心中也是不忍。自己是为她全家报了仇,她甘愿以身相报,也没人能说不对。他点点头:“严小娘子便住下了就是,我也只是问问。好好的,谁也不会赶你走。”

韩冈在这里跟严素心和韩阿李说话,而小丫头却不见踪影。自己回来都有一阵了,韩云娘也不出来,平常可不是这样。

心中有了挂念,他跟韩阿李告了声罪,起身往后院书房去。身后严素心跟出来,“官人有什么想吃的,跟素心说一声,素心好去筹办。”

韩冈摇头笑笑,“倒没什么想吃的,我一向也不挑。你看着爹娘的口味,随着他们做。”

一进书房门,就看着小丫头搬了张小木墩,靠着窗边坐着。手上拿着块尺许见方的绿色绸子,正一针一线的在上面绣着花纹。

韩冈开门进门,韩云娘头也不抬,专心于手上的女红。等到韩冈走到身边,她才问了一句:“三哥哥吃过了吗?”

只是一句很平常的问话,但韩冈还是从中闻到了一股子浓浓的酸味。

这还真是有些让人头疼。清官难断家务事,要安抚吃醋的女孩子,本就是桩苦活计。韩云娘性格温婉可人,并不代表她不会吃醋。想必韩阿李已经把她的想法跟小丫头说过了。韩云娘没有反对的权力,但心中肯定是不高兴的。

韩冈惯于单刀直入,一把将她抱起来,在她耳边笑道:“吃哪门子飞醋?”

“吃醋?没有啊。”小丫头靠在韩冈怀里,也不动弹,手上的针线却不停。

看着韩云娘捏在手指上的银针闪烁,韩冈的心中有些发毛。小丫头的身子骨还是孩子般的纤细,个头也只到自己的胸口,但闹起脾气来,却是跟大人一样,让人心惊。

“还说没有……”韩冈硬是把她的身子转过来。

小丫头与韩冈面对着面,手上的针线动不了了。但她头低着,就是不说话。韩冈略略强硬的托着她小巧可爱的下巴,强着她把头抬起来,清丽的小脸上什么表情也没有,但韩冈紧紧盯着她,就见着她的眼眶渐渐红了。

韩冈怜意大盛,轻搂着云娘纤弱的身子,轻柔在耳边说着:“你放心。”

“嗯……”小丫头轻轻应了一声,低头在韩冈怀里感受着从他胸膛传来的温暖。

韩冈仰头叹了口气,齐人之福还真是不好享,都是要靠水磨工夫了。不管怎么说,在他的心里面,云娘还是排在第一位的。不论是严素心,还是周南,都比不上。

到了晚上,韩冈见到了当日得了水痘的小女孩,现在她脸上已经看不到病时留下的痕迹。长得很清秀,很老实的跟着严素心请安问好。听说她也是父母双亡,也难怪同病相怜的严素心会收养她。

接下来的几天,韩冈白天去衙门里,晚上回来读书。严素心饭菜做得好,而且烹了一手好茶。分茶斗茶,韩冈在京城时,经常在路边上看到有闲人在比拼着技术。只是他对此一窍不通,也没精力和时间去学。没想到严素心倒是个中里手,也是让韩冈好好的享受了一番。

不知什么时候,严素心和韩云娘分了工,韩云娘人多在韩冈的书房中,严素心的主阵地则是厨房,闲暇时则都是跟在韩阿李身边做女红,而韩冈的夜宵则是两人一日一换的分担。另外,一是由于心中在意小丫头的感受,另一方面,韩冈也不想表现得太过急色,有些事并没有发生。

这段时间里,秦州城内则很平静。李师中虽然对王韶的自作主张上书进行了弹劾,但实际上,他在公开场合并没有再说王韶的什么不是。只有窦舜卿跳得厉害,有事没事就骂王韶。有一次韩冈在衙门里遇到,还被他籍故训了一通,让韩冈很遗憾为什么中风的不是他。

而说起中风,向宝却是令人惊讶的康复了起来。从他在永宁寨发病,到现在才不过十几天的功夫,他已经能站起来被人扶着走路了。这个复原速度实在让人吃惊不已。来给向宝诊治的几名名医,也都说他们从来没见过中过风后,还能恢复得这般快的。

不过等到他们听说向宝发病时,韩冈就在身边,便一齐摇头说着难怪难怪,那可是孙真人的弟子啊,难怪能保住向钤辖的性命。对于医生们的误会,向宝和他的亲信幕僚们差点大骂出口,韩冈那厮明明什么都没做!他根本就不懂医术。

但这番话一传出来,反而有人说他们忘恩负义。韩冈虽然说自己不懂医术,但他在疗养院救了不知多少伤病,今次随军出征,一来一去半个月,军中也没几个生病的,难道这些事情都是假的不成?

现在向宝中了风,却一转眼的功夫就又站了起来,不是向宝发病时就在他身边的韩冈的功劳,难道还是向宝他家常常烧香拜佛的关系?这世上中风得多,拜佛的更多,拜佛又中风从没少过,也不见他们转眼就能走。

韩冈听到这个传言,却是苦笑连连,向宝那是底子好,跟自己哪有什么关系。但人们总喜欢比较耸动的新闻,向宝因为身体好,撑了过来,当然不如孙思邈的私淑弟子妙手回春把人救起听起来有趣。

这真是人在家中坐,祸从天上来,韩冈最怕的就是有人把自己抬得太高,日后摔下来可不得了。害得韩冈去衙中的时候,都得跟人不停的解释——我真的什么都没做——但信的人还是不多。

就在韩冈跟着向宝一起大骂的时候,王韶终于凯旋而回,几辆囚车载着托硕部的族长首酋们招摇过世,而一众有功的蕃部首领也跟着一起回来。

