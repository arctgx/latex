\section{第39章 欲雨还晴咨明辅(39)}

一番议论之后,就在殿上将出兵高丽的事情定下了。

看起来是有些儿戏了,但那毕竟不是去打仗。不过几条船、千余人去巡边,仅仅是走得稍稍远了那么一点。事情不妙,转身就能跑的,辽国的骑兵也追不上来。

些许小事,宰辅们都不介意给韩冈一个面子,这份人情之后肯定是能拿回来的。

不过韩冈回去还要写札子,口说无凭据,公文上的程序必须要走。曰后查验档案,整理国史中的个人传记,也需要各方面的资料来证明。

当然,写在纸面上的文字,绝不会是跟辽国抢食,而是义正辞严、正大光明的去援助在辽国铁蹄下苦苦挣扎的高丽。

刚刚击败了辽国,又夺回了大片的土地,京城脚下的百万军民心气正高,援救外藩不被辽国侵扰,保其传承不绝,这就是天朝上国该做的事。

到了如今,京城内外还怕辽国的也没多少了,贤臣名将无数,三军又各个精锐,最不成气候的京营都能将辽人打得丢盔弃甲,河北运气差一点,但也是将辽军的主力给堵在了三关之外。

现在高丽使者求到了京城来,朝廷出兵表示一下,甚至能得到百万军民的欢呼声。

接下来的两三曰,韩冈都在炮制他的奏章,朝廷需要一些准备,高丽使臣那里也要稍稍晾一下,韩冈也就不用着急。

而这几曰,金悌、柳洪和他们的随从都被约束在驿馆中,免得一群人乱跑乱说,乱了朝廷的计划。

金悌在出发前,曾从国主手中接受了一笔丰厚的财物,让他抵达开封城之后,用来收买大宋朝廷里面的高官显贵,好为高丽说话。可他现在根本连门都出不了,到底能拜见哪一个?

但金悌所不知道的是,他一登岸,他带来了什么人,什么物,都被打探了一遍,宰辅们都很清楚,金悌手上的那些都是贵价货色,可那终究是在高丽,在大宋,可就要差上一筹。此外的一匣子黄金也算不上多贵重。想拿来买好大宋宰辅,根本是痴心妄想。不过这笔钱若是给下面的黑心官吏骗走了,朝廷的脸面上也过不去。

蔡确专门下令,让同文馆官吏严加看管,如果让高丽使者或随从溜出去,值守的官吏从上到下都是严惩不贷。政事堂那边还私下里传话,那个严惩不贷,就是流放沙门岛,绝不宽宥。

第二天,就有一名随从得到了一个机会,潜出了同文馆,但没有几步便被拦了回来。很快一名被打断了四肢的同文馆吏员便被丢上马车,在同文馆诸官吏的目送下,离开了京城。看他的情况,莫说沙门岛,就是开封府界都不一定能出得去。

朝廷如此激烈的手段,让同文馆上下为之一凛,不敢再为钱币所引诱。金悌也战战兢兢的好半天。万一他的动作惹怒了皇后和两府,这一回可就只能打道回府。不过大宋的朝廷幸好根本就没理会他,只有馆伴使来陪他说话。

馆伴使是太常礼院的人,很普通的一个礼官,完全没有当年来大宋时所受到的礼遇。

金悌与这名馆伴使聊天时,更发现对方是个话篓子,什么都往外倒,根本就没有过去接触过的馆伴使那般稳重和斟词酌句,小心的不在言辞中留下可用利用的破绽。

所选非人,从这一点可以看出来,大宋朝廷根本就没将他们这群高丽求援的使者放在心上。

几天来,礼官都带来了最新的有关高丽的消息。

就在今天早间,新的信息被送到了——开京被辽人攻破了。

虽然说礼官当时也讲了,‘到底是不是谣言,还得再加以验证。’

可金悌已经不再抱什么希望。

从一开始,就有说开京陷落的消息,不过从时间上算,辽军就算跑得再快,也不过刚刚能摸到开京,但那样可是要沿途都是毫无阻碍的平原,而且寻找食物和饲料还不能浪费时间。怎么看都是谣言。

之后开京陷落的消息接二连三的传来,但仔细询问之后,从来都不是亲眼所见,而是道听途所。

但今天的情况就不一样了,登州知州密报,辽国攻下开京的消息已经为十余人所证实。这些证人都是从高丽逃出来的商人、官员,并不是一处来的。不同渠道都证明了同一个消息,那么也不需要再多的怀疑了。

金悌已经开始绝望了,如果没有一个朝廷,至少一个国王在高丽维持局面,以证明高丽国依然存在,大宋是绝对不可能出兵相助,甚至原本已经答应下来的军器,都会有了反复。

他如同困兽一般在,尽管同文馆的馆舍,远比他在开京的住宅要宽敞的多,甚至除了规模略小,建筑、装饰、摆设都远远超过高丽的王宫,但他还是觉得喘不过气来。

“喜事!喜事!大喜事!”

礼官几天来,第一次不是按照规定的时间登门作陪,更没有让人通报,直接冲进了金悌的房中。身后柳洪和一群随从跟着,都没把他给拦住。

金悌转过身来,苍白的脸上没有一丝期待,声音干哑的问着,“什么喜事?”

“大喜事啊,金大使。韩宣徽今天递了札子,说是要请朝廷出兵援救贵国!”

“韩宣徽?!”

金悌和他的同伴们都惊呆了。

“怎么,还不知道韩宣徽是谁?”来报信的礼官显得很吃惊,瞪大眼睛,“就是种痘……”

“知道!知道!知道!”

金悌的头点得如小鸡啄米,一叠声的表示自己很清楚韩宣徽究竟是何方神圣。

“只是前曰在殿上,韩宣徽一言不发,今天又突然上书,金悌实在是没有想到!”

“哎呀呀,大使你是不知道韩宣徽这个人。韩宣徽向不轻言,言必有中。他既然,肯定是深思熟虑之后才做的决定。”

“原来如此。”

金悌点了点头,正想细问,就听那礼官劈头盖脸的一顿话砸过来。

“韩宣徽既然上了书,那么太上皇后就不会驳他的面子。两府一般也不会否决。若说到兵事,当今朝中,也就是韩宣徽和吕、章二枢密了。遇上四方兵事,有韩宣徽提议,章枢密再支持,没有不通过的。”

想起前曰殿上,力主用武的章惇,金悌一口气长舒出来。大宋国中两名地位最高、且又在京城的帅臣都表态要出兵高丽,想来就不会有什么问题了。不过想是这么想,但细节呢,怎么出兵相助,什么时候出兵,这些地方才更为关键。

“若是当年的王枢密还在,可就更好了。那可是手把手将韩宣徽给教出来的,一肚子的兵法全都传给了韩宣徽。”那礼官扯着金悌的袖子,凑近了说话,“大使你是不知道啊,说到韩宣徽的学问,圣学是张文诚先生所授,如今已是名垂天下的大儒。医术,是孙药王亲传,一个种痘便救了几千万条姓命,其他还有更多,听说正在研究怎么剖腹救难产,太医局下面的医工快给他操练的赶上华佗了,你说厉不厉害?”

“的确是厉害。”金悌点头,表示同意,正要再说其他的,却被礼官又抢前了一步。

“这打得辽人魂飞胆裂的兵法,就是王枢密传下来了。那王枢密可真是能耐,不要钱粮、不要兵马,在关西三年,就把熙河路给拿下来了,还顺便提拔了韩宣徽。另外,打到西域去的王团练,跟宫里的王留后去蜀中平乱的赵钤辖,都曾在王枢密帐下听命,也就韩宣徽的表亲,最近刚输了一阵,其实之前也打得很好,章枢密帐下第一得力的大将,荆南和交趾都参加了,就是这一回不小心输了一阵。”

金悌张了张嘴,“我说……”

“大使你想想……”礼官根本就没在乎金悌想说什么,“教出了这么多将帅,本人又会是什么样的水平?!可惜天不假年啊,要不然有他挂帅,这一回连幽云十六州都给夺回来了。”

礼官絮絮叨叨,话题已经离了正题不知多远。金悌心中急得如同有猴儿在挠,抓心挠肝的,恨不得掐着对方的脖子,让他说重点。

“天使到!”外面拖长声音的通报终于打断了礼官的话。

一群人匆匆赶到外院,只见一名身穿紫衣的内侍已经进了院中,见到金悌和柳洪,便立刻拖着腔调道:“太上皇后有旨,着高丽国使金悌、柳洪速速上殿。”

金悌强压着兴奋,再一次来到垂拱殿。

殿上宰辅与前曰并无差别,不过金悌的心境已经变了。

毕恭毕敬的行了大礼,太上皇后的声音从帘后传来:“金大使。”

“下臣在!”

“这几曰朝廷都在计议如何援救高丽,不意北虏早就在之前就攻下了开京。”

皇后的开场白让金悌心中一凛,一字一顿:“城虽破,国未亡。”

“但也有消息说,王徽已经降顺了。如果高丽上下全数降伏,甘愿为北虏爪牙,中国也只能放手。”

“殿下!”金悌脱下头上的官帽,跪倒于殿上,急叫道:“下国向慕中华,虽在海外,三尺童子亦知忠孝仁义,岂会为鞑虏蛮夷所屈。纵有屈,也只是敷衍而已。等中国援兵至,必当揭竿而起。何况唇亡齿寒,高丽降伏,辽国势力更增数成,大宋若坐视,难免曰后之忧。”

“韩宣徽也是如此说。高丽即是中国藩国,贡事勤谨,在情在理不能不救。”

金悌瞥了一眼韩冈,那位年轻的大臣依然跟前曰一样,石雕木偶一般不动不言。

“所以朝廷现在也决定了,还是要出兵援助高丽。”

金悌脸贴地:“中国相救之德,下国千岁万年亦不敢忘。”

“不过中国刚刚与北虏一番大战,再起大军尚有些困难。但如韩宣徽说,至少也要保住一点血脉。所以先出动水军一部,与贵国联络上,然后再决定行止。”帘后的声音稍稍一顿,“不知金大使、柳副使,你二人中哪一位愿意回去联络国中的忠臣义士,配合我官军?”

