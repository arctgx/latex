\section{第29章 顿尘回首望天阙(八)}

【今天就这一更了,很抱歉。】

王韶要走了。

当韩冈回到驿站,王韶也已经回来了,他的那座小院灯火通明,随扈们正在整理着行装。

“方才已经禀明了官家,明日的早朝,就要上殿陛辞。”王韶说着。

所谓陛辞,就是当朝官离开朝廷出外任官时,上殿辞别皇帝的规矩。不过相对于今晚天子对王韶的临时召见,明日早朝的陛辞只是个走过场的仪式。但规矩就是规矩,朝官离京,正常情况下都要走这一遭。

王韶让人给韩冈端来醒酒汤,一起坐了下来,指着忙忙碌碌的随从们,“明天他们也一样早起,就在宣德门外候着。等我出宫后,就直接离城返回通远军【古渭】。”

“这么急?!”

“已经在京城留得太久了。虽然近期河湟那里的蕃人当不至有异动,但离开通远军过久,也不是件好事。”

说起来王韶已经在京城待了一个多月,要不是天子留人,他早就走了。跟王韶一起来京城的俞龙珂、瞎药——现在已经改名叫包顺、包约——两兄弟,还有张香儿,早在韩冈还没到的时候就回了秦州。

王韶在京城留得越久,古渭寨里的高遵裕就能越加深入的控制起寨中内外事务,而且缘边安抚司中领军的苗授,又是高遵裕的人。当王韶和韩冈都不在的时候,只靠一个王厚,怎么可能跟高遵裕抗衡。

“而且我还担心横山的战事,会影响到河湟这边。夜夜都在想,头都疼了。还是要当面看到才行。”

王韶苦皱着眉,两手用力揉着太阳穴,看起来的确头疼着。

韩冈也知道以眼下的局势,王韶肯定是要头痛的。

河湟、横山都是关西主要的战略方向,两边自然有着千丝万缕的联系。从王韶本心来讲,他是肯定不愿看到韩绛、种谔春风得意的模样。

拓边河湟是什么,是‘断西贼右臂’!从侧面来牵制西夏军力。而横山,则是党项人的腹心。夺取罗兀,控制横山,就是一剑穿心。一旦韩绛功成,西夏国就要亡了,王韶在秦州以西的任务再没有继续下去的必要,砍死人膀子有意义吗?在河湟再多的大捷,也抵不过占据罗兀城的意义。

但‘善祝善颂’的话,王韶也不想说。他心中也许恨不得韩绛骑着一匹歪脖子的劣马,一头栽进无定河里淹死,但他也不希望看到损兵折将的惨败出现——那时候,西贼势力大盛,河湟那边的压力也会大起来。

王韶其实是左右为难,对于韩冈即将上任的工作,也没什么心情去想。

“今天入宫面圣,官家提到玉昆你好几次,话里话外都想见你一面。”王韶回忆起今晚见到赵顼时的情形,年轻的天子对韩冈重视,着实让他惊讶。王韶为韩冈无缘上殿而感到遗憾:“若不是冯当世在中间拦了一道,玉昆你今次得以入宫廷对,说不定就能特旨转官了。”

“此乃时也命也,也只能等下次了。”

韩冈叹了口气,看似豁达的笑了笑。不过他的心中不无怨声,‘现在说这些还有什么用?’

从选人转为京官,脱离选海,是每一个底层文官都梦寐以求的美事,韩冈何能例外?只要见到天子时奏对出色,总是少不了一份恩赏。韩冈的功劳已经积累得离转官只差一步,天子恩泽一下,转官当是定数。

只是自真宗起,大宋的历任天子都顾忌着后世名声,不想跟宰执打擂台。而几十年的宽和政治延续下来,文官们也少了顾忌。为了表现自己的刚直,一众宰辅能为一件芝麻大的小事,闹得天子下不了台。

若是为了新法倒也罢了,但为了韩冈一人,而让冯京闹将起来,赵顼当然不愿意。若是使得执政赌起气,闹到辞官要挟的地步,不论谁是谁非,都是皇帝输了。

‘这些文官都被惯坏了!’冯京一句话阻了他进步的道路,韩冈可是将其恨到骨头里去了,‘世事轮回,报应不爽,这件事总有回报的一天。’

不过韩冈若是以一介臣僚的角度来看今次的事,冯京做得其实也不算错,维护朝野定规,让天子不能恣意妄为,也是大臣的本份。

实际上,韩冈一直认为皇帝弱势一点没有坏处。若是能把天子变成后世的英、日等国那种装饰用的壁画,或是就像此时的东瀛倭皇,自己当上宰相的时候,也会痛快许多。

韩冈作为臣子,当然希望天子越老实越好。可是如果换作他韩冈是皇帝,莫说挤兑的权臣,他不可能容得下,就是普通结党的大臣,他都会拉一派打一派,让他们两边老老实实的听话受教。这就叫做屁股坐的地方不同,观点也自不同。

“对了,玉昆。”王韶并不知道韩冈现在满肚子都是反逆的念头,见韩冈突然沉默了下去,以为触及到了他现在的心情,安慰似的岔开话题,笑着:“你的风流之名已经传到了宫中,让花魁为你守节,不是等闲人能做得到。而周南出淤泥而不染,也算是能难能可贵了。”

在跟王韶的内侄女定亲的时候,与名妓勾勾搭搭,从情理上的确有些说不过去。但现在也看不出王韶他有什么怒意。而韩冈也是惊讶于他和周南的事,竟然已经传到了天子的耳中。

“是皇城司?!”

论起天子的耳目消息,在京城的百司之中,也只有皇城司负责也一方面的任务。

“不要想太多,皇城司也不是事事都能打探的到。”王韶误以为韩冈是那等憎恶天子侦缉臣民言论的士人,帮天子赵顼解释道:“只是玉昆你和周南的事传得广了点,所以传到了官家的耳朵里。今夜也是当笑话说了出来,并没有责怪你的意思。”

‘责怪?这怪谁得了?’

韩冈与周南未及与乱,并未违反法令律条。而引得名妓倾心,甚至为其守节,也只不过是一桩小小的风流韵事,天子也不至于大煞风景。

若是在面圣,说一声想拿功劳换人,请天子将周南放了。以当今天子赵顼的性格,当是人也送,财也送,功劳照样给,也好成就一段佳话,就像为宫人结今生缘的唐明皇一样。哪像现在,不上不下的。

‘可惜了。’韩冈想着。

第二天,出城十里送走了王韶,韩冈和李信重新回到驿馆。

王韶走了,韩冈感觉就安静了许多。他现在要做的,除了周南之事外,就是等待调任的令文递到手中。

“应该还有几天的时间。”韩冈轻声的自言自语。

如果是在缘边安抚司,办理一个调职的手续,除非从头到尾都有人盯着、催着,要不然都要花个两天走流程。而以中书门下的事务繁剧,平常调任走个十天半个月也是等闲。即便延州那里急着要人,而且只是兼职暂任,不需经过流内铨,韩冈估计也要三五日的时间。

而过两天,李信要试射殿廷,但他在三班院中,自几个来自京营、同样参加试射殿廷的军官那里,受了点气,心情有些不好。原本就不多的话更少了许多。也不出去联络三班院中能使得上力的官员,而是呆在驿馆中,习练武艺。

韩冈不禁为李信叹息,除非自己一直看顾他,否则他这样的性子一辈子也难升官,可惜了李信一身的好武艺、好兵法。

李信自己处理不好人际关系,韩冈也只能自己出头来帮他这个表兄的忙。先带着李信出去转一转,顺便去找周南,看看能不能打听到什么消息——官妓们的情报网,有时比皇城司还要厉害。

对于给周南赎身脱籍一事,韩冈不会把希望全数寄托在他人身上。光靠蔡确可不保险,章惇打得包票,也要少算个几成。以韩冈的行事习惯,必要的工作,自然是要双管齐下。李信昨日没有被宴请,韩冈也不便带上他,不过今日,见一见自家人也无妨。

不过韩冈正要有所动作,客人就来了。

韩冈身份不同了,不可能随随便便见人。驿卒给韩冈带了客人的口信,在驿站外面说是教坊司派来的。

应当不是周南的从人,否则一封信就足够。

韩冈对此也不意外。周南跟自己的关系,天子都听说了,东京城中早就传遍了的样子。这种情况下,教坊司若是还没有反应那就是白痴了。论情形也该到了。

韩冈没有叫人过来,而是直接到了外面的大厅中,一名干瘦干瘦的中年汉子正在那里等着。

先找了张空着的座位坐下,周围的几个官员便讨好的过来打招呼,韩冈很谦虚的一一还礼,丝毫不见傲气。坐定后,他让李小六把那干瘦汉子招过来,问道:“你因何事而来?”

“小人甘穆,今次来是为了周小娘子。”中年汉子在韩冈面前弓腰行礼,但口吻一点也不安生,“小人今次是奉了上命,还请官人莫要再来找周小娘子!”

