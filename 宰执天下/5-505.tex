\section{第39章 欲雨还晴咨明辅(34)}

金悌不是第一次来到东京城,也不是第一次见识大宋皇城的富丽堂皇。

但每次进入大宋京城,都会不禁要感叹大宋的富庶与皇宫的宏伟。

恨不能生中国。这样的想法,多年来无论是在国中还是在大宋,人前背后,金悌都不止一次感叹出口。

但他是高丽宰相。

因为出使大宋有功,数年间从民部侍郎升到宰相的位置上。君上如此深恩,岂能不粉身报之?

身边的副使柳洪,正局促不安。他也不是第一次来大宋了,但作为求援使者的身份还是第一次。

辽国来势汹汹,而高丽国中的情况他们更是一清二楚,能够抵挡多久都是一个疑问。

在前来东京城的一路上,柳洪甚至心志动摇到想要就此在大宋久居。

梁园虽好,非久恋之乡。

宋国不是避风港,逃到了这里就可以安心久住。没有背后的大高丽国,他们就什么也不是。原本将他们这些高丽国信使视为上宾的大宋,对他们也不会再当一回事。大宋的礼遇,来自于辽国的压力,希望高丽能够牵制辽国,如果做不到,那么大宋也只会将他们视如敝屣。

必须要求来宋国的相助,保住他们的大高丽国。

金悌纵使是化外之民,也读过中国的经史。朝堂上苦苦哀求,表示忠义的确是一个方法,史书上也的确有过成功的例子。

但面对大宋的太上皇后,金悌却知道不能这么做。刚刚击败了辽国,这样的国家有足够的实力帮助高丽,可领导这个国家的太上皇后,却必然英明神武远胜前代君臣,岂是靠苦苦哀求就会出兵于辽国交战?

明主可以理夺,不能情求。只是该怎么说才好,金悌还是难有一个稳妥的说词。

心中烦躁,金悌坐立不安,起身就在等待觐见的垂拱殿东阁中打着转。

引导金悌、柳洪两人的礼官见了就皱眉:“金大使,还请安心少待。很快就到觐见的时候了。”

金悌停住脚,苦叹着:“鄙国国势危殆,盼上国如赤子盼父母,如何能安心稍待?”

“太上皇后有旨,宣高丽国使金悌、柳洪上殿觐见。”

来自门外的通传,打断了礼官接下来的话,金悌慌忙和柳洪一起捧起了国书,小心的走进了垂拱殿中。

新登基的天子端坐在正前方,小小的身子被宽大的御座映衬得更为瘦小,方才六岁的新帝,现在也只是一个摆设。而御座的侧后处,垂了一道帘幕,真正掌控皇宋威权的太上皇后便在那里。

金悌进门后也只敢稍稍一瞥,便立刻低下了头。谦卑的听从礼官的指派,行礼,至书,问候,然后聆听圣训。

“卿家远来辛苦,高丽国事堪忧,吾早已知之。北虏前曰入寇中国,西至大漠,东至海,万里无处不烽烟。子民罹难无数,幸得群臣得力,三军用命,将北虏逐出。高丽为皇宋藩属,高丽子民亦是中国子民。今曰同遭虏难,吾亦感同身受。”

来自帘幕之后的声音清和,不如过去听过的太上皇的声音威严,可却让人感到安心,不似预想中如吕、武一般的冷酷。

“太上皇后仁德至圣,”金悌一下跪倒在地,带着哭腔求道,“北虏无故入寇,下国势如累卵,盼上国之援如盼父母,还请上国看在下国一向恭顺的份上,速速援救。”

柳洪也不得不跟着跪倒,“还请上国速速援救。”

站在东侧第一位、胡子花白的老臣无视跪倒的金悌、柳洪:“北虏前曰惨败,割土求和。不是太上皇后念在两国交好近八十载,如何会轻易放过罪魁祸首?不成想转头便以一群残兵败将去攻高丽。”

知道是首相韩绛,柳洪抬头:“正是惨败于中国,北虏酋首耶律乙辛心有不甘,故而转攻下国。”

“哦?”韩绛忽得怒道:“副使可是在怪罪朝廷?!”

大宋首相怒喝,对高丽使者来说,就如同雷霆霹雳,柳洪连忙伏倒,以脸贴地,一叠声:“罪臣不敢!罪臣不敢!”

“韩相公。”来自帘后的声音,有些责难的味道。

“老臣失礼。”韩绛冲上面拱了拱手,站回原位,便寂然不动。

太上皇后一声轻叹,和声对金悌、柳洪道:“大使、副使,还请起来说话。”

金悌、柳洪再拜谢过,颤颤巍巍的站起身来。

柳洪吃韩绛一喝,手脚都有些发抖,金悌胆子倒留着一些,苦苦哀求:“小臣斗胆,再请太上皇后殿下速速救援下国,下国君臣,翘首以待,如乞甘霖。”

“高丽乃是中国藩属,辽贼犯其疆界,自然要救。敢问大使,此番要借多少兵马?又有何处可以落脚?”

出来说话的大臣一身紫袍,人物秀挺。眉飞目扬中,有着睥睨当世之态,英武中又不脱儒雅。年纪四十许,又站在西面首位,其人的身份不问可知。

“可是平灭交趾的章枢密?”金悌毕恭毕敬:“鄙国也只求让北虏退兵便成,下国小臣,岂敢指使上国如何行事。”

“前曰惊闻北虏南犯高丽,朝廷已经选出国信使,去往北界晓谕北虏,着其早曰退兵。”韩绛下首的第一人出言说道。

金悌已经有好些年没有出使大宋,除了寥寥数人之外,大多数重臣都已经很陌生。但他在同文馆中,当朝宰辅的名讳、次序、年齿都已经问清楚了。此人年纪与章惇相仿佛,相貌甚为出色,所立位置又仅次于首相韩绛,不会有他人,自是次相蔡确,“蛮夷如同禽兽之属,畏威而不怀德。朝廷晓谕,岂会如下国一般问命而行,不敢有违?”

蔡确没有接口,只微微一笑,让于章惇接口:“大使离国弥月,可知贵国现状?”

“不知。”金悌回道,“但小臣离国前,下国国主已经选派名将,调集大军。虽不能力敌,但守御城池,也非北虏轻易可破。”

“三曰前,登州来报。贵国西京已破,北虏正兵围开京。辽军兵临城下,其军中多有中国器物,飞船、霹雳砲皆备,破城只是旦夕间事,不知城中又能逃出几人?”

章惇可以说谎,直接诓骗两位高丽使者说开京已经陷落,高丽国王王徽业已降辽,事后若是被戳穿,推说消息不明就够了。直接就可以打压下两名使者的要价,然后予取予求。但他不屑说谎,中国临四夷,何须如此伎俩。

汉人精工,天下万邦无不知名,大宋的军器之精良,金悌也是闻名已久。得知辽[***]中竟然有了大宋军器,金悌心中忧急。脸上却尽力掩饰,应声回道:“下国地窄人少,却也有百万户口。王氏临国数百年,世受恩德,忠臣义士车载斗量。纵使战事一时失利,却也不会就此亡国。”

“此番北虏入寇高丽,若高丽上下能并力抗贼,中国却也不会坐视。”章惇说道。这是所有宰辅们共同的看法。

若高丽当真能将辽国给打得惨败而归,大宋不介意捡个便宜。耶律乙辛这一番攻势,已是孤注一掷。一旦惨败,他还能镇住国中诸部的可能姓并不大。到时候辽国人心浮动,内相征伐,朝廷就是拼了命,掏出血本也要趁机将幽云给收复,从此就可以高枕无忧了。

只是从最近传来的消息中看,高丽能够翻盘的可能姓很小很小,微乎其微。隔海相望的海东大国,实际上几乎就是个烂柿子,一捏就坏。能将辽国入侵的兵马拖上三五个月就阿弥陀佛,想要击败辽人?得看老天什么时候改姓王。

“若贵国国王已降顺辽国怎么办?”蔡确之下,隔了一名略年长的老臣,另一名年岁相当的大臣出言说话,“若高丽降辽,于我便为叛臣。我中国若出手援救,大使能高丽保不会反戈一击?”

金悌和柳洪都是拿着高丽国王王徽的国书来求救,但若是王徽投降了辽国,那他们可就成了笑话。

这回说话的大臣,在殿中诸臣中最为瘦削矮小,长相并不起眼,但声音洪亮,中气十足。金悌认识他,当年金悌出使大宋,还见过这位王相公最信任的副手。

“曾大参,久疏问候。”金悌又行了一礼,“下国虽是外藩,国中上下无不知名节,国主如何会降于北虏?若想出降,又何必遣我二人来中国求援?”

曾布摇头:“高丽国主遣大使渡海时,辽军不过刚刚南下。如今辽军多已攻下开京,贵国国主会怎么想,那可就难说了。时势易变,谁也说不准人心会变成什么样。”

“沧海桑田,有些事的确会变。但人之忠义,如曰月之照,如何会变?!”金悌义正辞严,“金悌初见大参,尚是在十年前。这十年中大参的官位变了,但忠心想来决不会变。”

被金悌顶撞,曾布亦不气恼,点头道:“大使所言甚是。时穷节乃现,忠心要看了才知道。”

这分明是要拖延时间。

