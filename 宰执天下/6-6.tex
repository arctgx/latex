\section{第二章 天危欲倾何敬恭(下)}

“适者生存?”

用来描述自然的道理,却被用在了国家制度上,章敦眉头几乎打了结,“还是用顺天应人。顺应时势比较中听。”

“意思是一样的。”

“但听起来不一样。”章敦说了一句,又摇摇头,根究这等穷枝末节没有什么意义,咬文嚼字的,又不是汉代酸儒,“要顺应时势,所以要变祖宗之法。适者生存一说,用在国家法度上,也能够说得通。但是玉昆……你觉得现在是能变的时候吗?不是变法了,是变天啊。”

“天子还姓赵,天也就还姓赵,哪里变了?”韩冈笑了一笑,又道:“日本自言其国主万世一系,百代不易。子厚兄,他们变不变?”

太宗年间,日本国僧侣奝然渡海至中国,面见天子时,曾献《职员今》和《王年代记》两卷书册。其中《王年代记》里,记录了日本国主的谱系,自云二十三代王尊,六十四世天皇,一脉相承,传承数千载未曾断绝。

太宗皇帝听说之后,便对身边的宰相叹息道:‘此岛夷耳,乃世祚遐久,其臣亦继袭不绝,此盖古之道也。’

当年旧事本来没什么人记得,昔日的记录也都丢到了史馆的故纸堆中,埋得不知有多深了。可前日辽军渡海入寇日本,朝廷上下立刻翻箱倒柜,从各个不同的衙门中,将有关日本的记录都给翻找了出来。韩冈和章敦贵为宰辅,参议军事,这些记录都是必须通览的。

“四夷哪得与中国同?”

“不同?大者天下,中者国家,小者社会。就比如平日里都能看到弓箭社、忠义社,其中社首,都是公推耆老或是有名望的士绅来主持,并非官府任命。而蹴鞠、赛马两大总社,每年年初都要选举会首,票多者为胜。这一选举之法,却又与泰西古国类同。”

“泰西古国?大秦?条支?”

从出玉门关开始,一直到大食,都算是西域。西域之南,越过吐蕃,是西天诸国,也就是天竺。而西域之西,便名为泰西。章敦对那么远的地方没有多少印象,只记得后汉书里有大秦、条支。

“希腊。”韩冈说了一个章敦很陌生的国家,“小弟近年来搜集外国书籍,又使人翻译成汉文,增长了不少见识。泰西有古国名希腊,大约与周同时。其国文教昌盛,贤人辈出,在大食国的书籍中,至今无不称慕。希腊国中以成年有财产者为士,其国主由士人推举而出,任职定有时限,或三年、或五年,时限一至便须卸任,重归为士,不复为君。这个与现如今蹴鞠、赛马两大总社选举会首有多少区别?”

章敦猜疑道:“别的我不知道。蹴鞠、赛马两大总社的会首选举,莫不会是玉昆你借鉴来的吧?”

“借鉴?好吧,小弟就举个不是借鉴的例子。”韩冈又道,“现如今,希腊早亡。有景教本宗一统泰西。泰西诸国国王若即位,必须先上禀景教教主也就是泰西帝,由泰西帝遣使为其授冠冕。这与周时很像吧?天下分封建制,诸国皆奉周王为共主。”

韩冈说得简明扼要,省去了很多细节。章敦他家是福建大族,与海外交流很多。韩冈说的有关泰西,他隐隐约约有些印象,可是听起来很不一样,仿佛哪里扭曲了,只是用周时的分封建制套上去,却也的确能合得上。

“不同国家,有不同风俗。不同时期,文法、纲常也不尽相同。可有的时候,相距数万里却又有相同地方。这其实也映证了一点,无论哪国的文法,都并非天生,而是不断演变,因时而变,因势而变,符合实际。子厚兄,说句冒犯的话。之所以惶惶不可终日,是以为不会变,十年之后,又会恢复如旧。但实际上呢?只要习惯了,变了也就变了!”

韩冈说得如同绕口令,章敦却听得很明白。

“可能吗?”他立刻冷笑着问道。

“只要有那份心。”韩冈说道。

“玉昆。”章敦压低了声音,却是声色俱厉,“如果只是废立天子,那是一劳永逸,能富贵终老。而你的办法,却是要我等冒着杀头的风险,为后人开道鸣锣。愚兄再问你一句,可能吗?!”

的确不可能,因为以现如今的两府宰执,他们根本就不能指望。

后世的西方几大国,其推翻国王的革命,无不是形势所迫。在这一过程中,又不知付出了多少条性命?参与到其中的人,除了少部分野心家之外,基本上都是迫不得已,才会选择这样的一条道路。

韩冈想要做的事,并不是靠他一个人就能做到。便是集合众人之力,也同样可能性不大,必须要冒很大的风险。

而放在眼下,宰辅们哪个愿意选择冒着巨大的风险却好处不多的路。将犯了弑父之罪的六岁小皇帝换掉,这么简单的一件事,带来的好处却难以计数。前者风险和收益完全不成比例,而后者,收益率实在是太高了。

只是事到临头,可由不得他们了,他们必须做出改变。

难道现在他们还能有办法废去皇帝不成?就算他们能说服向太后动了念头,同意将赵煦给废为庶人,也必须找到一个合适的时机,以及合适的人选。这同样不是简单的事情。对赵煦的位置虎视眈眈的宗室决不会少,可有几个能让向太后看着顺眼的?

“的确是难。但皇帝凭喜怒决人生死,这是对,还是错?如今有机会将之改正,能放过这个机会吗?”

“改过去也会被改回来。”章敦十分清楚,对天子的畏惧在臣子们的心中根深蒂固。

不论大臣们如何控制朝堂,只要皇帝不昏庸,很快就会将权力抢回来。换作是五代时还好些,没人会畏惧年幼的皇帝。不过若是五代,皇帝只留下妇人孺子,早就是哪家的都点检来个黄袍加身了。

但瞅着韩冈充满自信的神色,章敦疑惑起来,小心的问着:“玉昆,你是不是知道什么?”

“知道什么?”韩冈反问。

“天子体弱多病,有不足之症……”

韩冈正色说道:“子厚兄。小弟的确能断人生死,但那时断案的时候。人的寿数,却不是韩冈能知道的。”

赵煦的寿数能有多长?历史上他驾崩的年纪似乎并不大。韩冈记得那本提及道君皇帝的名著中,还是端王的赵佶继位时,还是喜欢踢球取乐的纨绔,年纪不会大。从他身上推过来,赵煦驾崩时岁数也不可能有多大。

不过现实是赵顼没生出道君皇帝就死了,证明历史已经偏得让人完全认不出来了。既然宋神宗能变成宋熙宗,又比历史上早死。那么谁能保证赵煦不能多活上几十年?活到花甲,活到古稀,甚至能与梁武帝相媲美,活到八十以上?

而向太后的寿数同样难说,可能很长寿,也可能有意外。一旦意外发生,就算赵煦还没到亲政的年纪,还有高太皇太后,或是朱太妃,无论哪个,都能让宰辅们心寒到底。

章敦还想说些什么,但看着韩冈,心中想说的话,最后都化作长长一叹。韩冈的想法太过可怕,又难以实现,章敦不敢将自己的身家性命都放在他的身上。

“子厚兄,暂且不用担心。”韩冈安慰着章敦,“我等还是有足够的时间来想办法。”

“足够的时间?”章敦摇了摇头。既然寿数难以确定,哪里还有足够的时间。看着还有十年出头,可实际上,却很难说不会有什么意外。

婉拒了在韩冈家里留用酒饭,章敦心头压着沉甸甸的巨石离开了。

送了章敦出门,韩冈回到书房。

今天晚上与章敦说的话,充满了太多大逆之言,想必章敦他不会疏口泄露出去。不过连章敦都没有说服,更别说其他宰辅了。

可韩冈不是那么的担心。还有的是时间。向太后发生意外的可能性其实没那么大,而赵煦本人的问题,比起向太后也要大得多。

赵顼已死,赵煦也孝心可悯,但他从小身上就要背着沉重的罪孽,旁边人会怎么看待他?就算是贵为天子,也不能当做整件事并不存在。身处那个独一无二的位置上,世人的注意力都会集中到他身上,不论他做得有多好,都逃不过一句‘弑父’——他身上的标签已经定下来了。

宰辅们太过小瞧赵顼已经被曝光的弑父之罪,也太过看轻他们自己的力量。

他们的心性还撑不起局面,就是胆大包天的章敦也一样,对皇权的畏惧依然根深蒂固,远远无法与一肚子谋反之事的韩冈相比,否则根本就不用担心什么。但掌握在他们手中的力量,却是真实无虚。韩冈相信,随着时间的推移,他们慢慢会明白到底能做到些什么。

而现在的恐惧之心,也能让蔡确、章敦,甚至包括韩绛在内,都主动去做一些他们本不肯或不敢去做的事,只要他们觉得这么做对未来有利。

并不需要别人去多费心引导,在时势面前,他们会自己做出选择。

