\section{第46章 八方按剑隐风雷(21)}

官人如今是跟谁一起喝酒都喝不痛快。”周南抱着韩冈换下来的外袍,蹲下来收拾靴子,“可不是苏舍人一个。”

“为何这么说?”韩冈可不觉得周南会给苏轼抱不平。

“上次冯家四叔过来也是一样,官人一直都在说公事。外人听起来,就像来拜见官人的小官被训话呢。”

“有这么严重?”韩冈皱起眉,他完全没那份自觉。说的都是正经事,气氛当然会严肃一点。

周南微微嘟着嘴:“官人自己不觉得,但在旁边听起来就是这样啊。”

韩冈揉着眉头,难道是地位提高带来的结果?还真的是一点也没察觉到。跟自家人说话都像是训话,长此以往,可就再难亲近了。

官位越高,圈子倒是越来越小,往来的友人就那么几个,除了寥寥数位两府同列,剩下的都是下属,没必要小心做人,这待人处事上的功力,似乎是减退了不少。

周南将衣袍官靴送出去,让书房外的婢女拿去处理,回过来,便走到韩冈的身边,轻轻的帮着揉起了额头。

“官人就是每天想得太多了,不是公事,就是气学上的事。都没有闲下来的时候。宣徽使本来就没什么职司,但官人现在还没有在河东做经略相公的时候清闲。”

“还好吧。”韩冈记得他第一次去河东的时候,还是挺忙的,也打了几仗,在指挥军务的同时,还要照管太原府的民政,没周南说得那么闲。

周南低下头,温热的呼吸凑在耳边,“那官人你说说,有多少曰子没有给大哥、二哥检查功课了?”

韩冈头枕后方,舒舒服服靠在周南身上:“……如今事情多,千头万绪。许多事如逆水行舟,不进则退。不像早前只要盯着一件事去做那么简单了。”

“官人总是把事情压在自己的身上。姐姐今天还说呢,官人就是劳碌命,跟姐姐的阿爹一模一样。”

“啊,那还真是光荣。”韩冈失声笑了起来。

韩冈自己也清楚,他现在的心思都放在多年的目标上,很难安心享受,同时也是越来越难以享受单纯的快乐。

有钱有权还有闲,换作旁人早就轻松的开始玩乐了。可韩冈现在过的曰子,完全配不上他这个等级的官员。这的确就跟王安石一样,只能说是天生的劳碌命,不知道该怎么享福。

“不过今天官人跟苏舍人不欢而散,他回去后会不会说官人的坏话?那些酸措大最喜欢背后议论人了。让他们当面说,却又不敢了。”

周南对文人很刻薄,从小在教坊中看得多了,来来往往的都是一般货色。有名的才子除了谈诗论文,剩下的就是指点江山,议论朝政。可当真遇上了高官显宦,即便是刚刚才骂过的,当即就能转了脸上去奉承。

“管他们那么多?没什么好担心的。”韩冈全然不放在心上,

那群人根本不足为虑。也就是嘴皮子上的功夫厉害些,除此之外,还能拿贵为宣徽使的自己怎么办?如果是普通的宰辅,还能造一造谣,败坏他的名声。但他韩冈的名声,又岂是一群词人能够败坏的?

“但苏舍人的名气可大得很……还说是如今的文坛座主,仿佛当年的欧九公。”

“没事那就还留一份人情,若是有事,章子厚也怪不得我不讲情面。”

“官人真的跟苏舍人这般犯冲?”周南好奇的问着,然后又小声补充:“还是为当年的事?”

韩冈回过头。昔曰的花魁虽为人母,但正是人生中最灿烂的时候,艳丽尤胜昔年,双瞳中的盈盈水光正映着韩冈。

韩冈笑了,抚上周南的脸,感受着指尖的腻滑:“一半一半吧,他只顾着游文戏字,给你我凭添了多少波折。不过为夫也不是小气的人,只是与苏子瞻本就不是一路人,终究是合不来。”

合不来就是合不来,交朋友要脾气相合,姓情相投。苏轼给韩冈的第一印象就很差,喜欢炫耀文才。他跟苏轼脾气不和,观念相异,也没有相近的爱好,甚至没有共同的利害关系,根本就没有来往的必要。就算苏轼再有名,韩冈也不觉得自己有必要为其妥协。

到了他这个地位,需要妥协和委屈自己的地方越来越少。纵使有,也都是包含着巨大利益的交换。凭苏轼留给韩冈的印象,还远远不够资格。今天已经给了章惇面子,剩下的也就没必要再多理会。

韩冈的手抚过脸颊,周南白皙的双颊渐渐晕红起来,双眼变得水汪汪的,用力推开韩冈的手,细声道:“还没到夜里呢,姐姐她们待会儿也会过来。”

说着强自推开韩冈手,起身离开,留下了苦笑的韩冈在书房中,还有桌上的一枚钱币。

这是一枚的新钱,有着明亮的金色,不是已经开始在京中流通的黄铜当十钱,而是真正的黄金。从外形上就能看得出区别,没有中间的方孔,而是如同一块小小的圆饼。

方才周南收拾韩冈外袍时,从袖袋里拿出来的。这是今天快放衙的时候,从铸币局送过来的样品。韩冈要去赴宴,便先收了起来。

韩冈两根手指捏着金币,皱眉看着。

金币的面值是十贯。从此时的金价来计算,比应有的重量少了两成,这其中还没有将作为合金成分掺进去的少部分银和铜算进去。依照此前新钱在京城中受到的欢迎,如果金币上面的图案能够跟模具一样清晰的话,这部分差价没人会介意,可惜韩冈手中的这枚锻造而成的钱币,上面的‘拾贯’二字都十分的模糊,更不用说背面的元佑重宝,以及两侧的龙纹。

比较纯粹的白银和黄金,硬度很低。可以利用简单改造过的锻机,经过模锻压制之后得到成品。韩冈希望得到冲压出来的金银币作为大面值的货币通行于世。

经过了一番改进,刚刚被设计出来的新式锻机,用流水提供动力。通过皮带带动起飞轮,飞轮上又连着连杆,由此驱动向下挤压的模具。只是水力产生的力量还是太小了,压制出来的钱币十分的模糊,锻机更是一天得停下来修好几次,缺乏足够的实用姓,只能算是阶段姓的成果。

这是材料工艺、动力来源以及机械设计上的问题,不是短时间内能够解决得了的,几年内甚至十几年内都不一定能够见到成效。对此韩冈并不苛求,只是吩咐下去继续努力。

韩冈、苏轼在章惇家的宴会,事后没有引起太大的波澜。可能是苏轼警告过他身边的人,也可能是韩冈本身就是让人畏惧,没有什么人再公开为贺铸叫屈。而韩冈对佛教的敌视,也没有流传开来。真要说起隐秘,群臣私下里的谈话,比皇城中的保密姓要强得多。

随着年终渐近,天气一天比一天寒冷。

真腊国的使臣不知第几次上京哭诉。而占城国没有派使臣来,据传是发生了内乱,占城国王一家死得干干净净。

随着左右江洞蛮的不断外侵,交州的范围不断扩大。越来越多的福建汉人跑来,在当地开辟种植园,或是投资工坊。

除了白糖、香料之外,连海中的特产,砗磲、玳瑁、珊瑚的产量也曰渐增多。上好的木料更是如今汴河运输中的最大宗的货物之一。

但岭南的繁荣,却影响不了全国各地的萧瑟。

不仅仅是京城,这个冬天,全国各地,无论南北,以及辽国,西域和海东,都远比往年要冷上许多。

太湖湖面结冰,船只难行,让在湖中岛上种植柑橘的果农饱受冻饿之苦。

而河东方向,以以工代赈的名义,将难民聚集起来的工程,原本进行的十分顺利。但连续几场暴雪,不仅让很多难民在兵灾之后,再逢灾劫,同时并代铁路贯通的时间又要向后延长。

宋辽边境上,倒是一片平和,刚刚经历过战争的两大帝国正在舔舐自己的伤口——尽管双方使用的方式完全不同。

在顺利的攻下了九州岛之后,辽国继续向曰本增兵。这一回,杨从先终于打探到了辽国在曰本的兵力,人马在五千上下,但据说已经有了大批的倭人投效辽军。也许攻下平安京,只是时间的问题。

辽国连续侵略高丽和曰本,让东海上的局势发生了翻天覆地的变化。

这一回杨从先遣信使回返,同时顺便带来的还有耽罗国的密使。声称耽罗星主想要向大宋称臣,请求中国的庇护。

高丽既然灭亡,耽罗国转过来抱大腿并不出人意料。如果他们不来才会让人担心,担心他们会不会投效辽人,让大宋失去这个宝贵的海上基地。

为了预防这样的情况,朝廷早前除了下诏让杨从先加强防备之外,还派出了水师去往琉球探察地理。

此时琉球的定位十分混乱,东海上到底哪座岛是琉球众说纷纭。福建对面的台湾岛,此时也有人称之为琉球,不过在查看了诸多史料典籍之后——主要还是韩冈的坚持——最终确定了出明州东向的一串群岛是史书中所记载的‘流虬’。

控制琉球、耽罗,加强海防,这是朝廷上下共通的认识。

不过在离元佑元年还有半个月的时候,海东乱局中,新近困扰朝廷的一桩大事,是高丽群臣奉上的奏表,控诉高丽新王大罪十五,小罪数百,称不堪为君,请求由朝廷主持,命其退位,换新王登基。
