\section{第36章 沧浪歌罢濯尘缨(五)}

“枢密,白玉那边又有军报来了。”

韩冈正皱着眉头,黄裳拿着一份公函进了帐来。

“朔州的消息倒还真是多。”

“不是朔州,是白玉。”黄裳强调着手中公函的出处。

折克行本官、差遣皆在白玉之上,在白玉领命离开时,韩冈也是让他听从折克行的指挥。

白玉不敢违反韩冈将令,只是他将折克行的安排一五一十都写了报告回来,军营中的大小事也事无巨细都向韩冈禀报。这让在韩冈身边掌书记的黄裳觉得很人烦。

“白玉忠勤太过,并非正人。”

“成见不要那么重啊。”韩冈笑着接过黄裳手中的文函,扫了一眼后,见的确没有什么重要的事,便又放了下来,对黄裳又道:“他能将事情办好就行了。”

黄裳对白玉的确是有些成见,就是之前白玉来拜见韩冈时做得太难看。

韩冈将西军精锐一留多月,不给他们立功的机会,白昭信抱怨几句是正常的。韩冈也不觉得被冒犯了,鸡毛蒜皮的小事而已——一般来说,地位越高,对口舌是非就越不看重,而是会更加注重行动和实际。

而白玉为此给说气话的儿子一巴掌,也可算是无暇思索下的自保手段,韩冈也同样能够理解。但之后又踢一脚算什么,把他韩冈当成什么人了?

韩冈由此对白玉的感官降了一个等级,而当时在旁作陪的黄裳,对白家父子两人更是一下便没了好感。

白玉能力是有,在关西诸多将领中,虽不能说是第一流的,至少算得上中规中矩。可惜做人做事差了一筹,让人心里不舒服。

难怪以他在关西军中的资历,最后还是没人愿意用他。

郭逵救了他一命,恩同再造,却没设法将他调去河北做亲将。其中有自全的因素,但也肯定是觉得白玉不堪驱用。

若白玉不是多了那一脚,韩冈不介意在幕府中给他留个位置,但现在是不可能了。只会是有多少功劳,给多少奖赏,不会破格提拔,也不会将之视为亲信。

没有后台的武将,想要晋升,除了熬时间,别无他法。韩冈本想给他机会,可惜白玉没能把握得住。

不过话说回来,这一回若能成功,功绩也够让白家三代安康。一名年过五旬的武将,对此还有什么好奢求的?

白玉的问题想过就算了,现在问题是皇帝。

现如今的代州本郡,除了几座关隘外,基本上都已经收复,而且那几座关隘出口,皆有重兵设营把守,不虞辽军袭击。

而朔州州城处,折克行和白玉正在准备与辽军主力决战。整训兵马,修整城防寨防,试图在辽军来袭前,做好万全的准备。并不准备亲临朔州战阵的韩冈,只需要在代州城等待消息便可。

他的心神,多数都放在了从京城传来的急报那里。

复幽云者封王。

这可是生封王爵,而且必然是异姓王。

汉代是非刘不王,除了国初和末年,没有异姓封王的例子。唐时的情况又不一样,但异姓封王的依然稀少,安史之乱后才稍稍多起来。

以本朝开国后逐渐形成的规则,外藩不论,朝中的异姓臣子封王只能是在死后,生不封王。

之前得封异姓王的赵普、慕容延钊、高怀德、曹彬、潘美,要么是皇后的父祖辈,要么就是立有大功,无从酬奖。而这些人都有一个共同点,那就是封爵时已经不在人世,乃是追封。

韩琦相三帝、立二主,在他故世后,一直都有说法要将其封王,不过至今没有动静。

要想生封王爵,在如今,至少得有郭子仪那般的不世之功。

光复幽云者,勉强够资格。不过韩冈不论怎么想,都觉得皇帝的这句话,肯定不会是局限于本意。应该是针对眼下的局面,经过一番计算的结果。

或许是针对领军的几位帅臣,或许是针对现在的和议。反正皇帝必然暗藏其他心思。

但这分明是乱命!

皇帝哪里能随意的将一个王爵授予他人?朝廷自有制度在,容不得皇帝恣意妄为。

说句难听话,万一皇帝当真连下乱命,让韩冈自尽。那时候,他是自裁还是不自裁?

照常理,韩冈当然可以不理会,还没有成为正式的诏令,仅仅是口谕而已。

没了大政之权的天子乱命,虽不便呸上一口,也完全可以置之不理。

韩冈记得有传言说仁宗皇帝晚年多病,头脑时常不清醒,一次犯病时还当着辽国使者的面,高呼着皇后和宰相要谋害他,可也没见曹太皇和韩琦自寻白绫,倒是把皇帝弄进福宁殿养着。

不过皇帝毕竟只是重病,并非昏聩。世人皆知,天子的心智依然清明,要不然也不会在垂危之际,仍能洞悉二大王的奸谋,让皇后垂帘听政。

万一让赵顼说出什么话来,那就真的是无妄之灾,纵然可以来个趋吉避凶,可也是少不了一身骚。日后也定然会被人拿着当做把柄,时时敲打一番。

也幸亏现在是皇后主持大政,暂时可以不用担心这样的诏令砸到自己头上。

但韩冈可没打算就此放下心来。

就在桌前,展开纸笔,韩冈开始给王安石写信:

‘乱命不诤,流言不禁,上不谏君,下不安民。敢问平章,平得何章?’

奉命前来的韩中信,瞥了一眼后就张着嘴合不拢。要有多大担子,才会给担任平章军国重事的岳父写上这样的信。

“枢……枢密,真的要送这封信?”

韩中信结结巴巴的问道。虽然他已经得了敇命,但还不是正式的官职,必须要经过朝廷的许可才算正式进入官籍。

现在尽管战争还没有结束,但到了这个阶段,已经没有多少韩中信立功的机会,正好可以回去走一遭,顺便送几封不方便走马递的私信。

“我不是说给平章听!”韩冈不以为然,将信纸折好收起。

他这是要逼王安石表态。

皇帝虽然还算清醒,可已经有了神智恶化的迹象,现在尽管能拦住,如果日后再下乱命为何?纵然还没有通过两府,但保不准以后就会有。

未雨绸缪,还是先让世人明白皇帝已经无力处理政事比较好。

……………………

六七天过去了,消息想来已经传到了河东前线。可从福宁殿传出来的话,在京城中掀起的惊涛骇浪依然未有止歇。

太过惊人的圣谕,使得两府诸臣不约而同的选择了沉默。对于市井中的议论,一时间也是听之任之。

皇帝的心思根本让人猜不透。

随着他在病榻上睡卧日久,心思和性格都向人难以理解和揣摩的方向转变。不过转变的方向是可以确定的,只会变得更坏,不会变得更好。

王安石不会去奢望他们还能瞒着赵顼多久。谎言无论怎么编,都是有破绽的,时间这么久了,想来皇帝已经看破了真相。

人虽然躺着,可心思却是清醒的,看破了真相,然后下一个无法捉摸的命令,最后闹得上下不安。

他究竟想做什么?

多少人考虑过这个问题,当然后得到了答案多多少少有些差异。

唯一可以肯定的,是没人愿意去相信皇帝仅仅是希望夺回故土,才下了这样的诏令。

在经过了去岁冬至郊祀那风风雨雨的一夜之后,皇帝的心机、城府已经为世人所认同。他的思路必然是九转十八弯,让人很难琢磨透。

韩冈在河东的胜利,其实是打开了一扇大门,让人们了解到了如何去与辽军作战。也看到了灭亡辽国的希望。

虽然说绝大多数世人对韩冈的这个胜利并不了解其意义所在,但他们这些执掌天下大政的宰辅,至少都能看透韩冈用兵方略的好处。

大宋的优势究竟在哪里,如何利用大宋的优势来克制辽军的长处。都是在这个胜利中看到的。

收复幽云的希望就在眼前,可赵顼的话,现在却让人不敢稍动。

封王并非好事,对绝大多数两府中人来说,这句话没有任何错误。

封了王后,还能指望再留在政府中?军权、政权、财权肯定都要放弃。

以吕惠卿、韩冈的年纪,会甘心就此养老?养个乐班、造间别墅,以娱天年?

一旦被封王,必为众矢之的。天子会看着,言官会盯着,一言一行都会被有心人加以解读。就像狄青当年成为枢密使,而被言官以及更为庞大文官群体视为眼中钉,疯狂的加以攻击,以至于英年早逝,让人不甚痛惜。

以韩冈和吕惠卿的才智,肯定不会去争那个王爵,他们的路还长得很。等五六十往后还差不多,吕惠卿还不到五旬,而韩冈更是在朝堂上还有三四十年的时间。拥有这样的未来,会愿意被当成囚犯?

但只为国事,王安石就不敢冒险。

国家财计已经支撑不了,而战争结束看起来还遥遥无期。

现在不能放弃和谈的机会,更不能让战争持续下去。

王安石给韩冈去信,给吕惠卿去信,更联合韩绛、蔡确,给郭逵下了严令,禁止他去干扰和议,也禁止他去反击辽境。

战争必须终止,赵顼掀起的风浪必须得到停歇。王安石夜不能寐,他只盼望皇帝不要再出难题了。
