\section{第三章 墙成垣隳猿得意(下)}

辽国使臣遣人搜购《浮力追源》,这个消息不过一天,就在京城中流传了开来。

有了辽国的看重,使得韩冈的名望又高了一层。只是市井中也多了些担忧,生怕板甲、飞船这一干利器被契丹人学了过去后,反过来对付起大宋来。

比如飞船,这些天听着从西边传来的消息,连洛阳的酒楼都开始学着东京的七十二家正店,开始在门前造热气球为店铺打广告了。结构这么简单的东西,一家酒楼就能学得来,东京城中也有了专门为人造热气球的店铺。契丹人若当真想要将飞船学了去,实在是太简单不过。

不过韩冈的一番奏对也一起传了出去,世人受了他的灌输,明白了一件事,不论是锻锤、飞船还是板甲,辽国、西夏想学过去,在技术上没有难度,只是工艺和规模上差的太远,比不上大宋财大气粗、技艺精巧,名工大匠数以万计。

虽然不知其中有多少人相信了韩冈的这番言论,但至少能稍稍安定人心。而对韩冈的计划来说,一点紧迫感还是很有必要的。当契丹人开始仿造板甲、飞船甚至雪橇车、霹雳炮之后,宋人想要保持技术上的优势,是将自己治罪,还是给自己更大的权柄,这个选择想来还是不至于会选错的。

韩冈今天正值休沐,就将一干心事丢到了一边去,安心的修养。朝堂上为了两件案子该吵还是吵,轮不到他来操心,休息的日子他是万事不理。

在家中穿了身宽松的衣服,韩冈很是悠闲自在。上午在书房里回了几封书信,又读了一阵书。等到中午,吃了严素心精心烹调的佳肴,就在微煦的阳光下小睡片刻。一觉醒来,又与王旖在房中随意下起棋来。

韩冈的棋艺差劲得厉害,连着输了两盘之后,王旖让了他一车一马,第三盘才杀得难解难分起来。

只是韩冈在对着棋盘苦思冥想,王旖还有余力分神说话:“最近大哥身体不太好,前几天娘娘来信,说大哥前些日子心口疼得厉害,在床上躺了有十来天,连几部新义的修改,都耽搁了下来。”

韩冈这时正凝神的盯着棋盘,王旖的车落得位置正好,现在他要在丢马还是丢砲之间做个选择。想了一阵,终于还是选择将马给放弃。抬手将砲挪开,随口就道:“你那两个哥哥身子骨都不怎么样,仲元这两年风里雨里的忙着,倒是康健了不少。元泽那是读书写书用心过度,耗用心神太多。本来就得要歇下来一两个月,将养一下身子方才会好。”

韩冈说得事不关己一般,王旖顿时眉梢就挑了起来,啪的一声响,狠狠的吃掉了韩冈的马。

王旖常常闹些小脾气,韩冈笑了笑,不与她一般见识。应了一手,又道:“太医局的雷简前日送了两张药方,说是日常补身子的,正好岳父的生辰快要到了,礼物为夫也准备好了。过两天,就让韩礼带人一起送过去。”

听到韩冈说起药方,王旖追问着:“药方子有用吗?”

“听说挺管用的,官家最近喝的药汤就是改了这个方子。要不是雷简过去承了为夫多少人情,他也不敢将两张方子拿给为夫。”“不过这也只是治标而已。真正要养好身子,还是多活动。”王雱身体一直不怎么好,韩冈也不是没劝过他,都说了好几年了,

“呼吸导引大哥也是常年在做着。”王旖为兄长辩解道。

韩冈嘿嘿笑着:“动功、静功那都是要做的,怎么能可以偏废?没看为夫常年锻炼筋骨之余,还不照样学了些导引调息之术。这叫做内外兼修,你大哥走偏了路。”

听着丈夫信口开河一般的批评兄长,王旖有点不开心了,落子就不再留情,啪啪啪的几步下来,就快要将韩冈的棋给将死了。

韩冈皱着眉头盯住棋局,王旖则翘着下巴,鼻子里哼哼着,很是有点小得意的模样。

这时候,管家韩忠在外面通报一声,走进来:“舍人、夫人,外面有一个汉子,自称是蔡御史的家人,有急事要见舍人。”

韩冈没动弹,看着棋盘,信口吩咐道:“问他带来的是口信,还是书信。口信让他说出来,书信就让他交出来。”

他韩冈是什么身份,蔡确家的下人说见就能见的?再有急事,也不能失了身份,将性急表现到外面来。否则就是有失体面,贻笑大方。蔡确与自家又不亲近,他韩冈可不会将笑话漏给外人看。

韩忠听了吩咐,就连忙出去了。

不过蔡确怎么派人来了?韩冈有些闹不明白——棋盘就那么放着,他也无心去下了,反正也差不多可以确定这一盘是输定了。

冯京已经两天没有上殿了。因为事涉厢军聚众反乱一事,纵是宰相,也得照规矩避嫌在家中。不过冯京也不忘上表自辩,里面顺道将韩冈骂了一通——虽然现在是吕惠卿在兴风作浪,但整件事起头的还是韩冈。

可就是在这个时候,冯京的亲家却是跑来通风报信,是嗅到了什么风声?还是想做个称职的两面派?韩冈一时间,也想不出个头绪来。换作是王韶、章惇家的人,那就好猜了。

过了片刻,韩忠拿了一封书信过来,双手呈给了韩冈。

韩冈接过信:“没有其他的话。”

韩忠摇了摇头:“没有。他只是奉命来送信,说是要面呈舍人。小人费了好一通口水,才让蔡家家人将信交了出来。”

韩冈点点头,打开信封,抽出信纸。仅是展开一看,神色顿时就变得古怪起来。左手上的扇子不由自主的在棋盘上敲了一敲,叹道:“想不到终究还是到了这一步!”

“官人,是何事?”王旖好奇的问着。

“嗯,你也该看看。”韩冈抬手将书信递给了妻子。

王旖接过来一看,顿时就是怒容满面。她这一回是真正的被气着了,将信纸往棋盘上用力一拍,也不管棋子落了满地,粉面含霜的怒道:“他们怎么敢将二哥也牵连进来?!”

“既然已经牵到了李士宁头上,当然会把元泽和仲元牵连进来,总不能直接找到岳父的头上去,许多时候,要绕一圈才能走到目的地。”韩冈冷笑着:“根究此案的目的不就是这个嘛?要不然早就结案了。有什么好气的?”

情涉至亲,王旖心头有些慌乱,忙问道:“官人,那该怎么办?”

“人还没走吧?”韩冈转头问着韩忠。

韩忠摇摇头,“他正在门房那里等着官人的回覆。”

“去跟他说,我韩冈今日承了他主上的人情,日后必有回报。”韩冈说得直截了当,完全没有此时文人惯常见的委婉。不过能传递这般重要的信函,在蔡确家中肯定是备受信重的亲信,让他转述也不用担心太多。

韩忠恭声应了就要出门去,但王旖从后面叫住了他,“从帐房支五贯钱去,说是赏他喝茶的。”

韩忠正要点头,韩冈却道:“没那个必要,一贯就已经很多了!”

“官人!”王旖转头急叫道。

韩冈偏偏头,对王旖笑着:“给得赏钱太多,会让人误会的,不能表错了情。”对上妻子惶急的眼,他笑着安慰,“不用担心,天子怎么都要顾全岳父的体面。你不想想,岳父岂是寻常的落职宰相?”

“但二哥他说不定会被收进诏狱中。”王旖为兄长急得都快要哭了出来。被牵连进谋反案中,怎么可能不进牢狱走一遭?说不定现在范百禄那边就已经去白马县抓人了。想那牢狱之灾,岂是寻常人受得起?进去一天,就不一定能囫囵个儿的出来。

“那是当然的,就算天子不想动,下面的人却还是会照样做些事出来。木已成舟四个字,会写得人太多了。”韩冈笑容恬淡,“不过从京城到白马一个来回,少说也要两天时间。有两天的时间,足够为夫把这摊子事给处理好了。”

在丈夫脸上自信的笑容,王旖一颗惶急的心,渐渐平复下来。就像今年的上元节,韩冈被请去宣德门城上时也是这样的一副表情。从容的笑脸,仿佛任何难题都无法对他造成困扰。而宽厚结实的肩膀,也似乎能将任何事一肩给担下。

柔顺的倚着韩冈,双手紧紧抓住了粗壮的手臂,王旖低声道:“一切就都要靠官人了。”

感到怀中妻子现在的软弱,韩冈反手拍了拍王旖纤细的肩膀,轻笑道:“其实我也是得要靠着岳父的积威才能成功,狐假虎威罢了。”

王旖点点头,却聪明的没有细问,只是细声又问道:“要不要派人去白马县,跟二哥说一声。”

“没那个必要!……说不定外面就有人正等着为夫这么做呢!”韩冈拿着乌檀折扇一下一下的,有节奏的敲着棋盘,笑容也一点点的转冷下来,“要下棋就得照着规则好好的下,像现在这般不守规矩的乱来,就别怪我掀棋盘了。”

