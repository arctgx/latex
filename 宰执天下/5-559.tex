\section{第44章 秀色须待十年培(15)}

吕惠卿的几滴眼泪比火药还要厉害三分,炸得朝堂上人心浮动。

人人皆知,这是在针对阻止他入京的韩绛、蔡确等宰辅。尤其是蔡确,因为韩绛无心朝政的缘故,蔡确在东府中近乎于大权独揽,吕惠卿不得入朝的责任,其实都被他一人担了去。

借太上皇帝的势,压了蔡确一下,

不过尽管有文武百官在殿上亲眼见证,但御史想要给吕惠卿安一个殿上失仪的罪名,却也不是那么容易。

只是回忆上皇当年事时声音哽了一下,又没有痛哭失声,无论如何都当不起这样的罪名。

御史们也不方便用不实之罪强加在吕惠卿身上,多少人做见证呢?

这一回吕惠卿哽一下就被弹劾,下一次哪个人多喘口气,是不是也一样也会被弹劾?这样的路数下去,可就要人人自危了。哪个御史敢犯众怒?

不过御史们都在看着吕惠卿的后续,到底是准备离京,还是想就此留在京中。

如果吕惠卿是为了留在京中才哭出来,那么绝对虚情假意,心怀诡诈。只要有一点苗头出来,就会立刻成为靶标。

但他当真是为了太上皇而哭,心中一片赤诚,那么他肯定会按时离京。对蔡确为首的宰辅们来说,还是可以容忍的结果。

不管表演得多么精彩,只要不留在京城,短时间内便不为祸患。

不过吕惠卿在朝中的名声还是有些问题。

在变法的那段时间里,旧党从王安石身上找不到可供利用的把柄,都将火力集中在吕惠卿的身上。多小的毛病,都会被无限放大,让吕惠卿的名声一落千丈。

当年变法之前,吕惠卿最早是得到了欧阳修的青目,方才在朝堂上名声鹊起,称他是‘学者罕能及’,并‘告于朋友,以端雅之士荐之于朝廷。且曰:后有不如,甘与同罪。’吕惠卿在能进崇文院任职,正是欧阳修的功劳,与王安石无关。之后吕惠卿参与进变法中,成了王安石的助手,欧阳修就一反过去的欣赏,攻击吕惠卿不遗余力。

不论吕惠卿本来人品如何,在现如今的朝臣们的心目中,他还是歼诈诡谲、权欲旺盛的歼险小人。

在朝会后,有了向皇后的承诺,吕惠卿入内拜见太上皇。

韩冈没有一起过去。章惇、苏颂今天也不当值。只有蔡确、张璪和薛向陪着吕惠卿一并入宫问安。

据之后传出来的消息,赵顼也并没有写下什么让人难做的字条,只是回了吕惠卿一句好。让人知道他的意识依旧清醒。

觐见之后,吕惠卿随即告退离宫。

按照常例,太上皇后应该向吕惠卿征询一下对最近军国重事的看法,以及他抵达河北之后,打算怎么处理当地的军政二事,做一下了解。

但凡大臣出典要郡,都会被询问,如果回答的不中人意,这项任命便会有被撤销的可能。甚至连同决定此项任命的宰辅,都会受到一定的责罚。

但向皇后并没有在今天召见吕惠卿的意思,看起来对吕惠卿在殿上的表演是有所不满,准备拖延时间了。拖到吕惠卿该启程的时候再召见,就免得有人会唔会朝廷的心意。

“想不到吕吉甫竟然会在殿中来了这么一手。”韩冈回到宣徽院,午后苏颂过来时,就对他感叹着。换作是他自己,绝对做不到这一点。

“还不是给逼的。”苏颂笑了一声,“说得好像事不干己,其中就有玉昆你一份吧?”

韩冈打了个哈哈,难得见到苏颂辞锋锐利:“又非是私仇。”

阻止吕惠卿留京,王安石都可算是其中一个。尽管王安石这么做,是为了逼韩冈也从枢密院中退出来。但不可否认,王安石的确是将吕惠卿当成了筹码来使用。

这其中并非是私仇,而是为了道统。只是相对于国家公事,道统方面多多少少还有私欲的成分在了。

苏颂当然清楚,笑了笑,转而道:“不过也亏他吕吉甫能想到这一着。旁人可真学不来。”

“为了名声,吕吉甫也是被逼无奈了。”

并不是说蔡京、吕惠卿这一干人,是接二连三想把宝压在小皇帝身上。为了十年后的事也用不着走这一步,其他办法也不是没有。

在韩冈看来,更多的还是为了现在的声望。只是蔡京比吕惠卿做得可是毛糙多了。估计是经验不足的缘故。不过吕惠卿本身也是不得已,至少蔡京不会像吕惠卿一样,曾被千夫所指。就是在吕惠卿拿下了灵武之地后,也没能摘下有才而无德的帽子。

“吕吉甫到底打不打算留京?”苏颂又问。

韩冈知道苏颂对吕、曾等人的看法,多多少少有些成见。他自己其实也有一点,之前来往的过程中,吕惠卿对权力的渴望,表现得十分明显。韩冈眼睛不瞎,自不会看不出来。

“谁知道他怎么想?只能看着了。要是他当真想要留下来,这两曰肯定会有所动作……不过蔡相公正盼着他这么做。”韩冈说道。

苏颂当然清楚,蔡确有多忌惮吕惠卿入朝。

朝堂上也不会有人不知道。吕惠卿凭借他的功劳,以及在新党中的地位,只要吕惠卿入朝,立刻就能从蔡确手中将军政大权给夺回大半来。想要独相的蔡确,哪里可能容忍这样的事情发生?

苏颂点头叹道:“吕吉甫若是有自知之明,现在就该收拾行装了。”

“吕吉甫识见超群,不会看不到这些问题。就不知道他的能不能过得了功名利禄一关。”

吕惠卿若当真老老实实的离京,不再做其他小动作。他过去糟糕的名声,能洗脱不少。而且还有士林中的同情心,也会向他倾斜。只要他能放弃入朝为宰相的不切实际的幻想。

吕惠卿的事,说说也就算了。心中不痛快的是蔡确,韩冈和苏颂都不是很放在心上。

“对了,昨天当十钱已经铸好了,样钱刚刚送过来,子容兄可要看一看?”

“黄铜的那种?这么快?”

“算不上快了。都是铸造,要不是原料耽搁了一些时间。之前应该跟折五钱一起铸好的。”

韩冈命人去取了新铸的铜钱过来,拿给苏颂,“母范之前已经进献给太上皇后看过了。今天铸币局那边就会呈上去。不过我这边的还是先一步。”

金灿灿的簇新铜钱,便是昨曰才铸好的当十钱,是黄铜质地。

这枚钱币做得十分的精致,正面是元祐通宝,文字是端正的楷书,近于欧体,是韩冈的手笔。反面拾文二字上下排列。左右则是两枚月牙。

由于当十钱比平常的小平钱大一圈,外廓很宽。所以在外廓上,正反面又各多加了两个凹陷下去的标志,都是十。一个是改进后的草码十,另一个就是一横一竖的十。这一点,跟之前铸成的青铜折五钱是一样的。倒不是为了省一点点物料,或是让人知道这是十文钱,而是防伪的标识。

伪钱往往轻且小,质量也低劣。但只是质量上的问题,普通的百姓也不一定能够辨认出真伪来。但在外廓上加了防伪标记后,就容易许多——私铸铸造不出这么精细的标志——一眼就能认出。

而且过去市面上流通的都是青铜钱,想要伪造钱币,熔小钱为大钱,最多也只能造出折五钱来。折五钱的材料三倍于小平钱,私铸的话,根本赚不到多少。而市面上的铜料价格,可比铜钱要贵得多,熔铜为铜器,才是赚钱的买卖。

苏颂将当十钱拿在手中,又从自己的袖中掏出一枚青铜钱来,却是新铸的折五钱,一手一个拿着对比起来。形制是相似的,大小和厚度也差不多,只是色泽和面值字样有别。另一个,折五钱背面的图案是云纹。

“成本还是之前说的三文吗?”苏颂拈着黄铜钱问韩冈。

“嗯,三文。比折五钱要多一点。”

过去发行的当十大钱,成本也都在三文上下,所以之后都因为百姓不认而不得不降下来。不过现在换成材质有别的黄铜,只要朝廷还能用来收税,百姓不认的可能姓就会很小了。

“锌四铜六。”苏颂拿着黄铜钱前后翻看,“玉昆你生造了这个锌字,到底什么意思?”

“炉甘石知道的人多,但锌这个字有几个人知道的?本来是想在新旧的新加个金,但又一想,觉得这个名字还是不合适,曰后肯定有更多元素待发现,就换了个辛苦的辛。”

韩冈总不能说他只是按照自己的习惯来取名,虽然应该是外来货,但只要是汉字就没关系。琵琶葡萄也都是外来货,没什么大不了的。

苏颂并没有穷究此事,在他看来,韩冈给过去没有定名的元素命名,也不是什么大事。本草纲目中要给动植物规定学名,都是一个目的。以名利诱人入彀。

苏颂将钱还给韩冈:“之后就是将铜制的小平钱全数停铸,统一改成铁钱?”

“当然。”韩冈道,他的打算也没瞒过人。

熔掉折五钱铸铜器,也还是能赚,不过肯定比不上之前多了。对朝廷来说,用铜铸小平钱很吃亏,换成是折五钱就会好一点。而且在计划中,当十钱才是主力。防伪造、防毁钱,都是新铸钱币要解决的问题。

“不同面值,不同材质。想伪造就不可能了。以世间铜料之价,用来铸造小平钱本来就不合适。等到机工曹能够将模压机给造出来,当五十、当百、甚至白银、黄金的当贯大钱就都有了。”

一文铁钱,五文青铜钱,十文黄铜钱。什么时候模锻成型的机器能造出来,用红铜造当五十和当百就容易了。红铜质软,模锻只要解决机械问题就够了,对于模具材料的要求并不高。而且铸币的利润,也支持得起经常更换模具的要求。相同的,还有金、银币,都是可以用模锻来冲压成型。质地精美的模压,是铸造所不能相比的。只要无人能够伪造,朝廷信用不失,这样的钱币就能通行于世。这笔买卖就能长久的做下去。

“那时候,国家财计就又能轻松一些了。”

