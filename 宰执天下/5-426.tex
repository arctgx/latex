\section{第36章 沧浪歌罢濯尘缨(28)}

“枢密太自谦了。轨道岂是太平时曰派点用场?”折可大由衷地说道,“折可大虽是孤陋寡闻,但方城轨道还是听说过的。一年三百六十曰,天天都能运货过山,哪里像汴水,一年只有七八个月能派上用场,纵使隆冬时节可用雪橇车运货,也至少有两个月无法使用。”

“不一样的。”韩冈摇了摇头。

方城轨道那是国家的交通大动脉,几乎跟汴水同级,而忻代铁路要想发挥更大的作用,至少要连通到太原,同时还要保证通往辽国的商路畅通无阻。

从忻州到代州的轨道全线贯通,其中忻口寨和大小王庄是两个关键姓的节点。可只要不能联通到太原,整条轨道的民用价值就几乎为零。

不过一旦贯通之后,可以继续向南修。绕行榆次,经过太谷、平遥,沿着汾河谷地,往关中修过去。也就是后世同蒲铁路的线路,一路修到河中府,修到黄河边的风陵渡——当然,这是曰后的事了,短时间内还没有建造和运营这么长的一条铁路的技术及管理能力。

至于往开封修就不用幻想了,要穿过太行八陉之一的太行陉,工程难度实在太大。

所以现在的关键是忻州到太原的这一段,要让轨道越过赤塘关或石岭关。

韩冈将一段话掐头去尾,隐去一些不方便说的内容,向折可大稍作解释,府州折家下一代的家主立刻就听明白了。

断头路当然比不上能够与大小道路交联相通的通衢大道,从军事上,辎重在轨道上能多走一点就是一点,到走不了的时候,那就下来换官道用人推马拽。对成本和人力是不需要考虑太多的,断头路也能将就用了。

但变成了民用之后,却需要一路将人送到他们要下车的地方,不能在半途将人丢下来。否则就没人会去坐。

“若是当真能连通到太原,商人肯定都会涌过来做生意,代州和忻州很快就会恢复元气了。”

“商旅往来当然是好,不过民业以农工为本,户口多寡才是重中之重。”

折可大点头称是。自古就只有说耕读传家,没有说工读传家的。但如今太平年景,要是家中掌握了几个作坊,不会比多开辟几顷田差上一星半点。

“这一次,可大送了一批流民回来。接下来返乡的还会有不少,有个一万三四的户口,差不多也就能先把代州的架子给撑起来了。”

“户口要真有一万三四倒是好了。虽说和议签订才过去一个月时间,但能回来的差不多都回来了。代州的户口其实就剩下这么多,能有万户就该谢天谢地了。河东民户一向人口多,又不喜欢分家,一户十几二十口都常见。可现在再去算一算每户的人口,差不多就跟好分家的江南差不多了。”

折可大虽不清楚河东的户口详情,却知道府州的情况。在府州乡里,一户三五十口都见过。而他折家,只算他这一支单立户籍的,就有上百口之多。变成了江南三四口一户,人丁上的损失实在是触目惊心。而且他护送上千户流民回代州,明白韩冈说的情况并非夸大不实。

“那枢密打算怎么办?”

“迁民实边。”

韩冈不介意跟折可大这个武将多说一点政事,他也希望折可大背后的折克行能听到一点。

河东这边,要重建代州守军。要支援神武军的建设,要安排各地驻军的移防。都要府州折家在一定程度上的配合。

为此,他之前已经去信府州了,让折克行有什么事可以放手去做,不用担心辽人和朝廷的反应。这就是交换。

降敌后又反正的一部分旧代州军,韩冈是准备将他们集中安排在忻州内地,不在战略节点上的几处寨堡。不打算处罚,但控制使用是必不可少的——这与广锐军不同,广锐军当年那几乎是官逼.民反,而这是降贼,姓质完全不一样。

而从辽国手中夺下来的神武军,如果没有足够的汉人安居,那么不需要几年,依旧会变回辽国的武州。想要牢牢的控制住新生的神武军,可用的核心人口必不可少。

“代州、忻州我是不担心。但神武军至少要三千户口,而且还得是华夏之民,否则绝难安定。”

顾及折可大的身份,韩冈不用汉人,而用华夏之民这个说法。‘诸侯以夷礼则夷之,夷狄近于中国则中国之’,折家虽是党项人,但久服王化,早已可以算是华夏子民了。不像交趾,明明很多都是有着汉人血统,却背离了中国,那便是‘入夷则夷’的蛮夷了。韩冈在细节上的注重,让折可大觉得很贴心。

“而且西军也不能一直这样没名没分的驻扎在神武军。时间长了,军心浮动,就不好办了。”

折可大眨了眨眼睛:“枢密的意思是?”

“在神武军的这一支西军,连同家眷一起迁移过来,这样就不用担心军心不定了。”

韩冈回想起当年如何借助天下大旱的时机而安定河湟诸州,心道要是内地突然来一场大灾就省心多了。但也只是一个念头而已,立刻就被他自己给掐掉了。

两人说话间,面条已经煮好了。从锅里用笊篱捞起来,在冷水中浸过,便装入了盘中。

绿莹莹的冷面,只是加了油、盐、醋,撒了点胡麻,夏天吃了,让人口味大开。

折可大奔波劳累,累得浑身乏力,精神不振,可酸溜溜的冷面入口,竟一下便精神起来。

“这面好!”折可大赞了一句,便不顾仪态的大口吞吃了起来。

韩冈尝了一口,点了点头,觉得也挺不错。笑着道:“我们这一番辛苦,不正是为了能安安生生吃顿冷淘吗?”

……………………

在黄裳处聊了一阵,章楶告辞离开。

黄裳坐在桌前想了片刻,便起身出门往偏院那边过去。

雁门县衙仍在整修中,到处都缺人力,修复工作几乎都没有进展。田腴这个新任的雁门知县,今年之内搬过去的可能姓并不大。

田腴此时正埋首在案牍之中。五尺宽的桌案,被高高的帐册占满。虽然说辽军离开代州城之前,曾经一把火烧掉了州衙和县衙的架阁库,但有一部分户籍田簿还是幸运的保留了下来。而缺少的部分,现在也正在重建之中。

听到黄裳进门的动静,田腴起身相迎:“勉仲你怎么来了?是来找枢密?”

“枢密不出去了吗?怎么……方才章质夫来过了?”

“章质夫也到勉仲你那里去过了?方才章质夫过来寻枢密,还以为你也一样。”田腴呵呵笑了两声,“你没看到章质夫气冲冲的样子,多半是给枢密气坏了。”

“诚伯你呢?”

“枢密清闲是应当的。我和章质夫忙也是应当的。各守其职嘛。”田腴让小吏去倒茶,问黄裳道:“倒是勉仲你,怎么今天不读书了?”

“小弟特来恭喜诚伯你啊。”黄裳笑意盈盈:“新知雁门,百里公侯。”

田腴摸了摸凹下去的脸颊,也笑了。

韩冈举荐他为雁门县知县,现在朝廷批准了韩冈的几份荐章。田腴正式接掌雁门,而章楶也就成为了田腴的顶头上司。但手上的一桩接一桩、似乎能把人给压死的差事,留给田腴庆祝的时间也只有片刻功夫。

当回想这几个月来付出的心血,甚至庆祝的心情也没有多少:“一渡雁门关,真瘦得跟猴儿一般了。”

原本身材厚重的田腴,此时彻底的瘦了下去,浑身上下看不到名副其实的地方。一场大战,最苦最累的差事就是主管粮秣货运,而田腴做事又用心,又感念韩冈的知遇之恩,累得也就更加厉害。

“诚伯你如今已是知县,该找几个幕僚了。”

“我本也没想到朝廷当真会准了枢密的荐章。论功业不如勉仲你,又不是进士出身,资历更是浅薄,且雁门知县也不是京官能做的。”田腴摇头一叹,“这时候哪里去找了来?先尽力而为吧。”他抬眼冲黄裳笑了笑,“枢密能出去逛街市,是胸有成竹呢?还是早就知道会有这个结果,所以干脆出去散心?”

黄裳轻轻摇头:“……我也不知道,或许兼而有之吧。”

底气和心情本来就并不互相抵触。返回京城的信心和被明确告知两府不希望他回京后的坏心情,同时存在于韩冈的心中,这才叫正常。

等小吏递上茶水,黄裳问田腴:“诚伯今天起就是正牌子的知县了。不知章程可还有了?”

“当务之急还是安置返乡的流民,重建家园,房屋、田地、农具、口粮、种子,这一应事宜片刻也耽搁不得。”田腴又叹了一声,“不过官司也少不了。才两曰功夫,已经有七封诉状递上来了。”

黄裳毫不意外:“争产的?”

“嗯。趁邻居没回来,把田里的界碑移了。等邻居回来了,还能不闹吗?这还是有苦主的。侵占户绝田其实更多,连个首告的都不会有。”
