\section{第28章 官近青云与天通(16)}

吕公著和韩维离开了城南驿后,同行了没多久,也告辞分散,各自回府。

刚刚回到位于旧城左厢第一区的枢密使府邸,吕公著便得知刑恕已经回来了。

踏进偏厅,刑恕在其中已经坐了很久。不过一见到吕公著进来,便站起身,迎上前。

“和叔这一番辛苦了。”吕公著立刻让刑恕落座,温言说道。刑恕只用了十天,就到洛阳绕了一圈,说辛苦也的确是辛苦了。

“不敢。”刑恕恭声道,“枢密为国事劳烦伤忧,刑恕感同身受。既有命,自当效犬马之劳。”

几句客气话说过,刑恕看看吕公著的脸色,问道:“司马端明终于入京,枢密今日去城南驿,怎么不见喜色。难道是因为王介甫的缘故?”

刑恕在离开洛阳后去了嵩阳书院一趟,虽然比司马光早了一天出发,入京却要迟上半日。他进城后,径直来到了吕公著的府上。司马光已经抵京的消息,还是他到了吕府之后才听说的。王安石犹在城南驿的事,刑恕也一并知晓,自然有此猜测。

吕公著摇摇头,“王介甫也算是旧友,如果只论旧谊,倒也没什么关系。我旧日与君实、持国,以及王介甫相往来,情谊甚笃。今日能重聚,也是一桩喜事。倒是韩冈在侧,说了半日的闲话。”

“韩冈也来了?”刑恕声音一沉,带着怒气道,“难道他敢对枢密不敬?!”

“这倒没有。”吕公著摇摇头,韩冈要是那般浅薄倒是好办了,“韩冈在席上持壶倒酒,比公休【司马康】和王安石家的儿子都会做事。”

刑恕眼角抽搐了一下,随即厉声道:“大奸似忠,大佞似信,外似朴野,中藏巧诈!”

吕公著咧嘴笑了:“这是当年吕献可【吕诲】弹劾王介甫十大罪状时的话,现在倒是用在了他女婿的身上了。”

刑恕摇摇头:“以此言来攻王介甫,未免沦于诟骂。但用在韩冈身上,却是不为过当。”

“但韩冈正得圣心。更得人心。”

“……天子虽然卧病在床,但依然能发号施令,只是麻烦了一点。”刑恕笑道,“而且病卧在床久了,性情也会逐渐改变。俗言道:久病床头无孝子,也不光是子女孝心不足的缘故。”

吕公著皱了皱眉,刑恕说的虽是人之常情,但如果在公开场合这么说话,就不是御史弹劾那么简单了。而且听着也不舒服。

孝道重于天,不孝那是‘决不待时’的十恶不赦之罪。父母再有过错,子女都没有不孝的理由。要不然父母首告子女不孝,就不会是直接论死了。

刑恕正看着吕公著的反应,见他似乎有些不快,立刻改正道:“太子自无不孝之理,但太子纵然再孝顺,天子的心情也很难说会有多好。韩冈一直仗着药王弟子的名声牟利,眼下天子重病卧床,却不设法挽救,自是不知忠孝何在,枉顾君恩。所以说,一切还在天子身上。”

吕公著点了点头,对刑恕的话表示赞许。韩冈聪明就聪明在从来不承认什么药王弟子,但这一回为了定储之事,却硬是拿了药王庙来发配两位亲王。这样一来,有些事可就说不清了。
“若韩冈不是总是拿着药王弟子的名声来诓骗世人,什么计策都对他没有用。但眼下他既然放言出来,可谓是作法自毙。何况还有殷墟,那件事可还不算完。”刑恕冷笑道。不需要明韩冈之罪,只要让天子这么想就行了。

刑恕的为人品性,吕公著多多少少也能看得出一点,只看他在自己面前只提司马端明,而不是司马宫师,就知道他是个很聪明很小心的人物——东宫三师虽然平级,但太子太师还是要比太子太保高一点——至于君实先生之类的称呼,更是不见他用。

在吕公著看来,这个门客还是很有用的,不是读书读呆了的士子。若当真是个守礼君子,反而就不方便使唤到他了。

韩冈那边可以就按照刑恕说的去做,慢慢动摇皇帝皇后对他的信任。失去了信任,就算还是太子师,也不用担心日后。

而眼下,吕公著眼神陡然变得狠厉起来,还是得先将王珪赶出朝堂去!

……………………

东方天空泛起的红光撕破了夜幕,随着晨钟敲响,宣德门的侧门被缓缓地推开。

聚集在门外的朝官们随即鱼贯而入。不过在行走时,许多朝官的都在交换着眼神,仿佛有暗流在涌动。

今天的朝会,引人注目的地方,一个是十余年不见的司马光来了,另一个则是王珪这名宰相并不在场。

司马光进了皇城。他今天是入觐,不是入对。觐见监国太子和听政皇后的地点,并不是在崇政殿中,而是在举行朝会的文德殿上。

入觐和陛辞都是礼仪性质,重要的是入对,与天子议论政事,而不是听着阁门使或内侍呼喝,依照礼节在殿上拜礼。司马光这样的重臣,抵京后在宣德门报了名,第二天就能入朝上殿,但想要奏对,就得排队了。

不过没人怀疑司马光能不能入崇政殿奏对,昨天他初至京城,就连王介甫利韩冈都登门造访,与吕公著、韩维把酒言欢,怎么看都有资格网崇政殿中走上一遭。

至于王珪,这几天,上百封弹章砸在了他的头上,只能照规矩闭门待罪,不可能厚着脸皮来朝堂上。所有文武朝官,都知道到底发生了什么事才动摇到了王禹玉的地位,更知道,他脱身的可能性实在不大。

只要看看宣德门中几位正在监察入宫朝臣的言官的眼神,就知道他们肯定是不依不饶。

“玉昆。”章敦也在看着那几位言官,痛打落水狗的弹章,几位监察御史和监察御史里行这些天来没有少写。带着一丝幸灾乐祸,“王禹玉这一回可是脱身不得了。”

韩冈摇摇头。在朝廷中待了久了,这个气氛如何感觉不出来。他冷笑道:“乌台今天是要发难了。”
这是自然的。宰执之中最方便下手的只有王珪!

吕公著在冬至之夜的表现依然可以说是忠,因为他并不知道内情,但王珪就完完全全的首鼠两端,小人之尤了。
皇后总是将弹章留中,御史当然只剩下在朝会上发难一途了。甚至在几天前,皇后将弹章初留中,包括韩冈在内,就有不少朝臣预计到了会有这一天。
而且王安石今天还不在。王安石是五日一上朝,也就是跟所谓的六参官相似——一个月上殿六次,今天并不上殿。要是他在的话,定然会站出来整顿朝堂秩序,不会让御史打乱朝会。御史们当然知道王安石能起的作用,肯定时要避开他。
但章敦总觉得韩冈的语气有些怪,有些担心看着他,低声问道:“玉昆,你该不会保王禹玉吧?”
韩冈跟王珪关系不差,这是章敦一直以来都清楚的。无论如何,王珪旧年也帮了韩冈不少的忙,尤其是举荐张载入京一事上,是王珪搭了一把手。
而且章敦也知道,韩冈同样希望王珪能留在朝中。维持朝堂的稳定,韩冈的立场应该跟病榻上的天子差不多——痛恨王珪的皇后将所有弹劾王珪的奏章一并留中的决定,只会是来自于福宁宫中的授意。
但眼下王珪的困境来自于他本人的过错,向皇后对王珪恨之入骨,若是有谁帮王珪说话,徒徒惹上一身骚不说,向皇后那边也交代不过去。
“留王禹玉在朝堂上,当是天子的心意。但王禹玉犯了那么大的错,皇后也不可能为了保他而将御史台都清空。”
章敦瞥了韩冈一眼。这位新晋的翰林学士应该很明白,向皇后对他的信任度肯定是在朝堂诸臣之上。维护这一层信任关系,比起保住王珪更为重要,重要过百倍。
章敦说的,韩冈都明白。殿上发难,弹劾和被弹劾的双方非此即彼,无法再同立于朝堂,天子必须要做出个决定,再不可能用留中的手段来敷衍。可谓是形同要挟。若是仁宗那样的天子对此还能一笑了之,但刚刚得掌大政的皇后呢?韩冈不是歧视女子,但比起心胸,胜过仁宗的皇帝史上并不多见,更不用说皇后了。
“韩冈不是要保王禹玉,也不会保王禹玉。”韩冈摇摇头,“但今天是王禹玉,明天又会是谁?朝堂不稳,得意的又会是谁?”

前几天已经有弹章砸到了自己的头上,他跟御史台多有旧怨,尤其是张商英,现在已经是殿中侍御史,若是给他弹劾了王珪成功,凭这份功劳日后不定会怎么恶心人呢?

而且朝堂中暗流暗流,还不如一股脑的爆出来,越拖到后面,越是麻烦,不管吕公著有什么盘算,也不管司马光还有什么心思,韩冈可没有坐等他们发招,自己来个后发制人的想法。
……………………
朝会已经在进行中。

文德殿上的御榻空无一人,太子的座位在御榻下一阶的台陛上,赵佣端端正正的坐着。帘后的皇后则设座在御榻旁,只能看到影影约约的一个身影。

张商英双手捧着笏板,静静的等待着。紧张感传遍全身,心脏剧烈的跳动着,如同擂鼓一般响亮,他甚至不得不深呼吸,缓解这莫名的兴奋。

在御史台的计划中,他将是第一个站出来弹劾王珪的御史。弹劾一名宰相,将之逐出朝堂,这是一名言官莫大的光荣。而这首义之功,将会是他张天觉的。

群臣参拜太子、皇后。

辽国告哀使上殿辞行。

朝会上的事项一件件的按顺序往下执行。

待皇后颁下赏赐,辽国告哀使离开殿堂,接下来就是外臣觐见。当头的,自然是太子太师司马光。

听到内侍宋用臣唱着司马光的名字,张商英一下捏紧了笏板,腰背也更加挺直。御史们不想跟司马光为敌,并不打算抢在他前面。但等司马光结束了觐见之仪,就是他张商英的领衔出场了。

在殿中百官的注视下,司马光走到了大殿中央,但他并没有叩拜,而是持笏躬身,声音朗朗:“臣,判西京御史台司马光,有本奏于殿下!”