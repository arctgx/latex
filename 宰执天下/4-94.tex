\section{第17章 籍籍人言何所图(下)}

王旖往父母所在的院子里走去,两个随身婢女慌慌张张的跟在后面。

“夫人,仔细脚底下,别走得太快,小心动了胎气。”

尽管王旖其实走得并不算快,只是随身婢女在大惊小怪而已,但她还是小心的将脚步慢了下来。为了肚子里的骨肉,她可不敢有哪里疏忽大意。

如今韩家不仅是云娘和周南已经身怀六甲,就是王旖在韩冈走后没半个月,也被确诊了喜讯、有了身孕。

家中的主心骨不在,一下有了三个孕妇,府中里外事务只靠严素心一人也管不过来,而且妇道人家又是妾室,与外界打交道也是麻烦。到最后也只能依靠王旖的娘家,借用了相府的一间偏院,全家上下都住了进来。

多了一大家子,还包括三个小儿女,相府中一下就热闹了许多,不再像过去那般冷清:除了人数绝不算多的亲友、门客,以及不多的仆役,一家上下都不到十口人。

小孩子都在王安石和吴氏的院子玩在一块儿。韩家的三个儿女精神极好,拿着球跑跑跳跳,追逐吵闹着。而王雱的幼子虽然没有他的表兄弟们玩得这么疯,可因为近来学着韩冈家照顾孩子的方法,比起过去病恹恹的情况要好了许多,也是很有精神的在又叫又跳,红扑扑的一张小脸,很快就是满头大汗。

吴氏就在旁边含笑看着小孩子们玩闹,手中拿着张图样在绣着。虽然孙子外孙们闹腾得厉害,但家中多长时间没这么热闹过了。若不是因为王雱的病情一天比一天重了,吴氏的心里会比现在高兴得更多。

不过看见王旖进了院子来,小孩子们立刻就不敢闹了,金娘带着两个弟弟韩钟和韩钲乖乖的过来行礼。吴氏不高兴的转过来,责怪的说了一句:“二姐儿你现在有了身子,不要走动得太多,要好生养着身子。也不学学你家的南娘和云娘,晨昏定省后,就安安静静的养着多好?!”

“娘你是怕女儿管着金娘他们吧?”王旖笑着坐下来,让金娘他们到院子里稍远一点的地方闹去。她知道吴氏多喜欢小孩子,金娘他们在有着洁癖的吴氏面前,就算打翻了花盆,弄脏了床罩,砸坏了窗户,都是有对无错。又对吴氏道:“官人说着就是有孕的时候,也要每天走一走路,这样才能有力气生养,云娘和南娘在院子走动得也不少。”

“姑爷连这事也说!”吴氏不高兴,只是立刻想起来韩冈在稳婆中也是有着偌大的名气。

“官人也是担心女儿才这么说的。”

看着女儿提起韩冈一脸幸福的模样,吴氏心头哪有半分不快。二女儿夫妻俩人感情好当然是好事,比起大女儿不知要幸运多少倍,但家里面的事情,不光在夫妻两人之间:

“听你爹说,亲家公两任轮满,考绩年年都在上选,到了年中时也该上京了。自从你和玉昆成亲之后,两边亲家都没有见上一面,亲家公亲家母逢年过节都让人送了陇右的风物过来,我们这边却没多少合适礼物送回去,怎么看,都是我们这边失了礼数。”

“姑姑还让人捎口信来说,娘娘托人送去的团茶、缎子,家里都喜欢得很,还要女儿代为致谢。”

韩冈娶了王旖之后,两家依节庆也有礼物往来,虽说两边是一个是宰相,一个则是农官,但放在亲家这一条上就没有尊卑之分,王安石夫妇更没有高高在上的态度。

“借花送佛,也算不上诚意。”吴氏将绣花针往针插里面一收,对王旖正经起来训诫道:“亲家公和亲家母年纪也大了,就一个儿子还不在身边上。说起来也是你不对,二姐儿你既然已经是韩家的媳妇了,照规矩也该留在陇西守着,怎么能陪着玉昆一起出来。”

韩家夫妇两人就剩韩冈一个儿子,老夫妻两个留在陇西做官,身边都没有照应,只能靠着冯从义这个表侄来服侍着。要是传扬出去,对韩冈的名声免不了也有所损伤。

王旖有些委屈,当初可是韩阿李急着要儿孙满堂,才催着自己上京的,要不然她就算再舍不得,也不会不守世间的规矩。只是亲娘拿着纲常训诫,她也不敢分辨。

吴氏训了女儿几句,也舍不得再训了,自家的女儿能在身边常常相见,她哪有不高兴的,转而又道:“也不知你爹哪里到底要说到什么时候,病还没好透,就坐不住了!”

王安石一边要处理着繁重的公务,一边还为着长子的日渐沉重的病情而担心,加之朝堂内外、天南地北都没一个消停,累得一日.比一日厉害。勉强撑过了一个冬天,春天的时候就有些撑不住了。并不是王安石过去因为与赵顼之间的纷争,而做出告病的姿态来要挟,而是当真生了病,这些天下来,已经请了十天的病假。

天子派来的御医一个接着一个,而来自宫中的中使,一天要往相府跑上三五趟。王安石休息了一阵子后,身体也康复了许多。也能接见来探病的官员——因为已经是参知政事,要避嫌的情况下,吕惠卿就算以探病的名义也不便多来拜访王安石——传递朝堂要务的工作则是交给曾孝宽:

“章子厚、韩玉昆此次功绩非小,以天子的心意,章子厚当为端明,而韩玉昆则是从假龙变大龙。”

王安石坐在韩冈送给他的摇椅上,膝盖上盖着羊毛毛毡,春日午后的阳光洒在他和曾孝宽的身上。眯起眼睛,慢悠悠的晃着:“想不到都做了直学士了,才二十五啊……”

世间俗称龙图阁学士为老龙,直学士是大龙,侍制小龙,韩冈之前的直龙图阁为假龙,至于在直龙图阁任上至死不迁的倒霉鬼,则被人笑称为死龙。韩冈跨过了侍制一级,跳到了直学士的级别上,就如同鲤鱼跳龙门一般。

“国之大事,在祀与戎。祭祀天地神明先祖,只是日常之事。而在勋表之上,唯有军功最重!”说起韩冈升官的速度,曾孝宽虽然也是咋舌不已,但细细想过来,却是再合情合理不过:“当年王子纯接连得授资政殿学士、观文殿学士二职名,破了非执政不授的旧例,就是靠了在陇右开疆拓土的军功。如今邕州大捷,以千人之众破十万之敌,俘斩逾万,章子厚从龙图阁学士晋端明殿学士,由阁升殿;韩玉昆从直阁升直学士,也都是应有之理。”

“包孝肃可就是直学士到了顶,而侍制更是做了多少年,外面一提起包孝肃,可多是喊着包侍制。”包拯是王安石的老上司,当年在群牧监,王安石和司马光都在包拯手下任群牧判官。包拯名重天下,世所共仰,最后连个正牌的殿阁学士都没有做到。而自家的女婿年纪轻轻就与其平齐,尽管并不认为韩冈当不起,但也免不了心生感慨。

“包孝肃乃是时运不济,要是从枢密副使任上退下来,少不了一个资政殿。”

王安石一叹:“也有诤臣也不为人所喜的缘故。”

文学之职不是熬资历能得到的,除了少数要在馆阁中做事的官员,绝大多数是赠以臣子美名的加官。想得到更好的职名,要么靠功劳,要么得天子或是宰相们喜欢,要么就从翰林学士往上走,升到宰执以上。

包拯曾是龙图阁直学士,后来又做到了枢密院直学士,虽然清介之名传于天下,京城百姓至今仍感念不已,连河湟蕃部的首领都要求赐包姓。但他并无军功,殊乏文名,且时常教训仁宗皇帝、喷他一脸口水,加之做枢密副使的时间又太短且故于任上,所以在文学职名上最后也只做到了直学士。

相反地,许多籍籍无名之辈却因为各种各样的原因得到了学士的头衔,如龙图阁、天章阁、宝文阁、观文殿、资政殿等等的殿阁学士,现在朝中还有好几个让人记不住姓名的臣子得手了。

不过就像本官进了朝官序列后,品级的晋升意义已经不大。文学之职的升迁,过了侍制这条线,也就不算很重要了,只是听起来好听罢了——到了一定的位置后,资格已不再是阻碍,剩下的就是对差遣的争夺了。

“广西转运使李平一是从六品的司封郎中。玉昆的职名升了,本官依例也要因功超迁,为从六品的郎中。本官两人平级,而职名上则玉昆远远过之。以下凌上不合法度,转运使之位,肯定要由玉昆来接手。至于李平一,这一次也算是薄有微功,吕吉甫的意见是要他转去广东。”

“黄金满的职司怎么定的?”韩冈任转运使是情理中事,能力摆在那里,王安石只问着其他的封赠,“此次邕州大捷,他是出了死力,没有他的五千兵马,十万交趾军,就是玉昆也无能为力。”

坊间传言说是韩冈以千五破十万,其实是把黄金满的五千兵给省略的结果,黄金满和他手下广源蛮军在邕州大战中所起的作用,朝堂内外都很清楚。

“两府拟议将广源州由刺史州升为团练州,任其为广源州团练使,并广南西路蛮部都巡检,并官其子二人,黄元黄全皆有任命。”曾孝宽为王安石叙述着正在商议中的功赏:“另外供备库副使、荆湖南路都监李信,则是文思副使、权发遣广西钤辖——正好钤辖高卞战死在邕州;桂州军判苏子元说服了黄金满投效,孤身收复昆仑关,又在战中有辅佐之功,加上其父苏缄在邕州恩信卓著,且是阖门死节,当以其权发遣邕州,并由武资改文资、为太子中允——不过苏子元当丁忧守制,现在中书还没定下来要不要夺情。”

提起苏缄,王安石也免不了要黯然一叹。

他对阖门殉国的邕州知州有些映象。虽然苏缄最后一次进京时,他还在江宁,并没有碰上。但苏缄前几次进京,王安石都在政事堂中,也接见过他。虽然因为宰相一天要见的官员太多,没怎么多说话,但苏缄给王安石留下的就是一个才具出色、老于政事的边臣形象。只是之前都没有将苏缄的警告放在心上,如果当真听了一句,也许就不会有这一次的大乱了。

“先将苏子元招入京城来,天子应该想见一见他。而且对交趾的方略,问他也是最合适的。”

如果北面能腾出手来,自当发兵剿平交趾。但就不知道北面能不能平静下来。因为荆南军表现出了出乎意料的战斗力,让天子对禁军的信心大增,进而对萧禧的讹诈,态度一转变得强硬起来。看这势头,肯定是要收复丰州了。

