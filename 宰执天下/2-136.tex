\section{第27章 京师望远只千里(六)}

【等一会儿还有一更,求红票,收藏】

茶香袅袅,琵琶铮铮。

长安京兆府的驿馆中,韩绛盘膝坐在枣木打造的软榻上,闭着眼,和着琵琶声打着拍子。一袭青色的道服松松穿在身上,头上没带冠冕,仅插了一根木簪。留着一把长须的韩绛,现在看上去只是一个悠闲自得的老书生。

韩家世代簪缨,出身灵寿韩氏的韩绛,是决不输相州韩家的世家子弟。自幼传习家学,承受父兄之教,越是心浮气躁的时候,越是会表现出士大夫的气度来。即便是刚刚跟知永兴军的司马光——永兴军就是京兆府的军额——起了争执,他现在的脸色上也没有表现出半点不快。

韩绛以宰相之尊,而且是兼任昭文馆大学士的首相,当然不是他去见司马光,而是司马光来拜会他。所以韩绛住在了驿馆中,而不是府衙里的寅宾馆。

只是司马光和韩绛都是同一辈官员中的佼佼者,韩绛不过是先行一步而已,论名望,论资历,司马光绝不在韩绛之下。所以司马光来拜会韩绛,仅仅是将表面的礼数尽到,对于韩绛在永兴军路军事上的指手画脚,他都是冷淡而礼貌的全部拒绝掉。不生事,这就是司马光的政见。不论是整修城防,还是用兵横山、河湟,又或是推广将兵法,他都持反对的态度,根本不跟韩绛合作。

韩绛实则心头怒火中烧,这段时间,司马光没少在陕西军务上大放厥词,要不是大顺城那条路通庆州的路被大雪封道,他何苦到京兆府来跟司马十二碰面。

韩绛本是要去环庆路巡视,可是一场暴雪毁了陕西北部山区的交通,让他不得不绕行到长安来。因为已经向南绕行了几百里,再往北去庆州,就来不及在预定时间内赶回延州。所以韩绛现在是在等,等接到通知的环庆路的主要将领赶来长安。

“相公,王文谅到了。”韩绛的随身老仆进来禀报。

韩绛没有理会,只等一曲奏罢,带着颤声的尾音绕梁而过,渐渐消散,他才睁开眼,挥退了弹奏琵琶的随行家伎,让下人传话给王文谅:“让他进来。”

王文谅躬着腰碎步走了进来,完全没有在道边客栈中的狂妄,恭顺中带着一点拘谨,跪在地上行礼时,就像一条对主人忠心耿耿的忠犬。

“怎么这么迟才到?”

“正好在路上遇到大雪。马嵬驿的房子也全塌了,只能住到个客栈里面。想不到还凑巧遇上了秦州的韩冈,还有广锐军的吴逵……”

王文谅在韩绛面前,不像普通官员一样畏缩、不敢多言一句,而是不厌其烦地把事情都说出来。他也不隐瞒自己和吴逵的矛盾,以及在客栈中的一番争执,只是隐去了他那句狂妄的话,很巧妙的变成了跟过去争夺马匹一样,争夺房间闹出的乱子。王文谅先入为主给韩绛留下印象,日后再传出对他不利的话来,也可以说是吴逵散布的谣言。

王文谅当个旅途闲话一样说得轻描淡写,韩绛便没去多想,小事而已。“韩冈、吴逵没跟你一起来?”

“小人不敢耽搁,只待雪势稍减,就往京兆府赶来。至于吴逵和韩冈他们的行程,小人就不知道了。”

韩绛满意的点着头,这就是他看重王文谅的原因,“若人人都像你这般用命,何愁北疆不宁?”

“小人只是不敢有负相公的看重,当不起相公夸赞。”

“韩冈吗……能得种五【种谔】、赵公才【赵禼】齐荐,才识自是不缺。随军疗养、沙盘军棋,这些虽是小术,但对军中不无裨益,也难怪天子也看重他。”

‘只可惜不是进士……非经正途而出,此辈可用,却不可重用。’后半句韩绛留在了心底,并没有说出来。但不管怎么样,对于韩冈的到来——即便并不是到宣抚司来报到,只是经过长安赶去京城——韩绛也是乐于屈尊见上一面,看看最近暴得重名的韩玉昆,究竟是个什么样的人物。

……………………

从兴平县到长安城的八十里路,韩冈一行走了两天。他和吴逵带队紧赶慢赶,也没能追上王文谅,不过还是在重新上路的第二天午后,抵达了长安京兆府。

暴雪后的长安城,有着非同一般的喧闹。

就跟秦州下雪之后会组织厢军出来铲雪一样,当韩冈一行从西门进城来。沿途看到了许多厢军士兵扛着木铲,在清理大街小巷中的积雪。四十多步宽的主街,厚厚的积雪都堆到了路边。从横街的街口、巷口望进去,也都铲出了一条供人行走的道路来。就在雪停后的第二天,长安城的交通就已经回复,至少可以看得出司马光做得并不差。

韩冈上一次来京兆府,就是在今年的上元节时。当时他在驿馆中巧遇种建中、种朴兄弟,还有他们的叔叔种詠,谈天说地,畅快无比。可惜如今种詠因李复圭而瘐死在冤狱中,种建中和种朴兄弟现在正跟着种谔在绥德,再见之日,不知是何年了。

昨天,韩冈跟吴逵聊天时曾提到了李复圭造的那一场冤狱,酒后的广锐军都虞侯差点掀翻了桌子。李复圭为了掩盖自己指挥上的错误,斩了大将抵罪,并关押了种咏,致使其病死在狱中,这件事,关西官场无人不知。但种詠三人以下,还有十几名没有官身的军校也一起陪了上法场,这一茬却没有人提及。

相对于高高在上、从外地调来的三名将领,十几名环庆军中沉浮多年、亲朋好友无数的军官无辜被杀,才是让吴逵、乃至整个环庆军都愤恨不已的一桩痛事。

而如今韩绛信用王文谅,偏袒蕃人,广锐军上下没有不恨的。今次韩绛要巡视诸边军州,但环庆路近日大雪封山,北线大顺城无法走通,只能命令环庆众将到京兆府相会。王文谅从庆州收到消息急忙南下,而吴逵辛苦巡边回来,看到命令也匆匆赶往京兆府,这就是为什么两人会相会在兴平县的一间小客栈的原因所在。

与王文谅不期而遇,吴逵只觉得自己沾了一身的晦气:“王文谅这厮最是阴毒,惯会争功诿过。他手下有一蕃将唤作赵馀庆的,本是两人约期至金明故寨巡边,但王文谅走到半路,听说前面有敌,便退了回去。等赵馀庆抵达金明寨,发现没人来,也撤退了。这件事本是王文谅有罪,但王文谅却妄称赵馀庆失期不至,害得他到现在还关在牢里。”

韩冈暗自冷笑,这王文谅也是本事,把韩绛蒙得耳目双盲,偏听偏信,这样昏聩的主帅,真的很难让人放下心来。

韩冈和吴逵边聊边往驿馆行去,只是到了驿馆所在的厢坊中,两人就一下停住脚,整个队伍也一齐停住。

京兆府驿馆的周围,现在围着一圈护卫,少说也有两三百人之多。大门前的站着两名高壮如熊的大汉,一柄长柄的白色战斧,不过斧身要比普通的战斧大了一半去。

“钺!”

韩冈顿时明白了究竟是怎么一回事。节钺——符节、斧钺——是象征臣子代天巡狩的礼器,所以过去有个假节钺的名目,非重臣不与——这里的‘假’是‘借’的意思。陕西一地,得赐节钺只有韩绛一人。以宰相之尊开幕陕西,当然要赐节钺,张旌旗。

当今名义上的首相韩绛现在就在驿馆中。

吴逵、韩冈领头,一群人下马后,慢慢走近驿馆。守门的人群中出来一名军官,高高典起的肚腩看起像个将军:“此时大丞相行辕,过往众官不得妄入。”

“我乃邠宁广锐军都虞侯吴逵,奉命来此拜见相公。”吴逵从怀里掏出一份公文,递交给守门的军官。

军官正要打开了公文,一个三十上下的中年文官突然急匆匆地走了出来,转身时一眼瞥到了吴逵。

“吴逵,怎么现在才到?!”中年文官也不等吴逵谢罪,“还不快点进去拜见韩相公?”

“游军判,下官……”

“别磨蹭,还不快点给我进去。”中年文官毫不客气的指使着吴逵,“已经有人在驿馆里住了五天了,还想让人等你多久。”

韩冈在旁看了半天,先是觉得眼熟,过了一阵终于想起了中年文官的的身份,“可是游景叔?”他突然提气叫了一声。

“……在下正是游师雄。”中年文官疑惑的看着韩冈,虽然眼前的这位高个儿的年轻人是跟吴逵一起前来,但怎么看都不像是武夫。一时想不起究竟是在哪里跟他见过,中年文官终于放弃回忆,低声问道:“兄台是……”

韩冈笑了一笑,上前一步,躬身行礼:“小弟韩冈,拜见景叔兄。”

游师雄两眼一亮,惊喜叫道:“你就是韩玉昆!?”

韩冈轻轻点头,与游师雄重新见礼。吴逵在旁看得惊叹不已,暗道韩冈果然是横渠弟子,交友遍天下,哪边都能碰到熟人。

