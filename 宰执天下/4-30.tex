\section{第六章 仲尼不生世无明(上)}

王雱站在船头,一张削瘦的脸苍白中泛着青灰。近一个月的舟船劳顿,让他原本就不算健康的身体,越发的瘦弱了起来。

只是离开了半年多,终于重又回到了天下的中心,这份兴奋,让王雱又重新提振起精神,贪婪的看着沿途的一草一木、一人一屋。

汴河两岸的风物百看不厌,一座接着一座的横跨汴河水面的虹桥更是让他心潮起伏。已经出现在地平线上的城墙,没入云霄的铁塔,岸边的青青杨柳随风轻舞,无数行人车马走在路边、行在桥上。离着东京城尚还有十里,周围的屋舍便已经是鳞次栉比,富丽繁华之处,王雱经历过的州县,无一处可以比拟——这一座城市才是他立足之地!

巨大的官船在码头上停了下来,一名内侍就站在栈桥上。天子派了亲近出城迎接王安石这名宰相,并招王安石进京后即刻入宫相见。

王安石在朝野中的地位声望,如今已是极高。

没有朝廷安排,主动出城来的官员多达数百人。不仅仅有想在王安石面前混个脸熟的低品小官。连衣着朱紫之辈,也来了许多,不仅仅是几个与王安石关系紧密的官员。一见到王安石抵达,这些官员便蜂拥上前,只是看到内侍带来班直护卫,才不敢有所骚动。

与吕惠卿、章惇、曾孝宽,还有王安上、王旁和韩冈——弟弟、儿子和女婿——一一打过招呼,王安石跨上了内侍牵来的御马,在旗牌官和一部鼓吹的引领下,当先向着东京城而去。

韩冈与王雱并辔而行。今日再见大舅哥,瘦得脱了形的样子让韩冈吓了一跳。不过王雱的精神极好,在马背上左顾右盼,絮絮的与韩冈说着闲话,畅叙离情。

途径一座码头,王雱突然指着从栈桥下来的两条延伸至库房的平行线:“那是轨道?”

韩冈惊异的看了王雱一眼。轨道和有轨马车从提出到实现,总共也不过两个月的时间,韩冈尚未在送去江宁的信中提及此事,王雱怎么就知道了?

透过韩冈脸上的表情,王雱明了韩冈的疑问:“是前日在南京泊船时看到的,去年南下时还没见到,所以就找人来问了一问,没想到竟然又是玉昆你的功劳。”他又笑道,“难道玉昆你不知道汴河上每天有多少艘船北上南下吗?金陵的酒店门前,现在都挂着热气球。还有不少好事之人,四处张罗着要造飞船,上天看一看风景。”

韩冈呵呵笑道:“这不是我的本事,是七十二家正店的功劳。”

东京城中的七十二家正店,不仅仅在东京城中有着莫大的影响力,同时也是天下酒楼的仿效对象。七十二家正店门前扎起彩楼欢门,天下酒楼门前也都少不了用绸缎和竹竿凑个趣。现在既然七十二家正店都开始在门头出放起热气球,甚至用挂下来的条幅为自家打广告,汴河沿岸各城市中的酒楼,当然也不会甘于后人——

“说得也是。”王雱点着头。

新抵京城,王安石便被召入宫中,入内面圣。而王雱虽然是王安石的儿子,但在朝中就都是大宋的臣子,身份不同,地位有别,自然不能一起入宫。向一群相熟的友人告了罪,与请了假的韩冈在宣德门前候着。至于王旁,则是领着吴氏和家人去安置。

刚刚坐定下来,就见到一名内侍,背上帮着长条包裹,带着五六个班直向着城北面的陈桥门过去。王雱认识那一位内侍:“是刘有方……”

“大概是相州之事。前日韩稚圭又上辞表,诏不许。昨日听闻将由淮南节度使迁任永兴节度使,续判相州。升了一级,算是冲喜吧。”

韩冈说得很轻巧。他从来没有见过韩琦,自他任官之后,韩琦这位三朝宰辅、顾命元老,就已经出外,回到相州任官,再也没能重返政事堂。虽然韩琦在朝野之中的影响力极大,给王安石的变法事业平添了无数阻力,但对韩冈来说,这位他在千年之后并没有怎么听说过的前任宰相,也只不过是个并不关己的符号人物罢了。

“韩稚圭快不行了?”王雱的声音中则多了一份说不清道不明的味道。

“应该没多久了。”韩冈说道。

王雱的眼神追着刘有方一路向北。仁宗、英宗之时,韩琦权倾当朝,政令由其所出,逼太后撤帘归政也不过是一句话而已。

仕宦而至将相,富贵而归故乡。

刻在昼锦堂中的这两句话,是多少官员梦寐以求的境界,终身奋斗的目标。只不过一代新人换旧人,现在韩琦已经不行了,是他王雱的父亲王安石取代了韩琦的位置。

“最近朝堂上还有什么事?”王雱随口问道。

“还有?……”韩冈想了想,“还有就是日前王禹玉、吕微仲还有小弟,同荐家师子厚先生入京任官,只是尚没有得到批复。”

“什么!”

王雱脸色大变,双眼瞪了过来。韩冈则是半点不让的与王雱对视着,原本温情脉脉的气氛荡然无存。

王安石去年担任宰相时的府邸在他离任后便被收回,但并没有立刻安排出去,现在回来正好可以继续入住。

王安石复相的消息确定之后,开封府便派了人来打理府邸,屋舍草木都整理了一遍,还开了后花园中水门,将里面的池水也换了一遍。里里外外的打扫得干干净净,住进来之后,省了王家仆婢们不少的麻烦。

王安石一个月来车船劳顿,入城之后直接被召去面圣,回来脸上难掩疲惫之色。晚上,一家人围坐在桌边,王雱故意避开了有关张载的话题,但等到韩冈离去之后,王安石父子三人坐在一起畅叙离情,就免不了要说起推荐张载入京的事。

王旁知道此事,一五一十的跟父兄说了。

“判国子监?!”王安石听了之后又惊又怒:“玉昆怎么就能伸手要这个职位?”

方才韩冈根本没有细说此事,王雱这时候才知道韩冈竟然是荐张载判国子监,眉眼中也尽是怒意:“国子监决不能交给张载!”

“大人复相,没少了玉昆出力,如今连一份荐书都要从中作梗,难道不会被人说忘恩负义?!”王旁很清楚韩冈可是帮了自己免遭牢狱之灾,更明白若没有韩冈用计,自己的父亲也不会这么快入京为相,何况之前他还帮了新党不知多少忙,“张横渠之学,的确与大人相异,但玉昆毕竟是他的弟子,就算不喜其学,怎么也得让玉昆脸面上过得去。”

“二哥儿,不明白就别多说话。”王雱声色俱厉,“那可是判国子监!”

韩冈就算荐他的老子、王安石的亲家入国子监,在王安石和王雱眼里都不是什么大事,农事也算是一门学问。唯有张载不行,这是在刨新党的根基,在抢王学的未来。

整个变法集团是一个完整的机体。有负责立法的司农寺,有负责执行的中书检正公事,有编订变法纲领和理论基础的经义局,还有培养变法后继之人的国子监,以保证新法不至于人亡政息。其中的任何一项,王安石都不可能交到他人手中。

纵使亲如韩冈,只要他还不是王学的门徒,只要他还想着推崇关学,王安石和王雱就不可能让他如愿以偿,将国子监交给他处置。事关毕生的功业,就算要跟韩冈这个女婿反目,王安石都不会让步的。

绝不会!

“要判国子监,少说也要到侍制一级。文选荟萃之地,岂是微官能弹压得住?”王安石冷着脸,找着理由。至少在品阶上,张载要任这个职位也的确很勉强,“张载此前不过是个崇文院校书而已!”

王旁不敢再说,只是脸上写满不服气,这样不是要逼着韩冈离心离德吗?

王安石知道自己这样做很是过分,但他不能让步,对着闹着别扭的次子叹道,“玉昆那里为父会给他一个交代,其他的事都能应允,只是国子监不能让张载去管。”

王雱心头一阵火后,这时则稍稍冷静下来。回想着白天时与韩冈的一番对话,又听到父亲的话语,脑中忽然间一道灵光闪过,急声道:“大人,玉昆对儿子说的时候,只是说他与王珪、吕大防荐张横渠入京任职,并不是判国子监!”

王安石闻言一怔,将询问的视线投向长子,就见到王雱点了点头。得到确认,王安石绷得紧紧的一张脸也放松了一些:“……也算知道分寸。”

“嗯。”王雱点头表示同意。

父子两人这下都明白过来了。

韩冈在王雱当面不提国子监,只说入京任官,其实就是划出了底限。国子监只是张口报出的价码,王安石他们可以落地还钱。但如果连张载入京都不肯答应,那韩冈就当真要翻脸了。

如果韩冈直接要荐张在入朝为官,王安石和王雱心中肯定是很不痛快。而现在韩冈先是荐张载判国子监,到了他们面前则是退了一步,在王安石和王雱的心里感觉就好了不少,至少觉得韩冈并不是在挟恩图报。

王安石想了片刻,终于放弃一般的叹了口气,道:“张载名望已高,也不便阻止,就让他进京来好了,看看哪里能给他安插一个职位。”

“什么样的职司,是清要还是繁剧?”王雱问着。

“若是事务繁剧的差遣,张载不一定会接任,玉昆那里也会平添曲折。”王安石说道,“就在三馆中找一个清闲点的差事,让张载去做好了。想必玉昆也不能再多要求。”

