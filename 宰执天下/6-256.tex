\section{第33章 为日觅月议乾坤(14)}

觐见后的次日,吕惠卿登上了列车,离开了东京城。

对于他的离去,依依不舍者不乏其人,朝野内外,都有大批的人感到惋惜。

稳固的两府,稳定的朝堂,需要资历和人脉才能跻身的议政行列,理所当然会有大批所谓怀才不遇、认为自己升得太慢的人想要改变现状。

不过朝堂上,敢于将心情宣之于口的官员少之又少,只有国子监中,年轻气盛的学生们,才有臧否时事的胆量。

王寀从国子监出来,在附近找了个食肆坐下没半刻钟,就又听见旁边有人说起上京诣阙又匆匆离开的吕惠卿。

这算是什么大事?至于你也说我也说,说了一遍又一遍?

王寀觉得这些人真是闲得无聊,有空去赛马场和球场,要么就去甜水巷,或是各大瓦子,看百戏,看杂剧,或是逛街,从大相国寺万姓交易的集市,到日出即收市的鬼市子,打发时间的去处,京城中实在多的是。

但话还是往耳朵里面钻。

“就这样结束了?”

“太后好恶分明,吕宣徽也是有心无力。”

王寀撇了撇嘴,所谓好恶分明,就是在说向太后偏听偏信。

但他们也不想想,太后对章惇、韩冈、张璪等人信之不疑,完全是因为几人都是立有殊勋,是定策勋臣。

而吕惠卿,先帝发病之夜,他不在,戾王宫变之夜,他同样不在。身无尺寸之功,太后怎么可能信任他?

“朝廷会怎么处置吕宣徽?”

“还能怎么样?罚俸而已,照旧外任。宰辅就有宰辅的待遇。”

王寀有些烦躁的拿筷子戳着盘子里的木樨饭,实在让人没胃口吃。饭粒太软,鸡蛋太硬,葱花糊了,用的还是发黑的粗盐,吃起来有股子苦味,这样的厨师死后应该下油锅地狱,这样才能让他知道什么叫做火候。

这样的食肆究竟是怎么维持下来的?王寀真的很纳闷。而且旁边还这么吵。

像吕惠卿这样当朝哭出声来的宰辅,最多也不过罚个俸而已。心念天子,感怀先帝,难道还能说他这位忠臣不是?所以说吕惠卿奸猾。就是奸猾在这个地方。

“常言说君臣犹父子,子为亡父哭,越是动情越是合乎礼法。行止皆合礼节却一个劲的干嚎,怎么比得上真心诚意的痛哭一场。诚心正意四个字,在气学中与格物致知同样看重,御史台要惩治吕惠卿,韩相公可能厚着脸皮点头?就是他拦下来的。”

不拦下来又怎么样?平白给吕惠卿增添声名。

听得厌了,王寀刷刷的划着筷子,几口将难以下咽的午饭弄进肚子里,会过钞便出了店,打定主意下一次再也不来。

正在高谈阔论的几人,听到动静,回头看了一眼。发觉是个十几岁的小孩子,看着就是刚入学的学生,便不放在心上,回过去继续高谈阔论。

正午的街上依然喧闹,靠近南薰门的地方,如今从早到晚就没有安静的时候。本来就因为国子监位于此处而人声嘈杂,现在又多了往来车站的人流,就更加吵闹。闹得都有人在朝堂上提议国子监迁址,在外廓城换一个僻静的地方。

听起来是个好主意。外城和外廓城的房价差了两倍还多,房租的差距也差不离。真要搬到外廓城的话,在外租房的学生每个月还能省下一笔钱。

可惜王寀还知道一件事,如果国子监当真外迁的话,空下来的地皮将会改建一批提供给官吏居住的屋舍。

不仅仅是国子监,包括将作监在内——只除了军器监——绝大部分官作工坊都将会迁出新城旧城,进驻外廓城。由此置换下来的地皮可以兴修大批屋舍,无论是居住还是作为商铺出租,都是一笔好买卖。

京师的睦亲宅已经住了太多宗室,原本好几处名园,因为分家的缘故,被划分给兄弟几人,好端端的竹林、梅林,被一道道围墙所替代,京城之中没少了焚琴煮鹤之讥。宗室们早就盼着朝廷能新修一批住宅了。而京师的大小官员,因为朝廷提供的房屋不够分配,有很大一部分不得不在外租房居住,这一批人也同样盼着朝廷能够提供更多的官宅。

但工役之事,兴师动众,一向是能省则省。现在虽有意向,但到底何时能够实行,王寀也不清楚。反正这件事不易办,尽管他是在宰相府中得知此事,可王寀还是觉得即使有宰相推动,想要在京师中兴作工役,也是得旷日持久。不过,终究还是会办成。王寀倒是很确信这一点。

眼前一片熙熙攘攘的场面,或许等到几年后,就会稍稍清静一点了。而国子监搬到外廓城后,起居的环境也当会比现在更加适宜读书。

也许自己不一定能享受到新校舍,再有几年,自己早一步考上进士也说不定。

王寀憧憬着。

今日午后没课,但王寀又不想在街市上闲逛,想着是不是回去睡个午觉,然后再看会儿书,把功课做了。

再几日就是月考了,王寀虽不指望能初进国子监,就从外舍升上内舍,但两千名外舍生中,他也不甘心位居后列,总要往前百名中争上一争,积累几次高名,再在三次大考中保持成绩,明年进入内舍就不是难事了。

正在街上犹豫的时候,就听见背后有人在喊,“十三叔?”

王寀排行十三,但在京城中,称呼他王十三的不少,称呼他十三郎、十三哥的,回家就能听到,人数也不少。可称呼他十三叔的,可就寥寥数人。

王寀回头,看清来人就笑了起来,“哦,是钲哥啊。”

韩钲带着四名伴当,正穿过人群过来。脚步快中见稳,不徐不疾,把士人应有的仪态表现得淋漓尽致。

王寀听说韩钲小时候被他父亲放着养,心都给玩野了。稍长一点,没少被他娘亲责打,完全是靠了棍棒才把风仪练出来。

看见韩钲这副模样,王寀就忍不住想笑,待韩钲走近,他就抿了抿嘴,“钲哥,怎么走到这边来了?是出来置办行装的?”

韩钲再过半月就要出发西去,去横渠书院读书。王寀上一回去韩家就听说了,而且韩家的子弟日后都要去横渠书院,那是气学的根基所在。韩冈作为气学宗师,总不能连他的儿子都放弃横渠书院,而去国子监读书。王寀他的侄儿也同样在横渠书院读书,也不知是不是为了讨好他的岳父。

而王寀自幼丧父,在江西乡里侍候在母亲身边,并没有去横渠书院,年纪到了之后,又顺理成章的来到国子监中读书。但从学派上来看,王寀自觉更倾向气学,不过那些从横渠书院流传过来的数学题,尤其是一干奥数题,他当真做不出来。正如其名,奥数,实在是太深奥了。这让王寀也不好意思说自己是气学一门。

不过跟韩家的关系,还是一样的亲近,并没有因时间而疏远。

“置办什么?早就准备好了,都不用我动手。”

韩钲悻悻然的口气,让王寀了然。就是他自己,也不喜欢什么事都被父母打点好,自己只管坐着等。

“那钲哥你今天出来是做什么的?”王寀问道。

“就是来找十三叔你。”

两人的年岁相差不大,但从王韶、王厚与韩冈的关系顺下来,王寀的辈份理所当然的要长上一辈。

“三丈找我?是有何事?”

韩钲摇摇头,“不是大人,是娘要找十三叔你。”

“三嫂?”王寀难得吃了一惊,“三嫂找我何事?”

“前几日,祖母知道十三叔上京读书了,特地多送了一份特产来,娘本来是想让人送来的,后来一想,正好十三叔好些日子没登门了,就让小侄来请十三叔你。不知十三叔今天有空没有?”

也不等王寀考虑,韩钲就上前挽住王寀的手,笑道,“今天有空就去家里。前几日,大人还提起十三叔呢?”

“三丈怎么说?”王寀稍稍有些紧张。

“上次十三叔来家里,大人是怎么说的?‘进京上学半年了,除了一开始住了一阵,之后就登了两次门,这是把家里当外人看了?’记得十三叔当时说好之后会常来,这一个月过去了,也没见十三叔你上门,不知十三叔的常,是哪个常?。”

王寀苦笑起来,“这不是学业忙嘛。”

“再怎么忙,一天的空都抽不出来?”韩钲摇头,“十三叔你也别跟侄儿解释,等回去见了大人,你跟大人说。”

被韩钲强拉着脱不开手,王寀很是无奈的被一路强拉到了宰相府上。

两家是通家之好,韩冈与王厚更是情同手足,又约为婚姻,这让王寀根本不知该如何拒绝韩家的热情。

不过到了家中,韩钲拉着王寀来到韩冈的外书房前,守在门口的元随拦住了两人。

“二郎,十三郎,还请稍等一下,相公正在见客。”

“谁来了?”韩钲扬眉问道。

这个时候并非他父亲见外客的时间。休沐之日的午后,韩冈不是看书,就是写书,或是审核论文,除非有急事,否则根本就不会见客。

“是王家二舅来了。”

“哦。”韩钲回头冲王寀笑了笑,笑容中多了几分这个年纪不该有的苦涩,“看来亲事终于是定下来了。
