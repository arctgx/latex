\section{第21章 欲寻佳木归圣众(六)}

将铁路系统归于一个衙门来管理。

这件事韩冈已经考虑很久了,只是之前还不能拿上台面。

轨道如今还不长,但日后少不了上万里,之前建得鸡零狗碎,东一条、西一条,没必要多费心思。但如今京泗铁路贯通,京洛铁路也通车在即,正常来说也该出台一个管理办法来。总不能再随便丢给转运司,发运司,甚至经过的当地州县。政出多头,必然是有难事时,相互推诿,有功劳时,相互竞争。全部归于一个衙门,最好管理。

另一个,也是最重要的,如果像现在一样,几个衙门各管一摊,铁路内部的财税利益就不好分配。别的事小,这件事当真是重,与其等到日后人人都想插上一脚,将铁路弄得乌烟瘴气,还不如先行将制度给确定下来,免得他人伸手。

让沈括以参知政事的身份来主持铁路,也正是遵循了这个心思。偌大的一个衙门,只有宰辅一级才有资格控制住。

韩冈在宗泽走后,想了一阵这件事,忽而笑了起来。

专掌某一司职的宰辅,说起来也跟后世一样,官僚制度这东西,不论职位变得如何,本质上还是差不多的。

只是类同于铁路警察的军队,至少三四十个指挥,人数如此之众,终究还是有些犯忌讳。可若是政出多门,或是兵力不足,对于铁路的安全保护就是个灾难。

铁轨,除非是自家造反,拿去打造兵甲,只要是为了赚钱而偷窃,没有哪家的铁匠敢收购这等要命,倒是枕木,破坏轨道的罪行中,还是以此事最多。但找理由的时候,还是以钢轨为由更好一些。韩冈微又自得的想着,其实以他的权势,想要将此事通过,也不过一个说得过去的借口。

有宗泽来提出此事,就不用多想了,韩冈很干脆的放下了这件事。等到此事解决,这位状元郎,也该放出去历练一下了。

看看时间将近散衙,韩冈无心再去批阅永远解决不完的公文,唤人上来将凉汤换了,又考虑起铁路的事了。

不论自己将铁路总理衙门的架子搭得多好,没有足够数量的铁路,便依然是个笑话。

已经成型的京泗、并代,即将完工的京洛,加起来也撑不起一个宰辅手中的应有的权力,也容不下数万守卫大军。

现在韩冈已经在规划京洛铁路向东西延伸的计划。出洛阳、过潼关后的铁路轨道,经过长安,一直到凤翔府的宝鸡,都还算好铺设,但再向西去,难度就大了不少。

韩冈很想修一条自海东密州到河西兰州,再到河西走廊西部玉门关的铁路,一条横贯中国腹地东西的大动脉。这个愿望,近期做不到,但二三十年后,韩冈相信自己能够有很大机会看得到。

至于自玉门关至伊州,仅仅是星星峡那一段,韩冈就不抱希望了。穿越河西走廊的铁路轨道最终也只能停在玉门关处,将甘凉路的军事防区连成一线,成为中国本土的西大门,同时也能尽可能快的支持西域的同胞,便已经是完成所有的任务了。

兰州西去,直至将北庭、安西两大都护府都纳入铁路的运输之内,韩冈就不指望自己有生之年能看到这条铁路能够修成。

进入西域的第一关口星星峡就不说了,哈密附近的大风也是,后世的新闻上时常能听见百里风区的这个名词,如今在西域道上也同样有名,哈密附近最大的一间驿站,便名为避风驿。千年后的铁路车厢都能给吹翻,如今要是修了铁路轨道,保不准连路基也给吹翻。

在那里修建铁路的成本也太高,暂时只能用大规模的车队来运输。增加当地汉人人口,屯垦西域的工作一直在进行中,关西百姓有灵武之地可以移民,从绝对人数上并不稀少,而河北、河东、甚至京畿的百姓,想去代州也只要一句话,这两年,当初战乱造成的缺口也几乎快要填补起来了,甚至原本属于辽地的神武军,也有了上千人户。

只是想让人去西域就难了。如今但凡作奸犯科,只要过了杖责,不到十恶,全都是发配,靠南方的去岭南,北方的去西域,只是这样还是远远不够。想要将数百年的历史缩短到区区十载,这不是单纯依靠努力就能做到的。

操心的事实在太多啊。

目标,现状,各色事务交织在一起,便变成了让韩冈也不得不望而生畏的繁重工作。

纵是独相,下面也还有参知政事来分担事务,没有说公廨里就只有一个人来。苏颂年纪大了,懒怠理事,韩冈可是独力支撑朝政很有一段时间了。

几日后的廷推,不光是为了争权夺利,是政事堂真得进人了,只要是想做事的,他绝对的欢迎。

单纯做一个宰相对韩冈来说并非难事,但要实现自己的目标,又怎么不去操心?只是相较于总是重复再重复的公事,还是自家私活更有意义一点。

放衙的鼓声传进耳中,韩冈迫不及待的起身,尽管回府之后还要操心公事,但总比在衙署中松快许多。而且还能更多时间做自己的事。

走出门来,只觉得空气都舒适了几分。

韩冈心道,再这样下去,自己也得要变得怠政了,见到公文就头疼,可不是要变成苏颂一般了?嗯……还有太后。

……………………

李格非刚刚进门,就收到了太后派王中正出京去体量轨道工役的消息。

打发了报信的小吏,一一向同僚打着招呼,李格非往自己的公厅走过去的时候,头脑之中一如狂风般急速旋转。

赶在廷推之前,派人去查沈括的底,太后是不是在暗示什么?

不用说,这一消息传开,整个朝堂肯定都要轰动了。

沈括这是要在两府的大门前输上几次才甘休?

难怪方才在路上看见沈括过去的时候,他的脸色那么难看。

也不知道韩冈会怎么做?

硬顶着太后,继续推荐沈括?还是再一次承认现实?

想是这般想,李格非也只存了一点看热闹的心思,无论如何他不会去蹚浑水的。

不过他能够确定,台中绝对会有人趁机上书,攻劾沈括、甚至他背后的韩冈——这世上,总是不会缺乏想要希合上意的‘聪明人’。

方才他一路走过来,已经感觉到台中的气候不一样了。那种隐藏在阴暗下的浮躁,隐藏在每句话的中的兴奋,隐藏在一举一动之中的激动,都在说,机会到了。

有人心思活泛,也有人老成持重,但御史台中没有人不对这几年太过平静的朝堂腻烦透顶。他们是御史,如惊雷般亮相于朝堂的精彩,才是属于他们的天地。可明明两相对立,却始终维系着和平局面的两府,像一重压到头顶的山峦,不给人任何透一透气的机会。没人不想打破这个局面,太多前辈的成功,在诱惑着他们。

李格非也不例外。

可他们也不想想,要是韩冈连这点风浪都撑不住的话,还能够坐在现在位置上吗?

与其想从韩冈身上的捞声望,还不如多揣摩一下太后与官家的关系。

太后会不会在天子大婚之后还政,这件事让李格非踯躅许久,虽说还有几年的时间,可又不是七老八十只待致仕的耆老,正当壮年的李格非怎么会没有向上继续走的心思?现在不想想几年后的事,日后又怎么抓住那一闪即逝的机会,做出合格的应对?

前几日,从相州来的那人对自己说的话,李格非依然记忆犹新,每每想起,心肝依然要颤上几下。

“太后与章献不同。”

低沉而压抑的声音,透着凌厉的寒意。

章献明肃皇后权欲很重,而且在真宗晚年开始,便帮着真宗处理朝政,就跟武后当年辅助眼疾的唐高宗一般。历来穿着天子服去太庙的女子,除了武后,就只有章献明肃。相比起武后来,也只差了一个皇帝的名号。

而当今太后刚刚垂帘,直到宫变之后的一段时间,她还是很勤政的。兢兢业业,一日二日万几。但自从气学一脉掌握政事堂,与新党对掌文武大政之后的几年间,天下太平无事,人口日渐增多,财计平稳上涨。边州无军情,国中无变乱,朝中有贤相主持,地方又多忠勤王事,莫说是太后这女流之辈,便是如仁宗那样的贤君,也免不了开始怠政。

这一年多来,太后一直疏怠政事,早朝也变成了五日一登朝,基本上就是朱笔批个准字,如果是直送御前的章疏,也肯定直接转给两府。当初韩冈劝太后好歹多看一看奏章,过后没几天,太后就把李南公做三司使给否了,韩冈之后照旧还劝,却也没有之前那般苦口婆心了。也多亏了西南战事爆发,太后这才又重新开始认真的去看奏章。

在这个节骨眼上,太后偏偏去要去跟韩相公过不去,谁知道是怎么一回事?

说不定就是要让韩相公难堪,才使得沈括遭了池鱼之殃。

太后会不会最终收回自己的决定,或是再设法弥补韩冈,或是干脆与韩冈翻脸,李格非不知道,但他知道,现在还不是时候,

远远不是!
