\section{第35章 重峦千障望余雪(六)}

秦鸿是新近从太医局调来熙河路的医官,也是眼下陇西城中手段最为高明的医生。不过他在韩冈面前,绝不敢摆着京城名医的谱。韩冈的名字,在太医局中是跟孙思邈是挂上勾的,而且他主持的疗养院更是得到天子的称赞。

以韩冈在医界中地位,日后说不定就能兼管太医局,秦鸿哪能不小心侍候着。韩冈让他去疗养院治疗伤病,他就治疗伤病,韩冈让他编写一些军中合用的药方,他就跟那些只会做针线活的村医,交流医术心得。

今天被传到韩府上时,秦鸿也是诚惶诚恐。两名等他把脉问诊的绝色佳人,也是不敢多看半眼。

坐上交椅调匀呼吸,将三根手指搭上纤细的手腕。指尖上的触感一片腻滑,秦鸿却不敢有半分邪念。

闭着眼睛感受着脉搏跳动,半晌之后,他站起身,向着韩冈和韩父韩母拱手行礼,“恭喜机宜,恭喜老官人、老太君,两位娘子的确都是喜脉!”

“当真?!”韩阿李喜不自禁,但仍不放心的追问了一句。

说起医术,熙河路最高的其实并不是这个医官,而是僧人智缘。只是智缘现在跟着王韶去了熙州,韩冈也只能将秦鸿请来。

可秦鸿虽不比那些御医,甚至不比智缘,但喜脉是怎么也不会诊错的,他点头打着保票,“千真万确。”

韩冈封了一封丰厚谢礼,让下人将秦鸿送了出去。

回过头来,两女都含羞带怯,手抚着小腹,绽开幸福的笑容。只要有了孩子,她们的一生便安稳了,腹中还未成形的小小生命,关系到她们一声的幸福。

严素心、周南同时有孕,韩千六和韩阿李连声说着要到附近的寺庙中烧香还愿。自从老大成亲开始,两人盼了多少年了,到了今天,终于等到了喜信。而以韩冈一贯的冷然自若,竟也有些难以遏制的欣喜难耐。

“日后都要小心着了,不能累着。”围着素心、周南嘘寒问暖,韩冈只感觉着有些手忙脚乱,不知该做什么好。

韩冈两名妾室怀孕的消息,很快就在陇西城中传开了。听说了韩冈家中有喜,熙河东路巡检傅勍,就第一个带了礼物上门来恭喜。而后,苗授、赵隆、王惟新等熙河路中的将校官吏一个个都亲自上门,几乎踏破了韩家门槛。更下面的士绅商人不够资格上门,但也送了礼来。

韩冈还没正式成亲,就这么快有了子嗣,众人在恭喜之余,也是招来了一些议论。说韩冈早过弱冠之年,又晋了朝官,也该成婚娶妻,好有人来主持中馈。

韩千六的官职不可能再升到哪里,日后也是做封翁的份。韩冈的前途至少在现在看来一片光明,但联姻一名朝官,和找一个新进士做女婿并不相同,熙河路有资格开口的,却没有几人。

腊月廿三,送过灶神,年节也算是到了。该来贺喜的也都来过了,上门送礼的人也便稀少了许多。

韩冈在衙门中打理着今年最后的公务,前两天,衙门就已经封印了,直到一个月后,才会开印。长达一个月的休假,并不代表没有公事。只是需要盖上州中大印的要事不再处理,至于一干庶务,衙中官吏,也免不了要辛苦一番。

按理说,这些事都不是该通判管辖。通判是知州的副手,副署公文,监察州中公事。但现在王韶和高遵裕都被积雪堵在鸟鼠山对面,巩州与熙州的交通线还是没能打通,韩冈也只能先一个人挑起州中的政事——另外还有熙河经略司的,他这个机宜文字也是一堆事要他处置。

同样因为大雪封山的关系,巩州与东面秦州的交通也中断了。虽说也不是不能联络,但前日派了驿马出去,到现在也不清楚到底到底有没有抵达秦州。幸好秦州那边几乎在同时派出了信使,已经到了韩冈的面前。

韩冈手脚麻利的处理好了所有的公事,正打算回家,匠作营遣人来报,前日韩冈让他们制作的雪橇车现在打造好了,等着韩冈去验收。

听到此事,韩冈就在想着,是不是送点酒水去熙州,也正好可以展示他的力学原理是如何用于实际。

……………………

王厚是第二次诣阙了,但他进宫面圣却不止两次。就是刚到京城的第二天,天子就召见了他,而今天,大内又传话出来,把王厚叫进了宫中。

想想韩冈都成了正八品的太子中允,正儿八经能上殿参加朝会的朝官,竟然都没有见过天子一次,王厚便觉得,世事每每出人意表,当真是难以预料。

前些天,王厚抵达京城的时候,正值韩冈被推到了风尖浪口之上。王厚在驿馆中听到的,多少人都在议论韩冈。

熙河路的官员升官实在太快了。王韶是正牌子进士,高遵裕是太后的叔叔,可能是因为他们两人都没有多少攻击的余地,所以入官才两年就升任朝官的韩冈,便成了众矢之的。

年纪姑且不论,入官两载,便能上殿参加朝会。也只有开国之初,才会有这样的例子。即便是三十五岁就进政事堂的韩琦,他升任朝官的速度,也决没有韩冈这般迅快。

进用如此之速,嫉妒韩冈的人自然绝不会少。

他们不会去提韩冈立下的功劳,将他的历历功绩放在一边,说韩冈是党附权臣的一个幸进之辈。幸好韩冈没有入官面圣过,否则阿谀天子的罪名少不了。

倒是刚刚做了崇政殿说书的王家大衙内为人仗义,前日在樊楼赴宴的时候,明明白白对外面说,只要有哪个选人敢自称有韩冈一半的功劳,他当即回家向王相公推荐,荐他入朝为官。

诽谤韩冈的谣言就这么消失了,而他立下的累累功绩也开始在京中传递。

韩冈跟王家二衙内有些交情,这是王厚知道的。而王家大衙内,一向心高气傲,又是跟文彦博、司马光一般的早慧,能出头帮韩冈说话,当真是难得。想来多半是得了王安石的授意。

韩冈升为朝官,而王厚并没有转官。但他的本官也是一升再升,进用之速,也算是少有了。不过王厚并不打算继续作文官,准备着转成武资。做文官虽然安稳,但王厚有足够的自知之明,他在文事上没有多少前途。父亲王韶的才学他连一半都没学到,而韩冈在经义大道的见识,王厚也只有仰头观望的份。

如果考不上进士,又想在官场上高歌猛进,算起来还是转为武官的好。河湟周围,还有许多地方可以去开拓。王韶立威于此,自是能遗泽后世,日后当也有他王厚立功的机会。

一阵寒风吹来,王厚冻得瑟瑟发抖。不比他前次进京,夏天在崇政殿外候着,只是热上一点,而且还有穿堂风。但冬天守在殿外,却是冷得够呛。如果是朝臣,尚有资格在暖和的偏阁等候传唤,但他这样的外臣,还是老老实实的站在殿外阶下。表现得恭谨一些,只会有好处不会有坏处。

不知等了多久,崇政殿的大门终于打开,一众宰辅鱼贯而出。王厚连忙躬身退到一边,见着一只只脚从面前过去。

人流走尽,殿中又过了半个多时辰才有人出来,将王厚叫了进去。

崇政殿中,除了天子赵顼,下面还有一名大臣坐在绣墩上。身穿紫袍,腰缠御仙花带,面皮如黑炭一般——自然是如今的宰相王安石。

面圣,王厚早有多次经验。行礼叩拜,一点也不慌乱。

起身之后,王厚就听赵顼在问:“韩冈在疗养院中私酿酒水,不知王厚你知不知道?”

王厚一下愣住,这是谁传到天子的耳朵里的?!不敢偷看天子的脸色,他低头为韩冈辩解:“陛下有问,微臣不敢隐瞒。韩冈主持的疗养院的确是造了酒,但已得了家严的同意。且疗养院所酿之酒并不是给人喝的,而是用来清洗伤口。因为前次有几个好酒的将校偷了酒喝,韩冈还大发雷霆,说是烈酒阳气太重,可以用来驱除会让伤口溃烂的阴毒之气,喝了却会伤身。只能外用,不宜内服。”

韩冈这番话是用来吓唬王舜臣、傅勍那一干酒鬼的,王厚也知道这是胡扯,但拿来解释韩冈并没有私卖酒水的心思,王厚觉得更为合适。

“原来如此。”赵顼算是释然了。秦凤转运司传来的密奏让他看了很不痛快,他并不希望他所看好的臣子,会是个贪鄙的小人。王厚的解释,赵顼听着,觉得不会是临时编出来的,当不至于有假。

“韩冈一直都说他跟孙思邈没有关系,但这医理却是让人叹服……还记得他论跌打损伤的治疗,得用柳木做夹板,外敷石膏泥,水、土、木皆备,才能让骨头长得好。这一个方子传回京中,太医局里人人皆叹。”

王厚都没想到天子连这些事都知道,连忙道:“韩冈虽然不通医术,但医理的确让人佩服。”

“听说王厚你与韩冈情谊匪浅?”赵顼突然问着。

“……是。”

“那他遇仙之事究竟是真是假?”赵顼问得饶有兴致,就算是天子,也是有着一颗八卦的心。

“韩冈一直都是说,当初遇到的只是一个姓孙的道士。还说怪力乱神,君子宜远避之。”

“儒门弟子当不语怪力乱神。”王安石很欣赏韩冈的态度,就是真的遇仙又如何?如果韩冈总是把神怪之事挂在嘴边,日后对他的前途决没有好影响。

