\section{第29章 浮生迫岁期行旅(四)}

夜色已深,韩冈的书房中灯火仍明,王旖正拿着家里的账本给韩冈审核。

年终关账是定例,韩冈记得他曾经开玩笑说把关账的时间改在年节后,省得年节前一堆事挤在一起,但终究还是抵不过习惯。

王旖等几个妻妾辛苦,韩冈是甩手掌柜,家里的内账从来不掺合,听过结果就行了。王旖她们辛辛苦苦做好的账本,韩冈瞥了两眼就丢到了一边。

每到这个时候,王旖看到悠闲自在的丈夫,气便不打一处来。而且这两天正斗着气,白天韩冈去送了王安礼,又陪着小心,心情稍稍好转。可这一下,心情又坏了起来。

“这是何矩送来的信,是义哥的。”韩冈适时递上一封信。他知道,王旖喜欢自己跟她商量家里的事情,而不是一直被瞒在鼓里。夫妻多年,想让妻子心情变好,韩冈还是清楚怎么做。

王旖接过信,看了两眼就知道是在说玻璃工坊。有关此事,韩冈跟她说过不少。乍看是不以为意,但几行之后,一个巨大的数字让她猛吃一惊。

“论钱二十有七……万万!”王旖震惊于这个数字。

“说贯。说钱听起来像是朝廷给赏钱呢。”韩冈更正王旖的说法,“义哥是故意这么写,糊弄人用的。”

朝廷赐钱,总是习惯往大里说。一百贯、两百贯,‘百’字头打转,听着就可怜。十万钱、二十万钱,一用上‘万’字,那就是一个完全不同的境界了。

当初太子赵佣种痘成功,天子赐钱三十万。因为给太子种痘的缘故,韩冈得到了三百贯的赏钱。两种说法,自然是前一种更能体现朝廷的慷慨。

国初时,曹彬领军灭南唐,太祖皇帝赐钱二十万。曹太尉攻下南唐的赏钱就这么多,这是用使相【节度使兼宰相】的职位换来的。听起来有个万字,还能过得去。如果换成二百贯,朝廷待人苛刻可就昭彰于世了。

这可能是汉唐时遗留下来的习惯,总是一钱两钱的来算。但在如今的大宋,商业发达,远胜汉唐。民间的商业往来时,都是用多少千,多少缗,多少贯来计算,从来不用一钱两钱的单位。

不过话说回来,不论是二十有七万万钱,还是两百七十万贯,都是很惊人的数字。

仁宗皇帝大行后,朝廷给出的赏赐比这个数目要多一点。英宗在位只有三年多,因为收支一直都是赤字,国库空虚的缘故,大行后赐予臣下的钱绢,不到百万贯。当今天子对国库空虚的第一印象便来自于此,由此耿耿,一心变法又岂是无因?

“怎么会这么多?!”

“想想天下有多少人。”

眼下冯从义在信中说的两百七十万贯,是他估算的玻璃工业一年的收益。由此推证,工坊和配方的价值更在其数倍之上。

不过工坊的价值这个时代没有确定的评价标准,尤其是韩冈总是喜欢玩技术扩散,到了现在,甚至吸引了不少豪商,一起出钱研究更新的工艺,然后一同分享成果。就如透明玻璃的制造,要不是有二十多家雍秦豪商出人出力,很难这么快就利用巩州的当地原材料,找到能派得上用场的工业配方。

玻璃制品一旦普及开来,冯从义估算为一年最高能有两百七十万贯的收益,只不过这属于所有的参与者,同时还会有越来越多的人参与进来分账。

而掌握在韩冈手中另一数字则是五十万贯,是冯从义估算的最低值,却没有对其他商家公布。说起来这样的估算很不靠谱,几乎是拍脑门出来的,在冯从义那边的理解,让他估算出来一个二百七十万贯,是韩冈打算用来诱人入彀的手段。

“两百七十万贯……”虽说韩家的家底丰厚,韩冈过去也曾自言有形无形的资产价值千万,但再看看一年两百七十万贯这个数字,王旖还是吃惊非小,又问了一遍,“真的有这么多?”

“普惠天下,当然不是一蹴而就的。若是有个二三十年,甚至更长时间的发展,还差不多。这可是一个不下铁器的大产业。”韩冈笑道。

“官人就不想将两百七十万贯尽收宦囊?”

“笑话。吃不完用不完存在地里吗?为夫希望看到的是一门兴旺的产业啊……我没有岳父敢与天下士大夫为敌的胆魄,但如何解决问题的想法还是有一些。”韩冈很佩服王安石的品德和胸怀,更加佩服他的强硬,但韩冈自知学不来,他的倔和王安石的倔,完全是两回事,“新的土地,新的产业,更加畅通的水陆交通,这些都是能够给朝廷带来丰厚收入的手段,民不加赋,而国用自足。只是要有耐心,十几二十年的去培养和等待。”

这是以前便陆陆续续跟王旖提起过。她点点头,丈夫胸怀天下,这是最让王旖自豪的地方。

“但现在说还是空话,事情要一步步的去做。”韩冈的想法很多,但都需要大量的时间。

再比如雍秦商会,内部的信用借贷很早就有了,京城和陕西几个要郡的飞钱业务则在筹备中。尽管还是苗头,但正规化组织化的金融财团也就在这两年将会有一个初步的雏形,这是现实的需要。

冯从义已经写信来商量过好几次,韩冈倒是让他慎重再慎重,一步步走得稳一点。

说实话,通过质库、放贷得钱太容易了,对普通百姓的借贷要与官方的便民贷竞争,很是麻烦,而对商业伙伴的借贷,则没有任何阻碍。只要对借贷后的伙伴,分享一部分雍秦商会手中的信息和交流资源,坏账的几率将会很小。而飞钱,更是纸币的雏形,铸币税这个进项都能轻松超过几十万贯。韩冈只担心冯从义和其他有份参与豪商看到钱来得太容易,便把实业抛到了脑后去。

“治国平天下,为天下开太平,都不是空口说白话。当年天子问政,司马君实说是‘修心之要三:曰仁,曰明,曰武;治国之要三:曰官人,曰信赏,曰必罚。’但根本就没有实际施行的条贯,光喊空话谁不会?有用吗?”韩冈摇头,“哪比得了岳父,是真正愿意不惜声名去做事的。天子任用岳父,而将司马十二放到陕西,最后甚至安排在了洛阳,岂是无因?实在不能做实事。”

王旖微笑了起来,她能感受得到丈夫对父亲的敬佩是真心实意的。虽然道统不一,但依然是争之以公,而不是指斥对方人品那样下三滥的攻击。

看到妻子心情转好,韩冈也安心了下来。家宅不宁,可是让人头疼。

只是说给妻子听的,还是藏着掖着许多。韩冈现在已经清醒的认识到,他所主张的工商业再继续发展下去,必然会受到习惯与文化的拖累——无论是交州的种植园,还是熙河的工坊、棉田,对人口的需要几乎无穷无尽,但这个时代的文化还远远跟不上韩冈想要达到发展速度。

这段时间以来,韩冈一直都在思考,随着地位越来越高,看待问题的角度也就越来越宽广。必须要有一个纲领,或是说理论,来合乎情理的改变如今的意识形态,支持国家走上以工业扩张和发展的道路。虽然还不是很急迫,但这是一个无法逃避的问题。

话说回来,这些也不是当务之急。

当务之急是辽人。韩冈身上的差事逃不掉。

辽国的正旦使萧禧再过几日就要进京了,现在估计已经到了大名府。以耶律乙辛的老辣,或者说老奸巨猾,他不会糊涂到以为只用一个萧禧就能敲到多少好处,尤其是他派出萧禧前,还不知道大宋天子成了废人。一个理所当然的推断,就是边境上肯定会有动作。

不过亲自赤膊上阵的事,耶律乙辛多半还不会做,他应该还没有做好让宋辽两国陷入战争的准备,驱动附庸或是代理人的可能性更大一点——之前他就玩过这一手,现在自然可以继续这么做。

所以不是河东,就是陕西。

要么是在胜州的黑山党项,要么就是青铜峡的那一批余孽。

韩冈对此还没有跟章敦商议过,前些日子各种事忙得厉害,一时忘了。而且当也不需要商议,章敦不会想不到。即便想不到也会有人提醒他,刘仲武在鸣沙城不是没有来由的,章家的门客中也颇有几个了解西事、北事的行家。

在河东那边,韩冈有折家做耳目,陕西更是自留地,比只有刘仲武的章敦要强得多,至于河北,还有在定州的李信。

他准备明天向皇后和西府申请,查阅主管外交往来的枢密院礼房积存的情报。

再过几日,顺丰行那边应该会有消息传来,配合枢密院礼房的情报,至少对眼下的边疆形势能有一个完整的认识。比如如辽国在边境的动向等等,在与萧禧谈判时,是必须掌握的情报。

只是想到这里,韩冈突然间就有些担心了。

皇后准备好了没有?

让自己做馆伴使的理由是可笑的防止太子被契丹人煞气冲撞——说起来要是自家接了枢密副使的差事,馆伴使就做不得了,没有执政去陪客的道理——对边境上可能会有的冲突,则提都没提过。

要知道,辽人甚至会因为探知天子中风,而立刻大动干戈。

这样的准备,皇后做好了没有?!

……………………

皇后此时正在灯下苦恼。

擢韩冈为枢密副使的第五封诰敇已经写好了,但白麻诏书拿在手中,向皇后却是心烦如麻。

她知道,韩冈肯定还是不会接诏。

那接下去该怎么办?

总不能学王安石当年逼韩冈去横山那样吧,愿意做得做,不愿做还是得做。堂堂枢密副使,千万文武官求也求不来的清凉伞,能强逼着人接下来吗?这可是国家名.器啊,丢脸不是丢到家了

倒是王珪和吕公著的辞章,已经慰留了三次,差不多就可以应允了。她已经用朱笔批了两个可,下面就可以发下去,让翰林学士起草两人的新去处。

自家的丈夫到底是不是有心让韩冈做枢密副使?以韩冈的身份,做天子蒙师,侍讲资善堂绰绰有余,也是让人安心。偏偏要加上两个同僚,一个是岳父,一个是师长,这让韩冈怎么想?

给个枢密副使算是补偿吗?

想想人家的脾气。跟王安石一模一样。这样的手段能乱用吗?

前两天,一听调曾布回来做参知政事,拗相公直接就找上门来了,在御榻边气得黑脸变白脸,她在旁边看得都心惊。

最后肯答应下来,那是念在旧日情分上,看到丈夫现在模样,心里难受,不想再争下去了。王安石的心态变化,向皇后在旁边看得最是分明。

这是何苦呢?

向皇后想着,将心比心,难道就不能让王相公、韩学士这样的忠臣尽心尽力吗?

放下韩冈的诰敇,向皇后又拿起放在最上面的一份奏章,脸色陡然变了:“枢密院是怎么回事?!这件事怎么不早报上来!”

