\section{第40章 中原神京覆九州(下)}

【第三更,求红票,收藏】

出嫁的队伍走过眼前,韩冈看着心底纳闷。

但凡富贵人家嫁女送嫁妆,一溜三十六个大箱子在街上游走一圈,炫耀一下,也是此时习俗。但他看着箱子都是晃晃悠悠,扛着箱子的汉子也都是一脸轻松,很明显全是空的。郡公嫁女,好歹一个县主,这嫁妆怎么这么寒酸?

这北宋的婚嫁习惯,跟后世的中国不同,也可说跟后世的印度相似,基本上都是女方贴钱,男方的聘礼远远不如嫁妆丰厚。稍稍有点家产的人家,都不敢亏待女儿,怕嫁过去吃亏,嫁妆给得如流水。

还在秦州的时候,想韩冈来提亲的人家,都是把嫁妆单子一一列出,连着名帖一起请着媒人递过来。再如当日韩冈听王厚说的,曾经在陕西挣下个金毛鼠名头的冯京冯当世,他考上状元后,有家外戚想招他为婿,便是把他请到家中,把十几万贯的嫁妆箱子一个一个的摆在他面前。

反过来说,如果哪家嫁女儿不给足嫁妆,婆家便绝不会有好脸色看,打骂是轻的,直接休掉也是常有的事。如今若是哪家生了女儿多了,父母就等着哭吧!看到生下来的是女儿,直接溺死在水盆里,这样的事都不值得惊奇,尤其在江南,民风奢侈,婚丧嫁娶花费尤高,因不想十几年后为女儿的嫁妆倾家荡产,多少父母生下女婴后就丢进水里。

所以韩冈看着这一溜嫁妆队伍才觉得奇怪,难道县主就能摆这么大的谱?把个空箱子摆在外面走?他随口问着身边一个脸比马都长的汉子:“敢问兄台,难道箱子里面就是嫁妆?怎么我看三十多个箱子,好像没一个重的!?”

路明在后面用力扯了下韩冈的袖口,韩冈的眼神是好,但这话问的就丢人了。

果然,马脸汉子看韩冈,完全是看到乡下土包子的表情,一脸的鄙夷:“好叫秀才知道,别人家的女儿是赔钱货,但这宗室家的女儿,却是能倒收钱的!”

不懂就问,即便被人鄙视了,韩冈也不觉得有什么丢脸,他的自尊心可没这么脆弱。微微笑了笑,点了下头,算是在道谢,马脸汉子反倒看着一愣。

路明挤到韩冈身边,向他解释道:“宣祖生了三兄弟,太祖、太宗还有坏了事的魏王。依照太祖当初颁的旨意,他们的后人都是皇亲。太宗朝、真宗朝还好,但到了仁宗朝后,宗室便越来越多,也越来越穷,那些不成器的就打起了嫁女儿的注意。娶了宗亲,少不了一个环卫官【注1】,为了一个官身,愿意掏钱的人家不少。”

他又转头问马脸汉子:“兄台,现在一个县主的聘礼是什么价码了?还是一万贯吗?”

马脸汉子一声笑:“那是哪年的老黄历了?一万贯是皇佑时候的价码!早没那么值钱了,现今是五千贯还有得找。宗女更便宜,一千贯就能领回家去。”

出嫁的队伍走到城门口,并不出城,径自转往北去,一片锣鼓响,新郎官骑着匹马,护着架大红饰彩的花轿,走过了众人面前。韩冈看着新郎官,左看右看,怎么觉得这位胡子都有些花白的新郎,少说也该超过四十岁了。可王舜臣的例子摆在前面,让韩冈不敢乱猜,也许是少年白也说不定。

“原来是肖生药!”马脸汉子认出了新郎官的模样,立刻愤愤不平的啐了一口:“那鸟货,都四十八了,还敢娶个十七八的,也不看他下面玩意儿什么时候管用过!”

转过来,换上一脸猥琐笑意,他又对韩冈几人道:“肖白郎那厮自幼天阉,为了方便自治,便开了一家生药铺子,却也没用。平日里为了掩饰,却把小甜水巷常来常往,袖子里都不忘揣上几根角先生。他自以为掩饰得好,还到处吹嘘自己一夜不停腰,却不想他的底细早被甜水巷的婊子传遍了。嘿嘿……今天夜里洞房花烛,肖生药为了一展雄风,多半会把他店里没切过的鹿角拿来用!”

嘲笑归嘲笑,但韩冈看马脸汉子的神色倒是羡慕的居多。他出言问着:“肖白郎应该是做生药买卖的商人吧?宗室难道连亲家是商户都不在意?”

“在意什么?有钱不就行了?”马脸汉子冷笑着:“进士不肯跟宗室结亲,怕耽误了前程,荫补的官儿也不肯跟宗室结亲,同样是怕耽误前程——他们亲爹的。也就是些商人愿意结个亲家,好歹混个官身。进纳官要掏钱,跟宗室结亲也要掏钱,左右都是掏钱,当然选个带添头的。”

这添头是娶来的浑家呢,还是指的官身?韩冈嗤笑了一声,多半是前者。

“就像大桶张家那样吧?”路明说道。

“大桶张家早败落了……”马脸汉子看土包子的眼神同样砸到了路明的头上,嘴角歪歪的像是在嘲笑,“不过他家娶得县主是多。仁宗的时候一大家子前前后后总共娶了三十多个县主,小张县马,死了两任县主浑家,第三次娶妻还是个县主。虽说现在败落了,但在马行街南还有个大桶张宅园子,七十二家正店里排在前二十的。”

“这都能败落?”路明摇头感叹了几声,又问:“如今是哪家娶得县主多?”

“帽子田家!据说娶了十几个县主!正旦祭祖,田家祖宗的神主下面,跪了一地县马。”

“怎么都是县马?”刘仲武在后面听着,也听出了兴趣,挤上前来问着。

马脸汉子回头打量了刘仲武一下,看着像是韩冈一伙,便向他解释道:“公主、郡主人少,跟宫里走得近,太皇太后、太后都看着,商人肯定没份,皆是跟勋贵家联姻,用钱能买到的都是县主、宗女。”

“卖大桶的,卖帽子的,都能跟天家成亲家了。”刘仲武摇着头,皇帝在他们这样的边远小臣眼里,就是天上神明一般的人物。想到皇帝的亲戚都是跟商人结亲,心里总之有些很不舒服。

“大桶,帽子,都是张家、田家早年起家时候的事了。后来发了家,这两家哪家还会把旧生意做主业?”

“那他们现在做什么?开酒楼?”韩冈还记得方才马脸汉子说过大桶张宅酒楼,能名入京师七十二家正店之列,而且排在前二十,放在后世。五星级是跑不了的,日进斗金自不消说。

马脸汉子比起小拇指,“那是小头!旧业也能赚一点!还有在开封府十六县里买地收租佃,也是一份。可更多的还是放贷收息!”

韩冈心神一凛:“放贷?!”

马脸汉子很奇怪的瞥了韩冈一眼,再土包子也不该连这事都不知道吧,天下哪个军州应是都一样啊,“现在哪家做买卖的不放贷?别人家的田地产业,不贷给他钱怎么弄到手?”马脸汉子左右看看,侧过头神神秘秘的压低声音说着:“宗室家不敢出来做买卖,怕丢了天家的脸。但亲家就没问题了。王公家的余钱如今都是交给他们亲家去放账。还有外戚,也是一样。曹、高两家,哪家不是如此?!”

听到这话,韩冈心中越发的不看好王安石的结果。看看王安石要从什么人手上抢钱啊?!宗室、外戚,还有天子赵顼的亲娘和奶奶!光一个青苗贷就把这么一群人一股脑的都得罪了,变法不失败那才叫奇怪!

皇帝当然想富国强兵,因为大宋是他的基业。但他身边的亲戚臣子可都不想看着原本属于自家的钱钞流进国库去,毁家纾难的觉悟,韩冈不认为他们会有。大宋是官家的,铜钱才是自己的,这样的想法才是常例。

对了!韩冈突然又想起,除了青苗法外,均输法其实也是与东京城里的豪商有点关联,虽然具体的利益纠缠他没机会去深入的了解,但一个‘徙贵就贱,用近易远’,便是要平抑物价,抢走商家赚钱的机会。而商家身后的宗亲呢,对此又会有什么想法?

豪商与宗室之间的联姻,这绝对不什么好事,对变法派尤其如此!变革是最忌讳的就是京城动荡,首都是国之重心,一旦都城动乱,全国都不会安稳。统治阶级内乱,如果天子镇压不住,牺牲首倡者是必然,晁错不就是朝服腰斩于市吗?内外风雨交加,这青苗贷王安石还能坚持下去?!韩冈不知赵顼和王安石推行青苗贷的时候有没有考虑到这么多,但他清楚,要应付起来一点也不容易。

虽然从后世带来的记忆中,韩冈知道变法事业不会那么快失败,但只要王安石不能大杀四方,把所有反对者都从肉体上消灭,等到变法失败,现在被压服下去的反对派,反扑起来就会越猛烈。商鞅做得够狠了,把太子的师傅都杀了祭旗,最后的结局呢,车裂!

韩冈完全不看好王安石的结局,就算没有从前生带来的那点模糊记忆,只凭现在了解到的信息就能做出判断。车裂虽不至于,但落职却是免不了的,到那时,说不定就是树倒猢狲散。据韩冈所知,王韶的心中早早的就转着等到从河湟凯旋,便跟变法派一刀两断的盘算。

出嫁的队伍已经全部走过去了,御街上重新被行人占满。韩冈与马脸汉子拱手道别,正要往驿站去,人群中不知从哪里传来一个兴奋的声音:“听没听说!听没听说!王大参请郡了!”

注1:不是环卫所的环卫,而是环绕保卫天子的环卫官。旧时是给天子身边护卫的,后来逐渐变为给宗室子弟和戚里的虚头官职。

