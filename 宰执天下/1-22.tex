\section{第12章 大厦将颓急遣行(中)}

吴衍和韩冈此时正在州衙之中。

秦州的州衙就是普通的院落,也没什么特别的地方,不过占地大,屋舍多罢了。唯一有点特别的,就是周围的围墙高达一丈还多,形制如同城墙,有女墙,有雉堞,宽达五六尺。这是为了在城破后,能继续展开巷战而设计出的式样。

大堂,二堂等处于中轴线上的建筑,属于州衙的正主,也就是秦州知州。至于吴衍这位节度判官,则是拥有西侧的一间院落作为自己的公厅。但吴衍并没有带着韩冈去节判厅,而是带着他去找隔邻的节度推官。

如今北面战事正烈,经略相公李师中尚未回返。作为署理兵事的节度判官,压在吴衍身上的事情并不少。但作为第一责任人,他有义务在移交本案时,将事情详细向主管刑名的节度推官说明。不过此时推官厅中却没人值守,吴衍叹了口气,又把韩冈带回了自己的公厅。

“坐罢!”吴衍先唤了一名值夜的老兵,命他端茶上来。再指着下首的一张交椅,示意韩冈坐下说话。他对韩冈的印象很好,说话便甚为温和。

韩冈没有坐,反倒对吴衍跪倒行礼道:“学生有事要向节判请罪。”

吴衍纳闷,这算是什么话。他欠身问道:“你有何罪?”

“私开军库,取用器械之罪。”

吴衍失笑:“这算得什么事……”他话声突然一停,像是想起了什么,“为什么韩秀才你能确定刘三三人今夜会来?”

韩冈道:“因为学生今日说要清点库房以便交接时,带着学生来此的李留哥神情有异。朝廷下令清点州中财计,府君纵火焚烧账簿的事,学生也曾听过。若真有此事,给他们得手后,学生将百口莫辩,百死莫赎。所以多留了一个心,做了点准备。本以为只是有备无患,没想到他们竟然那般心急。”

韩冈说得并无漏洞,吴衍轻轻颔首表示同意,韩冈说的他都明白,这本也不是什么奇事。

韩冈就是被挑选出来的替死鬼。失火的罪魁死在了火里,守门的王五、王九判个流放,如果为了保险,在狱中灭口报个瘐死也行。至于军器库直属上司——兵曹和县尉担个领导责任,落职待审,如今的知县则是直接罢任。而押司陈举,则可以安安心心的跟户曹书办刘显坐在一起喝茶,黄德用也得到了他想要的小美人,李癞子几十年的夙愿得偿,一切都安逸了。

只可恨呐,韩冈这个反角为什么不按编好的剧本去演?一场好戏彻底给砸掉了!

韩冈心知陈举绝对是这么在想。而他在吴衍面前说出这番话,真正要对付的已经不是黄大瘤,而正是黄大瘤身后的陈举。当他射死了刘三,逼得王五王九献上了投名状,黄大瘤就已经是个死老虎了。但黄大瘤身后,还有传说中在成纪县一手遮天的陈举。

秦州州治便是成纪县。州衙和县衙都是在一座城中,陈举号称一手遮天,但正如韩冈前日对他父母所说,在秦州城中的一众文武官员面前,小小的押司根本算不上号人物。他的遮天,不过是像云翳一般,将百姓和官员分割开来,若真有人能冲破云层的遮挡,回头看看,其实也不过是层稀薄的水汽罢了。

陈举不似黄大瘤、李癞子,在城中的名声并不恶。坏事都让手下亲信做了,自己便能得个好名声。可是在组成了以自己为中心的利益集团的同时,却少不得会侵害到其他势力的利益。陈举在成纪县中三十年,得罪的人必然不在少数,只是畏他势力庞大,投鼠忌器而已。如果能从他在秦州布下的关系网上撕破一个口子,动摇到他的地位,在阴暗处涌动的潜流,足以把陈举的势力给劈成碎片。

韩冈已经做了个开头,没有理由不继续下去。也心知此时不得不搏上一搏。为了日后的安全起见,必须将陈举一棍子打死。

“是陈举吗?”吴衍的问题,如天外一剑,让韩冈猛然心惊。吴衍并非蠢人,在秦州任职也有两年。对陈举的了解,比韩冈还要清楚。之所以将韩冈三人带回州衙,而不是移交成纪县,也正是为了防着陈举。

吴衍不是不想对付陈举,但若是因此惹来一身骚,却又不值当了。陈举不是小人物,他的垂死挣扎,足以咬进一名从八品京官的骨头里。

虽然欣赏韩冈,但吴衍不会去冒险!

做官一任三年,但吏职可是能做一辈子。陈举从他祖父辈起就是在成纪县衙里做事,那时真宗才刚刚即位没多久。如今几十年过去,陈举本人都已经做了三十年的吏员,升到县级吏职中等级最高的押司,而且还有几个散官职,有个名目唤作银酒监武——银青光禄大夫、检校国子祭酒、兼监察御史、武骑尉【注1】。

虽然这几个名号都是给吏员的虚衔,审官院查无其人,官告院亦不录其名,仅是唐末五代时官制败坏后滥封官爵的产物,但能得到这等散官的,一个州近千胥吏中也没有几人。

同时此时还有个说法,叫官无封建,而吏有封建。如陈举这样祖孙几代在一间衙门里做事,所在多有,但官员任职不过是走马观花,往往一任未满便调往他任——有的时候,知州知县的位置上,一年能换个五六个官员——交椅还未坐热,就要赶着换岗,这样如何是下面这些人精的对手?

官员被胥吏瞒骗,弄到丢官去职的例子太多了,好一点,也是灰头土脸,就连包拯包孝肃,也照样被开封府的胥吏诓骗过。能压着胥吏好好做人的,泰半皆为名臣,他们整治胥吏的事迹,都能在正史传记中留下浓墨重彩的一笔!

天下胥吏皆可杀,这句话里含着多少官员的斑斑血泪!

看在横渠先生的面上,助韩冈一臂之力可以,但吴衍绝对不会赤膊上阵,拿自己去冒险!

……………………

昨日儿子独自入城,回家后韩千六在床上翻来覆去的一夜也没能合眼。第二天早上起来,浑家和养娘跟自己一样都是熬红了眼,一宿未睡。对于孤身留在城中,几乎是身处敌境的韩冈,家里没一个能放得下心去。韩阿李赶急赶忙的热了两块炊饼,韩千六拿在手上啃着就往渡头奔去。

大清早,阴风劲吹,天色阴阴,渡船上的空气也是阴郁的。韩千六坐在船头,双眼死死盯着坐在渡船另一头的李癞子。韩千六是个老实人,作奸犯科的事从来也不敢想过,甚至很少跟人斗过气,可他如今都恨不得将李癞子一脚踹进藉水里去。

李癞子在船尾坐得轻松自在,有个小厮跟在身边,他根本不怕老实做人的韩千六能做出什么。如果韩阿李在旁边那就不同了,现在不带上三五个家丁,李癞子绝不敢跟韩阿李打照面。

“韩老哥,是去城里看你家的三哥儿罢?”

李癞子没话找话,根本是怀着恶意的挑起话头。韩千六扭头看着河水,不去理会。可他这样反应正是李癞子所喜欢看到的,脸上的笑容更加得意。他亲家既然已经拍了胸脯保证了,那块河湾菜田,几天后就改姓为李,不再是抱养的,而是亲生的了。今天李癞子去城里,也是去探探消息的,去路上能碰到韩千六,不失一个打发时间的乐事。

藉水太窄,韩千六和李癞子都是还没坐热屁股底下的船底板,就只感觉着船身轻轻一震,渡船已经到了对岸。下了船,韩千六脚步匆匆,想把李癞子给甩掉。可李癞子带着小厮就是紧紧跟在后面,韩千六越是失态,他看着越是开心。为了河湾边的三亩菜园,他跟韩家争了二十年。如今终于即将如愿,李癞子的心情好得一路上哼着小曲,故意恶心着韩千六。

一路疾行,韩千六和李癞子一前一后走到城门下,就见着那里乱哄哄的,多少人被堵在城门口,要排着队才能入城,几个士兵反手拖着条杆棒,在城门外呼呼喝喝,整顿着队列秩序。入城的队列前进速度很慢,能看到每一个出入城门的行人和车辆,都是上上下下里里外外的搜查一遍才被放行。

李癞子扯住一个出来整顿秩序的士兵,塞了两文钱,冲着城门呶呶嘴,问道:“城里出了什么事?”

“好像昨天夜里有个姓韩的衙前杀了人,据说是烧军器库被发现了,可能是西贼的奸细。现在进城出城,都得搜一遍身。”

昨夜事发,到现在才几个时辰,除了相关人等,真实内情还没多少人知道。从衙门里传出来的信息都是支离破碎,都得靠着猜测和臆断来补全。

韩千六就在旁边,话声入耳就如五雷轰顶,就像陷入了一场恐怖的噩梦中一般,“不会的,三哥儿不会做这等事!”

李癞子也有些难以置信,但韩冈的硬脾气他是有所了解的。幸灾乐祸的笑容从他的脸上冒了出来,只恨不得狂笑一番来宣泄自己心中的快意。“韩老哥,你家三哥……”

“我怎么了?”一道很熟悉的声音突兀的在两人身边响起。扭头一看,李癞子惊得像只兔子一样蹦得老远。他刚刚提到的那人,不知何时竟然走到了身边。

注1:晚唐五代,官职泛滥。如银青光禄大夫,算是高品贵官,但小小的吏员也被封了此等官职。而宋朝建立后,除了将五代的苛捐杂税一并继承下来外,连胥吏带职的传统也有所继承。只不过胥吏的宪职,不通过审官院审核,不经过官告院录名,看起来再夸张,也只是好听罢了。像银酒监武这样的虚衔,宋廷一次就能封出一百多。而辽国也有着这虚头散官,用来安抚纳粟官(花钱买官)和匠作。只不过避辽太宗耶律德光讳,将银青光禄大夫改为银青崇禄大夫。

今天第一更,求红票,收藏

