\section{第24章 兵戈虽收战未宁(四)}

禹臧花麻退走,最得意的就是瞎药。原本还让他吃力应付的对手,转瞬间便成了受了惊的羊群,在他眼前四散逃开,往着任何一处能可能逃生的去处涌去。

瞎药大声呼喝,指挥着他的队伍纵横于战场之中,将所有不及逃窜的敌军全数歼灭。难得有机会欺负一下禹臧家这样顶尖的大部族,他越杀越是兴奋,刚刚把几队还保持着一点编制的对手给硬吞了下去,紧接着便追着一队逃出生天的幸运儿冲进了大来谷中,

韩冈脸色为之一变,连忙派出自己的一名亲卫:“去,快去!请瞎药巡检快回头。穷寇莫追,以防不测!”

可他的话还是慢了一步,瞎药和他的人在谷中转了个弯便没了踪影,过了一阵,则丢盔弃甲的回来了。韩冈派出去的亲兵,并没有来得及追上瞎药,只眼睁睁的看着他被禹臧花麻的一记回马枪,挑掉了两百多将士。瞎药的兵是从族中临时征发起来,比起禹臧花麻用来殿后的七八百精锐的常备兵,差了不止一筹。

幸好禹臧花麻无意在大来谷中与瞎药缠斗,逗留越久,越是危险。在给了瞎药一个惨痛的教训后,他便扬长而去,让瞎药咬牙切齿的吞下苦涩的败果。

“让他吃点苦头也好……”不知何时,苗授已来到韩冈的身边,“这些蕃人不让他们吃点苦头,就不知道天高地厚了。”

韩冈静静的看着瞎药垂头丧气的从谷中出来,慢慢点头:“都巡说得正是。”

天光将晚,夜色已经笼罩了东方,能隐隐约约的从夜幕中看到无穷无尽的繁星。只有禹臧花麻遁走的方向,还有着一幅横跨天际的红色彤云,宣告着黄昏尚未终结。

身处战场之中,敌军仅仅只是退走而已,并不能确定他们是否会回来。等天黑后,这片山谷前的开阔地,即便是对于仍驻留在这片战场上几千名宋军,也一样是危机四伏。但眼下的时间,已经不容许宋军再赶回渭源堡。何况一场大战之后,将士们的体力消耗极大,眼前就有不少人坐在地上不肯动弹,让他们连夜回师渭源,也显得太过不通人情。

所以苗授的第一件事,是遣人连夜赶回渭源堡,向翘首以待的王韶通报战事结果。而第二件事,就是派人收拾了禹臧花麻留下的营盘,重新加固外围防御,并安置下营帐。苗履奉了父命,带领得力人手打扫起战场来。兵甲、旗帜、战马都要好生收集,投降的敌军看押起来,而受了伤的,则直接给他们一个痛快。另外,苗授还派了帐下书办去点算各部的斩首,登记造册,以便回去后上报请功。

而韩冈则做着他的本职工作,把自己的亲卫还有王舜臣的亲卫,都集合起来,打发他们去帮着处理伤患。苗授听说此事,也把自己亲卫中,进疗养院培训过战地急救术的两人,也派了过来。经过一番手忙脚乱的急救处理,有不少伤员都幸运的保住了他们的小命。虽然伤亡人数至少到要明天才能有个准确的数字,但依然可以确定,比起过往的战事,今次的伤亡情况肯定要好上不少。

安排下一番琐碎杂事,营盘也已经整理完毕,韩冈和苗授便进了主帐。九月山中,夜风清寒。不过主帐内已经点起了火盆,使得帐中温暖如春。而且在火盆上,还架着一个铁锅,里面还烫着酒。锅中水已经沸腾,咕嘟咕嘟的冒着气泡,而酒香也随之四溢,充斥在帐中。

兵收戈止,苗授便收起了他在战场中表现出来到嗜血和疯狂,重又变得温文尔雅,问候过韩冈之后,便微笑着亲手给韩冈倒了一杯热酒,表示自己心中的谢意:“今次一战多得玉昆之力。若非玉昆你及时赶回,并抵挡了禹臧花麻的偏师,这一战还不知会有什么结果。”

“下官仅仅是跟偏师厮杀,而独力对抗禹臧家主力的还是都巡。论功劳,还是都巡更大一点。”韩冈自谦的说着。他跟苗授对饮了几杯,热腾腾的酒液下肚后,就仿佛有一团火在腹中传开,将渗入体内的寒气全都驱散。

熊熊火光映红了韩冈满面风尘的一张脸,想起刚刚结束的一番大战,他心中后怕不已。今日一战,虽然的确是胜了,但现在他回想起来,却胜得很险。若是禹臧花麻肯硬拼,胜负还未可知。他摇晃着酒盏,“其实禹臧花麻如果再能坚持一下,说不定我们就败了。”

苗授摇头笑道:“跟着禹臧花麻出战的都是族中子弟,又不是没干系的外人,哪里会真的硬拼到底?被他丢下的那群背时货,玉昆你也该听了他们的供词,都不是禹臧家的人,只是些附庸而已。丢下自家人,禹臧花麻回去后不好交代,但抛下附庸,让自家子弟得以安然回返,却能让禹臧族中老人们都闭上嘴。”

不知是酒意上头,还是无意在自己人面前虚言掩饰,苗授推心置腹的跟韩冈说道:“说句实话,我等为求一个封妻荫子,不会吝惜下面士卒的性命。但蕃人就不同了,正常情况下谁也不会拿着自家子弟跟人硬拼……玉昆,你可知道为什么过去的三十年,官军总是被西贼伏击?”

“贪功累事!”韩冈不假思索,这在国中都已是定论了。

“说得没错,正是因为贪功!”苗授盯着火盆中跳动着的明红色火焰,同样明亮的焰火也在他的瞳孔中闪耀,“任福、葛怀敏,哪个不是因为贪功才丢了性命?而相对于官军,西贼就很少会吃埋伏。他们出来征战,仅是求钱粮财帛而已,盯准了肥羊抢一把就走,遇上危险那就绕行。不想着博取功名、争权夺利,便不会跳入陷阱……”他突然一声嗤笑,“这大概也可以算是无欲则刚吧!”

韩冈喃喃的揣摩了一阵,起身向苗授道谢:“多谢都巡指点。”

苗授的确是在指点韩冈,他的话其实已经很隐晦的向韩冈说明了伏击为何会失败。

韩冈是把这群吐蕃人当作了跟自己以及他所熟悉的秦州文武官员来设计,但除了禹臧花麻等地位最高的几人外,剩下的其实不过是些强盗罢了,根本不会为了战功而让自己身陷险境。

前面设伏时韩冈竟然忘了这一茬,让吐蕃人跟在后面拣了一堆便宜。一直到了伏击圈,看到追击的对象都已经把身上的东西都丢光了,这群吐蕃人失去了追杀的理由,所以才会干净利落的退回去。若是少让人丢些东西,也许韩冈所设计的对象,真的会一直追到伏击圈中。

‘强盗的思维逻辑当真是让人难以理解。’韩冈心里想着。大宋周边的蕃部,一直以来都是把汉人当作肥羊来宰割,靠着劫掠来的财富满足自己的欲望,不论契丹,还是党项,都是一般无二。在韩冈看来,这些蕃人都是些养不熟的饿狼。

不过自从澶渊之盟后,契丹人就收手不干了,因为他们已经有了旱涝保收的岁币,而且他们从南京道——也就是幽燕之地——的汉人手中,也能收取大量的税赋,不需要因为钱财之物而跟大宋闹翻。

但西夏这边,却并没有南京道这样富庶的土地,而时有时无的‘岁赐’,却是逼得关西遭到年年入寇的主因。因为韩冈对西贼绝无好感,故而便能一刀斩了野利征。不过也为了避免日后的麻烦,他才会把这份功劳送给瞎药,这样就不会有人对他说什么两国交兵、不斩来使的笑话了。

相对于契丹、党项,吐蕃人早在唐时,就已经在抢掠汉人的财富了。比起建立了辽夏的民族,吐蕃才是领先数百年的老前辈。尤其是在旧年镇压西域的吐蕃王国灭国之后,残存在河湟之地的吐蕃人做惯了强盗,只剩下劫掠这一简单粗暴的手段了。

韩冈如果从这方面去入手,说不定就能成功了,但用战功来引诱,却是把媚眼做给了瞎子看。

韩冈与苗授围炉夜话,一点水酒,让他们聊天到了深夜。第二天,当两人领兵回到渭源,这场战事总算是宣告结束。

今次一战,交战的双方都吃了点亏,却都没有吃大亏。而且无论是禹臧花麻还是王韶,都实现了他们最初的目的,并安然的各自返回自己的地盘。

一时之间,和平也终于降临这片土地。但任谁都知道,围绕着河湟之地的争斗,其实不过是才开了一个头。

宋、夏两方都有染指河湟的心思。大宋这边,王韶咄咄逼人,让河湟的每一家部族都警惕起来。而西夏虽然光是为了对抗陕西四路和河东路,便已是有些力不从心了,但仅仅是禹臧花麻一家,就已经让王韶感受到了威胁。

而尚未归顺任何一方的吐蕃部族中,首当其冲的木征,他的动向和想法尤为让人困扰。没有木征的首肯,禹臧花麻绝对不可能借道武胜军,韩冈和王韶都在猜测,他是不是在暗示他必要时会投向西夏一方——从今次木征和禹臧花麻之间的默契来看,两人私下里的联络应该不少。

不过河湟的战局,仅仅是宋夏两国之间如火如荼的交锋中的一个缩影,在鄜延、在环庆、在河东,都有着同样激烈的战斗。两国之间新一轮的战事,此时刚刚拉开了序幕。

