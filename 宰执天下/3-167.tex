\section{第43章 竹纸知何物(下)}

除夕的钟声越来越近,京城中过节的气氛也越发的热烈了起来。

今年好歹度过了灾伤,有眼见着入冬后连番降雪,不用担心来春旱情,京城中的百姓也都恢复了旧时大手大脚的习惯。

到了腊月下旬,大相国寺每月五次万姓交易的日子也就剩两次。这个时候,就是除了年节之时以及四月初八佛诞日,大相国寺一年之中最为喧闹的时刻——如果是以殿前三门广场上市集的热闹程度比较,就算是年节和佛诞日,都远远比不上。

从大相国寺由太宗皇帝亲笔题额的牌楼下走过,大门之后就是贩售飞禽猫犬、珍禽奇兽的区域。穿过此处,在二门、三门处,则是日用、军器和零食,如蒲合、簟席、屏帏、洗漱、鞍辔、弓剑、时果、腊脯之类。

一边是要进来烧香的信众,一边则是要买年货、特产的顾客,大相国寺之前正挤得人山人海。踩掉了鞋,挤掉了帽的情况,都不少见。

作为一路帅臣,郭逵每年至少都要入京诣阙一回,过去也曾常住于此多年,东京城的繁华倒也并不陌生,而大相国寺逢到腊月时的热闹,更是一清而楚。但他作为一名武将,一辈子杀人无算,免不了要靠着礼拜神佛来安心,每次进京,都会来大相国寺一趟。

郭逵今日来大相国寺烧香礼佛,就是避开正门,从后门进来的。虽然后门处也是人声鼎沸,但都是些卖书画、珍玩的摊子,还有些摊位则是代售诸路罢任官员,从地方上带回京来的土产——郭逵一向喜好货殖之术,他这一次入京就也有些土产带回来,但这些琐事自有家人掌管,郭逵只要在家里看现钱就行了——所以顾客终究还是不如正门处多。

郭逵带着儿子郭忠孝在大雄宝殿中上过香,又捐了一批金帛香油作为供物,便闲极无聊的在寺中的殿阁间信步游逛了起来。

如果给耳朵长得跟兔子一般的御史听说他明明已经接受王命,却不赶紧去太原府上任,反而来闲逛大相国寺,肯定要奏上一本,但郭逵可不在乎。犯些小过被人弹劾,反而是好事。

他去太原府的任命也已经确定,进京不过是走过场而已。见到天子,更没什么多余的话说。不过是将原本因故被剥去的宣徽使一职,又还给他而已。这算什么酬劳?但郭逵还是做出一副大喜过往的态度来拜领了这份任命。他如今的地位太高,如果不加以收敛,落到狄青、曹利用的下场不足为奇。

慢慢的一路走到二殿天王堂。天王堂的外廊上,是一幅炽盛光佛降九矅鬼百戏的壁画,乃是仁宗朝翰林院画待诏的手笔,炽盛光佛身周光芒四耀,威猛无俦,而被起压制的九鬼,则是神态各异,或胆怯、或狰狞,或狂嚎,姿态个个不一。是大相国寺中,最为有名的几处佛图。

不过在壁画前,此时拥着一群人。其中有两个是官员,一个红袍、一个青袍,而剩下的看其穿着不类中国人氏,郭逵也认不出是哪里的人,聚一起在看着墙上的壁画,一边对着壁画指指点点的。

郭逵冲着他们呶呶嘴,一名伴当会意的上前去打听。片刻后转回来,道:“是高丽使臣金良鉴。听说今天是特地来大相国寺拜佛的。”

郭逵听说是高丽使臣,转身就绕路往前殿罗汉堂走。此等外夷使节,做臣子的根本就不能沾边。除了朝廷专门指定随行陪伴的馆伴使,否则瓜田李下之嫌,文臣武臣沾上都是个大麻烦。

走到罗汉堂,再往前就是三门处满是摊点的广场,郭逵本来就不怎么喜喧闹,也不跨出去,转头就准备欣赏起殿中的五百尊金罗汉来。

只是在一瞥眼间,郭逵却于殿门外不意发现了一个熟悉的身影。穿着青布襕衫,在一家卖彩灯的摊子前站着,手里还正拿着一盏孔明灯【注1】。

“韩冈?!”

……………………

外层是极薄的竹纸,而内里骨架则使用着极细的竹篾给撑起来。里面是一支手指长的红色蜡烛,四面绘着精美的花卉图案。这么一盏制作精美的孔明灯,现在就在彩灯摊前站着的年轻官人手中。

能在大相国寺摆摊,摊主本身就得有些能耐,眼睛也早就给磨得利了。

面前的这位年轻官人,只看装束,就像是个年轻的秀才。但他身上所着的襕衫所用布料,怎么看都不像是丝麻所制。再看他后面还跟着几个孔武有力的伴当,又像富贵人家的子弟,可是神情态度却一点也不似寻常的衙内,仅仅是随便一站,便是身居高位的气派。

相貌虽然不是此时受姐儿欢迎的秀气斯文的白面书生,但看着就像是文武双全的模样,加之身高体健,自有一番吸引人的气度。周围来上香的女眷,十个之中能有一半,往他这边看过来。

‘说不定能作笔大买卖。’想到这里,摊主心头就热了起来。

“这灯多少钱?”韩冈看了手中孔明灯一阵,终于抬头问着价格。

摊主听得发问,连忙回话道:“官人,这折枝百花灯一套二十五盏,只整卖,不单卖。”

“一套二十五盏?”

韩冈上下翻看着这盏四面绘花的纸灯,上面有一朵合欢,一朵栀子,还有两朵不认识,但做工精美,而且画工也是上成,只是想不到竟然是套装。

见着韩冈看似有了些兴趣,卖灯的摊主更加殷勤起来:“官人有所不知,这一套孔明灯,上绘折枝百花,是京中有名的灯笼张亲手糊制,而绘图的也是名师所作,是陈待诏的亲传弟子。只有小人摊子上有,别家店铺根本就找不到”

那卖灯一边推销着,一边指着灯笼一角给韩冈看,的确能看到鲜红的印记。

“寻常的孔明灯,就是个纸袋子,里面用粗粗劈就的竹篾架起来,居中放上一团浸了油的粗布。点着了,只能在天上飘个半刻钟。而小人的折枝百花灯,用的是上好西河竹的篾丝,还有敬玉堂的竹纸,里面放的是上品蜡烛,点起来飞上半个时辰都不会落地。这么一套,才不过三贯钱而已,东京城中哪里能寻得来?”

韩冈倒不管贵还是便宜,只要能飞就行。一套二十五盏虽然多了些,但拿回去摆在家里也不错。连讨价还价也不做,直接示意随行的伴当付了钱。付了帐,他又问着摊主:“这个灯笼张是什么人?”

摊主连忙道:“正是小人家传的名号,现在是小人之父用着。”

韩冈笑了笑,将手上的纸灯交还给张姓的灯笼摊主,“二十多盏灯带着太累赘,收市后一发儿送到常乐坊的韩舍人家。”

“韩舍人?”摊主闻言张大了嘴,他可听说过这一位。

韩冈已经踱着步子走开,摊主的惊异由他的伴当来回答,“如今朝中韩姓的起居舍人,可就我家舍人一个!”

买过了孔明灯,韩冈就又准备在寺中逛上一逛。他今天主要是来见刚刚升任左街正僧录,成为国中最高僧官的智缘。亲自下场买东西,却是一时起了兴致。

“可是玉昆兄?”

一个隐约曾有听闻的声音在身后响起。韩冈回头一看,先是一怔,然后方才认出是久未谋面的郭逵之子郭忠孝,“怎么是立之兄?”

“随家严礼佛还愿来的。”郭忠孝笑意盈盈,问道:“玉昆兄也是来烧香的?”

官员来大相国寺烧香拜佛的多,可逛殿前的集市却几乎没有。尤其是韩冈这等身份的官员,更是少见。都是要自重身份,也怕御史多嘴多舌。即便有,也仅仅是逛一下佛殿前的几家店铺——赵家的笔,潘家的墨,都是京中最受士人欢迎的文房用具。像两廊中,各尼庵师姑们来贩售的女红等饰物,绝不会有官员有脸挤在女眷之中去购买。

“来见故友,顺便准备买艘船回家。”韩冈说着让人不明不白的话,双眼则一扫郭忠孝过来的方向,登时就发现了负手站在罗汉堂中的郭逵。

听着韩冈的话,郭忠孝一时愣住,“船?”

韩冈没多解释,向罗汉堂走过去与郭逵见礼,“韩冈拜见宣徽。”

郭逵拱手还礼:“玉昆,久违了。”

郭逵比起当年要见老,但神采依旧,依然是大宋军中首屈一指的将帅。见着周围闲人都向他们看过来,郭逵眉头一皱,“且陪老夫走一走。”

韩冈跟在郭逵,差了半步的距离。听着郭逵在前面说道:“今守太原,本来是想拜一拜我佛,求一个安心。想不到竟然见到玉昆。”

韩冈笑道:“北虏张狂,不得宣徽坐镇北门,天子岂能安寝?”

注1:北宋时有关孔明灯的记载一时没有找到,但南宋范成大的《上元纪吴中节物俳谐体三十二韵》中有‘掷烛腾空稳’一句,从这句来看,孔明灯在宋时还是存在的,可能叫做掷烛灯。不过为了行文方便,文中还是以孔明灯为名。

