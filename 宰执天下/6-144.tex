\section{第13章 晨奎错落天日近(九)}

元佑二年的正旦过得平平淡淡。

没有了大朝会,大多数官员,也就能够在家中多睡上一会儿。

不过太后的病情,牵动着京城上下的每一个人心。

这两天,太后正逐渐康复的消息从宫中传了出来,说是再过些日子,就能垂帘理事了,朝臣们都听说了之后,总算是平静了下来。

至少是在表面上平静下来了。

正月初三天庆节,是真宗皇帝为了庆祝‘天书’降世,特地设立的节日。虽然‘天书’随着真宗皇帝一并葬入了永定陵,可时至如今,每逢天庆节时,朝廷依惯例休务五日。而同时设立的天祯节、天贶节,也一样保持了下来,只是因为要避仁宗赵祯的讳,将天祯节改为天祺节,同样放假。

不过宰辅们没有机会休息,天庆节要开道场设醮,宰辅们不仅要到场,之后还得去上清宫进香祈福。

韩冈和章敦都是从上清宫回来,两人很有一段时间没有一起好好聊一聊了。

对王安石说得那些话,韩冈没有瞒着苏颂,也没打算瞒着章敦。

之前抽空对苏颂说了,今天有了机会,半道上便原原本本的告诉了章敦。

听了韩冈说完,章敦沉默了很久,“这么说……玉昆你早就知道吕吉甫不能成事了?”

章敦没有怀疑韩冈对日本多产金银的判断,即使之前日本并没有这方面的名声,可他对韩冈言辞的信任,就跟王安石一样。

韩冈在这方面树立起的权威,能动摇对手的信心,而他一贯所表现出来的品行,也让王安石和章敦这等关系微妙的政敌,能够充分信任他。

“吕吉甫的盘算,也是最近才看透的。”从认为吕惠卿是要胁迫朝廷将他调回京城,到发现吕惠卿借势取利的想法,的确费了韩冈一些时间,“就是因为知道耶律乙辛不会困于财帛,所以一直都没往那个地方去想——坐拥日本的金银矿,耶律乙辛能做的选择太多了。不过……子厚你瞒得真好啊……”

韩冈瞅着章敦,对他笑着说道。

“三天的时间,足够吕吉甫知道这件事了。”章敦顾左右而言他。

都从韩冈这边得到了消息,王安石当然会写信给吕惠卿。从开封到大名府,的确不算远。

不过章敦把话题转移到这件事上,让韩冈感觉之前王安石并没有将计划向章敦和盘托出。以章敦的性格,肯定是不屑解释,但从他语气的变化,足以让韩冈看出底细了。

“吕吉甫会怎么做?”韩冈问道。

“不管他会怎么做,不得朝旨,谁敢妄动兵马,就是死罪!”章敦发狠的说着。

“谁能妄动兵马?”韩冈反问。

如果是辽军来攻,他坚守城池还好说。要是耶律乙辛不来,吕惠卿又怎么可能在朝廷还没有决定的情况下,出兵北上。就算他有那个想法,下面的将校也不会听他的吩咐。

即便是以种谔的胆大妄为,曾经背着枢密院出兵,但那时候,他的背后还有皇帝,始终也有密旨。即便事后两府要追究,至少还能保住性命。

而没有来自朝廷的诏书,只有大名府的经略安抚使的钧令,让边地各军州调集大军,主动攻向辽国境内。试问吕惠卿能够使唤动几个人?

“吕惠卿至少能骗过几个贪功的将校。只要能引来几部辽军兵马,至少事后能够糊弄过去。”

到时候还是能推说辽人先行犯境。雄州对面,驻扎了多少皮室军,不论是章敦,还是韩冈,心底都是清楚的。挑动大军来袭已经不可能了,但诱使辽国的边境驻军来攻,即便是官军先行越境,事后吕惠卿总有很大机会给遮掩过去。

“那还要他能打赢才行!”章敦冷声说道。

“如果真要打的话,还是得盼他赢下来。”韩冈叹息着。

士林中的风向,还没有改变。阻止对辽开战,依然是要以个人名声为代价。而站在韩冈和章敦的立场上,不可能去期盼吕惠卿惨败而归,那样损耗的都是国中精锐,还会影响到日后攻辽的计划。

这实在是让他们处在两难的境地上。章敦其实的不是很在乎,河北损失大了,大不了从关西调兵。而韩冈却不能不在乎,如果已经尽力去阻止而不成,他事后还能安心。可要是为了欲擒故纵,故意放纵吕惠卿,事后心中还是会堵得不舒服。

章敦能感觉出韩冈的话发自肺腑,毕竟是老交情了,“的确不能任凭官军损失,否则又是肥了辽人……不过吕吉甫不会这么想玉昆你,怕也是不敢恣意妄为了。”

“好像是说过什么贾文和吧?”

章敦闻言一笑。当年在王安石府上见了韩冈第一面,给他留下的印象就是敢作敢为、唯恐天下不乱的性子。

不过韩冈行事中对黎庶和士卒都十分看顾,这是日后与韩冈共事时才知道的。

“若是能如子厚兄所言,那就太好了。”韩冈笑着说道。

换作别人处在自己的位置上,直接就会下手了。牺牲几百一千人的性命,去解决掉一个难缠的政敌,绝大多数官员绝对不会介意。何况这本就是政敌自己寻死,只需要利用一下就可以了。

但是韩冈终究与这个时代的士大夫还差上一点,终究不能无视几百一千条人命。何况兵势如水,本无形状规矩,从来不会让人心想事成。说不定吕惠卿会坚持冒险,带来一场大捷,然后将整个国家卷进去,

也有可能会是一场失败,然后带来一场超乎预计的大战。

耶律乙辛手上有钱不假,可这并不代表他肯定不会来袭。来自日本的白银和黄金,只是让他的选择余地更大,不会为财帛而在错误的时间和错误的地点,进行一场错误的战争。

人心终究是没办法猜透的,尤其是处在吕惠卿的位置上。当他收到王安石的去信之后,还能作什么,外人是无法计算清楚的。

人的判断,在理智之外,还有情绪的干扰。

不过到了天庆节的休假结束,文武百官重新回到他们的岗位上的那一天,韩冈终于知道了吕惠卿的反应。

随着今日太后出现在朝堂上,像往日一样的说话,朝廷已经安稳下来。

太后的病情平复,前几天的慌乱,就像是个笑话。尽管肯定有异心萌动,不过现在还不会有任何人敢于去挑战得到朝堂一致支持的太后的权威。

不需要再留任宿直,韩冈也可以安心的留在家中,拆看最近收到的信函和拜帖。

作为一名执掌国政的副相,韩冈每天收到的信件和拜帖多不胜数。有求官的,有问候的,有讨好的,还有诉冤的。在往日,除了一些朋友的信件,其他的信,韩冈都是一扫而过,几十上百封,不会费去他太多的时间。

不过韩冈今天只拆看了放在最上面的一封信,他就停住了,久久没有动作,只有笑容出现在脸上。

王旖进来时,正瞧见韩冈看着信发笑,惊讶的问道:“官人,谁写来的信?怎么边看边笑?”

韩冈放下信,抬起头来,对妻子道:“是吕惠卿。”

…………………………

“吕吉甫昨天送了一封信来。”

前往内东门小殿的半路上,章敦突然听到韩冈丢出一句话。

韩冈这种冷不丁的抛出一个消息,然后看人反应的习惯,章敦一直以来都不是很喜欢。

但许多时候,章敦都会为这句话的内容所吸引,而忘记了表示不满。

他这一次也是一样。

“吕吉甫写了些什么?”章敦问道。

距离从韩冈口中,听到耶律乙辛底牌的那一天,已经过去了五天。吕惠卿要是有反应,这时候也的确应该送到京城了。

“什么都没说,只是推荐了两个人。”韩冈笑道。

“就是这么简单?”

“换作子厚你在吕吉甫的位置上,写封信过来,会怎么写?”韩冈反问。

章敦沉默下来,换作是他,也一样什么都不会写。单只是写信这件事,已经有太多含义了。

“玉昆,你打算怎么做?”他问着韩冈。

“当然是把信收起来。难道把这封信给家岳吗?”

“为什么不?”章敦反问。

这样的一封信送去给王安石,王安石虽不至于立刻跟吕惠卿翻脸,但也肯定会留下心结,至少知道吕惠卿绝不会跟他一条心。

“还是算了。不能齐家,如何治国平天下?”

“是怕葡萄架子到了吗?”章敦摇摇头,轻轻笑了起来。

韩冈至少还想留着一份情面,在章敦看来,这到底还是一件好事。

韩冈轻松的心情只维持到一封雄州急报送来之前。

“雄州急报,腊月廿九,雄州城外军铺被毁,守军击杀三名越界虏兵,观其甲号,皆是皮室军出身。”

张璪拍起了桌子,大怒道:“为什么这么慢?以急脚递送信,三四天前就该把消息到了!”

韩冈拿着急报,“因为州将刘舜卿要查验真伪,将这个消息压了整整三天。”
