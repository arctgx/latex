\section{第23章 铁骑连声压金鼓(八)}

【这几天一天一更,向大家说声抱歉。现在事情终于结束了,明天回复正常更新。】

“退得好快!”

远隔近十里,星罗结城也仅是近于地平线处的一块手掌大小的暗影,而围在城外的军队则更为模糊难辨。只不过星罗结城外由千军万马组成的阵列虽然让人费尽眼力,但他们逐渐从城下撤离的趋势,却是能让立马山道上的韩冈、智缘还有瞎药等人瞧得分明。

“看来预定的计划行不通了。”看了一阵,韩冈回头对瞎药说道。

瞎药阴沉着脸点头称是。他本想将功赎罪,谁想到禹臧花麻的主力竟然不在渭源堡,而在星罗结城下。

由于与前方阻隔的关系,韩冈和瞎药对禹臧花麻自渭源堡下的撤退不甚了了。不论是韩冈前日从王惟新口中听说的,还是从瞎药自己派出去的哨探那里,都仅仅是得知禹臧花麻自东出大来谷后便分了兵,一部围攻渭源堡,另一部分兵力在攻打星罗结城这样模糊的情报。

从情理上判断,渭源堡的重要性远远在星罗结城之上,禹臧花麻不可能去动用主力攻打一座破败的城池,而放过渭源堡这块肥肉。

不过现下几人远眺着星罗结城下的军势,只要稍通军事,就知道攻城一方的兵马数量究竟是个什么样的级别。少说也要超过五千人的数目,不论是谁来看,都是禹臧家的主力无疑。

而与禹臧家的实力相比,瞎药手上的兵力就很可怜了,他带出来的族中战士还不到一千。如果是在对方攻城的过程中突袭,或许能够像当初对付董裕时那样打出一场震动朝堂的大胜来。可是在禹臧花麻已经撤退的情况下,再想追上去偷袭却只会惹人发笑。

无论是托硕大捷,还是古渭大捷,取胜的招数都是趁敌不意的突袭。今次韩冈和瞎药本想着估计重施,先行歼灭围攻星罗结城的敌军,而后返身把禹臧花麻的主力聚歼在渭源堡城下。韩冈也不是能未卜先知的算命先生,哪里能料到禹臧花麻的主力丢下了渭源堡,而竟然是在攻打星罗结城。

从韩冈所立足的山道这个角度看过去,他分不清禹臧花麻的军队究竟是破城劫掠后的胜利回师,还是攻城不利下的无奈撤离。两种不同的情况,对他下一步的行动有着最直接的影响。

瞎药也是有些惶惑,问韩冈道:“机宜,现在该怎么做?”

没有足够的情报支持,任何战略运筹都只是空谈而已。韩冈本质上是个谨慎的性子,外在表现出来的锋锐,也只是因为他提前把可能后果都盘算清楚,使得他行动起来毫无犹豫的缘故。

在没有了解到更多的信息前,韩冈不会冒着误陷敌阵的风险,“先派人跟上去打探一番再说。我们也往前走,不过要注意地方路边可能会有的伏兵。”

“诺!”瞎药点头应诺,随即把韩冈命令传了下去。

十几骑干练的哨探,奔出了队伍,向东急速而去。很快,山谷两侧的山壁上,也重新回响起千军万马前行时隆隆如潮涌的踏步声。

“机宜……”身随中军前行,一直沉默着的智缘叫了韩冈一声,欲言又止。

韩冈知道智缘想问什么,智缘的问题他和瞎药前面都没有提上半句,因为这已是明摆着的事情,不需要浪费口水。他声音低了下来:“如果禹臧花麻不是蠢材的话,他应该已经发现了我们。”

……………………

禹臧花麻不是蠢才,相反地,他能成为禹臧家近几十年来最年轻的一位族长,能从十几名竞争者中脱颖而出,并不仅仅是因为他是前任族长的儿子,更是因为他的眼光和才能压倒了所有的对手。

在通向星罗结城的十几条大道小道上,禹臧花麻都远远的布下了足够多的哨探,放出了数十上百的游骑。有着星罗结部的漏网之鱼为他们指点地理,每一条让他们可能被偷袭的通道,都被牢牢地封锁和监视上。

靠着提前布置下的情报网,禹臧花麻在第一时间,便发现了从小道直奔而来的韩冈和瞎药所率领的队伍。千名蕃骑就算是古渭城中也点不住这个数目,离的最近的部族,也只有与兄长分了家的青唐部的瞎药。

当从赶回来的哨探口中听说这桩紧急军情后,禹臧花麻只骂了野利征两句‘废物’,就立刻放弃了近在眼前的胜利。因为他最清楚他所率领的士兵究竟是什么样的德性。一旦他攻破了城池,为了扫清城中的残部、同时再把冲进城去开始抢劫的部众给收回来,少说也要半天以上的时间。这段空隙,足以让他万劫不复。

抄小道赶来星罗结城的援军到底有这么什么样的意图,这一点并不难猜测。不是没有人向禹臧花麻提议将计就计,设伏将这一支援军给歼灭。但这个提议被他否决了,临时设伏做不到完美的隐藏,如果设伏失败,他就会落入两面甚至三面的包围之中——不论是渭源堡还是星罗结城,城中守军实力并没有消减多少,再加上生力军的瞎药,几方一起合作,足以把与禹臧花麻今次带来的数千大军尽数歼灭。

谨慎和大胆两种截然不同的特质,在禹臧花麻身上融合得很完美。他敢从兰州老家南下武胜军,并通过了大来谷,因为他相信木征有足够的眼光看清楚,如今究竟谁才是他最大的敌人。但他不会去冒险去猜度一个骑兵千人队会有什么样的实力。

原本在星罗结城下高高飘扬的战旗收了起来,低落的士气从行军中的沉默中就能感受得到。劳而无功的结局,让拼杀了数日的吐蕃将士分外难以接受。辛苦了这么久,什么也没能抢回去。禹臧花麻能清楚的感觉得,他在族中的威信正在一步步的下降。

“退到大来谷去……”禹臧花麻正要继续说着,但从被他抛到身后的星罗结城中,传来了威风凛凛的鼓声。

在过去的几天里,什么也没有做的宋军战鼓,终于第一次被敲响。轰然暴起的欢呼声仿佛重重打来的一个巴掌,让从城下狼狈退走的吐蕃士兵们羞愧难当。

星罗结城里的守军竟然敢出城追击!

禹臧花麻回望着依然留在宋人手中的城池,也暗自吃了一惊。恼羞成怒的心情中他依然不动声色,“先去大来谷,确保后路。”

……………………

乔四轻提着皮制缰绳,领着不到百骑的小分队,远远吊着吐蕃人殿后的军队。浓眉下的一双利眼,紧紧盯着前方,西贼用来殿后的军队。

就在半个时辰前,吐蕃人刚刚撤走,星罗结城中仅有的一支骑兵都的都头,便从王舜臣那里听说了他正打着的主意,“什么……追击?”

乔四明白,王舜臣不会用两条腿去追四条腿,那样很容易就会被反咬上一口,所以要动也只能动用得上他这个骑兵都。可苦战之下,乔四现在只想好好睡上一觉,无意去追求更多的功绩。而且以只剩几十骑的小队,去追着百倍于己的敌军,这分明是在找死。

乔四脸色都有些发白,设法想打消掉王舜臣的命令,“侍禁,这可能是西贼的陷阱啊……”

“本来差一步就能破城了,禹臧花麻先把人撤走,再设陷阱做什么?”王舜臣反问。

“……侍禁难道知道禹臧花麻是因何而退?”

如果是因为援军来了才退兵,当然要出城紧追上一阵,只要能扰乱了禹臧家的撤军行动,后面的援军足以把禹臧花麻帐下士卒留下一半来。可若不是援军来了,那究竟会是什么原因让禹臧花麻退走,这一点乔四很想弄得明白。

“就是因为不知道才更是要去看一看。”王舜臣说得像个好奇的小孩子,看到山里洞窟就像钻进去看一看。他其实想得很明白,不论禹臧花麻是为了什么原因而退走,自己这边只要让他的行动难以顺利就足够了,“盯着禹臧家的兵。如果他们不理我们,那就盯紧些。若是他们反过来追杀,那就直接逃远点。不要硬拼,相机行事。给我黏住他!”

乔四抵不过王舜臣的命令——阵前违令,在军法中是立斩不赦的罪名——若是惹得王舜臣翻脸,就真的要试一试军法了。

心怀畏惧,乔四不得不接下命令,在战鼓声的伴奏下,领着手下仅存的七十多骑兵出城追击。而让乔四大为吃惊的,是王舜臣竟然也带着亲兵一起出来。

王舜臣的两张战弓收在弓囊中,两支铁简则插在鞍前,一身鱼鳞细铠是有官品的武臣的标准配置。乔四相信,看到这一副铠甲,说不定能像磁石一般,把已经逐步离开的西贼再吸引回来。

百丈的距离看似遥远,其实对两支骑兵队伍来说是近在咫尺,只要西贼有心,把马头调回,几个呼吸间就能冲到面前。

乔四紧紧握着手中的长枪,他学不来王舜臣那般轻松的神态,咬着牙身子绷直:

‘大不了就拼了这条命吧!’

