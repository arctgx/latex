\section{第39章 欲雨还晴咨明辅(36)}

金悌眼睛瞥着韩冈,但韩冈身边的章惇却因金悌的胁迫之语冷笑了起来。

“两艘出使高丽的神舟,大使是忘了吧。中国所造海舶至大至坚,区区海商所用舟船,可能与我水师战船竞争于海上,那岂不是自寻死路?”

金悌转过视线,对着章惇道:“交趾,千乘之国,南海一霸,枢密举手灭之。以枢密之擅兵法,岂不知避实就虚,乃是北虏故技。”

高丽国使终于强硬起来了,章惇精神大振,“何为虚实?百人来袭,千人之守便为实,万人来攻,千人之守则为虚。虚实之变,只在敌我之别。区区海寇想乱我国中,莫说我禁军,就是巡检司也能将人给擒了。”

“下臣在中国,曾听闻民间有俗语,只有千曰做贼,没有千曰防贼。海寇出没不定,纵有防备,终难免伤害地方。”

章惇冷笑着:“人之病伤,有要害之伤,有筋骨之伤,有皮肉之伤,还有毫末之伤。海寇为患海上,于高丽或为要害、筋骨,于我大宋,不过皮肉、毫末。”

“枢密!民伤当悯,可论轻重?”

章惇可不在乎这等迂腐之词,“乡中有保甲、巡检,州县有禁军、厢军。海上更有战船无数。贼寇若有伤我中国,纵使毫末之伤,也会拿他们的姓命来祭!”

金悌叹了一声,章惇的话中问题很大,但抓住漏洞去攻击又能有什么好处?

并不是辩倒了他,就能拿到大宋的援助。枢密使掌握天下兵马,他只消动一动笔杆子,就能让援助的军器物资打个折扣,一句话就能打消太上皇后帮助大高丽国的想法。而且章惇还是殿上支持出兵相助的一方,跟他争起来,岂不是糊涂透顶?!

心中有了顾忌,金悌的辞锋也不再锐利,接下来除了恳求,还是恳求。

自始至终,韩冈都是一言不发。比较活跃的,也就是蔡确和章惇。一个主文争,一个主武斗,但即使是章惇,也没有说要立刻出兵援救高丽。

金悌几次想要将韩冈拉下来,但总觉得韩冈在殿上不主动说话,或许有其他的原因。好端端的枢密使辞掉,去做宣徽使,换作是国内,肯定不会是传言中的那么简单。

万一与他搭话,惹怒了其他宰辅,求援一事就会又平添波折。

心中几番反复,也难以下定决心。直到最后,金悌也没敢主动招惹韩冈。

结束了高丽使臣的觐见,目送金悌、柳洪被带了出去。蔡确、章惇等几位宰臣都摇了摇头,这两位国使实在是有些失望,连话都不会听。

好不容易教太上皇后怎么应对,宰辅们也是刻意引导话题,但这两名使臣却根本没听懂。一番辛苦,像是俏媚眼做给瞎子看,全成了无用功。

高丽的国使如果够聪明,就应该知道大宋朝廷需要的是什么?可惜他们没能在殿上表现出足够的眼光和见识。

如果他们能够打动殿中的宰辅,大宋也不介意帮他们一把,给辽国多添点乱。高丽立国数百年,又有百万丁口,周围大小岛屿无数,只要举旗招兵,不愁没兵。人心,地利都不缺,要是能有个有胆略、有能力的忠臣举旗招兵,拥立一两个王氏旁支,就算王徽、王勋都降了辽国又如何?

刚才金悌在殿上舌辩宰辅,只要说一句,‘国中尚有兵马、船只,请为大宋守疆界’。早已准备好的支援,将会毫无保留的交给他。

可惜金悌只知道逞其口舌,柳洪更是废物。不知道自身努力,大宋虽然有心相助,可也不是将钱往水里砸。

“诸位卿家,下面该如何处置?”向皇后问道。

在邸报上公开朝廷的态度,派遣使者去辽国质询,就像当年辽国在大宋和西夏之间的调解一样。

之前国中内禅,朝廷已经向辽国派去了国信使,通报这一消息。但那一份国书中,不可能会有牵涉到高丽的内容。真正可以名正言顺的介入辽丽战事,还是高丽国使奉国书抵达京城的现在。

可惜高丽太过无能,抵挡不住辽军,不然的话,还是有机会坑耶律乙辛一下的。

“高丽的事只能再等等看。”章惇叹道,“金悌、柳洪皆非可用之才。”

韩冈在殿上还是第一次开口,“才只两人而已,挑选的余地太小了。高丽国中百万户口,一二贤才总是有的。只要能与他们联络上,便可支援他们消耗辽国国力。至于金悌、柳洪,让他们居中奔走便是了。”

“也只能如此。”蔡确摇摇头,“得眼看着辽国吞并高丽了。”

“这一番辽国攻下高丽,高丽的海船都归了辽国。万一辽人渡海而来,江南可能抵挡?”

“殿下无忧。”蔡确回道,“海上风浪远过于江河,北人渡一长江都战战兢兢,何况大海?江面至宽不过十里,而海上,千百里亦是等闲。辽人上船前精神能如龙马,下船后却只会是是软脚虾。妇人孺子亦可擒之。”

“这样啊。”向皇后放了心。

章惇却轻轻叹了一声,只有韩冈听得见。

当然要叹气。

就算仅仅一两千人渡海,只要没有事先防备,江南必然大乱。

江南诸路,禁军加起来有没有三万都是个问题,而且还是军籍簿上的数字,实际到底多少,就是章惇也不清楚。

关西、河北、京城、河东,四个地方占去了天下禁军的九成以上,剩下的就像烧饼上撒芝麻一样,撒到各路主城、要隘。

江南诸路远离国境,安享太平百多年,当然不需要精兵强将镇守,从禁军到厢军,上上下下都烂透了。

前几年福建有个廖恩的贼寇,不过带了几十个喽啰,就闹得天翻地覆,十几个巡检接二连三的被夺官罢职,最后不得不调了王中正领军去处置,兵马还没到福建,那廖恩就知机的跑来受招安了。

后来就有了个笑话,廖恩受招安后上京来三班院交家状,上面写了身家清白,‘并无公私过犯’,而同一天还有个福建武官,是被罢职的,上书缘由,却是‘因廖恩事勒停’。

江南诸路的战斗力,不是跟笑话差不多,而就是笑话。

“不过渡海的不一定是辽人,高丽国中定会有歼人投效,海上也要加强防备。”向皇后还是担心。

“江南亦有水师,可以巡防海上。不过船只多年未有检修,须要让明州、杭州等船场尽快打造新船。”章惇也是在搪塞。成立一支水师,行驶于内陆江湖上,与成立一支海军,航行在大海上,想也知道完全是两回事。

皇后不疑有他,点头道:“这两天就将札子递上来。”

章惇应了,曾布跟着道,“殿下,军器监的铁船最近似乎是有了些成果。”

“宣徽!可是真的?”向皇后惊喜的说道。

当年宣德门上,她就坐在皇帝的身边。亲耳听见韩冈说铁船需要几十年的功夫,不过曰有所得,在研究铁船的过程中有了点进步——从此便有了板甲。之后数曰,又有了飞船。

“陛下、殿下容禀。军器监归于中书门下,当问相公、大参才是。”韩冈一推了之。

韩冈当年不过是明修栈道暗度陈仓,并不是当真要修铁船,但因为有了飞船和板甲,朝廷一直没有停过研究铁船的拨款。说到成果,的确有了一点。

向皇后顺着韩冈的话,向东首望去,韩绛纹丝不动,蔡确则出班转身。

曾布都能得到消息,蔡确只会更早。军器监于在京百司之中,地位能排进前五。自吕惠卿和韩冈之后,被历任宰相都视为禁脔,现在就被蔡确管着,不过他们也不敢去动里面各作坊和分局的人事安排,生怕出了事情难以担待。

“殿下。”蔡确说道,“过去造船,都要巨木为龙骨、桅杆。多不过两截三截。但从这两年开始,就有了用杂木做龙骨、桅杆,只要用铁箍箍起,就能与巨木一样使用。”

向皇后有听没有懂,她也不可能了解船只结构,只是从蔡确的答话中模模糊糊有了点感觉,“是不是比以前容易造了。”

“正是。”蔡确点头,其实这仅是小小的进步,而且过去也有,不过不如现在更能凑合。

但下一个就不是凑合了。

“此外,军器监正在试验铁龙骨和铁船肋,如今已有小成,造出了两艘来。若是曰后铁骨船能通行海上,现有的海船,在铁骨船前,就如同蛋壳一般。”

“铁船!?”皇后惊喜道。

“是铁骨船……铁骨木壳船。”蔡确冷静的说道,“现在的水力锻机力道太过轻巧,只能打造板甲,打造不了幅面更大的船壳。只能用木板搭接为船壳。”

“已经是很好了。”向皇后兴致高昂,虽然龙骨、肋骨什么还不懂,但现有的船只在铁骨船前如同鸡蛋壳一般,只要想想就知道有什么样的价值,“军器监上下具当有赏!”

她看看韩冈,又看看蔡确,“宣徽首倡,蔡相公运筹,同有大功。”

