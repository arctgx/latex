\section{第17章 观婿黄榜下(上)}

三月初。

正好是春来簪花的时候。

仿佛是一夜之间,大街小巷中的行人,头上无不多了一朵或艳红、或粉白,或花开争艳,或含苞欲放的鲜花。在发髻上、在帽子上,随着步子颤颤巍巍。

东京人喜欢簪花,到了仲春之日,不论男女都会在发鬓或帽子上,插上一朵应时的花卉。现在是山茶,再过半月,则是牡丹花在头上绽放的时节。也有绢花,以金丝缠绕,饰以碎珠,比起真花来多了两分贵气,只是火焰一般红艳的绢花插在一个满脸皱纹白发苍苍的老家伙的帽子上,不免让韩冈看得毛骨悚然。

先是庆幸着秦州没有这样夸张的风俗,又想到自己到也少不了要头戴绢花,在御街上招摇而过,韩冈多少就有点不寒而栗的感觉。越发的体会到王安石和司马光的心情来。这两个死对头都是不喜欢簪花,王安石从来不戴花。而司马光中了进士后,也不想簪花,只是被人劝说是天子所赐,所以不便推辞,勉强戴上。

从一朵朵插在头上的鲜花上收回视线。身边的同伴正僵硬骑在马上,挣扎、期待、彷徨,各色表情交替在脸上浮现,让人目不暇给。

慕容武患得患失的表现,让韩冈暗自摇头。

他闲来无事,陪着慕容武来看榜,这事先也是约过的。

说起来,曾经考中过明经的慕容武,他的才学水准并不算很高,如果是考得是诗赋,必然中不了,所以当年才选的明经。今次进士科改考经义策论,方才来碰一碰运气。

但中奖的可能性只有一两成,欢迎下次再来的几率则占了百分之八九十。已经确定了自己成绩的韩冈,陪着慕容武来看一看结果,只能算是尽尽人事而已。既然是师兄弟,当然要多加亲近。至于嫉妒什么的,韩冈却不会在意。

韩冈和慕容武向着南薰门内的国子监行去,越靠近国子监,街上的行人就越多。到了国子监外的礼部试放榜处时,那里早已是人山人海。韩冈听说过,历年礼部试放榜,有三更天开始,就跑过来坐守的士子。人数还不少,都想第一个看到自己的名字。五千名士子引颈而望,加上更多的准备来捉女婿的官员商人和富户,国子监门前的二十多步宽的大街,被车马行人堵得水泄不通。

“这下怎么进去?!”慕容武有些发楞,就算是上元灯会,似乎也没有这般拥挤的人群。比起前日应考时,堵在门前的人数犹要多上一两倍。

“官人,这里让小人来!”

跟在韩冈和慕容武身后,两名膀大腰圆的壮汉站了出来。

这是王韶知道韩冈要去陪人看榜后,特意下令让他们跟着韩冈一起去。皆是从熙河军中被王韶招揽下来,都有把子气力,从人群中挤过,就像战车碾过草原,风行草偃,挡在前面的,无论是士子还是其他人等,全都被硬生生的挤开。

有人被挤到一边后,转身就要怒斥,但一看到两名壮汉身上穿的红色号衣,便立刻住了嘴——宰执家的仆人,尤其是拿着一份官家俸禄的元随,都是有规定制服的。在宰执们上朝事,被这些身穿红衣的元随护卫着,国之重鼎的气派便出来了。

下了马,一路顺利的来到黄榜下。五大张黄色的榜单贴在墙上,密密麻麻的名字、籍贯,占据了大部分的纸面空间。

首先映入眼帘的是礼部试头名——也就是省元的名讳——邵刚。

韩冈对这个名字印象不深。不过去年腊月见过面的余中排在第三。

至于韩冈本人,早就知道了结果,排在了第一百五十七位,在礼部试取中的四百零八人中,排在中前部的位置上。在榜单上瞥了一眼自己的名字之后,心神只是微动,就帮着慕容武找起了他的名字。

至于慕容武,他早已经从头开始,在四百零八人中,寻找着自己名字。只是他越看脸色越白,一个个姓名过去,都是不见慕容二字。

心慌意乱之中,突然衣袖一重,韩冈一扯他,“中了。”

“我知道玉昆你中了!”慕容武不快的冲了一句,没理会韩冈。韩冈得中的消息,慕容武来找韩冈时就听说了,方才也看到了韩冈的名字,可现在是要找自己名字!

“我说思文兄你中了!”韩冈提声说着。

“玉昆,别戏弄愚兄了,根本就没看到啊。”慕容武的视线黏在了榜单上,却还是没找到自己的姓名。

韩冈无奈的一指前方,提点着:“从后面开始看。”

最后一页榜单,倒数第一的姓孙,不过不叫孙山,而是叫做孙中。至于倒数第二个,就是慕容武。

简简单单但三个字,慕容武看了一遍,两遍,揉了揉眼睛之后,又看了第三遍。

没错,就是‘慕容武’三个字。

“啊!”他一声大叫,“当真中了!”

这一声喝,顿时惊动了四周十丈之内的闲杂人等。如同一块鲜肉,抛进了狼群,几十人一下一拥而上。

韩冈见势不妙,疾退数步,任由成了众矢之的的慕容武被淹没在人海中。

慕容武不过三十出头,有着北方人的高大身材,加上为官多年,看起来气度也不差。这样的进士在四百人中也不多见。几十双饥渴的眼神盯着慕容武,仿佛久旷之身的寡妇看着赤裸着身子的精壮汉子。

一个仆役抢先喊了起来:“小人主人家的二小娘子,年方二八,貌美如花,温柔贤淑,德才兼备,正要招个可人意的郎君!不知官人意下如何?”

此话一出,周围一起投以鄙视的目光,这时候说这些废话作甚。一个富商模样的胖子将手一张,五根粗短的手指晾在慕容武面前:“我家女儿有嫁妆五千贯!”

同样鄙视的目光改向那名富商投去。捉女婿,进士是先决条件。在这之后,就要看年岁和长相了。两样都不行,陪嫁那就是千贯的最低价。再往上,五千贯则是平均数,提供给普通水准的进士。至于慕容武这样一看就是年轻有为的官人,可是五千贯就能拿得下?!

“我家女儿有八千贯陪嫁!”一名瘦削的乡绅喊着价码。

另一名腰缠金玉、最为贵重的菱花龟背竹纹蜀锦都穿在身上的商人,也掺了进来,“八千贯,在东明县还有五十亩水浇地的脂粉田!”

“一万贯,在陈留有个庄子,十五顷地!”

喊出最高价的士绅看起来更加有气派。穿着看似普通,但腰间的黑带其实是猪婆龙皮,身上的青袍更是贡绢。只要稍有见识,就知道这是一户跟皇亲脱不了干系的人家。

在喊价的过程中,慕容武被拉拉扯扯,头上的帽子也掉了。见着势头不妙,连忙扯着嗓子连声叫道,“家有糟糠!家有糟糠!”

此话一出口,人群刹那间就静了下来。接着便是卷堂大散,刚才还争得热火朝天的人们,这时各自摇头四散开去。

方才喊出一万贯的士绅正好经过韩冈身边,方才也是看着他跟慕容武站在一起,不免多问了一句,“不知官人可考中了进士?”

韩冈反问:“你看我像中进士的样子吗?”

士绅从头到脚打量了韩冈一番,相貌和年纪都不差,只是宁宁定定的表情,的确不似考中进士后应有的样子。摇了摇头,便弃了韩冈而去。

“玉昆,何苦戏弄人。”对于方才韩冈站干岸的行为有忿于心,慕容武质问着他,只想着让韩冈也来尝一尝差点被人挤死的感觉。

“小弟说谎了吗?”韩冈反问,“谁让他不会看人。”

“噫,中了!中了!”

一声尖叫打断了韩冈和慕容武的对话。一个花白胡子、差不多有五十岁的老贡生拍着手,大叫了两声,然后便咕咚一声栽倒在地上。

这副场景,东京人已是见怪不怪。熬了几十年,终于熬出一个进士,疯了的贡生都是有的。

哗的一声,一下涌上来一群人。泼水的泼水,打扇的打扇,还有一个五大三粗的壮汉,听了身前主人的吩咐,往掌心吐了两口唾沫,搓了一搓,就对着老贡生的人中死命一掐。

对阵下药,老贡生随即悠悠醒来。

壮汉的主人走上前,是个四十多岁的商人。他在老贡生身边蹲下:“官人,可是中了?”

“三百零四位的范庸就是学生。”名次排行,老贡生是至死不忘,就算是刚从昏迷中醒来,照样一口报出。

“是否婚配?”那商人又立刻追问了一句。此话一出,周围顿时鼓噪起来。有些人想拥上前。但却被跟着商人的几个壮得像头牛的伴当,死死的拦住。

“没有。”范庸摇头哀叹,老泪纵横,“求学四十年,无所成就。父母不收,昆弟弃我,哪还有人愿与我结亲。”

“没有就好!”商人更不多话,一招手,几个壮汉立刻回头来,横拖竖拽的将范庸架进了马车中,转眼就冲出了人群。来去如风,这绑架的手段显然是行家里手。

“不愧是榜下捉婿。”见着马车载着范庸转瞬去远,韩冈啧啧称叹。

