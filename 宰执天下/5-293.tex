\section{第30章 随阳雁飞各西东(23)}

“大帅,当真不要紧?!”

种建中不是质问,而是要配合种谔引出他的解释。

种谔满意的看了侄子一眼,道:“冬天水枯,有没有水还是两说。即便有水,也淹不到这里。去年夏天水才到哪边?现在天寒地冻,河水一旦流到平原上,很快就会上冻,现在掘了河堤,明天就能给冻上,照样能跑马,怕什么?”

种谔的话,立刻被传了出去,让营中士兵们稍稍安定下来。

种建中暗中松了一口气,但心情依然阴沉。

事前不是没考虑过辽人掘堤的可能,否则也不会远远的找个高地扎营。可是现在当真发现辽人准备挖开河堤,终究还是不会有什么好心情。

种谔照旧拿着酒碗绕行在各堆篝火旁,只是营中的气氛已远不如之前热烈。

慢慢的走过一堆堆篝火,种谔问着种建中:“十九,你觉得辽人什么时候会决堤放水?”

“官军开始攻城……”种建中想了想,“或是去抢夺堤坝的时候。”

决堤放水,总要选个好时机,能将宋军一起淹进来换一个大捷才算不亏本,不可能听到一点风声就开始吭哧吭哧的掘河堤。

种谔点点头,“就是这个道理。”

“不过辽人应该发现我们知道他们准备决堤的事了。”这样的情况下,只要有一点风吹草动,辽人就会立刻开始决堤。种建中低声问种谔,“五叔,怎么办?”

“明天绕个道吧,先往西南去。上了堤后再往西北走。追在耶律余里背后,那边怎么也不可能被淹到。”

种谔手上也有几名对兴灵地理极为熟悉的向导,有西夏国灭后投靠来的,也有在溥乐城下被耶律余里给抛弃的,还有过去以商人的身份来过兴灵侦查的间谍。兴灵的地理,种谔大体上是了解的。

紧追耶律余里,就能赶上他和党项人的决战。就算出了些意外——也不用从灵州川的来路往回走,那可是几百里没有半点人烟——改从青铜峡回去,甚至可以就地征粮。

不过种谔现在可没有为失败考虑后路的打算,除夕的夜空下,他放声笑道:“我还想做个渔翁呢。”

……………………

一口气跑回来了六七百里,耶律余里知道他麾下士兵已经快要支持不住了,但他更知道,最危险的时候已经过去了。

迁来兴灵的各家部族数万帐,虽说这一回带了不少士兵南下,但实际上不过是三丁才出一兵,剩下的还有许多丁壮。给党项人打了一个措手不及是不假,也的确让党项人毁了不少族帐,可安化州——也就是兴庆府——还是及时将州中的子民给**了起来,招入城中固守待援。直到耶律余里回援为止,安化州依旧安然无恙。

党项人就在二十里外。如今大军在外,重兵在内,他们几乎是被困住了。只要里外合围,西夏余孽最后的一点反扑,也会化为泡影。

“先好好歇息两天。”就在一座刚刚被党项人攻破的寨堡中,耶律余里高声的发号施令,“等恢复了气力,就去见一见仁多零丁和叶孛麻!让他们见识一下我大辽男儿的豪勇!!”

大昌嗣高声与众将一同气冲斗牛的应和着,但从帐中出来,望了望看不到月亮的夜空,他低声的问其父大公鼎:“也不知西平府【灵州】那边水淹到哪里了?”

“足够困住种谔就行。拖上三五日,就够我们杀光这群党项人了!”大公鼎语调和风一样的冷。西夏的国都可是他这一族的属地,被党项人攻打,也不知死了多少族人,更不知损失了多少牲畜。

河渠中冰层很厚,大公鼎也没把握掘开刚刚修复的那段河堤能放出多少水来。但今年修补堤坝时,大公鼎可是亲眼看见河床比堤外的地面要高,只要冰层下还有水,那是肯定能放出来,也就是多少的问题而已。

大昌嗣犹疑的问道:“可种谔都追上来了,鸣沙城的赵隆会不会也跟着会不会……”

“不论来与不来,我们都必须先赢过面前的贼人再说。”大公鼎望着夜空,声音冷澈,“只有一,才有二。”

……………………

同样沉黯的天空下,仁多零丁同样望着夜空。

听见身后的脚步声,他头也不回的说道:“今天可是除夕,这算不算守岁?”

西夏用的是宋人的历法,新年的时候,照样要团圆守岁,与汉人一般无二。但叶孛麻却没有一点好心情,“已经是孤注一掷了,还过什么年?”

仁多零丁转过身来,轻笑道:“还在担心?”

“能不担心吗?”叶孛麻反问。

突破青铜峡口的一开始,打得很顺利。辽人诸部分得很散,完全没有防备,无法抵抗并力北向的大军。不过等辽人反应过来后,抵抗一下就激烈起来了。兴庆府到了现在还没拿下。确切的说,仁多零丁根本就没有打算去硬攻兴庆府,而是试探了一下后,就开始坐等辽军回师。

耶律余里回来得狼狈,六七百里都没好生歇息,士气低落,马力也消耗极大。不过别看现在是师老兵疲,但只要给他们歇息上几天,回过气来,那就又是生龙活虎的一万精锐了。

仁多零丁心平气和,在生死决战之前,却看不见半点惶惑,“可知耶律余里驻扎的位置?”

叶孛麻停了一阵,才叹了一声,“……当然知道。”

“哪还有什么好担心的?”仁多零丁笑问道,“不是如事前所料吗?”

……………………

吕惠卿正在夏州。

丰盛却粗犷的年夜饭并不合他的胃口,只是吃了几块烤肉,喝了点酒,现任的陕西宣抚使便回到了后厅歇了下来。

俯身看着铺在桌面上的巨型沙盘,吕惠卿的心情跟夜色一般深沉。

怎么办?摆在吕惠卿面前的,是两难的境地。

是为种谔独走而背书?还是上书承认自己没能控制住这条疯狗?

必须要做出一个选择——谁让种谔都追到了兴灵去了?已经不可能追回来了。

当听说种谔领兵北上,吕惠卿砍人的心思都有了。如果种谔现在就在他的面前,吕惠卿是绝不会犹豫的。

或许在普通的文臣眼中,这完全是个博取功名的机会。将愤怒的耶律乙辛交给东京城中的天子、皇后和宰辅们去应付,自己只要享受夺占兴灵的功劳就够了。

但吕惠卿不能这么做。既然他的目标是宰相,那么他就必须站在宰相的视角去考虑问题。便宜行事的权力,也代表着相应的责任。

双手撑在沙盘上,吕惠卿默默看着沙盘上的荒漠与高山。窗外的鞭炮声充耳不闻。

就在这除夕之夜,他必须做出一个选择!

……………………

吕大临和游酢推门进来时,谢良佐正坐在桌边。

“怎么还没睡?”

游酢问道。方才席上,谢良佐可是以不胜酒力而先离席的。

谢良佐抬起头:“睡不着啊。”

“所以就占筮卜问吉凶?”吕大临看看摊在桌面上的蓍草,不以为然的摇了摇头,“真要卜筮,还不如烧乌龟壳,最近不是正时兴吗?”

“也是闲来无事。”谢良佐赧然说道。

吕大临皱眉道:“邵康节旧日欲将术数之学传授于伯淳先生,而先生不受。显道欲从康节之学?”

“不是不受,先生说欲通术数,非二十年之功不可,哪得如许时间?!”游酢更正道,“小弟曾经听正叔先生说起过,那是熙宁初年的事了。”

“熙宁初年,伯淳先生年齿几何?‘加我数年,五十以学易,可以无大过矣’。先生之心在圣人之易,岂在术数?”

岁末之时,程颢程颐回了洛阳。十几名弟子也跟着一同到了洛阳。现在都借住在洛阳城中的一间小庙中,离二程的府上很近。除夕之夜,聚在一起吃了顿年夜饭。等过了年,他们就准备跟程颢一同上京。

谢良佐是其中之一。就要去京城了,但他总觉得前方是一片混沌。忍不住就拿了蓍草想占上一卦,问一问吉凶。

不过卜筮之术,一向不被程门弟子看重,甚至轻视,听见吕大临如此说,谢良佐抬手就想将已经占出的卦象给拂了去。

“等等!”游酢抢上一步,看着桌上蓍草组成的卦象,脸色就是一变,下兑上巽,“这不是中孚卦?!泽上有风。君子以议狱缓死。这卦象可不好!”

谢良佐手停了,轻叹道:“是‘翰音登于天’啊……”

游酢脸色更难看了三分。

中孚卦的上九一条——‘翰音登于天’,卦则‘贞凶’,象曰‘何可长也?’说起来,程颢为太子师,说书资善堂,岂不是字面上的‘翰音登于天?’注疏根本就不用提了。

“中孚又如何?不过是‘志未变也’。利涉大川,利贞。”吕大临嗤之以鼻,“即云‘有它不燕’,一心一意也就够了。先圣有云‘人而无恒,不可以作巫医。’但若是有恒,又何须做巫医?”

挥袖拂乱了桌上的蓍草,吕大临决然道:“不占而已矣!”

……………………

王安石刚刚睡下,守夜什么的他根本不在意。如今就是按时睡按时起。虽然对西北战局担心,不过就算是辽人大举入侵,王安石也不觉得能赢得了国势正盛的大宋。唯一的期盼,天子要是能康复就好了。

蔡确与妻妾儿女团团坐着,已经是宰相之尊,他没有什么不满意了。剩下的,就是如何长保权位。看刑恕传来的话,洛阳的旧党已经是死老虎,一个赛一个的老实,估计是皇后把他们给吓到了。真正的对手,可就是每天抬头就能见到的同僚。

章敦悠闲的喝着酒。西北的战事并没有打扰到他的兴致,相反地,倒是让他心情很好。做了宣抚使后,吕惠卿不论是失败还是成功,都很难再继续担任枢密使了,明年的西府自然是自家说了算。至于辽人,他根本就不担心,不就是打上一仗吗,章敦可不觉得会输!

曾布新近抵京,尚未拿到他的官邸。正在城南驿中,独坐于灯下,看着奏章、札子和旧档的副本。郊祀后的两个月,内外动荡,朝局国政的变化,让外来者摸不着头脑。曾布自知必须要尽快掌握朝堂内外的动向,他的同列可都是吃人不吐骨头的大虫,半点也疏忽不得。

苏颂看着星空,他托人新制的望远镜就快要打造好了,过些日子就能送到自己手上,到时候,便又能沉浸在无穷无尽的星海之中。不过明年最重要的还是《自然》,韩冈想要推广气学,但苏颂最想做的,是利用这本期刊与同好交流。

韩绛、张璪、薛向,各有各的心思,却同在期盼新的一年。

可除夕之夜的深宫中清冷如冰。

病重垂危的赵顼完全没有恢复的迹象,自然也无人有心过一过新年。向皇后带着众嫔妃和一对儿女,向病榻上的皇帝祝过酒,便将他们都送回各自的住处,只有她一人留了下来。

夜色渐深沉,无心节庆的向皇后也睡了过去。

福宁殿内的杨戬正是当值,半睡半醒的守在床榻边。睁开一阵,又闭上一阵,抓紧一切时间休息。但他再一次闭上眼睛,就突然睁开了,方才他似乎看到了些什么。

并不是错觉,杨戬揉了揉眼睛,专注地盯着赵顼的手指。片刻之后,他就瞪大了眼睛,“官……官家能动了!!”

他一下跳了起来,放声大喊,“官家能动了!官家能动了!!”

向皇后一下惊醒,只稍稍迷糊,就扑到床榻边,看着突然之间就恢复少许的皇帝,她激动地难以自抑:“快宣韩学士!快宣御医!”

