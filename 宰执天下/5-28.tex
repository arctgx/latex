\section{第四章 惊云纷纷掠短篷(二)}

【第二更】

韩冈打马而过,却也不往灯山上多瞧一眼,正要横穿大街,从侧面冲来一名骑手,急匆匆的口中喊着让路,挥着皮鞭,将挡在马前的行人全都驱赶到一边。

韩冈轻提马缰,让了那人过去。

如同一阵风卷过,韩冈还没有看清那名骑手的相貌,那人就只剩下一个背影了。

但韩冈身边的韩信,分明就认识他:“那不是驸马王都尉家的人吗?怎么从宋门那边过来?”

“公主府中的人?”韩冈低声问道。方才那名骑手没走多远,又被一群人给挡住——初更天,街上一向不缺游人——正在几十步外,匆匆忙忙的挥着皮鞭赶人。

“年前朝会的时候,还在宣德门外见到过的。”韩信说道,“王都尉家的人,京中朝官都不会与他多牵扯,而他在宗室那边也不受待见。只能孤伶伶的站在一边,所以小人印象很深。”

“这么说你们就很受待见了?”韩冈笑着问,心中倒是很有几分惊讶,抵达京城还没有几天,怎么韩信就打听到了这些秘闻

韩信摸着头,嘿嘿笑道:“俺们也是狐假虎威,若没有龙图,京城里面的人,谁会正眼看俺们这群从关西来的缺胳膊断腿的赤佬?”

“你与其他人家的元随交好,这是好事。但要注意,不要一幅小家子气,也不要太过大方,平平实实的与人交往。不要抹不开情面,被人拉着做些不该做的事。”

“龙图放心,俺们绝不敢在外面丢龙图的脸。”韩信猛点头,又补充道,“在京城,就是要多交朋友,这样才能吃得开。”

韩信是个四海的性子,韩冈日常里了解到了。而行事稳重,韩冈也一样了解。对于韩信的为人处事,韩冈还是很放心的,“只要你们能时时谨记,这样我就放心了。”

韩冈叮嘱了一句,抬头望,王家的家丁已经不见踪影。走得还真是挺匆忙的。

“从宋门那边过来,该不会是王都尉没带足钱钞,在观音院或是第一甜水巷的婊子那里,被扣下来了?”韩信带着恶意的猜测着,其他几名元随,顿时都笑了起来。

韩冈轻轻地摇摇头。王诜在宗室中不受待见的理由,他也知道,京城里面传得很广,不过是宠妾欺妻四个字而已。蜀国公主性格据说很好,在宗室中很受尊重,王诜待她刻薄,自是不会受到待见。

而朝官们与驸马都尉的交往基本上也很少,瓜田李下的嫌疑总要避着。不过他跟苏轼关系不恶。

王诜恃才傲物,目无余子,诗画虽是有名,但因为总在烟花里行,便与苏轼唱和往来,交情倒是很好……韩冈身子猛然间震了一下,脑中灵光一闪,该不会是去通风报信的吧?

苏辙正在南京应天府为官,从东京到南京,一来一回,用快马的话,正好一天一夜。苏轼要被提审入京的消息也是昨天快入夜的时候宫中传出来的,王诜作为驸马都尉,耳目一向灵通,早一步派人去通知南京的苏辙,让他赶紧与苏轼联络,当是不在话下。

御史台派去捉苏轼的人刚刚南下,只要苏辙的人走得急,纵使不能借用朝廷的驿站,但早一步通知到苏轼应当还是有很大的机会。

可早一步又能如何?最多也不过是可以烧掉被弹劾的罪证。

回到家中,王旖和周南素心云娘她们正坐在一起说话,见到韩冈终于回府,连忙起身行礼。

“官人喝了酒?”严素心随口说着,接着就给韩冈端上了醒酒汤,有点烫,却不影响入口,温度恰到好处。

“正在说什么呢?”韩冈坐下来随意的问道。

王旖有几分好奇:“听说朝廷派了人去湖州,要捉苏轼回京审问,官人是不是真的?”

“哪里听来的?”韩冈反问。

闺阁中的消息传递,总比宫中慢半拍,可这一次,王旖的时间并不比朝堂上要慢。韩冈心中有几分疑惑。

王旖看出了丈夫心中的疑惑,连忙解释道:“方才六婶婶来了,正好提及此事,还说了苏轼不少好话。”笑了笑,“她也是喜欢大苏的诗词文章。”

王安礼在家中排行老六,如今在京城中应付差事。兼了好几个职位,从开封府判官,到权发遣提举三司帐司勾院磨勘司、此外还有直舍人院和同修起居注的差事。身兼四差遣,王安礼每天总是忙得跟陀螺一般转个不停,与韩冈的清闲形成了鲜明的对比。

“还有吴判官方才派家人来送信,说是请官人闲暇时,过府一叙。”

吴衍与苏轼是有些交情的,韩冈清楚这一次邀请,多半是为了苏轼。

韩冈也想救苏轼,但眼下还没有到那个时候,下台狱的官员什么时候少过,等苏轼进了台狱之后,再设法去保他的性命。

整件事事情闹得大了,就成了党同伐异的工具。韩冈觉得以当今天子的心意。应当不打算杀苏轼,

毕竟不是什么正事,只是出口气罢了。

如果只是敲打一下苏轼,没有什么关系,但弄成了文字狱……韩冈眯起眼睛想了想。其实跟他也没什么关系,从来不作诗作文,一干公文、奏折,都是写熟手的,没什么破绽可以利用。对韩冈的政敌来说,与其依靠奏章来构陷自己,还不如在其他地方搜集罪证,那样花得精力还少一点。

不过这个头不能开,一旦开了先河,日后就不知道轮到谁倒霉了。朝堂上的都是聪明人,谁也不愿落到那样的结果。

“要处置苏轼,可以用别的罪名定罪,但文字入罪是万万不可。除了为夫,世上的文人哪个不写文章诗词?”

“官人你打算怎么做?”王旖急匆匆的问道。她很想知道自家的丈夫是准备怎么营救当世闻名的才子。,

“找个机会韩冈会去跟天子说的,苏轼决不能以诗文入罪。但如果是其他的罪名,我就没办法了。”韩冈对身边的弟子们说,“不过眼下还来得及。就是看在章子厚的面子上也得救他一救。南娘,过去的事就让他过去好了,毕竟结果是好的,从结果上说,他也是帮了个忙。”

周南柔顺不过的说着:“一切都随官人的意。奴奴只要跟在官人身边就行了。”

“三哥哥要救苏子瞻?”韩云娘眼睛一亮,道:“苏子瞻名气这么大,诗文又做得好。三哥哥好好劝一劝天子,文曲星一般的人,不能杀的。”

韩冈抿着嘴,笑着摇头:“这话可不能对天子说,说了可是把苏轼往死里逼。文足以饰非,辞足以惑众。天子正恨他名声大呢!”

王旖、素心和周南都是先迷惑而后恍然,只是前后有别,只有韩云娘疑惑的歪着头:“三哥哥,那该怎么说?”

“天子重后世之名,往这方面说就行了。”见韩云娘还是一脸疑惑,韩冈就明说了,“苏轼自负才高而不得进用,腹中或有怨怼。但以言辞杀一儒士,不知后世会怎么看陛下?”

韩冈端起茶杯喝了一口,有几分自得:“这样才能救苏轼!”

“这算是揣摩上意吧?”王旖突然笑着问道。

“咳!”韩冈呛了一口水,“人家养猫,不顺着毛捋,难道还逆着来吗?!该直言谏争的时候就直言谏争,该婉转曲言的时候就婉转曲言。为政当以结果为上,那等为邀清名,故意让天子难堪的官员,为夫可没兴趣学他们!”

关于苏轼一案,韩冈本是打算先看看再说,天子也许只是要出口闲气罢了。

三条腿的蛤蟆不好找,说酸话的措大哪里都是,天子应该习惯了才是。都被恶心这么多年了,多苏轼不多,少苏轼不少,赵顼只是一时心头不痛快。

但现在看御史台的一封封弹章,是打算将苏轼的罪名钉死在怨望二字之上。

雷霆雨露皆是天恩,腹诽倒也罢了,说出来可就是自寻苦头了。尤其苏轼的名声很大,新作一出,天下传唱,讪谤之言也便一同流布天下。这么一来,一贯重视名声的赵顼,也不可能不怒火中烧。韩冈估摸着,苏轼这一次不死也要脱层皮。

文章憎命达,苏轼再一次受责之后,文才也许还能更上一层楼。韩冈记得当年他还想让章惇传一句文王厄而演周易的话,只是那时候觉得有些太过幸灾乐祸的味道,故而就没说出口。不过从结果上看,这个道理是没有错的,出外数载之后,苏轼的诗文水平的确是大有长进。就如李白、杜甫,如果一辈子的高官显宦做着,绝不会有如今的地位。

当然想归想,做归做,苏轼能不能在受责之后,文才一番磨砺更上一层楼,只是微不足道的小事而已——或许对后世的意义很大——但他如果因文字而得罪,对每一个文官来说,都是个危险的信号。

韩冈不惧,不代表他的朋友、门人不惧,这一次,必须得伸手拉上一把。

