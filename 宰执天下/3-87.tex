\section{第28章 临乱心难齐(二)}

何家争坟案结束,在周边不过是留下一份谈资而已,但对于韩冈来说,只是他辛苦的开始。

上午处理公务,而下午就去县外诸乡视察灾情。半个月来,白马县的十六个乡,韩冈都跑遍了。通过保甲法而设立的二十六个大保的保正,韩冈也都见过以便。而原本的积案,又都断了几个。解决两村田地争水的纠纷,兄弟争产的纠纷,也都加以安抚和调解。

另外就是新法的推行情况,由于秋税已过,韩冈就不用催逼百姓缴税,而是处理积欠问题。年前两浙灾伤,总计十来万贯五等户在便民贷上的积欠,当地官员奏请天子后,就此一笔划去。既然有先例在,没有说白马县的积欠不能赦免的。下户在便民贷上的欠账也不过几千贯而已,韩冈已经写了奏章呈递上去,当不会有不允之理——作为一县之尊,理所当然的要为自己治下的百姓争取利益。

不过作为知县的韩冈忙忙碌碌,下面的幕僚也是跟着在忙。魏平真坐镇衙中,帮着监督钱粮。方兴则跟随韩冈,到了傍晚才风尘仆仆的回来。

正好游醇也从县学中回到衙门。韩冈安排了游醇在县学作学官。游节夫虽然年轻,但他的文学水平的确出色——福建的有名才子到了北方的乡下地方,绝对是超一流的水平了——加之韩冈的支持,游醇只用了十天的时间,就已经让白马县的士子们心服口服了。

三人一见,各自都脸都瘦了,不由得也是摇头感叹,给韩冈做幕僚,还真是辛苦。

晚间吃过饭后,三人又坐在一起聊天,而韩冈则在书房中,看白马县旧时的陈案。

“总觉得正言在急着什么?”游醇很少听说过如此勤勉的知县,在他看来,韩冈已经忙得不像一个官了,“真要说起来,正言当头就把那桩争产案拿出来,就是有些急了。其实可以慢慢来的,用不着一上来就冒险。”

韩冈的心思,方兴则看得明白:“能不急吗?看眼下县里的情况就知道了,明年的大灾那可是不得了的。”

“这跟何家争产案有什么关系?”

“人望啊!”方兴长叹道:“正言要得就是人望,方才迎难而上。靠着潜移默化,你说正言要多少时间才能攒下如今的威望?能让小吏不敢欺瞒?能让百姓心悦诚服?现在呢,一个案子就够了!”

魏平真也跟着道:“没有足够的威望,怎么能在明年的大灾时,安定本县人心,如臂使指的指挥本县百姓救灾?如何能压迫那些为富不仁的大户,不要囤积居奇,趁势搜敛民财?!”

“但也不至于这般心急。”游醇声音转低,“正言该不会是要帮着王相公,才如此急进?”

这么大的灾伤,宰相必然要出来负责,除非今冬河北、京畿大雪连番大雪,否则灾情继续下去,明年王安石肯定要离任。

“正言要是真的支持他的泰山,就不会落到白马县来做知县。”虽然是从王安石那里转到韩冈幕下,但方兴说得一点忌讳都没有,“如果不举荐横渠、洛阳的几位师长,正言难道在朝廷找不到好位置?同修起居注跟在天子身边都绰绰有余,那需要什么资历?有天子看顾,有宰相支持,一个权发遣,什么职司拿不到手?!就是不和王相公亲附,所以才落到白马县来。”

游醇说不出话来。二程就是从韩冈的举荐中看到了希望,知道韩冈与他的岳父不是一路人。程颢介绍游醇来韩冈处作幕僚,也明白的让他时常劝谏,不能让韩冈彻底偏到新党一边去。

魏平真看着一脸倔犟的游醇,仿佛看见了三十年前幼稚的自己,微笑着,问道:“节夫你以为当王相公因此灾而下台后,如韩、富、文诸公会怎么做?”

“当然是拯危济难!”

“错啦!”“大误!”方兴和魏平真一齐暴笑了起来,游醇的说法实在太天真了。

“是党同伐异!“魏平真脸容一下转冷:”拿着一清积弊、拨乱反正为借口,尽废新法,将王相公的势力彻底铲除。说牛李党争那就太远了,想想庆历新政,吕文靖【吕夷简】对范文正【范仲淹】是怎么做的?‘一网打尽’啊,节夫!至于正事,那是排在后面再后面!”

方兴也冷笑:“反正所有的错都可以推到前任身上,怨有所归,有什么好怕的呢?反倒是如今的王相公,为保住自身和新法,肯定会竭尽全力来救灾。”

“今冬明春的灾伤河北肯定是救不了的,到时候流民过河而来,蜂拥向东京城,到时候,还是看乐子的为多。要不然,就是乘机攻击王相公。看看有几个会出主意帮着流民一解倒悬之苦?”

游醇不知该如何争辩,但他的心里,对方、魏二人的说法却是无论如何都不能认同的,不停的摇头。

见着游醇不服气,魏平真收敛笑容,问道:“一到荒年,粮价便是飞涨。节夫你说这世上是囤积居奇的奸商多,还是开仓施粥的善人多?”

“这……”游醇想说奸商多,但这又不合人性本善的道理,一时结舌。

“我告诉你,其实还是善人多!”魏平真几十年不得仕宦,胸中有着一股愤世嫉俗的心思在,“但善人多在乡野,而奸商之所以能为奸,就是他们背后有人撑腰,否则何敢为奸?!”

“朝中总有正人!”游醇兀自强辩。

“正人?”魏平真呵呵冷笑,“范文正算不算正人?晁仲约当年知高邮军,不知逐盗捉贼,反以牛酒犒劳过境巨寇,希图祸水外引。这等官当不当杀?但你知道范文正说了什么吗?……‘祖宗以来,未尝轻杀臣下。此盛德事,奈何欲轻坏之?他日手滑,恐吾辈亦未可保。’”他厉声质问:“晁仲约论罪足当死,但范文正为日后天下文臣着想,故而贷其死,不知节夫你认为范文正说的对还是不对?”

范仲淹此举无视律法朝规,而且开了一个极恶劣的先河。但从士大夫的角度来讲,做得也不算错。游醇一时也不知该点头还是该摇头。

“这个例子用的不妥。”方兴眉头一挑,冷笑道:“朝廷年年向西北二虏奉上岁币岁赐,近百万贯民脂民膏毫不吝惜,且天子还要与蛮夷叙亲。而奄奄诸公,不以为耻,反以为荣,乃称此是圣德事。晁仲约以牛酒奉盗贼,不过是上行下效罢了!当然不能降罪!”

方兴这话一出,魏、游脸色急变,连忙阻止他再说。这话传出去,韩冈都要担一份罪责。而心惊胆战之余,也没心思再争辩了,便摇头一叹,各自散去。

而到了第二天,该忙碌的还是要忙着。

魏平真算着钱粮上的帐,监督着户工诸曹,而游醇照例去县学。韩冈则带着方兴去视察县中的医馆。

照律条,州县城中都该有医馆,而且由官府支持,医生就在县衙边坐馆,医治百姓。同时按照敇令,每逢夏日,县中都有两百贯汤药钱,用来散给百姓防暑药物。到了冬天,若是无名路倒死尸,也是官中出钱将之收敛火化,然后掩埋。

这一条条律令定得其实极好,可有几个真个照着去做的?毕竟是善财难舍啊!

而韩冈现在就准备将之一条条的实行起来,该节省的节省,那些吃喝玩乐的费用都会投入到备灾上来,该用的则用,他最拿手的疗养院,就准备快一点将架子搭起来。同时已经在县外的一片空旷荒地上规划好了地皮,以备即将面对的成千上万的过境流民。

十一月初一。

天依然是晴着,一点云翳都看不到。只是不再发蓝,而是因为被风卷上天空的灰土而带着蒙蒙的黄色。

也就在这一天,第一股超过百人的河北流民,渡过了黄河,进入了白马县境内。

流民来得如此之早,让韩冈也不由得心惊。听了消息,就骑上马,带着随从往北面的白马渡方向去。

就在何双垣墓边不远,韩冈见到了这股背井离乡的流民,大包小包的背着、挑着,有的还推着独轮车,小孩儿们不是坐在箩筐里,就是坐在车上。

见着一队马队直奔而来,其中有许多还是跨弓带刀的壮汉,流民们一下都被吓得四散奔逃。

幸好方兴连声高喊,“各位百姓,不要惊慌,白马县的韩知县来探视各位。”这才战战兢兢的站定了下来。

韩冈先远远的下马,然后慢慢的走上前,几名护卫拿着刀要走到他的前面,却被他推开。

流民们各个面有菜色,衣衫褴褛。大人都瘦得脱形,而小孩子的腿脚更是都瘦得只能看到骨头。

何阗、何允文两家,他们都比这些流民要强得多。就算是何阗,他虽说贫寒,其实也是能吃饱穿暖的。却为了两顷田打了三十年的官司。而眼下的这群流民,却个个面黄肌瘦,摇摇晃晃的随时倒下都不奇怪。

看到这片惨状,韩冈只觉得怵目惊心。

面对着惊慌不已的河北百姓,韩冈尽量的将声音和气下来:“河北灾情,本官早已知之。已经奏请上闻,不日必有回音。就在县城外,本官也已经安排下驻地,搭建帐篷的材料也准备了。诸位父老尽管在本县安居,且等灾情过后,再回乡不迟。”

有些事,他根本不在乎。但有些事,他却不能不在乎。

