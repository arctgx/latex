\section{第41章 辞章一封乱都堂(一)}

【第一更,红票,收藏。】

韩冈在城南驿的大门前翻身下马,一名门吏当先迎了进来。

不同于接待辽国使臣的都亭驿和西夏使臣的都亭西驿,韩冈入住的城南驿是供进京的官员们居住的驿馆。为屋百楹,院落二十余座,比起长安、洛阳的驿馆,又要强出十倍。不过门吏的傲气也比长安、洛阳驿站的同行强上十倍,眼中藏着京中子民才有的自负,行礼虽是一丝不苟,但没有韩冈见多了的谦卑神色。

这也是情理中事,韩冈见怪不怪,让李小六带着驿马与门吏说话,自己则走进驿馆厅中。进了馆中,韩冈向着驿卒亮了一下驿券,驿丞很快就被找来——还是与长安、洛阳的情况一样,管勾驿馆的官员不会出面迎客,都是下面的小吏在跑腿。

“官人是来候阙的?”驿丞举止间有着官员的派头,在韩冈面前不卑不亢,也可能是看着韩冈不是高官的模样,所以少了些恭敬。他啧啧的叹着:“现在可是迟了。”

无论是到审官院还是流内铨,又或是主管武臣的三班院,呈名候阙都是在每个季度第一个月的上半月便结束了。如果有哪个想为自己弄个差遣的无职官员,如韩冈这般拖过了正月十五才到京城,就只能等到夏季开始的四月份了。

但韩冈不同。

“不,韩某的职司已经定下了。”韩冈摇了摇头。此时官多阙少,一个差遣或者叫职司,都是几个官在争,有官身没差遣的官员都需要候阙,可他的职司早就有了。

驿丞微微吃了一惊,又低头仔细看了韩冈的驿券,“十九?!”他惊得又抬起头。仔细看过才发现,他眼前的这些小官人的确面嫩,就是眼神甚深,眉峰太利,让人不自觉的忽略了他的年龄。

能在京城驿馆里做驿丞,眼力眼界都不会差,而朝廷最近的变动、新的条令法规,连便桥边站着等人雇的车夫都能够说出个一二三来,他更不会不了解。十九岁得官不难,但十九岁得差遣,却是难如登天——真的要登天!不把名字放到天子面前,哪可能会有差遣!?

态度一转变得恭敬,驿丞把韩冈一行安排在了驿馆一角的清净上房中,再亲自遣了人手来听候使唤,这才退了出去。

终于抵达目的地,韩冈躺在床上,近二十天来积攒的疲累全涌了上来。只闭了下眼,就沉沉的睡了过去。等他醒来,却已是日影西移,过了午时,肚子也在咕咕的叫着。

自从来到这个时代后,韩冈一直保持着一日三餐的习惯。这一点特别的地方,让王韶都感到惊讶,因为整个大宋,有着这样习惯的地方很少,其中也并不包括秦州。许多军州,甚至连一些富户豪门都是一日两餐。不过在东京,却不同于大宋的其他地区,即便是小民,惯常的也是一日三顿。而开门做生意的酒店、食肆,更是不在乎饭点,随到随吃,驿馆里也是一般。

在驿馆里随意的用过饭,韩冈考虑着今天接下来的行止。东京城中值得游览观光的地方很多,但他还是觉得先做了正事再说。此时天色尚明,但自己去流内铨,刘仲武去三班院,都已经算是迟了,只能明天请早。现在韩冈面前有两个选择,一个是去见王安石,还有个则是去找张载。

韩冈方才在街边顺耳听了一句,虽然消息模模糊糊,但他还是半蒙半猜的推算了大半真相出来。王安石请郡,并且是以称病的名义辞去参知政事一职,请求调往地方任职。王安石的这番行动,便是在大宋朝堂的政治【和谐第一】斗争上,标准的认输姿态。

但王安石究竟认输了没有?韩冈的判断是否定的。王安石正式开始变法,是从去年二月出任参知政事,设置三司条例司开始,七月颁布均输法,九月立青苗法,十一月,颁布农田水利利害条约。到现在,才一年的时间。

这么短的时间,变法才刚刚开了头便失败了,怎么可能在历史上留下那么大的名声?连革命导师都听过他的名字和事迹?好歹也要有四五年的光景,把所有的人都得罪光才对!——可惜的是,韩冈对历史不甚了了,要不然混水摸鱼,兴风作浪的机会就来了。他时有后悔,早知今日,当初历史课就不睡觉了。

如果方才的推论正确,那王安石的用意也就不难猜测。诸如此类官场上以退为进的战术其实并不出奇,职场上有,情场上更是所在多有。反正本质就是一句话,有我没他,逼着人作决定。二选一的场面,韩冈旧年经历过许多次,富有经验,但赵顼应该不会有。

——从目前的情况看,也就是赵顼现在要做选择,究竟是变法,还是不变法。

韩冈虽不知道究竟是什么事逼得王安石如此作态,但变法走到了关键的转折点这件事,他却完全可以肯定。因为这是一手逼不得已才会放出来的大招,若是有其他选择,聪明人都不会轻易的使出这招胜负手。这一招一拍两瞪眼,完全不给自己留后路,招数一出再没有转圜的余地了。

想必王安石现在是在府中等着结果,这种情况下去求见,多半是见不到。河湟的那点事,远远比不上变法事业的存续。韩冈想来想去,还是决定先去张载那里打探一下消息,听说张载弟弟张戬是御史,官位虽卑,却可以直接议论朝政,从他那里应该能得到第一手的情报。

出了房门,韩冈去跟刘仲武和路明打声招呼。刘仲武又蹲在马厩里,说不定今天晚上也不会出来了。而路明还在考虑着日后该怎么做。他为了科举花了一辈子的心力,自己放下了,但他的亲友、家人那里都还要他一一处理。不考试了,总得为自己日后想个能养家的出路。

韩冈劝他:“路兄,既然到了京师,不如今科再考一次,博个运气。如果不成,等到下一科,那时再考个特奏名进士出来。到时候,在西北的军州任个文学、助教之类的学官,拿点俸禄,也好养家糊口。不然不是可惜了你这个免解贡生的身份?”

路明摇摇头:“在下赌了三十年了,都是这个想法。总想着这一科如果不中,下一科就去试试特奏名。但真到了下一科,便又忍不住要考进士了……当断则断,不能再赌下去了。”

韩冈拍了拍他的肩膀,陪着他叹了口气。既然路明如此决定,自家也不便多嘴,便带着李小六出门去了。李小六手上还捧着礼物,学生探望老师,照理是要表些心意的。

张载和他的弟弟张戬在城东租了间宅子同住,韩冈从留守横渠镇老宅的老夫妻那里得到了具体地址。他在驿馆中将道路问得明白,不知为什么,被他询问的那名驿卒,看他的眼神甚为奇怪。等他骑着租来的马,到了地头,才知道为何驿卒的眼神那般怪异。韩冈完全没想到,张家兄弟在京师租得的宅子,竟然就靠着小甜水巷。

从城南驿到小甜水巷,中间正好经过大相国寺的北门。韩冈打马路过,没能进去见识一下何为‘棋布黄金,图拟碧络,云廓八景,雨散四花’,只看到这座天下第一的皇家寺庙,即便是后门处里面都是黑压压一片人潮如海。不过听一同陪着走的租马人说,今天并不是大相国寺每月五次万姓交易的正日子,只有些上香拜佛的香客,人数要少得多。

韩冈犹在回头望着大相国寺,便已经到了地头。他们在小甜水巷边下马,韩冈掏钱会了钞,租马人便带着三匹马回城南驿的门口去了。他的身份相当于后世的出租车司机,带着几匹马等着人来租用。如他这样的租马人,在东京各处的街口、桥边,都能看到。

小甜水巷口,韩冈抽了抽鼻子,空气中弥漫着一股子脂粉和头油的甜香味道,甚是腻人。此时刚过午后不久,小甜水巷看起来很清静,往巷中看,行人并不多的样子。但韩冈知道,等到上灯时分,情况就不同了。

东京城东南的甜水巷其实是四条巷子的合称——第一甜水巷,第二甜水巷,第三甜水巷和小甜水巷。其他三条甜水巷还算好,是开封东南的商业街,酒楼众多、店铺林立。而小甜水巷则是妓院一条街,中间夹杂着些食肆,相当于秦州惠民桥后的地方。驿卒多半是误认韩冈刚到京师便要尝尝京师佳人的味道,又不好意思明说,才故意找个临近的地方来问。

王厚过去没少在韩冈等人面前提起小甜水巷婊子的风情,顺带把惠民桥后贬得狗屎不如,惹得王舜臣如同十几只老鼠在抓心撩肝,只念叨着要去京城逛逛,而赵隆也是听得悠然神往。不过韩冈清楚王家的家教如何,王厚真的敢去逛青楼,两条腿都会被王韶打断。他所说的,自然是道听途说而来。

一阵香风飘过,一名装束艳丽的妓女从韩冈一侧擦身而过,匆匆走进巷内,还不忘顺带回头抛了个媚眼。韩冈对浓妆艳抹的女子向无好感,看了一眼便转头,但李小六已经渐通人事,又没经过阵仗,顿时眼都直了。

韩冈曾经听说过有位伟人为了锻炼自己的集中力,而故意在通衢大道上读书,现在张载张戬定居在甜水巷隔邻,离着不及百步就是妓馆,不知是不是在锻炼自己的毅力。

笑着摇摇头,这样想实在太不恭敬。他举步,慢慢走进张载家宅所在的后街小巷。

