\section{第二章 一物万家欢(上)}

高遵裕的确是病急乱投医了。

一向深沉的高遵裕,现在已是急得满头汗水,而且今天的天气也是闷热,搞得他浑身上下都是大汗淋漓。

在韩家听到消息的时候,羊水已经出来了。但这一夜都过去了,孩儿却还没有给生出来。在产房外,听着明珠的嘶喊声越来越小,熙河副总管是心急如焚。

这一夜,两个稳婆跑进跑出。领头的徐姓老稳婆是从秦州请过来的,做了三十多年的稳婆,接生下来的孩儿成千上万,秦州城里富贵人家的子弟看到她,都要恭声道一句徐婆婆。而她带来的徒弟,是她的儿媳,也有十来年的经验。但就是这两个稳婆,现在也快没了招数。

“头都看到了,但就是最后一口气下不来。贵家的娘子身子骨实在是弱了点……”

高遵裕不想听这些话,他阴声问着:“到底会如何?”

“可能大的小的都保不住。”

面对高遵裕,徐老稳婆也没有讳言。她见过的官儿太多了,韩琦家的孙子她都经过手——二十多年前,韩琦正在秦州做了知州——太后的叔叔又如何。

高遵裕脸色彻底黑了下去,眼下最放在心上的宠妾,还有快要出生的儿女,就落得这么轻飘飘的一句‘可能大的小的都保不住’。整个人的眼神都危险起来。

“不是听说药王弟子在这里吗?……韩机宜啊。他应该有主意吧?”

徐老稳婆的儿媳三十多岁在旁缓颊。说起话来没有她姑姑的底气,细声细气的,但落到高遵裕的耳中却是黄钟大吕一般。的确,韩冈虽说是从不承认什么药王弟子,但疗养院,治疗骨折,烈酒消毒,有关医道之事,他可是有过不少创见。而且韩冈多智,去找他的时候总能有办法。

不用人提点第二遍,高遵裕就连忙派了身边的管家去找韩冈。

韩冈听着高府的管家匆匆忙忙的几句话将事一说,苦笑不已。莫说难产,就是正常生产,韩冈都没亲眼见识过。

他倒是见过军马难产的,去左近的蕃部时正好撞上。为了保住母马,蕃人们处理的手段很粗暴,直接就把小马切碎了拖出来,这样就伤不到母马,是个好办法,只是用不到人的身上。要是他说一句把卡在盆骨处的孩儿切碎了拖出来,高遵裕能跟自己拼命。

但韩冈也不可能拒绝高遵裕的请求,怎么也得走上一趟。幸好为了自家的娘子也曾有些想法,他也曾私下里命人作出了实物,只是眼下不便拿出来——未卜先知的,坐实了什么药王弟子的名头可不好办——但那器物的式样心中有着谱,打造起来也容易。

跟了高府的管家出来,就正好碰上了两名挺着大肚子的孕妇。

“官人?出了何事?”

周南和严素心此时都已经起来了,被各自的婢女搀扶着,挺着肚子在院子中走着。见到韩冈跟着高府的管家匆匆而出,两女都有些疑问。

她们一贯早起,也按着韩冈还有医官的嘱咐,每天在院子中多走动。平日吃的也不算很多,就是为了防着头胎难产。眼前的两张如花俏脸,现在也只变得微圆,而这种略显丰腴的容颜因为充满了母性的光辉,反而更为吸引目光。

此时普通富贵人家的孕妇,都是鸡鸭鱼肉往死里狠补。但真正能请上良医的人家,就不会这么让人填鸭似的给孕妇滋补身体。而高遵裕的小妾明珠,应该得到了同样的医嘱。但高遵裕好象是偏好娇弱型的女子,韩冈曾见过明珠一面,身子骨实在有些弱,会难产倒不是没有听从医生吩咐的缘故。

明珠现在的情况,韩冈知道不能对周南和素心说,要是让她们受到惊吓,本来没事的反而变得有事了,“有事去高府一趟,你们散过步后,就去好好歇着,不要累到了。”

韩冈脚步匆匆的赶到了高府,高遵裕连忙迎上来,急叫着:“玉昆,你可来了。”

高遵裕浑身上下都向被水泼过一般,急得跳脚的样子,就算碰上了再危急的战事,韩冈都没有在他身上见过。暗叹一口气,毕竟是关心则乱。换作是自己,当周南和素心难产的时候,怕也是一般的不中用。

“总管莫急,你在这里急着也用不上力气,先缓口气,喝点水。”韩冈丢了两句话给高遵裕,反身问着站在一边的老妇,前日请来陇西时,韩冈就见过她——秦州有名的徐婆婆,“徐婆婆,里面的情况如何了?”

“都看到了头了,但娘子没了力气。如果官人没有办法,可能母子都保不住。”只有四尺多高的老妇硬邦邦的回着话。韩冈看她的模样,很可能是对自己这个药王弟子不以为然。那么把自己拖过来,也许几分分担责任的意思了。

韩冈啧啧嘴,想了一想,回头对高遵裕说:“总管,韩冈只有个不算主意的主意。”看着高遵裕一下亮起来的眼神,他继续道,“平日里是不好用的,但现在已经这步田地……”

“玉昆,你就直说吧!”高遵裕心急如焚,直跺着脚。

韩冈点着头:“以下官的意思,既然已经到了最后一步,只差了一口气。那干脆弄个钳子将胎儿夹出来。”

“钳子?”

“夹出来?”

高遵裕和徐老稳婆同时问着。

“不是普通的那等两枝尖长的铁钳,而是前端弧形,能卡着孩儿的头颅。”韩冈让人找来纸笔,随手就将产钳画了出来。

关于产钳,韩冈只听说过名字,但他还听过曾经西方一个助产士的家族,为了赚钱,将这个技术隐匿几十年的故事。为了自家的女人,他苦思冥想好久,才从记忆中翻找出来。知道用处,通过名字也能明白基本原理,要画出大概图样就不算很难,如果再有一个经验充分的稳婆在旁建言,那么要派上大用场也是理所当然的。

“把图纸和材料送到铜匠那里,用银子来打,半个时辰就足够了。”

已是危急关头,房中的声音都已经暗哑下去。韩冈毫不犹豫的下着命令,也不管老稳婆满脸的不以为然。两个稳婆要分出去一个,剩下的让产妇保住一口气。也好在打造时指点一下尺寸和式样。而且银料的硬度不算高,如果尺寸不合,只要临时用手扳上一下,也不用费多少力气。

高遵裕是何等身份,一句话的功夫,就将城里的铁匠、铜匠都调到了同一个工坊中,另外还多拉上了个银匠,他不是常驻陇西城中,前两日刚刚到这边来找口饭吃。正好高遵裕派出去找工匠的军汉刚刚帮浑家打了一套银饰,受命时提了一句,高遵裕和韩冈便异口同声地下令将其请过来。

众工匠到的时候,韩冈已经等在了工坊里。看到人来齐了,也不废话,就让他们按着徐稳婆的指点来打造产钳。

几个工匠心中疑惑,却也不敢多问。铁匠、铜匠、银匠同时动手,听着一个老迈龙钟的老婆子的指点,打造一个前端带着弧形,正好能套上孩儿头颅的银钳。

每个工匠都是老手,锻银子也比锻铁锻铜更为简单。先将银子打成银条,再铆在一起,然后弯成合适的形状。叮叮当当的一片声后,先是一个铁匠完工,接着铜匠们也完工了,最后银匠才结束了工作——他之所以慢,是因为他费了点时间,将产钳上的锋锐部位,都给磨了一遍——从高遵裕下令到现在,加起来也不到半个时辰。

打出来的产钳式样都差不多,徐老稳婆便拿着试了试手,挑出了两把,其中一个就是银匠的。

韩冈关切的问着:“徐婆婆,怎么样?”

老稳婆脸上的不以为然早就不见了,谦卑的回答道,“回官人的话,看起来还成,但得先试了再说。”

韩冈和徐老稳婆赶着回高府,随身还带着银匠。急就章的产钳,质量当然不会好,如果有所损坏,还得让他来当场改动。这段时间中,徐老稳婆的儿媳则在尽量的安抚着明珠。高遵裕则在院子中,来回的踱着步子。

听到韩冈和老稳婆回来的声音,高遵裕转头过来,“怎么样了?”

“先是了再说。”韩冈顿了一下,郑重的对高遵裕道,“高总管……”

“不用说了!我知道。”高遵裕皱着眉一摆手,“药医不死病,就算最后不行了,那是命,跟谁都没关系。”

高遵裕放得开,韩冈也放下心来,安心的指点着内外之事。

两个稳婆和打下手的丫鬟们全都是用烈酒洗过手,而两具产钳全都用开水、烈酒反复清洗过。依照疗养院的规矩,全都换上蓝色的罩衣,口罩,并带上了遮住头发的布帽——在疗养院中,这些手术必配用具都不缺。

穿戴好一切,徐老稳婆走进了房中。床上的孕妇蜡黄着脸,已是气若游丝。

虽然是第一次用产钳,但徐稳婆好歹几十年来也是接生过上万名小儿,也曾用手将出不来的婴儿拽出来过。知道钳口放哪边,该怎么使劲。试了两下后,她便稳稳的用产钳将拖拖拉拉不肯出世的高家小儿给捉住了……

高遵裕还在踱着步,左一圈右一圈,看的人眼晕。只是他突的脚步一停,侧耳听起了房内的声音,而一边的韩冈在这之前,就已经竖起了耳朵。

从房内传出来的声音虽然微弱,却的确是孩儿在哭。

