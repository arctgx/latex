\section{第32章 金城可在汉图中(21)}

折可大一步两阶,大步流星走上城头。

张俭提着官袍的衣角,紧随在他身后。但体力不足,跨上最后一阶的时候,却已经是呼哧带喘。一步没踩稳,木底的靴子便在带着青苔的砖石上一滑,人就向后摔了下去。双臂扬在空中,惨叫声刚要出口,后背便被稳稳的托住。

重新站稳了脚,差点从城上跌回城下的张俭心有余悸的回头,一名三十多岁、脸颊上刺了字的军校正伸手扶着他。

“韩指使,多谢了。”张俭冲着那名军校点了点头,出声道谢。

“韩宝不敢当,只是伸伸手而已。”军校语气平淡,并不为卖了经略使机宜文字一个人情而兴奋。见张俭站稳了,便收回了手,视线也越过张俭,投向了已经站在雉堞后折可大背上。

张俭得了提醒,连忙转身往折可大那边去,韩宝也跟了过去。

来到折可大的身边,扶着城墙的雉堞向外望去,有一桩显而易见的事实出现在张俭的面前。

太原城外已没有了之前几日的喧嚣,虽然还能看到契丹骑兵的活动,但数量明显减少了许多。

之前就算是分头去乡里打草谷,也没见城外的辽军少于万数,依然是旌旗招展,人马如海。可现在,就像是收割过了的麦田,变得稀稀落落起来。

“辽兵当真退了!”犹喘着气的张俭一下挺直了腰,惊喜到忘了阖上张开的嘴,想不到当真不是误报。

只是张俭的喜悦没有传给他的同伴,折可大脸上看不见分毫喜色,向着城外的一处眯起了眼,声音依然低沉:“没走干净!”

几处城门之外,依然有着为数不少的契丹骑兵盯视。可以说,太原城还是处在被封锁之中。以城中的军队数量,不付出大的代价,还是很难突破这样的封锁。

“好歹是少了。”张俭笑着说道。围城的军队少了就是少了,而且既然辽军主力已经离开,城下的这些当也只是殿后的军队而已,不会逗留太久。

折可适却仍沉着脸、锁着眉,心事重重。他左右回顾,周围官兵们的脸上都是一幅如释重负的神情,与张俭一模一样。他轻声一叹,终究还是少有人能多想一想。

“王知府可以少念几句阿弥陀佛了。”张俭双手合什,却是没什么虔诚的笑说着。

折可适皱了皱鼻子,想笑,却笑不出来,嘴角扯出的纹路填满了苦涩的味道。

太原府的王府尊在北虏围城的十几日间,整日价的只知念叨着阿弥陀佛,求着佛祖保佑援军能按时抵达,却没有在城防上作出多少作用。

在折可大的眼中,这两年王.克臣在太原府的治政其实也能算得上中上水平,只是因为有韩冈在前做对比才显得口碑不足。不过辽军一来,便把他不擅应对兵事的缺点给暴露出来了,举措多误,更无力安定人心,现在都没看出来辽人离开究竟是为了什么,终究是狗肉不上席面。

“还是拜托王府尊多念几句屙屎豆腐吧。”一直沉默着的韩宝突然开口,“辽贼可是奔援军去的。”

“什么?!”张俭的神色陡然一变,一下楞住了。

韩宝望着城外:“辽贼移动的方向是南方,如果仅仅是打草谷,不会出动这么多人。更不会集中在一个方向。只可能是为了援军。”

张俭终于反应过来,苍白着脸望向折可大。

折可大点头:“韩指使说得没错!”

张俭如同从天堂落到了地狱。以他的才智其实应该能看得出来,但辽军主力的离开,仿佛是搬走了一块压在心头的巨石,放松之余就只剩下一份狂喜了。

现在回过神来,头脑重新运转,终于发现局势并没有好转,甚至是更为险恶。辽人既然肯定是为了援军去的,那么只要他们能击溃了北上的援军,太原自然也逃不过。甚至局面会比之前更差,援军惨败+失去了信心的太原城,都不用辽军费力气去攻打了。

“不用担心。”折可大眼瞳中闪烁着光芒坚定如钢,“这是韩枢密故意将他们引走的。”

“为何如此说?”张俭连忙问。

折可大一笑:“知兵如韩枢密,为什么会公然声称二十日援军必至?就是为了让辽人记挂着援军啊!”

凭借蛛丝马迹,折可大几乎可以确认,辽人之所以会南下。完全是韩冈是拿自己做饵,硬生生的把辽军给吸引走的。

张俭的心情平复了一些,但折可大的又一句话,又让他难受起来:“但打仗的事,谁也说不准会不会有意外,依然的做好准备。”

张俭苦着脸,只听得韩宝也在旁帮腔,“府尊要念屙屎豆腐,没多少时间,还请机宜和通判赶快整备城防才是。”

折可大看着韩宝,眼中不掩欣赏。

这是他这段时间认识的新朋友,虽然仅仅是尚未入流的底层武官,但一个手握三百多士卒的实职指挥使,在现在的太原,地位已经很不低了。而且眼光头脑都不差,是个难得的人才,如果再历练一下,不是不可能成为一名出色的将领。

张俭此时已经收拾好心情,不再一惊一乍,必要的城府还是有的。最坏的局面也不过是恢复之前的情况,也没什么大不了的。何况两名武夫在前,他也不想太丢文官的脸。

“韩指使似乎不太喜欢佛门啊。”张俭低声问着折可大,他刻意岔开话题,好让自己能留下一份颜面,“是不是信道门?”

从韩宝说话的口气中,张俭能很明显的听得出来他对佛家的不屑。很少能见到军官对佛门这般厌恶的,这让张俭有了几分好奇。

折可大同样低声:“他未过门的浑家曾给个贼秃占了,怎么可能喜欢和尚?”

韩宝见两人开始说私话,便立刻挪远了,走到了十几步外等候。

张俭放松一点:“他是因为这件事犯了法才入军中的?”

折可大眉一挑:“怎么看得出来?”

“当然看得出来,脸上的金印不一样。”张俭微微一笑,“以他方才的脾气,当也不会隐忍。”

“原来如此。”折可大点了点头。

除了少部分特招的效用士,绝大部分士卒入伍时都会被刺字。刺字有刺鬓角的,也有刺额边的,还有一些乡兵弓箭手是刺字在手背上,当然,刺面颊的也不少。这不仅仅是身份的标志,同时也书名了隶属和番号。

犯法刺配军中的罪囚同样要刺字,不过金印的形状、文字和位置跟普通的士兵一看就有区别。轻罪的还好,跟士兵同样都是刺小字,尽量在脸颊的边缘,以求不毁人容貌。但重罪的罪囚——流配千里以上的基本上两边脸颊。额头上直接刺了强盗二字的配军,营里正好就有几个。这些都是给官府捉了之后,幸运的被赦免了死罪的强盗,在军营中脏活累活都少不了。

韩宝脸上的刺字就是最典型的刺配罪囚才有的金印,不像有些士兵的刺字,远远地看起来,还有几分像刺青图案。

其实刺青在世间是寻常事,夏天的时候到市井中走一圈,很容易就能发现有很多男子身上有着花式各异的纹身。周太祖郭威的脖子上就刺了一只雀,所以人称郭雀儿。折可大身上其实也有,就在胳膊上,但只有个粗糙的轮廓。

少年时的折可大曾经做过几天纨绔,跟他的十六叔折克仁以及十几个年岁相当的玩伴横行街市乡里,甚至还相约去刺了青。不过好一点的纹身不是一天就能完成的,刚刚刺了一个虎头的外廓过来,一回去便给拎去祠堂一顿好打,接下来自然就没有然后了。

“那和尚最后怎么样了?”张俭问着。

“那贼秃给削了子孙根,只是没入宫的运气,当天就咽了气。”

听到韩宝是从哪里下的手,张俭身子就是一抽,双腿也下意识的夹.紧了一点:“杀人?!”

“杀个淫僧!”折可大更正道,他当日听到韩宝当兵的原因之后,只觉得解气得紧。

“那时当还没有自首减二等的敇令吧。”

张俭对刑名认识非浅,甚至曾有过去考明法科的打算——本就是低人一等的荫补出身,若再没几分拿手的活计,在官场中也是混不好的——律令、编敇、案例或许还不能倒背如流,可当今天子颁布过的最有名的一条律令,他不可能不知道。

在自首减二等论法的敇令实施前,只要定了是故杀,再情有可原也当是绞刑,除非遇上大赦,或许还有那么一分可能免死。

“论理是死罪的,不过当时的县尊看他是条好汉,杀的又是在理,就批了个失手误伤。”

误伤致人死地,就是流刑了。张俭点点头:“倒跟狄武襄有几分相似。”

狄青也是伤人犯法,受刑后被收入了军籍。不过据说那不是狄青本人犯下的过错,而是帮他的兄长顶罪,而且人也没死,后来给救下来了。

“狄武襄军中可没什么人能比得上,但也算是条好汉了。”

折可大很看好韩宝,想将他拉入自家。方才多说了两句,现在就警醒了起来。只盼着张俭能将韩冈看低一点——毕竟是罪囚出身,文官寻常连武夫都看不起,何况罪囚出身的军汉?

罪囚或在牢城中干活,或直接就归入军中,同样被刺字。军汉跟罪囚在世人眼中就成了一类。他们这些将门出身的还好,世代从军能做到指挥使或是都头的也还说得过去,可普通的士卒根本就等同于贼配军。

当然,对从军的的看法也分地方。在穷困的边疆,吃官粮拿官饷是门绝好的营生。但在内地,可就是避之犹恐不及的恶差,正常士卒想要离开军队,甚至必须从族中找来一人顶替他的位置。

太原乃是富庶之地,说起来是国之重镇,河东的核心,不过百多年不闻烽烟,赤佬的地位自然不高。罪囚出身的赤佬就更不用说。

“且不说那一干败人兴的贼秃了。”折可大说道,“辽贼的主力既然南下了,就需要有人出去打探详情,究竟是胜是败。”

张俭也点头道,“若是韩枢密胜了,那么正好痛打落水狗。若是不幸有失,也能提前一步得到消息,警戒城中,以防有人谋图不轨。”

“多半能赢。”折可大更正道:“如果韩枢密当真能在南面的太谷县附近抵挡住辽军的攻势,甚至不求击败辽军,只要能拖住这一支兵马,待各路援军赶来,萧十三便是必败无疑,他手下的几万人马甚至有全军覆没在河东的可能。”

“只要韩枢密能做到。”

“当然能!”折可大对韩冈有着绝对的信心,毫不犹豫的断言,“韩枢密肯定能做到!!”
