\section{第38章 岂与群蚁争毫芒(七)}

范纯仁身份、地位都不低,亲朋故旧无数,于朝中名声也极大。不过这一点倒不算什么,韩冈是都转运使,监司官与亲民官不是一个路数,更有监察治下百官的职权,没必要巴巴的会上一面。

但范纯仁是范仲淹之子,而范仲淹曾经劝学张载,一代大儒实出于范文正公的一番劝诫。此事尽人皆知,这一份香火情,就算当年的当事人皆已不在人世,韩冈也不能翻脸不认。疑惑归疑惑,既然在唐州遇到了,在情在理都得见上一面。

所以在沈括设的接风宴上,韩冈见到了范纯仁。世人传说范家四子,以范纯仁最似范仲淹,今日一见,气貌纯粹,言谈举止的确不是普通俗吏可比。

范纯仁论年纪,可以说是韩冈的父辈。依靠父荫,他的起步比起韩冈当年要轻松得多,但这么多年下来,他的官职始终不高,总之是不合时宜之故,跟范仲淹一模一样。

原本他文学贴职还是直龙图阁,但因罪责授信阳军后,便连着这份贴职都丢了。如果范纯仁没有被降罪,他从名义上,应该是从属于韩冈这位龙图阁学士的手下了。

除了范纯仁之外,与会的并没有他的儿子、侄儿,只有走了顺道一起南下的新任辰州司户参军李之仪,说是范纯仁的弟子。从鄜延转调荆南,看来是贬任。韩冈似乎在哪里听说过李之仪这个名字,就是记不清是在哪里听到过,来自于鄜延路的种建中的信中,也没有提起过他。

在席上,韩冈和范纯仁初见而已,只是泛泛而谈,不过是说起两人长辈的旧日来往,以及两人都认识的熟人,拉一拉关系。

不过当不知内情的范纯仁提到入关中讲学的程颐时,韩冈还算是平和淡定的心情就变了有些坏了。

“纯仁自京兆府东行,于华州适逢程正叔聚众讲学。其入关中不过半月,关中士大夫便已是闻风影从,心向往之。还听程正叔提起玉昆你,说玉昆你曾于风雪中,立于程宅门前半日之久。积雪过膝,落雪满肩,问道之心可见一斑,尊师之举可为万世法。”

“韩冈曾于伯淳先生处聆听教诲,又是奉先师之命致信程府,于其门前自不敢有所不敬。”

由于苏昞和范育的来信,韩冈早已有了心理准备,范纯仁爆出的这个料并没有超出他的预计。

看来自己的预感还是没有错的,果然是被抄底了。张载去世,缺乏核心的气学,让入关中讲学的程颐给斩草除根,那是没得跑的。如果没有合适的手段加以反击,一两年前还在关中、京城兴盛无比的气学,就会是昙花一现,转眼就化为泡影。

道统之争本就没有任何私情可言,哪一位大儒不是深信自己走上的道路能直通天人大道?对于任何杂音,都有势不两立的想法。

韩冈对程颢依然尊敬,对程颐也保持敬意,但这并不代表他能忍受气学被程门收编。心情一变,与范纯仁的对话也就成了敷衍。

不过范纯仁的来意韩冈基本上也探明了,范仲淹的这位儿子在说话时本也没有隐瞒。

“种子正已然上书天子,意欲攻取西夏。如今关西兵虽精,然则不多,粮虽备,然则不丰。西夏母子相争,横山一役后,三年不敢犯中国,庆州百姓皆乐此太平盛世,岂有弃富贵而入行伍者。且西夏国力虽衰,仍坐拥甲骑数十万。争利山林非难事,用兵于兴灵,又岂是那般容易?”

“以二丈之见,当如何?”韩冈好奇的问道。

“息兵、消祸、止战、除役,但使彼国生灵,先感朝廷好生之德,则其酋首自无能为。”范纯仁的脸色变了一下,“否则兵祸一生,百万人流离失所,无所依归。”

范纯仁的公心,韩冈的确对此很佩服,但整件事就好笑了。明明有着足够的优势,却还要保持着守势,这一点韩冈首先就难以认同。他辛辛苦苦的打造板甲、神臂弓、斩马刀和热气球到底是为了什么?

更何况所谓疏不间亲,因为王舜臣、种建中和种朴的关系,种家对韩冈来说,是他在军中的基本盘,就算是有什么想法,也是私下里来交流,要吵架也是关起门来吵。跟范纯仁这外人,怎么也不可能交心。

别说是范纯仁,就是换作其父范文正公来,韩冈也不会昏了头脑,他早就过了遇上名人就晕头转向的年纪了。

对于范纯仁的忧虑,韩冈报之以畅快淋漓的大笑,“要攻打西夏,需天子首肯,两府无阻,千军万马又岂是那么好动的?且三军未动,粮草先行,这钱粮又是一桩。要想动刀兵,没有那么容易的事。”

转头看着范纯仁,他收敛了笑容:“自从官军收复熙河之后,种子正便接连上书要收复罗兀,那是熙宁五年的事,可横山一役收复罗兀城又是何时?是熙宁八年。如今种子正上书攻夏,即便通过了天子、宰相,想要点集兵马、输送粮秣兵甲,也不是旦夕之事,再怎么快也要两年——故忠献公旧年在陕西急于成事,才导致好水川惨败。有鉴于此,之后朝廷用兵,便谨慎了许多。王资政为河湟,筹划了五年;韩冈在广西,也用了一年,而西夏国力又岂是吐蕃、交趾可比?自当慎之又慎。”韩冈最后总结,“此事论之尚早,范二丈实是太多虑了。”

韩冈和范纯仁的这一次会面,说不上坦诚,更谈不上友好,只是礼节性的一团和气,说着不相干的闲话,最后也是维持着士大夫之间的礼节,看似亲热实则冷淡的相互告辞。

范纯仁双眉紧锁的走在前面,而作为陪客的李之仪跟在后面,两人骑着马往驿站行去。

李之仪脸上带着隐隐怒意,又有几分不解,“先生特意走唐邓,难道就是为了见韩冈一面?!”

“的确是为了见他。”范纯仁放着近路不走,不顾家人疑虑的绕路而行,究竟是为何原因,现在是终于承认了,“韩冈太过年轻,不宜居于朝堂之上,天子和两府,应该都明白这一点。所以等襄汉漕运打通后,他也不可能因功入朝。那么下一步,韩冈会被调到哪里?”范纯仁回头瞥了弟子一眼,“其实是不难猜的。”

李之仪瞪大了眼睛,惊道:“关西!?”

“以他的才干、功绩和官位,难道还不够一任边帅吗?直龙图阁已可为庆帅,直学士连开封府都能去了,何况龙图学士?”范纯仁自嘲的笑了一声,“随军转运一职,非韩冈莫属,更有可能亲领一路,让种五后顾无忧。”

李之仪这下完全明白了范纯仁的用意,双眼一亮,“若是韩冈不愿为之出头,甚至反对用兵,想必天子、两府,都会为之犹豫。甚至种谔本人,也会退缩。”

范纯仁摇头一叹:“……可惜啊,他也是一样,否则不会满口托词,却不言己见。”

李之仪怒道:“其心可诛!”

“端叔,当以责人之心责己,以恕己之心恕人。”范纯仁神色严肃。

李之仪低头受教,却又问道:“那先生打算怎么办?”

范纯仁语气平淡,眼神却是坚定:“割而可卷,孰为神兵;焚而可变,孰为英琼。宁鸣而死,不默而生。”

……………………

将范纯仁敷衍了过去,韩冈回住处时,浮荡在他眼前的还是范纯仁夹杂着愤怒、悲悯和坚持的眼神。

这应该算是偏执吧,绕了上千里来见自己,只为了阻止对西夏的战争,寻常人绝不会这么做,都已经被贬到京西来了。

因为在道德品行上无可指摘,所以行事、作为就是正确的。就因为自己是正确的,所以他人也应该赞同。看人如此,视己亦如此。这样的想法,实在让人哭笑不得。

什么叫一日三省吾身?

韩冈其实挺怵这等人,道理根本说不清楚。

摇摇头,便将范纯仁抛之脑后。

眼下襄汉漕运即将打通,只要荆湖的粮食能源源不断的运进京城,即便漕渠没有全线贯通,韩冈的任务都算完成了。

不过就算完成,也不会有多少有实际意义的封赏,韩冈很清楚,天子和两府中的绝大多数人,都不想他出现在朝堂上,无论功劳有多大,一个未及而立便离两府只有一步之遥的臣子,对眼下的朝局,还有国家的未来,都不是好事。

韩冈并不是为了他们而辛苦,更不是为了乞求功赏,他只是按部就班的照着预定的规划去做而已,受到。但这并不代表他会愿意坐在,,官位不是很在乎,但他需要回到京城。只是为了气学一脉的存续,他都必须回到京城,不借助开封的地理优势,他韩冈是压不住已经成了气候的程门,也凝聚不了气学一脉的人心。

眼下的当务之急,就是早点回到京城,并授课为人讲学。只是看起来,似乎很难。

