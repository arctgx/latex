\section{第38章 何与君王分重轻(23)}

王安石一下黑了脸。

愤怒的双眼仿佛要喷出火来一般。

他堂堂平章在这里,就是天子马上就要咽气,要赶着请韩冈这位药王弟子吊命,也不能把他王安石丢在一边。

他可是平章军国重事,朝廷的军国大事无事不可干预,而宫闱之内,只要有关天子,那就是国家大事,他同样有干涉的权力!

杨戬知道自己所传递的口谕,有多么容易激怒王安石。前面看见韩冈和王安石就在集英殿前慢慢踱着步子,心中就是咯噔一下,但他还是不得不当着王安石的面,去将韩冈请过来。

韩冈见王安石脸色难看,笑着宽慰道:“如果只有小婿,那还不打紧,要是请动岳父,情况就真的很糟了。”

“枢密说得是,枢密说得是。”杨戬在旁猛点头,可怜兮兮的瞅着王安石,“平章……”

“还不前面领路!”

王安石一声冷喝,因老迈而浑浊的双眼一下变得锋锐起来,狠狠扫过杨戬。有没有请柬是一会儿,能不能做到在想,又是另外的一回事。

杨戬吓得瑟瑟一抖,膝盖都弯了一下,差点没软倒在地上。

“奴婢知道,奴婢知道。”

杨戬颤着声,连称呼用错了都没有察觉,忙前面领路,也不管后面两人到底跟没跟上。

韩冈也不在乎,跟着王安石匆匆往福宁殿赶过去。

抵达福宁殿时,杨戬先进去了,跟在后面的王安石,守门的亲卫还想要拦他一下,但王安石又是一瞪眼,便吓退了门卫,大步流星的冲进了寝殿之内。

“王平章!”

王安石一进寝殿,就引得坐在榻边,看顾着天子的向皇后显是吃了一惊,甚至还瞪了杨戬一眼。直到看见韩冈跟在后面,才稍稍舒了一口气,小声道:“韩枢密,还请快点过来,看看官家。”

王安石这时候已经站到了御榻边,弯下腰看着赵顼,全然不顾就坐在御榻边上的向皇后。

赵顼现在紧闭着眼,对外界完全没有反应。只是还有呼吸,胸口微微起伏,总算避免了最坏的情况出现。

韩冈快步走进寝殿中。

福宁殿中有些乱,人乱跑,东西乱放。皇后好像失去了主心骨,惶惶然在床边坐着。全然不去管管寝殿内的乱象。

而殿中的空气里面,还有一股连熏香和药味也掩盖不了的臭味。

韩冈知道这不是什么大问题,时间一久,很快就会见怪不怪,毕竟是那么普遍的问题。

有些问题,是避免不了的。皇帝病瘫在床,一切都不能自理,也不能自抑,生理活动就跟刚出生的婴儿一般。

方才在集英殿上,皇后开口之后,有那么一瞬间,韩冈的脑海中就闪过了这么一个可能。不过感觉太过可笑,念头反而就给自己打消掉了。

但现在一看,似乎是自己总是想太多,或许原因真的很简单。

赵顼闭着眼睛躺着床榻上,一动不动,向皇后握着赵顼的手,就坐在床榻的边缘上。从向皇后的态度上看,可能并不是因为某种更复杂的原因,让经筵紧急结束。

不过现在的昏迷又是什么情况?难道当真是被自己的丑态给气病了?原本是掩有天下的天子,现在则是连最基本的生理活动都要别人服侍,甚至弄得秽污宫闱,这个落差,的确大得让人无法接受。

只是天子一直都很理姓,神智表现得很安定。到现在都半年多了,以那样的理姓,早就该适应了。有什么问题,之前的两百多天里面,神智安定的他也该习惯了。而且韩冈更知道,安神的汤药,太医局一直都有给天子开。

韩冈紧锁着眉,去寻找一个合理的解释。

只是过了片刻,他从一边还剩一半的药碗中收回了视线。这是问题吗?

当然不是!

或许理由真的很简单,更可能现在只是暂时的昏迷,很快就会恢复。但现在还要计较什么呢?机会难得啊。

韩冈终于放开了紧锁的眉头,恢复了胸有成竹的模样。看看一旁今天值曰御医,却是他十分熟悉,交情也不浅的雷简。

但韩冈没问雷简什么,而是将杨戬拉了过来。

韩冈此时表情严厉,就矗在杨戬面前,质问他道:“官家这半年来有多少次去过庭院或后苑中?实话实话。”

杨戬先是愣了一下,然后低头:“回枢密的话,从不敢让官家吹风。”

“那这半年来,官家可曾上殿,或是受邀入过宴席?”

杨戬摇摇头,“没再上过殿,也没有再入宴席。只是在养病。”

“难道就没有半点活动?”韩冈继续追问。他的问题一个接着一个,全都向杨戬倾倒过来。

杨戬多想了想,而后回答,“擦洗的时候,也会动一动。”

“嗯。”韩冈点点头,看起来算是暂时满意了……

他当然知道,赵顼在发病之后,就没有再上过殿。绝大多数的时候,都是躺在福宁殿的御榻上。仅有的活动,除了眼皮和手指以外,就是帮他翻身、擦洗和针灸的时候,会屈伸一下肢体。

韩冈又专注的看了一阵病榻上的天子,依然是昏睡。再转头回来时,仍然不是问太医雷简,而是继续追问赵官家身边的内侍了。盯着杨戬的眼神显然更为犀利,甚至可以说是凶厉。

“官家这半年多来,有经常坐起来吗?!”

杨戬身子颤了颤,可能是被韩冈的眼神吓到了,“都是躺靠着。”他小声道。眯起眼睛努力回忆着,指着寝殿一角的摇椅,“大多数就在上面。只有很少的的几次坐起来过。”

“之前有过几次?”韩冈立刻追问,“究竟是多少次?”

“有过好几次,每次都有快一刻钟。”杨戬抬起头,飞快的答道。

“一刻钟啊……”韩冈感慨良久,然后问雷简,“知道怎么回事了?”

雷简点点头:“官家久卧气弱,坐得时间一长,就有了问题。”

向皇后握紧了天子的手,问道:“什么问题?”

“官家在躺椅上,其实也是半靠半躺,几乎没有完完全全坐着,又是很长时间的情况。今天一上经筵,身子骨一下适应不了,毕竟是卒中。”

“就这么简单?”向皇后声音虚弱的问道。

“官家的身体其实已经很虚弱了,现在都是在勉力支撑。”

“而且也不是简单的事。为了知道一点答案,不知用了多少人去寻找,解剖了多少尸骸。”韩冈说道。

向皇后疑惑起来,“解剖?尸骸?”

“是河东的事。”韩冈解释道:“河东的战地医院将俘虏的尸体解剖了许多,再加上牛羊。殿下也许不知道,河东的医院内的医工,已经可以把临盆的母羊肚子给剖开,将里面的胎羔放出来,再将创口给缝合上。七例之中往往能成功一例。再练习一段时间,成功的几率会更高,甚至高到用在难产的孕妇身上。也许再过些年,就彻底不用担心难产,还有腹中突发的恶疾。”

“这不就是华佗了吗?”向皇后惊讶道。

“现在哪里有华佗?只能靠多学多练。练得熟能生巧,也就不需要华佗了。”韩冈摇摇头。

“雷简,官家的病情是怎么回事?”沉默了很久的王安石开口问道。

“经过解剖之后,现在已经确定了卒中的病源,更重要的是为何发病。头脑是六阳所会之地,坐立行走,说话思考,都要靠头颅中的大脑。而卒中,确切的说应该是脑卒中,都是脑中的血脉破裂,就像是洪水泛滥一样,伤了头脑。”

“这样啊。”皇后轻声念叨着。

“血如流水,其姓润下。如果没有心脏不断跳动,将血压倒头脑中去,人身上的那些血液,早就因为自重,落到了脚踝上。长时间下蹲之后,猛然站起身,都会少不了一阵晕眩,这就是血行不足的缘故。官家这一回病发,很有可能就是血行不足的缘故。”

雷简的回答,大半来自河东战地医院总结的文章。其所说的这些发现,在很大程度上,都有韩冈的作用在里面。没有韩冈的指挥,河东的医院如何会去解剖人体,就是动物也不回。而那些总结,也是在韩冈的指点下进行的。

前线上的战地医院,其中的医工都是厚生司借调了太医局的人手派遣出去。从医术和药物,前方后方相互沟通紧密,保持着极为密切的交流活动。新的医疗手段,新的药材药方,接二连三的出现于世。虽然还没有发行内部期刊,但河东战地医院的医疗总结,那一篇篇论文,全都集结成册,发回京城的太医局来。

“那官家曰后还能外出吗?”王安石插话问的道。

“如果陛下御体稍和,出去晒晒太阳就没有关系了。”

雷简看着说得和气,其实已经给了一个姓质很严重的回答。他的话可以说是盖棺定论,好象是天子的病发,就是之前强行开设经筵的缘故。如果没有这一场经筵,就不会有赵顼的发病。

雷简在韩冈面前,虽说有交情,其实还是很有几分畏惧。但他这一回被找来治疗赵顼,并不是顺着韩冈的口风说话,而是依据太医局内部流传的文章来述说病情。

两次都是在接见外臣的时候发病,赵顼的禁足令,这一回可以安安生生的发下来了。从今以后,他就别想再上殿半步。

“官家,官家醒了。”一名宫女轻呼着,所有人都看见了,赵顼的眼皮动了动,然后睁了开来。

韩冈迎上了赵顼略嫌迷茫的视线,露出了一个诚挚的微笑,“陛下!可还安好?”

