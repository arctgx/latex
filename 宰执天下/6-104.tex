\section{第11章 飞雷喧野传声教(一)}

时近五月,道边的绿意越来越浓,天气也一日热过一日。

路上的行人,也开始避开日头最烈的正午,而改在早晚赶路,但也有不得不在这个时候奔波于官道上的可怜人。

一队车马正顶着烈日,行驶在开封府向西去的官道上,除了中间的一辆马车,前后左右,都是骑着高头大马的护卫。

不过不论是往哪里去的行人,但凡看见道上有这样的一支队伍,都不会觉得他们有哪里可怜,只会立刻避让道旁。

纵然没有打出旗牌,可连仆从都是鲜衣怒马,穿着元随制式的衣袍,坐在马车之中的不是接近宰辅一级的高官显宦,又能是什么人?

也正如道边路人的猜测,坐在马车之内的是当今的参知政事,堂堂的宰辅重臣,作陪的也是朝中数得着的重臣。

目的地在开封府外三十里,骑马也要走上半日,韩冈能在百忙之中抽空出来,完全是因为公务。

“这天够热的。”

韩冈坐在马车车厢中,虽然车窗开着,可车厢中的温度也不见降低多少。

坐在对面的王居卿大点其头,身量中等的他,上车时的动静比韩冈都大,足见他的体重了,在闷热的车厢中坐了一路,早就汗流浃背。拿着手巾擦着额头:“可不是,才五月就这么热了,还不知道了六月七月会多热。”

韩冈微微一笑,好歹还没到夏天,一滴汗没落地就不见了踪影,现在至少还能落到地上。王居卿是有些好奉承,但人无完人,能做事就好。在韩冈的主张下,黄履终于让出了判军器监的位置,由王居卿接任。上任才几日,监中的大事小事已经处置得有条有理,治事之材,并不是靠吹嘘而来。

望着道旁远处一块块的金黄色田地,韩冈油然说道:“接下来的十天半月,要都是这样的天气就好了。”

“参政说的是夏收吗?”王居卿望着与韩冈同样的方向,田地中,可以看见一名名农人正在收割成熟的麦子,在他们身前,是还没收割的金黄,在他们身后,则留下了黯淡下来的土黄,两种颜色泾渭分明:“也的确只有麦子入仓,才能算是安心下来。”

此时正是冬麦收割的时节。晴朗干燥的天气,对收获反而是一件好事。如果接下来的十几天,都是晴天,一直到晒干的小麦入库都是如此,朝廷上下都会松上一大口气——今年的夏税可以无忧了。这可是每年到了夏秋两季,天子、太后和朝臣们都在祈求的事,在希望风调雨顺这件事上,他们与普通的农人没有半点区别。

这段时间京师附近的天气很不错,天朗气清,除了稍稍热一点,什么问题都没有。而京外各路,虽也有上报灾异,但也仅仅局限于一州一线,最多的江南东路春旱,也不过是蔓延到了江宁府、太平州、宣州这江南东路北部的军州,并出现没有席卷一路的大灾。至少今年,应该还是延续着元丰年间的好年景。

前面的马鞭响了两声,叮叮当当的一阵清脆铃声,马车随即转了一个方向,下了官道,转向了另一条道路。

相对于通往洛阳的官道,这条岔路就窄了近一半。不过刚刚整修过不久,车行十分平稳。从车窗望出去是一马平川,两侧的风景,除了路边的屋舍少了一些之外,依然是满目丰收的黄色。

又前行了五六里地,身宽体胖的王居卿早用汗将手巾都浸透了,而韩冈的额头上也能看到了些薄汗。这时候,前方终于看到了一座高出平原甚多的营垒。

营垒傍着一片树林,朝着韩冈这一面的寨墙有近一里的样子,大小可一座小县城相比,规模远超关西边防的千步城。

远远的就能听见轰轰的鸣响,再近一点,硝烟味就更为清晰起来,这是军器监火器局的城外基地。

韩冈眯起眼,望着还在一里多地开外的寨堡,“终于是到了。”

王居卿热得直喘气,附和着:“哈,终于是到了。”

韩冈与王居卿两人的目的地终于到了。

“枢密、天章,方监丞已经过来迎接了。”

韩冈一名亲随在车窗外弯下腰,低声的禀报着。

“让他们过来吧。”韩冈吩咐道。

马车刚停,一群人就迎了上来。

军器监丞兼提举火器局事的方兴走在最前,他早在一个多时辰前就在寨门处等候韩冈一行的到来。传信的士兵也派出去了好几位,等到韩冈、王居卿的车马从官道转过来,便立刻率领寨中能腾出手的官吏,远出寨门来迎接韩冈与顶头上司的视察——要不是知道韩冈不喜欢这一套,他肯定会迎得更远。

大概是因为火炮声伤到了耳朵,方兴以下,火器局的官员一个个声音大得能传出三里地。

韩冈从马车上下来,也不过三十里路,坐车就做了一个多时辰。尽管是高大轩敞的四轮大车,但做得久了,也不免有些腰酸背痛。只不过京师中污染重,又连着多日晴天,风大灰大,韩冈不想弄得满面土灰,便没有骑马,弄了辆马车,拉着王居卿一起坐着过来。

王居卿也累得够呛,他比韩冈年纪大得多,身体差得更远,更受不得累。和韩冈一起受了方兴等人的礼,回头望着来路,“这里离京师还是远了一点,一个多时辰啊,要是能修一条轨道直通此处就好了。万一此处有变,城中援军也可以尽快赶到这里。”

开封府附近没有高山,所谓的山,只是些小土包而已。除去黄河之外,也没有大河、深沟,仅有的几条河水,还都有人工的痕迹。一望无际的平原上,根本就没有什么阻隔。要说起铺设铁路轨道,当然是最好的地形。

“判监说得是。我等来回京城,就想着有一条轨道直通此处就方便了。”方兴附和着,脸上似笑非笑。

真要有外敌入寇,第一目标必然是开封府。而想要打到开封,不论从哪个方向都至少有上千里地,有那个时间,朝廷早就从这座新落成没几天的火炮试验场,把兵力和装备给抽回去了。然后在这里驻屯更多的军队,作为反击的战略要地,卡在敌军喉咙处。至于铁轨,在有外敌控制开封城外的情况下,直接将铁轨一扒,就一点用都没有了。

不过王居卿是新近投入韩冈旗下的重要官员,刚刚被任命为判军器监,方兴虽然是韩冈的心腹,在军器监中又管理着最重要的火器局,也不方便对他的话取笑反驳。

韩冈看得出方兴的真实想法,估计做了几十年官的王居卿也不是瞎子,转头就对王居卿笑道,“寿明你这话要是给沈存中听到,包管要闹起来。”

“这是为何?”

王居卿正为方兴的态度暗暗恼火,这时听见韩冈的话,却不明所以。

韩冈解释道:“沈存中也有意在京中修一条轨道。不过是在城墙内,在城墙根绕城一周。五十里长的铁路,平日可以供百姓乘坐,真要到了战时,运送兵员、物资也方便。”

王居卿立刻反应过来,“开封城墙内的那条路?”

开封城墙内侧,有着很宽的一条城防通道。尽管走的人很少,每隔几年都会对清理一番,禁止百姓侵占。若是作为轨道的基础,就算铺设一条复线铁路,也不会影响到正常的行走。

但他立刻又皱起了眉,东京城门人来人往,而且不止是人,还有马、牛、羊、猪,突然间一列有轨马车开过来,这让都不好让,“城门那边可不好安排。”

韩冈、王居卿两位说着不相关的话,火器局的官吏们不敢打扰,悄无声息的向外让了一步,然后又是一步,绝不敢让人误会是在偷听。

韩冈瞥了这群知情识趣的官吏,笑道:“沈存中正考虑着是下面掘地道传过去,或是架桥越过去。”

“那可难了。”王居卿摇头。

架桥是不难,但想要越过东京城的城门门洞,要架起多高的桥?万一马车从桥上摔下来,砸到人又怎么办?何况要经过四门正门,那是天子车驾经过的地方,如果是守卫倒也罢了,难道要马车从天子的头顶上过去。

而换成是地道,有没有先例不说,怎么让地道不坍塌下来?能并排通过两辆马车的地道,那该有多高,多宽?城门都是车来车往,万一塌下来,这又该如何是好。

“的确是难。但要是解决了,轨道遇到山、河就都有办法了。”韩冈笑道,“不过寿明你和沈存中能这么想的确好,轨道从来都是不嫌多的,要是能以开封府为中心,将府中各县先连起来,日后连接天下各路的干线轨道,就有了仿效的对象。”

“就像是邮政一样,一级级的传下去?”

韩冈道,“既然天子临天下,天威就是这么一级一级的散布下去,那不论是驿传、邮政,还是道路,当然也是得一个样子。”
