\section{第33章 旌旗西指聚虎贲(四)}

喝过了茶,从沉思中惊醒过来。韩冈从袖口中抽一封信。这是游师雄托人寄来的私信,今天才送到手上,还没来及看,就被王韶找了过去。然后听说酒场出事,又往那里去教训几个偷酒的贼人,一直拖到现在。

韩冈打开来看了一编,也没什么特别的。寻常的问好,说些学术上的话题,还有最近几件得意的趣事,顺便的,也谈及了眼下的关中局势。

自从韩冈在麦收时节,离开陕西宣抚司返回通远,到现在已经三个月了。在这段时间中,朝廷在关西地区的战略转移的态势已经非常明显。

陕西转运司一分为二自然是最为明显的实证,但鄜延路和环庆路的平静死寂,也证明了横山南北双方,都在早前的会战中伤到了元气。

尽管凛冬将至,早已到了一年一度的防秋时节,但今年西夏那边不需要太多担心,梁乙埋刚刚解决了几家豪族,虽是稳定了权位,但不得不窝在兴庆府老巢里舔舐.着伤口。

而鄜延路一线的横山蕃部,无论南麓北麓,皆在此前的大战中全数残破。南麓蕃部先是宋军大掠过一遍,接下来又给党项人抢走了几乎所有的存粮。而北麓的蕃部尽管宋人没去叨扰,可他们效忠的主子也照样把他们抢了个干干净净。

没了牛,没了羊,在开春时短了照料的麦田只有往年一半的收成,靠着这么一点粮食,连年节都熬不到。摆在一众蕃部面前的有两条路,一条是求援,一条是抢掠。

抢劫对横山蕃部来说,已是习惯成自然。每年跟着党项人一起南侵,在富庶的汉人身上分上一杯羹,早就是从祖辈传下来的惯例。今年党项人没来,横山蕃部无人领头,聚不起大队,小股盗匪便是层出不穷。

只是新任的延州知州、鄜延路经略使赵禼,以及兵马副总管燕达都不是好招惹的,层层布控,以新组建的保甲为核心,配以精兵强将,将一股股盗贼尽数诛除。鄜延路这段时间没有一次大战,但零零碎碎的斩首,竟然达到了一千四五百之多。千方百计,不让这些强盗抢到半点存粮。按照赵禼向朝廷的报告,只要形势如此发展下去,今冬过后,横山蕃部的人口少说也要减少两成。

也不是没有人选择归附,在正常的情况下,朝廷可能会慷慨解囊,拿出常平仓中的存粮来安抚。但眼下,永兴军路转运司根本挤不出一点存粮,光是白渠灌区的大规模减产,旧年一百四五十万石的收成,今年却仅有七十万石,光是这一项亏空,就让接手转运使一职的吕大防焦头烂额。

蓝田吕氏四贤,只有吕大防不是张载的弟子。但他跟关学一派也十分亲密。游师雄现在正在长安的郭逵麾下任职,而且已经是永兴军路节度判官。这段时间的几封信中,也提过吕大防几次。说这位新上任的权永兴军路转运使,对鄜延路赵禼、燕达的行动多有支持,希望能通过坚壁清野的战术,把时常骚扰宋境的横山蕃人多多饿死几家——即便饱学儒士,也不会傻乎乎的像个东郭先生一样,把仁心放在豺狼毒蛇身上。

横山局势如此,只论王韶出兵武胜军的时机,眼下的确是最为合适的。

在党项人养好伤口之前,穿越大来谷,走到鸟鼠山的另一面。先行打下临洮,控制住洮水,向北可以威胁西夏的西南重镇兰州,向西则直面河州。

天色将晚,韩冈将桌上的文字都收拾了,起身离开公厅。

走出门,望着西侧,漫天的红霞夺目刺眼。

薄薄的云翳被低垂的夕阳染红,仿佛天幕被人划开了一道伤口,殷红的鲜血浸透半幅天空。

韩冈近日多读武经总要,云气占术一篇中有‘赤气漫血色者,流血之象’等语。

眼下大战在即,自然少不得刀锋染血,只是不知这一‘赤气漫血色者’,

究竟是大凶,还是大吉?

……………………

残阳如血。

木征读过汉人的书,跟绝大多是吐蕃贵族一样,对汉人的文化心向往之。看到染了一层血色的天际,不由得想起了这个词。可他再仔细回想,却也想不出来是在哪本汉人的书上看见过。

但木征也不会像汉人的书生那般吟诗作对,看着漫天的红光,只是心惊于这颜色实在不吉利。恐怕也是上天在昭示着很快便是大战降临。

念了几声佛,收回视线,木征走进帐中。

帐内正中有一人跪着,见到木征进来,便立刻五体投地的将脸贴在地上,等着木征发落。

木征面无表情的看了他一眼,坐下来喝茶,也不搭理他。

这是他弟弟派来的求援使节,几天下来,已经看得厌了。

领有武胜军的弟弟瞎吴叱,这段时间以来,一天三封急报,一个信使接着一个信使。说鸟鼠山对面的古渭——现在已经改名做通远还是陇西的——已经聚集了十万宋军,转眼就要攻打过来。第一步是武胜军,下一步,可就是河州了。

瞎吴叱在求救的信中哭诉着,请他念在一母同胞的情分上,还有唇亡齿寒的关系上,拉兄弟一把。

要是真如瞎吴叱所说,宋人真的派来十万大军,饿都能饿死他们。如果饿不死,那就是他木征坐下来等死,拼不过的。

实际上,木征猜度着宋军最多也就是一两万之间,再多了,宋人供给不起——鸟鼠山中的大来谷并不是多好走。

可是莫说一两万,就是七八千就已经很让人头痛了。

武胜军能不能保住,木征并不看好。如今的宋人越来越难缠,这是无可否认的现实。

听说在今年的早些时候,宋军在横山把西夏国相梁乙埋亲领的大军,打得大败而逃。要不是当时宋人正逢上国中内乱,前线被迫回师,梁乙埋说不定都回不了兴庆府,得埋骨无定河畔。

相距千里,木征也分不清传言是真是假。但有一点可以肯定,宋人本来肯定是占了上风,只是因为内乱撤军,让梁乙埋得意安然回到国都去。

因为在来往河州的商队中,多有人在传说此事,众口一词。在他们嘴里,惋惜之辞溢于言表,深恨宋人没能把梁乙埋和他所率领的党项大军留在横山深处——在河西之地,不论是哪一族的商人,多不会对劫掠成性、惯于背信弃义的党项人有任何好感。

木征不想与宋人交战,打起来对他也没有好处。许多时候,木征还幻想着跟他的叔叔交换个位置,让他做着赞普的叔叔,来为自己堵着宋人和党项。而不是眼下截然相反的现状。

可是宋人现在咄咄逼人,都打上门来,也不能不应付。正如瞎吴叱说的,今天宋人夺了武胜军,明天就可能把手伸到河州来。木征很清楚他的居城地理位置有多好,只要宋人有心控制河湟,少不得把河州城占了。

木征慢慢的喝完茶水,把剩下的残渣一起倒进嘴里,咀嚼着里面的酥油和茶叶的清香,一点也不浪费。

思来想去,木征终于有了决断。他对弟弟派来的求援使节道,“跟瞎吴叱说,不要硬打。先避过风头,转到后面我会派人来帮他一起断了宋人粮道。饿着肚子,宋人待不长久。”

木征的话只用了一天便传到了瞎吴叱耳中。

“不要硬打?避过风头?那我这临洮城怎么办?”瞎吴叱脸上没有急怒之色,但语气的尖锐,明明白白的把怒火中烧的心情亮了出来,“难道留给宋人不成?!”他质问着。

使者跪在地上,不敢抬头,任凭瞎吴叱发泄着怒气。

董裕死了,排在老三的瞎吴叱,好不容易继承了他的这块地盘。一年来,他费尽心力的去治理武胜军的各家蕃部,只想把这片洮河边的土地,打造成不逊于青唐、河州的富庶去处。

但他一年来的心血结晶,长兄木征轻飘飘的一句话却要他放弃。看着人口渐多的城市,还有自己所居住的新修豪宅大院,瞎吴叱如何能舍得丢下这些他视若珍宝的产业,而窜入山间躲避宋人兵锋?

如果依照吩咐放弃了临洮城,他的大哥真的能派援军来救他吗?

瞎吴叱不愿把希望寄托在木征身上,但其他方法他又不好说出口,他环视厅中,他所领有的几十个大小部族的族长如今都在这里,他们中间有许多并不是只投靠了一家,相信他们中间,有人能先出头来,说出让他满意的意见。

“要不要向禹臧家求援?”帐下部族中的长老有人提议着,

这项提议让厅中的族长们都私声交谈起来,反对者有之,赞成者有之。而赞成者中,有人怕禹臧家来了就不走了,也有人觉得要请动禹臧花麻不是那么容易。

瞎吴叱咳嗽一声,阻止了下面的纷纷议论,他点一个聪明伶俐会说话的亲信,“你去带信给禹臧花麻,把唇亡齿寒的道理说给他听,并说如果功成,将把武胜军北面靠近兰州的那一片割让给他,请他率军来救援。”

不顾下面的窃窃私语一下响亮起来,瞎吴叱又道:“你走之前,去库中一趟,里面的财物觉得有用的尽管搬,只要能把禹臧花麻请来,搬多少都随你。”

瞎吴叱大方的说着,仓库里的财物是这一年来积攒下的,只要能保住临洮城,今天送出去的,一年后就能补回来。而割去的土地,只要等禹臧家与宋人打得两败俱伤,他也可以摇头不认的。

先把禹臧家的兵诓来再说!

