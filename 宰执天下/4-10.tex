\section{第一章 纵谈犹说旧升平(十)}

韩冈前几天在崇政殿上的提议,通过正式的奏章已经递到了赵顼的案头上。政事堂两方对立,支持的和反对的各占一半,最后只能交由天子圣裁。而赵顼到现在为止,也没能拿定主意。

虚外守中的国策,五代时臣弑君、下克上的混乱,还有契丹人始终存在的威胁,让缺乏安全感的大宋几代天子,都习惯性的将最强有力的东西留在身边。能放在京畿就不会放在外路,能搬进京城,就不会留在城外。不论是军队,还是作坊,都是一样。

军器监负责钢铁锻造的工坊,直接为大宋的百万大军服务,从机器到产品,都是赵顼恨不得能藏进宫里才放心的宝贝。韩冈要将板甲局的几个作坊移出城外,即便仅仅是放到已经归属开封府管辖的旧郑州,依然有着很大的阻力——在朝中,更在赵顼的心里。

“王卿。”这一日的崇政殿议事之后,赵顼留下了与韩冈最为亲近的王韶,“韩冈要将军器监的锻造作坊迁往汴口附近,以便利用水力,不知王卿你觉得此事是否可行?”

“当移!”王韶肯定地点头,不出意外的支持韩冈的提议,“近二十万京营禁军的家眷,有一多半就在京城之外,不见军中不稳。只要板甲局作坊留在京畿,何须在意是否在一道土墙之内。”

“但万一军器监中的机密泄露,又该如何是好?”赵顼忧心的就是这件事,“作坊身处城内,可以严加防守。而一旦出了城,又该如何封锁?”

“陛下有所不知。岷州滔山监虽以铸钱为主,但立监时起就开始使用水力锻锤,以用来修补甲胄和刀剑……这是由一名来自景德镇的配军所献。”王韶半真半假的说着,这等小事根本无法查证,“而景德镇用水力锻锤粉碎瓷石,已经有几百年,用者甚多,能造此物的工匠亦为数众多。守秘亦是无用。”

赵顼有些不理解:“那为何韩冈还要悬赏征求水利锻锤,又要请了苏颂出来?”

“打造板甲,与粉碎瓷石、修补甲叶、刀剑,在形制当是有所差别,故而韩冈才会再以重金悬赏改进的水力锻锤。”

话题就这样绕回来了。“既然是新式锻锤,那必然是机密,难道就不需要守秘?”赵顼反问着。

“韩冈才智虽是出众,但他的一干发明,都不是机巧之物,只是难以想到而已。飞船、锻锤、板甲、霹雳砲、雪橇车,无一不是制造简便,易于打造。自然,也就是轻易便能仿效,难以严守其中之秘。只看如今七十二家正店门前便知端的。”王韶偷眼看了一下脸色沉重下来的赵顼,“不过西北二虏国力远不及中国。中国能在两三年内打造百万兵甲,西虏北虏即使合力,十万亦是难及。与其遮遮掩掩,耽误时机,不如尽快给五十八万禁军整体换装板甲。等到国朝兵利甲坚,严阵以待,西北二虏又何敢再欺中国无人?!”

王韶站在韩冈这一边是没错,这番话大半也是转述,但如果韩冈说得没有道理,王韶也不会在天子面前为其张目。

赵顼沉默半晌:“依王卿之意,就是泄密亦无妨?”

“非也。”王韶摇摇头,“此举正是为了防止泄密。”

赵顼闻之一怔:“此话怎讲?”

“陛下明鉴。板甲所耗人工仅及札甲十一,所用人力当然也远少于旧时。札甲诸作转入板甲局者,只有三一之数,若是不为其余人等找一条出路,便会有数百上千名工匠成为冗员,最后被扫地出门。万一其中几人叛国而去,投奔契丹、党项,其后果当不下于张元、吴昊多少。”

“此事韩冈为何不……”惊讶不已的赵顼说到一半,就已经恍然大悟。

如果韩冈直接说原本打造札甲的几百名工匠已经一起没了差事做,如果不加以处理,就会被被军器监扫地出门,朝中必然会有人借题发挥。韩冈隐而不谈似乎也是有道理的。

“只是朕就这么是非不分吗?”赵顼有点不高兴。只是想想上元节的事,他又叹了一口气,韩冈当是怕了政事堂中的那几位,“此事朕就准了,不过军器监中工匠们都要安置好,不要给朕出乱子。”

“陛下圣谕,臣必会转告韩冈。”

……………………

七八天的时间不算很短,当初打造板甲也就几天工夫。但韩冈所要的轮轴轮毂却没有收到一个让他满意的回复,尤其是铁铸、钢铸的轮轴、轮毂根本不可能在几年甚至十几年内给弄出来,要在钢材的材质和车床技术上有大突破才行。韩冈也明白,能像如今的上等马车那样,在轮子外缘钉上一层铜皮就很了不得了。

木质轨道倒是出来了,现在只有二十丈长,占了一条僻静的巷道,十几名工匠正准备打造有轨马车,除了轮子,其他部件都已经准备完毕,与普通的马车根本没有什么区别。韩冈估计最多也就一两个月,便能见成果了——兴国坊的军器监中是天下最不缺高手工匠的地方,技术水平达不到那没办法,可只要技术条件许可,韩冈要什么,工匠们都能给个满意的答复——再试行一段时间,加以改进,便可以推广到矿山之中。

除此之外,韩冈的奏章也终于被批复下来,几个锻造作坊终于确定了可以迁往汴口,而当地的水力磨坊将会在一年间逐步撤除一半,以给军器监腾出空位来。

拿到圣旨,之前一直如同深海鱼一般在军器监中洄游的消息终于得到了证实。铁甲、钉钗、铁身、纲甲、柔甲、错磨、鳞子、钉头牟等八作的作头,加上十几个工匠头目,还有没有调入板甲局的数百名匠人,一起被召到了韩冈的面前,将正堂的大院,挤得水泄不通。

打造札甲的八个作坊中,水平出众的工匠早已被韩冈调动到了板甲局中,加上尽力塞进去的一部分小工,归入新局的人数占了其中总数的一半左右。而剩下的匠人并不说不能用,只是已经尽力扩充的板甲局中,塞不进更多的人了。

等待他们这些工匠的未来,拿后世的话说就是下岗,以如今的词汇则是沙汰——像筛沙子一样淘汰掉不再需要的冗员。

“……不过本官不是这样的人。”韩冈冲着几百名眼中满是期待的工匠们高声说着,“既然夺了你们的差事,当然会为你们找个出路。想必你们都听说了,本官奉旨设立板甲局之后,就奏请天子,将局中作坊逐渐移往汴口,以便利用水力。而军器监的铁器锻造,将会扩大规模,转出一部分打造农具,而将不仅仅限于军器。今日天子已下恩旨,你们之中只要想留的,就都能留下来!”

一片欢呼声猛然响起,几百名落选板甲局的工匠,担惊受怕了这么些天,终于可以安心下来陪着家人享受春光了。

韩冈挥挥手示意他们散去,笑着转身进屋,却见一人匆匆而来,附在他耳边低声说了几句。

来人是韩家的下人,周全认识他,见到韩冈闻言脸色微变,就立刻问道:“舍人,出了何事?”

韩冈神色恢复平静,淡然一笑,“一百多汴口水磨坊的人,方才进了城,正一起往家里去了。”

周全听着就顿时大怒,须发皆动,一声暴喝:“好狗胆!”

“拿根长棍子去拨树上鸟雀的巢,把它搞下来,雀儿也要叫几声。周全你也有一个巢,我把你的巢搞烂了,你要不要叫几声?”韩冈哈哈笑着,“世间的道理都一样啊,古今中外概莫能外,不给人一条出路,有多少人会忍气吞声?”

砸人饭碗,若是安抚不当,肯定会有乱子,韩冈当然不会没有心理准备。减员增效四个字韩冈当然知道不是那么简单。曾孝宽这段时间,一直在刻意减少来军器监的次数,以防与声势正盛的韩冈对立。曾孝宽的放权,也使得韩冈就必须一人担起责任。

听着韩冈的口气,仿佛在体谅磨坊里的人,周全就奇怪的问道:“那舍人为何要去抢他们的地盘?”

“树就那么大,能做窝的树杈就那么几个,不去抢,怎么做窝。”韩冈脸上的笑意,随着话声一点点的变得冷了下来,“我是判军器监,当然要顾着自家人。”

周全一个劲的点头,想了想,却又问韩冈:“那如果舍人在三司里做事,还会帮着军器监吗?”

韩冈哈哈笑了两声,并没有回答。却是反问道:“周全,你说现在该怎么办?”

周全一挥铁钩,恶狠狠的喝道,仿佛又回到了当年战前请命的时候:“舍人!让下官带人去堵着那些个磨坊里的驴货,废了领头的几个,看他们还敢再闹事!真当我军器监里的汉子都还是吃奶娃儿不成?”

“你有这个心就行了。”韩冈微微一笑,要想做事,有些事就是免不了的:“一个饭碗两家争,磨坊的人已经进城来闹了,你将人约束好就行了,韩缜不敢看我的笑话,你旧时的那些兄弟也不是白白吃饭的。”

