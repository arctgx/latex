\section{第39章 欲雨还晴咨明辅(20)}

让人给王安石送去了帖子,韩冈正在给奏章打草稿。

他当然不觉得给王安石写一封帖子就够了。

那只是通知,事先通个气,以免伤了和气——尽管不伤和气的可能姓不大,但场面上要做好了。

接下来就是怎么将铸币权给总于手中。

以韩冈在铸币和发行上的发言权,想要从三司使手中将他想要的任何位置抢过来,根本就不需要知会哪一位,只要上表就够了。

连视韩冈为仇敌的吕嘉问,都承认韩冈有关改变钱法的提议能行之有效,且准备去实施,那么还会有谁有反对意见?

他的身边,云娘正拈着一块墨,在一方外形古拙的砚台上小心的磨着。

一手捏着延州油墨,一手按住洮河砚,白皙修长的手指微微用力,平时有些跳脱的云娘,此时却是沉静而专注。

延州油墨是一个误会的产物。韩冈想要普及更简单的印刷技术,需要更适合印刷的油墨,但他写信让家里招揽匠师去制作,只是没有写清楚,所以最后弄出来的却是延州石液——也就是延安的石油——灼烧取炭而制作出来的墨块,并以此而命名为延州油墨。成为了雍秦商会中的又一具有地方特色的产品。

洮河砚则是如今洮州最有名的特产之一,色泽沉绿,质地坚密,在唐时就名传天下,与端砚,歙砚并称。柳公权在他的《论砚》中也说‘蓄砚以青州为第一,绛州次之,后始端、歙、临洮’。晚唐失洮河,临洮砚也就失去了源头。可自从熙宁五年收复了熙河之后,洮河河底的绿石就重新被发掘出来,许多制砚名匠被招揽了去洮州,制作砚台。

近水楼台先得月,韩冈家中所有用墨和砚,都是这两样。

韩冈不喜欢在自己写作的时候,有不相干的外人在身边,就是家中的婢女、仆役也一样。如果是在内院的小书房中写东西,就只会让妻妾过来打下手。

云娘帮着铺好稿纸,磨好浓墨,然后坐在一旁,静静地看着韩冈在书桌前凝神书写。

手托着香腮,微带褐色的剪水双瞳中,如同两汪深潭,映照着韩冈高大的身影。这是她平常最喜欢做的事。只看着她的三哥哥,心中便是一片平安喜乐。

韩冈静心凝神,正斟酌着落笔的文字。

在闭门辞官的时候为国事上书,说起来有些忌讳,不过他说的又并不是军事,也不会有人太较真。

坑冶铸钱之事,以及诸州铸钱钱监,是由三司铁案来掌管。

韩冈的打算,是依从三司盐铁司胄案升格为军器监故事,将三司盐铁司铁案中的有关钱币铸造的部分给分离出来,升格为铸币局。然后安排得力的人手去管理。

所谓三司。是盐铁司、户部司和度支司的统称。三司使总领三司二十一案,经理国家财赋、土木工程和百官俸给。是两府之下,最为重要的执行机构,没有之一。

三司使之下,是分管各司的副使。三司盐铁副使、三司户部副使和三司度支副使,各管朝廷财计的一滩事。

户部司,辖下两税、曲、上贡、修造、衣粮五案,也就是五个部门,掌天下户口、税赋、簿籍。除此之外,还有酒水专卖,百工制作和官服、军服的储备。

度支司,辖下赏给、钱帛、发运、斛斗、骑、百官、粮料和常平八案。掌诸路上送财赋总数,每年计量出入,以规划朝廷之用。从战马的口粮,到官员的俸禄,还有税赋的运输,都在其中。

盐铁司,辖下兵刑、胄、铁、商税、茶、设、课盐和末盐八案。主要是工商税收,包括朝廷专卖的盐、铁、茶。其中颗盐是陕西、四川的井盐、池盐,而末盐则是海盐。不过胄案早就改成了军器监,从三司独.立出来。其实现如今只有七案。

韩冈想要的仅仅是其中盐铁司辖下的铁案,掌五金、朱砂、白矾、绿矾和石炭的开采、冶炼,同时也包括铸钱的铁案。他给王安石的短笺中,也只是提及铸钱而已。不过各地钱监下面都有各自的矿坑,矿冶终究还是要涉及。

韩冈计划中的铸币局。想要从铁案中独.立,最好就是要将铜铁料的铸造,与采掘、冶炼分开。否则坑冶的人事管理都在兵刑案中,若是有坑冶连着钱监,韩冈的铸币局根本无法从旧时窠臼中脱身。想怎么安排就怎么安排。

只是现在,矿产的开采、冶炼和铸造——主要是铸钱,都是一条龙。很多钱监正管理着矿坑,要不然这些开采和铸造的大小事务,也不会都归于一个三级衙门来管理【三司——盐铁——铁案】。如果抛弃了矿冶而读力,那么铸造钱币的原材料就都要另外编订账目收购,手续上就要麻烦了一层。

这也是韩冈所要面临的两难境地。这个时代的官僚制度,约束了国家的发展,许多时候,甚至是有着反作用力。

国内绝大部分的矿场,都是属于国有,却包给私人。分成一家家坑户,各自冶炼。官府所需要的矿产品,是从坑户手中购买。收取两成作为实物税,剩下的则出钱收购。矿石从矿坑中运出来后,就交由坑户冶炼,然后官府收取成品。所以旧年徐州有十万冶户,江西的铜矿,同样是以万为单位。

不过随着钢铁工业的发展,煤铁共同体的出现,徐州的钢铁冶炼早就收归国有,矿工、冶工都从官府手中拿工钱。大规模生产所造成的成本降低,让许多地方的铁矿坑户破产,但也有些地方,坑户中的大户——号称——自己修建高炉,破产的有,成功的也有。

由于可怜的管理能力,大宋国内的钢铁实际年产量,根本是统计不了的,朝廷也只能统计出掌握在各级官府手中的数字。但即便是那个数字,也是如果给耶律乙辛知道了,大辽尚父的脸大概会变青的等级。

几个完全是雇工制的大铁场,已经从铁案中分离出来,归属了将作监管辖。但其余矿物的开采和冶炼还是在铁案之内,而韩冈现在着意的钱币铸造,同样如此。

这样的隶属关系看起来很乱,其实是很符合这个时代工业的地位。

国之大事,在戎在祀。

几千年保持着这样的理念,对于工农的认识,除了户口,就是税赋。重要的是礼仪、官僚和军队。对生产和发展并不放在心上——会在意生产,只是因为统治的稳定和国家的税收,除此之外再无它物。

没有足够的引导和需求,工艺的进步,都是如同蜗牛在爬行。

可一旦有了正确的引导和需求,技术进步的速度就会快到难以想象。

最好的例子就是钢铁业,自从有了板甲这个巨大的需求量之后,才几年功夫,从运矿出坑的轨道,到冶炼钢铁的高炉,全都一个个出现在眼前,这些都是过去不敢想象的发展。如果再向外拓张,则进步的速度可能会更快,说不定转眼之间就变得沧海桑田了。

铸币局如果发展起来,其中的钱息都是国家财计很大一部分收入。毕竟在这个时代,大宋的钱币是周边各国所通行的货币。有那么多坑可以注水,那么多铸币局那边放放水也很正常。造得越多,铸币局赚的就越多。尤其是高价的钱币,越是面值高,其中所含的利润也就越高,铸币局也就越赚钱在朝堂上的都是现实主义者,有好处的,谁都不会丢到一边。有那份需要,就会放在手边。

就像以京师、徐州为首的几大铁场,之所以能从三司盐铁司的铁案中分离出来,归入了中书门下直辖的将作监,正是因为钢铁业地位上升,能给朝廷带来更多收入和好处。而军器监从胄案中分离独.立,也是因为国家开始重视军器制造的缘故。

只要币制改革能够给朝廷带来更多的收益,那么铸币局从三司转入政事堂,那是顺理成章的一件事。甚至不需要韩冈多说什么,到时候,有的是人主动去抢。

所以现在,韩冈打算的仅仅是财物、人事分离,并不准备将之割离三司,这样一来,也能少上许多阻力。等到结果出来之后,人人争抢,到时候,三司也来不及后悔了。

一篇奏章一蹴而就,经过了这么多年的联系,韩冈在书写公文上的水平已经是一流的了,就算文采不高,可精确的用词,简练的语句,都是公文所看重的。

稍稍修改了一番,韩冈随即誊抄起来。抄写完毕,重新检查了是否犯讳,又是否有错别字和不通顺的语句,韩冈随即署上了姓名。

不过收起了这一份奏折,韩冈并没有离开书桌。

让韩云娘重新铺开一张稿纸,韩冈又重新提起笔。

不过他的笔停顿了很久,似乎是在犹豫。就这么对着空白的稿纸,过了好半天,韩冈才提笔在纸上写了两个字:

国债。

