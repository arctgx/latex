\section{第40章 岁物皆新期时英(三)}

“那莫不是韩宣徽?”

包厢之中随着这一句话,陡然间起了搔动。

五六名身着襕衫的士子,全都没了平曰的修养,一齐拥到了窗边,向下张望着。

唯有宗泽坐得纹丝不动,双眼直勾勾的看着桌上的菜肴。

以他在京城中的名声,如果想要去韩府上拜访,不会等待太久就会被招进去。只是他没什么兴趣,不打算刻意去拜见,现阶段的所要注意的重点,难道不是礼部试吗?

至于当下,宗泽现在更想着怎么趁这个机会多吃点,不能浪费了自己为此付出的十贯大钱。

要不是给两家快报报社撰写专门的评论文章,宗泽也拿不到那么多的现钱来贴补平时曰用。不过也不会被同一个住处的同窗起哄,最后不得不答应下来去内城的正店好好喝上一巡酒。

“怎么就收起清凉伞了。”有人挤在窗前,惊异的问着。

好几个人立刻一句接一句的回答:“听说韩宣徽在两府中,是出了名的最不喜欢用仪仗。除了上朝放衙,平时出外,都是只有十几人护卫。”

宗泽不忘吃菜,筷子不停地动着。难得来内城吃酒,却一个个只顾得看人,太浪费了这满桌子好酒菜了。

不过是见到一张清凉伞,何必这么一惊一乍。宗泽如此想着。

比起外城靠南门的国子监,内城之中,能遇上重臣的机会要大许多。州桥夜市上的,哪天都能看见一名名金紫重臣的从面前经过,但南薰门,见猪的几率可比见官的几率要大多的。如果再算上官员中那些跟猪没两样的一部分,这个差距就更大了。只是国子监中,没见过清凉伞的学生又有几个?五十里京城内的直到目送了韩冈一行拐进了另一条街道,在窗前看热闹的国子监生们,这才一个个回到了座位上,只是议论的重点,依然还是已经远去的韩冈。

“真的是韩宣徽。”

“若是能当面请教,聆听学问就好了。”

宗泽埋头吃菜,就算韩冈出面讲学,他也不觉得现在有资格去凑那个热闹。

要去听讲,先得要打好基础,得将有关气学一系列书籍都通览一遍,还要随时去《自然》上最新的文章。韩冈所主张的格物之学,不是读两本经书就能明白的,什么底子都没有,去听了也只是浪费时间。

《自然》已经刊行了三期,宗泽每次都在第一时间弄到了手,只是好多地方都看不懂。如果是有关生物、物理和化学的部分,依照文章中的内容做个实验或是实际观察一下也就能明了了,但若是是有关数算的部分,实在是看得头疼。

宗泽出身于两浙商人家庭,论起算学,在座的没有一个能比得上他,可惜就是七八岁便能将九九歌诀倒背如流的宗泽,也一样对《自然》中的那几篇有关开方、勾股,还有天元代数法的论文感到头疼不已。

相较宗泽而言,他的同学们就简单多了。国子监中的学生,最早对气学大多不是很看重,只是对那些实验有些兴趣,但还是视之为小道,但等到这一次韩冈回京,在殿上宣讲华夷之辨,鼓吹对外扩张,立刻就在国子监中掀起了轩然大波,太学生们对气学的态度就变了。

与宗泽交情比较好的一群同学,现在最喜爱也最欣赏的就是韩冈新近针对华夷之辨的一干理论。作为气学圭臬的横渠四句教中的为万世开太平一句,不再是空口说白话,而是有了切实的理论基础,同时目标和方向也都从中衍生出来。

“那些蛮夷,空占了那么多土地,却只知刀耕火种。换了我中国之人,开沟洫、辟田地,再差的地也能种出粮食来。”

“想想幽燕十六州,到现在才多少人口,如果换做我中国据有此地,又能安置多少人口?”

“浪费啊。那么好的地,那么大的平原,却给北虏拿来做牧场。这不是浪费是什么?”

“再说南方。南征平交之前,广西才有多少出产。现在呢,每年的粮食都有百万石。”

“岭外之地,出了州城,就是蛮夷的地盘了。想想吧,几十万、多不过百万的蛮夷,占了两路之地。只看官军南征灭交趾之后,两广的出产多了多少,就知道过去浪费的究竟有多少。”

“最好的办法还是改土归流。”

宗泽微笑的看着同学们的高谈阔论。

宗泽也挺喜欢横渠四句教中的气势,也认同韩冈对自然万物的看法,以及蛮夷、华夏的区分。尽管最后终究是要从四方蛮夷手中夺取土地,但必先‘老吾老’、‘幼吾幼’,方能‘及人之老’,‘及人之幼’的道理,他还是懂的。

不能提供足够的土地耕耘种植,死的就会是华夏子民,或是在襁褓里就溺死,或是在诚仁后,遇上灾异而饿死、病死。如果想要华夏子民能够安心的生活,就必须要将蛮夷手中的那些土地给夺取、并开发出来。

虽然说物竞天择、适者生存的道理真要计较起来,也可以把韩冈鼓吹的对外扩张一起算进去,与华夷之辨正好相冲,可谓是作茧自缚。但感觉上,韩冈的确是将世情给说透了。

就像现在的国子监,两千外舍争夺三百个内舍名额,而三百个内舍生则要去争夺一百个上舍生的空缺。而且上舍生要得赐进士及第、进士出身,不是一年两年的事,加上不进则退的一批人,每年空出来的上舍生的位置,也没有超过二十个。这样的竞争叫做什么?正是适者生存!不能适应的就要被淘汰掉。太过于符合现实,随时随地都能见到印证的例子,让‘物竞天择、适者生存’这八个字,在国子监的师生中‘于我心有戚戚焉’。

现如今韩冈在国子监中有许多支持者。都是年轻人,都是满腔抱负,都是亲眼见证国家从衰微转向强盛,也都羡慕着韩冈、章惇、吕惠卿等年轻一辈的功业和际遇。

既然如此,他们又怎么会选择保守内敛的旧党。新党,则在国家扩张上,并没有一个合理的理论为根基。而韩冈的华夷之辨的新解,却是给了朝廷一个名正言顺开疆拓土的道德基础和必要理由。

华夏不是蛮夷。

如果是蛮夷,只要有一个名气大的酋首振臂高呼,我们要去抢汉人,那么多丝绢、女人都是我们的,立马就能拉起几千上万的人马。而中国要征伐四方,必须要名正言顺,或吊民伐罪,或征讨不臣,总之,要有一个大义的名分。

韩冈所给出的名分,就是再充分也不过。

吊民伐罪也好,征讨不臣也好,国中都会由反对的声音,因为从理由上,那毕竟只是脸面问题,而劳动的却是百姓,消耗的则是国库。太祖皇帝的‘卧榻之侧岂容他人鼾睡’,不过是帝王想图个安心稳睡,与百姓无关,朝臣们想要反对,都可以拿民生、民心来做理由。

而韩冈的华夷之辨新解,却是由人口数字着手,得出的结论无可争议,让世人都能明白开疆拓土的必要姓,那是事关大宋和亿万百姓生死存亡的关键。

“汝霖,你怎么看?”终于有学生发现了宗泽一直没有发言,只顾着吃菜喝酒,“辽国和南海,究竟该先攻哪一边?”

“现在打辽国做什么?韩宣徽辛辛苦苦才祸水东引。”宗泽放下筷子,喝了一口淡酒漱口,不紧不慢的道:“开国时,太祖从赵韩王【赵普】先南后北、先易后难之策,才有了今天。若是一开始,万一败了,之后哪里还有力气去征讨南方?”

宗泽在国子监中以精通兵事而闻名,他这么一说,一名同学就得意的叫了起来,“我就说嘛,自然还是南海最好。”

“那朝廷成立水军,又这么急的派人护送高丽国使回国又是为了什么?!”

“还不是希望高丽能牵制辽国,官军南下可以后顾无忧。汝霖,你说是不是?”

问题又被传到了宗泽这边,宗泽想了一想,道:“不过征战之事,不是儿戏,未虑胜先虑败,即便只是南海小国,官军也不是不可能无功而返。当年太宗伐交趾,不也是困于水土,最后不得不放弃了吗?万一不幸出了意外,官军被追击到国中该怎么办?”

宗泽答非所问,几名同窗都没听懂,“汝霖,这是什么意思?”

“朝廷成立水军正是因为上面的顾虑,为了败不至患,选择不走陆上,而走海路。就是在海外战败了,也不用担心贼人的大军能反攻大宋。想攻则攻,想退则退,进退自如,对将帅来说,没有什么仗比这更好打了。比如现在对高丽的支持,这边能通过水军在海外消耗辽国国力,而河北、河东却不用担心辽国的骑兵。没有比这样的战争更为舒心惬意了。”

许多人因为宗泽的发言而陷入了沉思,但还有人不服:“可海上风浪,军器粮草能有多少送到海外的官军手中?”

出身两浙,宗泽对海贸多有了解,“就算是风浪再大,至少都能有一半抵达。而且船只曰夜都在航行,速度比陆路更快。《桂窗丛谈》都看过吧,里面说起各种输送粮草的方法和消耗,陆上千里转运,若只用人畜,一路上都要吃饭,最后能送到前方的不过十分之一。车运好一点,轨道和纲船水运就更好了。但最多也不过七八百料的纲船,还有不铺好路就不能用的轨道,能比得上动辄万石的海上巨舟?一艘海船抵得几百上千台车,百姓也不用受转运之苦。这样的战事,对国库财计的消耗是最少的,朝廷也打得起。”

没人还能提出反对的意见,宗泽重又拿起筷子,“以宗泽愚见,朝廷现如今所考虑的不是先南还是先北的问题。中国人口只会越来越多,战事也不会有休止的时候,长生不败的战法世上从来没有,那么就必须有一个败而不损的战术。高丽,正是这样的一个实验!”

