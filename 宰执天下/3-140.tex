\section{第37章 相叹投残笔(中)}

雨后的黄河波涛汹涌,浊流滚滚。

原本只在河床中心地带的河水,此时已经快要漫到大堤前,眼看着就要一波一波的开始冲击着刚刚夯筑好没有多久的黄河大堤。

河中的滔滔洪流,是来自于陕西、京西的秋汛,涛声如雷。滔滔黄河水尽管离着堤面还有半丈多,可比起另一侧的白马县地面,整整要高出了三四丈。如果大堤溃破,堤外的一片土地上,洪流将纵横驰骋,再无地势能阻。

站在大堤向下望久了,普通人少不了就会有些头晕目眩、双脚发软。而韩冈带着一群人走在比寻常官道还要宽阔几分的大堤之上,也是很注意的行在中间,尽量远离河面。此等洪流,如果落水根本就是没有救的。

此时的黄河大堤已经不复几个月来的热闹,放眼望过去,这一段堤岸上冷冷清清,只有韩冈一行三十多人。

就在一个月前,白马县一段的河堤提前完工,高度虽然只增加了三到五尺不等,不过厚度却平均增加了三分之一,并且在几处河道转弯、容易破堤的位置上,不仅仅特别加厚,于大堤内侧,更是增筑了几道用以阻洪、称为月堤的小坝。

宽阔的大堤内部主体还是黄土,不过外层则是用的是石灰、河沙加上粘土混合成的三合土,厚厚的夯筑起来,现在已经坚硬如石,不惧水泡。走在刚刚下过雨的大堤上,木质的靴底夺夺响着,如同踩着石板路上,一点泥浆也没有。

韩冈沿着大堤走了一阵,对这一工程质量很是满意。只要常年不懈的检修,大堤主体保上三五十年应该没问题。

王旁走得累了,停了脚,对着韩冈道:“今天又有一批流民北上返乡。恐怕不等到了冬天,人就都走光了,要筑内堤可是没办法了。”

说是这么说,可王旁脸上的表情与所说的内容完全不同,笑得如释重负。

“自由来去嘛。”韩冈也是很放松的笑了一笑。

流民逐渐北返,回家乡去播种,也就代表着他安抚流民的任务也即将结束,整整一年的辛苦,如今也告一段落。日后要筑内堤,拿钱征召本地民夫也没问题,并不需要今年赶着用流民来完成。

方兴跟着道:“如今洛阳、大名的外堤增筑都没有完工,北岸甚至大部分都没有开工。以眼下的进度,没有个三五年,外堤不能建功,内堤也难动手。”

“不过朝廷难得下了决心,要重新整治河防,即便要耗上多年时间,以亿万计的钱粮,天子当是心甘情愿。”王旁望着滚滚激流,半年多来的用心劳苦,神色中已多了一点深沉和稳重,“若能洪水不再为患,京畿百姓当也是乐意出上一份力。”

“回去还得想想到明年该怎么办吧。”韩冈说道,抬头看看天上乌云密合,又要下雨的样子,便开始往回走,“河北那边虽然能开种了,可还是照样要救上一年的荒。而开封这里,也都是一样。到明年五月收获前,赈济的工作还得继续。”

游醇叹道:“要不是蝗灾,白马县的春麦收成也不至于只能用到年底。”

方兴则道:“幸好雨下得是时候,要不然就只能吃到冬月。”

因为蝗虫的缘故,白马县春麦的收成只有应有的一半。只是有一点算是运气,县中的春麦刚刚收获并晾晒完毕,就开始下雨。如果雨下得早两日,就又会损失一批宝贵的粮食。

王旁道:“整个开封,白马县的情况已经算是最好了。其他诸县,补种的春麦也几乎都没有收成。”

“这些事还是回去再说吧。”韩冈说道。

从上堤的位置下了大堤,韩冈一行人骑上马向着县城去。此时将及傍晚,途经的两座流民营中的炊烟比起前些日子要少了许多,韩冈没有下马进去查看,而是从门前打马而过。

抵达县城时,天色已经黑了,不过雨还未下。

韩冈进了提点司衙门,留守的魏平真便迎了上来。韩冈一边与他说这话,就准备往公厅去,王旁就说道:“二姐就要生了,玉昆你还是多陪陪她。衙门里的事情明天再处置也不迟。一干文牍,我等整理好了就送来给玉昆你看。”

王旁如此说了,方兴、游醇、魏平真纷纷点头应是。

王旖此时已经到了预产期,挺着肚子,随时都有可能分娩。韩冈心里也担心着,不推辞王旁几人的好意,点了点头,“劳烦各位了。”

方兴哈哈笑道:“就要有官做了,累着也甘心。”

魏平真稳重,游醇矜持,但听了方兴的话,都忍不住有了点笑容。

如今可以肯定,因为安置流民之功,韩冈必然要受到嘉奖。而跟着他一路辛苦过来的方兴、魏平真和游醇,韩冈已经将他们的名字都报上去了,不出意外的话都能得一个官身。

做官可要比做幕僚强得多,光是从民籍升到官籍,就能让家人不再受赋役之苦,更别说日后有机会荫及子孙。有几个给人做幕宾的不愿意做官?就是因为做不了官,才给人当幕僚。魏平真和方兴跟着韩冈辛苦受累,就是看好他的前途。而游醇尽管也准备考进士,但他也不介意先得一个官身,这样得到贡生的资格也会容易许多。

至于王旁,因为他早就荫补为官——正九品的太常寺太祝——所以在七月的时候,韩冈为了方便起见,就荐了他入提点司,担任勾当公事一职。天子一开始不同意,说这个职位太过低微,当是以选人出任,而王旁已是京官的身份。不过王安石劝过之后,天子才点头下来。

韩冈回了内院,王旁与魏、方、游三名幕僚一起整理着今天送来的文牍档案。用了半个时辰整理好,王旁就亲自拿着,往后院去找韩冈。

走进书房的时候,韩冈正看着一封书信。听到王旁进来的动静,就抬头道:“沈存中要调回来了。”

“沈存中……是沈括?!”王旁见过沈括,熙宁初年的时候也经常进出家中,只是混在一群小官里,印象已经模糊了。见韩冈提起他,坐下来问道:“他前面在哪里任职?怎么调回来了?”

韩冈笑笑:“熙河路经略司机宜任满回京。他所制的舆图、沙盘,可比我所献上的当年要强多了,天子看起来就准备用他这个长处。”

沈括在熙河路经略司接替的是韩冈的职位,做了两年的机宜文字。在这段时间中,沈括走遍了熙河路六州,绘制了新的地图,并藉此打造了沙盘模型。韩冈亲眼见过,比起他当年主持测绘的路中全图又要精细了数倍,可谓是名不虚传,不愧是千古留名的沈括沈存中。

王旁听着惊讶,韩冈竟然对沈括近乎针对性的重制地图一事毫不在意。但他看了韩冈脸上的微笑,也就登时明白了。就是因为对自己充满自信,韩冈才能毫无芥蒂的夸奖沈括,并承认自己的不足。

“是因为契丹人的事?”王旁问道。

韩冈则反问:“现在还能有什么地方急着要整理舆图的?”

契丹人趁火打劫的盘算已经传遍天下,这一年来,京城里有好几次谣传契丹铁骑已经南下。

多少臣子都为此而上疏,表述自己的看法和意见。韩冈也不例外。他主张强硬回绝。契丹人欲壑难填,若任其予取予求,给了契丹人软弱可欺的感觉,他们只会变本加厉。化外蛮夷,畏威而不怀德,当严辞拒绝,并摆出不惜一战的架势,这样才能遏制契丹人的野心。

由于韩冈的态度太过强硬,赵顼曾有让其去河东与契丹人谈判的念头立刻就打消了。最后还是让能耐下性子与契丹人辩论的河东转运使刘庠,以及翰林学士韩缜,继续负责此事,并调了长于地理、文案的沈括,准备让他去与契丹人谈判。

韩冈虽不在朝中,但靠着王雱,得到了消息也是十分及时,也随之松了一口气,他可不想去河东。

不过上书的不仅仅是京城里的朝臣,还有外地的元老重臣:“天子问政元老,不过富彦国却给了一个笑话回来。”

“什么笑话?”王旁问着。

“‘边奏警急,兵粮皆缺,窘于应用。须防四方凶徒,必有观望者,谓国家方事外虞,其力不能制我,遂相啸聚,蜂猬而起,事将奈何?臣愿陛下以宗社为忧,生民为念,纳污含垢,且求安静。’”韩冈读着王雱的信,最后放声大笑,笑声越来越冷,“这算不算叫做内残外忍?”

富弼的奏章第一个送抵京城,上面要天子‘纳污含垢,且求安静’,若是与契丹人交战起来,国家内部必然有人心怀叵测,盗贼纷起。看到富弼的回答,韩琦、文彦博的奏章,想来也不会有什么差别。

“富弼这是明着欺君!”王旁恨恨的骂道。

韩冈对此不知是该笑还是该恼,堂堂宰相,不想着折冲御侮,却担心着与契丹人开战,会造成内乱。

这真是笑话了,能不惜一切的保护百姓的国家,怎么可能会有内乱?看富弼的奏章,真像是老糊涂了。

可韩冈知道富弼一点都不糊涂。

与只凭血缘就坐上帝位的天子不同,能升任宰相的没有一个会是简单人物。富弼出使辽国的时候,当年对辽人还算强硬,在仁宗皇帝、宰相吕夷简、已经烂掉的大宋官军,加上西夏李元昊一起拖后腿的情况下,添了二十万岁币将危机度过去了。

可现在国势大涨,军事力量远过于仁宗之时,却一转变得瞻前顾后,不是富弼变得胆小苟且,而是别有一番用心在。

韩冈冷笑着,这就跟自己一样,都是明知契丹人绝不会南下,所以所上奏疏中,都是掺着个人的政治目的。富弼要废新法,而韩冈则仅仅是不想去河东与契丹人磨嘴皮子。

从富弼到王安石,再到他韩冈,明眼人都知道契丹人不可能南侵,但天子不相信。只是从问政元老一事上,赵顼的态度已经很明显了。

如此一来,自己的岳父,可能当真要辞相了!

