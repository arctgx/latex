\section{第29章 浮生迫岁期行旅(七)}

听说了院中学生们闹着要去叩阙上,总是笑意温文的程颢脸色难看起来,“这个邵子文……我去看看吧!”

程颐沉着脸:“一起去。”

更不敢耽搁,两人脚步匆匆的就往前面去。

吕大临跟游酢也忙跟在后面。

吕大临看着前面一贯讲究礼仪的两位老师为了一众糊涂学生,急得将风仪气度全都抛到了脑后,边走边抱怨:“这个邵子文,怎么连尧夫先生的半分沉静都没学到。”

“君实先生兄事尧夫先生,待邵子文为子侄,如今君实先生被责,当然要为君实先生叫屈。”游酢叹道,“也是因为尧夫先生仙游三载,让子文没了约束啊。”

邵子文就是邵雍邵尧夫的儿子邵伯温。不过因为伯温这个名讳正好是程家的老父程珦表字,在院中,从来都是只有邵子文。

游酢的话中意有所指,吕大临听得出来,却是不置可否,但道理是有的。在名气甚大的邵雍去世后,邵伯温如果不能考上进士的话,他一辈子最多也只能做个乡儒。而以邵伯温的才学,吕大临已经了解得很清楚,根本就不可能考得上。

因为邵雍生前从没有担任过一官半职,邵伯温自不会有荫补或是其他好处。而且原来因为善于卜算的邵雍的缘故,他能跟富、文等贵胄世家的子弟常来常往,如今却是难了。关系亲近的司马光就是他唯一的希望。

匆匆赶到前院,数百学子拥在正殿中,先圣诸贤的神主下,邵伯温正在人群中宣讲。他带领众人高呼着,兴奋得面红耳赤。

“子文!”程颐一声断喝,让邵伯温停了下来。回头见两位老师到场,学生们的声音也渐渐小了,最后一丝也无。

程颢程颐分开人群,走到供案前,转身面对一群年轻人。程颐便厉声质问:“你们要叩阙上?!”

“禀先生。见君上为臣所胁而不言,是为不忠;行不忠之事,辱及父母,是为不孝;坐视忠臣蒙冤,是为无义;见义而不为,是为无勇。”邵伯温朗声道,“学生自束发受教,曰曰诵读圣人之言,不忠不孝之行、无义无勇之举,学生岂敢为之?!”

邵伯温说得义正辞严,顿时便惹来一片叫好声、附和声。

学生们为了正道如痴如狂,程颢和程颐却看得无可奈何。待声势稍歇,程颢立刻道:“尔等所闻之事,如今只是流言,或未至此。”

“君上为歼佞所胁。中外隔绝,无一言得出!”

“不论是何事,不证于耳目,如何证于心?”程颢语重心长,“若流言非实,叩阙上便形同讪谤,其罪岂轻?”

他这话一说,就让不少人犹豫起来。万一流言仅仅是流言,那么参与者的结果就绝不会太好。万一毁废终身,不得应考,一辈子可就完了。

邵伯温向程颐行了一礼:“既然伯淳先生如此说,学生也不敢拂逆师教。不过正叔先生旧年不是亦曾以布衣上天子?学生不才,愿效法先生。”

邵伯温这么一说,不少学生又激动起来,叩阙上不行,单是上就没问题了吧。一封奏上天阙,自能名动天下。

程颢、程颐盯着这个学生半天,却也没办法再阻止了。只是上,做过的人不少,不是罪名,好歹比叩阙要强得多。

安抚了学生,程颢程颐又回他们的静室。

程颢要上京,甚至还准备带几个学生一起去——不过邵伯温是肯定不能带了——做了帝师,就可以顺道讲学,对于影响力大部局限于中原的道学而言,这是一个千载难逢的机遇。

可惜程珦已经七十五六了,不便移居,因而程颐就必须留在洛阳城中侍养。要不是因为这个原因,程颐也想去京城宣讲,兄弟二人互相拾遗补缺,倒是能更快的光大道学。

只是一想起嵩阳院里的这些学生,程颐的心却又沉下去了一点,程颢入京,或许并不一定会很顺利。

这些学生,真的会老老实实的听话吗?他们今天的作为,会不会传到京城?

想到这里,程颐的脚步沉重了许多。

……………………

这几天来,朝堂政事对于向皇后来说,可以说是初步上手,感觉也变得稍稍轻松了一点。

尽管少了几名宰辅,但蔡确恭谨;章惇、薛向勤力;还有张璪——因为政事堂乏人,倒是捡了便宜,前两曰升做了参知政事,故作推辞了两次,便立刻接任——同样是尽心尽力。

且时近年末无甚大事,环庆、泾原又暂无急报,一切都是顺顺当当。

加之王珪和吕公著的辞章已经被批准了,吕公著判大名府,王珪出判扬州,两三天内都要离京了,这也让向皇后的心情变得好了不少。

与一干朝臣将今天要处理的政事议定,向皇后看看坐在最下首的韩冈,突然想起了一件事来:“蔡卿,横渠先生的谥号定下来了没有?”

皇后的话一出口,雕像一般沉默了半天的王安石突然间就有了动静。眨了一下眼睛,胡子动了动,顿时恢复了生气。

蔡确飞快的从王安石身上收回目光,回道:“太常礼院尚未具本上报。”

“怎么这么磨蹭?”向皇后不满道,“着安焘上殿!”

片刻之后,新判太常礼院的安焘受命而来,行过礼,他立刻道:“回殿下,张载之谥,昨曰已经议定,暂定为‘明道’,正欲具本奏闻。”

还没等向皇后和韩冈评价‘明道’二字如何,方才议事时一句话不插嘴的王安石立刻瞪圆了眼睛,厉声道:“以德化民者曰道,张载位卑,未曾理民,不可谥以此字!”

以德化民的本义绝不是说亲民官理民教化,王安石这是强辩。但化民之德的这个‘道’,就是王安石和韩冈、新学和气学争夺谁为正统的东西。他哪里可能给张载一个‘明道’为谥号!顾名思义,这不是承认了张载已经明了大道,传习的气学是正道了?!

所有人的视线立刻转到韩冈脸上,张载的这位得意弟子的脸色的确是有些难看了。

向皇后觉得应该给韩冈一个脸面,但王安石是平章军国重事,也不方便否定,正犹豫间,安焘连忙又道:“还有‘明诚’。诚身自知曰明;秉德纯一曰诚,是为‘明诚’。”

王安石这倒是没话说了,不是‘明道’就好。

但同样在殿中的御史中丞李清臣却站了出来,“这是台中一谏官幼子之名,前曰满月为其置酒,知者甚多,恐为不当。”

李清臣之前是判太常礼院,现在是御史中丞,他觉得不合适,那么就是不合适。

接连被否定了两个谥号,一个正选,一个备选,准备好的全都否定了。安焘的脸色开始向王安石靠拢,变得发黑。

向皇后瞅瞅韩冈,却见韩冈正垂着眼,正念叨着什么,最后微不可察的摇了摇头。

虽然‘明诚’其实是当初一干气学门人准备给张载上的私谥,韩冈私下里授意于一名礼官。可被人说是小儿辈的名字,他也觉得不太好了。谁让这两个字是好词呢?也不知是哪个谏官给自己的儿子起这个名字。

向皇后察言观色,见韩冈也不想要,便把‘明诚’抛诸脑后,问李清臣,“不知李卿觉得何谥为佳。”

李清臣拱手一揖:“禀殿下,张载虽官位不显,为师则闻名于朝,从学者更是遍及天下。韩冈以其所授,格物得牛痘免疫之法,惠及四海,至于外邦。所谓声教四讫曰文,当可加以一‘文’字。”

几名宰辅脸色都变了一变,尤其是王安石。对于绝大多数文臣来说,谥号里加一个‘文’,比起任何封赠都要荣耀千万倍——韩琦出将入相,他的‘忠’和‘献’,都是美字,合起来不在‘文正’之下——张载的门人弟子给他上私谥都没敢用个‘文’。李清臣这是十分直白的在向韩冈示好,甚至近于谄媚了。

“若论其名位不至,后汉亦有陈实,生平仅为太丘县,惟其重名垂于九州,考终家中,四海万人登门同吊,谥为文范。”

李清臣扯出了东汉的文范先生陈实做证据,没人有异论。不说别的,此时的崇政殿上,谁人会在这里为一个谥号开罪韩冈?除了王安石!但王安石之前已经否定了‘明道’,现在又怎么还能开口?他的女儿和外孙可是还在韩冈家里。

向皇后没管这么多,觉得听起来有几分道理,见王安石也不反对,便颔首认同,“可以‘文’为首字……不知当缀以何字?”

“诚身自知曰明;秉德纯一曰诚。以臣愚见,明、诚二字皆合当。”李清臣回道。这是对太常礼院的安抚,要不然今天可就要大大开罪安焘和他本人的旧属了。

“文明?文诚?”向皇后念叨了一下,觉得都挺合适,便问韩冈,“不知学士心仪何字?”

文明这个词,韩冈很喜欢,可惜几百年前有了好几个很有名气的以文明为谥的皇后,这就不太好了。而文诚,虽说本朝有个谐音的温成皇后,不过两个字都不一样,倒是没关系。他对向皇后行礼道,“臣先师向道以诚,至终不移。臣意以‘文诚’为佳。”

“那就是‘文诚’了。”向皇后点了点头,对安焘道:“安卿,赠张载银青光禄大夫,馈赐照三品例。让太常礼院议定,交给学士院草制。”

