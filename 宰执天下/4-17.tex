\section{第二章 凡物偏能动世情(四)}

当韩冈和章惇被熙熙楼的掌柜一脸殷勤的相送着从酒楼中出来,已经是华灯初上的时候。

落日的余晖已然散尽,但西边的天空还残留一抹带着丝光的深紫,瑰丽的色彩犹如出自湖州的吴绫,不需要任何纹路花样,便堪于最上等的蜀锦相媲美。

熙熙楼楼外的街道,也是一处夜市,虽比不得州桥夜市的繁华,但人气也不输多少。当韩冈踏足楼外,就看到一盏盏灯高高的挑了起来,整条大街给照得犹如白昼,街上的行人反比白天还要多上几分。

就在酒楼门边的摊子上,一名身处褐衣、头戴毡帽的小贩,唱着货郎曲儿,向来往的行人推销着摊子上一支支铜质的梳子和发簪。这个时代的酒楼,对摊贩很是宽容,这个小贩就在门边不远处坐着,也没人出来赶他离开。身处市口,加之卖的货物有些吸引力,他的生意倒还不错,竟围了五六人。

韩冈踏着台阶与章惇前后脚走出,只是顺带的看了摊子一眼,脚步就顿时停了下来。

“韩孝,你去买一支簪子回来。”

被韩冈点了名的伴当有些纳闷,这里明显的就是几文钱一支的低档货,自家都没脸买给婆娘穿戴,怎么舍人要买给家里的夫人和三位娘子?但心中疑惑归疑惑,他还是乖觉的上前挤进人群,自掏腰包,拿了九文钱,一点也不还价的依言买了簪子,想了想,就又买了一把铜梳回来。

将簪子和梳子一起呈给了韩冈,韩孝还碎碎叨叨的说着:“这家摊子的铜簪怎么这么便宜?往常买少说也要十五六文才对。”

章惇正等着酒店的小二将他的马给牵来,回头一看韩冈,竟然是在命下人买着地摊货。

“怎么了?”他很奇怪的走过来。

韩冈没作声,先用指甲刮了刮簪子的表面,见上面的铜色依然灿烂。就将簪子交给了身后的另一个伴当,示意他在地上磨上几下。就这么磨了两下,当铜簪重新拿到眼前时,当即就见到了里面银亮的铁来。

“是浸铜法。”韩冈将簪子拿给章惇看。又掂了掂掌中的铜梳,果然重量似乎有些不对劲,远不如他旧时家里用的差不多大小的那一柄。

浸铜法,也就是用铁来置换出胆矾水中的铜,是基础化学中的内容。如今在南方的铜矿中使用的为多,南方诸路生产出来的生铁,有不少用此法来制铜。虽然此事世间有着不少人皆认为此种制铜法制造出来的是伪铜,但从三司流传出来的传言却说,浸铜法此后将会大力推广,如江西铅山等处的铜矿,都会陆续采用此法。

而另一个浸铜法用得多的地方,就是军器监中用来给铁器镀铜色。韩冈上元节时拿出来的板甲,便是给工匠镀上了一层铜。除此之外,就几乎没人用,甚至知道这种方法的都少,当初工匠给板甲零件浸铜时曾对韩冈说,除了军器监的工匠之外以外,东京城中找不到第二个明白浸铜法的匠人。

可现在才过去几个月,就连路边摊贩卖的器物都用上了浸铜法,究竟是巧合,还是从军器监中学来的?

韩冈的视线转到了章惇脸上,翰林学士明了一切的神色,说明了他想到得正与韩冈一模一样。

章惇咳嗽了一声,现在出现的这个东西,也确证了军器监已经成了世人关注的焦点,有些技术上的特色就立刻会被偷出去。方才他对韩冈的话,看来也不是白担心。“玉昆,愚兄今日所言,还望慎思之。”章惇沉声说道。

“学士放心,韩冈明白。”韩冈一声轻叹。

技术扩散是好事,但自己的压力可就要大了。但他到了现在这个位置上,已经不需要顾虑太多,而且在飞船出现后,有点错处也是好事。且不管怎么说,他的一切发明,都是没有太多的技术含量,想仿效吗?看一眼实物就够了。唯一能让朝廷占据压倒性优势的,就是规模。这也是韩冈一直以来告诉赵顼的道理。

道别之后,章惇向东,韩冈向西。

身下的坐骑,四蹄哒哒的蹬着地面,漫不经心的向前走着。这匹阉过的河西马肩高四尺二寸,刚刚过了军马的及格线,并不能算是好马——好马也舍不得阉割——但胜在老实温顺,甚至是迟钝,在熙熙攘攘的东京城中,不会像另外一些河西马一般容易受到惊吓。行走得平稳,让骑着这匹马的骑手,在驾驭时都不会感到吃力。

沿着南门大街慢慢向西行去,前方天幕上的艳紫在一点点的蜕变成墨蓝,天空中,稀稀落落的几个星子还看不分明,但天色已经差不多都黑了下来。

天色将晚,已经可以看到街边的巷子中,更夫在敲着梆子,每走上几步就敲上一回。韩冈轻夹马腹,往家中赶去。只是刚到浚仪桥,就见到了一个熟人。

是吴充的二儿子吴安持,另外,他也正是韩冈的连襟。

这吴安持从得胜桥上下来,眼睛在街边左右扫着。似乎在韩冈看到他的同时,也发现了韩冈。但看他的态度又好像并没有发现,反正视线是茫茫然的一带而过,就想转身上马。只是从吴安持匆匆忙忙的态度上,韩冈估计他多半还是看到了自己。

“仲由兄!”韩冈远远唤了一声。见面了就跑,吴安持的做法未免太不给他面子了。

吴安持这下子跑不了了,只得下马回头,脸上堆起了惊喜:“原来玉昆贤弟!”

“许久不见仲由兄,不知向来可好?”

伸手不打笑脸人,韩冈笑着走上来,吴安持也不好说两句就走,却是被他拉着在街边说了好一阵话。既要叠起心思应对韩冈,也要防着一不小心被诳出一些不该说的话来,只寒暄了没几句,就是浑身是汗。

被韩冈耽搁了好一阵,甚至不由自主的答应下来改日一起喝酒的承诺,当吴安持回到家里的时候,已经快要到二更天了。

走进房中向父母问安,吴充就不快的问道:“怎么回来得这么迟?可是去青楼了?!”

吴安持不敢隐瞒:“儿子是在路上遇上了韩冈。”

“韩冈?!”吴充不意从儿子口中听到了这个名字。

“正是韩冈。”吴安持低头道:“他上来跟儿子搭话,也不便不理睬他。”

吴充脸色沉了下来:“说了些什么?”

“没说什么,就是闲聊了一阵。”吴安持见吴充脸上写满了不信,连忙将跟韩冈说得那些话,一五一十的转述给吴充。

吴充听了儿子一阵絮絮叨叨的废话,不耐烦的往外摆了摆手,“你下去吧,以后见了韩冈离着远一点。”

“大人……”吴安持没有动,反而有些迟疑的在背后叫了转身准备入内间休息的吴充一声。。

“怎么?”正如如今大部分做父亲的人一样,吴充在家中亦如严君,标准的严父慈母中的前者。只是微皱起眉头的回头一瞥,就让吴安持胆颤心惊。

“为什么大人要一直针对韩冈,他不是只在安心的打造军器吗?”吴安持大着胆子问着,“大人的对手当是吕惠卿,何必与韩冈结下仇怨。也许现在韩冈只是直阁而已,可一二十年后,未必不能升入东西二府。”

吴充的眼神如刀似箭一般的变得锐利起来,使得吴安持的声音越来越小,但听着儿子的话,他却沉默了。过了好一阵,方才反问道:“知道为什么天子喜欢孤臣?”

“……不结党营私,忠心事上?”吴安持的回答说到最后又变成了疑问句。小心翼翼的抬眼看着吴充,等着对回答的评判。

吴充不置可否,只是再问了一句:“见过孤臣做宰相吗?”

“啊!?”吴安持闻言一愣。

“一个都没有。”吴充冷冷一笑,“韩冈甚至连新党都不亲附,朝中上下无人,日后如何能升入东西二府?

王安石为官数十载,入朝任职虽然只有几年。但朝中亲厚之人无数,才学亦是一时之选。文宽夫、富彦国、欧阳永叔、包希仁,多少重臣元老看重于他?吕晦叔、吕宝臣、司马君实、甚至包括为父,又有多少友人与其来往唱和?其身在江宁,在今上耳边,还有韩维、韩绛为其做仗马之鸣。朝野上下无人不赞,安石不出,奈苍生何?!

可你再看看韩冈,他参加过几次诗会?上京以来,又结交过多少士人?朝中的几名重臣,他亲附过谁?就连他的岳父他都不理会!这样的臣子,天子当然喜欢。但想要做到宰执,根本是休想。宣麻一事,可不是天子一人说了算的!”

“可两府之中还有王韶。在关西,也有关学一脉。”吴安持小声的争辩道。

“王韶功劳不小,但开疆拓土,枢密副使就到顶了,没机会再升上一步,能帮到韩冈什么?更休提若关学,但凡关学有点底蕴,张载也不会一直守在横渠。”吴充再一声冷笑,“要不是有韩冈这名弟子,他的名声一辈子都别想流传到京城中!”

