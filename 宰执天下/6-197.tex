\section{第21章 欲寻佳木归圣众(17)}

“以神卫军左厢第三第五指挥,右厢第一、第四指挥为主力,总数三千人马的飞跃演习,已圆满完成既定任务,胜利凯旋。今日午间,军车抵达东京车站,枢密院李都承讳迪,前往车站进行迎接。”

宋国报纸上的遣词用句,受到韩冈的影响太大。《九域游记》在文学上的影响力已经渐渐浮现上来。

浅近的白话,少许的新词,耶律乙辛并非白丁,双眼掠过,就明了了其中的内容。

托如今宋辽两国民间越发紧密的商贸往来的福,耶律乙辛只迟了一个月就看到了主要发行地位于南朝京师的报纸。

不论是蹴鞠还是赛马,两家快报的报纸,差不多每隔几日,就能摆上耶律乙辛的案头。

耶律乙辛对蹴鞠毫无兴趣,对赛马则关注得多。每当发现宋国的赛马场上又多了几匹拥有西域天马血统的赛马,这位窃国大盗,心情便会恶劣上好些天。

不过这些报纸上的内容,耶律乙辛看得更多的,还是对于开封城中各色新闻的报道。尤其是重要官员的人事变动,各色新式发明,以及商业信息,都是他关注的焦点。

东海的盐、岭南的糖,南洋的豆蔻,西域的孜然,还有最重要的粮食和铁。

这些商品的价格变动,都意味着宋国哪个地方的局势有了异动。但在苏颂、韩冈、章惇秉政的这几年,铁器价格不断小幅下跌,而粮食价格则保持稳定,让宋人苦于抑配的官盐价格则降了三成。

配合稳定到乏味的官场,以及越来越多符合韩冈喜好的发明,这让耶律乙辛了解到了南朝如今的稳定和强盛。

今天除了报纸之外,还有一份报告,是另外一名细作发来,有关于这一次的演习。报告上称,在所谓的飞跃演习开始后,韩冈私下里曾对外透露,这次演习,仅仅是战术上演练,不针对任何国家和个人。

“不针对任何国家和个人!”耶律乙辛念着这句让他觉得十分拗口的文字,“张孝杰,你怎么看?”

“回陛下,此语其中必有深意。”

“深意?当然,这是说给朕听的。”

耶律乙辛自是明白,这句话肯定是要反着看的,当是韩冈在被人询问这是不是针对大辽的行动后,以开玩笑的口气,给出的一个肯定的回答,丝毫不介意这个回应,传到北方的邻居耳朵里,甚至就是打着让细作传话的目的。

韩冈隔空传话,带来的是饱含恶意的嘲讽。耶律乙辛有一种预感,宋国来袭的时间,也许不会太久了。

“三千人马,千里远征,六日往返。”

报纸上铿锵有力的词句,念起来,便让人心中腾起一股绝望。

三日五百,六日一千,那已经是骑兵一等一的快速了。拼着死掉一半战马,三日千里或许也不是做不到,但那样的话,到地头了,强如宫分军,也会变成任人揉捏的软柿子。

可宋国,行驶在铁路轨道上的马车日夜不息,每隔几十里就换上一批健马。兵马就在车上休息,除了局促一点外,就跟休息没两样了——何况车上再局促,也比马背上要宽敞——以这样的方式行进千里,到了目的地,下来就能作战。

而大辽的骑兵想要做到相当的水准,至少一人十匹马,这样才能换得过来,只要人换不了,一样还是笑话。

这一回,还仅仅是第一次的演习。要是时间长了,次数多了,到了铁路和禁军双方都习惯了这样的演习,又有了充分的经验,一万人,两万人,甚至三万人,五万人,同样能够做到类似的事。

以河北前线与南朝京师几百近千里的距离,宋人在五天之内,便能将上万精锐送抵激战的最前线。

轨道不好修,耶律乙辛一直想在析津府【今北京】和奉圣州【今张家口】之间修一条铁路轨道,派了人出去,回来后就只会摇头,说是不可能。

现在宋人都已经能够让铁路轨道跋山涉水了。几千里的长路,说修就修,而且细作还回报说,南朝连接各县的铁路,会交给地方上合股修建。多少大户垂涎欲滴,想着要分肥。

河东已经修成的铁路,让耶律乙辛不敢再打雁门关和代州的主意,一旦雁门有警,太原的守军最多三天就能赶来,就算打进去了也站不稳脚跟。再等几年,等到河北连县中都有铁路联通,原本可以容许契丹铁骑纵横奔驰的大平原,就变成了一张巨大的蛛网,让战马无从奔驰。

“这让人日后怎么打?”

只要大军开始在鸳鸯泺集结,甚至就在析津府集结,宋人一旦收到消息,还是能以最快的速度做出应对。

想要顺利的攻打宋国,让宋人猝不及防,首先得能够让数万大军悄然集结在南京道,然后以最快的速度冲入河北地界,破坏铁路轨道。

但耶律乙辛多年领军,不是赵括、马谡之流,自然明白事涉千军万马,想要瞒着百多年来的死敌,比登天还难。

“可以以计破之。”张孝杰道。

耶律乙辛抬起眼,“疲兵之计?”

张孝杰的表情立刻僵住了。

耶律乙辛呵的一声轻笑,眼前这个无解之局,张孝杰还能出什么计策?

隔三差五的在奉圣州或是南京道集结兵马,等宋人习惯成自然,就有攻击的机会。

这是个好想法,但宋人难道会看不懂?何况,他们已经在用了。

“这个计策,我们能用,宋人也能用。这个什么飞跃演习,一旦用在河北,几次下来,谁还能一直防备着。都用上同样的计策,到时候,吃亏的肯定是我们。”

金帐中放着冬季埋藏于地下的冰块,用以解暑。但这一回,寒气来自于心中。

“陛下……”张孝杰道,“南朝小皇帝年岁渐长,韩冈纵有才干,亦难安居其位,可静待其变。”

“但愿如此。”

耶律乙辛点点头。

韩冈成不了他耶律乙辛,一旦宋国朝堂不稳,不论是韩冈得胜,还是那位小皇帝得逞,最后都会让宋国现在咄咄逼人的局面,大为改观。

耶律乙辛看了看张孝杰,端正了一下坐姿,一国之君,不能在臣子面前垂头丧气太久。

拿起另一份报告,他问:“这个水力缫丝机是什么东西?”

……………………

有关的水力缫丝机和丝绸织机的消息,经过一个月的时间发酵,已经在京师的上层社会传开。

尽管冯从义在冠军马会透露消息的时候,各人都打定主意要将这个发财的好机会保密,但一个人的秘密才是真正的秘密,两个人的秘密,更不用说十几二十,乃至更多人知晓的秘密。

相较于近十余年方才异军突起的棉布,相传是黄帝之妻嫘祖所开创的丝绸,其在社会中的地位,远远超过棉布不止七八筹。很长一段时间以来,丝绸的另一个身份,就是充作一般等价物的货币,也是税赋中,除了钱粮二事之外,最重要的征收品。

棉布的利益,被西北吞占,尽管很多人看着眼热,也忍不住想要分一杯羹,但如果将棉布和丝绸放在一起让人做对比,至少有九成九的人会选择丝绸。

当水力缫丝机和丝绸织机——这种如棉纺织机的一般——能将丝绸织造的效率提高十倍的机器,出现在世人的面前,大宋朝野如同炉子上的一壶水,从平静渐渐转向沸腾。

只要有些身份地位,都对此抱着极大的兴趣。技术进步的意义,这些年来,逐渐深入人心。从板甲的制造开始,人人皆知,上好的机器,能够省下数倍的人工,带来难以计数的好处。

现在站出来的,都是那些在京师排得上号豪商的势力,但私下里,冯从义收到的帖子,甚至比韩冈还多。主动登门求见的,并不比他收到的名帖稍少。

看着情况不妙,冯从义早早的就收拾行装,上路回家,也不等到秋凉。

冯从义一走,他背后的韩冈便曝露在人前。

人们不敢直接向一国宰相诘问,但人人明白一切都是在韩冈的计划中,现在要考虑的,是韩冈会提出什么样的交换要求。

韩冈对人们的等待视若无睹,而是在经筵上,向太后与小皇帝讲授着气学的奥义:

“文诚先生所传四句教,为天地立心,为生民立命,为往圣继绝学,为万世开太平。此四句,乃真儒之行,也是四条习学的方向。所谓为天地立心,研究天地至理;为生民立命,研习经世济民之法;为往圣继绝学,那是经史;至于……为万世开太平。”

向太后明了:“就是用兵之法了。”

“陛下圣明。”韩冈点头,“等闲士大夫,能得其一,便可谓之人才。但寻常士人所学难得正法,故而世间乏人。陛下治国,当思如何得人。”

向太后道:“当广开进用之途,不拘一格,选拔人才。”

“陛下。”韩冈摇头道,“南方有蛮部,不识耕作,唯以采集狩猎为生。但这样得到的食物,少之又少,难得一饱。”

韩冈的比喻,向皇后还在考虑中,而小皇帝反应很快:“可是兴学?”

韩冈点头微笑,“陛下圣明,材士如粮,多种方能多得。”
