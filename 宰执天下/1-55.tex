\section{第26章 仕宦岂为稻粱谋(中)}

王韶四面顾盼,恍若未闻。却是王厚性急,直截了当道:“玉昆,你也别装佯了。愚兄和大人来此,为得甚事难道你还不清楚?”

韩冈笑而不答,反是王厚称呼王韶所用的‘大人’二字,让他听着感慨。

‘大人’这个词。韩冈穿越后只在王厚这里听过,因为此时尊称官吏,从来不会用到。大人一词可以用来称呼贤者,西汉的司马相如就曾经著有《大人先生传》。但最为常用的地方,还是用来尊称自己的父、祖。至于对官吏的称谓,高傲的汉人士大夫绝不会使用‘大人’,他们不愿也不会自贬为长官的儿孙。

就算到了后世的明代,甚至满清早期,对官员也不会有‘大人’之称——韩冈前世读过《西游记》和《儒林外史》,两部一个出自明代,一个出自前清的作品,都是证明了这一点——直到满清中期之后,汉人气节沦丧殆尽,大人一词才开始在官场上通用起来。

见韩冈若无其事的在前领路,并不回应自己。王厚心中焦躁起来,怎么一个个都是绕来绕去的脾气,他的老子是这样,连最为佩服的朋友也是这样。

王韶感觉着自己的儿子快要爆发了,抢先一步话出口:“韩贤侄,你这座伤病营看着就与他地不同。伤兵居于此处,当是不用多久就能痊愈。”

“机宜谬赞了,此事无他,不过是用心尔。”韩冈谦虚地说着,并不居功自傲。不过事实摆在眼前,功劳是丢不掉的,他越是谦逊,越是会为世人所尊重,“许多伤病,如果是在家里养着,有人悉心照料,根本不会恶化乃至丧命。院中如今的情况,并不是学生有什么功劳,而是这些护工们用心照料的结果。”

“贤侄太过自谦。”王韶笑说了一句,他看着几名护工就着流水,辛苦的清洗病号换下来的衣服,神色皆是认真专注的模样。又点了点头,道:“不过贤侄说得也对,不论做何事都要用心。若路中各城各寨的伤病营皆如此处,日后征战,也少了许多后顾之忧。”

“机宜说得正是。”韩冈道:“学生如今正在整理一份有关军中伤病疗养的章程,在甘谷城已经做的,还有准备做的,都会包括进去。届时各地伤病营若能依着章程办,营中的病殁人数当可大大降低。”

王韶有些惊异的看了韩冈一眼:“这算是在立言了?”

儒门弟子行事,讲究三立——立功、立德、立言。韩冈在甘谷城做得这一切,立德、立功都有了,只差个立言。但只要他把所谓章程给整理出来,立言这一条也算圆满完成。

所以他点头:“如此才不枉学生一番辛苦。”又笑了笑,“张都监荐学生管勾路中伤病事务,不论成与不成,现在将章程定下,日后各处伤病营也可以参考一二,不至再沦入旧有的境况。”

“玉昆!”王厚猛的叫起,王韶和韩冈两人围着正题绕来绕去,让他实在烦透了,“你当真以为张守约荐举于你,是因为看着你伤病营打理得好的缘故?他是为了向宝啊!”

韩冈看着王厚,先是愣了一下,后又摇头轻叹,似是感慨万千,“我知道……我知道的。”

王厚要说什么,韩冈都知道,王韶的用心,张守约的用意,他怎么会不清楚?

但这又有什么办法——他并没有生在相州韩家,不然凭着一个相三帝立二主的韩琦韩太师,莫说十八岁,就是八岁,也能身披官袍,领着俸禄。他也不是生在灵寿韩家,否则借助自仁宗朝的执政韩亿以下,八子皆为显官的荣耀,横行乡里也不在话下。他只不过是菜园韩家的幺子,想在秦州混出个名堂,先得找个好后台。

韩冈很清楚这一点,但后台他绝不会溜须拍马的去找,得让人自己送上门来。要想受人荐举,最重要的是名望,以及才能。韩冈把握住了出现在他面前的大部分机会,表现得足够出色,所以才引来了王韶和张守约的目光。

荐举本质上是一种利益的交换,必须要给荐举人带来足够多的利益——这个利益可以是名声,可以是权位,也可以是财富——否则谁会浪费自己的笔墨和信用,还要为他人担上责任。任何荐章的最后,都有类似于‘甘当同罪’的一段话,这是荐举人在向朝廷表示对被荐举人的信心,也意味着荐举人将和被荐举人休戚与共。

王韶想用他韩冈,目的不外是开拓河湟的助力。不同职位的官员,能荐举的人数都是有数量限制的,即便是统御万邦的天子,即便是执掌中枢的宰执,都不可能能想用谁,就用谁。以王韶担任的经略司管勾机宜文字这个差遣,他能荐举的人数,最多也就两三人。分给韩冈一个名额,王韶所想要交换回来的,绝对不会少。

至于张守约突然荐举他为官,明面上是因为他在伤病营的表现。可韩冈还不至于那般幼稚,张守约前日还特意问过伏羌城的事,韩冈人精一个,就算王厚不说,张老都监跟都钤辖向宝之间的微妙关系,他照样能看出来。

王厚爆发之后,三人陷入一阵沉默,在院中静静的走着。沿途的护工和伤病,见到韩冈陪着人走,都是立刻避开道路,站在路边鞠躬行礼。他们不是为了王韶和王厚,而是为了韩冈。王韶不禁惊叹,韩冈在甘谷的这段时间,当真是把人心都收服了。

病房前,雷简和仇一闻已经得到了消息,领着一众护工和能行动的伤病在门口候着。仇一闻穿了身易于做事的短衣,老脸上都是嫌麻烦的表情,而雷简则不愧是从东京来的,衣裳干净整齐,一脸的殷勤小心,腰背也躬得恰到好处。

韩冈上前一步,欲为王韶向介绍着两名疗养院中的主治医师。王韶笑着打断道:“不用介绍了,都是熟人。”

雷简是秦凤路四位军医之一,而仇一闻虽为民间郎中,但在秦凤军中比雷简名气大上百倍。王韶在秦凤路已经待了一年,当然不会不认识。

王韶被恭恭敬敬的请入病房内。新近打理好的病房干干净净,地面上无一丝杂物。被木板分割开的床位看起来整整齐齐,床单都是常洗常换。躺在病房中的重伤员也得到了精心的治疗,虽然无法起身,但也不是颓然待死的模样。放眼一望,偌大的营房整洁清爽,让人一看就觉得舒服顺眼。

王韶看了直点头,对两位大夫赞许有加。回过头来,又对韩冈赞道:“贤侄做了件善事。如甘谷疗养院般的伤病营,真是闻所未闻,见所未见。”

“如今仅是刚开个头,有许多还要改进的地方。”韩冈谦虚了一句,指了指地面,“就如这黄土地,完全遇不得水。但要在营房内铺设砖石也太耗费。所以等道明年开春,有了闲暇,还要改用石灰合了沙子来界平地面。”

王厚惊奇道:“玉昆真是博识。连江南豪民修墓墙的手段都知道。”

韩冈也是吃了一惊,他说的可是土制水泥,难道这个时代就已经出现了?他问王厚:“江南修墓不用墓砖?”

王厚解释道:“旧时江南王公墓中多用砖石砌墙,但往往被奸民所盗取。如今都学乖了,改用石灰合了筛土砌墙,干后便坚硬如石,不比砖石稍差。【注1】”

筛土就是沙子,从河边挖出的河沙都是含着石子石块,都要过筛才能使用,所以称为筛土。用石灰拌合筛土,便是最简单的水泥。韩冈真没想到,土制水泥在这个时代便出现了,亏他还想等把水泥造出来后,拿来炫耀显摆,如果能顺便赚点身家那就更好。

参观过两间病房出来,王韶让雷简和仇一闻继续做他们的事,不必再作陪。仇一闻掉头回病房,雷简腆着脸还想凑个趣,却被王厚不耐烦的斥了回去。

三人随意的在挂满衣物和床单的晒衣场边走着,王韶突然问道:“贤侄还记得裴峡中袭击你所率车队那些蕃人吗?”

“当然记得。他们听了西贼内奸陈举的撺掇,妄图截断粮道,学生也是深受其害。多亏了机宜当机立断,揪出幕后罪魁陈举、刘显。这个消息学生已经听说了,想必不数日,当日出兵裴峡谷的蕃部当水落石出。”韩冈顺着王韶的口气说话,他既然想市恩,自己捧个场又如何。

“当日在裴峡中偷袭你的是洛门山【今洛门镇】的末星部!自陈举的祖父辈开始,就跟陈家有几十年的往来。经略司已经从伏羌城和夕阳镇调出四个指挥的人马,又征发了附近的九个蕃部两千兵力,如果不出意外的话,就在这几天,末星部便要族灭。”王韶说得轻巧,漫不经意间,一个拥有近千帐幕的大部族便要灰飞烟灭。

注1:北宋江休复的《江邻几杂志》中有载:‘江南王公墓莫不为村人所盗,取其砖以卖之。是砖为累也。近日,江南有识之家不用砖葬,唯以石灰和筛土筑实,其坚如石。’这应是中国比较早的水泥记载了。

ps:中国古代科技水平不低,原始的水泥早就用来刷墙。除非是能工业化制取水泥,不然,不可能在古人面前显摆起来。

今天第一更,求红票,收藏

