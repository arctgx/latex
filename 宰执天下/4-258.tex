\section{第43章 庙堂垂衣天宇泰(五)}

【第一更】

“种痘法事关重大,小弟只想问一下哥哥,到底有没有用!”

黄庸的堂兄弟大约三十出头的样子,中等身材,相貌也不算出众,不过言行举止间的大家风范,却让人一见难忘。只是他现在颤抖着手暴露了心中的激动,急促的语调,也是黄庸平日所未曾见到过的。

自家兄弟的反应不难理解,当黄庸两天前终于确认种痘之术当真存在时也是激动得不能自已,只是听到了韩冈涉足其中之后才冷静了下来,恢复了作为一名宦海沉浮多年的官员的理智。

黄庸冷静得近乎冷淡,点头确认:“有伏龙山周围六个庄子,总计两千百姓亲身验证,应该是不会有假。”

“那哥哥你还等什么?!每拖延一天,就不知有多少小儿枉死于痘疮之疾,这事怎么还能等?!”

黄庸突然觉得自己选错了商量的对象,自家浑家都要比眼前人好一点:“兄弟,愚兄不瞒你。上门去拜会韩冈,这事倒不难,愚兄也不怕丢这个面子。怕就怕丢了面子还挣不回里子。换做是你我,会不会愿意在得到天子准许之前,先行推广种痘法?这是要冒风险的,韩冈暗中派人去伏龙山中,只是想做个验证而已,否则早就来请愚兄了。”

“哥哥……你好好想想,种痘法对百姓重要,难道对天家就不重要?如今皇子的排行已经排到了第七,却只留下了两个加起来还不到四岁的皇第六子,皇第七子!”

黄庸愣了一下,长吁一口气:“……愚兄倒是疏忽了!”但他立刻又皱起眉,“既然如此,那韩冈不就更不会让人分工了吗?”

“哥哥,种痘法虽好,但毕竟初行于世,献上去之后,试问天家敢不敢遽然使用?”

“伏龙山已经有两千人验证过了……”黄庸觉得堂弟的口气有些不对,这是帮着种痘法说话,还是反对啊,眼中闪起了疑惑。

“才两千人算什么验证?天下有亿万人啊。七万三千六百零四户,十七万一千一百一十二口,三十八万九千三百人,这才叫验证!……用一个襄州为王前驱,如此,才能让天子放心得下。”

自家兄弟只是在上缴秋税时,随便看了一眼,竟然就把襄州的户口人数全都给记住了。过人的记忆力,黄庸并不惊异,但面前闪烁着锋锐光芒的眼神,却着实让人惊讶。

“用襄州为王前驱……”黄庸重复着,沉吟起来。

“哥哥,须知合则两利!韩冈本与沈括交好,但他没有选择在唐州进行验证。不管是什么原因,这就是哥哥你的机会。整件事已经流传出来,想来韩冈也无须再保密。只要哥哥你全力支持,难道韩冈不希望种痘法在天子面前更有说服力?”他声音低沉了下去,“……既然韩冈选了伏龙山做验证,那就不能让这个功劳跑出襄州去。”

黄庸正在考虑着,书房外响起了唤门声:“老爷,小韩学士遣人送帖子来了。”

韩冈!帖子!

黄庸与兄弟对视一眼,眼中满是疑惑。

黄庸连忙让人进来,接过一片短笺匆匆一览,刨去无意义的辞藻,韩冈的本意,他已经看得分明。转头就道:“韩冈请愚兄过府一叙……”

他的兄弟立刻面露喜色:“真是太巧了!哥哥,看来韩玉昆多半就是想求哥哥助上一臂之力。”

但黄庸面色不愉,并不搭腔。他以知州之尊,就是贵为转运使,也不能召之即来挥之即去。想让自己帮忙,就只以片纸相招,当他黄庸成什么人了?

黄庸的堂弟暗叹一声:“哥哥。小弟知道哥哥不喜欢因人成事,可是事关谨哥儿和谕哥儿,舍了点面子又如何。”

黄庸的两个儿子,本房排行第一、第四,都还没到应举出仕的年纪,若说有什么地方能与他们有关联,那就只有荫补了。于国有功,不说黄庸本官的品级能上移一步,达到荫补子嗣的最低标准,让长子可以得受官职。就是天子那里,多半也会特旨褒奖,连老四的官身也一起解决了。

“而且这也是为襄州百姓着想,知道哥哥你亲自去求韩冈在襄州施行种痘法,全州上下近四十万人,哪个不会对哥哥你感恩戴德?”

黄庸踌躇了一下,终于点了头。站起身,道:“勉仲,你随我同去。”

这是态度问题。将自己还没有做官的兄弟带过去,是向韩冈表示自己不打算将两人的关系局限在官场往来上,结下了私交,许多事就好办了。

黄庸的堂弟心领神会,“是。哥哥请稍候,小弟进去换身衣服便来。”

黄庸抬起袖子看了一眼,家常的蓝布直裰穿得是舒服,但不是访客该穿的服饰。笑道,“愚兄也得换身衣服,不能太失礼了。”

黄氏兄弟应邀前往拜访韩冈,没有耽搁时间,就带了一队人马前往漕司衙门。

转进漕司衙门所在的街中,就发现一条路就跟上元灯会时一般的熙熙攘攘,上千名百姓拥挤在衙门前,可偏偏就没人敢于喧哗出声。

怎么聚集了这么多人?这个问题在两人的脑海中一闪即逝。

还用问吗?明眼人太多了,从伏龙山传出来的一切信息都指向有药王弟子之名的京西转运使,怎么可能还有人猜不出来?

在前面鸣锣开道的旗牌官抵达人群前的时候,围着漕司衙门的百姓立刻让出了一条道来。

这群人中不仅仅是平民百姓,还有不少在州中县中都说得上话的士绅,一等到黄庸的马到了面前,就抬头高声的喊着:“黄使君,还请代禀韩学士,如今即有治痘的良法,莫要敝帚自珍,当念生民困厄,早日颁之于众。”

这一声喊,惹得群情激动,甚至有人跪了下来,一齐求着黄庸,让他去劝说闭门谢客的韩冈。

黄庸停下马,环目一扫周围人众:“诸位父老放心,本官来此正是为了与韩学士商议此事。”

听到了黄庸这位知州的话,人群中立刻响起一片低低的欢呼声,人人喜笑颜开,连声向黄庸道谢。

黄庸高居马上,享受了一阵众人膜拜后,脸又板了起来,“不过尔等于漕司衙门门前,聚众数千,岂不有要挟上官之嫌?暂且归家,静候佳音!”

几个领头的士绅互相交换了一个眼色,低头领命:“谨遵使君之命,我等这就回家,静候佳音。”

说罢便起身纷纷散去,主心骨一走,绝大部分的百姓也都是感恩戴德的向黄庸说了几句好话,随之而去。上千人众,转眼就剩下二三十人,还站在衙门门前舍不得离开。

“你们是怎么回事?”黄庸略皱眉,竟然还有人敢不听他的吩咐。

一群人连忙跪了下来,领头的一名汉子重重的磕了一个头:“大府容禀。小人文三,与这几位家都住在城南。如今街坊中正闹着痘疮,已经有七八家的儿女都染上了,不治身亡也有三人。眼看着就会传到了家中的孩儿身上。不是小人不听大府的吩咐,就是想着能早一步看到方子都是好的。恳请大府体察小人一片舔犊之心,宽贷小人不恭之罪。”

可能是文三读过两年书、说话不算粗鄙的缘故,黄庸脸色缓了下来,点点头,驭马越过他们,直往衙门门前去。作为知州,朝令夕改肯定是不好,但也没必要不近人情,放着不理就是了。

黄庸片言散去了衙门外的群众,一下就被传到了韩冈的耳中。

“才送了信去,人就来了,看起来黄常伯已经有所准备了。”听到外面的消息,韩冈不禁唇角微动,露出一个了然于心的微笑。这么大的功劳,就是宰相都不免心动,何况区区一个知州。

“就是让黄庸捡了便宜去了。”站在韩冈身后的李德新阴沉着脸,“龙图为何不出去说一句话。现在外面的那群百姓,感恩戴德的都是黄庸了。”

“让黄庸得个好名声又如何?我这边是主动行事还是被动受邀,在朝廷那边看过来可是两回事。”韩冈笑了一笑,很不在意,“而且当真会影响到我在襄州百姓中的名声吗?……可不见得。”

韩冈不介意分功,襄州也好、唐州也好,越多的人参与进来,推广种痘法就会越顺利——世间的许多事,之所以不顺利,就是因为主事者吃独食的缘故——韩冈行事一向如此,他在陇西分了多少利益出去?顺丰行能发展得这么快,就是因为他拉到身边的人多了,少有人扯后腿,有什么阻碍能凭借巨大的势力直接碾压过去的缘故。

“我可是京西转运,为官一任,当造福一方。”

衙门外的司阍又进来了,这一回带着的是黄庸的拜帖。

韩冈微笑着接过拜帖,而司阍通报的另外一人的姓名,更让他呵呵笑出了声。

黄裳……虽然不知道是不是同名同姓的巧合,但对于过去依然有着依稀记忆的韩冈来说,的确是很有趣。

——也仅仅是觉得有趣。‘六五:黄裳,元吉’出自于易经坤卦中的‘黄裳’这个词,在这个时代是个很常见的名字,据韩冈所知,蔡确老子的名讳就是黄裳,唐代更有个叫杜黄裳的重臣,姓黄名裳也不算出奇。

“快请!”韩冈说着就起身,走到院中去迎接。

