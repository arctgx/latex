\section{第八章 欲谋旧地重兴兵(上)}

延州的夏天分外让人难耐。

不仅仅是因为树木稀少的缘故,各家各户烧着石炭的烟气弥漫在延州城的上空,还有人家甚至用那种黑糊糊的石脂来生火烧饭,烟味更是呛人。

这种烟气缭绕、熏得让人头晕的地方,就是种谔现在所在的城市。

虽然身在自家的书房中,种谔也没打算像延州城的其他官宦人家一样,点起香炉,用薰香来抵挡刺鼻的烟味,而是在烟尘中安之如素。

照样看书、照样写字,照样拿着块麂皮擦拭着刚刚得到一柄宝剑。

浅黄色的麂皮沾了点油后,在两指宽的剑身上抹过。剑尖就在擦拭中轻轻颤动,薄如纸页的剑身弯曲自如,竟是一柄难得一见的软剑。

麂皮拂过的剑身清亮如一泓碧水,莹莹光泽中隐见纹理,打磨得恰到好处的锋刃透着森森寒意,而这样的利器却是柔如丝缎,任谁来看,都是难得一见的神兵。

种谔前两日受到这柄剑的时候,也试验过一次,将之弯曲团起,甚至能放进木盒中。而拿出来时则一下弹开,重又伸得笔直。如果是爱剑如痴的郭逵见了,必然视如珍宝。

不过再好的剑也要着意保养,要经常上油擦拭,一有疏忽,就会很容易变得锈迹斑斑。

“太尉,王都巡在外求见。”种谔的亲随来到书房前。

“让他进来吧。”种谔继续低头擦着剑,专注在剑身上的眼睛透着冷漠。

片刻之后,先是种谔的儿子种朴,接着一个身材矮壮,坚如磐石的汉子出现在门口。满面的虬髯,双目神光湛然,因饱经风霜而变得黝黑粗糙的面颊,让不知情的外人根本就看不出他才不过二十出头。

刚刚从熙河路调任而来的王舜臣,就这么跟着种朴前后脚走了进来。

一走进来,王舜臣便冲着种谔大礼参拜:“王舜臣拜见五郎。”

五郎。

听见王舜臣用了这个熟悉的称呼,种谔冷淡的脸上终于有了笑容。

种世衡亲卫的儿子,少年时跟着种朴做伴当,当年因为殴伤贵人家的衙内,不得不连夜逃往秦州。只是七八年一过,如今的王舜臣已经是名震关西的大将,一手连珠神箭在天子的面前都挂着名。际遇之奇,也是让世人闻之惊叹。

只是现在两边的关系就有些让人烦心。王舜臣算是种家的家生子,但如今已经是一方镇将。按如今的世情这个名分还在,不过继续将王舜臣视为下人,就是亲家要便仇家了。

王舜臣乃是枢密副使王韶的爱将,在河湟开边时立功甚多,同时也与未来必然在两府中有一席之地的韩冈以兄弟相称,一手冠绝当今的神射更被天子所喜爱,又怎么可能像过去如仆役一般视种家为主。只是上下尊卑的观念在世人心中根深蒂固,种谔何能例外?故而心头一直转不过弯来。

幸好王舜臣的表态让心高气傲的种谔松了一口气,“坐!”

种朴见到父亲的态度软化,也放下了心,扯着王舜臣站起身,一起在种谔的下手坐下来。

种朴可没有他老子那么多纠缠的心结。他自幼与王舜臣一起长大,如同亲兄弟一般。王舜臣当年之所以远走秦州,其实也是因为在帮他种十七出气的缘故。这一次请调王舜臣至鄜延,虽是大伯种诂的建议,但若没有种朴在后面的推波助澜,种谔也不会这么容易上本奏请天子。

善待王舜臣,是现今种家上下一致的意见。不仅仅因为王舜臣本身还有很大潜力,也有王舜臣身后的韩冈这一重要因素在。有着种建中的同窗之谊,再加上王舜臣这位与韩冈兄弟相称的生死之交,就能与韩冈相与交好,不仅日后很可能会有几十年的依仗,现在就能在王安石和王韶面前再多上一条路。不至于在第二次攻略横山的时候,受到来自朝中的干扰。

种家可是吃够了朝堂无人的苦。种世衡当年在西军中,人人将他与狄青相提并论。起头时,两人所立功业也相差仿佛,修了清涧城、施有离间计的种世衡其实还更强一点。可是狄青占了几个宰辅看重,日后飞黄腾达,最后竟是靠着剿平侬智高之乱,而坐到了正任的枢密使。至于种世衡,则终官正七品的东染院使,横班只在眼前不远,可就是没能踏过去。有鉴于此,种家如今执掌家门的几位,如何会放过前途无量的韩冈不去交好?

王舜臣坐了下来,视线当先落在了种谔手上的宝剑,武将的习惯让他一时间忘了礼节,两眼发亮:“好剑!”

“前些日子才拿到手,是磁州名匠解良所造。”种谔说着来历,将剑反手递过去。

王舜臣接过来上上下下看了一通,又就手挥舞了两下,晶莹闪亮、柔韧如蛇,却不会因为太过柔软而妨碍挥舞的剑身让他啧啧称叹,“果然是好剑!也就只有磁州的刀剑大匠才有这样的好锤头。”

将剑双手捧着还回去,王舜臣笑道:“不知五郎打算将这柄剑起来起个什么名字?”

“剑就是剑!杀人的器物,要名字作什么?”种谔刷的一声收剑归鞘。作为一名武将,种谔当然也喜欢收集神兵利器,但要说他有多把这些刀剑放在心上,那倒也未必。抬起手来,就把剑再丢给王舜臣:“要想起,自己想个好名字去。”

“当真?”王舜臣也不推辞,喜笑颜开的起身拜谢道:“多谢五郎的赏赐。”

王舜臣外表看着粗豪,但为人却是精细,自小跟着种朴作伴当,怎么可能不学着察言观色。说话处事,也都保持着分寸,而一点点粗鲁,反而透着亲热。熙河路中的将领里面,他在军中的人缘是最好的。该一起骂娘的时候一起骂娘,该一起喝酒的时候一起喝酒,时常呼朋唤友出外游猎,在熙河路的军中,结下了多少铁打的交情来。

他若是说什么无功不受禄,那反而就生分了。现在虽是毫不客气的接受下来,但却更显得亲近。王舜臣自幼在清涧城长大,跟着种家也久了,也不会因为现在身居高位了,身后又有够硬的后台,就认为能与种家分庭抗礼。而且若是被人认为是坏了品性,那就别想再往上走多远了。

收下了剑,王舜臣喜滋滋的坐下来,“前日一听五郎要调俺来鄜延,俺当天就想骑着马赶来了。在熙河路的这两年,鸟都淡出来了。一张弓,射下来全是野鸡野兔,好一点的就是野鹿野猪,偶尔射了只大虫熊罴,就要敲锣打鼓了,跟不见来个贼人好让俺练练手的。对了,前两天还弄了张黑白纹的花熊皮,俺娘说给大郎旧时从马上摔下来的时候伤过腰,花熊的皮子正好用来护腰。”

王舜臣杂七杂八的说着,毫不见外,亲热得就是一家人,种朴也旁边帮着腔,种谔渐渐的话也多了起来。看着王舜臣的态度,就是自家的子侄一般。

喝了一巡茶,说了一阵话,种谔将茶盏一放,神色变得严正起来:“王舜臣,你可知今日我请调你来鄜延路是为了什么?”

王舜臣站起身,单膝跪倒:“请太尉指派,末将无有不从!”

“就是为了横山。”种谔前倾着身子,俯身对着王舜臣:“你也知道,从老太尉在的时候,就一心要克复横山,熙宁元年,我费尽心力将绥德城拿回来,也是为了横山。五年前,西军上下并立一击,筑起了罗兀城,那时已经是胜券在握,谁能想到因为庆州军叛乱而功亏一篑。”

“只差一步啊……”种谔至今说起当年事,遗憾、悔恨依然充满胸臆,要是能再坚持几天该有多好?!眼见着就要多得最后的胜利,却还是没能将之抓到手中。现在想来,错就错在他押错了宝,压倒了韩绛这个不值得下注的赌徒身上。

“你虽是延州东路都巡检,但治所年前已经迁到绥德城。绥德城中的鄜延路第七将的十一个指挥,四千五百马步兵归你管辖。”种谔沉声说道,“调你来此,不为他事。就是攻取横山时,由你来为全军打头阵。”

旧时的一个城寨里,通常都会有分属不同军额的军队,而且是有禁军、有蕃军、有乡兵,令出多头,指挥调动起来很是麻烦,经常会贻误战机。现在随着将兵法在陕西推广,则是按驻兵的地域划分,以三千到一万人为一将,将同驻一地的军队整编起来,自此可以灵活指挥。

鄜延路如今分为九将,王舜臣作为都巡检,为第七将的正将。手下管着四千五百马步兵,总共分为十一个指挥。这些事,王舜臣在接下调令时就知道了。

“当真让俺做先锋?!多谢五郎抬举!”

王舜臣听了又是大喜,跳起来又向种谔拜礼称谢,不是收到宝剑时的带着一点伪装的道谢,而是发自心底里的欢喜,他可是盼着战场上的血腥味盼了整整有三年了。

