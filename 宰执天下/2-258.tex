\section{第37章 青山声碎觑后影(一)}

已经是第三天。

昨日景思立率部上来了。泾原军的主力也抵达了狄道。依照王韶的分派,姚麟将会前往狄道北方的临洮堡,负责抵挡禹臧家的任务。而将剩余的秦凤军,一部分守护珂诺堡,剩下的将会和姚兕所率三千泾原军一起,与主力在河州城下会合。

可王韶所率领的熙河军现在并没有攻到河州城下。

在他们的下方,是离水支流所在的谷地。这条支流与离水的汇合处,便是河州城的所在。

河州城下,听从木征召唤而来的数万吐蕃士兵将离水河谷整个占据,并且瞅准了宋军攻来的方向,不断遣出骑兵进出于支流谷地。

木征军逐日上攻,只要宋军一有进驻谷地的动作,便立刻发动攻击。下山的宋军但始终无法顺利展开阵型,同样也难以在蕃军的攻击下,设下营地。总是很快就会被赶上去。

吐蕃人以骑兵在谷地奔驰,速度比起宋军要快得多。不论王舜臣和苗授选在在哪里下山——甚至是穿过山坡上的树林,刻意挑选让人意想不到的地方——可只要一露头,数千蕃骑也很快就会迎面杀来。

支流谷地成了战场。由于战场范围的限制,河州城下的数万蕃军只能有一小部分投入进来,但也有万人左右的数量。也因此,宋军纵然能够将战力在同一时间投入战场,但兵力上的劣势依然存在,想方设法也无力向前推进过去。

一时间,王韶被逼得只能在山坡上段,下令全军驻扎下来。

草木深重的山坡上,根本安顿不下加上景思立的秦凤军,总计超过一万的兵马。砍树伐木出来的空地,也只安排下其中的三千人。迫不得已只能在山南的险坡处,扎了几个连珠小寨。最后还是有两千骑兵,不得不回到香子城中。

再过两日,秦凤军的余部,再加上三千泾原军就要过来。如果不能在他们上来之前打开局面的话,这个脸可就要丢到整个关西去了。

王舜臣踢着脚回来,这些天他上马下马,在山道上来来回回的走着,牛皮缝制的马靴前头都张开了张嘴。心里想着要回帐换双靴子,但他还是往王韶的主帐走去。

王舜臣今天又试了一次。赵隆的三百名选锋,还有他在麾下本部中选取的四百射手,在晨曦未上的时候冲下山坡。

七百精锐的列阵速度很快,在吐蕃蕃骑赶来之前的片刻工夫中,就已经排下了阵势。用强弓硬弩轻而易举的就遏制了多达千人的蕃军骑兵的突击。

而后苗授也找计划随之出阵,但他们下来的速度实在赶不上骑兵的迅速,被另一支千骑上下的蕃兵硬是堵在山林中。西军步卒虽是堪战,但如果不能组成阵列,对上骑兵还是没有多少胜算。

眼见苗授那边支援不上,又发现了第三支吐蕃骑兵已经冲入了谷地,有反抄后路的打算。王舜臣当机立断,立刻将人撤了回来。断后的他大发神威,一张长弓射落了数十名咬着尾巴追击上来的蕃骑。就如单薄的堤坝,挡住了滔天而来的洪水。

只是个人再是武勇,无法改变计划的挫败。

‘难道真的要等泾原军过来?’

有了泾原军,加上秦凤军的余部,总计两万兵马,的确可以杀进谷地,修起与河州城对峙的营寨。但王舜臣明白,王韶需要的是对麾下战力的绝对的控制权,如果在窘迫的情况下,必须靠姚兕姚麟才脱离困境,他这个经略使的指挥权肯定要打折扣。

今晨一战,苗授和他不动用昨日抵达的秦凤军,而只是熙河军出阵,也是因为明了王韶的想法,才如此行事,只是最后还是失败了。唯一值得庆幸的就是没有太大的伤亡。

“怎么办?”进了营帐,王舜臣就听着有人在问道。

王韶、高遵裕都在,景思立和他秦凤军中的几个得力将校也在,苗授、赵隆比王舜臣更早一步安抚好士卒,也更早一步到了主帐中。

“还是夜间立寨。”出言提议的是王存,景思立麾下的将领,也是昨日刚到的。

在王韶的首肯下,王存说着他的计划。用半日的时间,将一方方木排从满是雪水的山溪中放下去,然后在平缓的山谷里捞上来,木栅一圈,什么都好办了。有一个晚上的工夫,可以很轻易的将营地的栅栏给竖起来。

“当我们没想过吗?第一天就这么做了。”王舜臣当即就翻了白眼,“别把木征当傻瓜,他们是敢夜战的!”

“只要有两三个时辰的工夫,就足以将营栅给立起。”

“木征一个时辰的空隙都不会留下来。”赵隆摇头,“这两天一到夜中,山下面全是火光。木征派出不知多少队游骑,日夜巡哨,那是真拼命了。”

“是吗?”在王舜臣进来前,就一直没有说话的王韶这时突然开口,“那你们呢?”

众人的视线集中在王韶的身上。

“不敢拼命吗?”熙河经略双眼如刀,温声问着。

……………………

天色将晚。

战鼓突然沉沉的响了起来。

先是一面鼓在响,继而几面,几十面,到最后,连同山谷间的回声在一起,在天壤间回响。

就算远隔了近十里,雷鸣般的重鼓,依然震颤着人心。听起来就像从天空中传递下来,如同夏日的惊雷叱咤,一声接着一声,并不停息片刻。

青谊结鬼章走出营帐,远远望着鼓声传来的方向。

被堵在山坡上扎营的宋军,让鬼章部的年轻族长心中不屑。见势不妙就不敢一赌勇力,宋人的确没胆。现在突然如此大的声势,可以想见七八分是在骚扰,只有小部分是为了出战而准备。

其实他更希望能直接宋人下山来,这样才能发挥出数万大军的作用。而不像现在,只是一千两千一队的骑兵与宋军进行短暂的交锋,最多也不过万人就填满了谷地,而更多主力只是在后面看热闹。

但木征要给宋人更多压力,以补偿之前放弃一连串城寨给他声望带来的不利影响,另外也为了将宋军引得更深一步而做铺垫。

回头望了望木征主帐前摇晃的旗帜,青谊结鬼章翻身回帐,聚兵的号角也不关他的事,现在还轮不到他上场,不如去睡觉。

鼓声中,王舜臣随着麾下的士兵,一步步走下山坡。

相比起骑在马上,王舜臣更喜欢脚踏实地时的安稳。只要双脚牢牢钉住地面,前方的千军万马,他都有信心用掌中的长弓一一射落。

王韶下令麾下大军分作数队,同时从山头上下来。这几日,为了能杀进谷地,宋军也在树林中开辟了好几条道路,并不是白白浪费时间。

蹄声压倒了鼓声,两队就在谷中巡视的蕃军,冲杀了过来。面对分成数部,同时攻入谷地的宋军,他们并没有分散开,而已一起钉住了最边缘的一队。

分散的宋军,等于是给了他们各个击破的机会,只要冲散这一队之后,就能势如破竹的紧跟着解决接下来的几支宋军。

他们撞上的正是王舜臣所部。

半数军卒刚刚走出山林间,只来得及排出两排单薄的队列,而后面还有一半没有出来。可直面千骑蕃兵,王舜臣没有丝毫退避的意思。

清早的时候无功而返,他胸口中正凝聚一团怒意。眼下只要拖住片刻,援军就能赶来,他坚定不移的站在最前面的队列之中,张弓,搭箭,高声大喝:

“跟我射!”

箭雨如注。

由于王舜臣擅长弓箭的缘故,他对帐下士兵的箭术训练要求最高。而且他从小就听说过种世衡如何引诱民众习练箭术的故事,用悬银为赏,谁能射中,就将银子赏给其人。借鉴了种世衡的故伎,只用了半年时间,王舜臣麾下的军卒箭术便都提高了一大截。

弓箭的射速远过弩弓,一轮两轮三轮的急速射击,吐蕃骑兵也刚刚前进了二十步。前排.射过,后排紧随而上。不射人,专射马,几轮下来,蕃骑的先头部队中,已经满是痛得疯狂乱跳的战马。

还是有蕃人冲到了近前,但纵然他们冲到了身边,纵然身边的兄弟被战马冲倒,但只要王舜臣还在阵前,这一支队伍就依然保持着稳定。

袍泽的鲜血溅在脚前,王舜臣又是一声高喝,指掌中的长弓散射出道道流光,一支支利箭几乎是在同一时刻,穿刺进眼前十几名敌人喉间。

宋军如此顽强,让吐蕃人顿时感受到了与前几天截然不同的压力。原本只需要一个冲锋,让宋人见到没有立寨的余地,他们便退回山上去。几天来,十几次的反复交锋,蕃人们也习惯了下来。但今天宋人一拼命,反而轮到他们节节败退。

王舜臣的坚持,让其他几路宋军有了结阵的机会。当木征从河州城下派来的援军终于赶到的时候,面对的已是四个完整的大宋箭阵。

“已经撑不住了!他们都快要压倒谷口了。”半个时辰之内,已经不止一个木征辖下的蕃部族长在他面前叫苦,“宋人的木排又从山上放下来了,他们是真的要立营!”

木征的声音毫不动摇:“再去!否则定斩不饶!”

他环顾悚然而立的诸将,“得胜太过轻易,反而会惹起怀疑。前面退了,现在就不能再退。拼过一场后,才能让王韶知道我固守河州的决心。”抬头看看已经化作深蓝色的东方夜空:“坚持到月亮上来的时候!”

