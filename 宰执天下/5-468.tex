\section{第38章 何与君王分重轻(26)}

“早作打算?”向皇后惨然而笑,“吾哪里敢!”

蔡确一惊,发现自己说错话了。不过也来不及悔改,只见皇后冷笑着,“不过是经筵上不让他丢人,就换了个皇后害我,要是早作打算,还不知会写什么!”

见皇后又有激动起来的样子,蔡确连忙的叫着,“殿下,殿下!还请息怒,还请息怒!”

皇后哪里理会他:“结发十几年,吾何曾负过他赵仲鍼。写什么皇后害我,相公枢密都在这里,干脆明说吾哪里失德,废了吾这皇后好了!做着也受气!”

蔡确向曾布、韩冈使眼色,他这个宰相手忙脚乱,曾、韩两人倒好,坐在一边看热闹。

“殿下。”韩冈起身劝到,“殿下的辛苦,臣等都看在眼里。殿下的功绩,天下臣民也都看在眼里。今曰之事,是天子失心,非是殿下失德。世间自会有公论。”

曾布也接上去劝着:“当年仁宗皇帝也曾经失心,上殿大喊慈圣和张茂则要谋反,但哪个朝臣不知,这是仁宗病糊涂了。张茂则至今犹在宫中,慈圣之德更是为后世垂范。岂会有人糊涂到不知是非的地步?”

“这半年,吾劳心劳力,天天都在担心受怕,为的什么?还不是为他赵家!”

“幸得殿下,辽贼入寇,国家方能得保全,否则河北必然糜烂,河东也难挽救。这件事陛下虽不知道,但天下又有谁不知。”

皇后哭诉了一阵,好不容易在三名宰辅的劝说下,抽抽噎噎的,终于算是平复下来。

只是片刻,蔡确和曾布就急出了满头汗,坐下来后相顾无言,跟妇人说话真是累。

向皇后呼吸渐渐平稳了,拿着手巾擦了擦眼,问韩冈道,“枢密,接下来该怎么办?”

“殿下,以臣之见,还是尽快招平章和两府宰执都入宫。”韩冈将方才蔡确说的早作打算抛到了脑后,不把人召集起了,怎么打算?

“韩枢密,此事不能操之过急。”

曾布表示反对,没开口的蔡确也轻轻的点了点头。

韩冈双眉一扬:“如何不急?天子突发心疾,怎么能不尽快通知各家宰辅?这岂不是要隔绝中外?!”

曾布哪里想到韩冈随手就栽了自己一个隔绝中外的罪名,他只是担心深夜招宰辅入宫会惊动京城,当然他也是不打算放过这个难得的机会。韩冈一句话将他气得七窍生烟,皇后对韩冈的信任显而易见,韩冈话说得这么重,“韩枢密是急着要让天子内禅吗。”

韩冈沉下脸:“韩冈可是有半字说内禅?参政如何以不实之罪污我?!”

“有与无,枢密心中自知!”

韩冈倒是不气了,心平气和的问:“就是周兴与来俊臣,想要入人以罪,也得先弄个大瓮放在火上。参政倒好,有罪无罪,要我心中自知。不知参政在外知军州数年,都是这般断案的?”

“玉昆,不要置气。”蔡确住来劝和,“子宣参政只是口误,并非真意如此。通知介甫平章、子华相公他们是应该,但也要防着人心动荡。”

韩冈霍然而起,“相公!秘而不发,正是人心动荡的原因所在!”

在韩冈眼中,今夜陪同宿卫的两人,一个蔡确、一个曾布,都是不能相信的,就是王安石也比他们更可信任。从曾布的态度上看,他很有可能想趁这个机会提出内禅,否则他不会这么抵触招宰辅入宫的提议?攻击自己的借口也是用了内禅,可见他心中至少才盘算过,所以才脱口而出。

“玉昆。皇城大门夜不轻开,现在派出内侍、班直去传话,京城军民恐怕不免会胡思乱想。”

韩冈根本就不理会蔡确在说什么。他要坐实赵顼已经发疯的消息,只有将宰辅们尽快招入宫中。要是明天上朝后才让王安石、韩绛他们知晓,心中有了疙瘩,问题可就大了。单是为韩琦抛下自己,单独逼迫曹太皇撤帘一事,富弼就记恨了一辈子。韩冈并不怀疑,一旦给了蔡确和曾布机会,让他们说服皇后,今夜就内禅太子,明天之后,自己会不会被其他辅臣恨之入骨。

而最重要的。宰辅漏夜入宫,京城上下今夜不知有多少人难以入眠,等明天,天子发病的消息传出去,人人恍然大悟,事情就坐实了。他早就有了定策之功,就算今天拥立太子,也不会增加多少功劳,反倒是当初没有参加进来的蔡确、曾布最盼望这个机会。

皇后现在气得发狠,很有可能被蔡确和曾布说动,韩冈是宁可当场翻脸,也要让皇后下诏将王安石、章惇他们给召进来。

韩冈就站着,也不继续反驳,只是冲门口看了一眼,又点点头。巡视宫掖的王中正就在那里,全副披挂,就是一副武将的打扮。

王中正一句话不说,低眉顺目,站在门后仿佛门神一般。但韩冈冲他点头,王中正就仿佛从雕像变回了人,重新有了生气,同样点头,回了礼。

向皇后没注意到这么细微的动作,但蔡确和曾布无法无视。那可是宫中兵法第一,半年内统帅班直的内侍大将,而且跟韩冈交情匪浅。他一点头,就意味着韩冈并不需要他们同意,才能将消息传到王安石、韩绛,以及其余宰辅那里。

“殿下!”蔡确大声道,“臣和曾布,并无阻止他宰辅入宫之意。都是怕连夜打开宫门,会让京城百万军民人心动荡,万一有贼子图谋不轨,恐怕会生出大乱。实在是不能不慎重!”

“韩冈岂敢怀疑相公。但吾等三人今夜宿卫,而王介甫平章、韩子华相公他们能安心回去,是相信天子若有不豫,我等能安定人心,并及时通知宫外。早一步让其余相公、枢密知晓宫中变化,更可安定国人之心。”

“枢密此言是正论。”向皇后擦干了眼泪,挺直了腰杆。“不能让相公们在外面担心。”

向皇后这句话一出口,蔡确和曾布都安静了下来,先后拱手道,“殿下所言极是。”

不过他们看韩冈的眼神就不一样了,已经是带着恨!

韩冈早就有定策之功在手,根本就不需要画蛇添足,可他们缺功劳啊。自己吃饱了饭,就不让别人吃。殊不知,这有多招人恨?

韩冈之前对王安石说,并不主张赵顼立刻禅让。但现在的情况又不一样了。

赵顼竟然说皇后害他。

也许是怨气曰积月累后的爆发,或是苏醒后精神混乱的,或是当真认为他这一次发病源自于皇后中断经验,但这一句话一说,向皇后就再也没有转圜的余地了。

“天子的病情已经到了这般田地,不论有什么想法,都是得早点有个章程出来。一旦议定,就算今天夜里人心不安,明天也会安抚下来的。”不管接下来会怎么样,有事情大家一起承担,这是韩冈的想法,“殿下,还是快派人吧。”

……………………“三更天了。”

听见外面的梆子响,蔡京确认了现在的时间。跟强渊明喝酒,不知不觉就喝到了半夜。虽然自家酿的葡萄酒并不是烧刀子那般能打着火的烈酒,可喝多了下去。照样还是唯有醺意。

“怎么,舍不得你家新酿葡萄酒了。”在后院的石桌下,与蔡京一起喝酒的强渊明舌头有些大,已经是喝了不少了,但还没有到醉倒的时候。

蔡京笑了笑,举起手中的酒杯,对着天空毫无遮掩的月亮,虽说喝葡萄酒最好的就是夜光杯,但有个玻璃盏装酒,瑰丽的红穿过玻璃之后,就又多了一层晶莹剔透。

“这葡萄酒又不比过去珍奇,现在家家都会酿了,能喝多少只看隐季你的酒量!”

到底怎么制作葡萄酒,不知何时就在京城中传开了。不需要蒸酿的酒具,也不需要酒药,只要将葡萄洗干净和交州白糖一层一层的放在陶瓷罐里,然后密封好放在一边,等一段时间就变成葡萄酒了。剩下的就是过滤和装瓶。

只要家里有葡萄藤子,又买得起交州白糖的人家,都忍不住去自酿些酒水出来。一时间,弄得京城的酒税跌了两成还多。

“既然这么说了,我也不客气了。”强渊明招呼着蔡家的仆人端酒上来,又对蔡京道,“元度还是不肯出来?”

“元度的情况,你也不是不知道。体质又弱,出去喝次酒回来就要吐几回。今天上殿吗,估计是中了暑,回来后就躺下来了。”

“真的是中暑?”

“喝你的吧。左右明天就知道了。”

经筵上究竟发生了什么,蔡卞回来后却不肯说。只是知道天子在经筵上发了病,然后宰辅们都进了福宁殿。具体的细节一概奉缺。

蔡京准备等到明天再去了解。反正大体情况猜都能猜出来,自家的兄弟,终究比不上韩冈。很有可能是吃了大亏,否则就不会一回家就躲进了房中,谁人也不见。

“三郎。”蔡京家的一名老家人从前院匆匆而来,附在蔡京耳边说了几句。

见蔡京脸色陡然就变了,强渊明立刻问道,“出了何事?!”

蔡京慢慢放下了酒杯,轻声道:“王平章和章枢密又入宫了。”

