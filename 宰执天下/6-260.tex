\section{第34章 道近途远治乱根(下)}

张孝杰终于出去了,耶律乙辛小小的松了一口气。

让人倒了一杯温热的鹿奶,咕嘟咕嘟的灌了两口。喝得急了,不小心呛了两下,内侍赶忙拿着手巾上来。

一阵撕心裂肺的咳嗽后,耶律乙辛感觉肺和喉咙火辣辣的,几乎都要烧起来了。拿开手巾,低头看着紫色巾帕上的奶白色的痕迹,大辽天子从心底里,泛起一股岁月不饶人的疲惫。

当真是老了。

他已经老了,不用照镜子,低头看手就够了。

手背的皮肤,青筋毕露,沧桑得仿佛就像一层陈旧的薄纸,靠手腕的位置更是悄然生出了代表年老的黑斑。

“大概人老了就开始念旧吧。”耶律乙辛喃喃说着。

不然会这样一再容忍张孝杰与自己唱反调?好像他说的那些话,自己不明白一样。

“陛下?”

正趴在地上,努力擦着地毡的内侍没听清,抬起头,疑惑的问着。

耶律乙辛轻轻阖上眼帘。

在篡位近十年后,从宣帝开始就跟着他的一干老臣子如今剩下也不多了。

有的告老,有的病故,有的战死,还有的因为首鼠两端被他处死,也就张孝杰还跟在他的身边。

从私心上讲,张孝杰不算贤德良臣,过去更是被视为奸佞。

但他有见识,有能力,这几年又刻意打造了一身直言敢谏的孤臣形象,谁都不亲近,也不追求自己的势力,耶律乙辛不用他用谁?

只是他对女直的提防,实在是让耶律乙辛无可奈何。

难道有什么事情是自己不知道的?!

耶律乙辛缓缓坐直身子,“去招燕王来。”

耶律乙辛次子封燕王,平日长居日本,领着八千本国兵马驻守在倭国都城平安京——如今已经改名做海安府——一般只有在年节时才会回本土。

完颜阿骨打跟着他的这个儿子,高丽、倭国,都是他们给打下来的。正好还有些事情,耶律乙辛也想问问清楚。

皇子们的帐幕离御帐都不远,耶律乙辛没有等待太久。

“父皇。”

随着声音,一人掀帘而入。修长笔挺的身材,年轻英俊得让人嫉妒。

相比起来,耶律乙辛的太子就略嫌文弱了。

“别拜了,又没外人,坐吧。”

让儿子在旁坐下,耶律乙辛半眯着眼睛,不紧不慢的问道,“你上次说的卖人给南朝的事,再给为父说说。”

近几年,辽国从各种渠道购买来的南朝丝织品,已经有两成是机织。这让耶律乙辛对南朝开办的丝厂十分有兴趣。

如果从‘两成’这个数字来推算,这几年,南朝丝织品的产量至少涨了有半成。

而以南朝的丝绢产量来说,百分之五也已经是个惊人的数目了。

尽管昔年宋人给付大辽的岁币中,那三十万匹绢帛不过是两浙治下区区一州贡赋之数。可仅仅是百来家新建的丝织厂,每一座工厂的产量就能达到一州的十分之一。这样的技术进步,当真是很可怕了。由不得耶律乙辛不重视。

尤其是在他在日本的二儿子写信来说,宋人要买倭人回去做工,这就更让耶律乙辛想要一探究竟。

“其实就是有几个南朝的海商,过海到海安府的时候,顺口提起的。说是南朝好些家丝厂招不到工,都嫌活计太苦,给再高工钱都不干。”

“在丝厂里面做工能有多苦?”

耶律乙辛知道工匠的辛苦。但南朝的丝绢根本就是另一种模样的钱。铸钱的工坊再苦再累,管事的也不会涸泽而渔,去催逼匠人。流淌在厂子里面的是不竭的金钱,而让金钱流淌的正是这些工人,谁会做杀鸡取卵的蠢事?

“孩儿也这么问的。那些海商说,做工时什么都是一板一眼按规矩,一点都不带通融,想喘口气都得被呵斥。那些做工的,一个个都是懒骨头,受不得这样的约束。后来听说倭人听话,就想到了来倭国买人。不过私下里,孩儿还听说,那几家丝厂都是年底才关账发钱。”

耶律乙辛听的都是一愣。

即使是住在家里的长工,不说按月结,也得按季来结清工钱。丝绢这种跟蚕茧季节走的活计,更是应该在冬季前就结账的。这到了年底,哪家的丝绢是到年底才上机织的?

这也太黑了吧?耶律乙辛都觉得匪夷所思,如果是要养家糊口,做这份工,等拿到工钱回去,就只能看见饿死的妻儿父母了。

耶律乙辛将话摁在心底,又问道:“那些海商是怎么说的?还真就是上次你在信上说的,不要男丁。”

“的确是不要男丁。除此之外,也不要四十岁以上,以及得病和有残疾的。而男童、女童,妇人都可以,只要手脚齐全就行。一月一贯工钱,且包吃包住,先给五匹绢做安家费,年底结账回家。”

耶律乙辛听得就露出一抹怪异的微笑,“他们买这些倭人,当真只是想要办丝厂?”

“应当不会有假,否则就该要男丁了。”

“我还以为他们是想要做善堂呢。”耶律乙辛冷笑着,“这么好的差事,怎么会招不到人来做?妇孺都能做的差事,这要有多简单多轻松啊?!”

“孩儿是听说抽丝剥茧是要将手伸到开水里,将线头从蚕茧上抽出来。那工厂里面,到处都是滚水——用锅炉烧开的。”

“原来如此。”

耶律乙辛点着头,这就水落石出了。

如果是养蚕户自家缫丝,端个水盆,一次只要顾好一头茧子。而工厂里缫丝,说是比蚕家快几十倍,那一次肯定就是要照顾几十头茧子。这手,当然就得不停地往滚水中浸、

隔三五分钟烫一回,一分钟烫三五回能一样吗?哪个人的手不是肉长的?好端端的人进去,最多也只消两三个月,手上的皮肉多半就煮烂了。

两只手废了,这人还能活吗?

完全是要人命的买卖,这才把所有人都给吓跑。否则好端端的,找那些连汉人的话都听不懂的倭人做什么?不就是因为骗不到附近的人了,只能找那些背井离乡的倭人欺负。

“难怪韩冈不做。”耶律乙辛叹着,“去了宋人的丝厂,一年下来,能有一半活下来就不错了。那些倭人妇孺,恐怕没几个能活到拿钱回家的那一天。”

“当真是作孽啊!”大辽天子悲天悯人的一声长叹。

“父皇……那倭人,我们就不卖了?”

“卖,为什么不卖?倭人的丁口卖得越多越好,男童也卖,但妇人、女童不卖,国人在倭国的人口太少了,没个百万,这片地占不稳。你回去跟那些海商说,高丽这边的人,也可以卖。”

“但没了丁口,这粮食?”

“多用牲畜,多请教老农,不用担心粮食。少个几万张嘴,还能多省下些地皮来种棉花。”

“种棉花?父皇是想要造棉布吗?日本多山,其实更适合植桑养蚕。”

耶律乙辛摇头,“丝绸对我国无用,真正有用的还是棉花。”

冬天的严寒,对这片土地上的任何生灵都是一种考验。

即便有了耶律乙辛对医疗制度和技术的重视,每年冬天,各个部族都要失去大量的人口。

棉布在辽国,乍看起来比皮裘卖得要贵。可若是按照面积来算,将一张张羊皮拼凑到一匹棉布的大小,价格可是棉布的近十倍了。

如果棉花不是来织布,而只用来填充被褥和衣料,这种种在地里、一年一收的植物,自然要比羊皮要强得更多了。

一亩好地能产两三百斤麦子,用来种棉花,往少说也该能收上百斤了,一亩草地能养一只羊吗?

耶律乙辛把自己的想法跟儿子说了,倭国的土地,应当用来养辽人,而不是用来养倭人的。

只是他说得兴起,最后儿子离开,耶律乙辛歇息下来时,才想起自己倒忘了问儿子对女直人的看法了。

不过这也不是什么大事,也就一句话就能处理了。

鸭子河冻结的冰面上,一群女直人凿开了冰洞,洒下了春日的第一张网。

号子声此起彼伏,由旦至暮。

河冰上,一片片银鳞闪烁。

夜幕降临,星空笼罩着大地。

河畔的荒原上,篝火多如繁星。

耶律乙辛的大帐中,数百部族的首领云集于此,将新年后,从鸭子河中捕上来的第一网鱼,进献给大辽皇帝。

大辽天子雄踞帐中,大部分时间都是半闭着眼睛,礼仪上的事务皆由太子应付,忽然间他开口:“就这么喝酒没意思。乌古乃,阿骨打,你们父子两给朕跳个舞吧。”

喧闹的帐中静了下来,数百双眼睛顿时汇聚在完颜乌古乃和他的次子阿骨打的身上。

要生女真节度使,几乎可以算做是生女真之王的完颜太师和他的儿子上场跳舞助兴?这是因为两人做错了事,现在要当众进行惩罚。

完颜阿骨打缓缓的放下了手中的烤羊排,抬头看着前方的父亲,却握紧了手中的银刀。

跳还是不跳?

女直诸部的首领都在这里,要是谄媚一般的跳了舞,这样的屈辱,即是几十年后,与各部相会,都会被人当成笑话提起。完颜部多年树立的名望,都有可能在转眼间崩塌。

却见完颜乌古乃欣欣然站起身,毫不犹豫的走到场中。

阿骨打只是停了一下,也放下了银刀,紧紧跟了上去。

当然要跳。为什么不跳?

听大辽皇帝话难道是件丢脸的事?

或许如此。

但听强者的话,那绝对不是件丢脸的事。

如今的大辽皇帝,只要一句话,就能毁掉完颜部,这样的强者,只应该跟随,而不能反抗。

在荒野上,即使是狼,也得群聚一处。跟随最强壮的头狼,是每一头野狼都会做出的选择。

但只要这头头狼依然强壮,那么其他狼都会跟随到底。

父子欣然起舞,没有半点犹豫。
