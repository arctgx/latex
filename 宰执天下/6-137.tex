\section{第13章 晨奎错落天日近(二)}

 太后有恙,而且是昏迷。

这是要重到什么样的程度,才会在上朝的半道上晕倒?!

而且看杨戬慌慌张张赶出来通报的样子,就知道太后并没有来得及给他吩咐什么,肯定还没有醒过来。

这样的病情谁也说不清是轻是重,但至少不是普通的伤风感冒。会不会变成难以治愈的重症,甚至到了最坏的局面,更是让人心中忐忑。

如果太后的病情就这么散布出去,天知道会变成什么样的情况。

恐怕有心人立刻就多了起来,至少圣瑞宫那边肯定要动心思了——朱太妃盼这一天不知盼了多久了。

朝臣们为此面面相觑。

这才太平几天功夫,怎么看着就要又乱起来了?

皇后垂帘,上皇驾崩,太皇宫变,宫外就不说了,宫里面的事情都是一桩接着一桩,完全没消停过,好不容易安生了一年,这就由出乱子了。

大部分朝臣最怕的还是宫中不稳。那时候,朝堂上少不了要乱上一阵,想自清自净的都免不了要卷进漩涡里去。一个不好,就会翻船,这辈子的辛苦都要打水漂了。只有那些想趁着浑水,挣上一份功劳的乱德之人,才会兴奋不已——机会终于来了。

不过吕惠卿等待的机会没了。

吕嘉问暗叹起来。他不是宰辅,没权力跟着往后殿去,只能随班退出宫中。

而在外的吕惠卿为了回到两府班中来,费尽了心思,想要靠军功,太后这么一倒,什么辽国都要抛到闍婆国去了。倒是回归两府的希望,却多了那么一两分。

被丢下的张璪看着已经走光了的前排,无声的叹了一口气。

押班的差事是宰相和参知政事轮班分领,在文德殿带着不厘务的朝臣向空无一人的御座行礼。而有实务的朝臣则是参加垂拱殿的常起居,这一朝会是天子和垂帘太后必至。

在垂拱殿率众押班退朝,张璪还是第一次,但韩绛、韩冈都跑了,他不出面,礼仪上其他朝臣真都不好走。

正常应该是资历最浅的一位留下来,可韩绛偏偏留下了他张璪。

也不知道这算是坏事还是好事,或许可以这么想,韩绛不希望韩冈留下来,让他有机会直接控制军队。

当然,更有可能的就是韩绛认为,韩冈作为传说中的药王弟子,越早赶到慈寿殿,越是对病倒的太后有好处。

只要不是韩绛觉得自己可有可无就好,张璪这么想着,一边出班领头向着空无一人的御座开始行礼。

杨戬正冒汗,撑在地上的手连打了两次滑,差点没一头撞在黝沉的金砖上。

宰辅亲自出手,真是寻常人一辈子都难见到的场面。

上一回,是蔡确遭殃,这一回就落到了他这个小小的内侍身上了。

王安石年纪老大,但力气可不小,尽管只是嫌杨戬挡道推了一下,但他跑过来之前,腿脚早都吓软了,别说体格高大的王平章一伸手,就是削瘦矮小的曾布还在这里的时候,也是一根手指就解决了。

爬了几下,杨戬好不容易才起身。

殿中张璪已经开始率领群臣参拜,杨戬小心翼翼的向后殿退出去。

抄近路直接穿过后殿,从后门出殿时,已经看不见先行一步的宰辅们的身影。

杨戬的脚步立刻就快了起来,已经通知到了,现在就得回去复命,还得尽快追上前面的宰辅。

慈明宫是新修,位置在保慈宫后侧,

方才眼睁睁看着太后晕倒,杨戬在旁魂飞魄散,整个人都懵了,跟着随行的宫女、内侍一起哭喊起来。还是多亏了同行的李宪反应快,一脚把杨戬踢醒过来,要他来垂拱殿通知一众宰辅。

返回慈明殿的路上,杨戬右边的大腿上给李宪踹得一阵阵的抽痛,但他不敢耽搁,拖着腿,一拐一拐的着向前快步走着。

沿途的班直禁卫都是一脸紧张,方才他们纵使没看见太后晕倒,至少也是亲眼看见太后的鸾驾在快到垂拱殿的时候,突然停了下来,然后就原路返回,往后面的慈明殿退回去。

杨戬相信他们都不糊涂,知道这意味着什么,更加上宰相、参政和枢密使们一窝蜂的往后面跑,再蠢也该明白了发生大事了。

当真是大事。

在犯过一次糊涂之后,好不容易才又重新得到太后的宠信,如今太后偏偏又病倒了。万一有个什么不测,让圣瑞宫中的那一位得了志,老天爷哪里还可能给自己第三次机会?

杨戬知觉得嘴里发苦,心中将阿弥陀佛翻来覆去的念了一遍又一遍。脚步尽可能的快速移动着,但前面还是不见几位宰辅的身影。

从垂拱殿后的侧门出来,往慈寿宫赶过去,杨戬身后响起了笃笃笃的木底官靴的声音,脚步急促,仿佛有债主牵着恶狗在追。

回头看看,张璪竟然已经追上来了。

明明除了韩冈一个,其他都是些五六十岁的老头子,怎么一个比一个腿脚利索?

前面还没追到,后面倒是追上来了。

既然已经看见了的张璪,杨戬不敢再往前走,停了脚,向路侧退了两步。

越过杨戬的时候,张璪扭头皱眉看了他一眼,却也没再理会他,一路向前,匆匆赶到慈寿宫。

太后所居的慈寿殿外殿中,王安石与韩绛领头,几位宰辅正面对着内殿的重门垂手恭立。

张璪感到有些意外,宰辅皆是男子,又非医者,自不便进入太后曰常安寝的内殿。

但韩冈身负大名,自是应该进去;而为避免瓜田李下之嫌,王安石也得一起跟着做个见证,做岳父的怎么也不会看着自家女婿犯‘错’;王安石都进去了,为了不失首相之心,韩绛也得跟着一起入内;都进去三个人了,其他人也没必要留在外面,一起探问太后,也免得各自心生猜忌。

张璪过来的时候,是这么想的,可他没想到自己过来的时候,韩冈还在外殿。韩冈不入内,其他人可找不到理由去闯一位寡妇的闺房。

心念如电转,张璪立刻提声问道,“平章,相公,太后可安好?”

他面对的是王安石和韩绛,却是向内发问。

“吾无事。”内殿中传来一缕细若游丝的回声。

“太后无事。”随后又是一名内侍尖细的复述。

张璪的心咯噔一沉。听太后的声音就绝不是无事,虚弱气短,与平曰里竭力想表现得稳重的声音迥然有异。只是音调中还能听得出是太后的声音。

“张参政少待,待吾更衣出见。”

把重病的太后逼着出见群臣,张璪不知道该说什么好了,连声‘不敢’,然后跟同僚们站在了一起。

但太后是必须要跟宰辅们见上一面的。

宰辅们不敢妄闯太后闺房,那么太后就得强撑病体出来见一下宰辅。

不出面让他们这些宰辅看个明白,谁也不敢保证门内正与他们说话的是太后,而不是声音相像的另外一人。就算只有百分之一的可能,他们都必须要去质疑、印证。至少这一次必须要见面。

“还有谁在里面?”

回到同僚身边,张璪低声发问。他问得不清不楚,不过他相信,在场的每一个人都是知道他在问什么。

“只有御医,还有天子。”

韩冈用同样低的声音回答道。

“北面呢?”张璪冲北侧努努嘴。

不用多问,韩冈也知道张璪指的是谁。

“没过来。”

张璪安心的舒了一口气,圣瑞宫中的朱太妃没过来,就是最好的消息。

要是她现在就在在内殿中,搂着天子跟太后说话,事情就麻烦了。

不!!!

张璪忽然皱眉,如果她不是因为怕事,而是心思沉稳,也不是什么好事。

门内传来脚步声,张璪的身子立刻绷紧,微低下头,用余光死死盯着门口。

脚步声很慢、很轻,不过很快就挪到了门前。

先出来的是两名内侍,两名宫女,还有牵着天子的老宫人,再下来就被两名宫女搀扶着的太后。

在低下眼帘的一瞬间,张璪的视线从太后身上划过,脸颊蜡黄的,看不到半点血色,完全没有化妆。衣冠倒是穿戴得整整齐齐,坐在外殿正中,就像是正式上朝一般。

“让诸位卿家挂心了。”

生病的时候还被逼迫着起来,太后也看不到什么怒气。只是有气无力,没有人扶着,便坐不稳的样子。但宫女和内侍却不敢近前搀扶,让太后坐下后,便松了手,以便让宰臣们确认。

“臣韩绛叩见陛下。”

确认了太后并未受人挟持,韩绛立刻俯身拜见太后。

韩绛领头,宰辅们一个个都拜了下去。

与其他宰辅不同,王安石很是无礼的盯着太后,直到可以确认为止。韩冈也盯着太后的脸看了一阵,而后才低下头去,与同僚们一起行礼。

即便是宰辅,也少有能看见太后真容的时候,确认太后不是冒充,确认其并未被人挟持,最后在确认还有比较清醒的神智,如此才让王安石和韩冈安心下来。
