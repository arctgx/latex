\section{第六章 仲尼不生世无明(下)}

入了三伏之后,天气越发的炎热了起来。

一个让人无法直视的火球挂在天顶上,给人的感觉却是有七八个太阳一起在散发着热量。天地之间都泛着白光,炫得人双眼发花。

虽然有风,但吹到身上依然燥热难当。无论人畜,无一例外都是没精打采的耷拉着脑袋,道边草木的枝叶也都是蔫蔫的,只有树上的知了,依然在吃饱喝足之余欢快用嘶哑嘈杂的调子在唱着。

虽然已经换了一身薄纱的袍服,韩冈头上的汗水还是涔涔而落,背后也湿透了。眯起被烈日的反光照得发酸发涩的双眼,韩冈有点后悔,如果可以选择的话,他是绝不会选在这个时候,出城来视察汴河边上新作坊的工地的。

“这天气还真是越来越热了。”韩冈虽是这么再说,却仍在烈日下边走边看。

新工坊的围墙已经画好了地界,而通向码头的道路也留了出来,规划得有条有理。韩冈示意随行的伴当,拿起撂在地下的钎子用力敲了敲,只落下了一小撮碎土,看起来厂房的地基也是用心去夯筑了。

“臧樟。”韩冈喊来在这里主持的板甲局管勾官,“的确做的不错。”

管着新作坊修葺之事的老工头正拿着手巾擦着汗,听到韩冈夸他,连忙将手巾往袖子里一收,小跑着上来,“多谢舍人的夸赞。要用几十年的房舍,下官哪里敢不用心。”他偷眼看看韩冈头上的汗水,“舍人,现在正是最热的时候,连小工都歇下来了。还是等到申时,暑气稍稍退了,再来看也不迟啊。”

“我身子还没这么金贵,一时的暑热也算不了什么。”韩冈笑道,他出城来也不仅仅是为了来视察工地,只是没必要对臧樟说:“如今可比当年读书时要好得多了。不比当年,坐在寓居的禅房里,冷了热了都是要硬熬着。”

“天将降大任于斯人也,必先苦其心志、劳其筋骨。”臧樟啧啧称叹,虽然是工匠出身,但显然是读过两年书,他堆起笑脸,“不过舍人这不已经是受了大任了吗?”

韩冈笑了笑。难怪只能在军器监中做事,臧樟的马屁功夫尚有待锤炼,话是说得没错,只是未免显得过于粗糙而少了含蓄。不过看着老工匠也是满头大汗的跟着,韩冈也是知道体恤下属,挥了挥手,“也罢,先回去歇歇吧。”

参与建设的工匠们,现在一个个都躲在树荫底下,享受着清凉。皇帝不差饿兵,韩刚这位判军器监也不能逼着手下的工匠们,顶着能晒死人的炎炎烈日上工。

入夏后的这段时间,这一片工地都是四更天便开工,到了巳时就停工,歇上到了午后暑气稍退则重新开工,一直干到初更。总计的工作时间不变,只是要躲一躲这火辣辣的太阳。

让臧樟回去看着他的手下,韩冈也带着随性之人,回到附近的凉亭中坐下来。

一等韩冈坐下,伴当连忙递上刚刚买来的解暑凉汤。京城人会做生意,只要有人聚集的地方,就能看到小贩们的身影。就在新作坊的工地边,这段时间,有不少小贩靠着从匠人们手中赚来的钱养家。

展开折扇,一边摇着,一边喝了两口已经微温、变得名不副实的凉汤,韩冈抬头看看外面亮得眩眼的汴河水面:“这天气一天比一天要热,河北那边也许会更严重。定州路的旱情,这个夏天也不知道能不能缓解。”

与他面对面坐着的,却是最近又回到三班院中任职的种建中,今天是有事随着韩冈一起出城来。

种建中则是大口的将凉汤喝光掉,痛快的呼了口气,“不管怎么说,比起去年要好多了,听说定州路还没有一起发蝗灾。”

“单纯比灾情大小,的确是不比去岁。”韩冈叹道,“不过这是紧接着熙宁六年七年的大旱之后的又一场旱灾,前两年也许还有一些人家能靠着存粮撑过去,今年就不可能了。”

按照转运使路来划分,河北分为东西二路,可若是按照经略使路——用后世的话说,就是军区——来划分,则是分为定州、高阳关、真定和大名府四路。这一点就跟陕西很像,设有经略安抚使司的,有熙河、秦凤、泾原、、鄜延,还有永兴军这六路,而转运使路,过去是陕西路,如今则是一分为二,分别是永兴军路和秦凤路。两种路份的划分,其辖区截然不同。

继前年去年的大旱之后,河北北部的定州路一带,今年又是遇到大旱。边境地区的灾情,怎么看都是让人。唯一值得庆幸的就是辽人的南京道也同样干旱,同样是流民在道。

韩冈不得不庆幸,王安石回来的时候迟了点,没人将灾情与他联系起来。可是只要新法依然推行,世人用天灾攻击新法的问题就不能解决。

水灾,旱灾,蝗灾,地震,说起来这几年的确有些不顺,这些灾害,大宋都经历了一遍。虽然从道理上来说,这是国家地域太广的缘故,加上气候上的偶然因素,但新党为此而损失的民心,却是怎么也挽回不了的。

只是种建中就没有这么多感慨了,“定州这一旱,就又有流民了……玉昆,祸福相倚啊,黄河金堤这下子又可以开始全力去修筑了。束水攻沙也能更早一步完工。”

“谁知道呢?”韩冈无奈的摇摇头,“不见黄河破堤,不见流民在道,就没人急着此事。朝廷到现在也没有定下谁来的都提举黄河工役,进度能快得起来就有鬼了。”

得了种建中的提醒,韩冈想起了到现在还没有完工的黄河大堤重修工程。

去年一个冬天过去,河北那里仅仅是将黄河北岸的外堤加固了,而且还没有完全完工。要想开始修筑内堤,还不知要等到什么时候。对此韩冈也不心急,慢慢来也不是坏事,反正事不关己。也有人跟韩冈说,这是都水丞侯叔献在案中捣鬼。

主管督促此事的都水丞侯叔献,的确与韩冈有怨——更确切一点,就是他怨恨韩冈,至于韩冈,则没多少闲空去跟他过不去。

两年前为了冬天从汴河运粮入京,侯叔献曾被王安石点将。不过他当初所献的碓冰船成了世人的笑柄,而韩冈所荐的雪橇车,冬天时,在北方则已经十分常见了。侯叔献因此对束水攻沙的治河方略有所偏见,也是人之常情。过去见过几次面,都只是保持着表面上的礼节。但治河之事任谁也不敢做手脚,说他故意拖延,那就是污蔑了,韩冈也不会蠢到去相信。

又在亭子中坐了一阵,看看日影西移,种建中对韩冈道:“玉昆,时候差不多了。”

韩冈点点头,他出来可不仅仅是为了,更是为了迎接张载一行。

出城五十里迎接张载,韩冈恭迎的心情是真心实意,但他不想惹人注意,便找了个视察工地的借口出城来。而种建中位低官卑,倒是没人会在意他的行动,请假也很方便。

韩冈招来臧樟吩咐了一句,便与种建中一起上马。张载在京城中的学生,主要的就是韩冈和种建中。其余大多只是听过一段时间的讲学,算不上是真正的入室弟子。就是因为这样,所以张载才必须出关中一行。

沿着官道,韩、种一行向西而去。

种建中抬头看着天空:“这个天气不宜出行,先生身体一向不好,车马劳顿也不知能不能吃得消。”

“也怪我太心急了。”

“不干玉昆你的事。”种建中连忙摇头,“不赶在令岳进京前先将此事递到天子的案头上去,就算到了秋冬,先生也入不了京城。”

韩冈并不打算跟着种建中一起批评自己的岳父,在这件事上,王安石怎么说也算是退让了。“多亏了吕微仲,要不是他出面,也难以说动王珪出头。”

种建中听了一笑,明白了韩冈话里的意思,便说道:“吕微仲前两天就去秦州了。他守秦州,经略秦风,不知他会不会有什么动作?”

“论性格,吕微仲并不会主动出击,他在河东的几年,也没有见他主动对付过党项人。何况如今夏主做了辽国的女婿,想要打狗也得顾忌着身后的主人。”

”说得也是啊。”种建中闻言一叹,“如今要对付西贼,需要顾忌的事又多了一层。”

两人骑马西行,从身边过去的车马无数,属于驿馆系统的也有不少,但都不是张载一行。一直向前行了十几里,前面又出现了一队车马。

韩冈眼尖,一眼就发现,他前几日派去迎接张载的家丁,就在前面骑着马,混在一队仆从之间。而后面跟着的车马中,竟有许多他熟悉的面孔。

“是先生!”种建中兴奋的说道,“连吕与叔和苏季明也来了!”

“还不快下马!”已经返身落地的韩冈提醒道。

两人连忙带着从人避让到路边,迎面而来的一行人就在他们前面不远的地方停下来了。几辆车中,还有马匹上,许多年龄各异的士人纷纷下来,看着韩冈、种师道两人,眼中都带着欣喜之色。

中间一辆马车的车帘这时一动,一个瘦削苍老的身影走了出来,韩冈和种建中一见,就一起在路边大礼拜倒:“学生拜见先生!”

