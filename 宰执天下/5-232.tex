\section{第27章 舒心放意行所愿(中)}

张璪放开了手中的笔管,揉了揉发胀发涩的双眼。

已经是后半夜。从寝殿内殿转到寝殿外殿后,玻璃灯罩里的蜡烛都换过了一茬,现在又差不多快烧到了底。家住得远的朝官,这时候多半已经起床了。

不过张璪觉得这一夜,满朝文武应该没几人能安然入睡,肯定都在考虑今天朝会上会是谁出来接受臣子们的参拜。只是能猜到结果的,想来应该不会有几人。

“内翰,已经写好了?”宋用臣见张璪停了笔,走过来问道。

“好了。”张璪点点头,将小桌案上的草稿递给了宋用臣。

这第七份诏书的草稿交出去后,张璪就松了口气。挺直了腰背,放松了一下筋骨。一夜写了七份诏书,还作废了几份,张璪只觉得今天将一年的心力都耗尽了。跪坐得久了,两脚也变得麻木,还不知道待会儿怎么站起来。

“圣人,王相公的太子太傅制书已经写好了。”宋用臣拿着草稿呈给了皇后。天子已经睡了,给王安石加赠太子太傅的制书,就需要让皇后来评判。

一夜之间,东宫三师全都被封了出去,换在任何时候,都能引起朝堂上的一场大风浪。而且还是分别给了新旧两党的党魁和赤帜。换在一天之前,不管是谁来说,张璪是绝对都不会相信的。

向皇后拿着草稿看了起来,张璪的心也提了起来。

今夜他所撰写的诏书,都不是普通的诏书。

如果仅仅是给普通朝官加官的制书,三五十个字就能打发,一个时辰写十份都没关系。但册封东宫三师也好,册立太子也好,还有皇子出阁任命资善堂侍讲,哪一篇不要绞尽脑汁?

更不用说这些诏书必然会成为朝堂上关注的焦点。在私人文集中,日后也要收录进来。加之关系到一众重臣,典故用得错上一点,可不仅仅是丢脸的问题。

但张璪的心中除了一点点紧张之外,就只剩下欣喜和兴奋。

对于一名翰林学士来说,这做梦也求不来的好事。

能在帝统传承的时候,站在了最有利的位置。在请立太子的时候,比宰相、执政还要要早上一步。这一点,肯定能被皇后以及未来的天子记上一辈子。日后在两府中立足,肯定是不在话下。

没见皇后和宰执们都在等着他草诏,根本都没招第二名翰林学士进宫来?张璪兴奋的想着。这就是信任啊!

皇后已经匆匆看完了草稿,将之交给宋用臣,让他转递给蔡确审核。

宰执们还在寝殿中,但二大王已经被劝去休息了。说是劝,其实更近于押送,甚至还特别派了蓝元震领人看守,以防他自杀。不过只要等到二大王出宫,回到府中后再去自杀,这间寝殿中也没人会在意了。

除了自己以外,殿中唯一不是宰执的重臣,就是站在一边,根本不说话的韩冈。宰执们与皇后商议大小事宜,他一句也不掺合。闭着眼睛,似乎跟天子一样,睡着了一般。

蔡确匆匆看完了王安石的制书草稿,又递回给宋用臣。

“参政觉得如何?”

蔡确回复道:“回殿下。不需要修改,可以直接书诏了。”

宋用臣遂又拿着草稿转回到张璪这边。向皇后却道,“还是先给内翰一条热手巾擦擦脸。”

茶早就赐了,但张璪怕内急没敢喝,不过热手巾就没问题。

张璪连忙想起身道谢。不过站起来时,两脚一阵发麻,吃不上力,软软得差点就此摔倒。还好被身后的两名内侍给扶住,这才站稳了脚。

拿着热腾腾的手巾擦了擦脸,张璪精神也为之一振。只能跪坐的小桌案,也换成了配着杌扎的几案。坐下来后,他立刻就动手将之誊抄。

今夜他写的七份诏书。任命东宫三师,就是三份。韩冈的资善堂侍讲,则又是一份。此外还有皇太子的册书及天子圣躬违和,由太子监国、皇后权同听政这两份。

剩下的,便是招司马光入京的诏书。这份诏书,并没有收回。王珪之前还特意请示了皇后,不过皇后转回去请示天子,赵顼则回了一个‘发’字。

“殿下。”张璪开始最后的誊抄,王珪这时候又站出来向皇后请示,“如今虽已承天子之意,定下了太子监国,殿下垂帘。但一众朝臣不知,其中或有不便。臣请先行将圣谕告知群臣,不知可否?”

王珪陪着小心的问着。

所有人都看得出来,王珪在心虚。而且他还眷恋权势,不甘就此退场。所以前面做事就没了分寸。甚至为了表示忠心,而拿着郑伯克段的典故来作比。现在又想同时示好皇后和还不知情的满朝文武——早一步通知,就能避免有人做错事。

只是张璪觉得,就算王珪现在这么卖力,皇后也不一定会饶他。

不在其位,不谋其政——这是圣人的教诲。但在其位,却不谋其政呢?

当年蔡襄在仁宗立储时,是唯一没有上书请立英宗为皇储的重臣。所以当这件事爆出来后,三司使立刻就没得做了。英宗甚至一看到他请辞的奏章,就立刻签书批准。照规矩,重臣非罪请辞,天子要先驳回加以挽留。韩琦为此劝英宗,但英宗却说,万一蔡襄不走怎么办?

可惜了蔡君谟,本来以他的资望和能力,其实有望晋身两府。但他出外之后,没两年,就在服母丧的时候病死了。

而王珪今夜犯的错,其性质比蔡襄更为严重。蔡襄当时不过一个翰林学士,有他没他都一样,而王珪可是唯一的宰相,他的沉默,差点就将皇后和太子逼入了深渊。

皇后并没有立刻回复王珪,却向其他执政咨询,“蔡参政,吕枢密,韩参政,章枢副,薛枢副,不知你们如何看?”

吕公著道:“正当如此。”

而蔡确、韩缜、章敦和薛向也纷纷表示同意,没人愿意得罪那么多朝臣。

“韩学士,你看呢?”

韩冈睁开了眼睛,似乎是醒过来了,但他说的话跟没说一样:“此事当由殿下和相公们做主。”

“那就这么办吧。”皇后点了点头,算是同意了。

吕公著瞥了最下首的韩冈一眼,眼神冰冷。

尽管半夜过来,得到的只是支离破碎的细节,但吕公著已经将上半夜发生的一切,连蒙带猜的给拼凑了出来。

通过七份诏书,可以得知天子已经有意退让,召回司马光,任命自己及司马光为东宫师保,这两件事便是明证。

虽然不知道具体的情况,在天子退让后,王珪很明显犯了错。但真正的逆转,还是来自于韩冈的一句话。

韩冈提起千里之外的两座药王祠,在天子和皇后眼中,当是为了让皇子能顺利即位,可在太后及雍王看来却是要雍王的命。

如此一来,天子的心意便完全逆转。甚至不惜召回所有宰执,当着太后的面确定让皇后来垂帘。

一言兴国,一言丧邦。

纵横家的本事或许还不如他。

吕公著望向韩冈的眼神再冷,也比不上他的心冷:‘此子可畏啊。’

七份诏书敲定,群臣便齐齐告辞离开寝殿,他们还要稍事休息,也要让皇后休息。

皇后心身疲惫回到了内殿,赵佣已经被带回去睡觉了,但赵顼却睁开了眼睛。向皇后忙走到床榻边,嗔怪的问道:“官家怎么不多歇一会儿?”

赵顼眨了一下眼。

‘是不需要吗?’向皇后立刻拿起韵书:“官家是不是还有什么吩咐?”

朝……堂……

“朝堂上怎么样?”向皇后屏气凝神,等着赵顼的吩咐。

一……切……如……旧……

“一切如旧?”向皇后点了点头,镇之以静,这是应有之理,“妾身明白了。”

但她立刻眼睛又瞪圆了起来,赵顼说不要改变朝堂上的人事,但还有一个人,她绝对是无法原谅的,“那王珪呢?!”向皇后双眉倒竖的厉声问道。

照……旧……

“还让他做相公?!”

向皇后完全不能理解。不说王珪这名宰相已经在东府里坐得太久,至少他也应该为今天的错误受到惩罚才对。

在立储的问题上,王珪犯了无可挽回的大错。向皇后不会忘记王珪的沉默,给她带来多大的恐慌。就像是被沉入了水底,只有冰冷和无尽的黑暗。

王珪的性格,圆滑、软弱、毫无担当,三旨相公的名号,向皇后在宫中都听说过。这也是官家一直用他的原因。相形之下,更有威望的王安石,才干过人的吕惠卿,武功赫赫的王韶,都没能在两府中待得太久。

向皇后无法原谅。

赵顼费力的眨着眼,长达六个字的一句话,让他很吃力:

使功不如使过。

……………………

蔡京已经到了宣德门前。他现在的袖袋里有两份奏章。左袖中的一份是请皇太后垂帘,右边一份则是为孙思邈请封。

天子重病,臣子们求医问药是理所当然的,宰执们还要轮班去大相国寺祈福。不过为慈济医灵显圣守道妙应真君再次请封,就是蔡京的投机了。看看究竟是皇太后出来,还是天子出来。

至于万一二大王出来,该有的贺表蔡京却没有写,有些事不能做的太急。不过宣德门前的有几人看起来很是心浮气躁,在传播着雍王昨夜没有出宫的消息,不知是不是连雍王登基贺表都写好了。

‘文德殿上的究竟会是谁呢?’蔡京默默的想着。
