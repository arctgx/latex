\section{第13章 晨奎错落天日近(八)}

向太后醒过来的时候,眼前黑漆漆的一片。

喉咙中仿佛有火在烧,头也沉沉的,身子没有半点气力,肚子却饿得厉害。

她记得她早上起来了,也记得自己去上朝,然后记忆就有些混乱了。

好像有人来,又有人走,有些闹腾。

“什么时候了?”

她的声音低得如同在呻吟。

“太后?!”

立刻好几个人抢到床前。有几个声音激动,甚至还带着哭腔,只是没人敢哭出声来。

“什么时候了?”

“已经四更了。”好几个声音同时回答。

“……吾是怎么了?”

“几位太医都过来把过脉。说是感了风寒,这段时间,又太过劳累了。”

“……当真?”

床前立刻跪倒一片,一群人指天誓日,“奴婢怎敢欺瞒太后?几位太医都这么说,韩参政也这么说。”

“哦。”向太后算是安心了,想想,又问,“官家呢?”

“太后放心,国婆婆陪着官家在西厢睡着。”

一个脚步声出了门去,很快就回来。

就听见杨戬在床帘外回报,“禀太后,官家还在睡。”

“是吗,那就好。”

向太后放心下来。

身边的侍女扶着向太后坐起来,

“太后,秦和安来了,要把下脉。”

民间传说宫中的太医能悬丝诊脉,以免亵渎后妃,不过那也只是传说,正常谁能

向太后躺着,只露出一截手腕,让当值的御医三根手指搭上来。

“脉象好了一点,不过还要再吃两天药。”

医官的诊断之后,是写字时拂动纸张的声音。

“太后。”杨戬小声问着太后,“要通知宿直的相公过来拜见吗?”

“今夜谁宿直?”

“有王平章,韩参政和郭枢密。”

“……算了。”向太后想了一下,“吩咐王中正过来,让种谔守好宫禁。与韩参政、王平章他们说一下吾已大安,请他们明早再来。”

门帘掀动,几人匆匆而出。

“太后还有什么吩咐?”

“之前还有谁来过?”

随侍在太后身边的女官一个个数着名字,向太后垂下眼帘听着,只是在听到朱太妃这三个字时,才动了一下眼睛。

王中正奉旨而来,拜见了太后。待太后喝了药,又睡过去,方才退了出来。走出门时,长长的松了一口气。

太后猝然发病,如同一块如山巨石落到了海里,掀起的波浪撼动了整个宫城和朝堂。

王中正自己也提心吊胆,自事发后便盯着宰辅们的一举一动。

不过韩冈和王安石说了些什么,依然不知道。他只知道到了两刻钟之前,王安石所在屋舍的灯都没有熄掉。

王中正吩咐着跟在身后的养子,“二哥,你去圣瑞宫,把人撤回来。”

“孩儿知道了……父亲,太后大好了?!”

“嗯。”王中正点点头,停了一下,又叫住了准备离开的养子,“等一下。”

“父亲还有何吩咐?”

“顺便让梁从政来见我。”

“孩儿明白……那蓝从熙呢?”

“他哪得能回头。”

虽然向太后没有说出口,可王中正也知道宫里面该注意谁。主导宫变的那一位在失败之后,已经没有了任何复起的可能。真正对病中的太后有威胁的,是住在圣瑞宫中的人。

也幸好太后的病情不重,否则王中正表面上虽不会说,心里可就要做些准备了。

当然,对象可不一定会是朱太妃。

…………………………

李信彻夜守在宣德门城楼上。

三千余神机军士,有一个指挥守在皇宫正门。

八门火炮在城楼上虎视城中,而门洞经过改造的耳室中,随时能用虎蹲炮发射出致命的铅弹。

而李信的十几名亲兵,则都背着一杆沉重的新火器,可随身携带,就像火炮一般发射铅弹。需要的时候在枪管口插上锋利的枪尖,直接当成长枪来使用,所以称为火枪。

虽然火枪比起神臂弓要沉重得多,可威力也大了许多。火枪发射出来的铅弹,可以力毙奔牛,打中人,基本上就该去找棺材了。

可惜的是,现如今火枪还不能大量制造,除了还在火器局中做实验的十几支,剩下的都给了李信的亲兵。

仅仅是带着弹力的簧片已经够麻烦了,而枪管则更加让人作难。

李信从韩冈那边听来的消息,火枪的设计,是与火炮一起出台的。可是火炮制作起来更简单一点,早早的就造出来了,而火枪则就复杂许多,想要制造出一根尺寸合度的枪管,就要占去一名工匠半个多月的时间。

标准化,度量衡,图纸,在火枪造出来之后,韩冈曾经就火枪的事说了很多。李信没怎么听懂,不过亲眼见证过军器监成立后手中兵器质量的飞升,他多多少少能理解韩冈的意思。在不能大规模生产尺度完全合乎标准的火枪前,这样的武器,是不能够出现在战场上,只能成为妆点。

不过不论是能上战场的火炮,还是不能上战场的火枪,今夜应该不会有需要它们上场的时候。

李信想着。

比起之前人各异心的宫中帅臣,现在统领宫禁兵马的帅臣和将领,都是对太后忠心耿耿,绝不会附逆。

唯一需要担心的就是太后万一有所不豫,那该怎么办?

今夜,自己的表弟宿卫宫中。李信更是打叠起精神,以防出意外。

一旦有变,该听谁的话,他可不会弄错。

之前李信就与王厚约定了信号,一旦宫中有变,立刻就率人将韩冈从福宁宫拉出来,

王厚已经派了人去福宁殿处守着,

已经四更天了,城东的方向上已经可以看到星星点点的灯火,那里是鬼市子的位置。到了五更天,鬼市子就会变得灯火通明,买卖衣服图画花环之类,至晓方散。

李信的视线在城中扫视了一圈之后,又回到了宫中来。

宫中的几处殿宇灯火最多,比得上城内的市口,而后苑中则是一片黑暗。

李信先看了福宁宫的偏殿,再转向慈寿宫,最后又瞄着圣瑞宫好一阵。

这一个晚上,他来来回回的盯着的就是这三个地方,若要出事,事端只会发生在那里。

看了一阵,李信的神情陡然一变,飞速的拿起了千里镜。

在千里镜中,可以看见一点星火正从慈寿宫中出来,转去了圣瑞宫的方向。

现在可是四更天!怎么也不该这时候去天子生母安居的宫室。

只是点着灯,却又有几分正大光明的感觉。

李信将千里镜紧紧压在眼睛上,看着那只灯笼在圣瑞宫的侧门口停下,过了片刻,才随着另一盏从圣瑞宫中出来的灯笼,一起往慈寿宫过去。

李信的眉头皱了起来,叫了人过来,让他去找王中正。

人刚走不久,从慈寿宫中,又出来好几点灯火。分别向禁中统军将帅的驻留地赶过去,其中有一盏灯笼,还正向着宣德门过来。

片刻之后,李信见到了童贯。

“太后大安。”童贯说道。

…………………………

王安石肯定是没有睡好。

韩冈可以确定。

应当是为了河北的事,这一点,韩冈基本上也可以确定。

自家岳父入住的房舍,一个晚上没有熄灯。

不过韩冈也没睡好。

夜里他和衣而眠,一直都没睡着,直到四更天的时候,得到了太后已经退烧的消息,方才安心的睡下去。早上再过去请安,太后已经醒了。

说了几句话,吩咐了朝事,宰辅们退出来时,就全都安心了。

听太医们的诊断,太后的病情已经好转,不日将会康复。

对此,韩冈是长舒了一口气。

太后的安危决定了朝堂是否能够安稳,韩冈无论如何都不想看到好不容易才安定下来的朝堂再生波澜。

几次宫中变乱实在是耗尽了他的心力,太后垂帘的体制,不知能持续到何时,总是让人不能安下心来。

其实昨天白天的时候,韩冈在一闪念间,甚至有一劳永逸的想法。不过真要去按那个想法去做,的确会有人支持,而且还不少,不过也不是那么容易。而且成功了之后,对他自己来说也不一定有好处,还不如先看着下去。

将心中的悖逆思想藏了起来,韩冈迎上了苏颂。

“玉昆,你昨晚也没睡?”

韩冈知道自己眼底都是血丝,看起来的确是有些憔悴的样子。而苏颂几乎一样,眼底同样都是血丝。

“昨晚睡是睡了,不过没睡好。在宫里面提心吊胆的……一点动静都要醒过来。”

苏颂摇摇头,“在宫外也差不多。”

两府之中,只有郭逵今天看起来心无挂碍,回去就安心睡了。其他宰辅,都是一连疲惫,从神态上看起来跟王安石和韩冈都没两样。

苏颂叹了一口气,“太后这一回也只是风寒而已,便弄得人心惶惶……”

“杯弓蛇影啊。”韩冈道,这两年,总是有事发生,当然人心不定,“都是惊弓之鸟,有几个能够什么都不在意的?”

“这样下去可不好。”

“总比习惯了要好。”韩冈笑道。

“也是呢。”苏颂也笑道,然后又叹起气来,“不过太后一病,北面的事能多耽搁几日了。”

“的确是耽搁了,不过不是在这件事上。”

这一事,韩冈并不打算瞒着苏颂。
