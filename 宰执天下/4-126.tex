\section{第20章 冥冥鬼神有也无(二)}

章惇坐在书桌前。

饱蘸了浓墨的笔拿在手上,却迟迟没法儿下笔。眼睛是在看着桌上的白纸,但焦点却不知落在何处。

墨水在重力的作用下渐渐汇聚在笔尖上,章惇不知这样呆坐了多久,凝聚在笔尖上的墨滴终于落了下来,啪的一下砸在微黄的纸面上,瞬息间就晕了开来。

好端端的一张纸给落下的墨水污了,如果是作为稿纸还是可以继续使用,但回过神来的章惇,将笔往砚台上一架,就将写坏了的纸张团起来往地上一丢,就仰靠在椅背上。腰背弓着,完全没了寻常时候的锐气。

一道水帘正挂在敞开的窗前,水流越来越细,渐渐的变成了一滴一滴的下落。

一场大雨刚刚过去,从屋檐上淌下来的雨水砸在檐下的青石板上,又沿着暗沟向着州衙后院的池塘流淌过去。如果在京中,此时已经是秋色降临,但在广西桂州,依然是夏日的气候,午后时不时一场滂沱大雨,让空气潮湿得让人很不舒服。

不过困扰着章惇的并不是桂州让人难过的气候,也不是眼下的战局。

虽然他紧锁着眉头,放在眼前的奏章草稿的稿纸已经被废弃了一张又一张,一团团的被丢在桌面和脚下,但章惇并没有失败,安南经略招讨司也没有失败,他不是在写请罪的折子。

尽管被鼓动起来的三十六峒蛮部在深入交趾境内的时候,有几个部族被交趾军漂亮的打了几个埋伏,吃了不小的亏,但其他部族的成功也将整体上的损失给找补了回来。

而且经过了几个月的扫荡,交趾边境往内七八十里的地方,都已是渺无人烟,成了彻头彻尾的鬼地。一群群溪洞蛮人如同蝗虫一般将他们经过的地方清洗得一干二净,只要是能拿回来的就一起搬回来,搬不动的则一把火烧掉。用竹子树木打造起来的房屋,在熊熊火焰中只剩下灰烬,成千上万人在这里生活过的证据,仅余作为地基的黄土。

而被拯救回来的汉人则有八千五百之多,除了一部分新近从邕州被掳走的,剩下的都已经被安排去了钦州、廉州屯垦。照着惯例,由官府分给他们土地,并借出种子、农具、耕牛和房屋,以尽快恢复两州的元气。

虽其中有许多并不是真正的汉人,只是会说官话;也有许多是汉人,只是他们是主动迁移到交趾;更有一些虽是被掳走,但靠着自己的双手已经脱离了奴隶的身份,在交趾境内娶妻生子——这也是章惇和韩冈不敢将他们尽数安排在邕州的缘故——不过章惇并不在乎他们怎么想,因为被迫做牛做马的汉儿为数更多,而且这个由官府发起的行动,也在南蛮地区确定了汉人要比蛮人高人一等的地位。

从他和韩冈拟定的方略上讲,这几个月一系列的战事,完全是一个辉煌的胜利,付出的代价可以说是微乎其微。

但这个胜利带来的结果,比起失败,还要让章惇心头憋着一口闷气:因为他赢了,皇宋也赢了。

不对!

章惇慢慢的晃了晃沉重的头颅,他并没有赢,大宋也没有赢——是‘不战而胜’,更是一个让人笑不出来的笑话。

——交趾上表请降!

就在五天前,交趾来进献降表的使臣乘船抵达了钦州。钦州的百姓没有报仇雪恨,让章惇很遗憾;他们没有选择从陆路过来,让章惇更是遗憾万分。

章惇真心想拦着交趾派来的使节,最好能剁碎了埋进他后花园的几株芭蕉下面。但他做不到一手遮天,交趾人只要泛舟海上,便能从广东上岸,照样能将降表送到东京城去。

他只能先行下令,让钦州将这一行使节留下来招待,自己则派人上京去禀报,看看朝堂会有什么样的反应。

交趾的这一手做得很漂亮,绕过安南招讨司直接去找宋国的皇帝。

因为他们知道,眼下能阻止官军灭国复仇的力量,并不在交趾国内,而在皇宋的朝堂之上。

章惇此时还不清楚丰州的异变,更不清楚已经划归他麾下的一万八千余名西军精兵,如今只能有五千到位。

但他很清楚,北方的情况很复杂,三国纷争的时候,任何一方的动向都会引起局面的瞬间改换。辽国态度的暧昧不明,使得他和韩冈必须想方设法的从天子和两府诸公的手里,把他们所需要的兵力给挤出来。

韩冈上京后,的确按照预定的计划做到了这一切,有两万精锐禁军在手,章惇相信自己身边的盟友会越打越多。

交趾自立国之后,就一直摆出小中华的姿态,将四邻视为必须降伏的藩属。不论是南面的占城和真腊,还是北面的诸多蛮部,都备受欺凌,不得不向升龙府进贡。

如今只要朝廷表现出足够的强硬,以灭国为目的,这些曾经被欺凌过的部族、国家必定会蜂拥而来,争先恐后的一效犬马之劳。

可若是朝廷犹豫不定,反复无常。畏于交趾积威,可以成为盟军的国家和部族,就会陷入犹豫和观望之中,甚至未免后患,而反投交趾,共抗官军。

章惇再明白不过,眼下他必须尽快有一个大捷来回报给天子,让天子认为自己能够给他带去一个破国灭族的荣光,可以自豪的去太庙朝见列祖列宗,而不需要感到任何愧疚;并且可以让王安石压制住朝堂上的一切反动。

也就是说,他必须尽快出战,而不能等着全师抵达。

但到底要不要打,能不能打,这就是章惇这些天来一直都没能打定主意的原因。

凭着先期抵达的五千兵马,加上荆南军和刚刚训练出来的新兵,到底能不能胜过严阵以待的交趾大军,章惇心中并没有底。进攻和防守是两回事,休整过的军队和邕州城下的疲兵也是两回事。同一支军队,在不同情况下表现出来的战斗力可以是天差地远。

但这个决心必须要下,如果自己都犹豫不定,更别想说服天子和朝堂,文字上只要稍有破绽,就会造成不可挽回的结果。

窗前的水帘已经不见了,残留的雨水要很长时间才会滴上一滴下来。厚厚的云层也散去了,午后的阳光从窗外西侧的屋檐下照过来,挂在瓦当下的滴滴雨露,闪着七彩的光芒。

章惇曾听韩冈说过,阳光本是七色,只是混在一起才成为白光。后来在许多地方也试验过,三棱形的水晶镜,让无数人亲眼见证了彩虹的成因。

如果韩冈在这里会怎么说。

章惇为之一笑,想都不用想的——他肯定会说要打!

韩冈是自己的下属。有这样的下属,作为上司,压力就会很大。不过章惇不会去嫉妒韩冈所立下的累累功绩,因为所有的功劳他都能占上一份。可总是占着韩冈的便宜,也许有人乐得享受,但章惇不会甘愿。

士气可鼓不可泄,若是半途而废,日后官府在广西的蛮部之中再没有威信可言,而交趾这个嘬尔小国就会更加张狂。从章惇个人的角度来说,晋身西府的大门就在眼前,不推开来走进去,他如何会甘心!

不过在这之前,必须要将自己的意见更进一步向天子分说明白,而不是任由反对者来捣乱他安南经略招讨司的事务。这是安南经略招讨使兼安南行营兵马都总管的工作,不能交给他人。

主意已定,也不再取出新的稿纸起草文章,章惇拿过禀于天子的专用奏折,振笔疾书。一列列整齐的小楷出现在纸面上,在奏折中将自己的心意表明。

……………………

丰州的局势影响着天下大局,但并不是说,鄜延路这里就能干脆了当的忽略不顾。

种谔不想做旁观者,成为衬托郭逵的绿叶;也不会让横山对面的党项人安安稳稳的守在他们的寨堡城池之中。

王舜臣终于得到了领军出战的机会。他的将旗还高高挂在罗兀城上,但他本人已经领着一千多精锐向着葭芦川的方向潜行过去。

这是一次尽量隐秘的行动,偷偷摸摸的样子就算让王舜臣自己来看,都是为了不惊动近在咫尺的西夏军,好去支援的河东路的战局,才做出来的姿态。

但每一个鄜延路的将领都知道,为了防止鄜延路为河东煽风点火,党项人不知派了多少对眼睛潜藏在山林中。而西夏安排在左厢神勇军司的两万多驻军,早就整装待发,随时准备截断通往葭芦川的道路。

看着道路两侧愈发的显得千丘万壑的地势,王舜臣知道离着他的目标越来越近了。

“该来了。”种朴突然就在旁边说着。

王舜臣沉沉的点了点头,离河东越来越近,也越来越接近葭芦川,对于西夏军来说,眼下这一段的地势是最好出击的地点,等过了葭芦川就来不及了。而从消息传递的速度看,神勇军司的兵马也只能来得及赶到这一段路上。

他麾下的将士们也都知道这一点,一个个走路时都在警惕万分的用目光搜寻着两侧的山林。

‘这样下去,再不来,可就没力气打仗了。’王舜臣的心中有些焦急。

就在这时,前方的一片树林中突的有大批的飞鸟惊起。

王舜臣一紧手上的战弓,掩在满脸的络腮胡子下面的嘴角向上勾勒出了一个笑容,“终于来了!”

