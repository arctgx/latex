\section{第19章 萧萧马鸣乱真伪(五)}

契丹骑兵出现在丰州的消息,由于斥候传回来的次数太多,并没能隐瞒起来。尽管下面的士兵还懵然不知,但将校们都已经听说了。

对于辽国、西夏联手对抗大宋的猜测,有人说真,有人说假,争论得厉害。一直到了出兵之后,私底下还是议论纷纷。

“管他真假,先打了再说!输也罢、赢也罢,总得先试上一试。也不能还没上阵腿就先软了。”折可适亲自牵着马,在山路上跋涉着,“我们几十年都没与契丹人打上一仗了,谁知道契丹骑兵到底是真货假货?”

跟在他身边的亲将摇头晃脑,“能不打最好不打,等打到兴庆府,回头再打契丹人。一起上来可吃不消。”

折可适反手往亲将的头盔上一敲,当的一声脆响,“头上有盔,身上有甲,弓弩刀枪全都配齐了,就差牙齿没安个套子。就算是卒伍身上的装备,契丹大将看着都要眼红,你还怕个鸟!”

“饭不是要一口口吃吗?就是去窑子里消遣,也是得轮番来的,要是一起上,一夜下来俺李铁脚也会两脚发软。”说是双脚发软,但看这李铁脚扬眉带笑的模样,也是浑没将契丹人的威胁放在心上。

折可适狠狠的瞪了亲将一眼,转头也笑了起来。

弓弩刀枪不缺,头盔甲胄俱全,让折可适并不会太过担心遇上辽人会有什么麻烦。只是三五百人,根本就不用在意,就算辽军大举来援,只要想退回来,谅那契丹骑兵也阻止不了。

这就是兵甲俱足的好处。

麟府军一向可以算是折家的私兵,装备一直以来都要比驻扎在太原、代州的上位禁军要差,更别说跟龙卫、神卫、天武、捧日上四军相比。

只是因为前一次兵败丰州,如今又要跟着郭逵出阵。河东经略司与枢密院打了一阵嘴皮官司,自家的父亲、叔父也都上书求天子一个恩典,终于让东西两府的几个相公大方了一次。

竟然一口气发下了两千套板甲下来,连神臂弓、斩马刀这样的神兵利器也全都配齐,一下将折家军最核心的五个指挥给装备到了牙齿。而他们替换下来的有些老旧的皮甲、札甲,又分派给另外的十几个指挥中的精锐装备,让这几千人的战力同样上了一个台阶。

折可适所在的前军之中,有一半是带甲的精锐指挥。而配有板甲的指挥,也有一个,正由折可适来率领。此时士兵们的甲胄都由骡马驮着,一步步走着上坡路。

骑兵前后奔驰在行军的队列中,来回传递着侦查来的消息。在林中穿行的小队,清理着潜伏的敌军哨探,这两天下来已经颇有斩获。前出十几里的游骑,则传回前方并没有发现敌军大队人马的消息。这让折可适松上一口气的同时,也免不了会感到有些失望。

李铁脚紧紧跟着折可适,他在山路上走得四平八稳的模样,到不枉了铁脚的诨名:“听说这两千套板甲不过京城军器监三五日的产量,想那军器监一年十五六万套精铁板甲,只要两个月,就能让我们麟府军全身上下都是铁甲。”

“别指望再多了,西军、京营、河北、河东,哪家不等着将铁甲配齐全了?再轮到麟府,还不是要到多少年后。”折可适抬手指了指李铁脚身上,“还是将你身上的这套板甲保养好了才是真的,省得过了两年锈了没得用。”

裙甲、护胫、护臂等零碎的配件没装上,不过李铁脚还是将前胸后背的两片护甲给挂上了,他的脚力惊人,不在乎行军时身上多了十几二十斤。偌大的拳头嘭嘭的在胸口敲了两下,“都可以当镜子用了。”羊尾油将板甲的前后构件擦得油光水亮,的确是能当镜子使用了。

底层的军官身上的板甲式样与士兵们相同,但都带着红褐铜色,胸前、背后的甲板还装饰了波浪形的花纹,头盔上的红缨也比士兵高上五寸,这身份一下就起来了。

到了指挥使一级以后,则是加配了护臂护腿,内衬更多了一层挡箭的丝绸。至于将军一级之后,配件则更多了,不仅有札甲的甲叶做活动部件,与板甲配合起来的甲胄,也有形似河虾的外壳,一截截的用钉子铆起来的甲胄。

而折可适在太原府,还见过更极端的板甲。将从头到脚全都包在铁皮之中,就连双手都用铁手套给包起来,足足有百斤之重,人进去就别想再走路。

据说这本来是准备给郭逵的,折可适两个月前去太原领京中下发的军器时,正好就撞上了。一套甲胄,上上下下不知怎么弄得金光闪闪。看着倒是漂亮,如果天上有太阳,隔着三五里都能认出来。唯一一点不好,就不能穿着上阵,郭逵看了之后直接了当的给推掉了。军器监的这一套在韩冈离开后新开发出来的产品,并没有能推销出去——郭逵倒是喜欢收藏兵器,但甲胄和重弩,就算他已是国之干城,世所公认的当朝排行第一的大将,都不敢干犯律法多留几套在家中。

“四郎是跟发明板甲、飞船的小韩学士见过吧?现在人人都说说他才是真真的文曲星。”李铁脚凑上前问道。

折可适只在宣抚司中见过两面,并没有深交,甚至仅仅行礼问候了两句而已,但当时他得官不过一年,就能顶着韩绛的威风,任谁看了都知道日后绝不是个简单的人物。

“韩龙学是不是文曲星,我肉眼凡胎看不出来。不过他是当真有本事。在他之前,军器监一年才多少套甲胄?如今又是多少套。现在京城只恨铁少,让造甲的工匠都变清闲。”

正说着话,迎面来风骤然紧了。折可适抬起头,却发现不知不觉之间他们已经到了山口。左右望望,还能看见一道半丈髙的土垄,在峰谷间蜿蜒。古长城的遗迹经过了千年依然留存,只要越过去后,就是到了丰州。

党项人没有在山口屯兵防守,如果他们当真打算死守着山口,一众宋军将帅的心中可是会乐开了花。

踏上山口,前面依然是望不到头的山林,但只要跨出一步,就是抵达了丰州。这一片土地原本也属于府州,只是在旧丰州被攻破之后,朝廷割了府州的土地重建了丰州。

折可适望着前方的一片旧时属于自家的土地,党项甚至契丹的大军都在前方等候着他们的到来,一股豪情壮志在胸中涌起。

“下去吧!”折可适冲着麾下的将士放声大吼,“我等即为先锋,便要将第一桩捷报当先报与天子!”

………………………

赵顼眉眼间满是喜色。

广西昨日上报,解救的汉人数目已经超过六千,不知道其中多少是货真价实的大宋子民,但只要有一半是真的,也是可喜可贺的一桩事了。

因为韩冈没有将溪峒蛮部对交趾边境部族的斩首算成是功绩,在广西经略司的军报中,也没有清楚的汇报这些天来的具体战果,只是从解救出来的汉儿数目来推断,交趾军的损失,当是数以万计。交趾小邦,这么大的人员损失,等于是在身上割了条口子放血,很快就会支撑不住了。

广西的好消息不断,自然就是韩冈的功劳。

韩冈今天上殿辞行,赵顼便是没口子的夸奖。看到王安石也在,跟着又道:“今闻韩卿喜得麟儿,相公也添了两个外孙,实是可喜可贺。韩卿劳苦功高,朕岂能吝于爵赏……”

云娘和周南就在昨天前后脚的临盆,给韩冈添两个儿子,心情也是正好。不过听到赵顼要给自己刚出生的儿子荫官,则是连忙推辞:“陛下厚赏,臣感激涕零。但臣此前以微薄之功,已有了两子得陛下厚恩。如今新功未立,如何能再觍颜邀赏,臣不敢受!”

王安石虽然刚刚丢了一个儿子,但一下又多了两个外孙——尽管是名义上的外孙——心情也好了许多。只是朝廷的规矩他还要维持,“陛下,韩冈所言极是。且尚在襁褓便宠以荫赠,也有折福之虞。”

“也罢。”赵顼也不坚持,“就等韩卿在交州建功立业,朕再一并赏来。”

韩冈本来就有了两个儿子,这一下又是两个,加上听说还在怀孕的正妻,也就是王安石家的女儿,说不定又是一个。赵顼兴奋过后,心中就有些郁闷,自己的儿子也不少,就是养不大。

一封奏报这时被匆匆送进了崇政殿,只有战报才有这个待遇。宰相既然在殿上,也不需瞒着,赵顼直接接了来看。脸上的笑容一下就收敛了起来,“丰州有辽军?!”

“辽军?”王安石和韩冈同样的变了颜色。

赵顼脸色惨白,让宋用臣将奏报递下去给王安石:“说是打着西夏的旗号,但实际上是契丹的骑兵,穿着、装束还有乘用的战马,都是契丹一系。”

