\section{第11章 安得良策援南土(四)}

由旦至暮,崇政殿中宰辅们依然给赵顼一个满意的结论。

因为收复罗兀城,夺取横山,虽是种谔的一力主张,但受到了新党的全力支持。王安石、吕惠卿都对此投了赞成票。到了这个时候,半点也不能退让。

吕惠卿始终坚持着他的观点,已经得到的罗兀城不可放弃,而失去的丰州更要将之收复:“丰州城余粮不多,西贼盘踞城中,是坐吃山空,来年开春必然粮尽。而同样的道理,河东、河北对面的辽军,也不可能一直驻扎在边境上。等到明年正月,辽主就要起身北上,将捺钵移往鸭子河,设头鱼宴镇服女真诸部。到时候,就可以一举收复丰州。”

“难道要放着邕州到开春才去援救?”冯京厉声质问。

“只要调动荆南兵将,就能稳住广西局势,击退贼军。”吕惠卿说道:“等到明年开春,再发遣天兵,吊民伐罪,便可一举平灭交趾。”

“江东盗贼蜂起,江西也难安稳,潭州守军如何能轻动?”

这就是为什么到现在也没有得出结论的原因。天南地北的困局,现在成了一个死结。

想要解决交趾,就必须调派大军。想要将调派大军,必须缓和了北方局势。这样才能从陕西、河东、河北腾出手来。否则有辽国、西夏虎视眈眈,赵顼怎么也不可能从北方调兵离开?京营禁军,也是要随时防着契丹铁骑南下,同样不可轻动。可是要缓和北方的局势,就必须在鄜延路上做出退让。

偏偏这个退让,王安石、吕惠卿都加以反对,同样是赵顼最不愿意点头的——就算他点头,也不能保证契丹、党项两家会放弃得寸进尺。一旦退让,就等于承认了国中的虚怯,两头野兽要不乘机咬上来,大宋也不会一百多年一直受到困扰——至于从潭州调军南下,赵顼也不放心江南的局势。所以他头疼得很厉害,到现在也无法下决断。

所以眼下的情况,现在依然是在僵持着——包括前线和朝堂。

另外还有推荐南征主帅一事——这是自上午的廷议之后,第二桩没有被确定下来的决议。交趾既然敢兵犯中国,当然得兴兵反击,没人敢说将这口气忍下来,待其自退。

尽管可用的兵源还未决定,是缓是急也同样没有争出个结果,但并不妨碍现在商议一下主帅的人选。

天下这么大,但真正能让赵顼放心得下、让他们统领大军远征灭国的也就那么几人。这几位究竟是谁,如今的宰辅们也都一清二楚,一掌之数而已。

吴充推荐郭逵领军。不过郭逵现在还在太原府,因西京道的辽人以及丰州之事难以脱身。所以吴充同时推荐到吕大防,郭逵在河东的职位,则由新近去了华州的吕大防来接任,这样郭逵就能腾得出手来。

但郭逵是首屈一指的宿将,是军中柱石,调他离开太原,要冒着很大的风险。吴充的私心赵顼看得很清楚,在狄青之后,哪一个武将都别想、同时也不敢去想枢密使这个位置。而换作是同样有资格领军的王韶、蔡挺,他们的功成之后,必然会晋身枢密使,取代他的职位。

只是王韶至今也没有表明态度,看他模样,不是有心去广西的样子。至于蔡挺,他本人的意愿不提——‘谁念玉关人老’的那一句,赵顼还记在心上——他的年纪也大了一些,已是年过花甲。六十多岁的老臣,赵顼怎么敢让他领军南下?若有个万一,失了主帅的大军如何上阵?

除了郭逵、王韶之外,赵顼心目中的另外一个人选——赵禼,现在正在环庆路经略使任上。相对于郭逵,将他给调出来可能性还更高一点。

王安石则是一力主张调动荆南兵马,虽然没有明言,可从他的心意上看,多半是要支持章惇。

荆南的兵马姑且不论,难道当真让章惇领军……

赵顼并不喜欢这个人选,章惇的经验太少了,平定荆蛮的功劳不足为凭。且不调荆南兵马,让章惇领军就毫无意义。如果要用章惇,就必须一并调动荆南兵马。这个决断赵顼要能下早就下了。

相对于主帅的人选一时难定,副手和幕僚就很好办了。

赵顼最为看重的将领燕达,最近刚刚调入京中,接了种谔的龙神卫四厢都指挥使的职司。他为副将,身份是足够了。就算主帅有个万一,他也能将一军的军心安定下来。

另外军中掌管转运的幕僚,人选也不用多想,论起经验、能力,以及过往功绩,除了韩冈不作第二人想,而且南方瘴疠之地,精通军中医疗的韩冈是肯定要随军南下。

冯京第一个出来举荐韩冈,“无论在陕西宣抚司中,还是在熙河经略司,韩冈主持随军转运并军中医疗等事,都备受军中赞许。南征大军,缺其人不得。”

而王珪也一并支持这个推荐,“有韩冈在,军中皆可安定。荆南之役,若无医疗随军,哪能如此顺利?”

接着就是吴充,就在殿上推举荐了韩冈任广西转运使:“等到大军南下,可以顺理成章的负责随军转运并军中医疗等事。”

赵顼对这个提议当然不会有任何反对。尽管冯京、吴充、王珪他们三人推荐韩冈是各有私心,但人选是无可挑剔的,连王安石、吕惠卿说不出反对的意见。唯一让人担心的就是军器监,是否在韩冈离开之后,还能保持如今的效率,以及接连不断给人的惊喜。至于韩冈本人的意愿,相信以他的性格,不至于会不答应。

只是主帅、军队……赵顼闭着眼睛喃喃自语,忍着剧烈的头疼,却怎么都想不出一个让他一切顺心的方案。

“官家……官家……”

仿佛从很远的地方传来的呼唤,让赵顼猛然间惊醒过来。睁大眼睛,就看见李舜举的脸满是关切的出现在面前,说着:“官家,该歇息了。”

赵顼茫茫然的看着周围,只见内侍和宫女,宰辅们一个不见:“人呢?”

李舜举一脸的疑惑:“官家要找何人?”

赵顼定了定神,终于发现现在自己是在寝宫之中。“什么时候回的福宁宫?”他问着李舜举。

“官家散朝后就回来了,已经有一个多时辰了。”李舜举就是看着赵顼一个多时辰都在发呆,觉得有些不对劲了,方才才会出言惊醒赵顼,“官家,是不是有哪里不适?”

“没有。”赵顼闷声的说道。他恍恍惚惚之间,思路都成了一团浆糊,连何时散朝的事都记不清楚了。眉头皱了一阵,抬起眼:“现在什么时辰了?”

“回官家的话,刚敲了二更三刻的点。”李舜举眼中尽是忧色,劝说着赵顼,“官家,还是早点歇息吧。熬夜伤了御体,可非国家之福啊。太皇太后和太后知道了都会担心的。皇后虽是在守制,但也是常常问着官家的身子。”

赵顼没理会,道:“去,速去传韩冈和章惇入宫。”

“官家!”李舜举惊道。

“将宰执们一并请来!另外还有燕达!”赵顼按了按额头,脑仁越发的疼了,“这事不解决,怎么能歇息得了?”

……………………

“吴充、冯京、王珪都是一个心思,要将韩冈请出京城去。”夜色已深,吕惠卿却没有多少睡意,就在烛台下,与弟弟吕升卿议论着崇政殿中的决议,“韩玉昆在京城中开罪的人可是不少,倒省了我做小人了。”

“可南征的战事不可能拖延太久,也就一年的功夫而已。”

“所以是广西转运副使。”吕惠卿笑道,“吴充倒是会抬举人。如果是任随军转运使一职,打完仗就能回来。但广西转运副使不同,有了功劳后,就能顺理成章的管起广西转运司。就算结束了南征之役,也少不了平靖地方的差事。至少三五年内,别想再入京城。”

“广西转运使……这职位可就高了。”

“大军南下,钱粮消耗难以计数。到时候,广西的账目要是能清楚明白就有鬼了。韩冈接手广西转运司,一个账目不明就能治他的罪。吴充想必就是这个心思。”吕惠卿笑道,“既然韩玉昆这般心急的推动章子厚南下,想必他也有心在南方振作一番,说不定吴充的推荐,正合他的心意。”

“但章惇……”吕升卿觉得他的兄长似乎太过在意韩冈,反而忘记了更为接近的危险,提醒道,“章惇可已经是翰林学士了。”

“他去他的枢密院,我在我的政事堂。十年之内,两不相干。”吕惠卿摇摇手,示意弟弟不用担心,“章子厚有了两次领军的经验,日后西、北两边有战事,就可以让他出去应付了,他可没空常驻京中!”

“西、北两边战事?”吕升卿疑惑的问着。

吕惠卿脸色冷了下来:“辽人如今已经彻底的站在了党项一边。天子想要剿灭西夏,就要做好抵御辽军的准备。而天子辛辛苦苦这么些年,为了使什么?他可能放弃吗?”

“辽国会支持西贼?!”吕升卿只听到了这一句,寒毛都竖了起来,心惊胆战,“不至于吧。”

“由微见著!这几年西北连番大战,西夏国力难支,试问辽人怎么可能会没有唇亡齿寒的想法?”吕惠卿沉声说着,“攻灭西夏,收复燕云,天子的心中的确是一步步的来,但对手可不会乖乖的站着挨打。”

吕惠卿正在和弟弟说着话,书房门外则有家人敲门,“参政,门外有天使带官家的口谕来了。”

吕惠卿连忙让人迎进来,就见那名小黄门说道:“官家有旨,召参政即刻入宫。”

吕惠卿没有立刻接旨,“敢问还有谁人?”

“两府的相公,还有燕太尉,张内翰和韩舍人。”

吕惠卿微变,沉吟了一下,道:“天色已晚,宰辅漏夜入宫必惹京人疑惧,请天使回去转告陛下,明日上朝后再说。”

