\section{第26章 鸿信飞报犹觉迟(七)}

“这就是盐场?好大的一片。”黄金满惊讶的瞪大眼睛。一片闪着白光的土地,沿着海岸线向南北延伸开去,至少有十几里,因为他站在盐场的入口,无论向南向北,都看见盐场的尽头。

“当然就是盐场。”韩冈点头说着,“不过这还只是一半而已。在北面还有一片差不多大的草场,原本是提供给煮盐用的草料的。”

黄金满伸出手去,指着充斥在眼前的一片反射着天上阳光的白色土地,手都有点颤抖,“这里有这么多盐,怎么还不发卖?!”

韩冈笑了笑,知道黄金满是误会了。随行在侧的知州马竺也是笑道,“团练有所不知,这一片看着像是盐的白色地面,其实全都是多少年来浸泡了盐水的土地。让牛马这样的牲畜来舔倒没问题,可怎么卖给人吃?”他向南指了指,“产盐的晒盐池,是在前面一点的地方,只占了盐场的一小部分而已。”

韩冈眼下还在交州,甚至有空带着黄金满来盐场参观——这其实也就是他在交州多留了十天的缘故,是交州盐场重开的问题。

盐是生活必需品,没有盐吃,人就会废掉。所以朝廷对于盐业的垄断所带来的利润,占到了财政收入的很大一块。

但要生产食盐,光有盐场是不够的,还需要有足够的人手。

京东、淮东、两浙、福建,乃至广东广西的诸多盐场,哪一座没有几百上千的盐丁。交趾的盐场当然也不例外。

但之前一场灭国之战,交趾沿海几个盐场的盐丁基本上都是各家溪洞蛮部给瓜分了。那时候,安南经略招讨司的心思,皆放在打下升龙府上。章惇和韩冈哪里有多余的精力,去想着煮海造盐的事。

等到交州平靖下来,章惇回京去做他的枢密副使,留在广西的韩冈就有的头疼了。为了重开盐场,他不得不从溪洞诸部手中讨回了一部分已经废掉的盐丁。

交州七十二部没人为韩冈的行为而抱怨。汉人要吃盐,夷人同样也要吃盐,在盐场重开前的半年多的时间中,积存下来的食盐都已经卖光了,交州盐场再不开,日后各部上上下下加起来,男女老少总共上百万,全都得吃淡食去。到时候,连重一点的力气活都不能干了。

对于食盐紧缺的事,留在广源州的黄金满都急了。他的部族过去吃得是交趾贩来的私盐,价钱便宜得很。而眼下用的钦州官盐,价格比过去吃的私盐翻了一番还多,口味还不见得更好。黄金满对此叫苦不迭,可是这钦州官盐运到广源州后的盐价,一点也不会因为他的煊赫身份而降低一文钱。

但他们送回来的盐丁人数远远不及过往,只有几百人而已。韩冈困于人手不足,不得已之下,不得不冒着风险,换了一个制盐的办法。

尽管韩冈对于如何晒盐的手法一窍不通,但知道大略的方向就能试验出来,就像当初制造飞船一样。不过这一次就不需要他来试验,关西最有名的解州盐池出产的池盐全都是晒出来的。

还留在广西的关西人还有几百人,倒有一个队来自于解州,虽然这一队并不是驻扎在盐池边,但其中有一半老家就在盐池附近。这一半人中,又有两人了解解州盐池是如何晒盐。

有韩冈统观全局,有两名专家来指点细节,这一次在交州盐场试行晒盐法,便是没走任何弯路的一举成功。

“海盐当真可以晒出来?”黄金满虽然没有走南闯北过,但他好歹活到了四五十岁,多多少少也知道些常识:“不是说盐全都是用大锅煮出来的吗?末将当初与那些挑着担子到洞里贩盐的私盐贩子讨价还价的时候,他们都是说煮出一斤盐,就要用上多少柴草,千里迢迢送来一斤盐,又要费上多少脚力。这贩来的花销多高多高,这卖给末将的盐价多低多低,自家还有浑家孩儿和八十岁的老母要养,实在是不能再低了,再低就只能全家去喝卤水去了。”

黄金满学着商人卖货的腔调说话,逗得韩冈为之一笑,哪里商人都是一个德性。敢拼敢杀的黄巢同行,做起买卖来,竟然也是脱不了生意人的口吻。

这生意人的口吻姑且不论,当初私盐贩子与黄金满讨价还价时,说煮盐要花用大量的柴草,为此增加了许多成本,这一点却是扎扎实实,并无半点夸大。

“只要少了柴薪之费,制盐的成本至少能减去七成。”马竺为黄金满解释道,“过去邕州的一斤官盐,要卖十四五到二十文,交趾的官盐也要卖到十文,而广源州……”

“八文。不过是私盐,”黄金满想起过去的事就愤愤不已,“交趾人将盐卖到广源,一斤竟敢要价二十五文!”

马竺笑着点点头,指着一块块如同田垄的盐畦:“现在换做了晒盐法,就是官盐以八文一斤来卖,官府赚的钱也绝不会比过去要少。”

望着海滩上的一方方随处可见白色盐霜的盐畦,黄金满欣喜之余,也是咋舌不已。要是官盐以八文一斤来卖,赚的钱都不比卖到十几二十文要少,那眼下一斤盐的成本,是不是就只有一文上下了?

这些盐畦都是用水泥抹过了池底和池壁,正好位于潮水线上。有一道水闸对着大海。潮涨时,将水闸打开,海水涌入池中,再将水闸关闭,畦中的海水就被留了下来,在阳光下逐渐蒸发晒干。

尽管旱季刚刚开始,但池中已经有些地方的卤水被晒干后,出现了白色的盐霜。而从附近的一条小河引来的清水由一条条前后有两道水闸的水渠与一方方盐畦连通。

这座盐场是在转运司名下,并不归交州管,马竺虽是海门知县,但他作为韩冈的前任幕僚,比起知州李丰,在盐场中下的功夫要多得多。多少日子下来,早已是一切门清:“这就跟解州的晒盐一样,等到盐霜析出后,就得用清水冲上一遍,将畦中的苦卤冲走,剩下的就是可供食用的盐巴。”

黄金满望着一方方已经可以见到食盐的卤水池,感慨不已,“末将一辈子多半都是守在广源州,都没见过海。见识是不多,一直都是以为盐只能是煮出来。想不到晒盐竟然如此省时省力。天朝上国的确不是交趾这等蛮夷能比。”

韩冈笑道:“你不知道也不足为奇,中国之中知道晒盐法的本也不多。这晒盐法也就在关西有,其他地方都是煮盐。眼下交州盐场晒盐成功,接下来转运司就会在钦州和廉州推广晒盐法,替换掉原有的煎煮之法。”

对于如今通行于沿海和蜀中的煮盐法,韩冈一直都觉得很是纳闷。这个时代已经有个更为节省人工和成本的晒盐法,为何没有给推广开来。若说是这个时代没有推进技术发展的动力,只是去看看如今的江西广东的几大铜矿,就知道这种说法是污蔑。全都已经用上胆铜法,以铁屑来置换铜了,皆是这几年推广开来的。

韩冈有时候不禁从阴谋论上去推测,是不是晒盐法太过于简单,只要有片大一点的海滩,加上一条干净的淡水河,就能将食盐给大批的制造出来。而煮盐法则是需要大量的草料,需要大量的人工,另外煮盐用得铁锅铁盘也都是官府提供,越大的规模,官府就越容易控制,比起晒盐法更能将盐业控制在手中,也就没有改变过去生产模式的迫切需要。

不过交州是偏远之地,出产的食盐也不会卖到外路去,倒也不需要顾忌太多。甚至钦州和廉州两地的盐场都可以推行晒盐法——广西内陆吃着钦州和廉州盐场所出产的食盐,但临近的路州,则自有其他地方的食盐来供给。

推广晒盐法之后,不再需要配合盐场煮盐的草料,广西一路的数万顷的草场都不用再去种草,从而节省下大批适宜耕种的土地。多了那一片草场,广西粮食的产量又能升上一台阶。

等到第一批食盐出来,韩冈让黄金满带了五十多匹驮马回广源州,运走了他能得到的所有的食盐。剩下的部族则是不得不耐下性子,等着下一批食盐的出产。

在这个过程中,章恂已经告辞离开了交州,他将交州的商号安顿好了之后,剩下的就没有别的大事了。至于与盐有关的事,那就跟商人无关了。

私盐在内陆是禁而不止,但交州是不用担心的,更不会有人从交州贩了盐去北方卖,用海船来贩私盐,这就是个不折不扣的笑话——并不是赚不到钱,而是利润太低,与冒的风险想比,实在是得不偿失。

食盐的事情宣告解决,摆在韩冈面前的已经没有多少事了。他连如今在交州的正在繁荣发展中的海外贸易都不怎么在意。

海外的购买力毕竟是有限,对比起大宋的经济和人口水平,海外诸国加起来都提不上筷子。海贸的规模能养活几千几万的海商,但对整个国家并没有太大的用处。

如果从商人的角度,在海外贸易上能赚到大钱,可以轻易成为一方豪富。但对于国家来说,他们能海贸分润到的钱钞,实在是少得可怜,真正应该着眼的还是国内的市场。

等到糖产业成为交州支柱,韩冈留在交州的一番心血,也就算是没有白费。他安排在此处的顺丰行分号的掌柜,接下来的任务可就重了。

过了两天,韩冈又收到了一封信,不是王安石辞相的消息,而是张载重病不起。

