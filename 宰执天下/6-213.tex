\section{第26章 惶惶寒鸦啄且嚎(上)}

 啪的一声脆响,茶盏在墙上碎成千片,落到地上的碎瓷片,已经看不出官窑出品的精致。

刚刚把心爱的茶具给砸得粉碎,龚原公牛一样喘着粗气,眼睛都赤红一片,妻妾就在壁脚看着,却不敢过来劝。

“看什么,还不过来收拾!”

龚原横眉竖眼的冲着妻妾吼了两句,铁青着脸,跨出房门,大声喝:“来人。”

贴身伴当陪着小心的蹭过来,龚原瞪了他一眼,“去准备车马。”

“是。”伴当不敢多问,应声后匆匆离去。

如今马车也便宜,过去的低品朝官,莫说马车,连马都买不起。现在挽马的价格便宜了许多,马车也便宜了。一辆车配上两匹驽马,只要供养的亲戚不多,每月俸禄能达到十贯的官员,想要配的话,都能配得起车马。只是在京师中能有一套屋舍,能放得下马厩和马车,比买马买车都难。

除非是住到城外,否则如龚原这个等级的官员,能有一套前后两进的屋子就不错了。哪里有地方放得下马车?就连马都养不了。

幸而官宦人家聚居的里坊,外面都会有很多赶车人、养马人等着人来雇车马。想要马车,不过是让下人多走几步路。

在等下人去雇马车的时候,龚原回去飞快的换了一身衣服,然后就在院中来回踱着步子。

木底靴踏着院中的石板地,哒哒的又重又响,恨不得将石板跺碎的样儿。一听到外面的巷子中有了声音,他便立刻向外走。

伴当慌慌忙忙的进门,差点就跟龚原撞上。

“怎么这么慢。”龚原瞪了一眼,说着就排开伴当出门。

出门下了两级台阶,弯腰进门坐了上去。

伴当连忙跟上,关了车门,一脚踩在车门外的踏脚上,稳稳的站定了。

“怎么还不走?”龚原隔着车窗,冲伴当道。

“这就走。”车把式耳朵尖,听到了,先照空挥了一鞭子,给了一个响儿,又赔话道,“只是还没问大官人要去哪儿。”

龚原声音低了一点,只说给伴当听,“敦义坊。”

伴当应了一声,抬头对车夫道,“去敦义坊。”

“是章枢密府上?”

“就是那儿!”龚原没好气。

伴当又高声传话,“就是章枢密府上。”

老道的车把式见多识广,哪个不知道眉高眼低。见龚原一副晚娘脸,气急败坏的样子,并不多问。一声吆喝,就赶了车上路。至于多少车钱,回来还是否要车,待会儿自跟伴当去算。

龚原压了一肚子的火,上车后还是感觉着心里烧得慌。

前些天,太后受了政事堂的唆使,诏命开封府满城去抓乞丐,皇城司的狗到处嗅,引了军巡铺的巡卒一家家的搜,闹得京中鸡飞狗跳。

打着追缉人犯的名义,冲进人家的不胜枚举。几天前,在东城开铺子的亲戚的儿子,跑到龚原这边哭诉了一番,说是本厢的巡卒冲进他家里绕了一圈,然后抢了一堆家当走,金银器皿好几套,连现钱都拿走了百多贯,还把亲戚本人给抓走了。

龚原听得火冒三丈,先是找台谏中的老朋友,回来后连夜写了奏章,上表给太后控诉,然后又写了信告到了开封府。

上表没有结果,他已经不在御史台,而是回到了国子监——这还是靠了金陵那边在章惇面前说了话,否则就出外了——普通朝臣的奏章,想要递到太后的案头上,必须要经过政事堂,想也知道,肯定是给那位权臣拦下来了。别说是龚原本人,就是御史台的三两封弹章,也给太后压下来了。

这本是在龚原的预料之内,如今太后根本就不理会台谏的奏章,对权臣偏听偏信。但台谏中有人上表,这声势就起来了——尽管上表弹劾的御史比他预计中的要少许多。

但开封府那边的反应就让他不能容忍了。

新任知府的韩忠彦直接将状子给了亲戚所在的厢中都巡检,然后那边到了今天,就给龚原写了个帖子。解释说,抓人是因为其与丐贼勾结,为丐贼销赃,而被拿走的东西,也是作为与丐贼勾结,为其销赃的罪证而被扣押的。现在查明其与丐贼并无勾连,只是误收赃物。除赃物之外,所有扣押证物将全数返回。

刚刚从狱中被放出来的亲戚只回去洗了个澡,换了身衣服,就上门来道谢了,还带了一堆礼物。

在龚原面前,亲戚是千恩万谢,第一是免了官司,第二是挽回了大部分损失,这已经是天大的喜庆了,寻常人进了开封府狱,不脱层皮,怎么可能安安生生的出来?更何况他实际上也的确贪图那些白天乞丐、夜里窃贼的丐贼所带来的好处——那些赃物实在是太便宜了。

但龚原不满意。他问了亲戚,东西是还回来,可并不是全部,细算起来只有七成多。

面子还能打折?当时龚原就火冒三丈。

要是他还在御史台中,别说在要还的东西中克扣,就是他家亲戚当日拦着门放声亮个名号,巡卒都能吓得爬着走,当事的巡检也得跑过来赔不是。

等亲戚走了,龚原就再耐不住心头火,当即就决定,到章惇府上好好说上一说。

韩冈如今越发的独断独行,仗着太后的宠信恣意妄为,视两府同列如庙中泥胎。

这一回对乞丐下手,明面是上是为了云南的屯田,尽可能的发遣人过去,但另一方面,也是进一步控制了京师的兵马。等到他当真达成目的,章惇还能在枢密院中安居?

一路上,龚原在心里组织着对章惇的说辞,怎么去说服这位位高权重的枢密使。

到了敦义坊,章府所在的那条街,依然是车水马龙,人满为患。

龚原就在巷口下了车,车把式跳下来,弓腰问道:“官人,可要小人等你出来?”

“不要等了,出来不知是什么时候了。”

龚原摇摇头,他要与章惇商议要事,回去也会有章府的车。让伴当与车把式会钞,便朝章府大门过去。

章府今日守门的两个司阍是龚原所熟识的,看到他,龚原便把脸上怒色稍收,让伴当上前去,“跟他说,转告枢密,史馆修撰龚原有要事求见。”

龚原如此做派,门前的其他人纷纷侧目。

门状不递,门房不守,站在门口就等着章府开门来迎。

这架势,莫不是章惇家的亲戚,还是因为有些身份门第?

认识龚原的官员,人群中也有,名号传开,立刻就有人上来行礼问候。

有人过来问号,龚原心中的焦躁渐渐缓和了一些,一边与人寒暄,一边等着两个司阍进去通传。

但两个司阍却都没动身,龚原的伴当已经又重复了一遍,但一人在门前冷眼看着,另一人迎了另一位官员进了门房。

转眼之间,本还在跟龚原寒暄的官员一个接一个的散开了,方才迟了一步上来的官员,就在一旁冷笑。

区区一个同管勾国子监公事、史馆修撰,怎么可能到了枢密使府上就能直接进去?

龚原心中的火头又蹭蹭的上来了,走上前,对其中一位司阍道:“余富!还不快去通报枢密,说龚原有要事相商,莫要耽搁了大事。”

那余富却只后退一步,向龚原行了一礼,卑笑道:“龚官人容禀。龚官人小人自是认识,但府中自有规矩,除枢密先行吩咐,或事前约定,他人想要拜谒枢密,须得出具名帖,待府内通传。还请龚官人让贵仆给小人名帖,免得小人难做。是官人来时仓促,一时未具名帖,门房里也备有空名帖和笔墨,官人可以进去写了交给小人。”

龚原差点把牙齿咬碎,他过去登门造访,无论带不带名帖,章惇都不会将他拒之门外。今天是走得仓促,没带名帖,但就么进门房,他的脸面往哪里摆?

他忍下气,寒声道,“吾向与枢密熟识,你去禀报了枢密便知。”

“小人知道官人与枢密熟识,也知道官人前些年常来府上,可小人是行伍出身,从荆南时起,就一直跟着枢密,只知将命不可违。枢密定下来的规矩,小人岂敢不遵?眼下小人让官人动怒,转头枢密定会打小人一顿板子给官人出气。但违了枢密之令,依军法处置,小人受得处置会比板子更重。还请官人体恤小人的辛苦。”

龚原的眼神彻底冷了下来,这个司阍完全是在针对自己。他咬着牙,“你倒是好说嘴。”

余富做了好些年的章府司阍,当然认识龚原。

最早的时候,龚原是王安石留给章惇的门人,章惇也才曾经打算重用他。可惜的是,龚原选错了路,已经不是府中主人的亲信,不过是个叛逆。这样的人,余富怎么不敢得罪?

“若小人拿了名帖却不肯通传,那是小人的错。但若是连名帖都没有,就想进枢密家的大门,可就是管勾的错了。难道去其他相公的府上,管勾也是这般无礼?”

龚原盯了他几眼,不再多话,转头拂袖而去。

这么多官员和官员家的下人都在看着,他的脸面可谓是丢得一干二净。

往巷口走,还听到有人议论。

“好个伶牙俐齿,难怪让他做司阍。”

“说得也没错,凭什么我家的老爷要递门状,这龚官人就能不用?我家老爷的官位还高一点。”

“把自己看太高了,枢密府上,连个名帖都不准备,当自己是翰林吗?”

穿过人群,走到巷口,龚原恼羞成怒,脸上红得发烫。

“编修。”

“怎么了?”

“这里停的都是他人的车子,小人要先去外面雇,请编修等一下。”

龚原一听,便欲发作,但最后他却是无力一挥手,“你去吧。”
