\section{第七章 烟霞随步正登览(一)}

“玉昆,听说你正与郑国公家议亲?”

这一日,韩冈正好休沐。难得一日清闲,他在家中用了一天的时间审核了新一期《自然》的稿件,没有去考虑朝堂政事。可到了晚间,王旁却跑了过来劈头便问。

王旁问得鲁莽,韩冈却并不以为意,摇摇头:“八字都还没合,还早得很。”

韩冈既然这么说,也就是有了。

不过这一桩婚事,并非是韩冈主动联系富弼。他对子女的终身大事,并不是那么的心急。都还不到十岁,现在定下来也不一定能够保证能够最后执行,就像当年王韶做媒定下的那门亲事一般。之前与苏子元和王厚定下的婚约,都是形势使然,并非是刻意而为。

联姻也要看孩子们的秉性,想读书的找个诗书传家的岳家。偏好军事的,就找个普通门第,好方便领军。而文武两端都不出众,只能谨守门户的儿子,就找个高门显宦家的女儿,这样也不至于在兄弟中吃亏。不可能那么早议定。

王旁坐了下来,“正巧跟玉昆你议亲,郑国公的运气也算是好了。”

“巧合而已,当初可没想到会有这一番风波。”

“所以说是运气。”

富弼的女婿是冯京,冯京的女婿是蔡渭,说起来蔡确也能跟富弼攀上亲。

如果走正常的司法程序,当然不至于连枝带叶,将富弼也一并带进来。可惜这样的案子,从来都是政治决定一切。现如今是新党当政,若要烧火,自是要往洛阳那边烧过去。这么好的机会,不将旧党连根断了。

韩冈之前与富弼议亲的时候,当然不可能预测到会有这一次的叛乱。只是在洛阳诸多元老中,唯有富弼,是韩冈比较重视和尊敬的。而富弼家的家风,在诸多元老之中,也是比较受到称赞的。

诗书传家的大族一般都是出色的联姻对象。进士频出的南丰曾家,晋江吕家,或是范文正范仲淹家,在议婚时,往往比宰相家更受青睐。而在宰相门第内,相对于富家,灵寿、安阳二韩,介休文家,就差了许多。更不该用说与韩冈算是同乡的吕家。

就比如东莱吕。吕夷简与庆历党人的恩怨由于时日已久,可以不论。但陈世儒弑母案在前,吕家的外孙女,竟然将丈夫生母给害死,不论有多少理由,也是辩解不了的。而吕嘉问,作为吕公著和吕公弼的侄孙,却背叛家族,做了家贼,亦可见吕家主支和支脉的关系有多恶劣。连族中子女都教育不好,吕公著、吕公弼两家的门风可见并不如他们表面那么堂堂正正。

丢下吕家的事,韩冈问道:“你家大哥定下来没有?”

王旁摇摇头:“还早呢,才多大?”

“说得也是。要议亲,关键还是要看人品和性格。孩子若年纪太小,一切都看不出来,还是等大一点的好。”

娶妻在贤,娶错了浑家,一辈子就毁了。不说别人,沈括那一家就是最好的反面教师。

“可惜玉昆你家没第二个女儿了。否则以玉昆你家的门风,我家的大哥肯定是要找玉昆你家的女儿。”

“寒门素户,哪来的门风?穿堂风倒有。”

“不是有二姐在管着?自然不用担心。”

王旁过来算是探听消息,坐了一阵便告辞离开。

待王旁离开,素心进来书房,问韩冈:“官人,可是要睡了?”

韩冈摇摇头,议亲的事可以先放一放,比较重要的还是最近在眼前的推举,“还要等一下。何矩差不多也该来了。”

……………………

“下个赛季的会首终于是定了,是博陵侯。”

何矩赶来韩府时已经两更天了,但一通禀,就立刻被引去了书房,韩冈还在那里等着他的消息。

韩冈亲手递过一盏茶去,“都选两天了,可是够辛苦的。”

顺丰行在京师的大掌柜千恩万谢的接过茶杯,陪着笑道:“还是国公的主意好,要不然就是二十天也决不出来。”

韩冈摇了摇头,这么称呼还太早了一点。他还没答应做齐国公,诏书依然在宫中和韩府之间往还。

在韩冈的本心中,一个莱国公就够了。但他立下这么大的功劳,朝廷不厚加酬赏,情理上也说不过去。反正齐国公也罢,莱国公也罢,全都是虚的,做一做也无所谓。

韩冈抱着这样的态度,便和坚辞不就时有所区别,很容易区分开。明白了韩冈的想法,这一两天,称呼他国公的人就开始多了起来。不过再过几日,估计也就不会有人再这么称呼他了——论起尊贵,朝中无如宰执,就是亲王之尊,见到宰相也是要先行礼的。

“我还以为最后会是阳泉侯呢,没想到会是章懿皇后家。”

何矩叹了一声,低声道:“太后正垂帘,谁还敢选向家的人?”

虽说是赵家人做天子,但毕竟是太宗的血裔,太祖皇帝的子孙来做会首,没人会担心。但阳泉侯向绍峰,可是向太后的叔伯兄弟,谁敢让他沾手会中事务财务?万一他起了贪心,会中可没人能压得住他。

一开始的确很有一批人想奉承他,阳泉侯得到的票数也最高。但随着一轮轮选下去,候选人一个接一个淘汰,越来越多的选票集中在他的对手博陵侯身上,最后便是李家胜出。

何矩还是叹着气:“要还是淮阴侯来做,谁都没有话说。”

抬眼看见韩冈脸上似笑非笑的表情,何矩就摸着脸上的一块乌青,叹道,“小人明白,小人明白,谁让他是宗室呢。”

正因为齐逆叛乱,朝廷上下去看宗室,谁都像叛逆,闹得现在所有宗室都不得不夹起尾巴。

从赛马联赛创办就开始做会首的华阴侯赵世将,便在选举的五天前宣布放弃参加下一赛季的会首选举。

如果仅仅是赛马赌球,这样的宗室一向最得朝廷欣赏。可赵世将在做会首的期间,将自己的收入拿出去资助了大批的宗室,连带着影响到了许多参与到两大联赛的宗亲们都一并出钱襄助族人。

在这其中其中有很多是因为王安石的宗室法,而失去了太庙留名资格的赵姓子弟。每个月都能多拿上一份钱,尽管有些远支宗室甚至只有一两贯收入,但救急之德,让赵世将在很多宗室心目中都有着很高的地位。只是诱人子弟参赌,让他在士林中的名声很糟。

光有宗室的支持,没有士大夫的称许,赵世将这么做也不算很犯忌讳。而且如果要阻止他资助宗室,那么朝廷就必须拿出真金白银来作为补偿。所以尽管这件事经常有人提起,但时至今日,还是赵世将

所以赵世将的退出,使得原本没有任何悬念的此次总社大选,一下就变得混乱起来。一下子就有六人打算参加选举。

一直以来都是赵世将高票当选赛马总社的会首,去年甚至是全票。但这一回的选举,六名候选者最高的一位也只有三成的支持率。

一开始的几次投票,各方的支持率或有变动,但最高一人的得票率依然没有达到五成。而随着选举的不断热化,各方也动了真火,虽说还没到后世议会里打作一团的情况,但已经开始丢茶杯、丢瓜果,何矩现在脸上的一块乌青,就是被误伤的。

眼瞅着这样下去绝对会引来外面的虎狼,许多中立派便说动了何矩,来向韩冈讨要主意。他们也相信,朝堂上正要开始推举宰辅的现在,韩冈不会坐视他提议的赛马总社会首选举变成笑话。

韩冈如他们所愿,想了个招数出来,而他给出的办法,在后世极为常见。

如果在选举中,没有一位选举人能够得到半数以上的选票,那么得票数最少的选举人就必须退出,然后进行下一轮选举。由此一轮轮的下去,直到有人得到半数以上的选票而当选。

其实韩冈给出的意见,之前在争议时,并不是没有人提出来,只是他们缺乏足够的权威来推行自己的意见。直到韩冈发言之后,赛马联赛总约的第四修正案才高票通过——在获得三分之二选票之后,才能够进行进行补充修正的联赛总约,有百分之八十的选票同意修改,绝对是绰绰有余了。同时这也是在马券收益分配和升降级制度之外,第一条有关会首选举的修正。

“也算是了了最后一桩心事。”何矩笑说道,“可以安心回去了。”

“总行那边会安排好的,不用担心什么。”

何矩在京师的时间已经很长了,再留两年,开封这里就成了他的自留地,这是韩冈和冯从义都不想看到的。

何矩点头:“小人明白。”

又说了几句,何矩便起身告辞。

等待多时,韩冈终于安心了。

适逢其会的赛马总社会首选举了,京城中多少人都瞪大眼睛看着。一开始虽有波折,最终却还是顺利的结束,想必很多人的想法会有些变化了。

更鼓声随风而来,听着响起的声音,已经过了三更子正,算是第二天了。

洛阳那边的消息差不多就该到京城了。韩冈对洛阳元老们的印象并不好,但对他们的政治智慧——确切点说,是政客智慧——从不怀疑。韩冈相信他们会寻找符合自己利益的道路。

距离宰辅推举,还剩下两天。

