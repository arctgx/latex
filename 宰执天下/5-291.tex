\section{第30章 随阳雁飞各西东(22)}

临到年节,国子监早就放了假。大门都关了,但南门附近的酒肆茶楼中,还是能看到许多衣着襕衫的士子。

明年就是科举之年,数以千计的贡生此时都从天下各州云集京城之中,国子监附近便是他们聚会最多的场所。

正午时分,黄裳也与几名士子坐在一间酒店的包厢中,虽然不是正店,也算是干净气派了。

“勉仲。”说话的士子跟黄裳有着同样的口音,他提着酒壶给黄裳倒酒,“西北之事,朝廷到底是个什么章程?”

“说要打又不像,韩学士还跟辽使谈着买卖,一群商人都要上路了。可要说不打,看西北的模样,却是明明白白的要打上一场。”

礼部试考的自然主要是经义,但策问一项,还是免不了要与时政挂钩。看眼下的形势,很有可能是有关辽宋之间的题目。如果不能把握好朝廷对辽的基本态度,那么落榜是没话说的。若是把握好了,那么熙宁三年叶祖洽叶状元的运气,说不定也能落在自己的头上。

黄裳是韩冈的门客,也算是气学门生,甚至他本人就住在韩冈府中,此事国子监中尽人皆知。韩冈现如今就是一个没名分的西府执政,在军事战略上,朝廷都要参考他的意见。

既然如此,黄裳当然就成了应考的士子们打探的对象。

黄裳也不知该说什么好,他甚至后悔今天应邀出来。

现如今,各色流言传得到处都是。有事实,但更多的就是毫无根据的谣传。

最上面的重臣,有能力有条件去伪存正,分辨流言的真伪。可底层的官僚,只能在浩如烟海的流言蜚语中区寻找真相了。

除了西面已经跟辽人对上阵以外这个事实被确认以外,其他便是众说纷纭。

想不想打,能不能打,该不该打,会不会打。朝廷的态度并不明确,韩冈也没有一个准话。既然朝廷本身就没有一个明确的说法,那么黄裳又怎么可能弄得清楚?

但身边的都是乡里,若是明着推搪敷衍,那可就是得罪狠了,日后少了乡党为助,怎么做官?

“学士平日里可不会多说朝廷的政事。”黄裳笑得深沉,举杯一饮而尽,将心虚掩住,“不过就小弟来看,朝廷可是已经做好了一切准备。和、战与否看的是北面,但无论北面是什么选择,朝廷都有应对之策。学士的性子强硬,更不会退让。有些事小弟不能说,但很快就该传出来了,各位可以拭目以待。”

黄裳故弄玄虚,不过他的话倒是有几分真实。

外面闹得不成样了,向皇后每天早晚听了皇城司的回报,信心也是越来越少。毕竟从西北传来的消息上看,边境上的小小争端已经快要往战争方向上发展了。

吕惠卿的宣抚司,在皇后眼中,已经成了战争的策源地。种谔的胡作非为,是吕惠卿在背后撑腰。

章敦和薛向无法为吕惠卿去辩解,而东府中的韩绛、蔡确,还有刚刚抵京的曾布,更是在推波助澜,加深向皇后的成见。

就是吕惠卿本人,他自己都不便自辩。一边是策划战争,一边是无法控制下属,两样必须要选择一样,放在谁身上都要头疼万分。

韩冈都在为吕惠卿感到为难,实在是不好选。

但为了让皇后安心,他和西府必须要拿出些实际的东西,否则就只能让东府继续兴风作浪。

幸好军器监有了一件好东西。

虽然是属于中书门下管辖,东府并没有报上来,但西府的耳目却早就伸了过去,毕竟在京百司中,军器监跟西府的关系最为密切。

是个很简单的机器,不是兵器,而是滑轮组的应用。

“是上弦机。”

滑轮组的用处,韩冈早就在私人的笔记中加以阐述,也很早就用在了港口中。军器监中,也有很多工匠想要利用滑轮组省力的原理,来改造弓弩。

第一目标,不是单人使用的弓或弩,而是床子弩,但这些工匠得到的却是连续失败。现在出现的,只是单纯的上弦机。

由一名栗姓的工匠献上的这具上弦机。利用绞盘和滑轮为主体。将神臂弓架在上弦机上,能省一半以上的力气,速度也更快,而且快得多。

若是改成驴子拉磨式,驴子或牛拉着绞盘转上小半圈,神臂弓的弓弦就给拉上去了。眨几下眼,就能上好一张弩。

一架机器,可以让一个十人的什,维持着正常发射的速度;让一个五人的伍做到急速射。如果专供一个人的话,就是一人成军。

虽说在韩冈看来,这完全是守城时才能用,野战时难道能把驴子和磨一起拖上阵?不过他明面上没多说什么,成与不成,奇思妙想都是件好事,任何一桩经过实战验证的武器背后,都有一堆被废弃的设计。只是私下里,跟章敦、薛向议论了一句。

不过其他人并不认同韩冈的看法,有了上弦机后,守城的确是方便多了。而且人的气力是有极限的,两只胳膊有千斤之力的猛将寥寥可数,他们也做不到一天拉上一千次的神臂弓。但上弦机只要有钱,想造多少都可以,若是不断加以改进,甚至一次能给几架重弩上弦。

“原本纺车只有一个纱锭,现在可是有十六个纱锭。”章敦笑着对韩冈道,“这不是玉昆你过去曾经说过的吗?如切如磋,如琢如磨,不仅是君子之道,也是器用之道——这也是玉昆你说的吧?”

所以就有了年节前两日的试射殿廷。

向皇后,两府宰执,以及韩冈,都在武英殿前看着这一架新奇的机器。

在一头驴子和一架上弦机和一名十二三岁的小黄门的辅助下,石得一拿着三张神臂弓交替射击,一刻钟的时间,就射出了整整两百发弩矢,将六十步外的十副铁甲射得千疮百孔。

“只要一个都,就能守住一面城墙。”实际上是不可能的,章敦纯粹是忽悠,但确有几分道理。

使用冷兵器和热兵器最大的区别就是对体力的消耗。弓刀之类就不用说,单是四石的神臂弓,想要拉开就不是随随便便就能做到的——按照将兵法的规定,要成为一名上位禁军,不至于一个月只能拿五百文钱的俸禄,必须要能拉开神臂弓,推行将兵法时的这一刀让整整三分之一的上位禁军被降等。

而要在战时保持射速,就更难了。

若是急速射,不要三五次就会腰酸腿软,直接脱力。即便是有时间回气的慢射,一天下来,也不可能超过百次,否则就会伤到筋骨,医治若不及时,严重时甚至能留下一辈子的暗伤。

不过现在上弦机的出现,却拆掉了原先横在大多数人面前的门槛。

“以骡马机器代劳,莫说被降了等第的下位禁军或是厢军,就连老弱都可以拿着神臂弓站上城头。只要他们能拿得动神臂弓,扣得动牙发。拿着同样的神臂弓,从老弱妇孺手中射出的箭矢,自与跟锐卒一般无二,一县之内,兵员何啻多了十万之众!老弱妇孺皆可上城杀敌,只要军械充足,又有何城可破?!”

这应该就是火枪的好处,现在放在有了上弦机的神臂弓上,却也一样说得通。

向皇后立刻就被说服了大半,剩下的一点不放心让她望向韩冈:“学士曾经做过判军器监,板甲便是学士之功。军器监一年千万斤钢铁,同样是学士之能。不知学士意下如何?上弦机是否可用?”

韩冈并不是反对畜力的上弦机,他只是认为此物不适合野战,肯定比不上火器。但如今要让皇后安心,说服同侪,只要证明在契丹铁骑来袭时,官军——主要是河北禁军——能将城池守住就够了。

何况章敦今天下了不少功夫,看看拿来做靶子的十副铁甲就知道了——全都是库存的旧货,只是外面给擦亮了,换成是现如今的板甲,不到四十步哪里可能射得穿?!——只看这一点,韩冈也不可能给他和军器监拆台。

“枢密之言,正在情理之中。就是城中户口不过三五百的下县,守城时城中若多三五十具上弦机,就等于多了上千精兵。”韩冈停了一下,又提议道,“臣请调选监中良工,在神臂弓局下,附设一上弦机局,由专人负责制造。若有品质低劣,可追查至一人,而两局的工匠们熟悉之后,互相配合,也可以不断加以改进……这两年新造的神臂弓、斩马刀和板甲,都比一开始时改进了很多地方,也强了很多。”

判过军器监的韩冈在这方面就是权威,他的肯定和补充推了最后一把。能拾遗补缺,甚至是提议设专局

打造,韩冈等于是在为上弦机背书,而不是单纯的附和了。

向皇后完全放下心来:“既然如此,就让军器监设局专一打造上弦机,越快越好。尽快送往河北各路,对于肯用心的匠人,财帛爵禄,朝廷不吝赏赐。至于发明了上弦机的栗忠……”她看了看东府的宰相和参政,“要重赏!”

曾布抢出班道:“可依神臂弓例,降一等给赏。”

向皇后点点头:“可。”

‘是不是该将热兵器弄出来了……’韩冈想了想,又否定了,‘还是明年吧。’先在《自然》上阐述理论,至于制造,让其他人去费心好了——军器监中的有心人想必多会订上一份。而且有了上弦机,火枪想超过神臂弓就更难了几分,倒是火炮还好说些。

敲定了上弦机,向皇后对河北的局势也稍稍安心了下来,不再寝食难安,而且这个决定很快传出去,同样让人心浮荡的东京城安定了下来。

有了神臂弓、斩马刀、板甲、飞船、轨道等一系列的成就,东京城的百万军民对军器监的新发明有着极大的信心,这一点是十几年来潜移默化造成的,很少有人能察觉得到,但却是实实在在的存在。

就在元丰三年的最后一天,鞭炮声响彻九州大地,东到大海,西到大漠,南至交州,北至云中,都是喜气洋洋迎接着新的一年的到来。

而沿着灵州川艰难跋涉了数百里的种谔……

他,看见了灵州。

