\section{第12章 锋芒早现意已彰(二)}

韩冈准备组建医学的计划书就放在桌上。

纸页的边缘卷起,明显的已经被翻看过很多遍了。

章敦站在窗前,望着无光的夜空,紧紧皱起的双眉,显是心烦意燥。

背后的章恂方才一番话说得口干舌燥,却也不知道他的七哥听进去多少。

舔了一下发干的嘴唇,章恂道:“哥哥,交州那边……”

“好了,我已经都知道了。”章敦沉声,打断了堂弟的重复,“黄金满他们想要更多的奴工,随他们去,但不得将朝廷卷进来。收买奴工也好,亲自去掳掠也好,朝廷都不会管,也不会插手。朝廷多少还得要点脸,你明白?”

依靠章敦的地位,还有以交州为主的贸易收入,章敦、章恂所在的这一房,在莆田章这个大家族中,已经超越了曾经做过宰相的章得象那一房,二者相辅相成,以章敦在政治上有些洁癖的性子,财富上的作用更大一点。但章敦可不想交州的事,动摇到自己的地位。

章恂是章敦在交州的代理人,掌握着福建路中数得着的大商号,可在章敦面前,连一句话都不敢分辨,低下头,“小弟明白。”

章敦转回身来,“而且也别以为我不知道,黄金满他们的手都跨海伸到三佛齐去了,现在想要朝廷跟占城、真腊打一仗,不过是嫌南洋奴工的价码贵而已。”他盯着章恂,“十一,你老实说,是不是上京之前,答应了他们什么?”

“哥哥莫误会,小弟也只是答应黄金满转述的他们请求,朝廷会怎么处置,绝不敢妄言一句。”

“你明白就好。”

“小弟这一次还带回了一些南面的药材,有几味正好能给九叔补一补。”

章敦绷紧的脸色稍稍缓和了一点,点了点头。章俞年纪大了,身子骨也越发的不行,去年还能夜夜笙歌,今年在家休息的时候多了起来,入秋后,好几个月没有出去了。

见章敦的心情转好,章恂笑着凑近了点,“小弟听援哥说,太医局那边一向给三叔诊治的医官现在不在京城,剩下的几位医术都算不上好。”

“熊日严前日去了洛阳。”章敦道,“富彦国快不行了,一个月中杨戬去了三趟洛阳送药,太医局中医术最好的几名医官都派去了。”

“何至于如此,难道东京这边就不需要良医了?”

“也只是一时而已。”章敦摇摇头,富弼没多少日子了,那些御医也不会在洛阳逗留太久,他们过去,只是想体现太后对老臣的优遇罢了。

自从九月,洛阳报称富弼病重,开封这边已经有好几批御医派过去了,都是翰林医官中数得着的好手。这些日子根据从洛阳传回的消息,很可能过不了这个冬天,即便能熬过这个冬天,明年也很难撑过去。

太后想善始善终,给元老重臣最后一个体面。章敦也都乐见于此。党争归党争,不能像牛李党争那样没了底限。再怎么立场相冲,现在苛待老臣,日后同样的待遇未必不会落到自己头上。

“少了富弼,想必韩冈要头疼了。”章恂说道,章敦与韩冈的疏远,他看在眼里,所以最近与顺丰行的日常联系都断了,他笑着:“没有富相公居中转圜,文相公怎可能多看韩参政一眼?”

富弼欣赏韩冈,不论东京、西京都不是什么秘密。

韩冈当日能得到洛阳的支持,一方面是旧党已经看不到希望,另一方面,世传有富弼在其中为其转圜——文彦博与韩冈关系极差,朝中尽人皆知,怎么看都不会是旧党改弦更张的主导者。

“有没有洛阳支持,对他都不会有影响。”章敦走到桌边,低头看着桌上的卷册,“他一向喜欢挟大势压人,一干过了气的老家伙,多他们不多,少他们不少。”

章恂张了张嘴,不知道该说什么好,过了一会,才小声问:“韩冈可又是要做什么了?”

章敦抬了抬眼:“利国利民的好事。”

见章敦心情不好,章恂发觉自己又说错话了。但他从章敦的语气中,也听不出有什么讽刺之意。

心中疑惑,却又不敢问。

“韩玉昆提议设立医学。”章敦揭开了底。

“原来是真的!”章恂脱口而出。

“哦,外面也在传了?”

“是。”章恂无奈点头:“的确有些流言蜚语。”

章敦变得饶有兴致起来,问道:“还有什么说法?”

“除了说医学之外,还有说要整顿武学和武举,另外又有说要在明法科之外,加增明算科、明工科、明医科,另外也有传言说进士科的试题要大改,不仅仅是殿试,礼部试和发解试都会加题目……很多一听就是无稽之谈,但偏偏有人信。”

“无稽之谈?”章敦笑了一下,“其实都没错,韩玉昆的确是打算这么做……只是迟早而已。”

韩冈要设立医学,其他人纵使想要反对,也没有能站得住脚的理由。

几十年前,范仲淹就让翰林医官在武成王庙给京中医者将《素问》、《难经》等医书,太医局中一直都在培养医官,九科归并之前,医生名额有一百二十人,之后又增长到三百人,等到韩冈将大方脉、小方脉、产科、眼科等九科归并重组为内科、外科、眼科、耳鼻喉科、妇产科、儿科、牙科七科,在太医局中就学的医生更是达到了四百五十人,已经是国子监的四分之一了,比武学的人数都多。而地方上也一直都仿效太医局的制度,设有医学博士、助教等职位。

韩冈现在不过是将之换个名目,增加人员,并稍稍更改一下制度。以韩冈在医道上的权威地位,没人能够从道理上反对他。。

而且朝野内外都希望能有一个更好的培训机制,让所有人都能受到更好的治疗。另一方面,东京的两家医院,收入都高于支出,负责种痘的保赤局同样也有收入,尽管两者的盈利都不算多,但终究不是从朝廷手中刮钱,吕嘉问也没办法为难韩冈,何况他也不敢犯众怒。

一直以来,技艺高超的良医从来都是游走于朱门,而顶尖的名医,则大多为太医局搜罗,服务于天家。两家医院的创立,让普通的百姓也有机会接触到高高在上的翰林医官们。每隔五日、十日,都要在医院中接诊的翰林医官,成了重病患者们最大的希望。

现在韩冈想要让更多的人得到名医或名医弟子的治疗,要是哪个朝臣敢在此时说一句不,这名声就别要了,家乡父老都能戳烂他的脊梁骨。

章敦很清楚一点,韩冈从来不在乎裹挟民意,如果有人想要阻拦,他肯定很乐意让此人千夫所指,然后趁势一脚踢出去。

但这是韩冈推进气学的又一步,不像王安石当初借助天子之力,强行将诗赋改成了经义,并以三经新义为本,而是从外到内,一步步的慢慢来,走得稳当,就像他的年纪一样,一点也不需要着急。

今天是医学,明年就是明算科、明工科,等到两科取士的人数到达一定数量,韩冈就有足够的支持者来改变进士科举试了。

特奏名考的头名,都会被授予同进士出身的资格,明法科也算是有正经出身的科目。明算科和明工科如果当真要创设,自不会比明法科要差。说不定还能连带着明法科一起受惠,仿效特奏名,让排在前几名的考生,得到进士的资格。

“进士啊……”

听到章敦无意中发出的感叹,章恂心中一悸,韩冈这是要对进士科下手了,难怪自家的堂兄是这幅模样。

章敦却没有在意自己的感叹,是不是让堂弟误会了。

新党虽遍及朝堂,可大部分外围成员和一部分核心成员,都是为了权力而来,趋炎附势之辈,蔡确就是最好的代表。另一种坚持的是新法,而不是新学,章敦他自己就是属于这一类。就是因为这个态度,所以章敦才不能得到王安石的全力襄助,只是依靠军功才得以进入两府。

最后便是绝不容忍气学与新学争夺官学地位的一批人,目下除了王安石之外,剩下的几乎都是在国子监。

可只要韩冈还没有动到他们头上,他们也不会主动去挑战。连王安石近来都偃旗息鼓了,终于是尝到了当初韩琦文彦博、富弼看到先皇和他君臣一心时的感觉了。

也不知过了多久,冷笑了一下,章敦才问道:“市井中还有别的什么传言。”

“也没别的了。”章恂道,除了韩冈要设医学、改科目之外,外面传说得最多的就是辽使耶律迪的丑态,“对了,外面还谣传说皇城司的人围了都亭驿,说是要抓躲到里面的辽人细作。说得有鼻子有眼。”

“十一你不信?”

“不是使团成员,就是有辽使耶律迪护着,也保不住他。哪个细作敢往死地中躲?”

“的确是这样……不过抓细作是真的,皇城司已经抓了不少。”

‘开封府怎么不管管,当街抓人,就算是细作,也不该是皇城司出手。今日可以当街捕贼,日后说不定就能横行闾里,抄杀官宦?’

章恂很想这么对章敦说。虽是行商,他身上也有个买来的官身,只是没有去候阙就任。拿着一份俸禄,便对皇城司这种组织分外提防,如今听说皇城司将拿着刀的手而不仅仅是耳目放出宫外,立刻就想到了未来的危机。

但他更明白,章敦如果当真介意此事,早就闹到太后面前了。看他的样子就知道,根本就没有。

而且这件事跟开封府的关系更大,但沈括是韩冈的人,看到了也只会往地里埋着头,只当做没看到。其他人还能说什么?

未来可能会发生的危机和眼下敌人派出来的细作,二者比较起来,太后那边肯定是觉得细作更危险一点,而御史台,现在几次换血,也变得不敢说话了。

“这么快就抓到人了。”章恂堆起笑脸,“皇城司还真是有一手,还以为他们只会竖着耳朵听内城各府里晚上的说话呢。”

“王厚还算会做事。”章敦平静的说道。

“不过城里搜捕细作,也不知怎么就传成了兵围都亭驿,这市井中的流言都只顾耸人听闻了。”

“什么流言都不会流传太久,而且,马上就有事要发生了。”章敦语调深沉。

“什么事?”章恂立刻问。

章敦摇了摇头,却没说什么。

转头望着窗外,也不知朝堂上有几人能看出来?
