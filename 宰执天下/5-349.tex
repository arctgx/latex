\section{第33章 枕惯蹄声梦不惊(六)}

园中春意盎然。

红的是花,绿的是叶,几只黄雀在嫩红色的枝梢上吱吱喳喳的叫着,间或啄起几只小虫。

文彦博手持黎杖,穿行在草木之间。

世所谓‘人间佳节惟寒食,天下名园重洛阳’,洛阳的园林之盛,远过于他处。但凡来此居住的元老重臣,无不经营园林,以作自娱。

自致仕后,他这位三朝元老纵然还能遥遥影响朝廷政事,但大半的精力也只能寄情尺山寸水中。院中亭台花朩,皆出其目营心匠,耗费了多少心神。

望着满园的姹紫嫣红,草绿水清,委屈在洛阳七八年的文彦博也是心怀大畅,步履也轻快了许多。

两名小婢在前引路,身后又跟着四名。

文及甫紧随在身侧。他跟着文彦博从最南的卧云堂,穿梭在各处亭台水榭、山石水脉之间,走了已有小半日。瞧着文彦博的兴致越来越好,却忍不住心中的担忧,“大人,今天已经走了不少路了,到四景堂中歇歇脚吧。”

“去荫樾亭!”文彦博兴致极高,“顺便看看你弄的那些牡丹怎么样了。”

文及甫迟疑了一下,“……还得几天功夫。”

“不用急,关键要办得好。只要能赶得上花会就行了。”文彦博说着,依然是往东头走,“旧时有所谓天下九福之论,京师是钱福、眼福、病福、屏帷福,吴越有口福,蜀地药福、秦陇鞍马福、燕赵衣裳福,而洛阳,则是花福。花会办得好,花福才留得住。”

所谓的福,自是冠于天下。东京钱多,风景多,人物好,有良医,屏帷是特产,吴越乃是太平地,在烹饪上的发展比北方要强得多,蜀地气候适宜药材生长,关西有好马也有好鞍,河北的织造名声大,至于洛阳嘛,特产的牡丹贵为花王,自然是天下第一。

“不过那也是国朝之初的事了。”文及甫说道,“京师多了赌赛福,口福不输吴越,秦陇衣被更胜燕赵,也就洛阳的花福无人争。”

“这也没什么争不争的,人心所向而已。”文彦博回头看看儿子,“归仁园的会场是你操办,不要输给天王堂花园子那边才是。”

归仁园在归仁坊,或者说归仁坊就是归仁园。洛阳城周五十里,城内苑囿众多,不过最大的还是归仁坊。单是其中的竹林就有百亩之多,乃是唐时宰相牛僧孺家的园林。相对于归仁园,,白居易的,,都比不上。司马光的独乐园则更小,

文彦博家的苑囿规模虽也不小,但与归仁园比起来,就差了远了。不过论起景物之盛,文彦博却不认为会输给归仁园,而且又是新起不过十几年的园子,比起牛僧孺的旧园自是更胜一筹。

三月四月牡丹开,一年一度的牡丹花会,也是在牡丹花开正盛的时候举行。天王院花园子是多年来惯例的集会之地。园中牡丹有数十万本。城中依靠牡丹为生者基本上都住在天王院附近。每至花期,花园子及其左近立成闹市,张幙幄,列市肆,管弦奏于其中,城中士女皆过而游之。

而今年富弼和文彦博突然来了兴致,旧日都不会去那些太热闹的地方凑趣的两人,却联手操办起了花会来。富弼挑头,邀了文彦博,把归仁园也借了下来,连同花园子一并当作了会场。甚至还打算模仿,邀请去岁洛阳蹴鞠联赛的头名和次名在校场中来一场比赛。

文府这边文彦博很上心,富弼那里更不用提。总之都是当成了一桩正经事来大事操办。

外人乍听时,都免不了要抱怨,北面都打成那样了,两位老相公倒还有闲心开花会。但上层都明白,富弼和文彦博是故意如此。

“富彦国既然有心,为父也不能后人。都在洛阳住了这么些年了,也不想看到河南府乱起来。”文彦博道:“再去问问其他几家,一起凑个趣好了。司马君实在地洞里住得也久了,该出来见见太阳了。”

“是,孩儿会修书去请司马君实。”

“还有金带围,也该从环溪里搬出来了。谅王君贶纵然再舍不得,也不会摆出张苦脸来。”文彦博手捋着胡须,咪咪笑着说。

扬州的金带围芍药,红瓣黄腰,如同腰缠金带、衣着朱紫的宰执。不过这仅仅是红色芍药的偶尔才见的变异,绝少出现,没有被培养成一个独立的品种,不过一旦有幸开花,世传就预兆着城中当出宰相。

韩琦旧年知扬州,却是一口气出现四支。韩琦算一位,已有声名的王安石、王珪当时也在扬州,却还缺第四人,正好陈升之路过扬州,便被拉上了宴席。四人簪花围坐,日后就出了四名宰相。

而洛阳这边的金带围为牡丹,却是已经成了固定的品种,每次开花都是上下皆红,中间一圈黄,虽然依然名贵尤胜姚黄魏紫,却也比不上扬州的金带围那样能成为有神异的传奇了。

不过金带围牡丹终究还是稀少,能出现一株,当也能为花会增光添彩,而且那一株正是出自王拱辰的环溪园。

王拱辰有着开府仪同三司的头衔,乃是洛阳元老中,富、文之下的第一人。他若能与会,对富弼和文彦博的计划也有好处。

文及甫的脸挂了下来:“王开府的家眷今早城门开时就出了城,说是去别庄小住。似乎不像是要参加花会的样子。”

“……随他去。”文彦博沉默了片刻,又往前走,“王君贶要走,也拦不住他。让他走好了。多他一个不多,少他一个不少。”

文彦博和富弼对牡丹花会的兴致,并没有他们表现出来的十分之一。不过是镇之以静,安定人心的手段。

尽管洛阳向北,度过孟津后,就是太行余脉。往太原去的路程,比东京去太原要少上几百里,可是在洛阳这边,却很难及时了解得到河东的军情。

富弼也好,文彦博也好,都是只能收到从东京城传来的二手消息,时间上能延迟个十余天。至于其他致仕的元老,当然更是不会例外。

这么长时间的延误,使得洛阳内外对于整个河东战局,总有着许许多多毫无来由的猜测甚至恐惧。在这个时候,一干元老重臣的表现,便决定了谣言的方向。

他们这些老家伙越是稳当,洛阳也就越安稳。而一旦元老们一个个将家眷往南方转移,那么河南府的富贵人家三五天内就能跑个精光。

不管怎么说,富弼、文彦博都是三朝元老,不打算丢人现眼的去让后生晚辈指责。只是如果王拱辰,以及另外那几位想跑,文彦博也不打算管,就随他们去好了。有了对比,反而是一桩好事。

侧头看看了小心翼翼跟在身侧的儿子,文彦博暗暗叹了一口气,他都到了这把岁数了,还要想方设法为儿孙铺路,真是天生的冤孽啊。

“枉食君禄,都不知道什么叫做休戚与共。”文彦博的脚步慢了下来,“韩玉汝【韩绛】的全家老小前几日还一起出门去看了球赛,你想得到吗?”

“嗯,孩儿听说了。”

这件事富绍庭也听说了。太原为敌军所困,东京城中的混乱只会比洛阳更盛。

韩绛作为首相肯定要为君分忧。不仅仅是韩绛,蔡确、张璪据说都有活动。至于王安石,他那个性格,却做不来这样的事,倒也没人会误会。

富弼和文彦博都在担心,若是后方乱起来,那么河东的局势可能真的万劫不复了。到时候,关西胜了又如何?河北甚至反攻辽境又能怎么样?还不是鸡飞蛋打的结果。

平日里拖一拖后腿倒也罢了,到了如今的局面下,还是以同舟共济为上。打烂了河东,让辽人入寇中原,谁的日子都好过不了。

更何况,文彦博的乡贯正是河东汾州的介休。

只是毕竟不是人人都有富弼和文彦博这等见识。建议与辽人媾和的奏章一封接着一封,士林中也多有批评,时局败坏如此,皆是两府之中奸人当道的结果。否则近八十年的澶渊之盟,如何会毫无征兆的就宣告破裂?

河东的责任说起来并不在韩冈,只是欲加之罪何患无辞?在许多人的奏章中,由于夺占了胜州,使得河东兵力偏向西北布置,代州失去了为数众多的精兵强将,才会如此脆弱。加之韩冈之前还在朝中时,曾经赞成河东派兵出援,由此使得太原兵力空虚,更是无法辩驳的事实。尽管事出有因,而代州的陷落无人能预料得到,可并不妨碍有人将罪名算到韩冈头上。

关西的主力离得太原,还有一段时间才能支援得上。河北似乎正打着围魏救赵的主意。京畿还要留着很大一部分兵力来稳定人心。韩冈手上的资源其实并不多。

这也就是为什么以韩冈的声望,还是有很多人不相信他。

“真不知道那些人究竟是怎么想的。乱了军心,自毁长城,就能让辽贼偃旗息鼓了?”文及甫都想不明白,他自知才智不高,能力不强,连进士都没敢去考,但现在看到一名名进士出身的官员,却恨不得自毁长城来让辽国尚父稍息心头之怒,真不知他们是怎么靠上进士的。

“谁能知道?”文彦博冷哼。他当然知道是为什么,但他没心思去体谅。什么时候能做什么事都不懂,衡量轻重的眼力都没有,也不值得文彦博多费一份心。

有句话他没说出口,搜遍朝野,能稳定河东、挡住辽寇的,除了近年安抚河东、深孚人望的韩冈,不作第二人想。

这个认知世所公认。虽然文彦博已经是一把年纪,往日又跟韩冈交恶,但他还没糊涂到会自欺欺人的地步。几次三番让他难堪。韩冈若是无能幸进,那他又算什么?
