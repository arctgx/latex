\section{第30章 肘腋萧墙暮色凉(17)}

辽人的插手完全出乎于赵顼的意料之外,让他猝不及防。一场宋夏两国的边塞之争,怎么会引起北方的注意,这让赵顼在震惊中,又百思不得其解。

摊在眼前的辽人国书让赵顼心烦意乱,挥手想扫到一边,却在不经意间把桌上的茶盏打翻。里面的茶汤洇湿了御桌上的国书,也溅到赵顼的身上,湿淋淋的直往下流。

随侍在侧的李舜举见状连忙上来收拾,把国书拿起来也不敢多看一眼,小心翼翼地擦干净上面的茶水折放起来。伴君如伴虎,虽说从真宗以后的大宋诸帝都是宽和的性子,但天子就是天子,一点小事触怒了他,就能让自己万劫不复。在服侍天子的时候,谨守本分是最重要的。

“官家,先换身衣服吧……”

李舜举收拾干净桌子,看了看赵顼的脸色,又轻声道。但赵顼却失魂落魄的什么都没听到。

在他数年的天子经历中,尚未跟辽国有过太深的接触。只是不止一次的幻想过收复燕云,实现太祖太宗也没有完成的事业。但对契丹兵马的恐惧,却也是深深刻在他骨子里的。

由于地形和国势的因素,党项骑兵突破不了关中。但辽国却是大宋被迫要与其并称南北朝,不得不结为兄弟之国的强国。从位于燕山南侧的辽国南京道,一直到东京城下,除了一道黄河之外,并无其他天险可以凭借。而辽国数十万骑兵举手可集的实力,让人想起来就不寒而栗。

从开国之初一直到到澶渊之盟订立,大宋虽然抵挡住了辽国的屡次进攻,但每次宋辽交战的战场,都是在大宋这一边。一旦没能在河北将入侵者堵住,契丹铁骑就将直逼东京城。这样的结局是每一个宋室天子的噩梦,难道赵顼很想每年送上五十万银绢给辽人?这是花钱买平安,不得已而为之!

王安石在下面看得直皱眉头,赵顼如此失态,让他这个宰相都看不过眼。心中也不由暗叹,究竟不是从小就作为皇储来培养的皇帝。

赵顼虽不是在深宫中养大,但也没出过富丽繁华的东京城。自幼时起就没有受过什么挫折。虽然梦想着能重现汉唐遗风,能如唐太宗一样,文成武就,成为名流千古的明君。但真正临到大事时,却远不如李世民这等经历过战争的帝王性格坚毅,情绪波动极易受到外界的影响。

“陛下!”王安石终于按耐不住,高声提醒着赵顼他的身份。

宰相责难的声调让赵顼仿佛是被先生斥责的学生,慌慌张张的想着:‘对了,要派人去应付辽人!’

“让冯京去做馆伴使!”赵顼连忙说道。

宋辽两国在对方国中,并没有常驻使节,不过在正旦等重要的节日,或是天子、太后的寿诞,双方都派出使臣去对方国中贺礼。朝中做过使臣去过辽国的大臣不少——王安石就去过辽国,还留下了几篇诗作——而为了接待这些使臣,就有了所谓的馆伴使。

依照双方地位对等的原则,受命接待辽国使节的馆伴使,一般都是选则与对方正使官位相当的官员临时充任,当然,也要考虑把能力和口才考虑进去。

不过现在赵顼也顾不得那么多了,应付辽人,至少要宰执一级。但王安石是宰相,绝不可能让他去;王珪是个软性子;而文彦博又是乐得接受辽人的条件。只有冯京勉强能充任。

“陛下!”王安石见赵顼完全陷入混乱之中,心头更是不快,高声提醒着,“仅仅是至书而已,并不是有使臣来了!”

“啊……啊!”赵顼这时才稍稍冷静下来,用手按着额头,问着王安石:“王卿,辽人这份国书,究竟该如何处置?”

“只是边塞之争,何预辽人事。明说是为了膺惩西人屡犯边塞之举便是。辽人只是虚张声势而已,何尝会为西人火中取栗?”

王安石虽是因为辽人插手宋夏之战,而赶在宫掖落锁前入宫,但他对辽人的威胁还是保持着强硬的态度。他见赵顼还有些犹犹豫豫,又加重语气说道:“眼下罗兀鏖兵,战事正烈,一旦朝中贸然下令退兵,罗兀城的上万守军,可能安然回返?”

赵顼慢慢的点着头,似是赞同王安石的言辞,但脸上的犹豫亦依然不减。

“攻取横山,谋划已久。积数年之功,因辽人一言而退,让外间如何看待,朝廷的体面可还要了?日后使北,使臣又如何在辽国抬得起头来?!”王安石的质问如同用鞭子抽打着赵顼的自尊心,“如果今次依辽人之言而退兵,日后整兵攻夏,难道辽人就不会再说吗?届时不知陛下意欲如何?”

赵顼终于被王安石说动了,他现在最在意的目标便是剿平西夏。若是总是要顾忌着辽人,日后那就不用再妄想观兵兴灵了。“王卿说得是!就依王卿之言。”

王安石走了,下定决心的赵顼又坐立不安起来。

他很清楚,只要这个消息传出去,出身于北方的大臣们,必然会群起上书,逼天子下令收兵。对于辽人的威胁,北方人有切骨之痛,而王安石这个江西人,却是隔了一层。赵顼能够想见出身河东的文彦博在朝堂上跳脚的样子。

幸好王珪和冯京都是南方人。要回辽人国书,光是天子和宰相点头还不够,必须要参知政事点头。没有执政的副署,诏令就不算合法,国书也不合法。如果有个北方人做参政,他们会不会同意王安石的意见回至辽人国书,那就可是难说得很。

直至夜深更漏,赵顼犹在灯下踯躅。福宁殿中,数十支龙涎香巨烛已经烧去了一半,却也不见赵顼有半分就寝的意思。刚刚病愈,便熬夜下去,这身体如何受得了?今日当值的李舜举劝了几次,却见官家是越来越不耐烦。无奈之下便想去让人通知太后或是皇后来规劝,但赵顼却突然开口,叫住正想悄悄去殿外叫人的李舜举。

赵顼问着李舜举:“若是要派人去鄜延体量军事。你觉得宫中谁人为好?”

“官家!”李舜举一听之下,慌忙跪倒,这事他哪敢插足进去?传出去,宰执班中没一个能饶他。他连磕了几个头,言辞恳切的劝谏道:“我等刑余之人,当时洒扫庭院,侍奉天家。鄜延战事事关重大,岂有我等内臣插言的余地?还请官家自朝中选取贤能正直之臣前去鄜延!”

赵顼摇了摇头,他需要的是准确、而不带任何偏见的情报。遣朝臣去并不是不好,但他们不像宫中的宦官,各自的立场都太过明显,回报也免不了要被他们的立场所影响。

赵顼瞥了言跪在地上的侍臣。李舜举行事素来小心谨慎,不敢稍逾规矩,这点是他很喜欢的。但今次赵顼却还是要听一听鄜延那里的真实情况,好决定在罗兀城后路受到威胁,而辽人又为西贼撑腰的情况下,罗兀城的现状到底有没有让他坚持下去的必要。

“你且起来吧!”赵顼先说了一句,又道:“你明日知会王中正,让他去鄜延一趟。”

……………………

“玉昆!可曾行了未?”

天还没亮的时候,韩冈就被一个略嫌苍老的大嗓门从睡梦中叫醒。摇了摇混混沉沉的脑袋,韩冈从硬邦邦的床铺上起身。昨天他是和衣而睡,也省得换衣服了,直接就着盆中的清水擦了擦脸,就走出门去。

站在门外叫醒韩冈的是一个须发已然花白,但筋骨依然强健,个性看起来很张扬的老家伙——张玉。

“劳总管久候了。”韩冈连忙上前行礼。

“不是让玉昆你不要这么多礼嘛?”张玉摇头了,摆出了很不高兴的样子。

他是在三天前,冲进了罗兀城的两千骑兵的领军将领。有了援军入城,罗兀城到底能不能守住,城中已经没人再抱有疑问。

张玉擅使双简,军中人称张铁简。今次就是他领军冲入被围困的罗兀城,而且还是冲在了最前面。当他进城的时候,手上的一对铁简还向下滴着血水和脑浆。

这老家伙倒有些自来熟,前日领军来罗兀的时候,虽然亲手敲瘪了几十个头盔和头盔下的脑袋,但也受了几处伤。进城后就被送到了韩冈这里,聊了几句,就立刻亲近得叫着韩冈的表字了。张玉是外路客将,虽然地位远在高永能之上,但也无意去抢他的指挥权。为了避嫌,也不住进城衙。就住在军营中,跟着韩冈的疗养院紧靠着。

除了上阵对敌,或是与高永能讨论兵事,就来找韩冈聊天。张玉跟着狄青南征北战,陕西待过,广西也待过,满肚子天下见闻,与同样广博的韩冈倒是相得得很。

看到韩冈把疗养院中处理的井井有条,张玉每每都说,要是当年狄武襄率领西军,南平侬智高之乱时,有韩冈处理军中疾疫,也不会十个人去,五个人回了。

聊了一阵,张玉自去找找他的兵去——西夏人玩了两日日夜攻城,损失的兵力就大感吃不消,只能摆出了长期围困的姿态。等到张玉领军入城后,城中军心重振,反倒是守军日日出城摆阵挑战。

韩冈看了看天色,等到再过半个时辰,今天的例行就该开始了。但过了半个时辰,传来的不是出战的战鼓声,而是主帅高永能的召唤。

面对城外的数万敌军,高永能没有变色。面对抚宁堡的烽火,高永能也没有变色。但走进主帐的的韩冈,现在看着高永能,却分明铁青了一张脸。而方才跟自己言笑不拘的张玉,也是板着脸,很阴沉的站在一边。

等到城中的文武官员一起到齐,罗兀城的主将张开口。只是他嘴唇哆嗦着,几次张口,却都吐不出一个字来。

“磨蹭个什么?!”张玉在旁边不耐烦了,厉声呵斥着高永能。

高永能被骂了一声,也终于能说出话了,但在场的所有人都不希望听到这个消息:“三天前,庆州广锐军兵变!”

