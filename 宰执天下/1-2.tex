\section{第二章 摇红烛影忆平生(上)}

韩父韩母貌似被说服了,就算明知李癞子是为了自家的田地,但与宝贝儿子比起来,田地又算得了什么?人没了,留下田还有什么意义?

‘不要卖!’贺方有些惶惑,这不是他的意识,而是莫名的从心底里爆发来的念头。郁愤充溢于胸臆,自责,愤怒,诸多情绪在心头交替浮现。躺在床上的这段时日里,正是这个公鸭嗓音不停的劝说家里将田地换成钱钞,去为他求医问药。到最后,就只剩下一块菜田,也不肯放过。

不知何时,李癞子已经走了,而韩父韩母又坐到了自己的床头前。夫妻相对无言,只为了儿子,倾家荡产也甘愿——可怜天下父母心。

“卖了吧,不就一块地嘛……把三哥儿救回来就好!总得试一试。”韩母叹着气,手掌轻抚着贺方的额头,全没有方才对上李癞子的刚硬。

韩母的话让贺方心中一阵酸楚,不知是出自于自己还是韩冈。韩母放在额头上的手很粗糙,像砂纸一般,但掌心却出奇的温暖。

韩父看着已经瘦脱了形的儿子,刚过四十就已经十分苍老的脸上有着掩饰不住的忧伤,家中只剩这么一根独苗,若是再没了,他夫妇俩还有什么活头?他点了点头,声音嘶哑低沉:“那好,就先把田典卖给李癞子,价钱贱就贱点……总得先把三哥儿救回来。”

“啊……啊……”贺方突然间挣扎起来,拼尽全力想挤出‘不要卖’这三个字来。但喉咙仿佛被什么东西堵着。久病的他很快便用尽了体力,在韩家父母惊喜交加的声音中昏了过去。

……………………

不知又昏睡了多久,贺方第三次醒了过来。这一次,他终于有了睁开眼皮的气力。张开双眼,首先映入眼中的是一片不停摇曳着的昏黄灯光,还有一股子刺鼻的气味。

‘是油灯!’明显的,只有不稳定的火焰才会摇晃。同样的,也只有点着油灯才会有一屋子的烟气。

‘果真是穿越了吗?’

贺方转动着双眼,巡视着自己身处的这个房间。房间很小,大约只有五六个平方,比韩冈记忆中属于自己的厢房还要小上许多。但房内的灯火是如此的微弱,以至于如此狭小的房间也无法完全照亮,就连头顶上的天花板也笼罩在一片黑暗之中——

‘哦,对了!可能根本就没有天花板。’贺方想着,因为在他身侧,还是黄土夯筑成的粗糙墙壁,表面上还有着因岁月而沉淀下来的黑色,但墙体土纹依然清晰可辨。想必这样的古代房屋,头顶上的应该是如同前世老家旧宅那样的房梁和椽子,而不是平平一片的天花板。

‘当真是穿越了。’

看清自己所睡的卧室,贺方苦笑着,终于确认了这个他并不想承认的事实。死于二十一世纪的空难,而在复活在千年前一名宋朝少年的身体中。如果是故事,说不定会很有趣,但发生在自己身上,那只能让人叹气了。

不过贺方还是暗自庆幸,死于空难,转生古代,其中祸福难分。福兮祸之所伏祸兮福之所倚。虽是老生常谈,却一点也不错。被匪夷所思的现实冲击过后,认清了自己现在的处境,贺方心神逐渐沉静下来。如果要在宋朝好好的活下去,就必须先了解这个时代。

他静下心来在脑海里细细搜寻,惊喜的发现身体原主人留下的记忆尚算完整。父母、亲友、师长、乡邻都能记得分明。就是这些记忆仿佛隔在一层薄纱之后,让他无法产生足够的认同感,就像是在观看一出冗长的电影,没法当成是自己的记忆。不过这样已经足够,贺方庆幸的想着,靠着这些记忆,只要谨言慎行,少说多看,并不用担心冒名顶替时会出什么大问题,就算有些差别也还可以推到病症上去。

如今是熙宁二年【西元1069】——对历史从来都是勉强及格的贺方来说是个很陌生的纪年。但靠着身体原主人留存在记忆中的宋朝太祖、太宗、真宗,和刚死没几年的仁宗皇帝、英宗皇帝,以及王安石、司马光、苏轼、柳永这些熟悉的名字,再加上契丹、西夏、大理这些更为熟悉的国号,还是让贺方确认了自己所在的时代。

在大庆殿的龙椅上坐了四十二年的仁宗皇帝于六年前驾崩,享国虽久,却并未留下子嗣——生了一堆公主,却一个皇子也没有。作为仁宗远房堂侄的英宗皇帝遂以过继皇子的身份入继大统。但体弱多病的英宗皇帝也并没能在皇位上坐太久,仅仅四年多一点的时间,便紧追着他名义上的父皇的脚步,撒手尘寰,将偌大的一个帝国交给了还不到二十的长子赵顼。

天子登基,便要改元。大宋的年号由此从治平改为熙宁,而今年正是第二个年头。而这位新皇帝,想来应该就是与王安石变法紧密相连的宋神宗……回想到这里,贺方心中猛然一凛。

对了!神宗是庙号,没死的皇帝还享受不到,若是贸贸然如此称呼当今天子,怕是不会有好结果。贺方暗叹一声,这又是脑内的记忆留给他的常识。

且不管该如何称呼如今的皇帝,赵顼对宋朝过去几十年来的积弊心中不满,意欲学习商鞅变法,从而富国强兵的打算,贺方是能够百分百肯定的。

就算没有他本身对历史一点浅薄的了解,只看这拥兵百万的堂堂天朝上国,每年竟不得不向辽国、夏国献上岁币,用钱来买一个安稳。号称中国,却为四夷所欺,泱泱大国受此奇耻大辱,一想起来,但凡有些羞耻心的宋人都会悲愤不已,连带着贺方也被残留的记忆影响着感到满腔怨愤。小民如此,更不用提大宋之主——毕竟——如今的皇帝赵顼才二十出头,正是勇于有为、无视陈规的年龄。

而贺方现在之所以会躺在床榻之上而动弹不得,追根究底,却也是因为大宋军力不振,屡受西夏相欺的缘故。

贺方所占据的这具身躯的旧主,姓韩名冈,有个表字唤作玉昆。名和字都是韩冈幼年时的蒙师所起,用的是《千字文》中‘金生丽水,玉出昆冈’这一句典故。

想到这里,贺方忍不住又要苦笑。他穿越到宋代的事情肯定是坐实了。不然脑袋里不会多出一堆他从没读过的古文和诗词,更不会知道什么典故。这都是那位韩冈自开蒙后,十几年来陆续背下来的。

韩家说不上富裕,但在与陕西路绝大多数乡村同样贫困的下龙湾村中,也算得上是上户人家。有百十亩地,一头耕牛。只是还算不上地主,平日都是自己劳作,只有在农忙时才会雇些短工来,而家中主业则是种菜。从河湾旁的几亩称得上是膏腴的上等菜田中,种些春韭秋菘【注1】之类的蔬菜,卖到仅是一河之隔、近在咫尺的秦州州城中,换来的钱钞维持着家中二十多年的小康生活。

韩冈是家中的三儿子,连着他的两个兄长,都很幸运的养到了成年。这在幼儿夭折率超过一半,连皇室也免不了因此而绝嗣的宋代,算是个小小的奇迹。

韩冈的长兄继承家业,二兄投了军中,而他本人则是自幼聪颖,家里便省吃俭用供他进学。八岁开蒙,十二岁便通读五经等诸多典籍,是十里八乡有名的秀才。到了前年,也就是治平四年【西元1067】,韩冈满了十六岁,便辞别父母,与此时的士子们一样,开始离家出外游学。

北宋承平百年,文风大炽。早一点的孙复、胡瑗,近时的欧阳修、周敦颐,还有如今的王安石、司马光、邵雍、程颢、程颐,有名的、无名的,学者大儒层出不穷。

而就在关西,也有一名开宗立派的博学鸿儒,姓张名载。张载在关中地区广收门徒,弟子众多,其创立的学派号为关学,韩冈便是投奔在他的门下,勤学苦读了整整两年。

韩家所在的路州并不太平——位于大宋西北边陲的陕西秦州。在二十一世纪,陕西的风土人情贺方见识过很多,却从来没有穿越战火的经历。但在北宋,陕西却因为直面西夏,故而年年兵灾不断。

在韩冈留下来的记忆中,二十多年前,李元昊继承父位,统领西北党项各部之后,便举起了叛旗。李元昊为人残暴不仁,又好渔色,连儿媳也不放过,最后也是死在了亲生儿子之手。但他的确是个人杰,抛弃了宋国的赐姓,为自己找了个鲜卑族的先祖,改姓嵬名。率领原本就已经是半独立的银夏党项,攻下了河套平原上的兴灵二州,自行登基称帝,建立了西夏政权。短短数年间,三次大规模会战,宋军皆以惨败而告终,十数万大军覆没,只能承认了西夏国的存在。

注1:韭是韭菜,菘则是白菜。这两样是古代最常见的蔬菜。