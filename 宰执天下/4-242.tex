\section{第39章 遥观方城青霞举(七)}

【第二更。】

将韩冈托付的工作完成的尽善尽美,从方城山返回襄州的李诫,正满心得意的站在船头。远远地就看见了已经成了襄州港中标志的龙门吊,心中的自得更加难以遏抑。

回头对着从家里带出来的一老一小两个伴当,舒心畅意的笑着:“这一回回去见到了龙图,也算是能交代了。”

两个伴当都是一团喜气,他们这等下人的脸面是靠着主人挣来的。一个是从李诫少年时就跟在身边服侍的苍头,一个则是读书时做伴的书童,最是亲近不过。李诫若能靠着韩冈得个一官半职,他们出来也有体面,日后李家分家,也少不了一个管家的位置。

老一点的说着:“韩龙图不喜多养清客,但一向对能做事的幕宾最大方。方管勾在王相公家几年,跟龙图已故的内兄交情匪浅,但到最后还是在龙图这里出仕。算起来,曾经正经八百被龙图收为幕僚的几位,全都得了官身,竟没漏下一个。”

李诫笑眯眯的,直点着头。

李诫的书童也跟着说道:“龙图的确是念个旧情的人。小人跟龙图家的周五哥最好,听他说起过龙图是怎么待。周五哥当年跟着龙图在熙河路征战,生死关头不知走了多少次,就是缺了点运气。最后腿也废了,手也抓不了东西了,当兵的也不知道积攒,到最后在军中也待不下去,只剩了些河湟开边功成后的赏钱。周五哥老子娘都死了,回去后,家产也分光了,还要听兄弟们的刻薄话,要不是韩龙图,早就饿死了。像他这样的身上有残疾的老兵,有许多都投到韩龙图家中。那个为龙图造了飞船的周全,也是少了一只手。”

他絮絮叨叨的说了一通,老伴当接口道,“小乙说得没错,韩龙图的确是念旧情。想想他对横渠先生,还有两个同样教了他几天书的名儒,四时八节上礼数从来没缺过。方管勾也就当过一年半载的幕僚。这一次为了能让他转官,龙图把发运的差事都交给了他。眼下道路通了,剩下的功劳就全是他的了。”

“这一条水道向北,尤其是方城山那一段,还有港口、官道,还不是二郎你督造,凭什么给他占便宜?!”书童为李诫打抱不平。

“谁让他是韩龙图曾经用过的老人。”老伴当道:“只要在韩龙图门下走得近了,这一次就算没有好处,日后也少不了一个官身。”

“什么日后啊,这一次还不肯定有!”书童立刻更正。

“官身什么也不指望立刻就能成事的,要是能派几个人出来迎一下,也算是不枉这一番辛苦。”李诫嘴里说的和心中想着截然不同,可当他看清码头上战的是谁的时候,脸色完全都变了。

李诫着实吓了一跳,韩冈竟然出城来迎接他,直率的性子让他一时间装不出感激涕零的模样,满是惊讶,“李诫怎当得起龙图来迎!?”

韩冈笑了一声,他出城并不是全然为了迎接李诫,不过也没打算解释什么。随性子来好了,有些事根本不需要在李诫面前解释,“明仲你一番辛苦,难道还当不起让我多走几步?”

“上下尊卑还是要讲究的。”李诫陪着小心,尽量不去看就站在韩冈身后的方兴。。

韩冈摇摇头,问道,“山阴山阳的两座港口,现在的情况怎么样了?”

“山阳山阴的两座港中的龙门吊都修好了,李诫也想早些通知龙图和方管勾。”李诫向韩冈解释着他的行动,“山阳港那里有沈知州亲自督促,山阴港也有汝州的方知州。两边都不缺人盯着,似乎又有一较高下的想法,估计最多再有十天八天的样子,两边的码头就都能用了。。”

韩冈看了看恭谨有加的李诫,回头又了一眼满脸喜色的方兴,满意的点头,“既然如此,也就该做好准备了。若是换个年份,这么点事,谁来做都不会手忙脚乱,倒是要让人感到太过清闲。今年剩下的时间不多,把刚刚收上来的秋粮送去京城,时间上就卡得很紧,需要你们同心协力才是。”

韩冈发话,方兴、李诫恭声都应了。

韩冈抬头看着龙门吊上的吊索来回移动,只片刻时间,就已经将一船的粮食都运到了旁边一艘纲船上。前面的小船刚刚从码头中退出去,接着又是一艘满载着粮食的货船,在一群纤夫的吆喝下,顺顺当当的进了码头之中。

看了一阵,韩冈回头对李诫笑着道:“本来都以为当真要用民夫和厢军充几天力工了,幸好明仲把龙门吊打造了出来。”

李诫哪里敢居功,连忙道:“李诫只是照着龙图的吩咐去做,自个想几十年都不一定能想得出来吊运货物上船下船能有这么方便。还有就是龙图安排来的那些大工匠,又都是各有各的绝活,此事得力于他们甚多。上有龙图照管、提点,下有大工们主持,就是十岁出头的黄口孺子,也能把龙门吊给造出来。”

“莫要自谦,你的辛苦,我看得也清楚。”韩冈说着,“这一辈中,我见过的,能与你比肩的可没有多少。”

李诫忙谦虚了两句,陪了韩冈走了一阵,心有所想,道:“龙图,学生有个想法。既然方城轨道已经竣工,连两端港口眼见着也都快修好了。也该想想树碑立传的事。在方城垭口中立块石碑,也好让后人知道龙图打通襄汉漕运的辛苦。还有山阴港、山阳港的门额,汝州、唐州两边都说要请龙图去提个字。”

李诫参与并主持了大半工役,当然想能留个纪念。虽然不敢抢韩冈和沈括的风头,但能在碑上付上自己的姓名,也不往他这半年多来的一番辛苦,也能让他在家里扬眉吐气,省得每次回家之后,浑家就连着几天每个好脸色。

“再等等,等到当真见了功,再去考虑撰文立碑的事也不迟。”韩冈看了一眼难掩失望之色的李诫,笑了笑,“如果漕运当真有成,于朝廷稍有补益,我写信给家岳求篇文章也能算是理直气壮。”

“王相公!”李诫惊讶的提高了嗓门,不过立刻就警觉得闭紧了嘴。他倒没想到韩冈的心思更大,求文直接求到他岳父那里去了。

王安石当世文章第一,旧日的文坛座主欧阳修就算依然在生,也不敢说可以压他一头去。尤其是如今退居江宁,听说他的文笔日臻老辣,一年更胜一年。有王安石这么一篇文章做下来下来,再选个名家来书写,流传千古有了三五分可能。

想到这里,李诫的心中就是一阵激动。虽然他的心思都放在工器营造之上,但李诫依然自命文人。文人想要的是什么,钱和权那都是暗地里,真正光明正大求着的就是一个‘名’啊!

“其实山阴山阳两个港口的名字也可以改一改。”方兴在旁笑着提议,“山南为阳、山北为阴,山阳山阴,天下间实在太多重名了。沈、方二位不是说要龙图题字吗?干脆就顺道将名字改了。”

“不用急。”韩冈还是摇头,“真正成事了,自会有人想给两座港口起个好名字。”

方城轨道两端的港口,北面被叫做山阴港,南面被叫做山阳港,韩冈从一开始就没心思给两个港口取个正经名字。他很清楚,这两处关键性的节点,很快就会被人给换上个好听又有口彩的名字。

当今的这位天子是个好事的人,再偏一点就可以说是好大喜功,喜欢给新修的寨堡、驿站起名,熙河、秦凤这十年来新修的寨堡,有三成是天子的手笔,比如甘谷城、再比如巩州陇西,熙州狄道,皆是天子钦定。

如果方城轨道当真能将今年秋天的六十万石纲粮顺顺当当的运抵京城,以天子赵顼的脾性,少不得给两座新港取个吉利好听的名字

韩冈没有进一步想两位幕僚解释,说天子长短还是不太好的,而且此事也没有确定,万一赵顼对此没有兴趣,自己可是要丢脸了。

在江边走了一段,李诫突然就看到有人在仔细翻着江滩上的芦苇荡,他疑惑的问着,“这是在做什么?”

“这是在检查水中是否有钉螺。”方兴道,“你也知道,钉螺传疫症——蛊胀,得了就是福薄命浅,不指望能治好。但重要的是预防。所以官人就下令,去寻找钉螺滋生的地方,将数量多的地方划出来,让人躲着走。顺便撒点石灰什么的,不要让其再害人。”

李诫忍不住赞着前面的韩冈道:“龙图宅心仁厚,京西百姓,靠着龙图终免了疾疫之苦。”

“这可不敢当。”韩冈回过头来,“尽尽人事而已。若是普通的头疼脑热,根本就不用官府掺和了,自去请医生问诊。但防治疾疫却是官府之责,不能视而不见。身为一路漕司,正好是分内事。”

