\section{第33章 枕惯蹄声梦不惊(二)}

乌鲁匍匐在地上,几次想抬头,却都被一刻也不见停歇的箭矢给沉沉的压了回去。

宋人从城头上推下了一团团燃烧的草球,让他所率领的儿郎们大半暴露在火光中。箭矢撞击着铁甲,一声声或沉闷或清脆的响声连绵不绝。

纵然装备了宋人的铁甲,那么密集的箭雨中,总有那么几箭是甲胄防御不到的。以神臂弓的力道,三四十步开外被木羽矢射中,除了更为厚实的头盔,就是以胸甲背甲的坚固,也无法做到彻底挡住箭矢造成的伤害。

而且在城墙之上,连成一个音符的弓弦声中,还参杂着沉郁而厚重的嗡鸣,那是犹在神臂弓之上的破甲重弩在射击。

不知在何时——可能是在确定大辽各支宫分军也开始换装铁甲之后——宋人为了保证他们最为精擅的强弓硬弩的效果,所用的箭矢都已经改变了形制。大部分的箭簇改成了铸造,形制如一。几百支三棱形的箭簇摆在一起,甚至连每一条微微外凸的弧线都一模一样。这些是用于无甲或轻甲的敌人。

但另一部分箭矢则依然是熟铁锻造,可经过了不知什么样的秘术,锋锐远胜过往,箭簇上总闪着精钢的光芒。而最好的箭簇,据说用的则是数十炼的锻钢,就是配合专用的比神臂弓还要大一圈的破甲重弩才制造出来,箭杆更长,而箭簇却变得更为尖锐。

之前在代州和已经攻下的几处关隘中的武库内,曾经发现了大批的箭矢,铸造的锻造的都有,并给各部瓜分得一干二净——即便是铸造的普通箭矢,也远比辽国国内生产的箭矢更为精良。而锻造的上品,更是争抢的目标。

这些日子以来,看着手中光色幽暗的箭簇,有不少人都发现了铸铁箭簇根本是从一个模子里出来的。这一发现让包括乌鲁在内的契丹勇士都不寒而栗。这样的箭簇,只要泥模泥范能备得上,一间铁场一天怕不有几千支造出来了。

通过破甲重弩射出的破甲矢,轻易洞穿了宋人装备的铁甲,更是让多少自负盔甲不输宋人的武将们都暗道侥幸。今日放弃在白天攻城,而选择了暗夜,乌鲁估计有三成的缘由就是畏惧宋人的破甲重弩。

可是,现在,夜色并没有如愿以偿的抵消宋人在弓弩上的优势。

每时每刻,乌鲁都能听到周围传来一声两声的惨叫,还有被压低了的呻吟。而乌鲁本身也被笼罩在箭雨中,只是因为趴在一条菜田的田垄下,深藏阴影内,方才幸运的没有成为目标。

但刚才躲避的时候,背心处却接连有了几下莫名的刺痛,乌鲁当时心凉了半截,直到躲到田垄下,才惊魂甫定的发现自己还活着。最坚固的背甲虽然没能挡下箭矢,但好歹减低了许多威力。但他多半可以肯定,他只不过是运气好,没有被破甲重弩给盯上。

不知何时,箭矢渐渐稀落了起来,然后就断断续续的射击。方才缭绕在耳畔的密集弦鸣一下消散了,但远处犹有微声随着风传来。

这边的宋人已经用光箭矢了,要么就是没力气再拉弓了。就连乌鲁的脑中都闪过了这个念头,但他几十年来的经验所锻炼出来的直觉立刻提醒他,绝不是这样。只是他手下的儿郎,却缺乏这样的直觉。

“不要站起来!”乌鲁的吼声迟了一步,几名族中的战士已经飞快的跳了起来,直接向城墙脚下冲过去。即便被宋人用强弓硬弩压制许久,但他们依然无所畏惧。

只是这样的勇敢,却形同鲁莽,他们仅仅冲前了两步,刚刚平息下去的弦鸣陡然拔高,近百架重弩同时发射,从前方射过来的利矢瞬息间贯穿了他们的身体,如同被正面猛击了一拳,倒仰着飞了回来。落地之前便没了声息。

乌鲁痛苦的一声低吼,用力的将脸埋进了土里。那几人里面可有他朝夕相伴的兄弟手足。一同狩猎,一同放牧,一同征战的手足啊!却在宋人的陷阱中送了性命。

冰冷的土壤中蕴含的腥气让乌鲁逐渐清醒过来。

那是彻头彻尾的陷阱!但只有守军在有效的指挥和充分的信心之下,才能为敌人设下这样的陷阱。

如果是初次上阵的平民甚至是士兵,有很多都会在第一时间将手上的箭矢全都射出去。哪里可能会用射击节奏的变化来欺骗敌人?被恐惧和紧张擒获的新兵,就在耳边回响的口令他们也是听不见的。

冷静,这是战阵上最难做到的一件事。

但宋人这一回做得很精彩,很漂亮。乌鲁可以肯定,现在在城头上的,必然是南朝军中的精锐。绝不会是初次上阵的新兵和刚刚被征发的平民。

乌鲁都三十岁了,自幼生活在临近北方草原的土地上,从九岁那年射杀第一个阻卜人开始,上阵杀敌已经不知多少次了。他再清楚也不过,族中许多初次上阵的儿郎,在紧张的情况下,动作会变形,行动会失误,甚至拉扯弓弦都能滑手。就如他本人,九岁的时候能射杀一名来袭的敌人,靠的是运气,而不是箭艺。城上的守军,绝对都是上过阵或是久经训练的精兵。

乌鲁埋着头,身子紧绷着,须臾也不敢放松。

大概是方才暴露了位置,射向他这个方向的箭矢比之前更多了。之前的箭矢密度与现在比起来,就像是春雨和夏末的风暴在作比较,幸好位置不差,能依靠地形来挡住大多数的箭矢。

城墙上面到底有多少人?

难道宋人在太谷城中有成千上万的士兵?!

三五千禁军厢军加上两三万逃进城中、没经历过战争的百姓——至于原本就居住在城中的坊廓户,数目并不算多,毕竟只是县城——这是之前从探马的回报中得到的判断。

虽然乌鲁并不是重要人物,但萧十三为了提高三军士气和信心,将太谷城中的军力数目都向下做了通报。

不过被箭矢压得抬不起头来的乌鲁已经明白了,那所谓的通报,根本就是胡扯。

神臂弓有多难拉开,在代州的武库中为族人争来了百来架之后,乌鲁同样也很清楚。必须坐下来靠腰力上弦的重弩,那不是小孩子用的玩具弓,都不用喘上一口气就能拉开来一次的。速度只要稍快一点,十次八次腰力就跟不上了。正常使用神臂弓,都是射上一箭后,歇上一阵,才会再射上一箭,需要保持着稳定徐缓的节奏。

但现在神臂弓的发射速度竟然比拉弓还要快。如果眼前的整座县城的每一段都能射出如此密集而稳定的骤风急雨,要说城中守军不到三万人,乌鲁是绝对不可能去相信的。

也就是说……萧十三那个贱种又在说谎了!

……………………

韩冈在城楼上拿着千里镜望着城外,借助着熊熊火光,可以清楚的看见来袭的辽兵被城上的箭雨完全给压制住了,甚至连城池都无法接近。

不过由此消耗的箭矢,也是个巨大的数字,完完全全的是用钱砸人,每时每刻都是几十贯上百贯的砸了出去。

“箭矢还够不够?”韩冈问道,眼睛没有离开千里镜。

“库中还有六十三万支。”陈丰应声答道,“其中破甲矢十四万。”

这几天来,陈丰对数字很敏感,可能是商贾家庭出身的缘故,钱粮计算上很有些水平。人都有长处,陈丰的这项长处也让他在韩冈的幕府中站稳了脚跟。

“暂时用不着。”韩冈摇摇头。

真正在城上拿着神臂弓射击的士兵不足三千,之前是将箭矢以一人四束分发下去。神臂弓配用的木羽矢以三十为一束,也就是说每人一百二十支,总计接近三十五万,就算以现在的射击速度,也足够使用了。

“不过节奏要把握好。”韩冈轻声吩咐。用箭矢压制敌人,不是只管闷头乱射,合理的节奏才能将箭矢的作用发挥到最大。

负责军事的黄裳回话道:“枢密的吩咐之前就传下去了,下面指挥射击的都头、都副,皆明白该如何做。”

韩冈点点头,类似的内容他之前就说过,在这两日的训练中都传达了下去。真要是能做到,这一战基本上就不会悬念了。

高昂的士气,合理的指挥,再加上充沛的军需,那就不用担心还会有失败的危险。

尤其是军资供给,区区三千士卒,就拥有多达二十具的床子弩,十倍于人数的重弩,两倍的铁甲,以及数量庞大的简易上弦器和畜力上弦机,如此充沛的军资供给,这在大宋周边诸国,甚至包括辽国在内,都是难以想象的数字。

箭矢这样的消耗品,就多达百万。大宋恐怖的国力使得周瑜刁难诸葛亮的难题都不再成为将帅们头疼的问题。京城的军器监不用说,一年之内轻轻松松就能让二十万人武装到牙齿。在边州和要郡,也都设有制作箭矢、弓弩的弓弩院。任何一座边州弓弩院,只要全力打造,各色箭矢一年百万也不成问题。而在边境的任何一场会战,后方都能轻而易举的调集来数以十万计的箭矢以供使用。

不过正常情况下并不代表太谷县这样的县城也能拥有如此数量的军器。只是在韩冈决定以太谷县作为决战之地后,大批来自京城的支援在抵达太原之前,就给韩冈直接在太谷县断了下来。

太原城中的军械库中有多少军器,韩冈比任何人都清楚。他离开河东的时间并不长,钱仓、粮仓或许会有很大的变化,但军器的库存在没有战争的情况下是不可能有太大的变动。

对于之后补充的军械箭矢,太原城并不是那么急缺。太原守军最急缺的信心,韩冈已经给了他们。还引走了围困太原的敌军,作为交换,借用一批军需自然没有任何问题。何况在辽军突破石岭关,进入太原府界后,再往太原运送支援物资,是肉包子打狗,是往漩涡中开船。

韩冈手中的千里镜小小的移动了一个角度,指向了远处的辽军兵营,萧十三应该就在那里,同样正在观察着战况。现在城头上的狂风暴雨,是大宋国力的体现,是宋辽最大的差距所在,看到了这样的战况,不知道大辽国的枢密使,此时是作何想?

也该认命了!
