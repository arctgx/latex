\section{第36章 万众袭远似火焚(四)}

刘源现在还会偶尔想起渭源追敌的那一夜,不仅仅是在清醒的时候。

就算时间过去了差不多有半年,他在睡梦中仍不时的会梦到率领麾下精骑冲入敌军阵营中的场面。

如同饿狼冲入羊群,追赶着不敢反抗的敌人,把长枪捅进他们的后背。

长枪不知挑过了多少人的性命,枪尖上凝聚的血腥,浓得就像整个人浸泡在血海之中。

刘源只觉得杀戮得从未如此恣意,成百上千的蕃人奔逃着,被他麾下的军队毫不容情的驱赶起来。

结吴延征在混乱中不知是谁人所杀,但瞎吴叱的那条胳膊,刘源依稀记得他曾纵马踏过许多落马的蕃军士兵。前一次见到瞎吴叱的时候,只剩一条胳膊的新晋熙州刺史,还拿眼睛瞪着自己。

那种敢怒不敢言的眼神,一直留在刘源的记忆里,想起就觉得痛快。

刘源浑家起身的声音,把刘源从梦中吵醒,变得半睡半醒的时候,不知不觉又想起被流放到河湟之地的那一天。

作为最后一批被流放到河湟的叛军罪囚,上千男女老少拖着脚,经过了漫长的跋涉,才终于抵达了他们的目的地。

那一天的天气很不好。

雨水很大,刘源还记得自己当时上上下下都沾满了泥浆,所有人都像是从泥地里爬出来的。即便是天气已经转暖,浑身肮脏的淋着雨,也一样容易生病。

每一个人都惶惶不安,但当时的缘边安抚司、如今的熙河经略司做得不错,一口热汤就让所有人放下心来。

他们被安顿在陇西城外只有一里地的一处由营地改建的村寨,周围是保护营垒的高墙,抬头是更为高耸的陇西城城墙。刘源知道,在那道城墙之上,有着一对对警惕的眼神。只要他们这群流囚预备在寨子中闹出点事来,转头过来,城中的骑兵就能堵上村寨门口。

不过这事也忍了,其实是两头害怕。陇西城里的官人们也害怕再把他们这群罪囚给逼反了。要缴的租税都按着正牌子的乡兵弓箭手来。分下来的田地有三成是已经开垦好了的,地里的麦苗都长了及膝了,

因为是主持此事的缘故,韩冈这个小官人,刘源跟他很熟悉。而之前韩冈去咸阳城中招降的时候,刘源还与他打过照面。看起来很和气,因为救了广锐军几千人的性命,加上又是主管军中医疗,人缘更是好的无以复加。他们这群叛军,几乎都要给他立长生牌位了。

而韩冈的父亲韩千六——韩谦益这个官场上用的大号,私下里也没人这么叫他——刘源也见过好几次。都是因为他们这群在军中混到老的军汉不会种地,收拾不好庄稼里的事情——他们做庄家的时候经常有,种庄稼的时候,却从来没有过——韩千六才每隔几日,就带着屯田所的官吏,来指点他们如何料理田地。

换在过去,对于面朝黄土背天的农夫,刘源他们这些军头正眼也不会看一眼,不屑一顾。但一次次跟在韩千六身后,刘源也不得不承认种地的学问的确不简单,绝不是松土播种、浇水施肥那么几条。

可能是因为韩千六性格和善的关系,在他的影响下,其他人投向刘源他们的视线,并不再是看叛贼的眼神,说话和和气气,也没人把他们在农事上的笨拙当作笑话来看待。

但亲自下地耕作,还是很麻烦,总比不上一弓一刀的挣口饭吃容易。

半睡半醒的任凭神飞天外,一声鸡鸣霍然响起,喔喔喔的带动全村的公鸡都跟着叫了起来。刘源先是捂着耳朵,翻了几下身子,见实在挡不住鸡鸣入耳,不得已皱着眉头从床上起来。听惯了营中的鼓号,总是在晨钟中起身,被嘈耳尖利的鸡叫唤起,总是一肚子的火气,更是莫名其妙的浑身发毛。

支开窗棱,看看屋外的天色,依然还是黑沉沉。从窗缝中传进了鸡叫声,更为猛烈的蹂躏起刘源的耳朵。

睡在身边的浑家现在大概是在厨房里忙着,刘源披着衣服,走出房门。家里养的一只报晓公鸡就站在栅栏上,鬼哭狼嚎的叫着。

“叫个鸟……今天就炖了你。”刘源撒气似的抬脚踢出脚边的一块石子,擦着公鸡尾巴飞了出去。

才一岁不到的公鸡扑楞楞的飞到另一根木桩上,歪着脖子盯着刘源。

“这扁毛畜生!”

刘源的下床气很大,又挑起一颗石子,抬手就要丢过去。

“这么大人了,跟鸡撒什么气?”一个苍老的声音叫住了刘源。

刘源连忙回身行礼:“爹。”

一个六十上下的老头子从西厢中走出来,看着儿子,摇摇头叹了口气。

原来刘源还有一个小妾,加上两个家仆,在出事后就遣出去了,跟着自己到河湟这里,也就父母妻儿了。

刘源一时糊涂,拖累了家人。但家里面对此却都没什么抱怨,浑家还是温柔贤淑,父母也是笑呵呵乐观得很。不像有的兄弟家里,因为被连累到流放边陲,家中人都不待见了,说话的声音都小了三分。甚至也有娶了个让人不省心妻室,闹到衙门中要判和离的。看到他们,让刘源觉得自己真是幸运无比。

就是两个儿子的前程让人烦心。刘源也没指望让他们现在就能从军做官。不管再如何努力的流血流汗,不管朝廷已经下旨把他们的过往罪孽用功劳都抵消了。但身为叛贼家的儿子,就算能从军,也不过是送死的份,至少要等到孙子辈。但眼下可以出外行走,而不用担心被人拘束,这一点,就让刘源很满意了。

“爹!”“爹!”

正想着儿子的事,两个小子也从东厢的房间里钻出来了。

“怎么这么早就起来了?”少年人贪睡,两个小子起得如此早,刘源都觉得奇怪。

刘源的大儿子摆了个架势:“早起要习武啊!塾里的先生说了这叫闻鸡起舞。赶明儿从军,再上阵挣个功劳回来。”

“挣个屁!要拼命,你爹我去拼。你们先正经把地种好,再跟着先生多识两个字。这辈子别想当官的事,到你们儿子辈还差不多。”

刘源骂了两句,训得两个小子失落得回了房去。

他才四十不到,两个儿子一个十二,一个十四,都还没有成年。旧年定下的亲事,给老大找的是邠州城里的商户,现在已经黄了。老二的则是刘源在广锐军中兄弟家的女儿,眼下就同在一个村寨中,婚约依然还在。看起来日后自家的大儿媳妇,也只能在本村中找了。

心情不好,胡乱吃了点东西,刘源就往出门校场中走。看到前面一个也往校场去的高瘦背影,正是他现在的邻居,过去的广锐军都头胡千里,刘源连忙叫住他,“胡四!”

胡千里闻声回头:“刘指挥……你今天起得早啊。”

“被只瘟鸡吵昏头了,睡不着,干脆起来。”

说着话,两人就到了校场上。村中最大的一片空场,叫做晒谷场其实更好,但村里人还是都习惯性的称为校场。同样也是过去在军营中的习惯,不需要点卯的时候,刘源这样的将校起床后就往校场走,打熬筋骨的事,一天都耽搁不得。

校场走,此时已经聚满了老老小小的军汉。各自拿着兵器呼呼的挥舞着,或者干脆练着拳脚套路。见到刘源到了,各自上前打个招呼,也有人诧异他为何能早起,刘源随口就把责任丢到了家里的报晓鸡身上。

走到自己习惯的角落,亮起随身携来的一杆长枪,双手用力一一振,就是几十朵枪花,如梨花瑞雪,绕身纷纷而落。

胡千里看着啧啧称叹:“以刘指挥你的枪术,在这一片地,也算是得上拔尖了。要不是因为一个叛字,凭着在渭源的功劳,老大名头早就挣下了。”

“叛贼都当过了,还争个屁名头。”刘源将手上的长枪又转了两圈,带起了一阵啸声。还是很不满意,“究竟还是不如吴都虞的铁枪。”

“吴都虞到底还在不在?”胡千里看看左右,凑近了压低声音道:“都说那具尸首是假的。”

丢下长枪,从一旁的架子上提起一柄重斧。甩手挥了两下,带出的风声把胡千里吓得连退了几步。刘源狠声道,“管他真的假的,过去受的恩情,前面都还清了。若他再敢出现在我们面前,就拿他的脑袋去抵数。”

胡千里见刘源口气说得狠厉,忙扯开话题:“听说马上就要大战了,不知道会不会把我们再征召起来。”

“没征召,该做什么做什么。有了征召,那么上阵就是。”重斧随着手腕转了两转,掠起的浮光如电,“挣不了军功,日他鸟的挣钱就是了,看谁敢克扣我们的赏钱?!就像去年在渭源打得那一仗,各自赏了几十亩地、十几贯银钱,其实也不差了。”

刘源正跟胡千里说着,一名骑兵出现在校场外。

“刘保正可在?”骑手高声叫着刘源,“奉经略司韩机宜之命,征召承恩村保丁随军应役!”

