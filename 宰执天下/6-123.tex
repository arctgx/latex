\section{第12章 锋芒早现意已彰(四)}

政事堂的堂官来到韩府的时候,韩冈正在家里给老大韩钟安排的独居小院中。

早上的时候,韩冈答应了教两个射箭。结果除了最小的两个,其他几个韩家的儿子,都拿着弓箭到家里的小校场上来练习。

前段时间,他们在王安石那边住了有一个月,王安石和吴氏这对老夫妻高兴了,可韩家儿子们的武艺却生疏了不少。

王雱留下的独生子王栴身体一向不好,王栴与韩钟的年纪相当,如今在王安石的教导下,连诗经都开始学了,但他就是学不来武艺,连带着王旁的两个儿子也没能习武。

所以说到身体,王家的表兄弟就不能跟韩家的子弟比了。弓马拳脚,韩家的孩子自小天天练,夏天就跳到后院的荷塘里游泳,韩冈只会鼓励,从来不阻拦。而王安石和吴氏,哪里可能让孙子脱光了下水游泳?以王栴的身体,夏天受点风也要咳许久的。

但在王家也有一个好处,文事上有人督促,比在家里的时候进益更多。就是老大是中人之姿,比王栴逊色了不少,读书不如人,作诗对对子也都差了些,家里的老二韩钲虽更早慧一点,可因为韩冈不重诗词歌赋,也没能超过王栴多少。

韩冈倒是不在意,他更在乎家里孩子们的健康。只有健全的身体,才有健全的精神,这是韩冈一向信奉的圭臬。至于诗词歌赋,太浪费时间和精力了,没必要去学。

看着儿子们张弓搭箭练了一番之后,趁着离中午还有一段时间,韩冈就在王旖的提议下,来到刚刚整修完毕的小院中。

在普通人家,只要还算殷实,当儿子大了,多少也会隔间房出来。而在官宦世家,那就是一个单独的院落。

男孩子年纪大了,就不适合在后院和父母一起住了。而且每个小孩子都会想要一个属于自己的空间。当听到自己能有一个独居的小院,韩钟兴奋了很久,当院子开始整修装潢,他更是每天都要来看一看进展。

而看见一直玩在一起的哥哥可以一个人住了,韩钲也闹着要自己的小院,不过他的抗议旋即被王旖镇压下去了,只能用羡慕的眼神看着自己的哥哥。

韩钟的小院,前后两进,位于府中一角。

进院之后的照壁,正面是一片素壁,背面则是用方瓷砖镶嵌出来的山水画,而正反两面的石基,则雕刻了凿壁偷光、囊萤映雪、悬梁刺股之类古代苦读士子的图画。

房子刚刚粉刷过,粉墙素壁,看上去分外整洁。梁柱皆是新漆过,玻璃窗被擦得透亮,屋顶上的瓦也都是换了新的。房中的地板给磨得光可鉴人,而院子里面,除了两侧的花坛和两株海棠外,皆是以小方砖铺地,整体平整而微带凸起,走在上面不虞滑倒。

韩钟本人起居的卧室、书房和客厅中,桌椅床榻等家具用的都是檀木,连文房四宝都是御赐的上品。书架上堆满了经史子集,有数千卷之多,皆是出自于国子监的印书坊。多宝格上的摆设虽不多,也各个都是精品,唐时的三彩瓶和漆器,汉代的铜器,秦时的剑和戈,非是满目金玉的灼眼,而是素雅中透着逼人的富贵。

在书房的旁边还有一间单独的实验室,有望远镜、显微镜、司南,一整套的玻璃实验仪器,以及一个小小的天球仪。这就是气学宗师的儿子,所能享受到的特权。

把手中的一方洮河砚放下,再看看挂在壁上的一幅巨然的山水,韩冈摇摇头,回头对王旖笑道:“娘子辛苦了。”

韩冈知道王旖怕被人说闲话,才在屋里堆满了古玩珍器。要是她亲生的韩钲也一起出外居住,肯定不会这么奢华。太用心了,其实反而能看到心中的在意。

“官人觉得如何?”

平日很豁达的王旖,询问韩冈的感觉时,却难得的有点紧张。

“素心,你看呢?”韩冈反问同行的素心。

“是不是太奢侈了?”

严素心小声的问着,东西是好,可她觉得韩冈的书房都没这么好的笔墨纸砚。

“贵重也好,便宜也好,都当寻常器物来用就是。”王旖笑道,“其实隔壁的小库房中另外还有一套日常用的器物,若是担心不慎损坏,就将那些换过来便是,这里的就收到库房里……官人,你看呢?”

“让大哥儿自己看着办,这是他的住处。不过……”韩冈交了一名亲随过来,“将我书房里的那一幅字拿来。”

韩冈有内外两个书房,只是男仆是进不了内院,更不用说里面的书房了。但外书房中所收藏和张挂的字画不止一幅,亲随犹豫着问韩冈,“参政,是哪一幅?”

“挂在对着门的那面墙上的。”

“文诚先生亲笔的‘君子不器’?!”

还没等亲随回话,王旖就惊叫了起来。

“就是那一幅。”韩冈点头,又吩咐着,“顺便把放在桌子左手边的两幅字也拿来。”

“官人,文成先生的字就太贵重了。”严素心连忙道。

这的确贵重。

张载留给韩冈的纪念品并不多。除去信笺和手稿之外,就更少了。张载亲书的笔筒,一直放在韩冈的书桌上,而他临终前留给韩冈的一幅字,更是被韩冈张挂在自己的书房中。

王旖和严素心完全没想到韩冈会将这幅字给了韩钟。

“没关系,大哥儿是家中长子,行事是弟弟们的表率,他能遵循子厚先生教导,比什么都好。字不重要,重要的是遵循教诲。”

“孩儿明白。”韩钟用力点头,十岁出头的他,已经能明白韩冈话中的意思了,“孩儿一定谨遵爹爹和文成先生的教诲,绝不违背。”

“那就好。”韩冈点头,满意的笑着。

张载亲书的‘君子不器’,张挂在了韩钟的书房中。

而另外两幅字,则是张载所著的《砭愚》、《订顽》两篇,即所谓的《东铭》、《西铭》,是韩冈亲笔所录,装裱好了,同样张挂了起来。

这是韩冈给自家儿子准备的礼物。

“院子的确是费心了。等再大一点就去横渠书院,与士人交流,还是从关西先开始。”韩冈低头对韩钟道:“你处道二叔家的大哥就在书院,日后你们兄弟得多亲近亲近。”

韩钟知道韩冈说的是谁:“是金娘的夫婿?”

虽只是定亲,但也是自家女婿了。可是韩家的准女婿才十岁出头就给王厚安排去了横渠书院,接受关中名师的教导。这一回王厚上京,还是将儿子留在了书院中,没有带出来。韩冈也没能见到。

“王家的大哥,虽比不上他的十三叔,但听闻也算是个聪慧稳重的孩子,日后当是大有前途。说不定没几年就是一榜进士了。”韩冈对妻妾们说着自己派人打探来的消息,

“若当真能如此,就真的是可喜可贺了。”王旖很高兴地笑了起来。

“官人,大哥的天资的确是差了点,其实日后还是转成武职好些。”严素心低声的说着,只让王旖和韩冈听道。

想要长保家门,转到武臣体系中,才是最稳妥的。尤其是有韩冈这个在军中人脉极广的父亲,别人要费十二分气力,只要有中人之姿,庸将的水平,也能完成名将辛辛苦苦才能实现的功业。

韩钟也不比韩钲,他是长子却不是嫡子,就算韩冈日后能封王,甚至还能让后代袭爵,那时候也是韩钲来接受,绝不是韩钟。

“现在说这些还太早。”韩冈摇头笑道。

他过去没有跟妻妾们讨论过儿子们的未来,可是时间过得飞快,一年年就是转眼的事,现在还可以拖一下,再过两年就连拖也拖不了多久了。

王旖还想再说些什么,这时外面报称政事堂来人了,韩冈出去接见,便听到了韩绛从政事堂发出的传话。

韩冈不便怠慢,与王旖等人说了一句,也没忘让王旖去安抚一下约好下棋的女儿,立刻就骑马出门。

来到政事堂,看到从边境传来的急报,韩冈苦笑着摇头:“耶律乙辛真的会抓时机。”

辽国这么大,虽然也有充任国使的无能之辈,但能打仗的还是不在少数。

耶律乙辛父子,也可算是虎父无犬子的典范。高丽、日本说打下来就打下来了,换作是官军来打,哪里有这么快的?

有了高丽和日本的收获,耶律乙辛已经夯实了物质和声望上的基础,剩下的,就是时机了。

而耶律乙辛的确是会抓时机。

“当真没错?!”韩绛皱着浓密花白的双眉。

“有五六分把握。”韩冈说道,除此之外,他想不到有别的可能,让南京道一下多出三万骑兵,并驻守在边境附近。

“既然玉昆如此说,那就没有错了。”韩绛叹息道。他和张璪其实早就做出了判断,只是想让韩冈加以确定。

“忍了这么多年,终于忍不住了。”张璪冷笑道。

韩冈摇头,“是时机到了……篡位的!”
