\section{第21章 欲寻佳木归圣众(11)}

赶在皇城城门开启前一刻,称病多日的苏颂匆匆赶到了。

为了避暑而称病了多日,苏颂今日的精神状态很不错。

韩冈看得心中一阵堵得慌。自从开战之后,两府的事务陡然增多,自家这段时间累得瘦了一圈,苏颂倒是将养得面色红润,人也胖了两分。

苏颂走过来的时候,韩冈没好气的问:“丞相安乐否?”

苏颂回道:“能者多劳。”

韩冈微微愣了一下,似乎有种既视感,好像以前有过这种对话。

“听说今天会有好戏看。”苏颂低声笑问。

韩冈颇有几分惊讶,苏颂开玩笑的时候可不多见,“只有猴儿戏看,子容兄看不看?”

“……真不想看,”苏颂沉默了一阵后说道,“台谏之中尽是此辈,吾等之过。”

“御史之任,本与宰相无关。何况能如三舍人者,世间又有几人?”

当年苏颂正做着中书舍人的时候,与同僚宋敏求、李大临共同拒绝起草李定迁任监察御史里行的诏书,缴还词头,最后被天子一起罢去,这是一场严重的政治事冇件,也是旧党对抗新党的过程中一次巨大的挫败。尽管事后苏颂等三人被旧党宣扬为三舍人,但旧党在中书中势力又缩减了许多。

苏颂扯了扯嘴角,韩冈这句马屁拍得可让他不舒服。

事实证明,他们当初的争辩,完全是一个错误,给人当枪使了。而且三舍人是三舍人,御史则是御史。中书舍人能缴还词头,能驳回诏书,可以约束天子,而御史则是天子克制权臣的利器,否则监察御史的任命,就不会绕过两府,不给宰相和枢密使荐举权,两者根本不好类比。

“御史台三番两次螳冇臂当冇车,玉昆你是不是厌了?”苏颂转移话题。

“总得让人说话才是,不让人当面说话,就会背后坏事了。两相比较,让人说上几句那还好些。”

“真是胸有成竹了。”

“非是韩冈有把握。有两条铁路为沈括做保,螳螂也罢、乌鸦也罢,都挡不住碾过来的车轮。”

‘历史的车轮吗?’苏颂会心一笑。

《九域游记》中的词汇,虽多无典故,语出不经,但如今当真是流传开了,时常能听到有人嘴里蹦出一两个来。

今日御史们敢在文德殿上发难,只是认为王中正的受命是太后的表态。而太后究竟是个什么态度,虽然苏颂也想也知道,但他更清楚,沈括的位置是靠实打实的功绩做出来的,即是做不成宰辅,也照样是朝堂重臣,轨道工役暂时还离不开他这个熟手。而御史们,若依然按照过去的惯例来行事,下场绝不会好到哪里去。

炮声响起,城门缓缓打开,新的一天,终于开始了。

韩冈扫了一眼城门洞前的几名御史,还有居于人后的御史中丞舒亶,对苏颂道:“冇该进去了。”

该进去了,韩冈回首,冲依然紧张的沈括点点头,与苏颂一同走进门中。

……………………

龚原在文德殿的殿角站定,握紧了手中的笏板。

今天的目的,并不是要掀翻韩冈,甚至阻击沈括的就任,他也不是那么坚持,龚原只想要让太后和皇帝记住自己,而是要扩大自己的声名,在新党之中,也能得到更好的认同。

韩冈力挺沈括,就是一个错误。而以韩冈的性格,也为了自己的威望,在御史们的反对声中,只会一错到底——这可不是李南公的三司使,是要铁路修造的主持者的人选,韩冈计划中最重要的一环,那位权相绝不会就此妥协。

不论出于是公心,还是私欲,打击沈括这个韩冈,都是一本万利的一桩事。

苏颂率领一众在文德殿上向天子和太后拜礼,一应的朝仪之后,朝堂中的气氛陡然紧绷了起来,太后的发话却让这个气氛为之稍缓,“太皇太后于今病重,吾当辍朝,为太皇太后祈福。从明日起,辍朝五日。苏相公、韩相公,请二位率诸位卿家去大相国寺为太皇太后祈福。”

说是辍朝,需要太后处理的要务还是会按时送到她的面前,只是没有每天早上的繁文缛节。

没有人出来反对,这只是很小的一件事而已,而且太后更不是出于对太皇太后的孝心,不论是辍朝,还是群臣祈福,只是不得不如此走个形式罢了。但苏颂还是领头出来,赞美太后的一片纯孝。又与韩冈一起,接下了去大相国寺的任务。

龚原屏住了呼吸,他对辍朝并不关心,下面就该是众所期待的廷推了。只要这一次能够成功,太后辍朝多少时间都无关紧要。

紧了紧手中的笏板,将汗湿的手掌擦了又擦,龚原越发的紧张起来,事到临头,这最后一步竟然如此难以他出去。但当他的视线掠过对面的文臣,定格在沈括的身上,他的身子终于停止了抖动。

…………………………

韩冈小小的挪了一下脚步,让自己的视野能够囊括边角处的御史们。赞美过太后的孝心,群臣回到班列中,今日最重要的一项议题,就要开始了。御史台如果要发难,差不多是时候了。

韩冈在御史台中没有怎么插手,他一向是认为做实事,比动嘴皮子更重要。尽管御史台地位很关键,但他夹袋中的人,基本上都是在做事的差遣上。

一个职司的地位高下,并不是固定的。比如枢密使,一开始只是天子近臣之任,如今却能与宰相分庭抗礼。又比如侍中之名,原本是宰相之吏,之后却变成了与宰相掌握的外廷对立的内廷官职,再后来,又一转变成了宰相之职。

监察御史一直都是天子克制臣下的工具,立国以来,这个工具一直都运作的很好,虽说渐渐的有了独立性,但在压制宰辅这个基本用途上,还是表现得十分出色。

可变法以来,御史台掣肘太多,先帝赵顼为了推行新法,将御史台几番折腾,而新旧两党为了控制朝政,打击政敌,也不约而同的去争夺御史台的空缺。经过了这些年的打压,御史台的素质愈见下降,大半都是投机主义者。乌台在士林中培养出来的声望,也是这些年打着旋儿的往下落。无论是韩冈,还是章惇,都不介意在这一过程中,再推上一把。

不过现在,御史台免不了还要在挣扎一下。

不仅仅韩冈这么在等待,章惇也在期待,下面的重臣、朝臣也都在期待着,

一场好戏,或是某些人眼里的一场猴戏。

瞪大眼睛,迫不及待。

一如包括一众宰辅在内的朝臣们所预料,当廷推开始,沈括的名字第一个被提出来之后,御史台首先发难了。

“沈括才干卓异,名著朝野,提举铁路工役,尽显其才,已无需赘言。论功论才,皆不让人。臣举沈括,为两府备选。”

王居卿的话声刚落,文德殿的角落处,立刻一声嘶声力竭的大喝:“陛下,沈括不可入选!沈括万万不可入选!”

龚原大步向前,前方正依班列恭立的朝臣,如同被分开的海水,给他让出了一条道来。

龚原的步子略大,又急又快,脚步声啪啪作响,转眼便走到了殿中冇央,

躬身一揖到地:“陛冇下,臣监察御史龚原有本奏。依故事,受御史所劾,纵宰辅亦得退避,以待裁断。沈括过犯,难以尽书,如今御史多有弹章呈于陛下,岂能容其安坐于朝堂上?”

韩冈立刻成为殿中数百道视线关注的焦点、沈括反倒没有收到多少的注意,纵有,也只是一晃而过。

谁都知道,这一次的弹劾,针对的到底是谁?

韩冈没有动,只是表情上看,是胸有成竹的模样。投向韩冈的目光立刻充满了疑惑,难道他不打算自己出头,而是安排了别人出来反驳?

王居卿、蒲宗孟,还是状元郎?

又或是别的韩党成员?

“那些弹章吾不是都留中了吗?怎么还来说?!”

一声呵斥,从大殿的正北方传来,带着浓浓的不满,让群臣心惊肉跳,让龚原脸色苍白。

是太后在说话。

“陛下……”龚原颤声。

一切听太后指示,这是一名忠臣应有的行为。太后既然表现对韩冈推荐沈括不满的意思,他们这些做臣子的当然要附和太后的心意。但刚刚放走了王中正的太后怎么会替沈括抱不平?

一时之间,鸦雀无声。原本预定要出场的御史,一个个停下了脚步,查看风色。而殿中的群臣,则是在等待着他们中间有人敢于出列抗辩。

“陛下!”第二名御史出列,赫然是新晋御史杨畏:“沈括壬人,虽小有才学,但人品实劣,不足以为辅弼。且外又有传,韩相公将诸铁路归于一衙,并欲以精兵数万护卫铁路,统掌军政刑名,由沈括执掌。数万大军于外,又有班直禁卫于内,要害皆为宰相腹心所掌,太后,须防肘腋生变!”

韩冈轻轻的啧了一下嘴。他的一番盘算,不过是稍稍漏了点口风,就这么快的传开了。不过也有可能是英雄所见略同,干脆是编造出来的。御史有风闻奏事之权,到底是听说,还是故意编造来攻击自己,韩冈并不清楚。

但韩冈想做的事,也是世间一众有心轨道的大族心中所想。令出一门,不论是找人疏通,还是插手实权,日后都方便许多。他可不在乎有人拿这事攻击自己。

但太后会是什么反应,这么多年了,韩冈也无法确定。

“吾听人读史,为什么明明有名将领军在外,却总是无法克敌制胜。国事就是给这等小人败坏的,南面的仗还没打完呢,就急着想要兔死狗烹了?别以为吾不知你们在想什么,看见王中正出去,就以为吾要查办沈括?之间怎么不见几人说?”却见太后勃然大怒,“舒亶,御史台中都是这等奸佞,你是怎么管教的?”

两上两下的李定,已经不在御史中丞的位置上了,现任的御史中丞舒亶抗声道:“不能为朝廷去贼,不能为太后辩奸,臣实有过!”
