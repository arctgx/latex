\section{第44章 秀色须待十年培(八)}

一边听着韩冈的解释,郭逵一边绕着火炮转了几圈。

脸上的怒气不知不觉间已经消失了,神情专注,变回了天下知闻的名帅的样子。

韩冈看着他伸手拍了拍炮管,然后抬头问道:“能不能试一试。”

“当然可以。”韩冈道,“不过要重新固定一下。”

他可不想再命中郭家的房子了。

郭逵点了点头,然后仔细的观察着火器局的士兵小心翼翼的调整炮管的角度。

之前放松木炮的架子早就给撤掉了,搭了装了一个滑轮组的外框,像是龙门吊一样,下面是土堆起来的炮座。想要调整角度,就需要用滑轮组吊着移动,在炮座上调整。也就是因为土堆得不结实,方才炮口才会角度偏移,让炮弹飞去了郭逵的宅邸。

只是他们小心过了头,看起来炮口冲着下面了。这是一朝被蛇咬十年怕井绳。

“这样不行。炮弹会掉出来。要平着。”

然后又是一阵忙活。好半天,终于调整好了。

郭逵的眉头皱得紧了起来:“这么麻烦?”

“之后会有支架。”

韩冈招招手,让人拿出几张图纸来。这是他命画师所绘制的火炮外形图。第一张就是韩冈记忆中的野战炮,有轮子,有炮架,旁边有炮手做对比,脚边上还有炮弹,韩冈拿着指给郭逵看。

有图就很直观了。郭逵看了两眼,就知道了带着轮子的新兵器的特点了。比起霹雳砲来的确更为适合野战,看样子就是为随军行进而设计的。

他拿着图仔细瞧,问:“这炮架够结实吗?火炮看起来不轻。”

“火炮炮管是一千一百斤,正式定型后不会差太多,正常是能撑得住。但炮弹发射的时候,会有一个很大的反冲力,所以炮架必须有足够的分量,不然会不稳。”

“两千斤?”

韩冈考虑了一下,“不会超过两千五。”

“那是要比大车轻一点。”郭逵轻轻点头。

四轮的太平大车,一般载上五六千斤不成问题,这样的马车在官道上时时可见。两千五百斤连炮身带炮架的可靠姓,当然也不会问题。

“发射呢?怎么做。”

郭逵发问,韩冈就比了个手势,这一回章惇不再上去点火了,几名士兵清膛、装药、装弹、装引线,点火,最后一声巨响,青烟腾起。郭逵耳朵嗡嗡的响了一阵,好半天才定了神下来。抬头见韩冈和章惇都望着自己,顿时就有些羞恼,勉强笑道:“真跟打雷差不多,年节时的爆竹可比不上。”

扯了一句,郭逵扭头看那标靶,心中的情绪转眼就不翼而飞。

三十步的距离上,炮弹竟然硬生生的打穿了三重木板,深深的陷入了院墙边的沙土堆中。

三层近一寸厚的松木板,里面夹着石棉,竟然一下都没吃住。郭逵摸着木板上的洞口,感觉就是包铁的城门,估计也挨不了几下。而且这发射比霹雳砲和八牛弩简单太多了。

郭逵是老行伍,这样武器放在他手中,能变出几十种花样,新战术当场就在脑中一个个蹦出来。这时候,郭逵倒是有些佩服韩冈了,亏他想得出。

射程能达到七八百步的距离,而且比八牛弩重新上弦要容易。一旦官军装备了火炮,辽国想靠近官军的军阵都难了。

不论是骑兵还是步卒,在出击前都要聚集,就是契丹人也不可能避免。难道要契丹人骑着上阵的战马在两三里地集合,然后冲击宋军的阵前?

速度快如赛马,在长距离的赛事中,经常能听说有赛马暴毙的新闻。就是普通的两里、三里疾奔下来,赛马也都会满身是汗。若骑在上面的是全副武装、连人带装具有两百斤的骑手,再好的马匹都吃不消两三里的狂奔。

这就是火炮的好处。直接在极远处就打散敌军的集结,打乱敌军的攻击节奏。其实类似的事,床子弩也能做到,霹雳砲也勉强可以。但火炮能做到随军野战,床子弩、霹雳砲可做不到。

霹雳砲,围城时才能派上用场。八牛弩,都只是在城池的攻防战中安身落户。而火炮,适用姓就宽广了许多。比起霹雳砲和八牛弩,更要灵活轻便。

这还只是试作品,现在就有着媲美霹雳砲和八牛弩的威力,当改进完成,可能要将这两样过去的国之重器,远远地抛到后面去了。

郭逵转过来看韩冈,好半天才叹道:“今曰一见,方知宣徽所言不虚。”

韩冈谦逊的笑了一笑,正要说话,郭逵的注意力却被韩冈手中的第二张图纸吸引住了。

野战炮的图纸在郭逵手中,而韩冈手上的图纸,郭逵本来以为是绘着同样的火炮,但从卷起的图纸露出的那一部分看,却完全不一样。

“那是什么?”郭逵立刻问。

“前面是野战炮,这一张是步人炮。”韩冈将图纸展开来,的确不同于野战炮,没有轮子,只在前面有个两脚叉开的支架。

他指着图上的作为标识的炮手,可以看得出来这种炮很轻巧,不到画在旁边的炮手的膝盖,其实就是后世虎蹲炮,“重量不超过四十斤,两人轮换背负,或是用马驼就可以,一个都至少能带上两门。”

“一个都两门。”郭逵双眼一亮,考虑了一下,问韩冈,“一伍照顾一门够不够?”

“足够了。”

一伍合当五人,但由于空饷的缘故,基本上四人为多,四个人照顾一门虎蹲炮,也是绰绰有余。

“与弩弓比怎么样?炮弹射程能达多远。”

“虎蹲炮若用炮弹,当能及百步。不过一般不用炮弹,而是用铅子的霰弹。十步之内,可比得上十张劲弩。重新发射也不会比弩弓慢。”韩冈见郭逵皱着眉,情知他没见识过霰弹,让人端了一盘铅子过来,道,“待会儿请太尉看一下,就知道了这铅子的威力了。”

郭逵抓了几颗细小的弹丸,掂了一掂,知道韩冈所言不虚。霹雳砲也有类似的砲石,用布袋装一包石子投出去,落地后能让十几步内的敌军都哭爹喊娘。

“怎么不先造?”这虎蹲炮看起来就很简单,应该先造的,而且当是大批量装备军中。

“这个得用铁制,而不是青铜。铜太贵了,用铁造,坏了不心疼。”

这样的轻型火炮只追求近距离杀伤,火药装量又少,不怕炸膛,用铁铸应当没关系。而野战炮,韩冈还是没把握,暂时还得用铜。等到技术过关,再用铸铁不迟。

章惇听着韩冈和郭逵的对话,主要都是战术上的运用,他也懒得插话。

“哦?那一门野战炮要多少钱?”

“一贯小平钱十二斤,炮管是一千一百斤,消耗的物料也就一百贯,在配料上纵有些参差,也不会超过一百五十贯。炮架是铁和木头,材料最多五十贯。剩下的就是人工和制造的花费了。”

“那也不贵啊!”郭逵甚至有些惊喜。

物料成本才两百贯,加上人工和制作上的开销能有一千贯吗?就是一千贯一门,对大宋来说真不是什么问题。从今年开始,西北那边的军费开支要打个大折扣,一千万也许不到,五六百万贯总是有的。能造多少火炮,只看工匠们能不能来得及,还有铜料能挤出多少了。

“是不贵。”

火炮的制造成本,以及之后的维护,都要比床子弩便宜许多。取代床子弩是必然。

便宜又好用,这样的兵器,没有谁不喜欢吧……除了敌人。

郭逵心潮澎湃了一阵,用先进的武器蹂躏敌人,那真是无比的快乐。

“第三张图上又是什么?”郭逵有些迫不及待。

“是城防炮。”韩冈展示给郭逵看。

这一张图上的火炮,按照与炮手的对比,要比野战炮要大一圈,看着就是分量十足。火炮的环境是在室内,通过细小的窗口对外射击。

看主图下面的附图,城防炮并不是装设在城墙顶上,而是城墙外壁向外突出几个堡垒,火炮设在堡垒中,分三层布置。最底一层接近地面,最高的一层则只比城墙顶端低上一点。

“这是守城用的火炮。不用随军行动,所以可以重一点,威力和射程都超过野战炮,正好可以克制住。曰后就是辽贼偷学了火炮,也不用怕他们能攻城。”

郭逵冷淡的瞥了韩冈两眼。

这话上殿骗骗太上皇后和蔡确这等外行人差不多。以城池的高度就是普通的野战炮也能拥有超过敌军的射程。而且以辽国的工匠水平,同样形制的火炮,能有大宋七成水平就不错了。所谓克制野战炮,不过是安人心罢了。

“在东京城、大名府安几十门就可以高枕无忧了,对吧?”他冷笑着问道。

韩冈和章惇交换了一个眼色,果然是瞒不过方家,

“也不仅仅如此。”韩冈说道。

“当然。”郭逵道,“这岂是能局限在守城上,而应该是放在进筑的寨堡中才对。”

“只要布置得当,火炮和弹药没问题,十倍的贼军都别想攻进来。”

郭逵自得的点着头。接着问道:“方才飞到寒家的炮弹有十多斤吧?”

“十一斤左右。径圆四寸。”

“这城防炮最大能造多大,八寸、一尺?”郭逵问韩冈。

韩冈皱起眉:“八寸口径的话,炮弹就快要有一百斤了。”

十多斤就能拆屋,一百斤大概就能将城给拆了。

直径四寸的圆形炮弹,在十一斤上下,百斤就是九倍。质地不变的情况下,重量与半径的立方成正比。简单点说,炮弹的半径加倍再多一点就行了。大约八寸出头的炮弹差不多就是百斤了。

“怎么可能那么重?”“百斤的炮能造出来?”

章惇和郭逵同时发问。

“的确就是那么重。要造也容易。”

后世有着十五寸十六寸的重炮。但那时候用在火炮上的长度单位,与这个时代的寸,在实际长度上肯定不会一样。而且那是战列舰的主炮,这个时代哪里能做得到?

而且超重型火炮能用得到的地方实在不多,四寸炮能跟着大军前进,但八寸炮呢?实在是太夸张了,火炮的重量不知要翻几倍才够。明显的无法随军在路上行动,就是用船装载或是守城的时候都勉强。

如果在船上的话,百斤的炮弹可是要让船员发疯了,有几个能抱得动?而要在船舱中安装借力的滑轮组,时间一长,船板可不一定能吃得消。

但还是有办法,韩冈记忆中有一种可能是专门用来攻城的臼炮,口径很大,炮身却很短,炮弹的射击角度是绝大的弧形,标准的曲射炮,重型的炮弹能轻易的发射出去,就是射程比较短。

这就是韩冈的第四张图。

