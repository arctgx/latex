\section{第三章 时移机转关百虑(八)}

从庆寿宫出来,赵颢已经是满身疲惫,在皇兄和祖母的身边,小半个时辰就像是过了一年半载,不过现在脚步则是轻快了许多。

祖母终于要死了。

想到皇兄赵顼在床边垂泪的样子,赵颢刻意表露在眼中眉间的沉痛和忧虑,几乎就要保持不下去了,笑容也快要浮起在嘴角。

掌控后宫的很快就只剩最疼爱自己的母亲,而不再有一个花甲之龄、曾经垂帘听政过的老祖母压在头上。

赵颢跟着两名内侍一齐往太后所居的保慈宫过去,心情是前所未有的轻松。

他每次入宫拜见名义上的祖母时,都是备受煎熬。太皇太后只看重长兄赵顼,对赵颢和他的弟弟并不假以辞色。

就是如今病恹恹的快要咽气,但从祖母眼中射出来的那种拒绝、猜忌甚至厌弃的视线,依然给着赵颢极大的压力。

幸好不会看到她太久了,病得只剩骨头了,哪里还有治好的可能。过不了两个月,就该上仙了。去天上陪仁宗皇帝,应该是皆大欢喜的一件事。

行走在阑柱相连的廊道上,赵颢游目四顾。

自从几年前被设计逐出宫之后,赵颢一年也难得入宫几次,但每次看到一座座或雄伟、或华丽、或精致的殿宇楼阁,他总会有些只敢在午夜梦回时方敢升起的念头从心底泛起,但转眼间就给他压了下去,不敢去多想。

没在庆寿宫中看到自家的兄长,不知是不是在保慈宫中。不过就是没在母亲那里看到自家的兄长,赵颢也不奇怪。

这个新年,天子、朝堂,乃至京城的百万军民,心思全都放在即将展开的战争之上。

上天赐予的良机,几乎是每一个人都认为该把握住。

虽说其中还有点杂音,可自己的兄长、当今的大宋天子,从治平四年年初登基,至今已有十二载。对于朝堂的控制越发得严密,威福自用几近刚愎,且从开始推行新法时起,就是一意孤行的性子,一旦他下了决定,眼下是谁劝都没用了。何况反对者寥寥无几。

但赵颢却觉得这次不一定能成功,毕竟反对者中有如今朝堂上仅存的两名知兵的重臣,一个郭逵,一个韩冈。一名宿将,一名新锐,都在主张慎重。如果情况正如两人所料,那可就有意思了。

赵颢嘴角扯动了一下,满是自嘲。他现在也只能这么去想,否则就是绝望。

当今天子自登基以来,大宋官军几乎就没怎么败过。就是熙宁四年横山之役,也是非战之罪,斩首数千,虽败犹胜。

纯以武功论,开国以来六代天子中,赵颢的兄长就仅次于太祖、太宗,彻底压倒了真宗、仁宗和英宗。

王师连番胜绩,让赵顼声望大涨。同时也使得旧党对于新法的攻击,不得不偃旗息鼓。

眼下更是有一举解决西北百年之患的机会,甚至之后还有可能连辽国都压倒,废除澶渊之盟,夺回燕云。

一旦给兄长做成了此事,登时就是盛世之主,别说旧党得俯首帖耳,就是自己这个做兄弟的,也得战战兢兢的活着了。对于威胁皇位的兄弟,对权力掌握得越发森严的皇帝,可不会有多宽容。

上天当真是不公平。赵颢从父亲赵曙成为皇储的那一天开始,就在这么想。只是出生前后的差别,让自己只能当个被养起来的亲王。

而仅仅是运气稍好,比自家早了两年出生的兄长,在过去,从没有表现出多出色的才智,身体也不好,就是心气高于常人。还是濮王府中一个不起眼的孙辈时,就想着建功立业,做了太子之后,更是到处寻找能富国强兵的良方和贤才。

但他不会用啊!登基之后,就闹得天下大乱,任用非人,整整用了十年时间,才让国家稍稍安定下来。

虽说在这些年中,王师连年胜绩,但运气的成分更多。尤其是这一次,宋辽夏三国打了上百年的仗,两国内乱,而另一国正好国势昌盛、兵精粮足的情况从来没有见过。

哪里能有这么好的运气?!分明是老天爷在偏帮。

赵颢从不认为自己比谁差,换作是自己坐在大庆殿中的那个位置上,绝不会为了变法而闹得满朝风雨,也绝不会让朝堂分裂,最后让一党独大。熙宁的十年,朝堂上大部分精力都用在争吵上了,要是能平复朝臣之乱,让他们用心于国事,三五年内,就能天下安定。

至于西夏、契丹,只要朝堂一心,名臣和衷共济,只要有如今一半的运气,平夏灭辽,哪里还有一点难度!

赵颢转着悖逆不道的心思,在前面领路的两名内侍却突然缓了下来。

“陈衍,怎么了?”赵颢神情一凛,警觉的问道。

说到太后最喜欢哪个儿子,只要看一看谁在给赵颢领路就知道了。进宫之后,赵颢照规矩先去拜见祖母,而他的母亲听到消息,就遣了身边的亲信宦官来庆寿宫殿外等候,就是三弟都没有这个待遇。

太后身边的亲信内侍陈衍回头道:“大王,外命妇正在谒见太后,不宜冲撞。不如稍稍等上一等。”

赵颢听了便放眼望过去。

前面便是太后所居住的保慈宫,眼下殿前正聚集了数百名外命妇,一个个按品大妆,身着真红大袖衣、外披霞帔,头戴花钗冠,依序入殿。

赵颢瞥了一眼之后,就远远地停下了脚步,不再上前。不论是有意还是无意,冲撞了外命妇谒见之礼,只会被御史一顿乱骂,最后坏了自家的名声。反正她们对自家母亲的拜见很快就会结束,不会耽搁太久。

正如赵颢所料,一次礼仪性的拜谒并没有耽搁他多长时间,命妇们很快就从殿中出来的。

一群外命妇早前就拜见过了太皇太后,现在拜见过太后,接下来就该去见皇后。新年的第一天,天子免了今年的大朝会后,她们的丈夫或是儿子,不再用辛苦上朝,但她们入宫探问却是免不了的。

她们依然是依序而行,但半路上却来了一名内侍,与队伍中的一名四五品装束的命妇匆匆说了几句后,便又离开了。

赵颢疑惑的多看了两眼,这群命妇是朝官之母之妻,不是宗室的亲眷,怎么跟宫里面搭上关系了。就是有关系,也不该如此明目张胆。他心里猜测着,眼睛也眯了起来,但离得远,眯起眼睛也看不清楚究竟是谁。

陈衍惯会察言观色,在赵颢耳边低声道:“是朱贤妃身边的吴白,他找上的当是龙图学士韩冈家的王氏。”

“原来如此。”赵颢点点头,不用多解释也知道他六侄儿的生母找韩冈的妻子究竟是为了什么。笑了一笑,“看来韩王氏当是常入宫了。每次入宫,都要到去后苑玉华殿一趟,还不知要多晚才能出去。”

陈衍正要陪着笑说两句,另一名由内侍回头道,“大王,宫里面的规矩严,奴婢没听说过有外命妇能在宫中久留的。”

那名内侍话刚出口,陈衍脸色就陡然一变,不敢置信的望向赵颢。

过去有个动辄留小周后三五日的太宗皇帝,要是有人传出韩冈之妻在宫里面留的时间长了,天子、韩冈和王安石的名声全都能毁掉。

赵颢皱着眉头瞥了这名内侍一眼,身材倒是高大,肤色黝黑,并没有多少阉人的阴柔气,看着倒像是武夫。

深深的盯了这名内侍,赵颢径自往前走,脸上毫无表情。

想要知道赵顼对弟弟赵颢有多猜忌,只看从入宫后,赵颢身边就没少过御药院的内侍就知道了。从庆寿宫去保慈宫,过去不知走了多少遍,又有陈衍陪着,还照样派了人来领路。

赵颢倒是无心,但这个在崇政殿中当差,在御药院中挂名的内侍却把他当成贼防着。年节时,甚至连那位六侄儿都不让自己靠近。赵颢心中恨到了极点,自己那位皇兄的身边还真的都是精细人,随便派出个人来,都是对自己如此提防,随口一句,都如临大敌。

童贯直到将赵颢送到了保慈宫,方才离开,回赵顼的寝宫福宁殿,等天子回来以便缴旨。接下去,有另外的人在保慈宫门外候着二大王。

作为御药院中挂名的内侍,童贯虽不能跟李舜举、石得一这等贴身亲信相比,也不能与在外领兵的王中正、以及师傅李宪,相提并论,但也是正当红的内侍。

对天子的心意,童贯把得很准。被皇帝派出来护送雍王去保慈宫,根本就有押送的意味在。随时都有天子身边的亲信内臣盯着,雍王即便有什么花样,都别想在宫中玩出来。

但童贯却是没想到雍王竟然还敢问一些别有用心的问题,幸好直接当面给点破了,否则日后会很麻烦。

既然自己已经点明了,谅这位二大王也不敢造次。如果在这之后,外面有什么流言蜚语,不管是不是雍王做的,立刻就能追到赵颢的头上。

童贯脚步轻快,见了天子之后,这件事还是要说一说。事君惟忠嘛,轻描淡写的提一句,天子明白就明白,不明白日后出了事也有说道。

童贯得意的轻笑,虽然这一次没机会去陕西挣功劳,但在天子身侧,何愁没有功劳可立?

