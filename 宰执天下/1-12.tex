\section{第七章 飞将庙中风波起(下)}

仿佛有极北冰原上的寒流从殿中刮过,殿中的一切动作都被瞬间冻结。

‘什么?……衙前?!’

所谓衙前,就是在衙门中奔走的吏员。只是这样的吏员有两种,一是长名衙前,他们长期把持吏职,能借着官威上下其手,是人人抢着干的好活计。但衙前差役便是另一回事,这是专门针对一等户的苦役,也是收割肥羊的用意,但凡摊上的富户,运气差的家破人亡,运气好的也要损失大半家财。

衙门里庶务繁芜,有些事都是大耗钱财,故而都想着法子转嫁到衙前身上,押运让衙前去做,看管库房也让衙前去做,只要中间有个亏空或是损耗,就要照数目描赔。这还是小的,衙前甚至还成了衙门里贪官污吏诈钱的对象,若是知情识趣,老老实实献上银钱,便能得个美差。若是少给了几文,好罢,韩冈曾听说有摊到千里迢迢向京中解银的差事,最后在东京城内待了整三年的倒霉鬼——而他所押解的银钱还不到一两【注1】!

只是衙前役一任便是一年,都是从年初当到年尾,除非衙门里突然事情多了,才临时发文摊派。现今也没听说有什么大事,最多是西夏人照往年规矩来打个秋风。没头没脑的,韩家如何会摊上这等破家的苦役?!殿中众人皆知其中必有情弊,保不准就是李癞子做的手脚。

韩千六想得明白,一拍桌案,怒道,“李癞子,你是想灭俺韩家的门不是?!用这等绝户手段!你不就是贪着俺家在的河湾边那块菜园子吗?不想让俺赎回去,占全了俺家的那块地,你家在河湾的地就能连一片了!”

“韩千六,俺这可真是冤枉了!”李癞子苦笑着摇头,说得七情上面,仿佛真是被人误会一般,“这几年,衙前役你韩家可一次都没轮到,也该到你家里。本来县中早两个月就要来提人,还是俺看在前面你家小子正病着,实在脱不开身,托了在县衙中做班头的亲家帮你分说了一番,拖累两个月。”

“你也少装模作样!”韩千六冷笑:“衙前役都是一等户充的。三哥儿一病,俺家早没了余财,田地只剩一亩半,当个四等户都是勉强,更别提三哥儿今年才十八岁,要到二十才成丁【注2】。俺家现在就俺韩千六一个丁壮,实打实的单丁户【注3】。衙前也罢,夫役也罢,哪个都摊不上俺家!”

“韩菜园,难道你不知道只逢得闰年才重造五等丁产簿,还有两个月才重造。现下在县里,你家还是有两丁的一等户!”

韩千六冷哼一声:“只要俺到衙门里报个备,不信还能硬押着俺这个单丁户充衙前?”

李癞子倒没想到韩千六这个闷葫芦竟然一切门清,愣了一阵,冷笑起来:“那也要俺这个里正为你具结作保才成!”

“你……你……”韩千六倒没想到李癞子竟然如此无耻。气愤填膺,指着李癞子的手抖个不停,说不出半句话来。他一辈子的好好先生。难得跟人红次脸,现在却被李癞子气得差点就要脑溢血。

“李癞子,都是乡里乡亲,何苦把人往绝处逼?”第一个跳起来的是韩千六的酒友刘久,他家中院子内有着一棵极高峻的古槐,乡里人称刘槐树,跟韩千六有着几十年的交情。

“唷,是刘槐树啊,你倒是会出来抱不平!”李癞子阴阳怪气的说道,“想代韩菜园说话,行呵,谁去不是去?!县中只是要人,也没说定是谁。今次县里的衙前,就由你刘槐树家出人好了。”

刘久愣了半天,以他家的身家,服一年衙前役家破人亡都是板上钉钉的,哪里敢应承。叹了口气,转头对上韩千六,“韩老哥,对不住了。”愧疚的低头坐了下去。

“还有谁想代韩家去服衙前的?”李癞子得意洋洋,视线扫过,偏殿中人人低头,竟没一个敢跟他对上眼的。

李癞子这下更为得意,“韩老哥啊,你也听俺一句劝,还是趁早把你家菜田断卖给俺,还有你家的养娘,也是个招人爱的。拿了钱到县里上下打点一下,辛苦两个月也就没事了。”

只是当他转到韩家人的那边时,却见到韩冈冷冷的一眼瞥了过来,眼神森寒如冰,激得李癞子全身四万八千根寒毛一下都竖了起。

韩冈双眉又浓又密,却并不粗重,浓黑得像是制墨圣手李廷珪亲造的珪墨描出,却没有卧蚕眉的粗厚,也不似过于挺直一端收尖的剑眉,而是匀称窄长,直如一对打造得既薄且利的关西快刀。有了这对如刀双眉,韩冈原本略嫌朴实的脸就立刻生动起来,只将两眼剔起,双眉飞挑,就像两把快刀捅将上去。

李癞子少年曾在山中被大虫盯过,凭着一点运气逃得性命。韩冈这一眼给他的感觉,却如虎视一般。被韩冈一瞪,李癞子的气焰便登时莫名其妙的低下去了七八分。这时候,厨房里的韩阿李、韩云娘正好得了消息,一起赶了出来。

“李癞子,你好胆!”一声震得殿顶天花承尘上灰土直落的暴喝,很难相信是出自一个四十多岁的妇人之口。韩阿李喝声未落,手臂一挥,一条虚影呼啸而出,带着滔天的杀意直奔李癞子而去。

韩冈的外祖曾经在一场战斗中,用三支投枪穿透了七名党项步跋子的身体,就此稳稳的坐上了都头的位子,在泾原路军中也是小有名气。韩阿李投出的东西也仿佛投枪,快如流星,只是以些微的差距擦过李癞子的耳垂,猛然撞在朝内开的庙门上。轰然一声暴起,震得众人耳中嗡嗡直响。虚影砰的落于地面,却是韩阿李从家中带来的擀面杖。

韩阿李气势汹汹的杀奔出来,李癞子被一根擀面杖吓得最后一点气焰也消失无踪,连忙干咳了一声:“韩菜园,阿李嫂,别道俺没说。两天后你还是老老实实的入城做衙前罢,要是不应役,你的板子少不了,你家三哥的前程怕是也要泡汤!

李癞子抛下句话,转身就跑着走了,韩阿李直追出门外,大骂着追着李癞子跑远,才恨恨而回。偏殿一片寂静,参加宴席的众人皆面面相觑,不知该走还是该留。

韩千六垂着脑袋唉声叹气,韩阿李冷着脸,紧紧攥着捡回来的擀面杖。韩云娘泫然欲泣,楚楚可怜,李癞子让韩家卖了自己的话,正好给她听见,心中顿如落进了冰海里,浑身都在发抖。她不由自主的靠近韩冈,几乎要贴到他身上,仿佛只有这样才能驱散心中的寒意。

韩家四人中,一人愁,一人怒,一人忧,只有韩冈若无其事,坐得四平八稳。握了握小丫头变得冰冷的小手,安慰了一下,轻声说道:“别担心,又不是多大的事!你三哥哥解决得了。”

安抚了小丫头,韩冈拿着酒杯站起来,灿烂的笑容中充满自信,“怎么了,宴席才开始啊……别让李癞子这蠢物败了大伙儿的兴致!”

“……三哥儿……”刘槐树茫然的看着韩冈,刚才没能帮上韩家的忙,让他很是愧疚,“可那李癞子的亲家……”

“黄大瘤又如何?”韩冈哈哈大笑,笑声中有着掩不住的杀机,“李癞子仗势欺人,鱼肉乡里,视国法于无物。日后自有王法处置他,到时诸位叔伯在旁做个见证也就够了。”

韩冈说得狂妄,但满是豪情壮志的气魄让众人不由自主的相信了他。他们仰头看着韩冈,就像第一次认识韩家的三哥儿。对了,他毕竟是个秀才,走到县里,县尹都要和和气气跟他说话的。黄大瘤虽是陈举的亲信,但也不能跟一个读书人比吧!

韩冈将酒杯举起,洒脱自如的姿态使得席上各人不敢怠慢。来客纷纷举杯,虽然不比开始时热烈,但一场酬神还愿的宴席终究还是顺顺利利的进行了下去。

韩阿李和云娘从厨房中跑进跑出,端上来一盆盆热菜,韩千六不住向宾客劝酒,至少在表面上已经看不出韩家将要面对的危局。

韩冈低着头,在他面前,筛过的酒水清澈透亮,在杯中轻轻摇晃,散着寒气的眼眸倒影扭曲不定,隐隐透着阴戾,一如韩冈的心。他轻声低吟:

“天作孽,犹可恕,自作孽,不可活!”

仰头举杯一饮而尽,抬起头来的韩冈,他脸上绽出的笑容如同春风吹拂,眼底的凶戾敛藏无踪,

‘天作孽,犹可恕,自作孽,不可活!’

注1:此是史实。宋神宗和王安石之所以要改革役法,也是因为这差役太过残民。

注2:北宋丁壮的年纪划分以二十岁为底线,六十岁为上限。

注3:按照北宋前期役法,单丁户,无丁户,女户,都是不需要服徭役的。

PS:文化商业繁荣的北宋,被许多人心往相之。但北宋是士大夫和小市民的乐土,而绝不是农民的。

今天第三更,求红票,收藏

