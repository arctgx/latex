\section{第37章 朱台相望京关道(12)}

这算是不辩而辩吧。

短短两千余字的文章,曾布很快就看完了,但坐在灯下,却是久久不动。

论其文采,不值得多议,可内容,却不能不深思。

京城两大报社的影响力越来越大,而手中的资源也越来越多,早在仅仅只有小报的时代,官员就是购买报纸的主力军,何况如今内容更加充实的两家快报。

聪明一点的官员都知道购买一份报纸来及时掌握市井中的变化。而手中权力更大的宰辅,更是能得到报社中人的投效,还能直接收买关键人员,往往能在第一时间得到重要的情报。

韩冈送到《逐曰快报》的文章,不仅仅送到了王安石的手中,曾布也十分及时的得到了一份抄本,由妻弟魏泰连夜送到府中。

盯着韩冈的文章,曾布久久不动,书房中只有玻璃灯盏内的烛火在闪烁。魏泰等了半天,不见曾布动作,眼睛也被烛火闪花了,耐不住低声轻问:“姐夫,韩冈的文章中果真有深意?”

魏泰连夜将韩冈的文章送来,之前已经先看了一遍。心知韩冈用心深远,不过效果似乎太强了一点。让他的姐夫竟然愣了半晌都没有动静。

曾布终于有了动作,探手一推桌案,靠在了椅背上。闭起酸涩的双眼,喟叹道:“义利之辩,韩冈是别出蹊径。甚至无一字涉义利,只从钱着手,孔孟之后,无人有此一言……气学一脉的着眼点始终与人不同。”

他又摇摇头,“不过究其本,还是孟子的‘王何必曰利’。以信义治民,而利自得。背信弃义,得之小而失之者大。尤其是将钱之为钱,归之于信这一点,完完全全是出自于孟子。气学崇孟,由此可见一斑。”

魏泰眨了半天眼睛,他可不是想听曾布说这些的,曾布要如何应对才是他的有兴趣关心的事,“那韩冈的用心呢,可是在攻击朝堂?”

曾布眼皮微抬:“还看不出来吗?气学讲究实证,既然韩玉昆说‘钱之本,实乃信’,那铸币的问题就得从信字入手。无外乎是在说,当他入朝后,就能让折五钱重新得到三军万民的信任。”

“他不是要朝廷停止铸造大钱、铁钱?”

“怎么会?只是以己之能,证我等之愚,不堪为朝廷辅弼而已。他一向是自负其能的……”

曾布哼了一声,韩冈的脾气他早就见识过了。从当年第一次见到韩冈时起,曾布就知道,这是个喜欢表现自己的才干,又不吝于铤而走险的危险人物。如今十几年过去了,多少事实都不断在证明这一点。

“可他也没说怎么办?”

“文章里已经说得再明白不过了吧?文宽夫的例子都拿出来了。”曾布看了魏泰一眼,“只要能保证朝廷的信用不失,铸币的祸患就没了。赋税征收以折五钱为重,或是铜铁各半缴纳。只要朝廷承认折五钱能当五文用,那市井中当然不会再折二折三。”

魏泰摇摇头:“似是故弄玄虚。”

“这就是韩冈聪明的地方。就算没有他的这篇文章,朝廷迟早会这么做的。但现在他的文章既然写出来,那么政事堂下堂札要,都会被说成是从韩冈而行。无能二字,可就少不了的。”

“这……”魏泰抿着嘴,阴沉着脸说道:“其心可诛。”

“担什么心?此篇一出,难做的可是王介甫。”曾布脸上不见有半点心结,“如果王介甫能排除偏见,当能看得出《钱源》的意义何在。可惜啊,他怎么可能做得到?”

“究竟是为何?”

“还没明白吗?韩玉坤在《钱源》之中说朝廷铸造钱币,必须保证钱币的信用。但反过来想,只要保证信用,就是纸片也能当钱了。”

“怎么可能?!”魏泰双眼一下瞪起,差点失声叫了起来。

“去过质库吗?知道押票吗?”

“没去过,不过还是知道的。”魏泰摇摇头,又点点头。

“押票是什么?”

魏泰顿时恍然,“啊!……原来是这个道理。”

质库的押票不仅仅是赎回的凭证,在许多质库,还可以直接拿着押票再押一次,拿来换钱。这也是门生意。绝大多数人把押票都押出去后,就不会再赎回了。而两次质押加起来得到的金额,其实还不如一次死当。对开质库的商人们来说,他们也更喜欢这样的质押手法。

押票不过是张纸,但也能换钱。因为押票的背后,是赎回抵当物的权力。是信用的一部分。片纸亦可为钱。

魏泰舔了舔嘴唇,只觉得喉咙发干,“朝廷如今财用匮乏,如果韩冈当真能以片纸为钱,可就在朝廷中站稳脚跟了。”

“站稳脚跟?”曾布一声冷笑,“东府之中,可就只剩一个空缺了。”

魏泰终于怔住了。

韩冈的目的……竟然是宰相?

……………………

凡民之情,见利则移之。这是司马光的话。小人喻以利,更是圣人之教。

求利则必以信义。失信去义,虽得利一时,却不能长久。要跟小民说话,当然只能以利为主,不能空谈大义。所谓见人说人话,见鬼说鬼话,就是这个道理。

让百姓得利,朝廷的信义也就树立起来了。

虽然不是韩冈文章中的原话,但章惇读过来,其中内容大体如此。

韩冈的文章像是在尊崇孟轲,但内里却总有一股子怪味。

明面上这是韩冈向皇后,向群臣,向三军万姓立下的军令状,只要他回京,因朝廷加铸大钱而动荡不已的物价,将会很快恢复平静。由此一来,跻身东府也是顺理成章。

东府中的空缺,只剩一个宰相了。曾布、张璪都还不够资格再上一层楼!

只是章惇觉得,韩冈并不会去奢想宰相之位,他只是故意让人这么去想罢了。

章惇很了解韩冈,既然公诸于众,那么背后肯定还有别的用意,不可能那么简单让人看破。

至少有一个破绽,两家报社可不是韩家开的,自己能提前拿到他的文章,其余宰辅不可能拿不到,只要阻止刊载,那些谋算全都要作废。

这篇文章,韩冈当真指望过能登上报纸吗?章惇可不觉得。

不知道平章府那边会不会派人去封掉报社,或是逼报社撤下含韩冈的文章。不过就算王安石不做,两府之中,总有人会去做的。

不管怎么说,加铸的决议是东府之中所有人都赞同过的,出了问题每一名宰辅都要承担起责任。等到明天报刊发行,政事堂可就要沦为笑柄了。无能二字,没人愿意落到自己的头上的。只要及时更正,对百官、三军、万民可都是一桩好事。

或许要等到韩冈进京之后,才能知道他的真正用心了。

章惇叹了一声,犹豫着,要不要先抓住他问一问。

……………………

情况看来又要发生变化了。

在韩冈名为奏禀,实同檄文的奏章抵达京城后,蔡确就觉得,他是要拼个鱼死网破了。

但以韩冈的行事为人,肯定还有后手。至少在他启程南下前,应该能预料得到,韩冈历年来任用和提拔的官员,也必然都会受到池鱼之殃,绝不可能再置身于外。

韩冈所提拔的官员,现如今还远远不能为他提供帮助。其中几乎都找不到一个有进士资格的成员。大部分还沉沦下僚,不知何时才能得到晋升的资格。论起做事的能力,他们都很出色,可缺乏出身是他们的致命伤。可是韩冈既然已经举荐了他们,就必然要受到他们的牵累,一旦被定罪,韩冈也难以自安。

而且韩冈的表兄李信也受到了波及。在战争结束后,便因那场惨败被夺职后召入京城,甚至连差遣都没给,直接在京城做起了冷板凳。现在更是受到了弹劾,要追究他战败之罪。

一直以来,韩冈与王安石虽有学派之争,但在朝堂上,却还是靠得很紧。王安石两次回朝,韩冈在其中出力甚多。蔡确多年在朝,未曾离京半步,凡事历历在目。

如今翁婿俩竟然到了公然决裂的地步,始料未及之余,倒是让人松了一口气。

只是蔡确没料到韩冈会从铸钱着手,而且着手的角度同样是出人意料。

朝廷通过加大货币发行的力度,将亏空转嫁给百姓,致使民怨沸腾。韩冈一篇文章浅显易懂,就算仅仅是粗通文墨,也能看得出来他的用心是要朝廷保证币值的稳定,以此来维系朝廷的信用。这是在彰显他的谋国之材。由此来反衬东府的无能颟顸。

要不要派人去报社,蔡确有些犹豫。同意加铸大钱和铁钱也有他一份。韩冈的文章如果正式公布,对自家的名声殊为不利。

只是考虑了一阵之后,他还是决定放弃了。

韩冈对朝廷铸钱一事的公开宣言,至少在明面上进一步与控制朝堂的新党分道扬镳。在朝堂上独树一帜,甚至可以说他已经与洛阳的旧党,开始两相唱和。靠近旧党,来保护自己的嫡系。

既然如此,韩冈就肯定有后手。就蔡确所知,洛阳可也是有报纸的。

或许这个时候,司马光已经看到了韩冈的这篇文章了。

