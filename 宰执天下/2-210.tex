\section{第32章 吴钩终用笑冯唐(17)}

四月中旬,暮春与初夏的交替之时,天气渐渐热了起来。

一天比一天更为炽烈的阳光,晒得石榴花红艳如火,开遍了长安城的大街小巷。

韩冈已经在长安城驿馆之中住了快有半个月,等待着东京城中传来最后的消息。相对于前段时间在生死边缘的忙碌,现在的生活可以说是清闲得过了头。不过这也是韩冈梦寐以求的,他还想着考进士。前面忙了几个月,功课都耽搁了下来,见缝插针的攻读经书,只能保证不会生疏,但更进一步的系统学习,也只有等到有了比较完整的空闲时间。

在韩绛离职,郭逵继任后,陕西宣抚司已经陷入了解散前的停滞状态。没有‘便宜行事’的自行处断之权,宣抚使就是一个空名,郭逵的主要精力现在都放在了他知京兆府及永兴军路安抚使的职位上。

他刚刚上任,有许多事务亟待上手,另外因为吴逵的生死不明,所有与他有过关联的设施、部署、人事,都要进行更迭或是检查,不论是缘边四路,还是关中腹地的永兴军路,都是一样。

为了处理这些大大小小、千头万绪的琐事,郭逵很是忙碌,根本无暇去理会停摆中的陕西宣抚司。他身为宣抚使所下的唯一的一道命令,就是将帅府行辕迁到了长安城中。

而在此之前,来自于缘边各路的平叛将领,都已经各自率部回返驻地。只剩被韩绛陆续征辟而来的十几个属官,与韩冈一样都住在长安驿馆之内,等着朝廷最后的发落。

他们中的大部分,都是各自邀约,每天出门闲游,白天骑着马转遍了长安内外有名的名胜古迹,夜里则去自唐时起,便广有盛名的平康坊去体察民情。也只有韩冈一人,独宿于驿馆中的一处偏僻小院,日夜攻读经传。除了被郭逵征辟,没有回邠州的游师雄,他也不去见任何闲杂人等,只是在读书。

读书累了,就起身锻炼一下身体。流了一身汗后,换了衣服,就又坐下来继续攻读。如此专注苦读,让途经长安的吕大忠赞赏不已。

关中有名的蓝田吕氏四兄弟——吕大忠、吕大防、吕大钧、吕大临,除了吕大防外,其他三个都是张载的弟子,其中吕大忠年纪最长,跟随张载也最早。他与张载同龄,却依然师事张载,是韩冈、游师雄的大师兄。吕大忠本是做着,最近届满卸任后,暂时没有去京城守阙,而是准备去横渠书院拜会张载。只是在路上听说了韩冈和游师雄这两位最近声名鹊起的师弟的名头,才顺道来拜访。

蓝田吕氏虽未出过宰执一级的显宦,但上溯数代也都是官宦人家,算是历代簪缨。在张载门下,不同于种建中和游师雄以兵法为主,吕氏兄弟则是专注于经术之上。

见到韩冈正在苦读经传,吕大忠便不顾旅途疲累的加以指点,连游师雄和韩冈为他办的接风宴上,也在说着经传释义。他的这位大师兄虽是为人谦抑,但学问精深,在周礼、史论上更是专精,给了韩冈不少指点。

而当吕大忠听韩冈说起‘以数达道’的想法,还有对‘格物致知’的新解,也不是嗤之以鼻,而是兴致盎然的详加询问,讨论了数日之久,甚至帮了韩冈弥补了他叙述理论时,几处用词上的漏洞,用更加切实的儒学语言来解释几条力学定理,使得力学原理跟张载的气学更加紧密的联系在一起。

这一番讨论,直到行程紧迫,吕大防方才依依不舍的告辞离开。临走时还让韩冈对此继续深入钻研。在他看来,自然之道是气学重要的组成部分,如果韩冈对格物致知的总结更加充分,便可以更加完善气学上这一方面的理论。所以他告别的时候是依依不舍,走时却是脚步匆匆,急着要跟张载去讨论。

吕大忠走了,韩冈继续安然坐下来读书。只是他苦读归苦读,等到留在绥德的周南,被种谔遣了可靠亲信护送过来后,韩冈也会在读书和锻炼之余,加进去一点娱乐活动。

没有外人的小院中,周南换了一身轻薄的青色罗衫,单薄的数层丝绸遮掩不住傲人的身材。踩着一双木屐,白生生的一对小脚露在外面。她坐下来的时候,背挺得很直,巴掌宽的绣花黄丝罗带系在腰间,更显得腰肢纤纤、峰峦挺拔。

韩冈坐在院中的石桌旁,头上的榆树荫荫如盖,遮挡着变得炽烈起来的阳光。低头看着桌上的书卷,默默的读着书上的文字。念完一句经文,便闭上眼睛去背诵有关的注疏。一段段的背过来,显得不急不躁。

而周南娴静地在一旁,拿着轻罗扇,轻轻的扇着风。持扇的小手,光洁如玉,褪到肘间的袖口又把玉藕一般的小臂露了出来。手臂轻挥时,闪着炫目的白光。

绝色佳丽就在身边,阵阵幽香从微敞的襟口处散了出来。此情此景让人沉醉,但韩冈依然不解风情的在读着书。专注而用心的神情,让周南痴痴地看着,不知时间倏忽而过。

一直到了快中午的时候,才有人惊扰到静谧而安宁的气氛,游师雄找上了门来。

听到外面游师雄的声音,周南连忙起身,快步走进了屋内,她的穿着不能见外客。

而韩冈把书放下,自己过去开门,把游师雄迎了进来。两人就在院中坐下,淡淡的幽香仍在原处,游师雄微微一笑,也不打趣韩冈的艳福,而是正色道:“玉昆,京里来的使臣终于要到了。”

“什么时候?!”

“明天……郭太尉已经派人去迎接了。”

“明天?!”韩冈惊喜着,“等了这么些日子,终于有了个了局!”

“可不一定是好事啊!”游师雄却叹了口气。

他在张载的弟子中算是出类拔萃的一个,中进士又早,与同窗们的联系比刚刚崭露头角的韩冈要多得多,如今又在郭逵的帐下,消息也自灵通不少,今天刚刚得到一点新情报,便赶着过来。

“为了评判今次一战的功过,据说王相公和文相公两边吵到天翻地覆,一个说罗兀得而复失虽是不无遗憾,但胜果累累,战功为多年仅有;一个则道,此战劳民伤财,激起兵变,哪有半分功劳可言。这弹劾和请郡的奏疏,一封接着一封,也不知道那边占了上风。”

韩冈摇摇头,冷笑着:“小弟不信文枢密敢吞没参战众军的战功?”

“枢密院当然不敢,所以倒霉的会是宣抚司中的文官。韩相公的处置已定,总的要有人出来负责——光一个吴逵,压不下悠悠之口。”

就算是文彦博等一干旧党,也怕不能以功封赏,以至于闹出兵变。他们打压的,只是宣抚司中的文官。宣抚司文官都是韩绛征辟而来,能力水准都不差,且绝大多数都是偏向于变法一派,如果承认了他们的功劳,等于是给新党添砖加瓦,文彦博他们怎么肯干?!

“秀才造反,十年不成。文枢密会怕逼反了武将,却不会怕得罪陕西宣抚司的文官,看起来真的是不妙了。”韩冈笑着,这对他来说倒是不差。

“玉昆你倒是胸有成竹啊……”

“跟景叔兄你一样。”

宣抚司中,韩绛的诸多幕僚,也就只有韩冈和游师雄的功劳是没人能抹去。游师雄前面担心的,就是他和韩冈独占功勋,而他人无赏,会惹得众人嫉妒。而韩冈放心的,也是因为众文官没有功劳,他拒绝封赏,便不会让人说成是沽名钓誉。

当次日,宣诏使臣李宪带着诏书来到长安,宣诏的内容,就是跟他们预计的一样。赵顼和王安石都没能压下文彦博等一干旧党重臣的反扑,不得不将宣抚司文臣牺牲掉。

宣抚司众文官,只有微薄的银绢用以酬劳,而没有任何加官进爵的功赏。唯有游师雄和韩冈两人例外。

游师雄的功劳没有任何争论的余地,在叛军气焰正盛时,给他们当头一棒,阵斩鼓动部众将吴逵救出大狱的贼酋解吉,保住了兵力虚弱的邠州城。从胆识,从才智,在官员中都是屈指可数,故而特旨转官。由选人转为京官,脱离了选海。

而韩冈,金银财帛一样不少,另外最为重要的一项,是跟游师雄一样,也是脱离了选海,被特旨转为京官。

接下来只要他们两人去京城走上一遭,依例面圣过后,就是正式的京官了。自此之后,便能走上升官的快车道。在为官刚满一年的情况下,便由选人转为京官,这在官场上绝对是个异数。

失落的众文官的眼神又嫉又妒,但他们却震惊的发现,韩冈并没有叩拜谢恩的意思。

李宪催促着:“韩冈……还不接旨谢恩!”

“玉昆,你……”游师雄也大惊失色。

围观的众人都不知道为何韩冈还不接旨。横亘在选人和京官之间的鸿沟,深阔如渊海,多少心比天高的臣僚,在一次次转官未果的情况下,最终失去了所有的动力,在选海中沉沦了下去。才二十岁就能成为京官,只有宰执家的嫡子受到荫补时,才有可能。纯凭功劳,韩冈可能是几十年来的头一份。

为什么要犹豫?还是说,他欢喜坏了,忘了谢恩?

韩冈沉默了一阵,终于开口。是谢绝,而不是接受:“罗兀之捷,在于精兵悍将,韩冈不过是随行而已,并无尺寸之功。说降叛军,乃是大军在外之故,并非韩冈之力。至于其余微薄之功,当不起如此封赏。诸多溢美之词,韩冈亦是愧甚。”

他再拜叩首:“下臣不敢受赏!”

