\section{第19章 虎狼终至风声起(中)}

【今天肯定两更。下一更在十二点之前。求红票,收藏】

郭逵还在京城。二十天之内,他已经四次被天子招入宫中问讯西北边事,每一次都至少说上一个时辰。一般来说,入京觐见的守臣,通常是面圣一两次就回去,而外放的官员陛辞,也不过是在朝会上叩谢天恩、说几句有用没用的话罢了。

而郭逵以地方守臣的身份三番四次入宫廷对,自赵顼登基以来,是从来没有过的恩数。世人本以为他因为跟韩绛相争,而被调离延州,是失了圣眷。可如今一看,天子对他的信任是依然不变。赶来登门拜访的客人一波接着一波,热闹得就跟宰执家门一般。

不过郭逵却有些不耐烦了,站在厅门外的台阶上,送走了今天不知第几批客人。他就阴沉了脸进厅坐下,拿起手边已经放冷的茶水,一口灌了下去。可凉茶还是压不住心里的烦躁,炎夏日落后的暑气也是一直不停窜入厅中。

内外交加,郭逵烦躁不堪。转过身,从身后婢女手上劈手夺过慢慢扇动着的绢扇,他就这么攥着扇柄,自己哗啦哗啦用力的摇了起来。

郭逵向以知人明事著称朝中,先见之明更是跟乌鸦嘴也差不多。他说韩琦行急进之策,命任福贸然出兵,是‘地远而食不继,城大而兵不多,未见其利’,而后便有好水川之惨败;他当着众人的面,说葛怀敏为人‘喜功徼幸,徒勇无谋’,‘他日必败朝廷事’,当时无人肯信,可转过头来,就是葛怀敏战殁于定川寨。

所以赵顼的想法,以郭逵的眼光便看得很明白。这只不过是天子安抚重臣的做法罢了。他是现今外放武将中稳坐头把交椅的重臣,又做过执政,不是等闲守臣可比。如今三衙中管军的几个太尉,论名位,也无不在他之下。他在延州起用燕达新败党项不久,便被韩绛逼离,天子对此当然要安抚一二。

不过天子多这个安抚,郭逵看得出里面又是带着一点小心思。他第一次第二次面圣还说了点正事,到了第三、第四次时,根本就是在武英殿陪着皇帝在摆弄沙盘军棋。

虽然在沙盘上向天子解说自己过往的战绩,的确是件光彩的事,可天子如此做,却多半是在担心自己到了秦州后赌气,另一方面,应当也是想给筹备缘边安抚司的王韶留一点应手的时间。

如果天子所为,不是有人在后面给他支了招,就代表年轻的皇帝陛下在坐上龙庭几年后,历练出了足够的城府和心机——两种情况都一样糟,这代表在天子心目中,他郭逵是个不能容人、心胸狭隘之辈。

郭逵越是这么想着,心中的烦躁就越盛。他现在已经是秦州知州,王韶就是他的属下,王韶听他的是理所应当。只要王韶肯遵从他的命令,他郭逵又怎会与其为难。可天子却偏偏不放心,硬是要留着他,为王韶让出路来。

即是如此,那还不如让王韶做这个知州,他去当缘边安抚!

郭逵手上的扇子越扇越快,带起的呼呼风声就像是他心里的怒意在燃烧,绢扇扇面上绣着的图案模糊了起来。当郭逵的儿子郭忠孝进来的时候,就看见他父亲手上的扇子啪的一声响,竹枝扇柄断了,扇面一下飞了出去,落到了郭忠孝的脚边。

郭忠孝轻轻叹了口气,俯身拾起扇面。郭逵这样的情绪他也不是第一次见了。

他的父亲,精于兵事,尤擅阵法,知人知兵之名,亦传与当世,断人成败如烛照龟卜,百无一错,且善抚士卒,深得军心。但在世人的评价中,可没有一条说他易于相处。

相反地,郭逵为人峻急,性格刚毅,甚至近于刚愎。一直以来都仗着眼光精准,行事少有错漏,很少采纳他人之言。而且随着地位日升,他独断独行的作风越发的强硬,根本容不得有人说二话。

他在延州统管鄜延军事,便把跟他性格相似的种谔踢到了一边站着,自己直接控制进筑横山的战略。而当韩绛以枢密副使的身份担任陕西宣抚使,就变成了一山难容二虎的局面。若是他在韩绛面前能稍稍退让,也不至于被赶出延州。

只是江山易改,本性难移,郭忠孝也不指望自己的父亲在现在这个年纪,还能把一贯以来的行事作风给改了。

“大人,孩儿回来了。”郭忠孝在郭逵身边敛手行礼。

“回来了……”郭逵把秃秃的一节扇柄丢到了脚下,问道:“李师中的那个幕僚怎么说?”他在家中亦如严君,对待儿子,就像对待手下的官兵一般,说话直截了当。

向宝此时身在京中,窦舜卿此时身在京中,给李师中打前站的家人也刚刚入了东京城。就像天子要向每一个诣阙的守臣询问地方上的大小事务一样,既然就要成为秦州的主事者,郭逵没有理由不跟他们询问一下秦州的内情。而郭忠孝今天宴请的姚飞,便是李师中手下最得力的幕僚。

郭忠孝道:“姚飞说的跟窦舜卿、向宝没有什么区别。但言王韶奸狡,而他手下的韩冈尤甚一筹,若要对付王韶,最好先剪除其羽翼。”

“哼!”郭逵冷笑一声:“这是李师中要姚飞代他说的话。是要我替他报仇吧?被属官灰头土脸的赶出了秦州,亏他还有脸来求人!”

郭逵在儿子面前没有掩饰他对李师中的不屑,郭忠孝心中有些惊异,“难道大人想听的不是这些?”

郭逵冷声道:“我想听的是秦州内外诸事,能派得上用场的消息,不是李师中、窦舜卿、向宝他们对王韶的怨恨。如果王韶老实听话,为父何苦要与他为难?如果王韶想跟为父打擂台,我自有手段对付他,又何须用一群丧家犬出的馊主意!”

“那韩冈呢,”郭忠孝又问着,“他是王韶帐下鹰爪,可是出了不少主意……”

“韩冈奇才!”郭逵打断了儿子的话,而他对韩冈的评价更是让儿子惊讶不已,“光是在军中设疗养院一事的功绩,韩冈就是转官都够资格的。受伤后能及时康复,少了后顾之忧的士卒,可比一群胆怯之辈有用得多。他若是在我帐下,为父怎么也要把他顶到京官的位置上。为父到秦州后主持拓边河湟,动起刀兵来,也少不得要用得到他。”

郭忠孝眨着眼睛,难以置信的看着自己的父亲。自他记事以来,几乎没有从郭逵嘴里听到如此盛赞一个年轻人的话语。就连自己,读书读得好,被西席先生赞了,换来的,也不过是郭逵的头点上一点。郭忠孝微不可察的皱了皱眉,一点嫉妒之心油然而生。

儿子嫉恨上了韩冈,而郭逵却还在大赞着他:“而且韩冈还造出了军棋、沙盘,用之推演过往战事,或是排兵布阵,可比起纸上谈兵要直观得多。常人能作出其中一项,已足以留名后世,他却轻轻松松的就拿出了两项、三项。”

赫赫有名的郭太尉在儿子面前,摇着头感叹着,“韩冈之才,在年轻一辈中少有人能及。能孤身夜入虏帐,说服俞龙珂,更是智勇双全的豪举,不比为父当年孤身入保州,说服叛军出降稍差。李师中那三人只看到了韩冈的心机智计,却没看到他真正的大智慧。”

郭逵对韩冈到所作所为啧啧称叹。作为知兵知人的名将,他对韩冈自入官以来的功绩,感受到的震撼可比那些文官要强出百倍。无论是让伤兵死亡率降到一成以下的疗养院,还是让天子——甚至还有他本人——都差点沉迷进去沙盘军棋,都是在军事上有着难以估量的作用——比起斩首个千儿八百,要强得不啻十倍、百倍。

而且韩冈还深得圣眷。在郭逵四次于崇政殿中面圣廷对的过程中,天子提到韩冈这个名字至少十几回,而在其中两次被带到武英殿偏殿沙盘模型时,提到的次数就更多了。

郭逵并不打算要跟韩冈过不去,相反地,更想好好的提拔他:“如此人才当为我所用,而不是把他当作王韶的羽翼个剪除了。”

王韶在秦州沉寂一年多,自从把韩冈延揽入帐下后,便一鸣惊人,接连两次大捷不说,还把秦州军中三位主官一起赶了出来。虽然李师中他们的调离,本质上体现的是天子的倾向,但能让天子作出决断,王韶……也许是隐在他身后的韩冈……在其中费了不少力气。

而他本人之所以会从延州任上被调去秦州,就是天子在他和陕西宣抚韩绛之间,选择了从没有带过兵的韩子华,让他主持横山战略。韩绛立功心切,他所倚重的种谔也是个贪功之辈,他们的想法,跟自己实行的战略完全相悖。

而眼下的,正在秦州施行的河湟开边,其中的各项策略,都是郭逵能认同的。既然如此,他也想着从中插上一脚……不,是全面掌控大局。

天子不是喜欢开疆辟土吗?

王韶能做到的,他郭逵一样能做到,而且可以做得更好……因为他是郭逵!

两天后,郭逵第五次入宫面圣,完成了他的陛辞,终于踏上前往秦州的道路,而与他同行的,还有带着圣旨和十几车赏赐,去秦州为古渭大捷颁发封赏的天使——并不是前次颁诏的王中正,而是另外一人

——李宪。

