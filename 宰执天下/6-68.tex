\section{第八章 朔吹号寒欲争锋(八)}

“倭国怎么就这么弱?”

次日在殿上,皇太后对才几个月的时间,日本连京城都被烧掉表示不能适应。

“不说是海东大国,方圆数千里,人口几千万?怎么连京城都丢了?辽军不是不会攻城吗?”

“太后。”章敦出班道,“倭人的国都平安京并无城墙,据闻倭人只在外围修筑了一道长墙和几处寨堡。”

得了章敦的说明,向太后隐隐约约想起上一回说起日本战事,就听枢密院禀报过有关倭国的情报,其中就有提到其都城的防御。

“居安思危,有空起名做平安,不如先将城墙修好,这样才得平安。”

“太后圣明。”

群臣一起向太后行礼。

国内也不是所有城市都有城墙。南方大部分城市都是没有的,甚至包括许多州城,只在官衙等城市重要建筑有围墙保护。理应设置城墙的位置上,就只有一道篱笆。有的是木桩,不过更多的是柳条,主要是因为其扦插便能成活,等其长成大树后,就是一条不错的防线。

但京城都没有城墙,那就是日本人自己的错了。至关重要的都市不修筑起城墙,这是开门揖盗。若不是日本有海水为外防,早不知给灭亡多少次了。

“可就算没有城墙,以倭人之大、人口之多,也不该败得这么快。他们的刀剑不是很出色吗?倭刀在京师中卖得也贵。”

“太后明鉴。”章敦持笏行礼,“军国之寄,非在刀剑一项,弓弩、甲胄,倭人远远不及中国和北虏。”

紧随在大宋之后,辽人现如今能给国中的主力骑兵装备上大批量的铁甲,尽管基本上都是护住胸腹要害的胸甲,以及一顶铁头盔,但足以让辽军和倭军之间本就有天壤之别的实力差距,拉大到加上天时、地利都难以扭转的地步。毕竟人和的一面,专心抢掠的契丹人,绝不会给倭人表现出来的机会。

“而庙算、阵法、训练,无一不是决定胜败的关键。”章敦继续说道,“此更非倭人所长。除了皇宋,现如今又有哪一国能挡得住契丹铁骑?”

韩冈也出班道:“契丹铁骑来去如风,寻常步卒如何克制?我皇宋禁军用了不知多少将士的性命和血汗,才换来应对契丹铁骑的经验。就算主帅为敌所伤,其下士卒也会在各自将校的指挥下,继续结阵抵御敌军。这就是经验。而倭人对此无从得知,就算有一二眼光卓异的将帅谋臣,也指挥不了一群茫然无知的士卒和将校。”

韩冈说的话,太后更能听进去:“参政说得是,我皇宋禁军的确非倭人所能及。只是吾亦知国中精兵强将尽在北境,辽人渡海后都能在旬月中灭去日本。万一辽人渡海绕过河北,从淮东、江南登陆又当如何?”

这方面的问题,朝廷很早就在说。但当时的形势还没有如此急迫,日本如此速败,也超出了所有人的预计。

“陛下勿须忧虑。日本速败,是闭关锁国的结果。倭人关起门来称皇帝,有渊海为防,自以为可以高枕无忧。可是当恶敌上门,便全无应对之法,除了降,也只有死了。”

“依参政之见,又当如何应对?”

“一是加强水师,并修造海舶。”

“自当如此。”

向太后点头,这是过去枢密院曾经提议过的。

日本和高丽,对大宋来说都是远隔重洋,不论是进攻还是防御,都需要一只能在海上作战的精锐水师。

“第二,便是驻军耽罗岛。应该加强兵力,并加快岛上的寨防建设。同时在耽罗岛上招收逃亡的倭人为军。”

参与到对倭战略中,韩冈回答太后问题时,不像是东府的参知政事,倒像是西府中人。

章敦在旁看着,暗暗摇了摇头。韩冈在军事上的话语权太重了,自己也难以与他相匹敌。

要是日后他做到宰相,西府多半就成了政事堂手下打杂的了。加上还有对韩冈言听计从的太后——

“当如参政之言。”

屏风后传来太后答应得干脆利落的声音。

“第三,便是修筑砖石城墙,以备贼寇。”

南方城市缺少城墙的原因,最重要的一条便是雨水太多,黄土夯筑的墙体很容易被浸泡损坏,只有换成砖石包墙的城墙,才能够保证墙体长久的安全。

但立刻就有两个反对声响起:

“这不可行。”

“只怕有骇物议。”

章敦和张璪一先一后的开口。

“朝廷猝然下诏修筑城墙,可知江南人心会乱至何等模样?!”章敦质问着。

张璪也道:“辽人渡海而来,尚属猜测。却耗竭民力去修筑城墙,届时臣恐家国之忧不在外而在内。”

韩冈摇头:“韩冈不知整修开封城墙如何会有骇物议?更不知为何会引起民乱?”

“开封?!”

“辽贼若渡海,只会是各路沿海军州先遇贼!”

“如今石炭价廉量大,故而砖价大减,正好可以用来整修京城城墙,还可以于城周设立炮台,用以御敌。从此京师可以不畏外敌。”

“善哉斯言。”张璪说道,“可这与防备海寇有何关系?”

“整修京城所用的青砖,可交由南方各处州县招聚工匠烧制,再汇集至京师。”

“参政的意思是,等开封府的城墙修好,那些工匠和砖窑正好可以继续用来修筑沿海军州的城墙?”

“正是如此。”韩冈道,“辽人攻下倭国都城虽快,但平定其国中还需数年之久。等到二三年后,京师城墙修筑完成,沿海军州就不愁墙砖难以烧制了。那时候,可能会有贼人渡海而来的消息也必然在当地民间流传已久,朝廷的举措便不会惊扰到百姓。而且这么做,万一辽人不能稳定日本,也正好可以省下来这笔开支,免得花上冤枉钱。”

韩冈的意见有很多值得商榷地方,不过加强开封城防、避免花冤枉钱两件事,其实就是政治正确,怎么说都不会有错。

而韩冈实际上需要的是对炮台的结构和式样进行试验,与其在边境上实验,还不如在京城这个火炮永远都不会用来杀敌的地方来做实验,就算有问题,依然能拥有足够的威慑力。

“陛下!”默然恭立一旁的王安石突然大声喝问,“臣不知倭国之事,要议论到何时?!”

“平章?!”

王安石发怒,让向太后为之一惊。

章敦闻言,也是一怔。的确是耽搁太多时间了,弄得科举之后的议题恐怕要拖到明天再议。

在日本方面的急报送抵京城前,今天崇政殿上预定的第一件事本来是科举。第二项议题,才是军事,也就是西南诸夷的问题。

卧榻之侧,岂容他人鼾睡。

太祖皇帝的这句话,就是行动的圭臬。

而西南夷也一贯的不恭顺,时叛时降,让朝廷渐生不耐。

让西军能够维持住战斗力,以西南夷为磨刀石。

章敦预计他的提议不虞有人反对。这不仅是他本人的意见,跟是他与韩冈、郭逵等帅臣的共识。

能与辽军中的精锐互有攻守的,六十万禁军之中只有西军可以。

在西军直接面对的目标被摧毁后,就算其中一部分可以转去河东,或是北上宁夏,西去西域,剩下的军力依然数量庞大。

不可能让他们远赴西域,要维持住西军的实力,只有离国中稍近一点地方——比如西南夷,以及之后的大理。

三千西军便能攻取西域,一两万西军想要胜过山中西南夷,也不是那样的难。但摆在禁军面前的第一道关卡,却是道路问题。

在成都府路周边行军,就是在群山中兜圈子,必然会有人以此为由,来否定这项战略。

要如何压倒他们,说服太后,便是章敦今天想要做的,只是被昨日的新消息给干扰到了。

王安石在殿中大声喝问,“天下最为贵重的便是人。周公一饭三吐哺,何为?得人也。倭国,偏鄙小邦。科举,国之大事。如今省试在即,五千贡生云集京师。陛下不在意省试之事,不想着今年又有数百英睿之士被收入朝中,却挂念着远在万里之外的小邦,何也?”

‘差点忘了。’韩冈暗里自语。

他并不是很关心这一科的进士科试。

这一届的考生中,关西士子没几个出挑的。尤其是陇西出身的贡生,基本上就是陪读陪考,很多就是等待参加过几次考试,得到特奏名的资格。气学门下莫不如此。

韩冈并没有地域的偏见,也颇有几个其他地方的应届士子曾经登过韩冈的家门。

可惜如今的科举不是唐时,若能行卷宰辅衙,让当朝宰辅对呈上去的诗文感到满意,那么多半就能拿到一个进士头衔。

但到了如今,科举越来越正规化。糊名、誊抄、锁院,一项项都是在针对考官徇私的手段。

在现有的,进士已经尽可能的做到了公平公正——那是连时代的局限性都算不上,千年之后,类似取巧的手段也是多如牛毛。终究越不了最后的关卡,而纵使位高权重,任凭哪一位朝臣,也不敢公然破坏选举的公平性和公正性。

就是韩冈当年入京参加考试,已经做了王安石的女婿,考官又都是新党徒众,本人更是连天子都看重的新生代,但他为了一榜进士,也是绞尽了脑汁。

所以他根本就没打算为哪位士子伸手,他更关心的是之后黄裳的制举考试。

不过王安石现在发火,方才话最多的韩冈也不得不站出来,

“平章请息怒。方才议论,不是在倭人,而是在辽国。若任由北虏肆虐,中国虽大,将摆不下一张安静的书桌。”
