\section{第31章 战鼓将擂缘败至(六)}

【不好意思,早上有事出去了,这一章迟了一点,下一章夜里照常发】

拥有远超对方的兵力,败得却竟然如此凄惨,不但越过罗兀濠河的近千名步跋子就逃回来了十几人,连也在冲击地方箭阵的过程中,丢了有三四百人。相对于己方几近一千五百人的伤亡,宋人那一边的损失,就几乎可以忽略不计了。

但西夏众将,连带梁乙埋,却是有着一种劫后余生的庆幸。就在战斗结束了半个时辰后,一支骑军出现在罗兀城的西北处在众目睽睽之下,回到了罗兀城中,那群骑兵后面后面还跟着一溜马车,赫然正是方才出城南下,引得梁乙埋尽起兵马的那一支队伍!

“汉人狡猾!”

“汉人当真太狡猾!”

被这次惨败打掉了所有自信的将领们,都在帐中一连声的叹着。只是他们的私心里却在庆幸着:

‘都罗马尾那死鬼死得好,要不是他被一杆枪箭轰碎了脑壳,梁国相肯定不会下令收兵。杀到最激烈的时候,这一千多骑兵突然出现在侧翼,不但攻城的步跋子要丢了大半在罗兀城下,攻打高永能的几千铁鹞子怕也只有一半能回来。’

‘可现在也是一样,粮草已经撑不住了。’

‘不是说契丹人会帮忙吗,怎么还没消息?’

‘事不关己,他们乐得看笑话,耶律乙辛的一封信能当真?’

‘……该退兵了。’

众豪族的族长你看看我,我看看你,交换着眼色,小声的递着话。就等着有人先出头发难,其他人好跟着上来说话。

梁乙埋坐在上首,对下面的小动作只能当作看不见。都罗马尾是他的亲信,虽然丢了罗兀,但他毕竟忠心,本来梁乙埋派他出阵是想顺便让他立个功劳,以便能重新大用,谁能想到最后竟会是这样的局面。

论情况,的确是撑不住了,但要他开口退兵,梁乙埋却很难下定决心。契丹人的承诺的确不靠谱,虽然他一开始也没有指望北朝真的能帮忙,靠着辽国权臣耶律乙辛的一封信,把已经因为宋人近两年来的强硬攻势而变得胆怯起来的豪族,重新召集在旗下,就已是达到了他最初的目的。可是眼下进退不得的窘境,却让他真切地盼望起契丹人真的能帮上他一个忙。

一名豪族的族长终于站了出来,向着梁乙埋道:“相公,这两日分到手的粮草已经越来越少,别说肉了,连干粮都只有那么一小口。孩儿们都喊着饿,再这么下去,就只能杀马充饥了。不知能不能先多给一点口粮,也好让孩儿们有力气上阵!”

梁乙埋暗叹了一口气,这是先用粮草为借口,接下来就是逼他退军。他两眼一扫帐中,众将都在等着他的回答。

一名守在帐外的亲兵这时突然掀帘悄步走了进来,到了梁乙埋身边,暂时化解了他面对着的危局。亲兵递上了一封用火漆封口的信笺,“是西面刚刚送来的消息。”

“西面?”梁乙埋狐疑的验过了火漆,把信封打开。

只是看了两眼,他便猛然的站了起身狂喜的叫起,“原来如此!原来如此!怪不得高永能突然玩上这一手……”他抖着手上的信纸,向帐中众将宣布道:“庆州兵变了!”

“庆州兵变?!”

“没错,的确是庆州兵变了!”

这条消息,对梁乙埋来说,仿佛是绝处逢生一般,而众将则是半信半疑,怀疑者是不是梁乙埋为了让他们继续守在罗兀城外,所耍得诡计。

“是真还是假的,该不会是误传吧?”

“再等两天看看有没有消息。难道你们心急得两天都等不了?!”梁乙埋的口气变得强硬起来,眼神森然,带着若有若无的杀意。宋人内乱,他现在便有了底气。

众将都沉默了下去,暂时不想在风尖浪口上去触霉头,反正是真是假,很快就能见分晓。

见到没人敢反驳,梁乙埋得意的扬起了头,“罗兀城要撤军了,今天只是他们在试探。”他说道。

“庆州那里没怎么打就兵变了,难道就宋国的官家和相公们就不担心罗兀城里会兵变?肯定要撤军了!”得到了庆州兵变的消息后,梁乙埋他已经把罗兀城中今日的举动前前后后都想通了,“今天的第一支是幌子,但第二队出城的车马,肯定是正主。他们是要撤军了,所以先把一些重要的人和物送走。”

“那下面该怎么办?”有人问着,“把罗兀城围起来?”

“让他们走,让他们走!”梁乙埋狠狠的说着,“走出城我们才好追上去,追上去才能把他们全吃掉!”

“要让他们一个都回不了绥德城!”

……………………

经过了几天在马背上的行程,王中正终于抵达延州城中。

让王中正感到惊讶的并不是比半年前见面时老了近十岁的韩绛,而是种谔这位主帅,竟然不在绥德,而到了延州来了。他也跟着韩绛,把领受皇命的一行人,迎进了延州帅府之中。

不仅是王中正来了,为了让文彦博等一干重臣闭嘴,赵顼不得不另外加派了一名朝臣随行——只让王中正这个阉宦一人去体量陕西,就连王安石都不支持。在反对宫中阉人插手政事军事上,新党和旧党实则是有志一同。不过挑出的人却是明明白白的旧党,做着开封判官的赵瞻,是陕西人,一年前还是陕西提点刑狱,因为对陕西局势了解,所以被赵顼看中。

王中正和赵瞻领旨之后,出了京城,便一路向关西赶去。只是当他们一行刚刚抵达潼关,从东京又来了一道金牌,带着几份诏书,把王、两人的体量陕西军事的差事撤了,而改成了到绥德宣诏,并督促剿灭叛军。

匆匆忙忙的改变任务,让王中正和赵瞻都觉得不对。当他们看了给他们两人的诏书,方才知道原来是因为庆州兵变。王中正和赵瞻前脚离开京城,后脚庆州兵变的消息就到了崇政殿中。紧接着就是金牌加急,在潼关终于追上了他们一行。

从天子亲笔写下的几分诏书上,王中正甚至能从中体会到天子的愤怒和惊慌。若论兵变,其实天下从未少过,但一次超过三千人的大叛乱,自贝州王则叛乱之后,这还是第一次。而且是叛军主体的都是经历过战事的陕西禁军中的精锐,这一点,尤其让朝廷上下都感到一阵恐慌。

这场兵变直接导致了横山战事的破局,王中正估计着,至少几年内不可能再有大的攻势。而眼下就在诏书中,原本为了得到横山而攻下并增筑的罗兀城,也将会被放弃。同时陕西的官场也会有一个大的变动。

韩绛是宰相,暂时不宜轻动,得等广锐之乱有个了局才会夺了他的职位。但叛军就在长安城不远处,曾上书反对修筑长安城防、增添邠州兵马的司马光,他的军事才能让赵顼无法信任。秦州知州、秦凤经略安抚使郭逵取代了他的位置,接下来将会镇守长安城,统领永兴军路。而负责剿匪的则是燕达,天子任命了这位一年来青云直上的年轻将领充任招捉使,他留下的秦凤副总管一职,则由张守约暂代。

从这两道任命中来看,缘边诸路的兵将,只有秦凤一路最得信任。而其他诸路,不仅是环庆,连带着鄜延、泾原的兵将,只要跟横山挨着边,官家都不敢相信了。

至于种谔,官家给他的任务就是弃守罗兀,并把一万多守军安然的带回来,让他把自己做出来的事,自己处理干净。

在大厅中,王中正和赵瞻并没有急着宣诏,而是先问起了眼下的叛军军情。

宣抚判官赵禼代表韩绛,回答着两位使臣的问题:“吴逵已经绕过了邠州南下,不过前日被燕达领秦凤军堵在了渭水北岸,没能渡河。而在吴逵的来路上,泾原兵已经抵达邠州,正在向南进军。现在叛军盘踞在咸阳城中,进退不得。”

“吴逵为何要绕过邠州?”赵瞻深悉陕西内情,听得有些生疑。

“因为叛军在邠州城外被伏击,损失不小。吴逵知道城中有了防备,所以才绕了过去。”

“哦……”赵瞻对这个答案有些出乎意料,“想不到张靖还有这个能耐。”

“是军判游师雄的功劳,张靖也就棋下得好。”赵禼对邠州知州张靖的看法跟赵瞻一样。“吴逵久在邠州,与他同举叛旗的广锐军卒,家室也多在邠宁二州,如果真的让他杀到城下,邠州当是难保。所以游师雄便率众出城伏击了吴逵的前军,逼得吴逵不敢攻打邠州,而不得不绕过去。”

“游师雄……”赵瞻点了点头,把这个名字给记下了,“秦凤、泾原,走得也算快了。”

从头到尾都是赵瞻在说话,王中正连句插嘴的地方都没有。正想说话,赵瞻却又问道:“为何种总管会在这里?”

种谔应声答话:“四日前罗兀大捷,斩首一千四百余级,并阵斩西贼都枢密都罗马尾……末将是来报喜的。”

