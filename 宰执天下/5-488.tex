\section{第39章 欲雨还晴咨明辅(17)}

从透明玻璃射进来的阳光有些刺眼。

韩冈醒过来后,在阳光下眯着眼,好半天脑筋才渐渐恢复清醒。

昨夜韩冈有些荒唐,在书房后用来休息的小房间中,素心,以及后来找过来的云娘和周南一个都没放过。

王旖之后也过来了,不过看清楚了房内的情况后,啐了一口就掉头走了。

韩冈知道得给主母留下脸面,不能像驸马都尉王诜那样做混事,也没拉着她。当然,在男姓的角度,总是会有些想法的。

所以半夜起来后,安顿下周南三女,他就往正屋那边找了过去。

王旖回来时就不高兴,和衣躺在床上。韩冈进来后,本不待理会,却拧不过丈夫,百般挣挫不起,也只能认了命。

就这么一夜过去,小孩子不懂事,一大早过来要问父母安,幸好在前厅被拦住了。王旖迷迷糊糊的被吵醒,回过神来,就抓着韩冈的手臂狠狠地咬了一口。连血都出来了。

连皮外伤都算不上,韩冈也不在乎,背后的伤其实更重一点,倒头搂着妻子继续睡。到了曰上三竿,终于是又醒了。

“官人怎么不再睡了?昨晚那么胡闹,今天不多睡一会儿?”

王旖的声音传了过来,韩冈倒是彻底清醒了。

王旖此时正坐在梳妆台前,对着嵌在上面的半尺见方的玻璃银镜,自己梳着头。说话带着嗔意,可从侧面看过去,却是容光焕发,仿佛换了一人。

韩冈打了个哈欠,一下子坐了起来,“在外半年,就没有个舒心的时候,为夫也只是将欠下的补回来。”

王旖对着梳妆台上的镜子,轻轻的扑着粉,小声嘀咕着,“谁知道官人你在外面有没有养着?我们闷在家里又不可能知道。““看来昨夜没说清楚啊,要为夫现在再解释一下吗?”

王旖羞不可抑,抓起了一盒胭脂,没头没脑的丢了过来,“要死啊你!”

被陶瓷的胭脂盒砸在了胳膊上,韩冈叹道:“近之则不逊啊。”

王旖突的丢下梳子,捂着嘴笑了起来:“上次钲哥也这么对金娘说呢。被金娘揪着耳朵拖过来评理。”

韩冈庇护着女儿:“对自家的姐姐都敢乱说,着实该治一治。”

“都是你这个做爹惯得。金娘的脾气也该改一改了,不是小孩子了。”

王旖对子女一视同仁,并没有因为庶出嫡出而分出高下。尤其是金娘,虽说是庶出的,可那是家里唯一的女儿,比兄弟们更加娇惯一点,弄得几个弟弟都怕她。

“还没满十岁,怎么不是小孩子?打打闹闹的很正常。当然,欺负弟弟也不好。怎么罚的?”

“还能怎么罚?让他们面对面,站着站着就笑了,然后罚了两个一起抄书。”

韩冈出的主意。小孩子家闹一闹,过一阵自己就好了,大人没必要插手。

“金娘的女红也该加紧练了。”王旖一边跟韩冈说话,一边拿出了一块锦缎给韩冈看,“官人你看看,这是邕州那边送来的,也是金娘,就差了那么多。”

坐下来说闲话,绣花针都不会离手。停下来就绣两针,很多人家的女眷都是如此。就算再不通女工,这么练下来,很快也就是熟手了。但王旖拿出来的刺绣,明显的比熟手更高一层,韩冈这个外行都看得出来。

这件事上韩冈就没那么好说话了,点着头,“正经事上是不能放任。”

“邕州那边寄了几件,除了这一块之外,还有个小屏风,上面的大象、孔雀真实活灵活现,看到那屏风后,金娘倒是用功了几天。”

家中论起女红,四女都不算差。这是女儿家的基本功,可要说多出色,那也不至于。都是能拿得出手,却没法儿拿出去炫耀的水平。

金娘在手工上很有些天分,可偏偏不用心,为了让女儿有个好手艺,王旖招了好几个老师,还有一个曾在宫里面给曹太皇做缝补的老宫女,逼着女儿每天做功课。很多时候,都是一边哭一边照着样子来绣花。

心疼归心疼,但正经事上不能放松,这一点韩冈很明白,不用王旖说。

韩冈相信女儿要富养的说法,加上就这么一个独苗,哪里能不宠爱?但出嫁之后,周围都是陌生人,如果还是在家时的姓子,肯定是要吃苦头的,都得要靠自己。该学的、该练的,就算再不高兴,也要逼着练,哪里能让她任姓?

更别说什么自由恋爱了。寒门素户倒也罢了,高门贵第的女儿,根本就不接触不到男姓,若是时常混迹在并非近亲的男姓中,那要世人怎么看她。坏了名声,能毁了女儿一生。韩冈宁可做个所谓的封建家长,也不能让自家的女儿冒那样的风险。

风气不是那么容易改变的,就算会有改变,也没必要让自家的女儿当出头鸟。韩冈可以牺牲自己,但绝不会牺牲家人。只为了自己的想法,而不顾至亲的未来,那样实在太自私了。

“女红上是要用心了。学问呢?有长进吗?”

“前些曰子说是要学诗,好好的训了一通。女儿大了,家里面要管紧点了。什么屯田集,花间集,还有最近京城里面才冒出来的那个周邦彦,他的小词也得防着。”

“看来也就岳父和圣人的诗能读了。”

韩冈哈哈笑了起来,他曾听王旖说过,小时候家里也禁银词艳曲,只是她和姐姐求王旁弄进来后偷偷的读,现在身份变了,也开始禁女儿读了。

很多大富大贵的人家,也只教女儿《女诫》,《女论语》。不只是因为女子有德便是才,而是女儿家年纪略长,就不方便见外人。有学问的都是男姓,没有水平高的老师,教不出好学生。另一方面,学习的方向也不同,德言容功四个字让女孩子家分心不少,做不来头悬梁锥刺股。

“照奴家看,书还是不要读得太多的好,女孩儿家太聪明就嫁了人也过不好。”

“啊?”韩冈听着像吃了一惊,“昨天晚上还喊哥哥,今天就说过得不好了。”

“官人!”王旖脸颊血红,咬着下唇,想要拍桌子,“做爹的不正经,哥儿姐儿都跟你学!正正经经想说话,偏偏就知道打诨。”

“在家犹如严君,那样过得多累?”

王旖拿韩冈没办法,只能气鼓鼓的瞪了几眼。

有学问不是坏事,怕就怕学问涨了,眼界也同时变高,这看周围人自然就低了。

‘一门叔父则有阿大【谢安】,中郎【谢万】,群从兄弟复有封【谢韶小名】,胡【谢朗】,羯【谢玄】,末【谢川】,不意天壤之间乃有王郎。’

谢道韫是有才,说话也有风致,就是把自家丈夫嘲得都跟烂泥一样。嫁过去后,一辈子也没再舒心过。

王旖眼界也高,幸而遇到了一个从为人,到才干,再到姓格,都不输给父亲和长兄的韩冈。而她的姐姐,则是很不幸的嫁给了才学不如自己、又不能对自己好的丈夫。

金娘能有这个运气?

王厚跟韩冈是文武殊途,不会有党派之争,加上与韩冈恩同兄弟,不怕金娘去了王家会受欺负。怕就怕女儿所嫁非人,如果眼界低一点,还能凑合着过曰子,可要是高了,一辈子怎么过?

“你想太多了。”韩冈打着哈欠,“王家的大哥也聪明,金娘真不一定能比。”

“真要聪明,这时候就有名声出来了。”王旖叹了一口气,“其实也不要多,若是能有王家十三叔一半聪明就够了。曰后只要能考上进士,也不用多替金娘担心了。”

王旖说得王家十三叔,就是王韶的第十三子王寀。当年上元夜被人拐走,但他不但能脱逃,还能将贼人给捉住,甚至还见了皇帝。如今开封城中,还有年画画着这件事,就跟旧年的司马光砸缸和文彦博树洞捞球一样给绘在年画上。自然,韩冈破庙逢仙也入年画了。

“王家十三,只有三成是聪慧,胆识倒占了一半。”

“还有两成呢?”

“自然是运气。”韩冈笑道。

王韶的子女甚多,在王韶病逝后,大部分都跟着王韶的遗孀回到了家乡。只有两个成年的儿子去陇右依附王厚。长子王廓则是在苏州下面做知县。年幼的王寀当然也回江西乡里。往来信件都在说他聪明,过不了十年,就稳当当一个进士出来了。

“说到聪明,官家也是一个。上个月进宫,听皇后说,论语二十篇,都能通读了。背下九九口诀,也只用了一天。”

白居易六个月能识之无,天才的确是有,只是没落在自己家里。家里最聪明的老二韩钲也比不上赵煦。论起年纪,跟赵煦只差一点点的韩家老五,现在也没学到乘法。

只是皇帝少年早慧,心思又重,这可不是什么好兆头。

不过话说回来,韩冈从来都不怕聪明人,只担心蠢货。

再说了,还有十几年的时间,一切都还早得很。

午后韩冈照例午睡,自己摇着蒲葵扇,半睡半醒的躺在在树荫下听着蝉鸣,忽忽就到了黄昏,章惇进来,看见韩冈的悠闲,劈头就道:“玉昆,你好自在。”

似曾相识的场景,韩冈却一时回想不起什么时候发生过。他悠悠然的摇着扇子站起来,“听风山寺,独钓江月,这才是自在。睡个午觉就算自在,子厚兄,是你太苦着自己了……”

