\section{第37章 朱台相望京关道(03)}

曾布意欲留下独对。

向皇后虽理政不久,可也知道曾布的请求绝不正常,多半是有话想私下里说。她为皇后,内命妇、外命妇不知见了多少,想要私下里说话的,也有不少人在谒见之后,单独请求留下来。

难道是为了召回韩冈?

曾布和王安石的旧怨,向皇后好歹也是知道了。王安石既然一直都拦着不让他的女婿回来,曾布当然就会支持召回韩冈。

可是之前曾布一直都是反对的……

向皇后有点拿不准。

但不管到底是为了什么,既然曾布想要留下来,答应就是了。先看看他到底想要说什么,说的不好就不答应。这没什么好多犹豫的,宰辅们不再同进同退,在向皇后眼中,也是一件好事。

“既然参政有事要奏禀,那就稍留一步。”皇后不再迟疑,留下了曾布,便开始逐客:“诸位卿家若无事,就先回吧。”

……………………

肯定是为了召回韩冈。

曾布话出口后,王安石正要挪动的脚步就定住了,脸色越发的黑了。

虽然说召回韩冈就意味着把吕惠卿也一并召回,两位勋臣如今已经被连在了一起,一荣俱荣,一损俱损。之前曾布便是因为不打算让吕惠卿回朝,而一并阻韩冈于外。不过看曾布现在的样子,多半是宁可看着吕惠卿回朝升任宰相,也要把韩冈给拉回来。

在短短几天之内,曾布的态度完全颠倒,剧烈的变化,让王安石觉得有哪里不对劲。难道说已经跟河东那边勾结上了?还是说,一直以来都是故意反对,现在赶在节骨眼上来迎合皇后?

曾布当年在自己背后捅了一刀的旧事,至今王安石仍然是耿耿于怀,不免将曾布往恶劣处去想。

一见皇后逐客,王安石再也不能等了,踏前一步,向着帘后躬身:“殿下。臣王安石,亦有有一事需奏禀,请留对!”

王安石垂眼看着手中写了几句今日议题节略的笏板,上面完全没有需要独对的条目。但他现在必须留下来。否则曾布一提议,下面墙头草再一奉承,让皇后确认她在两府中还有支持者,王安石就再也压制不住底下的异动了。

纵然他还能挡住韩冈,可之后政事堂中有曾布大齤事小事都配合皇后的心意,根基一去,王安石这个平章军国重事,真的就只能在‘重事’上发言——究竟什么是‘重事’,下定义的却是垂帘的皇后。

不过现在还来得及。不管曾布想做什么?只要不能单独跟皇后议论,那么一切阴谋诡计都别想成功。

原本王安石阻韩冈于京外的心思并非这般坚定,可到了现在已经是骑虎难下,不坚持到底,就是他无法再安坐于朝堂。

纵然亲如翁婿,关系到一生的功业,也没有多少人情可以讲。

就如当年吕夷简对范仲淹,出去了就别回来了!

……………………

王安石的回应,让韩绛、蔡确、张璪、章敦等两府诸臣心中赞叹不已,釜底抽得好薪,完全不给曾布机会。

曾布完了。

既然王安石硬是插了一脚进来,皇后还能赶走王安石把曾布单独留下来?就是天子也不能那么做。

王介甫这一回连老脸都不要了。这时明着欺负皇后年轻识浅,无力掌控朝政。可偏偏皇后和曾布都无力做出应对。

韩绛冷眼看着曾布,‘做事前也不多想想。蠢一次不够,还要蠢第二次!’

韩绛当年第二次任相时,也曾自请留对。但那一次与这一回不同,当时他是把王安石请回来,就算当着众臣的面来说,也没什么好怕的,无人敢于反对。

可曾布今日不同,他的提议有太多能让人出手阻止的空间,完全不可与他韩绛当年相比。

而且王安石的犟脾气,曾布了解得还是不够多。越是用小花招,惹起的脾气就会越拧。当年吃下的亏,难道都忘掉了吗?

终究还是个无能之辈!变法时身兼十数职的风光,其实不过是有王安石在背后支撑,自身还是上不了席面啊。

无意多看那废物一眼,韩绛想知道,皇后会怎么说。

宰辅们当面互驳,非此即彼。一旦定出胜负,输家必然要请辞。可无论如何,王安石的平章军国重事是辞不得的,动静太大了。离开的只会是曾布。区区一个参知政事,不会影响太多。皇后如果还想保住朝中有一个体己人,这时候就该说话了。否则已经把身家押上赌桌的曾布就要完蛋了。

最好的办法还是说一声今天累了,有事明日具本奏闻。

说留曾布议事,现在也可以不留,这点小事上反口只是小问题。真要让曾布将他想说的话说出来,他可就要成众矢之的了,王安石更是绝不会放过。

只是向皇后心中的弯弯绕明显不够多,没有把曾布从危机中给解救出来,“卿家有什么想法尽管说,吾洗耳恭听。”

曾布瞬息间脸色数变,低头道:“前日蒙朝廷深恩,官臣第四子纡。惟念臣子年幼,不堪受此重恩。臣父早殇,得长兄巩教养方得成人。如今兄子绾,学问精粹,性情厚重,然至今白身。臣请殿下恩典,愿将臣子之荫转予吾兄子绾,以全臣兄弟之德,臣不胜感激涕零。”

皇后声音阴沉,“是吗,就为了这件事?”

韩绛低头笑了。

曾布反应够快的,但实质上还是退缩了。

自反而缩,虽千万人吾往矣的勇气,曾布看来是缺得太多。王安石当年能不顾众多旧友割席断交而坚持新法,曾布只是被王安石瞪了一眼就软了。

真是废物。

……………………

‘缩得好。’

蔡确心中还是松了口气。

曾布若当着王安石的面提议要召回韩冈,王安石自是会反对到底。

皇后左袒曾布没关系,但如此一来,王安石可就要辞官了。若是皇后畏惧王安石,反过来就变成了曾布不得不引咎出外。

可无论是王安石这个平章军国重事辞官,还是曾布这个参知政事辞官,都不可能不告知还在病榻上的天子。

只要皇帝还卧病在床,王安石的地位将稳如泰山——毕竟垂帘听政的只是皇后,而不是太后。但要让皇帝开口让曾布辞官,之前一切的谎言却也将无法遮掩。从皇后到宰辅哪一个都逃不过罪责,善意的谎言也是谎言。何况到了后来,已经完全分不清是善意的掩饰,还是纯粹的不想让皇帝再接触朝政。

万一把赵顼给气到了,皇后罪过最大,三从四德都不遵守,如何母仪天下?若是没气到,那问题就更大了,宰辅们离得远,还能躲一躲,皇后往哪里躲?

届时不是皇后把天子给管束起来,就是天子改立皇后。不管会变成什么局面,对已经坐上宰相之位、已经没多少空间可以晋升的蔡确来说,都不会有太多的好处,反而是危机重重。千金之子,坐不垂堂,维持稳定才是他的利益所在。

这样撕破脸皮的手段,还是不用为妙。皇后和宰辅们互相妥协,这样才能维系朝廷的安稳,也能让天子安心养病!

幸好曾布缩了。

如此一来,大齤事抵定。连参知政事都无法动摇王安石,那么谁还敢为韩冈出头?韩冈既然回不来,那么吕惠卿又如何回京?

向皇后终究是拗不过拗相公,至少今年年内,吕韩二人他们别指望能回来了。

……………………

曾布败了。

刚刚得到消息的蔡京看了看桌上刚刚修改好的草稿,无可奈何。精雕细琢的文章,现在已经用不着誊抄了。就着灯火,直接点了。

看着稿纸烧得干干净净,蔡京轻叹了一声。王安石以他的执拗的脾气,硬是将曾布逼退,韩冈只能继续留在河东。

“曾布这一回在京城留不住了。”强渊明走了进来,一脸的兴奋,没注意房中还有着的淡淡烟味。

丢了这么大的脸,曾布若是还不请辞,御史台可就有活干了。

“不可能的!”蔡京摇头,“别忘了,曾布终究还是天子钦点的参知政事。”

就算王安石想要把曾布给赶走,他也得担心曾布破釜沉舟把事情闹到御前。

“即便乌台、谏院齐齐上本弹劾,王平章都得要保他。”

强渊明稍楞,“什么事都没有?!这可太便宜他了。”

“什么叫什么事都没有?中书门下自此而后,曾子宣他说话还有多少分量可言?”

终究只是首鼠两端的货色。远比不上章敦能另辟蹊径,也比不上的蔡确会看风色,更比不上吕惠卿的一意到底,当年市易务一案,已经让人看透了他的本性。

章敦能改走军功一途,吕惠卿把变法坚持到底,蔡确早早的就改抱了皇帝的大腿,只有曾布,在最坏的时机,做出最蠢的事。

“可惜了。韩玉昆这一回可是要在河东长住了。”

“有什么好可惜的,难道元长你想喝韩枢密的寿酒不成?”

三十岁的枢密使啊!远在天边倒也罢了,近在眼前岂不是让人心中堵得慌?

蔡京笑而不言。

虽然失去了一个难得的机会,但韩冈不回来不算是坏事,免得心中不痛快嘛,终究还是有好的一面。

因为崇政殿中的小小动荡,朝堂上一下变得清静了,一时之间,再也没有

直到三天后,河东制置使司一封奏报传来:

折克行大破叛国附贼的黑山党项诸部,斩首三千余。河东制置使韩冈为其表功请赏。
