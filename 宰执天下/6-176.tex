\section{第19章 登朝惟愿博轩冕(上)}

“首鼠两端……”

“章敦本与韩冈沆瀣一气,岂能依靠……”

“父子皆无士行……”

“不是其暗通韩冈,楚公当年如何会被迫出外……”

章敦边说边笑,龚原却是冷汗涔涔。

现任枢密使那挂在嘴角的淡淡笑容,在他的眼里,比暴怒时还要恐怖。

背着章敦,他可以和台谏中的同僚一起大放厥词,可当着章敦的面,却一句话也说不出来。

龚原甚至希望章敦大发雷霆,而不是现在的笑语盈盈。

他一向自觉胆大,并不畏于权势,只是章敦现在的表情实在太瘆人,让龚原不寒而栗。

章敦也终于收敛了只挂在脸皮上的瘆人笑意,眼神却变得更为狠厉,“敦本俗吏,居西府多年而无所建树,不得人心也不足为奇。可家父无辜,年已八旬,却还要受不肖子连累。”

龚原汗如雨下,无言以对。

御史台中几个人在一起骂章敦胆怯,以至于贻误良机,掉过头来,章敦却把那些话一个字不差的说出来。

不用想,这几个人中间,肯定有人向章敦通风报信了。

可急切之中,龚原想不出到底是哪一个……也许不只一个。

龚原只觉得自己背后又黏又湿,越来越难受了。

不但前面有敌人,背后也有敌人。早知道在御史台中做得如此憋屈,还不如留在国子监里。可惜自己在当年的太学案中吃了大苦头,尽管三舍法有自己的一份汗马功劳,但再想回去,也不方便了。

章敦看见龚原脸色发青,倒也不再逼他了。

这一群人,是靠了自己才能在御史台中站稳脚跟,可他们不思图报,反而在背后议论。这样下去,说不定过些日子,就会上表弹劾自己,以示身为御史的忠直。

这样的人,还能留吗?

当然,章敦并不打算将自己在台谏中的布置一扫而空,有的人是不能留,有的人还是可以再看看。

“你们啊……是利令智昏!真当韩玉昆不敢赶你们出去?”

龚原与王安石关系很好,当年变法,三舍法多得其力,在国子监生中很有地位。章敦觉得他还可以挽救一下。

章敦松了口,龚原却不服气的低声说了一句:“纵使宰相也不能随心所欲驱逐台谏!”

“如果要太后决定谁去谁留,太后会留你们吗?”

龚原呐呐难言,太后的态度谁都明白。

章敦冷淡的看了龚原一眼。这样的人,只知道添乱,且不是给对手,而是自己人。

“知道韩玉昆为什么当初不阻止你们进入御史台?……是因为你们坏不了事!”

“韩相公权势煊赫,我等无力拮抗,可枢密身居西府多年,又何必惧他?”

“我为什么要从尔等所愿,与韩冈为敌?!把韩冈赶走之后,靠你们帮忙,能把国事处理得比他更好?”

章敦当然想进政事堂,但他不希望自己进去之后,天天与人打嘴仗。

“我等虽不如韩相公多才多艺,可枢密若能进东府,岂会输给他?”

“工役、财计、军器,这些事我远不如韩冈。人贵自知,正是有自知之明,我才能在京城留到今日。”章敦微微冷笑,“深甫,你向来实诚,这挑拨的事情,你做不来的。”

龚原的脸一下胀红了,他方才说话的时候,的确带了挑拨离间的想法,挑动章敦的心思,“可韩冈当政以来,便大兴工役,劳师动众,年年不绝,地方上早已是民怨沸腾!”

“年年兴修工役,却不见百姓揭竿而起。”“你们搜集的那些东西,烧掉都嫌要扫灰,什么用都没有!你好好想想吧,为什么韩玉昆将铁路定为御道?!”

“铁路的收益并不是全数归于国库,而是有一半进的是太后的钱袋子。这天大的好处,放在变法时,不知要敲锣打鼓说上多少遍,可韩三提过几次?”

铁路轨道是朝廷建的,所以运输收费也归入国库,不过其所收取的商税税入则直接送进内库之中。不管之后政事堂会不会拿着国债债券,从内库将钱给挖出来,太后那边是实打实的看见钱入账的。

可是韩冈在呈与太后的一系列有关铁路轨道的奏章中,只有不到四分之一,提到了财税收入。提及保证纲粮稳定运输,占了三分之二,而军事用途,几乎每一份相关奏章中都有提及。

在韩冈的议论中,铁路轨道赚钱只是次要,仅仅是贴补一部分修筑铁路的之处。真正作用,是要在七八天内,将上万大军连人带装备送到千里之外。旬月之内,百万石纲粮从江南运入京城。

这是铁路的真正用途。既然大宋此前能年复一年的疏浚被黄河泥沙淤积抬高的汴水,能花费国库收入的六成来供给军用,那么修造铁路,保证京城的安稳,让国境上的守军能在最短的时间内得到援军,动员千万民夫,花费百万钱粮,都只是一桩小事而已。

而且在韩冈直接控制之下,几条铁路同时铺设,动用的民夫成千上万,却也并没有造成士民沸腾的局面,一切都顺利的进行中。这不是三五人搜集几份材料,就能扳回来的局面。

干线铁路的顺利,使得从东京城通往各县的支线道路早早的就进入了筹划阶段。由于京城的地皮极贵,经过的土地,田主也是成百上千,所以路线还在扯皮。但经过朝堂上的几次会商,决定允许田主购买股票,成为股东,以股权交换地权。

经此一变,京师的世家大族更是群起云涌,也许要经常会被军事占用的铁路轨道干线是个赔钱货,可联通京师的支线,想也知道,肯定是大赚特赚。不说他事,只是朝廷允许铁路沿途站点上可以自行设立墟市,就知道其中有多少油水了。

控制了如今最大的一块肥肉,韩冈正是如日中天,想动摇韩冈的地位,绝不是在现在。而章敦也不会糊涂到与现在的韩冈为敌,所以当他发现下面的人有所异动,才分外不能容忍。

还好,相比起其他几个人,龚原坏不了事,留他一个,也能搪塞一下。

章敦瞥了一下眼前的中年男子。

龚原此刻半是羞怒,半是迷茫,对章敦的话也没有回应。如果是辩解经义,他能滔滔不绝,半日不歇,可说到朝事政事,可就只能算是一个庸才了。

还是留着他吧。章敦进一步坚定了想法。

无伤大雅,无害于国,更确切一点,就是韩冈曾经说过的人畜无害了。不留他,难道留与自己同名的安处厚吗?

点汤送客,章敦在空无他人的厅中,只想叹气。

这些人,看着廷推在即,便一个劲的想要兴风作浪,也不看看局面。

政事堂只有两名宰相,已经必须增加人手。之前两次廷推无果,这一回,不会再拖下去了。

按照与韩冈、苏颂的商议,这一回是四选二,但最终的结果还是看太后。廷推的前四名可以送到太后的面前。太后可以在其中选两个,但也可以只选一个,或是一个都不选。

这两个名额之中,韩冈是肯定想要一个自己人来占据。

苏颂六十多了,不过身体好,又会保养,看他的样子,应当不会比富弼活得短。富弼八旬才去世,而文彦博也八十多了,还活得开开心心,这两人都是朝中让人羡慕的老寿星。

不过即便苏颂能活到八十多,他在朝堂中的时间也不多了。

朝中公认的致仕年龄是七十,也有律条规定,但也不是那么死板,一般来说能活到七十的不多,七十岁还没病没灾的更少,退与不退只看有没有人说。

当年曾公亮年过七旬仍留居东府,就是被一句‘老凤池畔蹲不去,饿乌台前噤无声’给骂走的。苏颂如果过了七十岁还不恋栈不去,自然会有人写新诗送他。

高处不胜寒啊。

现在的情况,苏颂一去,韩冈便是要独木擎天。本已是困难重重,日后将更加困难。他根基不厚,这是没办法的事,先天不足。像他这样父祖皆是庶民,靠着自身的努力跃过龙门的官员,想要独树一帜,自成一派,本来是几乎不可能的一件事。

不是韩绛、韩琦这样的世家子,想要在朝堂上长久立足,有所作为,就必须厚植根基。王安石是官宦世家,可他的父亲也不过是个中层,普通进士,所以根基不深,只能靠学术来聚集人才。

韩冈也是一样,不过他的儿子多,等他开始与人联姻之后,韩家的地位就能在朝堂上稳固起来了。

但这是缓不济急,韩冈目前重用的都是愿意跟着他做事的人。

黄裳依然在西南,李承之留在河北,改知大名府,署理河北防务。游师雄留京数载,也该出外了。他们都不可能被韩冈选上。而在铁路轨道营造上涉足甚多的沈括,必然是韩冈力推的人选。

如果有人这时候想从铁路上下手,韩冈绝不会容忍,那时候,好不容易才稳定下来的朝堂肯定有得乱了。

作为新党如今在朝中的领导者,章敦绝不愿意看到那一幕。

………………

“相公可还看过了司马光的遗表?”

内东门小殿中,向太后向韩冈问道。

“回陛下,臣已经看过了。”

“相公觉得如何?”

“其中多有激愤之言。”太后语气愤愤然,可韩冈心平气和,他要操心的事太多,没精力跟死人怄气,“人若挟怨,观人观物便难以公正。至于其说变法误国误民,臣等朋比为奸,陛下只看二十年前和如今的区别,就知道是否是事实。”

“但司马光临死都不忘上表污蔑,给侍中衔,是不是太高了?”

“如今司马光既已无害于国,就不宜太过苛责了。”

相对于司马光在洛阳一待十几年的悲剧,他死后的封赠可谓是备极哀荣。韩冈和苏颂商议过后,在太常礼院拟定的规格上,又加了一级。

朝廷给予司马光的追赠,远远超过了他应有的地位。作为司马光的嫡子,司马康也特旨加官一级。

都已经死了,给他多少好处都没关系,坏不了事了。至于会不会让人误会有什么政治意味,韩冈也并不在意。

“就依相公。”

向太后看起来还有些不快。不过给韩冈劝了一下,也没有多坚持。

就像韩冈说的,已经没法儿再败坏国事了,只冲《资治通鉴》,给个侍中也不算过当。

劝说了几句,韩冈见太后无他事吩咐,便告辞退下。

出来之后,韩冈心中犹有几分疑惑。

这几年,向太后处置朝政早已得心应手,今天找韩冈来说司马光的追赠,韩冈并不觉得她是被司马光的遗表给气的。只是把事情想复杂了,又不像是太后的性格,一时间,不容易想明白。

韩冈想得很开,想不明白就不去多想,太后迟早要说明的。

五年之期已经没有多少日子了,他当初制定的目标,现在正在逐一实现中。

户数和丁口稳定的增长,税赋也在稳定的增长,轨道的运输费用还没有到收入的时候,但冲抵日常开支,也不至于亏本。而修造轨道的支出,并没有超过朝廷的承受能力,在铸币局的运作下,朝廷的铸币数量大增,物价却保持稳定。只要工业品和粮食都保持相应增长,国家就能保证稳定的发展。

来自岭南的纲粮现在占了每年收入京师的纲粮的五分之一,而供给民间的粮食则更多。其地位重要已经不下于江南任何一路。

所以韩冈一直最看重的交通线,并不是铁路轨道,而是来自岭南的海运路线。同时对辽的前期战略,海战也占了很重要的一部分。在军事准备中,海军也是重中之重。

海军建设与轨道一样,这都是要砸钱的生意,韩冈是用自己的威望来推行此事。

而想要维持现有的政策不加变动,必须要有一个有见识的政府,两府中的新人人选,也必须加以考量。

想到这里,韩冈心中又是一动。

太后方才专为司马光找自己过来,是不是有所暗指?