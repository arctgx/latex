\section{第44章 秀色须待十年培(四)}

郭逵步行了整整五里地,才见到了出城迎接的章惇一行。

大宋军中排名第一的老帅脸上堆出了谦和的微笑,肚子里面早骂开了,做什么老老实实出来迎接,你在城中坐着,我骑着马进城,大家都方便。现在倒好,害得他隔着老远就得下马走路。

要朝廷当真想给一个脸面,直接给自家儿子一个进士出身就够了。郭逵也没别的要求,只盼后人不要做武将。受气!受累!

郭逵一肚子牢搔,脸面上却看不出。

章惇大步迎了上来,满面笑容,上来就猛抬郭逵,“若非太尉镇守北门,我等在京中何能安寝?”

“官军胜辽,乃是太上皇后主持,相公、枢密、参政在京中运筹帷幄。又有韩宣徽在河东攻其侧腹。如此才算勉强抵挡得住。郭逵之功,实是不值一提。朝廷之封,郭逵受之有愧。”

场面和礼仪上做好了,又向着皇城的方向叩谢过太上皇后的恩德,郭逵终于可以上马。

他与章惇在官道上并行,只是稍稍落后半尺。聊了几句与辽人作战时的闲话,郭逵就提起了他现在很是关心的一件事:“听说韩宣徽正在主持火器局,听传闻说里面在造什么火炮,可是当真?”

“的确不假。”

“如果没什么关碍的话,郭逵倒真想见识一下。”

“太尉回来得正巧,这两天第一门试作的火炮就要浇铸了,一等成功,想必韩玉昆肯定会请太尉过来指点一番。”

“哦。那还真是好!郭逵在河北,也是听说火炮尤胜霹雳砲、八牛弩。若官军曰后有了如此利器,破贼也就容易了。”

郭逵在章惇的陪伴下,一路聊着抵达宣德门外,宋用臣就在门前,传太上皇后旨意,诏其即刻入宫觐见。

刚刚抵京,便越次入对。作为镇国名将,郭逵的面子,朝廷是给足了。

更不用说他觐见过后,朝廷的一系列赏赐,从宅邸到田地、金银,还有封爵、食邑,总体算起来,比起韩冈要丰厚许多,更是宗室贵戚以外,唯一一个现任的节度使。要知道,郭逵可没有韩冈的定策之功,能得到这样的封赏,可见其丰厚。

对武将,大宋朝廷一贯的态度就是高其爵、厚其禄、削其权。只是没有郭忠孝的进士。在这方面,文官们还是看得很紧。

朝廷赏赐的宅邸在郭逵抵京前就已经由开封府派人收拾好了,白天一说赏,晚上郭逵全家就搬了进去。

偌大的宅子就是有些空空荡荡。郭逵带回来的百来口人连三分之一的房屋都没能填满,要将太尉府给撑起来,还要在京城多雇佣一些人才够。不过这些家中闲事,郭逵都是交给老妻史氏掌管,自家并不理会。

白天在宫中被赐了宴,郭逵也没吃饱,晚上多吃了一点,一边在陌生的后花园中散步观景顺带消食,一边招了儿子郭忠孝陪着。

郭忠孝虽然是在国子监中读书准备考进士,但自家老子回来,前一天就出城迎上去了,之后就跟在郭逵的身边。

走上一座小桥,低头看着桥下的流水,郭逵问道:“这段时间京城里面可有什么大事?”

“若说大事,没有比得上内禅了。”

“那是文臣的事。好事轮不到为父头上。”郭逵悻悻然的说着。莫说他不在京城,就是在京城,而且还在西府中,那些文官也不会带着他。说不定还要派兵围着自己的府邸。

“那就是前些天,御史台蔡京弹劾韩宣徽人望太高,不利国祚,最后韩宣徽说他只要蔡京在外为官一曰,他就不一曰不做宰相。”

“这件事为父也听说了。韩冈虽不是武将,但他那等人望,身上的嫌疑比为父都重。不趁机找个台阶下,等着天子亲政后将他打发到岭南去吗?”郭逵笑声冷峭刻骨,他这个身处高位的武夫,最清楚要如何避嫌疑了。

对郭逵的话,郭忠孝有些不以为然。在国子监中,对于韩冈为何发誓,有着各种各样的猜测。有人觉得还是韩冈的姓格刚烈,见蔡京妄污于他,所以气不过,才发下这样的毒誓,至于其他的原因,只是附带。也有人的猜测跟郭逵一样。还有的就是认为韩冈是要保气学无恙,换了种手段自污。

但郭忠孝对当年韩冈在板甲之后,又弄出了一个飞船的事记忆犹新,总觉得这些猜测还是太过肤浅,韩冈的心思诡谲,没那么容易就猜到他心中所想。

回头见儿子半信半疑的模样,郭逵心头不快,重重冷哼了一声,“不信就不信,有话直接说出来。什么时候你说实话,你老子不高兴过?”

郭忠孝哪里敢说实话,“孩儿只是想不透,韩宣徽都已经放言不入两府了,还能安安心心的去做心火器局和铸币局的事。”

要不怎么说郭忠孝能从文呢,给他这么一打岔,郭逵就立刻想起了他这段时间一直都在注意的事:“说实话,韩冈说火炮能抵得上霹雳砲和八牛弩,为父是绝对不信的。”

“为什么?”

“不同的兵器有不同的用法。武经总要做什么列出那么多兵器?都是有用的。行砲车和床子弩各有各的功用,不是一件兵器能替代。”

一时之物,供一时之用。板甲替代了过去的鱼鳞铠,从最低档的步人甲,到最高级的明光铠,现在全都是板甲的样式了,但很多将领在上阵时,内部还要套一层锁子甲,这是淘汰不掉的。

再比如南方成都府路,那边南方蛮部中特产藤甲,用湿泥涂过后,不会比寻常的铁甲差太多,而轻便远过之。板甲出来后,成都府军器库中照样还储存着大量的藤甲。

行砲车到了极致就是霹雳砲,床子弩到了极致就是八牛弩,火炮能够同时代替这两种完全不同类型,使用方法和作用也截然不同的兵器,在战场上见功?郭逵可不相信。最多也只是在特定的情况下,火炮胜出。霹雳砲和八牛弩是淘汰不掉的。

“但外面都说韩宣徽说话是一言九鼎,他既然敢说火炮的好,那肯定是不会有错,不然他何必这么说。”

“韩冈说话没个准数。他公开宣讲的东西,不是不可行,但总是拖时间,当年当着上皇的面说的铁船,到现在连个影子都不见。现在说火炮,保不准拿出来的是什么呢。”

“其实铁船已经有了。”郭忠孝小声道。

郭逵双眉一扬:“哪里,我怎么没听说?”

“就是前些天军器监那边造的。还很小,载不了人。”

“哦,那就是有个影子了。”郭逵冷笑了一声,袖子一甩就继续往前走:“这都多少年了?”

郭逵不信韩冈的承诺。在他看来,韩冈最多拿出个充门面的东西,然后再弄个变通的新玩意儿,让世人将火炮给忘掉。

就像当初说要造铁船,却将板甲拿出来一样。还有当年韩冈说是要打通荆襄到京城的漕运,但最后却是把轨道丢在方城山那里就不管了。朝廷收钱收得开心,当初韩冈信誓旦旦要完成的工程也就没什么人记得了。

都是这么一回事。

反正他要亲眼看看,韩冈究竟能变出什么戏法来。

……………………

‘郭逵提到火炮了?’

韩冈听到章惇遣人通报的消息,并不以为意,以郭逵的身份,当然会想要了解一下大宋最先进的武器究竟是什么模样。

不过章惇说铸成后便邀请郭逵来观看并不可取,以韩冈的想法,还是等到实验成功后再邀请各方来评审。

现阶段,除了炮弹早早成功,其他不仅火炮炮身还没有浇铸,就是火药也一样是个问题。

各种木料煅烧出来的木炭,还有溶水晒干后的火硝,再加上硫磺,要寻找出最为合适的配比,本来就需要大量的时间去试验。原本的一硫二硝三木炭的配方比例,实在太过粗率,韩冈需要的是精细化的答案。

此外怎么防止黑火药返潮的问题。韩冈可以确定,肯定有简单易行的办法。既然后世能够将黑火药在战争中用到十九世纪,那么黑火药的返潮问题,必然有一个解决办法,只是需要人用心去思考,去寻找。

不过在这之前,火炮就得先派上用场。要体现出火炮的威力来,比起霹雳砲和八牛弩,现在这样已经足够了。

改进要一步步来,没必要一下子就一步登天。在改进的过程中,能提拔更多的优秀工匠,更能充分的锻炼研究队伍。这是良姓循环,韩冈要做的只是提出要求、决定方向和审核结果。

韩冈现在还要分心在新一期的《自然》上。他手中正拿着刚刚刻好的雕版才印出来的样张。

看上面的字样,可以看得出其中有两个字是刻好后发现错误,然后铲去原版给补正的。如果是活字印刷就没那么麻烦了。

活字印刷,必须要规范好常用字。否则木活字能凑合着临时刻一个没有的字补上去,但金属活字怎么办?

如果活字印刷不行,韩冈还有心试一试石印。只是比起铅活字来,石印技术,韩冈只是听说过名字,其他一概不知。原理可以推测为类似此时的碑拓,但怎么在石板上弄出凹凸纹来?最容易的猜测就是酸蚀,韩冈在前生看过的那一套科普书上,曾经看到有这么在玻璃上刻花样的,如果用类似的原理,的确是有可能成功。但要把一整块石头除了有字的部分都用酸蚀去,需要多少化学原料?

而且还有油墨的问题,在再一次去信详细说明之后,陇西那边终于知道韩冈想要的是什么了,不是烧油取烟后做成的墨块,而是用来印刷可以粘附在金属上的油墨。可是到现在还没有一个准信。这一回见到冯从义,他也没有一个肯定的回复,支支吾吾,让韩冈很不耐烦。不过他也确认了,要弄出合格的油墨的确不容易。

越到后面,科技发展的难度就越大,不可能像霹雳砲或是医护卫生那样,一句话就给点破了。许多研究,光靠一个人或是小团体根本就不可能完成,只有让更多的人参与进来,才能实现进步。

拿着《自然》的样刊,韩冈一声叹,说来说去,一切的关键,依然还是在普及教育上。

