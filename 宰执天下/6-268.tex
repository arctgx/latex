\section{第36章 骎骎载骤探寒温(四)}

城上的守军顿时骚动起来,还没有等到命令,零零星星的箭矢便飞下城去。

“住手!”宗泽连忙喝止,仅仅五六骑的规模,不会是过来攻城的,“看看贼人有什么话说。”

一名信使被守军用筐子吊上城来。

连城门也不敢开,城中的心虚气短表露无遗。当信使走到景诚和宗泽面前时,整个人举手投足都能让人联想到趾高气昂四个字。

“圣公有令……”

“斩了!”

信使刚开口,景诚便一声怒喝,他身后的亲兵立刻扑出去,将信使扑倒在地。

景通判翻脸如翻书,突变如兔起鹘落,周围官兵都看得眼晕,不知景诚唱的哪一出。

那信使拼命挣扎,叫道:“两国交兵,不斩来使!”

“斩了此贼!”景诚大喝,“说书听多了,贼子也敢称使节。”

让贼人报上名号就够了,剩下的多听一句都嫌污耳朵。

如果在仁宗时代,搬出牛酒犒劳贼人,祈求其高抬贵手的官员,还能留下性命。

现在的地方官要再这么做,朝廷就算要留他体面,也只会是免了枭首一刀,白绫、鸩酒伺候。

二十年来的累累武功,民间也好,朝中也好,风气早就变了。对外敌、对内贼,只要态度稍软一点,那就是无能,少不了受斥责甚至罢官夺职。景诚直接了当的表态,便是不想落人口实。

“圣公……”景诚回顾宗泽,“看来贼子是蓄谋已久啊。”

“是啊。”宗泽点头,“狼子野心,于今昭彰。”

建制定号,坐实了反贼的身份。这一下子,责任彻底由卫康担过去了,所有对韩冈新政的非议,便可以彻底洗清。

就在城头上,信使被景诚的亲兵一刀站下了头颅。狂叫戛然而止,只剩噗噗的喷血声。

围观的官兵,基本上都是除了鸡鸭之外,没见过血淋淋的杀生场面。北方时常会围观刑场,南方却不多见。就在身边看见活生生的人被砍下首级,好些个士兵都吓软了脚。

宗泽虽为南人,但类似的场面还是见过不少,他倒是惊讶起景诚的这几位并不起眼的亲兵来。

一刀断首,刀法如此利落,非是积年的侩子手或是久经战阵的老卒不可为。宗泽用心打量起这几位亲兵,一个个相貌沧桑,皆是有别于南人的精悍。

“拿弓来。”

在一旁,景诚命人拿来了弓箭。借着一点亮光,对准还在城下的几名贼人,他张弓搭箭。

一声弦鸣,宗泽惊讶转头,只看见景诚持弓而立,城下一声惨叫悠悠传来。

“再来!”

景诚大喝,接箭张弓再射,又是一声惨叫窜起。

一柄长弓连张连射,惨叫声此起彼伏。景诚一箭一人,五箭之后,城下又重新陷入了黑暗之中。

宗泽鼓掌赞道,“好箭术,家学渊源,果然了得。”

“微末之技,不足挂齿。”景诚面无得色。

他的祖父景泰,是少见的文进士转武职的例子。

大宋文武殊途,朝廷中有文不换武的说法。文臣愿意领兵,但没人愿意转为武职。

当年党项叛乱,范仲淹、庞籍、韩琦等重臣前往前线镇守,仁宗皇帝便打算将他们转为武职,可以更名正言顺的领军。

但范仲淹和庞籍都找了借口拒绝了,而韩琦虽是接了圣旨,可还是委委屈屈的上奏表说,‘虽众人之论谓匪美迁,在拙者之诚独无过望,盖以寇仇未殄,兵调方兴,宵旰贻忧,庙堂精虑,使白衣而奋命尚所甘心’——虽然不愿意接受这个任命,但因为敌未灭,战方酣,天子和朝廷也夙夜谋划,他也只能起一起表率作用,以激励人心。韩琦在奏章中便是这么一幅相忍为国的姿态。

要么直接拒绝,要么便是满腔幽怨,故而不久之后,对臣子一向宽容的仁宗皇帝,便收回了这道诏命。

相较而言,景泰老老实实的从进士转武职,在重文贱武的朝堂上,真可以说是一个异数了。

但景家也由此转成了将门,从此离开了士大夫的行列,所得所失,只看景诚费尽心力去考进士这一事,便可知端的。

就宗泽所知,其实种家也有让自己子弟转换身份的想法,可惜种家实在没有有望皇榜的读书种子,即使其中有一个还算聪明的,拜在了当世大儒门下,与当朝宰相同窗共学,也只挣到了一个诸科出身,如今还回到了继承家业的旧路上。

经过了一番努力,终于从将门挣扎出来,重新回到了文官的队列中,景诚很少愿意提及自家的累累军功。中进士后,枪棒功夫也放下了。不过文官习练射术,却是如今风气,他便一直在练习。也幸亏如此,否则也没有方才的连珠箭。只是方才一展射术,神情依然淡淡,不见喜色。

景诚的反应虽是寡淡,可周围的官兵早已看得目瞪口呆。

火器还没有在南方军中普及,弓弩依然是军中校演的重点。景诚的箭术放在北方军中或许只是不错,但在南方,却已经是神乎其技了。

一名年轻的士兵兴奋的涨红了脸,振臂高呼:“通判威武!”

一名老卒眼神中充满了敬慕:“通判威武!”

一名跟着长官上城的小吏挥舞起细弱的臂膀,尖声高叫:“通判威武!”

几名亲兵相互交换了眼色,长枪开始一下一下的杵着地面,极富韵律的应和起来,“通判威武!皇宋万胜!”

南门城头上的士兵,一个一个加入进来,开始杵动他们的长枪,开始挥动起他们的臂膀,“通判威武!皇宋完胜!”

上百条长枪齐齐起落,他们心潮澎湃,他们意志如钢,“通判威武!皇宋完胜!”

咚咚的跺地声中,城头上,越来越多的士兵加入到呼喊的行列。从南门城头,沿着城墙,向东西两侧延伸过去:

“通判威武!!皇宋完胜!!”

“通判威武!!!皇宋完胜!!!”

片刻之后,已是全城齐呼。景诚方才炫耀的箭术遍传城中,虽然贼人尚在,城中的士气已经截然不同。

宗泽暗暗一叹。

若是有三百精兵,借着方才的一股锐气,就能杀出城去。城外的乌合之众,乍闻城中高呼,必然心怀犹疑,决然抵挡不了此时的袭击。

只可惜,城中守军亦是乌合之众,多少人连神臂弓都拉不动,上弦的机器不仅数量稀少,还都是坏的。

“五郎。”亲兵中最老成的一位悄然走上来,附在景诚耳边说了一通话。

宗泽见状,避嫌的让出了几步。

就见景诚的眉头越皱越紧,最后对那亲兵问道,“可有把握?”

亲兵低声道:“六七成总是有的。实在不行,也能退回来,不怕贼人追。”

犹豫了半刻,景诚眉头舒展开,点了头,“也罢,就这么办吧。”

让亲兵下去准备,景诚走回到宗泽身边。

宗泽轻声问,“何事?”

“我打算命人出城去冲一冲。明教妖贼刚刚起事,人心尚未归附,城外贼众心中定然不稳,若猝然受袭,必然大乱。”

“可有把握?”

“我那几位家丁,几乎都是是跟着先祖、先父和先叔父上过阵的,无不是弓马娴熟,武艺出众,把握不可能十足,但七八成还是有的。”

宗泽略一思忖,便拱手一礼,“既如此,小弟便祝兄长旗开得胜。”

得了宗泽首肯,景诚随即召集城中众官,将计划合盘托出。

方才景诚引弓杀人,着实将底下的一众官吏给镇住了。现在他说要派兵出去冲杀一番,竟然没有一人出来反对,绝大多数都表示赞成。

在这个节骨眼上,本就没人还能顾着争权夺利,景诚这位将门世家出身的通判,此时又表现出了过人的武艺,哪个不把希望放在他身上?生杀予夺的指挥大权顺利的给景诚拿到了手中。

宗泽本来还打算用自己的钦差身份来帮景诚一把,现在既然不用他多事,宗泽便退避一旁,看着景诚指派。

就在城楼上,景诚将任务一一分派下去。

半个时辰之后,百名应募而来的敢死之士,业已穿戴整齐,由景诚的八位家丁领着,排列在城门后的小广场上。

这些勇士一个个身上都披挂了铁甲,外面还套了一件甲衣,用来防止胸甲上的反光,头盔上的盔缨则换成了一簇笔挺的白鹅毛,用来识别身份。

景诚一身铁甲,手扶腰间长剑,笔直的挺立于他们面前。

在景诚身侧,是整整一箱新出的银钱,又用牛拉了整整一车绢帛。加起来近万贯,全是从城中大户手中募捐出来的犒赏。

宗泽立于城头,向下俯望。

只看见景诚不知说了什么,百名勇士一起高呼了起来。又见景诚捧着酒坛上前,亲自给每一人都斟上了一碗烈酒。

不愧是名将世家。宗泽不禁叹息。

相隔百步,当上百人同时饮下烈酒,摔碎酒碗的时候,宗泽犹能感觉到在那里,士气沸腾,战意如火如荼。

东门城墙处猛然灯火尽灭,片刻后方才又亮起,而那队勇士,则悄悄地从南面城上陲了下去。

景诚回到城头上,走到宗泽身边,一言不发,静静的望着城外。

宗泽也没了说话的兴致,一同望向星火满点的夜色中。

寂静中,平静的夜幕忽的起了一片涟漪,星星点点的火光忽然间在边角处黑了一片,然后喊杀声便传上了城头。

一支支火炬落地,一丛丛篝火熄灭,区区百人的队伍,在城外的贼军中掀起了一片惊涛骇浪。

景诚回头城内,千余士兵已经在城门后列队等候。虽说此辈多不堪用,但借着胜势,赶敌军,已经绰绰有余。

景诚举起掌中长剑,奋声高喝,“出兵!”
