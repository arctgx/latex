\section{第六章 见说崇山放四凶(九)}

尽管实际如今的御史台,已经被宰辅们控制在手中。可在明面上,推荐和决定御史人选的权力,宰辅们依然插手不得。

没有哪个宰辅能够自己直接上书,说某人适合做御史,这不是推荐,这是把自己也搭进去的陷害。

同样的道理,即将重归宰辅行列的韩冈,也不方便公然插手宰执的候选名单。

在暗地里撺掇,通过言辞来引导,推动某个人上位是可行的,也很常见,韩冈也做过。可当太后将人事权放到他面前,韩冈却不能不退避三舍。

“陛下既然属意臣,那两府阙额,便不宜再由臣推荐。”韩冈的回覆无比郑重。

“卿家如今可不在两府中。”似乎是找到了韩冈话中的矛盾,向太后的声音中隐约带了些微笑意,“这不正是要卿家推荐吗?嗯?”

韩冈只能苦笑。

话不是这样说的。

现如今从太后的话中来看,如果没有更深的用意,应该就是全心全意的信任自己。

但韩冈即便不用为日后考虑,他手中可以举荐的人选也显得太少。他对气学门徒的培养,还远没有到收获的时候。

能够晋身两府的几人中,游师雄年资尚浅。而资历合格的苏颂已入西府;至于沈括,韩冈在确定他的是否又犯了老毛病之前,不愿意举荐。

除此之外,还有一些曾经有过来往的上司、同僚,比如在河湟时的秦凤路转运使蔡延庆,在白马县时的开封知府孙永,在京西时的京西北路转运使李南公,以及现如今还在关西的赵禼,人数不算少,可惜他们都不适合。

而且若是保证日后太后的信任不再,再回想起此事时不会翻起旧帐。同时也为了自清,以避人言,韩冈还得顺手举荐一个自己看不顺眼,对方也看自己不顺眼的人选。可这不是吃饱了撑的吗?自己给自己找不痛快?

不过……韩冈抬起头,透过单薄的屏风,望着屏风后模糊的身影。太后明显正在兴头上,泼冷水也也不合适,那可是敬酒不吃吃罚酒了。

“韩卿?”

见韩冈迟疑,屏风后的太后催促着。

“陛下。余职且不论,若要臣来说,今日京师人心动荡,一如当初先帝发病的冬至之夜。为安定朝野人心,当先请一有夙望,能服众的元老出山来。”

……………………

王中正在旁听得心惊肉跳,动也不敢动弹一下。

太后对韩冈的态度,是信任还是猜忌,他至少还能看得出来。

只要韩冈一句话,朝堂上的局面就会大变样。

众宰辅都在宫中,可向太后视而不见。看这样子,如果韩冈不推荐,太后估计多半就会乾纲独断了。

不过也不是没有理由。

剩下宰辅虽没有参与到叛乱其中,但他们脱不了失察的嫌疑。王安石和韩冈一离开朝堂,太皇太后和二大王立刻叛乱,宰辅们不能防患于未然,向太后对宰辅们的信任自然不剩多少了。

而且今天能平叛,明天呢?

蔡确当初虽没有赶上册立皇太子,可后来也是拥立太子登基的一员。而薛向更是曾经参与过冬至夜定储一事,仅次于韩冈和张璪。

其他宰辅又能比他们多忠心多少?

从结果倒推原因,太后的想法基本上就是这么一回事。

不过事前根本就想不到,在被太后派出去请韩冈之前,王中正也以为韩冈从此圣眷不再。而现在韩冈在太后面前,举荐自家岳父的情景,也是王中正始料未及的。

……………………

“有夙望……”向太后自然知道韩冈说的是谁,“这不就是令岳吗?”

韩冈没有因为太后对王安石的称呼而避嫌,“以楚国公人望、威信,保扶圣君,稳定朝纲之任,非其莫属。”

“可是平章军国?”

“平章军国重事!”

韩冈着重强调了后两个字。没了‘重事’二字那还了得?

“啊,说的就是平章军国重事。”

若说有夙望。朝堂上除了王安石之外,再无第二人。便是韩绛也差了一筹。

而王安石今日立此殊勋,重归平章军国重事的旧职是在情理之中。怎么说也不可能让他再做宰相。韩绛是昭文馆大学士,首相,王安石总不能回东府将他挤下去,更不可能站在韩绛的下首。

“……楚国公的确劳苦功高。”向太后在时间稍长的一阵停顿之后,将话说了出来,“吾也考虑了很久,朝中的确需要楚国公。”

“陛下圣明。”韩冈低头行礼。

“不过楚国公是当朝元老,旧日又做过了平章军国重事,今日又立下大功。官复原职,是在情理之中。吾想知道的是蔡确、曾布、薛向这三位逆贼所留下的空缺,该如何填补?”

……………………

在吕嘉问的眼中,蔡确、曾布和薛向这一次最大的功劳,是让出了两府中的三个位置。

三人去后,政事堂只剩韩绛、张璪二人;郭逵可以不论,枢密院也只有章敦和苏颂两人。

两府中一下就多出了好几个空缺,正常情况下,可不知要等几年才能遇上。又正好自己在京师,位置也能够得上,这更是难得。

也许太后对自己感观不佳,但那也是因为韩冈造成的。现在韩冈明显在太后那边不再受信重,只要自己努力一下,进入两府的机会还是很大的。

想到毕生的梦想近在眼前,吕嘉问心中焦热如火烧,扯开襟口,推开书房门,望着黑暗中的小院。

王安石是否东山再起尚不可知,纵使回来,也只会是平章军国重事,而不会占去两府阙额。

而韩冈,从功劳上看,不让他重入两府,哪里都说不过去。

但太后会怎么安排他?

如果进枢密院,是与章敦并为枢密使,还是章敦成为宰相,韩冈接任?又或是韩冈继续做枢密副使,苏颂顶上章敦的位置?

这倒是可能性很大。以韩冈和苏颂的关系,还有年纪,韩冈完全没有必要。而从他过去的为人来看,也不会急着争夺枢密使的位置。

当然,韩冈也有可能进入中书门下。

他不会立刻就任宰相,再大的功劳也能用其他方式进行奖赏,只会是参知政事。

但不论韩冈进入东府,还是西府,都能够顺利掌握住权力,加上章敦和苏颂两人,都是韩冈的盟友,韩绛老迈、张璪无用,太后肯定需要一个能与韩冈对抗的人选。

太后需要什么样的角色,吕嘉问就会去成为什么样的角色。只要能够进入两府,因为这样是最容易的。

皇帝总是需要几个能做事的宰辅,然后再配上几个与他们合不来的同僚,这样才是一个稳定的,能让皇帝安心的政府。

不说前朝多少例子,只看本朝,就知道多少宰辅一开始是天子为了扼制权臣而被提拔上去的。这些年,吕嘉问可是亲眼看见冯京、吴充、蔡确之辈是怎么借着王安石的光,踩在他身上,一步步爬上宰相之位。

这条路多少人走过,是最简单易行,也是最顺畅的一条。

不过能看出这一点,不会少。至少李定肯定看到了。

尽管两府空出的位置有三个,但太后那边,并不一定需要塞进去多少人。对于将名额一个个都占满,并没有太多的急迫性。

真要说起来,最后除了韩冈之外,能入选两府的甚至可能只有一人。

“一个啊。”

吕嘉问悄声自语,然后转身回房。

今晚就得写封奏章上去,这时候,得尽快表态。不然肯定会输给做着御史中丞的李定。

……………………

三个空缺如何弥补?

韩冈能感觉得到向太后对自己的殷殷期待,只是他心有顾忌,也另有想法。

“蔡确诚奸佞之辈。得选入朝后,十年身登公辅。其善于作伪,长于体察上意,先帝一时失察,致使其能够祸乱朝堂。”

有今天的事,韩冈便敢当着太后的面指责赵顼用错了人。

向太后也没有为丈夫辩解的意思,点头道:“卿家说得是。”

“其余二人和蔡确一般。蔡确看惯风色,惯会见风使舵,小人也。曾布曾受家岳推举,数年便至三司使、翰林学士,但其为人反复,因而被逐出京城。薛向诚有才,财计之术,当朝无人能及,不过对圣人之学少见亲近。”

“卿家说得是。三人正是如此。”

“此三人非是朝列所望,却能罗列朝堂,乃是先帝权衡之策。”

“嗯,的确。”向太后点头称是。

秉政日久,向太后多多少少也能明白当初赵顼为什么将曾布给调回来。从曾布推及蔡确、薛向,他们被启用,差不多应该是一样的道理。

“以陛下之智,若再有三五载,必能将此等小人或用之,或逐之,进退由心。只可惜一场意外给了他们机会。”

向太后沉吟片刻,“卿家说的道理吾都明白了,卿家打算举荐沈括?”

韩冈张了张嘴,他什么时候打算举荐沈括了?

“……陛下误会了。”停了一下,他说道,“如今朝廷纲纪正需有德望者稳定,有才无德者可待日后视时势而用。所以依臣之见,既然陛下难以决断,不如让群臣推举,择其善者而用?”

