\section{第24章 夜雨更觉春风酣(中)}

江南棉商此举,在韩冈看来缺乏长远眼光。

应该是多年养成的习惯,但凡是给官府的货,江南的商人拿出来的,向来要差上一等。

也是跟民风有关,就像两浙、两江一带作为税赋收上来的丝绢,很多都是薄得一根手指便能洞穿,几乎与医用的棉纱布差不多。但丝绢尽管可以做得很轻薄,但如纱布和麻布一般粗糙,可就说不过去了。棉布的情况也是差不多的情况。朝廷要征收棉布为贡赋,那么江南的棉商就顺理成章的将过去的经验用上了。

这种事成了习惯之后,就连外售的棉布都会做手脚。为了降低成本,与陇右棉布比拼价格,这两年已经能看出江南棉布制造商们开始偷工减料的苗头了。

而陇右棉布,质量上名声的出来了。所以才能够卖得比相同等级的江南棉布更贵。

品牌这东西,是需要常年不断的去维护的。如果从个人角度来说,设法躲避苛捐杂税,无可厚非,但是从地区的整体利益来讲,所有人都这么做的话,江南棉布的名声也就坏了大半。

直接的竞争对手若也一个样,棉布这个生意还能长久的做下去,可是陇西这边,韩冈耳提面命要注重质量,上缴的布匹都是选了质量好的,为得是什么?还不是就是为了保证陇右在棉布上的垄断利益?

时至今日,陇右与江南之间的棉布之争,已经达到了韩冈最初的目标。即使是完全一样的棉布,挂了陇右的牌子,硬是要比江南棉布贵上几十文,这些差价,就是名声。

“如果江南的棉布,就做成小衣好了,穿在里面谁都看不出来。做外袍的话,还是自家的布。”

韩冈手指捻了好几下,直觉上觉得不该是江南布,但也没分清手上的衣服,到底是不是自家的。

要不是因为成了朝廷发下来的俸禄的一部分,江南的棉布也不会出现在韩家。自家就是陇右棉布最大的生产商之一,韩家当然不会向外购买别人家的棉布,但朝廷作为俸禄的一部分发下来的棉布,那也只能收下。

从南方征收来的棉布,军中也好,官中也好,都没人想要。堆在仓库中,最后只会成为账本上的红字,平白亏了一大笔。最后韩冈决定,这批棉布作为官员的俸禄,以一半陕西布、一半江南布这样的比例分发下去。包括宰相在内,重臣们哪个都没逃过。

拿回家后,王旖持家一贯不喜浪费,毫不犹豫的拿来裁衣,韩冈还特意让人给自己用江南布做衣服,不过到底做了没有,他之后也没在意过。但如果给他做了,妻妾子女都少不了,却不会将俸禄上损失转嫁给下面的仆佣。

“这两件衣服都不是陇西里的布,不过也是机织的。”

云娘在家中负责四季衣物等杂事,虽然治家的水平不行,平常还要靠王旖提点、周南帮忙,但看衣料的眼光可比不管家事的韩冈要强。周南、严素心都比不上。

“不是陇西的?”韩冈扯了扯布料,看不出有什么区别。

“陇西的棉花织成的布不是这个样子,但也不是江南布,江南布几乎都是手织,差得很多。”

陇西的棉布织造机械在韩冈的督迫下,年年都有改进。现在的发展水平,已经不是过去那样,看几眼回去再琢磨一下就能够仿制个七七八八。大量使用钢铁零件的纺织机,就是拿到现货,没一点技术水平也仿制不了。

“或许是其他地方的出产吧。”韩冈说道。以棉布为贡赋,眼下并不只有陇右和江南,只是其他地方少罢了。

“种棉花的就陇右、江南多吧,还有哪里种?”严素心问道。

“多了,荆湖、蜀中、河东都有人种,辽国都有。不过种得多的,当属淮东,”韩冈将衣服递给云娘,回身坐了下来:“淮东挺适合种棉花。早几年,海州、涟水军、楚州、泰州都有人去买地种了,棉花这东西,不怎么挑地方……”

他回想着“我记得商会中就有一家前两年就在盐城县买了六十顷地,今年是第三还是第四年,收成差不多有同样六十顷的江南棉田一半。”

“这么多?”“收成这么差?”

严素心和云娘几乎同时开口。

“论起收成的话,还是江南最多,”韩冈对云娘道,“陇右就差了许多,幸好陇右地多。淮东的情况也差不多,不如江南收成多,我记得是跟陇右差不多。”

陇西是新辟之地,平均一家能有三五十亩棉田,农忙时还能从蕃部那边得到相对廉价的雇工,大户人家动辄百顷,除了种植和收获,人工使用更少,所以最后结算下来,在陇西种植棉花的收益能与江南相当。

转过来又对严素心道:“买的田多,是因为没人种,都是荒地,所以便宜。”

淮东靠海,土地多盐碱,不利耕种,所以与河北的沧州一般,常能见大片大片的荒地,地价极便宜。但棉花耐盐碱,又不是海滩边上,地下都是咸水,淮南东路沿海诸军州的荒地,差不多有一半能种棉花。买下那些荒地后,一把火烧过野草,就又多了层上好的肥田肥料。

第一季的棉花就有了个不错的收成。不过去年秋后,去淮东买地的竞争对手是越来越多了。雍秦商会中,有十几家都去那里了。他们本来还想找冯从义一起去,希望能借韩冈的光,不过给冯从义婉言谢绝了。

“如今淮东种棉的风气渐起,等到淮东本地人都开始种棉花,那市面上争夺的就更厉害了。江南不一定能够比得上。”

“对家里要不要紧?”周南轻声问。

“没事,反正家里还有其他产业,棉花也不愁卖不掉,少赚点就是了。”韩冈笑道,探手捏了捏云娘细嫩的脸颊,“总少不了你们的脂粉钱,”

“官人!”“三哥哥!”周南、云娘同时嗔道。

严素心白了韩冈一眼,“官人,要不要到淮东去买地?”

韩冈摇摇头,“大饼一个人吃不完的,人总不能把所有的好处都占尽了吧?现在已经很不错了。”

江南的棉田最开始的时候,韩冈完全又能插足进去,但他给忍住了。淮东虽好,但他依然不需要。

韩冈又拿起那件新作的袍服,细腻的手感时刻警醒着他,现在的优势并不足以为凭。

江南棉布在手感和质地上,还不能与陇右的棉布相提并论,加上有意无意的缩减成本,让江南棉布始终竞争不过陇右。可换个角度来比较,江南棉布的质量比起一开始时,其实还是进步了许多。

现在的陇右棉布,主要还是依靠了技术上的优势才带来了成本上的优势。但技术是会扩散的,即是现在雍秦商会的各家都在保守这个秘密,可江南棉商想要收买一个人,总能拿出适合的价码。

现在江南棉商一心想着是如何压榨织工,每天出产更多的棉布。资垩本家的范儿,现在是一点也不输给另一个世界几百年后的同类。至少韩冈就没有看见,哪一家考虑到了工人的安垩全问题。

正想说话,韩冈突然心中一动。起身走到门边,看着门外院中,“下雨了。”

……………………

“下雨了。”

听着窗外的雨声,行人的惊叫,王旖悄悄的将车窗打来了一条线。

风雨带来的寒流一下就探进了车厢中。而外面的嘈杂也一下响亮了起来。

王旖透过车窗,观察着外面的风雨,但黝黑的夜幕下,风雨交加,连路边的灯笼走在风雨飘摇中,看不清道路两边的景物,也分辨不清已经到了那里。

终于是辞别了依依不舍的六婶婶,向六叔夫妻告别,然后匆匆上车往家赶过去

幸好听到了钟声,之后又传来消息说太皇太后上仙了,这样她才脱了身。

“到哪里了?”掀开前面的车窗,王旖问道。

“回夫人的话,到大图书馆了。”

马车经过了东京大图书馆,车窗外的噪杂声立刻又上升了一个数量。从车窗的缝隙中看过去,好几位士子在路边上奔跑过来,一路往大图书馆方向赶过去

红色的砖墙曾经是大图书馆的主体外观,不过前两年,被石灰粉刷了一遍,看起来没有任何厚重感觉,反倒像是一栋普通的建筑。

自从有了大图书馆之后,士人们多了一个流连往返的去处,而且可以说是最好的。有钱可以进,没钱也可以进,海纳百川一般欢迎所有人入内。

大图书馆每天一直开放到二更初,现在成了士子们竞相学习的场所。图书馆中珍藏各色图书二十万卷,不仅仅是流传到外界的书籍,还有《册府元龟》、《太平广记》、《太平御览》、《文苑英华》,这些从太宗、真宗时期便留下来的典籍,如今成了人们竞相抄录的目标。过去的雕版早浑碎了大半,现在除了一步一划的抄录,也没有别的办法解决问题。

