\section{第31章 战鼓将擂缘败至(二)}

自从京中回到通远军后,王韶的心情一直不算好,上元夜的宴会上先行离开便是明证。而之后的这一个月,王韶的心情也不见好转。

衙门中的气氛,仿佛是夏日暴雨前的空气,让人憋闷不已。大大小小几十名官吏,连说话都是轻声轻气。虽然在与王韶接触时,没有被迁怒。但缘边安抚司的安抚使那对如锥子一般挑剔的眼神,却让他们都是战战兢兢。

现在连着高遵裕也觉得衙门里太过压抑了,难以让人待得下去。名义上负责屯田,但向来不管事的他,也便赶着连日出城去检查各处村寨麦苗的生长情况。早出晚归,尽量不与王韶打照面。

高遵裕正在回返古渭寨的路上。都已是二月中了,背阳的地方,尚有着一点积雪,但大部分土地却早已冰消雪融。五颜六色的草花在道边绽放,而青茬茬的麦苗,在经历过一个冬天之后,也变得更加青翠。

阳光明媚,渭水潺潺,温柔的春风拂面而来,在田野上散逸的动人春光,让高遵裕都有了点作诗的兴头。比起阴郁的衙门,当然是外面更让人觉得心中畅快。如果是在东京,就已是到了踏青的时候。

告别了动人的春光,高遵裕回到衙门中。因为增加防御力的需要,而建得分外低矮坚实的衙门建筑,走进去后,便是有种压抑之感。走到正厅前,原本轻松的心情随着步子一点点消失无踪,高遵裕正想打个招呼就离开,却见王韶正拿着一份公文在那里看着,掩饰不住眉间唇角的喜色。

“怎么了?”高遵裕跨进厅中,惊讶的问道:“心情今天怎么这么好?”

“没有……”王韶立刻换了副严肃的表情,递过来一份公文,语气也突的变得沉重起来,“刚刚收到的消息,庆州广锐营三千人叛乱,副总管张玉正好领军去了罗兀,经略王广渊没能及时镇压住。宣抚司下令泾原和秦凤两路一起出兵,现在燕达多半已经往东面赶去了。”

“罗兀城危险了!”高遵裕立刻惊道,这是他听到消息后的第一个反应。而第二个反应,就是在想难怪王韶心情会变好。乍听到韩绛那里出乱子,高遵裕现在都有学着外面的吐蕃人那样,唱歌跳舞的冲动了。

“当然危险。”从神色上看不出王韶有半点幸灾乐祸,但说话中也却不由自主的带着几分轻快,“罗兀本就是孤悬在外,抚宁失陷后,又在被夏人围攻,已是勉力支撑。如今后方庆州再一乱,罗兀城很难在安守下去!”

高遵裕抿了抿嘴:“攘外必先安内,朝中怕是要放弃罗兀城了。”

“谁说不是?外患不过是癣癞之疾,内忧才是腹心之患。庆州远比罗兀城重要得多,罗兀能丢,庆州却乱不得。”王韶抬手指了指方才递到高遵裕手上的公文,“何况兵变的范围已经不再局限于庆州了。”

“到哪里了?”高遵裕边问边打开公文细看。

王韶没接口,让高遵裕自己看去。在衙中服侍的一名老兵正好奉茶进来,等到老兵把两杯茶放好,躬身离开,王韶才道:“叛军已经确认是前日被下狱的广锐军都虞侯吴逵率领,现在已经南下,当是到邠州了。”

“邠州?”高遵裕一目十行的将公文看完,摇头道:“吴逵胆子还真不小。再下面可就是京兆府了,不知邠州能不能挡得住?!”

“吴锐的职司全称可是邠宁广锐军都虞侯,把他救出大狱的多是从邠州调去庆州的兵,城中内外一应悉知。邠州城的守卫说不定都会投了叛军。”王韶又冷笑了一声,“还有,公绰你忘了前段时间,司马十二的几份奏章吗?”

“是司马光反对横山的那一份,的确给他说对了时机,现在韩绛失算,他的先见之明可就露脸了。”

“先见之明?!”王韶登时大笑摇头:“是另一份!反对加强长安城防,还有增加邠州守军的那一份!”

高遵裕啊了一声,终于想了起来:“……看来真的麻烦大了!”

“的确是麻烦大了……”王韶感叹着,“即便此次兵变能顺利平定。可广锐一叛,整个环庆和鄜延两路的军心都要受到怀疑。开拓横山的战略,可能要暂时搁置了。”

王韶和高遵裕对视一眼,两人眼底尽是隐藏不住的笑意。若论关西战略的优先程度,拓土横山远在河湟开边之前。朝中相公们不可能支持关西同时发动两场战争,就算他们有这个打算,钱粮物资也补给不上。

种鄂意欲修筑罗兀城,是建立在他熙宁元年收复绥德城的基础之上。有此战绩为底,所以这两年,横山方向一直得到优先支持,连主持全局的韩绛因为需要能够同时号令陕西、河东,而被升做了宰相。

而熙河方向,到现在为止还在纠缠之中,自从结束了渭源之战后,不论物资、还是人力,都是被削弱到一个仅能自保的地步,朝廷仅有的支持却是下令在古渭建立通远军而已。

王韶摸着滚热的茶杯,无限感慨:“我何苦要奏请在古渭寨开榷场,不就是为了让开拓熙河的行动省些钱粮,省得给人找借口。”

“但现在不同了!”高遵裕立刻高声道。

王韶又点头附和:“的确是不同了!”

横山方向既然已经失败,一直排在二线的熙河方向自然会顶上。关西已经没了其他选择,只要还想在军事上挽回一点颜面,天子和朝堂也只有选择支持缘边安抚司,选择支持王韶。

“一叶落而知天下秋,罗兀兵败,尤可卷土重来。但庆州卒叛,朝堂安敢再于环庆、鄜延点兵?横山之事已是彻底失败!”

“王相公需要一场胜利。官家也想看到一场胜利。韩绛、种谔给不了,但我们这里可以给。”

王韶和高遵裕你一句,我一句,几乎要弹冠相庆。一旦有了朝堂的支持,河湟这里随时可以动手。

“对了!”高遵裕突然想起,“韩冈不就在种谔帐下,说不定就在罗兀。他那里……”

王韶毫不担心的笑着:“玉昆是需要让人担心的人吗?”

“说的也是!”

高遵裕由衷的表示赞同。以韩冈的能耐,就算遇上了天崩地裂,怕也是能活下来。

……………………

韩冈却不认为自己的性命能做到跟在地球上生活了几亿年的蟑螂一样。他正为了自己的安全,而在罗兀城中费尽口舌。

用了一个晚上的时间,韩冈说服了张玉和高永能,让他们终于点头同意让伤兵们先行离开。于此同时,在城中的并不实际领兵的文武官员,都会乘着这次机会而返回绥德——只除了韩冈他自己。

得以幸运脱离苦海的诸人,在军议上听说是韩冈的主意,当即就让他收获不少感激的目光。

‘这就是人缘啊!’韩冈有些小得意的想着。反正留着他们也没用,早点送其离开,还能得到一份感激。

而且通过这一条与己有关的建议,韩冈顺利插手进了军务之中。等到全军要离开罗兀城,难道高永能会不问问他的意见?

张玉跟自己一见如故,算是忘年之交——话说回来,除了窦舜卿和向宝那几个之外,其他认识的武将,跟自己的关系一般都不差,郭逵、张玉、种谔莫不如此,与高永能点头之交也是有的。但插言不归自己名下的公事,却不是靠着人缘关系就能做到的,在官场上也是个忌讳,韩冈也是用了上一点心思。

不过韩冈更多的心思还是放在外面。在得到朝廷的准许之前,罗兀城绝不可能被放弃。他也不会奢求能在此之前离开罗兀城,否则就算能回到绥德,最后的结果也好不到哪里去。

现在的关键是军心要稳定,让伤兵先行离开,也是为此而来。

张玉久在军中,威望甚高。而韩冈最近也是声名鹊起,在士卒们心目中的地位也不在老将张玉之下。只要高永能这位主帅不走,张玉和韩冈又继续在城中坐镇,根本不用操心军中生乱。

但西夏人那里迟早会得到庆州兵变的消息,为了防着士兵们怕受伤后被抛弃,在撤退时不肯用命,需要先把隐患去除。

对着罗兀城周边的小比例精细沙盘,以高永能和张玉为首的罗兀众官,正在筹划着让伤兵和护送他们的队伍顺利回返绥德的计划。

但不论是谁,都没有想出一个能在西贼眼皮底下潜离罗兀的主意。讨论了半天,无论是种朴和高永能,或是其他参赞军务的幕僚,都有些颓然。

“一个两个倒也罢了,上千人的离开,要想西贼不发觉,除非他们全都变成了瞎子。”一名高永能手下的幕僚叹了一口气,摇头表示自己没有办法。

“那就以被发现为前提,把西贼引出来打一仗,让他们不敢追击。”韩冈一直保持沉默,在众人都放弃的时候,才站出来提醒他们换个角度去思考。他要树立自己发言的权威性,只会在正确的时机选择开口。

“如果一支有车有马的队伍突然悄悄的离开罗兀向南去,落在党项人眼里会是什么情况?”他向帐中众人问着。

