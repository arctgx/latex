\section{第40章 帝乡尘云迷(三)}

进门,行礼,落座。

吕府门外,等候召见的官员数不胜数。但韩冈一至,便立刻被请了进去。与吕惠卿在吕家并不宽广的内厅中,分了宾主坐下来说话。

吕惠卿和韩冈不是第一次见面,不过基本上都是在王安石的府上,单独会面的情况几乎没有过。

望着坐镇下首处年轻得几乎要让人嫉妒的韩冈,吕惠卿半开玩笑的责怪着:“玉昆可是让我久候了。这些天来,我一直都让人洒扫庭院,等着玉昆上京来。没想到一等就是半个月!”

韩冈在座位上坐得四平八稳,丝毫没有普通小官见到上官,只敢斜着身子,在座位上沾半个屁股的情况。

不过他的态度还是很有分寸,吕惠卿在言辞中刻意表现着亲近,他还是拱手告罪:“韩冈也欲早日拜见大参,只是身负王命未了,恐大参见责,才一直拖到现在。”

“玉昆欺我,你哪有这般胆小?!”吕惠卿摇头失笑:“想及当年初见,玉昆你便在介甫相公面前侃侃而谈,当时说的话,我现在还记着呢。”

回想旧时,两人心中的确也免不了要心生感慨。

五年前,两人在王安石府上第一次见面,王安石、曾布、章惇也都在场。

当时的吕惠卿虽然已经是新党的核心之一,却还没有多高的地位,且由于旧党重臣群起而攻,新法只在风雨飘摇之间,虽是都有鼎覆之灾。而韩冈那时更是不过一个刚刚做了官的小选人,在大宋官场上,不值一提的小人物。

时间易转,吕惠卿已经侧身政事堂,与当年的王安石平齐。韩冈也是靠着历历功绩不断攀升,在年轻一辈中,独占鳌头,将一干状元、榜眼远远的抛在身后。

现在的两人,一个是举足轻重的执政,另一个在朝堂中也算是有着不小的分量,对天子的影响力更是不能小觑。即便仅仅坐在一起说话,只要消息一传出去,也能引动朝中众臣的议论。

“当年年轻气盛,妄言朝政,没被乱棒打出去,那是韩冈的运气。”

“哪有岳父打女婿的?玉昆你数条对策一出口,就已经被介甫相公放在心上了。”吕惠卿笑道:“就连曾子宣,当时也是说玉昆你是贾诩。”

韩冈哈哈一笑,这个评价,章惇向他提过。但章惇当时说是吕惠卿,现在吕惠卿则说是曾布。真搞不清究竟是谁说的。不过这也不是什么值得深究的大事,摇摇头:“贾诩一句话,就让汉室再无挽回的余地。想不到曾子宣那么看得起韩冈。但一言丧邦的本事,韩冈哪里能有?!”

吕惠卿笑容微敛,感慨道:“不过若是尽数听了玉昆你当初的意见,新法的施行也不会有那么多反复。”

韩冈摇摇头,“事实难料,若是真的按照韩冈所言施行,更有可能会因诸法过于峻急,反而坏了大事。”

吕惠卿深深的看了韩冈一眼,从他的脸上看不出什么异样,一时判断不清这两句话是否有深意,道:“天子为韩富文之辈所蛊惑,畏虏如虎,使得相公不得不辞官。如今朝堂之上,群小猖狂。冯京今日又上本,说修葺黄河内外双堤,耗费钱粮无法计数,国计实在难以支撑。且束水攻沙的方略未有实证,贸然取用,未免太过冒险。乞天子只修外堤,内堤延至日后,待验证之后,再行处置。”

吕惠卿毫不客气的将冯京归为群小的范围,言辞中一点也不客气。

韩冈本是在等着吕惠卿的开价,却没想到吕大参当先做的却是讨价还价。但吕惠卿拿起这个话题,却是看错了人,也用错了地方。

韩冈先是摇摇头,继而轻笑道:“当朝之人所谋不及长远,乃是国之不幸。幸而政事堂中有大参在,韩冈也不用担心。即便大堤一时修不好,有大参坐镇京中,黄河当不至于为患。”

束水攻沙的治河方略的确是自己的提议,但天子就算不采用,韩冈也不会太过放在心上。开封一段的黄河堤坝已经修过了,但洛阳、大名的还没有完工,而黄河北岸的大堤甚至没有动工。外堤还没有修好,内堤就更是没影的事。

韩冈本来就做过预计,整修黄河中段,需要耗费大量的时间和人力物力,。韩冈不信黄河日后会不泛滥、不破堤,等到出了事,他的方略还是要提上台面来,根本不必急于一事。想拿这个当做交换条件,未免太过欺人了。

吕惠卿心中一叹,果然韩冈不是这么简单就能收服的。“玉昆任府界提点,所行诸事,安民无数,后人当效之。如今河北流民皆安然北返,在京者已寥寥无几。让天子、两宫安居无忧,此是玉昆之力。”

韩冈谦虚着:“大参之赞,韩冈愧不敢当。上有天子朝廷还有开封府指挥,韩冈也只是跑跑腿而已。”

“玉昆却是太自谦了。”吕惠卿笑道:“玉昆之材,世所罕有,非是一州一县所能容。”

韩冈身处新党之中,与吕惠卿和章惇是没有竞争关系的。年龄相隔太远,吕惠卿能因为升任参知政事,从右正言一跃成为右谏议大夫,韩冈就不可能。他只能按部就班的一步步走,三十多岁成为执政有先例,可未到而立就入政事堂,未免太骇人听闻了。

既然没有竞争,吕惠卿当然乐于拉拢扶持韩冈,来稳定自己的根基。

只是韩冈有自己的想法,他的地位不是因为希合上意、附和新法,靠着天子、王安石赏赐而来,而是自己一拳一脚拼杀出来的。旧党重臣能说当着赵顼的面说吕惠卿等人是新进小臣,但他们的弹章中有几个敢说韩冈是幸进之辈?不怕天子直接批回去?!

韩冈的一桩桩功业,许多身居高位的大臣都没能做到,他晋升之速,立国以来难有匹敌,是仗着功劳成就,而不是哪人的看顾。韩冈这段时间来,已经受过不少弹劾,但其中的最为激烈的言辞,也只是集中在行事的手段和他的人品道德,而不是能力和功绩上。

这就是韩冈的底气,让他可以抬眼直面吕惠卿投来的锋锐视线:“韩冈浅薄之材,为一府界提点尚且不足,惹来众多议论。到了天子面前,还得先行请罪,哪敢有非分之想。”

他在京府立此大功,擢升入朝本是应有之理,哪有什么必要承吕惠卿的人情?要想来拉拢人,得先拿出点实在的东西来。他也不是只有投靠吕惠卿一条路可走,毕竟他吕吉甫还不是宰相。

韩冈说得足够坦白,话中之意,吕惠卿不可能听不明白。

将猛然腾起的不快之意压在心底,吕惠卿微笑起来,端起茶盅:“玉昆还是这般谦虚。”

一番长谈之后,韩冈告辞离开。吕惠卿降阶相送,给足了韩冈脸面。

等他送了韩冈回来,一人从屏风后转出,是吕惠卿的二弟吕和卿,“大哥,韩冈此子似有异心啊……”

吕惠卿沉着脸坐了下来。

虽然经过时间不短的谈话,但这番谈话中,韩冈的态度依然不明确。

唯一能肯定的,是韩冈支持新法——这个时候,他不可能在背离新党。但韩冈会不会以自己马首是瞻,吕惠卿却没有把握,甚至已经不抱希望。

吕升卿在后面听到了全部对话,对韩冈的态度很不快,“韩冈桀骜不驯,宁可与其反目,也不能把腹心之患留在朝堂中。”

吕惠卿摇了摇头,“此事不妥。”

不能容人者无亲,吕惠卿虽然权欲旺盛,可还不至于无法容忍韩冈今天表现出来的独立性。

在王安石的面前,韩冈就一直是这个态度,始终都没有变过。要是今天突然变成了满口谀词,吕惠卿反而要警惕起来。

而且即便吕惠卿觉得韩冈在朝中是个祸害,要将他赶出朝堂也不是那么容易的事。

“要让韩冈出外,谈何容易!”吕惠卿长叹道,“不光天子那一关不好过,也要考虑王介甫那边的想法。他一去位,我就将韩冈逐出京城,王介甫会怎么想?天子又会怎么想?还有朝中,也免不了议论。为一个韩冈,却坏了自己的名声,未免不值。”

吕升卿恍然:“……难怪韩冈有恃无恐。”

吕惠卿摇摇头:“还是先想想自己吧。我已经准备荐二哥你担任崇政殿说书,若能才学,我是不担心二哥你。就是你素乏捷才,侍从天子时,恐难以应付。”

王安石主持编订三经新义,新党之中才学上佳的成员都参与了其间。吕升卿虽然不及其兄,但在福建乡里也颇有些文名,负责了《诗序》一篇的注解。他将诗经三百篇的总纲一句句的注释出来后,连王安石都没有怎么改动,而在书中全盘加以收录。

只是吕升卿反应慢,许多事要反复考虑过才能想明白。吕惠卿知道这一点,“我会安排沈季长跟你一起做。”

“沈道元【季长字】?他也做崇政殿说书?!”吕升卿闻言立刻问道。

吕惠卿点了点头:“既然我安排了王介甫的妹夫做了天子近臣,那即便对付起他的女婿,王介甫当也无法说什么了。韩冈的脾气,他应该明白。”

“大哥已经决定要对付韩冈了?”

吕惠卿面色阴沉:“那还要看他本人会怎么做了!”

