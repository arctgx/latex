\section{第33章 物外自闲人自忙(11)}

“既如此,你我可当说?”富弼微笑的反问着。

刘几说此事不便去问当事人,富弼便问着背后议论文彦博到底合不合适。刘几这下明白了,富弼的态度其实已经表明了他全无左袒文彦博的意思。

“说得也是。文宽夫的确是输了,逞一时意气,结果就是坏了名声。韩冈当真是不简单,后生可畏啊。”刘几抬眼看看富弼:“……彦国你当年在青州救了数十万流民,韩冈前两年也救了数十万。虽说他当时有开封府库为凭,又是帮他岳父收拾手尾,不及彦国你当年在青州冒着被猜忌的风险在石头里攥油,却也不差了。”

“韩冈长于政事,更长于军略。政事上也许他还欠把火候,但军略上我可是远有不及。”

富弼对韩冈的功绩毫无芥蒂的夸赞着,一点也不觉得输给一个比自己小了近五十岁的年轻人,有什么觉得丢脸的地方。

这话听在刘几耳中,就是没有任何挽回余地的拒绝,脸色也不由得微微变了一下。

而富弼则是端起茶汤来喝。不管刘几和他所代表的那几位准备做什么,富弼是半点也不想掺和,肯坐下来将这些事说开,已经是他这位前任宰相给人面子了。

文彦博这次算是大败亏输。在市井中的议论,是韩冈尊老,给足了文彦博的面子;在士林和官场中,则多半是认为韩冈为了襄汉漕渠而委曲求全;但在他们这群老臣眼里,基本上都能看得出来是韩冈赢了,文彦博家里的动静瞒不过他们。

有人幸灾乐祸,但也有人有着兔死狐悲、物伤其类的心思,看着文彦博被一个一个三十不到的后生晚辈压着连脸皮都被剥了——尽管表面上看着是韩冈低头,但实际上是什么样的情形,只看韩冈现在在京西的好名声就知道了——老家伙们当然看不过眼。

不能简单的说他们心胸狭隘。看不惯年轻人的行为,不过是老辈人最常见的现象。富弼很庆幸自己能清醒的认识到这一点,尽管当初听到有人拿韩冈救了数十万河北流民的功绩,与自己旧时的功业相提并论时,富弼心中也少不了有点不痛快,但理智很快就让他变得清醒起来。

富弼自知他跟文彦博是两个性子,文彦博不服老,至今不肯致仕,而他富弼看着当今天子不肯接受自己的意见,便干脆了当的回家养老,到了如今这个年纪,心里考虑的只有子孙了。

放下茶盏,富弼笑道:“近闻伯寿你开春后时常骑牛外出,嵩山之下,以铁笛伴春风,翩翩仿佛神仙中人,倒时让富弼羡煞。”

别掺合了——富弼的劝告不再隐晦,已经变得十分直白。

刘几看着富弼不容再劝的严肃神色,最后摇头一叹,“算了,也是受人之托……即是如此,此事还是放在一边。”便是洒然一笑,神色一下放了开来:“自去岁秋后,隔个一月便往嵩山一游,只是冬天大雪封山时停了一阵。回程后便在峻极寺留下一个标记,如今峻极寺墙上已经有六个标记了……若能九九归真,百岁可期。”

“此亦是养生之法?”话题终于转到富弼感兴趣的话题。

刘几在为官时,以知兵著称,几十年来多守边州。不过,除此之外,他还通音律,善养生,致仕之后,这两个特长,比起知兵有用得多。房中补道之术传了不少人,富弼还曾从他那里学了一手暖外肾的手法。

“彦国你牵扯甚多,难以轻动,却是难学来。若是当真想学,先把庄子搬到嵩山脚下再说。”

……………………

沈括已经在唐州就任了。他走马上任之后,除了点验府库等例行公事,他首先做的,便是检查百年前曾经为襄汉漕运而开辟的河道。

在韩冈收到的信中,沈括描述了襄汉漕渠唐州段的现状。正如韩冈几次往来京西所看到的大概情况,沈括巡视过的运河河段,情况都还不错。

四十余里的人工河道,需要疏浚和拓宽的地段并不多,原本就是为水利运输而开凿的渠道,经过的地段自然都是宜居宜垦、人烟辐辏的平陆,这些年来也免不了在水运上发挥着一定程度上的作用。

反倒是被襄汉漕渠利用的几处自然河流还有两个湖泊,有必要加以清理,同时需要整修堤防,只是沈括也在信上说了,这几处工役,并不需要花太多的人工和钱粮

除了方城山的那一段,襄汉漕渠经过唐州的运河和河流加起来总计两百余里的水道,大体上只要稍加处置就都可以使用,不会影响到整个进度。

此外沈括还依靠他在水利工程上的才华,发现了几处可以加以改进的地方,依沈括一番的估算,如果都加以改进,不但能加强水道的防洪能力,同时还能顺便淤灌土地,将四个县的一千七百余顷旱田,改造成水浇地。

韩冈对沈括在政务和水利上的水平抱有很大的信心,既然沈括如此保证,韩冈当然也愿意看到他成功。

唐州的情况既然很不错,那么越过方城山,在方城垭口的另一端,属于汝州的渠道,情况也不会比差的太远。

也就是说,一切正如韩冈之前几次经过京西的所查看过的情况,襄汉漕渠只要稍加处理就能派上用场——自然,前提是方城垭口那一段的空白能及早填补上,不对商道形成阻碍。

韩冈收起看了两三遍的信笺,离开洛阳南下的心思也越发的重了起来。

要不是还有富弼的寿诞要参加,几名老臣同样的得加以拜访,韩冈早就动身离开了这个满是浊流的漩涡之地了。

转运司中的公务,对韩冈来说,算是小菜一碟。绝大部分庶务皆有转运副使负责,韩冈不需要亲历亲为,只要督促一下就够了。

至于胥吏惯使的欺蒙上官的招数,韩冈已经见识过一次了。是在绝户田上做文章,想要将应该没入官库的无主财产给私分掉,不过给韩冈用笔在公文上,将一个个破绽给圈出来之后,登时就消停了——该怎么说呢,相对于东京城里的胥吏,洛阳的这些贪腐之辈一点想象力都没有,做事的手法还是太老套了。

“玉昆你就要南下了?”

既然手上没多少事情可做,韩冈便抽空又往程府这里来拜访了一趟,听见程颢相问,便点点头,“最多再过十天就走。既然学生受命提举襄汉漕渠,就必须待在这千里水路旁盯着。洛阳不在水道上,离着远了,消息传递也不方便。”

程颢想起韩冈上一次说的话:“不是说还要拜访其他一干致仕的老臣吗?”

“郑国公寿宴之后就各家上门,但也只能拜访城内,城外的就没办法了,”如果任职州县,就是住在山里的致仕高官,都该去拜访一次。但韩冈既然是转运使,世间的礼法就没那么苛刻,“不过独乐园是要去的。”

“若是要去独乐园,可以让刑和叔居中传句话。”程颢说道,“司马君实杜门谢客,见客的时候并不多。但刑和叔是司马君实的私淑弟子,由他居中传递,比起直接上门要更简单。”

刑恕这位游走于多家门下的士人,韩冈倒是有所耳闻。程颢、司马光和吕公著,刑恕都可算是他们的门人。

“他没去东京?”韩冈有些奇怪。刑恕也算是吕公著的弟子,而吕公著担任着枢密使,刑恕应该水涨船高才是。怎么不在东京,而到了洛阳这个养老地。

“他还是要去东京,仅仅是在洛阳歇上数日而已,不过他的亲友甚多,说是要歇息,但至今也不得一个清闲。”

“还真是劳碌命,就跟学生一样。”韩冈自嘲的笑了笑,“若有刑和叔居中联络,去独乐园倒是能省心许多……对了。”他突然想起了一件事,“敢问先生,吕与叔是不是回洛阳来了?”

“他还没有登门?”程颢惊讶道。

韩冈摇摇头:“没有。”

吕大临回到洛阳,已经有几天了,韩冈的名声如此响亮,以同窗之谊,也该上门拜侯一番。就算不想看到韩冈,韩冈的幕僚之中,也有好几位张载的弟子,总得见上一面。但吕大临却是硬着脾气,根本不来理会。

“也不知道他到底会不会来,想想还是学生过去见他更方便一点。”韩冈不是赶着要往人冷屁股上贴,而是吕大临手上有横渠先生的行状,记录了张载的生平、事迹和功业。

韩冈当然想看看吕大临写得到底客观不客观。一份出色的行状,能一开场就给人留下一个好印象。而被记录人的墓志铭、传记,都要依靠行状为本。吕大临能被选上,是因为蓝田吕氏投在张载门下最早,经历得也最多的缘故。

吕大临的文笔韩冈不能保证,但他应该是真心诚意的帮着张载和关学做着总结。由他写出来的行状,应该能让所有张载弟子满意。

