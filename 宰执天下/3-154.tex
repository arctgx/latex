\section{第41章 礼天祈民康(一)}

离着腊月初一的冬至日越来越近,开封府中的气氛也变得越发的紧张起来。

京中多条要道上的巡检,巡逻的人数、次数一下多了一倍。如果有人夜中在路上行走,少不了会被巡检们给抓个正着。

城门、税卡的检查,也变得森严起来。原本只要翻看一下、甚至有时看都不看一眼的行李、包裹,现在皆要打开来细细搜查。旧时行人可以随身携带的寻常兵器、如弓箭、短刀、棍棒,也都开始被严查,只要稍有逾制,就会被没收。

府中的两判官、两推官这些日子也都忙得不可开交,每天都要在衙门里熬到点灯时候才能回家。

京中那些泼皮、地痞,以及一些大户人家的浮浪子弟,过去在京中横行市井之中,只要不犯大罪,官府也没精力去理睬他们。犯点寻常的过错,被揪到衙门里,也皆是叱骂几句,敲上几板就放他们回去。可如今却是只要犯了事,不论轻重与否,随便问上两句就直接押进了大牢内,等着大赦诏颁布之后再放人。

为着这一场大礼,甚至连街道上的乞丐都能从官府得到一日三餐,不用、也不需出来乞讨了。

而知府孙永,每天要上朝面圣奏事,回衙门后要处理京中各种各样的大小事务。除此之外,他还要挤出时间来,去视察城外祭天圜丘的整修工作。

已是冬月中旬,还剩半个月就要到大典之时,孙永已经记不清自己是十次还是第二十次前往城南的青城行宫。

道边的榆柳落光了叶子,枝干光秃秃的,上面还有些残雪堆积着。风物萧瑟,倒是远远近近的屋舍上都是白茫茫的一片,比起去岁大旱时,灰土遮地要好上许多。

前两天的又一场暴雪,城中积雪盈尺。尽管这是个能让天子喜笑颜开的好兆头,可对于孙永来说,却不是那般可喜了。

用了两天的时间,动用了三千厢军,好不容易才将京城内外的几条主要官道给清理了出来。虽然雪橇车今年在京城中时常能看到,可不管怎么说,天子出宫去祭天,总不能让他坐雪橇出行。

孙永身下的坐骑,踏着两个月前刚刚重修过的官道。钉了蹄铁的马蹄,在三合土夯筑而成,如同坚石一般的路面上,发出哒哒的清脆声响。而在孙永的身侧,还有一串清脆的蹄声做着合奏。

与开封知府并辔而行的,是个只有二十多岁的年轻官员。身穿着绿袍,身姿矫健,控马之术水平很高。

从开封府一路行过来,此事已经出了南薰门。孙永发现两匹马的前后差距,始终保持一个马头到半个马身的距离上。这点差距不影响说话,却体现了身边这名年轻人对自己的尊重。

孙永很满意的轻笑了一声,抬头望了望天空,道,“玉昆,你看看这天是不是要下雪了。”

这个年轻官员自然是韩冈,他也跟着看了看天色。午后的天空,已经被铅灰色的阴云所笼罩。云层压得很低,离着地面似乎也没多远,再望远一些,就已经与灰白色的地面纠缠在一起,让人难以区分。骑在马上,迎面吹来的风更是刺骨。被寒风冻得一颤,点了点头:“可能真的又要下雪了。大府,看来得快一点赶到青城行宫。”

韩冈虽然只是附和着孙永的话,但孙永却信之不疑。

因为流民图一案,以及廷对十日后的一场暴雨,使得世人都相信韩冈有着判断天候的本事。

京城的百姓传说他是孙真人的弟子,所以能掐会算。而官场、士林之中,一般则是说他靠了农家出身才学到的能耐。‘吾不如老农’,‘吾不如老圃’,这是圣人说过的话,韩冈能做到并不奇怪。

反倒是现在都没人怀疑韩冈当初是在糊弄着皇帝,那一场雨,下得当真是再及时不过。

蹄声由缓转急,哒哒如同响板的清脆节奏,转眼就变成了夏日的暴雨,暴雨一般落在了路面上。

孙永、韩冈挥鞭疾行,带着后面的一行随从,开始紧赶慢赶,往着青城行宫而去。

两人都是能做事的官员,在为时一年的共事中,两人关系相处得很是不错,也有了几分交情。

韩冈这一年来,在公事上得了孙永的全力支持,若非如此,几十万河北流民,他安置得不会这般顺利。对于自己的这位顶头上司,韩冈有几分好感,也有几分尊敬。

而在孙永眼中,才二十二三岁的府界提点,行事虽不为礼节所拘,可他的身上从来不见少年骤贵的骄狂,说话处事的分寸把握得恰到好处,一点也不像初出茅庐的年轻后生。

不过韩冈也不是那等棱角在官场中被深刻打磨过的油滑,要不然也不会将安置流民这个苦差事担到身上。

韩冈在今年的流民安置上立功不小,但他在其中费了多少心力,孙永他这位站在最近处的开封知府,看得也是最为明白。换作是一般的官员,聪明的不会接手,而愚笨贪心的接下来也做不好。能如韩冈这样安稳妥当的将几十万流民都抚慰安置,也只有拿富弼当年来比。

国有贤臣,为人厚道又曾是潜邸旧臣的孙永,却是为着天子而感到高兴。

青城离着开封府城并不远,只有五六里的距离,出了城后,奔行不久就到了地头。

从性质上来说,将祭天圜丘包括进来的青城行宫,就跟后世的天坛一模一样。

韩冈当然不是第一次看到天坛,不说眼前的这一座天坛,就是后世京城的那一座,以及唐朝的那处被挖出来的,他都进去参观过。

此时所使用的天坛,和他前世在京城看到的天坛,形制完全不同,反倒是跟旧唐都城的那座很像。

同样是圜丘,韩冈眼前的这一座上下分为四层,并非是白玉栏杆,白石台基,而是用黄土夯筑而成,上面抹了白灰。同时圜丘一周,按照地支,有十二条走上台顶的陛——也就是台阶。其中以正南方的一条最宽,以供天子行走。

韩冈和孙永从着侧面的台陛走上圜丘顶部。立于圜丘之上,并没有一览众山小的感觉。天子祭天的这座建筑其实并不高,每层八尺一寸,加起来只有三丈多,还不及北面的行宫主殿端诚殿。

孙永和韩冈也只有现在能上去,真正到了祭天的时候,仅有天子,以及天地神主,加上陪祀的太祖神位,可以站上台顶。其余千万神明、文武群臣,全都得排于陛下。

两人在台陛上仔仔细细的查看了一遍,天上的乌云更加低垂,天地一片阴暗,才不过未时,就已经像是夜晚提前降临。

孙永和韩冈仅仅稍稍犹豫了一下,一片片雪花就已然随风在空中狂飞乱舞。急急的从圜丘上下来,退到了青城行宫中的偏殿——熙成殿前的宫门内。不过转眼的功夫,飞雪便是铺天盖地,视线中一片模糊。

看着宫中的仆役把门窗关紧,将风雪堵在了室外。孙永自己拍了拍身上的雪花,叹着气:“桥道顿递之事,不管你再如何操心,事情一场接着一场,总是忙不完。”

国家大典,三年才得一次,不会设立专门的官员,而是要安排临时性质的差事,让朝中官员负责其中的事务。

一般来说,由宰相兼大礼使,翰林学士任礼仪使,兵部尚书为卤簿使,御史中丞则是仪仗使,而开封知府则是固定不变的桥道顿递使。

五使之中最麻烦的就是桥道顿递这个位置,其他职司只要事前检查一下准备情况,基本上都是到了大礼当天,监督百官遵守礼仪法度就行了。只有桥道顿递使,是城内城外都要跑着,如果预定的路线上出一点差错,这罪过就能让人去南方过上三五年。

韩冈深有感触的点着头:“前两天才扫过雪,今天又下了,费了那么多气力,几乎都是无用。”又自嘲的笑了一声,“去年盼着下雪却不下,今想着能过了冬至再下雪,眼下却不见停。”

留守行宫的宫人这时为开封府的两名高官端上来祛寒的热茶。孙永坐了下来,端起茶来喝着。听着外面的骤雪不断的敲打着门扉,更是叹道:“京府大尹,天下亲民官中最为繁剧。任官一载,堪比他任十年。”

见到孙永已经坐了,韩冈同样欠身坐下,笑道:“冯相公治平初年为开封尹,任官年余,便接连上本自请出外。记得魏国公【韩琦】说,‘京领府事甚久,必以繁剧故求去尔’。即便是宰相之才,也是怕着开封府的忙碌。”

“谁让这里是开封呢……”孙永叹道。作为开封知府,权柄之重,远在寻常知州知府之上,即便只有重臣能够参加的崇政殿议事,都少不了他一个。

“冯当世还是做得不错的,韩稚圭不也是说了吗,他处事无过啊!”

“大府当不输于冯相公!”韩冈接口道。他倒不是溜须拍马,而是当真这么认为。这一次的大旱,冯京可没有经过。

“多谢玉昆称赞,老夫愧受了。”孙永笑道,“只可惜,不能与玉昆你多多相处了。”

