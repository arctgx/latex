\section{第18章 霁月虚明自知寒(下)}

司马光重病,其子司马康来京城求医,不想才抵达京城便被抓进了衙门里。

这件事说起来有几分好笑,由于轨道新修,因为不懂规矩而被抓进衙门里的本地人、外地人都不少,官员也不止一个。如果司马康坐官车抵京,那什么问题都不会有。

官人总是能得到更多的优待,上车下车,都会有专人引导,从不同的地方上下车,根本不会闹出今天的这场误会。可惜司马康心忧司马光的病情,并没有乘坐每天只有一班的专用官车,而是坐了最早一班车。

但这件事闹到了官府里,证明了司马康的身份,除了留下一个笑话之外,也就到此为止。

苏颂请动太后下了派遣良医的口谕,韩冈也命人从太医局选了两名高品的翰林医官,当天午后就让他们去了洛阳。

在韩冈看来,这件事就算结束了。

苏颂与司马光有交情,故而求到太后那边,而韩冈与司马光可是没什么瓜葛,甚至就见过两次面,其中一次还是在文德殿上,韩冈让人为他找了两位御医,尽一尽人事便算了事。

回到家中,把这件事当做闲聊的话题同妻妾提起,周南便冷笑道,“也亏他想要朝廷赐医!”

周南对朝中大臣一向不客气,尤其是跟韩冈过不去的。就算司马光重病垂危,也没半点关切,倒是先想到朝廷赐医上。

“朝廷赐医怎么了?”云娘不解的问道。

韩冈道:“苏子容是好心。而且医官是官,岂能擅离职守?”

王旖也说道:“朝廷赐医不是好事,只会乱花钱。真要救人,直接请医生过来就好。”

云娘圆睁着眼睛,一幅不明所以的样子。

韩冈笑了起来,都三个孩子的母亲了,却还存留着少女时的天真,像现在这样的歪着头、眨起眼,就让他想起刚苏醒时那段相依为命的日子。

“太后下诏赐医,医官要向太后交代,病家延医,医官是向病家交代,云娘你说哪种情况医官会用心治?”

大臣生病,天子惯常是赐药,下诏派遣御医上门治病的情况并不多见。平常御医给朝臣治病,都是病家主动上门去请。若是天子下诏赐医,宰辅们还好,下面的朝臣可就要叫苦不迭。

但凡朝臣重病,第一怕的就是天子遣医。御医受命之后,为了讨好天子,只会往贵里开药,到时候万贯家财能有一半留下就是好的了。第二怕天子遣人治丧。一旦丧礼落到了朝廷派来的礼官手中,也不会帮着节省,剩下的一半财产也不会剩下什么了。

或许对有些人来说钱财并不是大事,但轮到天子赐医的时候,多半就已经不治了,根本就救不回来,徒耗家财,可事后还要一本正经的叩谢天恩。落到哪家的头上,哪家不会是满腹苦水?

一番解释,云娘终于明白过来。

她点着头:“原来是这样啊。官家的赏赐,不全是好事呢。”

“当然。”韩冈笑了起来,“莫说天子所赐,就是上天所赐,顺风、甘霖之外,却也有水旱蝗汤。”

“汤?熟水?”严素心纳闷的问道。

“不是,说差了。”韩冈摇头,跳过这个不好解释的口误,“上天与人万物,也是有好有坏,天子所赐,怎么可能都是好事呢?”

“怎么官人不拦着?”韩云娘又问道。

韩冈道:“先不说当时没想那么多,事后想起来,总不能让太后将口谕再撤回。”

“而且前面官人也说了,司马侍郎远在洛阳,东京的医官不得调令,也不方便擅离职守。”王旖解释道,“而且太后对司马侍郎没有好感,这是人所共知。没有太后亲口下诏,去洛阳不一定会用心救治。苏相公是好心,希望被调去的医官能够尽最大可能的去治病救人。”

“何况把做人情的好事留给君上去做,这是忠臣该做的。”

韩冈笑着道。但他的话里完全听不出真心,甚至还有些许讽刺味道。

王旖眉头微皱,丈夫偶尔语出不逊,她其实已经习惯了,只要韩冈不在外面说就行了。但韩冈在整件事上的态度,却让她觉得不太合适:“官人是不是对司马君实还有怨恨?”

“司马侍郎应是怨恨为夫,但为夫为何要怨恨于他?”韩冈反问道,他与司马光只打过一次交道,吃亏的不是他,与旧党打过很多交道,吃亏的也从来不是他。

“嗯……”韩冈摸着下巴沉吟了一下,“要说旧怨,的确有一条可以算……《资治通鉴》交上来太早了,这让为夫和苏子容很不好做啊……”

王旖狠狠的瞪了韩冈一眼,都宰相了,依然不正经。

韩冈笑了一笑,又提起了其他话题。

在韩冈看来,这件事仅可供闺阁闲聊,但是余波却在不经意间开始泛滥起来。

次日韩冈案头上,便摆了一份弹章的副本。

弹劾的目标便是昨日误捉了司马康的铁路交通局。

弹章上面别说司马光,就是司马二字都没有提。只是在说东京车站的官吏以权谋私,妄捕良善,给贿赂便放行,若不给好处便关押起来,更进一步说铁路交通局管理混乱,上下皆是汲汲营私之徒。

交通二字,本是交相通达、交往、甚至还有勾结之意。赋予其运输新意,是韩冈的主意。整个衙门上下都是韩冈的党羽,这篇弹章想跟谁过不去,不用想就知道。

铁路交通局的品级虽不高,只是中书门下下辖的一个二级衙门,与火器局地位相当,但其重要性,没有人看不明白。只可惜给韩冈布置得水泼不进,这回有机会,有心人当然不会放过。

但这个弹章,韩冈完全没放在心上,只要车站每天的净利润还能保持在两百贯上下,只要这笔利润还能不断增加,只要还没捅下大篓子,一百本弹章都没用。

只是来自于铁路交通监内部的报告,让韩冈心头火起。

“那个小吏被抓起来了?好么……”韩冈将公函丢到了桌上,脸上不见喜愠,问身前的宗泽,“汝霖,你怎么看?”

“孝景皇帝被阻于细柳营外,未闻处罚了守门的士卒。”宗泽立刻道:“规矩就是规矩,无有规矩,不成方圆。如果是依照规矩行事,如何要治罪?司马侍郎虽有他处不是,但人品毋庸置疑,岂会因己病而迁怒于小人?”

司马康为父求医,不辞跋涉,这是至孝没错,此举中途却被人干扰,并非他人有意作祟,而是他准备不周,怎么能够怪罪车站里的小吏?

韩冈点头,“车站人流汇集,龙蛇混杂。不以峻法约束,迟早变成祸乱之源。司马康若是准备充分一点,岂会有昨日的事?”

东京车站建成才一年多,抓住的扒手就有上百个,全都被送去了西域。而逃票的旅客,也同样抓了不少,只是还不至于将他们也给流放,只是要补票。若是不肯及时补票,也会被拉去打上十几棍,然后让他们做工还债。

虽然说车站内的律法苛刻了一点,可韩冈还是坚持如此。那些敢于破坏铁路,盗窃铁轨、枕木的贼人,以破坏御道的名义,杀了都有数百了,然后是全家流放。

——严刑峻法,才是保证交通顺畅的关键。

这座车站位于城南,向西的一条线通往洛阳,向南则是直通泗州。往东的要跨过的河流不少,向北更是黄河,但这两个方向上,日后肯定还是要修铁路。待到东南西北的铁路汇聚于京师,可想而知东京车站到底会有多少人流量。如果现在不管严一点,到时候,就不知会有多乱了。

“不过……”宗泽沉声,“这一次是有心人想浑水摸鱼,并不是要剔司马侍郎打抱不平。”

“这是当然的。”

韩冈点头,期待着宗泽接下来的分析。但宗泽就没下文了,好像提醒了韩冈一句,已经还了人情。

韩冈心中苦笑,他知道宗泽的性格,不喜欢朝堂上狗咬狗一般的党争,能多说一句,足见人情了。可宗泽这个性格,若不是遇到国家危难的时候,一辈子都出不了头。

但这点小事,韩冈也不会很在意。

聊了几句公事,挥手示意宗泽出去,看着桌上的弹章和公函,韩冈的表情严肃了起来。

司马光是旧党标志性的人物,所谓赤帜。不管对他的感觉如何,若苛待朝廷重臣,免不了让其他文官有兔死狐悲之叹。所以这就是一个机会。

现在既然看到弹章了,这件事的性质,就已经变了。

富弼三年前去世,司马光病中垂死,文彦博更是苟延残喘,旧党的核心层几乎凋零殆尽。剩下的一些人,如范纯仁、吕大防,颇有些名气,却是连朝堂也进不了。

旧党怨气深重那是有理由的,他们想要一个发泄的机会,已经想了很久。而新党因为地位骤降,同样是怨气深重。,旧日旧党能为了跟新党过不去而支持自己,现在因为新党同样无缘政事堂,旧党把新党当做可以联手的对象,也不足为奇。

节操这东西,向来不存在政客心中。
