\section{第20章 冥冥鬼神有也无(18)}

【明日复明日,明日何其多,这话说得脸都红了。要补上的一更还是赶不出来,又要拖明天,还请书友们见谅。】

自从门州陷落,洪真太子殉国的消息传来之后,升龙府中立刻加强了防备,城上城下皆是戒备森严,而城中的官员、百姓,也都是收拾了家中财物,准备见势不妙的时候,就南下暂避。一时间,风声鹤唳、草木皆兵。李常杰对此也无可奈何,只能等着有人先跳出来后,再来杀一儆百。

升龙府的城头上,一队队巡卒绕城而行,李常杰站在北门的城楼上,向下窥探着。只发现守城的士兵们一个个都缺乏精气神,抬脚落步都是没有多少魄力。

‘这一仗可怎么打?’

没有自信心的队伍根本就排不上任何用场,吓破了胆的士兵连草寇都赢不了。

李常杰的双手紧紧抓着窗棱,脸色阴沉得如同雨季的天空,与这些天来艳阳高照的天气截然不同。

宋军携胜势大举南下,而一众溪洞蛮部也随着宋人的南下,一同攻入大越国的腹心地带。两敌皆是凶名卓著,尤其是溪洞蛮部,半年来连续越境屠戮百姓,让富良江两岸的百姓畏之如虎狼。紧接着又有宋人要对所有俘获的男丁都处以刖刑作为报复的传言出来。

尽管这让许多人坚定了抵抗到底的决心,但普通的百姓对宋人畏惧又更甚了一筹。如今城中上下,全都将抵挡敌军的希望放在了富良江之上,对大越官军,已经是一点信心都不报了。

李常杰对富良江也给予了极大的期望,地利本来就是最可信赖的倚仗。可当他略略移动视线,在雨季时,能涨到接近城墙的富良江,如今则曝露出了近百步的河滩,在烈日的照射下,一块块龟裂成龟背上的图样。宽阔的江面,只剩下一里半的宽度,水流也平缓得如同清风吹过的湖面。

视线又投得更远了点,隔着两里地远,根本看不清北岸的情形,但李常杰总觉得隐隐约约有哭喊声传进自己的耳朵里来,那里是来自北方的流民们发出的声音。

双眼凝视着水雾掩映的深处,李常杰久久也不稍动一下,如同一座雕塑,凝固在升龙府北门城楼的上方。

“太尉……”身后的亲信试探着李常杰的心思,“宋人还没到,还是先派船接一些过来。”

李常杰终于有了动静,他牙齿咬紧,从牙缝挤出声音:“片板不得过江,违令者族诛!”过了半天,他才又颓然低声,“他们来得太迟了……”

大越国东面有千里鲸波,西面有高山峻岭,防线都放在南北国境上。在富良江两岸的平原,每座城镇都没有城墙,好一点的用木栅栏,差一点的干脆就是竹篱笆。倒是北岸有座城池,是六十年前被废弃的旧螺城,也就是升龙府的前身,几十年没有整修一次,在一年年的雨季旱季的交替中,全都毁坏了。

无险可恃,在确定了宋军即将南侵的时候,李常杰就打算着在富良江设立防线,北面则坚壁清野,将江北的百姓迁移到江南来,让宋军和溪峒蛮军无所收获。大越气候炎热多雨,远胜广西,宋人不可能在北岸久居,只要手中有人有兵,等宋军撤退之后,可以轻易的将失土收复。

但李常杰哪里能想到,坚壁清野的命令传达下去之后,根本就没有起到多少作用,反倒是让北方的官吏们有了弃职而逃的借口。

坚壁清野四个字好写,也好说,就是不好做。

至今为止,已经一个多月了,逃到江南来的百姓数量,只有应有数目的两成而已。并不是百姓不怕宋人的威势,而是江北的州县官们太怕了,直接丢下他们就逃了回来。没有官员的传达、组织和驱赶,乡下消息闭塞的农户如何能知道朝廷的命令究竟是什么?如何知道该怎么去完成坚壁清野的任务?

据李常杰派出去的细作回报,北面的州县官们,绝大部分都只是在城门口将坚壁清野、回撤过江的诏令张榜公布,然后自顾自的收拾起行装,也不见他们派人下乡催促。一听到门州陷落之后,就带着家人仆婢以及金银细软,急匆匆弃职南归。而跟随他们一起回来的全都是城镇中的坊廓户,而不是擅长种地的农户,接近九成的户口都给丢在了北方。

直到此时,门州陷落,而宋军、蛮军终于攻入江北平原,南逃的难民才有一下多了起来,近十万人拥挤在富良江北百里江岸上,求着有一船渡江。但江面上,无论是渡船还是渔船,所有的船只都给收到了南岸,除了战船以外,全都拖到了岸上。

宋军想要过江,就需要船只。只要不在北岸留下船只,他们就不过不了江。宋人要打造战船,肯定是缓不济急。使用木排也不是不可以,但富良江上还有拥有几十艘大小战船的大越水师,小小的木排只要战船压过去,就能给压碎掉。

李常杰没有把握在陆地上击败宋军,而且就算他拍着胸脯说有把握,也没人会相信。之前败得实在太惨了,国中上下对宋军的畏惧已经刻到了骨头里。只有靠着富良江天险来阻挡敌军的来势,才能让人放心。江面上的船只,是升龙府中最后的信心所在。每个人都想救对岸的百姓,可万一北面的流民们混入了宋军的先锋,趁机抢夺船只,那可就是将天险双手奉上的最愚蠢的行为。

‘要是李洪真能多拖一阵就好了,真是个废物!’李常杰心中恨声不已。

他让李洪真去北方,本来就没有安着好心思,只是想着‘攘外必先安内’六个字而已。如果李洪真城破后被宋军俘虏,李常杰就能给他栽一个叛逆之罪,名正言顺的将升龙府内部剩余的隐患给去除。如果他不降,丢城失地的跑回来,也是一样的结果。若是出现了奇迹,让他守住了门州,就算只有十天半个月,那也是一桩美事。

没想到他竟然死战……应该叫战死——宋军攻下门州,就只用了一个时辰,没资格叫死战。这只能说李洪真完全是个废物,什么事都做不好。可他这样一死,什么罪名都加不到他头上了。

相处了几十年,李洪真无能而不自知的愚蠢,李常杰是一清二楚,可怎么都没有发现李洪真有着如此刚烈的脾气,竟然与城偕亡,以至于不得不对他和一同殉国的将领加以褒奖,并荫封家人。内部的裂痕,尽管变得更大了,李常杰也没办法下手消除隐患。

脚步声在身后的楼梯上响起,一名内侍匆匆上了城楼,跪倒在李常杰的身后,“太尉,太后有旨,请太尉速来黄龙庙。”

“又出了什么事?”李常杰慢慢的转过身来,“太后怎么去了黄龙庙?”

“回太尉的话,太后和皇帝是为了求雨才去的。”内侍恭声说道。

“下雨吗?”李常杰仰起头,旱季炽烈的阳光就照在江面上,波光粼粼。

升龙府原是唐时安南都护府治所。原名罗城,升龙之名,是因为如今李朝太祖李公蕴得国之后,从旧都华闾迁都来此,一日在江面上看见有黄龙升空,故改名做升龙府,之后的几十年,隔三差五的也能在富良江上见到黄龙出没。

几十年来,黄龙已经成了大越天子的象征,所以立庙祭祀。城中的黄龙庙,每年的祭祀从来没少过,求子嗣、求富贵、求安康,香火也是极盛。不过求雨的情况,倒是没有。旱季求也没用,雨季不需要求都是没日没夜的在下。

‘要真能下雨就好了。’李常杰想着。

到了黄龙庙,街道上已经被班直封锁,一阵浓烈的檀香味扑鼻而来,还能看见车驾和肩舆停在外面的庭院中。

在黄龙庙的正殿中,倚兰太后身着大礼时的祭服,正在神台前合什默祷,大越国的天子李乾德,则跪在一边,也是在祷告着什么。听见殿门处的响声,倚兰太后便盈盈转过身来。一层面纱遮住了玉容,宽大的衣袍遮住了身形,只有声音清越:“太尉来了。”

李常杰欠了欠身,算是行了礼:“不知太后有何事吩咐老臣?”

“太尉劳苦功高,我母子二人和大越国,全都要靠太尉主张,如何敢说吩咐。”倚兰一手搂着站起身来的儿子,“只是听说宋人能驱使鬼神,门州城一攻即破,就是靠着鬼神之力,不知太尉可否知道此言真伪?”

“无稽之谈!”李常杰很不客气。积威之下,李乾德身子微颤,向着他的母亲身边又靠近了一点。

出身低微的倚兰太后却没有害怕的感觉,轻声问道:“听说之前太尉下令处决了几名逃兵?”

“的确是老臣下令杀的。”李常杰有些不耐烦,他的确是把回来报信的士兵灭了口,但这又算得了什么,“只要宋人过不了江,待到雨季降临,纵能驱使妖魔鬼怪,也奈何不了我大越。”

