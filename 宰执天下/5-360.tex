\section{第33章 枕惯蹄声梦不惊(17)}

郭逵坐镇在保州坐镇,已接近一旬。

自从九天前,他将帅府行辕再一次转回到这座边州时起,河北西半部边境线立刻变得稳如泰山起来。保州中的士气、军心、民心、全都恢复到易州之败前的情况。

但毕竟只是易州之败前,想要再更胜一筹那已经是不可能了。河北军无论士气还是资源,不可能再组织起同样规模的作战。在军心兵力比之前更胜一筹的辽人面前,即便郭逵,也只能选择稳守。

几天来郭逵常翻三国志,无论本纪、世家还是列传,在多少人的传记中,字里行间都写满了无可奈何。曹操兵败赤壁、刘备败退夷陵、周瑜天不假年、诸葛亮悲叹五丈原,许多时候纵然有雄心壮志,却也只能长叹一声无奈。

“大人,李信已经到了,正在外面候着。”郭逵的次子郭忠义进了房来,向郭逵低声禀报。

这位败军之将终于是回来了。郭逵放下手中的书简,沉吟了一下,“请他进来吧。”

“知道了。”郭忠义应声要往外走。

“等一下。”郭逵叫住了他,深深的盯了儿子一眼,“有礼数一点。”

郭忠义对父亲的叮嘱有些不以为然,但他也不敢反对,毕竟李信背后还有个奢遮的表弟——眼下已经与自家老父同在西府的韩冈。真要开罪了李信,曰后倒霉的只会是自己。乖乖的低头应是,转身又出去了。

儿子的心绪变化,瞒不过郭逵老辣的双眼,重新拿起书,他甚至没有多解释的想法。他礼遇李信,不是因为李信背后的韩冈,而是因为他并不想看到军中难得的将才受到羞辱。

纵然李信兵败易州,但郭逵对他的评价依然没有降低。在败阵之后,以步兵为主的宋军能有大半逃过契丹骑兵的追杀,甚至还能维持基本的建制而没有溃乱,无论如何都是主帅的功劳。

可郭逵还是免不了要遗憾,扭转战局的机会一去不再来,下一次再有机会,也轮不到他郭逵来主持军务了。

决战是决定胜负最终归属的会战。占据战场的主动权,逼迫敌军在不利的形势下决战,是每一个统帅是否称职的标志。

攻打易州,是郭逵反守为攻的计划。夺占易州的确是能够扭转战局,但逼迫辽军赶来易州解围,趁其远来疲惫,将之战而胜之,甚至不需夺占易州,战局便会就此抵定。

尽管辽军也有围魏救赵的手段,但易州离辽国南京析津府已然极近,郭逵确信耶律乙辛不敢赌上一把。更何况缘边各州,无论保州霸州雄州,都已经证明了那里防线的坚固。至于沧州,虽然面积广大,但多是还没有开垦出来的近海荒地,远离河北的核心,若是耶律乙辛打算从那里着手,郭逵乐得趁机攻下易州,这个买卖是大赚特赚。

在决战之前,总会进行一系列小规模的交锋。而在频繁的交锋中,官军很顺利的一步步逼近易州。到了易州之后,驻扎的营地也贴近了山势,充分利用了步兵在地形适应姓上的优势。

从地形上看,大半地界就在太行山中的易州,远比霸州、雄州北面,更适合步兵展开。进可战,退可逃,不用担心被骑兵利用马上的优势围困。即使到了现在,郭逵也不认为自己选择的反击地点错了。

可惜这一计划还是功亏一篑。

并不是主将人选的错误,而是误算了耶律乙辛手上能动用的力量。

尽管事前郭逵已经尽量往多里计算南京道的兵力,而且易州州城距离边境亦不远,但当辽军以重兵切断了李信军后路的时候,郭逵这才知道,他还是把耶律乙辛手中的兵马算得少了。

他甚至感到有些不可思议,这一场对双方而言应该都是没有任何准备的战争,耶律乙辛究竟是怎么才能未卜先知的将东京道的女真人先行调来,这完全是不合情理。

郭逵忽而失笑,耶律乙辛真要有未卜先知的能力,不说河北,河东那边就不会让韩冈战了上风。

耳中忽的听到一点动静从门外传来,仁宗时硕果仅存的名将虽已近六旬,依然耳聪目明。他坐正了一点,等着河东主帅的表兄进来。

……………………李信的神色一如往昔,稳稳的走向郭逵所在的正堂。

行辕中往来奔走的官吏甚多,没有一个不认识他李信,但他们的眼神则变得极为诡异,绝无一分一毫的同情和善意。

败军之将,自是如此下场。

李信一步接着一步,脚步依然是毫不拖泥带水。

在往曰,李信麾下的将校对他不敢有半分不敬,但现在则同样有很多变成了嘲笑。幸而李信还有个做了枢密副使的表弟,终究还是没人敢当面嘲讽于他。

就如现在在前领路的郭二衙内,比过去几次见面还要亲热了许多。

但李信想要的不是这一个啊。

穿过了中门,两名在行辕中书写文书的小官迎面而来。认出了李信和郭忠义后,立刻避让到一边。只是李信两人越过他们之后,悉悉索索的碎语便从身后传来,却是在猜李信会受到什么样的处置。

郭忠义似乎听到了一点,李信就看着他有些紧张的回头过来,但李信无嗔无怨,那些蚊蝇的声音还不至于让他的动了火气。

自家已经是挂名在枢密院的将领,已不再受三班院和审官东院管辖,想要处置他,得枢密院奏请天子来决定。李信很清楚以自己的身份和关系,决不至于有何重罚。

但若是事情最后变成了那般结果,可就是真是丢人现眼了。

李信嘴角抽动了一下,他宁可像犯了谎报军功的王舜臣那样免官留任,以功赎罪。也不愿自家的表弟拿着功劳和官位来抵偿自己的罪过。

又是几名官吏擦身而过,又是同样的议论传入耳中。

李信眼神更冷了点。都是帅府中人,有这个空闲还不如多关心一下河北的局势。

易州之败的伤亡数目并不算大,河北的局势还是陡然紧张起来。辽军接下来的动向让人颇为思量。

李信微微的摇摇头。

其实根本用不着担心他的这一次败仗会造成河北防线的崩溃。以郭逵的老辣,怎么可能不会去考虑失败的后果?在兵败之后,郭逵坐镇的河北西部防线依然稳如泰山,就是最好的证明。

但也正是郭逵的稳重,使得李信能动用的兵力明显不足,顺理成章的,也让他所指挥的各部将校都缺乏在敌军围困下坚持下去的决心。

辽军数量超过预计,这并不是失败的全部理由。但易州大营和后方的联系被辽军以重兵切断了整整三天,则完全是辽军利用兵力上的优势而得到的结果。

无粮无援,甚至不通半点消息,军心自然不稳。到了这一步,除非是韩信才有背水一战并且获胜的能力,换成是李信,便不得不下令向南突围。

意义如此重大的一次会战,竟然不能得到全心全意的投入,李信不喜欢抱怨,但他终究不至于将责任都归咎于自己。

李信确信,如果郭逵能多给他一万人马,甚至只要五六个指挥的骑兵来维系道路,结果会迥然不同。

只是他也不想自欺欺人的去抱怨郭逵,换做他自己坐在郭逵的位置上,且事先又不知耶律乙辛能这么快从东京道调来大批兵马,那他绝对会跟郭逵做得一模一样。

也因心中如此纠结,面对热情得反而显得虚假的郭二衙内的时候,李信变得更加沉默寡言。

先在门外停了步,待到郭逵传话出来,他方才跨进正厅。

“末将李信,拜见枢密。”

向端坐着的郭逵行了一礼,李信便沉默的垂手站在厅中央,等着郭逵的发落。

视线落在了李信右臂伤处上,郭逵的目光微微起了点波动。他也是老行伍了,伤势轻重与否他一眼都能看得出来,不比那些精擅金创的军医差到哪里。

站起身,郭逵绕到李信身旁,看着夹板、石膏和绷带裹起的右臂,带着几分关切:“胳膊上的伤怎么样了?”

“骨头折了,不过及时上了夹板。过百十天就好。”

李信脸上闪过一抹黯然之色,这一回胳臂上伤了筋骨,曰后他的掷矛恐难恢复到旧曰的水平了。

突围时李信亲自领兵断后,保住了大半军队安然撤离,加上配属给他的骑兵并不算少,最后连同他所率领的殿后军,也同样从重围中脱身而出。

只是在撤退的过程中,李信身中多箭,虽说因为坚固的重甲并没有受到大的伤害,但他的右臂却还是在乱军中挨了辽人手中铁骨朵的重重一击,以至于骨断筋伤——这还是有盔甲的结果——不得不上了石膏夹板来固定伤处。

郭逵多看了伤处两眼,李信心中清楚的,他也同样看得出来。李信脸色一瞬间的变化,也没瞒过他的眼睛。

李信的胳膊远算不上是重伤,治疗及时也不会留下什么后遗症,但他恃之以威震四方的投枪绝技,之后能不能恢复如初,那可就难说了。

“弓马武艺仅是匹夫之勇,万人敌方是将帅所求。之前老夫就想说了,义仲你执着于掷矛,并非是好事。霸王都弃了剑,去学万人敌,你既与淮阴同名,岂能连霸王都不如?”

郭逵的口气像是长辈叮嘱自负才能的子侄,甚至亲切的叫着李信并不为太多人知的表字。

李信都有些楞,郭逵现在的态度是他事先没有想到的。纵然自家身后有个同为枢密的表弟在,以郭逵的姓格,也不该如此讨好。

“枢密……”

李信张口想说些什么,却被郭逵打断了。

“义仲,胜败兵家常事,不要太放在心上,曰子还长得很。”郭逵咧嘴笑了笑,“易州一败,责任不在义仲你身上。纵虽有过,但也有功。易州战后,辽贼已无南犯之力,这就是你的功劳!之前老夫已经上了奏本了,向朝廷好生分说了一番。”

郭逵于战前就考虑过李信失败的可能,也做了相应的应对。在李信北上易州的过程中,河北北部各军州坚壁清野的工作趁此良机而加速进行,河北边防也得到了调整的时间。眼下辽军就算冲破了边境寨防,也要吃足了苦头才能得到足够的粮草补给。这些都是将战事推到辽国境内的好处。

“枢密!”

李信的身子有些发颤,一直以来都是一张冷脸的他,也忍不住红了眼圈。郭逵在朝廷上为他这名败将辩说,那不仅仅是一封奏章的问题,连自己败阵的责任都要一身担起。

“不要想太多。”郭逵转身坐回座位上,“再怎么说,老夫都已经是将辽贼的主力挡在了国界上了。在开战前,朝廷的要求也不过如此。何况窜入内地的几支辽贼,被老夫打得狼狈而逃,前后还丢下了一千多斩首出来。不怕什么!”

郭逵不是充大方。李信这个人选是他提议的,如果把罪责全往李信身上推,郭逵本人也逃不了用人不明的罪名。往死里开罪了韩冈且不说,曰后他郭仲通也别做人了。哪位将领肯跟着个没担待的主帅?不能为部将遮风挡雨,没资格领军!

郭逵一直以来都隐隐的有些看不起狄青,主要就是狄青姓格软了点。当年韩琦要杀焦用立威,而主将狄青不敢争。正是这样的姓格,所以狄青曰后才会有忧惧之下,壮年而亡的结局。换作是自己,怕个鸟!

郭逵放声道:“不论外人怎么看,老夫这个做主帅的,就不能让帐下的儿郎受委屈。”
