\section{第33章 为日觅月议乾坤(11)}

屋外,堂吏来往奔走,偏偏脚下片尘不惊。

当宰辅们济济一堂,经过门前时,堂吏们就连大气也不敢喘。

不过有几分是深怕惊扰到屋中的宰辅,有几分是担心气喘粗了,听不见里面的争论,这还真说不准。

熊本倒是可以确定,现在至少有二十对耳朵朝着这间公厅竖着。

将注意力从屋外转回屋内,熊本就见邓润甫指着那片纸,质问上首处的韩冈,“敢问相公,何为童生?“

所谓童生,如果只是顾名思义,那就是小学生。这是谁都能明白的。但与进士归为一类,那肯定就不一样了。就像是秀才,寻常称呼读书人的,现在同样与进士写在一处,谁也不会还觉得是寻常的称呼。

“蒙学毕业,便是童生。”

当太后留下韩冈的时候,熊本就知道他们会怎么做。

想要让太后继续垂帘,第一就要收买人心。若韩冈不说,过了今晚,熊本也要上书。之前在殿上一同请太后继续垂帘,有几分算是形势所迫,可既然上了贼船,又跳不下去,也只能从贼了。

计议出了什么对策,还得尽快颁布,至少是得尽快宣传出去,要不然,今天在内东门小殿中的一番话传扬开,宰辅们少不了为人诟病。尽管理由光明正大,大部分朝臣还是会站在他们一边,可身上背负的骂名难道不是越少越好。

不过这套童生、秀才、举人、进士的玩意儿,熊本事前还真没想到。

蒙学的毕业考是由县中主持,算是官方的考试,过去按照通过人数多寡,蒙学的主办者,以及当地的知县、教谕还有更高层的学政都会都到相应的处罚和奖赏。而蒙学毕业的学生,过去并没有太注重,只是希望由此减少文盲数量——韩冈创造的这个词可谓是精到——实现有教无类的梦想。

但现在看来,韩冈怕是已经计划很久了,用以更进一步的收买人心。

“做了童生有什么好处?”

熊本能想到的东西,邓润甫也不可能想不到。

“好处?自然是丁税。”

“免征?!”

“蒙学的学生数目可不少。”

邓润甫和曾孝宽先后说道。

两人不是反对,而是想知道韩冈怎么解决。

韩冈道:“商税增加的数目,足以抵得上人丁税了。”

先不说读书人有多少,丁税数量本也不多,多的是附在丁税之后的折变。

丁税本身一般在两三百文之间,有地方少至百文,但也有地方多到四五百文,开国初年,甚至曾经有七百文的情况。如果是纳粮,则是三五斗不等,多的时候也曾达到一石的。

如果是在京师,四五百文最多也就半月苦工的收入。而贫困之地,丁税的数额一般也不会高。七百文、一石米的例子,基本上都是出自江南的鱼米之乡,而且是开国初年,延续了吴越、南唐的税收额度才会如此之高。

最重要的,是朝廷经常免去某个地方的丁税。因为天灾,因为战乱,都会减免丁税。相比起夏秋两税来,丁税的数量并不算多,减免一部分并不会影响朝廷的财政收入太多。

“从此之后,怕是人人都要让自家的儿子去上学了。”邓润甫叹道。

家中子弟去读书,只要通过县中组织的考试,就能拿到童生的资格。名额没有限制,达到标准就可以。只要三年,让七八岁的小孩子上三年学,这么简单就能免去日后几十年的丁税,有几个人算不过这笔账?

韩冈怡然颔首,而曾孝宽却又叹道,“日后作弊者必多如牛毛,无法禁绝。”

“有学政在,让学政去管。”韩冈毫无责任心的说着。

怎么杜绝作弊,还有在作弊之后怎么查出来,办法都是人想的,魔高一尺,道高一丈嘛。

“时日一长,丁税怕就是再也收不上来了。”

“丁税本就行之无据。”章惇突然说道,“本朝税制,上承唐之两税,而两税法,本就是将旧制租庸调归并为夏秋二税,身丁钱亦纳入其中。晚唐五代,天下战乱频频,为军资故,各地复征身丁钱,尤其以南方为重。而本朝开国之后,相因承袭,并未恢复旧制。”

“免行钱怎么办?”熊本问道。

这是免除夫役要缴纳的税收,说起来也算是人头税的一种,而且数量还不少,并不下于丁税。

“做了秀才便可免除,而且成为秀才之后,便可以游学天下,不需要地方开具的路引。”

方便做生意吗?熊本暗自摇头。韩冈看样子就是想要做到两全其美,一方面收买人心,一方面还要为气学张目,打得一手好算盘。

“秀才的资格是小学毕业?”曾孝宽问道。

“当然。”

“举人呢?解试?”“不过通过了发解试便是举人,那日后是不是都可以上京赶考了?”

韩冈笑道:“方才太后还有子厚兄都这么问过韩冈,温伯不必担心。”

邓润甫摇头,“润甫担心什么,要是有了举人资格,就能上京赶考,受益最多的就是福建、江西了。”

宋代地方上的解试,与明清不同,就是在于通过解试,却没考中进士,下次还得从解试开始。

福建、江西文风昌盛,不知有多少一榜不中然后回头再考的士子,要不是朝廷控制福建、江西各州每年取解的名额,说不定三分之一的进士名额都能给他们给拿去。就现在,平均每科都至少有十分之一的进士数量。

韩冈要是让那些通过了一次解试的士子,从此不用再过解试一关,那么南方、尤其是福建、江西两地的士子,怕是能占据朝堂的半壁江山。

韩冈当然不会这么做。

“想要赶考,当然还是依照旧制,有发解资格才能上京参加礼部试。但考中一次之后,就是举人。”

“举人的好处呢?”邓润甫顺理成章的推断,“不会是免征田赋?”

“田赋不能免,家中如果有工坊,工坊税收可减半。”

“相公这是要鼓励世人去开办工坊了?”

韩冈道:“一亩田一个人都养不了,但面积一亩的工坊,养上十几人都没问题。有恒产者有恒心,能吃饱穿暖,就不会跟着人去造反了。”

“相公是意在诸科?”熊本直言不讳。

韩冈理直气壮的点头,“诸科贡生当然也会有相同的权力。”

除去进士科外,明法、明算、明工诸科,也都有举试。韩冈怎么可能会忘掉自己的基本盘?

“相公的打算当不止于此?”

有能力开办工坊的士人可并不多。光是这一个税收打折的好处,吸引不了所有人。

“另有边地赐地三顷,只要愿意去边境,登时就是一个地主。”

曾孝宽摇头,“恐怕不会有多少人愿意去。”

“会直接发给授田券,允许其转让。”

授田券变成可以流通的有价证券,这也算是成为举人的好处。给钱,给待遇,至于参政议政之权,那就要靠他们自己去争取。

“会有人买?”

就连熊本都觉得有几分不靠谱。

“多少是笔钱。”

韩冈无意向同僚解释太多,让事实告诉他们就行了。

“此外举人还可为官,总不能让地方庶务操纵于吏员之手。既然吏员也有俸禄,其实官吏也无甚大分别了。县中六曹,都可以让举人去做。虽无品级,并不入流,但终究还是有积劳入流的机会。同样是士人,就不必像约束胥吏那般,一年才几十人释褐入官。”

“只怕读书人无人甘愿操持贱役。”

“只要有好处,迟早有人愿意去干的。先从一个地方试点,然后慢慢推行。”

举人与现今贡生之间的差别,就是身份固定。就算考中之后什么都不干,举人还是举人。而贡生却是一次性的,除非接连五六次不过,那才能当一个免解贡生。但在政治上,与其他读书人没有任何待遇上的区别。

举人可以做很多事了,不论是开工厂,还是去开荒,又或是去参与吏职,都是受到政府鼓励。

尤其是诸科举人,数量日后还在进士科举人之上,当进士科的举人皓首穷经的时候,他们也就有机会去把持地方庶务了。地方稳固,那么朝堂上也会有所反应。

“那么官宦子弟呢?”曾孝宽关心的问着。

照常规,官宦子弟或是官员本身,都要另外安排发解试,名为别头试,锁厅试。名义上是避免他们与寒门士人相争,实际上取中的比例远高于地方州郡举行的发解试,是彻头彻尾的优待。

“自然是一如既往。该锁厅的锁厅,该别头的别头。”

这是在示好天下士子,同时也不会侵犯宦门子弟的权益,想要去考进士,还是要通过当科的解试,官宦子弟在解试上的优遇依然能够保持,同时一个举人的身份也比其他人更加容易获得。

熊本心中暗叹。

古语曾言,小惠未至,民弗从也。如今韩冈欲普惠天下,民……从也不从?

熊本不知道,但他知道,吕惠卿这次是输定了。

看来没有押错庄。
