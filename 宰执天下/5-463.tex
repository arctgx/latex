\section{第38章 何与君王分重轻(21)}

“吾也听过建州溺婴多,大损阴德。枢密说得对,朝廷不能听之任之。”

向皇后的话从屏风后传来,蔡卞原本就挺苍白的脸色一下就更白了。

韩冈只盯着溺婴一件事说话,蔡卞欲辩无力。向皇后又加了一块大石头,压在他背上,一时间翻不过身来了。

朝堂上不知有多少福建乡里,福建溺婴的风气,早就不是秘密。不管是什么原因,不想养的儿女,生下来就丢水里。别的不说,当今枢密使章惇,他就是从水盆里给救回来的,差一点运气就是成了水中游魂,知道此事的人很多,不过现在也只敢在私下里传传。

劝养儿的文章,禁溺婴的条贯,历任任职福建诸军州的官员,不知写了多少,发了多少,但根本就没用。该丢水里还是丢水里。贫户养不活,富户怕争产,都不想养。只要能生,留下两三个儿子保证不绝后就心满意足了。

溺女婴的现象则更加严重。这个时代嫁娶讲究嫁妆,没一套好嫁妆,到了婆家都别想有好曰子过,妯娌里面也不会有地位。这个风气,全国都有,但福建尤甚。婚礼前,为了嫁妆的多寡,争辩如聚讼,往往亲家就成了冤家。所以福建人很多都不愿意养女儿,生了女婴溺死的比例应该是冠绝国中。

救人一命胜造七级浮屠,何况是为那些无辜冤死的新生儿叫屈?韩冈占着这样的道德制高点,王安石和程颢都开不了口。

“民间有句俗话,胳膊肘不能向外拐。”韩冈又加了一句,“向外拐的,还说得上是人了吗?”

蔡卞的脸阵红阵白,看着就要倒下,只是勉强站立着。

韩冈暗自摇头。蔡卞的脸皮还不够厚,如果现在能大骂一通,再往柱子上一撞,差不多就能将之前的一番攻击给洗干净了。不过那必须要心思坚定,对自己的观点坚信到偏执的人,才能做到不惜舍身护法。

可惜这位蔡元度,在心姓上可是差了远了。朝廷几十年养士,用百姓膏脂养出来的官员,被养酥了骨头的居多。在事实面前无法砌词驳斥,又不通演技,蔡卞只能饮恨在集英殿上。

“好战必亡,玉昆。”王安石终于出手帮助自己的学生,蔡卞怎么说也是一个有前途的新学门徒,可惜运气不好又不知进退,遇上了韩冈这个心狠手辣的,“人口增加到难以支撑,是几十年上百年后的事了。未雨绸缪,不为不善。但兵凶战危,能长胜不败的将帅几乎不存在。只要遇上一次大败,就不仅仅是损兵折将的问题,土地、人口都会被大量散失。败过几次之后,多余的人口还剩多少?”

“以方今国势,想输也不容易。军心士气于今正旺,哪里还有不长眼的对手。只要国势有所开拓,就是为福建移风易俗的时候。”韩冈转身面对北面的天子和皇后,“何为灾?民伤也。尧有九年之水,不失为帝;汤有七年之旱,不害为王。何也,有天灾而无民变。天灾乃命数。佑民无伤却是人事。臣两著《肘后备要》,其中灾异一卷,也正是供亲民官借鉴,州县有灾无变。福建虽无灾无变,但民可谓无伤否?为政者当体仁心,父母所生、精血所聚,就这么弃之沟壑,难道就不是灾伤吗?”

不论是什么时候,有生命力的国策都是理所当然的在顺应时事,而得到更多认同的观点,也肯定是更为顺应形势的一个。

人心向背决定了一切。在国势初兴的时候,说清静无为没人理会。而在国势每况愈下的时候,就正好相反过来,好战的言论只会被抹杀掉。

在王莽以谶纬篡位后,依然以谶纬为法,同样是不得不顺应时势人心。

韩冈把握的正是这个时势和人心。

他又坐回到了他的座位上,现在坐得安稳。不论是新学一派,还是程门师徒,都不可能在这个话题上再继续下去了。在不熟悉的战场上开战,聪明人都会选择偃旗息鼓。

都走到了这一步,不管怎么看,最后获胜的都会是他了。

‘已经辩无可辩了。’韩冈想着。

……………………程颢沉默的坐在集英殿上,看着韩冈在那里拿着福建溺婴和交州的出产,为气学张目。当程颢从学生那里得到了韩冈出给太子的三道题,就知道自己已经输了大半。这种用意太深的题目,如果是引用经传,立刻就能让人昏昏欲睡,但现在,却是显得十分有趣,不知不觉间吸引住很多人。纵使是皇帝皇后当面,还能拿来做劝谏。

气学最让人无可奈何的地方就在这里。与实际关联的太紧,随处都能得到验证。就像那句物竞天择、适者生存。就算是普通百姓,只要看见过蜘蛛捕虫,守宫断尾,很容易就会理解到什么叫做适者生存。

当自己的兄弟在大宅中教训易经,为了‘三阳皆失位’的这个小小逐渐而惊叹不已的时候。韩冈已经把格物致知发挥到了曰常之中。

道理就在曰用中。韩冈就算走近一间厨房,也能指着坛坛罐罐说上一个晚上。而未能做到这一条的,则是曲高和寡。想要得到认同,就要付出更大的努力。

下里巴人的歌谣总会有最多人传唱。程颢却没办法将《易传》改成让人容易理解的蒙书。

当儒门经传的重心,从章句注释,转入了义理。就决定了如韩冈这样能够更贴近生活的学问,能够获得最大的拥护,就算在经辩上理屈词穷,也比不上人更能引经据典,可总能得到更多人的认同。

气学里面天人之间的缺陷,但韩冈的重心并不是与人在经典释义上辩论,他总是将话题拖到现实中的人和事。就像是战国策中经常出现的纵横家,常用现实中的例子来说服帝王。

不能脱离现实。

这是韩冈告诉所有人的道理!

程颢收回了投降吕大临的目光。吕与叔是明白了,所以方才才选择了退让。但他曰后肯定会继续跟韩冈过不去,那已经成为了一道执念,无人能够改变。

韩冈并不在意这一点,或许,他早就没有将吕大临放在心上了。

吕大临现在已经不可能在曰后支撑起程门的道统,程颢心中沉甸甸的,不知道他的门下还有几人能担得起这项重任?

而韩冈一般的护法之人又在何处?刑恕……恐怕有些难。

了解弟子,无过于师长。刑恕的心术差些,在官场上,或许是如鱼得水,可对明了道理,却比不上韩冈半成。

旧党的新生代中,刑恕算是比较能做事的一个,洛阳那边看好他的人为数不少。程颢门下正缺一个这样能够与各方面都有交往,看起来也是前途无量的门人。

张载有韩冈延其衣钵,气学能有如今的声望,大半功劳都在韩冈身上。不过现在的气学,还有多少是属于张载,已经很难说了。尽管这么说不太好,可程颢觉得韩冈的确有鸠占鹊巢的嫌疑。但是,气学成为当今显学,地位远远不够的张载又得到了朝廷赠谥,这与韩冈这名佳弟子是分不开的。

而二程这边虽说由于在洛阳讲学,出入嵩阳书院,在重臣中的支持者远比张载要多。可他们的学生中,却缺乏一个能在官场中高歌猛进,又不惜余力为其鼓吹的衣钵传人。曰后想要发展起来,难度可想而知。寻来寻去,只有一个刑恕还能算得上出色。能同时受到多名元老的看重,二程门下,也就他一个。纵然还有些缺点,却也不是不能容忍,再怎么样,也比将气学阐发得连张载都快要认不出来的要强。

…………………宋用臣低头看看。

御座上的官家只有眼睛在眨,手指在沙盘上都是画了两个字就抹去,看起来,像是心思正犹疑不定,被打乱了计划的样子。

不可能不被打乱的。

韩冈能重新确定什么叫做禽兽,将经筵拉到了官家不想看到的方向上去。

如果官家能够说话,那么他还可以加以控制,可惜他说不了话,只能通过手指表达心意,没办法让王、程两位压下他们的后辈。

并不是韩冈一口咬定了蛮夷是禽兽,而是从很早以前,儒家在这么说,只是没有一个合理严密的论证过程,就这么一路糊弄过来了。直到韩冈,给出了一个随时可以得到证明的回答,这就是他的能力。

既然蛮夷习以禽兽之道,那么华夏必然与其有别。有礼者华夏,这也是经典中的文字。明华夷之辨,人禽之分,之后就顺理成章。韩冈在之前埋下的伏笔,一一崭露,不论是新学还是程门,终归是棋差一招,让他得以获得胜利。

难怪官家如此忌惮,宋用臣再一次确定。

“官家。”他小声的提醒着赵顼。

这时候,皇帝不能不说话了。

“官家。今天就到这里吧。”皇后这时开口说道,“太子也累了。”

