\section{第11章 城下马鸣谁与守(八)}

李清说得大方,不过这也是武贵的话说到他心里去的缘故。

【这几天调整工作,应酬多了点,缺了不少更新,实在是很对不住。今天三更,补上一点是一点,这是第一更。】

说进他心里的当然不是武贵的借口。对面熟人兄弟多?富贵功名面前,什么都是狗屁!

只是这时候,有必要在宋人那里留下太多血债吗?再怎么说大宋如今国势强盛,一次两次的败仗根本改变不了大局。

灵州城下的战绩,是宋国用错了人,做错了事,西夏这边又靠了计谋,而且没有辽人帮助,根本就没有赢的可能。换作是正面交锋,李清一点信心都没有。

就算在盐州城下再胜一次又能如何?不是宋兵不善战,而是宋国的皇帝用错了主帅。郭逵、王韶、韩冈这些知兵善战的主帅可是一个都没有出现,任用的都是李宪、王中正这样的阉人,以及徐禧、高遵裕一般贪功无能之辈。而种谔、曲珍等西军名将也被打压、约束。就这样还能一直压着西夏军打。一旦宋人走马换将,主帅、名将重新被起用,调来陕西主持大局,局面很可能瞬间逆转。

如同灵州之役式的胜利如果再来一次就不会再有西夏国了。不仅仅是李清,国中的任何一名重臣对此看得十分清楚。如今还在进行中的这一场战争,已经耗尽了西夏的国力。

这一仗即使赢了,耗尽的元气也不可能恢复。生民流离、部族残破、田地荒芜、商路断绝,村庄、城池一个个化为残垣断壁。坚壁清野的战略本来就是拼着自己重伤,也要赢下对手,是在灭亡和重伤两者之间做出的无奈选择。其造成的损失,几十年都没办法恢复。

等到宋人下一次攻来,只要宋国那边能一个中人之姿、又不贪功的主帅,就能将灭国的胜利给东京城中的宋国皇帝带回去。

既然西夏覆亡是在所难免,无论如何李清都不可能做得太过火,他还打算换个东家。而投降了大宋之后,朝廷或许不会计较他过去犯下的各项罪行,但他又不是有自己的部族,只能在西军中安生,到时候会有什么样的待遇,就得看自己的表现了。李清可不想将仇怨结得太深。

盐州到底能不能攻下来?他对此不是很在意,但私底下还是认为攻下来的好。要改换门庭,也得手上有个好东西可以卖出去。

李清其实很遗憾,要不是当初没有领到去灵州助守的差事,加上身边被安插的耳目太多,他也不会在西夏继续留下来。

之前李清没能领到去灵州的差事,本想等到宋军攻到兴庆府之后,便立刻发难。开城门的事不用人教也是会的。只可惜他没有等到宋军进抵兴庆府城下。

现在又是到了关键的时候。依眼下的情形如果宋人输了,那一个改变战局的机会,就能卖到更高的价码——锦上添花,哪及得上雪中送炭回报来得大?!

可惜这些盘算都是不能对外人说的。除了几个亲兵之外,李清他其他人都信不过。但要跟宋人联络,必须得有一个居中奔走的信使,要有口才、有头脑、能讨价还价,几个亲信没一个能适任。不肯与宋人为敌的武贵,自然是最合适的人选。

武贵来自于大宋,跟许许多多投奔西夏的汉人一样,都是犯了罪,逃避刑罚。一般来说,这些人其实绝大部分在西夏也混不出头来,有许多甚至都沦为奴仆。纯由汉人组成的撞令郎,更是西夏军中一支专门用来陷阵敢死的军队。

并不是人人都能比得上张元、吴昊,或是景询,乃至李清。他们这几位都是本身就有能力,只是在宋国时运不济,换了个地方就能出头了。

在李清看来武贵也是这样的人。武贵胆量很大,眼光见识也不差,还能识文断字,所以李清见了就一力提拔。

只是武贵身手虽然灵活,可惜武艺不算出众,日常演武,其表现出来的弓马皆是平平,也难怪他之前在宋国西军中混不出头来。没点好武艺,如何能在军中安身?文人在行伍间也颇受尊重,可武贵还比不上一个乡措大会咬文嚼字。文不成武不就,就是有见识有谋略,也没处投奔,最后犯了事,也只能来西夏谋个出路。

李清啧啧嘴,也幸好武贵在军中人缘不错,自己的越次提拔也没有让他惹来多少嫉妒,让他留守营中倒还正合适。

武贵没管那么多,也不可能知道李清在想些什么,只沉默的跟随李清往营中走。

李清回头望了望盐州城,“白天扎营时,城里面都没派兵出来干扰,想来今天夜里多半会热闹一点。”

“太尉尽可放心。天上有飞船里的人盯着,外面又有六十多暗哨,多是精通隐匿藏形、伏地听声,盐州城中大队人马一旦出城,立刻就能发现。盐州城左近能藏兵的地方,末将也都安排了人手去查探过,并没有发现被埋下伏兵。”

“如此甚好。”李清点了点头,却又道:“不过就是能防着大队人马,若是十几骑出来扰营,那就是防不胜防了。”

“疲兵之策从来都不会少。末将前面奉太尉之命,也安排好了人手绕营巡视,料无大碍,最多有些吵而已。太尉尽管放心,必不致让宋军有机会乱我大营。”

“多亏了有武贵你啊!”李清很是满意。武贵办事的确是妥当,比起之前连字都写不好的一群下属强得太多,他抬手拍拍武贵的肩膀,笑道:“今夜可就全靠你了。”

“末将遵命。”

将今夜营中的事务交代给武贵,李清便回中军大帐休息。不过他也不敢高枕安眠,只敢和衣而睡。

一觉就睡到天亮。安睡了一夜的李清穿戴梳洗好,打了个哈欠,掀帘出帐。对这一夜的安寝,他心中有着几分惊讶。

当守了一夜的武贵过来,李清就皱眉对他道:“感觉不对啊,怎么夜里这么平静。是不是宋人在半路上就被挡住了?”

武贵眼中还带着血丝,他摇摇头,“夜里没有人从城中出来。”

李清吃惊非小,“竟没派人来袭扰?……就是徐禧不懂,曲珍和高永能难道还不懂怎么守城。难道……”他的话突然顿住,忽而又笑了起来:“这仗的赢面看来又大了一分。”

“嗯,多半如此。”

宋军据守盐州,正常来说至少要考虑到提振城中士气,在西夏这边抵达时派出几百名骑兵,好歹干扰一下扎营的工作。夜里也该敲敲鼓,或是派人出城吆喝两声,没有说什么手脚都不做的。

驻扎在野地里的军营,就算再有防备,也不可能是所有人都枕戈待旦,那样明天就不要上阵了——这可是远道而来的军队,必须要好好歇上一夜。除非能确定敌军必然来偷营,否则必然是大部分士兵安寝,少部分士兵守营。这样一来,宋军若是当真来袭扰,造成的混乱不会少。以曲珍、高永能两人的老辣,不可能放过这个机会,可宋军偏偏就没有出现。可能性就只有一个,必然是被徐禧挡住了。

这可是个好消息,李清正笑,一记记鼓声隐隐传入耳中。军中厮混多年,李清对这种声音最为敏感。立刻抬头寻找声音的来源,却见武贵已经向盐州城的方向看过去。

“宋军看来要出战了。”武贵沉声说道。

“不扰敌,不袭营,一大早就出战,难道是想堂堂正正的打上一仗?”李清很有几分讶异的问着。

“倚城而守本来就是守城正道,只靠着一面城墙,那是坐困愁城。不能逼他们退守城中,就不能顺顺当当的攻城。”

李清点头道:“是先得逼宋军退守城池,到那时霹雳砲差不多也就能造好了。”

眼下暂时只有云梯,霹雳砲得再过几日。不过幸好还有板甲,神臂弓也是成千上万,就算其中有一半浸过水,但晾晒干燥之后,并不比没有浸水的另一半差太多。

即便拥有同样的武器,宋人还占了一个居高临下的优势。但有了板甲和覆面头盔护身,不接近到五十步之内,不用担心被神臂弓的木羽短矢射到重伤乃至送命。

“得先称量一下他们的实力。就不知出来了多少人?”

李清正抱着疑问,从飞船上就丢下一个小竹筒。打开看过里面的纸条,武贵道:“大约是三千兵马。”

“看来是不用指望叶孛麻那边相助了。”

灵州之役后,梁太后和梁乙埋都不便再驱动外姓大族的兵马打头阵,李清的汉军只能成为牺牲品。

“应该都吃过饭了吧?”李清望着营地一角还没有散尽的炊烟,多问了一句。

武贵道:“四更天就开始造饭,到了五更初,就将饭都送了下去,现在都已经吃过了。”

李清放下心来,站在营门前,提气高声,将自己的命令吩咐下去:

“击鼓,聚将点兵!”

军鼓咚咚的响了起来,营帐中顿时一片喧嚣。

片刻之后,李清麾下的四千余人马,已经在他们身前汇聚。一名名将校在李清面前俯首帖耳,静待号令。

亲兵牵了李清的坐骑过来,他随即一跃上马,从腰间拔出长刀,前指营门,一声髙喝遍及全军:“儿郎们,随我出战!”

