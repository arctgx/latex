\section{第20章 廷对展玉华(中)}

赵顼对韩冈很是满意,但韩冈却是对坐在上面问话的皇帝,却有着隐藏得很深的反感。不是针对赵顼这个人,而是天子这个位置让韩冈从骨子里感到忌惮和反感。

韩冈现在并没有逆反之心——以现在的时势,还是给人打工是正经——不过高坐在御榻上的那人,一喜一怒都会决定自己的命运。喜欢控制局面的韩冈,对于自己的命运要受到别人操控,便有股发自心底却又不能宣之于口的痛恨。

这种感觉,在王安石和王韶面前,韩冈都从来没有感受过。但论起才学、才智,远远逊色于王安石和王韶的赵顼,却是在这方面远远超越了他的宰相和执政。

此乃地位使然。

也难怪有人说伴君如伴虎。也难怪有人见了天子后,不是战战惶惶,汗出如浆;就是战战兢兢,汗不敢出。身处这种让人无法把握己身安危的状态,韩冈虽不至于如钟会、钟毓见魏文帝一般不堪,但也的确很是让人不舒服。

以韩冈的城府之深,不免受到一点心境上的影响。而这种影响,落到赵顼眼中,就是韩冈表露在外的拘谨。

但这点拘谨其实恰到好处,也让赵顼从韩冈身上,感觉到了作为臣子应有的诚惶诚恐之心。若于崇政殿中,韩冈还能保持着在王安石、王韶面前一般儿的态度,对天子来说,未免就显得太不恭敬了一点。

误会了韩冈的态度,赵顼更加满意,“韩卿自任官以来,屡有殊勋。不说河湟,就是罗兀和咸阳,也是靠了韩卿不顾自身安危的结果。”

韩冈躬身:“臣身受陛下殊恩,敢不鞠躬尽瘁。”

赵顼点头微笑。韩冈尽管是王韶、张守约等人所荐,但更是赵顼特旨授予差遣的。没有赵顼下诏首肯,走正常的路线,韩冈根本不可能十八岁就入官得到差遣。赵顼也曾为自己的眼光而沾沾自喜过,不要说韩冈,就是王韶本人,将他从选人直接提拔到朝官,又让他去关西立功,还不是他赵顼的独断?!

韩冈如此说,当然正搔到赵顼的痒处。不过赵顼找韩冈进宫,自不会是拉家常,说些你好我好的场面话,更不是要听韩冈的奉承。说好听话的阿谀小人,他身边也有。吹拍捧起来比韩冈要出色的多,不需要在这方面并不算很合格的韩冈来占一个位置。

“听说韩卿上京赶考之前,曾经在熙河又有所发明,以产钳帮了高遵裕一次?”

比起朝堂上,赵顼现在关注的事情一点也不逊色于新法的推行。他已经有过两个儿子,但没有一个存活下来。没有儿子,家业将会落于他人之手。对于普通的人家,所谓的家业不过是百贯千贯万贯而已。但赵顼手上的家业,却是一个拥有亿万人口、幅员万里的大帝国。

事关家国天下,韩冈也能理解为什么赵顼把此事当作第一个问题来问。他点点头:“不敢隐瞒陛下,的确是有此事。产钳一物,乃是去岁高遵裕内眷遭逢产难,求到臣的头上后,臣让人打造出来的。”

“想不到韩卿还有此等才能。”赵顼微微一笑,身子却是前倾,神情更加专注,“难道韩卿当真见过药王不成?”

“药王孙真人,臣从无缘得见,世间谣传而已。”对于民间谣言,韩冈当然是否认到底,又很谦虚的道,“真正能在一个时辰中造出产钳,一个靠着蜀地来的银匠,另一个靠着三十年接生万人的老稳婆,臣仅仅提领而已。”

“提领难道还不够?银匠打造了不知多少器物,稳婆也接生了三十年,但他们此前都没有想到。只有韩卿你的提领,才最终有了产钳一物。”赵顼对韩冈发明产钳赞赏有加,不是没有来由,“宫内的宋才人已怀胎九月,大约再过半月的功夫,就要临盆了。到时候,还得靠韩卿的产钳来立功劳。”

听说了宫中有嫔妃待产,韩冈暗道一声原来如此。先提前恭喜了一下赵顼,然后他正色道:“产钳乃是为防一尸两命,母子双亡而不得已为之。一旦用上,以人力钳颅而出,日后恐有痴愚之危。此一事,还请陛下明察。”

韩冈必须要打预防针,否则出了事他肯定要倒霉。就像医生到病人家中,多会将病往重里说,然后出了事,才能脱身。这是未雨绸缪之法。

而赵顼听了后,怔了半天,最终叹了口气,知道产钳不能用了。作为皇室,不像外面的官宦富户,能承担得起子嗣痴愚的危险。如果今次宋才人生下了儿子,但这个儿子却是因为用了产钳而变得痴愚,日后大宋的基业可就危险了。谁也不能保证,会不会有心怀不轨之辈,抬出一个宋‘惠帝’来。

赵顼对产钳的心冷了一点,但对于韩冈的才能还是赞赏不已:“沙盘军棋,霹雳炮,烈酒,还有产钳,韩卿的才能不仅是在军政上,这发明创造也是一般出类拔萃。虽然韩卿你说没有见过孙思邈,但这发明之才,也只有天授才能说得通了。”

“陛下有所不知。”韩冈为自己辩解,“不论是军棋沙盘,还是霹雳炮,又或是烈酒、产钳。都是格物致知的道理,运用到实物上后所得到的结果。乃是儒门圣人之传,并无鬼神之力!”

虽然韩冈一手创立了疗养院制度,而药王弟子的传言,更让他在军中和民间也是搏出了诺大的名声来。可韩冈从来没有打算分管太医局的想法。卫生管理和医道差得很远,韩冈很明白这一点,他不能给赵顼留下一个错误的印象。而自己发明创造的本事,也决不能跟神神鬼鬼扯上关联,必须嫁接到儒门大道之上。

“格物致知?”赵顼皱起了眉头,他的记忆中,郑、孔二人给出的解释,可是不会让人造出产钳的。“可是张载有何别出心裁的见解?”

赵顼猜得也不算差,韩冈便将如今格物致知的新解向他详详细细的做了一番阐述,最后又道:“不仅仅是家师,如今在嵩阳书院讲学的程伯淳、程正叔,也是在格物致知上多有开创。”

“这一新解的确是别出心裁……”赵顼慢慢的点着头,在心中对比着汉儒唐儒和如今儒者的两份不同解释。

他已经准备要设立经义局,准备‘一道德’,也就是准备让王安石的学术自如今的儒学百家中脱颖而出,成为朝廷钦定的官学。不过要是变成了学着汉武帝‘废黜百家、独尊儒术’的做法也不一定是好事,就如格物致知的这一说,他从王安石和王雱那里都没有听说过,可效用却是显而易见。

别出心裁这个评价,韩冈不能担上。新不如古,就像王安石推行新法,都要从三代上为自己找寻借口。

“伏羲见河图而演八卦,夏禹收洛书而分九州,仓颉见鸟兽蹄爪之迹,遂以构造书契。至于民间,也有公输般见丝茅而造锯的传说。此诸事,皆是格物致知的化用。臣之诸多发明,也不过是上承先圣之学而已。”

东拉西扯,将不着边际的事拉到一次,这是文人的天赋。韩冈多多少少也有了一点,至少说起来还真想那么一回事。如果能以此说服天子,格物致知的这个新解推广起来就容易了许多。而发明创造,便能挂靠在圣人之学中,当有人来攻击韩冈务于杂学,也便有了还击的武器。

时间过去得很快,从午后入宫,君臣二人一问一答,韩冈已经在崇政殿中待了一个多时辰的时间,这在过去赵顼接见臣子时,是很罕见的情况。除了几个重臣外,也没多少大臣能在陛前多留上哪怕一刻钟。

随着交谈的深入,赵顼越发的对韩冈看重起来。

现在在殿上的韩冈言之有物,见事明确,将关西的军政之事剖析得淋漓尽致。就算把过去的功绩放在一边,这样的臣子也是值得重用的。

“韩卿的本官现在还是国子监博士吧。”得了韩冈的承认,赵顼自言自语,“有进士后当转太常博士,右正言就不好办了。”他又抬眼问,“韩卿可有馆阁?”

韩冈摇摇头:“尚无。”

“此乃朕之误也,以韩卿的官阶,就是直秘阁也能当了。”赵顼心中歉然,“就是初任,不能升得太高。这样吧,先与韩卿集贤校理一职,且过一年半载,再转直秘阁不迟。”

身为朝官,尤其是天子重视的朝官,不可能没有馆职或是贴职。虽然名义上这些是为文学高选之士所备,但实际上,到了一定位置上,只要不是戴罪之臣,得到馆职是顺理成章——没有一个馆职贴职,到外面也不好意思说你是能上朝的文臣。

有了进士在手,韩冈被授予一个贴职,也是在情理之中。只是馆职,他就不去奢望了,崇文院里的那些工作,不是他能来处理的。更别提在入馆阁前,都要进行考试。不比科举的经义,入阁考试可都是考得诗词歌赋。

