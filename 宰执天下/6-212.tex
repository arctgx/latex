\section{第25章 鸟鼠移穴营新巢(下)}

一个军汉大步走来。身上的穿着,便与李老实等其他军汉不同,光鲜得多。可衣服虽然好,长相就不好了。

五官倒是不丑,但一对招子太渗人。眼睛挺大,黑眼仁却出奇的小,犹如蛇一般,看人就带着一股子阴狠。这样的一对眼睛,也许只有洗热水澡的时候,才会由雾气带来一点暖意。若不小心对视上了,登时就是一身冷汗。

此人虽是漫步走来,身形也不高大,反而有些干瘪,但他一亮相,还有些乱的场面登时就清净了,军汉们闪到了一边,纷纷行礼,口称指使。乞丐们走避不及,也不敢躲,犹如被蛇盯上的青蛙,一幅束手待毙的模样。

这指挥使在人群前站定,被绑好脚的乞丐们全都给赶到了他的面前。

看着乱哄哄的人群,这指挥使只一皱眉,下面的士兵立刻拳打脚踢,帮乞丐把队列排好。

等人都排整齐了,他方才缓缓开口,在他喉咙上有一道如同蜈蚣一般的伤痕,鲜红的,随着喉结活动,仿佛在张牙舞爪。

“现在你们应该要知道要去云南了。”话音徐缓而沙哑,好似砂纸磨着刀刃,“是韩相公要抓你们,也是韩相公要安排你们。边境上缺人手,要人戍边屯田。好人家的百姓,都有生活,没事谁也不会想去云南。但你们这些贱骨头,一个个只知伸手,不知干活,没事还作奸犯科,不抓你们抓谁?!”

乞丐们早就被骂惯了,指挥使的几句‘贱骨头’对他们来说不疼不痒,若不是被抓取云南屯田,乔二苟只会打个哈欠。就是现在,也没有伤到自尊心的感觉,他心中除了逃跑的念头,剩下的只是愤恨,恨高高在上的宰相,恨前日抓了他们的军汉,恨眼前要把他押去云南的士兵。

“我知道你们这群懒骨头没一个肯认真干活,等到了云南扶上犁头,没半刻就会想着逃跑,但我要说……”指挥使下巴微扬,“别做梦了!云南四周都是蕃人的地,距最近的成都府都有三千里,一路上山高水深,关隘十几处,官军边打走,用了小一年。你们想逃,先得看看蕃人是吃荤吃素,再问问那些关隘中的弟兄们答不答应!”

乔二苟脸色苍白,看起来到云南再逃是不可能了,要逃只能在路上。

“我知道你们中间,仍有人想着趁还没到云南先逃出去,但我告诉你们……这还是做梦!你们当这绳子是做什么用的?!”

指挥使缓缓走了过来,就像一条毒蛇卷起抓到的食物,乔二苟一直都自诩是曾经打下两条街的好汉,但被这人的双眼一盯,连发抖都不敢了,身子都是僵硬的。

抬脚踢了一下连接在乔二苟和叶小三脚上的绳索,指挥使环目一扫,“一人犯错,全队连坐。一人逃跑,全队皆杀,这就是本官的规矩。”

听到这一句,乔二苟顿时就没了想法,就是想逃,一条绳子上的其他人都会拖后腿,他老老实实听着那指挥使继续说。

“这一路上,行的是军法。犯了事,本官就要杀人。军法最大,州官县官都拦不住。本官在陕西、在云南杀得贼多了,杀得人也多了。就这一年买卖清淡些,刀子没发利市,谁犯在本官手上,别怪本官拿他祭刀!”

指挥使又慢慢的踱了两圈,乞丐们没一个敢大喘气。叶小三方才洗澡时受了冻,喉咙痒痒的,刚想咳嗽,旁边一只手猛地捂过来,咳嗽给压在嘴里,叶小三胸口一个起伏,苍白的脸一下就涨得通红。

指挥使瞥了乔二苟和叶小三两人一眼,“本来本官是不想多废话的,不过本官过去在韩相公麾下,学到了一件事,不能不教而诛,不把话说明白了就杀人不好。所以本官现把话说在前头,听到了最好,记住别做蠢事。没听清的,本官现在再重复一遍——一人犯错,全队连坐,一人逃跑,全队皆杀。”

“你!”马鞭点着乔二苟的鼻子,“姓名。”

乔二苟连忙弯下腰,任凭马鞭抵歪了鼻尖,“小人……”

马鞭倏的收回,立刻又猛抽过来。啪的一声,衣服碎片顿时横飞,乔二苟身子猛地一颤,却没敢叫出声。

他做乞丐的时候,被打的次数多了,疼归疼,但不能叫出来。盯着打他的人看,盯住了,没两下胆就寒了。该给钱给钱,该舍饭舍饭。太平时日,有哪个敢随意把人打死?遇上乔二苟这种滚刀肉,商家、民家,都只能自认倒霉。

不过对面森冷的双眼,让乔二苟明白,就算把人打死,那对眼睛绝不会有半点波动,现在是越老实越好。他低垂着头,不敢有任何怨愤的表现。只是他不明白,为什么会挨鞭子。

抽过乔二苟后,指挥使再一次举起马鞭,指着他的鼻子,“只问你姓名。”

“小……”

乔二苟刚开口,啪,又是重重一鞭。

血和着布片飞落,马鞭第三次指着乔二苟的鼻子,“姓名。”

乔二苟脸上的皮肉都抽搐着,身上一阵阵的抽痛,他现在终于明白了,连忙道:“乔二苟。”

鞭子没再挥来,“本官方才说了什么,重复一遍。”

“一人……一……一人犯错,全队连坐。一人逃跑……那个……那个……”‘那个’了两次,见到指使又提起马鞭,他慌忙大声叫道,“全队都杀了!”

尽管用词有些错误,但意思是没错的。

指挥使点了点头,放开了乔二苟,马鞭指向了另一人,“姓名。”

等到每一队都抽了人出来问过,指挥使双手持着马鞭杆的两头,一下一下的弯着,“本官的规矩看来你们都已经明白,若是犯了规矩被本官杀了,就不能算不教而诛了。”

他视线在排好队的一众乞丐身上掠过,“现在,都给本官上车,本官数到十之后,还有哪队有人没上车的,全队十鞭!”

话声刚落,便是一片混乱,乞丐们纷纷赶着上车。只是被脚下的绳索牵累,一个人摔倒,其他人跟着就摔下来。

军汉们一个个过来,又是一阵拳打脚踢方整理好秩序,按着顺序将乞丐们押上了马车,而最后一队便被拉下来一人抽了十鞭。

劈啪作响的鞭打声和惨叫声,车厢中听得分明,车上的十几位同伴一脸逃过灾劫的庆幸,乔二苟却是不寒而栗。他刚才虽然急着上车,可耳朵一直竖着,但他根本就没听见那指挥使在数数。

乔二苟心中悚然,这一位明面上是心狠手辣却讲规矩的人,不过实际上,他很可能根本就不讲规矩,只抓着杀鸡儆猴一条。

“二狗哥,疼不疼。”旁边的小兄弟小声问着。

“疼,好歹还有命在。”乔二苟惨笑道。

头顶上一阵声响。隔着车厢顶壁,能听到脚步和说话的声音。那上面本是装行李的地方,但有时候也可以坐人,现在应该是那些拿着神臂弓的军汉坐在上面,谁逃了,立刻就会被神臂弓招呼上。

乔二苟头靠在车厢壁板上,闭目养神。现在什么心思都不能有,一个不好就会被拉出被杀掉给人看。

没死在那个被追捕的雨夜,没死在监狱里,没死在公堂中,他现在可不想陪着那些丐头一起去下黄泉。

头顶上安静了下来,透过敞开的车门,能听见外面有人说话。

“指使下手是越来越重。”

“这伙鸟贼,不打不堪用,打死了也不冤枉,可了劲打就是了。他们做的那些腌臜事,去开封府听听就知道了,别都推到丐头身上,这一干鸟货,哪个身上清白。”

“老七说得没错,就是该打。神机营怎么样,照样打。我那兄弟在神机营里面,一日两操,夜里还要加餐,以他的脾气怎么那么听话,还不是打出来的!神机营的队列,你们也看过的,怎么样?金枪班都比不上!怎么来的?棍棒打出来的!”

“俺也听说了,走队列的时候,快一点,一棍,慢一点,一棍,歪上一点,还是一棍。”

“去年我那兄弟跟着李侯去了广西,就一千人,排了三排,前面是两万大理国的两万大军,就这么排着队迎上去过去,没过午就杀了个精光啊。”

“你那兄弟是第一次上战阵吧,都不怕?”

“哪可能不怕?人马过万,无边无岸。两万夷兵,放眼望过去,人山人海。其实也怕,但听我兄弟说,听到小鼓一敲,就不由自主的在走了。”

“你兄弟写信回来了?”

“请都里的文书代写了信,贴了邮票,就寄回来了,本厢的铺兵直接送到家门口。”

旁边几个人说话,方才那个李老实走了过来,手押着门,对里面轻声道,

“这一路上也别害怕,不要违逆指使就行。去了云南没那么容易死,朝廷还要你们屯田呢。到了云南后,你们就老老实实种地,日后地也是你们的,房子也是你们的,再攒些钱,从蕃人娶个浑家,这辈子还有什么求的?不比当乞丐强?!只要勤快一点,别再偷懒,能活得很好!”

车门轻轻关上了,外面的声音小了许多。乔二苟耳朵贴着壁板,对话声兀自传入耳中。

“想不到这一回,轮到俺们去云南了。”

“其实云南也有云南的好处,可知夷女多情,皮肉白净,只要给些好处,娶了来也方便……”

马车开始启动,车厢外的声音渐渐低得不可听闻,只能听到几声淫笑作为最后的回应。

要上路了,乔二苟心想。接下来应该是先到车站,坐有轨马车南下。

乔二苟只希望能好好的活下去。

透过细窄的门缝,他望着不断退后的街道,这辈子,也许不会再回来。
