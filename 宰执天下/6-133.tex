\section{第12章 锋芒早现意已彰(14)}

元佑元年腊月底的一天午后,大名府府衙兼河北经略安抚使、马步军都总管行辕中的大小官吏们,亲眼目睹了一府之尊、一路之帅、宣徽南院使吕惠卿吕相公,亲自将一位武官送出厅外。

那位武官,既不是吕惠卿的副手,也不是朝廷派来的军帅,不过是高阳关路的兵马都监。

一路兵马都监的地位不低,且又是一名统管数千兵马的正将,已经是能够上殿参加朝会的等级了。可放在曾入两府任官、现任的宣徽南院使的面前,那就什么都不是了。如此厚遇一武夫,而且还是蕃官,吕惠卿的做法未免显得太掉价。

此事不仅在府衙中传播,也在区区半日之内,传遍了整个大名府城中的各个衙门。

“何以至此……”河北转运使李常摇头。

河北转运使李常,是位性子老派的文臣,看不惯吕惠卿这种对武将太过放低姿态的行为。

“士宣,你怎么看?”他问着身边的下属。

“今日之事,运使何所问也?宣徽相公是司马昭之心,路人皆知!”

表字士宣的河北转运判官唐义问,乃是前参政唐介之子。变法之初,唐介曾与王安石在中书门下内为同僚,当时政事堂中五位宰执,有‘生老病死苦’的戏称。其中的‘生’自是生气勃勃的王安石,而‘死’,便是被气得发疽痈而亡的唐介。

因为唐介之事,唐义问对王安石、吕惠卿有着很重的心结,又是在李常的面前,倒是什么都敢说。

“士宣……”

李常苦笑起来。他不意外唐义问的愤然之言,但他这话说得太大声,外面都是耳朵,传出去就又是一场府漕之争了。

唐义问愤愤然:“今日是刘绍能,明日还有张绍能、李绍能。宣徽相公调进河北来的旧部,可不只一个。等到他们都来了一遍大名府,河北禁军就要过界河了!”

李常暗暗一叹。唐义问说的其实没有错。

这一年来,吕惠卿从关西调来的将官有好几个,甚至包括那位刘都监在内,还有两位已转入了汉官序列的蕃官。

本来世人都以为吕惠卿这是为了提拔旧部,给他们一个机会,免得在裁汰整编西军的过程中成为牺牲品。

就像韩冈,保了多少旧部继续留在关西军中带兵不说,又往河东、安西等处安排人,最近西南方向要动刀兵,他还从西军中点了一批有功将校去帮忙,好让他们挣些军功以保全自身。

可如今耶律乙辛篡位,宋辽两国局势再度紧张起来,就能看到吕惠卿心思的深远。可以说是到了河北之后,就开始为未来的战争做准备了。

有了这些将校任官边境之上,边衅随时都能挑起——只需吕惠卿的一句话。以吕惠卿对耶律乙辛篡逆一事的表态,战争或许已经不可避免。

就是粮草军资冇也不算什么问题,至少河北钱粮,皆在吕惠卿耳目所及的范围内,有多少他都一清二楚,河北漕司想要推托都很困难。

河北路转运判官不只唐义问一人,吕惠卿的弟弟吕温卿是另外一位转运判官。

吕温卿是在半年前被调来任上,尽管兄弟二人,一任亲民,一在监司,其实应该避忌。但吕惠卿地位太高,而转运判官则不值一提,也没哪个御史多说废话。可现在看来,似乎又是吕惠卿布局的一部分。

不过今天他听到传言之后,便告了假,急匆匆的回去要问个究竟。

……………………

“兄长待刘绍能礼遇过重,区区一都监,可能受得起?!”

对于吕惠卿的做法,吕温卿有几分不解,又有几分愤懑,对一个武将如此礼待,传到京冇城,成为笑柄还是小事,被人借题发挥,说吕惠卿有异志,可就麻烦大了。

但吕温卿的愤愤之言宛如丢进枯井的石头,连一点涟漪都没有泛起。

书房的院子前,吕惠卿正拿着把剪刀,弯着腰专心致志的修剪着一株盆松。

叶偃枝盘的矮松生长在青灰色的浅盆中,树下青苔碧绿,树上针叶苍翠。枝干虬曲,干上鳞生,如蟠龙潜卧。如此一株盆松,古意盎然,仿佛百年之物,却又因满盆苍翠而显得生气勃勃,实是匠心独运,大家手笔。

吕惠卿手持剪刀,就绕着这盆栽打转,过了半日,才用剪刀截下了小指尖大小的一截枝桠来。尽管少了只是一点点,但盆中的虬龙却更加生动了几分,仿佛有了灵气。

半天的时间,仅仅是动了一下剪刀,吕惠卿抬起头时,额头上已经蒙了一层薄汗。

吕温卿给憋得不行,见吕惠卿终于停了手,忙忙又要说话,不提防一柄剪刀突然伸到眼前。

吕温卿被吓得一个倒仰,这时却听到吕惠卿慢悠悠的声音,“三哥,你看这剪刀。”

“剪刀?”

吕温卿不明所以,却还是依言低头仔细去看。

正常的剪刀手柄与刀刃一样长,而吕惠卿手中的这柄剪刀,刀刃只有手柄的一半,且是圆头。

“柄长刃短,一臂是另一臂的两倍,用此剪,当可省上一半力道。”

气学如今已可算是显学,大凡士大夫,多多少少都对光学和力学上的知识懂上一点。力臂力矩之类的理论,只要上过街看过商贩称米称货,多少都能有些印象。

他看了看吕惠卿,小心猜测着:“……是事半功倍的意思?”

吕惠卿嘴角向下拉了一点,抬了抬手,道:“看刃口。”

剪刀刃口处隐见锋光,闪亮如银,与黝黑的剪刀刀身形成鲜明的对比。

吕温卿看得仔细了一点,抬头用不太肯定的语气问道:“这是夹钢?”

“是夹钢!”吕惠卿点头,又将剪刀架在盆松上。

吕温卿不明所以,却见吕惠卿手指一动,剪刀刀刃在盆松上上下一合,小指粗细的枝干应声而落。

吕温卿惊叫了一声,一盆能入画入诗的杰作,就这么一剪刀给毁掉了。

“看到没有。”吕惠卿手指轻轻抹过刃口,“连剪刀都用上夹钢来造了。再怕辽人,又是为何?”

的确,如今夹钢和折锻的技术即便是对州县中的普通铁匠来说,也并不是秘密,夹钢甚至百炼钢的刀剑,只要不是名家出品,最多也就十几贯。而锻钢同样不比过往那么金贵。斩马刀、腰刀几乎都是夹钢的,而官造剪刀,

“可刘绍能……”吕温卿欲言又止。

“此事愚兄自有一番计较。”

吕惠卿放下了剪刀,绕过书房,慢慢向后院走去,吕温卿连忙跟了上去。

大名府衙的后花园是文彦博任官大名府、做北京留守时翻建。那一次的翻建,并没有多修补建筑,反而拆了两栋破旧的楼阁,掘了池塘,以一道小桥将旧有的两片梅林连做了一处。

这样的改建没怎么花钱,改动也不算大,却让花园平添了一分大气。吕惠卿当初入住此处时,曾多次感叹文彦博为相多年,手底下的确有人才。

梅花此时未开,池塘则几乎连底都冻住了,只有几株松柏常青。

吕惠卿漫步在青石铺就的小路上,“你可知河北诸将之中,对先帝最忠心的就是刘绍能。”

“为何?”

“他是蕃人。”

“是。”吕温卿点头,冇这当然不是秘密,“保安军,横山蕃。”

“刘绍能世代居于横山之中,其父怀忠亦闻名西鄙。元昊叛时,曾以王爵诱之。怀忠斩使毁书,之后殁于王事。尽管如此,刘绍能依然免不了为人猜忌。”

“免不了的。”吕温卿点头道。同样的情况,发生过太多次。多少部族在宋夏两边来回反复,任何一个蕃官,在宋人眼中,首先是叛逆的预备军,然后才是可能中的友军。

吕惠卿走上小桥,凭栏而望,“而且西贼惯会用间,他这个蕃官没少受罪,数次面临牢狱之灾,还是先帝说了一句公道话“绍能战功最多,忠勇第一,此必夏人畏忌,为间害之计耳’。因为这句话,刘绍能对先帝忠心耿耿,几次上阵都不顾生死。”

“是这样啊……”吕温卿多多少少明白了一点吕惠卿的想法。

“如今愚兄厚遇此人,也是希望他能够感念这点恩德,有所回报。”

施恩望报,自不是什么美德,何论是以阵上拼命作为回报?吴起给士兵吸疮中脓水,让那士兵的母亲痛哭流涕;吕惠卿送人送上那么远,私心也是昭彰可见。

“士为知己者死。现如今,哪个士人能做到?”吕惠卿自嘲的笑了一下,王安石于他有知遇之恩,但他也不会为了王安石,而去赌上自己的未来,过去没有,以后也不会有,“倒是蕃人朴实,懂得知恩图报,在这一点上,刘绍能比一干‘正人君子’可要强得多了。”

“只是……”吕温卿道,“只是刘舜卿在雄州啊!”

“刘舜卿离得远了点。河东之将,再是有名,也管不到西军头上。同时三千兵马的正将,刘绍能没必要听他的话。”
