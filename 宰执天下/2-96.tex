\section{第22章 瞒天过海暗遣兵(一)}

新任的西路都巡检苗授,亦是四十上下,跟王韶、高遵裕差不多年纪,有着一副文质彬彬的好相貌。温文尔雅,言辞知礼,气质淳淳如饱学宿儒。连王韶这个正牌子的进士跟他比起来,都显得经多了风刀霜剑摧残,英俊或有过之,但文气却逊色不少。

如此气象,韩冈前日第一次见面,亦不由得暗赞了一句不愧是安定先生的弟子。也只有看到苗授惯常笼在袖子中的一对骨节粗大、青筋凸起的大手,还有从鬓角一直延伸到耳后的一条血红色的刀痕,才能发觉他终究还是武将的身份。

“授之一路辛苦了!”王韶站在内厅门口相迎。

“分内之事,未足为劳。”风尘仆仆的苗授谦虚着,行过礼,便被走下来的高遵裕拉起。缘边安抚司的两位安抚和西路都巡检谦让了一番,一起携手进了内厅中。

韩冈和王厚跟着进去,只是跨过门槛时回头一看,就见苗授的儿子苗履在院中犹豫着不敢跟上来。苗履与他的父亲有七八分相似,却没有继承下来多少儒雅之气,行动举止一看就是武夫模样。他尚无官身,不敢进入商议军机的内厅——放在是节帅帅府,那就是白虎节堂,谁人敢犯禁令。

韩冈看了,便招呼他进来,“慎之,何必站在门外,一起进来便是。”

“多谢机宜!”

有了韩冈的许可,苗履心中的犹豫一扫而空。两步便跨上台阶,跟着韩冈王厚入厅。他的年纪跟韩冈相当,但在古渭寨中,刨去了他的衙内身份,没官身的他就连赵隆、杨英都比不上。不过苗履性格沉毅,又会做人,倒跟王舜臣他们几个处得不坏,也跟王厚颇谈得来。就是面对声名远扬的韩冈时,还是有些拘谨。

内厅中,性急的高遵裕也不等人端茶上来,坐下来就问着苗授有关星罗结部的消息。

苗授说话声不徐不急,平稳如一的声调中尽透着文人儒士的闲雅。只是他说的话,却是豪气自生:“别羌不足虑,若能与末将精兵千人,当取其首献于二位安抚座前!”

王韶老成持重,“授之莫要小看别羌星罗结,宁可高看一眼,也不要轻视于他。”

“别羌不过是虚张声势、自壮其胆而已,非此不足以统领星罗结部。他拿着抵御官军侵袭的借口,已经杀了三个族中耆长,都是其兄康遵留下来的亲信……自乱家门,这是寻死之道。”苗授声音沉了一点:“真正需要担心的是西贼!光是兰州禹臧家的实力就不在木征之下,若是一个不巧,让别羌与禹臧家勾连上,进而交通西贼,诱得梁乙埋兵出青铜峡,越六盘山来支持他,河湟之事必生变数。”

高遵裕则长笑道:“现在西贼的心思都放在横山,等天气再转凉一点,鄜延那边就要点烽火了。”他停了一下,“也可能是环庆那边要先打个头阵。”

“兵出白豹,攻打大顺,阻断环庆、鄜延之间的交通,这样就可以安安心心的攻打绥德。声东击西,西贼来来回回也就这么几手。”苗授说着党项人可能实行的计划,言语间对自己的判断充满了自信。

白豹城是西夏人在横山南麓的重要据点,位于环庆路和鄜延路之间。自白豹向东南四十里,便是环庆、鄜延两路北线交通枢纽的大顺城。近三十年前,白豹城曾经被任福带兵夜袭过。此战斩首六百,自军战殁则只有一人,因此而来的白豹大捷,是三川口之败后,宋军盼望已久的大胜。军中士气大振,任福从此得以统领大军,可紧接着便是好水川惨败,任福战死,白豹城也得而复失。

而大顺城建立是在三川口之败的第二年,由范仲淹主持,在一个名为马铺寨的小军寨上扩建而成——只看马铺寨这名字,就知道是设立在交通要道上,拥有驿传铺递的寨子。

大顺城的建立,一开始并不是为了维持两路的北线交通,而是为了抵挡西贼铁骑南侵的步伐。一旦党项骑兵自白豹城南下,能同时得到鄜延、环庆两路支援的大顺城防线,可以将其堵在北方。即便西贼能设法绕过防线,有大顺城钉在后方,他们也不敢在南面横行无忌,只能劫掠一番便匆匆而退。

事实上,大顺城也圆满完成了这个任务,‘大顺既城,而白豹、金汤皆不敢犯,环庆自此寇益少。’四年前西夏前主嵬名谅祚领军南侵,便是惨败于大顺城下,传说他还在此战中中了一箭,很快便因伤而死,让梁氏兄妹得以掌控西夏朝政。

不过相对的,大顺城位于连接鄜延、环庆的北线要道之上,一旦大顺城被围,两路交通就只能依靠南方两百里的中线——子午山小道,还有更南面的长安道。无论哪一条路,都不足以让两路能顺利并及时的运送兵员。

故而当西夏人每次进攻鄜延或环庆的时候,都不会忘记派一支偏师攻打大顺城。每一次,白豹城都会成为一根木楔,牢牢插在大顺城的喉间,让环庆、鄜延的北线交通时刻受到威胁。

“既然西贼主力在鄜延,偏师会攻大顺城,如何要担心西贼出兵支援星罗结部?”王韶反问着,只是听他的语气,却没有多少否定的成分在,大概仅仅是想测试一下苗授的水平如何。

“横山为西贼命脉,朝廷亦是势在必得。如今朝中已有进筑罗兀之意,西贼对此不会坐视不理,今冬必有一番前所未有的大战。如此大战,绝不会仅仅牵制住环庆守军就够的。为了能让关西除鄜延外的三路都脱不开身,秦凤、泾原、环庆都少不了会有偏师来攻。

兰州的禹臧家虽然是吐蕃人,但一直都为西贼谨守西南门户。如果禹臧家受命南下,他们跟星罗结部当会是一拍即合。若不能先发制人,露骨山附近的蕃部都会投向西贼不说,甚至连临洮也会落入党项人手中。”

苗授一力主战,当下他出言说服在座众人的时候,声如洪钟、眼神咄咄逼人,完全没有掩饰自己的好战之心。揭开胡瑗弟子儒雅的外衣,藏在里面的,是对蕃人不共戴天的刻骨痛恨,还有对战争和战功的无比渴望。

大宋自三十年前起,连续遭遇了三川口、好水川、定川寨三次惨败后,关西军中精锐尽丧。直到如今,军中六十上下、战功卓著的老将寥寥无几。凭着一些残兵败将,二十年来只能勉强守着横山、六盘的防线。

但如今西军中的新生代都已成长起来。镇守缘边各路各州的中坚,基本上皆是苗授这样三四十岁的将领。无论是大名鼎鼎的三种二姚,还是刘昌祚、刘舜卿、曲端,都是近二十年来成长起来的少壮派。

依靠这些在官场上仍能算是年轻人的将领,自赵顼即位后,宋军一方猛然变得进取起来。进筑绥德、甘谷,拓土横山,开边河湟,甚至包括庆州李复圭几次失败的攻势,都证明了宋夏两方之间攻守易势的现实。而苗授这等少壮派的将领,也从中渐渐的感受到了最近从东京城中刮来的、与过去二十年截然不同的风向。

少年时一次接着一次的听着官军惨败的消息,不少人的父兄都战死在沙场之上。亲自上阵之后,又不断的被动防守,坐困愁城,看着西贼的铁鹞子在外耀武扬威。时至今日,新天子抱着观兵兴灵之心,让西军的年轻将帅终于可以一舒心中积郁,哪一个不是成日想着建功立业?!

燕达凭借绥德之胜升任了秦凤兵马副总管,而王韶和高遵裕因为连续两次大捷而受到的封赏,也同样让人眼红!为了博一个封妻荫子,当调令送到手中的时候,苗授便毫不犹豫地接了下来。从德顺军都监的位置上降了半级,当上了秦州西路都巡检。他心中念兹在兹的就是战功,而眼前一场大战正等着他,苗授哪有不将之紧紧抓住的道理?

苗授霍然起立,向王韶和高遵裕的躬身行礼,朗声道:“末将愿立下军令状,只要两位安抚能拈选千名精锐与我,若不能大胜而归,斩别羌之首而还,末将甘受军法处置,虽死不怨!”

看到了苗授放入燃烧着火焰的狂热眼神,王韶与高遵裕交换了一个眼色,各自轻轻点头。

王韶随即便道:“此事也不须瞒着授之。星罗结部不恭于国朝,我等皆有心一战。可惜钱粮人力欠奉,只能徒唤奈何。……不过如果依照授之的计划,以千人速战速决的话,这点钱粮还是能拼凑得出来。就不知授之对此战有多少把握?”

“用兵贵奇,只要是出其不意,必定能手到擒来!”这是苗授的回答。

王韶和高遵裕点了点头。可韩冈却摇了摇头,这样实在有些冒险。王韶惯是剑走偏逢,推荐韩冈时如此,团聚七部时如此,只要合乎他心意的人和事,便会毫不犹豫地去招揽、去施行。高遵裕则是被军功冲昏了头脑。但韩冈他不会把宝押在苗授身上,不是他觉得苗授能力不足,而是他只相信自己。

苗授说他对别羌能手到擒来,而韩冈对此的评估也有六成的机会。只不过,韩冈希望胜利的几率能更大一点。他咳嗽了一声,缓缓出言:“其实还是有足够出兵的钱粮的,即便是三千兵、一个月,也一样够用。”

数道视线一齐转到韩冈的身上,高遵裕惊讶的问道:“哪里来的?”

“渭源堡。”

