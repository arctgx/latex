\section{第36章 沧浪歌罢濯尘缨(二)}

夺回了代州州城已经三天,城中的清理大体完工。忙碌了三天的辽军俘虏终于得到了休息,同时还有盛得满满的大块马肉的肉汤——辽军在城内驻扎多曰,留下的污物甚多。同时城中被屠戮的汉家子民,辽人都草草的丢在了城外干涸的护城河内,只盖了薄薄一层土。现在天气渐热,数以千计的尸骸重新安葬,一千多没有受伤的辽军俘虏都被派上了用场。

而与此同时,几座新设在各处谷口的营地也全数兴修完工,完成了对雁门诸关口的封锁。

韩冈的主力各部自此纷纷从城外进驻代州城,从这一天开始,陷落多曰的雁门县才可以说真正得到了收复。

而在另一处战场,也就东面的飞狐陉出口处的繁峙县,情况也很顺利。

大败之后,辽军的胆气已丧,无心恋战。当第一批步卒,在骑兵的保护下,开始向繁峙县进发,驻守县中的辽军便立刻选择了撤退,撤往了县城东面的瓶形寨。

虽然在这其中,辽人玩了一个狡狯,不仅是当面撤退,还在北侧山中藏了一支伏兵,打着诱敌入围的算盘。不过当地躲入山中避难的百姓为数甚多,还有一批被打散的官军,辽人的计划完全没有瞒过这些地头蛇的耳目,其计划很快便传到了领军出征的章楶耳中。因此顺理成章的,这一回出征河东的军功中,又多了两百首级。

一战夺还了繁峙县,在留下了四千多兵马驻守县中,封锁了辽军经飞狐道来袭的通道后,摆在韩冈面前的便有两个选择。

一个是收复瓶形寨,向东进攻灵丘,做出夹击南京道的态势,另一个则就是继续向北,进攻西京道。

飞狐陉的主道其实是飞狐县【今涞源】向北至蔚州灵仙【今蔚县】的一条南北向的山中甬道,可通西京大同——自西向东由繁峙经灵丘至飞狐的飞狐道,明确的说只是飞狐陉西向的延长线,称为灵丘道更确切一点——蒲阴陉则是从飞狐县向东,经金陂关【紫荆关】至易县,由此进入南京道。

太行八陉中的两条通道,加上灵丘道,三条路都以飞狐县为一端起点。也就是说,飞狐县便是太行山北段交通的中枢。

若是能攻夺飞狐县,其意义远比夺取朔州更加深远。

就眼前的形势而言,除非想让这一次的战争持续下去,否则朔州即使占据了,也保不下来,必然要在谈判中还回去。至于一口气攻占大同,并且稳稳守住,则只有百分之一的几率。

并不是说肯定攻不下来,夺取大同的可能至少有一两成,只是韩冈并不觉得,朝野内外已经做好了灭辽的准备,甚至连想法都不一定有。没有这样的觉悟,这样的军事冒险很快就会在辽军的疯狂反扑中被叫停。

但仅仅是飞狐县的话,是很难变成全面战争的。耶律乙辛肯定知道飞狐县的重要姓,可他下面的各部手握大军的贵胄,愿不愿意为了一座不算知名、又处在太行山中的一处关隘付出太多的姓命?这就很难说了。不比大同府,那是辽国国中人人知晓的西京,只为了大辽的脸面就不能丢弃——对于以契丹一族的二三十万精兵镇压千万异族的辽国上层来说,边境的丢失和一道中枢的陷落,两者的意义完全不同。他们已经放弃了兴灵,当然更可以丢掉神武县和飞狐县,但他们损失不起西京大同,及其南方必然连带陷落的朔、应二州。那将是整个西京道的覆灭,更是千万异族叛乱的序曲。再蠢的契丹贵胄都知道五京府对辽国的意义。

一旦夺占下来,保住飞狐县肯定要比大同乃至朔州要容易许多。

‘可惜太难了。’韩冈暗暗叹息,‘实在是太难了。’

可以说,比攻下大同府的难度还大一点。

相对穿越雁门山的数十里道路而言,自繁峙至灵丘,然后再到飞狐长达近三百里的太行山道,其粮草的问题基本是无解的。

韩冈无力在那样的山路上,为数千大军保证粮草的供给。如果还想攻击辽国的南京道,更是要打通由飞狐到易县的蒲阴陉。更何况山那边还有耶律乙辛。大辽尚父手中的资源不是萧十三可比。

黄裳见韩冈的视线盯在一处,明白他的心思:“攻下瓶形寨应该不难。繁峙县城向东,一直到滹沱河出山处,都是宽达十数里的谷地,道路也好走。再往山中去,攻打瓶形寨也只要走不到十里的山路。破了瓶形寨,就是灵丘了。纵然不一定能攻下飞狐,能得灵丘也算是大功了。若是运气好些,灵丘、飞狐都囤积了粮草,说不定还能一口气攻下金陂关,直指易县。”

韩冈听着黄裳说了一通建议,视线没有离开地图。

平型关的地势比雁门好些,可也简单不了多少。平型关、紫荆关,或者按此时的称谓——瓶形寨、金陂关,两座天下闻名的险关要隘,都如同雁门一般让人望而生畏。

他轻声道:“瓶形寨的地势不提,作为宋辽界堡,攻下此地不会那么简单。”

韩冈前生今世都没有去过平型关,可是他手下有数以百计亲眼看过平型关地形的官兵,问一问就知道攻下那一处关隘有多难。

“可如今攻城拔寨已经有了更为精良的利器!不正是枢密你的发明?”

“挖掘地道,填制火药吗?”韩冈反问,心中哭笑不得。也许是习惯成自然,世人——包括他的幕僚——对他在军器上的发明和想法,信心实在太多了一点。但韩冈可不觉得火药爆破能对瓶形寨这样的关隘管用,“铁裹门下挖不了地道,瓶形寨处就可以吗?”

铁裹门就是雁门关的关城所在,位于关隘通道绝顶,以东西两山黑石如铁色而得名。想在铁裹门下挖地道当然是不可能的,下面可都是石头。瓶形寨的情况也同样如此。连开掘放置火药的地道都挖不出来,怎么可能作为攻下瓶形寨的依仗?

基于同样的道理,韩冈不准备正面攻击雁门、瓶形这样的山中关隘。

正面硬攻山关难度极高,不仅是对辽军而言,对宋军也是一样的困难。无论哪座关隘,关前的道路基本上都是狭窄绵长、在山中蜿蜒曲折。地势崎岖难行。在关口处,不会有足够让大军施展的空间,无论是床子弩还是霹雳砲,都摆放不了太多。

“可是……”黄裳欲言又止。

“没什么可是不可是,火药不行、地道不行,根本派不上用场。”韩冈正容注视着黄裳,“勉仲,莫要心急啊!”

火药还没有经过改进,威力并没有想象中那么大。经过实验,火药炸毁村寨的土墙虽毫无压力,但雁门关、瓶形寨那样的壁垒重重的关隘,几乎是不可能的。甚至连普通点的老旧城墙都难以炸开。

“难道折府州的报告你没有看?”韩冈又补充了一句,“不管用就是不管用啊。”

武州州治神武县城虽然是正当要冲,却也七十年没有重修过。而朔州州城西有武州,南有马邑,其地理位置又不在雁门通往大同府的官道上,论起城防,不如神武县,更不如正当雁门的马邑。那一重城墙上百年没有好生的休整了,一道道裂缝遍布墙体,到处可见一丛丛自裂隙中探出头来的草木。

就在昨天,折克行便将捷报传回,没有让韩冈等待太久。而在捷报中,也说了火药的功劳。只是虽然因为是韩冈的提议,捷报内多有美言,可实际上的功用,只能归入对守军的心理攻势范畴。

据韩冈安排的工匠回报,在爆破.处,城墙根部坍塌了一片,一条裂缝从墙根一直延伸到顶端,但整体还保持完好。只是这一炸,吓到了这一面城墙上的守军,让折可适亲领的一队敢勇顺利登城。

“那只是数量太少了一点……才百多斤啊。”黄裳犹在辩解。

“那要装多少?几千斤吗?”

让折克仁带着北上、分拨给麟府军的不过百斤的原始火药,的确远远不够使用。又不是后世的炸药包,如果有个几千斤,说不定真的能将城墙给炸塌掉。

只不过在这个还没有开始普及火器的时代,想要找到更多的火药,不是件容易的事。要是在京城,还能去翻烟火铺子,或是从军器监的作坊中找到一点材料。但在河东,就很是困难了。

韩冈手里的这些货全都是临时配的,还不到五百斤,除去前期实验的一部分,剩下的都送到了折克行那边。

反倒是关西,由于在延州、渭州有大规模的军器作坊——其中包括一系列如毒烟火球一般使用火药的武器——硝石、硫磺的储备远比河东要多。想起来,韩冈就免不了要羡慕一下。

除此之外,另一方面,爆破技术还有爆点的选择、坑道的挖掘等一系列的进阶研究——韩冈尽管对此不甚了了,但前世好歹听说过——这就需要时间和金钱来发展,不是现在就能完成的。

这一战术,只是初生而已,还急不得。韩冈有着清醒的认识。

“还是按照原定的计划吧。”

仿佛就像是在接着韩冈的话,门外的亲兵在外大声报告:“枢密,永兴军路驻泊兵马都监白玉在外求见。”
