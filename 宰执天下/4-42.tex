\section{第九章 鼙鼓声喧贯中国(二)}

横山南麓下的战鼓已经敲响。韩冈只从每天三四趟从新郑门进城,直奔皇城而去的信使,就知道鄜延路的战局已经是如火如荼。

“此战必胜。”

种谔在给天子的奏疏中,三番四次的重复着自己的信心。以收复罗兀城为最低目标,想要达成的确不是难事。

尽管唱反调的声音依然存在,在失去了韩琦之后,元老重臣们的声音并没有降低多少。不过他们的话语对天子的说服力已是越来越低——对朝堂的影响力,随着离开朝堂日久,而逐渐衰退,这是不可避免的结局。

对军情捷报的渴求,让天子吩咐下去,即便他安寝后,只要是鄜延路的急报,就立刻将他唤醒。

而以韩冈和王韶两人的共同判断,对这一战的估计,则是‘应该能赢’。虽然两人都是希望先拿下兰州,但并不代表他们会睁着眼说瞎话。

换做韩冈来为党项人考虑,也没办法找到他们获取胜利的钥匙。

这些年来,宋夏两国之间的国势、军力的差距越来越大;论起粮秣军械,宋军已远远胜过党项一方;军心士气也随着西夏的衰退而逐年高涨;加之参与河湟、荆南、西南几处开疆拓土的官兵所获的封赏,让西军上下都看红了眼,渴战之心无比旺盛。

党项想要胜出,就只能祈求运气。让种谔等领军将帅在战场上迭犯蠢行,使得大白上国的大军能通过战术上的成功,扭转战略上的颓势,最后取得胜利。

不能说没有这种可能,上阵作战,运气的确是很重要的环节。已经快要看到胜利,忽然之间因为一阵狂风而逆转,也是有可能的。

另外,党项人想要撑过此次大战,还有一个希望就是大宋国中有事。就像当年因为庆州军作乱,而功亏一篑的罗兀城攻防战。正好如今南方——当然不是广西——而是淮南、江东,今年又遇上旱蝗,以至秋来绝收。

韩冈依稀记得前些年有人跟他说过,大宋开国以来,水旱蝗灾一直不断,有国土广大的因素在,但也仿佛有着某种周期循环。每隔一段时间,便要闹上一次大的,连着几年,天下各地都有大灾。只是一时想不起来是听谁所说了。

这番话现在想来倒是真有几番道理,前年去年是北方加上两浙路大旱,赤地千里,飞蝗漫天,今年则是河北北部加上江东、淮南遇上旱蝗大灾。看样子,明年就要轮到荆湖、蜀中去了。

“玉昆可认识张玉?”王雱的声音在耳边响起,让韩冈一下回过神来。方才在楼下驭马狂奔而过的金牌急脚递,让他一时走了神。

“赫赫有名的张铁简怎么可能不认识?当年又是同守罗兀城,一起随军撤回绥德,中途还有个无定大捷,将追兵斩首上千级。这些年来,偶尔也是有书信往来的。”韩冈反问回去:“张铁简怎么了?难道觉得他上个月的大战在秦凤路指挥得好,准备将他调回京中任职?”

“玉昆说得正是!”王雱点着头,拿起酒杯比了一下,“张玉可能又要入京了。殿帅宋守约新近病殁,空出来的侍卫步军司副都指挥使一职,天子有意让其接任。”

“哦?那还真是可喜可贺!”张玉若能接手宋守约的位置,西军在军方声音又要大上一分,韩冈自是乐见,举杯与王雱对饮而尽。转又问道:“不过张玉兼着捧日天武四厢都指挥使,他空出来的位置给谁?还有秦凤路的副总管一职又给谁接手?”

“秦凤路估计是将燕达调回去,不用再加权发遣了。”

韩冈还在河湟的时候,燕达就被天子越次提拔为秦凤路副总管,只是因为资历不够,而加了权发遣的前缀。如今几年过去,在京中和环庆路绕了一圈后,就又升了一级,的确只要加个权字就够了,“这一辈的将领中,天子最是看重他,日后必是稳稳地一个太尉。”

“也是运数,强求不来。而且他出身京营,天子怎么都会高看他一眼。”王雱摇头感叹了一番,“至于捧日天武四厢都指挥使,依序应该是种谔,他的龙神卫四厢都指挥使也还没有移给他人。”

“西贼国势日蹙,但军备犹存,种谔要想得胜,也不是那么容易的事。不过此战要重夺罗兀倒是不难,只要罗兀城拿下来,种谔肯定是要升一级了,接张玉的班没人能说不是。只是他身上的龙神四厢,也是循序接任吗?”

“这就不清楚了,不过我看天子的想法,应该不会再循序而进,很可能会越次提拔。”王雱身为天子近臣,耳目比起韩冈要灵通得多,察言观色的条件也比韩冈优越。

“是谁?”韩冈给王雱和自己又倒了一杯酒,随口问道。

“我怎么可能猜得到?”王雱摇了摇头,“说真的,若只以军功论,高遵裕和苗授都是有可能的,在泾原的张守约也不是没有希望。至于河北、京营到是算了,没人有足够的功勋。如今天子拣选管军,已是以军功为上,不复旧日的寻资论辈。不论谁上来,对军中都是好事。”

“说这些也太多了。”韩冈哈哈的笑了笑,“不如喝酒。”

为数仅有十数的三衙管军,是大宋军方的最高将领,都是起居八座的太尉,殿前司、侍卫马军司、侍卫步军司的正副都指挥使和都虞候,再加上捧日、天武、龙卫、神卫上四军的两个四厢都指挥使,总共是十一个位置。枢密掌兵籍、虎符,三衙管诸军,各有分野,‘兵符出于密院,而不得统其众;兵众隶于三衙,而不得专其制’——郭逵当年做了殿前都虞候后,转为同签书枢密院事,就再也不能回去担任三衙管军了,所以王雱、韩冈也不提他的名字。

小使臣、大使臣,宫苑诸使,这是中低层的将校,最高的是正七品皇城使。再往上入了横班,就是军中高层,只有三十个名额;而过了横班,要坐上节度使、观察使,最低也要是正五品的正任刺史,才能有资格当上三衙管军。且三衙管军的十一个职位,还要被皇亲国戚和潜邸旧臣分去至少三分之一,真正能落到领军将帅手上的,最多也就七八个位置。争夺之激烈,可想而知。

只是对于韩冈和王雱来说,三衙管军的人选为谁,实在离得他们太远,只能算是谈资而已。从韩冈的角度,与他有着交情高遵裕、苗授和张守约都有希望入三衙,这也算是个好消息,与张玉入三衙一样值得庆贺,但也只是庆贺而已。

王雱与韩冈又喝了两杯,忽然响起了什么:“对了,还有件事忘了说。广西今日急报,邕州有变!”

“邕州有变?”韩冈看着王雱的神情,不见半分紧张,反倒带了几分戏谑,心知定然并非他曾经几次提到过的那一桩事。“是何事?”他问道。

“蛮贼聚众劫掠古万寨,就在九月十五的时候。”

“古万寨【今广西扶绥北】?”韩冈并不在枢密院中做事,没资格看到地图沙盘,古万寨在哪里都不清楚,只是看王雱的态度,好像这一次的事并不是很严重。

“就是邕州南面一点的城寨。因为当着路口,又在左水【左江】北侧,商旅往来,城寨周围户口甚众,所以一向算得上是大寨,富庶在当地也是有些名气,所以引得蛮贼寇城……这就是苏缄三番四次上书说的南方情势危殆,亟待庙堂垂顾。”王雱哈哈大笑一番,揶揄着韩冈。

韩冈皱着眉头,却是难以释怀。虽然他与苏缄只是数面之缘,但也能看得出他并非信口开河之辈,区区土蛮,怎么能让他的警报连传?

看得韩冈心里还有疑惑,王雱笑道:“玉昆,你可知道交趾李乾德今日亦有表至,表中请罪,道‘新有艰阻,不与通和博买,未敢发人上京贡奉’。这是在告桂州刘彝的御状呢!……李乾德尚在幼冲,其母听政。主少国疑,可能会北犯吗?”

西北两处,太后领军出战的事可不少见,只是韩冈也不觉得交趾人有这能耐,想想也只能放在一边,等到之后的消息来了再说。

“看来的确是我想得太多了,如果只是蛮贼,以邕州的兵力,当能顺利剿灭。”

“自是当然。”王雱呵呵又笑着,如今王安石秉政,朝堂上虽有杂音,也干扰不了正事,让他的心情变得很好,也能开开韩冈的玩笑,“玉昆你可没少帮邕州的忙,若不能顺利剿平,想那苏缄也没脸再见玉昆你的面。”

韩冈摇摇头,举杯让王雱:“还是只论杯中酒吧。”

酒足饭饱之后,让伴当去会钞,王雱、韩冈一前一后的走出酒楼。早已是交了二更,街市上华灯璀璨,行人如织。天穹星辰弥补,冬季大三角闪闪生辉。

韩冈跟王雱一起出门,两人的坐骑已经被拉了过来。王雱扶着马鞍,仰头瞧了一眼星空,就站着不动了。

“怎么了?”韩冈问道

王雱眉头皱得死紧,牙缝里透出声音:“是彗星!”

