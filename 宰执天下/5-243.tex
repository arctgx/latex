\section{第28章 官近青云与天通(十)}

独乐园中,司马光坐在桌边。

桌上放着已经整理好的行装。车马也在外面准备妥当,只要司马光说一声,立刻就能出发。

但昨晚才接下两份诏书的司马光现在却犹豫了。因为从昨夜到现在,自文彦博、富弼,甚至王拱辰、楚建中一干元老那里,他陆陆续续的收到了十几条有关东京城的最新消息。

尽管还不知道更进一步的情报,但皇后垂帘是确定的,王安石成为了平章军国重事也是确定的。这样的情况下,司马光自问再去京城已经没有太大意义。

上京再受王安石的羞辱吗?若是变成那般情况,司马光宁可去死。

虽说已经接了诏,但迟几天出发,路上推说生病,就此回洛阳也没有什么关碍。

只是,这可是十一年来第一次离开洛阳的机会!

司马光看着行装,一时间竟无法决断。

“大人,你看谁来了。”司马康忽然引着一人,进了正厅。

看清来人,司马光便心头一惊:“和叔,你怎么来了?”

那人上前一步,在司马光面前拜倒:“刑恕拜见君实先生。”

刑恕是司马光的门人,但也在吕公著门下行走,同时还是二程的弟子。这两年,因为吕公著一直都担任枢密使,所以刑恕一直留在京城中任职。

“和叔远来辛苦了。”司马光看着刑恕,就微微笑了起来。

英姿焕发的刑恕,是旧党新生代的中坚力量,司马光从不掩对他的欣赏。

虽说刑恕好结交,在旧党中友人遍地,在中间看风色的那一派中同样朋友多多,甚至在新党一脉那边也有能说得上话的朋友。但司马光和吕公著都不觉得有什么大不了的,在他们看来,刑恕这是为人阔达,有着不拘小节的脾性,只要能守住底限,也是无伤大雅。

当然,这也是因为他们甚为欣赏刑恕的缘故。换作是其他人,司马光和吕公著怕是立刻就翻脸了。

刑恕风尘仆仆,脸上有些花,这是刚在外面擦了脸却没有擦干净的缘故,

“京城中的事想必君实先生已经知道了。”刑恕坐下来就说话,“天子郊祀后,在宫宴上突然中风,实在是让人有措手不及之感。尤其是皇后垂帘一事,更是让人意想不到。枢密命学生告假来西京,就是来跟君实先生计议一下,还让学生去问一问富郑公和文潞公的意见。”

说完后,刑恕就端起司马康亲自送来的茶水,几口就喝了精光,喝完后,立刻又要了一杯。这番举动看着失礼,但司马光对此却没有半点不快。

司马光神色沉重,问着刑恕“晦叔是什么意思?”

“君实先生,别的不论。天子在垂危之时,依然记得要将太子太师之位赠与先生,可见对先生的看重。至于之后变成现在的这个局面,着实让人纳闷。枢密有言,当也宿直的只有王珪、薛向和韩冈。其他宰辅,都是后半夜才招进去了。”刑恕看看司马光,沉声道:“有些事在洛阳是看不清楚的。”

……………………

“刑恕在外求见?”

午后时分,正在小憩的富弼被儿子惊醒了。

“他刚从司马君实和文宽夫那边过来。”富绍庭顿了一下,补充道,“是吕晦叔命其告了假,从京中赶来的。”

“此人天生乱德,吕晦叔、司马十二都是长歪了眼。”富弼摇头,断然道,“我不见他,就说为父身体不适,不便见客。”

富绍庭愣了。‘巧言乱德’,这是孔子之语。刑恕会说话,这倒是真的,但说其天生乱德,未免过分了一点。

富绍庭知道,富弼并不是很喜欢刑恕,但也从来没有表现得如此尖酸刻薄。自家父亲对他人的评价,除了奖誉之辞外,都尽量不会外传,以免日后祸患。尤其是这些年,随着年纪越来越大,脾气也是越发的好了起来。

说起来,这个变化好像就是在韩琦病死之后。不管怎么说,明争暗斗了一辈子,至少在寿数上,终究是赢了韩琦一着。

但今天父亲的反应实在很奇怪。富绍庭还想劝上两句,但看着富弼不耐烦的挥挥手,就不敢多问了。只能先出去找个借口将刑恕打发了。

听说富弼身体不适,不便见客,刑恕便站起了身,拱手道:“既然郑公不适,刑恕岂敢再打扰?”

富绍庭有点难堪,陪着礼,送着刑恕出门。

刑恕的坐骑已经被牵来了,富绍庭将刑恕一直送到马边。二程的学生这个身份倒也罢了,但吕公著的心腹人,司马光的门生这两个身份,纵然是富绍庭也不便轻忽视之。

在富府的大门前,就要上马的刑恕拉起富绍庭的手,微皱着眉,轻叹着气,声调沉沉,语重心长,“刑恕素知郑公最重纲常,旧年英宗有恙,一时触怒了慈圣,正是有郑公直言劝谏。”

富绍庭楞然,他不知刑恕为何提起这番旧事。

虽然刑恕说得简单,但富绍庭清楚当年的事,‘伊尹之事,臣能为之’,是富弼当着不敬仁宗、忤逆太后的英宗皇帝的面亲口说的。富弼那时是在正告英宗赵曙,如果不守孝道的行为再继续下去,他就要学伊尹,‘放太甲于桐宫’了。之后外界的传言甚至变成了‘伊霍之事,臣能为之。’,那就是说富弼还要学霍光,废立皇帝了。

富绍庭发着愣,刑恕依然是语气诚挚的说话:“如今太后尤在宫中,却是皇后垂帘,郑公或许是因此而积郁在胸。”他看看左右,凑近了一点,“但眼下两府皆寂然无声,御史台也不敢多言,依刑恕愚见,郑公还是早日为太子上贺表的好。”

富绍庭不由得点了点头,也许就是因为跳过了太后,变成了皇后垂帘,才让自家的父亲这么恼火。至于那些传言,几分真,几分假还真说不清楚。

刑恕虽然是二程的学生,司马光、吕公著的门人,但并不是茅坑里的石头那般又臭又硬的那种人,权变的地方比较多。或许是这个缘故,自家父亲才看刑恕不顺眼。但这能说刑恕错了吗?当然不能!都如此推心置腹了。

拱了拱手,富绍庭真心诚意的向刑恕道谢:“多谢和叔指点。”

刑恕连忙回礼,连声说着不敢。谦让了几句后,便告辞离去。

在门前目送了刑恕骑马出了巷口,富绍庭这才转回来向富弼禀报。

富弼还是半靠在榻上,听见动静,才睁开眼睛:“刑恕走了。”

富绍庭点点头:“儿子刚送了他回来。”

“他走的时候说了什么没有?”

富绍庭立刻摇头:“没有。”

“知情识趣啊。”富弼抿嘴笑了一笑。

富绍庭拿不准富弼的心思,小心的问道,“大人,接下来该怎么办?”

富弼眼皮又抬了抬:“你说呢?”

富绍庭低头考虑了一下,道:“是不是给天子寻些药方?洛阳城这边也颇有几个名医,当不输太医局的几位翰林医官多少。”

富弼点了点头,“虽然天子不一定用得上,但该尽的心意,的确该尽。”

得了父亲的肯定,富绍庭胆气稍壮,又道:“既然皇太子已经册立,大人亦当上贺表才是。”

“这是理所当然的。”富弼也没有犹豫。

“剩下的,儿子就想不到了。”富绍庭虽然还有想做的,但他还是觉得不说的好,所以他问道:“不知大人还有什么吩咐?”

富弼没其他的吩咐,只从榻上欠身坐起:“拿笔墨来,贺表为父要亲自写。”

富绍庭忙忙的亲自找来笔墨纸砚,帮富弼在榻上小几上准备好动笔的一切。服侍着已经难得提笔的富弼写字,富绍庭试探的叹道:“如今一来,就不知太后和雍王会有什么结果。”

“这是你该操心的吗?”富弼笔一停,声音转冷:“朝堂上的事,都别去操心!”

……………………

“雍王发了病,太后也病倒了,司马光多半就要进京,不知大人觉得怎么处置?”

同样的夜幕下,数百里之外,东京城南驿中,王旁问着王安石类似的问题。

“这是玉昆……”王安石想了想,摇了摇头,“这事跟玉昆都没关系了,为父更没兴趣。有皇后在,有两府在,自然能处理得好。”

“……大人可是平章军国重事!”

“与集禧观使有区别吗?”王安石拿着一卷书翻着,从露出的一角书皮上,能看到作者的名讳——韩冈。

王安石早就对官场没兴趣了。两度宰相,十年都堂,早耗尽了他的心血,连最看重的长子都赔了进去,王安石除了维护新学道统之外,就只剩下优游林下的兴致了。

这一回撞上天子重病,要不是看在赵顼旧日的情分上,他根本就不会来趟这汪浑水。尤其还是平章军国重事!难道以王安石四十年官场的眼力,还看不出赵顼耍的帝王心术?就是一时心软啊,才接下了这个职位。

但话说回来,这个差事也有个好处,胜在轻松,凡事可以不理,每隔五日上朝,闭着眼睛站着就行了。换作是三度宣麻,再任宰相,他还不一定会这么痛快的接下来。

