\section{第37章 蒿目黄尘顾世事(下)}

【过个年比上班还累,竟然还断更了,真是没脸见大家了。幸好明天开始就回去上班了,更新也会恢复正常。】

夏日的横山深处,有青山、有流水、有鸟兽。森森草木、潺潺山涧、悠悠鸟鸣,还有有别于山外的清凉的和风。

如果是内地,比如是京畿或是江南,如此秀美的山岭,决少不了文人墨客遗留下来的痕迹。或是题字题诗,或是建在风景佳处的亭台楼阁,或许还有着几处用来避暑的别墅。

但浸透了血液的宋夏边界,正常人都不会将这片随时都可能爆发战争的土地,当做避暑的场所。就是突然喧闹起来的今天,也不是为了于此避暑纳凉,而是数百健儿跨马持弓的游猎。不过在过去的几十年里,但凡大规模的游猎,其目标永远都是两条腿的直立行走的猎物,少有瞄准山林中鸟兽。

这一日的射猎参与者人数众多,有红袍锦衣的汉家军士、也有金环辫发的蕃人,但他们都只是围观者,张弓的则只有一位。

个矮体壮,满面虬髯,一对持弓的手腕如同铁铸,轻快的扯动着弓力过百斤的长弓,呼吸之间便是数支飞出,却是毫不费力的模样。

离弦的长箭在虚空中如同珠链,瞄着同一个目标飞向前方。令人瞠目结舌的箭术,其箭矢的落处,却只是一只不幸从洞中蹿出来的灰色野兔。

能在山岭间灵活奔行的猎物,于箭矢落下时并没有来得及做出任何反应,一箭便被带走了性命,但接下来的箭矢,依然穿透了灰色的皮毛。

弓弦声连绵不停,每一箭的落下都能将灰兔带飞出老远,但下一箭却总能精准的命中飞出去的目标。

这根本就不是狩猎,仅仅是弓手单纯在发泄多余的精力而已。这两天的游猎过程中,几乎每一只不幸撞到箭簇前的野兽,都会在密如雨丝的飞矢下被射成一滩碎肉。

远远近近围在他周围的人们,一个个紧闭着嘴,看着弓手表演着自己冠绝全军的连珠箭技。而张弓射箭的那人则是毫不在意,哼着流传在军中的粗俗的歌谣,一箭箭的射出去,完全没有让同伴们一起参与到射猎的活动中来的意思。

慕安明看看那只可怜的兔子,又看看比自己矮了有一个头的弓手,心中满是畏惧。

新任的环庆路都监,到任后只花了两个月,就将庆州北端的横山蕃部全数收服,甚至还干脆了当的灭掉了两个据说是对庆州的号令阴奉阳违,与山北的党项人暗通款曲,打算做个合格的墙头草的部族。

天知道这位王都监是怎么知道那两个部族心怀叵则,慕安明也不知道王都监是怎么看出来的。

说起来这几十年横山蕃部没少跟党项人一起杀进汉人的地界中。如今汉人势大,横山脚下的各家蕃部不得已投靠过去,然真心给他们做狗的还没几家——抢钱抢粮抢女人,只要跟着党项人跑个腿冒点风险就能尽情享用,过去的好日子跟着汉人可过不上——说到居心叵测,又是哪一家能例外?或许是抓到哪个是哪个。

但不管是究竟怎样从几十家部族中挑出的这两家,眼下结果很明显,现在定边城里的王都监只要出来转一圈,沿途的各家部族都得出来小心侍候。也幸好王都监虽说是脾性暴躁,但不是贪婪苛刻之辈,老老实实听话受教,偶尔在围猎的活动上捧个场,就不用担心自家的性命安危,也不用担心受到盘剥。

慕安明知道,眼前的这位王都监,是个后台极硬的角色——也不仅仅是他,环庆、鄜延的蕃部,大多都知道此事——他的际遇已经可以说是一个传奇。原本只是种老太尉亲兵的儿子,是跟着如今种家第三代的伴当。后来犯了事逃到了陇西去,却是撞了大运,不仅跟着开拓河湟的王相公搭上了关系,据说他还跟未来肯定能做宰相的小韩相公,甚至是以兄弟相称。

两年多前,他从熙河路衣锦还乡,连旧时的主人都得好言好语的拉拢。听说去年刚刚死了浑家,才过了几天,种家就巴巴的将女儿嫁给了他。眼下才三十岁,就已经是一路都监,日后肯定是坐镇关西的主帅之一,只要奉承好了,迟早都能摊上点好处。

慕安明看看左右,跟他一般心思的部族子弟为数不少,若能跟在后面捞个官身,有份让人垂涎的俸禄,谁还会想着在穷山僻壤中的领着几百上千的部众,日夜与羊粪为伍。

一筒长箭射空,前后射出了上百箭的掌中长弓听着拉开时的声音也快到了极限,王舜臣也松开了手,将长弓丢给了身边的亲兵。

这边箭矢一停,喝彩声就如同爆炸一般的响了起来。欢呼叫好的声音吓走了附近山林中所有的野兽,也难怪只有一两只倒霉的兔子或是雉鸡,才成了王舜臣弓下的牺牲品。

没有经过轮回转世,就已经成了刺猬的兔子,当然没人去关心,只有连着张弓射箭,头上冒汗的王都监才是众人奉承的对象。一群人涌上来,端茶的端茶,递水的递水,打扇的打扇,还有人递上了刚刚在清凉的溪水里泡过的手巾。

拿着手巾擦过满头的汗水,王舜臣抬头望望北面近在眼前的山峰,王舜臣如今在庆州,已经做了一年的环庆路驻泊都监,镇守在刚刚进筑完工的定边城,也算是习惯了现在的生活。

定边城已经处在横山南麓的深处,往北不远就是边界了。

自从两年前的横山一役结束后,宋夏两国的国境便已经正式定在了横山的山脊上。西夏在横山南麓的千里之地丢失殆尽。这其中两国并没有签署任何条约,只是在连番败绩之下,党项人不敢再越界一步。

如今南麓归宋,北麓……大宋依然想要,只是暂时还没去攻打——定边城的山对面就是银夏之地,是西夏仅次于兴灵的命脉,如果宋军想要夺占,那就要面对党项人的举国之战,东京城中的天子和朝堂,至今还没有下定决心。

但王舜臣已经等得不耐烦了。党项人眼下的窘境,只要眼睛不瞎,谁都能看得出来。更别说兴庆府内部,还有梁氏和秉常的母子之争,说不定什么时候就会大打出手。

坐在一片如盖伞般的树荫下,王舜臣将手上的酒杯一摆,一旁随侍的亲兵连忙给他满上。对面几个蕃部的大小酋领,都老老实实的在他面前站成了两排。

不是没人劝过王舜臣要对横山蕃部宽和些,但王舜臣却觉得这些人就跟狗一样,不踹两脚,就不知道该向谁摇尾巴。

喝了两口冰镇过的米酒,王舜臣正要说话,但一名小校匆匆而来,附在他耳边低声道:“都监,种家的十九衙内到了城中,说是有事要见都监。”

“十九哥到了?”在环庆路经略司担任机宜文字的种建中没有任何通知就突然来到定边城,王舜臣大笑起身,“肯定是好事!”

听闻种建中到了定边城,王舜臣就要立刻上马回城,但回头看到一众横山蕃部酋领,脚步便停了下来,“这些日子尔等做得都不错,本将也没什么要多说的,慕家做得尤其好,打探消息及时、准确,这份功劳本将已经报上去了,不日庆州便会有所回覆。”看几名慕家的首领脸上掩不住的得意,他又提声道,“望尔等日后也勤谨如一,也省得闹得不痛快。”再一挥手,“都散了吧。”

王舜臣没再找横山蕃部的麻烦,起身后就带着随行的亲兵上马返程。只留下一众蕃部酋领带着一脸的如释重负。

只用了一个多时辰,王舜臣便赶回了四十里外的定边城。种建中就在他的老窝里,安安稳稳的喝着解暑的凉汤。

“十九哥,怎么来之前也不说一声,也让小弟能上有所准备。”王舜臣大步进门,顺便用手巾擦着额上和脖子里的汗水。

“还记得五叔上次说得事吗?”

“当然!”王舜臣刚刚坐下,便跳了起来,“难道成了?!”

“只是请走了庆州知州。”种建中说着耸人听闻的消息,一点也不介意被人听到。

王舜臣双眼瞪圆,似是不敢相信,但转眼就是双手合十,“阿弥陀佛,总算走了。”

“是啊,”种建中点点头,很是松了口气的模样,“总算是走了。”

王舜臣如今已经是种家的女婿,种家的许多事也不避他。而王舜臣也是好战,虽然距离上一次攻略横山才过去了不到两年而已,但总觉得已经是闷得好久了。

武将若是没有军功,官阶便是七年一转。诸司使、副使四十阶——确切的说是四十二阶——说得极端点,不依靠军功,单纯的熬资历,想要从最低一级的供备库副使,爬到最高级的皇城使,需要两百八十七年,这还得着近三百年时间里不给人抓到一点错,否则一个错处,就能降个几级下来,当然是个笑话。

唯有军功,才有一次三阶、五阶、七阶往上跳的机会,才能让人一望横班之路,最后晋身三衙中的那十几个职位。

眼下官军越发的强盛,而西军更是精锐。在交趾,万人不到,就能扫平一国。而西军堪战之兵,少说也有二十万。有了这样的军队,谁还会能忍耐得下北面的那一块肥肉?

要开战,向北收复失土,这是种家、乃至西军上下共同的心愿。

