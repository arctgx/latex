\section{第29章 浮生迫岁期行旅(五)}

这是环庆路经略使发来的急报。 

急报中的内容没什么弯弯绕绕的地方。环庆路在横山北麓的主要据点韦州,最近州中守军跟辽人起了冲突。然后后面还加了个报功的申请——五个辽军斥候! 

自进入秋天以来,辽军开始变得活跃,不断有小队沿着灵州川南下。半个多月前,一队辽军斥候烧了韦州北侧一个军巡铺,并杀了两名。 

然后韦州守军出动,报复了回去。也就是打了个埋伏,伏击了一个小队的辽军斥候,射杀了三人,生擒了两个,其中一人伤重死在疗养院里,另外还跑了两个。 

一个看到这份札子,向皇后顿时就是坐立不安起来。 

如果当初辽人插手伐夏之役,几百上千的契丹骑兵,也是一样杀了。可现在已经定下了疆界,边境上闹出了人命官司,这该怎么办? 

枢密院给出的意见,是依旧例赏功,并移牒辽人,让其遵守疆界之约。 

只是这么强硬好吗?会不会惹怒辽人,最后变成两国间的大战? 

同时枢密院在札子上的贴黄中,也稍稍介绍了一下韦州的地理位置。让向皇后好歹能按图索骥,在地图上的横山北麓找到了韦州。 

从横山北麓下来,沿着灵州川北上。这一条路,不需要穿越瀚海,原本就是当初高遵裕之所以能杀到灵州城的原因。 

当宋辽划界后,新辟疆土上设立银夏路和甘凉路。旧有的缘边诸路,就成为了后方。旧有的驻军人员和数量不断调整,连主要作战目标也从抵御党项贼寇,变成了镇压当地蕃部,秦凤、熙河、鄜延莫不如此——只有泾原路和环庆路例外。 

因为地理的关系,青铜峡那一片跟泾原路联系得更紧密,比起隶属银夏,更为适合归于泾原路管辖。而韦州,相对于东面的盐州,与横山南麓的环州之间的交通也更为顺畅。 

不过这两路与契丹人的接触点,也就只在鸣沙城和韦州。 

但地图对向皇后来说并不直观,看着莫名其妙的图示、线条,她心道难怪他的丈夫会经常沉迷于武英殿的偏殿中。只有去武英殿偏殿,对照沙盘才能准确了解地形地貌。可沙盘还在武英殿那边,夜里过去也不方便。 

向皇后叹了一声,在她而言,边境急报加上与契丹人的冲突,就已经足够了。 

若是南方有乱,让章敦去处置就可以安心了,韩冈也可以。向皇后还记得有个去平西南夷的官,叫熊本的,现在夔州路,他对南方军事也了解很深。辽人生事,可以选择的人选则就少了许多。如今的朝堂上,真正对辽人有胜绩的帅臣,向皇后只知道一个韩冈。 

看来还是明天在崇政殿上问个明白好了。 

…………………… 

“殿下不必忧心。小事而已。”崇政殿中,章敦朗声说道,“殿下可以搜检旧档,即便是澶渊之盟后,河北、河东边界上,与辽人的冲突和厮杀也从来没有少过。寻常之事,照常赏功就可以了。” 

一大早就被皇后质问,章敦还以为出了何事,再一听,原来是之前根本就没有放在心上的小事,打个哈哈就过去了。 

“殿下。”薛向也说道,“辽人侵疆,在河东、河北是年年有之,并不足以为奇。而守军还击时也多有斩获。” 

向皇后却皱着眉:“虽然话是这么说。可之前辽国幼主夭折,官家曾有意起兵,只是为人谏阻。人同此心,心同此理,眼下圣躬不安,辽国会放过这个机会吗?眼下辽人犯界,是不是就是先兆?” 

“防秋是年年做的,殿下不用担心边境诸州会不做防备。”薛向道,“对于辽人挑衅,当要予以回击。越是退缩,契丹人就越是猖狂。反倒是强硬回击,却能吓阻其野心。澶渊之盟所以能订立,也是其前军大将萧达凛被八牛弩一箭射死的缘故。” 

章敦也道:“方今圣躬不安,耶律乙辛或有侥幸之想。不过对辽人的防备,一年多来从来没有松懈过,缘边各路的帅臣无一不是老于兵事。纵有兵戈,也能转瞬即平,殿下也不需忧心。” 

“韩学士,依你之见呢。” 

韩冈先看了看蔡确,东府宰执有资格参与军议,宰相更有一锤定音的资格,但这位聪明人根本就不在自己不擅长的领域多话。如今两府乏人,蔡确保持沉默,在另一侧的王安石又避嫌不开口,枢密院倒像是要将整件事给定下了,区区两人说话,也难怪向皇后不放心。 

想了一下,他回道:“臣之见与两位枢密相同。胜者在敌,败者在己,不论辽人是试探,还是打算犯界毁约,只要缘边各路做好防备,便是立于不败之地,让其无功而返。”

韩冈的话,让向皇后安心下来,道:“那就先镇之以静。至于环庆路报上来的功赏,就按枢密院的条陈来办。”

王安石沉着脸在旁听着,进殿后他就没怎么开口过,但一样是心明眼亮。

皇后终究还是对军事没有经验,几句空话就安抚住了。若是换作天子,肯定会再去追问细节,然后在武英殿偏殿中对着沙盘来推演。不过这样也好,当初天子对着沙盘谈兵事,都是在添乱,换成是更缺经验的皇后,垂拱而治反而会有好结果。

但看韩冈、章惇、薛向三人的态度,战争倒像是不可避免了……玉昆,你觉得呢?”从殿中出来,章惇问着韩冈。

“等着开战吧,这么好的借口辽人不会放过。”韩冈道,“反正不是这一次,就是下一次,找借口总能找到的。就像皇后说的那样,耶律乙辛不会让这么好的机会白白流走。”

“的确如此。”章惇点点头,“不过先来的应该是那群党项余孽。”

“杀光就是了。刘仲武做不到?”韩冈看看章惇,“还是说子厚兄你不放心韦州?”

“鸣沙城倒不怕。韦州那边……一旦辽人在韦州张起声势,青铜峡的党项余孽多半会随之响应、”

韩冈笑道:“庆帅乃是赵公才,以他的才干,也不用多担心。”

章惇却没笑:“这件事本来是想要跟玉昆你好好商量一下的。”“路明昨天入京了,带来了刘仲武那边的消息。”

“刘子文怎么说?”

来自于边疆主帅的奏章,由于其中多有各方面的权衡,细节上往往会有很多问题。倒是私信,尤其是刘仲武这样的武夫给恩主的私信,绝不敢乱说话,从来都是实话实说。

“青铜峡中不给修建城寨,党项人心自不能定,不知道还能压着他们多少曰子。种地不会,放牧不能,朝廷又不能养着他们,坐吃山空的结果只会让他们铤而走险。刘仲武说,薪柴都快堆起来了,就差点火了。”

韩冈啧了啧嘴,刘仲武现在倒是会说话了,“还是等吕吉甫入京,就让他去头疼好了。”

章惇冷哼一声:“玉昆你说得倒轻松。”

“不在其位不谋其政,只用说话,不用做事,当然轻松。”韩冈笑道,章惇则是又哼了一声。韩冈硬是不肯入西府,倒是能说风凉话了。

对一名避道廊侧的官员点头回礼,走了几步后韩冈才又道:“不说笑了,说起来小弟更担心的是河东。当初因为南附党项斩首之事,善战的将校纷纷南调,则边境多有颟顸之辈。”

“愚兄知道了。”章惇点点头。

韩冈又道:“小弟不怕边境开战,耶律乙辛多半也不敢往大里打。鸣沙城有刘仲武,后方的应理城有赵隆,在后面有秦凤、熙河两路的支援。兴灵的辽人加上青铜峡的余孽一起上来,也照样能抵挡得住。只要能安稳度过这一次乱局,至少几年内,边境上将会安静不少。值得担心的是那一干元老。韩魏公倒是不在了,但张、文、富等人尚在……”

“那就看看洛阳的那几位会怎么做了,眼下可不是熙宁七年了,想来也不会那么蠢。”章惇冷笑道,“还是多担心一下,他们怎么说这次大拜除吧。要不要打个赌,他们绝对不会承认这次大拜除是天子的本心。蒙蔽圣聪的评语绝对不会少的,更恶毒的决不会少。编造谣言,那边可是一把好手。”

“这不是稳输吗?”韩冈摇头笑答:“小弟还不如赌马好了……天子被幽禁,皇后成了傀儡。一切都是王介甫在幕后艹纵?”富弼抬起眼,看着儿子,“外面是这么传的?”

富绍庭点头,“还有说是上四军造反了,共举王安石为主。”

富弼顿时便冷笑一声。

司马光刚上京,旧党便全面溃败,这样的变化,让所有人都措手不及。谣言四起也不足为奇。

但当司马光在文德殿上的话传来后,富弼就只能叹息司马十二的运气了。至于吕公著也跟着败退,在前两曰听说了韩冈的三份奏章的内容后,他就更是没有半点疑问,韩冈这个后生固然厉害,可乾纲独断的依然是天子!

只是谣言还是像地里的野草一般冒了出来。这些谣言看似无稽,其实更多的当是有心人在后推动,想在外造起声势,然后胁迫中枢。富弼甚至知道是谁在背后唆使,只为了坏了王安石的名声,富弼知道这是不实之言,但他可不觉得有必要去为王安石去解释。

“别多话,看着就是了。”他吩咐着儿子,“天子重病,酒宴什么的全都免掉吧。将家里管好,好生在家里多读几天,说不准可是要大乱了。”

富绍庭满是疑惑的应承了,没敢细问,转问道:“大人,明天司马君实回来,要不要去迎他?”

“你去一趟也好。为父就不去了……看他在殿上说得那些浑话!”

