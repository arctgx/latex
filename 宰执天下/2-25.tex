\section{第八章 太平调声传烽烟(四)}

【第一更,求红票,收藏】

王韶和高遵裕一起拜访沈起去了。虽然韩冈确信,沈起应该不会愿意自己被西贼吓得夹尾而逃的狼狈样儿,被他要调查的对象看见——这实在太丢人了。

而且韩冈更确信,王韶和高遵裕也同样能想到这一点。但他们是不得不去,既然沈起已经身在三阳寨中,那王韶和高遵裕就必须去拜访,礼节上的顾虑让人无法避免这种尴尬——反倒是韩冈,由于品级太低,反而落个轻松自在。

这事说起来还得怪沈起自己。当初韩冈做衙前的时候,甘谷城也一样传说着即将陷落的消息。而且真的离陷落只差一步,身为城中支柱的城主张守约带着主力甚至被引入西贼的伏击圈,并不像今次只是来抢粮和烧粮。

可当日在边境诸寨做着断头买卖的商人们,也只不过跑到了六十里外的伏羌城,就停下来等更进一步的结果。而这位陕西都转运使倒好,跑得够快,从甘谷城到伏羌城,逃了六十里还不够,又急急向东,才一天的时间,人都已经到了离甘谷城有百里之遥的三阳寨了。

不愧是做转运使的!

这一天一百里的速度实在让人惊叹不已。要知道沈起不是单人独骑,而是带着一队足够庞大的车马队伍,而且都是没有什么战斗力、只有服侍主人这一项能力的仆役。沈起能带着这样的一群累赘,在一天内走完一百里的路程,足见天子和政事堂的宰执们是慧眼识人。

讽刺和嘲笑在韩冈心中打着转,王韶和高遵裕都不在,他便是清闲得很,也不用去考虑在古渭真的碰上蕃贼来袭的情况——发生的可能性太低了。

王韶虽说是去古渭寨坐镇,以防蕃贼趁势作乱。但一切应对措施都有预备,只要木征、董裕不发疯一般的倾巢而来,就凭古渭寨现在的防御水准,加上刘昌祚留在城中的一千兵马,依然可以轻松应对。

而木征、董裕发疯的可能性,在韩冈看来,即便有,也不会大。木征、董裕兄弟俩有没有胆子承受攻下古渭寨后,随之而来的天子怒火故且不论,单是青唐部的俞龙珂,就不会任由他们在自己的地盘上恣意妄为。

古渭寨旁边就是青唐部,两处甚至是被合称为青渭。虽说俞龙珂现在抱着首鼠两端的暧昧态度,在宋、夏、董毡、木征四家之间玩着势力平衡的游戏,尽量想着哪边都不得罪。但宋、夏两家倒也罢了,他若是会容许木征、董毡踏足他的势力范围,他日后不要想在河湟诸部中再抬起头来。

既然会有俞龙珂和他的青唐部帮着防守古渭,王韶、韩冈又怎么会真的如表面上那么担心。也就是高遵裕这个初来乍到的新人,还不知渭河水深水浅的,才会对西贼一年数次、比女人来红还准的攻击一惊一乍,被王韶给诳到。

韩冈不知道高遵裕和王韶会在沈起那里扯多久,也许还会被沈起留下来吃饭。而自己却没事做,而且今天走得太急,也没能带几卷书出来。

左右无事,韩冈便抬脚往外走。走了两步,他却又转回来,找到孤身待在阴暗的营房中的王舜臣:“王兄弟,闲来无事,要不要出去走走。”

“三哥你去好了,我不想去。”王舜臣摇头拒绝。

因为种詠之事,王舜臣最近的心情很不好。除了前两天听说种诂和种诊联手扫荡边境的党项羌,他才叫了声好之外,其他时候都变成了个土胎木偶一样的雕像。不问他,他就不开口说话,性格跟过去的爽快比起来,完全变了样。

韩冈对此看得很不舒服。王舜臣现在往房间角落里一坐,他所在位置立刻就阴沉得像是培养蘑菇的暗房。连照进营房内的落日余晖,到了他的这一角后也显得黯淡了许多。

韩冈两步上前,抬腿就是一脚,把王舜臣从床上踹了下去,“闹个什么别扭,婆娘也没你这样长气吧?”

王舜臣猝不及防,砰的一声,从床铺上摔了下来。他爬起来,沉默的揉了揉痛处,却仍是阴沉沉的一张脸。他现在的心情,当头棒喝都没用,何况韩冈并不算沉重的一脚飞踢?

“说说吧……”韩冈在床边坐了下来,拍拍床沿,示意他坐下。韩冈看得出来,王舜臣对种家的感情很深,所以对种詠冤死一事才会难以释怀,“事情闷在心里并不好,有什么话都说出来。”

王舜臣对着韩冈鼓励的眼神,犹豫一番,最后点了点头,依言坐下说话:“……三哥你知道的,俺爹是紧跟在种老太尉身边的亲信,俺从小就在种家长大。就在几年前,我还跟十七哥,十五哥还有李家的八哥一起在四郎面前习练箭术。四郎是手把手的教过俺射箭,俺现在用的连珠箭也是他教的。每次射中靶心,四郎都会奖我们一个钱,可以去街上买几块糖。俺的箭术一开始在几个兄弟里面算是差的,就是因为想着四郎的奖励,才会变得这么好。谁想到,李复圭那个该被驴子日上千遍的贼鸟,竟然……”

说起过去的事,王舜臣眼眶又红了。他模样看着苍老,说话做事又是一副粗豪的作派,而平日行事心中都有个谱,心计其实也不差。内外皆是早熟,让人往往忘了他的年纪。可他今年的确才十八岁,比韩冈还小一岁。

原来如此,韩冈终于知道为什么王舜臣为什么对种詠冤死耿耿于怀。王舜臣的老子死的早,他这是隐隐的把种詠当作了自己的父亲看待。明白了王舜臣的想法,韩冈也知道该怎么劝了。他一指王舜臣的鼻尖:“你这像是要报仇雪恨的模样吗?!坐在房间里生闷气,就能把李复圭给气死?还是说你知道了李复圭的生辰八字,能躲在房中扎着草人就把他咒死?”

“但李复圭……”王舜臣欲言又止。

韩冈对此心领神会:“李复圭的身份贵重,已经是一路安抚使,连天子都不能把他说杀就杀。但他还有儿子孙子,你真想报仇,日后总有机会的。再说,种四郎的兄弟子侄都没说话,你发个什么狠?有事不先跟他们联络一下?上次见到种十七、种十九,他们还提到你来着,连封信都不给他们去?”

韩冈劝了几句,也不多说话了,拍了拍王舜臣的肩膀,起身走出房。出了门,回头看看,却见王舜臣也跟了出来。韩冈微微一笑,虽然说的都是些废话,但还是有些用的。他当先走在前面,想着逛一逛三阳寨。

不过此时的三阳寨,却没有半点可供游览的地方。几条街道上,都是脸色沉重的人流。站在三阳寨正中央的十字路口上,看着周围的人心惶惶,韩冈突然间有种旧日重临的感觉,

就在不久之前,他在伏羌城、安远寨,看着周围一片混乱,而他当时的心中,也是同样的惶惑不安。而现在,他已经不再是身份卑微的衙前,而是成为了官人。心中的底气已经不同,对未来前路,他的心里也更有把握。

这时前方的人群中突然混乱起来,一个瘦削干枯的汉子在人群中左冲右突,直奔着韩冈过来。

“抓贼啊!抓住前面的贼!”叫喊声跟在干瘦汉子的身后传来。

喊声入耳,王舜臣便伸手一栏,将快要跑过去的干瘦汉子抓住。汉子还想挣扎,王舜臣更不多话,随手就是一拳砸到了他的侧肋上。

王舜臣手重,干瘦汉子挨了一拳,差点闭了气过去。但老做贼的也有对策,他顺势翻倒,在地上打着滚,没口子的惨叫着:“打死人啦!军汉打死人啦!”

“做贼还有理了。”王舜臣捋起袖子,蒲扇般的大手一张,就把在地上打着滚的小偷给揪了起来。

失主这时气喘吁吁的追了上来,很年轻的一个后生,中等个头,相貌普通。他跑到韩冈他们前面,先谢了王舜臣,又一把抓住小偷:“把俺的钱还来。”

“谁偷你的钱了!”汉子回了一句,又按着肋骨惨叫起来,“打死人啦,打死人啦。”

韩冈在旁边不耐烦了:“王兄弟,废话那么多做什么?送他去见傅寨主。这里是军寨,行的是军法。军情紧急,竟然还有人敢在营中作乱?!直接砍了,悬门示众。”

“送衙门去受军法?”王舜臣都愣了一下,偷东西而已,没那么重吧,打一顿就够了。

“你以为我是在开玩笑吗?”韩冈脸上没有什么表情,只有一种不把人命放在心上的淡漠。被他瞥了一眼,那汉子浑身都抖了起来。

“也就是个小偷而已,何必要他的命。”王舜臣倒帮贼人说起好话。

“他把刀子拔出来时就不是扒手了。”韩冈反手一掌劈在干瘦汉子的右手上,砰的一声,一把匕首落到了地上。

“好胆!”王舜臣眼一瞪,怒喝一声,抬手一拳就在干瘦汉子脸上开了油盐铺,把他打了个发昏十三章,一个钱袋也从他的袖子里落到了地上。

韩冈在地上把钱袋捡起,也懒得查验,直接交给年轻人:“小心收好,别再给偷了。”

年轻人连忙收好,躬身向韩冈道谢,“小人冯从义,多些官人大恩。”

‘冯从义?!’韩冈听到这个名字就是一怔,上上下下打量了他一番。冯从义与其说是年轻,还不如说是年幼,看起来比韩冈还小个一两岁,“可是二马冯、从心所欲、义之所在的冯从义?”

冯从义被问得心惊胆战,小声的回答:“小人正是。”

韩冈眉眼一凛,正要追问。

“玉昆!你怎么在这里?”王韶的声音这时从后面传来。

急回头一看,就见着王韶和高遵裕走了过来。没空在追究冯从义的身份,韩冈赶忙迎上前去。

‘这世上哪有这般巧的事,应该只是同名而已。’他想着。

