\section{第14章 落落词话映浮光(下)}

“这是大肠。”

岳一山抬头看了一眼,“猪的。”

“这是心脏,你们看,两心房,两心室。动脉血和静脉血便是以此为枢,自肺而体,自体而肺。”

岳一山又看了一眼,哼道:“牛的。”

“嘘,小声一点。”邻桌的同学紧张得向上看了看,低声道,“给天杀星听到就完了。”

“听到又怎么样。”岳一山咕哝了一句,“还天杀星,连人的都没有。”

“当然只能是猪牛羊的,人的五脏六腑能随便拿出来吗?”

岳一山撇撇嘴:“天杀星生劏过几百人,分我们几个又如何?”

“岳一山!”

来自上面的吼声,让岳一山条件反射一般的跳了起来:“有!”

“肠胃是属于哪个系统?”

他飞快地回答:“消化系统。”

“肾脏呢?”

“泌尿系统。”

“肺。”

“循环系统。”

“人的脊椎有多少块?”

…………

讲台上连番质问,岳一山十分流利的都回答了出来,然后在怒视的目光下,平平安安的重新坐下。

岳一山进入代州医学已经有半年了,解剖学的课程也同样上了半年,不过人体解剖,没几次机会,全都是猪牛羊的尸首,解剖之后,便成了三餐下肚。

岳一山本是医家子,考入医学,就是为了成为名医。可这医学之中,伙食不差,就是自家变成了屠夫。整日剖猪杀羊,到现在为止,感觉自己除了一些新奇的词汇和胖了十斤之外,都没有别的收获了。这让他越来越渴盼真正的人体解剖,也越来越对现在的课程失去了兴趣。

重新坐下,岳一山拿起课本,不过他的课本下面,藏了一本书,在右边露出了最边上的一行文字——‘李逵拿起斧头,在石头上磨了几磨’

“小心一点,不要给天杀星看见了。”

“嗯。”

岳一山现在对讲台上的天杀星没兴趣,他只对私下看的这部话本里的天杀星感兴趣。

岳一山的同桌摇了摇头,教授的课可是数量很少的,他没有岳一山的成绩和胆量,更舍不得浪费这么珍贵的学习机会。

正在课堂上的教授,也就是岳一山这些学生嘴里的天杀星,是河东路上最好的外科医师,就是放到东京城中也是顶尖。据说河东道上,多少将校都受过他的恩惠,遇上赤佬的时候,一说教授的名字,少说也能使其让上三分。学中论医术,比祭酒雷简要强了不少。

从东京成开始,这几年,全国各路的要郡,都设立了医学院。

医学院分为医学和医院两个部分,在医学中教书育人,在医院中治病救人,这是一干医学教授、讲师的工作。

只要是教授,都是翰林医官,但天下医学院数十所,只有四京和江宁、成都、京兆和代州、邕州,九处的医学院拥有教授。

东南西北四京,江宁、成都、京兆,这七个地方的就不说了,是天下最大也是最富庶的七处州府。

而邕州能跻身其间,是因为地处岭南,同时也因为韩冈曾经任职于此。

至于代州,却是因为这里是数年前宋辽之战前后,野战医院的位置所在,多少辽人的尸骸都在这里被解剖,由此培养了为数近百的外科名医,其中有三分之一,成了翰林医官。即使到了现在,代州医学也是天下外科最好的一所医学,手术水平甚至还在开封医学之上。

太医局中外科的成员,有九成以上来自于代州。在代州医学中学习外科医术,就像是在东京医学院中学习内科和小儿科一样机会难得。

一个家世普通的医学生,没有资格浪费时间。他拿起笔,专注的记录着讲台上的授课。

‘天杀星。’

岳一山瞅了瞅在讲台上一手教鞭指着挂图,一手拿着牛心的中年人,又看了看书里,这一位天杀星可比书中的天杀星要差得多了。

话本里面说得那位神医李逵,解剖尸体数百,早年被世人误会,甚至有了外号天杀星,但他继续坚持,最后成为天下最顶尖的名医,医术堪比华佗、扁鹊,能拿斧头给人开膛破肚,从肠子上切下穿了孔的阑尾,再从内到外的缝合上,让人安然脱离危险。

李逵,书中这位天杀星的手段,岳一山不敢指望,只求能跟现在在讲台上的那位天杀星一样,什么时候能弄到一个官身,成为一名翰林医官——即使是没有品级的最底层的医官也无所谓。

一阵噪音让岳一山从幻想中惊醒。

看到周围的同学一个个兴奋的交头接耳,他纳闷的问着同桌,“怎么了?”

同桌也陷入了兴奋,“天杀星要去神武军巡诊,准备带两个学生去!”

“去巡诊?”

岳一山心中一动,放下手中的书。

“怎么,有兴趣?”同桌还有点紧张的看着他。

“比去城里听说书要强!”

虽是这么说,岳一山心中还是有些遗憾,要是跟着教授去神武军巡诊,就赶不上在春明酒楼说《九域》,‘浔阳江头,李逵大发神威,两把斧头,连开十六床手术’的那一段了。

……………………

啪!

一声惊堂木,让茶肆中变得寂静无声。

“滚滚长江东逝水,浪花淘尽英雄。是非成败转头空。青山依旧在,几度夕阳红。白发渔樵江渚上,惯看秋月春风。一壶浊酒喜相逢。古今多少事,都付笑谈中。”

这首词算是全篇第一,甚至放在那位大家的文集中都不嫌过分,可惜是出自话本,不过因为那名不肯列名的作者缘故,还是传唱天下。

王祥每次听到这首开场词,也不禁有些伤感的感觉。

“想那千古英雄豪杰无数……”

坐在茶肆最后面一点的位置上,王祥没精打采的听着。这里的说书人口沫横飞,但他比起京城说书人要差些,主要是掺水的能力不足,可见说书水平有差距。但他旁边的同伴却是聚精会神。

“女儿是水做的,男人是泥做的。能说出这等话,想那王英,必是浊世佳公子,不可不见。”

听到这里,茶肆中的客人不约而同的脸上浮起了微笑。他们也不知听了多少次同样的内容,但那种又期待又想笑的表情,每一次都会出现在同样的回目上。

这是整部话本中,不多的几次让人捧腹大笑的段子。

‘什么佳公子,不过矮脚虎罢了。’

王祥摇摇头,扯着同伴的衣袖:“走不走?”

同伴拍掉了王祥的手,仰头盯着说书人的两张嘴皮:“急什么?瑞麟你听过,我还没听过啊!再等等。”

王祥无奈,但他又不能丢下同伴先走,只能耐下性子等着他。

又过了一段时间——幸好比起京中专说九域的张三四要短不少——终于等来了意料之中的哄堂大笑。

然后就是啪的一声响,请听下回分解。

茶肆重新喧闹起来,王祥也迫不及待的站起身,“好了,可以走了?”

他的同伴也不耽搁了,也站起来,会了钞,出门还跟王祥讨论着剧情:“说起来‘女儿是水做的,男人是泥做的’这句话,乍听起来,的确非爱花惜花之人不能说。”

王祥没好气的说着:“其实下面还有一句。”

他的同伴很惊讶,因为《九域游记》他也看过,就这么两句,“没有吧?”

“有!”王祥很肯定的点头。

“什么?”

“女人有了男人,就是水泥了!”王祥板着脸,忍着不笑。

他同伴脸上的表情,僵硬了几刹那的时间,然后就更加放肆的笑声在街边想起,惹得周围的行人人人注目。

待同伴因为喘不过气,终于停止了笑声,王祥叹道,“好了,该回书院去了。”

他很是有几分无奈,方才同伴大笑出声时,他看见附近有好几个同门的师兄弟在对这边指指点点。

‘早知就不说了!’,王祥后悔不迭,这下脸丢大了。

可是他又不能把同伴丢下,自家兄弟,怎么能丢下不管?

十五岁志于学。

承圣人之教,王祥十五岁便来到横渠镇,来到了横渠书院。不久之前,他的同伴也来到了书院中。

兄弟二人,相互照应,每日苦读不辍。闲来无事时,有时逛逛街,或是看看《九域游记》之类的杂书。

走在回书院的路上,王祥的同伴还在说着《九域》,“瑞麟,你说《九域》中,哪个人物最有趣?!”

王祥毫不犹豫的回答:“当然是入云龙公孙胜!”

“就是那个总说别人有血光之灾的江湖术士?!”

“就是他。”王祥点头。

同伴笑了起来,对王祥上下一打量,“瑞麟,吾看你印堂发暗,脸色发青,今日当有血光之灾啊!”

王祥提起拳头晃了晃,“谁的?”

两位少年又哈哈大笑起来。

《九域游记》中的公孙胜,总是爱劈头对人说有血光之灾,若是吓得人信了,那就伸手要钱。要是别人不信,那就劈面一拳,看,血光之灾!还唬住了史进,不过给鲁达一顿好打,两拳下去,打得脸上油盐酱醋的铺子都开了个遍。但这公孙胜却是宁输人不输阵的,当着鲁达醋钵大的拳头和五尺长的火枪,他从怀里掏出了一块镜子,举着叫道,‘今晨梳洗时便知,今日会有血光之灾,果然是映在了这里!’

两人笑了一阵,突然听到背后有人叫,“二郎!”

两人循声望过去,却见王祥的伴当匆匆而来,递上了一封信。

王祥看了信封上的寄信人,然后立刻打开了信封。匆匆一览,脸色就稍稍有了些许变化。

“怎么么了?”同伴问道。

“恭喜了,岳父要升宰相了。”王祥脸上有着意味深长的笑容,他低头又看看信上的时间,然后对同伴道,“现在多半已经是了!”

