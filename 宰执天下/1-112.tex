\section{第42章 诡谋暗计何曾伤(五)}

【第三更,红票,收藏。由于个人原因,从明天开始,更新时间改为零点,十二点和十九点。三更依然不变,只是更新的时间提前。】

程禹一愣:“为什么?!”

“嗨……”刘易一叹,为程禹的迟钝,“谋杀自首,可减二等论处的条贯,《律疏》【即《唐律疏议》或称《永徽律疏》】上可没有!”

“啊!”程禹顿时恍然。

韩冈有才学!现在他们不得不承认,这一块西北来的昆冈璞玉,也许诗赋不成,但经义已烂熟于胸,王韶、吴衍和张守约推荐得没错。王安石的青眼也没错,皇帝的特旨更没错!

既然韩冈才学如此,就不能再抱着侥幸。不论是千头万绪的家产分割,还是证言多矛盾的田产纷争,都不一定能难得住他。宋承唐律,此时通用的《刑统》根本是成于《律疏》的抄袭,两人现在都不能保证韩冈没有看过《刑统》和《律疏》。如果拿出来的案子能用唐律上的条文解决,说不定会正中其下怀。

但阿云案不同,有伤者,有凶手,凶手还认了罪,看似很简单,但却有着一个陷阱在里面。

两人对视一眼,同时用力的点了点头,还没入官的韩冈,必然会踏进陷阱。

韩冈翘首以待,等刘易和程禹再次回来,他立刻露出如阳光般的和煦笑容。前面的几道关那么容易就过去了,最后一题的难度必然不会高。刘、程这两位韩冈还不知道名讳的流内铨令丞,算是他在官场上遇到的最为善意的几个人之一。对他们,韩冈心中好感大生。

韩冈脸上灿烂的微笑刺伤了刘易和程禹脆弱的心灵,在两位令丞的眼里,这位年轻的秦州选人笑容中充满了恶意的讽刺。刘易心中更恨,将好不容易翻出来的卷宗递到韩冈面前。

韩冈拿过卷宗一翻,笑意更盛,感激之情也更多了几分。正与他猜测的一样,最后的判案更为简单,不是繁琐的家产析分,也不是产业争夺,更不是什么无头公案,而是一桩杀人未遂案,罪犯在公堂上自承其罪,要求对此写出判词,写明罪名、判决结果,并所引用的法律条贯。

什么样的考试肯定能得满分?————事先知道标准答案的考试肯定能得满分。

韩冈简直要笑出声来了,这就像是高考考试时,发现所有的考题自己正好都做过,而且连每一题的标准答案也了如指掌。真不知是自己的运气,还是流内铨的铨试就是这么轻易。

这桩案子韩冈看过。登州阿云案,即便是以他对律法的陌生,同时一直以来对通行的《刑统》只是泛泛读过,并未精研,却也照样了如指掌。因为这桩案子,直接引发了变法派与反变法派的一次大规模交锋,从而震动了官场。

就在熙宁元年到二年,一桩闹翻了整个朝堂的杀人未遂案,确立了‘谋杀已伤,按问欲举,自首,从谋杀减二等论’这一条律法。如果是普通的士大夫,他们不会关心刑律。但无论前身今身,皆接触过此案的韩冈,又哪会不知?

这一案的案情其实也很简单:登州女子阿云居母丧期间,因叔父贪图聘礼将其许配于农夫韦高,而韦高本人相貌丑陋、年岁又大,阿云不喜,这位彪悍的山东婆娘遂趁夜持刀将韦高连砍十几刀。不过妇人力弱,只是将其砍伤。而当阿云作为嫌疑人被传到官府时,不待审讯,她便自吐其实。

谋杀未遂很好判,依律当绞,而阿云不待审讯和用刑便自承其罪,在此时算是自首,依天子早前的敇书当减两等。登州知州许遵判得便是流放。

只是这判决上到审刑院和大理寺复核时却被推翻,因为他们认为韦高是阿云丈夫,妇人谋杀夫婿,是犯人伦,属十恶不赦之罪,依律当斩立决。因韦高未死,可减一等,当绞。

而大理寺和审刑院的复审意见传到登州后,许遵则抗辩说,阿云是许嫁而未嫁,而且丧期定亲违反孝道,在宋律中是要杖责并断离的,因此她并非韦高之妻,当以‘凡人’论,也就是没有关系的普通人论处,许遵坚持原判。

大理寺这时又说,阿云在孝期结亲,是违律为婚,更当加罪一等,同时在《刑统》中,有‘于人有损伤,不在自首之例’这一条,不承认阿云算自首。

为了这件事,许遵和大理寺打起了笔墨官司,继而又惊动了整个朝堂。赵顼让刑部复审,而结果是支持大理寺和审刑院的判决——绞刑。而许遵仍然不服,坚持己见。

赵顼新登基不久,无法做出决断,遂同意让两制以上的高官一起参与讨论。王安石支持许遵,而司马光则支持大理寺、审刑院和刑部的决定。他们各自身后都有一批支持者,互相之间由辩论变成了争吵,简单的刑律断案,一直吵了一年多,到了新法开始推行,又渐渐变成了变法派和反变法派之间的政治【和谐】斗争。

而当刑事转为政治后,其结果便不是靠法律来判决了,王安石正得圣意,所以最后阿云被天子特赦,不是斩,不是绞,也不是流,更没有杖责,名义上是编管流放,实际上接下来的大赦就让她直接放归乡里。同时,‘谋杀已伤,按问欲举,自首,从谋杀减二等论’这一条出自赵顼敇书的律法,就压倒了《刑统》中的条文,成了通行世间的法律。

对于阿云案,韩冈的看法是与许遵差不多。阿云是在母丧期被其叔父聘于他人,所谓的未婚夫妇关系是非法的,不当承认这个关系。而阿云仅是斩伤韦高,其人未死,她本人认罪态度又好,减刑也是应当。

这桩案子在朝堂上闹了整整一年还多,发给地方的朝报也刊载了判决的结果。普通人看不到朝报,就连县一级的官员都看不到——朝报一般只下发到州中——但韩冈的老师张载却是渭州军事判官,他能看到,也让学生们讨论过这个案件,韩冈当然也参加了讨论。同学们的看法不尽相同,去问张载,张载则用笔写了个‘仁’字,没有直接回答。

等到重生的韩冈回想起这段记忆,闲暇时又跟王韶和王厚讨论过,两人所持的观点都与韩冈相同,法令即在,依律行事即可——另外,王舜臣当时正好在场,他的观点则正好相反,也直接粗暴了点——“这等毒妇,打死了事!”

宋代的法律,属于成文法,判案者虽说有一定的灵活权变的余地,但主要还是是依律条判案。既然法令清楚,当然好判。而且阿云案前后韩冈也是了如指掌。当他再次面对登州阿云的这桩杀人未遂案时,该怎判,甚至判词该怎么写,都不是难事——标准答案就在心中。如果考官敢判错,闹到天子面前,都是韩冈占理。

看着韩冈振笔疾书,一行行端正的三馆楷书出现在纸页上。看着韩冈的判词,刘易和程禹的笑容渐渐收起,而脸色则一点点的苍白了下去。

‘怎么可能!!?’

两人在心中一齐大吼,新近出来的条令,韩冈一介布衣怎么可能知道?他才十九岁啊,怎么可能向积年老吏一样对法令一概门清?!韩冈的三份荐书中说他杀人、说他救人、说他惊人,就是没一条提过他能判人!

‘该怎么办?’刘易和程禹面面相觑。韩冈过关斩将,走得顺利无比。这下……该怎么向上面交代。

“怎么回事?”

一道洪亮的声音突然间从门外传来。话声入耳,两人的脸色不再惨白,简直是泛绿。他们一点点的转回头,坚硬的颈骨就像久未使用的门轴一般干涩,“陈判铨?!”

一人随声踏进厅门。来人干瘦矮小,比韩冈整整矮了一个头去,而方才那道如洪钟一般的声音,却是出自于他口。瘦小的身体上,面圣所穿的朝服尚未换去。长脚幞头,黑犀腰带还有一身代表六七品的绿色官袍,宽宽松松的套了一身。在腰带一侧,还挂着一个银丝绣的小腰囊——银鱼袋。

韩冈躬身行礼,这名瘦削男子便是判流内铨事——陈襄。

陈襄进来后,谁也没理会。先走到桌边,低头看了看刘易出给韩冈的试题,又瞥了一眼脸色阵青阵白的两名令丞,摇头冷笑了一声,“难怪!”

刘易和程禹身子便是一颤,张了张嘴,却什么话也没说出来。两人都很清楚,他们的顶头上司,判流内铨事、秘阁校理陈襄,绝不是好糊弄的人物。在官场上沉浮日久,一些小手段根本骗不过他。要不然,也不会刻意等着他去崇政殿的时候,才把韩冈叫来。

刘程二人心中哀叹自家的运气太差,怎么陈襄去了廷对后,还会回来?

自来少见肯做事的官人,京中百司的判事们,极少听说他们在廷对之后,还会回本署理事的,多是放羊回家了事。做官本来就是这样,太辛苦就不是官,那叫吏!刘易和程禹平常有事,也是尽量推给下面的吏员的。

陈襄又拿起韩冈方才所作的墨义考卷,只一眼,便点了点头:“字不错!……就是少了点神韵。多买点金石拓本翻一翻,学着写,别做了抄书匠。”

韩冈点头受教。

陈襄一目十行,放下答卷,又赞了一句:“算是有才学的。”

陈襄见多了因为字写不出来而把笔管咬烂的荫补官,真的有才学有心气的人物,早就去考进士或是明经了。得人推荐、由布衣为官的人,其实数量很少,而真有才学的,数目更少。他在流内铨一年多,加上韩冈,也不过一掌之数——这还是包括了荫补官在内。

看完韩冈的前一张试卷,陈襄径自坐到了刘易的座位上,问道:“现在考到哪一步了?”

“……只剩断案了。”刘易迟疑了一阵,低声回答。

“判词写好了没有?”陈襄又问着韩冈。

韩冈上前,将卷宗和答卷一起呈上:“请判铨过目。”

陈襄先翻了一下卷宗,便抬眼扫了两名下属。又看了韩冈的答卷,当即一声嗤笑:“作茧自缚!”

四个字的评语,让刘易、程禹又涨红了脸。

而看到了这一幕,韩冈若还是不明白,那就太愧对自己的智商了。他明白了,也为方才自己的自作聪明而感到好笑,甚至还有一点后怕,幸好刘易和程禹小看了自己。

陈襄很爽快的拿起笔,在试卷上批了几个字。抬头对韩冈道:“恭喜了。”

韩冈心领神会,连忙行礼,“多谢判铨!”转过来,又向刘、程二人行礼,“多谢两位令丞。”

直起腰,瞬间放松的心情,一时间让韩冈忘记了礼仪,他长长的叹了口气。如愿以偿,却不见欣喜,心头唯有轻松自在:

“终于合格了!”

