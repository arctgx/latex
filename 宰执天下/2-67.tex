\section{第15章 前路多坎无须虑(三)}

【第一更,求红票,收藏】

感慨过后,韩冈重新静下心来读书。不过没过多久,他的读书声又中断了。李小六进书房来通禀,说是仇老郎中带着个徒弟来拜访。

‘终于来了。’韩冈笑了一笑,放下了手上的书本。

窦舜卿入京,窦解被下狱,将仇一闻徒弟弄进大狱的原告都不在了,韩冈半月前便抽了个空,将他从狱中弄了出来。不过那个倒霉的党项郎中在狱中颇吃了一点苦头,被拖出来时,就只剩下半条命,仇一闻忙将他领回家去调养。今天能上门来拜会,看起来应该已经大好了。

韩冈先让李小六出去把人请进客厅,又叫了云娘进来,帮着自己换上了一身见客用的衣服,才施施然的走了出去。

仇一闻正坐在韩家的客厅中喝茶,而坐在他下首处的三四十岁,容色憔悴,一脸病容的中年人,当然就是没能救下窦解的儿子,而被栽了个罪名的背时货。他虽是党项人,却唤作李德新。不过党项人多有汉姓,也并不足为奇。

见到韩冈出来,仇一闻连忙放下茶杯站起来,向韩冈拱手行礼,而李德新则抢上前,跪下磕头,为韩冈的救命之恩道谢。

韩冈站着生受了他们一礼,即便不论他的救命之恩,以他现在的身份,也当得起两人的叩拜、躬身。

两人起身后,寒暄了几句,稍叙寒温,韩冈便请了他们坐下。

等谦让了落座,韩冈不想再听了无新意的感激之词,便主动问着李德新,“只听着仇老说李兄出身党项,却不知李兄究竟是哪一部的?”

不知为何,听到韩冈相问,李德新没有立刻回答,反而吞吞吐吐起来。

难道有什么不能说的?韩冈的眼神一下变得锐利,若是出身自六盘山对面,那就不能轻轻放过了。

仇一闻看着韩冈的神色变了,连忙帮着徒弟解释:“小老儿这徒儿,其实是出身于金明寨。”

“金明寨?”韩冈皱起眉,他不记得秦凤路有哪座寨子叫这个名字,但却又感到莫名耳熟。

仇一闻叹了口气,向东面遥遥一指:“就是延州的那座金明寨。”

“啊!”韩冈恍然,一拍交椅扶手,笑道:“原来是铁面相公的族人。”

“不是族人。”仇一闻摇了摇头,“他是铁面相公的亲儿子。”

“哦?!”韩冈吃了一惊。想不到眼前这个党项郎中,就是导致三川口一役惨败的李士彬的亲生儿子。

金明寨的铁面相公李士彬,时至今日记得他的人已经不多,即便记得,也是骂声居多。但在三十年前,或者说在三川口之战开始前,却是在关西鼎鼎大名,受人敬仰。

李士彬是党项豪族的族长,世代居于横山南麓。他的主帐位于延水之畔的金明寨中,本身也担任着都监一职。而金明寨周围,又有十七处小寨堡,皆受其统管,控制着方圆百里的土地。号称部众十万,精锐数千。

李士彬靠着手上的军力,将起兵叛宋的李元昊硬是堵得不能接近延州一步。而且由于他治军极严,勇猛敢战,故而有了铁面相公的诨号。

为了拿下李士彬这块堵路石,李元昊竭尽所能。但不论是用财帛收买,还是设计离间,都是以失败而告终。

李士彬多年来从宋廷收到的赏赐,是李元昊这个劫匪开出的价码所不能比的,这个时代没哪家能跟大宋比钱多。而李士彬本人又对大宋忠心耿耿,自祖父辈起就世代镇守金明寨,深得朝廷和历任延州守臣信重,离间计也是个笑话。

最后,狡猾多诈的李元昊,便想出了一个骄兵之计。

他先派人散布谣言,大赞着李士彬的威名赫赫,又让自己手下的士卒一见到李士彬的旗号就丢下兵械转身逃跑,让李士彬心生骄意。

紧接着,李元昊又派了手下的得力之人,诡称敬畏李士彬的威名而投奔大宋。蕃部来投是常有的事,老于边事的李士彬也没有看出其中的问题,很轻易的就收容了这些归附者。

而李士彬本有铁面相公之名,平日里治军严格,动辄以军法处置,受过责罚的卒伍心怀不满者为数众多。李元昊靠着派进金明寨的奸细,花费重金收买了他们,以为内应。

一切布置做好,李元昊便举兵南侵,一战攻下金明寨的北面门户塞门寨,紧接着又南下攻打金明寨。不过到了金明寨下,李元昊没有不趁着白天攻城,仅仅是陈兵寨外。

李士彬本就因为中了骄兵之计,而分外看不起李元昊。见到他们不敢进攻,便更是得意,入夜后就丢下军务,直接回去睡觉。

接下来,就是很常见的内应作乱的故事,城门被打开,坚固的金明寨就此失陷。李士彬连坐骑的缰绳都被内应给割断了,欲逃不及,被李元昊生俘。韩冈听说他的结局是被李元昊割去双耳,带到了兴庆府去做展览,苟延残喘了十年方死。

韩冈感叹着:“若是当年没有内应作乱,金明寨得保不失,就不会有三川口之败了。说不定,一战挫了元昊的锐气,也没有后面的事了。”

李士彬的惨败和金明寨的陷落,使得延州暴露在西贼的铁蹄之下。延州告急,刘平忙日夜兼程的领军救援,这就正好落到了李元昊的陷阱中。党项人围点打援的战略大功告成,在离延州只有数里的三川口,刘平所部全军覆没。

三川口之败是宋军连续惨败的开端,也是西夏正式立国的标志。三川口之后,紧接着又是好水川、定川寨两次惨败,西军精锐为之一空,到如今,才稍稍恢复了元气。

韩冈的话中之意,隐隐有责怪李士彬的意思。李德新立刻为他老子争辩:“金明寨之失非是先父之过,是大范相公让先父把元昊的内应就地安置。若依着先父的意思,把他们安顿到,金明寨哪里会失陷?!”

对于范雍和李士彬的这桩公案,韩冈也听说过不少次,只要讨论起三川口之败,不可能不提到。当年李元昊遣人来做内应,李士彬的确是建议范雍将这些新归附的党项人安排延州的其他寨子,不要放在金明寨,而范雍却让李士彬将他们就地安置。

从明面上看,最后金明寨会陷落,范雍的责任至少占了七成。但实际上,他只是按着惯例去做而已。

李士彬作为归附大宋的党项守臣,就算心中再想将降人收为部众,也不能私下里处置,必须申请上命。而且因为李元昊的离间计,当时就有着不利于李士彬的传言。铁面相公为了自撇清,防着朝廷怀疑他扩充势力,也得对范雍说自己不想留人。

而范雍则是照着惯例,让李士彬就地安置。这番公文来往,一个要表示自己对朝廷的忠诚,一个要体现自己坚定不移的信任,其实都是官场上的虚应故事。就跟天子登基要三辞三让,重臣升任宰相要上表推辞,都是一样的表面文章。

若李士彬真的怀疑其中有诈,后来将之安排到一个偏僻的寨子里,也不是什么难事。可李士彬却是将他们中的大部分安排在金明寨主寨中,让这些奸细得以自由的收买内应。

不过其中的曲折,在李士彬的儿子面前就没必要说了,弄得大家不痛快,何况韩冈也不认识范雍。只见他点头道:“范忠献【范雍谥号】多谋少成,又不通兵事,最后害了李都监,也害了刘太尉。不过范忠献为人仁恕,曾经饶了犯法当斩的狄武襄一命,也算是勉强弥补了一下早前的过失。”

李德新脸色缓和下来,“官人说得是。”而后又紧张的向韩冈道起歉来,“小人方才口不择言,冒犯了官人,还望官人恕罪。”

韩冈呵呵笑道:“我只见到了李兄的一片诚孝,却没看到什么冒犯。”他笑了两声,又跟着问道,“不过我记得李都监的儿子在金明寨失陷的时候,被家人护送了出来。因为李都监最后在兴州殉国,各自都被赠了官。怎么李兄会跟仇老行起了医来?”

李德新听到李士彬殉国就垂下头去,仇一闻则又帮起他说话:“老头子这徒儿是铁面相公的庶子,被救出来时才五岁。等大一点,去京城找他的两个兄弟,却都不肯相认。最后没奈何,就跟着老头子来学些岐黄之术,到现在也有二十年了。若非如此,他也是个官人啊。”

韩冈看着仇一闻的神色不像是作伪,再看看李德新低下头去的沉重,也是真情实感,的确像是在为其父的死而感到难过,让韩冈的一点疑心散去了不少。

他说道:“仇老,再过一阵,我想在秦州城设立第三座疗养院。不过管事之人,朱中和雷简都没有空。若是换了个不知名的来,又不一定压住秦州城里的骄兵,除了仇老,我实在想不到更合适的人选。就不知仇老肯不肯屈就?”

仇一闻立刻道:“怎么叫屈就?官人有命,小老儿当然得听!正好小老儿年岁也大了,没法儿像过去那样在秦凤路上到处跑,也想歇一歇脚了。”

韩冈笑道:“也不是要仇老你亲历亲为,庶务可由李兄处置。等李兄一切上手,仇老你挂个名字也就可以了。不知李兄意下如何?”

李德新听了便站起身,弯腰恭声道:“官人于小人有救命之恩,敢不尽心尽力。”

“好好。”韩冈拍手笑道,“届时就要劳烦二位了。”

又说了一阵闲话,看看时候差不多了,韩冈命李小六送汤水上来。这是官场上送客的礼仪,就跟后世的端茶送客是一个道理。喝过两口严素心亲手做得的酸梅汤,仇一闻、李德新告辞离开。

韩冈把他们送到院中,盯着李德新的背影,残留在心底的最后一点疑心却始终挥之不去。但他始终想不出又哪里不对。不过最后,疑虑化为自嘲一笑,他都是什么身份了,何须为此等小事烦心,真闹出事来,两根手指捏死就是。

“路遥知马力,日久见人心,还是走着看吧。”

