\section{第27章 片言断积案(中)}

诸霖和他家三弟就守着清水沟边,他们的兄长诸立则是要跟着韩冈才能出来。

因为靠着裙带都有着一个官身,两人占得位置甚好,基本上就靠着何双垣的坟墓。只要韩冈真的过来审案,可以在最近的地方看到这位韩正言的好戏。

等待的过程中,兄弟两人时不时的还望着南面,他们知道这一案的原告和被告都有开棺就撤诉的想法,不知道韩冈会不会放弃掘坟开棺,带着原告和被告过来审案。

何双垣虽然死的早,但他积攒下来的身家很不错,要不然也不会有两顷一十五亩的祭田。坟头由于被大水冲毁过,后来不论何允文还是何阗就加以整修,现在周围四十尺的坟头,并不是一开始的模样。但三个儿子给他立的墓碑却是实实在在的有一人多高,乃是真正的青石所凿,还请人写了墓志铭,刻在墓碑后,就是没有孙子的姓名。

而就在何双垣墓的东侧,一片面积广大的土地方平如印。这片两百余亩的田地,在垄沟上有着一块块界碑,与周围的田地区分开来。不过更为明显的区别是土地的颜色,深黑色。前一次,十年来一直留在何允文名下,但由于何阗的干扰,这片地并没有开垦,只有烧荒还是可以的。十年下来,厚厚的一层草木灰混了雨水化入地里。

日头此时已经升得老高,以何双垣墓为中心,径圆半里的地面上,聚集了百姓成千上万。所谓‘连衽成帷,举袂成幕,挥汗成雨’也就是指得这个场面

县尉冉觉乃是文职出身,看见了这么多人,《战国策》中的成语一下就联想了起来。只觉得今天白马县的百姓可能都到齐了,比起三月三的大庙会人还要多。如果将他们捉将起来仔细分辨,县中所有逃避丁税的隐户大概都能给揪出来。

这么多人,若是出个意外,那就不得了的通天答案。冉觉提心吊胆,而韩冈也一样担心。昨天就让他带着县中的一半弓手出城,在何双垣墓周围划定地界,将白马县四里八乡的百姓们的位置事先给定下来。用白在地面上写了字,画了线,并用麻绳圈起。而今天则带了大半弓手来此,将来到此处围观的百姓,按着乡里保甲,安排到预定的地方,并维持着秩序。

也幸好白马县虽不是大县,但因为地位重要,他手下的弓手人数超过两百,勉强够用。而且更幸运的是,这两年保甲法在京畿一带的推行,让百姓开始有了纪律性,很容易就让他们按着乡中保甲站定。

“魏兄、方兄,你们看这样还行吗?”掏出汗巾抹了把汗,冉觉来到韩冈的两位幕僚身前,问着他们的看法。

站在两人身边的,一名三四十岁的中年人抢先一步:“冉县尉果然难得,近万乡民竟然安排得如此稳妥。”

县官不如现管,冉觉不敢接此人的腔,低头道:“文衙内夸赞了,在下只是听了韩知县的分派。”

与魏平真、方兴并肩而立的,居然是文彦博的六儿子文及甫。

文及甫受父命去京师,不成想刚度过白马津,就碰上了这一档子事。他对韩冈的才能算是认同,但好感却欠奉,王安石的女婿,当初还差点气倒自己的老子,没当成死敌就已经是他文文翰宽宏大量了。今日韩冈出来审案,总要看个热闹。文及甫故意暴露身份,站到众官员和韩冈幕僚的行列中,一个是想抢个好位置,另一个,则是审案过程中如果有什么不对劲的地方,他就可以当场指摘出来,给韩冈一个难堪!

清道的锣声终于传了过来,只见着从南面一队人马从人群中留下的道路,直直行了过来。在成千上万人瞩目下,韩冈一行来到何双垣墓前。

高高骑在马上的年轻知县,腰背挺直,昂首挺胸,气宇轩昂的姿态,给所有白马百姓留下了极为深刻的第一印象。

翻身下马,让衙役带着原告被告去墓前站定,而韩冈却带着游醇,过去先跟周围被请出来观审的乡绅士子打一圈招呼。等到了文及甫面前,稍作询问,听闻竟然是文彦博的儿子,也不禁小吃一惊。

文及甫拱手笑道:“及甫不请自来,正言不会觉得在下冒昧吧?”

韩冈回了一礼:“衙内得司空言传身教,韩冈素来敬服。能得衙内观案,韩冈正是求之不得。”

衙役和原告被告都在墓前站定了,而一干弓手,在人群中敲着锣鼓喊着肃静,也让这上万人安静了下来。

“正言,到底要怎么审?”审判就在眼前,游醇忍不住低声问道。

“虽千万人吾往矣。节夫,你认为世上有几人能做到?”韩冈温声反问,终于揭开了底牌。

游醇一扬脖子:“义之所在,当一往无前。”

“对,因此真的假不了,假的真不了。所以也有说法叫做‘千夫所指,不病而死’。”说完举步,向何双垣墓前走过去。

韩冈说出的话有些高深莫测,魏平真等三人看着周围人群,隐隐约约有些感觉。

而文及甫转念间却在想着:难道韩冈是要借着这里的上万百姓,来强压着何阗与何允文认同他的判决?这可当真是大胆,若是一个拾掇不下,可就是丢脸到了全县百姓面前了。

韩冈却不管身后人怎么想,也不理会并立在坟前的两名当事人,而是径自来到墓碑前。

捻起一炷香,点燃后奉在手中,对着墓碑朗声说道:“何双垣!你虽已身故五十年,可即投本案,便仍是本县治下子民。身后事一缠三十年,虽已居身土木之下,却仍不得安寝。汝之冤情,本县已知。天日昭昭,众目睽睽,今天就在青天白日之下,万众观睹之中,让本官还你的公道!”

一番话说完,周围众人都是脸色微变,而更远一点的百姓,也都是起了一阵喧哗。难道这位韩知县,当真能沟通鬼神不成?

韩冈全然不理会身后的骚动,直着腰,双手拢着香一拱手,算是行了一礼。让人将香火插在坟前。

转过身来,他一脸端正严肃,对着何允文和何阗道:“此案本官即要宣判,你二人也过来上炷香。等片刻之后,本县宣判,是子孙的,日后依时节奉着香烟血食,而没有瓜葛的,也就该一刀两断了。不管尔等是不是墓中之人子孙,打扰了三十年清净,也该来行个礼。何允文,你先来!”

周围再一次变得寂静了起来,成千上万对眼睛望着墓前的一举一动。

在上万人的注视下,何允文颤颤巍巍的上前,点过香,扑通一声跪在墓碑前:“爹、娘,孩儿不孝。爷啊,孙子无能,不能守着祖宗啊!孙儿不孝……孙儿无能……”哭到动情处,竟然膝行上前,一把搂着墓碑,一下下用头撞着,只两下,就已是头破血流。

眼见着何允文如此恸哭,人人为之恻然,韩冈却仍板着脸,命人将挣扎不已的何允文强行搀扶起来。

“何阗轮到你了。”

场中一下又静了,一起盯着此案的原告。

何阗也拿着香上前,尤留着血迹的墓碑前同样是扑通一声跪倒。但他的哭声却没有悲情,只是在嘶声竭力的干嚎着,头也撞着石碑,咚咚声响中却不见血。这样哭了一阵,人群中却是隐隐的一片低笑声响起。

“好了!何阗,你就不要再哭了!”

冷声将何阗从坟前叫了起来,韩冈环视白马县的一干乡绅和士子,沉声问着:“看到方才的何允文、何阗两人哭坟,这个案子,想必不需要本官来判了吧?”

还要怎么说?一个哭得要吐血;一个却是干嚎了半天,怎么都装不出个悲恸的样子来是,干巴巴的连眼泪都没怎么掉。这结果是明摆着的。

众目睽睽,天日昭昭。当着千万人的面,韩冈似又有沟通鬼神之能,又有几人会不心虚?就算想强妆出一幅孝子贤孙的样儿,也是镇静不下来,演不下去的。

不但乡绅们各自点头称是,就连原来支持何阗的士子,也都偃旗息鼓,根本都抬不起头来。何阗脸色灰败,而何允文却大喜过望,又是哭得老泪纵横。

不过只有文及甫眼神冷冰冰的。这毕竟并不是审案的正途,虽然是光天化日下明明白白的对比,可用哭来证明谁是谁非,却根本不合律条。文及甫自信,只要自己表示一下,得到支持的何阗还有反口的能力。

“韩正言,如此审案未免太儿戏了吧?!何阗不过是哭声不哀,就这样判他输了官司,试问这判词,审刑院能认帐吗?”

“想不到韩冈还没说,文衙内也知道谁输谁赢了。”韩冈冷笑一声,回头转身,面对着千万白马百姓,“韩冈敢问白马县的各位父老,这个世上可有哭父哭祖,却无泪无哀的孝子贤孙?”

十几名大嗓门的衙役将韩冈的话一起传了出去,立刻就得到了回答。七嘴八舌,前前后后的响了起来,“没有!没有!”

“有没有!?”韩冈再一次问着。

“没有!没有!”这次回答变得整齐了一点。

“有没有!?”

同样的问题用着更高的声音第三次重复,返回来的声浪也随时高涨,震天憾地:“没有!没有!”

等到声浪稍歇,韩冈又高声问道:“韩冈再问各位父老,这世上有没有父祖坟前不伤不悲的道理?”

“没有!没有!”

“有没有!?”

“没有!没有!”

“如有人自称坟冢之人子孙,却哭坟无泪,祭拜无哀,那他究竟是不是真的子孙?!”

“不是!不是!”

“是还是不是!?”

“不是!不是!”

一呼万喝,千万人的吼声连成一片,声势之大,仿佛地裂山崩,飓风海啸。站在韩冈身后,人人为之变色。文及甫脸色惨白,浑身上下冷汗涔涔而出,甚至双脚都在发软。

“是非自有公论,公道自在人心。今日三问,可见我白马县乃是方正之地,百姓亦是忠孝之民。方正之县,忠孝之乡,哪有容小人招摇撞骗的余地?!”

韩冈再一次转身,沉如山岳的眼神压着众人的心头。来自千万人的声浪犹然不止,合着他的话声,向着一干官吏猛扑而来,“本官今日将何双垣坟茔并祭田一并断给何允文。此案判决如此,谁赞成?!谁反对?!”

