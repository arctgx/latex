\section{第12章 锋芒早现意已彰(六)}

距离御帐的旷野上,此时汇聚了来自辽国各处的人们。

这些人中,刘霄认识很多。跟着朝廷巡游四方好几年,很多面孔是每年都要见到的。不是高官显贵,便是部族首领,至少也是他们的继承人。当他们出现在捺钵处,都是能够走进金色的御帐。

而刘霄也知道,也有很多人认识自己。

他是咸雍十年(1074)甲寅科状元,但他更是南京道上,四大汉人世家刘家长房的继承人。

南京道上的汉民,以韩、刘、马、赵四家马首是瞻。刘氏远祖刘怦,乃唐卢龙节度使。刘家世代在燕地繁衍生息,等到石晋以十六州归辽,刘家又投靠了契丹。时至今日,已是四代为相,而刘霄既然做了状元,日后也定然是宰相——汉人能做到宰相,在辽国基本上就到顶了,再想往上走,除非被特赐契丹的身份及姓名。

可上千人在场,却静无一声。绝无围猎时的喧闹,纵使相熟知交,也没有哪个人走出来与朋友打个招呼……

这些在辽国国中地位显赫的人们聚集在一起,以北面的一座高高耸立的土台为核心,向着南面排了下去,中间却空着,就像是上朝时文武官的排列一样。

站在高台上的理所当然是大辽实际上的统治者,尚父、晋国国王、太师兼太傅耶律乙辛,以及一批亲信重臣。

以刘霄的地位还不能站到高台上,位置也不能算近,但他站得相对靠前,位置也十分讨巧,在他的正前方,十几步外,正整齐的摆放着几具色泽沉黝体型巨大的金属物体。

从来没有上过战场,但已经是声名远扬。从南国流传过来后,便由国中最顶尖的匠师千辛万苦打造而成的神兵利器。

尽管火炮二字已经传遍了南北两朝,天下万邦。但在大辽国中公开展示,出现在来自全国各地的权贵们面前,这还是第一次。

但这并不是今天的重点,重点是火炮炮口前的一群被捆扎得严严实实的罪囚。

光是尚父殿下的威严还不足让一众权贵们噤口不言,便是在真正的天子面前,也有很多敢于大声笑的人物。寻常时至少还会有一些窃窃私语,是罪囚们的身份造成了这一切。

被绑在炮口旁的那群罪囚,刘霄比周围的权贵更为熟悉,尽管看不清楚模样,可他们的身份早就在刘霄的头脑中。

曾经的北院枢密使,前任的漆水郡王,被夺职的两位大王,也就是一部之首的夷离堇——太祖皇帝登上帝位之前,也只是八部之首迭剌部的夷离堇。

这是罪囚中地位最高的几人。

剩下的,有高官显贵,有一族之尊,还有他们家中的嫡脉子孙,地位低一点的庶子不是成了极北军州的牧奴,便是早一步被斩草除根,他们还不够资格被公开行刑。

是的,这是对叛逆们的公开处刑。

他们犯下的罪过,并非是背叛了大辽,仅仅是反对耶律乙辛称帝。但在尚父殿下执掌大辽朝堂的时候,他们就是不折不扣的叛逆。

刘霄在南京道的老家看过杂剧,开场前都会有一阵子锣鼓喧天的热闹。用敌人的血作为登上帝位的开场锣鼓,在刘霄读过的史书中,翻上几页就能看到一个。

耶律乙辛说了几句话,他身边的侍卫就面向众人放声传话,声音虽大,可风也同样的大,传到刘霄这里,已经变得很模糊了。说了什么听得不太清楚,不过他也不需要听清楚,他已经看见了。

一群虎背熊腰的侍卫走向火炮后的人群,从罪囚中拖了几人出来,一人对应一门火炮,用木桩和绳索牢牢的固定在炮口前,顺手还拉掉了堵在他们嘴里的布团。

破口大骂和哭叫声在那几人中响了起来,可没有人理会。

侍卫全都退回去了,每一门的火炮旁,都有一名士兵拿着火把走了上去。

刘霄喉咙开始发干,双手也紧紧攥起了拳头。不独是他,所有人都盯着那几支火把,沉默着,一股紧张感弥漫在空气中。

火把凑近了火炮的尾端。

刘霄正对着火炮,清晰地看见火把凑近的是一根长长的白线。

火焰点燃了白线,白线上闪起了火星,火星顺着线滑进了火炮中。

然后便是火炮炮口火光吞吐,腾起一片白烟,同时几声巨响此起彼伏,仿佛惊雷在耳畔响起,又像是重鼓就在头顶敲动,刘霄的耳中嗡嗡作响,差点就要摔倒在地。

随着火炮鸣响,人群喧哗着向后退去,如风行草偃,被惊得倒下了一片。

烈风鼓动旌旗,硝烟即时散尽,人群又恢复了平静。

再去看火炮,炮口前的罪囚已经不见了踪影。

正对着炮口的躯干没了,连同背后的木桩一起无影无踪。

下半截的木桩尚留在炮口之下,几位‘叛逆’的下半身也依然紧紧的绑扎在木桩上。

离刘霄最近的一根木桩上,殷红的断面有一尺径圆,表面上还能看见一点白色。他用了点时间才反应过来,那是脊椎骨的残余。

风,迎面而来,血腥气和硫磺味参杂在一起。

刘霄腹中顿时一阵翻腾,早上喝下的羊肉汤,几乎就要冲到了喉头。他不是没有见识过血腥的官员,可他也没见过如此惨烈的死状。

他立刻闭上了眼皮,扭开了脸。再次睁开时,刘霄就看见了一张干干净净、瞪着双眼的年轻面庞,距离脚边只有十余步。但那只是带着半边肩膀的头颅,张着一张他十分熟悉的面孔。

“萧……”

脱口而出的话声陡然中断,因为心中的忌惮,更因为再也忍不住的呕吐。

就在刘霄低下头的时候,近处也是一片惊呼,距离火炮位置最近的权贵们,即便有再多的见识,也从来没有面对过如此恐怖的场面。

来自不同地域、不同部族的权贵,如今都是一般的面无人色。不止一人呕吐出来,就像方才开炮的那一瞬间一样,人群再度向后退开,退得比上一次更远。

刘霄抬起头来,原本与他挤在一处的人群,早就退到更远的位置上去了。

再往下看去,就是一片被碎肉洒满的沙地。

双眼瞄到一块东西,刘霄再一次低下头去,那是一块形状完整的肝脏。刘霄并不知道那是肝脏,但他知道,片刻之前,那块内脏还在处在一个活生生的身体内。

火炮的威力,简直是骇人听闻。

刀能砍出一道伤口,枪能捅出一个窟窿,骨朵能将骨头粉碎,可没有哪件武器,能将人打得粉身碎骨,除非是从千百丈的高空掉下来,否则除非被乱刃加身,死状再惨,好歹还能有个人形留下来。

但挨了火炮就是没有,完完全全的粉身碎骨,仅仅是一炮之威,便恐怖如斯。

刘霄曾经听说过当年大军攻入宋境,逼得宋主亲征,在澶州城下,前锋大将萧达凛中了一击八牛弩射出的铁枪,半边脑袋不见了踪影,下葬时,脸上是盖着银质的面具。当时已经以为是惨绝人寰,可那样的死状,也比不上今天的恐怖。

又呕吐了口,咬着牙,刘霄退了几步,耳边尖叫声又高了起来。

转脸睁眼,看见的一幕,刘霄觉得自己终生都不会忘记。

只见青紫色的半截肠子挂在一名老者的脸上。隔了有一二十步,也不知是怎么崩了过去,硬是把一个看着矍铄硬朗的老头给惊得坐到了地上,旁边也没人记得帮他拿下来,只顾着尖叫。

那老者刘霄看着眼熟,记得是国舅帐中的司徒,但肯定不是尚父的支持者,否则就会常常出现在尚父大帐中,现在也会站在高台下,而不是与官位不高的自己站在相近的位置上。

国舅房中,尚父耶律乙辛的亲附者一直都没有多起来,但今日事后,明面上的反对者一个个粉身碎骨,即便是再如何反对耶律乙辛篡逆,怕也是绝不敢再妄生异心。

不仅仅刘霄这里,人群中很多处都是一片难以抑制的惊叫。位置稍微靠后的权贵们看清了惨烈的场面,都难以抑制自己的声带。

高台上的耶律乙辛突然有了动作,从座位上缓缓站了起来。

尖叫声戛然而止,就像是夏夜虫鸣中,突然有一人踏进了草丛,在一瞬间,变得寂静无比。

千百道视线重新汇聚到了耶律乙辛身上的视线中,平添了许多畏惧。

然后尚父殿下再次坐了下来。

在尚父殿下的眼前,又是一批侍卫上前去,将火炮前的柱子起出,将下一批罪囚绑在炮口前。

火炮不停地鸣响,一批批的叛逆在火焰和硝烟中粉身碎骨,化为一滩肉泥和残肢。

当地位最高的几人被拖到了火炮前,耶律乙辛这才重新站起身来,从身边的亲卫手中接过一个望远镜,静静的观看着。

直到炮声响过,硝烟散尽。

这是对所有反对势力的威慑。

这一日,鲜血染红了永平淀。

下一日,耶律乙辛即将昭告四方。
