\section{第十章 千秋邈矣变新腔(15)}

区区一府之地,就能达到衣被天下的等级,这决不是西北各路可以与之相提并论的。绝不可能像过去那样,依靠棉行内部同进共退而带来的规模上的优势,将一个个势单力薄的对手,

江南富庶,土地膏腴,粮食的亩产量往往三倍四倍于西北,加之江南的田亩数量也远不是千丘万壑的陇西可比,若是这样的田地转种棉花,天下棉布的产量翻几倍十几倍都是可以想见的。

高等级的陇西棉布,如今在京城市场上的地位,大约是寸布寸金的蜀锦那个级别,而普通一点的棉布,也相当与上品的丝绢。

即便如今市面上还有其他地方出产的棉布,不过从规模、品质、种类和口碑等各方面,都远远不如陇西棉布。所以陇西棉布才能维持着高高在上的地位。

与品牌优势带来的高昂售价相比,棉布远比丝绢还要低,甚至因为半手工半机械化生产还低于麻布的生产成本,根本不值一提。

不过一旦江南开始大规模生产棉布,以其土地数量和种植条件,很快就会形成压倒姓的优势。

简单的品牌效应,根本抵不过数量上的优势,当棉布不再成为数量紧缺的奢侈品,而与麻布规模相当,陇西棉布的好曰子就到头了。

就是因为了解到陇西在生产上的客观条件的不足,早在棉行成立伊始,一方面就对行会内独有的织造技术严防死守,严格防止外泄,并不断投入巨量资金,对织机、纺机、轧花机等有关棉布生产的机械进行研发和改进,让外泄出去的技术无法追及,另一方面,便开始对天下各路所有可能种植棉花的地区,展开监视。

将西北的特产运去天下诸路的关西棉商,都会将各地的见闻传回家乡,尤其是出现竞争者这等生死攸关的大事,立刻就会引发行会内部的高度关注。

韩冈之所以能够随时了解到各地棉花的种植情况,乃至于各地的商品价格的变动,正是来自于棉行的通报。

江南早几年就开始种植棉花,但棉花生产开始上规模还是去年的事,当去年年末,江南自产的棉布打到了十万匹的规模后,韩冈这里就不断收到来自棉行内部的请求。

很多人都希望韩冈能将敌人扼杀在摇篮中,但韩冈自己最清楚。他对江南的棉纺织业,能够做到的只有拖延,想要扼杀根本不可能。皇燕京做不了快意事,更别说韩冈这个参知政事了。

以韩冈的地位,他想要压制江南的棉花生产,的确有冠冕堂皇的理由——这也是棉行想要他去做的——那就是粮食。

棉花是从地里长出来的,又不像是桑树可以长在不适宜耕种的山坡上,天然的就要与主粮争夺田地。

来自于东南诸路的纲粮,攸关京师的粮食安全问题。如今每年已经接近七百万石的纲粮,若有个闪失,京师都要大乱。

如果江南粮食生产不足,纲粮就有可能不足,东南各路的常平仓也会无法补足缺额。一旦遇上大范围的自然灾害,东南各路就会成为火药桶。

在京师百万军民皆仰食东南的情况下,朝廷当然不会允许东南诸路有太多田地转产棉花。

在韩冈的眼中,粮食安全自然要比棉花更加重要,但在高额利润的引诱下,朝廷即使下达禁令,也无法阻止江南田主的向利之心。

更别说那些出身江南的官员,必定会为自家的利益而拼命阻止禁令颁布和执行,而地方上执行实务的吏员,也必定会出工不出力。

棉花的种植技术不可能不外流,江南也还有稻棉轮种的可能。只是在耕地上,江南不如西北多牲畜,但完全可以以人力替代。

政治手段,尤其是缺乏执行力的政治手段,根本不是经济规律的对手。从各种角度来看,江南开始大规模生产棉布,已经成了定局。

现阶段只有朝廷的和买手段来威胁,才能够几年内稍稍延缓江南开发棉纺业的脚步。

而对江南发展棉纺业的另一个阻碍,就是棉种问题。如今的棉花品种,对江南当地的气候能否很快适应,其实还说不准。据前往江南的不少行商探查得知,当地棉花的亩产量,现在普遍比陇西的平均水平还要低一点。

以江南的自然条件,棉花的亩产量尚不及陇西,可见棉种问题没有解决这个猜测,并非无的放矢——不过既然历史上江南地区能成为中国、乃至世界的棉布生产中心,棉种问题不可能困扰江南太久。

在朝廷和买的威慑下,棉种的问题解决之前,来自西北的棉商们还有几年的时间可以供他们准备。

在韩冈看来,想要与江南棉产业相竞争,必须做到两件事,一件事是降低成本,另一桩则是扩大规模。然后才可以做到与江南一较高下。

扩大规模,首先就是扩大棉花的种植面积。

由于棉布通行于世,棉行这个区域姓的行会,影响力早已扩大到全国。

而以棉行为核心的雍秦商会在襄州,以及襄州至京城这一线的商业圈中,有着很重的份量,尤其是仓储转运,在襄州是独家买卖。这两年,从京城和襄州这两个中枢节点,同样将势力探伸到全国。

但棉花产地,依然局限在西北。其中熙河路发展得最好,秦凤次之,甘凉、宁夏则是刚刚起步,而在天山南北,适宜种植棉花的地方甚至更多,在种植规模上,西北还有很大的开发余地。只是由于地域广大,棉花和棉布的运输,要占去大量的成本空间。

这就使得棉行必须同时推进先进的织造工艺,在制造成本上压倒竞争对手。

新式的棉纺织技术,各色机器,甚至还包括蒸汽机在内,韩冈将每年收入的很大一部分都投入到了研发之中,而在他的引领下,同时也因为尝到了甜头,棉行的成员也都没有吝啬通过棉布赚取的收入。

有了机器,现在在棉纺工场中,织造的效率能够做到几十倍、近百倍的提升。

棉花的轧制,棉纱的纺成,棉布的织造,在西北都已经在使用机器,减少了大量的人工,适应了西北缺乏人口的状况,同时也降低了大量的成本和时间。

这些纺织机器,以巩州为中心,从外到内,技术水平不断提升。巩州的几家棉纺织工场中所使用的机器,已经不是间谍看上一阵就能将技术给偷走的水平了。

就是拿到原型机,想要一模一样的模仿出来,都不是江南的一家一户能够做到的。至于江南的田主会不会联合起来,这并不是需要担心的事。大户开办织机工场,小民则种棉纺纱,这样的分工合作,符合江南田地零碎、地主家的田地都是东一块、西一块的特点,但与一家一户都能有上百亩连片田地,大户更是阡陌相连的熙河路相比,缺乏工业化的源动力。

只要将这些机器继续改进,西北地区的棉布产业,只会遥遥领先于天下各路的竞争者。

现在困扰织机进一步发展的就是动力的问题,水力、畜力是机器动力的主力。而蒸汽机,韩冈都不指望能够在十年内看到成果。

另外在织造机械的研发和修改之外,棉行内部还在集资实验种植各种棉花,试图从其中培育出更好的良种来。而在事前的约定中,出产的种子只供给所有行会内部的成员。

这也是西北棉纺织业的优势之一。

不过对手不仅仅是江南,河北方向上也会有问题。

河北种植棉花有着很大的希望,尤其是沧州。

沧州靠海,面积光大,基本上都是平原。只因为土地偏盐碱,才没有多少人来落足。

而且在河北,不仅仅是沧州,河北东路临海诸州都有很大一片荒地,那是因为水患和近海双重因素造成的结果。

但棉花耐得住盐碱,正是种植在当地最好的作物。

有了各地同时推广棉布,其取代丝绢,成为国内的主流织物,韩冈的想法也许在十年内成为现实。

……………………

京城的风中带着暖意,完全是春天的气候了。

韩冈在殿上的提议,已经得到了太后的应允。而猜到他本意的也有好几个,章惇在出来后,就冲着韩冈摇头了好一阵。王安石也是投来冷淡的一瞥。

道路上,在头上带着花的人也多了起来。京师之中,无论男女老少,都有拿着花做头上饰物的习惯。

对京城中这样的风俗,韩冈并不喜欢,看着不顺眼,想起自己曾经不得不簪花,越发的心中有抵触。

但新科进士簪花是从唐时便开始的习俗,探花郎之名,也是此中而来。有世所称羡的进士引领,想要改掉簪花的风俗,可是难得很。

韩冈自己都带过花,又怎么去干涉多年的习惯。?

骑着高头大马,韩冈从大街上招摇而过,往家的方向过去。

一路上,韩冈看到了不少士子避让道旁。

已经接近礼部试发榜的时候了。
