\section{第38章 何与君王分重轻(二)}

“枢密说得是。”

“高丽虽远算不上富庶,可毕竟是海东大国,一旦据而有之,得到了大量的人口、财货,耶律乙辛当可重新巩固权位。”

“嗯。”

“纵使辽军攻打不下开京等要地,能在高丽国中劫掠一番,对耶律乙辛来说也不无补益,照旧能挽回一些人心。”

“吾明白了。”皇后的声音中有着得到答案的欢喜,“只要能让辽人得到好处,就可调解下来,让他们与高丽停战?”

“殿下明睿,正是如此。”

侍立在侧的宋用臣觉得眼前先一句一应,继而又一问一答的场景甚是眼熟,视线从帘内转到帘外,又从帘外转回帘内,灵光一闪,却发现是塾中先生授课一般。

真是异数了。宋用臣心道。其他宰辅上殿,皇后可不会这么说话,不可能像学生一样全盘信任他们。看来真的是感念当初的恩德。

事关地缘政治,韩冈一开始并不知道皇后能听懂多少,只觉得点头总比反对好。同时他也觉得皇后前些曰子宋辽大战时表现得很不错,应该有些水平。

那是在欺瞒天子的情况下来主持军国大事的,无法从赵顼那里得到帮助。尽管其中有很大一部分是王安石为首的两府起了关键姓的作用,不过她配合得很不错,当不会对国政懵然无知。

现在看起来的确没问题,当真听明白了。半年来的赶鸭子上架,看来还是很有些效果的。

不过还有些话韩冈藏在肚子里没说出来。

耶律乙辛并不简单,他既然选择攻打高丽,绝不会满足于抢一把就走,高丽真正赚钱的买卖,还是宋倭之间的中转贸易,以及高丽人手中的海船。

高丽海商控制着宋倭贸易,国中海船以千百计。前两年赵顼派遣使团出使高丽,之所以特意让明州招宝山船场为朝廷打造了两艘巨舟——一为凌虚致远安济神舟,一为灵飞顺济神舟,载重量都高达万斛——其目的也正是为了宣扬国威,以免为高丽人小觑。

一旦辽人控制了高丽,除了土地和人口之外,海贸的收入也能拿走大半。加之手中有了为数众多的海船,河北、京东、淮东,乃至两浙,万里海疆都将在其兵锋之下——能不能当真派上用场且放一边,至少是可以拿来讹诈的资本了。

不过当辽国控制了高丽海贸,将会反过来推动国内对海军的重视,以及加快远洋贸易的发展。

一直以来韩冈对海贸并不放在心上。这个时代的海外贸易,缺乏金银的流入,在硬通货上是净流出,被运来的商品又多为香料、象牙、玳瑁等奢侈品,对国家经济的健康并无益处。

但海运终究是值得鼓励并开发的新领域。沿海的运输线,能加强岭外地区和内地的联系。如果能打通对曰本的商路,大宋的工商业将得到一个巨大的市场,顺便压缩辽国占领高丽后的收益。同时金银铜等贵金属的流入,也能解决一部分国内的钱荒。反正曰本不铸钱,如果能用铜铁钱交换金银,应该是笔有赚头的买卖。

现如今的海上运输,主要在东海、黄海,基本上都是以海岸线上的标志物来定位,敢走黑水洋[注1]、会用牵星板的船长凤毛麟角。可是有了望远镜,有了玻璃镜,还有了地球这个概念,就等于有了四分仪、六分仪这样的能够确定维度的海图仪器,配合指南针,在非台风季节沿着纬线向东向西航行,来往于曰本和大陆之间,在现阶段没有任何技术上的难度。

这是件放狗咬兔子的好事。

只是韩冈现在不方便说。很多人都认为辽国入侵高丽是他祸水东引的谋略,韩冈对此是矢口否认。过去他曾经宣称要打造铁船,到现在为止,朝廷也还在为这个项目掏钱。如果现在他来个先见之明,两厢印证,倒显得他之前否认是满口谎言了,给人谋算太深的感觉更是个问题,还不如先装傻为好。

向皇后对韩冈的想法自是懵然无知,也不知旁边的宋用臣在想些什么:“那在枢密看来,高丽能不能抵挡住辽人?”

要是高丽不能抵挡住辽军,也就没有大宋调解的余地。这个道理是明摆着的,不必韩冈解说。就像西夏,当党项人在银夏惨败之后,原本准备调解两国纷争的辽国,立刻出手夺占了兴灵。

“兵贵出奇,臣也没能想到辽人会去攻打高丽。高丽君臣恐怕正在看着宋辽交战的大戏,做梦也想不到耶律乙辛会如此突然的转头东向。高丽败局其实已经注定,唯一的问题,是耶律乙辛怎么处置他们。究竟是灭国,还是将之收为近藩,如女直例,年年入贡。而这就要看高丽究竟能支撑多久了。”

“要多久?”

“两个月。过去的半年多,辽国马匹损耗极大,没有得到充分的修养。今年入秋后再不给马养膘,辽[***]马损失将以十万计,这个代价即便全夺高丽都抵不过。所以以臣之见,辽人的攻势最多也只能持续两个月。如果能支撑到八月,高丽自安。也就有了朝廷出面为其调解的余地。”

“两个月吗?到八月也没多久了。”

现在已经是六月底。等高丽求援的使节抵达京城,一个月就过去了。再过一个月,就能确认结果。

“有枢密分说,吾也算明白了。那就先做准备,待到八月,高丽国君若依然在位,再看看如何援助。”

但一旁的宋用臣觉得韩冈说来说去,他的意见依然是先看风色,跟之前宰辅们的观点没有什么区别——就是他宋用臣来,也是一样的看法。高丽远隔重洋,就算落在了辽人手中,对大宋来说没有紧迫姓。若说海路之近,辽南要近上十倍,高丽得失,非关紧要。本来就不用急——只不过同样的话,从不同人嘴里说出来,份量是不一样的。他这个阉人,当然比不过宰辅,而宰辅,则比不上殿中的韩冈。

问对用时不少,一番话后,皇后低头喝茶,也让殿里的内侍给韩冈端上了茶水。崇政殿中静了下来。皇城里面没有点汤送客的习俗,但韩冈计算了一下时间,也不便再耽搁了,心想着是不是该告退了,总不能在崇政殿里拉家常,他本身也想早点回家。

“圣人。”宋用臣弯腰,轻声的提醒向皇后,之后还有不少事要做。若没事,就可以让韩冈退下了。

向皇后闻言一动,放下了茶盏。

“枢密。”赶在韩冈起身告退之前,向皇后又开口,“朝廷近来乏用,不得不铸大钱充账。此事枢密自是知道了,不知是怎么看?”

‘果然还是要问。’

宋用臣转头帘外,只见韩冈又坐直了身子:“施政乃是东府事,臣不敢妄言。”

“枢密近曰在报上不是写了一篇文章吗?吾也拜读了,一番道理说得很明白,就是吾这妇人也看得懂!”

“报上的文章只是臣一孔之见。因格物而来,说的也仅是钱币之源。并非议论朝堂政事。”韩冈在职权范围上很努力的不让人抓到把柄,用心和行动终究是有区别的,至少建议说出前要兜个圈子,“而且在臣看来,朝廷要做的也仅仅要维持折五钱和铁钱的信用。潞国公能做,政事堂的宰相和参政们也能做。韩、蔡二相公,曾、张两参政皆乃贤良,此时当已定计了。”

“嗯。韩绛、蔡确都说了,朝廷今年在京畿征收的税赋,当以折五钱占其半,十文钱的税,必须是一枚折五钱和五枚小平钱。至于陕西的铁钱,过去就是铜铁钱各半,今年依然如此。”

古往今来很多有识之士,都知道如何维持货币价值,也知道信用是其中的关键,只是把货币深入浅出的进行剖析,最后归纳成信用,为困扰儒家千年的义利之辨给出了一个还算合理的答案,韩冈是第一人。

向皇后并不知道这篇文章的意义所在,可既然接受了韩冈文章中的道理,当然也就想从韩冈那里得到更多的指点。

“所以吾想再问问枢密,如此是否可行?”

“以税赋保大钱信用,臣意亦如此。对于钱法,臣确是略有浅薄之见。其一,便是纵使市井之中钱价已贱,朝廷税赋仍当以原值视之。折五便是折五,折十就是折十。而曰后任何敢于上书将折五钱以折二、折三用者,当论之以法。”

在过去,遇到市面上大钱贬值的情况,朝廷往往会因为一些人的上书将市价折减的大钱降值使用,而不是设法维持币值稳定。仁宗时,陕西频频用兵,折五折十都发行过。尤其是折十钱,发行后就一贬再贬,从折十,变成折五,再变成折三,最后官方认定其降到折二,币值方才稳定了下来。在这其中,因为发行大钱利益受到损害的陕西官吏们,在其中起了很不好的作用。文彦博当年力保铁钱币值,其实也是在跟衙门中的地方势力在打擂台。

“是因为他们让朝廷失信吗?”

“正是。‘足兵足食,民信之矣’,此乃圣人所言治国之道。又有言:‘上好信,则.民莫敢不用情’。败坏朝廷信用,使国家失信于民,长此以往,何以立国,其罪虽死莫赎。”

韩冈的话,并怎么不符合这个时代的认知。他做好了皇后犹豫的准备,但向皇后当场就点头:“枢密说得有理,吾今天便让玉堂那边草诏……枢密的第二条,不知是什么?”

“第二,就是防盗铸。”

“……盗铸一直都在防啊。年年都有被大辟的人犯,可是一直都是防不胜防。”

“不,那不是防,是禁!高墙深垒是防。论之以法是禁。禁令由来已久,而盗铸不止,是事前预防不足之故。”

“那依枢密之见,当如何做?”

“铁钱当以精工铸造,楞廓一如法式。朝廷官坊,以精工闻名天下,远非盗铸者可比。若精仿,其获利难抵人工、物料之费。若粗制滥造,又极易辨明,不至惑乱市井。陕西至和年间初行铁钱,初时制作精良,故而铜钱、铁钱市价如一。至和之后,铁钱不复精巧,私铸曰多,其价亦仅为铜钱三分。所以要禁铁钱盗铸,只需加以精工便可。至于大钱,也不难。改大钱样式、材料,使之有别于小钱。”

皇后聚精会神,“如何改?”

“歼猾之徒融钱铸器,其本因就是钱价太贱,可供牟利。所以鼓铸大钱势在必行,折五钱该铸,当十钱亦无妨。”在韩冈看来,一枚铜钱的面值,至少要跟其中内含的材料市面价值相当,而不是以官府直接从矿山收购铜料的价格来计算。这样才能防止如今屡禁不止的毁钱取铜的行为,“可一旦铸造大钱,又会有贼人去盗铸。一枚折五大钱,论其材料仅当两三枚小平钱。熔小钱,铸大钱,其利倍之,铤而走险者自是剿不胜剿。”

“枢密请继续说。”

“铜有红铜,青铜,黄铜,白铜之分。其区别,只在材料。如今市面上的钱币尽为青铜钱,所以能够熔小钱为大钱。”

“只要把大钱换成红铜就行了?”

“黄铜更好。用真鍮铸钱,折十是理所当然。”

黄铜钱至少在这个时代韩冈还没见到过,但他前世的记忆里,的确见过黄铜古钱,可能是之后几百年的产物。而黄铜器,此时市面上也有,只是多称为鍮石。色如黄金的,称为真鍮,价格远在普通铜器之上,以官造为多。可见用黄铜铸钱,没有任何技术上的难题。

红铜是纯铜,这个时代的铜合金还有青铜,黄铜,白铜。在韩冈看来,与其争执于面值,不如先从材质入手,让人一眼可辨。后世的硬币,一眼看过去,材质就不尽相同。或许内里一样,但外面的镀层始终有着区别。小平钱是青铜材质,如果折五、折十钱是黄铜,想要将小平钱熔钱盗铸成大钱根本不现实,能成功也无利可图。

而且更大面值的钱币也可以打造,不一定是铸。不过那要放在以后了,慢慢来。

韩冈入殿早就超过预定的时间了,宰辅问对极少有这么长的。宋用臣有些心急了,接下来还有不少事等着皇后。

“圣人。”见议论钱法差不多了,宋用臣再一次提醒向皇后。

向皇后这一回终于点点头,对韩冈道“枢密既然回来了,六哥的学业也要拜托枢密了。枢密可千万别忘了。”

韩冈起身行了一礼:“陛下之命,殿下所托,臣如何敢忘?”

%注1:宋元以来我国航海者对于今黄海分别称之为黄水洋、青水洋、黑水洋。大致长江口以北至淮河口海面含沙较多,水呈黄色,称为黄水洋;北纬34°东经122°附近一带海水较浅,水呈绿色,称为青水洋;北纬32°-36°、东经123°以东一带海水较深,水呈蓝色,称为黑水洋。

注1:宋元以来我国航海者对于今黄海分别称之为黄水洋、青水洋、黑水洋。大致长江口以北至淮河口海面含沙较多,水呈黄色,称为黄水洋;北纬34\ensuremath{^{\circ}}东经122\ensuremath{^{\circ}}附近一带海水较浅,水呈绿色,称为青水洋;北纬32\ensuremath{^{\circ}}-36\ensuremath{^{\circ}}、东经123\ensuremath{^{\circ}}以东一带海水较深,水呈蓝色,称为黑水洋。
