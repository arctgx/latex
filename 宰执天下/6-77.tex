\section{第九章 旧日孤灯映寒窗(中)}

张驯在身边念念有词,似乎还是有关黄裳,宗泽对此充耳不闻。

决定一生命运的考试前,大部分考生都有各式各样的毛病。

有的求神拜佛,有的足不出户,有的茹素断屠,有的大吃大喝,有的出门必须要先用左脚跨出去,一旦错了,就立刻回去,接下来连着好几天都不再出门。

这么长的时间经历下来,张驯现在的毛病,宗泽完全能够体谅。

时间一点点的过去,宗泽也一点点接近贡院的大门。

远望着通过了门前检验的贡生,他忽然看到两名似曾相识的身影,那是宗泽在国子监中的同窗学友。

五千士子中,只有一百余人是来自于国子监,想要在这么多人中看见同学,几率并不算很高。

离着大门尚有些距离,宗泽还是认出了两人——钟世美、潘必正。

宗泽能认识他们,完全是因为钟世美与潘必正与他自己,同为监中今科上榜的贡生——前段时间,国子监判监,以及判监以下的官员、教授,将他们这些今科应考的贡生召集起来,好生的勉励了一番,这就给了宗泽认识新朋友的机会。

不过这两位的名气在太学中并不大,真正名声响亮的是这两位的一名好友。

当三人聚在一起,永远都是那位好友更为引人瞩目。

宗泽之前不认识钟世美、潘必正,却早早的听过了两人朋友的名字。

可惜如今进士科考的是经义,而不是诗赋,否则他们的朋友不说首冠鳌山,也至少能有前十的能耐。可是仅仅是在国子监中,那位朋友每一次考试都是居于末位,更不用说两千监生抢一百名额的解试,理所当然的落榜了。

宗泽听说他最近在写什么文章,准备进献给天子、太后。题目好象是《汴京赋》还是《汴都赋》,应当是模仿《两都赋》《二京赋》和《三都赋》的格式来写。

这本应是十分保密的一件事,不知何时已经在监中传开,并在监生中引为笑谈。

尽管那一位在诗赋上水平很高,在士子中的名气也不低,但终归不过是一名初出茅庐的国子监生,想要与班固、张衡、左思这样的千古之下仍栩栩如生的才士相比,除了东施效颦,就只有自不量力这个词了。

想到那位同窗,宗泽莞尔一笑。

进献赋文,其实与黄裳投身韩冈幕府也没有什么差别。黄裳能走出来,保不准那一位也一样能够自辟蹊径。没有必要在结果出来前大加讥讽。

宗泽还是第一次参加进士科考试,但他的心境却宁静平和。

或许是在京师接触到了太多,反而就没有了初次临考的忐忑。

纵然在学业上不算突出,但宗泽有着年轻人中难得一见的沉稳心性。越是到了关键时刻,他总是会有更加出色的发挥。

张驯需要通过攻击他人,将自己的不安发泄出来,而宗泽就不需要。

望着越来越近的贡院大门,宗泽心中越来越宁定。

不论考题难易,是否正合己意,他都会将自己最好的一面给发挥出来。

……………………

随着考生越来越多的进入贡院,蒲宗孟的心情就越来越是烦躁。

已经差不多该起身去外院了,但他和对面的李承之依然是对坐着,与一个时辰之前没有什么变化。

蒲宗孟几次想要站起来,可看见李承之不紧不慢,他又只能耐下性子与其对峙着。

作为知贡举,蒲宗孟接下来的工作是在贡院大门上锁之后,与其他考官一起,领着一众考生,拜祭先圣。然后再让吏员,将考生们领去各自的位置上。

开国以来,礼部试已经进行了几十科,一切制度都有可以遵循的方向。蒲宗孟要做的事,只要与他的同僚商量好一切如常就行了。

但现在的问题是谁为正、谁为副,朝廷并没有给予一个明确的认定。两人并为权知贡举,要是以贴职来看,当然是有学士衔的蒲宗孟在李承之之上,但职权既然没有确定,李承之就能争上一争,岂会甘愿由与蒲宗孟地位相当的权知贡举,变成权同知贡举?

仅仅是题目的问题,就让蒲宗孟和李承之争执了整整三天,直到最后关头才将考题给确定了下来。

虽说让考官在受命后提前入住贡院,一方面是躲避干请,另一方面便是让考官有时间准备考题,但今科礼部试,蒲宗孟和李承之本就是因为之前的考官都受到了大逆案的牵累才匆忙受命,拥有准备时间严重不足,就这样还花了三天才敲定了考题,那已经不是用浪费时间能够形容的了。

幸好李承之能做事,蒲宗孟也不算很差,一边争执,一边将其他与考题无关的准备都做好,这才勉强赶得上开考。

蒲宗孟还不想离开京城。前一次的廷推其实是帮了他,要不然蒲宗孟就得赶赴河阳府的任上,或是告病请求留在京师。但那样的话,也没可能再返回翰林学士院,即使能够上殿推举宰辅,但偶尔才有一次行使权力的机会,如何比得上日日在皇城中让人奉承?

如今知贡举,便是蒲宗孟不愿放过的机会。若是能够顺利完成,王安石和章敦肯定都要表示一下,蒲宗孟现在对两府暂时不敢保有奢望,但回归玉堂却是他日思夜想。

奢华的生活若是没有权柄相配,如何算得上完满?只此一端,就让蒲宗孟对这次的任务尽心尽力起来。

直到眼下为止,李承之会叛投韩冈原因,依然无人能够确认。蒲宗孟为了安全起见,李承之的任何意见都会翻来覆去的考虑清楚,只要有一点可疑之处,就必定会于李承之议论个明白。仿佛锱铢必较的铿吝商人,变得斤斤计较起来。

而李承之,也差不多是一样的态度。

现在只是看着对面李承之慢条斯理喝茶的样子,就知道现在仅仅是开始,之后还有的是纠缠。

虽说下面的考官基本上都是新党出身,只要他们不叛离,大部分贡生的命运都能够控制在手中。

经义上不过关,刷落。策论上不合意,同样刷落。

只要初考官和覆考官有着同样的意见,那份试卷在他们手中就会被刷落。

最后汇集到主考官面前的试卷,一般不会超过一千份。

但问题一般就会处在最后的名单上。

只要李承之不肯配合,通过礼部试的贡生名单便定不下来,考生的顺序也定不下来。

难道最后要去请太后裁量?

那是不可能,蒲宗孟绝不接受。

礼部试的结果不出,他们就离不开贡院。就算可以上书,连知贡举的任务都无法完成,他们在朝野内外的都会成为笑柄。

而且一旦让太后来做决定,不论太后接受了哪一方的意见,另一方就必须辞官,为自己的坚持负责,绝不可能厚着脸皮再留在朝堂中。

太后会选择谁,蒲宗孟对此并没有奢望。

……………………

“开宝寺那边差不多该开始了吧?”

苏颂难得听到章敦与自己闲聊。

虽然说与韩冈的关系都不错——至少曾经是——又同在西府共事多时,可苏颂与章敦没有什么交情。

不管怎么说,苏颂早在变法开始的时候,曾经上书批评天子对李定任用。可以算是旧党中的一员,至少不会被视为新党,与章敦绝不是一路人。

平日里与章敦的交流,只会是公事,少有闲谈的时候。

不过偶尔闲谈,苏颂也不会不近人情,他望了一下外面的天色:“这个时候的确差不多了。”

“等明天,秘阁那边也要开考了。”

“黄勉仲有才学,多半能通过,其他人,苏颂并不熟悉,不敢妄言。不过能够被推荐应制科,理应有些把握。枢密不也是如此?”

章敦很坦然的摇头,“把握有一些,却不如黄勉仲。”

章敦推荐了一名门人参加制科,但把握并不是很大。

关键还是在阁试上,能通过阁试,就代表有着通过进士科礼部试的实力。

但既然能考中进士,那又何必去做人幕僚,而不是直接去参加考试?

如黄裳这样满腹经纶却科场不利的士人不少,科场不利去做幕僚的为数更多,在给人做幕僚的过程中因功得到官身的也不是一个两个,但几项结合起来,这样的人却几乎是绝无仅有。

即便是贵为西府之长的章敦身边,又有几个才学能够在福建的某个军州,拿到解元的身份?

不可能有。

所以章敦只是为人所请,又看在多年相交的情分上,才答应了下来。而且也并不是黄裳的军谋宏远材任边寄科,

黄裳参加的制科太过冷门,军谋宏远材任边寄科,也只有黄裳这样已经在边事上有所成就的士人过来应考,才能应对世论质疑。

做一个言官,只要胆大就够了。

但军谋宏远材任边寄科,这可是要出典边郡,不仅仅应考的难有信心,就是朝廷也对缺乏临阵经验的士人没有信心。谁敢将一方边镇的军政大权,交给一个文采高妙、善于在纸面上指点江山的官员?赵括、马谡是前车之鉴,丢了盐州的徐禧更是就在身边。

章敦只能感慨韩冈的运气,能有黄裳这样的幕僚。

要是黄裳能通过制科,十多年后,韩冈在殿上就有多了一名助力。更重要的是,黄裳命运的转变,会给韩冈带来一大批自谓怀才不遇的低层官员,在这其中,不是没有珍珠。
