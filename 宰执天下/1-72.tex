\section{第32章 营中纷纷难止休(上)}

军营中的新年枯燥乏味。没有哪家商人会到古渭附近卖爆竹,就是竹竿都少见【注1】,半点也没有过节的气氛。也就驿馆外面的军营里,吆五喝六的赌博声最为响亮。

到了元旦那一日的午后,刘昌祚领着一群偏裨校佐过来拜贺,请着王韶和韩冈一行吃了一顿酒席,也便散了。

古渭寨平日里提供的酒菜着实提不上筷子,用的盐质量又不好,吃到嘴里泛着苦味。这里常用的井盐远不比上秦州通用的池盐——解州盐池和青白盐池所出产的食盐,放在大宋全境都是上等口感。

咸中发苦的菜肴,习惯清淡口味的韩冈根本吃不下去,王韶父子浅尝即止,赵隆和亲卫们也都是叫苦不迭。王舜臣不住的抱怨:“就仗着这鸟地方产盐,一斤一斤往菜里添,想把俺们做腌肉不成?”唯独李信一人,默不作声的吃了个干干净净。

刘昌祚待客虽然都是一板一眼按着礼节来的,可这一番款待却是不冷不热。王韶看起来全然不介意的样子,但对王韶性格已经有所了解的韩冈知道,他的举主恐怕心中早已狠狠地记了一笔。

韩冈心中也不痛快,他知刘昌祚忌惮向宝,心中便转着念头,想着用什么办法让刘昌祚恶了向宝,不得不投过来。

不过韩冈还是颇受古渭寨下层官兵的尊敬,见到他,点头哈腰的为数不少。服侍韩冈起居的士兵,也是嘘寒问暖,甚为殷勤。

韩冈在甘谷城的一番作为,几乎传遍了秦凤路的各处寨堡。数万秦州将士都知道,很快就要有个孙思邈孙真人的徒弟来管勾秦凤路伤病事宜——尽管孙思邈弟子身份的误会,韩冈绝不会在明面上承认,反而竭力澄清;但谣言传播的速度和广度,却不出他意料,正中他下怀。

吃着兵粮,守着边疆,谁也不能拍着胸脯说自己一辈子都安安稳稳地不受一点伤。刘昌祚顾忌着向宝这位顶头上司,但普通的士兵可不管那么多。高高在上的都钤辖连眼角都不会往自家身上瞟一下,何苦为他得罪日后可能成为救自己一命的恩公?

韩冈房中取暖的火盆,就算是到了后半夜也从来没熄过。而他晨起活动过筋骨后,便立刻有人送来大桶的热水请他沐浴更衣。骑乘的坐骑,被刷洗得油光水亮,喂得也是最上等的豆粨。吃得盐苦了,韩冈提了一句后,也好了不少,据说是改用了净水漂去了粗盐中苦味,经过第二次熬煮成的精盐。

这等待遇,连王韶都靠他沾光。王厚也看得眼热,私下里避过他老子,笑着对韩冈道:“玉昆你日后在秦凤估计都可以横着走了,真没人敢得罪你。”

韩冈笑而不语,这话他不好回。

以待人殷勤论,刘昌祚待王韶、韩冈一行的态度要倒着数,而古渭寨低层将校们的表现,则让韩冈想把刘昌祚揪过来,让他好好学一学。至于古渭附近的蕃部对刘昌祚的态度,则是略逊于后者,而远超前者。

最为亲附大宋的纳芝临占部早早的在年前就送来了几十只羊充当节礼,还特意给刘昌祚选了匹好马——一匹高大雄峻的枣红色河西马。到了正月初二,部族中的首酋们又在族长的带领下过来拜贺,在古渭州中,无一家能比他们更恭顺。

纳芝临占部本是古渭州最大的吐蕃部族,一度拥有附近的九条谷地,数万人丁。但如今势力大减,仅保住了其中的三条——这还是靠着他们二十年前第一个归附大宋所结下的善缘方才得以保住。

而取代他们成为古渭最强蕃部的,就是刚刚走进官厅的一群蕃人所代表的部族。

王厚、韩冈闲来无事,守在官厅外,看着一众蕃人鱼贯而入——主要还是韩冈拉着王厚,他希望能籍此对认识古渭的蕃部了解更多一点。在官厅外不过一个时辰,他对西北蕃部,已经有了更为直观的了解,掌握了第一手资料。这比坐在秦州官厅中,翻着故纸堆有用得多。

王韶人在厅中。他提举秦凤蕃部大小事务,既然他人在寨内,而蕃部又来了人,刘昌祚即便不愿意,也不得不让王韶坐进他的官厅。

“是青唐部的人……”

王厚附在韩冈耳边说着。这几年王厚跟着王韶在陕西缘边地区跑了许多地方,对各地的大蕃部都有基本的认识,不同蕃部拥有的旗号和装束都有细微的差别,韩冈看不出来,但王韶和王厚一眼就能分辨。

古渭的青唐部与吐蕃赞普唃厮罗和董毡所据有的青唐王城两不相干,只是恰巧重名而已。说起重名,韩冈前世曾经来过古渭,不过那时名号已是甘肃陇西,还逛过县城附近的首阳山,就是传说中不食周粟的伯夷叔齐饿死的地方。

但不仅陇西县,河北、河南、山西的很多地方都有个首阳山,皆自称是伯夷叔齐最后隐居之所。只是如今韩冈‘旧地重游’,却没听说古渭这里有什么首阳山,想必也是后人臆测生造出来的。

青唐部在古渭附近是人丁最多,据地最广,也是最为富庶的一个部族,甚至连带着古渭寨合在一起被世人称作青渭。其所据有的盐井,据说每天能给青唐部带来八匹马的利润。这可是个惊人的数字。

北宋马贵,一匹最普通的驽马也要十贯往上,而战马都是三十贯起头,往百贯上跑。即便以价格最廉的驽马计算,八匹马就是一百贯,而一年便能净入三万五千余贯!

王厚当日提起青唐盐井,曾经叹着气,若这三万五千贯年入归属古渭寨,不用下面的臣子提,官家自己都会要求古渭建军。

“青唐部不是没有归顺吗?他们怎么也来了?难道俞龙珂打算向朝廷要个官做?”韩冈有些想不通。他这些日子,也多方搜集蕃部的资料,虽不如王厚的见多识广,但还是知道青唐部的族酋究竟是何人。

青唐部并未归顺大宋,没有接受朝廷官职,更没有献土。按照大宋对蕃人的分类,他们属于生户,而投效了大宋的纳芝临占部则是熟户。一个生蕃部落跑来拜年,是惯例?还是特例?

“能关起门来称大王,俞龙珂当然不会愿意成为大宋臣子。但这不代表青唐部不愿与朝廷交好。平日结个善缘,也省得日后麻烦,许多蕃部也都是这么做的。何况青唐部除了盐和马,也不产其他东西,都要靠着来古渭的商队……”

“青唐部不是号称帐下超过十二万口?”王舜臣一贯的把蕃人当贼看,从来都是往坏里想他们,“俞龙珂那鸟货说不定想做个李元昊,前面磕着头,背后捅刀子,囫囵个儿的占了古渭州!”

“十二万口?”王厚不屑的冷笑一声:“的确是有!把羊算上去还少一点,加上狗那就多一点。再添个马,说不定能上十三万!”

韩冈也摇头失笑,这样的传言都是不能信的,秦州是西北重镇,汉人也才不过十余万丁口【注2】,一个蕃部怎么可能有与秦州相当的人力:“帐下十二万口当然是个笑话。古渭就这么点大,能容得下多少帐?

大小部族加起来,说不定的确能有十二万。单一个青唐部,能有个三万丁口,编组两三个装备齐全的千人队就不错了。董毡或木征的直领部族,估计也不过是十万上下!”

“但董毡和木征一声号令,三五万吐蕃精锐也是轻而易举。即便俞龙珂,也能在古渭凑个一万上下吧?”

“兵力多少无关紧要,”韩冈说道,若要拓边河湟,却连青唐部都打不过,那就别去想河州木征,以及青海畔的董毡了,“青唐部当道而立,要出兵河湟,绕不过他去。要么灭了他,要么就要收服他。决不能容许他首鼠两端!”

“可木征、董毡和西贼都派人去过俞龙珂的帐中。”韩冈对地理的认识,已经被王厚所敬服。而青唐部的战略意义,不必韩冈说,王厚也明白。就算他对地理不甚了了,但从木征、董毡以及夏人对俞龙珂的拉拢中,任谁都能看得出青唐部的重要性,“墙头草是两边倒,俞龙珂可是四方跑。董毡、木征、西贼还有朝廷,他都是逢着庙就烧香,一个菩萨也不得罪……”

王厚正不屑的说着青唐部四面拜佛的丑事,官厅门前人影一晃,身高体阔的赵隆从厅中走出来。赵隆的身材和相貌所具有的威慑力,要远高于王舜臣和李信,故而被王韶带在身边,与刘昌祚一起接见蕃部来客。而王舜臣和李信就只能站在帐外,守着韩冈和王厚。

注1:最早的爆竹,就是将干竹节放进火里去烧,听着竹节爆裂的声音,爆竹因此而得名。到了北宋后,火药爆竹才逐渐流行开来。

注2:古代统计人口,只记男丁数量,也即是二十到六十的成年男子数目。男丁十二万,换算成总人口,大约有三十万。

ps:王韶要拓边河湟的第一个目标出现了……

今天第三更,照例的红票,收藏。。

