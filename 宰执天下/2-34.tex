\section{第九章 长戈如林起纷纷(七)}

【第一更,求红票,收藏】

俞龙珂一旦决定开始行动,聚在在他身边的领军将佐,便一个个向着四面八方冲去,回到他们所在的队伍中。

也许是为了防止消息泄露,青唐部并没有吹响出征的号角,也没有擂动进兵的战鼓,但一面面高高举起的旗帜,已经向所有在谷地中的青唐部子民,宣告战争的到来。

俞龙珂显然是早就有所准备。昨夜能在韩冈点燃火炬的一刻钟之内,就点起数千人马,他的准备当然足够充分。如今他一声号令,又是区区一刻钟,数千等候已久的青唐部战士,便已经整装待发。

不过俞龙珂并没有动,他还在等着,所有的青唐部战士都跟他一起等着。

马蹄声初始时微不可闻,但很快就随着一个骑着马的身影一起变大了起来。一名高大雄健的骑手跨着一匹同样雄峻的战马,朝着俞龙珂直奔而来。他的马颈下,挂着两个圆球状的物体,韩冈都不用细看,便知道这两个应该都是不小心撞上了枪尖的倒霉蛋。

高大的骑手在俞龙珂马前跪倒,拎着两颗头颅献了上去,:“启禀族长,小人今天奉命巡视周围,斩获两名贼人哨探的首级,还请族长查验。”

韩冈在旁边看着两枚首级,都是蕃人装束,而且死不瞑目,龇牙咧嘴,从眼角、鼻孔还有牙缝中一条条渗出血来,样子甚是恐怖。

“既然是越格你带回来,也没有查验的必要。”俞龙珂把两枚同样来自蕃人的首级接过来高高举起,向着麾下将士们亮了一亮,“把这两个首级挂到我的大纛上去,今次就拿他们祭旗。”

拿出一条哈达赏给第一个带回敌军首级的游骑,俞龙珂又继续等着。韩冈现在明白,青唐部的族长是想把董裕派来的哨探都一网打尽,才开始向外出兵,就算董裕能从消失的哨探察觉青唐部出来问题,但他却不可能再凭哨探察知俞龙珂的动向,这样便能打个董裕措手不及。

继第一名游骑之后,一名又一名的青唐部骑兵紧跟着回来了。他们带回来的贼军首级为数不少,但这些游骑,也有不少人身上都带着伤。韩冈明白,他们的成功可是费了一番辛苦。也许还有些同伴,可能已经回不来了。

已经不再有游骑回来,而俞龙珂仍然在等,韩冈也保持着足够的耐心。而韩冈不催促,俞龙珂倒是奇怪:“韩官人不心急吗?”

“本官很心急。但用兵往往是越是心急越容易出事。今次一战,最好的结果是一战而定,让董裕无力再起。俞族长老于兵事,也无需本官多言。”

“呵呵,我明白了。”

俞龙珂当然明白,韩冈的话正说到他心里去了。不论怎么说,胜利还是第一位的。至于七家蕃部,能就则救,救不了拉倒,不用太在意。

终于……一缕尘烟自远方腾起,马蹄声随之传来,一队骑兵急速奔回,他们身上的披风和帽盔都是灰蒙蒙的,显然在野外有一阵子了。离着大队还有百十步的地方,他们便勒马停步。领头的队主缓缓上前,并没有献出贼人的首级,而是向俞龙珂禀报他打探的情报。

这名斥候的第一句话,就是石破天惊,“董裕的前锋已经到了古渭寨!”

王舜臣骑乘的坐骑突然长嘶了一声,仿佛在惨叫。在周围的蕃部将佐看过来之前,王舜臣忙把坐骑安抚,又不为人知的悄悄把方才揪在手中的马鬃给擦掉。

俞龙珂瞥了韩冈一眼,但在他脸上什么也没看出来。

“既然董裕的前锋已经到了古渭寨外,官人想回古渭就有些难了。不如这样吧,还请官人随我一起行动,等我家儿郎斩下董裕首级的时候,也好让官人做个见证。”俞龙珂提议着,让韩冈随他同行。

“这是自然。”韩冈点点头,又道,“不过本官还要派两人回去报个信。”

在随行的护卫红找了两个胆大心细的,韩冈让他们回去禀报王韶。两人领命走后,俞龙珂已经点起他要带走的兵马,虽然俞龙珂麾下战士数以千计,但他这时只领了族中精挑细选的八百人。而落选的三千多士兵,则给俞龙珂分作几队派了出去,用来在外面虚张声势,好让董裕把注意力转移过去。

八百骑兵汇集于谷地,分作了八个阵势,虽然做不到顶级精锐那样的队形齐整、阵列俨然,但也是一个个神气完足,气势昂然,丝毫不为即将到来的战斗担心。他们都在身上披挂着皮甲,而战马上也披着厚厚的毛毡。另外这些上阵用的战马都是牵在手上,他们现在骑乘的是另外一匹用来赶路的坐骑。

韩冈想不到俞龙珂竟然能拼凑出如此之多一人双马的带甲骑兵,虽然那些甲胄有新有旧,但青唐部的富裕已经可见一斑,每年三万贯的盐入看来也并不是光用来装饰俞龙珂的宅邸。

“有此近千甲骑,必能旗开得胜,全师而还。”韩冈说着通用的吉利话,以讨个好口彩。

而俞龙珂却道:“有韩官人压阵,旗开得胜当然不在话下。不过全师而还那就难说了,兵凶战危,今次跟随我出征的这些儿郎,能有一半安然返回那就是万幸了。”

韩冈轻轻一震,他怎么听着俞龙珂的一番话中好像有着言外之意。他望过去,却正好对上俞龙珂蕴意颇深的眼神。

韩冈会心一笑,想不到自己的名声已经传到了蕃人的耳中:“我观今次出征将士,都少有夭折之相,能安然返回的还是占了绝大多数。”

“听说韩官人是秦州有名的神医,管着秦凤路上的所有伤病营。好像连古渭寨都多了个疗养院。”

韩冈在古渭寨的名声不小,孙真人嫡传弟子这个谣言流传得也很广。俞龙珂虽然早前没听说过他,但青唐部中听过韩冈名讳的族人却有许多,昨夜稍一打听,也就把韩冈这个人了解了许多。

“神医绝然当不起。但管勾秦凤路伤病事却是真事。韩冈虽不能开方施针,但照料一下病人,使他们能早日康复,却还是有些能耐。”韩冈几乎是拍着胸脯说话,昨日他是为了朝廷去说服青唐部族长,而现在,他是为了自己要向俞龙珂结个善缘。

据韩冈所知,后世传得神乎其神的藏医藏药,此时仅有个雏形,如今的吐蕃蕃部是缺医少药,宋廷赐给归顺蕃部的物品中,除了金银财帛之外,还有很大一部分是蕃部急需的药材。

王韶在古渭坐享其成,韩冈却不想白白为人出力,最后却赚不到大头。王韶对付蕃部的手段是恩威并施,其中的恩,也就是善缘,不如由自己来结。因为他是管勾路中伤病事,无论古渭还是渭源,这一带的蕃部也在秦凤路中。

俞龙珂右手抚胸行礼,“若是今次出战的儿郎,能得到韩官人的救治,青唐部上下必然感激不尽。”

“何须感激,既然今次青唐部是为了朝廷讨伐董裕,韩冈为了受伤的将士出一份力也是理所当然!”

韩冈一直都希望自己能拥有更大的发言权,拥有着更多的晋身本钱,以便在天子和宰执们的心目中,成为一个对河湟事务有着深刻了解的官员。

在天子心目中成为某方面的专家,就代表一旦那里需要人,或是需要征求意见,朝廷就会第一个想到他。

富弼出使过两次辽国,而且是在紧急情况下临危受命,他凭着这份功绩以及对辽国的了解升任了宰相。韩琦担任过陕西宣抚使和秦凤经略使,关于西北的边事,朝臣之中他的发言权一向最大。薛向是荫补出身,而且专长是充满铜臭味的士大夫们所不屑的财计之事,可即便进士出身的士大夫经常攻击他,但赵顼当初考虑均输法是否应当实行时,却照样去征求他的意见。

王韶如今主持河湟开边之事,就算他最后失败了,日后一旦朝廷重提此事,多半也会启用他,而朝廷要征询有关河湟蕃部的意见,也会找王韶问话。

后世招聘通常都是有工作经验者优先,也是同样的道理。知识、经验、人脉、资历,无论古今,都是在职场上、官场上衡量人才的最重要的几个指标。

有这些人做例子,韩冈也在考虑着如何发挥自己的优势,在河湟蕃部中留下浓墨重彩的一笔。人情也好,威望也好,就算是能让小儿不敢夜啼的恐怖也行,总得留点东西下来。这就是资历,这就是本钱,这就是他未来有机会就可以继续插足西北拓边之事的证明书。

对于此事,韩冈原本已经有了点腹案,也做了准备,现在一看俞龙珂,便明了自己的想法还是可行的。

得到韩冈的许诺,俞龙珂便下令全军出动。依然没有号角金鼓,只是旗帜在舞动。一行八百余骑兵,走得也不是山谷中的大路,而是意图翻山而行。

八百骑兵穿梭在山间小道之上。山道蜿蜒,十步一弯,在人流中,前望不到头,后望不到尾。

由于地处黄土高原,地势是千丘万壑,往往走不了一里地就要经过是几条沟,隔着二十里,就能音讯数日不通。董裕是沿着渭水河谷进兵,要绕道他身后,却也有不少条路可以选择。

蕃人不缺头脑,也不缺对兵法的认识。俞龙珂很自然的就选择了最有利的战略。当一两天后,他们出现在董裕的身侧,便是大获功成的时候。

