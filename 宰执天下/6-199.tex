\section{第21章 欲寻佳木归圣众(19)}

初秋的金陵,终于有了些许凉风。

肆虐了两个月的酷热暑气,也在秋风中渐渐消散。

白天时还是有些热,不过到了太阳落山之后,很快就变得清凉了起来。

穿入窗中的凉风习习,穿得单薄了,王旁甚至还觉得有些冷。隔着袖子搓了搓手臂,他拿起一件袍子,走到穿得同样单薄的父亲身边,“大人,再添件衣服吧。”

比起年初时,王安石又苍老了许多。离七十古稀已经不远的老相公,须发全都白了,乍看起来慈眉善目,已经看不出拗相公当年的那股子拧劲。

“嗯。”

王安石透过老花眼镜,盯着桌上的报纸,只随口应了一声。

他正看着的那一版报纸的正上方,一篇文章被一道黑框框起。框内短文中,故太子太保、上柱国、申国公、司空、赐紫金鱼袋几个头衔极为显眼。

轻手轻脚的给王安石披上外套,瞥眼看到申国公这一封爵,不用看后面的名讳,王旁就知道了此人的身份。

吕夷简、吕公著父子相继申国公,可算是国朝的一段佳话了。

但吕公著的官衔和名讳出现在报纸上的黑框中,就有另外一种意味了。

尽管不知道这种标识是从何而来,又有何典故,但现如今,只要在报纸上看到黑色的边框,必然是噩耗无疑。

正如眼前的这一则——

让王安石在桌旁惆怅许久的,正是吕公著的讣闻。

吕公著的死讯登载在来自京城的快报上,反倒比遣送四方的朝报更早一步送到王安石的手中。

王安石、吕公著早就割席断交,吕家的子弟不会千里迢迢遣人来告哀,没有报纸,至少要到一个月后,朝廷议定了吕公著的追封,王安石才会得到吕公著的死讯。幸好有了报纸,又有了让江宁至京师总计二十二程的水路,缩减到六天的京泗铁路,能够王安石及早的为自己的老朋友、老对头开始哀悼——之前的司马光,他过世了,王安石也是通过报纸和朝报才得以知晓。

对王安石来说,吕公著和司马光即是老朋友,又是老对头,最早以为会是志同道合的好友,再后来,就是道不同不相为谋的死敌,可如今,剩下的就只有去日难追的怅惘。

当年在僧坊一起唱和,宴饮至日终的嘉祐四友就只剩王安石和韩维两人,这如何不让日暮途穷的王安石心中郁结难捱?

“大人,可要儿子去寿州一行?”王旁轻声问道。

虽然说吕公著致仕后,回乡隐居,就在寿州,可谓近在咫尺。但他去世的消息是先到了京师,再从京师传回来,现在不动身,过两天再走,就只能去坟上祭拜了。

王安石沉默的将桌上的报纸折了几折,叠起来放好。上面的讣文被掩去了,而下面的婚庆喜事的通告,倒是露在了外面。

现在好像成了习惯,王旁想着,大户人家的红白事,往往都会在报纸上买上一块版面,公诸于众。

王旁上一次在报纸上还看见了章惇家的长子章持成婚的消息,新妇是福建蔡氏出身,赶在跨马游街之后就成亲,真是一点不耽搁。

王旁也不知道自己的几个外甥到底会是什么时候去参加科举。不过以韩冈的性格,不会让他们去学习新学,这样连解试都很难过的去。

可谁让他们有一个做宰相的亲爹,而且还是执掌一派道统、身为当世大儒的亲爹。等到他们开始去参加科举的时候,想必进士科的科目,已经与现在截然不同,根本不用担心考不考得上,只有名次的问题。

不过就像如今的枢密家的两位公子,同一榜上高中,一个二甲,一位则在第四等,名次不高,但前十名的好处不过是方便进入崇文院,包括御史台在内,三馆秘阁和台谏等清职屡遭清洗,早无过去作为登天之阶的风光。现在东西两府都是务于实务的名臣主持,想要博取美职,先得从做事积累名声和经验,这样一来,就是后几名也不用太讲究名次高下了。

瞥了叠放起来的报纸两眼,王旁叫着默不做声的王安石,“大人……如果要去的话,孩儿这就去打理行装。”

王安石摇摇头,“不必了,既然寿州没有来人,你也不用去。”

司马光那边太远,而吕公著那里就很近,如果自家的父亲想要化解过去的恩怨,洛阳没派自己去,寿州那边是肯定应该派的。

王旁揣测着,是不是自家父亲担心韩冈会误会,认为他打算与新党媾和了,所以才不打算节外生枝。

“还有何事?”王安石硬邦邦的问道。

他的心情不太好,对自己儿子的夹缠不清,也有些许不耐烦。

“本路的提点学政使再两天就要到了。”

王安石板起的脸上终于有了一丝惊讶,“……这么快。”

他扶着桌子慢慢坐了下来,“想必是为了督办蒙学。”

韩冈在路中四监司的基础上,又加了一个学政。

帅司安抚衙门,漕司转运衙门,宪司提刑衙门,仓司常平衙门,现在又多了一个提学衙门。

新设的学官体系,即使将已有的学官归入其中,也是至少能安排一千以上大小官员的肥肉。旧有的县学、州学,不再受到当地州县亲民官的管辖,只不过在考试的时候,亲民官和衙中幕职,都有资格参与进去,作为副考官。

但韩冈依靠学官,进一步收拢人心,这不是王安石忧心地方。

王安石不想看到韩冈的人来到江宁府,直接管理一路教育和考试的学政,免不了要干扰到金陵书院在江南东路的地位。可是更让王安石感到棘手的,是韩冈正在推行的蒙学制度。

一个月之前,韩冈上书请求太后下诏,诏命天下诸州县共建蒙学,招收当地幼童入学,以三年为限,教授学童识字、数算还有天文地理等一系列的自然常识,当然了,也不会缺了《三字经》、《幼学琼林》,以及必不可少的《论语》。

如果修建蒙学是要朝廷掏钱,所有人都会看韩冈笑话。那可不是仅仅容纳几十上百读书人的县学、州学,而是一州一县,学生都要成千上万的蒙学。朝廷即使倾尽全力,也难以维持这样的支出。

但韩冈的提议却不用朝廷掏钱出来,而是倡议天下士绅共建,然后去衙门登记办学,朝廷只需要给所有的蒙学学生安排统一考试,然后给予毕业生终生丁税减半的好处。而蒙学的主办者所能得到的好处,则是要看他们所建立的蒙学,到底能有多少合格的毕业生而定,在这一方面,韩冈更不可能吝啬,不过大多都是让朝廷不用付出太多的实质性代价的奖赏,但也是有足够的吸引力。至少王安石觉得,给达到标准的蒙学的主人以士绅的称号,让他们可以见官不跪,能够吸引足够多的商人和地主。

最重要的,天下间的蒙学本来就是成千上万,根本不需要韩冈提倡就有人办,或是一族合办,或是一个村、一条街、一个里坊来合作,又或是一位士人自己来招收学生,这种额外多出来的好处又有谁不喜欢?只要再用心一些就好。而终身丁税减半的好处,又不愁那些孩子不用心去学,至少他们的父母会各种方法去督促。

韩冈五年内的目标是让每年能够十万人拿到蒙学的毕业证书,最终目标是天下男丁都能上学。当然,人人念书,就跟孔子的大同之世一样,只是一个梦想。不过一年十万蒙学毕业生,就算有一半是滥竽充数,剩下的也有五万了,十年之后,就是五十万,其中只要有百分之一能成才,就有五千人,即便只有千分之一,那也有五百人,五百才士,足以支撑起气学的未来。而更重要的,是天下的幼子,从开蒙时起便受到气学之道的熏陶。

这就是其他学派无法与气学相争的地方,不论是哪一家学派,基础都是建立在对六经的诠释上,而想要去研习任何一家学派,至少要熟读诸经,绝不可能像气学的格物一派,直接从开蒙便着手培养。

王安石这段时间的疲惫都是来自于此,韩冈不仅仅在道统之争上,开始学习王安石的故技,通过科举来操纵士林,大力援引同伴进入朝堂。甚至更进一步,开始培养后人,不怕时间久长,因为在宰相之位上的韩冈依然太过年轻。

而朝廷付出的代价,只是日后每年最多不到百万贯的税赋的损失,看起来很多,但没有人会怀疑,韩冈会弥补不上这样的亏空。

只要二十年,气学的地位将会无可阻止的压服诸派,不论是新学,还是其他学派,都会成为历史。

拥有高屋建瓴的手段,又有着十年生聚十年教训的耐心,王安石忽然发现这些年与韩冈之间的道统之争,似乎都只是自己落入陷阱后的挣扎,看着激烈,其实结局早已注定。

他到底能走到哪一步呢?

满心疲惫的老相公黯然想着。
