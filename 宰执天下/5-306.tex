\section{第31章 停云静听曲中意(13)}

天下上元放灯皆是三日,唯有京城五日。

起于十四,止于十八。

数日间金吾不禁,灯如山海。

但等到了正月十九的白天,人散灯灭,街巷上纵然如平日一般车水马龙,却平添了一层清冷之感。

不过这清冷之感并不是今日就开始的,元丰四年的这个年节,从一开始就比往年要冷清得多。

天子的重病,宋辽的纷争,自去年时起,便是京城中最为让人挂心的话题,

到了正月初八,种谔在为溥乐城解围之后,领军北上追击的消息传来,一下就让京城内的年节气氛降到了冰点。

原本官军火烧耀德城,焚辽人粮草的消息还让京城中不少人兴奋不已,但这一胜利仅仅是救援溥乐城的手段,可领军追击入兴灵,那就是扩大战争的举动了。

而就在两天前,有关种谔的最新消息传来,官军在兴庆府外,联合了两万多党项士兵,与多达八万的辽军大战竟日,最后双方皆是人困马乏,幸而收到了鸣沙城援军将至的消息,让官军鼓起余勇,一举击败了辽军。

南薰门内,国子监旁,黄裳和他曾任襄州知州的堂兄黄庸对坐于酒桌前。黄庸是诣阙抵京,正好于在韩冈门下的堂弟见上一面。不过两人现在都没说话,隔着一面木板墙,隔壁包厢中的声音清晰的传入耳中。

“我还从来没见过露布飞捷抵京,京城里面却人人忧心的场面!”

“开封离河北太近了。”

“也不能说是人人忧心啊,当轴诸公哪一个不是叫唤着要跟辽人决一死战?”

“一群南人,他们当然不担心!贪功好利,败坏国事,福建子就没一个好货!”

“蜀闽同风,腹中有虫,南人多是奸猾之辈,私心太重!岂不知战事一开,河北将有数百万人流离失所?!”

“韩三相公不是河北人吗?”

“就他一个,说话又有谁人听?!”

黄裳和黄庸就在隔壁听得分明,福建出身的他们,听到隔壁北方士子们的议论,也只能摇头苦笑。

黄庸低声问着:“国子监里,南北相哄的事多吗?”

黄裳张开双手,“一天下来,十根手指都数不完。”

中书门下和枢密院中充斥了太多了南方人,北方的士人对两府的人事非议很多。尤其是出身河北的士人,更可谓是怨声载道。黄裳在国子监中,听到地域攻击的次数不胜枚举。

拿起酒杯,喝了口滚热的黄酒,黄裳叹了口气:“等过两天,恐怕会闹得更凶。”

“这话怎么说?”黄庸立刻问道。

“露布飞捷就经过洛阳。这几日从洛阳来的全都是弹劾吕枢密的奏章。有文宽夫的。有吕晦叔的。还有司马君实的。这一回终于是给他们等到机会了。等他们的奏章都传出来,国子监里还能不翻天?”

“司马光还敢说?”

“他又怎么不敢说的?太子太师啊。”黄裳摇摇头,“这一回就是韩学士都在说想不到。种谔好赌谁都知道,但赌得这么大,还给他赌赢了,这还是头一次。”

“谁也想不到党项人也打回了兴灵。前些日子,还以为他们会跟着辽人一起南下。吕枢密用得好计策!”黄庸叹了一声,却突然神色一肃,凑近了压低声:“愚兄也听说这是种谔的计策,吕惠卿只是适逢其会。哪个是真的?”

“还真说不准。”黄裳摇了摇头,又道:“但依小弟从学士那里听到的说法,好像都不是。是青铜峡的党项人自行其事。”

黄裳这件事他听韩冈提起过,并不是如京中传言所说,是种谔或吕惠卿的计策。根本是党项人死里求活的挣扎而已。不敢攻打鸣沙城,却趁辽军攻溥乐,偷袭兵力空虚的兴灵。甚至在这之前,为了迷惑辽人,还故意放出了要背宋投辽的消息,瞒过了所有人,两府之中都是始料不及。

“还真是天欲兴宋啊!”黄庸拖长了声调。

“等攻下兴庆府再说吧。”

“也就这两天的事了吧?”

黄庸正说着。远远的,街巷上突然起了骚动,黄裳黄庸放下酒杯屏息静听,是来自城中心的方向。

声音由小渐大,一下就传到了近前。

王师克复兴庆府!

黄裳霍然而起,与同样蹦起来的堂兄相顾无言。

当真将兴灵给攻下来了?!

……………………

“终于来了?”章敦放下了笔,长身而起,油然叹着,“想不到真的给种谔做到了。”

“辽人在兴灵的主力都败了,兴庆府又如何能守得住?”薛向虽是如此说,但心中同样感慨万千。哪里能想到种谔竟然能全了两年前的未尽之功。

从种谔出兵,到吕惠卿为种谔的作为背书,枢密院这边一直都是抱着看戏态度。凡事终归是陕西宣抚司来承担,功罪与否,都轮不到他们操心。远的不说,就在十天前,只知道种谔北上的枢密院中,也没人认为种谔能攻下兴庆府,夺取兴灵——尽管这时候种谔已经坐在兴庆府的城头上,看着城中风生火起。

事前事后,没有一人能想到种谔仅仅凭借手上的两三千骑兵就达成了这个近乎不可思议的成就。皇帝皇后没想到,宰相参政们没想到,枢密使们同样也没想到。就是号称最知西事的韩冈,他之前也没说过这一回能收复兴灵,反倒是对青铜峡的党项人关心很多。

夺占兴灵,可以说是功劳,但更多的还是负担。这只是引燃草原的一点火星,接下来究竟是燎原火海,还是就此熄灭,谁也说不清楚。

只要将辽国牵扯进来,任何一桩事都不是区区一个边臣就能承担得了的。何况还是兴灵?宣抚使兼枢密使的吕惠卿都承担不了!这是宋辽百年纷争中,最大的一次收获。但也是彻底破弃了延续近八十年的澶渊之盟的举动。同意吕惠卿担任宣抚使的东西两府,谁也逃不掉这个责任。

从某种程度上说,眼下这个结果都是两府放任造成的。种谔独走不假,但吕惠卿既然为种谔收拾手尾,以枢密使兼宣抚使的身份将责任担了起来,那么朝廷这边也要为任命吕惠卿为宣抚使的这一件事承担责任。

不过这个责任在一开始并没有多少人放在心上,毕竟吕惠卿是兼任宣抚使的枢密使,一般的情况下,最多也只是将他罢职而已。但夺占了兴灵之后,可就是量变引起质变了。

“怎么办?”章敦回头问道。

“还能怎么办?”薛向笑着反问,却是苦笑居多,“前两天不是已经决定好了吗?”

种谔仰头哈哈哈大笑了起来,可笑声到了最后,也化为唇角边的一抹无奈。

处置还是褒奖,朝廷的处断在两天前就已经决定了。

两府之前也曾为此争执了两日,但当文彦博的弹劾送抵通进银台司之后,立刻在一刻钟之内达成了共识。

赏功。

而且是重赏。

原因无他,只有四个字——党同伐异。

如果没有旧党的掺和,两府之中,台上台下都少不了给吕惠卿下眼药。可现在洛阳的奏疏一到,那就必须要保吕惠卿了。

就像当年王安石明知道市易法弄出了大乱子,却不得不硬保吕嘉问和市易法。长河溃堤,坏于蚁穴,如果认同旧党对吕惠卿的弹劾,接下来两府之中的大半宰执都要一股脑的被牵连进来。

章敦利利索索的回到桌案边:“河北今天又奏表来吗?皇后肯定要问了。”

“郭逵的有一封,真定府也有一封。沧州、雄州都有。”

南京道的辽军已经有了异动,河北这几天,边境上的各大军州自然是连番上书报急。在天下四百军州的表章中,占了三成还多。

“广信军的呢?”章敦依稀记得李信也写了奏章上来,在桌上翻找着,“遂城可是辽人南下的必经之路。”

“好像昨夜就递进去了,皇后急着要。”薛向也在收拾着桌面,将来自河北的还没处理的奏表匆匆翻阅一遍,力争在被招入宫中之前,有个大概的印象。

“河北决不能出事!”他边翻看,边说着。像是说给章敦听,更像是在警告自己。

“这是自然!”章敦握着一份来自保州奏折,笑容冷然,“文宽夫、吕晦叔不正等着看我们的好戏吗?”

一旦辽军大举南犯,洛阳旧党的第一件事绝不会是同舟共济,而是借其声势将新党组成的两府都赶下台。

只要在台上,就必须为所有的事负责。内政外交,政事军事,乃至寒暑旱涝蝗瘟,都得由天子、宰相们承担起责任来。

至于在台下的大臣们,只要动动嘴皮子,什么事都不需要做,什么责任都不需要承担。写奏章指责不在话下,直接煽动人心,破坏当权者的名声,更是老套而又必然会用的手段。

种谔出乎意料的夺占兴灵,让两府终究还是陷入了被动。不趁这个机会下手,还等到何时?

“张枢密,薛副枢,皇后有旨,请两位枢密即刻入宫。”一名中使意料之中的来到了枢密院。

章敦和薛向相顾颔首,一同起身而行。
