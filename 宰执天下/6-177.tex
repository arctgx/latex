\section{第19章 登朝惟愿博轩冕(下)}

李诫刚刚从睡梦中醒来。

坐在宽敞的车厢中,钢铁制成的车轮不断咔哒咔哒的响着。

由于钢铁质地的轨道,会随着四季季节的变化,而改变长度,这使得一条条铁轨之间,都必须保持一定的长度,以防在夏日艳阳照射下,两条铁轨相互挤压,最后让轨道变形。

重复而单调的声响,不停地敲打着双耳,但只要习惯之后,便会不知不觉的忘掉这样的响声。

可铁轨却很容易在这样的撞击中损坏,在过去,经常是一段轨道的两端被碾压出裂痕甚至破损的缺口,使得方城山的铁路不停地换铁轨。幸好如今钢铁的质量越来越好了,铁轨的质地也越发的坚硬,铁路的维修费用这才降了下来。

睁眼望着窗外,窗帘已经被拉起,穿过透明的玻璃窗,一座房屋便映入眼帘。但下一刻,又刷的一下,离开了视野,被远远的抛到脑后。

车窗外的风景不停地变换,从房屋到田地,从田地到道路,一座座的房舍,一株株树木,从窗前接连不断的闪过。

当看见一座如彩虹一般拱起的桥梁自车顶上跨过,李诫心中一动。这是汴水京畿段最常见的虹桥,现在跨越铁路,也依然使用虹桥。

睡在车上,只能感受到轻微的摇晃,李诫也睡得很沉,不知不觉间,就已经进入了开封地界。

舒展了一下腰身,李诫看向对面的窗口。

对面窗外,高耸的堤坝连绵不绝,灰土黄的颜色,一直遮挡着视野。

从李诫自泗州上车开始,一路上,车辆前进方向右侧的窗口。一直便是黄土累积而起的大堤。几百里了,也不见发生半点变化。

这就是京泗铁路。

耗用了以百万贯来计算的金钱,以百万计的钢铁和木材,难及计算的人力,沿途诸州各县全力动员,历经四载方才修成。同时这也是在建的三条干线中,第一条全线贯通的铁路。

刚刚建成不久的京泗铁路,沿途市镇百余,车站总共二十三处,上上下下里里外外都凝聚着李诫的心血。

开通后的第一次行车,李诫便从泗州上车,准备一路抵达京师。

京泗铁路全长近八百五十里,完全沿着汴河修造,自始至终与河道平行的这条铁路轨道,将是皇宋未来的命脉,按照韩冈曾经使用过的医学上的比喻,就是连接心脏的主动脉,一旦有失,便是性命堪忧,神仙难救。

这样的比喻并不为过。

在襄汉漕运尚未打通,京泗铁路更不见踪影的一百多年里,若是没有汴水上的纲运,将南方的粮食不断运送到拥有百万军民的京师,皇宋的都城,根本无法支撑下来。整整一百万张嘴,不是一年两熟、亩产三五石的江南美田,如何养得起?

那时候,皇宋只有这么一条主动脉,所以举国上下都对汴水战战兢兢,每年都要差人去整修汴水沿途的堤坝和水闸。而且为了维持纲运通道的稳定,汴水两岸常年配备一支厢军,专门用来清理河道中淤积的泥沙,同时检查大坝是否损坏,投入的资金都是一个让人瞠目结舌的数字。

而这样的投入,并没有改变汴水逐年升高的势头,更让朝廷不得不年复一年的投入更多的资金去保证汴水畅通,以及堤防无损。

如今汴水已经与黄河一样,河床不断抬高,大堤也一年年的增长,也让人越来越担心汴水会不会哪一天彻底淤塞起来。但这依然京师上下不得不付出的代价,

幸而如今有了铁路轨道。

先是襄汉漕渠,因为有了方城山轨道而得以畅通,如今又有了贯通淮泗和京师的京泗铁路。

这就相当于在一条主动脉之外,又增加了两条主动脉。这样一来,即便是汴水断流,也照样不会影响到东京军民的生计,大宋朝廷也能够始终保持稳定。

按照最近修改的设计,日后来自南方的货物和旅客,都将会通过京泗铁路来运输。至于汴河,则只负责输送纲粮和一些大宗又不需要赶时间的南方货物。

在李诫看来,在京泗铁路开通之后,即使是中断了汴水的航运,只要能够及时调整,将朝廷过去灌注在汴水上的心力放一半到京泗铁路上,南北纲运也不会有多大的影响。



不过上面的宰辅们都觉得,还是将铁路和运河都拿在手里更安心一点。铁路可以用来赚钱,而汴水的运力,则就是维持京师的安稳。

也不知道这样的改变,会不会让那些水耗子们得意。李诫知道,自从掌控纲运最为得力的薛向因大逆案而被发配南疆,继而殒身于彼,汴水上纲运便成了贪官污吏嘴下的肥肉,这几年抓出的水耗子一窝接着一窝,但不论朝廷杀了多少人,还是灭不尽人心的贪婪。

不过李诫有一次与方兴喝酒,曾听他提起过,之所以韩冈不去整顿六路发运司,只是因为他想要在汴水上看到一个混乱的纲运体系,好用来逼着朝廷去修建京泗铁路。

不论是真是假,现在朝廷上下的确是对六路发运司颇有微词,而京泗铁路能够如此顺利的得到朝廷的全力支持。

一块石碑从眼前闪过。

陈留二字,刻在了石头上,朱红色的正楷烙在了李诫的眼底。

陈留到了。

这也是抵达开封前的最后一站。

但早在看到刻着陈留二字的石碑之前,李诫就已经通过鳞次栉比的房屋,以及几处重要的建筑,分辨出了此地究竟是何处。

车在站台上停稳,推开车门,李诫跳下了车,与迎上来的官吏一一招呼过,他便向前面走过去。

全线贯通的初次运行,这第一列车辆总共八节车厢。李诫独自占了最后面的一节车厢,甚至在里面睡了一觉,不过这个车厢也仅仅是普通的客运车厢。

李诫没有选择官车,他打算体验一下普通旅客长时间乘坐的感觉,而官车就太舒服了。

虽然官车车厢的大小,与普通车厢别无二致,同样不算宽敞。但每一节官车车厢都分做了内外两重,靠前的半截是内间,有着松软的床铺和精致的摆设,甚至还装了玻璃油灯,牢牢的卡在车厢壁板的凹槽内,燃烧后的油烟能通过事先安装的管子通到车厢外。靠后的半截是外间,夜里仆役打地铺,白天则可以见客、读书,而上下车都得从外间走。

韩冈曾经提议过打造一种新式的车厢,加宽车厢宽度,同时在里面安shang床铺,在车厢的一侧留下一条通道,可以连接前后,同时方便上下车。但那样的车厢太难制造了。加宽车厢宽度不算很难,可前后有门,前后车厢连贯相通,这虽是好想法,可惜现在还做不到。

所以如今想要到前面或是后面的车厢,要么等下车后再去,要么就是从车顶上走过去。

李诫自不能从车顶上走路,他走到中间的位置,在车门外通报了一声,便被迎了进去。

这里是全车唯一一节官车,布置和陈设都不是后面的车厢能比,日后将会供给上京的官员使用。

车厢内,五六人,但只有一人还在内间坐着,直到李诫走进来后,方才出来迎接。

李诫上前行礼:“李诫拜见端明。”

沈括自开封知府任上调职,便以端明殿学士的身份出京,都大提举轨道工役等事。

迄今为止,已有三年之久。

这三年间,沈括虽然不能说是天南地北的跑,可河北、河东、京畿等地的轨道工地,他也都跑了个遍。

眼下的几条轨道,在同时兴修的同时,还能够保证速度和质量,至少有一半是沈括的功劳。

沈括伸手扶住李诫:“说过多少次了,明仲你勿须多礼,坐下来说话。”

“礼不可废。”李诫坚持行了礼,方才依言坐下。

日以继夜的劳作,往来千里的奔波,李诫的外表,看起来比他的实际年纪还要大了不少。

沈括打量着李诫,感叹道:“转眼就到京城了。”

“最多一个时辰。”李诫道,“在车上也快三天了,再有一个时辰终于是结束了。”

“还不到三天!”

沈括比了一个手势,着重强调着花费的时间。

从泗州到,只用了两天多的时间,这的确是个让人瞠目结舌的速度。普通人出行,能达到日行三百里、四百里的高速,这是十年之前无论如何都不敢相信的,这本因是急脚递才能拥有的速度。

“主要因为是空车。此外,拉车的都是健马,车夫也没有顾惜马力。”

日后正常运营,有轨马车的速度可能会降到一半。不过四天走完八百里路,这也已经是过去难以想象的速度了。

从金陵至开封,总共二十二程,按照朝廷制定的驿传速度,二十二天才能从长江南岸的金陵抵达京师。而在京泗铁路修成之后,二十二天的时间能减去一半还多。

“这一回回去,当可以说一句幸不辱命了。”

李诫拱了拱手,“恭喜端明。”

“当是同喜才是。明仲你的功劳,可不是等闲可比。”

李诫自从当年被韩冈征召入幕,工作和官职便一直围绕着轨道。如今作为沈括身边最为得力的助手,李诫为此也付出了大量的汗水和时间,一直站在第一线督造。功绩不必说,光是时间,就投入得不比任何人少。

“非为此事。而是以端明之功,一张清凉伞不在话下。”

沈括谦退的笑了两声,“就是进了政事堂,也还是主管工役营造,跟现在也没多少区别。何况还没有开始选,现在说也还太早。”

“如今有两位相公在,哪里还会有什么意外?”

沈括只是摇头。

但韩冈已经明确说要支持沈括,而苏颂与韩冈也同进同退,

之前两次廷推,韩冈对沈括的支持力度并不够,甚至没有去说服太后。

沈括并不指望太后能对自己另眼相看,没有韩冈的关说,太后根本不会提起朱笔,在自己的名字上画圈。

但现在韩冈已经明确说会在太后面前为之美言,而且进入两府的人选又多了一个。有两个名额,又有韩冈在太后那边说项,再加上这一回适时的京泗铁路通车,沈括相信自己这一次肯定能够得到梦寐已久的那一张清凉伞。

恍惚间,沈括好像做了个美梦,清凉伞张挂在头顶,不论刮风下雨,都牢牢护在左右。

但当他清醒过来的时候,正好看见李诫望着外面。

“明仲……”

沈括正想说话,就见李诫指着窗外。

“已经到京师了。”

