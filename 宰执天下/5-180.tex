\section{第20章 土中骨石千载迷(九)}

十天。

东京距洛阳四百五十里。东京城中的新闻,传到西京洛阳,一般要五天时间。一般的奏章和公文传递,从洛阳送到京城,也同样要五天的时间。

而司马光请求朝廷派遣专人保护并发掘殷墟,以明先王之文的奏章,出现在通进银台司中,距离韩冈和苏颂公布有关殷墟和甲骨文的消息,只过去了十天。

公文传递的时间是不可能缩减的,半天都不可能。不是军情,不可能动用马递和急脚递,普通的步递铺兵,绝不会闲着没事的多走一站。

而无论如何,从东京将消息传往洛阳,速度再快也不会缩减到三天以下。

两天,一天,甚至可能只有一个晚上,让司马光来写奏章。这个时间对于一篇几千字的奏章,可以说是很少了,可司马光还是给写了出来了。

不仅是司马光,文彦博、富弼、范镇等洛阳老臣也都写了奏章。不过富弼的奏章据说只是一封谢上表,感谢赵顼前段时间赐下的药物,但也有说法,是跟文彦博和司马光等人一样,都想趁机踩上王安石一脚。

能惊动这一干人等,也在韩冈的意料之中。毕竟机会难得,毕竟在洛阳憋屈了很多年了。

新学是官学,把持着儒生们进入官场的权力。短时间内是不可能将之从台上推下来。除了天子支持新学以外,另一方面,不论是气学还是程门道学,都还没有一个如同《三经新义》一般系统化的儒门经典的新注解。

但对韩冈来说,纵然一时间不可能动摇新学把持官学的地位,也决不能让新学将儒门道统控制在手中。一旦给新学彻底站稳脚跟。百十年内,韩冈估计大概也只有痛失半壁江山那般剧烈的动荡,才能动摇得了新学的权威地位了。

“终究不是学术之争啊。”坐在家中的小院中,韩冈拈着一片枯黄的梧桐落叶,已是深秋近冬的时节了。

虽然也是道统之争,但更多的还是由政治决定。学术和政治所占的份量,有着一个指头和九个指头的巨大差别。

新法、新学、新党,是一体的,打击新学,就是打击新法和新党。赵顼无意改变新法,要维护现在稳定的局面,这样一来,也就是不会允许有人动摇新学的地位。不过同样的道理,有机会通过打击新学,连带着打击到新法和新党,旧党中人不会放过这么好的机会。

“官人,”周南在桌边剥着板栗,用剪刀将外壳剪开,将金黄色的栗子一颗颗的放到韩冈手边,“殷墟的事,官人到底打算做到哪一步?”

严素心和韩云娘正亲手为家里的几个孩子缝制冬衣,虽然完全没有必要,但也是平日里打发时间的办法。周南这一问,手上的针线活就停了。

王旖坐在炕上看着书,看都不看韩冈这边,但翻书的动作在听到这句话后,也一下停了下来。

王旖虽然不是在跟韩冈怄气,但心情不好已经有好些天了,这事连韩冈都没办法。

韩冈瞥了妻子一眼,“最好是将千里镜的禁令撤销。”

看了眼妻妾们一下变得惊讶起来的表情,他又笑着道:“当然,这是不可能的,天子和朝廷的脸面还是得要顾及……至少三五年之内不可能。而且就算是三五年之后,想要解禁,也得要有个合适的借口。比如辽国已经可以自产千里镜什么的。”

毕竟千里镜不是可以用来厮杀的武器,民间拥有了硬弩、甲胄和长杆兵器,就有用来编制军队的可能,光拿着千里镜,无论如何都不可能用来上阵厮杀。至于观察天象,只要不涉及谶纬,让朝廷睁一只眼闭一只眼也就混过去了。

但周南似乎是误会了,惊得掩住嘴:“官人,你是要将千里镜传给辽人?!”

韩冈愣了一下才明白过来,摇头笑道:“别误会,也不要小瞧辽人,更休提辽国现在掌权的还是耶律乙辛。依靠飞船,他已经占尽了便宜。在种痘法上,也享受到了足够的好处。雪橇车在辽国运用得比大宋更广,他又怎么可能会放弃仿造千里镜?”

“那样岂不是还要等很久?”严素心问道。

“是啊,所以禁令的事,只能先认命了,眼下为夫只求朝廷接下来不要干预太多。”韩冈将栗子一个个丢进嘴里,“如果仅止于学术,我是不怕任何人的,气学也不输于任何一家学派。”

韩冈的豪言,让王旖更行沉默,周南像是要转换一下气氛,问韩冈道:“官人想要天子怎么做?”

“这件事还是让天子去考虑吧,做臣子的可不能越俎代庖。”韩冈笑道,“只要愿意去发掘殷墟就可以了。”

虽是这么说,但他解开殷墟谜团,以及司马光和文彦博等人的奏章,其实都是没将天子太过放在眼里的表现。否则就该学习王珪,皇帝说什么,那就是什么。

所有的能经常面对天子的朝臣,都知道所谓皇帝都只是普通人而已,只是敬畏皇帝所代表的那份生杀予夺的权力,隐藏起来的悖逆思想仅仅是程度深浅不同罢了。

“这样一来,殷墟便是要毁于一旦了。”王旖放下了书,“官人可知晓,天下的盗墓贼决不会放过殷墟。”

“殷墟那可是一座都城,摸金校尉想要让一座都城毁于一旦,可得用上几十上百年的时间。”

盗墓贼的问题的确存在,但韩冈不会自己出面去催促天子早下决断。他之前已经做得够多了,继续出手,可是会过犹不及,甚至引来天子的逆反心理。

韩琦家就在安阳,安阳的土地有一多半是在韩家名下,外人想去盗墓,也得没那么容易。运气不好,就会被当地的保甲给捉住。不过当地的百姓,就地挖掘,然后将文物卖给外来的古董贩子,这样的事后世便禁绝不了,这个时代更是不用指望。

对于考古,韩冈只知道一丁点连粗浅都还够不上的常识,比如那种如同九宫格一般的挖掘现场,比如按时间排列的地层,还有通过残存的遗迹结构,可以去推测当时的社会制度、建筑制度。但细节一概欠奉。

但韩冈更清楚,考古学对遗迹发掘时的保护措施,是在不断实践中逐渐进步的。要让考古学真正称为一门有深度的学科,而不是由人随意挖掘,只去关心和研究挖出来的器物,需要大量的现场积累。而这一次的殷墟,如果朝廷能组织发掘,应当就能总结出大量考古学现场发掘的经验来,也能吸引大量研究金石的儒生。

只不过乱世黄金盛世古董,能玩金石的,都不是普通的儒生,全都是有钱有闲的主儿。考古学这东西,也只有和平年代才能让人静心下来研究,换作是乱世,生存和生产才是排在最前的重要课题。

这个世代也许还不算乱世,但若是继续发展下去,多半还是避免不了陷入乱世。所以他希望能尽快做到更大的影响。

当然,这些也是自我开脱的话。从本心上,韩冈重人而轻物。一边是殷商古迹,一边是普通百姓,两者放在一起让韩冈选择拯救哪一方,韩冈绝对是选人而不是选物。

不过韩冈完全可以说,他是学了孔老夫子的做法,仿效圣人而为。将三代留存下来的资料,删减到百篇,编纂成《尚书》。从商、周王室,到诸侯国,再到民间,搜集而来的数以千的诗歌,经过删修,就只剩下三百篇,编纂成《诗经》。还有《春秋》,这部鲁国国史,也是被孔子大加删改,以求微言大义,符合儒门之旨。大量抛弃和毁坏原始资料,修改成合乎己意的文字,都是孔子做的。

前生受到的教育,以及来到这个时代后,充斥于世间每一个角落,每日都能感受到的中央之国的自负,让韩冈绝对无法容忍那一个世界的历史重演。

韩冈自视是很高的,至少不缺乏改变未来的使命感——尽管这个使命或许可能并不存在。在韩冈心中,过去不是不重要,但远不及现在和未来重要。在压到一切的大义面前,区区一个殷墟的牺牲,韩冈觉得很值得,这个交换实在太便宜了。如果牺牲的是人的生命,韩冈免不了要犹豫再三,但换作是古代的遗迹,他却是一点心理障碍都没有。

若是这个世界的未来还是在重复着旧事,那么将先人的遗产继续留在地底还有什么意义?给千年后的域外蛮夷妆点自家的书房?还是连同一个伟大文明的耻辱一并陈列在博物馆中?

敦煌也拿下来了,但韩冈却没去动敦煌莫高窟里所藏珍宝的主意。有意义的牺牲,和无意义的浪费,他分得很清楚。

虽然历史已经确定改变,就算女真人能崛起,也不可能复制旧日的历史,但谁也不能保证不会出现其他的问题。要想确保未来能走向韩冈所期望的方向,那么就必须尽快让他能够发挥出自己的实力。

韩冈没有时间耽搁,他缺乏的正是时间。

