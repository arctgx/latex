\section{第39章 欲雨还晴咨明辅(11)}

吕嘉问果然还是来了。

章惇早有所料,今天在朝堂上都说起吕嘉问的去留,王安石硬是将韩冈的人选给踢了回去。今晚不找恩主拜谢一番,还等什么时候?

在吕嘉问进来前,章惇就起身告辞。虽然还有些交情,但他来拜访王安石是为了两府的人事安排,不是吕嘉问可以插进来的。现在见面,实在不太方便。

“子厚,你还是留一下。”王安石却出言留人,很是诚恳的说着,“望之有几桩事,都需要你那边做些配合。”

章惇苦笑,只得坐了下来。王安石方才当面说吕嘉问来了,就是要把自己拖进来,现在果然是走不掉了。

吕嘉问被王旁引进了书房,见到章惇也没惊讶,显然是王旁事先说了。

等吕嘉问坐定,寒暄了几句,章惇就问道,“望之,不知犒赏三军你打算怎么办?”

章惇是枢密使,最关心的也就是三军犒赏的问题。赏赐越是丰厚,百官三军就越是安稳。而且这件事要尽快,迟了就会乱了,官吏们还好说,但那群赤佬,可是不知道什么叫做相忍为国。

京营从来就不老实,现在才打过仗,更是骄悍了十倍。韩冈出面都不一定能压下他们,更别说其他人了。就在月前,因为犒赏事已经闹了一通。一番好杀之后,虽然闹事的几个指挥已经给压下去,可人心压不下去,若是给了他们机会,说不定会闹到不可收拾的地步。

表面上虽看不出来,但章惇的心里的确已是着急上火。吕嘉问不能给个让他满意的回覆,他当场就能翻脸。

“开内藏库。”三司使吕嘉问说得理直气壮,“天子践位,不开内库,难道还开国库不成?”

要买好百官三军的是皇帝,当然要掏自家荷包。吕嘉问打算直接向皇后摊手要钱,不过几百万贯的事,内藏诸库把老底掏出来肯定能支撑得起。

“这就是望之你的办法?”章惇半眯起眼,不冷不热的问道。

“太上皇后深明大义,只要与太上皇后辩说分明,必然不会推辞。”

“太上皇为了收复幽云,辛辛苦苦攒下的那点本钱全都要给捞空了。”

‘五季失图,猃狁孔炽;艺祖造邦,思有惩艾。爰设内府,基以募士;曾孙保之,敢忘厥志’。这三十二库,马上就要开始跑耗子了。

“聚天下之财,就要为天下之用。难道犒赏百官三军,为天子贺,难道不是用在正途上吗?国不安,何能御外侮?子厚,我向来佩服你的果决,此时此刻,可是犹豫不定的时候?”

天子既已践位,犒赏就得发下去,已经没有太多时间了,实在是拖不得。就像群臣参加正旦大朝会,回头就能拎回胙肉,没说要等几天才发下去。那样的话,下面穷困点的三班官,可都要清汤寡水的过年了。两府宰执,在这件事上都不可能置身事外。

吕嘉问抓住了这一点,根本就没有一点担心。

“那我还不如问问玉昆那边有什么办法。”

“不会有办法的。这不是变戏法,钱粮变不出来,韩玉昆来了,也只能伸手从内藏库中要钱。要点脸的,打个借条,不要脸的话,就直接要了。”吕嘉问冲章惇笑了笑,笑容甚至有些阴寒,“我现在,可是已经把脸皮舍了不要了。”

章惇深深的看了吕嘉问一眼,忽的一叹气。人身上下,最贵重的就是这张脸,吕嘉问不要脸了,那这件事还真的就能解决了。

他摇摇头,“这么办就够了吗?折五钱呢,铁钱呢,不仅仅是一桩事啊。”

韩冈在《钱源》中说得不错,钱币本质是在于一个信字。有了信用,纸片……不,甚至空口白话都是钱。什么叫做一诺千金,就是在说这个‘信’字。

只要抓住了重点,维持住朝廷信用的手段也容易。但吕嘉问能抓得住吗?他的信用,可远远比不上韩冈。

韩冈一篇钱源论,让折五钱立刻能当五文用了,但当他受阻于朝堂,折五钱就又跌回去了。这一跌一涨之间,正证明了韩冈的信用,在京中百万军民中,到底是个什么样的等级。

吕嘉问跟他怎么比?天差地远。就是当今两府宰执加起来,也不一定能比得过他。

“一件件来。犒赏事解决了,铜铁钱和折五钱也不难了。今年秋税,陕西是铜铁钱各半征收,京中则半数折五钱。这件事,就需要政事堂的配合。”

“伪钱怎么办?”

“只要重量不差太多,可以一并收下来。这个亏,三司认了。到时候,多铸些折五钱也就能抵得过了。”

“认下的是朝廷吧。”章惇叹了一声。但他也不能否认,这的确是个能解决问题的办法。

吕嘉问给出的办法,早已通行于世,也是韩冈的意见,只是之前执行不严。尤其是总有歼猾之徒用假币来冒充折五钱,使得下面的税吏都不肯收取——他们将税金缴上去后,被查出伪币,都是要自掏腰包补上的——这样当然会造成折五钱信用贬值,直到百姓不肯使用。

道理其实人人皆知,关键的还是执行。但只要朝廷肯吃这个亏,将不太过分的民间伪钱都给认下来,还是能够保证折五钱的信用。至于中间亏损的部分,保证了信用之后,可以通过增发来弥补。

但那个时候最苦的肯定是政事堂。

那些收上来的伪币,到底怎么处理,绝对是个大麻烦。

肯定是不能对外用,否则朝廷信用怎么办?可是要挑拣出来,就不知要消耗多少人工。说不定到时候就只能一股脑的化成铜水,重新再铸新钱。其中的火耗,能将铸币的钱息,一股脑的都给消耗掉。

整件事绝不会像吕嘉问说得那么简单。

吕嘉问看得出章惇心中所想,毕竟这其中的问题太大了,只要是明眼人,不可能看不出来。

“子厚放心,还有另一条手段,”吕嘉问笑道,“嘉问虽愚,还不至于如此糊涂。”

“什么办法?”

“发行大钱,以异色分铸。当二、折五,折十,折二十,不同币值,不同的质地。色泽不一,伪币就别想有存身之地。”

章惇的眼睛瞪了起来,看了看王安石,又转回来看吕嘉问:“这不是韩玉昆的提议?!”

“正是!”吕嘉问点头。

比起这几天来,为帝位而费尽心神的两府宰执,吕嘉问的心思则全都放在了如何保住自己的位置上。韩冈当初给向皇后的建议,他费尽心思的一五一十打听清楚,然后在三司衙门中,找来一干得力的亲信关起门来制定实行的计划。

不就是用不同材质铸造新币嘛,这个的确是个好主意。如此一来,那些贼人融钱改铸的老手段就行不通了。青铜质的一文钱是一个颜色,黄铜质的十文钱又是另一种颜色,就算十文钱的含铜量远不及一文钱的十倍,可融掉的一文钱重铸起来,能变成另一种颜色的十文钱吗?

“玉昆之材,远高于嘉问,《钱源》一论,旷古绝伦。义利之辨,由此而决。既然韩玉昆有良策,嘉问哪有不用的道理。”吕嘉问冲着章惇微微笑道,“国事为重,纵然会受世人耻笑,但嘉问受之,甘之如饴。”

“国事为重,一时荣辱只是闲事。望之能有这份心思,实在是难能可贵。”始终沉默的王安石,为吕嘉问辩解,“玉昆以为望之不能胜任三司一职,只要望之将此事办好,天下之疑不也就烟消云散了吗?”

吕嘉问轻轻点头,道:“为朝廷办事,也不能讲究那么多了。如果玉昆不忿,嘉问登门负荆请罪也可。”

只要把事情办好,管他是谁的意见,脸皮这东西,有官位管用吗?

吕嘉问当年好端端,被吕公弼大骂是家贼,那是为什么?

还不是因为他在吕家里面,既不是吕公著那一房、也不是吕公弼那一房,总是被同族兄弟给欺辱。所以一等吕公弼准备狙击王安石变法,他便毫不犹豫的偷出吕公弼的奏章草稿去找王安石。拿着吕公弼的奏章草稿,知道了吕公弼准备用什么名目来编排新法,在御前,王安石将吕公弼打了个丢盔弃甲。之后得知罪魁祸首的吕公弼将侄孙赶出了家门,并骂其是家贼。

时至今曰,这旧曰恩怨差不多快到了结束的时候。现如今,吕公弼死了,吕公著完了,只要吕嘉问能够跨进两府——就是爬进去都行——族中那些废物,就要过来舔自己的脚。失去了祖辈的护持,他们就是些废物!

吕嘉问等这一天等了很久了,只要再有一两次机会,就能身登两府之位,让吕公著在死前,亲眼看见他的儿孙过来奉承自己,捧自己的靴子。

到那时,这积郁在心中多年的旧曰恩怨,才会有一个终结。

吕嘉问望向章惇的眼神毫不动摇,三司使的位置,他是绝对不会让出去的!

