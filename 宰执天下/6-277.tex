\section{第39章 帝都先温春常早(一)}

东边叮叮当当的在响,一声声的传进韩冈的书房。

韩冈放下笔,无奈的看着面前只有十几个字的白纸,无奈的叹了一声。

家里正将空闲的东跨院给整理了出来,待韩钲成亲后住进去。韩钲原来在偏院的屋子,也收拾了起来,等下半年老四的生日过了给他住进去。

外院的书房本就偏东,离东跨院近了点。鸡犬相闻,那边修屋子,这边连木匠的咳嗽声都能听到。

韩冈唤了人进来把书房里的一干书籍和资料都收拾一下,准备换到后花园的小楼去写文章。

书房里面收拾东西,韩冈走了出来,抬头向东望去。两名匠人正跨在东院正屋的屋顶上方,一片片的铺着瓦片。这一次要全都更换, ( 顶点小说手机版  ) 早上的时候,才刚起了个头,现在吃过午饭不久,看着就快要铺好了。

真是一眨眼的功夫,儿女都要成家了。每次家里看着里里外外的准备,韩冈就忍不住在心里感慨一番。

早些年还是开国以来最年轻的宰执,如今已不再适合以年轻标榜了。

即使按照一般的标准来说,他这个年纪的朝臣,依然能被说成是新进,但从这个时代的平均年龄来看,韩冈已经接近平均寿命了。

而且韩冈也无意标榜年轻。身居宰辅前列,老成二字是必须拥有的标签。在杂剧中演大官的,无不是带着一把大胡子,这就是民间最朴素的认识。

当然,这个世上,有太多年纪老大,还依然轻佻不晓事的人。

润州知州杨绘——昨日韩冈在任免诏书上签名画押后,已经是前知州——当年在琼林苑上不顾尊卑,攻击韩冈一个小小进士,反而丢人现眼,不久之后又因为行事不谨,与宗室女近于亵乱,又遭到贬斥。

要是依照他的资历和早年的境遇,现在就可能与韩冈等人并肩而立,可惜性格决定命运,润州事后,杨绘责授润州团练副使,本州安置。

团练使是军中贵官,即使是遥郡团练使,也是军功煊赫的将领才能有的加衔,但加了副字之后,就是安置被贬责的官员的特有职位了。

但这一处置方案中,最为刻薄的一手,不是将杨绘左迁至润州团练副使的位置上,而是本州安置四个字。

润州团练副使的本州,自然就是润州。

润州明教之乱,州治丹徒县百姓伤亡惨重。纵使明教承担了血债,并全数偿还,但官府方面,总得有个官员出来承担一部分不可推卸的责任。

不能在事前发现贼人谋叛,不能在贼势刚发的时候扼杀在襁褓中,治郡不谨,致使明教能蛊惑人心。这一切,都是丹徒县、润州,乃至两浙路相应官员的责任。

景诚以及当地州县官通过平定贼乱的功劳,将自己的责任给洗清。让贼人一击得手,致使乱事扩大的丹徒县尉,用性命换来了朝廷的抚恤,以及百姓的谅解。两浙路的监司官,距离百姓太远,罚俸和延长磨勘时间的惩罚,已经可以揭过此事。

最主要的罪责自然还要落到事变前懵然无知,事变中躲藏不出,事变后还没出面安抚百姓的润州知州的头上。

润州也许还有人会觉得朝廷不能无过,但政事堂将杨绘丢过去后,便可谓是怨有所归,还残留下来的怨恨,就都落向杨绘的头上。

受贬责的官员只能拿到一半的俸禄,以韩冈听到的一些消息,杨绘在润州市面上,能不能买到吃的,那还当真存在疑问。

生老病死,本是常事。这些老骨头,于国于民,有百害而无一利,死了也算是好事。

放下杨绘,韩冈又想起宗泽。

宗泽这一回在润州,看到的那些事,似乎是动摇了他的信念。对韩冈所描述的未来,不再抱有坚定的信心。

这让韩冈有些挂心。如果是别人倒也罢了,宗泽的才智心性都是韩冈很欣赏的,而且又不缺决断,日后必为国之栋梁——这一点,在另一个历史中已经得到了证明。

是不是让他去辅佐沈括的工作,被现实所动摇的信念,最好还是由现实重新确立。

沈括这些年工作的成果,世人皆是历历在目。铁路给社会带来的变化,远远超过了修建铁路时,所付出的那些成本。

不论是谁,如果能更深入一点的去观察铁路对天下的影响,必然会明白谁才把握住了世界发展的流向。

不论是另一个世纪的历史书上,还是在此时的现实中,都不乏大宋商业发达的评述。

但在实际上,所谓的发达只是相对的。在铁路开始贯通大宋南北,真正起到大动脉的作用之后,大宋的商业,才真正发达了起来。

世人对产品的需求,一直被恶劣的交通情况所压制。直到有了铁路之后,他们的需求才爆发出来。

棉布在全国各地的热销,来自于方便的交通,降低了运输上的成本,相应也降低了各地的售价。

而丝绸价格的下降,也同样因为交通更加通畅。蜀锦的贵重,一方面来自其独特的美感和质地,但另一方面,也来自于难于上青天的蜀道。

而丝厂的工厂主们,便是因为需要通过大量的倾销来拓展市场,又希望能够从倾销中得到更多的利润,才穷凶极恶的对工人尽心盘剥。在工人爆发出了他们的力量之后,没有哪位工厂主会吝啬一小部分利润,而愿意冒着自家工厂被焚烧的风险。

工人们的待遇,会得到一定程度上的提高。看到这一点,再看到,工业发展给国家带来的变化,韩冈相信,宗泽会自己分辨回到过去还是继续发展,哪个对大宋更加有利。

不管怎么说,历史已经走上了韩冈所希望的轨道,韩冈有些得意的想着……

砰的一声脆响,突然从韩冈身后的书房中传来。

韩冈的思路被打断,回头进屋,却见三名仆人都低头,看着书房中的满地玻璃碎片,三个人全都愣住了。

韩冈进门的动静,惊动了三人。领头的仆人指着其中一人,“岑三,你是怎么做事的,那是相公最喜欢的玻璃盏!”

最喜欢的……

韩冈转头去看百宝阁,那件玻璃器物的确不见了。

在韩冈的书房中,没有特别贵重的古董,但大小器物,也都能算得上是珍贵。

现在打破的一个玻璃盏,从下到上,自蔚蓝逐步转为艳紫,色彩瑰丽,质地又晶莹剔透,毫无瑕疵,宛如真正水晶。

这玻璃盏出自陇西韩家自有的玻璃窑,却是意外中的产物,到现在为止,还没有研究出来到底是什么样的材料,导致了颜色上的变化。也就是说,是世上独一无二。

岑三已是面无人色,双腿一软,就跪了下来,“相公,小人万死……”

“好了,你这样吵得慌,不是什么大事,谁没个失手的时候?”韩冈让领头的仆人不要再责骂,又对岑三道,“继续收拾,注意手脚要轻些。”

岑三惶恐的抬头,甚至不能相信韩冈的宽大:“相公……”

“沙子做得器物,不值什么。你们先把玻璃收拾一下,小心别给划伤了手。”

韩冈说着就转头出门,但立刻就又回来,从书桌上取了个小屏风,丢下一句‘你们继续’。

三位仆人面面相觑,领头的仆人咳了一声,“相公宽宏大量,你们可别当成了习惯,都给打起精神,别再跌跌撞撞的。”

岑三一语不发,低下头去收拾东西。

“哥哥,相公拿出去的那是什么宝贝啊?”另一个仆人小声问着。

韩冈出去后又进来,是担心他们粗手笨脚,再弄坏了,将那件色泽稀世罕见的玻璃盏都轻轻放过,被韩冈拿出去的那件屏风,该不会是什么稀世珍宝?

领头的仆人瞪了他一眼,“你什么眼神,那是去年相公寿诞,大娘子亲手绣的礼物!”

韩冈在院子里,低头看着刚刚拿出来的屏风,庆幸不是这屏风遭了劫。

要是当真给摔坏了,以自家的宝贝女儿的性子,怕是会赶在出嫁之前,再绣出一幅来。断断续续近一年才绣成的屏风,韩冈哪里舍得女儿熬夜去重绣。

在六岁开蒙读书的同时,韩锳就开始学习女红。到了十二岁,更是被家里督促着缝制嫁妆。

绝大多数官宦人家的女儿在出嫁之前,都少不了这一项任务,只有工作量多少的区别。

在女红上,韩锳没有多少天赋,但有着从宫庭中请来的顶尖名师,本人又愿意下功夫去练习,进步速度自然很快。如今精心绣出的一干花样,水平已不输宫造的绣品。

韩锳各色陪嫁,装满了三十六只箱笼。其中整整一箱,都是韩锳亲手缝制的绣品。

韩冈书桌上的这面对开小屏风,就是他的女儿所精心绣制。

屏风底色纯白,用墨色丝线绣成。气学要义的《东铭》、《西铭》,以韩锳最擅长的欧体,绣在上了屏风上。

韩冈收到礼后,还对外炫耀了一阵。所以在名门众多的京师之中,韩家大娘子的女红水平,也是十分有名的。

“听说官人你书房里那件玻璃盏给打破了?”晚上的时候,王旖问起了白天的事。

韩冈低头看着报纸,应了一声,“嗯,不小心给打了。”

旁边素心就笑了起来,对王旖道,“姐姐你看官人说的,不知道的,还以为是官人自己打坏的。”

韩冈抬了抬眼,“让人收拾,自然为夫要负责。”

云娘惋惜道:“可惜那么好的颜色。”

韩冈低下头,继续看报纸,“打了就打了,只要不是金娘绣的屏风坏了就行。”想想又放下报纸,“南娘呢,还在陪金娘?”

王旖叹道:“转眼就没几天了,当然是舍不得。”

韩冈低头看报,素心瞅了他一眼,道,“都舍不得,但金娘总算是找了个好夫婿,祥哥可比越娘要嫁的那一位强多了。”