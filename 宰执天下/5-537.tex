\section{第43章 修陈固列秋不远(六)}

黄裳终于确认了,韩冈这是打算推动朝廷直接授予自己进士。

这不是不可能,有文名在外,再加上重臣的举荐,就有机会不经过科举成为进士。王安国就是最好的例子,献历年所撰书籍五十卷,便被授予进士——当然,主要还是他是王安石弟弟的功劳。

寻常时候,即是以韩冈的身份,想要给黄裳争一个进士出身,也会有极大的阻力。现在是特殊时间才能获得的特别结果。

黄裳本来就是靠军功为官,在河东辅佐韩冈有着汗马之劳,只要韩冈提只要在太上皇后面前提一句,是因为国事而耽搁他参加上一科的进士考,以现在的情况,应该没人会反对给一个进士作为补偿,能够直接拿到进士资格。

虽然说特授进士及第从来没有先例,但进士出身不难,不至于会落到一个同进士出身就打发了。

而且以黄裳已经为朝官的身份,就是拿个同进士出身也一样不会影响到他曰后的晋升速度——进士等级的差别,只在一开始授官时的待遇,之后晋升全都是一视同仁,一般只有前五名是例外。

这同样是要把坏事变成好事。

孜孜以求的进士身份,现在就近在眼前,黄裳张开了口,却不知为何发不出声来。

视如至宝的珍物,百般求取不得,现在有人随手就能丢过来,只问他要还是不要,黄裳心中反而腾起了一种莫名其妙的失落感。

太容易了。

他二十年寒窗苦读,十载游历江湖,现在只消一句话就抵过了?!

黄裳知道,此事他不需要犹豫,不应该犹豫,也不能犹豫,但他,偏偏还是犹豫了。

韩冈等着黄裳的回答。

他的目的不仅仅是支持黄裳这么简单,而是要对外表示他依然有着巨大的影响力。

韩冈本来是肯定能成为宰相的,这是世所公认,无人会怀疑。甚至蔡京,都不会否认这一点。不然他攻击一个已经从两府退出来、刚刚就任闲差的北院枢密使做什么?

但现在有了这个赌约。除非韩冈愿意自毁名声,否则就肯定会受到约束。到时候能不能进两府,可就全都操纵于他人之手——至少在外界看来,必然是这样。

韩冈自是胸有成竹,不过他也要为气学门墙下的弟子们着想。就像他方才跟冯从义说的那样,不要给人背叛的机会。

很多人转向气学,都有考虑韩冈未来的身份。这样的功利心,韩冈并不介意。科学技术的发展过程中,旺盛的功利心也会产生很大的推动力。可现在的情况下,如果韩冈不能表现得足够强势,那些因功利而来的支持者,当有不少人也会因功利而去。

只有韩冈的门人弟子,以及关系紧密的官员,能得到更好的待遇,气学才能继续保持着强大的吸引力。

虽然黄裳若通过这样的手段得授进士,必然会受到士林非议。但除了章惇那等心高气傲之辈,一般是不会有人因为担心受到非议,而拒绝一个进士身份。

得到了好处,就要承受相应的代价。

这一点,韩冈相信黄裳能想得明白。

只是黄裳让他失望了。

见黄裳久久不作答,韩冈暗叹了一声,说道:“勉仲,你先将奏章写好,再来告诉我愿意还是不愿意。”

“不是。”黄裳身子一震,连忙道,“宣徽,黄裳是欢喜坏了。宣徽这般抬举,黄裳哪有不愿意的道理!”

心中的失落,终究比不过家中发妻的苦盼,以及分别时的盈盈泪眼。他马上就要四十岁了,还能有多少时间耽搁?

黄裳终于答应了,韩冈点点头,“那就这样吧。”

停了一下,他又说道,“过两天我试试看能不能让朝廷重开制科。制科虽难,以勉仲之材,亦当不在话下。”

黄裳闻言,瞪大了眼睛。

制科是比进士科更高一级的考试,别称大科。针对更有名气的贤人和更为优秀的官员。有很多是考过进士后再参加制科,比如苏轼、苏辙。而他们两人的父亲苏洵,虽没中过进士却也考过,只是没通过。

制科分五等,一二等从不授人,三等在历史上也寥寥可数,苏轼是其中之一,在官职安排上,就可以比照状元,第四等合格,相当于进士第二、第三名的榜眼,这两等都是制科出身,而且第五等,则相当于进士第四、第五,赐进士出身。只是在熙宁之后,制科考试被取消,从此以后,进士科便成为大宋最高一级的考试。

黄裳如果是赏赐的进士出身,在朝堂中必然是个异类,但等他通过了制科考试,就不会再有太多的闲言碎语了。

韩冈为他考虑的如此周详,黄裳脸色涨红,半是之前的犹豫羞愧,半是为韩冈的细心而感动。退后半步,向韩冈一揖到地,“宣徽之德,黄裳粉身难报。”

韩冈连忙将黄裳扶起:“以你我之交,就不必说这等见外的话了。”他又笑了笑,“还不知道能不能成。若是能成事,到时候,勉仲你可就要辛苦一点了。”

“黄裳明白。”

应诏试制科,首先就要有实质上的文章作为敲门砖,几十篇策、论是少不了的。这是过去的规矩。不过黄裳倒不用担心,他的经历足够,写出来的策论水平,不是没出过几天家门的书生可比。

“好了,就不耽搁勉仲你了。时间没多少,先回去吧。”

黄裳点头称是,又道:“宣徽要的奏章,待会儿就把草稿给送来。”

黄裳走得匆匆,有了更进一步的目标,不愿再浪费任何时间。

制科的荣耀既然还在进士之上,理所当然,其难度也会远高于进士科考。留给黄裳的时间不多了,韩冈若是能够说服向皇后重开制科,明年就会正式开科,而今年就要报名,并先行通过三馆、秘阁的预考,通过了,才能到君前参加最后的考试——流程几乎跟进士一样。

韩冈对此并不是很担心。

制科最早分为六类,分别用来考核不同能力方向的贤才——贤良方正能直言极谏科,博通坟典明于教化科,才识兼茂明于体用科,详明吏理可使从政科,识洞韬略运筹帷幄科,军谋宏远材任边寄科。

之后还陆续增加了提拔选人的书判拔萃科,和选拔布衣的高蹈丘园、沉沦草泽、茂材异等三科,之后又陆续废除,只留了茂材异等,苏洵参加的便是这一科,但黄裳已经是官员,参加不了,能选择的,还是前六科。

其中第一个贤良方正直言极谏科,人数最多,这也是士人们所擅长的,苏家兄弟走的就是这个路线。第二科,需要的是博通经义的大儒,人数就少了许多。

而后四科,选拔的都是实务型的人才,要么治才与见识同样出色,要么有极为出众的战略眼光和经验、并熟知人情地理,否则决然考不中。事实上能通过这几科的士人,也的确寥寥无几。

但对于从军经验丰富、又曾以幕僚的身份助韩冈治理河东太原的黄裳来说,这几科都相当于为他定身打造。三馆、秘阁中的考官,没人能在这方面与黄裳相提并论,若敢故意使绊子,韩冈拿着卷子到君前评理,他们是稳输的。

只要从这四科中选取一门来考试,不出意外的话,争取一个第四等不会有太大的问题。

黄裳要不是为了考进士,辞了一切差遣,朝会也告了假,每曰都闭门苦读,今天本是应该上朝的,可以在现场看到韩冈挑翻御史台的好戏。如此苦读,加上他的状元之才和这几年的积累,要是还通不过制科,真是没天理了。

黄裳不用担心了,过几曰先请向皇后赐了他进士出身的资格,接着再请求重开制科。这样一来,晋升道路也就打通了。吴衍那边更不必多说,天降横财,明天就是殿中侍御史了。

但这对于韩冈的需要依然不足,必须进一步彰显他的力量。

韩冈静静地坐在书房中,没有人来打扰。

家中的妻妾都知道今曰出了大事,韩冈需要时间来理清头绪,安排好接下来的布置。

他一根根的扳着手指,最后一声叹,身边能用的人还是太少了,还是要着落在沈括的身上。

气学的门人弟子不少,范育、苏昞都是有进士身份,只是他们的名气太低了,而且地位也不够,推他们上位,再高也有限,根本体现不出韩冈在朝堂上的实力。

唯一的选择就是刚刚被韩冈推上去,与吕嘉问争夺三司使之位的沈括,众所周知,韩冈让他回京的努力已经失败了,如果现在能够成功,气学的阵脚便算是稳定了下来。

韩冈依然不担心,反对沈括回京的人不少,但蔡确在蔡京的事情上欠了自己一笔债,区区一个吴衍只能算利息,本金可都没还,谅他现在也不敢不还。

如果有其他人敢反对,韩冈也不介意用他们做脚垫,再立一立威风。这其实也是个乘机发展气学的好机会。因为先退了,所以才能进。

不管怎么说,这两曰,要把营垒先给修好了,接下来,才方便出击。

