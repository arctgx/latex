\section{第14章 贡院明月皎(中)}

参拜过至圣先师,文庙大殿前的广场上排得整整齐齐的五千贡生,顿时土崩瓦解一般的四散而去。在胥吏的引导下,前往自己所在的考场。

一张半新不旧的几案,一张掉光了漆的圆凳,这就是韩冈的位置。不知平日里,国子监的学生用了多少年,现在被摆了出来。整间偏殿中,六十多名锁厅贡生,分配到的座位都是一水儿的破旧。

在几案一角的贴了一张纸,上面有着韩冈的姓名,同时还书有籍贯、年甲。就算是同名同姓,只要籍贯不同、岁数不同,就不会坐错了位置。几案边还有个小桶,里面的清水是为了磨墨而准备的。

这等周密的准备,是百多年来的一步步积累下来的经验。不仅仅是座位的安排,从进门之前,韩冈就已经感受到了在抡才大典上,宋人所表现出来的组织水准。

不过他现在并没有多余的心力,去赞叹与后世已经相差不大的考试筹备工作。今科礼部试的考题,已经在文庙之前张榜而出。而其抄本,更被考官带到了殿中,高高挂起在众考生的眼前。

韩冈扫了一眼贴经墨义的题目,果然比起锁厅试来,难度要远远的超了出去。他事先已经有了心理准备,要从五千一百人中挑选出三百人,如此高的淘汰率,试题的难度必然大大加强,以便拉开名次距离,也让考官易于评判高下。

从小桶中舀起一点清水磨好了墨,韩冈张开刚刚发下来的草拟文字所用的纸张,开始向草稿纸上抄写今次的考题。

笔墨纸砚等文房四宝,是可以由考生自己带进来的,但文集、等书籍就不允许带进考场。不过韩冈在进考场时,并没有被严格的检查。并不是因为他是官员而被放松,韩冈看了其他贡生,也一样检查得很松。

进士科的考试长达一整天,大部分考生很少会快速交卷,基本上都是从凌晨一直考到点灯,这么长的时间,中途当然可以吃饭。几乎每一个考生都是带着篮子,装了笔墨纸砚和干粮进来。但搜检考生的士兵,也并没有掰着炊饼,看看里面藏没藏着小纸条。

大概是因为过去以诗赋取士,靠夹带做不了弊。今科是第一次改变,经义注疏这个考试范围,远比诗赋要小上许多。韩冈估计到了以后的考试时,防止夹带的搜检工作就会加倍的严厉起来。

韩冈运笔如飞,笔迹工整的将题目全部抄写了下来。虽然三十条经义出得虽然冷门,但对于精研甚深,又经常利用书信,聆听两名当世大儒教诲的韩冈来说,并没有太大的问题。而唯一的一道策论看过之后,也让他放心了不少。

策论其实是两种文体,策是策问,对某件政事给出一个可行的策略。而论,就是议论,对某事某人或某件史实加以评述。今次的考题并不是策,而是论。题目虽然读着拗口,本质内容则很简单——关于秦和商君。

商君就是商鞅。说起商鞅变法,以及秦兴秦亡、六国生灭。从汉时起,就没少被人提起。《过秦论》就不提了。《六国论》,老苏做过,大苏做过,小苏也做过。商鞅变法的成败得失,谢安石说过,王安石也说过。

韩冈还记得王安石曾经写过的一首论商鞅的诗——自古驱民在信诚,一言为重百金轻。今人未可非商鞅,商鞅能令政必行。

王安石推崇商君卫鞅,如今的变法也仿佛商鞅当年。在场的考生只要不是糊涂蛋,恐怕都会拿来做论题。

只是这个看似简单的题目,因为写的人太多,便很难写得出彩。看起来曾布吕惠卿就是用这等题目,一下刷掉大半考生。

而韩冈,则将这份题目轻轻放到一边,开始俯首写着贴经墨义的答案。与他人不同,对于关键的策论,他已是胸有成竹。

……………………

巡视考场内外的兵将来回走动,考官们则各自坐在正殿两侧的厢房,等着考生们完成他们的考试。

曾布、吕惠卿等几个主考官,现在能在殿后休息。而叶祖洽,上官均等小官,则是必须在殿门便的小角房中候着。

总共十几个官员,都是身穿最低一等的青色官袍。叶祖洽他们的差事是点检试卷,其实就是考校举人试卷,批定分数,拟定等第。也就是说,他们是批改考卷的第一道关口。

叶祖洽,是上一科的状元,上官均、陆佃是上一科的榜眼。这些监考考官,除了一两个例外,基本上都是上一科或是再前两科,排在前十名的进士。

二月初的天气,有些背离正常的年景。清晨时还好,但到了近午时分,就热得仿佛是三月末的暮春时节。陆佃坐在窗户边上,正能晒到太阳,官袍内的皮袄根本穿不住身,脱了之后,方才能按坐下来。

十来个前科进士,百无聊赖的坐在一起,除了闲谈也没有他事可做。

“不知今科状元会花落谁家?”叶祖洽很悠闲的问着,也只有他这个的状元公,才能用这等前辈的口气说话。

“殿试还早得很,还是猜猜谁是礼部试第一吧。”舒亶是治平二年礼部试第一,也就是省元。针锋相对的说话,其实也是在半开着玩笑。

“应该余中吧……他在国子监中名气不小。”龚原是国子监直讲,对于国子监内的情况很是了解

“湖州朱服名气也不小。”另一人说着。

叶祖洽立刻将之否定:“他的文风只合作第二,做不得状元。”

朱服是苏轼的弟子,叶祖洽能看得惯就奇怪了。

“叶涛的文章不差。”

“他的确有些可能。”

“还有邵刚。”

“文采识见都有过于常人之处。”

天下聚于京城的五千多贡生中,能在东京城中传扬开姓名的,多半都不是简单人物,大部分都有冲击状元的实力。余中、朱服、叶涛、邵刚都是其中的佼佼者。

“韩冈呢?”忽然有人冷不丁的提到了这个名字

论起名气,韩冈在今科贡生之中,是当之无愧的声名最盛。

陆佃是王安石的学生;叶祖洽在殿试的策问试卷上写了一堆关于新法的好话,差点就被苏轼给黜落。上一科取中的排名前列的进士,无一例外都是偏向于新党一边。但他们没有一个看好韩冈。

陆佃摇头:“韩冈恐怕不成。就是他真有才学,阅卷时能排在前列,拆卷后也会被强拉下来。瓜田李下的嫌疑,曾、吕二位,有哪个愿意沾的?”

“何况他从无文名,亦不见有何诗作流传。”叶祖洽也说道。

“说到诗作……”上官均了起来,“还记得西太一宫中的那首枯藤老树吗?”

“不可能,韩冈的年纪经历写不出来!”龚原一口否定,“世间不是流传说是一个久试不中的老举人吗?”

“传言没有错,这一篇当然不是韩冈的手笔,至少不全是。”上官均神神秘秘的说着,“韩冈只是加了四个字而已!”

“……夕阳西下!”陆佃脑筋转得快,一下惊道,“可是这四个字?!”

“正是!”上官均点头,“各位去西太一宫看那一首枯藤老树的时候,没觉得那四个字是后添上去的吗?”

“……的确。”龚原回想起来,的确是有这个感觉。可转念一想,又觉得有哪里不对,“但这首诗,他为何没有题名?!”

“因为只是添了一句,所以韩冈没有居功……但因为是韩冈妙笔增辉,所以那位老贡生也没有宣扬是自己所作。”

“真的假的?”叶祖洽还是有些怀疑,“莫不是在诳我们吧?”

上官均微怒:“当初小弟和蔡元长都在场,亲眼看着他们离开。墨迹都是新的,哪还会有别人来写?!”

但陆佃心头依然有着疑惑,“前次小弟去观题壁,怎么觉得‘夕阳西下’四个字与全篇的字体都是一样!”

“还是略有区别。大概是韩冈为了能配合得上前面的字体,而刻意贴近了来写。”

陆佃点点头,“如果这是真的,韩玉昆的才学当是毋庸置疑,画龙点睛不外如是。”

没有那四个字,整首诗作为王安石两首题壁诗的和应之作,连中平的评价都不够资格,只是怨气深重而已。写出这样的作品,考不上进士也是当然。可‘夕阳西下’四字一出,便是画龙点睛,甚至力压王安石一头。

“那位老贡生最后怎么了?”龚原追问起了原诗作者的情况。

“一首枯藤老树都写出来了,还会有什么想法?”上官均回想起西太一宫中的那首诗,就算少了韩冈添加的四个字,也能感觉到充满在字里行间的悲凉和沧桑,这一篇诗作的作者怎么可能还有心留意仕途,“此人姓路讳明。当年屡考不中,在西太一宫中留诗时,被韩冈四个字如当头棒喝般点醒,最后弃儒从商了,现在已是广有身家。”

“这……实是有辱斯文。拿着这首诗献于天子,怎么都能得个官职回来!”

“穷官可比不上富商。”上官均冷笑一声。又道:“要不是腊月时,蔡元长任满回京候阙,正好在章子厚家中遇上,也没人能知道其中的关节。”

“蔡元长上次还见过他,怎么没听他提起?”叶祖洽很奇怪的问着。

