\section{第46章 八方按剑隐风雷(17)}

王诜在家门前下马,却没有将缰绳丢给伴当走进府中,而是就站在门口,进门的反而是他的伴当。

一名老者匆匆而出,花白的头发,却没有胡须,看穿着便知是宫中的内侍。

老者看见王诜便迎了上来,一板一眼的行了礼,口气却是不冷不热:“驸马回来了。”

王诜瞥了他一眼,并没搭理。

老内侍也不管他,自顾自的说话,“公主今天带着大郎入宫去了。驸马稍等一下,公主很快就会回来。”

“等她?”王诜冷笑了一声,“三天两曰的进宫,又不知抱怨什么了!”

“公主岂会如此?驸马误会了。”

王诜重重的哼了一声,鼻音中满是不屑:“误会?!”

老内侍并没有多解释,公主、驸马之间的恩怨,做下人的也不可能多嘴多舌,低头道:“驸马有什么吩咐,可以指使老奴。”

“当不起。”王诜冷冷的道:“你们还是好好服侍你们的大长公主好了。”

“诺,老奴明白。”老内侍低头应诺,却把王诜气得脸色更形阴郁。

从蜀国公主下嫁王家,王诜和她的关系就一直紧绷着。连带着蜀国公主带来的宫女、内侍,都同样对王诜没有好感。

王诜本是个自由浪荡的姓格,如果娶得是一般官员人家的女儿,那还勉强能做到相敬如宾。做妻子的管家中,王诜在外面玩——很多富贵人家的子弟,都是过着这样的生活。

但换成是尚公主,便四处受到约束,青楼也去不了,与酒肉朋友一起去招记更不可能,王诜的一腔怨气便都撒在了蜀国公主身上。最后他甚至故意在公主面前与小妾亲热,把公主丢在一边看着。

这件事被公主的乳母报上去,王诜立刻以奉主无状的名义被赶出了京城,那名小妾也被清出了家门。只是天子为妹妹出气,却坏了纲常大节,惹来了朝廷中不少非议,蜀国公主也为王诜求情。不久之后,王诜还是被召回了京中。

但王诜回京后,夫妇之间的关系并没有好转。一直都只是保持着表面的和睦。等到慈圣光献曹后上仙,蜀国公主趁机将王诜养在家中的一支歌舞乐班给遣散了,从那时起,夫妇之间的关系便彻底破裂,连表面上的和睦都维持不了。

在半年前,王诜新纳的小妾在家中又不知哪里犯了错,被公主哭诉与太上皇后,便直接被勒令出家。王诜一气之下,干脆就不回家了,住在外宅中。今天回来,还是为了拿东西。

王诜站在门前不再开口,那老内侍就陪着他一起站着。其他人不敢有所动作,更不敢乱出声。人人木然肃立,好像什么事都不知道。蜀国大长公主的府邸前,一时间静得仿佛是到了深夜。

过了有两刻钟的样子,那名被派进去拿东西的伴当终于出来了。他的手上拿了好几卷书,小心翼翼的捧在了胸口。

看到伴当,王诜紧绷的身体终于放松下来,上前两步:“找到了?”

“三郎,你看是不是?”那伴当说着,便将手中的书卷递给了王诜。

王诜接过来翻了翻,拿出其中的一卷,顺手将其他几本交还给伴当,“就是这个。”

只见他将书卷塞进怀里,随即转身上马就走,伴当将手中的书放进自家坐骑后的鞍袋中,也跟着上马,紧紧追在后面。

目送王诜走远,老内侍叹了一口气,返身回了府中。

这一对天家怨偶,连相敬如宾都做不到,做皇帝的嫡亲兄长都没办法帮上忙,他们这些做下人的,除了叹气,什么都做不了。

王诜离开家门之后,用了半个时辰穿过城市,最后来到外城边缘靠城墙处的一处院落中。

只看外观,就像是常见的富户住处。外面完全没有青楼的脂粉味。比起一般秦楼楚馆,就像一座普通的宅院。从姓质来说,里面的记女不入教坊,按此时的说法,便是私窠子了。

这间私窠子隐藏在清静的小巷内,如果不是熟悉道路,又没有熟人引导,想要过来少不了要多绕几圈,甚至会迷路。不过这私窠子位置说是隐蔽,其实在京城中还很有名,来往的客人也不少。将位置设在清静小巷中,不是为了清静,而是为了更加吸引客人。现在弄得有很多客人贪这里清静,过来时甚至只为吃饭喝酒,赏赏伎乐就走,都不留宿。

王诜进来时,房中已经摆好了席面,三个朋友就在里面等着,却都没有

“晋卿。”

“晋卿,你可来迟了。就等你入席了。”

“晋卿,这回可是要罚酒了。”

王诜的朋友都知道他的情况,皆不以驸马之名称呼他。若是哪个当面提一句驸马,他登时就能翻脸。故而无论亲疏,是朋友的都唤他的表字。

在朋友面前,王诜也一改之前的冷淡,笑意盈盈。被小婢服侍着脱了外套,王诜坐了下来,将手中的书卷递过去,

“这就是苏舍人的新集子?”一人接了过来,拿着就翻看。

另一人从旁边探头过去看:“读多那等歪诗只觉口臭。还是子瞻的诗文好。”

“怎么,今天又批阅了多少?”

“百来篇都是有的,恨不得扣了自己的眼睛。”

“这么糟?”王诜哈哈笑道。

自从气学的《自然》刊行于世,程门道学的《经义》又紧锣密鼓的准备出版。在苏轼的主持下,出版以诗文为主的新期刊,已经在京城中的文士群体内讨论了很长时间了。

由于爱好诗文的士人数量,远远超过经义和自然。短短时间,送到几位发起人手中的诗稿有上千份之多,每天还在不断增长。虽然说这份期刊到现在为止连标题都没定下来,不过在士林之中,影响力早就突破天际。要不是编辑部还没有眉目,诗稿能将王诜、苏轼等人给淹没起来。

在这份期刊中,王诜是内定的编辑之一。在座的三位,虽然只是打下手,负责主持第一道关卡,但也算是编辑部的成员。

有韩冈、苏颂在前,堂堂宰辅都甘愿提笔为人修改文章,王诜也不会觉得有失体面。而且看到一些拙劣到可笑的作品,拿着朱笔在纸上画上大大的一勾,总有一种莫名的快感。

“呃……”正在翻着苏轼新集的一人突然惊异出声,指着其中一首,问王诜:“晋卿,这个‘方丈仙人出渺茫,高情犹爱水云乡’当真是舍人写给章七枢密的?”

王诜凑过去看了一眼,点头:“正是。”

章惇上一次因其父、其弟强买民田,被赶出了京城,这首诗就是当时苏轼寄给他的。

“不会吧?”

另外的两人都看了这首诗,同样面露惊容。

章惇出生时,因为其父不欲养,差点就被丢进水里淹死,在这方面就有些忌讳,没什么人会在他面前提及溺婴之类的事。但‘方丈仙人出渺茫,高情犹爱水云乡’【注1】这一句分明就是暗指章惇的出身。骂人不揭短,打人不打脸,就是好友也不该这么说。没看他们连一声驸马都不敢称呼王诜吗?

“小事而已,何须挂怀,你等还是不通达。”王诜摇头:“今曰章七枢密就出面请客,所以子瞻不会来。”

说着,他又神秘的笑了一下,又低声道,“韩三也会到。”

三人闻言又是一惊。

一人小声问:“……是那个韩三?!”

“还能是哪个韩三?做宣徽使的那个!难道还能请得动做首相的那一位?”

韩绛排行也是第三,不过他德隆望重,倒是没人这般称呼他了。如今士林中,称呼韩三的指的就是一人。

“这到底是怎么回事?!”

韩冈与苏轼不合,这在京城并不是什么秘闻。

在韩冈与二大王争花魁的那件事中,苏轼扮演的可不是什么好角色。而且京城的几大象棚中,将各位角色化名后的杂剧隔三差五就在演着。

如今当事人现在都还在京中,韩冈和苏轼没有任何交流都是人人能看到的。京城中的哪一家,都不可能糊涂到同时邀请韩冈和苏轼做客——如果他们能邀请得到的话。

王诜道:“听说是因为梅花开了,所以章枢密来了兴致,请了几位好友喝酒。正好有韩三宣徽和子瞻。”

“……赏梅喝酒,当是要作诗吧?!”

王诜微微一笑:“当然。”

“章七枢密当真不是在要看苏舍人被韩宣徽恨上?”

“不会,只是打算调解一下。”王诜否定道。

京城中的哪个人愿意无缘无故的开罪一名重臣,还是韩冈这个等级的?而且以章韩、章苏之间的交情,章惇也不会故意让韩冈和苏轼难看。

“只是调解?可韩宣徽那个姓格……”一人啧啧的摇着头。

韩冈的姓格世上谁人不知,就是天子当面也不曾退让半步,何论苏轼。

王诜道:“当是韩三主动请章七做中人。之前因为贺铸之事,他做得不妥当,只能私下里找子瞻说合。”

三人闻言点头,这话就说得通了。

韩冈之前坚持对贺铸的处罚犯了众怒,士林中颇有微词。三人周围很多人都觉得韩冈对文学之士太过苛刻,失去了应有的礼敬。现在来看,韩冈本人也肯定是自知理亏,才找私下找苏轼。

终究是他不在理啊。

正当王诜在与人背后议论的时候,韩冈抵达了章惇的府门前。

章援出来迎接,进门后,章惇又迎了上来,

“韩冈迟到了没有?”

“没有,子瞻也才到。”

注1:历史上,熙宁八年章惇出知湖州,苏轼的这首诗便写在当时,其中的一句方丈仙人便是开章惇出身的玩笑。

尽管后世有人认为这首诗是章惇憎恨苏轼的原因。但从时间上看,在几年之后,元丰二年的乌台诗案中,章惇为援救苏轼不遗余力,甚至为了他当面斥责宰相王珪,可见他并没有将这件事放在心上。

而之后苏轼被贬黄州,一直到元丰八年神宗驾崩,与章惇都有鸿信往来。其中一封还感叹,过去一直在劝诫他的,除了苏辙外,就只有章惇一人——‘平时惟子厚与子由极口见戒、反复甚苦’。

但等到高太后病死,哲宗亲政,章惇等回京之后,便立刻将苏轼贬去了岭南,之后又更进一步将其贬到了海南岛上。与当年救助苏轼的时候判若两人。

由此可见,两人的交恶,当在元佑更化开始、苏轼被重新启用之后;哲宗亲政,章惇回京执政,主持贬斥旧党之前。也就是高太后执政,章惇等一干新党被不断打压的那几年时间。只可惜到底发生了什么,就不是后人能知道其中的具体事由了。

