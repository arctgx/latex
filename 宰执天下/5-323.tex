\section{第32章 金城可在汉图中(二)}

韩冈一行抵达酸枣县的时候,城门早就关了,更鼓在城头上梆梆的敲着。 

不过在城下一通名,守门官便忙不迭的将城门给打开,低头哈腰的迎了韩冈进来。 

夜色深沉,不似京城的烟花繁华,根本看不到几点灯光,宁静的城市沉在睡梦中。 

但就在这寂静的夜晚,酸枣城内的街道上,突然一阵雨打芭蕉的马蹄声响起,带起了一片犬吠,从城南响到城西。在驿馆前,停了下来。 

酸枣离东京城近,入京的官员往往都会设法多赶上一程,住在京城里自然是要比郊县中安逸。驿馆中入住率不高,韩冈一行近百人,没怎么折腾便全都安顿了下来。 

知县这时得到消息,带着县中的官员赶来问候。韩冈没见他,让黄裳穿着官服出去接待,三言两语便打发了。 

不过这个知县倒也识做,退下后就从县中找来了几名大厨,为韩冈和他的随从们置办饭菜。

韩冈梳洗过后,匆匆吃过饭,跟黄裳商量着要,丢下大队,自己先行一步。黄裳想要劝,韩冈便问他:“驿馆中有多少马?” 

黄裳无奈一叹:“三十余匹。” 

这算是多了,酸枣毕竟是京畿大县,普通的驿站和军铺甚至连一半都养不了。 

“我是打算兼程赶去太原,但什么都吃不住一千多里的路程。一天一百三五十里,只消三五日,就能死上大半。可想要一路换马,沿途没有哪家驿站能支持得了?——人太多了!”韩冈摇摇头,他出京时太急,还是考虑得少了,“我是去太原坐镇,不是上阵。十一二个就差不多了,剩下的后面慢慢跟上来好了。” 

韩冈和黄裳讨论带着谁先走。韩冈跟黄裳商量,就是准备让他留在后面做领队。商议抵定,就听见外面一阵喧哗声传来。 

“去看看出了什么事,怎么闹起来了?”韩冈吩咐了一名亲信出去查看。 

他明天就要赶着上路,纵马兼程,正要睡觉养足精神,听到外面闹起来,心中便是不快。 

转眼就是一名在外守夜的班直进来报信,“是过境的金牌急脚铺兵,在城里换马的。看到了马厩里的马,就大骂驿丞欺人太甚,他身携军情急报,连夜赶往京城报信,驿中好马百十,竟然只拉了一匹劣马出来充数。” 

“这马不都是我们的,哪里是驿马?谁见过四尺三四寸的好马做驿马的?”韩信愤然道。 

回复 2楼2012-12-31 00:59举报 |

l97wang
小吧主11
韩冈一行带的马都是一流的,不论是班直还是韩家的家丁,都有好马骑乘,其中自然是韩冈本人的坐骑最好。且都比驿馆中能用来当做铺递替换的坐骑要强——军马分三六九等,好马通常就充作了战马,只有下等的军马才会充作驿马。 

这些坐骑一同放在驿馆的马厩里,被个懂马的铺兵看见,而驿馆中却从中牵了一匹劣马出来给他换乘,也难怪会闹将起来。铺兵虽卑微,可带上金牌的急脚递,就不好欺辱了,他身上的紧急军情是能送到天子面前的。 

“枢副,黄裳出去看看。”黄裳起身。 

“哪用得着你去?”韩冈笑着摇摇头,“韩信!你出去一趟。问一问他带了什么军情,报我的名字……跟他好好说,不要仗势欺人。” 

韩信恭声应诺,韩冈又对黄裳笑道:“要不是今天出京后就紧赶慢赶,坐骑耗了不少体力,直接把我的马借给那个铺兵也没什么大不了的。反正城南驿也不敢贪墨了我的马。” 

天下铺递都归于枢密院管辖,韩冈可是驿馆顶头上司的顶头上司,当然不可能有哪家驿馆敢贪占他的马,放在驿馆中,只会用好料养着。 

但韩信刚刚接了韩冈的吩咐,正要出去,另一名在外值守的班直就进来了:“小人刚报了枢副的名,那铺兵就嚷着要拜见枢副,说是代州的故人。” 

“故人?” 

韩冈微微一愣,这倒是有趣了。换作是陕西倒也罢了,微贱时自然会有地位不高的故旧。可他到河东时就已经是经略安抚使,掌控一路兵马,一个铺兵哪里有这个资格自称故人?不过话说回来,那铺兵既然敢自称是枢密副使的故人,好歹应是有些底气的。 

韩冈努了努嘴,一名曾经跟着韩冈左右、一同经历过河东的亲信就出去了。片刻之后,他就转了回来。 

“可是熟人?”韩冈问道。 

“是西陉寨秦寨主的儿子。” 

“……秦怀信的儿子?都已经回河东了?”秦怀信去年死在了夔州路任上,让韩冈惋惜不已,他的两个儿子韩冈都见过,也的确算得上是故人。只是变成了铺兵的身份,却让韩冈很纳闷,闻言便问:“是秦琬还是秦……秦……?” 

前任西陉寨寨主秦怀信的长子秦琬,当初虽只有一面之缘,却给韩冈留下了很不错的印象,是个很聪明又有见识的年轻人,日后当能在军中有所成就。至于秦怀信的次子,虽曾经代其父奔走报信,还多见过两面,可印象就是很淡薄了,韩冈连名字都没记住。 

“是秦玑。”亲信说道。 

‘秦玑。’韩冈点了点头,终于想起来了。 

“让他进来吧。”韩冈吩咐道。不管秦玑带了什么紧急军情,他都有资格问一问。

秦玑被领进来了。韩冈在他身上已经看不到旧日的影子。举止看着很是老成,并没有在外面吵闹时的浮躁,连相貌都有三十左右的样子。 

看起来这一两年间,他是吃了不少苦,才会让一个二十出头的年轻人一下苍老了许多。 

看着一身铺兵装束的秦玑,韩冈先是神色黯然,可紧接着心头便腾起一阵疑云:“秦玑,你父秦怀信再差,最后也做到了一州都监。纵然过世了,身份还在,怎么轮到你这个衙内做了铺兵?!” 

秦玑一听,顿时眼圈就红了,哭拜在地上,“想不到枢副还记得先父。” 

韩冈摇摇头:“你父也算是我旧部,怎么会不知道?好了,且起来说话。” 

秦玑擦了擦眼睛,依言起身,“小人跟在家严身边受庭训,一直跟到。不过家严去世后,小人就回了乡里。至于铺兵,是前几日家兄安排的。” 

“秦琬可还好?” 

“家兄现在忻州军中任指挥使,尚幸军中的陈都监是小人父执辈,过得还算可以。” 

‘指挥使?’韩冈点点头,算是明白了来龙去脉。军中并不讲究庐墓三年,比如边将,遇上父母丧无一例外都要夺情。秦琬回到乡中后,没有官职的他,能有一实差,出掌指挥使也算不差了。且他能在这个节骨眼上,让秦玑做铺兵上京,也可见秦琬还是有一定的活动能力——多半是来找自己的。 

“你父是可惜了。我这一回回河东,若要重整河东军,除了刘舜卿,就是你父秦怀信了。” 

秦玑眼圈又红了,用手背蹭了蹭眼睛,擦去了泪水,哽咽的谢着韩冈的看重。 

“还是说说正事……”韩冈神色严肃起来,“你今日带了什么军情来?” 

秦玑闻言,脸上感伤的神色一扫而空。咬起了牙,板着脸,一字一顿:“回枢副,是代州知州魏泽降贼!” 

如同石破天惊。 

房中自黄裳以下,连同班直和韩家的家丁,全都怔住了。 

不知多少年没看到有知州一级的大臣降敌了?魏泽那可是诸司使一级的将领,正七品的官宦,任职上州知州,在朝中、在军中,都不是那种一抓一把的普通角色。 

尤其是一干班直,更是一幅难以置信的神情。魏泽是京营出身,过去在京城时,地位和名气都不低。他们之中有好几个都见过魏泽。 

只有韩冈神色不动,在另一个世界,几十年后,投降的更多,地位也更高,区区一个知州,还真算不了什么。 

“这事怎么传出来的?”韩冈平静的问道。 

“前几日,魏泽和几名叛贼被辽贼派到忻口寨劝降。” 

提起魏泽,秦玑便是咬牙切齿,但说话间却也有几分快意。知州降贼,在大宋决不是件小事,尤其还是转头就帮辽贼做事,更是罪不可绾。少不了要牵连亲友。 

秦家世代在代北,军中根基深厚,魏泽在代州没少拿秦家的亲朋好友下手。这是现世报啊!

韩冈没去多注意秦玑的心情,沉着脸追问道:“忻口寨还没丢?” 

河东是由山脉和一串南北向的盆地所组成,太原正是一个大盆地,而在代州和太原之间的忻州,这也是一个,不过比较小。忻口寨便是忻州北界的关隘。忻州多山,也多关隘,本来就是太原北方的屏障,只要忻州还在,太原就不会丢。 

“现在多半已经丢了。”魏泽语气沉沉,“小人不敢妄说军机,但由于有代州在前的关系,忻口寨一直以来也没有多加整修,守不了太久。而且忻州兵力不足,所以在小人出来前,州中已经议论要放弃忻口寨,守住秀容【今忻州市】和定襄两城。” 

“糊涂!”韩冈脸色一变,“忻口寨守不住,秀容县和定襄县就能守得住吗?!” 

将忻州主力放在忻口寨,而忻州内部的空缺依靠太原的外地的兵马赶来填补。只要太原能派兵北上支援,完全可以稳守住忻州。除非…… 

“可是太原的王.克臣不肯派兵?”韩冈问道。 

秦玑苦笑:“派了。” 

“多少?” 

“两千。” 

韩冈一声叹息。王.克臣他还以为他就是一个太原知府吗? 

对忻州,韩冈已经不报希望了,失去了忻州,太原城北就只剩石岭、赤塘关等几处军寨。不过石岭关是太原的北界门户,兵家必争之所,亦是一等一的要隘,倒是还留着一线希望: 

“希望他能守住石岭关。”
