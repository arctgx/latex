\section{第34章 彩杖飞鞭度春牛(上)}

天色有些阴沉,韩冈抬头看了看,看起来要下雪下雨的样子。他不知道鞭牛祭祀在天气上有没有忌讳,看起来多半是没有的样子。只是在野地里举行的祭典,没遮没挡的,下起雨雪来可是会让人很不爽。而他明天就要往东京城去,更是不希望逢着雨雪。

大清早的时候,韩冈便来到秦州城的南门外一块被清出来的空旷场地上。周围已是人山人海,人群的中央,李师中带着秦州城内的一众文武官庄严肃立。他们的每只手中都拿根五色丝缠成的彩杖,围着一头披红挂彩的土牛。土牛边上还有泥塑的农夫和农具。

这头用泥土塑就,与真牛一般大小的春牛,雕得甚为精致。一个俯首拉犁的动作,连肩胛处鼓起的肌肉都刻画了出来。牛尾轻摆,貌似驱赶蚊蝇,竟然活灵活现。如此雕工,让韩冈很好奇这是谁家手笔。

在今天的仪式上,这头泥牛便是主角。

鼓乐声中,李师中带头围着春牛转了一圈,又抽了三鞭。一个个官员依序上前,与李师中一样的举动,转一圈,抽三鞭。旁边还有两名小吏用着秦腔高声吼着劝农歌,是令韩冈叹为观止的标准的原生态唱法。

这一套仪式,称为鞭春,又称打春,用意是祈求丰年。不但是秦州,天下南北十八路,四百军州,数千郡县,乃至皇宫大内,到了立春的这一天,官吏也好、天子也好,都要走出来,对着土牛屁股抽上三鞭子。天子还有藉田之礼,就是下田推犁,推上九下,以示劝农之义。

韩冈还没得到官身,不够资格参加鞭牛。但他的身份,让他占据了一个好位子,站在最前面围观。韩冈的高个子让身后的观众们愤怒不已,就听见他们一个劲的在后面蹦达。

还有许多行脚商,在人群中窜来窜去,高声叫卖着一个个泥塑的五色小春牛。小春牛巴掌大小,惟妙惟肖。最高级的小春牛甚至有个精雕细琢的小木笼子装着,笼子上还插着一列泥塑百戏人像。这样的一具春牛,往往价值四五贯之多。

不理会身后的动静,韩冈的注意力都放在手执鞭牛彩杖的官人们身上。能看到秦州城中文武两班的几十名大小官员同时出动,一年中也没有几次机会。

与官袍划分文武的明清两朝不同,此时参加仪式的文武官员身上所穿的服饰并没什么差别,只能通过身材体魄来分辨。韩冈一个个辨认他们的身份,其中有一多半他只听说过名字,从未见过面。直到现在才是第一次把名字与人对应起来。

“那么多官人,怎么一个关西人都没有?”人群中,不知是谁突然冒出来一句。

立刻就有好几人一起反驳:“向钤辖就是关西人!”

得他们提醒,韩冈再仔细观察了一遍。向宝的确是关西人,但向宝之外,在场的几十名文武官中,却真的没有一个陕西出身。若是文官倒也罢了,本就是四方为官,能守乡郡的都是特例。但守边的武臣就不同了,总得有些本路出身、熟悉人情地理的成员。

韩冈双眼从在场的武官身上一个个扫视过去,忽然发觉他们论年纪都在四十到六十岁左右——二三十岁的青年将佐官品都不高,本就是不够资格参加祭典。发现了这一点后,韩冈便释怀了。一点不奇怪,因为这个问题同样出现在关西的其他几路。在四十岁到六十岁之间,在陕西禁军中有个很明显的断层。

关西领军的中层将校中,包括诸多城主、寨主和堡主,但凡四十到六十岁之间的,大部分都不是在关西土生土长,或者说不是根正苗红的西军出身。

比如向宝是镇戎军人,但起家是在东京,并不被视为西军中的一员。郭逵、杨文广、张守约在关西多年,但他们也都不是陕西人。

造成这种局面的原因只有一个,就是二十多年前,李元昊起兵叛乱后,宋军在三川口、好水川以及定川寨三次会战的接连惨败,以及在其后多年间与西夏交锋中的连续失血。

这三次会战惨败,论兵力损失,加起来其实也没超过十万,但关西军中的精兵强将几乎被一扫而空,尤其是许多早早就被看好前途的年轻将校,都在三次会战中损失殆尽,使得西军元气大伤。以至于近二十年时间,多是被动挨打的局面。

狄青、种世衡这两位西军中的佼佼者,在面对党项人的时候,也是守御的时候居多。到如今,狄青、种世衡接连故去,宿将中郭逵、杨文广硕果仅存,还得靠张守约这等老家伙去边城驻守来撑场面。

至于刘昌祚,虽然祖籍河北真定,但自父辈起,便移居陕西为将,却是标准的西军一员。刘昌祚虽然四十出头,但还应该算在新生代这个层次,因为他是承父荫而得官,其父刘贺便战死于定川寨一役。

不过从庆历议和后,成长起来的西军将校如今都处在当打之年,刘昌祚、王君万,再到最近据说很得向宝赏识的刘仲武,莫不是如此。二十多岁,三十多岁的优秀将校,在关西数不胜数。王韶如要挑选参与拓边河湟的将领,可以选择的余地,便远比当年来关西救急的范仲淹、韩琦要强上了许多。

回头再看着站在官员队列中的王韶,昨日还纵马奔驰的经略机宜,现在也是手拿彩杖,排着队亦步亦趋的挪着上前。一个个平日里衣冠楚楚的官员,举着彩杖手舞足蹈,韩冈觉得有些无聊,即便当做娱乐节目,感觉上也不过如此。

但参加仪式的人众,包括李师中,包括王韶,都是一本正经。农为国本,仪式上出点差错,万一当年收成不佳,可是要受到全州县的百姓怨恨。捅到朝堂上,也是一桩罪名。

李师中已经站回了主持仪式的主位,端端正正的拢手而立,表情庄严肃穆,仿佛一具雕像,只要是在朝堂上待过两年,多半就会练出这身本事。隶属于秦凤经略司和秦州州衙的属官们,正依着次序上前鞭牛,还有好一阵才会结束。

李师中脸上维持着庄严肃穆的神情,视线却盯上了周围人群中的一人。吸引住秦凤经略使目光的,是站在人群最前面,一位身材高大的少年。

‘是韩冈吧?’

虽然王韶、吴衍和张守约的荐章,李师中都细细读过,其中对韩冈的才能、德行推崇备至,但李师中还是第一次看见韩冈本人。

的确出色!

李师中不得不承认,韩冈的仪容气质是秦州难得一见的出众,即便是在人才济济的东京城里,也能排在前列。站在数以千计的围观百姓中,让人一眼就能看到他,有种鹤立鸡群的感觉。

李师中忽的自嘲而笑,再怎么说韩冈都是文武双全,智计心性皆为一流的士子,若是泯然众人,反而是个笑话了。

韩冈虽然站在人群的最前面,却一副懒怠困顿的样子,完全没有沾染到半点在周围人群中弥散的狂热或虔诚,这也是为什么李师中只一眼,就把他从千百人中认出来的原因所在。

——‘毕竟是张横渠的弟子。’李师中不禁感叹。

张载虽然官位不高,资历也远逊于李师中,却是天下闻名的鸿儒,对礼制自然早已融会贯通。如今的祭春仪式与古制大不相同,还有许多媚俗的改动,难怪承袭张载之教的韩冈,会当作笑话在看,全然不放在心上。

“难得的俊才啊……”李师中的感叹终于发出了声,引得站在他身边的几人看了过来。李师中眼神一凛,让他们立刻低头避过。

视线重又投到韩冈的身上。韩冈所修纂的伤病营制度规程,去年腊月初被呈了上来,放到了李师中的案头上。

李师中猜测韩冈也许是抱着‘宁厌之于繁,勿失之于简’的想法。他修纂的制度规程总计有六大项、七十余条细则,共两万多字,厚厚的一摞五六十页,如一卷书一般。那份制度规程中,从外部建筑到内部陈设,从日常饮食到伤患救护,从作息规则到安全保障,与伤病营相关的方方面面的细节都有涉猎。

李师中只是随手翻了一翻,单是字数就吓了他一跳。北宋与千年之后的时代不同,千字上下的文章才是普遍情况。过了万字,就号称万言书,不是普通读书人能信手写出来的。而韩冈只花了一个多月,便是两万字之多。而韩冈在扉页中还明确说明这只是试行条例,具体的条款要在试行的过程中逐步加以修订。

尽管这份规程看起来繁琐了一些,但每条每款都自有道理,无一条可删改。能把这些方面都考虑到,李师中只觉得韩冈根本不可能才十八岁,四十八岁的老行吏还差不多——将规程中涉及的各个方面的学问都融会贯通,而且还留有加以修改的余地,根本就不可能是一个还未有过任何实务经验的少年。

ps:立春鞭牛是个很有趣的祭典,从中也可以看出农业对古代中国的意义。

第二更,红票,收藏。

