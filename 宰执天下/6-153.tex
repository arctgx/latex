\section{第13章 晨奎错落天日近(18)}

“坐。”

“多谢大参。”

边让躬身拱手,然后诚惶诚恐的听命坐下。屁股只搭在交椅的边上,不敢坐实在了。

看着边让战战兢兢的样子,韩冈笑了一下,“安心坐好,你这样子也不好说话。”

边让依言挪了挪臀部,向后退了一点,但依然有一半悬空。

他很庆幸自己顶头上司生了重病,短时间内不能理事,这才让他这个顺丰行京师分号的二掌事有了出头的机会。

也正因如此,当今夜边让被传唤到参政府时,便处处小心谨慎,打算尽可能的利用这一次机会,博得韩冈的看重。

出于职业习惯,边让在进门时打量一下布局,想知道主人的喜好,以便加以迎合。

一国副相的书房十分简单,简单得与主人的身份并不相称。读书、写作和休息用的两侧偏厢,以及见客的中厅,因为是两层小楼,上面应该还有藏书室,这是普通官员和富裕人家都有的布置。

从外面就能看得出来,几个房间都不算大。中央的外厅陈设简单素净,甚至可说得上是寒素简薄,寻常官宦人家,见客用的厅堂很少会如此布置。东侧的偏厢中摆着书桌,不是见客的地方,但被领进这里的边让,也不够资格当参知政事的客人。

不过房间中放了一些很特别的东西。有几种精致器械,显微镜,望远镜,黄铜的罗盘,另外的就是各种石头,绝大多数边让都不认识,只有一块因为上面有着如同草丛一般伸出的六棱型晶柱,使得他能分辨出那应是水晶的原石。

剩下的就是书架了,有一面墙从上到下被书架占满,一格一格归类区分,只有一点很特别,书架上的书卷都是竖放,让书脊露在外面。不大的一个格子中,差不多就能摆放几十卷书。若连同上面的藏书室中也是如此放置,光是这一栋小楼内的藏书,就至少有上万卷了。

简朴而注重实用,这是韩冈书房给边让的感觉。这让边让不敢将一肚子的奉承话倒出来,紧紧记着谨言慎行四个字。

边让的一点小心思,自瞒不过韩冈的眼睛。

这位京师分号的副职与三国时的一位历史人物同名同姓,只是这位文采肯定是没有,商才倒是不差,否则也不会被冯从义任命为京城分号的二掌事,但他的交际能力,还没有更多的表现。

不过韩冈现在也不需要他表现什么,只需要一个合格的传声筒。

“分号中的事情,最近还忙得过来吗?”

边让正屏息静声的等待着,听到韩冈的问话,立刻就挺直腰道,“禀枢密,年节的时候,铺子里一向是比较清闲的。只是蹴鞠、赛马两大总社这几天都有例会要开,雍秦商会三天后还有一场宴席,本该由熊掌事去的;另外平安号那边也有宴席……”

“好了。”韩冈打断了边让的回话,“熊泉病好之前,他的担子先由你代他担着。那几处都是熟人,也不会故意找茬,相信你能处理好。”

韩冈的一句‘相信你能处理好’,边让的骨头都轻了几分,心中欣喜欲狂:“大参的赏识,小人感激涕零。只是小人见浅才薄,当不起大参的称赞。”

“哦?”韩冈抬起眼,盯着边让,“是谦虚,还是当真没把握?”

灯火在韩冈的双瞳中跳动,泛出冰冷的光色,边让登时冷汗涔涔,忙指天誓日,“大参放心,小人定会用心去做好!”

韩冈点头微笑:“我那表弟不会用没才干的人,尤其是京师要地。既然他用了你,我也是相信你能代熊泉将分号中的事情给处理好的。”

“小人明白,小人明白。”边让连连点头。

“生意上还有什么问题?”韩冈问道。

“一切安好。有参政看顾,有冯东主指派,又有曲、熊二掌事先后主持,分号的买卖一天比一点更好。就是河北那边……”

河东是独立的分号,韩冈在并代之地留下的痕迹很深,也让顺丰行能够深深的扎根在那里。但河北,由于份额太小,基本上就是让京师分号代管。宋辽边境的局势陷入了紧张,河东轮不到边让操心,而河北就不一样了。

边让很明白,顺丰行京师分号的大掌事并不是什么单纯的生意人。值此风高浪急的时候,韩冈招自己过府,也不会仅仅是想要了解一下买卖做得如何。

韩冈满意的点了点头,若这点眼色和政治嗅觉都没有,他真得考虑换人了。

“耶律乙辛篡位,朝廷不会与他再有任何瓜葛,岁币会断掉,边境上的榷场也会停掉。”

“朝廷要禁榷了?”边让一下抓住了关键。

韩冈道:“朝廷不可能与逆贼有任何往来的。”

边让点头,他完全明白了。

河北官宦豪门,于边境榷场上得利甚多。每年的岁币,都会经过边境上的诸多公私榷场,回流到大宋国内。尽管数额在整间商行的盈利中占有的比例很小,可顺丰行也的确在其中分了一杯羹。如今河北即将面临战乱,这当然是边让必须关心的重点。

而韩冈一直都在说朝廷不会跟篡位的耶律乙辛打交道,乃至要断绝与辽国的关系,这已经说明了一切。

处在韩冈的位置上,当然不能说公归公,私归私,上面跟辽国断交,下面可以照旧去与辽人做生意。

韩冈若是这么说,就是授人以柄,等着被新党拿着当成把柄来弹劾。现在的一番话,已经说得够明白了,这让边让知道自己在参加两大总社及雍秦商会的会议时,该如何表明自己的态度。

见边让已经明白,韩冈该说的都说完了,便示意边让可以走了。边让起身行礼,恭恭敬敬的退了出去。

韩冈端起还温热的茶盏,喝了一口。要得到河北和京城豪门的支持,这算是其中关键的一步。

之前已经见过了一些外客,边让算是最后一个,他走了,今天也没别的人要见了。

尽管三天后就是决定胜负的关键时刻,但韩冈并不打算像王安石一样召集门人一起议论如何应战。

并不是他自大,只不过他要做的准备,与王安石不一样。

“官人,时候不早了,明儿还要上朝吧?”

外面传来一声娇媚的呼唤,随即门帘被掀开,周南轻步走了进来。

昔日的花魁,随着年岁增长而愈见风韵,眼下正是最为娇艳的时候,举手抬足都能让人心弦怦动。

徐步走到韩冈身后,周南熟练地捶打起肩膀来。韩冈舒服的半眯起眼,一股淡雅独特的香气充满了嗅觉。

享受了一阵,韩冈开口问道:“你姐姐怎么样了?”

因为与王安石之间的矛盾陡然恶化,夫妻关系这段时间也变得紧张起来,每天见面也就两三句话。

韩冈知道自己的妻子夹在中间很难做,但在正事上,他不会因为妻子的缘故而退让半点,先不顾情面的也不是他。

不过多年夫妻,韩冈可做不到绝情绝性。

周南的拳头重了一点,“若官人能多陪陪姐姐,姐姐肯定会比现在好。”

“还是等这件事结束再说。”韩冈的口气有些不耐烦,不过随即又醒觉,叹道,“你们有空还是陪她多说说话。”

“何须官人说?云娘和素心都在陪着姐姐。”

“那就好。”韩冈稍稍放了点心下来,轻声道,“等过了这段时间,情况就会好点了。那时候,也能抽出时间,多陪陪你们。”

“官人也要说话算话才好。”

肩膀上的捶打换成了揉捏,从袖口中露出的半截皓腕如玉,在翡翠手镯的映衬下,更加显得白皙娇嫩。

韩冈低头看着桌上,如果顺利的话,眼下的麻烦,的确很快就能解决了。过些日子,外面的压力少了,便能多陪陪妻儿。家里的孩子渐渐长大了,要加强教育,自己不能注意盯着,指不定就是一批祸害百姓的纨绔子弟。韩冈一贯律己,也注重名声,他可不想丢这个人,被人说治家无能,想为人师表,却连儿子都教不好。

“官人?”周南见韩冈没了声息,轻声道。

“好了。”韩冈回手拍拍肩膀上的玉手,指尖的触感滑腻如脂,“明天还要上朝,现在梳洗一下就得去睡了。”

周南低下头,凑在韩冈的耳边,呵气如兰,“那就让奴家来服侍官人。”

韩冈手背抚着细腻的脸颊,笑道:“谁服侍谁还说不定呢。”

次日晨起,韩冈忍着困倦起身上朝。结束了垂拱殿的常朝,接下来便是内东门小殿的议事。

殿上的气氛有些紧张,毕竟再有几日便是决战之时,空气中都仿佛有刀剑交击的声音,充满了张力。幸而双方都保持着克制,没有在今日殿上就开始前哨战。

只是等到一切结束,太后已经准备起身,宰辅们也在王安石的带领下,准备退出内东门小殿,韩冈却没有动脚步,而是对太后道,“陛下,臣今日请留对。”
