\section{第22章 瞒天过海暗遣兵(九)}

【第二更,求红票,收藏】

“但眼下的情况又是什么怎么回事?!”智缘百思不得其解,凭着他对河湟局势的一点了解,以及吐蕃、党项当年的恩恩怨怨,怎么想,也不觉得木征会彻底倒向禹臧家,只是眼下的事情却是明摆着反常,“如若不是有着木征的准许,禹臧花麻的军队如何能穿过武胜军?”

“西夏如今声势正盛,三十万大军一齐南侵,五路皆遭攻打。如此风头火势,想来木征是不愿触这霉头,故而便为禹臧花麻让开一条路罢了。这些蕃人看起来势不两立,其实私下里有交情的不少。”韩冈想了想,又道:“今次当是木征和禹臧两家互不侵犯的默契而已,真正投效禹臧家的,还是星罗结部。”

智缘虽然年纪比韩冈长上一倍,但他还是第一次经历这样的场面,尽管才智绝高,但临战时的心性却还未见磨砺:“不过不是听说禹臧家的实力已经可以跟董毡、木征相抗衡了吗?木征把路让开,禹臧花麻就能全力攻打渭源。渭源堡中的军力能支撑的下?”

“大师不用太过忧心,渭源至今也没有点起烽火,可见情况还不算危急。”

一旦点燃了烽火,就等于向人公开自己的失败。消息传回秦州,传到京兆府,传到天子的案头上。不论最后的结果如何,王韶最重视的河湟开边少不得会被被人打上失败的烙印。除非城破在即,否则王韶绝不会这么做。韩冈对王韶的性格了若指掌,不过他欺智缘并不知道这一点,胡说八道也不怕被拆穿。

韩冈不再理会智缘的打岔,他追问着王惟新:“贼军兵力如何?”

王惟新立刻回道:“在渭源堡上看到的是六千左右,不过小人出城时,西贼虽然派人阻拦,却很容易就冲破了,看起来兵力并不足。”

在通报敌情时,惯常的是要往多里说。但这是对付上面的做法。夸大敌军实力,要是胜了,功劳会更多,若失败了,借口也很好找。不过王惟新知道是王韶的亲信,知道韩冈的重要性,不会在数目欺瞒他。

“才六千!”韩冈转头对智缘笑道,“大师你看,才六千人!”

“六千怎么了?”智缘问了一句,突然想到了答案,“是不是因为兵力太少,攻不下渭源?!”

韩冈点点头,道:“攻城兵力和守城兵力相当,而前面攻打星罗结部时,消耗的物资又少,要想守住渭源轻而易举。禹臧家本部中能征观战的精锐少说也有一万,加上附属部族的份,总计能到两万五千左右。如果必要时,把十五岁到六十岁的男丁一起征发,少说也能动员起超过八万以上的军团……”

“阿弥陀佛,竟然如此之多?!”智缘由衷惊叹了一句。

韩冈看事的角度与智缘却不相同,“能征调起八万大军的大族,却只有六千人抵达渭源堡下。从这里面就可以看出,不管时局怎么样发展,禹臧花麻都不会信任木征。就算木征借了道给他,他的至少还有一半以上的心力要放在背后,只能腾出一只手来攻打渭源。这样禹臧花麻可能胜吗?”

智缘和王惟新细细思忖韩冈的一番话,很快便心领神会的连连点头。

韩冈不想在道边久留,说不定再过一阵,禹臧家的游骑哨探就会流窜到这里。他对王惟新道:“王惟新,你们有紧急军情在身,我也不能多留你们。你等速去古渭寨,把渭源之事通禀给高钤辖。不过不要惊慌失措,照平常模样进城,不得泄露军机。”

王惟新连声应是,更不多话,向韩冈、智缘道别后,就利落的跳上马,带着七八名护卫急急往古渭寨去了。韩冈也跟着翻身上马,不再是往渭源去,而是跟着王惟新往东走。

“机宜,去哪里?”智缘并不觉得韩冈要回古渭,否则就跟王惟新一起走了,只是韩冈想做什么,他却弄不明白。

“去见瞎药。”韩冈骑在马上,手持马鞭指着东北方的山峦:“幸好王安抚没有点烽火,不然真不知道该怎么说服那头饿狼。”

………………

“禹臧花麻去攻打渭源堡了?!”

原本半躺在绒毯上,跟兄弟瞎吴叱一起喝酒吃肉的结吴延征,脸色大变。猛然坐了起来。手上的酒盏一下没拿稳,全都泼在了身上。冰冷的酒水顺着衣服渗了下去,可结吴延征还发着愣。

瞎吴叱不以为意,仍旧舒舒服服的躺着:“禹臧花麻借道的事有什么好奇怪的?过去禹臧家也没不是没有打过渭源去。通渭、古渭,北面的可都杀到那里去过。”

可结吴延征并不是为这件事吃惊。前日他的兄长木征派他出来前,叮嘱过他要盯着大来谷,还让他注意北面,难道是早就知道会有今天的局面?!

虽然结吴延征没有想通,木征是不是事先就看破了一切。但他已经明白了,前日禹臧家往河州派去使者,其目的并不是要说服他的长兄,禹臧家要招揽的,已经确定,要收买的,也已经完成。他们实际的用意不过是打个招呼而已,以防木征反应过度。

‘难怪大哥对那使者根本就不加理会,直接就打发出去了。’

结吴延征对木征的眼力敬佩不已,但眼下要做的,木征却没有给他指示。结吴延征问着瞎吴叱,“三哥,下面该怎么办?”

“还能怎么办?先看着再说。”瞎吴叱很轻松地说着,“如果王韶败了,就跟着禹臧花麻去渭源转转。”

“要是禹臧花麻败了呢?”结吴延征追问道。

瞎吴叱用金匕挑起一块羊肉,连汁带水的塞进嘴里,含糊不清的说道,“兰州是个好地方!”

‘果然如此!’

瞎吴叱的回答并没有出乎结吴延征的意料。说起来,他的几位兄弟之中,董裕的野心排第一,而瞎吴叱则能排在第二。别看他现在跟禹臧花麻好得跟兄弟一样,连禹臧花麻带兵过路都点头同意,可若是禹臧家不小心把软肋露出在外,第一个上去捅刀的,必然是他的这位三哥。

结吴延征这是突然又想起,如果他按着木征的吩咐,把瞎吴叱在岷州的地盘接收下来。那么,在他北面的就不只是兰州的禹臧家,更为接近的是控制了武胜军的瞎吴叱。

想到这里,结吴延征悚然一惊,木征要他小心的,究竟是谁?!

………………

“花麻,下面该怎么办?”

围住了星罗结城,围住了渭源堡,但接下来是猛攻还是围困,如果是要攻打,又该先攻哪一处。这些问题都需要新近成为禹臧家族长的禹臧花麻来决定。

‘怎么办……’禹臧花麻望着数百步外的渭源堡,皱眉想着。

今次宰相梁乙埋以举国之兵南下攻宋,收到命令的禹臧家也不得不应付一下,否则日后被秋后算帐,他的族长之位就很难坐稳了。

禹臧花麻一开始时只想表现得好一点,正好他跟别羌星罗结还有瞎吴叱都有些交情——尽管这种交情并不可靠,但用甜头喂饱了他们,也就变成了凡事好商量的生死之交了——他就想着在渭源堡下领军绕上一圈,也就算尽了人事。如果有空隙,还可以突袭一把,把王韶用来筑堡的钱粮军械都抢回去。

‘只是王韶的动作太快了。’

局势变化得超出了禹臧花麻的计算,联络好的别羌星罗结竟然在一日之间被灭族,惯用奇兵的王韶再一次大获全攻。但在这中间,他便看到了机会。

大胜之后,宋军必然松懈;而王韶既然分心攻打星罗结部,那渭源堡肯定没有修好;接着禹臧花麻又打探到,王韶回师时竟然还分了兵,将一千军队放在星罗结城,用来扫荡余部。如此良机,禹臧花麻当然不会放过。

此次南下,禹臧花麻总计带了一万一千多兵马,其中有四千守着大来谷,剩下的兵力中,大半围住了渭源,剩下的则是看守着。不过渭源堡下的几千人中,真正的精锐只有他亲领的五百精骑,剩下的都是附庸部族的人马。而围定星罗结城的军力,则是他禹臧家本部的精锐——以上驷克下驷的道理,即便是禹臧花麻这个蕃人,也能说出个道道——另外在兰州与武胜军交界的马衔山,另有三千人守着他的后路。

后路无忧,禹臧花麻唯一要担心的就是粮草问题。幸好听说了星罗结部被灭后,他就紧急跟瞎吴叱达成了新的协议。原本臣服于星罗结部的一众小部族,他会留给瞎吴叱,但这些小部族必须提供粮草给他,而瞎吴叱也得提供一部分粮食。

有这些压榨得来的粮草,足以支撑帐下大军的消耗,而不论渭源还是星罗结城,其城防的脆弱,即便是不擅攻城的蕃人,也没有太大的问题。

桃子就吊在眼前,只要伸手就能摘下。回头看着眼前一对对发亮的眼睛,禹臧花麻知道军心士气可用,他一甩马鞭,下令道:“这里我来盯着,你们先把星罗结城打下来。等合兵一处,便来攻打渭源堡!”

