\section{第32章 金城可在汉图中(20)}

望着空荡荡的太谷水,萧十三的牙都疼起来了。

萧十三是第一批南下的,并不是他喜欢身先士卒,而是这一回近三万大军先后南下,他不可能留在后方,必须坐镇在大军之中。

现在光是中军就有近五千骑兵,连人带马挤满了太谷水的岸边。可奔行了几十里,人困马乏,却竟然没人敢去喝上一口河水的。堂堂契丹勇士,过条不没膝盖的小河都战战兢兢,仿佛河里流淌的不是河水,而是毒液。

毒是肯定不会有毒,一开始也是有人先喝了水,饮了马,才知道河水不对。但水里不干净,三五日后疾疫发作怎么办?

虽然看起来不过是上游堆了粪尿下来,使得水的味道不对,但实际上,谁知道韩冈在里面做了什么手脚。萧十三信神佛,但不信韩冈能有什么法力。可药王弟子做过手脚的水谁敢喝?萧十三自己也怕得病。而河水如此,井水就更不用想了。

“枢密。”萧十三的一名汉人幕僚走近了一点,用着献宝的口气:“小人看过韩冈的书,再不干净的水,烧开了就没事了,实在不行,只喝水汽凝结后的蒸馏水就绝不会有事。”

萧十三阴阴的扫了他一眼:“这里有三万儿郎,你去哪里找那么多柴草来煮水?就算人够喝了,马怎么办?!”

瞪走了自作聪明的幕僚,一名专责传令联络的亲将骑马奔来,“禀枢密,附近几条村子的水井都没有填,可全都倒了粪尿进去。”

果不其然的印证了心中猜测,萧十三低低骂了一句,然后立刻下令:“传令下去,切不可去喝井中的水。”

亲将应声行礼,转身上马走了。

萧十三脸色更加阴沉,甚至气得心口疼。这比用石头沙土填起来更麻烦。填起来的井不难重新掘开,但被污染的水井就不可能再利用了,甚至联通的水脉都会被污染。从现在的情况看,十里之内别想找到干净的水源了。

前几天打草谷的时候,在太原城周围的村里面,村民不过是将水井用土石给填塞起来,就是在韩冈的《御寇备要》中也是这么写,白纸黑字的许诺,填了多少口井,官府就会帮着补上多少口。

当时萧十三和张孝杰只是在感叹宋人的财大气粗,‘甚至都不要亲自派人去挖,一眼井补偿个十贯钱就已经很多了,最多不过十万贯而已,比起一场胜利又能算得了什么?宋人花得起!’张孝杰当时是这么说,不过两人都没放在心上,虽然他们没钱,可手下有几万精壮,一起动手将填起的水井挖开来也不会太费时间。

都是韩冈的书害的。

不仅仅是因为他四处散发那些小册子,让他出镇河东的消息完全瞒不了下面的士兵。而且那本小册子里面说了条条款款,竟然半个字没提用粪尿污水的话。看了书,又看到宋人一切依照书中所载行事,几次下来也就视若平常了,却忘了多想一点。

且韩冈的手段是不是仅此而已,萧十三更不敢断言。韩冈硬是以己为饵,他的计划会就这么简单?!

“枢密,怎么办?”几名将领已经先一步聚了过来,没了水解渴,人人火气上头的模样,“这水没得喝,可是能要命的。”

“人还好说,大部分水囊里还有些水,忍着个一天两天没问题。但马可不行,那不是骆驼。”

“什么怎么办?”萧十三怒声道,“水在哪里,粮草在哪里,我们就去哪里!”

“但那可不近……”

“再远也要先喝水!”

本来萧十三还说,如果是赶得及的话,甚至可以将汾州攻下来。这样一来,就能将宋国西军来援的道路堵上。但现在看来,光是一个太谷县就够让人头疼了。

不过就算为了水食往外退,也不可能放弃已经控制的城外村镇,必须在城下放上一支队伍,否则还没开仗被一番折腾,士气就完蛋了。

正在商议该怎么安排,又是一匹探马带着军情从远方赶来。

“盘陀的宋军出了谷口,开始北上了?!”心情刚刚平复下来的没多久的萧十三又是猛然一惊,“这么快?!”

在他的想法中,自家攻城三五日不下,师老兵疲,然后才是宋人援军杀出来捡个便宜的时候。当然,如果攻势猛烈,反过来就能逼得南面的宋军提前赶来救援。

但现在可还都没开始攻城。从距离和时间上看,根本是前锋刚到了太谷县,宋人就出动了。

萧十三钓过鱼,鱼刚动了钩子就提鱼竿可钓不上鱼来。但韩冈多精明的人,他会想不到这一点?还是说南面的宋军脱离了他的控制?

都不可能啊。萧十三不认为自己有那么好的运气。

既然韩冈偏偏这么做了,那么这里面肯定就有陷阱了。只是萧十三一时间就想到了很多可能,却无法确定是哪一种。

萧十三能想到的,他下面的将领们差不多也都能想得到。

“枢密,怎么办?”十几只眼睛望着萧十三,希望他能拿出一个主意。

肯定要让人去拦截,否则一旦让北上的宋军在近处扎营,这一回就不用打了,直接拔营回去吧。

深吸了一口气,萧十三平定了下来,他看看左右,“换个想法,韩冈既然做了这么多准备,肯定会以为我们无计可施。若是宋人都这么想,不是不可能将计就计。”

将领们听明白了,但萧十三的想法未免太冒险了。资格最老的一个试探的问道,“那我们该怎么做?”

萧十三断然道:“今夜就攻城。”

“夜攻?枢密,这可不容易!”

好几个将领摇头,白天攻城都难得很,更别说夜里了。

“对宋人来说更不好守!”萧十三双眼扫过麾下战将,在他的魄力下,没人敢于反对。

“把马先牵走就食,人留下。”萧十三说道,人能忍饥挨饿,能耐着渴,马不行,而且马匹对攻城没有太大的作用。逐水草离开,甚至还能给宋人以错觉。

“今夜就要破城!”比之前改了几个字,萧十三的语气更加坚定,“今夜就要破城!!”

……………………

拿着一柄银质小刀,切削着仍散发着热气、流淌着油汁的烤羊腿,韩信正张扬得笑着。

“这只羊不错,够嫩的啊!直娘贼的,在京里可吃不到这么肥这么嫩的定襄羊。”

“也是时候不好,弄只羊也得费一番手脚。等河东这边太平了,哥几个再请韩兄弟你到代州,太行山中的时鲜,又岂是定襄羊能比的?”

忻州被重重围困,但韩信却大模厮样的坐在城下的军营里。在他的面前,几名身穿铜色板甲的军官正陪着小心的咧嘴在笑,仿佛发自内心的关心韩信是否能吃得顺心畅意。

整间帐篷中,也只有在韩信身边的,前西陉寨主秦怀信的长子秦琬平静如常。不过韩信每次下刀切肉,总不忘分给秦琬一块,吃得嘴角流油,一点不比韩信要少。

“韩家兄弟。”秦琬跟韩信说话时半点没有衙内气。脸上的一道还没完全愈合的刀疤,甚至更是让他平添了几分匪气,“这一回可是多亏了你。”

“哥哥说哪儿的话。”几日功夫,韩信已经是跟秦琬称兄道弟的交情了,“我这也是狐假虎威,有着我家枢密的亲笔信,有几个还会跟魏泽一条路走到黑的?……各位哥哥说,俺说得是不是在理?”

几个军官自然是猛点着头,一片声的附和。

且不说韩信是宰相门下七品官,就是韩信他本人,也是武艺精强,胆识过人。之前出入忻州,坚定了城中稳守之心,之后只用了两日就在忻州左近的山里找到已经拉起一支队伍的秦琬。这份能耐,可谓是空空儿、聂隐娘一般的人物,岂是能以家奴视之?

对秦琬来说,自家安安稳稳的混入叛军营地去说降,光靠前西陉寨主的儿子的身份,那是远远不够。没有韩信他以韩枢副家人的身份佐证,拿出了盖着制置使大印的亲笔信,绝对做不到直接就说降了六个指挥使中的四个。

就着火堆一番吃喝,秦琬忽然放下酒碗抬起头,“魏丈人快到了吧?”

魏泽自从降了辽人之后,便在代州大肆搜刮民财,然后送到了辽军的营地里。原本就在辽人手中过了一遍筛子的代州百姓,又过了细细密密的一层纱,但凡有那么丁点油水都给刮出来了。

但最让人恨的,还是他将富户官宦家中有点颜色的女眷都给强抢了出来,加上一干官妓,全都献给了辽人。代州百姓对魏泽倍加称赞,说他做得一手好媒,是契丹的好丈人。不过辽人对他看重得很,在辽人的宰相耶律孝杰那里说话也很有分量。

“来了正好!”一个指挥使低低阴笑,“俺正愁没机会献份大礼给枢密相公。魏泽那个逆贼人头,也不知够不够入得了枢密相公的眼!”

“哥哥何必冒这个风险?不明正典刑,千刀万剐,怎么能让世人知道他犯下的罪过有多重?”

“……只是没功劳不好见枢密相公啊。”

“什么叫没功劳?幡然悔悟,这就是功劳。保住手上的兵也是功劳。等到我家枢密带兵打过来,直接从背后给辽狗来一下子,什么罪过都赎清了。须知夜长梦多,直接拖了人走最少麻烦的,否则时间一耽搁,让你我之事泄露出去,辽狗可就在营栅外啊!”

韩信放下了银刀羊腿,拿手巾擦了擦嘴,双眼扫视众人,正容道:“我家枢密也常说,做事最忌讳的一是贪大求全,贪心一起,原本拿在手中的好处也会丢个精光。二是就是凭空耽搁,一旦有空闲下来,就会开始胡思乱想,原本已经决定的事也会越想越觉得犹豫,最后便会一改再改,犹豫又犹豫,最后不了了之。议定了就去做,这样才是做事的套路!”

