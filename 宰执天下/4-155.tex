\section{第22章 汉唐旧疆终克复(上)}

【不好意思,迟了点】

韩冈的计划,章惇全程参与,他不必韩冈多加解释,但李宪、燕达对此并不了解,需要加以详细说明。

‘不。’韩冈看看李宪,他应该知道一点,毕竟整套计划虽然没有全数向天子汇报过,细节更是隐藏了许多,不过在韩冈,以及章惇上报的奏折中,其实也零零碎碎的说了许多。

要不然,安南经略招讨司在交趾一番行事,朝廷早就派人来阻止了,如何会到现在都是视而不见?赵顼遣人送来的密旨,也是说要以交州长治久安为目的,至于仁义道德什么的,暂时可以放在一边,等交趾人死光了再开始谈也来得及。

有了天子的支持,并不是说可以高枕无忧,毕竟有当年种谔收复绥德,拿着密旨行事,还是给枢密院给贬到了随州三四年的旧事。

不过章惇、韩冈都是文臣,只要政事堂那边不出问题,枢密院的话可以当做放屁都可以,顾着眼下就行。

眼下,就需要李宪、燕达的支持。而李宪则秉持了天子的旨意,必然不会加以反对。只要能向他说明清楚便可以。

“将交州的治所放在海门?”

李宪知道,现在放任溪洞蛮部在交州恣意妄为,大肆劫掠甚至杀戮,并不是要将好不容易才打下来的交州土地全都丢给这些蛮人。而是如同邕州的例子,设立诸多羁縻州,但朝廷在交趾必须得保留一块核心土地,以维系朝廷对交趾的控制。

而这些日子李宪跟章惇、韩冈共事,大体想法也是知道了少许,他们是希望能通过潜移默化,用三五十年时间,逐渐加强对交州的控制,进而将这块土地吞并下来,处置掉已经与中国离心离德的交趾人,也就是为了这个目的

这一方略的本源,就是熙河路。没有熙河路这些年来的成功经验,天子不会答应得如此爽快。

只是在江口的海门镇设立交州治所,岂不是说,日后朝廷要派官来治理交州,全都要走海路?

“因为走海路不用担心路上有阻。就算日后交州之中有人反叛,朝廷通过海路可以直接支援海门。若是当年党项李继迁【元昊祖父】反,割据银夏,围困灵州,朝廷无从援救,只能坐视李继迁盘踞兴灵。如果在升龙府设立治所,日后交人做反,朝廷就还要再劳师远征,从邕州一路打过来。岂有在海门的方便?”

章惇紧接着韩冈解释:“旧唐的安南都护府,就是在升龙府,当时就叫做交州城,但之后交人反乱,交州城陷落,而海门镇一直保持在中国手中,进而成为交州……啊,应该是叫行交州。”这个行,就是临时的意思。

“但海上风波险恶,日后都要是坐船上任,未免会有所损伤。”

燕达已经吃够了水上行军的苦头,在江湖上坐船,都差点会要人命,到了望不见陆地的沧海之上,纵使天下闻名的勇将,也是想想就感到不寒而栗。所以他很奇怪,韩冈也是关西出身,怎么就不见他怕水呢?

“走陆路难道就没有损伤?从钦州往海门走,贴着海岸走,也就两三日的水程。何况海上可没有这么多的蛇虫!”

章惇是福建浦城人,他对海路的认识当然远在燕达之上。章家是望族,有好几房做着海上贸易,赚的钱也不在少数。

“另外,就是在海门设立治所,开辟港口,甚至设立市舶司,朝廷、蛮部,还有商人都有利可图。不惧日后朝廷中有人因为‘徒耗钱粮’,而提议放弃交州。”韩冈微笑着说道。

燕达和李宪闻言都是神色一动,他们当然知道朝野内外有许多人都是反对开疆拓土,认为边疆之地得来无用,反而‘徒耗钱粮’。

既如关西的一些战略要地,就是因为支出多过收取的税赋,比如当年的绥德城、如今的罗兀城,都有人反对,上书要求朝廷放弃、甚至赐还给西夏——如今盘踞在西京的那些元老们,多有这样的想法。

两人都是在攻打交趾上出了死力,如今交趾克复,日后这份功劳就是他们在朝堂上的资本,但如果交州被放弃,所谓的资本也就变成了被攻击的弱点。即是两人再豁达,也是难以忍受的。

听到韩冈说起能让朝廷在交州有利可图——商人、蛮部什么的,他们并不在乎——两人一下都变得聚精会神起来,身子前倾,专心致志的聆听韩冈接下来要说的话语。

韩冈看见自己的话,终于吸引了两人的心神完全投入了进来,微微一笑:“想必逢辰、都知,这些天也都看到了,交趾的土地有多肥沃,一年两熟、三熟都是不在话下,更不需要精心料理。尤其是富良江两岸的平原,一旦开垦出来,就是几十万顷的良田,一年产粮足足能有千万石。而富良江上游的山中,巨木无数、出产丰富,都是北方急缺的商货。

可是要向将这些土产运出来,必须借助海路,陆路绝不可能。交州北方的那一片山地也都看到了,那样的山势,那样的道路,木材、粮食,这些货物如何能运出来?”

“粮食……只要有这一条就够了。”燕达长长的叹了一口气,“不过谁来开垦,靠着蛮部可不保险。”

“难道不能让他们来?”韩冈笑得眼睛眯起,看着阳光灿烂,但从他口中说出的话语却是阴森无情,“将交趾人全都送给他们使唤,并不是让他们自此无所事事,都是要有所指派的。什么能做,什么不能做,朝廷都会有所安排,可不是想怎么来就怎么来。要他们为朝廷种粮。旧年交趾人依仗土地肥沃,不事稼樯,都是漫种漫收,种到能填饱肚子就不再费力。而现在他们有溪洞蛮部盯着,还能再偷懒耍滑?”

章惇接口道:“交趾人的脚趾都被废了,除了耕地,也没别的事能做,蛮部当然不会白白养着一群吃白食的人,肯定会让他们做牛做马!交趾人拿着汉儿为奴为婢,这一次要让他们偿还一辈子,一代接着一代,永世不能翻身。”

章惇话声一停,韩冈便继续道:“这么多人力,这么多土地,日后就是大宋的粮仓。以大宋官军的声威,没有哪家蛮部敢于不顺服,凭海门一城,控制住整个交州也是轻而易举。且只要有利可图,商人必然会云集交州,海门镇并不需要朝廷驱动移民,也会自动吸引汉儿来此屯垦,时日一长,有了一两万户人口,交趾还能姓李而不姓赵吗?所以能说三五十年后,交州必将属于中国,这并不是胡乱臆测。”

李宪和燕达都陷入了沉思,章惇、韩冈两人的计划,看似狠辣,却是在考虑着几十年后的事,这份眼光、这份见识,能在这个年纪坐上这个位置,果然不是幸至,日后出入两府,也是情理中事。

而李宪久在宫中,这两年也多次前往熙河路,对于熙河路的经济发展,比起在秦凤任职的燕达更为熟悉。韩冈和章惇的计划,都是以熙河路为本,出谋划策的当是韩冈,而不是章惇。

这些年的熙河路,哪一家蕃部不是靠着贸易,赚到了过去想也不敢想的财富?几十万贯都算不上富了,几个大族的族长少说都是上百万贯身家。青唐部的包顺——改名前叫做俞龙珂的——能在蹴鞠联赛上一掷千金,还不是靠了茶马贸易、盐业以及棉花、油菜种植,进而暴富起来?

如韩家、王家、高家,当初占据的不要钱的荒地,如今地价都接近了秦州的平均水平,光是土地,就值十万贯以上。加上土地的出产,还有作坊、商行,那可是一个个都富得流油。要不是有太后家参与其中,加上王韶、韩冈的身份,这块肥肉不知会有多少人想来咬上一口。

有了特产、也富庶起来的熙河路,就像一块吸引蚂蚁的蜜糖,如同滚雪球一般,每年都能有数千人怀着一载暴富的心思,来这片梦想与传说中的土地。

如果交州能与熙河路一样,有了特产,有了财富,必然会有人不顾性命来此博一个富贵。这是人之常情,并不需要朝廷严加指派。

李宪心头火热起来,如果交州能发展起来,成为皇宋不会放弃的土地,收复汉唐旧疆的功劳也自然不会失落,日后他在宫中的地位,也不会一直被王中正那个运气十足却没有什么能耐的同伴,压着一头去。

“那我们要怎么做?”燕达沉声问着。

“只要盯着蛮部,让他们不要偷懒耍滑就行了。”韩冈笑着说道。

燕达有些疑惑的看着章惇,韩冈的话说得有些不明不白。

“先把升龙府毁掉,拆掉城墙,烧掉屋舍,迁走人口,在海门建立起新城。”章惇冷然说着,这是他和韩冈的计划第一步,“除了海门以外,交趾不需要别的城市!”

