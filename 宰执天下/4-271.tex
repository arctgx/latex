\section{第43章 庙堂垂衣天宇泰(18)}

“天子最怕什么?帝统有失,皇嗣不继。”富弼端坐着,自问自答,“就只剩一个儿子,天子还能得罪发明了产钳和种痘的韩冈?”

“韩冈难道事先算到建国公会出事?”富绍庭从他父亲的话中深思下去,再将两件事联系起来,脊背有些发冷,“时间上也太巧了。”

“谁知道呢?”富弼摇了摇头,道:“不过七个里面剩两个与七个里面剩一个有区别吗?”

富绍庭抿抿嘴。的确没有大的区别。从一个甲子以来,皇宫中的历史来看,加起来才三周岁的两位皇子,长成人的几率,与一位皇子是一样髙,也可以说是一样低,反正都是零。

“没能及时赶上救治建国公,天子恨韩冈是人之常情,虽没道理,却是免不了的事。有了,但他他还要谢韩冈,让均国公不用担心痘疮。否则光是痘疮,就很可能让两个皇子都夭亡。”富弼,“不靠韩冈,基本上一个都很难养活。但依靠韩冈,多半还能保全一人。”

“种痘已经出来了,要韩冈还有什么用?”

“过河拆桥?”富弼嗤笑一声,“韩冈拿出产钳的时候,没人知道他会种痘。韩冈拿出种痘的之后,你能保证韩冈没有其他更为高明的医术?你说只剩一名皇子的天子,会不会使性子去赌?……我告诉你,怎么都不会赌的,连逼问都不敢。”

“那可是天子啊。”富绍庭咕哝着。

“天子?……富笃!”富弼突然冲外面叫了一声,将服侍他的老管家叫了进来,“早上我问你的话,你再跟大郎说一遍。”

富家的老管家问道:“就是小韩学士的事?”

“没错。你说一说外面怎么传的韩冈。”

“外面都说小韩学士是得了孙真人的真传,制产钳,种痘苗,救治天下小儿;还有说小韩学士是药师王佛座下弟子,又受了观世音菩萨的托付,出世抚保小儿。现在外面有人从转运司衙门里弄来了小韩学士的签押,说是烧成灰之后,和水服了,能安胎。”

咳,富绍庭突然咳嗽起来,拳头抵着嘴,掩饰自己的笑意。

富弼没有笑,挥挥手让富笃下去了。

“你笑世人,韩冈笑你。你们都给韩冈糊弄了。”富弼因老迈而浑浊的双眼,是看透世情的锐利老辣,“如果从来没有读过《浮力追源》,对飞船飞天的道理全然不知,突然看到一艘飞船载了人在天上,你会怎么想?”

富绍庭哑然,不用说的,肯定是往神仙妖魔上靠。

“韩冈如果不将飞船、种痘说通说透,朝堂上没他站的位子。换个手法,就是太平道、弥勒教,能骗下不知多少愚夫愚妇,士人也会为他所欺,午门外的一把刀少不了他。但韩冈将原理一说,再跟儒门扯上关联,所有士大夫都觉得平常了——只要多看多想,就是凡事多格一格,其实自己也能想得通。”富弼垂下来的银须,掩住了嘴角的讽刺,“士人多自傲,慢公卿、傲王侯,看到韩冈能做到,多半会觉得我也行,是也不是?”

富绍庭脸红了一下,他是洛阳城中最早得到显微镜中的一人,颇费了点周折才弄到手。这两天,听说了种痘之事后,他将显微镜摆弄来摆弄去,就是想着也能有所发现。

比起与狐朋狗友聚在一起饮宴狎妓、大吃大喝,做一些让人羞愧的诗词附庸风雅,带着子侄在读书之余,观察泥土中的细小生命,绘制最精细的虫豸的图形,与同好们聊着树叶上的脉络,水中的微虫,反倒更有意思的。同时,如何能让显微镜的放大效果更出色,他跟几个朋友也召集了好些工匠来试验。

富弼瞅了长子一眼。他对自己儿子还有几个孙子的爱好心知肚明,虽然摆弄显微镜也花钱,可比之饮宴要便宜得多,心中还是比较支持的。

“对韩冈的成就不以为然,这其实也是人之常情。”富弼又开口,“离得远,自然是敬畏不已。可一旦离得近了,反而就觉得平常了。”

富绍庭看着自己的父亲眼望窗外,心道多半不是在说韩冈,而是在说皇帝。

富弼轻咳一声:“韩冈由人痘发明牛痘,如果他只说牛痘的事,不一定会有今天的麻烦,天子只会为建国公惋惜,不会心存芥蒂。但他偏偏将那位孙道士扯了出来,为什么?得了仙授良方,用了十年找到了比仙方更好地方子,他能做到的,世人也能做到。他能超越仙人,世人当然也能。从韩冈过去的行事来看,恐怕他就是希望士大夫们能这么想的。”

“为什么?”富绍庭很惊讶,韩冈绕来绕去,对他自己有什么好处。

“当是为气学吧。”富弼略皱眉,疑惑的口气有几分不确定。前面对韩冈的猜测,他其实也没把握。

摇了摇头,回到原来的话题:“士大夫都在韩冈的解说下,对飞船、种痘等事都看透了,明白是格物的结果。但百姓呢,他们会怎么想?你们有没有想过?……除非想跟韩冈结死仇,否则士大夫当都是嘲笑世人多愚,以深悉其理而自傲。所以说韩冈聪明啊……”富弼看儿子的目光是恨铁不成钢的无奈:“诳的你们所有人以为能跟他一样聪明。让天下士绅‘聪明’到看不清种痘法对黎庶们意味着什么?想不到韩冈现在在百姓们心中又是什么身份?”

富绍庭很是有些难堪,但他还是想不通。“这跟天子要维护韩冈有何瓜葛?”

“不是天子,而是宫中。宫中能有士大夫的见识和性子吗?妇寺之辈,看韩冈倒是跟外面差不多。不管传说是真是假,水快没顶了,一根稻草都有人抓。病急乱投医,何况韩冈还有那么多成绩在?”

富绍庭眉头皱了半天,突然瞪大了眼睛,惊畏之情也随之缠住了心脏。

“天子已经三十岁了,唯一的皇子才三岁,身体还不好。”富弼深吸一口气,摇着头叹出来,“不是人人都有真宗的运气。”

仁宗皇帝是真宗四十过后所生,当时诸兄皆夭,是独生子。原本真宗都以绿车旄节迎濮安懿王入宫抚养,准备养为嗣子,仁宗出生后,才箫韶部乐送还府邸。但仁宗皇帝就没有这份运气了,儿子生一个死一个,最后没办法了,才从濮安懿王赵允让那里抱了排行十三的英宗赵曙回来。

“三十过后,子嗣是越来越难生。当今皇帝身体又不好,为了儿子旦旦而伐,日夜操劳,不见得能过五旬。万一六皇子均国公再出了事,想四十多岁生个嗣子出来,真得要祖宗保佑了。以前车为鉴,当今天子难道还想再弄出一个濮议之争来?”富弼冷笑,“也许应该叫雍议才是。”

“雍议……雍王?!”富绍庭脑筋转了一圈才想通。

“还不一定只是从雍王那里抱个儿子那么简单。万一今上天不假年,有保慈宫中主持,立长君也不是不可能的。”富弼眯起眼,“二大王即位,后妃们还有立足之地吗?想想太宗皇帝是怎么待孝章皇后的,向皇后不会不知道。就算天子要治罪韩冈,除了刑婉仪这样病夭皇嗣的嫔妃,其他哪个会支持?生了皇六子的朱贤妃不用说,就是向皇后,也会拼了命的要把皇帝劝住!又不是亲生儿子,死了也不见得有多伤心,只要能保着一个庶子登基,她就是太后。换做是雍王即位如何?”

富绍庭听得直冒冷汗,要不是在家中书房里,他都要夺门而出了。

富弼根本不怕。雪夜看禁书,这是很痛快的一件事。在家里说些悖逆不道的话,也叫一个痛快。

富弼说得很开心。别说在家里,就是当着皇帝的面,犯忌的话他也不是没说过!

当年因为曹太皇和英宗之间的事,差点被韩琦和欧阳修害死,他积了一肚子火。年纪越大,当年的仇怨就积得越深,韩琦和欧阳修去世的时候,就富弼没有派人致礼、送上奠仪。

‘伊尹之事,臣能为之。’

伊尹什么人?殷商开国贤相,助汤建国。后商汤驾崩,其子太甲为君无道,伊尹便放逐太甲于桐宫,三年后见其悔改,才将之迎回——这是如今世上对上古历史的主流观点——他与废立天子的西汉权臣霍光是一向是被并称为伊霍。

曹太后对两府哭诉皇帝不孝,韩琦打个哈哈随口劝了两句当放屁,富弼可是冲着英宗这般出言威胁:不孝顺点,直接废了你。结果怎么样,每到富弼生辰,来自庆寿宫的赏赐最多,不是没有理由的。对比起来,韩冈献上种痘法迟了一步,又能算是什么罪名?

“当然,雍王即位的可能性的确不大。但以均国公的身子骨,天子肯定还是得想一想之后的事。”富弼扯着胡须,“从天子这边来考虑考虑,惩处了韩冈倒是不难,找个罪名发去远恶军州做个十年八年的知军州事,愿意为天子出口气的多得很,那几个御史不就是如此。说不定以韩冈的才干,还能让个没产出的下州转成富庶之地,生民安居乐业。可少了韩冈的一份力,万一绝嗣了怎么办?……过继吗?”

富绍庭沉默着,谁都知道过继的坏处。

