\section{第13章 晨奎错落天日近(21)}

简短的交流之后,韩冈和章惇便分了开来。

方兴远远看着章惇面无表情的离开,不由嘴角微扬。

章惇为新党统领枢密院,地位非曾孝宽可比。站在方兴的角度,恨不得马上就让章惇被贬出京城去,让韩冈的盟友苏颂掌管枢密院。

可惜章惇在西府多年,地位坚不可摧,方兴每隔几天都不得不在去了政事堂禀报日常公事之后,再去西府一趟,将同样的内容在章惇面前再重复一遍。而每一次,他都被章惇寻根究底问得很辛苦。

章惇在韩冈面前讨了个没趣,这不仅仅解气,也能看得见韩冈的信心。

不用说,肯定是说服了太后,否则哪来的信心。

直接在例会后自请留对,当真是神来之笔。

方兴心中暗赞不已,看看这一回,还有多少人站在王安石和章惇那边?

“方丞。”

“监丞。”

皇城门前,上来与方兴见礼的同僚络绎不绝。非是进士出身,却在主要由进士组成的朝官圈子里面混得如鱼得水。

方兴身上贴着韩家门下走马狗的标签,纵然品级还不高,也还掌握着至关重要的一个部门,没有多少朝臣敢于在他面前拿大,在同僚中,也是众星捧月一般。

而此刻方兴脸上的表情,让很多有心人看在眼里,也让他比往日更受关注。

……………………

“韩三必败!”

判都水监杨汲语气坚定,仿佛在陈述事实,“国运日渐昌盛,有识之人谁会去贸然改变成法?”

“还用说?”吕嘉问也满是自信,“一年多前的第一次廷推上,我等三人均分票数,就这样,他也才赢了一票啊……”

吕嘉问低沉的声调,强调着韩冈当初只赢了一票的事实。今日再无三人来平分选票,韩冈要一人独挡满朝遍野的新党,他如何能做到?

“韩冈惯于行险,行事往往剑走偏锋。”杨汲盯了韩冈的背影一眼,“但凡他有些耐心,何至于今日走到鱼死网破的境地。”

“今日且观他自取其败。”吕嘉问对杨汲道:“潜古,今日事了,戎监之中,可就要劳烦潜古多费心了。”

“军器监事务繁剧,远在都水监之上。汲乃斗屑之材,才不及任,惟蒙平章不弃,也只能竭尽全力了。”

“以潜古之材,何须如此自谦?”

吕嘉问哈哈笑了两声,与杨汲作别,招呼着刚刚走过前面的龙图阁直学士、判审刑院安焘,追了上去。

与吕嘉问分开,杨汲略嫌做作的激昂之色阴沉下来。

韩冈若没有把握,何至于如此行险?又何必如此匆忙?

深深的盯了吕嘉问的背影一眼,又望了望正往待漏院中去的章惇,他摇了摇头。

不远处,刚刚回京诣阙的知永兴军王存,望着阴沉下来的杨汲也若有所思。

吕嘉问的忙忙碌碌,分外体现了他的心虚。就连曾经支持过他的杨汲,都要在皇城外再叮嘱一遍,这样可让人很难对他有信心。

杨汲……王存屈了一下手指……这又是一个不安稳的。

今日之前,王存计算过好几次。在共商国是的会议上,的确是新党占据了绝对性的优势,但今天起来,看到了章惇、吕嘉问和杨汲,原来的笃定,现在已经不是那么有把握了。

再想一想,上一次推举曾、李、吕三人与韩冈竞争的重臣们,在这一年的时间里,已经有三分之一离开了朝堂。

依照比例,这算是正常的人事调动。因为当日推荐韩冈的成员,如范纯仁、李常等人,都不在京中,占了一半还多。但那几位只是为了对抗新党才站在韩冈一边,并非认同韩冈和他的学术,将他们调走,是韩冈本人的意见。而新党重臣向京外迁任,太后的主张至少占了一半。

太后对新党的反感显而易见,这一回,太后明显站在韩冈一边,否则前日的留对后不会连一点有价值的消息都传不出来。而新党这一边,除了通过一党合力,来压制太后,逼她按照之前议定的法度来施行。殊不知太后纵使一时屈服了,可在心里面,难道不会记上一笔?

该还的人情差不多都已经还清了,现在没有必要再去趟浑水。

“正仲!”

王存正想着,身后传来一个熟悉的声音,回过头来,只见曾孝宽正拱手为礼。

王存嘴角微微翘了起来,躬身回礼:“令绰,有礼了。”

……………………

“真够忙的。”

看着吕嘉问如穿花蝴蝶一般,在人群中左右穿梭,李常冷然说道。

李常昨天入夜前刚刚回京,因为河北边事,而被招入京中。主掌河北一路军事的吕惠卿不能轻动,那么掌管钱粮转运的李常就不能继续再安坐大名府。而这一次的协商会议,正好卡在了他回京的第二天。

就在日前刚刚度过白马渡,抵达黄河南岸的时候,李常就已经从韩冈派来的信使口中,得到了第一手的消息。今日来到宣德门外,就已经感觉到了一股山雨欲来风满楼的前奏。

就像夏日暴雨前的那段时间,深藏在池中的虫、鱼,纷纷钻出了水面来透气。

“心虚之故也。”韩冈笑道。

“或许。”李常应了一声。

李常绝口不问韩冈前日留对,在太后面前说了什么?有些事,他不便多问。他与韩冈只是因为反对新党才开始合作,说亲近,其实也算不上亲近。

韩冈如果想说,那就自然会说的。如果韩冈不想,问了只会平添尴尬。但有些事,还是可以问的。

“玉昆,你有多少把握?”李常问道。

“有一多半,六七成的样子,要不然,也不会我也不敢如此提议。”韩冈笑了一笑,“剩下的,就是尽人事听天命了。”

“天命?”李常皱眉,他不喜欢这样的说辞。

“天命?!”

尖锐的声音来自两人身后。

两人同时回头,李常看见了曾经共同支持韩冈晋身两府的李承之。

“奉世。”

韩冈与李常同时行礼,但李承之没有回礼,而是十分急切的上前一步,厉声质问:“天命?!”

李常终于反应过来了,回头剔起眼盯住韩冈。

“天命!”韩冈点头。

“如此最好。”李承之放心的点头,长长出了一口气,六七成的把握,再加上‘天命’,岂不是十成十了,至少也会有**成了。

李承之安心下来,韩冈的回答,以及之前的问对,让他不再担心。

李承之已彻底投靠了韩冈,现在任职审官西院,掌握了中层武臣的升迁任免。而他之前在河北的漕司一职,正是交给了李常。

韩冈没有跟韩绛、张璪去争夺主管文官升迁的审官东院和流内铨,而铨曹四选中的西班,决定中低层武官命运的审官西院和三班院,则成了韩冈的囊中之物。

判审官西院是李承之,而三班院他则是推荐了熟悉军事的游师雄,他在甘凉已经有好几年了,也该调回来了,只是这个推荐时日不久,恐怕消息才刚刚传到凉州。

不过李承之并没有在审官西院久居的打算,三司、开封府、翰林学士院、直至枢密院、中书门下,都是他的目标。只要这一回,新党败阵,那么,就是他享受胜利果实的时候了。

李常多少还有些疑虑,但他也不再多问,剩下的还是去看实际。他四面张望了一下,发现很多人挪开了视线。李常摇头一笑,王、韩二翁婿对决在即,韩冈现在的举动,自是众人关注的焦点。也不知道,韩冈此时的一副胸有成竹的表情,能让多少人改变立场。

南方的御道上,这时候有了一些小小的混乱,拥堵的人群纷纷散开,李常双眉一展:“玉昆,令岳到了。”

……………………

赶在宣德门开的前一刻,王安石终于到了。

喧闹的人声渐渐平息,千百道视线集中到了马背上的那位老者的身上。

可万众瞩目的焦点,却明显没有睡好的样子,衣袍一如往常的陈旧掉色,而脸色则是有些灰败,黝黑的面庞上有着很明显的疲惫。

王安石竭力做出恍若无事的样子,但熟悉他的人都能看得出来,王安石的心情看起来与他的脸色一样糟糕,甚至因为糟糕的脸色,更增添了几分老态。

宰相可以驭马进皇城,王安石来到宣德门下,也没有选择下马。

朝臣们纷纷向着王安石行礼,并退让到道边,就连在待漏院中的官员也全都出来了,上至宰辅,下至卑官,无不对平章军国重事的王安石躬身作揖。

韩冈也想着王安石行礼,王安石居高临下的盯了韩冈一下,却是连招呼都不打就走过去了。

刚刚从待漏院中出来,章惇看到了这一幕,脸色陡然一变,以王安石的性格,正常情况下怎么会这般无礼?那还是他的女婿,他女儿的丈夫,过去再如何对韩冈动怒,也不曾见今日愤怒。

章惇心中敞亮,王安石如此失态,事情绝不简单。

走了两步,来到韩冈的身边,他今日第二次向韩冈询问:“玉昆,你做了什么?!”

“昨夜让人送了一封信给家岳。”韩冈毫不讳言。

章惇神色一凛,追问道:“写了什么?”

“不是我写了什么,而是吕吉甫写了什么。”

章惇知道吕惠卿给韩冈写信的事,向韩冈推荐了两人,这是前几天韩冈亲口告诉他的。不过当时韩冈并不打算将这封信透露给王安石,说是为了家中安定。但是现在,韩冈却说,他昨夜送了一封信给王安石。

章惇能够想象得出,拿到韩冈送来的信后,王安石会是什么样的心情?

不,完全不需要想象,看王安石的现状就已经够了。

都是要图穷匕见的时候,也没必要再遮着掩着,该拿出来便很干脆的拿了出来。

章惇冷下了脸:“玉昆,没想到你会这样做。”

就在不久之前,韩冈还口口声声怕家里添乱,才不愿将这封信送去给王安石。可现在却把之前的话抛到了脑后,毫不。

“快刀斩乱麻罢了。”韩冈微微笑,“有些事必须要下决断了,不是吗,子厚兄?”

……………………

“平章。”

沉默的从韩冈身旁走开,章惇来到王安石身边,小心翼翼的观察着王安石的脸色。

王安石已翻身下马,脸上的皱纹更加深刻,但腰背一如往常的挺直。

“子厚,没关系。”

一直以来,章惇都不似吕惠卿和曾布那般为王安石所重视。对章惇,是可以重用的助手,而吕惠卿和曾布,则还要加一条——是能传承大道的学生。

所以吕惠卿、曾布,能够在朝堂中一路升迁上去,成为赤帜之下,处理实际事务的第二号人物,而章惇,则必须先去荆南平定蛮夷,以博取功劳。

但现在,曾布叛离了,吕惠卿也背地里与对手暗通款曲,尽管章惇也跟韩冈有所牵扯,还差点为韩冈所说服,可在王安石眼中,吕、章两人与他的感情终究不同。

这一次的打击,王安石心中决不好受。

但王安石的双唇紧紧抿着,拗相公的倔强清晰分明的展露出来。

钟声响起,炮声轰鸣,城头上烟雾缭绕,喧嚣中,宣德门的城楼缓缓开启。

“没关系的,子厚。”

王安石说着,向上望了一眼,然后驱马进入皇城。

……………………

“太后,时间差不多了。”

随着王中正在耳边提醒,向太后缓缓站了起来。

尽管今日只是一次既非正旦、冬至,又非朔日、望日的普通朝会的日子,但却决定了大宋的未来。

三两人谋于私室,不若开诚布公,让士大夫一同来计议。

这也能避免日后会变成熙宁初年的那种乱局,十几年前,她是亲眼见证了新法造成了怎样的乱象,让自己的丈夫夙夜忧叹,十几年后,处在当年丈夫的位置上,她不想重蹈覆辙。



