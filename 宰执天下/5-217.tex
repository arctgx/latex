\section{第24章 缭垣斜压紫云低(九)}

茫然看着祝酒的金杯从手中滑落,赵顼一时间不明白到底发生了什么,心中满是困惑。

但在赵顼身边担任宿卫和引导的王中正和石得一两人的眼里,天子的脸在陡然间变得僵硬,变得怪异而扭曲,最后定格在一种让两人毛骨悚然的神情。

“官家。”石得一抢前一步,弯腰捡起金杯,凑近了观察赵顼的神色。

只是石得一这一看,顿时就是分开八块顶阳骨,一盆冰水浇下来,从头顶冷到了脚跟。只觉得整条脊梁骨都像是变成了冰柱一般。正捡拾起金杯的手也像是抽了筋,刚刚拿起来的金杯,又砰地一声落到地。

赵顼眼神中透着惶惑,为什么眼前变得模糊起来,为什么有人就在身边说话,却听不清他们到底的再说些什么。

大宋天子的嘴张了开来,双唇哆嗦着,想说些什么,却是一个字也说不出来,仅仅是在喉间发出暗哑的咕哝。

耳边有如蚊蝇环绕,抢到近前的两人似乎是石得一和王中正,但眼前就像是蒙了一层纱,也分辨不清到底是是不是他们。

难道是中风?!

赵顼渐渐变得浑浊的头脑中,却有一道灵光闪过,终于想明白了到底出了什么事。只是赵顼宁可自己没有想明白。

眼前的视野忽然歪斜,赵顼并没有感觉到自己失去了平衡,可越来越近的地面清楚地告诉他,自己的确是摔倒了。

当赵顼从御榻翻倒的时候,殿下的朝臣们终于觉察到大事不妙。并不是金杯脱手的小小意外,而是很可能是要人性命的重症。

殿一时间没了杂音,文武百官连大气也不敢喘,只是紧张的望着台陛的天子。

还能将郊祀后的宫宴主持下去吗?

心中的恐惧如同潮水一般涌来,想将他埋入黑暗之中。赵顼的意识拼命的挣扎着。可他的挣扎,就像是陷入了蛛网的飞虫,完全没有达到应有的目的。赵顼并不是在一瞬间就失去意识,而是清晰感受到自己的身体已经无法控制,在明白了自己到底出了什么事的情况下,意识才一点点的开始模糊起来,只有对死亡恐惧留存。

被王中正扶住的天子,看模样已经不可能继续方才的任务。王中正和石得一对视一眼,对方的想法都已经了然于心。

“扶官家回内殿。”石得一说道。而不论是王中正,还是其他服侍在侧的内侍,完全没有反对的意见。

赵顼被搀扶进去的那一刻,让所有在场的官员都感到风雨欲来的危机感,极浓极重。不止一人将视线投向赵顼的两个亲弟弟。赵頵倒也罢了,跟其他望着内殿的官员差不多的反应。赵颢低头看着眼前的桌面,动也不动一下。可任谁也知道,他心里面还不知如何敲锣打鼓,兴奋得无以名状。

宴会怎么办?

天子还没有让皇子出来奉酒,预定中的程序没有完成,那么请皇子出阁读的奏章到底要不要递去?已经有很多人开始犹豫了。

如果皇帝还能恢复,肯定不会有人起异心。但赵顼的病可不是感冒发烧那般轻易,几乎是无药可救的,让人们没有了太多的顾忌。

手足麻痹,口不能言,这是典型的中风症状。

在赵家的前五位天子中,因风疾而不能理事的不是一个两个。真宗、仁宗、英宗,都是风疾而沉疴不起。

大庆殿中的文武百官里面,深悉医理的至少有十分之一,具备些许基本的医学常识的则能有一半。而什么是中风,几乎每一个人都有这个见识。

天子的离席,不仅仅是给朝堂蒙一层阴影那么简单了。

很多人还记得,就在几年前,当今的皇帝似乎曾经有过一次疑似中风的发病。一次中风还不一定致命,但两次、三次中风,可就跟一道道走过鬼门关一样,鲜有能撑过去的。

宴会的主人离开了,剩下的客人全都陷入了。这个时候,宰相应该站出来收拾局面了。章敦盯着斜对面的王珪,打着眼神催促王珪。但王珪根本就不跟其他人对眼,只顾伸长脖子望着通往内殿的小门。

章敦狠狠地咬紧牙。不能挺身而出,稳定局面,这还配做宰相吗?换做是自己,绝对不会做出这样的蠢事。

不知过了多久,内殿中终于有了动静,王中正匆匆从殿中出来,站到台陛下,“太后有旨,着王珪主席。”

王中正再没有别的话,王珪起身领命。有了吩咐,他就敢做事了。

只有王珪的主持,自然不可能让延安郡王赵佣出来面见朝臣。只用了小半个时辰,一场耗费巨大人数众多的宫宴便匆匆结束,臣子们从大庆殿中鱼贯而出。

但解散了宫宴,却并不是所有人都离开了皇城。原本宫宴结束后官员们就该四散返家——在开封,冬至日是一年中仅次于正旦的大节日,就算皇帝也不便耽搁臣僚们想早点回家与家人相聚的心思——可是今天却有许多人因为各种各样的理由想要等个结果而滞留在皇城中。

皇城中官员们的神色,完美的诠释了什么叫做人心惶惶。天子到底能不能撑过去,可是事关他们命运和前途的关键。

韩冈也没有离开皇城,而是直接返回了太常寺。太医局中的几名医官都已经被召去了福宁殿,为天子诊治。要有什么消息,这里是消息灵通仅次于两府的地方。而且韩冈相信,他肯定会被召进内宫,在太常寺这边等着最合适。

从架抽出一部有关生物学的科普读物的手稿,韩冈气定神闲的校对起来。中风不是心脏病,就算一病不起,至少也有两三天的时间做缓冲,总会有办法让局面不至于落到最坏的地步。

也正如韩冈所预料,刚刚坐下来没半个时辰,宫内派了人出来,请韩冈入宫中。来人是赵顼身边的内侍,虽然名字不清楚,但相貌很面熟,这也让韩冈多放了点心。

在就在这名内侍的引领下,韩冈走进了天子的寝宫。

几十支儿臂粗细的蜡烛将福宁殿的外殿照得透亮,但不知为什么,走进来的时候,韩冈却觉得这里阴气逼人。

东西两府宰执一个不漏的全都聚集在福宁殿外殿中。以张守节为首的四名殿帅,还有福宁殿中有差事的大小内侍,就算不将殿外的班直算进来,一眼望过去也有二三十人之多。但偌大的殿堂,比夜漏更深时的古刹深处还要安静。这么多人,或坐或站,竟然连个开口说话的都没有。宛如木雕泥塑的偶像,

王珪、蔡确眼定定的望着内殿的门口。薛向和其他几名执政坐在椅子闭目养神。只有章敦背着手在踱来踱去——这个时候也不管什么规矩了。

当韩冈进来的时候,章敦首先看见了他,几步走过来。

“玉昆,可有什么良策医治中风?”

“太医局中,会治中风就那么几个,现在都已经在福宁殿中了。”

章敦闻言,叹了一口气,不再多问了。

着王珪等人也看到了韩冈,平常还能够问候一句两句,但现在却都没人有心情说两句废话。

论理韩冈是不够资格加入到两府重臣的行列中,但他的身份特殊,不说太医局、厚生司都在他的管辖之下,光是传言中药王弟子的身份,就足以让向皇后遣人将他招进宫中。

且不论召来韩冈到底有用没用,对于病人家属来说,看到药王弟子站在病床边,心理总能得到一点安慰。

对自己成了庙里神座的塑像——再难听点就是安慰剂——韩冈并没有在意太多,能在天子重病时走进福宁殿,就有影响甚至扭转局面的机会——不管这个机会有多小。无论如何,韩冈都不希望自己呆坐在家里等待局势的发展,最后被人通知朝,然后就看到雍王赵颢出现在大庆殿中的御座。

给韩冈领路的内侍先行进了内殿,没过片刻,他就又出来了,“端明,皇后有旨,诏端明入内殿说话。”

韩冈只是稍稍犹豫了一下,便跟随内侍跨进了内殿中。

皇后、朱妃等嫔妃,就在床边坐着。向皇后抱着年仅五岁的赵佣,早就是哭得满面泪痕。稍远一点是太后,看起来热爱。而三名翰林医官也在内殿中,各自脸色都不太好。

经过施针和灌药之后,御医们已经把他们能做的都做了,接下来只能等赵顼自己醒来。如果醒不过来的话,那么就会在这几天了。

韩冈纵然对疾病的了解远不如他手下的御医们,但中风还是有所了解,如果不能在短时间内醒过来的话,那就没有希望了。

在唯一显得格格不入的便是站在太后身边的雍王赵颢。

说句实在话,韩冈和赵颢两人两人虽然有旧怨,但打过照面的次数屈指可数,而且大多数还是隔得很远的认个脸而已。眼下同在一殿,相距不过数尺,却是极难的的经历。

赵顼的另一个弟弟则不在这里。韩冈方才是亲眼看见嘉王殿下从宫中离开,以赵頵谨小慎微的心性,多半会就此杜门不出,直到皇宫这边有个结果。

拥有自知之明的人的确不讨人嫌,保持这样的作派,最后不论是维持现状还是换人台,赵頵都会为今天的行动受到奖赏。当然,如果赵頵不是排行第三,而是跟赵颢交换,排在第二,想来就会是另外一番表现了。

大概就会是赵颢现在的反应,暗藏着窃喜和期待,在兄长的病床前表现出自己的伤感和关切,然后安慰着似乎并不需要安慰的太后。

亲生儿子出了事,坐在一旁的高太后不是没有伤心的神色,但她的神情更接近于太后这个身份,而不是一位母亲。

好,这可以算是偏见。韩冈一直不是很待见,确切点说是敌视赵颢,以至于这个看法甚至牵连到高太后身——尽管没有表现出来。从有色眼镜中看到的人和事或许并不是事实,不过韩冈并不觉得需要更正自己的看法。从很早以前,在韩冈得知高太后硬是将两个成年的儿子留在宫中的时候,韩冈就已经抱着这样的‘偏见’了。

“韩学士。”皇后向氏这时候擦了擦眼泪,“朝臣中以你最擅医术,你来看看官家的情况到底该怎么样治?”

韩冈依言走过去,躺在床的赵顼盖着明黄色的缎子被褥,只有脸露在外面。紧闭的双目,呼吸也是极细极弱,原本苍白的脸现在更加苍白。从外相看,大宋的这位皇帝情况并不好,但病情似乎是稳定下来了。

十几道期盼的眼神望着韩冈,但韩冈只能给他们一个虚无缥缈的回答:“陛下奉天承运,必不致有大碍。”

韩冈话音刚落,满是惊喜的声音便在床边响起,“官家醒了!官家醒了!”
