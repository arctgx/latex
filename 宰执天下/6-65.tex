\section{第八章 朔吹号寒欲争锋(五)}

“西域都护府?”

韩冈也不看名单了,抬头问着。

“叫安西都护府也行。”韩绛道,“方才也说了,玉昆你对西域最是熟悉,王舜臣就更不用说了。”

“的确需要设立一个统掌西域的衙门。不过都护府也好,安抚使司也行。”

张璪摇头:“安抚使位太高,王舜臣年资不够。”

韩绛也道:“他要做,至少得再过十年。”

文臣好说,武将想要做到一路安抚使,王舜臣这个年纪实在是太年轻了。不过以他的功劳,十年之后,说不定还有可能进入枢密院,步郭逵后尘。

毕竟王舜臣是独立领军,打下了西域。三军易得,一将难求,而帅臣更为难得。

张璪问韩冈:“玉昆,王舜臣的表字是景圣吧?”

“没错,不知邃明兄何时听说?”

王舜臣就算可归入当世名将的行列,以张璪的身份也不会刻意去打听他的表字,而且王舜臣的表字流传的并不广。

“京中民间都在传说有个能定西域的王景圣。一说汉唐武功,就是定西域,如今国朝在民间能与汉唐相提并论,得王景圣之力甚多。”

“民间传言多有夸大,多是说书人的功劳。”韩冈沉默了一下,又感慨着,“他的表字还是王襄敏当年所赠。”

韩冈的话,让韩绛和张璪都沉默了,片刻后,韩绛叹道:“……王子纯是可惜了。”

韩冈沉沉的点了点头。

这不是他第一次听人这么说,韩冈这些年来已经听人说了同样的话很多次了。

因为王韶的早亡的确让人十分痛惜。

能领军而胜的主帅,不是那么好找的。尤其是王韶,是在内外交困的情况下率领一支偏师杀了出来。王韶若不是早早的病死,现如今至少能做到枢密使了,说不定,宰相都有得做。

“幸好王子纯还留下一个王厚。这一回平乱,他的功劳不小。”韩绛说道。

韩冈笑道:“功劳的确不小,但朝廷也没有亏待他。”

西上阁门使,兰州观察使,带御器械,提举皇城司,这是王厚现在官职。之前西上阁门使只是本官的官阶,现在又成了差遣。也就是说他即管着皇城司,同时也管着横行五司之一的西上阁门司。禁中兵权的三分之一在他手中,又管理着一系列有关礼仪性质的事务,实权在握,正炙手可热。日后不是三衙,便是西府。

只不过,韩绛和张璪都明白,韩冈说朝廷没亏待王厚,可不是为了说朝廷对王厚够意思。

“王舜臣现如今是东染院使、甘州团练使。”张璪表现得对王舜臣十分熟悉,“他现在能做安西都护?”

安西都护,不论是在汉在唐,地位都很高。唐时到了安史之乱前,安西大都护更是要么皇子挂名,要么就是权臣。

而且都护一职兼管文武,这个位置不能等同安抚使,更接近于宣抚使。以西域的幅员之广,同时道路因天山而南北相隔,在地理上也不方便安排为一路之地,至少的分成两路。

王舜臣又是武将,而不是文臣。

从人、从地、从官,从各个角度来看,王舜臣成为安西都护都不适合。

但从功劳和威望来看,却没有人比他更适合。

而且若是派个文官过去,到时文武相争,又该支持谁?

韩冈看着韩绛。既然是韩绛先提起王舜臣的事,想必他已经有了些想法。

“王舜臣若能安心在西域十年,可任其权发遣安西都护。”韩绛说道。

西域对朝廷来说是鸡肋。

看着地域之广,堪比中原,实际上是个窟窿,每年都需要朝廷大笔的投入。这是攻下了河西之地后,顺势而为的结果。不过能拒敌于国门之外,让中原之地在外围再多上一重防御,终归是一桩好事。

对外扩张,总比在家门口抵御敌人来得安心。之前多少儒臣说虚外守中,有很多都是因为开国以来,大宋始终居于弱势,不过是说葡萄酸的狐狸。现在国势复振,对外喊打喊杀的就越来越多。连寻常士人唱和的诗文中,提到卫霍或是班定远的次数都多了许多。

“但王舜臣的年资还是太浅。”

韩冈盯着韩绛。他不介意代王舜臣答应下来,三五年之内,西域方向,王舜臣走不开。而三五年后,韩绛和张璪都还不一定能留在现在的位置上。

但韩绛方才可是说了,安抚使可都要十年。

“将王舜臣调回来的,军心士气可能维持?”韩绛反问韩冈。

当然不行。

韩冈摇头道:“西域人心尚未归附,王舜臣最好还是得先在那里留上一阵。黑汗国惨败之后,必然心有不甘。等西域开春雪化之后,定然会出兵东来收复失地。”

“既然势必留任,又何必说年资?”韩绛道:“皇宋开国以来,安西都护从未授人。”

因为晚唐之后西域就丢了,大宋开国后,也没能重新夺回西域。

但韩绛说得韩冈也明白。

既然之前从来没有设立过,安西都护府是高是低,全凭政事堂来定。而安抚使的位置高下,早就确定了。

王舜臣十年后才能做一路安抚使,但现在就可以做都护——尽管是权发遣。

设立西域都护府,王舜臣任权发遣都护。

这看来是韩绛、张璪预付的账单。

不过还不够。

“记得朝廷之前封了伊州观察,西州观察。”韩冈说道。

那些投效大宋的土官,朝廷一向舍得赠与官职。光是一个西州回鹘,便是观察使、防御使封出去一堆。而王舜臣现在一个遥郡团练使,在职位上,就有着极大的差距。

“武臣外任的遥郡之封,也只能到观察使为止。”张璪说道。

节度使、节度留后不要指望。

他看着韩冈,笑得意味深长:“而且这件事,得跟章子厚商量。”

武官的差遣任命,归属审官西院和三班院,在政事堂管辖之下。但武将的本官晋升,则是在枢密院的权限下。

朝廷设立安西都护府,都护地位就算能与安抚使相当,超过了政事堂的人事权限范围,可举荐之权还是在宰相和参知政事的手中。不过想给王舜臣加一个观察使的遥郡,先得问章敦、苏颂同不同意。

韩冈点点头:“韩冈明白。”

这好说。

章敦就算要与自己对立,也不会将事情放在王舜臣的头上。

西域离京师实在太远了,也完全没必要。弄坏了在武将中的声誉,对章敦自己都不是好事。

见韩冈胸有成竹,韩绛和张璪都没有其他话要说了。他们所要做的,只是设立安西都护府和让王舜臣就任权发遣都护两件事。至于王舜臣的遥郡,那就由韩冈与章敦打交道去。

西域事毕,韩绛端着温热的茶水喝了两口。突然就叹了起来,“若不是种五重病,这一回就让他去做大都护了。想必王舜臣也不敢有二话。种五也该会乐意。”

肯定会乐意的。

朝廷在近年内不可能北上攻辽,对于此事,高阶将领没有不清楚的。

种谔那种没有战争就会浑身不自在的人,怎么可能不愿意抽空去西域跟黑汗人打?

只可惜种谔生了重病,种建中写了信来,向韩冈求医问药,韩冈之前已经向太后申请过了,从太医局中选了一名翰林医官去为种谔诊治,还特地赐了许多珍贵的药材。

“希望种五能吉人天相。西军有三种之说,可只凭种诂、种谊,撑不起种家。”

十年前韩绛曾为陕西宣抚使,统领大军往攻横山。虽说是惨败而归,但他对西军的了解,也不是寻常辅臣可比。

种诂曾经上表投诉庞籍贪功,如此胆大妄为,触动了每一名文官的神经,所以他这一辈子都没有被重用,一直都在边郡。种五能晋身三衙管军,但种诂永远不可能。而种谊,不论名气和功绩都与他的兄长差了太远。

“种子正的子侄中,也有几位将种。种朴、种建中、种师中这三人。”韩冈说道,“放在西军的年轻将领中,也算得上是出色。”

“种建中?”张璪像是想起了什么,“好像曾经看见过这个姓名。”

“种建中是明法科出身,之前随军在种子正身边参赞军事。现在又转回文资,灵州的灵武知县。”

边州的知县不值钱,种建中从武资转回文资,以从八品的京官大理寺丞担任灵武知县,根本就没人跟他争这个最近从西夏旧址上设立的新县。

“这就对了!那一次看得正是宁夏路。”

“这三人可用于北事?”

“种朴、种师中也在宁夏路?现在朝廷不正是在用他们镇守北疆?”

辽国迟早要解决,但绝不是现在,这是朝中所有宰辅的共识。

至少要等到耶律乙辛谋朝篡位,辽国国中人心不稳的时候。那时候打起拨乱反正的旗号。

两家皇室可是有亲戚关系,在澶渊之盟中,宋真宗、辽圣宗便约为兄弟。刚刚‘病死’辽章宗耶律延禧,与赵煦正好就是兄弟关系。

如果耶律乙辛篡位,向太后有充分的理由为自己的侄儿斥责这名叛臣。

但在这之前,守住刚刚扩张过的北疆,才是北方边军最重要的一件事。
