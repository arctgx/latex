\section{第13章 赳赳铁骑寒贼胆(上)}

“从秦州往甘谷城的路可不好走。”普修寺的厢房中,韩千六在灯下摇头叹气,“黄大瘤死了,李癞子服软,本以为再没事了,怎么还被摊到这桩差事。唉……”

“谁让孩儿得罪了县尹。”韩冈也是苦笑,“自来做官都是瞒上不瞒下,都生怕事情捅到上面,妨了自家升官发财的路。但军器库一案被州里截了去,死的、办的都是成纪县中的人。县尹因此吃了不少排头,少不得一个失察之罪,当然看孩儿不顺眼。”

“这……这……”韩千六给惊到了,已是初冬的天气,头脸上却腾地冒出豆大的冷汗直往下流。黄德用区区一个班头就害得韩家差点翻不了身。现在黄德用死了,但陈举还在,却又得罪了知县,他舌头吓得直打结:“这……这可怎生是好?!”

“爹爹不用担心。”韩冈安慰着,“孩儿现今与吴节判交好,若有什么事情,他总会帮忙担待着。县尹如今也不过是出口闲气,不会做得太过。左右就是一趟押运,避是避不过的,先走着看罢。”

韩冈这话是说给韩千六听的,实际上他面临的情况要危险得多。成纪知县不会要他的性命,但陈举可是要的。他在公堂上没能如愿,后续手段当是一招招的接着杀过来。而从这几天来跟吴衍的接触来看,韩冈知道,雄武军节度判官绝不会正面与陈举过不去的。

做官的都是怕麻烦,能少一件事就是少一件事。他能为韩冈移文成纪县,是他看着韩冈顺眼,能帮就顺便帮一手,但如果帮不了,那也就摊摊手,连句抱歉都不用说的。

不过韩冈本来就不是把希望寄托给别人的性子。他对吴衍的要求也不多,请他随便找个理由,遣几个可信之人假借去甘谷城送信的名义与韩冈他同行,算是随行护卫,应该不成问题。再多的,韩冈自信光凭自己就能解决。

陈举的势力在内而不在外,秦州城中他根深蒂固,可出了州城,陈举能动用的手段就只剩下几个选择,要防备起来也容易了许多,就是怕陈举害他不成,转去找父母和小丫头出气。

“别说这个了。”韩冈不想再在知县和陈举的话题上说太多,省得他走后父母和小丫头担心,他问韩千六道:“去年杨太尉修甘谷城。爹爹你也是应役的,从秦州到甘谷,哪段路平,哪段路险,应该有个数罢?”

韩冈嘴里的杨太尉,大名唤作杨文广,是当年威震云中的杨业杨无敌的亲孙,力克契丹的杨延昭杨六郎的儿子。韩冈不论前生今世,都是对这几个名字耳熟能详。

杨文广为将有勇有谋,不输父祖之风。如今已年近六旬,仍拼杀在对抗西夏的第一线上。他曾参加过平定侬智高的战役,当主帅狄青北返后,以邕州知州的身份镇守广西边境。在现如今的大宋诸多武臣中,杨文广算是硕果仅存的名将。

去年修筑甘谷城的时候,杨文广是秦凤路兵马副总管——总管则惯例是由身为文臣的秦州知州、秦凤路经略安抚使兼任——现在他正担任泾州知州,抵抗着西夏人的进攻。

当时为了能在西夏人反应过来之前,将处在战略要地的甘谷城——当时还叫做筚篥城——筑好,秦州的六个县几乎是全民动员。秦凤经略司一口气从秦州调集了七八万民伕参加,韩冈的大哥去了甘谷城工地夯土,而韩千六也被紧急征召起来运送粮草。

“去年为了给甘谷城运粮,你爹俺从秦州到甘谷,再从甘谷到秦州,来回跑了整六趟。说起来,那条路真是再熟也不过了。”韩千六叹了口气,感慨万千,“那条路啊,可不好走!”

韩冈点了点头,虽然甘谷城就在秦州州城的西北面,直线距离只有五六十里,但由于两城之间隔了一重高耸分水岭,一个在藉水河谷,一个在渭水河谷。这个时代可没有什么隧道或是穿山公路。想从秦州城运辎重去甘谷,必须先向东,沿着藉水走到陇城县【今天水市麦积区】,那里是藉水与渭水的合流处。

藉水与渭水虽然都是东西向,不过北面的渭水更近于西北——东南走向,与由正西向正东流淌的藉水有个不大的夹角。韩冈押运的这批军资便是要在陇城县由藉水河谷拐个大弯,转到渭水河谷,再从渭水上溯,改往西北方向去。一路要经过三阳寨、夕阳镇、伏羌城、安远寨,最后才能抵达目的地甘谷城。

“根本就是要绕个大圈子,多走上百十里地。”韩冈对秦州到甘谷的这条路,了解得就这么多,“而且渭水和藉水都不是一条直线,河道在山间曲折多变,看起来近,走起来却远得很。”

“所以说不好走啊!山路又长又窄,又是弯弯绕绕,不过隔着一重山,竟是要走上四程路。”韩千六用手指在茶盏中占了点水,直接在桌面上画起路线图来,“从州城到陇城,这是第一程……”

一程就是一天行程,韩冈打断韩千六的话,问道:“不过才三十里地,秦州到陇城的官道修得又好,怎地这就算是一程了?”

韩千六笑道:“三哥儿你不知道,从陇城往三阳寨【今天水渭南镇】的第二程这小六十里地太难走了,都是在山夹缝里,没得地歇脚。所以到陇城后须先歇上一夜,第二天四更天不到就得上路,一鼓作气到临夜时才能赶到三阳寨。”

韩冈点头受教,心知这一路陈举若有什么安排,应该先出现在第二天,如果第二天没有出现,那便会出现在第三天。“那第三程就是从三阳寨到夕阳镇【今天水新阳乡】喽?”

“哪得那么好事?!才二十里地出头怎么歇?还是四更天上路,巳时前能在夕阳上镇歇个半刻,再急脚赶过裴峡去,大约酉时能入伏羌城【今天水甘谷县城】歇息。”

韩冈再点头,又把裴峡两个字记在了心底。

韩千六看着韩冈老实听教,兴致一下变得极高,更是说得口沫横飞:“伏羌城那是甘谷水【今散渡河】汇入渭水的地方,这第四程便是沿着甘谷水向北去,三十里到安远寨【今安远乡】,再三十里方才到甘谷城。杨太尉在大甘谷口修得这座城,把整个甘谷都括了进来,少说也有数千顷的上等良田。

甘谷本是筚篥族世代所居,甘谷城刚修的时候也还叫筚篥城。不过十几年前他们给党项人逼走了,换了心波三族来占着。现在甘谷有一半的地是他们的,还有一半他们也想贪掉。听说如今正闹着呢,三哥儿你通过甘谷的时候,说不定还会碰到些麻烦。”

对于北上甘谷的路线,韩冈大体上已经了解了差不多,现在又从有过亲身经历的韩千六印证了一番,几个可能有危险的地方他都会做好防备,如果吴衍派来的人得力,保着自己安全抵达甘谷不成问题,即便不得力,他当日就在军器库找到了一些有用的东西,足以应对一些危急状况。等到安然抵达甘谷城,他有的是办法出头。

对于情报的搜集,韩冈也许还不如秦州城中惯谈着家长里短的妇人,但对相关情报的整理、分析、推断,这些在后世就算在商业活动上也是必不可少的手段,在此时的情报活动中,依然是块因少有人涉猎而缺乏系统的空白。

这些天来,韩冈对有关陈举的情报着力打探了不少,排除掉了一些明显夸张扭曲的信息,陈举所拥有的明面上的实力,韩冈大体上都已经有所了解。而既然看到了冰山露出海面的部分,那隐藏在水下的阴影也逃不过明眼人的追根究底。

陈家的田产遍布秦凤路的五州一军,其能动用的人力,至少在秦凤是个惊人的数字。而秦州城中的几家市口优良的出售吐蕃特产的商铺,以及面向蕃部的大型商号,证明陈举必要时还能动用蕃人的力量。与京中的联系,在各处城寨中的人脉,通过对陈举摆在明处的实力的解析,他所能动用的手段韩冈可以做到心中有数,现在他唯一担心的,就是父母和韩云娘的安危。

“爹爹!”灯火在韩冈脸上投下的阴影中满载着忧心,连一贯锐利的双眉也变得纠结起来,“孩儿这一去,陈举必然有花招要使。孩儿倒不惧他的龌龊手段,就是担心你和娘会有什么不测。舅舅如今在凤翔军中,陈举手再长也伸不到那里,不如你和娘带着云娘去投舅舅一阵子,等孩儿把这里的事处理好,你们再回来。”

“三哥儿你孤身一人对付陈举,可有多少把握?”

韩冈展颜笑道:“爹,你也看到黄大瘤的下场了。陈举势力虽大,在孩儿眼里也并非无懈可击。只要没有后顾之忧,孩儿有的是手段应对。”

“好!”韩千六没多考虑就点头答应了下来,李癞子和黄大瘤的结局,给了他很大的信心,也知道自己留在秦州只会给儿子添乱,“俺回去跟你娘说一声,去你舅舅那里避一避。”

ps:韩冈的后顾之忧解决了,各位兄弟也别让俺有后顾之忧。请多投一些红票,让宰执天下在新书榜上站得更高。

