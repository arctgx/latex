\section{第六章 千军齐发如奔洪(上)}

王舜臣还是第一次在长安以西立于黄河之滨。

没有高耸的堤坝,只有宽阔的河床,浑黄色的河水就眼前汹涌奔流,带来隆隆涛声。

眼前的滔滔大河,不是王舜臣过去入京时,在路上看到过的泥浆洪流。尽管依然浑浊,但一眼就能看得出与那一碗水半碗沙的泥浆水,到底有多大的差别。

“黄河之水天上来,奔流到海不复回。若是能够分身,真想再往上游去看看,看看黄河之源是从何而来。”

王舜臣循声回头,王厚不知何时已经走了上来。

王厚在王舜臣身边立定,一同眺望着黄河。他三十岁便担任了权洮州知州,兼熙河路钤辖,甚至之前还早早的转了文资,正八品的太子中允。在审官东院中,就是拥有一个进士头衔,一般也不可能在这个年纪便执掌一州军政。能做到这一步,也只有依靠军功。

在西北边陲历练了十年,留着两撇短须的王厚皮肤黝黑,但看着依然年轻。气质是沉凝浑厚,一双眸子既不锋芒毕露,也不是圆滑内敛,而是坚定如石。王舜臣看着他,就仿佛当年初见王韶时的感觉。

“记得当年玉昆曾经说过,黄河水中泥沙来自于陇西陕西的黄土高坡之上,雨水一过,便是泥沙俱下。到了下游之后,水流变缓,泥沙逐渐沉积,河床一日高过一日,水患由此而来。黄河之患,在沙不在水。要想从根本上治理好黄河泥沙,就得利用草木保持水土。”王厚笑了一下,“可惜做不到。也就自兰州往上游去,那里的草木几百年未有砍伐,情况要好一点。”

王舜臣当然也还记得韩冈当年所说的话。

这么多年过去,当年在军营的小厅内饮酒达旦的四人,各自都已经站在了他人几十年都难以企及的高度。这是当年想都没敢想过的。

他回望着河上,一声声号角开始为涛声伴奏:“赵大要过河了。”

王厚随即也望了过去,在两人立足的下游不远处,一条长链般的浮桥横在河上,被湍急汹涌的河水向下冲出了一个半圆的弧度。桥面在河上起伏,走在上面的车马看着就像是在挪动。

十九条大小渡船,加上一干羊皮筏子,这是兰州过去用来渡河的工具。不过在官军抵达兰州后,用了四天的时间,以渡船和羊皮筏子搭起了一条浮桥。

而在这之前,禹臧花麻就已经殷勤的帮官军将对岸的西贼一扫而空,让官军可以毫无阻碍的搭桥渡河。

昨日中午,浮桥刚刚搭建完成。可到了今天早上,半日加上一夜,官军就已经有一万多人马过了黄河。

赵隆作为王中正手下第一号得用的亲信大将,他的出动,代表着中军也终于开始渡河。

“等赵大领着熙河第一第二两将的八千人马过河,就该轮到蕃军了。希望他们别在桥上乱起来。”

对于这一次的战争,熙河路的蕃军都是不情不愿,他们种田养马就能赚大钱,闲暇时踢球看球赌球,有必要去卖命?可惜有朝廷的严令,从董毡以下,都不敢不从。上百个部落拼凑起来的一支军队,交由董毡的便宜儿子阿里骨统领。

想起那一支拼凑起来的蕃军,王厚也忍不住摇头苦笑。

阿里骨本人是个拖油瓶,没有吐蕃赞普家的血统,在河湟的吐蕃部族中没有多少威信,要不是他常年在巩州的蕃学混了个脸熟,根本轮不到他领军。

“不指望他们能上阵,能吓唬人就可以了。”王厚叹道。

“也不知禹臧花麻会不会派人一同出兵?”

王舜臣问着,两人都回头望了一眼在王中正身边露出谦卑微笑的禹臧花麻。

“多半会吧。”王厚看禹臧花麻的殷勤样,当不会漏下这个卖好的机会,“兰州拿下来,再随着官军打到灵州,稳当当的一个观察使到手。”

“这一次要不是仗着官军的威势,禹臧花麻怎么可能能这么容易就将兰州掌握住?”

“也是兰州城中党项人兵力减少的缘故。最多的时候,也就是熙宁九年、十年,兰州城中的铁鹞子有一万两千骑,粮草几乎都要他供给。那两年禹臧花麻一个月一封信求着经略司早点发兵打兰州,他肯定双手献上城门。也就是到了去年,才减少到八千。这个数目一直保持到战前,就在一个月前,才突然将其中大半调去北方,只留下了三千兵马。”王厚顿了一下,补充道:“而且几乎都是由小部族的成员组成。”

王厚在笑着,勾勒在嘴角的纹路中尽是讽刺,西夏高层这么做的用意再明显不过,绝不会硬顶着刚刚出兵后锋锐正盛的官军,而是打算利用艰难险阻的道路,逐渐消耗官军的锐气,拉长补给线,遣军截粮道,不断削弱官军的实力,最后才会决战。

“诱敌深入?”王舜臣冷笑。

“自然不会有其他招数。”王厚指着黄河,“不过这一战的关键之处,就在下游八百里外的灵州。兴灵本为一体,放弃了灵州,兴庆府不保。一旦官军攻下灵州,西夏就亡了,不论什么计策都没用……”他的声音忽而又低沉起来,“可如果官军攻不下灵州,那么西贼逆转的机会就到了。”

“三哥也是这么说。”

聚集到灵州城下的兵力越多,后勤上的压力就越大。一旦三十余万大军齐集灵州城,究竟会出现什么样的情况,王舜臣想都不敢想。

幸好自己是往西去,六千人马只要能翻越洪池岭,粮食要多少就有多少,不用跟自家人抢。

王厚喟然长叹:“明明能一点风险都不冒的将对手的家当一步步赚到手,偏偏还要赌上一把。输了就要倾家荡产,赢了也不过是能早两年得到对面的赌注,没见过这么做买卖的。”

王舜臣转头凝视着王厚,疑惑道:“不像是钤辖你说的话,怎么全是商人口吻?”

王厚随之一笑:“这是冯四说的。只是个商人而已,见识却比朝堂上的那些宰辅都要强得多。”

“毕竟是三哥的嫡亲表弟。怎么也不会差的。要不然,才这几年的时间,赚的钱都能赶得上一路财税了。”

王厚摇摇头,“不说这些了,等过了河就要上阵了,可不要犯了迷糊。”

“迷糊可不会再犯,能攻城掠地,就不在乎那几个首级了。”

在兰州渡过了黄河后,秦凤、熙河两路联军就要赶往东北的灵州,而王舜臣则是要率领偏师去攻打西北的凉州。

不过在这之前,必须要先攻下兰州北岸西侧五十里的卓啰城,那是西夏卓啰和南军司的核心,也是西夏用以控制黄河北岸的重要据点。不论向东还是向西,不将卓啰城控制住,不将卓啰和南军司铲平,粮道和后路随时会可能被截断。

王舜臣还记得前些日子被派到他麾下的一名充当向导的僧人是怎么跟他说的。

“打下了卓啰城之后,就溯喀罗川【今庄浪河】北上洪池岭。西贼在洪池岭【乌鞘岭】下设有一寨,名为济桑。这个寨子,是专门用来堵截想绕道兰州来的回鹘商人。当年西贼攻下甘州、肃州和凉州之后,为了收取过税,就强迫走河西道的回鹘商人过了凉州后继续向正东走,经由丝绸之路的北线抵达兴庆府,然后再往中原去,严禁他们走水源充沛的中线。”

“党项人的贪婪在回鹘商人中是鼎鼎有名的,收了税不说,有时候,有些部族还会假扮盗匪在半路抢劫,人和货都保不住。要不是为了躲避这些强盗,怎么会有从青海【青海湖】走的那条新路出来……那可是最难走的一条路,但吐蕃可比党项人讲信义多了。”

“夏天的洪池岭深寒如冬,六月都能下雪。想要翻山过去,要先将皮裘和御寒之物准备好,否则一不小心就会冻伤。”

不仅仅是从来往黄河两岸的僧侣那里得到河西的风土人情,顺丰行中也有许多有关各地地理的资料,冯从义早早的就遣了一名向导带着资料到王舜臣麾下报到。

为了翻越夏日一如寒冬的洪池岭,分配给王舜臣的六千人马,冬衣都是随身带在身边。

洪池岭的存在,阻断了兰州和凉州之间的粮道,王舜臣轻兵突进其实很危险。不过凉州是大城,周围的田地也不少,过去的时候正好到了六月,要就地征粮不算难。党项人的坚壁清野,不至于将凉州也归纳进来。

远眺赵隆骑着马从浮桥上疾行而过,身穿金甲的王中正也到了桥头前,王舜臣回身向王厚拱手一揖:“钤辖,舜臣要告辞了。”

王厚回了一礼:“王厚就在兰州静候佳音。”

王厚的送行到此为止,他不会随军继续北上,而是要负责粮秣调集。

地理位置至关重要的兰州接下来将会成为一个兵站,为前军输送粮草。在朝廷还没有正式设立兰州这个州级编制的时候,紧邻兰州的洮州就会将此城暂时归入管辖范围。

而洮州知州王厚,就是负责粮秣运输的主官,王中正到底能打到哪一步,有一半得看王厚的本事了。

