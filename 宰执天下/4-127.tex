\section{第20章 冥冥鬼神有也无(三)}

【有事耽搁了,下一章改在十二点发。之后应有的两章照常。】

秋色降临在北方的大地上。

一座由帐篷组建的城市,也依照惯例出现在辽国庆州西北伏虎林边。

人马车辆数以万计,大大小小的帐篷如同夏日雨后的蘑菇,一眼都望不尽。在最中心的位置上,甚至有了由高阔都有一二十步的大帐和一条条竹木组成的廊道缀连起来的宫殿——天子理政的省方殿,寝居的寿宁殿,接见部族藩属的八方公用殿等等。

这些以宫殿为名的帐篷皆以木竹为柱,以毡为盖,柱上施以彩绘,锦缎挂于四壁,绣龙黄布铺设在地面,窗、帘皆以毡为之,外面罩着黄油绢。

这就是辽国皇帝的捺钵。

大辽皇帝四时巡守国中,镇服四夷。所居住的行在,便称为捺钵。冬捺钵设在广平淀,春捺钵设在鸭子河,夏天在吐儿山,而秋天就是在伏虎林。狩猎、放牧、理政、接受领下部族的觐见,辽国的军政大事都是在捺钵中完成,这是沿袭了两百年的传统。不过换做了如今的大辽天子,就是在狩猎之事上用心多了一点。

一声声模仿着鹿鸣的号角,宣告着今秋的第一次狩猎即将开始。

辽国的权臣,如今的北院枢密使、被封为魏王、太师,赐姓耶律的耶律乙辛,也在自己的帐幕中换好了猎装,从帐中走了出来。耶律乙辛少年时以相貌出众而被兴宗皇帝和皇后看重并提拔,如今年岁虽长,但掌控朝政日久,使得他浑身上下的气度也更加不凡。

“太师!”耶律乙辛的亲信,北面林牙萧得里特正好来到帐门外,神色间有些惊慌,凑近了低声对耶律乙辛道:“太子那边似有异动。”

“不用慌,耶鲁斡翻不了身。都安排好了,这两天他近不了天子身边。他亲娘都因通奸之罪被赐死,他这个太子还能在位子上多久?”

耶律乙辛毫不在意的叫着太子的小名,拿着条鲜肉逗着站在左臂上一只彪悍骏捷的海东青。这只得自东海女真,又是由他自己亲自训练出来的猎鹰,是今秋狩猎的关键。要把想皇帝服侍好,稳固自己的地位,就需要随时服侍在身边,不能让那一位坏了兴致。

萧得里特脸上显着急色:“不仅仅是太子,还有宫卫那边……”

出于贵戚为侍卫,著帐为近侍,北南部族为护卫,武臣为宿卫,亲军为禁卫,百官番宿为宿直。侍卫、近侍、护卫、宿卫、禁卫、宿直,都是护卫天子帐幕安全的职位。不过真正做事的,主要还是护卫和禁卫,其他都是名义上的差事。

最近北护卫司就有些不稳,私下里隐隐的就有传言说,有人在中间挑头要刺杀耶律乙辛这位权臣。萧得里特也是听到风声就赶过来了。太子加上天子身边的护卫,耶律乙辛一个疏忽,就能送了性命。

“护卫那边有查剌在盯着,谁有心作乱,我也心里有数。”耶律乙辛冷笑着,护卫太保耶律查剌是他的人,根本就不用担心。若不能在天子身边安插上自己的亲信耳目,他枉为权臣了。皱眉想了一想,“好象是叫萧忽古。现在不需要动他,留着他日后有用。”

究竟有什么用,萧得里特不用想就知道,犹犹豫豫的开口:“可陛下就这么一个儿子。”

“不还有皇孙嘛……”耶律乙辛笑容中透着凛冽的杀意。

萧得里特悚然而惊,不敢直视耶律乙辛如同冰刀一般的笑意,低下了头去。只是他下移的视线,却发现耶律乙辛拿着鲜肉条、逗得猎鹰一对眼睛跟着直转的手,在不由自主的颤抖着,可见手的主人绝不似外表看起来的这般平静。

干咽了口唾沫,萧得里特也暗自发恨。要不是太子前岁预朝政后,事事针对耶律乙辛,始终敌视他们依附魏王的这一群人,魏王又何须下这等狠手。

现如今皇后已经被赐死,杀母之仇怎么都不可能化解得了。不除太子,死的就是耶律乙辛和他们这些人,已经是箭在弦上不得不发,再也退不得半分。他再一次凑近上前,就在耶律乙辛的耳边狠狠的说着,“太师,小心夜长梦多。”

耶律乙辛点了点头,这个道理的他当然明白,笑道:“你比张孝杰敢说。”

“那是因为小人对太师一片忠心。”萧得里特连忙拜倒,心中惶惶不安,不知道是不是说错话了。

张孝杰是北府宰相,汉人出身,不过最近被赐了国姓,改名耶律孝杰。不过张孝杰一向依附耶律乙辛,所以耶律乙辛还是该怎么称呼就怎么称呼——辽国与宋国不同,以文治国的宋国,宰相压在枢密使之上。而以武治国的辽国,宰相得站在枢密使的下首——但萧得里特的北面林牙还是远比不上宰相的权位,生怕耶律乙辛在说反话。

耶律乙辛伸手搀起萧得里特:“你的忠心我知道,所以事情都不瞒你。”

只是两句话,萧得里特便是感激涕零,眼圈都红了,看着就要哭出来,“太师青眼,小人必粉身碎骨竭力以报。”

耶律乙辛招了招手,唤来不敢听到不该听的话、一起避得远远的侍卫,吩咐着:“去找萧十三来。”

回过头,耶律乙辛对萧得里特笑道:“不知萧十三这位殿前都点检兼同知枢密院事,安排的事情办得如何了?若是部族军,死了万儿八千都没什么。但皮室军这边,一队人马都丢不得。”

“药师奴为人聪慧,当知进退。”萧得里特恭声说道。

耶律乙辛点点头,否则他就不会让此人去西京道办事了。

逗了半天,耶律乙辛终于将鲜肉拿着凑到了海东青的嘴边。羽毛蓬起的脖子一伸,如钩一般的鹰喙一口就将鲜肉吞了下去。

随即一对灼灼闪着寒光的鹰眼又盯起耶律乙辛的手指。耶律乙辛将手指在护臂的皮套上擦了一下,没有再多了。饿了两天的海东青,只是尾指大小的一小条鲜肉是远远不够满足空空如也的胃口,被引逗起来的饥饿反而会更进一步逼着猎鹰去参与到捕猎之中。

摸着价值连城的海东青,耶律乙辛道:“能不能帮党项保住丰州并不重要,战事不利先退了再说,只要能探出南人禁军的虚实就够了。”

萧得里特不屑的说着,“南人也就仗着兵利甲坚而已,哪能与我契丹铁骑相比。”

“南人还是有些能耐的。”耶律乙辛抬头望向寿宁殿的方向,巨大的帐篷上空,一个指尖大的黑点正悬浮在几十丈的高空中,“南人的飞船对我契丹铁骑来说,用处着实不大,不过用来寻找猎物还是很有些用,天子也是喜欢。”

萧得里特也跟着望了过去。他们所侍奉的皇帝,正乘着飞船在天上巡游。那不是从南朝买来的飞船,而是由大辽本国工匠制造。大辽幅员万里,丁口千万,有了模子和图样,要找出几个能制造飞船的工匠一点不都难。

“这玩意儿,只要海东青啄上一下这飞船就完了。”第一次看到飞船的时候,萧得里特甚至是惊骇得说不出话来,靠着飞船人竟然能飞上天!但等到他看得眼熟,也就觉得平常了,“冲过去砍了下面的绳索,风一起就不知道会被吹到哪里去。”

耶律乙辛将猎鹰移到鹰架上,眼神变得深沉起来:“飞船也好、板甲也好,加上神臂弓、斩马刀,这些军器都是配合南朝军队来使用的,与我大辽用处都不大。”

飞船不必说了,跟不上骑兵的行程,守城、攻城上才好用。板甲虽然是好,但大辽哪里来的那么多钢铁?人工倒是好说了。只有不缺铁,只缺人工的南朝,才有迫切的需要打造板甲。神臂弓、斩马刀更都是步卒使用,以骑兵立国的大辽学着打造,难道要装备给汉军和渤海军?

萧得里特的声音也低沉了下去:“南人软弱,本算不了什么。可配上板甲、飞船、神臂弓、斩马刀,的确是越来越强了。听说年初的时候,在南方还打出了一个千五破十万的大捷来。”

“那个倒没什么了不起。”耶律乙辛摇头冷笑着,“什么交趾国,什么千五破十万,那都是笑话。要定个高下,须得跟我契丹铁骑较量一番再说。”

“南朝的皇帝一直以来好像都是想着要夺回燕云,说不定什么时候就发了疯,当真起兵攻过来。”

耶律乙辛嘿嘿笑了两声,“只要还有党项人在,南人就不可能先行攻我。”

耶律乙辛无意主动对南朝挑起战争。他已经是一人之下万人之上,赢了他登不上帝位,输了他现在的位置不保,与宋人开战对他又有什么好处?又是何苦来由。

但南朝自新帝登基的这十年来,一直都是咄咄逼人,不断增强武备。若是对此置而不论,一直被他压制着的国中贵戚必然会心怀不满,最后有所密谋。

所以耶律乙辛他必须对南朝保持着强硬的姿态,索要代北地也好,将公主嫁给秉常也好,眼下帮着西夏攻打丰州也好,都是必须要表现出来的态度。身为权臣,对于南朝,他不能有半点软弱。

只是要把握好这个度,不能让局势滑落到两国开战的地步。耶律乙辛手上现在最好的武器就是西夏,如果利用得好,就是一条上好的猎犬,若是其陷入危局之中,就需要帮上一点忙。

“西夏是一条好狗!”耶律乙辛跳上自己的坐骑,不远处,萧十三骑着马过来了,“只要南朝天子还想着夺占兴灵,党项人就得乖乖做大辽的狗!”

