\section{第23章 弭患销祸知何补(九)}

一支浅紫色的玻璃小瓶,肚大口小如同一个含苞待放的荷花花苞,以纯银打造的瓶盖是世间绝无仅有的螺纹口。盖上软木塞子,拧上瓶盖后便滴水不漏。

瓶壁清澈透明,玉润光洁,看不到一个气孔。拿在手中,可以清晰的看到掺了金粉和珍珠粉的香水在瓶中摇晃。仅是外观,就是一件完美的工艺品。

而作为装载这支香水瓶的外盒,同样是一件工艺品。嵌了红纹玛瑙、金翠软玉的彩绘花鸟螺钿漆盒,可以放在任何一家珍宝坊的门面里,也丝毫不显寒酸。漆盒中垫了一层定州的黄绫,在绸缎下,是一方软木,凿了正好能嵌入香水瓶的槽口。将香水瓶放进去后严丝合缝,一点也不会晃动。

就算没有瓶中的香精,仅仅是瓶子和盒子,作价百贯亦不为过。而瓶中的香精,在配上瓶盖和盒子内侧的脂砚斋三个字后,更是价比黄金。

皇后向氏将香水瓶托在掌心,正细细看着,馥郁的桂花甜香自掌中飘散。虽说她贵为皇后,母仪天下,但她的丈夫不喜奢华,与之类似的奢侈器物身边都少见,往往都送到庆寿宫和保慈宫中中,自不用说将日常消耗品做成奢侈品的香精了。

不过女性的天性就算在宫中也是无法抹杀的,如此精致华美的一套香精,让向皇后爱不释手。

“圣人,”一名宫女进来向向皇后禀报,“朱娘子和淑寿公主到了。”

向皇后点了点头,命人传她们进来。香水瓶随手放进盒中,却没有让人收起来。

片刻之后,一名宫装少妇便带着粉雕玉琢的小女孩儿徐步进屋,面向向皇后行了礼。向皇后揽过宫中唯一的公主,搂在怀里,笑着让朱妃落座。

朱贤妃坐下来,顾盼生辉的眸子在阁中一扫,却见坤宁殿的东寝阁中只有皇后和女官,不见其他人。

向皇后看得出她的心思,解释道:“蜀国方才带了她家的益哥入宫来,正与六哥儿一起在后面玩呢。有国婆婆看着,不用担心什么。”

国婆婆是宫中的老宫女,是向皇后身边的亲信,有她照顾,倒也可以放心。

而听到弟弟和表弟都在后面,淑寿公主便不安分的在向皇后怀里扭着身子,想要到阁后去。向皇后笑了一下,随即便放了手,放了淑寿到后面去找她的弟弟。

“那蜀国哪儿去了呢?”朱贤妃问着。

六哥赵佣虽是她自己身上掉下来的亲骨肉,但为了国家未来的安泰,却是一开始就放在坤宁殿,由皇后亲自教养在身边。为防在皇后心中留下芥蒂,甚至不敢多问。只能去问应该在坤宁殿的蜀国公主。

“刚被召去了保慈宫。”向皇后说道,“才从保慈宫过来,坐下来还没说上两句话,说是二叔、三叔进宫来了,就又被召去了。”

“二大王、三大王难得入宫,平常在宫外也不容易见面,也难怪太后会急着招蜀国回去。官家从崇政殿出来,多也会去保慈宫。”

“不知蜀国会不会向太后说些什么,听说王诜在南面还是没改了旧性子。到扬州后,连着半个月都招了官妓饮酒作乐,前两天消息传到官家耳里,官家差点就要将他给贬去广南。”

朱贤妃叹道:“如今的几位大长公主和长公主,就数蜀国最委屈。”

向皇后陪着叹起气来:“天家的女儿能不委屈的,就只有唐人了。总不能学她们的样儿吧?大宋的公主只要沾点边,外面的言官就不会放过啊。”

朱贤妃明白向皇后说的是谁。

三十出头就病逝的仁宗长女秦国庄孝大长公主,她也只是跟驸马夫妇关系不和,换作是普通人家,早就去官府申请判了和离了,但天家的女儿却没办法离婚,只能分居了事。之后又被御史寻小过连番弹劾,以至于郁郁而死。

这还是仁宗最疼爱的长女,而且与自幼养在宫中的英宗如同一母同胞的兄妹一般亲近——英宗在宫中时,就是寄养在其母苗贵妃那里——但结局还是如此让人惋惜。大宋的公主,没有一个能如唐朝公主那般恣意妄为的。

如今的蜀国公主情况也差不多,夫妻之间的关系很是紧张,去年因为苏轼的案子,王诜之后没多久便被御史台盯上,连带的贬责出京。王诜带了小妾上路,而留下了蜀国公主。这样的丈夫,对女子来说的确是个不幸。幸好唯一的儿子种过痘一直都是健康活泼,成了蜀国公主唯一的寄托。

向皇后心中暗自叹息,王诜风流倜傥的名声,在京城中都是有名的。她的小姑尽量想要做普通人家的新妇,从不想有什么特殊的待遇。但她的身份在那里,又不能对夫婿伏低做小,还要维持天家的体面,如何能讨丈夫喜欢?

只是这件事说得也无奈,议论了几句,两人都不想再提了。

“前几天因为蹴鞠联赛赛后的那点事,御史台还说不能让韩冈出掌资善堂,但现在开封府断案,好歹是还了韩冈的清白。”朱贤妃并不是关心韩冈,她只是为了儿子。

前些日子新学、气学争道统什么的,向皇后和朱贤妃都不是很懂,但其权力斗争的本质,她们无论如何都不会看不明白。赵顼打压气学,她们不关心。但让韩冈去管资善堂,却是关系到她们的未来。

向皇后没有太多想法,自家自夭折了一个公主之后,还能再生的可能性很小,并不指望还能有个嫡子出来继承大统。眼下保住唯一的皇嗣,就是最大的心愿了。

皇帝的身体一向不好。为了能多几个儿子,又不得不旦旦而伐,宫中嫔妃雨露均沾的结果就是身体每况愈下,虽然一时间还没有大患,但以年过而立的皇帝,完全看不出正当壮年的精悍。正在后面与表弟和姐姐嬉闹的赵佣,便是宫中后妃眼下最大的希望。

“官家也说过,钱藻是能吏。”作为一门显贵的外戚,向家很早就被拉进齐云总社之中,向皇后虽然对家里面的作为并不喜欢,但事到临头,该站在哪一边还是知道的。何况,外戚飞鹰走马,本来就是免除祸患的不二法门,也不能说家里面的两位兄长做得不对。向皇后道:“官家不为别的着想,总是会为六哥儿多想想。蜀国今天来,其实也是想让益哥给六哥儿做陪读。”

“六哥儿和益哥的年纪都还是太小了一点,进了资善堂中,就怕他们心思不定。”

朱贤妃有些担心。出阁就学,便是正式昭告皇子拥有了帝位的继承权,这当然是好事,但也意味着赵佣从此,有点小过都会落到外臣的眼里,

“挂个名字就好,正式开蒙怎么也要到七八岁才是。”向皇后道,“有韩冈侍讲资善堂,也能对六哥儿多放心一点了。”

朱贤妃点了点头。

同一件事,不同的人,从各自不同的视角,得出的结论是不同的。皇帝也好,朝臣也好,资善堂的重启,对他们来说更多的是道统之争的余波。可对于向皇后和朱贤妃,资善堂的作用就是保护她们唯一的儿子,从身体健康,到未来能否顺利登基。

相应的,这也就是在保护她们自己。否则让了二大王得偿所愿,她们最后恐怕连个名分都没有——太祖的孝章皇后年纪轻轻便辞世,后事连应有的礼制都没有得到。谁愿意变成第二个孝章皇后?——至少在皇宫中,资善堂得到了大部分人的欢迎,纵然是向皇后、朱贤妃以外的嫔妃,皇弟即位还是皇子即位也是完全不同的两个概念。

在这件事上,她们是有共同利益的。

不过,也是仅此而已。宫里面的内命妇,面对朝堂上政局的变化,除非因此牵连到自家人,否则大多数人都是不关心的。她们只会在首饰、服装、脂粉、香水之类的东西上下功夫。

并不能说她们当真是不关心政事,但外有充斥朝堂的士大夫,内有正当年的天子,还有后宫不干政的祖宗家法在,宫中的女性,除了太后、皇后和生了皇嗣的朱贤妃,其他人不够资格去操心外面的政事,有精力还不如多为娘家人争取一点好处。且就算是向皇后和朱贤妃,她们所在意的,也只是皇嗣的地位问题,其余的纷争,不能议论,更不敢议论。

“对了,”向皇后突然响起了什么,“昨天招了东莱郡君今日午后入宫,差不多快要到了。”

朱贤妃带着些许惊喜笑道:“蜀国上次还说许久没见东莱郡君了。”

王安石的女儿能在宫中受皇后和嫔妃另眼相看,不受她父亲的牵累,一部分是靠了韩冈那个药王弟子的名头,但剩下的则是她本人的品行讨人喜欢。

“就是知道蜀国今天要来拜见太后,才去招东莱郡君入宫的。”向皇后说道。对于韩家的子女,不论是运气,还是神佛庇佑,总归是让人羡慕的。若能沾一沾光,又有谁不愿意?

