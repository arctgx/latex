\section{第32章 金城可在汉图中(15)}

风中的暖意越来越浓了。

战火如火如荼的日子里,春日的阳光却和煦得几乎让人忘了战争。

但也只是几乎。很多人都想忘掉,但没人敢忘。

太谷县中的气氛就像被拉开的弓,弓弦一点点的绷紧,几近崩裂,几乎让人窒息。

城中百姓脸上看不到笑容,酒店青楼更是没了生意。而太谷县城门因为附近已经有辽人的远探拦子马出没,也只在巳午未三个时辰开放。在这个时候,甚至连地痞泼皮、浮浪子弟都识趣的乖乖留在家里,让县衙变得好生清静。

整座县城中,唯一还有些生气的就只有韩冈的帅府行辕所在。只是行辕的位置并不是在县衙。

太谷县毕竟仅是县城,城中没有大齤规模的公共建筑,县衙和学校都算不上大。韩冈没去抢知县的地盘,也没理会几个富户的讨好,而是将他的帅府行辕放在了城南的名刹普慈寺,把一群和尚赶得到处跑。

太谷县大小庙宇数十,辽兵将至,城外光化寺、圆智寺等庙宇的僧侣大半都逃进了城中。理所当然的借住在城中的几间寺院中,挤得人满为患,连净信庵的比丘尼都不得不跟安禅寺的和尚做了邻居。

韩冈这么一来,普慈寺中上百个和尚被扫地出门,却是连一个秃头都不见踪影。大雄宝殿成了白虎节堂,就差把如来佛祖像给推了。倒是弥勒殿好一点,但也只是因为不算大且后面禅房足够住的缘故。

韩冈吃过午饭,在寺中闲逛消食,一时起意走近弥勒殿,就看见里面章楶跪在蒲团上。

“质夫求的什么?家宅平安?”韩冈笑着跨过门槛。

章楶仿佛没听到韩冈的玩笑,端端正正的拜了两拜,然后方起身回头:“都说枢副如六一,甚厌浮屠,今日一见果不其然。”

韩冈哈哈一笑:“至少我还没有给儿子取个‘和尚’做小名。”

六一居士欧阳修是有名的憎厌佛教,给儿子欧阳发取小名叫和尚。外人问他何故,却说贱名好养活,就跟平常百姓给子女起个阿猫阿狗一般。韩冈对佛门的态度虽与他差不多,却还不至于拿儿女的名字来开玩笑。

章楶抿了一下嘴,像笑又不是笑,显然对韩冈对佛门的态度不太适应。“方才章楶求的是战事顺遂。若这一回能胜过辽人,章楶愿重修金身为报。”

章楶信佛,韩冈却一点不信,不然也不会大模厮样的占了寺院。

抬头打量一下大肚带笑的弥勒佛,“质夫兄有所不知,这普慈寺在治平年间曾经重修过,金身一时用不着修。还不如修座塔,镇一镇辽人的阴魂!”韩冈笑意微敛,眼神有几分阴森,跟着却又咧开了嘴,“白塔其实不错,七层那是最好了。”

章楶皱了一下眉,却不打算细问韩冈究竟是什么意思,韩冈对佛门没有多少敬意,不会有好话的。

章楶拜佛起来,眼睛在弥勒殿转了一圈,却没找到一根没点过的香。韩冈将人赶得干净,他手下的亲兵将房子也打扫得干净,真的什么都没有了。

叹了一口气,章楶算是放弃了,问韩冈:“枢副可是已经有十足把握了?”

韩冈摇头笑:“打仗嘛,一阵风都能改变胜负,谁敢说有十成,那肯定是骗人的。”

章楶眼神专注的盯着韩冈,沉声:“但至少有成算,否则枢副当不至于冒此风险。”

“质夫你倒是对我有信心。”

“这几日看了枢副的布置,有了几分信心。”章楶说道,“当然,还有枢副过去的战绩。”

韩冈苦笑摇头:“胜负之望,不当归于一人。”

“可这一回枢副驻足太谷不就是希望北虏只将眼睛放在一人身上?”

韩冈闻言转头,对上了章楶迎过来的双眼。章楶的眼神中看不到挑衅,极是沉稳。


对视了一阵,韩冈方开口:“……我的确盼着辽人来赌上一把,就不知道能不能成功了。”

“枢副所说的成功,是北虏来攻太谷?”

“总不能看着辽人带着贼赃安然回返吧。”

章楶点点头,表示认同。至少在韩冈的话中,能听得出来,他对于与辽人在太谷决战,有着充分的信心。

“金太谷、银祁县,榆次的米面吃不尽。太原府也就是这一片最富。只从辽人的秉性上,就不可能放过这一片地方。”

韩冈说着跨出弥勒殿,章楶跟在他身后,“有枢副在,辽人或许会先放过呢。”

韩冈呵呵笑:“我好歹比金银更值钱一点吧?”

章楶已经五十出头了,几乎是王安石的那一辈人。不过中进士很晚,快四十方得中,所以官位并不高。莆田章家进士出得也多了,宰相、状元都出过,年近四旬方才踏入官场,升迁很慢,前途又不算大,让章楶在家族中也不是很受重视。不过倒是对了章敦的眼——章敦父子在族中一向是另类,纵然已经贵为枢相,还是没有太多的改变——这一回能担任韩冈的参议,也是章敦力荐的缘故。否则因为伐夏之役中所受的罪责,他还要耽搁几年才能重新被重用。

伐夏之役,章楶为转运判官,与已经去世的吕大钧为同僚,辅佐转运使李稷运输粮秣。伐夏之役未尽全功,战后议论功罪,负责粮草转运的官员没一个落了好,章楶也不例外。

对于章楶,韩冈了解得不算多,只是这几天相处下来,感觉还是一个很有能力的官员,尤其是在军事上,与自己很有些共同语言。

从弥勒殿出来,韩冈和章楶一同往大雄宝殿过去:“听说质夫兄旧年曾经游学天下?”

章楶点了点头:“整整十一年。河北、关西和成都都去过。”

“读万卷书不如行万里路,难怪质夫兄对天下地理兵事有此见识。”

“远不如枢副广博。”章楶的回赞是真心实意。几天下来,韩冈对天下地理的见识,让章楶深感敬服。甚至难以理解,深度和广度完全不像是这个年纪应该有的。甚至连福建的山水地势都能在他这个本地人面前说得头头是道。而且绝非胡诌,却像是亲眼见证过一般。

韩冈笑着摇摇头,这件事完全无法解释,幸好大雄宝殿就在眼前,也不需要解释。

已经有八百年历史的普慈寺的大雄宝殿殿门敞开着,在门外守卫的不是秃头,进进出出的也没穿僧衣,在释迦牟尼的注视下,依照地图刚刚制作完成的巨幅沙盘就放在大殿中央。

旁边还有一幅小一点的,则是太谷县的城防模型。

殿内的气氛很是紧张,以黄裳、田腴为首的一众幕僚,或围着沙盘,或坐在耳室之中,也有亲兵捧着,来回奔走。

韩冈新招募的幕僚陈丰也在这里,就在耳室中抄写公文。可惜韩冈传信回去,找一个叫宗泽的两浙士子,现在还没有消息。而另一方面,去了北方的韩信也没有消息,不知道他有没有遇上秦琬,甚至不知道他是死是活。

听到韩冈进来的动静,各有各事的幕僚和士兵全都停了下来,齐齐转身向韩冈行礼。

“都说过了,在这里,礼数就免了。”韩冈无奈的轻叹,“都去做事吧。”

回头看看,在众人行礼时,章楶早避让到一旁。

来到太谷县城的城防模型旁,韩冈停了步。方方正正的城池,完完整整的在五尺见方的沙盘上复制了出来。

太谷城周九里又一百步,城高两丈五,以县城的规模来说,已经很大齤规模了。如果放在南方,许多州城的城墙都没有这个高度——其实在南方,许多县城、甚至州城连城墙都没有,有一圈篱笆就算防御了——可放在北方,也只能说,毕竟只是县城。只是换作是州城、府城的话,辽人是绝对不会攻打的。

韩冈负手站在沙盘前。

之前他曾遣人带信去太原,对满城军民承诺说二十天内援军必至。现在距离预定的时间,还剩九天。时间越来越少,而韩冈的目的也越来越明确,如果辽人如其所愿的话,决战便已迫在眉睫。接下来就要靠这一座并不算雄伟的城池,来抵挡辽军的围困以及进攻。

“北虏真的会来吗?”章楶在后问道。黄裳等几名亲信幕僚也聚了过来。

“如果不来的话,就只能等陕西和麟府的援军一起到了,才能将他们赶出太原了。当然,”韩冈抬头对众幕僚笑道:“他们也就没有进一步扩大战果的机会了。已经打下了河东,仅仅是劫掠一番就北返,恐怕不是耶律乙辛所愿。”

不论大宋还是辽人,其实都在寻求决战的机会。只要能在决战,便可以打破现在的僵局,使得对手转为绝对的守势,接下来的几十年就可以为所欲为了。

就像太宗皇帝第二次北伐后的大宋,从那时起,即便是在澶渊之盟签订后,大宋都是处于弱势的地位。直到变法开始,经过了开拓河湟,南征交趾和灭亡西夏一系列战争,使得宋军的战斗力直线上升,方才改变了这一局面。

但在何时、何地决战,却是一个大问题。必须是有利于己,而不利于敌。

在河东,辽军占据了上风。韩冈很清楚,萧十三能有的选择,远比自己要多。即便援军安然赶来,辽人也可以施施然的返回代州。胆大一点,还可以利用机动力来个各个击破。

而韩冈,除了拿自己来做鱼饵,就没有别的更好的办法了。

他驻扎在太谷县是为了引诱辽人南下决战,《御寇备要》也同样是在逼迫辽人南下决战。

都是同样的道理。

大宋四方援军将至,而眼下,就是辽人最后也是最好的机会。
