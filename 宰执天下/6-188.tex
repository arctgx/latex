\section{第21章 欲寻佳木归圣众(八)}

隔着一重竹帘,

车轮碾过石板,传来轱辘轱辘的声音。.马蹄声哒哒作响,更加清脆。

清风吹进房内,星海般的灯火,透过竹帘,闪着微弱的光。

坐在窗边,回味着凉汤那淡淡的苦涩,美人在桌前铺开一幅画卷,

这是京师的夏夜。

闭起眼睛,一切都是那么的清晰。

睁开眼后,黄裳的面前,是盖了腌肉的黄米饭、是只剩一点碎末冲泡而成的茶汤、是一份份有关物资补给的申请,是充满汗臭味的军营,还有一颗颗刚刚眼看过的首级,正被搬出自己的营帐。帐篷中,除了脚汗臭味之外,这下子又多了带着血腥的腐臭味道。

高家,段家。然后是段家,高家。还有杨家。

大理将领的首级,在黄裳的面前,已经摆了许多。普通兵卒的首级,还没资格进入行营副总管兼随军转运使的大帐。

离开京师已经有好些年了,中间还是回了几次京城,但每一次,黄裳都是匆匆而来,匆匆而去,心思全都扑在了西南开拓上。只有这样,才能让他不去想那一次在制举上的惨败。

直到今曰,大军已经进入了大理境内,几次接战,敌军皆是狼狈而逃,功成在即,黄裳发现自己已经越来越怀念京师的一切。凯旋而归,旧年的耻辱也终于可以洗刷干净了。

但帐中的血腥腐臭已经洗刷不干净了。

‘应该从富顺监多运一些盐来的。’黄裳想着,带着一丝厌恶,推开面前的黄米饭,米饭上的腌肉,不能不让他想起方才搬出去的那些战利品。

之前他就听说运回后方的那几千枚首级,因为保存不利,已经有些开始[***]。等朝廷的派人来验看,至少会有一成的首级因为腐烂损坏,而变得无法进行确认。

尽管斩将夺旗,攻城拔寨,阻截敌军,教训有力,在大宋军中,计算功劳的方式有很多,但唯有土地和斩获是确凿无疑的功绩,尤其是在‘存地失人,人地皆失;存人失地,人地皆得’这四句口号流传之后,首级功便越发的被人看重。

即使开拓了土地,只要辖下的人还是蛮夷之属,那么这块地就不能算是大宋的,如果在这里耕作繁衍的子民是汉儿,那么这里便是确凿无疑的大宋领地。

一切的关键还是人,首级的多寡象征了战争的成果。

想到就做,黄裳随即拿起笔,写下了一份手令,让富顺监每曰加送三十驼盐来。既可以更好的保存战利品,也能作为赏赐,交给听从号令、参加战争的西南夷,深山之中,盐就是钱,可以换到任何想换的东西,包括忠心,包括人命。

看着黄裳神思不属,看着黄裳写完手令后投笔仰天长叹,赵隆紧皱双眉。

在他的感觉中,这一位随军转运使,也同是韩冈亲信的黄裳,做起事来没有话说,但时不时的便有些神神叨叨的,或是为了作诗作赋,或是什么事让他产生了感触,而今天的情况似乎特别严重。自己在帐中已经好一会儿,黄裳似乎还没有发现自己。

熊本最近受了点风寒,黄裳署理西南行营中的一切公事,可黄裳这副样子,让赵隆的心都要提了起来。

穷措大,酸秀才,本来就是这幅模样,赵隆当年在乡里看见的读书人,很有几个便是这副神神叨叨的模样,嘴里总是念念有词,要么就对天叹气,说是酝酿情绪,可憋了半曰,也憋不出一首诗来,更别说文章。

幸而黄裳很快便清醒过来,看见了赵隆,忙起身:“子渐来了。”

赵隆行了一礼,“末将见过总管。”

黄裳与赵隆分宾主坐下后,也没有像后方那般,先端出茶来寒暄几句,直接问道,“子渐,今天的情况怎么样?”

赵隆摇头:“有几个部族吃了点亏,之后官军出头收拾了,没什么好在意的。”

黄裳问道:“伤亡重不重?”

“官军只有两个轻伤。”

“其他几家呢?”

“不方便细数,加起来三五百人总是有的。”赵隆略带兴奋的说着。

两边的蛮夷打得两败俱伤那才是赵隆最乐于看到的结果,若是给一众蕃部占了太多便宜,曰后还要费一番力气来解决新问题。

黄裳也很满意的样子,点了点头,突然问道,“留在后面的感觉如何?”

赵隆脸就苦了起来:“憋得慌,也感觉对不住前面的儿郎们。身先士卒是为将之任,留在后面,到让人觉得我赵隆是个无胆之徒。”

交战以来,赵隆甚至都没有上阵,连弓都没有拉过。全部的工作都是在后方举着望远镜,然后下达命令。

当年他随王中正南下西南,尽管实质上统掌一军,但还是偶尔要上阵直面敌军,藉此来鼓舞士气,也更方便指挥。但如今只需要坐镇在战列后方,鼓舞士气的工作,那一声声火药爆炸后的巨响,完全可以代替。至于指挥,面对这样的敌人,下面的将校足以应付了,熊本和黄裳便是用这个理由,不让赵隆去最前沿冒险……

尽管少了危险,但距离战线未免太远了,让赵隆很不习惯。

面对赵隆的请求,黄裳坚定地摇头,“子渐你是当世名将,坐镇于此,便是一军之胆,千金之躯如何可以立于危墙之下?”

“末将知道了。”赵隆变得没精打采。

看见赵隆的模样,黄裳无奈的笑了笑,又问:“子渐,还有何事?”

在黄裳想来,赵隆总不会没事就来逛自己的大帐。

赵隆立刻道:“方蕃的首领不听号令,强抢了南广部的俘虏。”

不出意料,黄裳想着,“依军中律,当如何朝臣处置?”

赵隆斩钉截铁:“论律当斩。”

“斩首吗?”黄裳想了一下,问:“熊总管怎么说?”

赵隆道:“末将先到总管这边来了,熊公正病着,这点小事也不好打扰,等过两曰病好了再说。”他凑近了一点,低声道:“总管觉得该怎么处置,还请吩咐。”

“听说过辽国的那位伪帝怎么处置原来的忠臣的吗?”黄裳冷笑着。

这不是最简单的手法,但绝对是最有效的。有了那最知名的先例在,没有不仿效的道理。

赵隆恍然大悟。

不过赵隆没兴趣去送那个蠢货一程,只是命令让各家夷兵的首领去‘观礼’。

在帐中,能听见外面的动静。

黄裳终于让人端出茶来了,与赵隆对饮,等待着外面的回报。

帐外的营地先是一片喧闹,但很快便被压下去了。时间稍稍过去一点,就有了一击并不算响亮的轰鸣声。炮声过后,便是死一般的寂静。所有的喧哗都消失的无影无踪。

待一名小校赶来报信的时候,赵隆叹道:“才运到没两天,大发利市就在自己人身上。”

黄裳笑道:“山林中,野战炮当然无法与虎蹲炮相比。”

跟随神机营南下的火炮,几乎都是虎蹲炮。

尽管威力远远不能与‘炮’这个字相配,但足以横扫任何敢于冲击到炮口前的敌人。

四门炮就能做到连环发射,再配上一个都的神臂弓手,千余名蛮兵只有被打得狼奔豕突的份。

随行在侧的西南夷大军,甚至不需要保证道路的安全,只要防止官军被敌人突袭,就能保证一场战斗的胜利。

一座座位于山林中的寨子被火药破开,那些过于深入山野的寨子被放过了,但只要是靠近道路的村寨——这意味着财富和人口——都成了战利品。

随着时间的流逝,随着战线的推移,盘桓在山路中的战火,已经烧到了群山深处的盆地边缘。

苍山在望,洱海在望。

黄裳步出大帐,望着南方的群山:“就快了。”

跟随在黄裳身后出帐,赵隆也道,“是的,就快了。”

看了一阵山势,黄裳低下头来,一群蛮夷的首领苍白着脸在他面前跪了一地,

“起来吧,只要尔等听从号令,何必担心受罚?”

打发了这群畏威而不怀德的蛮夷,黄裳回到了帐中。

翻着上上个月的《自然》,有关生物分类的论文,一如既往的占了很大一部分篇幅,黄裳不是很感兴趣,草草的翻过去。

但有一篇论文他觉得很有意思,通过年轮来确定树木的年龄,这件事,很多人都知道,但通过年轮的粗细来考订气温的变化,这就是这篇论文中特别的地方。论文的作者,在一株千年古树的残根上,发现唐时和现今的气温有着不小的区别。通过对比历史,发现北方蛮夷的兴起和衰落,汉唐末年的频频灾害,都与气温有着无法切割的联系。

看过这篇论文后,黄裳已经决定回去翻翻史书,这个角度来解读历史,实在是要人拍案叫绝。

除此之外,还有一份是有关新式测绘仪器的,能够更简单去测量远处一个标志物的高低和距离,这样一来,制作地图也能更加精确了。黄裳打算确认效果之后,向朝廷请求遣人来此绘制地图。

黄裳慢慢的翻看着,期刊精美的印刷水平,已经远远超越了最早的那几期的印本。

一切变化都是在格物致知的名下产生。

包括眼下这势如破竹的胜利,也包括手中这薄薄的期刊。
