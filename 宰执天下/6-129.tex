\section{第12章 锋芒早现意已彰(十)}

山雨欲来啊。

“还要刮风下雨啊?再来几场风雨,这房子也撑不了多久了。”

王厚在火炉前搓着手,堂堂提举皇城司,就像外面做苦力的民夫,不顾体面的蹲在地面上。

“皇城司的衙门不知多久没修过,头顶漏水,四面漏风,脚底下直冒寒气,我当年住过的吐蕃帐篷,也没这么破败过。”

王厚好一通牢骚,轻轻的将李宪的话茬给撂开

王厚蹲着,同在厅内的李宪也不方便站着,一起蹲在火炉前烤火。

听了王厚的牢骚,他苦笑了起来,王厚的话太夸张了一点,皇城司提举的公厅,只要出现漏风漏雨的地方,肯定会立刻补上,但破旧倒是没错。

“谁让朝廷看得紧,宫里面但凡有点钱想修一修屋子,主要都紧着庆寿、保慈两宫和官家,哪里轮得到皇城司?”

王厚的双手搓得刷刷响:“但这改火炉的事倒是蛮快的。弄得人想烤个火,还得凑到屋子边上来。”

“先帝的事后,谁都怕炭气,只能这么改了。”

李宪坐在皇城司公厅中处理公务时,也不免觉得脚底板冷。换做过去,拖个火盆过来就好,可自从熙宗皇帝因炭气中毒而崩,宫中和衙门里的火盆全都改成了有固定通风通道的火炉。这么一改,就成了固定的设施,想烤个火,要么挪摆设,要么就是人凑过来,要么就干脆再点个火盆。只是这两日风大,关紧了门窗之后,即便胆大如王厚,也不敢再使用火盆——毕竟这间房,并不是像他说的那般一直在四面漏风。

“改也不知改好点。”王厚冷哼着。

“等日后再改吧……不知提举对辽事如何看?耶律乙辛篡位,辽国必然内乱。王平章求战,吕宣徽亦求战,偏偏其他宰辅都反对出兵,这事情,真是让人看不懂。”

李宪很直接的将王厚避过的话题又拉了回来。

王厚低头看着炉膛:“都知,非所宜言。”

“要是那两位国戚在,李宪是绝对不敢多说半句的。”李宪盯着王厚,“可眼下只有提举你我二人。”

自己几乎都挑明说宦官不当问军国事,李宪还如此坚持,那就只有一个可能了。王厚抬起头,笑了起来,“是太后想问?”

“自然!”李宪正色回答。不是奉了太后的诏令,他如何敢妄言半句?

“太后想知道韩参政的心意,为什么不直接问?”

“太后不是问过韩参政吗?韩参政也只是反对。”

李宪摆出了个‘你懂的’的表情,世间都在传韩冈等宰辅反对出兵辽国,不过是蒙蔽辽人在京城中的细作。最近皇城司抓了一批辽国细作,惹得朝野沸沸扬扬,正巧给印证上了。

“市井传言多自宫中出,太后知道韩参政心中有顾虑,也怕消息泄露,所以改命李宪来问一问。提举与韩参政最善,韩参政的想法就要托付给提举问个明白。待问明白了,再转告给李宪。此事出君之口,入宪之耳,除了太后和韩参政,决不会让第五人知晓。”

“韩参政正当面的回答,太后都不信,还要遣都知来问王厚,难道都知以为韩参政侍君不诚?!”

王厚依然笑眯眯的,但他的话让李宪不寒而栗。当真触怒了韩冈,太后绝不会保他。

……………………

李宪走了。

可以说是被王厚吓走的。

可将这位同僚和旧交识给吓跑,王厚只是轻叹了一声,转身离开了皇城司的衙署。

此时快要到放衙时间,离开皇城的官员渐渐的多了起来。

四门巨型的火炮,依然在宣德门内矗立着。幽黯的炮身,如磐石一般,坚不可摧、份量十足。

这四门炮是个摆设,但火炮绝不是。

有了这等军国重器,守住宋辽边界要比之前容易了千百倍。

可攻打辽国却没有那么简单。

要不然如章敦、韩冈这等知兵的宰辅,绝不会如此强烈的反对。

耶律乙辛不能失败,但大宋也同样失败不起。

在幽燕之地的一次决战惨败,就意味着数万精锐不得归乡,禁军承受不起这样的损失,也没有哪个宰辅敢于冒险。即便是王安石和吕惠卿也不敢。

他们的支持,只不过是党争又批了一层皮而已。

可直到今日,朝廷依然平静的很,没有人

王厚慢慢走上宣德门,就看见章敦和韩冈,一同向城门这边走来。

王厚心中一奇,这倒真是难得一见了。

……………………

章敦已经有好几天没有好脸色了。

不论是在朝堂上,还是在衙门中,又或是在家里。

黑着脸的表情,就像是凝固在了大宋枢密使的脸上。

王安石支持出兵,吕惠卿也支持出兵。

这明摆着是想要将吕惠卿从地方上拉回来。

谁敢将一名支持对辽开战的安抚使放在河北?

王安石和吕惠卿这么做,越来越像是当年旧党元老反对新法时的手段了。

“既然吕吉甫想要回来,就回来好了。”

韩冈虽是如此说,章敦没有在韩冈的脸上看到半点在意。

“当无此必要。只要朝廷不同意出兵,吕吉甫又怎么使动河北各部禁军?而且介甫平章和吉甫说的只是出兵与否,又岂有他意?”

章敦的话,他自己都不相信,只是他总不能公然附和韩冈,指责一名同在一党的宣徽使。

“如今民情奋发,吕吉甫想要开启边衅,将责任推到辽国身上,还真不是一件难事。”

辽国有内乱之忧,但敌国的危机,便是本国的机会。国子监已经为此沸腾,数千太学生都盼着趁辽国内乱之机,能够彻底解决北方的大敌。民情奋发,一心好战的念头,便是国子监第一个带领起来,

“玉昆!”章敦厉声打断韩冈的话,韩冈这可是在指责吕惠卿会无视朝廷的命令,擅自出兵。

“主动出兵不可能,要是辽军入寇后又如何?”韩冈笑道,“到时候,韩冈也只能顺水推舟了……不是吗?”

章敦脸色更拉长了几分,快要赶上驴子的长脸了。

太后如今对韩冈言听计从,韩冈要是提议调回吕惠卿,太后肯定会为此下诏。可要是韩冈提议下诏斥责吕惠卿,太后也绝不会拒绝。

一个好战的河北安抚使是朝廷所不需要的,但无罪又不当左迁,诸路安抚使又以河北最尊,调往他路亦不可行,将其召回京城就是最简单的办法。

可万一韩冈对王安石和吕惠卿的计划根本不理会呢?

只在背后顺手推吕惠卿一把,让其骑虎难下,这就是最好的应对办法了。

如果朝廷给与的物资越来越充分,吕惠卿当真就能够领军攻辽吗?就凭他这个从来没有指挥过千军万马的文臣?

章敦对此表示深深的怀疑。

……………………

韩冈回到家中后不久,王厚就找上门来了。

“玉昆,怎么样?”王厚略带紧张的问着。

“家岳真是越来越像洛阳的文相公了。”

韩冈摇着头,不掩心中的失望,王安石为了将吕惠卿调回来,还真是费了些功夫和心力。

或许这是王安石在不情愿的情况下配合吕惠卿,但既然王安石他既然这么做了,也证明了他将党争放到了比国政更重要的位置上了。

王厚一愣,他本来还以为韩冈会多说几句吕惠卿,或是与章敦商议的事情,没想到韩冈会直接指责自己的岳父。

“家岳大概以为我会谏言太后,将吕惠卿调回京来,免得他擅起边衅?”

“玉昆你打算怎么做?”王厚问道,“找个御史弹劾吕惠卿?”

除了请动乌台中人,王厚已经没有别的主意了,,

否定吕惠卿的提议,又让他不能调回京城,最简单的办法就是找专职之人,也就是御史。动用御史咬上一口,整治在外的吕惠卿,这样不是什么难事。但韩冈没有去做。

“找谁去?”韩冈摇头,御史台早就不是原来的那一座了,里面的一个比一个滑溜:“我是绝对不会先用御史,气学的根底比新党差多了。”

王安石和韩冈这对翁婿,一直以来没有撕破脸皮,都是靠了双方的克制,没有动用御史台这个凶器。一旦韩冈指使御史台弹劾吕惠卿,那么就是党争大戏的开始。气学门下的官员,远远比不上新党成员的势力和地位,一场大战下来,韩冈或许能保住二三核心成员,但气学在朝堂上的一点班底,怕是要给连根铲除了

没有意义的事,韩冈不会去做。两败俱伤,而且己方伤亡更重的战斗,韩冈更不会有兴致。

但河北诸将,又有哪个不喜战功?事到临头,总有愿意拼一拼,找一个封妻荫子的机会,

王厚也同意韩冈的看法:“若是辽国国中有几人愿意反正,打算为官军带路,的确没有多少将校能忍得住。到时候,被领进了陷阱中,可就不妙了……这样的例子,史书上从来没少过。”

“其实已经有了。”韩冈笑了起来,笑容如外面的天气一般冰冷,“真的很及时吧?”
