\section{第31章 离乡难知处(中)}

在汶子山上并没有多逗留,韩冈一行很快就下山返回县城。

——别说满目疮痍的黄河两岸,就是不停地传入耳中的叮叮当当的凿石声,在山头上也待不了太久。

汶子山虽小,也是白马县的一处名胜,但千不该万不该,就不该是座石灰山【注1】。此山翠石棱棱,山无余土,岩洞泉壑,堪称绝胜,可这等露在地表的石灰矿,在黄土厚积矿床深藏的白马县,看到了就不能放过。

不论是疗养院还是流民营中,用到石灰的地方都很多。韩冈当初来到黄河岸边,一看到这座小山上尽是洞穴,对文人风雅并无多少兴趣的他,就知道捡到宝了。现在离着汶子山只有半里地的石灰窑烟火不绝,每天都能出产上千斤生石灰。

也就是因为现在煤——或者按此时的说法,称作石炭——不足,使得石灰窑的规模不能扩大,否则一天上万斤也没问题。到时候不论是修桥铺路,还是修造房屋,都能派上大用场,而不是像现在,仅仅局限于日常消毒和简单的整修官道。

沿着官道,经过了两处流民营。营地规模都很大,但其中只有少数区域建起了窝棚,能看得见炊烟。不过现在县中的深井打得差不多了,这时候除了组装风车机械的,其他流民都开始拿着工钱在流民营内部开挖沟渠,以及窝棚的地基。

韩冈在第二座流民营停下马来,走进去。偌大的营地被纵横的主路分割成十几个片区。而片区之中,还有更小的巷道。其中一个片区已经有了住户,而其他区域,也能看到有人在挖着沟。

在营地偏东侧的地方,是深井所在。只见高高架起的风车旁,一群人围着上上下下的敲打。正是到了组装最紧张的时候,而周围的地面,由于井水的缘故。只是在此住持的王旁却是毫不在意的挽着袖子,穿着草鞋站在泥泞的土地中,完全看不出来他是宰相的儿子。

韩冈也不避泥泞,走过去道:“仲元,情况如何?”

王旁回过头,见着是韩冈。也笑呵呵的反手指了指已经架起来的风车,“玉昆你放心,等到晚上就能装好出水。”

韩冈看了看正在组装着风车的人们,皆是专心致志,并没有人注意到自己的到来。满意的轻轻点头:“多亏了仲元兄。”顿了一下,又道:“既然快要搭好了,这里就交给下面人收尾,待会儿仲元你跟我一起回城里。你也该歇一歇了,不然莫说你妹妹要怪罪小弟不会体恤人,回去后我也不好向岳父岳母交代。”

“玉昆你每天比愚兄忙得更累,也不见你多歇一歇。”王旁抬头望着高高的风车,带着自豪感的微笑中透着满足,“愚兄还是亲眼看着风车汲出水才能放心,现在回去可睡不好觉。”

不过十几天的功夫,王旁瘦了也黑了,但他的精气神已经不同过往的郁郁,眉宇间多了一份光彩。作为饱读诗书的士人,王旁终于等到展示自己才华的一天,当然是不辞辛劳。

虽然刚开始的几天出了点笑话,但接下来他遵照着韩冈定下的规条,来主持开凿深井和打造风车两件事,都是很顺利。关键也是在他宰相之子的身份上,没人敢糊弄他,反而要在他面前尽力表现自己的才干,故而这进度远远超出预计之外。

王旁又看了风车两眼,拉着韩冈稍稍走远了一些。指了指正在用竹子和木头搭建饮水道的匠人们,“玉昆,用了这么多竹木,是不是浪费了一些?直接在地上掘沟不成吗?河水还不是照样能喝,东京城中可是多少人家靠着金水河!有水井,或是向外买水的毕竟还是少数。”

“不一样啊。”韩冈摇了摇头,从深井引出的地下水要从井口利用引水道,引向营中每一个片区,虽然用了许多防洪物资,但绝不是浪费:“东京城中的饮用水除了井水外,都是靠着金水河。而金水河上都覆着石板,日夜有人巡守。可流民营中就不行了,若是饮水道设在地面上,污水流入,必致疾疫,只能用竹木搭起架子来。不管怎么说,人命比钱要贵重。”

五处流民营,尽管现在只启用了两处,但五座流民营都拥有至少一座深井,以及随井安置的风车,同时还搭建了引水道,保证供给流民们洁净的水源。另外还建有足够数量的公共厕所,加上消毒防疫用的生石灰绝不会缺少,对于在营中防止疫病的传播,韩冈有着足够的信心。

听着韩冈如此说,王旁也不坚持,只是问一问而已。“即是如此,那愚兄也会多照看着,督促他们不能偷懒耍滑。”

“那这里就拜托仲元了,等风车组装好,早点回城休息。”韩冈说着,又吩咐了王旁的随从好生照看,随即告辞离开。

离开营地,韩冈回头望去,还能看到矗立在风车下的王旁的身影。他摇头感叹着二舅哥的变化:‘终究还是要出来做事,否则闷在家中,心理当然会有问题。’

一路顺顺当当的回到县衙,县丞侯敂就迎了上来。如果不是穿着官袍,白马县中差不多也没人会记得除了韩冈之外,县衙中还有一个县丞。

韩冈是七品朝官,朝堂上官阶与他平齐或是在他之上的文臣,也不过三五百人。仅仅是选人的县丞侯敂哪有与他分庭抗礼的能力,几个月来被压制得一点存在感都没有。现在一说县里的官,就是小韩县尊,至于侯县丞,就是一摇头,他是谁啊?

倒是县尉冉觉的名气几个月来大了不少。

为了在韩冈面前表现,冉县尉每天都带着乡中的弓手,披星而出,戴月而归,巡视县城内外。一些原本横行乡里的所谓的江湖好汉,冉觉为防万一,也全都尽数敲打过。有产业有家室的加以训诫威胁。而无产的泼皮无赖,就直接提溜到大牢里去,不管有理没理先打上一顿,翻出过往罪愆,请韩冈审了,该流放的流放,该充军的充军,一点也不宽容。冉觉下手之狠,让县中的一众强人鸡飞狗跳、狼奔豕突,皆是偃旗息鼓,不敢犯事做过。一时之间,白马县倒给整治出了一个夜不闭户出来。

侯敂是个四十多岁的中年人,是荫补出身,已经在官场沉浮有二十年。他做事很稳重,也不爱出风头,平日帮着在县衙中拾遗补缺,勤勤恳恳,任劳任怨。

他们都是聪明人,当上司忙忙碌碌的没空坐下来吃饭的时候,有几个下属敢于安坐钓鱼台,懒懒洋洋的晒着太阳?也是同样忙得跟狗一样。更别提两人都还另外抱着着一份心思在。

向韩冈行过礼,侯敂立刻道,“正言,盛林乡大保保正方才遣人来报,上午的时候有了河北流民从野渡渡河,已经进入县中。”

野渡就是私人摆渡的渡头,而官营的渡口则称为官渡——不是三国时的官渡——白马渡就属于官渡,而白马县中这一段,也有几处野渡。不过通过野渡渡河,远比不上官渡安全。渡口之所以能建立,也是因为地理和水文的优越,否则天下行人商旅,何必聚集于此地渡河?

韩冈听了就问道:“人数有多少?”

“有七十多人。”

听着人数不算多,韩冈也算放心,笑道:“他们也是心急。我日前已经奏请天子,将白马渡的渡资就此免除,以免流民无力渡河。”

“这……”侯敂犹豫起来,小心提醒道:“白马渡渡资一日几近百贯,渡头上的艄公也是靠着分到的渡资养活家人的。”

“艄公的工钱县中会给他补上,但渡资肯定要免的。”韩冈坚持道:“任其流落河北饱受饥馁之苦并非朝廷之福,若是他们尽数移往野渡,甚至是私下里造筏过河,不知会有多少人出意外。”

“正言仁德,侯敂感佩不已。”侯县丞不吝谀词,捡着机会,就开始大拍韩冈马屁。

冉觉不是蠢人,侯敂又怎么会是瞎子?五座流民营,现今虽只有两千多,可每一座的规模都至少能安排下一万流民。这不是为了东京分忧又是为了什么?现在韩冈当面说得明白,更让侯敂这位县丞了解到他的用心,这一番折腾就是要留着流民在白马县。

既然知道顶头上司所想,聪明的官儿当然明白该怎么做。朝廷中的争斗,他们这等小官没得插手,而眼前这一位虽然地位还不髙,但很显然前途不会受到岳父太多影响的韩冈,他的大腿现在不抱,那还等何时?

冉觉清剿县中无赖、强人,而侯敂则是兢兢业业,与韩冈的三名幕僚密切配合,让韩冈可以顺心畅意的施展自己的才华。

注1:汶子山,后名为紫金山。与此时位于黄河中心的居山【后称凤凰山】都是由石灰岩构成的山体,如今已经被采石场挖成了坑,不复存在。

