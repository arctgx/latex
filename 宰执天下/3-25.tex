\section{第11章 立雪程门外(中)}

程颐与邵雍关系不佳,也不是没有缘由。程颐之父程珦,表字是伯温。而邵雍给他的儿子,起的名字也是伯温。要说避讳的话,不是一家人,也无需讲究这些。但抬头不见低头见,同时洛阳城中的闻人贤达,互相之间总得给个面子。儿子什么名字不能起,偏偏要用上程家老父的表字。

程颢性格洒脱,对此并不在意,大不了不去叫邵家长子的名讳就行了。而程颐是极重礼法,对于父亲的字号成了邵雍儿子的名字,一直隐怒在心。

程颢程颐兄弟俩性格差别显而易见。曾有一次两人去赴宴,在宴席,主人找来了一批妓女。程颢安坐如素,宾主尽欢;而程颐却是拂衣而去。到了第二天,程颐仍是怒积于心,而程颢则笑道,“昨日本有,心上却无;今日本无,心上却有。”

所以邵雍也只跟程颢走得多,程颐是附带而已。前日邵雍写诗,说起洛阳贤达,就是富弼、司马光、吕公著,然后便是程颢,没有程颐的份。

这一番内情,也算不上秘密,连张戬都听说。韩冈到盩庢县拜访他的时候,还被他叮嘱了一番,莫在程颐面前提邵雍。邵雍虽然是大儒,但世间流传的却是他算卦批命的本事。张戬也是担心韩冈兴头起来,跑去请邵雍算上一卦,算算他能不能考上进士——进士考前烧香拜佛的事很常见,张戬也不是白担心——让程颐听到了,可就不会有什么好脸色。

送走了邵家仆人,程颐回头跟韩冈告罪,言辞间不掩对韩冈的欣赏。韩冈的态度摆得很正,任何一个教授弟子的老师,没有一个不想见到能如此尊师重道的弟子。

问了几句张载、张戬的近况,程颐便道:“为天地立心,为生民立命,为往圣继绝学,为万世开太平。玉昆,最后一句你说得的确是好。”

前面翻阅张载来信时,程颐一眼就看到那四句必然光耀古今,为后世儒者明道的名言。虽然读信时气定神闲,但心中也是激荡不已。张载和他的弟子们喊出的这个口号,振聋发聩。张载一直提倡的‘大其心’,使得关学一脉的气魄,让其他学派难以企及。

“也是几位先生教授之功。”韩冈顿了一顿,“同时是韩冈在河湟数载所历种种之后,才有的一番心愿。”

“玉昆你的行事为人,子厚表叔在信中多有夸赞。在河湟战事激烈的时候,仍不忘揣摩大道,更是难能可贵。”

程颐客套了两句,便带出了自己要说的话。

韩冈冲着程颐拱手致礼:“格物致知一说,在子厚先生那里也有闻及。不过韩冈更多的,还是两年前在京城伯淳先生那里受教的结果。韩冈自得了伯淳先生的开悟,回去后便事事留心,风吹草动,马拉车行,皆拿去格。日久功深,也终于小有心得。”

韩冈并没有标榜张载,而是将提点之功归于程颢。但程颐明白,他和程颢所说的格物致知,却与韩冈所说的根本不是一回事。都是想自万物中找出永恒不灭的道,但各自走上的路,是截然不同!

在二程之前,无论是汉时郑玄、唐时孔颖达,都是把‘格’解释成‘来’,将格物致知四个字倒过来解释,知善事,来善物,知恶事,来恶物。到了今朝,汉唐的解法被宋儒抛弃,各家便有各家的解释了,但还是小家子气为多,比如司马光,将格说成是抵御——抵御外物之诱,然后方能知至道。

二程所言格物,却是穷究万物至理,格出来的是形而上的大道。这一点,可以算是他们所首创,也让他们傲视其余众家儒者。

而韩冈的格物得启于程颢,可格出来的道,却没有脱离有形之物,反而更近于形而下的器。所谓的力学三律,都是直接作用于外物上,从里到外都是张载气为本源的认知。大其心是大了,但未免太过于浅薄。

程颐毫无避忌的将自己的看法说了出来,并说道:“正如湖海之别,想那洞庭、鄱阳,虽然广阔如海,又近于世人,可究竟不如海之渊深。”

身为一代儒门宗师,必然已经拥有了自己的道路。在大道已经走得很远,又怎会为他人之言所影响?韩冈也没能指望可以说服程颐,而他也不想跟程颐这位主人吵起来。

“万事万物皆有道,皆是韩冈所欲知,吃饭读书时,亦处处可见。”韩冈微微欠身,不与程颐咄咄逼人的眼神对视,“力学三律,韩冈偶得之,不敢称知为大道,但推及他物,亦能得以验证。能知一物之源理,便可推而广之,此便是道。致知明道,便可以诚心用于天下。”

程颐气貌凛然,而韩冈则谦和有礼,但气氛却是紧绷着,大道之争不同于他事,不可和而同之,互相之间都难以说服。

程颐也知道,韩冈既然能从简简单单的四个字中,就自己开创出,虽是韩冈自己都说是要‘以旁艺近大道’,自承是旁门左道,但‘近大道’三个字,也可见其心,根本不会轻易改变观点,当然更不可能这么容易就被折服。

两边有些僵持不下。这时候,一名穿着仆佣衣服的老者,在书房门外敲了敲门,然后走了进来。

这是是程珦自少带在身边的书童,现在又成了程家的管家。他向着程颐和韩冈各行一礼后,便问道:“老仆受命来问二郎,今天家中可是来了稀客?”

“稀客?”

程颐看了韩冈一眼,张载的这位弟子也的确算是稀客了。毕竟不常见啊……

因为让老管家带话的是程珦,程颐站起来后才点点头:“玉昆的确是稀客。曾经在京中听过大哥的教诲,还带了横渠表叔的信。”

老管家冲着韩冈一躬身:“即是如此,那就请客人到正厅相见。”

……………………

“……那韩小官人立于门外,身上头上全是雪。程家看门的六丈出来后,请他进去,抬起脚,留下的印子怕也有一尺厚了。”

邵雍面前,回来的邵家仆人说得夸张,今天的雪也没那么大,但邵雍知道,至少韩冈冒着雪在程家门口等候了很长一段时间的这件事,是不会错的。

韩冈的名字,邵雍依稀也听说过一点,年纪轻轻的朝官当然受人瞩目。何况前段时间,河湟功成的消息传到洛阳时,程颢也提起过他。

听说了今天这一事,邵雍忍不住要感叹着:“不意横渠弟子守礼一至于此。程府门前犹如是,子厚面前当可知了。”他就站在一边的儿子邵伯温,“大哥儿,你也要跟着学一学。”

邵雍年过四旬方娶妻,生儿子更晚。虽说邵雍已经年近六旬,但长子伯温也不过十五六岁。

邵伯温一扬脖子,不服气的道:“所谓‘靡不有初,鲜克有终。’如今虽是谦抑,日后未必还能如此。孩儿听说韩冈近于新党,又奔走于王介甫门下。非此,如何得以幸进?”

邵雍一听就觉得不对劲了,立刻问道:“这话是哪里听来的?”

“只听着外面都这么说。”

“此时妒其得用的非毁之言。韩冈能出人头地,那是他用心国事,另外自有他的缘法在。”邵雍看着儿子点头称是,但神态中人不是如何信服,无奈的摇头。他暮年得子,儿子读书也算用功,打是舍不得打的,只能板起脸来,道:“年节前,你且在家安心读书,勿要再往外去,更不要多言妄语!”

‘富家也要少去。’邵雍却没把最后一句说出口。

邵雍并不算敌视新法,虽不认同,但也不会强烈抵触,算是温和派,至少不会像旧党的司马光、文彦博那般仿佛不共戴天的性格。也不会如富弼那般,一听到新法就皱眉头。

前次李中师【不是李师中】知洛阳河南府,推行新法时,上门考订富家的户等,并逼着富弼与普通的富民一样,缴纳免役法所规定的免行钱。

富弼三朝元老,新法要钱要到他的头上,这个面子就丢得大了,没听说相州知州敢收韩家的免行钱。富弼本人倒也罢了,年纪大,也算看得开,也就上书抱怨了一通。但富家的儿孙没有这个气量,私下里将王安石和李中师衔之入骨。

尤其是最近让王安石得赐玉带、彻底坐稳相位的王韶,以及熙河路的一众官员,在富家子弟嘴里,都没有一句好话。

“我邵家乃是诗书传家,旧年更是隐与乡里,不欲与外人结交。岂料因缘际会,方来到这洛阳城。承蒙几个相公不弃,多有亲近。但你父我究竟还是个白身,与官宦人家走得太近,可就会忘了自己站在哪里。”

邵伯温被父亲说得脸色发白,不是因为羞愧,而是暗恨着。回想起来,富弼的几个孙辈,与自己交往的过程中,的确没有太多的尊重。的确,宰相家的子弟,岂会真的看重自己。

他更是无法理解,以父亲的大才,为何不出来做官?

若父亲真要有个官身,他邵伯温日后岂会输于哪灌园小儿。

