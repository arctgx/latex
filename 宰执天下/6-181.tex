\section{第21章 欲寻佳木归圣众(一)}

关西大汉的一曲高歌,苍劲有力,然而不见婉转,更乏韵律,但熊本却静静的听完。

慢慢咀嚼着‘苍山如海,残阳如血’八个字,望着眼前的千里叠翠、如血残阳,直至夕阳入山,漫天的红霞渐次淡去,熊本方才回头,招了那名唱曲的士兵问道,“这是谁人所作?”

“回大帅的话,小人不知。不过以前曾听人唱过,觉得好听,所以学了来。这曲子词,在西军中传唱有好些年了。”

“大帅。”黄裳在旁插话道,“黄裳倒是知道这首曲子词的来历。”

熊本转过头来,讶异的看着黄裳:“哦,勉仲当真是广博。”

“非是黄裳广博,而是知道的人多。”黄裳回手指着身后的一群将佐中的一员,“大帅问一问他知道了。”

熊本一看那人,更是讶异:“周全?你知道,”

周全行了一礼:“回大帅的话,小人的确知道。”

一张络腮胡子的大脸,在一众粗鲁不文的军校中几乎没有什么差异,甚至很难分辨出来,但右手上的铁钩子,让熊本都知道周全这个人。

原本是西军中的一名小卒,后来因伤残离开军中,投到了韩冈的门下。再后来,被韩冈提拔去制作和实验飞船,继而因功授官,让多少旧日同伴羡慕。周全在军器监中多年,功劳苦劳都不缺,等到神机营成立之后,便被调到神机营中任指挥使。

寻常的禁军指挥使最多是三班借差,未入流品的杂阶武官,这还要是京营上位禁军中的马军指挥使。而神机营的指挥使,每一个都是有品级的三班使臣——天武军中的指挥使,都不一定有品阶。

而周全所带的神机营的这个指挥,也没有辜负朝廷给予的特殊地位。一刻钟的功夫,便炸毁了地处险要的石门关,南下第一功,周全和他的这个指挥,是拿定了。

熊本皱了皱眉,却没追问周全,而是转头问赵隆,“赵隆,你知不知?”

赵隆应道:“末将知道。”

熊本掉过脸,对黄裳道:“可是与韩相公有关?”

黄裳微微一笑:“大帅真是神机妙算!”

“哪里是什么神机妙算。”熊本摇头。赵隆与韩冈相交于微末之时,黄裳是韩冈的幕僚,而周全则是韩冈的家丁出身,三人都知道这首《忆秦娥》的来历,那还用再费神去猜什么来历?

“周全。”熊本再点了周全问话:“这首词可是韩相公的手笔?”

周全摇头,“回大帅,这首词不是相公作的。当初王枢密刚刚拿下熙州狄道,奉旨回京,韩相公主持熙河路公事,各州各县都走遍。相公做事,那是快得很,每天最多一个时辰,闲来无事,便爱游山玩水。小人跟着相公,在临洮的一处山壁上发现了这首曲子词。因为是用墨写的,字迹已经辨认不清,相公好不容易才分辨出来,只是落款没有了,不知是谁人手笔。”

熊本呆了半天,突然间哈哈笑了起来,“又是路边看来的?”

周全有些楞,“啊,是啊。”

“勉仲,你信吗?”熊本大笑着问黄裳。

“相公这么说,黄裳如何不信?”

熊本连连摇头,韩冈的医道,便是倒在路旁破庙中,被药王给救了。而当初西太一宫中的一首小词,因一曲道尽离人之悲,被誉为秋思之祖,却因为作者不详而传得沸沸扬扬。韩冈也曾被传为是作者,之后又有传言说这是韩冈在路边看来,随手写在西太一宫墙上的。不说韩冈到底能不能写来,这个路边看来的,倒真是可圈可点。

“周全,你家相公还说了什么?”

“相公找了工匠来刻字时,还说如此佳作,岂能不传于后世?”

熊本再问黄裳:“勉仲,你怎么看?”

“这首《忆秦娥》遣词用字不是今人腔调。”

这首词,文采不好说,仁者见仁,智者见智。用词是完全不合当今体例,黄裳虽然曾经猜测过是不是韩冈所作,但通观全篇后,就又否定掉了。韩冈不喜文辞,黄裳做了多年幕僚,怎么会不知道?而且韩冈本身的文采不足,同样是事实。就算一时偶得,也不会有不合今人腔调的句子。

熊本疑惑起来,“苍山如海,残阳如血……大漠孤烟直,长河落日圆,这是唐人气象。确非今人手笔。”

赵隆看着熊本皱眉苦思的样子,难以理解的摇了摇头,对他这等军汉来说们,这首词,只有单纯的赞赏。除了十八摸之外,还是这样的曲子词,唱着让人爽利。

一首《忆秦娥》,不过是战后小小的插曲。或许在日后的文人笔墨中,此时的一番对话远比刚刚夺取的关城还要更值得记录,但对于当事者来说,没有比战斗的结果更重要了。

在天色完全黑下去之后不久,被派去追击败敌的人马回来了。

官军拿下石门关后,残存的乌蒙部蛮兵向后方逃窜,赵隆便派来自熙河路的蕃兵追了上去。虽然乌蒙部的蛮兵无不熟悉道路,更善于山中奔行,但行走在山地中甚至还能骑马的番兵,也不会差到哪里。

在五尺道上逃跑,蛮兵们又是相互拥挤踩踏,绝大多数人甚至还没能发挥出他们所擅长的山地奔行,便被身后的人推倒踩踏,或是被一柄钢刀砍断脖子。

回来的番兵,给出了斩首千级的战果。最后打扫战场的工作,就交给了一直在做看客的蛮兵们去处理。

石门关后的十里血路,这就是乌蒙部大军最后的结局。

之后数日,官军稍事休整便向南继续进军,而清扫残敌的任务,则全部给了蛮军,其中以南广部和马湖部最为卖力。

乌蒙部是个大部族,披毡佩刀居住栏棚,不喜耕稼,多畜牧,其人精悍,善战斗,自马湖南广诸族皆畏之。乌蒙山上的一片草甸,是这个部族的中心,而后一干分支,分布在方圆数百里的区域内。

南广部和马湖部与其同为石门蕃部,道路最熟,恩怨也最多,他们领着官军和外来各部,将乌蒙部的老底全都给揭了开来。

乌蒙部于石门关上主力尽丧,残存的那点兵力,根本没有抵抗的能力。

而有官军在背后支撑,一众蛮部有了底气之后,更是士气高涨,原本见了乌蒙部的战士,顿时就要矮三分的南广、马湖两部,现在趾高气扬的,五六分的实力,都能发挥出十二分的水平了。

冷兵器的战争就是如此,士气高低在极大程度上决定了战争的胜负。乌蒙部惨败之后,人心惶惶,族长、长老等一干能聚拢人心的领袖皆尽战死,士气也泄得一干二净,死的死,降的降,跑的跑,连像样的抵抗都没有,石门关之战后十日,乌蒙部这个雄踞蜀地之南的大部族已经成了历史,子女,财货和土地皆被瓜分得一干二净。

当石门蕃部的战事抵定,大理国承诺的援军仍未到来,而皇宋一方的分赃都已经结束了。

南广、马湖两部得到了他们梦寐已久的土地,其余各部西南夷则得到了乌蒙部的人口,至于财货,则全数归于官军。

乌蒙部占着入滇的要道,看着不起眼,其实家底丰厚,给族人装备的武器,也远比周围各部精良——这也是乌蒙部能成为石门蕃部之首的原因所在,从乌蒙部的财产中得到的分红,参战的官军没有几个不满意这一次的收获。

用极微小的代价,便换来了巨大的收获,怀里揣着抵上半年俸禄的财货,天天吃着鲜嫩的牛肉马肉,数千官军的脸上哪个不是充满了喜悦和贪婪,这一下,每个人都在想着大理的好。

带着酒意,走过营地中的一堆堆篝火,从起身行礼的那些蛮人身上,赵隆也看到了同样的喜悦和贪婪。

赵隆将不屑和冷笑藏在了心底,这第一战是让这些蛮夷捡了便宜,但下一回,就没那么多便宜可以给他们捡了。

再想要好处,可就是得拿命来换了。

赵隆望着南方黑暗的天空,大理国的成色,先得用那些蛮人试试水。

……………………

来自西南的捷报,在十天后抵达了京师。

数千近万的斩首,也没能让京师百姓动容。京城中,对于这样的胜利,已经感到麻木。

如今京城百姓之中议论最多的,还是前一日,在一场球赛中踢进五个球的高季。剩下的话题,则被宫中所豢养的御马浮光的儿子,在一千五百步的赛道上三战三捷的喜讯所占据。

再有的,就是京泗铁路开通之后,从南方来的商货价格降了一成以上,包括江南产的棉布在内,这让京师百姓兴奋不已,不过粮价没变动,所以还不如蹴鞠和赛马的消息让人震动。

至于朝堂上,当然是越来越近的廷推占据了所有的话题空间。即使官军通过这一战,一举攻到了大理国境上,也不过是平平淡淡的几声称赞,没有告祭太庙,也没有群臣称贺,倒是派去点验首级,验明功绩真伪的官员,被早早的派了出去。
