\section{第19章 虎狼终至风声起(下)}

【终于赶在十二点前完成了。今天维持了两更。求红票,收藏/】

七月流火。

七夕节过后,别名大火的亮星心宿二开始向西移动,应和着出自诗经中的这一句,昭告着秋天的到来。

‘不过……’韩冈抬起头,就算隔着浓密的树荫,炎炎烈日的热力只剩斑驳的光影,可照在身上依然能清晰的感受得到。蓝色的天空被阳光映得发白,“白天是看不到星星的……”

“谁说白天看不到星星?”

来自身后的插话,让韩冈笑了一笑:“当然,太白昼现的时候从来没少过。”说着就转回头,就见着王厚几步并作一步,追了上来,与他并肩往王韶的官厅走去。

“看玉昆今天又是春风满面……”王厚看了看韩冈,便想开开他的玩笑。只是韩冈眼睛转过来这么一瞪,就让王厚咳嗽一声,正色道:“玉昆可是说岔了。十几年前,出现在毕宿天关东南的那颗客星,时交五月,正是夏天的时候。可是连着在白日里亮了二十多天!”

“是至和元年【西元1054年】的那一颗?”韩冈在前身的记忆中找到了答案,而在他自己从后世带来回忆中,也同样有着答案,‘是蟹状星云的超新星。’

韩冈对天文学只是稍有了解,不过这已经足以让他知道爆发在北宋,而在几百年后变成蟹状星云的这颗最为有名的超新星。

“玉昆你还记得啊!”

“那时小弟才几岁,怎么可能记得?”韩冈摇了摇头,“是后来听说的。说是开国一百多年,没有一颗客星能有这么亮过,比太白星还要亮。”

“现在想想,至和元年好像也没有出什么大事。”

韩冈总觉得王厚的语气中,好像隐隐有点遗憾。“客星、客星,既然是来做客的,那会跟主人家过不去?这恒星可没有反客为主的说法。”

“反客为主……郭逵来了,肯定是能反客为主的。”王厚突然压低了声音:“郭逵干脆别来算了!现在的李师中老实得很,日日待在后院里,只每天早晚各出来一个时辰视事。”

“怎么可能不来?!”韩冈摇头失笑。

王厚对郭逵可是顾忌得要命,而他的担心又不是毫无理由。天子对郭逵的评价是‘渊谋秘略,悉中事机。有臣如此,朕无西顾之忧矣。’

以郭逵的身份,就是一具大佛,放在哪里,哪里就会被他镇住。要想斗赢他,至少也得是枢密副使韩绛那个等级。

不过正如王厚所说,要是过去的李师中能跟现在一样老实,韩冈他们肯定巴不得他能留任。只可惜木已成舟了。

说话间,两人已经走到了王韶的官厅前。

王韶的官厅中,再没有了前些天的忙碌,厅内跑来跑去的胥吏,此时只剩两三人还在王韶身边服侍着。而因为一堆堆从架阁库搬来的旧档案,而一直都弥漫在厅中的灰尘,也被前两天的雨后清风刮得一干二净。

秦州这边该忙得都已经忙完了,古渭寨前两天王韶韩冈他们也去过了一趟。现在高遵裕尚蹲在古渭寨中,他是缘边安抚司同管勾,让他先处理一下衙门中的事务。而王韶则在这里收拾首尾。等着郭逵来后,也会搬去古渭。

韩冈、王厚跨过门槛,走进厅中。

王韶抬起头:“玉昆,二哥,怎么一起来了?”

“在外面碰上的。”王厚回了一句,跟着韩冈一起上前给王韶行礼。

韩冈直起腰后,道:“下官方才把秦州疗养院的一应准备又查看了一遍,应该没有问题了。等到郭太尉接任之后,请他把建造疗养院的营盘划过来,交给仇老郎中,下官就可以去古渭了。”

王韶点了点头,韩冈能把他管的另外一摊子事未雨绸缪的提前办好,这是最好不过。要是到了古渭,身边没了韩冈帮忙,有许多事都做不顺畅。

“哦,对了。玉昆你看看这个。”王韶想起了什么,递过来一份公文。公文露出的背面是由白色绫花的绸绢制成。

韩冈心中一动,接过来打开,便露出了里面的黄色纸面。

‘果然是敇!’。

他再习惯性的看了一眼最后的印章和画押,就看到了天子和政事堂大印,以及副相陈升之和以王安石为首的几个参知政事的签押。

有宰相执政签押,并奏覆天子,而由中书门下颁布的命令,就称为敇。而敇书,通常都是写在浅黄色的纸张上的。

不过敇书的质地倒没什么,关键是里面的内容。韩冈一目十行,看完后便抬头笑道:“终于来了。”

“是啊,”王韶也是轻松的笑道,“终于来了。”

这是韩冈前日撺掇王韶上的奏章的回覆。韩冈想给自家弄块地皮,手上却没什么钱财,便跟王韶和高遵裕商议过后,上了一份奏章,请求在古渭寨附近,划出一片宜垦荒地,供给缘边安抚司的官吏和古渭寨中驻军的将校们。

‘如果在古渭任职的官吏都不敢在当地置办田产,怎么能让招募来的百姓安心屯垦’——韩冈想出的理由光明正大,现在提前请了上命,日后也不怕跟御史打嘴仗。

同时,韩冈想要做买卖,让冯从义出面赚钱来补贴家用,但他手上没有本钱。幸好王韶有钱,他主管市易,手上有着数万贯的本金——韩冈前次用度牒作为借款抵押的提议,现在朝中的回覆也出现在这份公文中,同样得到了允许,三百份度牒,可以一半抵押给秦州、一半则抵押给陕西转运司。

——所以韩冈便又撺掇王韶在奏章上建议,朝廷发给缘边安抚司的市易本金,可以借贷给商人,用出息以佐军需——这是惯例——并请求允许官吏亲眷和门客借贷。不过他们借贷的利息要比普通百姓高上一成。

在外人看来,这是防止主持市易的官吏监守自盗的措施——因为基本上所有榷场的市易贷款,许多时候都是落到官员的亲眷和门客手上——故而在这份敇书上,甚至还能看到隐隐的赞许。

韩冈其实也可以不多此一举,私下里让冯从义从王韶那里借钱就行了。不过那等做法,常见却不合法。在朝中和秦州本地都始终有人用不善的目光,盯着缘边安抚司的时候,却不能这么将把柄送给人拿着。韩冈要未雨绸缪,为自己接下来的行动找来一个合法的名义。日后御史找起他的麻烦时,也可以一巴掌反手打回去。

多出一成的利息,他并不放在心上。边境回易,向来是高风险高回报。商队被抢掠的有许多,但满载而归的则更多。把风险和回报权衡起来计算,其利润往往有三五倍之多。

而在新开的榷场中,交易的风险大大降低了,而利润虽然也会因为要缴税而降低,但降低的比例并不多。官员在任职地经商,本身就有先天上的优势,可以把交易的风险压到几乎为零,而利润由于身份的关系,反而会增加。

最后能得到的利润,韩冈自己计算过,也让沿着渭河在永宁、三阳这一带,跑了一年多冯从义计算过,据韩冈所知,王韶让元瓘也算过,而高遵裕同样让他的门客计算过。最后的答案都差不多,就算要多给出一成利息,仍能保证有一倍半的利润。

“只多付了一成的利息,利润依然能保证,而且还有了朝廷的背书。这笔买卖做得也算值了。”韩冈笑着把敇书递给王厚,让他看。王厚则摇了摇头,他方才是出去办事了,这份公文其实已经看过。

王韶抬手收回了敇书,对韩冈笑道:“也是玉昆你才会想得这么周全。”

韩冈谦虚的躬了躬身,对王韶的赞许表示感谢。

王韶觉得韩冈这个人有时很难看透。勇猛直进、行事果决的情况不少。但很多时候,他又能把事情做得像几十年的老吏一般滑不留手,不留后患。这般行动处事的手段,张载是绝对教不出来,韩家夫妇也绝对教不出来,真不知他从哪里历练出来的。

而韩冈的这些提议,也是多方得利的典范。屯田之事就不用说了,给官员田地,朝廷肯定不吃亏,而韩冈给的借口其实也是事实。

市易贷款之事,朝廷也不亏,官员的亲属来借款,朝廷就能多得一成利息。至于官员本身,他们的利益也可以得到保证。

“最多四个月!……其实三个月就够了,七八九这三月,是商队来往最多的时期,光靠这三个月赚到的钱,足够吃上一年了。而榷场可是开办在古渭寨旁,光是占个好市口,就能财源滚滚。”

这是当日韩冈与王韶、高遵裕商议几条建议,元瓘这个假和尚表示支持时所说的。能合法合理的攫取财货,王韶也不会清高到表示拒绝。

世事通明,人情练达。王韶觉得韩冈当得起这八个字了。

几天后,从陇城县连夜传来了消息,新任知州郭逵,以及宣诏天使李宪,一行人已经在县城中,

当天夜里,就被派了出去。第二天清早,李师中终于从衙门的后院中出来,带着秦州上下的一众文官武官,远出十里之外,迎接郭逵和李宪。

随着夏末的烈日逐渐升上天空,昨夜派出去的迎宾骑手,也带着消息,一匹一匹的返回。

“郭太尉和李御府已经动身。”

“郭太尉和李御府已经出城。”

“郭……已经到了二十里外。”

“……十五里……”

“……十里……”

当最后一匹骑手回来,车马声已经清晰可辨。远远的一片灰黄色的尘头高高扬起,被一阵突入起来的狂风卷入云霄。

弥漫的黄烟渐渐散去,绵长的车马队伍出现在秦州官员们的眼前。让秦州上下等候已久的郭逵郭太尉,终于抵达了秦州。

