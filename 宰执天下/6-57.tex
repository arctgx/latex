\section{第七章 烟霞随步正登览(九)}

“龙图阁直学士、司勋郎中、判将作监王益柔——吕嘉问。”

当年王益柔参与了苏舜钦的进奏院祠神会,席上有‘醉卧北极遣帝扶,周公、孔子驱为奴’一句,使得范仲淹、富弼等庆历党人全都连累,被吕夷简一网打尽,庆历新政也因此不得不宣告失败。

之后王益柔在官场上郁郁不得志,为监税多年,直到王安石推荐其直舍人院方才有了起色,不过又为张方平等人所攻劾。

当初王益柔为知制诰,文章被人评价是‘野妪织机,虽能成幅而终非锦绣’,在士林中引以为笑谈。之后也因为答高丽国书不工,被罢去知制诰、直学士院。眼下判将作监,乃是作文不成,改做工了。

但他作为龙图阁直学士,依然握着扎扎实实的一票。

这是吕嘉问的第七张票,让他再一次保持领先。

而这也是新党三人总共获得的第十五票。

王安石望着屏风,十余年的开拓耕耘,新党的底蕴全都在这里了。

在范纯仁出面支持韩冈后,朝堂上参与的旧党彻底倒向韩冈。

原本以为他们不会选择任何人,或是干脆推举出一名旧党中人参选,不过在最坏的打算中,也预计到了所有不能确认去向的选票都改为支持韩冈。可即便韩冈能够自己投自己一票,新党的优势依然是在他的三倍以上。

现在就有十五票,之后还会更多。

如果新党的选票能集中在一人的身上,就算韩宗道和李承之依然叛离,照样能让韩冈臊得没脸接手太后的任命。

可是谁也不敢让新党众人一起来支持自己。就是吕嘉问也只敢出面联络个七八票便安分下来了。

那是自杀,实在是太危险了。就算是宰辅,也不当获得泰半重臣的支持。若当真如此人望,当着太后的面,结党的帽子戴上了就脱不掉。

党同伐异四个字,没人认为太后不会写不会念。

尽管如今新旧党争是很明白的一件事,要么站在新党一方,要么就是旧党一方,想站在中间,必须不起眼,学王存那样多年躲在馆阁中整理书籍。

而且之前流言已经传遍了京城之中,让人以为廷推之上,会南北之争再起。

但现在选票三分,尽管一众新党成员的身份依然不变,但票数的去向却是四分五裂,这样给人的感觉就会缓和许多。

只是韩宗道选择支持韩冈,让人觉得意外。而李承之的选择,更让王安石难以理解,他的一票谁也没有想到会落到韩冈的身上。这两票的去向,让人觉得新旧党争或许未现,但南北之争倒是确确实实的出现了。

尽管王安石因为不喜选举,没有插手,但可以想象得到,以李承之和韩宗道两人的身份,李定或是吕嘉问绝不会忘掉他们手中的选票,理应事先说定了才是。

现在又是怎么回事?

在王安石的疑惑中,李定又多了一票。

向太后看了看屏风,又低头看着手上的名单,忽然问道,“南人不可为相,太祖是不是说过这句话?”

杨戬悚然一惊,却不敢耽搁:“只听寇莱公如此说……只是真宗之后,南北并用,历任相公皆是大宋忠臣……”

杨戬越说越小声,因为向太后正扭过头盯着他。

太后的眼底看不出有什么样的情绪,杨戬浑身冒着虚汗,小腿肚子抖得快要抽筋。

盯得杨戬额头上黄豆大的汗珠扑簌簌的就要落下,向太后终于点了点头,“知过能改,你很好。”

得了太后的称赞,杨戬暗暗吁了口气,在袖口上将手心的冷汗擦干,这一次的赌博总算是赌对了。转头就看见屏风前拿笔的小黄门,在曾孝宽的名下,将正字补上了最后一划。

不经意间,又是一票被唱出。

这是曾孝宽的第五票。

还剩多少张票?

杨戬连忙开始搬起手指。

除去宰辅之外,有投票权的侍制及侍制以上的重臣,总数二十七人。

而现在的屏风之上,吕嘉问是一个‘正’字带一横一竖,韩冈则是‘正’下多一横,而李定和曾孝宽就都只有一个‘正’字。另外还有一张弃票,是韩冈投的。

吕七、韩六、李五、曾五,加上韩冈的弃票,业已开出的选票总计二十四张,也就是说,只剩下三票了。

杨戬望着王中正,之前在他面前堆成两堆的选票折子,都已经到了底,只剩下薄薄的三本,而王中正,已经在其中又拿起了一本。

只剩三票了,韩冈的选票还能增加吗?

杨戬记不得还有谁的票没有宣读了,依照之前的情况,很难说韩冈能否得到他的第七票。

他最后的得票数,很有可能也就是现在六张。

杨戬的手心又开始冒汗,如果当真是这样的话,问题可就大了。

接下来三票,若是全给吕、李、曾三人中的一人,或是吕嘉问占上两票,那还好说。但若是分加在李定和曾孝宽身上,或是平均给吕、李、曾三人,那就意味着,最后一名会出现票数相同的并列局面。

如果第一第二名票数相同,那没什么关系,保证第四名不如前三就够了。但两个第三名并列,或是三个第二名并列,问题难办了。

那时候,该怎么办?

好像之前也没有说好。

难道让太后挑一个第三人出来?

可这么做,这场推举还有什么意义?

原本就是让重臣们选出三人来让太后在其中决定,可现在连候选的三人都要太后来挑出,既然如此,根本就没有必要煞有介事的弄出一场廷推来。直接让太后处分不就够了?

杨戬难以想象接下来会是什么样的局面。

但李定的名下此时又多了一票。

“端明殿学士、右谏议大夫、知河阳府蒲宗孟——李定。”

这一位人缘极差,又性好奢侈,不过运气总是能好到躲过一次次风波。虽然说之前犯了过错,被请出了学士院,可还是很好运的留在了京中。不过他手底下的问题不少,李定为御史中丞,不知抓到了多少把柄。

至此,李定的票数已经与韩冈相同,而曾孝宽也紧随在后。

还没有开出的就只剩下两张了。

随着开票接近尾声,越来越多人都开始感觉到选举形势的微妙。

在齐云总社和赛马总社选举的时候,从来都不是匿名投票,每个人都要为自己的选择负责,也必须为自己的选择负责。今日朝堂之上,选票上也都是直书其名。哪一位重臣会选谁,事前大半都能猜测得到,所以韩宗道、李承之二人的变化,才那么让人吃惊。但没几个人会认为,接下来,这样的吃惊会接二连三的到来。而最后一名出现并列的情况,也越来越明显。

虽然没有事先加以约定,但廷推宰辅其实是从两大总社推举会首模仿而来。依照齐云总社或是赛马总社的例子,可以组织重臣们再投一次票,决定谁是真正的第三人。

只要如此提议,再直接问韩冈可行与否,韩冈作为廷推的提议者,难道还能说请太后决定?而韩冈同意,太后也当不会反对。若是太后反对,韩冈同样丢人现眼。

范纯仁表态之后,在很多新党成员的眼中,打击韩冈的威信,就是防止旧党沉滓复起。之前的情分早已荡然无存。

想那吕嘉问、李定与曾孝宽,此辈何能与韩冈相比。

若是韩冈不能不中选,恐怕京城百姓人人都要发此质问。

不过在朝堂上,决不能如此质问。

一旦被这样问,总会有人说,此乃为赵氏绸缪,韩冈稳居两府,恐日后不利于国家。

就算有蔡京在前面,但韩冈总不能拿着自己的前途去与所有没有投他票的重臣做赌。所谓罚不责众,韩冈的仇恨,也不可能分散到二十多人的身上。说不定只会回去跟他的浑家和岳父过不去。

黄裳相信,以新党的人才济济,肯定有人能够想到这一招。

这并不是黄裳对韩冈有没有信心的问题,而是现在面临的问题让他越来越担心。

而且也只能担心。

黄裳不无悲哀的想着。

他自问绝不会输给上面正被王中正唱名的一众重臣中的大部分。

如果二十多天后他能够顺利的通过制科,不要五六年,就能正式晋身重臣的行列。这一个是背景、功劳,另一个就是对自身才干的自信。但此时此地,黄裳却只能默默饮恨,自己无法协助到韩冈。

又是一票开出,群臣中一时按抑不住骚动。

同票!同票!!

竟然当真出现了同票。

这是投给曾孝宽的一票!

除了吕嘉问七票之外,韩冈、李定和曾孝宽这时候竟然均为六票。加上韩冈的弃权票后,正好是二十六票。

仅剩下最后一张没有开出。

杨戬脸色苍白的看着太后将手中的名单给一下握紧,但立刻又展了开来。

还有谁的票没有开出来?!

数百道目光在殿上交错,然后汇聚到了一人的身上。

天章阁侍制、都大提举市易司王居卿。

虽然比起其他重臣,他更加显得默默无闻,只是他身上的官职说明了一切。

都大提举……市易司!

