\section{第12章 平生心曲谁为伸(一)}

【第二更,求红票,收藏。】

城南驿中,一队车马已经整装待发。王厚、赵隆站在车马边,正与来送行的友人畅叙别情。

跟他们一起来的张守约因为早一步被任命为秦凤路钤辖,已经与两天前带队先走了。王厚之所以多留了两天,却是因为前日又被召入宫中,跟天子在新制的沙盘上又演练了一个多时辰——这已经是第三次了,可见天子对于军棋的痴迷程度,不下于当初的王厚他们。

京中近月,三次被召入宫中面圣。这样的恩典和际遇,除了一些个侍制以上的重臣外,也就是担任边地要郡守臣的臣子才有可能有这个荣幸。而王厚以一介微不足道的小官,却得天子垂青,在外人看来绝对是一个异数。在城南驿中,他一下变成了众星捧月的大人物,前来与他结交的官员也是络绎不绝。

“王官人,王官人。”清脆的声音从驿站的门外传来。

听到唤声,王厚欣然回头。

没错,他已经不再是王衙内,而是变成了王官人。虽然现在仍称呼王衙内也还可以,但终究没有官人中听。因为在托硕部之事,以及沙盘和军棋上的功劳,王厚恩受三班奉职,尽管并没有给差遣,可已经在三班院挂名了。

王厚整了整穿在身上的簇新的青色官袍,抬头挺胸,一副少年得志的模样。平定托硕部的功劳实在不小,几百级斩首摆在那里,托硕部的族长首酋又被送到京中,是当今天子登基以来以来,排在前三的大胜。

同时又因为没有动用官军,少费了国中钱粮,天子对这样以夷制夷的做法赞赏有加,在官职上并不吝啬。

不仅是王厚,跟随王韶参与此战的杨英、王舜臣、赵隆都因为此事而得了官身。王韶本人的本官也一下晋了两阶,是为从七品的左正言。而且散官和勋位都晋升了,一个是正七品上的朝请郎,一个是六转的上骑都尉,不过这两个名号全都是虚的,没职司没俸禄,仅仅是空名,只是让官员的头衔变长,听起来顺耳而已。

也就李信,因为先一步跟了张守约,没能沾上光。不过张守约如今已经是一路钤辖,他身边的人,说不得也会跟着水涨船高。李信现在还没个官人,不代表以后没有,也只是一两年之间的事。

唤着王厚的人从门外进来,跑得气喘吁吁,汗水顺着发丝不停的流下来,如初雪般白净的小脸上一片气促的晕红。是个才十来岁、娇俏的小女孩子。她身后跟着个面容朴实的汉子,手上提了三个包裹。

“是周小娘子身边的女使。好像叫墨文。”赵隆对王厚说着。

王厚点了点头,心中知道也该来了。他对身边的人告了声罪,和赵隆一起走上前:“小大姐,不知是否是周小娘子有书信要让王厚带给玉昆?”

“官人说得是。”墨文喘着气点头应了,又道了声万福,才从跟在后面的汉子手上拿过两个包裹,分别递给王厚和赵隆,“这是我家娘子让奴婢给王官人、赵官人送的饯行礼,且祝两位官人一路顺遂,无有滞碍。”

王厚并不推辞,这是沾了韩冈的光,当然不须推让,“周小娘子有心,王厚却之不恭,便厚颜收下了。请转告周小娘子王厚的谢意。”

墨文点了点头,“奴婢会转告我家娘子。”转身又接过一个包裹,“这是我家娘子请二位官人捎给韩官人的。”

王厚伸手接过,猜里面肯定放了信,点头道:“王厚必不负所托,回去请周小娘子放心就是。”

把要转达的话说了,要送得礼物送到,墨文又说了几句一路平安、一帆风顺的祝福,便告辞回去了。

赵隆掂着手上的包裹,对王厚笑道:“韩官人真是本事,在京中也就一个多月,什么人都认识了,连教坊里的花魁都倒贴了上来。”

王厚点了点头,看看周南巴巴的遣女使送到手上的包裹,笑道:“玉昆向为风流中人,气质出众,受到欢迎也不让人惊讶。”

“俺却是吓了一跳。今次上京为韩官人带信,几个官人都没什么,就是没想到最后一封是个花魁。……不过韩官人让俺带了五封信,如今就送到了两封,给横渠先生,还有张官人、程官人的信都没人收。”

“辞官的辞官,出外的出外,你送不到也没办法。”

今次上京,韩冈让赵隆带了五封信。有给章惇的,也有给张载、张戬和程颢的,另外就是给周南。韩冈在京中有私谊的几人,他一个不漏的都写了书信。

给章惇的信,赵隆送到了。也见到了韩冈救过的老章俞,在章家还受了不少赏钱——不,不能叫赏钱,而是以壮行色的川资——因为赵隆此时已经是个官人了。

但张戬和程颢这两个御史却在三月、四月时,与整个御史台一起,跟变法派大战了一场。最后两人都离京出外,而且不仅是他们被贬官,另外还有好几个御史都被贬了官,整个御史台都空了一半。

而张载从明州查案回来,看到自己弟弟和侄儿都被赶出京去,也跟着辞了官,回乡去了。这三封信,赵隆一个也没送到。他倒是顺路在小甜水巷好生享受了一番,把从章家拿到的银钱花了个一干二净。

以上四家,都仅是个官人而已,赵隆并不觉得有什么了不起。不过,当他去给周南送信时,一打听人家,却吓了一跳,收信人竟然是教坊中有名的花中魁首。

王厚当时在旁听了,也觉得有些不可思议,就跟着赵隆一起去给周南送信,他同时更担心没经历过多少风花雪月的韩冈,在京中被个青楼女子迷得五迷三道,最后坏了事。

不过当王厚看到周南把韩冈的信贴在心口,笑得一脸幸福的样儿,却发现事情跟他想得截然相反,反倒是这位绝色佳丽对韩冈是情根深种。

周南接到信后,就张罗着要请王厚赵隆会宴。但王厚却不敢留下,连忙拉着赵隆告辞。日后周南说不得会是韩冈的房内人,她这样的身份,王厚多说两句话都是失礼的,哪能留下来吃饭。

王厚这时幸灾乐祸的坏笑着,对赵隆道:“秦州家里两个,这边还有一个,家严在乡中又在为玉昆寻着个正室,日后韩家后院中事,有得他头痛的时候。”

……………………

赵顼此时身在武英殿的偏殿中。虽是偏殿,但一样面积广大,跟平常人家的两三进宅院也差不多大小。不过如今武英殿偏殿中,有了十几块沙盘七零八落的放着,倒占去了三分之一的地面。

赵顼在殿中漫步着,看着这些把天下山川浓缩进咫尺方圆的沙盘,心中有着一股掌控万里江山,身为天下之主的满足感。

而跟在天子身后的,却不是平常的李舜举,或是其他小黄门,而是跟着王厚一起进京的田计。他低着头,只看着赵顼的脚跟,轻手轻脚的跟在后面,神色间却没有多少紧张——说起这段时间面圣的次数,他比王厚还多得多。

“这就是河东?”赵顼在一幅新做好的沙盘前停下脚步,指了一指问道。

田计听问,抬头看了一眼。那块沙盘上,在崇山峻岭之中,从北到南,围起了几个盆地。道:“回官家的话,正是河东,另外还包括了云中。西侧的是黄河,东侧的是太行,中间的几片平原是太原等处,而北面的一片,便是契丹的西京大同。”

田计这月来奉旨制作全国各地的沙盘模型,在枢密院跟着翻看地图。他本人知道这是个难得机会,遂拼死拼活的去记忆,并不辞辛苦向来自当地的官员请教,才一个月不到的功夫,河东和陕西缘边各路的沙盘制作完毕,而田计也成了对北地山川深有了解的专家——至少可以蒙一蒙外行人了。

赵顼见着田计把大同也包括了进来,满意的点着头。回头看了看因为日夜辛苦、脸颊都凹下去的田计,对王命如此用心,赵顼心里想着是不是该给他加个官身。

李舜举这时却走了进来:“官家,东西二府的相公们已经在崇政殿等着了。”

“他们都到了?”赵顼微感惊讶,他只觉得自己在武英殿偏殿中走了两圈,没想到一个时辰这么快就过去了。

“田计,你先回去歇息两日,在月底前把河北的沙盘做出来就行了,也不用太着急。”赵顼说着,关心田计的健康。对于身边的臣子,从真宗下来的几个皇帝,其实都是很宽和的。

田计感动得跪了下来谢恩,赵顼则带着李舜举,往崇政殿去了。

虽然近一段时间,赵顼多往武英殿而来,摆弄沙盘军棋,但他还是能说抽身就抽身,不是真正的沉迷进去。

从内门进了崇政殿,赵顼的宰执们已经再等了,不仅仅是两府,连吕惠卿、章惇这些小臣也在场。今天要讨论的政事有关新法,他们也得以上殿。

不过枢密使文彦博却不管今天的议题如何,当行礼平身之后,他便给赵顼当头一棒:“陛下身负天下之重,如何能耽于游乐?!”

