\section{第11章 飞雷喧野传声教(11)}

加快修筑轨道的速度,在并代线投入使用之后,便成了宰辅们的共识。

轨道的好处,早就在方城山得到了证实,自修成后的数年间,投入的维护与更新的费用为数不少,但朝廷得到的更多。

汴河水运,每年朝廷都要投入巨额的维护费用:水门水闸的维护,堤坝的增修,纲船的修补与新造,治河厢军的日常开销,征发民夫的开支,加起来就是一个天文数字。这还不包括六路发运司这个庞大臃肿的衙门下,附在水运上的成千上万条水蛭。而朝廷收到的,就是每年由官方运输的七百万石纲粮,以及近千万绢绸、银钱等税收。数量更多的民间商业运输,仅仅是收税,而且收到的数量远不如偷漏的数量。

而轨道运输,在税收之外,光是运费就让朝廷收得眉开眼笑。同样的长度,轨道的维护费不会少于汴水,可这是不论冬夏,什么时候都能可以运输。不像汴水,冬天必须断航,雪橇也好,冰橇也好,那点运力都只能作为补充。

但韩绛、张璪看好轨道,却是因为韩冈曾经对外透露过的想法。

国家控制干线,地方豪强掌握支线,由此形成一张网,笼罩四百军州。

不论韩绛,还是张璪,都对轨道垂涎不已。留给子孙一条路,远比万亩田要安稳。

银山何如银水?山总有挖光的时候,水可是长流不息。

一条铁路轨道,就是流淌着金银的河流。坐拥轨道,没有耕作之苦,也不用担心,每年必定有收入,只要能将轨道维护好,这就是一辈子。更重要的是,轨道修建,轨道沿途两侧的土地就必须征用。打着朝廷的名义,许多犯忌讳的事,都可以做一做了。

从一开始就打算将各地豪族都拉下水,韩冈只怕韩绛、张璪不心动,如今见他们终于忍耐不住了,心中倒是以欣喜居多。在铁路轨道并代线业已完成,并成功的投入使用的时候,尽快开始京泗线的建设,让铁路轨道代替汴水成为运输的主力,就是眼下的当务之急。

“相公说的可是轨道?”韩冈反问韩绛。

“然也。”

“说起来,这的确是治本之法。”韩冈点着头,“轨道上也不缺奸猾之辈。但他们想要赚钱,私下里多挂上一节车厢就有得赚了,没必要将车给弄翻。翻在路上,不比水里,可以报个船只损坏,上面的货物就可以全数干没。粮食也好,绢绸也好,总不可能落了地没了。”

“玉昆说得正是。”张璪拍案说道,“所以方城山的情况才那么好,而并州到代州的轨道同样不差,要是京城到泗州的轨道修起来,又能如玉昆之言,太后和我等就再也不必为汴水之事日夜忧心了。”

“京城到泗州的轨道的确是该建了。”韩冈道,“其实这条路,在方城轨道运行之后不久,便已经开始准备了……只是被各种事给耽搁了,倒是让并代铁路抢了先去。”

“耽搁了一下,可也算是好事了。为王前驱,有并代铁路这条数百里的轨道在前,京城到泗州,也就是京泗铁路铺设起来,也就有了熟手可以使用。”韩冈给轨道的起名简单直接,韩绛倒是很赞赏,“河东的轨道开始使用也有一段时间了,这几个月,该出的问题都出了,怎么解决都有了眉目。这不是六十里的方城轨道能比。玉昆,你说是也不是?”

“相公说得极是。”韩冈点头,韩绛的确说得不错。并代铁路运行半年来,除了大灾大乱,能出的问题的确都出了。

不过作为并代铁路的运行情况并不能算好,经常出些意外。早到、晚点、前后相撞、脱轨,各种各样的问题,每天都层出不穷。

还有盗窃——从轨道上的铁轨,到货箱中的货物,都有贼人伸手。沿途州县颇杀了一批人,吊在铁路两侧的杆子上,也仅仅将河东盗贼的气焰打压下了一点点,需要投入更多的人力进行追剿。

不过从方城轨道调来的运营队伍,加上半年多的实践,一开始问题频频的局面,已经逐渐向好的方向转变。这就是韩冈敢于开始修筑京泗铁路的原因。

“只是韩冈之前的想法,是将并代铁路向南延长,一直通到河中府的黄河边上,京泗铁路还打算再等等。”

“玉昆说的是李诫之前的上书?”

“正是!”

才经过了几年,在参与方城轨道的修筑,并亲身主持打造并代铁路之后,李诫已经积功转为朝官,这是进士都远远不及的速度。要知道,在方城轨道完工前,李诫仅是韩冈的幕僚而已。

在并代线建设完成并交付使用之后,李诫已经成为闻名朝中的营造大师,姓名直抵御前。名气之大,比他的父亲——京西北路转运使李南公都不逊色。上一次李南公入觐,太后还称赞李诫是青出于蓝,世人也都认为他是雏凤清于老凤声。要不是李诫没有足够的文名和著作,一个进士出身早就赐给他了。

半个月前,李诫就具表上书,请求朝廷允许,将连接代州和太原府【并州】的并代铁路,从太原府向南延伸,经过汾河谷地,通向关中的解州与河中府,直抵要津风陵渡。在此处,便可借由渭河水路,直抵长安京兆府。

李诫对于轨道修造的上书,理所当然的惹起了朝廷的重视。这一条铁路,其实就是后世的同蒲铁路,只是避过了代州通往大同最为艰难的一段。通过这条铁路,关中与河东两大经济区就能贯通起来。虽然还没有蒸汽机车,但运输量也会十倍于前,对两地的商业和人员往来,有着难以估量的作用。

韩绛对关中河东连接一线的好处不能尽数知晓,但他明白,李诫之所以提出向南延长并代铁路,必然得到了韩冈的允许,甚至就是秉持韩冈的心意。

“李诫的提议不为不善。只是加上了太原至河中府,这条轨道贯穿了整个河东,直抵关中,长度差不多要超过一千五百里了,这运行上,恐怕还是有些问题吧?”

“相公一言中的。”韩冈叹了一口气,这事没有好避讳的,“说起来,还是缺人才啊。”

“也是少了历练。再多添一条京泗铁路,就可以历练出更多的人才。”张璪说道。

“从现今六路发运司的情况,就可以看得出人才有多大的用处。才一年的时间,发运司的法度就败坏到如此地步,不仅仅是薛向的缘故,也有他提拔的那些有才干的官员被调任……而此辈调任乃是正理,无可非议,但没有合格的官员填补进去,足见朝廷乏人。”

韩绛对韩冈的说法连连点头,“玉昆这话说得没错。军事、财计、刑名、水利、转运,这些事务皆需专才来管理。寻常进士就是书呆子,不经历练,贸然坐上正衙的位置,不说成事了,就只会坏事。”

韩冈道:“一榜进士四百,多是只明经义与对策,能够在实务上有所长才的,连十分之一都不到。”

张璪连连摇头,“辅弼良才,哪有那么多?一榜也就三五人的样子,再有一二十能做实事的,也就这么多了。剩下的皆是庸碌之辈,做老了官,或许能积累些许治才来。”

朝廷对官员的选拔,虽不能说唯才是举,可是但凡能够升任金紫重臣的进士,好歹都有一技之长。

最差也有些文学水平,至于在军事、财计等实务方面有一技之长的,其晋升的速度,远比寻常进士要快得多。

不说别人,只说现在已经完蛋了的蔡京,他能够在高中进士五年,就顺利转官,并进入中书任职,十年,就做上了殿中侍御史。靠得可不是长相和书法,他的才干不管以多严苛的标准,都是一等一的。

“玉昆,朝廷得人之难,尽人皆知,不知你对此有什么想法?”韩绛不想再多绕圈子了,要韩冈摊牌。

“依韩冈的一点浅见。如今有明法科,唐时有明算科,日后还可以加一个明工科。进士乃拔萃之选,其余诸科,则是招收专才,以供朝廷之用。”

“明工科,打算考什么?”张璪问道。明算科现在虽没有,但历史上有,倒是不用韩冈解释了。

“水利、河防、轨道,主要是工部所掌。”

这是给气学量身打造的科目。

争进士,西人争不过南人。明法科,以好讼闻名、拿《邓思贤》当蒙书的江西人考中的最多,七岁儿童就能在堂上引用律条,这不是陕西人能比。若是办起明算科,商业气息浓郁的江南也远远强于北方。

只有明工科,由气学教出来的学生,至少在水利、河防、轨道等营造才干上,不会输给任何人,军事上也不会逊色。这就是西人的晋身之阶,也是韩冈的交换条件。

只要能做官,正经的出身,比起特奏名之流,还是要强出不少。

几次贡举不中,就能得到参加特奏名考试的机会,然后混一个州学、县学的教授、助教,其实连品级都没有,只是朝廷给口饭吃。

若是通过了诸科考试,尽管比不了进士出身的官员,在选人阶段能够跳级晋升,但好歹身居前列就能有流内品官来做,即便不能入流品,只要做实事,得到入流的机会也不会少。

韩绛一声轻笑,“玉昆可真是一片苦心啊。”

韩冈欠了欠身,“此事利国利民。”
