\section{第31章 风火披拂覆坟典(二)}

眼前的物体已经超过了一切对火炮的认知。

整整三万五千斤的青铜,方铸成了此物。

耶律乙辛在巨炮下抬起了头,双手抚摸着炮管时,竟然带着颤抖。

这是他排除众议后得到的回报,这是他一意孤行的成果。

火炮的炮壁近七寸厚,耶律乙辛的手平放上去,两边都有黝黑的青铜露出头来。只看炮口,内径的尺寸大概就是两侧炮壁加起来。

一圈圈的铁箍,将青铜炮身牢牢箍住,以防火炮发射时,炮管爆裂开来。第一门重型火炮,在试射时将三十多名工匠、三名大工,以及两位官员同时送上了西天。第二门试制品加厚了炮管,但在试射了十余发之后还是发生了爆炸,炮管近底部的位置被炸开一个缺口,一名士兵就站在那个位置上,上半身被碎片刮过,整个都不见了。

之后这第三门炮,便加装铁箍,一圈一圈的如同箍桶一般的箍起来。到现在为止,已经试射了二十余次,试炮场的山石和城墙模型被轰碎了一次又一次,由此也摸索出了来一整套有效的降温办法。

直到这时候,这一门十三寸的巨炮,才宣告圆满成功。也上报到耶律乙辛手中,引得大辽天子亲自来的观看。

细算一下,至少要六十多头牛才能将这门重炮拖动。而行动速度更是缓慢,甚至还不如人行走的一半。想要从后方送抵几百里外的前线,得以旬日来计。

但这样的一门巨炮有着与其外形和重量相媲美的威力,能将数百斤重的炮弹送到三里之外,重重的挨上一下,什么样的城墙都会坚持不住。这是敲开那些如同硬核桃一般的棱堡的利器,就像是铁锤一般将又高又厚的城垒给砸碎。

相信只要这门火炮出现在城下,立刻就能在城头上惹起一片混乱来,

方才耶律乙辛已经看过了这一门火炮发射的场面,端的是惊天动地。

架着巨炮的台地猛地一震,一闪而过的火光从炮口喷出有一丈多长,浓浓的硝烟笼罩了数丈方圆,过了很长一段时间才被风给吹散。炮弹的呼啸声难以形容,仿佛天空中有龙在吼叫。落地的瞬间,就如陨石飞降,正好砸在作为靶子修起来的一道城墙上。两丈多宽的墙体爆出了巨大的裂口,从城墙三分之一的位置上开始,砌在墙面上的砖石哗啦啦的垮塌了下来。

“许卿家,你看这火炮如何?”耶律乙辛得意的回头,问着身后队伍最末尾的一名汉官。

这是从河北跑来的儒生,据说还很有些名望。耶律乙辛看在他一路辛苦的份上,给了他一个国子博士的官职。

在过去,宋辽双方有互遣逃人的约定,可是现在宋辽两国之间已经断绝了外交关系,只有商贸往来。过去的约定也没人再去理会。两三年间,跑到辽国来的儒生、工匠、乃至罪犯,超过了两千人。

耶律乙辛一口气接纳这么多南朝儒生,一个是让开封的南朝朝廷不痛快,另一个,也希望其中能发掘出几个贤能。不要韩冈这个等级,有张元吴昊的水平就可以了。尽管之后让他大失所望,但他还是安排了绝大多数儒生去教书,同时给其中名望最高的几人以官职。

在耶律乙辛想来,这应该是最合适的安排了。去教书的措大不提,那几个被封了官的应该感恩戴德才对,至少能像张孝杰一样,说话让人听着舒服。

但他今天失望了,那位许博士扬起脖子,大声质问:“敢问陛下,造此物者,国人欤?汉人欤?”

耶律乙辛被泼了冷水,脸色就阴沉下来,“卿家何意?”

看到耶律乙辛的脸色,张孝杰猛地一个寒颤。作为跟了耶律乙辛几十年的天子近臣,一看就知道,天子已经开始发怒了。不知道那措大是不是在玩欲擒故纵的伎俩,之后一个转折,让耶律乙辛心情好起来。

“既然此间汉人能造如此巨炮,难道南朝的汉人就造不了?”许博士却继续出人意料的抗声道,“南朝势大,陛下勤修武备不为错。但家国之固,在德不在险。兵多将广、甲坚兵利不足为凭。四民安定,百姓服膺方是治国之本。”

‘这措大,是在南面给惯坏了吧?这里乱说话是要人命的。’

张孝杰正想着,就看见耶律乙辛已经铁青着脸拂袖而去。

“把他抓起来!”

张孝杰刚开口,担任宿卫的完颜阿骨打便一把揪住这个新封的博士,手臂向上一举,将其用力掼在了地上。

咕咚一声闷响,张孝杰听着就觉得疼,头颅着地的许博士就这么昏迷了过去。

“读书都读傻了。”完颜阿骨打狠命的又踹了一脚,哎啊一声,人竟又给踹醒了过来。

“陛下,怎么处置他?”张孝杰赶上去,问着耶律乙辛。

耶律乙辛低头抚摸着火炮炮尾处的铁箍,“既然人说不服他,就让火炮去说服他好了。”

张孝杰愣了一下,忙点头,“臣明白。”

群臣汇集在耶律乙辛身后,大辽天子摩挲着火炮还没经过仔细打磨的粗糙外壁,缓缓说着,“朕的大辽,不需要腐儒,不需要读经读傻了的蠢货。只要有心灵手脚的工匠,善于种植的农人,懂得律法的官吏,勇猛敢战的将士,大辽将无所畏惧!知道为什么过去汉人打不过大辽?”他指着正在眼前被拖走的许博士,“都是这些东西害的!”

“陛下!别忘了南朝也出了一个韩冈!”

“我知道你们想说什么。”耶律乙辛回头冷笑,“韩冈是当世大儒吧!……狗屁大儒!韩冈和孔夫子从来不是一路人。鸠占鹊巢的把戏让他在南朝玩吧,我们就别上当了!”

耶律乙辛自觉说了一句很有趣的笑话,沙哑的笑了两声。

待群臣脸上堆起附和的笑容,他又将脸一沉,“大辽立国不是靠措大,治国也不是靠措大,如今开创更不需要靠措大。靠的是甲坚兵利,靠的是铁骑纵横,靠的是火炮凶猛。若有谁觉得朕错了,让他来见朕,朕会让他明白的!”

群臣悚然恭立。

耶律乙辛横扫一眼,又收回到心腹重臣身上,“张孝杰,你说这门炮叫什么名字好?”

张孝杰脑筋急转,“宋人放在皇城中的几门重炮皆以将军为名,其实加起来也不如此炮。上下四方、古往今来,能与此炮相媲美的神兵利器一个也无,依臣一点愚见,不如名为宇宙大将军。”

耶律乙辛现在对给武器起名,丝毫不感兴趣。叫狗屎也好,叫皇帝也好,其实都是这门炮。但一个好名字,肯定能激发起工匠的忠诚心,这是耶律乙辛所看重的。

“好!”他轻轻拍手,“就叫宇宙大将军!”

…………………………

太后的鸾驾,已经离开了金明池,回到了皇城中。

一年一度的龙舟竞标也决出了胜负。

同一天,在同一个地点,争夺锦标的两支冠军球队之间的那一场激烈较量,在京城的人们口中,已渐渐不再提起。

辽国实验新型重炮宇宙大将军的消息还没有传到开封,但早在一个月前,辽人的十五寸口径的超重型攻城炮制造成功的紧急军情,已经送抵到了两府每一位宰执的案头。

不过韩冈现在还不知道,辽国的那位伪帝将自己的老底给揭了开来——尽管耶律乙辛的本意,也许只是为了给韩冈添点乱,将大宋内部已经与政治密不可分的道统之争,搅合的更加混乱一点。

一封来自于江宁的急件,在这时候送抵韩冈夫妻面前。

韩冈看信之后,默然不语,而王旖更是立刻红了眼眶。与妻子商量了一下,韩冈找来了自己的嫡长子韩钲。

年满十五岁的少年身长玉立,相貌上更多的偏向于来自江南水乡的母亲的柔和。

已经是个大小伙子了。

韩冈欣慰的看着自己的次子,但他跟所有的父亲一样,并不会将自己的心情说出来,而是直截了当的命令道,“替你娘去一趟江宁。”

韩钲愣了一下,他过了夏天就要开始在横渠书院读书了,而这个夏天,他还要先去巩州一趟,拜见祖父母。去江宁探望外公外婆不在日程之中。

“是外公外婆身体有恙?”韩钲睁大眼睛问道。

的确是个机灵的孩子。

“是你外公,身体不太好。”韩冈看看哭过之后就沉默下去的王旖,“你娘一时不方便,得收拾两天再动身,你先坐车去江宁。”

韩钲明白,他的父亲是宰相,不可能去探望王安石,母亲也是宰相夫人,出外远行其实并不方便,而且说不定还要带着弟弟们,肯定是要准备一两天的。

“孩儿知道了。”韩钲一口答应下来。

外公的病是一桩,父母的吩咐是一桩,让他这么干脆的答应下来。但能够独自远行,去期盼已久的江南,也更让他期盼。

韩冈点点头,正要再说话,一名仆人通传进厅,“相公,通进银台司来人了,说是有紧急要事要通知相公。”

韩冈的话声停了一下,然后就扭头对王旖道:“朝廷现在也收到江宁那边的消息了。”

韩冈确信,发给自己的急件和江宁发给朝廷的急报肯定是走了同一班车。

卸任宰辅的身体健康,一向为朝廷所关注。王安石突然病倒,江宁知府若不能在第一时间报上,事后朝廷不会饶他,太后不会饶他,韩冈更不会饶他。

“去问一下是什么事,留下他的名字。”

韩冈自不会去见通进银台司的小吏,吩咐了仆人去确认,回头对妻儿说道,“南下的班车,今天最晚一班是晚上九点半发车。如果现在就收拾的话,可以赶得上。三十八个小时后将会抵达泗州。”

“三十八个小时,也就是后天中午能到泗州?”王旖终于有了一点反应,“那要在车上睡两个晚上了?”

“只用睡两个晚上。”韩冈觉得相对于汴河航运的速度,只用一天半就抵达终点站已经够快了。

这也是多亏了时钟普遍运用在生产生活中的好处,

在时钟被发明之后,就有了准确的列车运行时刻表,整条铁路的运行效率一下高了一倍。

在每一辆有轨马车中,最前和最后的车厢都装有一架座钟,可以让车长掌握好时间。经过每一座车站,都有固定的时间,只要控制好每一辆列车过站的时间,就能防止车辆在铁路上前后相撞。加上道岔和车站编组的运用,也使得重载的货车不会影响到客车的行驶。

所以也就有了一天半由京师抵达泗州,二十四小时,从京城直抵洛阳的高速。

韩冈前世从小说和历史书中知道,另一个世界里,几十年后南方有明教为乱,不过几个月内便为童贯所平灭。如今有了铁路,江南有变,五天之内,两万禁军就能抵达泗州,十日内全数过江。不管有什么叛乱,想要在大军杀来之前发展壮大,那完全是天方夜谭。这是技术进步最直接的好处。

“到了泗州,你再转乘车船南下金陵。”韩冈继续吩咐着。

一旦有了蒸汽船,再将铁路从泗州铺到长江边,从京师抵达江南的时间,还能再缩减三成。只可惜现在还没有。但韩冈安排给自己儿子乘坐的车船,速度已经很快了。

“官人,要不让二哥也跟奴家一起走。”王旖之前沉默了许久,这时突然说道。

想来想去,她还是不放心自己的儿子就这么独自出行。

“娘,孩儿可以!”韩钲立刻叫道。

“放心。二哥会带着贴身的伴当……石晟稳重点,就带他去。再寻一个去过江宁的老人,加上两个护卫,就足够了。一路官车、官船,还怕什么?”韩冈安抚下妻子,又撵着儿子回院子去,“快回去收拾东西,八点前就要出门。”
