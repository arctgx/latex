\section{第39章 太一宫深斜阳落(四)}

【第三更,红票,收藏。】

“韩官人果然大才!”路明读了两遍,便凑上来赞着,“实是难得一见的佳作。”

韩冈苦笑摇头,他眼不瞎,又老于人情世故,看得出路明的称赞言不由衷。的确,被篡改后的诗句,连韩冈自己读起来都感觉别扭,总觉得哪里出了问题,读得一点都不顺畅。

而与周围的和诗比起来,韩冈写下的这一首,如果不去考虑平仄,勉强算得上是可以入眼,但绝不算出奇。比起原诗号称一曲压故元百年的高度,可以说是生生被糟蹋了。

韩冈看了半天,叹了口气,终于看出了问题所在。他为了和着王安石两首六言诗的格律,将原作删了一句,却把一篇千古名词给毁掉了。马致远的原诗一唱三叹,动人心魄,韵味悠长。但韩冈删去了一句后,却让这首小令的节奏感乱了套。

王安石的‘三十六陂春水’一句吟来,语调宛转,韵味十足,而且说的是一个景色,带起最后一句‘白首想见江南’正为合适。而‘古道西风瘦马’,一句咏三物,跳跃感太强,后面又紧跟着‘断肠人在天涯’,少了一点缓冲,读起来当然不顺畅。要想改正,中间便必须再铺垫上一句。

韩冈摇头自嘲:‘终究不是写诗的材料。’

煅词炼句果然是大学问,难怪贾岛在推敲之间踌躇许久,也难怪欧阳修最近给韩琦写的《昼锦堂记》订最后一遍修改,只是在前两句中各添了一个‘而’字——将‘仕宦至将相,富贵归故乡’改成了‘仕宦而至将相,富贵而归故乡’,一字之别,宰相的雍容气度便在两句中透了出来。

沾了沾墨水,再度提起笔,韩冈在第三句后面又一气添了四字,退到路明身边,直笑道:“如此方好……”

“夕阳西下?”路明喃喃念着。

韩冈转头笑道:“本是想写在长安道上得遇明德兄之事,但在下诗才不足,不妄添四字便读不顺口。只是就不是六言了,世间也没这格律。”

路明却只听到前一句,对韩冈后面几句已经听不见了,他读着,看着,身子颤得厉害,难道这首诗里写的是他?!

“断肠人在天涯……断肠人在天涯……”路明一遍又一遍地念着,泪流满面,如陷疯魔。四十年读书,三十载试举,到头来一切辛苦却都是一场空。他每每在人前自吹自擂,但实际上是什么样的情况,他自个儿如何不明白。

“不考了……”路明低低一声叹,忽地又爆发般的吼出来,“不考了!”

“不考了?”韩冈楞住了,一时没反应过来。

“还考什么?!再去丢人现眼不成?”路明一副大解脱的笑容,“以官人之才,尚且不敢去考进士,路明才气不及官人万一,却还抱着奢望,考过一次两次还不够,一直考了三十年。梦也该醒了,梦也该醒了啊!”

他对韩冈一揖到地,“多谢官人当头棒喝,助路明得脱噩梦。”

古有观棋明理,有临水悟道,想不到今日得见读诗觉醒。路明为科举沉迷了几十年,竟然被一首诗点醒。韩冈一时间也不知该说什么,难道要说‘浪子回头,善哉善哉’吗?

路明直起腰,也不多说,返身便往外走,原本有点猥琐的身影,现在看来却变得高大了许多。

韩冈回头看了看墙上的原版《天净沙》,照规矩是要题款的,但他拿起笔,想了一想之后,却又摇了摇头将笔放了下来。

还是算了!不是自己的,就不是自己的。他自从来到这个时代,挣扎,争斗,最后挣到一个官身,一切靠的都是自己的本事。自家毫无诗才,靠着剽窃得来的名声却也没什么意义,还要为此提心吊胆,防着被人戳穿——这又是何必?

此诗是好,于己却是多余。

韩冈转过身,也大步走出了殿中,并不回顾。

片刻之后,一群人从旁门涌进偏殿。

大嗓门发出的声音在殿中回响:“蔡元长,你都到了西太一宫了,王大参的两首六言竟然没看?!”

“不是急着进来吗?”蔡京为自己辩解,“何况早记熟了。”

“如此佳作,如何不亲眼看一看正品?!”大嗓门带着人,在殿中一绕,便站在了韩冈方才站着的位置,“喏,就在这里!……咦,谁把纱帐拿下来了?”

“大概是方才在殿里的两人。”蔡京说着,方才擦肩而过的高大少年,给他的印象挺深。尤其是一对有些锋锐的眉眼,犀利得仿佛能看透人心,不似二十上下的年轻人应该拥有。

“好像留了和诗啊。”赵子正举着墨迹未干的毛笔,敲了敲还留着残墨的砚台。‘浪费笔墨!’他暗自摇头。王安石两首六言的和诗不少,但无一条能入人眼。说起来自家也是想和上两首,可用了一个晚上,一句合眼当都没憋出。王珪的富贵诗好学,顺耳的金玉之词往上堆就是了,图个亮眼顺耳。但王介甫的诗作,却是平淡中见真趣,没几十年的积累,怎么也学不来的。

“在这里!”大嗓门指着韩冈留下的手迹,几行字墨迹淋漓,显然是刚写出不久,他看过去,只看了两眼便大惊叫起,“……这是谁人所写?!!”

强抒仲也一把扯住蔡京的袖子,“元长,你看到是谁人写的?!”

蔡京也被这首新诗惊住,正默默念着,便被扯住袖口,他很不耐烦的甩开,“强抒仲,别闹!”

上官彦衡则高声读了出来:“枯藤老树昏鸦,小桥流水人家,古道西风瘦马,夕阳西下,断肠人在天涯。”读完,他啧啧嘴,像是在赞叹,却又摇起头,“不是诗,是曲子词,只是这个格律的小令从来没听过啊……”

“这‘夕阳西下’是后添的。”蔡京指着韩冈后添的一句,从墙上诗文的排列结构上,很容易就能看得出来。

“画龙点睛不外如是。”强抒仲感叹着,“四字一加。韵味悠长,就像是腌渍过的橄榄,越嚼越有味道。”

“神来之笔!神来之笔!”大嗓门对着‘夕阳西下’这四个字赞不绝口,“这四字是天外飞来,无可挑剔!”

“这究竟是谁人之作!?”一众士子大声叫道。此诗没有题名书款,但水平摆在这里,在场的一众士子,都是今科的贡生。蔡京蔡元长,大嗓门的赵挺之赵正夫,还有上官均上官彦衡,以及强浚明强抒仲和强渊明强隐季两兄弟,皆是一时俊才,自负才高之辈。在如今东京城中的数千举人中,多少有些名气。对他人来说,进士一第难如登天,而在他们几个看来,却如探囊取物一般。但他们现在看了这墙上新添的不合格律的新曲小令,却无不惊叹,自愧不如。

“是不是就是方才元长看到的两人?他们应该刚出去吧?”强渊明自己说着便冲出殿,左右看看,除了一个拿着扫帚的火工道人,并没有第二人,才转回过来问着蔡京道:“蔡元长!你不是看到了人吗?究竟是什么模样?”

“也不一定是他们!”蔡京摇头。他总觉得擦肩而过的两人都不是能写出这首小令的形象,一个太年轻,一个太猥琐,皆是不像。他去找来了在殿外庭院扫地的火工道人,还有宫里的庙祝,问道:“方才这偏殿有几人出来过?”

火工道人和庙祝对视了一眼,便拱手回道:“回秀才的话,就只有两个。”

蔡京愣了一下,难道猜错了,他确认着:“是不是一个二十上下的高个子,还有一个五十左右、面白无须的老儒士?”

“对!对!就是他们!”火工道人忙点头叫道,“今天午后,除了几位秀才外,就只有他们两个客人。”

‘两个人?究竟哪个写的?’赵挺之皱眉想着。他心中有些不痛快,如此绝品,放在王安石的两首六言旁边都不遑多让,怎么能不书款呢?若是自家写出来的,肯定会夹在名帖里到处递人啊,凭着这一首,宰相府都是能进的。

“究竟是他们中的哪一个?”强浚明问出了口。

“还用问吗?!”蔡京声音大得惊人,“‘断肠人在天涯!’刚成冠礼的后生晚辈写得出来吗?!”

众人一起摇头,这当然不可能!这首小令词义浅显,而蕴意颇深,不是久历江湖,身心疲惫的垂垂老者,怎么可能写得出如此文字?!

“他们可说是哪里人?”上官均问着火工道人。

火工道人摇头表示不知,而庙祝道:“方才听声音像是关西那边的。”

蔡京眯起眼推测着,他很喜欢这样动脑筋的活动:“五十上下,又是陕西口音……不是特奏名,便是免解贡生。这样的人不难找,每科加起来也就百来个。等考完一问便知。”

赵挺之、上官均、强氏兄弟和其他几人听后都是沉吟思忖了一下,很快便一齐点头,“元长说得正有道理!到了开考后,定然能知晓。”

蔡京回头又看了一眼墙上的诗句,笑道:“不过此等佳句,不须等到开考,怕是三五日内便能遍传东京。到时候,王大参说不定也要找他呢。”

【俺刻意写这一章的用意应该不难猜吧?】

