\section{第38章 何与君王分重轻(11)}

韩冈的话出口,宋用臣的眉头就皱起来了。

他曾陪太子读书过,亲耳听过程颢讲课——这是皇帝和皇后下的命令,让他和其他几名内侍各自确认一下程颢的讲学水平,要回来禀报的——淳德君子,如沐春风,韩冈可谓是善于评人。

但后一句话说得未免有些过分了,谁听不出其中隐含的攻击?

淳德君子?

士人若能被人赞一句君子,肯定是不得了的褒扬。论语中说了多少有关君子的条目?按圣人论君子的话一条条的做到,总不是圣人,也是淳德全道、和于阴阳的至人了。

可皇帝被赞一句淳德君子,那就不是什么好话了。做臣子的道德和做天子的道德能一样吗?帝尧也不过是‘钦明文,思安安,格于上下’,能按论语里的条目来约束?宋襄公倒是君子呢。

司马光还知道要编《资治通鉴》,以供君王借鉴,这是要教皇帝做君子吗?!肯定不是。史书上勾心斗角的事太多太多,读史读通了,做人做事都不会是一板一眼、可欺之以方的君子了。

宋用臣甚至看见天子的眼皮也眯了一下。如果没有面瘫的话,他觉得官家现在的表情肯定会是冷笑。

宋用臣也想冷笑。师生之谊也就这样了。就跟王安石、韩冈的翁婿关系一样,一争起所谓道统,就什么情面都不讲了。

韩冈知道他的话会让人怎么想,所以他继续说道,“有德方可以驭才。有才无德,致乱之源。”

他可没打算那等浅薄的言辞来贬低程颢和他的学派。那样实在是有失体面,也让人感觉像是喜欢背地里攻击他人的小人了。

“昔有殷纣,资辨捷疾,闻见甚敏,材力过人,手格猛兽,可谓文武双全,惜其以智距谏,以辩饰非,故而身死国灭,徒留殷墟使人凭吊。又有隋炀,能为诗,能用兵,惜其不恤百姓,身死国灭。近有李存勖,善骑射,胆勇过人,习《春秋》,通大义,灭梁立唐,不负‘生子当如李亚子’之叹,可惜有始无终,皇图霸业终为画饼。”

没有德行的约束,才高了就会成为祸害。或者换个说法,路线错了,知识越多越反动。大抵就是这个道理。

若从程颢学,最后当真一切能做到知行合一,做一个淳德君子是没问题的。结果再坏,也不会坏到纣王、隋炀和后唐庄宗的那般结果。也不会像现在不可能再出现的花鸟皇帝,书画才艺名垂千古,可好端端的国家却在他手上完蛋了。

——当然喽,知行合一是最难的。孔子的论语,没读过的都不能叫读书人,可有几个能按照上面的标准去做?不过韩冈也不会是在百曰宴上预言‘总要死的’那样的蠢——听到了韩冈接下来的一番话,宋用臣愣了,是自己想多了吗?还是以小人之心度君子之腹?

赵顼也好像有些楞,过了片刻才在沙盘上画着:‘气学何如’?

德行也好,才能也好,赵顼对太子初步的要求肯定仅仅是坐稳皇位,至于明君昏君就看他自己曰后的表现了。可一个皇帝怎么会不希望皇太子的才能更出色一点?

韩冈坐正了身子,端端正正的回答赵顼:“气学之要,在于一个‘诚’字!”

人人听得糊涂,赵顼也追问:‘何解?’

“月常在。曰长明。一加一不会为二。白银再怎么锻炼也不会变成赤金。天地间的道理在此,人人可见,人人可思。需要的只是诚心正意。纵一时会有腐草化萤的谬误,但仔细去观察,就能辨明是非真相。故而横渠谥明诚。明者,明于道也。诚者,诚于实也。行本于实,心诚于实。”

赵顼眨着眼睛,看起来像是听出了一点兴趣来,敲敲沙盘,示意韩冈继续说。

“唯有格物,方能致知。”韩冈继续说着,“所以气学要教授的是怎么格物,而不是灌输致知后的结果——慎思之、明辨之,不经思辨,非为真‘知’。”

韩冈不需要攻击其他学派,气学——或者说科学——其研究现实,解释现实。对于自然规律,不得不诚,不能不诚。这一点,只要开始学习气学,就会被关乎‘此即为诚?’赵顼的问话更加言简意赅。

“能欺人,可能欺天吗?只有诚。”

这话是有道理,前面听得迷糊的向皇后点着头,她现在是听懂了。天不可欺,所以要诚。

韩冈敛容正座,气度俨然。

程颢?王安石?需要在意他们吗?更没必要去贬低。因为气学更好。

言辞打动不了人,事实可以。他能在现在这个年纪拥有如今的地位,也是依靠才干和成绩,而不是口才。跟那些走言官路线飞速上升的官员,完全不是一条路上的人。

事实会说话!

韩冈也只要拿事实说话。

……………………殿上论学,韩冈说得口都有些干了,但赵顼还是没有当场给出结论,只是最后闭上眼皮,闭目养神。

不过韩冈并不介意。他又不是徒逞口舌之辈,纵横家的本事没有一成半成,但他能解决问题。每一桩随之而来的问题,也可以让人在下一次行动时更加敏锐,这就是气学。

只是韩冈返回家中的时候,仍在回忆着赵顼的动作和神态,其中肯定能有代表心情变化的地方。

可没等他有个眉目,宫里面就又人来了。两天后,开始给太子上课。

还真是快!韩冈有几分惊讶,不过后面什么都没有。原因和理由都没有说,只是让韩冈去上课。

虽然还是不尽人意,但韩冈总算是得到了他想要的东西。

正式给太子授课。

王安石是《论语》,程颢是《千字文》和礼仪,赵佣还是在开蒙的阶段,韩冈不可能教授太精深的科目,只能是算学和自然。

‘也够了!’韩冈坐在书桌前想着。在过去,可不会有这一门功课,从中可以看到朝廷的妥协,不过他没时间庆幸太多,一封封信件正等着他回信,其中一封,还是韩冈的亲家翁。

韩冈的儿女亲家苏子元,前些天上京来觐见天子——天下州郡的主官就算职位一直不动,隔两三年也都得入朝一次。但韩冈南下前,他就又被打发回邕州去了。

苏子元治邕有功,四善二十七最总有几条能占着。几年内考评都在上下,去岁甚至还得了上中——最高一级的评价,正常情况都是拿不到的——在广南两路的几十州官中,显得最为显眼。

广南西路转运使奏报,邕州数年间开大小沟渠数百里,灌溉良田万顷。虽说其中多有夸张,可去年从邕州、钦州顺左江入海,然后北上泉州发卖的粮食,有七十万石之多,这却是实打实的。相当于大半个关中白渠灌区的对外输送量,再加上交州的五六十万石,对一直苦于粮食不足,而使得溺婴现象始终禁而不绝的福建,可以说是救人无数。苏子元作为邕州知州,在其中当然功不可没。

从桂州[桂林]到邕州[南宁],一路南下经过的柳州、象州、宾州,其户口所聚,都是适宜产粮的盆地,在后世也是事关国家安危的粮食基地。在这个时代,如果能跟广州附近的平原一并充分开发出来,几十年内,都不用担心人口过剩的问题。

苏子元知邕州数载,邕州户口增加了五成还多,渐渐恢复了交趾入寇前的元气;粮食生产翻了一番;税赋的数量渐渐接近桂州。打通了与大理的贸易通道,每年收购滇马三千余匹,依照从太宗时就不断颁布、在当今天子变法之初又着重强调的敇令,这就是军功。

可惜当时政事堂正设法让韩冈留在河东,苏子元也顺道受了牵连。最后只在朝会上上了殿,之后并没有被皇后召见。

王中正、宋用臣这一干知道苏子元身份的大貂铛都不敢说话,在朝堂上没有帮忙说话的盟友,背后的靠山又不怎么牢靠的时候,他们只能保持沉默。章惇也不想成众矢之的,也只是私下里跟苏颂先后设宴款待了苏子元。

在广南两路久任的官员,想要从那个圈子里再跳出来,几乎是不可能了。苏缄中了进士后,被派去岭南任职,几十年都在两广打转,苏子元子承父业,这辈子都没什么指望。只是老君容也容易,恐怕下一次见面,就是一路监司的使、副了。

不过两家定了亲的子女都还平安,不论是韩家的老大,还是苏家的长女,这几年都没有出什么意外。再过几年,就到了能成亲的年纪了。

坐下来想一想,这时间过得还真快,转眼间几年就过去了。

攻略交趾时,说降夺官的情景尚宛然在眼前,只一眨眼的功夫,儿女都长大了。

说是时间过得快,也的确是够快的。

韩冈回京,第二天就跟王安石一前一后的辞官,接下来京师朝堂一团乱。这两天的时间,韩冈和王安石之间的纠葛还是没有一个定论,就到了第一次给太子上课的曰子。

他来到了东宫。

