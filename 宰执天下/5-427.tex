\section{第36章 沧浪歌罢濯尘缨(29)}

“户绝田啊……”

代州、忻州不知有多少户人家死绝了,房屋被烧,家财被夺,但田地可是烧不掉夺不走,都变成了无主的户绝田了。

依宋律,户绝田要收入官府,成为官田。但同村的邻居,只要还活着,完全可以趁机侵占甚至吞没这些土地。胆小的动一动界碑,胆大的直接把界碑拔了。

只要事后能打点好县中下去计点户口、土地的胥吏,就能安安心心的将田地侵占下来。如果还想要稳妥一点,再去伪造一张田契也就够了。

田契分为白契和红契两种。红契是在官府备案的,交过了契税,盖了鲜红的印章。白契则就没有备案,只有买卖双方和中人、保人。这两种买卖契约,在断案时都可以作为证据,不过红契和白契相冲时,还是以在官府中有存档的红契为准。只是如今的代州官衙,户籍也好,田契也好,都烧了干净。掏出一张白契来,就能证明田地的归属了。再交点钱,还能编进新订的官衙籍簿中。

黄裳自是知道现在代州乡里的情况,“那诚伯你打算怎么办?”

“什么都不办。当务之急是把田开垦起来,粮食种出来。只要能开辟出来,就是没田契也好说。”田腴苦笑着,现阶段,孰重孰轻必须要分清。他当然也想去整治一下那一干奸猾之辈,可雁门县现在最重要的就是尽快恢复生产,不再依靠朝廷的救济来维系百姓的生活。

“章质夫也这么想?”黄裳问道。

“我只要考虑雁门一县就够了。但章府君还要想着繁峙、五台和崞县。”田腴慢慢的摇头,他和黄裳都是韩冈门下士,但章楶不是,有一个知枢密院事的族弟,行事无须依从韩冈,“知繁峙县是陈丰,他还好说。但五台和崞县,枢密并没有推荐,新上任的知县会怎么想怎么做,章质夫免不了会有些顾虑。”

“……枢密若能回京中,与章枢密在朝堂上联手起来,想必章质夫就能放心去做了。”

当年广西邕州被屠之后,韩冈立刻组织了大量人力开辟渠道,对邕州的田地进行集体耕种,而无视原来田主的所有权。很多避难回来的大姓、富户,都对此颇有微词。甚至有人上告到开封,也幸好当时朝廷对平定交趾极为迫切,没有追究韩冈的责任。

而现在的情况,和议已定,辽军已退,就有了内斗的余暇。不说别的,京城中很多人正想找韩冈的把柄。纵然韩冈本身无懈可击,只要将韩冈身边的人放倒几个,他也肯定要受到牵累。章楶私心里肯定是不愿意为韩冈冒风险,不比黄裳和田腴,甘愿为韩冈冲锋陷阵。

“朝廷……”田腴摇了摇头。两府中那几位怎么可能让韩冈和吕惠卿回去。

韩冈、吕惠卿二人携临危救难和开疆拓土之功返回朝中,立刻就能聚拢起一大批官员投效,不费吹灰之力就能从在京的宰辅们手中夺下一大块实地来。可只要能拦住两人几个月,让其高涨的声望渐渐回落,让皇后、群臣和百姓的兴奋重新沉淀,想要投奔两人的官员就会少上许多。

而且两人既然不受已经在京中多日的同僚们的欢迎,那么下面的官员们想要投效就必须要冒开罪一位平章、两位宰相和数位执政的风险——而趋吉避凶的智慧,官员们不缺少。而雪中送炭虽好,但万一还没有等到收获的一天,便引火烧身可就不妙了。

在两府中争权夺利的背景下,韩冈的药王弟子光环现如今也发挥不了作用。既然他在外数月,皇太子都平安无事,那么再拖上两三个月也不会有太大的关系。

黄裳哼了一声,不屑之意溢于言表:“朝廷怎么想的不用管,反正枢密的准备快差不多了。”

“京营真的能成事?”

“既然诚伯你的职位都已经定下了,那么京营禁军的‘功劳’也肯定有了赏赐,朝廷岂会拖延?”

黄裳在功劳二字上加了重音。河东战事中,韩冈把京营禁军的作用发挥到了最大,但如果他们能有河东军一半的战斗力,早在太谷县,置制使司就能战役的目标改成全歼敌军,而不是退敌了。

“他们真有闹的胆子?”田腴仍有疑虑,“听说当年仁宗皇帝大行,英宗即位,京营曾以赏赐不足闹了起来,不是给殿帅李璋一句话就给骂回去了吗。”

这桩公案传得很广,往往士人评论军伍的时候,都会拿来做例子。

“那是他们没有上过战场,立过功劳。上过战场之后,自以为了不起的可是多得很。”

“……的确。”田腴点了点头。确不是一回事。同样赏赐微薄,有功和无功,闹起来的底气和声势都不一样。他又叹了一声:“朝廷诸公私心太重啊,枢密常说礼尚往来,如此行事也是不得已而为。”

“不过这都是我们在胡猜啊。”黄裳又道,“枢密到底是怎么想的,谁知道。”

田腴笑了一下,不置可否。这些全都是他们私下里的猜测。纵然一目了然,韩冈也绝不会向任何人承认他的私心。不过总有蛛丝马迹能看得出来。

身为韩冈身边的亲信,两人皆知韩冈本来准备在河东就开始清理军中空饷,可当他开始着手去做,并写信想征得王安石的支持的时候,却发现他的岳父有意让他留在河东。韩冈的想法当即就变了。

他本以为可以得到王安石的支持,可是现在没有足够的支持,反而会被同僚落井下石,这样的局面下韩冈可不会往火堆里伸手。不劳幕僚们苦劝,韩冈自己就很干脆的放弃了,战事一结束直接就把京营都打发回京。

但韩冈究竟有没有熄了之前的心思,那就谁都弄不清了。而这样情况下打发回去的京营禁军,究竟会给朝廷带来什么麻烦,也很容易看得清楚。

斩首、俘获还有经历过的战斗,韩冈在奏章中一点没有克扣,甚至还把功劳簿公开给了所有的将领观看,让他们自己来确认。最后还当面封存送去了京城,以示其公。

韩冈都做到了这一步,最后怎么封赏那就是朝廷的问题了。

“不过也有可能,枢密另有方略。以枢密的性格,不会将赌注压在一门上。”

现如今,朝堂中的紧要差遣,全都给人占了去,都没留给吕惠卿和韩冈一星半点。

按情理理说,如今就让吕惠卿及韩冈两人回京,他们一时之间也争不过根基牢固的其余宰辅。孓然一身的进了两府,只有被架空的命,存在感只在画押、盖章上。

可是韩冈和吕惠卿都不是没有基础的人,在朝中有门人、有奥援,本身又有年龄和功绩上的优势,不愁没人投效。

这两条强龙回朝,肯定是要抢班夺权的。这当然会引起已经大权在握的宰辅们的忌惮。且韩冈相对于吕惠卿,身上还多了一重公案,道统之争让王安石都不想他回京太早。

纵然皇后希望韩冈能早日回京,但只要宰辅们那边不同意,皇后一人是拧不过他们。因而直到六月艳阳高照,韩冈依然逗留在代州,不尴不尬的做着他的置制使。

换做是别人,这时候肯定是急得心中如火烧。可韩冈都是气定神闲,好像是一点也不担心回不去。

“枢密若是没有把握,今天就不会这般悠哉悠哉的去吃冷淘了。”

田腴的话有点盲目,但黄裳却觉得他并没有说错。

纵使亲近如他们这些幕僚,也没人能看得透韩冈韩冈。比如他的学问,比如他的见识,都很让人费解。世间都说是天授,但韩冈却总是振振有词的解释为格物而来。

这真是个好理由。

比起攻读经史,格物致知其实更需要时间去积累。黄裳喜欢兵法,对山川地理下过很多心思。真正要精研地理,就不能坐在家中翻书堆,而是必须脚踏实地的去各地探查。这也可以算是格物。其所用时间之多,远远超出在家中读书的消耗。

无论是天文地理,还是自然万物,都是需要消耗大量时间来研究的科目。可到了韩冈这里,很多颠覆了常识的见闻、道理,似乎没用太多时间就给他格致得到。

《桂窗丛谈》就不说了,前些日子曾与韩冈闲聊,不知怎么就谈起了酿蜜。黄裳最多也只能分辨不同蜜源的特点,而韩冈就不同了。

他不能分辨槐花蜜和桂花蜜的区别,但他却能将酿蜜的手法说得头头是道,好像比蜂农都要精熟。比如那王浆,黄裳从来没听说过。还有任何一个蜂巢中角度一模一样的格子,听到韩冈说了,方才惊觉竟有此事。而蜂群中的后、王、兵、工之分,如同人间的国度,更是让人匪夷所思,却无从质疑。

‘可能真的是天授吧。’黄裳想着。

不是说韩冈的识见,而是他格物的能力。别人需要长年累月的观察、积累,而他或许只要一瞥就能看透。天地之事如此,那人事呢,或许也能一眼看破吧……否则也做不到不及而立便身登两府。

而现在的情况也让人不得不认为,他真正的手段还没用出来。

“诚伯。”黄裳突然问田腴,“枢密那一日在张孝杰当面说的一番话,究竟……是对谁说的?”

“……只有天知道了。”


