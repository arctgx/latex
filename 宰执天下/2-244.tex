\section{第35章 重峦千障望余雪(九)}

虽说韩冈在给王韶、张载等人的信件中,没少提到他在格物学的新奇见解。但这些论述,都是他闲暇时的调剂,以及对未来的铺垫。他现在所真正关注的,是即将到来的科举,和近在眼前的决战。

“你这草料场最是要当心,几十万束草都堆在这里,明年大战的消耗全都靠着此处支持。若是出了半点差错,不是简简单单能了事的。”

韩冈嘱咐着,虽然语气也算平和,也不屑用威胁的口气下去人,但管勾草料场事的小官却还是心惊胆颤的点头哈腰,连声应诺。一众在草料场中听命的士卒,也都是恭恭敬敬的跟在后面听着教训。

寻常守着草料场的基本上都是配军的罪囚,但为了防备意外,经略司调来了一队军中精锐来看守。明年上万匹军马要靠着这里的草料,的确是半点差错都出不得。

巡视过草料场,韩冈又去了囤积军粮的常平仓。这两处是城中防火的重中之重,不亲自走一趟,看过两处的防火准备,他怎么都不能放心回家过年。

不过不论是草料场还是常平仓,里面划分了片区,片区之间都有着足够宽阔的隔火带,除非有人故意纵火,或是刮着能掀开屋顶的狂风,否则即便起火,也不会烧光。

绕了一圈后,韩冈安下心来。离开了常平仓,管辖巡城甲骑的王惟新正好带队从门前经过,曾经是王韶身边的亲信元随,现在也是经略司中有名有姓的将校了。

看到韩冈,王惟新连忙下马行礼,两年前的韩秀才,如今身份早已不同。就算有着王韶做靠山,他也不敢有任何怠慢。

扶着王惟新起来,韩冈盯着他的双眼,郑重的说道:“今夜城中安危,可就要靠惟新你来担着了。”

陇西小城,不似东京、秦州将事情分得那么清楚,潜火铺的铺兵和巡城都是一拨人马,王惟新就兼管着城中烟火事。

去年陇西县还是古渭寨时,年节的那段时间,城中有过十几次大大小小的火灾。今年雪大,屋上、地面积雪未消,火势难起。可入冬以来,还是烧过了两三次,韩冈不想在除夕时听到火警的消息。是以他早定下了巡逻的班次,以防除夕夜中走水。

听着韩冈说着郑重,王惟新忙不迭连连点头:“机宜放心,惟新敢不用命?!”

韩冈把手放开,“你用心就好。”

王惟新在熙河众将佐中,能力、武艺都算不上出色,但胜在勤谨,这也是为什么他能带着巡城甲骑的缘故。可也就是因为能者多劳,勤者也一样多劳,摊到身上的职司让他连过年都过不好,

但勤快又肯做事的人,总是能比别人升得快。据韩冈所知,转过年来,王韶就要把王惟新换个更容易立功的地方了。

别过王惟新,韩冈又去了衙门中,即便是除夕,他还有一摊子事要处置,还有明天的正旦大礼,也要再看一看准备的情况。

等他将手上的事批阅完毕,又到大堂检查了各项礼器,离衙返家时,天色早已黑了下来。前面家里等着着急,派来询问何时回家的仆人来了一拨又一拨。两个仆人站在韩府门前,掂着脚向州衙过来的方向张望着。一看到韩冈带着他的一众亲兵元随回来,十几骑组成的一队人马蹄声清脆,便飞奔进院,去通知韩家的老官人和老太君。

韩千六和韩阿李都换了身新衣,就在堂屋中正坐着。一个穿着官服;一个靠着丈夫、儿子得了封诰、一身官人家主妇的品妆,看着就官宦人家的气派。

终于见着儿子回家,韩阿李火烧火燎的站了起来,急声道:“怎么忙到现在?!就等三哥你回来了。这身皮穿着就不舒服,快点去祭了祖宗,让娘把衣服给换了。”

“娘这话说的,多少人求都求不来呢……”韩冈笑着跨门进屋,顺手解开斗篷的绳扣,韩云娘忙上来把他脱下的斗篷给收拾起来。

“还有多少人不喜欢做官,不是说有个跟素心一个姓的学究吗,官家亲自找去,都不待搭理的。也难怪,这份罪受的……”

韩冈哭笑不得,严子陵的名头倒也真是响亮。只是韩阿李虽然着急,但韩冈要打的招呼,却还是要尽到礼节。

在正厅中,除了他的父母之外,亲戚中就只有冯从义在这里——李信和韩冈的舅舅现下都在秦州。

“今年还是一个人,等明年可就要两人一起来了。”韩冈跟起身来见礼的表弟开着玩笑,“到了后年可就要三个人了。”

“从义要多谢表哥主持。不然也娶不到太后家的女儿。”

冯从义今年年中订的亲,聘妻是高家旁系的庶出女儿。论起身份比冯从义要高上不少,但以冯从义如今的身家,找个县主结亲都是没问题的。就是如果与宗室联姻,必定会连累到韩冈。所以无论韩冈还是冯从义,都不会往这个方向去寻找。

“倒不关愚兄的事,是高公绰主动提起的。”韩冈转头对父母道,“表弟经商的手段,高副总管是赞不决口,说他是白圭、漪顿之才,能。”

“义哥儿做买卖的本事,不比三哥做官的能耐差。顺丰行的名字,现在哪家蕃人不知道?”韩千六没口子的赞着冯从义,“他今天带来的烟花,可都是京城里专做药发傀儡的李家出产,官家都赞过的。”

冯从义立刻谦虚道:“药发傀儡实在买不到,只能用烟花顶数了。”

他所主持的顺丰行,在韩冈的支持下,今年一年就带来了上万贯的净利润。所以今天来的时候,不仅仅带了各色礼物,还顺便带了一箱子从东京城中买来的上品烟花。

韩千六看着用金银彩纸包装起来的烟花,脸上直带着笑。若在往年,花上三五个大钱买两三个单响、双响的爆竹,听个响,也算是过年了,何曾敢奢望过用上开封李家的特制烟火——听都没听说过。可现如今,他韩家也成了富贵长享的官宦人家了。

韩阿李也一样心情愉快。周南、素心就坐在她后面,身上的衣物都是宽松的款式,如今两位孕妇被无微不至的保养的,到明年就能给韩家添个后代了。

韩千六放下了烟花,对韩冈道:“三哥,也别耽搁了,先去祠堂吧。”

韩冈先祖的灵位就放在后院西角的小祠堂中。韩家在关西的这一支,现在能上族谱的也就三人。而祠堂中的灵位,就只有少少的几个。韩家夫妇带着韩冈在祠堂中上香行礼,而其他人都站在外面候着。

对于自己的祖父,韩冈一点印象都没有,但能在这个时代远行千里,来关西开枝散叶,不管是什么原因,都是让人佩服的。而且若不是他的祖父离开了家乡密州胶西,如何能有他的出场机会。韩冈此时突然惊觉,自己在选人的阶段,几任本官都是在密州附近。难道是官诰院或是流内铨特意的不成?

把这桩巧合放在一边,韩冈叩拜起身。随着父母出了祠堂来。

正事结束,韩冈一家在正厅中坐下,一摊宴席都已经摆好了,接下来就是等着年节钟声。

压岁钱如今也有,只是韩家还没有孙子辈,也就当女儿养大的云娘拿到了一份。韩冈私下里也让严素心和周南给了招儿、墨文一份,三个小女孩子拿着压岁钱,都是小心的收了起来。

给家中仆婢的红包也发了下去,韩家如今收入丰厚,给仆婢的赏赐在陇西城中,算是很丰厚了。韩家的几十名下人,一个个上来叩谢,拿到沉甸甸的红包,各自喜笑颜开。

家中的宴席热热闹闹的进行着,韩父韩母听着周南、素心唱着小曲助兴,云娘带着墨文在后面服侍。韩冈则端着酒杯,拉着身边的冯从义又聊起了棉花的事。

“明年的棉花将会扩种。当初秦州有好些家商行都在等着成功的消息,你要好生的去联络他们。那些商行有的能从把黎人亲手织造的吉贝布运来秦凤,让他们的脚掺进来,至少可以把黎人所用的纺机给弄到手。”韩冈说得不厌其烦。

棉花要织成布料,织机可以借鉴丝绸织机的形制,但前一步的纺纱工序,却是还没有一个妥当的着落。韩冈听说过飞梭、珍妮纺纱机,也在某个纪念馆见识过‘自己动手丰衣足食’时代的土制纺车,可要让他无中生有,还是很有点难度。

“就怕他们太贪了。”冯从义犹疑着。

“让利是必然的。饼做大了,大家才好分;根系扎得越深,就越难让人撼动。”见着冯从义欲言又止,韩冈心知这两年顺丰行跟王韶、高遵裕两家的商行,一起垄断了陇西榷场,让他这个表弟变得有点贪心了。“你放心,只要我还在官场中,就没人敢吞掉顺丰行的这一份。”

韩冈都如此说了,冯从义哪还能再说什么,点着头记下了。

酒宴上的时间渐渐的过去,韩冈特意安排人手的新年钟声,当当当的开始敲响。悦耳悠扬的钟声响遍了城内城外,在夜风中传得很远。

门外噼里啪啦的鞭炮声一下猛烈起来。

韩冈和家人一起走到院中,来自京城李家的烟花在空中爆开,五彩的图案照亮了夜空。

硫磺味扑鼻而来,并不算呛人。烟雾弥漫中,第一百零八下钟声敲过,熙宁五年终于到了。

