\section{第29章 雏龙初成觅花信(下)}

章惇犹未就寝,在一个多时辰前收到了消息,之后,他便守在书房。

连换用的公服都准备好了,一等宫来人,换上衣服,立刻就能出发。

但章惇一直没有等来宫的天使,只等来了安排到御街查探消息的家人。

“韩三出宫了?!”

章惇一下站了起来。

一个时辰。

韩冈在宫待了一个时辰,就这么出来了。

“枢密?”报信人不解的望着开始在房踱步的章惇。

章惇挥了挥手,“没事,你先下去。”

韩冈大摇大摆的进宫去,等于是明确的通报其他宰辅。这种自撇清的做法,让人纵然心怀不满也无从抱怨。

可到底是什么事,竟然让韩冈连夜入宫,然后只留了一个时辰,就从宫出来?

应当是急事,却不是大事,而且……

章惇脑灵光闪过,一下站定。

是天有变!

若是军国之事,肯定跳不过两府宰执,韩冈绝不会自把自为。

而宫事,除了皇帝,太后,其他都不可能惊动到韩冈。

但如果是太后出事,韩冈必须召集诸臣,至少是他能信任的宰辅,才能对抗顺理成章接手内宫的皇太妃。

若是天出了大事,太后先招韩冈不足为奇,可韩冈当也不会瞒着其他宰臣,想来也只会是天有事,多半是生病,不过病情应该不重,所以韩冈入宫后就……

“枢密。”

书房的门再一次被人敲开。

章惇回头,“什么事?”

“韩相公派人来了。”

章惇扬了扬眉毛,韩冈还是这么会做人。

“让他进来。”

来人只有二十出头的样,长得精瘦,背挺腰直,拜礼起身,行动如风,一看就是个行事很利落的年轻人。

章惇都不禁有些羡慕韩冈。他在陇西打下好大一副场面,又收拢了多少离开军队的卒伍在门下。现在新一代长成了,就是标准的韩家家生,可供他挑选的余地,成千上万。

再没有比这样的人更忠心听话了,通过人牙雇佣来的仆役,还是亲友推荐来的家丁,都无法让人全心全意的信任。章惇还有一个大家族能撑腰,韩冈能有这么多可用之人,真的就是靠自己双手搏来的功劳了。

这个年轻人说话也是干脆利落,“皇帝小恙,太后心忧,故而招相公入宫。”

“天得了什么病?”章惇一下就抓到了关键点。

“隐疾。”年轻人简单的吐出了两个字。

章惇眨了几下眼睛,已经明白了过来。可是明白过来,心还是觉得匪夷所思,“天才十二岁。”

韩家家丁没有说话,笔挺的站着。章惇并不是向他寻求答案,只是在表示惊讶罢了。

“真真是想不到。”章惇摇摇头,长长的叹了一口气。又问那年轻人,“玉昆还说了什么?”

“相公说,天得病,是身边人推波助澜之过,所以太后已经定下,将其人都发配西北,至于已得陛下垂顾者,则另外安排。”

章惇闭目略一思忖,就抬眼问道:“没有了?”

年轻人道:“相公还说,有人等不及了。”

章惇眼神一下便变得犀利起来,半身前倾,沉声问道:“有人?!”

“‘有人’。”年轻人点头,“小人转达相公的话与枢密,不敢有一字更改。”

章惇脸上的神色瞬息而变,随即又对年轻人道,“替我谢谢你家相公,说章惇承情了,让你家相公放心。”

年轻人收到了章惇的回复,便告辞离开,当章惇的重新坐下,他书房的门,今夜第三次被敲响,“枢密,韩相公去了苏相公府上。”

这是要亲自与苏颂商议?

宰辅之间,不便随意串门,但若是奉了太后诏命,就另当别论了。

恐怕是太后让韩冈去知会苏颂,否则韩冈只会像对自己一样,直接派个可信的家丁去通报。

章惇坐上躺椅,右手轻拍大腿,极有节奏。

韩冈的一段话‘有人’二字用得最是出色,章惇现在倒是知道了,为什么天这么小就开始近女色了。果然是‘有人等不及了’。

……………………

苏颂书房的小桌上,摆着两杯清茶,侍候在书房的苏家仆役,此刻都被赶出了厅门外。

苏颂一身道袍,须发尽白,清癯的面容,削瘦高大的身,望之犹如神仙人。

稍稍抿了一口茶,苏颂他才问道,“玉昆,究竟出了什么事?”

韩冈毫不避讳,“方才被太后召入宫去了。”

“……是天?”苏颂停了一下,随即问道。这其的关联,他一眼就看出来了。

“是。”韩冈坦然相陈,“天为宫人所诱,伤了身,今天在福宁宫差点晕倒。韩冈不合虚名在外,方才便被太后传入宫。”

苏颂听着听着就笑了起来,

韩冈说得好像自己因为是医道泰斗的缘故才被传召入宫,实际上,当然不是这么一回事。

宰辅之,太后更加信任谁,不用想就知道。

不过韩冈话透露出来的消息,让苏颂很快便收敛了笑容。

“玉昆,照你所说。可是天近了女色?”

“正是,而且过了头,今天在福宁殿差点就晕倒。”

“天才十二岁啊。”

“是啊。”

“真亏他们敢做。”苏颂摇摇头。

韩冈道:“除了已得天临幸的几位宫人,其他皆配军安西。容兄觉得如何?”

“被临幸的宫人是怎么处置的?”苏颂问道。

“依仁宗尚、杨二美人旧例,出宫居养,说不准十月之后,就有个皇皇女出世。”

韩冈做得面面俱到,苏颂没有别的话要说,但一想起赵煦十二岁就近了女色,他就忍不住要叹气,“这真是……”

赵家的皇帝都有这个毛病。

从真宗开始,仁宗,熙宗身骨,几乎都是被女色给掏空的。英宗被他的那位皇后管得严点,可那也是原本身体就不好的缘故。

就是有太后,也不可能时时盯着,臣们更是在宫墙之外,谁能管得住皇帝?

不要说皇帝了,苏颂家里的几个儿孙,虽不至于十一二岁就开荤,却也没有到了十五还不知肉味的。

与韩冈沉默的喝了一阵茶,苏颂突然又开口,“玉昆,你想过没有?”

“什么?”韩冈放下茶盏,抬眼问道。

“有此一事,天大婚可就会有人出来催了。”

韩冈毫不意外,冷笑道:“大概皇太妃会提吧。”

既然天已知人事,那么就该早点将婚事定下,免得嫡长还没生,前面就一堆皇皇女了。皇女还好,要是有那么一两个皇赶在前面,那日后就有得麻烦了——总会有人这么说。

等到天大婚之后,就能赶着太后撤帘归政。其最迫切的,便是天的生母,封了皇太妃的朱氏。

“玉昆……”苏颂先看了看门外,有些顾忌的凑近了低声道:“你看此事是不是玉华宫所为?”

韩冈苦笑着摇头,“说不清。但如果是有人故意如此,太妃的嫌疑最大。”

韩冈也考虑过是不是朱太妃故意在背后唆使天如此,逼着太后不得不让天及早大婚。

皇帝大婚,拖到十七八都可以,若是要早,十三议婚,十四成婚,也不是不行。关键就看太后怎么想。

如果太后无意学章献刘后,在皇帝大婚后还把持朝政,那么大婚后归政便是在情理之。

“在天亲近人,当然希望太后越早归政越好。用这种丑事逼宫,不是不可能。”

“但后面有些地方不通情理。”苏颂时候又要否决自己一开始的想法。

“的确还有些不通情理的地方。”韩冈点头,同意苏颂的看法,“要不是这样,早就可以认定了。”

只是有句话韩冈藏在肚里没说,女人嘛,思考回路跟男人完全不一样,身为男人,永远都不要奢想了解女人的想法……

有臣曾经上书唐太宗,提议唐太宗去佞臣,并建议唐太宗上朝时故作震怒,‘不畏雷霆,直言进谏,则是正人,顺情阿旨,则是佞人’。唐太宗回复说‘朕欲使大信行于天下。不欲以诈道训俗,卿言虽善,朕所不取也。’

而章献刘皇后,则曾经对宰辅们说,让他们将自家的侄名字报上来,她会赐予官位,以酬宰辅之功,等到那些重臣们一个个将名单列好呈上,刘皇后随即便一翻脸,但凡在名单上的都不用了。

有唐太宗故事在前,做皇帝的,哪个会这么做?刘皇后代行帝政,却自以为的毁了自己的信用,让自己在臣,变成一个爱施诈术的狡妇。须知孔都说过‘轻诺必寡信,民无信不立’。

但是,女人执政,想要他们去下权衡利弊,真的不是那么容易的事。

“不过现在还是太皇丧期,这么做对天的名声也无好处。”苏颂皱眉又说道。

“有恃无恐,还怕什么?”

如果是天刚刚犯下大错,就让他退位,不是不可以。但事到如今,太后和宰相都把宝压在了赵煦的身上,弑父的事都放过了,对高太皇不孝的罪名还有什么不能放过的?事到如今,已经不可能再放弃他了。

就是太后明知此事背后是朱太妃作祟,皇帝也是故意如此,也不方便就此发作,难道还能将皇帝退位不成?

苏颂拧着眉,这事还是有问题。

韩冈不等苏颂了,起身告辞,“是与不是,只看朱太妃到底会怎么说就可以了。这两天就能知道了。”

……………………

次日,早朝之后,议政重臣们齐聚崇政殿。

除了太后,就只有天的御医钱乙在场。

“钱乙,你把昨天的事跟诸卿家说一说。”有些话,太后不好当面说说,只能借助御医之口。

钱乙很快的将天的病情报告向殿所有重臣通报了一番,等他说完,不等宰辅们消化了这个消息,向太后便道:“昨日吾与韩相公商议,除了皇帝之外,其他人皆流放西域,而最近官家亲过近的宫人,则先养起来,看看情况。”

都是跟在赵煦身边的人,犯了错当然要处置,至于怎么处置,即使太后自己决定,群臣也不会有什么意见,除非太过火了——比如像赵煦之前说的,全都赐死。

不管赵煦身边人做了什么,只要不是有心谋害天,至少罪不至死。针对这种并非侵害了朝堂权力的内侍,士大夫们的态度一向很宽容,换作是王正这样的大貂珰坏了事,落井下石的绝对不会少。如果是掌握皇城司的都知,更是非要置之于死地不可。

现在的决定,既然是太后和韩冈共同做出的,又不是太过分的决定,也没人会跳出来的多说什么。

看着群臣无人有异议,向太后欣慰的点头,又冲着群臣道:“官家年纪也到了,这事也堵不住,万一有个喜信,日后也不好办。是不是可以为官家选后,大婚后,也能多个约束官家的人。”

韩冈隐晦的与苏颂交换了一个眼神,对面的章惇也是一副早在意料之的神色。

苏颂上前一步,“不知是何人向陛下提议?”

“皇太妃。”
