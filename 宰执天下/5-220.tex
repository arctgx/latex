\section{第24章 缭垣斜压紫云低(12)}

所有人的注意力都集中到了韩冈的身上。

韩冈平静如常,并没有失礼的放任自己的视线,而是垂下去看着脚前的地面。但当他开始向前迈步,前方的所有人如同分开的海水,全都让了开去。甚至是坐在床榻边,拉着儿子的手的高太后,都给传说中慈济医灵显圣守道妙应真君的私淑弟子挪开了位置。

赵佣和他的姐姐一起被抱离了床榻,在母亲的怀里张大了眼睛,仰着脖子看着这名走上前来的陌生人。

虽然赵佣年岁还小,但生长在宫廷中,周围人言传身教,还是颇为早慧。在他面前的人个头很高,肩膀很宽,面容微黑,跟瘦削白净的父皇完全不同。同样的紫袍金带,但比起见过好几次的王相公要年轻很多。当他抬起眼时,幽深的双瞳中却仿佛有光。

赵佣知道,这就是父皇为自己千挑万选的师傅。母后和娘亲还有国婆婆、几个服侍自己的姐姐和冯家哥哥【注1】,都经常提到他的名讳,尤其是这段时间,更是特别的多。

他是当世的大儒,将会在资善堂为自己开蒙授。又有武功为大宋开疆辟土。更是孙神仙的弟子,发明的种痘法让天下的小儿不用再担心痘疮,也包括自己和姐姐。而且姑姑家的益哥私下里还说,之所以那么大的功劳都没当相公,是因为父皇特意留下来给自己,好让他日后辅佐自己做个明君。

赵佣抬头望着父皇为自己留下来的宰相,目光中满是好奇。

韩冈举止舒缓,意态恬和,坐在榻边的,直接抓起了赵顼的手。并不是把脉,而是拇指按在赵顼右手掌心,其余四指压在手背上,一种让人看不明白的动作。

如此奇特却沉稳的举止,却让所有关心着赵顼安危的人们看到了希望。向皇后等人脸色依然郑重,却没有了方才的惶然。赵佣年岁还小,还不懂其中的曲折。但他也能够清晰的感觉得到,方才还僵硬的搂着自己的娘亲,双臂在不知不觉间已经放松了下来。

“蜀国公主到!”

静寂之中,突然传来了通报声。

“是姑姑,还有益哥。”淑寿公主轻声叫了起来。

双眼红肿,眼眶含泪的蜀国公主牵着儿子出现在寝殿门外。蜀国公主未施脂粉,连衣服都是家常的襦裙,外面套了一条褙子,看得出来她来得有多匆忙。

由于驸马王诜不在京中,宫内的母亲、兄长和嫂嫂又没有想起来遣人通知,蜀国公主迟了好几个时辰,才得知赵顼于宫宴上发病的消息。匆忙带着儿子乘着家里的马车一路赶来,已经是入夜时分。

赵顼因为子嗣不振的问题,对威胁皇位的两位兄弟有心结,所以将关爱都放在了蜀国公主这位胞妹身上。王诜慢待蜀国公主,纳妾嫖妓,日日笙歌,赵顼将其贬官、远斥,没少给妹妹出气。兄妹之间的感情,远比三兄弟要深厚得多。赵颢等着接手皇位,赵頵躲回家里避灾,只有蜀国公主真心在为他们的长兄哭泣。

可是蜀国公主的到来,这时候在赵顼的榻边却几乎没有人回头一顾,就是亲生母亲高太后没有多看她一眼,只有正好面朝大门的侄女注意到她。其他人,包括向皇后、朱贤妃在内的后妃们,包括蜀国公主见过的王珪在内的一众宰执,包括寝宫中所有的宫女和内侍,全都注视着站在御榻前,那个紫袍罩身、金带环腰的宽阔背影。

“陛下。”韩冈的语调平静冷淡,没有加入什么感情,就像是到普通人家按时问诊的家庭医生,“如果能听见,请眨一下眼睛。”

所有人都摒住了呼吸,盯紧了赵顼的眼睛。

眼皮果真眨了起来,一下,两下……

韩冈轻吁了一口气,紧绷的肩膀也松弛了下来,“还好。”

就这么简单?!

连章惇都瞪大了眼睛。这是儿戏吗?

从太后到宫人,所有人都在仔细聆听着韩冈的每一句话,关注着他的每一个举动,半点也不敢疏漏。可看到他这么简单就作出结论,人人惊讶莫名。

就是这么简单!韩冈轻轻的放开赵顼的手。抬头迎上众多疑惑的目光,回以他们淡然而肯定的微笑。

眼下的当务之急是什么?

是必须能让赵顼与人沟通,但更为确切的讲,应该是让世人认为他可以与人沟通,还有足够清醒的意识!

在赵顼无法通过语言和动作来传递信息的时候,怎么才能证明?

这就要靠专家来判断了。

赵顼的情况很糟糕,在韩冈看来——更多的从几位御医的脸色上来判断——最好的结果,都只是能勉强说话,重新下床走动的可能性都已经很低了。

恐怕赵佣之后,赵顼可能不会有再有其他子嗣了,宋徽宗赵佶基本上连成为受精卵的机会都不会有。瘦金体是不会有了,宋徽宗的鹰、赵子昂的马也不会有了。至于会不会再有靖康之耻,那倒是要看情况了……

虽然说只要睁开眼皮,瞎子也会眨眼,白痴当然也会,因为中风而导致的瘫痪和麻痹,一样要眨眼,完全没有意义。除了证明赵顼还活着以外,无法证明任何事。但韩冈有足够的资格来扭曲事实。

要想取信于人,一个专家的身份绝对少不了。而在这个时代的医学领域中,没有比韩冈更加权威的专家了,更明确地说,是类似于神明的存在。

普天下以百十计的药王庙中的塑像可以作证!天下四百军州,数以百万计种过牛痘的幼儿同样可以作证!

韩冈说赵顼是清醒的,那他就是清醒的,说他情况还好,那就不会太糟糕!

谁会质疑?

“皇兄!”赵颢带着哭腔呼唤着,赵顼又在眨眼——其实他自睁开眼后,就一直在眨。

“皇兄就只能眨眼吗?”赵颢回过头来质问韩冈。

他是在质疑……

“能对听见的话做出反应,瞳孔见光后有变化,手上也有感应,可见陛下的头脑还是清醒的。依常施针喂药,只要不出意外的话,陛下不久后当能开口。”韩冈对高太后和向皇后说道。

被韩冈丢在一边的赵颢脸色一变,又咬着牙隐忍下来。

膝跳反射的实验韩冈没有做,针灸看反应也不需要,只要握着手看看眨不眨眼,凭借身上的光环,韩冈做出的判断,区区雍王,根本没有能力来动摇。

因为韩冈的权威,因为有许多人愿意相信韩冈的结论。

——愿意相信和相信,是主动和被动的区别。两种情况,使得一个谎言,在成功的难度上有着决定性的差异。

赵顼的皇后和嫔妃们当然是愿意相信的。赵顼的亲信内侍们也是愿意相信的。他们的利益和个人安全与赵顼的安危紧密相连。

几位后妃看看赵顼,又看看韩冈。石得一、宋用臣几人更是紧张的望着这位药王弟子,赵颢上台,他们这群赵顼的贴身家奴,绝不会有好下场。

赵顼还拥有清醒的意识,不论真也好,假也好,对他们来说,都必须是真的。就算韩冈说的是假的,赵顼的后妃和家奴们也会将其变成真实无虚的事实。

韩冈沉稳的微笑,给他们莫大的信心,一个个放松下来。

韩冈也放松了下来。基本上,只要赵顼的病情不立刻恶化,撑上一段时间是没有问题的。

尽管赵颢点着头,说着‘那就好’,但神情中暗藏的阴暗却没逃过章惇的眼睛。

‘好一招翻云覆雨!’枢密副使暗自赞叹着韩冈的手段。

就算今天的事如果传到外面,赵颢尽力宣扬赵顼的病情,也是不用担心的。因为韩冈的结论,同样会传扬出去。

一边是声名狼藉的亲王,一边是皇后、嫔妃,还有药王弟子,哪个说话的份量更重,哪一位更能让人相信,这一点是完全不用多想的。

“阿弥陀佛。”念了一句佛之后,王珪终于是做出了宰相该做的事,“天子御体违和,臣等当依旧例轮值宿卫宫掖,以待陛下康复。”

高太后点点头,自然而然的作为主事者发话:“一切都交托给相公了。”

几位宰执很快各自的排班顺序,并招了仍在外殿的几名三衙管军进来,将他们的宿直的排班也当着高太后和向皇后的面给议定了。以张守约为首,几名太尉纵然军功赫赫,也只有俯首听命的份。

议定之后,不当值的宰执便要出宫,向皇后便点了石得一送他们离开。此时已是夜阑人静,皇城四门已经落锁,没有管勾皇城的石得一相送,几位宰执连大门都出不去。

一干文武重镇离开,韩冈却还在。他是作为医生进了寝宫,但也不方便久留。

向皇后看看韩冈:“官家如今御体尚未安康,韩学士可否宿直宫中,以备缓急?”

韩冈行了一礼:“臣谨遵懿旨。”

向皇后跟着点起两名大貂珰:“蓝元震,宿直的外殿,你去安排一下,都要准备妥当。宋用臣,你代六哥送一送韩学士。”

注1:宋代皇子对身边亲近的宫女内侍,常以婆婆、姐姐或哥哥称呼。在宋人笔记中便有记录,仁宗幼时便曾称亲近的宦官周怀政为周家哥哥。‘周家哥哥斩斩’,也是很有名的谶言。
