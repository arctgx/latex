\section{第35章 把盏相辞东行去(三)}

好好一场饯行宴给个厌物搅和得不欢而散,刘希奭送他们出来时,也只能苦笑着说等日后有机会再聚。只是这可能性不大了——韩冈自京中回来后,就是正式的秦州官员,走马承受碍于身份,便不可能再邀他一起小聚。自然,韩冈和王厚并不会在意刘希奭的宴请,只要秦凤走马在心底里给窦舜卿记上一笔账那也就够了。

别过刘希奭,韩冈、王厚、王舜臣等几人自惠丰楼一起往普修寺走去。还在年节中,又刚刚结束了春牛祭典,城中的大街小巷热闹非凡。噼里啪啦的鞭炮声不绝于耳,穿着新衣的孩童在路边笑闹着,而走亲访友的人们更是络绎不绝。

王厚左顾右盼,呵呵笑道:“都在扎彩灯了,再过几日便是上元。届时城中照例的放灯三日,只可惜玉昆你今年是看不到了。”

韩冈轻巧的避过一个差点撞上自己的小孩子,也笑道:“算下行程,上元的那一天,小弟恰好能赶到京兆府。长安的上元灯会,只会在秦州之上,不会在秦州之下,我可不会羡慕你们。”

“要是玉昆你能在上元夜赶到东京才叫好!”王厚放声说着,“天下上元放灯皆三日,唯有京城五日。从元月十四到十八,城中夜夜光焰冲霄,星光皆隐。御街之上溢彩流光,星汉银河如坠城中。那样的景色,天下四百军州,数千城池,也只有人口百万的东京城中才得一见!”

王厚沉醉于记忆之中,韩冈听着也是心向往之。百万人口的世界第一大城,虽然跟人口膨胀的后世没法儿比,但在韩冈心中,却自有一番魅力。

“那不是刘仲武吗?”转过一条街,赵隆突然叫了起来。

王厚、韩冈一起望去。只见赵隆手指之处,一个二十五六的青年军官被七八名军汉簇拥着,正往街旁的一家酒楼中走去。

“他就是刘仲武啊……”

刘仲武因为受到向宝的青眼,在秦州已经有了点小名气。被一路都钤辖关注提拔的新进,总是会受到多方的关注。

王厚一直目送着刘仲武走进酒楼中,这才转头对韩冈道:“刘仲武今次也要到东京去,与玉昆你一样都是明天启程。”

“向宝荐了他任官?!”

“不是!”王厚摇头,“刘仲武不是直接为官,他的功绩还不够。如果军功够多的话,就可以像甘谷城的王君万那样连转三官,一跃入了流品,做了一名从九品的三班借职。但刘仲武不够资格,他是去京中三班院参加试射殿廷。”

试射殿廷,顾名思义就是在天子面前考试射术。只要考绩优异,也可录名为品官。不用王厚解释,韩冈也清楚这条武官晋升流品的捷径,无他,王舜臣和赵隆过去没少在他耳边念叨。

韩冈忍不住叹了口气:“虽然不是直接荐官,但向宝为刘仲武争来的机会已经够难得了。王兄弟没捞到的机会,这刘仲武却是平白无功的便到了手。”

“如此恩遇,刘仲武只要不是生性凉薄之辈,对向宝肯定是感激涕零……何况还向宝还送了一个美人给刘仲武,在家为他缝衣做饭!”王厚冲王舜臣几人扬了扬下巴,“哪个不羡慕他的运气?”

王韶如今提拔的四个亲卫,都有将他们外放去领兵的计划。其中以王舜臣的职衔最高,再升一级就能转入流内官,只是年纪差了一点,要等上两年才能实际外任。杨英是王韶乡里,以殿侍的职衔担任弓箭手指挥使,其实是白领这一份俸禄,并不实际带兵,寻常便护持在王韶左右。

而赵隆和李信,两人在秦凤都是数得着的好武艺,轻而易举便能压制着手下的骄兵悍将。赵隆的相貌身材极有威慑力,王韶平常喜欢把他带着身边,但放出去带兵一样没问题;李信则为人寡言,重要的事情交给他便可以高枕无忧,是那种可以安心的把后方和粮道交给他的典型军官。

不过计划是计划,四人如今都还在王韶手下听命,要等到外放领兵,还有一段不短的时间。而刘仲武却眼看着就要达成目标了,只要他在殿前演武时有点好表现,一个流内官身便唾手可得。

“真真是好狗命!”王舜臣对刘仲武的运气又羡又妒。说起来,如果没有刘仲武,王舜臣应该有很大的机会获得去京城的名额——只要李师中和向宝届时不反对的话。

“王兄弟的军功其实已经够了,只是争不过向宝支持的刘仲武。几十个首级在身上,还换不来一次御前演射的机会,真是吃了大亏!”韩冈摇头又叹着气,他深为王舜臣感到遗憾。

说起军功,其实王舜臣很吃亏,韩冈更吃亏。在裴峡谷,斩首三十余级,在下龙湾村,又斩获过山风以下二十多个首级,两人都是亲历其事。寻常县尉捕盗得五人,已经可以加官一级,而军功斩首有个三五十级,足以让一名小卒得入流品,鱼跃龙门。如果上头有人,靠着五六十级的斩首,甚至完全可以吹出一个败敌数千的大胜来。

但韩冈刚刚因为前一次的斩首功以及在甘谷城的功绩,而受到荐举,后一战的军功并没有被录入下来。刚过了年,韩冈才十九,能入流品已是难得,进用太速反而不利日后——李师中便是这般说的。同样,虽然看起来有二十八、三十八,但实际上才十八岁的王舜臣,也是因为年龄的关系,而与从九品的流内官无缘。

所以最后的那点在下龙湾村里的功劳,便分给了赵隆和李信二人。王厚虽然适逢其会,但他也没有从赵隆和李信那里争功的意思。

“也不必羡慕刘仲武,以四位兄弟之勇武,又能耽误几年时间?说不定再过一年半载,就是几位官人了。”王厚出言安慰着有些丧气的王舜臣四人。

韩冈也道:“处道说得没错,以几位兄弟之才,只要有机会,何愁不能一跃龙门?……”他再一笑,“而在王机宜身边,机会又怎么会少?”

“说的也是!”王舜臣的兴致又高了起来,他走过路边的摊子,丁零当啷的丢下一把钱,捧了十几个橘子回来,分给韩冈他们一人两个。

王厚和韩冈要维持形象,把两个橘子收在袖中,而赵隆、李信他们,都是剥了皮,直接丢进嘴里。几人一边吃,一边走。

王舜臣吃着一嘴的汁水,顺着胡须向下流,含糊不清的说着,“三哥也是本事,都不知道你什么时候去查得药材市价。”

调查个鬼,韩冈当然没有去调查,但他前面把事情说得圆得很,没人会怀疑。不去问过石膏的行情,谁能看透天宁寺的豆腐是用的什么材料?

王厚也是摇头,指着街边的一家药铺:“这样的铺子秦州有二三十家,要是一家家药铺去问,我可吃不消。”

韩冈笑了笑,想避过这个话题。只顺着王厚的手指方向,却正见那间药铺中的伙计把一个抱着小孩的女子轰了出来。那伙计还插着腰,在台阶上骂着:“没钱还想抓药?!又不是开善堂的!没了钱赚,要俺们喝西北风去?”

那女子虽然头发都被推搡散了,遮去了容貌,但抱着孩子的背影看上去却是楚楚可怜,让人义愤填膺。见这么一对母子受欺,好事的王舜臣当即上前几步,揪住药铺伙计作势要打。

“别下重手!”韩冈淡然的说了一句,上前将那女子扶起,“小娘子可安好?”

被韩冈抓着手臂,严素心身子一颤,心中顿时又羞又恼。哪有这般无礼的?!方才想赊贴药而被轰出药铺,已经是不幸,想不到竟然还碰上了个调戏女子的泼皮。

世风严谨,男女大防虽然没有明清那么恐怖,但随意接触良家女子的身子也并不合适。王厚在旁边咳了一声,权作提醒。而韩冈扶起严素心后,便放开手,退了一步。动作自如,神色也是自然得紧。

严素心小心的抬起头,只见韩冈的双眼清澈深邃,神色也不带一丝淫邪,并不是趁机占便宜的浮华少年。而且这张面容,虽从没有正面相见,却早已深深的刻在心底。

“多谢官人!”严素心抱着招儿向韩冈行礼道谢,声音中有着一丝微不可察的颤抖。

官人?韩冈眼眉微动,又仔细看了严素心一眼,看起来她好像认识自己的样子。自家穿的是文士的襕衫,平常百姓看到自己,多半会道一声秀才,而官人,如果不是酒楼或脚店里的小二和掌柜,就只有知道自己身份的人才会这样称呼。

王舜臣这时退了回来,他并没动手,而是放手让药铺伙计躲进店中。赵隆奇怪的问着:“怎么不打?”

“三哥都说不能下重手,那还怎么打?!俺下手何时轻过?”王舜臣反问,他探头去看着严素心怀里的招儿,看轮廓应是个一个相貌很清秀的小女娃子,但她的头面上长着稀稀拉拉的水疱,而被扯开了半边衣襟,露在外面的上臂更是密密麻麻的一片浆疱。

ps:猜一猜小姑娘是什么病?

第一更,红票和收藏。

