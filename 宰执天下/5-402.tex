\section{第36章 沧浪歌罢濯尘缨(四)}

“记得那时候,将军已经是都巡检了。”

当年韩冈还在王韶麾下为幕僚,听说了李复圭为推卸罪责大开杀戒,除了不直李复圭没有担当的品姓,并为种詠感到遗憾外,也记下了好运气的白玉。

“是。”白玉低声应答。

“之后陕西推行将兵法,将军以秦凤都监之职任正将……是第三将吧?”

“是。”白玉声音更沉了一分。

陕西推行将兵法时,已经被调任秦凤路的白玉担任第三将正将。这时候,白玉已升任了都监。但十年过去,白玉的名位依然原地踏步。

在这段时间里,河湟已得,西夏已灭,大宋官军更是已经走到了天山脚下,将河西走廊纳入疆界之中。

多少原本微不足道的文武官员,在开疆拓土的过程中一个个飞黄腾达,王韶、韩冈先后晋升西府,种谔、燕达升任三衙管军,王舜臣、李信之辈从兵卒成为一路中坚,白玉这名西军中的宿将却什么都没能捞到,唯一能拿出来炫耀一下的,就只是在广锐军叛乱之后斩首两百余的一次胜仗。

但这样的胜利,放在眼下,在面对那些正当红的将帅时,甚至都不好意思提及。一众将佐坐下来夸功耀武,别人拿出来的,不是斩首上千的丰功,就是破敌数万的大捷。参加了平夏之战的一众将领,党项人的头颅拿到手软,两百多个首级甚至还不够一转之功。

这是白玉的伤心事,听见韩冈当面提起,脸色就免不了有些难看起来。他不敢现面皮给韩冈看,只能低下头去。

白玉心有顾忌,但在身后侍立的白昭信却不禁忿然,压抑许久的怒火终于再难抑制。他愤然道:“吾父若有枢密一般的机缘,岂会蹉跎至此?!”

“住口!”白玉脸上的血色一下就褪得干干净净,猝然起身,一巴掌把儿子打翻在地上,“枢密之功天下可有几人能比,生祠遍布关西,可是你配说嘴的?!”

说着,又狠狠地照面门踢了白昭信一脚。他下脚不轻,砰一声闷响,白昭信顿时便满脸是血。

“都监,你这是为何?”韩冈皱眉摇头。

白玉下手还真会选地方,踹身子容易出内伤,外面还看不出来,照脸去打,弄得满口鲜血,却不会有大碍,但看起来却是下手极重,已经体现了真心实意的歉意。

白玉收了脚,看了捂着脸的儿子一眼,转身低头跪倒:“小儿无知,冒犯了枢密,末将回去当重加责罚。”

“孝心岂可入罪?且令郎说得并没错。”韩冈过去亲自将白玉和白昭信先后搀扶起来,让人领着白昭信去疗伤,然后拉着白玉的手坐下来叹道:“都监缘数的确远不如他人。就是曲君玉【曲珍】,之前犯了重罪,如今也得吕枢密重用。可这一回都监来援河东,韩冈知都监宿将,用兵最稳,所以方以后路相托。只是耽误了都监立功的机会。”

被韩冈拉着手,白玉坐立不安,“岂敢,枢密既然信用末将,末将又如何敢不尽力?”

“说的好。”韩冈哈哈一笑,趁势放开了白玉的手,“正是多亏了都监尽力。稳定后方,我军方能安心与辽贼决战。这一战的功劳中,少不了都监的一份。”

“枢密之赞,白玉绝不敢当!”

白玉再一次躬身逊谢,但这一回,韩冈在他的神色中,却找到了一丝掩饰不住的愤然。至少在他的耳中,韩冈的话完全是托辞,做信不得。

这样的老家伙,胡子都花白了,人当然也变得固执。当然不可能因为几句空头话就改变看法,甚至感激涕零。但韩冈相信,白玉只要功名之心未尽,接下来就不愁他不上钩。

请了白玉重新落座,喝了两口茶后,韩冈才又说道:“现如今,代州后方已经为都监稳固,剩下的,也就是面前的贼寇了。辽贼退守雁门。险关要隘,攻打不易,都监宿将,惯习军事,当有以教我。”

“末将只知听命行事,恳请枢密吩咐。”

“武侯有云:集众思,广忠益,参署是也。都监为我僚属,当可直言无讳,共参益之。”韩冈再看了白玉一眼,“此是军令,都监勿再推辞。”

“……既然枢密这么说了,白玉斗胆,就说一说想法。”白玉停了一下,见韩冈点头,方又说了下去,“雁门为天下知名的险关,末将虽从未亲眼得见,可早已是如雷贯耳。孙子说过,‘攻城最下’,攻打险关自是等而下之。”

“嗯。”韩冈轻轻点头,示意他继续。

“所以以末将愚见,不如绕过去……从武州绕过去,与朔州城中的麟府军会合一处。”

白玉是在猜韩冈的心思,然后顺着韩冈的心思说话。

既然把他麾下的兵马调到忻口寨,却又不顺便东调代州,那么当然只会去往北面的神武县。顺着韩冈的心意说话,就是拍马屁,拍得韩冈这位枢密副使见兼制置使高兴了,也就能得到一个博取功名的机会了。

“说的好。都监之言正合我意。”韩冈拍了拍手,又道,“不过以精兵锐卒抄截辽贼后路固然为良策,可也必须从正面攻打雁门诸关以牵制雁门中的辽军。”

韩冈说着目光灼灼,盯住了白玉。而白玉一下就迟疑起来,他不清楚,韩冈是不是要让他去攻打雁门关,为折克行牵制辽军。

那样的话,比之前清除盗匪的差事还要痛苦。盗匪据守的破寨子怎么能比得上辽军坐镇的雁门诸关?功劳就更不用说了,兵马损失多了,甚至还会被治罪。

幸好韩冈很快就接了下文,让白玉松了一口气:“当然了,这件事不算难,代州这里的兵马已经足够了。用不到都监的西军。”

白玉的心提了起来,带着期待,“枢密的意思是……”

“辽军的主力已经退回朔州,麟府军虽是夺下了朔州城,但面对马邑周边的重兵,自保有余,进取不足。”

如果仅仅是驱逐辽寇,韩冈没有使用西军的打算。不过现在战事进行顺利,反攻入辽境在即,那么也就没有必要将白玉和七千西军留在后方清理匪患。

在代州、忻州,辽军人人思归,无心恋战,可当战场转移到辽国国内,就在家乡作战,那么之前造成士气不振的原因,也就不复存在。辽军真正的战斗力一旦爆发出来,即便因为战马损耗的缘故,实力大幅下降,也不是现在的京营禁军能够抵挡。

折克行之前在古长城处伏击辽军之后,便兵围朔州城。如果是在大宋境内,辽军一旦被围困,不用怎么打,自然会选择突围撤退。

可韩冈这两天接到的报告中,折克行却说辽军多次出城反击,此外他还遭受了两次夜袭。战斗意志和欲望比韩冈在代州这边有着显而易见的差别。

要不是辽军实在不擅守城,加上朔州城墙多年未有修补,即使以麟府军之精锐,也不可能那么容易便攻下了城池。可即便是在已经夺下,想要再进一步东进马邑,封锁雁门关北口,光靠麟府军的力量还是远远不够的。面对完全变了模样的对手,只有在战斗能力和作战意志上同样出色的西军才能让韩冈放心得下。

白玉起身,恭声问韩冈:“白玉当如何做,还请枢密示下。”

“折克行夺占了朔州城,辽贼定然心有不甘,必举大军来夺还。你去朔州,与折克行并肩作战,联手迎敌。若是万一辽贼不肯过来,你和折克行就分兵去清扫朔州城周边的部族、寨堡,逼萧十三遣军来攻。”

韩冈的计划就是背城决战,以他手下最精锐的队伍,来迎战辽国的疲惫之师。

朔州和大同府同在一个盆地中,对辽国来说肯定是不能弃置,只要占据了此处,就像是丢在湖中的香甜鱼饵,不愁鱼儿不赶着凑过来。

且朔州城背山而立,战事若有不顺,撤入山中,以骑兵为主的辽军便难以追击。立于不败之地,这一仗当然可以打上一回。

而且神武县方面,粮草之前都准备得差不多,通过轨道节省下来的民夫人力,有很大一部分用在了通往武州的几条通道上。而在翻越古长城抵达朔州后,更可以依靠就地征发来解决。光是一个朔州城,就足以供给两军一万四千余人以及五千多战马一个月的口粮。

万事无忧,只等辽军过来一战。

白玉等了十年,终于等到了一个机会,哪里还敢迟疑,当即行了军礼,“枢密之命,白玉何敢不从?愿立军令状,定将辽贼引至朔州城下,协助折府州与其决一死战。”

领了将令,白家父子便连夜回返忻口寨。次曰上午,便领军北上。两天后,全数抵达武州,继而开始向朔州进发。

在韩冈看来,战争已经到了尾声。

大同也好、飞狐也好,都难以夺占或是守住,退而求其次,能做的,也就是攻入西京道的核心之地,俘获更多的辽人,以换回被掳走的国人,同时以巨大的损失来扼制曰后辽国入寇的念头,使其不敢再南窥。

虽然比起之前参加过的河湟、征南和平夏诸战,这一回的对辽战争可以算得上很短暂,但给他的感觉,却是旷曰持久,耗费了太多的精力和时间。

这一路磕磕碰碰,总算是有了个了局。

不过这个想法只在他接到从京中传回来的密报。

‘这算什么?!’韩冈先是瞠目结舌,继而大怒,‘这疯了吗!’
