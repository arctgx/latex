\section{第32章 金城可在汉图中(三)}

韩冈思绪沉沉,而秦玑正战战兢兢。

方才他在外面争马,其实也是这秦玑近来受尽了白眼,今天撞上了,以为驿站驿丞没把他放在眼里,睁着眼睛说瞎话,所以仗着金牌急脚的身份一下就爆发了出来。

换作是普通点的官员,撞上了有金牌护身的急脚,就算再不痛快,也得让他一让。不过今天运气不好,撞上了是重返河东的韩冈,而且已经升到了副枢密使。

给韩冈留下个恶劣的印象还好说,秦玑看他的模样,应该不会有多少闲心来责罚自己。但方才是得罪了韩冈身边的人,不说日后没了进步的机会,转眼就会有大麻烦。

要怎么补救?该怎么补救?

秦玑这辈子脑筋都没转得这般快过。

韩冈紧抿着嘴,苦思着要怎么弥补失去忻州屏障后的艰难局面,黄裳等人都不敢打断他的思路。但秦玑却突然开口:“枢副可是担心太原的安危?小人倒是有一个办法。”

韩冈抬起眼,略带疑惑:“你说。”

“代州一地多山。在辽贼打进来后,有不少忠义的将士和乡民见局面已难挽回,便纷纷避入了山中,苦盼着官军能打回来。只要枢副遣人传话,联络上他们,此辈必然俯首听命。只要他们在山里能打起旗号,辽人就不敢冒着腹背受敌的危险。等到两军对垒,他们甚至可以从辽贼背后杀出来。”

秦玑说得头头是道,黄裳闻言两眼一亮,转头看韩冈。

韩冈也在轻轻点头,这可是标准的开辟敌后战场。

若能做到,也可稍稍缓解现在的压力。韩冈不指望他们能做到后世那般能包围起城市的汪洋大海,但只要能牵制辽人的一部分注意力,韩冈已经很满意了。何况他在正面,自问再差也会比后世要强,不会让士卒白白流血,浪费宝贵的民力。多分一份力走,韩冈这边就能多轻松一分。

而且逃入山中的将士终究只是少部分,真正的大头,还是那些已经投降了辽人的军队。

代州知州降贼,下面的官兵有了领头的,投降的自不在少数。不是魏泽人望多高,而是天塌下来有高个子顶着,最后就算官军打回来,总有魏泽在前面第一个被砍头。总之对大多数士兵来说,还以保住性命为重。

只是其中首鼠两端的肯定是占了绝大多数。这跟当年的广锐军叛乱不同。那时的广锐军都虞候吴逵是在下狱后,被部众救出来的。三千多广锐军卒都受到了不公正的待遇,心中怀着一股对朝廷的怨愤之气,一心跟吴逵走到黑。而在现如今的代州,有几个愿跟着辽人。韩冈代表朝廷松松口,转眼就能倒戈一击。

“而且还有为魏贼所诱,投降了辽人的将士。”秦玑也在说着那些叛军,“他们并不是真心降贼,只是被逼无奈,对朝廷还是敬畏和忠心的。”

韩冈皱眉想着,双眉向中间拧了起来:“联络到他们可不容易。”

“小人愿往。”秦玑立刻高声喊道。

韩冈想了一阵,看看秦玑的神色,最后还是摇头,“你带着金牌,有重任在身,怎么能随意离开?你且去朝中送信,回来复命后到我帐下来,我自有任用。”

秦玑闻言大喜,连忙跪下谢韩冈的恩德。

黄裳冷眼看着秦玑的神色变化。当秦玑跪倒拜谢的时候,他的眼神便冷淡了许多,甚至还有几分感慨。

这算是入宝山而空手还的典型了。要是秦玑坚持去代州联系,韩冈倒是敢用他了,可惜心性差了,一点诱惑都禁受不起,怕是得不了重用。

不过……黄裳又瞅了瞅韩冈,自家的这位恩主,心里应该已经有了人选吧?

“秦琬现在何处?是秀容还是定襄?”韩冈问道。

“在秀容,家兄受命在忻州城上城防守。”秦玑立刻回答。

秀荣县是忻州州治,从地理上看,正好扼守住忻州盆地到太原盆地的那段山中甬道的北方起点。所以城防修筑得甚为坚实,要强于东北面的定襄县城。但也正因为地理位置的关系,忻口寨一破,辽军的第一目标必然是秀荣县,而不是定襄城。

韩冈暗叹了一声,秦琬的运气还真不好,不过秦玑方才说所的父执辈,眼下也应该在忻州,不知他们能守上几日。但忻州还是得去一趟,后方的势力不能白白浪费,只是人数要大幅削减,最多一两个人。

韩冈的视线扫过房中,除了自己的亲信就只剩下人高马大的班直。不过一个个都低下了头,谁也不搭腔。然而还是有人愿意为他分忧解难。

“枢副!小人愿去忻州走一遭。”韩信向韩冈拱手一礼。

韩冈盯了韩信片刻,见他神色坚毅,才轻声一叹:“……这一路上可不好走。”

“小人明白,但小人愿为枢副分忧!”

“也罢,小心一点就是。”韩冈不再阻拦,他对韩信道:“你早一步去忻州。让秦琬放下手中军务,去联络山中的残兵,还有城中的叛军。”

“小人明白!”韩信连忙行礼。

“如果辽人已经破城,那么你就什么也别过问了,直接回来。”

“诺!”

韩信再一行礼,韩冈最后吩咐道,“你且先下去休息两个时辰,等天稍亮一点就先走。”

韩冈现在是三日并一日走,快一点的话,一天能走上两百里。不过飞马急报,速度则会更快,一天四五百里是正常的,只要马能换得及,人的体力能跟得上就行。

韩信告罪之后就下去了。韩冈转头对仍在屋中的秦玑道:“你那个哥哥做指挥使是屈才了。”

“吾兄才干胜小人十倍。”秦玑虽然不知道韩冈为什么会冒出这一句评语,但他的回答很诚挚,因为那是事实。

韩冈很欣赏秦玑背后的秦琬。当年他就把那个年轻人的名字给记住了,现在看看,果然是不简单。

能把秦玑弄来做铺兵,这个想法也算是别出心裁,普通人想不到的。金牌急脚身携紧急军情,一向是逐程换马,逐日换人,不可能一人从头跑到尾。秦玑能拿着金牌从忻州一路跑过来,根本就是不合规矩,虽不是没有例外,但一向难得,而秦琬偏偏能为他铺垫好。

忻州直面敌缨,秦玑这个从忻州来的铺兵,掌握了第一手情报,到了京城中,必然成为关注的焦点。以韩冈对章敦的了解,多半会直接将人招进院中去询问忻州的近况——纸面上的文字永远也比不上亲历者的述说来得直观——只要在东府、西府里面走一遭,回话合人意,就不会有人想着再治他的罪,甚至能捞个官做。

就算在最坏的情况下,忻州不保,秦玑也没有在京城中撞上一个好的际遇,更甚者,还落下一个罪名,但他到了京城,好歹也能保住秦家的一点血脉。

金牌急脚一事,不需要多大的权力,却需要手上有足够深厚的人脉。秦琬就是那等有能力也有声望、却偏偏官职不高的特例。有头脑,有手段,这样的人才极之难得。相对于老迈的宿将,韩冈倒是愿意多提拔如秦琬一般年轻有为的将校。

不过要想被越次提拔,关键还是看秦琬他能立下功劳?韩冈不会因为看重某个人就坏了规矩,就看他能拉回多少人了。给人一个机会,这就是韩冈看重人的做法。

又一问一答的说了几句,韩冈便让秦玑继续上路赶入东京,天亮的时候,正好能看到开封城。他另外还遣了一名班直与秦玑随行,算是为其保驾,让他能在两府的宰执中露一个脸。

秦玑感激涕零的拜谢过韩冈后便出去了,很快就听到了一阵马蹄声逐渐远去。

“枢副,怎么不问他雁门和代州是怎么丢的?”黄裳在韩冈身边问道。

“说不清楚的事。日后有的官司要打,哪里有时间查问?”韩冈叹了一声:“而且秦玑出来的时候,应该是得到秦琬的提醒了,谨言慎行肯定是少不了要吩咐的,多半还被叮嘱不要多嘴。要不然方才他应当主动提起代州的事。”

黄裳先是一愣,转而就是勃然作色:“他竟敢隐瞒!?”

韩冈不以为意:“事关天子,秦琬怎么敢让他的弟弟乱说话?”

黄裳脸色数变,最后一叹:“究竟是怎么丢的?”

“你知道吗,京官出外任官,必然会做的一件事是什么?”

黄裳皱眉,想了好几条都觉得不对,最后放弃了,问韩冈,“学生愚钝,想不出来。敢问是何事?”

韩冈没回答,却转向身边的班直:“黄奇,你应该知道。”

年轻的班直侍卫苦思片刻,犹犹豫豫的说道:“……可是置办家业?”

韩冈一笑:“正是如此!”

原来是刮地皮啊!黄裳恍然。也难怪自己想不到,还是比不上京城中土生土长的开封人。终归是见识少,没经验。

京中虽好,但开销也大,京官往往穷困。出京就任地方的官员哪有不靠山吃山靠水吃水的道理?到了代州后,魏泽等人恐怕也不知放出去了多少回易的商队来。捞上一笔,就可吃上十年。

但黄裳还是不敢尽信:“可魏泽应该知道轻重,不至于会为了钱毁了边防才是。”

“他当然知道轻重。不过能不能控制得住,那就是另外一回事了。”

刘舜卿、秦怀信都是在边关做得长久,回易的一切情弊早就了如指掌,能稳稳的把握得住。但魏泽呢?还有他任用的商人呢?

商人的节操不好说,魏泽等人的能力更不用说。初来乍到,却只想着奉承皇帝、清洗前任旧部,以及赚钱,什么事都有可能发生。

正常情况下,就算官职调动,继任者也不会将前任留下来的官吏给一网打尽,必然还要留下一批熟手。不过变成了秉持天子之意的魏泽,自然不会守那等陈规。没了熟悉内外事的官吏,任凭一群商人来往内外而不加监视,最后怎么会不生乱子。

“当然,这也只是猜测罢了。”韩冈笑说道,但这个猜测的根源是来自于后世的那几位做了皇商的晋商。虽然韩冈并不打算打击一大片,也清楚忠义之人依然是大多数,可树大有枯枝,总是会有败类的,只是不便明说,“还是等到了太原,就能全明白了,在这里多猜也没什么意义。”

黄裳闻言起身:“那学生就不打扰枢副安歇了。”

韩冈点点头:“早点睡觉,明天还要上路呢。”

只是话出口,就不禁皱眉,‘上路’二字似乎不太吉利啊。
