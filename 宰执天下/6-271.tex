\section{第36章 骎骎载骤探寒温(七)}

“可惜润州的丝厂厂主们没有想到这种办法。”

“想到也没用。”韩冈不屑的笑了一声,“你可知道,棉纺织工厂的工厂主和丝织工厂的工厂主之间最大的差别在哪里?”

“技术?”宗泽很清楚韩冈的观点。

“就是对技术的重视程度。”韩冈点头,“除了关西之外,所有厂家的丝织技术都是买来的,而棉纺织技术则是关西的棉纺厂自己出钱出人一点点攒起来的。”

宗泽补充道:“所以江南的丝厂厂主们,不会费心去想如何改进机器,让效率更加提升,而是想方设法的考虑如何压榨工人。”

“因为研究太费时间,也太费钱了。”韩冈很满意宗泽的回答,又道,“纺机、织机,每年的研究投入超过十万贯。相关的匠师接近三百人,这还不包括给他们打下手的小工,而且纺织工人也在厚赏之下,踊跃的出谋划策,寻找改进纺织机器的可能。”

研究缺乏基础,工厂主总喜欢多克扣工人一点,在要耗费大量钱财的情况下,谁有心从头开始研究?

开发新技术就是撞大运,并不是每枚铜板丢下去都能有回报,绝大多数时候连个回声都没有。与其花这份冤枉钱,不如抄袭和模仿。

最重要的一点,韩冈没有对宗泽说。如果其他地方的丝厂厂主当真开始研究新技术,当他成功的时候,雍秦商会立刻就会把等级相当的技术扩散开来,让其血本无归。

丝织技术有很大一部分与棉纺织技术共通,可以相互借鉴,以棉纺工厂主为主的雍秦商会,之所以能够大发横财近二十年,就是因为垄断技术所形成的成本上的优势。所以雍秦商会无法容许其他纺织工厂在技术上威胁到自己。

这不是韩冈的指点,而是雍秦商会上下共同的意见,打压外界的纺织技术的发展,牢牢把握住纺织科技的制高点。

尽管这样的垄断对科技发展不利,但韩冈没打算插手。他需要雍秦商会的支持,只要雍秦商会还愿意继续向技术领域大量投入,他自会继续支持。

“但现在出了润州的事,丝厂厂主会怎么办?”

无法进行技术改进而降低成本,又不能盘剥工人,这下子丝织的成本必然要上涨,尽管仍要低于手工织造,但凭空多了一份支出,少了一分利润,这对于工厂主们来说,可比割肉还痛。

韩冈道:“有件事,汝霖你大概还不知道。”

“什么事?”

“是秀州今上海那边出的新鲜事。”韩冈转身出了机房,“前两日消息才传到京城。说是秀州的几家丝厂,开春后不打算再雇原来的工人了。”

宗泽跟上去,问道:“难道要关张?”

“不是关门,而是改雇他人。”

宗泽很疑惑的说道,“一句话就把工人都赶出门,谁还敢再上门去?而且没有了那些熟手,工厂要生产速度肯定会耽搁的。”

“丝厂的工人,最多也不过做了两年而已,新人和老手也没差多少。丝厂里面,需要熟手的是修理工,缫丝之类的工作,新人来了,很快就能上手。”

有技术的匠师,不用担心失业,不用担心被盘剥,更不用担心有人敢克扣他们的工钱。而纯粹的重复劳动,则随便什么人培训一下就可以派上用场了。

韩冈可是记得,在他的前世,再早几十年前,也就在秀洲同样的位置上,有数以十计的棉厂、丝厂,在里面工作的包身工,基本上都是文盲。

“但他们能雇谁,手伤了就赶出门,谁敢上门做工?”

“有啊,倭人。”

“雇佣倭人?”宗泽的脸上尽是迷惑,出国打工这桩事,完全都不是这个时代的人们能够想象的,“这怎么可能?”

“已经不是可不可能的问题了,前几日,已经运了一船倭人进港了。”

韩冈现在的语气已经很平静了,但之前当他听说秀州的丝厂厂主买了倭人做奴工的时候,可是大吃一惊。

资本家追逐利益的本能爆发出来之后,当真是什么样的‘奇迹’都能产生。

“先不说外藩来人必须报予官府,倭国可早就被辽人占了,他们就不怕被说成是细作。”宗泽摇着头,这件事简直匪夷所思,爆出来的话,不是一两个脑袋能抵事的。

“只有妇孺,没有壮丁。说是为避辽人苛政,故此逃难而来。”

“此事当真?”

宗泽曾经听说过,辽人攻下高丽和倭国后,在当地横征暴敛,土著死伤无算,民不聊生。

若传言无讹,那有人逃亡大宋也不是不可能。又只是妇孺,没什么壮丁,想来也不会是细作。

“当然是假的。是辽人卖来的。”

类似的事,韩冈听多了,这不就是后世常见的为了顺利移民而用的借口吗?而这一批妇孺,更是辽人当做牲口一样贩卖来的。

“本来按照过去对入境倭人的处置,是要给付食水后命其返国。但如今倭国为辽人所占,回国必有性命之忧。强令其返国,乃是促其死,不令其返国,又有违法度。故而秀州州县均左右为难。”

韩冈回头看了一眼,见宗泽听得入神,笑问道,“汝霖,依你之见,当如何处置。”

“外番入国,风俗不同,恐与百姓相冲,不可留于中国。即有妇人,可遣往边疆配军,孺子则一并前去。若有贵胄,可送至京师,由朝廷处分。”

“既无罪行,又非自愿,强遣其戍边配军,此乃不仁。家国被夺,自万里之外而投中国,不加抚慰,反而行遣,此乃不义。不仁不义,朝廷安可为之?”韩冈摇头道,“汝霖,你没用心啊。”

宗泽欠了欠身,表示歉意,他的确只是随口说说,没多细想。他问韩冈:“秀州是把他们都留下来做工了?”

韩冈唇角挑起,带了几许嘲讽,“州县左右为难,一时不知该如何处置。幸而有义民为朝廷分忧,建议秀州官府仿效蕃坊,划分出一块无主荒地,设立倭人坊。在坊外修建围墙,禁其出入。不过因为逃人皆是身无分文,希望官府可以允许其做工,以供日用。虽说这些妇孺不能离开本坊,但可以让工坊开在倭人坊之中。”

“啊……”

宗泽轻叫了一声,甚至有一种恍然大悟的感觉。这样做,的确是想得周全。

不能遣返回国,又不能逐往异。地,只能就地安置。秀州不缺荒地,划出一块很简单,又不想看见这些异国之人随意出入市井,这样的安排是最妥当的。而且有了工作之后,还不用官府时时赈济。当真是两全其美之策。

“招收倭人的就是丝厂?”

“当然。”韩冈笑道:“你看……秀州只要拿出一块荒地,就能让这群妇孺自己养活自己,还有比这个更省事的办法吗?”

宗泽接口道:“正好明教借丝厂闹了一场,两浙州县都不想看到丝厂再生事端。改雇倭国妇孺,一来是外人,便生是非,镇抚时也不需多顾忌,二来皆是妇孺,闹也闹不出大事,三来,以丝厂的情况,几年后就不剩什么人了,不用担心里面藏了辽人的细作。”

“正是如此。”韩冈哈哈的拍了拍手。

“有此一策,秀州上下不答应都不成了。”宗泽叹服,“此计是谁人想出,才智绝非等闲。”

韩冈摇摇头,“听到铜板叮当一响,瞎子都能睁眼,蠢货也能变聪明。钱财之前,从来都没蠢人的。”

“朝廷打算怎么处置?”宗泽问道,“有此一例,仿效者定会越来越多。”

“口子已开,堵是堵不上了。”韩冈坦然的承认自己无能为力,“打着逃难的名义渡海而来,朝廷也不可能将他们赶回去——你想想开丝厂的都是什么人?朝廷要这么做了,在江南的名声可就彻底坏了。”

“那就看着丝厂里面充斥倭人?”

“交州这些年,种植园数以千计,人口不敷使用,早已开始雇请南洋人种地,福建富户,家中也少不了有几个南洋婢女。知道他们为什么喜欢用南洋女吗?因为死了也没人过问,”

韩冈自问自答,言语间有着淡淡的不快。

陈执中的小妾张氏——也就是前些年闹得沸沸扬扬的陈世儒弑母案的被害者——捶杀婢女,如果不是因为有人想踩陈执中立名,根本就不会爆出来。

而且最后仁宗皇帝对这件案件的判决,就是安排张氏进尼姑庵修行——这是在她又逼死了另一名婢女之后。

故而她被亲生儿子和新妇谋害了之后,很多人都说这是因果报应。

“再过些日子,这些倭人只会是被辽人贩卖过海。既然有了倭工,高丽婢当然也会有了。”

时隔几百年后,高丽婢再一次充斥达官贵人的府邸。那时候,没有律法约束的顾忌,不知会平添多少冤魂。”

“那该怎么办?”

“慢慢来,不要急。”

今天,工人们能为恶劣的工作环境怒烧丝厂。到了明天,失业的人们也能为一份相同的工作,而把工厂再一次烧毁。

韩冈对宗泽道:“有些事,急不得。”
