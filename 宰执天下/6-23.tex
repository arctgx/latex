\section{第五章 冥冥冬云幸开霁(二)}

章敦和韩冈都不担心宽衣天武是来支援叛军的。

守卫在天子、太后和太皇太后周围,主要是班直和宽衣天武的成员。守卫宫掖,同样有天武军。

作为更为贴近核心的禁卫,他们就跟班直一样,绝不可能听从石得一的指派。

没有一个强有力的领导者,宽衣天武即使站在叛军的一方,能起到的作用也极为有限。

而且若是那么容易就被收买去,张守约就未免太失败了。

最多也就两三个将校给收过去,在变乱时,稳住宽衣天武,不让他们出来坏事。

更有可能他们根本不知道是怎么回事。

发现皇城司兵马在攻打大庆殿,不知缘由下,决定静以观变。待到皇城司失败,觉得是痛打落水狗的时候了。

果不其然,这一批天武军正是来打落水狗的。拦住了几条通往西面的通道,将向西走的叛军一股脑给兜了起来。

原本因为班直的人数不足,不能将所有叛贼尽数擒获,现在终于是有了足够的人手。可这等虎口抢食的行为,也惹来追杀他们的班直愤愤不平,

韩冈没闲空去理会宽衣天武,他让人去太医局找担架,至少要尽快将张守约送到可以动手术的地方。

班直跑着走了。韩冈仍忧心不已,张守约年纪大了,不知道能不能撑过这一回。

章敦已经走到了一边,拉着一名班直再问些什么。

韩冈心情此时更加沉郁。

战场上共同出生入死才得到的来之不易的信任,这么多年才累积下来的交情,正因为这一次的变乱,而产生裂痕。

章敦是知道自己想法的,恐怕免不了要认为自己是私心坏国事。

之前章敦选择保持沉默,当也是希望蔡确能够继续说服向太后废立天子。只是他肯定没想到蔡确在失败之后,会毫不犹豫的将太后直接给抛弃掉,并没有去联合其余宰辅。

蔡确的选择不能说有错。

如果自己坚持要保幼主,不论反对这有多少,必然能说服向太后。太后的态度出去后,就没人会跟随蔡确、曾布。

当赵煦亲政后,其他人或许还能保条命,但蔡、曾二人是必死无疑,甚至株连满门良贱。从蔡确的角度来看,他是绝不会的愿意看到这个结果的。

只是还有其他地方有问题,有着说不通的地方,让韩冈依然很难理清一个头绪。

蔡确已死,再也不能确认他当初的想法,石得一也死了,皇城司的这一条线也算是断了。只能通过其余谋划者和参与者的口供来推测了。

想到这里,韩冈的神色又是一变,“留那些叛贼一条命,有话要问他们!”

当年庆历卫士之变时,当参与进来的禁卫失败后,仁宗皇帝曾喊着要留活口,好用来查明真相。但最后却是一个都没有留下来,参与进去的叛贼全都给杀了。

这一回可不要如此。到时候连追究都不可能了。

“玉昆,你打算事后穷究吗?”

章敦听到韩冈的喊话,便质问着韩冈的用心。

“该放的放,该抓就抓。可以不穷究,但必须要追究。”

“该放就放,该抓就抓。”章敦轻笑道,“倒像是魏武在官渡之后的作派了。”

韩冈脸色稍稍一变,章敦这是乱比喻。自己什么时候这样说了?

官渡之战时,曹弱袁强,曹操麾下多有写信联络袁绍。待袁绍惨败,往来信件被缴获,曹操没有拿着证据追究,而是一股脑的烧了自己麾下与袁绍方通信勾结的证据,从此人心安定。

但这一回的情况完全不同,韩冈也不觉得自己需要比同魏武帝。

“玉昆。你觉得这一回蔡确为何能够这么做?”章敦重又问了这个问题,用词稍稍有些不同。

韩冈为之正容。

向太后是相信了自己,所以才一力保住赵煦的皇位。否则她只消顺水推舟就可以了。

当所有人都知道必须说服王安石和韩冈,才能说服太后的时候,蔡确的心思转到另一个方向上也就无可厚非起来。

韩冈没办法洗脱自己身上的责任,总是外人不追究,他自己的心里也明白,必须为此事负责。

具体的细节,没有必要去猜测了。

自己在辅臣中给孤立了是事实。

章敦大概就是这个意思。可韩冈有另外一种看法。

苏颂新官上任不久,权力抓不到手中,论耳目消息,还不如自己。但章敦不是。

韩冈没有责难章敦态度的立场,这是他自己的选择。而章敦也不是他的下属,自然有自己独立的判断。

放弃两府中的职位是自己的错,怨不到他人头上。

放弃了中枢内的职位,就等于放弃了应变的能力。并不是所有时候,都能够后发制人的。

这一次,蔡确叛得仓促,但差点就给他成功了。

犯过一次蠢,他不打算犯第二次。

殿外的台阶上,寒风呼啸。

等不及担架过来,韩冈让人做了简易的担架,将张守约先抬进了殿中。

韩冈返身进殿,此时大势已定,殿中的气氛明显的活跃了许多。力挽狂澜的韩冈,更是得到了所有人的注目礼。

但韩冈看见宋用臣已经倒在地上,脸色就是一变。

“这是怎么回事?”韩冈皱起眉。

“自杀了。”王安石说道。

韩绛无奈:“咬舌自尽,谁都来不及阻止。”

“怎么就让他这么轻易就死了?”韩冈还有很多事要问,宋用臣作为太后近臣是个关键。

“可惜了。”章敦啧了一下嘴,“他可是值一个节度使呢!”

“二大王也是节度使!”张璪低声笑道:“太皇虽不能治罪,但制住伪帝和太皇亦是大功。要恭喜玉昆,还有你的表兄了。”

首先开出节度使价码的是高滔滔。之后王安石如报复一般,也为韩冈定下的四名首恶都开出了节度使的悬红,之后明确说擒杀宋用臣与石得一者为节度使的是章敦,不过其余两名首恶的赏格,自不会比

宰辅押下全家性命所订立的誓言,事后必然要让太后予以追认。

李信擒了赵颢,这个节度使,不出意外就拿定了——只要韩冈愿意去拿蔡确的赏格。

赵颢在韩冈捶杀蔡确之后,便失了魂,大喊着要杀韩冈。可李信一提了刀剑过来,赵颢就往他亲娘那边跑。

李信是个实心眼,追着赵颢直接就冲上了台陛,不仅赵颢给他用剑柄敲打了一顿,连赵颢的儿子也被他一把拎起,丢到了太皇太后怀里。

本来高滔滔看见儿子被李信踹倒,正准备去保护儿子,却给李信丢出的孙子一砸,又摔回了座位上,差点就闭过气去。

韩冈从一开始就没有将这一家三口放在眼里,没臣子理会他们,他们就什么都不是。由于宰辅们的刻意忽视,又有石得一在外,一时间就连班直都把三人丢一边,但李信一刀一剑控制了祖孙三人,郭逵用了大半辈子才拿到的节度使,李信现在轻松到手。

不过韩冈在让李信过去的时候,倒是没注意这一事。

宰辅们的誓言之中,悬赏只是其次,关键是只诛首恶和不从逆者有功无罪这两条。韩冈当时表面平静,心中可是紧张到了极点,王安石和章敦到底开出了什么价码,他还真是没注意。

倒是之后出了大庆殿,面对石得一带出来的叛军,郭逵宣布的悬赏,他却清楚的记下来了。当时自觉大局已定,心情已是轻松了许多。

“可惜没有一剑砍死这贼子,就是生擒也好啊,那可就是身兼两节度了。”

王厚半开玩笑的话中,稍稍有些酸意。

韩冈让李信去捉赵颢,却留王厚守着王安石。当时情况紧急,谁都没有多想,可现在几位首恶,或擒或诛,当事人的心中,就免不了有几分怨艾了。

从王厚的态度上,韩冈不得不庆幸,幸好宋用臣自杀了。

如果李信捉住的只是那一家三口倒没什么了,可要是他顺手将宋用臣也砍了,那可就成为众矢之的了。

节度使没有遥郡,遥郡官最高也只是遥郡节度使留后,武臣外任更是只到遥郡观察使为止。一说节度使,就只是正任官。三十出头的节度使,一般的宗室都难做到。

“说这么多废话做什么?”王安石忽然不耐烦,“当速去迎太后!”

得了王安石提醒,众人纷纷警觉,表示赞同,“我等一同前去。”

向太后和赵煦被囚禁在坤宁宫中,并没有受到什么虐待,时间也很短。

此番叛军的首脑皆已死。守在殿外的叛军全都投降了,不肯投降的则全数处决,

向皇后身边的宫人只有少部分参与叛乱,而不论哪一部分,现在全都不存在了。

当群臣涌进太后和皇帝被囚禁的室中。

向皇后从软榻上盈盈站起,没有哭闹,或是惊喜。

平静得让人难以置信,双眼只看着人群中的韩冈:“蔡确呢?”

韩冈低头:“被臣杀了。”

“太皇太后和齐王那逆贼呢?”

“全凭太后处分。”

