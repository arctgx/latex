\section{第40章 帝乡尘云迷(四)}

一阵寒流从北而至,透骨的北风刮了两天之后,阴云密布的天空终于放晴,而在河阳南门外流淌过的黄河之水,也终于冻透了底。

韩绛一早就安排了人手去河上探查冰情,回来报告时便说,黄河上现在已经有行人往来。冰层已厚有一尺,足以让车马能在其上通行。

韩绛等得就是这个消息,连忙点起了州中厢军,依照历年来的惯例,在冰面上用木板、草席铺设过河的道路。

当天午后,新任宰相韩绛便带着浩浩荡荡的家人和护卫,车辆数十,骑手上百,越过冻结的黄河,望着东京城急急而去。

韩绛可是急着回东京城就任宰相一职。

再过几日就是冬至的郊祀大典,若是误了时候,就只能让次相冯京代劳了。

他决不愿意这份功劳,落在了冯当世的手中。

郊祀是国家首屈一指的大典,侍奉天子、参与其中的官员都能得到丰厚的赏赐。而所谓的赏赐,决不仅仅是金银财帛那等俗物。官爵晋升,荫补子孙,都是应有之义。而主持整套典礼流程的宰相,更是能得到其中最大的一份。而且若能让大典安然结束,在天子面前,韩绛也足以证明自己是一个合格的宰相了。

不过韩绛现在考虑的,并不是怎么从冯京那里接手郊天大典的主控权,而是在与幕宾秦洳,商议着该如何顺利接收王安石留下的政治遗产。

一行车队中,韩绛所在的马车是最大也是最安适的一辆,是孟州驿馆中最好的马车。

车厢壁上辟出来隔间内点着个香炉,三条腿卡在凹槽中,车子晃得再厉害,也不动分毫。浓浓的檀香味从炉中飘散出来的同时,也将融融暖意在车厢中散布开来。

韩绛盘膝坐着,已经年过六旬的他现在不复当年在陕西,指挥着千军万马时的精神。须发皆已花白,脸上的皱纹也一天多过一天,只是腰背依然挺直,即便是在颠簸的车厢中,他也没有靠着身后的软垫。世家子弟的自幼练出来的仪态,任何时候都不会松懈下来。

坐在他对面的幕僚秦洳秦深秀,相貌清癯,身穿青布襕衫,做着儒士打扮。是一个也在往着暮年走去的老者,五十岁上下,颌下留着三缕长须,眼尾上挑的一对凤眼,幽深难测。

秦洳的声音平和澹然,将韩绛面临的形势娓娓到来:“相公离朝已有多年,朝中故旧不是出外,便是已经生疏。可冯京自今上登基后,便没有离开京城过。熙宁三年开始担任执政,如今在政事堂中已有四载,根基早已厚植。而王珪境遇也与其相类,都是在政事堂中时日久长。至于吕惠卿,他虽然年资浅薄,但他一直辅佐王介甫,在曾布叛离之后,他就是新党第二号人物,如今王介甫出外,新党中人当是就要以他马首是瞻。”

秦洳看了一眼默不作声的韩绛,直言道:“真要论起来,政事堂中的两相两参,势力却是以相公你这位首相最是单薄。”

这个道理韩绛当然明白,要不然他何必在摇晃的马车中还找来秦洳商量,依然保持着沉默,听着幕僚的后续。

秦洳继续说了下去:“相公是为首相,荐举堂除之权由相公总掌,而审官东院也脱不出相公的掌握。不过相公若是刚刚上任,便引用私人,必然会惹起议论,天子那里,怕也会失望。”

“所以要任用谁,提拔谁,都要有个准数,不能妄为。”这点官场上的常识,韩绛何须他人提醒,只是等着秦洳将答案给他,才耐下性子,顺着话题说话。

“相公所言甚是。”秦洳点着头。

秦洳他作为韩绛的耳目,这些年来多在京城中居住,常年写信通报。不过他是今日一早才过了冻结的黄河,见到了韩绛。对于京城中的大事小事,秦洳给韩绛写信说了不少,但有些话必须要当面说才能让人放心。

“朝中职位成百上千,可其中只有中书中的职位,虽然品阶不高,却最为关键。尤其是中书五房检正公事这一职,决不能让冯京抢过去!”

“那是自然。”韩绛点了点头。

只看中书五房检正公事这一个职位设立以来都是谁人担任,就知道这个位置的重要性了——吕惠卿、曾布、章惇,哪一个不是王安石的心腹,哪一个不是新党中的核心?

韩绛做了多少年的官,早知道要想在政事堂中,中书五房检正公事的职位上必须坐着自己人。

而秦洳此时话锋一变:“但即如前面所说,任用私人决然不妥,而相公举荐上来的人选也很难争得过冯京、王珪和吕惠卿。”

“哦……那深秀你觉得该用谁人?”韩绛饶有深意的问着。

“听闻相公是王介甫荐上来的,天子任用相公,当也有稳保新法的用意。所以相公荐上去的人必须是……”秦洳说到这里话声一顿。

韩绛立刻急问道:“新党?”

“不,必须是王相公的戚里,这样才能让吕惠卿不便反对,而不得不支持相公。同为一相一参,作为首相的相公,当能压倒冯京、王珪。而且京中也有传言,王介甫去任不以罪,天子甚有愧疚。”

秦洳终于说到了韩绛想听到的地方。

“可是王平甫【王安国】?”韩绛先说了一句,却又立刻摇头否定:“王平甫喜声色,为人轻佻,此人不合用。王和甫【王安礼】却是不错,他在河东的几年,做的事让人无可挑剔。”

“不是王安国,也不是王安礼。”秦洳摇着头。

“那是谁?”韩绛眼中透着讶异,还能有谁?王安石的另一个弟弟王安上任职的地方离着京城太远了,一时之间可调不回来。

“是韩冈!”

“韩冈?!”韩绛闻言一怔。

秦洳沉沉的点头:“正是韩冈韩玉昆!”

韩绛沉思不语,手轻轻拍着膝盖。

其实他对韩冈的评价不低,毕竟韩冈在罗兀、在咸阳所作的一切,韩绛都看在眼里,让他对王安石的这个女婿报着不小的好感。

经过了这么多事,尤其是安置数十万河北流民,使得韩冈已经被公认为是朝中为数不多的能臣之一。有富弼旧年在青州的表现,韩冈宰相之才的四字评语便无人能否定。不过世间多是夸赞韩冈的才干,也有称赞他说服叛军、扭转天子心意的纵横之术,但韩绛对韩冈的评价,当先一条却是为人正直。

韩冈曾经当着他的面,反对横山攻略,说其必不能成事。而后来传出的消息,韩冈更早一点的时候,更是对着王安石说,即便横山成事,他也不愿领那份功劳。

如果是寻常大臣说了这句话,即便不会暗地里使坏,也会消极怠工,不让自己日后成为笑柄。但韩冈却完全例外。他在罗兀城,皆心尽力,但凡当日一起被围在城中的将校,无人不赞其功。甚至可以说,没有韩冈,罗兀的战局在西夏大军围城的时候,就已经无可挽回了。就是靠了韩冈的谋划,才一直撑到天子诏令逼迫撤军的那一天,且也不见颓势,甚至犹有余力,打了一个伏击。

虽然反对某件事,却能不以私心坏国事,而尽心尽力的去完成。韩绛自问自己也难以做到,他所见朝臣之中,几乎无人能有这个气度。只是有个问题,让韩绛不便去考虑韩冈。

“韩冈的确可以大用。”考虑良久,韩绛抬起头来,对着秦洳说道,“但他未免太过年轻了一点。”

“年轻又如何?府界提点都当了,中书五房检正公事难道他当不了?!”秦洳反问道。他看得出来,韩绛其实是在推脱。

韩绛看了秦洳半晌,叹了口气,终于说了实话。他将心中顾虑告诉了幕僚:“以韩冈的身份地位,想必吕惠卿多半已经提了他的名字。以如今新党的现状,新党之中并无其他更为合适的人选。”

“那不是正好!”秦洳忽然笑了起来,“相公既然没有更合适的人选,不如同荐韩冈。相公以示公心的同时,也让新党安心,这样一来,新党中人难道还会都被吕惠卿给拉过去。相公可是宰相啊!”

“而且相公还可以多给韩冈一些职位,吕惠卿、曾布当年能做到的,难道韩冈会比他们差?!比如判军器监,现在是曾孝宽在做,他与吕惠卿关系不差。但韩冈若是进去了,曾孝宽绝对比不过他。有霹雳炮、雪橇车、沙盘军器在那里摆着呢!再比如判司农寺,韩冈是右正言,又是知州资序,难道还做不了?吕惠卿、曾布当年坐上这个位置的时候,不过是太子中允而已。只要韩冈得任要职,新党必然要分裂。吕惠卿绝容不下第二个曾子宣。届时,韩冈也只能投靠相公。”

听着秦洳之言,韩绛点着头,频率一点点的在加快。

眼见于此,秦洳知道自己成功了,便追加一步,“而且素闻相公支持新法,却对王介甫的新学有所保留。而韩冈的态度也是如此,将张载请进京中,韩冈、吕惠卿必然心生罅隙,这岂不是大妙!”

