\section{第44章 秀色须待十年培(一)}

辞别了岸上相送的人群,载着吕惠卿一家上百口的三艘官船,陆续放开了缆绳,顺水而下。

不过是路过,但洛阳城内的大小官员几乎都赶来相送,在运河畔的与吕惠卿依依惜别。

回想起几年前入关中,经过洛阳时的萧瑟冷遇,恍若隔世。

吕惠卿当曰出京,从开封入关中。经过洛阳时,无一人前来迎接。在洛阳歇息了一晚,吕惠卿一家一大清早便悄然启程,静悄悄的从洛阳城中离开。

那时他知道,想要看他落魄的洛阳元老不知凡几。以司马光为首的西京御史台,更是紧紧的盯着他。只要有一点错处,就会放大十倍的宣传。若是有些许嘈扰,便会有一封吕惠卿过境扰民的奏状递到天子案头。

幸好吕惠卿一直是以军法治家,不留一点破绽于外。就是从街上过,也是悄无声息。当他离开之后,多少洛阳官员甚至还不知道他已经走了。

时隔数载,当吕惠卿自长安回返,情况又发生了变化。

他已经不再是王安石越次提拔的新近,而是实实在在的功臣。击败了辽国,夺回了灵武,凭着这份功绩,就是在文彦博、富弼面前,不说分庭抗礼,都是反压一头都是可以的。

有了这一份功劳在,过往他所受到的攻击全都成了笑料。文彦博之辈,除了剿灭了一个跳大神的叛贼,还有什么可以炫耀的?

而且朝中,又换了天子。

对旧党有成见的太上皇后主政,十年之内都别想翻身。而富、文之辈,还有十年好活吗?树倒猢狲散,这还需要多说?人情一尽,就是富家、文家、吕家的子孙,都得贴过来讨好。不然,他们还能想着保几代的富贵?他们可不是韩琦。

还有六天。

从洛阳到开封四百里,急脚递一天能走完。单身赴任的官员,按照正常的行程走陆路,五天就够了。

而走漕运,从洛阳放舟至开封,由于水少船多,没办法曰夜行舟,总要比陆路慢一点,多花上一天。不过至少比反过来要好,从开封坐船到洛阳是逆水行舟,视情况要八到十天的时间。

但一大家子上百人走这条路,尤其是夏天,还是坐船最是安稳。不用车马劳顿,不用路途颠簸,坐上船,安安稳稳的就到开封了。只是到了冬天就不行了,一旦上冻,从淮南的宿州往上,一直到洛阳,这一条水路都要断绝。

吕惠卿已经换下了方才出城时的官袍,穿了一身略宽敞的道袍,站在船头。

官船沿着水路正中而行,随着浑浊的河水,一路向东。而在靠岸的地方,一艘艘逆水而上的船只正由着多至三五匹,少则两匹的挽马牵引着,沿着河渠一路上行。

马是多了。吕惠卿想着。

就是在几年前,拉着官船逆水行船的多还是纤夫,小一点的船则是靠艄工用竹篙撑着走路。

现在倒好了,马力替代人力,尽管没快多少,但省下了多少人工。一匹挽马能做到的事,至少要三五人才能抵得过。而一名力工如果不负责吃喝的话,一天就要一陌,七十八文。据吕惠卿所知,比起马料来,至少节省了一半。

吕惠卿对这个变化感受得很深。如今关中驿站里面的驿马很少再有缺额的情况,自从河湟拓边以来,军中和国中的马匹数量一路上涨,好马也多见了。若在过去,战马的肩高能有四尺,已经可以充入军中上阵使用了,四尺五寸的战马,往往都是主帅才有资格骑乘。到了如今,种谔的三匹坐骑,没一匹低于五尺。

当年来自西域的一匹浮光,如同锦缎般的皮毛,和高大神骏的体格,让京师人人称叹。可前几天,也就是吕惠卿动身离开长安前,四匹大宛天马从王舜臣那边送了过来。每一匹都是神骏异常,上下没有一丝杂色,说是进献给天子。而且这其中,有三匹是能充作种马的牡马。除此之外,还有稍逊一筹的六十多匹上等良驹,或是因为杂色,或是因为体型稍逊,但肩高都不在五尺之下,里面有公有母,可以想见,京城内外的马主们将会如何疯狂。

这就是胜利者的好处。所以当年辽国南下乐此不疲,而西夏也不惜民力的不断侵攻。都是因为能通过战争得到让人满意的收获。

而大宋就只能苦苦防守,将每年税入中的八成,投入到军中。

而那些元老,竟然说这样没有问题,是祖宗之法,要一直保持下去。文、富之辈目光之短浅,可见一斑。

只有彻底解决西夏,才能从不断将国力消耗在山野中的窘迫境地中脱身出来,否则只会越陷越深,到最后无法再支撑。东汉的灭亡,有不少功劳得归功于始终无法降伏的羌人。在陇西耗去了太多国力,让东汉朝廷不得不征收更多的税赋,加上昏君歼宦,最后再一场不合时宜的天灾,让步履维艰的朝廷再也无法支撑。

而现在,原本为了抵御西夏而设立的几个经略安抚使路,都要逐步撤销。而山中的成百上千的大小军寨,也得废弃大半。等到横山一线的军寨中,非关紧要的那一部分都改成屯田堡。整个陕西的军费消耗至少能减去四成还多。

就是朝廷那边对怎么划分疆省土还没个定见。一会儿是银夏路,一会儿是宁夏路,一会儿又说是灵武路,总之因为担心再出一个李继迁,想将那些蕃部都给分开,但始终找不到一个合适的分界线。

吕惠卿对此只觉得好笑。早点给一个确定的说法,将镇守关西的主力集中在几处战略要点上。这样陕西就可以安心的发展了。关中百姓受了几十年的苦,也该安心的休养一阵了。

大宋军事的重点必须要尽快开始北移。辽国国势因为耶律乙辛的缘故,正处在衰落中,短期内没有重新恢复的可能。这正是大宋解决百年宿敌的良机。耶律乙辛年纪不小了,他篡位迫在眉睫,十年之内,机会必然会到来。

一旦辽国内乱,大宋绝不能坐视,河北将会是其中的关键。从这一点上来说,吕惠卿还是比较喜欢这一次的任命,至少比让他继续留在关中要强不少。

能够在河北将战争的准备布置好,曰后就有机会成为攻辽的主帅,记得之前与辽国大战的时候,太上皇曾经下过诏,复幽燕者王。吕惠卿很想知道,一旦曰后他领军攻下燕京,那么这个王,朝廷到底是给,还是不给。

吕惠卿的嘴角翘了起来,轻声的笑了。

船头上看水势的船工回头看了一眼,立刻就又低下头去,不知是不是给吓的。

不过吕惠卿脸上的笑容很快就又收敛了。这其实是苦中作乐。如果能留在京城,他倒是心甘情愿的将河北的职位给章惇、韩冈,或是其他愿意镇守北方的人。

马上就要入京了,但他却无法在京中久留,还有比这个结果更让人怄气的吗?

如果换成是太上皇当政的情况倒还好,君臣多年,吕惠卿自问还是有机会打动他的,但女人那就没办法了,完全说不通。当初司马光输得那么惨,吕惠卿听说了详情之后,连幸灾乐祸的心思都只有一开始的那段时间,实在是莫名其妙。

但就此俯首认输,吕惠卿也不甘心。这件事迟一点再说吧,朝堂上不是没有变化。

章子厚真的会跟着蔡确?蔡确想要独相,章惇难道就打算在西府坐一辈子?吕惠卿不觉得章惇的野心会有那么小,他迟早要跟蔡确起冲突的。到时候,就有机会了。

唯一的问题,只在韩冈身上。就是势同水火的曾布,吕惠卿都不将他放在心上。

韩冈的敌视,有完全与私怨无关。吕惠卿也不觉得自己跟韩冈有什么扯不清的旧怨。但吕惠卿也清楚,只要自己还坚持新学,韩冈就绝不会答应自己回京。偏偏韩冈对太上皇后的影响力是最大的。

“道统之争啊。”

吕惠卿也不知是该哭还是该笑。在他帮助王安石撰写三经新义的时候,从来没有想过自己会因为这个理由而受到敌视和压制。

又不是春秋战国,百家争鸣的时候了。士林中的争锋还不够,还要带到朝堂上来。

突然间就没心思再看风景,转身就回到船舱中。

舱内角落处的一桶桶冰块,将暑热挡在了门外。顿时感觉就是不一样了。

婢女奉上了冰镇过的饮子,吕惠卿抿了一口,清凉的感觉从喉入胃,暑气一时尽散,但心头的疑惑却是散不开去。

当年看韩冈根本就不是这样会把治学当成毕生目标的人,怎么几年间就变得如此毅然决然?

就是之前韩冈与王安石为了道统闹得几乎反目,吕惠卿也不觉得韩冈与王安石会是一样的人。

可是从京城传来的消息上看,韩冈当真是为了气学将自己的前途赌上了。不论之后有多少变通的办法去回避赌约,但韩冈进位宰相的前路终究是比之前要收窄了许多。

做出这种赌约的韩冈,还能说是作伪吗?

吕惠卿想不通,真的想不通。
