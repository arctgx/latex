\section{第15章 自是功成藏剑履(三)}

天子的问题,只引来了殿中的一阵静默。

宰执们都低头看着手上的笏板,没有一人接口,贯彻着沉默是金的格言。

不是韩冈人缘好,而是自吕公著以下,多名执政在过去没将韩冈放在眼中时,或多或少都在他手上吃过亏。以两府重臣之尊,去针对一个新进,原本应该手到擒来的胜利,却每每被韩冈轻易翻转。吃一堑、长一智,眼下众宰辅中,曾经跟韩冈为敌过的几人,宁可让心急着踩人上位的监察御史们冲锋陷阵,也不愿公开表态,否则事情一个转折,丢人现眼的又将是自己。

殿中的静默令人尴尬,隔壁正殿宗室们的哭灵声清清楚楚的传了进来。赵顼一见得不到臣子的回音,脸色微沉,“吕卿家,你是西府之长,”

王珪尚在正殿中,唯一的宰相不在,赵顼便点起了执政中资格最老的枢密使。

赵顼的语气中带着冷意。韩冈在河东的行事,已经触犯了赵顼身为天子的忌讳。一路经略使,可以贪功好杀,可以为部将所挟,但不能明着愚弄朝廷。

韩冈有临机处断、便宜行事之权,但并不代表他能想做什么就做什么。要是韩冈能上一封密奏,说明情况,不论是什么理由,赵顼都不是不能体谅的。

区区两万黑山党项,又是多大的事?可是韩冈什么都没有做,只是命人飞捷入京。这是纯粹的态度问题,没将他这个皇帝放在眼里。而且韩冈的功劳已经高到不能不赏,赵顼正愁找不到理由挡着他晋身西府的机会。

天子的心思,殿中众人或多或少都感觉到了,这也正是他们沉默的理由。

吕公著在私下里直言无忌,但身在朝堂上,却不愿主动出头跟韩冈过不去:“回陛下的话。日前河东经略司上报官军于胜州大战南下黑山党项联军,斩首两万三千余级。枢密院已按旧日故事,遣人下胜州勘会。若其中数目不符,或当真有何过犯,自当回禀陛下依例行遣。”

王中正在天子身侧不远肃立着,听着枢密使吕公著一板一眼,述说着对河东军两万三千斩首的捷报如何处置。心道又是老狐狸一条。

身兼带御器械的名衔,刚刚回京的王中正他现在并不是以统帅的身份站在庆寿宫偏殿,而是一名护翼天子的宿卫。虽没有资格参与偏殿中的朝议,但在一旁看着韩冈成为御史们的众矢之的,而天子却不是直接驳回或留中,而是拿出来让辅臣们议论,王中正的心里也免不了有兔死狐悲的感伤。

王中正自知若是自己帮了韩冈说话,多半就会有人在天子面前进谗言了。但他是宫里面的老人了,知道如何说话才不犯天子的忌讳。要是这点本事都没有,能有如今这么大的名声?光是运气,如何能在天子面前得到这般信重?就是赵括、马谡,也要一副好口才,才能得人重用。

但到底要不要帮韩冈,或是帮到哪一步,是帮他脱罪,还是帮他缓颊,还得先看看官家的心意。要不然,让天子误会他与韩冈内外勾结,麻烦就大了。

王中正冷眼旁观,吕公著絮絮叨叨的说了一通,却都是应付故事,并没有直言要对韩冈下手。

赵顼耐着性子听吕公著说完,不置可否,转头看吕惠卿:“吕卿,你觉得当如何处置?”

天子若要治罪韩冈,吕惠卿并不反对。若能将朝廷的关注点从自己身上挪开,那还真是求之不得的一桩美事。不过他可不会为监察御史们的弹劾做背书:“以臣之见,西北一战,河东兵马功劳非小。如今虽有杀降冒功之嫌,但若是穷究治罪,非是优待功臣之法。军心一坏,日后如何再驱用其上阵杀敌?”

章惇眉头越皱越紧,吕惠卿的说法听起来总觉得不对劲。他避而不谈韩冈,看似是不想掺合,却又将河东军拿出来与韩冈拉上瓜葛,似有深意。

监察御史们的弹劾都在说着韩冈的错,但轻重有别。说韩冈贪功好杀,只是性格问题,与能力无关。而且杀降人,跟杀良冒功又是另外一码事。杀了两万黑山党项,也不至于深责,不过多在外留两年而已。但弹劾他为部将裹挟,那就是在攻击韩冈的能力问题了,一旦这个罪名坐实,别说晋身西府,就是再想做边臣都难。至于说韩冈故意拿军功收买河东军心,就更是会惹起天子的忌惮。监察御史还没有拿这个可以灭门的罪名来弹劾韩冈,吕惠卿却有了隐隐约约往这方面引的意思了。

对吕惠卿的话,赵顼还是没有表态,却又点起章惇:“章卿,你怎么看?”

“御史有风闻奏事之权,其论韩冈贪功嗜杀、为下将裹挟,并无错处。不过以御史片言,便问罪边臣,朝廷从无如此法度。此事当遣人至河东彻查,并下诏令韩冈自辩,以明是非对错。”

章惇摆明了支持韩冈。而他说得也是正论。就是过堂审案,人证物证俱全,也得给人犯开口自辩的机会。没有口供,如何能定罪?

赵顼当然知道章惇和韩冈交情好。只是没想到他这么干脆的站在韩冈一边。一旦事情变成了两边公开打嘴仗。就是原本对河东军斩杀降人而嫉恨的其他各路边臣,都要担心起日后会不会被御史援引此例,一封弹劾就会被治罪。很有可能会上本齐保韩冈,到时候,可就轮到如今弹劾韩冈的监察御史们被牺牲了。

赵顼心中不喜,怫然不悦:“若是他当真杀了来归顺的黑山党项又该如何处置?”

章惇正色回道:“陛下明察。记得之前辽人能够夺占兴灵,正是黑山威福军司的兵马引狼入室。陛下欲留其守边,异日辽人南侵,其未必不会倒戈相向。鞑虏蛮夷,岂知忠义?韩冈纵兵杀之,虽有小过,但以后事论,不为大错。焉知这一伙黑山逃人中日后不会出再出一个李继迁?”

赵顼一时默然。

大宋自开国以来,对武人都当贼防着,何况那些三姓家奴的黑山党项?莫说是黑山党项,西夏人的孑遗都杀光了,赵顼才能安心。但这话不能说出口,一旦说了,下面自认是仁人君子的臣子们都要骂上来。可谁能保证这批黑山党项中,不会出第二个李继迁?韩冈帮忙解决了让人头疼的问题,赵顼其实挺欣慰。

但韩冈这么做,也太讨武将们的欢喜。与那些想成为肉食者,却叫嚣着‘肉食者鄙,未能远谋’的低品官员不同,赵顼十分了解韩冈的能力,很清楚他绝不可能控制不了下面的武将。李宪的历历密奏中,也能隐约看得出韩冈对河东将佐们的掌控。这般得军心,如何不让人主忌惮?

而且以韩冈的功绩、能力,这一次西北战事终结之后,也只有西府中给他一个位置,才能说得过去。否则有功不赏,日后谁还会为朝廷卖命?

但那可是三十不到的西府执政啊……

赵顼一直以来压制韩冈的晋升,不正是不想看到这一幕吗?纵然让韩冈受了委屈,可为了大宋的长治久安,就不能开这个先例。韩冈在河东做得十分出色,军事政事都让人挑不出毛病,幸好出了胜州的一桩公案,让赵顼看到了机会。

韩冈迟早是要入两府的,但绝不是现在。在封赏上,赵顼绝不会吝啬,但官位上总要压上他一压。这也是为韩冈好,升的太快,后事当难以善终。

赵顼紧锁着眉头。吕公著说着场面话,吕惠卿顾左右而言他,章惇一力相助,至于其他几个没开口的,则是做了泥胎的佛像。

从他们的态度上可以看得出来,几名执政全都主张韩冈的罪名必须要先认定,之后才能治罪。可赵顼想得偏偏不是对韩冈明正典刑。从刑律上,只要还没有得到朝廷的应允,黑山党项就仍是敌国之人,韩冈杀之无罪。若朝廷当真成功的找到理由降罪韩冈,河东军也肯定要一并治罪,但这是赵顼竭力要避免的结果。

王珪不在,怎么就没一个能体贴上意的?

赵顼的视线在殿中臣僚的身上一个个划过,心情越来越坏,脸色更加阴沉。

“陛下,臣有一言欲进于陛下。”

突有一人出班说话,赵顼定睛一看,却是新近晋身政事堂的蔡确。

“蔡卿但说无妨。”

蔡确恭声道:“河东本有走马承受,又有李宪经制河东兵马,本有监察之权。其上报胜州一战,斩首两万,当不为虚。若仍有存疑,枢密院也已派人去查验真伪,不日便可知端的。”

赵顼皱着眉,不开口,看蔡确到底想说什么。

“以臣愚见。不论斩首是否来自于黑山逃人,都不宜深究。功疑惟重,罪疑惟轻。其人既非中国子民,陛前顺臣,杀之可谓之罪?”蔡确边说边偷眼看赵顼,见到天子脸色越来越差,话锋一转,“不过韩冈的确行事不谨,可由陛下内降密旨,严加申饬。想必韩冈能体会到陛下的用心良苦。”

