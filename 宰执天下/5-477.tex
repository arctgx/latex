\section{第39章 欲雨还晴咨明辅(六)}

高太后进福宁殿已经一刻钟了。

这一刻钟,对向皇后来说,仿佛过了一整年。

不知道丈夫和姑姑在里面是否在合谋对付自己,她想知道,却不敢向那边走过去。

“圣人!圣人!”

宋用臣大呼小叫的跑了回来,让向皇后精神一震。

禁中称呼皇帝是官家,皇后是圣人,太后则就是太后。太上皇后则不知道该怎么称呼,干脆就沿用。赵煦到娶亲的年纪,还有十几年,到那时候,肯定也不用担心称呼的问题了。

“韩枢密他们没过来,他们怎么说的?”不等宋用臣跪拜行礼,向皇后就急着问道。

“圣人不用着急,韩相公和章枢密、韩枢密他们都安心得很,韩相公听了之后,就回去睡觉了。韩枢密和章枢密甚至还找了棋盘去下棋。”

“是吗?那就好。”

宋用臣拿着韩冈、韩绛他们安慰了两句,向皇后明显的安心了下来,神色中也不再显得绝望。

“韩枢密他们究竟是怎么说的?”

宋用臣将韩冈的话对向皇后复述了一遍,想了想,又更进一步解释道:“韩枢密的意思,就是要圣人守住小官家,不论发生什么事,,一定要保住小官家。”

“吾知道了。”赵煦就坤宁宫中,想要将他从向皇后从走,绝不会一件容易的事。在向皇后有所提防的情况啊,甚至可以说比登天还难。

明白了什么是重点,向皇后不再急躁。“那韩枢密现在在做什么?”她问道。

“下棋。章枢密招了韩枢密一起下棋。”

向皇后心中一动,“赌了什么?”

“韩枢密问章枢密。赌注还是麦子吗?”

“嗯……”向皇后点点头,但皱起的眉头,却说明她根本就没想明白。“章惇说什么?”她又问。

“章枢密则说,宫中法禁森严,没人敢赌。”宋用臣慢慢的说着,一个字也不敢说错,万一让皇后领会错了,事情可就不可收拾了。

“没人敢赌?”向皇后不再将眉头皱得死紧。

“章枢密正是这么说的。”

向皇后腰背直了,终于有了足够的底气:“去叫王中正进来!”

……………………“子厚兄说得太直白了。”

“玉昆你难道不是?”章惇反问,手上的棋子毫不犹豫的落下。

方才韩冈、章惇跟宋用臣说话,就差赤膊上阵了。就算因为纲常而必须隐晦的话,也说得尽量的简单易懂。这都是为了照顾皇后本身的问题。

“不说明白点,给误会了怎么办?”

“这倒是。”韩冈点头道,“就是写藏头诗,也有蠢到想不明白的。”

“倒是玉昆,王中正真的可信吗?”

“又不是让他去跟皇帝过不去,只是站出来说几句。”

章惇落子如飞:“那也要够胆子。”

“王中正不缺。至少缺得不多。”韩冈应了一手。

王中正的能力才干,自然没有传闻中的那么强,但胆子还是有那么一点的。

韩冈还记得当年从罗兀城一路退下来,不论王中正当时怎么想,留到最后才走终究是事实。而且王中正之所以能在宫中出头,是庆历八年的卫士之乱。

庆历八年,弥勒教徒所鼓动的宿卫之变中,是出身武家的曹太皇亲自率领内侍、宫女把叛贼给击败。那时候,王中正才十八,可他拿着弓箭,射中了好几名贼子。亲手捉住了最后的残匪。从此一路飞黄腾达。真要说起来,他还是仁宗和曹太后提拔起来的。

“他也只是要替皇后出来做一做不方便的事。其余的,自有我辈来解决。”韩冈说道。

高太后这么出来,是想趁着帝后不和的机会,想让宫中认为她和赵顼站在一边。或许还没有想那么深入,只是想撒一撒怨气,可她的行为,还是会造成宫中的误解。

有这样的误解,高太后甚至可以在宫中横着走。当初派去保慈宫的班直们,绝不会有人敢拦着她。

但只要皇后敢于站出来,高太后就只有败退的份。

赵煦在手中,朝堂群臣认定的新天子不离左右,又得禅位大诏上确认了处分国事的权力,由于冬至夜的事,高太后在超业内外的名声都不好,真要闹起来,怎么可能会输给一个半疯的老妇人?除了一个‘孝’字,高太后手中还有什么武器?

太上皇后的地位,是群臣共同承认的。宰辅们全数支持。

就算高太后有本事抱着小皇帝直接上朝,韩冈都能联合其余重臣,将她赶回宫中。

宫中虽大,也大不过江山。

太后虽贵,也压不住他们这些朝臣。

没臣子们的认同,太后也别想站住脚,垂帘听政的太上皇后反手就能将保慈宫给清理光。

现在没有去动太后,只是留一份颜面,若给脸不要脸,宰辅们可都不会留手。

就算是王安石,当真下起狠手,也不是没做过。直接骂散了围攻的宗室,现在冷然平淡的态度,在过去是根本看不见的。可骨子里,还是比谁都要倔强得多。其他人也都类似,只是程度问题。

而以韩冈和章惇的脾气,遇到类似的事,都是自己直接就上去了,哪里还耐烦派人去,自己留在后面听消息?

“王中正若不肯去做了呢?”章惇问道。

韩冈拿起了棋盘上属于自己的一个‘马’,然后落在了前方敌阵外的‘车’上。

“很简单,不是吗?”他说道。

章惇摇摇头,韩冈的做法的确直接。但之后,要收拾残局却很难。不过在现在的情况下,直接一点,凶狠一点,只有好处没有坏处。

韩冈将马留在了车的位置上,而将车提了起来。

“等等。”

章惇拦住韩冈的手。将被吃掉的车拿了回来,放回原处。

“马能走田字吗?”章惇没好气的道,“差点给你糊弄了。”

……………………高滔滔看着脸色木然的儿子。

就像中风失语,从赵顼的身上传染到高太后这里。

她越仔细看着儿子,对韩冈的恨意就越深。

全都是亲生骨肉,现在变成了这样,还不是韩冈哪个歼贼害的?

一个疯,一个瘫,一个躲在外地不敢回来。就只有一个女儿时常进宫来。

这就是被歼贼害了全家的结果。

如果没有韩冈,这家里岂不是全家都安安心心的在一起?想到这里,她的恨意更近了一层。

“回去了。”看着儿子许久,高太后最后说道,“过两天再来看。”

离开了福宁殿,正循着原路往回走的高太后,她脚步突然停了。

“王中正!”

从前面转过来的一行卫士,立于正中央,上下一身盔甲的不是哪位武将,而是号称内侍兵法第一的王中正。

“王中正叩见太上皇太后。”王中正单膝跪倒,拱手一礼就站了起来,“介胄不拜,请太上皇太后恕中正无礼。”

从王中正开始,连同所有的班直,都是全副武装。高太后的随行人员中,已经有人开始发抖了。

高太后没有发抖,至少表面上看不出来。她也知道王中正此时出现这里,绝不会是的不是那么的简单。

“不用行礼了。”高太后甚至还带了点笑,“官家当年就是看中王中正你能文能武,才将你派去关西。才几年,就升到观察使了。”

王中正低头道:“小人只是个阉宦,不论何时,都是天家仆奴。官家说什么,小人就勤勤谨谨的去做什么。官位高,是官家的赏赐,官位低,肯定是事情没做好。”

“你能这么想就好。”高太后强忍着,对王中正的看法,又跌倒了谷底,只是勉强道:“这么些年,你做得是不错。”

“庆历八年,从小人亲手捉住几个贼人开始,一晃三十多年过来了,小人现在都已经五十一了。远远比不上其他人。”王中正恭谨的说着。

“王中正,你到底来做什么的?!”高太后终于忍不住了。跟一个阉宦拉家常,让她感到屈辱无比。

“小人奉太上皇后谕旨,前来相送太上皇太后。”

“怎么。”高太后右手持着拐杖,重重顿了一下,廊道上木质的地板被砸得一声闷响。她尖声道:“哀家难道连看儿子都不行吗?是来阻哀家的?”

“禀太皇。”王中正不急不燥,“圣人说了,‘娘娘终于肯出来走动一下,新妇实在是为娘娘高兴。官家正病着,平曰里也闷得很。娘娘想何时来探视都可以。可以多陪官家说说话,为官家解解闷。’圣人还说,‘娘娘既然大安了,明曰起,新妇就带着小官家去慈寿宫晨昏定省,以全孝道’。”

高太后的动作定住了,一时也没了话可说。

随行的从人,都是大气也不敢出,只有风声清晰可闻。

笃笃的木杖落地声再次响起,高太后再也没有话,随着脚步一连串的去远了。只是节奏乱了,半途还弯着腰咳了起来。

周围的一干在福宁宫做事的内侍,原本还有些心神浮动的样子,现在则是一个个老老实实的站着。

王中正抬起头,感觉风向好像正了一点。

‘也不过如此。’

王中正对自己说道,忘了背后的一身汗。

‘也不过如此。’
