\section{第十章 千秋邈矣变新腔(12)}

“不对?”

韩冈微微一笑,觉得不对就对了。

他拟定的那几道题根本就不能算是策问,韩绛用看策问的眼光去看这几道题,当然感觉不对。

“韩冈拟定这几道题的时候,的确是有些想法放在里面。”韩冈坦然相承。

韩冈的回答,正在韩绛的意料之中。

让韩冈去拟定考题的时候,韩绛本来就是对他将气学观点融入题目之中,有着一定程度上的心理准备。

以如今气学、新学相争的局面,韩冈不可能不利用这一点。所以他和张璪才没有与韩冈争夺拟定考题的机会。

进士科省试、殿试的考题就是士林中的风向标。

其中决定贡生是否能金榜题名的省试更重要一点,也是接下来的三年里,天下士子都需要揣摩精研的方向。

而殿试的考题对朝臣的意义更大一点,这代表着天子目前最为关注的重点,也明示了近期的政治风向。

韩冈要体现气学的特点,将气学的影响力推广到天下,进士科的殿试考题,就是近期内他所能拿到的最好的机会。

不过韩绛没想到韩冈会这么出题,看到这几道题目,蹇周辅会甘心去洛阳吗?

“但这几道题的确不适合做策问啊。”韩绛叹道。

策问,是天子问政。是要考生们从朝廷的角度来看问题,然后向天子建言,该如何去处理策问中提出的问题。

熙宁三年议变法,熙宁六年问阙政,到了九年,殿试的内容就是跟北方与西北的敌人有关了。

说的是大局,议的是天下。

而韩冈所给出的题目,却是具体到地方上和京中百司内的政务,将之拿出来要求贡生们进行分析。触及到实务,却与过去的策问完全不是一个模式。

即便这只是初步尝试拟定新题,之后真正殿试中的考题肯定不会与现在一样,但也未免太过超乎想象了。

在太后无法亲自拟定题目的时候,由臣子们所进呈的考题,不应该过于别出心裁,甚至应该为了避免成为士论的焦点,为人所攻击,还必须平庸一点。韩冈却似乎完全没有理会到这些。

“相公,考试的作用在于简拔人才。韩冈之前之所以对蹇周辅等四人给黄裳的考题有所非议,便是因为这样的题目根本不可能选拔出作为边臣的人选。”

“玉昆,你这话的确没有错,但体例亦当注重。”

韩冈偏过头,用余光扫了下后面,其他的宰辅都十分明理的离得很远,让韩冈与韩绛可以没有太多顾忌的说话。

走了几步,他轻声问道:“殿试不一定只能是一道题吧?”

……………………

韩绛和韩冈似乎在议论些什么。

章敦在后面盯了几眼之后,多多少少也能猜到了一些。

除了殿试的考题,应该不会有别的需要在路上就讨论的问题了。

更重要的,是方才迎面来风时,章敦听到了前面几句断断续续的话,尽管很模糊,不过已经足够用来确认了。

韩冈想要借以发扬气学,他的目标不可能仅仅是殿试,之后的御试也肯定不会放手。

不知在他的主张下,会将殿试的考题变成什么样。

如果韩冈所出的试题不能服众,对前两天还坚持将蹇周辅等人赶出京城的韩冈来说,就是最大的讽刺,对气学,也同样是一个打击。

当然,要是太后自此之后,一直都坚持让韩冈出题,恐怕也没人愿意牺牲自己的前途,而与韩冈的考题顶撞到底。

章敦扬了扬眉,他发现,又走了几步之后,韩冈和韩绛的神色变得更加严肃了。

到底说到了什么?

对韩绛与韩冈的对话,章敦想多多少少的了解一些,不过这一回,不论风怎么吹,都没有迎面而来的风向,章敦也便没能再听到前方两人的议论。

……………………

“……当然。”

停了片看,韩绛点头,明白了韩冈的用意。

在熙宁三年之前,进士殿试要为诗、赋、论各一,更早一点,是只有诗赋各一,而没有论。熙宁三年的殿试上,才变为了纯粹的一道策问。如今从一题再变回两题,根本就不算什么大事——也许对考生们来说是大事,但在体例上,没有问题。

“但既然有两道题,便会分出轻重。”韩绛边走边说,“昔年殿试上需为诗赋论各一,其中以赋为重,论次之,诗最末。只要赋写得好,诗、论即便仅仅中平,也能拿状元。反过来说,赋中有过,就算诗、论再佳也无济于事。玉昆你的岳父,就是在赋中写了‘孺子其朋’,因而丢了状元。你岳父当时写的论与诗,事后流传出来,做了状元的杨审贤比不了,可一样没用。”

韩绛说着,笑了一下,“这个例子其实不太恰当,仁宗皇帝阅卷前,令岳还是被排在第一的。不过本朝多少状元,皆是以赋夺魁,远的有王文正曾的《有物混成赋》,稍近的也有章衡的《民监赋》,恐怕玉昆你也听过、读过,而以诗、论夺魁的,玉昆,记得几个?”

他说着,回头看了一下走在侧后半步的韩冈,就见韩冈默默地摇头。

韩绛莞尔一笑,能教训韩冈的时候当真不多,“玉昆,如果你想要多加一道题,这想法虽好,但恐怕就会变成诗赋论的翻版,考官与考生,都只重其一,另外一题,就不会太用心。”

“韩冈另有一点浅见,想供相公参考。”韩冈胸有成竹,解决这个问题对他来说实在太容易,“以韩冈所想,不如以百分为满分,两道试题或六四,或七三,又或各居其半,待判卷之后,将两道题的得分加起来就是,总分高者优胜。这样给进士排序也简单。”

后世考试,韩冈经历得太多,一个百分制,就能解决韩绛的问题。只要将他所出的题目,折合进总分中,看哪个考生敢于放弃其中一题?

就算是三七开,韩冈的考题只占三十分,但对于考生们来说,另一道题做得再好也只可能拿到七十分,少了这三十分,或是只拿到了十几分,就不可能争得过两道题都拿到九成分数的对手。

……………………

苏颂离开前面的章敦有着几步的距离。

一直以来,苏颂都没有打算缩短这个距离。

对他来说,成为知枢密院事,是一个意外。不过苏颂并不清楚,对韩冈来说,这算不算是一桩意外。

太后对韩冈的信重,长了眼睛就能看见。不过要说韩冈在太后面前,推荐自己成为知枢密院事,倒是不好说了。

这一次的晋升,名义上是有了变化,算是枢密院的两位主官之一,但在实际上,与之前做枢密副使的时候没有什么区别,依然比章敦要低上一点。

从这一点来说,韩冈应该不会做这样的事。除了俸禄等事上,苏颂的权柄和待遇与之前别无二致,以韩冈的性格,肯定不会这么做。韩冈现在也没有太多的时间,去为了一个虚衔而与太后商量。为此耗费太后的信重,更加不值得。

而且韩冈现在还有更重要的事情需要考虑。尽管最近并没有与韩冈商议过,不过之前韩冈乘机发落蹇周辅等四人,苏颂已经从中看到了一点苗头。

既然以考试题目不尽人意为由赶走了蹇周辅,那么以类似的理由,顺便插手进御试和殿试的考题中也就是顺理成章。

借助考题推广气学,韩冈会怎么做,苏颂也能想到。

不过他就是不清楚,韩冈想要在考题中体现的是张载的观点,还是韩冈他本人的见解。

苏颂对张载的观点,并不是完全赞同。而对韩冈的想法,苏颂却觉得合理处甚多。比起张载一系列的著作,韩冈对自然的研究,更合他的胃口一点。

望着前方,稍稍分了前后的韩绛与韩冈,苏颂觉得韩冈应该不会犯错。

……………………

“……还是玉昆你想得周到。”沉默的走了一阵,韩绛点头,他已经想通了韩冈的提议,的确有其效力,“不过玉昆你另外打算出什么题?”

韩冈已经出了不像是策问的策问,难道还能再出一个论不成?倒还真不是没有可能,韩绛想着。

气学之中也有对经义的诠释,通过论来对这些诠释进行传播、发扬,不失为一个很有想法的主意。

【第二更】

只是韩冈如果当真这么去做了,大概就能看见他的岳父、自己的老朋友大发雷霆的模样了。

这样的改变,比起之前清理了蹇周辅等四名考官,更能挑动王安石的脾气。

韩绛都可以想象那时候的场面,只要韩冈真的这么去做了,结果多半会如此。

但韩冈却是摇头,“韩冈想要出的题目,就是这一类。至于策问,就请相公多费心了。”

韩绛闻言,脚步一顿,惊诧的看着韩冈,“玉昆?……”

“策问当然得另出。韩冈一开始就这么在想。”韩冈低声对韩绛道,“方才相公也说了,这几道题不能算是策问。”

“不是策问,又是什么?”韩绛不解的问道。

“子曰,申而论之……申论!”
