\section{第45章 成事百千扰(中)}

过了年,假期也结束了,不过人心还是散着——毕竟上元节还在前头。

韩冈回到军器监的时候,衙门中的气氛也是懒懒散散。而新年当头的第一件事,就是军器监丞白彰站出来说,今年监里的灯山在过年的时候不知怎么坏了,要赶紧修好,不然赶不上上元灯会,可是会在天子面前丢脸的。

‘哦,这可真是不得了!’韩冈将讽刺的话埋在肚子里,如今的风俗如此,他也无意顶着来。

元旦的热闹只在家中,上元节时的热闹却在街巷上。地方上的州县都是放灯三日,而京城则是放灯五日,从正月十四一直持续到正月十八。府院监司、皇亲贵胄,甚至各家行会,几乎都要扎彩灯、造灯山。

这些彩灯、灯山,从冬至之后就开始打造。到了腊月十五,便有许多家彩灯放到景龙门‘预赏’。不过府、院、监、司各衙门的灯山,则是上元节时方才亮相。

谁家的灯山在亮相时博得喝彩最多,谁家的面子上就有光彩,若能得到天子垂顾,那就更是不得了,一年之中都是个荣耀。真正要等到节日的气氛过去,那是要到正月十八之后。

曾孝宽今天人在枢密院,并没有来监中,韩冈也无意等待他的意见,直接问道:“灯山之事,监中由谁人主持?”

白彰恭声道:“正是下官。”

想想也是,若不是白彰主持,他也不会主动站出来禀报。韩冈道:“即是如此,那就由你全权负责。监中人事,你比我要熟悉,人手由你来点选。必要时可以日夜赶工,多出的花销则从公使钱账上走。”

“……”白彰犹豫了一下,并没有立刻应承。

“难道修补灯山要用到多少人手不成?”

白彰道:“灯山下官已经去看过,整个都垮了下来。新造反而比起修补还要容易一些。”

韩冈心头微感不耐:“那就新造!方才也说了,由你全权负责,我和曾都承只要在正月十四见到监中的灯山摆在御街上。”

白彰拱手接了命。

把灯山的事做了决定,将这个不着调的任务推回给了下属,摆在韩冈手上的还有监中一个年假积攒下来的诸多公务亟待处理。

军器监中的属吏并没有给韩冈玩什么花样,递上来的卷宗和文案,都是分好了类别,并将建议贴在了文案上,以供他参考。

许多衙中胥吏,为了给新任的上官一个下马威。往往都会将大量的公务部分门类的一起堆上来,让上官批不胜批,最后知难而退。韩冈本也有了心理准备,但军器监的属吏却是老老实实的找着规矩来——不知道这是不是吕惠卿和曾孝宽释放的善意。

不过事情毕竟不少,等到韩冈将手上的公务都处理完,已经是下午了。幸好并不是天天如此,要不然吕惠卿和曾孝宽也不可能将三四个、五六个,多的时候甚至有十几个兼差都给背到身上。

喝了杯热茶,歇了一阵,韩冈将门外听候使唤的小吏叫了进来:“去把金作和炉作的作头都找来。”

听了韩冈的吩咐,小吏忙跑了出去。不一会儿,大小金作、大小炉作,军器监中负责锻钢冶铁、打造铁质零件的四个作坊的作头都被找了过来。

这些作头都算是官员,身上带着的是武职,穿着一身青色官袍。不过有的已经入流,从九品、正九品都有,有的则是尚无品级的流外官——他们是被特许穿了青袍。基本上都是熬年资熬上来的,各个都有四五十岁,从外表上看,也都是工匠模样,与身上的官服一点不配。

等他们行了礼,各自坐下,韩冈开门见山的道:“想必诸位都听说了本官打算做什么了吧?”

一众点头回应,齐刷刷的回答:“下官明白。请舍人尽管吩咐。舍人说什么,下官们就做什么。”

韩冈打算打造铁船的消息已经在京中传播开,但军器监中的官吏都知道,至少在过年前,这位新上任的判军器监并没有动静,想来到了年后,肯定就要调集人马开工了。

“要造铁船,第一个就是要有上等好铁,必须要坚韧,易于弯折打造,能受风浪冲击。这是摆在头里的第一件事,所以本官想要问一下,炉作和金作能不能提供合适的铁件。”

坐在这里的四位都是真正专家,韩冈要想打造铁船,第一步就要听取他们的意见。

说句实在话,官员中不学无术的有之,只知道吟诗作对的有之,但的的确确也有许多出类拔萃的人才,而更多的官员,尤其是参与实务的底层官僚,对于手上的工作熟悉和精通程度都远远超乎后人的想象。道理也很简单,若尽是些无用之辈,如何治理一个偌大的国家?

大炉作的作头臧樟,就是这样的专家。他已经有六十岁了,在军器监五十一作中,名望不低,在四人中也最敢说话:“若说好铁,那就不能用石炭炼的北铁了。北方冶铁用石炭,南方用木炭,而蜀中用竹炭。石炭炼出的铁性多脆,南方和蜀中的铁便坚实许多。现在监中用铁,多从徐州来,斩马刀若是换作河北铁,斩不了几人就会坏了。也就铁鞭、马镫可用北铁。如果要造铁船,肯定要用徐州铁。”

军器监多用徐州铁的事,韩冈知道。铁矿石一般在矿场直接冶炼,矿石锻炼成铁后,再将生铁锭送入京中。徐州的利国监,有三十六冶,从事冶炼的工匠总数多达四五千人,而矿户更是有数万之多,乃是北方铁业的重镇。但徐州此时并没有发现煤矿,所以只能靠着木炭来冶炼。

不过用煤炭就炼不出好铁,韩冈就不知道到底是为什么了。

好罢,其实韩冈对于钢铁工业的认识,仅仅局限于高炉炼铁,平炉炼钢,炉渣可以废物利用,这些教科书中出现过的常识。仅此而已,对个中技术完全是一窍不通。高炉、平炉的模样都记不太清了。

对了,炼铁的原料是铁矿石和石灰石还有焦炭,这也是教科书中插图的功劳。另外他还知道,焦炭是煤炭干馏后的产品,副产品则是煤焦油和煤气。至于其他,真的是一头雾水了。只能靠着这个时代已经出现的技术和专家。

“是不是石炭的产地不合适?”韩冈问道。

“并不是北方水土不合。论铁性,契丹镔铁为最上。下官记得不知庆历还是皇佑年间,也就是仁宗皇帝还在位的时候,北使贺正旦的礼物中就有镔铁。”小炉作的作头谭运答道:“只是五行金木水火土,要锻铁炼铁,五行都不能缺。可石炭炼出的生铁,却是五行缺木,故而少了韧性。”

韩冈暗暗的摇了摇头。这个理由肯定有问题。他当年述说医治骨折伤时,就拿着五行之说作为论据,如今都已经被写入了太医局的医书。想不到眼下,炼铁的事上也跟五行掺合上了。

小金作作头紧跟着:“若说石炭,如今北方人家家中,绝不下于柴薪的使用。下官记得关中用得也很多,就如延州【延安】,寻常人家几乎都不用柴草了。”

韩冈对延州记忆犹新,当年他可是被王安石和韩绛逼着去了那里。对延州堪比后世的空气质量更是记忆深刻:“沙堆套里三条路,石炭烟中两座城。延州人的确都是用着石炭。”

“开封也是一样。”谭运接口道:“开封用得起木炭的尽为富贵之门,宫里更是多用不生烟的贡炭。不过寻常人家用的就都是怀州【今河南沁阳、焦作】九鼎渡运来的石炭了,就是因为便宜啊!”

九鼎渡是开封附近最大的一个煤炭交易和转运场所,河东【山西】的煤炭开采出来之后,穿过太行陉运抵怀州,再从九鼎渡由汴河水运进京城。

“如今河东、河北的多少富户都靠着石炭营生……”臧樟转头对着一直没有做声的大金作作头李泉,“李小乙,现在管着河南第九石炭场的,就是你的内弟吧?”

李泉点了点头,简短的回了一个字:“是。”

这两位说的河南,不是黄河之南的河南,而是汴河之南的河南。在开封城外,沿着汴河和五丈河,有河南第一到第十石炭场,河北第一至第十石炭场,还有京西、丰济等石炭场。

这些石炭场中,煤炭堆成了山,每天京城百万军民消耗的煤炭多达数十万斤,全都是从石炭场运进京城。住在城西的韩冈只要出门离了坊门,如今天天都能见到运煤进京的雪橇车,在汴水河道长长的拖出了一串。

不过今天讨论的可是铁,而不是石炭。话已经说偏了,韩冈将话题拉了回来,“”

“如果换成铸造如何?”“明道年间,宝相禅院铸铁佛,千手千眼。那可是一次铸成,手、眼无一缺失。就是李小乙他老子亲自监造的。”

“不行。”韩冈立刻摇头,“那样的铁船只能在水上漂的玩具。真正的船只,都从龙骨、船肋再到外壳,都是分部组合而成。不过龙骨和船肋,可以试试铸造,最好能用上钢而不是铁。”

“这可就难了。”臧樟皱着斑白的双眉,“如今的钢多出于磁州——团钢,也叫灌钢。用来打造斩马刀的就是磁州钢。可即便是斩马刀也不能都用钢来打造,千百钢刀倒也罢了,可一年就是二十万柄,完全用不起!只能在夹在刃上。”

