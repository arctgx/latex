\section{第31章 战鼓将擂缘败至(七)}

种谔的话差点让王中正给跳起来,连士大夫风度摆得让现在的韩绛也要自认不如的赵瞻,也不禁挑了一下眉毛。

“罗兀城竟然赢了?!”王中正尖声惊叫道。

而赵瞻也在同时问着:“围困罗兀城的西贼退军了?”

“西贼来攻的共有八万人马。”种谔辩解了一下,又连忙补充道,“但只要不能上阵,人马再多也只是累赘而已。”

种谔话说得很急,担心一句话说的磕绊,会给来自京中的两位使臣留下什么不好的印象,“西夏都枢密是被八牛弩射死,西贼士气已然大衰,一如澶州城下的契丹人。只要能再坚持一段时间,西贼那里必然要退兵,八万大军人吃马嚼,党项人的老本都快要给吃光了,如何还能支持得下去?”

种谔把前景描绘得很好,王中正听得都有些心动,但赵瞻的脸色却是依然冷淡。

“斩了一个都枢密又如何,等斩了梁乙埋再说吧!至于西贼士气大衰……”赵瞻不屑的冷笑了一声,“不知种谔你有没有看到关中也是军心大乱?你能不能保证其中不会有第二个吴逵?”

“末将可用全家性命担保!”

赵瞻冷哼一声,:“若真的有了兵变,你全家的性命就能敌得过吗?”

种谔勃然大怒,咬紧牙双手都在抖着,要不是赵瞻奉旨而来,他论地位可是在赵瞻之上!就算文武殊途,也轮不到一介开封判官这般无礼!

赵瞻却对种谔的愤怒全然不放在心上,长身而起,从身旁被派来护卫的班直手中,接过他和王中正今次所奉的圣旨。转过身,面南背北的大声说道:“种谔接旨!”

种谔脸色瞬变,踌躇了一下,终究还是低头跪了下去,但双手却是紧紧扣着地面,手背青筋迸起,指甲崩起开裂,血水从指尖丝丝流出,他却是毫无所觉。

把一封下令撤军的诏令,以嘲讽和讥笑作为伴奏,抑扬顿挫的念了出来,赵瞻最后把圣旨一卷,递到种谔的头上,“种谔,接旨吧!”

种谔没有动,他抬起头,侧过脸,望着韩绛,眼神中尽是企盼。但韩绛却是挪开了视线。

因为广锐军的关系,整个西军在朝廷的眼中,现在怕是已经变成了兵变的预备队了。种谔用全家性命来保证在日后数年,陕西再没有一次兵变。但韩绛做不到,尤其是在吴逵打出了诛杀王文谅的口号后,他更是没有了自信。

外患和内忧,哪个威胁更大,天子和朝臣们的观点,韩绛都很清楚,他无法硬顶,虽然他现在还是首相,依然有着便宜行事的权力,但如果他顶了今次的圣旨,下面就是立刻会有人来接替他的位置——天子能容许失败,但不会容许桀骜不驯,韩绛别无选择!

“退兵吧……”韩绛无奈的对种谔叹着。

韩绛退却了,赵瞻立刻得意的又一次高声厉叫,“种谔!接旨!”

谋划多年,历经艰辛,眼看成功在即,终究还是功亏一篑。种谔心丧若死,眼神中也失去了神采。渗着血的双手高举过头,接过了轻如鸿毛,却沉重得一下压垮了他数年心血的圣旨:“臣……遵旨!”

猩红的血液染红了圣旨背面的五色绫纸,赵瞻冷冷然的笑了一声,对种谔的痛苦甚是快意,“好了,下面该想想如何把罗兀城中的那一万多人给召回来!”

……………………

“终于还是来了。”

虽然这份命令,是罗兀城中的每一位官员将校都不想看到的,但当他们当真收到弃守罗兀的命令之后,也没有一人感到惊讶。只有无奈的沉默,和心血付之流水的颓然。

唯独韩冈缺乏这样的心境,他一直都认为罗兀城守不住,虽然已经给了西贼足够的教训——其中有自己的一份功劳——但结果终究没有变。在沉默的主帐中,他压低声线,对身边的种朴说道,“发现没有,这几天西贼的包围变得宽松了许多。”

“多半是被打怕了!一个都枢密啊!”种朴轻声的哈哈笑了两声,却看到韩冈板起的脸上并没有一点笑意,便笑不下去了。正色道:“玉昆你的意思是说,西贼已经收到了庆州兵变的消息,正等着我们离开?”

“还能有别的可能吗?”韩冈反问着。这样的推理是一条线下来的,明摆着的事实,“他们正盼着我们离开,好趁机缀上来,把我们追杀百里。”

“玉昆说得没错。西贼当是这么想的。”张玉点着头,表示同意韩冈的看法。

因为前日顺利把伤病一起送走,又有了如此辉煌的战功,韩冈的发言权因而大增,渐渐有了首席谋士的架势。现在罗兀城中的大小事务,无论高永能还是张玉,都要先征求一下他的意见。

“该把永乐川城里的守军收回来了!”韩冈提议着。

“罗兀城的撤军已经是明摆着的事了,党项人那边对永乐川寨的先一步撤防肯定是求之不得,完全可以光明正大的把寨中的守军撤出来。当然,必须要派兵出城接应,不然梁乙埋想必也不会介意在正餐前先吃点开胃的汤水。”

“不过这样要放弃罗兀的计划就瞒不住了。城中士兵如今因为前日的大捷而士气正旺,要是听说了朝廷要放弃罗兀,军心恐怕会不稳。”张玉看了看高永能,一齐点头,对韩冈说道:“玉昆,这事就交给你了。”

‘要兼任心理医生吗?’韩冈心中自嘲着,他的兼职是越来越多了。除了医生、护士之外,又多了心理医生的工作。不过他对自己在罗兀军中的声望还是颇有自信,要让他安抚军心,也不是什么难事。

“所谓疑心生暗鬼,越是隐瞒,情况可能就会越糟。以下官的想法,要趁此机会,把所有的事明明白白的和盘托出。据下官所知,将士们多是通情达理之辈,只要能开诚布公,相信他们都能体谅。”

“……是不是太过火了一点,没必要解释那么多。”高永能犹豫着。

“迟早要公布的,还不如从我们嘴里说出来。”韩冈很坚持。

把人当人看,这是他一直以来的观点。他可不会学着此时的官员,把‘愚氓’二字挂在嘴边。韩冈一向认为,这世上是聪明人居多。迟早会戳穿的谎言就不要说,转眼就会瞒不过去的真相,当是要主动爆出来。如今罗兀城中,底下的士卒对上层将校还是很信任的。主动说出不利的消息,能够加强这种信任。但若是东瞒西瞒,反而会把这层信赖关系给破坏掉。“有了被送走的伤病,所有人都该知道。我们不会抛弃一个,也不会放弃一个!”

韩冈的建议被采纳了。当做出了决定,行动便是很快。第二天,得到了命令的永乐川寨,就开始了撤往罗兀的行动,而为了接应他们,城中守军也出动了大半,保护着永乐川寨和罗兀城之间的通道。

“相公,宋人要退兵了!”一名党项将领兴奋的冲进了梁乙埋的大帐中,报着喜信。

但梁乙埋没有动,依然安坐着,慢慢的品着酒。他已经收到了消息,宋人仅仅是撤出了永乐川寨。而随着寨中近两千名守军的离开,寨子中冒起浓烟,接着火光也蔓延上了城寨最高处。墙倒屋塌,刚刚修好的城寨,就在火焰中走完了短暂的一生。

听说了宋人放火烧寨,梁乙埋终于走出了大帐。

“要不要打?”那名将领又问着。

梁乙埋远远的望了过去。在两里之外,为了迎接撤出永乐川寨的友军,罗兀城守军所摆下的阵型,那分明是要决战的态势。西夏国相又回头看了看跟在身后的诸多将领,毫不犹豫的摇了摇头,“不,让他们过去吧。”

现在的情况,让梁乙埋很难使动那些豪族的族长们。他们都是想等着捡便宜,怎么会在轻松拿到胜利的时候,改去与宋人硬拼?而梁乙埋现在也不想再消耗手上听命于己的军力,已经损失了许多,再伤下去,他可就自身难保了。

梁乙埋和党项族长们不肯火中取栗的想法,让守卫了永乐川寨多日的近两千将士,最终无惊无险的顺利撤到了罗兀城中。

把突然多出来的两千人安顿完毕,种朴在主帐中找到了正在埋首于沙盘之中的韩冈。“下面该撤军了吧?”

韩冈从无定河谷中收回了注意力,道:“不,先拖上几天。”

“为什么?!”

“当然是为了拖时间,磨一磨党项人的耐性。……出其不意嘛!”

“把永乐川城的守军撤出来是为了给西贼一个希望,让他们幻想着我们会立刻撤离,而安心的等待下去。但我何时说过立刻要走了?以城中的情况再等上五六天也行啊!”韩冈得意的笑着,“别说拖上五六天,就是十天半个月都没问题。在被西贼重重围困的时候,本就不可能说撤就撤。就算延州那里不断派人催着,也完全有充足的理由。熬下去就是了!熬到西夏人先退,我们才能安全的撤离!”

“但庆州叛军……”

“广锐军叛乱的事与我们有半点关系吗?!乱不到鄜延来!”韩冈的声音冷澈,“我们考虑自己的事就够了!”

