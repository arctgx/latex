\section{第28章 临乱心难齐(六)}

【不好意思,有事外出,迟了一点】

妹妹回门,还带着名义上的外甥和外甥女过来。这下,王雱倒有借口请假了。

“从开封往白马县,快马只要半天,现在走的话,入夜时就能见到玉昆。”

王安石想了一想,点点头,嘱咐道:“要先回府一趟,然后再出城去。”

“此等事孩儿当然明白!”

事态紧急,拖延不得,王雱随即辞过父亲,转身离开中书回家去。

自家妹婿的发明,还要从外人的口中得知,王雱心中不免有些后悔。明明知道韩冈多有发明创见,前几天应该去信问上一问,现在已经将开河口和碓冰船的奏章递了上去,还设法得到了天子的许可,弄得自己十分尴尬。另外心中也怪韩冈过去聊天时怎么没提上一句,不然也不至于眼下手忙脚乱。

能在冬天大雪封道的情况下,还能上路运输的车辆,竟然没有拿出来请功。不知是因为单纯的忘了,还是因为韩冈当初积压了多少功劳,却没有得到封赏,所以心思给淡了去。

陆行乘车,水行乘船,泥行乘橇,山行乘檋,大禹治水时,踏遍千山万水时所用的各色车驾,乃是视地形而定。沼地松软,车马易陷,而雪地也是一般,所以名为橇。韩冈给自己的发明所起的名字,望文即可生义。

在雪地上行驶的马车,又有了实际的使用经验,王雱怎么能坐得住。如果雪橇车当真能在汴河中派上用场,前面侯叔献所设计的碓冰船就不需要拿出来冒险了——那种东西,只是死马当活马医,看着就知道用处不会太大

‘不!’王雱念头一转……‘用处可是大得很!’

……………………

时隔半年多,王旖重新回到家中,还有韩冈的三个妾室以及一儿一女。

望着理应十分熟悉,却不知为何已经变得陌生起来的府邸,王旖的脚步变得慢了起来。出嫁之后,就是夫家的人,少女时在此度过的几年时光,现在虽然还是记忆犹新,但却像是几十年的事了。

跟在她身后的韩云娘则和素心一样,进了相府之中,就有些胆怯。低着头,脚步亦步亦趋的,不敢稍有错乱。她们是韩冈的妾室,普通的官员没什么好怕的。但轮到高高在上的宰相,只能在传说中听见的名字,就是感到一阵心虚。

倒是周南,当年在教坊司中见得达官贵人多了,神情平静如常。但三女之中,就属周南最不想来到东京城。除了韩冈之外,东京城的留给她的回忆,并没有多少值得留恋的。

御赐的宰相府邸很大,连着过了几道门,终于到了后院的花厅前。

吴氏正急着站在厅门口,和两名儿媳妇一起等着,只恨房子太大,不能让她第一眼看到女儿。终于等到披着猩红色斗篷的女儿绕过照壁,也不等她行礼,就一把拉过来抱住,心儿肉儿的叫着。

被母亲抱着怀里,王旖也忍不住眼中泪水直流。直到这时候,过去的感觉才又从心中回复。母女俩抱头痛哭了一阵,又跟王雱的妻子萧氏,王旁的妻子庞氏见过礼,王旖这才让过身子,将云娘她们介绍给吴氏和两名嫂子。

吴氏对女婿的妾室,并没有敌视的感觉,但也不可能亲近。很是疏淡的说了两句场面话后,就让下人们带着她们下去休息。倒是韩冈的一对庶生儿女,继承了父母的相貌,长得极是讨人喜欢,吴氏见着就抱了好一阵,也想着如果是自家的亲外孙和外孙女那就更好了。

等着两名儿媳妇识趣的找借口离开,吴氏拉着手和女儿一起坐下。问着她在夫家过的到底怎么样,舅姑待她如何,到底习不习惯关西的水土。听说在陇西过得很好,亲家那边也很是看顾,吴氏的一颗心方才略略放下了一些。

只是对韩冈这个刚刚结亲不久,就丢下妻儿跑出来做官的女婿,吴氏还是有些不满,“他们男人都是这样,为了做个官,甩手就丢下家里不管……”声音中,还带着几十年的怨气。

王旖拉着吴氏的手摇着,帮夫婿辩解:“这不是将女儿接来了嘛。”

“单是接来可没用。”吴氏慈爱看着女儿,二十一二了,还如少女一般娇憨。轻叹了一口气,“早点生个一儿半女出来,娘这边也就放心了。你看看那几个妾,都是惹人爱的,又都有了儿女,你虽然三从四德要守着,但也不要谦让得太过了,该争得也要争。”

王旖知道吴氏在说什么,红了脸:“娘啊,这些女儿知道。”

“你就是会说!真能做到就好了。”吴氏正说着,王雱的声音就在院外响了起来,“二姐可是到了!?”

“大哥回来啦!”王旖起身,向着大步进挺来的王雱福了一福,起身后对着王雱看了一阵,眨了眨眼睛:“大哥好像又瘦了些。”

“公事嘛,免不了要累着一点,”王雱匆匆的对妹妹道,“既然回来就,就多留两日陪陪娘。娘可是天天念叨着你。”

王旖眼睛红了起来。王雱则又对吴氏道:“娘,孩儿现在有事要急着出城去,等二哥儿回来,你跟他说一下,明天就留在家里。”

“怎么这时候要出去?”吴氏不高兴的问着。

“公事要紧!明天晚上就回来。这事爹爹知道!”王雱也不多解释。换了衣服,就从旁门出了府,混在一队自家的家丁中,出东京城,往白马县去了。

……………………

韩冈现在头疼的事越来越多,民生艰难,让他不能不操心——做亲民官的苦,就苦在这里。

白马县中的粮价开始涨了。虽然这个涨价是在预料之中,但幅度却超出了预计,韩冈让人打听了,那是因为东京城中粮价上涨的缘故。

比起往年冬天时的粮食价格,现如今的粮价高出了近倍,比起正常年景青黄不接时的价格还要高上一些去,而且还有继续上涨的可能。白天出城时,城门边的粮店中,米袋上的牌子还写着一百零五文一斗,回来后就是一百二十了。

熙宁之前,也就是仁宗末年和英宗朝时,粮价通常是一斗六十到七十文——这是以常见的十斗进磨、八斗而出的粗粮来计——到了熙宁之后,新法的推行并没有如旧党所言使得民不聊生,但也没有让粮价降低多少,还是保持着六七十文的水准。这个价格已经保持了有近二十年,一下波动得如此剧烈,百姓们当然难以接受。

这个粮价不可能不影响到百姓们的生活。本来因为旱灾,而使得百姓们不敢花钱。可现在粮价飞涨,省下来的钱却都要投进购粮中去。白马镇和县城总计有千户人家,在户籍上属于坊廓户,粮食基本都是靠着外购,不比农民可以自给自足。尤其是年关将近,高涨上去的粮价,必然会带动所有的生活必需品的上涨,这个年谁也别想过好。

韩冈对此并没有什么能立竿见影的招数,不仅是巧妇难为无米之炊。白马县离着东京城太近,他这边压粮价一点用处都没有。就算他韩冈绞尽脑汁成功将粮食价格压低,只要东京城过来几个粮商,或是传来两句谣言,粮食转眼就能涨回去。

因为灾情而导致的人心慌乱,整个京畿地区粮价都在飞涨,韩冈不会奢望自己治下能有成为置身事外的孤岛,但并不代表他不会想办法解决这个问题。

‘要给京城写信了。’韩冈想着。有着做宰相的岳父,当然要派上用场。想来王安石现在也是头疼。比起只关注白马一县的自己,放眼天下的宰相,要负责和关心的可是要多上许多。

从眼下粮食飞涨的势头上看,背后有人操纵,自然是肯定的。源头就是在东京城中,如果京城那边能将粮食降下来,白马县这里也会应声而落。

如果灾情延续到明年开春,粮食价高价低已经无关紧要,买得起的不需要买,买不起的还是买不起。而且朝廷那时必然要全力动员,从外路调粮进来打压粮价,并且开仓赈济百姓。但眼下,一系列策略都不可能实施,能帮百姓们省一点就是一点。

坐在书房中,韩冈斟酌着该怎么给王安石写信,同时又有什么办法帮着他将东京粮价给打压下去。正在思虑间,管家敲门进来,禀报道:“正言,东京的相公那里送信来了。说是主母和三个娘子,还有大郎、大娘都已经到了东京。”

“哦!?”韩冈闻言心中大喜,消息终于来了,放下笔立刻道,“快让他进来!”

不一会儿,管家领着一人进来,然后悄声退了出去。

韩冈漫不经意的一看来人,却惊得一下站起:“元泽,你怎么来了?!”

王雱脱下满是灰土的斗篷,对韩冈叹着:“玉昆,你瞒得愚兄好苦啊!”

韩冈一头雾水,王雱莫名其妙的跑来,又莫名其妙的当头丢下这么一句话:“究竟是怎么回事?”他不解的问着。

