\section{第34章 彩杖飞鞭度春牛(下)}

时至今日,王厚已经不会再吃惊于韩冈头脑的敏锐,很干脆的点头:“两个都是。是半年多前政事堂发回来的堂扎,里面附了李经略的奏疏。李经略在奏疏中说秦州渭水两岸有无主闲田万顷,可供屯垦……”

半年多前,那不是李师中刚到秦州上任的时候?!从他的奏疏中看,很明显是要向朝廷申请屯田渭源、古渭,这根本是在为王韶的计划背书。韩冈惊道:“经略相公原本是支持机宜的?”

“李经略刚来的时候,本就是支持大人的,连向钤辖都没二话——哪人不喜欢功劳?只不过等大人兼了管勾蕃部之职,又有了专折之权后,便一夜风头转向。”

“难怪!”韩冈叹了一句。管勾蕃部原是向宝兼任;而专折之权,意味着王韶在必要时,可以绕过经略司而直接向天子递上奏章。一个被夺了权,一个无缘分功,当然不会再支持王韶,明里暗里的反对,也是理所当然。

“也难怪当初机宜要在渭源筑城时,李经略不明加反对,而是叹着没钱没粮,说是要挪用军资粮饷来资助机宜的计划!”

“是啊,当时还以为他不想惹怒王相公。现在一看,原来是这么回事!”王厚的心情很好,王韶无意中揭破了李师中的底细,成了推动计划的最佳助力。

只要王韶用同样的言辞将渭源、古渭的屯田之利奏报上去,难道李师中还能覆口否认不成?如果他反口,王韶便更有理由向天子申诉李师中对开拓河湟的干扰。而‘奏报反复’这个罪名,也足以让李师中滚蛋。

“对了,为什么这事没早发现?”韩冈心中起疑,若是早点发现此事,王韶早前根本不会陷入进退不得的窘境。

王厚尴尬的笑了起来,这当然是王韶自己问题,“当时大人正带着愚兄在各城寨探风,一个月也会不到秦州一两次,没有想起要去翻看堂扎和朝报。”

韩冈眉峰微皱。孙子都说过知己知彼百战不殆,来自千年后的韩冈,更明白信息有多么重要。情报就在身边,但不去研读,就跟没有一样。朝报、堂扎都是蕴含着大量情报,怎么能因为忙碌,而忘记翻看?!这的确是王韶的疏忽。

“对了,玉昆……你是不是要抢春牛?”王厚岔开话题,左顾右盼一番,忽然问道。

韩冈点了点头,这才是为什么他一大清早就往城外跑的原因。以他的性格,才不会无故凑这种无聊的热闹,“家严是叮嘱过小弟,要带上一块春泥回去。”

“那就难怪了!”王厚点着头,又道:“愚兄便不凑这个热闹了。玉昆你待会儿要小心一点,别被踩着了。不然明天可上不了马!”

“别被踩着了?”韩冈喃喃的重复了一句,他回头望了一眼身后狂热的人山人海,猛的一阵寒颤,忙扯着又要挤出人群的王厚和王舜臣,笑道:“有王兄弟在,还轮得到小弟出手?”

强留下了王舜臣,韩冈和王厚往人群外挤去。踩踏致死的新闻,韩冈前世没有少听说过,万一出了意外,当真是死不瞑目。而王舜臣的重心低,底盘稳,身手够好,长相又是凶恶非常,即便在蜂拥的人群中,也不用担心他会有任何危险。

当最后一名官员抽过鞭子,转身而回,锣鼓声便喧天而起。李师中领着官员,向后退出了近百步。他们这一退,场中的气氛顿时紧绷起来,千百人蓄势待发。

锣鼓敲响了一个变奏,人群中央,一颗绣球带着条红绸往向空中腾起,就像点燃了烟花的引线,哗的一片狂躁声响,震动全场。如山崩海啸,如巨浪狂潮,千里长堤被洪水击垮,人流山呼海应,奔涌而上。

韩冈看得暗自心惊,若他还在疯狂的人群中,说不准就会被推倒踩死,难怪王厚要他小心一点。看着他们疯狂的程度,甚至不逊于后世那些追捧韩星的歌迷们。如行军蚁掠过雨林,又如蝗虫途经田野,更似洪水扫过大地,眨眼的功夫,与真牛一般大小的春牛便不见踪影。

韩冈满腹抱怨,他的前身当真是钻在书堆里拔不出来的书蠹虫,有关抢春牛的记忆,竟然一点都没有。要不是王厚提醒了一句,没有半点心理准备的自己,别说抢春牛,能保住小命就不错了。

无数只手从破碎的春牛身上一把把的往怀里揣着泥土。没能抢到的后来者,直接便将主意打道了已经揣着春泥往回走的幸运儿身上,因此而厮打起来的不在少数。

一块土,承载着百姓们对丰收的渴望,也难怪他们如此疯狂。韩冈叹了口气,他老子千叮咛万嘱咐,要他弄一块土回去,据说对养蚕很有好处,还能治病。不过,他今次要让父母失望了。王舜臣身高太矮,他的身影早在人群一拥而上时便消失的无影无踪。看他这样子,保住自己也许不难,想要弄回春泥怕是没可能了。

不过韩冈今次却猜错了。

“三哥,你真是好带契!日他娘的,没想到疯成这样!”

好不容易挤出人群的王舜臣,浑身狼狈不堪,在韩冈面前大声的抱怨着。他上下的衣衫都已经破破烂烂,蓬头乱发,连帽子都不见了踪影。

韩冈赔着笑,觉得自己是有些过分了。但只见王舜臣往袖中一掏,竟然摸出来海碗大小的一块春泥来。

王厚大笑出声:“好你个王舜臣,竟然藏得这么大的一块出来。亏你本事!”

韩冈也惊了一下,赞着:“王兄弟当真本事!”

“这算什么?”王舜臣拍着胸脯,放声大笑,“俺在千军万马里都能杀个七进七出,何况抢个春牛?把冲锋陷阵的事交给俺,保管放一百个心!”

王舜臣的官位虽卑,尚未入流品,但已经可以带上一个指挥的兵力。王韶已经透露要让他先去甘谷城领兵,积攒下一点军功,等河湟开边的战争正式开始,便能及时派上用场。王舜臣现在也尽做着统领大军,践踏敌阵的美梦。

春牛抢尽,祭春仪式也到了终点,锣止鼓歇,人群遂纷纷散去,只留下了一地鸡毛,一片狼藉。而在春祭仪式结束后,府衙里还有惯例的宴席。

一队在仪式举行时充作仪卫的骑兵,护送着地位最高的李师中和窦舜卿回城,剩下的官员也是三五成群,交情好的走在一起,往南门走去。只有王韶几乎是孤零零的站着,唯独吴衍陪在旁边,看他们的样子,明显的已经被秦州官场给排斥了出去。

当然,其中有多少是畏惧李师中的威势,有多少是真心反感王韶,其实并不难判断。在官场上,表面上言谈甚欢、情谊非常,背地里捅刀子才是常态。没有利益之争,很少会有人把事情做得这般绝——而与王韶利益相冲的,惟有王韶在经略司中的几个顶头上司,除了李师中、向宝,便是刚来的窦舜卿了,连张守约都乐见王韶功成。

王厚看着自己老子如今的人缘,也不禁苦笑。王韶要升古渭为军,就是在跟李师中摊牌,州中官吏选边站也是理所当然。从眼下的局面看,王韶与李师中的第一阵算是惨败。

“多亏了玉昆你的计策啊……”

“计策?”韩冈一向很在乎自己的形象问题。他并不愿意给人留下满肚子阴谋诡计的印象,这对他日后的发展全无好处。韩冈很明白王韶对自己有些看法,他并不想加深留给王韶的心机深沉的印象,“别说得跟阴谋诡计一般。真要说谋略的话,也是阳谋,不是阴谋!”

“阳谋?”王厚没听过这个生僻的词汇。与阴谋相对的谋略,就叫做阳谋吗?

“不是在暗地里谋算他人的诡计,而是以煌煌之师临堂堂之阵,光明正大的策略,放在光天化日之下说出来也没问题的策略,便是阳谋。即便明着告诉李师中,我们要上书朝中,他又有什么办法?正如下棋,落子在明处,但照样能分出胜负。陷其于两难之地,逼对手不得不应子,这便是阳谋的使用之法。”

“阳谋?”王厚再次念着这个陌生的词汇,韩冈的解释使他有了一丝明悟。比起阴谋诡计,韩冈所提议的计策,的确光明正大。但也是一样咄咄逼人,让李师中无法应手。再回想起韩冈于军器库对付黄大瘤,于押运之路上对付陈举,于伏羌城对付向宝家奴,还有……利用伤病营对付自己的老子,每一件事都看不到任何阴谋的痕迹,而是坦坦荡荡的行事,这样的作派无人能挑出破绽来,却也照样一桩桩的遂了韩冈的心思。

不愧是韩玉昆!王厚只觉得他今天第一次真正看到了一名士子心中的风光霁月。韩冈的心智才情,还有人品,都让王厚敬佩万分。

有助力如此,王厚也不再担心他父亲在事业上的能否成功。当初下的一点本钱,如今已经收获到了累累硕果。

王厚扯着韩冈的袖子,“玉昆,你明天就要去东京了,愚兄已在惠丰楼为你订下了一桌饯行酒。今天我们兄弟一定要好好的喝个痛快!”

ps:争权夺利,昨天是兄弟,今天就是死敌,这是常有的事。

今天第一更,求红票,收藏。

