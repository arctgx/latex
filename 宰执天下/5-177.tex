\section{第20章 土中骨石千载迷(六)}

距离韩冈和苏颂在政事堂及天子面前揭开殷商古文已经过去两天了,整个朝堂都在议论纷纷。

身在京城朝堂之中,必要的政治嗅觉,绝大部分官员皆不会缺少。新学和气学的纷争,眼下愈演愈烈。如火如荼的局面,让看客们大呼过瘾,甚至进一步的在猜测着这一战下来的胜败结果。

厚生司和太医局中也有些骚动,不过不是针对新学和气学之间的纷争的。对于韩冈能通过相州龙骨上的刻痕,考据出殷商王都遗迹,这份才学,着实让人敬佩三分。且这件事,多多少少因为韩冈的缘故,带上了些许神秘色彩,使得世人口耳相传时,又多了几分动力。

太常寺中则是一片喜气洋洋。殷商国都之中,最珍贵不是龟甲骨头,而是祭器礼器。那些可都是陪着天子祭天,也不会让人觉得失色的贵重珍宝。

韩冈让殷墟成为世间的焦点,朝廷已经有了开掘殷墟故址的言论。在太常寺中官吏们的眼中,一旦朝廷决定发掘殷墟,凭韩冈判太常寺的身份,发掘出来的殷商国器,只可能存放到太常寺中,交由太常寺的官员来审查管理,而不是惯常的太常礼院。

如此一来,太常寺不但能控制了殷墟遗物,捎带着还能从太常礼院口中夺一块肉下来。能多分一块肉,逢年过节也能多点荤腥,这对于苦哈哈的太常寺中官吏们来说,已经是难得的善政了。什么样的上司最受下属欢迎?就是能做事,又能让手下人一并沾光的那一种。

不论在厚生司还是在太常寺,韩冈的威望一步步的升高,使得他这两天工作起来更加得心应手,越来越顺利。

只是韩冈编纂药典,这两天的进度却远不如预期。每天来访的客人络绎不绝,希望看一看殷墟甲骨的请托接连不断。闹得苏颂直皱眉头,但也是无可奈何。只能盼着早点放衙,回家后再继续工作。

可韩冈就算回到家里也一般的不得清净,总是有人厚着脸皮来登门拜访。一般的访客,韩冈还能将他们拒之门外。但换成了是韩冈家的亲戚,以及王安石的弟子,那就没有办法了。

王安国的女婿、韩冈同榜的叶涛,王安石得意门生的陆佃,他们两人联袂造访,虽然有居心叵测之嫌,但韩冈也不能将他们拒之门外,只能将他们请进了待客的小厅。

换了身见客的衣服,韩冈走进厅中,呵呵笑着:“劳农师、致远久候了。”

陆佃和叶涛正默默的喝着茶,互相之间,没有一句话的交流。见到韩冈终于出面,两人同时起身,向韩冈拱手行礼,“端明。”

“应该是玉昆才对。”韩冈摇摇头,更正道:“非是在官衙中,农师、致远不须拘礼,直呼韩冈之字便可。”

陆佃与叶涛对视了一眼,便同时拱手行礼:“如此,请玉昆恕陆佃(叶涛)失礼了。”

两人与韩冈年岁相仿佛,甚至还更年长一点,陆佃和叶涛都不愿意在韩冈面前伏低做小。韩冈的话,让他们倒是心情一松。

三人就着热茶寒暄了一阵,韩冈悠悠闲闲的与客人谈天说地,话题遍及八荒六合,就是不提有关殷墟甲骨的话题。

“玉昆……”叶涛终于忍不住了,“听闻玉昆近来得到了一批殷墟甲骨,上镌古时文字。可是有此事?”

“正是。”韩冈点点头。

“可否让我等一开眼界。”

“殷商甲骨,绝大多数都在太常寺中。不过眼下家中的确还有几片。”韩冈也不拒绝,直接让人去书房去了实样来。

两片普通的龟甲,因为上面的文字,而变得价值连城。

陆佃和叶涛拿着两片龟甲上上下下打量着。可以看得很明白,龟甲上的图案,似图又似字。仓颉造字,效法自然,如图如画。眼前的文字,的确是古字的的样子。且仔细看来,有几个古字是能跟今字对应起来的。

“这应该是‘山’字吧?”陆佃有几分没把握。

“是‘山’没错。”叶涛说得十分肯定。

韩冈翻检了几百片的甲骨,其中日月山水等浅近的古字,还是给他找出来了。而叶涛和陆佃两人,也不须费尽心神考证,仅仅是看了几眼之后,就辨认出了其中最简单的几个字来。

翻来覆去的看了一阵,陆佃和叶涛双手沉沉的同时将两片龟甲放下。

叶涛抬眼与韩冈对视起来,“殷墟甲骨一出,若其物为真,对《字说》乃是如虎添翼,还要多谢玉昆了。”

叶涛的话出人意料,似是准备抢占甲骨文的释义权,将之归入《字说》中。韩冈则直截了当的问道:

“哦?这番话,致远可是代表家岳在说?”

韩冈的问题毫不客气,一点余地都不给人留下,让陆佃和叶涛脸色微变。而韩冈对新学的恶意也在话中表露无疑,今天当不会那么容易讨了好去。

王安石在金陵,吕惠卿在长安,新学的两面旗帜都不在京城中,纯凭自己和叶涛想来对抗垂重名于世的韩冈,陆佃的心中暗恨,韩冈当真会选时间,让他全然没有半点把握。

“在下和致远才疏学浅,尚不敢确认是否乃是殷商时物,也不敢就此惊动介甫相公。”

“农师之言,正合韩冈心意,我亦是做如此想。甲骨文要确定是殷商时物还是得经过更严的考证,逛啊好i韩冈一句话,还是,殷墟中出土的证物是越多越好。所以以我之见,当召集天下群儒,共通探究殷墟遗物,并加以考订。”

天下群儒?陆佃眼皮一跳。韩冈的做法损人不利己,只会让儒林的混乱更上一层楼。

韩冈静静的等着陆佃和叶涛的回答。虽然两人起不了任何作用,但他们的存在,代表着国子监里的两千余名儒生。

儒林之中,新学可以说是举目皆敌,想要将新学掀下台面的儒生数不胜数。只要有一个破绽露出来,游走在圈外的群狼就会立刻扑上去。

这是统治思想的宿命,总要受到各方的挑战,一步也退让不得。不论是拖延还是装聋作哑,都动摇自己的根基。势力强大如西方的教会,不也是被逼得只能放火烧人来堵人嘴。新学处在眼下的位置上,可就是成了众矢之的,等人群起而攻之。

不过今日是文辩,敬酒不成,可就是要换成是罚酒了。之后百来年的朱熹,也是一样官司缠身。韩冈也做好了应对的准备,道统之争,无所不用其极,直如后世的意识形态之争,你死我活而已矣。

眼下儒林中的纷争,哪一家占据了官学的位置,便决定了国家政策的方向。在治政上,韩冈站在新法一边,若是新学被气学打垮,新法还不至于会被废除,换作是其他学派,连带着新法一并都会完蛋大吉。

“圣人有云:务民之义,敬鬼神而远之。商人终归不得圣人之意。考订得再详细也无济于事。”叶涛说道,将韩冈的咄咄逼人,躲了过去,“若是周时旧物那就好了。”

“此乃争胜之论,非是论道之言。殷墟甲骨乃是殷人占卜后的记录。夏商周的三代治国,圣人皆有言及,不独是商。事鬼敬神而远之;率民事神、先鬼后礼。这些可都是在说三代之事。”

韩冈说的,是敬鬼神而远之的另一个出处。

夏道尊命,事鬼敬神而远之,近人而忠焉,先禄而后威,先赏而后罚,亲而不尊,其民之敝,蠢而愚,乔而野,朴而不文。殷人尊神,率民以事神,先鬼而后礼,先罚而后赏,尊而不亲,其民之敝,荡而不静,胜而无耻。周人尊礼尚施,事鬼敬神而远之,近人而忠焉,其赏罚用爵列,亲而不尊,其民之敝,利而巧,文而不惭,贼而蔽。

此一段出自于《礼记·表记》,里面孔子对夏商周三代的特点和弊政一个都没放过。

商尊神重鬼而后礼,其民放.荡而不静,好胜而无耻,这是孔子看殷商不顺眼的地方。但夏和周也不是没问题,两代都是事鬼敬神而远之,近人忠人,可一个其民是蠢笨愚昧,骄傲粗野,朴鲁不文;另一个的民众则是好利巧诈,文过饰非而不知羞耻。

叶涛笑道:“玉昆前日直斥《月令》【礼记中一篇】中腐草化萤之非,今日也论《礼记》?”

“《礼记》出自多人之手。有合道之正论,亦有外道之异说。”对《六经》进退取舍,将不合己意的篇章指称为伪作,方今各家无不是,张载曾论《十翼》,十篇易传中是孔子亲笔所做的,就只有彖象四篇,其余皆是他人所做,韩冈是张载弟子,也是得心应手,“《礼记》之中,《大学》、《中庸》两篇,可是深得圣教要旨。”

陆佃都开始头疼,各家学派对那些不和己意的经典,都敢视为伪作,眼下,正好又是一个例子,“不过此事别无旁证,也不能就此断言吧?”

“如今已经有了殷墟。《礼记》诸篇真伪与否,在殷墟遗物中,说不定也能见分晓。”韩冈说道:“若能精研历历甲骨卜辞,说不定还可印证《易》中的卦辞和爻辞。为圣学之助。”

“玉昆难道忘了象、数有别!”叶涛仿佛抓到了什么,精神一振,“卜法、筮法截然不同,一用龟甲,一用蓍草,卜辞如何能与卦爻相证?!”

