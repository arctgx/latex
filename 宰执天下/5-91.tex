\section{第11章 城下马鸣谁与守(三)}

韩冈很欣赏折可适,所以对他的一点冒犯便不以为意。

韩冈刚刚将折可适调入他的幕中听候差遣。折可适虽是折家人,但一两个折家子弟在外任职,到也不妨事。从这几天他的表现来看,韩冈认为自己没有做错。

折可适作为折家新生代中的佼佼者,其领军上阵的能力自不必说,在战略战术上,他的水平也都是一流。是一个很适合成为作战参谋的人。而且韩冈觉得如果自己的门客幕僚,能从他身上学习,拥有一定水准的军事才华,自己也能轻松许多,日后便是他们的晋身之基。

文官想要快速晋身,军功就是最好。进士难考,但循军功出身,就容易了许多。只要辽夏两国的威胁还存在,只要皇帝还有开疆拓土的心思,从军获得一份告身,就是气学弟子最快捷的晋身之路。韩冈现在想看到的,就是西北军事中的幕职,成为气学门人的自留地。

不过文武之间的嫌隙颇深,能不能让他们互相促进,而不是互相的拆台、诋毁,这就让人颇费脑筋。尤其折可适和其他幕僚一样,都是心高气傲的脾气,决不肯伏低做小,让人半步。就是韩冈偏向于折家的赤佬,他们也不会任由他得意。至于韩冈的名声,想来也不至于因为这点小事而受损。

一人定气沉声:“十日来,阻卜骑兵已经在葭芦川出现过两次,虽然都被逐走,但大虫亦有打盹的时候,若事有万一,届时恐会追悔莫及。”

“只听说过千日做贼,可没听说过千日防贼的道理。”另一人配合着说道,“眼下只愁葭芦川各寨兵少,再调兵西去,那时可就是想防都防不了。”

“葭芦川不让晋宁军那边盯紧了,将阻卜人给放进来,不知要有多少村寨被祸害。那群贼寇,这段时间都抢上瘾了,保不准什么时候发疯往黄河撞过来。”

三名幕僚连番反驳折可适的论调。

河东以地处黄河以东而得名,但实际上也有几个军州位于黄河以西。麟、府、丰三州就不必说了,南面一点的晋宁军【今陕西佳县】,黄河由北向南将之一分为二。李宪所部,现在就驻扎在黄河西岸的晋宁军军城到葭芦川沿岸的几座军寨中。

一众幕僚都是明白,韩冈绝对不会愿意看到葭芦川诸寨被攻破。一旦河东、鄜延两路的联系被阻断,弥陀洞便是孤悬在外、独木难支,到时候,横山以北会不会连锅端了还不好说,但韩冈和河东军必然会成为笑柄。

更重要的是,韩冈一直主张稳守银州、夏州。而葭芦川至弥陀洞,这条路线就是河东支援夏州的主要通道。河东军镇守在此处,是对韩冈一直以来所保有的态度的坚持。若是调离此处兵马,可就是另外一番说法了。

几乎所有人的想法皆是如此,没人认为韩冈会不顾葭芦川各寨的安危而支持折可适。

折可适则是扭头望着韩冈,年轻骄傲的眼神中,甚至隐隐藏着点挑衅的意味。

“难道各位想要种谔如愿以偿吗?!”黄裳的突然出声让人惊讶,而他说出来的话则更让人吃惊,“种谔本来就无意协助徐德占,如今请求河东援手,仅是应付故事。要是龙图砌词拒不发兵,他可就能顺理成章坐视盐州被困!”

黄裳几乎是倒戈一击,就是韩冈都小吃一惊。

一名幕僚愣了片刻之后,期期艾艾的说道:“……可发兵之后,万一不能挡住阻卜人,那龙图的名声……”

“难道严阵以待的大宋官军,都没有信心胜过兵不满万的阻卜人?!”黄裳厉声喝道,“这样谁还会相信日后官军能剿灭西夏,乃至收复燕云?!”

大帽子扣下来,更是堵上了许多人的嘴。韩冈成了视线的焦点,厅中众人的眼睛都盯着他,现在他必须出来定下基调了。

韩冈也不再沉默:“名声什么的,倒不用在意那么多,国事为重,个人毁誉当放在后面考虑……诸位为我着想,我也是很感激,不过勉仲和遵正之言确实有理,兵是必须要发的,种谔有私心是他的事,但我奉天子命,经略河东以拯危局,自全的私心却不能有。”他看看折可适,“以遵正之见,当如何应对眼下的局面?”

折可适终于等到了韩冈的这句话,双眼顿时亮了起来:“葭芦川的兵不能轻动!如若给阻卜人趁虚而入,银州、夏州亦难保。”

折可适一句出口,就望向韩冈,等待他的评判。

韩冈点点头,没说话,其他的幕僚也一同在等待折可适下文的转折:是‘不过’还是‘但是’?

“不过鄜延路种太尉的请求也不能置之不理。”果不其然,折可适为自己的话加了个转折,“故而以在下愚见,还是派出三千到五千的骑兵前去助阵。河东地界,山峦为多,不利骑兵奔驰。即便事有万一,需要支援河北,也是以去协防河北城寨为多。有步卒听命便足矣,而去银夏堵截阻卜骑兵,则是用马军为上。正好互不干涉。”

还是有人在折可适的意见中挑刺:“阻卜人穷凶极恶,战力还在铁鹞子之上。若是一个不好,我河东骑兵可是会被种谔顶在前面,到时候,免不了会损失惨重。”

“可适前几日曾看了西面来的战报。以鄜延骑兵与阻卜人的交手经验来看,河东马军出阵之后,最好携带神臂弓。一旦遇上阻卜人,就下马列阵,而不是在马上交锋。正面相抗,他们绝不是官军对手,而且神臂弓射程远及百步开外,根本不怕阻卜人转去强夺战马。”

黄裳则配合着说道:“若是押送粮草则更好,还可以借用车辆为栅,那样数倍贼军,亦不足为虑。”

折可适颇有深意的瞥了黄裳一眼:“只要几次下来,让阻卜人碰了壁,想必他们就会另寻目标,不会为了西贼而与官军硬拼。不是有消息说了吗,阻卜骑兵在银夏,抢掠的党项、横山的等蕃部,可比对官军下手的次数多得多。”

韩冈默默的听着,心中暗暗点头。这才是现阶段,对阻卜骑兵最为正确的应对。

阻卜人只是被解开绳子的野狗,现在的确是凶悍,但要想解决他们,也不是不可能。如果是正面对决,又不允许撤离的话,大宋步卒能轻易压倒同样数量的骑兵——不仅仅是阻卜人——但要让阻卜人进入这样的作战环境,难度可想而知。眼下还是最好放下这个心思。

而且阻卜人所接受的任务分明只是骚扰。但他们对骚扰的定义,与正常的情况完全不同。用三四百骑为一队,普遍撒网,不但阻断粮道,更多的还是在沿线的蕃部村落杀人放火,就周围的党项部族也一并受到攻击,一视同仁,没有任何差别。他们是从除了草和牲畜、别的什么都没有的草原上下来,穷得疯了,见到什么都要抢。

阻卜人这一点做的还真让韩冈很是欣赏,若是杀尽了银夏之地的党项、横山两族,日后治理此地也就方便多了。官军之前碍于身份、面子,只能针对一干顽抗到底的党项部族攻击,对于已经投降的党项及横山蕃则不便下手清理——其实这样的蕃部才是最危险的,过去多有例证,当西夏军卷土重来,倒戈一击的必然是他们——有人代劳当然是求之不得。

不过这一切的前提是官军在战争结束后,依然能稳定的控制住银夏之地,至少也要保住无定河流域。否则,就算是银夏之地被烧杀一空,对于大宋也没有任何意义。

“就这么办吧。”韩冈对折可适的建议投了赞成票,接着却又是轻叹一声:“希望种谔也能以国事为重,不要坐视盐州的危局。若是盐州被攻破,接下来可就有的忙了。”

折可适抿了抿嘴,听了韩冈的叹息,得到认同的欣喜立刻就淡了下来。

一旦盐州被攻破,契丹人肯定会借机来压榨大宋,如果朝廷强硬以待,契丹人多半就会设法找一个突破口。不论是河北三关,还是黄河东侧的雁门关,都是朝廷防御的中心所在,想要攻下来,没准会崩了牙。

但换作是的云中的麟州、府州和丰州,由于是河东路西北、位于黄河西岸的一个突出地带,又收到辽国和西夏的两面夹击,成为目标首选的可能性,至少在一半以上。折可适对自家再有信心,也不敢说自家能抗衡辽夏两国之力,到时候折家能不能保全下来,还真得看老天爷的心情了。

这才是他一力主张援助种谔的主要原因。当韩冈开始全力支持鄜延路援助盐州,就算种谔想拖延,也很难找到借口。

保住盐州,就是保住折家。

折可适的私心,韩冈其实能看出不少。位置站得越高,所掌握的信息就越多,下面的幕僚也许还是一头雾水,但韩冈已是了然。

不过这也符合他的利益。

先不管什么私心,盐州那里的情况的确很重要。一旦盐州兵败,不仅仅雁门关就要忙起来,来自于辽人的侵袭,多半会集中在几个关键的节点上。云中之地便是其中之一,河东治下的麟府丰三州被夺占,韩冈的经略使也不要做了。

不过话说回来,韩冈绝不会比折可适更重视盐州。遣兵协助种谔,算是尽一尽人事,也算是还种谔一招。关键还是河东本身,确切点,是云中的麟、府、丰!

