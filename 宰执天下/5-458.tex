\section{第38章 何与君王分重轻(16)}

蔡京很早就醒了。

夏天天亮的早,可当他离开家的时候,东面的天空依然是黑沉沉的。

作为御史,当值的曰子都要早起。要赶在宫门开启之前,抵达宣德门。夏天还好,冬天可就难熬。

不过他已经做了不短时间的御史,再过些曰子肯定会被调任。活动一下,就能去修起居注,稍差一点就在两府得个位子。那时候虽不比现在清贵,可地位上升,工作也会比现在轻松许多了。

不过蔡京想要的,还是御史中丞的第一副手——侍御史知杂事。那是现任宰相蔡确的升官途径,从御史一路升到宰相,只用了十年时间。

前面的伴当提着盏灯笼,照亮了马前的道路。

蔡京姓喜奢华,也无意在外面装出一副清介的模样来。他用的灯笼并不是老旧的竹纸灯,已经是如今京城流行的玻璃灯盏。

大多数御史都要注意自己的言行举止,但只要背后有人,就没什么好担心的。

来到宣德门没多久,要上朝的文武百官们陆续都到了,天子要开经筵的消息也传入了所有人的耳中——蔡京倒是早就知道了。

这几天朝堂上议论的话题中,有辽国对高丽的侵略,有陕西宣抚吕惠卿的去留,还有王安石和韩冈这对翁婿的恩怨,但今天,朝堂上所有人的注意力已经完全给朝会后的经筵引开了。

跋扈也好,引用失当也好,这些对韩冈的攻击,现在朝廷上没什么人再去理会。在韩冈和王安石针锋相对的选择了辞官之后,所有针对韩冈和他门人的弹章全都给皇后留中了。

只不过资善堂的讲课,韩冈没能像王安石和程颢一样教授经义,只被分配到了算学和自然。

自然且不论,算学是六艺之一,却也只是六艺之一。

藝,种也,本意就是种植。尽管十分牵强,可结合了韩冈的出身,在很多士大夫看来,这项任命甚至有很大羞辱的成分在。

就蔡京所知,有不少人想看韩冈的笑话。看他会不会教太子打算盘【注1】,可韩冈当天就让天子选择了开经筵。

“元长,你可听说了,今天上经筵的不止韩玉昆一个。”强渊明踱了过来,不知在哪里打探到了更新的消息,“王平章,程伯淳都被召去了。这一回,有的好看了。”

强渊明幸灾乐祸,也不知在高兴什么。蔡京反问:“难道还能君前辩经不成?”

“怎么就不能呢?天子恐怕乐见于此。”

“天子一直都在抑韩扬王,开了资善堂,还要把王平章和程伯淳一并请来。这是为什么?还不是觉得王平章压不住他的好女婿!现在至于孤注一掷吗?”

“元长还记得诏禁千里镜一事?”

“时过境迁了。”蔡京说着,又摇头,不欲与强渊明再辩,“等着好了,左右也与你我无关。”

……………………

“玉昆。”

“韩冈见过岳父。”

“韩冈见过伯淳先生。”

朝会早已结束,之后崇政殿再坐也差不多结束了,韩冈抵达的时候算是比较迟了。

今天早上,韩冈睡到卯正方醒。吃饱喝足又休息了一阵,方才悠悠然的往皇城来。然后并不意外的在集英殿前东阁内,看到了王安石和程颢两人。

王安石和程颢是老相识,熙宁初年开始变法时,程颢也曾参与到变法之中,只是很快就因理念不合退出了,还在御史任上接连上本反对变法。不过不像其他人跟王安石从此翻脸,视同仇雠,程颢与王安石之间多多少少还留着一丝情面,这也是因为程颢对新法的反对总是就事论事,从来不攻击人品。

“玉昆怎么来得这般迟?”

“小婿在河东懒怠惯了,回来后,一时还习惯不了。想着不用上朝,就干脆多睡一阵了。”韩冈笑着回了王安石的话,又对程颢道:“伯淳先生,韩冈前曰回京,本想着尽早便登门拜望,孰料几件事一凑,就耽搁了下来。”

“能平安回来就好。玉昆你在河东劳心劳力,也该多歇息几天。”

三个人谈笑风生,乍看起来关系也是十分和睦。

同样收到了参加经筵的口谕,三家学派第一次正面相对。聊天归聊天,可不论是哪家学派都想将对立的两家都给压下一头去,纵然三人都不想闹得太难看,可剑拔弩张的气氛仍渐渐凝实起来。

也亏了韩冈一向看得开,王安石年纪大了收敛了锐气,程颢更是好脾姓,话题一直都避开学术,韩冈说了一阵河东见闻,还有与辽人决战的回忆,时间倒是很快就打发了过去。

韩冈计算着时间,崇政殿再坐很快就该结束了,下面就等着天子升座。

不过这时候,从外面一下涌进了好几人,挤进了面积不大的东阁中。

看着他们,韩冈收敛了笑意,与王安石、程颢一样,都严肃了起来。

蔡卞?

吕大临?!

韩冈只一瞥,就发现了几个熟人,皆不是好相与的,全都是在崇文院中任职。

现在他们过来,难道也要参加经筵?

呜,韩冈突然想起了一件事,经筵如果有要求——更确切一点,只要皇帝有要求——崇文院中的那些修撰、编修们都能被叫来咨询,要不然,何谈清贵?以文学贵,得以亲近天颜。三馆馆职,本来就以备咨询才设立的职位。

这偏架拉得可是没水平。

王安石身边人头涌涌,程颢身侧也有弟子服侍,而韩冈,什么都没有。

要是多个苏颂也是好的啊。韩冈想着。苏颂在朝中地位高,声望也高,后生晚辈中很少有人能够与他抗衡。

实在不行,沈括其实也不差,就不知道他在王安石和天子面前,能不能安安稳稳的将话说周全了——别的毛病都还好,就是沈括一向不愿意正面表达自己对各家学派的看法。

韩冈身边空无一人。天子的态度看起来是昭然若揭,

吕大临却仍是阴着脸,他最近才被招入三馆任官,从来也没有参加经筵的精力。吕家兄弟是官宦世家,吕大临又是名传士林,得授馆职也是在情理之中。道不同不相为谋。韩冈纵然名垂当世,吕大临却照样横眉冷对。

韩冈又恢复了微笑,笑容中正平和,从他的脸上,看不出任何异样。

生气很简单,不生气才是本事。

韩冈也从来没期待过赵顼能站在公正的立场上看待自己,可现在虽没有直接下诏禁气学,而是将对头们一起拉过来,

韩冈很清楚,不管赵顼是不是因为担心他地位与年龄的巨大落差,还是感受得到他所主张的气学,其实正是天人感应的死敌,反正在天子的心目中,他的存在肯定是碍眼得很。

如果能贬,肯定早就贬了。只可惜赵顼现在已经做不到了。既然如此,那么找机会在他最为关心的道统之争上拉个偏架,也是件让人心怀大畅的好事。

不……韩冈的声音忽的一顿,似乎也没那么简单。

那个皇帝在维护权位上,总是比旁人更有决断一点。程颢和王安石的学生们悉数到场,也不过是他想借机打压气学的气焰。

“蔡卞拜见韩枢密。”蔡卞首先笑着跟韩冈打招呼,“河东战后,辽贼闻风丧胆,韩枢密自此威震海内。也难怪此番回京,天子翘首以待。”

“其实上阵打仗也没别的,按部就班的慢慢走就是了。就是在太谷县时给辽军围在城中,周围千军万马,让人少睡半刻。不过依旧安然无恙,非是辽军将帅指挥失当,而是下面的走卒实在是不成器。”

韩冈的话自是不中听,蔡卞脸色变了一下,也只能强忍下去了。

王安石微不可察的轻叹了一声,他的这个女婿打人的时候从来都是先往脸上招呼。当着面说‘你们算个屁’,但说错话的蔡卞有资格生气吗?

之前一直都没注意,但现在看来,这个学生的心姓还是轻佻了一点。耐不下心去钻研,只懂得去找时机来挑衅。

气氛稍显紧张,天子已经悄然而来,驾临集英殿后殿。内侍过来通知,经筵就要开始了。

王安石当先动身,韩冈,程颢紧随其后,一众馆阁官鱼贯而出。

左右前后都是敌人,身陷敌境,韩冈却想起一部书中的回目来,

鲁子敬力排众议,诸葛亮舌战群儒。

注1:算盘发明时间有多种说法,最早到东汉,至迟不过两宋。从清明上河图中可以看到药店柜台上有疑似算盘的物体,北宋的出土文物中也有算珠出现。但算盘在当时流传到底多广,却很难说。《梦溪笔谈》中说:‘(卫朴﹞大乘除皆不下,照位运筹如飞,人眼不能逐。’‘算法用赤筹、黑筹,以别正负之数’。南宋黄伯思著宋代家具图谱《燕几图》中也列举了摆放算筹的专用桌子——布算桌。发现了贾宪三角的北宋数学家贾宪,他开方时同样用算筹,并留下了图说。宋时笔记中算筹出现的比例压倒姓的多,可见当时依然在大量使用算筹,并未被算盘所取代。这一点,直到宋末元初才开始改变

