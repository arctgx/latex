\section{第38章 一夜惊涛撼孤城(中)}

夜色之中,几十名巨人盘踞在河州城的城门前,一高一低的撑着修长的双臂。时不时的便会甩出后臂,将裹着碎石的绳袋投掷出去。

绳袋重重地撞击到地面,袋中的碎石四散而开,弹雨飞溅,不幸处在落点附近的人和马,无不落得头破血流、骨断筋伤。吐蕃骑兵已经吃够了霹雳砲的苦,都远远的避开了砲车的攻击范围。

一部宋军就驻屯在河州城的城头上,姚兕一身明光铠,扶着腰刀,站在他的将旗下。身后的城市中犹升腾着缕缕黑烟,在夜色中已经模糊难辨,只有空气中弥漫的一股焦味,提醒人们,这里刚刚经过了一场猛烈的祝融之灾。

午后时分,在催毁了河州城的城门后,景思立麾下大将王存,第一个率军冲入了城中。经过了几次反复拉锯之后,更善于城市作战的宋军,逐出了城中的守军,如愿以偿地攻下了河州城。

但撤出城中的吐蕃人出人意料的纵火焚城,转眼之间,城中不多的建筑便都笼罩在熊熊烈焰之下。河州城中的居民早就在城门被破坏之后从西门撤离了城中。木征下起令来,虽有几丝心结,也没有太多的顾忌。

但对于宋军来说,一座被焚毁的城市,要收拾起来已经很是麻烦,如果是在敌军随时可能反扑的情况下,更是有可能变成溃败的主因。

而且一开始就留在城外的两支吐蕃骑兵,从开战时起就在反复的试探,冲上前,又很快的退回去。等到木征出城后,更是同心协力,意图阻断官军的退路,只是在霹雳砲的威胁下才稍稍安定了一点。但留守大营的队伍,在城中尚有火焰燃烧的时候,就已经派人来急报王韶,木征军一部正在翻越支流河谷南面的山丘,意图从山上攻打大营。

粮囤和后路受到威胁,这样的情况下,王韶最后便只能让姚兕统领泾原军进驻烟火稍息的城中,堵上各处城门,然后到城头上休息。至于主力,则是交替掩护着返回河谷中的大营。

“木征士气未衰!”

“那是因为他们还做着不切实际的美梦。”

“只要香子城能安然无恙,木征的梦就该醒了,那时他们将不战自溃。”

木征不会浪费兵力上的优势,这一点经略司上下早就有所预料。收到香子城的紧急军报,也并没有多么慌张。王韶召集众将来商议此事,但无论是苗授,还是景思立,都因为顺利夺占河州而仍留在兴奋之中。把木征对官军后路的反扑,看作是困兽临死前的挣扎。

“我们负责解决到河州城这边的敌人,至于后方自有得力之人去处置。”

“一切早有准备,不必担心。”

“有韩玉昆在珂诺堡嘛……”

“韩机宜可是最擅镇守后路,前日他还特地调了去年协防渭源堡的那群人上来。希望他能给我等留口饭吃,别像去年那样,一口吞掉大半的斩首功!”

苗履的话,惹起一阵哈哈大笑。笑声中,王韶满意听着帐下诸将信心十足的对话。但他还是保持足够的冷静,木征这个死中求活的计划,所挑选的队伍必然是精锐中的精锐,很有可能,香子城就会有危险。即便王韶看到香子城的军报后面有王舜臣的签押,也一样觉得这座距离河州城只有五十里的小城堡,有着失陷的可能。但私底下,王韶认为只要珂诺堡无恙,就算香子城丢了,众军军心不失,照样能轻易的夺回来。

不过能保住就当保住,王韶点起苗授:“授之,你率本部回援香子城,要稳一点,小心埋伏。”

苗授奉命领军出阵。

王韶安定下来,有三千人回去已经足够了,下面就看看木征还有什么花样!

……………………

珂诺堡。

将广锐军将校与他们过去的部下重新配备,并没有耗费太多的时间。一切熟门熟路,就算有些地方和过去对不上号,都能自发的进行调整。

当来自山间的马蹄声与夜风同时传到城头上的时候,为数千人的广锐军也赶到了城门下。刘源之前还在反对韩冈重组广锐军,但看到曾经战斗和生活的队伍又出现在面前的时候,就突然间沉默了下去,调过脸,扶着雉堞望向黑沉沉的城外。

隆隆的蹄声仿佛有千军万马在奔驰,就像是成百上千的战鼓,一同敲击在群山之中。

韩冈先回头让广锐军就地休息,而后才问着骑兵的专家。“刘源,你听着贼人有多少骑?”

听见韩冈的发问,前广锐军指挥使便开口回答。但张开口,发出的声音却莫名的暗哑。他连忙咳了两下,才用正常的声调回道:“听声音当在千骑上下。”

韩冈用着自己粗浅的骑兵常识计算着,“从河州绕道珂诺堡,少说也有百里之遥。奔袭百里,至少也要一人双马,那就是五百人。这未免太少了一点。”

“不,贼军当是千人。”刘源更正着,“夜袭冲锋时多带着一匹马,会碍手碍脚,不好控制,没人会这么做。”

“那他们的战马体力还够?!”韩冈连忙问道。

“一般来说,夜袭出营时,肯定带了备用的战马。但接近目标后,便会换了战马,然后将换下来的马匹藏在一个安全的地——这些贼人的马群,当是留在十几二十里外。”刘源侃侃而谈,对骑兵战术的精熟在言辞间展露无遗,“不过就像机宜说的,战马的体力有限,现在他们也已经冲了一阵,该停下来歇一歇了。”

刘源的话刚刚出口,接近到一两里外的蹄声就减弱了许多,只有之前的一般,又过了片刻,就彻底的安静了下来。

韩冈方才一直在分辨着传入耳中的蹄声的方向,从声音消失的地方来判断,那里是香子城所在的方向。

刘源也在指点着夜色中的略显模糊的山影,“贼人是分作了两部分,夹着去香子城的道路。”

“贼人是不想让我们去援救香子城。”韩冈拍手哈哈笑了两声,“幸好田琼走得快,不然给堵上可就麻烦了。”

“机宜,该怎么办?”

“……休息,等天明出战!他们不敢攻城。”韩冈无意冒险,也自知没有拿全军的安危去冒险的权力。

“这……”刘源像是不同意见,犹豫着。

就在这时,又韩冈继续道,“不过,也不能让城外的贼人顺顺当当的降息马力,得给他们点余兴节目。”

“节目?”刘源不知韩冈这个词是什么意思。

韩冈笑了一笑,便将一道道命令发布出去。

城头上的守军大部分撤了下去,只留着一簇簇火炬,让三五百人守在城墙顶上以作提防。连一千广锐军,也进入城门边的营帐中休息。

只有一个指挥的步兵,受命来到南面的城门处。

……………………

青谊结鬼章拿着一块干布,亲自擦拭着爱马身上淋漓的汗水。方才在接近珂诺堡时被发现得太早,不得不提前冲锋。但眼下已经顺利的堵上了通往香子城的去路,除了嘲笑宋人的迟钝外,就只剩对几日后追击败退宋人的场面的幻想。

但在瞥眼间,看着不远处暗影般的城墙中段,忽然透出了一星火光。

青谊结鬼章顿时停了手,这代表着什么他很清楚,“宋军竟然开城了?!”

中开的城门处,十几名游骑当先而出,他们一直向前奔行出几百步,向后打了声呼哨,然后,一条火龙蜿蜒出门。成百上千点星火组成的光流,离开了城墙的保护。

青谊结鬼章顿时紧张起来,看着宋军的游骑渐渐接近,连同他们身后的步兵,一步步的靠近自己所率领的这千名骑兵立足的地方。只是宋军步兵离着还有一里多地,就从行军时的队列,转为一座方阵。

无数星火织成的阵列,比起后方的城池还要气势凛然,战鼓声在阵中响起,看起来转头就要攻上来的模样。

“这是要准备拼命?”对于宋军的不智之举,青谊结鬼章万分欢迎,一声令下,千名精锐便严阵以待,随时等待出击的号角。

但过了小半个时辰后,却不见宋人前进,只有鼓声一直在响着。

青谊结鬼章这时终于明白了,这是宋人的骚扰试探,并没有作战的意图。

被愚弄的愤怒涌上心头,攻击的口令已经到了嘴边,但他竭力给按奈的了下去,‘岂能让你们如愿?!’

“城门是开着的!”身边有人提醒着年轻的族长。

要是能够击败眼前的宋军,紧摄在他们之后,趁势攻入城中,夺取珂诺堡也不是不可能的。那座敞开的城门诱惑力的确很大,犹豫再三,青谊结鬼章咬着牙,攥着缰绳的掌心湿漉漉的,最后却是什么话也没说。

“还以为他会忍不住。”

城头上,韩冈遗憾的叹了一声,他已经准备让人将就睡在城门边的广锐军给叫起来守门了,但想不到还是吐蕃人领军的将领这么谨慎。

“让他们回来吧!换人从西门往山里去,不用点火,声音可以大一点。”

“别让他们睡觉!”

