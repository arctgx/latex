\section{第31章 九重自是进退地(14)}

韩家家里正在收拾家当。

韩家的仆佣以他的地位来说,人数并不算多,男女老少加起来了也不过四十来人,都是做事的仆婢,没有养来赏玩的。

通常到了学士一级,蓄养一队家妓、一支乐班,都是再正常不过的事。如王安石那样清简完全是特例吗,但韩冈却是学着他的岳父,并没有在这事上费心思。

家中人口少,行装自然也简单,之前就开始在收拾,到了今天,绝大多数都捆扎好了,等着明天装上车。

之前已经经过了殿上陛辞这道环节,韩冈预订的启程日期也就在明日。在宜出行的好日子,韩冈就要带着全家老小向西出东京城,去他新的工作地点上任。

依常理,天子应该再见上韩冈一面,算是给他送行,并再次确认他上任之后的施政方针,这是重臣应该有的待遇。但都到了要出发了,天子并没有再次召见韩冈的意思。

除去礼仪性质的朝会,平灭交趾的功臣,在京城逗留的一个月的时间里,仅仅被召入宫中一次,韩冈失了圣眷的传言,在京城中甚嚣尘上。

一时间一股股暗流涌向韩冈,在京城中,总少不了有人会‘聪明’的揣摩上意,也总有人想靠着踩在另外的人身上,向上爬去。

“天子都只召见过一次,韩冈竟然还能做他的都转运使!”

“那是他在交趾有功,让天子不好加以处断。”

“天子当真看重他,怎么会让他外放?!”

“不是说他年资浅薄,所以天子要他在外做上数任。”

“天子既然有这番考量,岂不是正好?韩冈身上一点罪名都没背过,若是给他修成了襄汉漕渠,怎么还能再挡着他入京?我等上表弹劾,让韩冈戴罪立功,天子自当乐见。”

“如此倒是不错。本来不想多次一举,但都到了眼前,总不能放过。”

“韩冈得官前,都已经是快家破人亡,可眼下在熙河路,说起豪富,谁能比得过韩家?前些日子,在下查看熙河诸州田籍,韩家的田地已经多达八百余顷,这贪渎之罪是少不了的。”

“韩冈在熙河、广西都没少杀人,这嗜杀之罪也同样少不了的。”

“举荐皆同门,有结党之嫌。”

“这一干罪名给韩冈定下,谅他也难脱身。”

……………………

大事小事都已经处理完毕,在变得空旷起来的书房中,韩冈正抱着儿女,给他们说故事。却见王厚不带通报,就咚咚咚的疾走近来,看到韩冈悠然自得的模样,他急叫道:“玉昆,都出大事了,你还这般悠闲?!”

韩冈放了儿子女儿下来,示意他们出去。待家里的几个孩子,很守规矩的向王厚行过礼后离开,他方才问道:“出了何事?”

王厚也不讲礼数了,一屁股坐下来:“今天十几人接连上本,一齐弹劾玉昆你,贪渎、擅兴、好杀,要留身勘问,并乞诛之。”

韩冈一脸紧张:“啊,那还真不得了。天子是怎么说的?是不是依卿所奏?”

王厚板着脸瞪着韩冈,而韩冈则是反过来板着脸看着王厚。王厚眨眨眼睛,最后撑不住笑了起来,“当真跟家严说得一样,玉昆还真是沉得住气。”

“因为是说笑嘛。”韩冈微微笑道。

王厚呵呵道:“哪边说笑?是说愚兄,还是说弹劾玉昆你的那几位?”

“难道不都是在说笑?”

王厚纵声大笑起来,“的确都是在说笑话啊!”

当年司马光弹劾王广渊,一连上了八九章,说是要‘留身乞诛之以谢天下’。王广渊急得到处找人,最后找到了任起居注、随时都在天子身边的滕元发,询问天子当时的回复。滕元发的回答是:“只我听得圣语云:依卿所奏。”却把王广渊吓得魂飞魄散。

这当然是开玩笑,最后王广渊屁事都没有,英宗皇帝根本没理会司马光的弹劾,让王广渊升任群牧、三司户部判官,后来又加了直龙图阁,宠遇一时。

韩冈就是知道王厚是在开玩笑,才这般悠闲的回了这么一句。不过王旖她们却不知道,从儿子女儿的口中问了几句,四名妻妾就脸色大变的匆匆忙忙赶过来,却见韩冈和王厚正在哈哈笑着。

四女一头雾水,王旖疑惑的问着:“官人,王家二伯不是说出了事吗?”

“没事没事,放心好了。”韩冈挥挥手,“去准备酒菜,我和处道今天要共谋一醉。”

王旖疑惑的看看王厚,不知道韩冈是不是在故意说谎好让她们放心,王厚则忙站起来谢罪,玩笑开大了也不是好事,“乃是愚兄说笑罢了,不意惊动了弟妹,还望恕罪。”

韩冈的几名妻妾终于离开了,王厚不好意思的摇摇头:“早知会惊动到弟妹,愚兄就不开这个玩笑了。”

“家里迟早会知道小弟被弹劾的事,处道兄倒也不用太在意了。”韩冈笑着说道。

韩冈当然知道许多人都对他幸灾乐祸,年纪轻轻便身居高位,受到的嫉妒自然也为数众多。有人趁机上书弹劾,拿些捕风捉影的事来攻击自己,想趁机捞取名望,这一点根本是不用想的。

但做是一回事,能不能做到,则是另外一回事。想要弹劾他韩冈,也得先看准时机。

眼下可不是好时候。

“出征在即,杀几个不开眼的祭旗也是好事。”韩冈的笑容中的寒意让王厚都有些发冷,“如果此事不给我个说法,我可是要反过来讨要个说法了。”

……………………

“‘为官不及十载,田产已至千顷’这一条总算是有真凭实据了。”吕升卿一张张的翻看从宫中传出来的弹章,虽然在私下抄录的过程中,为了方便起见有所省略,但安在韩冈头上的罪状,倒是一条条都不缺的罗列了下来。

“韩家在熙河路,千顷田当是没有,不过数百顷倒真的有。”吕惠卿撇撇嘴,“可惜找不到田地的原主,全都是荒地开辟出来的,想要告他个强买民田都难。何况高、王两家在熙河路的田地只多不少,凭这个罪名,怎么都动不了韩冈。”

“当真是一群蠢货,真当韩冈好欺负不成?”连吕升卿都知道这一干人做的都是无用功,自寻苦吃,“也不看看韩冈的身份地位,现在正要做什么?哪里这般容易被弹劾的。”

“这样也好,朝堂上也能清静一点,天子可是要逐人了。”吕惠卿冷笑着。恐怕想打落水狗的那十几人都不会想到,天子赵顼竟然对韩冈这般看重。

皇帝对于臣子所上的弹章,一般有三种处理方法,一个就是转发有司,根究是否属实,以此来决定是否治罪;一个则是留中不发,留待后论;另一个则是并不根究真相,而是直接凭着弹章,将人请出去。

但赵顼对韩冈的态度,却是三条之外的第四条,竟是亲笔批驳,将弹劾韩冈的十几位官员一个个全都降罪外放,甚至还包括两个御史一齐发落。处罚之快之狠,今天的政事堂都一时没了声音。

“韩玉昆眼下要打通襄汉漕运,捅出天大的篓子,天子都会帮他挡着。”经过今天的这一事,吕惠卿重新确认了天子对于襄汉漕渠的重视,也知道自己之后该怎么做,“任谁敢干扰韩冈行事,天子都不会留手半分。”

“不都是看着韩冈失了圣眷吗?”

“圣眷。”吕惠卿像是什么好笑的话,咧嘴笑了一声,随即冷下脸来,“能不能进两府那是要靠圣眷,韩冈他一个龙图阁学士,做着他的都转运使,还要靠圣眷不成?!为兄若是出了事,外放之后,也少不了一个大郡郡守。”

身居高位的官员则都知道,所谓的圣眷,过了直学士一级之后,也就仅仅决定是否能进入两府了。一旦哪位得到了直学士的名号,就是在朝堂上政争失败,也至少能到地方做个知州。

韩冈都已经是龙图阁学士,眼下看似没了圣眷,但他京西都转运使照样做着。若是成功,保不准能因功进两府,就算不成功,降了罪,也至少一个中州知州。

朝堂上的交锋,下层的官员能贬去监酒税,但最上面的重臣,即便是失败也不会被痛责,几十年来,皆是如此。士大夫不能与凡人论,而重臣更不能与小官一视同仁。

经过了平南一役,从转运副使升都转运使,从龙图阁直学士升学士,韩冈早已经是实打实的重臣,靠着功劳打下的根基,哪里可能是轻易可以撼动的。

“本以为会留中呢。”吕升卿叹了一声。

“留中太过暧昧,天子不想再看到有人打扰韩冈,所以要想给个明确的回答。”

“知制诰应该会封驳吧?”

“孙洙已经封驳了。御史有风闻奏事之权,不当以言罪之。”

“那天子会如何处置。”

“多半还是放他们一马。总不能为了此事,让知制诰都一起出外吧。想必天子的态度也很明确了,不会再有人误会。”

