\section{第31章 战鼓将擂缘败至(九)}

罗兀城中的焰色冲霄,火光映红了半幅天空。

梁乙埋眼定定的望着夜幕下的红光,已经听到了宋军离城的消息,却有种猝不及防的感觉。按照他制定的计划,其实并不是伪作撤军,而已经是准备转调大半兵马去攻打内乱中的环庆路,只留一万精锐在银州候着——他其实已经无力再把麾下的数万大军,在罗兀城下耗下去了——银夏之地的多年储备已经全都挖了出来,用来供给全军。梁乙埋这时是孤注一掷,无论是罗兀城,还是环庆路的几座缘边城寨,他都想彻底解决。

先攻打环庆路,收集粮草,逼得宋廷更加慌乱,再回头攻打罗兀。这就是梁乙埋的如意算盘,只是他没想到,高永能撤离得那么快,让他的计划全盘作废。

从罗兀城守军开始出城列阵,到城中火起,也不过半个时辰的时间。如此快速的行动,实在是让梁乙埋惊讶不已。

因为是防备着宋军出城追击撤回银州的队伍,摆在外面做护卫的铁鹞子的四个千人队,看到了宋军在城外列阵,便不敢轻动,而且又退回了一点距离,不想离得太近。等到终于发现宋人也是在撤军,而且是彻底的放弃了罗兀城,再想整顿兵马出击,天色早已黑透了。

不过黑夜并不是问题,现在困扰着西夏国相的最大的问题是由谁去追击?

梁乙埋的麾下大军,有一半已经在白天回到了银州,而剩下的一半尚留在罗兀城下的营地中,原计划是在明天回返。留在营中的几家豪族听说了宋人撤离,便立刻叫着要追击。但已经抵达银州的几个部族,听到消息却派了快马匆匆的赶回来,就拦在营门处,不让任何人出营。

‘哪家不是都是饿着肚子等了许久,好不容易等到了肥羊离城了,竟然想吃独食?!’

要不是顾忌着最后会造成两部火并而两败俱伤,还有梁乙埋赶出来阻拦,两边早就开始厮打起来了。虽然现在是交由梁乙埋处置,可他也知道他必须公平处断,否则就别想再让下面的人听话。

但梁乙埋也很头疼,若让仍留在营中的几家出兵,已经饿绿了眼的其他豪族,说不定就能领军过来抢夺战利品。但要是等退到银州的军队全部回来,宋人早就走出三四十里了。

想了一阵,梁乙埋觉得还是快一点解决,看下面两边的架势,说不定等高永能进了绥德城,还定不下来。“能出兵的先出兵,追上宋人再说。在银州的各家,先出五百骑兵,明天清早之前抵达此处。得到了缴获按照各家兵数来分配。至于缴获……”西夏国相阴森森的眼神环顾一圈,“谁敢私下隐瞒,就拿谁的首级来一验军纪!”

梁乙埋一言敲定,被阻挡得快要失去耐心的寨中大军正欲立刻出寨追击,但翰林学士景询这时走到了梁乙埋的身边。景询本是汉人士子,因为犯罪当死而逃亡西夏,现在身居高位,也是梁乙埋在朝中的亲信。因为出战大军久无捷报回传,他就奉梁太后之命,带了一点酒水和银绢来阵前犒军。

才到了没两天,但把国中精锐拖得苦不堪言的罗兀主帅,已经给景询留下了深刻的印象。他上前谏阻着:“高永能为人狡狯,他大张旗鼓,趁夜宵遁,必有诡计,不可穷追不舍。当明日天亮后,再行追击。届时宋人一夜奔波劳累,正是败敌之时!”

一群迫不及待的将领立刻虎视眈眈的瞪着他,眼中尽是杀气。

“小心就是!”梁乙埋一摆手,欢呼声中,一队队铁鹞子便立刻从各个寨门奔涌而出,向着宋军离开的方向本去。

“相公!?这是为何?”景询急问着。

梁乙埋眨了眨眼睛,低声冷笑道:“第一个当然先死。但只要把宋军绊住了,后面紧跟上来,必然可以把他们全数留在无定河边!”

…………………………………………………………………………

行走在黑暗之中,只有一点火光照耀着脚下的地面,王中正这时才害怕起来。

在赵瞻的命令下,他来到了罗兀城。进城时,还有这不过如此的心思。但这回程有多么的艰难,直到出城之后,他才真切的体会到。数万敌军锁在背后,就像杀气腾腾的刀子在背心处比划着。

冷汗浸透了全身,周围就算围满了士兵,但始终沉默的他们,让王中正无法有上一点安全感,直接感受到死亡的临近。在离开罗兀城还不到半个时辰的时候,他已经把一辈子的悔恨都用光了,早知道就不催逼着张玉和高永能撤离罗兀城。

离开了罗兀城的队伍走得并不快。在无月的朔日,天上的星光黯淡,只靠着火炬,夜间奔马根本是个笑话。而急行军也是有难度的,领军的将校没一人会幻想在被西贼的骑兵衔尾直击后,正在急行军中的队伍还能坚持下来。

幸好无定河边坑坑洼洼的路面上,对党项人的影响肯定也是一样,在追击时,他们也别想骑快马。甚至得像宋军骑兵一样,下马牵着走。

离开了罗兀城后,在河边官道上,逶迤而行的大军,只走了十里就停歇了下来,并没有再继续前进。罗兀城的幕僚们推算过梁乙埋出兵的速度,正常的情况下,再过一阵子,西贼的追兵就该到了。

正在施行中的撤离罗兀城的计划里,有着如何应对追兵的一整套方案。不过并不是什么计策,而是要通过堂堂正正的战斗来击败对手,让他们不敢再追击。

击败西夏追兵,彻底洗脱罪责,这就是集合了众人之智,在撤离了罗兀城后实行的计划。是靠着这段时间以来,不断以孤城压制西夏国相所率领的倾国之兵,所带来的自信和底气。

在罗兀城下已经拖了一个月,并且还惨败过一次的情况下,梁乙埋带来的数万大军,还能有多少实力?

别以为大宋官军放弃了城防,一干党项贼子就能够恣意妄为。

当年的刘平在三川口中了伏击后,还是拼杀了一夜,甚至在李元昊的眼皮底下建起了一座营寨。要不是兵力实在太过悬殊,丢盔弃甲的该是李元昊才是。

而且要知道,绥德那里还是有援军的,当听到罗兀城弃守的消息,种谔为了消减自己的罪责,肯定是要出兵接应。

把带出了罗兀城的上百辆马车,卸下了车轮,整齐的叠放在来路之后,很简易的一道防线便告建起。虽然只是针对后方,防不了过河的敌军,但雪水解冻后的无定河。正值桃花汛时,水流湍急,难以度过。有此为屏障,只需防着后路便可高枕无忧。

“怎么还没来?”种朴等着有些心浮气躁。

韩冈也很纳闷:“什么时候西贼有这般耐心了?”

就算以韩冈的才智,或是张玉等老将对西夏人的了解诶,谁也不可能想到梁乙埋手下的,会因为决定谁出战追击,而耽搁了时间。

不过他们并没有等待多久,先是有伏地听声之能的斥候开始报警,接下来数以百计的敌骑举着火把,出现在道路北面。火炬多如繁星,充满了谷地,当他们被马车阻挡,追击的速度便为之一缓。

上百辆马车都载着引火之物,载物很轻,所以才能方便的在黑夜中的谷道上行驶。宋军把引火之物都集中在了一起,见着追兵已经跟了上来,便立刻把准备已久的火箭全数射了上去。

熊熊的火焰顿时燃烧起来,这也是一个信号。一阵锣响,道路一侧的山坡上一片箭雨落下,火光中晃动着的全都是目标,射击起来不费什么力气。

谷中一片声的惨叫,不懂党项语的韩冈和种朴却不知他们在叫些什么。

但惨叫声让种朴很兴奋,他笑着对韩冈说道:“也许我们该砍倒一棵树,上面写庞涓……不,梁乙埋死于此!”

“正主还没到呢!”韩冈摇了摇头,西夏人只是小挫而已,而且遭到射击的西贼,已经用着比来时快上数倍的速度离开了。

前军丢盔弃甲的模样,让后续的队伍为之警觉,也暗叹侥幸。他们终于回想起景询的话,不由得放慢了追击的速度,想着到了白天再来追击。这让已经扎下简易寨防的宋军主力,有了足够的休息时间。

“还是玉昆厉害,一下就吓得西贼不敢急着过来了。”

“侥幸而已,不敢称功!”

韩冈谦虚的笑了笑,计策的成功都是建立在军心士气还有战斗力均强于对手的基础上的;是建立在一巴掌一巴掌把坐拥八万大军的梁乙埋,打得只敢捡便宜的基础上的。若不是宋军能在野战中击败同等数量的西贼,想要党项大军面前从容退走,除非诸葛复生——而韩冈,那是连五根琴弦都认不全。

但现在,连夜奔驰的骑兵,对上严阵以待的对手,能有几分胜算,韩冈倒想为梁乙埋算上一算!

