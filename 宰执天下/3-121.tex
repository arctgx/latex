\section{第32章 忧勤自惕砺(中)}

回到家中,已是夜幕将临。

吃过饭,王安石将今日延和殿中的一番奏对,一条条的跟着儿子讨论了一遍。

王雱对于天子畏契丹如虎的态度,很是看不上眼。又对派谁去知定州并兼任真定路经略安抚使一职,与父亲讨论了一番。等到听说了赵顼并没有怪罪韩冈在白马县的打算后,放心下来之余,却又说道:“官家如此看重玉昆,不知会否如弥子瑕前后之遇。”

弥子瑕乃是春秋时卫国人,以男色侍奉于卫灵公,备受宠爱。一日,其母病危,弥子瑕假传了命令,用了卫灵公的车驾赶回去探视。这本是重罪,但卫灵公却道:“孝哉,为母之故,亡其刖罪。”——弥子瑕孝顺啊,为了母亲,忘掉了要砍掉脚的刑罚。过了几天,弥子瑕与卫灵公又去桃园游玩,吃到一个甘甜的桃子,吃了一半,将剩下的给卫灵公。卫灵公又感叹道:“爱我哉!亡其口味以啖寡人。”——他是多爱我啊,放弃了自己喜欢的桃子献给寡人。

可等到弥子瑕年老色衰,不再受宠,卫灵公就翻起了旧账,“是固尝矫驾吾车,又尝啖我以馀桃。”——他曾经假传命令驾驶我的车子,又拿吃剩的桃子给我吃。

王雱提着弥子瑕,是在担心现在韩冈受天子看重,所以行事无碍。但日后翻过来,很可能会被算旧账。

“此比不伦不类。”王安石听着不舒服,狠狠瞪了儿子一眼。

王雱呵呵的笑了笑,也不分辨,在自家里拿天子比卫灵公没什么大不了的,可拿韩冈比弥子瑕的确是不太好。“最近二哥在白马主持深井汲水灌溉之事,很有些成效,玉昆也来信说二哥帮了他大忙。”

虽然只是小事,但看到次子有所成就,王安石的心里也很是为其感到高兴。

父子两人正说着,管家进来通报,却是曾布登门拜访。

王安石神色一肃,“曾子宣这时候过来,必然有事!”

“说不定是来抱怨的。”王雱说着,哈哈一笑。因为吕惠卿曾丁忧三年,曾布在官位上一直稳稳的压着他一头。但就在这两天,吕惠卿升任翰林学士,而昨日王安石又将曾布判司农寺的差遣转给了吕惠卿,换作是任何人处在曾布的位置上,肯定都会不痛快。

曾布很快就进来,却还带着一人。王雱不认识,但王安石却见过他,乃是市易法的倡议之人魏继宗。

等下人奉了茶,王安石便问道:“子宣漏液来访,不知出了何事?”

曾布拱了拱手:“相公应该记得,年前京中物价飞涨,其时多有人言,‘市易务扰民不便着甚众。’曾布前日受诏暗访,如今已得探得确实。”

“哦,探查的如何了?”王安石端起茶喝了一口,问道。

“市易法本为良策。但如今主事之人专略其利,障固其市,只知聚敛搜刮,一切皆背初衷,都邑之人不胜其怨。”曾布几句话说过,示意魏继宗将其中情弊细细说来。

王安石听着双眉越皱越厉害,等到魏继宗一番话终于说完,他立刻问道:“事既如此,何以不及早告知?”

魏继宗回道:“提举日在相公左右,继宗何敢提及于此。”

魏继宗说的提举就是吕嘉问。吕嘉问的确经常跟在自己身边,王安石对此也清楚,不好说什么。

只是曾布来此说吕嘉问之事,王安石从中还是看到了其中端倪,潜藏起来的一份怨气,连着魏继宗久不迁调的怨艾混在一起。曾布肚子里藏着这口怨气,当是出在吕惠卿身上,加上吕嘉问,现在终于爆发出来,王安石对此也能够理解。

在王安石的心中,曾布和吕惠卿是他的左膀右臂,私底下甚至还更看重吕惠卿一点,毕竟在学术上,曾布还是不如吕惠卿。而且吕惠卿在政务上也绝不逊色。去年他接下判军器监一职,不过一年不到的时间,就从过去‘在京及诸路造军器多杂恶,河北尤甚’的情况,变成了如今的‘兵械皆精利’,这个功劳决不下于攻城掠地。曾布此时已经是翰林学士,吕惠卿当然也不能落后太远。正好翰林学士有空缺,王安石就奏禀天子,让吕惠卿凭着功劳补上这个位置。

但王安石对曾布还是十分重视的。前两天,将曾布手上判司农寺的工作转给吕惠卿,他也是有着一番更深的考量,并不是要让吕惠卿压着曾布一头。不管怎么说,王安石都不会去故意去挑起了左膀右臂之间的争斗。

明了得力助手的心思,他笑了一笑:“子宣你是三司使,不知准备处置市易务之事。”

曾布停了一下,眼神低垂,视线不与王安石交汇:“曾布明日当入对,欲以此尽数禀报天子。”

王雱听了一下怔住。而王安石脸上的笑容也凝固了,半晌之后,才勉强说道:“啊……是么,如此也好。”

厅中的气氛突然间变得让人难以忍受,虽然曾布和王安石两人都还在说着话,但已经变成了毫无意义的赘言。又东拉西扯的说了一段时间,曾布带着魏继宗起身告辞。

等到曾魏二人离开,王雱才一拍桌案,厉声叫道:“他这是要学蔡确吗?!”

王安石沉默着。心头有着火气,更多的还是酸楚。想拿起茶盏喝两口,只是手抖着,连滑了两下,都没有拿稳。最后干脆的放弃了,身子一仰,靠在椅背上。

蔡确叛离,王安石并不在意,但曾布不一样啊……

“曾子宣今日做的,就跟文彦博在大名府做的一样,都是一点错都没有。”王雱咬着牙,嘿嘿冷笑。

文彦博在大名府用着常平仓耗到最后,聚集在大名府周边的流民,听吕惠卿回来说至少有十万上下。眼下大名府仓中无粮,朝廷前些日子也因为黄河解冻,而无法将文彦博要得六十万石粮食都运上去。现在流民全都向南面涌来,不可能再回头。其中即便有错,也不是文彦博的,他在大名府养了流民一个冬天,又没有让他们闹出事来,一切做得无可指摘。

但文彦博做的事,仅仅只是普通官员该做的,能做的,却绝不是一国宰相该有的水平。文彦博不是普通的官员,他能做到一国宰相,治政上的才能就算是政敌也无法贬低。可他今冬在大名府做的,可有半分宰相的水准?还不如做着知县的韩冈。

同样是宰相处理灾情。富弼当年知青州时,也是遇到大灾流民,他却是很轻易将五十余万流民全都安置的一一当当,一年多的时间,扶生民,葬死者,一点也不给朝廷添麻烦。而且其安置流民的策略,也成了之后官府遵循的法度。所以文彦博在处置流民上的失色,即便他做得半点错也没有,也让人会有些想法。

而曾布也同样如此。

从为臣之道上,曾布行事并无错失可言,而且事先还跟王安石通了气,更是做得完满。作为臣子,忠心的只该是天子,下情不上禀,这是欺君之事,非是忠臣所为。事先禀报于王安石,则是尽了知遇之情。

只是在官场上的道理,可不是说给外人看的这些。曾布此举,政治意图十分明显。除了天子以外,放到谁人眼中,都是能从中看到见风使舵四个字。而方才跑来王安石府上通知一声,则就跟最后通牒一般。一番话、整件事,都是明明白白的依照朝规,让王安石根本无法开口阻止。

王安石不知沉默了多久,终于开了口:“此次大旱遍及数路,经冬不见雨雪,为父其实已经有了出外的准备。”

王雱闻言眉眼一动,就要说话,却被父亲的眼神阻止了。

随着王安石开始说话,他一直保持着冷然沉稳的神色终于松懈下来,就像解开了包裹在外面的甲胄,方才深藏起来的疲惫和伤感再难以掩饰,“为父出外无妨,但新法绝不可废。政事堂中必须有人来坚持施行,不至使奸人沮坏。代居宰相之位者,为父属意于韩子华【韩绛】。当年罗兀之事,也该是过去了。子华曾为昭文相【首相】,其代为父之位,有足够的资格挡着冯当世【冯京】和吴冲卿【吴充】。而且这个人选,想必天子也不会有意见。至于辅佐之人,为父则是在曾布和吉甫两人之间犹豫……”

现在就不会再犹豫了。

从父亲冷然又伤心的眼神中,王雱看得出来;从父亲对曾布称呼的改变上,王雱也听得出来。

不会再犹豫了。

其实王雱更清楚,如果要父亲在曾布和吕惠卿之间做个选择,到最后肯定还是曾布能胜出。曾布的资历要在吕惠卿之上,翰林学士之位,吕惠卿才是刚刚接手,而曾布已经做了一年多、近两年的时间。且过去数年,吕惠卿居乡丁忧,曾布一人身兼十几个职位的辛苦,自己的父亲更是都看在眼中。日前将曾布判司农寺的职位转交给吕惠卿,其实就是不想让他再纠缠于琐事,而是要负担起更全面也更重要的工作。

只可惜……曾布自己毁了这一切。百计求之,却不想会离着目标越来越远。

“就看他明天怎么说了。”王安石冷淡得仿佛在说一个陌生人。

