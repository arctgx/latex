\section{第44章 秀色须待十年培(18)}

将萧禧逼得入席,算是一个小小的胜利。

不过即使到了席面上,按照礼节的几巡酒后,言辞之间的交锋也没有停下来。

在都亭驿中,辽国的使团被约束得像是坐监一样,外面的消息进不来,里面的消息出不去。所以到了酒宴上,留给萧十三享受的余地很小,他得到的消息,往往都是宋人故意泄露给他的。

“如果贵国硬是要侵吞高丽,那就没什么好谈了。”韩冈的语气看起来很是强硬的说着,完全不留任何婉转的余地。

萧禧的态度并没有因为几倍水酒而软化:“高丽王家负隅顽抗,毫无向悔之心。征战之中,自留不得他们的姓命。”

“当可别立旁嗣。难道高丽宗室,无论远近,都给贵国杀绝了?”吕惠卿冷笑着着问道,

“宣徽当是不知,高丽王女不下嫁臣庶,必归之兄弟。数百口皆在开京城中,兵火一起,连城俱化为灰烬了。”

“数百口人没一个逃出来?!”吕惠卿当然不信,只是他在这着一会儿,可没权力干涉军务。

“贵国在交趾所为,其实跟鄙国在高丽做的有多少差别?”

韩冈反问道:“交趾杀我中国子民数万。国仇可复,此乃春秋大义。不知高丽杀了多少贵国子民?”

见宴上争锋相对,郭逵只顾低头看着面前的酒杯。

世人皆知,祸水东引的是韩冈。高丽国灭,也可以说是在韩冈的计划之中。只要能用高丽拖住辽国,那么谁都要赞一句韩冈运筹之妙已是出神入化,堪比管乐。韩冈现在的强硬,想来也就是为了达成这个目的。郭逵很是好奇,韩冈到底能用什么办法,将辽国硬生生的拖在高丽,只凭那些不堪一战的水军吗?

郭逵完全不出声,他是武将,倒也罢了。而吕惠卿,也同样不做声。跟辽人像商人一样讨价还价,丢脸的是韩冈。若是将萧禧给气走,犯下大错的依然是韩冈。既然如此,还插手做什么,干脆就交给韩冈好了。

萧禧当然不肯放弃,他挟攻下高丽的声势而来,但宋人根本就不在乎,自己是盲目乐观了。相反地,宋人的还击他却不能不在乎。那关系到高丽的战局。

各自都有心思,酒宴很快就结束了。萧禧不用人扶,根本就没醉,却等于是被押送回了都亭驿。

韩冈和吕惠卿跟郭逵一一告辞,郭逵今天在殿上一句话都没说,事后也没有松口的意思。但韩冈与吕惠卿有话说。

“之前玉昆你说要看辽国的诚意。要是有朝一曰,辽国的诚意充分,那么耶律乙辛打算篡位,玉昆你就是准备反对出兵了?”

“怎么可能?当然要出兵!匡扶正统,存亡续绝,这正是华夏有别于蛮夷的地方。”

至于幽燕,那是酬劳。还有比这更加名正言顺的吗?韩冈可是一直都在盼着耶律乙辛能够早一步篡位。

“回想当年,萧禧每次入京,朝堂上就要乱上一次,如今倒是变了,朝堂上安安静静,而换成是萧禧坐卧不安了。”吕惠卿感叹着,这样的结果,在几年前完完全全想不到。

“弱国无外交。”韩冈说道。

原本以军力算,辽强宋弱,所以辽国国使每每能逞欲于大庆殿上。但现在宋强辽弱,萧禧虽是外交上经验丰富的使者,在咄咄逼人的吕惠卿和韩冈面前,也只能进退失据。

“这一句说得好。”吕惠卿放开缰绳,双手拍了拍,“可以登载到报纸上了,给今天的事做个标题。”

“报纸?也不知道他们有没有兴趣。”韩冈看了吕惠卿一眼,有这一位在,快报就会当做没这回事。

吕惠卿到底是把京城中的宗室、贵戚和豪商得罪得太深。吕惠卿回京的消息,快报上没有刊登出来,更别说他在殿上的精彩演出。翻翻近曰的报纸,里面甚至提都没提到吕惠卿这个名字。倒是辽使被火炮吓得魂飞魄散,被翻来覆去的说。东京百姓最是喜欢这一套,所以那些编辑们都是不厌其烦的反复一说再说。

不仅仅是这一次,就是之前宋辽大战,两家快报的报道主力也放在河东、河北的战局上。而潼关以西的战事,就是王舜臣的名字都比提到吕惠卿的次数更多。在种谔和吕惠卿的对比中,种谔也是远远胜出。早在吕惠卿到山西前,他就已经是名震一方的大将了

由于宣传上有所侧重,在最底层的百姓中,有很多人都认为吕惠卿是借助了种谔才得到的灵武之地,只是捡了便宜去。相比起韩冈救难之功,当然是差了很远。与郭逵那边相比,也就占了个斩首多,夺回的土地多,其他也只是平平。

虽说这样的说法,擦了一点事实的边,但总体上说,没有吕惠卿在背后支持,最后的结果不会这么完美。贺兰山下的核心地区给官军牢牢占据,几家从青铜峡出来的党项部族根本无力与官军抗衡。这其中,吕惠卿起到的作用比种谔要大,而且是大得多。

“肯定是有兴趣。”吕惠卿笑说着,但很快就收敛了笑容。只见他又说道,“关西百年烽烟,于今终于是到了尽头。澶渊之后,河北得享七十年太平,现在也该轮到关西军民休养生息了。不过西军虽是精锐,可若马放南山,几年之后,也就泯然众人了,届时如同河北禁军一般,国家忧急之时,如何派上用场?”

“也不是不能用。”韩冈皱着眉头,“河北禁军训练一下,还是可以独当一面的。河东的官军就是如此,河北军也不会例外。”

吕惠卿冷笑了一声:“仓促训练,又能如何?关西,有良将强兵,又能驱使党项,故而胜得轻易。而河东,虽然一开始就是危局,但河东精兵是玉昆你第一次任官河东时就开始训练的,此番虽败,却非战之罪。只有河北的禁军最差,跟京营相仿佛。若是令表兄领军侵入辽境时,麾下皆是西军的话,不会有此大败。”

韩冈摇头道:“终究还是不训练的缘故。字一天不练手变生了,全都马放南山,不要说几年,半年、一年就废了。不过自来练兵,没有比战场更合适的地方了。西军的精锐,是用一年数次上阵换来的。现如今,元丰新约既已议定,短时间内无论南北都不会去破坏。没有战事,练兵也无从谈起?”

吕惠卿轻轻摇头,韩冈其实心中早有定见,现在只是装傻。

“大宋周围,可以用兵的地方多得是。不信玉昆你没有考虑过。”吕惠卿说道。

之前在王安石家,有许多话没能说得很细,但现在时间正好很充裕,可以稍稍详细的说上一阵。

韩冈的心思,吕惠卿看得清楚,却又感觉很模糊。说是清楚,韩冈一心要推广气学,这一点,吕惠卿早就看明白了。但说他模糊,却也的确模糊,韩冈的气学发展下去,究竟会变成什么样子。这一点,吕惠卿还没有想出个结果来。

不过现在既然要出外,一段时间内都没有竞争的必要。现在顺着毛捋,暂时倒是不难应付。

韩冈这个人,只要不主动去招惹他,几乎是无害的。就像是刺猬只要把刺竖起来,谁上来都要吃个大亏。

当然,刺猬不会经常忘水里丢石头,让人没法儿安生的过曰子。吕惠卿也明白,韩冈只是现在稍稍消停了一点,过些曰子,又不知会做什么事了。

种痘法就是韩冈弄出来的事,还有军器监各种行之有效的产品,还有方城山中的轨道,都是韩冈弄出来的事。只要稍稍放松一下,韩冈就会将手伸到各个角落,就是吕惠卿也想不清到底是从哪里来的那么多奇思妙想。

不过吕惠卿很快就放弃了猜想,安心想用韩冈的发明就够了。

韩冈也有些感慨,吕惠卿当真是想要在河北做一番事业,之前的猜测都有些以小人之心度君子之腹。

不过吕惠卿可不是省油的灯。王安石一走,就开始推行手实法。现在镇守河北,纵然不是宣抚使,但能动用的资源也不是等闲,想要做出点事情来还是很容易的。

但这关韩冈什么事?那是两府要操心的。吕惠卿要是能分心在两府身上,韩冈双手支持。

两人各有索取,很容易便达成了协议。自不会要书写合同,只要签字画押。点点头,在夜幕下,分道扬镳而去。

吕惠卿走了,走得十分的干脆利落。让很多人失望不少。

而沈括回京来了,苏轼也回来了,还有苏轼的死对头李定,官复原职的御史中丞,竟然跟苏轼同时进京。连两人所乘坐的官船,都是同时入港。去迎接两人的官员,见面时,少不了有些尴尬。最头疼的就是章惇,李定和苏轼他都要迎接,偏偏还撞上了。

萧禧却还没走,他还在等向皇后的第二次接见。

“这个秋天还真热闹。”韩冈拿着墨香阵阵的《自然》新刊,说得事不关己。

“京城一向热闹。”坐在韩冈对面,李信平静的说道。

