\section{第11章 城下马鸣谁与守(一)}

【还有一更,要迟一点了。】

“吕卿,可还有盐州的消息?”

“回陛下的话,昨日盐州来报,新修城墙已经增筑到预计的高度,只要再有十天便能彻底完工。不过就算现在西贼杀来,也不用担心他们能攻破城垣,党项人没有足够攻城器械,想攻破盐州这样的城池,只能长期围困。到时候,不论是环庆路,还是鄜延路都能派兵去救援,区区阻卜人只能骚扰粮道,却阻挡不了援军。盐州有金城汤池之固,内有必守之人,外有必救之军,党项人在风沙中跋涉数百里,岂是盐州官军的对手?”

大约一刻钟之前,赵顼才问过这个问题,再上一次是半个时辰前,今天自上朝后,对盐州最新军情的询问,大概重复了有七八次。每一次吕惠卿都是毕恭毕敬的详加回复,仿佛赵顼是第一次问起此事。

赵顼对于盐州军情的态度已经近乎于神经质——尽管吕惠卿并不知道这个词语——以赵顼近日来的表现,让吕惠卿不得不去猜想他的主君是不是有了心疾。

这绝不是胡乱猜测,前面的英宗皇帝、在前面的仁宗皇帝,都有得了心疾之后,胡言乱语以至于不能理事的时候。英宗更是在重病之下,让曹太皇得以出面垂帘听政。

现如今,太皇太后的病情越发的沉重,很有可能见不到元丰三年的太阳。一旦天子御体欠安,出面垂帘的必然是高太后。一贯反对变法的高太后掌权,面对病重的儿子,逼其退位,另立新君也不是不可能。至于新君是谁……自然不会是年方三岁的六皇子。

一想到这样的未来,吕惠卿就是不寒而栗。到时候,新党一脉还有几人能安居在朝堂之上。

值得庆幸的是的,天子的情况至少还不至于如此,也不见病容。除了隔着一阵就提一次盐州,就没有其他方面的问题。

天子正是年轻力壮,绝不会落到那种地步,吕惠卿为自己壮着胆子。徐禧也上表说城垣完固,必不致有失。只要守住盐州城,如今滋生在暗地里的一切魑魅魍魉都会烟消云散。

吕惠卿明白,在自己全力支持徐禧之后,自己的命运已经与盐州城紧密相连。这个时候,不可能让盐州撤军,只能咬牙坚持下去。

韩冈很早以前就说过,一旦官军夺占了银夏之地,瀚海就成了困扰党项人的天堑。这番话,是赵顼和吕惠卿同意徐禧方略的前提。

在吕惠卿看来,韩冈之所以要放弃盐州、宥州,只顾守着离国境最近的两座城池,并不是老成持重的反应,而是有私心作祟——韩冈之前一直都宣称攻占银夏是灭亡西夏的第一步,但现在却只要守住银夏之地东南角的银州和夏州,将盐州等区域置之脑后,怎么看都是有一份私心存在。

不论赵顼,还是吕惠卿,都曾猜度韩冈的想法。官军保住银州夏州,不仅是韩冈建言的成功,同时也证明了韩冈在战前反对速攻兴灵战略的正确性;而官军守住盐州,成就的仅仅是吕惠卿和徐禧。

吕惠卿当然不会愿意成就韩冈的名声,他选择了支持徐禧,可惜还是出了一点差错。

但如果说有谁能挽救这个局面,韩冈必然是其中之一

看来最好是要给韩冈写封信联络一下了。吕惠卿想着。恐怕韩冈最不愿看到的就是雍王继承大统,甚至有继承大统的可能都不会乐于看到。

把握到韩冈的弱点,吕惠卿觉得可以借用一下他的力量。韩冈如今镇守河东,面对日渐紧张的局势,他有很多选择。可以冷眼旁观,也可以遣兵援助,更可以在边境上闹出一点动静来。

打过了整十年的交道,吕惠卿依然对韩冈有着很深的忌惮,但对他的能力,则有着很强的信心。如果韩冈肯出手,至少能将如今的局面挽回一点。

被留下独对的吕惠卿离开了崇政殿,只剩赵顼在御榻上呆坐着。

盐州最新的军情什么时候能到?不由自主的,赵顼的心思又转到了战争的关键点上。

当决定守住盐州的时候,谁能想到辽国竟然光着膀子直接就上来了。事情没到最坏的地步是耶律乙辛并没有直接派遣宫分军、皮室军,仅仅是让阻卜人去援助西夏。

就这一点而言,代表辽国还没有立刻撕破脸皮的打算,耶律乙辛的目的很有可能就是众人猜测的要挟勒索——这段时间,辽使萧禧几次上殿觐见,都提到了提高岁币的要求——但谁能保证,当利用阻卜人没有达到目标之后,辽人不会赤膊上阵。

而且西夏竟然敢让成千上万的外族兵力进入其国中核心的兴灵。这是大宋君臣事先都没有想到过的。

在赵顼看来,党项人即便要投降,也会投降大宋,这样至少不用担心日日被勒索,而且还有各种各样的赏赐,就是进贡,也会有等价的回赐。而投降辽国,迟早被榨干掉。西夏的部族中,没人会愿意与每年都要索要走三万马驼的辽国打交道。

谁曾想西夏却还是将阻卜人引了进来。即便仅仅是阻卜,而不是契丹兵,可口子一旦开了,就像大堤上有了个小洞,迟早会变成一溃千里的缺口。

但赵顼无心去为西夏的未来忧心,他最想看到的就是西夏再无未来。所以在灵州之败后,他不甘心退守银州和夏州。

如果一场大战之后,仅仅是保住银州夏州的那一小片土地,那么又怎么对得起之前所动员的三十余万兵员,两倍于此的民夫,以付出的难以计数的银钱和物资?想想吧,以倾国之力,换回来的却是一个笑话,赵顼如何能甘心?

而且在辽国的支持下,西夏说不定还有更大的胃口,将边境线恢复到熙宁八年以前的状态,将横山南麓重新收回。如果当真出现了这样的要求,对于一心想要光复兴灵、收回燕云、恢复汉唐荣光的赵顼,不啻于当头一棒。

试问天底下可有割地失土、屡战屡败的天可汗?把唐太宗当成崇拜对象的赵顼,肯定不能接受这样的结局。

为了天子的颜面,至少要夺占了银夏和甘凉,将党项人压制在贺兰山下那一小片空间,如此才算不枉朝廷动员如此的人力物力。赵顼的脸面好歹也能挽回一点。

吕惠卿由此画出来的大饼,让赵顼心动不已。而且韩冈、郭逵都明确说契丹人——确切点说是耶律乙辛——带到鸳鸯泺,乃至南京道、西京道的二十余万兵马,绝不可能是用来南下侵攻的。

一声长叹,赵顼从御榻上起身,过去已经再难挽回,眼下就只能盼着徐禧守住盐州。

“官家,可是先去庆寿宫?”李舜举悄步走过来,提醒着赵顼下一步的行程。

赵顼点点头,动身往庆寿宫去。半路上,远远的看见前方的廊道,七八个人从前方横过,正往保慈宫的方向过去。

“是二大王。”李舜举在赵顼身边轻声说道。

应该是刚刚去庆寿宫探视过。赵顼想着,又往赵颢一行人的方向看了一眼,心中有几分不快,‘来得还真勤快。’

去了庆寿宫探视过昏睡中的太皇太后,赵顼没有接着去太后所居的保慈宫,他不想与赵颢打照面。

就在御苑的一片枫林边缘,脚下满地的红叶,面前是一片荷池,但池中只剩残枝枯叶。

扶着汉白玉雕成的阑干,望着萧瑟的水面,正想着盐州局势的赵顼,突然心口没来由的一阵剧痛如绞。紧紧的按着心口,身子也佝偻了起来。

李舜举觉得不对,立刻抢前一步,便惊见天子的脸色惨白如纸,额头上密密的出了一层冷汗。

他一下就慌了神,扶着赵顼,带着哭腔惊叫道:“官家!官家!可是哪里不适?”回头又冲身后的内侍们呵斥:“还不快去传太医!”

“朕没事!”赵顼挣扎着直起身,半倚着白玉阑干,坚持着不让自己倒下去,“朕没事,不要闹得人心惶惶。去取苏合香丸来。”

“奴婢知道了。”李舜举偷眼看了赵顼两眼,转过身,低声喝道:“谁捧着药,还不快点上来!?”

两个小黄门慌里慌张的快步上来,将自己手中的药箱打开来,捧给李舜举看。

天子在宫苑中行走,身后随行的内侍,从更替的衣物到安坐的马扎,从钓鱼用的钓竿到射猎兴致起时的弹弓,都会随身携带着,天子想要,立刻就能拿出来。

如菓子、蜜饯,熟水、凉汤,等零食饮品,同样有专人负责携带。而一些急救的药物,如苏合香丸这样芳香温通、能治一切气症,中风、中暑、心痛胃痛,诸般病痛皆可化解治疗的备急难的圣药,更是常备着。

赵家的几代天子都曾犯过卒中,跟在赵旭身后捧药的小黄门手里,就有专治卒中【脑中风】、心痛或是中暑等毫无征兆的急症发病时所用的丹药。

在几个小银盒子中,慌乱中的李舜举发现了苏合香丸、至宝丹、灵宝护心丹等合用的药物。他慌慌忙忙的选了苏合香丸,双手颤抖着捏开药丸外面的蜜蜡,倾入已经斟满烈酒的银杯里,也等不及用烈酒将药丸化开,就火烧火燎的递到赵顼的面前。

“官家,这是苏合香丸。”李舜举服侍着赵顼服了药,抚着赵顼的背,轻声问道:“官家,可好一点了?这里还有至宝丹和灵宝护心丹,要不要也服一颗?”

赵顼服了药,就闭起眼睛。过了一阵,感觉稍稍好了一点,摇摇头:“没必要吃那么杂,苏合香丸就够了。回了福宁殿,再将杨文蔚唤来。”

李舜举明白赵顼的心思,又低声问道:“官家,要不要坐肩舆回去?”

“在这里歇一会儿,朕走回去。”赵顼硬咬着牙,忍耐着脑中的晕眩,这时候,决不能有半点弱势。

就在荷池边,赵顼歇了好一阵,终于有了力气,在李舜举的搀扶下慢慢的往福宁殿去。正走着,他忽然道:“李舜举。”

李舜举低头应承:“奴婢在。”

“你且去盐州体量军事,如军情危殆,以全师为重。”

