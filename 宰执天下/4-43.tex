\section{第九章 鼙鼓声喧贯中国(三)}

十月初的汴京城,夜晚越发的寒冷起来。寒风呼啸着,让还没有来得及换上冬衣的人们,浑身都冻得如坠冰窟一般。不过冯宰相府后花园中的池畔小厅中,火盆中烈焰熊熊,让厅内温暖如春。

“还是煅烧过的焦炭火旺,比起石炭要强出不止一筹了。韩玉昆得判军器监,发明众多,可谓是如鱼得水。”说起韩冈,冯京言笑自若,似是心中已经毫无芥蒂,“当初他不肯接下中书五房检正公事,世人都以为他畏难,谁能想到他自有腹中锦绣。”

与冯京对坐的蔡确则是笑道,“只是为了霹雳砲泄露一事,天子心里可是很有几分不快。保不准哪天辽人手上就有了飞船,皮室军人人身着板甲。”

“韩冈硬是不认罪,天子肯定少不了心头有气。但现在只是小罪,若是以为认了无妨,日后板甲、神臂弓泄露出去,那就是重罪了。”冯京将温好的酒倒入杯中:“所以说韩冈这次也算是聪明了,宁可触怒君上,也不愿给日后留着后患。”

“说得也是。”蔡确点头附和,“现在不将有罪无罪确定下来,日后有得苦头吃。”

尽管在西夏的军队中出现的霹雳砲,是韩冈尚在河湟、并没有开始宣传格物致知的时候,就已经用在了阵上,应该也是在那个时候泄露出去。但韩冈却不能辩解说他传播格物之理与军器泄露一事无关。万一日后西北二虏的军阵中再出现飞船,士兵装备上板甲,那时又该怎么辩解?这是明明白白的陷阱,韩冈当然没有蠢到跳下去。

而且为了日后着想,韩冈也必须逼天子给个说法,因为格物理论的传播,让敌国学去了霹雳砲、飞船、甚至雪橇车、板甲的制造方法,到底是有罪还是无罪。

冯京拿起酒杯浅尝一口:“不论对错,天子现在都少不了要靠韩冈掌管军器监。他有恃无恐,自然敢于顶撞天子。”

不论从任何角度,韩冈肯定是有罪的。但是,朝堂上得出的结论不是看谁对谁错,而是看需要。天子觉得谁对朝堂更重要,谁就能留下来。过去也不是没有宰相犯了重罪,弹劾他的御史掌握着再充分不过的证据,但天子就是站在宰相一边,而让御史出外。

“只是细细算来,还是有些得不偿失……”蔡确一向看重天子的看法,韩冈的行为实是愚不可及,“韩冈虽然逼得天子改认其无罪,但终究还是有失圣眷的举动。”

“得失与否,各由心证。”冯京笑道:“我们看来冒着失去圣眷的危险是得不偿失,但在韩冈眼中,说不定还是合算的,他不顾毁誉也要推广气学,也算是用心良苦了。”

身为宰辅,冯京不便出外饮宴,只在家中请人喝酒。王安石复相之后,蔡确没有刻意与冯京疏远,作为御史台的主要官员,不与新党为敌就是善意,太过贴近王安石,反而惹祸上身。倒是与冯京,那就是亲戚间的往来,并非曲意逢迎。

不过冯京已经做了一年的宰相,蔡确在京城中已经拖不下去了,两次应付场面式的上书请郡,再来一次,多半就会给批准了。他不是吴充,能得天子信任,与亲家王安石对掌二府。冯京的相位一时间无可动摇,蔡确自知今年之内必然要出外任官,要找一个好差遣,就要靠冯京来帮忙。

酒过三巡,两人的话题已从韩冈身上转到了御史台中。

“邓绾前日荐蔡承禧为御史,今天应是他入台的日子吧?”得了蔡确点头确认,冯京便问道,“持正你观其人如何?”

蔡确摇摇头:“没看清他长相,只看到了临川二字。”

冯京笑了一声,也是摇了摇头。

与吕惠卿、苏轼、张载等人一样,蔡承禧也是嘉佑二年的进士。不过这一点不足为奇。真正惹人注意的是他的籍贯——江西临川。

临川是文学之乡,在江南西路也是以进士迭出而知名。蔡承禧的父亲蔡元导甚至中过制科里的茂材异等——制科第三等任官,等同于进士科状元,难度可想而知——只是因为触犯律条而被夺了功名,但他十几年后重新出山,又是轻轻松松的与儿子一起考上了进士。可如今一旦说起临川人中最为有名的一位是谁,则没有第二个答案,当然就是当今的首相,新党的核心、主持变法的宰辅王安石。

“朝廷之设御史,就是为了监督百官。所以宰相无权举荐御史,只能有御史台本身和翰林学士来荐,但蔡承禧的任命,少不了有王介甫的授意。”蔡确板着脸,也不避忌冯京同样是现任的宰相。

“若御史台也以王介甫马首是瞻,东府就当真成了王介甫的一言堂了。”冯京对蔡确的表态很是满意,“只可惜持正你已难在御史台中久留。”

蔡确没有接口,这就要看冯京怎么打算了。

其实对于冯京的小心思,蔡确私底下是不屑于顾的。天子喜欢开疆拓土的光荣,如果种谔能重夺罗兀城,再一次证明了新法的好处,又怎么会让旧党上台秉政。

自从新党秉政后,天下的变化——尤其是军队的变化——天子肯定是都看在眼里。韩琦、富弼、文彦博一干元老秉政时,对西夏胜果如何?如今官军对西贼的胜果又是如何?在登基后,就穿着金甲给太皇太后看的皇帝,怎么可能会抛弃新法?只要没有动摇到他的帝位,天子肯定会将一项项法度坚持下去。

蔡确不会依照新党、旧党的划分来选边,他只会站在天子一边。如果天子喜欢旧党,他就会贴着旧党,如果天子要坚持变法,那他就是新法最坚定的支持者。只要让天子满意,冯京能为相,他蔡确亦能为相,仅是要少待时日罢了。

看了蔡确一阵,冯京重又开口:“种谔领军北攻罗兀,北人那边需要遣使分说,只是这国信使的人选尚未选定。吾有意荐持正为正使。但持正乃闽人,不知耐不耐得风寒?”

西夏向大宋称臣,同时也是辽国的臣子。如今鄜延路攻打横山,照理也得向契丹人解释一番,故而要为此派出国信使。

蔡确不畏寒,他只怕坐冷板凳,出使辽国虽然辛苦,但只要不辱使命,带来的回报也是丰厚无比。现在他是殿中侍御史,出外也不应能得到上等官阙,但等他回来,必然有个更好的未来。他向冯京拱了拱手:“向知北地风物有别南土,愿往一观,亦为君解忧。”

冯京点头笑道:“有持正的话,我就放心了,明日上殿面君,我荐持正你为国信使。”

一番小酌之后,冯京亲自送了蔡确出来。

两名仆从手持灯笼在前引路,冯京和蔡确穿过回廊,走在疏影微斜的院落中。抬头仰望初冬的夜空,蔡确的两只脚顿时就定住了。

“怎么了?”冯京回身问道。

蔡确眯着双眼,抬起手遥遥指着南方的天空:“彗星!”

……………………

白天下了场雪,只是雪不大,只有两寸而已。且天气转眼就放晴了,到了晚间,漫天星子在天幕中闪烁着彩光。

这让种建中很有些失望。

种建中原来是在三班院任职,但在战前被他的叔父请调入鄜延路中,不过是以文官的身份。但只要随军,文官武官都能沾上一份军功。种家世代将门,第一代的种放又非进士出身,在文官系统中毫无根基。为了让没有进士头衔的种建中能顺利转官,也是破费了一番思量。

官军如今已经进逼至罗兀城下,浩浩荡荡的两万大军,已经将无定河谷给填满,只是没能将罗兀城给围困起来。

西夏人当道设了一个寨子,与罗兀城成掎角之势,城、寨之间只隔了百步。想围城,就要将寨子也围起来,想攻寨子,则需要同时赌注罗兀城的城门。而且更重要的,就是驻扎在山北的铁鹞子,战事一开,随时都能赶来,这让种谔不能放胆攻打罗兀城,只能先等着后方的霹雳砲运上来再说。

另外,他还盼着下雪,谷中一寸雪,山路上能下半尺厚,让西贼援军不能速至,就可以腾出手来,安心的用投石车在一城一寨上,砸开一条路来。

只可惜雪太小了。

踏着薄薄的一层雪,脚底下响着咯吱咯吱的声音。种建中向山坡上走过去,那里站着一个瘦高的身影,是游师雄。

游师雄是鄜延路经略安抚司的机宜文字,现如今正在种谔的幕中。当年他领军在邠州城外埋伏了叛乱的广锐军,保住了邠州,依靠这份战功,几年下来也是走上了晋升的快速通道。

“景叔兄可是在夜观天象,”种建中走过去,开玩笑的提声问着,“不知明日阴晴如何?”

“阴晴难测,不过吉凶或可知……”游师雄顿了一顿,“在看彗星!”

彗星!种建中心头一惊,抬眼顺着游师雄的视线方向望过去,立刻在南方七宿的轸宿星区中,发现了一道曳着长尾的星光。

果然是彗星!

“慧主兵灾。出天车,犯荧惑,长沙不显,双辖无光。”不知天文,不知地理,不可为将,种建中的声音沉了下去,“此乃兵丧之兆,难道是南方有变?”

