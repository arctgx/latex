\section{第13章 晨奎错落天日近(13)}

韩冈今日在文德殿常朝押班。

不厘实务的朝臣才会参加文德殿常朝,一般由宰辅一员押班,而天子或太后不会到场。

等童贯出来传话太后今日不视朝,韩冈率领群臣参拜过御座后,便被几位自感说得上话的朝臣给围上。

大宋官场虽然官多缺少,但那主要还是指低品的选人和小使臣,朝臣候阙的情况很少。没有实职差遣的朝臣,基本上不是宫观使,就是皇亲国戚。开府仪同三司的数量,比两府加起来还多。而节度使、观察使之类武臣中的贵官,也基本都在这里。

能围着韩冈的朝臣,基本上不是资格老到可以在宫观中拿张长期饭票,就是跟韩冈攀得上关系的外戚。至于那些有着国王、郡王之位的宗室,倒是一个个避之不及——军国事他们绝不敢掺和。

辽军叩关的消息已经传遍了朝堂,理所当然的让每个人都担心起来。之前喊着打过白沟去,拿下析津府的声音在朝堂中一点不比外面要小,可当真辽师兵临城下,慌张的还占了大多数。

使节叩关不打紧。只要朝廷不承认耶律乙辛的身份,就不会承认他派出的使节。辽方为此事移牒质问,边境上也会拒之门外,他们进不了关门一步。

可大军叩关就不一样了。

虽然雄州的急报中,辽人还只是大军南下,质问官军为了越界击杀,可大军顿兵界上,一言不合,难道会打道回府不成?

不过从这件事上看,辽人的脾气已经很收敛了,换做是以前,肯定是不管三七二十,直接杀将过来。弄个几十条人命,来祭祀亡魂。现在懂得先礼后兵了。

李格非他这个殿中侍御史里行也在殿上,正好听见韩冈在对围着他的人说着话,“些许小事,何须惊慌。”

跟韩冈对话的那人,李格非认识,是曹太皇的侄儿曹训——殿中侍御史有整肃朝仪之职,弹劾的潜在对象不能不认识——但旁边有人插话,曹训的话听着有些模糊,断断续续的能听得出还是在说辽事。

“亏得韩参政好脾气啊。”

身边传来一阵幸灾乐祸的轻笑声。

台中同僚的取笑,李格非恍若未闻。

消息灵通的朝臣,哪个会在这时候自找不痛快?现在围上来的都是不厘实务的,对宫中的消息虽是灵通一些,可对朝局变化的了解,却远不如在垂拱殿参加内朝的朝臣,生生犯了韩冈的忌讳。

而韩冈的回答,李格非听得很清楚。参知政事的声音低沉却清晰无比,在他说话的时候,也没人敢于插话,“一来,辽人尚未来攻,只不过是万余骑兵驻扎得离边界近了一点。二来,便是打过来了,边州中也还有精兵强将抵挡,国中的局势远胜两年前,吾知诸君心忧国事,不过大可放心,勿须忧虑。”

“可边境战乱一起,生民必受灾劫,士农工商,无论哪个都要受苦了。”

‘士农工商!’李格非心中冷哼了一声,他知道,曹训想问的肯定是宋辽边境互市的问题。京城中的皇亲贵戚,在其中投下了不小的成本。但这种时候,

“边州黎民若当真遭受战乱之苦,朝廷岂会置之不理?自当给予赈济。”

不用亲眼去看,李格非都能想象得出,曹训那张圆圆的肥脸上,现在会是什么一张苦相。在河北边郡有生意的京师贵胄不在少数,韩冈虽然已经明说日后会让朝廷对他们的损失给予补偿,但以曹家为代表的、在北地有收益的皇亲外戚们,更希望的是韩冈能够站出来阻止战争。

韩冈也肯定不希望打起来,可现在他都不敢答应曹训的请求,连一点暗示都没有。李格非摇了摇头,可见韩冈面临的局面有多不妙。

“参政。”曹训仍在试图说服韩冈而喋喋不休,“若能免除兵戈,也就不需要赈济了。”

“仗是朝廷要打得吗?!”

韩冈似乎有了些怒意,反驳的声音严厉了许多。

想要避免战争,就得平息辽人的愤怒;想要平息辽人的愤怒,就得为那三名死者给辽人一个交代。或许整件事当真是吕惠卿背地里指使,但无论如何朝廷也不可能答应给他们一个交代的。

按照雄州上奏,一切的责任都是在辽人一方,尽管这多半是吕惠卿的手笔,可即便是韩冈也不可能会拆穿,而且也不一定能拆穿。

河北禁军的名簿中,只有名字而查无此人的比例多的能有三四成,少的也有两成,从这里面随便挑两个出来,说是给越界辽军所杀,直接就能搪塞过去。要是想把事情做圆满了,随便杀两个人,再换身衣服,这下连人证物证都有了。

韩冈还能怎么做?难道让刘舜卿去彻查?

任谁都知道,如果朝堂上是两党分立,真伪与否,只看两边的实力。

吕惠卿如此恣意妄为,可章敦偏偏倒了回去。韩冈在朝堂上的地位已岌岌可危,他能赢得了新党吗?

李格非很看好韩冈,毕竟有太后在。

若韩冈和章敦携手,能够将吕惠卿压在地方上不得回京,让王安石也不能继续干预朝政。可现在章敦又倒了回来,势单力薄的韩冈必然会借助太后的力量。

可眼下的第一回合,韩冈却必须先退一步。

望着韩冈,李格非心中暗叹,这只能怪韩冈自己没看对人。

章敦这么快就反倒回去,的确出人意料。但王安石和李定怎么说服的章敦,就很让人感到好奇了。能坐到章敦那个位置上,而且是靠功绩才干而不是天子的赏识,心志坚定一条肯定是有的。

只是李定性格严重,谏官门一向都畏其三分。李格非与他关系也没好到可以谈论个人**,不好打探到底是什么样的情况。

朝堂上要起风了,不是普通的风,而是能杀人的台风。

看好韩冈一派的结果,但李格非确信在他胜利之前,朝堂上必有一番大动荡,

是不是要找个机会离开御史台?

李格非想着。

……………………

韩冈来到垂拱殿的时候,为北方之事,向太后与重臣们已经讨论了很长一段时间了。

不过因为苏颂的主张,以及太后的坚持,一直都拖着没有得出结论。

看到韩冈过来,太后有了主心骨。待其参拜后,就在殿堂上叹息着,“参政你说说,不过是三个人,怎么就会走到现在这步田地。”

“如果不是如今的局势,就是杀了北虏三十人、三百人也不妨事!”吕嘉问站了出来,“前两年,北虏的首级拿了不少,并不缺皮室军和宫分军的。可北虏今日是早有预谋,有借口会来,没有借口同样会来,不给足岁币,他们如何会甘心?”

岁币。

辽人会因为岁币而南侵,这是朝堂上公认的理由。

韩冈曾经告知王安石、章敦、苏颂等人的理由,不可能拿到朝堂上来说。没有明确的证据,全凭韩冈的片面之词。他能够在私底下让王安石、章敦相信,却无法挡得住政敌公开的驳斥。

当向太后向韩冈询问意见,韩冈没有拿日本的金银去驳回吕嘉问的话,“不论北虏是否会举兵南来,也不论到底是什么原因,陕西、河东,最重要的是河北的边州,必须做好迎敌的准备,三军、粮秣、军械,绝不能有半点差池。”

此前王安石和章敦就已经反复表达了相同的意见,向太后却一直都犹豫着,可韩冈这么一说,她立刻就点了头,“……参政此言有理。诸位卿家意下如何?”

一片声的回应:“臣无异议。”

韩冈并不想过于借助太后的力量。他一直想要维系的是朝中的平衡,是整个官僚体系能够将皇权排除在外,而不是想成为一个得太后宠信进而控制朝堂的权臣。尽管这样也不差,但等到太后身故之后,不论在位的是哪个皇帝,朝堂上必有一番波折,甚至会连累整个气学。

即便是现在,他也会尽量避免借用太多。至少不会在事关军国的要事上,借用太后之力,来压制政敌。不是什么国事为重,也不是什么兄弟阋于墙而外御其侮,只是他知道,该这么做才对自己有利。

“那河北需要怎么准备?”

向太后又问着韩冈。

“河北禁军几乎都有对敌的经验,而且两年间还经过了加强,绝不输于辽人。粮秣这两年不断进行补充,按照去岁十一月,河北转运与常平二司的奏报,应有一年以上的存粮。军资方面,则可问枢密院。”

“苏枢密。”向太后跳过了章敦,点了苏颂回答。

章敦沉着脸,看着苏颂出班回话,“河北兵精粮足,陛下可无忧。河北边地诸州军库,年前刚刚经过点验,甲胄、弓弩、刀枪、箭矢皆如数,足以抵用。而在册军马总计八万三千余匹,亦如数点验造册。”

“火炮呢?”

“火器局生产的火炮,已经运抵河北的轻重榴弹炮有一百三十门,虎蹲炮两千八百八十门,大小炮弹十万余发,各式火药药包二十万个。”

说到这里,韩冈顿了一下。大宋的国力有多深厚?看这一年来火炮的产量就够了。

这还不是军器监的全力,如果到了必要的时候,只需从军器监中调动人员,火器局就能够迅速扩大,在保证质量的情况下将产能加倍,榴弹炮能做到一天一门——六寸的城防炮或是四寸的野战炮,虎蹲炮一天二十门。在这个时代,这个世界,没有任何一个国家能够与大宋的生产力相抗衡,甚至接近也做不到。

只是由于不逊于八牛弩的威力,维护和使用更加简单,使得火炮在军中广受欢迎。不止一名边臣,上书要求朝廷及早给他们装备上火炮——其中既有文臣,也有武将。现在的产量还是远远不能满足所有需求。是的,远远不足。但是有优先权的河北、河东边州,是不用担心的,武备一向是绰绰有余。

“只要运用得宜,不虞河北城池为辽军攻破。”韩冈补充说道。

“有参政的话,吾便放心了。”太后道。

再有一两个月,河北河流湖泊解冻,千里陂塘防线恢复作用,那时候,辽人的威胁就只剩现在的一二分了。只要能先守住一段时间,辽军将不得不退。这个道理,她也懂。

“既然北虏南侵在即,依故事河北事权当归一。”排在下首处的蒲宗孟出人意料的站了出来,“臣请陛下于河北设宣抚一职,统御诸军,镇抚路中,以备辽人。”

殿中众臣纷纷侧目。王安石也脸色微变。这个蒲宗孟跳出来实在太会选时候了。

“何须宣抚使!”苏颂立刻出班道,“与一制置使便可。北虏入寇河东,亦不过是设一制置使抵御,如今北虏尚在边界外,制置使已经绰绰有余。”

“北虏已是箭在弦上,战事迫在眉睫。”蒲宗孟道。

苏颂当即反驳,“韩冈制置河东时,不知北虏的箭射到哪里了?”

王安石沉着脸,蒲宗孟这是明帮暗阻。提议设宣抚司,最合适的时机是辽军开始进攻的时候,现在什么只要下令加强防备,

统掌军政,才能独占功劳。如果只是掌军事的制置使,河北转运使至少要分去三成的功劳。

蒲宗孟怎么都不受待见,但总是能够留在京城,眼光和赌性好歹是有那么一点点。

不过设制置使也已经有足够的意义了。没有韩冈认可,苏颂不会出来。是迫于形势,还是没有底牌了?

几乎所有人都在猜测着。

太后询问韩冈的意见,“不知参政如何看?”

“北虏驻屯界上,设制置使统掌军事,以御敌寇,是应有之理。不过不知陛下可曾想过,澶渊之盟后七十年,河北军民不识何为兵戈,为何这几年来,辽人为何总是南侵?”

“为何?”

韩冈瞥了眼王安石,“乃国是之故。”
