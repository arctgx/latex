\section{第一章 纵谈犹说旧升平(一)}

三月的汴水,草长莺飞,岸边杨柳依依,河上船行如梭。

此时风光正好,正是踏青的时节。

城中士子、百姓,乃至官宦人家的子弟,多有头簪鲜花,踩着青青的草皮,在河畔的柳树下漫步。丝竹曲乐悠然河上,那是妓女陪着恩客荡舟水面。河边有几处帘幕重重,以丝缎圈起一块土地,这是达官贵人家的女眷休息的场所。

不过苏颂今日带着儿子苏熹出城,却不是为了踏青。也没有往河边的僻静去处,而是来到了城外的码头边——他是来迎一位客人的。

五十多岁的苏颂在官场上沉浮三十年,如今也算是身居高位,一个集贤院学士就让几千几万的官僚一辈子都只能仰望,而他很快便要就任的应天知府一职,也是大宋四百军州中,排在前五的要职。

虽然在码头上,认出身穿常服的苏颂的人不多,但十几个身穿红袍的元随,就已经是人人侧目,都在猜测究竟是哪路神仙,能让至少是两制一级的高官亲自出城来迎接。好奇的人们很快就知道了究竟。码头上每到一艘官船,苏缄的一名元随酒会上前去高声询问,问着是不是邕州苏皇城的船。

皇城使是武职,为正七品,是四十阶宫苑诸使中最高一级,离横班也只差一步。但这个官职很显然远远比不上文臣中两制官,绝不够资格让人亲迎。只会是来迎接亲戚长辈,多半就是同样姓苏。朝中两制以上的贵官,姓苏的不多。熟悉朝堂人事的,很快就猜到了码头上这位高官显宦的身份。

时间一点点的过去,每一次询问,都是否定的答案,随着苏颂而来的元随们也渐渐没了精神。到了午时前后,伴着几声锣响,又一艘从南而来的官船渐渐的靠近码头。苏颂的元随照例上前,有气无力的喊话,“可是邕州苏皇城的船?”

“正是!”回答声中气十足,反问道,“可是苏子容苏学士?”

苏颂上前一步:“苏颂在此!”

一个须发花白、面孔黝黑的老头子很快就从船舱中走了出来,六十多岁的模样,脸上的皱纹差不多能夹死蚊子。不过精神矍铄,腰背一点也不像这个岁数的老人一般佝偻。站在上下浮动的船板上,不见身子动摇半分。

随行之人都有着一副晒得黝黑的皮肤,甚至有一个六七岁的小女孩儿,也是微黑的肤色。而且有好些个仆役明显的是岭南的相貌,显然是从南方进京来的官员。

苏颂一见那老头儿,便在码头上拜倒:“侄儿拜见二十六叔。”

“子容,不必多礼。”老头儿等着船板搭上来,忙走上栈桥,亲手扶起苏颂,上下打量着:“这可是多年不见了。”

苏颂执着老头儿的手,相看泪眼:“昨夜侄儿接到二十六叔让人从雍丘连夜送来的书信,真是喜出望外。前几次二十六叔上京,侄儿在外任官都错过了,今次当真是赶巧。”

“谁说不是?上一次见面,还是仁宗时候的事,都十多年了。”老头儿和苏颂一起叹了半晌,终于想起了什么,回头招了两名少年和那个皮肤微黑的小女孩儿:“对了,这是你的侄儿侄女。”随后就冲着孙儿孙女喝道,“还不来拜见你们七伯!”

苏颂坦然受了他们一礼,问着老头儿:“都是元哥儿的?”

“嗯,都是大哥的。”老头儿点点头,“二哥家的两个还小。这次上京,顺道让他们见见世面,总不能一辈子都在待在广南。”

河上一阵风吹来,老头儿眯起了眼:“还是春天啊,在岭南待得太久,都不习惯北方的清寒了。”

苏颂笑道:“二十六叔三年四诣阙,怎么还是没习惯?”

老头子随之一笑,带着一丝苦涩:“若是当真习惯了,我苏缄都不知该怎么回邕州【今广西南宁】了。”

邕州知州苏缄,今年春天又是奉旨诣阙。

熙宁四年,交趾就闹了一次,有消息说准备北犯,不过后来证明是虚惊一场。但当今天子,还是将苏缄调去了邕州。自从中了进士出仕之后,苏颂的这位堂叔在南方诸路做了近四十年的官,甚至还参与过讨伐侬智高叛乱的战事。论经验、论资历、论威望,在广南都是排在最前面的。有他守着邕州,才能让天子和朝堂放心。

不过这也是苏缄的悲哀所在。

流内铨外的阙亭中,每天都守着几百位官儿,就是不见人去成潼利夔、福荆广南这八路去。寻常官员去了这八路,升官倒容易——别说选人做知州,如琼崖岛上的那几个军州,甚至都有吏员权掌州职——就是很难再回来了。尤其是去岭南任官,一旦在那里待得久了,再想回北边来,几乎就不可能了。

成都府路、潼川府路、利州路、夔州路与广南东路、广南西路、福建路、荆湖南路,这南方八路,由于地理偏远,中原之人多不愿去其地任职,常年是官等人,而不是一般的人等官。许多职位都是空缺的,只要有人肯做,这些职位任其点选,点到哪个就能做上哪个——这就是指射。

既然南方八路职多官少,朝中有无人肯去顶替,那么那几路仅有的一些官员,就不得不来回转任,根本就没机会回来。如苏缄,他中进士近四十年来,基本上都是在南方几路来回调任。狄青平侬智高的时候,苏缄他就已经是英州【今英德】知州兼广南东路都监,二十年过去了,他现在是邕州知州兼广南西路钤辖。一辈子全都消磨在岭南了。

苏颂看着苏缄神色郁郁,心中也暗叹一口气。他的这位二十六叔运气不好,一考中进士,就被发派到广州任职。偏偏苏缄没有拒绝,而是接下了这个职位。自此之后,官场生涯就再也离不开南方了。

“二十六叔,侄儿已经在家中设了接风宴,还是早点进城。”

苏颂说着。苏缄也只比他长了四岁,但辈份就是辈份。见了族中排行二十六的苏缄,苏颂也必须恭恭敬敬的道一声二十六叔,自称也只能是小侄、侄儿。

苏缄收起心绪,笑了起来:“劳子容费心了。”

“不敢……对了”苏颂谦让了一句又道,“二十六叔奉旨诣阙,得先去城南驿留个名,不过行李可先送去侄儿家里,省得来回搬了。”

苏缄点点头,“如此也好。”

苏颂这一次也是上京诣阙,然后就出京任职。不过他十岁随父进京,家早就安在东京城中,并不需要住在城南驿。同样的,苏缄也只要在城南驿留个名就够了。

待儿子与远房的族兄弟见过礼,苏颂便与苏缄同上了一辆车,其余人骑上马,一起返身回城。

一行人沿着大道从城东一直往驿馆来,沿途的富丽繁华的街市,让苏缄的几个从来没有见识过京师胜景的孙儿孙女,看得眼花缭乱。

与苏缄、苏颂同乘了一辆车的孙女儿,虽然守着礼仪安静的坐在苏缄的身边,但一对乌溜溜的眼睛一直望着车窗外。待到马车进城,突然扯着苏缄的袖子,叫了起来,“大爹爹!那是什么?”

苏缄随着孙女儿手指的方向望了过去,只见几个或大或小的黑点,远远近近的浮在空中。不过他已经老了,眼力不济,眯起眼看了两眼,没看清天上飞的到底是什么。不过身边的苏颂,虽然也是年纪一把,也老花了,但他知道天上飞的究竟是何物。

“那就是飞船。”苏颂转头对苏缄道,“想必二十六叔北来的路上,也听说了吧?”

苏缄点了点头,又将眼晴眯成了一条缝,盯着天上的一个个黑点:“听说了,在泗州换船时就听说了。是王介甫的女婿做的吧?只是没想到当真能飞天。”

“没错,就是韩冈。”苏颂感慨着,飞船送人飞天的消息如同一块巨石投进水中,在天下掀起的波澜,就算猜也能猜得到,“素日见着虫鸟在眼前飞,想不到这辈子还能亲眼见着人上了天!”

“听说是在二月中旬,金明池里面上天的?”

“二月中是第一次。这一个月来,金明池天天都能看见飞船上天,已经有几十个胆子大的坐上去过了。”

“那些都是带着人的?”苏缄抬手指着天上一个个圆球状的物体,随着马车前行,离得最近的飞船已经看得很清楚了。

“能载人的叫飞船,不能载人的,如今的诨名是热气球。现在城中天上的这些,其实都是热气球。”

苏缄很是惊讶:“才一个月的时间,怎么造的这么多?”

“只是没人去想,当真要造起来其实再容易不过,而且也不是军器监造的。”苏颂说起来都觉得有几分好笑,“第一家是紧邻着兴国坊的王家铺子,听说就在金明池飞船试飞后的第四天,两个热气球就带着招牌上了天,接下来就是日日宾客盈门——也亏他们想得出——之后才半个月功夫,七十二家正店,如今家家门口都开始悬挂热气球。旧时是彩楼欢门,如今就是气球悬门了。”

