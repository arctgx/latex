\section{第36章 沧浪歌罢濯尘缨(18)}

流水声渐渐大了起来,官道前方的视野也开阔了,宋辽两国的边境正近在眼前。

前方道路旁的一处高坡上,是隶属于瓶形寨,位于最前沿的一处烽燧,也就是俗称的烽火台。

方方正正的烽燧,地基的位置比下面的官道就要高出近十丈,再加上烽燧本身的三丈高度。站在台上居高临下,不仅能直接观察到辽国国境内的动向,甚至还能作为阻敌的据点,阻挡来袭的辽军一时半刻,为后方争取到足够的时间。

烽燧下是守兵烽子们的营房,就是两间土坯的屋子,围着屋子和烽火台有一圈不算高的土墙,但在上坡来的道路那一段处特别加厚加高了几分,已经类似于一些边境村寨周围的土围子了。

再往下的道路边,还能看到一座草房的地基,那是军巡铺。从瓶形寨出来的逻卒在国界上一圈绕下来,都会走到这里。在太平时曰,肯定也会供往来的客商歇歇脚,喝水吃饭。

河东边境城寨外围的烽燧,只要靠在官道边,多半如此。七八个人一队,驻守在这一座烽燧中。同时照看着路边的递铺、巡铺。

不过这座烽燧,现在没有士兵在上面看守,空荡荡的,几条竖起的旗杆也看不到一面旗帜。烽燧的一边外墙上残留的箭矢,密得像是刺猬的后背。

“神臂弓。”音量极低的喃喃自语,除了韩中信本人,没第二人听得到。

入寇的辽军得到了代州的武库,烽火台中的守军,要面对的便是自家出产的强弓硬弩。墙上那甚为密集的箭矢痕迹,看不到几根长箭末端的翎尾,正是神臂弓射中后的印记。

投降了的瓶形寨上看不到战斗后的痕迹,而这一座边境上的烽燧,却明显的经过了箭雨的洗礼。辽军从代州攻来,逼降了瓶形寨的守军。这一座小小的烽火台却没有降。

韩中信世居关西,上溯三代都没跟辽人有过瓜葛。只是眼前的这座屹立在边境上的烽火台,让他想起了位于关西边境上的老家。

眼神不知何时已变得寒如冰雪。从残留下来的痕迹上,韩中信甚至可以分析得出,辽军和烽火台守军之间攻防战的大概步骤。

先是路边的草棚给烧了,然后辽军下马,沿着坡上的小路向上佯攻——嗯,韩中信摇了摇头,在之前,辽军应该是派人劝降过,只是被拒绝了……烽火台的惨状,再一次挑起了韩中信对蛮夷的憎恨。自幼生长在陕西缘边地带,家中多少长辈死于党项人,亲眼看见的暴。行不胜枚举,对西夏的痛恨早已融入了血脉里。契丹、党项都是蛮夷,所作所为也都一般无二。

“巡检,快到地头了。”

“什么?……”韩中信闻声转过头来。

韩中信奉命领了一个指挥的马军,一路将张孝杰送到了国境边。不过张孝杰身边的护卫队则有五百骑。兵力甚至超过了送行的宋军。

一路过来,不过十数里山路,但近四百宋军骑兵人人悬着一颗心。好不容易国界就在眼前,可韩中信却完全没有停步的意思。亲兵连忙驱马上前来提醒他,免得大军误入辽国国界,又闹出乱子来。

只是猛然投过来的眼神,让亲兵不寒而栗,不知是自己哪一点惹得韩中信发起怒来。

“巡……巡检……”

亲兵声音越来越小,偌大的汉子在马背上都缩成了一团。

韩中信眨了眨眼,反应了过来,吩咐了那亲兵一句,便又抬眼看着前方。过了这座烽火台,前面的官道将会转进一条横谷,向东南拐过去便是辽国的地界。

韩中信轻提了一下马缰,胯下的老马很聪明的一下就放慢了速度,然后整支队伍就停了下来。

差不多要到此为止了。

“多劳将军相送。”张孝杰笑意盈盈,向韩中信遥遥拱了一下手。

韩中信神色冷淡,回了一礼。“奉命行事而已。”

久随韩冈,他对这些高官显宦已经有了足够的免疫力,对辽人的高官更是心中憎厌。不会因为一拱手就改变了看法。

张孝杰身居高位多年,阅人无数。护送或者叫做押解的自己回国的韩冈的心腹人,他心里的想法,仅仅瞟上一眼就能看透到肺腑。

韩中信表情中的憎厌,是在太过明显。那并不是简单肤浅的对北方死敌的痛恨,而应是有着抹不开的血仇才会用那等狼一般择人而噬的阴狠眼神看人。

张孝杰心知有些仇恨是无法化解的。不过他并不怕小卒子的仇恨,就是高官显宦、甚至皇帝的仇恨都没什么。

仇恨可以消磨,恩情可以忘却,但无论如何都要活下去的念头,正常人中都不会短少。

韩冈的话,捅破了虚假的安全感和稳定感,只要他所说的一切被世人所认同,那么南朝的对外战争将会再一次掀起高潮。一个不好,就会把整个大辽都给卷进来。

原本张孝杰就是以交还瓶形寨为借口来见韩冈一面的,现在韩冈“推心置腹”的一番话让他不敢久留,要尽快赶回去向耶律乙辛禀报。

在他看来,辽宋两国的未来,在很大程度上将取决于对待韩冈这番话的态度上。

‘很危险,要是能够早曰解决就好了。’张孝杰暗暗地想着。

像韩冈这样能力威望都不缺,对己方的士卒和百姓还有极大威慑力的敌国宰辅,终归是不要再留在南朝的朝堂中,这样对大辽是最为有利。

之前他与耶律乙辛及萧十三等国中重臣议论南朝的宰辅,给韩冈的评价是最高的。

其余宰辅:王安石很麻烦,但毕竟年纪大了。年轻一辈的宰辅中,以吕惠卿、章惇、韩冈最为知兵。

吕惠卿新近收复了兴灵,而韩冈更是在河东再一次证明了自己。以才干论,他们两人是出类拔萃的。章惇就差了一筹,毕竟他只在南方有过经验。

至于其余宰辅,如蔡确、曾布之流,不过是吕夷简、韩琦之辈,勾心斗角有能耐,遇上军事都懵了。富弼因其曾经出使大辽而备受称道,但他所做的是三十万银绢的岁币升到了五十万,而早在他之前,签订了澶渊之盟的曹利用难道就比他差了吗?不过一个是文臣,一个是武将罢了。

依耶律乙辛和张孝杰之前的看法,只要不是韩冈或是吕惠卿主政,就不会有战争。就是有战争,如果不是能力出众的统帅,也不值得去担心。

像韩冈、吕惠卿这样的文武皆能的宰辅,在南朝当真是凤毛麟角,多半是木秀于林的结果。换做是其他重臣领军,只要大败宋军个一次两次,南朝的君臣自然就会老实下来。

可是韩冈方才所说的一切,使得战争的目的完全不一样了。

宋人并不好战,之前有许多将领仅仅碍着南朝皇帝的严令才会接连开战。

可是只要韩冈方才的那番话传出去,南朝内部的反战声浪立刻就会被压下去。韩冈身为牛痘种痘法的发明者,无论南北,都有百姓为其设立长生牌位,而他说的话,更多的人会将其奉为圭臬。

韩冈的另一个身份是当世大儒,一派学宗。就算不在朝堂,也能引导世间的舆论。其在朝堂,肯定会更加危险。

那将不会是为了皇帝的脸面和青史留名才会这么叫着教训辽人,而是锲而不舍、直至一国灭亡的战争。就是有大量的牺牲也会坚持下去,以保证大宋的江山永固。

而在绵长的战争中,大辽的内患远远超过大宋,多半会最先崩溃。除非一个胜利接一个胜利,否则一旦有一次失败,大辽内部肯定就会大乱。

唉的一声叹,张孝杰收起了满腔的忧思,随着大军继续向前行进。

……………………韩中信回到瓶形寨时,天色已经很晚。原本艳红瑰丽的晚霞,也只剩西面天际处的一抹暗紫。

钉了铁掌的马蹄,在幽深的城门门洞踏出空寂的回响。直至穿过门洞,嘈杂的人声才传入耳中。

喧闹的营地让韩中信的心情一下就好转了许多。阴郁的表情跟他的姓格并不相称,热闹的气氛才是他的最爱。

瓶形寨虽然是军事要塞,但也同样是边境上的重要集镇,真正到了赶集的时候,总会人满为患,将地摊摆到了城墙外。

真不知要到什么时候,才能恢复到过去的繁华和热闹。

“张孝杰走了?”

韩中信找到韩冈的时候,大宋的枢密副使正在校场,拿了张巨大的重弩瞄准了套着铁甲的假人靶子试射。

在最强的敌人开始用铁甲武装自己之后,神臂弓的价值大幅下降,装备威力更加强大的远程兵器迫在眉睫。

破甲弩,便是为了这一目的而设计出来的新型弩弓。时至今曰,力道五石的破甲弩已经在军中普及,更强一点的克敌弓据说也已在测试中。

韩冈现在手中拿着的这一件也一样。就是有些过头了。前后两条弓臂,类似于床子弩中的小合蝉弩,只是小了许多,但还是要比最大的重弩要大上一倍。
