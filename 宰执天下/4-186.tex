\section{第30章 狂潮渐起何可施(下)}

【向2011说再见,迎接2012的到来,不知船票各位买了没有?待会儿还有一更。】

韩冈进城的时候都已经中午了。

就在城门口,撞见了守着自己的家人。一行人便骑着骏马,在东京城中穿街过巷。走到快到州桥时,队伍中就分出几个人来,赶去宣德门帮韩冈报到——韩冈进京是转任前的入觐,要先去报名等待轮对——不过他自己还是直往家中行去。

按道理韩冈应该亲自去宣德门报名,毕竟规矩如此,不过他倒是没太放在心上。这样的错误,许多臣子都犯过,算不得什么大事。

以韩冈如今的地位、功绩和声望,犯点小过错,背上两三份弹章,才是件能让人松口气的好事。不过恐怕赵顼都没脸看着在别妻弃子在广西辛苦两年的韩冈背上这个罪名。

韩冈一行人,身上都穿戴着简朴的行装,看不出是当朝重臣的模样。不过进了家门前的巷子后,将他认出来的一下就变得多了。

一支不大的队伍,却惹得人人侧目。韩冈就在一路的注目礼中,看到了久违的宅院大门。

朱漆的大门,让韩冈心脏跳得快了一些。韩府的正门此时已经中开,韩家的仆役在管家的带领下,迎出门来,在门前跪了一地。

就在门前,韩冈翻身下马,两步跨上五级的石阶,又是两步跨进家门。就在院中,王旖领着周南、素心和云娘,领着一众婢女,盈盈屈膝,向韩冈道着万福。而几个已经能自己走动的儿女,也一起跟着向韩冈拜倒。

韩冈先是一把将王旖扶起来:“辛苦娘子了。”

朝思暮想的脸庞就在眼前,王旖抿着嘴,已经是泪水盈眶。

韩冈又一手一个的将周南、素心、云娘都搀起来,“这一年多,也是苦了你们了。”

周南、素心也都是泪中带欣喜的笑意,如雨带梨花,颜色动人无比。

“三哥哥……”云娘细白的手指则绞着韩冈的袖口,眼中的珠泪不停地从脸颊上滑落。看得韩冈的心都痛了起来。

对妻妾安慰了一阵,韩冈又转过看着他的几个儿女。

长子韩钟、次子韩钲,此时都已经快到了该入学的年纪,行事的礼节都是自幼便被培养,向韩冈行礼时一板一眼,只是少了一份亲近。韩冈暗叹自己离开家的时候太多,陪着儿女的时候太少,疏远的都不像是父子了。不过韩冈一向疼爱的女儿,倒是一点也不生疏,缠着韩冈要抱,让他欣慰不已。

至于韩冈一年半前离开京城时刚刚出生的三子、四子,以及当时还在王旖肚子里的幼子,此时都已经在牙牙学语。却也是对韩冈很是陌生,当乳母将他们带过来,韩冈要抱他们的时候,都是一下就哭了起来。韩冈这个做父亲的也是一阵难堪。

只是儿女俱全,也足以让他感到欣喜和安慰。

不过二十六七,就已经有了五子一女,这其实算不了什么,但每一个都是长得健健康康的无病无灾,这却能算是一个不大不小的奇迹。韩冈的子嗣情况,连皇帝都要羡慕——赵顼也就在今年年初的时候,才有多了一个儿子,还不知能不能保得住。

时隔一年半,一家团聚在旧时的宅院中。

“不知官人能在家中待上多久?”

“好歹要歇到上元节。”韩冈说着。

“官家可不会让官人歇息。”

“当是不会。”韩冈搂着妻妾,笑道:“皇帝不差饿兵。为夫可是整整饿了一年多,强差出来,也上不了阵。”

“饿了一年多?!”云娘被唬住了,紧张看着韩冈的脸,“是瘦了好多!”

但其他三人可都明白,王旖和素心的脸一下又都红了,羞赧的,看着韩冈的。而周南钩钩韩云娘的袖子,凑到她的耳边嘀咕了几句,然后就看到云娘晶莹玉润的小耳朵就蹭的红了起来。返身抱着周南,就留给韩冈一个后背。

在广西,韩冈也不是没沾荤腥,只是隔着一段时间才有一两次调剂而已。这两年来,每天有忙不完的公事,一日也不得闲。

而更重要的是,他被养叼的胃口实在适应不惯南方的风味,而他本人更是没到饥不择食的地步。平日里多是演练拳脚、习练弓箭枪棒,作为消耗多余精力的手段,所以看着倒是瘦了,但身子骨,可是用广西从不缺乏的牛肉和海鲜将养得精力十足。

正如韩冈所说,当真是饿了一年多,饿得眼都绿了。

拉着云娘的手,又一把抓住害羞的想要逃开的王旖,素心、周南乖乖的跟在后面,竟是一起往内间去了。

感受着身边的四位妻妾温香软玉般的身体,韩冈他庆幸着,他终于可以好生歇息一段时间。

朝堂上正逢变局,自己除了几天后,礼仪性质的正旦大朝会上能见到天子以外,想指望赵顼能一听到他的名字,就派人来招他入宫,那几乎是幻想。

毕竟,作为一枚秤砣,他的份量已经太重,但想成为拨戥子的手,却还有一段距离。

……………………

韩冈抵达京师的消息,赵顼没用一个时辰便收到了。

只是他考虑了一番之后,便决定将韩冈暂时冷上一冷。如今四方安定,也没有什么紧急军情需要处置,不需要急着召他入宫询问。

韩冈是能臣,以他的功绩,不让他越次入对,的确会伤了人心。不过比起朝堂上的局势来,这点小事,赵顼还是只能放在后面了,当真不算什么大事。

要是韩冈在觐见时,帮着吕惠卿和章惇说些什么,赵顼可就会陷入两难了。直截了当的拒绝,比起现在的拖延会更加伤了人心;但若想含糊过去,这个态度被朝臣解读,那就会给他目前想要达到的目的,带来不可预测的变数。

至少在眼下,赵顼要极力维持他所做出的人事安排,直到朝局彻底稳定下来。要是韩冈在这时候插足进来,局面可就难以收拾。

一阁学士,不论什么资历、年龄,都已是重臣中的重臣,只是略逊于朝堂上的十几人罢了。外放的诸多经略使中,有学士资格的都没几个,也就侍制、直学士罢了。比韩冈强的,都是些出外的老臣了。

而且以韩冈的才能,就算没有现在的地位,份量也已经足够重了,当初几次帮着王安石扭转朝局,就是靠着他过人一等的手腕和才干。这个时候,赵顼也只能选择将韩冈放远一点。

“你先下去吧。”赵顼挥挥手,让来报信的童贯下去候命。

童贯心中惊疑不定,难道韩冈失了圣眷不成?不至于啊,京西南北二路再次合并而成的都转运使,想想这个位置,天下诸多转运使中,也就是河北东路、河北西路合并为河北路,或是永兴军路和秦凤路,重新合并为陕西路,才能压得过去。

韩冈得此重任,怎么看都不像是有失圣眷的样子。但天子不再像过去那样,韩冈一到,便宣其入宫,这究竟是怎么一回事,以童贯如今尚算浅薄的政治智慧,一时间还是想不明白。难道当真是因为如今要起用旧党秉政,而刻意将其冷落不成。

弓着腰,倒退着走出殿门,在直起腰的同时,飞快的瞥了一眼高居殿堂深处面无表情的天子,童贯带着满腔疑云,离开了崇政殿中。

赵顼有赵顼的想法,也许在外人看来吴充是铁杆的旧党,但赵顼觉得吴充只是是跟王安石这个亲家拧着来罢了。赵顼能在王安石担任宰相的时候,让他的亲家担任枢密使,可就是看到吴充始终保持着与王安石相抵触的态度。

但在赵顼看来,一旦王安石离朝,吴充对新法的态度就会缓和下来。如果他当真与新法势不两立,与枢密院有关的保甲法、将兵法,怎么可能顺利实施?早就辞位请郡。

吴充做上宰相之后,肯定会改变他旧有的态度。赵顼深信这一点。只要自己维持新法,吴充就会默认并执行下去——王珪其实也听话,但赵顼更相信吴充的才能一点。

至于吕公著的任命,更不用担心。枢密使无权干涉属于东府权限范围的青苗法、免役法和市易法,他只能对保甲法和将兵法发言。但吕公著虽是旧党,却是难得的支持保甲法的一人——旧党中人,可也不是见新法必反。

且这项任命,也能让西北二虏暂时释疑。世人都知道,旧党大多反对用兵于外,这项任命,应当能安定辽国和西夏两国君臣的心。

新法如今已见功效,最需要的是稳定,将行之有效的新法条款稳定的执行下去,而不是再推行新的法令。吕惠卿就是不明白这一点,一旦让他做了宰相,定然会设法标新立异。赵顼不想看到这一点。

而且朝臣之间能互相牵制着才是好事,若是吕惠卿当政,想要维持‘异论相搅’四个字可就有些难了,没见他连冯京都无法容忍吗?

赵顼正盘算着该如何稳定眼下的局面,一则来自于辽国,简短到只有六个字的消息,让赵顼惊喜得失声而叫——‘废太子浚暴卒’。

