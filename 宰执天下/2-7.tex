\section{第二章 边声连角不知眠(三)}

【今天第一更,求红票,收藏】

“玉昆,依愚兄之见,你还是到了古渭便停脚,就在古渭寨建你的疗养院,等前面送人回来,为他们诊治。不能跟在向宝身边。”

“这小弟当然知道。只是向宝若真的要跟小弟随军同行,小弟也只能听命。小弟真有推脱掉的本事,明天也可以继续病在家里,不去理会李经略的命令了。”

韩冈被李师中亲自点将,把他发配到军中。韩冈很清楚李师中想做什么,也知道自己到了向宝军中,向宝会怎么做,但事实是,韩冈现在完全没有拒绝的可能。

“爹爹,你说怎么办?”王厚焦心的问着父亲。

“玉昆,你心中可有成算?”王韶皱眉想了半天,最后有些无奈的问道。

王韶是在经略司军议后就直接来了韩家。上个月韩家乔迁时,他也来过一趟。不过前一次是喜剧,这一次就是悲剧了。

韩冈缓缓的摇头,“半分都没有。谁知道向宝会怎么做?”

王韶叹了口气:“那我跟你一起走一趟吧,有我在,向宝总不至于做得太过火。”

“多谢机宜……”韩冈冲王韶拱手致谢,却又摇头道:“只是向宝的心思不好猜啊!”

王韶听得出韩冈这是在拒绝,再仔细想想,自己跟着向宝走,也的确只会害得韩冈。营中主帅便是天,虽然这有时也要视情况、人物而定。但以向宝的为人强势,一旦他出阵为主帅,当然不会容许他人来动摇他的权威。如果他要整治韩冈,王韶就算为之出头去,也只会让向宝手下得更重。

“玉昆,你干脆还是称病算了,你一病不起,想来李师中也不能把你硬拖上马。”

韩冈苦笑着:“现在也只能兵来将挡,水来土掩。向宝毕竟不是,只要他不敢杀我,这一关下官还是能撑过去的。”

“军棍也没人能吃几下重的。”王韶提醒韩冈,“向宝少不得要挑错。”

“机宜说得是。唉……所以也只能求向宝挑不出错来。”

“自来做事难、挑错易,世上哪有找不出错的事?欲加之罪,何患无辞?”王韶摇着头。

韩冈笑了起来:“只要不做事,那就不会犯错。”

“不做事?”王韶带着疑问。

“不做事!”韩冈肯定的点头。

“不做事。”王韶明白韩冈说得不是怠工、罢工的那种不做事,而是军中没有伤员病人,让韩冈无事可做。

只要不做事,向宝如何能从中挑出错来?王韶头轻轻点了几下,这么想倒是有几分理。

韩冈的底气也就在这里。向宝要挑人错,总不能说看你面相不好,所以要打二十军棍,今天天气不好,所以该打三十军棍。韩冈是去为今次作战,做他的管勾秦凤伤病事宜的工作,只要这件事上他挑不出错,自家再小心谨慎一点,向宝还能硬来不成?韩冈本人可不是向宝想打就打,想杀就杀的人物!

韩冈不知道向宝究竟为自己准备了什么样的豪华大餐,但他对自己安全充满自信。若是真的觉得自己有性命危险,他能咬牙直接摔断自己的胳膊和腿,以躲避跟着老虎一起出游的疯狂。就是因为此去韩冈自信能保全自己,方才会点头。不过,保险肯定要加上,谁知道向宝会不会发个疯。

只听韩冈继续说着:“向宝出阵,目的是为了托硕隆博二部。但以两部的实力,根本用不到他,有古渭的刘昌祚就够了。听闻刘昌祚这几个月被向宝挤兑得很惨,而且李、窦二位也都不喜欢他……就是这么做,会让机宜……”

王韶听明白了,他打断韩冈的话:“我是文官,又是提举秦州西路蕃部,而且还有王相公在……玉昆你完全不必担心。”

………………

次日清早,也就是四更天刚过的样子,韩冈便起床梳洗,赶着去了衙门。军中点卯不至,那是要误事的。而向宝虽然在秦州没能弄到兵,只有先到永宁寨,才能接手他今次要率领的军队。但他既然已经接过了李师中的军令,那么只要他不缴令,向宝就可以拿着军法惩治他帐下的官兵,而不必顾及在何地。

韩冈到得算早了,抵达衙门口的时候,天还是黑的,就看着州衙的正门上挂着一溜灯笼,照着门前透亮。除了那些在衙门中奔走的胥吏衙前,韩冈作为官员,算是到得最早的一个。

韩冈站在衙门口,也不想傻等。上前叫开了门,直接进了衙中。只是今天他没去二进的勾当公事厅,而是径自去了第三进的东院。兵马副总管的官厅和都钤辖的官厅都在这里。

韩冈在东院等着,看着天空从墨蓝转为艳紫,又从艳紫化为鲜红,等到火烧火燎的霞光褪尽,浅浅的蓝色充斥于天际,东院的主人终于到了。

不过不是向宝,而是带着一队随从的窦舜卿。

窦舜卿每天起得很早,一个是因为年纪大了,睡眠少,另一个则也是因为年纪大了,许多会让人晚睡的节目都没法参加了。早睡,故而能早起。

虽然心中认为窦舜卿是老而不死,但他身份地位摆着,韩冈只能上前行礼问好。

“韩冈,你的病这么快就好了?”窦舜卿可能是闲得发闷,想拿韩冈来寻开心。不过韩冈一大早就守在东院,也的确给人以走投无路,想低声下气求个人情的样子。

韩冈脸皮老厚,见窦舜卿要挑他的毛病,当即咳了几声,“下官病其实还没好,可终究须得以国事为重。若是衙门中的琐屑之事,倒也能放下。但托硕隆博二部相争,若烽火连绵不绝,说不定会引得西贼再次入寇,整个关西都要为之震动的大事,下官哪还能躺在……咳咳咳……”

韩冈厚着脸皮装模作样,咳得像是得了肺痨,窦舜卿自持身份,也没办法拆穿他,又不能真的说,韩冈你带病出征,堪为天下臣僚之典范。只好几步走过去,不去看韩冈的惫懒模样。

韩冈继续站着等向宝,而秦凤都钤辖没让他久等,赶在卯时初刻,向宝也到了,与他同至的,还有他的几个提拔起来的跟班,都是要一起去古渭的。

看到韩冈,向宝同样惊讶:“韩冈,这么早就到了。”

韩冈又咳了两声,不过不是为了装病,而是清嗓子,“受命出征,哪有迟到的道理。”

向宝领着人走进自己的官厅,韩冈也跟着进去。一群人按着官位高低站了。韩冈没想到,以他的品级,竟然还能站在向宝左手最前面的位置上。看起来向宝把他身边得力的人手都荐了不少出去,现在他身边,有官身的就没几个了。

等众人站定,向宝当即高声道:“今次惩治恣意妄为的托硕部,有韩抚勾来就让人安心了。你们都给我听着,韩抚勾站在这里。上阵后你们也不必再缩着脖子,就算受了再重的伤,韩抚勾也能把你们给救回来!”

“钤辖误会了!”韩冈立刻毫不客气的指出向宝的错误,不论向宝的误会是真还是假,现在不明确指出,含糊过去,日后就是向宝出手时的刀子。他以谦虚的口气说着:“药医不死病,若是真个有谁能包治百病,那是仙,不是人。韩冈能做的,也不过让伤者病者少受点苦,卒伍中少死点人。”

向宝呵呵笑道:“韩抚勾你太自谦了,不是说你是孙真人的私淑弟子吗?”

“市井谣言,当止于智者。”韩冈神色不为所动。

“……事情是这样啊,”向宝的脸挂了下来,扬起下巴,用眼底余光瞧着韩冈,“亏外面传得神乎其神,原来也就这等本事。”

紧跟着向宝,他的几个亲信便是凑趣一般的哈哈大笑。

“医道之事本就是尽人事,听天命。韩冈的确就这点本事。”向宝的鄙视对韩冈没一点用,他一向谦虚。

“尽人事,听天命,你就靠着这六个字救我军中儿郎?”向宝的声音冷狠下来。

“是的。”韩冈点了点头,“钤辖久在行伍之间,当知军中伤病,至少有半数无法痊愈。若是时节、地气有差,病殁者便难以计数……”

“俺自从军以来受过七八次伤,却是此次都逢凶化吉,俺怎么没病死?”站在韩冈的正对面,一个三十出头、猛将模样的军官反驳着韩冈的话语。

“殿直军中素有威名,当然能得到最多的照顾,但寻常士卒,可就没有这么好的条件。受了重伤后,没有得到及时救治,最后人整个烂在病榻上的事,殿值应该见识过吧。”

猛将殿直看起来不是很会说谎,有着张口结舌。

“韩抚勾,”向宝冰冷的眼神如一片巨石沉沉压韩冈:“你倒是伶牙俐齿!”

韩冈毫不客气的针锋相对:“是下官理直气壮。”

向宝勃然做色,他的一众亲信当即齐喝:“好胆。”

韩冈视其走狗狂吠如无物,只看着向宝:“敢问钤辖还有何吩咐?”

向宝的怒气渐渐在脸上凝聚:“韩冈……真当我斩你不得?”

“以军法,军中可斩之行有四十七条,只是不知钤辖要斩韩冈的,是为了其中的哪一条?”

