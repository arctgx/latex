\section{第13章 赳赳铁骑寒贼胆(下)}

“杀你娘!别以为你杀了刘三,爷爷就怕你……”董超捋起袖子,就想给韩冈点颜色看看。韩冈是够狠,杀了黄大瘤和刘三的手段,他们这些市井中的无赖想都想不出来,但他董超也不是孬种。市井中常年打混的,讲究的就是狠字,嘴不能软,气不能短,不然如何在江湖上立足?

只是他刚上前,胳膊肘便给扯住了。回头一看,薛廿八正拼命朝他使眼色。董超脸色数变,最后重重哼了一声,朝地上吐了口吐沫,还是退了回去。薛廿八对韩冈笑了一笑,也跟着退回去坐下。

韩冈见董超和薛廿八缩了头去,心中凛然,能忍一时之气,可见他们肯定有什么算计在后面要施展。不过他顺带激怒两人的目的也达到了,等吴衍派来的人到了,出了城后,他自有手段对付他们。只是韩冈心中还是有些焦急,如果吴衍派来的人不到,那自己就只能孤身面对董超、薛廿八二人。虽然暗中已有自保手段,但手上只剩一两张底牌可打,让他总是有些难以安心。

韩冈低下头,正想将车子、骡子反过来再检查一遍,磨一磨时间,却听见一阵急促的马蹄声重重的从身后压了过来,声势急如奔雷。急回头循声望去,只见一名骑兵正直奔辎重队而来。

“好了。”韩冈终于放下了心头大石,他们所处的巷子并非要道,不是发送军资的日子便少有人走,这名骑兵明显的是冲着车队来的。他直起腰掸了掸身上的尘土,仰头看天,天色依然晦暗:“差不多该上路了。天色看起来不太好啊!”

董超朝韩冈这边吐了口痰。心道又不是你韩家养得狗,你说走就走,说留就留。他坐在地上就是不动弹。薛廿八则看出了来人气势汹汹的,势头有些不对。他跳起身,绕过韩冈,对来人喝问道:“是什么人?!”

“是你爷爷!”那名骑手远远的一声大吼回来,不但耳朵尖,看起来脾气也不甚好。

吼声很耳熟,身形也眼熟,韩冈只觉得其人的身份在脑海中呼之欲出,就是一时想不起来在哪里见过。

来人转眼间便越来越近,倒是董超先认出了他的身份,也惊得一下蹦起,叫道:“王舜臣,怎么会是你?你来这里作甚?!”

被董超唤做王舜臣的骑手也不多话,等几步冲到近前,他一勒马缰,手腕顺势一摆,马鞭刷的一声抽了下来。一条血痕顿时出现在董超的脸上:“爷爷的名讳也是你叫的?!”

跳下马,王舜臣对韩冈直接了当道:“你们是去甘谷城的罢。洒家奉命要送密信去甘谷,跟你们正好顺路。算是你们运气,有洒家保着你们一起走。”

“多谢殿直!”韩冈忙着点头,他不知王舜臣官位为何,但往高里说却是不会有错。韩冈一边说着,直盯着王舜臣看,只觉得面熟,却还是没能认出来。

董超用手捂着脸,指缝间都往外冒出血来。却一声也不叫痛。他算是个市井好汉,一个泼皮光棍,被陈举抬举了升入了县衙。圈养了许久,但泼皮破落户的脾气还没有改变。方才被韩冈逼退,已是怨愤,现在又挨了一鞭子,他更是心中发恨。冲着王舜臣一阵大叫:“王舜臣!你骑马,俺们走路,你跟俺们又不是一路的!”

“大道朝天,爷爷爱横走就横走,爱竖走就竖走,端看爷爷的兴致。难道爷爷走路还要向陈举那厮报备不成?!”

这腔调也是似曾相识。又看了王舜臣几眼,韩冈突然恍然,他不正是自家死中求活的那一夜,跟着吴衍一起来援救、隔门怒吼的巡城队官嘛!

吴节判说话算话。前天韩冈请他帮自己安排了个随行的护卫,他果然将人派来,还是有胆色的强手。

‘原来就是他啊……’

在宋代,唤作尧臣、舜臣的特别多,一抓一把。就像后世共和国开国时,起名叫解放、向阳的一样。这是思慕上古贤君所起的名讳。

王舜臣的名号普通,但相貌却极有特色。他脸很大,几乎比常人大一倍,手也很长,虽不比刘备,垂下来离膝盖也不远。宽厚如石板的身躯上,长着一张有些丑陋的脸。再加上留了一嘴乱丛丛的络腮胡子,眼睛圆圆,一瞪起来,几乎与传说中的张飞有五分相像。

只是王舜臣善用的不是丈八蛇矛,而是弓和铁简。

就在王舜臣的马鞍后侧左右,各挎了一只弓袋,里面装的角弓尺寸并不算大,可制作之精良,是韩冈生平所仅见。而在马鞍前侧,则是挂了两支四棱铁简,上面泛着油光,显是保养得很好。弓和简,便是王舜臣的主要装备,在宋军中,也是属于制式武器。

王舜臣身量不高,大约五尺二三的模样,双腿还是罗圈腿,两脚贴紧时,他的双腿仍然并不直。但这是常年骑马的特征。王舜臣双臂长而有力,从身体条件来看,他的弓术绝然不差。

“王舜臣!别以为身后有了节度判官就能保着你。出了差错,你担待不起!”

有董超为鉴,薛廿八不敢放些狠话,只能从利害方面入手,但王舜臣可不吃这一套,立刻反咬一口:“你两个鸟男女在这闹个甚,不知道甘谷城正等着这批酒水吗,还拖个鸟?!莫道洒家不敢杀你两个鸟货,军法立来可不是作摆设的!”

他骂着,马鞭再一挥,在空中噼啪作响,落到两名押运的长行身上,抽得他们满地乱滚。王舜臣在秦州凶名早著,也不怕两人敢还手。一顿鞭子,让董超,薛廿八趴在地上直哼哼,衣衫破烂,脸上手上多处血痕。不过王舜臣没下重手,并未伤到两人的筋骨,至少在秦州城中,他还不能把两人给废掉。

王舜臣将马鞭收起,猛然回过头来。拧着眉盯着韩冈,一双环眼精芒如电,浑身上下杀气腾腾,恶狠狠的道:“你就是杀了刘三那几个鸟货的韩三秀才?!”

“在下正是!”韩冈微笑着点头行礼,吴衍派来的这位可真是妙人,说下手就下手,又满嘴跑鸟。但这脾气,韩冈倒是喜欢。

没能吓住韩冈,王舜臣并不意外,手上都攥着三条人命了,哪还会被人瞪瞪眼便给吓到?韩冈在军器库中的杀伐果断,他是有点佩服的,“你这秀才倒是好胆略,陈举将了三人翻墙害你,却没成想被射死了一对半。三条人命,他陈举巴掌再大也遮瞒不过去。别看现在县里结案,等经略相公回来,照样能把案翻过来整死他。”

韩冈故作不解:“殿直何有此言,黄德用和刘三等人明明是夏贼在城中的奸细,又与陈押司何干?”

王舜臣啐了一口,“你们这些措大,就是阴在肚子里,明明白白的事还死咬着不肯松口。也算你做得好事。那陈举仗着自家势力大,身后又有人,从不把我们这些军汉放在眼里,都是呼来喝去。若是在荒郊野地里给洒家碰上,直剥了皮,囫囵丢进藉水里去喂王八。”

骂了几句,见韩冈也不附和,王舜臣自己便停了嘴,又对韩冈道:“韩秀才,俺只是个没品级的军将,离殿直什么的,还有五六级。别这么叫俺!洒家听不惯!”

韩冈低头逊谢。这王舜臣脾气粗豪,但却知道分寸,看起来心思也算细密,吴衍倒是好带契,给他找来一个够管用的保镖。这样一来,韩冈安然抵达甘谷城的信心又多了一点。

王舜臣既然到了,也不用再拖延时间。韩冈一声令下,大队当即启程,连薛廿八和董霸也被王舜臣一人一脚踢起来收拾了伤口,恨恨的跟上队伍。

在城门处验了关防,一行人径直出了东门,迤逦向东。三十多辆骡车一架接着一架,在官道上排出一列长队,而王舜臣骑着马,就跟在车队的外围。

跟着骡车快步前行,韩冈突然心有所感,猛回头,只见城头上,一个不算高大的身影正挺立在寒风中。

韩冈的瞳孔一下缩紧:“陈举!”

“真是陈押司!”一行人议论纷纷。

“他来做什么?”

“没看到这次是谁领队吗?韩三秀才啊,杀了刘三,逼死了黄大瘤的那个。陈押司能不来?”

听着队伍中的低声议论,韩冈淡然一笑,陈举来了又能如何?!

他现在最大的希望,就是想凑近了看看陈举现在脸上的表情。怕是陈举自己也没想过,在韩冈身边,会突然多了一个保镖,而且还是脾气够坏,但又不乏聪明的王舜臣!

朔风渐渐猛烈起来,韩冈外袍里面穿的羊皮背心是用双层皮子对缝而起,带毛的一面给缝在了里面。背心是对襟开,带盘扣,形制有别于此时的服饰。是用了韩冈的建议,韩阿李裁剪,韩云娘又用了两天时间一针一针的赶制出来的。今天早上,由韩千六赶着送到韩冈他手中。穿起这一件背心,不但身子暖和,连心里也暖洋洋的。

盘踞在韩冈心中数日的阴云,已因王舜臣的到来而烟消云散,心情变得很轻松,直如阳光灿烂。天顶虽是阴云密布,但前路却一片光明。

ps:王舜臣不是什么名人,在青史中只有寥寥数笔。但能以一人之力挽救全军危亡,在北宋后期,也就区区几人。

新角色出场,征集红票和收藏。今天第二更。

