\section{第35章 势颓何来回天力(上)}

“是吗?”

这是十二道金牌召岳飞吗?

不过来送信的不是金牌急脚,只是宦官而已。而他韩冈也不是武将,而是文官。

“大小王庄一战,辽军伤亡过万,这是河北、陕西所无法相比的大捷。”

“辽贼新逢大败,已无心坚守代州,只要稍待时日,官军必能重夺代州。此时罢兵,亲者痛而仇者快。”

夜中的追击,斩获的首级几近三千,逃窜不得,缴械投降的也有千余。

在以步兵为主的宋军追杀下损失了五分之一的军力,丢弃的战马、军械数量则更多,辽军的损失可谓是伤筋动骨,几年内都很难再回复元气。要不是其中不见多少辽军作为核心的宫分军和皮室军,韩冈直接就敢大张旗鼓去攻打代州城了。

章楶、黄裳、折可大,诸多幕僚,无论文武人人气急败坏。原本担心朝廷拖后腿,可一直不见有诏旨来,才放心下来。现在即将大获全胜,不意这时候却来扯后腿。

夺还忻代二州失土,这是多大功劳?!周世宗也才三州十七县啊。

被一屋子虎狼盯上,浑身颤抖的姜荣几乎要哭出来,身子抖着:“小人只是奉旨前来,军中事岂敢,若枢密,可具表奏闻,”

“朝廷诏令何在?”韩冈放过了他。

这位姜荣并不够资格宣诏,携带诏书过来的另有其人。

那名韩冈十分熟悉,却暌违已久的中使托着黄绫裹起的诏书,“枢密,此乃密旨,请排开无关人等。”

姜荣闻言一下怔住,瞪大了眼睛看着同伴,而韩冈立刻低头领命。

帐中的几名幕僚交换了眼色,纷纷退出帐外。亲兵摆下香案,也都退了出去,只剩下韩冈一人。

韩冈走到两位内侍面前,背朝帐门,面对阉宦,“请天使传召。”

“不敢。”那名宣诏中使向韩冈躬躬身,“小人出来前,圣人也只知枢密在修造轨道,并不知晓有此大捷。”

韩冈这段时间以来,一直都在慢悠悠的修造轨道,跟辽贼的对峙似乎看不到尽头。朝廷那边的耐性消磨得差不多了。

“若是枢密当真领下旨意,使得河东兵事功败垂成,绝非圣人所愿见。这份诏书,传与不传,其实是一样的。”

“是吗?”韩冈很惊讶,有此胆色的内侍可不多见。难怪名气能有那么大,地位能有那么高,人品不论,胆量就是超乎常人。不过拒不拒诏,那是他的责任,用不着他人来越俎代庖:“童阁门,既然是奉命传诏,还是不要耽搁了。”

童贯迟疑了一下,见韩冈心意坚定,也只能无奈的暗叹一声。好不容易回到京城,又好不容易拿到这个差事,却无法如愿以偿。

点了点头,他展开诏旨,宣读起来。

韩冈躬聆圣训,最后再拜起身,双手接过诏旨。

这份诏旨上缺乏细节,没有将事情说得很明白,不过和议将成、权且休兵的命令还是确定无疑的。

所以韩冈更为惊讶。

西府怎么会同意双方罢兵的动议呢?至少在眼下,大宋在河东的优势一天比一天明显,功劳近在眼前,无论是章敦还是薛向都不会甘愿放弃。

另外,他们就不怕吕惠卿的反对?复夺兴灵的功劳,吕惠卿这位枢密使怎么可能眼睁睁的看着被送还给辽人。

“猝然罢兵,必有缘由,朝廷究竟是为何事罢兵?”

韩冈还没有收到来自京中的信报,之前和谈的条件两府都不愿接受,但现在突然同意,肯定是辽人一方开出了什么让人无法拒绝的条件。

“禀枢密。”童贯回道:“是辽人那边改了和议的条款。”

“什么条款?”

“大半就是枢密提的条款。”姜荣在旁插话,韩冈的要求还是他上次带回去的,“岁币不增,而兴灵则是以银绢一百万匹两赎买下来,从此归于大宋。”

“银绢各五十万?”

“正是。”

“我之前说的是三五万贯吧。如今却是百倍于此。”韩冈话声转寒:“尔等怎以此事诬我!是我少说一句,勿过三十万,过必斩汝?!”

这话是寇准在曹利用去签订澶渊之盟前特意说的警告。‘虽有敇旨,汝往,所许不得过三十万。过三十万勿来见准,准将斩汝。’

韩冈一变脸,姜荣几乎吓得魂飞胆破,宰辅之怒岂是他这个小黄门能担当得起的。

“还请枢密息怒。此百万银绢并非岁币,乃是断买。就如世间买卖田宅,断卖本就要比典卖价高。相比日后岁岁鏖兵,百万非多。”

绢五十万匹,银五十万两,相当于两倍的岁币。不过这不是岁币,而是一次性的买断。

韩冈之前说的三五万贯的确是太少了,但这一百万匹两的买断费也未免太多了。如今的银价一两一千八百钱,相当于两贯半,而作为随笔供给辽人的绢绸,均价也在两贯上下。两百多万贯的现钱。以兴灵之地的税收不知要到哪一年才能收回来。不过也正如童贯所说,得到了兴灵,光是省掉的军费就不知多少了。

除了买断兴灵的代价过于高昂之外,本质上跟韩冈的要求一样。这两百多万贯的财货,也是照顾耶律乙辛的面子所给予的实利。

韩冈其实也不甚恼,比起岁币,这个一次性的买断的确不算吃亏。也难怪朝中不怕吕惠卿反对和议。他所争的,不正是兴灵吗?

“此外还有沧州北界增开榷场。解州刺史将为使常驻。”

这的确也是韩冈的要求。之前准备用来卖好耶律乙辛,同时收买京中豪门的手段。至于解州刺史,就是皇后的那位堂兄。

“武州呢?”

“辽人愿以代州诸关塞换回。甚至可以交还熙宁八年,河东北界割让的七百里失土。”

还真有创意,韩冈倒真想为耶律乙辛的想象力拍案,“……皇后怎么看?”

“圣人和王平章不同意。”

的确是不可能同意的,要是让天子赵顼知道,谁也说不准他是高兴还是羞恼。

“不知枢密你意下如何?”

“你觉得我们现在该同意吗?”

韩冈带人出帐,辽军的尸骸正收拢起来在野外焚烧,一道道浓烟在澄蓝的天幕下分外显眼。

“飞捷的奏报早已上京,请两位回去后奏于皇后殿下,和议可以,但武州绝不交还。”

韩冈望着南方的滹沱河水,“天时地利人和俱全,难道还要怕他辽贼不成?”

滹沱河发源自繁峙县瓶形寨附近的泰戏山,出山之后一路流向西南。途经代州雁门县、崞县抵达忻口寨。然后在忻口寨折往东南,穿过太行山进入河北。

滹沱河穿过太行山时落差极大,不能利用其来沟通河东、河北,但滹沱河在河东的径流,大半是在忻州、代州的带状盆地之中,水流平缓,河面宽阔,自古时起就有水运通航。《墨子》之中提起大禹治水,便有‘呼池之窦,洒为底柱……以利燕、代、胡、貉与西河之民’之语。这个呼池,便是滹沱河。

只是之前正逢春末,山头融雪和春日雨水汇合的桃花汛时节已过,滹沱河水位大降。三尺多勉强可以行船的水深已变得深不盈尺,河底浅滩一段段的暴露在外,整条水道完全无法利用。也许再过一个月夏汛到来时,这条河道将能重新派上用场,但那时候几乎连饮马都显得困难。

其实以堤坝抬升水位的工程技术手段,在灵渠中就有使用,用于滹沱河上,也一样可行。可是要弥补从代州到忻口寨的落差,在河面宽度能达到七八十丈的这一段河段上,就需要修筑高度至少在一丈的漫水坝,这个工程量比一条简易轨道大得太多,人工、物资和时间都不是韩冈能消耗得起的。

通过当地的土著了解到水文地理后,韩冈便决定放弃水运、修筑轨道。本就不擅长水路交通的辽军也同样只能望河兴叹。

现如今,情况发生了变化。从大小王庄至代州城,四五十里简易轨道的修筑需要大量的工时和物资。但随着时间进入四月,五台山中的雨水也渐多起来,滹沱河中水位的上涨让韩冈原本放弃的选择,又重新进入了他的视野中。相对于耗费过大的轨道,

当两艘载着百余石粮食的河船,在数匹挽马的拉动下,顺利的抵达了代州城下预备好的码头上,官军的补给线也变成了水路交通加上轨道运输的综合体。

这可谓是天时相助。

山中的雨水多了,但代州盆地中却没怎么下雨。这让逐步进抵代州城下的宋军能够很顺利的做起攻城的准备来。不管从什么角度来看,老天帮了大忙。

除了天时之外,人和的方面也同样越来越占优势。

在神武县的方向,麟府军之前就已经控制了古长城一线,彻底掌握了武州,直面朔州之敌。随着辽军兵败大小王庄,朔州守军士气大落,折克行乘势翻越古长城。

这也意味着宋军全面攻势的的开始。

一天后,宋军的骑兵出现在朔州州城外,并与退守此处的辽军几番交手。

出自河外的骑兵实力不弱,不过数目不比朔州的辽军,仅有千骑。一番缠战后见辽军越来越多,便选择了撤退。

两千多辽骑紧追不放,一口气追出了二十余里,直至山中,被引进了折克行摆下的伏击圈。

只可惜辽军熟悉地理,且领头的军官为人精明,麟府军的伏击最后只咬住了两百多人下来。

但这一战,也打掉了朔州辽军最后一点信心。面对宋军紧逼,已经没有人再愿意冒险远离朔州城的防护。

所以第二天,当第一支步军抵达城外,便顺利的安营扎寨下来。

从神武县翻过古长城进抵朔州的麟府军超过五千,并压制了城内的守军。

仍在大宋境内坚持不退的辽军,此时已经面临着后路断绝的危险。且朔州往北,便是西京道的首府大同府了。

此处战略要地,是不容有失。

要代州?还是要朔州?

韩冈正等着萧十三的选择。

