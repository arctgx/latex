\section{第六章 征近伐远方寸间(下)}

【第三更,求红票,收藏】

关于罗江、神狗之类劝诫,王韶说说也就罢了,他知道韩冈做事向来稳妥,提点一二足矣。

今天的正事不是训斥儿子,也不是提醒韩冈不要在公文中说到狗。韩冈会跟着王韶一起走,同样不是为了检验他军棋推演有多吸引人,而是为了准备招待一名客人。

韩冈另外一名举主,王韶在秦州仅有的两名盟友之一刚从甘谷城回到秦州,明日就要诣阙面圣,与王厚他们做一路走,王韶理所当然的要设宴款待。

也许,王韶的盟友现在只能算一个半,雄武节判吴衍如今渐渐与王韶疏离,连韩冈要求见他,都会被推三阻四。

韩冈对此也是无可奈何,看不清形势的官员秦州多得很,并不止吴衍一个。对王韶和他的平戎策,谁也不会有跟韩冈一样信心。

故而到了晚上,王韶设宴招待张守约时,吴衍便没有到场,而是韩冈跟在后面相陪。

“韩冈拜见老都监。”韩冈赶着对张守约行礼,起身后笑道:“韩冈看着老都监身子骨越发的康健了,精神都比我们这等小字辈要好得多。”

“就玉昆你嘴会说。”

张守约笑得眼眯缝了起来,被韩冈说得很开心。老家伙今年六十多,在军中超过四十年,但看精神的确比谁都好,至少比窦舜卿要好许多。

李信则跟在张守约的后面,也不知道他什么时候又转回去的。而王厚、赵隆他们也在李信旁边站着。几人都是熬了夜,有些萎靡不振。

王韶看着他们的样子,就有些不高兴:“玉昆是实话实说,都监看起来是比我家的儿子要精神!”

张守约回头,冲着王厚他们笑道:“昨夜玩得痛快吧?”

王厚呐呐难言,而李信的脸色变得尤其厉害。

张守约在西北军中向以识人著称,刘昌祚、燕达都被他称赞过,尤其是燕达,最近刚刚在绥德城立了大功——只是韩冈方才提起此事,王舜臣就骂了起来,说是郭逵刻意调走种五郎,而把功劳给了燕达。

王舜臣偏向性过于明显的抨击之词姑且不论,被张守约赞过的燕达和刘昌祚的确都是难得一见的人才,被他举荐的韩冈则是另一个成功的例子。李信能得他看中,日后前途必然一片光明。

也就是因为得到张守约的看重,李信更是分外在意他对自己的看法。

“处道他们倒也不是去逛了什么不干净的地方,”韩冈出头,帮着自家表兄解释。“昨夜都是在机宜家指点江山呢。”

“怎么个指点法?”张守约当即问道。

王厚得意的上前,把韩冈弄出来的这一套都跟张守约说了一通。

“挺有趣的。”张守约给沙盘和军棋推演的评价就这四个字,没看到实物,他也不会轻易下结论。韩冈本以为以张守约不见兔子不撒鹰的性子,只会说一句‘还不错’,而张守约的评语,好歹比他估计得要多出一个字来——虽然评价等级却是更低了一点。

不过也难怪张守约会不放在心上。

韩冈弄出来的军棋,本就是把规则简化而又简化的东西,甚至比不上后世的桌面游戏复杂——更复杂的规则,韩冈也做不出来,那要考虑到方方面面太多了,对数学的要求也更高——王厚他们玩得用心,是因为他们见识太少,而张守约老于战阵,性格也因为年龄更加顽固,当然不会对模拟的东西看得很重。

“玉昆弄得这个什么军棋推演,必须先查敌。多派斥候细作,知道对手的兵力布置、粮秣存放,还有地理人情,才能玩得起来。若是其中有一项变了,一切就会变成无用功。”

张守约不仅是顽固那么简单,眼神也很毒辣,一眼便看出了缺陷所在。

任何战前的军棋推演都得建筑在准确的情报上,情报错误,的确会一切都变成无用功。而有了准确的情报,在对付党项吐蕃的战争中,有没有战前推演过一番却也不重要了——有这个闲空,还不如把粮饷准备得更充分一点。

在韩冈想来,战棋推演反倒是在战后总结上的用处要大上一些。否则就必须不嫌麻烦,事前把所有可能的情况都推算一遍。

王韶引着张守约坐下来,他选的设宴地点,是新近开张的一家酒楼,人气还不算旺,王韶却就是要取着这里的清净。

韩冈在下首做陪,而王厚便坐得更下面。请人入宴,又是饯行,歌舞是少不了的。王韶找了秦州最好的几个官妓来给张守约劝酒,虽是不比东京歌舞妙丽,但也是有些味道了。

但在座诸人的心思,都不在酒宴上。

酒过三巡,张守约屏开几个歌妓,直言不讳地问着王韶:“拿向宝做幌子,径自去抄了托硕部的老窝,一举断了向宝的路。如此行事,不像是机宜的手笔,”

在张守约面前,王韶也不加掩饰:“一开始是玉昆的主意,但结果却是机缘巧合。事先谁都不会想到会把向宝气成中风,说起来还真是运气。”

张守约哈哈笑了:“运气也很重要。没有运气,老夫的骨头早就给党项人拿去熬汤了。”他又指着王厚、赵隆说着,“别看你们今次要押送入京的托硕部的那群首酋,现在一副倒运背时的模样,等见过天子,你们没一个能比得上他们。都是运气。”

张守约说话的声口有点倚老卖老,但道理却不错,王韶苦笑着敬了张守约,“都监说得没错……”

而韩冈也是一般的苦笑摇头。

别看王厚、赵隆明天就要雄赳赳气昂昂的押解着托硕部一众入京献俘,也别看王韶团聚七部把托硕部和背后支援托硕部的木征打得屁滚尿流。但到最后,比起官品来,还是被押送的那几位会高上一点。如今情况就是这样,只要表现得恭顺些,外藩进京总能弄个好名头,即便是被打败了,押解入京,也少不了用几个空官安抚一下。

王韶一心想算计的木征,现在正领着河州刺史的本官,还有个银青光禄大夫的加衔,是光明正大、正儿八经的大宋臣子。

另外木征在党项人那边也领着观察使的头衔,虽说是没俸禄的空名,无论宋夏,两边其实都不在乎,但官位就是官位。如果木征肯入朝,他在大庆殿上的位置,只会比王安石、郭逵这些执政或前执政低少许,而王韶就只能站在殿门口。

一夜痛饮,第二天,王韶和韩冈便送着张守约和王厚他们一行远去京城,而托硕部的一众俘虏,则是用囚车装着,一起运送过去。

王厚骑上了马,手提着缰绳对韩冈笑道:“玉昆,今次愚兄回来,我们兄弟两个可就是要同朝为官了。”

王厚对军棋推演和沙盘寄予了厚望,以他的身份,光靠献俘一事,已经能在天子面前混个官身了,如果再加上沙盘一事,说不定能一下就能拿到三班奉职,就像刘仲武那样。

“处道兄此去当能如愿以偿。”

“那也是玉昆你的功劳。”

韩冈跟王厚一样充满信心,毕竟比起如今的地图来,今次要献给天子的沙盘,要精美上许多,看上去不仅仅是准确一点点。

如果说韩冈在千年之后见识过的地图是写实型的古典主义画派的作品,那他在这个时代看到的地图往差里说是涂鸦,稍微美言一点,那就是印象派。看着此时的地图,找对地方比找错地方还要难上许多。

不管怎么说,越精细的作品——不是精确,是精细——就越能得到肯定,而其中的谬误,却往往会被忽视过去。

韩冈相信赵顼会对沙盘和军棋推演感兴趣。游戏嘛,哪个不喜欢?他自己也曾经有点着蜡烛熬夜打牌的时候。何况赵顼本来就是喜欢对军务指手画脚的性子,发到地方上的阵图,连秦州的架阁库中都有。以赵顼的这种性子,韩冈不信他能忍住在沙盘上指点江山的诱惑。

只要赵顼喜欢上了沙盘游戏,那王韶和韩冈想要在沙盘上透露的信息,自然也会被赵顼所接受。无论窦舜卿、李若愚说什么都没用了,究竟是万顷田还是一顷田,沙盘上不是一目了然吗,赵顼又怎么会相信窦李之辈的空口之言?

王厚走了,张守约也走了。王韶和韩冈在他们两人身上都寄予了厚望,毕竟他们今次都能见到天子。

到了当天午后,王韶把韩冈又找了来。

“高遵裕来了。”王韶的声音中有着很深的阴郁,在韩冈面前,他没有过多隐藏内心的不快,“分功倒也罢了,只希望不是来添乱的。”

“天子派窦舜卿来,目的也不是添乱。不过,窦舜卿听命于韩琦,而高遵裕却是只听命于天子。”

韩冈倒不介意高遵裕来分功,他一向看得开。将欲取之,必先与之的道理,他也向来是奉为圭臬。如今王韶求得是立功的机会,而不是功劳的大小。只要高遵裕能给王韶带来这个机会,又何必介意他把功劳分去一半?

“要做件事怎么就这么难呢……”王韶望天长叹:“只望一切能如玉昆所言。”

