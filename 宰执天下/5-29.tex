\section{第四章 惊云纷纷掠短篷(三)}

上元已过,年节的气氛已经荡然无存。

湖州地处两浙,很快就要开始春耕了。农为国本,无论衙门里的官员,还是田地中的老农,这时候,都要忙起来了。

湖州城外的何山上,却还有一群人悠然自得的在一座凉亭内外或坐或立。

亭外围着一群衣着统一的家丁,再外围更有一帮穿着各色衣裳的闲人,都是在望着凉亭内,脸上尽是期盼之色。

而在亭中,两只火盆里面烧着木炭,火苗舔得老高,滚滚热浪,驱走了亭中初春暮冬的湿冷。几名衣红服翠的妓女抱着琵琶笙箫散坐在周围,很是闲适的弹拨吹奏着,让轻柔的曲调从凉亭内传到了亭外众人的耳中。

“怎么还没有新诗出来。”

“苏学士已经进去好一阵了。”

“快了吧。”

人群中的议论,也随风穿了回来。

亭中的火盆边,两名中年男子处在所有人的中心处。

其中一人,留着三缕长须,笑道:“子瞻此一出,直如卫玠,恐被世人看杀……”

另一个留着一脸大胡子,拍着自家的肚子,“苏轼榔槺粗笨,最喜吃肉喝酒,可没那般娇贵。”

“也是子瞻如今文名传天下,才会惹得世人追随身后。”

“却似腐蝇逐臭肉。”

苏轼跟着接了一句,两人眼神对上,顿时一阵哈哈大笑。

现任湖州知州苏轼,拿着柄玉如意在手上轻轻敲着:“去岁曾携友挟妓共游何、道二山,道中遇风雨,憩于贾耘老溪上澄晖亭中,随兴命官妓执烛,画风雨竹一枝于壁上,并题诗一首:更将掀舞势,把烛画风筱。美人为破颜,正似腰肢嫋。此一篇,当为任官湖州数月以来第一。”

“美人为破颜,正似腰肢嫋。”坐在苏轼对面的中年人一笑,“子瞻其时兴致不浅啊……可惜王巩未能与会,诚可惜哉。”

苏轼手中玉如意一停,看着王巩:“不得定国相唱和,苏轼也是觉得不甚圆满。”

“王巩捷才不及子瞻,明日当敷衍一篇出来相和。”王巩在亭中远眺山下的田地,田中已经有农人赶着耕牛在犁田了,“眼下过了上元节,州中也该忙起来了,王巩过湖州,却耽搁了子瞻的公事。”

“定国来湖州,却是便宜了苏轼。”苏轼呵呵一笑,举着玉如意一挥远水近山,“我正病湖州山水,定国即来,正好可以下定决心告病数日。至于州事,交由通判祖无颇暂摄。”

“州厅、倅厅向来不合。尤记昔年钱昆求补外郡,人问其所欲何州,只云:有螃蟹无通判处即可。子瞻能放手州务,倒是比钱昆阔达多矣。”

苏轼放声大笑:“孟轲有云:‘为政不难,不得罪于巨室。巨室之所慕,一国慕之;一国之所慕,天下慕之,故沛然德教溢乎四海。’湖州巨室如今各安其分,苏轼又何须劳形于案牍之上。”说着一举玉如意,“定策安民,州将之任。至于琐事细务,交予通判又如何?”

王巩叹道:“若天下军州帅臣皆如子瞻一般豁达,国事早已定矣。”

“苏轼之才尚不足论。岂如定国,巨室世臣,家学渊源,若出而治世,何愁世事不定?”苏轼长声曼吟道:“所谓故国者,非谓有乔木之谓也,有世臣之谓也。”

这是孟子见梁惠王时的谏言,王巩摇摇头,叹息道:“不如当朝诸臣能得天子垂顾。”

“此辈何足论?”苏轼毫不客气,“平居无事,商功利,课殿最,定国诚不如新进之士。至于缓急之际,决大策,安大众,呼之则来,挥之则散者,惟世臣、巨室为能!”

王巩的祖父是真宗朝的名相王旦,父亲是仁宗朝的名臣王素。曾祖王祐也是太祖太宗朝的重臣。王祐封了晋国公,王旦封了魏国公,王素以工部尚书致仕,熙宁六年病逝,得赠谥号‘懿敏’。王巩是元勋世家,正是属于苏轼所说的世臣巨室的行列。

王巩眼睛笑眯眯,却是摇头,说着当不起、不敢当。

“如何当不起?”苏轼道:“嘉佑时,苏轼初识识懿敏王公于成都,其后从事于岐州。方是时,西虏大举犯边,边人恐惧,军不堪用。但一闻懿敏公将至,西虏随即解兵而去。公至,不过设宴犒劳而已。使新进之士当之,虽有韩信、白起之勇,张良、陈平之奇,又岂有懿敏公不劳军民,坐胜默成之功。”

王素当年什么都没做,只是正好撞上了西贼解围而已——甚至还不能说撞上,党项人抢得心满意足离开的时候,王素还没有到任,但人嘴两张皮,想推功于王素,苏轼有足够的才气做到。

苏轼说着,就站起身,“取纸墨笔砚来!”

随行的伴当就等着这一句话,在亭中架起了桌,铺上了纸,磨好了墨,将笔递到苏轼手中。

苏轼拿着笔饱饱的蘸了墨汁,回头对走过来的王巩道:“吾有一真赞,追奉懿敏公于九泉之下。”

随即落笔,一行行草书龙飞凤舞,出现在纸面上,苏轼的书法天下知名,文章更是冠绝当代,王巩凝神细读。

“堂堂魏公,配命召祖。显允懿敏,维周之虎。魏公在朝,百度维正。懿敏在外,有闻无声。高明广大,宜公宜相。如木百围,宜宫宜堂。天既厚之,又贵富之。如山如河,维安有之。”

王巩扬了扬双眉,眼中满是喜色。只有苏子瞻的文字,才配得上他的父亲。

苏轼运笔如飞:“彼窭人【穷苦人】子,既陋且寒。终劳永忧,莫知其贤。”

王巩微微一笑,更是点了点头。正是如此!那等小门小户的出身,狗苟蝇营而已,虽不为无用,却非是定国的贤才。

“易不观此,佩玉剑履。晋公之孙,魏公之子。”

最后十六个字一气呵成,苏轼抬手掷笔,直起腰哈哈一笑。

王巩通览一遍:“子瞻之誉,王巩本不敢受。唯论先人之德,不敢推拒……”

他喜滋滋的,将苏轼即席写下的赞诗读了一遍又一遍。

凉亭中,几名妓女轻挥丝弦,将苏轼为王巩之父王素所写的四言赞诗半吟半唱了出来。

苏轼此时兴致正高,看了看面庞丰泽、皮肤光滑、保养得甚好连眼角都不见鱼尾纹的王巩两眼,“苏轼又有一篇赠与定国。”

随即落笔,“温然而泽也,道人之腴也。凛然而清者,诗人之癯也。雍容委蛇者,贵介之公子。而短小精悍者,游侠之徒也。人何足以知之,此皆其肤也。若人者,泰不骄,困不挠,而老不枯也。”

很快,这一篇真赞也被妓女唱了出来。

“看到没有,这才是做官。”一个执掌蒙学的乡儒拍着弟子的脑袋,“好好读书,日后考中进士当了官,也能如此!”

“苏学士这两日告假携友重游何山,果然有佳作问世。”

苏轼仅是直史馆,尚不到侍制一级,离学士更是有千八百里,但外面的百姓却都是一口一个学士。

毕竟文曲星下凡……

苏轼在湖州不过数月,从秋至冬而已,山山水水都逛了一遍,已经有了几十篇诗词出来了。一篇即出,立刻就是城中传唱。

而在州衙之中,也无人称他知州,而是直史——苏轼文名广布天下,怎么能不以文学之职称呼?

但通判祖无颇就没那么高的声望了,苏轼在城外名胜之地吟诗作对的时候,他还在倅厅里埋头于公事之中。吃了一半的午餐放在一边,手上的笔始终不停。

案头上的公文堆得老高。年节刚过,湖州治下州县被耽搁下来的公事,一下呈了许多上来。而知州苏轼则是请了病假,和来访的朋友出去游山玩水。湖州衙门中的大小事务,也就全压到了权摄州事的祖无颇身上。

祖无颇一封封的批阅着公文,他的亲信幕僚,领着两名抱着账册的小吏进了厅来。

到了祖无颇身边,幕僚低声说道,“通判,刚刚过了上元节,州中公使钱已经去了两成。寒食、端午都少不了设宴祠神,若是再这样下去,恐不及年中便会用尽了。”

“反正之后会有人请他。”祖无颇头也不抬的说道,“苏直史在杭州任通判三年,视其为酒食地狱,吃喝之事,勿须为他担心。”

幕僚脸上现了急色,他哪里是为知州下半年没钱游宴着急,州中的公使钱可不仅仅是用来招待客人的。

这时忽然听见厅外一片声,“回来了,回来了!直史回来了。”

祖无颇抬头看了看天色,还不黄昏,略感惊讶:“今天还真是早。”

“好象是苏直史的兄弟从南京派了人来。”幕僚压低了声音,凑近了道:“好像有什么急事,前脚进了后院,后脚里面就派了人去寻苏直史了。”

祖无颇放下笔,“莫管他人家闲事。”说着,便出厅迎接知州‘病愈’归来。

从侧门进院的苏轼一行人脚步匆匆,感觉上都有些慌慌张张的。尤其是领头的苏轼,像是失魂落魄一般,全然没了旧时的闲雅。若在往常,如何会如此有失士大夫风范?

祖无颇心中疑云大起,心中揣测着到底是出了什么事。难道是小苏有什么不测?

猜测归猜测,亟待处置的公事却仍是少不了要向苏轼禀报,“直史,昨日衙中收到漕司公函,命州中督设保赤局,专一管勾种痘之事。种痘的痘苗将在二月初送抵州中。治下各县需遣人来州中学习种痘之事,最晚要在五月之前在各县中开始为百姓种痘。”

“此事由公方你全权处置。”苏轼很是不耐烦说了就走。

祖无颇还想说话,可苏轼已经大步流星的,转眼就进了知州一家居住的后院。

