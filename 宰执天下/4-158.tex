\section{第23章 天南铜柱今复立(上)}

王师交州大捷。

这个消息,几乎在一瞬间传遍了东京城中。

一开始,还有人半信半疑,都觉得毫无先兆的一举功成,未免太过突然——告捷信使所经过的道路毕竟有限,但等到从宫中传出消息之后,情况就不一样了。

宋人的确厌武,可那是对西北二虏几十年屈居劣势的结果。王师连连败绩,当然没有人会喜欢战争。若是能够百战百胜,又有谁会对战争感到畏惧?

多少人在赞颂着诸位将帅的功绩,更有人即席赋诗,呼唤着王师在平南之后,能征伐北方。

而当王旁带着这个捷报回家来的时候,王旖正在藏书楼中。

王安石为人邋遢,但他的藏书阁却是干干净净的,连丁点灰尘都不见。上万卷书依照着私家编订的目录,整齐的排放着书架上。

王旁快步的走进楼中,王旖听到动静,便举着手上的一卷书册,扬着给王旁看,“二哥,爹爹的这部《唐百家诗选》的手稿,怎么不见了第七卷?”

二十卷的《唐百家诗选》是王安石还没有入京时,集合了自己挑选出来的总计一千多首唐诗编纂而成。不过这部诗集并没有收集李白、杜甫、白居易等名家的作品,也没有孟浩然、王维、韩愈等人的诗句。

王安石放弃了诗集在世间多有流传的名家,选取了名气基本上都不大的若干诗人的代表作,免得他们因为声名不彰,而落得佳作失传的结局,他在序中还说:‘欲知唐诗者,观此足矣。’这一套诗选的刻印版如今国子监卖得正好,而王安石亲笔撰写和修改的手稿,更是只有藏书楼中唯一的一套。

不过王旁那还有心思去在乎少了一卷的诗选,“这时候还找什么第七卷!交趾那里有消息了!”

啪嗒一声,王旖手上的书卷落在了地上,她脸色煞白,颤抖的双唇满是紧张,“交趾那里怎么了?”

看到妹妹误会了,王旁连忙解释道,“赢了,官军赢了,已经攻下了升龙府!”

王旖脸上的血色终于恢复了,忙着细问内情。

“因为被雨水坏了道路的缘故,安南经略招讨司连着七八日派出的信使最后是竟是一起到的。”王旁将自己所得知的,今天白天时崇政殿中发生的事,原原本本的告诉了妹妹,更笑道:“要是提前知道章子厚和玉昆随军过了江,张方平也不会出来丢人现眼了。”

其实这段时间以来,朝堂的旧党们,已经见韩冈当成了瘟神一般,连御史们都聪明的不再去找韩冈的麻烦。无论遇到多大的麻烦,在韩冈的面前,都是如同举手可治的小事而已。也只有张方平这位对如今的朝堂来说,已经是个陌生人的所谓元老,才会糊里糊涂的去攻击有韩冈参与的事务。

张方平与欧阳修在政坛上是死对头。不过苏洵、苏轼、苏辙父子三人还未知名时,却是张方平将他们举荐给当时的文坛领袖欧阳修。他在文坛名声不恶,但在朝堂上却是被人群起而攻。而现在他与西京的元老们一唱一和,却是将自己的遭遇加诸新党之上。

王旖可不在乎张方平怎么样了,她只关心着丈夫的身体,还有丈夫的归程。

“旖姐姐,官军在交趾打赢了。”周南欣喜中满载着兴奋的声音远远地就响起,从楼外传进来,“官人快要回来了!”

周南从出外买菜的仆婢那里听到了一点风声,就忙着过来通知周南。只是没想到王旁也在。尴尬的停住脚,敛衽为礼。王旁点了一下头算是打过招呼,立刻就出去了,瓜田李下,这嫌疑他可不想背。只在藏书楼中留下了王旖和周南两人。

韩千六、韩阿李在王旖生的韩冈第五个儿子满了三个月之后,动身返回了巩州。这段时间里,老夫妻两人好生的将东京城里里外外逛了个遍,因为王安石和韩冈的缘故,他们倒也颇受礼遇。等到二老离开,韩冈的四位妻妾又搬去了相府中居住。

“想不到赢得这么快。”周南欣喜的说着,“对了,还要去跟云娘和素心说一说。”但她很快就忧心忡忡起来,“官人在交趾取胜的事到底是真是假?”

“方才二哥已经说过了。”王旖道,“等爹爹回来,就能确认了。”

等到王安石从宫中回来,已经是入夜时分。

在厅中做下,孙子和外孙都上来请安,虽然王雱的儿子年纪最大,不过韩冈家的长子韩钟,却总是更大胆一分,磕过头,就趴在王安石的膝盖上,扬起小脸:“外公,爹爹是不是赢了?”

“嗯!你的爹爹是赢了。”王安石将外孙抱起来,一本正经的与孙子辈说着话。不论是孙子还是外孙,他都是疼爱有加。而且有了女婿一家住进来,宰相府里面也算是多了人气。

抱着孙子说了一阵话,王安石带着王旁进了书房。

“玉昆是灭国之功,”王旁坐下来就兴奋的说着,“等他回京后,正好可以帮着爹爹。”

“玉昆接下来几年,多半是只能在外任官了。”王安石没有避讳什么,这基本上已经确定了,“灭了交趾,章子厚回来后,一个枢密副使是少不了的。而玉昆居中功劳不让章子厚,此番若是回朝,同样少不了一个两制官。……为父是宰相,中书门下同平章事,不可能让玉昆去做中书舍人,他的功劳也不只是一个外制官。只可能去做内翰……二十五六的翰林学士。”

王安石说着,忽然抿起了嘴,唇边的笑容有着让人难以捉摸的味道。只是说出口而已,但王安石还是觉得韩冈的境遇实在是不可思议。

翰林学士是踏上宰执之位的最后一级台阶,王安石、王珪、冯京、吕惠卿无不是如此。可王安石做到翰林学士,是当今天子登基时候的事了,而曾布、吕惠卿和章惇虽然都比他早,可也是三十多快四十的样子。二十五六的翰林学士,那他接下来晋身两府又会是多少岁?

不循资升官,在一般官员眼里就是新近。早间出了丑的张方平,他前日上表批评役法,里面还是口口声声的骂着眼下的新党成员都是新近之辈。韩冈若是身登内制,不知会惹来多少议论。

而且自家的女婿功劳虽说摆在那里,但年纪的问题,就算是天子也会感到忌惮。三十上下入两府,几十年的宰执坐下来,日后谁还能制得住他?

王旁当然不会怀疑父亲的话,不过想了想,就笑了起来,“不过说起来,玉昆也不擅文辞,这个翰林学士做起来也不安稳。”

“司马君实也不说过他不擅四六吗?”王安石摇头,司马光说自己不擅长做四六骈俪的赋文,当然也无法代笔些诏书,但天子不照样用他做了翰林学士。“更何况,不加知制诰的翰林学士也是有的。”

“原来如此。”王旁点着头。不过他立刻又觉得纳闷的问道:“那爹爹你今天心情不佳,就是为了此事。”

王安石沉吟了一下,对儿子道,“你可知道最近天子在福宁殿上的屏风亲笔题了杨大年【杨亿】的一首诗。”

王旁摇了摇头。自家怎么可能会知道宫里面的这等事,他管着在京粮料的库务,问问三司的家底还有多少,他倒是能说出个一二三来。

“是哪一首?”他问道。

“《闻北师克捷喜而成咏》。前面的都是空话而已,但最后的几句——前军临瀚海,后军缚阏氏。蓟北沙尘静,河南露版驰。河北诸父老,重睹汉官仪。”

这几句气魄倒是不小,但王旁听着就觉得挺奇怪,“杨大年不是一贯的缀风月、弄花草吗?这诗可一点都不像他写的。”

“杨大年一直都是主战的。澶渊之战,他是与寇莱公【寇准】一同促战。”王安石叹了一口气,天子将这首算不上多出色的诗句,抄写在寝殿的白屏风上,用意不言而喻,“收复燕云诸州,这是为父平生之愿,不过此事却半点也急不得。”

先是韩冈以千五破十万,如今安南行营又以万人灭交趾。若是说交趾人太弱,那么也有丰州和鄜延路,官军对上党项和契丹的胜绩。

短短的时间里,天南地北的一连串捷报,给人的感觉仿佛是在一夜之间,大宋就拥有了能压制、击败甚至并吞西北二虏的强大军力。

而天子本人也是明显在这么想着,对辽国的态度也是日趋强硬,从眼下的态势来看,同天节的时候,辽国的使臣多半就不会受到与过往一般的待遇了。

王安石对此十分忧心。要按部就班的来才是,但皇帝偏不。赵顼旧日对契丹畏之如虎狼,只是被契丹的使者讹诈恐吓了一下,便割让了代北的土地,这一桩事,也不过才过去了两年而已。

疮疤还没好透,眼下就开始转着攻打辽国的念头了。才两年的时间官军不会进步这么快,而辽国也不会极速衰弱,两国的实力对比并没有出现太大的变化。

王安石不禁暗叹了起来,如此变幻无定的心思,绝不是能做成大事的性格。

