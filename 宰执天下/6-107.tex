\section{第11章 飞雷喧野传声教(四)}

王居卿听得发愣,他倒是没想到,韩冈如此精通厨房炊具。

他不知道韩冈家里有个为了烧菜,能折腾着十八般兵器的大厨。

铸铁饭锅,锻铁炒锅,一出来家里就预备上了。铸铁饭锅烧出来的米饭,比起寻常的饭甑和瓦罐,口感要好不少。用来做炖菜,炖鸡、炖肉什么的,火候也同样不错。这样的铸铁锅,再镀了珐琅就可以拿出来加个十倍卖了。而锻铁小炒锅更不必说,天天都能派上用场,也让家中的菜单上增添了不少新菜。

“扯得远了,还是说怎么发明的铁范。”韩冈不好意思的笑了笑,示意方兴说下去。

“都说格物致知,我等不留意的什物,参政看一眼就记在心里,什么都能说上几句。像方兴这般,看了就过去了,转眼就忘光。”方兴跟韩冈有情份,笑说了两句,然后对王居卿道,“徐良离开了板甲局后,就被分配去造大锅。那种径长三五尺的大锅,不可能锻打,只能铸造。可他是锻铁匠转任,手艺不行,学人用泥模子来铸锅,总是有沙眼,而铁锅又不能太厚,铁料多了就亏本了,所以他铸出来的锅每次都会漏水。后来这徐良就不知怎么的便想到了先弄个铁模子,再在铁模子中浇铸个锅子出来,倒是给他闯出了一条路子出来。”

王居卿听得感叹不已,“也亏了那徐良在铸铁上是生手,否则还不会有这铁范。”

“是啊,要是他传习了泥范铸锅,知道怎么避开沙眼,就不会去琢磨其他办法了,更不会有铁范。有了铁范法,便比泥范要强出许多,大小质量都没得比,铸造的速度也比不上,所以东京市面上的铁锅这两年几乎都改官造了,往南、往北,也是官造的多。监中也知道有这么个人,等到参政提议要新设火器、铸币两局之后,臧监丞直接就把那徐良给调回来了。”

“现在徐良就在火器局中?!”

“不,是在铸币局里面。铁范法用在铸币上时有些麻烦,所以把他要了过去。”方兴叹道,“现在看起来,铁范铸器暂时只适合大的物件,小如钱币,就有些难了。”

“方才说徐良是为了铸锅才发明的铁范法……”

“的确。”方兴点头。

“但也该加以奖誉才是,毕竟这让朝廷节省了多少耗用?节省了多少时间?用在铸锅上就让朝廷得益甚多,用在铸炮上更是不知多少好处。当年区区一个酿酒的连灶法就给吕三司赏了官了,这铁范法可是远在连灶法之上。”王居卿连声为徐良抱不平,转头对韩冈道,“参政,若徐良还未得封赏,居卿今日回去,就给太后和中书上书。”

王居卿和吕嘉问的心结,从当年吕嘉问吞了连灶法的赏赐之后就结下了。尽管这些年来,王居卿知吕嘉问势大,故而隐忍不发。但这一回在廷推上,便从背后捅了吕嘉问一刀。现在一说起新工艺带来的好处,还不忘踩吕嘉问两脚。

“是准备给的。不过若是现在就给官,怕让北虏知道这铁范法的好处。”方兴说道,“就怕给契丹人知道,所以才调去的铸币局。”

“这样啊。”王居卿平静了下来,点头称是。

火炮这一军国之器,如今借助韩冈的名声早已遍传天下,辽人不可能不打探。有飞船、板甲和种痘法的先例在,辽人对火炮会有多重视也可想而知。为了防止他们了解工艺制造上的细节,什么样的应对都不嫌多。

“其实这铁范法就像当年以板甲代札甲、鳞甲,工时和成本都降到了之前的几分之一,馈赐当如神臂弓和板甲例。”韩冈说道。

自从神臂弓、斩马刀、板甲、飞船名震天下之后,每年献上来新式武器不知多少,但合用的寥寥无几,其中适合量产的更为稀少。而对于工艺上的改进,也同样如此,数量不少,堪用的不多。能与板甲相比的那是一个都没有。如铁范法这等军民皆宜的新工艺,这些年来还是第一个。

“只是怕北虏知晓,所以等徐良他在铸币局稍待时日,才会将他提拔起来,钱币铸造也需要一个有想法的大匠来管着。要是他能改进铁范法,用在铸币上,那就更是名正言顺了。”韩冈又冲王居卿笑笑,“铸一板制钱,就要换一次模子,这耗费也不少。钱币价本廉,能少耗用一分就是一分。”

这是下面的提议,韩冈总不能公然说自己不在乎辽人去制造火炮。而且相对于火炮的外观,工艺技术更为重要。看了火炮的外观,模仿起来不难,但工艺却不一样。韩冈还是希望看到当辽人用上同样的武器与官军对阵时,突然发现性能上天差地远、数量上天壤之别时的蠢样。

“参政说的是。”王居卿点着头,停了一下,忽然问道,“不知青铜炮可不可以也用铁范来铸?”

方兴回道,“尚在试作中。应该可以,最多只要改一改。”

“若是能成功,就是莫大的喜讯了。”王居卿赞了一声,转问韩冈,“虎蹲炮是铁制,野战炮和城防炮还是青铜质地。敢问参政,这野战炮与城防炮,是否能改成铁制,还是只能使用青铜?”

韩冈道:“铜价远贵于铁价,当然是用铁制的更好。只是因为野战炮与城防炮与虎蹲炮结构不同,发射时炮壁受力远过于虎蹲,担心铁炮像爆竹一样炸开,才改用更有韧性的青铜,日后技术如有突破,还是要改作铁炮方好。”

“铁炮若能成功,数量能十倍于青铜炮,一年当在千门以上。”

方兴在火器局中的时间不短了,而且是一手控制火器局的发展起来的元老功臣,对家底当然比谁都熟悉。

“若当真一年千余门野战炮、城防炮,床子弩就再也派不上用场了。”王居卿对韩冈笑道。

韩冈随即问道:“那时候,军器监打算怎么做?”

王居卿一愣,想了一想,“撤除吧。与青铜炮比起来,床子弩的成本也高了许多,更不用说铁炮了。能够使用的时间又短,保养上也麻烦。其实居卿这些天,正在考虑如何处置弓弩院中造床子弩的那一部。”

韩冈点了点头。在他的想法中,军器监专门制造床子弩的部门,将削减编制和生产量。最后最多只维持一小队工匠,以保证技术的传承。其他全都转去他处。王居卿也能这么想,就免得自己再费口舌了。

“那些库存的床子弩可不在少数。”韩冈又道。

王居卿这下就想不出韩冈想说什么了,欠身道:“请参政明示。”

韩冈说道,“已经造好的床子弩,还可以送去西域。面对黑汗国的军队,神臂弓和床子弩足够应付了,且床子弩拆开来运送,比火炮要轻便一点。”

尽管王舜臣那边也需要火炮,可暂时他还不打算运火炮过去。重量是个问题,同时对手是个文明程度不亚于中国的国度,与千年之后截然不同。由于道长途远,大宋的国力优势无法体现,当技术泄露之后,无法在数量上压倒对方,而质量上,以现有的水平还难以拉开一个让敌人无法企及的距离。

如何分派已经造好的军器,这是枢密院的工作,跟王居卿完全无关,与韩冈也没有太大的关系,但这并不妨碍王居卿附和韩冈的话:“枢密说得是,用床子弩对付黑汗人的确是足够了。”

韩冈笑了一下,与王居卿一起,继续参观起这座火器试验场,从试射场转到守卫和实验人员的房舍,再从守卫们的房舍转到仓库,又花了两人不少时间。

这两天,这里都是在测试虎蹲炮,青铜的野战炮、城防炮还没有拖过来,韩冈、王居卿两人也没有继续让火炮声继续折磨自己耳朵的想法。

来到宽近两丈的寨墙上,韩冈俯视着被城墙圈起的这一片试验场,问方兴道:“火器局现在还有什么需要?”

“火药的原料还是少。”

经过大批量的实验,火炮发射药的原材料,已经找到了比较合适的提纯方法,同时三种原料的配比也确认了,成品制造也有了规程,可以说已初步定型。

除了两个小组依然在实验威力更大的火药,其余工匠都被调去解大量生产上的设置与安全问题。

现如今,唯一的问题就是原材料的数量。

可既然几百年后,普及火器的两个王朝都不缺火药,韩冈也不觉得自己需要担心。

“中国之地,却也不缺硫磺,木炭、硝石数量更多,只要朝廷有需要,肯定不会缺。”

“参政如此说,方兴也就安心了。”方兴说着,又叹:“听说日本多硫磺,辽国这一回占了日本,造火炮可就更容易了。”

“学也学不像,赶不上中国,不用担心太多。”

韩冈不怕辽人偷学火炮,就怕他们不学。他是真想看到拖着几千斤大炮行军的契丹骑兵。不能随着骑兵机动的炮兵,辽军可没有相应的战术来配合。就像大宋官军,其战术体系完全是以步兵为核心,辽军的战术体系也是围绕着其最为擅长的骑兵来展开的。

“是。”方兴拱了拱手。

“火器局是军器监的重中之重,日后也要寿明多费心了。”

“居卿不敢有负朝廷与参政之托。”王居卿回顾方兴,“且有提举在,参政也不用担心。”
