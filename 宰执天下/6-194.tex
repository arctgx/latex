\section{第21章 欲寻佳木归圣众(14)}

“这么说来,现在朝堂上,还是韩、章二人的天下?”

两日之后,这一次廷推的结果,已经传到了相州。

“还用说吗?御史台那些不长眼的,才跳起来蹦跶两下,就给踩死了。”

昼锦堂中,刚刚从京师带着消息回到相州的韩正彦,正将他前几日在京中的见闻,原原本本的告知他正要顶替的现任知州韩忠彦。

身在龙图阁,职份小龙,韩正彦自有份参与这一次的廷推。尽管他再一次就任相州知州的诏命,早已签出,但现如今朝廷已有规定,只要预定的廷推日期在受命的一个月之内,即将外任的议政重臣,都可以留到廷推之后再走。而不用像一开始的时候,想方设法让自己病上一场——这样的情况多了,廷推的严肃性也不免受人诟病。

“不过也不能算是不长眼。”韩正彦补充道,“文德殿廷推上闹事,比平日里更显眼,官家坐在上面看着,也能记得清楚是谁。”

“是赌马赢多了?”

韩忠彦轻笑了一声,敢将身家性命压在当今天子身上的可没几个。不说他当年犯下的罪孽,先看看仁宗皇帝多大年纪才亲政,再看看当今天子的身子骨,这份赌注九成九是打水漂了。就是买球券、马券连中个十次八次,也比押中天子的几率更大一点。

心知韩忠彦在笑什么,韩正彦道:“自然,押官家一注是一回事,另一边,也是有人在背后推波助澜。”

“是福建子在背后指使?”

“不管是不是他,现在他已经改知江宁府了。”

“这么快!”韩忠彦惊讶道。

韩冈、章惇还真是一点也不耽搁。廷推刚结束,就敲定了吕惠卿的罪名,彻底要将他给压在地方上了。

“怕也是敲山震虎。没有金陵那边的同意,龚原应当不会冲得那么前面。”

“集贤相的老泰山都说是专注教书了,还听说为了跟他的好女婿打擂台,精神是越来越好。现在看起来,或许是好过头了。”

韩忠彦和韩正彦说着一齐哈哈大笑起来。王安石和韩冈这对翁婿间已经持续了十几年的明争暗斗,对大多数士大夫而言,实在是喜闻乐见的一桩趣事。

“说道女婿,”韩正彦问道,“家里的官家女婿咋样了?”

提起自家的亲弟弟,韩忠彦的脸上没了笑容,“还是成天玩他的那些瓶瓶罐罐,往东南角一走,全是怪味,哪里能住人?入夏后就去了城东庄子上了,烧了房子都随他。”

“喜好格物之学也不是坏事,嘉哥儿都做了驸马,也考不得进士了。多用些心思在瓶瓶罐罐上,比学小王都尉日后在脂粉阵里混要好。”

韩忠彦闻言发狠,“若是五哥当真跟王诜学卧花眠柳,腿先打断!”

长兄如父,韩琦不在了,韩忠彦可不会让自家的弟弟变成只知败家的纨绔子弟。教训起来,绝对狠得下手。

“嘉哥儿自小聪慧懂事,不至如此。”韩正彦劝了一下,又道,“不过雍国长公主那边还得派人去问个好。这情分越多越好,嘉哥儿要与长公主成婚,过上一辈子,总不能与那王诜与大长公主一般,成了冤家对头,弄得家中不靖。”

“四时八节,何曾失过礼?五哥也常写信……”

“这不是好事嘛。”韩正彦立刻叫了起来。

“可他信上说的什么啊?是硫酸浇糖霜,弄出一堆黑沫子来?还是用胆矾水给铁器上了铜色?拿着刀去给蛙啊、蛇啊开膛破肚,我都不好说了,血淋淋的东西,记下来让娇生惯养的小娘子看?!”

“这个……”韩正彦也不免张口结舌,这个了半天,终于道:“比诗词好,不是吗?”

韩忠彦哼了一声,却又没有反驳的话。已经订了亲的未婚夫妻,相互通信不是什么见不得人的事,日常使人登门问个好、送些礼物,更是再普通不过的交往。可是这些事,就要做得发乎情止乎礼,若是在信中写下一些挑逗性很强的诗赋,不会被罪,却也免不了一个轻佻之名。

先帝留下的第三女,如今的雍国长公主,早在几年前,便与韩琦的五子嘉彦议了亲。而韩琦在最近一次郊祀后,更被追晋魏王。虽说身为韩琦长子的韩忠彦,不仅仅即将成为皇亲国戚,还不断受到了亡父带来的荫庇,但他在意的事依然遥遥悬于天际。

其实当初与韩家议亲的时候,两边的年纪都还小,照常例,不到十四五,朝廷压根就不会为公主开始选婿。这般早早的定下亲事,一方面是当年太后初秉政,行事偏向新党,让西京难看,需要安抚元老重臣,另一方面,在韩忠彦看来,也有借助相州韩氏的余威,来压制把持朝堂的一众宰辅。

只是这么些年了,韩琦留下的余威越来也不管用。在相州,朝廷给足了韩家颜面,大宋开国百余年,何曾听说过堂兄弟来回在家乡担任知州,这可不是府州,或是南方的羁縻州,是河北重镇相州,是殷墟所在、京保铁路经过的相州。可是在朝廷上,最有希望的韩忠彦,一直被拒之门外,距离两府之位,也是有着一段遥不可及的距离。

“等交割之后,我就要去京城了,五哥在乡里,十三你平日里,还要多关照一下。”

“家中兄弟,何出此言?”韩正彦摇头,又道:“还望哥哥能在京城中心想事成。”

“唉……”韩忠彦颓然长叹,哪有这么简单。

韩忠彦有天家姻亲的身份,可毕竟还是隔了一层,加之有韩琦遗爱,正常来说,朝廷不会对其关上两府的大门。

只是韩冈和章惇的默契,是有志于两府的其他臣子的灾难。

但两位党魁并不排斥引入新人,进入议政重臣的行列,这一回廷推,总票数比第一次廷推多了五成,这不仅仅是因为侍从官以上的重臣,减少离京人数的缘故,也因为存放熙宗皇帝诏令、墨宝的显谟阁已经修好,光是议政重臣的数量,就比原来增加了三分之一。

韩冈秉政多年,当初他在廷议上做出的承诺,一个个的付诸现实。

韩冈完成了他的承诺,国库充盈,民生安定,朝堂安稳,国势日盛,对外又将有灭国之功,韩冈少不了一个运筹千里、决胜庙堂的评价。如此贤相,民心士心都不缺。

最重要的是,太后依然对他言听计从,这样一来,还能指望他过两年便被赶下台去?为其他人留下朝中的空缺?

从京师出发,向四方而去的道路,更是越发的畅通起来。韩正彦清早从京师出发,当天落日前就过了河,第二天还不到中午便抵达相州州城。这不是拿着金牌的急脚递,而是拖家带口近百人的大队人马。这事要放在十年前,谁敢相信?谁会相信?

这是韩冈带来的变化,亘古以来不曾听闻,如此功业,又得太后信赖,年纪更是让人绝望,韩忠彦真的觉得自己其实不用再费心了。

只不过,尽管一切都心知肚明,但他还是有着浓浓的不甘心。

韩忠彦陡然间安静了下来,韩正彦看着他,一切明了于心。

“其实还有一件事,方才没说。”韩正彦说道。

韩忠彦回过神来:“还有什么事?”

韩正彦低下声来,神神秘秘:“我出来的时候,正听闻政事堂和枢密院在计划要对京泗铁路进行压力测试。”

韩忠彦一头雾水,“什么意思?”

他完全不明白,什么叫做压力测试。离开京师这些年,难道就这么落伍了,连说的话都听不懂了。

“就是尽可能的给京泗铁路加担子,往他们身上压石头,将铁路上的人都累着,看看他们能撑多久。所以叫做压力测试。”

“怎么个测试法?”韩忠彦没问为什么,两府如此做的用意一眼就能看清。

“据说是将二十个指挥的禁军连马带装具一起送上车。一路运到泗州,下车休息一晚之后,再从泗州坐车回来,用时不能超过十天。”韩正彦停了一下,补充道,“这是我临出门时听说的。”

十天之内,将二十个指挥的禁军往近千里外地方,运送一个来回。

这是闲得没事干了吗?还是突然之间,脑袋被火辣辣的太阳给晒坏了吗?

韩忠彦突然发现自己完全无法了解韩冈和章惇到底在想些什么,这么多人的运输,地方上要鸡飞狗跳,京师也会人心惶惶,好用的钱粮更会是难以计数,

虽然说轨道修建的第一目的,就是为了抗衡辽国,让官军不至于在自家的土地上千里奔波,从而耗去了所有的气力。

可这个实验,实在是过于异想天开了。

这等于就是练兵,防止事到临头,所有人都没有经验来处置急务,但这样做的话,因此而带来的损失,将是难以计数。

“太后答应了?”

“不知道,想来应该不会拒绝。不是吗?”韩正彦冷笑着,“他们总有办法说服太后的。”