\section{第33章 为日觅月议乾坤(八)}

王旁在京师住了有好些日子了。

为了女儿的婚事,他在京城中的日日夜夜,都是在紧张和不安中度过。

他还记得刚刚抵京时,韩冈曾经设宴邀请他。宴后,韩冈与他说起这一桩婚事,直接就说当今的皇帝非是良配。

除非是娃娃亲,双方都成人时议亲,家世门第要看,但最重要的还是品性。

韩冈作为臣子,直接就说皇帝的品性不好,越娘嫁过去后,会被耽误一生。

韩冈当时就声明,这番话非为权位,也不是为了政争,只是为内侄女担心,否则他要阻止的话,当日先一步下聘就好了。所以他话只在私下里说,到了公开场合,他绝不会阻止越娘成为皇后。当然也不会赞同,什么都不会说,也不打算做。

王旁不知道韩冈说的是真是假。

尽管这段时间来,韩冈的确对皇后的人选不发一言,但以他的权威,只要一句话,不论是什么时候除非已经下了聘他都能将局面彻底翻过来。

心中烦躁,换了身衣服,王旁他也不带着人,独自一人出门去散心。

京师之中,消遣的去处很多。王旁上了一辆马车,走了半日下车来,随便在街边找了间酒馆坐下。

但即使是在外城偏僻小街中的小酒馆中,依然不缺乏指点江山的酒客,以及一肚子宫闱秘闻的闲人。

这就是京城的风俗。

王旁刚刚坐下来,还没点酒菜,就听到旁边的一桌上有人在说:“韩相公这次可是吃了大亏了。”

“何以见得。”

说话斯文,王旁看过去,却是一个有张毛胡子脸的大汉。

说话的人背着王旁,看不清相貌,“韩相公拦着,是他恋栈权位。不拦,就要受到拖累,韩相公怎么做都没好处。除非不选她做皇后,否则日后吃亏的地方更多”

“都是扯淡的话!”大汉捏着蚕豆,一点点的剥着皮,“只要韩相公不愿意,他轻而易举的就能将王楚公的孙女给否决掉。真当韩相公做了那么久的宰相、参政是假的啊?还以为退隐江宁的楚国公是当年在京师叱咤风云的拗相公?”他不屑的冷笑着,“韩相公连话都不用说,只要对门下的走马狗比个手势,就能让他们把事情给办妥了。”

跑堂的小二站到,等着王旁点菜。

等王旁随便选了一壶米酒,两份下酒菜,已经跳过了几句话,就听见那个背着自己的人说,“狄家的女儿也算是出色。”

大汉道:“什么叫也算是?一个两父三母,祖父还是武夫,另一个却是元勋之后,姑父更是权臣,两人现在评价相当,哪个更出色?”

狄家小娘子相貌在京师已经出了名,沉鱼落雁、闭月羞花之类的修饰词,用得都滥了。品性上,据说也是一等一的贤淑温良。但王旁岂会认为自己的女儿会输给别人?不由得就皱起了眉头。

“相貌太出众,其实也不好。”

“又不是她的错。”

“是天子心性,万一沉湎女色,为奸人所趁,国事不知会如何变了。”

那大汉失声笑:“几位相公怕是就盼着皇帝只在后宫生孩子,外面的事,全都交给他们去操劳好了。”

京城人什么都敢说的脾气,王旁算是又领教到了。但他说的,未必不是韩冈等人所想。

自家老父,是不是就看着这一点,才会让孙女去待选?王旁不清楚,王安石也从来没有跟他明说过。

不过王旁希望如此,他不想自家老父让越娘入宫,是因为看见自己不成材,想让王家有个更加安定的未来。

“宰辅刚才都被招入宫中了。”坐在角落中的一人转过身来,看此人身上的服色,是个积年的吏员,“今天曾参政休沐,方才就急冲冲的过去了,说不定今天就要把皇后的人选议定下来。”

王旁心咯噔一下,其实他也能感觉得出来,决定皇后人选的日子就在最近了。

难道就在今天?

……………………

‘今天看来是决定不了了。’

当韩冈的话一出,殿中顿时静无一声。

张璪一阵心惊肉跳,也亏韩冈敢说。什么皇后啊,什么嫔妃啊,全都得丢到一边去了。

韩冈这是直接要跟皇帝过不去了。

独夫谁人?商纣,夏桀。

齐宣王曾问孟子,‘汤放桀,武王伐纣,臣弑其君,可乎?’,孟子则回道,‘闻诛一夫纣矣,未闻弑君也。’

成了独夫,臣子杀之不为弑。

富弼当面说伊尹之事臣能为之,但伊尹也只是流放太甲,三年后还迎了回来,而韩冈却更进一步,明说君若为独夫,臣子杀之无碍。

这话别说让皇帝听了,就是让他这个做臣子的听了,同样让人不寒而栗。

他看着对面,曾孝宽、邓润甫都一脸惊容。

包括气学在内,新学、道学等如今流传最广的三家学派,都是思孟一系。但敢在朝堂上把独夫挂在嘴边的,可就韩冈这一位大儒。

但最上首章惇早就不会为韩冈的观点而吃惊了。

一心想要让皇帝垂拱而治的韩冈,没有抱着这样的想法,反而是奇了怪了。

那一句‘天下人之天下’正说进了他的心里。自家的产业,怎么会是皇帝的产业?就是皇帝自己,也不敢随意将别人家的产业变成皇产。

但将这句话光明正大的说出来,却是公然否定了天子对天下所握有的权力。

这绝不是一时意气,或是有感而发,自是有着深刻的用心。如果不然,韩冈就不配站在这内东门小殿中。

‘天下是天下人的天下,非是一家一姓的天下。欲以天下奉己身,非是天子,乃是独夫!’,传将出去,便是千古名言。

也许为了说出这句话,韩冈等着发难的机会等了很久了。

既然如此,章惇也不打算落于人后。

他举步出班,“韩冈所言正是。天下,亿兆万姓所居,天之属也。天子,代天牧守者也。岂得闻子可夺父产?又岂得闻代人放牧,可将所牧之物据为己有?太妃当慎言,以免累及天子。”

章惇的话,与韩冈前后呼应,拿着朱太妃的话做文章。说是不要累及天子,却明摆着要将事情牵扯到皇帝身上。

换作是皇帝,遇上两位宰相同时发难,也得低头服下软,除非想做鱼死网破,那倒是可以唤可信的御卫来将两个措大打杀。

但放在章、韩两位宰相身上,便是唤了御龙直的人上来,唤了金枪班的人上来,又有哪个敢对他们举刀?

张璪的双眼在韩冈和章惇身上来回打转,脑筋也在不停地转动,他们为什么不怕皇帝日后报复?

不管他们立下多少功劳,对皇帝都没有意义。再大的功劳,也抵不过侵犯皇帝权柄的罪过。而韩冈、章惇近乎肆无忌惮,那么理由只有一个,他们不担心。

至于为何不担心,原因就太简单了小皇帝或许根本就没有日后。

韩冈怕是早就诊出天子的寿数不长,活不过他,也活不过太后!

张璪的双眼亮了起来,既然这样,那自己为何还不敢插上一脚?

“陛下,臣闻狄氏女容色为诸女之冠,又曾闻天子曾于后苑携千里镜登高。太妃殿下一心想要为天子纳狄氏女为后妃,究竟是太妃所欲,还是天子所欲?”

张璪的话直指天子,质问其品性。太妃若是不肯认,那事情就得是天子担下来了。

连枢密使都出来了,文武两班的首脑一齐发难,朱太妃只有低头认错,难道还能将责任推到他儿子身上?

殿上气氛如同绞紧的弓弦,绷得越来越紧。

群臣都等着朱太妃的道歉。

只是屏风后,传来了一阵嚎啕大哭声,哭声断续,口齿又不清,只听得‘孤儿寡母……乱臣贼子……太后做主’云云。

几位宰辅顿时面面相觑,遇上女人夹缠不清,这下子还真难办了。

章惇皱眉,所以说牡鸡司晨就是麻烦,太后在旁边都不呵斥一声,就看太妃殿上失仪。

偏头冲韩冈使了个眼色,让他去处理。

韩冈抬头直视屏风,怒声呵斥:“先帝昔年病重,臣随侍在侧,权同听政之语,只闻予皇后,不闻予德妃。先帝内禅,臣同样随侍在侧,权同听政的诏命,亦只闻予太后,不闻予太妃。帘后何人,敢于在殿上放肆!”

韩冈这是有着几分把握,朱太妃最近太活跃了,几乎把皇帝的婚事大包大揽,而太后这位嫡母由于种种顾忌,反而插不上话。

而且内东门小殿,本来只有太后才能来,太妃今天跟过来,虽是有着商议天子婚事的名义,但也是侵犯了太后的权力,不信她心里会高兴。

呵斥声犹在殿中回荡,屏风后忽的就一声巨响,然后又是一阵慌乱,一个尖细的嗓门叫道,“太妃晕过去了!”

如果是太后被气晕过去了,那是真麻烦。但只是太妃而已,韩冈真还不在乎,“太妃当是为天子婚事操劳过度,须好生休养数月。”

睁着眼睛说了句瞎话,就听见屏风后,太后终于开了金口,“相公说得是。快将太妃搀扶下去,传太医来为太妃诊治。”

屏风后一阵乱,太妃被扶了出去,几个月之内,就别想再插手赵煦的婚事了。

好好的议政之地,给弄得鸡飞狗跳,向太后叹了一声,也不知该怨谁,心力交瘁的叹道,“今天就到这里吧,这情形也谈不了事了。”

韩冈却要留着她,“陛下,无关人等即去,还请陛下稍留片刻。”

向太后无力的问道,“相公还有什么事要说?”

“陛下乃是嫡母,天子的婚事本当由陛下做主。太后忙于政务,将之交予太妃,但太妃见识不及,臣恐所选非人,恳请太后细择之。”

向太后苦笑道:“就怕那孩儿心中有怨。再出了一个郭皇后,岂不是害了人家。”

章惇立刻高声赞道:“陛下心慈,实乃天下之福,万姓之福。然此处并无吕夷简。宫中亦无阎文应。纵使天子妄为,自有忠臣贤良阻止。”

撺掇仁宗废后,朝中是宰相吕夷简,宫中是御药院阎文应。御史台一众御史上表阻止,吕夷简直接拒收。之后郭皇后暴卒,据说也是因为阎文应担心其回宫,而设法将其给毒死。

“若只有卿等在,吾当然放心。但朝中重臣,并非与诸卿心意相通。”

韩冈道:“忠臣贤良,自会与吾等同心同德。但正如陛下所言,朝臣之中,不免奸佞之辈。若天子圣德,定不会受其蛊惑。唯恐天子心思不定,届时,必至祸乱。”停了一下,他接着说,“太妃方才所言,如果只是出自己意,有太后在宫中,当无害于天下。但天子若有此心,则大宋危矣,天下危矣。臣有一言,有犯圣颜,还请陛下见谅。”

“无妨,相公请说。”

韩冈图穷匕见:“太后日后撤帘,将如何约束天子?”
