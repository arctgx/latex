\section{第三章 岂得圣手扶炎宋(下)}

“保护太皇太后!”

第一个反应过来的不是殿中的班直,那位被韩冈抢了手中武器的御龙骨朵子直的禁卫,还抱着韩冈给他的官服,张着嘴愣在那里。

尖声大叫的是齐王赵颢。

他人会疏忽,但赵颢绝不会!

尽管亲生儿子坐在御榻上,正要通过这一日的朝会成为天下之主,可赵颢的注意力却一直都在韩冈身上。这是猎人审视陷阱中的猎物的得意,可他的潜意识中,也未尝没有残留着对韩冈的警惕。

韩冈是完了,当他今天随着百官走进这大庆殿时,就已经走进了绝境。赵颢提了多少日的心也稍稍放了下来。但狗急跳墙的事从来不少,多少人在胜利在握的时候,却被带着同归于尽。

赵颢在四面钉上棉胎的房间中,住了近一年。对害得他入此监牢的罪魁祸首,一直保持着最大的戒惧。

除非亲眼看到韩冈被砍下脑袋,否则就算是韩冈传出死讯,已要发送出殡,赵颢也要在灵堂上,把盖在韩冈脸上的白布给掀开来看一看——就像传说中仁宗对夏竦做得那样——不确认一下,谁知道他是不是诈死?

韩冈出身寒素,又不是一开始就有了种痘法得来的名声!他能得王韶看重,是他敢作敢为,敢杀人,能杀人,手上有多少条性命,可以驱用来为鹰犬。

当他看见韩冈从班直手中抢过了武器,隐藏在心中的那份戒惧,立刻让赵颢及时的警觉了过来:韩冈虽败,却还有同归于尽的一招。

来自齐王的一声尖叫,让台陛上下立刻有了反应。

台陛之上,不仅仅是高滔滔和赵孝骞,也有捧香拿扇的宫人,有奉礼的内侍,还有……御龙直的禁卫。

包括那两名抱着韩冈衣物的御龙骨朵子直禁卫,他们守护的位置只是台陛最下方。天子身边最近处,是御龙直的防御范围。这些班直,他们不关心到底是什么人坐在御榻上,他们只需知道,谁能安然坐上去,他们就守护谁。

韩冈离御榻虽近,却还隔着这几名御龙直的禁卫。

这是大庆殿,皇宫的主殿,是皇城中最为雄伟的建筑,不是大臣的唾沫星子能溅到天子脸上的崇政殿。

韩冈还在台陛下,有五六人挡在中间,他要冲上去,就要面对班直中也是最精锐的御龙直禁卫。也许他们杀人的数量加起来都不如韩冈,可是自幼每日都要操演武艺,又是祖孙数代娶妻皆以身材长大为上,连身量都不输韩冈,只要他们居中一堵,韩冈便再无机会。

数级台阶,十步之遥,却是咫尺天涯。

尖叫过后,赵颢就安心了下来。

这才是真的完了!

下一刻,就能看见韩冈被乱锤乱刀打死在殿上!还是名正言顺,让任何人说不出话来!

但韩冈没有冲上去。

他反而退了!

退后了一步,两步,退到了与宰相班列平行冇的地方。

那里有王安石、有韩绛,还有……

蔡确!

谁都没有想到韩冈抢到了铁骨朵后,却不冲上台陛。

韩冈在一番表演后,抢夺武器的举动,已经让所有朝臣都难以置信,而他这一退,更是出乎了所有人的意料。

包括蔡确。

当韩冈抢到了铁骨朵,赵颢大叫着保护太皇太后,警醒过来的蔡确便指着韩冈,惊慌的喝骂着:“韩冈,你想做什么?!”

布衣之怒,伏尸二人,流血五步。

在那一瞬间,蔡确脑中闪过的是战国策中的故事。

韩冈不是普通的文臣,他杀人放火什么事没做过?当年章敦将韩冈介绍给他,曾赞韩冈大有古风。这古风,可就是说韩冈有着战国之士一般的胆魄,不是手无缚鸡之力的文人。

蔡确惊惶得追了出来,韩冈拼却一命,说不定就能将他的美梦打得破碎无存。

但他只是刚刚跨出,却不意韩冈转瞬间就已经退到了他的身前。

蔡确怔住了,他不知道韩冈为什么会退。可韩冈随之转移过来的视线,让蔡确立时明白了,韩冈的目标到底是谁?

这是他从来没有见过的眼神。

除了平静和坚定之外,还带着冰寒刺人的杀意。隔了近一丈的距离,那咄咄逼人的目光,却仿佛已经如刀剑刺到了脸上。

蔡确不由得向后一仰,想退得远远的,远离那位杀星。

可是已经迟了。

韩冈身形一动,箭步冲出。形如虎豹,一步便跨到了蔡确的面前。

右手中的铁骨朵早已举起,随着跨步冲前,猛力挥了下来。

韩冈的动作如兔起鹘落,只在瞬息之间,无人来得及阻拦半步。

殿上殴斗,本朝不是第一回。

太祖之时,开宝八年的状元郎,还是靠摔跤决出来的。

殿上见血也不是第一次。

太祖赵匡胤也曾经一玉斧挥下,将冒犯了他的大臣的两颗门牙给砸了下来。

但当殿捶杀宰相呢?

亘古以来,又有过几回?

韩冈挥起铁骨朵,带起的风声猛恶,这时候,大庆殿中反而变得静了。

噗的一声闷响,并不清脆。

但锤头凸起的地方,已准确的命中了蔡确左侧的额头。

惊骇欲绝的表情顿时在宰相的脸上凝固,然后又随着头部的变形而扭曲了起来。

‘为什么是……’

蔡确最后的思维也凝固了,陷入了永远的黑暗中。

直落而下的钝器,上面还带着聚力撕扯作用的凸起,只要有了足够的力道,就能一击破坏铁甲,将敌人砸得骨断筋伤,这是克制坚固防御的最有利的武器。

当这样的武器不受阻拦的落在了人类的头部,蔡确的头颅便如同西瓜一样破碎开来。半边天灵盖被铁骨朵的凸起掀了开,远远地飞了出去。

猩红的血液和白色的脑浆泼溅在下首处的曾布、张璪的脸上、身上,热腾腾的,在寒风肆虐的大殿上,还冒着丝丝白气。

一声沉闷的落地声响起,蔡确的尸身,被锤头上蕴藉的力道带着转了半圈,这才扑倒在地。

“韩冈!你敢!”

来自屏风后的尖叫迟了一步,太皇太后推倒了面前的屏风,只能看见蔡确尸横当场,还有韩冈悠悠然瞥过来的一眼。

曾布转身就逃,跌跌撞撞,甚至来不及擦脸上的血迹,韩冈的下一个目标,只会是最近的他。

可韩冈没有追击,将宰相一击毙命,他便退后了半步,脸色也回复平和。

如果不看他身上的斑斑血点,还有手中还在滴着红白色浆液的骨朵,只从表情上,根本就看不出他是一个刚刚取走了一条人命的凶手。

这一回政变的真正核心究竟是谁?

不是高滔滔和赵颢,他们这对母子的名声,在民间都糟糕的很。也不是被控制了的禁卫,皇城之外还有更多的军队。

而是蔡确。

在韩冈的眼中,蔡确在这场政变中的地位,绝不在高太皇之下,同样是不可或缺。

并不是坐到了御榻上,便是皇帝了。向太后和赵煦的权力,来自于先帝赵顼的授权,又得到了群臣和天下士民的公认。

现在高滔滔以武力坐了上来,没有名正言顺的权力让渡,就只能靠文武百官的认同必须有蔡确、曾布、薛向以宰辅的身冇份,带领群臣向参拜,认同了她的身冇份,如此一来,君临天下亿万子民,指挥百万冇大军的权力,才会拿到高滔滔的手中。

在这中间,作为宰相的蔡确最为关键,是曾布、薛向所不能比。宰执虽并称,但在制度上,宰相的地位要远远高于执政。无论是从待遇,还是从官阶,都差之甚远。

杀了赵孝骞,高滔滔还能拿出另一个孙子,杀了赵颢,更是没有一点意义。

只有杀了太皇太后和蔡确,才能将悬崖边的局面扭转过来。而孰难孰易,不问可知。

看了眼前几日还同席饮宴的宰相,韩冈心神稍放,已经成功了大半。

此时,殿中已乱作了一团。

“来人!杀了这丧心病狂的贼子!”

“来人,杀了他!快杀了他!”

太皇太后尖叫着,与她儿子的吼声合在一处,还有曾布的惊叫,“救命!救命!”

殿中御龙四直的禁卫终于有了动作,一个个听话的向韩冈扑来。

韩冈一声暴喝,冲着所有的班直:“奸佞已然伏诛,天子与太后尚在,你们到底听谁的?!”

对的。

天子和太后还在人世、

主心骨还在,也就意味着还有效命的对象。

方才韩冈的质问,用意就在此处。

当时不论太后和小皇帝是否还在人世,高滔滔和蔡确都不能承认已经先杀了他们,只能之后再找借口。而他们的回答,殿中的所有人都听到了。

“杀了他!”

“快杀了他!”

必须尽快杀了韩冈,否则王安石一动,局面就要彻底崩溃!

“杀了韩冈!”

赵颢指着韩冈,嘶声力竭。高滔滔也在嘶喊着。

殿中的班直们在短暂的犹豫之后,还是围了上来。

他们都已经从了贼,哪里还有退路?

赵颢兴冇奋得都要结巴起来,不停地重复着,“杀了他,杀了他!”

轰的一声,炸响在赵颢脚边。惊得这位二大王差点给噎住。

低头看去,一支铁骨朵当啷落地,赵颢脚边的金砖上出现了一圈裂纹。

这是从殿门处飞过来的武器。

只见一名低品的武将提着锋刃带血的腰刀,向韩冈这边疾冲过来。

另一人跟在后面,左手骨朵,右手腰刀。

李信!

王厚!

但两人离得太远,已经来不及再赶上。

可面对就要围上来的班直,韩冈依然不动,只抬头放声:“列祖列宗在上,臣韩冈以全家性命为誓言,今日之事,只诛首恶蔡确、赵颢、石得一、宋用臣!自此刻起,不从逆者皆有功无罪。如违此誓,天人共戮!”

韩冈的发誓让班直们脚步稍缓,韩冈在民间的声望,当朝无可比拟,在军中亦是人人敬服。他立下重誓,动摇了班直们的心。

“笑话!谁会信!都想死不成?”赵颢大吼着,“快杀!”

高太后也尖叫着:“诛杀韩冈此獠,便封节度使!”

“王平章!”韩冈冲着王安石大叫道,“还不立誓!”

王安石毫不犹豫,“历代祖宗在上,臣王安石于此立誓,今日之乱,只诛首恶,余者不问。自此刻起,不从逆者皆有功!擒杀首恶者,节度使!”他盯着班直,怒吼着,“你们还不退下!”

积年宰相,平章军国,王安石的威望更压人一筹。班直们寻常见太皇太后极少,见王安石的次数却多得很。

他与韩冈先后立誓,又是一喝,班直禁卫纷纷停下了脚步。

毕竟他们并非参与到政变其中,只是听命而已。而宰相积威,更是让他们肝胆俱寒。

只有一名御龙直的成员依然提着刀冲了上来。

“韦四清,领头的是你!”

张守约在下面惊叫,却只能眼睁睁的看着御龙直的都虞候冲到了韩冈面前。

韩冈没有与这位都虞候比较武艺,一支长剑疾飞而来,穿透了韦四清的小腿,让他扑倒在韩冈面前。韩冈随即抬起一脚,正中剑柄,让他痛晕了过去!

韩冈没有给政变帮凶补上一锤,冲没再失手的李信点点头,转头望着韩绛,“韩相公!”

“快上啊!杀了韩冈,就是节度使!”

赵颢疯狂的叫着,让几名班直犹犹豫豫的又开始前进。

韩绛也没有第一时间回应。

“子华!”王安石一声大叫。

回望着几十年的老友,韩绛叹了一声,又点了点头。

都这么多年了,这一回,还是奉陪到底好了。

他刚刚开口,对面的章敦,却也跟着一并高声发誓。

“祖宗在上,臣韩绛以全家男女为誓言……”

“臣章敦在此立誓,今日只诛首恶,胁从不问。自此刻起,不从逆者有功无罪。杀石得一、宋用臣者,节度使。如违此誓,天人共诛!”

王安石、韩绛、章敦,东西两府领头的重臣,都跟着韩冈纷纷立誓。绝不会秋后算账。只要不再附逆,便是有功无罪。

班直们终于停下了,比起高滔滔和赵颢,王安石、韩冈、韩绛、章敦的威信可是要高得多,得到了他们支持的太后与天子,又岂是高滔滔、赵颢可比。

“你们还愣着什么?还不去杀了他!!杀啊!杀啊!”赵颢几乎要发狂了,“石得一!石得一!快进来诛贼!”

“闭嘴!”韩冈回头怒喝,闪着血光的铁骨朵遥遥指着赵颢,“我们在讨论民冇主,没你说话的份!”

一声呵斥之后,他又回过头来,对着殿上群臣:“何人可为万民之主?篡逆之贼?还是先帝的骨血,奉诏书,已经登基、称制的天子、太后?”

