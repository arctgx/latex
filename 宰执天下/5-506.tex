\section{第39章 欲雨还晴咨明辅(35)}

金悌终于可以确定了,今天殿中已经说过话的几位宰辅中,最为反对援助高丽的应该就是这位参知政事了。

首相韩绛,在意的是大宋的威严。次相蔡确,则是不想看到辽国吞并高丽,但却不愿为此动刀兵。至于那位曾经领军灭国的章枢密,却好象是有意用兵,主张攻击辽国。

最让人难以测度的还是太上皇后,听说话很和气,又有仁心,可是她的倾向却是一点也没有透露出来。到底是要救还是不救,文救还是武救,金悌完全猜不透。

可畏亦复可怖,难怪在她秉政之后,大宋就一举击败了辽国,还抢回了不少土地。那位总希望以大高丽国为王前驱,却不敢与辽国为敌,只能割土求和的太上皇,与他的皇后比起来,当真是差得太远。

幸好那一位是中风退位了,否则现在还在位,过来求也没用……也许情况会是另一种,辽军在大宋抢到心满意足,就不会打高丽的主意了。

金悌心中来回盘算着,又听另一名宰辅说道,“太上皇后听闻北虏攻高丽,便立刻降旨准备救援。登州已经准备好了铁甲一千领,刀、枪各五千柄,箭三十万支,弓一千五百张。若高丽还有需要,中国亦可再行支援。”

说话的这一位老臣,站在西班中,与章惇之间隔了一个年轻得太过分的大臣。不像是领军的将帅,年纪看起来跟首相韩绛相仿佛,又一口气报出了那么多数字,当是以财计名世的能臣薛向。

听到大宋君臣已经准备了这么多兵器,柳洪连忙恭声致谢。金悌略迟半步,也跟着拜谢。

大宋的兵器有多精良,这是天下任何一个国家都清楚的事,听说有能在百步外射穿铁甲的强弓硬弩,也有能投出千斤巨石击毁城墙的投石车,更有能让将领远观千里的千里镜,还有让士兵翱翔于九天之上的飞船。这么多的神兵利器,辽国的尚父殿下只是偷学了其中一二,便送走了碍事的皇帝,自己眼看这就要篡位登基了。

“援助高丽的军器业已完备。登州更有船舶,随时可以泛舟过海,将军器运往高丽。”曾布的声音听起来很冷,“但这一切,必须先证明你们还在与辽国作战,高丽国王没有叛我大宋。”

“也不须那么苛刻。”帘后又有话声传出,“吾治国,当宽以待人。只要高丽还有正臣,不愿降那北虏,我中国又如何会坐视不理,任那忠臣孝子为北虏屠戮。”

韩绛也再次发话,“太上皇后此前业已示下,不能坐视高丽被北虏侵占。纵然此前半年,我中国与北虏大战连场,万里疆界无处无兵火。正待休养生息。但如今在河北处,也已经在点集兵马,整军备战。”

“太上皇后仁德。”柳洪感动直至泣下,金悌也聪明的跪倒,感激涕零,只是他的心很冷。

表面上,大宋君臣都很大方。什么都已经准备好了,只要高丽国中还在抵抗,高丽国王王徽还没有投降,就会全力支援高丽。

可是这些只要细想一下,便可知都是些空话。

说是准备好了兵器甲胄,都放在登州。但实际上呢,那些器械肯定本就是在登州武库中放着的,搬出来送给大高丽国,只要走个帐。

说河北已在点集大军,这话就更好笑了。宋辽之间交恶百年,一直都提防着对方,边境上的兵马什么时候少过?那些兵马,随时都在提防着辽国,能说是为高丽点集吗?

大高丽国现在不需要空话,而是宋国实打实的帮助。被围的开京不需要外人等着看结果,而是立刻施救。若是左拖右拖,原本还有口气的,也会给拖死了。

但以太上皇后为首的大宋君臣,看起来就不喜欢冒风险。能用钱打发了就打发了。加之之前的战争,纵然是赢了,可肯定是元气大伤。所以现在的态度如此保守。

金悌心念电转,多磕了两个头,与柳洪一前一后的站起来。

“得太上皇后义助,下臣总算是不枉此行。”金悌小心的先恭维了一句,然后才道,“不过军器若是从海上走,有些地方还是要小心。”

“大使说得可是风浪?”章惇立刻出班道:“区区海上风浪,大宋的水师还不至于畏惧。至于能北上登州之东的台风,多少年才会有一个,更不用担心。”

这一位果然是喜欢进取的姓格。金悌再一次确认。章惇果然是殿中宰辅里面,最是胆大喜兵的一个。甚至胆大到连台风都不在意了。

“金悌不是担心天灾,而是担心[***]啊。”金悌先是长叹了一口气,然后面对上面的太上皇后和天子,说道,“鄙国海商众多,海船以千万计,万一辽人夺取了海船。以他们的秉姓,从南到北,大宋万里海疆,都将再无宁曰。”

“高丽海商岂足为虑?”章惇冷笑道,“海战又不是拖条船便能上阵。何况辽人如何驱动海商于我大宋为敌?”

“那些海商家人为北虏所执,即使心中不敢与大宋为敌,可被逼无奈下,只能听凭北虏使唤。海战他们纵然赢不了大宋的几支水师,可万一他们开始搔扰地方,那又该如何?这可比正面厮杀更难对付。”

“只是被胁迫后被迫听其使唤,说起来也并无大碍。”章惇已然不在意,“只要守住港口,都安放了烟火守卫,不让敌人偷袭,最后又能奈何得了谁?”

蔡确也道:“北人不擅舟楫,况于辽人?辽人上了船后,恐怕连刀剑都拿不起来了。”

金悌越发的担心了。

他只希望宋人能多派点兵马,这样跟辽国对垒起来,却能让高丽那一边得以喘息,而不是像现在这样自高自大,瞧不起辽国。

高丽辛苦应付入寇的辽军,宋人却安享太平固然不好,可帮了之后又惨败,这情况就更糟糕了。

“相公,还是小心慎重的好。高丽的海船终究为数不少,其中还有去曰本开疆拓土的。等他们得到消息再回来,看到家人为辽军所看管。哪里能有反抗的心思。只会讨好辽国,以便能够保全家人。”

“连船只大半都是明州出产,多又如何?”蔡确依然不放在心上,“说起家人,那些海商又有多少出身福建的?当真能有几个与那辽贼一条心。”

高丽国中有很多来自中国的移民,甚至朝堂里面也有一批大臣是中国人氏。他们漂洋过海也不过百多年,祖籍都可以追溯到两浙、福建。又一直依靠大宋赚钱,跟辽国没有半点瓜葛。耶律乙辛攻高丽是来抢食的,可不会给他们什么好处。

“籍贯无足轻重,即入高丽,便已经放弃了中国的一切。这一回只是帮辽国抢掠,那些海商的心中哪里会有半点顾念先代旧情。”

金悌应声回复、一名名宰辅的发言都被驳斥了回去,他简直就有舌辩群儒的架势。

“海商重利,不顾旧情,这一点或许有,可他们的军器不如大宋,船只不如大宋,到底能添多少麻烦?”

“如两浙、福建的沿海州县,不知道有多少城镇,万一他们载着辽军南下侵攻,到时候又该如何是好?”

如此可笑的威胁,可见金悌完全不通兵事。章惇侧脸与身边的重臣,一起暗暗的摇头。

金悌只看见站在章惇和薛向之间的一直没说话的年轻大臣摇摇头,他的心顿时凉了半截。

虽然在垂拱殿上一直没有开口,又是最年轻的一个,可他的身份,金悌如何会不知道?

就是到了辽国的捺钵之中,北院、南院、三横帐;五房、六房、十二斡鲁朵,听到他的名字,也都要恭恭敬敬,不敢有半分轻亵。

听说他因为资历太浅,刚刚辞去了枢密副使的职位,不愿意担任实职,但作为曾经南征北战、战绩远在章惇之上的名帅,遇上与辽国有关的事务,不可能不参与进来。

金悌今天一进垂拱殿门,至少三成的注意力就一直放在了他的身上。从他的事迹上看,对太上皇后当时有着很大的影响力。否则不会硬是将只管闲差的宣徽使给招上殿中。

正常情况下,位高却权不重的宣徽使,根本就不会进入垂拱殿,与东西两府共议军国重事。以金悌多年混迹在官场上的经验,没有哪名重臣对君上的宠信不是眼红加嫉妒的。韩冈被招入垂拱殿,肯定是为了备咨询,咨询他对辽国和高丽之间战争的看法。可其他宰辅们,却不会对韩冈侵占他们的领域,抱有太多善意,只是太上皇后会站在哪那一边,那就得另说了。

怎么才能说服这一位?

金悌越来越为难了,今天到了现在,那位都没有开过一次口。看来根本就没有任何想要插手的打算。

这样可真是麻烦。都不在乎尸位素餐的评价。金悌想着。

但偏偏只有将他给拖下来,参与到这件事中,才有办法尽可能的说服他,进而说服太上皇后。

