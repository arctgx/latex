\section{第19章 登朝惟愿博轩冕(上)}

写着偃师县三字的石碑在车窗外一晃而过,马车的速度便慢了下来。

“公子,到偃师了。”

不用伴当多话,司马康早在两个时辰前就收拾好了,忧心如焚的等着马车抵达终点站。

车刚停稳,车门才被拉开,他便突地一下跳下了车。

正准备拉开车门的车站工人吓了一跳,还在车厢中的陈易简、孙奇对视一眼,一同摇头苦笑。

还在车上的时候,司马康就一幅火烧火燎的表情。每次列车在沿途车站停下来的时候,他都不耐烦的捶着车厢内壁,就连夜间也是如此。

这样性急的病家他们过去见了不少,非是至亲不会如此,以司马康的情况,绝对算得上是至孝。只是万一不治,也肯定是最不好说话的。

陈易简拉着孙奇,小心翼翼的从车上下来。

司马康站在旁边,眉头紧锁,想催促,又强自忍下。

陈易简和孙奇都看在眼里,被司马康满是血丝的双眼盯着,心中忍不住暗暗叫苦。赶急赶忙的下车,都能不耐烦,恐怕自己耽搁半步,都会被记恨上。

还没等两人站定,司马康便上前来,先行了一礼,然后就说道:“两位大夫……”

“衙内。”陈易简抬起右手,“救人如救火,不用多耽搁了,我们还是边走边说。”

明知司马康会说什么,陈易简也乖觉,直接提起腰囊说要走,堵住了司马康的嘴。

如今的翰林医官有了具体的职阶,在医学院中有住院医师、主治医师和主任医师之分,在太医局中,则相应的有着和安、安济等大夫的级别。

两位顶级的御医,还有匆匆跟在后面负责拎着大件的医学生,便与司马康主仆一起匆匆忙忙的往车站外走去。

偃师的车站远没有东京车站的规模,官民之别也不是那么泾渭分明。

与一帮主要是商人模样的旅客前后出了车站,就见门前停了一排马车,正对门的一辆,与其他一个模子出来的载客大车完全不同,装饰精美,质地精良,外形也是尽善尽美,一看就是富贵人家的私车。

这辆马车前,一名锦服老者和车夫看起来已经等候多时,额头上尽是汗水。可看到司马康一行,老者的眼睛就亮了起来,急忙走来,迎面一礼,问道:“可是司马侍郎家的衙内?”

这位老人,司马康只觉得眼熟。应声点了点头,打量了一下,尽管上下皆是丝罗而制,但装束还是仆佣模样。

老者又行了一礼:“小人文砚,是在文老相公府上听候使唤,今日奉老相公命,特来迎接衙内。”

“啊。”老者自报家门后,司马康就想了起来,“是文管家,”

老者点头应是,转头对上两名医生,“两位是京里来的太医吧,还请一并上车。车里也坐得下,行李可以放在车厢上。”

“可是……”孙奇回头看了一眼跟在后面的学生。

“太医不用担心。小人已经安排好了,贵属可以坐官车随后赶来。”

文砚指了一下后面,在他的马车后,还有一辆马车。虽然与前面的大车没有太大区别,可车厢上的印记是官府,与其他车辆迥异。

陈易简和孙奇暗暗一声赞,面面俱到,不愧是文彦博家的管家。

只是司马康上京请医生,这文彦博家的管家半道上来接人,这里面可就让人不禁要往坏处去想了。

司马康也正是如此,“文管家……老相公,是否……是否寒家……家严……”

他面色陡然间变得惨白,说话也混乱了起来。

“衙内莫急,小的只是奉老相公之命过来迎接衙内,仓促离城,侍郎的病情如何并不知晓。”

陈易简与孙奇交换了一个眼神,各自明会于心。

这位来接人的文府管家,在提到重病的司马光的时候,甚至没有说半句宽慰的话,如不是当真危急,至少也应该给司马康一点安慰。现在这样,等于是让司马康先做好心理准备了

司马康一时间摇摇欲坠,眼看这就要晕倒,文砚连忙上前搀扶,然后让那位体格粗壮的车夫扶着他上车去。

陈易简和孙奇也匆忙跟在后面,上车的时候,孙奇忍不住多问了一句,“赶夜路没关系吗?”

看这位老者的模样,肯定是不会在驿馆里耽搁时间。但要是夜间还在路上奔行,一个坑就能要了全车上下所有人的命。

“太医放心,这辆车整个车的底盘都是将作监出产,之后也是名匠打造,颠簸都很少。偃师通洛阳的官道去年也都重新整修过了,走夜路不用担心。”

孙奇半信半疑,但他还是上了车,他区区一个翰林医官,没能耐为了一点风险,不理会文与司马两家的邀请。

一夜的路上颠簸,司马康终于回到了洛阳城。然后更是没有耽搁,直接就前往司马光在城中的居所。

司马康依然是第一个跳下马车,两位医师同样跟在身后。他们的仆人还远远的落在后面。正如文砚所说,这辆车,的确不怕走夜路,在车夫的控制下,车行得很是平稳,没有出一点差错,颠簸也只比在轨道上行驶的列车稍多一点。

但连个两天的车马劳顿,甚至连睡觉都还在行车,这样的日子,仅仅两天,就让陈易简和孙奇他们两个都大伤元气。

跟着司马康的身后,走进司马光的宅邸,却看见正厅中,一位须发皆白的老人正扶着拐杖,静静的等着。

看文砚上前向那老人行礼、回话,陈易简和孙奇立刻就认出了他的身份。

“论礼数,老夫现在是不该来的。”文彦博拄着拐杖,连腰也不弯,慢条斯理的说话,却让人喘不过气来,“但想想这些年,志同道合的知交,各自七零八落,死的死,退的退,归乡的归乡,就剩这么一两个与老夫一样的死心眼了,却又不能不来。”

司马康呆呆的站在文彦博身前,整个人的都没有了反应。

“先进去吧。”文彦博一声喟叹,示意身后仆人将司马康带进去,见见司马光。

“可惜了司马公休的一片纯孝。”

当两名医官也跟着进去之后,文彦博的身后传来一声感叹。

“与叔,孝心没有什么可惜不可惜的。他做了,我们也看到了。心性是没话说的。至于孝行,虽然没有完满,但也是一等一了。”

“相公说过的是。”

有文彦博在,厅中的其他人,都被忽视掉了。巨大的存在感,让其他人立刻成了视线不到之处的龙套。吕大临并有可以隐藏自己,但司马康三人,仍是没有一个注意到他。

“司马君实……”文彦博看着自己的手,轻轻屈起了一只手指,“富彦国这十几年都对王安石堵了一口气,可临走之前,还是跟王安石的女婿结了亲。韩稚圭家的老六,又被苏颂提议,做了天子家的娇客。我这个老贼还……”

文彦博没事自己骂自己,吕大临听得坐不住,叫道:“老相公……”

老而不死是为贼,文彦博知道,不知有多少人这么骂他,想避也避不了。

文彦博哈哈笑道:“老贼什么的,有人想要还要不了。老夫精神还好,准备活到一百岁。只要能活到百岁,”

真宗时以文学知名的杨亿当年三旬便入翰林学士院中,另外两名同僚年老,所以杨亿每每以某老来戏谑。有一人反击道:“且待将来,以此‘老’与君”。另一人却道,“不要给”。而杨亿果然就没能活到五十。

“以老相公的身子骨,百岁不为难事。”

“谢与叔吉言了。”文彦博笑了笑,又道:“令兄吕微仲当世贤才,若在先帝时,早入朝辅佐天子了。可惜如今……”

吕大临面沉如水,没有搭腔。文彦博也不以为意,“有件事,要拜托你走一趟,”

“是去金陵吗?”吕大临平静的问道。文彦博最近想做什么,并不是什么秘密。

“见王安石作甚?”文彦博眉毛都挑起来了,“去见吕惠卿那厮啊!”

“吕惠卿?!”

“王安石说不通的,吕惠卿却不一样。”文彦博悠悠然说道,“看着章敦久居西府,他的眼睛早该红了。”

……………………

司马光病逝。

这个消息,没用太久便传到了京城。

去洛阳的两位太医并没派上用场。

不过京师、在朝中,司马光早在当年先帝发病、太后初垂帘时便已经死了。

但在不少弹章中,司马光这个名字还是使用着。

朝堂中的有些声音,认为是车站中延误,让司马光没等到太医局的医官。

章敦丢下一份的弹章的副本,冷笑着,从小事开始,向上一直追究到韩冈身上,这是某些人的如意算盘。只要韩冈想要保住整个铁路交通局,他就别想脱身。

可是,章敦没打算掺合进去。

四天中往返洛阳与京师,这个速度在五年前根本无法想象,没有铁路,哪里会有这样的速度?车站中最多多耽搁了一个时辰,而铁路节省的时间,又是多少。

最重要的,是太后不喜欢司马光。

