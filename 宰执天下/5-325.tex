\section{第32章 金城可在汉图中(四)}

城外终于平静了下来,但忻州城中依然紧绷着。

几天来绕着城东冲西突的辽兵突然间消失不见,在城头上枕戈待旦了半月之久的守军们,自然是一头雾水。辽军的消失,在人们眼中,。

一名披挂着全身浸铜铁甲的军校趴在忻州城头的雉堞上,向城外张望着,在他身旁,套着同样式样的甲胄的军校,则抬头看着悬浮在二十丈空中的飞船。

如果是熟悉军中武备的人来看,只从甲胄的式样和纹路上,就看得出两人都是指挥使一级的军官。他们正好分派在城西守卫城门和城墙。

一人身高六尺有半,肩宽体阔,虬髯横生,极是威武雄壮。有他的对比,旁边的一个也有六尺身高的军官,却是一点也不显出众了。而且矮一点的军官肩膀又窄了点,相形之下,便是更显得精瘦。

削瘦的军校从飞船上收回视线,半天过去了,飞船上都没传下消息来,看起来辽人真的是不再附近了。他皱着脸、皱着眉,问道:“要不要出城看一看?”

“直娘贼的,出他娘的城呢!”高个子军校粗大的手指,指着城外近处的村庄,“没看到那边村子里的烟。忻州城外的活人除了辽狗还会有谁?”

“可要这真的是辽狗的陷阱,那他们为什么还能让烟冒起来?”

“谁知道辽狗怎么想的?撞上个蠢货也说不定。”

瘦军校摇了摇头,这就是强辩了。不论在大宋还是在辽国,没哪个将领会蠢到一边埋伏,一边还生火做饭的。就是有蠢货,可下面终究还是有精明的人。

“我看还是报上去吧,去那个村子看一看究竟,好歹也能放心一点。”

“要是报上去,贺知州肯定会说了,‘那你们俩就去查看一下,探明之后速速回来禀报’。”高壮的军校提着嗓子,学着知州说话的声音,惟妙惟肖,接着脸一板:“你去还是我去?!”

“……那还是算了吧。”瘦军校叹了一声,又摇摇头,“反正其他几面城墙都能出人,也没必要先出头。”

两人可都不愿去送死。以忻州城中的军力,若是出城碰上了辽军,那是必死无疑。若是换做一个有人望的知州,为他拼一拼命倒也没什么。可现在的知州?还是为家小守住城才是正经!

且不说出城,就是这些天来,辽军在城外来来往往的劫掠乡中,都把忻州上下吓得够呛。辽人是不擅攻城,可秀荣县已经六十多年没修过城了,被雨水淋坏的墙体就有好几处。

忻州州治所在的秀荣县,虽然正当要道,可惜的是北有代州、南有太原,绝大多数的军事资源都被两个战略要地给吞吃掉了,正当中的秀荣县城——也就是忻州城——只有残羹剩饭。

而想要靠本地的财税整修城防和军事,库中没那份多余的钱粮,让商人富户报效,更是不可能,毕竟是北有代州、南有太原,夹在中间的忻州加固城防、整备军力做什么?有哪个能想得到代州会有破关失城的一天。

“唉,换作是韩学士和陈知州在的时候,哪还需要俺们在这里担惊受怕?”

“若有韩学士在,代州怎么会丢?不说韩学士了,就是有刘太尉在,代州也不会丢啊。”

少说两句吧。”一个懒洋洋的声音从两人身后传来,“贺知州可要来巡城了。”

两人连忙回头,说话的是个二十四五的年轻人,坐在支撑敌楼的一根大柱下面的柱础上,四平八稳的。他没有穿铠甲,头盔也没带,白巾裹头,一身结束整齐的白色军袍,在人人贯甲、身着赤色甲衣的城头上很是显眼。

不过他手上能把脸埋进去的粗瓷汤碗更是显眼。他稀里胡噜的往嘴里倒着掺了醋的汤饼【面条】,说话却一点不耽搁,“知州身边小人可不缺,要是你们说的话给传到知州的耳朵里,赶明儿赌桌上可就没人给俺送钱了。”

听到这个年轻人说话,两个指挥使立刻警觉的收了口,左右望望,附近也没什么人。松了口气稍稍放了心,凑了过来,搓着手叹道:“秦兄弟,其实若是有令尊秦老寨主在,好歹雁门寨不会丢啊。”

“少打岔,把欠俺的赌债还了再说。”秦琬横了陪着笑脸的两人一眼,手一翻,把最后的一点汤水倒进了肚子里。用手抹了一下嘴,放下了面盆般的海碗,恨声说道:“雁门丢了、代州丢了,忻口寨也丢了,都这时候了,说这话有屁用啊!”

秦琬的话不中听,高瘦二军校也不着恼,那是再真切不过的事实。论起弓马,秦琬只是平平,但眼光见识却让身边人人敬服。

秦琬倚仗做忻州都监的老父执为靠山,一来就占了个好位置,本身却没有出众的武艺,一般来说很难会被被同僚所接受,可实际情况正好相反,来忻州后不到半个月,上上下下的一干袍泽对他都服气得很。没点本事,怎么能把秦玑赶着趟儿弄去做铺兵,还是配着金牌的急脚?

“秦兄弟。”两位指挥使贴着秦琬坐了下来,“按你说,这城外的辽狗怎么突然不见了?是陷阱还是真的走了。”

秦琬让亲兵上来将碗收走,舒服的伸长了手脚,“当然是走了。”

“但外面大王庄上的烟呢?”高个子军校立刻诘问道。

“那是骗人的!最多也就三五百人。那点人,只是拿来盯着忻州城,省得我们在后面给攻打赤塘、石岭两关的辽狗坏事。若是出动全军出城,说不定能把他们都吓跑掉。”

“……那辽狗去了哪里?”瘦军校问道。

“南边。”秦琬挪了挪身子,让太阳继续晒在身上,“大概是打石岭关和赤塘关主意。”

“怎么可能?!雁门关是辽狗趁人不备才攻下来的。现在哪还有可能攻下石岭关和赤塘关?!”

“石岭关、赤塘关、百井寨。从忻州往太原去,一路上都在山谷中,关隘军寨倒是不少,但你们说说,这些关隘在太宗皇帝后修过几次?庆历年,石岭旧关后修了一个烽火山城,赤塘关也将坏掉的城墙重新版筑,仅此而已。”

“忻州城不也几十年没修过了。辽狗真要有攻城的本事,直接就来打忻州城了,绕什么路?”

“只有去打石岭和和赤塘关,辽军才有希望攻下忻州城。”秦琬没多解释,反问道,“派了两千人来支援忻州,你们说那位王经略会怎么安排太原的人马?”

“……怎么安排?”

“石岭关支援两千,赤塘关放两千,百井寨再放一千,从忻州到太原,逐寨分兵把守,一直守到太原城。”

“不至于吧?!”瘦军校大惊,。

“不可能的。”高个子军校摇头哂笑。

“王经略真要够聪明就不会只派两千人来忻州了,要么干脆不派,要派就少说也该有两万兵马。”秦琬摇摇头,“现在也只能求他不会这么做了。”

“两万?太原都没这么多兵马!”

“所以说干脆就不派啊。就是两千援兵不得不派,也应该直接将那两千人放在石岭关上。”

石岭关属于忻州,赤塘关属于太原府,两个并列的关隘,相隔只有二十多里,互为犄角之势。守住两座关隘,太原可保无忧。但太原府支援忻州的兵马并没有放在石岭关上,却送到了忻州城来。换作是秦琬,他绝对不会这么做。忻州城中不缺这两千人马,缺的是一个屯有重兵的后方。若能以重兵稳守两关,辽军的动作不会这么肆无忌惮。

“忻州跟代州太近了,又没有关山险阻,本来就难守。既然没能及时守住忻口寨,就不用指望能保住忻州全境了。若是只能守住秀容、定襄两城,多两千兵马跟没多一样。还不如用重兵稳守住石岭关和赤塘关,护住太原。有石岭、赤塘两关在背后,忻州城守上一个月不成问题。有一个月的时间,援兵早到了。”

秦琬正说着,城墙的另一头隐约的有一队人走近,两名小卒张张惶惶的跑了过来,“太守来了,太守来了。”

一高一瘦两个指挥使连忙直起身,秦琬也猛地跳了起来,闪进敌楼,唤过两名亲随,让他们将硌手硌脚的甲胄帮忙给披挂上。

秦琬整理着衣襟,心中盘算着自己的计划。他已经不准备在忻州城待了,若有可能,他今天就准备离城。这些想法他没有对外人说,只告诉了兄弟秦玑。

秦琬的父亲秦怀信还没升到能荫补子孙的地位就病死在夔州路上,历年来积攒的功劳也还差了那么一点,秦琬之前所立下的功劳让他晋身了武官。只不过是杂阶,离品官还差了一级,也就是所谓的不入流。

以三班差使充任指挥使,是委屈不错,但放在秦琬身上,还是多亏了在军中有长辈照顾。否则指挥使位置那么多,凭什么他就能出任最精锐的一个骑兵指挥的指挥使?——军籍簿上兵额五百一十三,北方军中大多数的指挥差不多都在这个数字上下,但实兵数目能达到三百七十人之多的骑兵指挥,可是凤毛麟角!即便当年韩冈在河东合兵为将,各将中的一个步军指挥,平均也不过四百上下的样子,骑兵指挥通常更是只有三百。

只是秦琬如何甘心?秦琬打算将手下的人马拉出去,他对代州地理了如指掌,父亲秦怀信在代州又有声望,他去了代州,以带出去的骑兵为核心,拉起一彪人马不成问题。骑兵困在城里,战马白白消耗粮草不说,等到了围城日久,说不定还会被杀了吃肉,太浪费了。

只要一个都的骑兵就足够了,忻州和代州之间大道小路不知有多少条,想要躲着辽军走,对秦琬来说并不是难事。

现在的问题就只剩下说服知州。

整理好衣甲,秦琬出了敌楼,望着渐走渐近的知州一行,他深呼吸几下,将心神安定了下来。

该怎么说呢?不过是富贵险中求罢了。

