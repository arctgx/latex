\section{第28章 大梁软红骤雨狂(一)}

腊月的京师喧闹无比,宽阔得横过来都能用来跑步的大街,都被堵得水泄不通。比起前次韩冈上京时,更是热闹的一倍都不止。

韩冈从新郑门进来,沿着今年年初时走过的路线,向城南驿行去。还有半月就是年节,置办年货的热浪掀到了最高潮。街市上面车水马龙,一辆辆由十几匹马拉动的太平车,在街巷上往来穿梭。

车上堆满了各色货物,坛坛罐罐里面装的是酒、油、醋和盐菜,而装在大大小小的木箱中则通常是布匹丝绢。除了这些寻常的货车,还有运煤的、运菜的、运盐的车辆。倒是运柴禾的没有看到,韩冈听说京中生火只用石炭,看来真的是这样。

骑在马上,在人群中艰难跋涉,韩冈虽然心急,但也只能耐下性子慢慢的向前挪去。他自出长安后,就一路向东急行。本来预定在洛阳城还要拜访一下程家——虽然程颢此时正在澶州任镇宁军节度判官,但程颢的父亲程珦前日刚刚诣阙,现在应该在家。

韩冈打算感谢一下程颢前日对他的照顾和教导,好好的联络一下跟程家的感情。可是既然从游师雄那里听说要调任延州,一时失了心情,急着往东京城赶,这一计划也便是作罢。

望着道路上的人头涌涌,韩冈觉得东京城中的百万军民是不是今天都上了街来,要不然怎么御街上都挤满了人。

李小六也是对眼前人流给惊到了,前次他跟着韩冈上京,已经震惊于东京城的繁荣和拥挤,而今次比前次还要多上数倍,“挤成这样,这地方怎么能住人?”

“居长安大不易!东京城也一般。只要是京城,便没有一个好住人的。”韩冈微微笑着,他前生后世经历过了的两座首都,没有哪一座能让人轻轻松松住下来的。无论是北,还是东。

韩冈主仆二人穿越了拥挤的御街,经过了满是店铺的街道,向着越来越近的城南驿方向行去。

在他们背后,一个十三四岁、娇俏可爱的小女孩儿,从道边的胭脂铺中跑出来。她掂着脚望着韩冈骑在马上、逐渐远去的背影,可爱的歪着头,眼中先是转着疑惑,但很快就变成了惊喜。

“小娘子!小娘子!”胭脂铺掌柜这时追了出来,喘着气对着小女孩儿叫道:“你还没付帐呢……”

小女孩儿有些迷糊眨了眨大大的眼睛,抬头看看急怒中的掌柜,又低头看看自己手上,还抓着一个螺钿胭脂盒,顿时恍然。她很不高兴的嘟起嘴,把胭脂盒塞回掌柜的手上:“又不是不买,连着方才看过的杭州平云斋的胭脂,都包起来送到安仁坊小周娘子那里去。”

“安仁坊小周娘子?”掌柜确认似的问了一句。‘小周娘子’这四个字如今在东京城中可是很有些名气,不知道是不是小女孩说的那一个。

小女孩儿气哼哼的反问道:“教坊司难道还有第二个小周娘子?”

“快点送,别忘了。”丢下了这句话,小女孩儿向街边招了招手,一个看起来就是沉默寡言的大汉赶了一辆车过来。小女孩儿跳上车,一声鞭花响过,马车转眼就去得远了。

胭脂铺的掌柜看着车马走远,隔壁家卖镜子的老板凑过来,冲着远去的马车扬了扬下巴,“张二哥,方才说的小周娘子,是不是亮出匕首,把高密侯吓跑的那个小周娘子?”

“多半便是。”胭脂铺张掌柜点着头,“李大镜你还没听说啊,高密侯强要梳拢小周娘子,想不到人家小娘子性子烈,把匕首一亮,说要是强来那就一命换一命,一下就把高密侯给吓跑了!好事不出门,坏事传千里。这件事从教坊司的娘子们嘴里传扬开来,据说已经有好几个月没见到高密侯出来了。”

“高密侯就没有想着报复?”胭脂铺旁边绸缎铺的掌柜也凑了过来。

挤过来的绸缎铺掌柜脸上都是一颗颗麻子,仿佛洒满了胡麻的烧饼。他也是在这条街上做买卖的,在家中排行第五,本来外号麻皮老五,但叫着叫着就变成了麻老五。现在外人都以为他姓麻,倒没几个知道他真姓名了。

“他有那个脸吗?教坊司中人按律是不陪夜的。”张掌柜嘲笑着。

李大镜也说道:“强要官妓陪夜,这件事若是闹将出去,高密侯肯定要去大宗正寺走一圈。”

“何况这事都传遍京中了,高密侯也没那个胆子敢下手。”

三人背后传来一道沙哑粗糙的声音。张掌柜等人回头一看,却见是一个跟腌制过的萝卜一样缩了水的瘦汉。是常年在这条街上打晃的泼皮,不过这泼皮跟街上做买卖的生意人井水不犯河水,两边倒是能谈得来。“原来是高猴子你啊。”

高猴子晃过来,也挤到三个八卦党中间:“多少闲得没事干的官人都听说了,不少人都佩服她贞烈,谱了诗词的都有。若是高密侯敢害小周娘子,肯定有人会出头。”

麻老五感叹着:“宗室都看不上眼,这小周娘子眼界还真高。”

“那要看什么宗室了。高密侯下一辈就已经出了五服,王丞相前年定的宗子法,出了五服后就不算宗室了,不赐名,不封官,除了姓赵以外,就是平头百姓了。这样的宗室谁看得上眼?”

“话说回来,别的不论,王相公在宗室上真的做了件好事。俺听俺那在三司衙门做事的小舅子的岳父的姨侄说,熙宁元年,在京三千宗室的给俸,一个月就要七万贯,两千多官人,就只要三万贯,而二十万京营,则是十一万贯。想想吧,不做事干拿俸。”李大镜的口气说不出的羡慕。

“说得是啊。”“说得正是。”“宗室的确拿得实在太多了。”

听了李大镜的这番话,虽然都不是第一次听说这几个数字,但依然让张掌柜、麻老五连连点头,从心底表示赞同。

倒是高猴子不高兴,他一肚子的秘闻还没说呢,现在硬堵着,比便秘还让他难受:“都说到哪儿去了?正说周小娘子的事呢……”

麻老五反问道:“周小娘子怎么了,名声又出去了,高密侯又不敢为难她,不是好得很?”

高猴子嘿嘿冷笑,“她不理高密侯啊。但现在盯上她的那一位宗室,她可没法儿不理了……”

“是哪一家的宗室?”三人齐声追问道。他们都是典型的东京百姓,赌博、喝酒之类的爱好只是寻常,就是宫闱秘辛是他们的最爱。

高猴子脸上泛起了一种神秘的微笑,拿着架子摇头不说。

“开国县公?”李大镜问道。高密侯论爵位,是开国侯一级。比他还要强的宗室,在理当是比开国侯要高上一级两级。

高猴子继续摇头。

麻老五开口追问:“开国郡公?”

高猴子还是摇头,还瞟了麻老五一眼,眼中尽是嘲笑。

“难不成是开国公?”

“比开国公高,那就是郡公了?!”

“郡公都不是?!不会吧……是国公?!!”

张掌柜、麻老五、李大镜三人把十二品封爵一级一级往上报上去,但高猴子自始至终都在摇着他的那颗干巴巴、皮包骨的瘦脑袋,就是不肯开金口。

张掌柜已经张口结舌,要不是他清楚高猴子不爱吹嘘的脾气,早就哼哼哼的嘲笑起来。但现在,他背后因为兴奋或是紧张,都已经被汗水给湿透了。连国公都不算高,下面可就是王爵了。“该不会是个郡王吧?!”他小心翼翼地问着。

“呿,郡王?”高猴子把下巴一抬,不屑用鼻子哼了一声,“郡王算什么?!太庙东廊里的牌位,上三层,下三层,金字描的全是郡王,十四五张供桌都排不下,”他再重重哼了一声,“郡王算什么!”

胭脂铺张掌柜和其他两人,都被高猴子从鼻子里一声接着一声的不屑一顾的态度惊得抖了起来。郡王都不够格,那就只剩下一个答案了。

各自脸上浮起一种想听又不敢听的表情,三人犹豫了半天都不敢发问。但最终还是京城百姓对宫廷八卦的喜好占了上风。李大镜出了头,一条能说会道的舌头,仿佛被米浆浸了三天三夜,硬得发僵发挺,结结巴巴的问道:“是……是……是哪一家的大王?”

瘦高个的泼皮凑近了,压低声音,神神秘秘的比出两根手指,吐出两个字来:

“雍王!”

竟是天子嫡亲二弟——雍王赵颢!

……………………

韩冈并不知道,他已经跟当今天子的弟弟成了情敌。仍是淡淡定定、安安稳稳地抵达了城南驿。

刚刚下马,向驿丞通报了自己身份,王韶就已经脚步匆匆的赶着迎了出来。

如今炙手可热,正得天子宠信的王韶亲自出迎,城南驿的大厅中,顿时响起一阵嗡嗡的议论声。每一个人都想知道,这个高个子的年轻人究竟是何等身份?

只是韩冈刚刚跟王韶相见,一个仆役打扮的中年人就挤到了两人的面前,他一句话就让驿馆中的隐波顿时变成了惊涛骇浪:“小人奉王相公命,请王官人、韩官人过府一叙。”

而韩冈的回话,更是推波助澜的把浪涛化作了海啸:“尘垢未净,不敢拜见大丞相。且稍等片刻,待韩某沐浴更衣。”

说完,韩冈转身进馆,竟把王安石家的仆人晾到了一边。

