\section{第32章 金城可在汉图中(五)}

这一日,刑恕自国子监出来,便向城西走。出了内城,小半个时辰后,来到了一处不大的院落前。

遣了伴当上前叫门,司阍却说主人不在家。刑恕皱着眉,想着该到哪里去找人,身后就听着一声唤:“刑和叔?”

刑恕闻声回头,英俊倜傥的蔡京正从巷口过来。

“元长兄,还以为你在家呢。”刑恕大喜上前,笑意盈盈的行礼:“幸好会来得巧,差一点可就要空跑一回了。”

刑恕长袖善舞,到处都有朋友。蔡京跟刑恕也有些来往,只是交情也谈不上太深厚。今日见刑恕如此殷勤,就知道绝无好事。不过蔡京为人圆滑,不会随便得罪人,请了刑恕进门落座,上茶寒暄之后,方问道:“和叔此来可是有何指教?”

“唉。”刑恕苦笑着一声叹,“元长勿见怪,刑恕的确是无事不登三宝殿,今日是特来求情的。”

蔡京略一思忖,便大致有了底:“和叔是为了国子监的事?”

“元长你都听说了?”刑恕似是有些吃惊的模样,随即又是恍然,叹道:“不愧是御史台。”

“和叔是想找元度吧?”蔡京道,“可惜舍弟已去了平章府上。”

“令七弟可不好说话。”刑恕唉声叹气,“刑恕是打算先请元长做个中人,方好向元度求个情面。想不到,元长你在御史台都听说了。”

‘鬼才相信。’蔡京肚子咕哝一句,很无奈的看着刑恕:“做中人好说,这一事大事化小最好。但怎么就能吵起来了?”

“是啊,怎么就能吵起来呢?”刑恕再一次长叹息,“这下是如了韩枢副的心意了。”

蔡京眨了眨眼,没去附和。

韩冈临走前将程颢的弟子荐去了国子监,让皇后亲下了诏。程门师徒不便拂逆皇后,都答应了下来。只是他们实在与其他新学门人合不来,才几天功夫,从经辩变成了争吵。尤其是今天,闹得最厉害,午中会食时大吵了一家,几名程门弟子舌辩群儒,好生的了得。

但蔡卞这些个在国子监中教授弟子的新党,自是不会将胳膊肘向外拐,更不会乐见。若是给报上去,让程颢的几个弟子被国子监赶出来,一辈子都能给耽搁了。

刑恕早就知道会有这么一天,对那几个同门实在不想搭理,却也不得不来做个样子,“前些日子,嵩阳书院中就有一群糊涂鬼想着叩阙上书,尚幸给大程小程两位先生拦下来了,没翻出大浪,只有一点余波。不过在下也听说了,这件事都给传到了宫里面去了。若是今天的事再闹到皇后那里,事情可就难化解了。”

“我是不看好你的那位老师,别看韩玉昆现在不在京城,等他回来后,照样能翻过来。偏偏还带了些不稳重的弟子来京城,”蔡京摇头,“张明诚病殁,苏昞接掌横渠书院,可是喊出了‘功成便是有德,事济方是有理’;韩玉昆也说道理要‘以事验,以实证’。相比起气学来,二程之学,未免太重口舌而少实证。”

“韩枢副自出机杼,常人所难及,但他意欲让学生从自然中自寻大道,却是强人所难。大道渺茫难寻,还是得贤者传道来得方便……何况韩枢副还不知要到什么时候才能回得来。辽人可是都在南下攻打忻州了,还不知能坚持几日。”

“再这样下去,不等他回来令师就在京城里待不住了……”

俗谚说山中无老虎,猴子称大王,可不论是王安石,还是程颢,在学术界的地位上,都是绝对的虎狼。韩冈这只猛虎离山,留下的空间自然免不了会给抢走,当他回来后,不一定能将丢掉的地盘给抢回来。所以韩冈临走之前,才荐了程门弟子入国子监,就想挑动两家相争。人人都能看得清的事实,却还是让他如愿以偿了。这是韩冈奸猾,还是某些人太愚蠢?

蔡京理解不了所谓对大道宁折不弯的坚持,更没有兴趣去理解,他对刑恕道:“好了,这事也不是你我能掺和,至于今天和叔你的事,等舍弟从平章府上回来,我会跟他说的。和叔你也可以放心,好歹他们是韩玉昆荐入国子监的,不看僧面看佛面啊。”

刑恕起身谢过蔡京,再坐下来时,眉宇间的沉重看着就少了许多。蔡京耳目灵通,人又精明,对刑恕的为人和行事有所了解,暗赞他演技倒是好。

解决了心中事,刑恕笑问道:“元度今日去平章府上,可是为了给太子开蒙的事?”

皇太子前日出阁读书,王安石和程颢都开始了他们的课程,不过教授已有根基的弟子和给童子开蒙完全是两回事。虽然两边都是很做了准备,可王安石和程颢教授的课程,对皇太子来说依然是艰深了一点。这在京城里面,也成了最新的笑话,据说皇后那边很是不高兴。

“蒙学有蒙学的教法。平章虽说是博通六经、深明义理,但教五六岁的孩童读书,可不是那么容易。”

“要说难,哪个不难?难道大程和韩玉昆曾给孩童上过课?只是没转过来而已,过几日就好了。韩玉昆也没做过塾师,还不是亲自编写蒙书,听说关中的蒙学中,有《乡礼》、《三字经》、《算术》、《自然》,大半是他编写的。”

刑恕摇头:“气学蒙书的课程太多了,须知贪多嚼不烂。蒙学是扎稳根基,当从一字一句着手。现在囫囵吞枣的塞进去那么多,既非圣人之言,也非圣人之学。日后想要学以致用,却是难了。”

蔡京笑着:“总比江西只学《邓思贤》要强。”

刑恕一笑点头:“说得也是。”

为什么江西号为难治,就是因为许多进士出身的官员在律法上还不如治下的百姓熟悉。一个是学四书五经开蒙,另一个则学《邓思贤》这样的法律教材识字,当然不是对手。

“亲民官只要教导百姓遵从王法就够了,以《邓思贤》受学,平日里与邻里相争如斗鸡,上堂又一争口舌,乱了尊卑之序,更是败坏了风俗。”蔡京严肃起来,正色说道。

刑恕神色也同样严肃:“正人心,厚风俗,此是治世之道。教人以讼辩之术,人人好胜相争,虽兄弟亦不肯相让。如此,家无宁日,国亦无宁日。”

在士大夫的普遍观点中,平民百姓知法懂法,连四尺童子都能在庭上舌辩,自然是对地方的教化不利,坏了一方风气。

百姓要对王法有足够的敬畏、尊重和信仰,这远比知法懂法更重要!

——说白了,一旦了解了朝廷律法,草民都能利用其来维护自己的利益,这当然让高高在上的官员头疼不已。更不用说许多亲民官本身还没足够的律法常识,若是在庭上被草民驳得张口结舌、面红耳赤,岂不是伤了官人们的体面?

当然,后面的一段只有少部分官员能深刻入骨的认识到这一点。大多数士大夫只是习惯性的将百姓知法和有伤风俗教化等同起来,将这一观点视为理所当然。

蔡京和刑恕并不是普通的儒家士大夫,诸子百家之学他们皆能了然于胸,韩非子的法术势,他们一样熟悉。对于世间的观点,其实是不屑一顾。

‘法者,宪令著于官府,刑罚必于民心。’就像汉高约法三章,杀人者死,伤人及盗抵罪,是公诸于世,以简练公正而得民心。

‘术者,藏之于胸,以偶众端而潜御群臣者也。故法莫如显,而术不欲见。’犯了什么罪,要受什么罚,这是法,让百姓明白这一点就够了。但江西蒙学中教授的《邓思贤》,却是律讼之学,是运用法的术,不能让百姓知晓,而必须当操之于上。

只是儒门子弟怎么能用法家的话来做论据?‘主卖.官爵,臣卖智力’,这样的君臣关系,无论如何儒臣都是不可能接受的。纵然韩非子说得再鞭辟入里,也不能宣之于口。

两人相对着摇头一叹,跳开了这个话题。

“不过韩枢副编写蒙书,其所图甚大,所谋亦是甚远。”刑恕说道,“十年树木、百年树人。百年之后且不知,一二十年后,关中出来的士子,可都是一片声的格物致知了。”

蔡京眯起眼睛:“现在都已经是枢密副使了。别看今天国子监中吵得一塌糊涂,等过些年韩玉昆做了宰相,将新学一股脑的打翻,换成了气学做大堂,到时候,一般儿都是天涯沦落人。”

“谁说不是?”刑恕又是一叹。

蔡京曾经在厚生司中做事,韩冈、苏昞所代表的气学,还是比较合他的口味。毕竟蔡京是靠才干出头的,虽说他现在是言官,但他可不会瞧得起身边的同僚中,那几个只有一张嘴的废物。

功成有德,事济有理。

若韩冈这一回能功成事济,那德和理,便就在都他的手上了。

只是忻口寨难保,忻州亦难保。河北那边打得血流漂杵,更无法支援河东半点。

蔡京疑惑起来,这一回,韩冈真的能功成事济?
