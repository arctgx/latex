\section{第29章 顿尘回首望天阙(十)}

【第二更,求红票,收藏】

风声传了几天,开封知府终究还是换人了。

前日被言官弹劾后,韩维就已经照例避位在家待罪,并上了本子,请求出外。

蔡确是由韩绛荐到韩维门下,他的管干右厢公事,也是韩维特意提拔而得来的。韩维去职虽早有征兆,蔡确向章惇靠拢,也是因为想重新找个靠山。但没想到事情发展得这么快,真正事到临头的时候,蔡确还是有些慌张。

“韩大府不是天子藩邸中人吗,怎么官家这么快同意他辞位了?”

蔡确看了韩冈一眼,现在他有求于自己,不可能是在说风凉话。可天子怎么想的,蔡确虽是心知肚明,也不便放开来说给别人听。而且天子决断之速,也的确是出乎他的意料。

韩维与他的兄长韩绛不同,现并不支持王安石的变法。其实韩家八兄弟,除了老大韩纲因为曾有弃城而逃的重罪,而被夺官之外,其他七子皆为显宦,但他们的政治立场都不尽相同。

现在地位最高的韩绛,稳稳站在王安石一边。他的首相之位,说到底也是王安石让出来的。一个在外领军,一个在内处置政事,配合得很是默契——韩冈也是因为这个默契而被牺牲的。

而韩维虽然跟王安石有着极深的旧交,当初还是他在尚是太子的赵顼身边任记室参军时,不停的推崇王安石,才让赵顼了解到世间还有一个不合流俗、有心振作的良相之才。可是如今韩维已经跟王安石分道扬镳,对新法在开封府的推行多有阻碍。

不过韩维虽然是因为跟王安石不和而去职,但换上来的新知府分明还是个旧党。而且竟是跟韩绛不对付的前河东都转运使刘庠。从这人选中看,赵顼走马换将,并不是站在王安石的一边,以保证新法在开封府的顺利推行,而是在防着韩家兄弟。一个是领军的宰相、一个京城的大尹,为了避嫌,韩维的确该走人。

听到了这个消息,蔡确这顿酒就没有喝好。顶头上司倒得太快,新的靠山还没确认,蔡确的心情一时间也很难振作。

将东京城化为几个厢,让各厢的管干公事处理庶务,就是韩维所倡议。如今韩维去职,新上任的刘庠究竟会不会将这个制度继承下来,谁也说不清。

不过蔡确还是向韩冈再三保证,会把他托付的事情办得妥当。如今的情况下,王安石面前的红人——章惇和韩冈——都挂心的这一事,他也必须重视起来。

要想在王相公面前受到看重,当然得先卖力做事才行。

蔡确很清楚这一点。

……………………

次日。

离着任命刚刚下达不过半日,新任知府刘庠就已经到了开封府中。

卸任的韩绛与刘庠一起对验了公帐,办好交接之后,便推辞了新任知府没有真心的酒宴邀请,毫不犹豫的告辞离开。

韩家的家丁从后门处搬着箱笼,十几辆马车在后街处一字排开,开封府的后花园已经不属于他们。而府中的胥吏,则袖着手在旁边看着热闹,就没一个上前帮把手。

在东京,有‘忤逆开封府,孝顺御史台’的说法。开封知府和御史台的台官,是朝中两个最容易犯错而去职的位置,但他们卸任后从旧时僚属那里得到的待遇,却是天差地远。

御史台的台官,因弹劾不被接受而转任后,多半很快就会回到朝堂上,而且往往会有所晋升,以酬奖他们不避权势、勇于任事的功劳,所以御史台的胥吏对上即将出京的前任台官,照样殷勤无比,比亲儿子还孝顺。

而治理京城的开封知府,无一不是治事之才,所以才能被托付给这个繁琐却重要的工作。但东京城毕竟是多方势力交错存在的地方,府中胥吏也多是各有各的后台。为了表现出自己的才能,知府们实际处理政事时,都不免对胥吏们采取强硬的手段。所以当他们因故罢官,就没一个人会搭理他们。

看到府中胥吏一改往日的殷勤,而冷眼看着韩家的笑话,蔡确也只是叹了声时过境迁,没去打扰韩家人的搬家工作。明日韩维上路东去,他也会去送行。辞别的话语,也无必要在这里找韩维去说。

“听说了没有。今日来的刘大府,可是前些天,王相公指名等他去拜会的那一位。可人家就是脾气大,根本不理王相公。”

“刘大府倒真是硬脾气,说不去就不去。”

“这刘大府看起来跟文相公是一家的,都是看新法不顺眼。”

“那俸禄怎么办?给俺们吏员加俸可也是新法,刘大府不喜新法,那明年会不会加?”

从廊下经过,偏厢里的窃窃私语传入耳中。当蔡确抵达內衙三堂时,继任的刘庠已经坐在了知府的正位上。

开封新知府上任,照例衙中从官都要行庭参之礼。也就是如蔡确这样的开封府官员,都要趋步进官厅,向新知府跪拜。如果是文官,知府就站着接受;若是武职,则要自报官衔姓名名,知府坐着受礼。

蔡确当然不想向刘庠跪拜,因为昨天的一件事,他心中有了些想法。刘庠与他的举主不对付,而方才无意间听到的一番话,也证明了刘庠根本没有去拜会王安石。把握到了这两条,蔡确要做的就很简单了。

庭参之仪,按步骤依次序进行中。刘庠站在公厅中的座位前,而衙中官吏则按着官位高下,一个个小碎快步的进厅,向其跪倒拜礼。

先是通判,继而是两位开封、祥符两县的知县。接下去,是录事、判官、推官。等他们都结束了,蔡确便与诸厢管干公事,一起上前。

顺着赞礼官的口令,一众官员向新任开封知府拜倒。可是就在刘庠的面前,蔡确却硬挺着身子一动不动。在人群中独自站着的蔡确,加上他身侧向刘庠跪拜下去的开封府属官,合在一起看,就像一个山形的笔架。

身边人扯着蔡确衣角,压低声音急道:“还不下来庭参?”

“庭参?”蔡确像是听到一句很荒谬的言论,脸上有着难以描画的嘲讽般的笑容,反过来大声诘问道:“何以要庭参?!”

刘庠眼眉一紧,他在官场中混迹多年,心里很清楚,这位分明就是来挑事的。他慢慢的开口,像是每一个字都是深思熟虑过一般:“百年来有此故事。”

“唐时藩镇僚属皆为节度征辟,方有庭参之仪。如今同为朝臣,辇毂下比肩事主,此故事安可续用?!”蔡确的声音提得更高,丝毫没有参拜的打算。

刘庠沉下了脸。蔡确所为有悖常例,他见韩维时难道没有庭参吗?!

“你下去!”刘庠甩手一拂袍袖。蔡确此举,犯了他府尹之威,刘庠是必须要在天子面前讨个说法的。

蔡确仿佛打了胜仗一般出了开封府衙,这种行事手法还是韩冈提醒了他。事情闹得越大,对他越是有利。他蔡持正旗帜鲜明的跟刘庠划清了界限,无论是韩绛还是王安石那边,都能卖得上好。而且说得是又是正理,摆到天子面前,也不能说他蔡确错了,最多一个不敬上官的罪名而已。

不过经他这么一闹,开封府肯定是待下不下去了,必然要离职,就看王安石和韩绛会酬谢他什么职位。还有韩冈托付给他的事情,申状都已经放在了自己的案头上,但现在也不可能回去再办了。

虽然感觉有些对不起韩玉昆,但在蔡确心中,还是示好韩绛和王安石更为重要——能直接凑上去,何必间接的绕着走门路。

……………………

“蔡持正好大的脾气。”

走在开封府衙的幽深廊道间,说着这句话的官人不过三十多岁。但他留着一把大胡子,眉目俊秀,举手投足间透着潇洒不羁。如果没有留须,年纪应当比他现在要年轻许多。而沿路的小吏看到他,都立刻避道,恭恭敬敬的向他行礼。

这位官人在州衙中的地位很超然,实际上,也很少到官厅中来帮忙。他虽然常常受人邀约,出外喝酒的时候居多,但仅余的一点时间,他总能把公务做得妥妥贴贴。

今天蔡确跟刘庠闹翻了,蔡确手上的公事都要移交给他人。现在属于蔡确的公务,不知为何都压到了这位官人的案头上。尽管免不了有些抱怨,但仍然很卖力的开始处理起来。

“这是?”他处置了几桩急务,随后从公文堆中随手拿过一张文书,展开了一看,竟然是周南脱离乐籍的申状。他从上到下全看了一边,摇了摇头:“周南既然是花魁,这如何能走?一花飞去,恐百花颜色皆尽矣。”

提起笔,他龙飞凤舞的写下了判词:“慕周南之化,此意虽可嘉;空冀北之群,所请宜不允。”

半日后,韩冈拿着判状,拍案大骂:“好你个苏子瞻,不许就不许,何苦以文字戏人!?”

