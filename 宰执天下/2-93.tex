\section{第21章 山外望山待时至(二)}

【求红票,收藏。】

同中书门下平章事,同中书门下三品,这些官称名目,都是代表着宰相的职位。王安石自己都还不是宰相,却毫不犹豫地把位置推了韩绛。

赵顼吃了一惊,回头看着王安石,却见他神色恬淡,当是言出由衷。赵顼犹豫了一阵,最后摇头:“……且再等等。等过两个月后再提此事不迟。”

王安石为人无私,毫不犹豫的推荐韩绛为相,但赵顼却不能不在意王安石的身份。赵顼所依仗这位重臣,在去年富弼离职后就可以升任宰相。但他却把机会让给了陈升之。

不爱名位是德行高致,值得颂扬。但王安石如今是以参政之位来主持国政,名不正言不顺,赵顼也希望能尽早把王安石提到宰相班列之中。

首相曾公亮已经因为李复圭的诗文以及御史们的弹劾,上书请辞宰相之位,申请出外。同时照惯例杜门不出,不再上朝,以示待罪之意。

赵顼并没有留下他的意思,只是曾公亮有定策辅主之功,赵顼为了不让人说他刻薄,还是照规矩慰留了两次,等中使从宫中到曾府,再来回个五六趟后,就可以批准其出外了。而曾公亮一走,王安石和韩绛便可晋升宰相,加上陈升之,昭文、史馆、集贤三相正好一个不缺。

有着这样的想法,在曾公亮正式离职之前,赵顼暂时就并不打算把宰相之位给韩绛。

而天子要把事情拖上一拖,王安石也无意反对。宰相为众臣之首,礼绝百僚,宣麻拜相绝不是张嘴就来这么简单,天子需要权衡的地方很多。只要能赶在罗兀城开始修造前决定下来,不耽误事,王安石不会催促。

赵顼再看了一眼无定河流域的沙盘,起步踱到了秦州的沙盘前。沙盘上有着一面面小旗和一个个木雕的兵人。

这是他最近最喜欢的一副沙盘,这段时间以来。他命王中正和李宪,把他们听到托硕、古渭两战的细节,在这副沙盘摆了又摆,重新推演了许多次。每次都让年轻的天子看得听得热血沸腾,恨不得指挥战事的是自己。

赵顼低头看了沙盘一阵,道:“郭逵到了秦州后,脾气好像改了不少。王韶和高遵裕举荐德顺军都监苗授为秦州西路都巡检,他也没反对……”

地方中层将领的调动,并不经过中书门下,走得是枢密院,王安石无从得知。听到赵顼的话,他有些惊讶:“军都监去做都巡检,枢府那边同意了?!”

赵顼摇了摇头。文彦博现在最恨的就是让他差点中了风的王韶。前些日子还为了是否设立缘边安抚司一事,在朝会上对出头提议的章惇冷嘲热讽,被殿中侍御史弹劾他君前失仪,最后也就罚了半个月的俸了事。

但凡有关秦州王韶的公案,文彦博鸡蛋里面都要挑出骨头,何况今次举荐又不合常理。事情直接在枢密院就被否决了,赵顼甚至能想象到文彦博兴奋的拿起笔,在奏折上写下几行极尽讽刺之能事的批语的场面——那份被否决的奏折现在就在崇政殿的御案上,写在上面的批语的确称得上尖酸刻薄。

不过,缘边安抚司的征辟虽然枢府给否决了,不代表赵顼不能把事情转圜回来。罗兀筑城在即,横山战事将开,韩绛这个陕西宣抚都是坐镇在延州,接下来的一年,陕西的资源全都得以鄜延前线为最优先的考量。

在无法给王韶更多的物质支援的情况下,赵顼能做的,就是满足他们在人事上的要求。但天子直接出面否决枢密院的批文并不合适,需要政事堂为此先提上一句。

王安石心领神会,但他并不了解苗授,不能随随便便就答应下来,“不知苗授才具如何?”

“枢密院称以都监为巡检,非是优待功臣之道。”

枢密院虽是反对,但用词却进一步证明了苗授的才能和功绩。王安石相信王韶和枢密院不会同时看错人,“即是如此,臣明日便提一下此事。正好秦凤兵马副总管一职依然空悬未定,两件事可以一起说。”

“秦凤兵马副总管的人选,枢密院已经有了推荐。”

“是谁?”王安石问道。

赵顼低头看着沙盘,没有说话。

王安石脑中灵光一闪,顿时惊怒:“燕达?!他只是鄜延都监,这资序差得未免太远了!”

武臣任职统军,跟文官一样,都讲究着资序。正常的依照资序升迁,是‘由正将而边守、州钤,由边守、州钤而边帅、路钤,由边帅、路钤而都钤、总管’。一路都监相当于边守一级,与一路副总管差了两个阶级。依照正常的升迁磨勘次序,就算朝中有人,没有十几年功夫,也根本爬不上去,若是无人,更是一辈子也别想指望。

秦凤都监张守约好不容易才升为钤辖,而燕达的资历远低于张守约,枢密院竟然要让他做副总管?!他的前任窦舜卿可是正任的观察使,而燕达连个遥郡都没有。

王安石觉得文彦博好像是疯了!他要怎么做才能让燕达把两堵高墙给跳过去?!

“权发遣。”赵顼轻轻吐出三个字来。

大宋立国之后,官僚社会已持续了百年,体系内官员的迁转调动都有规则可循。相应的资序对应着相应的差遣,一般来说不会有所差池,不过高职低就和低职高就却也常见,但职和位的差距通常不会超过一级。而要区分这三种情况,只要看一下加在差遣前的前缀就可以明了。

高职低就为‘判’,平级的称为‘知’,而以低超一阶任职则冠以‘权’字。平级的‘知’,事情而定,可以不加。如韩冈是管勾缘边安抚司机宜等事,而王厚跟他同职,但资序却低了一级,所以是权管勾。再比如现在在亳州任职的富弼,他是以前宰相的身份做亳州知州,所以他的差遣是判亳州,而不是知亳州。

资序差上两级情况也是有的,为了让年轻资浅的官员能早点担任要职,便会给他们一个‘权发遣’的名头。燕达的资序并不足以让他担任秦凤兵马副总管这个职位,但变成权发遣秦州兵马副总管,却是勉强能够说得过去。

不过以文彦博为首的反变法一派,用来攻击王安石的几条罪状中,都少不了任用新进的这一条。因为属于变法派的官员,往往资历甚浅,就是吕惠卿、曾布、章惇等人,入官也不过十几年。为了把他们安排在主持变法的各个要职上,都不得不在职官前面加上‘权发遣’的字样。

守旧因循的反变法派,一直都很反感年轻官员的超迁。一步登天的情况,让排了多少年队、等着按次序依次升官的老迈庸官愤恨不已。

而现在文彦博推荐燕达为秦凤副总管,日后他再想用‘任用新进’四个字来攻击王安石,可是要被人一巴掌打回来的。王安石相信以文彦博的老谋深算,肯定不会看不到这一点。而他还这么做,可见这项任命,必然会给文彦博带来足够的利益。由此推断,可以被安排下来的燕达就很可疑了。

只是王安石看赵顼样子,却是很看好燕达:“燕达的才具是足够了,功劳也不缺。加一个权发遣的名头,秦凤副总管一职他也能充任了。”

燕达在绥德城,有着一日连破八堡,斩首数百度战绩。而在世人眼中,党项比起吐蕃来,还是要强上一筹。从斩首数上来看,王韶的托硕、古渭两战,要高于燕达在绥德城的战果。但朝堂上下,却是把燕达的功劳看得比王韶的两次战功都要重……而且是重的多。

而且燕达今次入觐诣阙,在奏对上,给赵顼留下极好的印象。韩绛要清理郭逵留下来的影响,他排挤燕达的心意,赵顼也看出来了。既然如此,把这位才能卓异的将领安排到合适的位置上,以期能够立下更大的功劳,赵顼的想法却在情理之中。

“就怕他功利心重,日后变得跟李师中、窦舜卿一样,只知道争权夺利。却不知道辛苦做事。”

“日后的事,日后再说。”赵顼不想再多谈此事,问道:“前日王韶上书,备言蕃人虔信佛法,如今结吴叱腊伏诛,剩下的蕃人和尚连金刚经都背不下来。正是安排大宋的僧人去蕃部传道授业、招抚蕃部的良机。”

“人选已经定了。就在昨夜才答应。“王安石并不隐瞒赵顼,“当初蕃僧结吴叱腊便靠着他的身份,游走各个家蕃部之中,甚至撺掇了董裕起兵攻打古渭。如今结吴叱腊已经成为了王舜臣的刀下冤魂,僧录司要透过拣选西使吐蕃的高僧大德,来说服各家同属于边地的蕃部。不过还有一僧人主动上门自荐,此人才学过人,精通医术,又浸淫佛法多年,舌辨无人能及。”

赵顼听了便欣喜的问道:“此人是何许人?”

“是京中有名的高僧——智缘。”

