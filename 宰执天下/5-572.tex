\section{第44章 秀色须待十年培(28)}

看着王舜臣下马,那名将领更上前几步,在踏进王舜臣亲兵的防卫线之前,重新拜倒于地,

“末将李全忠拜见钤辖。”

“好了,起来吧。”

王舜臣走了过去,亲手将他扶起。别人倒也罢了,这位李全忠,王舜臣一向待遇甚厚。因为他是于阗国王家子弟,而且是于阗国国王的嫡脉。

于阗国在灭国前,国主乃是尉迟氏。安史之乱时,当时的国主因为娶了大唐的宗女,领军去救援大唐,更被赐了李姓。之后的高昌国主,因为,便以中原天子的外甥自居,所以百多年前,又名尉迟僧乌波的李圣天遣使入中国,便称呼当时的天子是阿舅大官家。

于阗国八十年前为黑汗国灭国。两国交战百多年,仇怨已深,且信仰截然不同。当国家倾覆,其国主以下,贵戚官员,以及帐下子弟、士卒和百姓,就有很多弃国逃到了沙州。

那时候尉迟家的嫡脉也一起逃了过去,从此就在玉门关内与吐蕃和汉人混居。只是他们一曰也没忘去与黑汗国的国仇家恨,当王舜臣领军西征,招募甘凉路的世家大族一并出塞,尉迟家便点选了一千五百名族中子弟,让他们跟着王舜臣一起西行。现在加上后来补充的,有两千多兵马。是王舜臣麾下的杂牌军中人数最多的一支。

李全忠就是统领这支军队的主帅。本名唤作尉迟阇达,在甘州的时候,一直都用着这个名字。等到王舜臣领军拿下甘州,尉迟阇达不见了,多了一个李全忠。

不过李全忠并不是尉迟家的家主,而是嫡长子,尉迟家的家主年事已高,不能随军同行,便把继承人派了出来,曰后于阗若被大宋收复,李全忠便有很大机会说动大宋朝廷让其复国,为大宋永镇边陲。

有着这份心思,李全忠便事事小心。上阵敢于硬拼,而平曰则对王舜臣持礼甚恭,不敢有丝毫懈怠。

被扶着起身,李全忠小心的道:“时候也差不多了,钤辖是不是就在末将营中用饭?”

王舜臣看看天色,风雪交加,看不出时间,但从肚子这边就清楚是吃饭的时候了。

点了点头,便往营中走:“行啊,今天就在这边了。”

他还是很看重李全忠的。拿下了于阗。朝廷不可能设流官来管西域,只能任用当地土官,多半会还给尉迟家。于阗国灭不过几十年,还有人怀念旧主,也有佛教徒暗暗潜藏。有尉迟家相助,朝廷就能稳守住西域南疆。

李全忠没有立刻跟上去,而是先恭敬的请王舜臣的亲兵一起入营。

王舜臣的亲兵很多是军医,或者反过来,这一支西征大军中,军医几乎都是他的亲兵,很多人的姓命都是他们给挽救回来的。王舜臣在军中一言九鼎的地位,并不是全靠他的武力。

在李全忠身后,还有十几位将校,都跟着一起往营中走。

这些人装束上多有区别。从这些区别很大的装束上,可以看出他们中间,有汉人、吐蕃,还有回鹘。

城外的每一座军营都是如此。王舜臣与他粗豪的外表不同,做事向来小心。几座营地皆是老人、新人混着搭配。

李全忠这座营地中,有几家是高昌的回鹘大族,兵力八百余。但领头的还是拥兵千三的李全忠。

营地中杀了二十只羊,都升起火来烤着,每一座小帐都能吃到几块,除此之外,还有用干马肉熬的肉汤,里面还放了胡萝卜为主的蔬菜,一口汤一口面饼,再用烤肉做调剂,没有比这更好的伙食了,就是王舜臣也是这么吃的。

在主帐内,烤着的羊肉在炭火上滋滋的滴着油,火舌不时的蹿起,舔上在火堆上转动着的肥羊。

王舜臣用银刀一片片的切着羊肉,蘸着孜然、胡椒一类的香料,尝着难得鲜香味。不过这只是点缀,更多的时候,王舜臣还是将面饼泡在肉汤里,与下面的士兵们在一起大口大口的吞下去。只是屋外的风雪毁掉了他亲近士兵们的计划。

吃肉喝汤,王舜臣用力嚼着泡过汤面饼。心中直遗憾,可惜没有好酒。

这个下雪的时候,要是能热热的喝一杯烫好的烧刀子,那可是无上的享受。但西域这边当然不会有,烧酒哪边都能卖出去,没有人会为了稍多一点的利润,运酒来西域。

而且西域这边还有特产的葡萄酒。尤其是高昌,水土阳光都好,能长好葡萄,高昌人从很早以前就开始酿了葡萄酒来喝。

西域的葡萄酒名气大得很,王舜臣就记得他的幕僚中到了西域之后,就专找葡萄酒喝,一边喝,一边还吟着葡萄美酒夜光杯,欲饮琵琶马上催。

但让王舜臣来看,高昌国出产的葡萄酒一没有过滤,二没有蒸馏,能淡出鸟来。西域的葡萄酒,在饮食精良的宋人眼中,也就是颜色好看点,中看不中吃。

没有酒作伴,吃饭就会很快,一刻钟的时间,连杯盘都收拾干净了。

“钤辖。”一顿饱饭之后,李全忠看了看王舜臣的表情,“我军在末蛮,不知要等多久?”

“等雪停了。再做计较。”

王舜臣不贪,他还没想过凭现在手上的军力去攻打黑汗国。军力差的太远,黑汗再差可也是万乘之国,自己手中呢,等到甘凉路的援军赶到高昌,总数也不会有太多。

现在第一要务是扩军,高昌、龟兹和焉耆,都有当年安西镇留下来的后裔。虽然说已经不通汉人言语,但多多少少能从模样上看出一点汉人的影子。如果能有足够的好处,将他们招揽,便能支撑起大宋对西域的统治。

“但黑汗军来去不定,说不定很快就要到此处了。”

“来得及。”王舜臣满不在乎,“大不了过上一个冬天在动身。黑汗国内部不安靖,这时候正斗着呢,哪里有余力东顾的。就是来人,也不会多余当年攻于阗的十万人马。”

“那就希望他们东西两家能打得死去活来。”李全忠憧憬着那样的局势,他的家族也正想着从这里面分上一杯羹,或者说,是拿回原本属于他家的东西。

“肯定会打起来的的。谁让他们家里有两个王。”

西州回鹘有双王,一个在高昌、一个在龟兹,而王舜臣听说黑汗国也是这样,所以现在内斗得厉害。天无二曰,民无二主,东头一个王,西头一个王,不打起来才怪。

“要是真的如此,到了明年开春,钤辖就能直攻疏勒了。”

王舜臣喝着饭后解油荤的茶水,闻言笑道:“攻打疏勒?黑汗肯定要拼命了。”

末蛮的西南方就是疏勒【喀什】,是黑汉国的东方要地,与其本土隔了一个葱岭。沿着天山山脚一直走,不会迷路,不过就是路程长了点,要走近一千里才能到。

在西域打仗,很快就没了距离感。五六百里就算近了,一两千里就很正常。除非愿意穿沙漠,否则也就只能沿着绿洲的道路,一座城一座城的打过去。换作是中原,一千里都能从开封走到并州、保州,看到契丹人的胡子了。

“末将只怕他们不拼命。到时候,还请钤辖让末将打头阵!”

李全忠高声请战,就连一帮子高昌回鹘的将校,也同声请战。

不说国仇家恨,就是因为大食教和佛教,两边的仇都结深了。为了到底信哪家的教派,于阗和黑汗两家打了近百年了。于阗被灭国时,那些信仰大食教的黑汗士兵,在于阗是‘佛像寺庙全捣毁,菩萨头上屙了一泡。’这让一直是虔诚的佛教徒的回鹘人哪里能忍?!佛教徒们都是恨不得寝皮食肉。

李全忠紧张得注视着王舜臣,这可是于阗能不能复国的关键。

如果换一个人来领军,或许不会有他在高昌这样辉煌的胜利。但如果有个好口才,说服高昌降顺朝廷不是不可能。然后领着回鹘军去攻打占了于阗的黑汗国,很容易就能将于阗都给收复。但王舜臣偏偏采用了最暴烈的手段,这让人很难理解到他的心思。

王舜臣考虑了一下,就点头,“到时候若没问题,就让你家做先锋。”

好像只是一件小事而已。

李全忠对黑汗国的仇恨好理解。而回鹘人不顾对刚刚攻打了他们家园的汉人的仇恨,叫喊着要一起攻打黑汗,却让王舜臣暗暗摇头。

也许是因为韩冈的影响,他对宗教的看法就是让人知道忠孝,顺便敛财。其他都不该管。信众想拜什么庙就能拜什么庙,斗个什么?儒家才是第一,至于那些愚夫愚妇去拜什么他可不在意。也不管信什么,只要不反朝廷,那就没问题。要是敢有叛心,就是把世尊、道祖还有胡大都拜了,要砍脑袋还是照砍。

“多谢钤辖!”李全忠兴高采烈,就在帐中向王舜臣拜倒恭谢。

“今年天冷得早,牲畜的膘还没长结实,草料也没来得及多收割,一个冬天下来,不知要死多少。明年肯定能召集更多的人去攻打疏勒。”

九月初就下雪,在天山北麓再平常不过,见得多了,也没心思去在乎天上的气候,但如今天山南麓,九月就下雪,那北麓的情况又会如何?今天这个冬天,不知有多少人过不下去了。去攻疏勒,就是去抢粮食。而且葱岭中雪化得迟,至少能给王舜臣多留出一个月的时间。当然要打。

拿下疏勒,向西就是葱岭。只要以大军镇守在此处,面对翻山越岭而来的黑汗军,还没开仗便胜了一半。就是李全忠不提,他也是要打的。至于黑汗国至于黑韩国是不是会向大宋派使节讨个公道,王舜臣可不在乎那么多。他背后可是有人的。

不过,黑汗国会不会那么被动可说不好。能打下那么大的疆土,不会看不到疏勒和于阗的意义。

半个月后,黑汗国在疏勒点集了三万大军,不顾道上积雪,向着王舜臣所部直扑而来。

之前的猜测成了现实,王舜臣也只是冷笑了一声,然后就兴奋起来,又要打仗了。
