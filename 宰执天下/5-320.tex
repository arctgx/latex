\section{第31章 停云静听曲中意(27)}

军情紧急,韩冈和章敦、薛向连午饭都匆匆吃了两口点心垫饥,一边派人去联络政事堂和三司,一边翻找本院中的档案籍簿,将河东现有的官吏、将校、兵马、军械、钱粮一条条都列了出来。

韩冈打算在抵任前,就对河东的现状有个大体的认识。

原本他就对河东有着很深的了解,今天再对比一下现在的数据和人事,大体的情况基本上就了然于心。到河东后,手中拥有什么样的牌,差不多也就有数了。

不过这仅仅是河东本身的情况,必要的支援也是不可缺少。

韩冈和章敦、薛向一番讨论后,很快就选定了京营出援河东的兵马,总共五千人。但仓促间这五千人不可能立刻北上,韩冈也没时间等他们收拾好行装。出兵剩下的手续和事务,他都交托给了章敦和薛向。

章敦都笑说像是当年南征交趾时的翻版,韩冈领兵赶着去救火,而他在后面整顿战备。只是说了之后又觉得不吉利,倒又向韩冈赔了罪——当年韩冈领军南下,紧赶慢赶也没有将邕州给救下来。

至于必不可少的军资,河东府库里不知堆了多少,绝大部分不必从京城这边赶着运。尤其是铠甲,早两年禁军全数换装完毕后,军器监依然是一年八九万套的生产,全都堆在了河北、河东和陕西的各州军库中。斩马刀也是差不多的情况。纵然大战就在眼前,韩冈也照样觉得这是在浪费铁料,完全可以减半生产,节省下来的钢铁还能多打造点农具。

倒是上弦器和上弦机,现在还很稀少,不管有用没用,韩冈则张口分了一半走。这段时间,军器监都在日夜赶工,河北那边形势还算好,暂时也用不到许多。

等到午后,事情商议得差不多了,章敦、薛向便忙着他们手边的一摊子事。韩冈则还在翻看着三司那边抄送来的河东钱粮账籍,手上端着热茶,眼睛却仍不离开纸面。一名小吏这时进来,说是宣诏的天使到了。

韩冈起身迎出门,就在枢密院的院中,迎接宣诏使臣的到来。只是当他看见前来宣诏的是王中正,这让韩冈多多少少吃了一惊。

章敦和薛向在内听到了消息,也都各自惊讶。宣诏的内侍依官职高下也分三六九等,皇后选了王中正来,可谓是给足了韩冈脸面。

就在庭中,韩冈行礼如仪,聆听王中正宣读诏命。直接在枢密院中拜领晋身西府任命,他这样的情况,可以说是极少见。

王中正展开墨迹淋漓的敇命诏书,抑扬顿挫的念着。一篇四六骈俪的文字,任命韩冈为枢密副使,但另一个差遣,却并非说好的河东宣抚,而是河东路制置使。

河东路制置使。依诏书中所言,得授节钺,可便宜行事,且知州以下不从号令者悉可斩之后报,知州以上官,亦可先行夺职。

从职权上来看,这一个制置使基本上可以说是宣抚使了。但制置使的名号,对官制并不算了解的韩冈只有模模糊糊的印象。似乎是唐朝时曾经设立过的职位,在国朝之初也曾授人。所谓制置,规划、处理而已,并不一定属于军职,统掌军事。


国朝初年,朝廷曾经在陕西设立过制置使一职,执掌青白池盐事——这就是韩冈能有印象的缘故——不过从诏书上的词句中,倒也说明了在唐时算是军职的范畴。可终究还是要比宣抚使低一等。

王中正比预计中来得迟了,韩冈估计是诏书写得慢的缘故。玉堂那边多半并不熟悉这个差遣的管辖范围,为了草诏自是要费了番功夫从故纸堆中翻找了一通,也亏天子能想得到——一直都有传言,当今的天子有心改易官制,恢复唐时制度,从制置使这件事上,也可见一斑。不过改易官制的事现如今是不可能了,几百年演变来的官僚制度,没大精力大决心是改不来的——只是天子宁可翻出一个没什么名气、百多年没人用的差遣,也不肯将宣抚使授予韩冈,其中隐含的猜忌和冷淡,任何人都能看得很清楚。

章敦和薛向都站在厅内观礼,听到王中正读着诏书,脸色都有些难看,更是担心望向韩冈,生怕他跟王安石不相上下的牛脾气上来,场面可就难看了。

不过等王中正念完那份并不长的诏书,韩冈却再拜而言,“枢府之职,臣才具浅薄,恐难符其任。惟殿下重托厚望,故臣不敢推辞。”

方才宣诏时,王中正念到河东路制置使这个官名时,声音就磕绊了一下,知道事情不好,以韩冈脾气上来,多半会不肯接旨。现在韩冈干脆了当的接了旨,倒是让他为之一愣。

赵顼的心思,韩冈自然明白,但他现在没心情跟半死的皇帝打擂台。赵顼并不知道河东局势有多么危险,可韩冈知道,现在不是赌气的时候。不过他说的话还是明摆着有怨气。

正常敇诏,前来颁诏的内侍都少不了能得一份喜钱。但韩冈一时间忘了,王中正更不会提,上前拉着韩冈叹道:“枢副之材,世所共知。有枢副坐镇河东,驱敌逐寇,京城上下才方得安寝。中正只恨不能在枢副帐下竞鞭,与辽寇一决高下。”

韩冈摇头笑道:“如今的皇城中,又如何少得了观察?”

两人你一句我一句,气氛倒是缓和了不少。谦逊了两句,韩冈就在正厅中接受了枢密院中属吏们的参拜,正式就任枢密副使,成为了宰执班中的一员。

整整十一年的官场生涯,终于一级级的攀到了宰执的位置上。自此有了一张清凉伞,为士人所艳羡,可韩冈却发现,自己并没有想象中的兴奋,就如顺水行船,自然而然,顺理成章。心中莫名的平静。

而且他这个成员马上就又要离开,连自己的桌案都还没摆下来。偌大的西府,如今分给枢密使们的院落大半都是空的。西府不过五人而已,这一回倒有大半在外面跑。吕公著离职时是章敦、薛向撑场面,到了现在还是两人在院中撑场面。朝廷中,西府如今被东府压着,人数看着多,但实际在朝中的只为东府的一半。

颁诏之后,王中正需要回宫复命,没有多耽搁,对韩冈道:“中正这就回宫向圣人复命。不知枢副还有什么话要中正转告的。”

“边地不稳,国家不宁。韩冈今天就走。军情紧急,耽搁多一日,太原就多一分危险。”韩冈对瞪圆了眼睛的王中正道:“还请观察代为奏禀皇后殿下,臣韩冈今日请陛辞。”

王中正为韩冈的决定惊叹了一阵,告辞离开。

韩冈转头过来,对章敦道:“对了,方才忘了说,还要托子厚兄跟郭逵说一声。不要放萧禧回去,等之后朝廷的处断。”

“扣住萧禧?”章敦在提到那个奸猾的北院林牙名字时,加重了语气。

“正是。”

“我会跟蔡持正说一声的,让他传话给郭逵,”章敦点头笑了一笑,“把折干放回去。”

韩冈也笑了,章敦的头脑不是一般的敏锐。打归打,谈归谈,两者并行不悖。这一回宋辽两国间的事,最后还是要靠谈判来解决,所以需要一个居中的传话者。折干是个好选择,自然,萧禧这根搅屎棍是不能让他回去了。

韩冈在枢密院中要做的事很快就结束了,向章敦、薛向告辞后,就直接出了大门,准备往宫中一行,陛辞离京。

只是刚出门,就见到了伴当韩信领着七八人,带着十几匹马迤逦而来。

骑手们各个雄壮,他们都是韩冈准备带去河东的亲随。队伍中无人骑乘的四匹马,除了韩冈本人的坐骑外,其余三匹背上都是大包小包。韩冈一问,这是他方才传话回去,让家里帮忙收拾的行装。

“怎么这么多?简单点就行了,还当真能冻着我?”

“夫人和娘子们说了,河东天寒,不比京城,多带一点家里也能放心。”

“府上送了这么多寒衣来。”章敦袖手站在门前,笑着对韩冈道,“玉昆,可不要学小宋。”

姓宋的人很多,别称小宋的也有不少,不过章敦能拿来说笑话的自然是名气最大的那一位。兄长做宰相,自己则做了翰林学士的那个小宋——宋祁。

宋祁好声色,被他的兄长看不顺眼,不过他是以真心待妻妾。有一回陡然天寒,妻妾十几人一同送了寒衣来,拢拢总总十几件,不论穿谁的其他人都会伤心,宋祁难做决断,最后干脆就谁送的寒衣也没穿,冻得瑟瑟发抖的回家了。

章敦拿着他说笑话,韩冈也只能一笑了之。而且另外还有同来的一人,不属于韩冈的元随队伍,却是韩冈极熟悉的——黄裳。

“勉仲,你怎么来了?”韩冈惊讶着,“这一去至少半年,你这一科不准备考了?”

黄裳向韩冈行了一礼:“国家有难,士人岂有安坐的道理?黄裳虽不才,亦愿尽一份绵薄之力。”

韩冈熟视良久,见黄裳神色诚挚,点了点头,“也罢,你这份心意难得。”

黄裳闻言大喜,韩冈转又对韩信道:“韩信,你将这封信送去光禄寺苏学士那里,跟他说,事情紧急,来不及告辞了,编修局中一应事务都交托于他。”

