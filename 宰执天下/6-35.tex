\section{第六章 见说崇山放四凶(二)}

乍听到章敦推举沈括,李定立刻将视线投向韩冈。

章敦这是要拿沈括下手?他与韩冈的关系决裂了吗?

但以韩冈的性格,以及资历,应该不会跟章敦他去争宰相位置的。既然不争,那还有什么理由两人决裂?

之前沈括临危受命,去扑灭石炭场火灾,沈括有苦劳,也有些功劳,不过因为他仅仅是让大火烧光了石炭场的煤炭,最后自然熄灭,又拆毁了数百户百姓家宅来防止火势蔓延,致使民怨沸腾,颇闹了些事,还是有些朝官指责他办事不利。

沈括出任开封府是被赶鸭子上架,受命平复危局,寻常官员都是避之唯恐不及。只是沈括本来就因为人品备受歧视,仅有韩冈愿意接纳。韩冈同意他出知开封,便不敢推辞,不得不接下来。

这一回沈括是再一次被推到了风尖浪口上,而韩冈又同意了,这不是逼着沈括离心离德?

鬼才会相信韩冈会容忍有人拆自己的台,而那个人更不应该会是章敦。

不过听到韩冈之后的回话,李定顿时恍然。

开封府审讫,交由诸法司复核,乍听起来是把沈括给牺牲了。

如果一切都这么按照正常的程序来。开封府批出判词之后,上覆大理寺复审,再送审刑院详议,由于事关重大,又多人论死,所以还有刑部复核的一道关,而御史台更将会依例全程监审。

开封府的判决一路上要过关斩将,想要顺利通过根本不可能。

可再之后呢?

如果法司将开封府的判决给驳回去,沈括是不可能就此罢休,必然要就此申诉。双方各执一端,接下来要么就是请两制以上官详议,要么就是请太后处断。

换做自己站在韩冈和章敦的位置上,肯定会选择密奏太后,让向太后直接进行赦免。

如果当真依律判决,参与叛乱的主要成员,还有直系亲属中的男丁,必然不离斩绞重刑,腰斩也会有几个。如果蔡确、宋用臣和石得一还活着,更是逃不过千刀万剐的凌迟极刑。

前两年的李逢、赵世居谋反案,就各凌迟和腰斩了两位伎术官。那一回,连谋反的谋字都算不上,只是赵世居家中藏了兵书和谶纬图书。往来书信上看不得一个阴谋。这一次,是实打实的谋反,砍下的头颅当是赵世居案的十倍。

但只要说服了太后,赦书一出,什么先例故事也就无关紧要了

宰辅们要放过一众叛逆,本来就是要请太后降赦诏。想要名正言顺的颁诏,必要的审判程序就少不了。只有先行定罪,才能赦免。

配合的倒是好。

韩冈和章敦根本就没有让步,只是先拖延一下。有了对一众叛臣的处置,另一面曾布和薛向的处置也就有了依循的标准,接下来再议论,可就脱不出宰辅们划出的底限。

李定差不多明白了韩冈与章敦的一点盘算,但他清楚,绝不会这么简单。

一切的核心还是在太后身上。

吕嘉问眼神阴冷,盯着韩冈和章敦。两人明目张胆的相互配合,绝不止是暂且拖延,以逞其谋算那么简单。

宰辅们在挫败了叛党,救回了太后与天子之后,已是功高难赏,如果再表现得太强势,在太后眼中免不了会被认为是咄咄逼人,骄横跋扈。

才经过一场叛乱,尤其是倚为心腹的石得一、宋用臣的叛离,太后免不了会疑心重重,对权力也将格外执着,此人之常情。

女人本就多疑,天子的疑心病只会比女人更重,刚刚被背叛的人则总免不了以猜疑的目光看外界,如今垂帘听政的太后是三事叠加,猜忌的程度将会是之前的十倍、百倍。

如果有人触动她的心结,之前的信任不论多深厚,也会立刻变成猜忌。

吕嘉问敢于随着李定一同顶撞诸宰辅,正是想借用太后这样的心理。

可韩冈、章敦现在已经退了一步,这边再咬着不放,太后猜忌的对象可就会转过来了。

吕嘉问此时更加确定,只要还有章敦和韩冈在,两府中空出的那几个位置就像水里的月亮,看似触手可及,却抓不到手中。

难道就这么认输不成?

吕嘉问紧紧咬着牙关。

王安石、韩绛、张璪,以及苏颂,这四人都比不上韩冈、章敦的年轻,精力早已不济。时间一长,朝堂事务必然会渐渐落到韩冈、章敦两人手中。

如果自己能在近日进入两府,还可以跟章敦、韩冈争一争朝堂大政。但若是不能及时填补上那几个空缺,待朝局安稳下来,以章敦和韩冈的能力,当能顺利的处置好军政两方面的国家大事,让朝堂上下——最关键的是太后——觉得没有必要补足两府的阙员。

到时候,想要再挤进去,就没那么容易了。即便太后有意扩充两府,牵制章敦、韩冈,也要与外路的一应重臣相互竞争,哪里有现在的机会好!?

如李定、吕嘉问一般咬碎牙关的重臣不在少数,皆是有资格跻身两府的一干人。他们或前或后,就自问已经看透了韩冈和章敦的把戏。

不过此时苏颂心中与王安石一样疑惑不解。

绝不是什么默契和配合,苏颂极为熟悉韩冈的性格,他和章敦先后发言,反倒有着些微争锋相对的味道。

从资历和官阶来说,韩冈不会与章敦争夺宰相的位置。两人要心生嫌隙未免还太早了一点。

难道是出了什么事吗?

自韩冈开口支持开封府作为主审之后,殿上一时就静默了下来,人人都在猜测韩冈的用心。

韩冈看见每个人的表情从狐疑到恍然,好像都已经看明白了自己和章敦的想法一样。

真的能想明白?

韩冈暗中冷笑,真正明了对方用意的只有自己和章敦两人吧,谁让自家曾经向章敦透露过自己的打算?

在经历过一场叛乱之后,太后的心性到底会产生什么样的变化,韩冈很想知道。

不过不论事情怎么发展,他回到两府的位置上已经成为定局,对于朝堂的影响力会恢复到之前的水平,多半还有超过。

到时候,是一点点的撬空皇权的基石,还是现在就在殿上立下法度,这不过是手段缓急的差别。

不知两府中剩下的一两个空缺,能吸引住多少人渴求的目光。

吕嘉问权衡再三,眼神坚定起来,他从窃出叔祖父的奏章草稿,投奔王安石,被称为家贼开始,至今已有十二年,将他赶出家门的两位元凶都垂垂已老,他没有第二个十二年可以耽搁了。

只是当他准备站出来的时候,只见一名内侍匆匆跑进殿中。

冲太后行过礼,内侍高声贺喜:“蔡确子弟,蔡硕、蔡渭以下十七人,并从党蔡京一人,皆已全数就擒。其中蔡渭本是逃脱,却为开封府判章辟光及西上阁门使王厚与蔡京同时擒获,已经械送开封府。”

‘擒获?’

‘怎么给弄到开封府去了?’

直接就砍了了账的事,竟然还给拖到了开封府去。

就算王厚见到蔡京蔡渭,二话不说,将两人砍了首级下来,韩冈也照样能给他报上一个不留遗患的上上之功。

办事真是不利索。

韩冈暗暗摇头,王厚是不是在陇西养尊处优太久了,天天看人赌球赌马,现在连杀人放火的老本行都忘了?换作是当年,这么好的机会在眼前却给放过了,不用别人说,王韶回头就会好生的用家法教训一下自己的儿子。

不过韩冈也没有权力让王厚为自己赴汤蹈火。

王舜臣、李信肯定会做的事,王厚却不一定会。这就是差别。

幸好有了一个章辟光。

韩冈脸色古怪。

章辟光这一回可就是露脸了。

首倡驱二王出宫,之后就被暴怒的高太皇赶出了京城。这一番折磨,就是他的资本。从心性上,章辟光就是一个会投机行险的人物,

与蔡确的区别,就是一个先走鸿运后遭灾,而另一个则是应了孟子的苦其心志、劳其筋骨——这一回‘天’就要降大任于他了。

“韩卿,这蔡京该如何处置。”

向太后问韩冈,当初蔡京与韩冈正争吵不休的时候,她算是其中一个参与者。

“那是沈括的事。”韩冈很干脆的推给了开封府知府。

“若蔡京是幡然悔悟,自是既往不咎。如果不是,依国法便不可轻饶。”

韩冈的后半句才是重点。

当着太后的面,他自是不能说让蔡京早点去投胎,但韩冈的态度十分明确——不可轻饶。

他无论如何都不会让沈括审出一个‘幡然悔悟’的结果来。

而且总不能进了开封府的所有人,最后都因为宰辅们的誓言得到赦免,总要有一两个例外,来验证国法的森严。

尽管当初的誓言中有足够多的漏洞,其实根本约束不了韩冈。就算沈括又一次叛离,对韩冈也没影响。

而那个蔡京,甚至不用审,直接下狱报个瘐死很容易就了事。

又不是台狱,犯官吃的住的,比京城人家还要好几分。

这里是府狱,皋陶的神主之后,就是暗无天日。谁敢在里面多待?

市井中的泼皮无赖,但凡被捉进了狱中,第一件事就是托人赶紧通知家里,早点拿钱将自己赎出去,半日也不敢多留。哪个不是屁滚尿流?不用上刑,住上三五天,出去后就病死的也不是一个两个了。

尤其是最深处的几间牢房,专门是用来弄死人犯。都不用见血、也不用牢卒亲自上阵,丢里面几天,出来就只剩一口气了。

