\section{第39章 铜戈斑斑足堪用(下)}

清晨的时候。

王韶已经站在了大营中的最高处,在他的身边,属于他的旗帜正在风中猎猎作响。拂面而来的晨风,稍显激烈,依然带着烈火后的焦臭味道。

河州城上,宋军的战旗依然挺立。昨天夜中,王韶亲耳听到河州城那里传来的厮杀之声,不过城头上始终未有变化的火焰,证明了姚兕的指挥才能。

视线换了个方向,眺望着数里之外,被万军簇拥着的那面已经模糊得看不清的旗帜。王韶心知在那面大旗下的木征,当也是跟自己一样,等着后方传回来的消息。

计算时间,苗授的回报差不多也该到了。除非出了什么意外,否则就算仍是在激烈交锋中,苗授也应该派人带信回来。

王韶难得的有些焦躁不安,即便心知只有在苗授全军覆没的情况下,才会一点消息都传不回来,即便是惨败,都会有败兵返回,但他的思路还是忍不住要往最坏的情况划过去。

如果是在战前,多想想最坏的情况倒是思虑周密的表现了,但眼下一军主帅因此而坐立不安,未免要让谢东山笑煞。

王韶这么想着,很努力的要将自己的心情放轻松一点,不过还是没有什么效果。熟悉他的亲兵们这时候隔得老远,生怕一个不小心,就撞到了经略相公的火头上。

一名从后过来,,递到了王韶的手中。

就在这名信使身后,高遵裕和景思立不知怎么得到的消息,都急急忙忙的赶了过来。

王韶打开蜡丸迷信,一览之后,就将纸条紧紧的捏在手中。也不理正在迫切等待消息的高遵裕、景思立二人,对着亲兵下令道:“击鼓,聚将!”

亲兵匆匆的跑走了,高遵裕、景思立连忙上前一步,急问着:“怎么了?香子城出了何事!?”

“须得速速出兵,不能让木征给逃了!”王韶回过头来,对着两人展颜笑道:“香子城安然无恙。虽然有数千贼人来袭,但王舜臣坚守城池,杀敌无算。而苗授之到了之后,贼人见势不妙,便只能趁夜远遁。”

“好!”景思立兴奋得一声大叫。正要说话,便听着聚将鼓隆隆的被擂响。他看俩看王韶和高遵裕,稍一犹豫,还是先一步赶往主帐,这是他的身为部将的义务。

高遵裕与王韶搭档了数载,已经很熟悉对方的潜藏在平静外表之下的心绪波动。景思立走了之后,他当即脸色一变,追问着王韶:“贼军退走后去了哪里?……是不是珂诺堡?!”

王韶没说话,右手松开,将信报传给高遵裕。

“田琼死了!”高遵裕一看之下,心头大惊。再一看,偷袭后路的贼人竟当真是去了珂诺堡。“玉昆那里还剩多少兵?”他忙问着。

“坚守城池当是足够了,前两日还有一千叛贼到了珂诺堡中……就怕玉昆领军出来援救香子城。”王韶并不清楚珂诺堡也被攻击,还以为昨夜吐蕃人的目标仅仅是香子城。这样的情况下,韩冈很有可能重蹈田琼覆辙。

鼓声越来越急,大营中的诸多将校被鼓点催着,骑着马从面积广大的营地的各个角落赶了过来。王韶、高遵裕同时向主帐走去。高遵裕问道:“木征知不知道香子城没有打下来?”

“从时间上看,应该还不知道。”王韶摇摇头,“他毕竟是绕路,我们这边才是行程最短的路线!”

“这就好了!可以……”高遵裕声音一顿,惊问道:“所以子纯你现在要聚将?!”

“先下手为强。”王韶凶狠的说着,“木征军的士气已经差不多见底了,只要将他们夜袭香子城失败的消息传开,木征军转眼就会崩溃。我们至少有两个时辰的时间!”

……………………

天亮了,韩冈醒了过来。睁开眼睛,眼前的天花板没有房梁和椽子,而是半拱形的穹顶设计。

穹顶上是黑黑的一片,不知被多少盏油灯的烟气沾染过。韩冈稍稍愣了愣神,终于反应过来他睡的是城门门洞旁的耳室中。

坐起身来,低头看了看褶皱起来的袍服,还有上面的污渍,韩冈想起自己这几天都是和衣而卧,好长时间没换身干净衣服了。

“我睡了多久?”他问着身边的亲兵。

“回机宜,才两柱香的时间!”

“还真是短。”韩冈揉了揉仍旧有些困顿的头脑,抱怨着,但他终究还是从长条的木凳上站了起来。

旁边就有稻草铺成的铺垫上,可尽管韩冈在军中推广卫生制度,但除了疗养院以外的营房中,虱子、跳蚤在士兵的床铺上依然都不少见。韩冈宁可睡在长凳上,也不会躺到可能会有一群把自己当作大餐的虫子的床上。

从耳室中出来,帐下的文吏就递上了计点出来的斩获和战利品的清单。

韩冈接过墨迹淋漓的纸张,低头看着,随口又问道:“刘源他们回来没有?”

“回机宜的话,刘源还没有回来。”

“也不知他那边怎么样了,应该有个三五百斩首吧?”

“只会多,不会少。”文吏躬声说着。

刘源不回来,这斩首数就不能确定。毕竟列在清单上的区区三百出头的斩首数,怎么也跟昨日的敌军数目差得太远。战阵所获远远比不上追击,吐蕃人的战马吃不住连夜行军的消耗,只要刘源追摄在后面,很有可能咬下一块肥肉来。最差的情况,也能逼着敌将像壁虎一般短尾求生。

‘要是田琼没有出事就好了。’

韩冈心中感叹着。若是昨夜敌军败退的时候,他有着一个指挥骑兵在手上。前后来袭的那四五千吐蕃蕃骑,他少说也能留下一半来。

如果香子城那边,苗授能及时的堵上吐蕃人的退路那也可以,但韩冈依然知道这个想法不现实。

贼将青谊结鬼章和栗颇都不是笨蛋,从珂诺堡向南七八里,就有小路进山,他们肯定会立刻进山,而不是继续向栗颇败退的地方前进。而且苗授肯定赶不上来,以他用兵的习惯,他应该在解救香子城后就留驻在城中,休整兵马,等到天明后才会珂诺堡这边赶来。

贪心不足啊……

大捷之后,韩冈的心情很好,笑着反省着自己,又想着自己派去王韶那里的报捷信使,现在应该到香子城了

……………………

晨曦的微光照在苗授的脸上,正向出城来送行的王舜臣道别的笑容中,很有着几分得意。

这可是解救全军危亡的大功,辛苦的赶了半夜的路,终于是落到了他的手上。

香子城能保住,镇守城中的王舜臣虽不无功劳,但也是他苗授及时回援的缘故。城下的几百具尸体他无意跟王舜臣和守军争抢,但这退敌之功,他苗授当是要占到大半。

强令王舜臣驻留城中,又留下了三分之一的兵力帮着他协防香子城,苗授天刚亮就准备领了两千步兵赶往珂诺堡去。休息了近两个时辰,士兵的体力恢复了一些。苗授又许了倍于往日的厚赏,赶上二三十里应该不成问题。

昨夜退走的贼军现在当是还在珂诺堡下,木征面临的困境,让这些蕃人不会稍稍受挫之后,就立刻退军回返,总要搏上一搏。以韩冈守城的能力,应当可以保住珂诺堡不失,等到自己赶到珂诺堡下,正好可以给吐蕃人最后一击。

希望韩冈能把他们给拖住。

“小心一点,要多注意道路两边的险地。”

苗履带着百来名骑兵,要先于大军出发,防着路上可能会有的伏兵。苗授多番叮嘱,担心这个儿子一时大意,给到手的大功抹上一层黑灰。

苗履拱手承命,“孩儿明白,不会让蕃贼有机会伏击。田琼的错,孩儿不会犯的。”

苗授点了点头,仰首望北,“韩玉昆也是心急了一点,要是他能多叮咛一句就好了。希望珂诺堡不会有事!”

“有三哥在,珂诺堡稳如泰山!都监大可以放心!”王舜臣满面虬髯,比几年前更是浓密了许多的胡须的遮掩下,让人看不出他眼下的表情。

“如此最好,如此最好。”苗授长笑着,“也只是多叮嘱了几句,这小子可远不如王都巡你。”

熙河都监用力一拍儿子的头盔,砰的一声响,将苗履派了个趔趄,“还不快去,若有差池,决不饶你!”

苗履跳上马,照空抽了两响鞭,领着麾下骑兵远去。

苗授冲着王舜臣拱了拱手,正要道别。却听见渐渐减弱的蹄声突然停了,然后就看着刚刚离开的儿子带着一名骑兵转了回来。

那名骑兵明显不是苗履麾下,脸上满是灰土,就是几条汗水冲过的小溪能看到尘埃下面的皮肤。他到了苗授和王舜臣身前,便下马跪倒,“小人石勇,拜见都监、都巡。”

“你是跟着三哥的石七?!”王舜臣认出了来人。

“正是小人。”石勇叩了一个头,“小人是奉命传捷报来着。”

“已经赢了?!”苗授脸色微变。

石勇挺起胸膛,自豪的说着:“昨夜月下,韩机宜领兵出城夜战,大败木征麾下大将栗颇及鬼章部族长青谊结。如今已经遣了得力将佐正在追击残寇之中。小人就是奉机宜之命,去经略相公那里传捷报的。”

