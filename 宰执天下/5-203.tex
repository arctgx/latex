\section{第23章 弭患销祸知何补(11)}

“阿弥陀佛,佣哥儿终于是要出阁读书了。”

赵頵坐上了马车,便忍不住低声念了一句佛。

中午赵頵还在保慈宫的时候,赵顼过了饭点才匆匆而来,陪着他的母亲和弟妹一起用了膳后又匆匆而去。天子操劳于国事,连坐下来好生说说话的余暇都没有,这番忙碌是赵頵这等宗室平日里是绝对不会有的。但一顿饭的功夫,至少让赵頵确认了他皇兄的心意。

轻巧的四轮马车,由将作监精心打造。钉了铁皮的车轮,碾过鱼鳞般的青砖地。咕噜咕噜的声响中,行驶得极为平稳。

赵頵在车厢中闭目凝神。

赵顼要为赵佣开资善堂,赵頵他可是完完全全的支持。如今亲耳听到兄长予以承认,赵頵心头上的一块大石终于是落了地。

皇兄仅存的儿子能出阁读书,又有药王弟子在旁庇佑,如此一来,赵颢即位的可能性就越来越小,他藏在心底里的那点小心思也可以就此偃旗息鼓了。只要赵佣能安安稳稳的长大成人,那他赵頵未来的生活也将会安安稳稳。

已经是放衙的时候,从皇城中离开的官员越来越多,两府、三馆、三衙和内外诸司的官衙皆在皇城之中,每当到了黄昏之时,宣德门和左右掖门内外竟是满目朱紫,让人不禁惊叹,哪里来的那么多官儿。

人流汹涌,赵頵的马车也不由的慢了下来,不过并没有会跟他争路的官吏,两匹骏马拉动的四轮马车,依然平稳的向前。就像重启资善堂、为赵佣做好铺垫的大势,没有任何人能够阻挡。

说实话,就算自家的那个侄儿有什么三长两短,赵頵他也是宁可看到赵顼从外面找个宗室子弟做太齤子,也不愿见到赵颢继承皇位。

不说别的,老大赵顼做皇帝,那是理所当然。嫡长子继位,天经地义的事。但赵颢想做皇帝,赵頵就想问一句了——凭什么?!都是英宗的儿子,都是太后生的嫡子,两个都做了皇帝,他这个仅存的一个就能甘心吗?

而且有太祖太宗和秦悼王的先例在,若是赵颢能继承皇位,赵頵知道自己不会有太好的下场——莫名暴死的可能性至少有七八成,甚至连子嗣都不一定能保得住。

‘如今倒是可以放心了。’赵頵想着,他正犹豫着是不是将自己的儿子也送到宫中。

自家的长子,年纪都与赵佣差不多,资善堂开讲,若是能做个陪读,事先与未来的天子打好关系,总是一桩美事。就像他的姐姐,已经决定将他的外甥送进宫中陪读,日后总比外甥的那个不靠谱的父亲要强。

……………………

前两天才在踊路街上见过一面的马车从不远处驶过,出右掖门离开了皇城。

刚刚放衙的韩冈眯了眯眼睛,嘉王殿下这段时间入宫还是真是勤快,才隔了几天,就又入宫了。

看起来为了保护自己手上的利益,两大会社中的宗亲们是使足了力气。请动天子的亲弟弟几次三番的出来游说,付出的代价可不会太小。

但赵顼最终会做出什么样决定,可就说不准了……幸好韩冈对此并不在意。

离开皇城,韩冈上马回家。

就是天子将蹴鞠联赛禁了又如何?想把宗室都得罪干净那也是他自家的事,韩冈可不会为赵顼多担一份心。

纵观历史,一个正常在位的皇帝,登基十年以上之后,其控制朝堂的能力基本上就达到了巅峰,很少再有朝臣能够与这样的皇帝抗衡。其到了晚年,更会是重臣们的灾难,能臣、诤臣,能有好结果的不会太多。

当今的天子独揽大权的倾向早已是显而易见,韩冈巴不得这个皇帝能成为一个名副其实的孤家寡人。

不过在韩冈看来,做了那么多年的皇帝赵顼不会糊涂到去做损人不利己的蠢事。而自家为了维持在两项联赛中的影响力,也必须在天子面前尽力为齐云总社辩护。方才的那一番陈词,似乎赵顼也听进去了。如此一来,赵顼对联赛的判决,恐怕也不会拂逆人心。

回到家中时,王旖已经早一步从宫中回来了。

跟皇后、贤妃的聊天也没什么好说的,跟过去进宫时的话题没什么两样,纵然不同的身份地位的女人在一起,可聊天时其实还是以废话居多。

王旖也没有事无巨细的转述给韩冈,挑了几句重点提了。只是在提起坤宁殿中那一件韩家出产的香精的时候,脸上却不免带上了几分忧色,“官人,须知匹夫无罪、怀璧其罪的道理。家里素称寒素,可不过十数载,便富甲一方,如果传到外面,必招人嫉。”

“所以为夫才会将各项技术扩散出去。领军多年,为夫如何会犯孤军奋战的错?”韩冈笑着安慰妻子,“不用担心,过些日子,玻璃器皿会变得跟瓷器一样便宜,香精的制造方法也已经流传天下,到时候,就没那么惹眼了。”

韩家的豪富如今也不能算是秘密,幸好自己有个药王弟子的身份可以压得住阵脚;顺丰行眼下只做批发,不做零售,仅在小部分人中有名;而韩家的根据地更是远居边陲,隔得远了,只凭传言而不是亲眼所见,招来的嫉妒也就不会太多。

但为了以防万一,韩冈还是早早的就将手上的技术扩散出去。逐步扩张、乃至更名的雍秦商会,就是依附在各项新产业和新技术的基础上逐渐成长起来的。如今韩家的顺丰行在其中,也不过是拥有一个副会首头衔的普通成员罢了。韩冈要的是影响力,而不是控制权,而且只要他的位置不动摇,会首和副会首并没有什么差别。

顺丰行的香精眼下的确有无可比拟的优势,但在技术扩散之后,这个优势保持不了几天,再过几年,香精作坊的利润就该是细水长流了。

韩冈说得虽然有道理,但王旖却难掩心中的隐忧。别的她不知道,但女人购物看重名气的习惯,她如何能不明白。没条件的倒也罢了,若是有条件,肯定会去追求那些名牌。

洗面药,一定要张戴花家的,皮靴,要大鞋任家的,买珍珠,去梁家珠子铺,染布料,得去余家染店,买香粉,则是少不了要到李家香铺逛一逛。如今的香精,自然就是要有一个脂砚斋的牌子。

用酒精萃取香料的手段,在韩冈的吩咐下,在韩家的香精作坊成立后的一年内,就向雍秦商会中的所有成员有偿公开,收取的技术转让费很是低廉,只有百贯而已。

但其他作坊刚创立时,是打着大食香露的牌子,唯有韩家的作坊例外。若是打起大食的招牌,也许能将香精卖到黄金的价格,可是一旦工坊规模扩大,很容易便会被拆穿。

所以从一开始,顺丰行辖下的香精工坊就想着自创品牌。而利用酒精对香料进行萃取,这样的技术从一开始就是独家的。虽说如今技术已经扩散开来,可品牌的优势已经建立起来了,后来者短时间内没有办法动摇到脂砚斋的地位。

除了玫瑰香精之外,脂砚斋还陆续开发出了百合、木犀、桂花、栀子等不同香型,而且还有利用不同比例混合起来的香水。如今宫中都在用,而教坊司以及小甜水巷中的名妓们更是无不趋之若鹜。有她们引领潮流,自然就在全国范围内流行开了。

虽然有人仿造——甚至宫中专门制作脂粉的工坊都造出了一模一样的香水——只是脂砚斋这个牌子的名气既然打出来了,仿冒品也只能去分食中低端,卖贵了没人买账。就是宫里面的嫔妃,也宁可用脂砚斋出品的香露,对宫中的出产反而不屑一顾。

京城中做香精的利润有多丰厚,主管家计的王旖,至少跟韩冈一样一清二楚。每次翻看账簿,看到家中财富积累的速度,总是免不了要心惊胆战。太多的金钱是致乱之源,甚至有灭门之祸。

王旖忧心忡忡,但韩冈宽慰了妻子几句,却并没有将这件事放到心上。

到了第二天,王安石领下了朝廷的诏命,将会接手发掘殷墟的消息终于由驿马送抵京城。很快就要见到父母,王旖一时间放下了心事,而韩冈也不会再提及。

王安石将不日抵京,这个新闻,一下就吸引了所有人的注意力,就连朝廷公布了对蹴鞠联赛的处断结果,也被许多人抛到了脑后。

藉此良机,重新出山的王安石会不会第三次出任宰相,这是很多人都想知道的,也是很多人都在担心的。

在当天的《蹴鞠快报》中,头版是全文刊载朝廷允许蹴鞠联赛重启的诏令,前提是京城中几座举行球赛的校场,由齐云总社负责改造成保证观众安全的球场,但王安石东山再起的新闻,却是稳稳的坐在了第二版上。

韩冈将看完的报纸叠好,嘴角有着莫名的笑意。

《蹴鞠快报》本身,已经不仅仅局限于联赛的赛报,这世界上又多了一把堪比匕齤首和投枪的利器。抢占舆论制高点,肯定是日后党争中的一个重要手段,未来的朝堂,那是会越来越有趣了。

