\section{第28章 官近青云与天通(11)}

捧着盛着半杯热茶的茶盏,呼了一口带着茶味的热气,韩冈惬意的眯着眼睛,身子也终于暖和了过来。

在刚刚进衙的韩冈面前,太常寺的官吏们分列左右,一个接着一个,一如既往的禀报着衙中一干琐碎无谓的日常事务。

尽管朝局动荡,韩冈还是照常每日往太常寺来。在明年开春,资善堂正式开讲前,他一时间还没有更多的工作要操心。

耳边的噪音终于停了下来,韩冈睁开眼,“没了?”

“回学士的话,就这些。”韩冈的副手恭恭敬敬的回答道。

“那就下去吧。”韩冈捧着杯子,也不留人。

属僚们悄无声息鱼贯而出。

如果说起变化,就是韩冈的下属们,行事和态度比过去更为恭敬了。

向皇后前日曾经想让韩冈参与崇政殿议事。对于皇后的这项提议,除了吕公著外,其他宰辅都没有否决,不过给韩冈拒绝了,顺便将第二封加赠他资政殿学士和翰林学士的诏书给挡了回去。但由此一来,韩冈究竟多得圣眷,这一点,朝堂上已经不再有人会误会了。

桌案上需要处理的公事,只有可怜的四五件,其中两件还是有关郊祀赏赐的事,若是郊祀祭天之前,更是只有现在的一半而已。

判太常寺的日常工作,韩冈处理起来,一天也只要五分钟。

随手拿起公文,韩冈提笔批阅,由于多了几件,用了大概一刻钟的样子。

再捧起依然滚热的茶盏,看看外面的日头,韩冈觉得苏颂也该到了。光禄寺的日常工作,其实也只要半刻钟。

“学士,相州急报。”一名小吏快步进厅。

“怎么了?”韩冈笼着茶盏不放手,“相州出了何事?”

“相州知州、通判及安阳知县联名奏闻,前日安阳县中,发掘除了一具高四尺五寸,长四尺,宽三尺,重达两千斤的青铜方鼎,经考证,乃是殷时祭天之物。此乃天赐祥瑞,欲呈于陛前。”

以韩冈的城府,乍听到这个消息,也差点失声叫起来。手上一抖,盏中的茶水也险些给泼出来。

竟是司母戊方鼎!

不,韩冈心中立刻否定。

殷墟里面的青铜鼎不一定就是司母戊,很可能是其他性质类似,重量相近的礼器。

但看着这名吏员脸色涨红的兴奋,韩冈就明白,两千斤重的殷商青铜礼器,对于朝廷有多大的意义。

不管怎么说,如此等级的巨鼎被发掘出来,其意义也只比夏禹九鼎和传国玉玺差个一筹两筹。对于帝王来说,这是上天的赐予,是货真价实的三代礼器!祭天时在圜丘下一摆,往台陛上走的脚步都能高上两分。

所以相州知州、通判和安阳知县连署,将这件祥瑞之物,呈献上来。

这可是出乎意料的变化。

恐怕现在不会有人记得王安石上京是为了去相州主掌殷墟的发掘工作。说实话,要不是这一条消息提醒,韩冈自己都忘掉了。

谁让几天前,赵顼突然发病呢?

韩冈轻叹了一声。

他很清楚,计划和实际从来是两回事。别的不说,他初掌厚生司、太医局时,本是打算改建四座疗养院,将之一并设为医院。但最后由于要照顾京外诸县,人手无法调配,只在京城里先设立的东西两座医院。

现在王安石已经成为了平章军国重事,自然不可能再去主持发掘殷墟。而自韩冈拿着龙骨跟新学打起擂台,已经是两个多月的时间过去了,相州的殷墟,纵然还没有给挖成了满是兔子洞的草原一般,也不会比长安周围山陵差多少了。

真不知道该让谁去收拾残局?说真的,韩冈心中是有些愧疚。

不过话说回来,随着自己成了太子师,爱拉偏架的赵官家躺在了床榻上,气学大兴已是必然。相对于殷墟中的那些损失,韩冈自信他所主张的科学,绝对能改变历史的方向,避免出现那些让华夏损失更为惨重的未来。在过去和未来之间,哪一个更为重要,他绝不会弄错。

收拾心情,韩冈慨叹道:“恐怕相州遣人上京时,当是还不知道天子抱恙的消息吧?”

相州知州、通判,安阳知县,以及所有在奏表上联名上报祥瑞的官员,这一次恐怕都会失望了——谁也不可能预计得到赵顼会在郊祀后的宫宴上突发卒中。

不过到了今天,相州上下应该是已经听说了,真不知道他们现在会是一个什么样的心情。

小吏嘴角扯了一下,幸灾乐祸的笑意一闪即逝,转又立刻恢复了毕恭毕敬的神情:“或许能给官家冲个喜。”

“若当真能如此,那就太好了。”韩冈挥了挥手,示意小吏下去。

放下已经空了的茶盏,韩冈沉吟着,突然间两手一拍,要上京的司马光这一回应该是有处去了。

司马十二对金石的爱好那是有名的,在史学上的造诣更不必多说。就是不知道岳父大人答不答应了——司马光肯定是不会放过踩上新学两脚的机会。

……………………

御史台这段时间很活跃,这让向皇后很是头疼。

原本亲眼在宫宴上看见天子发病,台中准备请太后垂帘的御史占了大半,但第二天,一听说是皇后垂帘,就变成了回去针对二大王写弹章。可再等到听说二大王发疯,刚刚写到一半的弹章,也就写不下去了。

弹劾一个疯子,没有任何意义。甚至对于皇后和太子——包括还有清醒意识的天子——来说,公示二大王发疯,比证明他是装疯,有更多的好处。

御史们不是蠢货,绝大多数都立刻将自己的奏章给烧了,只有一两个糊涂鬼。不过他们将奏章地上去后,也很快就反应过来,当向皇后将之留中,反倒是松了一口气。

只是当那动荡的一夜中的更为具体的细节从宫中流传出来后,乌台上下又重新躁动起来。王珪——这名身负皇恩,却在定储之事上犹豫不定的宰相,成了御史们的新目标。

一个个将笔杆化作投枪和利箭,瞄准了王相公的脑门,使足了气力射了过去。

相州的祥瑞早就丢到了一边,向皇后看着桌案上高达尺许、来自御史台的弹章,脑中就是一阵阵的抽痛。

是该留中吗?

向皇后犹豫不决。

如果有足够的时间,纵使是中人之姿,也能锻炼出掌控朝堂的能力。可在眼下。没有任何处理政事的经验,甚至不知该如何保持朝堂上的平衡,就算是后宫,在过去,管理者也是太皇太后和皇太后。

或者宰相有足够的威望和能力,也能帮着做出决定。但王珪不行,向皇后绝不会去信任他,也不可能让王珪自己处理弹劾他自己的奏章。地位更高的王安石又不可能为这件事而开口。其他宰执更是怕惹火烧身,躲还来不及。而可以信任的韩冈,碍于身份,却不肯答应参与朝廷政事。

她只能去询问她重病的丈夫。

得到的回答是留中。

向皇后依言做了。可是到了第二天,向皇后只能看得见堆在自己案头上,来自台谏的那一叠奏章,又比昨日高了一截。

她所能做的,就是继续留中。

而王珪在这时候,也只能避位待参。

“若是天子听政,做出了留中的决定后,他的态度,就不会让人误会了。”韩冈在家中对妻子叹着。

“但皇后的确不想让王相公出外。”王旖的声音低了点,“这也是官家的意思。”

只是饭后的闲聊而已——刚刚从宫里回来的王旖,向韩冈提起了向皇后为了那些弹章到底有多头疼。

“你当是明白,卒中一病,其病情是很难好转的。一旦疾作,即便救回来,也只能苦捱拖时间。”韩冈在自家妻子面前并不讳言,“一年?两年?还是只有半年,甚至三个月?”

王旖脸色发白,韩冈的话若是传出去,对他们一家老小来说实在是很危险。

“当然,为夫是觉得天子能够吉人天相,但御史台中人可能并不这么认为。”韩冈笑了,笑得冰冷,他自然能感觉得到天子赵顼想维持朝局稳定的想法,但天子的想法,现在还能压得住多少人,“在乌台众人看来,这个时候,讨好皇后是最重要的。即便他们明知天子想要维持朝堂稳定。”

讨好皇后才会最要紧的。或许天子要保住王珪,但皇后呢?聪明人都知道皇后对王珪是什么样的看法。

皇帝已经病重!太子依然年幼!听政的……现在是皇后!

朝局的稳定,并不取决于皇帝或垂帘的皇后。还要看他们本身的执政能力,以及朝臣们在这个局面下个人的想法。

“除非能有则天皇后的手段,否则就必须倚重宰臣。”韩冈冷笑着,“是吕文靖【吕夷简】之于章献明肃,是韩忠献【韩琦】之于慈圣光献。可现在呢?”

韩冈一开始没有想得那么多,但当他看到这几天朝堂上的变化后,就算再漫不经心,也是看明白了。

骤雨将至,人力岂能挽回?

人心思乱,又岂是重病待死的皇帝所能阻止的?

“西京那边,又岂会甘愿坐视?”


