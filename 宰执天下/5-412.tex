\section{第36章 沧浪歌罢濯尘缨(14)}

“那……”

邢恕刚要继续问下去,就听到一个怯生生的声音响起:

“两位官人,可要听时新的小曲儿?”

蔡渭的脸色顿时沉了下来,竟然让不相干的人进了自己的包房,门口的伴当玩忽职守竟到了这般地步。宰相家中规矩大,犯了错的下人自是要以家法惩治。只是火气上头后,他立刻又想了起来,门外并没有伴当守着。

为了不惹人注意,两人特地选了不属于七十二家正店的小酒楼,虽也是外楼内院,可档次终究不高。外间多是贩夫走卒,内厢也不过是附近有点身家的居民。不过这对邢恕和蔡渭两人来说,正合心意,他们连伴当都留在酒楼外,免得给人看出破绽来。

邢恕抬起眼,只见一个提着琵琶的老头子陪着名正当妙龄的少女站在半掩的厢房门前。

老头子颤颤巍巍,而少女只有十五六岁的样子,看上去像是祖孙多过父女。

那少女大大方方的上前来,向蔡渭和邢恕屈膝福了一福,“两位大官人,可要听小曲儿。奴奴会各色时新小调,秦太虚新写的几首曲子词,奴奴都能唱得来。”

在酒楼上,席上常常会有不请自来的歌女。只是一般来说,正店的包厢不会让人随意进出。可这毕竟不是正店,管理得并不是那么严格。

歌女的声音娇柔婉然,蔡渭不禁多打量了她两眼,但见那歌女容色并不出众,便又收回了目光。从袖中随手掏了几个大钱,丢过去打发那一老一少出门。没听曲子本也不需要给钱,可是宰相家的财大气粗不是普通官宦能够比得上。

七八枚大钱落得满地都是,但那名女子并没有低头去捡。她仿佛受了羞辱,双颊涨得血红:“小女子虽然在外抛头露面,可也不是乞丐。这位官人太大方了,小女子受之有愧。”

那歌女丢下话后便不顾而去,老头子抱着琵琶急忙追在后面,出了门后才想起来要回头行个礼。

被一个歌伎顶撞了一回,蔡渭脸色讪讪。他可没脸摆出宰相家的威风来,传出去肯定是他没理,何况他还不可能让这件事闹起来。

“相貌虽然不入流,这脾气倒是樊楼的。”邢恕谑笑着,顺手给蔡渭倒了一杯酒。

蔡渭人面广,人头熟,随即接话道:“樊楼的赵宝儿,张齐齐,还有三十娘,脾气的确也都算大了。方才的那个也不输他们。”

“终究还是比不过韩玉昆家里的那一位。”

邢恕抿了抿嘴,“那谁能比?雍王还疯着呢。看看这个仇结得有多大?”

邢恕说这话,顺手悄然摸了摸袖中,里面倒有两串用来结账的大钱,还有几个零散的元丰重宝,是折五大钱,还是簇新的,刚刚发行不久,就跟方才蔡渭给那名歌女一模一样。

他现在的差事不是有油水的官职,崇文院中的校书清贵归清贵,宦囊羞涩也是实打实的,家里人口多,。实在比不上宰相家的衙内随手就丢出两个大钱。

“不说这个了,先喝酒。”

邢恕放下心事,与蔡渭对饮了两杯,就听见方才刚听过的声线就从隔壁传了过来。有曲有乐,的确是最新的小词。

两人对视一笑,并不介意旁听一下不花钱的曲乐,这样一来,他们说话的声音也可以放开了一些了。

“被掳走的人口,耶律乙辛能还回来多少?”打断的思路重新接上,邢恕继续问道。

“代州、忻州和太原被掳走到辽国的户口,能有损失的三分之一就不错了。”

邢恕点点头,他不是不经事的人,强盗劫掠过后的惨状也颇看过几次。区区山中强贼都已如此,被数以万计的契丹精骑洗劫后的代州、忻州,情况只可能会更惨。

那些被掳走的百姓可能还算是运气好的,因为剩下的不是死于战火,就是在之后的逃难中出了各种各样的意外。

“而且换回来肯定还要打个折扣。”美貌的女子,有才能的士人,技术高超的工匠,这些人都很难换回来,蔡渭也不瞒邢恕:“按河东那边的说法,多半不会超过五千户。”

“是韩玉昆的密奏?”

“嗯。”蔡渭又点点头,“韩玉昆在奏章中说,代州和忻州要三十年才能恢复元气。”

“不是说避入山中的人户有不少吗?”

蔡渭嘿的一声嗤笑:“都不会超过三千户,而且没一家不用披麻戴孝的。”

“韩玉昆此前好像是上奏说,要重新河东版籍,并五等丁产簿。”

“好确定户绝田的数目,用来安置移民。”蔡渭接着道。

邢恕轻叹一声,摇了摇头:“这可是桩难事。”

战前的代州,不算近两万各自拥有家庭的驻军,都有三万民户;忻州虽小,民户也近两万。三千户在其中只占了小半。何况这些民户,没几个能达到户均五口的平均线。也就是说,实际拥有的人口比正常的要少得多。

在诸多土地的原主阖门死难的情况下,重新分配无主土地成了忻代两州的当务之急,韩冈早在屯兵忻口寨时,便安置难民在忻州去就地补种口粮。现在也只不过是之前的延续和深入罢了。

不过这一件事,其实已经超出了韩冈的职权范围。置制使是军事方面的临时差遣,之前能够允许置制使司插手地方政事,也仅是因为忻代战乱未止,韩冈以宰辅的身份权宜行事罢了。现如今,兵戈已止,置制使司再干预政事,就很难再说得过去了。

“……记得昔年蜀中大旱,韩忠献曾为益、利两路体量安抚使。”邢恕低头考虑了一阵,然后说道。

“正是如此。”蔡渭一击掌,笑道:“家严也是这么想的。”

韩冈现在的差事的确不能署理民政,既然如此,蔡确就像干脆顺水推舟弄个新差遣给他,随便找个名目,比如体量安抚使什么的,加个大字也行,体量安抚大使,

韩琦曾经受命体察并救治过蜀中的旱情。这个就是先例。有先例在,安排韩冈这等重臣,便有了名目。

相比下来,吕惠卿就比韩冈好安排多了。

只要保持宣抚使的名号,直接让他来治理陕西。宣抚使军政皆可理会,吕惠卿手中的权柄虽大到碍眼,可照规矩做事就不会有越权一说。

要酬奖吕惠卿的功劳,一个宰相之位是少不了的。不过若是能晾上几曰,却有很大的机会寻他个错处,让他的宰相梦再拖上个几年。

当然,如果脸皮厚一点,拿着曹玮平南唐的旧事,几百贯赏钱也就打发了,回来后照样只能做枢密使。

只不过要说动皇后拉下脸来,难度肯定要比让她从国库中掏个两三百万贯出来,或是给一个宰相的位置还要高。而且皇后也不可能只让吕惠卿回来,将韩冈留在外面。

政事堂想要厚此薄彼很难得到皇后的同意。皇后不画押、不盖印,就是有王安石这名平章军国重事在,也奈何不了。

那么蔡确到底想要让自己做什么?邢恕翻来覆去的想着,忽然一道灵光闪现:“是要让吕吉甫去河北顶替郭逵?”

蔡渭神色变了一下,但随即就恢复了笑容。

邢恕现在依然在司马光门下,奔走在两京之间。在洛阳,他的名声都还不错。是许多旧党元老所看好的的

旧党是不可能退出朝堂的,只要南北之争犹存,主要成员皆出自南方的新党就不可能将北方的士子给整合起来。

有人,有势,旧党纵然在两府中失去了位置,可在中层,依然不输给新党。尤其是在京朝官的序列中,旧党及其同情者的人数是要远远超过新党。只是多在地方,而难以在朝中立足。

可是随着时间的流逝,上没有宰辅统率,下则是在国子监中学习三经新义的太学生越来越多,迟早有一天,旧党免不了要分崩离析。

邢恕眉头皱了半天,正想要开口,外面呼的一阵喧闹,正是在门前的大街上。将他的话给堵住。

一名骑士穿街而过,身后飘起的旗帜上书写着墨迹淋漓的捷报。很难有人能看得清那一晃而过的文字,但露布飞捷的信使都会在穿过市镇时,向人群散播大捷的消息。

随着外面的议论渐起,邢恕和蔡渭终于了解到了到底是哪里又传来了捷报。

“王都监大破高昌?”

“王都监是谁?”

“高昌……高昌又是哪里?”

“是西域吧,芝麻大的小国。”

“还不及辽国腰上的一根汗毛粗。胜之不武,赢了也好意思叫大捷?”

“好歹是西域,走过去不容易啊。”

隔墙的议论仅仅持续了几句,喧嚣声便重新响起,唱曲的依然唱曲,弹琴的继续弹琴,并没有因为这一道来自于西域的捷报而受到影响。

如果这一回王舜臣的捷报出现在宋辽开战之前,当还是能够惹起相当程度的关注。但现在,远在天涯的胜利,相对于宋辽两军数十万大军交战的激烈,未免就显得太过微不足道了。

看外面的动静,似乎连成为酒桌上议论的话题的都远远不足。

蔡渭打了个哈欠,转回头来问邢恕:“刚才……说到哪儿了?”

