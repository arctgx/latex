\section{第40章 岁物皆新期时英(二)}

韩晋卿在刑名上闻名朝堂,甚至是到了不可或缺的地步,不过韩冈没有跟他打过交道。他家的子弟还没有到惹是生非,最后落到大理寺手中的年纪。

“韩伯修怎么了?”韩冈问章惇。

从大理国联想到大理寺,但大理寺那么多官员,至于只想到一个韩晋卿?崔台符这位判大理寺的正卿尚在任。

崔台符虽然不如韩晋卿在大理寺、审刑院中的时间长,不过他是明法科出身,而且从熙宁三年开始,同样是在法司的两个衙门里来回任官,同样可以说是律法专家,只是据传闻在能力上稍逊一筹而已。

但那位韩少卿一向与王安石过不去,斗鹑案反对王安石的观点,阿云案站在司马光一边,而大理寺卿崔台符则是始终是新党的支持者,熙宁初年,王安石与司马光争阿云案,崔台符支持了王安石。这就是为什么崔台符能从河北监牧使的位置上,直接入京权知大理寺卿事,始终在大理寺和审刑院压韩晋卿一头的缘故。

“近曰大理寺狱空,崔平夫上表的事知道吧?”章惇满是讽刺的口气。

章惇为什么会有这样的语气,韩冈明白。

“听说了一点。”他点头,又道,“不过这只是运气吧,就跟黄河水清一样,偶尔的自然变化,跟人事无关。”

天气暑热的时候,大理寺的人犯数量本来就会少不少。又不是御史台狱,对官员的恩惠能泽及狱中,以大理寺狱的卫生条件,一到夏天,人犯死个一半都是正常的。为了防止出现大量犯人瘐死狱中,以至于被御史盯上的麻烦,在夏天之前,大理寺就会尽量处理积案,将人犯打发出去。而这一回大理狱空,不过是个惯例之后的巧合而已。崔台符、韩伯修等大理寺官虽有功,却也当不起郑重其事的上表奏闻。

“黄河水能清?!”章惇问道。

“前些年不就有奏报吗?”

“听说过不少次,就是从来没见过。”

“黄河向上游去,未入陇右时,水就是清的。黄河泥沙都是从陕西的黄土高原上冲刷下来,什么时候陕西干旱一年半载,黄河水就会干净不少。”

章惇摇摇头,“闹灾荒时候谁还会管黄河是清是浊。”

“所以说这些谶纬、祥瑞之类就该丢一边去,自欺欺人,又有什么用?”韩冈发了几句牢搔,发现话题扯偏了,道:“狱空可证谳狱清明,依例似乎是当减磨勘,中书门下是怎么定的?”

“政事堂那边定的是正卿减两年磨勘,少卿减一年。不过太上皇后说,官家新登基,大理狱空更要加一等赏,崔平夫特晋一阶,而韩伯修减了两年磨勘。”

韩冈拿着酒杯,在手中转了两圈,抬眼问道:“……两年?”

见韩冈立刻就反应过来,章惇冷笑的点头,“减两年磨勘是不少了,但跟进官一级就差远了。不患寡而患不均啊。他和崔平夫明争暗斗十几年,现在都快致仕了,本来就已经差了很远,突然间又是差了这么一级,哪里能甘心?”

到了韩冈、章惇这个级别,磨勘不磨勘根本不用放在心上了。进了两府,都能直升正四品的谏议大夫,再熬上去也不过多那么一点钱,称呼好听而已。而大理寺的正卿,也就是崔台符,同样是右谏议大夫,只是没进过两府,非特旨不能再晋升了,这一回正好是特旨。但少卿韩晋卿,资历、能力都不逊色于崔台符,却还是正七品的员外郎一级,“他怎么了?”韩冈问。

“没什么,”章惇说得很平淡,“只是有人看见他家的下人向御史台那边通消息。”

“……崔平夫说的?!”韩冈怔了一下之后,才有点惊讶的问着。

若真是崔台符通风报信,足可见崔、韩两人之间关系的险恶,都要派人去盯着对方一举一动了。但若当真是崔台符做的,却是不折不扣的糊涂,将自己也给陷进去。有谁愿意用一个敢派人跟踪朝廷大臣的官员,皇城司就已经很让朝臣们反感了,若哪位朝臣敢在这个原则问题上犯错误,他的下场绝不会好。

“不,是蹇磻翁跟蔡持正说的。是他家的下人遇上了一个熟人,也就是韩晋卿家的家丁,这才发现了韩晋卿在做什么。”

“这还真乱。”韩冈笑了两声。

蹇磻翁就是蹇周辅,现在在三司做度支副使,是吕嘉问的三位副手之一。明明是三司衙门的人,却插手到大理寺中,尤其是在吕嘉问吃了大亏,在三司中声望大跌的时候,他向蔡确示好的举动,不可能没有其他用意。

不过蹇周辅曾在大理寺做过少卿,也曾在御史台任官过——不是进士,进不了御史台,这就是崔台符和韩晋卿比不上他的地方——他家的下人认得韩晋卿家下人和御史台中人,多多少少也能说得过去。而且他与大理寺两边都没有利益牵扯,也不用担心有人会怀疑他是不是派人去监视韩晋卿、甚至是崔台符。

“的确是乱。”

“这是送上门的刀子啊。”韩冈呵呵笑着,“难怪蔡相公会这么笃定呢,原来有这一桩事抓在手中。”

朝堂上但凡牵连多人的大案,基本上都是从不怎么起眼的小事开始罗织罪名,然后一点点从缝隙处的撕开盖子,最后一网成擒。

不说御史台与大理寺卿相勾结,就是。结果最差也不过是崔台符也给拉下去,与韩晋卿同归于尽。

与章惇对饮了一杯,韩冈又随意的问道:“打算保崔台符吗?”

章惇啧了啧嘴,叹道:“……那真要看情况了。”

大理寺那里,崔台符这位判大理寺卿事,蔡确、章惇是肯定想要保的。刑名系统中再找不出与他资历和地位相当的新党支持者了。而想要调人进去,想去的没资格,有资格的,却不会有几个愿意接受。

进士出身的文臣,最怕的就是案子多,不得清闲。最喜欢的就是清要之职。出典州县,遇上诉讼多的地方,做不了多久就会活动调任。而天下刑案汇聚的大理寺、审刑院,派哪个进士去,都是不会愿意久任,只想拿来做官路上的一个跳板。

但韩晋卿与御史台暗中联络,手中肯定有崔台符的把柄,若是真的彻查下去,崔台符也会一并被牵扯进来,到时候双方都得完蛋。只是罪名的轻重问题。

一杯酒饮尽,章惇持银壶给自己和韩冈倒酒,往韩冈脸上多看了两眼,“玉昆看起来对此事没什么兴趣啊?”

“何以见得?”韩冈问。

“能这么问就知道了。”章惇摇头道:“真要有兴趣,至少会问一句乌台中到底是谁与韩晋卿私下交通。”

“谁?”韩冈喝了半杯酒,问道。

“蔡相公没细说,所以也没问。看着就好了。”

冲章惇的口气,就知道他也是一样没兴趣掺和。不论蔡确用什么手段将御史台上下清洗一番,空出来的位置,都少不了他章子厚的一份蛋糕。

韩冈更是事不关己,反正不要再推荐张商英那等愣头青上来就好了。

……………………韩冈在章惇家喝到初更,方告辞出来。

他现在不是宰辅,没有太多的顾忌。拜访两府中人,可以更加光明正大。

虽然说已经过了立秋,但还是夏天的感觉。

夜风依然燥热,前几天稍稍凉快一点,但这两天就又热了起来。

在章家多喝了两杯水酒,虽说度数不高,可热风一吹,就感觉有些醉意上涌。

前面有旗牌喝道,街上的行人车马都避让到路边。原本挺热闹的街道,先是一阵鸡飞狗跳,然后就一下就静了下来。

韩冈觉得有些不舒服,回头看看清凉伞还在背后张着,便不高兴的说道:“太碍眼,又不下雨,打什么伞,收起来吧。”

亲随不敢违逆,忙收起了清凉伞,前面喝道的旗牌官也不那么张扬了。

没了前面吵吵嚷嚷的吆喝声,韩冈感觉上就好多了。

就还是热,抬头看天,繁星密布,明天看起来也不是阴天,更不会下雨。

幸好京畿种麦的多,早收割了。要是种稻子,不知会有多少家哭。

今年天下各路比不上前几年风调雨顺,除了兵火带来的[***]之外,还有天灾降临。陕西的旱情比较严重,江南东路和荆湖南路报了洪涝,邕州上个月连续十几天阴雨,当地损失不少。

不过以大宋的疆域,哪一年都少不了有几个地方闹灾,之前三年多的无灾无祸,其实也只是路一级的安定,下面的州县还是有灾情的。

想到这里,韩冈不禁叹了一声。如果没有与辽国的连番大战,还有帝位交接,今年的灾情根本算不了什么,但现在朝廷没钱,内库也没钱,赈济一时间只能靠地方的库藏来支撑了。

也不知州县中,会有多少官员将医疗卫生放在心上。大灾之后有大疫,救灾并不仅仅是让灾民吃饱就好的。

太医局和厚生司,早就交给他人了,不过里面的程序还是按照韩冈过去制定的方针在实行。只是那些研究工作,少了韩冈来指引方向,一个个都陷入了停顿。

让他们自行开拓,对未来的发展是有好处,但放在现在,浪费时间是小事,就怕耗尽了朝廷的耐心。

韩冈心中犹豫着,是不是要推动成立一个学会,每年拿出一部分钱来支持各个方向上的基础研究?

