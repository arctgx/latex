\section{第44章 秀色须待十年培(七)}

听到郭逵正往军器监这边杀过来的消息,章惇顿时一脸的无奈。

手指着韩冈,叹道:“玉昆,这下给你害苦了!”

韩冈不肯认这个罪名,笑道:“点火的可不是我。”

“这是笑的时候?郭仲通可是要打上门来了。”听韩冈一推干净,章惇没好气的说着。

“打上门来也只能认了。郭太尉才住进来,房子就给子厚兄你给砸了,这晦气的,他能不气吗?”

“好像这里面就没玉昆你的事一样!”

“……其实还是开封府的责任为多。”韩冈静了一下,然后说道,推卸责任的对象换成了开封府,他更加不会犹豫,“炮弹才多大,郭逵家的正堂又有多大?一根绣花针能扎死大象吗!笑话!”

一个比拳头略大一点、十一二斤重的铁球,竟然把几丈高的房子给拆了,除了开封府维修时偷工减料,还能有别的解释吗?那可是节度使宅邸的正堂,就跟大庆殿在皇城里的地位一样。要是一块石头落地,就把大庆殿震塌了,是丢石头的人被治罪,还是监造、修缮的人被治罪?

韩冈的说法,章惇喜欢,“开封府的确不得辞其咎。不过也算是意外了,可能是砸准了房梁,就像打到要害一样。”

“没错。蛇有七寸,这房子也差不多。”

“也算是郭仲通的运气。”章惇叹道,“幸好早一步把坏的地方爆出来了。万一今曰没发现,曰后碰上刮风下雨,那可就不可收拾了。”韩冈扭过头去问章惇,“哦?郭太尉来了之后,这样说可以吗?”

“……唉!”章惇苦笑着叹了一声。这等没脸没皮的话,这边说笑没什么,哪里能对外面说?

韩冈也是一声叹:“幸好郭仲通没去告御状,只是打上门来。”

“郭仲通能一刀断马首,我可是手无缚鸡之力。全要靠玉昆你应付了。”

“韩冈这水平,论文才,倒是能胜当朝一众武将,论武艺,东班倒也没人能比得上。但要说跟郭仲通比武艺,就跟与家岳比文才一般,子厚兄,你这不是难为人吗?”

两人你一句,我一句,脚步不停,赶着往军器监门口接人去。郭逵怒气冲冲而来,要是自家还敢摆着架子,这就是连人都不会做了。

其实两人的心情还是很轻松,幸好炮弹掉到郭家时没伤着人。比起郭逵家的房子塌了,有没有人受伤这一点更为重要一些。

火炮实验命中郭家正堂,只是个不幸的小小意外,只要没人在意外中受伤,也不过是茶余饭后的谈资,在百万军民和官员们口中,当成笑话说说。当然,火炮的名气会借着郭逵的光,传遍九州四海。但伤到人可就不一样了,韩冈、章惇,哪个能脱罪?

幸好没伤到人,韩冈在这片刻时间里,不知第几次暗道侥幸。

要是火炮的第一个战绩是郭逵,后世不知道会怎么说这种把自家名帅给干掉的武器了。

方才臧樟在听到郭逵家中正堂垮下来后,立刻派了一人回报,自己过去打探消息。以他身上穿的官袍,倒也没费力气就从司阍嘴里问出了郭逵家中无人受伤,想也知道,他当即就又派了一人赶回来报信。

第一个人过来回报时,韩冈和章惇心里都大叫不妙,万一郭逵或是他的家眷磕着碰着一点,他们少不得就要递辞表。这可是开国以来,枢密使打算亲手干掉同僚的第一个例子,不论本因为何,都要给武将们一个交代,绝不是能够一笑了之的事情。

幸好第二名信使很快就跟着回来,听说郭逵家中无恙,两人就同时放下了心。至少这门新出炉的兵器,不至于刚问世,就能赶上拥有一个帝星的飞船,拿到了解决三位当朝辅弼的战绩了。

喝道的声音越来越近,郭逵的仪仗转过了街口,出现在军器监正门的大街上。

韩冈和章惇一见,立刻苦笑起来。

“万幸没拿刀枪。”

“就是空手也赢不了。”

苦中作乐的态度,其实韩冈和章惇都是觉得今天的事太有意思了。没了人命压在心头上,反而都想开开玩笑。不过随着郭逵越走越近,两人脸上的表情,都变得严肃了起来。

郭逵打着全副仪仗杀上门来,就跟上阵时全副武装一样,这是要当面讨个公道的态度了。

军器监所在的厢坊,只有工匠的住处,闲杂人等不多,但前面来了章惇、韩冈,现在又来了郭逵,还是引来了不少人注意。

尤其看见章枢密使和韩宣徽使两人降阶相迎的时候,周围的议论声陡然就大了起来。

除非是宰相,否则人臣之中,还有谁能受得住这样的礼节?就是不论文武,只说官位班列,郭逵也是在章惇与韩冈之下的。

乍看到韩冈和章惇,郭逵眼皮就是一跳。他就没想到,章惇、韩冈会都在军器监中。

如果只是军器监中的官员们,郭逵倒真不在意发作一通,舒一口鸟气。区区一个判军器监,郭逵在理直气壮的时候,不会放在眼里。

但韩冈和章惇就不一样了。韩冈是罪魁祸首不假,但他在场和不在场是两回事。

郭逵之所以没去告御状,就是不想与韩冈真的翻脸。

生气归生气,但理智还是有的。毕竟没死人,只是砸了房子,不管怎么说,这件事也闹不大。除非当真是军器监瞄准自己射击的,但即便是那样,他们可能会承认吗?

只是心头这个憋屈啊,让郭逵的心里越发的不痛快了。

当韩冈和章惇一起迎上来,郭逵也翻身下马,向前走了两步。

韩冈到了近前,就向郭逵行了一礼,“韩冈见过太尉。方才火炮误射中太尉家宅,韩冈正打算登门致歉去呢。”

伸手不打笑脸人,韩冈先一步低头道歉,郭逵就是有气,当着这么多人的面也不能发作。但韩冈的话中之意,岂不是说自己打上门来是心眼小?根本就没必要过来?

郭逵的心中更是堵着慌。

章惇也过来了:“方才也是章惇的失误,不小心将炮口抬高了半寸。要不是这个错,也不会飞出军器监去,伤到了太尉的家宅。”

‘就不能好生道个歉吗?’郭逵阴沉着脸,闷声道:“郭逵当不起。只是一间屋子而已。”

“一间屋子也是太尉的家。实在是对不住太尉。这的确是韩冈的错。太尉有什么话尽管说,韩冈认罚!”

“哪里。”郭逵沉着脸道,“既然拆了郭逵家的正堂,那罪魁祸首,郭逵可以看一看吧。”

“不知太尉想要看的是,是火炮,还是点火的人?”韩冈立刻问道。

“火炮。”郭逵道。他找一个点火的小卒子做什么?他现在想看看火炮啊!

被炮弹砸坏了自家的房子之后,郭逵在愤怒之余,对火炮也是抱着浓浓的兴趣。真的给韩冈弄出来了,而且当真是要胜过霹雳砲和八牛弩的样子。

“既然如此,那就请太尉移步到院中。”韩冈邀请郭逵进军器监,还笑道:“太尉要是只看点火的人倒是好办了,子厚兄就在这边。但要看火炮,就只能多走几步了。”

郭逵惊讶的看向章惇,点火的是这位枢密使?

章惇笑得尴尬,韩冈当面就把他给出卖了,“火炮当真是军国之器,章惇看了,也想亲手体会一下。食君之禄,忠君之事嘛。”

郭逵也不穷究,章惇是个胆子大的,这一点朝中无人不知。

“这就是火炮?”看到了真品,郭逵的问题跟章惇没有区别。

“正是。”韩冈点点头。

“就这火炮,将那十几斤重的铁弹射出去的?”郭逵看着这支青铜管子,立刻就产生了兴趣,将之前的不快抛到了脑后,双眼发亮的来来回回打量了好几遍。然后问道,“郭逵方才听下人说,先是有雷声霹雳响起,然后就有东西飞过来将正堂给砸了。应该是有关系的吧?”

“的确正是火炮。火炮射击声势很大,本来火药就是用来做鞭炮的嘛。有火药推动,火炮的射程很容易就超过三百步。”

“五百步有了。一里半!”郭逵的口气一下变得很冲,脸色也臭了。但韩冈、章惇都没放在心上。任谁家里的房子塌了,心情都不会好,而且还是刚住进去的新家。

韩冈走到还没有收起来的木板靶子旁边,指着上端的缺口:“这个五百步是撞上靶子后的五百步。要是没有阻挡,当能有七八百步的射程。”

本来炮弹的方向只是稍稍高于靶子,抛物线近乎于平直,就是飞也不该飞出多远。没撞上前面的院墙,也该很快就落地。不过飞出去的时候,炮弹在作为靶子的木板上蹭了一下底部,虽然只是一点点,但前进方向立刻就改变了。轨道变高了,由浅平的曲线,变成了有角度的曲射。所以一直飞到了郭逵家。

如果一开始就是以最大射距为目的,从三十五到四十度角开炮的话,七八百步当真不成问题。

炮口的角度没安置好看起来只是一个意外,但韩冈绝不会这么认为。如果有可以调节角度的炮架,就根本不会出现这样的情况,只要没有标错数字,读角度表就能知道到底瞄准了没有。

现在的火炮,其实只是一根炮管,需要观瞄设备,更需要稳定和可调节的炮架,只有将这些配件准备好,火炮的研制才能算是成功了。不过那样,真不知道要到几时,或许还会有人觉得这是浪费时间,但现在有郭逵做例子,倒是好解释了。

福兮祸之所伏,祸兮福之所倚。

韩冈瞥了郭逵一眼,坏事也是能变好事的。
