\section{第12章 锋芒早现意已彰(13)}

从雄州来到大名府,用了刘绍能两天的时间。

只比急脚递稍慢一点的速度,让刘绍能的双脚在下马的时候都直打颤,也让这位高阳关路兵马都监灰头土脸的走进大名府衙门的门厅时,惹来了一片嫌恶的目光。

厅中尽是文官,都是陌生的面孔,没有一个认识的人。刘绍能安静的在靠门口的位置坐下,距离最近一人也隔了三四个位置。他被人打量了几眼,然后再无人关注,一群文官已经自顾自的交谈起来。

刘绍能静静地听着,议论的话题最多的无外乎北面的那点事情,还有一些各个衙门各自的事务,偶尔有人说个官场中的笑话,然后便惹来一阵压抑的笑声。

在门厅中等了大约一刻钟,刘绍能便见到一名身着皂衣的胥吏出现在门前。

厅中的小声对话中断了,十几名文官立刻向鹅一样的伸长了脖子,引颈期盼,而那位胥吏却叫着另外一人:“高阳关路兵马刘都监可在。”

刘绍能连忙站起身:“绍能在此。”

“宣徽相公有请,都监请随小人来。”

吏员两句话说完,很是干脆爽利的转身就走,刘绍能连忙跨步跟上。股间一阵剧痛传来,让他的脸皮抽搐了一下。从雄州任上一路换马南下,两天走了五百里,已经一年多没如此长时间的骑在马上,让刘绍能现在吃足了苦头。但他随即便恢复了正常,恍若无事的跟了上去。

举步出门,背后便是一阵窸窸窣窣、交头接耳的声音。刘绍能估计自己的大概要被人用目光给刺穿了。

厅中都是等待宣徽相公接见的官员,凭什么他只等了一刻钟就被提前唤了进去?免不了要让人猜测。

刘绍能也听到有人为自己的身份所惊讶,相互间打听着。

高阳关路是河北四路之一,不过自前岁辽军入寇之后,高阳关、定州、真定府、大名府四个经略安抚使路,便合为一路。吕惠卿为河北路安抚使,二十余万河北禁军、厢军,皆奉其号令。但除了各路的经略安抚使和马步军都总管、副都总管被撤除之外,下面的钤辖、都监则依然维持着原样不变。

‘真的要打了……’

刘绍能对自己受到的待遇安之若素,微微一笑,以现如今的局面,是本该如此。

被胥吏领着穿门过户,一直来到一座堂屋前,抬头看了看匾额上的字,刘绍能谦卑的低下了头来,尽管已经是一路都监,掌管三千兵马,但他能走进这里的机会依然不多。

胥吏进去禀报,随即里面便传话让刘绍能进去。

顿了顿脚,去了鞋底还残留泥土,又掸了掸身上腿上的浮灰,他这才上了台阶,走进门中。

堂中有七八人,中心处,是两位金紫重臣。一个五十上下,身材挺拔,纠纠不群。另一个,则有七十多岁,满脸皱纹,双眼浑浊,显得老态龙钟。

判大名府、河北安抚使兼马步军都总管,同时还是南院宣徽使的吕惠卿吕相公,以及他的副手冯行己。

刘绍能上前向两人行礼,“刘绍能拜见相公、太尉。”

“及之远来辛苦了。”吕惠卿过来将他给扶起。

冯行己则打量了刘绍能两眼,笑道:“刘二,你这是忙得连衣服都没换啊!”

“相公和太尉有召,绍能不敢耽搁。”

吕惠卿状似满意的点头,而冯行己在后面则挑了挑雪白的眉毛。

他对吕惠卿拱了拱手,道:“相公有事,那下官就先告退了。”

吕惠卿一板一眼的回礼:“今日劳动防御了。”

“不敢,不敢。”冯行己说了两句,转身便要出门。

卫州防御使、河北路马步军副都总管冯行己在军中是个颇有名声的老人,旧年曾得韩琦举荐,在河北任职多年,只是没什么大的军功。之前辽军入寇河北,表现不过不失。不过一来他名气大,二来地位也高,最关键的他是真宗时故相冯拯的儿子,不为文臣视为异己,所以能安然做上吕惠卿的副手。

只是这两句对答,让刘绍能感觉,冯行己跟吕惠卿并不是一条心,至少在目前的事上,这位副都总管不愿意趟浑水。

“相公,太尉!”一名胥吏恰好出现在门前,挡了冯行己的路,递上一封公函,“雄州遣急脚递入京过境,并有密函转呈相公。”

“雄州……”

听到这个地名,冯行己也不出去了,和吕惠卿一起转向刘绍能。

刘绍能摇摇头,表示自己不清楚。

急脚递肯定要比他走得快,通常一天一夜便能走完雄州到大名府的道路,接下来,再用一天半夜的时间,将消息送到京垩城。看着是前后脚,其实至少比自己迟了半天。

“或许是有关辽使。”刘绍能猜测着,“绍能出来前,正好有一名探子说看见辽国那边有一队人马往白沟来。当时猜的或许是辽国的使臣。”

吕惠卿闻言便笑道:“尚父殿下如此心急?”。

“都等了多少年了,能不急?”冯行己也笑着道。

“说得也是!”

吕惠卿看了冯行己一眼,随手撕开了火漆密封的信函,抽出信纸看了起来。一个让人眼熟的名字,再一次出现在他的眼前。

吕惠卿嘴咧开得更大了一点:“果然是辽国的国信使,还是熟人。”

“是谁?”冯行己问道。

“萧海里……”吕惠卿一声嗤笑,“又是萧禧这厮。是辽国无人了,还是觉得他做不了正事,就被打发出来了。”

“到底是为何派他出来?”冯行己没笑,追问道。

深深的吸了一口气,吕惠卿用平淡的语气说道:“自然是耶律乙辛篡位了。”

国信使带来了耶律乙辛通过禅让得登大宝的消息。宋辽乃兄弟之邦,辽国新君践位,当然得尽快通知大宋。不过吕惠卿可以肯定,太后绝对不想认耶律乙辛这门亲。

吕惠卿问冯行己:“防御,你看这当如何处置?”

“此事只能由相公决定,岂是下官能插话的?”冯行己提声反问,“不敢耽搁相公的时间,下官先告退了。”

他一推了之,跟着便告辞离开。

刘绍能微微抬起眼皮,只看见吕惠卿望着门口的眼神冰冷如冬。

转过脸来,吕惠卿的表情已经变得如春风般和煦。他笑着让刘绍能坐下来,又道:“也怨不得冯防御,他这把年纪,着实受不得累。”

刘绍能唯唯诺诺,不敢应声。两位顶头上司的明刀暗箭,轮不到他插话。

对冯行己的躲闪,吕惠卿暗暗冷哼了一声,不过也只是冷哼。

冯行己已经七十多岁,若不是吕惠卿需要这么一个不管事的副手,一早就使人上表弹劾他恋栈不去了。而依靠冯行己来指挥军队的想法,吕惠卿从来都没想过。

冯行己尽管缺点甚多,但他有个好处,就是知道分寸,懂得谨守本分,将大部分的心思都放在边境的榷场上,而不是衙门中。

郭逵贪财,那是为避免狄青的下场,而冯行己的贪财,就是货真价实的贪财了。冯家的奢靡,从冯行己之父冯拯开始便有名于国中,几乎能与寇准相提并论。为了维持奢靡的生活,光靠家乡的产业是远远不足的。

这样的副手,虽不能引为助力,至少可以不用担心在大垩事上拖后腿。

可吕惠卿还是希望能多一点人手可供驱用,他手上的长于吏事、财税、刑名、水利的幕僚很多,愿意将自己的前途赌在他身上的官员也有不少,但在军事上可以提供帮助的,无论是幕僚还是官员,都是寥寥可数。

可惜杨家没人了,来到河北后,吕惠卿不止一次的遗憾着。杨文广在临死前,尚上表献阵图并取幽燕之策。可杨文广死了这么些年,几个儿子却没一个人能出头,皆是庸碌之辈,不堪一用。否则以杨家在河北军民心目中的地位,吕惠卿不介意提拔一两个杨家子弟起来。

如今吕惠卿手里面,只有几个从西北调来的将领可以用上一用。就像眼前的刘绍能,就是吕惠卿从陕西调来的旧部。刘绍能虽然过去是蕃官,但几代为将,又愿意脱离族人,已经可以视同汉官,故而能在河北任职。

“连个确凿的消息还没传回来,辽使就在叩关了。”待下人送上了茶,吕惠卿笑着对刘绍能道,“细作腿脚慢了点,不过也可见辽国那伪帝的心急。”

“相公说得是,细作的腿脚的确慢。绍能在陕西的时候,也有好几次将来袭的西贼击退了,派出去的探马才绕回来。”

“伪帝想要朝廷的认可,不过朝廷不可能答应他。如此一来,伪帝恼羞成怒,必然要发动大军。”

刘绍能挺起胸膛:“相公放心,绍能只怕他们不来!打过白沟也只要相公的一声吩咐。就是……”

“就是什么?”

“就是粮秣一定要足。当年跟西贼打仗,总是因为粮草跟不上来,大军出去几天就要回来。”

“这你不必担心。如今朝廷绝不会让人饿着肚子打仗,大名府这里囤积的粮草也足够三军吃上一年。”

“这件事,及之你不用担心。”吕惠卿信誓旦旦。

他最近刚收到吕嘉问的私信,信上除了说朝廷内部政争,也说了钱粮绝对不是问题。吕嘉问在三司看过全国的账簿,有把握为河北河东提供足够的粮秣和资材。
