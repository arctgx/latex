\section{第32章 金城可在汉图中(22)}

天快黑了。

贺胜正站在敌楼上,拿着他还不熟悉的器物,透过透明的镜片,观察着城外远处的敌营。

辽军是直接将营地安在县城附近的村子里。从千里镜中可以看得很清楚,几处离城最近的村子里面,进进出出的全是细小如蚂蚁的身影。

从地理位置上看,村庄一般都会建在高地上,以防雨后积水。同时村庄直接连接道路,交通也便利。又有房舍,免得搭帐篷。再从防御上,有坚固围墙的村庄也远比临时搭建的营地更为可靠。要是村中再有食水,更是绝佳的落脚地点。现在也只不过缺粮草水源罢了。

镜筒的一端紧贴着一侧的眼眶。贺胜他之前自是没有机会接触千里镜这样的贵重军器,只是听说这样的一具千里镜,随便在哪个地方,就能价值百贯以上。拿着黄铜镜身的双手,就像守财奴死死攥着金砖一般。攥着镜筒的手也让黄铜镜身变得温热起来,掌心渗出的汗水润湿了镜筒,贺胜在衣服上用力擦了擦手,蹭去了汗水,又紧紧的攥住了镜筒,盯着人头攒动的敌营。

从下方扫过贺胜的眼神中,多有带着羡慕和嫉妒的。

贺胜靠了姓名讨了巧,现在在人们眼中贺胜不是那个傻愣愣的小赤佬了,而是标准的祥瑞。既然如此,那就不能让他出事。否则在战阵中他中上一支流箭,那可就是大吉转大凶了。

本有人提议给贺胜一个小官,提拔到制置使司中,不过给韩冈否决了,甚至不同意将他调离城墙,以防坏了军心。无功提拔并非治军之道,韩冈在这方面极有原则性。但又要保住他的安全,所以还是有人想了办法,让贺胜做了望远观风的斥候,拿着千里镜在飞船上向着敌阵远观。

飞船的安全性其实很不错,只是曾经摔落下来的遇难者实在太有名了,让许多人对跨上飞船都有一份畏惧。贺胜战战兢兢的上了狭小的吊篮中,只是还没到黄昏,空中的风就变大了许多,飞船在天上被刮得看着都快横了过来。守御这一段城墙的将校连忙下令收起了飞船,差点连苦胆都给吓出来的贺胜也终于被放下来,改在了敌楼上侦查。不过这风刮得也不尽然都是坏处,辽军的飞船同样也没办法使用,探查不到城中的动静。

整整半日多都在拿着千里镜,贺胜已经是双眼发花。酸涩的眼睛眨了又眨,突然有了发现,村庄中的那些蚂蚁一般的黑影正在一批批的离开他们的营地。

“辽……辽贼那里有动静了!”贺胜眼睛终于离开了镜筒,回头在楼中大叫起来。

“辽贼攻城了?!”就在敌楼中的一名军校一步跨了过来,劈手抢过了贺胜手中的千里镜。

“好像是走了……”贺胜在已经举起千里镜的军校身后小心翼翼的说着。

‘果然是去找水了。’敌楼中的其余十几名官兵,立刻就小声的议论了起来。

‘也许是撤退呢。没水没粮,这样根本没法儿攻城。’

‘真能那样就好了。’

“胡说什么?!”军校回头过来一声吼,铜铃般的圆眼在楼中瞪了一圈,让敌楼的最高层陡然间安静了下来。伸手将千里镜塞回贺胜的手中,他便往楼下走,还不忘丢下一句,“走的是马,不是人!以后学着分辨。”

贺胜拿着千里镜,愣愣的点头,就听见噔噔噔的下楼声,急促的消没在楼下的最底层。

……………………

“辽人有动静了!?”

“有大批的战马离开?”

“只是战马?”

普慈寺的大雄宝殿中,一群人围着一条长桌,沙盘、地图,城防模型都被放在一旁。十几对眼睛望着赶来报信的军校,

黄裳、田腴,还有太谷知县一个接一个出声发问。

在韩冈的幕府,或者说参谋本部中,来来往往的人很杂。有韩冈带来的幕僚,也有军中的将校——**品的小使臣、甚至还有没品级的指挥使——另外,太谷县本地的官员,知县、县丞、县尉、主簿都参加过韩冈主持的军议,并且还被允许发言乃至提议。

韩冈这样的做法极少见,大多数将帅都是依靠自己和幕僚制定计划,征求各方面的意见,然后分派命令下去,以求将资源和信息全都控制在自己手中。而不是如韩冈这样集中不同方面的负责人来集中参与决策,本人只单纯的控制着战略的大方向。

稳定城内,共抗外虏,军民一心是守住太谷县的前提,而要将事情做好,则需要所有人的通力合作。韩冈制定的一系列计划,少不了本地官员的配合。单纯的下令,最多也只能让人将事情做到七八成,如果是本人参与到其中,事情就不一样了,最明显的就是主动性大大增强。最后得出来的方案,不能说是最好,但在韩冈的控制下却是最稳妥的。

“只是战马,随行的骑兵并不多。”那名军校给了太谷知县肯定的答复。

“看来是准备将战马拉走了。”一名参加军医的将领说道。马要是没水喝,死得有多快,稍稍熟悉马性的人都知道。

“既然不敢在河中饮马,想来辽人本身也不敢喝水。”黄裳回顾韩冈道,“这比预计得还要好一点。”

韩冈还没说话,太谷知县就立刻道:“辽贼敬畏枢密如神,自是不敢拿性命。”

韩冈笑而不言。心中却道‘哪是畏我,是畏疾疫啊’’。

试问谁能不畏疾疫?谁敢不畏?辽人也一样是人!”

为了污染河水,粪尿,甚至腐尸都往水里倒。不论敢不敢喝,即便流水冲得再干净,这个心理压力是免不了的。现在连马都牵走去逐水草,辽人当然更不敢去喝河水。

如今世上对疾疫的认识,基本上都出自韩冈的一系列防疫防病的科普书。而对名为病毒实为细菌的致病源,一知半解反而更让人增添了恐惧之心。对疾疫的恐惧是来自于牛痘在辽国国中的推广。如果换在过去,河流的一点脏水真的不至于让他们干挺着。

“有看到炊烟吗?”陈丰忽然问道。

“有,不多。”军校回答道。

韩冈明白陈丰的用意,对太谷县丞袁介点头赞许道:“袁县丞,这事你做得好。”

太谷县丞是个五十多岁、没功名的老官僚,听到了韩冈的夸,脸色一下涨得通红,下巴哆哆嗦嗦,都结巴起来。

当然值得夸奖,能将太谷城周围的村民都安然撤入城中,并且销毁了无法带走的柴草秸秆,整套工作都是这位县丞来主持的。想对他的经验和能力,进士出身的太谷知县就差了许多。

“看清辽人的马了吗?”一名与会的武官问着。

“千里镜可看不见,要问出城的游骑了。”太谷知县笑道。

黄裳立刻接话上去:“不用问了,游骑之前的回报中,很多都说了辽人探马的坐骑掉膘掉得厉害。”

“不掉膘才不正常。”田腴说道,“寻常的年景,北虏哪有春天出兵的道理。这一回回去,还不知要死多少马匹。”

要不是形势使然,耶律乙辛也不会出兵南下。哪个契丹人不知马性?消耗了一个冬天,马匹的体质下降得厉害,就是顿顿精粮,用黄豆好生将养着,也一样填补不了消耗的体力。春天时一千里两千里的远距离跋涉,体质稍差一点的战马都撑不过去。

“只是离开的只是战马而已,大部分士兵都留了下来,看起来打定了主意,可能是要准备攻城了、”

“不是可能,而是肯定。方才你们也听到了。”

“辽贼会怎么攻?”

“依靠人数垒土成山不是难事。而且攻城材料并不缺,有房子就不会没木料。太谷县城的城墙并不高,稍长一点的梯子很容易就能搭上来。而且还有城外的那一片屋舍呢。”

“……”太谷知县沉吟着,最后点了点头。

太谷县是位于要道上的县城,人烟辐辏,商旅往来频繁。这一点便使得太谷县与边境上的军城,以及太原那样的战略要地有了决定性的不同。

太谷县有城壕,很算得上宽阔,可其中有很长一段已经壅塞了很久,城门外跨越濠河的也是宽阔的石桥而不是防御性质更浓的吊桥。

自城门延伸出来的官道两侧,是鳞次栉比的商铺酒家,以城门桥外最为密集,甚至形成了一座比城内还要繁华的商业区。而在城墙内侧,也多有紧贴着墙修造的房屋,这样能省下一面墙的砖石和人工,但对守城来说,实在是糟透了的一件事。

贴着城墙内侧的建筑使得调兵遣将和运送军资必须通过城墙顶端的通道,同时攻城时往城中射些火箭进去,是人人都会保留节目,这些建筑还会因为太过靠近城墙而成为火灾的源头。而外侧成百上千的店铺屋舍,更是会成为辽军攻城时的隐蔽物和攻城器械的资材来源。

不过这件事在众人眼中还是很好解决的,城内的另说,至于城外的那一片建筑,“不过是一把火的事。”

说出这句话的并不是韩冈,而是秉承了他心意的黄裳。

打仗没有不牺牲的,不过是些房舍,人都躲进城来了,有什么不敢烧的?黄裳跟着韩冈,可以说是老行伍了,人都杀了成千上万,烧个几百间空屋自不会多眨一下眼。但在正常情况下,这件事都只会放在心里,打仗的时候什么事都能发生,没必要明着说出来。

“如果辽军想借用这些屋舍,直接点火烧了便是。”黄裳低声道,“我们主动毁屋,怨恨就归结在我们身上。因为辽人开始攻城,而百姓就自然归怨于辽贼。”

城外的屋舍可能会被辽人拿来当做攻城的跳板,或是拆卸下来分解为物资,今天晚上一把火烧了,自然就不用再担心。若是能连着辽人在一起烧了,就更好了。那时候,可就不是简单的大捷了。

想起朝廷对军中的赏赐,众人一时浮想联翩。

“都准备准备吧。”这一次军议上韩冈是第一次开口,沉稳的声音将众人散出去的心神拉了回来,“多半就在今晚了!”

“那今晚城下可就能多上一堆旺火了!”黄裳语气昂扬。

……………………

夜色渐浓,灯火如星,绕着城墙的顶端串了起来。

远眺着暗夜中的太谷城,城下的连片阴影远比城墙更加深黯。

“知道什么叫灯下黑?”萧十三收回了投向远方的目光,回头问着。

早已将今夜的任务分派下去,萧十三的身后只剩下他的亲信将领。本等着最后的吩咐,但一群将领没想到萧十三会问出这一句,愣了一下不知道该怎么回答。

萧十三紧抿着嘴,但嘴角的笑纹却分外狰狞且得意,“亮者越亮,暗者自然就越暗。如果不点灯,暗处的的东西还能勉强看见轮廓,但点了灯后,不受光的暗处却会更加看不清了。”
