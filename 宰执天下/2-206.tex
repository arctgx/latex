\section{第32章 吴钩终用笑冯唐(13)}

从城头上很快赶回驻地,吴逵将一直提在手上的铁枪交给门口的近卫,犹豫了一下,然后跨步走进厅门。

曾经同行了数日的韩冈,正坦然的坐在厅中。喝着茶,与陪在旁边的几个叛军军官聊着天。听着他们说话的时候,韩冈时不时的端起茶盏,抿上一口,微笑着,自在得就像是来串门的朋友。

见到吴逵进来,众叛将退到一边,韩冈也站起身,拱手行礼:“吴兄,久违了。”

韩冈风采一如往日,就跟当初长安相辞时一样。再看看自己,吴逵不由得叹了口气,上前回礼:“韩机宜,久违了。”

韩冈被吴逵请着坐下,看着他变得苍老了许多的一张脸,感慨着:“真是造化弄人啊……当日长安一别,本想着回来后再与吴兄一叙,想不到竟然变成了眼下的局面。”

吴逵默然无语,劝降的一上来就戳着痛处,让他不知该说什么好。

吴逵的手下跳出来帮着他解围,“都是王文谅进的谗言,韩相公又不辩是非。不然如何会变成如今的局面?”

“王文谅已死,而韩相公的一番心血也因尔等付之流水。此事孰是孰非,世间自有公论,韩冈今日来此,也不是来做评判的。”

韩冈的反驳让厅中的气氛有些僵硬,从言辞上已经算是委婉,但与之前的陆渊相比,仍是强硬了许多。众叛将都有些不适应,连吴逵都怀疑他是不是故意过来摆下马威的。

见吴逵为首的众叛将都不说话,韩冈笑了几声,出言缓和紧绷的气氛:“韩冈自进城来,见着沿街的各家宅院、店面都是完好无损,看来吴兄在咸阳城内也是管得甚严。”

“此地皆是我等乡里,平素与邠州、宁州往来甚多,军中也多有亲戚在此,”吴逵答着,“兵变的罪名的确是洗脱不掉,但祸害乡邻之事,吴逵也不会去做。”

“吴兄谨严自守,此事做得甚是。如果一路烧杀抢掠,那就真个是贼了。”韩冈点头赞许。忽而直起腰,双眼一扫四周众叛将,提声道:“尔等即是未有自弃,如何不早早率军出降,求朝廷一个恩典?还在此迁延时日,岂不知,拖得越久,祸事越深的道理?!”

韩冈跳过吴逵向众叛将说话,吴逵本人脸色却毫无变化,神色如常,让韩冈心中讶异不已。

只听得吴逵问着:“前面陆渊来劝降,听他说起如果能开城投降,无论是本人还是其亲属,都会免去死罪,而仅仅是流放河湟……还说是韩机宜你的提议。”

吴逵问到了关键的问题上,众叛将十几只眼睛立刻盯住了韩冈的脸,韩冈越发的纳闷,‘怎么一点都不在意他自己?’

“没错,正是韩冈的提议。”心下犹疑,但韩冈的回答一点也不迟疑,“在下于宣抚司之中,仅是管勾伤病事,但在秦凤缘边安抚司中,在下却是参赞军事的机宜文字。对比过这两条,我想诸位无需怀疑在下的诚意。”

这三千叛军如果真的被同意流放通远军,那他们将会被全数打散,安插到各个屯田堡中。并实行彻底的兵将分离,叛军中所有队正以上的武官,全都要另行安置。这样的处置方案,当然会引起叛军的反弹,所以韩冈是不会说的。他只是要他们相信自己而已。

“可机宜你也只是缘边安抚司机宜……”

“但韩冈的提议,已经得到了韩相公的准许……吴兄你也不要怀疑韩相公的心意。横山攻略功败垂成,说起来韩相公的怨恨是最深的。罗兀城那里连番大战,斩首两千余,阵斩西贼都枢密,眼看胜利唾手可得,可就是因为尔等在庆州之叛,而不得不放弃罗兀,全师回返。要说韩相公对你们没有怨心,那都是骗人的。”

众叛将一阵骚动,每一个都是一脸的惊容。他们没料到韩冈说的这么直白。而他们更没想到的,是罗兀城会有如此大的战果。

吴逵的眼神沉了下去,以他对横山战局的了解,如果罗兀城真的守住,横山肯定逃不出大宋的掌心。如今因为自己引导的兵变,官军却不得不放弃罗兀,让韩绛的一番心血化为泡影。

这仇……报得的确是痛快无比!但怨恨也是越结越深。按照韩冈的说法,韩绛心中的怨恨是最深的。那他会不会事后反口,每个叛将心里都转着疑问。

韩冈看了眼他们的脸色,又道:“但韩相公也不会因私心坏国事。横山事已至此,杀了你们也挽回不了。但若能助河湟一臂之力,对官家、对朝廷,也算是有个交代。”

韩冈的说话几乎都是针对吴逵之下的叛军将领。叛军中的绝大部分官兵,都是被谣言鼓动起来而已,一时被冲昏了头脑。现在后悔的绝然不少,只是因为上了贼船,跳不下去,才不得不一条路走到黑。只要说服他们,完全可以把吴逵丢在一边。

韩冈本是做好了吴逵反驳和干扰的准备,可他没想到前任的广锐军都虞侯就放着自己来撬墙角,这态度真的很奇怪。

按理说,在正常情况下,招降谈判时,吴逵应该将主动权把握在自己手中,把手下的将校排斥在外才是。可他偏偏相反,将主要的叛将都招呼了过来旁听。这是嫌自己死得不够快吗?

如果是他控制不了手下的军队,还算是个理由。但眼前的情况,吴逵很明显的将三千军卒把握在手中——能约束不伤百姓,军纪差一点的官军都做不到,更别提叛乱的军队了。虽然韩冈不知道他用的是什么办法,但这手腕肯定是没话说的。

事有反常,必有妖异。这吴逵究竟是想怎么做?

韩冈分心二用,一边猜疑着吴逵的盘算,一边详细的回答着叛军将校的疑问。一句也不提对吴逵的处置。吴逵本人也像是忘了,根本不问。心照不宣的避过这个话题,可是最终,还是有人问起了宣抚司要如何处理吴逵。

韩冈双眼锁住了吴逵的表情变化,直率的回答了这个问题:“我只敢保证除吴兄之外的三千人的性命。韩相公也已点头,一旦尔等放下兵器,出城投降,便会上书朝廷。如今天子仁德,尔等并无杀伤百姓,足见尔等不是穷凶极恶之辈,见到不动刀兵便解决此事,官家定然欢喜。至于对吴兄的处置……韩冈不够资格参与。”

韩冈说得很明白了,只是没有捅破最后一层,但足以让人明白等待吴逵的是什么结果。

叛军将校立刻喧哗起来,多为吴逵而感到愤愤不平的,甚至还有人说,既然吴逵不能被赦免,干脆就不降了。只是吴逵一声呵斥,便让他们都住了嘴。平静如水的面庞上,看不出一点点情绪上的动摇。

‘视死如归?’

韩冈看吴逵的样子,实在平静得过了头。可是他锐利的眼神,绝不是放弃了一切的模样。到现在还在想着拼出一条活路吗?还能有什么招数?难道眼下的情况,他还能从城中跑掉不成?

‘算了,’韩冈放弃了多想,吴逵若是真能跑了,他也是乐见其成,‘只要三千叛军不跑就行了。’

想明白吴逵必然宁有盘算,韩冈便没有继续去说服叛军立刻出城投降。更没有当年郭逵入保州劝降时,以己身为人质的想法。留话让吴逵和一众叛将好生考虑一个晚上,便在他们的礼送下,出了咸阳城。

在城外的大帐中,听过了韩冈的回报,赵瞻立刻发作起来,“死到临头,贼人竟敢如此怠慢,如此狂悖,如何还能招降?!”

如果不是赵瞻说话,韩冈就会建议韩绛不要耽搁时间,今天照样排出投石车,以打促降。只是现在赵瞻抢先说话,韩冈也就没必要出头去附和他,有逆反心理在,韩绛不会答应的。

不过还有一部分原因是他也想看看吴逵有什么办法能从这天罗地网中脱逃,故而才缄口不言。

到了当天夜中,一个急促的声音将韩冈惊奇,匆匆穿衣出帐,就看见咸阳城中一片火光。

“起火了!咸阳城起火了!”

营中一片声在喊着,还有人乱哄哄的跑着。

韩冈眉头一皱,正要怒喝,就听着身后一声暴喝,“不要乱!”

竟是种朴和种建中出来镇压局面。

本就是不关城外官军的事,营中的乱局很快就平息下来。

到了下半夜,城中的火势消减,逐渐收止。天亮后,咸阳城门打开,城中的叛军鱼贯而出,在城门口,丢下了手上的武器。而领头的,只是不见吴逵的身影。

“吴逵呢?”韩绛厉声问着。

烧毁的县衙废墟中,只有几具烧焦的尸身,其中的一人手边横着吴逵惯用的铁枪,依然黑黝黝的,与攥着铁枪的烧焦的手一个颜色。

韩冈摇头,焦臭的尸身让昨日的疑问得到解释。吴逵的反常也有了理由。只是这金蝉脱壳、李代桃僵的手段做得实在很烂。

“搜!”韩绛很显然的也不相信眼前的焦尸是吴逵,他怒声叫着,“把城外围墙守好,将城中每一个角落都给我搜遍了!”

