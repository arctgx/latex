\section{第28章 官近青云与天通(一)}

王旖一夜没有阖眼,就在孤灯前坐了一夜。

周南、严素心和韩云娘也都没睡,陪着王旖在内厅中坐等着消息。

只是全都是让人心惊胆战的传闻。

王安石那边传话是韩冈确认了天子还有神智——这是转自章敦的第一手消息——也就是说,皇帝除代表神智的眼皮外,就没有其他能动的了;而另外两条,雍王留宿宫中和宰执在三更的时候全数入宫,则是家里的家丁派去御街打探来的。据说当时御街两边的各家家丁,跟老鼠一样,一窝窝的。

晨曦越过了院墙,照了进来,在地面上镀了一层金红,但厅中压抑的气氛却一点也没有被冲散。

“朝会快开始了。”韩云娘突然说道。

片刻之后,严素心才像是回应一样,“过一阵子官人应该就会有消息了吧。”

“嗯。韩信做事伶俐,应该能打探到。”王旖点点头,像是说给自己的听的一般。

周南则一直都沉默着,从昨天天子发病,韩冈被留在宫中之后,就已经没什么话了。等到二大王留宿宫中,继而,更是一个字也不说,一口水也不喝,更不用说吃饭了。

这个时候,韩钟和金娘带着弟弟们,一起过来拜见母亲。

几个孩子在膝前拜倒问安,平常王旖还要问一问功课,但今天没有心思,挥了挥手,让乳囘母带着孩子们下去吃饭。

韩家的次子起身后就扯着王旖的衣袖,“娘,爹爹昨晚没回来?”

“二哥真笨,昨晚吃饭的时候,不就没见到爹爹。”金娘指着韩钲的鼻子,细声细气的说道。韩家的儿子多,女儿少,家里面最受宠的就是这个女儿,惯得胆子大了。

“我才不笨。”韩钲立刻大叫了起来,“三字经我都背下来了。”

“九九口诀我也背下来了啊。”金娘哼哼着,扬着下巴,“二哥你只会加减吧。”

韩钲不服气:“姐姐你比我大,等我跟姐姐你一样大的时候,也肯定会背了。”

“别闹了。”韩钟阻止弟弟妹妹,“娘娘正担心爹爹呢。”

“爹爹没事。”金娘立刻说道。

韩钲叫了起来,“娘娘还不知道,姐姐你怎么可能知道!”

“我就是知道。”金娘的声音也变大了,“爹爹绝不会有事!”

几个小儿女闹腾着,王旖正心烦呢,眉梢顿时就挑了起来,在扶手上一拍,沉下脸不说话。

韩家的儿女们登时就给吓到了,一个个跟老鼠见了猫儿一般屏声静气起来。

“别闹了。”严素心连忙赶着他们的走,训斥着下人:“你们怎么看着哥儿姐儿的,还不带他们走?回去收拾一下准备今天功课,别耽搁了。”

刚将孩子们赶走,一名家丁就小跑着过来通报了,“夫人,韩信回来了。”

王旖一听,急忙道:“还不快让他进来!”

韩信匆匆而至,“夫人,已经打听到了。是六皇子为皇太子监国,由皇后权同听政。”

“皇后……”“到底打探确实了没有?是皇后,不是皇太后?”

“回夫人的话,确定是皇后。小人问了好几遍。”韩信答话道:“当时是两队天使出宫,一队往南面去了,另一队则往西走了。就是他们当着上朝的文武百官的面公布的出来的。”

王旖的脸色缓和了,立刻有追问起了韩冈:“学士呢……有没有学士的消息?”

话声没落,周南就念了一句佛号,“阿弥陀佛。”就软软的靠在交椅上,身子却斜斜的看着就要倒下。

“南娘姐姐!南娘姐姐!”韩云娘忙上去扶着她,却发现周南竟然是昏迷了过去。

“来人,快送南娘回去歇着。”王旖招了两名妇人,将周南送回去休息,继而一声轻叹,“苦了她了。”

韩云娘和严素心惶惶惑惑,弄不明白这究竟是怎么一回事。

韩信没敢问是怎么了,跟王旖回话:“学士的情况还没打听到。小的这就回去打探。”

“你快去吧。”王旖挥挥手,让他走了。

韩信一走,王旖整个人也松懈了下来,卸下了心头重担的叹了一声,然后对严素心和韩云娘笑道:“这一回不要紧了。官人可是有了福报。只要小心谨慎,就能几十年长保富贵了。”

“姐姐,这话怎么说的?”韩云娘听不懂,跟严素心一样,一脸的疑惑。

“是皇后权同听政,而不是太后。”王旖笑着解释道,“为了太子,官人肯定是做定了资善堂侍讲,皇后那边也会恩遇始终。不像太后,因为最疼爱的二大王,一直看官人不顺眼。”

“原来如此。”严素心和韩云娘算是明白了。若是太后垂帘,韩冈肯定要受打囘压,更不用说二大王登基。现在自家官人的弟子成了监国太子,跟王旖关系还不错的皇后垂帘,韩家当然是安稳了。

也难怪周南会昏倒,没吃没喝的紧张了一夜,而且因为二大王的事,是最紧张的一个。终于有了好消息,突然之间放松了下来,当然容易昏过去。

“弄点吃得来。坐了一夜,都饿得慌。”王旖提声吩咐着下人,回头当即又点了一名亲信家人:“将学士的事知会城南驿的老相公一声,说请勿忧虑。”

“相公会不知道朝堂上的事?”

“爹爹那边即没有人手,又不能入宫,耳目还不知道会闭塞成什么样。”

王旖可是很清楚,王安石这一次上囘京,身边就跟了寥寥数名亲信,早年的门客早就打发光了,纵有门路千万,没人手怎么打探消息。

但这个家丁去了才半个时辰,就忙着赶了回来,一见王旖就立刻说道:“夫人大喜,老相公得授太子太傅。”

“已经知道了。学士还正式得授资善堂侍讲。”这是韩信刚刚传回来的消息。

周南现在也恢复了,从房里出来,正喝着加了药材的小米粥。王旖对着她笑道,“不知道还会不会有什么好消息传来。”

好消息没有立刻跟着来,紧接着的却是有人上门送礼。说是来贺韩学士,递了名帖和礼单,将礼物在门房一放,请了回执之后就走了。

若是半刻钟前,韩云娘还会觉得惊讶,但现在却一点也不会了。

原本韩冈功劳太大,年纪太轻,所到之处皆有开创,天子不得不出手压制一下。甚至因为道统之争,逼得皇帝将千里镜归入军器,明显的做出打囘压气学的姿态。韩冈虽位高,但家门即便不能说门可罗雀,可上门的宾客的确不多。但眼下形势一转,当然就变得炙手可热,气焰腾腾起来了。

还不到中午,已经有十七八个人家派家仆来送礼了。这些都是耳聪目明的,知道韩冈接下来必然大用。不过王旖还是照老规矩,先将礼物封存,礼单登记造册,等韩冈回来再做处分。

只是才交午时,韩家的门前车马又翻了一倍。

——韩冈昨夜辞了参知政事,而且是领头请立太子。

……………………

就在王旖为突然而来的喧闹而头疼不已的时候,比平日要延长不少时间的朝会终于结束了,韩冈回到了太常寺。

比起皇城里各司的僚属吏员和城内的官宦人家,被圈在文德殿中一上午的朝臣们,他们得知真相的时间还要迟上一点。

不过到了中午的时候,差不多所有人都了解了昨夜大体的内情。

“玉昆,传言是真是假?”苏颂早就绕过来了,见到韩冈劈头就问。

他的光禄寺跟韩冈的太常寺一样,郊祀大典前没多少事,郊祀大典过后,却要为太常礼院擦屁囘股,一般要花上三五天的时间来收拾手尾。

但苏颂实在是坐不住,事情太大了,不找韩冈做个确认,怎么都不可能安心下来处理公务。

韩冈也不瞒他,反正像立储这么大的事,为了避免谣言滋生,宫里面也得主动让人去传播真实的消息甚至内情。现在可都是正午了,太阳移到了正南方,恐怕再吃一点,连逍遥洞里的老鼠都该知道昨夜福宁殿里发生了什么事——所谓的逍遥洞,那是开封府下水道的别称,贼来贼往,逍遥得很,由此得名。

“前面还不知道是怎么回事,这一下子终于是明明白白了。”听完了韩冈的叙述,苏颂摇了摇头,又赞道,“玉昆你好手段啊!乌台那边肯定是没话说了。就算是皇后垂帘,也没人敢说不是……不过玉昆你一向厌说鬼神,这一回却要装神弄鬼,不知是怎么一个说法?”

“事急从权,要什么说法?”韩冈笑道:“耀州、祁州的药王祠的香火可是要旺了。可是没办法跟他们分账啊!”

苏颂指着韩冈,摇头道,“就是两边的药王祠跟你分账,玉昆你敢拿吗?”

韩冈笑了起来,有着以往没有的轻松。但这时候一名内侍在外通报了后囘进来。

“韩端明。奉皇后口谕,请端明速至崇政殿。”

“臣谨遵懿旨。”韩冈行礼接旨,又问道,“不知是为了何事?”

“是契丹的正旦使萧禧到了霸州。因端明素知兵事,圣人想请教一下端明的意见。”小黄门说话低三下四,身负圣谕,却连请教二字都说出来了。

“萧禧又做了正旦使?”韩冈摇摇头,“又是来敲竹杠的吧。”

苏颂也几乎在同时道:“该不会是契丹人过来打饥囘荒吧?”

两人对视一眼,点了点头。

韩冈一边让伴当帮着整理着衣帽,一边道:“契丹国内有事,耶律乙辛弑幼主,不从大宋这边捞笔好处回去堵人口,他也不好过年啊。”

“若是给萧禧听说了天子病重,他肯定会狮子大开口。”

‘太子年幼,皇后又从无署理政事的经验,这一回可不好办。’有内侍在,这一句话两人都没说出来。

宋辽是兄弟之邦,年节时都要互遣正旦使,这是年年都有的情况,哪里需要如此戒备。但一来辽国最近的局面不对,任谁都知道耶律乙辛会想从大宋这边沾点便宜好安抚国内,韩冈知兵法,知道怎么应付辽人;第二,就是皇帝的情况不对了,皇后心里没底。剩下的,自然就是韩冈昨夜立功的好处了。

只是韩冈看了看苏颂,苏颂可是出使过辽国的,本身也不是不通兵法,既然招自己,苏颂更应该一并找过去备咨询才对。

这个念头才升起,立刻就有个内侍冲进来了,喘着气对着苏颂道:“苏学士,苏学士,你可让小人好找。圣人有命,请学士速至崇政殿。”



