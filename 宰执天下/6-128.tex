\section{第12章 锋芒早现意已彰(九)}

风雪中,耶律乙辛正一步步的走上封禅台。

毡冠挡不住雪片,锦袍也抵御不住风寒,却挡不住耶律乙辛向上的步伐。

这个风雪交加的日子,是耶律乙辛最后一次穿戴着臣子的服饰。

再过片刻,当身份转换,准备已久的实里薛衮冠、络缝红袍,就能派上用场。

举步踏上台阶,每一步都仿佛有千钧之重。

咚!咚!咚!

如同重鼓,如同炮声。

远在封禅台的脚步声没有一丝声息传来,却响彻每一个人的心中。

昨日耶律乙辛刚刚将擒获的反抗者在千百朝臣面前轰成了齑粉,今天他便登上封禅台。

用敌人的血来作为登基的台阶,这是北国的惯例,耶律乙辛纵然是通过篡夺得到了帝位,可他向上每一步,垫脚的石阶下也绝不缺少亡魂。

没有比这更为符合北地的风俗了,也没有比这更为正统了。

这样拿到大辽的权柄,成为统御万里疆土的皇帝,耶律乙辛没有半点心虚。

每踏下一步,都是稳如泰山。

……………………

早就习惯了北地的风刀霜剑,萧十三眯起了眼睛,视线追随着逐渐走上高台的身影。

多少年了,总算是走到了这一步。

虽然接下来还有更多的问题亟待解决,可最大的难关也不过是日后必然会来入寇的宋人。宋人对西京道和南京道觊觎已久,不禅让,宋人会来,禅让,宋人也会来。既然如此,又何必去顾虑北上的宋军。

至于已经在国中的宋国使团,不论怎么处置,会打过来,即是把他们当成祖宗供上,也肯定会打过来;不会打过来,即便都杀光了,宋人也不会因为他们而出兵。

——区区几名的使者而已,只可能成为战争的借口,决不会成为决定两国命运的战争的起因。

宋军到底会来不会来,朝中众说纷纭,便是在耶律乙辛的亲信中,也有不一样的判断。萧十三不打算去猜,反正契丹的将士从来不怕战争。

在外劫掠时的作战,和护家的战争,是两回事。

如果宋人不明白这一点,萧十三很乐意告诉他们。

……………………

张孝杰正满心期待的注视着耶律乙辛走上封禅台。

天是阴的,雪也在下着,寒意无处不在,但心却是热的。

如同铁浆沸腾一般的热。

之前他奉命修筑封禅台,在参考了宋人的史书之后,张孝杰竭力做得尽善尽美。

他等待这一天,已经等待很久了,自从亲自带人收拾了宣宗皇帝被摔成肉酱的残骸,他就已经在全心全意的盼望着耶律乙辛能够成为皇帝。

今日之后,他将不再是权臣的亲信,而是堂堂正正的朝堂重臣。在过去,他以一位宰相的身份,却不为国族各帐的那些老人所重,但今日之后,还有谁敢再看低他耶律孝杰一眼?

不会再有人了……

那些心存不满的,早就在火炮下被打成了碎肉。没有比火炮更适合当成行刑用具的武器了,刀枪斧钺哪一个能做到一炮轰去便万马齐喑的效果?

而这火炮,正是在他耶律孝杰的主导下铸造出来的。

昨日看过了火炮的登场,不用再提醒,每一个人都知道大辽北院宰相叫做耶律孝杰!

……………………

这是第几个了?

刘霄远眺着独自登上禅让台的身影,想着。

尽管只有这一回是亲眼所见,比不得南国五代时的朝臣经历丰富,可对于耶律乙辛的篡夺,并没有引动刘霄太多的情绪。

大辽立国以来,争位、谋逆之事,从来没有断过。

眼下耶律乙辛正在做的,不过是又一次而已。

与之前争夺帝位的区别,不过是当事人不是太祖皇帝之后,同时他的运气和耐心远比一众先行者要好得多罢了。

昔年太祖皇帝耶律阿保机临朝十年后驾崩,淳钦皇后述律氏便支持次子耶律德光即位,是为太宗。而理应即位的长子耶律倍避走后唐,这是第一次。

之后太宗耶律德光死在南征归途,淳钦皇后又选了幼子李胡继承帝位,耶律倍之子耶律阮遂起兵反叛,击败了李胡,软禁了淳钦皇后,是为世宗,这是第二次。

再后来耶律察哥弑杀世宗皇帝,耶律璟被拥立登位,这是穆宗,不过‘睡王’的称呼则更为有名。穆宗在位十八年,亦为臣子所弑,世宗次子耶律贤被拥立,也就是景宗皇帝。

开国五十余年,还没有哪一任天子是平平安安名正言顺的即位。

也就是景宗长子圣宗有母承天太后护持,才得保无恙。

而最近一次争位之乱,是耶律乙辛起家的圣宗次子皇太叔耶律重元之乱。从圣宗长子兴宗皇帝耶律宗真亲封的皇太弟,到宣宗即位后封赠的皇太叔,兴宗、宣宗几次三番说要将帝位传给耶律重元,却始终没有践诺。

当年兴宗和重元两人的生母法天太后打算废长子兴宗,改立次子重元,还是耶律重元主动告发,才免去了一场变乱。兴宗为了酬谢重元,将其封为皇太弟,答应传为于他,可最后还是。耶律重元被兄长、侄儿骗得这么惨,他起兵也是常理。

失败者众多,而成功者虽少,也不是没有。

世宗皇帝是一例,眼前正向禅让台上走去的,就是最近的另一例。

至于叛乱,就是数也数不清了。

大辽幅员万里,国中大小部族比天上的星星还多,朝中为了避免他们势力扩大,也会尽可能削弱各个部族的实力,每年要求上贡的牛羊马驼,都要将各个部族手中的余力给压榨出来。

怨恨也就一年年的积累起来,没有哪年没有叛乱。不过所谓的叛乱,只要大军一到,便立刻土崩瓦解。有数十万铁骑坐镇,大辽从来不用怕叛乱,只要让叛贼没有闹大起来的实力就够了。

武力决定了一切,兵强马壮者为天子,这是北国的通则。

所以刘家一直都是对契丹人忠心耿耿,只要宋国不能有绝对的胜利击败契丹铁骑,那么他们就会继续做大辽的忠臣。

秉持着家训,尽管刘霄的两位叔祖还是圣宗皇帝的女婿,但他完全没有为正统天子尽忠的想法,只要臣服于胜利者就够了。契丹人想要控制燕蓟之地,就必须得依靠他们这些汉人世家。既然如此,谁做皇帝还不是一样?反倒是宋军来了,才会让刘家无法保持如今的权势。

——不管以什么标准,刘霄都不觉得自己这个状元能在宋国考中进士。而没有进士,在南朝便难有权势。一直以来,刘霄和他的家族,很清醒的认识到这一点。

……………………

禅让台的阶梯,只有八十一级,却漫长得仿佛永远也走不到头。

但耶律乙辛终于走了上来。

并不算大的禅让台顶端,有几名侍卫,一名宫人,除此之外,便是天子。

大辽皇帝还不及十岁,如果说大宋幼主赵煦是胎里带来的瘦弱,辽国的幼主就是病弱了。穿了一身天子大祀时的祭服,瘦小的身子很勉强的才撑起沉重的衣冠。

在禅让台上,小皇帝双手抱着用黄绸包好的国玺,已经等了有一个时辰之久,冻得小脸都青了。

看见耶律乙辛终于走上台来,他立刻双手颤抖着将怀里的国玺高举过顶。

还没到移交国玺这个环节,台下的唱礼官刚出声便变了调,小皇帝身后随侍的宫人,也紧张的上前去提醒。但他的动作,被耶律乙辛的目光制止了。

耶律乙辛举步上前,劈手将国玺夺过去,迫不及待。

他已经等待得太久、太久……

还不到十岁的小皇帝,被耶律乙辛吓得连退了两步,咕咚一声仰天摔倒。

在台上的宫人忙上前将他搀扶起来,开始动手脱去金文金冠、白绫袍、络缝乌靴的天子服。

耶律乙辛则根本就不去理会,转过身去,面向千军万马、文武群臣,将黄绸包裹的御玺高高举了起来。

之前几位汉臣说了什么规矩都忘了,耶律乙辛全都给忘了。

汉人大臣为今日的禅让所写得那些诏书、文章,四字一句、六字一句,看起来整齐得很,可念起来软绵绵的,有什么意义?

要做的早就做了,要说的也用火炮说了,最后一步,也不要什么繁文缛节了,少念了几句废话,难道他就做不了皇帝了?!

耶律乙辛才不信一堆废话,比得上武功、财帛更能慑服人心。

拿到国玺,穿上冠冕,赏赐百官三军,然后大赦天下,最后……便是等待宋军的到来。

这就是要做的事的顺序。

举着国玺,眼前忽的明亮了起来,久违的阳光让耶律乙辛眯起了眼睛。

不知何时,雪停了,风也止了,一线阳光从云层的裂隙中透射下来,照在了封禅台上,照在了耶律乙辛的身上。

无数人目瞪口呆,难道耶律乙辛当真天命所归。

不知谁是第一个喊起了万岁,但片刻之后,千军万马都在发出了响彻天际的吼叫。

大陆北方千万里的土地,在这一天,换了一个新主人。
