\section{第21章 涉川终吉黄龙锁(上)}

大势已去。

一听到南门被人献了出去,李常杰脑中便是一阵天旋地转,身子晃了晃,要不是身后的亲兵连忙搀扶,就要一头栽倒在城头上。

可他被扶起来后,照样是头晕目眩,一时看着一众惶然失措的部下,也不知该说什么好。

“大越完了。”一名将领悲叫着,双腿软软的瘫到了地上,就坐在地上抬起头,面上再无一点血色:“太尉,降了吧。”

李常杰记得他当初跟着自己一起鼓吹着北攻邕州,等宋人南来后,又是一力主张举兵与其决战。是军中最为强硬的一个,现在都城已破,真面目就露出来了。

外无援军,内无守心,这样的城池如何保得住?!现在连城门都被人献了,就是富良江中的黄龙亲自下来助阵,也无济于事了。

就在这时候,一直密布于天空中阴云突然裂出一道缝隙,云层上空的艳阳投射下来,为裂开的云朵镶上了一条金边,雨水虽然并没有,但眼见着就变得小了起来。

李常杰之前拿着天兆来鼓动群臣和军中坚守城池,甚至还将库中的兵器全都发了下去,将满城的男丁都组织了起来,连着退入城中的军队,加起来几近十万人。这全都是因为天上的雨水在不断的落下,富良江的江水一日盛过一日,这才是仅仅以一城之力,还敢在宋军面前坚守不降的主因。

一套连李常杰本人都是确信无疑的谎言,成了支撑国中士气的仅存的根基,现在眼看着就要云收雨散,谎言也不再有用,这时候还有谁能再坚持下去?

换作是几天前,有人若敢在李常杰面前劝他投降,李常杰肯定会二话不说就拿他斩首示众,以儆效尤。可现在当真有人在面前说着投降二字,李常杰也没有力气再惩治于他。

一切都完了,连上天都抛弃了大越,还有什么能让大越不至于国亡族灭的手段?李常杰想不到,而跟随着他的所有的将领也同样想不到。

完了!

大越完了!

主帅和将领变得失魂落魄,很自然的就会影响到下面的士兵,防守的力度几乎在短短的时间内一下就降到了最低。

南门处的欢呼声并没有传过来,但东门城头上的守军中已经看不到几个还在向城下射箭的人了。

在前面指挥作战的燕达,作为一名久经战阵的宿将,在第一时间发现了城中情况定然有变,立刻遣人返回中军,向章惇、韩冈通报敌情。自己则是趁机加强了攻势,希图能由自己一举攻下交趾国都。

“难道是城中内乱?还是李常杰出了什么事?”章惇皱眉自言自语。

“看眼下这个样子,城中的守军也差不多到极限了,交趾人不会有死守到底的胆量。”在官军过江后,城中的朝臣们有许多还派了亲信来联络投降之事,只是等到雨季到来,暴雨如注之后,他们立刻就没有了动静。不过现在,想必这些墙头草、变色龙,又要换一个倒伏的方向,换一身别的颜色,韩冈很肯定的说着,“多半是内乱!”

就在燕达加强了攻势,而章惇、韩冈甚至包括李宪在内,三人都在猜测着,城中究竟出了何事,让城上守备的力道一下就变得软弱了许多。

派在南门处压阵的一个指挥的指挥使,这时候派了一名传令兵骑马过来。可能是在路上摔了一跤,虽然不像是有筋骨上的重伤,但浑身上下污糟得都是跟泥猴子一般无二。这位让人看不出面目的传令兵,被主帅们的一众亲卫死死的拦在外围。

那名传令兵被远远地拦住,一时心急如焚,高高举着同样泥泞的令旗,向着人群中的主帅大声喊着,“章大帅、韩副帅!小人奉刘指使军令,前来上禀二帅,升龙府南门守将已经开城了,现在官军已经控制了城门,还请二帅速派援军!”

听到吵吵闹闹的喊话,章惇、韩冈一时都没有反应过来,直到那名传令兵又将南门开城的消息高喊着重复了一遍,就没人再拦着他了。

来到几位主帅面前,传令兵行过礼,便将他所带来的消息重新复述了一遍。

韩冈听到了,脑中就像有什么炸开一般,一阵狂喜从心脏传遍全身。一年多近两年的辛苦,如今当真就要到了到头。

看看左右,章惇抿着嘴用力捏着拳头,李宪却是张着嘴发出无声的大笑,只要是听到这个消息的人,都是难以掩饰各自眼中的兴奋,周围的将校士卒,就更没有太多的顾忌,一下子全都欢呼起来。

官军集中在升龙府东侧也不是没有别的打算,也有将城中主力都引过来的想法。官军主力在东门与李常杰决战,而其他四门则交由能够把握时机的合适人选来负责。而眼下的结果,也证明这个选择并没有错。

“还是要小心。”章惇定了定神,从狂喜中恢复神智,“虽说是南门献城了,但只要官军还没有彻底占据升龙府,俘获或是确认倚兰、乾德和李常杰等人的死讯,就不能放松戒备,越是这个时候就越是要小心行事!”

章惇严令要麾下将士冷静下来的一番话,却是因为他自己太过于兴奋而忘了该下达具体的命令,传令兵有点发楞,韩冈在旁喝道:“章帅的吩咐没听到吗?速回南城,让刘冬守紧南城城门便可,至于溪洞诸部,让他们进城去,闹个天翻地覆最好!”他转过来又向章惇提议道,“子厚兄,最好能再派一队人马去南门协防,这样也稳妥一点。”

“当是如此。”章惇终于全然冷静下来,枢密副使毕竟还没到手,他随即从预备队中点起了一个指挥,让他们立刻赶往南城。

“西门、北门要盯紧,以防城中有人潜逃出外。”李宪也插了一句嘴。

韩冈心下笑了一笑。城外围了几万蛮兵,城里的重要人物哪里可能在这时候跑出来?只要出来一个,就会被捉到一个。根本就别想有什么机会逃离升龙府的地界。不过这时候也没必要驳李宪的面子,反正他也只是表现自己的存在感而已。

实际上经过几个月的相处,韩冈也知道李宪本人的军事素养绝对不差,甚至可以说是在一般都能算得上优秀的将领之上。王舜臣、赵隆、李信这一干年轻的将才,谈起天文、地理、兵法、政事来,其实都不如他。

章惇跟韩冈一个想法,又点了两名亲兵,掷下令箭,让他们速去知会北门和西门。转回头来,他厉声大喝:“南门开城,东门这边也得加紧进攻,加快垒土,不能让这里有调兵救援的余暇!”

半刻钟之后,升龙府破城的消息传遍了整个战场,虽然不知道为什么这个时候还要运送土块去城下累积,但正在奔走转运的蛮兵们已经变得一个个脚步如飞,为即将到来的盛宴而雀跃不已。甚至连为敌效力的一众交趾百姓,也主动加快了速度,希望自己能早点获得解脱。

命令都传递了出去,这时候,就只需静等整座城市落入官军的手中。

‘终于……终于算是定了。’

韩冈长舒了一口气。叹气的声音却大的让他吓了一跳,却是身旁的章惇也是在同时长声一叹。

两名主帅对望了一眼,章惇面容疲惫的笑叹了一声,“玉昆,这下总算是定了。”

韩冈深深点头,“的确是全都定了!”

不容易啊!

自从降雨之后,主管营中各项事务的韩冈,一直都是心神不宁。保证南下大军的身体健康,并不是说上去那么容易。今天的攻城,也绝不是表面上这么简单。天时地利全都不在手中,一旦今日攻击受挫,士气短时间肯定恢复不过来,而且只会在渐渐让人难以忍受的气候条件中,越来越低落。

雨季开始后的交趾,空气潮湿得几乎能直接挤出水来,身上的衣服总不见收干,许多人的皮肤上都起了疹子。喝得水也总是带着异味。军中带的用来止泻的黄连在过江之前,只动用了半成而已,但一过江,雨水一至数日间便少了三分之一。而且蛇虫也多了起来,韩冈今天早上就在自己的帐篷里一条近半尺长的蜈蚣,营中被各色毒虫毒蛇咬伤的士兵人数加起来已有两百多。

另外还有军器上的问题,斩马刀就算日日上油,也照样是一天要磨上一遍,盔甲也是一般,神臂弓就不用说了,连霹雳砲上的组装铁件,只几日功夫也都是变得锈迹斑斑。

韩冈在今天的攻城前还在想着,如果能在这两天转移到城中安置,利用干净的水源和饮食,还能勉强保证近万官军的健康和卫生问题不至于恶化得太厉害。要不然也只能选择撤退了,等到明年再卷土重来——这里的自然条件,对北方人实在不利,要是等到拿不起刀枪,再想撤退,可就没有多少机会了。

不过眼下可是最好的结果,听着城中一声接着一声欢呼胜利的号角传来,韩冈就听见身边的章惇说道,“看来,是可以进城了!”

