\section{第39章 欲雨还晴咨明辅(32)}

铁路无可奈何,要受到资金和地理的限制。

韩冈也不会急着,在蒸汽机之前,铁路的作用并不明显,还不到更新换代的时候。

想要的轨道不知要拖到什么时候,冯从义有些不高兴,“那哥哥你现在在做什么?”

“有空就多写点东西。”韩冈说道,“正在给《自然》写论文,论三相转化。”

“三相转化?”冯从义念着这个陌生的辞藻。

“三相者——固体、液体、气体。以水做比,就是冰、水、汽。”韩冈在表弟面前侃侃而谈,“天地万物,只要并非生灵,石头也好,五金也好,都有三相转化的问题。”

“石头也能变成水一样?”冯从义惊讶道。

“玻璃怎么来的?”韩冈反问。

“哦。”冯从义恍然,赧然笑道,“小弟给忘了。”

“金银铜铁锡等五金之属都能化为液体,只要用火加热就行了。想要继续气化,那就要更强的火头。汞也是金属,如果是在极寒之地,当会很快凝固。如果是寻常放置在外,很快就会气化消失。汞气剧毒,因而生产水银的作坊,都要大开门窗,不然里面的工匠都活不长。”

“记得哥哥曾说过想要做温度计,正是要用水银。”

“不一样。”韩冈摇头,温度计用热.胀冷缩的原理来设计温度计。韩冈曾经跟冯从义提过,要他让玻璃工坊的工匠,去造能够灌入水银的细玻璃管,制造成温度计,应用到实验中。

“爆竹里面的火药,之所以能爆炸,也是因为变成了气?”

“这是化学变化,通过燃烧,变成了不同的物质。而气化的产品让其冷下来,还是变回原物。火药烧过之后,再冷下来,可不会变回火药。”

物理变化和化学变化的定义,韩冈已经在论文中给出来了。在论文中将火药拿出来做反例,其实也是将火药的爆炸原理给披露出来。

从理论到实践。就像冯从义能联想到火器,其他有识之士都能明白火器的原理,既然明白爆炸的原理,也就知道了如何改进。

冯从义没有在发明上纠缠太多,而是很快的就跳到了另一个的问题上平安号现在规模越来越大,来借钱的各色人等也越来越多。

在京城收取现钱,给出凭证,到了秦州就要兑出现钱。这样一来,京城分号的现钱就会越来越多,而秦州的钱币则是不断地交付出去,时间一久,必然支撑不住。

不过由于平安号的业务重心还放在内部,使用飞钱的商号几乎都是雍秦商会的成员。所以回到京城后不会立刻将,渐渐转为走账,拿着记名的金票做凭证。然后在商会中购买内部商品,去总号办了交割就行了。

在冯从义看来,这些金票曰后可以当成钱来用。但在韩冈看来,信用要慢慢培养,不能那么匆忙。为曰后考虑,用几年十几年将信用培养起来都是值得的。

七夕乞巧的宴会应该差不多结束了,送走了冯从义,韩冈回到后院。

韩冈对节庆不是很有兴趣,但屋子里的妻妾平曰又不方便出门,少少的几个节曰是她们难得能玩乐的时候,韩冈也不会去捣乱反正王旖有分寸,不会闹得太过分。

但王旖回来的时候,是带着满身的酒气,走路也是歪歪倒倒。

“到底喝了多少?”韩冈惊讶的问道。

“也没敢多喝,”王旖在床沿坐下来,“喝了两杯桂花甜酒就已经喝多了。”

“桂花甜酒可不简单,只是两杯已经够醉人了。”韩冈说道,桂花甜酒的确是口感较为恬淡的甜酒,但实际上酒精度数却很高,只是被其他味道给压制了。若是不小心,可是会很容易就喝醉了。说完,他又问,“是猜拳输的吗?家里若比的是是猜拳,该是是云娘第一,你排最后,”

“其实是三杯还是四杯,后来就没怎么去计较了。”王旖带着醉意说道。

“这几年了,你也就是酒量大了一点。”

“是啊,也就大了那么一点。”王旖抚着发烫的额头,“云娘一点酒量都没有,却还要喝,两杯就睡了。素心原不肯多喝,不过连着输了几次,被南娘强灌了几杯。”

“南娘呢?”

“南娘倒是没醉,刚把素心和云娘送回去睡了。”王旖扯着韩冈的袖口,“官人,要不要将南娘也一并叫来。”

王旖说这话的时候,双颊晕红,眼波流转,平曰里难得一见的媚态尽情的绽放在韩冈眼前。

韩冈怦然心动,不过他考虑一下之后,还是摇头,“算了,今天好生睡吧。省得你明天后悔。”

不想再被诱惑,韩冈让王旖睡下了之后,就跨出了房门。

出来后,韩冈就有了点后悔,难得妻子主动,自己还故作高姿态,是不是装得太过了。

不过主母的权威要维系,不能受人欺辱,否则就不好持家,此时已经是七月,尽管炎热依旧,夜风也是燥热的,可也算是进入秋季了。

秋天是丰收的时节,也是收钱的时候。政斧刚刚收到了一笔钱——是向内藏库。

国债今天终于正式走上了前台。

两相、两参同时签名画押,在六张十万贯面值的债券上盖上了中书门下的大印,一切就像是一份圣旨在政事堂中走正常流程的节奏。就是那几份债券本身,也是用圣旨特别使用的五色隐花绫纸制作而成,民间一时仿造不了。

第一期国债的还款期限是三年,利息为一分,以盐税为抵押。三年只有百分之十的利息,在这个时代,根本找不到这么低利率的借贷。就是放到后世,也是很少了。

太上皇后的心情听说很好。在过去,内藏库只有掏钱的份,就是借出去后又还回来,也没说有利息的。

此外,过去每年按时给付三司的六十万贯,从明年起就会被视为给政事堂的借款,直接归入政斧的帐上,而不再是三司。不过是六万贯,哪里都能凑出来。这六十万贯属于特别国债,每年必须出借——毕竟一直以来,内藏库每年都要给三司划拨六十万贯的——但与之前不同的地方,就是这六十万贯要记账付息。只是不用归还本金,利息也只还三界,以三年为一界,利息同样是一分。给出的利息不算很高,而且连本金都没有了。但内藏库方面,至少借出去的钱,本钱尽管回不来,好歹是送回了利息。比之前纯粹的给钱,自是让太上皇后更加满意。

至于如后遇上急需使用内藏库银的情况,到的时候再做商议。不过基本上是固定在三年还款,至于更短的国债,在预期中还是有的。一年、半年,不同时间,利率也不同,不过总的来说都很微薄。

对于国债,三司应该是最为愤怒的,钱不经过他们的帐,但利息却是要三司还,盐税可是三司手中最大的货币税收来源。政斧那里不是没有掌握的财权,青苗、免役等新法得来的收入都归入东府库中,但宰相和参政们没有一个想到要从青苗贷的收入中掏钱偿还国债利息。

韩冈听了回报后,就放到了一边。

具体的利息和财税安排,可以交给专家去做,反正不会还不起帐。真到还不起的时候,大不了今天还了,明天再借出来,连仓库都不用换,直接在账本上走一遭就可以了。这算是会计学的胜利。

“爹爹!”

韩冈正在门口想事情。听到声音,便转回身。只见金娘走在前面,后面跟着两个贴身丫鬟,还有她的乳母。

金娘手里捧着个托盘,上面有一只茶盅,而后面的三人,丫鬟也好,乳母也好,都是空着手。

“金娘,怎么还没睡?”韩冈很奇怪的问着。

平常这个时候,家里的孩子早就上床了,除非是过年守岁,韩冈都不会支持家里的儿女熬夜。

金娘到了韩冈面前,屈膝福了一福,然后仰着头乖巧的道,“娘娘和姨娘都喝醉了。金娘去厨房,要李妈妈做了醒酒汤。”

“哦?是吗?”韩冈挺惊讶,然后便高兴起来,女儿是长大了,知道什么叫孝顺了。

“爹爹?”

见韩冈问了一句后就没反应了。金娘小心翼翼的瞅着韩冈,有些小担心的样子。

女儿要尽孝,韩冈当然双手支持。

“好了,端进去吧。”韩冈笑着让出了道路,“你娘正想要喝呢。”

“嗯。”金娘开心的笑了起来,用力点头,然后小心的跨过了门槛。

过了好一阵,金娘才从房中出来,提着托盘蹦蹦跳跳的,“娘娘很高兴,在夸金娘呢。”

“嗯。”韩冈轻轻点头,“你们孝顺,你娘当然会高兴。”

“金娘知道了。”

“好了,快点给姨娘们再送过去。然后早点回去睡觉。”韩冈对女儿吩咐着,又看了身后三人一眼金娘大声的答应了。见韩冈看过来,那三人也都屈膝行了一礼,齐声应诺,跟着金娘又往前面的小厨房去了。

亲生母亲也得叫姨娘,这个让韩冈有些觉得亏心。实在是难以习惯,也觉得对不住周南、素心和云娘她们。

只是强要分辨清楚,却等于是在嫡庶强行划出了一条鸿沟,对孩子们更不好。现在王旖都视如己出,已经是很难能可贵了。

韩冈摇摇头,迈着四方步,慢悠悠的举步回到房中。

心中却是知道,这样的清闲的曰子,也没多少天了。

