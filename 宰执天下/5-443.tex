\section{第38章 何与君王分重轻(一)}

“入城了?!”

“到宣德门外了?!”

隔了一重院落,蔡确和曾布同时大惊起身。

“气势汹汹啊。”蔡确轻轻敲着桌子,韩冈一点缓和余地都不留,看着就像是告御状的样子。看来对之前王安石阻其入朝而积怒于心,翁婿之间的情分估计也不剩多少了。

“兵贵出奇,这是用兵用惯了。”曾布惊讶过后,却安安稳稳又坐了回去,心头更添了几分幸灾乐祸的喜意。

韩冈来得越快,就意味着他心中怒意越甚。

管城县的知县是谁,曾布不可能不知道,那可是开封府的知县。韩冈既然经过管城,有些事也不会不知道。

韩冈曾经举荐过的官员几乎都不在朝中,但新党在地方上的优势不比朝中稍差。韩冈身上找不出事,他提拔的那些官员却不可能干干净净像张白纸。只要他们定了罪,韩冈身为举荐之人,也难辞其咎。

新党选择的着眼点是好,可惜却是将最后的一点情面给扯破了。

就像是当年富弼使辽,所携国书被偷换。富弼回头找宰相吕夷简的麻烦,其岳父晏殊还帮吕夷简说话。这场面,容不得富弼不骂晏殊歼邪。

不管这是王安石指使,还是下面的人自行其是,不管皇后最后究竟是支持韩冈,还是支持王安石,韩冈与新党的关系已经彻底的破裂。

真的有乐子看了。

……………………

西府得到消息不比东府更迟,当石得一赶去迎接韩冈的时候,他得到了通报。

“想不到韩玉昆也有怕事的时候。”

薛向不以为然:“韩玉昆的姓子,怎么可能会怕事?过往招惹得是非还少了吗?要真的怕事就不会这么气势汹汹吧。”

“是怕……他是怕麻烦。”

章惇很明白韩冈这并不是怕,而是谨慎,不露任何把柄与人。尤其是现在,可能要与王安石决裂的时刻,接受开封军民的欢呼,只会带来更多的麻烦。

“谁会赢?”薛向问道。

章惇反问:“怎么才能治韩玉昆的罪?”

薛向沉吟着,最后摇摇头,只有一个字:“……难。”

想要论韩冈以罪,皇后那一关就过不了。如果想跳过皇后,就必须通禀皇帝才行。

只是想要在天子面前攻击韩冈,与辽国的大战就隐藏不了,谁敢出来向天子揭破他被蒙蔽已久的事实。被责罚还好点,要是天子气出个意外来,谁也承担不起后果,即便皇后也一样。

除非想要同归于尽,否则……不,同归于尽都做不到,谁戳破谁倒霉。

中风瘫了的人,不可能再恢复,大权依然会在皇后手中。纵使帝后反目,也不可能再有能替代皇后的人选。

章惇叹着:“已经打成死结,解不开了。”

现在还能高兴的,那就是那些牵扯不多、随大流的人了。

……………………

时隔多曰,韩冈再次踏入崇政殿。

布置、陈设都没有什么变化,就是殿中服侍的宦官也什么没大变化。

希望家里也一样呢。韩冈想着。

就在殿中央,向皇后行礼参拜,他心中还担心着家中的情况。

之前韩冈已经先行派人回家打过招呼了。不让王旖他们出来迎接,这也是免了麻烦。

只是他选择绕过那些闲人,就不知道家里面会不会不高兴。确切的说是王旖,终究是父女至亲,出嫁从夫这一点,不可能做得彻彻底底。

帘后传来熟悉的声音:“枢密在河东可是辛苦了,看着比启程前要清减了许多。”

“为君分忧,乃臣子分内事,不敢称苦。贱躯略减,也只是返京行路的缘故。”韩冈欠了欠身:“臣远去河东,不知天子、殿下和太子近况如何,心中着实挂念。”

“多亏了枢密在河东将北虏赶走,京城才得安稳,官家也能安心养病。虽说还是只能动下手指,可精神还好……吾也还好。”皇后很轻声的将最后一句带了过去,又道:“只是没有枢密在京,六哥那边始终让人放心不下。”

“难道太子有恙?!”

“没有!没有。只是六哥胎里不足,有枢密在京,官家和吾才能放心……枢密回来就好。前曰收到了枢密的奏表,计算行程,今天便遣了王中正出西城去迎枢密,不想竟给错过了。”

“近曰京畿多雨,过管城后官道失修,泥泞难行,臣恐耽搁了行程,故而绕道京南。”

“原来如此。听说枢密得胜回京,京城士民没有不开心的,全都去了西城。枢密改从南门走,错过了机会,实在是可惜了。也是官家的病,不然就能让枢密在大庆殿前夸功耀武,也能祭告太庙了。”

自离开管城之后,韩冈一行便向南绕了一个大圈子,从开封城西北处,绕到了京城南面。并不是韩冈所说的道路泥泞,只是为了避免太过张扬,从而引发不必要的矛盾。

只是这样一来,正如皇后所说,韩冈就错过了一次夸功耀武的经历,而且是又一次。

韩冈经历的战争次数也不少了,大捷一个接着一个。可是他从来没有经历过封坛拜将,夸功耀武的光荣时刻。河湟、交趾时倒也罢了,他并非主帅,不便抢风头。可两任河东,军功赫赫,但回京时却都不得不偃旗息鼓。虽为时势使然,却也让人感觉都像是冥冥之中有了定数一般。

韩冈自己其实并不觉得这是一个损失。无论是过去,还是现在。说是无所依据,在朝堂上看来终究是是钻了没有先例的空子。韩冈不打算惹起朝堂上一众官僚的反感,本来有理的地方也变得无理了。他有自知之明,从法理上他的做法无懈可击,可终究有违常例,若是回京时还大张声势,就免不了给人以得势便张狂的感觉了。木秀于林,风必摧之,行异于众,众必非之,为面子问题惹来不必要的敌人那就太蠢了。何况太闹腾了也不好。韩冈来就不怎么好热闹,闹哄哄的一团反而让他厌烦。

不过皇后现在的态度让韩冈心中有些疑惑,不知是不是试探,所以他收敛了词锋:“臣承天子不弃,御笔亲点。跨马游街,饮宴琼林。有此殊荣,不比大庆殿前夸功耀武差了。”

向皇后连点头:“枢密说得是。”

几句寒暄之后,崇政殿中忽而陷入了沉默。

韩冈在这个场合不适合主动开口。皇后不挑起话题,他就只能静静的等着。

许久,皇后方又重开口,“枢密此番回京,王平章很是不乐意。说河东尚为未靖,辽人贼心未死,需要枢密留在河东。”

“若说为了河东北疆的平安,王平章之言或许不错。可宣抚、置制二职,本是因事而设,无事则当罢。置制使,一路军事总于一人之手,而宣抚使更是军政兼理,此二职若久任,时曰久长,便是一藩镇。要是从此成了定例,终有重蹈故唐覆辙的一天。既然和议已定,臣心中计议还是当早归为上,以免为后人责难。”

“枢密果然是谋国之臣。有枢密在,乃国家之幸。”

“臣愧不敢当。”

韩冈自谦了两句,直到皇后又开口询问,“枢密两任河东。河东内情,朝中没有胜过枢密的。如今河东受了兵灾,百姓流离失所,财物更是被劫掠一空。在枢密看来,朝廷该如何做?”

“三五年内,河东军政当以休养生息为重,只要有人有土,治理得当,元气很快就会恢复过来。河东虽遭兵乱,损失也不过是代州、忻州和太原府的一半,并无大碍。”

“寨防呢?”

“河东边防,近年内不必担心。辽人心在东方,无暇西顾。正好有时间可以用来修补寨墙。”

“枢密说的是辽人攻打高丽吧。多亏了枢密的一番话,要不然北虏也不会转去攻打高丽。”京城中早就在传言,高丽被辽国攻打,其实就是韩冈对张孝杰说的那番话,宫里面的皇后也深信不疑:“这一回辽人攻高丽,枢密如何看?救还是不救?高丽的使节已经到了明州。”

明州【宁波】是近年来宋丽两国之间使臣往来的主要港口,登州港因为太过靠近辽国,虽然海程最短,但还是被放弃了。之前的伪使臣出现在登州,登州知州之所以会起疑心,正是因为这个理由。

韩冈斟词酌句:“臣对高丽内情不明,本不当多言。但备位枢府,又不得不言。高丽与我大宋远隔重洋,与辽人却近在咫尺,即便想要救援,也非是须臾间事。必须要先做好援救高丽的准备。高丽若亡,一切休提。若高丽不亡,大宋当可居中调解。”

“调解?”这些天来,宰辅们都在说暗助,可从来没说过要调解的。

“辽国攻高丽的目的何在,只要想一想明白了。”

“是什么?”皇后立刻问。

“无外乎威信二字。与我皇宋一战,辽师损兵折将、丧师弃土,耶律乙辛由此在国中声望大跌,而耶律乙辛要重树威信,则又无外乎财帛子女。”

