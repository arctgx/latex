\section{第12章 锋芒早现意已彰(五)}

“那贼子果然要篡位了!”

向英闪进帐时神情紧张。

十几道目光全都落到了他的脸上。

大宋的国信副使面颊被寒风吹得潮红,急促的喘息着,干咽了口唾沫,又慌张的重复道:“那贼子要篡位了。”

帐中无人惊讶。

不论是在辽国,还是在大宋,有多少人不知道,耶律乙辛要篡位?皇帝杀了两个,太子和太子妃也干掉一对,他将那个不知从哪里找来的小皇帝赶下去,自己坐上辽国皇帝宝座不过是迟早的事。

但每一个人的脸色都更加难看。

就算明知耶律乙辛肯定要篡位,却也没人预料到他会选在近日,让自己撞上。

这是自投罗网。

作为正使的王存脸色灰败,“确实吗?”

向英苦着脸:“下官方才看到一队北兵牵了白马青牛走过去,往那座土台去了。”

青牛白马是契丹祭祀始祖时,必不可少的道具。就像汉家祭祀时所用的太牢、少牢一样。军队开拔时,也会杀青牛刑白马,以此为祭。

现在辽人拉了青牛白马,其实十分正常。不正常的是位置。

今年辽国的冬捺钵依然是在上京道永州的永平淀上。此地距离临潢府不远,土地乃沙质,草木稀少,而地气甚暖,周围有水源,北面又有山峦挡住寒风,适宜作为驻地。所以辽国自立国后冬捺钵便设在此处。而冬捺钵设于此处的另外一个原因,便是在永平淀北侧,离此不远的木叶山上,建有契丹始祖庙。

故老相传,有神人乘白马,自马盂山浮土河东行,有天女驾青牛车由平地松林泛潢河而下。至木叶山,二水合流,相遇为配偶,生八子。其后族属渐盛,分为八部。这位神人,便是契丹始祖奇首可汗,而那位天女,便是他的可敦——契丹语中的皇后。始祖庙分南北两庙,一座供奉奇首可汗,一座则供奉奇首可汗的可敦。庙中还有二圣及八子的塑像。

始祖庙设于此地,辽国皇帝的冬捺钵当然也只会放在这里,等到正旦时,正好可以就近去祭拜始祖。

可木叶山再近也有几十里路,要用青牛白马祭祀始祖,也该直接送过去。

难道说这几天有什么突然发生的要事需要行军出征的,又或者说,过年了,要杀青牛和白马各一匹,来犒赏三军?

若是如此,往那座高台牵过去又是为了什么?

契丹一族没有久远的历史,所谓始祖追溯不了几百年,过去也没有什么禅让,而是直接动手抢。而已经实际上掌握了辽国军政大权的耶律乙辛,想要做皇帝,杀了小皇帝未免太粗糙,禅让就是最好的办法。

高高筑起的土台,从来不是辽人的风俗,在汉人眼中,却是熟悉得很。现在连青牛白马都牵来了,要说那不是禅让台,也要帐中上下肯信。

自进入辽境之后,使团上下就觉得气氛有哪里不对。只是使团里面的官员,都是第一次出使辽国,根本无从分辨。但到了永平淀,拜见了耶律乙辛和辽国幼主,居住在千军万马中间,又怎么可能看不出辽人中的异常,好歹眼睛都不瞎。

“内翰,此事当如何处置?”向英问着正使王存。

大宋出使辽国的使团,无论是正旦使,还是生辰使,都是以一名正使、两名副使为首。副使分文武,文副使必是自厚生司出身——这是近年才形成的制度——以医药通好辽人。之前的几次出使,文副使全都是厚生司的判官,最早是蔡京,继而是吴衍。自辽国回国后,蔡京去了御史台,吴衍晋升为同提举厚生司,之后的两任判官,也都各自加官进爵。

向英出身太后家,在厚生司也只是占个位置,被选入赴辽使团,只是贪慕使辽回京后能得到的好处,另外又对堂兄在河北榷场上的收益眼热,希望有借口能去分上一杯羹。可从来没想过要近距离参观耶律乙辛篡位的大戏。

“不必自己吓自己。且继续看了再说。”王存在叹气之后,也只能这么说。

当来到这里之后,他便感到气氛迥然有异。可即便明知道耶律乙辛就要谋朝篡位了,但他们这群使者,也只能干看着,无法作出任何反应,甚至他们最盼望的,就是辽人上下将他们全都给忘掉。

“季高,辛苦了。”王存对向英道,转头又对另一位副使道,“彝叔,使团中以你最擅兵法,麻烦你去看一看辽人的军势。耶律乙辛若当真动了异心,辽国不免内乱,其麾下大军是否堪战,还要你看一看。”

种建中起身答诺,王存的要求其实是扯淡,又不是打仗,也不是射猎,能看出什么来?想要观察宫卫立营的布置,也得辽人允许自己可以围着捺钵绕上几圈才行。

种建中离开了营帐,身后身前的十一顶帐篷,便是辽国的‘都亭驿’。

外面一圈绳子,括起了方圆百步,这就是日常行动的范围。除非辽人来请,去拜见天子、尚父,或是参加射猎等活动,否则使团中的任何一个人,都不可能走出绳圈之外——有一支千人队护卫、或者说看守着使团,观其旗号是宫分军中的一支。种建中不认为自己能够排除他们的干扰,观察到辽军的虚实。

不过能从大帐中走出来透透气,倒是一件好事。

此处距离御帐有一里路的样子,但金色的大帐,就算隔了五六里也一样显眼。

辽国的朝廷于国中巡游四方,到了驻地之后,便将数千支长枪扎进土里,再用皮索拴住长枪,由此圈出一块地来,在其中立起御帐。

长枪、皮索组成的栅栏外,又有宫卫搭起一圈圈小帐,以作护卫。

数以万计的宫卫,一圈圈的围绕着御帐,千军万马凝成的气势,看起来比起金城汤池还要坚固数分。

种建中向远远地眺望了过去,久经沙场的他,对宫分军的驻地没有太多的感想。只是有一件事让他感叹,那座大帐的主人,过几天就要换人了。

作为副使,种建中负有统帅使团卫队的职责,同时在各项活动中,遇到辽人挑衅时,给予相当的回应。射猎、论武,武臣使节都得有些水准,免得为辽人小觑。不过这一回来辽国,种建中完全没有运用到自己才干的地方,只是按部就班,一步步的北上,抵达永平淀。尽管一路上感觉到了异样,没有使辽经验的他,直到在捺钵中安扎下来后,才察觉到了有什么事即将要发生。

绕着绳索慢慢走了一圈,身后响起了沙沙的脚步声,回头看时,是向英凑近了过来。

“彝叔,看出了什么没有?”向英小声的问着。

虽然他是文官,但向英毕竟是靠太后的关系才得重用。而同为副使的种建中,与韩冈是极亲近的师兄弟,叔父又是贵为太尉的种谔。即便是太后的亲族,向英也不敢对种建中有任何失礼之处,反而有事没事就表示一下亲近之意。

种建中虽不敢与太后家人太过接近,可也不会拒人千里,叹了一声:“就是看出了也没什么能做的。”

“王内翰只知道等,但现在再等下去,可就没好结果了。”向英心急如焚。

大宋的臣子,除非得到朝廷的准许,不可能参与到权臣谋逆的行动中去,不管耶律乙辛本人怎么涂脂抹粉,本质上还是一个篡字。若是他们这几位使节参与了耶律乙辛所谓的禅让大典中,回到京城,朝廷绝不会轻饶。

出使外邦,使臣即便仅仅是说错了一句话,走错了一步路,回到国中都免不了要受到责难。要是参加了耶律乙辛的禅让大典,这辈子就完了。

都是代表大宋的使节,出现在禅让台下,让异国异族的贡使看到了,还以为大宋承认了耶律乙辛谋朝篡位。

“但我等身处狼窝之中,又有什么办法?”种建中摇头,“难道还能阻止耶律乙辛不成?”

“怎么可能阻止,只是怎么躲过这一劫?”

看眼下的架势,说不定这两天就要禅让了。就算不参加禅让大典,等到递交国书,耶律乙辛穿着天子服坐在御榻上,这国书是交还是不交?

最好的办法就是装病,可正副三名使节同时生病,想要耶律乙辛能一笑了之,完全是个奢望。

怎么办?

“直接说不!”种建中只有一个字,“我等国使,耶律乙辛就是做了皇帝也不敢贸然杀戮。”

向英的脸垮了下来,当真这么做了,或许就是被扣下几十年的结果。

朝廷绝不会承认耶律乙辛篡位之举,宋辽是兄弟之国,皇帝之间都有着约定百年的亲戚关系,耶律乙辛篡位上来,是想让太后喊他大伯吗?更重要的是,辽国是大宋承认的帝统,承认了耶律乙辛的篡位,那大宋朝廷当如何自处?这是大是大非的问题,没有哪位臣子敢于触动的纲常大节。

一旦朝廷严辞叱责耶律乙辛,他们这些使节如何能保住自己不成为苏武?
