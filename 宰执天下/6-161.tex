\section{第14章 落落词话映浮光(上)}

船行汴水之上,离开开封城已有二三十里了。

在践行宴上稍稍喝了几口酒,头就有些发晕。端着一杯清茶,王安石便坐在主舱中。

窗口竹帘卷起,暮春的阳光照进舱内,稍稍有点热,不过有河上清风,让人感觉很是舒服。

出京之后,仿佛卸下了心头重担,望着汴水两岸上的垂柳,兴致渐渐高昂起来。

这三个月里,王安石的心情,也已经从愤懑变成了洒脱。

一切都看开了。

回头看看,自己的确是做错了点什么。

本来局面不至于如此。就像韩冈所说,他是以十年为期,不至于这么快便见分晓。

幸好韩冈本身也没有,有章惇在,新法和新学在朝堂上还是有人照料。韩冈暂时也不可能用他的气学,取代新学。

至于其余,王安石已经不想再多想了。

京城的事,就留在京城好了。

窗外,时不时便有一艘船只,与官船交错而过。单独的一两艘,是官船;三五艘成列,多是民船,而一连十艘同样形制的,则是纲船。

当年薛向主持,为了避免监守自盗,将纲船和民船混编,不过自薛向成为叛逆之后,他留下的一些法度不论好坏都被废去,曾经重用的官吏也先后被寻了罪名,或罢职、或治罪,以至于纲运败坏。

王安石曾经听说韩绛、韩冈都曾为此大发雷霆,今日看来,昔日良法的确恢复了一点,只是少了那群干练的官吏,六路发运司还没能恢复到过去的水平。

以韩绛、韩冈的地位,不至于找不到合格的官员来管理,但现在仍未好转,或许是为了修筑京泗铁路在做铺垫。

有了轨道,天下就变了一个模样。河北的轨道修好后,就不用再担心北虏。

尽管之前北方的紧张局面,并没有维持多久,但只要北方还有强敌在,大宋军民的心就不能完全放松下来。

王安石喝一口清茶,收复故土的功劳已经与新党无关,就看韩冈如何去实现他的目标了。

放下茶盏,王安石也一并丢下了所有的烦心事,看着岸上的春光,却没有多少诗兴,想了一想,也不唤人,就自己进内舱把女儿说得那部书给拿了出来。

《九域游记》。

这是女儿王旖送上来的书,一共十卷,一看就知道字数可不少。

只看封面,就知道不是手抄本,才出来的书,竟然已经付梓了。

韩冈这是想要让多少人看他的这部书?

书名很朴实,不知是不是说天下州郡的地理人情。不过要是这一类的内容,就不该被说是小说家言,也不该是佚名了。

随手抽了一卷出来,翻了一页,就看见最右边的一行是‘第十九回,宋公明远赴海外,吴加亮回返故乡’。

王安石一奇,然后摇头皱眉,这个体例没见过。不过估摸着就是说书人一次说得数,就是这么一回。

的确是小说家言,根本就是给说书人的话本,在题目后面应该加个评话二字才对。

放下对体例的琢磨,王安石去看内容,然后又是一皱眉,内容文字完全是白话,的的确确就是话本了。

再放下对文字的看法,他耐着性子继续读了下去。

这一回说是一位姓宋名江字公明的山东士子,在游学江南时,因为怀才不遇,在酒后愤而于店中题了反诗。

‘心在山东身在吴,飘蓬江海漫嗟吁。他时若遂凌云志,敢笑黄巢不丈夫。’

看到这首诗,王安石一声冷笑,是个不安于室的,放在今日,就是张元、吴昊。

不过宋公明被官府抓到之后,只是被县官一番训诫。

书生造反,十年不成,酒喝多了的昏话,谁也不放在心上。

但这宋公明是个有心气的,出来后就对好友吴加亮说要去海外拓殖。

‘朝廷有百万雄师,的确是英雄无用武之地。可想那大海对岸,除去一二港口和农场,便是朝廷兵马不及之处,凭吾胸中十万甲兵,做个不受管束的外藩之王又算得什么难事。’

吴加亮劝他,‘海外之王,可比得上一个神都的城门吏?’

‘只凭一个逍遥自在。’

‘有汽轮船往来于南海之上,移民一日多过一日,即使做了藩王,如何当得长久?’

这番对话除了一个生僻的汽轮船,内中的核心,就是韩冈的海外拓殖之策。

让多余的人口去海外生养,能活下来最好,活不下来,至少也能少一个潜在的反贼。

人不畏死,奈何以死惧之?百姓吃不饱,就是官府的责任。如果只是一时灾荒,就通过赈济帮百姓熬过去,如果的确是田地出产不足,养活不了那么多人口,那就得将其疏散出去。

书中的内容,完完全全体现了韩冈的思想。

看到这里,王安石已经明白了。这部书,大概就是子虚赋、大人先生传那一类说着子虚乌有的故事,然后在其中承载自己观点。不过韩冈采用了与司马相如、阮籍完全不同的体裁。采用话本,让庶民亦能了然,这亦是韩冈一贯的观念。

不过汽轮船是什么?

只看了两三千字,王安石就发现了很多陌生的名词,比如汽轮船,比如后面提到的蒸汽车。

蒸汽车看起来跟汽轮船类似,只是这个名气完全让人看不懂。马车用马拖,牛车用牛拉,蒸汽车,就是用蒸汽来拉。是仙家手段,还是别的什么?

随便翻看了几页,王安石的好奇心渐渐给引起来了。

合上了没头没尾的这一卷,他拿起了摆在最上面的第一卷。

没有跋、没有序,翻开来就是正文。

以回目为题,以诗文开篇。

只是书中的诗句是街头卖诗文的水平,一如既往的差劲。

开篇的故事,说的不是宋江、吴用,而是兰陵县的一名姓史名进的秀才,因兄长游学岭南时亡故,需要将他的棺木迎回家乡,跟刚才的那个要去海外的宋江完全不一样了。

去岭南迎回棺木,开篇就是难事,这让王安石有了兴趣,心道不知是用汽轮船、还是蒸汽车。

于异国他乡病故,如果是火化了还好说,要是将尸身和棺木都运回来,却是千难万难。

韩冈的老师张载,幼时丧父,父亲病死在蜀地任上,他与母亲一起扶灵归乡,出蜀到了横渠之后,就没钱继续走了,只能草草安葬在横渠镇边上。

同样的情况,王安石见了不少。寄放在寺庙里几十年不能回乡的棺木,哪家庙宇都不少。

不过书里面,史进父母还是命他去岭南扶梓而归。

这史进也没有称难,提了行装,别了父母,到了县中,便去车站坐车。

当然是有轨马车,坐上去先到州城,然后再从州城转车南下。在史进和送他的友人对话中,可以看到出现了蒸汽车。

‘自县里到州中,一百八十里地,得入夜才能到。’

‘不知何时可通蒸汽车,届时,半日便能到了。’

看到这两句,王安石一声轻叹。

铁路通到县中,寻常百姓出行,一个白天就能走出近两百里地,即使是骑马也就这个速度了。

韩冈想要做到的,就是这样的情况。

而且还能更快,只要换了那什么蒸汽车。

如果真的能半日两百里,不论天下哪里有了叛乱,五七天内,大军就杀到了。试问谁敢叛?

可惜……不知要多久才能实现。

‘快走了,快走了,再上一人就要走了!’

到了车站,在车主的招呼下,史进很顺利的上了车,在最后一节车厢里坐了下来。

在史进与同车之人的对话中,王安石又发现了几个陌生的名词——神都,顺天府

神都是洛阳的别称,不过东京开封府,又名汴梁、汴州、大梁,也有文章称为神京的。

但顺天府是哪里?

书中说是兰陵北面。兰陵县古有今无,如今只有丞县,不过王安石记得还有一个兰陵镇。

或许是应天府改名?

王安石知道韩冈不想惹麻烦,所以故意曲笔。

到现在为止,他连个朝代都没提。

提到天子,也就是说了一句‘如今圣天子在位’,另外还有一个泰康三年的年号。

这些都是枝节了,重要的还是小说的内容。

的确是小说家言,所以韩冈连名都没留,但看着的确有趣。

韩冈这是立了一个样子,告诉世人,他将会让大宋变成一个什么样的国家。

不过不是冷硬的文字,而是让人饶有兴味的话本,而且多有枝节。

比如一开始史进要远出郡外,在坐车前先去县中拿了关防路引,当时县中正在断案,一名县学中的学生写了一部有伤风化的话本,在县衙中被斥责,逐出了县学。

扶灵事急,却加一缓笔,让这话本显得有肉有骨。乍看是无关紧要的情节,却让文章增色不少。

至于上车后,描写更是精道。

脚下踩着货担、见人就奉承,是寻常走家串户、今日去州中置办货物的游商;

十五六岁,紧紧抱着包裹,不言不语,只啃着冷硬的炊饼,这是初次离家、要去州中寻工的小儿;

就着烧酒,啃着油纸包的烧鸡,露着圆滚滚的肚皮,满头满脸热津津的油汗,这是要去邻县收租的和尚;

坐在史进对面,高谈阔论,让史进畏而缩足,却把郁郁乎文哉说成是都都平丈我,牛头不对马嘴,是自称要去州中拜见做知州的座师的士人。

史进问那士人,‘澹台灭明是一个人,两个人。’

‘二人。’

‘尧舜是一人、两人。’

‘自是一人。’

‘且容小弟伸伸脚。’

看到这一段,王安石也撑不住笑了起来。

真是那等不学无术、却又拿着书本吓唬百姓的那等士人的嘴脸写得活了。

不是生长自民间,见惯了市井百态,写不出如此文字。

而且那个和尚,也是写得绝了。模样似盗匪,酒肉不离身,满口乡下土财主的口气偏要加一句阿弥陀佛。

想不到这世间还有此等人物!更想不到,文章还有这种写法。

不知不觉间,王安石已经沉浸了进去,浑忘了时间。

