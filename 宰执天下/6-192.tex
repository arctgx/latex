\section{第21章 欲寻佳木归圣众(12)}

不能为朝廷去贼,不能为太后辩奸,臣实有过。’

沈括低头看着笏板。

他知道,舒亶这句话后,肯定有许多人的视线在自己身上打转,视线的主人多半也都是在须髯下藏着讥讽的笑容,笑他贪婪,笑他不自量力。

沈括早料到这一回的廷推,必有波折。此前两次廷推,韩冈都没有公开表态,世人都知道他选不上,也就没有引来太多御史的关注。但这一回,因为轨道之功,韩冈出面支持,仿佛捅了马蜂窝,太后那边又有成见,派了王中正去巡查,更是火上浇油。

这几天来,沈括光是听说上表参劾自己的言官,已经占去了总数的一多半。厚厚的弹章在御案上堆得老高,太后会怎么样看?

没有了太后的选择,空有韩冈的支持,又能顶得了什么事?

更别说现在龚原、杨畏、舒亶,一个个都出来了,看这阵势,是打算连廷推都不让自己参加了。

不过,方才太后训斥杨畏、龚原,让沈括心中多了几分期待。他悄悄侧过脸来,用眼角的余光观察着台陛之上的动静。

向太后的脸色在听了舒亶的话后,更难看了两分。

她冷眼看着舒亶,明着说反话,仿佛在斥责前面的龚原、杨畏,实则却是在攻击他人。

类似的场面她见得太多了,这些人,做了台谏官之后,仿佛就不会正正常常说话了,总要拐弯抹角,实际上呢,还不是党同伐异。

“哦?那依中丞的看法,朝堂中谁是奸佞?苏相公、韩相公,还是章枢密?”

向太后的话中,分明满是怒意,殿中一片寂静,不闻一声。

太后怒气勃发的回应,舒亶一人在殿中央承受着,不见有丝毫慌乱。

“回陛下,御史台中,臣之属僚,多有此辈。”

出乎意料的回答,殿中一阵骚动。

沈括身子晃了一晃,抬起头来,呆然望着殿中的舒亶。龚原、杨畏也都愣住了,完全没有反应过来。

‘说错词了吧?’

若不是身在御前,必定会有人叫出声来。

不是在批沈括,以及沈括背后的韩冈吗?为何舒亶会将炮口返身对准御史台?

李格非也差点叫出声来。

舒亶此言一出,分明是要将台谏上下清洗一遍。龚原、杨畏他们到底做了什么,为什么会让章惇都下定决心抛弃他们?!

这时候,李格非方才回想起来,在这两天御史台的骚动中,舒亶这位一台之长,似乎消失了踪迹一般,完全没有出面来控制局势。

难道是陷阱吗?

他望向班列的最前方。

站在他的位置,只能看见西班的反应。

曾孝宽神色惊异,但旁边的章惇却是面无异色,仿佛一切的变化都在他的预料之内。

他是什么时候与韩冈联手起来的?!

李格非心中惊惧,若章惇与韩冈联合,之前还能利用两党之间嫌隙而勉强存身的旧党孑遗,这下子在朝堂上怕是没有立足之地了。

但他立刻就醒悟过来,只是宰相之位上的争斗,韩冈和章惇就不可能并肩携手。而且御史台的主力是新党,韩冈基本上没有插手台谏的任免,章惇根本没有必要为了迎合韩冈而自毁手脚。

一瞬间,李格非的心中已经有了答案,暗自庆幸起来,幸好自己没有趟浑水,否则,这一次绝难讨好。

太后亦是惊奇不已,“御史台中有奸贼?”

“不错!”舒亶仰头道:“沈括,反复小人,惟其有才,先帝用之,陛下用之。如今见功于社稷,足见先帝与陛下用人之明。惟其品行卑下,纵有殊勋,亦不当委以宰辅之任。今日廷推上,臣绝不会推举沈括。但廷推是朝堂大事,岂能横加干扰?沈括是否委以宰辅之寄,自当在廷推来决定。且为陛下拾遗补缺,裨赞朝廷方是言官之任。窥伺上意,掇拾臣下短长,以图幸进,岂是言官当为?故而臣言,御史台中多有奸佞。”

龚原依然仿佛雕像,舒亶的反戈一击,猝然而来,他的头脑如同被卷进了飓风,天旋地转。

杨畏则及时的从混乱中反应过来,不顾殿中的礼仪,大声叫道:“陛下!舒亶身为御史中丞,却迎逢宰相,罔顾圣恩,陛下明见,可知朝中奸佞乃是何人?”

杨畏满怀期待,盼望有人紧跟着自己发难。御史中丞竟然背叛了御史台,甚至攻击台中御史多为奸佞。这是捅马蜂窝,怎么可能没人出来一起反驳?

但殿中静静的,寂静仿佛在嘲笑杨畏的幻想。

头脑中的混乱或许已经平息,但观望之意却浮上了心头,没有绝对的把握,御史中丞为何要攻击御史们,明知已经掉进了陷阱,还有谁会轻举妄动?

太后也没有理会杨畏:“舒卿说台中有奸,苏相公,你怎么看?”

苏颂淡淡定定,朝堂上幻变迷离,他过去见得多了。

听到太后垂问,随即便出班道:“陛下,以臣之愚见,奸佞二字极重,当就事而论,不当妄言——舒亶、龚原、杨畏,所论皆有失。”

苏颂的发言,稍稍缓和了一下气氛,至少没有方才那么剑拔弩张。李格非吐出了憋在胸口的一口气,这是要做和事老吗?

太后也在问,却不是息事宁人:“龚原、杨畏二人方才说沈括,相公是另有看法?”

沈括悚然一惊,紧紧盯着苏颂。

苏颂道:“沈括品行的确难孚众望,但廷推既定,材与不材,当由陛下与重臣在廷推上共定,非是一二小臣可以干扰。待沈括就任之后,监察审视,方是御史的权责。”

李格非微微皱起了眉头。御史无事不可论,但苏颂的话若是成立,那么日后如果遇上了廷推,御史就不能在尘埃落定之前再有议论。

不用说,这必是秉持了韩冈的心意,在此维护廷推的威严。

“相公言之有理。”

太后的赞许从帘幕后传来,杨畏的脸色阵青阵白,却没有撞阶自辩的勇气。

“韩相公,你如何看?”

问过了苏颂,向太后又向另一位宰相征询意见。

韩冈徐步出班,他正等着向太后的垂问。

这一次的廷推,他完全没有担心过。不说他之前的安排,只为了沈括手上的差事,太后也不会允许有任何意外。说服她容忍沈括的,韩冈不觉得仅仅是自己的言辞,更多的应该是对铁路的需要。

仅仅是一条京泗铁路,已经给朝廷带来了天大的好处。原本从汴水北上的民船,征收不到多少税入,但换成铁路就不一样了,什么货物也逃不过。而且汴水缓而铁路疾,等到整条铁路运转磨合得差不多了之后,除了纲运之外,还能运送更多的南北货物,运力远胜一年有近一半时间要断流的汴水。

沈括这样的人才,无论在政治上犯了多少蠢事,只要朝廷还离不开他,他就不可能被一群御史给打倒。

现在大局已定,顺手推上一把,韩冈岂会吝惜气力?

来到苏颂身侧,韩冈躬了躬身,道:“昔年御史,非一任知县,不得入台。积年亲民,能知上下情弊,又能通达人情,故而可以裨补时政,查奸防阙。而如今御史,却常年居于京府,并无半点历练,不知下情,凡事纯凭己意猜度,故而行事每多荒谬。”

韩冈的话,比苏颂更加尖刻。只有嘴而已,韩冈只差没明说了。

“相公说得是,总有这么一般人,不知做事的苦,爱挑别人的刺,可到了自己做事,却是一塌糊涂。”向太后冷笑着,“既然台谏都上了弹章,说沈括做得这不好,那不好;那就去修轨道去,看看你们能做得怎么样!”

苏颂、韩冈,杨畏、龚原同时变了颜色,理由自然绝不相同。

“陛下,此事万万不可。”韩冈连忙道:“铁路乃国之命脉,不选能吏用事,却以舌辩之士为官。若事败,此等人死不足惜,但损失难以胜计,日后弥补起来,苦的可又会是缴纳税赋的百姓。”

苏颂亦道:“汉武帝时,匈奴请和亲。博士狄山以和亲为便,御史大夫陈汤则称其是‘愚儒无知’。狄山攻劾陈汤,武帝为之怒,问狄山:‘吾使生居一郡,能无使虏入盗乎?’山曰:‘不能。’再问:‘居一县可乎?’对曰:‘不能。’武帝复曰:‘居一寨可乎?’山不得已,曰:‘能。’就任后不及月余,便为匈奴斩其头而去。如龚原辈,便如狄山,百无一能,唯有口舌。今使其监理修造,若事败,难道要斩其头而去?”

苏颂、韩冈,严辞反对,向太后也不好坚持,点头道:“相公说得也是。以二位相公之见,当如何处置?”

苏颂、韩冈对视了一眼,韩冈便朗声道:“风闻奏事,台谏之权,论劾朝臣,亦是分内之事。唯龚原、杨畏二人,阻挠廷推,不可不惩,然此事未酿恶果,也难重惩。以臣之见,可去西京御史台任职。罚俸等事,可依例而行。”

龚原、杨畏面色如土,全然不见方才当殿弹劾沈括的威风。

这两年,秉政的韩冈、章惇将洛阳交给了旧党,大多数的知县都是旧党中人,只有京西北路转运使等寥寥几个监司位置,是新党,而韩冈门下,更是远离。让两名新党成员去西京御史台,盯着旧党官僚,两边都别想睡好觉。相较而来,龚原、杨畏更加危险。要么叛投旧党,要么就是众矢之的,绝难有任何好下场。

太后却觉得不够:“去西京御史台?只龚原、杨畏二人?”

韩冈听得出来,太后似乎对御史台近日的弹章骚扰厌烦透顶,不过将其他御史送去西京,并不是很合适。

“若陛下认为御史台近日所论无理,可事后与御史中丞、知杂事及翰林学士共议。台谏之任,非不得已,宰辅不当议论。”

“也罢。”只听得太后道,“就依相公,此事等廷推后再说,也别耽搁了。”
