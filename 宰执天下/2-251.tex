\section{第36章 万众袭远似火焚(七)}

杀良冒功之事说说也就罢了,毕竟不是今天的正题。

韩冈到了狄道城,安排下秦凤军后,片刻也不歇息,就赶来康乐寨,也不是来讨论如何杜绝杀良冒功的问题的。

“经略,大军自过洮水,究竟带了几天的口粮。”韩冈问着王韶,军粮补给才是能让他如此不辞辛劳的关键。

自度过洮水后,宋军连克两座寨堡,一口气冲出了六十里。这就跟当初种谔从绥德城出发,攻打罗兀城的冒险没有什么两样。虽然眼下的补给线看着稳定,可一旦木征反扑过来,这条粮道的安全,可就是个大问题了——不只是头痛那么简单。

“九天。”王韶的话证明了在他指挥下,实际情况并没有偏离之前与韩冈一起商讨的计划。

可韩冈疑惑起来:“那怎么下官到了狄道后,王都知说经略你催着要快一点把粮秣运上去。”

“的确是有让狄道运粮过来。计划中不是要以康乐寨为兵站吗?”高遵裕微微皱眉,“就不知王中正那个阉人是怎么听得,就会大惊小怪。”

王中正因为前次在熙州、在渭源分到的功劳,受了不少赏赐,甚至还跟高遵裕一起,多了一个带御器械的加衔。这个相当于天子宿卫的贵重职衔,本是代表了天子的恩宠,但跟一个阉人同时得授,使得高遵裕在私下里对王中正没有半句好话。

虽然王中正在兵事上没有什么才能,但他至少在熙河这里谨守本分,比起历史上的那些监军好得太多。韩冈觉得高遵裕说得就有些过头了。

但他今次听了王中正的话,便急急地跑过来的确有些冤。自嘲的笑了笑,道:“当川堡和康乐寨都要改为兵站,但眼下的兵力不知够不够?”

从王韶现在手上兵力来说,如果将两座刚刚夺下来的寨堡都改成兵站,可能就不足以留下足够分派的军力了。至少在泾原军赶来之前,光靠秦凤军和熙河军的力量并不够。

熙河今次出兵总计七千,秦凤的第一批是万人,接下来还有四千,要跟泾原的一万人马一起过来。总计三万军势,要在攻下河州的同时,抵挡住来自兰州的禹臧花麻甚至党项人,并且守住各条道路上的兵站,还是很吃紧的。比起几个月前攻打熙州时,路程上河州远了有两三百里,要补上这一段的缺口,人力、物力的消耗绝不是个小数目。

“洮西这一带的蕃部前几个月已经被木征和我们各自清理了一遍,没有多少剩下的,可以少放点守军,不用怕有人在中间劫道。”

“但康乐寨到当川堡,当川堡往珂诺堡,这两段路,守军和民伕可都少不得。”王舜臣不知何时进来了,他也算是有头有脸,在军议上的发言权只比苗授稍低。只听他插话道:“这一程路末将已经听回来的人说了,马车肯定上不去。只能使用独轮车,而且是两个人一推一拉才能,要不然就是要牲畜。”

韩冈道:“还是照预定计划,先下珂诺堡,回头从河谷中走。”

珂诺堡和香子城其实都在洮水支流边。洮水是自南向北汇入黄河,如果从狄道向洮水下游走上五十里,就是流经珂诺堡、香子城的支流汇入洮水的地方。再从交汇处向西上溯,很快就能抵达珂诺堡。

可无论是王韶还是韩冈都不会选择走上这条河谷道。就跟他们去年在攻打熙州时,没有选择抹邦山、竹牛岭的南线,而走得的鸟鼠山北线是同样的情况。

从地图或沙盘上看,如果忽略掉河道、道路上避免不了的蜿蜒曲折,只取主线的方向。河谷道和眼下官军要控制的山道,就大略组成了一个等腰直角三角形。河谷道因为是从洮水主流转向支流,就形成了三角形的两条腰,而山道则是底边。

起始地和目的地相同,行程却差了近一倍。平坦而悠长的河谷道路,是和平时期最为便捷的运输通道,但在战时,却是危机四伏,随时可能会被谷地两侧埋伏的敌军给切断。相反地,路程更短、地势更为险要的山路,才是更为稳妥、须先一步控制在手中的战略通道。

所以想顺利的用马车来运送粮草,还是等到攻下珂诺堡,再回头打通河谷道也不迟。

“那到底让谁来攻打珂诺堡?”王韶问着,看着韩冈。

王舜臣将胸一挺,他巴巴跑过来可不是就为了此事。

但高遵裕都没理他,“赵隆带的是选锋,康乐寨、当川堡都是他们攻下的。只是眼下连下两城,已经失去了锐气。”他转过来对韩冈道:“玉昆你已经征发了广锐军的那一队将校是吧?他们现在是不是在狄道城?”

‘难道已经商量过了。’韩冈有了点疑心,“……下官带在身边的是乡兵弓箭手而已。”

“玉昆,”王韶抬起眼,眼神沉重,“别舍不得,能用就多用。”

‘果然!’

韩冈知道,王韶、高遵裕,乃至朝堂诸公都是想着尽量早点把他们这群叛贼中的首领给消耗掉,一死百了,所有人都不用再担心他们的事。可韩冈跟刘源他们来往得多,完全没有这样的想法。

他故意装作不知道王韶和高遵裕的心思,轻笑道:“但他们立得功劳太多,犒赏起来可就麻烦了。”

“朝廷不会吝惜一点田地和银绢的。”高遵裕同样深深看了韩冈一眼,“玉昆你可以放心。”

刘源他们在渭源的表现,压倒了熙河军中每一个指挥。可越是光芒四射,宋廷对他们这群叛将的顾忌就越深。能叛一次,就能叛第二次,下面士卒或许是有着各种各样逼不得已的原因,但上面的将校可就找不到多少借口了。

王韶和高遵裕都不希望有人认为他们对这群叛将太过看顾,这关系到他们的前途。

韩冈心中暗叹,看起来是很难正面保住刘源他们了。对于王韶和高遵裕心中的不快,他没有多解释,急急辩解自己对这群叛将并无看顾之意,那反而是着相了。

“玉昆你回去知会刘源,让他来我这里。”王韶自觉前面口气硬了点,缓和气氛似的说着话,“要以快打快,眼下停步不得。攻下珂诺堡后,就有个休整全军时间,然后在河州城的门户香子城下与木征决战。相信木征是不敢把香子城都让给官军的。”

………………

当韩冈找过来的时候,刘源等一众旧日的广锐将校正在营地中安安静静的吃饭。

没有人敢打扰他们。扎营之后,非得上命,各营之间不得走动。否则就是一顿军棍,甚至就是军法从事。但在营地中巡逻的卫兵们,在经过他们这一片时,都忍不住要好奇的看上两眼。

刘源等两百五十人依然坐得很安稳。吃得好,睡得好,谁会在乎外面的人怎么看自己?

韩冈进去之前,曾经吩咐下面的亲兵为他们找块营盘歇息下来,而他的命令被百分之两百的完成了。

韩冈手下的亲兵,有传言说他们各个都精通急救之术。刘源等人实际看见的这一位,虽然没有表现他医术的机会,但他不仅仅帮着定下了今夜的驻地,还顺便将晚餐一起让人给准备好了,甚至还弄了两坛酒来。这办事的手段,算得上是出色了。

刘源想着,跟着治事之才出了名的韩机宜久了,的确是学到了几分下来。

今夜的饭菜有酒有肉,而且前面从陇西往狄道城开进过来时,在几个兵站中,吃到的热饭热菜也同样是不缺荤腥。不仅仅是他们这些广锐将校,整整一万人马的秦凤军,再加上两千多、近三千的民伕,他们在几处兵站歇下来的时候,都是吃上了带着荤腥的热饭热菜。

这件事说起来很简单,而且酒就一口,肉也就那么一块。但刘源毕竟从军已久,知道这番布置有多么难得。能让上万人都吃上带肉的热饭菜,要准备多少柴薪,多少牲畜,皆是个惊人的数字。但在韩冈的预先安排下,却是一点不漏的给完成了。

——这个‘预先’,其实最是难得。

一路过来,刘源就跟在韩冈身边,根本就没看到他费神费力的去安排大军的食宿,所有的事不带吩咐,都有下面的人完成。听说韩冈写了一套有关兵站的规条,就跟传说中让他一举得官的疗养院规条一样,什么事只要照章处理就行了,只是刘源无缘得见。

见到韩冈过来,所有的人齐刷刷的站起身,就算没有注意到的,也都是被旁边人提醒或是直接拉扯起来。

就算在吃饭的时候,手无寸铁、身无胄介,但两百多号人站在一起,便是一股千军万马临阵时的气势。让韩冈不由暗叹,果然是一群熊罴虎狼,只是因事蜷伏起爪牙罢了,也难怪王韶、高遵裕这般忌惮。

韩冈示意他们继续吃饭,然后走到刘源身边,似是平淡的对他和他的副手说了一句,“王经略点了你们的名。”

刘源的回答简单直接,就是一句反问:“是去哪里?”

“珂诺堡。”

