\section{第27章 更化同风期全盛(上)}

最后一批南下云南的流人,已经坐上西行的列车。

三千人充实云南,仍在动荡中的新疆土,晃动的幅度也会小上一些了。

在韩冈看来,这一次的清洗京师的行动,是一个新时代的标志。

人力资源,在如今的社会认识中,已经变得极端的重要。为了充实新夺取的边疆,想方设法移民充实,成了社会主流的认识。物尽天择的理论,也渐渐深入人心。

故而以安养为名的法令,便在此时正式颁布于众,推行天下。

京师的乞丐被流放云南,以此为开端,天下各路,千百城镇的乞丐都将成为过街的老鼠,成为被捕捉的对象。云南,广南、西域,每一处需要移民的地方,都将是流放乞丐的场所。

同时安养法,也成了加大流刑施行范围的法律依据。

依照刑统,五刑之中,笞刑、杖刑、徒刑之后,方是流刑,只比死刑轻上一级,而韩冈所希望的,就是小偷小摸,只要被抓住,也要判一个流放,另一方面,则禁止笞、杖之类,会毁伤身体的肉刑。

在安养法出台之前,各地所判处的流刑,绝大多数没有依照刑统和编敇,故而名不正言不顺。而安养法施行之后,窃盗之贼被送去云南,便不再是流放,而是安养——朝廷怜其身无分文以赡自身,不得已而行窃盗之事,故此依照安养法所定,将其交付边疆,分配土地,让其复为良民。

就如数百年后,极西之地的岛国,将国中的罪犯大批送往海外领地,百年之后,岛国国势大衰,但岛国的苗裔却还是占据了更大的几片国土。韩冈想要达到的目标,正是如此。

“今年年内,至少能有两万人抵达云南、广西两路。”

韩冈与章惇对坐在家中,很有几分欣喜的向他说着。

“玉昆你就不怕他们作乱?”章惇问道。

“要是盐枭我还会担些心思,如今只是一群乞丐,就算给了他们弓刀,他们还能揭竿而起不成?”

军事训练丝毫也无,有声望的头领几乎都被杀了个干净,人心不齐,又身在险地,不依附官军生活,还能做什么?

“希望玉昆你能说中。”章惇想了一想,“军巡铺那边你打算怎么做?”

“开封城的军巡铺肯定要大改,但绝对不是撤除。”

光靠衙役、快手、弓手,根本不可能维护城中安全。调遣禁军维护城中治安,一开始是不得已的安排,到了如今,已经是必不可少。

但这么多年来,军巡体系已经越来越难以满足京城中治安的需要。

军巡铺的巡卒们,在满城搜捕乞丐的同时,闹出的那些烂事,让韩冈脸上毫无光彩。虽然用了更大的声音遮掩过去,也放弃了追究,但这不代表他韩冈不会事后弥补。

“是不是打算成立新衙门?”

“新衙门?”韩冈笑着摇头,他的确动过这样的想法,连名字都考虑过,市容管理或是城管?

若当真有这么一支队伍,的确很有趣。不过只是为了有趣,就在军巡系统之外,再增设个一个衙门,韩冈觉得暂时没有那个必要。且他正准备将军巡铺和潜火铺合并起来,交给合适的人去管理,又怎么会再多开一个衙门。

“这可是子厚兄你的差事,办新衙门也要子厚兄你来考虑。”韩冈直接推给了章惇。

章惇的脸上是自矜的笑容,“现在还不是。”

“就是日后有所变动,这些事也还是得着落在子厚兄你的头上。”

章惇在两府待了有十年了,不过只要朝廷的大局不变,就不会有大的变化。所谓的变动,就是从西走到东而已。

“尚无定论。”章惇还是摇头。

“算了,换个话题。”韩冈不逼着章惇了,“太皇太后的谥号也该定下了。”

章惇听了,就感觉头疼起来。

之前向太后曾经想过,不给太皇太后上谥号,甚至不让她与英宗合葬。

但苏颂领头,宰辅们一阵苦劝,才把太后劝住。

向太后虽然对她的姑姑衔之入骨,但也不得不承认臣子们说得有道理。这么几年都忍下来了,对太皇太后礼数就没怎么缺过,已经是最后一步了,难道要功亏一篑不成?

英宗皇帝只有一个皇后,先帝更是太皇太后肚子里出来的,怎么可能在礼数上欠缺太多?

所以依然是合葬,谥号也交给太常礼院来拟定。

真宗的刘皇后,谥号是章献明肃,仁宗的曹皇后,谥号是慈圣光献,现在的太皇太后的情况太特殊,谥号就不免让人费神了。

按照最低标准,只要在出殡前将谥号议定就够了。但实际上,太常礼院不可能将事情拖到那么后面,过去拟定谥号,甚至庙号,都是几天之内就交上来。太常礼院接到这份差事后,一直就没个回信。

“这件事,子厚兄你如何看?”

“在太后面前我已经说过了。臣子议天子谥,尚不为君父隐,桓、灵可证。太皇太后所作所为,人所共知。其传,秉笔直书,其谥,依实而论。”

“这样啊。”

“玉昆,这句话你问过几个人了。”

“除了子容相公和子厚兄你,其他人还没问过……”

“要是问了,大概会跟国子监一样吧,两边打起来吧。”

韩冈摇头笑,其实没有章惇说得那么恐怖,国子监打起来次数并不多。

国子监中,有气学和新学两派,各执一端,每日相互攻讦不休。尽管讲师几乎都是新学成员,可气学如野草一般,在荒野之地茁壮成长。当然,论起势力高下,自是新学一派更占优势。但有苏、韩两宰相把持朝政,气学人数虽寡,却也没有哪个老师敢用手上的权力去打压他们。只是国子监是新学的自留地,所以最后科举,韩冈多也会设法多夺几个名额,

“国子监也不是没有人。”韩冈猝然问道,“子厚兄,你可知道秦少游?”

“秦少游?”章惇一时茫然,难道是名人?但他所认识的秦姓的名人中,没秦少游这个人。

“‘山抹微云’。”韩冈提示道。

章惇登时恍然:“‘岂在朝朝暮暮’的秦观?他不是字太虚吗?”

“听他说是前两年改的。”

“‘务外游不如务内观’?”

这是《列子·仲尼篇》中的一句,秦观的字与名正好都在其中。名字出自子部,章惇之博学,

韩冈摇头,“他自陈是欲学马少游,故而改太虚为少游。”

汉伏波将军马援的堂弟,劝告志向远大的马援时,曾留下一段名言,‘士生一世,但取衣食裁足,乘下泽车,御款段马,为郡掾史,守坟墓,乡里称善人,斯可矣。致求盈馀,但自苦尔。’——士人一生,吃饱穿暖,有车有马,守乡为吏,造福乡里,便可算是圆满了,若是追究更多,只是自寻苦恼。

独善其身的想法,在自觉不遇的士人心目中,有着很强的共鸣。秦观屡考不中,又受连累而不得科举,年届四旬仍只能在国子监中游学,虽然说已经得到了韩冈的看重,可在少年即闻名乡里,长成之后更以文学知名的秦观而言,如今的境遇,岂能没有怀才不遇的无奈。

“太虚为天,以观天为名字,心不可谓不小,如今到底是知道自己是何人了。当初他投于子瞻门下,吾也曾与他见过几面,还得到他的几部兵书。”

“如何?”

韩冈问的时候,已经有了答案。

果不其然,章惇呵呵冷笑,“狗屁不通。”

看了几部兵书,就打算指点江山的士人太多太多,而能沉下心来做实事,十个里面也没一个。诸葛亮光会隆中对,能成为一代名相、陪祀武庙吗?章惇一直都不待见这种只有嘴皮子的文人,说话也刻薄得很。

“《孙武子》《战国策》害人不浅。”韩冈轻叹,“心比天高,命比纸薄。所以如今不穷太虚,只愿为少游了。”

章惇没有半点同情:“装可怜吗?”

“他的两个弟弟,一字少仪,一字少章。”

章惇顿时哈哈大笑起来。

少游二字,与其兄弟表字首字相同,而太虚就是完全不同的类型了。真要细推敲,说不定少游才是他被起名时就定下来的表字,而太虚则是他长大后自取,如今日渐日蹙,知道了何为现实,故而改回了长辈所赠表字。

秦观拿着旧表字在韩冈面前装可怜,没想到一下子就穿帮了。

章惇摇着头,为秦观的坏运气而乐不可支,“他大概不知道玉昆你一贯是求真求实的脾气。”

或许秦观只是真的心灰意冷才改了表字,而不是章惇和韩冈想的那种情况。但他和章惇这种人,凡事都会往坏处想,事也好,人也好,皆是如此。这是多年来不得不养成的习惯,也是实际的需要。

“左右我评价人,是看他做而不是听他说,也没什么影响。”韩冈没有对秦观表示太多的反感。

“怎么,入了玉昆你眼缘了?”

章惇起了好奇心,真要说起来,对文学之士不假辞色的毛病,固然有他自己自傲的一面,但更多的还是从韩冈那边染上的。

韩冈当年都不愿与苏轼结交,更视周邦彦、贺铸等才子如无物,现在怎么会对秦观另眼相看。

“秦观他作兵书,我不曾见识。诗词近年变了不少,很有几篇能流传千古,我于诗词之道也不甚了了,不敢妄作评价。”

章惇笑笑,不说话。不懂诗词还能说秦观的词流传千古。要是懂了又会是什么情况?

“只是秦观他也努力,前日将如何养蚕写了书。就叫《蚕书》。”

“写得如何?”这次轮到章惇相问。

“有心是好事,也是难得了。”

秦观能写下《蚕书》一篇,的确是很难得了注1。但如果以论文的要求而言,他写的未免空泛了一点,缺乏足够的细节来让人研究。所以秦观给《自然》投了三次稿,前两次都给否定了,第三次投稿,还是韩冈看在秦观本人的代表意义上,才放了行——不过还是先找人好好将论文改了一番,才发表出来。

“看来他还是去学柳三变卧花眠柳比较合适。论文需要的平实和缜密,不是写丁香笑吐娇无限的笔能写出来的。”

“日渐日新,得许人改正才是。或许三年之后,他就能让子厚兄你刮目相看。”

注1:真实的历史上,秦观也的确写过《蚕书》,是为如今研究古代养蚕业的第一手资料。
