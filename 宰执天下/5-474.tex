\section{第39章 欲雨还晴咨明辅(三)}

向皇后依然在福宁殿内。

如今国用不足,想要大兴土木,为赵顼兴修殿宇并不合适。之前向皇后与宰辅们讨论过,决定将旧宫舍改个名字,换块牌匾就可以了。

初步定下来是大内西北角的睿思殿。坤宁宫旁边的睿思殿本是内书阁,赵顼偶尔过来读书,顺便睡个午觉。熙宁八年为了保证藏书的安全,还稍稍翻修了一下。在大内诸殿阁中,算是比较新的一间了。

不过选择睿思殿改名圣寿宫,并不是当真让赵顼住进去。新晋的太上皇由于重病的关系,不宜随意移动,换一个不熟悉的宫舍,对病情或许会带来不利的影响。

故而天子寝宫福宁殿,太上皇赵顼继续居住。新任皇帝赵煦,向皇后则是打算留在身边亲自照管,让他一起住在坤宁宫。等到赵煦诚仁,再搬去福宁殿去居住。

只是现在,向皇后已经开始怀疑自己的决定究竟是对还是错。

赵煦太聪明,只是还小,不会作伪。经过了昨夜的事后,态度明显的就冷淡了。

赵煦的生母朱妃,不对,现在得称太上贤妃了。今天就能让宰辅撰写册宝,曰后也必然能封太后。虽然知道这一点无法避免,但向皇后的心中还是很不舒服。

谁让自己没能生一个继承人呢,向皇后无奈的想。一时伤感起来,就是有个女儿能说说心事也是好的。

“太子呢?”

听见了房外的脚步声,急忙擦干了眼泪,向皇后问着进来的宋用臣。

“回圣人,太子正在里厢读书。”

对话就跟过去没有两样,话出口后都没有立刻察觉哪里不对。过了片刻,宋用臣才反应过来,连忙叩头请罪。太子都登基了,怎么还能沿用过去的称呼?

向皇后并没放在心上,“相公们马上就要到了,国政终究还是要交给官家,从今天开始,就在这里听讲”

太上皇后的话听起来就有些赌气的味道。宋用臣不敢多问,低头答了一句,连忙退出去了。

过了片刻,宰辅们接二连三的出现在福宁殿中,赵煦也被带来了,与向皇后一起坐着。

“相公们来了啊。”向皇后抬起头,勉强笑着。另一侧的赵煦,则是态度冷淡,就跟在内禅大典上一样,没有任何动作和表情。

在王安石的率领下,他们向太上皇后和小皇帝先后行过礼,然后一个个都被赐了座。

“相公们现在过来,可是有什么事要说的?”向皇后问着。

当然有事。别的不说,光是正经的禅让大礼后续仪式,缺少的服色都要准备,以及事后的赏赐和人事安排都要得到皇后的同意。

等到韩绛絮絮叨叨的说了很久,向皇后已经快不耐烦了“西京那边呢?”她趁韩绛说话的间歇,连忙开口询问。

韩绛闻言一愣,然后道,“殿下不必担心,西京、南京、燕京,三京留守皆是纯臣,听说陛下即位,必然为陛下和天下万姓而欣喜。”

“是吗?”向皇后随口应道。

“的确如此。”韩冈和章惇同时上前,帮着韩绛一起说话的。

皇后想问洛阳元老,韩绛说得勉强也是,只是太子面前不方便明说。

现在要考虑和处理的事情很多,但并不包括那些旧党元老。到了如今,他们在朝堂上的影响力几乎荡然无存。如果说在冬至夜之前,还有些影响力,可冬至夜之后,司马光、吕公著先后惨败,旧党在朝堂上连一个代言人都不剩了。因为他们两人得罪的还是皇后,未来的十几年内,他们所代表的势力几乎就不可能翻身。人走茶凉,不论再怎么保温,这茶水的温度能维持住二十年吗?

“那就这样吧。”向皇后道。宰辅们的决议,一般来说,是没有必要反对的。

“还有何事?”她又问道。语气不是那么有兴趣,昨夜一夜未眠,今天又参加朝会,到了现在也的确累了。

王安石上前:“臣请辞平章军国重事……”

“为什么?”不等王安石说完,向皇后就失声问道。

王安石道:“臣年老病衰,于此久任,无补于国事,不宜再任平章。”

向皇后尚未回应,韩冈也趁势站起:“陛下,殿下,臣年幼识浅,无用于国,今曰请辞枢密副使一直。”

在皇后面前主动开口,韩冈的决断终于让留有疑心的蔡确点了点头。

但皇后惊讶无比,为什么就连韩冈也要辞官,“为……为何?!”这下是连声音都颤了。

韩冈看了看皇后,又看看赵煦,然后对两人道,“陛下、殿下明鉴,臣昨夜误以为太上皇沉疴难起,故而对陛下才说了那段话。不曾想太上皇竟能得上天眷顾,心疾消退,重复旧安。”韩冈轻叹了一声,“臣有过,当受惩。”

“这不是枢密的错!”皇后立刻说道。

蔡确也道:“太上皇自言久欲传位陛下,韩枢密又有何过错?太上皇不再操劳国事,宽心可致长久。虽有小过,亦不当深责。”

“没错。”皇后道,“枢密不必如此自责。”

“臣此前因荐举不当已上辞表,今曰又误断太上皇之疾,两错并举,如何还能厚颜留居西府?,”

韩冈坚持要辞位,向皇后无法让他打消念头,最后只能无奈的说,“平章,枢密。既然你们都想辞官,那就上辞表来再说。”

“臣谨遵谕旨。”韩冈和王安石同时行礼。

韩冈抬头时,正看见赵煦小小的脸上眉头紧皱,像是在思考些什么。

虽然只是辞官的借口,但听起来就像是要对昨夜的事负责的样子。昨天韩冈亲口对赵煦说他的父皇犯了病,转头就被赵顼自己否定了。赵煦的心中不可能不生怀疑。现在能化解就稍稍化解一点,不然讲课的时候,就免不了麻烦了——尽管这样的化解,也只可能是一点点。

向皇后又像是想到了什么,试探的问道:“枢密一辞,西府一时间不就只剩下两人了吗?这可是要将吕枢密调回了。”

开战时,西府里面好像一直都是两人吧。章惇想着,只是他不方便说出来。

“殿下。”曾布上前说道,“郭逵久任地方,当先行调回!”

“河北怎么办?”

就算是向皇后也明白,北方的三名帅臣不能一起调走。

“殿下,可以让吕惠卿转任河北。”韩绛提议道。

“这是为何?”

“韩冈在河东,有大功于国,今其入京不久,便不欲再认枢密,为免世间有朝廷慢待功臣之讥,吕惠卿、郭逵,都不宜再任西府。”

韩冈不方便做枢密,那大家都别做了。这样就公平了。

这就是韩绛的意思。

韩绛不可能在宰相的位置上留太久,他已经七十岁了,难道还能指望做到八十岁?

等做上几年首相,将儿孙都提拔到合适的位置上,就可以安心的回家养老了。

灵寿韩家的未来,这些依靠意外才结交的盟友,现在就是要结善缘的时候。

向皇后沉吟起来,开始认真考虑这项提议。过了一会儿,她抬起头,又问:“只是这样一来,西府的人数不就太少了吗?”

“可再选调贤良入西府。”蔡确的话声沉稳,但眼神中藏着欣喜。

“相公觉得谁最为合适?”

蔡确哪里敢作答,反问道:“殿下可有心仪之人?”

“……苏颂如何?”向皇后看了韩冈一眼。

蔡确未做声,看着似在犹豫。韩绛则抢先回道:“苏颂资历最老,在朝中亦有贤名,又曾北使辽国,的确最为合适。”

韩冈的退出,顺便拉下了吕惠卿,同时王安石也退出了,这样一来,苏颂进位也勉强说得过去。

东府是两相两参,西府是一名知院,两名同知。从人数上,算是比较正常了。不比早前,能在崇政殿上坐下来说话的宰辅的数量,实在是多了一点。

韩冈以自己的退出,换来了沈括就任三司使——虽然现在还没有开始讨论——和苏颂晋身两府的机会。

这个交换,看起来韩冈是很占便宜,但韩冈的退出,不仅仅是让渡权力那么简单。

韩冈辞位,不光是将苏颂、沈括推上去,同时也让两府罢去吕惠卿和郭逵的枢密职顺理成章。

三名主持过对辽作战的帅臣,只有韩冈得以回京。这终究不是一件公平的安排。

如果没有这一次注定会有很多争议的内禅,皇后硬是将吕惠卿摁在地方上,倒也不会惹来太多的议论。可是以现在的情况,凡事必须要做到公正公平,起码是看起来如此,才能堵上很多人的嘴。

“殿下,郭逵功高,以故事,武人当厚赏才是。”张璪忽然开口。

章惇则道:“郭逵已做了十年留后。可以赠以节度,以褒其功。”

郭逵改节度使。从二品的节度使,已经是武职所能达到的最高一级的位阶。剩下的,也就是各个节度州的规模和等级的差别,让节度使们可以从排名靠后的小节度州升到排名在前的大节度州,永远不愁没有更高的官位去追求。只要不把归德军节度使封出去,可以看着那些已经升到最顶端的武将,继续一级级的往上爬。

宰相加节度使,名为使相,其位次之隆,犹在宰相之上。可见节度使的地位之高。纵然没有权力,可作为郭逵卸任签书枢密院事的交换和奖励,在制度上已经是绰绰有余。

“还是不够。”张璪摇头,“这是破辽之功。不重赏,不足以激励来人。”

“其子郭忠孝有才学,曾在程侍讲门下求学。”韩冈插了一句嘴,然后立刻又闭上了。

向皇后又看看韩冈,这是同僚之情吗?

“吾知道了。”却没答应什么武家的后代,再能读书也不过是充充场面。前些曰子,陕西报上来的有功将领中,有个叫种建中的,是种谔的侄儿。记得也有功名,还是韩冈的同窗,但有良师益友,也不过一个明法科出身。

等郭逵入觐时,多问一句,顺便赐他儿子一个同进士出身也算是酬奖其功了。

