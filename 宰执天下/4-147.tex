\section{第20章 冥冥鬼神有也无(23)}

夜色如晦。

伏在岸边,黄元眼前一片黑暗,耳朵里全是哗哗的水流声。

想从江水的流淌声中,将战船破水的声音给区分出来,黄元没有那个能耐。但他身边的一名左江上跑几十年船的老船工,却是很明白的在说,“已经来了,至少有十条船。”

老船工的话声带着颤音,显然是对即将开始的战斗感到恐惧,但黄元的心中一片火热,“想不到交趾人当真过来了。”

他前日受命领军攻向上游的如月渡。那一段的河道最为宽阔,相应的也是水势最为平缓的一段,是富良江上有数的大渡口。他本以为大军会从如月渡过江,可他刚刚占领了毫无抵抗的如月渡后,却又被招了回来,镇守在船场中。

当然,作为广源州出身的将领,黄元不可能是独立镇守。但主将燕达亲自领军坐镇船场,跟在他身边,黄元也就没有半点受辱的感觉。

眯起双眼,紧盯着不远处的水面,好不容易才看见了一团团黑影正缓缓逆水而上。黄元狠狠的咧着嘴无声的大笑着,终于能够与交趾人好好打上一仗了。

扯过身边的亲兵,命他立刻回船场通报,黄元转过身来,又望着那一支即将踩进陷阱之中的船队,渐渐驶向船场水道的入口。

……………………

只借助水面上反射的星光,十一艘交趾战船平平稳稳的驶进了富良江的支流。

阮陶立于船首,昏暗的夜色中,他的双眼看不清水面上情形,他的船队是冒着搁浅的风险才进入了这一条只有五六十步宽的河道。船上的舵工虽说是配了熟悉水道的向导,但没有一艘搁浅,还是让阮陶喜出望外,也松了一口气。

向后招了招手,阮陶问道:“离着船场入口还有多远?”

战船无法从芦荡中穿过去,宋人打造好的船只,也不会走芦荡中出来。宋人船场借用的深水塘,本来就是有主的。用以停泊渔船以避风浪的水塘主人,现在也就在阮陶的船上。

“回将军,过去这片芦荡,再往上一里就到了。”水塘主人回答的声音颤抖着,但阮陶已经很满意了。进入船场水道就在眼前,如果能一战功成,日后的富贵自不待言。

如果是陆上决战,给阮陶十倍的兵力,他都不会过来。换做水战,他可就不怕了。宋人从北方调来的援军,想要在水上称雄,绝不可能那般容易。

船尾的大橹缓缓的摇着,尽量不发出过大的声音,推动着战船溯流而上。

“那是什么?”

阮陶忽然发现,就在河道右岸,有一座一丈多高的黑影,看轮廓明显是人工建筑,但绝不是房子。再仔细去看,又在另一侧发现了两座,三座。静下心来再找找,惊觉同样的建筑竟然有十来座之多。

“那是宋人在江边上修的望楼。”紧跟着水师统帅身后,探查船场消息的细作回答着问题,“总共有十四座,都是跟船场一起修起来的。但都还没有修好,可能是因为要过年所以就停工了。”

阮陶皱着眉头,“这就是你之前说的望楼,当真是停工了?”仔细的观察了一阵之后,确认了细作的回话,心情也更加好了起来。这么重要的建筑竟然没有完工,宋人当真太过于自大了。外面的耳目都这么疏忽,里面的防御想来也不会严密,选在过年时来偷袭,实在是选对了时间。

终于接近了入口。

咚咚的几声轻响,十一艘战船小心翼翼的在河道中央下了碇石。一艘艘小船从战船上放了下来,转眼就是四十来艘。藏身在船舱中的一干精挑细选出来的敢勇,也涌上了甲板。

“到了?”一名与李常杰又七八分相似,只是年轻了数岁的将领踏上甲板,很不客气的问着。

“回节度,已经到了。”阮陶恭声说道。

浑身上下结束整齐的将领是李常杰的亲弟弟李常宪,都到了这个时候,李常杰也不能让自己的家人留在安全的地方。

李常宪也不多话,顺着拖下去的渔网,安静的降到小船上。当敢勇们全数在船上落定,便以刀代桨,飞快的驱舟向着进入水塘的水道划过去。

这一过程中,压低了呼吸的阮陶急速喘了几口气,脸上终于绽起了笑容。都到了这个地步,宋人竟然还没有发现,看起来这一次偷袭赢定了!

……………………

“昨日除夕不来,今天终于还是到了!难得的客人,可要好生迎接!”听到通报,燕达一声笑,长身而起。拿着自己的头盔,带领一众部将走出简陋的营房。

正月初一的夜晚,天上没有月亮,只有星光。进入腊月以来,都是晴天居多,交趾人要想来偷袭船场,也只有选择没有月亮的朔日前后。

船场上的空地中没有看不到什么动静,几堆篝火平静的燃烧着,七八支巡逻的小队绕着营地的各个角落。隔上片刻,就有一队从篝火边穿过,一切都跟过去的一个月没有区别,让人觉得这一个夜晚毫无异样。

可如果是换作熟悉军事的将领们来看场中布置,就能发现每一座营房的修建地点,都是放在最容易攻出去的位置上,并不是营地中最为安全的地方。

燕达从营房中出来,身后的将领随即无声无息的散开,回到他们的队伍所在的营房中去。

十几个亲兵跟随着燕达,看起来就像是普普通通的巡卒。

燕达举目打量着船场内外,坐镇于此半月有余,终于可以回去交差了。

只用了近一个月的时间草草成立的船场,楼船是不用指望的,艨艟也造不出来,只能打造渔舟一个等级的船只,最多也仅能运送十几个人的小船。就是单纯的运兵,没有任何作战的能力。

十几人的小船,只要一起动手运桨,过江也不会慢。只是来自于关西的主力,能游泳的都不多,站在船上都直不起腰来,根本就不能指望他们上船后还能用桨划船。从左江上调来了一批船工,加上来自广西的新军,靠着他们来划船。

只要能毁掉交趾水师的几艘船,甚至只要能镇住他们,就可全军出动,与蛮军同时强渡富良江。官军和数万蛮军一旦抵达对岸,接下来就是北面的翻版,渡过大江的官军根本就不用担心粮草,只要一门心思的攻城就是了。

喊杀声猝然响起,敲碎了元日夜晚的宁静。自通向河中的水道上,千百人的声音传了过来。紧接着,一声号角划破天际,听着有几分急促,但落在船厂内所有人的耳中,这就是开战的信号。

幽暗的营地一下变得灯火通明,上千人从营房中抢出,早已是装备整齐,顺着事先划定的路线赶往各个要点。

……………………

号角声在岸边响起,而来自于交趾士兵们的喊杀声,则立刻又将号角声给压下。

“杀啊!”

李常宪意气风发的挥刀指向前方。此次偷袭,宋人全然无备,这一声声急促的号声正代表了守军的慌乱。

领头顺利的进入了水道,接近了船场的水门。而所谓的水门只是两条简陋的绳索。绳索之后的船场,静得只有几点长明灯火,除此之外,一点动静都没有。直到砍断绳索,冲入船厂内部之后,船场守卫们才反应过来,赶着吹号,可这已经来不及了。

“烧!烧光宋人的战船!”李常宪得意的拿刀指着黑暗中停在水面上的一艘艘小船,还有堆在岸边的造船材料,“全都要烧光!”

几十条船上的数百敢勇一齐呼应的大喊着,营造着千军万马来袭的声浪。手上也不耽搁,一个个都在给随身携带的火箭点火,要将宋人的船场用无数愤怒的火箭给烧个干净。

嚓嚓的火石声中,船场亮了起来。

李常宪瞪大了眼睛,这光源并不是来自船上的火箭,而是来自船场的每一个角落,照亮了水上,照亮了船上。

扑通一声,李常宪握在手中的佩刀落入了水中,双腿软软的,整个人都坐到在船板上。

脚步声自四面八方响起,千百人从设在船场各个角落的房屋杀出,更为雄浑的吼声由地面反冲回水上,紧接着就是一声声弦鸣。

一支支弩箭从身边划过,每一瞬间都能带起一声惨叫。

望着水道两岸密密麻麻活动中的人影,听着越来越多的重弩发射时的弦响,李常宪悲愤的大叫起来,“这是陷阱!”

“不对……这是陷阱!”

阮陶脸色一片惨白,只看着在一瞬间亮起来的船场,他就知道事情不妙了。如果宋军不是早有所备,怎么可能会如此整齐的亮起灯火?

水师统帅当机立断:“失败了!吹号,全军速退!”

可就在他开口大喊的时候,几支火箭破空而至,在阮陶的视网膜上留下数道鲜红的轨迹。轨迹的末端不是船只而是水面。但落到水面上的一点点火焰,却仿佛像是火星落进了干草堆中,一道火光猛地蹿起,转瞬就扩散开来,化作了一片火海,映得河面上刹那间亮了起来。

河面上浮着的竟然全都是油,来自下方的火光照得阮陶脸色忽明忽暗,船上也是一片混乱。

一声大吼喝止了船上的乱象,阮陶厉声髙喝:“不用慌,这点火烧不起来!砍了缆绳,退出去!”

每一艘船上收放石碇的绞盘边,站着的都不是水手,而是手拿利斧的军汉。阮陶本就准备着一旦战机不利,就砍断拴着石碇绳索,全速逃离。这个预备,现在看来并没有错。

可是已经迟了。

咚的一声巨响,一道水柱就在阮陶的身边腾了起来,哗的一下将带着腥味的河水全溅到了他的身上。

这不是石碇落水的声音,而是来自岸边的一块块石块。

