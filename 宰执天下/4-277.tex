\section{第45章 仁声已逐春风至(中)}

【待会儿有事,下一更会迟些】

拜访过王珪和苏颂,没有什么事要做的韩冈,就变得十分的清闲。

其实算一下时间,等自己捱到了时间点去群牧司报到,离着在京的各大衙门锁印放年假,也就七八天了。

都进入腊月了,韩冈暂时不想出门访友。上门的人就够多了,想要他回书安胎的更多,看着厚厚一叠短笺,他的牙都疼起来了。

就带着人,在自己的新居里转悠着。而开封府的户曹参军桓修仁正好来办交接,便一边为韩冈做介绍,一边一同在占地十几亩的前高平侯府中逛了起来。

绕过石灰斑驳的照壁,出现在面前的正堂高达近三丈,左右八个开间,前厅中六根巨柱都有一人粗细,梁柱间的绘饰虽然也同样是斑驳脱落,但只从残留下来的图案上看,肯定是出自名家之手。不是宗室戚里,也难有这样的规模。

“高平侯府源自秦康惠王,太平兴国八年修建,在天圣二年被焚,四年重修过。重修之后,前后十一楹,一百零六间,十四半间、二十一含、厦十七、过路一百二十九、披四、挟二十二、龟.头总计十五所。虽然比不上咸宜坊、常乐坊,但在京城中也算是排在前面的大宅院,开封府的官产中,这样的宅院也不过三十一处,只比宰执官们的宅子小些。至于更大的宅院,就不是官产了,而是皇产了。”

桓修仁对自己管理的数字如数家珍,韩冈眼中闪过一阵惊异,他当年担任开封府界提点,这一位还没有调来任职,他接触过的开封府中的属僚,没一个有他的水平。

跨过中门,进入内院,一下就变了形制。左右厢向内缩进,让内院的正院显得十分狭促。

桓修仁向韩冈道:“原本宅院中没有这么多间房,就是重修后,也只有八十一间。不过在皇佑年间高平侯府分家,连整间宅院全给分了,东西跨院和后花园都分给了各房居住,所以另外有所增筑,直至如今的规模。”

原来如此。韩冈看着院中的建筑,“难怪新旧不一。”

高平侯家家大业大,分家后人口更多,不过以韩家的人口却用不到这么多房子。除非能把巩州的庄子一起搬来,不然能有一半空屋。

在后花园的入口,有一座小院,应当是老主人闲居之所,屋舍用料考究,甚至比起正堂都要强上几分,挂了个退思堂的匾额,也是韩冈这般猜测的原因。

韩冈挺喜欢这个名字,出自左传,‘林父之事君也,进思尽忠,退思补过,社稷之卫也’,而且落款还是蔡襄。千年之后,是让人打破头的宝贝,遂让人取下来送去重新上漆涂金。

桓修仁凑着趣说着:“当年蔡君谟以书、茶二事闻名京中,开封内外,不少地方都能看到他的墨宝。”

而位于后花园内的一座小楼、一座独院,分别名为小琼楼、听雨阁,这就有些恶俗了。韩冈看了看,倒是有心摘下来当劈柴烧。

后花园中,一个坑接一个坑,宛如月球表面。如果时人能有高倍率的望远镜,应当能与韩冈有同样的感慨。不过有一件事让韩冈很奇怪,不知为什么,显微镜都有了,望远镜却到现在没有出现,不知是有人发明了之后藏私,还是这层窗户纸太严实。但韩冈没打算去捅破,应该是迟早的事。

“高平侯家的花木很是有名,不过几十本名花名木在搬家时就挖走了。”桓修仁向着神飞天外的韩冈解释着。

韩冈低头,一个一丈大小的深坑就在眼前,里面还积了半坑水。抬头看看桓修仁,这是哪种的花木?百年老树吗?

桓修仁想了半天,才说道:“应该是为了窖金。可能是过去髙平侯府有钱时埋进去的,搬家时不得不重新挖出来。”

说得没把握,韩冈也没放在心上。望了一圈前高平侯府破败的后花园,他就摇头笑了笑,不单是后花园的问题。

韩家的新居没有人居住的时间不过一年,将杂草杂木都清除干净后,看起来也就勉强能住人了。不过许多地方,甚至正屋中厅向院庭突出的龟.头——很早以前,韩冈为这个名字笑过,在后世应该叫抱厦的小间——连天花上的承尘,都给泡烂了。

韩冈和桓修仁两人从前到后绕了一圈,用去了近一个时辰的时间,有问题的地方发现了许多。

在韩冈看来,整间宅院里里外外都应该整修一下,否则不定什么时候,一阵风吹来,就能吹倒几间房。不过这间宅院属于官产,要整修也该房东来。

韩冈没指望开封府能帮他将整间宅子全都修上一遍,但好歹将朽烂的木料给换了,这也是房东的责任。

可惜桓修仁却摇头:“龙图,道理是这么说,但实话说出来不怕龙图你生气,除了桐油、青瓦以及瓦当以外,府中没有其他材料可以提供。而且这三样都要。但实际上还是要花钱来买,府衙里也不会这笔钱负责。依府中的惯例,只要能住人,就是椽子都烂光了,也不会主动去整修。”

韩冈很意外:“记得我过去住在京城时,也是租了官产,怎么没听说这回事?”

“龙图,那里的一片可都是新宅,建起来才十年不到,哪里是这间五十年的宅子可比?”桓修仁叫苦道:“虽然下官当时还没有调来京畿,但龙图能在那里得到一件宅子,说起来,当是龙图当时就已经有了赫赫声名,所以衙中不敢相欺。”

韩冈听了倒也罢了,不打算为难人。将交接办好,让人送了桓修仁出去,紧跟着就来了一名访客,竟是童贯。

“龙图,天子在崇政殿有召,请龙图即刻入宫。”

韩冈领了口谕,心中却满是疑惑,弄不清楚天子怎么这么急,竟让他去崇政殿。他才刚刚打定主意,对朝堂上的政事不去多费心神,做个合格的旁观者。想不到转过脸来,天子就让他去崇政殿,应当不是任实职,但肯定有事要咨询他。

但凡中使,没有不擅长察言观色的,童贯哪里不知这是与韩冈结深善缘的良机,低声道:“是轨道的事。方城轨道成果斐然,官家心中欢喜。”

韩冈心中有了底,换了一身干净的公服,便出了门,与童贯一起往宫中去。

崇政殿中,除了正在被御史弹劾的吕公著和章惇,宰辅们现在都在。赵顼投过来的眼神中,没了前些天的冷淡。

待韩冈行过礼,赵顼就连声说着:“韩卿可知道方城轨道这个月收取的运费是多少?”

韩冈当然知道,但不方便说,“还请陛下示下。”

只见赵顼兴奋得两眼发光:“收入两万四千一百余贯,除去人工、牲畜食料、以及修补损耗,净入整整两万贯!此皆是韩卿之力!”

为了区区两万贯,至于吗?

当然是没问题的。

因为赵顼看到不仅仅是一个方城山。

不当家不知柴米油盐贵,赵顼做了多年的皇帝,一早就知道钱的重要性。经过了王安石的多年熏陶,赵顼并不在乎当着儒臣的面,谈论收入、财计之类的话题,不怕丢面子。

他在景福殿库以八句四言诗为分库库名,‘五季失国,玁狁孔炽。艺祖造邦,思有惩艾。爰设内府,基以募士。曾孙保之,敢忘厥志’,当做扑满用来积攒军费,多一份钱,就离胜利更近一分。

吕惠卿接口:“方城轨道在纲运结束之后,才转为民间使用。这才过去一个月,就净入两万贯,而且还是刚开通不久,没有什么名气。到了明年,当有三十万贯到四十万贯,五十万贯也是可能的。”

去年全国的商税收入也不过一千万贯,其中东京都商税院是四十万贯,预计明年的收入不会比去年增加到哪里,而区区方城轨道,就基本上能与东京一城的商税持平,相当于天下商税收入的二十分之一到二十五分之一。

“而且方城山中的轨道才不过六十里路!”赵顼激动的补充。

“陛下。”韩冈看不下去了,泼起了冷水,“时值年终,行旅商人往来众多,是一年中的特例。若要计算全年收入的话,不当以其为本。”

“韩卿有所不知,方城轨道客运收入倒是不多,但货运营收甚多。等到了明年开春,只会比冬月腊月的收入更多。”赵顼说话的口气像个二道贩子,宣扬着自己手上的货物。

吕惠卿说道:“用渠道,只能收个百分之二的过税,山南、山北各收一次,也不过百分之四。但利用轨道,不但能收过税,而且还能收运费。同时在船只向有轨马车的转运过程中,查税也能变得很方便,不怕有人夹带隐瞒。”

“虽然货运收起来多了点,而且没了夹带,但总体算来,还是要比从扬州绕道汴河的那一段要省钱。”元绛也跟着道,“其实,能做到这一点的就够了。”

这是典型的双赢。

