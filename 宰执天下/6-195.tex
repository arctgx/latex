\section{第21章 欲寻佳木归圣众(15)}

“怎么尽是些赤佬?”

“南面出大事了?”

“大概是去大理吧?到了长江再转水路很快就能入川了。”

“大理国不是赢了吗?”

“谁知道是真是假?”

自从京泗铁路通车开始,东京车站的站台便总是熙熙攘攘。

作为已经往返京师与泗州十余次的商人,彭义已经习惯了人来人往、嘈杂喧嚣的车站。

但今日的混乱局面,也是他所没有见识过的。

刚刚验过票,走进站台,他就发现站台上的旅客们正三五成群的聚集在一起,大声小声的议论着。

向站台的另一个方向望过去,与这边有着一道木栅栏相隔,便是议论的缘由,也是混乱的来源。

同样是水泥铺砌的地面,高大的棚架也同样将站台遮掩。

但棚架下方,站台之上,不是带着大包小包的商贩、旅人,而是一名名身披军袍,头顶铁盔的士卒、将校。

一列马车停在站台边,这群军汉正排着队,从车门上鱼贯而入。

士兵们的腰上挎着刀剑,背上背着的行囊,有的人的行囊上,还能看见几双新扎的草鞋。大大小小的旗帜,旗面被紧紧地扎了起来,旁边放着战鼓,弓刀、盔甲也都捆扎好,一起送进了车厢中。

彭义甚至能看见那一处的站台上,在一张张涂了沥青的油布之下,被包裹得看不出外外形的物体,被人小心翼翼搬了进去。但只要是东京人,又有哪个认不出来?

“那是火炮吧?”

“是虎蹲炮!”

“这么多,肯定是上四军的。”

一个普通的京营步军指挥,如果已经换装,那么就能拥有五门虎蹲炮。但若是上四军,那就是十二门。任何一个指挥辖下,不论马军还是步军,都会配备连驾驭骡马的马弁在内,整整一个都的炮兵。

看到月台上,整整齐齐七八排九十列的虎蹲炮,稍有些见识,都知道这必是上四军出动了。

“是神卫军。”

彭义自家出身就是军营,长兄还在军中吃官饷,虽说只是普通的虎翼军出身,但分辨一支军队到底是什么根脚,不需要看第二眼。何况,他已从别的渠道得知,这一次出动的是神卫军的四个指挥——分别出自左右两厢。

“这位兄弟,都没看到指挥旗,你怎么看出来的。”

彭义回头,几张凑到眼皮前的大脸让他不由得向后一仰。

这年月,就数皇城根下的百姓,最喜欢议论军政,若是有些干货在手,一开口就能引来一群人。京师的茶馆酒楼,之所以多如牛毛,也正是因为京师的百姓太喜欢摆龙门阵了。

彭义随口的一句神卫军,立刻就让周围的人觉得这是一个懂行的。

面对几对闪闪发光眼睛,彭义保持着京冇城人的习惯,能炫耀的时候绝不卖关子:“昨天在冠军马会的宴上那边听说的。冇”

立刻,周围人投过来的眼神就不同了。

冠军马会中的马主,哪一个不是有几十上百万贯的身家,穷一点的郡王都养不起一匹冠军马,到现在为止,马主也就那么二三十人,人人都是手眼通天。眼前的这一位,肯定是没资格做马主,但能参加冠军马会的宴席,肯定也是有些身冇份或是关系的。

只不过,都是见多识广的东京人氏,拉虎皮做大旗的骗子也见得多了,彭义张口就是冠军马会,改变的眼神,倒有一半往看到骗子的方向变去。

“这位官人既然能去西十字大街去赴宴,肯定是知道的,”说话的人忽视了彭义身上的并不华丽的布衣,改了称呼:“枢密院到底在做什么?好端端的,作甚弄得人心惶惶?”

另一人插嘴:“不止是枢密院,还有政事堂。”

“自然有政事堂。没韩相公点头,枢密院敢调京营的人马?”

“有了两府,肯定要禀明太后了。太后两府下令,调兵南下,可是南方有变?”

几个人七嘴八舌,等着彭义的回答。

“诸位啊,南方太太平平,一点乱子都没有,就别胡思乱想了。小弟也是凑巧听到了,这事本也没什么好瞒着人,过几日京冇城也就能得到消息了。”彭义慢悠悠的说着,“太后和几位相公呢,是打算趁纲粮已清,新米未收的空闲时间,试一试这条铁路到底能运送多少大军。日后要打辽人呢,在国中肯定都是在铁路上走,十几万大军,到底怎么走,不是说一句立刻就能拔腿上路的。人吃马嚼,总得有个章程来。所以啊……”

“所以什么?!”

“所以就要先多历练历练,免得事到临头,弄得手忙脚乱。据说,”彭义双眼闪烁,“这一回至少二十个指挥,一万人南下,到了泗州后,再坐车回来。十天之内,要走个来回。”

……………………

“幸好之前,否则还不知会被言官怎么弹劾。”

韩冈笑叹着气。民间的流言,他已经从冯从义的口中了解到了三五分。

都已经有人在说,朝中出了奸臣,要削减京师的守军,好趁机作乱。

传言就是如此无稽,韩冈也是无奈。

“也幸好这一次没有当真调动上万人出京。也许数字只有两三倍的差距,但千与万,给人的感觉毕竟不同。”

尽管韩冈的确像来一次像样的压力测试,试试看以京泗铁路现有的运营水平,到底能做到哪一步——铁路贯通只是意味着具备了硬件而已,而日常运行则是软件,缺乏足够的运营能力,便是有了铁路,也会造成巨大的浪费——运力,以及利润。

不过他还是不可能仓促之间进行如此规模的军事调动,而且神机营正有一半在外,只是为了皇城中的人心安定,他就不可能将护卫皇城的主力,再分出很大一部分南下。

神机营只剩五个指挥留守宫禁,当然不能动,最多也只能动用上四军。

在整军计划中,上四军都是要成为真正能上战场的强军。天武军,日夜在皇城值守。捧日军,天子、太后若有行动,都要护翼左右。而龙卫、神卫两军,调动稍微容易一些,所以率先进行整备。其中龙卫军是马军,而神卫军是步军,所以最终还是选定了神卫军南下。

“但乱子还是免不了。哥哥你立身之地难以自清,不免启人疑窦。”

韩冈道:“更戍法是祖宗良法,京营若是不堪战,如何镇服地方?太祖太宗的时候,禁军调动频频,也不见有人胆敢作乱。”

冯从义道:“可官家和太后,毕竟比不了太祖、太宗。”

“所以也只是先让人习惯一下,之后才是正戏。”

“习惯?国人,还是辽人?”冯从义笑道。

韩冈道:“自然都有。”

这一次的调军南下,是两府共同议定,要为与辽国的大战开始准备,更重要的这也是在警告北方的邻居,不论他们的马有多快,大宋铁路,绝不会比他们慢上一星半点。

想要威吓敌人,亮一亮肌肉是必不可少的环节。

江南有变,须臾可至。河北有警,转瞬抵达。

尽管大宋与辽国,自耶律乙辛篡位之后,彻底断绝了官方往来,可是边境上的榷场依然没有变动,往来依旧频繁,辽国细作在大宋腹心之处的,至少冇数以百计,其中的大半都会紧紧盯住这一次的演习。

当然,这一切的前提,都是演习能够圆满完成预定任务。

“有把握吗?”

“以沈存中为主,以他的才干,应付得来。”

沈括将会以枢密院新成员的身冇份主持这一次的演习,铁路的军事作用必须通过这一次的行军演习来进行体现。

九月初,今年的纲运将会提前宣告完成。到时候,就会使万军齐发,用最快的速度

“一出一入,也用不了半月,钱粮是小事,不过,赏赐应该准备好了吧。”

“整整八十万贯。”韩冈道:“债券上签名签得让人烦啊。”

朝廷已经准备好新制钱八十万贯——铸币局一个月的努力,再被政事堂用国债从太后手中借出——用来作为演习成功的犒赏。

韩冈从袖中掏出一枚金灿灿的钱币,丢给冯从义,“这一回,不光铜铁钱,连金钱都准备好了。”

冯从义一把接住,翻手一看,就笑道,“是熊猫金币啊。”

金质的十贯大钱,拥有最精细的制造工艺,也用上了最好的提炼手法,沉甸甸的压手,也金灿灿的炫目。

金币上的图案是韩冈定下来的,圆滚滚、胖乎乎,看着像熊,却又不是熊,是传说中的貘,俗名食铁兽。

也就是后世的大熊猫,现在也是叫做熊猫。

外形如熊,性子像猫。在院子里打滚,又懒洋洋的晒太阳,真要说起性子,也的确像猫。

因为韩冈的缘故,比起貘和食铁兽,熊猫之名流传得要快得多。

一点点自娱自乐的心情,让他放弃了如意、莲花之类的寻常图样,

一只金色的熊猫趴在金色的竹林,竹叶清晰可辨,眼圈也清晰可辨,这是现今锻造工艺的最高水平。

冯从义用手指轻轻弹着金币,“一副吃肉的牙口,却只吃竹子。难怪大相国寺的说这熊猫天生有佛性。”

“熊猫偶尔也会吃些荤的,鸡,羊都有。”

冯从义将金币还给韩冈,哈哈笑道:“越说越像那些贼秃了。”

“好了,不说笑了。”韩冈放下了手中的熊猫金币,正色道:“义哥,这一回,让你顶着暑热上京来,有正事要你帮忙。”
