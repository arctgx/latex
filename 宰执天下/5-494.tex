\section{第39章 欲雨还晴咨明辅(23)}

在列的宰执都不是蠢人,韩冈只是不想担责任。只看他不正面回答吕嘉问的问题,就知道了。

章惇忍不住想笑一下,难得看见韩冈被逼得顾左右而言他。

国债这东西,第一要有信用,第二要有信用,第三还是要有信用。

但信用是不是强迫来的,没想好还钱的办法,就要发行债券,到时候失败了,可是要用朝廷信用来补上。

王莽是怎么败的,是从信用开始败的。

反过来说,明内外之分后,就能用更好的手段从内藏库中拿钱。

眼下将内藏库一口气掏光,那是特例。大部分情况,还是皇帝紧紧握住了财权,只从指头缝里挤出一点油水出来。每年六十万贯的补贴,相比起内藏库的总收入又算得了什么?迟早要回到正常的情况下。而现在韩冈的提议,就等于是留个了后门。

给了借据,又有抵押,还有还款的期限,且既名为债,又不可能不用偿付利息。这样一来,曰后就可以多从内藏库中借钱,给几张国债债券做凭据就好了。

“嘉问敢问韩资政。市井借贷,无不要保人。敢问这个国债的保人,是否是中书门下?”

由中书门下具结作保?开什么玩笑。这成何体统?!

吕嘉问一提,宰执们倒想起了这一茬。借款总有收不回帐的时候,那样的话作保的一方可就要连带着倒霉了。

见韩冈没有立刻回答,吕嘉问气势更高,“三司借钱,中书门下作保。万一还不起帐,是搬了政事堂的桌椅抵数,还是把架阁库中的字纸给卖了?”

见吕嘉问趾高气昂,韩冈轻叹。他根本就没想到自己会被招上崇政殿,不论是韩绛还是蔡确、曾布、张璪,只要他们看到了自己的奏章,就会立刻明白他到底想说什么。

只是可惜得很,他之前依然是枢密副使,奏章直抵御前案头,而太上皇后,看起来也并没有将自己的奏章给下面的臣子们传阅一番。

“中书门下不是作保。”韩冈淡然笑着,“天下至信之文,无如圣旨。圣旨起头都是门下,又有什么公文能比得上圣旨更有信用?历数朝堂,也只有盖着中书门下的钤记,才能让人信服片纸可值千金。”

他早就说了,这是要找补。吕嘉问既然从自己手中抢食,那也别怪他不给面子。

蔡确眨了眨眼睛,再看看韩冈,怀疑自己是不是听错了。

只是看见吕嘉问一下涨红了脸,才确定自己并没有听错。

绝大多数圣旨,不论是践位大诏,还是赏赐、调职、救灾、礼仪,其抬头,都是‘门下’。

这是传承唐时圣旨的格式。唐代中枢,最早是三省并立,尚书、中书、门下。其中门下省审查诏令,签署章奏,掌封驳之权。所以天子的诏书,都是发给门下省。故而抬头为门下。

如今三省六部制只存空名,但政事堂的正式名称依然是中书门下。旧时门下省的封驳之权,依然留存。

既然借钱的是政事堂,出钱的是赵官家,那么要三司做什么?

章惇在摇头,薛向低头看着笏板,张璪双眼发亮,曾布反倒皱起眉来,瞪着韩冈。

除了前面的韩绛看不清表情,其他人的反应,尽收蔡确眼底。基本上都是知道韩冈的心思了。

章惇轻轻摇头,韩冈这是破门拆屋啊。

三司的设立,就是为了分两府的财权。治权、军权、财权分立,天子就能稳坐钓鱼台。

熙宁变法前,财权稳稳的控制三司使手中,计相为名,名副其实。但熙宁变法开始后,常平、农田水利、免役、保甲诸法皆本于司农寺,而由此得来的收入也归入司农寺背后的中书门下。三司财权从此为宰相分割。曾布当年与吕惠卿、吕嘉问不合,以至最后生变,正是开端于他贵为三司使,掌天下财计,而吕嘉问主市易,却只报与在中书的吕惠卿。

这就是财权之争。

没了财权,三司又算什么?

而现在,韩冈丢出所谓国债,不是站在太上皇后一边帮着说话,也并非打算推行国债敛钱,这分明是将钱跳过三司,直接送给中书门下。就算只是每年六十万贯也好啊。

天降横财!

蔡确轻咳一声,迈着方步慢慢走出班来。

因人成事,实是受之有愧。可既然韩冈送过来,他也就却之不恭了。

“殿下,臣以为韩冈所言甚是!”

“殿下。三司之立,本是分宰相之权。如今财归政斧,宰相之权何人能治?”

吕嘉问要做孤臣?也不看看太上皇后待不待见他。

“原本南方几大钱监每年所铸新币,都是先送进内藏库,然后再由三司支借出去。”韩冈顿了一下,“臣请设铸币局,专司天下铸币事。”

韩冈的意见是将铸币的终点放在内藏库,而支取就是以国债的行事直接调拨。如果钱价涨,就多散出一些,钱价跌,则少支取一点。

看看,这财权不是回到了天子手中?

崇政殿议事结束了,三司成了大输家。向皇后认为自己是赢了,之前对她不恭顺的吕嘉问被韩冈削得没多少差事了,内藏库依然是被借空了,但至少有了借据。两府宰执也认为自己赢了,他们手中的财权进一步得到扩张。韩冈也觉得自己赢了,他的计划接下来正在慢慢发酵。

“玉昆,掌管铸币局的人选就交给你了。”蔡确知道投桃报李,不与韩冈争这个从三司分离出来的位置。

铸币的本质仅仅是铸造,是个苦活计,需要的是一些工匠和善于器械的官员,都是底层的职位,也就有一些油水惹人垂涎。

可是在宰辅们眼中,一两个有油水的差遣,怎么比得上拿到手中大权?韩冈若只要这一点报酬,给了他又何妨?就算韩冈想通过铸币局达到什么目的,到时候,事情出来了再做计议也不迟。

“总得相公拍板。”韩冈笑着谦让。

不过也只是谦让,实际上的控制权,他不会让给别人。

铸币局的作用在于固定币值,让钱币能以面值通行,而不是其中的材料定价。

过去曾经有过因为新铸币制作精美,百姓爱用,然后就有官员奏请,将新钱由一文改为当两文使用。

对为了一点好处,却破坏朝廷信用的官员,韩冈可以说是深恶痛绝。伤害的不仅仅是朝廷的信用,更是伤害了当地百姓的利益,破坏了商业秩序。给地方造成的损失,改铸新式硬币,将面值固定下来,在钱币上表明。该是多少,就是多少。不因私心而变化。硬币还是硬币,但实际上,大额硬币的实际成本远小于面值,其中大部分价值,已经是归于国家信用。以一文、两文的小面值钱币,来保证钱币的信用。然后通过大面值钱币的铸造,来赚取钱息。

而旧有的借款,一旦改用了国债形式,而不是旧有的支借,借新帐还旧账就成了。到期连本息一起还清,然后再借入。等于是将朝廷财计,逐渐正规化,向民间借贷的方式转移。

这两件事定下,国家财计就有了些更有意思的变化。

与蔡确等人分开,一直与韩冈沉默的并肩走着,快要告辞的时候,章惇最后才突然开口:“玉昆,明修栈道暗度陈仓的事做得好啊。明内外,好得很呐。”

暗度陈仓?韩冈微微一笑,“天下是天下人之天下,非是一家一姓之天下。”

章惇为之敛容:“乾称父,坤称母?”

“正是。”

这是天下所有人都如此。

天子只是天地嫡子,但并不是说他一人就能继承所有。

按照大宋法令,父母去世,兄弟们要均分遗产,与嫡庶无关,此外,在室女——未出嫁的女儿分得的比例是兄弟的三分之一。

所以顺理成章的,天下大家都有份。是天下人之天下,非是一家一姓之天下。

天子临国,那是代天行事,并非可以将天下当成自己的私产。

任何一个皇燕京不会喜欢将自己脖子套进绳索里,气学永远都不可能受到天子欢迎。这就是为什么王安石现在不愿与韩冈争辩的原因,气学本身的缺陷让其难以走进国子监中。

有不少人都说,气学类墨。就算亲疏之别这一最为儒家诟病的部分不一样,但约束天子的部分却并无差异,甚至远远过之。

墨家尚鬼,以鬼胁人。而气学弃鬼神,以道理压人,章惇也明白,不论韩冈此前说得多么冠冕堂皇,本质上还是想要遏制天子。

韩冈的提议有几重用意在内,包含在最内部的一重,虽然一开始就明说了,‘明内外之别’,但真正的用意,还需要结合对气学的了解才能看得清楚,但其他宰辅多多少少都会有些感觉,不可能一无所知。越是精通财计,感觉也就越明确,见薛向今天说了话了吗?那位朝中财计第一的能臣,第一眼就看明白了。

可谁会多说什么?

只是看到小皇帝的样子,有些事,现在的确得开始未雨绸缪了。

