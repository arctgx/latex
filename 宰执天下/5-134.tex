\section{第14章 霜蹄追风尝随骠(18)}

入冬后难得的一个晴天,温煦的阳光让人不禁想眯起眼,好好享受一下冬日的温暖。但北院枢密使萧十三所在院落的气氛却是阴沉沉的。

耶律罗汉奴、萧敌里领着西京道诸将正等着出现,已经半个时辰了,但萧十三所在的房间却没有任何动静。

终于,房门一动,不耐烦的将领们立刻又振奋起来,可出来的并不是萧十三。而是萧十三最为亲信的幕僚。

耶律罗汉奴寒着脸,踏前一步,正要揪住人好生考问一下,但那名幕僚却径自比了个噤声的手势,让罗汉奴小声一点说话。

萧十三这两天的心情糟糕到了极点,

韩冈毫无顾忌的下杀手,拿着党项逃人的首级来充门面,让他的计划成了笑话。哪有这样的经略使?宋人的文臣不是应该认为蕃,

如果在草原上,针对各部族的减丁的策略,自开国时起,便从来没有停止过。或是挑拨内乱,或是逼迫上缴大量的贡品,更有直接动手,割上一遍草。但宋人还做得如此顺畅,可以说少见至极。

萧十三驻守西京道,还听说许多西北手边的文臣,将蕃人视为比汉军更强的战力,而给予许多优待。几曾听说过,会嫌归附蕃人太多的例子?就是缺粮草也该想方设法挤出来,从南面运一批来能有什么麻烦?不过耽搁时间而已。

记得只有宋人开国时,在北汉做过。但换作是蕃人,最多只有一两个部族被剿灭,可从来没有一口气将十几、几十个部族全数毫不留情的屠戮。

现在萧十三派出去的三部兵马有两支被确定被歼灭,耶律世良的那一队,甚至没有一人逃出生天。而宋人出手灭杀他们,甚至都不需要什么理由,只是嫌人多而已,可以说是冤枉得不得了了。

不能浪费太多的时间。

幸好最后派出去的一队人马幸免遇难,而且经过联络确认,他们被安排在最前沿的柳发川修筑城寨。柳发川紧邻东胜州的武清军,这也是为什么能这么快联络上的缘故。

修筑柳发寨的党项苦力大约在三千上下,而驻扎此处的宋军也只有五千不到。另外一个直面辽境的前沿据点,则是在屈野川北向的支流浊轮川边的暖泉峰。同样是党项苦力与驻军数量相差不大。宋军的主力,基本上都在唐龙镇、子河汊以及更名胜州的旧丰州的这第二道防线上。

党项人不擅营造,如今被逼着做工,累死累活,对宋人的怨恨不会少。虽说比起对大辽怨恨或许会差些,但毕竟宋人就在眼前。离得越近的仇人,这恨意只会燃得更旺,只要一点火星,就肯定能烧起来了。

耶律罗汉奴紧紧揪着萧十三幕僚的衣襟:“枢密是不是想立威,敲打我们这群没眼色的一下?刚刚清理完那群黑山党项,连歇也不准歇,就从黑山下把我们叫到武清军这荒地来。行啊,我们就在这院里跪下了,跪到枢密出来可以吧?这应当就能让枢密看到我们一片赤心了吧。”

耶律罗汉奴脸上笑得灿烂,双手却越收越紧,将萧十三的幕僚勒得脸色血红,脚像青蛙一样一下一下的抽搐着。

其他将领则冷眼看着耶律罗汉奴闹事,在黑山下两个月的征战,充实了他们的家底,但也将战马使用到了极限。这时候应该是回家去向妻儿炫耀自己的功绩,顺便休养生息。对于萧十三的命令,没人是心甘情愿。

房门吱呀一声响,萧十三推门而出,看到这一幕,脸顿时就沉了下来:“罗汉奴,在闹什么?!”

耶律罗汉奴松了手,让快喘不过起来的幕僚双脚落到地上,抬头笑道:“枢密终于出来了。末将只是疑惑哪里犯了过错,恶了枢密,正想问一问明白。”

对官位比他高得多的北院枢密使的愤怒,耶律罗汉奴也没有多少胆怯,只要手上有兵,就是耶律乙辛也不可能随意责罚。便是兴灵之地,五院部穷迭剌的儿子也必须分一半给六院部。身为六院部的夷离堇——也就是南院大王——的亲弟,并不用担心得罪耶律乙辛帐下的宠臣。

萧十三没跟耶律罗汉奴说话,先探手将幕僚搀起来,好生抚慰了几句,让他先下去休息。然后才对下面的将领说话。不是耶律罗汉奴,而是仅次于他的萧敌里,“宋人正在柳发川边修筑城寨,距此只有三十多里,有三千多黑山党项做苦力,看守他们的宋军在五千上下。”

“枢密是想攻打柳发川的宋军?!”

萧敌里的问题,没有一个人感到惊讶。被叫到紧邻旧丰州的武清军,不可能还有别的差事。

“柳发川那里我已经安排下内应了,只要大军一到,他们立刻就会起兵。”

“是耶律世良?”耶律罗汉奴在旁冷笑着:“听说他被派出去了。不过他一向糊涂,枢密不怕误了大事。”

“不是他。”萧十三无心多解释,耶律世良全军覆没在宋人手中的消息,他并没有说出来的打算,“内应在柳发川的宋军营寨,伪装成党项人,被宋人当成苦力使唤。”

“好个内应,一天宋人给多少工钱?”耶律罗汉奴挑起眼眉,“末将等人打过去,不是耽搁了他们赚钱的机会?”

周围的将领扯起嘴角想笑不敢笑。

萧十三只当没有听到耶律罗汉奴的挑衅:“不,只要你们各领本部贴近柳发川和暖泉峰就够了。剩下自有人去做。”

“就这样?”萧敌里疑惑的问道。其他将领脸上的嘲笑也都转成了困惑。耶律罗汉奴则是一点不信的摇着头。

“够了。只要你们装装样子就够了,剩下的自有那群黑山党项去做。你们到了武清军之后,宋人逼着黑山党项日夜赶工,仇怨结得可就深。了”萧十三转头看着罗汉奴,挑起眉,咧开嘴笑道,“辛苦了两个月,怎么会让你们再上阵拼杀?”

……………………

五千。

一万。

一万五。

两万。

数字不断的增多,多到让人难以置信的地步,最后停止在两万三千,直至让人麻木。

韩冈将整理出来的军功簿放下,对身边的李宪道:“不过真正属于被官军亲手斩杀的部分,大约只有一万出头。”

李宪五味杂陈的感慨着:“别说一万,就是五千。放在过去,已经可以让天子告祭太庙了。”

“可惜西夏灭国后的党项人,成了斩首功的行首,只做大买卖了。”韩冈笑道。

“记得种谔之前在盐州,于西夏军内乱之际趁势掩杀,斩首也是超过两万。”

“其中有多少是死于内乱,有多少是死于官军的斩马刀,恐怕种谔自己也说不清楚。”韩冈说道。

而且眼下最大的问题,有种谔在盐州的斩获在前,这些斩首功,很难换来更多的收获。

就跟当初刷交趾兵的斩首一样,党项人的首级越来越不值钱,贬值得很厉害。过去一个首级也许能换五匹绢,但如今能换来一匹绢就了不得。在元昊领军肆虐的时候,十几级党项人的斩首就能换来官阶的晋升,但到了如今,已经得从一百开始起跳。

十分标准的通货膨胀。

韩冈呼了口气,摇头苦笑了一下。什么时候辽人的首级也能如此低廉,就可以算是功德圆满了。只是不知道还需要多少年。

而且现在最让人恨的,是斩获的七百契丹骑兵。如果他们的身份能够让人信服确认,甚至能抵得上十倍的党项人,可是眼下能明确其身份的证据,除了武器、发式、战马和一些随身的小饰品以外,便没有更多了。这些证据,供韩冈等将帅做出判断已经绰绰有余,但战后论功却远远不足以让人确认,最关键的,就是没有旗号。

纵然河东、麟府两军心中发恨,也是没用。韩冈不会支持他们上报,而且即使上报,朝廷那边也不可能承认,否则日后保不准就是几千上万的契丹斩首冒出来了。

不过这样一来,等到朝廷功赏下来,可能会有些乱子。幸好城寨的修筑正在进行之中,还是有弥补的机会。以辽人一贯的作风,还有萧十三本人之前表现出来的性格,多半不会放过。

只是在数万辽军大张旗鼓的进驻武清军后,胜州这边上上下下就提高了警惕,辽人当也能探查得知。这么一看,辽人会不会来,又值得商榷了。

韩冈正想着辽人到底会不会来。黄裳匆匆跨进厅中:“龙图,柳发川大营急报,辽人举兵来攻,兵马上万,请求龙图及早发兵救援。”

李宪霍然而起,拍案大叫:“辽人果然还是来了!”扭头看向端坐如初的韩冈,“龙图……”

韩冈摇摇头,叹一声,抬头问李宪:“不知都知跟人赌过吗?”

李宪愣了一愣,“……偶尔有之。”

韩冈笑道:“看来都知不是赌徒。所谓赌徒,赢了之后,总想赢得更多,输了之后,则想着翻本。永远都被黏在赌桌上,最后倾家荡产。”

“龙图!”黄裳急叫道。从时间算,柳发川现在已经接战了。

“不用着急。”韩冈笑了一笑,道,“传令折克行,按之前议定的方略去做。至于浊轮砦和暖泉峰……”他抬眼望着李宪,“就拜托都知了。”

