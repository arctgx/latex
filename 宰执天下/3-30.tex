\section{第12章 共道佳节早(四)}

夜已深。

没有月光的月初之夜,玉宇澄清,并无一丝云翳。一条星河横贯天际,天穹上繁星点点,比起平日,数量竟似多上了十倍。

王雱抬头望着星光,辨识着天上的一颗颗星子。

紫微垣中,帝星明亮,辅弼诸星也同样灿烂。就是相星,不知是不是心理作用,在王雱看来,就显得有些晦暗。

昨日慈寿宫中的一番争吵,早就传出了宫来。

对于天子都说出了‘汝自为之’这句话,王雱也能知道赵顼当时被气成了什么模样。

当时若是雍王不多上一句话,天子恐怕还是会低头聆听,只是不去照做。但长年累月的的耳旁风刮着,天子终有撑不住的一天。但现在,由于昨日的事情,天子不可能短时间内放弃市易法,怎么说还能保着一阵。

市易务只要能再撑上一段时间,那些自持背景深厚而不肯合作的豪商们肯定要低头了。吕嘉问已经信誓旦旦的说了好几次,王安石和王雱,都决定相信他的判断。最近又给他加了一份差遣,已经准备重用了。

除了吕嘉问的位置有了些变化。曾布身上的七八个差遣,到了明年的科举之后,也要有变更了。

他将去担任三司使,替换即将前往秦凤路转运司担任都转运使的薛向,而让吕惠卿接手判司农寺的工作。不过中书五房检正公事的职位还是得让曾布兼着,不然曾布那里肯定不会答应。

这个调动,想必曾布也能明白。司农寺和中书五房检正公事这两个关键的位置,不可能一直留在一个人的手上。之前是因为吕惠卿丁忧,章惇出外,才造成的逼不得已的局面,现在当然要改回来——新党之中,并不需要两个核心!

王雱向自己的小院走去,刚刚穿过一重小门,一阵激烈的争吵声,便从二弟王旁的院子中传了出来。有弟妇庞氏的哭泣,也有弟弟王旁的叫骂。

夫妇两人的争吵声打碎了深夜的寂静,王雱摇了摇头,带着身后的小厮一起走快了几步。

听见王雱回来的动静,萧氏从桌前站起,迎了上来。房中听候使唤的两个婢女已经睡了,萧氏便自己上前去,帮着王雱将身上防寒的斗篷脱下来。

将猩红色的大氅挂到墙边,萧氏随口说着:“二叔那边好像又在吵了。”

“别去管他!”王雱难隐心头的不快,重复了一句,“别去管他。”

王雱一喝,萧氏低头整理着王雱的衣袍,聪明的不再提起此事。

王雱换了身轻便的衣服,坐到了桌边上。看着萧氏坐在对面,拿着一块布料在飞针走线,看着渐渐成形的样子,是一件给小孩子穿的外袍。

王雱心里还想着二弟王旁的事。王旁自从儿子出生后,觉得儿子跟自己长得不像,就天天跟浑家吵架,弄得家中鸡犬不宁。王雱作为兄长也不好去劝,只好躲远一点。

只是日闹夜闹,实在不成体统,昨日还把娘给气到了。这件事要传出去,外人又该怎样去看?

国事就已经够让人烦心的了,家中却又是让人不得安闲,。王雱突然觉得心脏有些发慌,按了按心口,脸色也白了起来。只是他怕着妻子担心,竭力保持着平静自如的神色,让她给自己倒了杯热茶

喝了几口热茶,王雱感觉好受了不少。左手不用再按着心口,脸上也多了点血色。

萧氏没有觉察王雱一瞬间的不适,低头绣着儿子的小外袍,问着丈夫:“听说荆南那边昨天又有好消息传回来?”

“章惇前日降伏了梅山的苏甘,设了安化县。等过几日他回来,朝廷就会又有一场献俘大典了。”

说着章惇的功绩,王雱口吻中不脱讽刺的味道,章惇在荆南的表现,不如王韶远矣。梅山蛮也没有吐蕃人那么凶悍。就是有两仗打得可圈可点,但领军的两个主要将领可都是陕西人。

萧氏可不管丈夫对章惇是什么评价,手上的针线一停,追问道:“那愿成大师可以得授紫衣了吧?”

“应该吧。”王雱点了点头,“这样给的紫衣才是名正言顺。”

前些日子王雱儿子日夜啼哭,便是愿成给治好的。不过愿成想靠这个功劳就想讨上一件紫衣,未免就太过了一点。

救治自家孩儿,那是私恩。而高僧大德才能得赐的紫衣,却是朝廷的恩典。要是因为,把朝廷恩典当做私恩与人,试问如何可以服众?

公器私用的事,韩琦、文彦博做过,他们做宰相的时候,还举荐过两名得他们欢心的医生为官。但王雱知道,自家父亲绝不会答应,而王雱本人也不愿这么做。

就是浑家萧氏有些不高兴,自家儿子日后说不定还要求人,怎么能如此吝啬一个官位。

正好此时章惇从荆南寄信来,说荆蛮畏惧符咒,要向王安石讨要个有口才的道士去荆南。愿成虽然不是道士,但他的口才很好,又会符箓咒术,就正好派得上用场。

愿成自到了荆南,便事招摇得很,自号经略大师。只是跟着李资、明夷中一起进山去劝降荆蛮的时候,吃了大亏。李资、明夷中等官吏全都被杀,只有愿成因为荆蛮虔信浮屠、崇信鬼神,才被放了出来。

这样的和尚,当然远远比不上在熙河路立有殊勋的智缘,想必他也不敢要求太多。

‘一件紫衣,也该满足了。’王雱心里想着。慢慢阖起了眼睛,最近想的事太多,头有些疼,精力也有些不济。

萧氏这时拿起手上的衣服,对着灯火比了一比,左比划,右比划。放下来后,对丈夫道:“这吉贝布还真是让人喜欢,比起绸子可要厚实多了,又暖和又轻柔。照着火,根本都不透一丝光。”

“吉贝布?”王雱睁开眼睛,不快的问道,“怎么买这么贵的布料?!用朝中发下来丝绢做衣服不行吗?”

“不贵啊,这又不是琼州黎人的吉贝布。听说是陕西今年刚出来的,自熙河来,价格低了不少,而且一点都不差。”萧氏又举起了只缝起了一半的衣服,给王雱看着,“官人你不是,难道王枢密和那个韩玉昆在给中书的公文里面都没有提?”

王雱仔细,好像没有这么回事,等明日去中书查一查旧档好了。若京城市面上的吉贝布,真的有了出自熙河的货品。靠着足够的税入,河湟很快就能平定了下来。

想着此事,王雱都有些佩服起在熙河开拓了两三年的王韶和韩冈,“一边攻城略地,一边种田织布,这一步步,走得还真是够快的!”

“谁说不是,前两年还听说是要朝廷用几百万石来养着熙河路的兵将,转过脸来,现在就有布料出来了。”

“王韶和韩冈能点石成金啊……他们在熙河之事上用心之深,由此也可见一斑。”王雱感慨着。

开荒种地很多官员都知道,但种什么才能稳固根基,这不是普通官员能想到的了。光是种粮食,不过让一路百姓吃饱,多上一点税赋,根基只扎在当地。但换成是棉布,运到京城发卖后,天下人都知道熙河路有这个特产了,根基已经是扎进了京中。

再过几年,熙河吉贝布的名号传遍天下,就算是文彦博,也不敢轻言放弃。

“是韩冈吧。”王雱猜测着,“其父韩谦益管着熙河屯田事,这两年的熙河丰收都是他的功劳。想必木棉的种植,也少不了他的一份。”

“对了……”萧氏放下了衣服,“说起韩冈,奴家要问问了,小姑的事该怎么办?不是说要跟韩冈结亲吗?前两天还听娘在叹着,这一耽搁就耽搁了一年。小姑转年可都要到二十了。”

王雱一下皱起眉头,脸也沉了下来。提起这事,他就有些心头火。那个韩玉昆,宰相家招他做女婿都不肯一口答应,偏偏要考上进士才肯给个明确的回复。

若是在其他地方,都是女方要求男方只有考上进士,才能成婚。如果考不上的话,女方就另寻他人了。如果女方都不要求女婿的功名,便将女儿嫁过去,哪个士人不是忙着点头答应下来,可有一个像韩冈这么做的?!

从王韶传来的话里,韩冈是不肯被人说成是借助宰相岳父的门路才考上的进士,所以要拖到明年的三月后。才高之人,心高气傲一点王雱能理解,但韩冈这一拖,妹妹可就又大了一岁。

“要是三年前,在进士中挑上一个就好了,小姑十七岁的年纪也正正好。”

“现在还说这些有什么用,都过去三年了。”王雱说是如此说,不过在他的心中,当时若他在京城,肯定要帮着妹妹的选个合适的。绝不可能一拖三年。

“小姑心里怎么想的,你们去问过了吗?”

“二姐还要怎么样?能跟韩冈差不多的人选,现在也不好找了。”王雱抬眼问着妻子,“怎么,是不是二姐跟你说了些什么?”

“倒不是。”萧氏摇着头,“二姐倒没说,心里在想什么也看不出来。”

宰相之女二十而未嫁,外面肯定有不少不堪的猜测。对于二妹王旖,王雱心中也有着一份愧疚,“明天我去问问她好了,看看二姐是个什么心意。”

