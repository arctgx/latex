\section{第30章 随阳雁飞各西东(一)}

“韩学士?!” 

“正是韩冈。” 

面前这位极为年轻的宋国重臣只是拱了拱手,萧禧的身后便是一片倒抽冷气的声音。 

接伴使是韩冈的连襟,吴充的次子吴安持,交接时倒是没花什么时间,但亲自面会辽国使臣时,却不意有了这样的反应。 

“韩学士名垂天下,鄙国亦是尽人皆知。为种痘一事,都是感念颇深。今日得见尊颜,不免惊讶,倒是让学士见笑了。” 

“不敢当。韩冈一点虚名,所谓的功劳也是遵循师道的结果。”韩冈笑了一笑,“倒是萧林牙的大名,韩冈也是闻之久矣。” 

辽国使团抵京后,将会在都亭驿中先休息三五日,然后安排时间,上殿递交国书。见于紫宸殿,宴于垂拱殿。到了正旦大朝会,则是再上殿一次,在大庆殿参与朝会和宫宴。之后离开前还会有一次陛辞和宴席。这就是最基本的待使礼仪。 

不过今年的正旦大朝,因为天子重病的关系,很有可能不再举行,而是辍朝祈福。这样一来,就可以将三次减少到两次。但韩冈这边的压力只会更大,大朝会都不举行,等于是下大赦诏,代表大宋的皇帝差不多就剩一口气了。敲起竹杠,就只会狮子大开口。 

“萧禧奉君命而来,不知何日可以上殿递交国书。”携手走进都亭驿中,萧禧便迫不及待的试探道。 

韩冈暗中无奈,终究是不可能瞒过去的,“圣躬不安,暂且林牙少待时日。” 

萧禧脸上顿时就掠过一丝狂喜之色,他连忙拱手,“不意贵国天子竟然御体违和,请恕萧禧妄言之过……不过有韩学士在,当能保无恙。贵国皇帝乃是鄙国天子之叔,尚父平日里可是一直都挂念着。” 

韩冈脸色不愉的看了回去。澶渊之盟中,宋辽约为兄弟之邦,天子之间的关系以世以齿论,赵顼是辽国前后两代皇帝的侄儿和叔叔,也是如今辽国新帝的叔叔。而耶律乙辛只是辽国尚父,虽然带个‘父’字,不过是礼敬老臣的称号罢了,想跟赵顼攀关系,够不够资格?!等篡了位再说吧。 

被韩冈这么一瞪,萧禧低了低头,道歉道,“萧禧失言了。” 

难得辽人低头,韩冈点点头,也没就此事纠缠。转而道:“大使远来辛苦,一点薄水酒已经布置好,还请先行入席。” 

韩冈和萧禧并肩走入厅中,丰盛的接风宴上,两人并没有再提起有关公事,仅仅是喝酒而已。 

宴后,韩冈将朝廷依照惯例赐予辽使的钱物颁下,便告辞离去。不过明天他还要过来,在萧禧离京前,韩冈都得每天都来做陪客。 

副使折干在韩冈面前显得局促不安,连话都没怎么说,等接风宴后,韩冈告辞,他才算松了口气。急急忙忙的跟萧禧抱怨:“怎么来了这尊大神?!” 

萧禧也想问!韩冈是什么身份,怎么纡尊降贵做了馆伴使!?若他出使北朝,尚父都得迎出府门外。前面韩冈刚一到场,将姓名一报,自家的气势立刻就被压制住了。 

北朝的医疗水平远远不及南朝,缺医少药,痘疮这个疾疫,比起宋人这里远要严重得多。大辽国中,萧禧也好,折干也好,甚至耶律乙辛本人,他们的家中都有为数众多因痘疮而夭折的子女。 

在种痘法确确实实减少了天花的发病之后,韩冈这个名字,在辽国,即便是最北边的生女真那里,都是响亮得很。无数贵胄对韩冈崇敬有加。虔信浮屠的辽人,基本上都视韩冈为药师王佛帐下弟子转生,连药王庙都立了像。有几个能在他面前能不恭恭恭敬敬的? 

一时间萧禧头疼兼牙疼,连胃都隐隐疼了起来。

不要说折干,使团中的其他人眼见着都视韩冈如神明,这可怎么谈判?说不定转眼就能将底给泄露了。这不是没有前例,在过去,就有使宋的使节因为跟宋人关系处得太亲近,将谈判的底限全都泄露了不说,还帮着宋人减少讨价还价的阻力,屁股偏得不能再偏了。 

苦恼了半日,萧禧忽然抬头,“前面韩学士报的官职是什么?” 

他问着下属。 

一名记性好的从官立刻报道:““资政殿学士、翰林学士、判太常寺。还有东莱郡开国郡公,检校……” 

“这些虚名就不要报了。”萧禧抬抬手,前面两个学士衔就已经足够了,他摇摇头:“怎么有这种兼差法?” 

以萧禧对宋人官职的了解,虽然很浅薄——那等复杂的官职系统其实也没几个宋人能弄得清——可也是知道资政殿应是给宰执官的,而翰林学士,则明显低了一级。 

“也是好事。”想了一阵后,他突然说道,“以韩学士的身份,绝对是应该晋身两府的。让一名资政殿学士做翰林,做馆伴使,这不是褒赏,而是贬责。由此可见,宋人心的有多虚。” 

“但韩学士可是药师王佛……” 

“那又怎么样?!”萧禧一口打断道,他咬着牙,狞笑道,“南朝的太祖还是弥勒佛转世呢,还不是给亲兄弟害死?” 

不回天上,那也不过是个常人而已,又有什么好怕的。割肉刀都亮出来了,萧禧可不会收回去。何况尚父已经安排了人手配合自己,南朝君臣在年节前就会收到消息! 

年节前的羊是最肥的,割上一块就好回去了过年了。 

…………………… 

吕惠卿站在长安城城头上的敌楼中,远眺着白茫茫的无尽天地。 

他已经看了不少时日的关西风月,如今终于到了要离开这片天地的时候了。 

“枢密!” 

吕惠卿转回头,淡淡的看了身后出声的随行官员一眼。 

平日里就已是极为谦卑的属僚,今日更是加倍的恭谨,“城上风大,还是先下去吧。” 

“再等等。”吕惠卿笑了一笑,“也没多少时间能在这里看风景了。”他挥了挥手,示意这名官员先下城去。 

面对永兴军路的帅臣兼京兆知府,和面对统掌天下军事的枢密使,当然是两种截然不同的情形。 

世人都道做官好,但其中到底有多好,只有亲自侧身其间,才能感受得到。 

重又得到重用,吕惠卿的神色间却不见喜愠,这让下面的官员更加敬佩了三分,宠辱不惊的城府和心胸自然都是宰相才能拥有的。 

尽管吕惠卿连辞让都没有就立刻接了下来,可谁都知道,在这一局面混乱的时候,身在外地的官员只要稍作推辞,就有可能发生翻天覆地的结果,远远不比人在京城时那般犹有余暇。

属官弓了弓腰,便退了下去。 

“枢密。”吕惠卿最为亲信的门客郑希,这时候神色正激动着,“这可就是要回京城了!” 

“是要回去了!”吕惠卿是没有想到,他竟然能这么快以枢密使的身份回到京城中。之前朝廷的安排,明明是想维持稳定,可转眼之间,就变成了旧党彻底崩溃的局面。朝堂之诡谲,让人不由心惊。 

司马光形同受责,吕公著被贬出外,王珪这一回更是连家门都赔进去了。新党大兴……吕惠卿忽而冷笑。天子的决断力,在之前听说冬至夜前前后后之后,吕惠卿已经感受得很清楚了,现在不过是进一步的确认罢了。 

“果然是圣天子在位啊。” 

细微的声音从前面传来,郑希惊疑不定的看着吕惠卿。他这是在称赞吗?神色像,但语气怎么听也一点不像。 

“昨天的从环庆路传回来的消息,平伯你可知道了?”吕惠卿忽然道。 

“是辽人在兴灵的军队南下?”郑希问道。 

“三千兵马可不是小数目。”吕惠卿慢慢的说道,“还请平伯帮忙起草一份札子,请朝廷速速任命知兵良臣镇守永兴军,迟恐不及!” 

“郑希明白了。”郑希点头。 

如果是正常的人事更迭,是接任的官员先到任,现任官员要等顺利办理交接手续、并查对过账目之后才能正式离开,而眼下吕惠卿接受的诏命,却是非常形态,并不需要等待继任者。甚至很有可能,继任者还有十天半个月才能抵达。对帐封库的对象,将是代掌军政事的副手和幕职官。 

吕惠卿在长安身兼三职,手挽一路军政,但他一旦现在便离任上京,京兆府的政事将由府中通判处置,永兴军路的军令归于兵马副总管兼经略副使,不过由于是武将的关系,经略司中的军政,则是由经略安抚使司判官来代掌。三家分权的局面将会持续到下一任永兴军路经略安抚使兼京兆知府上任时为止。 

“在新官接任之前,就先在京兆府留上几日,行装收拾慢一点,当也不会惹怒了皇后。”吕惠卿又说着。 

郑希本是想点头,但脖子却僵住了。吕惠卿既然已经接下了枢密使一职,现在却还要拖在关西,这是想做什么? 

郑希的惊讶看在眼里,吕惠卿淡然一笑:“兴灵辽师南下,兵胁关西,我忝为枢密使,西府之长,怎么能这么甩手就走?” 
