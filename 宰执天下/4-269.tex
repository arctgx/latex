\section{第43章 庙堂垂衣天宇泰(16)}

所谓明君,不是说赵顼能明察秋毫,明辨是非,只是说他是个聪明人而已。

聪明人当然也会为一时激动的情绪所掌握,但在激动过后,还是会恢复冷静,会做出符合自己利益的选择。

赵顼即是一个父亲,也是一位君王,他的利益是家国天下,所以韩冈根本不担心他最后能将自己怎么样。

召入京城,给个高位是肯定的。襄汉漕运和种痘免疫,两件事加起来这么大的功劳,论理也该给予封赏,只是权力要削减一些。所谓宠以厚禄,削其权柄。毕竟功高盖主,总是会惹来忌惮。不过谅赵顼也不敢做得太过分,韩冈就有这份自信。

而且这也正合韩冈的心意,接下来的时间,他打算多分一些在学术上。没有多少士子敢于投奔被治罪的学者门下,传习他的学术。但如果是因为功劳太大而被供奉起来的人物,又是在交通往来最方便的京城,来自天下四方的士子们肯定会趋之若鹜。

回到内院,韩冈在小厅中坐了下来,对着身前的几位妻妾笑道,“过两天就可以把手上的俗事都放一边去了。”

“要回京城了?”周南问道,她的身子已经开始显怀,盈盈可掬的腰肢也圆润了许多。

“应该是先被弹劾。”韩冈撇了一下嘴,摊开双手,状似无奈的笑着:“建国公的事实在是太不巧了。”

王旖她们在内院也同样得到七皇子的消息了,在韩冈回来时,脸上都有着掩不住的惊讶。都是老夫老妻了,她们很容易就从韩冈的神色语气中,发现他根本没将七皇子病夭当一回事。联系起韩冈上书的时间,两件事巧合到难以想像。

王旖犹豫了半天,小心翼翼的问道:“官人是不是事先就知道建国公会在这几天夭折?”

“怎么可能?!”韩冈愣了一下之后,笑得差点把嘴里的茶水喷出来,忙将手上的茶盏放下:“为夫也不是能掐会算的半仙,不会把弄蓍草,更不会烧乌龟壳,怎么可能预料得到建国公会在这时候出事?只能说实在是太巧了。”

“真的是太巧了。”王旖叹了口气,她当然知道自己的丈夫对算命、占卜不是很放在心上,更没见他摆弄过算命的蓍草,书房中唯一跟占卜有关的器物,还是堪舆用的罗盘。

“三哥哥,弹劾不会有事吧?”云娘关切的问着。

“可能一开始会有些小麻烦的,但天子毕竟是明君啊,岂会让不实之罪加在我这个功臣的头上?”

韩冈伸懒腰时的轻松自在,让四名妻妾看不到半点忧心之色。

他拿出来的牛痘之术,还有过往的发明,就是他手上最大的保证。襄州城中的卫生防疫局,如今成了最热闹的去处,每天在门外打转的小贩都有几十人,驱逐都驱逐不了,最后只能放着他们赚钱,十天下来,光是小买卖就让他们已经赚了不少。而种了痘的小儿更是以一天五百人的速度增加,感激韩冈的家庭也在飞速的的增长。

“对了。”韩冈伸过懒腰后,坐直了身子,惫懒的神色收敛了些,“东跨院那边这几天可能有人家里会有急事,也许是父母重病、也许是幼子夭折,反正都要必须要紧急离开的事。别忘了准备一些盘缠以壮行色,以一人五十贯的标准。”

韩冈的吩咐,王旖和周南最先反应过来,王旖点头答应了。素心很快也明白了,摇摇头,没说什么。而云娘片刻之后,才终于想通,顿时柳眉倒竖,“平常好吃好喝的供奉着,才到了关键时候就要开溜,那些酒菜全都喂狗了!给他们五十贯做什么?五十文就够了!”

“那怎么行?”韩冈笑眯眯的摇头,“只用处几百一千贯,就能换个好名声,实在是太便宜了。”

“官人哪里还缺名声?能让离开的人没脸再回来就够了。”严素心粉面上挂着冰霜,也是一脸不快。

住在东跨院的都是投奔到韩冈门下的宾客,平日里好吃好喝,按月支俸,逢年过节也都少不了一份钱物,换季时还有几套新衣。花钱养着,不求他们同生共死,但那么早就往外逃,还真是让人心里觉得呕得慌。

“平日里冲着三哥哥跟狗一样一个劲的摇尾巴,到了主人家遇险的时候,却没了看门狗的忠心耿耿。真还不如多养几条狗。”云娘咕哝着,只让韩冈一人听见,

“疾风知劲草,板荡显忠臣。谁值得用,谁不值得用,今次之事上,便能见端的。”韩冈神色淡然,半点也看不到芥蒂,“这样好的机会,可是用钱都买不来的。”

“官人放心,奴家会安排好的。”王旖再次郑重的回答。

韩冈站起身:“好了,除此之外,也没别的大事了。今年的纲运算是成功了,六十万石全都通过了方城山,过两天就能全数抵达京城。不知还有多少人在意这一件事了。”

“官家和政事堂总不能干没了官人的功劳!”周南说道。

“当然不会,只是多半会耽搁一阵。”韩冈展颜笑道,“还是想想这一次为夫能得到几份弹劾吧,不知能不能达到两府的水平?”他冲着几位妻妾开着玩笑,“几年内,为夫是没办法晋身两府。若是这一次能在弹章上能与宰辅们一较高下,也算是提前享受一下两府的待遇了!”

……………………

赶在京畿水道封冻前,襄汉漕运的六十万石纲粮终于成功运抵开封城西的合口仓中。

但在这个极具象征义的日子,原本应该在京城中引发轰动,掀起一片喧嚣的成果,却被更大的轰动给遮掩过去了。

没人能低估种痘对世人带来的影响。京城内外,酒楼茶肆,到处都在议论着此事。

怀疑种痘法的几乎没有。韩冈的盛名在外,世所公认的医道权威,传说中的药王弟子,普通百姓立刻就相信了八九分。而他对种痘法由来的详细描述又在情在理,化解了士大夫们心中的疑虑。

当然,世人相信种痘法的关键,还是因为韩冈已经在京西开始推广,据说行之有效,在上书之前,便已在几千人身上试用过了,并非空口说白话。

只可惜明明是一桩可喜可贺的美事,偏偏因为七皇子的夭折,让种痘免疫法平添波澜。原本应该催促天子尽快应允了韩冈的奏疏,早日在京城和天下诸路推行,却因七皇子建国公之事,一时间没有人敢于上奏,触天子霉头。反倒是御史台中御史,从中看到了机会。

何正臣得意的端着酒杯,与同僚对饮。

他们这些做御史的,被安排到这个位置上,就是用来威慑重臣。能成为监察御史、或是监察御史里行,基本上都是年轻气盛的低品京朝官——在官场,四十岁以下都可以归入年轻、新近的范围——而御史们的弹劾,不论成功还是失败,都是打响名声的良机。有了足够响亮的名声,是晋身高位的前提条件,大半高官显宦,都是从选作御史开始起家,数年便身登侍制的不在少数。

韩冈是个例外。而且他年齿之幼、官位之高,实在是让人嫉恨不已。不过他之前能一路飞升,在何正臣看来,主要还是靠了得到圣眷,加上一点机缘,将一分的能耐,说成了十分。世上有才有能者为数众多,不独韩冈一人,为何偏偏他能够例外?——还不是天子看重的缘故。

一直以来,天子对韩冈的看重,让御史们对他无可奈何。别看韩冈去岁从广西回京之后,天子对他冷淡得很,但御史们送上去的弹章根本都不批复,全数留中,不让任何人干扰到韩冈在京西。

“不过眼下韩冈圣眷已衰,”何正臣冷笑着,“该是彻查他的时候了。”

“韩冈是奇才,他是靠能力得到天子眷顾。”同为御史的黄廉,对韩冈的评价比何正臣要高些,“可惜他辜负了天子,要不然,我们现在还等不到这个机会。”

都是有子有女的人,天子的心情也能体会一二,要不然也不会这么快就写了弹章上去。

“不过天子还是会用他的,”黄廉继续说着,“毕竟种痘免疫法还是需要他来主持。”

“使功不如使过。”何正臣提着酒壶给黄廉和自己倒酒,“推广种痘法的功劳太大了,天子怎么可能放心得下?”

“世上多有因人废事,为了种痘之术,韩冈不会被贬斥去远州,多半还能留京。不过天子对韩冈,肯定是会先责问,夺了他的官称,再让他戴罪立功……恶了天子,两府、两制都不用指望了。”

“政事堂那边有消息了。”一人匆匆闪进两名御史的包厢。

何正臣抬起眼:“是派人去京西体问,还是招韩冈入京?”

“是招韩冈入京,并遣中使去京西体问。”

何正臣和黄廉相视一笑,接着同时板起了脸,“中使?彻查朝臣不法事,自当出自御史台,岂是阉人可以干预,京西走马承受枉食君禄,不能纠举韩冈,这个帐还没找他们算!”

