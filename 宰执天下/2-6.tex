\section{第二章 边声连角不知眠(二)}

【第三更,求红票,收藏】

王厚见韩冈事事为自己父亲着想,心中欣喜:“愚兄也是这么想的。家严已经有所准备。”

韩冈不似王厚那般乐观:“能证明机宜先明之见的,是不是就只有元旦时,发给经略司的两封急报?”

“三月初,两部调集族中大军时,家严当时在永宁寨,听说后又发文给李师中,提醒他加强防备。”

“也就三封啊。”韩冈沉吟了一下,道:“得去架阁库,把机宜这几封有关托硕隆博二部的文字,都拿出来保存好,以防莫名其妙就不见了。”

“家严已经做了。”王厚笑道:“吃了那么多亏,哪能再糊涂。没了文字,那就任李师中泼脏水了。”

对于王韶这么小心谨慎,韩冈可以理解。王韶的才智本高,自己能想到的,王韶当然也能想到。何况对于李师中和窦舜卿的阴毒手段,王韶可是切肤之痛,当然会预防着。

韩冈点点头:“既然机宜早有准备,我就放心了……”他也笑道:“机宜的先见之明,传到京中,让人知道他这蕃部提举也不是白做的。”

王厚失笑,韩冈拍马屁的时候可是难得,只是他的脸色又正经起来,“不过玉昆你有所不知,秦州的蕃部提举可是就要再多了一个。”

“再多一个?这话怎么说?”韩冈惊讶道。

如今管理秦州缘边蕃部的官员已经有三人,王韶是提举西路蕃部,向宝是管勾西路,张守约则是管勾东路。就这么点大的地方塞了三个人来管,张守约管着东部,那里没什么事,当然,功劳也少。但西路其实就是指的河湟开边的,王韶、向宝,一个提举、一个管勾,就是在为此争着。如果再添一人,不可能是在事少功少的东路,只会是在功劳多多的西路。

这是还觉得秦州不够乱吗?

“天子钦点西京左藏库副使,阁门通事舍人高遵裕,为秦州西路蕃部同提举。”王厚说道。

高遵裕这个名字韩冈好像在哪里听过。不过他最近接触过人多,说得话多,听过的名字也多,使得其中许多只留下一点模糊记忆。他问道:“这高遵裕是什么人?”

王厚反问:“太后姓什么?”

得到提醒,韩冈想起来了,是高太后家的人,“……是太后的叔叔。”

太后亲叔为秦州西路蕃部同提举,往好处想,赵顼把自家的舅公派来秦州,当然不会是为了跟王韶打擂台,相反,算是为王韶准备的一大助力。但坏处貌似也不少,外戚在士大夫中并不是很受待见,王韶即便得到高遵裕的支持,朝堂上反变法派的重臣们的立场也不会改变,反而会更加兴奋。

而且,为了满足高遵裕的功名心——能放弃京城的优厚生活,而到秦州喝西北风,他就不可能是个视功名利禄如粪土的人——王韶就必须在一些事情上迁就于他,还要推让功劳给他。而且高遵裕不会单人上任,他有门客、有幕僚,有亲友,这些人,同是会来分大饼的。。

这下有好戏看了,韩冈想着。他从不怕与人争功,只怕没有立功的机会,反正高遵裕来秦州,第一个头疼的并不是他韩冈,也不是王韶,而是向宝。

………………

韩冈和王厚说着闲话,而经略司中,李师中和他的属官们也都在商议着如何处理隆博、托硕两部的问题。

正厅上,李师中居中高座,右手边,窦舜卿坐在第一位,只是眯着眼似睡非睡。窦舜卿的对面是向宝,秦凤都钤辖双眼如电,神色中满是跃跃欲试,迫不及待。而后,参议、参谋、机宜等幕僚官坐了一片,王韶的位置就在他们中间。

李师中开门见山:“隆博、托硕以细故起大兵,渭源至古渭百数十里,皆有其兵马出没,厮杀无一日而绝。现今两部的使者在西北各部中四处奔走,厚赂求盟。如不及早平息乱势,秦州以西怕都免不了要烽火连绵。不知诸位对此有何高见?此二部又该如何处置?”

“管他们为得什么事,即乱我秦州,那就一个也不放过!”向宝豪气迫人,他对蕃部一向秉持着强硬的态度,对不恭顺的蕃部,总是想着先打一顿再说,“经略,且由末将带兵去,管把他们教训得服服帖帖。”

李师中不置可否,转去问王韶的意见:“子纯,你意下如何?”

王韶心中正骂着,两部即将开战的文报早早的就被呈到了经略使的案头上,若李师中早做准备,说不定今日两部之乱都可以消弭于无形。但李师中一拖几个月,连点预防都不做,现在事情闹大了,王韶觉得更应该先追究李师中的失察误事之罪。

当然,王韶知道自己的想法不现实。他只能提醒:“河州木征那边呢?他的弟弟董裕娶的是托硕部的女儿,他不会不出兵。”

“朝廷行事哪能顾忌那么多,瞻前顾后,岂不徒惹蕃人笑?!”

“子纯,”李师中唤起王韶的表字,亲热得就像叫着自己的老友,“你还是觉得该慎重起见?”

王韶不上当,“出兵与否,经略一言可决。但未虑胜,先虑败,夫庙算多者恒为胜。如今只是庙算而已,还要问问在座各位的意见。”

“子纯说的是。”李师中遂一个个的问起僚属们的意见,而他们见解,无外乎谨慎行事和大胆用兵两种看法。最后也就窦舜卿还没发言,只是看他花白的双眉下,一对眼睛紧闭着,让人觉得他的意见有不如无。

“好了,”李师中最后总结陈词,“皇城是要立刻出兵,王子纯则是觉得要谨慎一些……”

“不,”王韶突然打断李师中的话:“经略误会了。职部倒是同意向钤辖的意见,平乱以速战速决为上,但必须要防备好木征。”

听见王韶支持自己,向宝先是愣了一下。然后满不在乎的笑道:“木征小儿辈,不足虑。即便他敢来插手两部之事,我也能让他丢盔弃甲而走。”

王韶当即反对:“真的要出兵,对付两部倒不需要钤辖出马,牛刀杀鸡反为不美。不如钤辖领军屯于永宁,以防备木征。古渭寨本有三千军,且西路都巡检刘昌祚素有威名,让他直接带兵去压服两部,也就足够了。”

“刘昌祚只箭射得好,一勇之夫,怕不如木征心机多。”窦舜卿今天第一次开口,在座一齐心道,原来他没睡着啊。

“不然,刘昌祚久历兵事,勇武智计皆为长才。木征不过一蕃人,如何跟我军大将相提并论?”

“老夫看他倒是寻常。”窦舜卿慢悠悠的说着。

秦州以西的蕃部,本归王韶、向宝两人统管,论地位,向宝一路都钤辖,当然在王韶之上。但放到蕃部这件事上,王韶的提举要比向宝的管勾高上一级,换句话说,在秦州西路蕃部事务上,王韶的话语权是要高于向宝的。但窦舜卿位高权重,他的话,份量犹在向宝、王韶之上。

“那此事就交予皇城了。”向宝的本官是皇城使,李师中不知因为什么原因,一直拿向宝的本官称呼他,而不是钤辖差遣,“最近西贼在环庆蠢蠢欲动,庆州的李复圭又是个好大喜功的性子,那里可能要出些乱子。秦州的兵要防着,一人一马都不能动。”

李师中以好大言著称,也就是一个大嘴巴,说起临路帅臣,一点也不避讳。在座的都在想,这话传到环庆经略使李复圭耳里,恐怕秦凤、环庆两路就要打起嘴仗来了。

李师中从不在乎这些,说完秦州不能调兵,继续道:“甘谷城要防贼,伏羌城又要支持甘谷,都不能轻动。本经略能给皇城的,就只有永宁、古渭二寨中的兵马。不过皇城还是管勾西路蕃部,有需要时可以征调周围诸部兵马。”

向宝耐着性子听着李师中絮絮叨叨的说了一通,只听到最后几句,闻之大喜。他一直能盼着出兵。,

“不过,”李师中给出兵加上前提,“必须确认木征开始匡助托硕部时才可动手。如果只是两部相争,由得他们自去。本经略会传令缘边各部,让他们不得插手托硕隆博两部之事。如有蕃部敢违我帅命,本经略自会遣人理会。”

不得不说,李师中做事还是有些分寸,不是按照向宝的意见,将两个斗气的蕃部一齐处置,也不让他立刻动手,而是等待木征的行动。

向宝对此略略有些不满,但还是上前接令:“末将遵命。”

“对了。”李师中像是突然想起一件事,“韩冈不是精通军中医术吗?他在本路军中也颇有些名气,有他随军,应该能稍稍安定军中人心。正好这也是韩冈的职司,就让他跟着向皇城一起去古渭。”

“韩冈今天病了,恐怕近日无法随军同行。”王韶为韩冈开脱,不然他进了向宝帐下。向宝只要动动嘴,就能将他治了罪。

“那就请他抱病出征。”李师中的决定毫不动摇,“为国岂敢惜身,相信韩玉昆有这份忠心。”

