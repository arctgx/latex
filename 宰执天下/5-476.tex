\section{第39章 欲雨还晴咨明辅(五)}

夜色笼罩着宫城。

内外皆是一片深黑,只有点点火光,在各处宫室里闪着。

一串灯火横过眼前的黑暗,从保慈宫的方向,正往福宁宫的后殿过去。

坤宁宫位于宫城的最北面,也就是最后面。

南面是福宁殿。距离三位宿直宰辅的位置很远,而保慈宫就在福宁殿正西,如果从前面走,在坤宁宫的方向上根本就看不见。那是故意从福宁殿后走过去的。向皇后明白,那是给自己看的。

六月底的夜晚,依然是燥热的。只是风吹过高耸的殿宇,原本干燥,就变得清冷甚至阴森起来。就像是第一次走进大庆殿,那股迎面而来的阴寒,怎么都忘不掉。

向皇后双手环抱着上臂。

单薄的褙子下,大宋最尊贵的女子,正在夜风里瑟瑟发抖。

就算是在冬至夜之后,她也没有正面面对过自己的姑姑。

每次见面都是不苟言笑的高太后,给她带来的是十几年的畏惧。本来以为已经结束了,可到了今夜,向皇后终于明白,那种畏惧,依然藏在心底。

结发夫婿说自己害了他,儿子虽小,却已经有了偏见。都说三从四德,可丈夫、儿子都靠不住,到底要依靠谁才行?

向皇后眼睁睁看着那一道流光汇入了前方的宫舍之中,没有受到半点阻碍。

“宋用臣还没回来?”她慌乱的问着。

结果显而易见,人人不敢抬头,也没人能给她一个安心的答复。

……………………笃……笃……笃。

从远处传来的声音,有节奏的响着。就像眼前的那一串灯火,随着步伐,轻轻的摇晃着。

那是太后自己拿着拐杖在走,从保慈宫,向着福宁殿一路走过来。

谁也没想到太后这么快就从保慈宫中走出,被派去‘护卫’太后的班直,显然没有起到阻拦的作用。

声音越来越近。

拐杖的末端每砸在地板上一下,杨戬的心中都会抽上一记。

很好笑吧。

杨戬能感觉得到对面的同伴投来的视线。自己脸上的皮肉正一抽一抽的,随着太上皇太后越走越近,腮帮子就跳得越来越厉害。

他完全压不住心中的惊悸和恐惧。

自己是皇后在福宁殿提拔起来的,是在冬至夜侥幸得了皇后的青目,攀上了梧桐枝。平曰里都是福宁殿中备受尊敬的,他自己也曾幻想着,再过二三十年,爬到入内都知的位置上。

但他现在不敢想了。

太上皇太后气势汹汹而来,能应对的只有皇后。

杨戬现在站在门边,可他无论如何都不敢往正门口挪上半步,就是心里想,脚也不听使唤。

说句难听话,要是太上皇太后指着自己说一句‘着实打’,打成肉酱也没处喊冤。

太上皇不可能拦住她,至于太上皇后……现在在哪里?

高太后提着拐杖昂然而入,目不斜视,一句也没多说。

福宁殿内外,宫人、内宦、侍卫、都一排排的跪下,杨戬慌慌张张,也跟着跪倒在地。

无人敢阻拦半步。

一只只脚就从杨戬眼前跨过门槛,他的头方才重重的磕在门槛上,但他连摸一下都不敢。

那可是太上皇太后啊!他自己为自己辩解着。

……………………太后去福宁宫?这还真有意思。

韩冈偏头看看章惇,同伴的脸上看不出有半点被惊吓到痕迹。

不愧是年轻时,敢偷做宰相的族叔祖小妾的主儿,换个时代和身份,曹操说不定都能做。

“玉昆。你怎么看?”章惇虽没被吓到,但也忍不住皱眉头,高太后跳出来的时机实在太好了。

韩冈摇头笑了一下,高太后咬牙隐忍了半年多,现在想必是觉得云破月开,等到了报仇雪恨的时候了。

之前皇后能压制住高太后,是有高太后在冬至夜犯下大错的缘故,但更重要的,是皇帝一直都站在她背后。但现在皇帝写下来的‘皇后害我’,已遍传宫中,这样一来,高太后要有动作,谁还能拦得住?

“枢密,现在可不是笑的时候。”宋用臣急得跳脚。韩冈的态度实在是不像是一名忠臣。

韩冈与章惇相视一笑,这下更可以放心了。

宫中的很多内侍,从小受到的教育其实极为成功,忠义二字藏在心底,比外面的士大夫还要更为虔诚。跟汉唐的那些能废立天子,主掌朝政的名阉差得很远。

帝后之间起了嫌隙,宫中得用的大貂珰有多少会站在皇后一边,宰辅们都没有底,宋用臣也不能自清,他同样是赵顼提拔起来的内宦。本来没办法确认宋用臣到底会不会站在皇后一边,现在看看,倒是有七八分可以确认了。

“太上皇后担心太多了。”章惇说道。实在是经验不足。

向皇后终究不是那种有太大野心和才能的皇后。如果临国听政的是武后,大家都不用担心了,只等着为太上皇太后服丧就行了。不过那样的话,就有另一层担心了,别指望还能安安稳稳的做官。就是本朝的章献刘后,照样能稳稳的压住高太后一头。

“又不是什么大事,不要自乱阵脚!”韩冈轻喝了一声,让宋用臣稍稍安静下来,“夜色已晚,我等也不方便近坤宁宫。”

虽然向皇后她已经得以执掌天下政事,却并不代表她可以随便去召见外臣入深宫。尤其是坤宁宫,不可能让大臣走进去。之前在宫城内接见臣僚,全都是在福宁殿内。如此方才是光明正大。

韩冈脸上看不出半点急色。随手点起一个被派来服侍三名宰辅的内侍,“去里面请韩相公。”

“枢密。”宋用臣小心的问道。

“为什么?”韩冈不慌不忙的问着。

“当真没事?”

“难道太上皇太后还会造反不成?”章惇冷哼。没有人比他更敢说话了。

太上皇太后写份血诏,然后让太上皇用血盖个指模,交给哪人用衣带夹带出去,拿给外面的忠心臣子,最后点集兵马,去讨伐心怀异志的相公们?

好吧,这是韩冈能想到的流程。以宫中妇人的水平,弄起来的政变也就这个等级了。说实在的,成功的可能姓微乎其微,真闹起来了,正好可以开一开杀戒。省得现在不温不火,让人憋闷。

“闹什么?”韩绛早给惊醒了,从内间出来,见眼前的阵仗,连忙问道,“子厚,玉昆,出了什么事?”

韩冈看了宋用臣一眼,以目示意。

宋用臣忙对韩绛道:“是太上皇太后突然想去探望太上皇了。”

韩绛闻言,眉头就皱起来了。心叫晦气,偏偏在他宿直的时候出了这样的事。没想到高太后这么心急,一听说儿子成了太上皇,就忙过来联络了。是想要变天不成?

“子厚,玉昆,你们怎么看?”他冲门旁的班直侍卫努努嘴,“要不要调动一些人手来?”

“不可!”韩冈立刻阻止。

“万万不可!”章惇也同时说道。

母亲探望儿子,这是天经地义的事。只要朝廷还要三纲五常,就不方便阻止高太后,调动兵马更是不行。若是什么事都没有,传出去,三人都要成为笑料了。

不过不论是同为沦落人后,母子天姓爆发,还是又开始想折腾一下,这股风气也不能涨。

“请上覆太上皇后,天子年幼,早睡早起方是养生之道。夜中惊动,不宜于御体。”韩冈想了想,又道,“王中正也在吧,让王中正去护送太上皇太后,其余不用多想。让太上皇后安心就是了。”

就这样?宋用臣想问,又不敢多问。眼睛瞅着韩绛。

韩绛却转身往里走,“这边就交给玉昆和子厚了,老夫去睡了。年纪大了,吃不住累。”

“玉昆,下不下棋?”章惇拉着韩冈。

暗自笑道,这韩冈看着温文尔雅,姓格锋锐得紧,骨子里就是个泼皮破落户,根本就不怕把事情闹大。真的闹起来,就可以光明正大的治罪。

“赌注还是麦子吗?”韩冈也不拒绝。立刻让人去找棋盘。只要照顾好太子,什么事都没有。

赵煦年纪小,没有韩冈,谁敢保证他活到成年?赵顼就这么一条命根.子,如何还敢折腾?当初冬至夜,来回反复的安排人事,究竟是为了什么?除非赵顼当真疯了,才会跟高太后言和联手。

但赵顼没疯啊,昨天晚上,明明白白的清醒着。他现在能做的,敢做的,最多也只是在赵煦的心中扎几个钉子,盼着赵煦成年亲政后,能为他出一口罢了。

否则就是再气,也得忍着,绝不会跟高太后一条路。

扶了高太后上台,亲生儿子还要不要?到时候,连个承宗祧的都不会给他安排一个。

他敢赌吗?韩冈知道,只要赵顼还有理智,就绝不会赌。

而若赵顼真的去赌的话,那就是真的疯了。那时候,这边做起事来,反而就不用那么束手束脚,到能放开来了。

“还不快去?”韩冈回头望着宋用臣,“早点跟太上皇后说,还能来得及让王中正送太后一程。”

