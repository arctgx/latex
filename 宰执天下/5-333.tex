\section{第32章 金城可在汉图中(12)}

“韩…冈…要…来…了…”张孝杰一字一顿,低沉下去的语调仿佛在承受着极大的压力。

“不对。”萧十三摇头,面黑如铁,“是已经来了!”

“这消息可是确实?”张孝杰已经顾不得宰相的风度,急声追问,“是从哪里得来的!”

萧十三将手上的信件递给张孝杰,“这是在太原城外截下一名信使身上搜出来的……人是往南面去的。”

张孝杰接过那份盖着太原府大印的信函,匆匆一扫,信上所说的不过是对韩冈之前承诺重新加以确认,并恳求援军越早越好,甚至还说了许多辽军攻城给城中造成的伤害。天知道,现在射向太原城头上的箭矢还不一定超过了十支。

“二十天……”张孝杰攥紧了手上的信纸,汗水从掌心中冒了出来,浸湿了纸张。

“从信上看,韩冈说二十天援军必至,已经是三四天前的事了。”萧十三数着时间,“就算以三天来计算,也只剩十七天了。十七天后,宋国京畿的禁军便要抵达太原。”

“怎么来得这么快?!”张孝杰想不通。宋人的进兵速度过去都是有记录的,应该不可能这么快。

萧十三却道:“河东事关宋国存亡,太原则事关河东存亡。眼见太原有失,宋人肯定是拼命赶过来。”

“可那是一千多里地啊!”张孝杰仍是一幅难以置信的表情。

数以万计的步卒怎么能以一天八九十乃至一百里的速度前进?三五天就会被拖垮。就算是骑兵,如果不是一人三马可以轮换着来,也撑不住这样进军的速度——宋国京畿能有五千一人三马的骑兵吗?

“但这可是韩冈说的。”这是最有说服力的一句话,张孝杰哑口无言。

不管怎么说,即便剥去了所谓药师王佛弟子的光环,在辽国的传言中,韩冈也都是那种不妄言妄语的南朝名臣风范。他说二十天援兵将至,那就肯定会有援兵在二十天内赶到太原。

“……韩冈现在在哪里?”张孝杰忽然问道。

萧十三摇摇头:“不知道,只是想来应该是在铜鞮县吧。”

“……铜鞮县?”张孝杰手上就有河东舆图,翻开来一看,就了解到了几条出入河东的主要道路,“皮室军能不能攻下铜鞮县?”

“铜鞮县中到底有多少禁军?”萧十三反问,这决定了铜鞮县到底有多难攻的关键。

“威胜军的驻泊禁军不过两个指挥。”张孝杰手上不仅有舆图,还有活口供,河东各州的军事他基本上都了解到了。

“只要有韩冈守在哪里,一千人至少得当成三千人用。想要攻下铜鞮县,至少要两万兵马。”

“没那么多兵马啊。”张孝杰苦恼着拧起眉,“忻州还没打下来呢。”

忻州的宋军已经是坚持到底了,连士气都莫名的高涨,这让辽军上下觉得很棘手,如同面对刺猬一般避之唯恐不及。

“而且太谷县也有两个指挥。”他又继续说道。

“太原府的兵马虽众,但大半都被调去了河北,留给其他县城的兵马,只有一些老弱病残。不足为虑。”萧十三道,“依我说,还是先夺了太谷县,南面的路也好看得住。”

“要是西军从河中府北上,自汾州出来怎么办?”

“自太原和太谷县出兵围剿,”萧十三停了一下,“只要能打得下太原和太谷”

“若是从太原和太谷县发兵,完全能让北上西军吃上一个大亏,只是有那个机会吗?”张孝杰也摇头说道。

太原四通之地,东南西北都有路通过来,必须分兵把守。可这偏偏是辽军最不擅长的一桩工作。

如果是在平原上,可以充分利用骑兵的速度,直接跳出包围圈,但在群山环绕,只有数条道路通向外界的河东,骑兵的最大优势完全发挥不出来。

不过萧十三的脸色却,耶律乙辛的态度是见好就收,但现在进入太原的除了一部分宫分军和皮室军外,也有不少头下军。这些贵人名下的军队没能赶得及在代州捞好处的机会,此时更加富庶的太原府就在眼前,不亲眼看见宋军杀来,是不可能让他们放下眼前的肥肉转身回撤的。

如果强逼着那些贵人和头下军北返,他们肯信这是救他们的性命才有鬼,必然会喊着要求补偿。若抛下他们直接北返,不论事先有没有警告,回到国中后都会兴起众怒。

“看来还是要打一仗了。”萧十三沉默了一下,又道:“其实这件事也不难。”

张孝杰点头表示同意,“的确不难。”

两人交换了一个眼色,同时点头:“就让来援的宋军葬身在太原附近!”

宋军四方援军被催逼的赶往太原来,却不可能同时抵达,在这时间差上就是机会。只要稍通军事,就不会不知道各个击破的好处。

可以说是运气,好就好在那个‘二十天’!

……………………

“守内虚外、内外相制自开国以来一直便是被定为国是,而京中也的确是屯有重兵,但在实际上,由于边患不绝,为了防备辽夏,放在边境上军力,无论是陕西还是河北,都不逊于京城。”

韩冈的帅府行辕已经设在了太谷县,黄裳等衙署幕职官也陆续抵达,帮着他将整个制置使司的架子给撑起来了。为了与入寇的辽军决一死战,制置使司中的上上下下都拼命做着准备。几日来每日聚会公厅,无一例外都是面色凝重。

只有韩冈看起来甚为悠闲,还有心将起兵制现状的利弊来,“不过除此之外,不论是南方,还是北方,广大的地区都是极为空虚。”

“兵力之差只是一方面。在枢密院的籍簿上,各路军队的数量并不少,纵然没有禁军,也有厢军充数,纸面上看起来甚为可观,可在实际上,北方边军通常只有兵额七八成的实际兵力,而其余各路能有五成就不错了,在南方,甚至三成四成也是有的。当年我去广西时,当地的几支厢兵甚至不及两成的也有。

而且论起兵备,不论是军寨,还是城池,也都是显而易见的年久失修。只是在庆历年间,因为朝廷为了与西夏作战而加重了税赋,使得国中盗贼遍地,各军州的防御体系都不得不进行了一番大的修整。这一仁宗皇帝时留下的遗产,便一直吃到了现在。”

韩冈向着幕僚讲着古,他对现在的兵制有很多看法。如果要总结经验教训,改正过去的错误,这一次河东半壁沦陷,就是一个最好的机会。

“不过有些事,只有失败中才能学得到。之前的历次胜利将很多.毛病都掩盖了,所以才有了今日之败。只要事后能扳回来,就不能算输。”

韩冈语气轻松,仅仅说着要如何总结经验教训,这个态度让那几个刚刚被借调来,而并不了解他的官员都稍稍安心了一些下来。

从好处想,这是军队积弊的一次大爆发。藉此机会,韩冈可以提议对军队进行一次大的改革,而不是置将法那样的修修补补。至少将士官的提拔和培训制度提上台面,并加以推行。

韩冈正说着,外面却送来了一封信,而且是来自于开封。所有人的目光都聚集在了那封信上,是王安石还是韩绛,又或是别的宰辅的吩咐。

韩冈在众目之下拆开信,看了一看,然后就向着厅中众官笑道:“家叔书而已,数日前,家中又添了一小儿。”

官吏们同时一愣,真不知道该有什么样的表情。该恭喜吗?但时候似乎不对,若是不道喜,恐有得罪韩冈,想想还真是两难。

而且现在是说这事的时候?

富弼当年出使在辽国的时候,看到家书就直接点火烧掉,说徒乱人意。现在辽军已经占了代州,围了太原,身为制置使的韩冈却在这里慢条斯理的说着又多了一个儿子。

可是不知为什么,厅中的大小官员紧绷的神经却为之松弛。

“不说那些无关的闲话了。”韩冈终于将议论的方向扯回了正题上,手上又拿起了一封小册子,黄裳等幕僚都知道,这是韩冈几天来日夜赶工的作品,只是内容还不知道,韩冈一直都在严守秘密,直到现在拿出来。

“敢问枢副,那是什么?”黄裳领头问道。

“如何发展敌后抗战的指导书。”韩冈一笑,“不能让北虏顺顺当当的抢钱抢粮,再来跟我官军对抗。公诸于众,每一座州县都要尽可能多的将之刻印散布,让河东所有百姓都知道,该怎么与北虏斗争。”

“是要动用乡兵、弓箭手还有保甲?!”

“北虏肆虐,河东百姓无不受苦,自然是要全民抗战,人人拿起武器。”说道这里,韩冈抿了一下嘴,“当然喽,与辽军对抗的真正主力还是官军,拿了那份军饷就该做事,不能将责任推到没军饷可拿的乡兵身上。”

“但若是按枢副所言,每一座州县都来刻印散步,肯定会落在辽人的耳目中。”

“辽人看到反而好。我正希望能广而告之。”

黄裳眼睛一亮:“可是能吓得走北虏?!”

韩冈笑着摇头,“吓不走。”

已经不是弦高的时候了,能用骗把敌军骗走。从那时起,战争艺术发展了一千七百年,在才智上小瞧敌人那是最蠢的行为。

黄裳有些遗憾和失落,韩冈看在眼里,暗暗摇头。他这个幕僚,终究还是没脱了文酸气。文人总有爱用计的坏毛病,总想着能一策定江山。但真正的克敌之法,还是稳扎稳打,以势压人。

韩冈想要的,便是这个‘势’!

