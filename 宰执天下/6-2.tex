\section{第一章 一年穷处已残冬(中)}

约定了见面的时间,章敦便先走一步。王厚向韩冈告罪了一声,改送章敦出去。韩冈则径直入内探视王安石。

书房内有着浓重的药味,王安石正皱着眉头的喝着黑乎乎的药汤。

今天的王安石虽说已经能够起身见客,但从气色上看,与前两天没有什么改变,脸上的皱纹也似乎比往日更深刻了几分。

不过看见韩冈,他却露出了许久未见的轻松笑意:“玉昆,坐。”

韩冈行了礼,依言落座,“岳父今天感觉如何?还有像昨天那般气闷了?”

王安石呵呵笑着,捶了捶膝盖,“年岁大了,哪里没有毛病?不过是胸口憋闷了点,你们就是爱瞎担心。”

“这件事,小婿还是听御医的。该吃药得吃药,该扎针得扎针。岳父你说了不算。”

王安石摇头叹了口气,“玉昆你啊,还真是……”不过说了半句,却又莫名的跳开了话题,转问道:“外面的雪停了没有?”

“刚刚停。”透过格栅细密的玻璃窗,韩冈看了眼外面灰色的天空,“可能是今年最后一场雪了。下一场雪,可能就是元佑元年了。”

王安石又叹了一口气,“想不到这么快就到元佑元年了,感觉才定的年号。”他看看韩冈,自嘲的笑道,“年纪大了,叹气的时候就多了。”

“是岳父为了国事思虑太多了,心里放不下。”

“是放不下。”王安石哼了一声:“有玉昆你一份功劳。”

王安石这话里话外显得积怨不浅,韩冈苦笑起来:“岳父说的小婿可万万当不起。”

“你还当不起?”王安石摇摇头,不禁又叹,“谁能全然看得开,放得下?真要有人能做到,那可要成圣成佛了。”

王安石叹气时疲态毕露。惨淡的日光透窗照进来,映在蜡黄的脸上,老人斑也越发的明显。看得出来,王安石的身体状况正日渐恶化,现在就算让他回任平章、宰相,恐怕也没那个能力了。

自从王雱去世之后,韩冈就感觉他老得特别快。加上赵顼、赵煦两父子接连出了意外,这对将毕生功业的未来寄托在赵煦身上的王安石来说,打击之大可想而知。

真要论年纪,才六十出头的王安石,远比不上韩冈当初出任京西时在洛阳见过的几位元老。富弼、文彦博都是年逾古稀而精力不衰,王安石可是差得远了。

韩冈也明白他的情况。之前卸去了平章之位,心中还有一个念想,一心想要教出一个明君来。可课程才开始,‘明君’的未来就不复存在了。灰心丧意之下,这一回退下来后,可能不会再复出了。

韩冈不是王安石,从来没有想过要‘致君尧舜上’。他会去做太子师和帝师,也只是想借资善堂和经筵这两个平台,来增加气学的知名度,对教出一个明君可没有什么想法。甚至可以说,越是明君越是麻烦。

发源自西方的科学,由于教权和王权经常性的对立,敌人主要是禁锢人心的宗教,许多时候还能受到世俗政权的保护。但韩冈现在推广气学,探究自然的行为,最大的敌人则是将皇权建立在绝地天通上的天子。祭天祀地,册封天下神明,言行举止能影响灾害,这种给自己套上无数神秘光环的统治者,就是自然科学的死敌。

只不过他的想法,可不是能说出来宽慰人的。

“圣人要能放得下,何须奔走列国,立道统于世?佛祖也不用传教授徒了。谁都有放不下的事。太上忘情,谁是太上?”

王安石盯着韩冈看了一阵:“玉昆,你是斗嘴成了习惯?”

韩冈猛然醒悟,现在可不是在跟王安石辩经,不由得苦笑起来:“好像真是习惯了。”

韩冈认得干脆,王安石都不知该说什么。他有时会想,自家是不是没积德,招个女婿都不省心。

沉默了一阵,又喝了口热茶,王安石提起章敦:“方才章子厚带着大赦诏来。”

韩冈还没有看到赦诏,不过诏书的内容基本上都是大同小异,但有件事是他要关心的:“赦诏上怎么说?‘常赦所不原者,一并放罪赦免’?那流配者怎么处置?”

“流配者还是就地安置。”

“那就好。”韩冈放心下来。

帝位更替,正常都要颁布赦诏。当天子或是太后、太皇太后重病——有事也会为了生病的皇子——为了祈福,也同样会颁布赦诏。不过赦诏也分等级,有的赦流刑以下罪,有的则是将十恶之外的死罪全都给赦免了。前一次大赦才过去几个月,这一回又是个大赦诏。三番两次的折腾,监狱里面还不知有没有人了。

之前熙宗内禅,赵煦即位,大赦天下的诏书中,在韩冈力争之下,有关重罪流配的犯人都是就地安置。这两年,长距离流配的罪犯,目的地只有一个——西北。纵然是广州那边一个三千里流配的犯人,三千里一走都到了中原繁华之地,但实际上的落脚点照样是西北的熙河、甘凉以及宁夏三路。西北蕃人多而汉人少,即便是罪犯,也没什么好讲究了。也不怕他们闹,反正朝廷在当地屯有重兵,又是天下有数的重法地,再不老实,刀子、棒子都是有的。

“不过时间划在腊月初一之前。想必玉昆你是明白的。”

“当然。”韩冈自然明白,“若是什么罪过都能赦除,朝廷纲纪可不就是笑话了?”

普及天下罪人、犯官的大赦诏中都会订一个时间点,某年某月某日之前,犯下的罪行可以一并赦除,如果犯人没有归案,只要在时限内过来自首,也便不会追究。这个时间点,一般都是赦诏的颁行日期。

只是现如今,熙宗皇帝崩于炭毒一案,除了赵煦之外,还有许多人都要受到处置。就比如韩冈和王安石,正因为没有将赵煦教育好,让他犯下如此大过,故而引罪请辞。还有福宁宫中的内侍、宫女,他们同样要论罪。

只是事故而已,纵被牵连,也并非十恶不赦的重罪,全都在赦免的范围内。现在若是一道赦诏下来,什么罪过都免了,难道赵顼就这么平白死了?连个负责的都没有,那皇帝还有什么威严可言?

章敦今天过来,多半是受了太后的私下委托,前来向王安石进行解释,韩冈这边,虽然还没有收到消息,多半也会派人来解释一番。

与韩冈提起大赦的时候,王安石一直在仔细观察着他的反应,现在终于是确定了大半,韩冈应该是真心打算辞官,没有任何勉强。

这不一定是好事。高官显爵说丢就丢,可见心神都在学问上,还是要跟新学为难。

“……除了大赦诏,还有一个是熙宗皇帝山陵的事。”

“熙宗皇帝……”这个词念起来,就是韩冈自己也觉得别扭,“的山陵,这是子华相公该去操心的事,墓址之前也已经定下来了,材料也都备好了。不要操心什么事情了吧?”

赵顼重病一年多,早已经点了所谓吉穴,选好了墓址,就等着赵顼的梓宫移去下葬,哪里还有什么事情要来问王安石。就是韩绛,他真正要头疼的,还是这一回上皇驾崩,是不是还要犒赏百官、三军,以及能拿出多少来犒赏。

“权同管勾司天监周琮上表,说之前选定的墓穴不吉。”

“之前的墓穴是判监事的丁洵选定的吧?”韩冈问。

有关天文、历法,以及卜问吉凶,都是司天监的工作范围,选择墓穴也同样如此。

“不是他还有谁?”

两边相持不下,影响到了赵顼的身后事,故而来向王安石通报。纵然他引罪辞官,但地位还摆在那里,切切实实的国之元老,鼎鼐重臣。

韩冈微微皱起眉头:“两人斗了二十多年了,还在斗?”

司天监中的天文官,属于伎术官范畴,不入文武两班序列,很多职位都是父子相承,而一个官员能在同一个位置上坐上几十年。权判监事的丁洵统管司天监三十一年,周琮做权同管勾也快三十年了,两人一主一副是从仁宗皇帝的时候一直在司天监做到了现在。韩冈记得前两年,两人因为近三十年不领磨勘,不得晋升,故而特赐恩其子孙,允许两人各荫补一子孙入学。只是两位老同事的关系据说是恶劣得很。从这两人搭档的时间上来看,倒也不难理解。

他随之又冷笑起来:“这两位是想做邢中和吧?”

邢中和是真宗时候的判司天监,当年真宗驾崩,他跑去对修治山陵的雷允恭说之前选定的墓穴差了一点,要移动百步才是最佳的吉穴。雷允恭信了他,征得了刘太后的同意。可邢中和指点的新位置开挖时却冒出了泉眼,喷水不止,他最后是用脑袋抵了罪过。雷允恭这位有拥立之功的大貂珰,也同时丢了性命。

韩冈一向觉得所谓点吉穴,发后人的说法是无稽之谈。墓穴只要不透水,不生蚁虫,不易为人盗掘就行了,哪有那么多弯弯绕绕的。而且他对司天监的不满一直都有。

“不管周琮是不是找理由,也不管到最后谁做了邢中和,事关大行皇帝,岂是小事?容不得有半点意外。”

“自是当然。”

韩冈的态度还是瞒不过王安石,气学讲究实证,自然对这些神鬼之事嗤之以鼻。

王安石其实也没什么兴趣,要不是事关赵顼身后事,他根本都不会在意,丢掉了那个无趣的话题,他问韩冈:“已经好几天了,外面是怎么说的?”

