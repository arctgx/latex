\section{第20章 千山红遍好凭栏(上)}

车停了。

李诫从车上下来,两脚刚落地,腰上便是一阵剧痛。

“明仲兄,腰又痛了?”

看见李诫右手撑腰,倒抽凉气的样子,前一步下车的宗泽立刻关切的问道。

“好多了。前天晚上才叫痛。”

“那今晚宗泽去将梅太医请来,再扎上几针?”

李诫疼得钻心,脸上蜡黄,不见一丝血色,额头上也密密的出了一层汗。只觉得昨天刚刚因为针灸才好了一点的腰,又疼得让人恨不得用锤子用力的捶上几下。

听宗泽一说,他连忙道:“多劳了。”

宗泽过来扶着李诫:“明仲兄还是先坐下来歇歇吧。”

“别!”李诫连忙伸手拦住宗泽,“这腰上的毛病站一会儿就好,坐不得。”

宗泽没放开手,扶着李诫靠着马车车门,让车夫不要急着走。

靠在车门站了好一阵,李诫的脸色也不再蜡黄,笑着道:“还好是坐车,换作是骑马,当真是能要了这条老命。”

“若是明仲兄不嫌麻烦,明日可与宗泽去西十字大街的车店去看一看,那里专一贩卖各色马车,最好的不比宫造的差,车底用了软钢缓冲,比这辆马车要强上不少,用以代步,绝不会伤到腰。”

这几年的辛劳,让李诫伤了腰。骑不得马,出行只能坐车。幸而这两年,京城内乘坐马车已经蔚然成风,多少官员在外皆是用马车代步。不比过去,从宰相到卑官,骑马的占了绝大多数,即便已是老迈,也会尽量骑马。谁也不想坐着马车或是肩舆出外,平白送把柄给御史台。

可如今一方面是马车造得越发得舒适,躺在车里与躺在床上也差不多,另一方面,京师的空气日渐污浊,在马车中也能避避灰尘,此外最重要的,则是世风日渐奢靡,没有一匹血统优良、高大英俊的好马,让人也无颜骑马外出。骑着驽马,脸还不够丢的。换作是乘车就好了许多,一辆外表光鲜的马车,不比好马贵,却更容易保养,挽马也不用河西马、大食马。

正因如此,出租马车的车马行,如雨后春笋,一个接着一个冒了出来,开封府光是收马车的牌照费,一年也能有上万贯。相对的,昔日在街口、桥头等待客人的租马人,则一个个消失不见,不是转业,就是加入了车马行。

这番变化,倒让抵京后,一直坐车的李诫不那么显眼。

不过李诫没打算买车,“不用费心了,过几日就要离京,买车又有何用?”

“明仲兄这腰上的病得好好养。而且相公前日也说了,这一次明仲兄你回来,当在京城好好将养上一阵。”

“竟有此事?!”李诫心脏扑通扑通的跳了起来,这是过河拆桥,还是另有重用?随即他摇摇头,试探道:“相公于李诫有知遇之恩,这番恩德,留在京师安养如何能回报?汝霖你也不用担心,腰疼又不是病,要不了命。”

“相公应当更想看见明仲兄健步如飞的样子。”宗泽笑了起来,清楚明了的说道:“相公之前一直在叹无人可用,明仲兄这一回回来,相公可不会放人。”

“才如汝霖者,当世凤毛麟角,万中无一,但如诫一等,却是车载斗量,除了卖卖苦力,也没其他地方能为相公助力了。”李诫安心下来后,谦虚了两句,便回头看了眼身后,“好了,我们还是快进去吧,不要让相公久等。”

话是这么说,但相府门前的巷道一向是车水马龙,行人如织。马车停在巷口只是一小会儿,后面已经有人开始不耐烦了。

一名仆役装束的男子走了过来,对两人行礼道:“还请两位秀才稍让一让,我家官人有事要进去。”

宗泽和李诫都没有穿官服,又是租用了马车,但京城中龙蛇混杂,又是在宰相家门前,谁知道穿着一身襕衫的两人,究竟是累试不第的士子,还是有背景的官人?说不定就是累试不第却同样背景深厚,保持着应有的礼貌是一名官宦人家家丁最基本的常识。

宗泽向仆役的来处望去,一辆装饰朴素却质料出色的黑色马车正停在后面,等待这边让出道路。

看到马车和前面两匹的挽马,宗泽心道,车子的主人必然家底不差,估计官位也不会太低。

正想拉着,却见马车的车门一下被推开,从里面蹦出一团红色,再定睛一看,却是一名身穿五品朱袍的官人。

正赶人的仆人吓了一跳,却见那官人没站稳便一声笑,“可是中书兵礼房的宗状元?!”

这官员本来是冷淡的等在车中,让仆人来处理前面的堵路人,可一看清了是宗泽,便立刻换了一副表情上前来。

宗泽一眼认出了来人,拱手相迎:“宗泽见过王直阁。”

王同老连忙回礼。

仁宗嘉佑时参知政事王尧臣之子,王同老比宗泽年岁更长,资历更深,地位也更高一点。但有一个中过状元、做过参知政事的老爹,王同老很清楚,像宗泽这样及时的从三馆秘阁和翰林学士院跳出来的状元,未来的道路会更宽广,更别提宗泽的后台。

对行了礼,王同老又打量起李诫:“这位应该便是主持修筑京泗铁路的李明仲了吧?”

李诫一本正经向王同老行礼,只是腰痛未消的缘故,行礼时腰便少弯了几分,王同老却看不出有任何芥蒂的样子,笑容也丝毫未减。不说拿到状元的宗泽,即使是没有进士资格的李诫,仅仅是以韩冈的看重,再过几年,也绝对有机会参加廷推了。

最重要的是,韩冈对沈括、李诫的工作十分满意,要不然两天前也不会亲自出城出城迎接抵达沈括、李诫一行。

王同老听说,金牌加急的特快专递,通过铁路将信息先一步送到京师,得知了沈括抵京的时间,韩冈便毫不犹豫的出城相迎。

那可是宰相郊迎啊!

这份荣光过去要么是大功返朝的帅臣才能享受到,要么就是殊勋元老的专利,普通官员哪里有这样的待遇?

但韩冈还是坚持出城去了。他亲自出城来迎接沈括、李诫一行,也是想让世人明白,两人所立功绩到底有多大。

一条京泗铁路,让国家命脉不再被汴水的涨落而控制,主持修造成功,其功绩岂在平贼败敌之下?

京泗铁路开通的意义,已经不用任何人多费唇舌。在一篇篇的报纸,一段段的评话,方城轨道多年的运营,以及韩冈的《九域游记》的宣传下,京城中,便是七岁小儿也知道,从此以后,汴水的地位已经不是那般重要了。

韩冈再进一步迎接,除了抬高京泗铁路通车的意义,也是为了给沈括撑腰,这是拉开架势要支持沈括入两府了。

而韩冈这样的态度,立刻便传遍京城,也让李诫提前一步享受到了让人敬畏的感觉。

王同老一番寒暄,热情洋溢,还特地邀请宗泽、李诫几日后的一个聚会,让宗、李两人废了一番功夫才得以脱身。

而后王同老便用羡慕的视线,目送宗泽、李诫避过了韩府的正门,转向了侧门的位置。

正门的门房中,有太多官员等待韩冈的接见,韩家人大多数的时候,都是从侧门进出,包括韩冈的亲信在内都是如此。

当宗泽和李诫自侧门悄然入内,很快便被引到韩冈的书房。

“李诫拜见相公。”

“宗泽拜见相公。”

刚进书房,两人便几乎同时的像韩冈行礼,而韩冈却也不谦让,大喇喇的站起来,拱拱手当做回礼,转身说道:“汝霖、明仲,且随我来。”

“相公……”宗泽吃了一惊,韩冈的心思之前可不是这一个。

“铁路上的大小事,昨天都在政事堂中议定好了,汝霖你也用不着多劳神了。”韩冈熟练的说道。

韩冈昨日已经在中书门下接见过李诫了,对铁路的事务说了很多,自然,今天就不必再说上一遍。而有关格物的内容,比如马车的速度,比如蒸汽机带来的便利,韩冈昨日只是泛泛而谈,随口提到了两句。

韩冈没有多说废话,回头从内间找出了一份图纸,展开来放在宗泽、李诫的面前。

“可曾见过类似的图纸?”他问道。

李诫只瞥了一眼,双眼就定在了里面。

“这是蒸汽机的图纸?”

他很快就分辨出来。

李诫的才干不仅仅是在修筑铁路上,工业机械上面的才干也同样突出。李诫画得一手好图,他所画的三视图,如果不是图纸纸制的缘故,几乎就像是后世的设计图纸了。当然,细节上的差异,也不是韩冈这个外行人能弄明白的。

“难道已经造出来了?!”宗泽大惊,之前怎么连一点风声都没有,而李诫也皱着眉,他也没听说有这方面的通报。

“还早得很,只是有些眉目了。”韩冈低头望着图纸,专注而用心,“自古农为国本,天下无粮不安。可如今的局面,就是一团乱麻,梳理不清。”
