\section{第31章 马鸣萧萧辞旧岁(下)}

“好冷!”

王厚用力搓着手,脸冻得通红,耳朵上都生满了冻疮。滴水成冰的天气,三天里骑在马上跑出几百里,迎面的风呼呼地直往衣襟里钻,把他冷得够呛。

“是够冷的。”韩冈随口答着。他里面穿的是对襟的双层皮袄,露在外面皮肤都抹了油,倒不如王厚那般受冻。王厚也是自找,韩冈让他弄些羊油抹在耳朵上,他嫌恶心没肯用,这下在外面一跑,便冻出毛病来了。

王韶没理会两个小辈,他站在盘山道上,向下俯视着渭河河谷。一众亲兵在王舜臣的指挥下,散开在周围,小心的护卫着王韶。

一个多月的时间,王舜臣和赵隆已经得到了王韶彻底的信任,而两人的实力也通过王厚传到了王韶耳里。包括刚刚得到任命的李信,如今王韶身边最得他看重的四名亲将中,有三人都是韩冈荐上来的。

王韶现在已经在为日后的进兵河湟点选将领。秦凤路,甚至是关西四路有名的将佐,他都已心中有数。但这些从外调来麾下的将领,肯定不及亲手提拔出来的军官易于指挥。王舜臣、赵隆、李信三人对王韶来说,其实助力不在韩冈之下。

盘山道的下方便是古渭寨。其所在的位置,是夹在群山之中的一片宽阔的谷地,也是渭水上游难得的一片沃土。从汉至唐,千多年都在此处建城设州,从无迁移,自然便是因为此处优越的地理条件。

冻结的渭河白色一片,但衬在河道两边的雪地中,冰结的白色却分外显眼。河上的冰面高低不平,宛如丘陵起伏。这是湍急的流水在冻结时交相推挤,才有了现在的模样。由于冰面挤压破碎,冰层上裂隙处处,行走在冰上,一不小心就会落入冰层下的河中。

而古渭,正是建在渭河边。

古渭,顾名思义,就是古时的渭州。不同于如今位于秦州以东的渭州【今甘肃平凉】,隋唐时的渭州就在韩冈现在立足的地方。汉晋之时,此地名为襄武,直至隋唐,亦是渭州州治襄武县之所在。只可惜安史之乱后,吐蕃势力扩张,将此地占据,不复为汉家所有。从那以后,渭州的位置自西向东迁移了五百里,这正是汉人王朝势力大幅消减的最有力的证明。

从高处俯视,地形上的细节被模糊了去,但却能统观全局。至少在河谷中分辨不出来的唐时渭州城的遗址,在盘山道上,却能看得很清楚。古渭州城的城墙已经尽毁,不过城基即便掩盖在雪地中,依然十分显眼。六七里长的大城,比起不远处的古渭寨要雄伟上许多。只可惜几百年前的繁华州城,各色人种纷至沓来的街市,如今仅剩一片残迹。

从盘山道上下来,一支兵马迎面而来,在最前面引路的杨英是王韶从德安带来的一名乡里,也是他的贴身亲信,在经略司补了一个不任实职的弓箭手指挥使。而跟在后面,领着一队骑兵的是驻扎在古渭寨中的秦凤西路都巡检,他同时还兼任着古渭寨主一职。

“刘昌祚见过机宜。”

在王韶身边拜见的西路都巡检,高大的身材是标准的北地男儿。相貌说不上英俊,线条冷峻,却极有男性魅力。他身穿着一身远比韩冈王厚等人要单薄得多的外套,在寒风中全无瑟缩之意,健壮的身材显露无遗。

刘昌祚应该超过四十岁了,比王韶还要年长一点,不过从他外表上却看不出来。他的父亲刘贺二十年前战死于定川寨一役,因此受了荫封,被录为正九品的右班殿直,主管威远寨。刘昌祚二十年在边陲,累立功勋,到如今才刚刚升做内殿崇班,与王韶同品阶。不过因为文武之别,在王韶面前还要低上一头去。

见着架在刘昌祚身后坐骑上的一张长弓,王舜臣有些跃跃欲试。那是一张闻名秦凤,全长超过四尺的巨弓。据称力道有三石之多,搭在弓上的长箭也是特制,径圆半寸许,又比普通的两尺箭矢长了近半。当刘昌祚将他的巨弓拉满,弓弦与弓臂的距离,也只有如此长箭,才能搭得上去。

按说四尺长的巨弓不可能在马上张开,但刘昌祚以箭术闻名秦凤,却硬是能做到。据说他骑射时甚至能箭出百步之外,能一箭洞穿战马。蕃人捡到他射出的箭矢,都是拿回家去供奉起来,以为神箭。

刘昌祚与王韶互相行过礼,又与王厚相见。到了韩冈这边,听了他自己的通名,刘昌祚身子便轻轻一震,眉头也不自觉的挑了起来。韩冈的名讳在秦凤路上已经够响亮了,让向宝有苦说不出的人物,动动手指就灭了一个蕃部、毁了一个豪族的策士,刘昌祚早有耳闻。他对韩冈拱了拱手:“韩抚勾。”神色间并不是很亲热,向宝是他的顶头上司,不敢跟韩冈太过亲近。

经略安抚使司勾当公事,是韩冈预定的差遣。王韶、吴衍和张守约三人的荐章已经得到批准,韩冈的任命也在半个月前下来了,等过年后他去京中流内铨应个卯,便是真正的官人了。抚勾就是经略安抚司勾当公事的简称,就像王韶的管勾机宜文字,可简称为机宜和帅机一样。只是韩冈总觉得这个简称,就跟上海吊车厂、自贡刹车厂的简称一样可笑。

韩冈深深的还了一礼,道:“学生尚未拿到流内铨下发的官诰,当不得都巡称呼。还请都巡唤韩冈本名便是。”

刘昌祚点了点头,转身对王韶道:“机宜,末将已在营中做好了准备。天寒地冻,请机宜早些入营歇息。”

“都巡有心了。”王韶谢了一句,与刘昌祚并肩走了。韩冈等人跟在后面,一行向古渭寨中而去。

快过年的时候,王韶当然不会无事前来,但用心不在古渭,而在秦州。古渭升军的风声他已经暗地里放出去了,很快就会传入李师中耳中。他当然得到古渭寨走一遭,以便取信于李师中。

官场相争,争功诿过是少不了的。在如今的情况下,王韶有李师中居中掣肘,河湟开边始终未有开张。功是没得争的,但过却必须要诿。大言诳君,让天子苦候不得,这个罪名,王韶不肯担在身上,也不能担在身上。韩冈给王韶出的计策,便是让皇帝赵顼明白,究竟是谁在给河湟开边的战略捣乱。

上弹章攻击李师中没有任何意义,经略使说话的分量总比机宜文字要重上许多。所以让李师中自己蹦出来给赵顼看,才是最佳的策略。从古渭建军,退到屯田市易,再退到屯田或者市易,只要李师中一步不让的姿态做到了天子眼前,谁还能再责怪王韶一年以来毫无动静?如果李师中在其中退上任何一步,却又遂了王韶的心思。

说实在的,能想出这样让对手进退两难的计策,王韶觉得韩冈比他还要像一个宦海沉浮多年的老官油子。

不过为了让李师中上钩,必须让他深信秦凤经略司机宜文字是真心的想在古渭设军。现在都快要到送灶王的日子了,再过六七天便要过年。这时候还往古渭跑,李师中再精明,疑心再重,也肯定不会怀疑王韶的真实用意。

‘也到了该摊牌的时候。’走在刘昌祚的身边,王韶下定了决心。

……………………

狂风吹得门窗哗哗作响,雪花被狂风卷着,从门缝中钻进屋内,屋中火盆里的火苗被风压得只在木炭表面跳动,半点暖意也散发不出来。

原本王韶预定着在古渭住上两天,就赶回秦州。可以赶在除夕之前,回到家中。可一场暴风雪突如其来,打断了他回程的计划,不得不暂留在古渭寨里。

王厚拥在火盆旁,双手几乎要伸进火盆中央,南方人怕冷,王厚尤甚。他在关西的几年,最怕的就是冬天。他的两只眼珠随着在屋中来回踱步的韩冈左右晃动,最令他气结的是韩冈踱步的时候,手上还拿着一卷不知何时带来的诗经在默读。

“看起来要在这里过年了。玉昆,你也别晃了,看着眼晕!”

韩冈笑道:“闲来无事,只有读书消磨时间了。”他看看蜷在火盆边的王厚,又道:“处道你还是起来走一走的好,坐着反而会更冷。”

王厚站起来,学着韩冈的样在屋中来回走动,走了几步,又没话找话的抱怨起来:“这刘昌祚真真是讨人嫌,玉昆你好心要去帮他救治伤病,他倒好,哼哼哈哈的就是不肯答应。不然,倒有些事做。”

“他也是怕向宝,等到告身下来再说吧!到时我便名正言顺的能做点事了。”王韶在里屋休息,刘昌祚又提防着自己,韩冈没事可做,也只能读书。

过年时要敬天,要祭祖。但被暴风雪堵在军营中,这些礼节也便没人去搭理。没有爆竹,没有烟花,在狂风骤雪声中,熙宁二年即将宣告结束,熙宁三年很快姗姗而来。

听着外面军营中的喧闹,韩冈放下手中的书卷,推开了屋门。一阵寒风卷入屋内,让王厚冻得一声惨叫。王厚在别人面前,一贯谨严守礼,性格郑重严肃。只不过与韩冈惯熟了,才会露出了真性情。

韩冈微微一笑,走到了屋外院中。不知何时,已是云收雪散,繁星重新闪耀于天际。韩冈站在院中,仰头向天,深邃的天穹有着无尽的神秘。仰望天际,慨然兴怀。再过几个时辰,就是新的一年,这是他在这个时代度过的第一个新年。不知数百里外,父母和云娘是不是也在仰看同一片天空,也不知道,留在另一个世界的亲人,是否也能看到同样的星空。

王韶出来的时候,正看着韩冈独立在院中,一种遗世独立的疏离感笼罩在身周,神情有些落寞,不知因何而伤感。韩冈献计献策,手腕老辣,步步算计人心。虽然是帮着自己,王韶却暗中有了几分顾忌。只是现在看着韩冈望天伤怀的样儿,王韶的心情不由得一松,心想他也许是想家了缘故,

‘毕竟还是少年人……’

ps:刘昌祚出场了,在西军中,他是能力屈指可数的大将之才。只可惜没有上司运。

今天第二更,求红票,收藏。

