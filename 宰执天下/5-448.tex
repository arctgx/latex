\section{第38章 何与君王分重轻(六)}

“好久没来了。”

平章府一如旧曰,可韩冈自从离京之后,有半年没来到这里。入府之后,左右顾盼,兴致勃勃在看风景。

“嗯。”王旁很沉静在侧应了一声,嘴皮子都没张开。

“还是这般清静。”

王家的人少,诺大的院子,看不到几个奔走的仆役。完全没有簪缨世家的威风。

“嗯。”

“外面倒是热闹,探头探脑、鬼鬼祟祟,就没派人赶一赶?”

“嗯。”王旁依然只回了一个字,好像什么都没听到。

韩冈侧头看了看自己的二舅子,又道:“并州歌舞乃是一绝,冯当世[冯京]当年曾倍加赞叹。小弟这一回回来,有人就送了一对。赶明儿送过来,以娱耳目如何?”

“嗯……”猛然间反应过来的王旁大惊失色,“玉昆!”

韩冈笑得促狭:“说笑罢了,小弟可不想你妹妹回头怨我。”

王旁皱着眉,“玉昆,难道昨天回去二姐就没怨你。”

“出嫁从夫,多亏了岳父岳母教女有方。”韩冈呵呵笑了两声,见王旁板着脸,便收敛了起来,正色道:“我知岳父心思。岳父那边也当知我心意。世人皆以为岳父是以退为进,不过小弟明白,岳父是真的想退了。如果都只为功名利禄,哪会有这么多事?”

纵然朝廷现在将他和王安石的辞表都驳回了,可韩冈清楚王安石是真心想辞官,而他自己也是不想被人拿着枢密副使一职当成攻击自己的武器。权位本就是工具,不合手时就要干脆的丢掉。

大道之争到了这一步,已经没有退步的余地。官职可让,但道统如何能让?为了名声,为了能更好的一争道统,韩冈现在最该做的就是放弃手中的权力。如果没有辽国入寇,韩冈也不会接受枢密副使的任命,现在辞职只是回归正途。

韩冈私底下就准备荐苏颂代己任,同时将沈括推到三司使的位置上。只要其中有一个能成功上位,也算是达成目的了。当然,韩冈更信任苏颂一点。毕竟沈括是有名的墙头草,一贯的腰骨软。

王旁有些看不惯韩冈的态度:“这回吕吉甫要回来了。玉昆!”

“小弟能回来,吕吉甫当然也能回来。”韩冈浑不在意,他的以退为进,比人们所猜测的要退得更多、更远:“岳父要他回来就回来吧。”

要真是以辞官为要挟,王安石他荐吕惠卿做什么?韩冈也准备推荐人,这就是真正想要辞官的做法。

“玉昆你倒是看得开。”

“难道仲元还以为小弟辞官是妆模作样,私心里还恋栈权位不成?”

韩冈不在乎一张清凉伞,王安石是更不在乎,可他不信吕惠卿能跟他一般想法。韩冈本就想跟王安石开诚布公的谈一谈,以眼下的局面,当然是越早越好。

书房内,王安石正坐在桌前,翻阅着刚到手的新书。那张巨幅的桌案也完全被书卷和纸张给遮盖了,甚至有好些书都掉到了地上。

王旁见状忙走过去,帮忙收拾起来。

“岳父好兴致啊。”韩冈则笑盈盈的上前行礼。

同样上表辞官的王安石并没有敌视韩冈的意思,转过身,正面对着韩冈:“玉昆,你来了啊。”

“是的,韩冈来了。”韩冈又躬了躬身。

王安石老了,皱纹和老人斑越来越明显,从外相上看,他比半年前至少老了五六岁。可见王安石这半年多来,为了朝政付出了多少。

“江州司马青衫湿,梨园弟子白发新。”韩冈走到桌边,低头看着王安石摆在桌上的文字,“岳父又是在做集句?”

王安石喜欢集句,也就是把别人的诗作词作,东拉一句,西扯一句,拼凑出一篇诗文来,或者就是凑一副对联。算是文字游戏。不过王安石水平高,凑合起来的诗词,多有超过原篇的情况。

只不过王安石是有名的两脚书橱,撰写诗文的时候,典故、韵脚什么的,根本都不用翻书,全凭自身的积累。将书铺了满桌子的情况,十分少见。一句一句的摆上去凑,苦吟之态,更有几分贾岛的味道。

这是准备要悠游林下吗?当真将事情都交托给吕惠卿不成。韩冈心中犯嘀咕。

王安石怅然一叹:“前曰做联,这一句始终对不上,幸亏有蔡天启来。得了他的指点。”

“蔡天启?”韩冈没听过这个名字。

“蔡子雍的儿子,名肇。上一科中了进士。这两年在国子监中。”

韩冈惊讶起来:“蔡渊的儿子都中进士了?!”

蔡子雍,韩冈是认识的。其名为渊,与韩冈同在熙宁六年中进士,不过年纪偏长,整整四十。有个元丰二年中进士的儿子,现在想想也不足为奇。

蔡渊是丹阳人,曾在王安石门下听讲,也难怪蔡天启能够随意的进出韩家。

王安石眼皮耷拉着,看着就没什么精神,只有叹气声响亮:“人老了,记姓也差了。集句起来越来越难。”

“岳父如何现在就称老?‘风定花尤落’这一句,不是岳父别人也对不上,岂是今曰可比?”

风定花尤落是静而动,世人过去认为是绝对,很难在过去的诗文中找到合用的下联。但王安石却轻易的找到了,而且是传唱极广的一首。‘鸟鸣山更幽’是动而静。两句并列比‘蝉噪林逾静,鸟鸣山更幽’对仗得更工整。

“说到对仗工整。记得过去也曾有一绝对,最后是石曼卿[石延年]给对出来的。”

“是这个?”王安石伸手去翻桌上,翻了半天翻出一张纸来,上面写满了诗句,大概是集句时来凑句子的。其中给他指着的一句让韩冈很熟悉。

“天若有情天亦老?”

“正是。”王旁应声道:“记得石曼卿对了一句‘月如无恨月常圆’。”

王安石摇了摇头:“义蕴甚浅,相去不可以道里计……”转过来,他对韩冈道:“集句多是百衲衣,游文戏字罢了。便是做得再好也有些突兀的地方。”

“……说的也是。”韩冈不知何故迟了一步才反应过来,“不过之前岳父寄来的《胡笳十八拍》,却是浑若天成。”

“玉昆你什么时候会评诗了?”王旁在旁笑问道。

“君子远庖厨,小弟还知道酒菜好吃难吃呢……”韩冈笑了一声。看看王安石,笑意又浮了起来,“岳父倒是要例外。”

王安石从来都是盯着面前的一盘菜吃,此事亲朋好友中无人不知。曾有一次王安石赴宴,只盯着鹿肉吃,有人以为他喜欢鹿肉。不过韩冈的岳母让人鹿肉挪远,换成另外一盘菜在面前,王安石就又只盯着那盘菜吃了。还有在仁宗面前做御制诗,苦吟之下无意中把鱼食一颗颗都吃下去。他吃饭不论好坏,这例子一一数起来,可不是一天半天能说完的。

“老夫例外不了。玉昆,你才是例外。”

韩冈不通诗词,他对外界一直都是这样的宣传。不过很多人都认为他其实是不想因诗词而乱正道,所以他故意掩盖了真正的水平,本身还是很有才华的。

王安石却不那么看。毕竟一遇到诗文的话题,韩冈往往都会避开。不但不作诗作词,就是评诗评词也没有过。从他平常的文章和奏表中,也能看得出韩冈在文学才华的匮乏。彻头彻尾的不做诗文,是异类中的异类。

“诗言志,歌永言。诗词昭人心。韩冈只需看看诗词中的志向,用不着有好才华。”

“志向?程颢的志向,玉昆你知不知道?”

“伯淳先生在京已半年,岳父倒是不介意。韩冈要回来却半点不客气。”韩冈拉下脸来询问,他很想知道王安石到底为什么极力阻止自己入京,“为何如此厚此薄彼?”

“此辈不足为虑。”

韩冈拱拱手:“承蒙岳父看重。”

韩冈与王安石,一见面就闹起了口舌之争。你来我往,让外人看的过瘾得很。

只是王安石变得不耐烦起来:“乾称父,坤称母。何谓天,何谓地?”

‘乾称父,坤称母’出自《订顽》[西铭],是张载亲撰的气学总纲。但这一篇文字,却与韩冈主张的格物之道无法融合。从韩冈的理论中,完全推导不出君臣纲常——天子为天地嫡子,大臣乃天子家相:‘大君者,吾父母宗子;其大臣,宗子之家相也’——差得太远了。天人之论与格物致知之间的裂隙,大到无法弥补。世界观分道扬镳,这是气学最大的漏洞。

“天地者,自然也。人存天地间,就是生活在自然之中。至于抬头看到的天,近的是地外云气,远的则是虚空星辰。”

“不见圣人之言。”

“韩冈从不认为有万世不易之法。纵使先圣之论,合于道,则承习之,悖于道,则摒弃之。传抄千载,谁知道里面有多少与原文相悖之处?”

“玉昆,你就这么跟太子说?”王安石口气轻松,神色却严肃起来。

“如何不能?”

“外公!爹爹!要吃饭了。”软糯糯的声音打断了韩冈与王安石的争论。

自家的女儿适时的出现在书房的门口。

韩冈不禁微笑。自家的女儿总是在最合适的时候登场呢。王安石的神色也同样缓和了下来。

每次韩冈登门拜访,一进王安石的书房,最后被派来找翁婿二人吃饭的都是怯生生站在门口的小丫头。

王安石孙辈中唯一的女孩儿,不仅是在家里,在王安石夫妻这边,也是最得宠爱的一个。王安石和韩冈私下里见面,少不了都要争上几句。能把剑拔弩张的气氛缓和下来的,也只有韩家的大姐儿了。

“知道啦。”韩冈立刻把跟王安石的争论都丢到一边去,走过去把女儿抱了起来。

王安石也理了理桌子,不准备跟韩冈争了。朝堂上有吕惠卿,资善堂还有他自己,总有办法压住韩冈。

“对了,岳父。”韩冈出门前又回头。

“什么?”

“石曼卿对得那一联,其实小婿也有一句下联。”

“哦?那就要洗耳恭听了。”

“天若有情天亦老,人间正道是沧桑。”

韩冈随着话声离开,房中一片寂静。

人间正道——

韩冈和王安石争得就是这一事。

到了最后他都不肯让去半步。

王旁干笑道:“玉昆的这一句对得一点都不工整啊。”

“工整?”

王安石哼哼着站起了身。手扶着椅背,将佝偻的腰杆挺直,僵硬的关节几声闷响,整个人忽的精神焕发起来,

“他是在说走着瞧!”太子太傅的声音前所未有的洪亮,冲着儿子嚷嚷:“走着瞧啊!”

