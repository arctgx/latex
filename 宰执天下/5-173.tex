\section{第20章 土中骨石千载迷(二)}

赵顼拿着刚刚草拟出台的药典凡例,细细读着。

所谓凡例,就是发凡以言例,一部典籍的宗旨、体例和结构,还有一些需要特别说明的地方,都要在其中加以阐述。

虽说对韩冈有所成见,但他所主张的自然之道,赵顼也知道其中有着极大的价值,一边细细翻看,一边听着苏颂的解说。

“药者,治病草也。声从乐,以勺切。乃治病之草之总名,是故药典号本草。”

赵顼低头翻看,随口道:“苏卿也解字?”

苏颂深吸一口气,将纷乱的心绪收拢,沉声道:“臣于此道不敢称能。不知源流,如何解字?正如不溯其源流,便无法给药物分类一样。”

赵顼抬眼深深的盯了苏颂一下,“苏卿是在评《字说》?”

“王安石的《字说》,只循楷书为解,却不知圣人书文,用的乃是大篆。至于大篆之前,更有仓颉所创古字。此可谓刻舟求剑。”

眼前的文字方才还让人放不下,可转眼间便被苏颂败了兴致。赵顼放下了札子:“难道苏卿你找到文字的源流了。是仓颉之字?还是嬴伯益之字?又或是太史籀之字?”

天子质问的声音凛凛生寒,苏颂摇头:“四五千年前的东西……不过的确找到一些比周鼎更早些的实物了,也是得了陛下的福佑。”

……………………

苏颂去崇政殿为天子讲学,而韩冈则是为厚生司中事来政事堂拜见王珪。两天后京城的两座医院就要正式开张了,许多手续必须要经过政事堂走上一圈。

迎接韩冈的不仅仅是王珪,蔡确不知是巧合还是故意,竟也在场。下面还有几名中书辖下的官员,在旁边听候指派。

“一边要编纂《本草纲目》,一边还将太常寺、厚生司和太医局打理得一丝不乱。医院建好了,灾民也救治了,玉昆这一个月来可是辛苦了。”

王珪跟韩冈关系不差,迎了韩冈进来后,笑呵呵的说着好话。

蔡确在旁也附和着:“玉昆一人兼数任,的确是辛苦。不过能者多劳,论才干、论器识,朝中比得上玉昆的也没几人。”

韩冈欠了欠身:“辛苦倒是不辛苦,救治灾民乃是司中分内事,有法度可循。《本草纲目》眼下还是在整理药材,除了一个分类和凡例外,需要韩冈动手的不多。倒有空坐下来读书。”

“哦,玉昆最近在看什么书?”王珪随口问道。

“《字说》。”

韩冈此话一出,王珪和蔡确便相视一笑,果然是不肯安安生生的做事。这一回,到时要洗耳恭听韩冈的高论了。

蔡确一副很好奇的模样:“《字说》乃是令岳王介甫心血所寄,难道玉昆你准备在里面挑出什么错来?”

“挑错?”韩冈大笑,“从根子开始就是错的,如何去挑?仓颉造字,鬼神夜哭,自此上古之民不须再结绳记事,而有文字可传承。从石鼓文、籀文和周鼎上的金文中可以得知,文字在春秋为一变,是为大篆。至秦一统,又为一变,是为李斯小篆。等到汉时再一变,隶书成了主流。至于如今通用的楷书,始于汉末,到了西晋方才通行于世。字体演化,如同草木之生,乃是渐进而成。故而解字,需追本溯源,不当以今字论之。”

韩冈话声朗朗,“许叔重【许慎】何以将籀文录入书中,不正是为了返本溯源?不从上古圣人创字时寻找本意,一部《字说》也只是刻舟求剑之文。船都行出数百里了,怎么能指着船帮子上的刀痕说我的剑就在这下面的水里?都已经过去几千年了!往前数千年,仓颉所创之字,是与金文相类,还是与石鼓文相类,抑或是蝌蚪文。更甚者,乃是别有另一番书体?必须明了此事,方才可以解字。”

“难道玉昆你找到了仓颉的实物?”蔡确故作惊讶的问道,脸上却写满了不信。

……………………

苏颂对赵顼隐隐的怒意并不放在心上:“不知陛下可知何为龙骨?”

“龙骨?”赵顼一时间疑惑起来,不知苏颂的葫芦里面卖的什么药,没事提船只的底梁做什么。随侍在侧的宋用臣附在他耳边轻声说了两句,让赵顼随即反应过来:“苏颂你说的是《本草》上的龙骨?”

苏颂点头:“陛下明鉴,这一次的发现正是从龙骨中来。”

…………………………

“龙骨?”王珪不知道韩冈为什么要提到这一个药材。不过被戏称为至宝丹这剂名贵成药的王珪,对医书也的确有几分了解,“龙骨主心腹鬼注,精物老魅,咳逆,泄利,脓血,女子漏下……”

宋人有不为良相便为良医的说法,士大夫们是以一个比一个更深悉医术。政事堂正厅中在座的五六人,人人都通读过《神农本草经》。

这边王珪刚把药用背完,蔡确又接下去背起了产地:“龙骨生晋地,山谷阴,大水所过处,是龙死骨也,青白者善,十二月采,或无时,龙骨畏干漆,蜀椒,理石。龙齿大寒,治惊痫,久服轻身。”

韩冈笑了起来:“看来相公和参政比韩冈更适合去主编《本草纲目》。”

“在玉昆面前说医书,那可是贻笑方家。”王珪摇摇头,不开玩笑,“玉昆,你说的龙骨又有何意?”

韩冈收笑容,正色道:“龙骨生河东,隐于山谷溪涧之下。出产稀少。所以世间所用龙骨多是从各处地下随意挖出来的,极少有河东珍品。近日韩冈编纂《本草纲目》,要检视药材,另外也有几张验方须用到龙骨,所以让人从城中的药房搜集了一批来……”

……………………

“历代《本草》中所说的龙骨,都是误以为是死龙的骨骸,但其实乃是兽类的骨骼,埋入土中多年后化石而成。龙身似蛇,四足五爪,而掘出来龙骨,腰肋乃至腿骨,拼接起来后,大者形似犀象,小者也似野兽,并非龙形。”

苏颂似是跑了题,赵顼耐着性子听着他说。

“不过药名之误,也没必要多计较,只要有功效便可入药。如今的龙骨若是用河东正品,一剂少不得也要两三百文,所以东京城中的药方里面,多有用他处龙骨冒充河东之物。效果也不算太差。前日编修局中搜检天下药材,便传话让各个药铺里的龙骨按着产地不同都找了几份样品来合药……”

苏颂停了一下,见赵顼虽皱着眉,但还是听得神情专注,安心下来继续道,“但臣与韩冈使人将不同地方的龙骨找来,大多与河东相差仿佛,可只有一个地方出产的龙骨却不对。

“怎么不对?”赵顼有几分不耐烦,“难道是龙骨上生了字?”

“的确生了字,且那里的龙骨,质地有别,种类亦有别。并非是犀象之种,乃是龟鳖甲壳,以及牛的肩胛骨。”

……………………

“鳖甲,牛骨?”王珪和蔡确听到这里,已经隐隐抓到了一点头绪。

韩冈微笑:“殷人尚鬼神,重占卜,每欲出战,非卜胜不出。敢问相公和参政,殷人是怎么占卜的?”

“似乎是拿龟壳或是牛骨放在火上烤,看裂纹。卜者,灼剥龟也,象灸龟之形,一曰象龟兆之纵横。”这是《说文解字》中的解释,王珪论才学也不稍逊与人,倒是一口就背出来了。

‘易有圣人之道四焉:以言者尚其辞,以动者尚其变,以制器者尚其象,以卜筮者尚其占。’夏易曰《连山》,殷易曰《归藏》。文王衍八卦,另得《周易》,虽有自出机杼的成分,但也不可能与《连山》、《归藏》有着截然之别。殷人的占卜之法,必然是在《归藏》中。”韩冈悠悠然的问道,“敢问相公、参政,在占卜之后,殷人又是如何做的?”

……………………

“刻卜辞于其上以记之……”赵顼霍然而站,指着苏颂,嘴唇直在发抖。“难……难道……”

“陛下可知那堆龙骨出于何处?”终于解开了谜底,苏颂像个真正的老师一般问着赵顼。

“不是河东……”赵顼的声音干涩,对苏颂和他身后的韩冈的用意,已经一清二楚,“是亳殷,还是商人建都的其他去处?”

赵顼的脸色阴阴泛青,为了一争是非,竟然掘了商都?他倒不怀疑韩冈会作假,但同样的,他也不会相信事情真有苏颂说得那么巧。

天子的态度,苏颂并不在意,很平静的回答:“在相州,安阳。”

……………………

“相州!安阳!洹水之南!”韩冈平和冷澈的声音在政事堂中回响,“……殷墟!”

蔡确和王珪都定定的望着韩冈,脸上阴晴不定,都是没想到韩冈竟然还有这一手。

不过真伪尚不得而知,谁知道是不是韩冈让人伪造出来做证据的。这样的例子过去实在太多,别的不说,《尚书》的今古文之争,就是在争一个孰真孰伪。

一名陪侍在侧的中书门下的官员出声反驳:“端明,盘庚五迁,治于亳殷,殷墟当是在亳州。”

“章邯降楚,盟于‘洹水南,殷墟上’。”

