\section{第21章 欲寻佳木归圣众(九)}

身边放着心爱的望远镜,空气又是难得的澄澈,七夕的夜晚,韩冈正在窗前。

托京城中越来越重的煤烟污染的福,入夏以来至少有一半的时候,太阳仿佛是隔了一层纱一般的黯淡——虽然气温还是热得能把沥青融化,开封城北的铁场附近的那几条用沥青拌合煤渣铺成的道路,已经被来往的车马碾出了一条条黑色的沟壑。

但韩冈没有拿着望远镜,去应时应景的看一看被银河横隔的两颗亮星,或是试试能不能找到一直想找的仙女座大星云——天知道,要把希腊时代的星座对应到三垣二十八宿中来,到底有多难?十二宫倒是很早就传到了中国,占星术中很常见,佛经中也有出现,一开始译名有些差异,如今与后世就只有些许区别了。可是其他星座的难度就太高了,尤其是对韩冈这个半调子都算不上的所谓的天文学家来说,更是如此。

韩冈正在审查新一期《自然》的小样。

《自然》是自然学会的核心刊物,也是气学格物学派的宣传阵地,更是如今士人心目中,一切有关自然议题的权威书刊。

从三年前开始,《自然》每年都会出一套合订本,将一年来,期刊上物理、数学、化学、生物、地理这五个分科的论文,按照学科的不同,分别集结成册,用以对外出售。

《自然》,包括合订本,只要是自然学会的核心会员都能免费收到,普通会员只要缴纳会费也能得到——会费中已经包括了期刊的费用,而不属于学会的普通人,也都能在大多数城市中的邮政局来购买和订阅,至于无钱购买,还能通过各地州学、县学中的公共图书馆,借阅、抄录——各地图书馆中,自然学会都捐赠了大量书籍,只要是学会出版的书籍,都能在这些图书馆中找到。

有了合订本,日后进行研究,想要查询相应的论文来,就容易了许多。而且在学会的计划中,将会五年一修论文目录,刊印论文的题目、作者和主要内容,以便学者们进行检索。

江南诸路的大城市,《自然》以及衍生刊物,销售量是个巨大的数目,也是如今初创的邮政系统最大的客户,每个月的销售量都超过了八万份,年内有望达到十万份。

在这样大的销售量面前,雕版印刷已经无法支撑印刷上的需求。木质的印版,无法承受住万次以上的印刷,往往几千次,印版上的字迹便会被磨光,学会总不可能为一页纸,刻上几十近百块雕版。

所以就有了韩冈力主的对活字印刷技术改进,但在活字印刷术出现他所期待的成果之前,已经出人意料的,在另一个方向,有了让韩冈惊喜的成果。

现如今,成本最低,印刷效果最好,不是韩冈让人去研究的铅活字印刷术,而是石印技术。

韩冈手上的这本小样,便是石印技术最好的展现。

石印主要利用的是对水和油的亲疏关系这个原理,在石头上刷上一层酸性的胶液,最后再利用油墨来印刷。

这是一个出乎意料的成功,也是到现阶段位置,韩冈通过各种途径进行技术扩散的最好的成果之一。

油墨,出自韩冈,在石油制墨的那个笑话之后,真正的油墨很快便问世。而在处理印石的化学药品,没有三酸的出现和上规模的制造,也不可能被发明。如果没有人通过酸液研究石灰石的成分,自然连印刷的底板也不会有。如果没有以《自然》为主的刊物进行知识的扩散,又有谁能将这些技术结合起来,发明石印?

石印技术自面世之后,在韩冈的力主下,很快就流传了出去。

才两年时间,不说京师,杭州的石印坊都已经有了五六家,而福建的建阳——也就是粗制滥造有名、而印刷数量更有名的福建版的产地——则是一下涌现了近二十家石印坊。

就连国子监的书坊也开始采用石印技术。而自然学会名下的印书坊,已经正在试行彩色套印。

同时由于韩冈化名在《自然》上的鼓动,现在世间不知有多少人将煤焦油,然后用酸碱去处理。虽然不指望能够有立竿见影的成果,但时间长了,十年、二十年、三十年、五十年,化学会继续进步,迟早会有化学染料的出现,韩冈只要活着就会继续推动。而在这过程中,又会有多少发现和发明?

如今已经有了印刷精致的石印,等到什么时候水印技术有了突破,韩冈计划已久的国债债券,也就可以向外进行发售了。

翻过了这一期的小样,没有什么惊天动地的发现和发明,只有对已有的知识进行更深入的探讨和研究。

韩冈一直觉得这是个好现象,知识的突破要靠积累,没有巨人,哪有肩膀可站?

由于韩冈的缘故,这个时代的自然科学太偏太狭,同时也太快。这就需要更多的人来填补空缺,夯实基础。

看到一篇对金龟子一生的养殖观察记录,韩冈不禁端起桌上的杯子,大大的喝了一口。

这篇论文,当浮一大白。

韩冈觉得,若有人能写下《昆虫记》,甚至放灯三日都配得上。

端着杯子,韩冈又啜了一口,沁凉甘甜的感觉,让他心中更是一片舒坦。

大号的玻璃杯中,盛满了红殷殷的液体,不是酒,而是西瓜汁。韩冈不喜欢将西瓜吃得汁水淋漓,但他喜欢西瓜,故而家中都这样处理。

玻璃杯的外壁上挂着晶莹的露珠,这是里面掺了冰块的结果。韩家贮藏的食用冰块,是用深井井水再烧开之后冻成的,因其比较清洁。河冰用来降温,却不会下肚。大户人家、包括宫中,对冰块都是如此处理。

放下杯子,收起小样,下面还有一本大小厚薄、乃至纸质都十分相似的印刷品。

不过不是小样,而是已经付梓的印品。虽然不是石印,但印刷的水平已经是雕版印刷中的最高等级。

扉页上的刊名,也许是受到了《自然》的影响,同样也只有两个字——

《科学》。

迥异于后世,科学二字的意义,是科举之学,而且比较生僻,典籍中出现的不多。

这是一本刚刚创刊的新期刊。以《科学》为名,内容也理所当然的科举之学。

韩冈拿起书,飞快的翻着。

里面尽是某某名儒、某某学官点评每一科高中的试卷,还有各地解试中出现的题目,以及对拔贡贡生试卷的点评。同时还有国子监中,日常考试的题目,以及监中教授讲学的内容。

可想而知,此书一出,必将洛阳纸贵,受欢迎的程度不在《自然》之下,等到几期过后,销售量多半就会超越《自然》。就是在未来,也没有多少书能卖得过教辅教材的。

国子监中还是有能人的。

“陆佃果然有些能耐。”韩冈轻叹着。

“官人,是想吃什么吗?”

严素心正好进门来,不知把韩冈的自言自语听成了什么。

韩冈回头,微笑着问道:“结束了?”

“都结束了,姐姐她们正在收拾。”

一年一度的乞巧,是这个时代唯一独属于女性的节日,月亮刚刚升起,家中的女眷便都到了后园中,祭拜祈祷,弄些没来由的仪式。

韩冈伸了个懒腰,起来在房中活动手脚,道:“今天还真是快。”

“都二更天了,哪里还快了。”严素心上来帮着收拾。

《科学》丢在一旁,韩冈也没去管,就这么一回事,除了占了这个名字让韩冈觉得可惜,没什么好让人担心的了。

金陵书院中,王安石正在发挥余热,努力教育新学的下一代。国子监中,陆佃弄出了《科学》,要让更多士人来研习新学。

在科举改革上,韩冈做得并不算过分,尽管解试加考,士人也只要通读《幼学琼林》就行了,里面的内容也是皆有实证的自然常识和算学知识。而当年的王安石,以自家的三经新义为钦定的释义,不管你是哪一派的弟子,甚至已经是饱学鸿儒,想要考中进士,就必须低头,放弃自己原有的学问。

但科举过后,还有谁去在意新学,不是要参加科举,又有几个士人会去研习《三经新义》?而自然科学的爱好,不仅可以贯彻终身,更能普及大众。

韩冈不介意,未来是在他这边的,陆佃的《科学》也好,王安石的金陵书院也好,都不过是垂死挣扎罢了。

就像上了刑场的人犯,即使如何挣扎,也逃不过枭首一刀。

正这么想着,一封紧急军报送抵到了宰相府上,很快便被人呈到韩冈面前。

“官人,奴家先出去了。”

严素心连忙要退出,不敢打扰韩冈处置公事。

韩冈拆看后,扬扬手道:“没关系的,是捷报,西南行营又赢了一阵……不出意外的话,再有一月半月,大理就要归附了。”

“当真?!”严素心欣喜道。

“看那边的战事发展,当不会有问题。”

不知道听到这个消息时,御史台中那些蠢蠢欲动的蠢货,又会做出什么样的反应?韩冈想着。
