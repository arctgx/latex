\section{第十章 却惭横刀问戎昭(20)}

【第二更。】

“龙图声名早已闻达于天下。听到龙图要亲审,一干人等多半就慌了,等到一过堂,哪里还嘴硬到底的胆子?当年包侍制权知开封,许多时候,他问上一句,被审的全都是老老实实的答话,没人敢于欺瞒一句。”

李宪拿着包拯来比韩冈,算是一句奉承,但黄裳倒没觉得不合适。包拯最多也不过一个直学士,韩冈可是正经八百的学士,且在民间的声望,也不比阎罗包老稍逊。

“不过这个案子如此定案,刘六回去,就不知道能不能养得大。”黄裳叹道:“就怕有人贪于刘家的产业,铤而走险。小儿易夭折,一般而言,是没人会深究的。”

有些富户晚年得子,成年的兄长为了不至于减少自己能分到的家产,对于幼弟往往都是杀之而后快,这样的案子其实并不鲜见。但只要是无人首告,官府根本就不会去追查。尤其是一般人晚年得子,往往都是妾室所生,没有可以依仗的娘家,绝大多数时候都是任人摆布。因为种痘法已经在河东逐步推广,居高不下的幼儿死亡率会有一个大的降幅,但要说降得多低,还是不可能的。

但李宪知道,黄裳的铤而走险,说的不是刘六的两位兄长,而是其他贪婪的黑手,“万一刘六夭折,刘家可就不妙了。就是当真是病夭,也会被说成是二兄害死。府中、县中的豺狼虎豹,不将刘家拆解入腹,恐怕是不会干休的。”

“正是这个道理。”黄裳点头,“刘六一旦夭折,到时候,刘王氏肯定是要与刘大和刘四拼命。渔翁得利的不知有多少。”

说了半日,天色已近黄昏,黄裳时不时的偷眼去看韩冈,心浮气躁起来。

李宪在公厅中坐了半日下来,他带回了有关陕西的最新军情,韩冈却连问都没有问,一直低头在审阅着判状。黄裳虽然虽然在谈笑,但他的胃都开始疼了。陪客陪得太久了,难道要等到夜里不成?但韩冈依然是稳如泰山,将卷宗一份份的查看。而李宪则是并不在意的样子,与黄裳聊着天。

韩冈宁可晾着李宪,也要先行审查太原各县呈送上来的判状,直白明确的表明了自己的态度。不过李宪就坐在公厅中等候韩冈,他的态度也同样明确。

黄裳再没有经验,也能明了两人不合常理的举动所代表的意义,但他并不是很明白,为什么韩冈突然间对西北的战局冷淡了下来。

最终韩冈还是在暮色降临前结束了他的工作,放下手中的卷宗和毛笔,亲手整理好,将之交给一名小吏,装订好,然后送去京城。

“劳都知久候。”韩冈走过来,先向李宪表示歉意。

李宪笑道:“龙图乃是为了国事,李宪自当静候。”

韩冈和李宪终于接上话,黄裳如释重负,松了一口气下来。

李宪带回来的消息可算是一个噩耗,增援西夏的确定是阻卜人。而这个证明,付出的代价是两个骑兵指挥受到重创,种朴以下两百余名官兵伤亡,夏州通往宥州、盐州的道路由此中断。

“种朴是在出巡时被阻卜人突袭的。”李宪向韩冈详细的述说着他所掌握的情报——河东陕西互不统属,陕西发生的事,从正规渠道来走,必须要在京城绕一圈。李宪来得虽迟,却比京城的信报要早,“当时阻卜人的兵力,据说是种朴麾下官军的三倍,又是风沙阻碍了视线,没有及早发现,故而受到突袭。”

下面报上来的敌情,尤其是与败阵消息同时传来的时候,都是要打个折扣来听的。由此来推断,种朴的对手应该与他的麾下人数相当或略多。不过受到这个数目的骑兵突袭,而且还是阻卜人,没有全军溃散,也可以算是很了不起了,当时种朴的功劳。

李宪的叙述正好韩冈的想法相一致,“幸好种朴应对有方,让全军下马列阵。”李宪说到这里,轻声慨叹,“官军终究还是不擅长骑战,比起换马冲锋,倒是更习惯下马列阵。”

韩冈听着也摇了摇头。骑兵竟然下马作战,又不是龙骑兵。配了一人双马是做什么的?但话说回来,种朴选择了下马并没有错。错的是宋军骑兵的战斗力,依然是个悲剧。

“官军列阵之后,尽管只有弓箭、腰刀,连神臂弓都没有,却依然让阻卜人吃了不小的苦头。阻卜人见战局不利,便立刻撤退了。不过他们离开之前,却将官军聚集在后方的战马抢了一半去。而种朴就是在这时候,受的重伤。”

阻卜人劫掠成性,却没有硬拼的打算,如果发现反抗过于激烈,想要达到目的付出的代价太大,就会直接撤离。这一点,可以在麟州受到攻击却没有被攻破的那个村庄上得到证实。但种朴的重伤,实在是运气不好,韩冈只希望他不要落下什么伤残。

至于战马的事,韩冈很通情达理:“被抢了战马,那还真是没办法。四百多人列阵,要想守住八百匹马,几乎是不可能的。只被抢了一半去,也算是运气。”

“但那毕竟是战马!”李宪着重强调,大宋可是不缺人,只缺马,“种朴作战不力的罪名肯定是逃不掉的。”

韩冈不太喜欢这样的观点,人比马更重要,上过阵的老兵更是如此。种朴麾下能以步弓抵挡骑兵,那都是能派上大用的精锐了。

“另一个指挥呢?”韩冈转过去问道。

“另一个指挥据闻是正面与敌军交锋。总数两百多的伤亡,有七成是他们的。”

韩冈闻言叹了口气,官军的骑兵还是用来当斥候好了。与党项的铁鹞子斗一斗还没问题,遇上真正的强敌,依然差得远。想破北方的骑兵,从步兵上挖掘潜力吧。

“阻卜出动援助西夏的兵力,应当是在一万上下。”所谓有识之士,所见略同,李宪的看法与韩冈差不多,“不可能再多了,党项人支撑不起太多的援军。多了反而会生乱。如果要向朝廷证明辽国对西夏的支持,有一万阻卜人也已经足够。而出现的仅仅是阻卜人的话,日后耶律乙辛推个干干净净,朝廷也没办法。”

韩冈摇头,笑了笑,“耶律乙辛本也没有打算隐瞒什么。只要有个借口,我们这边的朝堂上,谁也不敢主动撕毁澶渊之盟。”

耶律乙辛的确不简单,辽人一向善于乘火打劫,而他做得尤为出色。前些年的代北边界之争,已经表现出了他过人的眼力和手段。眼下插手宋夏两国的国运之争,在他的眼中,除了唇亡齿寒的原因外,更多的还是想利用压榨大宋的成果,来巩固自己在国中的地位。如果当真给他成功了,过一两年,辽国多半就要换上一名新君——不会是父祖两代四口,都死在耶律乙辛手中的耶律延禧。

如果西夏兵败,官军占了银夏,还要防着耶律乙辛直接出兵占据兴灵。若是官军败了,多半会被他逼着放弃横山北侧,甚至更多的土地。

这样的局面,还是天子和宰辅们送给他。

赌徒赌输了之后,总会想翻本。他们的想法,韩冈无法理解,但他见得多了。天子既然选择了继续他的赌博,那么压上去的赌注被对家吃掉,也没什么好惊讶。

“龙图好象是一点也不在意。”

待李宪告辞离开,黄裳便忍不住出言试探韩冈。他觉得韩冈的养气功夫着实让人佩服,阻卜人已经断了盐州的援兵通道了,怎么之前的焦急现在却一点不见了?

韩冈端了杯茶水,不紧不慢的喝着,“事已至此,还有必要心急上火吗?”

黄裳脸色一白,想不到韩冈已经是认命了。

韩冈慢吞吞的说道:“徐禧守不住盐州的结果,也不过是契丹人逞威风而已。当年是元昊领军来攻,连着三次全军覆没,也不过给契丹人讹去了二十万岁币。如今只是攻夏不克,远比当年的情况要强上不少,能给契丹占多少便宜去?”

若是官军能守住盐州,那是最好。对韩冈来说,让吕惠卿、徐禧得意去也无所谓,至少西夏灭亡定了。可若是守不住盐州,只要能退保银州、夏州,接下去不过是就是暂时换回守势而已,对辽国的战争讹诈也不用害怕,最多也不过是一些边境的冲突,整体上依然是属于外交的范畴。

在韩冈的理解中,所谓的外交,不就是扯皮?双方就各自的利益讨价还价罢了。

“我就不信,耶律乙辛当真敢撕毁澶渊之盟!当年承天太后能打到黄河边,这一次,我让他的西京道都丢掉!”

黄裳听韩冈如此强硬的说着。但在韩冈的脸上,他却发现了深深的遗憾。

黄裳心中不禁感慨起来,金玉良言被天子置之脑后,如弃土石,而一干祸国殃民的激进之策,却成了天子,落到如今这样的结果,难道不该悔恨没有听从韩冈、郭逵这样老于兵事的臣子的建议?

也难怪韩冈会遗憾。黄裳想着。换做自己,早就心灰意冷,请郡去南方佳处休养了。才不会来河东烦心,为天子和一干国蠹收拾残局。

韩冈静静的喝着茶,他私心里的确很是遗憾。

可惜自己是坐在太原府中,无法亲眼看见赵顼和吕惠卿听到这个消息时的反应,这当真是让人十万分的遗憾!

