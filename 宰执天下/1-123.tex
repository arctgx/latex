\section{第45章 樊楼春色难留意(三)}

【第二更,红票,收藏。有谁能告诉俺,玉|娘两个字到底是哪里犯禁了?】

现在是白天,离午时还有两刻,樊楼中相对于夜中,却是安静了许多,没有妓女在桥廊上待客。不过所谓的安静,也只是相对而言。实际上,就在一楼的散客厅中,还是有二三十张桌子坐着人。

见着韩冈、章俞他们进门,楼中跑堂的小二——俗称‘大伯’的——就迎了上来。

“福泉!”章俞侧头唤了一声,他身后的伴当便会意上前,拦着小二道,“我家老爷今日请得贵客,找个清静的院厅。再看看哪位行首得空,也一并请来。”

小二听了,忙答应着。找了人过来吩咐了几句,自己则引着韩冈他们往北楼走。

上了北楼二楼,被领进一间宽敞的包厢中。韩冈打量着包厢内的装潢,的确素雅清净,而且处处都能看到菖蒲的花纹,无论家具摆设还是门窗墙壁。韩冈心中了然,京城中的酒楼,包厢庭院多以花为名,也有的取自典故,樊楼自不会例外。但每一间包厢的布置,都是这般有着独一无二的配置,可以想见店主在其中花费的心力和钱财,肯定不在少数。

众人一番谦让,就此坐定。很快,专管点菜的茶饭量酒博士,便领着几个小子端着一些果子冷盘上来,又奉上了热茶。福泉去外面点了酒菜,韩冈听着他说了好一通,也不知点了多少。

先喝了热茶暖身,几壶筛过的酒水被拎了进来,放在开水壶里热着。酒香散入厅中,章俞为之介绍:“京城七十二家正店,家家都可自酿酒水。樊楼所酿,一名‘眉寿’、一名‘和旨’,眉寿入口浓烈,后劲十足,是老而弥坚之意。而和旨甘润,正如圣旨天霖。老夫不知玉昆酒性如何,便把两种都端上了。若是都觉得不适口,让人去外买些好酒亦可。”

韩冈不打算像刘仲武那样醉昏了头,道:“在下酒量不济,还是清淡一点。”

“那就取和旨来!”

章俞、路明陪着韩冈喝起清淡的和旨酒,刘仲武还在宿醉中,却说要用更烈性一点的眉寿来解酒。四人吃着小菜,说着闲话,就等着樊楼歌妓上场。

也没听到脚步声,敲门声却突然响起。李小六跳过去拉开门,四人一起看过去,无论是韩冈还是刘仲武,又或是路明,都有些期待。

门开了,一名歌妓出现众人眼前,后面跟着的小丫鬟双手捧着一柄曲颈琵琶。歌妓相貌朴素了一点,身材也不算出色,穿着也是素净为主,脂粉下的年纪怕是有三十岁了。

刘仲武眼中透着失望,而章俞却一副惊喜的模样,甚至冲她欠了欠身,“竟然是玉堂秀来了!”

玉堂秀当是花名,看着章俞的样子,看来她的琵琶技艺应该不错。虽然长相略逊,但自来色艺难两全,这也是常理中事。

玉堂秀进来向众人行了礼后,更不多话,坐到一边的绣墩上,接过琵琶,信手一拨,曲声便充斥于厅中。曲乐轻快,叮叮咚咚,恰如珠落玉盘,却是一首行酒令的小曲。

章俞配着曲子敬了韩冈一杯酒,压低声音说着:“玉小娘子的琵琶,可比之唐时的康昆仑,当年在富相公的甲子寿宴上,也是深得赞许。京中能与她一较高下的,也不过三数人。”

韩冈笑道:“在下不通音律,分不出好坏,听得顺耳便可。以在下看来,玉小娘子弹得的确不错。”

两人刻意压低的声音,被刘仲武听到了,他不屑道:“酒楼里的只有小姐,哪来的娘子?!”

宋时的习俗,娘子是对良家女子的称呼,而娼妓之流,就只称为小姐。只是坐在人家的地盘上,这么说可不好,是想让人在酒菜里吐口水吗?刘仲武宿醉犹未醒,说话不经大脑,声音还大得惊人。韩冈见着玉堂秀神色虽不变,但弹出的琵琶声中却分明添了两分杀气。

韩冈先瞪了刘仲武一眼,正色道:“论人当观其心。青楼中未必没有出淤泥而不染的女子。读了圣贤书的,也不是没有负心背义之徒。”

玉堂秀听得脸色一缓,神情间有了点笑意。

“官人说得正是!”一句悦耳动人的声音从门外传来,清亮中带着几许缠绵悱恻。

众人循声望去,正见一名少女,低着头,轻提裙裾跨过门槛。上提的裙裾,将一只蝶舞双双的绣花鞋露在外面,小脚纤纤,仿佛一掌可握。

跨了进来,少女双手拍了拍襦裙,呵的一声轻叹,像是完成了一项艰难的工作,放松下来后的感觉。诱人的嗓音,轻盈的体态,带着一点俏皮的动作,还没看到长相,就已让人心动不已。等她将脸轻轻扬起,众人无不惊叹出声,果然是绝色佳丽。

少女也就十七八岁的样子,松松地挽着发髻,只用一根白玉簪别住,另外也就是腰间系了一枚玉佩,除此之外,再无其他饰物。闭月羞花的白皙俏脸上亦是脂粉不施,却更显得清丽无双。少女一举手一投足,像头小鹿一般灵动,双眸隐约含情,顾盼间又能把人心都勾走。

“是花魁周小娘子!”章俞声音很轻,但惊讶并不比看到玉堂秀时稍差。

只见少女在桌前盈盈行礼:“小女子周南,拜见四位官人。”

听见周南这个名字,韩冈便笑了。这名字起得好!《周南》是《诗经》中的一部,下面有诗十一篇,最有名的就是《关雎》《桃夭》。他带着调笑之意,上上下下看了周南一通,然后赞道:“果然是窈窕淑女,灼灼其华。”

周南抿嘴轻笑,动人的媚态一瞬间绽放开来。她含嗔带喜的横了韩冈一眼,眼波流媚,又屈膝对韩冈福了一福,声音宛然如歌:“官人才是振振公子,福履绥之。”

两人的对话让章俞、路明会意而笑,刘仲武则听着有些摸不着头脑,“……你们打着什么哑谜?”

韩冈微微一笑,却也不作答。他从《关雎》《桃夭》两首诗里各摘了一句,合在一起恭维周南。而周南也同样从同属《周南》一部的《麟之趾》《樛木》两篇各摘一句,把恭维还给韩冈——

周南的敏锐反应,让韩冈一时间为之激赏。只是他见周南虽是在笑着,但一双似是含情的眸子,往深里看去,却是清如寒水,不生涟漪。

韩冈能明白原因,周南她这个名字起得是好。但凡读书人,没有不读诗经的,来来往往的文酸听到这两个字,都免不了要说笑两句。还有方才自己说得几句,也是欢场上常见的恭维,怕是她这样的对话听得多了,也没了感觉。

章俞突然拍了拍韩冈的肩膊,向两名歌妓炫耀:“老夫的这位韩贤弟,年未弱冠已是名动关西,得了王大参的青眼,请动天子亲下特旨,擢其为官,不是等闲可比。”

韩冈摇头:“韩冈不过一驽钝之才,那当得起四丈如此夸赞?”

周南轻轻道:“官人能得天子特旨,却不比进士们差了。”

“岂止不差?!”章俞提声道:“玉昆文武双全,不输当年张乖崖。老夫前日在关西道上遇上了一群饿狼,足足数百条,若不是玉昆和这位刘官人之力,老夫现在就成了狼粪了。”

周南小嘴微张,吃惊的看着韩冈,眼里透着崇拜:“官人竟有如此武勇?!”

一名绝色美人用崇拜的目光看着自己,韩冈免不了有些心旌动摇。只是一想到这样的神情至少八成是装出来的,心中又是一阵逆反性的厌烦。

“好了!”章俞拍了拍手,“玉小娘子和周小娘子,都是名传京师的花魁行首,今日齐至,却是老夫有耳福了。玉昆新近入官,正待大用,二位可有什么好曲子,为之一赞?”

“不,”韩冈立刻道,“四丈年尊。先以一曲赠四丈。”

“那就选晏相公的‘龟鹤命长松寿远’吧……”周南选定了晏殊的一首小词。韩冈和章俞也没有别的意见,点头允了。

周南粲然一笑,如百花绽放。步履轻盈的退了两步,俏生生的站在了厅中央。玉堂秀则调了调琵琶弦,定好了音。

两女正要唱曲助兴,但一阵歌声不知从何处传来,不是娇柔婉转的少女,而是带着沧桑和悲凉的老者。

听着歌声,辨清了歌词,韩冈顿时心中一凛,便抬手示意周南和玉堂秀不要干扰,自己静静的听了下去。

枯藤老树昏鸦,

小桥流水人家,

古道西风瘦马,

夕阳西下,

断肠人在天涯。

短短的二十八个字,不过五句,就听着那苍老而又沙哑的声音翻来覆去的唱着,伴奏的乐器也换成了胡琴,咿咿呀呀的拉着悲吟。

歌声流淌,樊楼春色顿无,却多了秋冬暮年的萧瑟。

韩冈苦笑摇头。才几天工夫,这首《天净沙》,怎么就传唱开来了?

但在樊楼中唱这种曲子毕竟不应景,很快便有人出来抗议:“哪家遭瘟的贼老不死,唱这鬼曲子败人兴?!要哭丧回家哭去,在樊楼里唱算什么?!!别打扰爷爷喝酒!”

