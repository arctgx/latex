\section{第17章 夜顾茅庐访遗贤(上)}

演员们已纷纷退场,但在刚刚结束了一出闹剧的戏台附近,却有两人正若有所思的看着韩冈远去的背影。两人身边,围着一队骑兵,各自下马候着,看他们的身形气度,都是精兵无疑。能有如此精锐护卫,两人自非等闲之辈。

“有风骨!”两人中的年轻人忍不住赞道。

“好聪明!”大约四十上下的中年人也不禁赞了一句。

对同一人、同一事的评价截然不同,年轻人诧异的问道:“大人这话如何说的?那韩秀才气节风骨那是没话说,但聪明可谈不上!一个服衙前役的乡秀才,得罪了一路的都钤辖,哪会有好结果?没听过向宝心胸有多广……”

“你还太年轻!”中年人摇摇头,“不过那韩秀才看上去跟二哥儿你也差不多大小,可人家的心机可比你深多了……”

“……怎么可能……”年轻人眨了眨眼睛,想明白了父亲说的意思,却不肯相信,“韩秀才又不能未卜先知,怎么知道向安会过来赔礼,而不是带着一队家丁来。”

“所以说他心机深啊!”中年人叹着,这样的年轻人当真是不多见,自己年轻时也是差得老远,“才智狠辣都不缺,还敢拼命,真是难得!”

年轻人左右晃着脑袋,韩冈的年纪与自己差不多,他怎么也不信韩冈的才智出色到能把向安的反应都算计进来。

知子莫若父,中年人呵呵笑了笑,道:“韩秀才到底人物如何,二哥儿你去跟他一谈便知。”

“大人要孩儿去跟他谈谈?”年轻人眼睛一亮。

中年人微微点头,道:“今晚你就去跟他聊聊罢,看看他的学问如何。如果真的是张子厚的学生,能帮一手就帮一手,任读书人服贱役,总之有辱斯文。若是看着他吃亏不理,日后到了蔡经略面前,也不好意思去见张子厚。”

“那孩儿直接过去好了。大人你先去歇息吧。”年轻人神色跳脱,巴不得甩开自己的老子。

“那二哥儿你就去罢。我毕竟老了,比不上你们年轻人有精神。”

中年人叹了口气,眉宇间有着深深的疲惫。韩冈与向荣贵闹得正欢的时候,他刚好进城,却被堵着了,正好看着一场好戏。中年人长得黑黑瘦瘦,不仅是因为这几个月来奔波劳碌,他本来也不是身强体壮之辈,今天一天他都在马上,到此时也支撑不住要去睡了。

一众士兵跟在中年人身后去了城中央的知城衙门,那里有专供来往官员们休息的寅宾馆,只有两名士兵留了下来,看他们的动作,像是要护卫年轻人的样子。年轻人轻轻摇头,示意两人不要跟来。整了整自己的衣服,当真如其父所说,去拜访韩冈。

………………

韩冈、王舜臣一行在赵隆的带领下在城北的一座营寨中歇了下来。往日还算空旷的营寨中,此时却挤满了商人和他们载货用的车马。这片营地,论道理就是成纪县往北方各城寨运送粮饷和犒军物资的车队规定的驻扎场地。可这些个商人鸠占鹊巢,竟把营房都占了去。赵隆领着辎重队在营内绕了一圈,硬是没找到一个落脚的地方。

赵隆看着不耐烦,卷起袖子,就要上前撵人。韩冈一把拦住他,笑道:“用不着动手。让我和军将来试试。”

“给爷爷让两间房出来!否则有你们好看!”这是拿着马鞭唱黑脸的王舜臣在表演。

“不知兄台能否让贵属挤上一挤。我等只住今夜,明天一早便上路。”韩冈则唱着红脸。

韩冈和王舜臣一软一硬,逼着占据了最大的两间营房的一名商人赶快滚蛋。两人心中都在盘算,若这位商人还敢推三阻四,就直接把尚存在车斗里的人头丢到屋里去,看他让还是不让!

“想叫俺让房,也不看看俺是为哪家官人奔走!?”商人正要发作,却被一人拉过去咬了一阵耳朵。等他再回来的时候,肥肥圆圆的一张脸上,已经堆满了职业性的笑容,看向韩冈的眼神也自不同。

“让!让!俺立刻就把营房让出来!”他点头哈腰,连声价的说道。

才就一眨眼的功夫,两间包括军官偏厢的营房就给腾了出来。民伕们一拥而入。有胆略,有能耐,会体恤人,又够威风,对韩冈,他们愈发的崇拜。

“秀才公,王大哥,你们先歇着。俺去弄点酒菜,马上就回来。”帮着众人在房中安顿下来,赵隆忙不迭地说道。他殷勤无比,差不多跟民伕们一样,都对敢落都钤辖面子的韩冈心生崇拜。

“多谢敢勇。”韩冈拱手谢过。越是在细微的地方,他越是小心在意,半点礼节也不疏忽。

赵隆出去没一会儿,半刻钟都不要,就带着一个提着食盒和酒坛的小二回来了。韩冈正在安顿受伤的民伕们休息,又安排了其他民伕去吃晚饭。见赵隆回来,韩冈抢先会了钞,自己没动,却把这些酒菜送到了民伕那里,还让小二再送一些好酒好肉过来——反正董超、薛廿八身上带的钱不少,已全给韩冈他笑纳了。

“这……”赵隆发起呆,民伕们也有些犹疑。

韩冈笑道:“今日在裴峡谷中,人人奋命,没有一人临阵退避的,若非如此,这里的各位,包括我韩冈都没一个能活!在军中,一场战后,总要弄些好酒好菜犒军,我们也不能例外……等今天的事报上去,肯定还有赏赐下来,诸位放心,韩某绝不会贪墨一文。”

“多谢秀才公!多谢秀才公!”民伕们感激涕零,连声道谢。

韩冈则回过来对赵隆道:“赵敢勇,我们还要先去城衙,把裴峡一事报上去。裴峡中的蕃部开始听命于西贼的指使,这不是一件小事,必须赶紧通报上去。”

……………………

一个时辰后,三人围坐在厢房中的桌边。三人的脸色都不好看,王舜臣怒色难掩,赵隆皱眉不屑,而韩冈看似平静,心底也是在破口大骂。

“你那个鸟副城,为了招待个鸟官,连军情大事都不理……难怪他说话没人听!”王舜臣砰砰的拍着桌子,满肚子火却无处撒气。

“副城跟俺有什么鸟关系?!”赵隆愤愤不平,“那个鸟货伏羌城上下看不过眼已经很久了。若上了阵,有机会哪个不想射他一个背上开花?!”

韩冈摇着头,不想说话,将没什么味道的淡酒一口喝下。他和王舜臣、赵隆三人去城衙通报军情,本以为留守伏羌的副城,听说连接秦州的要道——裴峡——出了贼人,会立刻接见。不曾想里面传出话来,副城有上官要招待,没时间理这等小事。‘才百八十个贼人也叫事?甘谷那边八千还要翻番!’直接就把三人给赶出来了。

赵隆又叹道:“也不知方才过来拜访秀才的小官人是哪里的,我们白跑一趟,却把秀才的事给耽误了,真是可惜。”

韩冈不介意的笑道:“若是有心,自当再来。若是无意,那也就罢了。”

“说得痛快!”王舜臣拍案叫了一声,便端起碗,“当痛饮一碗。”

韩冈连忙按住王舜臣,不让他喝酒:“军将你受了伤,不能喝酒!”

王舜臣不快,抱怨道:“光吃菜,不喝酒,那还有个鸟滋味!”

韩冈想了想,还是放了手。此间的酒水都是只见水少见酒,又不是蒸馏过的高度酒,喝一点真没什么关系。

大碗的粟米酒,大块的烧羊肉,味道算不上多好,但吃起来确实痛快。酒过三巡,虽然醉意不多,但气氛也热闹了起来。

赵隆指着王舜臣,说起了两人相识的经历:“这泼皮本是鄜延路的,不知犯了什么事,就是今年年初的时候,慌慌张张的到了秦州。到了秦州也不安生,一根马鞭闹得城中鸡犬不宁。俺找上门去评理。可这泼皮明明比俺还小,却死硬着不肯低头。最后在城外狠打一架,却是不打不相识,一来一往倒有了些交情。”

赵隆和王舜臣方才与韩冈说的大同小异,不过有一点让韩冈惊讶,王舜臣竟然比赵隆还小一点!他吃惊的问着赵隆:“不知敢勇如今年齿?”

“十九!”

韩冈呆了一呆,反过来对王舜臣问道:“军将你还不到十九?”

王舜臣干咳了两声,摸着脸上的络腮胡子,“洒家……那个……俺其实是壬辰年【西元1052,仁宗皇佑四年】生的,属龙。”

“你比我还小一岁?!”韩冈当日推算王舜臣的年纪不到二十四,本就有些难以置信,但现在当真是惊呆了。

王舜臣恼羞成怒:“俺是长得有点老……”

‘有点?’韩冈强忍着没把心里话说出来,但他的眼神还是暴露了他的心思。都说古人早熟,但早熟到王舜臣这份上,还是让他吓了一跳。

“但俺的确才十七!”王舜臣悲愤得大叫。

“好罢,好罢!”赵隆安慰的拍拍王舜臣的肩膀,嘿嘿坏笑:“就为十七岁的王军将喝一杯。”

ps:历史上的名人要登场了,虽然听说过他的并不多。但他所主持的战略,却贯穿北宋后期,多少人因此而发迹。韩冈要想在短时间内快速晋升,只有搭上这班快车。

今天第三更,求红票,收藏。

