\section{第20章 冥冥鬼神有也无(14)}

就在誓师出战的前一日,安南经略司和安南行营的文武官员,正围绕着一幅面积巨大的沙盘,对作战计划做着最后的确认。

真正的作战方案,自然是要尽量详尽,将方方面面都得考虑完备,而决不是像章惇昨日对蛮部洞主们在场的军议上那般说得——‘方略很简单。’

负责解说的陈震很是有些紧张,尽管早已经过了韩冈的耳提面命,又对计划有了充分的了解,并不要他对计划做深入的阐述,只是简略单纯的复述和总结,且每一位参加会议的文武官员手上都有一本手抄的小册子——那是今次的作战方案——但他的手还是忍不住一阵阵的颤抖。

一根细长的木杆拿在陈震颤抖的手中,指着沙盘上的一个个标识,“邕州南方军寨,古万、太平、永平三寨已经重建完成。现有荆南军四个指挥沿途坐镇。运送粮秣的船队将会由从邕州上溯至太平寨,再由马队转运到边境的永平寨中。永平寨现有存粮八万石、草两万束,太平寨三万石、草八千九百束。古万寨为转运点,存粮只有一千,草三千,但也足够为在左江边拉扯船只纤绳的四百军马提供两个月的粮料。且永平寨又有八队共六百九十八匹役马,且随时可以再投入五百匹备用军马,为官军入交趾后沿途运送粮秣。此外,盐、酱、菜、酒水、布匹、药材等资材,皆随同军粮一并运送,在此并不赘述。”

“逢辰。”等陈震的叙述告一段落,“你觉得关于粮秣转运一事还有什么要补充的?”

燕达的视线从作战方案的小册子上抬起,摇摇头,简短的回答:“没有。”

章惇又看了一眼燕达身侧的李宪,没有对他开口。走马承受没有资格被一军主帅询问战策方略,另外章惇也不会给他说话的机会。扬了扬下巴,示意陈震继续说着下一条。

陈震干咽了一口唾沫,润了润嗓子,又拿着木杆指着沙盘,“从国境的南下,第一步就是交趾的门州。据昨夜最后一次细作回报,驻守门州的主帅已经换了人,但新帅尚未抵达。这是三天前发回来的消息,想来现在新任主帅应该已经抵达门州。依靠章、韩二帅的谋划,从永平寨到富良江下游的平原,从北至南总用近两百里的山路,如今只有门州一处关卡上能抵抗。除此之外,东西千里的一片山林之中,所有的州县都已被毁,已经没有部族能够支援门州。只要攻下门州,就能够一举攻入富良江北岸的平原。”

“逢辰?”章惇又问着燕达,“首战攻打门州,你还有什么疑问或是意见?”

“没有。”燕达又摇头:“打下门州,就能与交趾人隔江对峙了。”

章惇瞥了一眼韩冈,韩冈会意开口:“就在昨天,思琅州的洞主也已经启程,邕州城中所有的洞主都已经返回本峒。依照计划,他们将会用最快的速度向交趾腹地进兵——为了比他人抢到更多的战利品,蛮部洞主们不会耽搁。但官军也要尽速南下,压制住交趾军的主力,以防止蛮军被各个击破。”

章惇再望向燕达,只见他在安南行营中的副手继续摇头,“战事有大帅、副帅运筹谋划,末将等只需依命行事。”

燕达的态度说是恭顺也可以,说是有几分腹诽,也同样合理。不过章惇和韩冈都不在意,就算燕达并不心服口服,只要他没有旗帜鲜明的表示反对,那就已经够了。

燕达本身是声震天下的名将,担任着征南行营兵马副总管一职,又是属于军中高层的横班成员,只是因为身为武将,在主帅章惇,以及副帅韩冈两名文臣的压制下,他对于南征交趾的方略和战策,都只有建议权,而没有决策权。

对于这个待遇,燕达也早有心理准备。章韩二人都是如今有名的通晓兵事的文臣,要想从他们手上抢到一份决策权。如果就跟着打就是了,如果章惇、韩冈的方略有所差池,那他也不介意趁机拿回一部分决策权。

只是让他站在一边点头应是倒也罢了,章惇和韩冈竟然提拔了多名行营参军,来处理军中诸多事务。有本属于经略司和行营的属官、将校,也有章惇、韩冈甚至燕达本人的幕僚。他们作为行营参军,参与草拟军中大小事务,甚至详细到行军路线、粮秣安排,由韩冈本人主持,并交由章惇拍板,至于燕达,则只有参与发言的资格,并不比行营参军强多少。

召集军中将校、属僚,共同谋划方略、战策,如此行事,其实几年前燕达就听说过。

第一次横山攻略失败,为了顺利的从罗兀城南下,困守在罗兀城中将领们从麾下召集了几十名年轻有为的将才,来拾遗补缺、参与军中细务,而提出这项制度的正是韩冈。

虽然在横山攻略之后,行营参军的制度很快就销声匿迹,也仅仅在河湟战事上冒了点头罢了。使用自己的亲信幕僚,行事向自己负责,这是多少年来将领们养成的习惯。尽管韩冈的做法是对军事有所裨益,但对于将领本人则免不了觉得很郁闷,一旦给自己不能控制的幕僚插手进来,比如冒领军饷,使唤军士为私家行事,等一系列违法之举那就不可能欺瞒下去。

哪一名将领也不喜欢这样的人晃在身边,这些事有自家幕僚去做就够了了,自己的阴私随时有着被人揭穿的危险,也有被人轻易架空的可能。就像安南行营,因为有着一众行营参军,所有的事务就都给章惇、韩冈抓在手中。

韩冈低头在看着沙盘,但他的心中却是在考虑着燕达的心思。

他将燕达的幕僚纳为行营参军——也就是实质上的参谋部——本来就是给燕达一个表述他自己心中构想的机会,有这位名将的意见参与进来,南攻交趾的计划可以更加完备。至于再多的权力,章惇不会出让,韩冈也不会出让。

实行参谋制度的前提本身是剥夺将领对麾下军队的控制权。

尽管早已不用担心将领如五代故事,带着麾下的士兵随意举起叛旗,但朝廷一直还是将将领们时常迁调,不让他们熟悉手下的军队。之所以会如此去做,就是因为将领在有着莫大的控制权。在军中,从装备到财计都是领军的将校们说了算,朝廷的检查制度如同孔目稀疏的筛子一样,只能偶尔筛几个倒霉蛋。。

实际的兵力只占兵籍簿上的几分之一,多出来的粮饷成了将校们的囊中私物;理应上阵杀敌的将士却成了将帅门下的走卒,洒扫庭院、做工务农;边境地带的将帅,他们名下的一支支回易商队都是用着麾下的兵员为主。

——这一桩桩、一件件,都是发生在现实中的恶行劣迹,看到他们的所作所为,给将帅们的权力不够吗?所以才必须经常调动,这样至少还能让那一干执掌军务的将帅们有些顾忌。

世间所说的将领频繁调动,造成将不识兵、兵不识将,这的确是现实;但要说对军中的战斗力造成了多大的恶劣影响,让官军不堪一战,那就不能一概而论了,真实的情况远比写在奏章上的一句两句批评更为复杂,从来不是一面倒的好与坏。

韩冈虽然年轻,却领上阵军多年,对军中情弊一目了然。世上的事,从来没有那么简单。任何已经成型的制度、规则和惯例,之所以难以变动,因为这些制度、规则以及惯例的背后,写满了两个字——利益。所以参谋制度,他直到南下作为经略招讨副使后,才开始重新推动起来。

也幸好这是行营,以战争为目的临时设置的机构,在行营中设立参谋制度,不会引起将校们的反弹。主帅章惇一心建功立业,而燕达、李信也都是心怀高远的年轻将才;加上官军的几个部分,要么是兵力与兵籍相差不大的精锐,要么就是刚刚组建,还没来得及败坏的新军;所有人的主要利益都在平灭交趾之上,而不是对士兵磨牙吮血,这样的行营推行,就会很简单。

这也是为什么当年从罗兀城撤军的时候,可以那么容易,死到临头,哪里顾得什么约定俗成的旧时规矩。换个时间、换个地点,韩冈的提议不是会被某个老将哈哈哈的拍着肩膀说句后生可畏,然后就被抛到一边去;就是背后遭人下阴招,落得不明不白的下场。

燕达虽对此也是无奈,只能加以接受。有了行营参军考虑着方方面面的事务之后,他身上的担子就轻松了许多,但他对麾下军队只剩下临阵的指挥权,除此以外,一切都是由安南经略招讨司说了算。

‘就看看行营参军能做出多少事来好了?’燕达想着,就算手中的权力实际上被夺走,只要作战指挥还在手中,他也勉强能满足了。

不论章、韩二位谋划计算了了多少,到了最后还得要让自己来击败敌军,有着这份想法,燕达倒也能感到几分舒心畅意。他可不想来广西白捡功劳。

