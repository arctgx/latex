\section{第31章 停云静听曲中意(19)}

“李使君!”广信军的通判紧追在李信的身后,苍白的脸气急败坏,“城外的辽军可是过万数了!怎么能出战?!”

“我知道。”早换上了一身军袍的李信大步向前走着,虽说是在军州衙门里面,可迎面而来的属吏还不如穿着军服的将校多。一见到李信,立刻退到一旁行起军礼。

通判是文官,可因为李信的背景,从来不敢在他面前大小声,而李信对通判又很尊重,一年多来搭档得很和睦,只是今天却变了模样。

“李使君!南下的辽师气势汹汹,锋锐正盛。眼下兵力便已多达万余,后面可能会更多,遂城兵马远不足以拦住他们,不能出战啊!”

“我知道。”跨过一重黑漆的大门,前面的军官士卒就更多了。绕门后的照壁,在宽阔的院子对面,白虎节堂就在眼前。

通判喘着气,他跟李信的身高差不多,李信的步幅也不大也不快,却让他追得汗流浃背,“而且遂城兵马半数为马军,乃为了反攻辽境,并非用来阻截辽军。”

“我知道。”李信终于回头说了一句,“仲和兄勿忧,李信自有万全的把握。”大步跨入白虎节堂的正厅,双眼左右一扫,厅中人满为患,围着中央沙盘站了一圈人,广信军中有差事在身的将校大半都在这里。此外,还夹杂着几名文官,青色官袍在一群武将中分外显眼。

待行过礼,李信站在沙盘的上首处,通判在他身侧。李信低头看着自家的地盘,沉声问道:“谢坊口的人回来了吗?”

一名年轻的军校立刻回道:“回钤辖,谢坊口村的人刚刚从南门进城,连同当地军铺的人马全都回来了,百姓和铺兵都没有怎么折损。”

李信紧绷的脸稍稍缓和了一点:“已经安排下了?”

“梁安国方才已经去了。”

“还有哪家没回来?”李信低头,眉头又皱了起来。上面的属于宋军的小红旗,在遂城以北,已经看不到几面了。

“加上谢坊口,北界的军铺今天回来的已经有三家,除了已经确定被辽人攻破的黑山村、釜山村两铺……”那名军校低头看了下沙盘,然后抬头道:“北界十九铺就只剩庞家村和广门村两处。”然后那名军校的声音又低了点,“就在谢坊口的军民进城后,辽军缀着他们将城围上了,就是庞家村、广门村的人逃出来,现在也进不了城了。”

李信暗中叹了一口气,又道:“李参军,城中人口计点出来了?”

一名文臣装束的官员随即回道,“昨日下官奉命计点城中,连妇孺在内,共计三万又四百二十四人。今天入城的还没点算,但不会过千。”

李信轻吁了一口气,心情沉重从脸上就能看得出来。

广信军如果不算军户,民户总计三千八百,人口不到两万。但若是加上军户,则人口正好翻上一倍,军民总计四万。而现在退入遂城县城中的人口,点算出来只有三万出头。除去一部分南逃的百姓,那么至少有五六千人陷落在辽军手中。

广信边界上的军铺都是放在村子中,离遂城都不算远,但离辽境则更近。在数日前,辽军猝然犯境之后,有很多都没有来得及逃离。

若是韩冈的信或是朝廷的通报来得能早上一天,至少能免掉上千名百姓死伤,而北面的黑山、釜山两个村子,也不会一个人都跑不出来。庞家村和广门村眼下看起来也是凶多吉少,也不只还剩多少下来。

李信手扶着沙盘,站直了身子,横扫厅中的眼神犀利锋锐:“我想这几天下来诸位都清楚了,辽国这一回是想往大里打了。”

厅中静了下来,等待着李信的下文。

李信指着沙盘上:“雄州霸州那边情况尚未得知,但从地理来看,已经不是百年前了。依靠塘泊堤防,三关险固远胜以往。倒是广信军这边,只有稻田树木,是边界上最大的缺口,必然是辽军主攻的目标!”

有几个文官变得脸色发白,但武将们大多神色如常。

广信军东面是安肃军,安肃军的东面则是雄州,霸州。‘河北自雄州东际海,多积水,戎人患之,未尝敢由是路。’这是时人的说法。自从大宋这边想出了放水开塘以作边防的策略,“因陂泽之地,潴水为塞”,壅塞九河中徐、鲍、沙、唐等河流,形成众多水泊,河泊相连,位于雄州、霸州的瓦桥、益津和淤口三关便有了一条几近千里的水上长城。

虽说千里塘泊现在都是冰冻状态,可地势还在,堤坝也在,自是不利冲奔。虽不能说是大军难渡的天险,可也是够难熬的。而广信军这边,没有深阔的河道,只有依靠开辟水稻田来阻敌——就是隔邻的安肃军,在其北面还有一条黑芦堤,那是故燕长城的遗址,能做堤坝挡水,自然比起广信军要安全一点——现在水稻土冻得生硬,只靠田边的沟渠和栽种的榆柳,对辽军来说,只比过一趟小树林要方便。比起太宗、真宗的时候,广信军的战略位置已经更加危险了。

宋贤是广信军辖下的都巡检使,李信之下的第二人,他主动询问,“钤辖打算怎么做?”

“我们出战!”李信眼神更加犀利。武将们的反应都还算平静,甚至有几个年轻的都是一幅跃跃欲试的表情。

李信就任广信军知军之后,就陆续选拔了一干能力出众的军中子弟,任命这些人参赞军务。李信的本意是依照韩冈的心意建立一个参谋部,同时也打算通过这些军中子弟了解一下本地和本军的详情。换作其他外来的将领,打压当地势力还来不及,但李信他本人的背景、地位、能力、功绩和声望在,完全不担心被架空。加上李信本人手脚大方,从不克扣军饷来中饱私囊,只是一心练兵,倒是一下就拉拢住了一群希望有所作为的年轻军官。

方才说话的司户参军迟疑着,反对道:“知军,如今城外辽军势大,当是得以守城为上。”

“话的确如此,我也打算暂以守城为重。”李信顿了一下,“不过也不能任辽军放肆,趁其主力未至,给他们当头一棒。人马、军备都已经做好了准备,李参军不需担心。”

另一位文官则质疑道:“可依照方略,遂城兵马一开始只需坚守,待辽军南下后,再北进才是!”

“如果想坚守的话,除非被围困一年半载,否则绝难攻下遂城。但辽军南下,我等守边有责不说,总得试试对手的成色。若是我等连城也不敢出,岂不是让他们小看了?”

广信遂城并不是中原腹地或是江南水乡的那种城防脆弱的城市,而是河北有数的坚城。是河北防线的前沿要地,在对辽方略中,更是反攻入辽境的出发地。在河北西路,是数一数二的坚城,不亚于同在边境线上雄州城、霸州城。

如果要防守的话,李信完全可以安然高卧,什么事都不需要操心,将事情随便交给哪个部将就足够了。

说起来,只要河北的城池,都给修得跟铁桶一般,守将如果有决心坚守的话,没几个月别想攻下来。旧年的贝州王则之乱,官军占尽了人力优势,城中叛军也只是乌合之众,照样是费尽气力用了近一年才攻下贝州——虽说此事跟当时河北军军力废弛,以及平叛的主帅军事才能有关,但也可见河北城防的坚固。而且贝州并非边州,不过是河北内地一中州罢了。

即便边境诸城,因为澶渊之盟的缘故,被禁止增修加筑,即使修座城门,辽人也会遣使来质问。可历任守将都会想方设法找空子,整修城防体系,而朝廷也不会为此而吝啬钱粮。

但这样的闷守,并不合李信的性格,就像他的说法,总得先试一试成色。

“宋贤,在我出战后,你配合闻通判来守城!”

“巡城之事,交给梁安国。若有城中有异动,杀!”

“周和,习玉。你二人率本部随我出城!”

“秦继忠、康肇,你二人带所部马军一并出战!”

“选锋也随我出战。”

李信几句话就将任务分派下去,其实一切都做了预案,只要稍作修改就可以了。

一名名将校领命而出,李信回头对一言未发的通判道,“现在犯我广信的辽军有万余骑,如果我等不出城,他们必会分散开去劫掠乡里。只为百姓,也得出城一趟,拖上半日也是好的,也能让更多的百姓逃生。”

李信都说到这一步,通判也没有办法了:“使君有多少把握?怎么才五个指挥,实兵一千六七。”

李信笑道:“我不会离城太远,不然被围住就回不来了,只是要吸引辽军来围城。人少一点,正好调动,战事不遂,退起来也方便。”

“还望使君小心为是!”

“要小心的是辽人!”李信沉稳的声音中充满自信。

