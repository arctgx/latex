\section{第43章 庙堂垂衣天宇泰(四)}

吃饭的时候,襄州知州黄庸冷不丁的被知州夫人问了一句:“最近听说襄州来了一名名医……”

黄庸手上的筷子顿了一下,夹了片鱼脍,沾着酱料,漫不经意的问道:“没头没脑的,从哪里听来的。”

“城中都传遍了,堂堂太守竟还不知道?”都是老夫老妻了,黄夫人的话中没有太多的恭敬。

“听说是听说了,但还不知真伪。为夫已经派人去查了,若有招摇撞骗,欺瞒世人的,决不轻饶。”

“还扯什么,真当我不知道外院的事?”黄夫人冷笑了一声,“老爷该不会是知道跟小韩学士扯上关系,所以不想插手吧?”

黄庸暗叹一声,结发夫妻之间,想瞒一下的确不容易。放下筷子,正正经经的说道:“韩冈年纪轻轻,就是一阁学士,弄出这么多事,也只为更上一层楼。既然如此,就随他去折腾好了。为夫年纪也大了,只想着安安稳稳地做官,不想去蹚浑水。”

前两天,黄庸就听到了一个流言。当时已经在市井中沸沸扬扬,多少人都在议论着。

痘疮算鬼门关一级的重症,随便找户人家问过去,不是自家有人因痘疮而亡,就亲戚邻里中有人死于此症。现在传言说有了种痘之术,可以防治天花。能保住千万人性命的传言,若不能惹得满城轰动反而是奇了怪了。

而听到这条流言,襄州知州先是摇头,接着便直叱荒谬,但他又不敢完全否定。种痘免疫事关重大,一旦被证实,当能惊动天下,若是能由自己通报进朝廷,好处肯定是少不了的。

黄庸遂立刻遣人在市面上细加查问,最后得知这流言是从襄州城西南的伏龙山中传来的。

在襄州,伏龙山不算多幽深的大去处,比起南漳县向西去的群山差得不知多少里,山下几个村子也不是与世隔绝的桃源乡,甚至还因为据说是诸葛武侯的故居所在,还很有些名气。

如果在伏龙山中有贤人隐居,各色的小道消息很快就能传播开来,远不如终南山深处清静,真正有心避世的大贤,不会选择伏龙山。但若是喜欢热闹,附近也不是没有城镇,离着襄阳并不远。

所以莫名其妙的冒出个名医来,怎么想都觉得不对劲。哪里有名医会在个半大不大的小山包下的村子里治病救人?

黄庸当时心就冷了一点,派人先去伏龙山查问,才知道那名医离开有一阵子了。按道理那位名医应该会来襄州城,但直到如今,却都没个消息。

乱猜也猜不出个眉目,黄庸又派了身边的亲信第二次去伏龙山打探,这一回才打听到会种痘的神医身边跟的伴当都是一口的秦腔,连着神医本人都是关西口音,甚至那李姓的神医私下里跟村民说的一些话,也探问清楚了。

整件事确凿无疑,种痘法竟然当真存在,这一点的确让人兴奋,但这件事又跟传说中的药王弟子脱不了干系。事涉韩冈,黄庸就不会去想着争功了。

虎口夺食的事,若能夺到手,黄庸还真敢去做一做。但撇开韩冈的官职地位不说,他可是传说中的药王弟子,就算种痘的神医不干韩冈的事,自己又抢在头里将种痘法献上去,只要韩冈说一句这是他的功劳,就没人会相信自己。

何况整件事怎么看都是韩冈弄出来的,自己傻乎乎的凑上去,是给人搭架子在自家头顶上耍百戏吗?黄庸权当自己不知道!

可黄夫人却不甘心这么好的机缘从眼前飞掉:“老爷,你也不想想。小韩学士跟唐州的沈知州好得跟亲兄弟一般。沈知州犯了事,本来是要贬到南方,是小韩学士说服了天子,才定了唐州。沈知州家里有事,他长子被赶出家门,还是小韩学士把人接到身边来安顿。”

沈括和韩冈的关系,虽然世间有所流传,但毕竟传得不广,黄庸也只是模模糊糊的知道一点,却没想到浑家竟然全听说了,“韩冈要沈存中帮他整治襄汉漕渠,所以才会帮了沈存中一把。”

“当真是这样?”黄夫人反问,“如果没有沈知州,方城山的轨道难道就修不起来?我怎么听说主持工役的是被小韩学士征辟的李运使次子,主管发运的则是小韩学士门下出身的幕僚,就没见沈存中出多少力气。”

黄庸张了张口,却没话可说。

见黄庸一时回答不了,黄夫人将得意小心的藏起,郑重的规劝道:“老爷你想想,韩冈和沈括这么亲近的关系,为何他却没有将李神医放在唐州,而是放在襄州?这一个,当是沈知州的声名有瑕,另一个就是怕方子在报功之前被人偷了去,所以要放在身边近处才能放心得下。”

黄庸摇着头:“说这么多,又有何用?难道还要为夫求上门不成?”

“求上门又如何,人情往来总是少不了的。何况知州的帮忙,韩冈总不便拒绝。”黄夫人好声好气的劝着:“老爷,这功劳不能让给人。与其等之后天子下诏,还不如趁机早点与韩冈联手,帮他在襄州之中将事情做好了开头,也好附骥尾面见天子。”

黄庸板着脸,不肯松口。

他本来也有心跟韩冈结交一番。韩冈为了打通襄汉漕运,扩建襄州港口时,没少请动黄庸。黄庸在其中尽心尽力,花了不少功夫——当然,这也是因为襄汉漕渠是通了天的缘故,否则黄庸就算不找理由将自己摘出去,也不至于那般殷勤。

韩冈派人在新港周围清理滩涂,焚烧芦苇荡,襄州州衙连句质问都没有。闹得外面笑话,说州衙里面不见知州,只见两个通判。听到这传言之后,黄庸倒是跟韩冈冷淡了下来,对于一名望州知州来说,韩冈的大腿还不够粗,抱上去没好处的同时,还要承受同僚的攻击。

监司官和亲民官由于工作的缘故,不可能和睦相处。钱粮上的纷争使转运司跟地方军州如同乌眼鸡一般互相看不顺眼,这样的情况,以转运司治所最为严重。

在襄州城中,自然也不会例外。两边的官员虽算不上针锋相对,但也是泾渭分明,两家的官吏甚至连日常去的酒馆、青楼,都是不一样的,尽量不碰面。

所以苦了州衙中一干低品的选人,他们的日常考绩不仅要靠上官来评判,就是转运司这里也有评判之权——这就是监司中的‘监’字的由来,而且转官需要的五份荐书中缺了路中监司的那一份,那就别做梦了。而京朝官的身份就不同了,被打压换个地方做官就是了,就是被人污蔑,也有自辩的能力。

“韩冈在京西又留不长,指不定过几天就去了陕西。种谔在鄜延路求着要打西夏,正愁一个帮他们守后路的,韩冈正好跟种家有份交情在。”黄庸虽说已经离开了东京很多年,但故旧在京中人数不少,耳目也灵便,“反正朝堂上没他的立足之地。就算将种痘法献上去,皇帝还能赏他一个宰执来做?他才二十七!”

“甘罗还不是十二岁拜相。韩冈若是成亲得早,儿子都能跟甘罗一个年纪。”

“甘罗十二岁做太宰,那是形势迫人。眼下的朝堂中,排着队眼巴巴的等着被天子抬举进两府的不知凡几。天子手边又不缺人,哪里能让韩冈占上一脚。几十年后,两府之中就还有别人站的位置吗?”

“老爷。现在说的不是小韩学士的前程,而是老爷你的前程和黄家的将来。”黄夫人柔声劝道:“种痘法只要有效,肯定要推广于世。天下人都要为此感恩戴德,只要能在其中沾一点光,那就是天大的福德,海深的善庆,遗泽子孙后世。老爷你就不说了,谨哥、谕哥他们兄弟凭着这份情面,任凭到哪个地方,下面士绅还不得恭恭敬敬?”

黄庸还在沉吟着,自家夫人的话,的确让他动心,但能不能从韩冈手上分到一份,这可是个大问题。平白无故的,韩冈凭什么将这泼天的功劳分出一份?上门去自讨没趣,这又何必?

见丈夫还是犹豫不定,黄夫人无奈的叹了一口气,“二叔不是就在府中吗?难得他来访,眼下这件事,是事关黄家举族兴衰的大事,你不信我这妇道人家,去问问二叔的意见如何?”

黄庸的堂兄弟正好游学至襄州,眼下就在府里住着,过两天就要上京,参加明年的礼部试。

“去问勉仲?”黄庸犹豫了一下,点了点头。

他的这位叔伯兄弟才学尽有,见识眼光都不差,也就是偏偏在科场中缺些运气。十四岁就在福建乡里通过了解试——要知道在福建考中贡生,比贡生中进士的几率都小——可他的这位堂弟二十年来,一次次举试都能拔贡入京,就是与皇榜无缘。否则多上一名进士,在延平乡里,他黄家也能更安稳了。

