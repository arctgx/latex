\section{第11章 五月鸣蜩闻羌曲(一)}

【第二更,求红票,收藏】

天色彻底黑了下去,战场上的搏杀也终于告一段落,董裕一方能逃的都逃了,逃不了的都被杀了。火炬陆续都点了起来,晃动的火光,照得这一片屠场与传说中的地狱又接近了几分。

几十个蕃兵提着刀在尸堆来回走动,时不时的向看上去还算完整的尸体踢上两脚,确定他们是不是真的死了。只要听到一点呻吟之声,他们便会走过去,分辨清楚身份,一旦确认了不是自家人,便抬手补上一刀。

其余的青唐部战士开始在战场上搜集战利品,董裕连续灭了七个青渭的部落,他们都是因亲附大宋而在交易中获取了丰厚利润的部族,在一家的收成就抵得上河州的四五家。董裕带来的蕃兵连续辛苦了三五日,抢来的财物几乎压垮了他们战马的腰。被突袭时,又舍不得丢下这些赃物。最后被瞎药的兵打得毫无还手之力,一点也不奇怪。

而董裕分出来的前军和后军,大概也是因为这个理由,听到中军被袭,也不出手援救,直接就跑掉了——董裕的本部几乎都在中军中,而前军后军却皆是跟风跑来的其他部族,这一点,已经是在方才战斗时确认了的。

董裕军收成如此丰厚,青唐部现在黑吃黑,也是吃了个肚儿溜圆。浓浓的血腥气中,还能听见一阵阵的欢声笑语。黑夜之中,火光昏暗,一时难以细细搜检。最后瞎药下了命令,青唐部战士便把董裕军的尸体一个个扒得精光,砍下头颅,把残躯丢进渭水。

如此一来,打扫战场的效率就令人吃惊的变得飞快。上千人一起行动,战场上的尸体飞速的减少着。俞龙珂和瞎药又各自派了两支百人队去周围,防着敌人卷土重来,毕竟他们击败的也只是董裕的中军。虽然可以确信前军和后军都不会再回来,但无论谁人都不敢冒这个风险。

战后的处置告一段落,俞龙珂和瞎药携手走了回来。同样微笑的两张脸上看不到他们之间有半点芥蒂。瞎药的背后,掌旗官还举着他的将旗,不过如其说他这是为了指挥全军,还不如说他是想炫耀自己的战功——董裕的脑袋还挂在上面。

“瞎药见过韩官人。”

抢前一步,在韩冈身前俯身行礼,瞎药收起狂傲,一转变得谦恭起来。而他一口纯正的秦州腔官话,也比起俞龙珂的吐蕃口音要强出不少。

韩冈上前拱手回礼,露出一副职业性的笑容:“昔日在古渭与将军只是擦身而过,已是惊讶于将军的英武。今日一战,将军大展神威,以千人之力败数倍之敌,阵斩董裕,我和俞族长都是看得惊叹不已啊……”

现在的情况下怎么称呼瞎药都有些问题,韩冈看着瞎药身上打磨得闪亮的鱼鳞甲,还有身后一直举着的将旗,觉得还是称呼他一声将军更合适一些。

而韩冈的这一番赞词,表面听上去是赞叹,但内里却透着深深的抱怨。瞎药听着便露出一丝讽刺的笑意,而让俞龙珂听着,却是觉得韩冈是站在自己的这一边。

俞龙珂正想说些什么,却听着身后突然乱了起来。瞎药和俞龙珂一起转身,却见着一个老蕃僧带着一个小蕃僧向他们这里走了过来。而在两个和尚的旁边,一群青唐部众围在他们周围,却不像是押送,倒像是护送一般。

“是结吴上师!”看清那个老蕃僧的模样,俞龙珂一下惊道,脸色全都变了。

“结吴叱腊!”瞎药的声音也沉了下来,跟着又低低念了一句,听口气,却像在疑惑他怎么没死?

韩冈听说过结吴叱腊这个名字,河湟地区有名的僧人。吐蕃人虔信佛教,僧侣地位也就极高,连董毡、木征也不愿与他们为难。王韶和韩冈想在东京城中找几个高僧到河湟弘扬佛法,也是为了对抗这些蕃僧

不过这些蕃僧念经的时候少,害人的时候多,结吴叱腊就是有名的爱掺和政事,据说还在木征和董裕之间搅风搅雨,在河州闹过一阵子。现在他又是跟着董裕一起杀入青渭,谁也不清楚他在里面扮演了什么样的角色。

青唐部上下都是佛教信徒,在河湟有高僧之名的结吴叱腊,无论俞龙珂和瞎药都不想得罪。但结吴叱腊与董裕一起入侵青渭,就此放过,他们也不甘心。

见着结吴叱腊带着弟子大摇大摆的走到自己面前合十行礼,口诵佛号,俞龙珂、瞎药一时之间两兄弟都是头疼起来,两人对视一眼,叽里咕噜又说了两句蕃话。韩冈看他们的神色,应该都想把这个烫手山芋丢到对方手中。

“不如把他们交给我好了。”韩冈出言帮两人解除烦恼。

瞎药立刻点头:“也好,就让韩官人招待结吴上师。”

俞龙珂犹豫了一下,却也跟着点头:“就拜托韩官人了。”

丢下结吴叱腊,两人立刻走开。这个僧人对他们来说是烫手得很,当然是离着越远越好。至于韩冈要把结吴叱腊煎炸烹煮,那就随韩冈好了,他们是眼不见为净。

俞龙珂和瞎药走远,韩冈便上前几步。漫不经心的看着这个在河湟搅风搅雨的老贼秃两眼,慢慢开口说道:“本官不是吐蕃人,也不信浮屠。自幼承袭圣人之学,所以结吴上师那些佛旨之类的话,就不必说了。”

“阿弥陀佛,礼佛不敬可是要入畜生道的。”

韩冈冷笑一声。结吴叱腊这等蕃僧,怕是连金刚经都不一定能背熟,竟然用平常恐吓蕃人的口吻来跟他说话?在大宋,怕是也只能用钱来买度牒了。他也不理结吴叱腊说什么,自顾自的说着:“想必上师你也明白,俞龙珂和瞎药把你交给我,就有任我处置的意思。还请上师把今次之事的来龙去脉说个清楚,要不然,我不介意让我手下人再多个斩将之功。”

天气闷热,战事又已经结束,王舜臣此时已经卸了护身的皮甲,又将外袍给脱了,精赤着上半身,露出了精壮的胸膛。听到韩冈的话,他便把两只拳头用力一攥,向着结吴叱腊展示了一下自己的肌肉,又歪着嘴狞笑了两声,作为伴奏。

如此低水平的恐吓当然吓不倒见多识广的结吴叱腊。他又是半躬下身子,双手合十宣了一声佛号,“韩官人,贫僧平素里只是吃斋礼佛,哪里知道什么秘事。今次跟董裕来青渭,也是想劝他少做杀孽,防着死后下了地狱。”

“既如此,那就请上师早点轮回去劝董裕吧。”韩冈冷冷看着满口胡言的结吴叱腊一眼,转身下令:“斩了他。”

王舜臣毫不犹豫,呛啷一声,拔刀出鞘。一道弧光寒如钩月,划破夜风,一闪即逝。刀声犹在耳中,结吴叱腊的颈项处,血水就犹如涌泉般喷了出来。

王舜臣听着韩冈的话,直接出手就把人杀了,他的这一刀,把结吴叱腊的脖子砍去了大半,就剩颈骨处的那一小段还连着上下。火光照耀下,蕃僧的脸上带着不敢置信的惊愕,翻到在地。

看着结吴叱腊在地上滚了两圈,抽了两下,不再动弹了,王舜臣这才转回来问韩冈:“三哥,这秃驴在河湟好像有点名气,杀了不太好吧?”

‘这话你应该在杀人前问吧?’韩冈好气亦复好笑。不过王舜臣对自己的命令形成了条件反射,听着就动手,这倒也不是坏事。

“这等僧人虽然是有些用场,但非我族类,其心必异,他在吐蕃人中名声越大,就越是危险。过阵子王机宜肯定要从京中请个大宋的高僧来渡化蕃人,如果结吴叱腊还在,必然会跟他起竞争,那多麻烦?即是如此,还是早点请结吴上师轮回去,也好给我们的大宋高僧腾出位子来。”

结吴叱腊被韩冈不管不顾的直接斩了,周围的吐蕃人起了一阵骚动,但立刻就被俞龙珂和瞎药给镇压了下去。而结吴叱腊的弟子则吓软了腿,扑通一声跪倒在地,向韩冈咚咚咚的如敲木鱼一般磕着响头。

审问一个被吓破了胆的和尚,并不费什么力气。韩冈只提了个头,他竹筒倒豆子的把所有的事全都抖了出来。

韩冈终于明白为什么董裕来得如此义无反顾,原来他是被结吴叱腊撺掇了想做赞普。

吐蕃赞普最重要的是血统,继而是实力,然后是声望。只要三样皆备,自然就能当上赞普。董裕是松赞干布传下来的嫡系后代,又是前任赞普唃厮罗孙子,血统上有证明书,剩下的就是实力和声望了。

董裕的目的是想收复渭源到古渭这一片的蕃部,有了这近百里方圆的一二十万蕃人的支持,他的势力必然大大扩张。

可王韶前次给他的当头一棒,让董裕几年的辛苦化为泡影,今次领众前来报仇雪恨,也是为了取回丢掉的声望。

不过董裕的野心也就到此为止,首级被挂在了瞎药的旗杆上,增加声望的也变成了瞎药。他留下的地盘和势力应该会给木征接收,木征的实力更为膨胀。

青唐部内部分裂倒是不坏,但要对付的河州却也变得更强。韩冈叹了口气。当王韶听到这个消息,恐怕又要头疼了。

