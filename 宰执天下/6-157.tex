\section{第13章 晨奎错落天日近(22)}

侍从官十七,全部到齐;

两制及以上官八,尽数在列;

两府七,实到六人、一人告假;

平章一,已在殿中;

韩冈已经在殿中站定,犹有余暇的一个个数了上来。

今日殿上,拥有投票权的总计三十二人,除了一个来了也不会投票的郭逵,人都到齐了。

事关皇宋未来国运,当然不会有人愿意错失。如果韩冈能将会议的时间拖上一个月,甚至半个月,在南京、西京任职、有资格参加会议的诸多重臣,怕是都会设法回朝。

之前韩冈在奏章中,将这一次共商国是的会议,定名为政治协商会议,虽有一两分玩笑的意味,但真要细论起来,他的记忆中,没有比这个名词更恰当的词汇了。

五年一次的例行会议,三四次后,便能形成习以为常的制度。朝臣享受过的权力,就不会甘愿放弃。日后的国家大政,便必须在此一会议上通过,才能得到推行的权力。

以来有了一把压制天子的利器,掌握在合适的人手中,就能让其不能逾越雷池一步。当然,处在皇帝的位置上,想要压制这样的权力并不算难,若是一位大权在握的皇帝,废除这样的制度,或是利用这个制度,甚至不用太费周折。

但日后想要与天子争权,需要当时朝臣们自己去争取。这不是前人立下制度,就能让后人安享余荫的,天底下没有那么好的事。

韩冈之所以这般绕着弯子做事,更多的主要还是为了自己方便。

……………………

净鞭声响过,向太后带着小皇帝从后殿进入前殿。

今日是决定国家大政的日子,在太后面前,臣子们的表现显得更加恭谨。

曾孝宽起身之后,才从腰背上的一阵酸疼中,发现自己的腰比平日弯得更深了几分。

他很早就明白韩冈的依仗,今天则更加确定。

向太后垂帘的时间,已经足够长了,足以让许多首鼠两端的朝臣,将自己的立场附和在她的看法上。

但韩冈未免显得过于自信了。

即使是有太后的全力支持,可是在韩冈与王安石彻底决裂的现在,让占据朝堂大多数的新党成员中的大部分,改弦易辙,彻底站在王安石的对立面,还是难了一点。

而且王安石并不是那么倔强,需要变通的时候,依然能够变通。

曾孝宽低头看了一眼笏板,古尺二尺六寸长,三寸宽的象牙笏板上,提纲挈领的写了几行小字。

这就是今天他要做的事。

他抬眼看了看韩冈。

对于国是,宰辅都有提议的权力,这正是韩冈的提议。

从韩冈的身上,转而向上,当视线落王安石的脸上时,曾孝宽的心中猛地一跳。

比起方才在宣德门外时,王安石现在的神色更加冷硬。熙宁二年,驳斥司马光和一众元老的谬论时,他就是这副表情;熙宁七年,面对曾布背离、旧党借用天灾兴风作浪时,他也是这副神情。

只知进,不知退。面对敌人的进攻,绝不会退让半步。

这就是当年博得拗相公之名的王安石。

曾孝宽心中不安起来,不过一个晚上,到底出了什么事?

……………………

朝会的前半段进行得很快,转眼就过去了。

这两日,从雄州传回来的军报中,还没有战事爆发的消息。国界对面的辽军,不知是在等援军,还是下台的台阶,总之没有任何动静。

唯一稍稍惹人注意的就是知南平军的黄裳,上表表功,说是击败了罗氏的叛逆,斩首四百多,为此上表献捷。

罗氏是地名还是族姓,殿上知道的不多。少数了解的,也是因为前两年,在熊本的主持下,平定了一次夔州路的叛乱,其中就与罗氏有关。

不过四百多斩获,在西南,或许代表了几个部族三分之一到四分之一的青壮,可放在近年来的战绩中,却根本不值一提。即是将比较的对象,局限在西南,也不算是什么大不了的成就。

除此之外,便无他事值得一提。

待一切琐碎杂事结束,今日真正的议题才正式开始。

左右两班的朝臣近三百人,只有十分之一多一点的臣子能够参与到会议中来,剩下的,都是旁听。

这也是廷推宰辅时的体例,一切都放在明面上,而非暗室之中。

既然是国家大事,当然要光明正大。

太后在帘后俯视着群臣,然后开口,“吾闻一时之法当一时之用。夏殷之法,难用于文武之时;子虽殷裔,从周而不从商;祖宗之法,先帝革而新之;先帝之法,今日又当如何?还望诸卿详议之。”

太后的声音并不高亢,甚至仅仅传到了台陛下。但随侍在侧的王中正随即带她将话传了下去。

王中正代太后传达口谕的声音,在静寂的殿堂中发散出去,清晰的传到了每一位朝臣的耳朵里。

一众朝臣有的吃惊,有的冷笑,有的欣喜欲狂,有的则是若有所思。

太后一边要群臣共议国是该不该变、能不能变,一边却直说要变,这根本就是拉偏架,彻底站在了韩冈的一边。

吕嘉问更是瞪起了眼睛,差点就要骂出口。

‘今日又当如何?’这不是已经明说了吗?革而新之!还问个什么?

‘三代之法,难用于文武之时’,引申开去,就是‘周不法商,夏不法虞,三代异势,而皆可以王’,这是商鞅的话。

‘一时之法当一时之用’,这更是出自韩冈之口。

说是要问政,却先一步定下了方向。吕嘉问早知道太后会偏袒,但也不能这般不要脸皮。

不,不要脸皮的肯定是韩冈,这番话,太后说不出来。韩冈这两天的奏疏中肯定有这么一段,前日自请留对,也必定一字一句的又给太后灌输了一遍。

吕嘉问望向王安石,一开场便被太后定了调子,王安石再不出来,这一场干脆认输好了。却见站在文臣班列首位的老臣,这时已经走了出来。

“陛下!”王安石紧紧攥着笏板,“易有‘穷则变,变则通,通则久’之语。穷则须变,却不可为变而变。熙宗皇帝初登大宝,国库空虚,财不足用,二虏猖獗,兵势不振,当变也。如今中国国势昌盛,西虏覆灭而北虏内乱,朝中却哓哓之声不减,此非是国是有瑕,实乃国是未明之故。”

“平章。今日殿上,诸卿在此所议,便是国是。须变还是不须变,平章当与诸卿共议。”

向太后对王安石立刻就跳出来有心理准备,几句话就推托出去。

“平章所言谬矣。”韩冈出班助言,“天下岂有无暇之物,而不需切磋琢磨?便是先圣,至七十,方能从心所欲不逾矩。先圣古稀之前,于古稀之后,可谓无暇否?。”

王安石瞥了韩冈一眼,冷着脸,都不想说话。

太后口谕中的这一句,的确是出自韩冈奏章中的原话。昨夜韩冈遣人送去的一封信,把王安石给刺激到了。但这股子怨气,没有砸向了吕惠卿,而是落到了韩冈的头上。

吕嘉问见状,忙走出班列,反问韩冈,“夏殷之法,不可用于文武之时。敢问韩参政,那三代之法,可否行之今日?”

“不可一概而论。可用则用之,不可用则弃之。”

“那井田可用否?”

“韩冈曾闻平章有言:‘古者井天下之田,而党庠、遂序、国学之法立乎其中’。平章昔年所喜,惜乎未行之于今日。”韩冈看了看吕嘉问,他知道吕嘉问到底想说什么,不给其机会,直接接下去道,“而气学求实。验一事是否可行,不本言辞,只求实证。故而先师文诚于乡里试行井田,以验其是否可行之于今世,与他人叶公好龙大不相同。”

当今儒者都在说井田,盱江李觏要推行井田,横渠张载要推行井田,王安石的新学承袭了李觏许多观点,同样赞赏井田,洛阳二程一样喜爱井田,但那么多儒者中,只有张载真正去做了。叶公好龙四个字,王安石的确当得上。

吕嘉问微微冷笑,又问道,“敢问结果如何?!日后参政当政,是否要推行天下?”

“横渠井田,施行有年。田地出产高于寻常农户,井田诸户更是能够安居乐业。但这一切的前提是先师买地与人,若无这份田地,井田便是纸上谈兵。不过……”韩冈话锋一转,“上古之时,地多人少。今日中国,则地狭人多。中国人口日繁,田地开垦日多,但田地增长之速,却远远追不上人口。若行井田,须从地主手中夺田,实乃虎口夺食,难如登天。此事既难行,井田如何可行?可若是国有闲地,使民常有土地可种,井田自可复。”

“北虏在侧,岂容安寝?”吕嘉问出班,“两虏在,则中国不可安。两虏去,则皇宋百姓不再受征伐之苦,方可安享太平。如今西虏已灭,北虏国中不靖,正需要一鼓作气,将之倾覆。皆是,天下安定,参政也可有闲暇推行井田之政。”

加强军备,以期一战决定两国命运,这是新党计划中的一劳永逸。

