\section{第21章 欲寻佳木归圣众(13)}

沈括终于如愿以偿,与出身新党的另一位老人——邓润甫,通过了廷推,被太后点选为新任的两府成员。

沈括签书枢密院事,而邓润甫则是参知政事。

政事堂不再是韩家天下,而枢密院也不再由新党独霸,两边相互牵制的局面越发得明显起来。

沈括激动不已。

他早年以博学闻名,才干亦是超乎同侪,不知有多少人都赞许其是未来的宰辅之备,一张清凉伞不为难事。要不然,士大夫家嫡女,为何会嫁给一名鳏夫?

可是自背王投吴的那一桩事之后,他就彻底成了世人眼中的反复小人。不仅开罪于权相,亦遭天子厌弃,青云之路至此断绝。

幸好有人看中了他的才干,这样才一点点的从深渊中爬了回来,直至两府门外。

不论这个签书枢密院事来得有多侥幸,也不论这个任命有多么不得人望,在入选诸人中,票数倒数第一,清凉伞是不会有任何区别的。

面向御座,伏地而拜。沈括颤声道:“御史之论,臣不敢辩。日后唯有鞠躬尽瘁,以报陛下垂顾之恩。”

向太后不喜沈括,可沈括的任命毕竟是韩冈力推,她也知道沈括是个能臣。让沈括主持轨道修造,至少能比其他朝臣更为让人放心。而邓润甫是新党老臣,资历老,人望也说的过去,至少比李定等人看得顺眼。

让沈括和邓润甫起身,向太后看了看在前面坐得端端正正的赵煦。自朝会开始后,他的姿势几乎么有变过。

《九域游记》中有立如松、坐如钟、行如风之说,称男子行动当以这九字为圭臬。

立如松、行如风两句且作别论,但坐如钟这三个字,赵煦肯定是完全符合的。

廷推让朝会延长了这么多时间,也苦了赵煦,能安安稳稳的坐在御座上,纹丝不动。

看着赵煦的背影,多了几分赞许,也带了几分怜惜。

皇帝小一点的时候,还会忍不住内急。御座后要藏一部鼓吹,锣儿、钹儿都得准备好。到了忍不住的时候,皇帝便会起身,到后面方便,锣钹敲上一阵,用来掩盖声音。

现在已经不需要准备乐器了,不再是小孩子,能够克制自己。再过几年,更是该大婚,娶妻生子。

就是这身子骨,向太后望着前方削瘦的双肩,怎么还是这般瘦弱?

补品从来都没有断过,甚至为了防病,每逢换季,如今被视为神仙药的人初乳都连着半月不断。以天子之尊,想要什么补药,都会有人贡献上来。可牙都换完了,个头、体重还是远远不及同龄孩子的平均水平。

这两年,厚生司让下面的医院给宗室和官宦人家的子女都设了一份个人病历,不仅仅每次生病后,症状、诊断、治疗,以及药物都会记录下来,以作参考,而且每年都要测量体重、身高,以确定成长情况。这种无微不至的关心,让厚生司成为在京百司中最有口碑的衙门,但也让向太后知道,小皇帝的生长发育在同龄人中,是个什么样的水准。

比起从两百多同龄少年身上统计出来的平均数据,赵煦的个头差了两寸多,体重也轻了近十斤。也幸好小皇帝一直按照韩冈的要求,每日在后苑走上三五里路,再打上一路拳脚,使得皇帝没怎么生病,伤风感冒都少有。

不过无论如何,赵煦先天便有不足之症,若不是朝廷中出了一位药王弟子,又有儿科圣手服侍左右,说不定就跟他的六位亲兄弟一样保不住。可之后不论怎么进补、锻炼,都无法达到正常应有的水平。

难道真的是心思太重的缘故?

向太后忧心忡忡,多年来一直萦绕心头的隐忧,这一次,又浮上了水面。

……………………

“今天回去,沈存中当能保住他的胡子了。”

“真有人这么说?”

韩冈有些惊讶,一半是苏颂也说八卦,另一半,则是这话是怎么给当朝首相给听到的。从首相嘴里传到自己这边,倒不是什么事了。

苏颂如今是首相,昭文馆大学士兼监修国史,韩冈则是集贤院大学士,若再添一名宰相的话,韩冈倒是能去监修国史了。可惜现阶段,新宰相的人选暂时还不会出台。

苏颂点点头,“的确有人这么说。”

“骂人不揭短,打人不打脸。”韩冈皱眉,“本来我以为会有人说‘沈存中这个参知政事当得好生没趣,又不能长居政事堂,也不能诏书上列名,不过是给个使唤地方的名分’,没想到,这话比我想的还要刻毒几分。”

“士人说酸话,能熔金蚀骨,与硫酸一般,哪有不刻毒的?”苏颂端着茶,也不嫌热,小口抿着,道:“沈括的参知政事就算只是给他一个使唤地方的名分,多少人连这个名分都没能有。岂能不让人含酸挟忿?”

看着苏颂的茶盏里,腾腾而起的热气,韩冈感觉自己都要帮他出汗了。

但苏颂也是知医理的人。觉得天气越是热,越是不能贪凉,若是寒气痹体,使得体中湿热不散发出去,肯定容易生病。所以今年入夏之后,韩冈都没看见苏颂喝政事堂中最受欢迎的冰镇紫苏香薷饮。韩冈也知道老年人不能与年轻人比身体,这样的保养,也不过是不求生病罢了。

苏颂这样的想法,韩冈自不会平添波折,而是继续笑着对苏颂道:“都说沈括侥幸,岂不知这一回他是必定能晋身两府。有沈括主持轨道修造,好处将会源源不断的流入国库,太后怎么会将这个散财童子给丢下?”

韩冈很早就知道,这一次不可能有任何意外。仅仅是一条京泗铁路,已经给朝廷带来了天大的好处。原本从汴水北上的民船,大量的船只用各种方式避过税卡,朝廷征收不到多少商税,而换成铁路就大不一样了。而且汴水缓而铁路疾,等到整条铁路运转磨合得差不多了之后,

仅仅从朝廷财计一项,沈括的作用就是不可替代。工程进度耽搁一天,就会少收入几千贯,有谁会嫌钱多烧手?去找个不懂行的人来代替沈括?

“说得是啊,”苏颂叹了一声,不想再说沈括,“沈括倒罢了,邓润甫来做参政可不一定是好事——邓温伯差不多该来了。”

“当然,沈括不留在京师,西府那边要轻松些,邓润甫可就难说了。”韩冈渐渐低下声来,“枢密院还可以多塞几个人进去,而政事堂也会继续收纳新人,沈括、邓润甫两人绝不是最后一个。”

“等到了新人来,老夫差不都该让贤了。”

苏颂悠闲的喝着茶水,仿佛这不是在说自己离开政事堂的事。

韩冈立刻惊叫道:“子容兄,你春秋正盛,何必弄什么急流勇退?!”

苏颂是他韩冈主掌政事堂最优秀的队友,怎么能说走就走?韩冈舍不得这么好的搭档。

苏颂轻轻笑了起来,“莫羞老圃秋容淡,要看黄花晚节香。”

韩冈对诗词没有什么鉴赏力,但这两句话中之意很浅显,一听就明白。能让苏颂如此感慨,这两句还做不到,多半是作者的身份,让苏颂腾起了维护晚节的心思。

“这是谁人手笔?”韩冈问道。

“是韩稚圭。”

“啊……难怪。”韩冈低声道。

苏颂笑了一下,“政事堂中有了参知政事,可谓事有所归。日后若有文学事,玉昆可问东厅,让他来处理。

邓润甫是旴江先生李觏门下,最为得意的弟子。因为王安石的新法很多地方都借鉴了李觏的理念,邓润甫一直都是王安石的坚定支持者。

邓润甫虽不是以诗文著名当世,但文章水准也是朝中前列。诗词或许稍逊,可官样文章几乎无人能比。翰林院两出两入,每一次就任翰林学士的时候,绝对是玉堂中手笔最快的一位。

“有了邓温伯,文学上的事就有人管了,子容兄你我,也就能多喘两口气了。”韩冈顿了一下,“不过政事堂中,还需要一个熟知朝堂掌故的参知政事。”

苏颂会心微笑,这是朝中流传已久的故事。

昔年韩琦为首相,次相是曾公亮——也就是曾孝宽的父亲,赵概和欧阳修参知政事。四人共同主持国政。

凡事事涉政令,韩琦便让人去找曾公亮:“问集贤”;有关典故,“问东厅”,去找赵概;若是文学上的事,自是由天下文宗欧阳修来处置,韩琦只会拿着笔向西一指,“问西厅”。至大事,韩琦方自决。

只从这一点上来看,韩琦也是一名称职的宰相了,再加上他对政事的处理,支撑着大宋朝堂渡过了仁宗传英宗,英宗传熙宗,两次帝位传承的艰难阶段,故而被许为是开国以来数得着的名相。纵使韩冈对韩琦的才干一向颇有微词,也不会否认这一点。

至少韩冈承认,韩琦即便不可出将,却绝对能入相。主持政事,钧衡朝野,单纯从这个时代对宰相的要求上来看,韩冈绝对没有韩琦做得好——当然,韩冈对自己的要求,也从来不会苟合这个时代的流俗。
