\section{第31章 风火披拂覆坟典(五)}

被方才还言谈甚欢的人指着鼻子说成是骗子,韩钲还是第一次。

而且还被说成是假冒宰相家的衙内。

韩钲一阵楞,他从头到尾都没说过自己的身份,一个宫里的玉佩,也不可能是宰相家才有的器物,就算是姓韩,朝堂里面还有好几家呢。

‘我没说过啊。’

他偏偏头,想不出自己什么时候露出了破绽。

几名拿着武器的军汉就在面前,可他根本就没害怕,连生气都没有。

除了疑惑之外,就只觉得有趣。

货真价实宰相家的公子,被指认成骗子,这可是京师中遇不到的趣事。

可韩冈的几个仆人却不会看着一个审刑院的小官在他面前指手画脚、胡说八道。

两名护卫阴着脸站到了韩钲身前,立刻便让几名军汉一阵紧张。

领头的军汉甚至把刀都抽了出来:“做什么?!想闹事?”

“你们退下。”老都管排开两名护卫走上前来,顶着刀尖,对王珏道:“王官人,你这是何意?”

“还能是什么?你们马脚露出来了。”王珏悠然道,“你们这些贼子胆子不小,可惜运气不好。可惜本官是在审刑院办差,二十年都没离开过法司。你们这些”

“领教了,原来审刑院是这般断案的。”老都管拱拱手,“也难怪官人二十年不能出头,”

“好个尖嘴利舌!”王珏脸上一阵青气泛起,“等到了衙门,杀威棒打过就好了!下一站是哪里?!”他冲着几名军汉怒道:“把他们押解下车送官”

听到王珏要人将自己押解下车,韩钲立刻就不觉得有趣了,“我没空跟你们闹了,我这回去江宁片刻都耽搁不得!”

“二郎!”老都管一声喝“出来时,夫人是怎么说的?小心夫人知道了会不高兴。这件事,让老头子来处理。”

韩钲扭过头,怏怏不快的闭上了嘴。

“叫车掌来!”

老都管呵斥着几个军汉,可是却没人动身。不管怎么说,王珏的投诉给他们先留下了深刻的印象,而这边一老一少,看起来都不如已经是朝官的王珏更让人感到放心。

老都管见状,也不气也不恼,拿出了车票,耐下性子对几位军汉道:“人你们不认识,告身给你们也认不出来,票是假是真,你们能认出来吧?”他抬手指着王珏,“别听着风就是雨,告对了没话说,要是他弄错了,他是朝官,脱身容易,你们呢,不死也得脱层皮。”

王珏嘿嘿冷笑,看着老苍头的表演。

“从头到尾,我家二郎什么说过他是宰相家的衙内。”老都管摊开手,直指车厢中的每一位看客。方才说话,他们可都是落进了耳中。

“哦?”王珏拖长声调的一声感叹,“你那二郎不是相公家的衙内?”

“二郎又什么时候说过不是了?”老都管用袖子掸了掸床铺,弯下腰,“二郎,坐。”

精乖的老家伙。

王珏仔仔细细的打量了一阵王德,看作派倒像孙子似的服侍那公子哥儿,装得不可谓不像。但说不准,他才是亲爷爷,没看小骗子对那老苍头么恭敬听话?

哪家十四五岁的小子不招人嫌,自家的儿子也差不多这个岁数,自己面前老实些,到了下人面前——其实家里就两个下人,还是从家乡里带来的族亲——立刻变得肆无忌惮。宰相家的儿子,可能会老实听话,但不可能这么老实的听仆人话!

“前些日子,本官审了一个案子。”王珏轻轻摇起折扇,笑着说道:“人犯抵京后便自称来自华山,陈抟老祖嫡传,身有长春方,能驻颜不老。活了一百二十多年,看起来就像三十多岁。有人登门拜访,先出来了一个胡须花白的老头子迎客,迎进门后,那人犯出来,先大骂那老头儿一顿,回过头来,就对客人说,这逆子一贯懒怠,修炼不勤,才八十多岁就老成如此模样。”

“客人一瞧,老头八十多,其实也就五六十,偷懒都这么有效,认真练了又会是什么样?看看那一百二十多的老神仙就知道了。一时间引来了多少人要学那长春方,甚至引动了好几位宗亲。

“只可惜他的事见了报,偏偏就惹动一群从不信鬼神的气学门生,上门刨根问底,却发现他连陈抟老祖的《太极图》都不会知道,就这么给拆穿了。到了公堂上一审,却发现那老头子才是父亲,那神仙竟是儿子。”

王珏习惯了在公堂上黑着脸,口才并不算好,但他说的这件事,京城中知道的不少。而且类似的骗子,在京师里面从来都没断过。如果把用长明灯骗香油的贼秃们算进来,那就是数都数不清了。

“这位衙内。”王珏如同老猫逗鼠的看着韩钲,“你对家仆是不是太恭谨了一点?”

这下是抓住真把柄了,王珏笑眯眯的盯着韩钲。

韩钲浑没在意,“家严有言,待人须有礼。何况王公公还是家慈的奶公。难道王刑详是以法治家?这可真是稀罕!”

“二郎!”老都管先回头瞪了韩钲一眼,这么不小心,如果没人在旁边看着,家里的老底都能给漏个精光。转头又对王珏道:“去了泗州的铁路衙门自然水落石出,你又急什么?难道还怕我们逃下车跑了不成?沈枢密或许不一定在泗州,但方判官肯定在衙门里。想必你们也知道,方判官是哪一家出身!”

方兴!

铁路衙门,有兵权,有财权,有事权,还有法权,主事的还是西府中人,除了两府和廷议,根本都不用理会其他人。

沈括因为要负责督办铁路,得四处巡游,所以不能留在泗州。所以主持铁路衙门一应公事的,便是做判官的方兴。

也许车中做护卫的士兵不知道方兴这个人,但领着他们的小校却不可能不清楚。顶头上司的顶头上司的顶头上司,也不知得绕过几层,才能与之对上一句话。

所以车掌很快就出现在了这节车厢中。

看到车掌过来,一群人七嘴八舌的,有的认为韩钲是骗子,也有人认为不是,而老都管却不管不问,“前面三号、四号车厢,究竟是哪家的!?”

车掌被老都管给镇住了,老头子威风得很,到了他面前连一声客气话都没有。

车掌低声道:“是去太平州做通判。”

“原来呢?”

车掌摇摇头,这种消息他不可能知道。也没人会拿出来随便乱说。

老都管皱起了眉,花白的双眉眉头几乎拧在了一起。

“怎么了?”

韩钲和王珏异口同声,但韩钲带着关切,而王珏则尽是冷嘲。

“没事,他们还没资格拜见相……老爷。”

“王公公!”

看到老都管如此说,韩钲忍不住叫了起来。

老都管却没理会他,“还有,老头子记得没错的话,律条中有诬告反坐一说。诬告人什么罪,自己就要受什么罪!方才听官人说,是在审刑院中办差,想必刑统和遍敇是能倒着背的。不知假冒官亲……不,二郎是以自家的告身拿的票——太常寺太祝——说二郎是骗子,就是在说二郎是假冒命官。敢问这是什么样的罪名,要怎么判?”

……………………

王安石重病的消息已经在京城中传开。

很多官员都开始思考失去了王安石之后,朝局将会发生什么样的变化。

但在韩冈家中,却是心系至亲,在次子韩钲连夜出发之后次日,王旖也带着全家儿女一起南下,这一回坐得是专列。

府中一下子变得空空荡荡起来,晚上只有狗叫才增添了些许人气。

但韩冈没能得到一个清闲,公事之外,还有自家儿子在铁路上闹出的案子。

幸好泗州有人,沈括给韩冈逼成了劳碌命,四下奔走。但方兴在泗州,有他证明韩钲的身份,这场误会立刻就给解开了。

“相公。”宗泽见韩冈手上没事,便问,“泗州那边问,王珏该如何处置?”

“放了吧。不过是误会而已,我家那小子从小就没受过挫,吃点苦头也好。着方兴好生抚慰,不可折辱。”

“下官知道了。”宗泽点头,又皱起眉,“这到底是怎么回事?怎么会弄错?”

“警惕心太高了一点。”韩冈笑道,“我家那小子一上马,就爱飞奔。就怕他晚上起码出了意外,才让他去坐车船。不比马车慢。没想到却被人误会了。”

宗泽连连点头,又问,“不知泗州那边,怎么处置那王珏。如果有所折辱,到时候可不是一两个官职就能打发得了。”

“为什么?”韩冈摇了摇头,似乎完全不明白。“国家名器不可以私故与人,日后以财货偿还便是。”韩冈靠上椅背,“汝霖,我这么说你满意了?”

宗泽低头道:“是宗泽想太多了。”

“汝霖你说得也不错。”韩冈笑道,“不过日后若要劝谏于人,要么说直话,要么就再委婉一些,半调子可是最差的做法。”
