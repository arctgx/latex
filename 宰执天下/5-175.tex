\section{第20章 土中骨石千载迷(四)}

从政事堂回到编修局,已经是快黄昏了。

离开政事堂的时候,王珪和蔡确亲自将韩冈送到了庭院中。蔡确还好说,王珪身为宰相,礼绝百僚,他在政事堂中降阶相送,可是极之难见的殊礼。

迎客送客,是在正门前,还是在院中、厅中、阶上、阶下,这都是有规矩的,尤其是在官场上,些微的改变都免不了会惹起他人的猜疑。

在富弼之前,普通的官员受到宰相接见后离开,宰相最多也只要站起来就够了。当富弼做了宰相,送客无论尊卑,皆起身相送到公厅门前。自此之后,成了定制延续了下来。不过,今日王珪和蔡确送韩冈,规格则更高了一层,从礼仪上,可以说将韩冈对等看待了。

这基本上可以说是王珪在表态,对韩冈这一次的举动表示支持。

与王安石同榜高中,则一直以来,对王安石都有几分竞争心理存在的王珪,不愿一直处在王安石的阴影下。能打压一下新学,对王珪来说,是乐见其成的一桩妙事,尤其出手的还是王安石的自家女婿。多走几步路,也算不了什么。

王珪的小心思,韩冈看得很明白,也是早有了解,要不然也不会跑到政事堂来,与在崇政殿的苏颂配合着拉开戏幕。

只是韩冈对王珪的支持能坚持多久不是很指望。王珪一向是以天子的心意为依归,就算现在看起来是支持自己,只要赵顼头一摇,到时候还是会改弦更张。而蔡确也是差不多的类型。这样的人根本没办法让人相信。

说起来,沈括也是类似的人,要不然也不会被赵顼和自家的岳父给打入另册了。相对而言,还是苏颂的人品更值得让人信任,将苏颂请来做副手,也是由这方面的心思在。

韩冈今天来政事堂披露甲骨文和殷墟之事,也只是想与崇政殿的苏颂同时出手,等天子正式表态,消息早就传开了。

踏进编修局的小院,浓浓的药草味就扑面而来,充斥鼻端。这段时间下来,韩冈也闻得习惯了。

几名小吏上来将韩冈的马给牵走,韩冈将缰绳和马鞭递过去,问道:“苏侍读可回来了?”

“回端明的话,苏侍读半个时辰前回来的。”

“只有苏侍读?”

“就苏侍读?”

小吏疑惑的摇摇头,不知道韩冈为何这么问问:“没有别人了。”

韩冈点了点头就往里面走。苏颂回来的比自己的还早,也没人来取走存放在这里的甲骨,看起来情况不是很顺利的样子。也算是在意料之中,以赵顼他做了十几年皇帝后逐渐变得威福自用的性格,也不可能一下就反过来赞同韩冈。

韩冈走进厅中,听到他回来的动静,苏颂从内厅里走了出来,一句韩冈便问道:“玉昆,情况如何?”

韩冈摇摇头:“跟子容兄你那边比,当时要好一点,不过也只是一点而已。”

苏颂也是摇头叹气,与韩冈一同进了内厅,

内厅中,刻有文字的甲片骨片装满了两个木箱子,本来是免得天子降旨要这些甲骨时手忙脚乱,事先给收拾好,但最终还是无用功。

苏颂拍了拍箱子,又是叹了一口气。

韩冈将箱盖打开,珍贵的甲骨用贫民在冬天垫鞋子的褥草小心的一片片包起来,充当缓冲。为了以防万一,事前做的准备不少,生怕在路上给颠坏了。从相州运回来的时候,也是这么做的。作为药材,碎了裂了不会影响药效,但作为珍贵资料的记录文件,碎了可就再也弥补不回来了。

“这是什么?”苏颂突然探手从箱子中翻出一张纸片,这是他这两天所没有注意到的。打开来一看,就是很简单的点和线组成的让人莫名其妙的图案。

“我说怎么找不到了,原来是掉到了了这里面了。”韩冈瞥了一眼后,与苏颂道:“这是安阳县甲骨出土最多的位置,算是舆图。”

“玉昆果然是心细如发。”苏颂看了之后,便叠起来顺手放到了桌上,“有了这份舆图,日后发掘也方便了许多。”

“也不能只局限于舆图。其实若真的开始发掘,就是出土时,周围的土层地样,都得让人给画出来。”

“嗯?这是为何?”苏颂不解的问道。

“占卜的位置,占卜的仪式,很有可能从埋藏的地点中找到痕迹。要是光是注意这些甲骨,忽略了那些残迹,虽不能算是买椟还珠,可也是把宝贝给丢了。”

韩冈说得只是些后世粗浅的常识,但苏颂听得却是连连点头,一幅有会于心的模样。毕竟这还是在发掘工作完全是由贪心的盗墓贼或是村民们来完成的时代,士大夫只会坐在家里研究,最多也只是拿着放大镜来查看铭文和式样。

不去实地观察,怎么可能会总结出考古时必须遵守的规矩?韩冈最不喜欢的就是这个时代坐在家里自以为能知天下事的无知措大。这样的人,可是标准的韩非子的五蠹。

苏颂点着头,觉得韩冈说得很有道理。

什么叫礼?可不只是祭祀仪式和待人处世的规则,官制、乐制,乃至建筑规格,全都包括在内,都属于礼的范畴。

城南青城的祭天圜丘,外形、大小、高度和台阶的数目,皆有定制,一点也差错不得。几千年后,后人看到圜丘,当也能从中印证到此时的郊天之制。

“不知道殷商时的仪制究竟是什么样的,要是能由此知晓一二,也不放这一番的辛苦。”

“商礼和周礼肯定是有区别,但必然也会有共同之处。要不然圣人也不会说,‘周监于二代,郁郁乎文哉!吾从周。’”

“说得也是。”苏颂点着头。

出自于《论语》的这三句话,后两句确立了后世王朝遵循周礼的规则,一切都仿效周礼中的定制来,纵有差别,也是万变不离其宗。但前一句,可就是圣人承认周制是参考了夏、商二朝的制度。

这里的监通鉴,乃借鉴之意。所以文字、经籍和礼制的源流必须要追本溯源的这个主张,能从孔子那里得到足够的依仗。

任何人想要反对韩冈的论点,就必须先证明圣人的话有错,或是用另外一种与韩冈相逆却又还能让人信服的解释来取代现有的诠释。而且身为殷商孑遗的孔夫子,有关夏商两代的言论,还有更多。韩冈倒想看看,谁有这个本事,能有理直气壮的理由来压倒圣人之言。

“不过这件事得尽快才行。”韩冈很是忧心,“消息现在传出来了,就怕被人赶去相州一通乱挖,想想洛阳和长安,那里的古人坟茔都变成什么样了!”

“天子可能还要多想一想。”苏颂不便说赵顼的坏话,也只能留在肚子里腹诽。

“此事虽是在意料之中,不过若是让殷墟受了我的连累,那韩冈可就是名教罪人了。”

韩冈一声轻叹。看着珍贵的文物被人盗掘,卖给了那些像松鼠一般只喜欢收藏的士人或是富商,完全不去研究其中的价值,最后在几百年上千年的历史进程中散佚无踪,那可是让人惋惜之极。

《尚书》也好,《竹书纪年》也好,全都找不到原本了,如今能看到的,可全都是各方拼凑出来的结果,使得许多地方让人不免有杜撰、伪作的感觉

但韩冈也知道,天子迟早会坐不住的,根本不需要着急。

今天他和苏颂在崇政殿和政事堂中所说的一切,两天之内,就能传遍京城。十日之内,相州城中能涌进一大帮子古董商。

研究碑文和篆刻的金石学可是当下最热门的学问之一,别的不说,跟韩冈过节极深的吕大临在金石上的造诣,就是第一流的。身在长安城边,只要有心,能得到的古董数不胜数,将兴趣培养成能力,吕大临已经可以算是一名一流的金石学专家。。

金石这么热门,靠着那群有钱有闲的士大夫赚钱的商人也是不在少数。有些身家的士人买些金石之物,或出钱拓印,这些开销,便是古董商们的利润所在。

对于士大夫们来说,这是打开儒学源流的一把钥匙,对于那些贪心的古董商来说,那是一片金矿,而对于安阳的百姓,也是谋生之外多了一项赚钱的买卖。

当殷商的礼器逐一出土,甚至司母戊大方鼎之类国器都从地里给抛出来,那时候,无论是天子,还是宰辅朝臣,都不可能再继续拖下去了。

西周的祭器都少得可怜,能确定是商朝的器物,皇宫中都没有多少,那等能摆在太庙或是祭天场所的上古礼器,无论如何都不可能让其流失在外。

要知道,今不如古,是此时儒生们的通病。天子去圜丘祭天时所乘的玉辂,还是唐高宗时制造的,有名的古物。赵顼曾经想换台新的玉辂,但刚刚做好后,就在正旦大朝会的展示上,自行垮塌下来。使得赵顼只能继续利用旧时之物。

古物的诱惑力是无穷的,过上三五个月,自然能见分晓。而韩冈接下来要做的,就是等世人对此的反应了。

