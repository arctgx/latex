\section{第30章 肘腋萧墙暮色凉(六)}

【下一章中午发,老是夜里发两章,损失不少点击。这里预先说声抱歉。】

大宋首相的年齿,据韩冈所知,应该有五十了。不过从外表上看不出来,须发都是黑油油的,脸上皱纹也不多,保养得很好,打理得更好。作为世家子弟,韩绛的言行举止也是出类拔萃。就算好像被韩冈的一句话给堵在心口,但那种被糯米糕噎着的表情,也是一闪即逝,眨眼工夫,就恢复了平静。

韩绛视线越过韩冈,望着厅外,似是追忆身处远方的友人,“欧九向来读书最勤,手不释卷。马上、枕上、厕上,他的这三上之说,还是当年他先对我说的。”

他略低下头,温和的望着下首的韩冈,摆出一副长辈的姿态,“玉昆你能学着欧九的样,得空便刻苦攻读,我这幕中的年轻人里,倒少有能比得上你。也难怪你能有如此大的名气,也难怪天子垂青于你。”

韩冈略略放心下来,看起来虽然在王安石家中的私语没有暴露,但韩绛应该是已经知道了他今次在京中闹出来的这一摊事来。他谦虚道:“天子重恩,韩冈粉身难报。相公的夸赞,韩冈也是愧不敢当。”

“没什么不敢当的。玉昆你是我用两份奏疏调来的,你说‘愧不敢当’,岂不是说我没有识人之明?”韩绛哈哈笑了两声,“今之横山,牵动天下时局,玉昆必有以教我。”

韩冈的眼底闪烁着疑惑的光芒,他可不会被人一捧,骨头就轻上三分。政客说的话,从来都是不能当真的。前面把人晾在外面坐冷板凳,说是要磨磨性子,现在却又好脾气的问起话来,韩冈心中立刻有了几分戒备。低下头去:“军国之事,非韩冈所宜言。”

只要是底下官员被询问,基本上都会这么先谦虚一下,韩绛只当韩冈也是如此,笑道:“玉昆你即为我幕中属吏,有何不可说。但说无妨!”

韩冈却是坚持着,“韩冈不才,仅仅是稍通医理,世人之赞,往往夸大其辞,盛名之下,其实难副。相公帐下皆是深谋远虑之辈,赵公才之于谋略,种子正之于战阵,无不是一时之选。将帅谋士,车载斗量,岂是浅薄如韩冈可比。”

从心底来说,韩冈对韩绛是有戒心的,平白无故磨着自己的性子,心里到底转着什么念头韩冈也猜不透,总得防着他引蛇出洞的把戏。

‘这是在说不在其位,不谋其事吧?’韩绛却是心下冷笑。他在官场中浸淫已久,套话、隐话都是熟极而流。韩冈的一番推搪之词落到他耳中,便觉得面前的这位年轻人,果然还是不满延州管勾伤病一职,在变着花样要官。

韩绛慢慢的端起茶喝了一口,一举一放,世家中人的气度让人看了都有自惭形秽之心。他温文尔雅的笑了笑:“玉昆之才,天子心知,我亦心知。区区管勾伤病事,的确是屈才了,确当加之重任……就不知玉昆心有何属?”

韩绛的笑容中仿佛隐藏杀机。韩冈心中一凛,这是无妄之灾、欲加之罪了,他何尝有着要官的心思,要是真的被钓上了钩,日后想脱罪都难。转瞬便打定主意,不管韩绛有着什么盘算,他都要一推了之。

他欠了欠身:“相公的看重,韩冈实不敢当。凡事有先后,韩冈又是才具浅薄,管勾伤病一职尚未上任,亟待处置的各项事务千头万绪。若是再妄求重任,恐难符相公所望,当会拖累相公识人之明。”

韩绛阴沉起来,仿佛下一刻就要翻脸的样子,厅中的空气也紧绷着。换作是别人,听到宰相下问,哪个不是谦虚两句,就眉飞色舞的指点江山起来。就这个韩玉昆倒好,什么都推的一干二净,油盐不进的样子,韩绛看得心头火起。

‘这厮好大脾气,当真是不肯低头了!’

他对韩冈感觉并不好,现在则更是有看法了。本是种谔、赵禼大力推荐,韩绛才上书朝中调韩冈来延州。后来因为各种原因,又上了第二封奏疏。自家只是想稍稍磨着他的性子,也好任用,却没想到他就在外面玩出那等花样。现在自己不耻下问,好话说尽,他非但不感恩,竟然一点脸面都不给。

只是韩绛暂时拿韩冈没有办法,这厮是他上书请天子调来的。若是当下就处罚于他,等于是在说自己识人不明。想到这里,韩绛越发的心头火起,韩冈方才的话中,好像也提到了‘识人不明’四个字。

‘这是在提醒我吗?!’

韩绛咬牙,真想随便找个罪名把韩冈处置了。可是他一向很顾惜自己的名声,不想因为一个选人而坏了自家知人善任的名头。‘算你命好,换作是六哥【韩缜】,棍棒早不管不顾的下去了!……’心中发狠,‘过阵子看你还能再硬着脖子!’

不再强逼着询问什么,士人真要犟起来的,天子的脸面都可以不给,韩绛也不想再丢脸了。声音冷了下来:“也罢,既然韩冈你不愿,我也不强迫你。种谔几次三番求我调你来延州。既然你已经到了,那就直接去绥德,不要再耽搁。”韩绛语气随即又加重了几分,“此战攸关国是,若你在其中有何疏失怠慢,我必不饶你。”

韩冈立刻起身,在厅堂正中,向韩绛躬身领命:“韩冈谨遵相公之命,敢不尽心尽力。”

再没什么话好说,话不投机,韩绛又是贵人事忙,随即便点汤送客,韩冈也顺势告辞出来。就算背着身子,他也能感受到韩绛带着怒意的目光,正冰冷的盯着自己的背后。

这一次见面,韩冈很直接的表明自己的立场和态度。他的工作仅仅局限于完成他的差遣所带给他的任务。除了军中伤病方面的事务,其他公事,他绝不会插手半分。从中也可以看得出,他完全没有亲附韩绛的想法。这样决绝的表态,加上在王安石府上的发言,日后罗兀沦陷,横山局势糜烂,也半点罪名牵连不到他头上——以王安石的性格,在天子面前不会隐瞒韩冈当初的立场。

当然,有得必有失,韩冈今天毫不给面子的态度,因此也彻底得罪了韩绛。不过话说回来,如果不是韩绛先用了手段,韩冈也不会回绝得这么直接——因为担心着韩绛会给自己下套,越强硬的拒绝才会越安全。

开罪了宰相,韩冈倒也不是很担心。反正至少在短时间内,韩绛不可能找茬整自己。他的两封请调的奏章,现在还在中书门下的架阁库中放着呢。也许过上几个月,现在的这份护体金身当会褪去颜色,但那时候,韩绛可不一定还能在现在的这个职位上。

在重又变得恭敬起来的门房恭送下,韩冈踏出帅府,一点冰凉忽而落在脸颊上。他抬头天际,晦暗的云层已经遮蔽了一切。鹅毛大的雪片,洋洋洒洒的落了下来。

探出手,指头大小的雪花打着转落在了掌心中,随即便融化消失。收掌握拳,些微寒意从掌心的肌肤中沁入,韩冈微微冷笑:“果然还是下雪了!”

回到驿馆,种建中并没有去访友。而是站在庭院中,也是抬头望着天,头发肩上落满雪花,脸色与天空的颜色一样阴沉。

韩冈毫不惊讶种建中的心情变化,脚步随即放重了一点。

听到韩冈回来的动静,种建中回过神来,“玉昆你这么快就回来了?见到韩相公了?!”

“见到了。”韩冈略一点头,却道:“延州下雪,不一定绥德、罗兀也有雪。隔着快两百里,不必太过担心。”

种建中挤出了一个苦涩的笑容:“玉昆你是不知道,绥德、罗兀与延州,天候变化许多时候都是同时的。而且延州这里下场小雪,往往绥德哪里。反倒是山北的银州,天象却是与咫尺之遥的罗兀城不尽相同。”

绥德、罗兀既然处在延州上游,地势理当比延州要高。三地既然同在横山南侧,气流受到山势影响,也的确是位置越高的地方雪会越大,绥德大过延州、罗兀又大过绥德。反倒是有山势阻隔的山北银州,情况会好上一点。

“秦岭的气象好像也是南北不一,同在秦州,山北成纪县就与山南的天水县有很大差别。”韩冈说着,“如果真如彝叔你的说法,那绥德、罗兀现在也当是下雪了。不过既然选在正月用兵,事先不会没有预计到会有现在的情况吧?”

“预计是预计到了,但……”种建中又看了眼雪片越发的大起来的天空,摇头苦笑:“再怎么预计,看到下雪,心里总是不爽利。这场雪,不知要给筑城之事添上多少麻烦。”

韩冈安慰似的拍着种建中的肩膀,掸去积下来的雪花:“往好处想,雪下得越大,西贼那里也不好进兵。”

“但愿如此。”种建中抿了抿嘴,却不见半点宽慰。又叹了口气,问韩冈道:“玉昆既然见到了韩相公,那你接下来的行止如何?”

“韩相公已经下令了,即刻启程,去绥德令叔帐下报道。”韩冈拱了拱手,笑道:“还望彝叔多加提点。”

