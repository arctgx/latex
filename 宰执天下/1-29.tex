\section{第14章 辘辘尘道犯胡兵(下)}

“原来真的是十九哥的同学!……”这下轮到王舜臣吃惊了,他本以为韩冈自称是横渠弟子不过是吹嘘,要不然早就开始拉关系了。却没想到韩冈竟然一口报出种十九的名和字,真的是十九哥种建中的同窗学友。

韩冈笑了,王舜臣先前的怀疑和现在的惊讶,他都看在了眼里,“说是同学,其实也不怎么亲近,先生的弟子众多,我和彝叔话也没说过两句。韩某是个书呆子,白天受教,夜里回去抄书,论起亲近的同窗,还真是不多。”

“那也是同学啊……”王舜臣豪爽的拍了拍胸脯,“秀才你放心,既然你是十九哥的同学,那就不是外人。别的洒家不敢说,只是外面的那两个鸟货,洒家保管他们这一路上别想闹出什么花样来。”

韩冈低头称谢,王舜臣如此保证,那这几天就可以安心了。

有了种建中这层关系,两人自感亲近许多。举杯跟王舜臣对饮了三杯,韩冈突然想起了一件事,问道:“对了,军将。有件事想要问一下,如今种家里,有没有大名唤作师道的?”

王舜臣想了一下,摇了摇头:“没有。”

“确定没有?”

“当然,除了这两年新出生的,种家的其他人洒家都清楚,肯定没有一个叫种师道的。倒是七郎家的二十三,也就是十九哥的同胞兄弟,名叫师中。名字有点像,但年纪才十三……【注1】”

……………………

在陇城县歇了一夜,第二天刚交三更二鼓,韩冈等人便起身。随便吃了点东西,再次启程,转向西北而行。黎明前的黑暗中,几支火炬照着前路。在身侧滚滚而流的,也不再是藉水,而是更加汹涌浑浊的渭水。这一天是沿着渭河走的一程,山道狭促,极是难行。不过有一点值得庆幸,就是天上看着要下雪,但最后却没有下下来,反而放晴了。

这一天,韩冈提着心思,随时准备解决薛廿八和董超两人,在他看来,从秦州到甘谷的四天路程中,第一天是通衢大道,而第四天行走在守卫严密的甘谷中,都不会有危险。可能会出问题的只有第二程和第三程。但一路上什么事也没发生,顺顺当当的抵达了目的地三阳寨。两天来,薛、董二人很老实跟着队伍在走,韩冈故意和王舜臣几次联手整治他们,可两人都是忍了下去。

看着两人的反应,韩冈越发的确定,危险的确是越来越近。有王舜臣在侧护翼,自己又是有着几条人命在手,董超和薛廿八却还是很有自信的样子,那唯一的可能,就是他们还有外援存在。

等到了启程后的第三天,又是三更多便启程,从三阳寨出发,用了几个时辰穿过峡谷山道,在中午时抵达夕阳上镇【今天水新阳乡】。一行人在镇子边找了个日头好的地方,停下来歇息。

夕阳上镇位于群山围绕的一块盆地中,是渭河这一段河道中难得的平坝,有不少商旅经过此处时顺便歇脚,形成了一个繁荣的市镇。而在其西北五里,还有个夕阳下镇,那里驻扎了一个指挥的禁军,权作防护。

王舜臣大马金刀的坐在骡车上,揉着脚腕。他虽然是骑兵,但战马难得,也舍不得多骑耗费马力,他的这一路来,反倒是走路的时候居多。他揉着脚,一边道:“到了夕阳镇,今天的这一程就已经过半。歇息个两刻,快一点过了裴峡,到了伏羌城就可以好好歇歇脚了。”

韩冈却是站着的,他遥遥望着西面的裴峡峡口,眉头紧皱:“要说险要,我们这一路几个峡谷是以裴峡最险,如果有什么贼人想劫道,也只会在裴峡里。”

“韩秀才,你在说什么呢?”王舜臣大笑道,“劫道?谁敢!”

韩冈侧头看了一下躲在二十多步外的薛廿八和董超两人,“韩某杀了刘三三人,又逼得黄大瘤自尽,为了尽快结案,陈举花了几万贯。他是恨我入骨,不可能让我韩冈安安稳稳地将这批军资运到甘谷城……”

王舜臣并不在意:“怕什么。若薛廿八和董超两人想做鬼,洒家帮秀才你找个借口弄死就是了!正好裴峡河窄水急,报个失足也就是了。反倒到了甘谷城后,秀才你该小心点。”

韩冈当然知道甘谷城里不会没有陈举的人,但到了甘谷城内,陈举不可能不会担心韩冈也许会有的后手。几次交锋,陈举还没能在韩冈身上占到什么便宜,若他以为能动用一下甘谷城里的自己人,就能解决韩三秀才,未免就太自大了。再怎么说,韩冈都是得世人敬重的读书人,而不会顾忌这一点的,只有愚昧无知的蕃人。

二中选一,挑选出一个方案解决韩三秀才这个心腹之患,陈举也许还要考虑一二。但一个是双管齐下,一个则是只靠甘谷城里的盟友,那就不必多想了。多一个手段,多一份保险,一直都在暗中盯着薛廿八和董超的韩冈,他现在有九成把握能肯定裴峡中有埋伏。

“陈举手下可不只薛廿八和董超,听说他还能驱使蕃人啊。”知己知彼,百战不殆,韩冈自从与陈举结下死仇,很是费了一番心力去打探陈举的情报,“陈家的店铺跟秦州西面山上的几个蕃落生意做得可不小,私盐、私茶从来不少的。”

秦州西面的山地,其实就是藉水和渭水之间的分水岭。若没有这重分水岭,那秦州与夕阳镇的直线距离,就只有三十多里,根本不需要绕上两天的路。所以与陈举常年买卖的蕃落所处的位置,应该就是裴峡正南方的山上。

王舜臣嘿嘿笑了两声:“秀才你想太多了。传说而已,谁也没见过!”他再一指周围,“何况军资又不是好劫,就算那些蕃贼有这个胆子,也没那个能耐。”

从秦州到甘谷,除了一些盘山道外,都是三丈五尺的军用驰道,不到两百里地,沿途大的城寨就有五个,小的堡子、烽火台随便在哪里抬抬眼就能看见几座,各处寨堡驻扎的军队加起来足有三四万人。这是一条以一连串寨堡组成的防线,拥有多达百里的纵深,其防御力并不比长城稍差,而攻击性则更高。这条寨堡防线,绵延两千里,宋人用了一百多年也没能修筑完成,但已经足以让西夏的铁鹞子望关中腹地而兴叹。

“总得小心为是……我们出城时,陈举正在城楼上看着。有军将你庇护,这一路韩某不需要再担心薛廿八和董超。陈举若想杀我,等我入了甘谷城可就迟了。韩某不信他能看着军将你跟我一起上路,还能把宝压在薛董二人身上……很有可能陈举会通知他惯熟的蕃落,在路上劫个道。

沿途寨堡防住西夏一点问题也没有,但说起蕃人,军将你也知道,这条路上平日里有多少蕃人在走?!别的不说,经略相公前段日子坐镇陇城县,为的什么?还不因为有四千石的粮秣,在往笼竿城的道上被蕃人给劫了!”

“真来了那更好!”王舜臣眼眉挑起,摩拳擦掌,兴奋得不骂上两句就感觉表达不出自己的心情,“日他娘的,陈举那鸟货要是能给洒家送些功劳,洒家可不会客气!”

……………………

在渭水沿岸,所谓的峡谷,就是被水流切割出来的黄土沟,一条大沟两侧有无数条如肋骨一般排列的小沟,而小沟两侧又有许多【和谐万岁】毛细沟。好好的一片黄土高原,被冲刷得千丘万壑,许多地方寸草不生。不过此时的裴峡两侧,树木却不在少数,丛丛密密,从东侧峡口一直延伸到西侧峡口。

裴峡并不算长,只有不到二十里,但顺着河岸边的山道赶着车子,少说也要近两个时辰。走在队列中央,韩冈提着一张六七斗力道的猎弓——临行前,韩千六交给他的不仅仅是钱钞,还将那张旧弓保养了一次换了弦后送来——他不时抬头看着谷地两侧的沟壑和密林,那里都是能藏人的地方。

“都给我打起精神来,走快一点。这里可是有蕃贼出没!”韩冈催促着手下的民伕。王舜臣自信得过了头,但韩冈却是小心谨慎,若真来了劫道的,就算只打碎了坛酒,到了甘谷也是桩麻烦的事。

没人敢说韩冈不是,但民伕们都是暗暗摇头,只觉得韩秀才太过杯弓蛇影。可世事从来都是没有最糟,只有更糟,事情总是会往更坏的情况发展。

“有贼人!”不知是谁人在前面叫了一声。下一刻,前方道路一侧的林木中,便突然间杀出了一群手持弓箭长刀的蕃人来。这些蕃人行动极快,几步冲出林子,跳上官道,直接杀奔过来。

民伕们战战兢兢,看着韩冈的眼神也自不同,心中皆是抱怨:‘这秀才是盐酱口,一说蕃贼,蕃贼就来了。’

“怕是有四五十人。”韩冈的脸色郑重无比,陈举的影响力超过他的想象。四五十人听起来不多,但这个数量的贼人出现在前线要道上,甚至能惊动到李师中。如果贼人身份泄露,他们的部落恐怕都被视为谋反而被官军荡清,这不是没有先例。当年曹玮曹太尉守边的时候,用这个罪名灭了不知多少蕃部。不知陈举许给了他们什么愿,竟然如此不顾后果?!

韩冈一瞥身侧看不出什么惊慌神色的薛廿八和董超二人,一支白羽箭随即搭上了弓弦,‘攘外必先安内!

“鸟蕃贼!”王舜臣则大喝一声,提弓在手,喜上眉梢,“送功劳的来了也!”

注1:种建中就是种师道。他之所以会改名,是因为他要避徽宗年号建中靖国的讳。在徽宗登基之前,并不存在种师道这个名字。

ps:果然有人猜中。种建中就是日后的种师道。老种经略相公在此时也不过是毛头小子,而他的名字在因为要避宋徽宗的建中靖国年号而修改之前,始终都是种建中。在神宗朝,不可能出现种师道这个名字。

今天第二更,继续征集红票和收藏。

