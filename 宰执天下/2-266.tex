\section{第39章 铜戈斑斑足堪用(中)}

刘源收到命令后,便纵马而出,直奔曾经的归属之处。

广锐军被韩冈带出城来后,就被安排在后方。一直没有等到韩冈的命令,如果是配属在他人麾下,能在后面纳凉,根本就是求之不得的美事。

但韩冈毕竟于整个广锐军有恩,自从流放到河湟之地后又多方照顾。现在跟在韩机宜的身后,前面打得热火朝天,而自己却坐着冷板凳,捞不到一个上阵的机会,许多人都微微的有了些怨言。

当他们看到刘源终于狂奔过来,手上还掌着一面红色的旗帜,广锐军上下差点就要欢呼起来。就算捞不到封妻荫子的机会,能得到金银财帛还有土地,也是一桩美事。跟着韩冈,他们的功劳绝不会被人贪墨。坚信着这一点,广锐军才会紧随着韩冈,毫无半点动摇。

广锐军的番号和旗帜早已成为沉寂,世间再也找不到曾经让千万人跟随过的,那面镶着黑边、绣着广锐二字的血红大纛。刘源带来的旗帜没有任何纹饰,就是一面纯而又纯的红旗。但聚拢在同一面旗帜下,依然让千百名为广锐之名而厮杀过的战士心潮起伏。

红旗招展,随着旌旗所向,潜藏洞中的毒蛇终于亮出了毒牙。

绕过了前方刀箭激烈的战线,从两翼插进了吐蕃人的阵列。在千名广锐军的一击之下,被分派上来迎敌的蕃军竟然毫无抗力。

原本青谊结鬼章就已经顶不住宋军的攻势,只是在援军抵达之后,胆气复壮,才跟对面的宋军对拼个有来有往。但现在广锐军这个生力军投入战斗之后,情况立刻急转直下。

重新恢复了完整形态的广锐军,虽然眼下的编制和人数,都只有旧时的三成不到。但一个完成的作战体系所展现出来的战力,充分证明了他们当初是如何凭着区区三千人,就震动了整个关中,让天子夜不能寐。

青谊结鬼章本已经难以支撑,本来他的目的是缠住宋人,让他们不能撤退,可现在情况完全逆转,变成宋军死命的缠住他的队伍,不让他回撤。

战线犬牙交错,越来越多吐蕃战士都放弃了战马,紧密的站在一起,拿起刀枪与宋人面对面的厮杀。阵列前都是一刀一枪的交换,每砍倒一名宋人,自己一边就同样有着一名族中兄弟倒下。

无数血水从不同的源头汇聚过来,在人们的脚下化作一条小溪流淌。踩过一汪汪血水,溅开的红色液体将一双双眼睛染得血红。他们投入惨烈的搏杀,狂吼着将手上的凶器挥砍出去,全是因为相信援军在稍事休整之后,就能投入战斗,将胜利带回到他们的手中。

可是当先赶来的不是自家的援军,而更是精锐、更为勇猛、也更为疯狂的宋军!

原本就已经变得十分脆弱的堤坝,挡不出新的一波更加汹涌澎湃的浪涛。已经是勉强支撑的战线,最后也抵挡不了新锐战力的冲击。

见着广锐军如同热刀插入黄油一般切开吐蕃人的队列,韩冈暗叹要不是当初吴逵选择了坐守咸阳,而是流窜于关中,别说现在能直攻河州,就是能保住陇西不被放弃,便已经是万幸了。

吐蕃人被广锐军的疯狂给吓住了,青谊结鬼章的命令不再管用。有许多人想往山里逃窜,但他们立刻发现,从现在的位置前往最近的一条进山小道,免不了要通过宋军的战线。

“拼了!”

栗颇一声大叫。他已经一日一夜没有长时间的休整,加上自香子城下转进的这几十里,整个队伍早就人困马乏,失去了大半的战斗力。他本来以为到了珂诺堡后,还可以在城下休息一阵,然后再与青谊结鬼章并力攻城。靠着珂诺堡中的几处暗道,还有全是步兵的宋军赶来之前的时间差,当能顺利的攻下这座城堡。

可谁能想到,竟会直接被卷进了战场之中?连个休息的时间都没有?

栗颇将悲叹丢到脑后,举起手中的腰刀,命令自己麾下的战士前去援助溃败中的青谊结鬼章。

可是,就只有很少的几十骑响应了他的命令。

都太累了。不论是人还是马,过度疲劳的情况下,一旦歇息下来,再想催动他们,那是比登天还难。人能感受到迫在眉睫的危机感,在危急关头能迸发出难以想像,但四条腿的畜牲却没有这么骑手们一次次的挥鞭,换来的也只不过是胯下坐骑的团团乱转。

“下马!下马!”

栗颇当机立断的发号施令。

但听他的话的人却不多,都是自家的战马,丢了怎么办?一时的犹豫却立刻造成了无法挽回的结果。溃散的鬼章军纷纷逃往道旁的山野之,只有少部分逃了会来。腾出手来的禁军步兵,用神臂弓隔空清理了一通栗颇手下的战马,随即手持刀斧的广锐军,便一头扎进了阵中。

混乱之中,任凭个人再是武勇,在如泰山崩石一般横扫过来的军阵面前,也只有被碾压的份。冲入敌阵中的广锐军,单看其个体,是一片纷乱杂乱无章的行动,其实却是在军官们的指挥下,于大方向上井井有条。

为数千人的精兵悍将,如同春蚕啃食桑叶,不断侵蚀着吐蕃军的阵线。长刀横扫,大斧纵劈,群狼掠过的地方,只留下一片血光。不住的被挤压,眼见得阵脚再也支持不住,栗颇终于忍耐不住,派出了紧跟在他的身边,族中最为精锐的一支百人队。

从数万人精挑出来的百名战士,又是聚集在一起。只是并排站成数排,用着长弓,连续射击数轮。就让怎么也无法阻挡的广锐军的攻势,终于为之一滞。

尽管覆盖式的射击,让与宋军拥挤阵前的自家人付出了很大的代价,但战局一时间的扭转,给了栗颇组织兵力重新反扑的勇气和信心。

“兵力还是我们这边多,解决这几百名汉人,剩下就不足为虑!珂诺堡肯定能打得下来!”栗颇面目狰狞的吼着。

而在他前方数十步外,刘源正恨恨的看着自己挂在胸腹腰肋处的数支长箭,他方才领军冲得最猛,要不是韩冈赐给的精铁甲胄,今天少不得就要躺着回去。

“攻上去!别让他们有喘气的机会!”

刘源将挂在盔甲上的长箭用力一把拔出,用着比吐蕃将领更为响亮的吼声,吼着自己命令。眼见着吐蕃人有重新整顿战线的动作,深悉军事的将领便不会给他重振旗鼓的机会。

但改变战局的不是广锐,也不是重新组织起来的吐蕃军

原本在开战时就被韩冈派出到山上的两支偏师,这时终于迂回到位。虽然一边各只有一个指挥,但当他们终于出现在战场,出现在吐蕃人的后方。甚至还没有来得及射出第一批箭矢,他们的目标就已溃不成军。

“栗颇,退!”

青谊结鬼章领着残部从还没有合围的宋军之中狂奔而去。离开之前,还秉持着一点香火之情,高声提醒了栗颇一声。而愤怒中的木征家大将,恨恨的一挥马鞭,同样调转头来,也跟着青谊结鬼章,向着他方才的来路狂奔而去。

韩冈远远望着前方摇晃的战旗,听着万众高呼的声音,终于长舒了一口气,摊开手,湿漉漉的掌心证明了他方才的紧张和忧心。

一名骑兵狂奔而回。

“刘……刘……刘……”刘源派回来的信使不知用什么词来称呼现在广锐军的领军人,‘刘’了半天,才想起了他现在的身份,“保正让小人问机宜,是否要他现在回撤?!”

刘源这话问的,分明是想继续追击。吐蕃人的战马都已经没有了气力,韩冈甚至在看到几个当场倒毙于地的例子。这种时候,追上去就能让吐蕃人的大部分战马完蛋大吉。如何不追?!

“去跟刘源说,让他追下去!”韩冈立刻下令。

传令兵接下命令,转身就骑着马跑了。

虽然配属给广锐军就十几匹用来传令的战马,但韩冈相信,今次一战,刘源肯定能夺下几倍几十倍的马匹。

在等待前线的回信中,宋军已经开始打扫战场。死亡的战马、被累垮的战马数以百计。而蕃人的尸体也同样是以百来计数。

这又是一个胜利,当韩冈看着被人呈到自己面前的田琼遗骸的时候,也终于能说一句‘你可以瞑目了。’

几名俘虏被押解过来。他们已经问明了身份。在没向他们问话之前,就拿了两个吐蕃俘虏开刀。直截了当地询问手段,甚至连用刑都没有,就让他们都如竹筒倒豆子一般的全都说了出来。

香子城已经解围了,而一开始就在香子城下吃了大亏,让韩冈也为之吃惊。

“王舜臣干得不错,难怪要转进,已经被击败的败军还想在我这里找回面子,这还真是好笑。”

韩冈哈哈两声大笑,带起了周围将士的一片笑声,笑着吐蕃人的自不量力。

