\section{第三章 时移机转关百虑(三)}

元丰元年对于东京城中的百姓们来说,应该是记忆深刻的一年。

这一年让他们产生惊奇的事和物,实在是太多了。

尤其是进入冬季之后,先是襄汉漕运打通,六十万石纲粮只用了一个月多一点的时间,便送到了京城。同时轨道的运用,也让世人看到了不输水运多少的另外一种运输方式。

继而又有了种痘之术,害死了无数人的痘疮,终于有了预防的方法,朝廷为此设立了一个衙门,专门负责种痘,家里的儿孙就此有福了,至少不用再战战兢兢的害怕他们被痘疮夺去性命。

人们本以为惊喜到此为止,谁成想,赶在过年前,又有了更让人欣喜不已的消息。辽主为权臣所害,从百丈高空坠落,摔成了一滩肉泥;而西夏国母则是囚禁了她担任国主的儿子。只要心明眼亮,没人会看不出来,辽国即将面临一场内战,而西夏也同样人心涣散。

辽国和西夏同时陷入内乱的消息传来,助长了民间和士林谈论兵事的风气。

已经不是仁宗的时候了,被辽夏二虏逼得近乎走投无路,只能卑辞厚币来讨好。如今的大宋,坐拥六十万甲士,有着灭国之力。再加上几场战争都是胜得干脆,于民无伤,对于甚嚣尘上的讨伐西夏的战争,支持一派远远多过反对者,仅有的争论,也只是速攻和缓攻的区别而已。

只要打下了西夏,到时候辽国也就没胆子敢南下犯界,太平的日子便能安稳永享。

“要不是担了这份差事,其实下官也是想去陕西随军出征。”李德新对韩冈叹道,“先父为元昊所害,此乃不共戴天之仇。要是能亲自去兴庆府走上一遭,为先父报仇雪恨,当是一桩快事。”

“如今保赤局中,可离不了易一你。”

李德新叹气声更重了,“这个年节过得好生无趣。也就除夕和正旦能歇上两天。本来想早些来向龙图拜年,谁想到保赤局给人种痘一天都歇不得。祭灶后就放假说不过去,但都到腊月廿七了,刚准备关门,几个侯伯就告到了天子那里……”

韩冈笑道:“谁不担心自家的儿孙在年节时出意外?早一天种痘,早一天放心。皇子公主都种了痘,也没人想再等等看了。”

“龙图教训得是,是德新的眼界太浅了。”

李德新说了两句话,留下了一份礼物,就匆匆走了。他如今已经将家眷接到了京城,而且他的几个兄长也住在京中。李德新认祖归宗后,除夕要祭祀先祖,不能耽搁时间。

韩冈目送着李德离开,韩云娘从厅内小门出来,向着客人离开的方向张望了一下,嘟囔道:“上一次来家里还陪着小心,怎么今天就敢在三哥哥面前抱怨了?”

对于在除夕还上门来拜访的客人,韩云娘说不上有好感。一年中的最后一天,除了出门燃放鞭炮的人们,街巷上的车马行人几乎绝迹,本来就该是一家人坐在一起的时候,却还来登门拜访,岂不是惹人厌?而且对于保赤局这样占了韩冈大便宜的衙门,李德新在里面功成名就,韩云娘本来就有几分不待见。

“他不是忙的吗?给人种痘,连个好好歇息的时候都没有,今天才放了假。”韩冈帮着解释了两句,“李德新他也算是出头了。天子那里挂了名,皇亲贵胄没有不认识他的。”

身为厚生司保赤局中掌管种痘诸事的医师,李德新的地位已经赶得上太医局的翰林医官。入宫给六皇子和淑寿公主种痘,受到的赏赐有上千贯,为雍王的子女种痘,他得到了汴水边一套两进的宅子。除此之外,还有其他高官显宦、皇亲国戚的馈赠,都是丰厚异常。转眼间的功夫,李德新在京城中,已经是有房有车有地位的成功人士了。

“三哥哥,是不是又有什么事要出来了。”韩云娘发现韩冈眉宇间的忧色,是淡如轻雾却化解不开的那一种。

“我在担心陇西。过了年后就要开战了。陇西也会征兵和调遣蕃军。”韩冈叹道,“这些年下来,青唐羌各部族长、耆老眼下基本上都是富家翁了,各个身娇肉贵,有几个愿意领军出征?他们族中的男丁皆是棉田的主力,一旦出兵,少了人手,就是几千几万贯的亏损。坐在家里看看球赛,隔三差五的来个怡情小赌,小日子多惬意?已经不是愿意拿性命去博富贵的时候了。”

“这都是三哥哥的功劳。”

韩冈摇摇头,他不知道会有多少人相信自己的话,但他已经做到自己所能做到的,多多少少的也算是尽了自己的一份心力。

郭逵在前一日已经同意去河北。依照之前在崇政殿中的商议,郭逵应该是加官一级,升了枢密副使,去河北担任宣抚使。

但这项任命还是有人反对,说针对性的意味太强了,担心会引来不必要的麻烦,而赵顼也同意了。在韩冈看来,应该是担心一旦辽国当真能分心南下,郭逵又击败了他们,使得赏赐最后不好给。

这真的是该叹气了。

已经是黄昏时分了,鞭炮声突然响亮了起来,仿佛摁下了开关,房间内也开始韩冈起身回了书房一趟,拿了一封信出来。

一家人已经团团坐内厅中,一家之主终于到了,气氛顿时就跟外面的烟火爆竹一样热烈了起来。

“是苏伯绪的信?”等韩冈坐下来,王旖看了一下他手中信笺的封皮,上面就有苏子元的签名。

“在李易一来访之前,正好伯绪遣人送来的这封信到了。”韩冈说着。

这的确是苏子元从邕州寄来的信。苏子元在信上提到了邕州这一年来的现状。户口已经有了战前的六成,二三十年后多半就彻底恢复了。

另外还感谢了韩冈派人为邕州送来痘苗,金娘已经种过了痘——韩冈在为牛痘上书天子的时候,也派了人带了疫苗去广西,李信的三个儿女、还有苏子元的女儿、韩家的儿媳妇,自然是越保险越好。

“怎么到得这么迟?是不是有什么事给耽搁了?”严素心问着。老大是她亲骨肉,最是关心邕州,“去年就没有到得这么迟过。”

“谁知道,信上没有写,他派来的人也没有细问。”韩冈摇着头。

结合了顺丰行搜集的资料,以及李信和苏子元的来信,韩冈对广西的情况有了更急一步的了解。绝大多数的问题基本上可以归结为户口稀少的缘故,广西和交州能不能安定下来,都要看日后的人口增长,能不能满足朝廷的需要。

苏子元的来信上,邕州关于增加户口的措施,被他详详细细的解说了一遍。韩冈案看了之后,不置可否。然而来自于邕州的信并不只是一封,韩冈从信封中抽出另外一封信,笑着递给老大韩钟,“还有这是给大哥,是金娘亲笔写得。”

家里的老大抓着韩冈的衣袖,轻轻摇着:“爹爹……女儿没有给钟哥儿、钲哥儿写信。”

韩冈和四名妻妾闻言,就一起笑了起来。周南笑着搂住女儿:“是广西的金娘。”

“是不是该给金娘起个闺名了?”王旖问道,“转了年,三哥儿他们三个就要荫补封官了,正好都要起个正经的名字。”

“是啊,三哥哥。”韩云娘说着,“大哥儿、二哥儿都有名号了,金娘和三哥儿他们总不能还是叫着小名。”

“记得以前曾经说过,家里已经有一口钟,一个钲,再来三个,就能凑齐一个班子……”韩冈话说到这里,望着几名妻妾一点笑意都没有的眼神,“说笑罢了,自家的儿女,可舍不得让他们成为笑柄。”

韩冈用手蘸了茶水,在桌上写了三个字,锬、铉、钦。写下了这三个字,他笑道:“其实这件事,我已经考虑过了。三哥儿韩锬、四哥韩铉,老五则是韩钦,就这么叫吧。”

钦字是常用字,锬和铉两个字却都生疏。王旖看看韩冈,心中堵着一口气,不问这个不负责任、拿儿女名字开玩笑的父亲,却叫着身边听候使唤的使女,“去拿说文解字来。”

说文解字就是此时的字典,书一到,王旖就开始查起来了。

锬是长矛,铉则是古代的举鼎器具,其状如钩,可以用来提鼎之两耳。

两个字说好不能算好,但至少比之前韩冈开玩笑时起的名字,要强上千百倍。几名妻妾互相之间却皆是点了点头,都不反对这一提案。

“至于金娘,也从兄弟一起排行好了。”韩冈想了想,“钟声为一人而鸣,锳这个字不错。叫做韩锳如何?”

还是不算多好的名字,韩冈没有起名的天赋,在从钅的字中,适合做名字的也没几个。不过王旖他们也没反对。

等过了年后写信去陇西,让几个孩儿在族谱上登了名字——尽管只有韩冈这个独苗——这件事就算有了个结果。

鞭炮声如春水般连绵不绝的响着,由三个大一点的孩子开始,韩府中人一批批向韩冈夫妇跪下磕头,问安。然后接过今年的红包。

元丰二年,终于到了。

