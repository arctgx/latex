\section{第28章 大梁软红骤雨狂(三)}

王安石的宰相府如今仍是他先前任参知政事时的旧邸,也是他三年前入京后,就从官中分发下来的宅院,一直没有变动。对于只有二三十个仆妇的王安石家,这间宅子本也是足够了。只是如今升任宰相,以礼绝百僚、群臣避道的宰相之尊,显得太过寒酸,有失朝廷体面。天子便赐下了新邸,就在皇城边上。

只是新邸虽赐,但王安石还是上表给辞了——这是天子恩赐,就要照规矩来的,需要辞让一番,才能接受。对王安石来说,他其实觉得很麻烦,要是天子不再重复下诏赐宅就好了。现在的宅子已然过大,换个更大的不是更麻烦?

不过对于拥挤在王府门前的官员们来说,他们还是觉得王安石家最好早点搬迁为上。只有六七步宽的这条小巷中的车马,比起夜中的小甜水巷,万姓烧香时的大相国寺,都要拥挤得多。数百名官员,加上更多的伴当,还有一样多的车辆马匹,把王丞相府门前的小巷堵成了暴雨后的下水道——天天如此,无一例外——唯一能让人欣慰的,是这里还算安静。在丞相府前,说话也要屏气静声。

腊月十五的这一天,随着王安石从宫中散值回府,一个个要拜谒他的官员陆续赶来,将车马停在了巷口,让仆人上去递了门帖,就在门口等着。到了夕阳西下的时候,又是一名仆役穿过人群,向宰相府的偏门挤过去。

不少人嘲笑的回头望着巷口处刚刚赶来的两名官员,他们来得实在太迟了,现在才来,今天根本不会有机会了。

但让所有人吃惊的是,这名仆役并没有在偏门处向门子递上主人的名帖,而是打了个招呼,就走了进去。而片刻之后,一个年轻人就跟在那名仆役身后从门中快步出来。认识年轻人的官员不少,当即起了一阵骚动,窃窃私语的声音,就像被人捅了一下的蜂窝,顿时嗡嗡嗡的响了起来。

“是王二衙内!”

“是谁来了,怎么是王家二衙内出迎?!”

王旁在家丁的引领下,快步从人群中穿过,迎面的官员纷纷避让,脸上浮起谦卑的笑容。数百只眼睛追着王旁的身影,一直到他停步的地方,就是方才遣了那名仆役进王府的两名官员。

这时终于有人仔细去辨认两人的身份,有见识的官员不少,最近甚得圣眷的王韶,四入宫掖,认识他的人很多。

“是王韶!”

“河湟王韶……上平戎策的那个。”

“……难怪了。”

“后面的那个高个儿是谁?”

“……跟班吧,大概王韶要举荐的。”

但接下了的一幕,更是让人吃惊。王旁的确是先跟王韶见礼,但很明显的,他与跟着王韶的年轻官员更加亲密。王安石家的次子一向阴沉,不喜与人结交,这是世人皆知的。可现在眼下众人看到的,却与传言差了不少,浮在他脸上的笑容比起跟其他官员见面时要亲切得多,

“玉昆兄,向来可好!”

韩冈笑着拱手回应,“托仲元兄的福。今天刚入城,放下行装,换了衣服就过来了。现在肚中正空,可是叨扰一顿晚饭了。”

王旁呆了一呆,转眼就更加欣喜的笑起来:“不敢让玉昆你饿着肚子,晚饭早已备下了,等与家严见过之后,当共谋一醉。”回头他便对王韶道,“家严正在家中见客,少待便有空闲。不敢让王安抚和玉昆在外久候,还请两位随在下先进家中稍等。”

几百只眼睛又妒又恨的看着王旁带着王韶、韩冈从偏门进去。看到王旁跟韩冈的亲近,王韶也是有些愕然。他只是听韩冈说过,跟王旁见过面下过棋,却没想到竟然如此惯熟。

韩冈跟王旁的关系当然不至于如此亲近,但他了解人情世故。王旁这样接触的多是别有用心之辈的衙内,只要用对方法,肯定是要比历尽宦海的官员更加容易接触。韩冈表现得越是洒脱不羁,不拘俗礼,王旁就越是不会摆出宰相之子的架子,反而会更添几分亲近感。

三人在韩冈所熟悉的偏厅分宾主坐下,让人进去通报了王安石。王旁跟王韶有些生硬的寒暄了两句,转头便问着韩冈:“听说玉昆你在蕃部中斩了一个西夏的使臣,是不是真有此事?”

韩冈神色不变,反问道:“这事是怎么传的?”

“秦凤走马承受传回来,还是天子聊天时跟家严说起的。”

“难怪!”韩冈点点头。关于他一剑杀了西夏派到瞎药那里撬墙角的使臣,明面上的功劳他的确是送给了瞎药,但私下里流传的话,却没有让人去禁言,也禁止不了。反正只要自己不承认,谁也不能把这事栽倒他头上。但熟悉韩冈性格的人都认定了他,他的性子刚毅果决,而且过去也不是没有先例,杀人放火,韩冈本就是行家里手。

王旁的眼神中透着好奇,见韩冈不否认,立刻追问道:“难道是真的?!”

韩冈笑了笑,正要说话。一名仆人走了进来,“相公已经在书房中等候,请两位官人过去。”

向王旁告了罪,在王家二衙内失望的目光中,王韶和韩冈被领着进了书房中。

今次书房里面,只有王安石一人。再一次见到这位千古名相,韩冈发现他已经憔悴了不少,黑瘦黑瘦的,颧骨下的阴影又重了许多,看容色,也显得很是疲累。

行礼落座,王安石也是先跟王韶说了几句话,但很快,就转到了韩冈这边,“玉昆,关于韩子华征辟你的奏文,你应该已经听说了吧?”

韩冈点了点头,“已经听说了。”

王安石也不绕圈子说话,直率的对韩冈道,“横山战事即起,所以韩子华幕中需要玉昆你去安顿军中伤病。连上两本奏文,可见其对玉昆你渴求之深。而战事一开,损伤难免,也的确需要你去主持。这件事,你就不要推辞了。”

不成想王安石竟然直截了当的命他去韩绛那里报道,韩冈想了想,便道:“光靠下官一人可不够,至少要调集秦凤上下三个疗养院中所有四百余人,才敷使用。”

“这么多?”王安石对疗养院不甚了了,听说韩冈一下要调去一个指挥的医疗团队,顿时吃了一惊。

“横山胜败未可知。罗兀城易取难守,若是不幸战败,恐怕四百多人还不够!”

王安石略显困顿的双眼一下睁开,锐利的眼神在点着烛火的内室中,如同闪电划过,“战败?!玉昆你说今次出战罗兀会战败?!”

“未虑胜,先虑败,此是兵法要旨。”韩冈停了一下,便正面回复王安石,“非韩冈战前出不吉之言,只是不想看着朝廷空耗钱粮,官军劳而无功,而陕西又平添无数孤儿寡母。罗兀易得,横山难取,此一战,还是输面居多!”

韩冈说得决绝,王安石眯起眼睛,“城罗兀,东连河东,南接陕西,二路并举,横山可定。韩冈,你说此战输面绝多,可是有何缘由?!”

“西贼不擅守城。韩相公坐镇延州,种谔出兵绥德,其余各路支援鄜延,以此规模,攻取罗兀当不在话下,击败西贼赶来的援军也不难。但要一年年的稳守下去,抗住西贼的反击,却是千难万难。”

“不还有横山蕃部在?罗兀一下,横山蕃部当会将尽投大宋。”

“与其寄望于人,不如求诸于己。即以河湟论,若非有古渭三千官军压阵,哪一个蕃部会老老实实的听命?蕃人可用不可信,更不可全然依赖,若是认为有着蕃人助力,就可以让西贼败退。这种想法,韩冈不敢苟同!”

韩冈语气激烈,王安石不由的瞥了王韶一眼。而王韶则是眼观鼻、鼻观口的默不作声,任由韩冈在前冲杀。王韶所在的位置让他不能肆意攻击韩绛,只有韩冈,因为要被调任鄜延,才有资格说话。

暗叹了一口气,王安石道:“种谔统领大军攻取罗兀后,已定要扩建罗兀。罗兀城中大军毕集,近处又有河东、鄜延可以支援,要慑服众蕃,击败西贼,当不至于有何困难。”

韩冈也叹了一口气:“下官方才也说了,夺取罗兀容易,击败援军不难,但守住罗兀却是难得很。因为罗兀城中能驻扎下的兵力,跟城池大小无关,而是取决于运送到城中的粮秣数量。”

“从绥德到罗兀不过六十余里。六十里转运,快则一日,慢则两天。城中的粮秣当不至于匮乏。”

“怎么会是六十里?!”韩冈立刻摇起头,毫无顾忌的反驳着高高在上的宰相,“绥德到罗兀的确是六十余里近七十里,但清涧城到绥德却是八十多里。罗兀城的一切用度,起点都是清涧城,而不是绥德——绥德本身的需用就要靠清涧城转运。也就是说,供给罗兀城的粮秣所运输的距离,不是六十里,而是一百五十里!”

