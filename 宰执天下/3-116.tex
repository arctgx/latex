\section{第30章 众论何曾一(九)}

不移时,一行便已抵达县中。

安排下住处,曾吕等人就先向韩冈告辞。他们在风沙地里奔波了一天,急着要去沐浴更衣。

韩冈也有事要做,王旁带来的两名木匠已经被王旁献宝一般的介绍了,尤其是俞皓的重孙俞正,更是被他推重。

俞皓在民间颇有一番神秘色彩,他曾经说开宝寺木塔受过百年西北风后就会被吹正,必定会有人想,那再过百年,木塔不就会向东南倾斜?可事实却是刚过百年,开宝寺木塔立刻就被烧掉了,再没有被风吹得向东南倒得情况。变成这样的结果,很容易就会让人联想起俞皓当年的一番话——难怪他不说百年之后的事。

不过这个时代,再有名的工匠,也比不上一个庸庸碌碌脑满肠肥的官员。俞正在韩冈面前小心翼翼的,韩冈让他坐下来说话,也是摇头说不敢。

也不强迫两名匠师,问了几句有关风车的事之后,韩冈吩咐了下人将他们安顿下去好生款待。过了一阵,方兴来报,说是接风宴席已经布置好了。韩冈命人去邀请曾布、吕惠卿等人入席。

韩冈今日要接待的,不仅仅是曾布、吕惠卿和王旁。还有两位随行的官员。其中一人韩冈没有印象,但另外一人——魏继宗的名号,韩冈可是如雷贯耳。

韩冈不认识魏继宗,但听过他的名字。在便民贷、免役法、保甲法顺利推行,而河湟开边又大获成功,使得新党地位稳固、朝堂终于平静下来之后,将两党战火重新点燃,惹起了这一场轩然大波的罪魁祸首,韩冈怎么可能没听说过他?

魏继宗从布衣被拔擢入官,靠得就是他市易法首倡者的身份。客一部市易法惹来了如此多的纷争,甚至使得新党的政治根基都开始被动摇。从东京市易务中一年得到的几十、上百万贯收入,看似不少,可对于新党来说,其实还是得不偿失。要不是为了新法整体的安危着想,即便是以王安石这位拗相公的性子,也肯定会将之废止。

魏继宗在东京市易务中被投闲置散,其原因根本不需要多想。可如今曾布、吕惠卿却又带着魏继宗一同上路……一同前往河北体量市易务,其中不知到底有什么考量。

等到五位客人应邀到齐,韩刚请了他们入席,他的三名幕僚也入内陪席。官位最高的曾布理所当然坐了上首,等到各自都坐定,韩冈举杯道:“此番酒宴过于简薄,还请各位海涵一二。”

韩冈的话不是客气,而是当真简薄。分席制的宴会,一开始摆出来的开胃菓子,就只有两样,更没有什么看果之类纯摆设的看菜。开场决定了后续,后面的下酒上来,也不可能多奢侈。招待过路官员的所有花销照例都是从公使钱账上走,一县之地也不会有太多的公帑供韩冈招待客人。若是花得太多,就得等着御史开骂了。

曾布举杯回应:“玉昆哪里的话,我等正是要去河北察访灾情,若玉昆当真铺张开来,曾布可是不敢入席的。”

吕惠卿也道:“天子如今已居偏殿,减常膳,我等不能为君分忧也就罢了,如何还能违逆圣上之意。”

曾吕两人都没指望韩冈会坏了自己的名声而大肆铺张的设宴招待。开封府人多官多嘴也多,盯着韩冈这边的眼睛更是太多,若是有哪怕一星半点的不是,韩冈也会被拎出来穷追猛打,更别说在如今的情况下大开宴席。曾布和吕惠卿两人都会感到忌惮,即便韩冈敢于摆下奢侈宴会,两人也不敢入席。

举杯行过三巡酒,说了一阵闲话,话题也逐渐转到正事上来。

“不知粮商一案处置?”韩冈问着,这一案有他的一份功劳在,虽然现在没他的事了,可也是他关心的焦点。

将酒杯放下,曾布道:“追毁出身以来文字这是肯定的。”

所谓出身以来文字,说白了就是官员得官的个人档案。就算是发配岭南,只要出身以来文字还在,即便所有的职位都被撤了,依然还是官。而毁去了出身以来文字,便是将粮商们从官籍彻底打回民籍。

吕惠卿不以为然的笑了一声:“也只是做给外人看,过两年就能补回来了。”

粮商们娶了宗室,翻身的可能性还是有的,碰上一次南郊祭天,大赦诏书一下,过往罪愆基本上就会被赦免。到时候又会跑出来让人碍眼。

“杀几个,流几个,放几个,也就是这样了……”曾布冷声说道,“还是要订立法度,以防日后奸人为乱。”

“低买髙卖,囤积居奇,乃是商人天性,也是常理,立法岂能扭转?”韩冈却道,“事关百姓的盐与酒都是官营,若立法度,只要放在粮食上就够了。至于他物贵贱变动,倒不至于影响民生。”

对于朝廷控制商业的做法,韩冈并不是很认同,就连市易法他都不赞同。利用经济手段让囤积居奇者血本无归,才是正常手段。此次使用刑律直接处置粮商,乃是被逼无奈,如果就此成为定制,迟早会越用越偏,韩冈只望能仅仅保持在粮食这等必需品上。

“市易法本有常平之意,本就是为了平抑京中物价而设。只是今次本金不足,以至奸商为乱。以现下的情形看来,立法度和加给市易务本金应当同时而行。”吕惠卿转头问曾布,“子宣,你看呢?”

曾布笑了笑:“说到市易务之事,还是要去问望之【吕嘉问】才对。”

“哪里的话,学士可是三司使!”韩冈摇头表示不同意。

“三司如何管得了市易务。”曾布冷淡回了一句。

“还是先问问酒水之事。市易务已经将酒药的价钱涨了五成。等几位回来,白马这边可是连酒都摆不起了。”韩冈心中的疑惑得到了答案,见着气氛有些不对,举起酒杯笑呵呵的敬了一轮。

互相敬了酒后,表面上还是一团和气。魏平真和方兴使尽浑身解数,尽量的让宴席上的气氛不至于冷场。

但此前曾布的说话和表现,可见他与吕惠卿嫌隙已深。两人不像同心同德的同志,而是各自异心的仇敌。方才曾布的话中,不无怨言。听口气好像吕惠卿侵夺了曾布的权力。连话语间都按捺不下这口气,看起来曾布和吕惠卿两人很可能快要撕破脸皮了。

‘是要争夺王安石留下的空缺吗?’

韩冈不是瞎子,王安石如今的危局一直都看在眼中。他不觉得他的岳父能支撑过去。如此大灾过去百年间当然是有过,宰相没有因此去位的情况也有。可在宰相本来就因施政而饱受争议,却正好碰上席卷半个国家的灾情的时候,要想稳坐相位,韩冈能找出的例子只有治平年间的韩琦!

韩琦韩稚圭,住在相州昼锦堂的那一位,治平年间是保扶英宗坐稳帝位的功臣,他虽然在濮议之中备受指责,又遇上了一场淹没了京城、且冲走了宫中上千军士的洪灾,但靠着定策拥立之功,没人能动摇到他的地位。

但韩琦的条件,王安石并不具备。他对赵顼的影响力,这两年一直在逐渐衰退中,也不比当初的韩琦——刚刚登基没多久的英宗,还要靠着这一位宰相在曹太后手中保住自己的位置。

以如今的现状,不论王安石怎么努力,想要安稳度过了这一场灾情带来的危局,几乎是一桩不可能的事。即便他处置了一干造成京中恐慌的粮商,但这场粮食危机也仅仅是序幕而已。

新法推行至今,王安石一开始预订实施的政策,差不多都已经出台。这个时候,赵顼还到底需不需要他,其实很多明眼人都能看得出来——曾吕之争,多半也缘于此。而且只要灾情还在继续,皇帝说不定也会有将其抛出来安抚民心的想法。

不知道王安石本人怎么想?

韩冈觉得他自己也该有自觉,眼下恋栈不去,可是会丢了卷土重来的机会。只是这话韩冈问不出口,向谁说都不合适。不过宴会后,王旁给了韩冈一封私信,一看封皮上的字迹,竟是王安石的。

王安石很少直接给韩冈写信,与韩冈联系多的是王雱。当着王旁的面,韩冈展开信笺。

一目十行的看过之后,韩冈也不得不承认,王安石能走到宰相的位置上,的确并非幸致。一般来说,看清别人很容易,看清自己却很难。王安石能正视自己的处境,比起韩冈冷眼旁观得出结论可要难得多。

这一封信,王安石已经隐隐透露出自己在宰相之位上坐不长久了。但关键是用什么形势去职,是因罪离任,还是功德圆满的自请出外,两种情况关系到新法会不会人亡政息,也关系到他能不能再次为相,由不得王安石不重视。

一切的关键还是在今次的大灾如何度过,问题还是落在河北流民上!

