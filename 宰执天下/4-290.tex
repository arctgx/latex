\section{第47章 天意分明启昌运(上)}

【说补更肯定会做到,夜里还有一更】

围绕着一间不大的宅院的火焰,数百兵士的喧嚣,取代了年节时的鞭炮。

明明就快要过年的时候,兴庆府中却没有半点的年节时应有的气氛。

大门敞开着,哭喊声从宅院中传来,很快,一名似是有些身份的中年人从宅院中被架了出来,按在门前的街道上。从门中跟着冲出了几名男女,但立刻又被拖了回去。一个士兵拿着大刀一挥而下,人头轱辘轱辘的滚了老远。

哭声更响亮了。

李清站在院子中,望着院墙上的一片红光,脸色木然。被抄家灭门的那一户官员,与他家只隔了两间宅院,平日里也时常见面,过年过节时,也是少不了人情往来。今天早上出门时还打了个招呼,谁想到转眼就成了刀下游魂。

牙关死死的咬住,手轻轻抖着。他恐惧,他害怕,邻居的命运随时能落到他的头上。今天是枢密院直学士,明天就是他夏州团练使。

“老爷。”随着声音,一只温软的手,握上了李清正在颤抖的拳头。

李清侧头,对上的眼眸,透着关切。这是他的妻子。

李清娶得是个小部族的女儿,但温婉的性格更像是汉女,并不似党项一族的女子。

“没事。”李清摇摇头,却是攥着妻子的手并不放开。

这是两个月来,第十一位被抄家论死的官员。如果从半年多前,翰林学士景询被杀开始算起,已经是第十三位了。

除了景询之外,被处置的都不是够资格站在紫宸殿上议事的高官,但无一不是身在实权位置的官员。

此外这些天,还有加入了班直的几个小部族族长和长老们的下一代,作为天子亲卫,莫名其妙的死在了宫廷中。

大夏不是宋国,朝廷内的争斗仅仅是以一方出外而告终。是跟契丹一样,从来都是用刀说话。胜者活,败者死,没有第三个结果。而在双方决出胜负之前,被漩涡卷进去的鱼虾不知还要死伤多少。

李清不想做下一个。但他投靠的梁家却至今没有大的动作,任凭国主继续拿着屠刀,一刀刀的砍杀梁氏在朝堂上的支持者。

第一次罗兀之役,梁乙埋虽胜尤败,回来后就在国中大杀一通,将反对者斩草除根。那时的狠辣眼下全然不见,让李清的心一天天的沉下去。

每一次上朝,李清都仿佛是在鬼门关前走过一遍。作为偏向梁家的汉臣,他自知随时都有可能落到那十三人同样的下场。唯一能自我安慰的,就是现在还没有杀到武将的头上。

朝中的武将各有各的后台,手上兵权在握,的确不易触动。之前秉常处置的也基本上都是文官。

能在西夏朝堂上担任文官,绝大多数都是汉人的身份,有很大一批是从陕西跑过来的士子,因为考不上进士,得不到官职,所以干脆一咬牙投奔西夏,以张元、吴昊以及景询为榜样,求一个富贵。

之前梁氏秉政,这些文官全都是匍匐在梁乙埋的脚边。现如今秉常亲政,也就将清洗的目标,先放在了他们身上。

十三个实权文官一去,朝中本就不多的文臣已经寥寥无几。

从院外传来的声音渐渐小了,碎乱的马蹄声却在门前不断掠过。

李清叹了一声,回头看着衣着单薄的妻子,“先回去吧,外面太冷了。”

看着妻子没有动,他又一笑,牵着手,往温暖的房中走去。

刚在火盆边坐下来,一杯热好的烧刀子已经递了过来。

接过热酒,李清看着虽不美貌但却贤惠无比的妻子,终于放开了紧皱的眉头。

“不用担心,不会有事的。”李清对妻子说着。也是对自己在说。

秉常真的疯了。

为了铲除梁氏,对契丹人奉承得太厉害。一年三万匹马、驼,如果卖给宋人,至少五十万贯的收入。

不但没能挽回梁氏主政时对宋国的颓势,反而为了借助契丹人的力量,将大量的牲畜送给辽国。拼命的讨好辽国的结果,是使西夏国中越来越贫困

国中对于刚刚亲政没有多久的这位新皇帝的期盼,在数月间已经沦入谷底。没有什么情况比现在更糟了。

秉常难道不知道这样做的后果?李清天天都能见到他,知道他绝不是蠢人。但秉常想控制朝堂,就必须下狠手铲除梁家的势力。一开始杀了景询这位谋主,就是他的宣告,

作为一名身居高位的将领,李清很清楚如今的国计是如何窘迫。已经到了难以为继的地步。军饷有两个月没有发了,李清更是有很久没在军饷中伸手,反而向外掏钱帮着没钱养家的麾下将士贴补家用。

身为大将的情况都如此窘迫,其他底层军官的情况只会更差。如果不能从宋人那里得到足够的收入,大白高国土崩瓦解,也就是转眼间的事。

……………………

虽然收到的情报,与潜伏于兴庆府的细作发出时有近一个月的延迟,但这并不影响赵顼推断出困扰大宋多年的西北死敌,正在为自己的棺材钉上钉子。

自从景询被诛之后,西夏朝堂的分裂已经不可避免,这一点显而易见。远在东京的赵顼,不用熟悉西夏内情的臣子向他解释,也能看得分明。

赵顼是天子,对西夏当今国主的心理,自是能体会上一二。换作是他处在秉常的位置上,一边是近在眼前,掌控了朝堂并压制自己多年的母族;一边是远在千里之外的敌人,会做出现在的选择,其实也不足为奇,只是行事的手段尚待商榷而已。

宋人不一定会攻打大夏;就算攻打大夏国,也不一定能打到兴庆府城下;即便打到兴庆府城下,还有辽国的岳父可以依仗。辽国能眼睁睁看着大夏国灭亡?所以秉常可以不去担心在横山边磨刀霍霍的大宋官军。

而梁家的势力就在身边,随时都可能让自己失去所有的一切,举目朝堂,全是之前紧紧跟随梁氏兄妹而被提拔上来的朝臣。在母后垂帘听政的时候,对自己全无一丝敬意,多年积怨,秉常哪里会继续忍耐下去?

这对赵顼来说是好事。尤其一年来,西夏接连派出使节,充分的向赵顼表示善意,并恳求大宋皇帝为两国百姓的安定生活着想,放弃进攻西夏的念头。这样的举动,充分满足了赵顼好大喜功的心理。

‘该备战还是备战,等准备好了就出兵。’赵顼在武英殿的偏殿中,绕着沙盘转着。

赵顼自不会是空谈仁义的宋襄公,更不会耽于虚名,谈判和备战两不误。这边谈,那边打,才是正常的事,要不然澶渊之盟怎么来的?城下之盟全都是打出来的,何况赵顼打算给党项人准备的前途,是灭国,而不是简单的称臣。

赵顼最想看到的就是西夏内乱,眼下西夏使臣的软弱也正合他的心意。

“官家,西夏贺正旦的使臣抵京了,正在都亭西驿中安歇。”李舜举带着消息回来了,“馆伴使正在接待他们,是否另外赐宴。”

“西夏的使臣没说别的?”赵顼从沙盘上抬起头。

“没有。”李舜举知道赵顼想听到什么回答,但西夏的使臣的确没有别的话,“应该只是来贺正旦的。不过贡物带了很多来。”

“外藩上贡,哪一次朝廷不是赐还价值相当的财物?带多带少又有什么区别?”官军年年胜绩,赵顼早已不将西夏放在眼里,“身为藩属,不修贡事。能给辽国一年三数万的牲畜,就给了朕五十匹马?!朕不想见他们,遣其出境。”

李舜举低头,没有接旨。

赵顼回身瞥了李舜举一眼:“去传元绛来。再看看知制诰谁人当值,一并传来。”

李舜举这下才应声,匆匆出了殿。他让赵顼看重的地方就在这里。如何对待西夏使臣是该直接吩咐给中书的宰执,他一个宦官当然不能越俎代庖的接旨。

元绛应诏上殿,吩咐一番过后,赵顼看看时间,便往庆寿宫去。

太皇太后曹氏,在八月的时候,生了一场大病,最近才稍稍好了些。只是身体越发的差了,赵顼晨昏定省,日日都来探问病情。

进了庆寿宫,只见曹氏半躺在榻上,看身上的衣服,是刚刚起来过。

“娘娘怎么起来了?”赵顼问着曹氏身边的人,“方才谁来过?”

“蜀国刚来过,现在去保慈宫了。”曹氏靠着迎枕,头发尽白,皱纹横生,比半年前要苍老了许多,“刚刚说了他家大哥儿种痘的事。”

赵顼在曹氏床边坐下来:“蜀国家的大哥儿也种过了?”

“就排在你二弟后面。”曹氏抬起眼,“京城里面,这些天来有几千人种过了痘。据说有人之后用痘疮病儿的痘浆抹了身子,都没有一个得病,看来的确是有神效。淑寿和六哥都不能再耽搁,拖一天就多一天风险。要是发了病,怎么都来不及了。”

这番话也只有曹太皇方便说,无论哪个嫔妃,乃至向皇后,都不敢拿着唯一的皇嗣冒险,替赵顼下决断。

