\section{第四章 力可回天安禁钟(上)}

天惟时求民主,乃大降显于成汤。

所谓民主,便是天下万民之主。《尚书》中的这一句,正是民主二字的缘起。

谁堪为民主?

殿上一时寂静,无人回应,只有赵颢嘶声竭力的叫喊。

极短暂的冷场之后,章敦立刻接口:“一年以来,天下士民皆知太后临朝,退北虏,安国事,有安邦定国之功。此乃朝中文武,天下军民所公认!先帝之崩,事出偶然,纵天子不得无罪,太后岂有罪过?违先帝之命,逆天下人心,荒悖如此,岂能听国政,立人主?”

章敦成功的连上了韩冈的质问,让他可以继续将话题延续下去。

“张参政!”

待章敦话声一落,韩冈立刻看向张璪。

张璪不敢怠慢,连忙声明立场,“臣张璪……太祖太宗…列祖列宗在上,臣张璪以全家性命为誓……”

在韩冈本人、王安石、韩绛、章敦四人接连立誓之后,殿中班直虽不能说全数反正,但也都不会再听从高滔滔和赵颢的吩咐。虽然他们每个人都清楚这已经等于是站在宰辅们的一边。但什么都不做,远比去冒风险做出些什么更简单。

现在张璪的立誓,就是只是在表态了。这个时候,容不得文武两班的重臣中还有人能保持中立。

韩冈却没去细听张璪结结巴巴的誓言,方才的冷场是怎么回事?

完全出乎他的意料,却让他对这场变乱的起因,终于有了答案。

韩冈偏过头,蔡确的尸骸还在远处,血水还在往外涌,地面上的红黑色渐渐扩大了范围。

难怪敢蔡确会参与进来,甚至成为主谋,难怪石得一、宋用臣会反叛,也难怪赵颢会有那么大的信心。

正如章敦所言,向太后自听政后,一切皆无可指摘。可韩冈方才问谁有资格为民主,赵颢之子自不能,但从章敦的话可知,就是他也都认为赵煦没有资格做皇帝——他攻击的是太皇太后妄立天子的行为,指斥的是叛贼们囚禁太后的举动,对于赵煦本人,则是‘天子不得无罪’。

自己知道事情很严重,可实际上,整件事比他想象得还要严重十倍、百倍。

韩冈微微苦笑,就算已经融入了这个时代,但观念上看来依然还有着很大的差距。也难怪当初劝说章敦时,他能点头也只是勉强。而蔡确,更是没有被自己的言辞所打动。

“……凡胁从者皆放其罪,只诛首恶数人……”

韩冈转头望着台陛之上,太皇太后愣愣的坐在那里,没有任何动静了。

高滔滔自幼生长在宫中,自然知道公然叛乱的下场,就是她能无事,最疼爱的儿子也必然没有好结果。她能参与到其中,也是对这场叛乱充满了信心——不,在她的心中,不是叛乱,而是顺天应人,拨乱反正。

这三纲五常,这还真的是天条一般。

如果是在千年之后,因尽孝而害了父亲的赵煦,反而能博得很多同情——为他必然要背负终生的罪。可是在现下,却是被世人认定是无可饶恕的重罪。

也许自己的坚持是错了,韩冈想着。他想将自己目标建立在人心的叛离上,却没想到对程度的错判,导致了最恶劣的形势。这一场叛乱,正是他坚持保留赵煦帝位的结果。

不过,事到如今,必须将错就错,坚持到底。

韩冈眼神转利,望着殿门处,那里已经聚集了不少守在殿外的班直禁卫。

他心中稍定,看来王中正并不在叛军之中。

守在殿外的班直,听到了殿中的变乱,便都赶了过来,但宰辅们接连立誓,却让他们大多放弃了支持叛乱,选择了中立。不过还是有几个冲了进来,但他们在众目睽睽之下,冲了几步,就犹犹豫豫的停下了脚步——他们毕竟心虚,听了宰辅的誓言,又都起了侥幸的心理。

会有如此可笑的情况,只会是因为群氓无首。若有其中有声望颇高的王中正领头,不至于如此。

不过有石得一领兵控制皇城,内部又有御龙直的韦四清,蔡确、曾布、薛向更是站在了太皇太后一边,这场政变想要成功,条件已经足够了,甚至绰绰有余。只要封锁皇城消息两个时辰,在京所有朝臣就会自己走进大庆殿,向太皇太后和新帝参拜,这真是太容易了。

如果不是自己能够将蔡确一击毙命,根本就不会任何反击的机会。

“……过往之罪皆不论,当下不从逆者即为有功,事后如有反复,天地共诛。”

张璪誓言刚落,不待韩冈点名,苏颂那边就跟着上去,“列祖列宗在上,臣苏颂以全家性命为誓……”

韩冈这时挪了两步,到了王安石身边,低声道:“岳父,须请速请郭枢密和张太尉率一部班直出殿。”

韩冈拿起铁骨朵击毙蔡确,不过眨几眼的功夫,再到现在几名宰辅接连发誓,也就两三分钟而已。围在殿外的叛军还没反应过来是正常的。

宫中自有规矩。五重禁卫也都是各有值守范围。

为了在朝会前不让朝臣们警惕起来,最外围的皇城司亲从官,不可能接手宽衣天武和诸班直的岗位。而且皇城司亲从官除了镇龘压宫中,还要严防有忠心向太后的宫人潜出宫城,也不能分心。

石得一纵知殿中有变,他要将局势扭转过来,不可能依靠人心不定的宽衣天武和诸班直,只有调来他手下最为亲信的队伍。

这就需要时间。

但石得一即使再耽搁,也不可能迟到哪里。

不能再拖了。

王安石会意点头,眼下在殿中的宰辅们,以他名位最高,威望也是无人可比。要指使郭逵和张守约,他远比韩冈合适的多。

韩冈总不能拿着铁骨朵去命令两名位于军中最高位的将帅,逼文臣发誓还是简单点。

“郭逵!张守约!”待苏颂誓声一落,王安石随即发话,点起了两名老帅,“吾恐禁卫诸军,尚不知殿中贼乱已平,你们和韩冈一起出去,晓谕众军。两班宰辅皆已立誓,从者皆放罪,只诛首恶石得一一人。”

韩冈只让王安石找郭逵和张守约一起出去,安抚军中,也让他们两人互相监视。这时候,事关军权,决不能有半点大意。但王安石又加了一个韩冈,这样的安排,更能让各方面更放心。殿内的局面,也不需要韩冈了。

“是。”韩冈点头,对身边的李信道,“让二大王闭嘴……别伤他性命。”

李信过来后,就守在了韩冈的身边。听了韩冈的吩咐,他却犹豫起来,“三哥……”

“没关系。”韩冈急急催促着,“别耽搁了。”

李信点点头,抬手就将御龙直的都虞候韦四清腿上的长剑拔起。这是他冲过来时顺手劈翻了一名班直,顺手抢到的,救了韩冈,也决定了成败,否则事情将败坏得无法想象。

腿上的长剑被拔起,韦四清啊的一声惨叫,痛醒了过来。李信抬脚一跺,又把他踹晕过去。左剑右刀,李信直接就上去了。

“杀了他。石得一,还不快来护驾!”

赵颢的嘶喊,在殿内群臣耳中已经只剩噪音。

“他也配姓赵。”王厚冲着赵颢啐了一口,转头道:“玉昆,我陪你去。”

韩冈摇摇头:“处道,你护着平章。”

说着便追着郭逵和张守约往殿门处跑去。

郭逵听到王安石的命令,没有犹豫,便立刻往殿门处过去。

而张守约多吼了一声,“还有份忠心,跟着老夫来!”

先是两三人,然后五六人,之后十几人,当耽搁了几句话的韩冈赶过来的时候,殿中大半班直都已经追随在郭逵和张守约的身后,就连殿外的班直禁卫,也几乎都投到了两人的麾下。

除了半只,还有十几名自觉勇武有力的武官,都是站在大庆殿上的,至少也是正从七品的诸司使,却一个个跟着出来,要争一份功劳。

韩冈走到门前,就看着张守约在点派人马,这里面他人头最熟,郭逵在旁边看着。突然又听到背后有人喊。

“韩相公,韩相公。”

什么时候韩绛过来了?

韩冈惊讶的回头,却见两人跑到了面前,然后被想要讨好韩冈的几名将领给拦住了。

其中一人正是被韩冈抢了武器的禁卫,他手上还有韩冈的官袍,与另一个拿着韩冈官帽和腰带的同伴,小跑着追了过来。

被拦在人群外,两人陪着笑,将韩冈的衣袍和帽子递过来:“韩相公,这衣服还是先穿上吧。”

韩冈接了过来,没有官袍也的确不像样。

就在大庆殿门口,韩冈穿戴起自己的衣服。只是久被人服侍,他连自己穿衣服都不那么顺手了。

那两位御龙骨朵子直的禁卫和几名将领见状,连忙帮着打下手,整理好韩冈的衣袍,戴上长脚幞头,围上腰带。

韩冈这边穿着衣服,而皇城司的人马这时候正赶了过来。

人数多达四五百,冲过大庆殿前的广场,直奔正门而来。

而冲在最前面的二三十人中,正有石得一的身影。

