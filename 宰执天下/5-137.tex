\section{第14章 霜蹄追风尝随骠(21)}

折克仁和折可大叔侄二人并没有太过关注营地内的纷乱。

那是一场注定不会有任何意外的战斗。

整个工程全面开始才半个月,但绝大部分参与工役的黑山党项苦力已经被繁重的工作折磨得不成人形。当然,隐藏身份混迹在这群黑山党项之中的契丹人也不会例外。

每天累死累活,半个月下来,连个休息都没有,吃得也不算饱,怨气的确积累了,但力气却还能有多少?若是从府州、麟州征发起来的民夫,还能得到一点照顾,可换作是黑山党项,那就尽情使唤了。就是折家,也是党项出身,但那点关系在几十年对抗西夏入侵之后,残留下来的只剩下憎恨。

气力早已耗尽来的苦力们,愤然一击或许还有着万军辟易的能力。但官军根本不与他们正面交锋,只是将其死死的困在苦力营中。周围的一片火焰,只留下一个缺口,而想从那条缺口出来,面对的却是官军神臂弓的封锁。

想来那些党项苦力在火起之前都没有去深思,为什么存放木料、草料,乃至存放通过屈野川运上来的麟州石炭的场地,都那么靠近苦力营。

火焰在眼中闪动,折可大和折克仁的脸上都映着晃动的红光。攒动的火苗,升到了半空高,这样的大火,两人可都从未见识过。

“幸好都是夯土,烧一烧反而能更硬。说起来还是石炭的火烧得旺。”折可大转头对折克仁道,“从麟州运来取暖烧火的石炭,果然是运对了。”

“神木寨出产的石炭比太原府和徐州的都好,用来炼铁肯定能出好铁,就这么烧了实在太可惜。”

“反正地理,刨开一层地皮,下面全都是石炭,比石头都多。”折可大转身望着北方沉浮如海的点点星火,“十六叔,小侄先去料理那些不开眼的辽人,这里就靠十六叔主持了。”

“小心一点,不要冲得太前。”折克仁叮嘱道。

“十六叔放心。”折可大咧开嘴,大笑道,“小侄会注意的。”

……………………

耶律罗汉奴正是志得意满的时候。

麾下的大军冲破了毫无防守的栅栏,再后面便是毫无所备的宋军营垒。只要穿过那道尚未修造完工的寨墙,正处在乱事中的宋军大营,便暴露在铁蹄之下。

当勇冠三军,精锐屈指可数的前锋越过营栅的时候,守在营垒中的宋人们还纠缠于猝然而起叛乱,甚至连一点反击或是抵抗都没有出现。

什么韩冈才智过人,什么的宋人不可轻辱,什么不可妄自进兵。耶律罗汉奴哈哈大笑,根本就是个笑话。

不过萧十三虽然对宋人畏之如虎,只敢做些下作的手段,又是个舔穷迭剌儿子脚丫子的废物,但这一次好歹还派对了人。

只是耶律罗汉奴笑声未已,却听到前方的一片惊呼,冲在最前面的一队人马突然间就矮了下去,不见了踪影。

“总管,宋人在栅栏后挖了一地的坑!专陷马脚。”前军派人赶回来报信。

宋人挖出来的陷马坑仅有海碗大小,只能陷住战马的四蹄,又没有遮掩,白天时一目了然,是用来迟滞骑兵的冲锋。正常情况下应该设在营栅外——在接近营栅前,耶律罗汉奴麾下的前锋兵马都是很小心的前进——可谁能想到营栅之内还会有这样的陷阱?

浑没想到在营地中还有陷阱的存在,正在兴头上耶律罗汉奴恨恨的磨起了牙,恨不得将主持修造这座营垒的宋将放在几颗大牙上磨碎嚼烂。直到前军派人回报说仅仅是最前面的百来骑中了陷阱,脸色才缓和了下来。

夜色将陷阱隐藏,冲在最前面的一排战马最远也没有冲出十丈,便全都被绊折了蹄子,背上的骑手也全都被抛了出去。幸而后续的骑兵没有跟得太近,加之人人骑术高超,却皆顺利的在陷阱前停下了脚步。

“小心一点!再中这样的陷阱,定斩不饶!”耶律罗汉奴呵斥着,让脸被吓白的小校回去传话。

可能是当真听到了耶律罗汉奴的吩咐,前军放慢了前进的速度,借用半轮上弦月洒在地面上的清辉,依靠超人的马术,轻巧的避开一个个小小的陷坑。

前军慢了下来,后面的兵马虽然没有挤上去,乱了队形。但前后彼此间的间隔,却几乎消失不见。一个接一个越过已经被砍倒的栅栏,向着正前方匍匐在地面上的黑影攻过去。

两丈髙的土墙从谷地东侧的山峰延伸到西侧的山峰,在月色下,如同蹲伏起来的巨兽。这是数千人半个月日以继夜不停劳作的成果。远未完工,但已经可以看到日后震慑百里方圆、抵御北面强敌的一座雄城的雏形。不过此时寨门还没有装设,只用一道活动的鹿角来挡着道路。

到了这时候,守城的宋军终于反应了过来。但出现在城墙上的,仅仅是百来人的阻击,从墙头射下来的箭矢,稀稀落落,宛如几片树叶落入河中,没有兴起半点涟漪。

而靠近了城墙,城寨内部的混乱则更为清晰地传入辽人的耳中。已经领着中军,接近到营栅前的耶律罗汉奴的笑声亦更为欢畅。只要再加把劲,就能攻入宋军精心打造的营垒,夺取开战以来最大的收获。

冲在最前的战士已经开始拿出马弓,与城头上的宋军开始对射。而堵在城门口的士兵则下马移动起沉重的鹿角。

多少人攥紧了手中的武器,马刀、长枪、铁鞭、骨朵,长短轻重,不同的兵器却被握得同样的紧。当鹿角被挪开,就是杀入城中的时候了。

从半空中传来几声重物破风的呼啸,数百人同时疑惑的仰起头,却立刻发现劈面就是一片落石如雨。砸中了额头,敲中了面门,击碎了鼻梁,打落了门牙,一片痛叫声响起,中心的位置,更是人人抱着头,与坐骑一起鬼哭狼嚎。一蓬蓬石弹劈头盖脸的不停歇的落下,惨叫声亦是不停歇的应和着。

“出了什么事?!”耶律罗汉奴在后面闻声大叫。但下一刻,从前方两侧的山坡上,亮起了十几点星光。这十几点星光赤红如火,在空中急速移动。就像火流星一般,从半空向着前军骑兵最密集的地方坠落。

是火油罐!

已经可以想象这些火油罐落到地上的惨状,耶律罗汉奴不由得闭上了眼睛。但那十几只燃烧的火油罐,于空中坠落时,已经在他的眼底留下一道道殷红的血线。

轰轰轰的十几声剧烈的鸣响,比之前凄厉十倍的惨叫声响了起来。隔着一重眼皮,耶律罗汉奴也能感觉到眼前一片赤红发亮。火油罐的释放出来的光和热,冲击到了百步之外的营栅前。

“总管,我们中埋伏了!”

“总管,这是宋人的计策!”

有人冲着耶律罗汉奴大声喊叫。

但耶律罗汉奴睁开眼后,却死死盯着前方的一片火海。在火光中,翻滚嘶嚎的全是他过去引以为豪的帐下勇士,就是被砍上一刀射上一箭都不会皱一下眉头。可是在浑身燃起的油火中,他们还是忍不下去剧烈的疼痛。

“总管,已经救不出来了。早点撤吧。”

“是啊,必须得撤了!迟了就来不及了,总管!”

看见主帅盯着前方中伏的同袍,更多的人苦苦哀求。

耶律罗汉奴犹豫着。若是这样回去,不说受人嘲笑了,萧十三少不了会乘火打劫。到时候,在黑山河间地立下的功劳不仅全都要抵消,就是那些战利品,也全都得砸进去来为自己脱罪。从没有吃过这样的亏,耶律罗汉奴如何甘心就此向萧十三那个小人低头服输?

眼下只是前军受困,出手反击的也仅是山头上的霹雳砲,宋人营垒中混乱依然。可见党项人的叛乱还没有被宋人镇压下去。他手上还有三四千兵马,还没有到山穷水尽的时候。

可就在耶律罗汉奴的犹豫中,他忽然间就觉得那里不对劲,然后才发现眼角余光捕捉到的火光多了许多,而周围也静了下来。

耶律罗汉奴瞪大了眼睛,左右回顾。便发现后侧方的山峦上,火光一片片的亮起,转瞬间,便照亮了整个山头。战鼓声从两峰山巅处响起,隆隆的如同天上的雷鸣。

果然中埋伏了!

耶律罗汉奴手脚冰冷,在马背上一阵摇晃。他终于可以确认,前方宋人城寨中的混乱,只是诱人上钩的饵料,而自家竟然硬是咬了钩子跳了进去。

对于可能出现的伏兵,耶律罗汉奴还是做好了应对的准备,中军外围还有拦子马护翼,但在前军成为陷阱中的猎物,全军的注意力都被吸引了过去,突然出现在身后的宋军,顺利的在辽军中造成了巨大混乱。

有人抓着耶律罗汉奴的缰绳:“总管,再不走就来不及了!”

“已经来不及了。”

身处高地的折克仁眯起眼睛,露出得意的笑容。

