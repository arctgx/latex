\section{第24章 缭垣斜压紫云低(六)}

从驿馆中出来,已是满天星斗。

迎面而来的料峭寒风,被驿馆内的香烛烟气熏得有点发昏的头脑转瞬间便为之一振。

‘今天算是解脱了。’韩冈暗自庆幸。

赵顼在城南驿逗留到两更天才起身回宫,他在正厅里与王安石说话,韩冈也不方便离开,只能在偏厅里候着。直到赵顼回宫,他才得以向已经很累的王安石和王旁告辞。

大大小小几个孩子早就沉沉的睡了过去,一个个被抱上了车。三辆车子,从前到后缓缓启动。车厢中无声无息,只有包铁的车轮碾压着地面。

韩冈陪在王旖所在的主车旁,骑着马向家中去。只隔了一重布帘,听见车厢里面传来了妻子的声音,“爹爹精神还好。”

“嗯,精神是不错。”

除了见老以外,王安石的精神状态比起之前最后一次见到他时,要强出许多。丧子之痛,已经看不到多少。在宰相之位上积累下来的疲累,也已经全都在金陵的山水中消散无踪。

“不知道娘怎么样?孤身留在金陵那边,实在让人不放心。”

“如果岳父在相州定下来,应该就会接岳母过去……而且岳母的身子骨只会比岳父好,不会比岳父差。”

韩冈的岳母可是个脾气极硬的人,又有洁癖,要不是逼着王安石时常换衣洗澡,以王安石的性格,个人卫生的情况只会更糟。

说起来韩冈的父母也是母亲那边更强势,不过同样很是和睦。对韩冈这个儿子也是关怀备至,一月一封的家书总是厚厚的如同一本书。韩冈发自心底里盼望他们能健康长寿。

回到家中,稍作收拾,就到了三更天,只能睡上一个时辰多一点的时间。

韩冈知道,天子亲临城南驿,必然会引发无数猜测,朝堂上的人心也会乱上一阵。事不关己的韩冈,倒是迫不及待的想看看会有什么样的发展。

……………………

“王禹玉是怎么了?”前脚出了政事堂的大门,后脚苏颂就忍不住问道。

郊祀大典就在两天后,皇城中,在各个衙门里面进进出出的官员一下多了起来。出任大礼使、礼仪使、卤簿使、仪仗使和桥道顿递使这五个大典临时差遣的几位大臣更是忙得脚不沾地。

担任大礼使的王珪在早朝之后,就接二连三的接见一应官员,再一次与他们确认各自在大典上的任务。

在礼仪性质的大典上,本已经成了虚衔的六部九寺的主官,却是有着与官职相对应的任务。分别掌管太常寺和光禄寺的韩冈、苏颂两人,也免不了要往政事堂去走一遭。

见到王珪之后,苏颂完全掩饰不住自己心中的惊讶。

众人面前的当朝宰相的脸色很是难看,心情恶劣是一方面,另一方面,气色也不对劲。双眼眼袋浮凸,泛着极明显的青黑色,整个人老态毕露。

尽管王珪个人能力在国朝历任宰相中被人说是从后往前数肯定能排前三,但他风姿仪态上的水准,在韩冈见过的重臣中,却是只有冯京能相提并论。富弼、文彦博这等名相都比不上他,更不用说衣服脏了都不知道要换的王安石了。

至宝丹这个评价,不仅仅局限在他金玉满堂的诗文上。十分注重仪表的王珪,每天总是光鲜得……如同一颗圆润光滑的至宝丹。简单地说,完全不像六十开外的样子。

认识了这么些年,韩冈都没见过王珪在仪容上有所疏忽,只有今天例外了。而看认识王珪更久的苏颂的表情,估计也是没见到过几次这样的王禹玉。

“多半是一夜没睡的缘故吧。”韩冈以袖掩口,打了一个哈欠,说道一夜没睡,其实他也是。

“是为了天子昨天去城南驿的事?”

“当然。”打过哈欠之后,韩冈却感觉更困了。强行忍住浓浓的倦意,他说道:“天子给家岳如此恩遇,王禹玉怎么可能睡得着?”

天子做客臣子家,都是难得的恩遇。何况亲自到驿站中做客?这是将王安石当诸葛亮来对待了。恩荣一时无两,自然在外人眼中怎么看都像是要复相的样子。

韩冈今天早上往太常寺衙门过来的时候,一路上遇到了十七八人在问昨夜城南驿中的内情。纵然对京城中流言传播之速早已知晓,但今天的这个消息穿得这般快,还是让韩冈吃了一惊。整件事才不过过去两三个时辰而已,就已经有不少人听到了传闻。

身为重臣中的一员,苏颂自然是其中之一,而且他还清楚韩冈也是当事人之一。侧过脸,看着倦色难掩的韩冈,“看玉昆你的样子。是不是也是一夜没睡?”

“天子不走,难道做臣子的还能自顾自的离开?”韩冈又是叹了一口气,“等到二更天后才解脱,到家都三更了。”

赵顼能打着斋戒的幌子,上午连政事堂都没去,估计是在补眠。可韩冈这个做臣子的就没有这等好事了,常朝不需要参加,但再怎么说也不能旷工。四舍五入,也才睡了两个时辰不到。韩冈纵然因为常年不懈的坚持锻炼而精力过人,但犯困依然难以避免。

苏颂闻言便会心一笑,难得能听到韩冈抱怨。

“做得过头了。”苏颂是难得站在新党一边,“若天子当真要让王介甫复相,这番恩遇也算不得什么……只是,看起来并不像是要对王介甫宣麻拜相的样子。”

“不是像不像的事,天子夜访城南驿,不过是宠遇老臣罢了,何曾说过要让家岳复相了?”
苏颂轻叹了一声:“还是因为前几天的事吧?”

“多半是。”

谁让王珪领着东西两府让天子下不了台的?直接将赵顼对北方的野心挡回去,是三旨相公难得一见的大胆举动,但由此惹怒了天子,当然会被敲打一番。

不过这话并没有说出来的必要,苏颂明白,韩冈也明白。

帝王心术本来就是要使得臣子因难以预料天子的心意而感到畏惧。不过只要能够从局中跳出来,像赵顼这般刻意,作为旁观者看着便是觉得好笑了。尽管当事人是很认真的在做。

王珪的相位建立在对天子的迎合上,与依靠个人能力而得到的地位截然不同。天子的喜怒,对两类臣子的意义也同样是截然不同。

身在局中,王珪一时间失魂落魄当然不出奇,只不过相对于‘不以物喜不以己悲’的境界,当然是差得远了。

苏颂和韩冈并肩走回太常寺衙门,韩冈只稍稍拖后了小半步,以示对年齿和资历皆在自身之上的苏颂的尊敬。一路上与不少朝臣擦肩而过,一个个都是忙忙碌碌的,只是当他们见到韩冈和苏颂并肩而行,都立刻闪到了路边,不敢与两人争路。

苏颂向迎面而来的官员们一个个行过礼,转头问着老神在在的韩冈:“玉昆,这一回要真的令岳复相又该怎么办?”

“新学、气学之争,如今是靠权位就能分出胜负的吗?”韩冈笑着反问,顺便向一名在路边行礼的将作监官员回了半礼。

苏颂摇摇头,当然不可能。

天子为了维持新学的地位,几次三番的出手偏帮。但最终也没有变成让他心满意足的局面。甚至可以说,新学在风雨中岌岌可危,而气学一直都在稳定的扩张中,王安石被任命主持殷墟发掘,正是证明了气学在学派之争上让新学狼狈不堪的现实。

“既然不是,那又有什么好担心的?”韩冈的笑容更加恬和,跟方才两人见到的王珪截然不同。

从功绩上,如今的天下大局,可以说是王安石一手主导而成。没有变法带来的西北拓张,韩冈也不会得到施展自己才华的机会。韩冈从来没有否认过王安石的功绩,纵然在学派上对立,但对王安石的敬重却是从来没有缺少过。

但在学术上,韩冈却绝不会退让半点。来自后世的眼光,让他绝不会认同王安石的主张。争斗将会是漫长的,而韩冈有信心笑道最后。

回到太常寺,依然是去《本草纲目》的编修局。

不过韩冈先行处理了一下衙门中公事,将国家卫生和医疗事务全数掌握在手中,当然比不上苏颂本职上的清闲。

大典上已经有了充分的安全保障——这主要是开封府职责范围,开封知府的桥道顿递使正是负责此事。而且在球赛后的那一次惨案之后,盲动的人群会带来什么样的悲剧,已经深深的印刻在许多人的脑海中,对于安全工作,这一回甚至到了苛刻地步。

但在紧急事故的预案中,医疗急救是个很重要的环节。几近十万人参与其中的典礼,谁也不能保证不会出一点意外——确切一点的说,肯定会出意外,有区别的,仅仅是大是小、是多是少的问题。为此韩冈已经将任务分派下去,让医官们去配合开封府的工作。

将最后一人打发了出去,韩冈忽然发现倦意不知什么时候已经消失无踪,整个人清醒得很,完全没有问题。

“大概是不用担心了。”韩冈对苏颂说道,“该做的准备都做好,出了什么事都能及时应对。”

