\section{第18章 秉烛待旦已忘眠(上)}

赵隆无心的插话正说到点子上,韩冈得他提醒,精神陡然一震:“攘外必先安内!若身后掣肘太多,如何能成就功业?开榷场,行市易,不为不美。唯秦州官吏、世家多有回易之事,若遽然而兴市易,断人财路,必惹众怒。当弹章交加而上,又有谁能安心开拓河湟?”

韩冈正正说到王厚的心结上,他双眉微皱,有些无奈。看了看韩冈,他欠起身虚心问道:“所以先要屯田?”

“比起市易一事,屯田便不算困难,秦州沿边地广人稀,只要见缝插针,在屯垦处筑堡而守,两三年内便有小成。通过屯田兵来震慑周边蕃部,打击悖逆之辈,再公平处断蕃汉纠纷,赐亲我汉家之蕃酋以官职,以收人心。使其为我用,而不为西夏所用。日后攻打西贼,他们也便是助力!”

韩冈说的安定边疆的方法,从古到今,一脉相承,也算不得什么独创的见解。但王厚已被韩冈前面的话所打动,不住的点头,只觉得眼前的韩秀才实是有大学问,大见识。

韩冈不再说屯田市易之事,能说的都说了,再深入说下去自己就要露底,话头一转,轻轻叹道:“不过关西早非胜地,出产已远不及汉唐,否则也不需辛辛苦苦的去屯田。多少上好的田地,都被黄河的流水冲掉了,而黄河也因此变成了黄色。这可不是好事!不仅关中良田尽丧,连天下都遭其患。”

韩冈说得郑重,王厚身子前倾,用心聆听。

“如黄河,一碗水,半碗沙,沙土皆是从关中而来。若是在潼关之前,黄河水流湍急,泥沙随水而流,但出了潼关之后,河水顿缓,其中所带泥沙便会沉积下来。”韩冈向王厚举起酒碗,没有过筛的浊酒中,许多酒糟随着酒碗的晃动而载浮载沉,‘绿蚁新醅酒’说得正是这种没有滤过的酒浆,“听说汴河便黄河水而泥沙淤积,必须年年清理河道,可即便如此,也是赶不上河底抬高的速度。”

王厚点头称是,他去过东京汴梁,也知道在汴河连接黄河的河口附近,堤内的纲船甚至比堤外房顶还高,都是因为黄河泥沙倒灌的缘故,为了疏浚汴河河道,每到冬天就要驱动大批民伕和厢军。汴河两岸的百姓,为此苦不堪言。

韩冈把酒碗放下,碗内的浊酒渐渐定下,而酒糟便沉到了碗底:“你看,只要水流轻缓起来,水中的沙土自然便沉淀下去了。欲治黄河水,先治黄河沙。欲治黄河沙,则得先从沙土来源着手。否则任凭你堆高河堤,掘深河底,也不过是治标不治本的应急手段,决堤改道也是或迟或早的事情。”

“韩兄说的正是。”听得韩冈说得通透,王厚不自觉的喝了口寡淡无味的浊酒,叹道,“庆历八年【西元1048年】六月,黄河在澶州商胡埽【今濮阳县】决口,改往北流,直入渤海。朝堂的相公们为了是填塞决口,还是顺势将河水导往北流,闹了几年也没见分晓,后来勉强行事,也没成功。

到了嘉佑五年【西元1060年】,大名府魏县第六埽决堤,分出一条支流,由笃马河向东入海。黄河经由东流与原来的北流同时入海,号为二股河。黄河一分为二,是堵是疏,还是任其流淌,从仁宗朝吵到了现在。富、韩、文几位相公,没少在廷上争辩过。

还有梁山泊!八百里水面又由何而来?还不是后晋开元元年【西元944年】黄河在滑州决口,水淹曹、单、濮、郓诸州,洪水积蓄在巨野,巨野泽才变成了梁山泊。”

“听说几个月前,黄河好像又改道了?”赵隆插话问道。

“没错。就在八月,北流填塞失败,许家港河决。水泛大名、恩、德、沧、永静五军州。淹死军民数以万计。”王厚长长叹了一声,“为了这条河,不知费了多少钱,也不知死了多少人,但终究无法根治。”

韩冈低头抿了一口酒。只看王厚这一段议论,绝对是在河防上下了苦功。韩冈自知在黄河水利等细节上,他是肯定不如深有研究的王厚。不能再往细处谈,韩冈把话题拉回到自己擅长的水土流失上:“这就是泥沙过多的危害所在,南方雨水十倍于北方,而长江水势自是远过黄河,为何长江少有决堤?还不是长江沙少,黄河沙多的缘故。砍了太多树木,山上没有草木固土,雨水一来便会泥沙俱下。看看泾水之清,再比一比渭水之浊,是何故方有泾渭分明之语?”

“泾原树多,可以固土,而渭河自伏羌往上,全是光山。”王舜臣抢答道,韩冈说得深入浅出,他也能听的懂,想得透。

“说得好!”王厚抬手敬了王舜臣一碗酒。王舜臣哈哈一笑,很洒脱的接下了饮了。

“王军将虽然年轻,却在关西走得多了,各地地理了解得不少!武艺也是过人一等,连珠箭术更是一绝。”韩冈拍着王舜臣的肩膀,向王厚介绍了一下,几句话便让王舜臣感激涕零。

屋中三人越听越是入神,此时少有人能把黄河水患从根源处说得如此明白。韩冈说得一时兴起,一把扫开桌面的杂物,用手指蘸着酒水,就在光桌上点画起来。先一笔画出了一个尾部上拖的‘几’字形。韩冈指着道:“这就是黄河!”

穿越千年,真正有用的是什么?是对江山地理的认识!——至少对韩冈现在来说,的确如此。

一本千年后只值十几块钱的地图册,放到千年之前,莫说千金,万金亦可换。那可是动员了千百万人次的测绘工程和各种先进仪器所绘制出来的地图,不是等闲可比。

韩冈历史并不好,对日后的历史细节发展懵然无知,但他对于地理学上的认识却十分的出色。加上他的口才,就算千年的时间,导致对地名的了解有所偏差,可要蒙过王厚这毛头小子,却是轻而易举,不费吹灰之力。

单是这一笔‘几’字,就已经让王厚更加佩服韩冈。不看过大量的地学书籍,并仔细推演过江山地理,这世上有几个知道大江黄河流向的?世所流传的《水经注》上,可从没天下舆图这一页。王厚能了解到黄河、长江的大致走向,还是沾了父亲王韶的光,从渭州知州兼泾原路经略使的蔡挺那里,见识过复制自崇政殿中张挂的天下舆图。

“黄河是这个样子?”王舜臣和赵隆也都好奇的看着桌面,他们虽然都看过黄河,也天天喝着黄河支流的水。但让他们将黄河说出个一二三来,绝对是两眼一抹黑,支吾半天也不定能迸出个字来。

“对!正是如此!”王厚帮韩冈证明,他在‘几’字的右下方点了一点,“这里就是东京。”

“这里就是东京啊……”王舜臣和赵隆专心的点着头,却不知他们到底有没有听懂。

有了千年之隔,具体的地理名词有许多都发生了变化。韩冈说不定在地名上还不如王厚,但大的区域韩冈凭着前身的记忆,互相印证过后,却也熟悉了下来。他指着‘几’字右边一竖的右侧空处,“这是河东【今山西】。因为位于黄河东侧,所以有河东之名!”

手指再从河东往上推,停在‘几’字头上一横处,王厚立刻道:“是契丹的西京道。”

韩冈又蘸了点酒水,横着一拖,把‘几’字下面的开口几乎封起,“这是渭水。而我们现在就在……”

话声轻轻一顿,王厚便聪明的在代表渭河的一横下点了一下,沉声道:“伏羌城。”

“而西贼就在这里。”韩冈指着被渭河和黄河括起的一片土地,“这一片地,被黄河三面环绕,形如布套

。故而我称之为河套!”

“河套!?”王厚重复着。他在嘴里喃喃念了几声,仿佛在咀嚼着词义。最后他才重重的点头,“起得好,起得好,的确像个口袋,正是套子的样子。”

韩冈直起腰,双臂夸张的张开,放声道:“黄河百害,唯利一套。党项人占着此处,兴灵【注1】一带水网交织,直如一塞上江南,不论耕种还是放牧,都是远胜他地。而兴灵之外,又有瀚海阻隔,使外敌难侵,此天险尤甚长江,广如渊海。要想直捣西人老巢,先要考虑如何穿过七百里瀚海,还要考虑如何保证粮道畅通,否则便有全军覆没的危险。”

王厚接口道,“从河东、鄜延、环庆几路往攻西贼,必定要受阻于瀚海。若从秦凤、泾原向北仰攻,又有天都山和兜岭阻隔。就算诸路同时出击,只要凭借天险,西贼将兵力分散亦能防守得住。但若是在更西一侧,比如兰州,放上一支奇兵,却能让西贼首尾难顾。”

“兰州?那是西贼占着的罢?”赵隆问道。

注1:兴庆府,灵州,即现在的银川、吴忠。

ps:言语的组织比实际内容更有用,许多演讲乍听来十分出色,但事后细细一想,也不过是些陈词滥调。

今天第二更,求红票,收藏

推荐一下三水的九霄天帝,书号69321:

这是一个吞云吐雾、炼气修行者为尊的世界

这是一个强者如林的世界,其中强者肆意妄为,弱者逆来顺受。

少年方兴,与炼气修行无缘,郁郁不得志。

一具水中女尸,让他的穿越者灵魂在沉睡百年之后终于觉醒,从此他踏上了成为至高者的永恒之路。

