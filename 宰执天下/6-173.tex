\section{第18章 霁月虚明自知寒(中)}

从车上下来的时候,司马康有一种天旋地转的感觉,耳朵里,还有一阵阵咔哒咔哒的响声。

“公子!”随行的伴当连忙上前扶住司马康,“没事吧?”

“没事。”司马康轻轻推开伴当的手,站直了身子,环视周围。

拉车的十几匹挽马满身是汗,在车厢中的旅客尽数下车之后,便被人赶着从站台旁继续往前,拖着车厢进了前方的一处厂房内,而那座仓房中,又有一列马车驶出,停在了对面的站台上。

身侧行人川流不息,有挑着担子的货郎,也有摇着扇子的书生。有拖儿带女的家庭,也有孤身上路的旅人。站内的役工在下面检查铁轨。几名手臂上套着警察袖标的士兵,手持短棍,在站中来回巡视——这是铁路警察,新成立的厢军。

人流汹涌,仿佛街市。

而站台一旁,高高架着巨大的牌子,远近可见——

东京车站。

尽管知道脚下就是开封府的土地,可亲眼看到牌子之后,司马康仍忍不住心中的震惊。

才一天,他就已经从西京洛阳抵达东京开封了。

从偃师一路坐车到此,上车时是七月初八的卯时初,下车时则是七月初九的辰时中,一天多一点的时间。这速度快得惊人,甚至要超过过去的急脚递——急脚递尽管也是昼夜不歇,可也做不到昼夜同速。

自洛阳出城,到抵达偃师,就用了司马康一天的时间,而从偃师到开封,七八倍的距离,时间却是完全相同。要不是洛阳到开封的四五百里轨道还未完全贯通,靠近洛阳的几条跨河大桥尚未修好,有轨马车只能从开封走到偃师,可能用时更短。

洛阳有个好处,就是有关塞险要。这就是为什么太祖皇帝始终想要将,但换成现在要修路,就是让人头疼的一件事了。相形之下,开封无险可守,四塞平野,在如今反倒是一件好事。

车站内人来人往,一间间商铺也生意兴隆,转过脸去,还能看到几个身穿绿袍的小吏,提着一袋袋的口袋往另一个铺子去,袋子上写了邮包二字,而铺子上方的牌匾则是邮局二字。

是通过邮车送来的信件,在偃师上车时,司马康也看到了这些邮包。看那些袋子的数量,可知其中的信件是成百上千。

也就是东京城会如此。司马康想着。

这世上,会离家远走的人并不多。这个时代,绝大部分人的交往范围不会超过百里,想要给亲友送条消息,只要多走几步路就可以了。有钱的让仆人走,没钱有时间的自己走,没钱没时间的,还可以托人帮忙。

让传递军情的驿传改送民信,希望从这其中收钱。不说朝廷面上无光,也是一件空耗国力的蠢事,且一旦边境有警,被这些邮件拖累,又怎么将警信传回京城?

可惜这世上,总是鼠目寸光之辈居于朝堂。

不过司马康完全没有买东西的心情,更没时间多感慨。

“公子,下面怎么做,去太医局吗?”伴当问着。

“这边是戴楼门,出去后租两匹马,我们进城去。”

司马康说着,主仆二人脚步匆匆的沿着路标向出口走过去。

车站里面,到处都有路标,出口,入口,公共厕所,急救站,还有写得到处都是的‘严禁逃票’,‘随地解手、罚款一陌’,‘禁止喧闹’,‘禁止嬉戏’,‘小心财物’等告示。

到底有几人能看得明白这些字?

司马康冷淡的想着,脚步更快的往出口走去。

出口处人流慢了下来,上百人拥堵在门前。司马康见状,眉头就皱了起来。

老练的伴当立刻上前去,推开前面的人群,嘴里喊着:“借过!借过!有急事,别挡着!”

司马康就跟在伴当后面,轻松的向前走。已经可以看见门外,从门前向外望出去,远远地能看得见北面新垒了砖石的开封城墙,还有新增筑的炮垒,已经不是司马康记忆中的用夯土铸成的城墙。

‘江山在德不在险。外敌当真能打到这里,这些炮垒又有什么用?’

司马康还记得老父当年听说朝廷又要大耗人工去修京师城墙的时候所说的话,但一声呵斥打断了他的回忆。

“你们干什么!到后面排队去!”

守在出口前的吏人指着伴当和司马康,很不高兴的样子。

“看你的衣装,也是读书人。怎么这巴掌大的字都看不懂?”那吏人呵斥着。

他旁边的警察用手中的短棍啪啪的打着墙上的字条——请有序排队。

“出战要查票,你们不排队怎么查?还是说你们想趁乱逃票?”警察的短棍挪向了墙上的另一张标语,“逃票须补票,违者解官。若没买票赶快去补,否则三十大板少不了,该付的票钱也别想逃。”

伴当当即大怒,尖声叫道:“我家公子乃是官人,尔等岂敢无礼!”

司马康没有考进士,但他靠了父亲司马光的身份,还是得到了一个荫官。

“官人?”查票的吏人看了一下司马康的模样,犹疑起来,“官人该坐官车,今天从偃师过来的官车不是这一趟!”

司马康耐下性子,忍下了这等粗鄙小人的冒犯:“有急事,先买了最早的票。”

吏人随即指着前面,“官人走错了,这里是平民百姓的出口。官人要出站,请去前面的大门,那边是官人专用的出口,出去后还有官中的车马,直接送去驿站里。”

警察跟着加了一句,“只要有告身就行。”

司马康脸色难看了,“出来的匆忙,没带告身。”

“不是匆忙吧。”小吏冷笑起来,盯着主仆二人空空如也的双手,视线变得锐利起来,“你们的行礼呢?”

周围的旅客都是大包小包,可司马康主仆却只有一只褡裢,形象太过特别。

警察用短棍拍打着手心,笑容与旁边的小吏一样的冷冽,“总有一些作奸犯科的,看到出站检查得严密,便把会暴露身份的行囊给丢了。你们不是第一个了!”

“无礼!我家老爷可是礼部侍郎!”

警察脸上的冷笑已经变成了狞笑。

生长在皇城脚下,京城人对官阶高低最为注重。侍郎是本官官阶,能做到这一级,都几乎是宰执了。但他们不知道,这是司马光上交《资治通鉴》后,朝廷给予的赏赐。

“来人啊。”小吏的喝声与警察嘴里的木笛同时响了起来,“把这两个贼人给我抓起来!”

七八名警察随即扑向了司马康主仆。

……………………

“两个宰相同编,十年弄不出一部《本草纲目》。这个速度快赶上司马十二了。”

“想不到玉昆你也听到了。”

政事堂中,两位宰相正对坐着喝茶聊天,处理了今日的公务,苏颂和韩冈总会设法抽出一点闲空来,聊聊天,或是说一说格物之道的最新发展。

“怎么可能听不到。”韩冈叹着,“范纯仁前回为《资治通鉴》上书,几乎就是指着鼻子骂了。”

“这事可不怪老夫。谁让玉昆你的心思都放在《幼学琼林》上?”

“《本草纲目》为先帝所托,不可不慎。《幼学琼林》就简单多了,都不用动脑,每天修改几笔,只当休息了。”

“哪里简单了?”苏颂笑着摇头。

《幼学琼林》属于蒙书一类,提供给小学生阅读。但作为实质上的科普读物,韩冈更希望天下士人都能来读一读。除了解试,日后的铨试,他也不会放过。

考中进士与诸科后,释褐注官,还要过身言书判一关,正是授职,也还有铨试。这些考试,都可以是逼迫士人去学习自然常识的大好良机。

韩冈苦心积虑要推重气学,怎么可能有太多的精力放在《本草纲目》上。

“相公。”一名吏员匆匆奔进厅中,打断了两人的对话。

“什么事?”苏颂问道。

“司马光的儿子因为在站内闹事,给铁路警察收押了!”

“司马光的儿子?”

韩冈想了一下,他对此人有些印象,好像是叫做司马康的。但又不是司马光闹事,司马康闹事至于要惊动两位宰相?

“司马光病重,他是赶来京城求医的,但在出站时被小吏给耽搁了。”

韩冈和苏颂脸色同时变了,对视了一眼,苏颂问道,“为何不报请河南府发急报?司马君实就这么一个儿子吧?”

“大概是来不及。”韩冈道:“过去有马递,自己上路无论如何都比不上,如今可是有铁路了。”

两天之内抵达京城,顺利的话,再有两天便能回去。而通过官府转呈,则时日久长,说不定消息还没送到,人就已经不在了。

“洛阳的医院,也有御医主持。”苏颂皱着眉,难道司马光已经快要不行了吗?

“想来,总是觉得他们比不上京师太医局里的医官。”

韩冈冷声道,在后世,病人和家属肯定也是更相信大医院而不是社区里的小医院。但管理西京医学院的医官,也都是太医局中顶尖的名医。他们治不了,京里的医生也同样治不了。

“这也没办法。”苏颂叹着,“人之常情。”

“去放他出来吧。子容兄,你……”

韩冈回头去跟苏颂说话,苏颂已经站起身来,

“我这就去见太后。玉昆,你去安排御医去洛阳。”

苏颂随即起身远去,只留下了几声叹息声。