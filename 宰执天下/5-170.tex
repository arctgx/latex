\section{第19章 此际风生翻离坎(中)}

画下密押,拿起略嫌沉重的金玺,在高一尺三寸、长两尺的黄纸上盖下一方九叠篆‘诏之宝’的鲜红印记,诏禁天下民间私藏千里镜的诏,便被宋用臣恭恭敬敬的捧了起来。

片刻之后,这封诏便会发去政事堂,如果没有被几名宰辅给驳回——其实在诏起草前,就已经确认过了政事堂中宰辅们的意见——那么从明日开始,千里镜便归入擅兴律第八条的管辖范围中。

依照这一律条,除弓、箭、刀、盾、短矛之外的兵器,皆为刑律禁私藏之列,千里镜将归入其中。私藏禁兵器者,徒一年半。从今而后,若是从谁家搜检出这一千里镜,至少就是一年半的徒刑。

当然,赵顼还是给了人悔过的时间,诏发布后的一个月之内,允许上缴或自行处分——其实这也是依照擅兴律第八条的律条:‘即得阑遗,过三十日不送官者,同私有法’。尽管是律条,也不会太过不近人情。

将千里镜算成是禁兵器中一条,赵顼并不觉得能从此禁绝民间私有。诏发出去没人理会的,也不是没有过。

千里镜是军国之器,但由于韩冈很早就将透镜的原理说得明明白白,加上几年的传播,天下间能够磨制镜片的工匠不知凡几。将千里镜拆开来,无论是凸透镜还是凹透镜,说是给人做眼镜,能说是禁物吗?甚至干脆安在支架上,说是不太好用的显微镜,一样没办法计较。

这样的情况下,总不能禁眼镜和显微镜?眼镜就不用说了,就是显微镜也很难禁止。病毒之说,已经深入人心。天下不知有多少人设法用显微镜来钻研医术,希望找到除痘疮以外的其他疾疫的治疗方法。这对所有人都是有好处的,与千里镜完全不同。若是赵顼敢下诏连显微镜一并诏禁,就是在政事堂中主持政事的是王珪和蔡确,也一样会将诏封驳回来。

另外依照律条,除非是甲胄或弩,其他禁兵器,只要拥有的不是完整的兵器,也不会论罪——也就是‘勿论’。这其实也是一个变通的手段。

枢密院的章惇也在说千里镜根本禁不掉,是徒劳之举。并拿历法做例子,诏禁千里镜唯一的作用,就是十年后,让辽人千里镜比中国的更好。赵顼猜测,这多半是从韩冈那里得到的借口。

但这是态度问题。通过这份诏,赵顼表明了自己不会坐视韩冈动摇新学地位的态度。而且历法一事并不能作为论据,军器监和将作监的风气,远比人浮于事的司天监要好得多。板甲、斩马刀、神臂弓为首的诸般兵器,不断得到改进,这让赵顼对禁军日后装备更好的千里镜深具信心。

诏被送去了政事堂,赵顼抽出了御案上最上面的一份奏章打开。眼神随即一变,这是开封府呈上来的灾情报告。

……………………

前几日一场突如其来的暴雨让东京城中的房屋倒塌了四百四十余间,而整个开封府地界的二十县中,则有数千之数,一时间无家可归的灾民当不会少于三万,亟需朝廷的援助。

韩冈并没有因为修纂《本草纲目》而耽搁到该做的正事,他主掌的厚生司在这场暴雨带来的灾害中表现得十分出色。
当这一场暴雨已经确认成为灾害,他就开始与开封府联系。两个衙门相互配合,在受灾各县中,为无家可归的灾民设立安置营地,暴雨方才结束,物资和人员便已经全都准备完成。光是为了防止伤寒、痢疾等疫症传播,而用来给灾民烧煮熟水的石炭,就划拨了两万石之多。

检查过派去开封府各县防治疾疫传回来的消息,绝大部分的灾民都已经返回家园,仍都留在营地的已是为数寥寥。韩冈基本上可以确定,这一场洪灾及后续的问题,当是平安渡过了。

不过在安置灾民的过程中,还是有不少问题需要改正。开封府当地衙门的问题,韩冈不便插手,甚是多说几句都不方便。但属于厚生司的问题,韩冈却不能置之不理。

最大的问题还是人员不足。开封府各县的灾民防疫工作,都是依靠从京城中派出去的医工,县中的人力没有用起来,药材储备也不足。这一次是因为靠着京城近,所以才表现得很完美,但如果换作是外路的州县,厚生司就鞭长莫及了。

幸而厚生司辖下有着遍布天下州县的保赤局,让保赤局中的人员专司种痘,其实很是浪费人力。里面的医疗资源,韩冈没打算放过,至少要加快培养他们应对灾疫的能力。还有在各县悬壶的医生们,也应该纳入厚生司的管辖范围,以便在灾害时能派上用场。

不过这件事,不是一时一日之功,而且还要经过赵顼和政事堂那一关,还是准备充分一点比较好,省得给人挑出错来。现在想找自己麻烦可不是一个两个,韩冈听到的类似风声可是充斥于耳,连日不绝,似乎真的成了众矢之的了。

“玉昆。”

韩冈正在考虑该怎么准备给天子和政事堂的札子,吴衍在外通报后走了进来。

见是吴衍,韩冈站起来相迎,见吴衍脸色不对,问道:“出了何事?”

“诏命……”吴衍说话的时候有几分迟疑,“……禁私藏千里镜的诏命发出来了。”

听到这个消息,韩冈轻笑了起来,“终于还是出来了。”

“玉昆,没关系吗?”吴衍微皱着眉,为韩冈担心着。

“当然不是……不过有关系也没办法啊。”韩冈状似无奈的摇着头。

虽然韩冈对这份诏命并不是很在意,但家里的望远镜,过几天还是得上交过去。

千里镜很是贵重,最普通的型号市价都要三十余贯。这是开封府地界中三五亩上等良田的价格,或是城外一间不临正街的两层小楼的价格。

同样数目的成本,能换来两套精锻板甲,五具神臂弓,六柄半斩马刀,或是一匹上等战马——由于西夏覆灭,官军夺取了西夏的半壁江山,战马的价格也降低了许多,原本最贵的时候,肩高四尺六寸顶格的战马,其收购价能有百贯之多,而现在就只有三分之一。

不过对于京城中的富户豪门来说,就是一两百贯也不算什么大不了的,拥有千里镜的为数甚多。民间私藏的千里镜上缴官府,只是在京城中,怕不有会近千具,这些也可算是真金白银了。说起来,这一封诏命还真是过分了。

处理了今天的公务,将剩下的事务交托给吴衍,韩冈照例赶去太常寺。

太常寺中的属僚们在面对韩冈这位主官时恭谨如初,远远地就开始行礼,但当韩冈经过之后,背后似乎就能感觉到手下官吏们投来的好奇视线。

朝廷要禁绝私家拥有千里镜,天子还没有正式下达诏令,但这件事已经在京城中传得沸沸扬扬。太常寺是个清闲的衙门,可越是这样的衙门,似乎对传播小道消息,好像就越发的严厉起来。

尽管千里镜不是韩冈发明,也没有遭到朝廷废弃。但从天子先是让《字说》上了经筵,又将开始将千里镜归入禁兵器的行列,朝堂中只要消息稍稍灵通的官员,就知道天子这是对韩冈借编纂《本草纲目》之机与新学相争表示不满。

而今天诏书终于颁布,多少天来的猜测被确认,这样一来,韩冈的立场也越发的微妙起来。跟随着他一起编修药典的助手们,当然也会受到同样的影响。

太常寺中没有太大的变化,但成了天子警告的对象,《本草纲目》编修局中的气氛就大变样了。韩冈走进编修局小院后,便是一派暮气沉沉的感觉,比之无所事事的太常寺,都还要阴沉上几分。

而且当韩冈检查今天,便发现比起昨天的进度来,今天的工作没与半点进展。不过前些天当千里镜可能被禁开始,编修局中的效率就已经是一日慢过一日,到了今天这一步,于韩冈来说,却也不是什么值得让人惊讶的一件事。

韩冈的助手林亿和高保衡在旁陪着小心,小心翼翼的关注着韩冈的反应,深怕韩冈会就此突然发作,一旦真的变成了这样的局面,谁都没有自信能拦住韩冈。

但出乎意料的。韩冈对此似乎毫不在意,对变得效率低下的《本草纲目》编修局完全没有什么反应。进了屋后,就坐下来,照常继续昨日未竟的工作。

“玉昆,你倒是严子陵稳坐钓鱼台。”听到韩冈的消息,早一步来到局中的苏颂,便直接走了过来串门。

韩冈放下手中的笔,起身迎接苏颂,笑道:“看起来像这样吗?”

“怎么不像?如此沉稳,世间可是少有得很。”苏颂叹了一口气,坐了下来:“国子监中现在都是一片批驳气学的声音。玉昆,风向变了。”

