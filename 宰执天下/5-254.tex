\section{第28章 官近青云与天通(21)}

蔡京结束了朝堂上的喧哗,使得朝会能够重新继续下去。

在御史台已经确定要集体出外的现在,蔡京肯定是要继续向上走了。

不过今天的事并不算完,崇政殿那边才是决定司马光此次上囘京的最后结局。

当然,朝会上上演的这一幕活剧之后,也没人能认为司马光还能翻身。最多也只会给他一个体面。即便是皇帝还想维护平衡,也没办法保住司马光。谁让韩冈和司马光彻底撕破脸皮,没有留下任何转圜的余地。非此即彼,赵顼若是敢偏袒司马光

两府宰执要参加崇政殿议事,其他朝臣则不需要。

可当朝会照着正常的流程结束后,韩冈回到太常寺,从宫中来的一名内侍却也追到了衙中。说是崇政殿再坐改在午后,让韩冈依时与会。

这名叫杨戬的小黄门离开,韩冈想了一阵,摇摇头,不知道皇后有没有通知司马十二。

如司马光这样外地上囘京的重臣,在朝会之后,天子肯定是要抽时间在崇政殿问对。

一方面体现对重臣的看重,另一方面也要藉机了解一下地方上的详情,并征询重臣对当前朝局和政令的看法。兼听则明,偏信则暗,中主以上,都知道该这么做。

若皇后根本就不遣人通知司马光,那么这位太子太师就彻底没戏了,完完全全失去了皇后的信任。

…………………………

定下了午后再去崇政殿,向皇后先回了福宁殿中探视赵顼的病情。

赵顼此时沉睡未醒,但脸色看着还不错,让她放心了一些。从内殿中囘出来,向皇后在御书房囘中坐下一开口就是司马光:“给司马宫师送些药过去,过几天就让他回洛阳!”

“奴婢知道了。”以司马光今天在殿上的表现,只让他回洛阳已经是很宽厚的待遇了,赐些药物也算是让这位老臣面子上能过得去,宋用臣应声后低头又问:“圣人,给司马宫师送什么药?”

“韩学士之前给雍王开的是什么药方?”向皇后冷着脸说道,“就给司马光送一模一样的过去!”

宋用臣没动弹,他的脚沉得像灌了铅。皇后竟然对司马光厌弃到了这一步!

“怎么?没听明白?!”向皇后见使唤不动宋用臣,顿时柳眉倒竖,声调高了八度。今天在殿上,她已经受够了大臣们的气,想不到现在连一名阉人都使唤不动了。

宋用臣连忙跪了下来,脸贴着地上的金砖,俯首帖耳,急声道:“圣人!好歹也要顾全一下太子的体面吧。司马光是太子太师。他本人虽不足论,但如此待遇东宫之师,传将出去,岂不是让世人觉得太子不尊师道?实是有累太子名声啊!”

向皇后冷眼瞪着宋用臣。背后传来的莫名刺痛让这名大貂珰汗水湿透了衣衫。

不知过了多久,终于,向皇后缓了口:“那就照规矩来好了,寻常给几位相公赐的什么药,就给司马光送什么药去。”

“奴婢明白,奴婢明白!”宋用臣连忙起身,倒退着出了殿。

宋用臣退下去了。向皇后仍是面如寒霜,犹自怏怏不快。喝了一口饮子,稍稍压住了心头火,又想起了在殿上的事。随即点起了勾当皇城司的石得一:“石得一,王中正现在在哪里?”

石得一道:“王观察现在应该在会通门那边。”

“速去找他过来!”向皇后有事要问问王中正。

王中正正在禁中宫城的南大门会通门处镇守。

这些天来,他身为带御器械,与三衙中几位太尉配合着一同谨守宫掖。仗着不弱的名声和观察使一级的地位,让手底下的班直和禁军一个个老老实实,没有起来闹囘事。

此时朝会上交锋已经传到了他的耳朵里,局势将会由此发生什么样的变化,到了这一步也算是看得分明了。

所以当一名小黄门带着皇后懿旨来传召的时候,他便气定神闲的起身,然后又脚步轻快的往福宁殿处赶过去。


能跟垂帘的皇后多接触,是王中正梦寐以求的事。可惜他官品已高,不方便常留于宫中。

内侍的官阶在升到从八品的内东头供奉官之后,便到了顶。之后想要再往上升,只能转入武官序列——这其实也是为什么开国以来内侍往往能名正言顺的领兵上阵的法律依据。

不过由此一来,控制内侍升迁的权力,便转入了政事堂和枢密院的手中。所以本朝的内侍不能为患的缘故便在此处——地位不高时,可以由宫中掌控,可地位一旦升格,便要受外廷牵制——这也是为了避免唐时阉人废立天子的局面。

王中正都已经是正任的观察使,若是现在就死了,越两级追授节度使都有一半的可能。自然,王中正肯定不会拿性命去换一个节度使,他还想安安然然享受荣华富贵。当日后再有战事,他这位内侍中的第一名将,也肯定是要为君分忧的。

只不过一朝天子一朝臣,当有心进取的天子中风病倒之后,他这个以知兵闻名朝野的内宦名将,能不能得到皇后的认同,其实是很难说的一件事。要是皇后厌武喜文,治事保守,那他可就全无用武之地了,最好的情况,也只能是去期待日后太子秉政会改回当今天子的作风。

快步来到了福宁殿,皇后就在外殿中的御书房里翻看着奏章。

除了人不同以外,御书房囘中的摆设,都是王中正旧日所熟悉的一切……其实还是有一点区别的,御桌旁的白屏风,已经添了一块新的。旧的屏风上,大半幅面已写满了人名,有百十人之多——这些全都是赵顼看好,准备任用的低品臣僚,除了福宁殿书房这里,崇政殿那边还有一面。而新的屏风上,则只有寥寥数人的姓名。

出乎王中正的意料,向皇后的召唤却是针对当年的开拓横山和广锐叛乱:“……记得王中正你当年是奉旨去的陕西,那时的情况,多多少少应该还知道一些吧?”

王中正虽惊讶,但也不慌不忙。蔡确拿着当棍棒敲打司马光的那段旧事,方才他也回忆了起来。他慢慢的组织语言:“微臣的确就在陕西。当其时,庆州兵变,关中动荡。故微臣奉官家之命,赶往延州宣诏,召回罗兀城中的精锐去平叛。而韩学士,当时受宣抚陕西、河东两路的韩大观征辟,为宣抚司管勾伤病事,身在罗兀城中。”

当年的陕西河东两路宣抚韩绛,眼下是以宰相的身份在外,得受观文殿大学士,故称之曰韩大观。但这位韩大观,向皇后多多少少知道他并不算称职。

向皇后还记得十年前发生的那些事。当时的皇帝先是因为修筑罗兀城成功而欣喜不已,继而前方兵事不顺,就变得忧心忡忡。等到庆州叛乱的消息传来,京中一日三惊,皇帝也茶饭不思,日夜守在武英殿中看着沙盘。而后宫中,也同样是人心惶惶。

那时的皇帝还年轻得很,登基才几年就开始主动攻向西夏了,但也是冒险得让人难以安心。从回忆中抽囘出思绪,向皇后继续听王中正说过去的故事。

“那是罗兀城下西贼由西夏国相梁乙埋领兵,兵力几近十万。幸而城中良将强军云集,又有知兵如韩学士的文官辅弼,所以能从数倍于己的西贼眼皮下顺利撤出,甚至连伤兵也一个不落的都带了出来,城中资材粮秣全都烧光,只留一空城与西贼。”

向皇后听得全神贯注,王中正可是当年那一战的当事人,说的虽然简略,但听着却有惊心动魄之感。

“那一夜,微臣与韩学士、张总管领后军而行,西贼衔尾追来,却被我殿后的数千官军设伏大破之,一战斩首千余级。战后论功,韩学士的功劳,便只在主帅张玉、高永能之下,微臣也忝居其后。”

“那是你的功劳。”向皇后道,“你一个内侍敢于殿后,没丢了官家的脸面,受赏是应该的。”

王中正闻言登时满心欢喜,通过这个态度,他已经把握住了向皇后的想法。

“一回到延州,从罗兀城回来的官军便立刻向咸阳赶去。这都是鄜延、环庆的两路精锐,能让天子安心的,也只有他们。”

王中正不着痕迹的跳过了他自己回延州后就称病的那一段,继续说平叛的事。

“仁宗以来,能兴兵据城的叛乱也就那么几次。贝州王则,保州韦贵,而后就是这庆州的吴逵。不过王则、韦贵,都是据城而守的守家之犬,叛后皆坐守城中,待死而已。惟有吴逵,破庆州后立刻南下,如囘狼囘似囘虎,锋锐难当。要不是在邠州有游师雄设伏阵斩劫吴逵出牢的首恶,打掉了叛军的锐气。准备过渭河的时候,又得当时的秦凤副总管,被任命为招捉使的燕达阻截于咸阳,长安恐其不保。”

向皇后点了点头。这是司马光的运气,多亏了燕达和游师雄。

燕达她是知道的,是她丈夫之前最看重的将领,年纪轻轻便担任了三衙管军,前些年尚在京任职的时候,轮班守卫宫掖,颇见过几面。

游师雄这个名字虽然向皇后很陌生,可就在一个多时辰前,她在殿上才刚刚听人提起过,也正是她想向王中正征询旧事的原因之一:“听说游师雄是韩学士的同门?”

“正是!不过他比韩冈投入横渠门下要早多的。游师雄,字景书,是治平二年的进士。在陕西文臣中,以知兵而著称。旧年邠州破贼,便是最好的例证。之后在缘边各路任职,也都有上佳的表现。”

