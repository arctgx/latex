\section{第24章 南国万里亦诛除(二)}

铜柱一宴后,让诸多蛮部等候已久的真正的盛宴狂欢也终于开始了。

近百位洞主带着他们的亲信,总计千人,随同官军重返海门港。在这座正在兴建之中的港口,让他们敬畏不已的经略章相公将会开始瓜分交趾土地,将他们渴盼已久的土地,依照功劳多寡给划分下来。

“海门港周边五十里,这是直属于中国的土地,属于海门县,为交州治所,隶属于广南西路。不会分给任何人。”

作为蛮部之中官职最高的一人,黄金满被允许第一个进入设立在海门镇的招讨行辕。行辕正中,是一幅巨大的沙盘,东面是海,西面是山,南面一条细窄的通道联通,这是交州的沙盘地形图。

黄元已经从他的顶头上司那里得知了大体的分派方案,就在沙盘前为自己的父亲解说着将会怎样分配交州的土地。

“有了海门港,官军就能控制着交州全境。日后若有人反叛,官军可以直接从海门港出兵,而援军也可以顺海路前来,不惧有人居中阻碍。不费吹灰之力,便能将海门港守住。”

黄元侃侃而谈,黄金满满意的连连点头。他已经确定将如何分配自己手上的两块互不相连的土地。他年纪最长的两个儿子,黄元、黄全,一个将继承在广源州的部族,另一个则是统治新的领地。

自家的儿子跟在章惇、韩冈身边,耳濡目染之下,看得出来他比过去进益了许多。将这一片土地交给他,黄金满也能放心下来,

“至于升龙府,由于有铜柱的存在,章帅说了,也不会分配给任何人。那里是连接南北的交通要道,据韩副帅说,将会设立一个军寨——听说起名做河内——安排下兵力来看守。除了升龙府外,如月渡等几个大渡口都是如此。”

黄金满对此并不意外,官军不可能只守着海门港,就像是邕州,除了邕州城以外,顺着左右江沿线,设立了一串军寨,控制住周边的部族。

在另一间厅堂内,章惇、韩冈、燕达和李宪列坐其中。

“……通过富良江,将两岸的部族给控制……不对,”章惇摇摇头,“是监视住。”

燕达能够理解,如果想要控制,兵力就得放上许多,但换成是监视,只需几百人就够了。

“交州的重心在海门港,即便是旧日的升龙府,也不会放上太多的兵力。”韩冈也说道,“当初邕州就是边境诸寨安置的兵力太多,使得邕州城内无兵可用。当永平、太平、古万被攻占,邕州城内的兵马连城墙都守不全。如果这些兵力大半聚在邕州,也许就能多保着三五日了。”

燕达和李宪都点着头,他们对此并无异议,此前此事其实也已经有了定议。“不过既然要掌握住富良江,那江上的水师肯定就不能少了。”燕达说道。

“不仅仅是富良江,就是海门港,也需要在一支能巡守海上的水师。”章惇道,“海门东面的海上,有多个岛屿,已沦为一干亡命的渊薮,官军最好能早日上岛清剿,不能他们侵袭海路。”

“南面的占城怎么办?”李宪问着,“交趾国灭,也让他们捡了便宜去。南面的几处州县,都让他们给占了,不能任由他们猖狂,置而不论吧。”

章惇的脸板了起来。

就在官军攻克升龙府的同时,与交趾素有旧怨的占城国也趁机出兵,侵占了偌大的一片领地。

之前经略招讨司因为一系列的事务,上上下下忙得不可开交,竟然把他们给忘了,直到前些天派了人去交州南方,却发现各地的旗帜已经换了人家。

“本官已经修书一封,让他们退出侵占的交州领土,这就让人送去占城。如果他们胆敢置之不理,交趾的下场就是他们的下场。”章惇声音阴狠。将交趾灭国,是他可以光耀一声、遗泽后世的壮举。如果有人想在他的荣光中抹上一层污秽,那么他绝对不会退让半分,必然会狠狠地报复回去。

“得到了交州,占城、真腊两国,自此与中国成为邻居。”韩冈则是微笑着,“如果他们不想老老实实的做个安分的皇宋属国,也不介意将他们变成占城州、真腊州。”

“就让末将领兵去好了。”燕达起身请战,“得让他们明白,官军是不会离开交趾的。”

“从河内寨向南六百里,在长山以东,全都是大宋的疆土。”韩冈指着地图,交州的北方,是东西数百里的山区和平原,不过到了南方,属于交州的土地,已经是狭长的一条,紧邻着海岸和高山,“这里在唐的安南都护府时,是驩州、爱州的地界,以古罗江与占城的前身扶林为界。交趾虽是几次征伐占城,国境线也基本上稳定在此。”

“即是汉唐旧疆,自当寸土不让。”李宪也是豪气干云,“若敢凌犯中国,纵然有万里之遥,也当发王师以诛除!”

章惇霍然起立,用力挥动手臂,“先将各家的土地分配下去,等到划分完成之后,占城对本帅的信函置之不理,还不肯退出他们侵占的土地,就立刻拈选精锐出兵,打到佛誓城去!”

分配土地的会议,基本上能够算得上是顺利,尽管人人都想多拿到一分,但章惇、韩冈镇着场子,许多纷争在还没有开始的时候,就已经宣告结束。熙熙攘攘的闹了十天之后,最终的分配方案终于敲定下来。

黄金满分到的土地最大,是交趾南方、从清化一直延伸到边境,与广源州相隔千里。不过连同他在内,分到土地最多的十一家部族,互相之间三两聚居,身边都有同样大小的部族,互相牵制着。

而围绕在海门、河内周围的核心地带,则是一干小部族的领地,零零散散的几十家,犬牙交错的分布着。这样的安排,可以让任何一家起了不轨之心的时候,都要先担心他们的邻居会不会背后一刀。

这些分派是在地图和沙盘完成的,并没有太多的准确性可言,不过利用交趾当初分置州县的界碑,倒也不至于为了边界的确认,耗费经略招讨司上下一干人等太多的精力。

占城并没有退出他们侵占的土地,不过占城王制矩献上了许多金银财物,抱着一丝侥幸,希望能默认。不过金银之物,大宋的数量更多,却不会为了钱而放弃土地。

在这样的情况下,燕达领军出阵,没走陆路,而是走水道,用了三天的时间,抵达南方的边境,在那里下船。而与此同时,秉持经略招讨司的令旨,交州诸部联军起兵南下,两方合力,将占城的侵占交州南方的军队给全数消灭。还没等到燕达返身南下侵攻占城,打下王都佛誓城,占城国王制矩已经被吓得魂飞胆丧,亲自带了人和财物来乞降。

章惇将之教训了一番之后,制矩和俘虏们被放了回去,因为这一战的耽搁,就又是半个月的时间。

这半个月中,新任的交州知州也确定了下来,是章惇的一位幕僚,姓李名丰,在这一次的军事行动中,作为行营参军,功劳和苦劳都立下了不少。通判和军事判官,都是从广西调来的官员。至于海门知县,韩冈推荐了自己的幕僚马竺。

在区划上,海门港是一个县,如今正是百废待兴的时候,土地荒废,户口缺乏。

章惇和韩冈计议之后,将主意打到了从交趾国中解救出来的汉家百姓身上。他们之中,将会有两千户迁移到这里,分配土地和种子,并从官府这里借贷了农具和耕牛,等收获后配上利息加以归还——基本上就是各地安置移民的翻版,不费多少手脚。

而交州也会安置一批邕州训练出来的新军,总共一千一百人,将会移防于此,同时还包括他们的家属。

加上之前的两千户汉儿,也就是说,交州的海门县,已经有了三千户口。这对于汉人数量稀少的广南两路,已经是一个数得上的大县了。

“该做的事都做完了,差不多该回去了。”

章惇已经将桂州放下了有半年之久,不过有通判看着,其实并无大碍。但章惇说得回去,并不是桂州,而是东京城。

“还是先去邕州。”韩冈说道,“坐船到钦州,从那里走要方便的多。”

进入了四月之后,雨水又多了起来,韩冈望了望外面,因为雨水的缘故,大部分的工程暂时都停了下来。一时间,也没有继续开工的可能。自己也可以趁这个时候,一起返回邕州一趟。

章惇点着头,“宗亶他们在狱中已经休息得够久了,也该送他们上路了。”

最后的圣旨已下。

虽然不能将已经死了的李常杰拉起来再斩首一遍,但在邕州城下、忠勇祠前,当初曾经领军侵攻过大宋疆土的将领,都会将之明正典刑——可惜广源诸帅太识时务,否则就一并拉过来陪斩。

当曾经凌犯中国的罪人在他们肆虐过的地方,用性命来赎清他们的罪孽。自此之后,南方当能保有二三十年的平安。

