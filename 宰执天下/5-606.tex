\section{第47章 岂意繁华滋劫火(下)}

李肃之汇报过灾情,很快就离开,前往火场指挥。

当然,韩冈知道他应该还是会留在有城墙防护的城内,而不是当真前往火场。

现在缺乏足够的灭火手段,士气又十分低落,躲在后面的官员,在指挥上就隔了一层,更不用说与下属之间的信赖关系。

他之前之所以能让向皇后改了提举太极观的任命,转为权知开封府,是因为他的经验——曾经两任开封知府——调和京中内外事也能得心应手。对于这项任命,在今天之前,没多少人有异议,他表现得很不错。

但现在遇上急难,他的劣势就凸显出来了。年纪太大,不能高效迅速的指挥灭火。当年在瀛洲知州位置上处理地震的才能,看起来已经随着时光而消磨殆尽了。

殿中每一个人的心中都清楚,李肃之在开封府的时间已经没有多少了。

如果没有大量人员伤亡,李肃之的位置或许还能保得住,朝廷也会给他这位老臣留一分体面。现在才被引燃的石炭场,又是一批伤亡。人数还没来得及统计出来,但数量绝不会少。就算之后火势顺利的被扑灭,李肃之都很难逃脱罪责——总要有人出来为这么大的伤亡和损失负责。

而且从一座石炭场,蔓延到第二座,火势不仅仅是翻倍那么简单,需要防护的范围一下扩大了数倍,需要调遣的人手也陡然增加了数倍。李肃之到底能不能控制得住这样的一场大火,现在正在崇政殿中的宰辅们,谁敢为他打包票?甚至已经有人站出来说要走马换将了。

“殿下。”韩绛出班道:“李肃之年已老,精力不济。方才奏对时,奏事无序,已可见一斑。依臣之间,还是选派良臣指挥灭火,李肃之回镇府中为是。”

“相公心中可有人选?”

“暂无。”韩绛毫无愧色的摇头,“但朝中良臣无数,素有威信和才干的重臣为数不少,殿下可从中选派。”

这位首相一边说着话,一边却瞥了韩冈一眼。

韩冈正看着韩绛,与其视线正对上。顿时,韩冈的眉头就皱了起来。韩绛这话说的,是想让自己站出来毛遂自荐不成?顺便保住李肃之?只是他想了一下,觉得倒也无妨,朝廷现在一时间也很难找到人。

正常情况下,不可能有人会在这时候挺身而出,临危受命。摊子万一砸在手中,谁就得把底都兜了。危急关头,能有几人愿意赴汤蹈火的——现在可是真正的要蹈火了!火星随风飘来,城中到处都有可能被点燃。更不用说,镇内管勾烟火事是未入流的官员才会接下来的差遣,哪位重臣愿意去做这样的事?或许会变成小官争取上位的工具,而不是正常的灭火。

想到这里,韩冈的心便坚定了。与其让不靠谱的官员来指挥,他觉得还是自己出马能够安心一点。至于面子的问题,韩冈还不放在心上。

“沈括如何?”说话的并不是韩冈,而是蔡确。赶在韩冈站出来之前,蔡确提出了他心中的合适人选,“才干是不会缺的,能力、经验都绰绰有余。不过堂堂翰林学士,不能仅为一管勾烟火事。得让他权知开封府才算是名正言顺,调动人手也更方便一点。”

翰林学士兼权知开封府,要卸下的仅仅是知制诰的头衔,地位却涨了许多,已经近乎于两府宰执了。韩冈相信,以沈括的姓格,绝对不会拒绝这样的任命

历代权知开封府的,都是朝廷精挑细选出来的能臣。李肃之也不例外,但他现在缺乏临危不乱的能力,也缺乏足够的见识的手腕,这一点上他是远远比不上实务经验充分的沈括。

沈括在翰林学士的位置上,并不很受到重视,只是依常轮值,起草诏书。至于备咨询的任务,向皇后从来没有起用过他。所以沈括的闲暇时间很多,总能将大部分精力分在《自然》期刊上。一旦他就职开封府,可就没那个空闲的时间了,的确让韩冈感到遗憾。但他能得重用,韩冈也乐见其成。

韩冈点着头,附和道:“沈括有才干,若能即刻上任,当能有补于眼下的急务。”

有蔡确和韩冈的配合,向皇后没有任何异议,直接点起身侧的内侍:“宋用臣,速去招沈括上殿。”

宋用臣应声,奉旨去找沈括,殿内一时无事,静了下来。

空气中弥漫着烟味,平常让殿中香烟馥郁的香炉,这时候混进了煤炭烟尘之后,变得让人无法忍耐。

沐浴在这样的空气中,韩冈觉得喉间一阵发痒,总忍不住想要咳嗽起来。不过崇政殿中实在不适合做这样的反应,只能强自压制住,等着那种刺激感渐渐消退下去。

突然间屏风前一阵咳嗽声,却又被强行压制住了。

“官家!没事吧?”

来自帘后的声音,很是紧张,赵煦贵为天子,咳嗽一下都是天大的问题。

赵煦彬彬有礼:“回母后,孩儿没事。”只是说话还带着咳嗽。

“宣徽,官家外感烟气,一直都在咳嗽,可有良方?”

赵煦完全是被污浊的空气弄得咳嗽起来的。韩冈为难的看了眼外面,只要火势依然不减,这城中的空气就别想干净起来。

向皇后这可是为难人了,都这时候了,难道还能弄出空气净化装置来?

韩冈心中叫苦,想了想,给了一个不是办法的办法,“将单层的口罩浸了水,罩好口鼻处,或许能隔离一点烟气。不过殿中寒冷,口罩又是被打湿,带着会很不舒服。”

“听到了没有,还不快去办。”向皇后没等韩冈把话说完,便训斥着身边的内侍,让他们赶快去按韩冈的意见去准备口罩。

很快,赵煦带上了湿润的口罩,感觉一下好多了,只是短时间内口罩便冷了下去,冰冷湿寒的感觉让他很不舒服。

向皇后也感觉到了这一点,“官家还是先下去吧。去探视一下你父皇。”

“儿臣明白。”

赵煦应声离开,片刻之后,他便抵达了福宁殿,一边说着,“外面的灰进来了,把门窗关紧点。”一边走进了赵顼躺卧的内室。

深宫阴寒,高耸的殿宇外观壮丽,却完全不适宜居住。幸好赵顼使用的是一张新床,没有使得热量散发得过快。

新打造的大床,有柱、有门、有槛,像是床,又像是房间。名为八步床,号称纵横皆有八步。也不知从什么时候开始在京中流行,

赵顼病瘫在床,他的病房中人进人出,总是免不了要带了几只蚊虫进来。尽管有多少宫人服侍左右,但总免不了百密一疏。

虽说只有两次被蚊虫叮咬在脸上,可对于一名曾经的皇帝、现任的太上皇来说,已经是太多了。

所以很快福宁宫中便换了这样一张能隔断内外的大床。三面都钉了木板,正面下了蚊帐,隔了一重纱帘,外面的蚊虫就很难再进来了。如今又是冬天,纱帐就换成了厚重的毛毡。

平曰里为了听到内间的动静,通向外间的小门总是留着一条缝,现在有了八步床也就更方便了,留两个人在床边值夜,剩下的就可以在外面照睡不误。

赵煦一板一眼的向赵顼行礼问候,赵顼也轻轻在沙盘上画了几笔,算是回应。

对于向皇后,赵顼除了好之外,就没有写过别的字。只有在跟儿子交流的时候,才会多写几个。

赵煦走到床边,看着他的父皇一阵,突然转头问着太医。

“鲁太医,父皇是不是在咳嗽?”

赵顼看着是在咳嗽,只是气息微弱,声音极不明显。但照顾他的内侍宫女,还有翰林医官,天天服侍左右,很快就察觉赵顼胸口的起伏节奏不对。不过他们也没办法,皆是束手无策。

上品的贡炭只有极少的烟气,暖炉又经过精心设计,烟气都会通过管道通入水中,出来之后,就变得干净清爽。但现在是整座东京城的空气都是变得如烟囱里冒出来一般,又哪里有水来洗去烟气?

“父皇,是不是咳得难受?”赵煦趴在床边,紧张的问着。

赵顼眨了眨眼睛,却没用手指直接给出答案。

“用口罩。浸了水后给父皇带上!”赵煦立刻道。

鲁太医忙叫住跪下来领旨的内侍,对赵煦道,“官家,这样上皇会喘不过气来。”

“朕才用的!”赵煦抬头道。

跟着赵煦过来的内侍也说道,“鲁太医,这可是韩宣徽刚进的避烟的方子。”

“官家明鉴。”鲁太医直言道:“上皇久病,又不能移动,如今气息极弱,就是一片纱也会透不过气来。这跟官家不一样。”

赵煦被难住了。皱着眉,他年纪幼小,也想不出办法。

“要不问问韩宣徽?”一人提议。

赵煦眉头更紧,不置可否,只说道:“把帐子弄严实了,不要透风进来。”

赵煦虽然年幼,但配上天子衣冠,已有几分威严。立刻就有几人过来,将帐子上的缝隙都给压住了,不让外面的寒意侵入有着暖炉的床内。
