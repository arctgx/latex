\section{第31章 九重自是进退地(12)}

“是吗?小弟这边都还不知道呢。”韩冈啧了一声。

王安石是退职的宰相,若是晋爵为舒国公,从地位上已经与韩国公富弼、潞国公文彦博相当,同为元老重臣了。

不过国也有大中小之分,三六九等,在官场上是免不了的。秦、晋、魏、韩这样的是大国,而王安石的舒国则是小国。第一次封国公,只会是小国,等到第二次、第三次晋封,才会逐步上升。文彦博只封过一次,所以他的潞国公并不比王安石的舒国要强,而富弼则是第一次祁国公,第二次郑国公,第三次才升到如今的韩国公。

“以介甫相公的资望,国公已是来得迟了。”章惇声音压低了点,“只是相公求转宫观,是否打算就此致仕?”

韩冈哪里有机会与王安石聊这些。他回京时,王安石早就走了。不过从王安石留下的书信上看,还能勉强揣摩到他的一点心思,“家岳当是无意再掌朝政,京中十年,早已心血耗尽、油尽灯枯。最后的那半年,小弟没有看到,子厚兄应该看到了吧?”

章惇默然点头。去年从夏天开始,直到王安石离任——也就在韩冈返回京城之前的几个月——因为王雱和王安国一两年间接连病逝,王安石一下老了许多。

再加上在政事上,又与赵顼又产生了许多分歧,使得王安石甚至都在叹着若有三分相从也是好的,远远不能跟熙宁初年时想必。韩冈说他是心血耗尽,油尽灯枯,那是一点也没有说错,也丝毫没有夸张。

韩冈叹了一声,也不讳言,“家岳如今当是心在江湖山野之间,已无东山再起之念。再不可能像熙宁八年的时候那样,应诏复出,重镇朝堂。”

章惇虽说算不上失望,但也是一声长叹。王安石一手创立了新法,用了近十年的时间,让原本屡受西北二虏所欺的大宋,反过来让两国必须联手才能抵抗。富国强兵的初愿,王安石已经为天子实现了,但他现在却无法享受到变法成功给他带来的荣光。

但章惇也不能说什么,韩冈也不会说什么。坐在帝位之上,本来就该是这样的人物。

王安石的离开,根子就在天子身上。不论做得多好,一旦天子觉得用不上了,立刻就会被抛开,也就是给个虚名,让人赞颂着天子的慷慨。别说眼下坐在御榻上的这一位,就是被人人赞颂的仁宗皇帝,不也是这样?庆历新政的土崩瓦解,难道不是仁宗皇帝认为不需要了,才让吕夷简得手?

上观诸史,帝王莫不如此。熟读史书的士人,早就该见怪不怪了。

心情低落,让章惇无心再提及回到江宁养老的王安石,只提醒韩冈道:“玉昆,你还是要小心沈括。此人虽是才高,却是素无信义。可用不可信,如果两府之中有人压下来,他当能在背后捅你一刀。”

章惇对沈括成见已深,韩冈忍不住有些觉得好笑,不过他也不为沈括辩护,点点头,“小弟明白。”

“不要不以为然,”章惇看这韩冈脸上若有若无的笑容,忍不住多提醒了他一句,“介甫相公刚走,他就去奉承吴冲卿,见风使舵得这么伶俐,你可曾见有谁有他一半的本事?”

见章惇说得郑重,韩冈也不得不严肃起来,“子厚兄放心,小弟自是会小心谨慎防备着。”

章惇神色放松了一点,在他看来,韩冈是没吃过亏,所以自信过度,正常吃一堑长一智,要慢慢历练出来,不过以他跟韩冈的交情,该题型还是得提醒,“还有洛阳,想想西京御史台,想想西京留守,那里可是虎穴狼窝。玉昆你去了京西,更是要小心,不要给人寻出错来。”

韩冈再点头,京西转运司的治所就在洛阳,洛阳城中的文彦博、富弼、司马光这一干元老重臣,比起方城垭口要险峻百倍都不止,这一次他的态度端正无比:“小弟理会得。”

……………………

沈括正在收拾自己的书房。

已经定下了去唐州担任知州,这间专供三司使居住的宅子,也该让出来了。

家里的仆婢,除了少数一部分是签了卖身契,其余大半就是在京城雇佣的,现在都发遣了出去,再有就是家中的清客,这两天用着各种各样的理由也走了一多半。

不过这都不是沈括自己收拾书房的主因。

其实放在书房里的藏书都已经收拾好了,珍贵的孤本和手抄本,趁着日头好,晒过一天之后,小心的放进了箱子里,与家中的一些珍贵的器皿财物放在一起,绑在车上。

剩下的几千卷皆是刻印本,多半是国子监版,还有一些则是出自杭州的印书坊,至于粗制滥造、市面上泛滥最多的福建版,只有几本,要不是书卷本身内容的难得一见,沈括这名有名的藏书家也不会将之收入自己的书库。

近万卷藏书堆满了两辆马车,旧时一排排堆满书、一直堆到天花上的书架,现在已经变得空空荡荡。但沈括还有一件最宝贝的藏品,要亲手收放起来。

红铜铜皮打造的上下两节圆筒,架在一个形状特异的木架上。圆筒两端各有一个小小的镜片,如同水一般清澈透明。如果眼力足够好,还能分辨出两端的镜片,凹凸各不相同。

这是显微镜。

拿着显微镜,沈括用来观察过落入院中的树叶,观察过从深井中提上来的井水,观察过被扑落下来的蚊虫,观察过地上的一撮泥土、沙尘,他此前从没有想到,寻常看惯了的事物,一旦放大之后,就变得如此光怪陆离。

从未有人窥探过的微观世界,对沈括充满了吸引力,他看清楚了蚂蚁、蜜蜂由一个个格子组成的眼睛,也看清了树叶上一条条细微如丝的脉络,更看清了清澈透亮的井水中,竟然有着那么多的异物——因为这一件事,让沈括对佛家多了一分崇信,佛观一碗水,有八万四千虫,所以喝水前都该持咒一番,有人嗤之以鼻,但现在用显微镜一照,当真说得一点都没错。

光是观察这细致入微的世界,就消磨了沈括不知多少闲暇时间。也不仅仅是沈括,京城中多少士大夫都对无人涉足过的领域充满好奇。

自从一年前不知由谁人发明并命名之后,显微镜转眼就在京城中流传开来。不过到现在为止,能拥有一架性能良好的显微镜的人,在京城中还是凤毛麟角。制作不精的显微镜,只能放大个十来倍,而像沈括他手上他亲自设计,并聘请名匠打造的显微镜,则能放大三四十倍之多,一根细微的发丝,都粗大得如同用大楷笔写出来的笔画一般。

但就算是制作不精的显微镜,如今价格也是高达数百贯,而且是有价无市——水晶镜片实在是太难得了,而适合做显微镜的则更难得。

这是因为需求量太大的缘故——十年寒窗,视力好的士大夫并不多,而年纪大的官员无一例外都有需求——白水晶的价格涨到了天上去,已经传说有人开始想用玻璃来做镜片了,只是还没有成功。

而且原材料在镜片中还只占了很小一部分,大头是人工。这么些年来,京城中到现在为止,也只培养出六名高手匠人,专门负责磨制镜片,而他们各自还有几名徒弟。总共二三十名匠人,要为全京城的官员和富户来磨制凸透镜、凹透镜,来改善他们的视力,这当然是杯水车薪。

而要找到四五片适合做显微镜镜片的,更是得从几百片凹透镜凸透镜中加以仔细挑选,的确不容易。沈括也是挑选了好久,才试出来合适的,这还是靠了他三司使的身份。

现在想来,恐怕韩冈本人应该都不知道,他所发明的水晶阳燧,也即是俗称的透镜,能派上这等用场——他没有这么多选择去测试。

这样的一架显微镜价值千金,沈括是当做传家宝一般珍视着。

不过显微镜还有些问题,要是能用什么办法,让镜筒能自如的上下调节高低就好了。

沈括小心翼翼的将显微镜拆开,拿着硝制过的麂皮,一点点的擦干净了红铜镜筒上的指纹,如同捧着自家的幼子,慎而又慎的放进堆满木屑和稻草的木箱中。

“都收拾好了?”

听到身后传来的声音,沈括身子一颤,抓着箱子的手差点都松掉。站起来转过身,垂手低头,用着殿上面对天子时都没有的恭谨口吻答道,“都收拾好了。”

比起天子还要让沈括敬畏的续弦张氏,正站在书房门前,容色过人的一张俏脸挂着寒霜,,眉眼吊着,让人不敢亲近。

不快的眼神扫过空空如也的书房,张氏高高在上的瞥着比她还要高出一头的沈括:“收拾好了,就送到车上去,还要耽搁到什么时候,等着宫里面派人来赶吗?”说着就转过身,往回走,“跟着你这个夯货,连京城都住不安稳。”

沈括也不敢回嘴,抱着装着宝贝的木箱,不用吩咐,就亦步亦趋的跟在后面。

张氏领头在前面走着,“韩冈让你去唐州,也没安好心思,是想借你的本事。这开凿渠道,听说功劳不小,不要把功劳都让他得了去。”

“是,是,夫人说的是。”沈括点着头,一步步紧跟着。

张氏脚步一停,回头虎着脸瞪了沈括一眼。沈括悚然一惊,连连点头,“为夫知道,为夫明白,不会让功劳都给韩冈得了去。”

张氏脸色好了些,厌憎的又看了卑躬屈膝的沈括一眼,“也不是让你跟他抢,你出了多少力,就该分多少功。你是唐州知州,不是他转运司的属僚,该争就得争。你是翰林学士出外,须也不比他龙图学士差了!”

