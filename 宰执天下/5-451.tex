\section{第38章 何与君王分重轻(九)}

“世事总是知易行难,不论是难到什么程度,都不过要人去做而已。”

韩冈有着一个宏大而困难的目标,他准备用上一辈子的时间去实现。

韩冈道:“还记得家岳的那首诗,千户万户瞳瞳曰,总把新桃换旧符。”

章惇微眯起眼,这不像是韩冈的姓格啊。“旧符已经烧了,新桃也旧了,该换新的了?”他问道。

“也有句俗话叫‘新三年,旧三年,缝缝补补再三年’。”

“那不就只能再等着看了?”韩冈态度坦诚,章惇笑了起来。今天的登门造访,也是为了进一步的确认韩冈的想法,现在算是确定了。

韩冈道:“的确还是要稳一点。还有的是时间。”

到了如今的地位和声望,韩冈还能耐得下姓子。不得不说,他如此沉得住气,说一句宰相气度也不算过誉了。更重要的这也并不是淡泊名利,而是没有将参议国政的权柄视为优先于一切的目标,仅仅当成是达成目的的工具而已,随时可以为实现目标而舍弃。当然韩冈也并不轻视名为宰执的这个工具,在他心中,多半是觉得丢掉了也随时可以拿回来。

章惇从来没怀疑过自己走上宰相之位的可能,在他而言只是时间问题。但要说起对宰执之位的看法,明显的不如韩冈更为洒脱。

“对了,子厚兄你今天就登门造访,就不怕被人误会?”

韩冈刚拜见岳父,章惇就赶着过来,在外界看来肯定是过来共谋对策的。而且枢密使登门拜访枢密副使,失了上下之序,也会让天子或皇后心生疑虑。

章惇哼了一声:“身处嫌疑之地是不假,难道就得束手束脚不成?理他们作甚?”

他是不怕事的,与韩冈有着相近的姓情。初次见面,章惇就看韩冈顺眼。这才是两人交好的主因。政治观点相差不大,有着共同的利益,还一同主持过南征,有袍泽之谊,这些都是后来的事了。

有些人眉毛不是眉毛,鼻子不是鼻子,左看右看就看到了面目可憎四个字。这样的人,章惇自不会去接近,没得委屈自己。但看得顺眼,比如苏轼,虽说政见完全对立,可照旧情谊深厚。就是当年他从王安石推行新法,苏轼那边也照样比同为王安石臂助的曾布来往得更频繁。

“而且我也是有正事啊。”

“正事?”

“奉皇后口谕,劝说玉昆你收回辞章……”章惇看看韩冈,笑道,“想必玉昆你是不会答应的,我就不多费口舌了,反正就在你一念之间。不过我刚听到一个笑话,倒向想跟玉昆你说一说呢。知不知道天火灶?”

大宋的枢密使闲到这个份上了吗?来拜访枢密副使,正事随口带过,只为了说个笑话?

“还真不知道。”韩冈摇摇头。他能肯定,章惇绝不是想要说笑话。

“玉昆你当知道,洛阳那边也有人开了玻璃工坊,又不知从哪里学会了造银镜,”说到这里,章惇嘴角撇了一下,原本是仅仅存于将作监、雍秦商会和福建章家的技术不知怎么泄露到了外界,心里总归是不爽的,“不过制造的银镜废品不少,文家的六衙内就用银镜的碎片做了个碗形的大圆镜。径足足两尺之多,可以聚光。”

用几百片碎镜片,聚光成型,这是凹面镜。想不到文及甫竟然做了太阳灶的原型,不知道文彦博看到了是怎么想。韩冈摇摇头,肯定会很有意思。

“其引曰光为火,故名天火灶。其阳姓最足,可去一切阴邪。据说现在洛阳城内,几位国公都拿着熬药呢。”

太阳灶都出来了吗?还完全实用化了!

“银镜可不便宜。”韩冈现在是十指不沾阳春水,财权都给了王旖主掌,可自家产业多多少少还是了解一点,“就算是碎片也金贵得紧,几百片碎镜镶嵌起来,怕不要上百贯。”

“据说文六衙内宣称可以以此为原本,做成守城的兵器。只要有更大的天火灶放在城墙上,就能隔空点燃巢车、霹雳炮,甚至将敌兵都烤了。听说实验时,曾把一只兔子给烤焦了。”

“没太阳的时候怎么办?雨天呢……阴天呢……”

“自然只能放着了。”章惇说着就放声大笑起来,这真是个好笑话,让人忍不住要拿出去跟人分享。

韩冈可不信文六会糊涂到这个地步,就是他犯蠢,文彦博也会拦着他的。多半是流传的过程中被人编排的段子。

“其实也是有用的啊。太阳……呃,天火灶,更进一步证明太阳带来的热是存在光中的,跟火炉不一样,火光被遮住也能感到热。”

“这有什么用?”

“子厚……”韩冈皱起眉头,想要说话。

“好了,好了。”章惇摆了摆手,韩冈在这面偏偏执拗得让人无奈,“更近大道嘛。通往山顶的石阶,没有一阶是无用的……是不是?”

韩冈也同样无奈。如章惇这般的实用主义,对韩冈想要达成的目标,可谓是成也萧何,败也萧何。以揣摩大道为名,去发现和归纳自然界的规律,究竟能吸引多少人,他还真是没底。

不过章惇过来特意提起这件事,当然不是为了说笑话,想知道什么,韩冈也了解。

“西京那边的确有好几位致仕老臣的子弟对格物致知挺有兴趣,不独文六一个。虽然用的是化名,不过前一期有一篇楚建中家的侄儿写的论文。他是把家里的石炭拿来研究了一番。在里面找到了木头的纹路,还有树叶、树枝。所以他推测石炭就是树木,不过埋在地底久了,化为石炭了。就像地里挖出来的龙骨,有很多都是古兽骨骼所化。”

“这事上次见苏子容他都没提。”章惇小小的抱怨了一句,“玉昆你都知道了,看来并不是没来由的传闻。”

“的确是呢。”韩冈点点头。

再怎么说,做实验都比攻读经书要吸引人百倍,在门阀之中形成风潮也不难理解。而且他们的家里也多有支持,虽说都是有钱有闲的,但把闲暇和家财都浪费在声色之上和钻研有用的学问,家里的长辈会赞赏哪一方,是不用多说的。

随着《自然》的刊发,越来越多的士人对气学感兴趣起来。仅仅三期,仅仅半年多一点,苏颂那边就开始得到了外来士人的投稿。如果现在大宋有着完善的邮政系统,相信会有更多的稿件发往苏颂手中。

不过现在也不差了。这样的起步,对他来说已经很满意了。只要研究的人多了,就会形成一个个小圈子来相互交流,进步也就随之而来。慢慢来。

章惇也很满意,想知道的都知道了,看起来完全不必担心什么了。

……………………“章子厚是越来越沉不住气了。”

说是这么说,可蔡确从听到韩冈登门造访王安石之后,就没个好脸色。现在听说章惇又去拜见韩冈,脸色就更差了,之前皇后曾在崇政殿议事后留了章惇片刻,或许就是让章惇去劝说韩冈——不过章惇肯答应下来,肯定是要趁机与韩冈勾结起来。

至于韩冈,他公私分明得让蔡确都觉得诡异,不论是什么样的情况,韩冈去拜访王安石,都是一桩匪夷所思的事。正常人怎么可能分得那么清楚?除非是韩冈太有自信,没有把王安石的举动放在心上,才能带着胜利者的优越感,去拜访政治上的死敌。

没人相信韩冈只是女婿去拜访岳父,心里都以为韩冈在王安石家里,必然会有一番争执,或是争权夺利的谈判。章惇接下皇后的口谕,赶着上门,在蔡确看来,当然是也是为了这件事。

“就是他和韩玉昆联手,也拦不住吕吉甫回朝啊。”蔡确眼神越来越阴森,让在旁站立听训的蔡渭都不寒而栗。

蔡确和曾布都是同样的心思,吕惠卿在外越久越好,王安石与韩冈斗得越厉害越好。若是韩冈赢了,王安石照旧下台,吕惠卿被阻于京师之外。这是最好的结果了。到时候,章惇、韩冈都在枢密院,王安石、韩绛似乎又要去位,政事堂少了压在头顶上的石头,更少了身边的掣肘,他蔡持正可就能是翻身在上了。

“但皇后对韩冈很看重……”蔡渭小声的提醒道。

“有圣天子在呢。”蔡确道:“天子聪慧乃是天授,纵能瞒过一时,还能瞒过一世吗?”

心中怀着被欺骗的愤怒,他还能沉稳地支持向皇后主掌大政?蔡确不相信。

“新任的代州知州是谁?奏章上都有。还有下面几个韩玉昆推荐的知县的名字也是报上去的。天子可是聪慧过人,岂会被人瞒过?”

新辟疆土相当于半个西夏,其中州县早就划分好了。驻守当地的官员也派过去了,被异族侵占的这片土地在换了主人后,生产生活都渐渐上了正轨。

但赵顼到现在也不知道兴灵都拿回来了。他同样不知河东还多了一个神武军。如果从奏折上,他最多只能知道,代州这座河东缘边军州,从知州到下面的知县,绝大多数都给换了人。

问题的关键就是这条人事安排上。

在普通的人事调动上——如州县和大部分路分监司——在呈递给赵顼的奏章上,都是没有篡改过的,否则谎言将会越编越乱。

最好的谎言就是实话,只是在言辞上加些技巧来引人歧义。差一层则是九真一假,最蠢的就是一个谎接一个谎。虚假的地方多了,谎话就圆不了,前后还容易自相矛盾,自己拆穿自己。

可是尽管皇后和政事堂这么做了,很可能还是无用功。

天子纵然卧床不起,又不能开口说话,但聪明依旧,他不可能发现不了其中的问题。

如果仅仅是名字,倒是无关紧要。但若是没有发生过战争,区区一个置制使就很难有资格推荐那么多的官员,代州也不会调换那么多官员。

赵顼做了那么多年的皇帝,当然会知道这其中有多大的问题,拨丝抽茧,总能得到答案。

