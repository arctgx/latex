\section{第31章 停云静听曲中意(二)}

“臣韩冈拜见陛下。”韩冈向赵顼行过礼,又向向皇后行礼,“臣拜见殿下。”

“学士。”向皇后招呼韩冈,“你来看看官家,是不是就要大好了。”

韩冈应声上前,上下一打量,皇帝除了手能动以外,比之前也没什么大的区别,只是眼中的确多了些神采,“官家的气色是好了许多,不知现在感觉如何?”

赵顼还拿不住笔,但用手没问题。摆在床边的是真正意义上的沙盘,盘子里平铺着一层沙,扶乩用的。巫蛊一类的东西,在宫中一向最为忌讳,也不知从哪里翻出来。

赵顼的食指就在沙上划着字,韩冈看着笔画,应该是个‘好’。

手指画字,这可比眨眼方便多了。现在看起来的确是病情好转的样子。只是回光返照也不是没可能。

韩冈看看站在床边的御医们。

御医们没人敢说这是回光返照,只是恭喜天子病情好转。至于赵顼会不会康复,或是情况变得更好一点,几名御医却都不敢给出明确的答复。

将药方写成鬼画符是韩冈记忆中后世医生们的专利,而兜圈子说话是这个时代御医们最擅长的事,将一句话铺陈至千百句,最后让人在言语的迷宫中晕头转向,其水平跟资深的官吏也差不多。絮絮叨叨说了好半天,给出的回答是再看一看。

韩冈同样不清楚赵顼到底是好转的征兆,还是回光返照。但即便是好转,以赵顼瘫痪了两个月的情况,想要恢复如初是绝对不可能的。可韩冈也不方便明说。而且御医们能兜圈子,他也能。不过韩冈没有拾人牙慧的打算。

“陛下病势稍可,当为天下同喜,相公们不可不知会。”韩冈找了个合理的借口,“还请陛下、殿下遣使。”

向皇后啊了一声,她只顾着招御医、招韩冈,却忘了宰相和执政。

韩冈看着皇后招来一名名亲信内侍,让他们带着口谕招宰辅们入宫。

方才自己入宫时,御街上放烟花的市民们可都看到了,等宰执们入宫,看到的人恐怕会更多。也不知在真实的消息传出去之前,外面到底会传成什么模样。

说起来韩冈其实有些惊讶,在自己应诏之前,为什么没有去通知宰辅们。皇帝也清醒了,就算皇后没想到,他也应该主动提出传召宰执。

真不知皇帝现在在想什么。韩冈也算是擅长察言观色,只是赵顼除了眼神中多了神采,僵硬的脸上却看不出表情,依然木然。

可能是习惯成自然,皇后发号施令时并没有向赵顼请示。韩冈虽没能发现赵顼的神情变化,不过还是感觉到从他身上传出一阵阴寒。

或许不是错觉。韩冈相信自己的眼力。

天子现在已经能够移动手指了,或许过些天还能开口说话,估计现在已经在幻想日后能够重新下地走路了。那么执掌天下大政的权柄,恐怕也不会甘愿放在皇后的手中。这与夫妻之情无关,至高的权力之前,没有亲情可言。

向皇后则是什么都没感觉到,心情很好的样子。将通知宰辅们的人都派出去后,更派人去通知其他嫔妃,像是要将好消息传给所有人。

两府宰执的住处,离皇城都不远。大约半个时辰后,宰辅们陆陆续续都到了。蔡确跑得最急,曾布第二,其他人则是差不多时候同时到了。

应该也是在奉召时就听到了消息,众宰辅进殿时脸上都堆着喜色,蔡确甚至热泪盈眶,拜礼时声音哽咽,几至泪下。

只是韩冈能看得出来,大多数人脸上的喜色都有些勉强。有着一种很微妙的感觉。

“病气之去如旧岁之辞,陛下新年康复。此诚不胜之喜。”

“当颁赦诏,为陛下贺,为天下贺。”

宰辅们说着善祷善颂的话,赵顼听了一阵,动着手指,画了一个‘种’字。

杂音顿时没了。

种谔。

天子果然还是最挂念西北的军事。

蔡确道:“种谔已尊奉陕西宣抚之名,领兵救援溥乐城。有陕西宣抚司在,陛下勿须担忧,可安心养病,静待捷报。”

瘦小干枯的曾布也立刻附和:“有吕惠卿坐镇,种谔依令而行,必不致使辽人得意猖狂。”

陕西那边的动向,从种谔和吕惠卿的奏报中就能看得出有问题。蔡确和曾布就是将事情全都往吕惠卿身上推。两人当然不是在帮吕惠卿确立宣抚使的权威,是等着看吕惠卿被种谔弄得灰头土脸。

这几个都在等着看吕惠卿的笑话了。吕惠卿一门心思想要证明自己的能力,可撞到种谔这名一贯爱自行其事的下属,纵是有万般韬略,也施展不出来。

这时候,种谔兵发兴灵的消息尚在半路上,连溥乐城解围的捷报、青铜峡党项人北进的八百里加急,都同样还没有传回京城。韩冈自然不知道种谔会做到哪一步,不过种家五郎的脾气朝野内外哪个不了解?天子定好的出兵日期,他偏偏敢提前出发。他还有什么事做不出来?

赵顼在沙盘上划了四个字,将宰辅们的小心思点了出来:奏报不合。

宰辅们暗暗心惊。他们都知道皇帝天天都听人读奏章,但能对比两人奏章,找出其中的破绽,可知赵顼的头脑依然清醒。

这算是下马威吧。

章敦恭声道,“陛下若有生疑之处,还请明示,臣也好移文质询陕西宣抚司。”

韩冈向章敦投去感激的一瞥。这是帮他确认赵顼到底是准备针对哪一个。种家还是吕惠卿。

种家跟韩冈的关系很深。只论军中将领。李信、王厚、王舜臣、赵隆,这几位是韩冈在军中的铁杆支持者,可以说是一荣俱荣、一损俱损。种家虽算比不上李信他们那么亲厚,但有王舜臣、种建中乃至种朴这一层关系在,种家可算是韩冈在军中的基本盘,相对的,韩冈也是他们在朝中的主要依仗。

韩冈不在乎吕惠卿是否受罚,也不在乎种谔的结果,但种朴和种建中这样优秀的苗子,韩冈肯定是要保住。就算他们一时受了种谔的连累,韩冈他也还要保证种家内部有人能出来递补。

赵顼又在沙盘上划字,不是回答问题,只是一个简简单单的‘冈’。

这是要自己表明立场吗?跳得也太远了。韩冈想着,同时说道:“臣未见两者奏章,不敢妄议是非。不过以臣之见,宣抚司和前线大将,一为帅、一为将,对战局的看法必不能完全一致。若无大的参差,当在情理之中。”

河北可能得全?赵顼活动着手指,又跳到了后果上。

这是要否定向皇后的决策吗?向皇后低着头,脸对着床铺内,让人看不见他脸上的表情

韩冈立刻道:“以直报怨,以德报德,圣人之教也。辽人既然背毁盟约,中国也不能任其猖狂!否则辽人得寸进尺,不仅是陕西,河北、河东也将再无宁日。”

“臣亦为此担心。辽人造衅,理当回击。可就怕溥乐城救下来之后,宣抚司那边还是不依不饶。”耶律乙辛的尴尬地位,让他必须维护自己的声望。可大宋这边还没有做好全面战争的准备。事情若当真到了不可挽回的地步,先倒霉的肯定是河北。作为两府中唯一的河北人,韩绛当然担心家乡,不过他虽然顺着赵顼的话,却依然死扣着陕西宣抚司,“辽人弃韦州而攻溥乐城,其实还是有分寸的。只是讹诈,并不是想毁盟。宣抚司遣种谔救援溥乐城,不知之后是否有应对之策?”

溥乐城是边境上的军城,周边没有村寨。辽人来围攻,只要城池不破,就不会有太大的损失。辽人的本意还是给东京城施加压力。可如今种谔领军援救溥乐城,如果仅仅是驱敌还好说,要是他将南下的辽军一股脑都解决了,或是杀伤过众,耶律乙辛可就没办法压制国内的激进势力了。若种谔一时兴起,进一步杀进兴灵,战争便无法避免了。

韩冈不会代吕惠卿答话。只是韩冈没开口,章敦则道:“此事可移文宣抚司。或是遣使问询。”

赵顼没写字了,手略略抬高了一点,指了指韩冈。

“当付有司。”韩冈照样推掉,不过看了看赵顼和皇后,他就更加明确地表明态度,“关西的臣子,自寇准寇忠愍以下,无一人主张对外敌委曲求全。臣亦不例外!”

韩冈说得有几分自负,其实是为了种谔,但这的确是关西人的骄傲——至于陕州夏县的那一位倒是不能算。陕州那是中条山外,早出潼关了。

没人对韩冈的态度惊讶,陕西人一直都是对外敌强硬到底。连大儒张载都曾打算领兵开拓河湟。这是在帮种谔说话了,只是韩冈用的理由,却让他对种家的袒护显得是公心而不是私心。

赵顼的手指停了下来,半天后不见动作,也没人主动开口。

“官家是不是累了?”皇后低声问着。

赵顼还没反应,蔡确就已接上皇后的话,“深夜劳神不利御体,请陛下稍歇,臣请明日再聆听圣训。”

赵顼字写到哪里,宰辅们的话说到哪里。没有一个主动提起国事。皇帝的病情彻底好转的可能性不大,想要重新得掌大权,说不定就要跟皇后为敌。赵顼的态度诡异,看出来的不少。可谁知道赵顼还能活几日,这时候开罪了皇后,日后可是难有好下场。

如果排除其他因素,单纯的在皇帝和皇后之间做个选择的话,他们多半会选择没那么强势的皇后——估计只有王安石会除外——自觉或是不自觉的要将赵顼排除在外。

韩冈微微而笑,幸亏现在台面上大多是跟他脾气相近的。
