\section{第32章 吴钩终用笑冯唐(五)}

王文谅战死的消息,很快就传回了泾阳县中的帅府行辕里。私下里还有燕达的抱怨,通过游师雄传到了韩冈的耳朵里——没事给城里的叛军加士气做什么,嫌朝廷钱多,围城事少吗?

王文谅的战死,都在意料之中,也没什么废话。韩绛派了人去整顿蕃军残兵,防着他们作乱,又让人送了酒菜去善加抚慰。倒是没让韩冈去照看跟随王文谅出战的蕃军伤兵,也是怕出意外,韩冈自是不会反对。

至于燕达那边的抱怨,韩冈也只能苦笑,但也燕达也只是私下里抱怨而已,以他的才智,要看不透其中的问题,那就有鬼了。

不过从表面上看的确是他韩冈推荐的王文谅。王文谅战败身死,从官场规则上说,他少不得要摊上一份罪责。但帅府众人都知道这是什么情况,其中的缘由也是能半公开的挑明到天子那里,就算有人故意跟韩冈过不去,天子也都不会让这份罪名落到韩冈的头上,别提还有王安石和韩绛。

结束了没有什么新意的军议,韩冈回到了分配给他的在泾阳城中的驻地。是县城东南的一处寺院,不大,但颇有些年头了,院中的几株老松有尺半粗细,想来两三百年总该有了。本来应该是个清净的去处,可最近被宣抚司中的官吏占去了大半,倒把寺庙的主人挤到了柴房、厨房里去安顿。

寺庙里的和尚心情如何,韩冈没兴趣关心,他被人领着,到了安排给他的厢房的时候,种建中和种朴都在院中。三人一起来的,便被安排在一间院子里,而种谔却是睡在营地中。

韩冈和种家兄弟所在的这间小院,天井只有两丈大小,韩冈住了东厢,种家兄弟睡在西厢,正屋则是游师雄先住上。正是因为游师雄安排,所以韩冈才会住进了这里,不然就去临时的疗养院去住了。

不过游师雄现在却是又去了前线的燕达那里听候指挥,他现在颇得韩绛、燕达看重,许多事都压在他身上,天天忙忙碌碌的。

韩冈回来得正是时候,种朴和种建中正在房中吃喝。在桌子上摆了不少酒菜,驴肉、羊肉,还有烧得正好的鸡鸭,几乎都是荤的,五六斤一坛的酒也喝了近一半去。

虽然因为拓土横山的战略宣告失败,种谔今次肯定是要被降罪,但此败非战之罪,甚至斩获数量比历次大捷都多,天子也好、朝堂也好,想来都会体谅一二。而且事已至此,再有什么变故,也只剩直面而已。所以种建中和种朴便放开来喝酒吃肉,也不去想多余的事。

听到韩冈回来的动静,种家兄弟就把他来过来一起吃喝。韩冈也不推辞,他的肚子也饿了,径自扯了凳子坐下来,在旁服侍的土兵拿了干净碗筷。

坐下来被敬了一杯酒,吃了两块烧驴肉,就听着种朴说:“玉昆,你今天可露了大脸,一句话就把王文谅那鸟货送去投胎了。再没别人有这本事。”

“这一招上过阵的哪个想不到,有多少人做过要不要我点出来?”韩冈端着酒碗笑着反问:“堂上都在看韩相公的笑话,就任凭王文谅乱攀扯,连令尊都是。小弟要不出头去说,王文谅这厮还不知要蹦达多久!”

“都说不提这事了,还提什么。”种建中在旁说着,“王文谅是惹人厌,吴逵也的确是给他逼的。但罗兀那里好歹没丢人,砍了两千多首级回来,今次韩相公就算贬官,也不会贬得太厉害。”

“总管也当无事。”韩冈略一点头,韩绛不会多重的处罚,那么种谔更不会有太大的事。一切都能推到王文谅和吴逵的恩怨上,现在王文谅为国尽忠,罪名就全是吴逵的了,“环庆路的事都跟总管无关,又有罗兀城的功绩在……”

“也多亏了玉昆。听十七哥说,玉昆你在罗兀的那段时间,运筹帷幄,军心士气大振,梁乙埋几次大败,玉昆你出了多少力!”种建中举碗敬韩冈,“就祝玉昆能鹏程万里、青云直上。”

“对,当敬玉昆。”种朴也举杯相和。

“罗兀城一事谁没有出力?嵬名济是怎么上当的?岂是韩冈一人之功?”韩冈给碗中倒满了酒,“要庆贺也是三人一起。”

酒碗一碰,三人兴致高昂的对饮了几杯。

放下碗,韩冈才又道:“不过封赏也好,责罚也好,都要等咸阳城里的麻烦事都给解决了,才会有余暇去提。”他叹了口气,“也不知要围城到什么时候,左近的地全都荒了。”

“过几天就该开始挖地道了吧?反正赵郎中除了照葫芦画瓢,也没别的本事了。”种朴的话把赵瞻埋汰得厉害。只不过赵瞻在咸阳做的,也的确是当年明镐、文彦博平定贝州之乱的翻版。

当年弥勒教王则起兵叛乱,占据了贝州城。前后两任主持平叛之事的明镐和文彦博,就是采用先筑墙围城,然后再挖掘地道,最后用了近四个月的时间,终于把孤城贝州给攻破了,而后贝州被改名恩州,换了个吉利名字,直到如今。

有成功的先例在前,赵瞻便有样学样,只是这么做,拖延的时间可就长了。

“贝州无论是从粮秣兵械的数量,还是城防的完备程度,都远远比不上咸阳城这座长安的北大门。而且城中的叛军可都是精锐,不是几十年没打过仗的河北禁军可比。真的这样磨下去,一年半载都有可能。”

听到韩冈这么说,种建中也点头表示赞同,“吴逵也不是蠢货,贝州怎么败的,他这个做都虞侯能不知道?看到城外一圈围墙,就该知道下一步该怎么防了。”

“赵郎中尾巴一翘,吴逵就知道他要拉什么屎了。”种朴与韩冈早已惯熟,当着他的面毫无顾忌的嘲笑着赵瞻。

“筑墙围城当真是失策啊……”韩冈叹了一口气。

赵瞻这一闹,今年白渠粮区怕是要闹饥荒。而且一年灾往往是要三年去补,陕西的常平仓储备两三年内眼见着都要吃紧。虽然陕西诸路战略重心西移如今已经可以确定,但没有了关中的支持,等于又是一条绳子拴到河湟开边的脖子上。也不知古渭那里屯田和市易的事办得怎么样了。

种建中道:“方才我是听人说了。赵大观【赵瞻字】这是两害相权取其轻。把咸阳城围起来,耗用民力是不小,还有可能有灾荒,但若是让叛军逃出去,散诸四野,兵灾如焚,当会比现在闹得更大。”

“那也要吴逵能冲得出去才行。前日咸阳城下的一场火,如果不是赵郎中乱来,哪会被烧去那么多精锐?!早被死死的围在城里了。”种朴分外看不起赵瞻这等乱指挥的文官,“燕达也算是有点本事的,全让他来指挥,咸阳早被打下来了。今天王文谅可都上城了,除了赵郎中,谁没有看到?!”

王文谅领着一群蕃兵都能一举上城,其实这就是一个信号,吴逵手上可用的兵力实在太少了。真的要攻打咸阳城的话,以现在围城的兵力来算,已经是绰绰有余了。只是韩绛这位宰相不开口,其他人也压不倒赵瞻,只能任他瞎指挥。

韩冈不知道韩绛还会忍耐赵瞻多久,可别看赵瞻虽然现在插手了许多事,但韩绛真要计较起来,他只有靠边站的份。王文谅战死了,兵败的瓜葛韩绛能洗脱不少,现在就看他何时能振作起来。

咸阳不是贝州,陕西也不是河北,乱的时间不能长。要是真的拖上个几个月,等党项人舔好伤口,就要杀来大宋这边来给自己补血了。更别提契丹人,他们趁火打劫是有一手的。再继续拖延下去,会不会变成招安叛军的局面谁也说不准。

喝了半夜的酒,三人也就散了。第二天起来,韩冈先去宣抚司点卯。拜见了韩绛、见过了赵瞻,接下来他便跟着种谔率领的鄜延路大军,一起向咸阳进发。

平叛主力现在皆在咸阳城外,韩冈照常理也是得在咸阳城外大营建立他的随军疗养院。

围绕着咸阳城的一圈围墙,已经垒到了近两丈高,厚度与城墙没有区别,与本来的咸阳城墙的距离大约有百步左右。看这架势,大概是要给咸阳弄出个内外城来。就是一丈多深、两丈多宽的壕沟挖在围墙内侧的这一点,让人觉得头疼。

种谔带兵过来,与正在领兵围城的招讨使燕达会面。因为郭逵的缘故,两人素来不和,见了面也只是稍作寒暄。不过燕达有个好处,他虽身为招讨使,统管平叛军务,但并没有自高自大的,把与他同为一路副总管的种谔,当作下属来看待。否则,以种谔的脾气,多半大帐中就有好戏看了。但两人之间,还是仿佛有电光雷鸣,隐隐交锋之势。

韩冈自有正事要做,没有在大帐看热闹的意思。向种谔、燕达两人请示过,便径自去了随军的疗养院中。

