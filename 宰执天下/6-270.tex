\section{第36章 骎骎载骤探寒温(六)}

宗泽刚刚走近锅炉房,一阵热浪便迎面而来。

两名锅炉工,正站在不断飞窜出的火焰前,一铲一铲的将煤块送进炽热的炉膛。他们皆**着上身,黑色的煤灰将肌肤染得看不出原色,而不断流淌下来的汗水,又在灰黑的底色上冲出一条条白色的印痕。

浓烟自屋顶上的烟囱里滚滚而起,烟熏火燎的气息,即使隔了一层口罩都遮掩不住。

蒸汽机运转的声音更是震耳欲聋。轰轰轰轰,仿佛站阵前的鼓点,不断重复着单调的节奏。

其实这一切,宗泽都能忍耐,可还是有一件,让宗泽对这间被铁与火所充满的屋心生畏惧。

不论是飞速旋转的铁轮,还是不断屈伸的连杆,都让宗泽平添几分怯意。只是他所了解的,光是因为机械故障导致的零件飞出,这两年来就造成了不下十宗血案。

有一击毙命的,也有在医院病榻上缠绵多日最后咽气的,还有一个被打碎了头盖骨,却奇迹一般的活了下来。当那人脱下铁头盔,将被摘去碎骨,以至于凹陷下去的天灵盖露出来,自诩大胆,过去也的确从来未曾畏怯过的宗泽,次日惊醒时浑身都出了一层冷汗。

但最让宗泽畏惧的还是蓄满了滚水的锅炉。锅炉的强大压力,让锅炉变成一枚填满火药的炸弹,由此产生的伤亡,并不比火药工坊少到哪里去。而且为了能够造出功率——这个新词依然是韩冈所拟——更大的蒸汽机,锅炉的压力也越来越大,两个大气压的蒸汽机已经准备大规模制造,三个大气压也已有了第一台实验机,五个大气压的蒸汽机则刚刚开始设计,但未来,还将要有八个、十个大气压的蒸汽机。

仅仅一个大气压,就已经让内部抽成真空的两个半球,用八匹健马也拉不开——五年前的这个实验,让世人见识到了何为气压,以及气压的力量。

两个大气压,业已造成了数百人的伤亡。那么,三个、五个,乃至八个、十个大气压,又会造成怎样的后果?

宗泽听韩冈说过,增加一个大气压,相当于潜到水下三十尺,增加十个大气压,是水面下三百多尺的压力,大概是将两丈多厚的水银,或四丈多厚的铁板压在身上的重量——足可以将人骨碾成碎粉。

十个大气压的锅炉如果爆开来,那样的画面,宗泽根本不敢想象。

吐火冒烟、能发出雷鸣般的吼声、而且还会吞噬人命,这简直就是故事里的凶兽。

如果工厂里面都是类似的环境,也难怪明教妖贼只是稍稍煽动了一下,丝厂的工人就开始造反了。

当然,现在蒸汽机还没有投入工厂使用。但已经很恶劣的生产条件,加了蒸汽机之后,那可就是变本加厉的糟糕了。谁能忍受得了?!

一直以来宗泽都很支持韩冈的一系列治政方略,也认为治国之要最基本的就是让百姓吃饱穿暖。达到温饱了,人心方能安定。人心安定,方能做到政通人和。

但如今天下的变化,越来越超出他想象的极限,这个世界将会变成什么样?

千里之行,三日而返。千石之物,一车可载。这些都是直到十几年前,任何人也无法预料到的,只在《域游记》有所预言。

可是在《域游记》,也没有工厂被烧,工人困苦的章节,反而充满了对工人生活富足、稳定的描写。宗泽没有去过两浙,但他家的亲友,可是有人亲自去了几次工厂。在信的述说里,丝厂之的工作环境,已经远远突破了宗泽预计的下限。

或许两浙丝厂之变只是歪嘴和尚念歪了经的结果,韩冈主导下的棉纺工厂正如他书描写的一般上下一团和气,家家吃饱穿暖。

可更大的问题是工厂的规模。

若是有人说,只要有需要,朝廷从开封铁场拉出三五千人的军队,宗泽一点都会不惊讶。因为开封铁场之的五千多工人,基本上都是身体强健的成年男丁,为了生产上的安全,举手投足都有规矩约束,能够轻易的适应军的管束。

虽说比不上开封铁场,但普通的一座工厂也有上百人,若是有个百十家工厂,那就是上万人了。这些工人都接受过了纪律的约束,比散漫的农民要容易训练十倍。一旦乱起来,岂是农民比得上?佃农闹佃时也的确会有骚乱,但绝无可能达到工厂的规模。

并不是宗泽不能理解韩冈的治国方略,就是因为太了解了,才让他产生了对未来的惶恐。未来就像是面前的这座机房,让他一时间望而却步。

但就在宗泽犹豫的慢下脚步的时候,韩冈已经轻快地走进了机房之。示意两名锅炉工继续铲煤,也不顾飞扬的尘土,很是愉快的打量着这台已经稳定运行天半的机器来。

这些天来,韩冈的心情显而易见的好。

两浙事变之后,乍听闻伤亡,他的心情的确是有些沉重。能够将责任归咎于明教固然是韩冈所乐见,可丹徒县民伤亡如此惨重,却非其本愿。当时安排一部京营禁军南下时,完全没想到事情会恶化的那么快,爆发得那么突然。

明知日后类似的事情只会更多,但情绪这回事,总是不受自己控制的。纵能收敛得旁人完全看不出来,可自己总是明白的。

不过能这么简单的解决工厂纵火一案,以及明教教众叛乱,韩冈也很是欣慰。

虽然润州上下没能阻止事情的发生,但将之扼杀在襁褓之,也算是应对及时了。若是给了卫康一点成长的空间,或许就是一个波及一州甚至一路的大乱。

韩冈可是还记得几十年后的方腊是怎么兴起于江南,能进教科书的农民起义,规模绝对不会太小。

就算如今丝厂兴起对江南百姓的伤害,远不如那位今世没能出生的画家皇帝的花石纲,可立国百多年来做积累下来的矛盾,爆发出来,一样能闹得江南天翻地覆。,

这一回叛乱者肆虐的范围,仅仅是润州治下一县,当真是不幸的万幸。否则韩冈也免不了有些被动。

只是欣慰和庆幸,还是抵不过乍闻百姓伤亡的沉重。

直到蒸汽机在进行了小幅改进后,在实验时技术指标又有了进步,韩冈心情方才变好了起来。

经过改进的蒸汽机,如今最长的运转时间,已经接近十天。而平均正常运转时间,最近的几台实验型号,也能维持在一天以上。绝大多数导致停机的故障,也能在一个小时内修好,然后重新开始运行。

在绝大多数工厂都没有夜班的情况下,作为动力源的蒸汽机最长的运转时间也只需十一二个时辰。如果现有的蒸汽机量产后还能保证现在的质量水平,那么纺织厂、钢铁厂,都可以将水力机器送进垃圾堆了。

“恭喜相公了。”

韩冈闻声回头看了一眼,是宗泽进来了。

他方才注意到了宗泽的犹豫,不过看起来还是克服了恐惧走了进来。站在这种并不稳定地机器前,的确会让人心平添一分惧意的。宗泽的反应十分正常。

韩冈又转回去看蒸汽机,“不,这还远远不够。”

“不是可以上车了吗?”宗泽问道。

“上车有些希望,但上船就不知是何年何月了。蒸汽机需要不断加水,船行海上,哪里来的净水使用?”

韩冈想要的是一台能支持千吨轮船穿过太平洋的蒸汽机。但现有的蒸汽机的结构来说,完全不能适应海船上的条件,必须采用新的架构来设计新型蒸汽机。

“不过最重要的是能够驱动纺纱机、织布机、重锤,从此工厂不用再受限于水力了。”

“……可是相公,”宗泽犹豫了一阵后说道,“工厂增多,日后难免再一次润州之乱。”

韩冈点头,“此事我当然清楚。”

南方的反贼近乎于笑话,但工人阶级的力量,却绝不是笑话。迟早有一天,工人们的怒火会再一次将工厂点燃,甚至席卷一地。但到那个时候,谁也不敢开口说:呐,我们干脆把工厂都关掉吧。韩冈的保护,也只需要维持到工业发展壮大的那一天。

“现在许多地方的矿井,矿工们罢工的情况时常有之,但解决问题的办法,还是坐下来慢慢谈。除非要价太高,除非矿工们破坏工具,否则哪一位官员都不会轻易动用武力。而矿工们也不一样会克制。因为双方都知道,把事情做绝了会是什么结果。他们有着太多的经验教训。”

类似的话,宗泽听韩冈说过一些,就在他南下之前。

“相公的意思,就是南方丝厂的厂主和工人,安坐下来谈判的经验太少了?”

“当然。之前跟汝霖你说过吧,办工厂的目的是生产和赚钱,用两浙的压榨手段,远比不上善待工人得到的收益。记得当时,我还举过关系棉纺织工厂的例。”

棉纺织工厂的工厂主,都是雍秦商会的成员。雍秦商会的这些工厂主们相互之间都有同进同退的协议,工人们的工钱,不低于某个限度,当然也不能于超出预定的上限。

除了按照产品数量和质量确定工资等级之外,工人们的年资,以及技术水平,都能决定一部分工资的高低。若是利润高于预计,还会下发一部分红利。

要是工人们对生产和酬劳有什么意见,工厂的管理者也会跟他们坐下来慢慢谈。不会强加什么罪责。

只有敢于带头闹事,违反工厂规定的工人,才会毫不留情的被开除,而且任何一家工厂都不会再招他。

棉纺织工人的收入,比起做农活要高得多,收入高,待遇好,因而工厂的气氛也很好,工人们也都有干劲,工厂主们的得益自然更多。
