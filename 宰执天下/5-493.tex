\section{第39章 欲雨还晴咨明辅(22)}

章惇啊的一声轻叫,一下就明白了韩冈的打算。

这是要借钱啊。

章惇对面,蔡确、曾布也是一脸恍然。

听向皇后提到的那几句,就可以知道,韩冈本意根本就不是要朝廷给内藏库开什么借据。

在表面上,那个什么国债,的确是给内藏库的凭证。但实际上呢?十万贯、一万贯,这样的定额债券,又有抵押,又有还款期限,还有利息,完全可以卖给其他人,甚至是当钱来使用。这在民间都很正常,那些质库的押票,都能拿去换钱的。抵债的时候,借据也同样能算钱。

当然,十万贯一份,除了天家,天底下没人能买得起,就是买得起也不会买、不敢买。一万贯一份,能买的多了,可一时间却没人会买。但谁说只能是十万贯和一万贯的?更少一点呢,一千贯、一百贯,都是可以的。

只要朝廷首开先河,现出来做了样子,让人信服之后,就可以放开手脚,去发行国债。大不了用盐来抵押,直接拿债券去换盐。那些给付入中商人的盐钞,也一样用盐做本金。

这就是韩冈的打算吧?

朝廷行事,最重要的还是一个信。

这一奏议从里到外都体现了韩冈最强调的‘信’。

朝廷诚信,百姓有信心。从此以往,只要发行一界国债,就能有个几十万贯的现钱,那样朝廷做事也方便了。而且好借好还,再借不难。

想到这里,章惇悚然而惊。

万一还不上怎么办?就算有抵押,但盐也不是说有就有。生了乱子,可就闹大了。

‘让吕嘉问跟韩冈打擂台吧。反正最后还是要他们来拍板。’章惇想着。

“殿下。防微需杜渐。”吕嘉问果然抓到了重点,“所谓债券,并非金铜,只是一张纸而已。可以伪造,可以损坏,也可能会不小心丢失。万一除了这些事,怎么办?”

铜钱都有伪造,国债怎么防伪?若是债券被毁坏、遗失呢?还有万一遇上大事,朝廷大肆发行债券,甚至强行从富民手中借贷,事情可就难以收拾。

“仁宗时,元昊起兵。关西兵事紧急,朝廷为了运送更多的粮草到边地,便给付入中的商人各色钞引,凭据可以到京城换钱换盐换茶,可商人们到了京师后,朝廷却因故使得钞引不能及时偿付,朝廷信用毁于一旦。”吕嘉问语气沉重,“殿下,立券事小,而信用事大。不以账目,而用国债,臣亦不敢多言。但国债并非借据,又岂是专给内藏一家,百姓亦会受累。曰后国家之乱,由此而启。”

曾经一门心思要变法聚财的吕嘉问,现如今满口的国家未来,就跟当年的旧党一样,其维护既得利益的态度十分明确。

“殿下。”曾布站了出来,“不如招韩冈上殿询问。”

曾布不喜欢韩冈,但对吕惠卿、吕嘉问,他则是恨。当年便是吕惠卿和吕嘉问联手,逼得他反戈一击,最后不得不饮恨出外。单单是之前将吕惠卿拒之京外,已经让他欣喜难耐,现在若再能给吕嘉问一记耳光,他不介意站在站在韩冈一边。

“韩冈已经请辞。”吕嘉问说道。

曾布瞥了吕嘉问一眼,就这么怕韩冈?

“为国事,须推脱不得。还请殿下速遣人招韩冈。”韩绛说道,“我等皆在此等候。”

韩冈很快就来了。

进门时看着就是三堂会审的样子。

宰辅们都看着自己,吕嘉问像是在发脾气。皇后依然在帘后,而赵煦在御座上坐了不短的时间,似乎是累了,看起来没什么精神。只是身子依然挺直,坐得四平八稳。看起来宫中的礼仪教育,将他培养得很好。至少没有需要议论的地方。

韩冈参拜过皇后和赵煦,向皇后就赐了他座位。但韩冈没有落座,拱手对皇后道:“臣本已在家待罪,岂可与宰辅坐而论道?”

“韩枢密?”向皇后吃了一惊,“枢密这是何意?”

“陛下,殿下。臣因罪无颜留居西府,请辞枢密副使一职。朝廷却至今不允……”

向皇后明白了,也即是说,除非答应韩冈辞官,否则他干脆就在这里装哑巴了。

向皇后久久的方一声叹:“……韩枢密既然无意留任,吾也不便强求。不过枢密在河东任上,拯危救急,解太原之围,光复代、忻,夺占神武,又多次大败辽贼。如此功绩,朝廷岂能无视,当加赠食邑四千户。”

满朝文武,有几个能像韩冈一样公忠体国?天子一逊位,立马就欺上门来了。现在太上皇跟死人差不多了,吕嘉问的做法就跟欺上寡妇门没两样,宰辅之中,只有一个韩冈出来主持公道,不管事成与不成,皇后肯定要对韩冈另眼相待。

但这个另眼相待未免过了头。

宰辅中一下有了搔动,他们都参与讨论过吕惠卿、韩冈和郭逵的封赏,知道他们三人现在具体的官职、差遣,以及其他一系列名爵和头衔。

韩冈现在已经是郡公了,食邑八千户,再加四千户就是一万两千户。依故事,食邑过万户必然要封国公。向皇后的心意也肯定是要将韩冈封做国公,要不然,也不会放下其他虚衔,直接先说加封食邑。

可这是在太夸张了,就是皇后愿意给,韩冈也不敢接受。

蔡确这个宰相现在都还不是国公。想要封国,要么是熬时间熬资历,要么是因故离任后得到朝廷赠与。宰相都不到国公,韩冈被封国,可不是要成众矢之的?

为了虚名而受累,实在太冤了。他连忙道:“殿下错爱,臣实不敢当。还请殿下收回成命!”

韩冈态度诚恳,向皇后皱眉想了下,“那就等之后再议……枢密……学士的奏章,吾已经看了。”韩冈正式辞官,她也换回了旧时的称呼,“国债之议,吾也觉得甚有道理。只是有些地方不太明白,枢密可否现在再给吾说一说。”

韩冈左右瞥了一下,宰辅们都在等他说话。也不再卖关子:“去岁的郊祀,之后又是半年多的战乱,战后的赏赐亦为数不少,还有现在的情况。内藏库多年所积,惟余空簿,此事朝中人尽皆知。”

“谁说不是。”向皇后叹道,“都没想过会这么少,都被搬空了。”

如果不是赶着要让赵顼退位,向皇后也不会面临现在的窘境。韩冈用眼角的余光,瞅着小皇帝。赵煦正瞥了一下嘴,只是很不明显。韩冈觉得,也许是有成见后的错觉。

将注意力从小皇帝身上收回来,韩冈继续道,“但内藏并非左藏,左藏也非内藏。当年太宗皇帝以左藏北库为内藏,内外之分由此而始。”

天子私产和官产不一样,并不是一回事。关于这一点,无论天子和朝臣都有认识。

当然,双方的认识是有差别的。在朝臣而言,天下都是皇帝的,朝廷要为天下用钱,怎么皇帝怎么能不管,所以没钱就往内藏库伸手。但皇帝想要动用国库,那就两样,不能用天下之财,供天子穷奢极侈。

而在皇帝看来,这是我的钱,不是官府的,外廷本有税赋收入,内藏库是为了兵事、典礼和救急用才设立的。国之大事,在戎和祀,加一个灾荒救急,当用在这上面。是国家的储备,是为国家大事准备的。平常的支出,当由外廷自行解决。

总之都是善财难舍。

一般来说,在过去,皇燕京是为一己之欲,侵夺国家财计。或大兴土木,或巡游天下,或封禅泰山,使得朝廷难以支撑,百官叫苦不迭。

现在之所以会反过来,是天子钱多,而国库钱少。富人不贪苦哈哈手中的那点钱,只想守好自己荷包。但朝廷百官哪个不是绿着眼睛,想从栓钱的绳子都断了的内藏库中掏钱?

“正是因为有内外之别,所以诸多封赏,都是从内藏库开支。如今新天子践祚,依例也当是自内藏库中开支。”吕嘉问反瞪着韩冈,半点也不退让。

“天子践位,不是国事,难道是私家事吗?难道不是一部分出自内藏库,一部分出自三司?如今内藏库的份不是说不发,而是三司要向内藏库借贷。却还不想留下借据。可公函往来也是要给回函的,人情往来更是要回书。难道回一份债券就这么难?”

“学士说得好。”向皇后点头,“从真宗皇帝开始,近百年了,每年都要从内藏库中给予三司六十万贯,以补国用。这还不够?几千万贯都给了,现在只是要个回执罢了,到底有什么不敢的?”

在向皇后的理解中,韩冈不打算让三司对内藏库再予取予求了。不能惯着他们随时的伸手要钱,账目也必须清楚。说起公忠体国,能体谅天家的难处,也就是韩冈了。

吕嘉问阴声问道:“嘉问只要韩学士说一句,国债是否仅止于内藏?”

“太上皇当年查内藏库账,发现内藏库财物进出无关防,便用李舜举守库,又曾诏江淮发运司,内藏库物移用,需关牒本库照会。太上皇这么做用意何在?一者,防盗,二者,明内外之分。”韩冈回道,“韩冈之意也只是内外分明四个字,至于其余等事,非韩冈所知。”

