\section{第28章 临乱心难齐(八)}

当一切敲定,窗外已经是雄鸡三唱。朝东的窗户,透进来清晨的霞光。

心神放松了下来,韩冈喝了口走了味的凉茶,看着尤是精神抖擞的大舅子,问道:“此事我们这边就算定下了,不知元泽你准备什么时候奏禀天子?”

王雱刚刚松懈下来的神经又绷了起来,苦恼的神色又出现在脸上,答非所问:“这件事不能瞒着天子。”

“自是当然!”

欺君乃是重罪,王安石和王雱都不至于犯这般愚蠢的过错。前面上书要在冬季开河口,又要造碓冰船,王安石在崇政殿中费了好一番口水,才让天子点头应允。现在回过头来,又变成了用雪橇运粮,出尔反尔,天子必然心有不快。

但如果瞒着赵顼不说,情况会更糟。这件事肯定要爆出来的,躲得过初一、躲不过十五。如果作为九五至尊,变成最后一个才知情,皇帝肯定会更为愤怒。所以必须要加以补救。

对于上位者来说,手下的人可以蠢,可以笨,可以有私心,甚至触犯法令条律,只要不太过分,还是可以容忍,但只有欺瞒蒙骗才是最大的忌讳,让人忍耐不得。

“但要怎么说还是得好生斟酌一番啊。”

王雱点头:“等回去后与父亲再商议一下。”

的确不好说。出尔反尔,下了决定后又立刻更改,这就叫做行事轻佻。世间对于宰相的要求,是沉稳、稳重,能如柱石一般稳定朝廷大局,面对危殆局面,也能将国事支撑起来。如澶渊之盟时的寇准,如曹后垂帘时的韩琦。朝令夕改的作风,出现在宰相身上,那就是要给人戳脊梁骨的。

王安石一向倔强,别说朝令夕改,在外人眼中,就是知错也不会改的,否则就不会有拗相公这个绰号了。现在他主动改弦更张,身上要背负的压力可想而知。

就要看看王安石要什么样的办法去取得天子的谅解和理解了。韩冈倒是老神在在,反正不管自己的事。何况以王安石几年来的君臣相知之雅,赵顼再怎么样也会对他优容一二,不过是丢点面子而已。

王雱也放下了这件烦心事,外在的面子问题不是关键,关键是先要将事情做好。先得有里,才能有外,“运粮上京,绝非易与。更别说还是用雪橇车来运送。不知玉昆是否有心转调六路发运司,主持其中诸事。以玉昆旧年在熙河路的表现,家严和愚兄也能放得下心来。”

到任两月就调离的前例有得是。认为韩冈到白马任知县就是为了来熬过一任资序的人,本来就很多,现在他转任也不会出人意料。但韩冈却无意改换职位。

简直是开玩笑!招之即来,挥之即去,他韩玉昆难道是王家养的狗吗?!

“先不说小弟资望浅薄,在六路发运司中根本毫无根基可言,短时间内根本使唤不动那一干官吏。且明春河北若有流民南下,白马县便会首当其冲。如今我在这县中也算薄有声望,就算有流民蜂拥而入,也能安排得下来,倒也不怕会出乱子。要是小弟离开,不知准备换谁来顶替?”韩冈反问着,又道:“不如这样吧,我来上书天子,将雪橇车呈递上去。至于后续的主持工作,还是要劳烦岳父和元泽你另选贤能为是。”

韩冈的推脱也不出王雱意料,叹了口气,两件事中,他也不能确定哪一桩更为重要。

“即是如此,那玉昆你就没有必要上书了。政事堂里肯定有过去熙河路呈上来的奏报,有关雪橇车的事也能找得到。”王雱笑笑,“当时没人放在心上,现在想起来了,重新给翻了出来——这等借口,想来也能说的过去。”

上书提议用雪橇车运送粮食入京,即便此事成功,功劳还是拿不到大头——六路发运司才是首功。但若是失败了,过错却要摊上大半——将责任对到雪橇车不堪使用上那是最简单的。韩冈既然不愿意参与进来,就没有必要让他冒这个风险,好歹也算是自家人。

“就让薛向来好了。六路发运司他管了几年,现在威望还在。让他来主持此事,不虞会有变故。”王雱说道。

“薛向可是三司使!”韩冈闻言惊讶不已。从六路发运司升到了三司使的位置上,现在难道要将他降回去?三司使可是大宋计相,六路发运使却是一个苦力活。

王雱微微一笑:“但他想入政事堂。”

说着他站起身:“时候不早了,愚兄这就要走。二姐现在就在家中,过两日,就将她们一起送来。”

天色已然大亮,带着韩冈画出来的图样,王雱就要告辞离开。有了图样在手,他并不担心打造不出来。

雪橇车仅是一个创意而已,但对于大宋那些手艺超乎后人想象的工匠们来说,他们也只需要一个创意。就像韩冈让人改造投石车,还有当初打造雪橇车的时候,他都是只提了几句话,熙河路的工匠们就将顺顺当当给造了出来。这些器物并不超越时代,仅仅是创意别出心裁,捅破了窗户纸后,将之付诸实现,一点难度都没有。

“那就劳烦元泽费心了。”韩冈瞅着王雱眼中密布的血丝,又道:“我还是让人找辆马车来好了,元泽你正好可以在路上睡一觉。”

推门而出,冬日的清晨,寒冷异常。可清寒的空气扑面而来,昏沉的头脑一下就能变得清醒过来。

韩冈唤了从关西带来的亲信去为王雱准备车马,又让厨中置办了早饭。半个时辰后,王雱带着一夜的收获,悄无声息的从偏门离开了县衙,上车返回东京城。

与披着连帽斗篷的王雱擦肩而过,刚刚走进偏门的诸立,又奇怪的回头向他盯了一眼。只是那人很快就上了车子,转眼就往城门处去了,让诸立没能在看清到底长得什么模样。

只不过这匆匆一眼,那人的面相就已经给诸立留下了深刻的印象。白马县的诸押司怎么看都不觉得与身上所穿的庶人服饰相匹配。气质差得太多,应该是个官人才对,而且官位绝对不低。一般的选人,若是不穿上官袍,就跟普通人没两样。只有在官场浸淫日久,颐气使指惯了的高官,才会有让自己在一瞥之间就为之胆寒的气质。

诸立在县衙中,三教九流的不知见了多少,论眼光他有足够的自信,绝对比如今坐在县衙中的韩冈都要毒。既然自己看着像是个官人,肯定是个官人。就是不知道是有什么大事,竟然让一个地位不低的官人纡尊降贵,装扮成庶人来夜访县尊。

诸立用脚趾头都能想得出来,肯定不是件小事。对于他们这等地位卑微的小吏,有些事还是不知道的为好。只是诸立却有心一探究竟。

韩冈如今在白马县已经是说一不二,给诸立的压力远远超过过去三十年,来白马做知县的几十位官员。让他睡觉都睡不好。若能抓着韩冈的把柄,就算不用来对付这位韩正言,能拿来当个舒服点的枕头,让自己睡个安稳觉也是好的。

诸立心中暗暗计较着,该怎么从韩府的下人们那里,将昨夜到访的客人身份给打探出来。边走边想的他,很快就到了偏厅中。

是韩冈昨日让诸立一早来县衙,他有事要询问。

由于陈举的缘故,他对县衙中的押司的感觉并不好,诸路这位押司当然也就在韩冈上任后,就立刻打入了另册。不过自他到任之后,诸立为人勤勉,接到的命令都好不推诿拖延的给完成。这让韩冈对他感官渐渐好转。

不过这段时间来,韩冈也已经打探得明白,诸立在白马县就是条地头蛇。陈举在成纪县的地位,就是现在诸立在白马县中的地位。他之所以老老实实,是因为自己能控制得住场面,加之身份地位太高的缘故。要不然,陈举能做的事,诸立也能做得出来。

诸立垂着手毕恭毕敬的站在韩冈面前,韩冈用手握着盛了滋补药汤茶盅,掌心传来的热流,让韩冈全身都暖和了起来。

等着药汤稍稍冷下来的过程中,韩冈问着白马县衙的押司,“诸立,你家是不是开的粮行?”

诸立心神一紧,但神色保持如常,“回正言的话,小人家中的确在城北门内有一家粮行。”

“这些天来,白马县的粮食可是噌噌的往上涨,这其中,诸立你家的粮行功不可没啊!”韩冈笑眯眯的说着诛心之言。

诸立连忙跪下,趴在地上连连叩首:“正言明察。粮价不是小人一家涨,开封的行会一起都要涨。若是哪一家敢不从,日后不论买粮卖粮都别想了。”

韩冈冷着眼看着诸立为自己辩解。这个惯使风的老吏,当真是能屈能伸,姿态摆得这么低,但实际上却不肯让半步。

“这事我也知道,只是问问而已。”韩冈说道。

