\section{第九章 拄剑握槊意未销(八)}

【昨晚有事,没能更新。抱歉了。】

图穷匕见。

韩冈兜兜转转终于还是绕回来了。

稳扎稳打,缓攻兴灵。韩冈战前的建议,在战败后又被重新提上台面。而曾经拒绝了这份建议的自己,又会在外面落到什么样的评价?

酝酿在赵顼心中的情绪,悔恨、愤怒、羞恼混杂在一起,一点点的沸腾起来。

空寂的崇政殿上,坚持己见的年轻臣子垂着双眼,摆着一副谦恭姿态,一盆盆冷水却浇了上来。

“不知韩卿的心中可有具体的方略?”赵顼忍下一时之气,向韩冈垂询。

“具体的方略当征询领军将帅的意见。”方略可以在朝堂上议论的,但具体的战略、战术还得让精兵强将去处置,“不过以臣之愚见,最好暂时将鄜延军的兵力收缩回来,保住夏州、银州一线。河东军也暂且退回到弥陀洞为上。”

“其他军中都放弃?盐州、宥州都已经在官军的手中了。”王珪立刻出声反对,“可知盐州的青白池盐有多少是西夏国中所用?有多少是通过回易来赚钱?怎么能就此送还给西贼。”

“送还之前毁掉就是了,去向问老盐工,看看他们有什么招数。”韩冈回道,

“那宥州呢?这么多的土地难道要送回去吗?”王珪质问。

“无妨。迟早能夺回来的。有了一个立脚点,西贼才能源源不断的派兵过来,但眼下连番大战,当地存粮早已消耗殆尽,如果西贼来攻,能否越过瀚海不说,就是在银夏之地,也是没有粮食可以补给的。西贼只能设法速战速决。但官军稳守城池,西贼速战的下场,将会是灵州之役颠倒过来的结果。”

王珪看到赵顼深思起来的表情,便知道事情不妙。如果按照韩冈的方略将银夏保住,自己作为宰相的立场就有问题了。

天子的愿望只有一个,灭亡二虏,谁能帮他做到,他就会支持谁——至于会不会过河拆桥,那是日后的事——为了富国强兵,天子曾经对王安石言听计从,如今灵州战败,只要能挽回现如今的颓势,终究还是会听从韩冈的意见。

“河东怎么办?”王珪终于寻到了一个借口,这是他之前绝不会去做的。

“雁门关没那么容易攻破。尤其是在官军已经提高戒备的情况下。”韩冈回道,“要不然之前也不会那么放心让李宪领兵参与进攻取西夏。”

赵顼终于还是给说服了,他的目的就是灭亡西夏。韩冈的方略虽然缓慢,但终究还是往那个方向去的,“就依韩卿之言。”

韩冈心情一松,终于还是控制了最后的战局。这样一来,至少能帮种谔一把,否则权衡之后下令退兵,种谔一辈子都不会再没有机会。

虽然为此开罪了天子,但等到成功的时候,这点怨气很容易就能化解。

……………………

从前一日的早朝时开始,朝臣们就在私下里议论天子对灵州之败会做出什么样的反应。是退兵还是继续下去?

对此有着各种各样的猜测。有人认为天子会选择坚持到底,继续打下去;但也有人认为在契丹人虎视眈眈的情况下,赢了还能继续,输了就只有撤兵一途了。而到了最后,从两府和学士院中传出来的消息,算是在情理之中,也有几人料中了。

但还是有人感叹,“终究还是韩冈赢了。”

的确是韩冈赢了。

利用灵州之败,韩冈再一次宣告了他对西北战事的权威。与西北战局有关的问题,眼下的朝堂,只有向韩冈咨询。

蒲宗孟很清楚韩冈在西北战事上的权威,但当他听说了韩冈在崇政殿上究竟是怎么对天子说话的时候,却忍不住爆发式的狂笑起来。

韩冈一世聪明,偶尔糊涂起来却能要命。只要推上一把,或是漏几点火星,便能让韩冈就此一蹶不振。

蒲宗孟今日正在崇政殿中撰写诏令,却恰好有一封是给河北郭逵。他是个急性子,便闲闲的添了一笔,赞他有先见之明,料敌观己如烛照龟卜,军民共服。

郭逵和韩冈当初同论不当急攻灵州,蒲宗孟特地将这一点给点出来,当然不是为了说郭逵的好话。

项庄舞剑,本就是意在沛公。

当然,这话说得很隐晦,不是心有定见是看不出来的。但足以在天子心中定下一根钉子,日后再一步步来。不管能力如何,开罪了天子,让皇帝心生芥蒂,才能再高也在朝中待不下去。韩冈就是自视太高了,要不然,也不会干脆了当的让天子下不了台。

蒲宗孟将起草好的几封诏令送了上去,等着天子的评判。

赵顼翻着蒲宗孟刚刚写好的文字,突然间就停住了,半天也不见动上一下。

蒲宗孟没有得到他想得到的反应,照理说,稍稍看过就签押才是正常的情况,自家的私心这时候应该看不出来才对。心存犹疑,遂偷眼看去,却正对上一双冰冷的眸子。

赵顼从上而下的眼神仿佛是极北寒冰,从嘴里挤出的话更是冷得如同寒流一般,“朕看起来就那么像袁绍吗?”

蒲宗孟张口结舌,他想不通,整件事不知怎么跟袁绍有了牵扯?天子到底看没看明白自己力透纸背的用心。

一身冷汗的从崇政殿中出来。跨出殿门,太阳一照,竟然一阵头晕目眩,双腿如同得了疟疾一般抖个不停。他心中惶惑不安,更是满头雾水,天子提袁绍做什么?又不是诸子争位,意欲立幼子的情况!

难道……蒲宗孟忽的一念闪过,顿时心头一悸。跨进学士院中,他勉强将神色恢复正常。像是什么事都没发生过一般,随手招来一名小吏,吩咐道:“去找部《三国志》来。”

学士院中藏书甚多,翰林学士们撰写诏书时,时常都要检查典故用得对错与否。没过多久,六十余卷的《三国志》便被全数搬来。

国子监版的史书在外面都是论贯卖,不论刻板、印刷,还是纸张、装订,都是第一流的,质量远比东京、杭州印书坊的版本要好,更不用提以粗制滥造著称的福建版。哪一个读书人买回去,不是小心翼翼的珍藏起来?读书时更是轻拿轻放,唯恐损了纸页。

但蒲宗孟挥退了搬书的小吏后,却是从一堆书中一本本的捡起来哗哗哗的急速翻着,寻找自己的目标,毫不在意书册是否会在粗暴的动作下损坏。

从《魏志》中找出了《袁绍传》,蒲宗孟还没翻上两页,脸上最后一点血色就都退了个干净,果然是这一条!一卷书从手上掉落袭来,书页舒展,几行正文暴露在阳光下——绍军既败,或谓丰曰:“君必见重。”丰曰:“若军有利,吾必全,今军败,吾其死矣。”绍还,谓左右曰:‘吾不用田丰言,果为所笑。’遂杀之。

蒲宗孟愣愣的坐了半天,听到院中的声音才一下惊醒过来。不过他心中终究还有着几分侥幸,从亲信中挑了一个办事得力的,“去打探一下最近有什么比较特别的事发生了……”

亲信派出去了,蒲宗孟等着他的回复。如果没有什么特别的,或许……这仅仅是天子随口的话而已。

蒲宗孟抱着三分期盼,幻想着事情能如他所预料的情况发展。一个下午都无心做事,在他的案头上堆满了亟待处理的公文,效率慢到已经可以说蒲宗孟是尸位素餐的地步。

派出去的亲信到了蒲宗孟散值时都没有回来,一直到了深夜,他方才悄然回到了蒲宗孟家中。到了被找到的一家之主面前他就低声道:“学士,好像有些不对。灵州兵败的消息已经在京城里面传开来了,都在议论此事。不过外面现在也在议论还关押在台狱中的苏直史的案子。”

“怎么议论的?”蒲宗孟连忙追问。

亲信道:“外面传言说朝廷出兵前,苏直史曾经说过此番用兵必败,所以恶了天子,被关进台狱。现在果真战败,天子无颜见他,据说已经降旨要将其赐死了!”

这不正是袁绍、田丰的故事吗?!蒲宗孟手足冰冷,不过改个人名而已,根本是一模一样!

难怪天子会质问自己的用心。袁绍、田丰的故事套在苏轼身上还有点勉强,套在韩冈身上却是正合适。

皇城司不是聋子、瞎子,传言必定早已传到天子耳中。正是为此而恼火的时候,自己的话却让天子产生了不该有的联想。

完了,完了。

想透了一切的蒲宗孟如同五雷轰顶,自己竟然在天子面前成了进献谗言的小人,这是蒲宗孟怎么也没能想到的。纵然实情没差太多,但谁也不会愿意自己跟小人扯上关系,

一旦奸臣的形象在天子心目中留下根,日后便会是麻烦不断。这个罪名可以说是毁了自己多年的努力。恰到好处的一道流言,将自己拍翻在即,让自家连脱身的机会都没有。

“这究竟是谁传出来的?!”

蒲宗孟嘶声力竭的大吼,从窗口传出,转瞬就散入夜空。

