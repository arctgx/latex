\section{第16章 夜凉如水无人酌(中)}

邕州城满目疮痍。

到处是过火后的灰黑色的痕迹。一年四季都是热闹繁荣的街市,到了早晚饭点就腾起缕缕炊烟的人家,还有学校、仓库、兵营、寺庙、道观,邕州城中的建筑,大半都烧得精光。

屋舍树木的余烬,被前几日的雨水冲刷过后,在街角的低洼处汇集起来,变成了一滩滩黑黑的污泥。一具具尸骸散落在街道上,房屋中,水池里,还有就是与被烧毁的房屋一起化入火中。

苏子元呆呆的站在一片瓦砾堆前,愣愣的看着眼前的残迹。就算跟随在韩冈身边,听说了邕州城破,父亲殉国,苏子元也拒绝承认,可现实就在眼前。

楼阁数十楹的邕州州衙,只有被烧得发黑的八字墙还留有着半截。

每日里数百人出入不息的门房没了;处断一桩桩大案,举办年节宴席的大堂没了;处理日常琐碎公务的二堂同样没了。

苏子元仿佛幽魂一般,穿过前院,往后院的废墟中走去。在满地的瓦砾中蹒跚的走着,深一脚,浅一脚。在滑腻的灰烬中一失足,跌倒在地。掌心被突出的钉子划破了,鲜红的血涌了出来,低头看着伤口,却感觉不到痛。被身后赶上来的亲兵搀扶起来,他又继续往前走。

花厅前的两株芭蕉烧了;后院他喜欢的一片竹林烧了;府里的书房,里面的近万卷书,当初来邕州的时候可是装了半船舱,现在也没了;父母的正厢,二弟、三弟所居的偏厢,还有自己回来时所住的小院,全都成了灰烬。

生下自己、将他苏子元教育成人的严父慈母;相伴着嬉戏、学习、成长的二弟和三弟;会在自己读书理事时倒上一杯茶的妻子;做事一板一眼、像个老学究的长子;读书时爱偷懒、让自己每每大发雷霆的次子;还有年纪最小、也最讨全家喜欢的七娘,这些人全都不在了。

房屋、花木、陈设、还有里面的人,邕州州衙的一切不复存在,除了他心中留下的回忆,什么都没了。

苏子元神色木然的看着这一切,浓浓的要将心撕裂的悲痛。可他摸着脸,干干的,没有泪,只有掌心是湿的,那是血。

哀至则哭,可他现在却不知道该怎么哭出来。

他希望这是梦,只要睡醒了,就能看到父母兄弟和妻儿的笑脸。但他告诉自己这不是梦,从今以后,他就是孤身一人。

“可是大郎?”莫名耳熟的女声在苏子元的身后响起。

“何人?!”韩冈派给他的亲卫跟着一声大喝。

苏子元转过身,眼中映出了一个熟悉的身影,是在家里带着女儿的乳母。当他终于看清了抱在妇人怀里究竟是谁,一下就睁大了眼睛。

他颤颤巍巍的走上见,不敢置信的问着:“七姐儿?是七姐儿!?”

小女孩儿睁着大大的眼睛,抬头看着苏子元。直到被抱在怀里,才抓着苏子元的衣襟,哇的一声哭了出来:“爹爹……爹爹……”

“自从大郎走后,交趾贼就一直围着城。府里面许多人都上了城。温哥儿上城后就……就不在了。二郎后来也是不在了。三郎更早一点就病倒了。但城一直守着,一直到贼人堆了土上城后才守不下去。到了城破的那一天,城里到处都是火。唐通判、谭观察还有高钤辖他们都死了。老爷见再挡不住,就让我们剩下的人都离开,然后……然后就跟老夫人喝了毒酒。二郎、三郎一家都喝了。老都管本来要将清哥儿带出去,但清哥儿不肯走。说……说他是苏家的子弟,不能丢苏家的脸。最后夫人就让奴婢带着七姐儿出来。说只有七姐儿是女孩儿,可以带出去……出来后,就一直躲着……七姐儿一直都没有哭。”

妇人断断续续的哭诉着,苏子元紧紧的抱紧了女儿,不知何时他的泪水终于涌了出来,这老天,至少还给他留了一个女儿下来!

……………………

两天后。

韩冈率军抵达邕州城。

没有亲眼看见邕州城的惨烈,邕州城中所发生的一切,对韩冈来说也只是交趾俘虏的口供而已。不知道唐子正与敌偕亡的决断,也不知道苏缄投入火中的毅然,更不会明白守住这样的一座城池究竟有多么艰难。

当韩冈走在邕州城的街道上,望着两旁的断壁残垣,才亲身体会到这一期。愤怒、伤感,五味杂陈的感觉,让他只觉得心头堵得慌。

尽管贼军攻入城中仅仅只有一两天,但宋人用了二十多年才从侬智高之乱的废墟上重建的邕州城,大半地区都化为了灰烬。站在城中唯一一座没有被烧毁的五层木塔上,放眼望过去。在纵横交错的街道分割下,是一处处灰色黑色的地块。

邕州已经毁了,无论人民还是城市,都要再从头来过。

城中还有人,都在收拾着被烧毁、被劫掠过的家园。

交趾军离开已有时日,逃进山中的居民也回来了一部分。等到苏子元进城后,让人在城头上悬挂起的宋字大旗,昭告着大宋官军重新回到了邕州城中,返回邕州的居民又多了许多。

只是如今回到邕州城中的百姓,苏子元之前让人去清点过,不过区区一万多。就算还有一部分没有返回,可加起来当也不会超过三万。相比起旧日邕州城的户口,还有在交趾军围城前逃入城中百姓,三万人实在太少了一些。

死在城中的百姓究竟是三万还是五万?

精确的数字已经无法去数清。但邕州百姓的尸体,只要走入城中抬眼可见,就算闭起眼,窜入鼻中的浓烈气味,也在提醒着人们,这里究竟有多少亡魂。

韩冈闭起了眼睛,旋又睁开,所有的情绪都压在心底,恢复到冷静自若的状态。

“要立刻将防疫工作做起来!还有疗养院,也要同时设立。”

“城中所有人都要动手,不论有主无主,所有的尸体就必须在五日内全数运出城去掩埋或是火化。”

“在城中清理出一片干净的居住地。如果城中找不到合适的地方,那就选在在城外。无论如何,不能与尸体居住太近。”

“要确保干净的水源,另外柴薪必须得到保证。”

“必须要有石灰来消毒,邕州城附近要尽快建起石灰窑。”

“还有粮食,城中的粮库都烧了,附近的村庄也没了,要尽快从武缘县或是宾州运粮来邕州。”

如果韩冈和李信身边没有足够多的通晓部分医术的亲兵;如果韩冈手下没有足够多听候使唤的士卒;如果韩冈不是因为有了击败了李常杰、斩首上万的功绩,而在邕州军民中一下建立起了足够的声望;如果他不是有着足够权限的转运副使。他所要做到这么多事,绝不可能顺顺利利的施行起来。

不过韩冈权力、声望和人力皆备,就像将水轮放进流水中的水车,立刻顺利的运转了起来。

将亟待措置的事务吩咐下去,韩冈来到已成废墟的州衙,来到站在废墟之中的苏子元身边。张了张嘴,想出言安慰,可千言万语不知该如何说起,到最后也只挤出一句:“伯绪,节哀顺变。”

苏子元静静的站着,没有任何动静。韩冈进城前的两天,他专心处置着城内的事务,等到韩冈来了之后,就将手上的事情转交韩冈,再一次回到了家人所在的地方。

韩冈低声一叹,苏家全家三十余口的性命,岂在轻飘飘的一句节哀顺变?转身看着一片焦土的州衙废墟:“没能救下邕州,是韩冈来得太迟了。”

“运使何须自责?子元跟随运使一路南下,中间究竟有没有耽搁,子元都看在了眼里。”苏子元回头露出一个凄楚的笑容,眼中尽是悲色,将怀里的小女儿抱紧,“能保住一点骨血,已是运使予我苏家的大恩大德。”

韩冈看着他抱在怀中的小女儿,正仰着头,默默的伸出小手上去,为苏子元擦着眼泪。不过一年不见,相貌没有什么变化,只是历经大劫,就像长大了许多,原本就是让人喜欢的小女孩,而现在更是乖巧得惹人心疼。

韩冈叹了一声:“令尊为邕州而死节。伯绪你的全家,以及城中近二十名文武官,也一齐殉国。我已经写好了奏折,准备报请天子为此建庙立祠。日后能长守邕州,佑护万民,想必令尊泉下有知,也不会拒绝。”

“……多劳运使。”苏子元向着韩冈衷心道了声谢,能名垂青史,对于士人来说已经是最大的褒奖了。

“这是应该的。忠臣孝子,自当请旌以植纲常,以维风教事。光耀千古,作训后人。”

拉着苏子元离开了废墟,韩冈对等候已久的一队士兵嘱咐道,“可以清理了。只是小心一点,不要伤到苏公和家里人。”

……………………

“苏缄、唐子正、谭必、周成、薛举、刘师谷、高卞、周颜、陈琦、丁琦、邵先、梁耸、李翔、何泌、刘公绰、刘希甫、欧阳延、王亢、苏子正、苏子明、苏直温。”

韩冈念着名单上长长的一串姓名,一个姓名,就是一个殁于王事、殉国死难的官员。轻轻放下名单,邕州城中在籍官员,都在这里,一起选择了与城共存亡。

十四日的月亮还不算很圆,有着小小的一个缺口。

映在杯中,也是一轮并不圆满的缺月。

夜色已深,二月的邕州夜晚仍有一分清寒。韩冈坐在小院中的石桌边,手上是一杯倒满的酒杯,在他的对面,同样放着一杯水酒。只是无人共饮。

他去岁与苏缄在京城中结识,相交甚欢,算是忘年交。比起京城中勾心斗角的官吏,与苏缄这位外任的州官来往起来更为舒心。谁料想一别之后,原本谈笑不拘的忘年之交,如今已是一缕忠魂。

宁死不屈的英雄,与他守护的城市一起消逝。

光是清理城中尸体——仅仅是露在外面的——就至少要五天的时间;将邕州城内的废墟清理,把所有的尸骸都寻找出来,韩冈估计至少要两个月;而要让邕州恢复旧观,还不知道要多少年。

这是一座多灾多难的城市。缘边的大城,就算是位于关西的城池,都没有这样一次又一次的被人攻破过。家有恶邻,这一边的,一千多年来,交趾人所窃据的地方,一直都是中国的交州。只是到了五代才分离了出去。千年之后,又让中国的子弟在那片土地流尽了血。

韩冈无意去考虑千年后的问题,也暂时搁置了对交趾的仇恨。只有面前一杯水酒,敬着逝去的友人……

