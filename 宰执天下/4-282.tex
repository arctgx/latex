\section{第46章 了无旧客伴清谈(四)}

韩冈装糊涂,章惇却不会信他真的听不明白,“有些话,玉昆你是说得太多了。所谓画蛇添足,要是玉昆你能藏去一半话,这一次的风波也会小一点。”

他霍然站定,一下变得锐利的眼神压着韩冈:“别说什么不想欺隐,不想遮掩师长的功德,要是你那位孙师当真有心功名,这么多年来,早就该站出来了。你将人痘之法瞒了十年,从道理上,没人敢说你有错,御史台中这一次也的确没人敢在这方面做文章,而天子,更是不能以此降罪于你。但在天子留下一根刺,从来都不是什么好事。这一根刺的份量,在许多关键的时候,能一下改变天子心中的决断。玉昆……你是聚九州之铁,铸此一错啊!”

韩冈也站定了,毫不动摇的与章惇对视,“小弟也不瞒子厚兄,在决定怎么说之前,韩冈是犹豫了很久,不过权衡利弊之下,还是选择了现在的做法。不敢贪师长之功为己功,这的确只是一部分理由。更重要的,小弟只是想表明,莫说是师长,就是真仙,也不能是说什么就信什么。必须是有所思,有所辨。做学问嘛,必先博学之,继而审问之,而后慎思之、明辨之,最后一切了然于胸,方可笃行之!”

“……听说洛阳的小程已经进关中了?还有蓝田吕家为其鼓吹?”章惇默然片刻,问道。

韩冈沉默的点了点头。

章惇摇了摇头,忽而一笑:“还是明白不了你的想法。不过有玉昆你在,气学大兴可期。”

韩冈同是摇头,发自内心的感慨,“还差得远啊!”

此时雪停了,天色渐渐亮了,云层也一点点变得发白。对于平民百姓来说,一天的奔忙也开始了,路上的马车多了起来。

章惇的双眼追逐着一辆四轮马车——这是近两年轨道车出现之后,才在京城中兴起的,

“听说军器监已经造出了铁轮车了。车轮外装设铁瓦,车轮内毂以方形铁条为车锏。能耐磨损,使用可以长久。”

“铁轮车?”韩冈一脸惊讶,“都做到这一步了?”他都没想到,军器监在钢铁制造上的技术进步,竟然已经到了这一步。

“玉昆你还不知道。”章惇见韩冈摇头,笑道:“玉昆你颁下的悬赏,天子也认可了。这几年,军器监的工匠们为了一个官身,哪个会不拼命?”

他冲着韩冈又笑了笑:“不过现在还只有个铁轮车,不知玉昆你所说的铁船什么时候能问世?”

“……恐怕还要很长时间。”韩冈声音略沉,“都得熬时间……”

“愚兄的情况跟玉昆你一样,年资浅薄,都得再熬上一阵了。”章惇对着天空叹了一口长气,“终比不上吕吉甫的运气。”

章惇现在才四十四岁,过了年四十五。尽管比之韩冈的确年长许多,甚至可以算是两辈人,但在宰执官中,依然年轻的让人嫉妒。

吕惠卿四十七岁,做了四年多的执政,但他想要升任宰相,恐怕还要有番不小的波折,甚至说成是狂涛巨浪也可以,不一定能渡得过去。

“吕吉甫的手实法已经推行有一阵子了。”韩冈低声说着。

章惇转头过来,微带讽刺的笑说着:“玉昆你之前是京西都转运使吧?”

“之前在京西,心思一直放在襄汉漕运和种痘法上,这些事全都丢给了下面的人去管,也没得去多问。”

章惇摇摇头:“吕吉甫的情况不太妙……玉昆你在京西,不理手实法之事,应当也不只是忙得没有时间吧?”

韩冈也不瞒章惇:“免役法、便民贷、市易法,对富户已经是刮了一层又一层。不是不能刮,而是太招人恨,家岳镇得住,可吕吉甫他压不住阵脚啊。前面几条法度已经将富户的浮财刮得大半下来,该见好就收,省得人家拼命。可吕吉甫倒好,现在还要将人的命.根子都要剐下来,能不狗急跳墙吗?”

手实法是让百姓自己申报家产,以确定户等和税负,基本上是针对富户的。先不说自住的房屋和非租佃取息的自耕田只折算成实际价值的五分之一,就是吕惠卿为防止财产申报不实,张榜鼓励告发,告发成功的以隐瞒的财产三分之一来犒赏告发者,也是明明白白盯着富户。

试问有几人会去告四等户、五等户隐瞒财产?告一次还不一定能拿回一两贯的奖励。全是盯着一等户、二等户来,甚至胆子大的,盯着形势户和官户。

这是动摇官绅们的根基,将他们变成众矢之的。地方上的反弹,可想而知。现在反对手实法的第一条,就是败坏地方风气,儒家重教化,败坏风气的罪名吕惠卿压不住;第二就是借助民田买卖频繁,不易计算来做理由。软硬兼施,抵制吕惠卿的手实法。

前天在崇政殿上,韩冈就发现吕惠卿太过于沉默了,这个他一向喜欢统掌大权的性子完全不合。

想必他也是感觉到了身上越来越重的压力吧?

在有王安石的时候,一切压力都由王安石这根顶梁柱承担了,他们只要做好自己的一摊事就够了,不用担负起多少抵御外敌的任务。当王安石离去后,遮风挡雨的参天大树不再有,推行新法的一切压力和后果都要自己担负,吕惠卿就明显的压不住阵脚了。

人总是高估自己的作用,而忽视他人的功绩。在吕惠卿开始推行手实法之前,有没有考虑过自己能不能担负得起王安石的角色?有没有考虑过,王安石能将新法坚持到底,到底消耗多少政治资本?韩冈估计他多半是没有,不然也不会兴冲冲的推行手实法。

如果吕惠卿能放弃自己的那一份雄心壮志,做到萧规曹随,维护王安石留下的法度,最多也只是稍作休整,那么在便民贷、免役法、市易法的阻力都给铲除了的现在,他会做得十分轻松愉快,升任宰相也是指日可待。

可惜的是吕惠卿的心气太高了。也许是他想证明自己的能力,但选择的手段完完全全的错了。眼下手实法一旦失败,作为主持者的吕惠卿。在政事堂中,也坐不了多久了。

韩冈暗叹,这么一个聪明绝顶的人物,只因身在局中,就变得一叶障目,不见泰山。

眼下对手的反扑可以说是十分激烈,从章惇身上就可以看得出来。像他这样的高官会不得不离京就郡,从来不是经济原因,而只会是政治原因。

“吕吉甫有说要来吗?”韩冈问道。

“兔死狐悲,如何会不来?”章惇叹了一声,“昨天已经派人来说过了,从崇政殿出来后就会到,如今京城中也没几人能要他相送了。”

韩冈一瞥眼,捕捉到了章惇眉宇间浓浓的忧色。

的确是没几人了。当年跟随王安石起家的新党成员还剩多少?

贬斥的贬斥,叛离的叛离,现在还在朝堂上的那些人,基本上都是新党大兴之后,依附过来的投机者。

如蔡确之辈,他们对新法的认同,永远也不可能比得上吕惠卿、章惇这般坚定。这一干盘踞在台上的朝臣们,只要天子还偏向新法,他们就会坚持新法,同时借用新法的名义打压政敌,来维护自己的权力。可要是天子开始厌弃新法呢,又有多少会坚持到底,毫不动摇?

在外界看来,他们的确是新党,可在章惇和吕惠卿眼中,要说他们是新党?那就是笑话了。

韩冈为眼下新党的处境感到遗憾,这可以说是典型的劣币驱逐良币,真正有心于国的逐渐被压制、驱逐,而投机者却趁势而起,占据了越来越重要的地位。

章惇却突然振奋起来:“凡事必有波折,潮起潮涨也是自然之道。眼下虽有颓势,并不代表日后不能卷土重来。愚兄试问玉昆,到了眼下这一步,新法可废否?”

“……当然不可能!只要天子在一日,这新法就会留一日。”

韩冈的话有几分悖逆了,章惇瞥了韩冈一眼,就听他继续说道:“推行新法,虽是家岳、吕吉甫和子厚兄并力施为,但更是天子一意坚持下来的结果。如今天下的大好局势,都是因新法而来。换作是仁宗、英宗之时,哪里可能会想着一边抵御契丹,一边出兵攻打西夏?”

章惇点头:“恐怕只要契丹一表现出支持西夏的想法,朝堂上的宰辅们都会立刻心惊胆战的派出使臣,送钱送绢,说上满口的好话,将雄心壮志就此打住。”

“就是几年前的情况也是如此,幸好将新法坚持下来了。”韩冈说道。

“所以说,眼下离开就离开吧,相公不也是有过一落一起吗?只要新法能够坚持下去,不出意外的话,两年后的战事,就能收回兴灵故地。接下里就是更为重要的燕云,那时候才是大丈夫的用武之地,试问眼下朝堂上的那群蝇营狗苟之辈,又有哪个能担得起这份重担?”章惇指了指自己,又指了指韩冈,“舍我其谁!?”

