\section{第38章 何与君王分重轻(29)}

快四更天了,福宁殿中依然灯火通明。

只看衣服都没准备好,就知道想要在明天立太子为帝是如何仓促了。

“玉堂的人呢,怎么还没到?”

“已经派人去催了。”

“太上皇退居何处?”

“以后再说。”

“大庆殿上,谁来确认真伪?”

“礼仪使赞礼的位置近,到时候子宣多看两眼,认清了再说话。”

“那就要靠子宣了。”

“平章和相公离得也近,都要看清楚。”

太宗驾崩,遗诏太子赵恒即位。曾经帮助太宗登基的内侍王继恩,谋图另立被废为庶人的前太子赵元佐,却被大事不糊涂的吕端给关在了政事堂中。

到了次曰,真宗御殿,受群臣拜贺。吕端为宰相,于礼当率群臣罗拜。可因为王继恩的关系,吕端多了个心眼,他是近视眼,硬是凑到御驾面前,把真宗披下来的头发拨开了,看清楚真的是赵恒后,才下来带着群臣叩拜成礼。不过也有另一种说法,真宗当时不是披头散发,只是坐在帘后,吕端让真宗撤帘,坐到正前,让朝臣都看清楚了,而后方行礼。

不管当年的真相究竟是属于哪一种说法,都是在提醒后人,于新帝登基前,必须要做好万一有变的准备。凡事要多看一眼,多想一下。尤其是在大庆殿,群臣罗拜于下,高高坐在上面的新皇帝还真不容易看清楚。所以宁可看起来是小心过度,也比疏忽大意拜错了人要强。

心浮气躁的对话不时的响起。两府宰臣已经将所谓的宰相气度丢到了脑后。

登基大典,是一国之中顶级的大典。要做的事千头万绪。但有些事必须最先完成,有些事则可以缓一缓。

当务之急,是群臣朝见新天子。

派去请太子的人已经走了,找裁缝的人同样走了,去学士院请翰林学士的人还没回来了。

等待的时候,宰辅们并没有闲着,内部很快就分派了各自的任务。

宫内的事交给宫内安排,宫外的也就一个群臣朝见是当务之急,其他都可以放一放,交给薛向理清,之后交付有司。

主持仪式的礼仪使是参知政事曾布,流程由他主控。

首相韩绛起草太上皇帝册文,由参政张璪书写。次相蔡确起草太上皇后册文,曾布兼职书写。此外太子的生母朱贤妃,为太上贤妃,其册文,章惇撰,韩冈书——她们要转太后、太妃,得等太上皇上仙。

而最为重要的禅让大诏,翰林来得太慢,已经没时间再等了。

“介甫,还是你来吧。”韩绛看了一圈,找到了闲着没事的王安石。

王安石咕哝了一句。

章惇、韩冈离得近,听到了,同时抬头看。韩绛也是苦笑。

“投名状吗?”王安石就是这么小声抱怨着。内禅一事,他现在已经不反对了,可心中依然难以释怀。

但韩绛还是坚持让王安石来写。大家都有事,王安石如何能清闲?同样都是赞成内禅,王安石当然不能置身事外。

王安石要了笔墨,片刻时间草草写就了一篇,拿着笔,盯着草稿,过一阵就动笔修改几个字,也没用多久就敲定了全文。这时候,其他人都还在咬文嚼字,苦思冥想。

论文才,这里所有人加起来或许都不及王安石一个。

“太子还没到吗?”

王安石心中浮躁,急脾气的他放下笔,“来了,来了!”

守在门口的杨戬闪了进来,太子赵佣终于到了。

衣服也拿来了,一个通天冠、绛纱袍,一个赭黄衫袍,还带了两个擅裁剪的宫女。

被抱过来的太子没戴帽子,刚剃的头皮泛青,剩下的头发撮了两个小角,眼睛迷迷糊糊的,显然是没睡醒。

“殿下,太子殿下。”宋用臣过去轻声唤。

赵佣慢慢张开眼帘,眼睛还没适应,就看着周围一群人,“天亮了?真早。”

“殿下。”宋用臣和声道,“请试一下衣服。”

宋用臣捧着通天冠和绛纱袍,赵佣就死盯着那件袍服。

“殿下,殿下。”见太子突然僵住,宋用臣害怕出了事,忙小声的喊着。

赵佣醒过神来,急着叫道::“父皇。父皇怎么了?”

“好聪明。”韩冈就听见旁边的章惇低声说。

的确聪明。韩冈也这么觉得。

看到了新制的天子服,一下子就明白出了什么事,应该可以说他逻辑推理能力比较强。

“殿下,天子尚在安睡,殿下勿忧。”宋用臣劝道,“快点试下衣服,要拿去改。”

赵佣不理,扭这身子要下地,“我不要试衣服,我不要做皇帝,我要父皇。”

太子一闹,向皇后拿着手巾捂住脸,又低声哭了起来。王安石叹了一声,感觉又老了几岁。

身下的宰辅们各自有事,只有韩冈这个做老师的最适合开口劝说。

韩冈上前两步,叫道:“太子殿下。”

“韩先生。”赵佣不敢闹了,老老实实下地,想向韩冈行礼。

“殿下。”韩冈不顾仪态的蹲了下来,与六岁的赵佣对视着,“殿下可知陛下到今天,已经做了多少年天子了?”

“……十五六年了。”赵佣想要计算了一下赵顼登基的时间,数着手指,用了不少时间。

“没错。这十五六年幸亏有天子,使得大宋比起太祖、太宗的时候,又兴盛了很多。西夏灭了,辽国败了,这在仁宗、真宗时,实难想像。但这都是天子的功劳。”

“嗯。”赵佣很高兴的点头,这是夸他的父亲。

“从王平章,韩相公,再到臣韩冈,无一不是得陛下所提拔,方能一展才华。如果天子的情况还能挽回,没人愿意放弃努力。可惜,不行了……”韩冈抬起头,对周围旁听的同事问道:“那张纸条呢。”

“玉昆!”韩绛惊叫,而在他的惊叫声用,也掺杂了皇后的惊讶。

‘不用担心。’韩冈向所有人作保证,拿起纸条,放在赵佣手中,赵佣果然认识字,“皇后害……”

“剩下的一个字是‘我’,陛下说,‘皇后害我’。”

“啊。”赵佣惊讶,看看赵顼,又看看向皇后。

“皇后是不可能害天子的。这点不用怀疑。”韩冈正色对赵佣道,“殿下!当今天子是史上难得的英主、明君。但现在的情况若是传出去,不说成了世人笑料,也会使天子过去十几年的辛苦全都成了泡影。太子,你能眼睁睁的看到这样的情况出现吗?”

赵佣终究年岁还小,几句就绕糊涂了。他摇摇头,“不能!”

“所以我们也一样不想看到。”

顺利的跟太子沟通,所有人都放下心来。

‘还真会说。’薛向咕哝着。

该怎么劝,薛向也知道。但他在太子面前出现的次数太少,留不下印象。见韩冈次数虽少,却肯定是印象深刻,皇后、王安石不说话,当然就只有韩冈出面。

章惇点点头,算是附和。

颠倒黑白是官僚的基本功。一件事,若做不到能正说反说,那就别写文章了。既然能将六岁的太子给说迷糊了,也肯定有这方面的特长。不过敢大着胆子直接将那张纸条拿出来,章惇自问也要多想一想,其他人就更不用说了。

“如果陛下还能恢复。我们决不会这么做。但陛下的病,天下都没有能治的方子,都是能看天数。”韩冈叹息着,然后对赵佣正色道,“殿下,你愿不愿意为了天子分担一下责任?”

“知道了。”赵佣用力点头,小拳头握紧,“愿意!”

这能并不是谎言,而且合情合理。当然能让人相信,不过只是小孩子而已。现在赵佣肯定是还醒悟不过来。等到他长大,如果没忘,则肯定会明白。但不论是忘了还是不忘,却又能如何?

赵佣被说服了。这算是最好的结果,否则小孩子闹起来,真的没有办法解决。

章惇就在韩冈身旁,安心的长叹了一声。

‘啊’!来自内间的惊叫声打断了所有人的激动,“官家醒了!”

宰辅们一个个抢进内间,向皇后也跟着,只是进去后没走上前。赵顼已经睁开了眼睛。原本一直在手边的沙盘早被拿开,现在手指就只能在床褥上划着。

‘什么时候醒的,汤药怎么没用?’宰辅们的心中乱作了一片。

“把沙盘拿过来。陛下有话要吩咐。”韩冈上前道。既然躲不了,就干脆正面迎上去。

拿到沙盘,赵顼开始在上面画字,‘六,哥’。

是在叫太子。而且看起来很冷静。

“陛下。”王安石有些激动,又强自忍耐。

宰辅们都屏住了呼吸,只听赵佣大声道,“儿臣在。”

‘改、名’。

啊。差点都忘了!

韩冈差点出声。

赵顼的名字就是登基时改的,之前叫做赵仲鍼。皇帝名字都要世人避讳,所以登基改名,基本上尽量用生僻一点的字,免得给世人添麻烦。

不过不管怎么改,终究还是会添麻烦。就是武瞾那样生造的字,也照样要避谐音的讳。比如山药,唐以前名为薯蓣,当唐代宗李豫登基后,就不得不避讳,改为薯药。到了上代的英宗赵曙为帝,又不得不改为山药。之后再没有改过,沿用到千年之后。

赵佣不太明白简单的两个字‘改名’是什么意思。王安石拉着他,详细的解释了一番。

并不狂躁的天子,各人望着无不心中生寒。

赵顼若是继续发狂,那还好说。现在一下就变得如此冷静,实在是出乎意料。毕竟是皇帝,纵使知道他再无爪牙可用,但积威尚在,不是寻常人可以轻辱。

只有韩冈放得开,他并不担心赵顼还能将他怎么样。安心的看着赵顼最后的表演。

“就是不能用旧名了?”赵佣点头,“儿臣知道了。”

他很机灵的跪下来,对赵顼道:“请父皇赐名。”

“‘煦’【注1】。”赵顼吃力的在沙盘上划着字,‘早、已、定、好’。

‘硬是留了一根刺下来呢。’韩冈想着。

注1:尽管现代,神宗、哲宗两父子的名字发音相同,只是音调不同。但在古代,赵顼的顼,在韵部中属于‘入声二沃’,而宋哲宗赵煦的煦,则是属于‘去声七遇’,发音相差很远。

