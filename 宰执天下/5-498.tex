\section{第39章 欲雨还晴咨明辅(27)}

“先吃饭。”

韩冈说着就起身,不理会冯从义的问题。

“哥哥,你这不是吊人胃口吗?这样小弟可吃不好饭。”冯从义忙跟在后面,抱怨道。

“怎么吃不好?”韩冈说着,“并不关你们的事,只是天家和朝廷之间的问题。不要提,不要问。”

冯从义本还带着点玩笑的口气,可听韩冈这么一说,不敢再追问了。

既然韩冈说是天家和朝廷之间的问题,那就是天家和朝廷的事。不是他们这些商人可以掺合进来的,至少还不到掺和的时候。

“吃完饭后好好歇一阵,不要总是酒宴,对身体不好。”

“小弟知道了。”冯从义诚诚恳恳的回道。

韩冈叹了一声,“也难得有个歇息的曰子,今天晚上当能安生一点。”

但晚饭之后,韩家却没有得到一个安生的夜晚。

宫里遣中使至韩府颁诏,以韩冈于枢密副使、河东制置使任上,有拯危救难之功,赠功臣号,推忠协谋同德守正佐理翊戴功臣。擢礼部侍郎。晋光禄大夫。赐检校太尉、上柱国。封莱国公。加赠食邑三千户,食实封一千户。并再赠韩冈诸子官。

此外还有赐田一千六百亩,并宅一第。

紧接着是第二封诏书,授韩冈宣徽北院使,掌院事。

这并不是战后的封赏,之前就已经给过了——尽管很微薄,但给了就是给了:曹彬平南唐,太祖皇帝给了他两百贯,少归少,可也是赏过了——而是给韩冈辞官的赠与。

宰辅去位,只要不是重罪,那么都要加官进爵来以示宠遇,并向外界表明这位宰辅并非以罪辞官。王安石当年辞相,寄禄官便从礼部侍郎直升吏部尚书,食邑、勋号皆有加赠。

如此一来,只要韩冈不推辞,他的官职就是推忠协谋同德守正佐理翊戴功臣、资政殿学士、宣徽北院使、礼部侍郎、光禄大夫、检校太尉、上柱国、莱国公、食邑一万一千户,食实封四千户,贴职没变,依然是资政殿学士。

宣徽北院使虽是闲差,可也是正式的差遣。

散官阶只决定服色,从二品的光禄大夫已经是执政能拿到的最高一级。

检校太尉是十九阶检校官中的第二阶,与勋号最高第十二转的上柱国,以及功臣号一样,都是给着好听,可以在墓碑上多刻几个字,除此之外就没别的意义了。

食邑一万一千户,而食实封则达到了四千户,超过三分之一了。正常的食邑和食实封的户口之比,差不多在四比一,多不过三分之一的样子,毕竟前者是虚的,后者则是实打实要给出现钱。

但问题不在这里,而是国公。

以韩冈的资历和年龄不可能得到封国,就任国公,这跟韩冈预先了解到情况不一样。而且之前上殿,他就已经辞了皇后的加赠,现在的任命他不可能就这么答应下来。

韩冈再拜而起,辞而不受。给了宋用臣一封礼金,就把苦笑的他打发回去了。

这就是所谓的礼,来来回回多少次了,有时候还是挺烦人的。

至于所赐田宅,细节诏书中不可能有,但很可能就是属于皇家的一个小庄子。皇家的庄园,不是位置绝佳,就是土地肥沃。而且天家田地更有个好处,就是都是整地,不是外面拆得零零碎碎,这边三分、那边半亩,拼凑起来的田土,而是一大片完整的田地。

在市面上,超过百亩以上完整的田地,比同样面积的零碎田地要贵上近倍。而完整的面积越大,价格就越高,如果一千六百亩是一整片,或者只分成三四块,只要不是薄田,离京城不论远近,十万贯都能卖。

毕竟是太稀少了,多少重臣、勋旧都想将自家的土地给连成一片,可成功的几乎没有。除非是皇庄,否则在京的田地,只要转过两手,就不会有这么完整的土地。

只是韩冈贪这点地皮做什么呢,家里的三百顷田地分据了几条河谷,前后四个大小庄子,论其出产,太上皇后所赐田宅,绝对比不上家里的土地。

虚名都嫌多,更别说在他而言不算很值钱、却烫手得紧的田地和宅邸。韩冈想想都觉得麻烦。

地位到了他这个等级,就不指望什么功赏了,立再大的功,也不可能能升多高。要考虑的是家族的延续和子孙的未来。平均年龄五六十岁的宰辅们,在私事上要考虑的也就这些事了。

不过韩冈才三十,立下的功劳又多,现有的官职和封赏制度,对这样的异类,却很难安置妥当。韩冈早有所悟,太上皇后的赐田宅,放在老臣那边,等于是叫人上表颐养天年。不过放在韩冈这边,则是给他的补偿。

可终究是是个麻烦。

“真是给人出难题。”等中使离开,韩冈摇头对王旖道,“这样的任命为夫怎么可能会接受?”

王旖点头道:“的确是个麻烦,还是不要为好。”

周南的姓格则更直一点:“官人想要就要,不想要就不要。反正总能当得起,只是看不上眼。”

“三哥哥不想要就罢了。”云娘也在说。

“但赠官就是辞了,也不会就此罢休。接下来肯定还会有诏书来。”严素心让下人端着凉汤来,也说着,“这两天都不得安生了,那香案干脆就摆在前院吧。省得搬来搬去。”

“等辞过两三次,明白为夫的心意不可更易,就会减下去了。那时也好顺水推舟的接下来。”韩冈笑了笑,“没个差事,也不方便留在京城。宣徽使就宣徽使,就算是抢了王状元,也不打紧。”

“大哥、二哥的荫补不会辞吧?”

“算了。”韩冈迎上周南、素心和云娘紧张的眼神,心中一软,做母亲的哪有不关心儿子的道理,“都接了。”

“这样一来,三哥、四哥可也是京官了。”严素心很开心的说道。

在官宦人家久了,京朝官和选人的差别早就清楚了,还不会写字,就与积年的进士相当,虽然不合理,但确实让人惊喜。

如果韩冈接下诏书,他排前面的四个儿子,最低一个都是正九品的太常寺太祝,直接就是京官了。而分别是长子和嫡长子的韩钟、韩钲,则是已经是从八品的大理寺丞。作为荫补的宰辅弟子,他们的官阶不能再高了。

除非是宗室,否则升朝官是不可能靠荫补来。而即使是宗室,也是武选官,不入文官序列。

“可惜啊,如果大哥二哥年纪再大一点,有了点文名,就能授进士了。”韩冈说笑了两句,又道,“郭仲通如果能辞官的话,他的儿子郭忠孝当会被赐同进士出身,吕吉甫那边当也是类似的情况,只要能辞官,给他的赠予不会迟,也不会少。”

如果是论功劳,韩冈有在京参与拥立,以及领军征战得胜两份功劳,理应比吕惠卿要多一点,那样才是正常的。但定策功现在还没有封赏下来,两府诸公都觉得要等一等,等风头过了再说。而之前就已经拿过了对辽功赏——比吕惠卿要少。现在是辞官,吕惠卿若是知情识趣,两府不会亏待他,肯定要比韩冈这个主动请辞的要多。

至于差遣,韩冈是宣徽北院使。吕惠卿是南院使,比韩冈高半级。郭逵为雄武军节度使,则更胜一筹。

“只是寄禄官晋升为礼部侍郎,还以为给个给事中就打发了。这下子每个月当能多拿几贯钱了。”

韩冈现在的本官是右谏议大夫,是执政的最低一级。不论原来的寄禄官多低,只要被任命为执政,那么立刻就会升到右谏议大夫的位置上。当年吕惠卿就是自正七品的低品寄禄官,一下跳到了右谏议大夫。

但升到这个位置之后,想要再晋升,一个,是做宰相,最低就是礼部侍郎。要么便是熬资历,时间长了,迟早就能升上去。最后,在卸职时会得到赠与。

王安石第一次做宰相,就是礼部侍郎,然后罢相后,从礼部侍郎一下升做了吏部尚书。跳过了户部、吏部侍郎,尚书左丞,工部、礼部、刑部、户部、兵部五部尚书,整整升了九阶。现在又经过了一次任相、罢相,做平章,又辞平章,则是司徒。

“就算打几个折扣,最后也不算少了。”韩冈对妻妾们笑道,“不枉为夫辞了这个差事。”

冯从义在后面听着苦笑,枢密副使哪里是宣徽使可比?

而且以韩冈的功劳,去东府争一下参政的位置,完全合乎情理。要知道,他之前就已经辞过了参知政事的任命。现在入东府,哪个都说不了闲话。

只可惜他的这位表兄,心思总不肯老老实实的放在做官上,总是想要去宣扬自己的学问,甚至为了学术,将大好官职都给丢了。

不过话说回来,如果韩冈不是这样的姓格,也不可能有现在的成就。

有人千方百计都求不来,有人却偏偏视若敝履。

这个世上,有意识的地方可就在这里。

