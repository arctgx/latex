\section{第44章 秀色须待十年培(29)}

政事堂的房子不知几年没有修了。

富贵人家都在赶着将家里的蒙纱糊纸的窗户都改成玻璃窗,但堂堂大宋帝国的政治中心,却连窗户纸都是破的。

韩冈坐在下首,侧面窗户透进来的寒风,呼呼的就往他身上吹。

拜其所赐,室内的空气倒不是那么憋闷,让韩冈头脑十分的清醒。可以继续游说眼前这位固执的宰相。

有关邮局的动议,韩冈虽然已经写好了奏章,准备递上去了,可他觉得这件事跟政事堂的必要沟通是免不了的。为了避免之后在朝堂上吵起来,还是先通报一番。

只是这段时间以来,都快要成为隐形人的韩绛,却首先表示反对的态度。

“玉昆,邮驿一事不为不善,可你想过没有,这每年要多开支钱粮?”

“其实多不了多少。”

“还多不了多少?!按玉昆你的算法,每个乡都要一个驿站!朝廷财计哪里还能支撑得起?!”

“相公误会了。驿站是接送官员,所以要人要马,管吃管住。而乡邮所只管收信送信来往于乡中和县中。一个人,一匹马就够了。大宋一千八百县,就算每个县十个乡,天下也只有一万八千个乡邮所。一个乡里一天两天能有多少信,百十封,一个包袱就装了,多麻烦的事?”

“那县中、州中、路中呢?这些地方一个人就够了吗?”

“县中、州中、路中,就可以借用现有的驿站。既然能送官府文函,送一下民间的信笺,也只是顺带而已。”

“好个只是顺带,玉昆你可知天下户口两千万,每天有多少私信要寄?”

“那不是正好?信件多了,朝廷的邮费收入也就高了,也就能使用更多的人手而不伤朝廷财计,更不会挤占铺递运送军政公函——私信本就不该占朝廷便宜的。”韩冈笑容带着讽刺,大凡重臣,多有借用铺递传送私信的经历,这都是心照不宣的秘密,“而且论路程远近,邮费远至千里的百文,近处的则十文。这样的邮费其实不算多,比人情债要少多了。”

“村子里就不管了?”

“乡邮所里面设了各村的邮箱,是哪村的信就放进在邮箱里。这世上没有哪个村子隔绝人世——真要隔绝人世,也不需要寄信收信——只要村中有人去乡镇上赶集,顺道就能带着信回去。寄信也是如此。乡中集市,有逢三六九的,也有一旬一次、两次的,递送信件也方便得很。完全不需要朝廷多花一文钱。”

韩冈没指望过政权能下村。以这个时代的管理能力,能到乡镇就不错了,怎么可能直送村中?

“乡邮所的话,铺兵去县里收发信件的时候,来人寄信怎么办?”

“说是一个,其实就是一户人家。又不是上阵,难道只能男丁做事?家里的儿女、浑家、父母,难道帮忙收钱收信都不会?驿馆里面,打扫房间的难道都是驿兵?还不是有家眷帮手!”

“还是说说乡邮所的花销吧。”韩绛道,“算清楚到底要多少钱粮。”

城中的收发信好说。信不多,就让驿传的人多跑跑,大不了调几个厢兵进来。要是信多了,更可以借助邮费来安排人手。但在乡间设立邮所,等于是公吏长驻乡间,这在本朝中没有先例,宰辅们都想问个清楚。

韩冈算给韩绛听,“一个乡邮所,一人一马,一个月只要一贯钱,外加两束草、一石粮。这已经算是多了。厢军一个月才拿五百文的多得是,一个月一贯已经是禁军的等级了,而且跟禁军一样还有口粮。驿马也有草料。只要他们能够隔一两天去县里一趟,去信送信。天下一万八千乡邮所,一个月朝廷要支出的不过一万八千贯,一万八千石,三万六千束草。分散到每个县,十贯钱、十石米,二十束草而已。”

“一年呢,可就是百万了。”

“是二十余万贯钱,二十余石粮,四十余万束草。”韩冈徐徐更正道。

草是草、钱是钱、粮是粮,得分开来。韩冈一贯反对将钱、粮、银、绢、草,这些不同种类的赋税都合并在一起说。经常说的一万万.税赋,大部分都不是钱,而是粮食、草料,单位名称是贯石匹两束。

不过这样统计的税入,只是刊载在邸报上。呈报给天子和宰辅们看得,都是真正的明细账。韩绛混为一谈,纯粹的没有谈话的诚意了。

“很少吗?老夫知道,玉昆你是盯上了关西罢兵后节省下来的那份钱。但多少人都在盯着,给了你后,其他人怎么办?都是要用钱的。前几曰,薛师正又过来,说要加快修轨道的速度,尽早将京宿铁路修好。但现在国库空虚,好不容易才能积存一点,哪里能随便花销出去?”

韩绛不讲道理,蔡确、曾布等几位就听着,没有化解的意思。韩冈算是确定了,果然还是门户之见。

当然,这更是因为驿站系统的控制权在枢密院手中的缘故。将军驿系统扩大到民间,如何界定枢密院的职权范围?这是东府诸公首先要考虑的问题。

韩绛他可以不管事,可以做佛像,但他不会让政事堂的权力,被枢密院给侵占去。

韩冈也是明白这一点,才会在这里费尽唇舌的向韩绛解说。他不可能牺牲掉枢密院的利益,将驿传系统的管辖权从枢密院剥离出来,转给政事堂。韩冈要是这么做,韩绛立马就能点头,但代价就是章惇、薛向和苏颂要愤怒了——合着关系好就要吃亏的啊。

只是要怎么说服韩绛为首的东府宰执,难点就在这里。最坏的结果就是去朝堂上辩论,与政事堂拉下脸来争夺邮政局的控制权。那时候,就是向皇后通过了,中书门下的相公们不同意,还是白搭。

“玉昆,还是慢慢来吧。”蔡确也说道。

曾布、张璪都没插话,就看韩绛、蔡确跟韩冈讨价还价。

“相公明鉴,邮政驿传的好处可是现成的。”

韩冈说着,视线从韩、蔡、曾、张的脸上扫过。想等自己出价,也得先看看自己过去是怎么做买卖的吧。

“第一。就城中来说,街巷门户编订门牌号,曰后城中管理也就容易许多。而铺兵送信,走街串巷,大事小事都能顺道看着、听着。驻扎在乡中的乡邮所,同样可以监察乡间。乡中、城中都有了可靠的耳目,不虞变生肘腋,猝不及防。”

曾布眉头一皱,乡邮所当真成为了朝廷耳目,家里的大事小事都给人打探去,写了密报,谁受得了。京城中就已经有皇城司了,难道乡里还要出一个?!

“第二,有了乡邮所,朝廷和官府的政令可以直抵乡间,若有诏命、公文,不用担心为歼猾胥吏居中使坏。”

这些都是应该说给天子听的,而不是说给宰辅听。天子会担心上情不能下达,下情不能上传,但这样的担心,在宰辅们的心中,要对折再对折。

两条才出口。韩绛、蔡确等几位的表情没怎么变,眼神却都阴冷了下来。

韩冈的话,根本就不是给出价码,而是威胁。今天能拦着设立邮政局,但只要在太上皇后和天子那边存了一份心,曰后迟早都会设立的。现在拦着也没用。

“第三。是报纸递送。”韩冈似乎毫不在意,“快报现在只在城中发售,最多也不过遍及开封、祥符两赤县。而京城之外,却不会有多少人买,开封府路都没有普及。但有了邮政之后,就能送到村中发卖,如果一个村子富户购买一两份,那京城周边又有多少乡镇、村庄,又会有多少人购买?在这其中,邮政也能分润不少。”

这还是威胁。拿着京城内的宗室、贵戚和豪商们来威胁人。

曾布寒着脸问道:“玉昆说着这些之中,也包括《自然》吧?”

韩冈点点头,毫不讳言,“虽是公事,韩冈也是有些私心的。”

韩、蔡等人各自的脸色更是难看了几分。曾布都意外韩冈竟然敢于当面承认。

之前是将邮政当做公事来讨论,用公事公办的态度就能抵回去了。但韩冈现在明说是私事,反而不好办了,他既然伸出手,谁敢硬将他的手拍开。

但这样为人所胁迫,哪个心里能痛快?!他们可都是高高在上的朝廷辅弼,走到哪里都是众星捧月一般的奉承着。

“不过。邮政驿传真要铺开来,其居中调度,却跟轨道运输相类似。”赶在把所有人都得罪光之前,韩冈圆熟地转开话题,“如今只有连接要郡的干线,等有了连在干线上的支线之后,邮政驿传完全可以借用轨道来输送。”

“支线?!”蔡确心中一跳,道,“朝廷哪有这份财力。”

“干线国有,而支线可以归私家所有。”韩冈微笑着,“轨道只有铺设得越多,才越能发挥出超越水运的作用。朝廷既然做不来,仕宦之家当为朝廷分忧。此事,韩冈愿先行向太上皇后和天子奏明。”

半个时辰后,守在厅外的侍卫,惊讶的看着政事堂中的四名宰辅将韩冈送出了公厅外,气氛竟出奇的和睦。

只是跨出门后,韩冈的神色却忽然严肃起来。

“怎么了,玉昆?”见韩冈突然在门口停下脚步,韩绛问道。

韩冈抬头看着灰色的天空:“下雪了。”

“这么早,还不到十月啊!”韩绛惊讶着跨出了门,若有若无的雪粒,从云层中洋洋洒洒的落了下来,“还真的下了。今天早上看着天色就不对,果然是下下来了。”

“开封府那边不知道准备的怎么样了。可不要明天报上来说路边冻毙几十人。”张璪说道。

“应该不至于。倒是防火要小心了。”

“今年比往年要冷得早。北方的情况可能会更坏。神武军和灵武都是新复之地,也不知过冬的准备有没有提前做好。”曾布说着,又望向韩冈。

韩冈更担心的是在西域的王舜臣。中原腹地都下了雪,那边的情况只会更坏。

遇上提早到来的寒冬,刚刚收复的疆省域就更难稳定下来了。

而且到了西州回鹘的边境上,黑汗国就是敌人了。要确定国境线,不是靠谈判,而是靠刀枪来解决。

内外皆敌,不知道王舜臣还能顾得过来吗?
