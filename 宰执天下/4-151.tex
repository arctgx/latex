\section{第20章 冥冥鬼神有也无(27)}

【中午欠下的,终于在今天赶上了。】

在号角声渐次消散在雨幕中的时候,来自官军大营之中的战鼓擂响。

沉重的鼓声穿透绵密如瀑的雨水,第一次压倒了奔流不息的江水轰鸣,传遍了升龙府城外的每一座军营,让进军的号角重又划破云空。

鼓号的助威声中,抱着用芭蕉叶或布匹包裹起掘出来的土块,数以千计的蛮兵冒着雨沿着道路冲向几个指定的地点,更有几处则是交趾百姓被身后蛮军用刀枪逼着,往城墙下涌过去。

在他们身后,是几个由三五百人组成的军阵,这是支撑溪洞蛮部能够奋勇上前的最直接的原因。不同于衣着简陋的蛮人,他们都穿着铁甲,外面还披挂着单薄但易于行动的玄色油布雨披,看着略显臃肿,黑压压的却像是一头头穿梭在山林中熊罴,只有杵在身侧的一人高的斩马大刀,泛着白亮的光芒。

几座军阵虽然人数不多,可沉甸甸的如同是装在船舱最底层的压舱石稳定船只一般,控制着战局不会偏移到其他方向。

升龙府四面八方同时受到攻击,而主攻的方向,正是宋军大营所在的东面,也是李常杰现在所正对着的位置上。

城外多为积水一两尺的稻田,只剩寥寥数条道路可以攻到城下。等于是有了数里宽阔的壕沟,可以将用来守城的兵力集中到几座城门来。这样的防御力当然远比分散守城要强得多。

“也只是如此而已!”

李常杰纵声大笑。他早就想到了宋人可能使用的攻城手段,召集如此之多的杂兵前来,怎么可能还有别的用处?虽然他要防着宋军乘隙攻来,不便派主力出城驱赶城下的杂兵,但他还有其他更为合适的军队可以指派。

看到正当面的蛮军驱赶着自家子民抱着一包包泥土向城下涌来,李常杰毫无动摇的下令道:

“象军出阵!”

东城的城门吱呀呀的打开,正押队冲到城下的一队蛮军探头向里面一看,立刻脸色大变的向两侧逃去。

一头头庞然巨.物从城门中鱼贯而出,在城门前稍稍整理了队型,便立刻向前猛冲起来。

奔驰的巨兽根本不需要在乎只能没过它们膝盖的积水,在积满雨水的稻田中,踢起一团团水花。

大象的身躯看似笨重,但飞奔起来,却是势如奔马。每走出一步都能引起地面的颤动,数十头一起狂奔,犹如山头上滚下来的万斤巨石,其势无可阻挡。

从城中奔出的象军,直直冲进了正要将土块堆到城下的蛮军和本国百姓之中。也不管是敌是我,将人卷起来,又用力甩在地上,继而再一脚踏上去。一头头嗜血的战象,高高抬起长鼻嚎叫着,用力踩踏脚边鬼哭狼嚎的敌人。脆弱的人体就像是不小心落到脚底的泥人一般,被踏得稀烂。

看着一众蛮兵拼了命的逃散开来,空留下一地摊开来的血肉。在城头上扶着雉堞的李常杰,只略作犹豫,便猛地向前一指,“给我冲一下!”如果在这时候能给宋人迎头一击,接下来的防守就能轻松许多。

听到城头上的将令,正沉浸在杀戮的兴奋中的象军,就毫不耽搁的调转方向,对准同一个目标直奔压阵的宋军而来。

数十头大象并肩而行,速度越来越快。近万斤的体重冲刺起来,地面上的积水都开始震颤起来。

城头上的守军也配合着开始高声唱起凯歌,他们的歌声隔着远了,本来应该是很模糊,但不知为何,却清晰的传到了观战的章惇、韩冈和李宪等人的耳中。

“南帝山河南帝居……”李宪的脸上露出一个不知该说是狰狞还是扭曲的笑容,“嘬尔小国,蛮夷之邦,也配称帝?普天之下、莫非王土,率土之滨、莫非王臣。天下万邦,哪里不是中国之属?还敢说‘截然定分在天书’!”

“他们是以为能胜过官军罢了!”章惇冷笑着,“李常杰大概还不知道,在这之前,官军已经杀了多少头大象!”

“人总是有赌性的。这个时候,李常杰也不得不赌一把了。”韩冈为李常杰感到遗憾。

他的选择不能算错,大象畏火,可雨天连火把都点不着,加上宋人最为依仗的神臂弓也当在这些日子的湿气重损坏了许多,正常的将领怎么会放过这么好的机会。但错就错在他只掷出过一把六个六,接下来的赌局,都是最差的点数。而他的对手,也就是官军,总能稳当当保证每把都是大数。

大象狂奔而来,当面的蛮军纷纷逃窜,可正当面的数百名宋军却并无半点动摇。在甲胄的外面是一层油布雨披,藏在雨披下的,是一张张已经张开的神臂弓。

浸透了雨水的神臂弓甚至比平常的一石弓威力还小,但特意用油布裹起、又用猪油密封了装在放有石灰的箱中,用来以防万一的五百张神臂弓,却还能保持着水准以上的力道。

不过当箭矢穿过滂泼大雨,虽然大半命中,却也失去了杀伤力,仅仅射入了皮肉半寸,这份痛楚则更加刺激得一只只战象猛冲过来。

“好!”城头上的李常杰大声叫好,身后的守军将士也是一片欢声雷动。神臂弓果然已经排不上用场,一旦进入接近作战,区区百十斤的人,如何能与近万斤的大象相比!

但他们的笑容只是存在了短暂的刹那时光,宋军对抗战象的手段远远不止神臂弓一项。

霹雳砲当先出动,六具虽少,抛掷出来的石弹也只命中了一颗。但这一颗能撞跨城墙的石弹,落到一头战象的前额上,咚的一声就是一记闷响,战象登时倒在了地上,比起箭矢来更有效果,

“掷矛!”就在霹雳砲开始攻击的同时,亲自指挥作战的燕达也是一声低喝。

接替神臂弓的是李信麾下最为得力的标枪手,上古的人类拿着石块打磨出枪头的简易标枪,将大地上所有的巨兽都放进了食谱之中,如今换成了钢铁为尖刃的掷矛,只是从侧面奋力一击,飞掷而出的投枪就深深的穿透最前面的五六头大象柔软的腹部,

受了致命伤的大象仰天咆哮,引得象军队形一时大乱,站在最前面官军将士,随即挺起斩马刀反冲了上去。

当头挥来的象鼻如同一根铁棒,轰在士兵们的身上,一口血就喷了出来。但身手灵活的士兵,却还来得及在战象的鼻子挥舞过来之前,跳到一边,手中的斩马刀一挥而下,直接将鼻子砍下半截。

对人来说,是十指连心。而对大象,则是长鼻连心。最敏感的部位受到了重创,发了疯,受了伤的象鼻狂挥乱舞,当空喷撒着血水

斩马刀发挥了作用,长枪也同样是对付巨大目标的利器,尖锐的枪头扎进战象的腹部,只需用力一绞,就能让战象倒地不起,比起的斩马刀还要简单。

大象是极聪明的动物,甚至各自都有不同的个性,听到同伴受伤的哀嚎,嗅到浓浓的血腥味,有的战象冲上前去,有的则停下脚来观望。就是寻常经过训练的战马也做不到如臂使指,象背上的驭手在这样的情况下,也无从控制,无疑成了神臂弓最好的目标。

气势汹汹的象军出击,攻势却只用了片刻的时间便是土崩瓦解。

纵观史书,中原军队与象兵交战,败阵的例子并不多,牲畜永远都比不上人类的灵活。在已经有过对垒战象经验的宋军面前,从升龙府城中攻出来的象军,全然不是对手。

战象四散逃离,而被驱散的蛮兵便各自围了上去。他们都是见惯了大象,只要对上的是单独一头,都能轻而易举的解决。

完胜了几乎算得上是杀手锏的象军,除了天上的雨水,再也没有能阻止官军迈向胜利的脚步。

李常杰惨白的脸,被雨水淋着的样子甚为狼狈,但他还是不甘心认输,如果能借助城墙抵挡得住宋军的攻势,只要三五天,没有干净食水的生活就能让军营中疾疫大起,到时候他们甚至连退兵回国都做不到。

只是宋军没有给他这个时间。蛮军以及被他们驱赶着的交趾百姓,正用巨大的蕉叶层层裹起土块,不惧雨水冲刷,堆到城下,就像是一块块砖头,逐渐垒起官军攻上城头的道路。

没有弓箭的威胁,城中的守军又不能出击,垒土工程的进度远比李常杰在邕州城下时要快得多,从城头上徒劳的向下方射着软弱无力的弓箭,但射倒的又多是被逼来堆土的交趾百姓。

李常杰低头望着城下逐渐高起的土坡,恍惚间又回到了一年前的邕州,那时攻下邕州城似乎也就在这个时候,当时的苏缄是不是也跟自己一样的绝望?

不知怎么的,李常杰突然想起宋国民间的一句谚语:‘举头三尺有神明。’

“报应……”身后的将校中,不知是谁人说出了所有人的心声。

李常杰转身回头,看着面目仓皇的众将,还没开口,“太……太尉。”一名小校疯狂的冲到的面前,“南门……南门开城了!”

