\section{第29章 百虑救灾伤(九)}

“薛师正好大的手笔!”

政事堂中,吕惠卿拍着手,大赞着今天终于让新党一派扬眉吐气的功臣。

从未时开始,一辆辆满载着纲粮的马车沿着汴河,从南面抵达京城。最新的消息,抵京的雪橇车已有一百五十列之多。从已经点算出来的那一部分来推算,预计今天抵京的粮食数量当在两万五千石上下。

这两万五千石粮食,就像王安石狠狠甩上来的一耳光,让朝堂上下,所有摩拳擦掌、准备彻底掀翻王安石以及他追随者的政敌们,顿时没了言语。

作为同判三司,曾布也为此而欣喜万分。

曾布如今已经开始展望王安石离任后他自己的定位。据他所知,吕惠卿也在考虑着这个问题。这段时间,吕惠卿与吕嘉问走得很近,有什么盘算不问可知。市易务归于三司管辖,但吕嘉问有事不是去找王安石,就是去找吕惠卿,从来不理他曾子宣这位三司总计。

不过从新党的共同利益上来说,曾布必然要支持今次的行动。否则倒台的很可能不会是王安石,而是整个的新党——究竟如何,还要去看天子的想法,但曾布绝不愿意去赌这一把。

“两万五千石!若是水运倒也罢了,谁能想到用马车也能一日将如此之多的粮食运抵京城。”曾布轻松的笑着,多月来,这般轻松的心情已是难得一见。

“禀同判。”刚刚抵京,就被提到中书来禀事的押运官小声提醒着,“明天开始就不会有这么多了。”

王安石轻轻敲了敲桌案,就算没有押运官说明,他也知道真实的情况——六路发运司每天都有报告送抵中书门下,而薛向也都有将内容更为详尽的私函送到他的手中——如果不是薛向特意安排,抵京的粮食数量绝不会有今天这么多。

今天能一下有几百列雪橇车抵达京师,是因为薛向刻意要引起朝野的轰动,故意调整了运送的时间,使得这些车辆归并在同一天抵京。如果时间推移下去,每日抵京的雪橇车数量,就会恢复到正常的水平——大约一日八十辆到一百辆左右。

“以一车额定一百五十石的运载量算过来,也就一万二到一万五千石上下。”押运官说着自己所掌握的数字。

虽然比起今天的几乎是打了个对折,但一万两千到一万五千这个数字,也已经让王安石喜出望外。不但是王安石,吕惠卿、曾布,以及闻讯而来的吕嘉问都是欣喜难耐。

吕嘉问笑着,对着王安石:“自此之后,汴河的冬天不会再冷清了。”

“自是如此。”王安石笑着点头,又对押运官道:“再说说薛师正究竟是怎么安排你们运输粮纲的。”

押运官立刻回道:“小人等出来时,都受了学士的严令。在路上一刻也不得停,就算其中有一节损坏,就直接将卸下来,留下人看管和修理,而车子继续上路。到了每一天的落脚点后,也会将各车重新编组,恢复到一列四丈长、载重一百五十石的定额上。”

听说薛向的一番举措,吕惠卿半开玩笑的说道:“薛师正如今的龙图阁直学士做不久了。”

王安石连连颔首,薛向的确是没让他失望:“当奏禀天子以奖誉之。”接着他又问道,“一路上可有什么阻碍,道路的情况如何?”

“回相公,如今汴河水都已经冻透了底,比起最好的官道还要平整,一点麻烦都没有,跑起来轻快得很。就算冰道上有坑洞,以橇板的长度直接就跨过去了,很少会像车轮一样陷下去。”

汴河中的渠水正常的当是在六尺深,作为运河,河中的水源当然来自途经的各条河流。南段是长江来补水,过了洪泽后的中段是靠淮河,而过了宿州后的北段便是黄河。这几段由于地势高低不一,中间是靠着斗门【注1】来调节水深。到了冬天,连着黄河的河口为防冰凌,惯例都是要堵上。只要黄河河口不放水,从宿州到东京的这一段,残留的底水就只有一尺到两尺来深。

今年冬天还特别的冷——冬天的时候,越晴的天往往就越冷——南方传回来的灾情报告说,洞庭湖都上了冻,没法儿走船,在湖中东西二岛上种橘的百姓,甚至因为粮食送不上去已经有人饿死。故而到了汴河这边,更是早就给冻透了底。

天时害人,有时也能助人。‘祸兮福所倚,福兮祸所伏’,老聃的话自有至理在其中。

王安石闻言放松了一些,靠着椅背,笑着问道:“第一次走这条路应该很难吧?”

“禀相公,今次领头的都是老把式,虽然从来没有在冰上走过,也只花了一两天工夫就习惯了。其实跟路上走也差不多,稳着点就行了。”

“这一路过来,雪橇车究竟坏了多少?”吕嘉问跟着发问。

押运官道:“这新打造的雪橇车的确容易坏,坏得还不少。可这玩意儿也容易修,坏的地方基本上都是在支脚和雪橇上。就算不是木匠,换根木条也不过就是敲着钉子而已,不算多难,只是将粮食搬上搬下要耗费人工罢了。”

王安石一下坐直了身子:“那纲粮又有多少损耗?”

押运官皱眉想了一下,道:“回相公的话,不算多,大概一成左右,跟均输法实行前纲运的损失差不了多少。”

王安石与吕惠卿对视一眼,各自都点了点头,的确比他们预计的要好多了。

均输法实施前,运载粮食的纲船经常会在只有六尺深的汴河中莫名‘遇浪翻沉’,或是‘水侵舟上’,然后船上的粮食就由此飘没。六百万石纲粮外,还要加拨六十万石。后来均输法实行,加上薛向的铁腕治理,路上的损失这才下降到百分之二、三。

现在利用雪橇车运送纲粮的损失,虽然与均输法实行前相等,但这一个新奇的运输方式,主要损坏的是车,不是马,更不是车上的粮食。薛向在六路发运司多年,等到他教训发运司上下官吏,逐渐适应这一运输方式,途中粮秣损失比例应该还会下降不少。

该问的都问了,心中的问题都得到了解答,王安石抬手示意押运官离开,“好了!你下去先歇着去吧。今次尔等是辛苦了,改日朝廷必有封赏。”

宰相的赞许和许诺,让押运官大喜过望,磕了头后,连声谢着告退出去。

雪橇车的运力,今天到京城的数额不能作为依据。但这个冬天都能保持如今日一半以上的水平。也就是说,大约是纲船运力的一半左右。于此同时,付出的人力、物力和资源,则是水运的三倍以上。只考虑成本,当然不合算,但如果加入政治方面的考量,这份代价就实在是太便宜了。

王安石安心的长舒了一口气,不枉他一直相信薛向的才能。

儿子王雱从白马县回来后曾说,韩冈出主意的时候,多次担心六路发运司无法组织起这样大规模的运输活动。但薛向从一个背景浅薄的荫补官——乃是靠着祖父的恩荫为官,其父寂寂无名——一路毫无阻绊的走到了三司使的位置上,让无数进士咬牙切齿却只能暗自饮恨,他在治事上的才能,朝中首屈一指。所谓‘计算无遗策,用心至到’。即便王安石拿自己来比较,也只能甘拜下风。

王安石当日就知道,若说朝中有人能将此事做好,除了薛师正再无第二个人选。就算调了韩冈过来,他也差了薛向在六路发运司中的威望。他那个女婿是太小瞧人了!……不过说起智术,韩冈却是绝不输于薛向——

“好了。”王安石双眼一扫他的几位得力下属,“下面就按着既定的策略来做!”

纲粮抵达京师的消息已经在开封府中传开,百万军民昂首企盼。但出乎他们意料的,朝廷已经在城中开始平价发售运抵京城的粮食,可是能买到这些粮食的普通百姓却寥寥无几,第一批抵京的纲粮,几乎都被京城中的官宦人家给全数买走。

中书为此两天内连续发文六道,严令各处发售点,单人购粮的数额不许超过一斗。但这个命令却无济于事,京城的粮价并没有因此下降,甚至作为标志的米价,反而又涨了五文上去。

每天抵达京城的纲粮不断,可已经是腊月十九,剩下的时间中,即便发运司上下都不放年假,能在年节前运抵东京的粮食也是十分有限。而天子,是绝不会允许斗米一百三十五文的价格一直维持到年节时。

这一点,王安石知道、文武百官知道,粮商们也都清楚。虽然百姓们都在持币观望着,店中的粮食全都卖不出去,可粮商依然坚持将粮价维持在高位,定要逼迫王安石敞开常平仓!

粮价居高不下,散放纲粮亦是全无用处,今日的朝会上,便有人跳了出来。一名御史当着天子百官,高声质问着王安石,为什么还不敞开常平仓!

王安石容色平静,在朝会上直面着文武百官的质疑,眼神如同太行山上的花岗石一般坚硬。

当真他没有招数了吗?!

中书五房检正吕惠卿缓步出列,持笏向着赵顼一礼:“关于放粮平抑粮价一事,臣有一言请奏。”

注1:斗门,就是船闸的古称。在秦朝开凿的灵渠上便有使用,而在宋代沟通了三大水系的汴河上,蓄水隔水的斗门已经是保持运河通航必不可少的部分。

