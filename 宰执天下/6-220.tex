\section{第28章 夜钟初闻已生潮(中)}

“实在是太大方了。”

苏颂脸上的笑容渐渐化为苦涩,重复着,“实在太大方了。”

“啊,是啊。”韩冈深深吸了一口气,叹了出来,“实在是太大方了。”

上个月从辽国那边传来消息,辽国的伪帝耶律乙辛昭告天下,只要有人给他献上能实际使用的蒸汽机,他愿以王爵相酬。

此言一出,当即就传遍了天下。

不仅仅是辽国内部,就连大宋也在一两个月之内,传遍了几乎每一个州县。

尽管有很多人怀疑这个传言,但经过不同渠道的确认,这条消息是千真万确。

多少儒生冷嘲热讽,说辽国伪帝不分尊卑,覆亡指日可待,但这个天下,已经为之沸腾。

事先决没人能想到,辽国的皇帝会给出如此之高的悬赏。

一个郡王的头衔,就是在如今的大宋,也就只有赵姓宗室,或是已亡外戚才能得到。

其求贤若渴的态度,仿佛战国时的燕昭王。

区区燕国,一筑黄金台,便引来了乐毅这只金凤凰。当辽国皇帝筑起了黄金台,自命千里马却怀才不遇的有几个不动心?甚至愿意做马骨的,都是成千上万。

有不少得到消息的河北士人觉得,工匠只是马骨,他们士人才是千里马。

既然耶律乙辛能为从事贱役的工匠给出郡王的名爵,那么求贤若渴的大辽天子,肯定愿意拿出更高的回报——甚至有可能是说书人嘴里的一字并肩王——送给投奔他的国士。

仿佛是当年张元吴昊所引发的陕西士子投奔西夏的再现。

只这一个月,私下里穿越宋辽边界的士子已经抓到了十几人,没有被抓到的或许更多。这种的情况,不免让人觉得啼笑皆非,韩冈和苏颂都期待,这些表错情的士人都去了辽国,能多节省一些粮食下来。

可是当大宋的工匠开始听到这样的消息的时候,他们会不会心动?这根本就不需要多猜。再过几年,说不定辽国仿大宋设立的将作监中,都充斥了来自大宋的工匠。

这虽然是个笑话,但想到耶律乙辛对技术的重视,对比大宋内部的情况,实在让人笑不出来。

也许投奔辽国的工匠造不出实用的蒸汽机,但他们肯定能够大幅提升辽国的工业技术水平。

或许过两年辽国就能造出万斤以上的巨型火炮来,或许辽主身边的神火营又能扩大规模,或许,下一场宋辽之间的战争,首先响彻天地的,就是千百门火炮齐鸣的声音。

苏颂冲韩冈苦笑,耶律乙辛能给出的价码,他们给不了。除了严防死守,他们别无他法。

“没办法,”韩冈摊开手半开玩笑的说道,“一个是东家一个是掌柜,能做的主当然不一样。”

就算大宋不承认耶律乙辛他是皇帝,可他依然是货真价实的辽国之主。

他能随手拿出一个郡王相赠,而韩冈和苏颂,即使贵为宰相,就是拿出一个九品官作为悬赏,都会有人议论,他们是轻忽君子,重视小人。这怎么跟辽国比?

宋辽之间的差距再大,宋国的九品官还是远远比不上辽国的郡王。大宋的城门官或许能比欧洲的国王过得更舒适,但绝对比不上辽国的郡王。

“说起来,听到这个消息的时候,我都心动了。”苏颂半开玩笑的低声说着。

“子容兄你是开玩笑。”

苏颂的微微眯起眼,“一半,一半。”

韩冈现在终于感受到战国时候奔走列国的士人是什么样的感受了。在本国受人轻视,敌国却备受尊重,纵有拳拳报国之心,也经不起太多次的引诱。

所以商鞅会去秦国,所以乐毅会去燕国,所以大宋的工匠,日后会投奔辽国。

就是在韩冈看来,除了年纪大一点,耶律乙辛的作为,活脱脱的是让俄国崛起的彼得大帝。

“怎么办?”苏颂问道。

“静以观变。”韩冈说道,他对苏颂又说,“幸好有耶律乙辛只知道蒸汽机,就算他得到了能够实用化的蒸汽机,短时间内也很难运用到更多的地方上去。”

“因为辽人不种棉花?”

苏颂知道,韩冈给蒸汽机设计的诸多用途中,就有为他家产业扩张铺路的一条。

“现在的水力纺机,可以一个人照顾上百个纱锭。现在让棉布产出不能增加的缘由,只有棉花不足,水力不够。”韩冈拉着苏颂从水运时计台的小楼中出来,在司天监中慢慢走着,“棉花不足,可以扩大种植面积,西域、尤其是天山脚下及伊犁河谷,有着足够多的荒原可供开辟。但水力不足,就必需要蒸汽机了。”

苏颂慢慢点头。与韩冈相处许久,韩冈的一些惊世骇俗的观点,早已在潜移默化的改变苏颂的认识。对工业的重视,对只有嘴皮子利落的士人的鄙视,现在的苏颂,若是给二十年前的他看见,必然绝不相信这会是他自己的看法。

“士为首脑,农为脏腑,商乃血脉,兵乃肌肉,百工则就是骨骼,支撑起这个天下。等到蒸汽机出现于世间,铁和火支撑的骨架,能让汉家子民走遍这个世界。相信耶律乙辛也是看到了这一点,方才为蒸汽机给出悬赏。”

蒸汽机早就出现在《自然》之中,之后在《九域游记》里,甚至连原理都已经放出来了。活塞、曲轴、飞轮、锅炉,瓦特等人要费尽心血才能发明,只不过是韩冈过去教科书上的一张图片。

只是很长一段时间以来,制造出来的蒸汽机,只具有爆炸性,还不具备实用性。直至如今,方有了一点眉目。辽国想要通过悬赏更早一步造出蒸汽机,这不是不可能。在这个工业技术刚刚开始冒头的时代,一名天才的灵光一闪,能抵得上一百名工匠的绞尽脑汁。苏颂在捅破摆钟的最后一层窗户纸用了五年的时间,换个人,说不定回去想一个晚上就突破了。

但只要有人开了头,其他人追上去却很容易,辽国通过蒸汽机占据不了太多的优势,甚至连改变宋辽两国之间的实力对比都很难。

科学技术的整体发展,不是光靠少数工匠就能做到的,不论招揽了几十几百名工匠研究蒸汽机,这数量依然显得太少。

韩冈成立了工学,鼓励读书人成为技术人员,他更扩大了蒙学,希望日后从中出现更多的人才。合格的工程师,只要有了足够的数量,就不是一二天才能够弥补得了。

而且韩冈的工作重心,已经从轻工业,转向了重工业。

在过去,矿冶业的主角,都是民户。徐州铁冶的三十六冶,就是由大户承包下来,参与到开采之中的冶工动辄以万人计。朝廷从中课税,然后再视需要多少进行和买。

如今因为工业化生产的需要,各地矿业都逐渐变成了由国家控制下的大规模生产。徐州成了北方排名第二的钢铁基地,三十六冶变成了大大小小十一座高炉,下面的矿工、冶户,都被朝廷吞了下来。

随着钢铁业的扩大,如今在技术上,已经达到了新的瓶颈。在过去,只是进行进行微小的改变,即能带来丰厚的收益,但现在,开发新工艺的投入越来越多,风险性也在加大,韩冈为此投入更多的精力来实现他的目标,而成果也在一一显现。

时钟并不包括在内,但技术的进步,时钟仅仅是其中之一。

“以辽国的技术水准,当他们开始蒸汽船的时候,大宋这一边也能够将蒸汽机放在火车头上了,这是底蕴上的差距。什么时候辽国能大规模制造蒸汽机,我们大宋绝不会迟上一年半载。”

这是韩冈的自信。

苏颂笑着点头,但很快又叹了起来,:“朝堂中还有人说把工匠都抓起来关好,让他们用心去做事,要赶在辽人之前。”

韩冈在来到这个世界之前,曾经听说过有所谓的囚徒设计局。因为不想让政治运动的波浪破坏国家急需的新武器的研究,掀起政治运动的主使者,便将整个武器设计局中的所有人都关进了集中营,让他们以囚徒的身份从事各自的工作。

这个时代,竟然有人能想到类似于囚徒设计局的点子,当真是超越了时代的局限。只是一下子超前了那么多步,当是又疯又蠢的白痴无疑了。

“这种蠢话提都不要再提了。一边是做奴隶,一边是做王公,白痴都知道该怎么选了。”韩冈冷笑,“朝中重臣,鱼袋狨座,尚不如蛮人有见识。”

“肉食者鄙,虽不尽然,却也有几分合乎道理。”苏颂忽然郑重起来,“不过玉昆,千万不要小觑耶律乙辛。”

“子容兄放心,都能篡国权奸,如何还敢瞧不起?”

提出蒸汽机的是韩冈,让蒸汽机超越火炮成为一个标志的也是韩冈,当耶律乙辛将悬赏高高挂在蒸汽机上的时候,这件事怎么会不牵连到韩冈身上?

总有人想藉此发难,或是将韩冈拖入浑水中沾上一身脏。

耶律乙辛把悬赏拿出来的时候,当也正看到了这一步。
