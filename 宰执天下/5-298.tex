\section{第31章 停云静听曲中意(五)}

“曾参政今天可没法儿出门见客了。”

“是新来的曾参政?他怎么不能见客了?”

“还有哪个曾参政?他家昨夜走了水,听说他是胡须都烧光了,头发也不剩多少。”

“也不止曾参政一家,去东十字大街后看看就知道。整整烧了四个半坊,旧城左军第一厢才几个坊?宝积坊、安业坊几乎都没一间好屋子了。”

“……这还真不得了!”

“出事的人倒没听说有多少。方才俺才从马行街那边过来,东城医院里也没收多少烧伤的。”

“烧了房子倒是小事,参知政事唉,朝廷还不得贴补他一座新宅。房子也是官中的,烧掉也是官家来心疼。只是曾参政家的家当全没了,从火场逃出来的时候连件袍子都没穿,一家老小就剩了一件小衣。现在的外套都还是借了大相国寺的屋子,寺中和尚送上的。”

“其实听说一开始火是从景宁坊烧起来的,隔了一条街,本是烧不到宝积坊去,谁知道突然刮了一场风。原本都快灭掉的火头一下就又烧起来了。差点就把将作监”

“曾参政的运气还真是不好。”

“谁让他家将马槽放在西北角的,草料都堆在那里,一起火,浇水都来不及。楚国大长公主和高平郡公两家同样都在宝积坊,他们两家可是将箱笼都搬出来了,搬出来的细软占了整整半条街。”

一场火后,半真半假的流言一如既往的在京城中传播。

宝积坊中,一个直学士、一个开国郡公、一个大长公主,再有一个参知政事,这是宅院全都被烧光的。至于烧了一半的住户,烧了三四成的人家,更是遍布了旧城左军第一厢贴近皇城城墙的那一个角落。

当事人可谓是哀鸿遍野,可聚在街头巷尾的人们一说起事不关己的八卦来,空气中就充满了快活的气氛,人人眼中闪着幸灾乐祸的光芒。甚至一时间都忘了去讨论皇帝的病情好转的消息——毕竟只是几根手指。

并不是说曾布等人的人缘很糟,京城的百姓有几个能记得熙宁六年七年时的翰林学士究竟做了什么?一个普通的官员在外面度过三五年,对京城来说,就是一个陌生人了。大长公主、开国郡公也同样离得很远。只是任何一名高官显宦落得灰头土脸,人们都会津津有味的咀嚼再三。

韩冈估计自己或许能例外……也只是或许。

夜里曾布等人的家宅刚刚被一把火烧了个精光,今天一早皇后就已经一股脑的派出十几人出来慰问,并探查灾情。不过韩冈觉得,曾布他们现在更需要的是一间新宅子,还有如何弥补被毁损的家当。

宅子还好说,基本上都是官产,可家当就难办了。这个时代可没有保险业,烧光了家产,朝廷给点补助和赏赐已经是天恩浩荡了——也不能说完全没有,如今在南方的海运中,货主为上船的货物额外向船行支付一笔保金,可以说是保险业的雏形——不过这件事跟曾布扯不上关系。

何矩是代表顺丰行上门来拜年的,又在韩冈面前说起了这桩新闻。作为顺丰行在京城的第一交椅,在蹴鞠、赛马两大总社中拥有投票权的代表,能进出大多数权贵家门的豪商,他带来的消息就没有多少谣言的成分了,而是更近于真实。

“曾参政家实在是运气不好,隔了怕不有二十丈,一阵风就把火头给卷过来了。根本是天降横祸,连家当都没收拾。也幸好曾参政刚刚回家,家中也无人入睡,否则还不知道有多少人会葬身火海。”

“现在曾子宣借住大相国寺?”

“就是当年狄武襄借住的那间院子。”何矩神秘兮兮的压低声线。

韩冈楞了一下。狄青当年可就是因为借助大相国寺时犯了火禁,京师水灾时又爬到大雄宝殿的顶上观水情,被御史们抓到了把柄。

“他还真不怕忌讳!”

何矩满不在意:“反正朝廷肯定会赐个新宅的。学士也不必为此担心。”

谁担心了?韩冈可没多余的同情心。他端起茶盏喝了一口,很是舒坦的眯起眼睛:“你们打算怎么做?”何矩说得这么详细,总不会是因为八卦心作祟。

何矩正看着韩冈茶盏中那汪黄绿色的茶水。韩冈喜欢喝炒制的山茶,而不是贵重的龙凤团茶,这一点在京城中很有名。而且现在也已经有很多人家学着他去和炒青山茶了。谁让韩冈担了个药王弟子的名头?连韩家的饮食习惯都成为外界模仿的对象。要是能出个韩家食谱,肯定能大卖特卖。

摇头挥去心中的杂念,他回话道:“南丰曾家在江西亲朋故旧无数,会中有不少家想跟他结个善缘的。”

“顺丰行呢?”韩冈悠然问道。

何矩老老实实的回答:“学士和东翁都曾吩咐过,行中只要抓住西、南、中三条线就够了。”

韩冈是雍秦商会的总后台,连带着顺丰行也成了商会中的头面角色。韩冈和冯从义并不打算将顺丰行的商业网络扩张得太厉害,巩州、交州、襄阳、京城,有这四个点也就足够撑起天下顶级大商号的架子了。按韩冈常说的话,钱是赚不完的。但其他商会的成员却有不少想将手伸入南方。何矩这位大掌柜行事时也不免要为整个商会的利益去考虑。

“要不要我写几个帖子。”

“学士亲笔写的帖子,可没人会送出去呢。”何矩开了个玩笑,却是婉拒了韩冈的提议。

韩冈点了点头。他也明白,商业上的事,他去插手其间,政治意味就会显得太浓了一点。不过这个何矩,心性倒是很让人欣赏。

“这一回将作监倒是没事。”何矩其实也有些紧张,急急的换了话题,“换作是一个月前,围墙没有增筑,那可就说不准了。”

“可惜吗?”

何矩也不瞒韩冈,舔了舔嘴唇:“像烧制玻璃那样的技艺,能再流出多一点就好了。将作监总是遮着掩着。”

韩冈微微一笑,又低头喝了一口茶。

将作监的几十个工坊运气很好,逃过了一劫,这是让韩冈……怎么说呢,十分遗憾的一件事。

百万人口的京师,对火灾的应对在这个世界上应该是能稳坐第一的。虽然是烧了四座坊,但要不是有专业的消防力量,烧掉整个旧城左军第一厢都不是不可能。

京城就这么大,偏偏皇帝家还要将各色作坊都压在京城中。要知道,将作监在很大程度上是专门为皇家服务的,所谓的宫粉、宫花,各色上用的器皿都是将作监辖下的工坊来制造。更甚者,酱菜、酱油、酒水之类的日常消耗,也都有专门的作坊来为皇家制作。而最重要的是打造玻璃、车辆的工坊,同样有专门为皇家设立。

要是一把火烧了两三个作坊,不说京城中还能省下一片地来,韩冈还可以建议将大部分工坊都迁出京城——如今就只有官窑等作坊是在京外——一旦迁出之后,他也就能更方便伸手进去了。

韩冈希望工匠们的才智也能发光发热。以皇家工匠们的技术水平,不应该在技术进步上输给韩家的草台班子。可巩州那边连半尺见方的平板玻璃都出来了,已经都能用在温室修建上,而将作监的玻璃工坊却还在继续制造鱼缸和花瓶。

许多秘藏在将作监中的技术应该更加公开,就像金属丝拉制的技术,将作监用来制造金银首饰,韩冈却觉得可以用在更实用的地方——也不知道能不能用在铜铁这类平价的金属上。

当然了,将作监韩冈是不在乎、甚至盼望过一遍火。但军器监可就一点火星韩冈都不想看到,他不敢想象要是军器监被烧了,会对大宋造成什么样影响。若当真发生了,可以说是灾难性的后果。

只是这些话韩冈只会藏在心底,不会吐露出来,

“学士。”何矩与韩冈又说了两句闲话,神色一肃,低声问道:“这一回钱大府能不能逃过一劫?”

钱藻吗?

韩冈沉吟了一下,反问道:“外面怎么说?”

“外面都说青城行宫的那一桩事还没处置,现在又贴着皇城烧了四座坊,钱大府过不了这一关了。”

火灾没有造成太大的人员伤亡,这的确是万幸。不过起火的位置是皇城外。达官贵人受灾严重。按照官僚系统的老习惯,在收拾残局之前,往往会有更为重要的一桩事要去做——推卸责任。

在官僚们的心目中,将过错推诿于他人,这比起争取功劳都要重要。

在郊祀之后,钱藻就因为桥道顿递使的差事没有办好,受到了御史的弹劾,甚至有人指责营房损坏致使多人伤亡,才是造成天子暴病不起的主因。不过之后紧接着朝堂动荡,御史台大换血,让他躲过了一劫。可现在终于是躲不过去了。

大年夜的这一场火一烧,钱藻就算再恋栈不去,也不得不上表请辞。何况开封府这个苦差事,基本上任何人做上一两年就想要活动着离开了。钱藻他也不例外。

接下来谁做开封知府?
