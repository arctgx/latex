\section{第一章 庙堂纷纷策平戎(四)}

“恐难说服天子。”

韩冈摇头,拒绝得很干脆。赵顼是不会信的,当然韩冈也不会相信。

郭逵想回太原的打算,让韩冈觉得很意外。好端端的执政不做,怎么想着出外?

何况推荐郭逵出外,韩冈自问也没有这个能力。先不说他自己的资格够不够推荐,即便是旁敲侧击,为郭逵敲边鼓,也显得太过突兀。

而且韩冈也不会想看到郭逵去河东。鄜延路紧邻河东,郭逵去了太原,等于是跟种谔打擂台——确切点说,是郭逵单方面踩种谔,鼎鼎有名的种五还没有上西府执政擂台的资格。

再怎么说,郭逵都是两次晋身西府的老资格执政,军方将帅中的第一人。一旦他到了河东,参与进平夏之役中,自然而然的就会侵占到种谔的职权。等到各路兵马合兵一处,郭逵必然坐在中军帐中,而种谔等一干西军将领,就得老老实实的左右站着。

本来各路兵马为了占据灭国的首功都会互不相让,郭逵这么一去,内部更是就要先斗起来了。这是将平夏大军往悬崖下面推,韩冈哪里可能会支持?!

以郭逵的见识,不可能会想不到这一点。因此韩冈对郭忠孝的转述满心都是怀疑。郭忠孝说得越多,越详细,韩冈的疑心就越重。

“立之。”韩冈打断郭忠孝的辩解,“太尉到底是想去河东还是河北?”

“确实是河东。”郭忠孝停了一下,偷眼看了看韩冈,就有几分尴尬的说道,“如果河东去不了,就只能退而求其次。家严世受国恩,无论如何,也不能留在京城坐看危局。”

起身送了郭忠孝出门,韩冈立刻就挂了脸下来。

郭逵是不甘寂寞,所以才求到自己的头上。但遮遮掩掩的态度,让韩冈很不喜欢。

虽说郭逵想去河北的打算,只要深思一下,不难猜的出来——任谁都明白,天子不会给他去河东或是陕西的机会,正如郭忠孝所言,去河北的确是退而求其次——当韩冈更希望郭忠孝能坦率点说出来,而不是玩弄什么纵横之术。

只是郭逵去了河北又能如何?

郭逵为此拿出来的交换条件,可是熙河路几个重要军职。郭忠孝帮老子开条件时,脸都红的。

尽管不是王舜臣、赵隆和李信他们能担任的职位——他们的身份已经太高了,要调动肯定的经过天子这一道关口——仅仅是低层将校的人事安排,但对于想要在熙河路厚植根基的韩冈来说,却是再合适不过的回报。

这样价码,郭逵求的却仅仅是韩冈说上一句话,似乎太大方了一点。

难道有什么是自己没看到的?还是说郭逵手上掌握着自己所不知道的情报?

回到书房,几个孩子早就回去睡了,韩冈靠在摇椅上冥思苦想。摇椅前后摇摆,但却没有摇出韩冈想要的一闪灵光。

严素心端着夜宵进来的时候,就看见韩冈闭着眼睛躺在摇椅上,像是睡着了,但看到他紧紧皱起的眉头,就知道还是想着让人费神的公事。

“官人。”

轻柔的声音在耳边回响,韩冈睁开眼,眼前的是一盅冬日温补用的羊肉枸杞汤,正袅袅冒着热气。诱人馋涎的香味,随着热气一起飘散开来。

美食在前,关切的眼神让韩冈展颜一笑,暂时放下了心事。

端过今晚夜宵,浓白色的汤中点缀着几颗鲜红的枸杞,还有两段鲜绿的小葱。严素心做菜,在外观上也很下功夫。红色、绿色加上做底色的白色,以及杯壁上的,小小的一盅羊肉汤,还没有开吃,就已经觉得赏心悦目了。

严素心做的羊肉入口即化,没有膻味,而带着枸杞的淡淡甜味。

“炖了多久?”韩冈又喝了一口汤。汤中的鲜香浓而不烈,正和他的口味。

“一天。在小灶上炖的。”素心在韩冈身边坐下,带着笑,看韩冈连汤带肉的大口吃着,“用的是腰肋上的肉,枸杞是前些日子从陇西送来的。”

韩冈一边吃,一边有一句没一句的和素心说着话。吃饱了之后,韩冈突然发现之前困扰自己的问题,现在想想,却也没必要那么去追究答案。

也算是自家的老毛病了,什么都要追根究底,越是想不通,就越是会全神贯注的去寻找答案。其实只要等一等,郭逵有什么盘算便能一目了然。手上信息不足的情况下,想得再多也平白耗神而已。

还是再等等看,既然一时想不透,就等着看郭逵的葫芦里面卖的到底是什么药好了。

反正最后的决定权在赵顼手中,韩冈也不介意帮郭逵说句话。如果北方的局势有什么变化,有郭逵在河北,还能让人放心些。

……………………

赵顼又是熬了夜。

到了快三更的时候,他都还没有去安歇。

纵然眼睑都已经透着疲劳过度的青黑,但赵顼的双眼晶亮,精神依然旺健。

武英殿的偏殿中,灯火通明,大宋的当今天子正守着一幅沙盘,专心致志的摆弄象征一支支军队的小旗。

跌宕起伏的构成,代表着西北地势。居于中央的西夏,被大宋六个经略安抚使路所包围。围绕在沙盘上的西夏国周围,现在是一圈密密麻麻的小旗,代表着六个路,加起来少说也有四十万的总兵力。

这一次灭亡西夏的战事,将会是六路同时出发的行动。赵顼决心用一次狮子搏兔的攻势,将所有失败的可能全都给堵上。如此庞大的兵力,是如今正陷入困境之中的西夏君臣所不能抵挡的。从过往的战绩来看,任何两路的合力,都能正面击败西夏全军,而赵顼将要动用的是六路!

在过去,在大宋的西北边陲,也从来没有过一次规模相近的战争。宋夏两国数千里的疆界上,将会有至少三十万的兵力出战。是真实存在的兵力,而不是用来恐吓敌人的浮夸。

赵顼脸上浮现出一丝自得,也就只有现在,经过十年的变法,由此积攒下来的财富,才能支撑得起这一规模的战争。

粮草、军饷、兵甲、战具,都是堆积如山,随时可以去取用。将领、士卒,无一不是经历过战争的精锐。这是用了十多年才积攒下来的成果。一旦投入下去——赵顼有自信——就是全盛时期的辽国,也要暂避锋芒。

从还在做太子的时候起,赵顼就一门心思的想着灭亡西夏,击败辽国,收复兴灵和燕云。能让大宋,像汉、唐一般让四夷宾服。

尽管步履艰难,但自己还是一步步的做到了。到了如今,旧时梦想已经是近在眼前,仿佛触手可及。

上天都在帮他,大宋的敌人,一个两个全都陷入了内乱。这么好的运气,就是赵顼过去在睡梦中,也从来不曾去幻想过。

世上哪有这么好的事?!

如果有人事先对他说:日后有一天,辽国会因权臣害死皇帝,西夏会因母子权力之争,在同一时间发生内乱。赵顼能给予的回答只会是一阵开怀大笑,也许会因当时心情的不同,给予处罚或是赏赐,反正是绝对不会相信的。

但现在,却是个摆在眼前的现实。

两个死敌都陷入了混乱,西夏的灭亡,也已经是指日可待了。

赵顼心如烈火焚烧,恨不得立刻就能听到官军攻入兴庆府的捷报。恨不得现在就看到秉常和梁氏母子两人,被械送到自己的面前。

这么好的局面,朝中竟然还有人说要小心、慎重,就不怕贻误战机。

郭逵老了,韩冈因为功劳挣得太多,也没了上进的动力,两人现在一味求稳。

想那韩冈,两年前,他和章惇从京城一路南下,抵达桂州后,又马不停蹄的杀到邕州城外,大败李常杰。当时可没说半句要慎重行事。

难道西军众将都不通兵事,为何他们都说如今正是一捣兴庆府的良机?

‘王中正也该到了。’赵顼想着。

论起军事,王中正当是内侍中的第一人。不论是在横山还是在河湟,都有着赞画辅佐之功。独立领军,也能一战平复西南。本人又有胆略,当年在罗兀城被西贼围困时,竟能主动入城。就是稍差一点的西军将校,也难比得上他,也只有种谔等寥寥数人,才能勉强压过他一头去。赵顼想听一听他的意见,这一次,也可以让他独领一路。

赵顼屈起手指,王中正一路、种谔一路、高遵裕一路,这三路的主帅可以定下来,但剩下的三路,该怎安排,得好好想想。如今将才不缺,帅才却难得,要将六路兵马的主帅都安排下来,还要有一番头疼。

赵顼并不准备设立指挥全军的宣抚司或是总管司。数千里的国境上,从河东到熙河,消息往来都要一个月,设立一名统括全局的主帅,根本没有任何意义。他打算让各路各自对付面前的敌军,最后汇聚到灵州城下,自然而然的,就能将胜利抓到手中。

“官家,该歇息了。”李舜举再一次过来,规劝赵顼早点休息,“连着几天都睡得这么迟,肯定会惊动了太皇太后和太后。”

赵顼应了一声,却站在沙盘边动也不动。

李舜举苦着脸正要再催促,突然间就听天子道:“对了,去选个好日子,就在年前给六哥和淑寿将痘种了。……上天垂顾我大宋,必不会看着朕的皇嗣再有任何意外。”

