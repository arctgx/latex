\section{第26章 鸿信飞报犹觉迟(二)}

今天见了两名疍民首领,韩冈一看到他们身上的穿戴,就知道这两个人是不能用的。通过正面的交流,更是确认了这一点。

虽然他们也是疍民,但却是压榨贫苦疍民的吸血鬼,自己吃得脑满肠肥,却不顾他们之下的族人。

两人手上的上千户疍民,就是他们最大的一笔财富。每年都能靠着疍民得到几千上万贯的收入,在钦州城中还有一份产业,试问怎么可能放弃这一切?

不仅仅是钦州的两名疍民首领,南方沿海诸路的疍民首领,应该都不能为己所用。

疍民有户籍的不多,基本上都是大小头领才会有。需要交纳税赋和劳役时,官府都直接找这些首领,再由首领摊派下去。没有户籍的疍民,其实就相当于首领们的部曲,死活都是各家首领说得算的。

对部曲拥有生杀予夺的权力,想要一句话就让他们放弃……这件事,只要稍微想一想,就知道完全不可能。他们尽管看起来的确是畏惧自己,可若是触动到他们的利益,也是会拼命的。

韩冈也不想再看着他们磕头求饶了,“算了,你们两个且起来罢……关于此事,本官也不会强逼尔等。愿意去也好,不愿去也好,一切都由你们自行决定。回去花上十天半个月聚起来商量一下,问问你们下面的人,想去的就直接去交州,不想去的就留在钦州好了。不要急着给本官回复。”

两名疍民首领就这么被人领着下去了,看他们脸上的表情,都是一副疑惑不解的样子。韩冈未免太好说话了,可一点也不像传说中,一句话就让交趾男丁都成了残废的小韩相公。

韩冈对他们脸上的疑惑笑了一笑。这也不怪他们,他只是不心急而已,所以看着好说话。

接下来他可是要张榜公布,并让,想必武福、俞亭二人无法将下面的疍民耳朵和眼睛全都蒙上。只要有一两个人感兴趣,并同意迁往交州就够了。

所谓的手段不过威逼利诱四个字。但要用得恰到好处,却不是那么容易。一干绊脚石,强行拔出只会生乱,得先让疍民中有了不同的想法,官府才好插手进去。最好的办法是先塑造两个典型出来,拿他们做范例,只要有好处,总会慢慢吸引人来。

如果那时候武福、俞亭这两名首领们还敢于阻拦,正好可以一并解决——钦州被攻破时,疍民们乘火打劫的事,韩冈可是还记得清清楚楚。这两位首领纵然不是直接的煽动者,也肯定对此进行了默许,若要问罪,少不了他们一份。

这一桩事,需要把握分寸和节奏,韩冈也只能从转运司中直接插手处置,如果交给地方上来管,多半就会将告示一贴,然后强迫疍民迁往交州,最后好事变成坏事。

韩冈自嘲笑了一笑,他在广西还有的是时间,可以等到疍民慢慢来投。

……………………

章恂抵达交州的消息,传到韩冈手上的时候,他正好结束了对廉州、钦州的巡视,抵达了邕州。

在廉州,韩冈也招来了当地的疍民首领询问,不出意外的得到了否定的回复。除此之外,倒是趁着偶然一日的晴天,去了海角亭一游。

此时的天涯海角不在海南,而在钦州、廉州。钦州有天涯亭,廉州是海角亭。不过也无甚特异之处。眼下论起大宋诸多军州,哪里一座更靠南,答案当然不会是钦州和廉州,人人都知道交州更南面一点。

廉州的知州还邀请韩冈顺便去合浦的断望地一游——那里产的珍珠,才能被称为合浦珠。不过韩冈想想还是算了,婉拒了盛情的邀约。他对珍珠没兴趣,瓜田李下的嫌疑也没必要沾。

在广西南方诸州绕了一圈,韩冈终于回到了邕州。

做着邕州知州的苏子元也算是能吏,治事手段很是出众。加上他并不光是继承父亲的余荫,本人也是立下了大功,在邕州名望极高,吩咐下去的事,无人会拖延推诿。

在他的治理下,同时也是经过了近两年的休养生息,邕州也算是热闹了起来。街市上人声鼎沸,几条商业街,都是挤满了人马和车辆。

城中的废墟早就被清理干净,而空出来的地皮,则已经大半被人买了去,到处都能看到有人在置屋建房。

韩冈进城后一路走过来,满意的点着头,比他三个月前来的时候,城中可是又多了许多建筑。

韩冈在邕州威望尤髙,仅次于不在人世的苏缄,为了省些麻烦,他进了邕州城便偃旗息鼓,只派了个亲随去通知苏子元,自己径直去了州学安身。

州学中的学生们也多了一些,他们刚刚考过了月考,正是呼朋唤友,准备出去放松一下。只是韩冈一到,便把他们都吓得乖乖的守在学校里。

韩冈看了一下他们的试卷,题目和答案都偏向关学,而且也多了水利、农事和兵法方面的条目。

受了韩冈的影响,苏子元对于经义方面的理解逐渐偏向于关学,而刚刚走马上任的邕州学官则是韩冈的幕僚李复,他因功得官后就被韩冈推荐到了这个位置上。在韩冈、苏子元和李复的影响下,广西的士林风气渐渐偏近于关学一脉,

而且研习关学,对他们也有实际好处。广西的士子基本上都不指望能中进士,只是若能在州学里出人头地,那么也是能出来做个摄官——虽无正式告身,任官也不经流内铨考核,人称‘假版官’,故而名‘摄’——但经过几次磨勘,也是有转为正式官员的资格。而摄官考试的主考官,就是转运使,也就是他韩冈。

在李复的陪同下,韩冈对几个出色的学生加以褒奖,又对考试不合格的学生则先是训斥,之后又勉励了一番。过了一阵,得到消息的苏子元,从衙门中过来了,要为韩冈接风洗尘。

苏子元此时已经是韩冈的亲家了,也就在半年前,在忠勇祠前处斩了一批屠戮邕州的战犯之后,韩冈便代长子韩钟向苏子元的女儿提亲。

尽管孝期未过,不便议论婚嫁。但以韩冈和苏子元两人的交情来说,口头上的约定已经足够了。等除了孝,再去完成通名、纳彩的定亲之仪也不迟。

接风宴之后,也就是晚上歇下来的时候,韩冈收到了章恂到了交州的消息。

此外还有一份由顺丰行交州分号送来的汇报,上面说了这个月在交州的业务开拓情况,另外还提到了入股一名福建商人的香药买卖——米彧这个名字,韩冈依稀还有点印象,似乎是表兄李信提起过的。

随着章恂的到来,章家便算是在交州扎下根来。不过正式出面组建商行的当然不会是章恂,章家在商事上也是有其代理人的。

韩冈也不会为章恂的事多费心,交州知州李丰就是章惇的门客,当然一切会照顾好。他唯一的企盼,就是希望章家能戒了急功近利的心思,能安心下来置办产业,而不是局限在贩卖转运的行当上——只是章恂一到交州,便急着询问各色香药的出产,韩冈的忧虑就不是杞人忧天。

烛台下,韩冈提着笔,考虑着该怎么给章惇写信比较合适,疏不间亲,这措辞上就得很费思量。

在摇晃的烛光下,用了大约两个时辰,韩冈终于搁下笔来,揉了揉又胀又痛的双眼。也算是知道为什么欧阳修近视得近乎是睁眼瞎了,马上、枕上、厕上,连这三个地方都不把书放下来,眼睛不坏才有鬼。

在信中,韩冈对章家过于关注处在风尖浪口上的香药贸易劝谏了两句,不过更多的还是透露了一点关于白糖制取的技术,并提议两家在交州设立制糖作坊,联合七十二部一起垄断交州糖业。晓之以理、诱之以利,双管齐下,总是最行之有效的手段。

韩冈与章惇之父章俞有救命之恩,有这份恩情在,章、韩两家如今才会走得如此之近。加之韩冈和章惇又是政坛上的盟友,两边的关系也更加牢固。不过恩情会渐渐淡忘,政坛盟友也会因为政见不合渐渐分离,但若是有着金钱的浇灌,这份盟约当能更加牢固,维持的时间也更加久长。

另一个对章俞有救命之恩的刘仲武,章惇到枢密府上任之后没两天,就找了个机会在天子面前提名,准备将他调去陕西边关立功。不过刘仲武运气甚好,在廷对时让赵顼给看中了,留在了京中任职。

相对而言,王舜臣就没那个运气,他上京的次数比刘仲武还多,半年前还因为葭芦川大捷的缘故上京面圣,可也是年纪的缘故,依然是留在都监一级——虽然得官时是改了年纪的,但还是显得过于年轻。加上功劳不比李信那般光辉耀眼,便跟留在熙河路的赵隆一样,都是都监都巡,无法再往上走到一路钤辖的位置上。

但话说回来,王舜臣和赵隆两人都已经是从七品的诸司副使,以他们这个年纪来说,军中也没几人能比得上。日后只要有机会,打上几场胜仗,三衙管军的几个位置,照样能争上一争。

自己处身官场,位置已然甚高,日后有望跻身政事堂。军中则有亲朋好友,加上自己的声望,一切更不在话下。再加上自己在工商二事上的布置,钱财方面也不会缺少。还有自家在关学一脉中的地位,人才也不会少。

人、财、权,都不缺,未来在朝堂上的地位可以想见。韩冈在幽暗的灯火下沉吟着。这样的布局还是差了一点,依然不够牢靠。

不过,要想更为牢靠的办法也是有的,眼下也有了眉目,离着成功也为之不远。只要能够成功实现,自己的地位将会比眼下牢固百倍!

