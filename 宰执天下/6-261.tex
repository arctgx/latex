\section{第35章 历历新事皆旧史(上)}

刚刚过了年,距离春暖花开、草长莺飞、万物生发的时节,至少还有一个月的时间,但韩冈已经发现自己手上的事情也像地里初生的野草一样,一个劲的冒出头来。

天下虽云无事,之前一年又是一个风调雨顺、四方安宁的年头,铁路正在延伸,时钟和蒸汽机也在推广和改进,《自然》推出了科举专刊,专门解说科举中有关气学与格物的考试内容,销量再一次突飞猛进,一切都在顺利的进行之中。

可韩冈想做的事情很多,手上的事情当然少不了。

吹了半年的风,科举制度的改革将在今年正式开始。各地学官、各路学政对此欣喜欲狂,没有几个官员会嫌自己手上的权力变大。不过韩冈不仅要为气学去争夺最大的那块饼,还要提防着新学从中掏掏摸摸,当然是有的忙。

科举制度的改革,不仅仅是为了扩大气学的自留地,也是为了减少官吏对工厂盘剥,赋予工厂主以地位,或者说鼓励工厂主去追求地位。

也因此,有关开办工厂的事务也多了起来。通过科举改革,朝廷上已经开始鼓励各地开办工厂。每多一个工人,就会少一个流民,地方上人口渐多,而土地数量增加缓慢,工业吸收劳动力的作用在韩冈的鼓吹下,越来越为世人所认知。只为了推进工业发展,韩冈也闲不下来。

最重要的,还是韩冈打算将预算制度需要提上台面了。总不能继续过量入为出的日子,更不能量出为入,去盘剥百姓。只是这么做的话,财政制度要大改,相应的,也会牵动许多官员的职位,而且,不论在谁看来,这都是宰相侵夺财权的手段——就是韩冈自己也不会否认——想要达到目的,韩冈当然要下更多的功夫。

虽然韩冈已经因此而忙忙碌碌,可除了这些政事之外,还有好几桩喜事等着他。

新的一年里,家中,有长子、长女的婚事,朝中,还有天子和内侄女的婚事。

听着喜气洋洋,实际上却是家里家外都是忙得脚不沾地——当然,这不是韩冈。

家中的婚事,韩冈让王旖去主持,流程和细节上则交给了专业人士。

京师的红白事,主家只要给钱,从仪式到宴席,全都可以给你办得妥妥帖帖。主家只要听着吩咐去做就是了。

即使是官宦显贵家的婚事,礼院中也能找来一批惯办红白事的礼官来主持。亦不须主家多费精力。

韩冈的麻烦,主要麻烦在他乃当世大儒,在礼法上到底是要遵循古礼,还是今人礼节。

韩冈没有多费心思,将田腴、邵清几个在礼法上有想法的同窗请来,共同议定婚仪,基本上,还是以如今通行的仪式为主,只是去了一些恶俗的环节。由此也作为气学门人的礼仪标准,就像乡规民约一样,愿不愿意遵守,就看各人了。

不过皇帝的婚事,就不能像家里一样来处置了。

“官人……越娘的婚期就托付给官人了,可别真的让她刚嫁过去,就多了个克夫的名号。你一向与二兄交好,二兄都上门求了你,你可要帮帮越娘啊。”

韩冈今日出门时,王旖难得的拉着他殷殷相求。

韩冈半开玩笑的说着,“我要看人面子,也是看我家娘子的,可不会看他王仲元的面子。”声音又柔和了起来,在妻子耳边道,“昨天晚上不是就说了嘛,你放心好了。为夫一定尽力的。”

王旖点点头,放开了手,笑着目送韩冈离开,但眉宇间,却又是一幅难以释怀的样子。

终究,韩冈也只说了一句‘尽量’,没有做出保证。

皇帝的这桩婚事很是磨人,已经不是宰相想怎么样就能怎么样了。

之前先是为了皇后的人选争执了许久,等到人选确定,在商议婚礼细节时。先是礼官对婚仪吹毛求疵,朝臣争执良久,继而,钦天监那边也给人添乱。

也就在年前,天文官为天子选定了大婚的黄道吉日,定在了今年的五月十六。

但这五月十六却是世间所谓的天地合日。依如今道家的说法,五月十六是天地相合之日,夫妻之间不得敦伦,甚至得相背而睡,否则便会遭逢不幸,尤其是丈夫,更会减损寿数。既然连周公之礼都行不得,更不用说婚礼了。

这个日子一出,朱太妃那边就闹了起来,说是天文官受奸人唆使,要害天子。

这番话传出来,又犯到了臣子们的忌讳。原本对钦天监弄出来的‘好事’还抱着反对心思的朝臣们,现在却都坚定了信念。

说是五月十六,那就五月十六。

儒门圣教面前,哪有那些歪门邪道站的地儿?朝堂政事,也容不得太后之外的妇人插话。

即便朱太妃是天子生母,议论的又是天子的婚事,她也没资格多嘴多舌。朝臣们有志一同,不能惯了她的脾气。

只是这么一来,要嫁给皇帝的王家女儿的立场就尴尬了。

对韩冈来说,的确有几分难做,皇帝要求娶的毕竟是他的内侄女。

韩冈本来想等着看王安石怎么说,但王旁和王旖先后相求,他也不能无动于衷。

只是现在朝臣们要给朱太妃难堪,尤其是在朱太妃的名声给韩冈、章惇等人踩了又踩之后,是个朝臣都想在她身上捞点名望。

考虑过前因后果,韩冈在宣德门外找到了章惇。没有首相的帮忙,他一个人想要实现对妻子和内兄的承诺,还是有些麻烦。

“太后是什么想法?”听过韩冈的请求,章惇问道。

韩冈道:“太后也要脸面,不想被人说她是非。”

为了天子婚期,朱太妃再一次上蹿下跳的闹腾,向太后尽管看不惯她的样儿,却也不想被世人说成是要害庶子的嫡母。

“既然如此,那就换个日子好了。”章惇没问韩冈言辞的真假,很干脆的说道。

得到章惇的承诺,到了殿上,再一次议论起天子的婚期,韩冈便出班表明自己的想法,

“所谓吉凶之日,本是附会而已。天地合乃是世间流俗,钦天辨历日观吉凶,也一样是流俗,不过是古传罢了。以臣之间,选什么日子都可以。夫妇和睦与否,在人不在天。所谓吉日、凶日,大可不必在意。”

“不过以臣看来,五月中旬,天已暑热。烈日下种种仪式,于天子御体有碍,不若选择春秋之时,气候宜人,不劳圣体。”

韩冈的话,差点引得满堂大乱。若不是韩冈一向跟太妃和皇帝不对付,他这番话出口,可就要千夫所指。

向太后倒是松了一口气,之前朝臣赶着要给朱太妃难堪,站在她的立场上,也是左右为难,幸好韩冈给了她一个台阶可下。

但她也知道,眼下的阵仗,光有韩冈还不够,便问向章惇,“章相公,韩相公之言,你意下如何?”

章、韩二相,大事总会相互协调,彼此拆台的情况几乎看不到,既然韩冈表态,章惇一般也不会有相悖的意见。

的确正如向太后所料,章惇出班回话,“韩冈言之有理,以臣之见,还是改期为是。既然五月中有暑热,不若就四月初八好了。至于神鬼之说,实不必理会!”

向太后全然没听到最后两句,只记住了章惇改动的日期,“四月初八,那不是佛诞日?!”

在佛祖诞辰举行婚礼,比起五月十六天地合似乎还要离谱,殿上人人吃惊。

佛祖从没说过他的生日不许世人成亲,也没那么多忌讳。只是寻常人都少不了念几句阿弥陀佛,到了佛祖生日时,去寺庙里焚香念经,求取开光的利物还来不及,哪得闲空去参加婚事?故而极少有人会选在这个日子。

将天子大婚的日期改了,朝臣们是退了一步。但朱太妃那边,却也不能让她得意。章惇改在了四月初八,顾全了臣子的脸面,也让朱太妃更没台阶可下。

既然你说五月十六不成,那改成四月初八,如果再闹,那可就是得寸进尺,做臣子的可就更有话能说了。

“韩相公?”

“臣无异议。四月初八,只是寻常日子,释迦摩尼既然没有阻人此日出生,自也不会阻人此日成婚。”

韩冈觉得既然没了什么克夫的忌讳,那也就没什么要避让的了。即便是时间,也不会嫌太仓促——皇帝婚礼上的一切准备,早就在筹办了,别说四月初举行,就是三月初,也一样不会有问题。

“陛下,皇帝本是现在佛,此日成礼本无忌讳。”

当年太祖皇帝去大相国寺上香,如来佛祖像面前曾问是否要叩拜,当时有个小沙弥机灵的回答——现在佛不拜过去佛。佛门从此视天子如佛祖。既然如此,自不用担心皇帝选在佛祖生辰成亲会触犯哪路神灵。

“就依相公吧。”向太后也没有别的意见了,若宫里面还有人不甘心,就让这两位宰相去应付吧。

韩冈和章惇的一番话后,天子的大婚日期便给改在了四月八日,不犯道家,而是去跟和尚过不去。

得了这样的结果,韩冈也觉得王旁和妻子那边也能说得过去了。

回到政事堂,心情比早上好了许多,只是当他看到了从江南送来的一份报告,脸就又挂了下来。

招来堂吏,他吩咐道:“去请宗汝霖来。”
