\section{第46章 了无旧客伴清谈(二)}

更戍法。

早在南征之役的时候,为了填补抽调西军南下后横山防线出现的兵力真空,韩冈就已经向赵顼提出了更戍法,将河北和京营的一部分兵力调往陕西去镇守各路寨堡。

尽管河北、京畿两地禁军的战斗力已经不比厢军强到哪里去,远远比不上西军和河东军,甚至不比陕西的乡兵组织弓箭社和忠义社,除了装备。

但由于南面的战局进展太快,安南行营只动用了第一批南下西军就解决了交趾,加上各地军头,尤其是河北、京营的禁军将领,明里暗里的抵制,鼓动士兵们跳出来闹事,这件事最后也就不了了之。

眼下为了应对即将到来的辽国威胁,又仅仅是河东和陕西换防,情况倒也不会变得跟前次一般。不管怎么说,河东军和陕西军都比较听话,也不畏战事,而不是像京营和河北军那般,烂到了骨头里。

话说回来,将河东路的下位禁军与陕西西军中的精锐换防,加强河东这个关键节点,这是保证对辽防务的关键,是当务之急,就是下面的将校士卒有异议,赵顼也会强压下去,

——只要赵顼同意自己的建议。

韩冈双手持笏,等着赵顼的决定。

赵顼并不反对韩冈的提议。更戍法对重整河北、京营禁军的战斗力和加强朝堂对禁军的控制力,都是有好处的,而且有了这个例子在前,开了头之后,今天用西军替换河东军,明天就能将京营和河北交换去陕西、河东。

只是要调动多少兵力才能达到预期的目的,就要好好想想了。

他望向同签书枢密院事,“郭卿?你看此事是否可行?”

郭逵低头想了一下,道:“要想加强河东兵力,做到韩群牧所说的西制党项,北当契丹,东援河北,那么至少要投入二十到三十个指挥的马军或是有马步人才堪使用,如果再将更戍法调出河东的兵力算进来,那更是要调入三万到四万精锐堪战的西军。”

‘二十个指挥!’

“三万到四万?!”

不论赵顼,还是政事堂中三位宰执,都对这数字吓了一跳。

三万到四万的禁军,相对于天下近六十万禁军总数来说,看着的确是不多。但如果将精锐堪战这个条件考虑进来的话,这就是陕西缘边的一个经略安抚使路所能掌握的全部作战兵力。再加上其中有二十到三十个指挥的马军或是有马步人——也就是靠马匹机动,作战时下马的军队,京城中有一名为龙骑的军额,——这个限制,几乎是抽空了一路兵马。

赵顼倒吸一口凉气:“三万、四万的精锐西军,其中还有一万以上的骑兵,都能抵得上半个河东路。”

‘是大半个!’韩冈心道,如果刨除折家的军力,这个数目的精锐西军,其实已经跟剩下的河东军的实力相差无几。

郭逵应声道:“契丹控弦百万,随时能抽调南下的骑兵至少十万。四万西军中,能随时抽调出来援助河北的也就是包括万余骑兵在内的二万人而已。”

赵顼皱眉想了片刻:“三十个指挥的马军或有马步人未免调动得太多了一点,将其中一半改成蕃骑如何?”

郭逵和韩冈同时摇头。

郭逵道:“蕃人可驱之作战,用以驻防名城,恐会生乱。”

韩冈也道:“蕃军善战,可惜难受约束,陕西汉蕃之争从未止歇,贸然移防河东,又是一致乱之源。”

“更戍只限禁军,要是将蕃军移防,朝野内外牵扯的事就太多了。”吕惠卿也反对赵顼的这个想法,“安史之乱殷鉴不远,蕃军可用不可信,不可使其常驻中国。”

京城中的禁军行列里,其实也有归明渤海、契丹直第一,契丹直第二,土浑直,这样的由渤海、契丹和土浑为名的军额,但兵力从来都不多,而且那还是开国之初利用俘虏建立的外籍军队。契丹直已经取消番号,而没有取消番号的,如今里面的人早就换了几茬,虽然还有当年俘获蕃人的子孙,可全都跟汉人没有任何区别了。

而且这些蕃军都是成立在开国之初,其时制度未立,那倒也罢了。现在没有人敢冒这个风险,而且也没那个必要,汉军的战斗力在加强了装备之后,能轻而易举的压倒蕃军。

被两名通晓军事的重臣,以及一名参政同声反对,赵顼也息了调用蕃军的心思:“三十个指挥就三十个指挥吧,陕西诸路还是能挤出来的。”

“更戍之制,不仅是地方领军之将在地方坐大,其实也有练兵的用意在。千里跋涉,都是征战时常见而日常训练不到的,此乃祖宗训兵之良法。陕西、河东换防,正好当做开战前的练兵。”

对韩冈这番话,赵顼听得进去,也能理解得了。

调动数万大军换防,本身就是对军队战斗力的一个巨大考验。也是对地方州县支援能力的考验。

陕西通河东,肯定走的是汾河谷地,从晋州穿阳凉关入太原,两边加起来七八万大军,包括上万马骡,地方上要筹备的粮草和资材都是个天文数字,一来一去,领军的将校和沿途州县的官员,谁合格谁不合格,有能无能,在这一次的换防中,就能大体上看得出来了。要是开战后才爆出问题来,对战局的影响可想而知。

薛向沉默了半天,这时候也开口道:“过年后就开春了,契丹即便想要用兵,也得等到秋后,用半年时间,在陕西、河东之间移屯更戍,三四万兵力不算是太大的问题。”

“那就这么办吧。”赵顼最终拍板,为了灭亡西夏,一点小问题还是可以克服的。

只是当他看看崇政殿中的几位臣子,大宋天子的脸又挂了下来。

陕西、河东差不多有一百个指挥要对调驻防地点,肯定少不了枢密院来主持。但枢密使和枢密副使都在家里待参,但是怎么签发公文都是一桩难题。

这么重大的决议,从枢密院出来的文件,若是只有郭逵一人的签押根本没有任何效力,就是上面还盖了天子的印玺也是一样,必须要有枢密使或枢密副使的签押。

但不管怎么说,先行在河北修筑轨道,沟通南北,与此同时在关西囤积粮草,并轮换河东陕西守军,静待时机的策略就这么确定了下来。

接下来一些人事安排就不是韩冈该插话的议题了,但赵顼还是留下了他和薛向,以备咨询。

这基本上就是韩冈今后一段时间内,在朝堂上所能起到的作用。

尽管一个同群牧使,是实职差遣中是难得的闲差。但日后只要是有关关西、广西的军事,轨道的修造,军器的发明生产,以及钢铁行业的发展,照样还是得来咨询他韩玉昆。

王韶出外了,章惇在家闭门待参,赵禼、熊本都远在边州,眼下的崇政殿中,韩冈已经是唯一一个拥有统帅大军出战经验的文官了,除了担任签书枢密院事的郭逵以外,在军事上以他的发言权份量最重。

虽然不管事,但他照样能在军国重事上参政议政,这是权威的份量!

韩冈也在用余光扫视着崇政殿中的几位同僚。两府加起来就四人在殿上,这个人数实在是可怜了。不过这个问题应该很快就能解决了。

为了签发两路移防的公文,至少得有一名枢密使或是副使出来签字画押。两名待参的枢密使和副使,吕公著事涉为人伦大案关说,短时间内别想出来,章惇不过是购买民田时出了点问题,被御史给咬上了。从程度上,章惇身上的问题是远远轻于吕公著。

既然天子需要人在公文上签字画押,那么章惇被解放出来也就是理所当然。

韩冈这么想着,只是当他的眼睛瞥到了薛向身上,原本很有自信的想法,却又变得不是那么有把握了。

隔了一天,应了韩冈当时在崇政殿上不妙的预感,薛向升任枢密副使的任命也下来了,吕公著和章惇依然在家中待着。紧接着,前一日在崇政殿中做出的决议,走过了一系列流程,从政事堂和枢密院中发布出来。

然后,群牧使韩缜兼任翰林学士的任命也跟着出来了。

韩缜是群牧使,韩冈是同群牧使,一为正任,一为副职。可韩缜只是龙图阁直学士,而韩冈是龙图阁学士,虽然两人的学士、直学士都仅仅是不厘实务的贴职,但韩缜从名义上说还是韩冈的下属,朝堂上合班站位,韩缜得站在韩冈身后。

这一点当然成问题。韩缜可是王安石一辈的老臣,出身自灵寿韩家,一门显贵。加上从差遣上算,韩缜也是高于韩冈。所以赶在韩冈走马上任之前,韩缜成了翰林学士。

韩缜靠了自己才被提升,韩冈倒是没有什么闲心去开玩笑。因为章俞购置田地时所犯一干违反律条的错误最终被定罪,使得章惇终于辞去了枢密副使的职位。

