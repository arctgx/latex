\section{第36章 万众袭远似火焚(九)}

韩冈前日将王韶的命令传达给刘源,便回到熙河的中心——狄道。

狄道城中,驻扎了一万秦凤军。王韶比预期中提前了半个月的时间,预定中的军队,有一半的还没赶到。不过等攻下珂诺堡后,稍稍修整两天,所有的军队就能到位了。

但韩冈并不知道,他担任同一职司的同僚什么时候能到。

同管勾军中转运事,沈括他这个人选其实是来得迟了。按理说,随军转运使在正月底、二月初的时候就该到熙河的,这样就能有一个月以上的时间来熟悉工作。看起来朝堂上为了争夺这个位置,浪费了不少时间。沈括能脱颖而出,一是他本身当有能力,另外就是他在新党中也有了些地位。

其实更为合适的人选当是在陕西路中选取,但对于军功的激烈争夺是在朝堂上。秦凤转运司还是永兴军转运司,两大漕司中的官员都没能在这场争夺战中占上什么便宜。

沈括的事暂且丢一边,韩冈估计他至少还有半个月的时间才能来报道。王厚人在陇西,由他出面接待也并无不可,而且蔡延庆必然同至,不必操太多心思。现在韩冈要注意的是眼下城中的情况。

正想着,门外亲兵通报,景思立景都监过来了。

“景都监。”韩冈起身相迎。

“用不着这么多礼。”景思立摆了摆手,急躁之色就凝在眉宇间,“在下已经在歇了三日,不知什么时候,王经略才会传回消息,让我军出动?”

“必须等到珂诺堡的消息传回来。”韩冈不急不躁,再三请了景思立坐下,“如果珂诺堡攻克,那就可以移师北向。将河谷道给清理出来,并防备禹臧家的突袭。”

他边说,边猜度着景思立的想法,这位秦凤都监大概是不想在后面等着残羹剩饭,他下面的兵将肯定也在催着他。半年前的临洮之战,率部来援的泾原姚兕,可是半点便宜都没沾上。

可是韩冈必须要让他执行熙州经略司制定的计划,“如果没有攻克珂诺堡,都监所部的行止就要视情况而定。最差的情况就是,珂诺堡始终未克,而禹臧花麻带着党项铁鹞子来攻打熙州,那时就要靠都监你来助守北关堡和狄道城了。”

见景思立嘴唇一动,韩冈又抢先一步说话,“不过都监不用担心,珂诺堡主堡位于河谷中,地势低凹。只要占据了山头高地上的几处营垒,位置不利的珂诺堡转眼可破。”

王韶和高遵裕将广锐将校拉出去不是没有缘故。单纯的要消耗人命,也不会放在这一场关键的战斗上。韩冈前面也听回来的游骑们说了,护翼珂诺堡的几处位于山头高地上的敌寨,大军难以展开,派遣精锐的小队才是攻下这些寨子最有效的手段。

只要能攻下珂诺堡外围的据点,景思立就可以领军北上,前往经过香子城、珂诺堡的支流汇入洮水的地方。

韩冈正安抚着景思立,一名匆匆走进。韩冈把蜡丸捏开,展开里面两寸宽、半尺长的纸条一看,笑意便爬上了嘴角。

他抬头对上景思立急切的视线:“这是王经略传回的消息,今晨官军已经攻下了珂诺堡。”将纸条递给景思立,“景都监,现在你可以北上了。”

……………………

暮色渐深,快到了收市上灯的时候。

被阳光薰了一日的春风还带着融融暖意,连着柳絮,一起吹进了秦州转运司衙门中。

转运使蔡延庆正主持着一场接风宴。一支支巨烛照得厅中犹如白日,从教坊司中请来的名妓坐在一角轻拨着琴弦。酒香、菜香,随着琴声乐曲浮动。

战争正在进行之中,蔡延庆无意将酒宴办得太过奢侈。并没有世间豪宴的初座、歇座、再座之分。菓子脯腊的随便上了八盘,作为正餐的一盏酒两道菜,也没有弄出个十六巡、二十巡出来,只是很简单的十二道菜,敬了几回酒,也就算个宴席了。

蔡延庆举着酒盏,对身边的中年官员,歉然道:“存中,今夜一宴算是简慢了。且明早尚要启程,延庆不敢多劝。”

中年官员大约四十上下,白面留须,看起来很是英俊。他拱手谢过蔡延庆,“今日运使盛情足见,沈括本也不胜酒力,待到功成,再谋一醉不迟。”

蔡延庆是不想惹得御史和走马承受的关心,想来沈括也不敢抱怨着简单的饭菜。再看看下面埋头吃喝的将校,这些赤佬有酒有肉就行了,何须多耗公使钱钞。若是给人说成是奉承,御下无状,可就没脸见人了。

蔡延庆款待的不仅仅是沈括,连同泾原路的将领也在一起——姚兕、姚麟两兄弟都来了。只是厅中的气氛很是有些压抑。沈括和姚氏兄弟都是为了河州决战而来,但到了秦州后,听到的消息却是王韶已经提前出兵。

王韶的这番作为,当然惹得众人不快。就连蔡延庆前几日听说熙河经略司先斩后奏的消息时,就算以他泰山崩于前而目不瞬的好修养,也差点当场摔了杯子。

现在蔡延庆的火气虽然消下去了,但他也担心沈括会在心中藏着芥蒂,最后坏了国事,“存中,今日传来的捷报,说苗授所部,已经攻下了河州门户的珂诺堡。珂诺堡自狄道远出百里,离着陇西,又是百里。如果再往河州去,还有一百二三十里。三百多里虽是路途遥遥,可官军仰籍天威,不会输于蕃人。唯有粮秣转运之事,乃是胜负之关键。”

蔡延庆的话中之意,沈括听得明白:“下官即奉天子之命而来,正欲粉身以报君恩,哪有不用心的道理。”

蔡延庆点了点头,正要说话,却听到一句歌喉悠扬:“谁念玉关人老……”

他脸色微微一变,顿时停了杯盏。

“太平也,且欢娱,不惜金樽频倒……”唱曲的营妓拖长了声调,将最后一句反复唱来。

沈括也是微微变色,听得最后几句,这首词已经很有些味道了,就是在这个场合唱着着实让人不痛快。

那营妓又开始从头唱起:“霜天清晓,望紫塞古垒,寒云衰草。汗马嘶风,边鸿翻飞,垄上铁衣寒早。剑歌骑曲悲壮,尽道君恩难报。塞垣乐,尽双鞬锦带,山西年少。”

听了曲调,辨出来词牌,蔡延庆杯子再拿不起来。虽然不合时宜,但听这上一阙就已是难得的杰作,惊问道,“这首喜迁莺是谁的做的?!”

蔡延庆问沈括,算是问道于盲。他摇摇头,“在下没听过,京中并无传唱……‘垄上铁衣寒早’、‘尽道君恩难报’,当是关西为官者所作。”

有别于上一阙的慷慨,唱到下阙时,歌声一下变得低婉起来:“谈笑,刁斗静,烽火一把,常送平安耗。”

沈括听了便道:“此番口吻,非是卑官者可有。后面还有句‘不惜金樽频倒。’”

蔡延庆轻轻点头,“也就十几人有这资格。”

歌声继续:“岁华向晚愁思,谁念玉关人老……”

啪,蔡延庆用力一拍桌案,苦思的问题终于想出了个结果:“多半是蔡子政【蔡挺】!!”

他的一句高喝,顿时惊散了歌声。招来唱曲的营妓,蔡延庆问着这首词的出处。

在蔡延庆面前,营妓没有惊慌失措,在宴席上常有人会问起所唱词曲的出处,“这是前日泾原路的蔡相公在宴上所做,刚刚流传出来的。”

一言中的,蔡延庆有些小得意,对沈括笑道:“蔡子政在泾原已经五六年了,也难怪他要说‘谁念玉关人老’。如今存中西来,河州兵锋正盛,正是大有为之时,当不能‘不惜金樽频倒’。”

他重又举杯,起身对着厅中一干文官武将:“夜已深,今日且尽此杯,来日功成,再与诸位痛饮。”

众人轰然应诺,连着沈括一起,都站了起来,将蔡挺的喜迁莺抛到脑后。

一夜酒宴过后,沈括和泾原诸将相聚在秦州城外,周围千军万马如山似海,从各处营地中行出,一波波向西开进。

等了片刻,蔡延庆领着转运司的主要人马,也在知州沈起的陪送下,出城而来。

蔡延庆也要往陇西去。这就是为什么韩冈可以随军去狄道,而不必留在陇西。熙河经略司的属地,也是秦凤转运使路的辖区。就算没有战事,以例蔡延庆每年也都要到巩州、熙州走上一趟。

如今王韶举兵攻夺河州,关系到数十万丁口,方圆几千里地的归属地决战,无论韩冈和沈括都不够资格主导军中后勤,也只有蔡延庆才够资格——也就是说,韩冈和沈括这两名随军转运使,同时接受王韶和蔡延庆的指挥。

沈括骑在马上,跟随他去熙河上任的只有七八个家仆,身边跟着一辆碧油小车,车帘低垂,不知坐着何人。

见到蔡延庆和沈起出来,沈括当先迎过去,几番寒暄,只见旌旗一摇,浩浩荡荡的便往西行去。

