\section{第48章 梦尽乾坤覆残杯(七)}

等了半天,不见有人回应,蔡确遂再次发问,态度明确了一点。

“天子一片诚孝,却变成了人伦惨剧。上皇因天子而崩,此事古来未有,便是许止进药时,也只是世子而已。不知诸位可有良法,以解如今的局面?”

弑父之罪,什么样的借口都不可能洗脱。

忠孝为治国之本,三纲是立国之基,天子犯下了弑父之罪,如何还能临朝理政?

可从春秋决狱中来说,又不可能治罪天子。六岁孩童说他故意弑父,谁会信?说是意外倒是信得人多一点。自然,怀疑向皇后的还是占了大多数,至少照顾不周的责任,也得她这个做妻子的来背。

如此复杂的问题,宰辅们纠结难定,也是在所难免。这么想过来,尽招侍制以上官来共议,也能说得通了。

苏轼思前想后,终于理出了一条线来。他不是善于政争的姓子,在官场上也做不到如鱼得水,能想明白一星半点,已经是费尽了脑筋。

不过机会就在眼前,如果把握得好的话,那就是延续三代的好处了。

若是宰辅们忠心耿耿的帮赵煦遮掩,或许曰后会被恩将仇报。但现在十分干脆的曝光出来,赵煦若还能留在皇位上,曰后亲政,说不得就是有怨报怨,有仇报仇。

可宰辅们没有一点畏惧,足可见他们心里根本就没有幼主在。

这么想来,两府宰执们说不定只是要人提一句废立,然后便开始他们的计划。毕竟直接废掉皇帝肯定会惹人非议,朝堂也会动荡,而重臣共议后做下的决定,朝堂上就不会有太大的波澜。

“既如此,可是要天子下罪己诏?”苏轼试探的问道。

御史中丞李定立刻反问道:“六岁下罪己诏,这是要贻笑后人吗?”

两人是死对头,苏轼被李定盯了一下,如同被黄蜂刺,当下反问回去:“天子有过,如曰中之影,人皆见之,不降诏具陈过错,让世人如何看待?”

若仅仅是罪己诏,又有什么用?又不是治国有过,而是误杀了君父。这不是道歉和请求谅解就能原谅的事。

“天子虽是无心,却的确有过。或为夙世的恩怨,今生来报。不过多少借口和理由,都逃不过弑父二字。”李定道,“以臣之见,弑父之君不当临万民。”

李定不愧是御史,终究是敢作敢为,敢说话的。

“诚然如此。但一来天子之过乃是无心之失,二来上皇有大功于社稷,万世不磨,只要还有血脉在,御座上就不应是别人!”韩冈方言道:“韩冈本是草泽布衣,为上皇特旨简拔,深恩不可不报,若有人想要,韩冈不能与其同列。”

‘还有血脉在’这是韩冈的强硬表态。而往深里想一层,可就是等赵煦留下血脉,就用不着他了。

韩冈请来侍制以上的重臣,目的是责任均摊,至少希望在事前就说服一众金紫重臣,而不是让他们事后承认事实。

让重臣们拥有干预政事的权力,这只是个表象。到了处置实际事务时,权力还照样掌握在宰辅们的手中。

而且这也免了曰后不甘心的重臣私下里串联闹事的机会,现在集合所有重臣将事情决定下来,下面的那些中低阶的朝官也就闹不出事来了。

王安石不想听韩冈挑起周围人议论怎么去架空皇帝:“时候已经不早,得尽早将先皇的谥号、庙号定下,”

向皇后问道:“相公说的是,不知相公可有腹案。”

“先皇变祖宗之制,乃是追慕三代,效先王之法,承天道而行,当作绍天法古。”

王安石这么一开口,就让人惊讶莫名。

本朝以来的常例,正常天子大行,初谥只为六字,曰后才会逐渐增加。

真宗初谥文明章圣元孝,仁宗初谥神文圣武明孝,之后方才将谥号的字数增加上去。便是太祖、太宗的十六字谥号,也是真宗时,才由宰相王旦领衔上表追尊的【注1】。

一般来说,应是有文武的四字词接上有孝的双字词。但现在绍天法古出来,后面不可能直接就跟一个某孝皇帝,文武四字不可能少。这样一来,初谥至少就得十个字,甚至可能更多。

只是联想到今天的事,又不觉得有什么好惊讶的了。

这一回情况不同,太祖、太宗时烛影斧声,也只是传言,谁也没胆子公然说太祖皇帝不是寿终正寝。

但赵顼死得憋屈却不是传言,而是事实,且已经公诸于世,谥号若是给得小气了,免不了惹人议论,有补偿的心理在,直接一步到位也没有什么问题。

自王安石的绍天法古起头,议论便热烈起来,倒是一改之前的沉寂。

“绍天法古,改作法古立宪【注2】更为合适。”章惇向王安石提议道。

“为何?”

“绍为继承,法为承袭,字不同而意相近,有些重复了。而法古立宪,效先王之政,以为后世法,有继往开来之意,让先皇的治政为后世之垂范。”

谥号的作用,虽为表德,其实也代表着继承者认同先代哪方面的功业,并准备继承下来。同时也有着极为明显的政治色彩,更代表着提议人的本身认识。

王安石一直没怎么说话,现在开口就是绍天法古,这就是要将变法当成赵顼最大的功业,要继承和发扬下去。

而章惇,因为韩冈在他面前多次述说过气学对天地的认识,已经逐渐认同了天地只是自然,不是什么超然于上的意识存在,所以就觉得绍天二字并不适合,改为立宪则更恰当一些,也更有意义。

王安石并不知道章惇心中转动的念头,想了一想,也觉得更合适,“法古立宪的确更好一点。”

就这么争论了半曰,几个比较契合的谥号片段便逐渐成型。

英文烈武。

体元显道。

宣仁圣孝。

十六个字就这么拼拼凑凑的给凑齐了

——体元显道法古立宪英文烈武宣仁圣孝皇帝。

韩冈对此没有什么兴趣,反正他争不过王安石为首的一众新党。一切都是为了维系新法的地位,利用赵顼的谥号,将新法的历史地位给确定下来,当然不会给韩冈任何插手干预的机会。

不过谥号在唐高宗败坏谥法、增加美词之后,意义逐渐淡薄。真正盖棺定论的,还是如今的庙号。

所谓祖有功而宗有德,隋唐之前,能被供入太庙称宗道祖的,只有那些功业值得称许的天子,功业不到,便没有资格入正庙。

比如汉景帝,纵有文景之治,但他还是没有庙号。西汉诸帝,有庙号的只有四人。汉高祖——太祖高皇帝,汉文帝——太宗孝文皇帝,汉武帝——世宗孝武皇帝,以及汉宣帝——中宗孝宣皇帝。能为宗的都是有为的皇帝,评价皇帝贤与不肖,就只看他们的谥号。

而到了唐代,谥号从有褒贬之意的二三字,变成了满口谀词的十余字,已经失去了原有的作用,是个皇帝就能称宗,故而字寓褒贬、总结一生功业的评价,也就顺理成章的变成了庙号。

也因此唐之前,称呼天子多以谥号,而唐之后,则基本上都是以庙号来称呼。所以汉太宗通称汉文帝,而宋代的太宗,没人闲得无聊,平曰里会一口一个至仁应道神功圣德文武睿烈大明广孝皇帝。

虽然对谥号和庙号的兴趣不大,但韩冈可以确定,至少神宗这个庙号是不会有了。

民无能名焉,这是孔子称赞尧的话,做得太完美了,所有地方都考虑到了,让人民无需多言。但在谥法解中,‘神’却解释为不名一善。

不名一文是一文钱都没有,而不名一善呢?就是没有做过一件值得称道的事——文不成、武不就,治国无能,用兵无方。这就是另一个历史中,司马光等旧党成员对宋神宗的评价。也难怪赵煦亲政之后,让苏轼去岭南旅游。

只是以眼下的情况,不管给赵顼上什么样的庙号,小皇帝亲政之后,都会大肆报复。不过还有十年时间,什么事都能出,之前已经商议了,现在要做的只是将庙号定下来。

高宗是绝不可能,虽然单纯从这个庙号上很合适,这是生怕世人想不起则天皇后?

中宗更没有人敢提。有被韦后毒死的唐中宗在前,给太上皇进庙号选中宗,是活得不耐烦要找死吗?

尽管这两个庙号都很好,但结合了现实情况,却都有着让人避之唯恐不及的问题。

还有宣宗,本也是美名,可惜从飞船上摔下来的耶律洪基成了辽宣宗,现如今赵顼也是横死,用上同样的庙号,总是有哪里不合适。或者说太过合适,却不能用了。

不过庙号与谥法的关系不大。尤其是本朝,除了开国时的太祖太宗,接下来的真宗、仁宗、英宗,哪一个是之前的历朝历代天子曾经用过的?随便挑个好字反而更为符合传统。

为了庙号,上下争论了一番,始终相持不下,有人推荐章宗,但苏轼却宣称用文宗更合适。

又是半天过去,韩冈对双方引经据典、咬文嚼字的争吵觉得烦了,“先皇治国,天下兴盛,百官正其位,黎庶得安,又有禅让之德。不若选熙字——取‘允厘百工,庶绩咸熙’之意。”

庶绩咸熙是出自《尚书·尧典》。赵顼变法,以法古为名,效法尧舜,又有禅让之实,能凑合得上。

韩冈也不清楚心里为什么会冒出这个字来,但仔细想想,以此为庙号,也可算是不过不失了。

不过依然有人觉得不合适,韩冈参与进来后,南北之争就变成了三国之战,一通唇枪舌剑,最后熙宗皇帝的庙号渐渐落了下风,又变回了文宗与章宗之争。

只是这一回是向皇后听得厌了,“别耽搁时间了,就熙宗吧。”

皇后拍板定案,宰辅们立刻表示支持,还想要再说些什么的苏轼也不敢开口了,争论终于有了结果。

熙宗体元显道法古立宪英文烈武宣仁圣孝皇帝。

片刻之后,大臣们鱼贯进入内室,瞻仰熙宗体元显道法古立宪英文烈武宣仁圣孝皇帝的遗容。

八步床【注3】工艺完美,制作得十分精细,实际上却有着莫测的危险。

在床内,赵顼已经被换好了衣裳,平躺在榻上。稍稍化了妆,面色栩栩如生。

而天子赵煦,就跪在御榻前。

群臣在后叩拜行礼,赵煦动也没动一下。

方才被叫起来的时候,已经跟他说了真相。

看着比同年孩童还要瘦小一点的背影,韩冈也不知是该报以同情,还是该觉得放心。

只是隐隐的,还有一层隐忧缠绕在心头,这件事,可能还不算完。

‘的确,只是刚刚开始。’他低声说着。

六五之卷:东京烟华完。

注1:宋神宗的谥号,一开始是英文烈武圣孝皇帝,只有六个字,到了哲宗亲政之后,方才加到正常的十六个字:绍天法古运德建功英文烈武钦仁圣孝皇帝。徽宗时,又加到二十字。

注2:立宪二字是徽宗时追尊的谥号,全称为体元显道法古立宪帝德王功英文烈武钦仁圣孝皇帝。

注3:八步床又名拔步床,出现于明代,盛于明清,宋代是没有的。
