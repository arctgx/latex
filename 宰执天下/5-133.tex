\section{第14章 霜蹄追风尝随骠(17)}

【今日第二更。】

缭绕在鼻端的血腥气,浓得化不开去。

数百人组成的队伍,在道路上驻足良久,仍没有重新起步的打算。

穿过了荒漠,越过了丘陵,在翻过了最后一道山峦之后,那龙部残存下来的六百多人不约而同收紧了缰绳。

堆叠在路旁的是一具具被扒光了衣袍的无头尸骸。白森森的肌肤暴露在外,肩膀上是空空一片。从尸身上流出来的血将雪地染成了浓浓的红褐色,几近深黑。

那龙阿日丁自幼在杀戮中长大,但他从来没有看过眼前的这一幕。比起一路上的累累伏尸,宋人边界寨墙之前的一段道路边,如柴禾一般整齐排列的尸骸,更是让人心惊胆寒。

或是为了争夺草场,或是为了争夺水源,或是为了争夺人口,或是为了为祖上延续下来的仇怨,或是有人挑拨,各部族之间经常会因为各种各样的理由,拿出弓刀来厮杀一场。有时候一句擦肩而过的口角,就能挑起两族间的战斗。

黑山下的部族之间,从来都不是一团和气,战斗一年年的永不停歇。但几乎不曾有过赶尽杀绝,以灭族为目的的战斗。杀了族长一家,吞并部族,基本上就是底线了。

尸骸夹道,那龙部的不知是该继续驭马上前去投奔宋人,还是就此掉头回返黑山下与契丹人拼个生死,或许后一种才是正确的选择。

但已经有人迎了上来,隔着拦在路前的一道栅栏,一人操着党项话远远地高声喊着:“你们是哪一部的?”

想退已经来不及了,那龙阿日丁上前回道,“我们是那龙部、”

紧跟着另一人喊了起来:“是不是阿日丁?”

接着两人就越过栅栏,走了过来。一个当是宋军中的军官,一名小校;另一个则是党项人的打扮。

来人辨认出了那龙阿日丁,而阿日丁也认出了陪同小校过来的老熟人,是另一个部族的族长,“是阿息保?!你还活着。”

阿息保是个三十多岁的壮汉,声音洪亮,“阿日丁,这话该我问你。还以为你们那龙部要跟契丹人死拼到底。”

两人算是故旧,旧日也有几分交情。异域相逢,阿日丁心情大好。

但小校咳了一声,“叙旧可以先等一等,有些话要先说的。”

阿日丁点点头,转过来听小校说话。只是一转眼,却见老友是一脸的惶恐。阿日丁疑惑起来,虽是寄人篱下,但也不至于听了一句话,就这般惊慌失措。

正想着这个问题,就听小校道:“尔等被辽人赶出家园,我大宋天子也是感同身受。河东路韩经略得了天子准许,将尔等收留。只是契丹人狡诈,遣人暗藏尔等之中,还有一些不成器的家伙,被辽人赶得家破人亡,却还听辽人使唤,想来骚扰胜州。所以路边上的尸首你们也看到了,全都是那些贼人的,希望你们不是。”

阿日丁点头哈腰:“小人是真心诚意来投奔大宋。”

“是不是真心诚意,得看你怎么做了。投奔来的部族各个表现了诚意,来得早的,帮着运粮。来得晚的,就要帮着去修城。”

“修城?”那龙阿日丁发了怔,“这个……我们那龙部连房都没有修过,只支过帐篷,没做过工。”

“不愿意也成,韩经略也说了,此事纯凭自愿,不会勉强任何人。天下间可没有吃白食的道理,想必你们也能明白。”

小校完全没有用上威胁的口气,只是和和气气的在说话,但那龙阿日丁心头一股子寒气咕嘟咕嘟的不停的冒出来,将整条脊椎骨浸在寒水中。在满地的无头尸骸边说出这段话,分明就是六个字:不听话便去死。阿息保在后面一个劲的使眼色,分明是要让他快点答应下来。

那龙阿日丁甚至没敢再犹豫,“小人明白了,愿去修城,愿去修城。”

应声答诺,几乎让人窒息的紧张气氛才松弛下来,可那龙阿日丁的脑中,依然一团混乱。

并不是觉得奇怪,而是感到恐惧。宋人在传说中对逃人很是优厚,曾有参与过南侵的族中耆老提起过,投奔宋人之后,手上只要有百来骑兵,就能混个领俸禄的官来做。可是那龙阿日丁今天却全然没有看这一点。宋人的盘算是全然的未知,阿日丁不知道他们接下来会怎么做。

但在眼下,也只能走一步看一步了。

……………………

“又是五六百人,修胜州的人数差不多也快够了。”

使用党项人来的修城立寨,就不要动员麟州、府州的民夫。这对于安定河东西北的局势,不用多说,就该知道有多少好处。

艰苦的工役,是消耗人命的磨坊。所以在动用民夫时有许多顾忌,但换成是黑山党项就方便了许多。

折可大站在高处俯视着被押送往胜州去的这一支兵马:“不过也算他们运气好,要不是识情识趣,他们的性命一个都别想留。”

“嗯,不听话的人,也不适合去做工。”黄裳应声道。

折可大转身对奉命来体问的黄裳笑道,“两侧山头劲弩攒射而下,加上抄截后路的两部兵马,将这个什么那龙部一口吞掉换成斩首功,根本不在话下。饶了他们,可也是少了六百斩首。子河汊大营那边不在意,但柳发川这里,六百斩首可不是小数目了。”

“折将军说哪里的话,如今怎么会缺斩首呢?”黄裳瞥了一眼折可大,嘴角的笑意突然诡异起来,“折府州当与李太尉商量好了吧?两家是对半分,还是四六分账?……又或是三七?不过那样李太尉可就太过分了。”

折可大闻言,脸上笑容顿时一僵。在旁作陪的折克仁神色也变得尴尬起来,想否认,却在黄裳诡谲的笑容中一个字也说不出来。过了半天,才言辞干涩的问道:“龙图知道了?!”

黄裳转回视线,又追着渐渐远去的那龙部人马:“南下的黑山部族差不多有三五万之数。一路上的艰辛,加上各部之间争夺食物、财物,倒毙于途有三四成之多。这些事可都是一五一十报到龙图那里的。现在又是冬天,尸首短时间之内不会腐烂。那些首级割下来,谁知道是死在什么人的手中?难道不是这样?”

折可大、折克仁脸色难看已极,韩冈当真是知道了。本以为还能多瞒两日,谁能想到尚未开始就给点破了。

“那可是成千上万。”黄裳顿了一下,唇角又翘起,“少说也要超过一万。一万斩首啊……”

拖长了音调说罢,黄裳呵呵两声笑,把折可大的心肝都带颤了。

自来冒功没有这般肆无忌惮的,真要给揪住,罪名可就大了。

现如今,是李宪和折家两边分账,河东军和麟府军的将领们等于是互相拿了把柄,不怕对方戳穿。等到事情做出来,再报与韩冈。到时候,韩冈也只有选择支持了。但在动手前,就给韩冈拿住,这件事还怎么继续下去?谁敢认为总是很和气的韩冈,不会下手杀人。

折克仁长叹了一声,“此乃家兄和李经制一时糊涂,尚幸犹可挽回……”

“没那个必要。龙图说了,分账的时候好生商量妥当,不要闹出事来。闹到朝廷那里,前面的功劳可就没人认了。”黄裳又是微微一笑,他的笑容越发的让折家叔侄二人看得心惊胆战,“多少贼人都是因为分赃不均,闹将起来后,被官府拿住的,可不能犯了跟他们一样的错。”

黄裳说得刻薄,折可大更行尴尬,而折克仁叹了一声,“龙图洞烛幽微……”

“你们怎么会奢望此事能瞒得过龙图?就是远在东京的朝廷那边也难以瞒过。龙图前两天就说了,‘斩首即便多个一万两万,功赏也不会按照比例增加多少。让他们开开心也好,当真以为朝廷好欺瞒不成?也不看看伤亡才多少,能瞒得住谁?当真以为销了空额就能糊弄过去?’”

折家叔侄二人被黄裳的转述批得哑口无言。韩冈的确是行内人,说的话一针见血。

就听黄裳接着道:“龙图还说,西军善战,不过谎报、冒功的本事也不差。就如荒漠里的那些黑山党项,放在陕西也一样不会放过。所以是见怪不怪了。龙图还说了,你们去割首级的时候小心一点,契丹人的马军说不定就在哪里游荡着。”

折克仁态度端正,欠身道:“多谢机宜提点,克仁必转述给家兄。”

黄裳摇摇头:“折府州那边,是遵正兄受命去的。我这边只是奉命体察军情,说这些话是多余了。”

“啊,是吗?原来如此。”折克仁还以为折可适没来,是为了避嫌,原来是去了暖泉峰,找折克行去了。

“那李经制那边呢?”折可大问道。

“李经制现如今就在胜州城中,龙图直接跟他说。”

……………………

送了李宪回来,韩冈微微一笑。

方才与李宪交谈,感觉他也是不愿意冒这份功劳。但下面的将领要安抚,必须要跟折家协商分赃。否则他一个阉人,凭什么能顺顺当当的带兵打仗?

河东这边冒功谎报的本事,终究还是不比陕西差多少。靠着那些自相残山的黑山党项,一口气就增加了一万多斩首,连同各部亲手屠戮的数字,加起来超过了两万。这个数字即便在对折后再对折,都是很吓人了。传到京城,也肯定会引起轩然大波。

说起来,肯定会有人指责韩冈草菅人命,屠戮辽国逃人。不过也没什么大不了的,这点指责动摇不了韩冈的地位。

下手重有下手重的好处,任何骚乱,都是人数越多乱得越厉害。不听话的都杀了,听话的不是在运粮,就是在筑墙。在大军的镇压下,就算有心做反,也掀不起波澜。

而且这两天还确认了被屠杀的部族中,有两支是契丹人的伪装。不过灭口灭得太顺利,一个活口都没有,也不知道萧十三究竟派出了多少。

韩冈估计不会太多,千八百的伤亡,以萧十三的身份压得下去,再多了,可就是罪名了。而这两支契丹人的军队,人数在七百上下,只比韩冈的估计少一点而已。

差不多该结束了。韩冈想着。如果萧十三除了在东胜州堆兵马以外,没有别的后手的话。

