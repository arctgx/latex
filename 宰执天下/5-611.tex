\section{第48章 梦尽乾坤覆残杯(五)}

夜风一个劲的刮着。

穿过殿阁楼宇之间宽窄不一的间隙,风声就变得高高低低,听在耳中,犹如鬼啸。

寒冬腊月的夜晚,寒风如刀。

宋用臣从温暖如春的寝殿中出来,慌慌张张的没有加衣服,一阵风过来,顿时就遍体生寒。

贵为御药院都知,没人敢挤在他身边。一丈之内,都没其他人站着,没遮没挡的,给冻得直哆嗦。

瞅瞅稍远一点,挤作一团的低阶内侍和宫女,不由得羡慕起来了。挤在一起不仅能取暖,还能壮壮胆。不会像他,身子冷,心更冷。

不过人群之外,围了一圈班直,冷也好,热也好,最后的结果不会有什么区别。

王中正受了韩冈的命令,带着人看守着从福宁宫中出来的同伴。

那群身量高大的班直,一个个手拄刀枪,腰跨长弓,将里面被围着的一群宫人,都当是反贼一般的盯着,想逃都没处逃。

“正卿,冷不冷。”

刘惟简不知什么时候挤了过来,叫着宋用臣的表字,亲热的就像是老朋友。但两人的关系,可从来都没有和睦过。

不过宋用臣这一回懒得与刘惟简争闲气了,长叹了一声,声音压低:“要是冻上一阵就能保平安,再冷一点也没什么。”

“……谁说不是。”

刘惟简抱着膀子,哆哆嗦嗦的说着。被韩冈从福宁殿里赶了出来,等着里面的裁决,现在心都冷了大半截,不知自家的姓命能不能保得住。

他和宋用臣现在都是脸青唇白,三分是寒风,七分是害怕。

太上皇突然驾崩,还带了三条人命走,从韩冈方才的询问中来看,怎么想都是做儿子的官家想要尽孝弄出来的祸。

刘惟简明白,如果宰辅们要帮天子遮掩,杀人灭口是最简单的做法。

就算曰后外面满城流言,也可以装没听到。谣言这东西,除非被有心人利用,否则没有一点意义。

烛影斧声,金匮之盟,太宗皇帝得登大宝的谜团,在世间传了上百年,但现在坐在御榻上的还不一样是太宗的后人?

可是这上皇暴毙当夜值守在殿中的宫人,则很难有好下场,否则人多嘴杂,曰后总会有各种各样的问题。

这不是多难想象的一件事,放眼望过去,小黄门一个个惶惶不安,宫女们也都在低声抽泣着。

只有王中正昂首阔步,呵斥着几个找地方避风的班直们。

一众从福宁殿里出来的宫人,只有王中正是不用怕,韩冈让他领兵做看守,摆明了要保他,如果朝廷要动刀子,也肯定砍不到他这个正任观察使头上,最多也只是让他找地方去养老。

两对眼睛遥遥望着王中正,羡慕和痛恨的情绪在视线中交织在一起,只是这些情绪很快又都收了起来,掩藏得妥妥当当。

石得一过来了。

不过被王中正拦在了外面。

刘惟简眯起眼睛看过去,不知两人在说些什么。只看见石得一先是发怒,然后就是一脸吃惊的样子,转身就要走。

“有人出来了。”宋用臣忽然紧张得说着。

刘惟简忙向殿门望过去,韩绛和郭逵从里面走了出来。

除了值守的班直之外,殿前所有人都躬身向宰相行礼,石得一闪躲不及,也只能一起向韩绛拜揖。

站在台陛顶端,韩绛自上而下的俯视着,高大的身躯,在火光中,有着难以言喻的威压感。

本来就已经是很安静的人群,更是静得呼吸可闻。

“都进去吧。”

韩绛开口,声音不大,可每个人都听得一清二楚。

宋用臣浑身一颤,看起来太上皇后和宰辅们已经做出了决断。可是他从韩绛、郭逵的脸上完全不看不出任何端倪,一时间心乱如麻。

只听见韩绛接着又补充道,“方才在福宁殿中的,现在都进去,太上皇后有话吩咐。张守约、王中正……哦,石得一你也在,你一起也进去!”

石得一不敢分辩,从贵为观察使的大貂珰,到没品级的小内宦,还有女官、宫女,在首相面前,都没人敢多问一句,听话顺从的拾阶而上,鱼贯进了殿中。

韩绛偏过头,对郭逵道,“下面的班直,拜托仲通先约束一下。”

“郭逵明白。”

郭逵拱手一礼,下了台阶,很自然的顶了王中正和张守约的班。当朝第一名将,只是用眼睛扫了一圈,桀骜不驯的班直全都屏息恭立,没有一个敢抬头对视。

宋用臣领头在前,身边是刘惟简,一步步的走上台阶。

刘惟简一对眼睛直勾勾的看着前面,嘴里念叨着什么。宋用臣细细一听,却不是在念佛,而是在含含糊糊的在祈祷着:“……不要那么忠心就好,不要那么忠心就好……”

虽然没头没脑,但宋用臣一听就明白了,在心中暗暗念叨:“不是亲生的,不是亲生的……”

他和刘惟简一样,都盼着太上皇后和宰辅们会有私心,而不是对小皇帝忠心一片。

毕竟就算宰辅们帮着小皇帝遮掩下去,可谁也不知道小皇帝会不会曰后觉得知道内情的宰辅们不是那么的保险,然后再来一次杀人灭口,将知情人彻底清除干净。

韩绛早一步回到殿中。

一群宫人极难得的被两班宰辅们包围在外殿正中央。

纵然人数比两边的相公、参政、枢密们多上几倍,但被他们一围,跪在地上都发起抖来。

蔡确出班,站在人群侧前,

“今已查证,上皇大行乃是意外。尔等虽无不赦之罪,但疏忽失察之过,却不能轻饶。”

意外!

宋用臣听到这里,绷紧的腰背就一软,差点就瘫倒在地上。身边的刘惟简也是一阵摇晃,悬在脑门上的大锤终于没落下来,让两人彻底没了力气。

有意外,就必须要有人负责。既然说殿中众人无不赦之罪,那么要负责的就不在他们中间。

一同死掉的三人也不可能,在殿上的太上皇后和宰辅跟不可能,那么剩下的,还会是谁?

虽没有直接点出来,但已经足够说明一起了。

证实了猜测,宋用臣还是有些不敢全然放心。用眼角瞟着王安石。

在这殿上,对小皇帝最为忠心的,只会是王安石和韩冈,而且真要是小皇帝做出来的事,他们两位都不脱不开干系。

不过韩冈肯定要差上一筹。方才韩冈审问宫人的时候,不可能没发现小皇帝对他心中有芥蒂。

可王安石现在只是沉着脸,完全没有反对的意思。

他对官家的忠心,看来也是看在上皇的份上。

而且能够对宫人公开宣言,可见向皇后已经被说服了。

终究不是亲生的。

宋用臣心想,这一下心头大石终于能放下了。

现在都在这里如此坦然的说明真相,之后也肯定会向朝臣公开。否则岂不是朝臣地位都不如他们这些天子家奴了?

但宋用臣随即发现自己猜错了,公开真相的时间,不是之后,而是现在。

蔡确报了长长一串名字,全都是当朝的金紫重臣,要即刻通知他们入宫。

宋用臣脸贴着地,一个一个记下来。

若太上皇正常因病驾崩,现在宫里面就会是忙忙碌碌。

收殓上皇遗蜕,更换陈设,布置梓宫。由首相韩绛出任山陵使,主持一应仪式,蔡确辅佐。并遣使告哀辽国。大赦天下。还要派人去通知在京寺观,为上皇敲钟祈福。

这些事朝廷早就有准备了,只要一声令下,宫里宫外立刻就能行动起来。但现在却是召唤在京重臣共议。

连夜招重臣入宫,这么做,难道是顺势要废了官家,重立天子?

想到这里,宋用臣打了个寒战,不敢再多想,与刘惟简、石得一一起领了旨,连忙出去,分头安排人去通知所有在京的重臣们连夜入宫。

王中正随后也领了旨意,去请天子赵煦。

安排下其余人等,皇后入内守着她丈夫尸身,宰辅们也都在外间坐了下来。

虽然入宫的时间并不长,才过去了一个时辰还不到,但自王安石以下,两府宰执一个个都是身心皆疲,在座位上呆然坐着。

“玉昆。”章惇偏过头,低声问韩冈,“这样真的好吗?”

“不然还能怎么办?”韩冈反问。

章惇怔了半曰,说不出话来。

行废立之事,要下决心不是那么容易,而且赵煦被废之后,若是活着,谁也不敢保证曰后不会卷土重来。若是死了,他们这群宰辅在青史上也别想留下什么好名声。

为伊尹之事,放太甲于桐宫。但赵煦不是太甲那样耽于嬉乐的皇帝,而是好心办了错事。才六岁的小孩子,仁孝聪慧,又有多少地方需要反省和悔改的?

最后只能叹气:“这等事,东京城一个冬天都出不了几起,偏偏落到了福宁殿里。”

就算完全是意外,赵煦都要背上一辈子的罪。好端端的聪慧天子,长达之后,还不知会变成什么样。

韩冈没指望过赵煦这个学生能为气学张目,也从未期待能教出一个言听计从的皇帝。

福宁殿中的变故的确让人感慨,但感慨之后,他剩下的念头,就是该如何利用这个机会了。

韩冈望着门外的黑暗,就等着他们来了。
