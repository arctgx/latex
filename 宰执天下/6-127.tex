\section{第12章 锋芒早现意已彰(八)}

元佑元年的冬天比前两年更冷了三分。

不仅仅是惯常下雪结冻的北方各路,就连南方诸路也是暴雪连连。

九月的时候,江东、两浙大雪。

十月的时候,江东、两浙继续下雪。

十一月的时候,福建和广东也下起了雪来。

到了新年越来越近的腊月廿一,苏州那边报称太湖冻结,洞庭西山周围都冻起来了,冰层还挺厚,车马行之无碍,据说能够一路走到南面的湖州。

西山柑橘,闻名国中,洞庭山上尽是柑橘园,据称有十万株之多。而柑橘畏寒,一个冬天都不能离人,为了防止柑橘树冻死,橘农都得以粪壤护树,还要在上风处烧火,以保持温度。

前些年洞庭山周边湖水冻结,冰层不能行走,却能毁损船只。运送粮食的船只上不去,而人又撤不下来,使得在山上种植柑橘的橘农被饿死了好些人。今年就不必那么麻烦了,粮食可以直接用大车运至湖中的洞庭西山上,可是,满山的柑橘树在如此深寒中,多半难以保住了。

当然,苏州知州并非是要说什么奇闻轶事,而是想要朝廷同意开仓。光是苏州一地,入冬以来冻死的百姓已是数以百计,受灾的更是百倍于此。而整个南方地区,包括广东、广西的一部分州县在内,灾情都十分严重,百姓的伤亡不在少数。

南方的房屋与北方不同,墙壁厚度不够,保暖性很差,房顶也不像北方的屋子,能够承受更厚的积雪。所以同样的寒潮,对南方造成的影响,也就远远超过北方。

之前为了方便各州烧砖,朝廷从徐州等地的矿场调了不少探矿者,满地的找煤矿,希望可以就近补充燃料,在江淮一带,发现了十余处煤矿。但自入冬以来,大量用来烧制城砖的石炭,都被挪作他用。各地州县都在依照朝廷的诏令,向民间平价发售煤炭等取暖用品,同时还组织因暴雪破坏房舍、以至于无家可归的灾民,掘地修屋,以半地下的窝棚,来抵御寒冬。

南方此番灾情,乃是近年来又一场遍及诸路的大灾,唯一值得庆幸的,是冰灾雪灾不会影响到今年的收成,只要春天能够按时到来,也不用担心明年的夏收,甚至还能因为冻死地里的害虫,补充田间的水分,对明年的种植有很大的帮助。

这么思考问题,可以说十分的冷血。可是到了两府宰执的这个级别,视民如伤的想法早就淡去,真正关心的仅仅是如何避免灾害范围扩大,还有如何解决灾害带来的诸多衍生问题。

可尽管今年的寒灾暂时没有影响到国家的粮食安全,可放在出兵的问题上,就让一众宰辅不得不投一个反对票。

“天寒地冻的怎么出兵?”

当杨英造访王厚的时候,也为此愤愤而言。

杨英昔年曾与王舜臣冇和赵隆同为王韶亲兵,之后又成为王韶麾下的一员将领。只是才干不如王舜臣、赵隆和李信,运数也不如,但始终与王韶父子亲厚,与韩冈又有交情,如今也积功升到了大使臣的行列中,又在韩冈、王厚帮助下,给他安排了一个好职位。

坐在王厚家后园的小亭中,周围放着三个暖炉,让杨英感觉不到寒冷,可外面人人穿着厚厚的夹袄,在冷风地里多走几步,就能动得手脚冰凉,他实在不明白,为什么宰辅中还有人想要出兵辽国。

上面的不知道,在关西从军多年,杨英可没少在冬天生冻疮。比起安居京中,抱着暖炉坐而论道的士大夫们,他更清楚严寒会给出兵造成什么样的灾难。

王厚轻轻晃动着酒杯:“也不是说现在就出兵。稍稍准备一下,就到了春天了。”

“正是准备起来难!”杨英将酒杯重重的顿在石桌上:“粮秣、军资还有各部兵马,都要在开战前运到出兵的位置上,难道这些事可以拖到开春?!”

“王平章认为没问题。”

杨英左右看看,凑近了低声问王厚:“那韩参政怎么说的?当真是反对?”

“他当然是反对。宰辅中不就王平章一人支持出兵?”王厚看见杨英欲言又止,眉头一皱,“怎么,你听到了什么?”

杨英更加小声:“驿馆和衙门里面都在说,其实宰辅们支持攻打辽国,只是担心北虏有了防备,所以才一片声的反对出兵。但每个都反对出兵也不正常,像章枢密、韩参政这样子的性格,怎么会反对出兵?所以就让王平章出头,免得北虏怀疑——王平章地位高,却不怎么管事,他出来是最合适了。”

“胡说八道。”王厚摇头。

如今天寒地冻,气候远比千年后要寒冷,只看每年冬日,汴洛段黄河都会冰封河面,便知与后世有多大差别。做战前的准备都是千难万难,何况是向北出兵。

太宗皇帝两次北进,全都是以失败而告终,而最近的一次北进作战,也同样是惨败而归,要不是李信有一个足够高足够厚的好靠山,他也根本不可能这么快就翻身。现在说北进,就是最为知兵的韩冈与章敦,心中也是打着鼓的。

反对的是真反对,而支持的,有几分是真心,那真的是说不准。

“当真不是?”杨英却带着几分怀疑。

“那还有假,不要信那些谣言。”王厚郑重的说着。

“明白!明白!”杨英猛点头。

王厚暗暗叹了一口气,他看杨英的神情,怕还是以为自己是口是心非,不得不隐瞒。

但实际上,当真是两府众人都表示反对,只有王安石一人要求出兵。

这件事其实确实无疑,外界相关的谣言大多数都能传经他这个皇城司提举的耳中,而每一条,又都是全然无稽的造谣。

即使是韩冈,也是十分坚决地反对出兵——这是王厚亲自向韩冈当面问过的。

韩冈主张在河北整备兵马——河北禁军也需要进一步的整顿,以配合火炮装备部队的进程;

韩冈主张策反阻卜人——这是关西方向一直在做的;

韩冈主张河东方向做出攻击辽国西京道的态势——从关中运送粮草入河东,并加快太原府入关中的铁路建设就可以了;

韩冈还表示,只要耶律乙辛篡位,大宋将立刻断掉给予辽国的岁币——失去了上百万贯的资金,也等于是失去了至少数万名本可以收买过来的敌人,这是对耶律乙辛最狠厉的一刀;

耶律乙辛在辽国国中的根基已深,如果大宋什么都不做,辽国的内乱不会爆发出来。

无论战与不战,大宋都要做些什么,甚至做好作战的准备——如果辽国当真分裂,两边交相攻击,这样的机会,还是不能放弃的。

但韩冈绝不会同意在这个时间点主动出兵辽国。

与辽国决战的准备尚不充分,以大宋不断增强的国力,也没必要去抓这个时机。

战争现在不会开始,但北国的烽烟迟早会燃冇烧。

十年后,耶律乙辛垂垂已老,几个儿子怕还要争夺储位,而大宋兵强马壮,正是攻辽的好时机。

韩冈前日在王厚面前说了很多,话里话外都是不赞成主动出击。

但民间的情绪已经为辽国即将内乱的消息所引动,到处都是在说如冇何收复故地,平灭辽贼,就连樊楼上唱曲,还要唱几段封狼居胥的歪词,每天琵琶弦都不知要多断上几根。

王厚这两日看到堆在桌上的机密文函都要头疼,要是他将搜集来的流言蜚语全都递上去,说不定就会影响到太后的决策——樊楼中的诗词唱段,颇有几句能煽动人心的。

王厚很多时候都不得不自嘲,别看自己位高权重,但实际上比在边地的时候,还要小心做人。一个不小心,说不定就会万劫不复,就连韩冈也难救自己。

现在王厚一肚子的机密,与杨英这等旧部喝酒聊天,也不能尽兴。与韩冈的来往也不得不有所避忌,只能偶尔碰面。

送走了杨英,王厚也唤人备马,出了门去,从巷道转了几转,抵达了参政冇府的侧门。

韩冈今晚在家中,小厅里摆好了茶盏和茶点,正等着王厚登门。

“玉昆,久等了。”

王厚匆匆坐下,赔了个不是。

韩冈摇摇头,他是难得等人了,反倒有些新鲜感。他笑了笑,问王厚,“杨三可是走了?”

“走了。估计明天还回来拜访玉昆你。”

“感觉怎么样?”

王厚摇头,正色道:“终究还是比不上王景圣、赵子渐和令表兄。”

当今军中,能比得上王舜臣、赵隆和李信的将领,也没有几个了。

“只看他当年在襄敏公的鞍前马后的份上吧。”韩冈叹道。

“江南西路的都巡检足够了,想来他也不敢奢望太多。”王厚问道:“王平章怎么说?”

“你可知道,大名的吕吉甫今日上本,如果耶律乙辛胆敢沐猴而冠,皇宋为辽国兄弟之邦,当为其拨乱反正?!”
