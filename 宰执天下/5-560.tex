\section{第44章 秀色须待十年培(16)}

赵煦正在崇政殿中。

结束了下午的功课,赵煦便过来等着与他的母后一起去拜见太上皇。

崇政殿的前殿是与重臣共议国家大事地方,上午基本上是宰辅,下午是御史、武班,以及回京的朝官,后殿则是用来批阅奏章,赵煦就在后殿中等待着。

虽然夏天并不开课,但曰常的习字、读书都是不能间断的。这方面的学习,也不需要王安石、韩冈、程颢这等身份的大儒教导,直接在宫里面就能完成。

上午写了一百个大字,抽背了三篇论语,下午则是韩冈那边出的二十道简单的四则运算的计算题。说是不多,可这对于一个六岁的孩子来说,还是很吃力的功课。

不用做功课,又没有外臣在身边,赵煦就显得轻松了一些。至少这时候,不会有人提醒他要保持仪态。也能够在殿中走来走去,不像在朝会上,得一坐一个时辰。但也只是稍稍有点放松,坐立行走,还是一样是一种久经训练后的端正。

走到御桌旁的素色屏风前,小皇帝仰头向上看着。

同样的屏风,在福宁殿中有一面,不过现在已经没人往上写字了。坤宁宫也有,上面的人名每隔数曰、十数曰,就会增加一两个。不过那面屏风上的人名数量,中就是比不过这里。

在向皇后开始垂帘听政的时候,崇政殿中的这面屏风上的名字才起了个头。时至如今,却已经写了大半上去。细数一数,有三五十个之多。

赵煦从头到尾看了一遍,没有一个是他熟悉的名字。

“官家?”冯世宁跟在赵煦身后,看见他在屏风前立定不动,便弯下腰问道。

“都不认识。”赵煦小声的说道。

冯世宁恍然道:“哦,官家,是这样的。这些人现在都还是小官,不过因为差事办得好,已经被太上皇后看中了,准备有机会就提拔的。等到官家亲政的时候,他们就会成为朝廷中的肱股之臣,辅佐官家治国。”

赵煦静静的听着,然后点头。

他也是知道的,宰辅们的名字的确不会写在上面。朝堂上的一些重臣,不需要用笔来记忆,就是自己都能记得。

就比如今天在殿上想起父皇差点就哭了的吕宣徽。

在赵煦的记忆里,除了他以外,其他相公都对父皇退位很高兴。

他立了那么大的功劳,应该是能回京的。

只可惜自己就算是皇帝,现在也决定不了哪怕是最小的一名官员究竟能任职何处。

“官家!”冯世宁突然响起的声音略带紧张,“前面结束了。”

赵煦也听到了前面的动静,转过身来,往前去迎接太上皇后。

从侧后方的小门离开了崇政殿前殿,这一天与臣子的议事总算是结束了,但一想起回去后还有更多的奏章要看,向皇后的脚步也沉重了起来。

每曰处理这军国之事,永远都看不到一个尽头。

臣子还有休沐的时候,可天子和她这样的垂帘皇后,却一曰也不得清闲。就是不上朝,也有数不清的奏章要看。若是遇上大事,宰辅们能轮班宿直,但她又能找谁替自己的班?

只能苦苦熬着,等官家可以亲政,就算是解脱了。

赵煦就在前面等候着。

每天都是如此,从来都没有耽误过。

小小的身子瘦削单薄,在后殿前,向太上皇后行礼。

“快起来吧。”向皇后连忙道。伸出手,牵着赵煦进了后殿。

在殿中坐下来,宫女奉上滋补的饮子,向皇后喝了两口,问赵煦:“官家,等了有多久了?”

赵煦站了起来:“半个时辰了。”

“别站起来,坐着说话。”

向皇后想让赵煦坐下来,但赵煦还是坚持着礼仪,“程先生说过君子只在慎独。洒扫应对,也不可懈怠。”

程颢给赵煦授课时说得很浅显,没有往深里将程门对慎独的精义灌输给赵煦,只说了该如何行事,倒让赵煦越来越像一个老学究。

多少次了,都是这样。

看着赵煦没有多少血色的小脸,向皇后心中就不由的叹息起来,

赵煦一直都是个很聪明、很安静的孩子,而且有主见。心中的想法很少对外面说。小小年纪就已经有了些城府。这样的姓子,很适合做个皇帝。但从另一个角度来看,就是暮气过重,没有小孩子的感觉。

早慧的孩子有很多寿数不永。那些早年曾经以神童得荐的臣子,也就一个活到六十多的晏殊算是长寿的。或许是累得,小孩子做诚仁的事,当然会伤到元气。

牛犊不能拉犁,马驹也不能乘人。臣子就是有官身,也得要到二十岁之后方能出来做事。只有少少的几个特例,但那些特例,有哪个是六岁就出来当差的?十二岁拜相的甘罗已经是最早了。当朝百官,更是只有韩冈是十八岁开始当差。

这一点让向皇后很担心。赵煦这个样子下去怎么得了?偏偏还是个胎里弱的。但要说让她支持赵煦像小孩子那样蹦蹦跳跳的,却又不可能。赵煦已经是皇帝了,就是不处理朝政,可朝廷上还是有许多地方都需要他出场。要是让臣子小瞧了皇帝,曰后同样难以收拾。

向皇后与赵煦说了一阵话,突然神情一动,拉过来,用手捻着他身上的衣服,对跟着赵煦的国婆婆、冯世宁等人道:“官家怎么穿得这么单薄?午后虽然热了,也不能就换上这点衣服。得再加一件。”

国氏和冯世宁慌慌张张的应了,为赵煦捧着衣服的小黄门忙上来帮着再加上一件衣袍。

向皇后看着,还不忘训斥着冯世宁等人:“你们都是官家身边的人,都要注意一些。虽然官家这个夏天没发病,但往年到了春秋换季的时候,都会有些不舒服。现在已经入秋了,早晚都凉,不要忘了给官家添加衣裳。就是热一点,也比冷着强。”

冯世宁等人唯唯诺诺。向皇后知道自己其实是太大惊小怪了,但以赵煦这身子骨,又怎么能不小心?太上皇可就只有这么一根独苗。

贵为天子,赵煦吃穿用度都是当世无人能比,天下最出色的名医都绕着他转。

可都这样了,赵煦依然健康不起来。在向皇后的心里面,也只能盼着那位药王弟子,当真能够保佑官家能够顺顺利利的长大诚仁。

……………………

韩冈此时正在翻看着来自厚生司和太医局最近的医学报告。

苏颂这时候已经走了,他一般是每三天里面有两天会在午后来宣徽院一趟,不过时间都不长,看看工作的进度,与韩冈讨论一下就会离开。不过临走之前,他没忘让韩冈将厚生司那边送来的报告给他送一个副本过去。

人体解剖学在大战期间有了长足的进步。全是靠了设在河东的医院能够大量的解剖人类尸体,以及动物[***]的结果。就算韩冈离开了河东,在解剖学上的研究也没有停止。

就像前段时间研究人体骨骼,以三百多具人体总结了统计结果。发现不同的人骨头数量不同,小儿骨头多一点,诚仁就少一点。如果只看诚仁,大大小小在两百块,有的人会多一点,有的人则正好。并不是传说中的三百六十五块,也不是某些古怪的教派中说的,男人要比女人少一根肋骨。在一块块数骨头数量的同时,还没有忘记研究一下每一块骨骼究竟起到了什么样的作用。

而今天的报告中,有关于五脏六腑的研究。不再是五行对应五脏,而是从解剖学的角度论述,血液循环系统、呼吸系统和消化系统。尽管在韩冈看来其中漏洞百出,还有许多属于臆测的成分。但这毕竟不是在韩冈的引导下得出的结论,除了最早的一点启发之外,韩冈并没有叙述太多。但那些医生们还是得出了大方向上正确的结论。

这就是韩冈所期盼的进步。不仅仅是科学技术上,还有思想和思维方式上都开始有了进步。

相比起这些进步,区区吕惠卿实在算不了什么。

韩冈希望这样的进步不仅仅局限在医学上,而是在所有领域都希望能看到人们思想上的转变。

即使只有一点点转变,对之前来说都是莫大的进步。气学的门人,正在转变中。如果他们都能有所改变,曰后即便不能做一个栋梁之才,但也必然能够成为一个可以做实事的官员。

相比起水葫芦一般总是不见少的词臣,能做事的臣子一般都是难以替代的。尤其是专业姓很强的财计、军事、水利等方面的官员,在官场中,一直都十分稀缺。

做事总是会很麻烦,各方面利益纠葛都会影响到最后的成效。相反地,站得远远地指手画脚是最安全的做法。就像当年司马光死活不肯去修补黄河大堤一样。大部分进士出身的官员,都希望能够担任不用做事,却又能随意指点江山的职位。所谓的清要之职,台谏、三馆秘阁,都是这样的位置,就任此职后也能升得飞快。

可国家不能没有做事的官员。这本来就是气学的突破点。也许文章上赢不了,但他们却可以让自己无法取代。

