\section{第18章 青云为履难知足(19)}

【又迟了一点,还是高估了自己码字的速度。下一章是明天早上七点。】

吕惠卿回到家中的时候,都是深夜了。

“想不到大哥在介甫相公那里留了那么久。”吕升卿来到书房中。白天时他已经先去过了王安石府上,并没有跟他的兄长走成一路。

弟弟走进书房,吕惠卿一点动静都没有。半眯起的眼睛,若有所思的看着烛台上跳动的火苗,视线的焦点却不知落在了何方,久久也不见开口说上一句话。

吕升卿觉得他兄长的神色有些不对劲,拉过一张圆凳坐了下来,问道:“大哥,出了何事?”

吕惠卿过了好一阵,才抬眼回应:“没事!”只是他又发了一阵呆之后,突然又道,“王介甫似乎已有南归之心。”

吕升卿眨了眨眼睛,一时没有反应过来。等到他将吕惠卿的话想明白,却又难以相信起自己的耳朵来。

“是辞相?”吕升卿眼睛瞪得老大,惊声问道,“王安石想要辞相?!”

“虽然王介甫没有明说,但方才与他说话的时候,的确有这个意思。”吕惠卿抿起嘴,王安石方才那副带着颓然、伤感的,现在还留在自己的眼中。

吕升卿连连摇头,“是不是大哥你误会了,这怎么可能?”

这根本不可能!坐在宰相的位置上,不是因为不再受天子信任,或是备受弹劾,有几人会主动辞去相位?王安石现如今再怎么说都是首相,两府之中,无人能与他的权柄相比。天子对他的信重,从给王雱的赠官上就能看得出来——从太子中允追赠到左谏议大夫上,一口气晋了十几级,比起在邕州殉国的官员受到追赠后,跳的级都多——加上新党的根基已经稳固,王安石只要不犯错,他的相位就是稳如泰山。

但吕惠卿很确定自己的判断,“王介甫不是恋栈不去的性子。复相后的这一年来,天子越来越自把自为,全没有过去的言听计从。前些天,王介甫还叹着,说若有过去一半也是好的。”

吕惠卿跟在王安石身边这么些年,对王安石的性格、为人很了解。当初为什么王安石不肯入朝,就是因为不能一展雄才。等到当今天子,他才肯出来任官朝中。如果做得不顺心,王安石就会干脆撂挑子。根本没有恋栈不去的想法。

“而且没了王元泽在身边辅助,王介甫也是难以为继。”

丧子之痛让王安石难以专心于政事之上,但更重要的是王安石现在身边已经没有可以信赖、且能力足够的助手了。当年的吕惠卿、曾布、章惇都因为各自的原因,而与王安石的目标有了分歧。不会再如早年一般,众心如一。

曾布不必说,他的背叛让王安石至今衔之入骨。

而吕惠卿在新党中独树一帜,虽然没有跟王安石的势力进行竞争,但他坐在参知政事的位置上,就算只是为了自己的地位稳固,也不能是王安石说什么他就附和什么,都要表现出自己的能力、才干和眼光来,总要表述自己的意见。这样的情况下,王安石也不能像过去那般,将他当做紧跟在身边的亲信助手。

至于章惇,他始终不是王安石的第一选择。虽然王安石对章惇的才干很看重,一直当做值得倚重的亲信看待。但章惇少年时过于放纵,恶劣的名声一直流传到此时。吕惠卿知道,王安石从没有想过让章惇顶替自己或是曾布的位置。章惇本人也是知道这一点,所以他会去荆南、会去广西,都是为了能够从枢密院走出一条路来,而不指望能如他吕惠卿,直接身登东府。

至于曾孝宽之辈,都可以独当一面,是方面之才,但并不能总揽全局,能力不够、威信也不够,论亲近更是远远比不上王雱。孤家寡人,没有商量国事、政事的助手,王安石的心哪能不累,又如何不退?

如果王安石要退的话,只有将新党托付给自己。除了他吕惠卿以外,别无他人!

“不是有韩冈吗?”吕升卿奇怪问着,怎么看都不能忽略掉王安石的亲女婿吧?

吕惠卿脸色微微一变。他很想要忽略掉韩冈,一直都下意识的避开这个让他心神不宁的名字。但吕升卿既然提起,吕惠卿也不会躲避:“韩冈如果愿意改换门庭,放弃关学,王介甫当然会着力抬举他。”

吕升卿摇了摇头,韩冈在学术上,不仅是秉承张载的关学,而且独有创见,已经渐渐有了一代宗师的名望,如何会改换门庭?格物致知四个字,现在在士林之中的名气,可是响亮得紧。

“他与王介甫根本就不是一条心,对新法也并不是全心全意的支持,王介甫把女儿嫁给他都没有扭转了他的心思,还能指望别的吗?”吕惠卿笑道,王安石是不会将政治遗产留给韩冈的,他很确定这一点。

“就算不是一条心,也已经是转运使、直学士了。再过几年,都能进两府了。”

吕惠卿嘴角抽搐了一下,或许是无意的,但吕升卿的话的确触动到了他的自尊心。

吕惠卿一门心思想从王安石那里接手新党。而韩冈则并不需要接手王安石的势力,他做到如今的广西转运使,都是靠着自己的功绩。

吕惠卿他被王安石提拔起来的时候,已经三十多岁,可攻击王安石和他的一份份弹章上,无一例外都会指责他是新进,远远不够资格。但韩冈二十五岁,升到广西转运使、龙图阁直学士,就没有一个谏官敢说他是新进。

仰之弥高的功绩,让御史们根本无法去拿韩冈的年龄说话。虽然私底下可以有许多手段让天子忌惮韩冈的年轻,但那些言辞是拿不上台面来的。

不过吕惠卿还是没有改变自己的想法,韩冈不会跟着王安石,能接手新党的只有自己。

…………………………

隔了一日,韩冈再一次被招入宫中,被天子询问着广西和西北的边事。

像他这样入京的边臣,在留京的一段时间中,被多次招入宫中咨询,并不足为奇。

韩冈本身也有许多建议,都不是短短一次面会,就能完全说得明白。

“……还有在邕州右江上游的横山寨正式设立马市。用兵一事,最重要的就是粮秣转运,滇马是上等的山地马,在交趾北部的山岭中,粮秣的运输都要依靠马匹来运输……”韩冈向赵顼说着自己的打算,“臣北上之前,已经遣人在横山寨试探的问过,只用了一个月就收购到两百匹好马。如果正式开办茶马互易的榷场,至少一年能收购到三千到五千匹军马。”

“马市之事,便如韩卿所奏,朕准了。等回去之后,你将具体的条陈呈上来。”

如果三千五千是战马的话,赵顼会兴奋得跳起来,不过从韩冈对滇马的叙述上看,都是些只能用来运送货物的驮马,能选为战马的寥寥无几。虽然军中也需要,但毕竟不如吐蕃马、河西马多矣。如今熙河路的茶马贸易,可是一年有两三万之多,可以上阵的战马占了其中十中二三。

等着韩冈行礼谢过,赵顼又问道,“朕听闻交趾土地肥沃,从不缺粮秣,不能因粮于敌吗?”

“交趾人也不算富庶,而且如今三十六峒蛮部都杀进了交趾国中,已经是因粮于敌,官军再杀过去,粮食早就没了。”

接下来的几天,韩冈几乎是隔日就被招入宫中一次,而广西的消息也一个接一个的传回京城来,让韩冈也能把握住广西局势的变化。

他在上京前安排的一切,已经到了收获成果的时候。

三十六峒在交趾国中的战绩,韩冈不去计算他们的斩首功,但如果他们将被拘禁的汉人解救出来,就依人数给予丰厚的回报。敌军首级会作假,但能说话的汉人,则完全做不了假,每一个都是真实的。而想要从交趾人手中抢回做工种田的摇钱树,也只有与交趾人拼命一条路。

不知不觉之间,被解救出来的汉人在飞快的达到两千人之后,用了稍多的时间超过了四千,现在容易救出的目标越来越少,所以人数增长也越来越慢,不过已经渐渐达到了五千之数。

三十六峒蛮部打得都是趁虚而入的主意,绝不与交趾军的主力硬拼。一碰到来援的交趾大军,就立刻会回撤。交趾军一开始还紧追不舍,希图一劳永逸,可当追得最是起劲的一部人马,落入了事先安排好的陷阱中,被守候已久的李信一口给吞掉了之后。交趾军对三十六他们也不再紧咬不放,而只是赶走了事。

为了抵御三十六峒蛮部对交趾国中的渗透,交趾派了大批的军队北上,想要将缺口给堵上,这样的动作,对其国力的消耗是致命的,但他们也不得不防。

不过传回来的也并不都是好消息,生病的还是有许多,已经造成了一成以上的病员,唯一值得庆幸的是没有多少病死的士兵,大部分都在康复中——良好的卫生习惯和严格的卫生条例,让南来的大军,不至于因为莫名其妙的疾病不战而败——只是药材不够了,需要紧急调运。

随着时间的过去,两府关于设立安南道经略招讨司的议案,也终于有了定论。

