\section{第37章 异乡犹牵故园梦(下)}

看过了工地,来访的两位贵人没说太多。

那位小官人本是一脸好奇,但去看了倭人所住的窝棚之后,表情也变了,似乎是对居住环境很有几分不满。转去看今天工地上的午餐,脸色就更不好看了。

在平一郎看来,工地上今天的伙食已经比平日好得多了。竟然都是干饭了,还有咸鱼萝卜汤,这在之前一段时间,是大小工匠们的伙食——大工的伙食还更好一点,能多一盆肉菜。而打杂的妇孺,只有稀粥和小块的腌菜和咸鱼吃。

可落在那位小官人眼里,竟然还是皱眉,“今日只如此,可以想见平日里是什么样了。”

旁边他的叔叔冯大东家则很会做人,安抚道:“等他们开始上工,自然能吃上好菜好饭。陈东家能舍得老本,供给他们一日三餐,已经是难得了。许多地方农忙帮工,地主家也只会给一天两顿。”

那韩小官人虽仍是不满,却也不敢不听他叔叔的话,点头受教,但又不甘心的暗暗瞪了平一郎的主人两眼。

平一郎的主人看着气氛尴尬,忙低头弯腰,上前陪着笑说了好些软话,又猛打眼色,让平一郎在旁帮腔,这才把这位小祖宗给敷衍了过去。

不过韩小官人的好奇心还是收敛了起来,变得跟他的叔叔一样没有太多的话,只看不说。

平一郎的主人只能搓着手,陪笑着请两位客人先上船。

船是江船,之前载着一行人从松江旁的别院抵达这边的工地。到了中午的时候,上面已经准备好了酒席,就等着主人和客人们入席。

正要开席的时候,其他几位预定在倭人坊安家立业的大东家,都不知从哪来得到了消息,纷纷跑了来。

其中一位大东家,比平一指的主人还要胖三分,个头只能到胸口,长得就像一颗球,平日里,走一步路都要喘三口气,可平一郎望向岸边的时候,却震惊的发现,他竟然是骑着快马过来,下马的时候不仅仅他喘得快要断气,连下面的骏马也一样快断气了。

其他大东家的情况也差不多,一个个都是步履匆匆。有两人共乘一艘车船,用人力脚踏,在水面上速度如飞。另一人也是乘了快舟,两排桨手将这艘前面有个龙头的细窄船只,划得几乎跃出水面,顺滑得仿佛就是在冰面上滑行。

待这几位走上船来,只跟平一郎的主人冷嘲热讽的寒暄了两句,便忙不迭的上前向两位客人行礼。

平常一掷千金,或是爱吹嘘自己的兄弟在京师有多高身份的贵人们,在两位客人面前,就像是下仆见到了主人一般谦卑,说尽了好听话。

看到这一幕,平一郎哪里还会不明白,今天过来的两位客人,身份有多么尊贵。只是在一干东主的寒暄和问候中,却都不约而同的避开了两位客人身份,竟然一个字也没提到。他们这么做,也从另一个角度,让平一郎了解到客人们的地位。

凭空多了几位客人,但宾主入席并没有耽搁。船上的大厨是平一郎的主人从扬州城特意聘来,早就预备好多余的材料。

平一郎捏着筷子,坐在最下首,虽然说他是丝厂未来的管理者之一,但依旧是仆从的身份,在这里能有一个位子,的确是被抬举了。不过以他旧日的身份,仅仅是能够入席,平一郎也不至于到受宠若惊的地步。

这是正式的宴席,看盘,干果鲜果,咸酸果脯,冷碟、热菜,按照正式程序一道道端上来,一巡酒过后,就换上两道新菜。一道道菜换得让平一郎目不暇接,即是几年前,他还是极尊贵的身份,在宫廷中,也没有享受过如此丰厚的宴席。

也难怪当初在日本时候,他的主人带着他去赴契丹大官的宴,出来后便不屑冷笑。当日的宴席,已经让平一郎为之惊叹,不敢视契丹为蛮夷。而今日,契丹人的宴席,又不知差了多远。

不过让平一郎来说,这次的酒席还差了一点。

尽管就在江边,尽管离东海也不远,但这一顿午餐,摆上餐桌的全然不见最受欢迎的鱼脍,即使有了鱼和虾,也全都是蒸熟,烧熟的菜肴。

平一郎的主人平日里最喜鱼脍,去日本时吃海鱼,回到中国就吃江鱼,据他主人说,天下鱼脍味道最好的还属开封熙熙楼做的黄河鲤鱼,但那只有入京的时候才能吃到。

为了就着贵客的口味,竟然连菜谱都换了。巴结到了这副田地,那两位的身份到底尊贵到哪个地步?平一郎的心中越发的好奇了起来。

饮了十七八巡酒,那位冯大东家似乎酒有些上头,指着菜盘子问平一郎的主人,“听说陈东家你最爱吃鱼脍,每餐无脍不欢,今日怎么不见?”

平一郎的主人陪着笑脸,“害怕贵客吃不惯,也就没上了。若是大东家想要,在下这船上,也有刀工最好的大厨,可以用现钓的江鱼割了做鱼脍。”

“罢了,不用劳烦了。”冯大东家摆摆手,“除了在京师和乡里,我就只吃热的熟食,水也只喝烧滚过的开水,要不然,有几人能走南闯北十几年没生过什么大病?”

一位东主连忙拱手道:“多谢大东家,在下可是又偷学了一招。”

一群贵人哈哈的奉承笑着,冯大东家不在意的摆摆手,“学就学吧,我那表兄盼着人人都学。陈老兄你这好吃鱼脍的习惯,倒是跟欧阳六一公一样。六一公家有一厨娘,最擅做鱼脍。猢狲入布袋那一位,每隔几日就提着鱼上门,要尝那厨娘的手艺……”

冯大东家讲起了古,几位东主赔笑点头,凑趣的说着话。

平一郎却不知道他们说的是谁。如果是唐人的名家,他一准知道,但日本与中国久不通往来,今人轶事,却是懵然无知了。

他只注意到了冯大东家的说话口音,在酒后已经与方才听到的标准官话截然不同。

不论是闽语还是吴语,平一郎都能听能说,这也是来日本的商人最常说的中国方言。而中国的官话洛阳雅音,虽然接触的时间不长,可他也勉强能够听和说了。

而冯大东家现在的口音,则与平一郎所了解的几种汉家方言都不一样,只是与洛阳雅音近一些。

应该是北方话吧。

平一郎饶有兴致的一点点分析着对方的身份。

不过知道酒席结束,客人们乘醉而归,他也没能从酒席上的言谈中,找出透露了对方身份的关键。只知道是很尊贵的贵人。

即使是平一郎,知道大宋朝中有一人姓韩,身份极为尊贵,据称是菩萨转世,在契丹人口中也不敢有分毫不敬。但平一郎并不认为两位客人会是那位贵人家的人,他们怎么可能会与商贾厮混?

也许是因为晚上想起过那一位,所以次日无事,平一郎便上街寻了一间书坊进去。在种类繁多的书籍中,专门挑选了那一位署名的著作。

就在他拿起一本那一位没有署名,却被书坊小工赌咒发誓说是其亲笔所撰的小说,便听见背后有人说,

“这些书籍都禁止夷人购买,要是你买书的事给人报了官,就是你家主人也会受到惩罚。你还是谨慎些个好,你家主人不是没对头。”

平一郎忙放下书,回头看时,却见是昨日的那位韩小官人。

相貌英俊、身形挺拔的少年,身上依然朴素,看不见任何饰物,只有手上,不合时宜的拿着把折扇。在他的身后,有两名精悍的伴当,正警惕的打量着自己。

平一郎忙上前见礼,韩小官人大喇喇的受了礼,指着他方才挑选的几部书,“不要买这些书,会害了你家主人。朝廷的这条禁令,平常虽没人管,但若是有人首告,衙门里也不可能不理会。”

平一郎还没说话,旁边的书坊小工就听见了,忙不迭的挤到平一郎和书架子中间,警惕的瞪着他。

这下子平一郎自不能再买他心仪的几部书,失落的从书房中出来,不敢对韩小官人有何怨言,但回头望着书坊里面,也不免还有几分依依不舍。

“想看书的话,先拿到户籍再说。”韩小官人说道,“有了大宋的户籍,就算是中国之人了。不再是外夷,自然是想买什么书就能买什么书。”

“多谢小官人指点。”平一郎恭声道谢。

韩小官人一摆手,“你也别谢得那么快,这的确是个办法,但一个外人,想要拿到中国户籍,哪里有那么容易。还不知道要几年呢。”

“但总算是个念想。”平一郎叹道。

韩小官人眯起了眼睛,眼神中又泛起好奇的神色,“你之前说你自己是倭国的普通百姓,我就不信,你说话一点都不像。这下子,可更难让人信了。能识字读书的,大宋的男子中也不过十之一二,我可不觉得一个普通的倭国庶人,能通读汉家书籍。”

平一郎张口结舌,吞吞吐吐了半天,也没说出个所以然来,“……小人……”

“算了,反正现在也没倭国了,你是什么身份都不重要。”

韩小官人善解人意的轻轻放过,却让平一郎脸色苍白起来。

啪的一声惊堂木响,侧前方的酒楼上传来抑扬顿挫的念白声,“话说那宋江……”

韩小官人被转移了注意力,抬头看过去,那间酒楼二楼坐满了人。“原来是说九域的,就是你方才要买的《九域游记》。”

“是韩相公所撰的《九域》?”

“难道还有别家的吗?天下分九州,九州之下又各有九州,大宋所居这赤县神州只是小九州中的一个。韩相公早年逢仙,周游了这九九八十一州,回来后就无所不知,创下了好大一份事业,做到了宰相。韩相公心胸一向宽大,并不敝帚自珍,又曲笔托名写了这部《九域》,希望别人也能有他这份际遇。”

韩小官人边说边偷笑,抿着嘴的笑模样,让他的话只剩一两分可信。但他的这番话,在别人那里也听说过。

“韩相公还说要造蒸汽船。”

“蒸汽船,现在虽没有造出来,但蒸汽车很快就有了。辽人都在学造蒸汽机。辽国的皇帝都说了,只要能造出蒸汽机,便可封王,倭国、高丽任选其一。”

韩小官人充满自豪的说着,不过他说着说着,又抿着嘴笑了起来。平一郎却听着心中一痛,耶律老贼竟然要把自己的国家送给匠人。

但那如果是像火炮一般的利器,就算裂土相赠,也绝对是值得的。

“可惜他们是白费心思,还是我大宋先把蒸汽机给造出来。”韩小官人得意的说着,“你若是有心,就每天抽小半个时辰听一听,定会有所得。”

平一郎看着他,突地跪下来,郑重的拜了一拜。无缘无故,怎么会害自己,只可能是指点,想起昨天韩小官人在工地上的反应,他就更加确信了。复国和治国的方略,也许就在这里面。

“多谢小官人,小人定会仔细去听。”
