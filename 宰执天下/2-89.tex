\section{第20章 心念不改意难平(八)}

【补昨日第三更。看起来下一更得到十二点了。】

王韶现在很忙。

忙得不可开交。

在一个一切都已上了正轨、已经正常运转了数百年的职位上任官,与白手起家、把一个衙门从无到有建立起来,这难度完全不同。

最直观的,就是胥吏的数量。在秦州州衙中奔走的胥吏人数,是官员数量的几十倍,多达三百,衙中几乎所有的庶务都是由他们完成。许多吏员都是父子传承,熟悉故事,贯通条令,公务到了他们手上一切都能做得妥妥当当,官员只需做好监督工作就足够了。

但古渭这边就不同,原本就是军寨。连书办、文员,都是吃着兵粮。衙前吏员的数量不是屈指可数,而是根本就是零。王韶奉旨设立缘边安抚司,就算把原来吃兵粮的文吏也统括进来,也是不敷使用——何况他们的编制属于古渭寨,而不是缘边安抚司。要不是新任寨主傅勍听话,王韶都没借口驱用他们——最后他想到的办法,就是从周围的千来户汉人弓箭手中招募。

做事的人少,能做事的人更少,这就是王韶所面临的现状。

偏偏王韶要头疼的不只是缘边安抚司的军事政事,管理屯田和市易都是需要大量人手去指挥。

屯田的工作,王韶很干脆的让给了高遵裕,让他手下的门客去头疼。而主管市易的人选早就确定,但元瓘能力毕竟不如韩冈。城寨外的榷场虽然早早的建立起来了,但王韶去看过几次,觉得里面乱糟糟的,没个应有的秩序。尤其是他从榷场回来后,顺道探望了几个来古渭养病的蕃部首酋,到了疗养院中转了一圈后,这样的感觉就更明显了。

刚刚把一个连九九口诀都背不好的应募文吏骂了下去,喝着凉茶,滋润着已经沙哑的喉咙,王韶越发的怀念起在秦州州衙中那群虽然总是少不了贪污受贿,欺压百姓,但终究还是能做事的胥吏。

‘也该找些门客来了。’王韶想着。在他还是机宜文字的时候,要养门客是浪费钱财。但现在他管着一个安抚司,若是没有些门客来帮着做事,光靠自己实在忙不过来。而且他在古渭,要把自家人安插进军中吃官饷,直接也比在秦州要容易。

王厚这时走进了厅中。王韶放下茶盏,问道:“韩家那边安顿好了?”

王厚点了点头,自家老子这两天火气见涨,让他说话声都轻了不少,“都已经住下了。孩儿遣了四个老兵去听候使唤,都是老实勤快有家室的。韩丈还让孩儿带话,要多谢爹爹关照。”

“韩玉昆说过他父亲精于农事,这事我已经跟高公绰提过了。明天……”王韶想了想,“还是后天。后天请他去高公绰那里,看看要开垦的荒地。韩家的那几顷田该从哪里划出来,任凭他挑选。”

“孩儿明白。”

“还有韩家的吃穿用度,你都要安排好,不要等他们自己去找人。”王韶继续叮嘱着。

王厚继续点头:“孩儿已经提前办好了,粮油肉蔬都让人送了上好新鲜的过去。韩家还有些不便携带的家当留在秦州没有带来,孩儿也早就安排了备用的。”

虽然已经从王舜臣那里听说了郭逵对韩冈的看重,父子两人在交谈时却绝口不提此事。韩家都搬到古渭了,两家也定了姻亲,韩冈的立场一般来说不可能轻易改变,并不是初来乍到的郭逵能动摇得了。

王厚倒是很佩服韩冈的魄力。官员上任最多带个妻妾儿女,把全家都搬到任上的很少见。此时官员调职很频繁得很,平均下来也就两年上下就得到另一处任官,带着全家老小奔走,其实是件很麻烦的事。就像王厚的继母和兄弟,都是被留在德安老家中,侍奉他的祖母,也就是王韶的亲娘。

“还算想得周全。”见儿子办事妥当,王韶口气松了一点,“跟韩家说,有什么需要可以尽管提,自家人不需要客气。”

“孩儿知道了。”王厚应声后,等了一下,见王韶没有其他话吩咐。便又说道:“孩儿还有一件事要禀报大人。玉昆的表弟冯从义,现今在元瓘那里做得也挺卖力的,这几天,已经听说他已经联络上青唐部,就是……”

王韶打断了儿子的话:“此事韩玉昆已经跟为父说过了。不是要借钱嘛,他要借就让他借,不要超过千贯就成。但利息不能少,而且年底前至少要把半年的利息偿清。一切照规矩来,为父不会为他徇私。”

“孩儿会转告给冯从义的。”

王厚答得痛快,让王韶有些不放心起来,“冯从义年纪轻,见识少。这世上又是人心险恶,保不准就会被人骗了。我不便叮嘱他,你去与他说,凡事多于元瓘、黄察商量,不要妄信他人。”

王厚忙点头答应了。若是韩冈不在古渭的时候,让冯从义给人骗了,他们也不好见韩冈,“不过大人也无须担心,冯从义找的人是俞龙珂和瞎药担保的,谅他们也不敢诓骗玉昆的表弟。”

所谓靠山吃山,靠水吃水,靠着新开辟的榷场,古渭的官员自然都在此有份买卖。王韶的那份在元瓘处,韩冈则是找了冯从义,高遵裕也有自己的代理人,也就是王韶说的黄察。三人都不是清正古板之辈,既然占着这个位置,在为朝廷卖命之余,从中分润一部分利益,没人会觉得不对。只要不犯国法,自己不明着出头做买卖,谁也不能籍此说事。

说完韩家的事,王厚一句闲话也不说的就出去了。韩冈不在,他身上的大小事务等于凭空增添一倍,跟王韶一样忙得脚不沾地。

王韶继续处理他好像永远也忙不完的公务,过了一阵子,高遵裕找了过来。王韶放下手中笔,又与他说起公事来。

屯田的事虽然王韶说是全权委托给他,但高遵裕却不能不与王韶商议。而王韶手头上的重要事务,也得通报给高遵裕这个安抚司同管勾。不然时间长了,两人之间必生嫌隙。

两人互相交流了一阵各自手上的公事。高遵裕突然提起新任古渭寨主傅勍,“傅勍自从当了知寨后,做事勤勤恳恳,不辞辛劳,也不见他再酗酒,韩玉昆这个人选推荐得确不错,挑他接刘昌祚的任是挑对了……只是刘昌祚留下另一个职位——西路都巡检——却得商量出个对策。傅勍官位太低当不了,也不能让这个位子空着,不然总会被人惦记着。”

“可实在没人啊……”王韶在秦州虽有几年时间,但一直被压制,难以结交将领,在秦州军中也没个体己可信、够资格担任西路都巡检的武将。

王韶本来听了韩冈的建议,想让傅勍兼任西路都巡检一职。但给朝廷否决了,宁可空缺也不让他暂代——比起当初有资格直登朝堂的刘昌祚,傅勍的本官实在太低,即便让他暂代其职,冠一个‘权发遣’的名目,也是不够资格。王舜臣现在倒是勉强够资格,

‘但他的资历实在太浅了。’王韶暗自叹着气。凭他个毛头小子,压不住手下的骄兵。

“我倒有个人选。”高遵裕突然道,“不知子纯意下如何?”

王韶略一犹豫,问道:“……是谁?”

“苗授。”

王韶听说过这个名字:“可是德顺军的苗授之【苗授字】?!”

“庆历元昊做反,苗授之父苗京死守麟州城,殁于王事,便因荫补而得官。他又是胡翼之【胡瑗】的学生,曾在国子监就学,是个文武双全的人才。”

高遵裕说得王韶都知道,“可苗授的本官已是供备库副使,在德顺军作着兵马都监,秦州西路都巡检怕是安不下他。”

供备库副使是诸司官,从七品。犹在大使臣之上,比当初守的刘昌祚还要高上一等。向宝的本官皇城使也属于诸司官,不过是最高一级,供备库副使则是最低一级。一般来说,到了诸司官之后,就能统帅一州或是一军的军务。

“秦州是下府,而德顺军则仅仅是军,级别差得这么多,德顺军的都监也只比秦州西路都巡检高出一线而已。再加上又是驻扎在古渭,不愁没有军功,苗授岂有不愿之理?”

高遵裕说的一切,王韶当然知道,而且他更清楚,以眼下拓边河湟的热度,就连刘昌祚都不会介意高职低配,放弃秦凤路兵马都监一职,回来做个西路都巡。不为别的,只为军功。

王韶想要一个亲信来统率缘边安抚司的军队,但他手上实在没人。出色的将领王韶知道不少,可眼下能保证在他手下俯首帖耳的却找不出一个。要是找来个跟自己不对盘的对头来,岂不是让李师中他们笑掉大牙。

王韶不得不感叹,比起在军中的底蕴,他这个江西进士终究比不上三代将门的高遵裕——高遵裕会推荐苗授,便是因为他父亲高继宣就是当年领军援救麟州的主帅。苗京的功绩还是高继宣报上去的,苗授得到荫补,也得承高家的一份人情。

王韶权衡了半天,最后终于点头。这个位子给高遵裕的人,总比给别人要好,“我这就给秦州发文,请郭太尉把苗授之调来古渭。”

