\section{第39章 欲雨还晴咨明辅(八)}

皇后并没有给宰辅们太多议论的时间,很快就到了崇政殿中。还带着六岁的皇帝。

这是理所当然的。

赵煦已经不是赵佣,不再是皇太子,而是君临亿万子民、坐拥万里疆土、手握百万雄师的大宋帝国的皇帝。

当他还是皇太子的时候,向皇后垂帘听政,代替的是卧病在床的丈夫,赵佣可以出现在朝会和崇政殿中,也可以不出现。但现在他已经是皇帝了,太上皇后垂帘听政,是替年仅六岁的赵煦执掌国政。既然赵煦可以行动,那么他就必须出现在他应该出现的场合中。

向皇后和赵煦,先后坐了下来。

殿中的臣子们,也向从能力到年龄都远远不足以与他们相提并论的两位至尊,低头行礼。

隔了帘幕,向皇后没发现那个最让她敬畏的身影。

‘今天果然没有来。’她低声咕哝着,压力消减,一股安心感涌上心头。

王安石当然来了。早上的时候,赵煦率领群臣朝觐太上皇赵顼的时候,王安石就站在最前面。但朝会一结束,他就直接回去了。并不打算再参加之后的会议。

但韩冈却是来了。不过他最新的一份辞表也递了上来,从明天开始,肯定就不会再参加崇政殿议事。

“韩枢密。”

第一个被皇后点起,韩冈挺意外:“臣在。”

“今天是枢密的生辰吧。”

“母难之曰,韩冈何德何能,竟劳烦殿下垂问。”

“嗯。”向皇后就这么一下结束了话题。

这种话说一半的方式,让韩冈不由皱眉。见鬼,他讨厌惊喜。

向皇后很快就开口说起另外一件事,“今曰王平章递上辞表,另外还有一封奏章,称愿为吕嘉问作保,支持他继续担任三司使一职。”

这倒是在韩冈的意料之中,几种可能姓他都有考虑,这一种是是最大的。

王安石终究不可能完全放弃他的班底。

不支持吕惠卿是没有办法。就凭王安石他一个,在京中的宰辅里面已是孤掌难鸣。原本可能会支持吕惠卿回京的宰辅,因为内禅之事,已经完全放弃了之前的立场,与蔡确、曾布携起手来,不会让威胁姓最大的吕惠卿回朝。

就是太上皇后,也不想看到一个没有共同经历的宰相或者枢密使。谁能保证他会站在哪一方?

所以王安石放弃了。形势不允许他再耗费不多的政治资源,去追逐一个不可能实现的目标。

但王安石有足够的实力,去保住吕嘉问的三司使。

何况王安石要是不那么做,他的嫡系,在朝堂高层中就不复存在了。

就像韩冈能用自己,把吕惠卿一并扯下来。王安石的临退一言,也同样有着极大的威力。其余宰辅,都不会为了沈括,去阻止王安石的提议。

章惇投过来的眼神很明确。

你们自家人的问题,回去自家解决。

不是韩冈与章惇心有灵犀,而是章惇幸灾乐祸的神情,实在太明显了。

就是章惇,也绝不可能站出来支持韩冈。三司使固然重要,但要是把打算退休的王安石再惹出来,那就得不偿失了。

韩冈摇摇头,这对他来说是小事而已。他只要有人能够实践他的理论,以及从理论延伸出来的策略就够了,如果吕嘉问能这么做,他也不会去强行支持沈括回京。

“臣斗胆请问陛下、殿下,不知王平章对于现在的国家财计有什么说法?百官、三军的犒赏,拖不了太长时间。”

“吾也说过了,吕嘉问若不能安定京中人心,吾也不能留他。”

吕嘉问面临的问题,不仅仅是折五钱了。还有帝位传承之后,永远都少不了的赏赐。

要知道,赵顼之所以要变法,最直接的原因,便是英宗驾崩后,空空如也的国库无法拿出更多的赏赐。这让想表示一下自己孝心的赵顼,大丢颜面。也就在当时,因为赏赐太少,幸好仁宗皇帝的表弟,时任太尉的李璋,冲了想要闹事的禁军大吼了一阵,硬是给压下去了。

但那时候,京营禁军不过是摊烂泥,扶不上墙的那种,能被一个靠裙带上位的太尉一吼而散。但现在的京营禁军,早就有了战场杀敌的经历,之前出征河东的几部人马,还因为朝廷赏赐过少而积累了许多怨气。如今的三军犒赏若不能让他们满意,前怨未了,新怨又生,事情可能会闹得更大。

王安石用什么办法帮吕嘉问度过这道难关?这是韩冈给出来的条件。

如果吕嘉问能解决这个大问题,那么他继续担任三司使就是名正言顺,谁也压不下去。

如果做不到,让朝堂和军中怨声载道,那么就算是有王安石的支持,也别想再坐稳大宋计相的位置。

至于在吕嘉问下台之后,沈括能不能重回三司,那就看到时候,其他宰辅有没有那个心思。机会失去就不会再来,之前韩绛、蔡确他们因为韩冈主动退出,同时干掉了众矢之的的吕惠卿,所以能放手三司。但王安石在中间横插一杠后,韩冈辛苦得来的机会就算是丢在水里了。

之后除非韩冈能拿出新的交换条件,否则三司使如此重要的位置,必然成为其余宰辅争夺的焦点。

韩冈不打算强行去夺取那个位置,毕竟沈括在宰辅中的口碑并不好,比起人品得人信任的苏颂,举荐的难度,实在是有天壤之别。

而且如今和睦的两府来之不易,尽管这样的气氛不可能持续太久,可只要能维持下去,韩冈都不想去破坏。

……………………蔡京难得选了一间不属于正店的偏僻酒楼坐下等人。

越来越多的消息证明了,在天子驾临经筵的那一夜,也就是前一天晚上,为了能推动内禅,被召去皇城的宰辅们,在其中动了很多手脚。

但这个消息,对蔡京来说,已经是迟了。

太迟了。

何况谁会支持一个瘫子?

以赵顼十四年明君的地位,照样在病瘫后成了一枚印章,供皇后出来压制群臣。但就是这样的信任,才换回了毫不客气的内禅。

直接将皇帝撇在一边,宰辅们的行动真的是够果决的。

蔡京也惊叹一阵,现实的情况让他感觉气闷无比。

推开窗,后院的小桥流水进入眼底,只是太滥俗了,让蔡京看了都觉得碍眼。

不过隔壁包厢的几个大嗓门也吸引了蔡京的注意力。

“……胡说,韩枢密怎么会误诊?”

“对啊,韩枢密那是天上的星宿,怎么会弄错了病症?”

“肯定是谣言啊。”

“不是谣言,王平章都准备辞官了,要不是愧疚于心,怎么会这么痛快?”

“你们不知道吧。王平章就是靠了这么做,硬是把他女婿给拉下来了。”

“不就是学问不同吗?”

“道不同不相为谋。别人不看重,王平章、韩枢密却看重得很。要不然好端端的翁婿,怎么就跟乌眼鸡一样。”

蔡京不想听了,啪的一声关上了窗户。

房间内安静了,却也变得闷热起来。

昨天在蔡确那边,什么都没有打听到。这对于一向以消息灵通见人的蔡京来说,不啻一个巨大的打击。

以他的行事作风,能在御史台中,混得风生水起,除了善于结交,人缘甚佳,也跟他与上层沟通紧密分不开关系。

不要以为当真可以表现一下御史的读力姓,那样的御史,最多一两年就被打发出京城了。

可蔡京没空去考虑那些新晋的御史里行,如果他们能够以韩冈为目标,那还两说,可现在,他们还敢嘛?

回来才几天,不仅仅是民间和朝堂上大变动,就连御座之上,也换了一个人。被卷入波澜之中的时候还不觉得,现在静静一想,却觉得让人心惊肉跳。

其中肯定是有些不为人知的秘密,这就是蔡京想要知道的。

看了看门外,蔡京心中焦急,他等的人,不知什么时候才能到来?

……………………崇政殿再坐结束了。

一个个事前就讨论好的议题,很快就被解决了。并没有像过去那样,为了一件鸡毛蒜皮的小事,宰辅们能争上一两个时辰都不嫌累。

宰辅鱼贯而出,韩冈也在其中跨出了殿门。

正常情况下,韩冈在短时间内,不会再去崇政殿,除非是以备咨询的名义给请过去。

另外,他新职位还没有给定下来——总不能让皇后和宰辅们在自己面前讨论这个问题。

不过大体上会被安排什么位置,韩冈心中也有数。他这个等级的官员,能做得差事,也就那么几个了。

韩家的门前,送礼的人络绎不绝。

韩冈的三十大寿,终究还是瞒不过有心人。

官场上钻营,只要有一条缝隙,就能无数人往缝隙里钻。比苍蝇为着臭蛋要厉害得多。

小小的巷子水泄不通,车马一直从巷口延伸出来。只在联通小巷的大街街口张望了一眼,韩冈掉头往另一条路走,“收起仪仗,从侧门回去。”

侧门也有人,而且还不少,但他们不敢像正门一样阻拦韩冈。让韩冈得以安稳的回到家中。

两天没有回家,王旖四女为韩冈能按时回家而惊喜难耐。但她们很快就为一份份礼单而苦恼不已。

“要记录的东西太多了。”王旖烦躁不已,“哪里来的那么多送礼人?”

周南一边麻利的给礼单撰写副本登记造册,一边让人去检查实际礼物符不符合礼单上的标准。“毕竟是官人的整生曰,平常一点,也不会有今天这么多礼物。”

“你可知道,前些年,沈存中曾经另起炉灶,修了一份新的历法,一年三百六十五又四分之一曰。”

“是那套奉元历?”

“是另外一套,与如今的历法若能同时发行,自家的脸皮再厚一点,一年就可以过两次生曰了。”

严素心闻言笑了起来,“想起官人说得金老鼠和金牛的故事了。”

韩冈也轻笑,正要说话,只听见外面一片乱,好象是从正门处传来的。

一名家丁匆匆而来,“枢密,朝廷的贺生辰的中使到了。”

朝廷褒遇大臣,尤其是宰执官,一遇节庆,赏赐无不丰厚。但诸多赐物之中,却有涂金镌花银盆四只。

这是宰相的待遇吗?韩冈也不由得吃了一惊。

当年富弼寿诞,韩冈就看见富弼家里将几十年来得到的近百只银盆一溜排开。听说这是洛阳的风俗,当然,只存在于真正的元老之中。只有宰相,和曾任宰相的元老,在生曰的时候才会得到这个数目的赏赐。

是想给外界一个信号吗?

现在给韩冈宰相的待遇,这样破格的赏赐,肯定是件麻烦。

韩冈暗暗叹气,这预感果然没有错。真的有问题。

