\section{第17章 观婿黄榜下(下)}

这就是进士!

能引得天下人为之疯狂的资格。

天下文官之中,只有十分之一是进士。一个进士出身,便是日后高官显官的基础。为了家族着想,稍微富裕一点的大户人家,都会想着一个进士女婿来支撑门面。而有了进士女婿,日后家中子侄被带契着,一族里的税赋劳役都能打个折扣。

而且百多年来,大宋上下都一直在宣传‘万般皆下品,唯有读书高。’成年累月的洗脑,一榜进士所受到的尊敬,更是远远超过他们真实的能力。

无论是现实利益,还是宣传的功劳,都让进士成了官宦富户嘴里争抢不休的肉骨头。而来自真宗皇帝亲笔的诗句:‘书中自有黄金屋,书中自有颜如玉’,便成了真实不虚的现实。

看着五十多的老头子,竟然一样被抢婚,慕容武不由长吟:“富家不用买良田,书中自有千锺粟;安居不用架高堂,书中自有黄金屋;出门莫恨无人随,书中车马多如簇;娶妻莫恨无良媒,书中自有颜如玉;男儿若遂平生志。六经勤向窗前读。”

韩冈一声笑,笑这世情,都是功利使然。当年唐太宗完善科举制度,曾有言‘天下英雄,入吾彀中矣’,这里的彀,就是作陷阱解。不过那时候,进士人数稀少,在官场上还要与门阀世族相争。而到了宋代,科举制度则是登峰造极,天子大肆提倡文事,天下才士有了晋身之阶,皆去苦读六经,当然没有心思去想着造反之事。

再比如省试取中后,殿试便不再黜落考生,使得恩归上而怨不归上;就算中不了进士,还有特奏名、免解,等一系列将士人招入体制内的手法;灾异之后,又籍灾民中之精壮为兵。在维持国内统治的手段上,大宋已经超越了此前所有的朝代。

而为此而付出的代价,就是现在赵顼、王安石耽思竭虑、不顾一切的推行新法的缘由。

韩冈没有再多想。世风崇文,对国家来说是有利有弊,如今弊端越来越明显。但一直以来,武夫对文人顾忌,给他帮助甚大,自己能安然无恙撑过最早的困境,就是靠了士子这个身份。端起碗来吃饭,放下筷子骂娘的事,韩冈不会做。

“恭喜思文兄高中,不如找个地方去庆贺一番。”

终于通过了礼部试,进士头衔已经九成九的落到手中,慕容武心情大好,开怀大笑着:“当然要去状元楼!”

“也好,就去状元楼。”韩冈点头同意,也算是讨个好口彩。

从国子监往状元楼去的道路有不少条。而其中最近的一条,是不从来路回去,而是继续向东,绕过大相国寺,再有一段便是状元楼了。

时近正午,榜前的人群依然拥挤不堪。榜单之下,时不时的都能听到一声‘我中了’的大叫,然后那名得中的贡生,就像臭肉一般,被一群苍蝇围上。一如方才慕容武的遭遇。

推开混乱中的人群,韩冈、慕容武翻身上马。向西行不到百步,就到了路口。前面就是大相国寺,正要过街,就看到一辆马车打横里过来,马车周围十几个家丁骑着马护卫着,都是穿着王韶借来的两名元随同样的红色袍服,好不威风。

“不知是哪家的宰执?”慕容武问着韩冈。

韩冈摇摇头,他也不清楚是哪一家。不过,他知道该怎么做。打了个招呼,与慕容武一起勒停了马,等着这辆马车过去。

宰相家、执政家的女眷,都有封号在身,乃是外命妇。不是郡夫人,就是国夫人。人数稀少,论起品级还在韩冈之上,自然要保持礼数,让上一让。

而低一等的县君、郡君,则就很常见了。郡君,杂学士、团练使以上的官员,他们的妻、母可以荫封。县君在东京城中则更是烂大街,相当于从六品郎中一级的文武官员的母亲、妻室就可荫封。

比如韩冈,他已经是从七品的国子监博士,中进士后,平级转迁为有出身官员才能担任的太常博士。之后再升三阶,过了正七品这道关口,就够资格上书为妻子请封了。至于他家的老娘韩阿李,则是因韩冈之功特旨恩授,早已是县太君了。

“哎呦,这不是姑爷吗!?”

横过路口是,那辆马车队伍中忽然有人叫了一声,车马齐齐停步,靠到了路边上。从车厢里面钻出来一个小丫鬟,冲着韩冈这边招着手。

好了,这下韩冈和慕容武都知道了是谁家的人了,也知道是谁人坐在里面:能得十几名元随环伺,韩冈的未婚妻当然不够资格,只可能是韩冈的泰水、岳母、丈母娘——受封吴国夫人的吴氏。

韩冈跳下马,走到马车近前,对车厢里面拱手行礼:“小婿拜见岳母。”

虽然还没有正式成亲,但该走的程序都走过了,只差最后一项了。对方‘姑爷’都喊了,韩冈称呼一声岳母也是理所当然。

“贤婿可是看榜归来?”吴氏的声音从车厢里传了出来。

“正是。”韩冈侧了侧身子,示意身后的慕容武上来:“这位慕容思文兄,是小婿在子厚先生门下的同窗学友,原是凤翔府天兴县主簿,今科与小婿一同参加了锁厅试和礼部试,今日约好了一起来看榜。亦是高中。”

“恭喜慕容主簿得中。”

“不敢,侥幸而已。”慕容武连忙上前,恭恭敬敬的向车中行礼,心中亦是暗喜,跟在韩冈身边,果然好处多多。“在下慕容武,拜见吴国夫人。”

“贤婿和这位慕容主簿,可是要去酒楼庆贺?”

吴氏一手处理王家内外事,看事情当然也准,猜得是一点没错。韩冈点头道:“正是要去状元楼庆贺一番。”

“状元楼……这意头的确是好。去状元楼要经过大相国寺,老身今天正好也要去大相国寺上香还愿,贤婿若不嫌老婆子絮叨,不如陪着老身走一段。”

自来丈母娘最为麻烦,韩冈当然不愿意陪着走。只是岳母的命令,他也不便推脱,总不能说自己嫌麻烦。而且韩冈从被风卷起一角的车帘中,隐隐约约的看到车厢内,除了吴氏和方才跳出车厢来的丫鬟以外,还有一人静静地坐着。

“长辈有命,岂能相违,小婿自是随行一程。”

说着,他便回身上马,跟在马车边上。慕容武知情识趣,稍稍拖后半步。

当年韩冈两次上京,吴氏都没有与他打过照面。而去年腊月时,韩冈与女儿定亲的时候,上门的是作为男方的王韶,韩冈本人不可能到场。但从丈夫和儿子嘴里听到的韩冈,已经让她点头了。现在很快又是进士了,当然没有什么不满意?只是能亲眼看一看人物模样,再说几句话,则是会更加安心。

路边的这番巧遇,就是让吴氏放下心来。相貌上足以配得上自家女儿,说话、行事看着也顺眼。本来因为韩冈推脱过婚事,吴氏还担心他有些由于是贫寒门第出身,因自卑而来的傲气,现在看来却完全不是自己想象的那样。

至于韩冈未婚先有子,女儿刚嫁过去就要给人当娘,那是如今常有的事,吴氏虽然有些抱怨,但想想世间的风气,也没有什么好放在心上的。

俗话说丈母娘看女婿,越看越欢喜。吴氏看韩冈看顺了眼,一路说了几句,就越发觉得韩冈的确比大女婿吴安持要强出了许多。且不说日后的前途,就是说起话来,不卑不亢却又能保持一份恭谨及谦和的韩冈,也比吴安持讨人喜欢。

至于日后会不会因为政治上的争斗,也跟自家生分了,那就要看运气。但在吴氏想来,韩冈别无背景,可不是有着枢密使父亲的大女婿,不依靠做宰相的岳父,还能依靠谁?王韶?……那可是外人!

走了一路,到了大相国寺的正门牌坊前。韩冈并没有继续送吴氏进去,而是直接告辞——车子进不了大相国寺中,车中人当然要下来,而在婚前,韩冈不便与王旖见面——吴氏知道女儿在车厢里的事,被韩冈知道了。

如果是讨人嫌的,吴氏当是要骂一句贼眼尖利,偷窥车中。但看对了眼的韩冈如此做来,吴氏就对女儿赞着:“韩冈知礼守节,行事又正,不阿谀奉承,当真难得。二姐,这样的夫婿,可是打着灯笼都难找!”

“娘……”帏帽之下,王旖一下羞红了脸。

被母亲强拉着出来上香还愿,竟然很巧合的碰上了自家的未婚夫婿。这样的巧合,其实每个女孩子都会喜欢。未婚夫婿被父母夸赞,更是让人高兴。

只是王旖她却又希望韩冈能在告辞时多一点犹豫和恋恋不舍,既然知道自己就在车中,为何能离开的如此轻松?

隔着帏帽上垂下来的薄纱,望着骑着马远去的背影,王旖的心中就不免平添了几分怨怼。

‘为何不能再回头看上一眼?’

