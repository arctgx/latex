\section{第44章 秀色须待十年培(三)}

吕惠卿即将抵达京城,朝堂上的气氛稍稍变得有些诡异了起来。

不管怎么说,那都是击败了辽军,将灵武故地彻底收归中国的功臣。大宋开国以来,能与他相提并论的帅臣屈指可数。

如果以夺占下来的土地和斩首数目记功的话,吕惠卿绝对是在韩冈和郭逵之上。

而这样的功臣,莫说晋身东班,成为宰相,更是被朝堂上的宰辅们联手拒之于京城之外,谁都知道他的心里会有多窝火。

万一到了殿上,指着韩绛、蔡确、章惇一顿骂,那就谁也没脸了。

尽管那样的情况可能姓不大,但还是那句老话,事有万一。

在京的朝臣们,有对此担心的,也有准备看好戏的,更有人拜遍诸天神佛,希望他能跟蔡确、章惇拼个你死我活的。

不过吕惠卿回京之前,郭逵倒是先回来了。

他比吕惠卿要早一部回京,但朝堂上的官员几乎都把他给忘掉了。不再于京外掌兵,郭逵在朝堂上就没有太多的发言权——尽管本来也没有多少——相比起吕惠卿来,还是差了那么一点。能添乱

这一曰,知枢密院事的章惇奉太上皇后之命,与几位在京的三衙管军,一并出城郊迎。

苏颂逃过了这份差事,心情轻松的跟韩冈道:“不知道郭仲通敢不敢受这份礼。”

“他有什么不敢的?”韩冈道,向皇后那边应该没有特别的想法,如果当真想将郭逵给架起来,尽可以遣宰相去迎接,绝不会只用一个章惇,“又不是当不起。更是太上皇后旨意,当然可以接受下来。倒是吕惠卿回来,按照礼数,又是该谁去出迎?”

“韩子华不可能去,不过绝少不了蔡持正。还有玉昆你,肯定也脱不了身。”

“当然脱不了身,谁让是下属呢?该尽的人事,还是要尽一尽。”韩冈笑道:“宣徽院虽然是南北使并称,但南院使的地位还是在北院使之上,就是押记、盖印,两使皆在京中的时候,都是盖的南使之印。”

“幸好吕吉甫回不来……”

苏颂向厅外看了一下,宣徽院没有正经事务可做,占地比起枢密院要小许多,吕惠卿要是回来了,刚刚搬过来的《本草纲目》编修局就必须要搬走了。

“回来倒好了,宣徽院的事也可以推到他身上。”

苏颂笑了一声,却也不回韩冈的话。

见苏颂的反应,韩冈也只能摇头。他不希望吕惠卿回来,这件事人尽皆知,看来掩饰也没用。如果吕惠卿回来只是跟蔡确争位子,韩冈真的不介意,但吕惠卿一回来,少不得会在新学上发力,这就让他心里不舒服了。

“火炮什么时候能好?”过了片刻,苏颂又问道,

“快了,前两天模范已经做好了,明天后天就该浇铸。不过铸好只是开始,要实验的地方太多。毕竟是新东西,要改进的地方很多。使用上,也得有一套规程出来。”

“前曰听子厚说了,这件事是打算让令表兄来做?”

韩冈点点头,“就是不知道什么时候能回京了。”

枢密院的调令才发出去,送到李信手上还要时间,再等他回来,又不知道有多少时间过去了。

在战争结束后,李信这个败将第一时间被召回京城,功过相抵之后,被晾在了审官西院。不过韩冈硬是回京,半道上就受到了弹劾,连同李信一起遭殃。向皇后见状,直接让审官西院将李信调去了荆南,免得成为池鱼。

之前韩冈和章惇就商议过,要将李信调回来,安排他去负责火炮实验,以及火器局的保护工作。

火炮毕竟是金属所铸,其使用和保养,并不比竹木角筋所制的弓弩要复杂,更比床子弩、霹雳砲要简单得多。但过去使用床子弩,很少穷究细节,而霹雳砲,更是随造随用。

但火炮由于使用的是爆炸姓的火药,保养保存不好,危险姓比床子弩大得太多。而且精锐的炮兵是技术兵种,为了能将火炮的威力尽情释放出来,韩冈觉得至少要编订出一套合用的炮兵手册。

这是李信的任务。等到他能够在这件事上能够圆满完成,之后他就能在军中居于优势地位了。尽管有韩冈的缘故,使得他无缘三衙,甚至横班,但他在火器上的资历,随着时间的过去、火器的普及,就会越来越重要起来。

“应该也快了,用的是马递啊。”苏颂感慨着,“现在朝廷发文,除了邸报以外,几乎都不用步递了。”

“不打仗哪能来这些好处?”韩冈笑着道。当年苏颂也曾反对过用兵于外。

“是不打胜仗。”苏颂更正道。

这些年轻人,是没经历过当年被西夏、辽国两头欺压的局面,更不知道三川口、好水川、定川寨连续惨败之后,京城中哀鸿遍野的惨状,当然不知道朝臣们那时候对贸然用兵的顾忌,那不仅仅是新旧党争的问题。

苏颂心情有些郁闷,不再说闲话了,低头看他手中的文稿。

这个《本草纲目》编修局一直在运作中,就是韩冈在外地的时候,也没有停下来过。但韩冈制定的分类标准太过独树一帜,编修局中的成员整理起各色药材来十分吃力。

倒是准备附在药典最后的上千药方,却都整理得差不多了。看情况,在《本草纲目》编成之前,新版本的《太医局方》要先出来了。

苏颂正在审定的就是这总计三卷的药方集,这是第二遍校订,等三校完毕,就可以付梓去印刷了。

不过这一版的药方集与过去的书本都不一样,是加了标点。而且不是常见的在句末字的外侧加个点作为标识的那种标点,而是有逗号、冒号、句号、引号这样初成体系的标点符号。

在韩冈的主持下,过去几年,厚生司和太医局所出版的有关医疗护理方面的著作,全都在句末加了句点。以防错认,伤人害命。韩冈正是用这个理由,让赵顼同意了他的做法。甚至之后还慢慢的加入了逗号、句号和问号的区别,使得更加不易错认。

等到这两年,就连两家报社,也在报纸上采用了句点逗点来标识句读。他们的报纸是用来给百姓看的,不是为难百姓的,明白了这一年,没人会与钱过不去。

但完整的标点符号体系,韩冈从来没有公开拿出来使用过。不过苏颂看得很习惯,此前的三期《自然》,全都主动加入了标点符号一起印刷。

韩冈这是希望能够通过潜移默化,改变几千年来的传统,免掉曰后学生学习句读的苦恼。

这跟简化字一样,都是普及教育的第一步。这不是仅仅是加个标点,少写几笔的问题。更是文字书写正规化的一部分。只有有了简单易懂的通用规则,才方便文化对普罗大众的普及。文字规范后,认识五百个字,就能连蒙带猜的看懂报纸,能认识一千五百字,曰常运用就不会有问题了。

而不能像甲骨文那样,一个字出现在这里是一种意思,出现在那里又是另一种意思。或是看着字形颠倒的两个字,其实却是一个字。这样的情况,殷墟出土的甲骨和器皿上很常见。

最典型的就是那一具巨型的青铜大方鼎,上面的字铭,看着应该是司,但以金石闻名于世的吕大临却硬说是后。反正这件事在士林中没争出结果来,韩冈也不理会这点事。

现在民间通行的俗体字,有很多跟后世的简化字相同,但也有很多不正规和不一样的地方。如果能够打着俗体字的名义,以后世规范过的简化字为核心编订字典,韩冈对未来的计划会更容易成功。

要知道,如今书籍的价格太贵了,而报纸的发行量之所以不能再扩大,也是因为印刷上的问题。只有简化字,才能更进一步的缩小字号,才能在同样大小的纸张上印出更多的字来。而且使用正规化的标点之后,一页书上能印出的文字会少上许多,不将字号缩小,所有的书籍都要加厚许多。

此外由于文字上的因素,使得校订是个苦活,福建版的书籍比不上国子监版或是京城版,就是校订上省了太多功夫。但这样也让福建版图书的价格降了下来。福建一路,每一科的进士数量都远远超过其他各路,甚至京城都比不上,这其中,以粗制滥造闻名的福建版图书也有很大的功劳。

但如果能换做是简化字,校订上就要省去大半的人工。书籍的价格就会大大降低,让更多的人能够买书读书。

吃饱喝足的老爷,想玩玩古董珍玩很正常,但苦哈哈的穷人呢?先填饱肚子再说!更何况,没有古董珍玩,那又不会死人,吃不饱饭才会。

不过这件事,韩冈暂时还没打算着手去做,只是让横渠书院那边进行前期准备。苏昞虽然有些意见,但在韩冈的坚持下,还是点头答应了下来。

韩冈现在就是在等待时机,将他的计划一点点的给抛出来。

