\section{第21章 飞逐驰马人所共(下)}

韩冈一行人就跟着何矩分开人群,向赛场内的包厢走过去。

他们这一行人中,领头的韩冈年纪轻轻,看起来像是高官显宦们的衙内,加上家眷都是带着帷帽隐藏相貌,一看就是到是大户人家,一路上人人为之侧目。不过当他们从贵人们专用的通道进入赛马场之后,关注韩冈一家的视线就消失无踪。

所谓的包厢,就是用木架子在看台上连成一片,搭建起来的一个个遮风挡雨的观赛点。外观和内饰都算不上奢华,甚至可以用朴素、寒酸一类的形容词来装饰。可比起外面毫无遮挡的看台,一座头上有顶棚的房间,还是很和士大夫们的口味,也符合他们的需要。

西晋石崇与人斗富,用锦缎布置出五十里的步障来。虽说如今不可能再有能恣意炫富而无须顾忌太多的世族。但春来出城踏青,大户人家却也时常在风景好的去处拉起一道步障来。十几丈、二十丈,不算什么大不了的事,关键是不能让外人惊扰到家中的女眷。故而到了赛场边的看台上,也就有了隐秘的需要。

韩冈被引到地头后,并没有引起什么人的关注。包厢外的一群人众,看模样就是各个等级的官宦子弟在其中占了大半,除了在旁边服侍的伴当,里面应是没有一个布衣白身的普通人。

比赛就快要开始了,听着悠悠响起的号角声,十二匹赛马已经进了栏中。

这些衙内和有官身的富户大半只关心自己参与到其中的赌局,正争论着今天的第一场到底哪一匹能获胜。对又来了一个带着家眷、占下的包厢还是边角处最低一级的新人,根本没什么人注意到。听到赛马进栏的声音,便一个个回到各自的包厢里去。

只是也有人对明显与这爱玩好赌的衙内富户不属于一类的韩冈很感兴趣。在韩冈一家过来时,就有一对眼睛钉在了在前面领路的何矩身上,而后便在韩家人身上逗留不去。

何矩领着王旖等家眷进了厢房,而韩冈走慢了两步,打量着在包厢外大声争论的一群人。待到号角声起,人群散去,一个身材跟何矩有的一拼的胖子没有跟着这些人回包厢,而是从人群的边缘走了过来。

韩家的家丁本来是要挡着他接近,不过韩冈冲领头的韩信使了个眼色,韩信便不动声色的将手下人给按住,不去阻挡。

那个胖子近前来,向着韩冈行了一礼,一口大约是京东的外地口音:“在下密州曲礼,任官浏阳主簿,不知官人贵姓?”

荆湖两路大部分县、监的名称韩冈都背不全,但浏阳县无论如何都不会忘掉,这个名字在千年之后也是十分‘响亮’,现如今则是标准的下县。

一个下县的主簿,基本上就是打发纳粟官的地方,是官,而不是差遣。纳粟官几乎不可能得到油水丰厚的实职差遣。交钱粮买.官能有的好处,一个是免了劳役,另一个是全家转入形势户的籍簿,提高了身份,仅此而已。想通过官位来牟利,将买.官的付出都收回来,这样的想法一点也不现实。

不过韩冈也没有崖岸自高,依然回答了问题,只是比较简短:“免贵,姓贺。”

韩冈微服出游,只是不想被人围观得走不动路,本没有隐藏身份的打算。但他名气虽大,可当面能认出他的人,在京城中毕竟还是不多。面前的这个胖子既然没有将他认出来,韩冈也不打算自报家门。随口报了个旧姓,却也不说多说细节。而且有一件事,让他有点在意,韩家的祖籍就是密州胶西,这个曲礼自称是密州人氏,算是同乡了。

曲礼仿佛没有感受得韩冈的冷淡,仍带着笑问道:“不知官人是在哪里高就?”

“朝廷的恩典,倒是不算很忙。”韩冈刻意说着让人误会的话,将这个胖子的思路给带偏掉。因为这个曲礼来自密州的缘故,他倒是不介意与他闲扯上两句。

王旖三女带着孩子们已经在包厢安坐了下来,几名护卫则各自守在门口,他们也是领会了韩冈的心意,没有像往日那般仿佛在守着中军帐一般的严肃。韩冈也不介意与陌生人随意扯两句闲话,这是在官场上很难得到的悠闲和放松。

曲礼正紧张的猜测着这位贺官人的真实身份。他认识领人进了包厢的何矩,能让顺丰行的大掌事亲自领路,身份绝不会低,而且关系亲近也是显而易见的。

要不是那一位身居显宦,不会轻入市井,且按照最近在城内城外到处乱飞的小道消息,在这个时间段里面,应该还忙着整理药典,跟他的岳父一争高下,并准备成为皇子的老师。曲礼还真要将这贺官人当成是在世人眼中如星宿下凡般一的那一位。话说回来,朝廷重臣都要讲究着个体面,哪一个出行不是前呼后拥?那一位可只比执政低一级了。

这个年轻后生,身边的人虽不多,但护卫看起来个个精悍,家世底蕴可见一斑。绝非包厢外面那群衙内的等级——能一日接着一日的声色犬马,全都是被惯坏的纨绔,换作是根底深厚的世家大族,对不肖子弟早就上家法了。越是高门,管束得越是森严,都被逼着辛辛苦苦的去考进士以维护家门不堕,哪里有空出来找乐子?

尤其是在对不明身份的陌生人极为警惕的这一点,更是让曲礼确认了自己的猜测。正常与别人通名,哪有就报个姓出来的?对人抱着高傲和提防的态度,又没有夸耀富贵的浅薄,绝对是某个累世簪缨的官宦人家的子弟,有个荫补的官职。今天大概是抽空带着全家出游。

作为一名纳粟官,同时又是一名身家丰厚的商人,曲礼很擅长把握落到面前的机会,“贺官人是第一次来这里?可否有看好的赛马?”

“没有看好的,只是来瞧瞧热闹。这些天听人说起过多次,才开办几个月,就已经快追上蹴鞠联赛,的确是有几分兴趣。”

“贺官人也爱蹴鞠?”曲礼立刻问道:“不知住在城中哪一坊?支持的是哪一队?”

基本上东京城中的球队都是以厢坊来划分,街坊邻里很少有人说不支持身边抬头可见的邻居,而去支持外人的。如果对蹴鞠联赛有一定了解的话,看一个球迷支持那支球队,一般就能知道他住在哪里了。

韩冈却摇摇头:“虽然寒家住在信陵坊,不过支持的是天泉坊的球队。”

信陵坊!曲礼闻言便是心头一跳。那可是内城中的厢坊,勋贵云集的地方,虽然他不知道到底有哪些达官贵人住在里面,但能住进去,肯定身份不简单。

不过他所支持的天泉坊的这支球队可有名的很,曲礼惊问:“可是棉行的喜乐丰?球场就在北面的?”

京城外西厢天泉坊是棉行在京城的,其球场就在赛马场不远处。

“乡里乡亲嘛。”韩冈点头笑道。

棉布行会不是顺丰行一家独大,连球队的队名,最后公推决定的也是十分喜庆、却让韩冈和冯从义直皱眉头的喜乐丰。对于这支球队,韩冈也不可能说那是自家的队伍。

与土生土长的开封人不同,外地迁来的人家多有支持乡里所组成的球队。京中的外地人很多,在京城的两百多、近三百支蹴鞠球队中,非京籍的占了十分之一。

这些球队在比赛中往往受到歧视,能在甲级联赛中出头的寥寥无几,能经常出入季后赛的,更是只有一支天泉坊的棉行喜乐丰队。这支队伍中有一大半是关西人,本来是一样要受到歧视,但蹴鞠联赛从赛制到规则,都是从关西传来,并由棉行发起。现如今连齐云总社中都有一名副会首是由棉行行首兼任。在所有外地球队中,棉行的球队便是独树一帜。名气也是最大。

但听到韩冈的话,胖子就笑了起来,“官人的口音可是一点都听不出来出身关西,倒似是开封府这里土生土长的。”

恭维了韩冈的口音之正,他却又多盯了韩冈几眼,心却有几分发颤。关西,那一位可也是关西的啊。

何矩这时已经将韩家人安顿下来,从包厢里快步出来。跨出门,就听到韩冈跟曲礼说道:“贺方也认识几个密州的朋友,论口音,倒是曲官人你最贴近官话。”

何矩看外形跟曲礼相似,笨重榔槺,但心思灵透,要不然也没资格执掌顺丰行在京城这边的一应事务,刚出来就听到韩冈的自称,到嘴边的招呼立刻就转了口:“原来是曲官人。怎么与贺员外在这里说话。”

曲礼听到何矩的话,终于彻底打消了对韩冈身份的怀疑。

这位名为贺方的衙内,他的员外绝不是市井中商家对客人的招呼,而是真正的员外郎——诸司的员外郎通常就是一名显宦子弟升到高位后得到的官衔。

砰地一声的号炮响,惊动了包厢外正在说话的三人。韩冈抬起眼示意了何矩招待这位密州来的曲礼,自己则再告辞之后,走进了包厢中。