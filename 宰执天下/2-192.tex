\section{第31章 战鼓将擂缘败至(十)}

对于前阵的失败,紧随在后的几支追兵随即提高了警惕。

远远向前放出斥候,又减慢了行军的速度,把拉长的队列收缩集中。

直到夜半时分,他们才抵达了前军败退的地方。被点燃的马车已经只剩无数余烬,闪着熄灭前黯淡的红光,而空气中,还弥漫着燃烧后的味道。

还有血腥味……

几十具尸体横七竖八的丢在路上,但都没有头颅,只有脖子以下的残躯。

西夏王族新生代的将领、同时也是右厢朝顺军司的团练使嵬名济,并没有下马,就着火炬看了一眼,便下令道:“都收拾起来!”

收到命令,一队铁鹞子下马,将被斩首后的袍泽尸骸抬到路边上。

但嵬名济当即皱起眉,提声道:“都送河里去!”

死得只是些部族里的丁壮而已,嵬名济并无多少物伤其类的心思,反倒是担心尸体摆在路边会伤了士气。他知道斩首记功是宋军的惯例,因为党项人则只在乎抢到的财物的多寡。心道被几万大军追在后面,还不忘斩首取功,宋人倒是胆大得可以。

扑通扑通的声音冲乱了哗哗的流水声,几十具遗骸消失在黑黢黢的无定河中。向前探路的斥候这时候赶来回报,就在前方三里处,宋人已经扎下了营寨。

“好胆!”嵬名济冷喝一声,手上的马鞭向前一挥,“追上去!”

……………………

宋军结下并不算是营寨,只不过在北面的来路上打下了一些半人高的桩子,缠上些绳索,充作栅栏而已。栅栏从山坡一直延伸到河边,前后三重,虽不坚固,但用来阻碍追兵却已经足够了。

此时殿后的队伍正在轮班休息。张玉带着亲卫,巡视在士卒之间。而韩冈和种朴正陪着王中正,也坐在了大军之中,皆是披挂甲胄,完全看不出三人身份上的区别。

谷地狭长,从罗兀城撤离的两万军分作四部,各部前后相隔半里有余,扎下了营盘。长蛇一样的阵线,的确是很有风险,只要后阵没能阻挡追兵,败退的队伍就能一起把前阵都冲散了去。不过镇守后路的是老将张玉亲自领军,至少张铁简的名望,能让前面的队伍歇得安心。

敌军随时可能到来,但宋军依然照常的点火取暖,火堆上架着锅,里面烧着开水。只要带过兵、上过阵或是行过军的将领们都知道,一口热水对于在春寒料峭的谷地中行军和驻扎的士兵们来说,究竟有多么宝贵。

道边山坡上的树木无人樵采,因而草木丰茂,枯枝败叶也多,拖下来就能点起来。树枝在火焰中噼里啪啦的作响,王中正也是就着火,只不过喝得却是热酒。

天子身边的近侍现在是豁出去了。如果官军被追兵击败,不论是在前军、后军还是中军,都是一个结果。还不如跟着张玉拖在后面,只要能顺利回到京城,当能得个勇于任事、临危不惧的评价。

一口热酒灌下肚,顿时就觉得在夜风中快要凝固的血脉顺畅了起来。哈了口酒气,王中正望着一堆堆篝火边,就着热水啃着干粮的士兵,对韩冈和种朴赞叹着:“追兵将至,大军尚能如此安稳,实是平生所仅见!张老总管,高都监,果然是军中柱石,深得军心啊……”

韩冈轻笑了起来,“总管和都监能得军心也不是没来由的。”他指了指周围士兵们,“都知可以问一问他们,究竟为什么能坐得如此安稳。”

“难道有什么缘故不成?”王中正有些好奇,在周边的人群中随便挑了一名看起来很老实的年轻士兵,让亲兵把他招过来问话。

年轻士兵看起来被王中正的召唤吓了一跳,到了面前,便跪下来连连叩头。

“好了,别做磕头虫了!”种朴不耐烦的把他叫起来,“王都知要问你话,站好回话就是!”

年轻士兵束手恭立,等着训示。

王中正便把他心中的疑问道了出来。

年轻士兵身上的胆怯不见了,一昂脖子,很骄傲的说着:“为什么要怕?!俺们本来就是赢的,打得党项狗屁滚尿流。就是广锐军那些贼子造反了,要不然哪轮得到党项狗追俺们。现在虽然是退出罗兀城了,但张老太尉要带俺们杀一个回马枪,再挣些功劳,俺们心里也快活。顺便还能出口怨气,让梁乙埋知道俺们官军的厉害!”

“说得好!就该让西贼知道皇宋官军的厉害!”王中正鼓掌赞了两句,便让亲卫拿了钱赏了年轻的士兵。看着他欢天喜地的磕头离开,回过头来,王中正却是不无犹疑的责问韩冈道:“怎么这等军情都说与卒伍?!”

“为了取信于人!不信人,如何让人信?”韩冈向着王中正解释着:“为将五德——智信仁勇严。要想军心稳定,‘信’是关键。圣人亦有言及与此,足兵足食,却皆不如一个‘信’字。”

种朴在旁帮韩冈敲边鼓:“先祖父当年自清涧移知环州,曾与一尚未归顺朝廷的蕃部族酋约时造访。不过到了约定的那一天,却天降暴雪。那名族酋以为先祖父肯定来不了的,便躺在帐里睡觉,谁想到却被冒雪而至的先祖父一脚给踢起来了。自此之后,他便举族归附于朝廷,听候使唤,全无半丝异心。”

“可是牛奴讹之事?!”种世衡的一诺千金、言出如山的名声,王中正也听说过。种朴只提个头,他就立刻记了起来。

种朴点着头:“正是其人!”

“正是因为一个‘信’字,所以种公虽已仙去,可遗泽犹在关西。”韩冈说着。

如果不是对将领们的信赖,相信高永能为首的将领不会抛弃他们,在黑夜中,当西贼追上来的时候,身边的这群士卒恐怕就会溃不成军。而不可能像现在这样,安安稳稳的等着反击敌军的追杀。

王中正深有感触,沉沉的点着头。

一通急促的鼓声,忽而随着北方的夜风传来,安稳的营地内,顿时响起了一片兵甲交鸣。

种朴当即跳起,眼望着北方的深黯,王中正也急急忙忙的扶着膝盖站起。

韩冈却是一口把手上的热酒喝干,站起身,整了整身上的甲胄,很沉稳的说着:“终于来了!”

……………………

在转过了一道河湾之后,远处如同火龙一般,在河谷中向南延伸开去的无尽星火,已经烙在眼中。嵬名济紧盯着那条火龙,心中迫不及待。可就在此时,沉重的鼓声在道边山坡上响起,顿时惊起了道路上的党项骑兵。

“是伏兵!”一群人大叫着。

因为一直都在防备之中,嵬名济手下的队伍并没有慌乱,而是纷纷下马,借助战马来抵挡山坡上可能飞来的箭矢。而离得近的,便立刻张弓搭箭,向着鼓声传来的黑暗处劲射而去。

铮铮的拉弦声此起彼伏,连绵不绝。过了好一阵,鼓声消失了,可应该有的伏兵却没出现。

“是报警的鼓声!慌个什么!”嵬名济骂了一句,一鞭抽到身边的亲兵身上,“继续向前!”

大军重新起步,因为感觉受到了戏耍,愤怒的情绪在军中蔓延,行军的速度快了许多。

行了不过半里,道边坡上忽然又是一通鼓声响起。

嵬名济毫不理会,提缰前行。可也有许多人下马,试探了射了几下,但很快就在周围人嗤笑的眼神中,红着脸上马加鞭。

当前锋已经冲到了卸下了绳索的栅栏边,看到了列阵以待的宋军的时候,嵬名济的中军离着张玉的将旗也只剩一里的距离。这时候,山坡上第三次传来了鼓声。

没人再去理会,只盯着前方的敌阵。可是这一次,却是当真有一片箭雨从山坡上落下。森森草木间,隐藏了宋军数百射手,他们在鼓声中,尽情的向下方的敌兵倾泻着利箭。

嵬名济由于身边的火光最亮,一开始就被上百张弓锁定,当鼓声响起,顿时就连人带马被射成了一只刺猬。党项王族的新生代出师未捷身先死,他所带领的队伍立刻一片大乱。而此时,前方宋人阵列中的战鼓也响了起来,先是神臂弓的一阵攒射,紧接着,一群锐卒提着大斧冲出了栅栏,如群狼入羊群,在敌阵砍杀起来。

“俺的计策怎么样?!”种朴从阵前的厮杀中回过头来,兴奋的向韩冈问着。

“这是种殿值的计策?!”王中正立刻惊问。

“正是!”韩冈鼓掌而笑。如今任何一个方案都不是韩冈一个人的功劳,皆是群力群策,他只是主持而已。种朴出的这个主意,成功的麻痹了追兵,让他们忽略了山坡上的动静。而响亮的鼓声又掩盖了伏兵的声响,从而让得到了最大的成果。

韩冈对着种朴赞道,“今夜一策,不辱种公令名。”

“不要耽搁!继续向前!”张玉此时就在将旗下大声呼喝,让传令兵把他的命令向前传递出去。

一队队宋军赶着混乱中的铁鹞子,逼着他们向北方逃去。突如其来的反击,轻易的打穿了追袭的敌军队列。跟在后面的几千铁鹞子奔逃而回,却又在狭窄山道上,冲散了更后面的追兵。

当初升的阳光洒满了山道,一名名大宋士卒高唱着得胜歌,带着党项人首级凯旋回返,重聚在张玉的大旗下。从张玉立足的地方,向北延伸五六里,倒伏着数以百计的党项人的无头尸骸,鲜红的道路,以血铺就!

