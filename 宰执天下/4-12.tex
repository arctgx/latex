\section{第一章 纵谈犹说旧升平(12)}

三月风光正好,春风被日头晒得暖洋洋的。连一贯阴森彻骨,总有些阴气不散、让人畏惧的开封府衙,也因为春日的阳光,而变得有了几分温馨。

焚上一炉香,倒上一杯茶。就在茶香、檀香之中,在散射进来的阳光下,慢慢的读着一本让人齿颊留香的好书,这是韩缜打发闲暇时间时,最喜欢的一种手段。如果是在家中,更可以招来两三名家伎,让她们以琴韵相伴。

慢慢的翻着书,轻轻的啜着茶,韩缜很是享受春日下的宁静时光。只是难得的闲暇并不长久,很快就被人打破了。

一名府中通传消息的老吏在外面求见,道是有急事禀报。

“什么事?!”将老吏招进来,韩缜的问话中就带着几分愠怒。

老吏在开封府衙中多年,惯能揣摩知府的脾气,知道此时撞到了韩知府的火头上。不敢浪费时间,用着尽可能快的速度、尽可能简洁的语言,向韩缜将事情说个明白:“有一百多汴河水磨坊的厢兵方才进了城,往常乐坊的韩舍人府去了。说是韩舍人要抢占汴河水磨坊,断了他们生路,没了饭吃,要去讨个说法。”

“汴河水磨坊?”

老吏点点头:“正是!”

“还真是太平啊。”韩缜笑叹了一声。

韩冈为安置军器监裁撤下来的工匠,抢了官营水磨的金饭碗,可到了最后,水磨坊就来了区区百来人的小打小闹,反而让人觉得今年春天的京城,实在是太平了过了头。远远不如一年多前,新党与粮商们的那场差点掀了东京城的激烈交锋。感觉就跟几十年前的太平年景差不多,内外皆是平静。只为了该不该裁撤三司之中不合格的冗员,朝堂上硬是扯了好几个月,最后还闹出一团乱子。

从今日这场看起来根本就是场闹剧的行动中,韩缜觉得政事堂中的几位应该并没有掺合进来,而是那些个得利的宗室和皇亲在背后推动——如水磨坊这样充满着油水的差事,往往都是交给远支的皇亲和外戚来管辖,这就叫做肥水不流外人田。

“也就这么大的一点事。”

老吏纳闷着,不知道韩缜脑中的想法到底是怎么转的,不敢搭话,垂着头等着韩知府的吩咐。

“让右厢的甘徽领人将其驱散,不要闹大了。”韩缜冷淡的赶人出去,又低头看着书。京府中的事务一向最为繁剧,能歇下来的时候并不多,他可不想在无谓的事上浪费难得的闲暇时光。

在京城中聚众上百,事情说大不大,说小也不能算小,惊动到天子倒是可以肯定,韩缜就不打算去凑那个热闹了,让人驱散就算完事。京城外的官营水力磨坊,属于宫苑诸司的地盘,与开封府不搭界,闹得大了也是韩冈的事,至于谁是谁非,还是让天子和政事堂来处理。他的兄长做着宰相,而他这个权知开封府的位置也只能算是过渡而已,正常过上两个月就要出外了,何必多扰是非,看书才是正经。

只是他手上的书卷才翻了一页,桌上杯盏里的茶汤还冒着热气,方才出去的老吏却已经转了回来。

“甘徽已经去了?”韩缜没有抬头。言辞举止、里里外外都是在对老吏说着‘说完了就快滚’。

“不,那个……”老吏的声音透着迟疑。

“怎么了?出了什么事?”韩缜抬起了头,皱眉问道。

老吏神色似乎还是有点恍惚:“去韩舍人家闹事的几个为首的厢兵,现在都已经被送到府里来了。是韩家的家丁给捉到的。并告他们啸聚为乱、白日破门、图谋不轨之罪。”

“什么?!”韩缜将手上的书卷一丢,差点将桌上的茶盏给打翻。

一百多人呐,就这么给韩冈家的家丁给捉了?又不是乡里的豪门世家,一举手就有三四百庄客可以驱用。京城中,恐怕谁家也找不出上百人能打能斗的家丁!

“此事当真?”韩缜不敢相信的追问着。

“千真万确。”老吏用力的点着头,“人现在就在外面。”

“好本事啊!”韩缜摇头惊叹。闹事的人不但没能成事,反而被打断了腿被韩家的家丁押送过来,当真是出乎意料之外,韩家的家丁真是有一套。

去官宦人家闹事的人,被苦主捉个正着,又押到了府衙中来。案子已经摆在了面前,韩缜虽然百般不情愿,也不得不亲自去二堂审案。

以周桂为首,几个领头闹事的此时都趴在二堂的地上不停地呻吟着。腿骨给根铁棍敲了,无一例外都是骨折,别说站了,连跪都没法儿跪。

一听到‘威武’声起,韩缜走上堂来,呻吟声就立刻大了三分。其中一个干瘦的汉子,更是哭嚎起来:“韩大府!韩大府!要为小人做主啊!韩家穷凶极恶,只是上门评理,就将小人的腿打残了……”

“小人参见知府。”

韩家的家丁则是向韩缜行了礼,动作划一,仿佛犹在军中。这几人,有高有矮,有老有少,但个个看着都有几分精悍,而且似乎都有些伤。领头的一个一眼看过去,韩缜就发现他的左手上少了两根手指。

传言中,韩冈将疗养院里没法儿再回军营的病残士卒,都揽入门下做家丁,看来倒是真的。因为飞上了天,最近刚得了官的周全也是个残废,手腕上装个铁钩子,换作是正常情况,他根本就没机会做官,都是靠了韩冈的抬举。不过韩冈家这一干病残家丁也是够厉害了,就这么几个竟然一下子就解决了上百人。

虽然对案情心知肚明,但韩缜也需要对此进行一番询问,也好将此事禀明天子。坐下来,一拍惊堂木,“究竟是怎么回事?尔等为本府细细道来。”

……………………

军器监中此时气氛紧张。周全在约束监中工匠时,当然就不可避免的将整件事给透露出来。听说了汴河水磨坊的厢兵聚众去了韩家闹事,旧时的札甲八作的作头、工匠都跑来向韩冈请命,要去跟他们杀个痛快。

不是为韩冈,而是为自己,要是事情给他们闹大了,天子收回成命,到时候没了活路的可是自己。而其余作坊也是同仇敌忾,同在一监之中,当然不能看着自家人最后丢了饭碗。而韩冈这名判军器监,也颇得人心,工匠们也都希望他能在这个位置上坐得久一点。

只是很快又传来消息,说堵在韩家门口的那群厢兵被打得屁滚尿流,领头的几个都被押去了开封府。原本拿着锤子、斧头的工匠们哈哈大笑一阵,就各自散去了。那等废物,不值得军器监中的汉子们动手。

等到众人散去,周全却变得坐立不安起来,藏在心底的不安掩藏不住,低声问着韩冈:“舍人,真的不要紧?”

韩冈命他去将军器监里的工匠约束起来,省得他们去与人针锋相对,他也的确去照着做了。只是听到家中急报,韩忠他们已经将闹到家门前的水磨坊厢兵,全都打断了腿送到了府衙里去。在感到痛快之余,周全也为这一粗暴的处理手段,而心中多了点忧虑。

“怕什么?杀到家门前了,不下狠手还以为我韩冈好欺负。”韩冈一点也不在意,“也并不是什么大事,只是打断了腿而已。不伤人命,这点小事没有关系。”

既然这一件事已经飞快的压了下来,那就什么都没关系。若是没有压下来,闹得京城乱了,不管有错没错,韩冈他都要受罚,御史台也不会放过他。

“如果真的闹起来,说不定还会怪罪到舍人头上,四哥还有几个兄弟也说不定……”周全声音一顿,仓促的转过话锋,“还不如让小人领着监里的工匠去跟他们火并一场,须怪不到舍人的头上。”

“错了!”韩冈笑着摇头,他听得出来,周全没说出来的话,其实是在怕韩忠他们被牺牲掉,“家人护家,那是忠心护主,不会有任何罪过。但换作是你带着工匠去跟人火并,那就是本官弹压不力、管束不当了。如今可不是你在军中的时候,打架斗殴都没有关系,只要能赢就不是罪名。”

周全恍然大悟,低头受教。只是当她抬起头,却见韩冈站起了身,整了整衣服就往外走。

“舍人?”周全疑惑着跟了上去。

“我要去入宫请罪啊,这件事还是早一点捅上去比较好。”韩冈边笑边走。

时代已经变了,如今不是仁宗庆历年间。天子和朝堂对于在京中聚众闹事的容忍度已经不一样了,按照老经验来做事,那是刻舟求剑,缘木求鱼的愚蠢之举。只要捅上去,幕后的黑手多半就少不了一份重责。而此事轻而易举的就被弹压下来,韩冈这边只要及早进宫向天子分说明白,根本就不会有事。

也正如韩冈所料,赵顼好歹也有了几年做皇帝的经验,当然能明白谁对谁错:“此事非关卿家的事。今日聚众闹事之人都在军中,每月都不缺俸禄,朝廷何曾亏欠他们!”

但正好论对在殿上的吴充却阴阳怪气的说着:“韩冈你家的家丁真是好武艺,不过三五人就大败百名军卒,若有个百来人,怕就是万军难当了!”

