\section{第30章 肘腋萧墙暮色凉(14)}

韩绛是为了亲眼看一看,在河东军中伏惨败后,种谔这里的军心士气而来绥德的。当然,另一个原因就是他也想了解清楚种谔的新计划究竟是否现实——光靠书信和公文往来,做不到这一点——种谔的率部回返绥德的确得到他的同意,但并不代表韩绛能就此放心原定计划作废后,种谔所订立的新方案。

已经是二月初。左厢神武军司的动作越来越大,前几天甚至有一队多达百人的骑兵,绕路抄到罗兀城的后方,逼近了抚宁堡。很明显他们是得到了西夏主力。

开战在即,韩绛心急如焚。随行而来的护卫军甚至还没安顿好,他就已经催促着在城衙的大堂中召集众将官来此议事了。

韩绛端坐在大堂正中,紫袍犀带,长焦幞头纹丝不动,但脸色焦黄的,唇角也因为心急上火而生了燎泡,世家子弟的闲雅舒缓的气质消没无踪,微皱的双眉给额头上添了好几道纵向的皱纹。

宣抚判官赵禼和种谔分据韩绛左右,其下陕西宣抚司的文武官员各自按官位高低站着。

韩绛等着众官一起行过礼,便忙催促着种谔把他的计划都说出来。

种谔在韩绛这里指手画脚的解说中自己的计划,甚至还把七八尺见方的大型沙盘搬了过来,拿着佩剑的剑鞘,在上面指指点点。厅内的七八位听众尽是有资格上朝面圣的高官,种谔也不虞他的计划会被泄露出去。

“……今次的守御还是要以罗兀城为主,西贼不善攻城,罗兀新城的城墙和壕河已经完工,以城中的兵力足以抵挡。不过环庆、泾原甚至秦凤,还望相公能在西贼来袭时,督促他们出兵,扫荡附贼村寨。让西贼不能专心攻打罗兀……”

“……前几日在罗兀,高永能在北去接应河东军时,顺道把沿途不肯降伏的蕃部都清理了一遍。没了横山蕃人支持,西贼也不可能久攻不退……”

种谔在大堂上朗朗做声,韩绛听着微微颔首。而以军事方面的才能而著称关西的赵禼则在旁直挺着高瘦的身子,略薄的双唇向下弯出了一个饱含了怨怒之气的弧度。

赵禼很后悔他没能劝住韩绛的催发河东军的调令,如果河东军不是为了赶时间而走的神堂道,不至于出援的近两万人损失了大半。就算归于河东修造的四座寨堡,最后只修起了一两座,或者干脆就没有修起来,但只要河东方面有兵,有一支随时可以出动的军队,绥德城就是安全的。而不像现在,必须要从罗兀城调兵回来。

临战分兵回师,本就不好打的仗,现在可就更难了,真亏种谔还能说得头头是道。

赵禼正腹诽着种谔的夸夸其谈,韩绛却突然点了他的名:“公才,子正的这套计划,你的意下如何?有何要补充的?”

‘补充?我这宣抚判官是给人缝缝补补的吗?!’

韩绛这话问的,分明就是已经同意了种谔的计划。赵禼心头火起,不过他一直都挂着脸,也没人注意。

“子正领军回镇绥德,这是极稳妥的。有子正守着绥德,此城当不至有失。但留在罗兀城的高永能。他的威望不足以震慑众军,一旦西贼攻至城下,不知罗兀城中的军心是否能稳得下来?!”赵禼说着他心中担忧的事,借机讽刺了一下种谔临战前离开罗兀。

种谔脸色略沉,正想出言反驳,但一直沉默的站在班列最后的王文谅,却忽然开口:“末将听下面的士卒们都在说,军中现在有了韩管勾,就算上阵拼命都安心了。绥德这里必须要有种总管坐镇,但罗兀城那里也须得安定军心。不如变通一下,让韩管勾去罗兀,也能帮着高监押一把。”

听到韩冈的名字,韩绛眉梢就跳了一下。他可不喜欢听到这个名字。但刚才种谔才赞过韩冈,说韩冈他在罗兀安置伤病,加之一系列建议,为罗兀城的顺利修筑立下了大功。

韩冈有此才能,韩绛也不会因人废事。他剔起眼皮,问道:“韩冈现在在哪里?”

“韩冈就在城中。”负责后勤的陕西转运判官李南公出来回答,“前几天他押了罗兀城的伤病,刚刚回绥德来。现在在城东南设了疗养院,把伤病都安顿下来了。”

“让他再去罗兀。”韩绛毫不犹豫的下令,“既然他能提振军心,还是留在罗兀城好一点。”

韩绛的视线从厅中众人身上一扫而过,并没有人出来反对。这个时候,能添一分胜算,就是一分。种谔也不反对,但他对提议人的身份却有些奇怪,王文谅好像跟韩冈没有什么瓜葛,但他说话分明是没安好意。真不知韩冈是在哪里得罪了这个小人。

不过种谔对王文谅的手段嗤之以鼻,也深感愤怒,难道现在去罗兀是送死吗?

罗兀城绝不会破!

关于韩冈去罗兀,仅仅是一件微不足道的小议题,后面还有许多亟待讨论和敲定的计划。一场军议从中午,一直开到了深夜。散会后,种谔回到了书房中,他在大堂中解说了半日,早已是喉咙冒烟,口干舌燥。正大口的喝着降火的药汤,种建中不知从哪里得到了消息,跟着种朴一起过来。

见了种谔,种建中开门见山的就说到:“五叔,我也一起去罗兀城。”

种谔的双眼危险的眯缝了起来,随手把茶盅放在一边。他这个侄子一向精明,怎么今天发了浑?知道他跟韩冈关系好,但有何必要同去罗兀城?难道罗兀是绝地,一起去送死表示负责,这感觉很悲壮吗?

“韩冈去罗兀,能稳定军心。你去做什么?!”种谔隐含怒意的质问着。

“五叔,今次从罗兀城回镇绥德,知情都明白五叔你是因为河东军大败,迫不得已而为之。但外面总有不知内情的,说五叔你是……你是……”种建中突然变得吞吞吐吐起来。

种谔的脸冷下来:“是什么?”

种建中鼓足勇气,抬起头:“是临阵脱逃!”

种谔一听之下,便大怒喝道:“谁说的?!”

种建中却毫不畏惧的与种谔对视着,过了片刻,种谔转过头去,脸上的怒色也褪了。种建中的说法,的确是有道理。不明内情的还还好说,真正怕的是那些故意传播谣言的。若是被他们宣扬出去,他种五承袭自种世衡,并在战场上熬打了几十年,才在军中积累下来的威望,可就要打水漂了。

转过头来,种谔又盯了种建中一阵,眼神锐利,心中却有几分欣慰。他的这个侄儿是想去罗兀,以自己的身份来证明他种谔战前离开罗兀绝无怯战之心。但种建中去是不成的。

“十七。”种谔叫着自己的儿子。

种朴立刻跨步上前,弯腰拱手,称呼公私分明:“请大帅吩咐!”

“你与韩冈一起去罗兀城!”

要稳住罗兀军心,已经颇有声望的韩冈有资格,但作为添头的种建中并不够格。不过他种谔的亲生儿子种朴,却还是能顶一点事的——儿子总比侄子要亲。

当种朴恭声应诺,接下军令,拉着还想辩说的种建中离开书房后。种谔靠在交椅背上,望着屋顶的梁椽,略显颓然的低声道:‘这样总不会有人说我有私心了吧!’

……………………

当两天后,韩冈和种朴重新返回罗兀城的时候,已经可以听到传自北方山间的号角之声,

这一路上,韩冈虽然都有跟种朴谈笑不拘,宛如常时,但心中一直都是颇为沉郁。回想起周南送他离开的时候,一直强忍着没哭出来,但红掉的眼圈却更透出了心里的悲伤。

本来已经在绥德城中安坐,笑看涛生云灭。想不到,王文谅在军议上竟然插了一句嘴,自己就必须再到这虎口险地走上一遭了。现在骑虎难下,只能求着种谔的计划真的能够实现。幸好种朴就在身边,种谔为了取信于军中,把嫡亲儿子都送到了最前线,也不会有人说他回镇绥德是临阵脱逃了。

种谔虽然有好几个儿子,但种朴的才能却是其他几人所不能比,在种家的第四代里,也是不输种建中而出类拔萃。种家损失不起这个未来之星,或者此时的话说——将种。当罗兀城有险,必然会倾力来援,也不枉他前日在外听到军议后,匆忙间耍得那些心机。

二月初八。当日头越过正南方的最高点,开始向西偏移的时候,一阵尖利的报警号角声传遍城中。当韩冈、种朴随着高永能匆匆走上城头向北望去,一队三百多人的党项骑兵,已然出现在罗兀城外四五里地的位置上。

“是铁鹞子!”

种朴看着他们的旗号,就对韩冈低声解释着。

这一队铁鹞子气势汹汹,因为就在昨日,位于最前沿的赏逋岭寨仅仅抵抗了片刻便告陷落。当时韩冈和种朴也像现在这样站在城头上,看着北方山峦中的一缕烽火,仅仅燃烧了半个时辰的时间便消失无踪。当时韩冈的背上一阵发凉,都说党项人不善攻城,但一座新修起的堡垒如此轻易的就为之陷落,这让他对于这条传言有了很大的疑问。

不过看到那队铁鹞子慢悠悠的开始向罗兀城逼近,韩冈的目光重又坚定起来。仿佛回到了一年多前,还是一个要服衙前役的穷酸措大的时候,为了自己的性命,而在陈举一手遮天的势力中奋死拼搏的那一刻。

‘都放马过来好了!看看谁能站到最后!’

