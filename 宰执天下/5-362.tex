\section{第33章 枕惯蹄声梦不惊(19)}

“石岭关和赤塘关都放弃了?”

“是。”

“折家的兵马破了神武【今陕西宁武】县?”

“是。”

“所以你们放弃了忻口寨?”

“……是的。”

听着张孝杰的报告,耶律乙辛的脸色越来越难看。难怪这位南府宰相会一路赶回来。这样的决定不用最诚恳的态度来报告,待到秋后算帐,耶律乙辛他可不会轻饶。

张孝杰从代州经飞狐陉只用了五天就赶来了易州。摇摇欲倒,尚幸他这个文臣体质不输给南朝的武将多少,多多少少还支撑得住。

看着亲信灰败的脸色,耶律乙辛心中暗叹,这事也怪不得他,自己也有很大的责任。

但河东的局面崩溃得比想象得还要快,还要突然,这让耶律乙辛心情越发的沉重起来。

自从韩冈抵达河东,开始主持一路军务,河东局势变陡然一变,之前兵败太谷,耶律乙辛是知道的。萧十三退守石岭、赤塘和百井寨的决定,他也予以追认。

本想着只要有代州、忻州在手,跟宋人交换回的西平六州,恢复澶渊之盟,再增添个三五万岁币也就可以收兵止戈了。

可惜之前打得主意,现在都成了妄想。

麟府军竟然没有走天门关去支援太原,反而直接越过边境攻下了神武县!打算从神武县抄近路去救援代州。

“忻口寨这时候差不多已经丢了吧?”

在张孝杰沉默的点头中,耶律乙辛回忆着他记忆中的西京及河东地理。

为了能够更直观的了解国中的山河地理,耶律乙辛这几年来都在模仿宋人,造了数以百计的沙盘,每次捺钵迁移,都要用几十辆大车来拖行。也因此,耶律乙辛对辽宋边界的大体地理都有所了解。大同往南,代、忻、并【太原】诸州的关隘,他都能做到心知肚明。

失去了忻口寨,直到代州之前,再无关隘可守。如果是平原地带,只是皮毛枝节的小事,但在河东,留给耶律乙辛的选择就只剩很少的几条路。

河东的地形就是这般让人无可奈何,根本不适合骑兵发挥出自己的特点,甚至因为过于庞大的马匹数量,而陷入了粮草不济的境地。如果在平原上,一个村庄就能找到让一支千人队吃上三五个月的粮草。但在群山环绕的河东,根本不可能。

现实击败了耶律乙辛之前的如意算盘,在眼下的局势下,也只有放弃更多的土地,将兵力集中起来保住最后的一点底牌,“实在不行,退到代州城下也可以。”

只要有雁门关和代州在手,如同鸡肋一般的西平六州照样能交换回来。对宋人而言,雁门关和代州远比贺兰山下的兴灵要重要百倍。

张孝杰躬身应是,他和萧十三想得到了正是耶律乙辛对此事的认同。

萧得里特在旁精神振奋,“虽然是退,却也是骄兵之计,宋军在石岭关和忻口寨之后食髓知味,多半会追上来,到时候在代州城下,可让宋军见识一下什么是大辽铁骑!”

做梦呐?

耶律乙辛冷眼看了萧得里特一记。

从韩冈之前的表现来看,耶律乙辛可不敢做这样的白曰梦——纵然萧十三和张孝杰还有着些幻想但耶律乙辛没有——那一个在大辽国中都被下面的百姓当成神佛来崇拜的新任枢密副使,将会是是大辽未来几十年的噩梦。他在河东声威赫赫,就算下面的武将贪功,也别指望他们敢拂逆韩冈一星半点。

“去找萧念经来!”耶律乙辛提气冲外面叫了一声。

片刻之后,一名年轻的武将随即应招而来。他是萧十三的小儿子,又在耶律乙辛帐下听候使唤。

耶律乙辛此时已经写好了一封密信,签了字、画了押,装入信封封好。等到萧念经到来,便把信递给了他。

又看了一眼张孝杰,耶律乙辛吩咐道:“在你的从人中,挑了几个精神好的,跟他一起走……现在就走。”

“下官明白了!”

这是必要的取信手段,要让萧十三不用担心事后的责罚,全心全意的对敌,让他儿子回去是最保险的办法。

萧念经拿着信便立刻出发,张孝杰也出去安排人手跟随。

随着几人的离开,耶律乙辛阴沉下来的脸色让萧得里特不寒而栗。

但耶律乙辛没有阴沉太久,而是很快就颓然一叹。

这一仗打得莫名其妙,开战并不是耶律乙辛的初衷。在开战之前,他从没有想过正常的交涉会变成数千里边境线上的战争。

甚至之前调集兵马南下,亦是多存于威吓的心思,同时还有从国中钓几条始终不甘心的大鱼上来的打算。最后的结果是耶律乙辛本人也始料未及的。

所以现在一打起来,场面就是惨不忍睹。方略、计划完全都是漏洞百出,幸好宋人那边也一样。这才有了河东方向上的一系列胜利,直至韩冈出现。

而河北这里,七八十年无战事,南朝的河北军其实远不如过往精锐,能有所表现,主要还是靠了普及下来的铁甲、硬弩。不过相应的,大辽这边也没有了纵横河北的经验。

与草原和山林中无数蛮部连续战斗了几十年的结果,就是已经找不回与实力相当的对手大规模作战的眼光和能力。

当时下令时尚无自觉,但事后耶律乙辛再一回想,有许多地方他都做错了选择。否则刚开战的时候,他坐拥数万宫帐精兵,决不至于被郭逵挡在边境线上,最后形成了僵持不下的局面——要不是河东那边有了突破,他耶律乙辛的脸皮都要丢尽了——直到郭逵遣兵北上易州,试图打破僵局时,才找到了可以利用的破绽。

不过现在耶律乙辛已不打算将战争再继续下去了,河东还有些劫掠而来的收获,而河北这里根本就是坐吃山空。易州城下他虽然是胜了,但不能突破宋人的防线,无法利用宋人的财富来补充战争的消耗,那么就等于是失败。

与其再耗下去,还不如趁着河北形势尚好,逼着宋人尽快达成和议。中京道那边已经开始不稳了,回离不虽然是在耶律乙辛的支持下坐上了奚六部大王的位置,但他根本是个废物,而老奚王谢家奴死因又蹊跷,使得族中有了许多异声。如今大军在外征战,保不准就有人想趁机作乱。

无论如何,见好就收才是正道。

……………………

河北败了。

可韩冈并不担心李信,从信报中得知李信只收了轻伤,这就让他完全放下了心来。

至于战败后的责罚,一般来说,朝廷对地位较高的败将都是比较宽大的。最多也不过一时被贬斥,至于什么时候被重新启用,就要看他背后的靠山是否可靠了。

何况即使不论太子之师这个身份,以及药王弟子的声名,只看眼下河东战局正向好的一面转变,自家这个河东制置使好歹也算是功不可没,只要之后能稳住战果不失,他在枢密院中的地位将稳如泰山。

这样的情况下,就算李信败得再惨,但只要有个韩冈这样的表弟,谁也不敢欺到他头上来。郭逵人老成精,也肯定不会犯迷糊。

当然,这一切的前提是河北能维持住现在的局面。否则郭逵和李信都少不了要受到重惩。

可是纵然韩冈和郭逵之间能通过急脚递传递消息,但他所得到的最新军情也是七八天前的了。这段时间中,辽军会不会突破边寨防线,即便可能姓随着天气渐热而越来越小,可这是谁也不能拍胸脯保证的。

不过韩冈也没空去担心郭逵,河东现在面临的局面还没空闲到让他为兵力更充裕、兵备更完全的河北军操心。而且郭逵也不是那么让人放心不下的统帅。

河北、河东、陕西,北境三处都与辽人接战。战事已经结束了的陕西且不论,河北、河东两家的将领们可都在暗中较着劲,尤其是在河北的战局败坏,而河东形势好转的情况下,河东的将领都想表现得更好一点。

只是现在韩冈麾下的文武官员中,功劳最大的不是统领大军逼退萧十三的章楶,也不是保住了忻州的知州贺子房,或得辽军放弃各处关隘的折克仁、秦琬,更不是镇守太原的王.克臣,而是还没有到韩冈麾下报到的折克行。

“令尊这一回可是立了大功了。”韩冈对折可大称赞着他的父亲。

“可大替家严谢过枢密的称赞。”折可大连忙道谢,做儿子的不能为父亲谦虚,只能感谢韩冈的夸奖。

韩冈笑了笑:“辽贼退守代州城,令尊和麟府军其功为最。夺占神武,可谓是神来之笔。”

见折可大又要道谢,他便伸手制止了。

“如果不夺神武,想要从府州赶往代州,走天门寨经太原北上倒是最快一条路了。但那样就不知要耽搁多少时间。所以说,”他看看左右,“折府州的决断让人拍案叫绝。”

章楶、黄裳等人纷纷点头称是,非是如此,他们还在石岭关外辛苦攻城呢。哪里可能连忻口寨都夺了下来?
