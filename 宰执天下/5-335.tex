\section{第32章 金城可在汉图中(14)}

萧十三和张孝杰已经抵达了太原。正遥遥望着太原城的城墙,惊讶于的这座城池的宏伟,那是绝不下于东京辽阳和南京析津的雄城。

“城内还能守上几天?”

“别的倒不清楚,只知道粮草足够。”

“普通的办法攻不下来呢。”

耶律乙辛的态度是见好就收,尚父殿下的亲信和军队离国过久过远,保不住后面就有人动起坏心思。

只是这并不代表会放弃土地,太原府固然不会占据,可代州是要跟宋人交换回西平六州的质物,不可能说丢就丢。

虽然隔得远,其实兴灵仍然可以算是黑山河间地的前沿屏障,萧十三和张孝杰两人都清楚,已经将自家的宫卫安置在黑山下的耶律乙辛,不会放弃贺兰山下的土地。

将国境线恢复到战争开始时,并从宋人那里得到让人满意的补偿,对于大辽就不算吃亏,对于耶律乙辛,那就更是让他在登临帝位时减去更多的阻力。

在萧十三和张孝杰而言,只有耶律乙辛登上皇位,或是他能做上周文、魏武、晋宣,他们才有未来。否则最终也只会是家破人亡的结果。

“敌进我退,敌驻我扰,敌疲我打,敌退我追。”萧十三满心的感慨,他从来没有指挥过这样的战争,也没面对过这样的敌人,“说起来这十六个字对我大辽全民更加有用。”

‘乌合之众,不值一哂。黔驴技穷,技止此耳。’

这是张孝杰刚刚听说韩冈准备发动河东义军时所下的评语。但当真看到韩冈使人在河东散布的所谓《御寇备要》的时候,却就再也说不出这样的话了。

并不是里面的内容吓到了张孝杰,而是他在那本小册子里面,看见的是韩冈的声明。是向所有河东百姓做出保证,朝廷不会与辽人善罢甘休。

不管怎么说,里面说的那些东西,能不能有成效是一回事,会给河东百姓带来什么样的刺激则是另外一回事。

韩冈在散布河东的这本《御寇备要》中是怎么说的?不要与侵略者硬拼,而是用不停地骚扰加以拖延,不要让强盗轻易带着赃物离开,否则日后将再无宁日。拖得时间越久,宋军赶来的援兵数目就越多。到最后,将会站到一个压倒性的优势。

韩冈在书中向所有河东百姓做了保证,一定会让贼寇血债血偿。这是把自己的脸面和声望与胜利挂钩起来。

这也代表了宋国朝廷绝不会息事宁人,甚至很难达成任何有利不利的。为了皇统承续,南朝的朝廷必须顾全药王弟子的脸面。只要宋国的皇太子还没有成人,两府就算做出了对韩冈不利的决定,那位正垂帘听政的皇后也能逼着王安石、韩绛把诏书扯碎了吃下去。

当然,这一切的前提是河东这边的局面不会继续恶化。能将帅府行辕放在太谷县,韩冈看起来就是一副很有信心的样子。虽然太谷县在南面挺远的地方,但萧十三和张孝杰同时望向南方时,仍是一脸的凝重。

或是营地中,在半夜突然出现一支响箭;或是行进时,道边飞来几支箭矢、石块。这已经不是一起两起了,受伤的也不止一个两个。得到了韩冈的公然许诺和教导,宋人中,越来越多的乡兵甚至百姓,开始用他们最拿手的办法来骚扰辽军。

虽然说现在仅仅是癣癞之疾,但宋人几乎是人人都有潜在的动手能力,如果他们当真群起而攻之,纵然精锐如大辽,多多少少也会吃一个亏。蚊子多了也能叮死牛的!

而想要解决这个问题,充足的兵力是第一要务。

可遗憾的是,经过飞狐陉来援的也不过八千骑兵。对于现在萧十三和张孝杰想要达成的目标来说,只能说是不无小补。而且不会有再多了。

大队的骑兵从狭窄的山道进军,远比同样数目的步兵更为艰难。在崎岖的山道中行进,骑兵不会比步兵更快,而消耗的粮草则是五倍十倍——具体数目要看战马和骑兵的比例——而战马,只要还想让其能够继续作战,是不可能拿来驮运粮草的。

前一次来自于河北的援军,使得飞狐陉上各军寨所储存的粮草全都消耗一空,短时间内不可能再从南京道得到更多的援助——无论是粮草还是兵员。而西京道剩下的兵力要牵制宋人的麟府军,更加不能轻动。

现如今西京道上能动用的兵力差不多都来了,再多就得往更远处找。可不说路程远近的问题,仅仅是人马调动后,为了稳定地方而必不可少的人事和兵力上的调整和安排,就决定了不可能那么简单就把人直接都抽走。草原上的阻卜诸族可没一家是省油的灯。

萧十三和张孝杰纵然身居高位,也想不出有解决的办法,只能摊着手苦笑。谁让这一场战争事发突然,远出双方预料?从一开始,辽国这一边就没有动员举国之力的打算,现在仓促调兵,哪里能调来多少?

“不如将阻卜部的人马一并招过来好了。”就在帐中,萧达摩得意洋洋的向一文一武两位主帅提着自己的建议。

这位萧十三族中的年轻人正是春风得意的时候,劫掠而来的战利品充斥了他的营帐,而代州首功的功绩也让他出尽了风头。这份持续了多日的兴奋,让他忽视了萧十三眼中猝然腾起的怒意。

“主意是不差,若阻卜大王府能调来两三万兵马,宋人在河东也就完了。”张孝杰先是笑着点点头,却又叹道:“只是他们离得太远啊,再快也要到一个月之后。如果这一回顺利的击败了宋人的援军,根本就不需要那些不听话的异族出力。若是我军反过来被宋人援军所击败,那么阻卜人来了反而会是大麻烦。”

“我大辽怎么会败?!”萧达摩的声音高亢,“纵然不胜,也能平手,到时候,赶来的阻卜人便能派上一点用场了。”

张孝杰在旁眯起了眼,几乎在冷笑。萧十三则恨不得用力给这位让他丢人现眼的族人一脚:“你丢铜板,有几次铜板竖起来过?!”

萧达摩被训得一头雾水,然后便抱头鼠窜。萧十三狠狠地瞪着他的背,回头过来,脸上又是尴尬的笑,“让相公见笑了。家里就没一个聪明的,一个个蠢得跟牛一般,除了敢拼命,也只剩个听话的好处了。”

“能听话难道还不够吗?”张孝杰哈哈笑,“我家几个小子,肯听话的可不多。”

耶律乙辛的信使在这一天稍晚的时候赶来了,带了耶律乙辛的吩咐,也稍带了他在过了石岭关后,在山道中被乱箭射击的消息。

“一路上挨了七八箭。都不是强弓,应该是宋人的百姓。”那名信使说得十分淡然,仿佛没什么大不了的。

难道韩冈的那本小册子已经传到忻州?萧十三眼中有着遮掩不住的疑惑:“这才几天啊。”

“足够了。”张孝杰沉声道。

“是在哪里遇上的?”萧十三眼睛一瞪,“此等贼子当立刻剿灭!”

不过来自北方的信使摇摇头:“这并非目下急务。”

这是个好答案,萧十三和耶律乙辛都不准备改变得太多。耶律乙辛尚在南京道,并没有西来的打算,萧十三和张孝杰都觉得尚父殿下应该还另有一个计划。不过既然没有透露给自己,两人也没打算去猜测。现在集中兵力,将太原夺占下来,才是最重要的。

但这时候,又是一名亲兵滚着进来:“三交镇外的草料被烧了!”

萧十三蹭的跳了起来,张孝杰也差点没能坐住。

太原城外的草料场早一步就被宋人放弃,里面二十余万石的干草刍豆被一把火烧个精光。虽然早在预料之中,可萧十三听到这个消息之后,还是恨不得将城中下此命令的守将全都吊在路边的树上。

三交镇位于太原和忻州之间的山中通道的南端出口,是交通枢纽。名为镇,却有着不弱于军寨的防御能力,相当于城池。镇中存放的粮草数目极多,接近三十万石,有本来便存在库中的,也有新近劫掠来的,是萧十三现在敢于继续打着太原主意的底气之一。

“现在不知两位有何打算?”信使低声问道。

萧十三略作思忖,而后说道,“粮草只要去抢去夺,就不会缺少。可错过的机会就难以挽回了。我等依然会按照之间的计划,将宋人的援军给歼灭了。少了援军,太原城将会不攻自破。至于那些所谓义军义勇,更是不成气候,韩冈岂会将希望放在他们的身上?”

放开了让援军过来,只要一支接一支的将来援的宋军击败,太原城中守军很快就会不战自溃。不同方向同时进军,必然会有先后,只要有个两三天的差距,就足以解决任何一支宋军。

“这并非急务。”信使依然重复着这句话。

“那什么才算是急务?”萧十三心中不痛快已经表现在了脸上。

“尚父有何吩咐?”张孝杰则更为敏锐了一点。

“在我出来前,尚父只问了一句。”信使看了一下两名面前的两名重臣,耶律乙辛的左膀右臂,“韩冈在哪里?”

