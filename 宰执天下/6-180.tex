\section{第20章 千山红遍好凭栏(下)}

李诫走在去将作监衙门的路上。

陌生的面孔和前面领路的中书堂后官,让往来于途的官吏们都不禁多看了他几眼,身为近日都中名人,李诫的身份立刻被人认了出来,成了议论的焦点。

多年来,李诫住在城中的时间不会超过三分之一,没有了城市的喧闹,李诫的耳朵变得极为敏锐。即使现在回到了吵杂的皇城之中,还是能够支离破碎的听到沿途的闲言碎语。

“工匠都成官了。”

李诫扫过一眼,这话出自一名须发皆白的吏员,撇着嘴对着身边人说着,眼睛还往这边瞟过来,但对上李诫的视线后,就吓得一抖,连忙将身子转了过去。

李诫冷笑,这一位多半就是积年为吏,不得一官,故而心怀怨怼,怨言出口,岂不知祸从口出的道理。

他是不记人,陌生面孔没那么好记,但前面的堂后官似乎耳朵也很好,认识的人也多,回头便对李诫将那位老公吏的身份透露给了他。

之后是报复,还是放过,那就看自己的心意了。

“这还真是将作大匠了。”

说怪话的是一名青袍的官人,年岁倒不大,但看着满脸的傲气,定当是进士出身,说不定还进了崇文院。

将作监的长官,汉时官名便是将作大匠,听人这么说,当然就是指自己是匠人。

李诫看了一眼后便不屑一顾。自家父亲都已经做到了知京兆府,即使自己不是长子,也是有荫补在身的。官宦世家的子弟,还真有人当自己是工匠?何况农夫之子都能做宰相,做到什么官,如今也不看出身。

“他不是进士,可他能经世。在韩相公眼里,这就是经世济用的大才啊。”

这话语带讽刺,玩着谐音的游戏,不过几名官员聚在一起,李诫没能找到究竟是谁说的。

“他姑母嫁出去了吗?”

又是一句戏谑的话语让李诫的脸沉了下来,不再左顾右盼。

他父亲李南公当年曾经被御史们群起而攻,主要原因是因为支持新法,而御史们所用的借口则是李南公的同母亲妹——也就是李诫他的姑母——年过三十都没有出嫁,而李诫的姐妹们都嫁了出去。世风奢靡,如今女子出嫁都要丰厚的嫁妆,李南公嫁女不嫁妹,是舍不得嫁妆,私德有亏。

自家长辈的事,李诫不好多说,但原因并不是御史台说得那么简单,不是同一个母亲的姐妹都嫁出去了,一母同胞的亲妹妹怎么会舍不得嫁妆,而让她寄住在亲戚家里?以李南公的身份,再如何舍不得嫁妆,也比不上他的面子重要,更比不上御史台的一份弹章。

但泼上来的脏水,没那么容易洗干净。贼咬一口,入骨三分。李诫也知道,朝廷中一说起他的父亲李南公,立刻就会想起那位因为同产兄舍不得嫁妆

而嫁不出去的李家女。

当年在韩冈离开京西都转运使的位置后,就任京西转运副使的李南公,便成为重新划分开来的京西北路转运使,之后又遍历地方,资历已经老得不能再老,可两年前韩冈想推荐李南公担任三司使,却遭到了朝中一众大臣的反对,甚至连太后都觉得不合适,后来给了一个宝文阁直学士的补偿,被打发到关中的京兆府去了。

尽管韩冈让李承之重临三司使的位置,维持了局面,又将反对最力的几位官员都打发到了地方上去,可是李南公经过这番折腾,离开朝堂就越来越远,眼见着年纪往七十走,这辈子恐怕也没机会再回去了。

父亲李南公在外任官,长兄李譓中过进士,也在外任官——因为做事偏激,为上官弹劾,所以至今没机会回京——李家的门楣,现在只能落在了李诫的身上。至于家中旧事,更是只能这么让时间去消磨掉人们的记忆了。

不管官吏们如何泛酸,李诫在抵京五日后,就任权同判将作监的任命,已经得到了太后和中书门下的批准。尽管因为资历不足,官职前面加了两个前缀,但李诫成为将作监最高长官之一,却是确凿无疑的。更重要的一点,就是李诫自此成为韩冈最重要的亲信之一,为世人所共知。对于那些嫉妒,是没有必要在意太多的。

将作监的官衙不远,没有太久便到了。将作监丞以下十余官员,近百胥吏,皆在门外迎候。而判将作监事赵子几,也在门中迎接同僚的到来。

嫉妒的眼神在将作监中官吏的脸上,比外面少了许多。他们绝大多数都是懂行的人,而且这两年也没少打交道,知道李诫博来这份差事有多不易。

判将作监事的赵子几是新党中人,不过在韩冈面前也算守规矩,没有因党派之争而找麻烦,在轨道的修筑过程中十分配合,故而能安然留任。而李诫这一回虽是,但他的工作与赵子几并不冲突,赵子几出迎时,亦是笑语殷殷,发自内心的欢迎李诫的到来。

明工科需要一个传习本业的官办学校,所以朝廷将正式设立工学院,专门用来培养技术官僚,为参加明工科做准备。

这件事,将作监中已经传遍了。而工学院的提举,据闻正是将由李诫来兼任。与此同时,据传李诫还将会主持修订一部有关堤坝、寨防、轨道、运河等工程修造的典籍,作为工学院教学的课本。李诫身兼多职,当然也不会有太多的时间来处置将作监的公事,更不用说与赵子几争权夺利。

就像如今的同判厚生司温杲出身医官,兼管勾太医局,同时还提举医学院。他主要的工作正是在太医局和医学院上,厚生司中的工作,由判厚生司吴衍一人处置——这也是为什么正式的敇命未下,便有那么多人认定李诫将会担任提举工学院一职。

只要李诫当真能如温杲一般,谨守本分,赵子几巴不得这个新同僚能在将作监中久一点。

进了大堂,照流程验了敇命、告身,送李诫来上任的堂后官拿了赏钱告辞,赵子几便一一向李诫介绍衙中的官员。

李诫上京次数不少,衙中大部分官员他都打过交道,每一个人,李诫都温言的说了几句,拉了拉交情。

当这番介绍到了最后,赵子几指着一名肤色微黑、满面风霜的中年人,“这位是提举内中修造所公事杨琰。”

没有介绍表字,也没有介绍其父祖辈的身份,就这么简简单单的官职和姓名。

内中修造所,是负责宫中建筑的修筑和修缮的衙门。提举公事,一般是内侍官担任,只有偶尔才会让三班使臣充任。内中修造所辖下有千名雄武军,

充做军匠。管理这么多兵员,提举公事的地位其实并不低,如果是内侍官担任,将作监对其的管辖权微乎其微。

内中修造所的地位绝对不低,而提举公事,更是不应该放在最后才介绍。但厅中官吏视若平常,脸上堆满了谦卑笑容的杨琰本人,同样没有反抗这个待遇的意思。

看见杨琰,李诫却带了几分惊喜,“可是杨琪的兄长?”

杨琰明显的愣了一下,然后才点头说道:“正是下官的弟弟。”

李诫这下更为热情,拉起杨琰的手,笑道:“吾受命主持修造铁路轨道,君弟为辅佐,监理工程。韩相公也几次赞许,称令弟为人勤谨,营造上也不输昔年的大工俞皓。京泗、京洛两条铁路,令弟居功甚伟。”

被李诫拉着手,杨琰局促不安,但他也不敢将手给拉出来,只能战战兢兢的等着李诫将话说完。不过听到李诫转述韩冈的话,还是不禁开心的笑了起来。

杨琰、杨琪两兄弟,都以木工闻名京中,是有数的木工大匠,擅长修造大型建筑。后来熙宗皇帝赵顼还以杨琰修造有功,将其提拔为殿值,做了武官。

前两年,杨琪被派到了李诫的麾下,辅佐修筑轨道。洛阳到京师的铁路轨道上,已经完工的几条木桥,正是杨琪所规划修建。那几座铁路桥,虽然是木质,但坚固稳定,重载的列车也能够从上面安然通过。现如今,杨琪也追随其兄的脚步,被授予了官职,同时还正在研究如何将木桥,改造成使用年限长久的石桥。

表能工巧匠为官,这件事肯定会一直做下去的。只要有足够的才干,立下足够的功劳,即使是出身卑贱也有机会为官。韩冈的心意,李诫当然明白,他虽是官宦人家出身,可做了那么多年事,绝不可能会去歧视有专长的人才——本身都被歧视着,李诫又怎么可能将之加诸他人身上?

一番介绍停当,再交托了其司掌的一应事务,待到中午时分,便是例行的官宴了。

衙署中的官员们各自入席,而吏员们则纷纷下堂回避,只留下一干服侍的。酒过三巡,他们才会再上来奉酒祝寿。可李诫抬眼看过去,已经有了官身多年的杨琰,却是跟着吏员们一起打算下堂去。

旧为吏人,虽作诸司使副,见旧所服事官,不与同坐。这是官衙中的习惯。即使是杨琰已经做到了提举内中修造所公事,依然不敢与同僚同坐同食。不过李诫却并不打算看着杨琰这么离开,立刻出声叫住了杨琰:“杨提举,请留步,今日官宴,衙中有官身皆当入席,提举何故离开?”

转头又对赵子几道:“三班使臣,理当列席。”

赵子几眉头微皱,一时没有回应。而杨琰,已是连连摇手,连称不敢。而将作监丞也在旁说道,“此乃条贯。”

李诫不以为然,朗声道:“当初令弟授官后,官宴上依然不敢入席。沈学士便说了,一经王命,便是王臣,已非旧时卑贱之身,如何不能于宴?吾亦曾听玉昆相公提起过,当年熙宗皇帝和王安石对此便颇不以为然,古人立贤无方,不闻秦王以五张羊皮而贱视百里奚,也不闻傅说不入殷高之席。太医局的温提举,前次在韩相公家,也照样安然入座的。”

李诫搬出了沈括没什么,回去养老的王安石也没什么,早就入了土的熙宗皇帝同样不打紧,可李诫把韩冈都搬了出来,这就没人敢再多说什么了。

赵子几也是圆滑得很,立刻对服侍左右的小吏道:“还不快给提举布席?!”

一通忙活,杨琰的座位给放在了最下首,真要计较起来,他至少还可以再向前挪几个位置。不过李诫不为己甚,没有再多的要求。

看着杨琰诚惶诚恐的入席并跪坐下来,李诫只觉得真是好累,初上任,都得这么走上一遍,为了能安稳的坐在这里,总少不了先勾心斗角一番。虽然是常例,但总归是让人心神俱疲。

酒宴开始了,席前的一番小波折,很快便被众人抛到了脑后。今天的主角成了敬酒的目标,纷纷上来劝酒祝寿,言谈间,多是拍着胸脯向李诫保证,不管是什么样的情况,他们都会让工学院顺顺当当做下去。

只看他们殷勤的模样,李诫就知道,他们是想要通过自己与宰相攀上交情。想也明白,如果能在韩冈面前露个脸,能会有多少好处。为了哪些好处,这些人肯定是不惜任何代价的。

觥筹交错,李诫小声说,大声笑,一杯接着一杯,与同僚们的交情如飞一般的上涨。

可是他越是喝酒,便越是明白,韩冈最近的注意力暂时不会放在工学院上,而是别的事情,现在献再多殷勤也没有用。

李诫在前日拜会韩冈的过程中,多多少少了解到了一点韩冈最近在关心些什么样的问题。

一个是不日举行的廷推,决定两名晋身两府的新人选,这同样是事关轨道建筑的要事。

做了自己几年顶头上司的沈括,日后多半依然是都大提举轨道工役的差事,但他晋升东府参知政事的任命,最多再有半月就该有喜报了。

李诫不认为这其中会有什么意外,如果做了宰相,还不能让沈括的名字送到御前,那韩冈这几年在朝中就是白费心了。以韩冈的权势,以及太后面前的地位,怎么可能还有人能从中干扰?

沈括或许并非是最佳的选择,他在哪里都不受到待见,也因此才几次在败在廷推上——李诫私下里觉得这是韩冈故意如此,故意败上几次,也免得世人认为他已经能够只手控制朝堂,更免得太后的忌惮,再多的情分也经不起消磨——不过韩冈也不会一直让沈括失败下去,他手中就这么一个合适的人选,以参知政事的身份,都大提举轨道工役,除了沈括之外,肯定没有其他人愿意去做。

另一个就是远在西南的大理之战。

李诫虽不是与军事有关的官员,但进出韩家家门,来往的官员都是能够接触到机密的显贵,更多的消息在京城中也不是秘密。

熊本已经走马上任,黄裳更是成了西南行营的大管家,而领军南下的赵隆,也已经率领四千关西精兵和两千吐蕃骑兵,在时限内抵达西南行营的大本营所在。此外还有神机军的两个指挥,也于同时抵达了前线。

从作战计划上,这将是南征之役的翻版,征发起降顺的西南夷,以数以万计的仆从军来配合主力精锐的进攻。

但从作战方式上,这将是火器的第一次大规模运用,若不是近距离内,没有更强的大国来成为火器的试验场,神机营根本不会走上大理的战场,而西南行营的辎重中,也不会有高达两百门的虎蹲炮和十五门野战炮,以及相应的炮弹和火药,还有各色的炸药。

以官军的威势,想要一举破敌不难,难就难在练兵上,据李诫所知,韩冈最近对西南方面可是关心备至,表面上充满信心,所以毫不介怀,但私下里,每一封军报都要翻看再三,在他的指示下,前线上的要求,政事堂都是百分之一百二十的满足,如果这样还不能赢,熊本、黄裳之辈,可就是愧对了朝廷、太后和韩相公了。

……………………

西南的战火早已点燃。

就在京师的君臣百姓都在期待捷报早传的时候,熊本、黄裳为主的西南行营,都已经离开成都府路好一阵子了。

之前他们在成都多日,等来了西北的精锐主力,又等来了奉诏齐聚的蕃部兵马,更等到了无数辎重,以及备受瞩目的神机营。

待三军齐集,熊本便于岷江畔筑台,歃血誓师。随即数万兵马如山洪泄地,顺着入滇的各条道路,开始了南下的进程。

主力南下十数日,先抵达了距离前线最近的戎州,面前的第一个敌人,不是大理国的军队,而是控制了石门关和五尺道的石门蕃部。

黄裳此时正跟随在熊本的身侧,沿着山谷间的羊肠小道,慢悠悠的前行。

前方数里外便是石门关,赵隆已经先行率主力抵达关下,照常理,他们这两位正副主帅,只需要在后方等着捷报就可以了,但这开头的第一战,两人都不愿意在后面等消息。

“庄蹻入滇,是自黔中郡引兵而进,渡沅水,克且兰,灭夜郎,一直攻打到滇池。也多亏了勉仲你,高家父子,大概都以为我们也会先入黔,再攻滇。”

熊本慢条斯理的说着,半点不为即将开始的战斗而担心。两人的身后,一群武将、幕僚亦步亦趋,更后面一点,还有一群头梳椎髻,衣着各色的蛮人紧紧跟随。

自古入滇的大路就那么几条。两条从成都南下,其中以石门道这条路为主,另一条则自渝州南下,经遵义至黔州,再转向西南。也就是熊本所说的战国时,楚将庄蹻率军入滇的道路。

而最后一条路,则是走广西,过去虽不好走,也很少使用。但自从广南两路平定,这一条入滇的道路,便有越来越多的商人经过,滇马一向是大宋军马的重要补充,这两年,滇马入中原最多的地方,却是在广西左江畔的横山寨,那里是韩冈开辟出来的马市。在邕州,沿着江水上行,最后再一路向西就行了。

广西土兵和禁军都是南方有数的精锐,如今正云集在横山寨处,还有左右江各家洞蛮的配合,摆出了随时入滇的姿态。

而这两年黄裳在黔州一带弄出来的动静很大,声势甚至压倒了成都府这里,大理如果要守,这几条路都必须守住,但士兵调动有主次之分,何处主力,何处偏师,必须事先安排好。一旦三军就位才发现计算错误,再想调动回来,可就没那么容易了,差一步就是万劫不复。

“黄裳倒是觉得高氏父子应该能猜对我军主力要走的道路,毕竟有大帅在此。”

“勉仲你倒是会说。”熊本笑得眯起了眼。

“大帅威名素著,西南各部无不畏服,岂是黄裳能比?听说大帅到了,高智升、高升泰父子,怎么敢不加防备。”

熊本脸上的笑容更加鲜明起来,黄裳的话只是说得好听,但他话中表明的态度才是最关键的。

熊本与黄裳之间没有什么好争的,地位、年龄都有差距,而黄裳更是知道分寸的一个人,两人之前已经有过交流,彼此之间印象都不差,现如今相互配合,更是相得益彰。

一道平路已经走过,走上一道颇为陡峭的台阶,熊本拄着手杖,边走边道:“兵者,诡道也。但更重要是实力。猜对也好,猜错也好,如今官军三路齐发,不管哪一路,都要大理国全神应付。”

黔州那边,有一部兵马。虽然是偏师,但实力并不算弱,是这几年在黄裳麾下,以各部蛮军历练出来的强兵。

广西的兵马更是调了李信去亲领,他虽然只带了一个指挥的神机营南下,但李信在广西多年,威望素著,由他指挥大军,是如臂使指。

至于熊本、赵隆亲领的主力,则是从成都府南下,沿着岷江河谷,途径因盐而兴的富顺监,过戎州南下石门,走在秦人所开的五尺道上,只要拿下了石门关,通向大理的大门便由此中开。

石门关的道路,秦时修的五尺道只剩路基,之后汉晋重修,名为僰道。道边山崖上有悬棺,传说是僰人安葬之所。之后唐伐南诏,又将已经破损的旧路重修了一遍,到了近年,因为贸易繁盛,不仅大宋这边修路,大理和各条道路上的沿途蕃部,几乎都将道路重修。

只是这些道路,都是在群山中蜿蜒曲折,修得最好的,也不过是让人行走,马能过、车不能过。攻打大理的难度,也就在这些险道上,而不是大理国军队的反击。

赵隆已经做好了作战的准备,只是在等着熊本、黄裳的到来。

自戎州开始,南下石门关的五尺道仅容二人并肩,石门关更是险峻。关前百步,便是一路台阶上行,关墙虽不高,但这一路上坡,着甲的士兵冲到墙下,基本上都要累得半死,更不用说云梯等攻城器械全都无法使用。而想要用蚁附攻城的战术,只要看一看关前仅有五六尺宽的道路,便知道会有多难,不管手上有多少兵马,能够在同一时间上阵攀城的士兵,最多也不会超过五个。剩下的士兵,只能用弩弓仰射城头,而这样的射击,也因为山道的蜿蜒和崎岖,只能容纳百多人施展。

这样的情况下,只要城寨中的守军有足够的信心,以及足够的物资,完全不用担心有人能够攻破。

之前当赵隆亲眼看过石门关前的地形后,也推演了一下,如果让自己来守的话,基本上是粮食能吃多久,这里就能守多久,山上有泉水,至于守城的物资,这山里,石头从来都不缺。

自然,这是不用神机营上阵的情况。

或许这是入滇的第一道难关,但现在赵隆的手中,有着足够的手段,来应付这种万夫难克的险关

一旦突破石门关,攻取大理的战争才到了正题上,不论是为了功劳,还是为了之后战事的顺利,赵隆都有必要用最小的代价拿下这座险关。

熊本、黄裳虽是缓步而行,可也没用太久,便来到了赵隆一处缓坡处设置的临时营地。

山上道路艰险,却又清泉淙淙,更有飞瀑自悬崖而下,在石壁的凸起处,几跌几撞,最后落到了路边的水潭中。小小的水潭只有一丈方圆,聚起的山泉水清澈见底,几匹战马正在池畔饮水,牵马的士兵原本懒洋洋的在旁坐着,看见熊本、黄裳一行而来,连忙跳起来行礼。

赵隆问询匆匆赶来,熊本没有浪费时间寒暄,直接就问:“本帅看你飞船都没有放上去,关中的情况探明白了吗?”

“山间风大不适合飞船,末将便派了人,爬上山壁去探查。”赵隆说着,抬手指着一旁的山上。
黄裳拿起望远镜,顺着赵隆的手指望过去,登时在山壁上发现了好几个身穿红衣的身影。

“贼人没有在山头上防备?”黄裳惊讶的问道。几个斥候太显眼了,如果山头上有敌人,丢下几块石头就能清光他们。

“没有。”赵隆摇头,“贼人全都缩在关门后。”

平地里交战,飞船总是飞得很高,只是这一次,在山谷中烈风劲吹,气球不能上天。但道旁山壁高耸,赵隆早选了军中善于攀援的健儿,让其爬上去观察关中。而且在派人的时候,千叮咛万嘱咐,让他们小心贼人在山头上设下的据点。只要是有点头脑的将领,肯定会设法事先占据山顶的有利地形,监视敌军,包围自身。不过赵隆的交代白费了,山头上根本就没有敌人。

赵隆并不惊讶,他当年随王中正南下平乱,遇到的也是这样的对手。士卒有勇气,敢拼敢杀,但领军的酋首却太无能了。即使能用些战术,也是可笑得紧。

“这样的敌人,就算不用火器,仅只是夜袭,末将也照样能破敌。”赵隆自负的说着,他对此有着充分的信心。

熊本摇摇头,“赵子渐你能这么说,都是靠了在关西、在河东用人命换来的经验。这群蕃人,哪里有这样的机会?官军过来时躲到山里,官军离开后再回来,这才是他们该用的战法,想要据险而守,他们还要多学几年。”

赵隆唯唯,点头称是。

黄裳把玩着望远镜,道:“石门关城狭窄,周围甚至不及百步。贼军的主力当是驻扎在关后。”

“正是。末将也这么想。”赵隆点头。

黄裳道:“一夫当关、万夫莫开的地势,也要不了多少人来驻守。”

“不过山中有小道,一时查探不清,贼人熟悉地理,有可能绕道我军的背后。”赵隆又补充道。

黄裳随即道:“马湖部和南广部的人都来了,他们都是石门蕃的成员,中间应该熟悉山中道路的人。”

“不然。”熊本摇头,“百里石门道,在乌蒙部控制下已有数百年,其他两家决没有乌蒙部那般熟悉。”

石门蕃部以三家为主,西北的马湖部,居于岷江支流马湖江左右,东北的南广部,在南广河附近聚居,剩下的一家乌蒙部,人口最多,土地最广,为石门诸部共主,据传始祖乌蒙自蜀汉时便来到此地定居,从此繁衍生息,至今几近千年之久。

“赵隆你说怎么办?”黄裳问道,他确信赵隆肯定有了主意。

“以末将来看,当尽快攻下关城,让贼人的伎俩没有施展的余地。只要石门关城一破,石门蕃便再难顽抗官军。”

赵隆充满信心。

这一次官军南征大理,石门蕃部中的乌蒙部不肯降顺,遂退守石门关,等待来自大理的援军。

这两年官军没有少敲打西南夷各部,水西罗氏鬼国给打得分崩离析,戎州、茂州叛乱的几个部族,更是被屠了个干干净净,乌蒙部不信熊本的话也是正常的。假途灭虢的典故,即是蕃人没听过,聪明人也会想到官军会不会这么玩上一手。但朝廷要惩治大理篡国的奸臣,想做拦路石,也得做好被碾碎的准备。

熊本和黄裳各自点了点头,黄裳对熊本笑道,“裳曾闻石门关下,五尺道旁,有唐大夫袁滋奉旨出使南诏时留下的墨宝,不知现在还留存了没有?。”

“是贞元九年的那一次吧。”熊本博闻强记,立刻就想到了黄裳在说什么?“那副摩崖就在石门关下的道路旁,袁德深以书法名世,碑文若是拓印下来,拿回京中,不知会有多少人争抢。”

黄裳连连点头,而熊本却突然一声断喝,“赵隆。”

熊本冷不丁的一声叫,赵隆立刻抱拳躬身,“末将在!”

熊本冷下脸,喝问道:“你想让老夫等到什么时候?”

“末将麾下将兵,早已准备停当,只等大帅之命。”

“打得好看一点!”熊本淡然吩咐道:“五百里外的育井监山前后长宁等十郡八姓都来了,近处的水西诸蕃,更是一个不拉,更有同属石门蕃的南广部和马湖部,不让他们好好看一看皇宋天威,这尾巴就又要翘起来了!”

方才熊本与黄裳一路说笑,有一半是要给后面的蛮部洞主、鬼主们看看,眼前的险关只是个抬脚能过的门槛而已。但赵隆如果没打好,之前的一番表演,可就要沦为笑柄了,而且是蛮人的。

“末将明白!”

随着赵隆走上前线,一声声号角响彻云霄,山道上的军势立刻活跃了起来。

一队队官军整装待发,士兵们检查着自己身上的装备,是否结束整齐,是否有所遗漏,而将校们更是一个个检查过去,严防有人疏漏。

号角声刚落,鼓声立刻紧接上,重鼓敲击后的一重重回音,响彻在山谷间。

就要进攻了。
熟悉官军攻击节奏的熊本和黄裳同时想到,也同时赞叹起赵隆治军的手段。

从告知麾下各军即将投入战斗,到正式攻击,只用了小半刻的时间。

后面的蕃部洞主、鬼主,一时惊骇莫名,就这样便进攻了?官军气势汹汹,看起来当真是想尽快攻下官城。

“或许不用太久。”黄裳低声说道

也的确没有让黄裳等待太久,只过了一刻钟,一声比惊雷还要响亮,比还要震撼人心的爆鸣,猝然响起,然后在山谷中不断回荡,一蓬蓬碎石扑扑簌簌的从山壁上落下,惊得道上的人马一阵乱躲。

一块人头大小的石头翻滚而下,砸中了一匹挽马,直接击中头部的重击,让挽马连惨嘶也没有,便随着落石摔落到了官道旁的深渊中。

黄裳此时心中一动,回头望去,各部鬼主、洞主全都惊白了脸,咬着手指,这一声,并不能说出乎意料,但这一击的威力当真是太大了。

熊本不顾落石,哈哈大笑,“赵隆这杀才,也太卖力了点!”

“报!!!”

一声拖长了的叫声,随着一名身背小旗的小校疾奔到了熊本的面前。


前方已是千军齐呼,一时间小校的禀报声完全给遮住了,隔着数里地,亦能分辨得出呼声中的兴奋。不是攻下了石门关,又会是什么原因!?

这才多一会儿啊,赵隆刚刚领命开赴前线,转眼就把石门关拿下来了。石门关有多险要,各部的成员都是看见过的,但如此坚固的堡垒,竟然转眼之间便被官军拿下,这样想来,此处各部,有哪个能守住自己的老巢?

各自的心思千折百转,方才刚刚受命上前的赵隆,此事又转了回来。

赵隆颇有几分后悔,他事前对炸药爆破还是没有太多信心,否则完全可以早点开始解决。

刚刚走到近前,便听到熊本的一声喝问:“石门关拿下了吗?”

赵隆重重的一抱拳:“禀大帅,石门关已经被官军拿下!乌蒙部残寇逃窜,末将已经安排人手追击下去,不给他们喘息之机。”

“好!”

“好!!”

“好!!!”

熊本交好声,一声比一声高,“自古攻城拔寨,未有如此快者,赵子渐你这一回,可是破天荒的第一快!”

赵隆倒是喜色不多,叹道:“非是末将的功劳,乃是火药之威。”

“哦,是吗?”熊本笑了一声,转头对黄裳道:“我们上去看看吧。”

一行人随即拾阶而上,转了两道弯,石门关的关城便出现在众人的眼前,但已经没人能认出来了。

碎裂的墙体,仍有袅袅余烟,城墙上的敌楼则不见了踪迹。整座石门关,前半段都成了废墟,而守在城中的乌蒙部的蛮军,泰半死在了瓦砾堆中。

关门前的道路,只有靠山的一半还残留着,另一半随着碎石一起坍塌了下去。残存的道路仅容一人行走,若不是火药炸得城中一片死伤,想要拿下石门关,还得费上一番功夫。

一群蛮人目瞪口呆,望向赵隆的眼神中满是畏惧。马湖、南广两部的鬼主反应最激烈,竟是全都跪了下来,嘴里念念叨叨的不知在说些什么。

熊本望着残迹,也没了之前的沉稳,呆然道:“火药之威,一至于斯。”

黄裳知道一点,赵隆用来炸毁关门的炸药,不完全是硫硝混合的黑火药,还有别的东西,运过来颇费了番功夫,本来只准备炸个城门,却没想到连城墙都没了。

一群士兵在瓦砾中搜寻着敌军的尸体,三名将帅带着幕僚,走在关城的遗迹上,等着出去追击逃敌的大军的回音。

日头一点点西斜,夕阳的余晖染红了西面的天空,也染红了远近的群山,

“或许能回京过年了。”熊本站在关城南面的城墙上,叹息着,有了火药为助力,这一场战争,恐怕会结束得很快。

“或许当真能如大帅所料。”

“年纪大了,都不想动了。”熊本感慨着,“几年前老夫奉旨出陇西,听到有人唱‘年年柳色,灞陵伤别’,老夫还笑其看不开。而今,倒是想听听有人唱此曲。”

“是李太白的那首《忆秦娥》?”赵隆突然问道,

黄裳惊讶起来:“不意赵子渐你还懂一点诗词。”

“只是稍知一二。”

跟在熊本身后的一名幕僚忽然引颈高歌,音声苍苍,曲调悠长:“箫声咽,秦娥梦断秦楼月。秦楼月,年年柳色,灞陵伤别。乐游原上清秋节,咸阳古道音尘绝。音尘绝,西风残照,汉家陵阙。”

赵隆皱了皱眉,他不喜这样的曲子词,让人心中平添几分悲凉。战场上,应该是更雄壮威武的曲调的天下。也就在这时候,一曲用着同样的调子却更为激越的《忆秦娥》,从前方的士兵中传来:

“西风烈,长空雁叫霜晨月。霜晨月,马蹄声碎,喇叭声咽。雄关漫道真如铁,而今迈步从头越。从头越,苍山如海,残阳如血!”
