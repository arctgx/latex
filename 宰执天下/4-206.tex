\section{第33章 物外自闲人自忙(二)}

“司马君实司职西京御史台,玉昆你去拜访他恐怕不太好吧?”程颢犹疑着。司马光的身份不一样。

韩冈笑得平和,对程颢、程颐解释道,“司马君实司掌西京御史台,学生身为监司,上门拜会本来是有些不妥当。不过……他毕竟是司马君实,学生既然身为前相之婿,前去拜会,当不虞被人误会。”

他需要去见文彦博,他也必须去拜会富弼,还有范镇等一干身在洛阳的致仕老臣。这些元老,不论韩冈想见或不想见,依礼数他都该去拜会。

先来见二程,只是因为程颢对他有半师之谊,放在第一位,不会让一干致仕老臣认为韩冈失礼。可是若是他始终不去拜见那些老臣,京城里面的皇帝,都要以为韩冈崖岸自高、不会做人了。

唯独司马光,却是韩冈不需要见,且因其司掌西京御史台,也不该特意去拜见,但他却想见上一面的。

倒不是因为来自于后世的记忆。那些记忆之中,有关司马光的,除了《资治通鉴》就只剩砸缸的故事了。

而是这些年来,韩冈通过各种各样的渠道,对司马光有所了解后,因此而产生的兴趣。他想登门去瞧一瞧,看看司马光到底是何样的人物。

自家的岳父对韩琦、富弼、文彦博都不是很看得上眼,唯独对司马光,却是看得极重。

王安石的那封《答司马谏议书》,可谓是变法的宣言和号角。

‘受命于人主,议法度而修之于朝廷,以授之于有司,不为侵官;

举先王之政,以兴利除弊,不为生事;

为天下理财,不为征利;

辟邪说,难壬人,不为拒谏。’

几个排比句如同床子弩射出的一枪三剑箭,一记一记的扎向旧党的心窝。

这短短几百字的文章,王安石将他超绝于世的文采挥洒得淋漓尽致,韩冈至今都能背下全篇。在正文中的最后一段‘如君实责我以在位久,未能助上大有为,以膏泽斯民,则安石知罪矣;如曰今日当一切不事事,守前所为而已,则非安石之所敢知。’此等煌煌雄辩之言,尤其让韩冈激赏不已。后来他受到监安上门的郑侠弹劾,上殿自辩时,也顺便借鉴了一下。

但一个巴掌拍不响,王安石能写出这一篇佳作,全是靠了司马光几封书信的刺激,韩琦、富弼和文彦博可都没有一个能做到。

而且王安石还说司马光是反变法派的赤帜,当时文彦博可就在枢密院中,担任着枢密使。对新法反对最为激烈的文彦博,都已经喊出了‘为与士大夫治天下,非与百姓治天下’,但在王安石眼中,依然不是赤帜。可当天子要任司马光为枢密副使时,便就是为异论立赤帜。王安石对司马光的看重,由此可见一斑。

不过韩冈觉得,司马光应该不喜欢王安石的看重。

他是想要有所作为的官员,距离宰执曾经只有一步之遥,世人也都视其为宰相之才。正常来说,五十到六十岁,应该是一名官员站在一生最高点的时候,王安石便是如此。吴充、冯京、王珪也无不是如此。可司马光却因为政见相异的关系,却硬是被王安石逼得在洛阳写书近十年。

看见曾经的好友执掌一国大政,成为能在天下郡国呼风唤雨的人物,司马光在家里挖个地洞进去写书的心情,韩冈也能体会得一二。

当初富弼初回洛阳,曾问邵雍近日洛阳城中有何新奇之事,邵雍回答说,有一巢居者,有一穴处者。前任执政王拱辰在自家中修了三层高的中堂,而司马光则是在独乐园挖了个地窖去写书,所以一个叫巢居,一个叫穴处。富弼在大笑之余,心里还不知怎么翻腾了。

换作是他韩冈,要么就是将恨意积蓄在心底,或者就是心灰意冷,从此以山野为念。但从韩冈听说的司马光的近况中,可是半点也不像是心灰意冷的样子——虽然司马光应该是君子,而韩冈不认为自己是君子,但人性应该是共通的,韩冈并不觉得司马光的想法会与自己太大的差别。

所以韩冈对司马光很有些兴趣,想面对面的了解一下司马家的另一位史学大家。

韩冈对司马光的态度让程颢、程颐有点纳闷,怎么也不可能想得到韩冈他仅仅是好奇的缘故。

不过以韩冈为人、心性和才智,两人也不觉得他会做出什么样蠢事来。独乐园也不是龙潭虎穴,韩冈拜访一下司马光当也不会有什么大事。

午后的一席谈,并没有讨论什么经义要旨,多是韩冈在说他去了岭南的一些见闻,还有在交州施政方略。程颢、程颐仔细聆听,并不时询问详情。

听说了章惇和韩冈在河内寨交趾旧王宫主殿的遗址上标铜立柱,两人还没有什么反应,但听到夺下交州的第一年粮食就能够自给自足,程颢、程颐却开始为韩冈的治事之材而感到惊叹。不过韩冈立刻就解释道,这不算是他的功劳,而是交趾水土好,水稻生长快速,一年两熟一年三熟都是很平常的事。

韩冈也顺便问了一下几名留在洛阳的同门的现状,没想到吕大临现在去了嵩阳书院。嵩阳书院在登封,离着洛阳稍微远了一点,程颐程颢本来也是在嵩阳书院授徒,只是每个月都会返回洛阳城省亲。韩冈也是到了巧了,迟上数日,就只能看到程珦和程家的孙子辈了。

到了傍晚的时候,韩冈被留了下来,程家为其设了家宴款待。

韩冈与程家是通家之好,家里的女眷也不避他。家宴上,韩冈见到了程颢和程颐的夫人,还有程家的几个女儿,也包括韩冈很早就见过的排行二十九的程鄂娘。

看到她,韩冈都愣了一下,惊讶的望望程颢,打算说什么,但想想又闭上了嘴,只是与女大十八变的程鄂娘见了礼。但心中很是有些疑惑,程鄂娘都已经十八九了,怎么还没嫁人?虽然他的夫人王旖嫁过来的时候更迟,但那是各种因素引起的特例。

不过些许疑惑,很快就被程家平和的家宴气氛给冲淡,韩冈是在得官之前便与程家来往,现在身份地位的差别算不上一回事,说起话来也是如同自家人一般亲近。

在宴席上,程珦的兴致很好,还念了他在同甲会上做的诗句,“藏拙归来已十年,身心世事不相关。洛阳山水寻须遍,更有何人似我闲。”

韩冈为着这首诗里从心所欲不逾矩的悠闲自在向程珦敬酒,程珦老怀大慰,满满喝了一杯,接下来就被程颢、程颐给劝住了。

程珦算是从仕途上解脱了出来,诗中的悠然自得也是透纸而来。不过这首诗与精丽繁缛的西昆体或是雄豪奇峭的险怪体都不一样,很是平实,而且还不是王安石那样平淡中隐现峰峦叠翠的平实,只是大白话而已,水平当真不能算高。说起来,韩冈经过了这么多年时代风气的熏陶和浸淫,费些脑筋,眼下也能做出水平差不多的。

吃过了饭,看看天色已晚,韩冈遂起身告辞。

送了父亲入房休息,等儿子也送了韩冈回来,程颢、程颐来到书房,点亮油灯,在灯下回忆今天韩冈说的话语。

今天都不想因为经义大道执之争而闹得不开心,所以他们和韩冈都尽量不提及这方面的话题。但韩冈还是透露了一些他现在的想法。

“经世济用。”程颢回味着韩冈今天说的一番话,“从还在熙河路开始,玉昆就是在讲究着事功。有几分胡安定【胡瑗】设治事斋的味道。经世济用四个字正好概括了。”

“要不是有着这份志向,也不能说出为万世开太平的话语。玉昆的心性,远比那一干小人争权夺利要好。”程颐不掩对韩冈的欣赏,“玉昆做事也有分寸,从来都是以实事为上,没听说他掺和那等腌臜之事,要是他想靠着新党幸进,当年就会去兼了中书都检正一职了。”

韩冈在世人看来一直算是新党核心成员,王安石的女婿这个身份就不用说了,这几年来他多少次帮着稳定了新党的根基,一系列的功绩也是在新党秉政后拿得出手的成果中,占了很大的比例。

但在程颢和程颐眼里,韩冈却不能算是新党的中坚人物,只能算是若即若离的边缘。

韩冈一直以来都坚持着关学,总是想方设法的将张载举荐入朝,在经义局中为关学争夺一席之地,他在道统之争上,从来都没有向王安石退让过半步。比起韩冈这些年来所立下的功绩,他在学术上的倾向,在二程看来才是确定他政治坐标的关键。

“与叔过两天就要从嵩阳书院回来了。”程颢忽而问道,“也不知道子厚表叔的行状写得怎么样了,草稿差不多也该定下来了。”

“前几天从书院回来,只看到一个开头,下面的草稿改得很多,就没细看了。估计还要费些时间。”

“玉昆虽然没有明说,但他估计也是急着看呢。”程颢长声喟叹,“子厚表叔好福气啊。”

