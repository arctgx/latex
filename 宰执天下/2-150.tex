\section{第29章 顿尘回首望天阙(二)}

【这个应是昨天的第三更,现在补上。】

韩冈听见章惇发问,却也不便把自己的真实理由说出来,想了想,只好随便找个理由搪塞一下。

“凡事分阴阳,阴阳皆否,内外皆困,便无一事可成。如韩相公统军攻横山。昨日在王相公府中所言诸事皆为外因,至于内因,则是韩相公御下不正,大损军心士气!其中尤以环庆一路为甚!”

章惇脸色一变,沉声追问:“这话怎么说?!”

韩冈便把他经过关中时的一番见闻,还有汉番两军之间的险恶关系,都一五一十的说了出来,“韩相公不能秉公而断,让军心怨艾沸腾。天时地利人和,这其中有哪一条韩相公能对西贼占上风?”

“吴逵?……广锐军的……”章惇仰头想了一阵,对韩冈道,“玉昆你所说邠宁广锐军都虞侯吴逵,在前两天宣抚司送来的急报中,已经被下狱收监了。”

“怎么会?!”韩冈大吃一惊,他瞪大眼睛,“前日过长安京兆府时,下官尚与其同路,那时尚且安好。怎么下官才上京,这吴逵下狱收监的公文就已经到了?!”

“陕西宣抚司的公文,全都是走得急脚递。日以继夜,千里一日而过,从京兆府至东京,不过一千多里地,一两天就能走完,可比玉昆你一程程的乘驿马走上十几天要快得多。”章惇起身,从摆在桌案旁的架阁上翻出了一份公文来。打开来看了一眼,低声冷笑:“果然就是这一份!”再看看写在公文最后的标识,“看时间,是五天前的事了。”

他转回来,把手上的公文递给韩冈。韩冈连忙翻阅着这份前线急报,越看越是觉得火大。上面说,吴逵曾与王文谅同出寨,共击一贼。但接战时,连呼吴逵不至。并说吴逵‘扇摇军士’,谋图不轨。因此将吴逵下狱。这其中每一条罪名,都要治吴逵于死地。

“王文谅这蕃人,分明是挟怨报复。”对急报中罗列的罪名,韩冈决计不信。若是真有其事,当日在道左客栈中,两边争执起来的时候,王文谅怎么不说出来?

章惇这时从脑海中搜索着记忆,王文谅这个名字,有好几次出现在他的眼前过,“关于王文谅与从官争执,尚记得好像还有一个赵馀庆,是个蕃官……”

韩冈点点头,他也是记得:“就是被王文谅说成是约期不至,以失期的罪名下狱的赵馀庆?”

“对!”章惇一拍桌案,他终于全想了起来,“官家当时曾亲下手敇,诏释这名蕃将,让他戴罪立功。但韩子华却还递了好几本奏章回来,说是要严加处置,以正军法。不过因为官家的坚持,所以最后赵馀庆还是被放了。这件事里,延州、开封之间文字往来好几次,因而我还记得。”

韩冈摇头叹息,“王文谅仗着韩相公对他的信任,恣意妄为。赵馀庆之事,已经难以查清真相。但王文谅与吴逵不合,以至于差点大打出手,在下是亲眼看到的。想不到以韩相公之智,也不免被王文谅这蕃人所蒙骗。想那吴逵在广锐军中威望甚高,所以他才会给吴逵加上一个‘扇摇军士’的罪名。”

章惇很清楚朝廷对武人的顾忌和偏见,“如果这一条坐实,吴逵当会被一正军法了。”

“本来就是子虚乌有之事,但吴逵在广锐军中威望甚高,说不定会弄假成真……”

章惇沉吟起来。他现在已经开始支持杀吴逵了,至少不能让他继续留在环庆。这样威望甚高的将校,又受到了不公正的待遇,一旦有了反心,就会很危险——过去多少兵变都是由此而来,由不得章惇不担心。

不过,最终他还是轻轻地摇了摇头,这不是他能干涉的事。

韩冈看破了章惇的想法,他问道:“关于吴逵和王文谅之间的纠葛,检正还有王相公应该不会跟韩相公提吧?”

章惇笑了一声,却不回话。都心知肚明的事,就没必要说得太清楚了。韩绛在外领兵,王安石只会全力支持,却绝不会插手其中。别说吴逵的一点冤屈,即便韩绛本身有什么问题,在即将展开的大战之前,都是不值一提的小事。

韩冈也清楚这一点,暗暗叹息,“想不到还是得去延州。”

章惇则让韩冈放宽心:“玉昆你可以放心的去延州。如果今次战事真的一如你事前所料,最后是损兵折将劳而无功,王相公必然会代玉昆你在天子面前分说明白,绝不至于降罪于你。”

王安石的人品,韩冈还是信任的。有王安石在宫中为自己缓颊,就算韩绛大败而归,对自己来说结果还是好的。但若是韩绛得胜而归,那他可就要丢脸了——王安石或是韩绛不会真的一点功劳都不给他,可如同丢下来的骨头一般的功赏,比起责罚更让人难以接受。

也幸亏韩冈对于自己的判断,有着决不动摇的信心,才能微笑着向章惇表示感谢。不过他还是有些无奈,他今次来中书,可不是为了聊天的。

章惇像是看透了韩冈的想法,笑道,“王相公不到午时不会从宫里回来,就算回来,事情也不会少,你的事也不会有空处置。冯当世那边,玉昆你也不必去见,他好像一直都不喜欢你。直接就在这里帮你把召令给缴了。……还有,玉昆你既然不想跟韩子华那边有瓜葛,我会帮你再劝一下王相公。将你去延州的职司改为临时的差遣,原本在秦凤的职位都不会变动。这样玉昆你应该可以放心了吧?”

章惇这也算是为韩冈尽心尽力着想了,不管他实际上是有什么打算,但从受到帮助的方面来说,都是值得感谢的事。韩冈遂重新起身,向章惇郑重行礼道谢。

章惇很看好韩冈,难得的经世济用的人才,文韬武略皆有所长,而非是只懂得说嘴的清谈之士。章惇对前途有着自己的一份考量,光是跟在王安石身后按部就班的晋升,满足不了他。而在他的计划中,韩冈可是一个很重要的助力。

一番深谈之后,又订下了晚间的樊楼之约。原本章惇是要让路明去请韩冈,谁想到韩冈今早就送上门来,便也一并说了。

韩冈被章惇送了出来,而且是一直送到了院门外。见着章惇下了门前石阶,与韩冈殷殷告别,周围中书门下的官吏们都吓了一跳。

在一般人眼里,章惇这位检正中书五房公事,素来自负才高,都是倨傲无比,极少看得起人。能让他出门相送,一个月也不一定能有一个。

“那个究竟是谁啊……”

“傻了吧,这都不知道。天子昨夜要见,被冯大参堵回去的那位。”

“天子要见?!难怪章检正这么看重他。”

“到底他立了什么功劳,让天子都要赶着在夜里传谕?”

“不知道前些日子上京来的那群蕃人吗?都是他帮着王韶给捉来的。”

周围一片窃窃私语,章惇视线横扫了过去,脸色微沉。显然对这些紧咬耳朵根子却不去做事的胥吏们有些恼火。这群胥吏都是在中书门下混迹多年,论起察言观色的本事,比起一般的官员都要精深许多。被章惇一瞪,情知不妙,便立刻卷堂大散,转眼周围就不见人迹。

“检正果然御下有方。”韩冈不禁赞了一句。

“还是多亏了玉昆你,加俸一议,让这等小人都转而拥护新法,使唤起来也顺手了许多。否则就算上面推行,底下人给你做手脚,照样什么事都做不成!”

“并非在下之功。动嘴容易,动手才叫难。在下只是说了一句话而已,真正让新法得以推行,让衙中胥吏俯首帖耳,当是靠着王相公和检正的一番心血。”

章惇笑了一笑,不再多言,与韩冈拱手告别。韩冈在章惇招来的一名胥吏的引领下,沿着刚才进来的路,向外走去。

走上繁忙的廊道,韩冈回想着方才的一席话,其中章惇示好之意溢于言表。在韩冈看来,光是一个父亲的救命之恩,不足以让他如此殷勤——刘仲武也是救了章俞的一人,而且是主力,但现在他却还在偏僻深山中的者达堡内数星星呢!今次也不见章惇提起他。

即是如此,那就是章惇有用的到自己的地方了。作为一枚棋子,有被人争抢利用的资格,也算是值得欣慰。越是重要的棋子,其位置就越是牢固。王韶、韩绛、王安石,还有现在的章惇,都看重自己的才能,韩冈至少不用担心他会被人当作弃子。

不过韩冈还是喜欢做棋手。在古渭,韩冈虽然地位不比王韶、高遵裕,也算是棋手中的一员,不过到了京城,就只是一枚棋子。一边做棋子,一边则也是棋手,两边的身份并不矛盾。前次韩冈来京城,就出手帮着王安石下了几步,今次局面虽已与前次有别,但他也照样能做出一番事来。

韩冈微笑着,和煦如春的笑容中,看不到半点他心中的阴寒。韩绛既然一个劲要他过去,那就去延州亲眼见证一下,见证自己的预言究竟是如何得到实现!

