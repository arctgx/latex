\section{第13章 晨奎错落天日近(24)}

韩绛出面支持韩冈,这让原本新党稳拿稳的胜利,变得不那么确定起来。

不过李定在短暂的震惊之后,还是恢复了平静。

或许韩绛能让一些反复无常之辈改弦更张,但韩冈过去的支持者,却肯定有很大一部分不会再投他一票——就是来自于那些丧家犬的一部分。

韩冈在廷推之后,将支持他的旧党众人,一个接一个的打发出去,这种过河拆桥的做法,不可能不引起旧党的怨愤。

旧党之所以会选择韩冈,并不是说他们能认同韩冈的理念,只不过是怨恨王安石罢了。任何能够给王安石造成麻烦的人和事,都会让那群丧家犬一同狺狺做声,然后幸灾乐祸看新党手忙脚乱的样子。

更重要的是他们也希望根基不稳的韩冈掌握权柄之后,能够为了巩固根基而启用旧党。

但韩冈的背信弃义让旧党彻底放弃了那点奢望——韩冈进入两府之后,根本就没有去与新党争权夺利,而是只顾着自己的那一亩三分地,以他的能力和手中的人才足以照顾得来,也就不需要引入更多的支持者——也从此对韩冈不抱有任何指望。

不管今日韩绛彻底站在韩冈一方,让韩冈多了几张选票,一旦减去旧党的票数,也就与之前韩冈进入两府的那一次廷推所得到的票数相当,依然改变不了大局。

李定摇了摇头,有太后和韩绛在,事情还是会有些波折。

在韩绛的站位大幅度改变投票结果之后,不仅仅意味着有些人打算投机,也意味着走中间路线的可能性变大了。

恐怕当有好些人会选择中立。毕竟他们除了王安石和韩冈之外,还有一个选择,那就是弃权。

本来决定国是的诱惑,让一干重臣不会放弃投票的权力,至少在这第一次会议中,不会放弃。

但现在形势过于险恶,不论是站在那一边,都是要将自己的未来给赌上去。

许多新党成员,只不过是一些见风使舵之辈,遇上现在的环境,不必奢望他们能够坚定信念。

李定仔细计算着新党一方现在能够获得的支持数,一个、两个,他盯着一张张熟悉的面孔,估计着他们立场。最后放心下来,依然可以占据优势,确保胜利。

但这份安心只持续了很短的时间,韩冈的发言让他一下陷入了混乱。

‘继续变法!?’

就在李定刚刚为韩冈与己方巨大的得票率差距安心的时候,韩冈的发言石破惊天般的传入耳中。双手一抖,手中的笏板差点就这么掉到地上。

李定简直不能相信自己的耳朵。

韩冈的大胆,完全出乎他的想象。

这是要从王安石的手上,抢过变法的大旗。

韩冈之前口口声声说要,现在就是要继续变法,当然,如何变,就要看他韩玉昆的喜好。

一股怒火从心头腾起,然后很快消失,现在已经不是发怒的时候了。

不说他事,只说旧党。就是他们再对韩冈如何恼火,可看见新党都要被人鸠占鹊巢,旧党中人定会毫不犹豫的支持韩冈,然后回到家里,开始哈哈大笑,直到笑疯掉或是笑得喘不过气来憋死自己为止。

而新党……

李定眉头紧紧聚拢了起来,比起单纯反感新党而聚合而成的旧党,新党成员的心思复杂十倍,他们现在的想法,在遇上任何人都无法预料到的情况下,根本无法去揣测。

…………………………

尽管感受到了那么一瞬间惊怒交加的视线,不过接下来,韩冈并没有在王安石的一张黑脸上找到太多表情。但他也只是扫上一眼,没有仔细的观察——王安石的心情,现在并不在韩冈的关注范围之内。

韩冈要改变国是,以气学为纲,将新学击败,将新党请出朝堂,本质上,是要以新兴势力,赶走旧的既得利益者,

而所谓变法,到头来同样也是利益分配的改变。

本质上是一样的。

但同样的本质,换上一种说法,却能给人以不同的感觉。

过去十几年的变法,旧党损失最大,也叫唤得最凶。或者说,损失最大又没能及时在变法中攫取新的利益的那一批人,组成了旧党。

而现在韩冈喊起了继续变法,有了过去的经验,韩冈相信,现在能立足在这座殿堂中的朝臣,大多都能明白,机会又来了。

一旦他得以成功,许多关键性的位置,将会迎来新的主人,然后在未来的很长时间中,掌握着朝堂。

这样的诱惑,距离两府只在数步之间的朝臣们,有多少能忍得住?

……………………

继续变法?

在蒲宗孟的眼中,丢出这个惊雷般的言论之后,韩冈仍是怡然自若。

蒲宗孟彻底放心了,之前的赌博,算是给他押中了。

在韩绛做出决定之后,结果变得难以预期起来。到底选谁更好?

握有选票三十二人中,只有少部分毫不动摇的坚持自己的选择,更多的,则陷入了迷茫。包括曾经支持过新党的王存、杨汲,也包括投靠韩冈的蒲宗孟。

选择了投票的目标,也就等于多了一个死敌。这是要拿自己未来的地位,去冒风险。

蒲宗孟之前只是抱着赌一把的心情去做,可他现在已经完全放心了。

韩冈能不能做得更好?从韩冈之前的表现来看,当然不用担心。

但拥有投票权的重臣们,他们做出选择的时候,绝不是抱着忧国忧民的想法——

——韩冈的提议能给他们更多的利益吗?

蒲宗孟确信,韩冈能够做到。

新党此时地位已经稳固,而新法行之有年,过去旧党想要维持的按部就班、论资排辈的晋身之序,现在已经重现在新党之中。变法之初,‘新近’频出,像吕惠卿三五年身登两府,蔡确六载京朝而至宰相,现在根本不可能做到了。

如王韶、章惇、韩冈那样依靠积攒军功而晋升两府,对绝大多数朝臣来说完全是天方夜谭。没有上面的提携,没有足够的空缺,怎么可能走进两府?

韩冈现在根基不深,手中乏人,这是劣势,也是优势,想要最好机会。否则有章惇、吕惠卿、吕嘉问、李定、曾孝宽等人在,其他人怎么跟他们争?

韩冈三十出头不假,可韩绛、苏颂,乃至张璪,年纪都不小了,等他们的空缺,比起与吕惠卿、李定等人竞争,可是要简单上数倍。

……………………

曾孝宽本来准备跟在韩冈之后发言。

他与王安石、章惇、李定等新党重臣商议过后,也总结出了一份提案,交由曾孝宽在今日的殿上提出来。

因为韩冈的提案,肯定是在军事上坚持以守御为主,维持与辽国的和平,同时在国内进行大规模的建设,用轨道将联系起来,并改革官学和科举,打开气学门人进入朝堂的通道,对新法和新学都进行考订和修改。

王安石和章惇的意见依然是保持现有国是不变,此外加强河北、河东的交通,同时对辽保持攻势,浅攻诱敌,蚕食辽国主力,不让耶律乙辛有喘息的机会。

新党的提案与韩冈针锋相对,既然韩冈任何时候都不忘要挖开新党的根基,王安石当然也坚决不给气学出头的机会。

而且有了选举资格的朝臣们,肯定都会在这第一次会议上试用一下手上的权力,那么就不应该被动等待,而是应该去主动利用。

曾孝宽对这份提案还是比较有信心,毕竟愿意冒风险的朝臣并不多,尤其是已经身居高位的那一批,没有几个愿意拿身家性命做赌注。

但韩冈的发言,改变了这一切。

韩冈不再是简单的要推翻新党、新法,而是要从新党手中,抢过新法,夺得主持变法的名分,按照自己心意去改造。

之前准备良久的一番陈词,被曾孝宽抛到了脑后。

——掌握在新党手中的变法大旗,绝不能让韩冈夺走。

现在韩冈才三十岁,一旦给他掌握了变法大业的主导,那就没新党的事了。

相反地,如果让韩冈铩羽而归,吕惠卿,甚至章惇就有机会在对辽战事中立下殊勋,不必一举平辽,或是收复燕蓟失土,只要有些功劳,北伐事权便可以控制在新党手中,日后也才能让新法继续维持下去。

“陛下。”曾孝宽不能耽搁,紧跟在韩冈身后出来。

只是他素乏捷才,短短的时间,很难找到一个有新意的腹案,更别说胜过韩冈。他正准备借助慢悠悠的动作,来挤出一点思考的时间。

但是他忽视了一个人,吕嘉问几乎是与他同时出班,仰头抗声道,“陛下,变法者,先帝与平章所拟,行之有年,中国日渐昌盛,军事渐强。国用偶不足,不过是因为北界乱事,其实已远过于熙宁之初。”

吕嘉问想要驳斥韩冈,阻止他去抢夺变法的大旗,只是一时兴起,却没有自己相应的提案。

可这样直接攻击韩冈的行动,惹怒了一人,“吕卿,拿出你的提案,由诸卿共议,孰是孰非,自有公论。”

太后的愤怒,恰到好处,到底该选谁,很多人的心中,已经不再犹豫。

