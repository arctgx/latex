\section{第23章 铁骑连声压金鼓(五)}

【熬夜码字,迟了一点的第二更,求红票,收藏。】

王舜臣坐在城头上,紧抿着嘴望着城外的一片火光。

一名亲兵正帮他裹着大腿上的箭伤。虽然中箭的位置是因为没有,但隔着套在外袍内的两层重绢,他所中的一箭只不过入肉一寸而已。而且有重绢隔着,箭头上的锈渍污物也没多少沾到伤口。

取出箭头,用盐水洗过,涂上止血的伤药,甚至不用缝上伤口,直接就包扎了事。王舜臣的亲兵都是在疗养院经过培训的,甚至有一个是从护工直接被调职,皆有一定的急救术水平,处理王舜臣腿上的箭疮,完全是游刃有余。而现在他们奉了王舜臣的命令,大部分都在战地医院中帮忙处理伤病,为王舜臣去争取士兵们的好感。

不仅是王舜臣的亲兵,韩冈的、王韶的、高遵裕还有赵隆、王厚他们的亲兵,每一个都是在疗养院中学习过战场急救。秦州那边王舜臣并不清楚,至少在古渭,让亲兵学点医术都成了一股风潮。经常上战场,身边有几个懂点医术的亲信,上阵时也可以安心一点。而且还有就是跟王舜臣一样的想法,在军中,医生总是最受人尊敬,亲兵在士卒们中间结下善缘,对将领来说也是件好事。

就像现在的王舜臣,他虽然夜袭失败了,自己还中了一箭,但士兵们依然保持着对他的敬意。一方面是他王舜臣有着秦凤路中能跟箭神刘昌祚一教高下的箭术,另一方面,也就是他的亲兵为他建立起来的人望。

‘但终究还是吃了亏!’王舜臣不忿气的捶着城墙。

方才出城劫营时,王舜臣完全没想到以吐蕃人的头脑,竟然能提前猜到并设下埋伏,害得他不得不狼狈退回城中。不过他还是成功的把绝大多数跟着他出城作战的士卒都带了回来。

在最后的一段道路上,王舜臣展露着如鬼神一般的武勇,领着四十多名强弓手拦道而立,借着微弱的星光,对着紧追上来的吐蕃人一阵迎头激射。在夜中依然精准如神的射术,用箭矢换来了一声声惨叫,吓退了追兵,让王舜臣施施然的回到城中。

王舜臣所率领的这四十名强弓,是从手下千名将士中精挑细选出来,射术皆为一流。靠着他们最后时刻的精彩表现,还有他守护着伤员们的亲兵,使得城中的士气犹存。另外,他在夜袭前,还从党项人那里骗来了一批箭矢,让自己的手上,多了一份守下去的本钱。

王舜臣很感激曾经在他面前聊起过历代知名战例的韩冈和王厚,今夜的计划就是模仿张巡守睢阳的战例而来。乘着月色晦暗,把一束束草人垂下城墙,骗来了一批箭矢,直到最后因为太过贪心的缘故,没有将之及时回收,让敌军发现了破绽,临时造出了一批火箭,烧掉了十几束草人。

尽管有所损失,但最后弄到手的箭矢,也有近万支之多,相当于每名士兵都能分到十支箭。想到这里,王舜臣心中释然了,今夜总算没白费气力,明天也可以用这些箭矢给城外的西贼点颜色看看。

王舜臣对城外的敌军营地重重哼了一声,想要攻破他的城池,也显得看看他手上的长弓答不答应。

…………………

夜色中,韩冈和智缘各自坐在一张小马扎上,两人面前的火堆,驱散了九月山中的肃肃寒意。他们以及瞎药和他的一千兵,现在刚刚向西走了有六十里的。这个距离,如果一开始走的是官道,现在就应该已经出现在渭源堡下。

不过瞎药和韩冈都没有直接救援渭源堡的意思。前面韩冈从王惟新那里已经听说了,王舜臣正被困星罗结城。如果能帮王舜臣解围,再顺势堵住大来谷的贼军退路,来个关门打狗,今次一仗就不会有悬念了。

渭水上游流域的山路众多,就算不走官道还是有其他道路能通往星罗结城。解救星罗结城中的王舜臣,接着封锁住大来谷,是韩冈和瞎药的如意算盘。

瞎药和韩冈都是谨慎的性子,还没出发就已经派了得力人手去探查道路,现在有的继续向前试探,有的则是回来报信,至少在他们已经探查过的地方,不需要担心党项人的伏兵。

现在韩冈正追求着更多的战功,能让他能早日转官。如果他是仰仗王韶鼻息而任官,那他最好也是最稳妥的选择就是去渭源堡。但韩冈自认他与王韶更接近于志同道合的盟友,便不会在乎这些小事。

“不知机宜为什么要把今次的功劳让给瞎药。”智缘拿着根粗树枝挑着火堆,把火拨得更旺上一点。他的声音中多了几许恭敬,韩冈的今天对付野利征的手段,让他叹为观止,但他还是不明白韩冈的用心,“机宜方才一刀,不让昔年班定远,此功若是报上去,天子和王相公必然喜欢。”

“一切以稳定为上,把功劳给他,就是让他继续臣服与朝廷,也更容易调兵……不如此,如何能救得出王舜臣和王安抚?”

这个功劳韩冈并非不想要,如果是在还没有决定归属的蕃部中,遇上西夏使节,韩冈定然会直接了当的一刀斩了,随之而来的功劳他也会乐于接受。

但自从古渭之战后,王韶和高遵裕软磨硬泡而来的俞龙珂瞎药兄弟俩,就已经被视为大宋的臣子。瞎药作为宋臣,其摇摆不定的态度肯定会连累到推荐他的王韶等人。一旦韩冈在瞎药城中斩杀西夏使者的消息传出去后,他和王韶不会被褒奖,而是会被追究之前欺君的罪过。

“所以这个功劳只能让给瞎药来领了。”韩冈也跟着智缘拨了下火堆,让其保持在现在的火势上。接着对智缘道,“早点休息吧,明天还要早起。”

…………………………………………

大约四千左右禹臧家战士,正重重围困着驻扎了大军的营寨。而三里外的渭源堡,也有近两千人在围攻。不过他么点兵力根本不够用,吐蕃人不擅长攻城,一队骑兵冲上去,射过几箭,再退下来,这就算是完成了一次进攻。没有足够的人数,眼下能做到的进攻就只剩下这一种。

看到他帐下士兵的种种丑态,禹臧花麻放弃了破城的幻想。开始盘算着该如何才能顺利的退兵。

名将的基本条件是知进退。何时该进、何时该退,进退时机能了然于胸,不为眼前之利所迷惑。做到这一点,就可以去争取名将这个称号了。

禹臧花麻自认还算不上是名将,但他在战场上对进退时机的把握还是很有一套。试探性的攻了一天,一举攻破营寨和城堡这等美事,他不会去幻想,但连城防上的破绽都没找出一处——更确切点说,上面的破绽不是他手下的几千兵能利用得上的——这让禹臧花麻彻底放弃了在渭源堡这块肥肉上咬下一块的念头。

虽然并不太清楚分据在营寨和渭源堡中的兵力究竟有多少,可能很多,也可能很少,但禹臧花麻无意去用人命去赌一把。梁乙埋命他出兵河湟,他已经做到了,不必在继续为梁乙埋拼命。

禹臧花麻不会把自家在国中以之立足的本钱,丢在对他来说毫无意义的渭源堡下。如果他丢掉了对他道命令俯首帖耳的上万精兵,禹臧家顿时就会从国中排在前五的豪门,沦为人见人欺的杂碎。而且禹臧家跟董毡的仇怨极深,若是听到他们倾巢出动的消息,不趁机来攻上一次,除非董毡突然变成了吃斋念佛的贼秃。

而且瞎药那里也可能会有变数,禹臧花麻不会把希望全数寄托在一个说客身上,更确切地说,他本就不相信野利征能够成功,仅仅是随手布下闲子而已。

他最希望的是能收到星罗结城的捷报,但到现在为止,也只传回了一切顺利进行中。禹臧花麻心知事情不对,当即便萌生了退意。

“该退了。”找来了领军的将领,禹臧花麻说出了自己的命令。

他在他的部下中有着极大的权威,禹臧花麻很容易就驱动了他们为自己服务。有着外来的帮手,一切便处理得得井井有条,不见一丝慌乱。

一场战事虎头蛇尾,不过对双方来说,他们最初的目标都已经达到。王韶要的是别羌星罗结的脑袋,而禹臧花麻则是想着应付一下朝廷。而论起损失,如果只算眼前双方都差不多,攻城和守城双方都损失了近三百。而星罗结城里的王舜臣那边,只要城池未破,伤亡的人数最也只会提高上一倍。

没有吃大亏,已经可以酬神拜佛了,可王韶还是高兴不起来,因为从禹臧花麻能出现在武胜军,有一件事已经可以确定——那就是木征倒向了西夏,否则禹臧花麻绝不会来去得如此轻松。

这是王韶在河湟与西夏人的第一次接触,可以预见的是,这绝不是最后一次。在日后攻打木征,慑服董毡的过程中,必须留上一只眼睛盯着被北面的兰州。

迟早要分出个胜负来。

不论是王韶,还是禹臧花麻,此刻都有了觉悟。

