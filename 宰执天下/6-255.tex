\section{第33章 为日觅月议乾坤(13)}

先帝。

吕惠卿带着哭腔的声音刚入耳,赵煦眼眶忽的就是一热,只感觉泪就要流出来。

想不到时至今日还有人记得他的父亲。

赵煦已经很久没有听到有人提起自己的父亲了,除了要祭拜太庙,或是教训自己的时候,身边的人都绝口不提熙宗皇帝,仿佛大宋的第六任皇帝根本不存在。

自己的耳朵里,只有太后、太后、太后。

让北虏不敢南窥是太后的功劳,国泰民安是太后的功劳,甚至这几年的风调雨顺也是太后的功劳。

先帝兢兢业业、宵衣旰食了十几年,平定南蛮和西贼的功业一下子就没人提了。

围绕在太后身边,尽是忘恩负义的奸贼,没有先帝将他们从草莽中简拔,哪里有今日的风光?

每每想到这里,赵煦的心中就仿佛有火在烧。

幸好有不惜一生令名,也要保护自己的王平章,也有看到自己长大成人就按耐不住情绪的吕宣徽。但这两位忠臣都不在京城之中,能留在京城内的,只有那**贼。

“果然啊。章相公说的没错,真的是哭起来了。”

背后传来的声音,让赵煦不寒而栗。没有任何缘由,甚至没经过头脑,他的身子就抖了起来。

在赵煦的记忆里,这样的声音他没有听过几次,只有提及那一位戾王的时候,才会有着如此让人深寒刺骨的冷笑。

仿佛身后的温度降到了冰点之下,赵煦感觉到自己背后起了一片鸡皮疙瘩,寒毛全都竖了起来。

这是对吕惠卿有多深的成见?!如果不是吕惠卿有大功,又无把柄与人,

还有那章惇,竟然能够先一步预料到吕惠卿会在朝堂上哭起来。

赵煦先是难以置信,但看到章惇看着吕惠卿,如同猫儿戏鼠时的眼神,又猛然醒悟过来。

吕惠卿为什么要哭?

不会完全是因为心情激动,他毕竟是做了几十年官的老臣。

一番话说得动情,但细想下来,其实就有宣称自己已经成年的用意,这是想让自己早日亲政才说的话。为了将这番话说出口,吕惠卿甚至不惜牺牲名望,还冒着被御史台弹劾君前失仪的风险。

如果说刚才吕惠卿的泣诉,让赵煦觉得是这位远离京师的宣徽使有着一颗他人所不能及的赤胆忠心。现在明了了吕惠卿的话中之意,赵煦的心中依然有着同样的感动,那同样是忠臣之为。

就像金陵的王平章,为了让自己能够早日亲政,为了给自己撑腰,把孙女都推了出来。谁不知道,家族中出了一任皇后,身份就从士大夫转成了外戚。王安石为了他赵煦,赔上了整个家族的身份。

什么叫做忠臣,这样的才是。不计一身毁誉,为天子不惜自身。

可惜吕惠卿的这个计策,被章惇给预计到了。

这也不足为奇。忠直之臣,怎么可能斗得过那些奸佞之辈?勉强想出了一个计策,立刻就被人给看破了,反倒是成了把柄。

但接下来该怎么办?

御史台肯定会出来攻击吕惠卿殿上失仪,太后就可以趁机处罚这位忠臣,甚至可能会被改派去疫症多发的地方做知州。吕惠卿看模样都六十多岁了,这样一去,还能活上几年?

赵煦的心抽紧了,王老平章已经时日无多,再失去一个吕惠卿,朝中有威望的忠臣还剩几人?

一定要保住吕惠卿。

赵煦完全没有犹豫,在瞬息间便下定了决心。

若是太后要重责吕惠卿,他要义正辞严的站出来为吕惠卿辩护,怀念先帝怎么能是罪名?

大不了也学吕惠卿,当殿哭上一场父皇,看看太后还能不能处置自己?

想到那个场面,赵煦就兴奋起来,众目睽睽之下,即使是太后、权相,也不能违逆人情,他这个皇帝出面保护感念先帝的臣子,纵使不符礼仪,却符合孝道,赵煦可不信现在就在殿上的那位儒学宗师,能不要脸皮的说自己错了。

以子之矛,攻子之盾。赵煦心头一片火热。

干涉对吕惠卿的判罚,这是听政的第一步。日后渐渐对朝堂政事发表自己的意见,迟早会聚来大批忠心的臣子。

太后能垂帘听政,是因为先帝的诏书。而先帝给她的权力,不过是权同听政,能够名正言顺听政问政的只有自己。就算太后不愿归政,自己问政的权力谁敢剥夺?

赵煦想着,就看见殿中侍御史李格非步出了班列。

“好了!”太后冰冷的说着,打断了李格非正准备要说的话,“吕卿家是什么意思,吾已经明白了。你是想让官家亲政是吧?”

什么?!

如同晴天霹雳在赵煦耳边炸响,太后怎么能这么说?!吕惠卿分明没说得这么明白。

赵煦看向吕惠卿,就连这位忠臣怔住了,愣了一下方才说道,“……官家年岁已长……”

“好了!”向太后再一次十分粗暴的打断了臣子的话,纵使有苏张之辩,也得把话说明白了,吕惠卿被太后这刻意打压,一番谋划还没有正式实施就终结了。

“官家,你怎么看?!”向太后突兀的向前方呆坐的赵煦询问。

赵煦没有回答,他的心中已如一团乱麻,根本不知道该说什么好。

为什么不是要责罚吕惠卿?这让自己怎么说?脱离了事前的计划,赵煦突然发现自己做不到随机应变。缺乏经验的他,根本就不知道这时候该给出什么回答更合适。

“官家,你说如何?!”

太后没有给赵煦思考的时间,更加强硬的问着。

赵煦发觉自己难得的成了殿中的焦点,臣子们的视线都投到了自己的身上,甚至能感觉到其中许多还带着责难。似乎是在责备他没有即刻回答太后的问题。

‘为什么要责怪朕?还当朕不知道真相?’

赵煦怒火中烧,火焰烧灼着五脏六腑,血管中也好似有岩浆在流淌。

世上无数人都在说自己是弑父弑君的罪人。自己的祖母和叔父,都借此为由,要致自己于死地。

可父皇卧病在床,谁最为得利?父皇驾崩,又是谁最为得利?

父皇驾崩,被太后和宰相直接归罪于当时只有五岁的自己,说是阴差阳错,孝心做了坏事。

赵煦曾经对此深信不疑,但随着年纪渐长,就越发难以相信此事。

将罪名归咎到一五岁小儿身上,也亏他们有脸说出口?随口一句就害死了自己的父亲,天下哪有那么巧合的一件事?难道不是控制着福宁殿的人最有机会,也最有可能?

‘官家,姐姐今天说的话你记好了,别对他人说……你父皇驾崩有蹊跷。’

亲生母亲只在自己耳边说过这句话,也仅仅说过一次,没头没脑,更没证据,但已经牢牢刻在了赵煦的心里。

当时福宁宫内,父皇身边都是太后安排的人,死掉的御医又是那位韩相公所安排。给自己定罪的,是他们两人,父皇驾崩后,最后得益最多的,也同样是他们两人。

自己当时只是五岁孩童,看不出情弊,但之后想过来,什么话都是他们说的,一句话定了罪,自己就成了弑父的罪人。

赵煦曾想过,迟早有一天要将真相揭露给世人,洗脱身上的冤屈,让世人明白谁才是真正的凶手。

但现在还没有到哪一天,来自太后的催促,是赵煦所不敢忽视的。

仿佛张开大嘴的青竹丝,又仿佛亮出尾针的黄蜂,面对太后的质问,赵煦的双唇已全然不见血色。拳头握紧又放开,低下去的面孔有着这个年纪所不该有的怨毒和狰狞。

他想说一句朕要亲政,却怎么也开不了口。到了嘴边的话,竟变成了,“孩儿尚年幼无知,又未成婚,并非亲政的时候。”

话声从牙缝中挤出来,旁边的小黄门听见,立刻放声传达了出去。而赵煦也仿佛失去了所有的气力,一时间瘫软在御座上。

……………………

章惇十分遗憾。

赵煦这个岁数,正是年轻气盛,爱闹别扭的时候。现在为群臣凌迫,发脾气的可能性自是更高一点。

只要他敢说一句请太后撤帘或是朕要亲政,不孝的罪名,立刻就能加到他的头上。

没想到他这一次会这么知情识趣,章惇眼中有掩藏不住的遗憾。太后询问赵煦自己的意见出乎意料,可如果赵煦闹起脾气,倒是能彻底解决了他,但现在,却是要多等些年了。

只是眼角的余光中,章惇发现,韩冈的眉心微皱,显是对这一结果并不满意。

……………………

韩冈还是想通过臣子们的选举得到结果,而不是因为皇帝自己想法而继续垂帘。

可惜,太后的无意之举,破坏了这一次重挫皇权的机会。

赵顼和赵煦两父子给了他太多机会了,要不然不会有重臣议政,不会有廷推,韩冈只会尽力去推动技术的进步,推动生产力的发展,将变化交给未来,而不是主动去改变政治制度。

但不知是幸与不幸,在几次变故中,不想放弃机会的韩冈,走上了一条他早年完全没有打算走上的道路。

现在,已不可能再回头的他,也只能继续走下去了。

他望着台陛之上。

不管对手是谁,也不管前路有多曲折,也只能继续走下去了。

……………………

吕惠卿已回到了驿馆中。

但他进门后,上来奉承的官员一个都没有,份外让人体会到孤家寡人这一现实。

吕惠卿自嘲的笑了,等到今天在殿上发生的一幕传扬开,身边怕就是更萧索了。

但吕惠卿笑得很开心,

失败了?

不,成功了。

经过今天的事,天子和太后之间的嫌隙越发的深了。

这才是他日后立足朝堂之本。

只要再稍等时日。
