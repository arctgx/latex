\section{第二章 边声连角不知眠(一)}

【第二更,求红票,收藏】

想挑拨着别人出头敲自家仇人闷棍,但最后动手的事却摊到了自己头上。读书不多的王启年说不出作茧自缚这个成语,却是在叹气自己搬起石头砸自己的脚。谁想到窦解会是这样的人物?王启年苦恼了一夜,想出的几个计策,没一个能用得上。一夜辗转反侧的到了第二天,又是发生一桩出乎他意料的事情来——

韩冈生病了。

更明确点说,是韩冈告病,请假在家养病。

可谁都知道韩冈根本没病,他是在抗议。没人能想到,拥有官身才不过几个月的韩冈,连这一招都学会了。

韩冈前面他不生病,那是为了自己名声着想,一上任就生病当然不好,少不得被人说闲话。而半月之后,经历过日夜处理繁重的政务,把衙中一应琐碎杂事无一处不处理的妥妥贴贴。这样的情况下,他已经可以生病,给自己放个假,李师中没脸拿这事来指责韩冈。

李师中、窦舜卿与王韶之间有恩怨,而韩冈则是被连累的。现在是韩冈吃了半个月的辛苦,而且还有暗地里遭陷害的危险。他等于是在为王韶挡着箭。他已经抗了半个多月,没有理由再为王韶扛下去。韩冈对王韶已经做到了他该做的,剩下要出生入死,陷自己与险地的事他可不干。

对韩冈来说,他已向王韶表现了自己的忠诚,他已向李师中、窦舜卿表现了自己的坚持,他已向整个秦州官场表现过了自己的能力,那他还有什么理由再卖傻力气?

五个人的勾当公事厅只有韩冈一人,他一力支撑官厅半个月,已经够久了,所以韩冈很爽快的病了。

依照时节,四月就已经可以算得上是夏季了,不过秦州的气候比起中原、江南都要冷一些,气温依然留在春天。晴日的时候,天气仍是清爽宜人,阳光和煦而不炽烈,无论出行,还是在家中,都是一年中最舒服的日子。

韩家小院中的梅树已经长得郁郁葱葱,片片叶子翠绿,一颗颗只有指尖大小的梅子藏在树叶丛中。韩阿李说是等这些梅子熟了后,就可以自家做些梅酒来喝。

一大清早,让李小六去衙门里帮自己告了病后,韩冈就靠在梅树旁的一张躺椅上,阳光透过树荫照在他的身上。他手里拿着一本书,很悠闲的翻着,一看就知道病得很重——懒病。

躺椅还是韩冈前些日子刚搬进来时,请木工打得,连油漆都没用,纯粹的原木色。虽然这并不是摇椅,但形制在此时已经算是别出心裁。韩冈在三月寒食节后踏青,出城后看到的游人都是坐着小杌子、能折叠的交椅,或是干脆席地而坐。即便在家里的院子中,如王韶家,也只是一张交椅坐着,哪像韩冈让人做的这种斜靠背、带扶手、而且足够结实的躺椅。

靠在躺椅上,韩冈享受着难能可贵的悠闲时光。半个多月来,他一直埋头于沉重繁琐的公务,现在的清闲是他前些日子做梦都在想的。这才是官员应该有的生活,奔波劳碌的是胥吏,不是官!

其实韩冈第一天就想生病请假了。虽然用繁琐的公务来整人是衙门中常见的手段,许多只擅长诗赋的新晋进士,往往就是这样吃了大亏,栽得灰头土脸。也有许多奸猾胥吏,为了让长官知难而退,使得自己得以把持政务,往往也会用上相同的手段。

但李师中、窦舜卿实在做得有些过火。四个同僚找借口出去,自己留守在厅内,像个傻瓜一样。但刚上任就请假,实在招人物议,故而他忍了七天。等他跟王厚的一番话后,韩冈想了又想,还是决定再忍个十天,至少把自己的才能多展露一些。到时候再放手,不会有人怀疑是自己的能力不足,而是明白他韩冈不想陪李师中他们玩了。

管你有什么阴谋诡计,我照样说一句恕不奉陪。韩冈打算歇个两天,直接跟王韶去甘谷城,在那里考察一下,把伤病营的这摊子事做起来,这是他的职司之一,李师中也说不了他不是。

韩冈垂下手,从躺椅边的小几上端起一杯微温的茶汤,喝了一口。一只白脸山雀扑楞楞飞到了梅树枝上,尖声叫了两声。清风拂过,树叶在风中沙沙作响,阳光照下的树影变幻不定。韩冈打了个哈欠,这样的安逸清净,实在让人沉迷。

后厅一个陌生的大嗓门,打破了宁静,传入韩冈耳中,也把枝头上的白脸山雀惊飞了去。韩家新宅只是精致,并不算大,只要门窗一开,声音就能随着风穿过来。韩冈也不用猜,这是韩阿李找来的牙婆,好像是姓柳。

韩冈听韩阿李说过,别的仆役可以暂时不要,首先得找个懂女红的厨娘。韩冈已经是官人了,都是老夫人的韩阿李自然不便在下厨,韩云娘一个小丫头忙里忙外的,实在忙不过来。韩冈不管这些事,听过也就算了,反正家宅里的事情都是韩阿李在管。

大嗓门在后面大声谈笑,这些三姑六婆都是在各家后院走门串户的多,还有的顺便卖些针头线脑的小玩意儿,顺便说说闲话,传些八卦,也是大户人家的女眷为数不多的娱乐活动。

在韩冈的理解中,她们大略是水浒传里王婆一样的人物。只不过像王婆那样即做媒婆、又做牙婆、还做产婆,私下里又能帮人撮合偷情做马泊六的,也算是极品了。这世上的三姑六婆大部分还是循规蹈矩的居多。

低下头,翻着书,将噪音从脑中过滤出去。韩冈低头读着由唐时大儒孔颖达注疏的《周易》。他还是有心在三年后考一次进士,在七品以下,进士出身的官员要比无出身的官员晋升速度要快一倍。无出身的官员只能一级级往上爬,而进士却可以一次跨两级,而且到了七品之后,对于无出身的官员,还有一道透明天花板存在。这就是为什么,进士在天下文官中只占了十分之一多一点,但在朝官中,却绝大多数都是进士。

后院正房中,秦州有名的柳牙婆走后,韩阿李支开小丫头,就对韩千六道:“云娘太小,还要一两年的时间。三哥偏偏在这方面又不开窍,但家里的香火事不能耽搁了。这厨娘也不要她多会做饭做菜,只要能生养,看着人品还好,就让三哥收了,明年就能抱上孙子了。”

“那还不如让三哥先娶了亲,再收妾不迟。你前些天不还是说要三哥先娶亲吗?”

“你懂什么,三哥他去京里都拜见过当朝的相公的,日后肯定,能随便娶一个吗?”

自从前两天,韩冈无意中说出自己在东京城跟如今有名的王相公说过了话,韩阿李的心气顿时变得高了,秦州城里的那些上门提亲的现在都不放在她的眼里。只想要一个正正经经的官宦人家的媳妇。

韩冈还不知道韩阿李正在算计着自己,他读了几句拗口晦涩的经文,对其中几句的句读有了很深的疑问。正想起身回书房,找另外几卷周易的注疏对照的看一下。守在外院充当门子的李小六,这时却领着王厚进来了。

王厚进院就看见韩冈舒舒服服的躺在院中晒太阳,当即便笑道:“玉昆,你病得好悠闲啊。”

韩冈站起身:“处道兄,你这不是探病时该说得话吧?”

“你也没真病。”王厚看着韩冈的躺椅:“你这张交椅还真不赖,看着就舒服,上次就想问了,究竟是在哪里打得。等过几天我也找人打一张,给家严表点孝心。”

“是牛栏街小李木匠。”韩冈也不提这躺椅是自己的主意,“他的手艺挺不错,榫头用得尤其好。”

王厚绕着看了两圈,又坐上去晃了晃,点头道:“果然够结实,比那些摇摇晃晃的交椅好多了。”

躺椅虽然好,可院子里只有这么一张,总不能一人坐着,一人站着,韩冈便引着王厚到书房去说话。

在书房中坐下,韩云娘听到声音便捧了茶过来,王厚接过来喝了一口,便道:“玉昆,你这病请得好,家严说你行事自有分寸,让愚兄不用担心,果然没说错。”

“机宜是过奖了。我这也是实在不能再忍,干脆放手。”

“李师中、窦舜卿本来就是跟玉昆你过不去,你一人做五分工,他们就是想看你笑话,你早该放手的。现在才放手,已经仁至义尽了。”王厚说了几句,便正色道:“玉昆你今天就在家好好歇一天,家严让你官厅里的事就别管了,明天一起去古渭。”

“古渭?昨天机宜和处道你不是才从古渭寨回来?”

“硕托、隆博两族终于打起来了,方才才到的消息,家严管着秦凤西路蕃部,当然脱不了干系,不得不再走一遭。”

“两族争斗事小,要小心李师中、窦舜卿籍此使坏。”

硕托、隆博两族的争斗,早在三个多月前,在古渭寨过年的时候,王韶就已经移文经略司,提醒李师中做好准备,但李师中却什么事都没做。虽然其中王韶本身挑不出一点错来,但保不准会给栽个罪名。

“窦舜卿的那等弥天大谎都能得到支持,还有什么做不出的?”韩冈这并不是在危言耸听。

