\section{第12章 大厦将颓急遣行(下)}

韩冈本打算趁大清早回家报个信,然后再赶回来。没想到一出城门,就看到了自家老子【注1】和李癞子。

韩千六又惊又喜,一把抓着儿子的双臂,上上下下来回打量了好一阵,像是古董收藏家将珍藏的瓷器不小心磕着碰着后,上下检查有无损坏的那样紧张:“三哥儿,你没事吧?”

韩冈笑着反问:“孩儿像有事的样子?”

“你没杀人?!”

“这事啊……”韩冈轻轻笑了起来,横着瞥了李癞子一眼,在韩千六眼中,儿子现在的眼神就跟方才李癞子的没两样,“孩儿的确杀了人……”

韩冈的话在这里顿了一下,韩千六的脸苍白了起来,李癞子则仿佛被金块砸到了脑袋,又高兴却又疑惑。而韩冈立马为他解惑:“刘三、张克定、肖十来。这几位,里正应该都认识罢?”

现在轮到李癞子脸色苍白了,双脚软绵绵的毫无力气,亲家的小跟班他怎么会不认识:“他……他们……”

“昨夜孩儿接了看守军器库的职司,没成想半夜里这三个贼子竟然偷偷闯进来意欲纵火,便给孩儿杀了。”韩冈快意的看着李癞子的脸色由白变青,因与陈举结下死仇的一点担忧,在看到李癞子这番表情后也轻松了不少。兵来将挡,水来土掩,自己能做翻了李癞子和黄大瘤,照样能掀翻陈举!

“刘三三人都是里正姻亲的跟班,他们去军器库放火,贵姻亲怕是也逃不过罪责。我出来前正好模模糊糊的听一句,黄德用……”韩冈的声音很轻,细微的话声却如同晴天霹雳在李癞子耳边炸响,“已经畏罪自尽!”

……………………

时间过得飞快,而州中对军器库案的审理也是速度飞快。

十天前的那一声警号,已经从秦州百姓的家常闲谈中消失。刘三等人的死所造成的影响也渐渐沉寂。罪魁黄大瘤畏罪自杀,一切罪责都担到了他的身上,家产尽数没入官中,而他的妻女也被充入教坊司,而两个儿子则莫名失踪。州衙只发出了两张海捕文书,为两个儿子定下了五贯的赏格,便宣告一切结案。

陈举曾经拍着胸脯,要保着黄德用的妻儿——他做到了。他保着黄德用的儿子改名换姓远走高飞,而黄大瘤的几个妻女,刚进教坊司还没过夜便被高价赎走。为了从州中得到一纸脱籍文书——官妓的从良必须要得到官府同意——陈举费的钱钞不在少数。

通过安抚黄德用的身后事,陈举略略安定了身边的人心。接下来要对付的,便是害得他损失了三成多身家,又欠下多少人情的外敌。韩冈不死,人心不安。

一个稳定的官僚社会,其各个部门的权利划分,已经有了常年积累下来的定规。以节度判官的威风,却也压不住下一级的地方官。

这些天来,韩冈日日在普修寺苦读不辍,间中拉弓射箭来调节心情。唯有去吴衍府中与他的闲谈,方算得上休息。韩冈如此用功,让吴衍更加看重。只是他帮韩冈做得身份证明,想求一个单丁户的认定,成纪县丝毫不理。而成纪知县发来的一纸文书,韩冈却不得不走进县衙中。

绕过空空当当的大堂,走在通往县衙二堂的石板路上,韩冈的心中有些不好的预感。自缢而死的黄大瘤他曾去看过,脸皮紫得发黑,舌头吐得老长,颈上的那颗瘤子却干瘪瘪、皱巴巴的如同一个放久了的苹果。不同于十天来,几乎天天过河来探视的韩千六,韩冈心里并没有胜利的喜悦。因为这只是陈举为了自保而断下来的壁虎尾巴。毒蛇尚在身后吐着信子,他夜里依然是睡不安稳。

一名长得慈眉顺眼的老胥吏领着韩冈向里走,另一名身上披了白麻孝服的青年与他擦肩而过。韩冈记性很好,记得那正是被他顶了位置的周凤。这几天来,韩冈一想起周凤,便不得不感叹他真是好运气,若不是自家惹来黄大瘤,他少不得落个烈火焚身化焦尸的下场。

领路的胥吏见韩冈回头望着周凤,笑道:“这小子也是运气,他老子前夜上吊了,他家成了单丁户。今天县尹开恩,便放了他回家。”

韩冈神色微动,“真巧……”

“这等巧也没人喜欢,今年就剩两个月不到,如何不能再忍一忍。”胥吏摇头叹道,感慨万千。

韩冈冷笑,‘若不是你们这些胥吏贪酷,周凤之父又何必自了性命,只为了将儿子保回来?’

两人走到二堂前,老胥吏没直接进去,而是转头对韩冈道,“韩秀才,人死万事空,黄德用已死,一切过节都该揭过了,那李癞子还请放他一马,让他退了你家卖给他的田也就罢了。”

韩冈愣住了,这唱的又是哪一出?这几天听每日入城的韩千六讲,虽然株连是株不到姻亲上,李癞子却也被提到州衙中好生拷问了一番,过了三天出来后,秋天的蛤蟆变成了春天的蛤蟆,瘦得整整一圈,家产也损失近半。这一番折腾后,他被韩冈的手段吓的魂飞魄散,天天上门赔罪,还要送回当初强买的田地。若李癞子有陈举撑腰,又何须如此?

只是疑惑归疑惑,该说得话还得说:“黄德用既然死了,韩某哪还有仇人?李癞子那是更是小事,卖给他的田地日后我家自会用钱赎回,不会占他一文便宜。”

“好!好!好!秀才果然宽宏大量。”老胥吏笑道,“即是如此,俺就提醒秀才一声。今天县尹传唤,可能是要派秀才你新的差事。你进去后将家里事禀报县尹,报称单丁户,也可今天跟周凤一样径自回家去。想想李癞子,他现在也没胆子不帮你具结作保。”

韩冈躬身道谢:“多谢陈押司!”

陈举神色一凛,再仔细打量韩冈。只见他还是普通的士人装束,外表上温文尔雅,其风仪,秦州的士人少有能及。唯其眉眼如刀,在斯文中平添了许多锐气。但陈举还记得,当黄大瘤的尸身从家里抬出去的时候,这一位秀才就站在门外的围观人众中,如同鹤立鸡群。当时他凌厉的眼神不是看着黄大瘤,而是盯着自己。双眉如刀,眼神如剑,阵阵寒意从体内升起,自家的皮肤都被激起了一阵战栗,心中只念着不愧是名师弟子。若不是已经结下了解不开的死仇,他真是不想招惹横渠先生的学生。

“好说,好说!”陈举干笑着打着哈哈,陪同韩冈跨入堂中。

一圈衙役围在二堂内,明镜高悬的匾额下,一个三十上下的年轻人端坐着。正是如今的成纪县知县。韩冈进来后,他忙着签书文件,发落子民。只等到半个时辰后,他得空下来喘口气,一抬头,便看到了仪容出众的韩冈。

韩冈穿着青布襕衫,头戴方巾,一身读书人的装束。高大的身材,鼻正眉直,双眼清亮,一看便气度不凡。

对上读书人,成纪知县不愿失礼,温言问道:“你这秀才,姓甚名谁,来衙中又有何事?”

韩冈恭声行礼:“学生韩冈。得招来衙中候命。”

“韩冈?”成纪知县脸刹那间冷了下去,不复方才的温和。

德贤坊军器库的事让他吃了不少挂落,今年的考绩少不得要判个中下,磨勘时间又要延长一年。他从陈举那里听了不少小话,几乎把韩冈恨到了骨头里。什么事不能县里处分,偏偏闹到州里去!张载的弟子又如何?张横渠不知收过多少弟子,只听过两次讲经也能算是学生!这样的灌园小儿,又有什么好后台!?

“你就是韩冈?!”成纪知县又追问了一句。

“学生正是韩冈。”韩冈恭恭敬敬的行礼回话。

知县的脸板着,冷声道:“韩冈,你既然应了差役,却只做了一天的监库。我成纪县事务繁芜,也留不得闲人。如今正有一批犒军的银绢和酒水要送去甘谷城,就由你来带队。”

‘要不要继续担任衙前?’若是担任押运,运输途中的损失都得自己来承担。但他韩家可没半点多余的钱钞。

对于韩家来说,卸了衙前苦役,是最好的选择。而一起跟进来的陈举,则是温和的笑着,冲韩冈投过来鼓励的眼神。韩冈心底却在冷笑:‘若真的有心,现在就该帮我说话了。’

这肯定是陷阱!

单看现在这种情况,周围衙役都是虎视眈眈,而且也不知陈举是怎么在成纪知县面前编排的自己,那位年轻的进士知县看过来的眼神也是颇为不善。也许自家只要说个不字,大概就会被掀在地上,碗口粗的杀威棒伺候。不管以他现在的身体条件,还是没生病前的状况,都是挨不了几下,就要一命呜呼。

陈举倒是好演技,但群众演员们的水平就差得多了。韩冈在他们眼中看到的尽是杀机,不是‘也许、大概’,而是‘肯定’!杀人灭口,顺便收拾人心,陈举的确好算计。

‘但若是我答应呢,你还能当下动手?君子不吃眼前亏,就是暂且应下又何妨。当着我的面把周凤放了回去,想的就是让我这个单丁户说个‘不’字罢?如何会让你如愿!’

心念转动,韩冈便一口应承下来,“既是明府之命,又为得国事,韩冈自当遵从!”

不得不应下押送犒军的差事,韩冈脸上如同挂着寒霜,只当他看到陈举的脸色也是一般的难看时,才让他的心情好上了一点。

出了二堂,他抬头仰望灰色的天空,自己命运自己不能把握,而是被人操纵着。如果能有个官身,陈举之辈如何能动他分毫。发自内心的感叹喃喃出口:“还是做官好啊!”

注1:关西人俗称父为老子。所以有小范老子【范仲淹】,大范老子【范雍】的说法,这是尊两人为父的意思。而为了让儿子免去服差役,老子上吊的事,也非杜撰。

ps:新的旅程即将开始,漫漫官途前,请投一票为我们的主角壮行。

