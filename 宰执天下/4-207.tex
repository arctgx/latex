\section{第33章 物外自闲人自忙(三)}

文彦博最近心情很烦。

作为三朝元老,就是天天不做事,整日拿着衙门里的公使钱喝酒饮宴,都不会有麻烦,就是有小人上报给天子,天子也只会派中使来询问公使钱还够不够用——这就是元老——但他的儿子文及甫不是元老,现在的麻烦很大。

如今东京城中,御史台中那群报丧的乌鸦正在穷究相州之狱,整个大理寺都被牵扯进来,而自己的这个不成器的儿子,却是因为一封干请的信函,被牵连进这件明显有人在兴风作浪的案子中。

文及甫不与自己商量,就写信为自己那个不成器的小舅子陈安民说项。年纪早过而立了,办事还这么糊涂。

看过了文及甫寄出去的那封书信的底稿,文彦博差点要挥起拐杖将这不成器的六儿子痛打一顿,官场上说话可以直白一点,反正也是留不下证据,但文字上怎么也得阴晦啊,这都写了什么?!

“他是你舅舅,难道不是为父妻弟?!难道告诉我,我会看着他受罪不成?!”

文及甫低着头不敢搭腔,自家父亲的脾气他最清楚,越是多加辩解,责罚就会越重,最好的办法就是老老实实的低头手脚,如此才能安然度过。

文彦博果然在发了一通火后,喝了一盅宽中快气的香澄汤后,外表上也没那么生气了。文及甫松了口气,连忙亲自为文彦博又端过一杯药汤过来,小声的说道:“儿子知错了。本来以为不过是关说两句,不是什么大事的。”

“事大事小是没定数的。没人惦记你,贪渎巨万都是无事,遇上有人惦记,就是多耗了几分公使钱,都会被御史弹劾。你也不看看你岳父挡了几个人的道,政事堂、枢密院、御史台多少只眼睛都盯着他。关说有司,平常时不过是阵清风而已,说句话嘛,现如今却能掀起巨浪!”文彦博再瞪了儿子一眼,声色俱厉:“可就是寻常时候,信上也不能写得这么直白。当吴家子弟没读过书吗,需要像对小学生一样解说的那么明白?!”

文及甫唯唯诺诺,文彦博恨恨的又重重哼了一声。因为儿子办的蠢事,府中的公事全都耽搁了。

昨日没有让属吏去迎接韩冈,也是他的一时气话。其实文彦博出了口就后悔,但他并没有去反悔,朝令夕改反而会让人将他小瞧了去。

些许小事他可不会放在心上,虽然会对他的名声有所影响,虽然会与韩冈结下死仇,不过,那又怎么样?

文彦博会后悔,也只是因为会有损声名,但他身为元老,受封国公,从先祖到子孙全都得到封赠,名声好点坏点又有什么影响?开罪韩冈,他则是全然不在乎。

韩冈什么人,灌园子而已,寒门素户,连个书香门第都算不上。他文彦博三朝元老,日后都有机会与皇家联姻,自己的孙辈中,也不是没有人才,门生故旧无数,姻亲更是遍布朝堂。韩冈一个宰相女婿算什么,他面前这个不成器的儿子,即是宰相子、也是宰相婿,娶了吴充的女儿!

韩冈就算日后暴发起来,还能当真将他文家灭门不成?!要是韩冈当真将此辱放在心上,日后处处与文家为敌,保不定就此止步了。只是个年轻小子而已,要是有了心胸狭隘的名声,日后也别想有什么成就了,文彦博恨不得韩冈会如此做。

文及甫只知道自己的事情办岔了,只是简简单单的说情,最后却变成了一桩惊动了整个御史台的大案,现在京中已经派人来询问,下一步多半就是会将自己提去开封审问。

虽然自己有父亲在上面镇着,可说不定什么时候就会被人捉进大狱去。父亲虽然是要保自己,但如果京城来提人,就是现任宰相都不能拦,也拦不住,肯定要先去台狱走上一遭。

文及甫已经是京中之鸟一般,现在又开罪了韩冈,韩冈司掌漕司,有监察一路百官之权。自家的父亲得罪他狠了,要说他会宽宏大量的一笑而过,文及甫可不信。年少得志的韩冈能有这般器量,到时候少不得会落井下石。

幸好此时还能化解得了。虽说因为昨日之事,文家与韩冈仇怨已深,但韩冈为人是有名的尊师重道,文彦博与张载有推重之恩,张载第一次在洛阳讲学也是文彦博的安排,这份香火情虽然不在了,但重新提起来也不是没有用。而且还有二程,韩冈昨天甫一到任就派人送礼到程家,今天就去登门造访。如果找二程居中调解,韩冈的尊师重道无论是真情假戏,都必须给程伯淳和程正叔一个面子。

文及甫这一回被吓得够呛,他出生时,文彦博都已经做了宰相,从来都没有吃过苦,出门在外,文府的六衙内到处都能受到奉承,如今不意却碰上了对文家的权势毫不放在心上的对手,想想会被提进御史台狱中,胆子一下就小了许多。

偷眼看着父亲,文及甫想着该怎么措辞,却见文彦博已经不理不睬的拿着一封信来看了。看见了在拆开来的信封上有着包绶顿首的字样,文及甫便知道,是与他家关系甚为亲近的包拯次子的来信。

文彦博将信上下看了一遍,抬头对文及甫道:“包家的綖哥儿一年丧期已满,说不日会来洛阳造访。綖哥儿去岁丧妻,中馈不能无人主持,也该续娶了。为父曾与包兼济【包拯】定有秦晋之约,只是各种事给耽搁。十一娘年纪只比綖哥小了几岁,也算是正合适。”

文及甫愣了一下,“将十一娘嫁过去?”

看着儿子似乎是有反对的意思,文彦博火气又起来了:“难道已经不记得了?!我文家与包家是世交,从你祖父开始就是如此。綖哥儿是个正人,十一娘嫁过去也不会受苦。”

包文两家的交情不用文彦博多说,文及甫自幼都是耳熟能详。

文及甫的祖父文洎,当年与包绶之祖、包拯之父包令仪同在馆阁,交情匪浅,而包拯和文彦博又在一起准备进士科举,日后两人在天圣五年【1027年】同科取中——同科的还有韩琦、陈升之、吴奎;与十五年后的王安石、王珪、韩绛同在的庆历二年榜【1042年】;以及又十五年后的吕惠卿、章惇、曾布、二程、二苏、张载所在的嘉佑二年榜【1057年】,是仁宗朝收获最大的三次科举。

包拯先字兼济,后改希仁通行于世,可文彦博偏偏就一直用前一个表字称呼他。父辈是知交,两人也是自少订交,因为这两层关系,包文两家就约为姻亲。

虽然包拯担任谏官的时候,也抨击过时任宰相的文彦博,但之后文彦博被罢相,一个理由就是他结交后宫,送了重礼给最受仁宗宠爱、后被追封为温成皇后的张贵妃——另一个就是阴结身为言官包拯、吴奎。

“当初为父与兼济定下来秦晋之好,愿相与姻缔,你的几个哥哥年纪都不合适,包家的大姐儿便嫁给了你的堂兄。只可惜他家大哥当时已经娶妻,而兼济故世的时候,綖哥儿才五岁,剩下的一桩亲事就一直都没提了。前次綖哥儿娶了张家的女儿,也是成了亲了才来信,否则为父肯定要抢先一步。”

文彦博回想着当年:“为父因唐介第一次罢相,过了几年之后,兼济因故被贬居池州,当时为父已经复相,就写信去池州。还记得为父写的什么吗?”

文及甫被问了个措手不及。他隐约记得,文彦博当时是写了一首七律过去,但他想了半天,才想到了最后的两句话:“‘别后当知昆气大,可得持久在江东?’”

文彦博怒哼了一声,明显的是对儿子很不满意,整篇七律记不得倒也罢了,但连记得的最后两句也都错了,甚至让意思变得截然相反,“是‘别后愈知昆气大,可能持久在江东?’!”

就跟朱馀庆临近科举时给张籍写了‘画眉深浅入是无’一样,文彦博知会包拯很快就会将他调回京师时,也是采用了隐晦婉转的曲笔。

包拯在池州只待了八个月,便调往江宁,在江宁知府任上做了不到一个月就又调回东京,回来后就担任了开封知府。开封知府包龙图的传说便是从此处发轫。

儿子背不全的这一首诗,可是文彦博的得意之作。可文彦博想起了当年旧事,就一下子就气冲天灵起来,横看竖看儿子不顺眼,拿着手指狠狠地点着文及甫的脑门。他不是要求儿子有自己或是朱庆馀的水平,文彦博的要求很低:“你就不能写得隐晦点吗?你就不能写得隐晦一点吗?读了那么多年书,做起诗文还不一定有韩冈强!”

文及甫嘴皮子动了动,想喊‘是可忍孰不可忍’,再差也不至于会比韩冈还差吧,但还是忍住了。

相对于韩冈的累累功绩,他的诗文水平在士林中更为人所乐道,就像日中黑影,有那么一点缺点就分外显眼,总是会被人拿出来当笑话说。

正说间,一名仆役匆匆而来,禀报道:“老相公,漕司那里递了帖子来,说新上任的韩龙图想明日登门造访。”

“明天?”文及甫闻言一惊。

“才一天就赶着来上门了?是想来查账吧?让他来好了!”文彦博纵声而笑,韩冈的急不可待让他心中快意无比:“想不到竟然这般沉不住气,韩冈如此心性,谁说此子能做宰相?!小器速成,纵然小有才具,日后也不过如此!不过如此!!”

