\section{第24章 南国万里亦诛除(五)}

【假日的时间总是身不由己,拖到现在,真的没办法。】

钦州知州本来算是个玩笑话,但经韩冈这么一驳,顿时面红耳赤起来。

韩冈也不是看不出他是玩笑,只是拿着别人的悲惨境遇当笑话说,如果是仇敌贼寇倒是没关系,拍手称快都可以,韩冈决没有什么‘人性、道德’之类的矫情,可放在治下百姓身上,哪里能让人笑得来。

他还有个身份是广西转运使。执掌监察路中各州政事的漕司,是钦州知州的半个顶头上司,随便挑出个错处,一份奏报就能让他丢官去职,正常也不敢在韩冈面前硬气。

不过钦州知州却没有服软,没有像想象中一般的低头认错,而是梗着脖子问道:“下官有一事不明,敢问龙学与章端明领军南征,到底是因为何事?”

钦州知州犟着嘴反驳回来,韩冈微微一愣,旋即恍然,“交贼入寇时,疍民在钦州做了什么?”

“倒也没什么……”钦州知州板着脸,表情却决不是在说没什么,“不过乘火打劫而已!”

在交趾入寇时乘火打劫……这个罪行,株连全家都不冤枉。

想来也不足为奇。疍民之中,除了若干首领能算得上富裕,绝大多数都是穷困潦倒,看到钦州城破,又没有了官府和官军的约束,不趁机抢上一票那才叫奇怪。而在这过程中,他们的手上当少不了沾上血腥。

年纪大约做韩冈父亲都够资格的钦州知州陈永龄,硬着脾气顶撞年少得志的转运使。身后的州中属吏,都为他捏了一把冷汗。

韩冈文武双全的才干闻名天下,在朝臣中也是排在最前面的出色。但这样的年轻人,往往都是锋芒毕露,很少能容忍他人的触犯。陈永龄当着多少人的面让他落了面皮,万一

落在后面的李宪脚尖动了动,想站出来缓和气氛,但看看前面的章惇都没动弹,犹豫了一下就定住了脚。

不过不同于众人的臆测,韩冈很干脆的向着陈永龄拱手一礼,致歉道:“韩冈不知此中情由,妄言冒犯,还望陈郡守勿怪。”

陈永龄没想到韩冈会如此,忙侧身避过,回礼道:“不敢,下官方才所言失当,运使责备正是!”

韩冈并不认为认个错有什么大不了的,他的自尊心和地位也没这么脆弱,不过陈永龄明显的有些感动。其余官吏们在松了一口气之余,投过来的眼神也有了几分变化。

“好了。”章惇插话进来,脸上带着点笑,韩冈的表现不出他所料,“玉昆仁心爱民,本是没有错了,只是不知内情罢了。有些罪囚并不值得同情!”

“说得也是。”韩冈叹了一口气。

陈永龄在前面殷勤的领路,章惇与韩冈并肩前行,随口问着:“既然知道了疍民之前的所作所为,玉昆你打算怎么处置?”

“疍民其罪当然得到清算,可眼下的情况,想查也无从查起。”疍民的团结,在沿海还是又有些名气,韩冈听说过不少传言,并不指望他们能将参与过劫掠的罪人给交出来,“总不可能像对付交趾那般,管他有罪无罪,一起砍了了事。”

“谁让他们是中国之民。”章惇摇摇头。

屠戮叛民和异族与杀戮国中子民,完全是两回事。眼下的情况是罚不责众,只能放着,或是推到交趾人身上。

“最好还是能将之编户齐民,州县中多了户口不说,留名在籍,日后犯了罪也别想逃脱。”

“疍民世世代代生活在船上,要想编户齐民,只能将他们迁移到陆上安置。”章惇侧过脸远眺着望不到尽头的蓝色的海,“但他们习惯的过来吗?”

尽管韩冈的想法有着很重的功利成分在,但对于朝廷和疍民本身都由足够的好处。

不过章惇说的也没有错。

生活在水上的疍民,尽管并没有多少人将其视为异族,但他们扎着椎髻,穿着短衣,光是服饰装束就与汉人截然不同。

且一代代的生活在水上,就算招揽他们上陆生活,也不一定能习惯的来。种地都是一门学问,打了一辈子的鱼,突然给了,谁又能很快上手?

只是韩冈眼下穷得慌,既然有着合适的目标,就不能轻易的放过。

在工业体系还是镜花水月的时代,人力就是一切。所以四夷攻打中国,最重要的任务就是劫掠人口,让擅长农工的汉人,为他们做牛做马,源源不断的创造财富。

几十万疍民生活在水上,甚至连户籍都没有,生老病死全都不经外人之手,这样的人群不加以收服,将其纳入官府的统治之中,实在是太过于浪费。

“但要防着日后再生乱却是必须的,只是不必急在一时,钦州沿海的疍民有上千户,没有一个妥当的策略,贸然行动肯定会出乱子。”

韩冈有时间也有耐心,为此等上一阵。等到安南经略招讨司的差事交卸,作为广西转运使来处置此事。

眼下就是要尽快赶回邕州,将南讨交州的战争做一个最后的交代。

……………………

在八九尺髙的石墙上,是一个只有一尺见方的小窗。窗口被三根手腕粗细的木棍等分,只留下窄窄的缝隙。粗大的木栅摇一下都不容易,想要从这样的窗子逃出去,那不是人能够做到的。

窗内是一间一丈方圆的房间,三面墙是土石砌起,而窗口对面的一面,则是全数由木栅组成。房间中只有稻草和一张薄薄的毯子,而净桶就放在房间一角,毫无阻隔和掩饰。

这里是邕州的大牢。自从被宋人从国中押送到邕州之后,他们这一干曾经攻打到邕州城下的交趾将校,都被送进了狱中。

躺在地面上的稻草堆中,到处都是阴湿的霉味,宗亶当真不知道,宋人到底是怎么想的,打算怎么处置自己,但自己的命运却是掌控在宋人的手中。

在牢狱中,他们至少能填饱肚子,也没有受到虐待。这让一众俘虏,有了几分侥幸的心思,只是宗亶不敢抱着这样的奢望。

从升龙府被押送邕州时,就在一旁的韩冈,那名将交趾国覆灭、却年轻得让人咋舌的官员,眼中尽是冷漠。而同样的眼神,也出现在每一位看守他们的狱卒身上。

“回来了!”也不知过了多少天,从牢房的窗口,突然传来了一片喧哗,“经略相公和转运相公都回来了!”

终于到了吗?宗亶抽紧了心,就算有了最坏的准备,但临到头来,还是发现自己心中一片惶然。

不知自己即将面临什么样的结局。

只有一死!

对于一众罪囚,却并不需要审判。发回来的圣旨已经敲定了他们将要受到的惩罚。

至今为止,忠勇祠前的祭品,只有一个徐百祥而已。这个数目,与交趾人在邕州犯下的罪孽相差实在太远,远远不能抵消他们造成的仇恨。

只是投降就想免死,这世上哪有这等好事?圣旨中唯一给出的恩典,就是从凌迟降格为斩首,算是对他们及时投降的回报。

供奉着苏缄和一众死节的邕州官吏,以及数以万计的百姓的忠勇祠,这一日,聚集了所有生活在邕州城中的大宋子民。他们都是劫后余生之人,一年多前的劫难中,侥幸逃得性命,不过每一人都有亲友葬身火海,至今一想起那一场大劫,至今难以安寝。好在官军为他们报仇雪恨,将仇人捉了回来。

嘬尔群獠,不知忠孝之道,惟逞枭獍之心。虽云宋臣,贡事不修。朝廷恩赏未已,兵势已犯中国。三州生民,十不存一。朝廷待汝甚厚,汝待朝廷何其薄也。其罪难恕,依律当以论剐。惟念其出降,当减其刑一等。以斩论之,决不待时。

章惇、韩冈等人列坐监刑,而苏缄的儿子苏子元就站在庙前,读过判词,一个个念着当处以斩首之刑的罪囚的姓名。

每念到一个姓名,两名军汉就会拖着一人走上临时搭起的刑台。拔掉插在颈后的木牌,强压着按到斩首台上。

山呼海啸一般的声浪,由数万愤怒悲恸的人们同时喊出,冲得台下待决的罪人们难以站稳脚跟。

侩子手上的斩首大刀,一个接着一个挥下,将一枚枚头颅扬起,然后送进忠勇祠中供奉在神台前。

台下待决的罪囚渐渐减少,送进忠勇祠中供奉起来的首级越来越多,直到最后的一人。

宗亶没有让人拖着,自行走上刑台,回头望望,无数充满愤怒的视线正盯着他。黯然一叹,成王败寇,也该有此报,引颈受戮。

宗亶之后,最后一个上场的并不是活人,黑黑的如同风干的腊肉,离得近了都还能嗅到一股子中人欲呕的臭味。

但干尸的出现,却引发了行刑以来,最大的一片声浪。这是李常杰的尸体,一直被保存到现在。

侩子手手起刀落,让罪魁授首。干枯的头颅高高吊起,就在台下,多少百姓就地烧起了陌纸,呼唤着逝去的冤魂。

“就算死了,也得到行刑台上走一遭。”章惇厉声,“敢于凌犯中国,绝不放过一个!”

“虽远必诛!”韩冈随之说道。

