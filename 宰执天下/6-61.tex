\section{第八章 朔吹号寒欲争锋(一)}

“尧夫。”

自殿中出来,范纯仁稍稍整理了一下因跪拜而偏移的襟口,就听见身后有人在叫他的表字。

回头时,就看见一名身形矮胖的官员,是李常。

“啊,公择。”

范纯仁倒不意外李常会叫住他。

昨日同在殿上投给了韩冈一票,范纯仁就想着李常什么事会过来找他。若李常不来,范纯仁也会过去拜访他。

范纯仁与李常是同年而生,同时还是皇佑元年【西元1049】的进士,政治观点又相近,同属旧党,自是早有交情。

而且昨日夜中,孙觉便来拜访了范纯仁,说了不少话,这让范纯仁也想找李常这位老友聊一聊。

这一回支持韩冈,范纯仁虽说不上迫不得已,但毕竟也是形势使然。

若不是旧党被一网打尽,朝堂上满是新党,根本听不到半点杂音。加之赵颢、蔡确叛乱,将旧党中人给牵扯进去了许多,连文、吕之辈都无法置身事外,如何会找王安石的女婿?

不过这也是韩冈主动提供了一个机会,否则也很难这么简单就与他搭上关系。

双方各有需要,所以一拍即合。

“尧夫今夜当还在驿馆吧?”李常问道。

范纯仁点了点头,“若公择不弃,今晚纯仁便洒扫庭院,烹茶煮酒,以待公择。”

“久未与尧夫共叙,今日当共谋一醉。”

这里并不是方便说话的地方,只是约定了夜里的会面,李常便告辞而去。

范纯仁却没有跟着一起出宫。

范纯仁前日抵达京城,昨日参加廷推,然后今天才上殿谒见,整个顺序都反了。

谒见之后正常还须问对,这是在宰辅们与太后的议事之后,必须要等上一段时间。

仰头看着屋檐外的春日细雨,范纯仁还记得昨夜孙觉来访,说起了为何会投韩冈的票。

范纯仁拿到的是富弼的信。而孙觉是因为接到了文彦博的信,而且还是吕公著的儿子吕希哲遣了儿子吕切问转送来的——吕希哲此时正寓居京师,孙觉前几日刚刚拜访过他。

的确是很有意思。

富弼一直很看重韩冈,他写信给范纯仁不足为奇,但文彦博、吕公著那两位,一直以来却是跟韩冈斗得鸡飞狗走。

文彦博与韩冈旧日曾经传说已经相逢一笑泯恩仇,还好说一点,而吕公著就难以理解了,一年多前,先帝卒中之后,韩冈可是一手将吕公著给赶出了京城。

也许在外人看来,富弼、文彦博,甚至吕公著,现在不过是想长保儿孙富贵,家门不堕,早没了过去的锐气,可在范纯仁、孙觉眼中,这一干元老重臣,还没山穷水尽到要有求于后生小子的时候。

不过孙觉昨日也对范纯仁说,这或许是吕希哲自己的判断。‘不主一门,不私一说’,比起做官,更在乎学术的吕希哲一直抱着博揽众家,择其善者而取之的态度。前日孙觉拜访吕希哲,两人讨论起经义,吕希哲也曾拿着气学中见解来与孙觉辩论。他对韩冈,或许并没有吕家子弟的成见。

元老们的态度是一回事,李常和孙觉本人的想法也很重要。

范纯仁很清楚,李常和孙觉都不是仅仅为了门户利益就会接受元老们指派的人,就像自己一样,对党同伐异四个字,还是抱着敬而远之的想法。

所以就在昨夜,范纯仁很直接的向孙觉询问了他为何会接受文彦博信上托付的原因。

孙觉当时则是反问范纯仁,问他时隔多年重回京师,感觉到京城有何变化?

当时范纯仁的回答是繁华尤胜,但烟灰多了许多,快要赶上延州了。

延州多用石炭,到了冬天,城市经常陷入烟雾中。范纯仁旧年随其父范仲淹至延州,对周围环境除了兵戈森严的紧张之外,感受最深的就是让人喘不过起来的空气。

时隔数载,范纯仁再次回京,呼吸到京城的空气,当年延州留给他的印象,立刻就在记忆中复活了。

本来范纯仁以为这是个人的感觉问题,只是想引出孙觉的回答,但他没想到孙觉正想说的就是这一点。

京城污浊的空气,民间使用越来越多的石炭仅仅占了其中的一部分,更多的还是因为日渐扩张的钢铁场而来。

孙觉告诉范纯仁,去岁钢铁产量几近十年前的五倍,而朝廷从中获取的收益多达三百余万贯。这还没有将节省下来的甲胄、兵器等费用计算在内。

新党对钢铁业极为重视,但发展到如今的状况,他们起到的作用,应该是不及韩冈的。尤其是炼钢炼铁的大规模扩张,还是在韩冈就任军器监之后。

从这里说起来,韩冈可谓是京城污染的罪魁祸首。但只要想到大宋在军事上的强势,有很大一部分来自于远远超过西北二虏的钢铁产量,听到辽人入寇,依然能维持着前所未有的安心,也就能够对此释然了。

范纯仁当年就任信阳军,曾经特地去见了在方城山修轨道的韩冈一次的。孙觉也与韩冈见过几面。不过两人对韩冈的感觉,依然是蒙了一层很厚的纱,完全看不透。

只有一点可以确定,在新法的推行中,有着汗马之功的韩冈,日后主持朝政时,也绝不会完全废除新法。变法即是国是,如今新法根基已成,韩冈不可能否定掉自己之前的心血,恢复旧日的祖宗之法

不过剩下的地方,不论是韩冈的目标,还是层出不穷的手段,又或是对新旧两党的看法,都让人捉摸不透。

相对而言,还是自己的想法最容易明白。

孙觉昨夜就问了范纯仁对新法的看法。

经过了这么多年,遍历州县的范纯仁在地方上也看到了很多。

在他看来,新法推行有好的一面,也有坏的一面,当然不会像新党所说的一样,是亿万生民欢呼鼓舞的德政,却也不能全盘否定。

若是自己来的主持朝政,只会是合人意者留之,不合人意者去之。不会因为是新党所倡导,就全盘敌视。

最少最少,现在实行的役法还是比过去的衙前役要好很多。

差役法伤民之处惟在衙前,纵是富民,一任衙前,也往往破产。而雇役法虽无衙前之累,但不须服差役之五等户及女户、单丁户,亦须出钱。

范纯仁昨夜这般对孙觉说道。只是平心而论,雇役法只消稍作修改,便能万民称便,而差役法却是积弊甚深,已是积重难返。

所以在新法一事上,不会与韩冈找不到共同点。

此外,还有新学的问题。

道统之争,是王安石与韩冈翁婿势不两立的主因。

虽然说很多旧党成员,包括孙觉、范纯仁都对经义另有见解,可相较于把持了士人晋升之路的新学,处于弱势的气学还是更适合的支持对象。

或许当日后韩冈主持朝堂,也会学王安石一样以私学为官学,但现在毕竟还没有。而且气学还没有在南方流传,北方士人若能早一步加以钻研,在日后的进士数量上,也许能够胜过南人。

春天的细雨冲刷着殿前的青石地面,从脚踝处能感受到上浮起来的清清寒意,不比冬雨的刺骨,范纯仁却还是觉得自己今天的衣服穿得少了。

应该多穿一点才是。范纯仁想着。另外,站在这里也许也太久了。

“范侍制,原来在这里。”

一名内侍远远地叫了一声,然后匆匆走了过来,看他脸上的焦急,可见是找了很久。

“怎么了?”范纯仁转身问道。

“请速去崇政殿,快要轮到侍制了。”

“这么快?”

范纯仁惊讶道。就算自己发了太久的呆,也不至于这么快。

难道今天没有多少事情需要太后与两府及重臣们商议?

只是想归想,范纯仁的双脚已经动了起来,跟着那名内侍,来到了崇政殿外。

走过来时,范纯仁看见李定和吕嘉问等人,连御史中丞和三司使都入内与太后禀报了今日各自衙署中的要务,如果今日谒见的顺序与平日相同,太后与宰辅们的议事早就结束了。看来的确是耽搁了许久。

在殿外通名之后,范纯仁没有等待太久,随即被招入了殿中。

出乎范纯仁的意料,殿内还留了一名宰辅——韩冈。

范纯仁抱着心中的一点狐疑,向屏风后的太后行礼如仪。

“范卿先坐下说话。”

向太后先赐了范纯仁座位,看起来对其很是看重。

“当初吾年幼时,听人说起本朝名相,就知道了范文正的忠节。不说范文正几次不顾,上表劝谏仁宗皇帝。就是西事,也多赖范文正。若非卿家之父镇守关西,当年西虏也不会被阻在横山之外,”

没有旧党中人传说中的刻薄,也没有另一种说法中的没有见识,范纯仁不知这是不是太后奖誉亡父后,自己心中激动后的错判,但他现在听到向太后对亡父如此赞誉,的确觉得她是女子中难得的英明。

连忙起身拜谢,范纯仁的双眼中已经有了酸涩。

待范纯仁回到座位,向太后又问道:“听说范卿与韩卿也是有渊源的?”

