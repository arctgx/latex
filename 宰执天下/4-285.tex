\section{第46章 了无旧客伴清谈(七)}

拉车的马,耕田的马,驮货的马,做邮递员的马,更差一点的,甚至只能作为肉用的马。

来自于民间的马匹,几乎全都是用于生产生活方面,军事上不用太指望。即便是上好的战马苗子,放在民间不要一年,基本上就会完全废掉。跟南方福建养在海岛上的州屿马差不多了——泉州、福州、兴化军的外岛上,总计有十来个牧场,但出栏的马匹,能做驿马都是好的。

毕竟如今已经不是五陵少年都能跨马游侠的时代了。

韩冈把看得让人生气的资料丢到一边去。

现在朝廷不论高下,是马就要——总有能派得上用处的地方,实在不堪用的大不了转卖出去——官员能得马三千匹便可转一官。熙河路转管马政的一干官员,一任之内,能接连迁转三五次,如果是熬磨勘的话,迁转一次可就要三年!在群牧司中低层官员中,最受欢迎的差遣就是在熙河和广西,换作是其他衙门,怎么可能会有这样的事。

因为这一条政策,群牧司的上上下下,基本上都在想着怎么弄马上来换官位。

虽然这些天,韩冈只去过两趟群牧司,但有些事多多少少还是听说了一些。下面的官员正在捣鼓着什么户马法,要求民户各计家产养马,坊郭户家产二千贯、乡村五千贯者,须养马一匹,家产增倍者,增加一匹,最多不超过三匹。

强制富民买马养马,这比便民贷的抑配还要糟。便民贷或者说青苗法的抑配,就是当常平仓中预备的贷款额度没有用光时,强迫不需要借钱的富户申请便民贷,由此强行取息,这是地方官为了追求政绩的结果。旧党拿着此事大骂出口,控诉便民贷扰民,朝中则是三令五申要禁绝此事。

现在强迫富户养马,而不是保马法的自愿申请,这等于是强制性的摊派徭役。而且是普遍性的摊派,至少是针对适合养马的北方,如开封府、京东西、河北、陕西、河东这几路的富户。不像是市易法,只针对一小部分豪商;免役法,收的钱对富户是九牛一毛,并让民间的中间阶层得以宽纵;便民贷更只是不让富户赚钱,决不是直接从富户口袋里面抢钱;就连手实法,从本质上也是让富户将隐瞒的财产公布出来,以便朝廷公平征税——能瞒家产的,总归是有势力的富人而不是穷人——在法理上是说得通。

富人应该承担更多的责任,这一点,韩冈是绝对支持的,可直接摊派,吃相未免太难看了一点,而且最后少不得会将罪名算到新党和新法头上,这一点更让韩冈觉得不舒服。吕惠卿现在困于手实法,再糊涂也不会节外生枝,真不知最后会是谁来接这个手。

应该不会是韩缜。

韩缜只在他这边试了一下口风,就被韩冈立刻顶回去了。不能乱来的,韩冈明说了,还是早点放弃的比较好。

而韩冈听韩缜的口气,发现他其实心中也有几分没把握,现在得不到自己的支持,多半是会偃旗息鼓了。

不过那些底层官员是如何的会钻营,韩冈再清楚不过,为减一年磨勘,杀人放火敢做的,想要他们就此放弃,绝对是不可能的。就是不知他们最终会唆动谁来上书。

韩冈叹了口气。王安石在台上的时候,还尽量想着要‘民不加赋,而国用自足’,现在上来的这一批,只顾着抢钱抢粮挣政绩了。

作为同群牧使,有关马政的事,必然会受到征询。对于户马法,韩冈不可能点头同意,肯定是要反对的,就不知道到时候自己能不能挡得住了。

就手拿过来一张白纸,韩冈将几处军马的来源依次写在纸上。保马法,青唐羌,大理,沙苑监,州屿,一个个都列了出来。看来看去,各有各的缺点,都是难当大用。

韩冈提着笔,皱着眉头,看着白纸黑字,盘算了好一阵,忽然就听见书房外有人在喊:“龙图!龙图!冯家四老爷来啦。”

韩冈猛一回神,从书房中走出来。就看见冯从义站在自己的面前。满面风尘,身上的斗篷都是灰蒙蒙的。

韩冈瞪大眼睛,惊讶道:“义哥,你怎么来了?我在京西收到你的信,不是说是过了清明再上京吗?是爹娘出事了?!”

冯从义正想行礼,却被一个劲追问的韩冈劈手抓住,忙道:“三哥放心,不是姨父姨母的事。小弟是在陇西听说了三哥你献上了种痘术,又听说七皇子因痘疮病夭,就立刻动身来京城了。”

“哦,原来是这样啊。”韩冈神色缓了下来,“让义哥你担心了,不过愚兄没事的。天子是明君啊,怎么会责怪愚兄?”

韩冈微微一笑,与冯从义进了书房坐下。

“啊……是,天子的确是明君。所以三天前走到洛阳,听说了三哥就任同群牧使,小弟当时就放心了。当时传了信回去,总不能让姨父姨母没办法安心过年。”冯从义笑说着,看见端茶上来的是五大三粗的汉子,问道,“嫂嫂和钟哥儿、钲哥儿他们还没有回来?”

韩冈道:“还要过几天才能到。你家的霖哥和大姐儿呢,还好吗?”

“都好,能跑能跳。三哥你弟妹如今又怀上了,再过半年就要生产。……如今有了牛痘,也不用担心痘疮了。”冯从义望望窗外,凑近了低声道:“三哥你既然身怀奇术,怎么不早点说出来。熙河路牛不缺,人不缺,要是早点吩咐人去找,说不定早就找到了,不定还能找出个马痘来。”

“哪有那么容易。”韩冈摇着头,“还说在熙河路找牛痘,马痘!根本都别指望,不是南方哪有那么多疾疫?人痘又太损阴德,说不定祸延子孙,怎么敢用?要不是愚兄在广西到了最后凑巧才发现牛痘,永远都不会提起人痘的事。找到牛痘后,愚兄也是先在京西试验过后才敢公诸于世。没个验证,贸贸然的谁敢拿自己儿女的性命当赌注?更不敢乱说啊。”

韩冈的感叹发自肺腑,牛痘哪里是想找就能找到的,可当真是快要绝望到准备拿交趾人制作人痘疫苗的时候,才碰巧在邕州横山寨发现了,“要不然早就拿出来了,钟哥儿他们也是才种上痘没几天。”

韩冈一番解释,是为了化解自家人的疑心,有些疙瘩得早些解开才是。

冯从义听了之后,正色点头:“原来如此,三哥所言极是。损阴德害子孙的事的确不能做,说不定,孙老神仙就是想看看三哥会不会去做这等恶事。若当真做了,多半会直接收了传给三哥你的仙方。”

这都是哪儿跟哪儿啊,韩冈简直是哭笑不得,不搭说胡话的冯从义的话茬,“朝廷新近成立了主管防疫救灾抚民的厚生司,这几天已经在开封设立了保赤局,专一负责种痘之事。种痘用的痘苗,也送去了熙河。”他声音也低了些,“其实半个月前,愚兄已经派了心腹人带了痘苗去陇西了,肯定是跟义哥你在路上擦身错过了。”

有好东西不先紧着自家人,韩冈可没那么穷大方。而且之前还瞒着种痘的事,怎么也该弥补一下。虽说跟朝廷送去的痘苗只是半月之差,但给人的感觉就不一样了。想必王厚、赵隆他们,甚至正好在路上的横渠书院众人,都能感受到他的诚意。王舜臣那里不顺路,但也派了亲信去。像在广西的李信,还有亲家公苏子元都有人带了牛痘去照应——他可不想外人救了,却把自家人给漏了,外人看笑话,自家可就是悲剧了。

前日韩冈向吴衍询问是否将痘苗送去熙河、广西两路,只是装装样子而已,作为牛痘的‘发明者’,他手上怎么会没有多余的疫苗。

韩冈的一番话,让匆匆赶来的冯从义,彻底放下了心头事。不顾仪态的伸了个懒腰,整个人都松弛了下来,冲着韩冈笑,“这一趟跑下来,都快赶上马递的速度了,都快累散了架。”

“谁叫你性子那么急。”韩冈的笑意温和,“方才已经安排人去准备酒饭了,待会儿吃过饭梳洗一下就好好的去休息,歇一觉醒来就好了。”

“好,看看三哥这里有什么好酒菜。”冯从义在交椅上扭了下身子,换了个舒服的姿势。

自家表弟惫懒的样子,韩冈笑了一笑,就当没看到,问道:“爹娘都还好吧?”

“都好得很。”冯从义道:“姨父领头捐钱建了一座普济院,正院供着药师王菩萨,偏院又供了李将军,请了当年秦州普救寺中的老和尚道安做主持,平常多去跟他聊天。隔三差五的还去看球赛。姨母平日里带着小弟浑家主持家务,偶尔也请两个说书的女先儿来家里。姨父虽然致仕了,但城里没人敢不给他面子,九月的时候,新知州上任,还亲自登门问好。”

