\section{第15章 焰上云霄思逐寇(16)}

李信慢慢的走在山道上。身前身后,都是他麾下的士卒。

渐渐向下的坡度让他走得很轻松,身上的甲胄仿佛没有重量。低头看着脚下,小心的走在泥地中。

前面的士兵突然停止了移动,李信停下了脚步,抬起头望着前方。

最前沿的战斗再次激烈了起来,厮杀声回荡在山间。刚刚被打散的交趾兵,重新组织又起了一道防线,奋力阻挡住了宋军继续前冲的势头。

一名身着皮甲的交趾军官就在那道防线之后,挥舞着头上的长刀,大声的在吼着什么。在他的指挥下,越来越多的交趾兵恢复起了些许勇气,再一次投入战线中。

李信瞅着他,冷哼一声,向身侧摊开右手,一柄投枪就递到了他的掌心。

右手掂了掂,使他熟悉的重量。深呼吸,雨后林间带着血腥的味道的清新空气,让他精神为之一振。左脚用力踏前,在跺出了深深的脚印。随着一声暴喝,他右臂奋力一挥,一道流光便从手中飚出。脱手而出的标枪划破空气,直奔敌阵而去。

那名军官似乎是吸取了几名前任的经验教训,在第一时间躲闪开来,让他身后的一名亲兵代他承受了李信灌注全身气力的猛力一击,胸部洞穿的倒在了地上。

不过交趾官的好运也就到此为止。在没有弓弩的情况下,宋军的战斗力少了至少五成。也许交趾人就是因为抱着这样的想法,才敢于一直压到昆仑关下。但宋军现在手上的弓弩虽然已经难以施用,不过他们还有投枪。

就在李信将标枪投出之后,紧跟在他身边的一队标枪手瞄准了同样的方向,掷出了他们手上的投枪。划着近乎一模一样的抛物线,二十几支标枪从天而降,军官在他身边亲兵的保护下正要向后撤退,可迟了一步的他们,如同被收割的麦子一样被放倒,浑身上下都是被沉重的掷矛穿透。

阵前一片呼声,指挥前线防御的军官再一次被李信一举击杀,宋军这边的士兵们无不是士气大振,纷纷向前冲去。

“第几个了?”李信恢复到早前的步速,慢悠悠的问着。

“回都监,是第七个!”亲兵很是兴奋,拼命提高嗓门的回复着,让李信的功绩传遍全军。

“才七个啊……”李信难以接受的摇了摇头。神枪出阵后就已经解决了七个交趾军官,连同他们身边的亲兵都一起,但他还是不满足。

对血的饥渴燃烧在他胸膛中,只嗅着战场上浓重的血腥味,就让李信一下兴奋起来,再杀十个八个才足够!

亲自上阵厮杀基本上都是低层军官们的任务,到了指挥使之后,就要开始指挥全军。而坐上都监的位置,基本上就不会还有厮杀在第一线的机会。李信如今还能亲自上阵,斩将夺旗,还得多亏了这一次身边只有区区数百人的缘故。

得多谢自己有个好表弟了,李信想着。看着前阵,他下令道:“廖四,带着你的人把程宗尧替下来。”

“末将领命!”

从阵后冲上来一队生力军,绕过李信身边的一群标枪手,直抵最前线。代替了体力消耗过大的程宗尧所部。这对交趾兵是百上加斤,受到了廖四所率领的百名精锐的冲击,本就因为再一次失去了前线指挥官而节节败退中的交趾人,再也抵挡不住宋军的冲锋。这一道、连同下一道的防线全都在一瞬间被冲垮,如同突破河堤的洪流一般,将所有挡在面前的障碍一起扫平。

狭窄的山道上,少数精兵的作用远远超过兵力上优势,交趾军的前阵不过刚刚抵达关城前一里的地方,还没有稳住阵脚,就被从关城中杀出来的宋军将士一举冲垮。

顺着倾斜的坡道,猛冲而下的宋军势不可挡。投枪手在其中立下了大功,尤其是被簇拥在后的李信,虽然他出手不多,但每一击都是盯准了在最前沿指挥抵抗的交趾军官,指挥这一支队伍的将领几次试图稳定战线,但在前线上的军官屡次被精准的标枪击杀击伤,混乱中的军队根本无力抵抗。

廖四带着他手上的一群的步兵呼喝着,冲上前去。他手中的长枪锋利异常,在扎穿了七八人的胸口之后,枪尖依然闪亮。无人能抵挡得住这样的冲锋,但返身逃遁的下场就是会被身后的敌人一一刺杀在拥堵的道路上。更多的交趾兵选择了冲进两边的山林中。只有这样才能躲开宋军如同解牛刀一般犀利的反击。

李信重新恢复了慢悠悠的步速,跟随着他的士兵继续向下追逐着败退的敌军。

“都监,前面就是贼军在小石坡上的营寨。”

转过一道弯,李信就看见了一座正搭到一半的营寨。营寨的位置离开昆仑关只有四里地多一点的样子,从大央岭驿进兵的交趾人,就以一座石头坡为中心,设立营地。当看到宋军反冲而来,把守营地的一支队伍,已经提前排下阵势,等待着宋军的到来。

要想攻城,除非有把握一举突破城防,否则就必须在贴近城池的地方设立营地,这样才好让攻城的将士们得到充分的休息,并给城上更大的压力。攻打昆仑关城没人会去幻想能够一蹴而就,交趾军当然也少不了逼近到关城近处就地设寨。

但韩冈和李信如何会让李常杰在离关城四五里的地方设立营寨?看透了李常杰潜藏于表象之下的真实计划,他们首先要做的,就是给幻想着昆仑关城中分兵宾州而不敢出兵作战的李常杰,一盆当头泼去让他清醒的冰水。

“杀!”

廖四将手中的长枪一摆,毫不畏惧的继续向下冲过去,就算前面有着千军万马,他也一无所惧。

倚城而战是守城的铁则,困守城墙那是难以力敌的无奈之举。在敌军来攻的时候,守军只要有余力,都会立刻出兵进行迎头痛击,以遏制敌军的汹汹来势。

李常杰对宋军出战有所预料,对自己的派出去的前军实力也算了解。就在官道上,连着排下三道拒马鹿角,中间只留了一丈宽的缝隙让自己人逃过来。

宋军紧追不舍。逃在前面的交趾兵,顺利的穿过了拒马防线,但还有百十名士兵,被堵在了最后面,在宋军畅快淋漓的砍杀中发出凄惨的哀嚎。

守在拒马鹿角之后的交趾弓箭手,纷纷张弓射击,尽管他们的长弓威力同样很小,但太过接近,依然有受到致命伤的可能。李信舍不得让他下面的士兵受伤,一声号令,正杀得兴起的将士停了手,在交趾人的面前耀武扬威一番,洋洋得意返身回关。

‘看来还没有到。’疑惑积蓄在李常杰的心中,难道他派出去的那一队人马出了什么意外?就算是李常杰都没有想到,宋军竟然会在分兵之后,依然选择了直接出城逆袭。

“继续扎制鹿角,将昆仑关下来的官道堵住,只要在昆仑关下扎下营盘就够了!”

李常杰的打算就是给昆仑关中一定程度的压力,让城中守军的注意力集中在他的身上,剩下的就交给两支偏师去完成。有着看似行之有效的策略,他怎么会冒着士气大落的风险去全力进攻?而且在雨中驻扎在简陋的营地里四五天,又要沿着烂泥道路正面进攻一座关城,李常杰也很清楚这样只会浪费宝贵的兵力。

‘很快就会有好消息传回来了!’李常杰想着,只要从横州绕过去的七千人到位,前后受敌的昆仑关根本无从应对。接下里的几天,慢慢的将拒马鹿角的防线向昆仑关移上去,只要保持着对关城上的压力就够了,不需要费气力与那几百名荆南军硬顶。

出战的李信率军回到昆仑关中,等候已久的韩冈亲自为他奉上胜利的美酒。

“好了!”李信将银杯中的美酒一饮而尽,同样用酒遍赏过出战的将士,回到正厅中,“歇一下就该出战了。三哥儿,你这里真的不要紧?要不要再多留一个都。”

“有黄全的两千兵驻守昆仑关,抵挡住李常杰没有问题。至于我身边,留下一个都装装样子就足够了。”韩冈正色对着李信道:“剩下的五百人就交给表哥你,要尽快与黄金满和苏伯绪会合上。”

“末将遵命!”

这才是真正的出兵。

方才出城作战是要打下交趾兵的气焰,争取三五天的空当,将绕道横州的交趾偏师给解决。而昆仑关这里,接下来的几天,就借着方才出战告捷的余荫,压着交趾人的攻势。

尽管两边兵力差得太远,但幸好敌军为了攻下昆仑关分了兵。在这样的情况下,只要能够各个击破,胜利也同样能抓到手中。

批亢捣虚,相比起李常杰这一部,那支偏师更容易解决。

韩冈重新让人送让一杯酒,双手奉给李信:“小弟就以此杯预祝表兄马到功成!”

