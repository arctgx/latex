\section{第一章 一年穷处已残冬(上)}

上六:龙战于野,其血玄黄。

******************

“九哥,看来是没指望了?”看着成九失落的从总社中出来,张胜就迎了上去,“怎么样,小弟说得没错吧?”

成九带上斗笠,挡住天上的飞雪,闷声闷气的嗯了一声,没精打采的踏着积雪往回走,没好气的说着,“嗯,是没错。都要禁个干净。”

“没说其他的了?上皇死得蹊跷。”

“再蹊跷也有韩宣徽看着,能做什么鬼祟?这回被烟呛死的人还少了?”成九不屑的撇撇嘴,便又没精打采起来“本指望年前能翻个本还酒帐呢,这下子全都完了!”

“不论翻本不翻本,九哥你都少不了被九嫂夜里罚顶夜壶。”

“扯你娘的淡,我成九什么时候怕过那婆娘了?都是让她的!”

“前天躲到李三哥家里也是让她?”

“当然。”看着张胜似笑非笑的表情,成九脸皮红了一下,用力咳了两声,愤愤然说道:“上回曹太皇上仙,就逼得改时间。这一回又是百日。哈哈,都别玩了!”

蹴鞠联赛的赛季如今都是跨年,尤其是腊月正月的时候,季后赛以及正旦大比,已是与鞭炮、桃符一样,成了东京龘城中过年少不了的风景线。之所以从年底结束变成现在这般,就是当年慈圣光献曹皇后上仙,依天子例天下禁乐百日,顺便把京龘城的两大联赛也禁了。

皇帝要恪行孝道,这顶大帽子压下来,没人敢触霉头,两大总社的会首们一个比一个乖觉,老老实实守了百日的丧,才重新开始比赛。赛马还好说,少比赛几场不影响结果,但蹴鞠联赛赛制摆在那里,一轮轮的比下来,再赶时间也不可能将三个月的空缺给补上,只能顺延下来,不跨年的联赛变成跨年的。

而这一回,又是三个月拖下来,还不知之后的赛程会怎么改。总是拿中奖后的奖金做零花钱的成九哪里能不抱怨,他还有每个月从蹴鞠总会拿到手的一笔月例收入呢。。

“华阴侯身边的赵虞侯前天都说了,两家联赛肯定要暂停。上皇驾崩这么大的事,能不停办吗?九哥你偏不信。”

“谁说不信的,不就是报个万一的念想吗?”

“九哥,小心后面!”

正偏过头跟成九说话的张胜突然一声大叫。

成九立刻向路边闪开,回头看过去,有四五人骑在马上,正好就在身后。

只不过他们虽骑着马,速度却不快,隔着也还有三四丈,实在不用大惊小怪。

“咋呼个什么?没撞到,倒你被吓到了。”成九反过来抱怨张胜。

张胜呵呵笑,扯着成九让路。

几人中间的那个领头的骑手,冲着张胜、成九点点头,似是感谢他让开路,看起来和气得很。

张胜松了一口气:“这是哪家衙内,这么好脾气?”

“马不行,肯定不是大户。”成九摇头。

领头的骑手像是个有身份的,但要是奢遮一点的大官,喝道的隔着老远就会叫得比老鸹叫还吵。

而且几个人骑得都是四尺五不到一点的河西马。如果在十年前,绝大多数的马军指挥使还没这么好的坐骑。但现如今,没匹近五尺的大食马,这还叫京里的衙内吗?出门都不好意思跟人打招呼。

“幸好不是。”张胜叹道。

要真是高门子弟,撞了也就撞了。人活着赔点汤药费,人死了给点烧埋钱,也就这点赔偿了。

天下官官相护,将事情瞒下来的还是占大多数,会被捅出去的都是因为自家的老子或是叔伯开罪了人。

“对了,前日小弟去喝酒的时候,听了宣翼军的李都头说了,火器局年后就要挑人去操练火炮,在京各军,但凡禁军,愿意转去神机军的都可以去报名。”

“这事俺也听说了。”成九点点头,不过他一贯的好逸恶劳,不想去掺和,“一日双操,十天才能歇一天,你吃得消?”

“俸料钱给得多就行。而且这还是新军额,做得好了升得也容易。九哥你不是说要还账吗?日后花用也多,总得多赚点。”

“让俺再想想。”成九犹豫着,去了就有钱,说不定还有权,只是那份辛苦让人吃不消,“让俺再想想。”

张胜不催他,反而掀开了斗笠,望着天空:“雪停了。”

……………………

“雪停了。”

韩冈出来时天上还有些米粒般的细雪,不过现在终于是停了,天空灰不灰白不白,也不知什么时候还会再下起来。

烧了三天的石炭场大火,昨天就熄灭了。

不过并不是人力,新任权知开封府沈括指挥京龘城军民做的,仅仅是防止火势蔓延,真正灭去大火的,还是从前天夜里开始下起的大雪。

飞雪到了火场上,顿时就化为了雨水。水火相激,立时便是烟雾弥漫,火势反而更甚,给救火的工作带来的了不小的干扰。从前天夜里开始,到现在近两天的时间,东门外都是迷雾锁城。

但终究是水克火,雪下的多了,水就积起来了。慢慢地就将火势压下,然后一点点的熄灭了。

雪是灰色的,落到地上很快就堆积起来。空气中的烟味,被洗去了不少,只是萦绕在鼻端的淡淡味道还提醒人们,前几天到底发生了什么。

这一场火灾中,直接因火灾而丧生的军民超过两百,受伤的更是数倍于此。

也就在这几天里,京龘城乱象频生,开封府抓住的盗贼有百十个,因各种各样的原因被当场处决的贼人,也有那么十几名。

同时为防止火势蔓延,石炭场周围被拆掉的屋舍大大小小有上千间,因此而流离失所的百姓有四五千之多,沈括为收拾残局,忙得焦头烂额。

不过这一场大火,留给世人最深刻的记忆,不是这几百死者,数千灾民,而是另外一人。

起火前的太上皇,起火后的熙宗皇帝。

一场大火,将离着火场三重城墙的太上皇给顺道带走了,不能不说是让人匪夷所思的一件事。

有人在庙号谥号中找寻那微言大义,整日琢磨着宰辅们到底是什么打算。有人则干脆认定了有阴谋,背地里痛斥宰辅不能匡扶社稷。

但更多的士子想要问,今年是不是还照常开科取士。大部分人对赵顼大行的原因并没有太多的非议。

因炭气满门死绝的传闻并非是市井传说,而是年年都有的常例,百来人中总有一两个知道的。何况这一回还有更多的人因浓重的烟雾而发病,其中急症不治的也超过一百人了。赵顼在其中,不算特别,只是事后的影响力有别于普通人,不仅是驾崩,还连带着将天子的形象给拉下来一大截。

这些是韩冈能够预料得到的,只是没想到东京军民会这么快就接受下来。

不过这些暂时也不管他的事了,到了目的地,韩冈翻身下马。

司阍出来迎接,韩冈将缰绳递过去:“岳父今天好点了吗?”

“相公已经好多了。”

王安石自离开宫中之后,便因伤感而卧床不起,并请颇重,韩冈心中担忧,这两天都登门探问。

他与司阍一问一答,没有问上几句,王旁就已经赶出来迎接了。

看到韩冈身边就几个人护卫,王旁顿时变了脸,“玉昆,怎么就带这点人?”

“现在是无官一身轻。要那么多人作甚?”

赵煦犯下大错,有心也罢,无心也罢,以赵煦的年纪,王安石和韩冈作为天子的老师,都不能辞其咎,必须要对此负责。

韩冈由此引罪,上表辞了所有差遣、并请降本官、散官、爵禄等一应名位,就是资政殿学士这样的贴职都放弃了。

虽然向太后还没批下来,但韩冈已经不去宫中,连紫章服、金鱼袋也都不再穿了。

王安石和程颢也跟韩冈一样,都上表辞去了经筵官为首的一应官职,放弃了差事。

所有经筵讲官,无不如此。一日之间,原本阵容强大的天子教育团队,现在只剩个牌子了。

韩冈这两日出外,都是不穿公服,不举旗牌,轻车简从。身边跟三五个伴当,骑着的河西马,走在街上一点也不起眼。真要说起来,在京龘城占了几条街的那些自号大侠的泼皮头子,排场还更大一点。

“玉昆你路上还方便?”

“走小路,人不多,车马也不多。不过路上也不见多少积雪,沈存中在这个位置上,可算是适任了。”

“只要不要耽搁桥道顿递使差事就好。”

韩冈笑道:“以沈存中的才干,还不至于这点事就手忙脚乱。”

正常天子大行,梓宫奉入山陵之前,必须先要整修官道。桥道顿递使一职就是开封知府推卸不掉的责任。沈括为了表现自己的才干,也没有推脱,现在是脚不沾地,但终究是游刃有余。

“今天还有别人来?”韩冈边走边问。

“章子厚来了,父亲正在里面见他。”

韩冈在内院前停步,只看见章敦正从里面迈步出来,看见韩冈,眼神倏然转利。

这是几日来韩冈第一次见到章敦,与王厚同迎了上去,见了礼,章敦便告辞先行,经过韩冈身边时,丢下话来,

语气冷且硬:“玉昆,待会儿我有话问你。”

这叫什么……兴师问罪?

韩冈暗叹,也该有这一出。
