\section{第44章 岂惧足履霜(上)}

韩冈的话,郭逵仅仅是报之一笑。这等信口的恭维,他听得太多了:“玉昆的话也说得不能算错,老夫去了太原是为了让天子心安,但也只是让天子心安而已。”

郭逵如此坦率,到让韩冈深感意外。叹道:“能让天子心安已是足矣。如果年中时,朝中文武能让天子心安,又岂会有代州割土之事?”

“木已成舟,此事就不便多说了。”

郭逵其实这两年坐镇关中,渐渐的也熄了功名之心。若是当年换了现在的心境去秦州,多半就不会起意与王韶争夺开拓熙河的控制权了。就算争来了机会又如何?得了功劳,朝廷的封赏他又如何敢要?

现在的官场上,郭逵作为武将,几乎已经走到了可以到达的最高点。虽然上面还有一个枢密使,但他若是当真做了这个职位,当即就是狄青的下场。别说真的坐到了西府中的主位上,即便起了一点心思,又或是天子露出一点意头,文官们都绝不会饶他。

郭逵在大相国寺的内廊中慢慢走着,“玉昆你如今判军器监,老夫倒是盼着玉昆你能在军器监有所成就。疗养院、霹雳炮、雪橇车,还有军棋沙盘,都是发前人所未发,任何一项都不输于神臂弓。若是,使得甲坚枪利,军中所用无不精良,只要稍作校阅,中国军力必当能震慑四夷。”

“韩冈的确打算在军器监做出一番功业,也有了预想。只是如今尚未见功,不敢呈于宣徽。”

郭逵回头瞥了韩冈一眼,眼神中的锋锐丝毫不减当年:“素知玉昆你言不虚发,有你这句话老夫就放心了。”

就在大相国寺内,郭逵使人定了一桌上等的素斋,邀了韩冈坐下来一起吃饭,韩冈很爽快的就答应了下来。

原本智缘准备请韩冈一起吃饭的,但宫里来人将他传了宫进去——曹太皇最近身体不好,御医的手段不见成效,需要向外延医问药,另外又要让京中的僧人为其念经祈福。智缘这位身着紫衣、在河湟蕃部中为大宋招揽人心数载的名僧,不但医术名满京中,又是左街正僧录,自然是第一个被点上。

一餐宾主尽欢,吃完之后,闲聊片刻,韩冈便起身告辞,郭逵也没有多留他。

韩冈与郭逵不可能走得太近,他也没必要与郭逵走得近。

郭逵只要不犯文官忌讳,谁也动不了他。他外面有着个贪于财货的名头,其中有几分为真,又有几分是以秦将王翦为榜样,外人都无从得知。但韩冈与郭逵太过接近,却会引起士林的议论——士大夫难以容忍一个投效武夫的士人——这对他的名声不利。尽了人情就行了,君子之交本就疏淡如水。

辞了郭逵、郭忠孝父子,韩冈离开依然熙熙攘攘的大相国寺,带着一众伴当上马返家。

回到位于旧城右军第一厢的常乐坊的家中,却见章惇正坐在偏厅里,冯从义下首陪客,另外一名客人则是很久不见的路明。

见到韩冈走进来,章惇也不管着厅中还有冯、路二人在场,劈头就道:“玉昆,你好悠闲!”

韩冈依然悠悠闲闲,跟路明打过招呼,坐下来问道:“不知出了何事?”

“何事?”章惇都为韩冈发急,“就是你太悠闲出的事!”

论起知情识趣,察言观色,商人不会比官员差上半点。见着章惇的口气不对,冯从义和路明立刻找了个由头,便一起走了出去。

章惇对于朋友,算是掏心窝子的性格。苏轼经常因为乱说话而得罪人,章惇就时常写信去告诫。他与韩冈的交情虽然参杂了许多政治利益上的成分,真说交情还没到推心置腹的地步。但韩冈的为人行事,章惇很是欣赏。过去两人互相帮了不少的忙,政治利益紧紧相连,现在眼看着韩冈的态度被吕惠卿所疑忌,便不能不为他担心。当然,也是怕着让人渔翁得利。

章惇知道韩冈自有盘算,乃是按照预定的步调在走,但别人可不会按照他步调来行事:“玉昆。若是别人判军器监,天子绝不会有多余的期盼,只要能看到军器精良就够了。但你可是在天子面前亲口许诺,要在军器监一展长才,现在半个月不见动静,连封文书都不发,天子难道会没有想法?!”

韩冈早是胸有成竹,章惇的焦急一点也没传染到他身上,只是在风清云淡的笑着:“韩冈一早也说过会萧规曹随吧……”

韩冈轻描淡写的态度,弄得章惇仿佛是皇帝不急太监急,心头怒意上涌:“玉昆,我不会问你到底打算做什么?只是想你早一点有所动作,至少让天子能看到一点东西。否则以天子的心性,不免会认为是有人在暗中阻挠你行事,吕吉甫也免不了会以为你现在的安静是在针对他。还是说,你当真有此心意?”

韩冈一笑,知道吕惠卿多半是有些受害妄想症,对自己猜忌过甚,也许转了年过来,他就要找个由头来整治自己了,以便将祸患提前给排出,故而才惹得章惇如此火急火燎。不过也有可能是吕惠卿故意摆出要针对自己的姿态,好引得章惇过来探底,至于章惇,或许也有顺水推舟的成分在。

可不管是什么情况,韩冈的计划无可不对人言,本来就是阳谋,无人能挡得了,并不需要多猜测对方的心思。随即站起身:“请直院随韩冈来。”

章惇半带着疑惑,随着韩冈一路走到书房中。

分了宾主落座,章惇打量着房内。韩冈书房的布置十分朴素,并没有多少摆设,仅仅用石灰粉了墙壁。房中的藏书也并不算多,刚刚摆满了一边墙壁的书架而已。靠着窗户的书桌,则是摆着文房四宝和几册书卷,整理得十分整齐。且又有淡淡的幽香漂浮在房中的空气中,这不是薰香的味道,而是女子所用的香粉味道,看起来韩冈红袖添香夜读书的生活,过得很是惬意。

只是在房中的圆桌上,却放着一个木盆,大小像是用来洗脚的。出现在书房中,让人感觉很是别扭。而盆中还盛着水,水面上飘着一块木头,还有一艘雕工十分粗糙的小木船。

“这是?”

看见盆中的木舟,章惇就想起了韩冈对他说过的话,那个‘船’字是不是就应在这里。

韩冈拱了拱手:“韩冈想请教直院,不知直院可知为何木舟能浮于水上?”

章惇知道韩冈不会白白发问,左思右想却想不透韩冈问话的用意,以及陷阱何在,犹犹豫豫的说道:“因为木头比水轻……”

“说的没错。不过确切的一点说,应该是同样体积的木头要比水要轻。不能说这张桌子,比盆里的水要重。……固定体积的重量,我称之为密度。比如说一升水,一升银,一升铁,一升木头的重量都不一样,也就是说它们密度都不尽相同。”

对于各种单位的定义是物理学的重点。重量、质量的差别暂时还不便提出来,但密度、速度等单位,就必须加以明确定义。

章惇听着点点头,虽然没有完全明白,但大体意思还是了解了,“也就是说密度比水轻的会浮在水上,而比水重的,会沉在水底?”

“正是这个道理!石头密度大于水,所以沉于水底,而油密度小于水,故而浮在水面。”韩冈很欣慰的说着,他这两天给妻妾灌输密度的定义,可是费了一番功夫。不比章惇,说了就明白了——自然,其中也是因为有了经验的缘故。

韩冈拿起桌上的一个小银碗,丢进盆中。只见着银碗浮在水面上飘飘摇摇,“现在问题来了。银的密度远比水要大,也就是同样大小的银要比水重得多,那为何银碗能浮于水上?”

“……银碗中空,压平了就沉水了。”章惇沉吟了一下,方才给出了回答。抬眼反问韩冈,“此一答当是人尽皆知。”

“的确,银碗能浮于水上,就是因为中空之故。所以将银碗改成铜碗,也当同样能浮于水上。”

“自是当然。”章惇的回答越来越干脆。

韩冈点了点头,又问道:“如果换成铁呢?”

“铁?铁碗……不对,是铁船!”章惇终于反应过来,猛然间蹦起,目瞪口呆的指着韩冈,“玉昆!你这是要打造铁船?!”

“只要算准了船只的自重和尺寸,行驶在水上的铁船也的确能造得出来。不过这仅仅是一部分而已,辨明了其中的道理,能造的东西多了,可不仅仅是铁船。”韩冈看着章惇的目光宁宁定定,“直院可知其中道理何在?”

章惇坐了下来,沉声道:“玉昆,你就别卖关子了,直说好了。”

章惇对韩冈一心倡导格物致知之说的坚持,其实也算是挺佩服的。当初韩冈在御前亲手验证了轻物重物同时落地,将格物之学搬上台面。章惇在荆南听说之后,对此也生了兴趣。但当他回去对着院后的一从竹子看了一个晚上,怎么也格不出个眉目。竹子随风而摆,吟诗作词不难,可换成是格物,却到底要格个什么?章惇想不出来,脑筋也始终转不过来。

韩冈倡导的学术,看似平平常常,平日里都随处可见,可只有说破了才让人恍然大悟。章惇已经放弃了在这上面花费时间和精神,他要做的事太多,可没有韩冈分心多用的本事。

韩冈微微一笑,将摆在桌上的一叠绢纸装订而成的册子递了过去,封皮上只有简简单单的四个字——《浮力追源》。

