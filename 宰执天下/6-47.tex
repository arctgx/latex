\section{第六章 见说崇山放四凶(14)}

文彦博日常起居的小园院中多了一群人。

他们全都围在院中龘央的一株高达四五丈、数人合抱的桂树下,被掘出的一个土坑周围。

土坑有一丈见方,最深处有五六尺,桂树的半边树根暴露在外。

刚刚从坑中上来的管家一身的土,“相公,这树根子还是好的,肯定能再抽枝。吕三?”

还在坑里的园丁吕三连忙点头:“对!对!相公,根子还有些青色,最好还是再等两年看看。这枯树发芽的事常常有。”

文及甫在旁边看得清楚,根子从皮到芯全都干了。不只是树心有了空洞,就是表面上的皮也坏了。

这株老桂,夏天叶子落光,秋天也没有开花,本想赶在开春确认一下病灶,好进行处理。却发现已经完全死了。

“大人?”

这是文彦博很喜欢的一株老桂,当初文彦博买下这间宅院时就已经在院中。之后改建的时候,也没有将这株历史和时代不明的桂树给砍了,而是以桂树为核心,在后园为文彦博建了一座小院。

文彦博对此极是喜爱,亲笔题名作桂园,还在主楼上题了个与月同馨的匾额。这两年,文彦博大多数时候都住在桂园中。

文彦博珍爱的老树病死,看着老态龙钟的老相公,每个人表面上都若无其事,但每个人心里都在念着‘不祥之兆’四个字。

“……砍了吧,留着也碍事”

文彦博面无表情的起身离开,留下文及甫与众人面面相觑。

“这……”管家为难的望向文及甫。

“……先留着不动,再等一等。”

文及甫也不清楚文彦博是不是说着气话,左思右想了好半日,才丢下话转身追过去。

片刻之后,他在家里的玻璃温房处,找到了正靠在椅上晒太阳的文彦博。

用玻璃拼出的透明窗户,尽管已经在高门大户中流传开来。文、富这一等的元老重臣,各家几乎都换掉了旧有的用纸或纱糊起来的窗户,而改用了更为透亮也更能遮挡风雨的玻璃窗。

不过顶部完全使用玻璃建起的温房,技术难度比单纯的玻璃窗高了不止一个等级。目前平板透明玻璃最大不过一尺见方,而且是要靠运气。且就算能建起来,也很难保证度过春夏秋冬的四季变化。所以当不知何处传出有人想要造出一间连墙壁都是透明的房屋,并早早的提名为水晶宫,便惹来许多人的嘲笑。

不过富弼和王拱辰两家还是修建了一座玻璃温房,让两位元老能够在里面安稳的晒着太阳。大不了隔三差五就给屋顶换一套玻璃,对普通富户算得上是难以想象的奢侈,在元老们的生活中,自出现后就已经成了必需品。

冬天出来晒太阳最舒服不过,可年纪大的人多吹了一点寒风,就很容易生病。熬不过冬天的老人,这世上很多,前几年的吕公弼便是一点外感小疾,然后暴毙。既然有能让元老们安然的享受着冬日阳光的玻璃温房,又有什么理由不让他们用上?所以文彦博等其他元老也跟在富弼、王拱辰之后,将玻璃温房给修了起来。

宽厚的毛毡披在腿上,文彦博正闭着眼沐浴在阳光中。光线透过无色的天花照射下来,让室中变得温暖如春。温房中有数十本畏寒的花木,都是市面上见不到名品珍品,在在此处却探手可折。

文彦博显然对名品花木不感兴趣,听到儿子过来的动静,文彦博忽然开口:“砍了吗?”

“大人。”文及甫小心翼翼的劝着,“还是等一等,说不定过几日就能看到新枝了。”

“新枝?”文彦博依然闭着眼睛,“死了就死了吧。当年买下这座院子的时候,也没指望能一直养活。”

文彦博饶是如此说,但文及甫知道,文彦博最后会选定买下这座宅院,就是因为这宅子里面的各色花木让他父亲十分中意,而当时正逢花期的这株老桂更是起到了决定性的作用。

月下丹桂怒放,宅中皆浮动着醉人的甜香。这比经历过多任达官显贵,藏下窖金的几率近乎百分百的宅子更让人觉得物有所值。文及甫当时的想法,也是如此。

洛阳乃千古名城,唐时为东都,深宅大院不计其数,位置好一点的宅院,往往都有数百年的历史。

在洛阳,经常能听说有人在翻修宅院时,从地下挖出一坛金银,或是数千贯钱币。也有杂剧中演,拿着做为本金去行商,又或是买了田地来个晴耕雨读,由此考中了进士,从此浑家有了,房子有了。

不过在文彦博、文及甫父子看来,地下挖出的窖金再众,也不如一颗老树来得让人欣喜。

可这株数百年的桂树,成为文家所有不过数年,便已经化为枯木。

“怎么还不去?”

文彦博没听到儿子的动静,终于张开了眼。

“大人……还是再等等,说不定……”

“什么说不定?天下万物皆有其寿数。寿数到了,等也无用,难道还能再回魂?为父也没多少时间了,寿数亲等桑拿倘若当真能如此,为父倒是有许多人想要再见见,再问问。”

每一个的冬天过去,文彦博过去熟悉的朋友、敌人、上司、同僚、下属都会少掉几个。当然,失去老相识的季节,也包括春天,夏天,秋天。

多活一年,对这个世界就陌生上一分,这就是每一位长寿者都要面临的问题。不过文彦博从来没有觉得这是一个问题,能活得长久才是赢家。

论起寿数,文彦博是赵顼的两倍还多。英宗、熙宗先后两位天子,加起来也没文彦博一人的寿命长。

文彦博早就不去求神拜佛了,在他看来,能活这么久,就是纯粹的天命——清醒明晰的头脑可以作证,换作是其他人活到他这个岁数,早就老糊涂。

嗯,没错,就是富弼那样。

“听说富弼老得都开始犯迷糊了?”

文彦博突兀的问话,让文及甫完全反应不过来。

“啊?……儿子没听说。”

“不是说他想要跟韩冈结个亲?”

“的确有这件事。”文及甫点头,“但韩冈这不是连宣徽使都没得做吗,富府大概是想要雪中送炭。”

而且之前文彦博还让家中的子弟研习气学,怎么现在韩冈一出事就立刻有了反复。只做锦上添花,却不去雪中送炭,文及甫怎么想都觉得自家的老子才是犯了糊涂的一个。

从儿子脸上的表情中,寻找了他心中想法的蛛丝马迹,文彦博的脸色立刻难看起来。

“让你们学气学,可不是去巴结韩冈,是为了日后考进士,免得遇上气学题目措手不及。”

在文彦博看来,让自家儿孙去学气学,那连志同道合都算不上,既然气学有成为显学的可能,那么让子弟去接触一下也并非坏事。万一日后气学拾新学之故技,将进士与气学挂上钩,那时候,难道要干瞪眼不成?

尽管对已经完全与五经拉不上关系的气学懵然不解,可文彦博就从这里得到了结论。气学是必须要去认真钻研的,否则很快就会看不懂《自然》中的一篇篇文章。

一旦气学入主进士试,就绝对不可能像旧时经义转变到新学上那般轻易,没有多年功夫的浸淫,看到考题也会是一头雾水。文彦博这也是在为家中子弟考虑。而且所有道理都是通过格物来验证,将实验放在最高的位置上。这对学生们的财力要求更高,对高门大户出身的士人也更为友善。

只是说起对韩冈的态度,文彦博觉得自己是始终如一。

而富弼那边却是恨不得将脸给贴上去,连孙女也舍得丢出去套狼。

文彦博一肚子冷嘲热讽要宣泄出来,但午后的阳光下,一件来自京龘城的紧急情报让文彦博猛地跳了起来。

“大人,大人!”文及甫惊出一身冷汗,“要小心,千万要小心!”

“慌什么。”

文彦博随即很不耐烦的说道,只是心中还是在为韩冈在殿上的神勇惊叹不已。

匪夷所思的平叛手段。亘古以来未曾见。

文彦博又不屑的撇着嘴。蔡确一伙还真是无能至极,都控制了朝堂,还能给他输了。

还能让那个灌园小儿上殿?另立新君,群臣仓促进拜,这等时候半点异声都不能有,像韩冈这样肯定会大闹朝廷的人,直接就在宣德门就拿下了。

既然已经让太皇太后垂帘听政,又拿到了国玺,难道写封诏书捉拿逆贼韩冈就那么难?!

若是想拿韩冈的人头立威,那就更是蠢透了。当韩冈跳出来后,王安石、章敦肯定不会甘于寂寞。

这么做的确要冒风险,但韩冈的危险性,难道不比这个风险大?当初文彦博只一个错失便被韩冈揪住,被逼着喝了十几盅消风散,从那时开始,文彦博便再不会小瞧韩冈。

在年轻一代中,韩冈的才干能力冠绝众人。文彦博纵然不喜韩冈,也不能不承认这个评语。

这一回,韩冈的又毒又利的眼睛,一眼就看清关键是在太皇太后和蔡确两人身上。

只是当庭挟持太皇太后太难了,危险性也高,不如直接杀了蔡确最为简单。不要太高的武力,有那份胆子比什么都重要。

