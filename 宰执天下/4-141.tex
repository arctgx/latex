\section{第20章 冥冥鬼神有也无(17)}

作为交趾北方的重要据点,门州城防工事的水准一直仅次于升龙府,是交趾最为重要的关卡要塞之一,加上地处要津,城中的储备也是甚为充裕。

但在宋军的攻势下,门州却是一战而下,城中的守备脆弱得难以想象,宋军顺利的登上城头,又毫无阻碍的占据了城池。

虽然不能说在看到飞船之后,守军的士气便崩溃了,但画着鬼面的飞船给予交趾士兵的心理打击是超乎预计的,已经习惯了看到飞船和热气球、并了解其原理的宋人,难以想象敬畏鬼神的四方蛮夷,对飞在天空中的异物的畏惧。

这一战中不多的伤亡,基本上都集中于城破之后,对城中守军的扫荡这一过程之中,理应最为艰难的登城,轻重伤加起来也不过十几人。

“交贼甚至连守城都不会。”

战后的第二天,章惇和李宪也赶到了门州。与韩冈、燕达一起坐在厅中。外面有一众行营参军正统计着各项数据,为这一战做着总结。

“是没有经验吧。”没有实际的经验,读再多兵书都没用,何况具体到攻守战术上的兵书,这个时代可不是想找就能找到的,韩冈一直都是看不起交趾人的战术水平,“一百多年了,除了太宗皇帝时被攻打过一次,其他时候,都是平平安安的,哪里知道如何守城攻城?相对而言,野战反倒是更擅长一点。”

“野战不还是没在玉昆你手上讨过便宜?”章惇冲着韩冈摇摇头,又哼了一声,“太平的日子不去过,偏偏要来找死。中国岂是他们这等蛮夷挑衅得起。自作孽,不可活!”经此一战,安南经略信心高涨,意气风发的笑着,“玉昆带着千五荆南精兵,和五千广源军,就大败李常杰十万大军。现在拥兵五千的门州城,我两千虎贲亦是一战即克,想那升龙府,也挡不住我大军一击之力!”

李宪也开怀笑道:“天子若是得知门州一战的战果,不知会有多欢喜!”

“门州一下,直至富良江边,沿途州县据说都没有城墙,就不用官军多费心了。得让一众蛮部多卖把力气,将交贼引过江来。”燕达亦是轻松的微笑,“想必此战之后,蛮部会更加卖力了。”

这一战,直接参与攻打门州城的宋军只有两千出头。此战刻意减少出战兵力,就是为了进一步震慑诸多蛮部,并给官军中的将校士卒们以足够的信心。

不过在外清扫山林中、道路上的斥候,阻截可能会有的援军,投入进去的兵力,有五千之多。而且还有好几支蛮部近万人跟随,他们也起到了同样的作用。稳定了外围之后,门州战场就变成了官军展示实力的舞台。

被吓到的绝不仅仅是门州城中的交趾兵,看到官军不费吹灰之力便攻破了门州城,被特意找来观战的一众蛮部代表虽然都是吃惊,但绝比不上看见飞天的神器之后受到的惊吓。在听说过飞船是韩冈所造之后,对韩冈的敬畏也为之更加重了数倍。

燕达还记得昨天一群蛮人来拜见韩冈,最后离开时,是挪动着双膝倒退到大厅门口之后,方才起身出厅的,“因为飞船一物,这些蛮人对副帅敬畏若神明,只要有副帅敦促他们尽快进兵,必定不敢懈怠。”

“岂是畏我?是这些蛮子怕鬼啊!换作是京城,有点手艺的自己都能缝出个气球来卖钱,谁会被个飞船吓到。”韩冈摇头笑了笑,“不过这样也好,有着鬼神撑腰,想必他们的行动可以更加大胆一点。”

外厅中,行营参军们的战果总结总算宣告完成,章惇的一名幕僚拿着一沓写满了文字的纸页递上来给章惇,他是经略招讨司机宜文字,也是行营参军的代表,“学士,这是统计出来门州一战的战果,包括斩首、俘获,俘虏的敌将清单,缴获的军粮、物资,一切都已罗列。后面还有军中的伤亡情况,详细的立功名单也一并附在最后。”

章惇拿着战果和战损的统计,专注的看了起来。

韩冈则抽空吩咐着:“另外还要早点将这一战的经验和不足给总结出来。即便是大捷,总有些不尽人意的地方。你们行营参军要多往下面去走走,去问问士卒、队正和都头们,问问他们觉得这一战有哪些地方不足,哪些地方需要改进。这些都是他们的事关性命,一般不会乱说话,要仔细聆听,用心记录。一日三省吾身,这行军打仗也要经常反躬自问。”

韩冈是行营参军们的直接领导,章惇的幕僚对他的吩咐也不敢有所轻忽,而且都是之前韩冈曾经说过的话,“龙学的吩咐,下官这就去做。”

只是他走出去的时候,向章惇望过去。章惇则没理会,低头看着手上的统计报告,“战死十一人,轻重伤五十六人,其中重伤十九人。斩首一千一百一十三,俘虏四千一百。”章惇将报告中记录伤亡的一页递给韩冈,“伤亡多在入城之后,争夺北门以外的其他三座城门的时候。想逃出门州的当有许多,但官军伤亡人数却只有这么一点,反抗的未免有些少了?将交趾丁口都处以刖刑的消息,难道还没有传到门州?”

“应该已经传遍了才是。”韩冈想了想,“想必是李常杰将与他不合的对头都丢了出来,这些人大概是不敢回去,被封了城门后也就不想拼命了。门州守将的洪真太子,是乾德的亲叔,乾德若死,在乾德幼弟之后,就轮到他接掌交趾王位。”

“洪真太子……”章惇低头又看看战果,斩首一页的最上面,就是洪真太子的名字,“可惜已经死了,早点投降多好。就算不可能坐上交趾国王,至少也能在京城中安享晚年。”

除了李常杰等参与北侵的臣子们,交趾国王——仅仅七八岁的李乾德——还有太后倚兰,以及一干宗室,以中原王朝的惯例,都会好生的在京城中豢养起来。就如李洪真,怎么都能捞个封爵和官职,得到一间京城中的宅院居住,根本不会受到多少伤害。

可惜他偏偏死硬到底,就是官军已经登城,还立于在城头上,都没有退入城中,更不用说逃跑了。结果就被率先登城的军中骁勇给一斧头砍死。

燕达叹道:“虽说此人是无能了些,但也是个有胆子的好汉。”

章惇翻着报告,满不在意的说着:“管他是好是赖,都已经是死了,一个首级而已。”

韩冈提议道:“还是将其好生安葬,即是忠臣孝子,身首异处也不太好。”

皱纹出现在章惇的双眉间:“难道要鼓励交趾人效法其人,奋死抵抗不成?”

“洪真的名声近期也来不及多远。”韩冈笑着:“日后传播开来,如若交贼要抵抗,他们又会抵抗谁人?”

韩冈可是希望交趾掌握文化的上层,全都学着李洪真的样子,拼死反抗。只要他们都死了,交趾这个国家就算毁了。没有受过教育的民众,是无法传承文明和文化的。

章惇默然点头,吃亏的只会是之后割据交趾的蛮部,将手上的总结报告递给燕达和李宪传阅,说道:“门州比起永平寨,位置更佳、城防也更为完备。这座城池日后也占下来,得尽早加以清理。”

“城中的交贼都要弄走,不然日后就是麻烦。”韩冈道。

“满城的俘虏有四千之众,而因为近几个月三十六峒蛮部扫荡,避入城中的交趾百姓也有一万多。”燕达问道,“要怎么处置他们?”

“让诸部拿人来换。”韩冈和章惇一开始考虑过这个问题,他们并不打算直接对俘虏们处刑,事情是谁做的,怨恨就会归于谁的身上,两人也都不想由自己来背着,“用人来换,就是一个换一个,尽量将他们手上的汉儿都换回来。日后的俘虏都要用来交换被掳走的大宋子民,如果最后还有剩,就作为封赏的一部分,按照出力多寡分予诸部。”

如果是用俘虏来交换金银财帛,少不了会被讽刺为人牙子,也免不了要被御史们弹劾。但现在是用来交换被掳走的汉人,则是难得的德政,不论是谁人提出、谁人主持,都是能沾光的。韩冈和章惇之前可都是因为用银绢等物买回了近万名汉人,受到了朝廷的赏赐和褒奖。

报告、总结、休整,战斗之后,一系列的琐碎杂事并没有耽搁安南行营的多少时间。

在轻松攻下门州两天之后,官军便继续高歌猛进,毫无阻碍穿过了交趾北疆的群山,一望无垠的三角洲、交趾国的核心地带出现在宋军的面前。

与此同时,一同南下的蛮部大军,也从几十条不同的大小道路上,杀入了这一片冲积平原。平静了百年的膏腴之地,在数日之间便被复仇的血与火所吞没。七八万人分作上百支队伍,流窜于乡野与城镇之间,杀戮与洗劫的剧目,时时刻刻都在富良江北岸的交趾国土上上演。

告急文书雪片一般的越过富良江,送进升龙府,北岸数十万子民的哀嚎充斥在字里行间。

救还是不救,这是个问题。

