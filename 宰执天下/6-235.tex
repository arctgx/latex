\section{第31章 风火披拂覆坟典(四)}

官车也有官车的规矩。

带着一大家子上任的官员,按照人数多寡,品级高低,能分到一节或半节车厢。如果是单身上任,就只有一个小房间。只不过,这是普通朝官才有的待遇。

至于议政重臣,骑马狨座,乘车八驾,上车……也自然有专列了。

上一回显谟阁直学士王安礼南下江宁,他家中人口少,仅仅占用了两节车厢,但照样是十六匹挽马拉着上路,后面还拖了六节空车厢。京城的商人们为此找上门,只这一趟就让王安礼赚了一大笔。

只不过,如果真是议政重臣家的子弟,好歹该有一个包厢吧?王珏疑惑着。

“衙内二字不敢当,小门小户罢了,不值一提。”

对陌生人的谨慎和提防很正常,但这副口吻,就更像是大户人家的子弟了。试问哪个小门小户的子弟,会这么说自己家?

王珏心中好奇,“请问贵姓?”

那公子犹豫之后方才吐出一个字:“……韩……”

王珏悚然一惊,甚至感觉到周围的目光也热切了起来。

韩是当世大姓,朝中望族。

安阳、灵寿、陇西,此三韩于朝中最为知名。做宰相的韩冈不说,韩琦、韩绛的子孙、族人,都有大把的在京师任官,议政重臣之中,安阳、灵寿二韩,可是各占两席。

不管是哪一家的子弟,这条大腿都是明法科出身的王珏双臂抱不过来的粗。

“在下王珏,在审刑院中办差,此番是要去楚州办一件案子。”

“在下蒋英,要去湖州上任。”

“在下文玉,是回乡守制。”

车厢中的官员,你一句我一句说了姓名和目的地,兜转了一圈,王珏小心翼翼的问着,“不知韩衙内此番南下,是要去何处?”

……………………

“去江宁。先到泗州,然后再转乘车船。”

韩钲无事老都管的咳嗽声,说了自己的目的地。

又没说家世,又没说名字,只提了姓氏,又有什么关系?

韩钲手指摆弄着腰间的玉佩,微笑着与那些目光灼灼的官员聊着天。

这御赐之物。韩钲幼时随母入宫,得太后所赐。只要有些眼力,看了之后就该知道这是御用之物。

韩冈早前因为他将要去横渠书院打好了预防针,又拿着隐姓埋名在学习的兄长来激励,韩钲也不觉得炫耀自己的身份是件好事。但自己的身份虽不当去炫耀,可适当地表露一点,也能免去小人的惦记,这也不是坏事。

……………………

车子已经出发了,韩衙内带来的四名仆人,也在无人反对的情况下,找了三张空床位安歇下来。

而韩衙内兴致颇高,谈兴极浓,在一众官员刻意的奉承下,滔滔不绝的从赛马聊到蹴鞠,从蹴鞠聊到射猎,从射猎聊到火器,从火器聊到钢铁。

“精铁需坩埚,此非辽国所能有,所以不论是铁路还是火炮,辽人即使再用心,也比不上我泱泱中国!”

每个人都似乎在为韩衙内对军事上的博学而赞叹,但所有人关注的焦点,不在他说出来的秘密,而在他对钢的称呼——

精铁!

这可不是钢!该说钢的时候,却说精铁,分明是刻意避开‘冈’这个发音。

世人避父讳,有的是临文避讳,有的就是说话都避讳。司马光之父名为司马池,所以他喊表字持国的韩维都是叫韩秉国。

眼前此子,一提到钢铁,就避开提到这个钢字,未免太着痕迹。他的身份也就呼之欲出。

据王珏所知,韩冈家中几个儿子,应该有一两个是这个年纪。

王珏眼睛亮了起来。

宰相家的公子,不管是什么原因上了这辆车,这条大粗腿不抱上,以后还可能有这么好的机会吗?

车速慢了下来。

王珏转起头,透过小窗望着窗外渐多的灯火,“前面的站要换马了。”

“这趟车只要换二十次马,就能到泗州了。有要方便、吃饭的,可以先下去。”

官车上没有热食。这是防止车上火灾。只有到站停车,才会有热食送上车来。也没有方便的地方,这是为了车上的卫生着想。所以吃喝拉撒,只能等到列车进站换马时匆匆完成。

拉运火车的挽马换得勤,而拉客车的马就可以少换几次。

但这客车的速度真要计较起来,其实并不算快,也就跟普通的马车差不多。当然,大赛马场中,那种被顶级赛马拉着满场飞奔的轻便双轮马车,肯定不是普通的马车。

赛车比赛中所用的马车,都是出了名的轻。马主都恨不得用篾条去编出一辆车来,好减轻一些重量,让赛马跑得更快一点。这样的车子,只能勉强站上一个人,剩下的就只剩不能缩减的重要零件了。

一个小时二十里路,也就是一个人小跑着的速度。但铁路最大的特点,就是不用多停留。除了到站换马,其他时候都是奔跑在铁轨上。一天十二个时辰、二十四个小时、九十六刻钟不停地奔行,一天四五百里,两天就是一千里了。

换作是快速客车——主要是以官车为主——那就更快了。一个小时差不多三十里。所以四十个小时不到,就能抵达泗州。

还没有铁路的时候,官员和他们的家眷上任、离任、进京、离京,在驿站中连吃带占,花费的成本表现在帐册上时就是一个鲜红色的无底洞。快车虽然消耗马力,但驿传系统节省下来的成本,却让年终审阅账目的三司使、宰相和太后,脸色都能好上许多。

“要是铁路能通扬州就好了,免了还要再换船。”

“该通真州才是,江对面就是江宁,还能少修几里路。”

“扬州的好。”

“还是真州好。”

车厢中稍稍起了些争执,只见那位韩衙内摇头嗤笑:“朝廷上争了两年都没争出个眉目,想看到京泗铁路南延,可是有得等了。”

扬州在泗州东南面,但泗州到扬州,如果是水路的话,过了泗州之后,必须先由淮水往东北方向走上一百里,抵达楚州,再转向南行,最后抵达扬州。这是因为必经之路淮水在这一段是西南、东北走向。

如果改成铁路联通,那就可以走直线,而不需要绕上一个大圈。不过由于朝堂上对于泗州向南的铁路,到底是通往扬州,还是江宁对面的真州今南京**县,还有着巨大的争议。

扬州过江就是苏杭运河通往扬子江的出口,两浙的纲粮、商货不必再上溯江水,而福建、广东的海货也同样如此,至少能节省一天的水程。但江宁府更是江南重镇,军事和政治上的意义不是扬州能比。

这样的争议闹了有两三年,出身两浙的沈括希望铁路能走扬州,两浙的货物经过运河之后,渡了江就能上车去往京师。福建路的宰相、枢密虽不表态,但福建出身的官员还是多数支持扬州线的方案。

而江西和江东路出身的官员,则全数希望能走江宁。而且北方出身的重臣,也觉得江宁地势更为重要,朝廷的兵马能更快抵达江宁府比什么商货更重要。

两边势均力敌,身为宰相、又分管此事的韩冈又不说话,一切全都推给廷议,所以京泗铁路的南延线也就一直难产到今天。

“其实也是跟两浙、江东之争有关。铁路修到扬州,对面是两浙路的润州今镇江,而铁路修到真州,对面就是江东东路的江宁。多经过一个州府,就等于凭空涨上两分的过税。如果是跨过一路,实际上,成本就要上涨一成。所以两浙、福建多是希望修到扬州,而江西、江东,包括淮南西路南方的黄、舒等军州,乃至荆湖南北两路偏东的军州,则都盼着江宁线。”

听了韩钲的一番话,王珏对他的身份再无怀疑。周围也是一片的赞叹声。

不是宰相家的子弟,如何能有如此真知灼见?这不是他们自己的东西,是从父兄长辈那里听来的。

“韩公子为何要连夜南下?”

这问题换来了一声黯然神伤的叹息,“长辈有恙。”

长辈?

王珏晕晕乎乎的点起头。

当然是长辈!亲外公嘛,人就在江宁。

以那一位的身份,做外孙千里迢迢去探望也是应当的。会上这一辆夜班车,挤进现在的车厢,多半是为赶时间,只能上这一列没有多余车厢和包厢的南下列车了。

用上一天半的时间抵达泗州,之后或按其所说转乘车船,又或是坐马车,抵达江宁,也就再两三日的功夫。

等等!王珏悚然一惊,为什么那韩公子之前要说转乘车船?!

如果不是王老相公或是那位楚国夫人突发恶疾,用不着宰相家的衙内连夜赶去探望。

要是王老相公和楚国夫人发了急病,要赶去江宁,理应在泗州换马南下,从瓜步镇渡江,这样至少能省下一天的时间。

转乘车船,这完全不合情理!

肯定不对!

起了疑心,王珏再回忆起之前的对话,登时就觉得满是破绽。

哪家的衙内不是嘲风弄月的行家里手,就算家学谨严,这个年纪也是读书用功的岁数,日后考中进士,也能保守家门不堕。再出色一点的,也就是多了解些天下大势,增广见闻,以备将来之用。但分心实务,却绝不该是贵人家的子弟该做的。

成本多上两分、一成,哪家十四五岁的衙内会关心这等事?试问这行商治家之学,对宰相家有何意义,可比得上一个金榜题名的进士?

更重要的是,方才一瞥之间,王珏看见那位韩家公子的手掌上,竟然有着一层厚厚老茧。

韩家公子手背细皮嫩肉,脸皮白皙粉嫩,牙齿更是整齐洁白。这是要钱养出来的,天生再好,也得靠日常保养才能维持。贵人家的子弟,从小养尊处优,才能养得起这副好皮囊。所以一见之下,就没人怀疑他的身份。

但手掌内老茧就完全不对劲了,有哪家的贵公子会是每天劳作,弄得满手老茧?

外面光鲜,里面寒酸,这样的人也是有。如果是天生之质,就算操劳了十几年,只要好生保养上一年半载,也能变成眼前这幅模样,就是手掌心上的老茧一时间褪不下去。

这样的人,王珏见过,是一些走偏门的青楼特意养起来,提供给好男风的客人的。当初王珏在聚会上见识过一位,一身女装亮相,比花魁还要娇艳三分。眼前的这位倒好,不装女人,而装起衙内了。

难怪以宰相之子的身份,只能来这里寄身。肯定是因为那车厢、包厢都拿不到,更别说专列了。

至于那一番有关铁路的真知灼见,还不知是在哪里的酒宴上听到的。或许还翻了翻京师的小报,又听多了酒楼茶肆中的传言。再细想,之前提起这个话题,可不就是这位韩衙内先起得头。

‘真是利令智昏啊!’王珏想着。

什么叫‘多半是为赶时间,只能上这一列没有多余车厢、包厢的南下列车了?'人都没说,自己就帮着把破绽给补上了。

但现在既然发现了,可就不能放过。

身为审刑院中司法官,王珏知道,这可是一个让自己的名字上达天听的大好良机。

王珏微笑着起身,对谈兴正浓的几人告了个罪,悄然离开

有人随意的瞥了他一眼,也只觉得这王珏是去方便了。

但过了片刻,王珏回来了,身后还跟着四名随车的警卫。

警卫们手持兵械,在门口一站,车厢登时就没有了声息。

人人狐疑的转过脸来,那名骗子衙内也是一脸的迷茫。

‘装得好像。’

王珏冷笑一声,当先走过来,指着韩钲的鼻子,“就是他们,一伙骗徒,竟敢冒充宰相家的衙内!”
