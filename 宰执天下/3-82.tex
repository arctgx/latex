\section{第26章 任官古渡西(七)}

胡二在堂上的一番话,很快就传到了诸立兄弟耳朵里。

“胡老二倒是好心啊,跟韩正言说了这么多。”诸家老三冷笑着,胡二看似老实,但他们兄弟三人都不亲近,总有些不顺服的心思在。

“他哪是好心,不过是想做韩家门下的走马狗罢了。只可惜人家年少气盛,不肯听劝。”

诸立低着头,一手把着茶瓶,一手拿着茶筅,小心的将滚水倾进杯中,双眼专注于茶盏之上,嘴角却是带着一丝笑容。

一切尽如他所料,而且发展得比他期盼的还要好。

何家祭田案比起其他争产案更为麻烦,没有证人、没有证物,全凭两家在争吵。争了整整三十年,比起韩冈的年纪都大,他怎么审这陈年旧案?

三十年来,多少精于刑名的积年老吏都在此案折戟沉沙,最后退避三舍。韩冈再有能耐,也只是军事上、医事上有着偌大的名气。刑名与治政、用兵可是两码事,书写判词跟做文章关系也不大。在判词中,用错了一个典故没什么,若是错了一条律令,整个案子就会打回来重审。

诸霖很是想看看明天的乐子,巴不得天早一点黑下去:“偏生这个案子名气极大,从县里打到州里,从州里打到监司,三十年的积案,怕是连审刑院都听说过。新来的韩知县要审此案,这消息一传出去,怕是整个白马县都要给惊动了。”

滚水细如一线,注入莹润的青瓷茶盏中,茶杓顺着水流轻搅着盏中的茶膏。热腾腾的白色茶汤上,一层浮沫粘着盏壁,一点也不散去,“竟然咬盏了!”

欣喜的将难得成功的佳绩亮给两个弟弟看着,诸立漫不经意:“我们也帮帮忙吧,帮韩正言好好的宣扬一下。明天是他到任后第一次审案,总得讲个排场。”

……………………

太阳刚刚升起,橘红色阳光冲淡了初冬凌晨的寒意。

由于何双垣祭田案的名气,还有诸立兄弟的宣传,加上白马县民对于韩冈这位新任知县的好奇。第二天一大清早,在县衙门前,聚集起大批的士绅百姓爷也就不足为奇。

两名五十出头的老头子,胡子都是花白了,并立在县衙的门前,中间却隔了老远,互相之间看都不看一眼。

这一案的原告和被告都到了。

“打了一辈子的官司。还真是不嫌腻烦。”人群中一阵冷嘲。

“两百多亩地啊,要是就是一个坟包,外人谁会去争?”

“不知今次能不能断出个眉目来。从十年前开始,可是连着六任知县没敢接这个案子了。”

“也不看看衙门里的那一位是谁?那可是今科进士第九,二十岁的进士。立得功劳不知多少,一句话就说降了叛军,张张口就帮着平了吐蕃。这么大的功绩,连着宰相都抢着做女婿,过去的知县哪一任能比?”

“就不知会断谁赢?”

“同是寒门素户出身,苦读之士,肯定不会偏向那富户。”

“说的也是,听说韩正言当年求学张横渠门下,下雪天站在书院门外,直到没了膝盖,才被收为弟子,真的是苦读啊!”

“真的假的,怎么听着那么像慧可祖师求学达摩祖师的段子。”

“千真万确!正是因为在雪地里站久了,韩知县落下了病根,所以回乡时倒在庙里,正好被孙真人救了。想想这天下倒在路边庙中的有多少人,可曾有一个能得到孙真人的救治?若不是同是天上旧相识,孙真人修道几百年,早就看破了生死,又怎么出手救人?”

“原来如此,受教了,受教了!”

大门紧闭着,无数或真或假有根无据的传言在人群中散布着,引得来此围观审案的白马百姓期盼之心更为旺盛。

从人心上来讲,人们都是喜欢看个热闹。韩冈的身份、经历,很有些传奇的味道,被人津津乐道。现在他来白马任知县,第一案就落在就难断的案子上,白马县百姓当然都想看一看传说中的韩正言,他的名气是否是货真价实,能否明察秋毫。

随着升堂鼓从衙门中响起,衙门外的人群渐渐安静了下来。

县衙正门吱呀呀的打开,紧接着向内几十步,大堂的正门——仪门也随之打开。连开二正门,体现了新来的知县开堂公审的心意。

二十名衙役一身皂服,结束整齐,都带着方帽,手持上红下黑的水火棍,挺胸叠肚的分立在大堂东西两侧。而同样数目的弓手,亦是分作两队,跨着刀,从大堂一直拖到正门。

水火棍咚咚敲着铺在大堂地面上的青石板,在威武声中,韩冈身着绿色公服,头戴长脚幞头,从后方侧门走上堂来。

衙门的观众,堂内的胥吏,齐刷刷的跪了下去。

在主桌上放着惊堂木,只有巴掌大,黑沉沉的,上面刻着龙纹,韩冈估摸着应该是枣木。他做管勾、做通判、做机宜,这玩意儿可都没上过手。现在拿在手上,才有了一点百里侯的感觉。而七品知县,在整个大宋,怕是也只有寥寥数人。

在主桌旁边,只有做记录的文书,虽然是陈年积案,但从分类上并不是大案,依照律条,县丞和县尉都不需要到场。若审的是杀人要案,那就不一样了。不但县中官员都得上堂,甚至要知会邻县,派官来监审。

韩冈坐定下来,而堂内堂外,也都拜后起身。

拿着惊堂木,在枣木方桌上用力一拍,韩冈提声道:“宣何阗、何允文上堂。”

韩冈的命令一路穿了出去,原告和被告都低着头,脚步匆匆的上了堂来。

韩冈双眼一扫两人,长相都不是作奸犯科的模样,穿着儒士服的何阗,相貌清癯,的确是读过书的。而被告何允文,虽然有些富态,但身上的装束也是素净,没有多少饰物,显然是不肯露财,惹得别人有成见。

“本县士绅,可容二十人至大堂外旁听。”韩冈先放了二十名有份量的听众进来。

等到观众到位,他一拍惊堂木:“本官受天子命,来白马任职,正欲一清县中政事,以报陛下恩德。近有本县何阗诉同乡何允文一案,但言葬于清水沟畔之何双垣,乃是其祖,欲求何允文归还先祖坟茔以及祭田两顷又一十五亩。此案拖延日久,本官无意留给后进。你二人且将各自凭证一一道来,本官自会依律做个评判。”

得到韩冈到命令,何阗、何允文各自上前,将自己的理由一一叙述,一切都与胡二昨日所说的一模一样,都没有证据,只凭一张嘴而已。

何双垣死得早,在他的墓碑上并没有刻上孙辈的名字。若是寿终正寝,孙子、曾孙的名字一起上了碑面,也就没有那么多事了。就是因为他只活到三十七岁,连长孙都没看到,所以才有了这一桩纠缠了三十年的争产案。

两人的一番叙述,韩冈在中间夹杂着疑问,耗用了近一个时辰的时间。

“小人虽是鄙薄,却也不会乱认祖宗。有证人,有系谱,怎么就断不明白!”何允文说道动情处,几乎就要哭出来。

“系谱可以伪造,证人可以收买。学生无钱收买证人,但祖宗不得血食,学生岂能无动于衷。还请县尊明断黑白,一正是非!”何阗理直气壮,外面的一群士子在外面鼓噪起来,纷纷为何阗助威,

韩冈一拍惊堂木:“堂上断案,堂下岂有喧哗之理。”喝止了儒士,他又道:“系谱其实可以伪造,证人也可以收买,更别说田契什么,何阗说的的确是有几分道理。”

韩冈说到这里,声音停了一停。就看见何允文了脸色一下变得发青,而何阗脸上泛起了红晕。

“不过。”韩冈话声一转,“终究还有一项是伪造不了的。清水沟边的两顷一十五亩田地,那都是祭田,跟着墓中人而来,只有何双垣的亲孙能够继承。”

惊堂木一震堂中,“何阗!何允文!”

韩冈提气叫着两名当事人的名字。

“小人(学生)在。”两人一起躬身等着韩冈的发话。

“你们都自称墓里的何双垣是自己的祖父,可是如此?”

两人又是异口同声:“正是小人(学生)祖父!”

“那就好!”韩冈满意的点着头,“既然如此,也不需要多费唇舌,更不需要去找证人、证据了,只要确定一下何双垣究竟是谁的祖父就可以!”

不论原告被告,堂上堂下,一下都愣住了。人都死了五十年了,又没个证人,怎么查验?难道要牒送城隍,传死人来上堂不成?早就转世投胎了吧。

韩冈却没有解释,却只见他再一拍惊堂木,“三日后,本官将亲至清水沟畔何双垣墓前再审此案!今日就到此为止,退堂!”

将大堂之外的哗然议论抛在脑后,韩冈径自回到内厅,吩咐着服侍自己的仆役:“本官接下来要斋戒三日,下面这三天,让厨中只送蔬饭即可。”

仆役摸不着头脑的受命离开,而魏平真追过来,问着韩冈:“正言,你这可有把握?”

他的东主三天后要做什么,魏平真自问已经可以猜到了。可就怕韩冈太过自信,反而会出岔子。方兴和游醇也盯着韩冈,生怕他自信过度三日后出错。

韩冈给了他们一个沉稳而让人安心的笑容:“‘古者言之不出,耻躬之不逮也’,韩冈承袭圣人之教,若是做不到,就不会说出来!今天问案只是走过场而已,关键还是在三天后。还请拭目以待!”

