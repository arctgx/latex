\section{第48章 辰星惊兆夷王戡(中)}

掀开车帘,周南正躺在车中。

身上盖着厚厚的棉被,周南脸色有些不好,如玉一般的双颊少了光泽,是病态的苍白。就是柔红如染的唇瓣,也泛着白,不见血色。

“怎么就病了。”韩冈心中一痛。

看见韩冈坐进来,她睁开眼,勉强的展颜笑道:“官人,奴家不要紧。”

长距离的旅行对孕妇来说很是吃力。幸好已经进入了稳定期,要不然韩冈也不敢让她上路,但看现在的模样,还是动了胎气。不过周南身体底子好,又不是头胎,韩冈总算能稍稍放心一点,等回去后,请两个御医来,调养一下应该就没事了。

理了一下周南散乱的发丝,将她身上的棉被盖严实了,韩冈温声道:“先歇一会儿,到家就好了。”

周南轻轻的嗯了一声,乖乖的闭上眼睛。如果还在路上,不论是王旖还是素心、云娘作陪,总会胡思乱想。但在心中最重要的人身边,她就能安心入眠。

离开车厢,王旖过来,在韩冈身边轻声道:“官人,南娘妹妹是路上累的,到了家就好了。”

韩冈点点头:“那就都上车,早点回家休息。”

送了王旖她们上车,韩冈换上了一匹马,陪在车边原路返回。

方才韩冈休息的小酒馆的老板已经出来了,看样子想过来说话,但被韩信拦住,不敢造次,只能悻悻然的站在一边,暗恨自己错失了良机。

回头路走了半个时辰。离开京城一年,家里的三个大一点的孩子,都兴奋的趴在车窗上向外看,小脸冻得红扑扑的。

当韩冈陪着家人回到新的府邸,却见门口停着三匹马。而原本聚在门前的访客,却离得远远的站着,且人数少了许多。

韩冈心中生疑,正猜测着究竟是为什么,就听到门前一叠声的在喊,“龙图回来了,龙图回来了。”是韩家司阍的声音。

在司阍的引领下,一个不认识的小黄门匆匆迎了上来,“韩龙图,韩龙图,你可让小人好找。”

韩冈一愣,翻身下马。宫中的内侍,自不会无故上门,难道是天子终于决定要给儿子种痘,想让自己去现场做个见证?

“官人?”马车中王旖惊疑不定。

“没事,你们坐着好了。”韩冈低声安慰,“天子召见,一个月总有个三五趟。”

但王旖如何能安心,让天子空等可不是好事。何况韩冈是在坐衙的时间里跑出来迎接她们的。肯定少不了一个处罚,加上七皇子的事,天子肯定有心结,小事都能变成大事。

小黄门在韩冈面前站定,尖着嗓子:“天子有旨。龙图阁学士、同群牧使韩冈,即刻入宫陛见。”

“臣恭领陛下圣谕。”韩冈恭声领旨,随后回头冲韩信使了个眼色。

韩信心领神会,上前往小黄门的手中照例塞了一份赏钱,凑上去问道:“这位黄门,官家此时召见龙图,不知有何要事?”

小黄门收了钱,低声对韩信道:“辽国出了大事,两府宰执都到了崇政殿,除此之外,官家就只遣人招了龙图入宫。”他吊足了胃口,才解开谜底,“是辽主驾崩!”

接旨之后,韩冈吩咐了家人几句,就上马往宫中去。但听到的消息还是震得他心中阵阵惊涛骇浪,推演着天下大局将会产生的变化。在路上也没有快马加鞭,任凭坐骑小碎步走着。

“龙图,快一点。”小黄门急得恨不得给韩冈的马两鞭子。他抬头看着天色,日头西垂,都已经近黄昏了。

“不,慢一点才好。”韩冈慢悠悠的说道,手上提着马缰,稳如泰山一般。

小黄门惊疑不定,脸色忽青忽白。但看见韩冈的平和淡定的表情,在宫廷中受到的教育让他立刻就醒悟过来:“呃……小人明白,是不能快,是不能快,惹起谣言就糟了。”说着就主动将马缓了下来。

韩冈微微一笑,“黄门明白就好。”

心中还是嗤笑的多。又不是仁宗时,西北连番大败,河北边境又有契丹虎视眈眈,京城中人心惶惶,一夕三惊。那个时候,就是有了紧急军情,宰辅们也必须在路上慢慢走。甚至直接将天子夜中传召的圣谕给挡回去,等到第二天上朝后再议论。

但眼下情况可不一样,到了明天,辽国国主驾崩的消息就能传遍京城,宰辅重臣急入宫,自不会有人会担惊受怕。韩冈现在走得慢只是为自己。慌慌张张、毛毛躁躁,可不是以两府为目标的重臣该有的行事作风,而且正好多一点时间想一想。

当韩冈抵达崇政殿的时候,时间已经很迟了,瞧殿中宰执们被赐了座,赐了茶,可见他们之前已经费了不少口水和力气。

看到韩冈耽搁了近一个多时辰才到,赵顼很是不快的问着,“韩卿今日非休沐,怎么不在群牧司?”

“臣妻子今日抵京,故而待司中事务处理完毕之后,臣便告了假。不意陛下于此时传召,臣有过,请陛下责罚。”

对于迟到和请假的原因,韩冈一点都不隐瞒,把信用消耗在小事上是最蠢的。

“哦,是吗?”赵顼嘴角抽搐一下,没说什么。

总不可能用这等小错惩罚重臣,尤其是现在离不了韩冈的情况下,借题发挥也不可能,最多罚铜而已。对于普通官员,同时代表着磨勘期限延长的罚铜,代表着他们可能要在升迁上多耽搁三年。可韩冈的本官,都升到了非宰执官能坐上的最高一级的谏议大夫,磨勘对他已经完全失去了意义。

“辽主驾崩之事,韩卿应该听说了吧?”赵顼问得也很干脆。他的臣子们接旨之后,不可能不会向传诏的中使私下里询问,相信韩冈不会例外。

韩冈点头:“仅是知其驾崩。”

“不知韩卿如何看此事?”赵顼追问。

“辽主正值壮年,又常年游猎。中国使辽的正旦使、生辰使常年不绝,亦不见有人回报其疾病缠身,身体当是康健。忽闻其暴毙,实在是难以置信。不知是因为何故?”

对于耶律洪基的死,说起来韩冈也是吃惊不小,意外性不说,其所带来的结果就是先前的战略规划,也必须重新进行修订。在进入崇政殿之前,韩冈已经想明白了。

赵顼的回答自是不出韩冈预料:“辽主死因,尚不知晓。不过耶律乙辛把持朝堂多年,故太子又因其谗言枉死,国中积怨甚深。且辽主只有一孙,小字阿果,年方五岁,若强立其为帝,主少国疑,又有众宗室虎视眈眈,耶律乙辛当难以控制朝堂。”

这大概就是之前众位宰辅议论之后的结果。听赵顼的口气,当不会放过这个机会了,王珪当是心中乐开了花。

韩冈向王珪那里瞟了一眼,当朝宰相正巧开口:“陛下之言极是。辽国一乱,西夏便不在话下。若是待其国中稳定下来,可就没有现在这么好的机会了。”

听着王珪的话,赵顼微笑点头,这正是他的想法。他又望向韩冈:“韩卿,你熟知兵事。依你之见,如今局势当如何应对?”

韩冈是求稳的性格,但不代表他会愿意放过机会,只是现在的机会在韩冈看来,还是不太稳妥,将希望放在敌人还没有发生的内乱上,未免太过一厢情愿。就是当真内乱,也没必要抢这个机会。修好轨道,练好士兵,备足兵甲钱粮,就是辽、夏两国实力完好,也没什么可怕的。

只是依眼下赵顼说话的口吻,想必‘昔之善战者,先为不可胜,以待敌之可胜。’这一句,是听不进去的。

“辽主暴毙,不论其是否留下遗诏,耶律乙辛皆当扶幼主登基,以期继续秉政。”韩冈顿了一顿,“可耶律乙辛现在是否安好?如果耶律乙辛同时出事,即位的就又会是谁?”

即位的不一定是耶律洪基的孙子,耶律浚的儿子。耶律乙辛虽然是权臣,但他的权力是嫁接在皇权之上的,不一定能压得住阵脚。而且说不定耶律乙辛跟耶律洪基一起死了,或者耶律乙辛跟着耶律洪基死了,到时候能即位的肯定不是阿果。

韩冈的言下之意。赵顼听明白了:“韩卿的意思是要稍等?”

“以臣愚见,最好能等到辽国内乱开始。”韩冈回道。

“五院、六院,二部皇族哪一个都不会看着耶律乙辛挟天子以令诸侯。遥辇九帐、横帐三父房、国舅五房,也都不会坐视。辽国内乱可期。”

“辽国必乱!”元绛也说道,“契丹幅员万里,其下属国大者五六,小者百余,皆常年受其压榨。一旦其国族内乱,其下属国自是难免离心离德,甚至揭竿起兵。”

“韩卿?”赵顼盯着韩冈。

“王相公、元参政旧日皆曾出使辽国。论起熟知辽国内情,韩冈安敢望相公、参政项背?”韩冈回道,“王相公、元参政即有此言,想来当是如此。”

韩冈只在对西夏事务上有发言权,元绛去过辽国和高丽,王珪也出使过辽国,两人在对辽事务上,可以轻而易举的压倒其他人的声音。

韩冈放弃了在辽国事务上与人相争,但又顺便将自己的原因归结到王珪两人出使过辽国上。待会儿说到西夏之事上,自己可有得话说。

