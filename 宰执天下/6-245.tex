\section{第33章 为日觅月议乾坤(下)}

韩冈正要上车的时候,一名堂后官气喘吁吁的跑了来。

“相公,相公。”他手里拿着一沓函,递给了韩冈的从人,“这是章相公命小人送来的。”

一群元随将他阻隔在人群之外,韩冈伸手接过函翻了翻,却是刚刚整理出来的新一批的开国以来宰辅与管军的嗣名单,除去了外戚,只有外臣,同时还限定了本人须在朝为官。

一页三人,二十二页,最后一页只有一人,十四位显贵之后。比前几批都要少,理应是最后一批了。

“好了。”韩冈合上页夹,“回去跟章相公说,我收到了。”

堂后官应声离开,韩冈也转身上车回家。

回到家,韩冈梳洗更衣,出来后,周南正在翻看他放在桌上的件。

她抬头问着韩冈,“官人,人就这么多了?”

韩冈习惯性的往躺椅上一靠,惬意的闭起了眼睛,“这就是最后的一批了,这些人家里面可有合适的?”

红婚白丧,两件事一向并称。天出殡,宰相都要为大礼使,而天大婚,当然也是要由宰相主持。

在外宰相,在内太后、太妃。为了年届十四的天的婚事,朝野内外都动员了起来。

堂后官刚刚送到韩冈手的这份公,便是新一批入选的名单。只有出自这些家庭的适龄少女,才有资格成为皇后的候选。

“都没什么印象。”周南将合页夹放下,靠进韩冈的怀里,扭了一下丰韵十足的身,让自己坐得舒服一点,“还是姐姐更熟悉些,奴家寻常又不出门。”

“谁让你们姐姐去了江宁,现在又不好问她。”

韩冈熟极而流的将手顺着衣襟插进周南怀里,摸着里面腻滑如脂的肌肤,没了王旖管束,孩也离开了,他在家里便放得更开。

“难道朝廷就不查吗?官人还是派人去打探吧,我们这些妇人又不拿俸禄。”

周南打着哈欠伸了个懒腰,在韩冈怀里又是一阵厮磨。

韩冈呼吸稍稍急促了些,“朝廷选人,仓促间哪里能查得清楚?平日里的口碑,还得靠你们。”

周南腻声道:“聪明人谁会把家里的女儿往火坑里推。能结亲的都赶着结亲了,没结亲的也推说家里的女儿貌寝颜陋,不堪为天良配。”

“你们姐姐以前回来的时候好像说过几家女儿还不错的。”

“再好都不如金娘。”严素心边说,边端着刚做好的饮进来,看着楼着韩冈的周南,轻哼了一声,“姐姐不在家,就变这样了。”

周南仰起依然绝艳不可方物的俏脸,笑着拍了拍韩冈另一侧:“这边还有个位置,”

“我可不凑趣了。”严素心捂着嘴笑,“隔天再换一个躺椅,让人怎么看?”

“怎么又要换椅了?”

云娘紧跟在后进来,看见韩冈和周南,也笑着啐了一口,“等姐姐回来知道了,看她怎么说。”

“还是先说这件事吧。”

不是因为这件事韩冈要征询妻妾们的意见,周南三女也不敢随意过问国事。

“官人,真要找不到意的怎么办?”严素心问道。

韩冈也挺头疼,“十全十美的自然不易寻,还是以品行为上。”

已经半个月过去了,皇后候选并没有想象的那种蜂拥而至的情况。

做皇后母仪天下的确光荣,更重要的是整个家族都能受益。

向太后的亡父向经,祖父向传亮,乃至曾祖向敏,都追授了王爵,而且还不是郡王,乃是国王——韩王、唐王、陈王。

还活着的兄弟、堂兄弟,叔伯、侄,也无一例外都授予了官职。

向宗回、向宗良,也就是太后的亲弟弟,他们姐姐做了皇后,便是正任刺史,之后团练使、观察使一路升上去,如今已经是节度使。而正式领兵的将帅,只有郭逵、种谔和王正有节度使的身份。

正所谓一人飞升,仙及鸡犬,韩冈亲眼所见,曹家、高家的两家外戚,亦无不如此。

但眼下的选后,毕竟是在垂帘听政的向太后的主持下进行。

这样所选出来的皇后,之前正好有一位——仁宗的郭皇后。

那是最好的前车之鉴。

天圣二年,在章献太后的主导下,仁宗娶了郭氏为后。

但在章献太后上仙后,郭皇后当年便被废为净妃,出家入道,赐号玉京冲妙仙师,再过了一年,郭皇后就猝死在长宁宫。死后才被念旧的仁宗追复皇后之位,但没有赠谥,也没有祔庙。

自然,郭家也没有因此而飞黄腾达,仁宗固然念旧,又感念郭皇后早殇,可也只是追赠其祖,赠其父兄,并没有兼及亲属。

郭皇后会落到这样的结局,本质上还是仁宗与章献太后之间的矛盾。

二十四岁方得亲政,仁宗十几年的积怨不能发泄在已经去世的章献太后身上,当然只能找身边的郭皇后。

以赵祯对身边人的宽厚仁爱,甚至在驾崩后得到了‘仁’为庙号,却容不下一个郭皇后。

那么当今天呢?

除了冬夜里的那一场意外,以及两年前福宁宫让人啼笑皆非的一桩事,赵煦在臣民心目留下的,仅仅是个模糊的形象。

不过人们至少没看出来,他有堪比仁宗的仁慈之心。

要是在太后的主导下,把家里的女儿嫁过去,即使一时贵为皇后,也不代表能一直持续下去。

一边是前车之鉴,另一边是富贵荣华,自然让许多有心靠女儿争一个富贵的人家,一时间难以做出决断。

当朝宰辅、重臣,乃至自认有前途的朝臣,自不可能让自家的女儿、孙女参与到皇后候选去。而许多有识之士,都没有一个愿意将女儿给献上来。

有几个少小便在京命妇圈闻名的女孩儿,早前也没听说过与他人议亲,但当朝廷遣人问询的时候,不是业已字人,就是已经许人,或是有了夫家。

“门第之选能否稍低几分?”周南问道,“再多上百千家,更易择人。”

“是啊,”云娘拿着件走过来,找了张小凳在韩冈身边坐下,“章献皇后家,温成皇后家,都不是什么高门显贵。”

“那是从嫔妃上被册立为后。正经聘后,无不出自高门。”韩冈拍拍云娘的背,“当今太后的曾祖父向简是太宗、真宗年间的宰相,慈圣乃是平南唐的曹彬孙女。而庄肃皇后,亦是名将高琼之后。无论哪一家,都是武两班顶尖儿的一批人。”

“那现在怎么办?”云娘问道。

“就是这么办。”韩冈拍了拍她手的合页夹,叹道,“强买强卖!”

这本就强买强卖的生意。

想要选为皇后,曾祖或祖父,至少得做过宰辅或节度使。所以朝廷如今搜检天下名臣之后,一个个列出名单,然后遣人去询问。

开国以来武两边的显贵,一两百总是有的。几代门第,一个个妻妾众多,孙辈、重孙辈,数以千计。其适龄的女儿家,也就是十二到十之间的,三五百总能找得出来。

不过再加上没有许人、相貌还要过得去这两条,最终可以入选的范围还将大幅度缩小,也就百十人。

即便是士大夫家娶亲,百十人也算是很大的范围了,可是放在坐拥天下的皇帝身上,却又嫌太少了。

严素心忽然道:“庄圣不是慈圣的侄女儿?太后嫁出去的几个姐妹,家里都有女儿吧?”

韩冈摇头,“就怕恃宠而骄,再出一庄肃。”

没有从小被养在宫,视同皇后女,待遇如公主一般。岂会养成高滔滔的刚愎脾气?

高太皇太后去世之后,因其旧过,谥号便只有两字。

在为了显示向太后的孝心,不能给予恶谥、平谥的情况下,太常礼院费了点心,拟定了庄肃二字。

太祖、太宗、真宗,三代皇帝的皇后谥号,都与皇帝谥号有关联,从其最后四字里取出一字。即所谓皇后谥冠以帝谥。

比如太祖三后皆有孝,太宗四后皆有德,而真宗五位皇后的谥号一开始还没有如此拟定,但庆历年间为礼官所言,故而纷纷改易,皆带有章,章怀、章穆、章懿、章惠,以及章献明肃。这是因为太祖谥号有‘大孝’,太宗谥号有‘圣德’,真宗谥号有‘章圣’。

但曹后的‘慈圣光献’四个字,却没有一个是从仁宗谥号最后四字选出,甚至与仁宗的‘体天法道极功全德神圣武睿哲明孝’无一字相同。

有此先例,那英宗皇后也没有什么必要从英宗谥号的最后四字再取一字。

庄肃二字,皆是美谥,但用在高太皇身上,却是明褒实贬。

而且这个谥号其实刚好占了章献皇后最早的谥号庄献明肃的头尾二字,韩冈没去细追究,太常礼院礼官的心思就跟肠一样曲十八弯,谁知道是什么用心。

有这一位在前,任谁都不想再出一位庄肃高皇后。

向太后更不想将自家的侄给牵连进来。

韩冈当然也不想,所以当云娘突然拿着名单倒数第二页指给他看的时候,顿时火冒三丈。

“官人,这个……不是二舅吗?”

王安石的次,正在江宁任职的王旁正列名其上。

而王旁家里,正好有一个还没许配出去的女儿。

“越娘满十二了吧?”严素心幽幽说道。

‘章惇怎么弄的?’韩冈恼火的想道。

王安石的亲孙女,他韩冈的内侄女,可是能送进宫做皇后的?
