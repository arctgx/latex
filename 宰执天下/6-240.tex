\section{第32章 江上水平潜波涛(上)}

一千零二十四。

一千零二十五。

徐玑低着头,数着脚下的步子。

走过了崇政殿前宽大的青石板,穿过了庆寿宫前细密的小砖路,福宁殿和庆寿殿宫墙相夹的小路,刚刚进过翻修,全部是青色的雕花方砖。木底的官靴走上去,就跟踏上殿宇中的金砖一样,笃笃的脚步声回响在两侧的宫墙之间。

黑色的锦缎鞋面上还有一条缝补过的痕迹,不过除非已经知道或是靠得近了才能看得出来,否则就是一双八成新的好鞋子。

黑色鞋面左右左右的出现在视野中,徐玑心中泛着淡淡的暖意。自己老妻巧手织补,又省了一双官靴的钱。

说起来,自过年后,家里就没裁过新衣。换季后朝廷发下的衣料,都拿去换了钱物。妻儿身上的衣服全都是旧的。

已经是翰林医官,隔三差五就能入宫,在医院和太医局中能拿两份俸禄,还有诊金的分账,可徐玑这日子还是过得紧巴巴的。

要是没铁路就好了。

徐玑忍不住怀念起几年前的日子。

那时候,虽然还没有通过主任医师的考试,在太医局中得到官身,但身为西城医院最优秀的内科主治医师,徐玑的收入,可也是能让妻儿隔几日就是一身新衣,自己也能隔三差五与同僚去甜水巷逛上一逛。

可自从两年前开封通向陈州的铁路开通之后,徐家的生活水平陡然直落千丈。

徐玑乡贯在陈州西华今周口西华,过去从西华上京一趟太难,一来一往半个月就没了,如今只要买票坐车,一天一夜就能抵京。故此亲戚乡邻便如潮水一般涌向京师,一年到头,徐家的客人都络绎不绝。

现在家里面时常都住着几名乡人,吃穿用度都要徐玑来负担。

尽管家中的生计已经很吃力了,但徐玑还是咬着牙坚持着。要是怠慢了,这些人回乡一说,他在乡里就没法儿见人了。

“徐太医,走这边。”

前面引路的小黄门正转向右方,走向一道小门,却发现徐玑没跟上来,惊讶的回头叫道。

“啊……啊,走过了。”

徐玑惊醒过来,方才心神恍惚,走了上百次的道路差点就走错了。点点头,回身急走两步,忙跟了上去。

一千五百二十。

一千五百二十一。

耳畔变得吵闹。

两边的士兵和内侍也多了起来。

空气中更多了一股子桐油的味道。

而宫中特有的那种连夏日的阳光都驱散不尽的阴冷,似乎也因为人气而消散了泰半。

坤宁殿到了。

尽管天子纳后连个消息都还没传出来,但空了好些年的坤宁宫,几年来第一次开始大规模的整修。

工匠们在坤宁宫中整修殿宇,外面守着宽衣天武,又有内侍警惕的盯着那些士兵。

每年朝廷都会专门拨出一笔款子,用于皇城的日常维护。但宫殿的翻修,则必须从内库拨款。

宫门边堆放的砖石在阳光下泛着金属的光泽,徐玑瞥了一眼后,就收了回了目光。

这样的一块金砖,就能抵了他十天的俸禄。宫门前的这么一堆砖石,够他做上一辈子了。

一千九百零三。

一千九百零四。

钉刨锯凿的噪音消失在身后,桐油气味也淡了下去。

前面的小黄门停下了脚步,是今天的目的地睿思殿到了。

小黄门上去回覆,徐玑等候在殿前。

片刻之后,里面传出话来,让徐玑进殿。

睿思殿规模不大,远小于天子的寝宫福宁殿。但这里本是先帝书房,比起福宁殿更让天子感到自在。

徐玑已经来过此处多次,几乎每隔五日,他就要入宫一趟,为天子检查身体。

得到这一重任,不仅身份地位就此不同,也让他得到了许多同僚的羡妒。

但这依然是一份让他战战兢兢的工作。

天子、太后、宰相,还有……

“徐卿来了。”

赵煦已经脱离了变声期,完全是长成之后的嗓音了。

徐玑在天子略带放松的声音中谦恭行礼。

“臣徐玑叩见陛下。”

“好了,平身吧。赐徐卿座。”

应该不是错觉,徐玑觉得,比起他的那些医术丝毫不逊色于自己同僚们,天子更要看重自己。大概是因为自己最为谦恭,大多数时间都是低下头的缘故。

“还是老样子?”

待徐玑坐下,赵煦熟练伸出手腕,问道。

天子的手腕纤细白皙,指掌细长。与其说是男生女相,还不如说是自幼体弱的缘故。

“是,还是先号脉。”

徐玑说着探出手指,轻轻按住手腕上尺关寸。

感受着指尖上搏动,徐玑闭目不言,身边陡然静了下来,这就是给天子日常问诊时心情最平和的时刻了。

片刻之后,换了一只手,又把了半日,徐玑点头睁眼,却没立刻放开手。

“官家今天的精神不好,可是经筵上布置的功课太多了?”他信口问道。

指端的脉搏陡然间有了变化。

“是多了点。”

赵煦故作平静的答道。但明眼人一看便知是谎话。

“陛下。不要太过劳累,尤其不得熬夜。若是做不完,便先放着,明日再做也可以。”

“但今天功课最好今天做完。今日事,今日毕嘛。”

“但陛下御体更重要。就是到太后、相公们的面前,臣也是得这么说。”

赵煦没再多话,徐玑也放开了手。

身边的内侍问道,“徐太医,官家今日脉象如何?”

“脉象一切正常,只是血气稍弱而已。”

手边已经摆好了日历和笔墨,徐玑提起笔,写了几个字。

天子的身体情况,从这每天都要记录的日历中,便能搜索得到。日历之后,徐玑还有一份病历,以及写了字的部分,也是记满了赵煦每次问诊和日常用药的情况。

早在韩冈还是提举厚生司的时候,他就开始在京中医院推行病历制度,京城高官显贵人人都有一份病史档案,如今更是普及到了官员们的子女身上,皇帝更是不可或缺。

“官家的血气就补不回去?”

“胎里痼疾,得常年累月的调养。不过陛下的血气,也只是比富贵人家的同龄人稍弱,比起贫寒之家,还是要胜过不少。”

徐玑收起笔,一根干净的扁木条便递到了他的手中。

“陛下请张口。”

徐玑查看了赵煦的舌苔,又伸手飞快的扒拉了一下赵煦的双眼眼皮,

“陛下恕罪。”徐玑说道。

赵煦用力眨了两下眼睛,“每次都难受得紧。”

徐玑说道:“但眼疾得从一开始就预防,防微杜渐,比病发后再治要简单得多。”

“徐卿说得是,早就该防微杜渐。”

徐玑赞过赵煦的英明,又问,“陛下这几日的胃口如何?”

一本厚重的册子便摆到了徐玑的眼前。

不同于日历,天子每日的饮食自有另外一份记录。徐玑也用不着看前面,只看最近的饮食记录。

同样是一切如常,没有什么变化。

徐玑合上记录本,一根喇叭型的听诊器立刻递到了他的面前。

每隔五天的问诊,徐玑的习惯也给人摸透了。,

赵煦已宽衣解带。将听诊器的大头压在他的胸前,徐玑便侧过脸仔细静听从胸腹处发出的每一个声音。

“正常。”徐玑说道,放下听诊器,接过送抵眼前的一张纸片。

“陛下的身高,体重。”内侍解说到。自从有了病历之后,这些数字就成了关键。

徐玑看了一眼,便仔细的将几个数字抄录在日历上。

“陛下还是要多注重御体康健。”

比起两年前,赵煦的确完全变了一个人。改头换面得十分彻底。身高,体重都没有变化。

身高、体重在一天之内免不了有些变化,但这是每天都要测量的数据,最后都是要看平均值。

赵煦此时的身高跟同年龄的少年相差不大,但体重至少轻了五斤以上。而且见多了少年人,皇帝的身体情况,徐玑从对比中也能知道大概情况。

一旦髭须生发,男子就很难再继续长高了。

徐玑自己就是这样,十六岁之后便没再长高过,仅仅五尺三寸的身高,也是他心中的一块疤。

但赵顼身子骨更差,发育也太早。

从半年前起,皇帝的身高便没有太大的变化了,如果画成太医局中常用的纵横图,以身高为纵,年岁为横,天子身高的变化线,最陡峭的时期是在十三岁之前,尤其是十一到十二岁半的那段时间,之后便平缓了下来。仿佛从山地走向了高原。

但他现在仅有五尺一指,还不到五尺一。

在朝堂上,身高七尺的重臣都有,上次进京的太原知府吕大防便是一例。六尺出头的更多,出身北方的文武官员有十分之一超过六尺,做宰相的韩冈便有这么高。剩下的朝臣们也大多都在五尺五寸以上。

在军中,禁军基本上也都是五尺五寸。

太祖募兵,定等长杖,不如杖高者不取。真宗时将杖细分五等,最低也要五尺五寸。仁宗战事起,募兵渐滥,武肃、忠靖等下位军额,五尺者亦收。但这样俸禄有别,

至如今,募兵又重回真宗时。不及五尺五寸者不取,若是身高仅有五尺一寸两寸,只能入下等厢军,俸钱两百。五尺七八,或许能入上四军,俸钱也超过一千。

而身高仅有五尺的官员极为罕见,实在是太矮了。

不过这不是徐玑问诊的重点,确定了心肺活动正常,徐玑稍稍松了一口气。

每日的问诊就这么简单,徐玑检查之后,便准备回太医局。

但赵煦先一步叫住了他,“徐卿,等等。”

“陛下。”徐玑回身行了一礼。

“徐卿,”赵煦使了个眼色,让下面的小黄门递过了一个丝绢质地的小手袋,晃动间还能听到一二声叮当脆响,“卿家几次见朕,衣料总是如此陈旧,简朴虽是好事,却也不可太俭省,此物聊表朕的心意。”

徐玑愣了片刻,然后跪下叩头,砰砰有声。

接过赐物,他红着眼圈千恩万谢告辞出殿,返回太医局。在快进门口的地方,徐玑停下了脚步,一名吏员悄然出现在他身旁。

“官家的病历呢?”

那吏员很不客气的问着。

“你要原件?”

徐玑微微皱眉,不满溢于言表。

那吏员转得很快,立刻换上了一幅笑脸,“徐太医,我们又不是要你做什么,仅仅是担心官家的身体,所以才要病历一观。”

“给你。”徐玑不想再听,将病历飞快的塞到那小吏手中,“抄好后快点还回来。”

“放心,放心。”那吏员打开病历只在新页上扫了一眼,随即转身离去。

徐玑脸色苍白,泄露病家的身体情况,还是天子的,更为了钱财出卖,他早就自暴自弃,根本就不去想买家会拿这份报告做什么。

只是,那人的背后到底是谁?徐玑还是猜不出来。
