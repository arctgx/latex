\section{第27章 京师望远只千里(八)}

秉常今年虚岁十一,虽然苍白瘦弱了一些,看起来不像是个蕃人的模样,但他做皇帝——西夏国主对宋辽两国皆称臣而被封王,但在国内都是自称天子,青天子兀卒——也有三年多了,与宋国如今的天子登基的时间差不多长。

对年幼的秉常来说,每隔几日的朝会,就是一桩痛苦的工作。他背后就是垂帘听政的母后,秉常唯一的任务就是得像一尊土偶木雕一般老老实实、安安静静的坐在御榻上。除此之外,再无他事。朝臣们的奏报、面请,虽然都要带着对他的称呼,陛下、陛下的叫着,但实际上他们说话的对象,却是秉常背后的那人。

一旦在御榻上坐下来后,秉常就不能乱动,只有等到朝会结束后,才能放松下来。秉常其实很不满足于自己现在的任务。这个国家就是他的,他应该有权利执掌朝政。每次听着母后跟他的臣子们讨论政事,秉常都很想试着在其中插上两句,表现一下自己的看法。他的确这么做了,但一旦这么做了后,他便要对上自己母亲的冰冷眼神,以及接下来的责罚。

一想起因自己的轻率而受到的惩罚,秉常就有些不舒服。尤其是坐着亲生母亲的背后,就像有刀子在划着,不由自主的就扭了扭身子。

坐在用着玛瑙珠串串起的帘幕之后,当今西夏太后梁氏很不高兴的看着自己的儿子,像是背痒一样扭着身子。一对细眉微微皱起,吊起的眼角透着厉色。她的容貌如果放在宋国,的确算不上多出色,只能算是普通的美人。但在西夏这里,却没有几个党项女子能比得上。相貌出众,又有心计,若非如此,她也不可能勾引上前国主谅祚,而自己的前夫一家全都送到了九泉之下,自己当上了皇后,乃至现在的太后。

不过要坐稳这个位置,可不是像自己的儿子想得那么容易。蕃人不像汉人那样讲究什么忠义,单纯的弱肉强食,再无别的道理可言。如果不能让下面的这群豺狼虎豹满意,莫说坐稳现在的位置,甚至随时都可能把她和整个梁氏家族都给彻底毁灭,绝不是扭着身子就能解决。

前次举全国之兵五路南下,除了打下了大顺城周边的几个小寨,基本上没占到半点便宜。禹臧家负责的河湟,渭源一战是无功而返。而无定河那边,紧贴着银州修了罗兀城,两地只隔了一重山,在国人眼里,这就是步步退让的胆怯之举。

尽管自退兵后,梁氏兄妹付出了不小的代价,才换来了国中局势的安定,和梁家地位的稳固。但每次上朝时,都少不得有人拿着前次的失败来说事。

国相梁乙埋拿着一份奏报在朝堂上念着:“静塞军司嵬名讹兀急报,近一月来,又有三家部族南逃环庆。自此半年来,叛逃到部族已经超过了十家。如此下去,静塞军司恐其难保,不知诸位有何高见?”

一个声音随即响起,“在青冈峡修城便是。”

梁乙埋脸色变了变,又拿出一份奏章,“绥州都监吕效忠急报,东朝德顺军聚兵意欲北犯,奏请朝中派兵援助。”

同一个声音冷笑着:“在赏移口修城便是。”

梁乙埋被挤兑得脸色铁青,终于按耐不住,一手指着阴阳怪气的捣乱者:“都罗正,这城你去修?!”

都罗正是国中豪族都罗家的重要人物,其兄长都罗马尾领军在外,为一方大帅,军中地位甚高,连带着都罗正也是气焰张狂。他一向看不起梁氏兄妹,对梁乙埋领军的几次劳而无功的出阵,从没有半句好话,“还是相公修得好。绥德城外修了八座连堡,坚固万分,宋人望而生畏。离着银州那么远,还是把罗兀城修起来了……”

西夏的朝堂就是如此,完全不像大宋那样有着殿前侍御史紧盯着朝臣的言谈举止。只要背后有着足够的实力撑着腰杆,就不必给梁乙埋兄妹面子。

而被都罗正如刀一般的言辞划着脸,梁乙埋脸色由青转红。他正要发作,高高坐在最上面梁氏终于忍耐不住了,她不能看着她的朝堂变成妇人吵架的菜市口,“两边要出兵,今次不打,日后宋人可不会收手,肯定变本加厉,步步进逼。”

一闻此言,一位老臣顿时倚老卖老的叫起苦来:“刚刚打过了一仗,再想把部众点集起来没那么容易。何况下面的孩儿们多累啊,还是歇上一个月再说罢。”

有人领头,其他朝臣也便一起叫起苦来。不见兔子不撒鹰,不看到好处,就别想让他们动刀兵,这就是西夏部族的习惯。

不过梁氏兄妹在朝堂上也不是没有支持者,梁乙埋使了个眼色,方才没派上用场的十几人,一个接一个站了出来,与对手打起了嘴仗,顿时把模仿宋人起名做紫宸的大殿,闹成了菜市口。

‘可惜浪遇不在,不然没人敢乱说话。’梁氏低头看着朝堂上的乱局,心中惋惜的想着。

前任都统军嵬名浪遇资历极老,是景宗皇帝曩霄【李元昊】的亲弟弟。浪遇在曩霄被太子宁令哥所弑之后,本有资格问鼎帝位,但他却支持了尚在襁褓中的谅祚。他统领西夏大军垂三十年,是宗室中少有的没有私心、忠诚天子的臣子。如果有嵬名浪遇在朝堂上坐镇,只要出来瞪一眼,就没人敢再废话。

不过浪遇就是因为他的威望太高,对梁氏秉政也多有为此,最近被梁氏兄妹联手打压得很厉害,兵权一削再削,已经让他回家养老去了。

一场朝会没有商讨出个结果,便不欢而散,不过梁氏和梁乙埋倒没有灰心丧意。这只是通报而已,在政治上要作出决断,全得要靠在台面下处理的手段。

少了嵬名浪遇这个位高权重的重臣,在梁氏眼中,方才殿中的拿些碎嘴的废物仅仅是听着烦人。而要分化这些鼠目寸光之辈,也不是太难。

东边的仁多、西面的禹臧,两家都不是梁氏的支持者,但两家的族长没事都不会到兴庆府来。仁多零丁、禹臧花麻,这两人都不是简单的人物,而除去他们两个,剩下几个,却没几个能拿得上台面的。指挥军队的水平一个比一个差劲,只是要起赏赐来,却一个比一个贪心。

不过是诱之以利罢了。

“这些都是小事,两三千人就能处置得了。”

在朝会结束后的,在梁太后实际处理政务的御书房中,梁乙埋的脸上已经看不到方才被挤兑后的狼狈,仿佛方才的变幻莫定的脸色仅仅是装出来的一般。

“真正危险的是无定河,是横山。”

接口的是与梁乙埋一起被留下来说话的翰林学士景询。他是自张元、吴昊之后,又一个投靠西夏的汉人。

景询本是延州人氏,犯法当死,所以逃亡西夏。因为本有才学,受了先王谅祚的看重,授其为翰林学士。景询就跟张元、吴昊一样,最为穷凶极恶,日夜为西夏谋算,惹得大宋先帝英宗亲下谕旨,‘捕系其孥,勿以赦原’,把他留在宋国的妻儿都捉了起来。

其实不仅是景询,所有在西夏的汉人,对付起宋国的同胞来,都必须比党项人更加狠辣,否则在这个蕃人为主的国家,就不会有他们的立足之地。就像梁氏,她纵然是太后之尊,也无法像东朝皇帝那样高高在上的命令臣子。

所以三年前,重臣们逼着梁氏兄妹下令,用景询交换绥德城的嵬名山的时候,梁氏没有半点犹豫的便点头同意。尽管景询是梁氏兄妹的支持者,但牺牲他一人换取党项豪族们的支持,梁氏兄妹不会有半点迟疑。不过到最后,由于宋臣郭逵的反对,这项交换不了了之。景询继续做他的翰林学士,也没表现出半点芥蒂来——他不能,也不敢。

现在景询依然是梁氏兄妹的谋主:“近闻陕西宣抚韩绛已兼领河东宣抚,又得授同中书门下平章事,兼昭文馆大学士。他以首相之尊宣抚陕西河东,岂会甘心于守成?若真的要防守,何必要他来陕西?甘谷、绥德、河湟,”景询一根根屈起手指,“自东朝英宗晏驾,新天子登基,宋人在这几处步步紧逼。最近又有消息说,陕西缘边四路要整顿兵马,分二十万守军为五十二将。这是即将举兵犯境的先兆!”

“敢问学士该如何应对?”梁乙埋问着景询。

“河湟那边,可以联姻董毡。臣闻董毡有一子最得宠爱,可选宗室一女妻之。”景询将自己计策献了出来,和亲就是最简单,也是最节省的方案。

“董毡会愿意?”

“宋人步步紧逼,虽然尚有木征为其做屏障,但木征还能为他挡上多久?难道不会投靠宋人。董毡怎么会不担心?若能联姻大夏,岂有不愿之理?”

梁氏皱着眉头:“不过东朝势大,又即将北犯。纵然交好董毡,他手上的几万兵,对我大夏不过是杯水车薪。”

“太后勿忧,臣亦有良策可备宋人。”

“学士可有何良策?”梁氏有些好奇的问道。

景询抬头看着才不过十一岁的秉常,露出了一个一切尽在谋算中的得意笑容,“陛下年岁已长,转眼已到了婚配的年纪了。臣请太后至书北朝,为陛下请婚!”

