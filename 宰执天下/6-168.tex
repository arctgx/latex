\section{第16章 山入四荒更郁苍(中)}

“汝霖,你怎么看?”

韩冈很看重宗泽,这是朝**知。就是杨从先向韩冈禀报公务的时候,宗泽也被韩冈留下来,在旁旁听。

宗泽刚刚送了杨从先回来,想了一想,道:“杨总管想必是明白了相公的心意。”

“什么心意?”

“港口。”宗泽言简意赅。

韩冈抬了抬手,示意宗泽继续。

“高丽、日本,还有辽东,北海之上,诸多港口可供选择。一旦朝廷意欲平辽,水师便能泛海而攻。可没有相公支持,杨从先到时候何从立功?”

“话是这么说,可水师从来没打过像样的一仗,让人难放心。”

“辽国攻占了高丽、日本,需要防备的港口一下多了许多。只要挑软柿子捏,辽军如何防得住?”

韩冈点了点头。

杨从先口口声声说想打下哪个港口就能打下哪个港口,就是想在韩冈面前先讨个好。他是章惇所提拔起来的,身上也有新党的烙印,尽管之前在韩冈面前也算是很被看重,但杨从先就算是武人,也知道什么叫做党同伐异。现在在韩冈面前多说几句,日后攻辽,也能够避免来自政事堂方面的阻挠。像他这样的武将,想要立功于外,没有一个好后台、没有一个好人缘,根本就立不了功。

不过杨从先的想法,不是韩冈让宗泽琢磨的重点。

他起身,亲手拿出一幅地图来,让宗泽张挂起来。

天下九州舆图。

舆图上不仅仅有大宋诸路,还有四方诸国。北有辽国五京,东有日本高丽,西面已出葱岭,而南方,南洋周边小国尽在图上。

但这不是唯一的地图。在这一幅地图之下,还有一副图,韩冈同样让宗泽张挂起来。

辽国五京舆图。

韩冈退后几步,指着地图,问宗泽,“若朝廷攻辽,以汝霖之见,水师当先攻何处。”

宗泽抬眼看着地图:“兴城,觉华岛。辽西走廊。”

兴城、觉华岛,辽西走廊上的外岛。数百年后,抵御北方渔猎民族大军南下的战略要地之一。而走廊之名,出自韩冈之后。先是河西走廊,继而辽西走廊。

“为何不是日本、高丽?”

“水师之用,不在克敌制胜,而是兵胁敌国软肋。”宗泽斩钉截铁,“日本远在海外,三五艘巡洋舰便能将日本封锁在外。而高丽虽为辽国据有,但人心不附,朝廷当真要攻辽,可让高丽国王自耽罗渡海复国,吸引当地辽军南下,不需要官军直接攻占。而且这两处,离辽国本土太远,远隔山河大海。若要让辽人在河北河东不敢用上全力,只有用水师迅速的拿下辽东,进可攻打辽阳,退则稳守苏州,直接威胁辽国的腹心之地。辽军虽众,可一旦分兵辽东,用兵可就捉襟见肘起来。”

韩冈听着,连连点头。这些话,听起来简单,说起来就不简单了。这个时代,对辽国地理和海军应用,同时有着一定认识的人,可并不多。宗泽的见识,可以说是很难得了。

眼界的高低,能够影响日后成就的高下,杨从先虽不是什么名将,但他至少知道如何应用水军。这也是韩冈为什么看重他的缘故。韩冈希望他能够给初创的水师打下一个坚实的基础,以便未来的发展。

“就是不知道枢密院那边到时候会怎么使用水师了。”

“可不要小瞧章子厚。”

“宗泽不敢。”宗泽连忙说道。

“我曾经与章子厚商议过,攻辽最大的问题,就是他们的骑兵。”韩冈拿过自己的茶盏,喝了一口,道,“骑兵,离合之兵。想要取胜,就必须让其在我方选定的战场上进行决战,抵消聚散自如的优势。”

宗泽皱眉沉吟:“兴城海岛,无法吸引辽国骑兵。”

“兴城也很合适。不过以水师的实力,没必要局限在区区几处。”

宗泽回头看了看地图,问:“还有苏州?”

辽国有很多地方的名字是直接抄袭大宋,比如益州、银州、辰州、武昌等地名,都是在东京道上。而辽国的苏州,就是后世的大连。辽东半岛的尖端。

“汝霖,好好想想,这可是举国之战!”

宗泽考虑一下,点头道:“……宗泽明白了!”

攻西夏是六路齐发,若是攻辽,河北就要分东西两路,河东同样会自代州、神武军和胜州一起出兵。北海水师并不是一支独立的力量,尽管海陆有别,坐拥多艘火力无与伦比的炮舰,但也要配合陆上进兵的方略,而不是一家的单打独斗。

韩冈的问题就是一个陷阱,不管回答攻打那一处都不是完美地回答。

“枢密院那边也很重视水师,而且同样的是想利用水师克制辽人。”韩冈对宗泽说道:“如果在前两年,与辽人打起来的话,就准备按照汝霖你所说的,去把觉华岛占下来。断绝辽西走廊,兵胁东京、南京两道!汝霖,你说说,要是这么做了,辽人会怎么做?”

“……海外孤岛,攻打不易。当招聚大军,佯攻觉华,伺机南下。”

“没错。最早的时候,我与章子厚所拟定的攻打辽国的方略,是以守待攻,逐渐消耗辽人国力。用水师的优势攻占渤海外岛,逼辽人兴兵南下,在河北边境上进行决战。”

“此乃良策。”

不管怎么说,以举国之兵北上攻辽,最大的风险就是辽国的骑兵,大军行军到半路上,一队宫分军杀来,即使装备再精良,败阵的可能都不小。

若大宋水师攻占觉华岛,将直接威胁辽西要道,对辽人来说,是骨鲠在喉。可这块骨头难以拔下来,那么摆在辽人面前的手段,就只有南下攻宋,待占据优势后,逼宋人自己退军。

“不过这么做有两个难题。第一个难题是战船,要封锁区区数里的海峡,让辽人无法抗衡的战船必不可少。第二个难题……”韩冈停住了,抬眼对宗泽笑道,“汝霖,你说是什么?”

“寨堡。”宗泽立刻回道。显然他心里已经考虑过了。

“的确是寨堡。”韩冈满意的点头笑道,“如果战争按照预期开始,为了稳固觉华岛,就必须在岛上修建城寨。可是觉华岛附近的海面,有半年的时间封冻着。从十月到三月,战船无用。深冬时,更是能踏冰登岛。不能在短短几个月时间中,将城寨修好,等辽军踏冰而来,这一仗不用打就输了。”

“现在都不是难事。”宗泽道。

近年来,河北的州县城墙正在大规模的改建,修筑成使用火炮的棱堡。时间一长,能够指挥工役的官员数量就多了起来,其中有不少被韩冈所看好。

而且修筑的棱堡多了,怎么修建也都有了经验,时间、人力、物力,在看到地形、得到要求之后,就能有大概的预测。半年时间,修筑一座驻扎一两千兵马,控制岛屿内外,同时让敌人难以攻破的棱堡,并不是多难的一件事。尤其是对经验丰富的河北军民来说,绝非难事。

“的确不难了。不过现在的官军的实力更强了,比当初谋划的时候要强得多,而且每一天都在变强。再等几年开战,就不需要这么麻烦,直接挥军北上。到时候,水师的用途,就不再是辅助,而是主攻的一路了。”

“在觉华岛上岸?还是直接去攻击榆关?”宗泽又盯着地图,皱眉问道。

榆关就是山海关。两京锁钥无双地,万里长城第一关。这两句,在此时,前一句还能够凑得上。榆关控扼辽国南京、东京的要道,也是辽国国内最为重要的关隘之一。

拿下海上的觉华岛,是在北虏的尾巴上点了把火,而攻下榆关,却是对准尾巴下面的洞,挺枪直刺进去。

辽人会发疯的。

“这也是一个办法。”韩冈点点头,接着却又笑道,“不过,汝霖,你可知道桑干河和辽河,都能通行船只。”

宗泽惊讶问道,“战列舰能入河口?!”

他可是听说战列舰吃水不浅,有些港口根本进不去。

“进不了,巡洋舰都难。但北海一带最多的平底防沙船,进入内河不用担心搁浅。快速的逆流而上,也不只风帆一个办法。”

“火炮也能装?”

“当然。”韩冈喝了一口茶。

“那就好办了,只要这样的战船数量多一些,运上一两万兵马至辽阳城下,或是析津府城下,猝不及防之下,直接攻下两座京城将会轻而易举。”

韩冈微笑着点头。

“但难就难在猝不及防。”宗泽敛容道,“隐秘二事,不容易做到。想要攻下两座京城,就要有所准备,一旦开始准备,必然会有泄露的可能。”

“泄露也无妨,看着情况不对,上船就走,辽人也拦不住。”

“可河中水文不明。”宗泽道,“主航道或许容易行船,但辽人的船,封锁河面不易,可要堵住河中主道,却也不难。”

“辽河和桑干河上的商船可不少。”

宗泽不说了,那些商船的背后,肯定是间谍无疑,而这些事,不是他该知道的。

“船只已经修好了吗?”宗泽问道。

“还要等几年。这件事不用急。越往后,北虏与中国的实力将会差得越远,这不是一两个明君贤臣能改变的。”

