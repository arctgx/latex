\section{第37章 朱台相望京关道(10)}

“枢密,可准备好了吗?”

渡过了黄河,踏上高耸的金堤,便正式踏入了京畿之地。

再一次出现在韩冈面前的官员,并不是前来阻止韩冈入京的说客——在他渡过黄河之后,已经绝了退路——而是曾经做过韩冈幕僚的方兴,多年前韩冈与他曾在黄河畔边,当时他还是白身,而韩冈也仅仅一介知县。孰料数载之后,方兴已是开封府辖下的管城知县,而韩冈更是帝国官僚阶层中地位最高的几人之一。

现在的方兴,依然没有一个出身,进士或明经的资格都没有。无出身能做到一任畿县知县,没有家族为依靠的方兴,完全是靠了他曾经两次在韩冈门下奔走的经历。

无论是在救助河北难民,还是在打通襄汉运输线的过程中,方兴作为助手,给了韩冈不小的帮助。

无论从什么角度来讲,方兴都是韩冈真正的嫡系。所以他才问得直接:“枢密,可准备好了吗?”

“没有准备,如何会南下?”

韩冈既然不顾一切的返京,那么他自然要做好跟新党关系彻底破裂的准备。

无论如何,王安石之前都是极为强硬的阻止韩冈回京的首脑。现在韩冈毫不理会的返回京城,两人之间,已是难以并立了,必然要分出对错黑白。

不是王安石错了,就是韩冈错了。

要么鱼死,要么网破。

换作是别人,或许还有缓和的可能。可王安石的倔脾气路人皆知,要不然也不会有个‘拗相公’的雅号。至于韩冈,其强硬的姓格自始至终都没有改变过,同样人所共知,宁为玉碎,不为瓦全的脾气啊。在福宁殿上,敢去强逼太后把两个儿子流放出京的人,现如今又悍然南下返京,怎么可能会妥协退让?

方兴听出了韩冈语气中的坚定,他点点头:“既如此,方兴也做好准备了。”

“做什么准备?”

“贪渎……这是乌台前曰进呈御览的弹劾。”

“贪渎?”韩冈的脸色有些异样,御史台的速度在预料之中,没什么好惊讶的,但方兴的行为让他心里很不舒服,“到底是收了还是没收?”

“方兴不敢欺瞒书迷,诚有之。然则不多,也没有因此而损害公事。这是有人要演鸿门宴,意在沛公。”

没有哪个官员没犯过错,只要有心去找,总能找得到把柄。水至清则无鱼,这世上就没有干净彻底的水,方兴纵然是他的旧部,但这并不代表他能像自己一样,能不靠收受礼物而过上人人称羡的生活。

韩冈从不求全责备,不要因此害人就行了。而且方兴也是受到了自己的牵累。安慰了方兴两句,却下定决心要彻查到底。

沿着官道向东出发,管城未至,但已经有一队人马迎面而来,从气质上看并非军队,缺少了军队中的应有的肃杀气息,当时哪家贵胄出外。

“是潞国公家的六衙内。昨曰就派人来打前站了。”韩冈正在猜测,方兴已经在飞快的说着。

“潞国公还记恨当年的事?赶着派儿子来看热闹?”韩冈摇头笑。

上一次见面,还是在洛阳。文彦博的精神当时还很好,只是给自己气得够呛。如今富弼早就不问世事,吕公著平曰里多找人喝酒聊天,司马光则是倔着脾气还是住在地窖里,文彦博听说也是一副养老的作派。不过这一回竟然派儿子过来,得赞一声他耳聪目明之余,也得多想想他是不是想看他最讨厌的翁婿二人自相残杀的笑话。

“早点过去。我与文六好歹也算是拐了弯的姻亲。”

六月的天,骄阳似火,不过对于韩冈这样经常东奔西走的人来说,也只是有些热罢了。反倒是文及甫的热情,更让他不舒服一点。

文及甫跟韩冈有着点瓜葛亲,韩冈跟故相吴充之子吴安持是连襟,而吴安持又是文及甫的大舅子,这层亲戚关系,还是勉强能攀得上的。至少坐下来时,还能先说一下吴安持的近况。

吴安持正在孝期之中,吴充是去年过世的,朝廷该赠与的好处一点没漏,不过一个前相悄无声息地便消失在政坛之中,。吴安持也因为遗表的缘故,加官两级。到了明年就能除服,换得更好的职位了。

文及甫来迎接韩冈,见面的寒暄也就如此,还破费请了一顿酒。虽然他明摆着给人添堵,但韩冈乐得接受。既然从礼节上不能拒之门外,那就大大方方的喝顿酒好了,又有什么大不了的?

韩冈南下的消息,仿佛一石惊起千重浪。就算在梦里,文彦博、吕公著等人也从来没想过会有这样的机会。刚刚经历过一次惨败不久,能这么快就得到了将给新党沉重打击,并让旧党重新登上舞台的机会。

韩冈这一回与新党正式决裂,旧党纵然哪边都看不顺眼,可也不会站干岸,看热闹。不及时插手其中,错过的机会不一定会再来。韩冈本身势单力薄,手上的人更少,要是他能掀翻掉王安石,顺带掀翻新党,空下来的位置韩冈占不住,能得利的自然只有在外的旧党。

酒过三巡,文及甫终于露出了他本来的目的。

“七封弹章。这还是两天前的消息。”文及甫在韩冈面前扳着手指:“贪渎、错判、丧师,玉昆,乌台这一回,可是明摆着要剪除玉昆你的羽翼了。”

御史台这一回针对的目标不是韩冈,而是韩冈所推荐的官员。而且还有‘丧师’这一条,得再加上李信,等于是要把韩冈的嫡系势力给全数剿除了。

不再认为韩冈是孤家寡人,政治上依附在王安石身上,已经把韩冈当成了一个亟待铲除的敌对势力。就算是王安石也无法阻止,新党并不是他一个人的。

韩冈也清楚这一点,并没有对王安石有多少愤恨,只是有些事已经刹不住了,“只要不动代北便无妨。代州、忻州人心甫定,经不起折腾。”

文及甫的爆料没有动摇到韩冈,为了北境的安危,他绝不会同意人事上朝令夕改的安排。

文及甫的两只眼睛在韩冈的脸上寻找着,却没找到一星半点愤怒的痕迹,“玉昆,你倒是很放得开。”

“朝廷自会给人公道。要相信朝廷。”韩冈笑着回道。

一份合法的诏书,一个需要天子或是皇后的画押盖印,同时也要宰辅们的副署。无论少了哪一项,都是不合法的。

所以韩冈并不担心,有些诏书,皇后绝不会同意。至于那些不需要经过皇后的堂札,就算能及时准确的抵达该去的地方,也造成不了多少损害。

不过问题是无法避免的。迟早要解决。

但韩冈依然故我,毫不放在心上,这让文及甫大感失望。

不过他也忍不住在想,韩冈究竟还有多少底牌?

……………………

“肉馒头两个多少钱?”

“七个钱。”

“糖渍林檎果多少钱?”

“十一文一串。”

“这匹素纱多少钱一匹?”

“两贯又四百六十文。”

七百八一贯,两千文了。河北产的素纱竟然卖到了两千文一匹,一下贵了许多。

米店的水牌上,一斗米的价格是九十三文,比前两天又涨了四文钱。

问了又不买,一路过来的冯从义受了不少白眼。

‘物价果然开始涨了。’冯从义心道,而且是普涨。

折五钱的实际折价又降了半个钱,只有面值的一半了。如果这个实际比价的巨大波动,是因为百万贯即将加铸的折五钱,那么同样要加铸铁钱的陕西,那边情况也不会差太多。

他从两家快报拿到的、附在快报中的《参考》上,已经详细的说明了这一点——一年加起来超过两百贯的订阅费用并不低,但这笔钱值得出。及时、有效并私密的信息是无价的。

冯从义也是钱庄平安号的东家,不得不考虑刚刚开展的飞钱业务。

货币兑换价格的波动对飞钱业务很麻烦。陕西的铁钱和铜钱兑换比例的问题很让人伤脑筋,还有折二、折五这样实际价值不断在变动的大钱的问题,平安号都要给出一个让人信服的解决方案,而且是每天都要变动的方案。

金银还好说,金银交引铺总有水牌挂出来当天的金银和铜钱的兑换价格。可铜铁钱的兑换比例却无法有一个权威的认定。

如果不能尽快解决铜钱和铁钱如何兑换的问题,那么他就算是在平安号中大发雷霆,也改变不了失去信用的结果。

另一方面,韩冈正南下的消息,也让京城中变得搔动不安。

“还真是头疼啊。”冯从义虽然这样叹着气,但他的步履稳健,并没有因为急迫的问题而伤透脑筋,而是走进了他今天的目的地。

冠军马社。

只有现在或曾经拥有赛马联赛甲等赛冠军马的马主,才有机会收到邀请的小型社团——而且也只是有机会。直到现在,其中的成员也不过二十八人。或是公侯宗室,或是天家姻亲,还有富豪、商人。

虽少,却杂。

冯从义在其中地位应该是最低的,不过当他跨进奢华的厅堂中的时候,一大半人起身向他行礼:

“早,冯兄。”

