\section{第38章 何与君王分重轻(12)}

赵佣仰头看着新来的先生。

从阁门外照进来的阳光,被高大的身影遮住了大半。

作为太子,赵佣与宰辅之一的韩冈已经见过许多次了。

其中印象最深的当然还是半年多前,冬至夜的那一个晚上。

跟所有相公一样,来到父皇的病榻前,都是十分严肃的样子,但是独一无二的年轻。说话声音不大,可不知为何,让赵佣一直都很害怕的祖母却一直瞪着他,最后还大发雷霆。

那一夜,赵佣一开始并不是很明白发生了什么,但从那一天开始,喊他六哥的人少了,都开始称呼他做太子了。也是从那一天,身边的人都开始在说‘幸好有了韩枢密’。

待年轻的韩枢密行过礼,赵佣立刻恭恭敬敬的回了一礼。

储君亦是君,纵然是贵为平章军国的王安石也要先行礼。但赵佣被耳提面命,对宰辅们要尊敬,决不可有失礼的地方。

宫中有专门的人来教授礼仪上的知识。赵佣在这方面做得很完美。

不过他旁边的王益就不行了,向韩冈行过礼后,就定着不动了。

赵佣侧过脸想看看怎么回事,就感觉的王益悄悄的扯了扯自己的袖口。手指微抬,指着阁中一侧的一张桌子。

赵佣看过去,就在那张桌子上,正放着一根尺子。一尺长的木尺,仅仅有小半寸搭在桌子的边缘,尺身几乎都悬在空中。而尺子的下方,还吊着一个锤子,用一根细绳连接。锤头是生铁的,看起来就很重,使得铁锤的木柄高高的翘起,抵住了木尺。

‘这是怎么回事?’

赵佣一下瞪大了圆溜溜的眼睛。这么放在桌边上的尺子,肯定应该掉下去的。

先一步过来的宋用臣早就盯着桌边上很长时间了。

锤子、尺子,还有绳子都是他让人拿来的。将锤子绑在尺子上,再摆好在桌边,都是跟着他的小黄门动的手。韩冈只动了动嘴皮子,却像是戏法一样让人不可思议。

这是什么鬼法术?宋用臣想问又不敢问。跟他一样,阁中的内侍,还有刚刚进来的赵佣、王益以及跟随他们的内侍、宫女和乳母,一大批人都瞅着,一脸的不可思议。不过他们都是瞧了几眼后,就端正了身子,只用眼角去瞟,又用眼神交换着自己。

‘肯定是胶。’

‘假锤子。’

‘是韩枢密啊。’

韩冈知道,肯定会有人想不通。就是在后世,多少学过物理、好端端从初中毕业的聪明人,都一口咬定决不可能。在现在这个时代,又有几人能想得通透?

这就是他的目的,先声夺人。

韩冈咳嗽了一声,两声,将所有人的注意力都拉过来。他是老师,不能任由自己的学生陷入迷糊之中。而且,他也要上课了。

韩冈看过赵佣以前做过的习题,其实算式和记录采用的正是如今世间通行的草码数字。

草码原本是商人中所用的,直接画在搬运的货物箱子或是麻袋上,箱中货物的数量看看外面的标签就知道。有时候,简单的账簿记录也用草码。

不过现在通行于世的草码,已经经过了改进。旧草码的一二三就是简单的将文字一二三扭转九十度给竖起来,而改进过的草码一二三无法通过添加笔画来篡改。结合了一部分阿拉伯数字进行的改进。使得有人篡改数字,也很容易看出破绽来。这是韩冈主导的缘故,所以在关西许多小学校中,都在用这本便宜又好用的算学蒙书。

当然,在真正的账簿中,不可能是单纯的草码,还必须有大写的数字。

不论是民间还是朝廷,账簿上的数字,作为确认标准的都是大写数字,甚至于都不用草码和小写的一二三,只用壹贰叁。这是从唐时就流传下来的习惯,如今更是普及到全国各地。尤其是官府——‘今官府文书凡计其数,皆取声同而画多者改用之。于是壹、贰、叁、肆之类,本皆非数,直是取同声之字,借以为用,贵点画多不可改换为歼耳’。

“乘法和除法,殿下和团练应该学过吧。”韩冈问着两位身份尊贵的学生。

“九九歌,我都会背了。佣哥比学生会得还早。”王益很自豪的说着。

宋用臣也在旁补充:“太子聪慧天生,现在是百以内的加减乘除都没问题。”

韩冈早就听说赵佣早慧,小小年纪就沉稳过人。他对此早就心中有数,可亲眼看见还是觉得惊讶。

会背九九乘法表其实不算什么,还没把一张大半部分只有加减的卷子做完的王益其实也会背,可根本就不会灵活应用。韩家家里的老大老二,也都是在四五岁的时候就被王旖逼着背熟了,只是在运用上是整整花了两年去练习。

但赵佣现在却能进行百以内的加减乘除!

六岁啊,韩冈暗暗惊叹着。

正常的小孩至少到九岁才能拥有的才能,赵俑现在就拥有。

跟九岁左右擅长数学的小孩子差不多。赵俑也可以算是天才,但还远不及数学史上的那些怪物,比如高斯之流。也比不上自称八岁就能看懂《海岛算经》的沈括。不过一年以后,四则运算肯定是没问题了。

“那就好,先把这几题做了,看一看到底学到了哪一部。”

韩冈第一次上课,就拿出了一张考卷,将数学上容易遇到的难点都变成了考试的内容。只要学过一阵数学,就应该能答得上来。

一盏茶的功夫,赵佣先做完了,而王益则吭哧吭哧算得满头大汗,看看卷子,不过写了一半。

果然是聪敏过人。韩冈心道。

不过赵佣虽然聪明,但体质比寻常的孩童要瘦弱许多,脸也是苍白的。跟同年的王益,也要矮小半个头。相较起来,韩冈的儿女们脸颊都是红润有光,入夏之后,因为学习骑射的缘故,老大老二甚至都晒得发黑。体质上的差别实在是太大了。

赵佣自出生后就多病,半年多来韩冈不在京师,幸而没有大碍。否则皇后能把提议韩冈出京的人给流放到海外去。

也真是好运气了。

韩冈拿着卷子一眼扫过,发现是几乎都做对了,只有两题是错的。

想想还真是难得。

从卷子的上来看,赵佣至少是后世小学三年级的水平了。学完韩冈给蒙学编订的教材,差不多也就是在这个水平线上。但赵佣可才六岁。当真是聪明呢。

做完题后,赵俑百无聊赖的等着韩冈的发落。而看了看王益还有很长一段才能写好,韩冈便对赵佣道:“殿下若有空,就再做道题好了。”

赵佣点了点头,说了声‘好’。

“题目很简单,一加二,再加三,再加四,就这么一直加到一百就可以了。也就是一到一百,这一百个数字的和是多少。”

赵佣立刻拿起笔,在纸上计算起来。

又过了半刻钟,王益终于写得差不多了,赵佣却还没算好,看起来还是差一点。

“先歇一歇。”韩冈示意赵佣停笔不要再算了,他竟然选择了最麻烦的死算。可以说终究还是差了一筹,比不上那些真正的数学大家。

“臣说件旧事吧。跟象戏有关,也跟数算有关。”

王益立刻丢下了笔,竖起耳朵听故事。而赵佣仍是端端正正坐着不动,就是眼睛眨着,还是很有兴趣的样子。

“不知殿下、团练可知象戏?”

赵佣和王益用力点头。

如今世间象戏的种类很多,大象戏、小象戏,七国象戏。但最流行的还是韩冈所创的楚汉象戏,规则简单,布局也简单,加上韩冈的名气也大,所以很快就流行开来。在宫中也多有人会下。赵佣和王益至少都看过人下棋。

“当年臣跟枢密院的章惇打了赌,臣若输了,就赔出百贯彩头,若是他输了,那他只要赔麦子就够了。”

“百贯的麦子?”

“好像很多的样子。”

赵佣和王益交头接耳,宋用臣也在心底计算着麦子的数目,但韩冈的接下来的话实在是出乎意料:

“是按粒来算。第一个格子放一粒麦子,第二个格子放两粒,第三个格子放四粒,第四个放八粒。就这么一格加一倍的加下去,将六十四格都放满就行。”

‘这么少?’阁中的每一个人都闪过了同样的念头。

“然后呢?”

“然后章枢密便说,除非将赌注交换,否则他绝对不赌。也就是我出麦子,他出钱做赌注。不过这就轮到臣不干了。”韩冈笑了笑,“后来臣又用同样的条件,打算跟曾经做过三司使的沈括下棋。可是他一听之后,就不干了,说倾家荡产也赌不来。”

“先生,只是几粒麦子啊。”王益忍不住开口。

韩冈脸色严肃了起来,“做学问,讲究的是诚实。诚于实。最不好的是只凭空想说好坏。真的只是几粒麦子吗?究竟是多少,还是算了之后再说!”

韩冈之前都是带着笑,看着也和气。虽然上课前,都被耳提面命要老老实实。但韩冈没摆出师长的架子,王益的胆子也就大了起来。不过现在韩冈脸稍稍一板,他立刻就老实了。

第一次上算术课,韩冈只是让做了一份卷子,又说了一个故事,最后再把卷子的错误之处给指了出来,一一加以讲解。不过两个学生的底细算是摸透了,主要是数学方面的才能,王益比赵佣差一点,而赵佣再过几年,会变得更出色——毕竟人聪明。

结束了一个时辰的课程,赵佣和王益一同向韩冈行礼,表示自己的感谢之意。

韩冈回礼之后,指了指依然稳定在桌沿上的尺子和铁锤,“那个就不想知道缘由?”

“还请先生赐教。”王益连忙道。阁中所有人都精神一振,他们都已经纳闷了一个时辰了。

“这就是课后的习题了。”韩冈却没有直接给答案的好心眼,“今天就三条,一个是‘从一加到一百是多少’。一个是‘章、沈两人都不肯赌的缘故,棋盘上要放多少粒麦’?最后一个就是‘到底为什么尺子不会掉下来’?下一次上课时,把答案准备好。”

赵佣和王益发着愣回去了,宋用臣立刻填补了上来,他问着韩冈:“韩枢密,这真的不是戏法?”

韩冈一下就变得脸色阴沉。宋用臣一个激灵,立刻反应过来,韩冈最讨厌的就是怪力乱神。

“但太子才六岁,肯定不知道怎么做。”

“有什么难的。想不通就不能问人吗?只要亲笔写好答案就够了”

宋用臣吓了一跳:“可以问人?”

“有谁能事事皆知。有不明白的地方当然要问人。最怕的就是自以为是。”

不是每件事都要充专家,而是学会寻找专家来咨询才是上位者该做的事。至于怎么挑选,相信谁,这就是关键。

相信宋用臣会明白,他背后的两个人也会明白。三道题目更是出给他们看的。

韩冈要教授的并不局限于知识,更重要的是学习的方法。怎么做事,怎么思考。

他的目标,就是给赵佣塑造出科学的人生观、世界观、价值观和方法论来。

做事,先学做人。正心,先正三观。

这是韩冈的想法。

