\section{第44章 文庙论文亦堂皇(五)}

【第二更,求红票,收藏】

王旁冷淡的说着客套话,韩冈开始后悔方才的试探,多说了两句话就把王旁得罪了,现在他说话都是冷冰冰的,与自己交换着没有诚意的恭维。这样的气氛,化解起来难度不小,让韩冈说起话来感觉很累。吃力的与王旁继续说着没意义的废话,却一眼瞥到了摆在坐榻一角的一个带脚棋盘,就放在手边,显然是经常使用。

韩冈顿时有了主意,刻意把视线逗留在王旁身后的地方。王旁心有所觉,回头一看,却见是自己常用的棋盘。大概同样是因为跟韩冈说话太累,王旁回头看到棋盘后,立刻如释重负,提议与韩冈手谈一局。

“不知韩兄会不会下棋?”

围棋韩冈当然会下,不过就是个半吊子,无论前世今生。而且宋代的围棋规则与千年之后差别很大,韩冈也只是凭着前身的记忆,以及后来跟王厚等人下过的几局,粗略的了解到一点。王旁如己愿提议下棋,韩冈当然不会拒绝,心想干脆趁机输个几盘,缓和一下跟王旁的关系也好。

这么想着,韩冈便拱了拱手:“在下棋艺疏浅,还望王兄手下留情。”

“哪里,在下的棋艺也不高。”王旁谦虚着,让人撤去了榻上的茶几,又亲自把棋盘和两个装棋的木盒子搬过来。

棋盘和棋盒都有些破旧,面子上有不少划痕,看起来颇有点年头了。放好棋盘,打开盖子,里面的棋子是陶瓷烧制而成,底部露胎,只有上半部才有釉面。虽然有些陈旧,甚至一眼看过去,发现有好几颗都崩了口子,但材质优良,摸上去温润光滑,应该出自于定州或磁州的名窑。

坐到棋盘边,王旁神色便是一变,庄重肃穆,全神贯注,精气神简直是换了一个人。王旁能主动提议下棋,水平当然不会差,但看他现在的模样,韩冈便是心中微微一惊,莫不是碰上了个国手吧?

韩冈过去跟王厚下过几盘,但王厚的棋艺差劲得可笑,先是乘着韩冈规则不熟赢了两局,接下来,便一路败下去,毫无还手之力。跟韩冈下不赢,王厚又转过去找王舜臣他们下。

谁知道王舜臣和赵隆虽然连棋盘十九路都数不全,但李信却是高手,跟王厚赌了一子十文的彩头,一局就从王厚那里赢了四百个大钱。李信赢了钱不敢要,王厚倒是赌品甚好,老老实实的把赌帐给清了,还千叮咛万嘱咐,千万不能让他老子知道。不过自此之后,就不敢跟李信再赌棋。

韩冈也跟自家表兄下过,每次都是在中盘就输得一塌糊涂,从没有拖进官子过。现在看着王旁的模样,比起李信下棋时还要更有高手风范,韩冈此时已经不是想着输个几盘,缓和一下关系了。而是要争取表现好一点,不至于输得太惨,免得丢人现眼。

韩冈远来是客,便执白先行。两人在棋盘的四个星位各自放下两子,这四个子称为座子,在开局前就放下,也是此时围棋的规则之一。

从棋盒中拈起一枚,韩冈右手落下,啪的一声响,一颗白子就摆在了棋盘上。王旁摆子相应,方寸之间的战场上,顿时燃起了战火。

韩冈喜欢下快棋,很少长考,没想到王旁同样爱下快棋。在棋盘上两人落子如飞,只听得啪啪的放下棋子的声音。几步下来,韩冈就发现王旁也不比自己强到哪里,都是半桶水的水平。韩冈的棋风一直以攻为主,全凭蛮力,这也是半桶水的通病,而王旁竟然也是一样,在棋盘上,两人杀得难解难分,一时间甚至找不到一块完整的棋形。也不过半个时辰的功夫,就到了收官的盘终。

宋时围棋规则并没有‘目’这个说法,只算地盘,占了多少实地,就算多少。空也好,子也好,一股脑儿都算进去,只是不计眼位。最后两人一算,韩冈在盘面上差了王旁一个子,但韩冈的棋型分作四块,比王旁琐碎的六块棋要少上两块。照规则王旁得还回两颗子,这叫还棋头。如此一算,韩冈反而赢了一子。

“承让!”韩冈拱手笑道。

王旁与韩冈一般的烂水平,正好旗鼓相当。厮杀得痛快无比,下得兴致高昂,即便输了也不计较。他等不及的叫着:“再来!”

两人换了先后手,这次由王旁先落子。方才韩冈饶了先,却只赢了一子,轮到王旁先手,他便是信心十足。一番酣战,这次倒真是让王旁赢了韩冈三子。

一胜一败,连下两局之后,王旁兴致尤高,他很久没有这么痛快的下过了。找的棋友几乎都是因为王安石的关系,对局时都让着他。这样赢了王旁都觉得没趣。只能闲暇时跟自家妹妹下几手。现在碰到跟自家水平相当、棋风相似、又肯全力厮杀的韩冈,当然不肯轻易放过。

但韩冈却不想下了,他过来又不是来下棋的。听着外面的更鼓,都要往三更走了,王安石那里还没个消息,想来今天是见不到了。韩冈不打算傻乎乎的等下去,那样反而会降低自己在王安石那里的评价。

“难得下得这般痛快,真想再多下几盘。”韩冈笑着站起身,“只是时候已经不早,在下得告辞了。”

王旁惊讶的陪着站起:“韩兄不是来见家严的吗?怎么现在就要走?!”

“现下已近三更。相公今日刚刚病愈复归,明日又要早朝,韩冈再不晓事,也知不能耽搁相公休息。左右在下最近还要留在京中一段时日,好等官诰下来。等过几日相公有闲,使人往城南驿传话,韩冈必会再来求见……哦,对了,”韩冈从袖中抽出王安石的名帖,“相公的名帖韩冈实在担不起。”

韩冈作风强势,而王旁虽然是执政的亲子,但生活在光芒四射的父兄长辈的阴影下,他的性格中其实有些软弱。被韩冈先声夺人,王旁也不知该说什么好,却糊里糊涂的送了韩冈离开。

而王安石这边才刚刚说完,吕、曾、章三人分别把自己衙门中最近的一些要事向王安石做了汇报,又商议了一下接下来的对策。等到一切抵定,吕惠卿才道:“参政,韩冈方才到了,由仲正陪着,要不要见他?”

“韩冈?!”王安石还没说话,章惇却先一步问道,“是哪里人氏?”

“是秦州来的。由王韶所荐,河湟的事都得向他问个清楚。”

吕惠卿说着顺带看了章惇一眼,却见他面有讶色。吕惠卿有些奇怪,这章子厚不是会大惊小怪的脾气,过去他跟苏轼一起游山,走到一座独木桥边,苏轼胆小不敢过,而章惇却大摇大摆的走过去,还在山壁上题了名。怎么听个名字就这么吃惊?

“他的表字是不是玉昆?”章惇继续追问。

“当然,玉出昆冈嘛。”

王安石也看出章惇的神色有些不对,“子厚,你认识韩冈?”

“是家严认识。”章惇收起惊讶,回复了从容淡定,正容道:“家严昨日刚刚自关中访友而回,听他说起了韩冈。前日家严在官道上不幸碰上了狼群,车子被上百条狼围在中央,几乎性命不保。若不是韩冈和另一位唤作刘仲武,准备试射殿廷的军汉,一起杀退了群狼,家严怕是要葬身狼腹,这是救命之恩。”

“竟有此事?!”王、吕、曾闻言均吃了一惊。

章惇道:“我听到此事时也是不敢相信。可毕竟是家严亲身经历,不会有假。”

曾布在政事堂奔走,自是知道韩冈这个人,他对章惇道:“看王韶的荐章,里面说韩冈在押送军资时,曾领着三十余民伕,击败数百埋伏于道左的蕃贼,斩首三十一级,缴获军械近百。还说他当时亲手格杀了两名蕃贼内应,勇武是不用说的。当初我也是有些难以置信,但韩冈既然能在群狼中救出尊翁,那就是板上钉钉了,不会有假了。”

王安石道:“韩冈据称文武全才,王韶的信中将之比为张乖崖。”

吕惠卿点点头,“王子纯【王韶】说的不错。韩冈亲笔撰写的一部伤病营管理条例,我正好看过。两万余字的条例,六大项,七十余条,条理分明,事理详细,方方面面都考虑到,治才在他这个年纪无人能及……他可不仅仅是武勇。”

“韩冈的德行也不差……”章惇感叹道,“他救完人后,上马就走,也不留下姓名。若不是家严紧赶慢赶,一直追到驿站,怕是连他身份都不会知道。后来送得谢礼他也是一分不要。家严回来后就一直在说,此子大有古人之风。”

几人把有关韩冈的信息合在一起,一个文武双全,品德高致的青年俊杰的形象便出现在眼前。王安石一拍桌案,为自己的怠慢后悔,“如此英才如何让其枯坐偏厅,来人,快把韩冈请过来!”

可片刻后,却是王旁走了进来,道韩玉昆已经走了。

“怎么就让他走了?!”王安石有些生气。

王旁讷讷的低声回答:“他说是大人明日还要早朝,不敢再打扰。等大人何时有闲,他会再来拜访。”

章惇笑道:“想不到这韩玉昆还是有点脾气的!”

若是没有方才的那段议论,几人说不定会因此而对韩冈心生反感,但现在一看,却真觉得韩冈的确是才高气壮,所以才能来去无碍。

“无妨,三哥儿你明日亲去城南驿,把韩玉昆好生的请来。为父也有许多话要问他。”

