\section{第33章 旌旗西指聚虎贲(三)}

对于王中正来河湟监军,韩冈说不上多欢迎——并不是源于文臣对宦官天然的歧视——仅是认为多一个人来分功,其他人的份量总会少上一点。

但这个职位落到王中正身上,倒也勉强能说是不幸中的万幸,总比其他阉宦来监军要好。至少王中正在罗兀撤军时,做得还算不错。虽不是主动到罗兀来,却也没有像边令诚之于潼关、鱼朝恩之于北邙那般插手军务而坏事——要韩冈来评价,可以说是本份。

至于王中正当初到秦州宣诏时的贪财受贿,那就是小毛病了,以现今陇西榷场的利润丰厚,怎么都能填得满他的胃口。

虽不是最好的结果,但勉强也能接受,这就是韩冈还有王韶、高遵裕对王中正来监军的看法。

不过王厚初闻乍听,对天子宠信宦官,而不信任地方守臣,倒还是有些愤愤不平,连声抱怨。

韩冈哈哈笑道:“就当他是走马承受好了……日后改为经略安抚司,也仍是会有阉宦来此,免不了的事。”

王厚回以一声长叹,苦笑着,终究对此也是没有办法。

打马经过泾原援军的营地门前,众军的呼喝声震内外,营中的那一个指挥的选锋依然是操演未休。

王厚朝里面呶呶嘴:“姚武之来了,玉昆你知道不知道?”

韩冈失笑:“泾原选锋的驻地还是我安排的,你说我知道不知道。”

王厚也笑了,自己是糊涂。韩冈是安抚司机宜,王韶、高遵裕的助手,这些琐碎的细务本该是他来处理。他回头望望被抛在身后的大门,姚兕现在多半已经在营中。“以玉昆你看来,姚大比之种五如何?”他向韩冈问道。

“姚兕和种谔?!”

韩冈微带惊诧的扭头,只见王厚点着头,“即见过姚武之,又与种子正熟悉的,这里就玉昆你一个啊……不问你问谁?”

“……过去或许并称,但现在两人已经没法比了。”韩冈皱着眉,斟酌着词句,“用兵上,种子正早已是放眼全局,其攻取绥德,进筑罗兀之举,都是为了夺取横山,进而攻灭西夏。而姚武之只是安心做他的都监,从来都是听命行事,从没有听说他有任何进取之举。向种谔当年不待上命,就出马夺下绥德,姚武之做不出来。”

“种谔可是奉了密旨!”王厚立刻指出了韩冈的错误,“而且还是高公绰居中传递的。”

韩冈冷哼一声:“不是枢密院的命令!”

王厚为之结舌——韩冈说得并没有错。

边将出兵攻打敌城,要么有枢密使的签书,要么是经略使的命令,否则便是擅兴兵事。即便有天子的密旨,但在缺少枢密院副署的情况下,也是不合法的。随便哪个文官,只要胆气高一点,就能丢到一边去。

所以当年种谔在夺下绥德之后,便差点被枢密院以生事之罪而诛杀,而他夺下的绥德城也要还给西夏。要不是郭逵看在绥德城的份上为其背书,天子也保不下他来。可种谔终究还是被治罪,居中传递消息的高遵裕,也连带着收了责罚。种谔因此事蹉跎了两年之久,直到韩绛宣抚陕西才把他从编管之地给捞出来。而接下来,便是他在韩绛的支持下,主持进筑罗兀、攻取横山的战略。

相比起种谔,姚兕可就差多了。从过去的经历看,姚兕当是一名合格的将领,可其作为帅臣的本事,还没有展露过一次。

这就是差距。

王厚沉默了下去,得得的马蹄声一路响着。过了一阵,他忽然又道:“想不到玉昆你对种子正的评价这么高。”

“高是高一些,但小弟可不希望种五来通远。来的姚大能听命,来的若是种五,即便不论现在的身份,他的那个性子,谁能压得下他去?”

“呵呵……”王厚莞尔一笑,“说得也是!就算带了选锋过来,姚兕怕还是比不上种谔一个人。”

王厚的话让韩冈忽然之间灵光一闪,莫名其妙的想到了什么,“说起来,通远并不缺良将精兵,也该编一个选锋指挥出来了。安抚手上有一队能信用的精锐,临阵时也方便许多。”

王厚正经起来:“玉昆……你跟家严说过没有?”

“刚刚才想到的,不知处道兄意下如何?”

“此事当可为!”王厚断然说道。

韩冈的一现灵光,便让两人快马挥鞭,一下便回到了衙门中。

正厅中,依然是王韶一人坐着,批阅着文书——高遵裕如今入京诣阙,人在东京——几个胥吏环伺在旁,一名低阶的文官在其面前,恭声禀报着公事。

“回来了?”听见动静,王韶抬起头,挥手让几个官吏退到一旁,问道,“酒厂那里出了何事?”

韩冈先瞥了几名官吏一眼,几人立刻识趣的告退。

等到厅中只剩三人,韩冈才苦笑着几句话把事情解释了。

王韶皱起眉来,难怪韩冈不想当着外人说。傅勍、王舜臣他们偷鸡摸狗的事未免也太丢人,一个个都是起居有体、亲卫环绕的官人了,怎么还做这等鸡鸣狗盗的事。可为几十斤酒,也不方便责罚他们。他正要说些什么,忍耐不住的王厚站了出来,把方才韩冈的提议向父亲说了。

王厚最后沉声说着,“通远军别的不多,就是精兵强将多。就算不在军籍中的保甲中人,拉出来也都是能上阵的精锐。挑选起选锋来,比起其他几路,只会嫌挑选的余地太大,不怕会挑不出人!”

王厚期待的眼神看着父亲,可王韶却是摇了摇头。

“大人!选锋一军,诸路皆备。可见上阵时实有大用。为何不同意?!”

“不是不同意。”王韶安然的笑着,“你们不说,我也是准备要做的。只是领军的人选难定,高公绰不在,这时候我不与他商量下令挑选选锋,保不准他心中会有芥蒂。”

王厚欲言又止,而韩冈在旁劝道,“高安抚已经走了一个半月,算时间,该是和王中正一起回来。权且稍等一等,也没几天了。”

安抚下王厚,韩冈又转过来,“安抚,高安抚不在,挑选将校主持选锋的确不便,不过下面的士卒挑选一下应该没问题。士卒先定下来,等高安抚回来就决定领军的人选。这样也好让本司选锋赶上出兵的时间。”

王韶略作思忖,点头首肯:“也好……这事我会交给苗授去做,明天我会知会他的,你们就不要管了。”

韩冈从正厅中告辞出来,王厚则被留在了里面。

姚兕新近抵达通远,按道理该为他举行接风宴。可接下来的十几天,援军将会一支接着一支的抵达,要是来了一家,就办一次接风宴,王韶口袋里的几千贯公使钱转眼就会给翻得底朝天。所以是先办一下简单的家宴,等到全军集齐,誓师出兵之前,才会把众将聚在一处,将接风洗尘的事一起办了——既然是家宴,当然交给了王厚去措办,韩冈也就没必要插手。

走在韩冈尤在想着王韶的决定.看起来王韶对高遵裕还是很是尊重,怕他心中暗生芥蒂,连选锋士卒的挑选都是交给高遵裕一派的苗授。

不过王韶这样做得也对,换作是自己也是会如此去做。

迎面走来的几个胥吏,看到韩冈过来,连忙退到一边行礼。韩冈心不在焉的冲他们点点头,仍在心中暗赞王韶的老于世故:

现在把选锋军卒的挑选之权交给苗授,等着高遵裕他回来,就不得不投桃报李,不去跟王韶争夺率领选锋的将校的人选归属。这等轻描淡写就把主动权掌握在手中的手段,还是在韩冈提议后的一转眼间就冒了出来,现在想想,还真是让人佩服。

回到自己的公厅,几个属吏连忙迎上来,服侍他坐下。韩冈端着他们奉上来的热茶,随手翻着摆在案头上的公文,都没什么大事。有关出兵的一应事宜,全都已经筹划好,不会临阵慌了手脚。而且现在才来了姚兕一家,更不用担心会突然出些个乱子,让人措手不及。

身无余事,韩冈一口口的啜着雪白的茶汤,在缓缓升腾起的水汽中,想着即将到来的战事。

说起来,今次出兵规模的确不小。通远军原有的五千兵马,去除留守的驻军,仍要出动三千以上,加上两路派来的六千左右的援军,总数接近一万——都是上阵厮杀的队伍,而不是,寻常连民伕一起算进来的号称人数。

如此军力,要击破武胜军的吐蕃人应当不难。但就跟罗兀城一样,要长久的稳守住临洮,却是很有些麻烦。要想保住临洮,控制住洮水流域。在武胜军少说也要驻守上四五千士兵,同时还要在几处关键的战略地点安置下城寨。这就需要征发大量的民伕来运送粮草、修筑城防。可屯田之事事关通远军日后的发展,也不能就此耽搁,在今年冬天还要组织开辟渠道,人力不能随意抽调。

人力、粮食,两桩事困扰着通远军的发展,相对而言,反倒是战争就显得不是那么麻烦了。

手扶着温热的茶盏,他暗自叹着:知易行难,要把一件事做好,当真不容易。

