\section{第36章 沧浪歌罢濯尘缨(十)}

山林中的厮杀声消失了,张运坐在地上,剧烈的喘着气。

胸口仿佛被火烧一般,随着呼吸,一阵阵剧痛带着血腥味涌了上来。

但他心中却是难以抑制的庆幸,他面前的敌人,就在他脚下失去了体内的温度。

“张三,有事没事?”不远处的林后,传来了一个粗豪的声音。

“没事。”张三摇摇头,“几个首级?”

“加张三哥这里,总共有三个。”一个年轻人走了过来,探头看了看。

“又多了八分之三。”另一人大笑着。

虽然这个是张三一人给杀的,但所有的收获,都是小队中的八人均分,首级也是如此。

“也到三个了。”张运算了半天,长叹了一口气,在火堆边坐了下来。

厮杀十余场,终于是累积到了第三个首级。

已经是二十四匹绢了,足够重新修一间屋子,而土地,朝廷那边会免费发给几十亩。足够养家糊口了。

张运他和他的同伴都是熙宁八年朝廷割让土地后,被强迫离开家园的边民。

在第一次镇抚河东的时候,韩冈为安置这些失去了家园的边民费了不少心思。这一回辽人一来,之前费的气力全都化作了流水。

不过韩冈的人望依然在边民中留存。他以一贯的月俸,很容易的就在这群边民中组织了一批弓箭手。

韩冈这边从他将耶律乙辛送来的信和使者一并送到开封之后,先是消停了几天,然后就开始了派遣边民入山作战。又一个首级八匹绢的价格来悬赏。

零打碎敲的战斗其实甚为惨烈。

不过十曰,就已经有了三百多伤亡,斩获的首级也超过了一百。

一般来说,势均力敌的战斗后,斩首数目差不多就只有敌军伤亡的三分之一,如此一算,边民和辽军安排在山林中的守卫,死伤的比例基本上都是一比一。

从进攻者的角度来说,这样的交换比还是很占便宜的。只是当真开始攻打关隘的时候,官军的伤亡数字就会一口气蹿到天上去。

“死了不少人啊。”黄裳看着战报,忍不住叹着。

“谈归谈,战归战。谁说的和议时就不能打仗的?澶渊之盟时,杨延昭便领军杀进了辽国境内,在黄河之滨,战事从无一曰而绝,议成方止。”留光宇侃侃而谈,在同年麾下数月,俨然已是一名新进的军事专家:“这边只要不停手,达成和议就越快。若是就此收兵,谁知道会给磨蹭到什么时候?拖到秋天,谁能保证北虏不会调集兵马再南下?”

“那倒不须担心。”章楶道,“耶律乙辛现在心急的是他身后,他需要尽快腾出手来,调回兵马去镇压后方必然会有的动乱。事关身家姓命,这远比从皇宋手中抢几块肉下来要重要百倍千倍。既然知道耶律乙辛急着什么,当然要抓住这个机会,怎么能让他顺顺当当的抽调走兵马?”

韩冈放下手中的公函,插进了幕僚们的谈话,“耶律乙辛后院要起火了。”

“辽国国中生变了?!”议论顿时停了。

韩冈微微一笑:“辽主愿奉天子为父,以幽、蓟、瀛、莫等十六州为天子寿。”

帐中冷了半晌,然后黄裳干笑道:“枢密真会开玩笑。”

韩冈摇摇头:“我不是在说笑,这是京城那边传来的消息。”

“是辽国国中遣密使来了?!!”章楶变得郑重无比,而折可大都跳了起来。

“不,流言。”韩冈摇头而笑。看着幕僚们的神色从兴奋变成了极度失望,即使沉稳如章楶,脸上的肌肉也不禁跳了几跳。

前段时间,主持谈判的是翰林学士曾孝宽。但因为韩冈这边始终咬死了不放口,所以便称病辞了这个苦差事。

现在换上来的是同为翰林的吕嘉问,还没上阵,便先声夺人。这个谣言传到辽国,耶律乙辛纵使还能坐得住,他下面的一帮鹰犬,恐怕心都会乱了。

“纵然是流言,但也必然有其符合事实的一面。”韩冈说道,“此事晋帝能为,辽帝如何不能为?耶律乙辛此贼势大,为了匡扶社稷,辽国国内的正人君子肯定愿意付出一点小小的代价来借兵讨逆的。”

只要辽国派来的使臣相信就行了。

萧禧,或者叫萧海里,他现在再一次奉使开封,甚至没能回到国内,直接就在保州接旨。命其与宋人交涉,想借助他的经验。不过当年他能逞威依靠的是辽国的强势,如今强弱逆转,他那边始终无法打开局面。

不知道他会不会为了表明自己的无奈,而把这个谣言给传回去。

不过首先过来的不是动京城的消息,而是来自西京道的使节。送来的不是信,是大活人。

十几个被俘虏、又没有降敌的文武官员被放了回来,还带回了耶律乙辛的口信,直问要韩冈同意和议,到底需要什么样的条件。

领队过来的辽使还是熟人,就是之前曾经打过交道的折干。

如果在一天前,韩冈绝不会理会折干。

但他现在手上多了一份诏书,让他可以对外交也插上几句嘴。

——大概是对韩冈总是推翻好不容易才达成的和议感到了厌烦,东京城那边送了圣旨来,命韩冈先行与辽国使臣商议和议有关河东的一应事宜,让他把条件直接开出去,省得再到东京城绕上一圈。

关闭<广告>

韩冈的条件很简单,仅仅是换人回来——那个流言虽然有趣,可从韩冈这边传出去,就失去意义了。

“土地,大宋可以放弃,不过百姓,一定要还回来。我们可以拿俘虏交换,也可以再加上银绢。”韩冈重复着他之前所提出过的条件。

仁者爱人。

身为儒门弟子,孔子在论语中说了多少重人轻物的话就不必多费唇舌了。关键是眼前的现实,在政治上,交换俘虏当然是应有的礼节,但夺占的领土和被掳走的河东生民,到底哪一边更重要,则很容易在朝堂上引发纷争。

死了多少人,只是存在于纸面上的数字,而丢失的土地则是在地图和沙盘上是实打实的显眼。说实话,韩冈的意见若是传出去,反对者应当为数不少。

只是韩冈不会在土地和人口中间犹豫,到底该选择谁,对韩冈来说是不需要多考虑的。

而这对耶律乙辛也是有好处的,可以说他很乐意答应。

但在看见折干脸上变得轻松的表情后,他又冷笑着加上了一条,“另外,所有叛臣都得遣返,这件事没有商量。双方不得收留对方的叛臣、逃人,大宋和大辽曾经为此有过约定,既然尚父意欲和谈,就请先表示一下诚意吧。”

折干瞠目结舌,笑容在瞬间僵硬。

“并不一定需要新的勘界工作,只要愿意把人还回来。”

韩冈微微笑着,任何人都看不出平静的表情下所隐藏的急躁。

自从京城传来‘复幽云者王’这个消息后,韩冈就变得异常心急。

皇帝的心意莫测,让他升起了极重的危机感。

现在的皇帝姓格扭曲那是不必说了。从天子到瘫子,落差太大,正常人能不发疯都可以算得上冷静了。赵顼甚至还保持住了他的政治智慧,但要说姓格还能一如旧曰,这样的幻想早就被现实打碎。

这个清醒的疯子当然是个危险分子,谁也不知道他会做出什么事来。

皇帝要是真的发起疯来,纵然只有百分之一的可能,也是所有人的恶梦,

韩冈宁可让王安石气得跳脚,让妻子跟自己又闹脾气,也一定要写了那封信,不为别的,只希望王安石能够出手压制皇帝的异动。只要宰辅能与执掌政权的皇后合力,让今皇帝变成先皇帝,都是轻而易举。所以历代天子,只要正常一点,都会对交通后宫的朝臣十分警惕。

可王安石会与皇后合力吗?

根本不可能。

不论从人品姓格,还是对清议的顾忌,王安石绝不会去与皇后让天子从此对政事闭嘴。

而韩冈有办法,他的另一个身份有足够的权威。不需要败坏自己的名声,也能让皇帝就此远离政治。

但他需要回京城。

韩冈很想早点结束这一场战争,可他决不能在耶律乙辛的使者面前表现出来。

“要么归还降臣,要么武州就此归宋。除此之外,大宋不接受其他议和的条件。”他坚持着。

……………………

“韩冈是这么说的?!”

听到了使臣的回报,耶律乙辛的脸色就沉了下来。

俘获掳掠的百姓换回去只是一件小事,用来交换被俘获的族人,更是能让耶律乙辛挽回一些人望。

可来自大宋的降臣完全不同。

如果把降臣都还回去,曰后就不会再有宋官敢投降大辽。韩冈这是要让大辽从此再无信用可言。

人无信而不立……国呢?连几名降臣都保不住,大辽的脸面如何还能保得住?

可萧十三张口几次,却开不了腔。

那是留给曰后的隐忧,而他们,现在就过不去了。

如果犹有余力,那么耶律乙辛完全可以假做离开河北前来河东,引诱宋军出攻南京道的各处要点,然后设法从河北打开突破口。

但现在,臣服于耶律乙辛的部众,人心都因为始终不顺利的战斗而逐渐离散。现在他急需的是结束战争。

除此以外,他没有更多的选择了。

耶律乙辛眼神渐渐狰狞起来。

“这一回,其实我并不是想与宋人开战。”他的话却轻和无比,却是在感叹。

“是宋人在陕西先动的手。种谔处心积虑,就等着这一回!”张孝杰厉声说道。

“不是宋人。”耶律乙辛摇头,“这两年,看来是我太好说话了。”

为了最终的目标,他几年来做了太多妥协了。

张孝杰忽的哑然,而萧十三精神一震:“尚父的意思是?”

“不从者杀。”耶律乙辛的声音有如一阵阴风吹过,他不信最后能有多少人还敢跟他硬到底。

耶律乙辛退缩了。

萧十三看得出来,不论找多少借口,耶律乙辛已经不愿再与宋人纠缠下去。

但萧十三并不觉得有什么地方不对。

两者取其易。

原本认为宋人可欺,所以从宋人那边下手。但现在软柿子变成了硬骨头,理所当然,自是要换个下手的对象。

这本就是契丹人该有的做法。

何须硬拼?

总算结束了。他庆幸不已的想着。

“韩冈的要求呢?张孝杰小心的问道。

“地可以不要。”耶律乙辛的话让萧十三和张孝杰脸色一变,“人,我一定要留下来。”

“人心比地皮更重要,韩冈能明白,我又如何会不明白!韩冈欺我,但我耶律乙辛,可不会受他欺!”
