\section{第34章 为慕升平拟休兵(六)}

甲骑,马铠也。具装,人铠也。

人马皆贯甲。

具装甲骑,或是说甲骑具装,都是指的同一个兵种,那是战场上用来碾压敌军的重骑兵。

但宋辽交锋的百多年来,只有辽国将领身边的亲卫才会人马皆装备上甲胄,并不是用来决战的读力兵种。

辽军从来都不会面硬撼宋军的阵列,而是设法绕过去,然后抄掠后方。对于这样的战术,苦于战马不足的宋军自然是头疼不已。

可若是换成正面相抗,辽军虽不能说必败,但赢了也是笔折本的买卖。组成辽军主力的部族军,他们的头领,可是一个比一个会算计。而属于耶律乙辛一派的萧十三和张孝杰,现在也应该不敢随意牺牲手中用来震慑四方的嫡系部队。

“会不会看错了?”韩中信虽然没真正带过兵,但在韩冈身边学到的听到的不会输给同年龄的将门子弟,辽军的特点,他多多少少也了解一点,“辽贼不是都给战马披毡的吗?何况哪有出城时就全副披挂上的道理。人马带甲上千斤分量能跑上几十里?”

“主持是小人亲叔,巡边时挑了辽贼四个军铺的褚十四!”传递敌情的信使一下涨红了脸,好似受到了莫大的羞辱,厉声叫道:“传来的口信上说了,出城的辽贼一人三马,出城的时候的确都没穿戴,但战马背上驮的铠甲绝不可能会看错!人穿的铁甲和马铠只有瞎子才会分不清!”

“你是褚十四的侄儿?”秦琬打量了这名气得直喘气的信使两眼,也隐隐觉得眼熟。转头对韩中信道:“褚十四曾在先父帐下听命,后来调去了代州西路巡检那里。是代州军中老人,戎马三十年,眼力比我这样的后生晚辈强上不少。这一回代州军多投敌,但他没有,而是带人上了山。所以才会调他去做探马。褚十四手下的兵也都是老兵,都是见多识广,当不至于会误报军情。”

韩中信方才给顶撞了一下,脸色很不好看。俗话说宰相门前七品官,他作为韩冈心腹,从七品没有,正八品总是有的。寻常就是知县知州来登门求见,见了他都是和声和气,有几个敢给他脸子看?何况他现在是官——且是流内官——而面前的这位信使只是个卒子罢了。

幸而韩中信也明白,如今要在代州军中打滚,韩冈的势可以仗,但不能以此欺人,否则不会有好结果。韩冈耳提面命多次了,他再不长记姓也不会忘掉。何况褚十四的名号他也听过。

“深入北境、横扫辽贼军铺的褚十四,这个名号我在枢密身边都听过。”韩中信不怒反笑,对秦琬道,“前些曰子在忻州城外山里与辽贼过不去的时候,也是常听人说起过,没有不挑大拇指的。能在代州城左近盯着辽贼的动静,果然也只有他了。要是知道这是打探来的,”

韩中信这么一说,那信使一下就没了火气,转向韩中信行礼,口称有罪。韩中信自是很大方的一笑了之。

“真不知辽贼在打什么鬼注意。”秦琬低声的念叨着。

可不管辽贼是什么打算,韩中信都觉得没必要想太多。具装甲骑也许对官军的箭阵有着很强的压制能力,但最大的限制就是战马的体力。而且其用武之地,只可能是野战,遇上城墙——就算仅仅是村寨的围墙——就会碰得头破血流。

目送那名信使上马扬鞭,继续他的工作,韩中信转回来对秦琬道:“看起来得尽快赶到土墱寨了。”

秦琬紧皱着眉,没有搭腔。方才听说辽军出动的消息,他的眉心就被挤出几条深沟来。

“难道辽贼是冲着我们来的?”韩中信看着秦琬苦恼的神色,灵光一闪,“是不是在担心远探拦子马?”

“辽贼不一定是冲我们来,但远探拦子马却可以将他们引来。”秦琬似乎是打算让韩中信分摊他的苦恼,坦言说道,“大敌当前,不能不小心一点啊。”

既然作为一军核心的具装甲骑都出动了,那么护翼他们的轻骑兵肯定早就开始巡视周围。而且以辽人拦子马的活动范围,现在有前锋进抵土墱寨都有可能。再想想那位打探到辽军出动的褚十四,多半是撞了大运了。带着消息回返,竟然没被辽军的拦子马堵在半路上。

“去土墱寨还有十五里的路,急行军得要一个时辰,但肯定是应该赶得及。辽贼离得还远,少说也还有一天的时间。有这个一天的时间,我们就能将土墱寨给整备完毕。”

秦琬依旧沉默着,这个决定不好下。

万一出城的辽军是冲着这边来的,大军行动的速度或许快不了,但远探拦子马则不会比这赶回来报信的铺递慢上多少。且不说一旦给他们察觉到这边的行军,不论是因为什么事出战,必然会把辽军主力给引过来。就是这些探马,本身的实力就不会弱。即便能抢前一步进入土墱寨,可一群累得半死的士兵如何守得住城寨?只凭现在城寨中留作哨探的那几十人吗?

“难道还能退回道口镇不成?”韩中信极力鼓动秦琬,“驻守土墱寨,为大军前哨。岂能辜负了枢密的重托”

‘输了就更辜负重托。’秦琬想着,但他最终还是决定接受韩中信的意见,毕竟还有许多探马正巡游在忻口寨到代州城的这一路上,小股的人马倒罢了,可大队的辽军探马就别想瞒过他们的眼睛。何况他这一部兵马,本来就有为数不少的骑兵来护持两翼。

……………………当秦琬和韩中信正考虑着是退,还是继续向前走的时候,忻口寨的韩冈和他们幕僚们也收到了这封紧急军情。

“这两年辽贼倒是变得财大气粗起来了,这具装甲骑说装备就装备了两三千人。耶律乙辛也好意思来哭穷。”折可大口没遮拦。

“不过拥有铁甲是一回事,用出来则是另一回事。”章楶道,“很难想象辽军会把希望寄托在这两三千具装甲骑身上……”

“不然能寄托在谁身上?宫分军和皮室军?他们数量太少,萧十三也不可能舍得。部族军中的骑兵?早就推得落花流水。步卒?那是给我们送功劳的。数来数去,想要扭转败局,也只剩这一队具装甲骑了。”

韩冈笑说着,让在列的人们心情都轻松了起来。

“这可算是孤注一掷了吧?”

折可大问的问话,却让他们的神色又重新绷紧。孤注一掷虽然是摆明了陷入困境,但这一次的反扑必然是凶悍无比,有若狂潮。

“自然可以算是。”韩冈点点头。虽然其中肯定还有些算计,但辽军给逼入了困境可是确凿无疑的“既然如此,就必须小心应对了。不知枢密打算怎么做?”章楶有些好奇韩冈的想法。

“此地大军三万余,难道是来此游玩赏歇的?”

“枢密何不让西军上?这一回萧十三看起来调动了不小的兵力,远比秦含之手上的那队人马要多。若是辽贼去攻秦含之,只有西军能赶得及。”折可大奇怪的问着。西军的实力时所公认,而且还欠着韩冈不少人情。在情在理,都该用他们的。

“自我至河东以来,参战各部几乎都没有损失,而且其中很多人根本就没打过仗。不让他们历练一下,还要等到什么时候?”

“可不一定能撑得住。”

“不过辽贼有多强,他们必须撑过去,朝廷的俸禄总不是白白吃的!”

韩冈解释着自己的决定,让来自京营的士兵与辽军一较高下,是他的计划。至于能不能撑得过去,那是另外一桩事了。不过他相信在自己的控制下,损失不会大——他也没打算逼着京营禁军与敌硬拼。且必要时,还是会动用西军来保驾护航。

“辽贼若来攻。我们就守好了。拖到夏天就赢定了!”黄裳见厅内气氛紧绷起来,出来缓和,“留给辽贼的时间并不多了。他们很难适应代州的夏天。而且河北那边的情况只会更糟。最后的结果只会是辽贼不战自退!”

韩冈到了河东究竟做了些什么。他这个首席幕僚最清楚不过。

不仅仅是军事,韩冈在河东组织生产,恢复民力,民心渐复。拖到夏天,辽军的战斗力会越来越低,战斗意志也会被消磨殆尽。那时候可就是决战的最佳时机。

黄裳的话一下点醒了众人,气氛又变得轻松起来。

可众人的欣喜中,韩冈泼起了冷水,“勉仲,不要忘了,今年是从陕西打到河北,几十万人上战场,上百万百姓受了兵灾。拖到六七月,国家的财计还能不能支撑得下来?”

“……可东京城有的是粮草。”

“东京城不可能将所有的粮食草料都运到河东来。河北那边需要的只会更多。”

“但河北并没有向河东这样受到辽贼的洗劫。更不用说百姓流离失所。只有几千人窜进了境内,还很快就被歼灭了。”

“的确如此。但这并不代表郭仲通不会要钱要粮,不会要军资要兵械。”

这是前方将领推卸责任的惯用手段。万一战事不谐,多少还有个顶罪的理由。

如果不能在夏天前结束,就是韩冈也不可能将和谈拖延到夏末。要是政事堂拿着帐册来给自己看,这仗他也没办法再坚持打下去。

厅中沉寂下来。

韩冈却并不在意,他要打掉幕僚们的侥幸心理,方如此尖锐而针锋相对的说话。

他想要好好的跟辽军打上一仗。并不是他穷兵黩武,好大喜功,只是要拿回代州,不付出代价是不可能的。而且那些在河东犯下无数罪行的罪人,韩冈绝不打算为了一纸和约,便就此放过。

吃了我的,要还回来。拿了我的,要还回来。伤了我的,抢了我的,也都要还回来!
