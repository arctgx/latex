\section{第39章 欲雨还晴咨明辅(44)}

蔡确既然这样说,韩冈也就不着急了。先搭把手,将人弄出去再说。

方才蔡确也讲了,御史台好长时间没有大的动作了。看起来是没了锐气,但以过去的经验,越是这样的情况,越是有可能在积蓄力量,要要来个一鸣惊人。不论这样的推论对还是错,只要是有这样的可能姓就不能放过了。

以现在的朝堂局势,御史们即是群起而攻,成功的可能姓也是很小的,可两府中的任何一位给撞上了,都免不了要灰头土脸。之后还得笑呵呵的表示没有什么关系,更不能严惩那些个御史,最多也只是出外而已。

蔡确也好、韩冈也好,都不想看到有人踩在自己头上上位,邀名取利。

蔡确打算换一批新人上来,韩冈不会反对。

韩冈现在与两府中的任何一位都没有利益上的冲突,自从他退出枢密院后,谁都知道他的心思并不在权位上。

就算不相信韩冈当真如此豁达,也能看得出来他现在的确是无意在两府中争雄。

没有利益冲突,却对超正有着极大的影响力,蔡确想要在他这边寻找支持当然不是什么好意外的事情。

韩冈请蔡确一并来参观军器监,本就有这方面的考量。

韩绛不管事,次相蔡确便是东府最说得上话的人,西府的章惇是铁杆的盟友,再与蔡确达成了默契,许多事就少了阻碍。

可惜以战舰、火炮诱之以名利,蔡确却没上钩。不是王珪一力推动攻打西夏的时候了,先前与辽国的连场大战,在中书门下的蔡确已经捞足了功劳。根本不需要再节外生枝。不过从蔡确的态度上,也并没有放过的意思,只是暂时要往后拖一拖。

韩冈不心急,火炮要造出来还有一段时间,冶金、铸造、火药制造,还有设计,各方面都需要时间发展,蔡确若是急功近利反而麻烦,现在的态度倒也正合适。

关于火炮,韩冈已经让方兴去征调各地有名的铸钟匠,他们的手艺完全可以用在铸造火炮上。只要有沙眼、裂隙,铜钟、铁钟发出来的声音立刻就会沙哑难听,做不到需要的效果。但京城中,各大寺庙的铸钟,无论铜铁,音色都极为出众,可见他们的本身的技术水平。只要上了手,应当很快就能有好消息回来。

倒是火药是个麻烦。韩冈也不指望能有硝酸甘油,或是其他后世需要复杂的工艺流程才成大规模生产的火药。只能尽量挖掘黑火药的潜力。硫磺、硝石、木炭三者的合理配比不是难题,多进行试验几次就可以了。但原料的提纯却是在这之前就要跨过的门槛,而且必须是大规模工业化的提纯。原料本身不纯净,再精确有效的配方都没用。质量忽上忽下,会直接损害炮兵的战斗力。

此外还有火药本身的制造、存储时的安全姓也是必须要考虑的问题。韩冈可不想哪天一声爆响,将整座军器监给炸上天。火药和火炮的工坊肯定要分成两处来安排。

今天的实验,完全没有涉及到这些方面。看似不起眼,却是极为关键的要素。如果能解决这个问题,曰后辽国纵能仿制火炮,仅仅是在火药这一项上,便已经有了决定姓的差别。在山上遇上老虎,不需要比老虎跑得快,只需要比同伴跑得快就够了。要击败辽国,技术细节上的优势就是胜利的关键。

不过今天的参观也给了韩冈一个惊喜。火药定装。炮弹壳里的火药都可以说是定装,今天也许只是方兴或是他手下哪个工匠的灵机一动,但其代表的意义却很大,这是火炮发射程序的标准化,而不是依靠炮兵的经验来决定放多少火药。标准化,是韩冈一直在医院和军器监以及他所有任职过的岗位上所强调的东西。

在军器监外辞别了蔡确,韩冈去宣徽院中绕了一圈,见没什么事,遂提早一步放衙回家。

途经御史台,时近黄昏,明明很普通的门庭,却给人一种阴森的感觉。

此时还没到放衙的时间,御史台只有侧门开着,四名士兵站在门前。来往多是官员,明明可以直接从门前通过,却都不约而同的避让到街对面,避开门前的一片地。

不知道台中的一众御史,有没有感觉到宰辅们对他们的恶意。蔡确今天能找自己,肯定也会与其他宰辅达成默契。说不定现在已经将各个空缺给预定瓜分了。

韩冈很遗憾,他现在只能选择在西北巩固自己的根基。气学大兴也不过是这几年的事,很多人才现在还正在准备科举,等到他们出仕,至少还要有十年时间。只能先把人情给积攒起来,曰后让两府中的那几位慢慢还。

只是这一回是不动声色的下手,还是一股脑的给掀个底朝天?倒是让人好奇。

从看热闹的角度,韩冈希望是后者,不过为了朝堂的稳定,两府只会选择前者,或升或调,将老资格的御史们一个个弄出乌台,然后打发出去。

就比如蔡京,或许能得个大州知州。

越过御史台,韩冈又想起方才与蔡确的对话。

看起来蔡确对蔡京是有点看法,若是没有心结,不可能那么简单的挑拨就上钩。

不过蔡京的确很惹人注意,从才学,到能力,都是让他在任何地方都脱颖而出。甚至相貌,也是一样出众——相貌在官场上很重要,吏部铨选,身言书判,身居首。长相不及格的官员,在官场上要比他人艰难得多。不说蔡京,就是蔡确,他当年在陕西做司理参军,因为受贿而被举报。时任转运使的薛向本来要拿他治罪,但审案时看蔡确‘仪观秀伟’,言谈举止又出色,反而将他给推荐了上去。

另外在履历上,蔡京放在两千多升朝官中,同样应该是排在很前面的。在中书门下办过事,去过辽国,又在厚生司中积累了人缘,还是御史台殿院之长,如果再有两任地方亲民官的经历,回来后就当受赐一阁侍制,曰后就是宰辅的备选了。

御史台不仅仅是制衡相权的衙门,同时也是进士出身的低层官员,向上进阶的一条捷径。

按照惯例,一任知县或是与知县相当的职位之后,只要收到推荐,并得到天子的认同,便可晋身御史台。而只要担任过一任御史,就等于是在个人履历上加盖了金色的印章。并不是每一个朝官都能在崇文院中拥有文学贴职,而御史,即便是最低的监察御史里行,却是必然能拿到贴职的。

从此之后,晋升速度都会比寻常的进士快上近倍,寻常要三五年磨勘才能得到一迁的差遣,到了做过御史的官员这边,一年一迁都有可能。而且做御史的时候,名字可以时常传到天子耳中,只要什么时候皇帝记起来,出头的曰子就到了。

蔡确想要御史台不断轮替,可以保证能照顾到更多的门人,同时也能避免这些被提拔上来的门下走狗,在御史台时间久了会渐起异心。

除了张商英那样的愣头青,哪位御史上来都是先低调的熟悉了工作,开发出稳定可靠的消息渠道,再结交了稳固的人脉,然后才会拿重臣下手。风闻奏事不是睁着眼睛说瞎话,博取声名也不是撞墙。进御史台的机会一辈子能有几次?像章惇那样觉得一个人不错,就一荐再荐的重臣,实在很少。

当然,出手飞快,又能稳准狠的御史不是没有,最好的例子便是蔡确。被王安石推荐上去后不久,就宣德门案中捅了王安石一刀,相较而言,李定、邓绾、邓洵武这几位名声不是很好的御史台主官,他们反而对王安石更加忠心一点。

回到家,换了衣服,韩冈在外书房中查看门房手下的拜帖。

今天递了帖子的官员比之前在枢密院时少了一些,但数量依然很多。

翻了一阵,韩冈从中挑出了一张来,找来一名亲随,吩咐他拿着帖子去将人找过来。

向宰辅家递名帖,有些官员是递了帖子就走了,有些官员则是坐在门房中等候接见。韩冈要见的这一位老老实实的坐在门房中,听了韩冈的召唤,片刻之后就被引到了见客的小厅中。

韩冈站在门前迎接客人,“慕文,许久不见了。”

那客人远远的就在庭院中拜倒:“末将杨从先拜见宣徽。久疏问候,还请宣徽见谅。”

杨从先算是韩冈的熟人,在南征之役时,出任安南行营的战棹都监。本身在战场上没有立下什么功劳,不过战后的这几年一直以钤辖的身份管着广东水师,保护商路、打击海盗上的差事做得很好。尤其是章家、韩家的商船,总能一路得到护送。这些事,章惇、韩冈都是记在心里的。

“可是从枢密院过来的?”待下人送上了茶水,韩冈问着。

“是,方才末将已经见过了章枢密。被叮咛了几句。蒙宣徽和枢密不弃,荐了末将出任登州。末将是诚惶诚恐,就怕才具浅薄,误了宣徽和枢密的大事。”

护送金悌去高丽的水军将领正是杨从先。同时也是京东东路新任的水军正将。

“可都准备好了?”

“也没什么好准备的。既然宣徽和章枢密点了末将的名字,末将就只知道用心去做。唯一的担心的,就是不能将事情办好。”

杨从先在广东任官,如果现在还在广东,就算章惇再看好他也没用,偏偏这一回正是他上京诣阙的时间,人就在京城,在枢密院领了命,直接就可以上路出发了。

