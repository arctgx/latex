\section{第三章 墙成垣隳猿得意(上)}

“袁十二平常都闷不吭声的一个人,怎么就敢在京城里面放火呢?”

“烧了又能怎么样?一口气出得是痛快,但那轨道才值多少钱,听俞六丈说,修起来只花了一百五十贯而已,九牛一毛啊……今天烧了,明天就能给造起来,转眼就能用上。没看这几天,李木匠都不在吗?给请去打造轨道了,现在汴河、五丈河上,多少家商行都准备要用上轨道。但袁十二他们呢?他们又是什么样的下场!”

“听说韩舍人在天子面前帮他们求了情,除了袁十二肯定要刺配以外,其他都是在开封府里杖责后就放了。可日后谁还敢雇他们做事?”

“说得没错!要是雇了人之后就不能开逐出去,以后谁还敢招人做工?”

“也不能一下就将十几人一起赶出去,一家老小要吃饭啊!前些日子,城外水磨坊的那些厢兵去韩舍人家,也不是要讨口饭吃吗!?吴枢密说得也是有道理,东西是好东西,但坏了人家的生计,于朝廷并非好事。”

“有把子气力,哪边混不到一口饭吃!?更别提那些推磨的驴货,不做事都有俸禄领的,你陈二锤子一天苦下来的钱,人家还不带正眼看,你发个鸟善心!”

“卖苦力的活也不是多好的事。累得五痨七伤,年纪一大,别说是抗包了,就是走路也是一步三喘,什么毛病都上来了。还不如趁年轻学门手艺。为了这点事就放上把火,把自己都陷进大牢里去,何苦呢?何必呢?”

事不关己,一些话说得便是轻巧无比。几个闲人坐在小小的酒家,一边吃吃喝喝,一边议论着最近京城里的新闻,那是平常得不能再平常。

“想不到殿上说得几句话竟然都传到外面来了。”

酒店里间的包厢中,外间的议论声透过并不厚实的板壁传了进来。听在韩冈本人的耳中,便付之一笑。

“谁让玉昆你现在一直站在风尖浪口上,木秀于林啊!”王厚叹着气,用了一个听起来十分严重的形容词,“轨道和有轨马车乃世间所无,玉昆你才智天授,让人打造得出来。可区区商贾却借此从中渔利,惹得京城生乱,责任最后都要摊到你的头上去!”

“他们是听过了轨道车的原理和大概式样,自己让人打造出来,其实与军器监中正在打造的有很大差别。”韩冈对那些绸缎商人的做法倒挺欢迎,要是什么都靠来军器监偷,自己不动脑筋,那就没有技术扩散的意义了,“只凭这一点相似,根本无法惩治那些个商人。也是因为我弄出来的东西皆是构造简单,想到难,学会则不难。军器监的轨道是为了更顺利的转运运到监中的货物,迟早也是要摆出来了,届时只要看上一眼,肯定都能偷学到一个大概。早一步、晚一步,并没有什么区别。”

小小的酒家并不是七十二家正店中的任何一家,只是靠着州桥近一点,就在与御街隔了两座坊的一条巷道中。很是僻静,但往宫里去却很方便。

王厚这一次奉旨上京诣阙,昨天赶在入夜前刚刚入了京,今日只在宣德门前报了到,当即就有吩咐下来,让他午后入宫觐见天子。时间赶得紧,他与韩冈两人也只能在宣德门外将就一下了。两人桌上连酒都没有,只有几盘精致的小菜。倒是外间的几个伴当,有酒有肉,饱了口福。

“但罪责落于人手,迟早会被人用上的。”王厚摇了摇头,前日他从王韶那里听说了韩冈到底跟多少名宰执关系不睦,心脏都差点停跳了,“玉昆你得罪的人未免也太多了一点,谨言慎行才是自全之法!”

“小事而已,也不能把我怎么样。”韩冈以茶代酒的敬了王厚,笑道:“倒是处道你,怎么变得这般谨慎了?”

两年不见,王厚变得越来越像王韶,不论从相貌还是气质,都与当年韩冈王韶初见时,有了六七分相似。王厚这两年坐镇西陲,手挽大军,多多少少积累了一些功劳。在这次诣阙之后,很有可能就要调离熙河路,转任他职。

“玉昆……这可不是小事啊!你能想象神臂弓流传到契丹人或是党项人手中的情况吗?”

“泄露也无所谓,难道契丹、西夏就不会造弓弩,为什么在这两样武器上还是远比不上大宋?只有三五具的神兵利器对军队是没有意义的,只有十数万、数十万的装备起来,才能提振一国军力。神臂弓泄露也好、水力锻锤泄露也好,想一口气打造出以千万计的兵甲,契丹和党项都做不到,他们没那个实力。”

技术谁也不能保证不外流,又不是发展到后世的那种精密的仪器,偷学起来一点也不难。唯一能让大宋朝廷占据压倒性优势的,就是规模。而真正决定国战胜负的,也正是国力的较量。

无论是辽国还是西夏,都不可能在整体国力上与大宋相抗衡。同样的水力锻锤,对三国军力的加成,绝不在一个数量级上。只要能将大宋经济上的优势转化成军事上的优势,以举国之力压倒西北二虏,其实不在话下。

“而且学也不一定能学得地道。现下外面码头上,正在打造的那些轨道,我已经让人去看过了。不论是车子,还是下面的轨道都有许多区别。就像腰开弩和神臂弓的差别,本质如一,但实际上还是差了很远。就在京中,传去的轨道都已经与原案不一样了,传到了兴庆府或是析津府【今北京,辽国南京】去,说不定变得我都不认识了。”

“关键还是玉昆你现在的声望不一样了。飞船出来后,多少人都是在竖着耳朵听着军器监中的动静。愚兄过横渠镇,你那些师兄师弟,都转托我问问,你还有什么宝贝没拿出来。”王厚唉地一声感慨着,“如果雪橇车是你现在拿出来,转脸说不定就能传到契丹东京的辽阳府去了,哪里会在陇西用了两三年,京城里面还没人听过?”

“这就是权威啊!”韩冈半开玩笑半认真的说着。

别看现在朝堂上的几位宰执都看自己不顺眼,前两天被招入宫中问话,赵顼还特意让韩冈单独奏对,省得他跟冯京、吴充他们在崇政殿上吵起来。可只要在机械上的东西,韩冈说好的,就没人敢说不好。在板甲、飞船出来之前,是有人敢说铁船是无用之物,但现在谁敢这么说。

就比如水力锻锤,今天能出京城,明天就能出京畿,再过一阵,天下的铁匠铺都能用上了。不一定要仿造军器监的式样,其他结构的水力锻锤一样有人能造的出来,只要听说了造出飞船的韩舍人说水力锻锤好,那么,天下铁匠就会趋之若鹜。再比如蒸汽机、火炮,只要韩冈他摆出原理,说这两件东西有用,天子都不会怀疑。朝堂上会对他全力支持,民间也同样会涌现一大批发明家来,沿着韩冈指明的方向去研究、探索。

“又不是一言九鼎,他们只是看到钱而已。这样的权威,有不如无!”王厚将韩冈的话全都当成了玩笑。

“凡事有利必有弊,所以行事就要权衡利弊而为之。处道,有些事我还是有把握的。”韩冈说道,“好了,不说这些事了。这些日子天天跟人解释来、解释去,都没一个清闲。”

“你是自找的。”王厚一点也不客气,以他和韩冈的交情,也不需要太多避讳。要不是他下午要进宫面圣,饭就直接在自己家或是韩家吃了。两家是通家之好,请客请到家门外,就未免太过生分。

韩冈干笑了一声,转又问道:“熙河路那边的情况怎么样?”

“只是在练兵而已,这两年都没有超过千人的大战,还是蕃军动手得多,比不上缘边四路那里还经常有些动静,倒是生意做得大了。官中拿着以茶酒交换马匹,去年是一万五千匹,今年就要看秋天了,说不定能往两万匹上走。”王厚抿了抿嘴,“不过最近跟兰州走得近了,禹臧花麻也有些心动。要不是因为辽国竟然当真将公主嫁给了秉常,两家做了的亲家。这一次上京,我都想提议出兵逼着禹臧花麻将兰州让出来,到时候就能往玉门关一路攻过去了。”

“两年之内都不可能动手,北方禁军全都要换装板甲,预订的计划是三年。但原本就有铁甲的几个军是不用换的,其实两年时间就能全数完成。”韩冈顿了一下,“只要生铁的产量跟得上!”

“还要等两年?不打仗的话,军队可都会烂下去的。就像你家的事,听说是来了一百多厢军,竟然被几个残废给撂翻了。”王厚啧着嘴,很是不满的说着,“如今也就是蹴鞠联赛上还能见点血了。”

“舍人,都监,时候差不多了。”韩冈的伴当这时敲了敲门,在外面提醒道。

王厚忙应了一声,拿起筷子三口两口的将盘子里的菜吃了大半,原本行动举止有着官宦子弟一般稳重的他,已经变得跟武夫一般粗俗,边吃边说:“京里的菜,陇西的厨师连提鞋子的资格都没有。”

放下筷子,他站起身,神色郑重:“玉昆,别怪愚兄多话,吕惠卿拿着你的事来起大狱,绝不是好心!家父昨夜也跟愚兄说了,你最近的情况,他可没有帮你担待一点。”

韩冈回以一笑:“你放心,吕吉甫的心思,我还能看得出来。不管他怎么想,我还有着一招撒手锏,已经差不多是时候,再过几天可就要用出来了……”

