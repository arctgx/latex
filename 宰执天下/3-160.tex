\section{第41章 礼天祈民康(七)}

“家岳德行高致,岂是微臣所能及万一。贸然仿效,便如东施效颦,遗人笑柄。微臣所以不敢轻受诏命,非不为也,实不能也。”

冯京暗中使绊,天子心生疑窦,可越是这个时候,韩冈就越是不能改口,必须一意孤行到底。

“韩卿你也只是资望稍逊而已。论才干,当不会输给王卿刚为官之时。”赵顼的话虽是与之前的话没有怎么改变,却已经隐隐透着猜疑。

“陛下所言甚是!”冯京登时高声附和,对着赵顼持笏拱手:“韩冈之才,如今少有人及。罗兀撤军、咸阳平叛,当日安石、韩绛强要韩冈入宣抚司,可算是做对了一件事。”

赵顼脸色又沉了一分,韩冈则是冷然一笑。冯京为了毁了自己在天子中的形象,当真是卖足了气力。

这可不是在赞他韩冈以国事为重,更不是再附和天子,而是在向赵顼证明,韩冈绝不是刚硬到底的直臣,一样是个会屈服于权势之下的软骨头而已。

韩冈不可能去解释他为什么当年最后去了韩绛的麾下,因为当时他答应去的交换条件之一是周南,还有与章惇合谋的一些秘事,都是见不得光的。而摆在外面的理由,却洗不掉冯京泼过来的脏水。

但他岂会没有办法应对?

“汉高得天下,以萧何、张良、韩信为首功。萧何治政,张良建策,而韩信领兵,故而三数年间便江山一统,有了炎汉四百年天下。试问汉高若以张良治政、萧何领军、韩信建策,可否赢得以范增为臂助的楚霸王这般轻易?”

韩冈见赵顼神色稍动,抢在冯京开口之前继续道,“伯乐之所以不常有,便在于此。知人有才不难,可用人恰如其才却是千难万难。诸葛武侯为人至正,非以私亲用人,马谡于其帐下,向日岂无功绩?可武侯用之于街亭,便致使北伐功败垂成。”

说着他又一拱手,“臣虽小有才学,往日也薄有微功,却也是陛下用臣恰如其份的缘故。若将臣换个位置,恐怕不但难以建功,反而要见罪。正如今日的中书检正一职,断非臣所能胜任。”

韩冈这番话,既拿了汉初三杰做正面的例证,又拿了马谡做反面教材,就是在明着说任命他去担任中书中的职位并不合适。只用汉初三杰,未免过于自大,如果仅用马谡,那就成了自污。一正一反却是恰到好处。

赵顼皱起眉:“马谡姑且不论,但萧何、张良、韩信换个位置,未必不能成事。”

韩冈立刻回道:“若任用得当,十分才学能有十二分的功劳,若是所任不当,十分才学就只得施展个五六分。”

赵顼从孙永那里的确知道韩冈的真实想法,见到韩冈的坚持,叹了口气:“韩绛荐韩卿你判军器监?不知韩卿你意下如何?”

韩冈拱手致礼:“臣受格物致知之学于师长,于此事上多有心得。若能去军器监,当能不负陛下之望!”

绝大部分的官员都是愿意留在朝中为官,这样才能接近天子,早些升官。所以王安石屡召不起,清要之职全数推拒,始终要在外任官的行为,才能得到士林的交口称赞,人望就是这么来的。

韩冈如果要学他岳父,光是推辞中书检正一职并不够,还要出外才行。而韩冈推脱中书都检正,却只是为了求一个判军器监,那么理所当然,冯京的指责便不成立。

——可冯京其实并没有指责韩冈,他只是信口的插了一句,不经意间惹得天子心中起了猜疑。这算是陷害手段上了境界了。

‘年轻人还是太嫩啊!’

冯京悠悠一笑,上前一步对赵顼道:“陛下,韩冈既然胸有成竹,之前又有韩绛之荐,不如便让他去军器监一展长才,想必很快便能有所成就。”

眼下韩冈尽力撇清他辞官以博名望的指控,也便在一两年内失去了去中书担任五房检正的可能。将韩冈堵在中书之外,这正是冯京今日的首要目的。他今日说的、做的,其实就是要让韩冈去不成中书,就算日后改了心意,也转不回来。

只要韩冈不是去中书门下,不论他是出外,还是去其他监司,对冯京来说都是件好事!更别说猜疑这颗有毒的种子,一旦种下,就没有连根拔起的可能。

“放大镜、雪橇车、霹雳砲、军棋沙盘,得韩冈主持,想必军器监所造军器当会更胜过往!”冯京步步紧逼,一点也不给韩冈喘息的机会。第一个目的达成,那第二个目的自然就要浮上台面。

所谓判军器监的‘过往’是谁?

——是吕惠卿!

想想吕惠卿接替曾布判司农寺的职位后,第一件事做得是什么?是下发了一道公文,说此前司农寺中‘官吏推行多违本意,及原法措置未尽,弊症难免。’这份公文,是在曾布叛离新党的过程中,很是出了一把大力。

难道吕惠卿不担心韩冈会有样学样?!

当一个参知政事出手干扰,韩冈又怎么在吕惠卿的固有地盘上施展他的才华?

所以说,年轻人还是太嫩了!

冯京得意无比。

一名宰相推荐,一名宰相附和,当事人又极力争取,虽然明知韩冈就是怕了中书里的麻烦事才不肯去,赵顼也不可能由着脾气一口给否决掉。同时,韩冈对于判军器监这个差遣如此迫切,也让赵顼心中也有了些期待:“既如此,军器监一事,便交由韩卿你来统管!”

“臣谨受命。必竭心尽力,兢兢业业,不敢有丝毫懈怠。”韩冈叩拜下去,他去军器监的任命如此便算是定下来了。只要之后中书签发下来,他就是继吕惠卿、曾孝宽之后的第三任判军器监了。

闹了一通,想不到最后还是让韩冈如了愿,赵顼摇头苦笑。天子说得话不及臣子有用,他的心中免不了有些芥蒂,“不去中书门下,却求着要去军器监。韩卿所求,朝中当是不会有第二人了。”

天子语气中的抱怨,韩冈如何听不出来。要不是冯京陷害,也不至于今天在殿上的窘境。他想着,就瞥了冯京一眼。

不去中书蹚浑水,而是去军器监博功劳,这是他韩冈的本意,现在看来,却也是如了冯京的心意。冯京端严肃正的表情下,那抹藏得很深的得意,让韩冈看得很不舒服。

一直以来,他所保持的习惯,或者说在天子面前保持的风度,是尽量不攻击他人,仅仅是就事论事。

当日在君前驳斥郑侠的指控,那时正逢赵顼盛怒,他也没有直接反驳,而是曲言分辩,只是最后闲闲一句,将郑侠送去了恩州——说起来,倒也有些像冯京今天的手段。

不过今天,过去的原则却要改一改了。

“陛下所言,微臣实不敢当。”韩冈谦虚道。冯京今天没有一句正面指责,的确不便反咬,但要给他上点眼药也不难。他微笑着一望冯京:“微臣今日的选择,却是学着冯相公。”

“学得哪里?”赵顼半是顺口,半是好奇的问道。

“微臣今日的心意,与冯相公当年严拒宣徽使张尧佐相仿佛,不愿多受牵累,只愿一展所长。”

说自己选择军器监,去跟冯京当初拒绝做张尧佐的女婿是一个道理,这个比喻不伦不类,更是明明白白的讽刺!

冯京当年不做温成皇后亲叔张尧佐家的女婿,而是娶了富弼家的女儿,难道是不畏权贵?还不是不想受到牵累!当了外戚的女婿,想顺顺当当的升官,除非御史都变成了哑巴——更别说张尧佐当时还不受官场待见,被包拯领头三番四次的敲打,仁宗皇帝被喷得满脸口水就是这个时候。

他韩冈是为了能更好的施展才华,为天子效力,所以才弃了中书都检正一职,选择了判军器监。但冯京弃张家女而娶富家女,又是为什么呢?是为国为民吗?

冯京牙齿咬了起来,韩冈也是宰相女婿,难道他自己的身上有多干净!?

但对于韩冈的讥刺,冯京却不能针对性的反击。韩冈的攻击实在太直接了,直接到以宰相的身份甚至不便直接反斥回去。否则宰相在殿上与一名小官斗起嘴来,丢脸的只会是宰相,是他冯京!

而韩冈如此说的用意……冯京偷眼向殿上望去,看到天子的脸色,心头便是一惊。

赵顼眉头紧锁,韩冈这算是十分直白的攻击,他如何听不明白?这未免太过分了一点,想着便要斥责。只是看到台陛下的两名臣僚的神色,到了嘴边的话却突然给堵住了……韩冈为什么要攻击冯京!?如此莽撞、直白、甚至是粗糙的攻击,这跟他的为人、才智完全不符。而且原因何在?

不见赵顼出声,韩冈就知道他成功了。

赵顼不是蠢人,又做了这么些年皇帝,让人牵着鼻子或许一时察觉不了,但只要有人点破,当然立刻就能反应过来。韩冈最后针对冯京的话,其实就是在点醒赵顼,让他去想想冯京到底说了些什么。

点破就足够了。

心怀叵测,以言辞扇摇君心——是一个判军器监的右正言危害大,还是一个宰相的危害大,想必天子自己能得出结论。

‘冯相公……’韩冈一瞥脸上阴云渐聚的冯京,双眉一轩,‘来而不往非礼也!’

