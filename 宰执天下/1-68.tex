\section{第30章 臣戍边关觅封侯(三)}

“谁说有了官身就不能考进士的?”王厚放下酒碗,奇怪的对韩冈反问道,“宰执家的子孙七八岁就受了荫补,但照样有出来考个进士的。尊师横渠先生的举主吕中丞,是吕文靖【吕夷简,仁宗朝宰相,谥号文靖】之子,早有荫补在身,但还不是考了个进士出来。有官身者参加科举远较普通士子方便,只要通过锁厅试就能得个贡生名额,可比参加州里的解试容易许多。”

韩冈一听,忙加追询,这是他前身留下的记忆中所没有的信息。王厚很惊讶为什么韩冈对此茫然不知,却还是一边喝酒,一边向他细细解释。

所谓锁厅,顾名思义就是锁起公厅,也就是官员将自己的官厅锁起,放下手中的职务,去参加科举的意思。

天下意欲参加科举的士子有百万之众,东京城可容纳不了那么多。所以必须在地方加以选拔。这种选拔称为解试,都是在科举之年的前一年在各个州军举行。秦州的解试,便是在今年八月,韩冈躺在病床上时结束的。通过解试的士子称为贡生,而第一名就是解元。有了贡生的资格,便可以去京里参加科举。

而京城的进士科举又分为两个步骤,第一步是省试,又名礼部试,将从天下四百军州的数千近万名贡生中,挑选出三百名左右的合格者——也有时是两百或四百——如果能成为三百名合格成员之一,基本上进士的资格就确定了。因为如今第二步的殿试,不会再黜落考生,只是决定名次高下的考试。

“这还要多谢张元!”王厚笑道:“西夏的这名张太师,就是从殿试上被黜落,最后愤而投奔西贼的。‘韩琦未足奇,夏竦何曾耸’【注1】,两名宰相之才,竟然被一个黜落的贡生打得颜面无光,几万将士因此葬身好水川畔。自此之后,殿试再也不黜落一人,就算犯了杂讳,也不过降至最低一等的同学究出身,照样给官。还有特奏名进士,也是为了安抚屡考不中的贡生而特加拔擢。”

所以要当上进士只有两道难关,第一道是解试,第二道是礼部试。而韩冈有了官身后如果还要考进士,一样要通过解试。只是因为他的官身,就不能与普通士子一起在州中考试,而是在路中参加特别为官员举办的锁厅试——这里的路,是转运使路,而不是经略安抚使路,也就是韩冈要去陕西路的路治京兆府【长安】去参加,而不是就在秦凤路的秦州——

“名义上将锁厅试放在路中,是为了不与地方上的寒士争位,但实际上州中贡生选取比例,在江南诸路是百里挑一、两百挑一,在陕西也是二十、三十选一,可锁厅试却是三五人里就能出一个贡生,最多也不过七中选一。”

王厚说得口干,给自己满上酒,一口喝下去。用丝巾擦擦嘴,继续道:“不仅是官员参加的锁厅试,还有官宦子弟参加的别头试,也是举着不与寒士争位的名义,可实际录取比也是放在十比一以下。想想家严,当年参加江州解试,可是近三千人争十七个名额!”

“三千人争十七个?”这差不多是后世公务员考试比较热门的职位的录取比例了。这么低的比例,竞争的确够惨烈的。而且贡生跟做官无关,不是明清的举人,就算今次考上,如果不能得中进士。下次照样打回原形,得重新再与三千人争去。

“就是三千争十七。”王厚以为韩冈被惊到了,遂更加得意说起,“这还算是少的。你到福建路看看,尤其是建州、福州,那里是五六千人争夺十几个名额!哪一科不是杀得血流漂杵、尸积如山!”

王厚说得夸张,引得韩冈轻笑起来:“可礼部试是一视同仁,不论身份家世,不论地望出身,解试困难也好,容易也好。到了礼部试中,都是一样的考题。”

“没错。”王厚很自豪的抬起头:“江西、福建的贡生都是从独木桥上杀出来的,而陕西贡生走的则是通衢大路。可到了礼部试上,十名江西贡生就能出一个进士,而陕西贡生一百人也出不了一个。”

韩冈感慨道:“所以啊……到最后,特奏名进士大半都是陕西人。”特奏名进士,就是年过四十、屡考不中的贡生,由地方统计名单呈到朝廷,参加一次很简单的考试,赐给他们一个官职,去州学、县学中做个文学、助教,省得他们投奔西夏、辽国去。陕西考贡生容易,中进士难,所以特奏名中,多是陕西人。

王厚知道韩冈为何感慨,他安慰拍拍韩冈肩膀,举起酒碗:“反正特奏名也与玉昆你无关了,来喝酒,喝酒!”

……………………

一顿酒不知喝了多久,韩冈酒量甚豪,还保持着清醒。但王厚没什么酒量,已经晕头转向。但他仍是颤颤巍巍的举着酒碗,对韩冈道:“玉昆,真是可喜可贺!尊师张横渠,今月初九已经擢了崇文院校书,日后必然要大用啊!来,我们再喝一碗!”

“处道,这已是你说的第三遍。该贺的也贺了,该喜的也喜了。你就别喝了!”

“多喝一点没关系。喜事嘛……等横渠先生在朝中水涨船高,来向你提亲的人可会越来越多……哈哈,玉昆论相貌也不输那金毛鼠多少,就是少个状元及第,要不然,宰相家的娇客也能做。”

“锦毛鼠……”韩冈大吃一惊,“白玉堂?”七侠五义中的名角难道真的出现在正史中过?!

“白玉堂是谁?”王厚抬起醉眼,茫茫然问着。

“啊……曾经听说过中原江湖中有个强贼,匪号锦毛鼠。”韩冈随口解释了两句,心中疑惑,难道北宋有另外一个锦毛鼠?

王厚醉得糊涂,也没去分辨真假,哈哈笑了笑:“想不到玉昆你交游如此之广!”

“只是些口耳相传的谣言罢了。也记不清究竟是在寄居的寺庙还是在茶肆中听到的,连什么时候听说的也记不得了。”韩冈将之一推了事,结交匪类的罪名他可承受不起。

“愚兄说的是皇佑元年【西元1049年】己丑科三元及第的那一位,他前几年不是来关西知京兆府的吗?”

韩冈啪的一声拍了下脑门,给王厚这么一提,他终于想起来了,“是冯当世啊……”

冯京,字当世。皇佑元年己丑科状元,乡试、省试、殿试皆第一,是历史上不多的几名三元及第的状元郎。冯京才学过人,相貌出众,但不知是不是因为商人家庭出身的缘故,对钱财十分看重,在京兆府任上大肆聚敛,被长安士人暗嘲为‘金毛鼠’——‘金毛’指得他仪容出色,而‘鼠’便是说的他聚敛之行。

“没错,没错,就是他!”王厚醉态可掬的笑着,说起话来舌头都大了,“当时冯当世中了状元后,几家贵戚一起在争他这个女婿,摆出来的嫁妆几万贯,最后还是给富相公捷足先登,而富相公又是太平相公【晏殊】的女婿……若是玉昆你能找个好亲事,说不定日后也是个宰……宰……”嘣地一声,王厚一头栽倒在桌上。

韩冈有些无奈的看着自己房里的醉鬼,话说到一半,就醉昏了过去。苦笑着摇了摇头,他放下酒碗。也许是习惯,韩冈不由自主的又开始去推断张载此番在京中为官,能给自己带来些什么。

张载是受吕公著的举荐而入京的,半年前韩冈回家奔丧时,张载已经打理行装准备东行。当时吕公著还是翰林学士,但如今吕公著已经是御史中丞,掌握着朝中的监察大权。

而张载的弟弟张戬,韩冈也见过,一样进士出身,在朝中做了吕公著的下属,任监察御史里行一职——担任监察御史的官员如果资历不不够,就要在官名后面缀上里行二字,意为试用——有着举主和兄弟在朝中护持,韩冈的老师应该能在京中多待两年。

但韩冈方才又从王厚这里得知,吕公著能升任御史中丞,完全是王安石王相公想把枢密使吕公弼赶出东京。韩冈对此完全能理解,兄弟两人一个是军方的首脑,一个是监察系统的老大,这在哪个朝代都是很犯忌讳的一件事,吕公弼识趣的就会自己辞职,如果不识趣,御史台中保不准会造吕公著的反,兄弟两人一起被弹劾。

如今的朝中局势错综复杂,谁也看不清,韩冈也一样。张载的后台与王安石不合,但张载本人帮着蔡挺改进的将兵法,却是深得王相公的赞许,也不知他本人对变法的看法又如何。但韩冈很清楚自己的立场,王韶在朝中的最大依仗就是王安石,自己如今的依仗则是王韶,对于变法,只有赞同,不能反对。

王厚不知什么时候又醒了过来,拿起酒坛子晃了晃,听着里面没有水声。便拍着桌子,口齿不清的怒道:“怎么没酒了?!”

“都给你喝完了……”韩冈无奈的叹了口气,王厚来他这边喝酒,有时是自带酒菜,有时候便是蹭吃蹭喝,韩冈大手大脚,手上的一点钱钞都给耗光了。今天回去,没好意思向家里拿钱,现在是囊中空空,“今天是没钱添酒了,等明天再说。”

“钱?……”王厚吃力的抬起头,“没问题,等到青苗贷正式实行,我们这里就该有钱了。”

注1:张元投奔西夏后,辅佐李元昊在好水川全歼了三万宋军,而当时主持关西军政的便是夏竦和韩琦。好水川之战后,张元在题诗一首——‘韩琦未足奇,夏竦何曾耸’,一泄多年怨气。

ps:陷空岛五鼠只有一鼠,就不知道锦毛鼠白玉堂的名号原型是不是冯京。

今天第二更,求红票,收藏。

