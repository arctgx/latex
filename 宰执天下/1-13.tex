\section{第八章 破釜沉舟自专横(上)}

李癞子离开李将军庙后,径自回到家中。李癞子家的宅子是有着四进六院的大宅,他回来后没有往后院走,而是去了接待亲朋好友的内厅。

内厅中,一名身穿皂色公服的衙役正坐着品茶。不是别人,正是李癞子的亲家,八娘的舅翁【注1】,在成纪县衙中做班头的黄德用黄大瘤。自来只有起错的名字,没有起错的绰号,黄大瘤人如其名,脖子上正有个鸡蛋大的肉瘤子,上面青筋外露,头一动就是一阵摇晃,看着让人作呕。

“亲家回来了?”见着李癞子进来,黄德用放下手中的粗瓷茶盏,仍大剌剌的坐着,一副反客为主的模样,他问道:“李将军庙里的那顿酒喝得如何?”

两人虽是亲家,但李癞子只是个土财主,而黄大瘤在县中却是陈押司的亲信。黄德用的无礼,李癞子也只能视而不见,拱了拱手,笑道:“还得多谢亲家的计策,韩菜园连脸都青了。”

坐下来,等下人奉上茶汤,李癞子叹了口气,道:“不过如今一来,俺可是把韩菜园给得罪狠了。”

黄德用哼了一声,对李癞子的担忧不屑一顾:“其实本不需如此,但韩菜园既然不识好歹,也顾不得什么了。反正韩菜园又不是陕西乡里,不过是个外来户,没个亲族支持,怕他作甚?!”

“韩家的三哥在宴席上都是冷着眼在看,连句话都没开口。他在外游学两年,也许认识了几个奢遮人物。就怕他会坏事啊……”李癞子眉头皱着。韩阿李的擀面杖躲远点便没事了,但韩冈方才在宴席上的眼神和表情,让他心中着实有些发毛。总觉得有哪里不对劲,无法安下心来。

“十几岁的毛孩子,能认识什么人物?再奢遮能奢遮得过陈押司?”黄德用毫不为意的冷笑着,“亲家你操个什么心,你想想这么多年了,秦州可曾出过一个进士?”

李癞子摇了摇头,这还真没听说过。他嘿嘿笑道:“……破落的措大倒是见得多了。”

“中不了进士,进不了学,那一辈子就是个村措大。运气好的,从现在考到四五十岁,让官家看着可怜,弄个特奏名。在哪里当个文学、助教什么的。那等寒酸措大,不需劳烦陈押司,俺一根手指便碾死了。”黄大瘤口气狂到了天上,仿佛自家不是区区一个县衙班头,而是手握数万强兵的大将。

李癞子也算是有些见识,知道什么是特奏名。也就是那些入京履考不中的举人,年龄至少要在四十岁以上,地方上特别奏其名入朝中,由天子特下恩旨,聚集起来进行一次远比进士试要简单的考试,再给合格的一个不入流的小官做做。

特奏名进士以陕西为多,也是怕他们投了西夏。当年在殿试上被黜落的张元还有屡考不中的吴昊,领着李元昊把陕西闹了个天翻地覆。就是现如今,西夏的朝堂上也还有不少从陕西跑过去的汉人臣僚。那些个怨气深重的读书人最是危险不过,自得给块骨头安抚安抚。

“抬头看天,秦州这里看不到文曲星。韩三最多也只能熬出个特奏名来。想中进士,除非他家祖坟上冒青烟!”黄德用摇头晃瘤给韩冈判了命,确定他是一辈子的穷措大。

李癞子笑道:“听亲家你一说,俺的心也就定了。那就还按着前日商议的,把韩菜园弄到县里去,给个亏空多的差事,逼得他把田给断卖了。”

黄德用拍着胸脯:“亲家你放心。一切且交给俺黄德用。只要那韩菜园到了县中,包管把他收拾得服服帖帖。”

李癞子心愿得偿,笑容也变得得意起来,“韩菜园种田是把好手,有他指点,村里的庄稼长得硬是比隔邻的几个村子好个那么一两成。要不他的那块菜园子把俺家的河湾田分成两半,卖了之后还打着赎回的主意,俺何必做个恶人。”

“一亩麦田一季只要一车粪。但种上一亩菜园,少说也要三车粪肥。韩家料理那块地快三十年了,施下去的肥料能把三亩地给埋起一人多高。怕是比江南的上等田还要肥许多……”黄德用意味深长的说着。

“亲家你放心。”这次是李癞子对黄大瘤说放心,“北山的那片地就算是我家八娘的脂粉田【注2】,过两日就把田契给你那儿送去。”

“嗯……”黄德用不动声色的点了点头,还是并不满意的样子。北山的田可比不上河湾田,出息和田价都差得远了。

“……还有韩家的那个养娘。等韩菜园逼到急处肯定也会卖掉,到时便送到亲家府上服侍。”

黄德用终于笑了,脖子下的瘤子抖的厉害,“一家人何必说两家话。亲家但凡有事托俺,俺黄德用什么时候没尽心尽力去办过?北山那块田是给新妇【注3】的,俺岂会贪你的?韩家的养娘俺也只是看着她伶俐罢了……”

李癞子听着黄大瘤假撇清,心中都觉得恶心,忙举起酒杯笑道,“亲家说得是!说得是!来……喝酒!喝酒!”

两人举杯痛饮,提前庆贺自己心愿将成。觥筹交错,喝到三更方休。一个癞子,一个瘤子,倒也是好搭配。

…………………………

李癞子和黄大瘤正算计着韩家。而将军庙中的宴席已经结束,韩家四人聚在正屋里,也在商讨着应对的策略。

“李癞子先说是县中刚刚行文,上面有俺的名字,后又说看在三哥儿的病上,帮俺拖了两个月,等到跟刘槐树说的时候,又变成了县中没有定下要谁去应差役,哪个代俺去都可以。几句话的工夫,连变了三种说道,根本就是睁眼扯瞎话!”

韩家的正厢中,韩千六气哼哼的说着。李癞子方才在李将军庙中,说谎也不待眨眼,明明白白的要夺他韩家的地,连脸皮都不要了。

“人善被人欺,马善被人骑。李癞子在将军庙里胡扯的时候,你怎么不一凳子砸死他!照老娘说,抄起刀子,去他家拼个你死我活!”韩阿李的脾气比爆竹还火暴三分,点着就着的那种。粗重得跟支铁简也差不离的擀面杖还紧紧攥在手中,一边说话一边挥舞,只恨方才李癞子跑得太快,没来得及给他一记狠的。

“胡说个什么!那要吃官司的!”韩千六摇着头,韩阿李妇道人家说个气话没什么,他可不能跟着昏头,“三哥儿的前程要紧。”

韩冈沉默着。在将军庙里,他笑语盈盈,充满自信,从庙中回来,也是一派安稳,气息宁定。将心中的熊熊怒火藏得无人看出,只有收在袖中的拳头握得死紧,如刀双眉微不可察的颤着,似是要出鞘斩人。韩冈如今杀了李癞子全家的心都有了,李癞子打他家菜园的主意不提,如今又把手伸到云娘身上,用得还是如此恶毒的手段,直欲逼着韩家家破人亡,这事他如何能忍?!

不过,这也是韩家没有权势的缘故,如果他是相州韩家的子嗣,谁人敢小觑他一眼?如果他现在已经名动关中,又岂是李癞子之辈所能欺辱?

‘不会永远如此的!’韩冈恶狠狠地想着。如今的情况下,不论用什么办法,总要为自己弄到一张官皮来护身。只恨李癞子逼得太急,却也不是整理理论的时候了。

但即便没有了慢慢做学问的时间,韩冈也照样无所畏惧。这个时代毕竟是文人当家,秦州城里官员百十,有多少文官在!自己有学问、有才能,外形又不算差,还有个名气够大的老师,岂是李癞子能动得了?韩冈本想着走稳一点,但有事临头,那就稍快两步也无妨。总得让人知道,惹到他韩冈,究竟会有个什么结果!

韩冈突然开口,对韩阿李道:“娘娘,只捅上李癞子几刀那样太不解气,还要把自家搭进去。照孩儿看,莫名其妙多了一份要衙前的文书,这一切的根源肯定就在城里,李癞子也不过是借了黄大瘤和陈举的虎皮罢了。不如先以应役的名义去城中走一遭,总有办法可想,留在村里只能是坐困愁城!”

若是这话让韩千六说,韩阿李肯定要发火,但由最心疼的小儿子说来,她却能听得进去。犹豫了半天,方不情愿的道:“难道真要让李癞子得意不成?……也罢,你爹在城里也认识几个人!”

韩冈笑着摇头:“爹爹年纪大了,还是让孩儿去城里走一遭罢!”

“那怎么行!?”韩阿李和韩千六脸色大变,就这么一个儿子了,再出点意外日后谁给他们送终?韩千六忙道:“三哥儿你病还没好利索,又才十八岁,怎么去得了?!”

韩冈仍然坚持己见,现在这种情况下,留在村里毫无机会。只有走出去才能杀出一条路来,不论是整治李癞子以及他身后的黄大瘤和陈举之辈,还是为自己博一个功名,都必须走出去。许多村人不敢离开乡土,任凭县里的胥吏和本村的里正欺辱。

这等贼子就是靠着隔绝上官和百姓,从而内外渔利。但韩冈不同,士人周游天下,是从祖师爷那里传下来的传统,他又来自后世,更是把离乡背井视作等闲。出村进城,为自己讨个说法,就像吃饭喝水一般简单,根本不算什么。

注1:中国古代,大约是元明之前,媳妇称呼夫家父母不是公公婆婆,而舅、姑。所谓‘待晓堂前拜舅姑”,便说的是洞房花烛后出外拜见公婆。

注2:宋代嫁妆田的另一种说法,以助出嫁女儿脂粉花用的名义,让女儿带一块田地出嫁。

注3:宋代的新妇大略是媳妇的意思,与新婚与否无关。嫁人十几年只要没熬成婆婆,照样是新妇。

ps:韩冈就要入城,高潮将至,求红票、收藏助阵。

