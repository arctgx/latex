\section{第15章 焰上云霄思逐寇(一)}

周围尽是火光。

火焰已经笼罩了邕州州衙。前后六进,左右皆有偏院,有楼阁、有花园,是邕州城中最大的建筑群,而此时,则化为了火海。

苏缄穿着公服,带着长脚幞头,一步步的在熊熊烈火的环绕下,用脚上的厚底官靴丈量着地面。端正的容装一丝不苟,就算立刻去觐见天子都不会失礼。

举步越过门槛,踏过仪门。身后的大堂被大火吞噬,攒动的火蛇在屋瓦上游动,每一扇门窗都在向外面吞吐的着火焰。

苏缄还记得他来到邕州后,第一桩案子就是在大堂中审的。他历任地方,很少有一上来就碰上一桩谋杀案。为了审那桩案子,苏缄可没少辛苦,光是往返与州里、县里以及桂州的公文就有十几斤重,用了一年的时间,才将定案的判状呈送东京,让天子勾了名字。现在想来,也就是靠着这桩案子,让自己的威信在邕州树立了起来。

之后的数年里,不论是审理要案,还是举行年节酒宴,都是在大堂中举行。熟悉的建筑很快就要不复存在,苏缄却发现自己却没有太多的伤感。

踏过侧门,二堂也窜起了火苗,堂中闪着火光。几点火星跳了出来,又攀上庭前一角的刺桐树。刺桐已经开花了,凝聚了血与火的树木上,朵朵红花就犹如火焰一般。传说此树若开花不依时节,邕州必遭兵焚。许多人信之不移,不过今天便可知传说的虚妄了。二月之初,正是刺桐花开正盛的时候。

州衙外面一阵喊杀声传来,这是守护州衙的士兵们在尽最后的努力,只是很快就沉寂了下去。

‘王亢也殉国了。’

就跟这座邕州城一样,坚守了近两个月,终究还没有坚持到援军的抵达。

苏缄在空寂无人的庭院中慢慢走着。

往日里,这座庭院之中,总会有着上百官吏往来奔走,喧嚣不绝。从早至晚,由夜达旦。但到了最后的时候,邕州的文武官员中,还在这里的就只剩下他一人。

唐子正昨夜战死了,在斩杀了多名攻入城中的交趾贼寇,于城墙脚下上流尽了最后一滴血。‘不过一死而已’,他的副手言出如山。

观察推官谭必死了,录事参军周成也死了,当城南的军营今早被攻破的时候,营中就立刻起了火,他们都选择了自尽殉国。

都监薛举是最早战死的一个,为了阻止交趾人垒筑上城的高台,他领军出城,第一次成功,第二次成功,第三次就中了李常杰的埋伏。也就在那一天,另一位都监、西头供奉官刘师谷也战死在城外的另一个方向。

在之后争夺城墙的几天中,钤辖高卞中箭而亡,宣化县尉周颜则是死于上城的交趾军长枪。陈琦、丁琦、邵先、梁耸、李翔、何泌、刘公绰,州城中的大小武官在这些日子里,也都陆陆续续战死。

城破之后,都监刘希甫回守城南军营,今日与谭必、周成一同殉国。宣化知县欧阳延在昨夜就与他的县衙一起投入火海。自己的次子苏子正,前两日在城头上被砍断右臂之后救治不及。长孙苏直温因荫补而挂着武职,上阵后不久就中了箭,也没能救回来。

如今的州衙之外的最后一道防线,是武缘知县王亢在把守。因为他在交趾来攻时,放弃了自己的职责,逃进了邕州城。被苏缄痛斥之后,却是立下了死志。会让他把守州衙,也是因为他此前已经在城墙上受了重伤,上不了阵了。

到了最后的关头,他苏缄的属下中,没有一个懦夫,也没有一人退缩。

一阵热浪随风卷起,苏缄的视线也模糊了起来。热流划过脸颊,探手抹了一下,落入指尖的却是濡.湿的触感。

真的很热。

州衙之外,已经全是人声,乱乱糟糟的不知在说着什么。苏缄听不懂交趾土话,但夹在在土话中自己的名字却不会听错。

想必是要活捉自己吧。

苏缄像是想到了什么好笑的事,咧开嘴呵呵的笑了起来:“吾乃大宋守臣,岂能死于贼手?”

一声剧烈的轰鸣响过,一阵狂风从身后飚来。苏缄缓缓转过身。是大堂塌了。坍塌下来屋顶,砸得火光一黯,但转眼火焰又直冲而上,窜起了有十余丈高,然后又落了回来,上下闪动了几个来回之后方才又开始稳定的燃烧。

大堂塌了、二堂也被祝融吞没,前院已成火海,红灿灿的映着夜色中的天空。融石铄金的热量向着天地四方全力散发出去,郁郁苍苍的树木,都在发出干柴在炉膛里燃烧时的噼噼啵啵的声音。

后花园和柴房也烧起来了,苏缄家里不缺忠心的仆佣。在守城的日子里,有许多都拿起了弓刀,上了城墙。而剩下的老弱妇孺,苏缄在城破后都让他们逃出了州衙,能不能躲过这场劫数只能看他的命运。

“老爷。”

穿过了宅门,自幼服侍着苏缄的老仆迎了上来。

‘还有人迎接自己啊。’苏缄走了上去,责怪着:“不是让你们走了吗?”

“小人一辈子都跟着老爷,老爷去哪里,小人就跟着服侍。”

苏缄看着眼前几十年来一直都在眼前的面孔,叹了一声,不劝了。问道:“三哥儿他们都走了?”

“嗯。”老仆低下头擦着眼睛,不让眼眶中的泪水流出来。

三子苏子明会一点医术,苏缄让他学着管理城中医疗急救。日以继夜,没能撑到最后就病倒了,最后的一段日子只能躺在家中。

“二哥儿一家也走了?”

“嗯。”

“大哥家里呢?”

老仆撇过脸,低头看着地面,声音小小的:“都一起喝了酒。”

苏缄一瞬间又老了几分,更加憔悴,嘴角只有惨淡的笑容,“他们不合是苏家的人。”

“不关老爷的事!”老仆猛抬头,几十年来第一次对着苏缄大声:“都是沈起、刘彝造的孽!”

“这时候就不用再说了。”苏缄慢慢的向前走着,老仆扶了上来,“还记得小时候,一起下海,也只有你敢与陪着我去。”

“回来后老爷就被老太爷打得不能走了。”老仆笑着,一起回想着的当年,“那时候都没想到老爷能做到知州,当时连进士都不知能不能考中。”

“快五十年了。过得还真快。”苏缄叹着时光变迁,旧日的记忆在脑海中如同走马灯一般一一掠过。

“可不是吗……小人也没想过自己也能活到六十。”扶着苏缄走到正厅前,老仆放了手,“老爷,小的要先走一步,下辈子再服侍老爷。”

他跪下来重重的磕了一个头,站起身蹒跚的走入着了火的后院。

望着火焰封起的门扉,苏缄叹息着:“没能救了满城百姓,这罪过不知有多大……你下辈子投的胎肯定会要比我好啊。”

外面更吵了,一阵沉闷的锤击声响了起来,似乎是有人用着檑木或是重锤撞着院墙。

苏缄皱眉看了看声音传来的方向。自己应该是吸引不了这么多人卖力,大概是怕府中积存的财物一把火被烧干净吧。他们肯定要失望了,官财私财在守城的日子中,都已散尽,哪里还有留给他们这群强盗的。

苏缄快不行了,随着火势越大,空气也越来越憋闷,呼吸进肺中的都是火辣辣的炎气。步履维艰的走进正厅中,慢慢的在自己熟悉的位置上坐下。

周围的火焰渐渐升了上来,飞窜起来的火苗,已经舔舐到了房梁之上。梁柱上的涂漆很快就被引燃,噼里啪啦的响着。柱子和天花上的彩绘受热之后,一块块剥离掉落,就在地面上燃烧着。

在明道年间这座小楼重修时,梁柱天花上就绘上了彩绘,精美之处远胜衙中的其他建筑。只是经过了几十年没有修补,苏缄来上任时,这些彩绘早已是斑驳不堪。曾有人向苏缄提议要修补一下,否则太难看。但苏缄算了一下开支之后,就把这个提议丢到了一边去了。还有后院的凉亭,两年前也在风雨中被倒下的树木砸开了半边,苏缄也没有让人去修。舍不得乱花钱啊。

官袍的衣角被火舌舔了一下,转眼就烧了上来。苏缄没有理会,拿起早已放在桌上的酒壶,给自己倒满了一杯酒。

拿起盛满酒的杯子,火焰又蹿髙了一点,可苏缄已经感觉不到身边的热了。

有些吃力的转头看看隔着一张桌子,伴了自己一辈子的老妻,闭着眼睛,就像睡过去一样。四十五年结缡相伴,本来想致仕后就回家乡闭门读书度日,夫妻两人过完最后的日子,谁能想到竟然在这里同生共死。

举杯一饮而尽,火热从喉间渗入腹中。苏缄想不到这酒的味道还不坏,就是只能喝上一次。一家三十七口,除了长子苏子元一人,还有战死的两个儿孙,其他人一起都在这座州衙中喝了同样的酒。他们不合作了邕州知州的家人啊,要不然也不会造此劫难。

腹中更热,火焰的颜色充斥在眼中,苏缄对家人的愧疚渐渐散去,最后只有一个念头在心中徘徊。

只恨没救了满城的百姓!

