\section{第八章 四句千古传(上)}

“为儒者,当为万世开太平!”

振聋发聩的一句从身后传来,惹得厅中的学子们人人向后张望过去。

只见着在正厅门外,一人驻足矗立。逆着光,让人看不清他的相貌年岁,只能看得出他身材高大健硕,不似普通的士子,却仿佛一名冲锋陷阵的勇将。

‘是谁?!’

近百人的头脑中疑问丛生。

此一句,不但将儒者的最终目标为之点明,还与前三句相互呼应。能接上这提纲挈领的第四句,可见是对横渠之学已是融会贯通。

‘究竟是谁?’

厅中大部分学子还没有弄清楚突然冒出来的这一位究竟是何方人氏,疑惑还未有解清,门外的那人已经跨步进厅。脚步不停,口中亦不停,一步一句:

“上辅君王,下安黎民,外服夷狄,内平贼寇,开万世太平之基业。此数事,非儒者谁人可当?!”

铿锵有力的声音中,潜藏着几分激昂,充满着鼓舞人心的力道。

来人走进厅中,厅内的人们终于看清了他的相貌,亦是眼前一亮。

大约只有二十出头,年轻得过分,双眉平直,鼻梁挺秀,眼中神光内敛,却隐含威严。肤色略黑,是常年风吹日晒后的痕迹,与一般在家中苦读的士子截然不同。身着普通的儒生外出游学的行装,可几步走来,举手投足中表现出来的气质,却明显的只有身居高位之人才能拥有,与他的年纪对不上号。

年龄与气度之间的巨大差距,使得来人的身份已经呼之欲出。横渠门下弟子众多,能年纪轻轻就身居高位的,数来数去,也只有一人。

许多人都惊喜得站了起来,其中就有弟子中年岁最长的吕大忠。

“当为万世开太平!”

来人一边说着,一边穿过纷纷避让开来的学子,一路走到同样起身相迎的张载面前,他跪下来大礼参拜:“韩冈拜见先生!”

‘果然他就是韩冈!’

‘难怪!’

原本韩冈在张载门下弟子的心目中,已经是一个让人赞叹不已的同门。在发明创见上,医疗制度,军棋沙盘,还有被天子命名的霹雳砲,加上让张载都受到启发的格物之说,都可以看出韩冈的才学。而经世济用的手腕上,又有辅佐王韶得成开疆拓土的功业,非等闲士子可以。

在张载门下,很有些人都把韩冈视作未来的名臣。日后光大横渠门楣,非此人莫属。

而今日韩冈的出现,如同奇峰突出,一句话就坐实了他张载门下杰出弟子的身份。几可与吕大钧、苏昞和范育这些久随张载的师兄们平起平坐。

“好!好!好!”

张载开怀大笑,亲手将韩冈扶了起来。

为万世开太平。

韩冈的这一句,正说到了关节上!

张载抬头看着自己门下最为出色的弟子中的一人,欣慰的点头赞着,“这数载玉昆你在熙湟助王子相威服青唐,收编众羌,安抚熙河之局既定,围攻党项之势将成,此一句非你不得言!”

儒家讲究着内圣外王,修身齐家治国平天下,无处不讲究着这内外四字,太平盛世也并非只靠武力便能得来。但对于饱受党项贼虏侵扰的关西来说,外服夷狄才是开太平的前提。

韩冈拱手行礼,谢过张载的赞许。

张载站上前,对着众弟子道:“班固有言:儒家者流,盖出于司徒之官,助人君顺阴阳明教化者也。游文于六经之中,留意于仁义之际,祖述尧舜,宪章文武,宗师仲尼。此一段……误矣!”

李复脸一红,听着张载继续道:

“儒者立于天地之间,格万物而体至理,习大道而治天下,岂是此数言可拘?”

“为天地立心!为生民立命!为往圣继绝学!为万世开太平!行此四事者方可为儒!”

在今天的这一场特别的讲会上,张载欣慰的了解到了他的学术可谓是后继有人。吕、苏、范几个大弟子不算,年轻一辈中,也出韩冈这般难得一见的人才。

而且这几人都已经将关学所传融会贯通,给出的答案比他预计得还要出色。心怀大畅,张载讲学的时间也便比平日还要长了许多,不但宣讲,而且还不住解答学生们的疑问,直到日影西斜。

一声玉罄响,今日的讲学结束。对着已经喉咙沙哑的张载,吕大钧领着众弟子向他恭恭敬敬的拜谢下去:“谢先生传道!”

……………………

学生们带着好奇的目光离开了,各自回书院中的房间去了。

虽然他们还想跟韩冈结交一番,但很明显张载要与韩冈先说说话。

被张载单独留了下来,就在正厅之中。韩冈用眼角余光打量着这一间可比得上中等寺庙大雄宝殿的建筑,高丈许,横阔皆有数丈,中有八根大柱支撑,容纳下方才的近百名学生,并不显得拥挤。只是几乎没有纹饰,仅仅上了一遍漆——毕竟还是要省钱。

在大厅左右双牅上,果然篆刻着《钉顽》《砭愚》二篇。这两篇是关学的关节要目,大纲一般的文字,韩冈都已经能背熟了。要想了解张载的学术观点,就得从这里入手。

见到韩冈在望着这座厅室,张载微笑道,“这一书院,多得玉昆之力。若不是玉昆你,也修不了如此堂皇之地。”

韩冈立刻站起来,垂手而立,“不敢。先生对于韩冈的教诲,难以报之万一。一点身外之物,当不起先生的谢。”

张载笑着示意韩冈重新坐下,“不必如此多礼。”他顿一顿,“玉昆,你今次过横渠,可是为了要上京科举?”

“学生正是要去京城考个进士出来,日后若能有所成就,也可为先生之学做个护法。”韩冈对自己的野心并不讳言。张载是君子,却绝非可以欺之以方。以师徒之亲,有话直说便可。

“若玉昆你当真能建功立业,那也是大善。若无朝堂上的支持,关学一脉,传承不远。”

张载非是慕于权势,但他很明白,没有权势的辅佐,任何学派都长久不了,也光大不了。要不然,夫子又何必游历诸国。

关学不似淮南学派,有王安石这个宰相撑腰,有整个新党的势力为后盾,未来的几十年,在士林中,传习王学必然是蔚然成风。除非有甚变故,让王安石名望尽丧。

关学也不似洛学。洛阳位于天下之中,大宋西京,文人才士咸聚于此。居于洛阳的两个表侄,能与富弼、司马光交游。他们在交往的过程中,必然能得到这一干朝廷重臣的宣扬。且两个表侄现今又在嵩阳书院中宣讲,传承数百年的嵩阳书院,不是草创不过两载的横渠书院可比。

关学在大宋学术界的地位,也就跟如今的蜀学差不多,偏居一隅,苟且而已。

为了能将关学一脉传承下去,张载绝不会矫情。

这几年来,张载一直多病,尤其是肺,是个治不了的病。现在看似没有大碍,但自己的身体自家最为清楚,并非药石可挽,只是拖日子,看看能不能多拖个几年。故而传说中的药王弟子就在身前,张载也没多问一句,甚至还要刻意离着看重的弟子坐远一点。

‘存,吾顺事。没,吾宁也。’

《钉顽》一篇中的最后两句,说得就是张载对生死之事的看法——活着,顺天应人,死了,只是安宁的时候到了。

对于生老病死,张载看得很开。他现在所挂念的,就是不想身死而道消。

孙复过世,泰山之学不之传也;胡瑗去世,世间再无经义、治事二斋;李觏病殁,盱江学派虽仍有流传,但也渐次式微。

张载不想看到他用尽一生的心力才开创的事业,因他的去世而变成陈迹。他还希望‘为天地立心,为生民立命,为往圣继绝学,为万世开太平’这四句,能随着他的学派而发扬光大下去。

太上立德,其次立功,再次立言。

将著述留于后世,只是立言而已。但若能让后世儒者传习大道,便是立功、立德的大功德。

诸生之中,以韩冈年纪为幼。说到传承关学一脉,就算从年龄上,韩冈都的确有这个资格。而且‘欲以旁门近大道’这句话,他也是当真能说到做到。格致万物、究研物理,此一说别出心裁,已经远远不同于二程的理论,而是韩冈对自家之言的饯行。不过,张载还是希望韩冈能在正途上也同样多下一点功夫。

要想光大关学门楣,要韩冈本人有这份能耐,对经义大道都要深入钻研。推广学术的权势须有,但本身的学问也要深厚。须知学术才是根本,权势仅是辅助。

只是现在的张载,对于韩冈所说的格物致知的理论,也是迫不及待的想要更深入的了解。从韩冈在信中提到的初步成型的几条理论,就已经可以看得出来这一套学说规模之宏大,意义之深远,自然万物的运转之道即囊括其中。如果能顺利的创立,并融入关学之内……

大事抵定矣!

朝问道,夕死可矣!

一起吃过饭,张载不顾夜色已重,连同三吕、范育、苏昞几人一起,拉着韩冈到了书房中:“玉昆,你且将前日在信中提到的力学三律,再与为师细细说来!”

