\section{第44章 本无全缺又何惭(中)}

“你能这样想那是最好。”韩冈很满意李德新的回答。过去的旧怨的确没必要留着。

因为当年窦解做的蠢事,窦舜卿被连累的夺官两阶,不过他在军中势力深厚,用了七八年又升回来了,还求了天子的恩典,从武职转为文职。如今正在京城中,担任宫观使的闲差。

尽管品阶还是正三品的高官,但对现下正当红的名医李德新,他连句重话都不会敢说。而对于韩冈,窦舜卿更是只有躲着走的份。韩冈估计过些天他就该自请致仕,回相州老家去了,这京城可不好再留。

“过两日你认祖归宗,到时候收拾收拾,回家里去住。兄弟之间,不管过去有什么事,如今也不要放在心上了。”韩冈又说道。

李德新是铁面相公李士彬的幼子,当年来认亲,却被他的几位兄长给赶了出去。但以李德新现在的名望,不需要韩冈说话,他的几位兄长都是主动上门求着他认祖归宗。

“小人明白。”又叹了口气,李德新遗憾的说着,“若是仇师能来就好了,小人一直都想侍奉他老人家左右。”

天子本欲给李德新赠官,李德新本来打算推辞这份官职,请求转给他的老师仇一闻。仇一闻在河湟拓边时就有了一个医官的身份,但不是正经的官员。如果有了正式的官身,仇一闻的晚年也能过得更好一点。

可惜李德新被授予这份官职的理由,是朝廷承认了他李士彬儿子的身份,将荫补的官身给了他,而不是种痘的功劳——要是奖励种痘,不可能跳过韩冈——荫补的官职,只可能让给族人,不能让给外人,这让李德新很遗憾。

“你有这份心就够了,仇老将你当儿子看,看到你如今功成名就,会比自己做官更高兴。”

“都是龙图提携。”

“你若是没那个能耐,谁提携你都没用。”这些感谢的话,韩冈听得也厌了,“话也不多说,你下去歇着吧。过些天专门负责厚生司就要成立了,里面少不了你的一个位子。到时候有的是要忙。”

韩冈就这么闲了下来,上门拜访他的并不少,多是想帮家里的儿女早一点能种上牛痘,让李德新也一通忙活。而且也不知为什么,经常的就有人写了短笺,让仆人一起连着礼物送来,然后就等着韩冈写回帖。

得到礼物,按道理就该写回帖。就是不收,论礼节也得写回信感谢。一天总有十几起,两三天下来,韩冈就觉得有哪里不对劲了。叫了随身的伴当出去打听。

他派出去的伴当,是个很伶俐的一个小伙子,才十七,其父伤了腿后投入韩家,连姓都改了,算是韩家的家生子。他名字很威风,在忠孝仁爱礼义廉耻八个字都用完的现在,代表着另一桩美德。

“龙图,打听到了,打听到了。”没过多久,活猴子一般的伴当跑了进来,“外面都说龙图的签名画押有安胎的功劳,只要烧成灰服用下去就行了。”

“啊?”韩冈就这么愣了半天。

自己在民间被神化的确是在他的预料之中,包拯能日审阳夜审阴,韩琦病逝,大星落于庭中,他韩冈做了那么多事,跟神仙挂上钩根本不出奇。但变成他的签名画押能安胎,这就未免太可笑了。

“龙图。”伴当嘻嘻笑着,“小人的浑家也快有身子了,看在小人一向勤谨的份上,赏给小人一个压宅的吧!”

“年终封红包,你那一份看来是不要装东西了,只要我签名画押就够了。是不是啊?”

伴当涎着脸笑道:“龙图是天上星宿下凡,签名画押都是能镇邪的。有了这宝物,没钱也甘心啊。”

韩冈无奈的摇了摇头,他待家人一向宽和,倒是不怕跟自己开玩笑。

麻烦事的确不少,天天都要回好几十份拜帖,但来拜访他的高官却不多。

与他交好的章惇正被御史台弹劾,只好杜门不出。可能是受了韩冈的牵累也说不定,在弹劾韩冈被天子阻止之后,他们的精力就转到了章惇身上,看风色,不久之后章惇很可能要出外。

而另一个关系不错的苏颂,担任权知开封府的他则是陷于一桩人伦大案而无暇分身,说实话,弑母案在这个以孝治天下的时代,的确是很耸人听闻,又是事涉宰辅,牵连更多。

其他人则基本上都知道韩冈现在的情况,暂时都在观望风色。

韩冈对此则是付之冷笑。

他的行事的确触到了天子的逆鳞,不过天子眼下唯一的一个儿子却也攥在他的手中,这可以算得上是人质,没什么好怕的。

而且只要在赵顼看来,自家的亲弟弟的威胁性比他韩冈更大就可以了。话说回来,他韩冈一不带兵、二不秉政,能威胁到皇权的只有雍王赵颢。

因为当年之事,韩冈与雍王接下的仇怨很深。基本上可以这么说,如果雍王登位,韩冈必死无疑,不论是被病死、被自杀、还是别的原因,肯定是一个死字,全家也都不会有好下场。

故而所有朝臣中,最热心保护皇嗣的只会是他,最希望赵顼的儿子继位的也是他。这样的人,又怎么会故意拖延不上缴有效的医方,最后让皇嗣不得及时医治而死?只会是为了能让医方更加安全而努力。

这么简单的逻辑关系,会有人想不通吗?也许吧。可亲手将弟弟赶出皇宫的赵顼要是想不明白,未免就是个笑话了。但他心情总不会好就是了。

不过只要皇帝不认为自己是心存恶意,他的心情好不好,韩冈倒也不会太在乎了。自命正直的士大夫有几个会在乎天子的心情?做皇帝的也该习惯了。

韩冈有的是正经事要做。

在他进京后没两天,就传出消息要将原本合为一路的京西转运司重新分为南北二路,到时候,韩冈的京西转运使之职自然也就不复存在。

这件事很快得到了确认。赵顼都向韩冈询问了他对分别在洛阳和襄州的两位转运副使的能力和品行的看法。基本上可以确定,因为在襄汉漕运上的功劳,他们将各自被扶正,分别统辖京西南路和京西北路的转运事务。

只不过襄汉漕运不归他们管。连同需要继续开挖的方城山渠道,新成立的襄汉发运司负责所有与襄汉漕运有直接关联的事务。不过眼下纲运因为隆冬而中断,发运司据说要到明年元月才正式成立,但从京城的传言中,第一任发运使多半会是沈括,而方兴、李诫都会在其中任职。

至于韩冈,他的差遣就很难定夺了。出外是不可能的,但在京城中,以他的官阶、贴职、资序,在两府中任职早够资格了,翰林学士这一级的两制官更是绰绰有余。不过韩冈不指望自己能有什么实权差遣,肯定是要消停个几年再说。

翰林学士是文学高选,就是不兼知制诰,也即是不用草诏的不在院学士,韩冈也不好意思去做。担任宫观使养老领干俸,韩冈更是还不到年纪。他估计自己多半是去崇文院,读书看报整理文牍档案,磨个几年性子。

这也是韩冈所希望的。之前的十年,他走得太快,正好可以停一下,歇歇脚,培养一下根基。而时间多了,也正好放在学术上。

因为自己依靠格物收获了可以免疫天花的牛痘,再加上《桂窗丛谈》的付梓,肯定能引发一股对自然科学的流行性的热潮。只要趁热打铁,当能将根基动摇的气学给稳定下来。

天子并没有让韩冈多等,他的差遣很快就被定了下来。

只是结果让韩冈很是惊讶,卸下京西转运使是在情理之中,但转任的职司却是群牧使——统管天下马政的群牧使。

该不会是前两天在崇政殿拿来作比较时给天子的灵感吧?韩冈惊讶之余不禁这么想着。

群牧使为群牧司长官,专领本司公事。大事与群牧制置使——此一职位由枢密使或副使或同签书枢密院公事兼掌,现在做兼职的是吕公著——同签署,小事遣副使处理,余事专决。

这可不是一个普通的官职,与掌管财政的三司是平级的中枢机构,并不归于中书门下管辖。许多时候,担任群牧使的官员还能兼一个翰林学士,与参知政事只有一线之隔。现任参知政事元绛就是从群牧使任上被提拔上去的,当时他便兼着翰林学士一职。

这是个很轻松的衙门,在新法将手触探到马政之后,许多的原本属于群牧司的职权,转到政事堂的手中。使得群牧司只剩下监查诸马监和掌管茶马互市的权力。

这也是个很有油水的衙门,‘三班吃香,群牧吃粪’的俗语,就源自于可以贩卖粪肥充作小金库的群牧司。

不过群牧使这个职位,并不是很有权力的位置。‘余事专决’中的余事,基本上可以解释为无事。大事是枢密使吕公著来处理,韩冈签个名画个押就可以;小事是群牧副使的职权,韩冈签个名画个押就可以;至于‘余事’,如果有的话,基本上也就是签个名画个押就可以处理了。

所以说,夹心层的官儿不好做……

进冰箱了,韩冈想着。

但胜在轻松,他倒是蛮高兴的。

