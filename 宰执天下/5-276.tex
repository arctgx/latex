\section{第30章 随阳雁飞各西东(七)}

韩冈主动揭破了辽兵压境的消息,却把主动权攥在了手中。而吕惠卿出任宣抚使更是给萧禧的强力一击,即表现了绝不退缩的意志,更证明了大宋并不担心失败的可能。

韩冈倒要试试看,萧禧到底有没有胆子一硬到底,还是说他有把握,耶律乙辛愿意将战事由此升级?

不管怎么说,萧禧也只是一个讨价还价、卖嘴皮子的使者,打掉他背后的支撑,就像剪掉了悬丝傀儡上的绳线,剩下的,就只是一截截竹子木头而已!

不过萧禧处理宋辽外交早就是行家里手,并不为韩冈的言辞所迫:“若贵国要破盟,鄙国绝不畏惧。若说鄙国故意背盟,在下也绝不敢妄自承认。是与非,不是内翰向在下骂上几句就能定下来的。至于河东,在下倒是只记得六年前。”

“林牙也是大宋的老朋友了,还望能坦诚一点!”韩冈忽然转怒为笑,剧烈的变化让人怀疑起之前的愤怒有几分是真几分是假,韩冈则并不在乎,有些时候不要脸皮反而能得到更好的结果,“就像鄙国使臣,出使前天子必有交代。林牙南下前,贵国尚父也必然有所嘱咐。林牙以正旦使南来,若只有这个差事,那韩冈就没有别的话好说,正旦之后,送林牙北返便是。若另有所图,还请将那一份国书拿出来,不知事前是否已经准备好了?”

萧禧终于是愣住了。他虽以正旦使的名义南下,但本质上还是要跟宋人讨价还价。实际上过去也莫不如此,萧禧曾以生辰使南下,后确定了大宋内部不稳,就一转开始索要土地。摇身一变,变成了‘泛使’——身负临时差遣的一般性使节。

但现在韩冈硬扣着他的身份,只要他说一个‘不’字,那么多半可以确认,宋人将只会理睬他为正旦使的任务,对其他言辞一概不理。过了元旦,便将他遣送回国。纵然萧禧他还可以照过去做的那样,硬是留在馆中不走——宋人也不可能强行驱逐——但只要不接触、不交谈,那就还是没办法。

可要说‘是’,那不就是自己打自己的脸,将前面喷出来的口水,一点点的从地上舔回去?若是这么认了,弱了气势,可就没法儿谈了。

一时间,他左右为难。

韩冈一见占了上风,便更加咄咄逼人:“西北之事,是否是贵国尚父的谋划?若是贵国尚父不知,那兴灵之地乃是妄自兴兵,鄙国不介意为贵国灭掉这群乱臣贼子。若确为贵国尚父的谋划,那鄙国也只有反击一途。”

“看来不论是萧禧怎么说,贵国都已经认定了西北之事的缘起,乃是曲在鄙国?!”萧禧被韩冈挑起了火气,一时间都忘了自己已经避讳改名的事,“内翰所言两种情形,届时都要与鄙国之兵厮杀到底,不知有何区别?!”

“若只论西北,自然是没有区别。但对于宋辽两国,却是截然不同。这关系到澶渊之盟是否应该存续下去!”

“贵国朝廷打算废弃澶渊之盟?此事易耳,只消说一句明年断了岁币就够了!”

“还请林牙听分明了!”韩冈扫了副使折干一眼,视线又回到萧禧的脸上,“韩冈问的是贵国尚父的想法!”

“萧禧奉朝廷之命南来,全权在我,此便是尚父之意!”

“很好。”韩冈点头,又看了折干一眼,然后道:“明日林牙上殿,还请如此说来!……不敢耽搁林牙休息,韩冈告辞。”

话声一落,他便转身而去。

只留下萧禧,

……………………

天越来越冷了,来自于北方的风也越来越激烈。

种建中站在盐州城的北城上,迎面而来的风卷着沙土,劈头盖脸砸来,但他也不过稍稍眯起了眼睛。

盐州城的风沙里,本只带着来自盐池的咸味,但如今则有掺进了更多的血腥气。

每到战事将起时,种建中总能从空气中嗅到一股浓浓的血气。

大战将要开始了。

不知溥乐城那边的情形如何了。种建中眼望北方,却担心起就驻扎在西面百多里外的堂兄弟来。

宋辽瓜分西夏后,种谔便被任命为银夏经略使。种朴由于是种谔的儿子,不方便留在银夏任官,却是给调去了环庆路的韦州。而种建中倒是得以留在了盐州,盐州知州兼西路都巡检。

论起距离,两边相隔的并不远。当初徐禧加筑盐州城墙,环庆路的民夫,就是从韦州过来。不过两州各有各的上司,分属不同的经略使路。想做到守望相助都必须征得后方的同意。

如今溥乐城被围,种建中想领兵救援,却平添阻碍,到了现在也没能离城一步。

“都巡。太尉有命,速至白虎节堂。”

背后的声音惊醒了种建中,“知道了。”他十分简短的回了一句。再多看了北方的茫茫沙原一眼,便转身下城。

就在昨日,种建中的顶头上司,也是他亲叔叔的种谔带着百十亲兵从夏州无声无息的进抵盐州。这一位太尉的吩咐,种建中绝不敢耽搁。

到了白虎节堂,种谔俯首正在沙盘边,听见动情,却头也不抬,只是问了一句,“来了?”便继续看他的沙盘。

“太尉,溥乐城那边……”种建中欲言又止,这些日子以来,已经为此事争辩了好几次,但每次都被种谔训上一通,但他还是想说。

“玉不琢,不成器。”种谔抬起头来,如石雕铁铸的面容没有一丝动摇,“十七若撑不过去,那就是他的命。撑得过去,那才能成大器!”

“五叔!”种建中叫道。

“你们兄弟几个从军也有十几年了,何曾吃过苦,又有几次在生死之境上挣扎过?不趁现在锻打一番,难道还要靠我、你爹,还有你的叔伯再撑上几十年?!种家的门户终究还是要靠你们撑起来,没个好身板怎么撑?!”种谔冷然说着,“十七是你兄弟,可别忘了,他更是我儿子。”

种建中无可奈何:“侄儿明白。”

种谔又瞥了侄子一眼,低头再去看沙盘,眼神也渐渐变得兴奋和狂热,最后他一拳捶在沙盘边。

“辽人不来则罢,来了就别走了!”他的语气森然,“区区三五万帐,到兴灵也不过一年而已,不好生扎下根基,这么快就想南侵?小心我翻了面皮,将兴灵也夺下来!”

……………………

辽人大军南下了。

这是溥乐城主种朴十天前,接连派亲信向韦州和盐州通报的紧急军情。

若是说位于环庆路北方防线最前沿的韦州,其实防御辽人南下的第一道关卡。那么溥乐城,便是韦州北方抵御辽人的第一道防线。

其位于韦州的北侧偏东,控扼辽人南下的主要通道灵州川。之前韦州边境上的几次冲突,大部分都发生在溥乐城附近。

十四人死,二十一人伤,还有八人失踪,这是到辽人南下之前为止,种朴手下斥候游骑们的全部伤亡数据。

能在溥乐城中成为一名斥候,无一不是可以以一当十的精锐骑兵,但在与辽人越来越剧烈的冲突中,仍是不断的受到损失。

要运回乡里的,在火中烧化,只留下骨殖做纪念。但更多的,则直接埋在了溥乐城边的墓地中。

所以在半个多月前,他设计埋伏了一支契丹人的骑兵小队。但随之而来的发展却让他失去了炫耀这份功劳的想法。

辽人竟然群起而动。数千辽军,在溥乐城北方札下营盘。看似是在围城,其实向南留下了很大的缺口。是造声势逼迫城中守军自行行动,甚至希望他们能南逃。

望着城外随风招展的旗帜,种朴心情更加阴郁。

每一名士兵的伤亡,都在挑战着种朴的自制力。之前伏击辽人成功,的确有了一点赏赐,但更多的,还是私下里的训斥!现在他是在勉强压制自己的愤怒,却也不知什么时候就再也压制不住。

种朴运气不佳,另外也是少个文官的出身,比起官运亨通、中过明法科的堂弟来,眼下仅为环庆路第七将的正将,环庆北路巡检使,镇守在溥乐城。麾下兵马三千五百,实际上有三千出头的兵力,大约军籍簿上八成五多一点,算得上是精锐了。

在北方七十里外,与溥乐城遥遥相对的,是辽军盘踞的耀德城。这两座城砦,皆位于灵州川畔,是韦州通向灵州的道路上的中继点。在皇宋开国之初,党项人还没有占据灵州的时候,这两座城砦,维系着灵州城的补给线。

等到了李继迁叛宋,占据了兴灵、银夏乃至横山,溥乐、耀德变成了西贼南侵时,往来兴灵和韦州中途的落脚点。

而从一年多前开始,在西夏灭国之后,则是分别为宋辽两国占据。两城中间的位置,便是宋辽的边界。

耀德城从三天前起,便不断有一支支辽军骑兵从北而来,陆续汇入城中。据斥候们的回报,从装束看,其中有契丹人,有库莫奚人,有渤海人,甚至还有党项人——打了一年多的交道,辽人中的不同族类,斥候们倒是分得一清二楚了。

“城主,去南边的人回来了。”一名心腹小校过来轻声禀报。

“让娄七来见我,其他人下去休息。”种朴挥挥手,让人退下。这是他派去南方的一支斥候小队,目的是试探辽人的动向。

但种朴的心腹小校没有动,而是低声道:“他们撞上了一支一百多人的辽人斥候。”说着,他就从下面领上来一群士子。

种朴终于明白了,冷喝一声,“看来辽人是不准备过年了!”他看看回来的这几名斥候,却发现少了一个熟悉的人,“娄七呢,是受了伤?”

那名小校低下头,声音也同样低了下去:“城主,娄七不成了。”

种朴一听,转身便走。下城后,就直接冲到了疗养院的重病房,只见正中央的一张病床上躺着一个人。但这个人脸上盖着一幅白巾。身上满是未干的血迹。

缺少两根手指的左手露在外面,正是自己过去的亲兵娄七。将他派到外面做斥候,是打算让他立些功勋也好提拔,哪想到会这么没福气。

揭开白布看了几眼,种朴飞一般转过身来,脸上不见悲恸,只有愈来愈盛的愤怒。

“且去准备,晚上随我去杀上一番!”他阴森森的低喝道。这段时间以来,都只是反击而已——你来打我,然后我还手——还没有主动攻击过。之前的伏击,也是居于这样的想法。但现在,种朴决定改变,“拔掉辽人的几个营寨,逼辽人动手攻城。”

“城主,我们这里可不好主动出手啊!万一城外的辽人大军当真开始攻城……”

“怕什么?!枢密府中的相公,可不是文彦博!”

