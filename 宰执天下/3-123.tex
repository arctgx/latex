\section{第33章 道远难襄理(上)}

曾布背叛王安石,在饱受争议的市易法上反戈一击,其影响远比表面上的纷争更要深远。

这些天来,京城之外久旱无雨,朝堂上却是风雨大作。

原本除了一些外围的趋炎附势之辈,内部还基本上能保持一致的新党,终于暴露出难以弥补的裂缝来。

曾布的背叛,让很多人都认为是新法覆舟在即,所以王安石倚为臂助的心腹才会在突然间抛弃了新党。而且因为曾布曾经掌握新法的制定和施行,他提拔起来的底层官吏不在少数。他这一下起事内乱,让新党中挂着曾系招牌的官员变得无所适从。

朝中政局由此而变,尤其是在京旧党,对于曾布对市易务的指责如获至宝。一时之间,奏章交加而上,与曾布同声相应,同气相求。

韩冈身处漩涡之外,对于朝堂中事,无法做出准确的判断,仅能从京中传来的片言只语了解其中的变化。

王雱在信中,让韩冈安心做事,不要有太多的顾虑。而近两天,一些最新的消息,也让韩冈嗅到了风向急转的味道。

新党毕竟根基还在,王安石对天子的影响力犹存,而吕惠卿更非易与。当赵顼点了吕惠卿和曾布的将,让他们一同根究市易务违法之事时开始,京城中的局势就渐渐开始对曾布不利起来。

曾布追查吕嘉问违法之事,甚至追及到仍挂着三司使一职的薛向头上。但吕惠卿则直接从魏继宗着手,指称他曾为曾布辟为指使,诳言欺君,追着魏继宗穷追猛打,攻其一点,让曾布对市易务的所有指责全数成为空谈。

韩冈这边就没有那么多麻烦了,早前的周全准备,让他应对起蜂拥南下的流民来举重若轻。在一切都上了正轨之后,他就回到了县城,安坐在县衙之中。一干事务,自有得力的下属和幕僚来处置,他只管每天一探流民营就够了。

至于浮桥之事,倒也好办。有先例,有人力,开封府那边又有钱粮支持,天子对于韩冈的建议也从无驳斥之说。只是重造浮桥,事涉京畿、河北两地,以韩冈的权限自是不够资格跨越路界,但赵顼还是降诏让韩冈全权主持此事。

“也该如此,黎阳知县只是太子中允,京官而已。”方兴的言下之意,河对岸的黎阳县知县与韩冈那是差了十万八千里。

韩冈并不在乎这点职权之争,他关心的是京中的支持:“只盼朝堂诸公不至于忘了流民之事。”

尽日听到南面一百多里外的朝堂上,政局一日三变的消息,韩冈想着是不是要让王旁回京去提醒一下自己的岳父,不管曾布怎么可恨,旧党如何的攻击,目前最为重要的还是流民的问题。

主要矛盾和次要矛盾的关系,韩冈学得还是不错的。

市易务之事的确是要争个明白,但那件事决不是关键所在。市易法的动摇,不过是在堤坝上打个口子而已,但若是流民生乱,黄河大堤都要塌了。且一旦大股的流民抵达东京城下,那就是压垮骆驼的最后一根稻草,现在王安石、吕惠卿奋力保护的一切,全都要化为泡影。

有了诏书,白马浮桥很快就建起。

浮桥的结构简单,搭建起来也并不费时费事,当韩冈联络了黎阳县之后,用了五天筹办浮桥必须的绳索、船只和木板,接下来就只用了两天便将沟通黄河两岸的浮桥给建了起来

白马浮桥并不是一条绳子直接拉到对岸去,那样实在太长了,中间很容易出现因黄河水流而被冲断的情况。故而在中段有个周转,就是河中心的居山。

架在黄河中的浮桥分成两个部分,一段从汶子山下延伸到居山之中,另半段则是从居山延伸到对岸。

韩冈立于浮桥边,听过一片鼓乐响,加上噼里啪啦的一串鞭炮声,桥上的最后一片木板钉了上去。在河水中随浪起伏的浮桥,被水流冲出了一个弧度,摇摇晃晃的很不安稳。可比起渡船来,却是更为安全。

浮桥一通,徘徊于对岸的流民都拖家携口,顺着浮桥南下而来。韩冈在渡口处,望着一条人龙跨过黄河,抵达白马。县中的流民越来越多了,不知道什么时候朝廷才能有权限更高的任命——他手中权柄所能达到的极限就快到了!

……………………

大名府。

文彦博八子,或为官,或居乡,现在就只有六子文及甫跟在身边服侍。

文及甫现在的任务就是孝顺父亲,同时也是传达内外消息的包打听。他脚步匆匆走近文彦博的书房:“大人,黎阳津那边的浮桥已经建起来了!”

文彦博坐在书房中,读着一本前人笔记。和煦的春日从窗户中照进来,正映在书桌上。黝黑的桌案纹理沉沉,在阳光下泛着微晕的光芒。

大名府常平仓耗尽,府内流民尽数南下。如今文彦博也就轻松了许多,冷眼看着京中的笑话之余,也能抽空看看闲书,到了他这个年纪,经史典籍已经看不进去了,也只有些许杂书还有些兴致。

见到儿子回来,文彦博也不管什么浮桥,指着正看着的书卷上的一段文字,对儿子道:“昨日见朝中祈雨文,文字寡淡,殊乏余味,只可付之一笑,却难求得雨来。”

文及甫不知父亲怎么突然提起提着祈雨文,呐呐的停住脚,一头雾水的站着。

文彦博素知自己的这个儿子一向反应慢,也没有等着文及甫回话,继续道:“如今朝中文学之士,多以朴素练达为上,不饰文采,反倒让了王禹玉的金玉满堂占尽了风流去。就是王介甫,偌大的名气其实也是一般。要说道文字,本朝还是以违命侯为上。看看他做的祈雨文,只一句‘尚乖龙润之祥’,就将这一年来的祈雨文全压下去了。”

文及甫当然知道父亲说的是谁。大宋的违命侯只有一个,那就是南唐后主李煜。李煜的文采自不必说,能一篇词将自己的小命送掉的,也算是独一份了。只是他揣摩不出父亲究竟想说些什么。

尴尬的站了一阵子,文及甫想不出个眉目,只能点头,“大人说的是、大人说的是。”

文彦博无奈,抬眼问道,“黎阳的浮桥修起来了?”

文及甫头点得更频,他如今十分关心白马县的一举一动,“已经跟白马连上了。现在黎阳境内的流民全都通过浮桥往白马县去。”

文彦博一声冷笑:“他手脚倒快!”

“大人。”文及甫上前一步,郑重道:“只看韩冈奏请搭建浮桥,就足见他根本就不怕流民入境。再看白马县中如今尽凿深井浇田,而开凿深井的井师,竟然是从蜀中富顺监而来,可见韩冈对大旱已是早有准备,措置亦是有条不紊。”

“哦,是吗?”文彦博神色淡然的应付了一句。

文及甫自从被父亲教训之后,对韩冈的态度,从贬低一转就变成了凡事都高看一眼。韩冈的行事,文及甫总能从中看出奸谋和深意来。见父亲不为所动,他进一步说道:“富弼能在青州做的事,韩冈当然也能做。若他当真将流民安置妥当,日后说不定又是一个富彦国!”

文彦博则是一点也不担心,摇摇头,“要应对河北南下的流民,至少是一州一府之力才能有足够的人力物力。从去年延续到如今的大旱,不仅仅是河北受灾,京畿也同样受灾。试问白马一县如何能支持?”

判大名府的前宰相说着指了一指堂外,春日的阳光毫无遮挡的洒落于庭院间,“现在不过是开春而已,整个河北的流民也才二三十万。可等到五六月时,吃光了家中存粮、又没有新粮补充的百姓,将不啻百万。到时候,从河北两路南下的流民,可不是冬天时围在大名府之外的那么一点点。”

“大人,韩冈可是右正言!”文及甫提醒道,“要是朝中有人提议恢复滑州,韩冈足可担任。”

文彦博终于放下了手中的书卷,垂下的寿眉压着因阳光而半眯起的眼睛:“记得当初将郑州、滑州并入开封之事,还是曾布所首倡。现在王介甫腹心内乱,曾布反戈。说不定还真的让韩冈当上了滑州知州,只不过……那又如何?”

文及甫欲言又止,只听着文彦博慢慢的说道:“要想处理好几十万的流民安置之务,绝不是一人之力便能完成,需要足够的助手和威望。韩冈虽然才高,但他人望不足——无论手边的可用之人,还有震慑僚属的声望,都实在太少了……”

富弼担任青州知州的时候,已经在朝中积累下了足够的资望,能顺利压制住治下的知县们,而且当时富弼手上也有不少得力的幕僚,这才将一场大灾平安度过。五十多万流民,若只凭富弼一人,如何能做到?!

文彦博老于政事,见过的人才数不胜数,即便是治世之雄才也是见得太多,可有哪个能以一人之力,解决一州政事——都要有人作为帮手。就算以太祖之绝世无双,也得靠着义社兄弟的辅助,才能在陈桥黄袍加身。

文彦博他决然不信那位让他多次吃亏受辱的陕西士子,能有独力擎天之能。

“韩冈或有治国之才,可如今王安石相位难保,他即便当上了滑州知州,又凭什么来让下面的知县对他的吩咐一一依从?年纪太轻、资望浅薄的缺点就在这里!”

