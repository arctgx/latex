\section{第24章 兵戈虽收战未宁(一)}

【昨天夜里写着写着就睡着了,今天早上起来把最后一段补上了。这是昨天的第二更】

被一条如毒蛇一般难缠的敌人盯住,禹臧花麻已经无法再退,也不能再退。尤其是在一场消耗了大量士气和体力的战斗后,又经过了长距离的行军,如果再退下去,最终的结果就是不战自溃。而且为了自己的声望,他也必须取得一个说得过去的胜利。

这一点,一开始时,好像并不算难。收买了星罗结部,又跟木征一方达成了默契。以兰州和渭源之间的距离,禹臧花麻相信王韶不可能会着意提防自己。以有心算无心,在猝不及防的情况下,他当是能在渭源堡大赚上一笔。

事先禹臧花麻怎么也不会想到,他今次面对的敌人竟然这般难缠。他也没有料到王韶那么快就拿星罗结部开刀,让自己的计划一下落空。虽然趁着王韶分兵与星罗结城,又没有什么防备,禹臧花麻试图打下星罗结城,并作势围攻渭源堡,以便给部众、给朝堂,有个能说得过去的交代。

但两边的战局皆是劳而无功,辛苦了一场,却不得不在胜利即将到来之前匆忙撤退。一点回报也没有的战争,让随他出战的附庸部族的族酋们暗地里怨声载道,也让辖下部众向他投来不信任的目光。为了挽回眼下不利的形式,禹臧花麻也只能选择一战。

“花麻,你真的有把握?”禹臧花麻的身前,十几名族酋和长老们追问着,他们是禹臧家的实权人物,失去了他们的支持,任谁也坐不稳族长之位。

即便是身为族长的禹臧花麻,也不得不耐下性子向他们解释:“对于我们来说,的确不像汉人那么擅长攻城守城。但若是改换成野战,不知各位叔伯有谁会认为我们会输给汉人?”

没有人会承认自己的无能,暗地里交换了几个眼神,便一齐首肯了禹臧花麻的决定。一个老头子对禹臧花麻嘱咐道:“花麻,这一次一定要胜,禹臧家的名声可都靠你了!”

禹臧花麻诚恳的点头应下,眼神中却是一片阴寒。

就像关西绝大多数的山谷一样,大来谷中也是有着一条河流,是洮水的支流,从谷中一直延伸到临洮。不过这条河的源头出自于谷中的一侧山峰,所以禹臧花麻所在的大来谷东侧出口,并没有河道的存在。这让准备交战中的两方,有了一个足够大的战场空间。

吐蕃人就在大来谷口扎下了营盘,韩冈驱动了瞎药也来到了大来谷口。粗制滥造的营地,看起来一冲即破。不过在营盘之前,是已经列阵而出的禹臧军。

古渭之战是韩冈第一次亲自走上战场,今天,则是韩冈的第二次上阵。前次的古渭大捷,说起来本质上就是一次成功的乘火打劫。董裕已经被瞎药倒戈一击,内部乱做了一团,而俞龙珂的出现,对董裕军来说是百上加斤。失去了指挥全军的控制力,董裕就像摆上砧板的鱼,任人煎炸烹煮。

而今天,韩冈算是真正见识到了什么叫做严阵以待。望着一里外摆下阵势的吐蕃人,韩冈分外感受到双方人数上的差距。两边隔着一里多的距离对峙着,相对于围绕在禹臧家大纛周围,超过六千的军势,韩冈这一边看起来就弱小得太多。

数倍于己的敌军,真的拼起来并无幸理。韩冈已经命人在后方用马匹拖着树枝来回奔驰,搅起漫天尘烟,装出大军行进的模样,让禹臧花麻为之畏缩。

但对方并非蠢人,韩冈也明白,他这招数瞒不了多久。不过渭源堡应该已经收到了他的消息,就看王韶派出来的援军什么时候能到了——渭源离大来谷口不算远,半日即可到达,禹臧花麻就是怕被两面夹击,方才一听说瞎药出兵便匆匆撤退——现在关键的问题是如何在禹臧家的面前拖延时间。

“不如就按着方才的做法,攻来就退,走后再追。”乔四,也就是王舜臣手下的骑兵都头,向韩冈提了一条建议。对于这种骚扰战法,他方才已是食髓知味,还想再模仿一次。

“不成!”王舜臣摇头,“前面俺领着的几十骑,都是军中的精锐,又一起经历了大战,他们都信任俺的指挥,故而能如臂使指,来去自如。而三哥此次带来的蕃兵,却都是心怀犹疑,若是让他们忽进忽退,只要禹臧花麻在关键处推上一把,那就是兵败如山倒的局面。”

韩冈意外的看了一眼王舜臣,虽然他一直以来都清楚,他的这个兄弟仅仅是外表粗豪,实际上却有着内秀。但王舜臣现在能分析得这么透彻,却是他过去所做不到的,看起来自从做了官之后,日夜用功学习兵法,果然是进步了不少。

不过韩冈则笑道,“没有关系的,只要让禹臧花麻认为我们会如此做就够了。”

韩冈不算知兵,但王舜臣说的道理,他也是明白的。越复杂的战术,就越需要主帅和将士们之间的互相信任。只有上下一心,有着紧密的信任关系的军队,方能进退自如,无坚不摧。如果是没有牢固的信任关系,基本上就是能进而不能退,打不了硬仗。

韩冈不想去实验他有没有这个能耐,也不想在这个节骨眼上,去测试他在青唐部吐蕃人心目中的地位。但禹臧花麻肯定明白,他韩冈现在的目的绝不是作战,而是在拖延——拖延到渭源堡的援军到来。所以利用此前王舜臣留下的印象,让禹臧花麻以为继,为了让他们这样去想,韩冈接下去命令瞎药摆开的阵势,甚至都是以方便撤退为目的,而他也把将校们都招来,向他们解释自己的用心。

现在双方的对峙,实际上是封锁了禹臧花麻对渭源方向的侦查,禹臧花麻并不清楚,渭源堡究竟出兵了没有。这一个顾虑,就像一根绳索绑在禹臧花麻的脚上,让他不敢放下心来对付这边的千多人。

如果他敢杀过来,韩冈便会向星罗结城退去,“要知道,大来谷口是位于渭源通往星罗结城的大道的中段。从这里向北是星罗结城,向东南去,则是渭源堡。如果我们退向星罗结城,禹臧花麻是追还是不追?”

如果追,大来谷口就很有可能会被渭源堡出来的军队封锁。如果不追,只是赶走了就回返,韩冈就能整顿军队再杀回来。虽然还有分兵追击这一条选择,但禹臧花麻有几分把握敢确定,通往星罗结城的道路上,没有韩冈设下的伏兵?要知道,星罗结城本就有千人左右的守军,谁知道会不会埋伏在路边?他能分出多少兵力来?

“而事实上,方才本官也已经派人去星罗结城传令了,让城中守军出来择地埋伏,以防万一。如果禹臧花麻真的分兵过来追袭,这一份大礼,我们也却之不恭了。”

韩冈的一番分析说得鞭辟入里,正反两个方面都考虑了周全,自王舜臣、瞎药以下,将校们纷纷点头称是。

而下一刻,号角声响起,从禹臧军阵中分出了一部人马,看起来一千五到两千人的样子,杀气腾腾的直扑过来。

“想不到禹臧花麻不智如此!”韩冈放声长笑,“就按本官方才所说,安抚住军心,把他们引到我们埋伏的地方去。”

王舜臣和瞎药率领一众将校齐齐躬身受命,韩冈高居马上,生受了他们的这一礼。运筹谋算,此刻,他也有了一份主帅的威严。

全军掉头回撤,由于韩冈事先做好了准备,千军万马的隆隆撼地之声,便显得有条不紊,退而不乱。不过为了引诱来敌继续追击,在韩冈的命令下,他们便把军旗和用不到的军械,还有影响马匹速度的干粮、盔甲,一点点的抛下去。摆出了一副丢盔弃甲的狼狈模样。

韩冈处在全军的护卫中,低着头,纵马狂奔。对于禹臧花麻的分兵,他心中还是有些疑惑,暗道自己难道是高估了禹臧花麻的头脑。

不过靠着父祖的恩泽忝居高位的废物姑且不论,在弱肉强食的蕃部之中,能坐上族长之位的,没有一个会是蠢货。按说禹臧花麻虽然围攻渭源堡和星罗结城皆失败,但他知进知退,却没有受到什么损失,这份眼光和决断不是蠢人能有的。

虽然想不通禹臧花麻究竟在打什么主意,韩冈还是暗自庆幸,幸好方才他没有把话说死,不然现在就不是撤退,而是逃窜了。

一口气追出了七八里,双方的速度渐渐缓了下来。追出来的蕃人也在顾虑着是否有伏兵的存在,不敢追得太急,而且两边都是因为之前长时间的行军,战马的气力消耗了许多,无法再保持高速。

不过韩冈让人丢下的东西越来越多,王舜臣带出来一队骑兵,都把身上的盔甲分解开来,一件件丢下去。这样的收获,让后面的追兵难以割舍,紧咬着不放。

而韩冈设下的伏击圈,已经近在眼前。

