\section{第45章 千里传音飞捷奏(上)}

“前面就是狄道城了!”

蔡曚在马上遥指着,吕大防顺着马鞭所指,望向前方。

但吕大防并没有看到狄道城,不知是出了何事,前面竟然也是一片尘头,正与自家的队伍相对而来。在烟尘的阻挡下,吕大防不知蔡曚是怎么看清的狄道城的模样。

“终于能见到那个奸猾之徒。韩冈为人狡诈,素性狂妄,今次抗旨不尊、伪传诏令,定然不能轻饶了他!好歹也要让他去乌台大狱走上一遭。”

蔡曚咬牙切齿,吕大防冷冷的看了他一眼,却并不接口。不过吕大防的正职是殿中侍御史,韩冈做下的事也不能不报上去,否则也是他的失职。

两人在陇西城听到的消息,韩冈不仅仅是将李宪传诏给顶了,更是伪传诏令,将天子要求退军的旨意,变成了鼓励众军进兵的奖誉,胆子不可谓不大。

蔡曚从吕大防嘴里将此事证实后,上窜下跳,没有少宣扬。而听说了韩冈如此行事,吕大防心头也是不喜。换个情况,这是臣子风骨的体现。但韩冈今次的所作所为,在吕大防看来,却是一条路走到黑,不知悔改。

与他的三个兄弟不同,吕大防并不是张载的弟子。但对于张载门下的学生,多多少少也有些香火之情。今次的宣诏,他本不想接手,但好不容易将他推到殿中侍御史的位置上的那几位,却不容许他拒绝。

不过吕大防最后答应下来,并不是因为有人催逼。如果真的从心底里反感,直接辞官就是。以他的脾性,根本不会受任何人的要挟。只是他真的觉得河湟之事不能再继续下去了,对国力的消耗实在太大,所以才点头下来。

吕大防曾经在陕西宣抚司中见过韩冈一面,虽然没有来得及交谈。但前前后后了解到的韩冈的情况,也当真是个难得的人才。就是与自己好像不是一条路,行事偏向新党一边。不过尊师一向却做得极好,兄长吕大忠的家信中屡次称赞了他,不是个忘本的人,而且在学术上还多有开创。

从不同渠道得来有关韩冈的情报,在吕大防心中组成了一个让他难以理解的形象,行事、才学、为人、性格,都绝不是二十出头的年轻人该有的模样。

不过这都与他无关了。今次见到韩冈,是来做仇人的。如果能劝一劝,还是讲一下人情也好,若是不听劝,那就秉公处置就是了。张子厚和兄长那里,在处理公事的时候,吕大防却不会多考虑。

不过……吕大防扭头看着与己并辔而行、嘴巴正一张一合、不停歇的秦凤运判,微微皱起眉头,这蔡曚可真是个厌物。今天风向也不好,竟从身后刮来。要是刮着西风,当能让他住嘴。

终于与出城迎接的队伍汇合。

李宪显然是到了很久,看到吕大防,便走上来迎接。吕大防下马后,淡漠的瞥了他一眼,并没有理睬。

这些阉人插手国事,却尽是坏事,韩冈伪传诏令,他竟然给默认了。要不是自己跟着来,恐怕韩冈还会继续错下去。

而韩冈的模样却是刚刚赶到,身上还有浮灰尘土,而随行众人骑乘的战马,更是浑身上下都是汗珠。

吕大防不多话,也没有寒暄,而是拿出了随身携带的圣旨。许多事晚做不如早做,他直接就在离着狄道城十几里外的地方展开了手中的诏书,

“韩冈,接旨!”

来自于狄道城中的每一个人,都对着天子的诏书拜倒了下来。韩冈更是长跪,聆听着天子在诏书中的训示。

听着吕大防,蔡曚越来越是得意。蔡延庆正在忙着为赶去德顺军的秦凤、泾原两路的援军筹划钱粮军资,便把这接收之职交给了自己。

急不可耐的等到吕大防终于从起头的‘门下’二字,将整卷诏令念完,秦凤转运判官立刻提声叫道:“韩冈!还不接旨!”

狄道城众人一片鼓噪,但韩冈却回手阻止了随行者的喧闹。跪伏恭声:“臣遵旨!”

接过诏书,韩冈站起身。

蔡曚更上前一步,“韩冈,还不将印信缴上来。”

吕大防一皱眉头,提声道:“运判!”

李宪同样心头不快,而身后又掀起一片吵闹声,仍是韩冈回头一眼给瞪了下去。

蔡曚却不理会。韩冈既然接旨,就没吕大防的事了。他蔡曚现在是着熙河路的主管,没有必要听别人的插嘴,更不用在乎下面小卒的鼓噪。他摊开手,强硬地问着:“印信呢?”

韩冈面无表情,从腰间的印囊中掏出一枚数寸见方的铜印来。

蔡曚摊着手,等着韩冈将经略司大印放到掌心,他很享受这个时刻的快乐。翻手一看印文,他终于笑了一笑。抬起眼,冷起脸盯着让他丢人现眼了半年多的死敌:“韩冈,你且回去待罪听参。抗旨不尊,伪传诏令,须饶你不得!”

韩冈却是笑了,如同猫儿看到鱼上钩的笑容,“先得让韩冈向御史和运判介绍一下随行的几位将军再走不迟。”

“不必了!”

蔡曚硬邦邦的拒绝,韩冈却不加理会。

拉过身后正怒瞪着蔡曚的虬髯的矮个将校,韩冈向吕大防介绍着,“这位是熙河东路都巡检王舜臣,是今次临洮堡一役的主将。”

吕大防一听,连忙追问:“临洮堡已经解围了!?”

韩冈道:“临洮堡大捷。虽然西贼有马逃得快,但还是斩首两百六十余级。”

“这又如何!?”蔡曚厉声呵斥,“韩冈,你还想罪上加罪不成!”

不过是临洮堡赢了而已,有什么好絮絮叨叨的。大局已定了,还想垂死挣扎!?蔡曚心下冷笑。

韩冈却仿佛没听到,让出了身后的另一人:“至于这位……”

吕大防和李宪看过去。黑黑瘦瘦,脸上胡须乱蓬蓬的,身上的衣袍都是有些破烂。

只看这个破落汉子上前拱手:“末将王惟新,在王经略帐下听候使唤。见过御史,运判。”

‘哪个不是在王韶帐下听候使唤……’蔡曚更是不屑的一撇嘴。

但吕大防却惊得手都抖了起来。李宪更是抢先一步叫道,“王韶……可是从王韶那边来!?”

韩冈笑了一笑:“王惟新是刚刚从洮州回来的,只比御史早了半个时辰。”他再向李宪歉然一礼,“不及知会都知,还望恕罪。”

李宪哪还会怪罪这些小事,另一边的蔡曚,终于知道不对了,身子也更着抖了起来,眼睛不眨的盯着王惟新。

“王子纯赢了吗?”吕大防慢慢问着。

王惟新挺了挺胸,难得的抬头与官位远远在他之上的文臣对视着,“回御史的话,王经略、高总管领军穿越露骨山,行程千余里,大小数十战,如今已经收复洮州蕃部四十三家,总计两万余帐,人口、牛马一时难以计数。”

“木征呢?!”李宪厉声问着,看他的模样,是恨不得揪起王惟新的衣襟,把想知道的消息给逼出来。

王惟新用着更大的声音回复道:“好叫御史、都知,还有运判知晓。木征被我官军逼得穷途末路,已然自缚出降!”

话声未落,周围的人群中就是一片爆然响起的万岁、万胜的呼声。方才在城中已经欢呼过的人们,又再一次欢呼起来。

韩冈瞥了张口结舌、脸上还挂着一副呆滞表情的蔡曚一眼。他自重身份,讽刺的话没说出口,但谁都知道韩冈这一眼究竟是什么意思。

——蔡运判,你还有什么说的?

蔡曚说不出话来,但韩冈最终还是有话要说。

“好了。”他拍了拍手,对蔡曚和吕大防说道:“西贼已退,木征归顺,河州平复,洮州降伏。数年心血,也终于有了结果。接下来,就没韩冈的事了。”

他看了一眼蔡曚兀自捏在手中的印信,那是王韶连同一路重任一起转托给他的。而他韩冈在交出去之前,并没有辜负了王韶的信任。

笑了一笑:“下面经略司中之事,就交由蔡运判来处置了。韩冈前日抗旨不遵,伪传诏令,也该回去闭门待罪。”

木征即已降伏,河湟大局已定,形势不可能再坏。或者说,就算蔡曚倒行逆施,也坏不了现在的局面。既然如此,韩冈干脆放手,正好他还嫌没时间读书,这道诏令来得正是时候!而王韶的捷报,到得更是时候!

吕大防看着韩冈的作态,却没说话。他知道这是他兄弟的小师弟就此发泄,但韩冈做得的确没有任何可以指摘的地方。人家接旨、待罪,都是理所当然的,又有什么地方能指责他?

只听得这位让吕大防也看不透的年轻人一声长笑,“韩冈待罪之身,恕不能接待了,还望勿怪。”

利落的翻身上马,一串轻快蹄声便渐渐向狄道城而去。

吕大忠望着一人一马远去的背影,又看了看仍楞得如土偶木雕一般的蔡曚,仰天摇头一叹:“世事难料啊!”

“……世事难料!”李宪同声说道。

