\section{第17章 籍籍人言何所图(中)}

被契丹人给惦记上的事情,韩冈无从得知。

不过就算他知道了,也不会放在心上。要对付敌国的大臣,就跟老鼠给猫儿戴铃铛一样困难。就算萧禧有心要害自己,能动用的人力、手段,都极为有限,甚至排不上用场。

他要操心的事还很多。不过他已经为自己手上的工作找到了下家,“这里的事都要劳烦伯绪了。”

“下官分内事,不敢称劳。”苏子元说了一句,急急的就要出去。

“令嫒可还安好?”韩冈随口问着。

说起女儿,苏子元的脸上就多了一些笑容,“劳运使挂心,小女近日已经好了许多。”

韩冈对苏子元的小女儿很看好,也想给自己的儿子订门好亲。不过苏缄刚刚离世,自己赶着提亲上门,实在是有乖人情。还是等个一年半载再说。

说起子女,他出来的时候,云娘和周南都有了身孕,不知道现在的情况怎么样了。如果一切都安好的话,再过半年多,他就能又多了两个子女。

章惇看着苏子元离开:“看来邕州知州只有苏伯绪能做。”

“这是当然,这里还有其父的恩荫在,也只有他做得。”

“对了,玉昆。你听说了没有,交趾贼军能破邕州靠的是一名汉人出的主意。”

“听说了。”韩冈点着头,脸也冷了下来,“不过此等贼子,只要官军压境,一封信就能让交趾将人给交出来。到时候千刀万剐,明正典刑那是不在话下。不过最可恨的还是交贼贼性不改,回程的路上竟然还敢杀人放火。”

李常杰领着最后一批交趾军渡江返回国中,是顺着当初宗亶的来路,而就在他回去的路上,顺道将永平、太平等一路上四五个寨子中残留的百姓全都给杀光了。

原本宗亶领军攻打邕州的时候,虽然抢也抢了,杀也杀了,而且还放了火,但没有做到连根拔起的地步,有邕州在前面等着,都无心浪费时间,几个寨子中好歹还留了些人下来。

只是日前交趾败军回师路过这些寨子的时候,却毫无顾忌抵达来个斩草除根。将还没来得及逃散得百姓。几个寨子加起来有几千人之多。

章惇的眼神也变得冰冷起来,“看来还是的依着玉昆你的说法,以琼玉相报,‘永以为好’!”

……………………

结束了一天的工作,吕惠卿带着满身的疲惫回到了府中。

早间崇政殿议事,为了几条敇令,他与吴充、冯京好几次顶了起来,最后的结果是押后再议;而等到了下午的时候,吕惠卿又突然发现自己在公廨之中,也同样少不了与人交锋,王珪和冯京都不是省油的灯,如果有一点倏忽,就是难以挽回的结果。而到了傍晚散值后,他又去了王安石府上,直到二更天方才回来。

“大哥,你回来了。”吕惠卿回到书房,唯一还在京中任职的弟弟吕升卿正在等着他。

吕惠卿点了点头,一下就坐在自己的位置上。手撑着头,满脸的困倦。虽然要保持着宰执官的气度,但身体里的疲劳怎么都遮掩不住,眼袋都出来了,下眼皮泛青,一眼知道看得出他已经有好几天没有好好休息了。

“大哥方才是去看了王相公吧?明天还能不能上朝?”吕升卿问着。

“多半还不行。”吕惠卿摇摇头,“今天去了相府探视过,虽说是差不多快好了,但还要歇上两天才行……就是多说了几句邕州的事,才拖到了现在。”

“邕州大捷的消息,王相公当是昨日就该知道了吧?”

吕惠卿道:“前天夜里,天子就让人将捷报抄送去相府里了,毕竟是翁婿。”

“韩冈的捷报写得也是有趣。”吕升卿冷笑着:“朝廷调了两千兵,韩冈在捷报上却说是一千五。这空饷之事就这么捅了出来,韩冈就不怕他麾下的那些指挥使,因为这样的小错而丢官得罪?。”

“军中空饷的事哪个不知道,只是装聋作哑而已。韩冈敢这么写,是他有恃无恐,一场大捷,吴充冯京都不敢在这时候触楣头,谁还管这些小事。”韩冈越来越会在奏折上做文章,这让看着韩冈从九品选人做起的吕惠卿感慨万千,“一千五与两千之间的差距,可不仅仅在那五百人。一个是以‘一’开头,一个则是以‘二’起头。两边给人的感觉不一样!一千余人大败十万贼军,两千余人大败十万贼军,差得很远,看在天子眼里评价也是差得甚远。你去问问外面的百姓,那种说法更合他们的口味。”

“还不如写八百人呢……”吕升卿悻悻然的说着,“前面几仗不都是两个指挥的官军加上几千归降的蛮兵打得吗?八百破十万啊,不比千五破十万要响亮得多?”

“那章子厚还不得跟韩玉昆翻脸!”通过联袂南下的这一桩事,吕惠卿已经很确定章惇和韩冈之间有着盟约,而且关系紧密得超乎他的想象。“不把章子厚派来支援的最后两个指挥的功劳说得大了,这一次邕州大战哪有他分润功劳的份?韩冈日后有不少地方要联手章惇,哪里会吝啬几分战功。”

“……章惇算是捡了个大便宜了。”吕升卿莫名难测的神色中,说不清是嫉妒还是羡慕。

“章子厚派兵的时间也抓得好,这就是他的功劳。如果他与韩冈有隙,或是犹豫了一下,将两个指挥留在桂州,韩冈就算不败,也会大受损失。”吕惠卿长吁了一口去,“天子可是对章子厚的及时遣军南下一直赞不绝口,说此次大捷,章惇虽身在桂州,但其功不下于领军上阵。明天诏书就要下来了,章惇和韩冈两人推荐的官员,全都批准了。他们两个,也都有封赏领。”

“封赏?!都还没有派中使去确认过吧?”吕升卿歪着头,疑惑的问着,“哪有这么快就定下来的道理。记得当初的河湟和荆南,都是几次三番的派人确认战绩的。”

“广西走马已经确认过了。”吕惠卿从书架上上翻出一本书,百无聊赖的翻了起来,似是对这个话题不再感兴趣。

“广西的两个走马承受说的话哪里能算数!过去不都是宫中选一人、朝中选一人,两人一同出外审核?”吕升卿没有注意到吕惠卿的神色,只是沉吟在自己的推测中。不管从什么角度,他都觉得这一次的事实在很奇怪,里面的名堂也让他难以揣摩了,整件事透着怪异的氛围,似乎是天子和朝廷都急着要将此战的结果给确认下来,“到底在急个什么?”

吕惠卿叹了一声,放下看了几页都没有看进去一个字的书卷,沉吟了一番,最后他跟弟弟说了实话,“章惇和韩冈在广西的功绩真假问题并不重要,天子和朝堂都需要这个胜利,提振民心士气,也好让契丹和党项两边都别想再拿交趾之事做文章。比起韩冈在邕州的谎报军功些许小事,眼下最大的问题是罗兀城和丰州。所以捷报上的数字绝不会有人去追查。你在外面难道没听说吗,今日宫中天子可是对萧禧不假辞色,所有的要求全都拒绝了。”

吕升卿听后愣了半天,最后摇了摇头,“真不知道章惇和韩冈是不是事先想到会有这一出,才抢着南下……对了,他们两人升的什么官职?”

“章惇从龙图阁转为端明殿,韩冈则是升为龙图阁直学士,这两项已经是定下来了。”

吕升卿吃大吃一惊:“韩冈跳过侍制了?!他原本不是直龙图阁吗?”

“王雱都是升侍制了,韩冈的功劳难道还比不上他。”吕惠卿对韩冈晋升倒是并没有多少偏见,只是有些感慨而已,“本来韩冈就是因为年资太浅升不上去,功劳早就攒够了,只是一直被压着。现在立有如此殊勋,哪有不赏的道理。”

“王元泽那是给他冲喜吧……听说他的脚已经都不能动了,大哥今天没有顺便去看他?”吕升卿问着王雱病情的最新消息。

“没有。里面正好是陈安和在施针,就没进去了。”吕惠卿摇摇头。王雱自从去年上京开始,就一直有恙在身,时常告病。入冬之后,病势更急。天子送医送药,不过回来的人都说,基本上是没有救了:“不过就是陈安和当也救不回来,没多少日子了,前面几个御医回来后都这么说。说不定过几天,他能跟韩冈一起被提为直学士。”

“少了王元泽,章惇、韩冈又远在五岭之外,介甫相公身边也没多少可信用倚重的人才了。”吕升卿的脸上看不出喜忧,只是语调中有些怪怪的味道。

吕惠卿的眼神凌厉起来,但一下又变得平静无比。王雱是王安石的长子,也是王安石的助手,他在王安石身边出谋划策甚多。自从王安石第二次入京后,王安石身边定策之人,已经从吕惠卿变成了王雱。

“不过是少了个王元泽而已,还有韩冈呢。介甫相公若得韩玉昆襄助,三五个王元泽都比不上。”

