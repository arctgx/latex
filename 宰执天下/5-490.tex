\section{第39章 欲雨还晴咨明辅(19)}

韩冈心中很是觉得可惜,高层的空缺,不是随随便便就能有的,而同时还能在下面捞到几个位置的机会就更少了。

可惜韩冈现在手上最缺乏的,就是文臣朝官一级的中间层,否则他还能有些办法,通过向皇后控制几个重要的位置,将吕嘉问直接给架空掉。

高层有他本人镇着,苏颂算是自己人,沈括也勉强能算得上,章惇是盟友,蔡确之流也可以交换一下利益。而底层的选人,在关西有不少,通过同窗、同事的关系,勾连成一张网络。但不及侍制的升朝官,韩冈能够指派得上,又可以充分信任的,数来数去都不超过十个。

这个以韩冈为核心的小团体,现在远远比不上新党。

十余年变法,在京百司,路分四监、州郡、县治,由两千升朝官所组成的官僚体系的中坚阶层,已经完成了更新换代。旧党在地方上已经彻底被取代,新党,以及认同或老实执行新法的官员,占据了从知县、通判、知州等大部分亲民官的名额。

在京百司,更是没了旧党的立脚点。中书门下,枢密院,两府众官,不论是不是挂了招牌的新党,都不会有人站出来表示支持旧时法度。

就是吕嘉问,只要他想要人,随时都能挤出几个合适的人选供他挑选。而韩冈,便是手中有位置,也要左右盘算怎么将位子给填满——横渠门下,把稳守关中、陇右的一批人去掉,也不剩几个了。

势力现在还差得远啊。

不过以一个做官不过十年出头,本身又是寒门素户出身的官员,想要跟新党比势力,实在是有些可笑。但没有足够厚实的根基,韩冈的目标终究是镜花水月。再怎么可笑,终究还是要往哪个方向努力才是。

“想必吕望之现在是在笑吧。”韩冈心中计算了一阵,忽然轻声说着。

“嗯?”王旖没听清楚韩冈的话,手停了下来。垂下头,在韩冈耳边问:“官人,你说什么?”

韩冈给王旖的呼吸.弄得耳朵痒痒,用手搓了搓,“待会准备派人去跟岳父递个帖子,跟岳父说一下。三司那边我要两个位子。吕望之若是不干,为夫可就强抢了。”

“官人!”王旖吓了一跳,官场上面或许有将官职私相授受的,但哪里能做的这么明目张胆,“这行吗?”

周南的手也停了,仰起素净绝美的小脸,惊讶的看着韩冈。

韩冈咧嘴而笑,整齐的白牙露在外面,“想占我便宜,我可以让他占,但一点好处都不分,那可没门儿!”

周南依言去取纸笔,王旖却皱着眉头,“这不像是官人。”

韩冈从来都没有这么赤裸裸的去抢官位,以前纵然跟人争执,目的都是为了能将事情做得更好,而不是为了一两个位子。这样的韩冈,给她的感觉很陌生,行事作风像是变了一个人一样。

韩冈呵呵笑了起来,伸手拍了拍放在自己肩膀上的妻子的手:“抢位置为夫也只是开玩笑。不想让吕望之将事情做坏了。”

“是异色铸币?”王旖想了一下,很快就反应了过来。

“没错。”韩冈应道。

韩冈想要把握住每一个机会去发展技术,而不是重复几千年来不断重复的工艺。

即便仅仅是铸币,韩冈也希望在其中能有些让人惊喜的地方。以便能推广到其他行业。

元素化学、机械加工,都是可以涉及的领域。而旧有的铸造,同样可以去精益求精,在成分辨析、配料和铸造工艺上,加以研究并实践。

否则他为什么要跟向皇后提起铸币?

怎么降低铸币的成本?怎么减小铸造过程中的损耗?

恐怕吕嘉问心中,只有压榨铜山和工匠吧。反正只要将事情做圆满了,何必多费心力?万一研发不成功,岂不是又多了一桩交到他人手中的把柄?

这就是韩冈看不起官僚的地方。

开源和节流,不应该是从人事上着手。

科学技术才是第一生产力。

不是为了党争,也不仅仅是为了争权夺利,如果吕嘉问有能力,让他一手主控所有事务又何妨?

但既然可以肯定做不好,还是不要占着坑了。

……………………

韩冈的帖子送到的时候,吕嘉问正在王安石府上,正絮絮的说着怎么去铸造新币。

至于给百官、三军的赏赐,太上皇后已经答应了,不需要他再费神从左藏库中搜钱。

“韩玉昆的见识,嘉问是极佩服的。从过往旧事来看,可以说不轻言,无妄语,却言出必中。既然韩玉昆提议以各色金铜铸钱,以防歼人融钱盗铸,那么肯定能够推行于世。”

吕嘉问不介意在王安石面前,夸一夸这位卸任平章的女婿。都抢了人家的生意,回点好话也是应该的。这也显得自己是一片公心,行事正直。

王安石听了也是很安心的样子,他就怕吕嘉问起了胜负心,想要在新钱法施行的过程中,再进行不必要的改动,最后让整件事都变得不可收拾。

现在吕嘉问既然已经舍了面皮,完全采用——或者说夺取——韩冈的建议,那么他也就没必要强行彰显自己,想方设法将韩冈从那一份改铸新钱的提议中抹去痕迹。

吕嘉问的目的仅仅是三司使的位置,将事情做好就是最重要的一步。

韩冈之前既然没能出来反击王安石,那么自己采用他的方略,韩冈就是生闷气,也很难再出来与自己作对。只要之后面对韩冈时,再公开表现得谦逊恭敬一些,韩冈就算依然恨意难消,也只能困于士论,不能对自己怎么样。

吕嘉问在王安石这边坐得很安心,章惇都愿意去说服韩冈了,还能有什么问题?

只是没多久,从韩家送来一封短笺,让王安石的脸一下就僵硬了。

看着韩冈的私信,王安石脸色变得十分难看,“人呢?”

他张扬着手中那封短笺,问送信过来的王旁。

“大人?”王旁躬身问道,王安石的话没头没脑,让他根本不知道是什么意思。

“送信的人呢?”王安石怒声道。多少年了,已经很难的看见他发这么大脾气。

只是他很快就从儿子的惊慌失措中想到了自己的错误,他已经从平章军国重事的任上退下来了,就不应该再干涉朝政太多。帮吕嘉问一把,已经是破例。是看在过去的情分上,以及新党的统治根基,才伸手帮的忙。

轻声一叹,王安石的声音和气了一点,“玉昆派来送信的人呢?”

王旁战战兢兢,一边暗自抱怨韩冈,一边回道:“孩儿写了回帖,已经打发他先回去了。大人,可是有什么不妥?”

“这样啊。那就算了。”王安石摇头,带着嫌恶,又去看韩冈的帖子。看了几眼,就递给了吕嘉问,“望之,你看看。”

韩冈在帖子中,只是问候一下王安石。

然后谈了一下吕嘉问准备铸造异色钱币的事,表示自己乐见其成,接着回忆了吕嘉问当年在市易司的功绩,认为吕嘉问有足够的能力去完成这项工作,并祝愿吕嘉问在王安石的支持下取得成功。

直到这里还没有问题,除非有人多心,认为韩冈是‘善祷善颂’,但接下来韩冈却又谈了一下突然改动铸币的工艺,会造成各大钱监无所适从,必须要选拔贤能来执掌一应铸币事宜——也就是盐铁司铁案。

从字面上,这当然不是要钱。就是想上奏说韩冈讨要官职,也不可能拿这份帖子来作证据。

可实际上呢?

就是在明说要官。一点也不隐晦,却别想拿来当把柄。

吕嘉问做得是过分,但韩冈的反扑也同样过分,为了争功,将国家名.器当成什么了?

吕嘉问一目十行的看了一遍,抬头强笑道,“不过是个芝麻粒大的差遣,让韩玉昆又如何,皆是为了国事,难道韩玉昆还会故意坏事不成?”

“真的这么想?”王安石直接问道。

吕嘉问笑得开怀了一点,“又不是要三司开拆司的判官,有什么不能给的?”

三司开拆司是承接中书门下和枢密院下发的宣、敇,以及诸州申报三司的文字,并分门别类,依照归属发放到盐铁、度支和户部三司,同时还有清理积欠,驱催文书,并管理勾销簿历——即是各式人事档案和账簿。是维系三司正常运转的部门。

判三司开拆司公事,也就是判官,相当于中书门下的中书五房检正公事,或是枢密院的都承旨。是本衙门的核心官员,掌握三司内部和外部的文牍往来。

相比起三司开拆司判官,区区一个铁案,给了韩冈的人又如何?

要坏事的办法太多了,要收买一个人的办法也太多了。

吕嘉问自信,只要韩冈派人来,他转眼就能将人弄下水。

到了他的手底下,想怎么整治,当然就可以怎么整治。

王安石定定的看着吕嘉问一阵,很是疲惫的叹道,“望之,希望你能记得这句话。为国事,息纷争。玉昆做错了,我去说他,你却要坐正了。”

“平章放心。”吕嘉问欠身道,“平章的话,嘉问会谨记在心。”

