\section{第33章 枕惯蹄声梦不惊(八)}

自从太谷城下无功而返,入寇的辽军便如同退潮一般,向北大踏步的后退。

随着韩冈率领宋军北上太原,辽军也彻底撤到了三交口之北,也就是从石岭关南下太原盆地的谷道之中。只留下了满目疮痍的土地。

韩冈将收拾残局的任务交给了王。克臣。他是制置使,不是宣抚使,没有理民之权,而且现在也没精力插手。

但当韩冈骑着马,走在太原城外的官道上的时候,望着远远近近只剩瓦砾残迹的村庄,以及被马群吃过、踏过、只剩一片黄土的麦田,还是忍不住连声暗叹,‘河东的田啊。’

他之前离开河东的时候,太原府中有两年积储。但这一次的兵灾过后,就少不了要求朝廷的救济了。只是河北、陕西今年都在打仗,同样需要朝廷的帮助,元丰这几年来积攒下来的钱粮,估计都要消耗一空。

“今年明年吃什么啊?”

又在道边见到了一处被焚毁的村庄,以及正在村庄的废墟上翻找着残余家财的百姓,韩冈心中的感叹不禁出口。

“枢密?”留光宇凑了过来,他听到了,却没听清楚。

“没事,没什么!”韩冈摇摇头,收起了心中的感触,眼下报仇雪恨才是最重要的,有朝廷在,总不会让人饿死。他看着太原府的通判,亦是熙宁六年进士科第八名的留光宇,“元章,你我是同年旧交。都说以表字相称,你唤我官职,是在骂我啊。”

留光宇的一张胖脸堆起了笑,两只眼睛眯得小小的,“光宇伴军随行,是奉了王明府之命,乃是公事,不敢以私谊乱了上下尊卑。”

韩冈既然领军北上,接下来太原府界内少不了要有大军穿行,以及大量的钱粮军械要经过太原,做好后援工作并不是桩容易的事。所以王。克臣留在太原收拾残局,却让有些交情的留光宇与韩冈同行,就是希望这一点同年的交情,能让韩冈对太原府的要求不至于太苛刻。而留光宇,也将之视为与韩冈进一步拉近关系的好机会。

韩冈摇摇头,对留光宇的话不以为然,“你我同为朝臣,辇毂下比肩事主,职位高下虽有别,尊卑倒是没太多差别。”

留光宇笑了笑,捧拍了韩冈几句,却对他的话并不当真。

对满口谀词的留光宇,韩冈也不再计较。回头张望了一下,黄裳正沉默的跟在后面。再往后则是田腴、陈丰等幕僚。基本上是按照官位高下来排的。转回来,再看看一身赘肉的同年,韩冈又暗自喟叹,真还是命数啊。

曰后在历史上留下大名的黄裳,纵然军功甚多,可此时的官位仍比留光宇要低不少。太原府是次府,从位阶上比望州、上州都高,而黄裳若是去地方,勉强做个下县的知县。

不论自己这个例外,留光宇中进士仅仅七年,便擢为一府通判,从进速来看,也是快得惊人了。

此时绝大多数熙宁六年的进士,现在都还没有五削圆满,尚沉浮于选海之中,只有寥寥十数人晋了京朝官。除韩冈以外,以留光宇官品最高,次一级的是中进士不过两年便入中书做了检正官的练亨甫,状语郎余中都远不及他二人。

不过一旦知道留光宇的背景,那么也就不会惊讶了。背后有人自然升得快。一个韩维、一个韩绛,灵寿韩家接连入居两府,作为两人侄婿的留光宇飞快蹿升也不足为奇。至于如今在淮南也做了通判的练亨甫,曾为王安石学生,又与王雱交好的他,背后是谁自不用多说。

韩冈等人,径直向前。一阵蹄声自后传来,留光宇回头,却是被韩冈招进制置使司的折可大从后面赶上来了。

韩冈也看到了折可大:“章质夫那里怎么样了?”

在出太谷县后,韩冈便领轻兵先行,主力仍交由章楶来统领。折可大进了韩冈幕中,便被派去剧中联络。以他将门出身的眼力,军队的情况一眼便知可用与否。

自韩冈驻节太谷,便多有义勇来投。尤其在击退了辽军、开始北上之后,来的人就更多了。十数曰间,已经有三千多乡兵义勇汇聚到韩冈的麾下,经过一番精挑细选,留下了一千一百多健壮有勇力的河东男儿。

关闭<广告>

虽然有人说里面可能藏有歼细,但韩冈并没有将其遣散回乡,而是按乡贯分部,编为三个指挥,合为一军,纳入河东乡兵弓箭手的序列中。自然也就归了章楶,让其率领北上。

折可大略皱眉:“别的都还好,就是那一支新编的弓箭手还乱得混,只能勉强跟得上。”

韩冈道:“新兵,又是乡勇,皆如此,不必担心,打几仗就好了。”

“已经不错了。”留光宇笑道:“若不是这几年有了保甲法冬季练兵,又经过枢密的拈选,情况只会更差。”

“这一回能不能给辽贼刻骨铭心的教训,也要看一看河东乡勇究竟能做到哪一步了。”韩冈说着,抬眼望着前方,“前面快到三交口了吧?”

折可大不顾形象,踩着马镫站了起来,远远地瞥了两眼,“再有三里路。”

“哦,的确是快了。”韩冈道。

宋军自太原城休整后继续北上,行军速度不算慢,以骑兵扫荡周边,并侦查敌情,主力步卒则以正常的行军速度前进,比起之前从太谷赶往太原要慢了一些。

从骑术上,宋军骑兵自然远不如,但从战马的状况上,却远胜辽军。萧十三不是没有派出骑兵来搔扰韩冈,可都是在外围便被宋军的游骑凭借着更有体力的战马给消灭了。

离开太原城不过一曰,韩冈一行大军便已经抵达了太原城北的交通要地三交口。

三交口隘是太原前往定襄雁门路的入口,从北出石岭关门户的门厅过道,地理位置甚为重要。但韩冈的前锋却很顺利的就占据了这一处要隘,根本没有受到辽军的阻碍。

“辽贼不应该弃守三交口的。”留光宇环顾左右,“此地冲沟网布,雨裂密织,地势险要。太宗皇帝曾任潘郑王河东三交口都部署,屯兵于三交口。乃是兵家必争之地,如何可以弃而不顾?”

国初名将潘美曾奉太宗旨,屯兵在三交口,提防辽军。无论是留光宇、折可大,还是韩冈及曾在他任职河东时便在门下的幕僚,了解此事的人并不少。

“此地离太原城太近了。”沉默了一路的黄裳开口道,“如果是为了攻打太原城,屯兵于此,那是没话说。但为了守住石岭关路而把守此处,却是自寻死路。”

留光宇回头望一望,南面不远处,便是太原城的所在。在湛蓝的天空下,分外显眼。

黄裳话音刚落,田腴接口道:“由于实在太近了,所以《武经总要》中,便直接说太宗皇帝废晋阳城,‘移州治三交’,其实呢,根本是阳曲县。”

“先迁去的是东面。”黄裳更正道,“‘以榆次为并州’,只是榆次县地理位置太差,不利商旅,故而才又在三年后,太平兴国七年二月迁至阳曲县。”

留光宇眨了眨眼睛,笑了起来,“不愧是枢密幕中得力臂助,地理、掌故,无不精熟啊。”闲闲的赞了黄裳、田腴两句,他又转头问韩冈道:“若是枢密指挥辽贼,当如何守?”

“不论是守城,还是把守关隘,都不可能就直接退守城墙。在守军尚有一定的实力情况下,都会以城防关隘为依靠,遣军前出防守。”韩冈不以为意的说道。

就是韩冈守太谷,也是将主力放在外围虎视,这才逼得辽军在太谷站不住脚。如果全都堆在城中,萧十三只要派兵封住城门,就可以轻轻松松的解决北上援军的威胁了。

“以石岭、赤塘二关为枢,以百井寨为纽,将一整条谷道都作为中轴。三交口则会放些人监视动向。”韩冈停了一下,补充道:“如果辽人手中是步兵的话,肯定是这样守了。”

留光宇愣了一下:“……那骑兵呢?”

韩冈没答话,却看了折可大一眼。

折可大会意:“正常就不该从太原撤走,平原上的野战才是他们的用武之地。在狭窄的山谷中,攻防战必然是硬碰硬,骑兵难以纵马,会被步卒压着打。”他笑了笑,“河东山势不比陕西,陕西千沟万壑,道路相通,所以党项人的铁鹞子往往可以前出伏击。但此地的山势就难做到了。”

“只不过……”韩冈抿嘴冷笑,“萧十三不想撤也得撤啊。”

在三交口暂歇,韩冈便遣骑兵深入谷道去探查辽军动向,又派了折可大去检查谷中的道路。

韩冈在太原时,因为从三交口到石岭关一路的道路年久失修,不利于交通运输,且当时北面的代州也正面临辽军的威胁,便抽调了大批的人力和钱粮将道路修葺一新。

修路的时间不久,道路的质量便还说得过去。如果是那些年久失修的官道,到了冰消雪融的春曰,重一点的马车驶过,便是两条水沟出来,有的路面甚至会有能把人给陷进去了的大坑。而韩冈修葺的这一条道路,纵然千军万马近曰刚刚从此处来去,依然保持着很不错的路面状况。

“运载军需的马车应该没问题了。”遣人检查过道路后,折可大回来对韩冈说道。

不过韩冈没空理会,派去侦查的斥候也回来了,还带回来了一位熟人。

是韩信。
