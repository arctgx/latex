\section{第15章 焰上云霄思逐寇(17)}

烛火幽暗。

李常杰端坐在帐幕中,紧闭着双眼。映在帐篷上的身影,随着跳动的烛光忽长忽短。

横州那一支偏师占据了永平县之后,就已经整军北上。但有关他们的情报,由于传递需要时间,都是几天前的旧消息。今天已经到了预定计划中的日子,但昆仑关上,还没有他们抵达目的地的征兆出现。

知悉这一计划的交趾将领们,私下里都在议论纷纷,他们的自信心在这些日子里都被宋人粉碎了,都在担心是不是出了意外。

而李常杰则依然稳如泰山一般,他沉稳的态度镇住了浮动的军心,也是因为他这边的情况要好一些。

用一重重拒马鹿角将道路封锁,关城中的守军即使想反击也只会被阻止在栅栏前,这让畏惧宋军突袭的士卒,夜中能够安寝。只是这样看起来,反倒是他这边更像是防守的一方。不过他麾下士兵每天夜里都将鹿角更上移一段,几天下来,已经压倒离关城只有两里的地方。

区区两里的距离,只要一个冲锋就能冲到城下,但城中的守军始终没有出战,看起来愚蠢的攻城法倒还是挺有用处,让城中的宋军有力无处施展。

但城头上的守卫依然严密,李字将旗在城头上挂着。如果他设置的障碍再往前推进一里,恐怕就要受到宋军的攻击。自己都是夜中让人去迁移鹿角,只有一里两里而已,宋军当会直接出来夜袭。

不过这就是李常杰目的。他要的就是试探宋军的反应,看看昆仑关中究竟有没有分兵出外。宋军前日在归仁铺撤退,虚虚实实的伎俩让李常杰丢人现眼,现在城头上虽然毫无动静,说不定就是宋人制造出来的假象。今夜他会再往前推进一里,如果没有动静,明天就可以开始攻城。

计算着利害得失,帐外这时有了动静。

听到声音,李常杰睁开眼睛:“你那边怎么样了?”

“广源军已经全数撤离,派了人一路盯着刘纪他们,不会让他们乱来。”宗亶走了进来。

横州的偏师虽然重要,但李常杰并没有让他来统领,交给了自己的心腹将领李玢。不过宗亶倒是不在乎这么多了,以李常杰的为人,绝不会将胜利的希望寄托在他这样的外系将领身上。

“倒是昆仑关这边怎么样了?李玢还没到吗?”

“没有消息也不能说明他没到。”李常杰长身而起,“只看今夜宋人如何应对,就能知道昆仑关中虚实。”

宋人手上的兵力不足,即使将昆仑关中所有兵力都调走估计都不会超过六千,比起横州的偏师还少。如果没有八百名宋军作为核心主力,只凭借黄金满的那点兵力,根本无法与李玢的七千兵马相对抗——这可不是之前逆袭疲惫不堪的追兵,也不是反叛后偷袭友军,在正面堂堂正正的作战上,大越官军还不至于会怕广源军。尤其是去横州的那七千人,并没有经历之前两次失败,不会畏惧广源兵。

不管那位韩运使怎么分派,就他手上的那点人数,要么是关城兵力不足,要么就是抵挡偏师的兵力不足,不会有两全齐美的可能。

“只要宋军主力不在关城中,就是用土来堆,我们也能一口气堆到城头上!”

……………………

“已经到了两里外,再近一点就要关城底下了。”黄全在韩冈背后低声说着。

韩冈扶着雉堞望着远方,交趾人的确是一天比一天更接近,就在眼皮底下忙忙碌碌的样子也的确让人看得烦心:“看着样子,李常杰今天夜里也不会停手。”

“今夜要不要小人带兵出关去?”

“守着关内就好。”

要是黄全突袭失败,关城中的实力露底事小,关中仅有的两千广源军士气大落可就麻烦了。

就算李常杰反应过来,直接来攻打关城……即便关中只有两千兵马,想要攻下昆仑关,也不是那么容易的一件事,少说也要数日时间,而他韩冈也只需要有几天的缓冲。

天天都有信使将行程传到韩冈的手中。昨天他得到的是个好消息。

之前突袭宾州的七百交趾兵在韩廉所率骑兵小队的干扰下,被拖慢了半日行程。给苏子元和黄金满率领的队伍咬住了,差了一步没能蹿进邕州东南的群山之中。

就在山林外的平原上,双方展开了一场战斗。尽管这七百兵是李常杰精挑细选出来的精锐,但先是在雨中翻山越岭,继而又被日夜骚扰,加上山林中的退路就在眼前使得人无战意。让黄金满很是轻松的就击败了他们。

不过也是因为离着山林太近,还是给交趾兵跑了大半进去,战后计点,连杀伤带俘获只有两百出头。跑出去的近五百人,不是毫发无伤,就是一点点皮肉轻伤。如果有人能够将他们重新组织起来,还是会有着一定的战斗力,但短时间内,不会造成太大的威胁了。

就在这份军报传到韩冈手上的同一天,李信与黄金满两部会合的消息、近万交趾军出现在宾州东南的消息,也在稍迟一点的时候传到了他的手中。从时间上看,两边决战多半不是在今天,就是在明日。

这是一个漂亮的右勾拳,不过要想打在自己这边的腰眼上,就必须足够隐秘且出人意表,很可惜李常杰并没有做到。虽然兵力相差甚远,但一方是严阵以待,一方则是偷袭失败,加上战力有别,韩冈不会怀疑官军的胜利。他现在要防备的是李常杰正面打来的直拳。

当天夜里,交趾兵又开始了向关城逼近的动作,将拒马、鹿角等拦截物向前推进。夜幕中他们发出的声音,就在关头上都能听得分明,但关城中的守军恍若未闻,就让交趾人自由自在的行动,一口气将防线推进到关城一里之内。

“关中无兵!”

就在两道鹿角之后,李常杰抬头望着关头上猎猎飞扬的大旗。眯起的双眼中满是得意,他昨夜为了防备城中守军杀出,辛辛苦苦做的防备全都没有派上用场,但终于试探出了宋人的底细。

不用再浪费时间,停歇了数日的战鼓重新鸣响,一名名交趾士兵抱着一包包泥土向着关城冲过来。只看他们跑动时的样子,城上的守军就知道到底是准备怎样攻城了。

“运使!让小人出关迎战!”黄全急声叫道。

“没那个必要。”韩冈摇头依旧,“在关中守着就可以了!”

韩冈没有同意让广源军出城迎战。他们不是官军,遇上逆境并没有咬牙坚持到底的可能。出关后,如果战事胶着起来,一时不能获胜,他们溃退的可能性很大。也只有在关城上,他们才能稳住阵脚与交趾人对垒。

关城中的沉默让交趾兵更加兴奋起来,两条腿奔跑起来更加有力。

看着第一批交趾兵已经接近到城下,城头一声鼓响,就是一片箭矢射下,将准备垒土上城的交趾兵射倒了一地。关城上箭矢如雨,将打头阵的交趾兵射得鬼哭狼嚎,逼着他们退逃回去,方才

发给广源军使用的弩弓都是这些天来被湿气所侵染,尽管临时用火烤过,但仍远远不及正常的威力,而且损坏几率则高出许多。仅仅一刻钟的射击,就有两成神臂弓断了弦,甚至有二十多张连弩臂都断了。

“果然没错!”李常杰再一次肯定了自己的判断,作为一名久经战事的将帅,他对战场上的一些事还是很敏感。昆仑关上弩弓射击的节奏感不对。与他之前见识过的宋军箭阵,有着截然不同的区别。如果是宋军,就算是用着状况不佳的弓弩,也只会是不能追击逃远的敌人,发射箭矢的间隔不会这么长,对目标的选择以及发射的时机,也不会这么乱,“只会是广源兵!”

更加确定了关城中并无宋军,交趾兵很快就举着防箭的巨型木盾,再一次攻了上来了。不仅仅有着抱着土包的士兵,还有几人用着大嗓门高声喊着广源土话,试图动摇城中军心。

“运使,怎么办?”黄全下去弹压军心回来,忧心忡忡的问着韩冈。

“只要能守住两天就够了。”城下的土堆一点点高了起来,韩冈依然保持着平静。

虽然没有石灰、没有油料、没有床弩,除了一堆长了青苔的礌石滚木以外,没有一切该有的守城装具。但靠着城墙和弓弩,以及两千守军,维持着一定水平的士气,要保住关城两天,还是绰绰有余。

城上箭矢不断,而城下则依然坚持着将土堆累积。

到了傍晚,李常杰望着已经堆到城墙一半位置上的土坡,回头对着身后的众将露出得意的笑容:“赢定了!”

离着李常杰直线距离只有一里,就在关城之内,韩冈将刚刚收到的一张纸条攥紧在手中,低头看着单膝跪在身前的一名军士,脸上有着同样的笑容,“赢定了!”

