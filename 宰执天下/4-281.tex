\section{第46章 了无旧客伴清谈(三)}

天阴着,细雪若有若无,从云层中洒落,又随着微风散开。

雪粒细细的,不像柳絮,却似盐末,落入大地,瞬息便不见了踪影。

冬日清晨的空气,没有因为降雪而变得湿润,干冷而又清爽。

天色已经大亮,菜市早就喧闹了起来,卖汤饼和炊饼的摊子在街边隔着不远就是一摊,章惇望望东十字大街的方向,“鬼市子差不多要散了。”

韩冈与章惇并肩走着:“听说鬼市子中杜家的羊头汤有名得很,黄幺儿的赤白腰子也是一绝,要不要去尝一尝?”

章惇不舍的望了一眼远方,摇头:“算了,到了那里都不知道是什么时候了。将就找个地方坐下来吃点好了。”

鬼市子就是城中早市,开在潘楼东面的东十字大街,五更开张,天明收场,卖些古董书画还有衣物饰品,货物的来历,有的合法,有的则不是那么干净,并不是正经去处。不过到了天亮之后,就变成了早点一条街。

不过鬼市子里面卖的早点,有不少在东京城中都有些名气,连宫里都派人出来买过,章惇和韩冈都很熟悉。

只是离得太远了,韩冈和章惇就汴水水边散步,等着城门的开启。而东十字大街则是在数里开外。下着雪,走过去都不知是什么时候了。

腊月不是出行的好日子,章惇是刚刚辞位的枢密副使,要出知池州,就算他拖到年后,过了上元节再南下,天子都要给他这份体面。

可章惇从枢密副使,变成了池州知州,门庭一下就冷落起来,也是让人灰心丧气。身居高位,突然间落入深渊,这是心高气傲的章惇所不愿面对的。与其留在京城丢人现眼,还不如早一步离开的好。

据韩冈所知,王韶的情况也差不多。

自从由京城回到南方之后,王韶身体似乎一直不太好。上次还写信回来,问着在南方湿润之地该怎么保养,不过韩冈觉得他的情况应该是心情上的问题,江西人问陕西人在南方怎么养生,根本是个笑话。但也不能说不对,毕竟韩冈的名气在。

王韶当年考上进士后,就弃官不做,游历陕西。他不屑做琐事,摒弃普通官员按部就班的路线,选择了更为艰难,但收获也更为丰厚的道路。而他也用才能和功绩证明了自己的选择,在嘉佑二年的进士中,他第一个晋身两府,比吕惠卿还要早三年,声威一时无两。

只是到了现在,去职出外,心气高傲的王韶,失落感只会比眼下的章惇更为强烈。即使王韶的心中很明白自己是不可能一直留在朝堂中,日后也还有重新返回两府的希望,但心情上的巨大落差还是免不了的。韩冈只希望他能放宽心,否则那样的心情对身体不会有好处。

韩冈陪着章惇在汴河边漫步着。现在两名天下闻名的重臣,都是穿着一袭襕衫,外面套了半新不旧的绰子,看着就像两个东京城中最为常见的不第士人,一大早起来,借着早上清醒的头脑,沿着河道回忆昨日的功课。

汴水之滨的码头,从清早就开始就是一片忙碌。

行驶在冬季的变水上的不是船只,而是一辆辆雪橇车。当年用来紧急运送纲粮的雪橇车,如今已经成了冬日随处可见的一景。安静的泊在码头边,卸货装车,通过轨道运往不远处的仓库。

章惇的双眼追逐着在轨道上穿梭的车辆:“从港口到矿山,再从矿山到方城,如今又从方城到河北。玉昆你的这轨道可比飞船更能排得上用场,薛师正【薛向】言其可当十万大军,并非夸大之语、”

“还早得很呐。”韩冈摇头,“河北轨道七百里路,修起来就不容易,运行起来问题还会更多。”

章惇偏头看着一步外的韩冈:“以玉昆的胸襟,眼光所及应该不止河北、京西。”

“只是有些想法而已。”韩冈谦虚了一句,“小弟最想看到的是天下州郡都有顺畅的交通联络,让朝廷的政令能用最快的速度抵达最边远的州郡,能让官军在最短的时间,进驻到每一处遇敌的边疆。”

他指着脚边冻结的水面:“说到运输,水道其实是最好的,运力大、耗用少。但天下地势起伏万端,水道不通的地方,最好修造轨道作为替代。”

“朝廷的钱粮不一定能供给得上。”

“轨道货运收入不少。通过第一条的收入,来推动第二条轨道的建设,等第二条建成,又可以用推动第三条轨道的建设。”韩冈顿了一下,“而且也不一定全都要官府攥在手中,以官政、行旅、商事往来的多寡,区分干线、支线。干线收归官有,支线交予民间。抓大放小嘛……”

“两位员外,小人这里有热腾腾的炊饼,可要来上一块!”一声吆喝打断了两人的对话。

两人一起循声望过去,离着两人不远,是个五短身材的小贩,挑着个担子,歇在路边上。

同时被章惇和韩冈扫了一眼,卖炊饼的矮子一下打了个寒战,话都说不利索了,在担子边上心惊胆战:‘这两个措大眼神好不骇人,莫不是杀过人放过火的。’

正好猜个正着的小贩,结结巴巴的挤了两句卖炊饼时的货郎词,“热腾腾的十字炊饼,甜津津的油蜜炊饼。两位员外,要不要一块。”

章惇以眼神阻止了略远处的护卫,走上去问道:“有没有馒头?”

“有,有。”小贩点着头,“有上好精肉做的肉馒头。有家里浑家亲手腌的梅干菜馒头。还有上好的交州糖霜熬的馅料做的糖霜馒头,面白馅润,咬一下便是满口糖汁,再香甜不过。”

“几文一枚?”章惇站在担子边,很有些兴致的问价格。

“肉馒头五文一枚,梅干菜的三文一枚,糖霜馒头十二文钱。”小贩麻利的掀开厚实的白布,里面的炊饼、馒头热气蒸腾。

“玉昆,要不要尝尝白糖馒头。”章惇回头微笑,“可是交州来的。”

“小弟不喜甜食,梅干菜的就可以了。”

“那就算了,我也不吃甜食好了。钱要省着点花。”

小贩亮起的眼神又黯淡了下去,眼前的两个客人看着相貌不俗,都有几分官人气派,没想到都是穷鬼。

也正应了小贩的腹诽,韩冈摸摸袖子,再摸摸怀里,手巾倒有一条,就是一文铜板也无。

章惇从袖子里掏出几个制钱来,对韩冈笑道:“出来能不带钱?”

韩冈回之一笑:“早就不知道钱包有多重了。”

章惇帮韩冈付了帐:“下一回可是要还的。”

“没问题,等子厚兄回京,小弟当在樊楼还席。”

‘穷措大还想去樊楼。’小贩肚子里咕哝着,用个竹夹子夹了两个梅干菜馒头,拿干荷叶包了,递给两位金紫重臣。

韩冈和章惇各自拿了一个干荷叶包着的梅干菜馒头,在河边边走边啃。馒头热得发烫,拿在手中,啃了一口,身子很快就暖和起来了。

章惇还笑呵呵的,“给御史看到,少不了要弹劾你我无大臣体。”

“管他们那么多!”韩冈狠狠咬了一口手上的馒头。说实话,口味还真不错。回头看看,跟着他们两个的十几名伴当也都去买了馒头来吃,让那个小贩笑得看不见眼睛。

三口两口热腾腾的梅干菜馒头便下了肚,在河边静静走了一阵,章惇忽然道:“这一次便宜郭逵了。”

韩冈先是一愣,要便宜该是便宜薛向才是,非进士的文官晋身两府,而且还不是高门世家子弟,这可是多少年也难得一见。但很快反应过来,“应该不会吧。”

“讨伐西夏,还有谁能统领大军。”章惇很有几分不忿。

想要统领平夏大军的官员将领数不胜数,但数遍朝中,够资格的也就郭逵、王韶和章惇三人——至于韩冈,能力没人怀疑,但资历还是浅了一点,赵顼也不会让他再立功劳。

王韶、章惇如今都是引罪出外,当然领军的可能性都不大。但话说回来,郭逵是武将。若是他平了西夏,还有什么位置能安排得了他?

“两年后的事,谁也说不准。”韩冈摇头,“说不定那时候子厚兄就卷土重来了?”

章惇付之一笑,不提这个话题,“在河北修筑轨道,是为了抵御辽国。但以眼下辽国朝堂上的局势,要是敢赌一把,派一个曾经见过辽国故太子的旧使去贺生辰,在见到辽主时提上两句故太子,说不定就能掀起辽国的内乱。年纪一大,舔犊之亲尤深,杀了独子,由不得辽主不后悔!只要他将耶律乙辛恨上,辽国内乱可期。”说完,他瞥了韩冈一眼,“不过这等做法,玉昆你大概不会放在心上。”

章惇的话饶有深意,韩冈只当没听明白:“与其等待敌国内乱,还不如加强中国实力。只要中国兵精粮足、将兵堪用,以大宋的国力,就是辽国上下万众一心,也会像螳螂一样在战车车轮下被碾得粉碎。”

