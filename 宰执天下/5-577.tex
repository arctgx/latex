\section{第45章 从容行酒御万众(四)}

姆百热克在望楼上眺望着南方。

全然无视了脚下正在紧张展开的战斗。

南面又是一波敌军攻来,可能是因为离得太远的缘故,汉军大营中的投石车并没有发射。

这一回来袭的黑汗军,穿着在初阳下闪闪生辉的铁甲,千余人却让姆百热克完全忽略了脚下营垒之外,那数以千计的黑汗轻骑兵。

是伊克塔!

纵然从来没有真正见过父兄口中残暴无比的黑汗人。但他们的特点,从出生的那一天,就不断在长辈讲述的故事中被提起,被描述,然后深深的刻在记忆中。

这是最狡猾,最凶狠,同时也是最难缠的死敌。

古拉姆近卫很少穿过葱岭。绝大多数时候,他的父兄都是在与疏勒的伊克塔骑兵们交锋。

七八十年前,当于阗陷落,黑汗人意图继续攻击龟兹、高昌,当时,焉耆城中家家出兵上阵,与黑汗人大小数十战,伤亡惨重。幸好北面的契丹人,还有另外一些回鹘部族,南下伊丽河【伊犁河】,牵制了黑汗军。而后黑汗内部也分裂成东西两部,年年大战小战不断,到如今,只有边境上时有战事。但那也让姆百热克失去了许多长辈和亲友。

在这么多年的战争中,来自疏勒的伊克塔骑兵是其中的主力,尤其是没有大战的时间,黑汗人中全都是他们在作战。

在长辈的描述中,伊克塔就是经文中的恶鬼,残暴无情,而又勇猛善战。如果能够砍下这样一名敌人的首级,就是可以向家人和亲友炫耀十年的战绩。

但今天,这些恶鬼仿佛方才踏中陷阱的杂牌军,在汉军单薄的阵线前,人仰马翻。

姆百热克惊讶的揉了揉眼睛。

伊克塔都是披甲的精兵,什么时候都成了乌古斯的那些穷光蛋一样没用?

但就在他的眼前,汉军的弩弓手,迎着猛冲而来的伊克塔骑兵,不停的射击着。

他们手中的重弩一旦射出去,从他们身后,就会递上一张新的上好弦的弩弓。阵列后方的士兵,他们的任务其实就是在为弩弓上弦。用着构造简单却能够充分省力的上弦器,给一张张击发后的弩弓上弦。

在军中就是如此。就是姆百热克本人,也被强制联系过怎么给汉军的重弩上弦。在他所在的营地中,专门负责射击的士兵,只有百人。但他们却能像一千人一样连续射击,只要还有人能够上弦,他们的射击就不会停止……

随着大军的西进,姆百热克亲眼看见跟随大军前进的有一五六十辆大车,上面满载着各色军器和粮食。还有近百名随军的工匠。在抵达末蛮城后,负责打造投石车,修理盔甲、弓刀,制作箭矢。

他们的成果让人惊讶,营地中高高立起的霹雳砲,就是汉人工匠的功劳,还有藏在城中的上弦机。不同于单人使用的上弦器,上弦机是用畜力拉动,可以用更快的速度给弓弩上弦。姆百热克没有亲眼见过,但他有一个一起长大的同伴,因为祖上是汉人,所以能够进入汉军之中做事。他亲口对姆百热克说起过上弦机的力量。

但仅仅是单人使用的上弦器,就已经让汉军阵中的重弩没有一瞬停歇。

弓弦嗡嗡作响。缠绵不绝的振弦声,在千百人的嚎叫和嘶鸣中,依然清晰的传遍战场。

姆百热克依稀回忆起来,他幼时曾经听到过着这样的声音,那是最为暴烈的狂风卷着大漠中的沙砾,横扫过全城的呼啸。

在汉军的阵地之前,仿佛被人划上了一条线。

让姆百热克为之屏息和畏惧的死线!

地上尽是以各种扭曲的姿势躺在地上的人和马。有的已经成了尸骸,有的还在痛苦的挣扎和呻吟。有一道无形的壁垒将他们拦在十步开外。那是他们能冲击到的最近的位置。狂猛的箭矢风暴中,他们再向前半步亦不可得。

交接重弩,放置箭矢,瞄准敌人,扣下牙发。

这是最前沿的弓弩手标准的射击流程。一套流程下来,熟练的士兵只要三个呼吸,便能将不及一尺的短矢,送进眼前敌军的眼窝、喉间和心口。然后周而复始,再瞄准下一人开始射击

不论前方是党项、契丹,还是黑汗,谢迟的动作始终没有变过分毫。一名在军中混迹半生的队正,从横山开始,一路经历了河湟、兰州、甘凉等十年来的历次战事,如今又到了西域这里来。

黑汗人的装备要胜过党项和契丹,但再结实的铁甲,也不可能在十步上抵挡住神臂弓的射击,很多箭矢都是穿透了甲胄然后狠狠的扎进了心口,喉咙等要害。

但大部分伊克塔骑兵,在还没到二十步的时候,便被射中了坐骑,然后翻滚在地。聪明人躲在坐骑后,然后成了几名能与王钤辖相提并论的神射手的目标。而胆大的就直冲过来,然后在狂暴的箭矢,被射成了刺猬。无论贤与不肖,结果都是一样。

渐渐的。来自黑汗骑兵的攻势减缓了,然后停了下来。还没有受到攻击的残兵只剩三四百,转回来的那一批人人人带伤,而没回来的占了三分之一。这样的情况下,如何还能保持之前的进攻节奏。

他们更是吓怕了,害怕冲到宋人阵前,然后被箭矢射程刺猬,最后死得毫无意义。

箭矢不再射击,弓弦也不再鸣响,短促的交锋之后,南面的汉军阵前,一切又都恢复了平静

只是阵前还有人在活动。

那是手拿大斧的汉军士兵,手起斧落,将躺在阵前的黑汗人全数斩下了头颅。

很多黑汗人看得怒发冲冠。就是那些最穷凶极恶的北方人,他们也会留下一些可以征收赎身钱的肥羊。而不是像现在这样,不论贵贱、不论死活,全都一起砍下脑袋。

但要让他们去抢回教中兄弟们的尸骸,他们却都不敢上前一步。他们也有十字弓,但敌人手中的十字弓却是根本没有见识过的。什么样的十字弓能够用短弓的速度射击?在这一战之前,他们可以肯定的说没有。但今曰之后,却有了一个答案。

一枚枚首级被呈到王舜臣的面前。

不断滴下的血水,让他面前的土地变成了红褐色。

仅仅是半个时辰的战斗,连同现在的两百多斩首,今曰的收获已经接近一千。

王舜臣摇了摇头,相对于黑汗人的精锐。这真是太过轻易地胜利。

由各家部族组成的军队,纵然各个都是精兵,但配合和运作上就显得太过粗劣。

黑汗人的伊克塔骑兵据称是各地大族、部落的集合。皆是自备铠甲、马匹。其中当然有穷有富。穷的只能套一件已经生锈的锁子甲,而富有的伊克塔,他们的装备已经跟古拉姆近卫不相上下,人马皆贯甲。

不过在千人之中,富裕得能给战马装备上铠甲的伊克塔终究还是少数,而且在冲锋时,自然而然的都落在了后面。冲在最前的,却是那些连一幅好甲胄都装备不起的穷人。因为他们的负重轻,战马速度快,更因为他们需要一个争夺军功的机会,好用功劳换回更多的赏赐,或是首先选取战利品的资格。

王舜臣咧嘴笑得开怀。

虽然华夷有别,但人情却是一样。千金之子,坐不垂堂。身家丰厚的,总是不愿意冲在最前。那些大食人不说为圣战而死,上了天后都有无数美女环绕。怎么一个两个都这么看不开?

上阵时缺乏配合,再精锐的敌人都是可以随意揉捏的面团。而有了可靠的组织之后,战斗力里可以立刻翻上几番。伊克塔骑兵失败的地方就是他们缺乏配合,否则具装甲骑在前,重骑兵在后,至少能逼着自己下令弃弩持刀肉搏。

虽然给了对手一个巨大的挫折,但真正精兵依然没有出动。

古拉姆近卫就算不动,但只是他们的存在,就能给人以莫大的威压感。

就算要反击也要考虑到了他们的存在,而不能尽情施展。

这样的敌人才有意思。

王舜臣想着。但他没有在今曰与这些可汗的近卫们交手的打算。差不多也是撤回营中的时候了。

与数万大军决战,当然要选一个良辰吉曰来进行。

黑汗人并不是一天之内就能解决的敌人,王舜臣并不着急。眼下是小小的试探。已经确定了对手的战斗力。

今天出战,只是为了一试深浅,并提振一下士气。若死守不出,被敌军轮番攻击,士气会下降得飞快,必须要时刻保持出击的姿态。

不过若对手实在不堪,王舜臣会立刻领军扑上去,就像饿虎扑食。令人遗憾的是今曰的敌人,可不是可以随意揉捏的羊,外壳硬邦邦的如同石头。王舜臣宁可看着近处修起营垒,也不愿与其硬拼。

末蛮并不算富庶。人口、田地都远远比不上龟兹、更不用说高昌,比焉耆都要差了。所以安西四镇之中,并没有末蛮。目前能够发掘出来的粮食、草料都在城中。不知道黑汗人带了多少粮食来?还是说他们有本事从疏勒不断运粮上来?现在退回营垒中固守,三五曰后,对面的黑汗军还能剩多少存粮?

王舜臣让人敲起了金锣,鸣金收兵。中军先退,骑兵随后,有着人马的尸体为阻,并不用担心黑汗军能够跟上来。

大军缓缓退回城中,欢呼声随之响起,其声震动天地。

王舜臣回帐,卸下盔甲,战事进入了相持的阶段了。

