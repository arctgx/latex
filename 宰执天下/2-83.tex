\section{第20章 心念不改意难平(二)}

【还是没能赶在十二点前完成,个人因素造成的断更真的很抱歉。俺承诺的每日两更总是因为一些事给耽误了。不过这个月二十八天,总计五十六更肯定会完成。不计这一章,接下来的八天,一天三更。】

“郭仲通到底在打着什么主意……”

酒宴过后,自家的客厅中,王韶皱着眉。今天在酒宴上,郭逵很明显的向着王韶、韩冈示好。完全没有他们事先猜想的那样摆出泰山压顶的强势。事出反常,总是让王韶有些难以安心。

“大概是因为李宪在吧。若是郭太尉表现得太过跋扈,他回去后少不得会对天子提上几句。”

韩冈今天在酒宴上被人多敬了几杯,面皮泛红,有些酒意上头。端起王家下人送上来的醒酒汤,啜了一口。满嘴的酸苦味,差点让他把喝进去的醒酒汤给喷出来。不过酒倒是彻底醒了。王家的厨子水平不够,醒酒汤的确能醒酒,却是因为难喝的缘故。

“这点我知道。”王韶也端起茶盅喝了一口,大概是喝惯了,没什么不良反应。只是他一口把醒酒汤喝完,也不放下茶杯,就在手中转着,“以郭逵的身份,也用不着玩什么下马威。在秦州,无人敢对他有丝毫不恭。”

“可郭太尉也没必要表现得这般殷勤,只要礼数到了,谁也不能说他的不是。”王厚像是在反驳他老子的话,可他一边说着,一边却偷眼看着韩冈的反应。

韩冈低下头去,对付起比起严素心的作品,要难喝上几十倍的醒酒汤来。不过这一次,他喝得心不在焉,一点感觉都没有。

其实郭逵今天表现出来的殷勤,有七成是对韩冈的。王韶、王厚都看在眼里,但在韩冈面前,他们有些顾忌着,不好明着说出来。故而言辞间,都有着旁敲侧击,刺探韩冈心意的意思在。

韩冈心下暗叹。这是何苦呢,生辰八字都换了,可以说就是一家人了,有话直接说不就可以了。不过再想想,换作是自己处在王韶的位置上,怕也是一样不会明着说。越是聪明人,心中的计算就越多,反而难以放得开,倒也不可能怪王韶。

“礼下于人必有所求,郭太尉当是想在河湟之事上有一番作为吧……”韩冈还是选择了把话题捅破,表明自己的态度,省得王韶、王厚给自己绕着说话,“郭太尉今日越是殷勤,日后心愿不逞时,攻击之声怕也越是激烈。”

从韩冈的角度来说,他当然想着能左右逢源最好。同时在王韶和郭逵手上得到好处,才能把他的利益最大化,尽可能早的从选人转为京官。

选人转为京官,正常情况下必须拥有五名路一级的监司官的推荐,一份荐书称为一削,五削圆满,号为合尖,此时方可转官。

如果不走正常路线,只依仗军功,也不是不能转为京官。不过在韩冈看来,现在朝廷大概是抱着压制王韶和自己的心思,不让他们进用过速,以防日后功成,难以封赏。

以至于他在古渭大捷上的功劳,都换不来一个京官。除非河湟已复,否则韩冈都不指望他能靠军功脱离选海,而王韶更是不用指望还能再升多少——其他功劳立得再多,也不过是增添食邑,把检校官、勋、散官这些没什么用的虚衔提上几级。

王韶那边韩冈是管不了,但如果他自己有着郭逵相助,把五份荐书搜集到手,朝廷还能再压制他吗?明面上的事情总不能做得太过分。功赏之事还有商榷的余地,只要有说的过去的借口就可以随心调整,但若是已经五削圆满还不能转官,谁还会再辛苦卖命?

只是韩冈的如意算盘是建立在王韶和郭逵同心协力的基础上的。如果要他从王韶和郭逵之间选择一个,那他就只能站在王韶的一方——王韶荐他为官,尽管韩冈对王韶的帮助,已经足以回报这份恩德,但世间,会被人指脊梁骨的蠢事韩冈不会做,何况他跟王家很快就是姻亲,没有胳膊肘往外拐的道理。

王韶听出了韩冈的言外之意,终于放下了手中的素瓷茶杯,笑道:“还是按玉昆的说法,察其言观其行。看日后郭仲通究竟会怎么做吧。”

“大人说的是。”王厚也轻松起来。

今天看到郭逵在酒宴上不顾身份差距,对韩冈举杯敬酒,他的心都提起来了。韩冈是王韶的谋主,他有多少才能王厚最清楚。要是他被郭逵招揽去,对王韶的打击几乎是抽梁扒柱一般,几乎就是毁灭性的。

见两人放下心来,韩冈便换了话题:“郭逵这边且看着日后。而李御府那边,好像也是对河湟之事很上心的样子……”

“李宪方才已经说了明天就去古渭。”王韶说道。

“这么急?”韩冈抬了抬眉毛,以示自己的惊讶。

王厚回想起了王中正,便笑道:“王都知上次来,还在秦州待了两天,收了点孝敬。李御府今次走得这么急,可是要少赚不少,真不知他怎么想的。”

“不管李宪怎么想,既然他明天要去古渭寨,我也得与他一起去。”王韶转过脸对韩冈道,“玉昆,你在秦州还要待几天。”

韩冈考虑一下:“疗养院这边的事有些棘手,不知安抚能不能让处道兄在秦州留上几天,帮着处理一下。等此间事了,我和处道兄一起再往古渭去。”

韩冈要留下王厚,这是他要自证清白,心中并无任何改换门第的心思。但王韶能顺水推舟的答应下来吗?当然不可能!这么做可是明摆着不信任韩冈。

所以他说道:“古渭有许多事急着要办,衙中少了玉昆你,就不能再少了二哥儿了。玉昆你把秦州疗养院的事安排好后,也尽速赶去古渭。李宪在天子面前很受看重,今次机会难得,你与他多说上几句,在御前也能得几句好话。”

王厚也道:“愚兄可是同管勾机宜等事,玉昆你这正牌子的机宜不去上任,愚兄再不去,不知会耽误多少事情。如今已是入秋,古渭寨的榷场再不快点开张,明年的日子就难过了。还有屯田,不趁这两个月招徕一批人来,就来不及垦田种麦了。”

“就让王舜臣先跟着玉昆你。”韩冈已经说了自己缺帮手,虽然只是安人心的借口,但王韶却得把明面上的事做圆满了,“有什么事,要他帮你处理着。他现在可是右侍禁了,反压在傅勍头上,去了急了反而有些麻烦。给傅勍几天时间,等他把寨中事务处理好,王舜臣再来不迟。”

王舜臣和杨英比郭逵一行要早上两天回到秦州。据他们所说,在路上跟郭逵、李宪的车队擦肩而过,不过没敢上前打招呼,直接从路边超了过去。

今天他们也参加了酒宴,而且坐得位置还不低。整个宴会上,就听着王舜臣举透着兴奋的喝酒、说话,纵声大笑,说话的声音也吵得直传上了天花板。最后喝得酩酊大醉,路都走不稳了,被人抬着送了回去。他最后的模样,就跟好酗酒的傅勍差不多去,让韩冈看得担心不已。

在所有参与了古渭之战的官员中,王舜臣是今次晋阶最多的一个。他护送韩冈去青唐城,又直接参加了伏击董裕大军的作战,手上还有一个斩将之功——为董裕奔走,招徕从逆部族的蕃僧结吴叱腊就死在他的刀下,虽然实际上是杀俘,但知情的都保持沉默——官位就因此一口气跳了四阶,从最低的三班借职,一下跃居右侍禁。

韩冈倒不会去嫉妒王舜臣晋升得比他还快。在北宋,文武两班是截然不同的两个系统。武臣有战功,往往都是几阶几阶的跳级,如果没有战功,靠熬资历的话,七年才能升一级——这是为了鼓励武将奋勇杀敌——不过若是犯错败阵,跌下来也容易。

可王舜臣还没到跌得时候,他现在正是春风得意。韩冈曾建议让傅勍权知古渭寨,让王舜臣等人则负责具体军务。可现在王舜臣的官阶已经彻底压倒了傅勍。这让在军中蹉跎已久的新任古渭寨主,怎么指挥他?

而且参加了古渭之役的杨英也是一样跃居傅勍之上。虽然他从头到尾都没上过阵,只守着王韶。但瞎药送了他五个斩首的功劳,而俞龙珂听说之后,立马又送了他十个斩首,虽然王韶没有看着他们乱来,只让杨英从俞龙珂两兄弟手上各收了五个首级作为战功,但杨英也是因此而越阶超转,压在傅勍的头上。

秦凤路中,甚至是秦州本州,都不是没有其他可以适任古渭知寨一职的官员,可以名正言顺的指挥着王舜臣和杨英。但现在木已成舟,王韶和高遵裕一力提拔傅勍的奏章刚刚得到批准没两天,又要将之换人,那会让人看笑话的。

“不知王舜臣到古渭寨之后,还会不会听着傅勍的指派。”王厚现在就有些担心,“两人官阶差得这么大,王舜臣不去理会傅勍的将令,也不好说他不是。”

“先做着看吧……”王韶此时也显得有些无奈,对他来说,王舜臣肯定是要比傅勍亲近,也比傅勍可信。如果王、傅两人相争,他很难去为了傅勍而责罚王舜臣。

韩冈眨了眨眼睛,也没说什么,这其中也有他的一份责任——毕竟傅勍是他推荐的。

只好有空就多提点提点王舜臣了,韩冈想着。

