\section{第十章 却惭横刀问戎昭(六)}

【白天都在走亲戚,晚上吃过饭才回来。今天剩下一更,明天上午补上。】

秋风起了。

两峰夹持,峰下谷中溪水潺潺。一支马队,在水畔迤逦而行。马颈下一串串铃声清脆,随着忽起忽落的马蹄,在峰谷间回响。铃声中,迎面而来的山风清凉,带着些许秋意。

过了忻州【今忻县】的忻口寨,五台山的峰峦叠嶂就出现在韩冈一行人的眼前。

时近八月,天上的日头也没有了半个月前的那般炽烈。骑在马上走了一程,身上竟然仍是清凉无汗。

远山近水,映在人们眼中的,依然是一片片或浓或淡的绿色,但队伍中每一个都能切实感受到将临的秋意。

‘秋天到了。’

任何一名驻守北方边州的官员,都不会太喜欢秋天。一年之中,春夏两季的悠闲之后,便是秋冬两季的紧张和忙碌。

粮秣军资要完成预定的储备,驻军要前出至边境的军寨,烽火台的缺额要填上,兵器甲胄要检查、修理和补充……等等等等。这就是所谓的防秋,北方边界诸州都要在这半年里支起耳朵、瞪大眼睛,握紧手中的刀枪,时刻准备着与寇边的贼虏拼死一战。

即便澶渊之盟订立之后,辽人对边境连骚扰都很少;当今天子登基以来,西夏更是如江河日下,根本无力侵边。但每到秋冬,还是无人敢于疏忽大意。就算有人疏忽大意,天子和两府每年到了七月必然下发的诏令,也会提醒他们不要糊涂。

今年的防秋,应当是近年来最紧张和危险一次。战火虽说是在西夏境内燃烧,但静极思动的辽人却有让河北、河东、陕西乃至京城,都一并陷入恐慌的能力。

一旦宋辽两国当真进入战争状态,同时进行两场全面战争的大宋,接下来的半年可就会很难过了。

所以韩冈知道,如果辽人仅只于骚扰的话,朝廷绝对不会同意为此大动干戈,甚至对于其掠边的暴行视而不见都有可能。

当然,如果韩冈当真如朝廷之意对此事姑息,他一样好过不了。朝中多少人正愁没有他的把柄,能有这么好的机会捅上他一刀,绝不会有二话。

韩冈正是不想出现这样的情况,才会提前前往代州。

辽人犯界一事说大不大,说小不小,但偏偏卡在这个节骨眼上,本就已经准备好代州一行的韩冈,不得不提前了几天动身。他将手上的事务丢给了通判,大张旗鼓的往代州去。

一般来说,亲民官是不能随意离开所领州县,知州、知县都得老老实实的蹲着,不能往外地乱跑。不过这一次韩冈任职太原府,身上背着经略使职衔,加上又是战时,所以一些约束守臣的规矩和法度,根本就没有实行的可能。

黄裳很是紧张。韩冈前往代州巡视,并不是坐在代州城中,任凭将领们胡说八道,而是要往最前线的寨堡去。若有个万一……黄裳用力甩了甩头,这种事可不能乱想。

韩冈对此倒没有太多的想法。他只是想看看北方的守备情况,顺便对辽人有来有往而已。

——现下坐镇云州【大同】的听说是北院枢密副使萧十三,耶律乙辛的心腹亲信。他想必是愿意给新上任的河东经略一个下马威的。

韩冈没有将自家的想法向下属们和盘托出的打算,这是他个人的想法而已,表面上仅仅是激励士气,同时顺便查个帐。

现在就看刘舜卿是怎么做的了。他到底是怎么处置犯下一桩桩罪行的辽人?是妥协退让,还是打定主意要报复回去?这将决定韩冈对他的态度。

过了界碑,就是代州的地界。

韩冈的行程早就先期传至代州城。迎接经略使一行的人马,已经在路边等候多时。

刘舜卿不可能在边界等候韩冈的到来,他必须在城中坐镇。但他派出了州中的节度判官,可以算是展示善意了。

迎接的流程几乎就是定式,韩冈本人都懒得多说废话。也就靠了知情识趣的节度判官竭力奉承,才让场面不至于冷下来。

在忻州之北,结束了一系列可以算得上是繁琐的仪式,韩冈的一队人全都是重新上马。

行不了多久,迎面忽而尘头大起。韩冈身边的护卫顿时就紧张起来,但派在前面的探马提前一步回来。

看着几名探马,韩冈笑着大声道,“不用担心,是自己人。”

出现在韩冈一行队列前方的骑兵,大约三四百骑,多半是一个指挥。每五骑一横列,沿着官道一列列缓步行进,直至在韩冈马前站定。

停下来的骑手们,他们的战马也跟着停下来,安安静静,连队形都没有乱,一匹匹老老实实站得很是稳当。

整齐的队列,出色的控马,甚至将韩冈带在身边的府中精锐都比下去了。无论从军容军貌,还是从他们展现出来的军事水准上来看,都可以算得上是精锐了。

‘练兵倒是不差。’

韩冈暗自点头。刘舜卿并无多少能够夸耀的军功,却偏偏能在天子面前受到看重,依之立足的能力还是有的,算得上是真材实料。

想想赵顼,军中将领升迁都要到他的面前走一遭,刘舜卿尚能在其中脱颖而出,可见其人的能力的确有那么几分。

不过擅长练兵的将领,并不代表善于领军上阵。队形整齐的队伍,不代表他们克敌制胜。

京营的上四军,步操阵列漂亮得够资格站到后世的长安大街上。随驾出宫时,一队队骑兵步卒在御街上的行军队形,能羞得西军将领一个个都掩面而走。可一旦说起上阵打仗,西军可以用鼻孔看人。

如果赵括、马谡之辈,只让他们做个幕僚,说不定能成就一番事业,可惜的是,明明有才能的两人没被放对地方,害人害己,空留了千古笑名。

韩冈一时间变得很想见一见以胜擅练兵,声名鹊起的代州之守。

而刘舜卿当真就到了,带着幕僚和自家子侄,跟着他的四百骑兵一起,仅仅是前后脚而已。

刘舜卿五十上下,一副饱经风霜的摸样。其外在的气质,比起韩冈见过的种家兄弟、姚家兄弟,都有很大的差距,更不用说远在三种二姚之上的郭逵。

韩冈没有跟刘舜卿多寒暄。直接就问到了他最关心的议题,“辽人不断寇边,不知团练如何应对此事?”

听到韩冈的问题,刘舜卿的幕僚、子侄们都紧张起来,生怕有什么地方开罪了这一位,让他看的不顺眼。

刘舜卿脸上的神色一如往常,“下官曾移牒辽人,要他们将劫掠过去的人口归还。同时要他们交出凶手。不过辽人那里没有动静。故而下官也就扣下了两名辽国商人和他们的货,让他们拿凶手来换。”

这算是对等报复吗?

韩冈很是意外刘舜卿竟然保持着强硬。“这什么时候的事?”他问道。

刘舜卿迟疑了一下,道:“……就在昨日。”

说完他忐忑不安的望着韩冈,他的幕僚、子侄也都在关注着韩冈。新任经略使的回答,将决定刘舜卿的命运。

众目睽睽,韩冈脸色沉郁,双眼只在代州众人的脸上扫过,每一个代州人的信都提了起来。最后她忽的哈哈一笑,板起的面容犹如春风化冻:

“做得好!”

刘舜卿倒是愣了,韩冈的回复未免太过于简略和直接。。

“做得好!”韩冈又强调了一遍。他也知道自己没将话说清楚,所以进一步解释道:“以妥协求和平,则和平不可得。以斗争求和平,则和平可致也。要想太平度日,许多事只能睁一只眼闭一只眼,唯独对于辽国和西虏,却是一点也不能手软。”

韩冈的回答让刘舜卿好一阵都没有开口,最后才点头:“……经略所言甚是,蛮夷都是畏威而不怀德,不可对他们退让半分。”

“龙图果然不负知兵之名。”刘舜卿的一名幕僚插话道,“‘以妥协求和平,则和平不可得。以斗争求和平,则和平可致也,这句话可是深刻入骨。”

“倒也不是我说的,乃是先贤之言。”韩冈并不解释是这句话是出自哪一位先贤,“先圣不也说过吗?当以直抱怨,委曲求全的心思要不得。”

韩冈的表态,为刘舜卿的行动做了背书,不仅刘舜卿带出来的部下、子侄紧绷的神经松懈了下来,一个个露出了如释重负的微笑,就是这位已近五旬的宿将本人,也算是松了一口气。

新上任的经略使,因为年轻,必然气盛。加之世间流传的一干传闻为佐证。当不会甘愿受辽人之欺,多半会针锋相对。这是韩冈上任之前,刘舜卿就与他的幕僚、子侄一起推断出来的。

但韩冈到底为人如何,对自己自作主张扣下多名辽国商人的举动如何看待,以及在天子那里接受了什么样的命令,在亲眼见到韩冈之前,就是刘舜卿自己心中也没有底。

如今终于确定了韩冈的性格、为人,以及应对辽人的基本观点,作为下属,刘舜卿知道自己可以放心大胆的去做事了。

韩冈则是对刘舜卿的行事很是欣赏,甚至有几分惊喜。

之前他在太原收到的代州急报中,并没有发现刘舜卿对辽人犯界行凶的应对和处置。韩冈因此而对刘舜卿有了成见。认为他凡事上请,必然是个坐视辽人犯界烧杀,而不敢正面应对的庸人。现在一看,倒是个敢作敢为的。所以刘舜卿的一些小心思,韩冈也就大大方方的放过了,只当看不见,没打算去计较。

不过刘舜卿在这件事中所表现出来的仅仅是性格,其能力究竟如何,却不可能从这点小事中看出来。现在唯一能确定的,也就是刘舜卿精擅练兵的名声并非虚传。

但韩冈并没有为此而多费心神,在刘舜卿的陪同下,穿过代州城南门,缓步进入城中。

反正李宪就要领军回返,到时候依照之前的约定,将守御北方的事情交给他也就是了。有足够的兵力和武备,而且是以防守为主,那么将领的能力就算差点,倒也不用太伤脑筋。

