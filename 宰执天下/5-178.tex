\section{第20章 土中骨石千载迷(七)}

韩冈端起茶杯请抿了一口,心里只觉得好笑。

就是以他本人算不上高明的儒学水准,也知道叶涛的质问完全是个笑话,是为辩而辩。韩冈看得很清楚,叶涛旁边的陆佃,眉头也在皱了一下。

占卜占筮怎么就能将分得这么清楚?儒门从夫子开始,哪一家不是将卜筮合在一起说的?太史公的史记里还有一篇《龟策列传》,龟自是占卜用的龟甲,而策便是占筮时所用的蓍草。《左传》中,筮不吉则卜,卜不吉则筮,这样例子可也有不少。

《尚》夏商周,《诗经》三百篇,其中有关商周旧事多如牛毛。且别说儒门经典,先秦诸子又有哪一家能将殷商丢一边?研究甲骨文,必然要联系到先秦的一干典籍。不用考据,只凭逻辑,韩冈就能这样确认。

在这一方面,甚至东方和西方也不会有区别。比如唯物辩证法,向上是黑格尔,再追根溯源,应该还能上溯到古希腊的一干哲学家,亚里斯多德、苏格拉底什么的。几何学则是《几何原理》——这是韩冈最近让大食商人去找的。后人总是踩在前人的肩膀上,西方的那位大贤早就说过了。而同一时代的不同学术之间也都会有着联系,化学和物理之间,瓜葛有多深?

用孤立、静止、片面的观点解释问题,看不到世界的事物和现象之间的普遍联系和变化发展,这种思维方式似乎有个名号……叫机械唯物主义。

韩冈如今再一次深深的感觉到,当年让他睡了不少好觉的课程,学的时候没人放在心上,但回到现实生活中,却每每能够得到印证。

当然,这番辩辞是没办法对陆佃和叶涛说的。但韩冈在儒学上的水平也没差到张口无言的地步,况且他早有所备,可不惧人质疑。

“凡事皆有传承,武王伐纣灭殷,文王还是殷商时人。周公秉政,亦是在周初。《周易》的卦辞爻辞中,无论如何都不可能抹得去殷商卜辞的痕迹。任何学派都有其源流,从文字到经籍,皆是如此,何独《周易》能例外?”

“圣人生而知之,《周易》何须用商法?卦爻之中只见蓍草,何曾见龟卜?”叶涛辩道。

“生而知之,不代表不学。难道孔子不是圣人?他问礼于老聃,学琴于师襄,又是为何?”韩冈道,“博采众家善者而学之,此乃圣人之德也。文王演《周易》,如何会将夏商二代之易摒弃于外?”

“空言无证。”叶涛一口咬死,绝不改口。旁边的陆佃几次欲言又止。

韩冈估计这一位是辩得糊涂了,否则绝不会忘了《周易》中的内容:“周易的卦辞爻辞之中,牵涉到殷商和周室先人故事的条目所在甚多。别的不说,泰和归妹两卦中的帝乙归妹,说的是哪一家的事?”他轻哼了一声,“谁敢确认说,殷墟遗迹之中,一片片龟甲牛骨里面,就没有能够印证这一条的卜辞?!殷人卜不胜不出,用兵、出行乃至婚嫁,可都是一样要占卜的。\footnote{利用甲骨卜辞对周易卦辞爻进行印证,甲骨文的发现者王懿荣似乎就说过,但记不太清了。在个人记忆中,能确定的最早应是顾颉刚在二十世纪初时出版的《周易卦爻辞中的故事》,帝乙归妹的考证便在其中。甲骨卜辞对更多经典的作用,则可以参看王国维有关殷墟卜辞的一干著作。这是早期的,如今则更多,殷人卜辞和先秦典籍之间相互印证是研究中国上古史最重要的线索之一兴趣可自去搜索,在此就不多做科普了,例子实在是举不胜举。但他也不便反驳,王安石对孔子为注《易经》而著《十翼》,一直都抱着深深的怀疑。}”

“但也不能说一定就有可作明证的卜辞。”

“所以才要去将殷墟发掘出来研习揣摩。”韩冈笑着说道,“暴秦焚坑儒,先王之文不之传也,家岳惜之,韩冈亦惜之。如今更近于先王之文的殷墟遗物重新出世,可谓是国家和儒林的一大盛事,亦弥补了家岳和韩冈的遗憾。”

王安石在《字说》序中说‘秦烧《诗》《书》,杀学士,变古而为隶,是天丧斯文’,而他考订字说,是‘惜乎先王之文缺已久’,‘天之将兴斯文也’。这自许之言,是王安石撰写《字说》,以之占据道统的依据。韩冈这么说可就是拿着王安石的原话来给人添堵。

陆佃深吸一口气,似是在强忍怒火。身为王安石的弟子,却亲耳听着王安石的女婿隐带讽刺的说话,得费尽心力,才能克制住燃烧在心胸中的熊熊怒意。

叶涛一般的心中盛怒,声音低沉的问着韩冈:“若是《字说》之论在甲骨文中得证,玉昆你又当如何?”

“难道在致远心中,韩冈是知错不改的人?”韩冈故意反问。

当然!陆佃和叶涛心中大叫着,但哪里敢明说出来,只能低头:“不敢。”

韩冈笑了一声:“殷墟重现人世,正是‘天之将兴斯文也’,可与汉景帝时,圣人故宅夹壁中的《古文尚》现世相提并论,甚至犹有过之。不论之后是印证了《字说》之论,还是得到相反的结论,只要能传承先圣之学,对儒门来说都是好事。若《字说》能得以明证,韩冈愿俯首就学。”

韩冈如此说着,神情中则是带着自信。

王安石可是为了一统儒门才撰写的《字说》出来的。眼下只要新学一脉去研究殷墟甲骨,那便是他韩冈和气学的胜利。而且韩冈也绝不会认为《字说》得到甲骨文的印证。能不能印证,并不是新学一脉说得算的。

瞥了桌上的甲骨一眼,陆佃心中暗叹,韩冈定然是胸有成竹,才敢如此放言。

王安石的《字说》,本来是要取代所有前代的训诂解字之,成为新学一道德的基石,现在却要证明中释义的正确性,这不正是落入了韩冈的陷阱了吗?不唯、不唯上、只唯实,这正是韩冈一直在提倡的格物!

过去韩冈的学问近于自然之道,使得儒门中人都小瞧了他,认为他所学只得一偏,可如今就渐渐图穷匕见了。但从道理上,各家拿着经典自述己见,当然比不上一件三代遗物来得更有说服力。

仗着殷墟遗物带来的优势,如同滚刀肉一般的韩冈,让陆佃和叶涛一点办法都没有。韩冈目的就是搅混水,让儒林去争论,除非一口否定殷墟的,否则只要去研究。

但能否定吗?

天子绝不会出面的,只有新学为朝廷同风俗一道德,没有天子为新学冲锋陷阵的道理。而政事堂中,王珪亲自将韩冈送到。

若是王安石和吕惠卿有一人在京中,眼下的局面还能好一些,可是让国子监中的一干学官跟韩冈放对,未免太过强人所难了。叶涛和陆佃都有力不从心的感觉。光是地位上的差距,就让他们在辩论上落尽下风。何况韩冈本人一副无欲无求的样子,搅了浑水就坐下来看热闹,不在殷墟古文上争高下,让人根本无从着手。

不是韩冈小瞧陆佃和叶涛,他一直以来的目标都是王安石或二程、司马光这样的人物,始终都是小心谨慎,半点破绽都不敢露,绝不会在那几位擅长的领域中与其较量。而当他对上陆、叶这等籍籍无名之辈,凭借自身的地位和身份,辩论的时候可是天然的占着优势。想说什么,都不用太多顾忌,可以明着欺负人。

“卜为象,筮为数。物生而后有象,象而后有滋,滋而后有数。周易卦爻虽说尽为占筮,卜的是数,但何以以‘象’为名?”说到这里,韩冈微微一笑:“且系辞亦有云:易者,象也。这里的易,仅仅是周易吗?”

韩冈扯淡的话,终于让陆佃忍不住了,勃然作怒,“岂有此理!”

叶涛亦怒道:“此言大谬!”

但陆佃和叶涛的愤怒,韩冈如同清风拂面,根本不在意。反而又端起茶盏,悠悠闲闲的再呷了一口。

六经注我的风气在这里,韩冈硬说《周易》中不称卦数而称卦象,是因为承袭重卜的《归藏》,陆叶二人的愤怒归愤怒,之后世人的驳斥归驳斥,但他说出来照样没问题。

韩冈对如今的学派之争,本来就是希望越热闹越好。儒林围绕着殷墟遗物闹上几十年乃是他梦寐以求的心愿,煽风点火、兴风作浪、火上浇油、推波助澜,只要能将气氛炒热起来,有些事韩冈也不介意多做一做。

反正他的根基可不是在儒门经典上,这是其他学派截然不同的地方,从一开始他就立于不败之地,无论局面向什么方向转变,对韩冈来说都没有差别。争得越厉害越好啊,这样才有趣。

让儒生们皓首穷经去好了,韩冈会在这段时间里继续发展他的自然之道。当格物对社会对生产的作用越发的显现,便是气学将道统攥在手中的时候!

