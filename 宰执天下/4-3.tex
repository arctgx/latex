\section{第一章 纵谈犹说旧升平(三)}

【这些天状态有些差,今天只有一更了,明天应该能恢复。】

李逢谋反案,说起来并不是什么大案,没有揭竿而起,也没有私藏铠甲兵器什么的。这件事说来也好笑,就是正月的时候,京东沂州一个叫朱唐的平民,首告前余姚县主簿、徐州人李逢密谋造反。

京东提点刑狱王庭筠受命去查证,找到的证据也只能证明李逢说了些诽谤朝政的话。而出首的朱唐,他的动机则很可疑,一是因为有旧怨,另一个原因,就是贪了首告谋叛的赏钱。

这样的案子,也并不少见,但以陷害为多。反逆之言,哪个没说过几句?怨怼也好,谤讪也好,只要不是当真是做出了谋反之事,都可以一笑了之。尤其是说文官谋反,更是个笑话。文人造反不是不可以,但也要他有这能耐才成啊……

所以王庭筠给出的判决是两人都编管发配。李逢‘谤讟朝政,或有指斥之语及妄说休咎。虽在赦前,且尝自言缘情理深重,乞法外编配’,而首告的朱唐‘告人虛妄,亦乞施行。’

但事情的发展不想王庭筠所想,而是变得激烈起来——只因为赵顼不肯接受这个判决。

派去审案的王庭筠照老规矩要息事宁人,赵顼却是不依不饶。又加派了一名御史蹇周辅去陪审。这蹇周辅秉持了天子的心意,将案子往大里操办。也就在前两天,不仅将李逢谋反的罪名给敲定,甚至还将打击范围扩大,把一大批官员都括了进来,甚至还包括一名宗室。

其中有几分为真,几分为假,那就难说了。

三木之下什么供状得不到?周兴、来俊臣的手段,如今诏狱之中也不是没有人承袭下来,要收押的犯人攀咬谁就攀咬谁,这点手段一点都不稀奇。

现在李逢攀咬出的宗室赵世居是太祖的四世孙,右羽林大将军兼秀州团练使。另外还有试将作监主簿张靖;做医官的翰林祗侯刘育;最后一个是出自司天监的学生,似乎是姓秦,叫什么韩冈给忘了,反正司天监这个身份,掺和进了谋反案中,就决定了他绝不会有好下场。

方才殿中又争吵一阵,韩冈也不知道到底是谁赢了,反正天子终于点头,换上了知制诰沈括和同知谏院的范百禄代替蹇周辅。

看蹇周辅行事,说不定就是一个来俊臣,能换上沈括和范百禄多半会好一点,至少以两人的性格不至于妄起大狱。就是不知道御史中丞邓绾会是什么想法。据韩冈所知,他跟范百禄关系不睦,而沈括一个外人掺和进御史台中事,说不定会引起他的反弹。

不过随着韩冈趋步进殿,李逢一案便被他抛之脑后,此事与他无关,他更不愿掺和。

但赵顼显然是方才被几位重臣给压得苦了,见了韩冈就抱怨了起来:“韩卿,朕向来待宗室不薄,想不到竟然还有人心怀不轨,甚至搜集图谶、兵书,《星辰行度图》、《攻守图术》,这两本书,也是宗室家该有的?”

若说起大宋到赵顼为止的几代皇帝,哪一个最不得宗室所喜,赵顼肯定能夺冠,而且能将第二名抛下三五圈之多。这叫待宗室不薄?

而倒霉的李逢和赵世居因为一本星图,一部图谶,而将叛逆的罪名给坐实,也只能怪他们自己太不小心。这也是为什么韩冈不想自己将望远镜拿出来的缘故,与天文扯上关系,等于是将把柄送给人。没有追究时,那便无事,可一旦开始追究,就是罪名——实在太危险了。

韩冈绝不想插言此案,而且宗室对赵顼的怨言,也是因为新法。直接跳过赵顼对这个案子的抱怨,只拿着图谶说事:“谶纬之学,背于六经,以文其私说,杂以图记,证以占验。天行有常,岂在图谶?!此物如今多为妖言惑众者所用,陛下当施以重责,以戒后人。”

“天行有常,这可是荀卿之言。”赵顼听着就笑了起来,倒忘了方才的抱怨。

韩冈传习的关学算是思孟一派,这点赵顼是知道的。引用荀况的话,听来未免就有些滑稽了。

“荀卿一脉亦源自先圣,并非全然无理。单只是天行有常四字,就是至理。”

其实韩冈对荀况的‘制天命而用之’这一句话,还是很有几分认同。如果将天命解释成自然规律,可以说得上是唯物了。而韩冈也希望关学能从天人感应这四个字中解脱出来。

赵顼笑道:“若依韩卿所言,司天监可算是无用了。”

“推算历法,考订节气,司天监之言可用。但若以星辰之变,妄说吉凶,则无用。”

韩冈的回复,一棒子就把司天监的日常工作给打没了。赵顼只觉得有些好笑,在这一点上,韩冈跟他的岳父是一个脾气,“可是天变不足畏?”

“民心即天心,可畏者民也,非天也!若陛下勤政事,抚黎民,天变何足畏?若是荒于政事,耽于嬉乐,以至民不聊生,纵使祥瑞频出,又岂能不畏?”

“韩卿此言是正理,朕当记之。”类似的话,赵顼听得多了,随口就应付了过去。

对于韩冈,他还算是信任。毕竟韩冈能造出送人上天的飞船,却不用来迷惑世人,而是直接说破了其中的道理,让世人知道此事只是寻常而已。这样的臣僚,可比整天拿着上天来恐吓天子的大臣要让人舒心得多。

“若是朝臣皆如韩卿,朕也可安心。”赵顼感叹着,“偏偏李逢等人,坐食朝廷俸禄,又无功于国,。”

赵顼又像怨妇一般喋喋不休起来,似乎是对赵世居和李逢谋图不轨之事,在心中放得极重,可在韩冈看来,赵顼纯粹是因为心虚而变得话多。

李逢的错不在他说得那些悖逆不道的话,也不在交结宗室,私藏图谶上,而是在于他说话的时机。

若是有人刚刚生了儿子,上门道喜时却说‘怎么你家的儿子跟你不像,反倒跟你家邻居阿三很像?’那他挨打也是很正常。朝廷刚刚割了地,却说若太祖皇帝在位必不致于此,这不是让天子难堪吗?

这等丢了祖宗脸的事,赵顼恨不得天下人都给忘掉。可李逢的话正好戳中了赵顼的痛处,当然是一头撞到了枪口上。另外李逢还在去年的大灾时,说天降灾祸是朝廷德政不修——其实这也是当时人人都有说的——但如果要罗织罪名,这也能算是一条——妄说休咎。

既然做了就不惧他人议论,这等厚脸皮,赵顼是没有的,只因他心虚,所以只能自欺欺人。不但李逢倒了霉,跟他有来往的赵世居也一并倒了霉——说文官谋叛,有些说不过去,但勾结宗室就是铁打的罪名,赵世居可说是无妄之灾。

这等没来由的大案,最后的结果只看天子的心情。赵顼的心情顺了,当个屁放掉都可以;若心情不顺,那就同案之人一起赴黄泉。

不过以韩冈的评判,仁宗皇帝的好脾气,赵顼肯定比不上。仁宗皇帝能给写反诗的老秀才一个官做,但赵顼绝不会原谅戳他痛处的官员,涉案之人的结果也就可想而知了。

韩冈只希望这件案子不要牵扯得太厉害,否则这样的瓜蔓抄下去,也不知道谁会给抄出来。按照后世的说法,世界上的任何一人,都可以只通过六人连接,就与另外的任意一人拉上关系。而在北宋的官场上,中间的传递点还要减个两三层。

抱怨了一阵,赵顼终于想起了今天他找韩冈来崇政殿到底是为了什么,“韩卿,你前日上书欲以水轮机驱动锻锤,朕已询问过多人,好像不是很方便啊!”

“水、风、人、畜,这些都能给机械、车船提供动力。若无动力驱动,不论是车、船等出行代步之物,还是磨、碾等农具,都是一架死物。而在水、风、人、畜,这等动力之源中,以水利最为便利,也最省成本。否则水碾、水碓不会大行于道。如果能将如今作坊中的以畜力,可以用上更快更重的锻锤,能让打造甲兵的速度再加快一倍,使军器监中成本大大降低。而节省下大量的人工和时间,还可以作为官营铁坊,打造农具、器物,其利不在少数。”

“但京城的水流当用不起水轮机。”流经开封城的河流,基本上都是运河,没有多少可供水力利用的能力,这一点,赵顼已经向苏颂询问过。“难道韩卿准备将板甲局的作坊搬离京城?”赵顼可不喜欢这个主意。

虚外守中是国策,韩冈并不指望他能说服赵顼,将几个重要的军器制造局搬离开封府,不过郑州如今已经划归京畿,也算是开封府的地界:“不同于其他军器,如板甲、斩马刀、神臂弓等作坊,的确不宜离开京师。只是旧郑州河流众多,当有几分可用之处。”

“旧郑州有梅山、嵩渚山,为须水、索水诸水之源,如果将作坊设于密县、新郑和管城,的确能派上些用场。但这三县水运不稳,比不上京城通畅。”赵顼对水运有着清醒的认识。徐州铁从五丈河运抵京城,而河东石炭则沿着汴河而来,论起交通便利,旧郑州有河流发源的三县远远比不上京师。

这一切,赵顼能想到,韩冈当然也都考虑到了,他说得可不是那几条小河:“陛下,汴河亦流经旧郑州。如果能将汴口以东的官营水磨作坊撤销一部分,就可以用来安置工坊。至于官中损失的收入,完全可以由铁器作坊来补足!”

