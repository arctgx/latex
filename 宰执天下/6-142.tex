\section{第13章 晨奎错落天日近(七)}

“不需要岁币?!”

王安石无法再保持沉默,韩冈简直是在说笑话。

三十万两银,二十万匹绢,相当于百万贯的收入。羊皮裹身的契丹人,会甘愿舍弃这么一大笔财货?

三次为相,五度宣麻的王安石,对朝堂财计如何不清楚。即便是在岁入数千万贯的大宋,价值一百万贯的银绢也是一笔极大的数目。

当年为了补充国用,连渡口都买扑出去,锱铢必较,连几百上千贯的花费都要精打细算,莫说一百万贯,放在十年前王安石做宰相的时候,谁要跟他说有一笔岁入十万贯的买卖,他也是要红了眼的。

可韩冈说得斩钉截铁,王安石想笑,笑容却凝结不出。

……………………

“上位禁军一员,一岁俸钱十八贯,绢绸布帛十匹,丝绵半斤。五十万匹两银绢,足以养上两万精锐禁军。放在辽国,收买的部族至少能抽出十万丁口。”章敦咧开嘴,似是在笑,似是发狠,“所以吕吉甫才不担心辽人不主动攻过来!”

到了这一步,河北那边差不多已经是图穷匕见了。能坐在两府这个位置上的,也大多都已经看透了,章敦自不例外。

只是他还是为了自己之前被吕惠卿和王安石所蒙蔽,一时间怒火中烧,直到回到家中,歇息下来,方才消退一点。

吕惠卿的确自知之明,并不是当真想要与辽国一较高下,不过是想利用既定的事实,给自己涂脂抹粉罢了。

如果仅仅是防御,辽军就算以举国之力南下,只要有十万兵马驻守边境,辽人也只能无功而返。正是凭着这一点,吕惠卿才敢高喊着北上北上,去抢终止岁币的功劳,完全不担心输了会怎么办。

章恺听得眼睛越瞪越大,“七兄,吕吉甫当真这么有把握辽人当真会南下?”

之前章恺还以为只要朝廷驳了了王安石和吕惠卿的提议,这一仗肯定就打不了。但现在听来,却是肯定要打的样子。若当真打起来,那这两年他在北方的一番布置,岂不是都要落空,损失可是要以十万计,这可不是用伤筋动骨四个字就能概括的。

“朝廷承不承认是小事,脸面也是小事。对耶律乙辛来说,只有岁币才是大事。”章敦冷笑道,火气虽收,可语气却难掩胸中的愤懑,“年年遣使大宋,大宋的街市再繁华,他也是看着吃不着,唯有一年一度的真金白银送来,才算是他自己的。”

“可高丽、日本刚刚打下来,收获无数。就算肯定会打,辽人也应该休整一年,明年再来才是。”

章恺皱着眉,他真心期盼耶律乙辛在登基之后,能在后宫里多宠信一阵子各族美人,给他一年半载的喘息时间。

若是有这一年的时间,足够他将家里在北方的生意给安排好,该卖的卖,该出典的出典,等战事结束之后再来捡个便宜。

“抢去再多贫瘠之地又有什么用?千里蛮荒之国,百万化外之民,即使有再多,能比得上中国的一州之地?两战虽是得人得地,可是没有得财。你还不明白?辽人根本就不缺人和地,缺的是养兵的财货!”

章恺脸上的焦急不见了,迎上章敦的视线,沉声问道:“这么说辽军肯定会来?”

章敦叹了一声,点了点头。

“那王平章到底是怎么想的?!这件事他一开始就知道了吧?!”章恺的声音中充满了愤怒。

如果王安石还把章敦当成可以倚重的助手,就不该在这么大的事上瞒着他。可是王安石偏偏帮吕惠卿瞒着章敦。章敦若是为此离心离德,王安石怨不得任何人。

“就是想不明白。或许怕为兄跟吕吉甫争胜。”

章敦冷着脸。

王安石如今最担心的就是新学,而不是他的位置。或许在王安石的眼中,只有吕惠卿才是交托衣钵的传人,三经新义之中,甚至还有吕升卿的手笔,却没有章敦的。

“韩三那边呢,他想到没有?”

“现在肯定想明白了。”

章敦基本上能够肯定,韩冈之前没想到这一点,吕惠卿表演得实在是太像那么回事了。之前要是想透了,韩冈的应对绝不是强硬到底的反对。

“既然现在他知道了,那他会不会知会……”

章恺说到一半便停了,反手指了指北方。

“那么聪明的人,会将这么大的把柄送给契丹人?他为了什么?”

“……气学?”

章敦笑着,然后摇头。

为了坚持大道,不惜牺牲生命,这是值得尊敬的先贤。

可一旦韩冈与辽人的勾结败露,气学可就完了。不惜牺牲生命和名声,顺便还将学派的未来给押进去,韩冈有蠢到这个地步吗?

就是为了权柄,也不需要勾结辽人,他现在离宰相之位只是时间的问题。

“韩三都没办法,那这便宜就让吕惠卿占了?”

十万贯不是小数目,由此连带的损失,现在还算不清楚,但只会更多。章恺看起来已经平静了,其实依然心痛如绞。本来朝廷只是囿于正统,拒绝承认耶律乙辛的身份,可还是有和平解决的可能,但吕惠卿在里面作祟,这一下子,大宋与辽国之间再无缓和的余地,大战一起,之前在北地的那些投入,都得灰飞烟灭。

“这个便宜就让吕吉甫占去好了。”章敦眯起眼睛,笑得开心极了,“等他回来,会知道现在的这个朝堂跟他离开时不一样了。”

……………………

耶律乙辛之前做权臣的时候,由于不得人心,不得不通过战争的胜利来维持自己的地位。

而现在他成了皇帝,反对派也已经给他清除得不剩多少,只要再用些财货就能维系住他的人望。

“一手拿刀,一手拿钱,如此便能有人心。”韩冈如是说道。

“然后,”王安石听着韩冈的话,笑了起来,“玉昆你说,耶律乙辛他不需要岁币?”

“当然。因为耶律乙辛他有钱!”

韩冈一口咬定,却又不说明,总是绕弯子,王安石渐渐不耐烦起来,“哪里来的钱?是仿效玉昆设铸币局,还是学了你的钱源论,准备发行国债?”

韩冈正想说话,一阵脚步声传来。是一队禁卫巡视至此。看见王安石和韩冈在说话,远远的行了一礼,然后绕了开去。

等他们走远,韩冈才道:“都不可能。辽国铸钱一向不多,矿冶也少。而且辽国朝廷,若以信用论,远远不如中国。没有信用,如何发行国债?”

“那玉昆你说,耶律乙辛的财货到底从何而来,以至于让他连百万贯的银绢都看不起?!”

韩冈依然没有直接说出来,他反问道,“不知道岳父可曾看过小婿的《桂窗丛谈》?”

王安石没有回答,一双眸子反射着灯火,牢牢盯住韩冈。

“书中倒数第二卷,是外国的风物,主要是道听途说。”

“是日本还是高丽?”王安石问。

“倭国多火山。火山,地之裂隙。地下有高热,金石化液,如冰下之水,奔涌不息。往往于裂隙处喷薄而上,积于地面而成山。山为金石所凝,故而多矿藏。”

“倭国多硫磺,亦肇因于此。”王安石将韩冈的话接了下去。

他当然记得韩冈在《桂窗丛谈》中写得那些轶闻,韩冈方才的复述与书中有异的几个字,他甚至还能辨别出来。《桂窗丛谈》从题材上只是私人笔记,表面上看不过是搜罗了一些奇闻异事,以及韩冈对这些事的解释,由此集结成册。但实际上,这部书,已经是气学一脉中的根本教材,

“此等秘闻事关军国,怎么能公布出来?辽人攻日本,当有玉昆你的一份功劳。”

“当时还没有火炮。而且更重要的是小婿可没说。比如金银矿,以及铜矿。五金之属,只有铁最难熔融,而金银铜则要容易上许多。故而从火山之中涌出的矿藏,少铁而多金银。这一条,小婿从来没在哪一本书里写过。但此事小婿去不写,耶律乙辛占据日本之后,难道会不知道?”

辽国矿冶之术,不下于中国,远胜于那一干岛夷。若辽国的炼银之法用在日本,一年百万两银,岂是难事?百万两银在手,还有金矿、铜矿,耶律乙辛每年手中能多出三五百万贯的财货,他又怎么会明知道吕惠卿在激怒他,却还会为了区区百万贯银绢,怒而兴兵?

“吕吉甫大喊着要攻辽。若是辽人并不因为岁币来攻,他是准备继续往辽国境内杀过去吗?”

有了钱,就有了控制力。如何对付大宋,在耶律乙辛手上就有了更加充裕的时间,战略上也有了更多的回旋余地。而一开始只准备迎击辽人进攻的吕惠卿,怎么可能应对得了这样的局面?

“吕吉甫能不能赢,韩冈不知道。可换成是韩冈,绝不会冒这风险!”

王安石紧紧皱着眉头,没有注意到韩冈告辞离开。

韩冈回头,王安石犹在灯下。

自己说得太多了,可有用吗?

韩冈摇摇头,根本不可能!吕惠卿都做到了这一步,已经不可能有退步的余地了。
