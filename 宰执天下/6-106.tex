\section{第11章 飞雷喧野传声教(三)}

十八门火炮喷吐着火焰。

当一门火炮刚刚轰鸣过,下一门火炮紧跟着就接了上去。宛如罕见的夏日雷暴,不断震撼着双耳。

即便这些小炮摆放下来,还不到王居卿膝盖的高度。可耳畔已经持续了半刻钟之久的轰鸣,还是让他不禁心旌动摇。

装填了十几颗铅子的散弹,由铁砂组成的霰弹,还有单独一个的铁球弹,甚至最简单的石子,都被装填进了炮口中,隔着棉纸托压在药包上,然后点火发射出去。

新换上的标靶,转眼就从光洁一片,变成一片片的麻点。在独头铁球弹的轰击下,靶上木屑横飞,有几块被击中了边缘或角落,登时就多了一个缺口。

每五名士兵操纵着一门虎蹲炮,其前后进退皆有制度,井然有序。各个炮组的装填和发射的手法比较起来,看不出有什么区别,速度也几乎相当,显然经过了一番严格的训练。

十八门火炮的炮组加起来才不过百人,炮阵的宽度却长达七十余丈,近一百五十步。按照步军的标准,是稀疏了一点,而如此宽松的阵列,什么骑兵能通过这样的防线?

王居卿看过火炮的发射,但他没有看过如此之多的火炮同时射击。今天实验的结果尽数落入他的眼底,集结成军的火炮的威力,让他咋舌不已。

待实验稍歇,他向军事经验十分丰富的韩冈征求意见。

“只要火炮还能够发射。只要骑兵还是血肉之躯,就不可能通得过。”韩冈极有信心的说着。

“一个都通不过。”方兴也重复强调着道。

“可惜一个指挥最多就只有十门。”王居卿知道预定中的火炮编制,每一个百人都装备两门,一个指挥五个都,也就十门而已,“根本就不够用。”

“指挥使手上还会留上六门火炮,加上人手一柄的神臂弓,足够应付任何敌人了。”

“这样的一个指挥有多少人?”王居卿问道。

即便是西军中的精锐指挥,其人数也只能按军籍簿上的八成来计算。每个都实际上仅有七八十人,配上两门火炮,就是少了八分之一的步卒。而指挥使手中,也就掌握一两个队,作为护卫和传达号令。要是指挥使手上再加六门火炮,这三个队三十人从哪里来?

更不用说那些吃空饷吃到一半的普通指挥了,连使用火炮的足够人手都找不来。

“当然是满编,比五百人还要多一点……过些日子,要跟章子厚商议一下。”

“满编?比五百人还要多?”王居卿太了解军中的情况了,他管过修堤水利等事,禁军、厢军打过交道。这是要把军中的编制做大改的架势。

韩冈不介意对王居卿多解释一点:“寿明有所不知,京中率先装备火炮的军额,只有神机军和上四军。神机军的吃穿用度都是比照上四军来安排,两边用不着吃空饷。至于其他地方最先装备虎蹲炮的步军指挥,河东、河北,刚刚经历大战,都有几支拿得出手的队伍。而关西,想来寿明你也知道,更是用不着担心空饷问题。”

上四军由于要在天子面前走动,又经常校阅,人少一点一眼就能看出来,空额的情况比西军中的选锋都还要好。

而西军正要裁汰部分兵员,很大程度上也是要挤水分,将只存在于军籍簿上的士兵给清除出去。为了不削减太多的实际战力,会将挤过水分的指挥合并,成为真正的满编指挥。

不过这样的指挥数量也多不了,吃空额早就在西军中形成了一个庞大的利益集团,一时间还扭转不了,如今也是借裁军为名进行整编。韩冈希望最后能够组成三十个到五十满编的步军指挥,作为初期装备火器的军队,而划拨给他们的军饷,也会提升到上位禁军,乃至接近上四军的水平。

韩冈之所以敢让黄裳去西南,正是因为一旦开始平定不顺服的西南夷,就会从关西调拨三四个指挥的新式禁军,作为核心主力。西军的战斗力,加上轻便的火炮是如虎添翼,对付起夷人来,把握更增加了几分。

方兴只是在旁听着,这些事跟他无关了。而王居卿却是有机会进入西府任职,枢密使和副使不敢去想,枢密院直学士、枢密院都承旨这些都是有机会的,没有放过接触更高端信息的机会,“京中、关西、河东、河北。如此说来,虎蹲炮需要的数量不会太少。”

“军器监还没收到枢密院的文函?”韩冈反问。

“还没有。”

“怎么这么磨蹭?”韩冈摇摇头,向王居卿透露道,“日前已经定下来了,虎蹲炮到年底之前至少要造出一千门出来,越多越好。火药与炮弹的制作,也要加紧。”

“这样啊,下官明白了。”王居卿转问方兴:“一千门虎蹲炮,火器局人手够不够?”

“判监放心,时间足够了,不会误事。”

王居卿自上任后,只来了火器局一次,韩冈特地带王居卿过来,正是不想让火器局游离于军器监之外,也让王居卿不要有太多的顾忌,方兴对此也心领神会。

“这么有把握?”

王居卿知道方兴不敢在韩冈面前胡吹大气,只是想知道这么有效率的原因——火器局中的细节,他没有多打探,但人员数量,他这位判军器监还是清楚的。

“现在局中是用铁范铸炮,比泥范更快了许多,也少了许多重新制模的人工。”

“铁范铸炮?”

王居卿没听过这种说法,但他能理解其中的含义,只是理解之后,就更为迷惑。只听见方兴说,“泥质的模具用一次就毁一次,而铁质的模具就没这么多问题。”王居卿立刻就追问:“两边都是铁,不会熔在一起?”

“不会。”方兴说道,“铁范里面涂了一层隔热的灰浆,铁水灌进去后,不会熔在一起。而且用铁范造出来的火炮有个好处。”

“什么?”

“就是大小如一,绝无沙眼!”

这时候,火炮阵地上的硝烟已经散尽,方兴引着韩冈、王居卿上前,十八门虎蹲炮,一门门的查看过去。

“这些虎蹲炮都是一个模子里铸出来的。”方兴为两名顶头上司解说着,还让炮手将三四十斤中的火炮给抱起来。

虎蹲炮真要说起来也不重,用木架子装好背起来,走着不会太碍事。若是走远路运输,可以用独轮车一边一个,也可以干脆架在牲口身上上,一边两个,加起来才一个胖大汉的重量,一个人抱着也不吃力。

方兴让韩冈、王居卿两人凑近了看,“参政、判监请看,炮膛内不需要多磨,就跟镜子一样滑顺。”

虎蹲炮的色泽深黯幽沉,炮膛虽是满手灰,但用长杆刷子往里面擦了几擦之后,就变得十分光滑。虽不能说如镜子一般,可也跟刀剑、板甲的表面差不多了。

王居卿很细致的一门门查看,大小的确一个模样——本就是一个模子出来的,而且从近处看过去,做工也十分细致。就是不知道这是特意拿出来看的,还是之后大批量制造还能保证现在水准。

不过这番心思,王居卿也不会说出来,说出口的就是赞美。好一通夸之后,他才问方兴,“铁范铸炮是早已有之,还是近来的发明?”

“是军器监中的一名工匠发明的,姓徐名良。这徐良原本是造板甲的匠人,后来被调去造铁锅。铁范铸器就是他在造铁锅时发明的。”

“铁锅?”王居卿听着发楞,这两个差得未免也太远了。

方兴点头笑道:“正是为了造铁锅,才有了铁范的发明。”

“若居卿记得没错,板甲是锻造的,调去做铁锅,也应该是锻打才对,怎么改铸锅了。”王居卿扭过头,对韩冈说道。

“这不是没办法吗?”韩冈无奈叹道,“那时候军器监中用不了那么多工匠,又不能解雇了他们,让辽人占了便宜去,只能硬着头皮上了。锻也好,铸也好,其实也差不多。”

斩马刀和板甲在发明之后,只用了三年多的时间便给六十万禁军全数换装完毕,接下来就只需要正常替换损坏的甲胄,不需要一年二十万套的打造。为了给剩余的产能一个去处,韩冈给出的意见就是军转民,将造军品的工匠拉去造民用的器具,让天下百姓可以用上廉价的铁制品,也可以继续扩大京中铁场的规模。

但这样的转产,总少不了有些让人啼笑皆非的事。造甲胄的工匠改行去铸锅,这已经是再正常不过的调动了,虽不比造斩马刀的工匠改去打造菜刀那般适任,但总比原本造皮甲的匠人,改去将作监给军中缝皮袄要强。

“而且他们做得也不错。铸锅就不说了,军中和衙门里的厨房所用的六耳铁锅,从二三十斤,到一百多斤的都有。又有内径不到一尺的双耳铸铁饭锅,连盖子都用的铸铁,烧汤、做饭比过去都好。而且还有了纯粹锻打出来的铁锅,厚仅数分,坚韧轻巧,一手就能拿起来,最适合用来炒菜,如今外面的酒店,炒菜越做越好就是因为有了这锻铁锅。”
