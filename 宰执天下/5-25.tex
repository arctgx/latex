\section{第三章 时移机转关百虑(11)}

韩冈一番雄论,王旖沉默了一阵之后,轻声道:“官人这番话,该是在朝堂上说的。”

韩冈猛然间哈哈大笑:“这个道理,你当天子不知道,你当群臣不知道?多少人都是揣着明白装糊涂。为夫前面说过了,知易行难。道理人人都懂,可想要做到,不知有多少道难关要过。商鞅变法,秦人因此富强,法度就那些,都是摆在明面上的,为什么山东六国死到临头还不去学?依样画葫芦也成啊……实在是学不来!”

他摇着头:“岳父变法,还远远没有到商鞅的地步,就已经是天下沸腾了。这是利益之争,所谓善财难舍,有几人能舍小家为大家?为夫都没那么无私,只是想要做到公私两便而已。像秦孝公和商鞅那般杀得人头滚滚,山东六国做不到,岳父也做不到。所以眼下就只能跌跌撞撞,岳父的境遇,也与此有关。”

王旖沉默了下来,如今国势昌盛,按理说是自家父亲主导变法之功。连丈夫都说,没有父亲在朝中的鼎力支持,河湟开边不可能成功,没有新法富国强兵,交趾不可能平定。但父亲不及六旬就不得不出居金陵,日后回到京师的可能性也是微乎其微。

“所谓变法,从本质上说,就是改易利益归属。所以岳父说变法之要在于理财,就是这个道理。一旦变法,在一部分人得利之时,总会有另一部分人失去他们的利益。商鞅变法,得利者秦王,失利者则是一干卿大夫,无军功不得授爵,公卿大夫们哪能不恨商鞅?岳父的变法,得利者天子,失利者是谁,就不必为夫说了吧?”

“难道爹爹推行新法,百姓没有得利?”王旖惊讶的问道,“官人也认为爹爹是与民争利?!”

“与民争利的民和平民的民不是一回事。普通百姓能吃饱就不错了,仅剩的一点油水刮下来,说不定会官逼民反。有恒产者有恒心,没了产业家当,铤而走险就没了顾忌了。岳父何曾做得那么绝?岳父争的利,绝大多数都是来自人数只占小半的富民。但凡攻击岳父与民争利的,多是拉着与平民百姓为幌子,为个人私利张目罢了……”

“司马君实清介,从没听爹爹说过他品行上有过错。还有子厚先生他们,都不是谋求私利之人。”王旖很是正直的为人辩护。

“这里的个人私利,不是一个人的私利,而是他代表的一个群体的私利。也许作为赤帜的某人会很清正,但是他所要维护的那群人呢?就是子厚、天祺、伯淳和正叔几位先生,他们都是糊里糊涂的帮了人出来打旗打鼓。文太师不是说过吗?‘为与士大夫治天下,非与百姓治天下也。’”

“原来如此。”王旖对丈夫的话全盘接受了下来,“原来他们反对爹爹,都是为了一己私利……”

“在军国政事上,私德从来都是枝节。只要能顺便记得帮百姓一把,不是认为盘剥民力是理所当然,就已经是很难得了。”韩冈双目清冷,盯着前方的虚空,犀利如刀的眼神仿佛能扒皮抽骨,将人看到了骨头里一般,“可惜这样的士大夫实在是少。”

王旖不太喜欢丈夫现在的表情,勉强的转过话题,“那官人不喜苏子瞻的诗词,就是因为他说过出来做官就是为了享受?”

“谁说的,最近的诗作为夫还是很喜欢的,只是不喜他早年的作品。”韩冈辩解道,他前生所喜欢的东坡诗词,在眼下只出现了一半,都是出外任官之后的所作,“苏子瞻早年的诗词,也就只是有文采而已。同是咏明妃,他的那一篇就远比不上岳父之作,失之浅薄。”

同样是咏王昭君,王安石的两首《明妃曲》传唱一时,人人争相唱和,就是司马光都和了一首。‘君不见,咫尺长门闭阿娇,人生失意无南北’,‘汉恩自浅胡恩深,人生乐在相知心。’前一首,叹世事如一,无论中外;后一首甚至藏了良禽择木而栖,臣亦能择君的想法。而苏轼的‘谁知去乡国,万里为胡鬼。人言生女作门楣,昭君当时忧色衰。’说浅薄已经是很宽容了。

而且苏轼在反对改变役法时也说过,没了服衙前役,在官员家中免费做工的百姓,官员家中就未免显得‘雕弊太甚,厨传萧然’,‘则似危邦之陋风,恐非太平之盛观’。士大夫‘捐亲戚,弃坟墓’,为了取乐是占了很大一部分原因,不仅仅是为了天子和国家做事。

按照后世的话说,早年的苏轼,缺乏人文主义的关怀,对百姓只是挂在嘴边的符号而已,触犯到自己的利益就会抱怨起来。直到出外后,在外任职数年,才有了些改变。

“苏子瞻近来的作品,佳作连连。‘明月夜、短松冈。’可不是寻常笔力能写出来的。”

“倒也是。”王旖点点头,苏轼的这一首悼亡词,伤痛感怀之处不输元稹,意境则犹有过之。此一篇一出世,便在旬月间传遍了大河南北。

“‘会挽雕弓如满月’更是值得痛饮一大白。”韩冈笑道,却见王旖神色淡淡,知道这等豪放派的诗词,不合此时大部分人的胃口,“如今他在湖州的任上,想必又有佳作。”

韩冈对苏轼自从出外之后文风的改变很是欣赏——不仅仅是韩冈,士林中对苏轼的评价也是越来越高——不过韩冈从没打算跟苏轼做朋友,而从苏轼那边来说,当然也不会喜欢连诗词都不会的韩冈。他身边来往的友人都是文采风流的才子,韩冈可够不上标准。就算没有旧时的一点过节,完全不同类型的两人也不会有多少交集。

“嗯,多半如此。”王旖感觉水冷了一点,唤人进来兑了一点热水,道:“年节一过,西北就要谋划攻夏。不知道熙河路粮草还够不够,去年天下五谷丰登,要是今年也丰收就好了。”

“今冬北方各路都不缺降雪,不出意外的话,今年会又是个丰年。如果时间把握得好,攻打西夏的时候,陕西的存粮用光后,正好能用新粮接替上。”

“只要粮草能供给得上,熙河路就不用担心了。”

“还要担心由谁统领熙河路汉人番人的六万大军。要是定了王中正,就让人头疼了。”

王旖安慰道:“不是说他是福将吗?到了哪边,哪边就不会输,若是由他领军,总比一干贪功不惜士卒性命的将校要好。”

韩冈呵的笑了一声,“说得也是,到时候,就得看他的福气能不能保佑熙河路的兵马了。”

就连家中的闲聊都少不了西北的战事,被请去吃饭的时候,韩冈回想与妻子的聊天,都觉得好笑,人家赵括好歹也是纸上谈兵,他今天算是什么。也怪眼下除了战事,朝堂中也不会有什么大事。

过了年节假,郭逵就启程去河北了,韩冈送了他之后,照旧去衙门上工。

为了西北之事,枢密院那里忙了起来,在枢密院挂名的韩缜自然也是整天不见在群牧司露个脸,韩冈身上的担子稍微重了些,不能再像刚刚上任时那样,每天用上一刻钟签字画押就了事。现在他要负责征调各处军马,以补充陕西转运及驿传的马匹缺口,工作时间也就从一刻钟延长到了一个时辰。

在几次开边的战争中,韩冈负责的都是粮秣后勤,说到战时转运,薛向都要靠边站。下面的人的一些小心思,以及在账籍中做的手脚,全都瞒不过他的眼睛——能玩花样的地方,韩冈一清二楚。

韩冈之前凡事不理,只当个合格的橡皮图章,让衙署中的一干属吏产生了一些不该有的误会,这时候没费什么力气,就将几个不长眼的揪出来。韩冈并没有责罚他们,而是转手交给韩缜处置。而韩缜待下一向严苛,一顿棒子,将四个人废了双腿,又全数开革了。其中有一个,被拖回去后当天晚上就在家里暴毙。

如果换个时间,韩缜少不了要吃挂落,一旦被政敌揪住,下台出外是免不了的。但眼下朝堂上的重心全在西北军事上,杀两个贪官污吏祭旗,也正合天子之意,御史台里面的乌鸦都不会蠢到帮他们叫两声。

整顿过了风纪,手下的人开始战战兢兢的老实做事,韩冈手上的事上了正轨,做起来就很轻松了。

边疆上厉兵秣马,朝中也是紧锣密鼓,国事的重心彻底偏向了陕西。但大事没有,小事还有那么一两桩。

先是陈世儒弑母案在大理寺、审刑院和御史台三方会审后终于定案,夫妻都论了死罪,而领命出手杀人的婢女总计十七人也全都是死罪。案子的判决结果,基本上跟苏缄当初的判决没有两样,有区别的地方,就是没有参与此事,且事先不知情的七名婢女则是被杖脊,编管远州。

而后到了正月初十,御史中丞李定上表弹劾知湖州苏轼,言其讥切时事,讪谤天子,‘伏望断自天衷,特行典宪’!

