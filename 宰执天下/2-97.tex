\section{第22章 瞒天过海暗遣兵(二)}

【第一更,求红票,收藏】

“渭源?”高遵裕等人各自把这个词在嘴里念了一下,当即一齐反应了过来,脸色无不为之一变。

“是要动用修筑渭源堡的钱粮?!”苗授惊问。

“还有人力。”

论军事才能——尤其是战术层面上的能力——韩冈并不算出色。也就战略眼光还可以,摇着鹅毛扇、运筹于帷幄之中没有什么问题,若真要让他上阵指挥,肯定要抓瞎。但他并不缺官场上和职场上变通的头脑,缺钱怎么办,很简单,就两个字——

——挪用!

放弃渭源堡的扩建,把建设专款挪作军费,保证战时的供给。

为了扩建渭源堡,王韶所准备的钱粮用来作为军费是绰绰有余,而调用秦州民伕的申请也早早得到秦州城的批复。本来增筑渭源的计划,就是利用十月冬麦播种前的时间,这些筑堡的民伕,完全可以用来运送粮草军械。

韩冈的建议不算出奇,厅中的几位其实都能想得到。但苗授是不敢去想的,韩冈作为安抚司机宜能说的话,他虽是地位更高的都巡检却不能说、不能想。而高遵裕和王厚两人,大概是思维走向上有了定势,没有往那方面去思考。

只是王韶……韩冈总觉得在他说出自己的建议的时候,他的顶头上司的眼神闪动了一下,之后的神色也没变成高遵裕和苗授一样的惊讶。

以王韶的才智,能想到这个主意也不会让人奇怪,韩冈就觉得很正常,他猜度着,大概是早就想到了,如果他没说话的话,王韶就要自己提出来了。

“那渭源堡怎么办?”王厚追问着韩冈,“总不能不修吧?”

增筑渭源堡的方案早早的就递到了秦凤经略司和枢密院,连天子都在关心着此事。王厚担心着如果不能依时完工,朝廷肯定要降罪。但韩冈一无所惧,胜利者不受指责,“只要此战得胜,朝廷自会重新拨钱下来。”

“若是败了呢?”高遵裕问道。

“当然会被降罪。”韩冈斩钉截铁地说着。

高遵裕神色间顿时多了点阴郁,韩冈的回答虽然是实话,却不是他想听的。

“结果都是一样,”王韶低沉的声音响起:“如若授之用兵不顺,有个参差的话,星罗结部势力必然大张。那时候,即便有钱粮有民伕,也一样不可能在他们眼前安安稳稳的把渭源堡扩建起来,还是会被治罪。”

正如王韶所言,如果出战失败,韩冈和苗授的两个方案其实都是一个结果,渭源堡不可能建起来。既然失败的后果一样,而出兵成功的几率,则是韩冈的计划要比苗授的更高一点——再怎么说,三千大军总比一千人冒风险要强——那该选择哪一个方案,自然不言而喻。

王韶出言为韩冈的计划背书,高遵裕想通后也点头表示同意。渭源堡要扩建,这一点连别羌都知道,还以此为借口,四处招揽盟友,试图与朝廷拮抗。故而就算渭源堡突然间多了几千民伕和士兵,又大车小车的在官道上来回穿梭,别羌星罗结也不会紧张过度。只要夯上两天土,让星罗结部放松警惕。接下来,便是三千奇兵突袭露骨山下的星罗结城。

用兵贵奇,这一招瞒天过海,无论是从可行性,还是成功率上,韩冈的计划的确是要比苗授高上一筹。苗授对此也没有什么不高兴的,如果有更稳妥地方法,他也不愿去冒风险,毕竟到时带兵上阵的肯定是他。他是西路都巡检,对苗授来说,渭源堡是否扩建都不干他事,只要有仗打就成。

只不过苗授还想确认一下自己领军的权力,他试探地问道:“秦州那边要不要事先知会一声?”

韩冈看了看高遵裕,又看了看王韶,两人都是木无表情。谁也不想看着郭逵在这件事上掺和一手,功劳本就不多,小小的一块饼,以郭逵的身份必然要分了大半去,说不定他还会派燕达来主持。秦凤经略司这么一口咬下来,作为下属机构的缘边安抚司就只剩残渣碎屑可以舔食了。

王厚将询问的眼神投来,韩冈道:“还是等到钱粮、民伕以及出战的各军到位后,届时再提也不迟。”

按照韩冈的计划,即便是出战,钱粮照样要送去渭源,民伕也同样得送去渭源,再以护卫筑堡的名义派出军队。无论是钱粮、民伕还是护卫,都是筑堡规划中已经确定的步骤。既然开头做的是一样的工作,就不必向经略司明说这是为了开战,而不是为了筑堡。等到把前期工作完成后,找个借口通知一下郭逵,也不会有什么问题。

“到了渭源堡后,别羌星罗结肯定会不断派人来渭源刺探。到时说他有心反乱,必须先发制人,也是顺理成章。”

韩冈把话摊开了说了。欲加之罪、何患无辞,这两句放在哪里都是管用的。只要想打,出兵的理由很好找。一旦蕃人出现在渭源堡附近,不管他们是哪一部的——即便是实打实的商人——都可以说成是别羌的奸细。

为了保护渭源堡的安全,缘边安抚司不得不出兵,谁能对此说不是?

……………………

八月下旬,秋风渐起的时候,第一批四百民伕抵达了古渭。

来自于成纪县的这群民伕,被安排在城中的一处空营中住下。不过为了查验是否有所逃亡,在入住前都是要进行一番清点。

虽然这四百名民伕看着乱哄哄的一窝蜂,但都按着户籍所在地的不同,分成了一个个小团体,乱中自有其秩序。而等到领着这群人的武官大喝了几声,便都静了下来,没几下,连队列都排好了。赶了几百里,每个人精气神却不差,而且都是些精壮汉子,看起来秦州那边应该是事先挑选过,并没有用些老弱病残来充数。

不过这也是在情理之中,边地筑堡是军中要事,郭逵当然不会不重视。历朝历代使用民伕加起来已经有了几千年的历史,被宋建立后,关西缘边大兴土木又非一日。若是对民伕连最基本的组织都做不好,怎么可能在崇山峻岭之中,打造出一条绵延两千余里、纵深上百里的筑垒地域。

点验民伕的工作由王厚负责,用了一刻钟,他笑着回来,“一个也没逃,全都到齐了。下面就看玉昆你的了。”

“朱中!”韩冈叫来古渭疗养院的主事,“你先在疗养院里挑两个干练的医工,明天跟着民伕一起去渭源,把随军医馆的架子先搭起来。过几天等渭源去得人多了,还要从你手下调一队过去,你要提前把人选定好。”

被韩冈从民伕中简拔出来的朱中,对韩冈的吩咐视同圣旨一般,忙不迭地点头,“机宜放心,小人一定仔细挑选。”

“三哥,要多挑几个好郎中,省得他们留在古渭闲得慌。”王舜臣方才跟着王厚一起点验过民伕回来,明天为全军打头阵做先锋的就是他。预定中,除了第一批的四百名民伕,王舜臣还要带上一个指挥的骑兵压阵。

他跟着韩冈久了,知道军中医疗救护的好处。不论是叫郎中还是医工,有从疗养院中出来的他们主持营中的卫生医护,可以防止疫病给他手下将士带来不必要的损失。

“朱中,听到没有?”韩冈对朱中说道。

朱中一个劲的点头:“小人明白,小人明白。”

王舜臣攥了攥拳头,骨节嘎嘣嘎嘣的响了几声,一张丑脸笑得狰狞:“有了军医,就不用怕伤病了。今次好歹再斩个几百首级,也让州城里的燕太尉瞧瞧……”

“低声点!”王厚急忙提醒着王舜臣,恨不得踢上他一脚。

今次出战,三千大军由苗授亲领,而王舜臣则是副将。虽然实际年龄比韩冈还小一岁,但如果不计入高遵裕的话,王舜臣的官阶在古渭寨内的武将中,其实仅次于苗授。他虽然还不能参与最机密的军议,不过会后,名为筑堡、实为突袭星罗结部的计划还是很快通报给他。

但除了古渭城中的几个文武官外,所有人都只知道今次仅仅是要增筑渭源堡。斩首几百级的话,连一个字都不能提的。王舜臣知道自己失言,撇了撇嘴不多话了。在他眼中,燕达是偷了种五郎功劳的小偷,郭逵则是幕后主使,若非他们两人,今次来秦州做副总管的,应该是种谔才是。在王厚和韩冈面前,他根本不去掩饰自己对燕达的不屑。

“王兄弟,你今天早点回去休息。明天……”韩冈话声一顿,与王厚一起,向王舜臣身后看去。脚步声随即从后传来,王舜臣跟着两人的视线转身,却见来人是王韶身边的亲随王惟新。

王惟新快步走到韩冈王厚身前,匆匆行过礼,道:“有个和尚来了,说是奉旨而来。王安抚让两位机宜快点回衙门去。”

“和尚?”韩冈与王厚对视一眼,问道,“他法号为何?”

“智缘。”

