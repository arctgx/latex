\section{第36章 万众袭远似火焚(六)}

在父母和妻妾依依不舍的泪眼中,韩冈和景思立率部启程。经过了七天的长途跋涉,终于抵达了熙州。

就在熙州城下,韩冈和景思立听到的第一个消息,就是王韶在前日已经领军打过洮西去了。赵隆所领的前锋,在熙州和河州交界处的当川堡击败了两千蕃军,将木征打回来河州。而王韶本人,现在正驻扎在康乐寨,离着洮水有三十多里。

想不到王韶的动作如此之快。韩冈用了很短的时间将景思立和民伕们都安顿下,便带着刘源一众作为护卫,度过洮水,赶往康乐寨。

康乐寨旧时不过是吐蕃人筑起的一个土围子,现在却被大军填的水泄不通。

韩冈在暮色降临前抵达寨门处,正见着一队游骑从外回返,看着他们空空的箭囊,以及马颈下挂着的首级,就知道他们刚刚经历过一场大战。

见到韩冈,这群游骑纷纷下马,但有一人却坐在马上,年纪不小,完全没有骑兵应有的精锐。他叫着韩冈:“玉昆兄,你也来了。”

韩冈停步,怔了一下后,认出了人来。是高遵裕家七拐八绕的亲戚,本来投到高遵裕帐下做个幕僚,但乱出了一通主意后,被火大的高副总管踢到下面做游骑了。是个考不上进士、明经的村秀才,人称高学究。

只是韩冈苦思冥想,也没能想起这一位的名字究竟为何。‘是高明什么,还是高什么辉?’想了一阵,他还是放弃了,拱了拱手:“原来是高兄。”

“前日离开陇西,没能向玉昆兄辞行,实在是失礼啊……”高学究絮絮叨叨的说着话,装着跟韩冈很熟悉的样子。

韩冈看得出他是用自己的身份来压其他的士卒,心头便有些不喜。他跟这位高某某,连点头之交都算不上。

他冲着挂在马脖子上血淋淋的人头指了一下,问着带队的骑手,“这是今天的斩获?”

“正是!”高学究抢着回答,“这是方才愚兄领队巡游,看到一队蕃贼,便冲过去杀了他们……”

高学究自吹自擂,而同行的游骑脸色都很难看,冷眼看着他。韩冈心知,这一位怕是什么都没做,就等着分功劳的。

不仅仅是高学究,现在王韶和高遵裕身边,都多了许多亲戚,吵着要投军。看着这群人,韩冈很难不联想到,大草原上从狮子、猎豹嘴里抢食的鬣狗那样的生物。

‘不过抢食之前,要小心背后啊。’

只是韩冈想了想,并不打算劝诫这位他记不起名号的高某某。有些人指出了他们的错处后,反而会恼羞成怒。能心安理得的抢夺他人的功劳,眼前的这位高家的亲族多半就是这样的人。

与高学究他们分开,韩冈低声对自己身边的一名亲兵嘱咐道,“速去记功的那里,让他们不要偏袒得太过火,记录时要公平一点。记住,不要让高学究看到。”

亲兵连忙应声去了,韩冈打算做的就是这么多了,毕竟是认识的人,让他被人从背后捅了总不是件心情愉快的事。

韩冈继续往里去,沿途的守卫看到了他过来,都连忙把路让开。走到王、高所在的厅堂外,还没进去,就看到王舜臣面朝内地站在门边,而王韶怒气难遏地叱责之声就从厅中传了出来。

韩冈向里一张望,只见苗授正低着头,听着王韶愤怒的责骂。

韩冈走上前,拍了拍王舜臣的肩膀,就见他猛的回头过来。

“怎么了?”韩冈同时在问着。

“三哥,你来了?!”王舜臣转头,看到韩冈,是又惊又喜。他如今刚满二十岁,几年来大大小小的战功,就让他与韩冈同样成了能参加朝会的官员。当然,王舜臣在军籍簿上的年纪,比他实际年龄要大得多,而外在的相貌也不会惹人疑窦。

“究竟是怎么了?”韩冈又问了一遍。

“还能是什么?苗都监下面有人杀良冒功,给抓个正着。”

“这事有什么好大惊小怪的?发这么大火?”韩冈一下愣住。

这根本不是什么大事嘛。杀良冒功的事,在任何名将手下几乎都难以避免,只要斩首记功的规则依然存在,人们的私心,就会像地里的杂草一般永远也烧不干净。

只是他又很快醒悟过来,低声急问:“是不是不长眼杀错人了?”

“斩了一个青唐部长老的弟弟,连同一队护卫都杀光了。”

“混帐东西!”韩冈听了就一声怒骂。洮西的蕃部尽管杀,熙州的部族杀几个可权当威慑,但杀到自己人的头上来充功劳,任何一个将领都容不得这等人。“包约怎么说?”

“前面来抱怨过一次,高副总管答应他要把人找出来以军法从事,现在查出来是苗都监下面的人。”王舜臣声音中多了几分沉重,苗授堂堂一个都监,照样被王韶骂得头都不敢抬,让他这个熙河南部都巡检有了种兔死狐悲的感慨。

“只是今次因为被杀的不是一般人,才闹起来的,换作是普通族丁来,包约说不定都咬牙给认了。如今在熙州的哪一军没有这等事,真要查起来,小弟下面说不准也有人做过。管他是青唐部,还是青盐部,左右都是蕃人,装束打扮都没区别。脑袋斩下来后,不知自家亲眷来辨认,谁也说不清是哪一部的,呈上去后,最少都是五匹绢。”

王舜臣又是苦笑一下,“这也是心浮气躁给惹得祸。现在大战已开,外出的游骑见到一个蕃人就杀,从来不多问。但若真的要先分辨再动手,失了先机,反倒是官军的游骑要吃亏了。这可都是精锐,哪能舍得啊?”

韩冈听了,也有些皱眉头,这种事的确不好解决——是两难啊。

厅中,王韶训了一阵后,有些气喘,端起杯子喝茶。韩冈瞅了这个机会,立刻走进了内厅中。

“玉昆,你怎么来了?!景思立人呢?”

看到韩冈,王韶和高遵裕都有些惊讶。前面听说景思立和韩冈已经抵达熙州,他们都以为韩冈今天会留在狄道城中安顿秦凤军,不会赶着过洮西来。

“秦凤军的驻地已经分派好,食宿也安排妥当,景都监指挥他的人马安顿下来,一时不克分身。而下官没什么要事,听说前方大捷,看着天色尚明,就赶着过洮水来了。”

韩冈进来打岔,王韶也没心思再骂人了,看了苗授一眼:“授之,你回去把那几个杀良冒功的都按军法处置,首级悬门三日……此风绝不可长。”

苗授躬身应承袭来,又唯唯诺诺的告退出门。熙河如今还没有钤辖,他这个都监已经算是军中的第三号人物了。但出了错,王韶照样也不会给他面子。

王韶眼下很是无奈,杀良冒功的事所在多有,可一旦闹出来、闹大了,就等于是给了朝中的政敌们一个把柄。那些人可不会就事论事,推而广之是必然的,熙河路过往的战绩在他们的嘴里,那就不知道会被说成是什么样,打上多少折扣。以王韶在河湟投入心血之多,立下了的功劳若是被人抹黑了,他哪可能忍得住?

一举夺占木征在洮西最后的据点的喜悦,在闹出这件事后,便烟消云散。他叹了口气,问韩冈道:“玉昆,你一向主意多,能不能想什么办法把这风头给杀下去?”

杀良冒功的事,韩冈不是很放在心上。这事本来就不好查,只要不杀到自己人身上,睁一只眼闭一只眼就是了。反正这河湟之地,哪有一个良民?至于青唐部那里,让他们自己小心着,一般来说,只要能及时报上身份,应当不会有大碍。

况且今次只要能打下河州城,斩首数就算有点水分,也无关紧要,天子不在意就行了。在他看来,王韶其实就是太过求全责备。另外,当还有一些原因是因为王韶跟自己所处的位置不同的缘故。

不过虽说心中不以为然,但王韶的要求韩冈也得去回答:“要不要试试刺字?……脸上有个涅记,怎么也不会让人冒了功来。”

韩冈算是乱出主意,高遵裕皱眉正想说话,但王韶却像被他的信口开河给提醒了,一拍桌案,“这个主意不错。刺了面之后,就可以算上是正式的蕃军弓箭手。多个三五千不要粮饷的蕃军,天子也当会乐见。”

“刺面恐怕不行!”高遵裕连忙反对,“蕃人没有这个习惯,要他们在脸上刺字,说不定会闹将起来。”

“那就刺耳后。耳朵背后总没事了吧?我们也只要留个记号而已……可以防止被官军误杀,也可算是蕃军弓箭手的标记。”

“但光是说防止被杀了取首级冒功,可能这些蕃人不会太甘心刺字。”

“那就以利诱之……肯于耳后刺字的,给米三斗,茶两块。这一点支出,还是能挤得出来,向朝廷申领也是断无不付之理。”

“万一木征那边的蕃人也学着样来在耳朵上刺字,日后可就麻烦了。”

“蕃人没那么机灵,我们这里声势小一点,不要让他们的知道就行了。”

韩冈随便出的主意,王韶和高遵裕讨论得一本正经,而且看起来马上就要上书朝廷,请求实行了——设立蕃军并刺字,必须要有朝廷的同意,即便是将帅也不敢多做。

