\section{第四章 惊云纷纷掠短篷(四)}

【第二更】

湖州通判铁青着脸站在院中,他没想到苏轼会如此无礼。

祖家也是书香门第,代代有进士。祖无颇的族兄祖无择可算是当世名臣,资历极老,仁宗时都做到了权知开封府。只是运气不甚佳,由于当年与王安石同做知制诰时留下的龃龉,十年来别人的官越做越大,祖无择的官则是越做越小。但祖无颇依然不是苏轼可以无礼的对象。

祖无颇的幕僚这时走了过来,附耳低声道,“肯定是出大事了,否则苏子瞻必不至如此失态。”

“失态……”

祖无颇念着这两个字,神色也缓了下来。若真的是苏家里面出了什么大事,苏轼方才的失礼也算不得什么了。人这一世,都会有这个时候。看向内院屏门的时候,眼神中也多了几分同情之色。

“保赤局的事,既然苏直史已经交予无颇全权处置,那就立刻转发漕司公函去县中,想必不会有哪一县会在此事上拖延。”

幕僚点头应下,随即便笑道:“肯定不会有人敢拖延,此事拖上一天,治下的百姓都能把他吃掉。说不定,还不待催促,就派了人上来。”

祖无颇叹了一口气:“要是夏秋上缴税赋时,他们能一半痛快就好了。”

幕僚摇着头:“善财难舍啊……”

宾主二人说着闲话,就准备回通判理事的倅厅去。还有一堆公事等着要办呢。

可衙门正门外,这时候却又传来一阵喧哗。门前司阍的衙前随即连滚带爬的跑了过来,“京里派御史来了,说是苏直史犯了事,要从中门进来。”

“中门?!”祖无颇脸色大变。

州衙的大门有三扇,从来都是只开边门供人行走,就是知州、通判,平常也是走边门。正中的主门,也只有新知州上任,还有元旦祭礼、立春鞭牛等仪式才会打开。当然,朝中来人身负如宣旨这样的重大使命时,也会要求大开中门。

如果没方才苏轼慌慌张张的样子,说不定祖无颇还能以为是苏轼得了圣眷,将要被大用了。但现在看来,肯定是噩耗。

心知来人多半身负皇命,祖无颇不敢耽搁,连忙派了人去大开中门,将来使迎进了州衙。

来使身穿朝服,手持笏板,立于庭中。双目阴寒,左右顾盼。他身边有两名伴当护持,都是白衣青巾,腰悬铁牌,只要对京中官场稍有了解,便知他们的出身,正是官员中人人闻之生畏的御史台——是御史台的台卒!

听到消息,州衙中的大小官吏,除了苏轼之外,全都出来了,领头的祖无颇战战兢兢,虽知今天的事多半跟自己无关,但看见乌台中人,心中还是免不了发慌。

只听一名台卒厉声喝问:“监察御史里行、太常博士皇甫僎在此,知州苏轼何在?!”

内院没有动静。

再问,还是没动静。

一众官吏的眼睛都望向了祖无颇,祖无颇无奈,出列道:“知州近日因病告假。”

“还请去催一催!”台卒吩咐道,“抬也得抬来!”

祖无颇抬眼去看皇甫僎。京城来的御史连个正眼都不给,丝毫不加理会。

湖州通判暗叹了一口气,却只能听着台卒的吩咐,去敲后院的屏门。

黑漆的大门吱呀一声就开了,让祖无颇走了进去。黑压压一群人就站在屏门内,就连苏轼也在其中,人人面色如土。

“究竟是出了何事?”祖无颇问道。

苏轼惶惶不安,“不瞒公方,是御史中丞李定弹劾苏轼讪谤朝政。方才才得了舍弟子由的急报,谁料想现在人就到了。”

祖无颇听到缘由之后,反倒一点也不惊讶了,讪谤朝政这件事,没有才是怪了。叹道:“事已至此,无可奈何,须出见之。”

“啊……说得也是。”苏轼全然没了主张,抬脚就要出去。

“直史……衣服!衣服!”祖无颇连忙提醒。

苏轼低头看,穿在身上的还是出外游玩的便服。摇摇头:“既有罪,不可穿朝服。”

“未知罪名,仍当以朝服见。”祖无颇提醒道。

“……多谢公方提点。事发仓卒,苏轼已经乱了方寸。”

苏轼随即依言换了朝服,手持笏板出去见京城来使。在他身后,祖无颇一众官吏左右排开。

可等到苏轼一众站在面前之后,皇甫僎却不开口,如鹰如狼的眼神扫视着湖州上下官员,像是在搜寻着什么。而站在他身后的两名御史台台卒,也同样默不作声。如此作态很是奇怪,让每一个在场的湖州官吏的心中,都越发的不安起来。

苏轼的心一点点的沉了下去。虽说得了弟弟苏辙的通报,但苏辙本来就是听了王诜的急报,加上王诜和苏辙都不敢留下文字,只让人传话,中间经过一番周转,早就面目全非。加之几千里匆匆赶来送信,任谁只会往重里去想。

其中一名台卒手上,攥着一根尺许长,如同棍状的东西,外面用青色的锦缎打着包裹。可能是写着诏命或是牒文的卷轴,但那样的形制,也可能是匕首——不少人心中都有了同样的猜测,该不会是赐给苏轼自裁用的吧?

苏轼脸色灰败,持笏的双手都在颤着:“苏轼自来疏于口舌笔墨,着恼朝廷甚多,今日必是赐死,死固不敢辞,乞归于家人诀别。”

后面的祖无颇心神一松,他看不见苏轼的脸色,只道苏轼心神终究还是恢复了清明。

不先把皇甫僎的底细探听明白,说不准就是曹利用被杨怀敏迫死的结果。这么放低姿态的一问,皇甫僎怎么都该回答了。

皇甫僎也的确不好再装哑巴,简短的回答道:“不至如此。”

终于让皇甫僎开了口,下面就该追问到底是什么罪名,准备如何处置了。可祖无颇几乎将苏轼的后背用视线烧个洞出来,也不见他的上司再问上一句。

祖无颇忍不住了,出头道:“大博奉命出京,必有被受文字!”

皇甫僎眼神一下又尖锐起来。

这句话分明是警告!湖州通判用本官官阶,而不是监察御史里行的差遣称呼他皇甫僎,分明是在警告,在场的知州、通判,品阶皆在他之上,不是可以任人欺辱的低品官员。

上下打量了祖无颇好一阵,皇甫僎语气阴森的缓缓问道:“君乃何人?”

祖无颇只当是同僚间的通名,拱手行了个礼:“通判祖无颇,如今权摄州职。”

皇甫僎又盯了祖无颇两眼,探手向后一招,台卒心领神会的将青绸包裹递给了他。

青色的丝绢一层层的打开,露出来的东西让所有人都松了口气,不是匕首,也不是绫纸做底的诏书,素色的纸背仅仅是普通的牒文。而内容更是让人放下心来,只是寻常的追摄行遣而已,不过是以苏轼以诗文讪谤朝廷,提他入京审问罢了。尽管性质依然严重,但总算比赐死什么的要好得多。

苏轼浑浑噩噩的低头领罪,当场脱了衣冠。

苏轼认了罪,湖州便以祖无颇为首。暂摄州事的差事眼见着要做上好几个月,暗叹了一声,祖无颇上前对皇甫僎道,“御史远来辛苦,在下这就命人安排食宿,权且少待。”

“不必劳烦。”皇甫僎冷然说着,一个眼色过去,两名台卒就抖开一条素练,将苏轼的双手给绑了起来。

庭中一片哗然,祖无颇也惊问道:“这……这是为何?”

“身负上命,岂敢耽搁片刻?皇甫僎这就要回京复命。”

皇甫僎转身就走,两名台卒用力扯了一把手上的素练,苏轼被拉了一个踉跄,跌跌撞撞的跟着去了。

内院的屏门中开,在里面听消息的苏轼妻儿跑了出来,哭喊着要跟上去。

苏轼的续弦王闰之抱着小儿子苏过,长子苏迈、次子苏迨同追在后面,滕妾仆婢一起涌了出来。苏家的侍妾以美貌著称,向来为同列所钦慕,但现在也没有人去多看她们两眼。皆是望着苏轼踉跄远去的背影,陷入兔死狐悲物伤其类的情绪之中。

王巩也跟着王家的人,脸色惨白,整个人都是呆滞的。苏轼因诗文出了事,跟他相唱和的朋友恐怕也讨不了好去。

此时消息已经传到了外面。

苏轼喜欢游宴,带着妓女和乐班,湖州境内的风景名胜处处都有了他的足迹。一听说苏学士要设宴作诗,有空的都跟过去的凑趣。

几个月下来,苏轼的名气在湖州大得没边,诗词一首接一首,城中百姓也都喜欢听苏学士的新词。这时知道苏学士被朝廷捉了去问罪,一时都赶了过来,却没人敢挡着皇甫僎的路,只能目送苏轼被一步步的拉向城外,许多人都眼中含泪。

在一片混乱中,只有祖无颇还保持着清明,先一步拦着苏家的人。

“得派人跟着直史。”祖无颇提醒道,眼睛看着苏轼的长子苏迈。

苏迈立刻就领会了祖无颇的用意,回身就对王闰之辞行,“娘,孩儿跟着父亲大人在旁随侍,必不叫大人有失。”

王闰之擦着眼泪,匆匆忙忙的点了两个平日里惯得用的仆人,“你们跟着老爷和大郎,好生服侍。”又忙叫人回去收拾衣物和银钱,要让苏迈带着。

苏家上下忙忙乱乱一阵,当苏迈带着人跟上去时,苏轼已经被绑着双手拖到了官船上。皇甫僎竟然当真是一点不肯耽搁,当天就要往京城去。

跟在后面,见着御史台台卒拉一太守如驱犬鸡,祖无颇不寒而栗,而皇甫僎最后投过来的深深一瞥更是让他心底发冷——

这件案子小不了,可别把自家栽进去。

