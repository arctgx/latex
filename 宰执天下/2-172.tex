\section{第30章 肘腋萧墙暮色凉(七)}

从保慈宫中出来,走在通往自己寝殿的廊道中,赵颢与天上皎洁的月光截然相反,始终阴沉着脸。王妃冯氏也是脸色木然的走在身后两步的地方,结缡三年后,夫妻两人的关系越发的紧张。而抱着赵颢一对儿女的两个宫女,还有一群内侍则不敢靠得太近,远远吊在后面。除了嚓嚓的脚步声,一行人行动间没有半点的声响,宛如在沉默的行军,气氛压抑得堪比守灵的夜晚。

一名给高太后端着药汤的小黄门迎面过来,见到赵颢这一路发丧一般的气氛,便缩了缩脖子,连宽敞得足以并行马车的廊道都觉得太窄,慌忙两步退到廊外,在雪地里跪下来等着雍王家一行人过去。

赵颢脸色沉沉,连瞥都不被瞥那小黄门一眼。他的心情七分愤怒,三分憎恨,对外界的变化,丝毫没有一点关心。刚刚在保慈宫中挨了一顿训,而他的兄长、如今的天子却在一旁做作的劝着发怒的娘娘。

赵顼言辞恳切的为赵颢辩说,劝着娘娘息怒。但赵颢知道,他的兄长现在的心中,就好像跟宫外一样,一个劲的在响着欢快的鞭炮声。

在外,横山大捷、罗兀克复,熙宁三年的连绵战事有了一个完美的总结;在内,新法顺利推行,去年的税入减去支出之后,有了近百万贯的结余;比起英宗年间,一千五百万贯的亏空要好上许多。而且这还是建立在熙宁三年战事不断,而且又开始给胥吏增发俸禄的基础上。

就算宫中刚刚诞下的是皇女,而不是内外盼望已久的皇子,也没坏了他大哥的心情。反而刚出生的皇二女,转天就被封为宝庆公主。

而他赵颢就很倒霉,不但因为一点芝麻大的小事,成了世人口中的反派,而且现在还被朝臣连番弹劾,说他有损天家体面,不宜久居宫中——‘先把你们自己的裤裆管好,好意思跟我比哪个更不要脸!?’赵颢倒是想这么骂。但是,他可没那个机会,想跟朝臣对骂,先得坐上皇帝的宝座。今次的上元夜观灯,赵颢也是没心情去了,站在宣德门城楼上给人指指点点,他还没那么好的气量。

但这一切是谁造成的?赵颢并不会恨错人。

韩冈是起头的,赵颢心里牢牢记着。明着说要把事情压下去,私下里却是推波助澜的兄长,赵颢也一样记着。

不就是要把他赶出宫吗?兄弟情分全都丢一边去,真是把李世民的样学到了十足十。

赵颢知道,他的大哥一向崇敬李世民的丰功伟绩。听说当初王安石第一次面圣,问他崇过往帝王何人之功,赵顼的回答就是李世民。

不过真要说起李世民,恐怕他大哥也要担心他赵颢有这份心思,正好也是老大、老二、老四三人这么排着。不过赵颢不是疯子,心里有想法,也不是在现在。

‘真的要被赶出宫去了。’

赵颢回到了分配给自己的寝殿,冯氏领着两个儿女到里面去了,也不搭理他。而赵颢在外面坐下来,望着头顶上雕饰斑驳的梁柱椽子。都是老旧的货色了,几十年过去,并没有修补过几次,就跟中书省的建筑一样,破败得连外面的酒楼都不如。

可是,这是皇城里的殿宇。就像是古董,唐时的三彩,就是比现在的官窑要值钱,价值不是在质地上。

但这座宫舍很快就跟他无缘了。群臣上书,一面倒地声音,新旧两党之间的矛盾都看不到了。赵顼乘势逼着娘娘点头,正月过后就要在宫外开始修造二王邸。等到两座王邸建成,就是他赵颢,还有老四赵頵搬出宫中的时候了。

堂堂一位亲王,因为一个丢脸的理由,近乎屈辱的被赶出宫去,就算明面上做得再漂亮,可在民间,他已是声名狼藉。

“茶呢?!”赵颢越想越怒,用力一拍几案,怒吼着。

……………………

正月十五的上元夜,韩冈是在罗兀城度过。

厚厚的积雪的覆盖了山头和谷地,天地间白茫茫一片,反倒让夜色变得不那么深沉。天上的明月皎皎,城下的工地上灯火辉煌。如果是站在罗兀旧城的城头上,低首下望,漫漫的篝火辉光闪耀,被山坡上的积雪反射回来,就仿佛有天上的星河映于地表,在山谷中流淌。

只要高高在上的望着,就算是东京城中的上元夜,也难以见到如此壮丽的景色。穿着皮裘,拥着火炉的文人墨客,也许会诗兴大发。

但对于韩冈来说,他不会欣赏——深冬寒夜的赶工,让他的工作又加重许多。对工地上,连夜赶工不得休息的民伕们来说,他们也不会欣赏——他们只想待在家中,就算只有一盏油灯,只要能看到妻儿父母的笑脸,那就够了。

“现在已不仅仅是冻伤的问题,这几天,自残的民伕已经超过了三十人,而且还有逐渐增加的趋势。”韩冈从临时搭建的战地医院中出来,面色沉重的对种建中摇着头,“彝叔兄,罗兀城之重,小弟心知。我不会劝你说夜里让民伕休息,把工期拖上一阵。但眼下的现状如果不能改善,情况将会越来越糟,恐会欲速不达啊!”

种建中紧皱浓眉,方才他跟着韩冈一起在医院中走了一圈,看得也是怵目惊心,知道这样下去不行。这里都是精壮的汉子,真要闹出民变,麻烦可就大了。

“不知玉昆你有什么办法?”

“雷简!”韩冈没有立时回答,反是回头向里面叫了一声,一名三十左右的高瘦医生连忙跑了出来。韩冈对他嘱咐道:“我要去大帐一趟,这里你先看着。”

雷简本是派在秦州甘谷城的医官,后来在韩冈手下,主持甘谷疗养院。不过前段时间调任庆州为医官,但转眼就又被调来了前线,跟着种谔一起出征罗兀。在韩冈到来之前,这里的军中医疗之事,就是由他全权负责。

雷简的医术不差,而管理水平在甘谷历练了一阵后,也勉强算是不错。但他没有开创之才,只有因循而为的本事。韩冈当初在甘谷定下的规矩,他老老实实的继承下来,做得还算不坏。但调到种谔麾下,本意是让他先给韩冈打个个头阵,不成想却是弄得一团糟。还是韩冈到了后,花了两天的时间,为其收拾首尾,费了番周折,才有了点眉目出来。

把伤病营中的事务交给雷简,两位年轻的官人就从设在城下工地边的临时疗养院,向城中的种谔主帐走去。所走过的道路上,积雪都已被铲清,只有被踩得发黑的地面。道路两边,用木架子插着一束束火炬,照亮了整条道路。

“玉昆……”并肩和韩冈沉默的走了一阵,种建中犹犹豫豫的开口,“你是不是还对今次出兵罗兀有所反对?”

“彝叔,你不必担心什么。我既然接下了这个差遣,只会用心做得最好。”韩冈没有正面回答,但已经表明了心意。

他走快了几步,反过来问着沉默下去的种建中,“彝叔,你们有没有考虑辽人那边的反应。西贼向大宋称臣。但他们也向辽国称臣。如果西贼求上了辽主,云中、河北那里的辽军有所异动,就算不出兵,这边难道还能安稳得起来?”

人落水的时候,就连稻草都会抓。何况党项人都不是傻瓜。但这番话说过,韩冈却发觉种建中脸上的神色没有一点变化。

“你们这是在赌博!”韩冈一下惊道。

也许韩绛没想到,可种谔肯定是考虑到了。也有可能是韩绛、种谔都想到了,但两人绝然没有在给天子的奏文中,提上一句。否则,这项危险的提案,必然在枢密院那里难以通过。

一旦牵扯到辽国,什么计划都要完蛋。大宋对西夏还有一些心理优势,就算当年李元昊闹得最凶的时候,宋廷都没有想过要加固潼关防线,以防高喊着要攻下长安的李元昊真的夺占关中——在宋人眼里,党项始终是边患,癣癞之疾而已。

可辽国那边只要个风吹草动,东京城中都要发抖。就算澶渊之盟后,宋辽之间已经近七十年不闻战火,但畏惧辽人之心照样存于骨髓里。

种建中停住脚,摇起头:“西贼自立国后,少有求上辽人的时候,亦多有桀骜不驯的时候,辽人何尝会为其出头。”

“辽人趁火打劫的事,不是没先例吧?”韩冈反问道,“澶渊之盟一开始只定下了三十万银绢,现在呢?五十万。没有元昊起兵,会多出这二十万?”

“那也不过是二十万岁币而已。不及每年消耗在缘边四路上的一个零头!”种建中指了指北面,“把西夏的岁赐转给辽人也就够了。”

韩冈叹了口气,没再争辩。反正他能确信西夏国祚尚长,不会就此灭国。今次之战,不论韩绛、种谔如何努力,都只会是无用功。与这里争论不休,毫无意义。

“走吧……先去见种帅。把眼下的事解决掉,辽人那里也不是我们能担心的。”韩冈叫着种建中,走进城中,一直走到大帐前。

“太尉!”种谔的亲兵见到韩冈、种建中齐至,便立刻向着大帐内高声通报:“韩管勾、种机宜求见!”

