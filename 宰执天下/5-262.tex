\section{第29章 浮生迫岁期行旅(二)}

韩冈是跟着王安石来送王安礼的。

但远远地看到已经围在王安礼身边的一群人,王安石的脸色就不好看起来。

王安石这一辈亲兄弟七人,活到出仕的四人,老三王安石居长。下面是王安国、王安礼、王安上。王安国前几年病逝了,王安上常年在外任职,而王安礼则多在京府。

所谓长兄如父,看到一手拉扯大的兄弟放.荡形骸,跟一帮同样性格的官僚厮混,明明能力出众,偏偏就在操行上坏事,王安石要能看得过眼就有鬼了。

幸而一见到王安石的旗牌,王安礼身边立刻就清净了,三丈之内不见余人。

王安礼上来向王安石问好,接着韩冈则过去向王安礼行礼。

看见韩冈也一并跟着王安石过来,听了这几天京城里风传的流言,从王安礼开始,每个人都忍不住面露讶色。

韩冈也没办法,他的三份奏章的确是实实在在的跟王安石翻了脸。

前一天席上倒酒,后一日就上本分道扬镳。王安石的心情不会比文德殿上司马光好到哪里去。

韩冈不想因为学术之争,而坏了与王安石的私人情谊。今天主动过来给王安礼送行,也是有修补关系的意思——不过,也有两三成是给王旖逼过来的。

之前下手太狠,消息传出来后气得王旖哭了一夜,两天没说话。修身齐家,方能治国平天下。韩冈也是赶着要灭后院的火。

被王安石拉着说话,像小学生一般被教训着,王安礼神色中的不耐烦,韩冈为避嫌虽站得远,看得倒是很清楚。

王安礼太过轻佻,喜好声色,跟苏轼那一帮人走得近,心性与王安石、韩冈截然不同。一面对王安石就不自在,跟韩冈更没有话说。

虽说亲戚终归是亲戚,可王安国的丧期刚满,王安礼便如同解脱一般,立刻招呼妓女来宴饮。肆无忌惮的作风,让韩冈看得心中不喜,自然不会亲近。

对于其家中的一滩烂事,王安石上京后,韩冈也从来没提过,只是王安石也有他自己的渠道,不知是从哪里听说了。

王安礼几乎是被王安石逼走的,但韩冈觉得,更多的还是王安石想保护他这个弟弟。地方上的事,大事可化小,小事可化了,而在京城中,再小的事,在有心人鼓动下,也有很大可能变成滔天巨浪。

王安石终究还是要给王安礼这个弟弟面子,教训的话私下里说没问题,当着外人和晚辈的面可不就方便了。

在路边酒楼中,送客的宴席早已摆下,王安石便拉着弟弟入席,其余人等鱼贯而入,韩冈排在前面,由王安礼的儿子王防陪席。

只是在送行时,照常例都要写诗相赠,以表离情。可是见了韩冈,最擅作诗作赋的这一群人,却变成了锯嘴的葫芦,倒不出一个字来。倒是王安石无顾忌,作诗送行,转眼就是一篇七律出来。

可王安石敢不顾他女婿的脸面,其他人哪里敢当面来?背后嘲笑韩冈是不作诗词的进士第九没问题,可眼下本人就在眼前,谁敢犯忌?

一时之间,就只有王安石的一篇亮着,其他人不是拿着筷子盯着盘盏,就是想在酒杯里看出一朵花来。

韩冈见冷了场,便起身笑道:“韩冈素乏诗才,世所共知,不敢献丑,今日且为各位做刀笔吏。”

说罢,便让人撤下自己席上的酒菜,摆开了文房四宝。拿起笔,随手写了几句序文,说了前因后果,时间地点人物,便开始将王安石刚刚的作品誊录下来。

他十几年练字不辍,气韵自华,一笔行楷虽远算不上卓然大家,却也不会再被人说是三馆抄书吏,给一个匠气十足的评语。

当韩冈开始抄写诗文,席上的气氛终于活跃起来。

王安礼的交友圈子跟王安石、韩冈差得太远,诗酒风流的一干骚人墨客和宰辅重臣从来都搁不到一个篮子里。不过在宴席上,活跃气氛倒都是一把好手,送别诗随着一杯杯酒下肚,一篇篇的传了出来。

王安石在上席处看着低头写字的韩冈,忍不住暗暗一叹。

只看今日这点小事,便足见其器量恢廓,世所罕比,要不是脾气又臭又硬得跟茅坑里的石头一般,这个女婿真的是没得话说了。

韩冈并不知道王安石的想法,就是知道也不会觉得自己的气量真有那么大。他只是不在意这点小事罢了。真要犯到他在意的人和事,二大王是什么下场?司马光、吕公著又是什么结果?

韩冈动笔抄写,心无旁碍。长兄如父,王安石在那边又拉着王安礼谆谆叮嘱。送行宴持续了两个时辰,最后还是曲终人散,将王安礼送得远去江南。

席散之后,王安礼的朋友们纷纷告辞离开,王安石却对准备早点回去销假的韩冈道:“玉昆。你陪老夫走一走。”

韩冈没奈何,迈开脚步,陪着王安石往南门行去,其他人则识相的远远避开。

从青城行宫外一直走到南薰门处,王安石一直都没开口,直到前面窜出一群猪——活猪进城,只能走南薰门——把前路一挡,一群‘痴宰相’让群臣避道的威风施展不开,王安石这才停下脚步,回身熟视韩冈良久:“玉昆,你这枢密副使当真是不想做吗?”

“岳父大人明鉴。小婿的心思,可是从来都没隐瞒过。”韩冈笑了笑,将话题丢回去,“而且天子的想法,岳父也不会不知道。否则为何招曾子宣入京?”

听到韩冈提起曾布的名字,王安石脸色顿时一沉,但随即又化为苦笑,摇摇头,不说话了,给韩冈堵得够呛。

待南薰门重新畅通,王安石和韩冈上马入城,穿过内外两重的城门,王安石才又开口:“吕与叔要走了。”

韩冈一笑:“张邃明【张璪】,蒲传正【蒲宗孟】写的好文章。”顿了顿,又补充道,“孙巨源也不差,今之贾谊,不比当年的司马十二丈逊色到哪里。”

王安石这下又没话了。

在大拜除后这十天里,给韩冈的白麻诰敇连下四道,给吕公著的慰留诏书也连下了三道。纵然皇帝、皇后都恨不得他早点离开,可以枢密使兼太子太保的身份,也不便一脚就将他踢走。当吕公著连本上奏请郡,翰林学士院便奉圣意接连书诏慰留。

若是脸皮厚一点,吕公著就此不再上本,短时间内还真是拿他没辙。但知制诰的翰林学士是什么人?乃是天下文萃华选。

就像当年司马光帮赵顼起草的慰留诏书,能将王安石气得七窍生烟一般。以张璪为首的三位内翰,各自起草的慰留诏,明褒实贬,字字诛心,不比司马光的功力逊色到哪里去,让吕公著没脸以假作真,厚着脸皮硬是留下来。

韩冈仰头看看天空,这几年来,随着石炭运用得越来越多,京城的冬天也越发的雾气缭绕。晴朗无云的冬日,天空中却仿佛被蒙上了一层纱。

只是在韩冈的心里,该走的都要走了,该来的还没来。腊月初的京城,倒是暴雨后的园林,污秽一扫而空,空气清新宜人。

……………………

当蔡京三人赶到西门时,大部分的御史都到了,幸好李定还没来。

蔡京过去打招呼,他人缘甚好,无论入台迟早,都是跟他有说有笑,与宫中殿上那一张张死人脸,完全是两个模样。

过了片刻,又是一主一仆骑着两匹马远远地赶了过来。

三十多岁的年纪,与蔡京相仿佛,就是形象上差了许多。蔡京见到他,便迎上去:“李文书,怎么来得这般迟?”

“格非来迟,还望各位恕罪。”李格非连连拱手告罪,道:“吕宫保已经在收拾家当,不方便从他家门前过,只得绕了点路。”

蔡京闻言便笑道:“文叔果然是为人敦厚啊。”

他拉着李格非过来,一群御史的脸色却都冷了下来,漫不经意的拱手行礼,却一点亲近之意都没有。

李格非尴尬得很,要不是蔡京跟他聊上两句,倒是连站都没出站了。

李格非是李清臣所荐,似乎是在相州韩家那边的关系。不过熙宁九年的进士,五年不到就转京官,而且还做了权监察御史里行,说起来实在是让赵挺之和强渊明这几位熙宁三年的进士嫉恨。而且照规矩,监察御史里行至少得是一任知县后的资深京官,但李格非根本就没做过知县,刚刚转官而已,只能加个权字。

要不是因为眼下御史台乏人,又因为是李清臣力荐,李格非根本就不可能出任此职,早就给骂回去了。当然,最重要的是新任御史的审批权是牢牢握在在天子手中。而以天子的精力,最多也只安排了一个御史中丞,剩下的人选全都是皇后批准。私下里,御史们都在议论,皇后根本不知道监察御史到底需要什么样的资历。

蔡京会做人,连不受待见的李格非也招呼到,又过了小半个时辰,一名不知是谁家的家丁跑来,说是李中丞来了。

御史们立刻放弃了闲谈,赶过去迎接。

李定一家的车马,很是简单,承载家当的马车只有两辆,仆婢也没几个。沿着大街一路过来,一点也不起眼,要不是一群御史群聚,根本都不会惹来任何目光。

这位御史中丞很早以前就被旧党视为攻击新党的突破口,不孝的传闻遍及天下。可李定赈济同族不遗余力,家无余财是显而易见的。从廉洁程度上,绝对当得起御史中丞这个位置。

而且他还统领乌台上下,好生整治了一下爱胡说八道的一帮词人。也让总爱仗着文才臧否人物的他们知道,有资格评判官员贤与不肖的,只有御史台!

比起攻劾宰相,这样的弹劾同样让人痛快,甚至还要更高。无论如何宰相是进不了诏狱的,但在苏轼住了多日之后,乌台东西两狱的名声,这两年来却已经能够威慑百官了。

而且是这次他是受连累的,有王安石、吕惠卿在,转眼就能回来。御史们哪个也不会枉做小人。

一艘河船这时从西水门进了城。一名一身绫罗,仍有着几分儒气的老者立于船头处。大约五十多岁,看着倒是挺富态。

西水门和西门新郑门比邻而立,眼尖的蔡京一眼就看清了那个老者的相貌。蔡京曾在章敦家见过两次。更早一点,则是在西太一宫打过照面。

是章敦家的门客。

路明。

