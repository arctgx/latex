\section{第14章 贡院明月皎(下)}

“他当面没能想起来,听了章惇对路明的一通介绍后,也只是隐隐觉得有哪里不对。直到回去之后,才醒悟过来,前两天刚刚写信给小弟。”上官均说道,“当日韩冈、路明,一同上京,还有最近配属在章子厚麾下,在屡立功勋的刘仲武,当日也是同行,正好在陕西道上救了章子厚之父的性命。”

这一段近乎传奇的故事,众人都听说过。虽然有许多人嫉妒韩冈,甚至下意识的贬低他的才能和功绩,但有关韩冈威震关西的传说,却流传得更广。

军库灭贼、道上驱狼,蕃部斩西使,京城夺花魁,加上他在熙河历次领军破敌,韩冈智勇双全的形象,却是早就在官场上树立了起来。以至于有人除了贬低韩冈一切只靠运气之外,就仅剩下攻击他出身这一条发泄嫉心的窗口了。

至于在民间,韩冈的名声却是跟着孙思邈联系在一起,与士林、官场上的形象,有着很大的差距。尽管士林中,多多少少都会相信一点鬼神之说,韩冈传习了孙思邈的医术也有不少人信,但却没有人会将韩冈当成可以保佑不生病灾的活神仙,仅是当作闲聊一个话题而已。就算韩冈救治了数以万计的军中伤病又如何,比得上一首名震天下的诗赋吗?开疆拓土的不世之功又如何,还不是要来考进士!

也就是因为韩冈表现得近乎一名懂得一点政事的武将,才让士子们这般嫉恨。若是其人广有文名,情况反而会好上不少。

“韩冈、路明上京后,一路到了八角镇,顺道就去了西太一宫。正好被蔡元长给撞上,小弟就差了一步,只见到了背影。”

上官均说得合情合理,众人都信了五六分。在座的都是上一科进士中的佼佼者,前面说起韩冈,皆在心中抱着几分看不起他的意思——文学高选,向来都是眼睛长在头顶上,韩冈不以文才名世,而靠着战场厮杀而博来一个朝官,在他们的心目中,便是非我族类。

但听说西太一宫中的那首枯藤老树,有着韩冈的一份功劳。画龙点睛的四个字,多少还是能证明他本身还是有才学的。再提起韩冈,至少不会再觉得他是个异类了,感觉上也便顺眼了一点。不过真的要确信无疑,还得等到真正见过韩冈本人。

就在也叶祖洽、上官均等人闲聊的时候,太阳已经移到了正南方,终于有了一位考生交了卷。

听了胥吏来报,叶祖洽就笑了起来,“比起上一科,还是慢了些。记得强渊明当日只用了两个时辰便缴了卷。”

上官均道:“今科变更法度,才思敏捷之士多是在州中便折戟沉沙。贡生中还是以老成稳重的居多。”

“说得也是。”另外几个考官一齐点头。

陆佃不愿在此事多说,说不准就会让人以为他们在抨击新法:“且不论今科捷才如何,已经有了第一个交卷的,下面三位封弥官可就要忙起来了。”

“呵呵,张户判、盛御史还有梁校勘的确是要忙了。”

正午过后,张讽、盛陶还有梁焘的确开始忙碌起来了。所有交上来的试卷,不是送给考官们,而是先送到他们手上处理之后,再送去文庙中。

三人是封弥官,主管试卷的糊名誊抄。不但要监督下面的胥吏糊起考卷上考生的个人资料,让人去誊抄。装订时还要打乱试卷誊本的装订次序,以防止负责阅卷的点检、考试、覆考三道关口的官员,能从考卷的顺序中,确认考生的身份。

一份份试卷送来,糊名的胥吏开始动手,用事先裁好的厚纸将考生姓名、籍贯给贴起来。遮严实了,再在纸上写上编号。糊完的考卷被送到另一个房间去誊抄。而誊抄完毕,书上同样的编号后,还有专门的人员来对照正本和抄本,看看誊抄后的文字是否有错讹,以防考生因胥吏的错误而被黜落。当一切审查完毕,才会五十份一摞的装订起来,然后送去给考官们们批改。

身边交卷的考生越来越多,座椅移动的声音,一声接着一声。连坐在左边的慕容武也起身缴了卷子,出门时还回头望了韩冈一眼。但韩冈一点也不心急,帖经墨义这一部分,他自觉答得很好。而最重要的一篇论,也已经在草稿上推敲了好几遍,又将词句一遍遍的修改。

韩冈文才向来平平,从来没有一挥而就的本事,要想写出一篇合格的文章,就必须一遍又一遍的反复修订。尽管其中大半内容,都是此前猜题作文时觉得有用而记下来的,现在正好借用过来。但要将之串联起,还是颇费一番思量。

时间一点点的过去,日影西移,越来越的贡生走出国子监的大门。他们神色,或是放松,或是失落,有悔恨,也有企盼,不管怎说,攸关命运的考试已经结束了。

考场中,除了韩冈以外的考生们,已经走了一干二净。监考的胥吏,已经把蜡烛给韩冈点上。他们不敢催促韩冈,在三更之前交卷,都还是合格的。这是依着唐时的故事——唐朝的时候,考生们对着定体限韵的诗题咬文嚼字,进士考试经常拖到半夜。

一篇史论其实已经写好了,比初稿时,修改得面目全非。韩冈宁宁定定的将草稿上的文字誊抄进试卷中,一个字一个字端端正正的出现在纸面上。墨磨得很浓,深黑的字迹直透纸背。但韩冈却不敢将笔蘸得很饱,而是每写两三个字便把笔放到砚台中蘸上一下,生怕落了几点墨迹,污了卷子。这么一来,速度更是不可避免的慢了下来。

“还有多少人没有交卷?”

曾布这时已经吃过了晚饭,喝着消食的茶汤,问着邓绾。

“大约还有百来人吧。”邓绾方才去外面的考场上绕了一圈,看了看情况,“不过锁厅贡生那边,就只有韩冈尚未交卷了。”

“素闻韩冈此人有急智,为人敏锐,怎么拖到了现在?”邓润甫放下了手中的茶盏,很是奇怪的问着。

邓绾道:“才思不同于才智。韩冈机变过人,但文章当非其所长。”

吕惠卿点头道:“旧日曾经看过韩冈写的疗养院暂行条例,以及一些公文,他的文字缜密得近于繁复,想必他写文章也是如此。写得时间长一点,也是情理中事。”

月亮也升起来了,初十的上弦月攀上了院墙,挂在树梢上,银色的辉光照进了偏殿中。烛台上尽是烛泪,烧到尽头的蜡烛闪了起来。胥吏连忙走过来,给换上了一根新的。想了想,他将烛台放在韩冈前面的一张桌上,以便照得考卷亮一点。

但韩冈这时却放下了笔,揉起了酸涩的双眼。

“韩官人,可是写好了?”两名胥吏连忙上来问道。

“请稍待。”韩冈不慌不忙的说着。他的确是写好了,但还没有检查,这如何使得?

韩冈从头到尾又看了两遍,贴经墨义的答案,还有刚刚完成的史论,一个字一个字的扣着。确定其中没有错字、漏字,同时也没有犯着杂讳。过了好半天,新换上的蜡烛又烧到一半,外面已经敲起了二更的鼓,这才将卷子交给了等在身前的小吏,并报以一个歉然的微笑:“劳两位久等!”

“不敢。不敢。”小吏上来将韩冈的试卷给小心的收起来,其中一人忙不迭的将卷子送了出去。他们多等了近三个时辰,才等到韩冈的交卷。

从考场中走出来,已是月上中天。天上的繁星被月光所遮掩,黯淡了许多。

可能是最后几个交卷的考生,韩冈出来的时候,周围已是安安静静。一盏盏灯笼挂在屋檐下,照得国子监内外灯火通明。踏着路面上的灯光,韩冈慢慢的走在贡院中,让过前面一队巡逻的士卒,对他们投过来的惊奇目光视而不见。他心思,却还在留在方才的考场中。

今天的考试,他准备了整整三年。虽然三年中,他经历颇多,放在书本上的时间只有一小部分,但他灌注其中的心力,自信决不比任何人要少。

在试卷上,每一题的答案,每一个字,每一句话,他都是仔细考虑再三,方才写了下来。

不会有问题的。

虽然没有章惇那样支撑起自信的文学才华,但韩冈已经把自己的能力都发挥了出来,相信最后的结果必然会给他带来惊喜。

韩冈一步步走出国子监的大门,门前守着一队来自于上四军的士兵,听到门后的动静,纷纷回头看了过来。而同时迎过来的,竟然还有王厚和慕容武。

跟着老远,王厚就喊着:“玉昆,你可叫我们好等!”

韩冈心中涌起一股暖意,向着他们一拱手:“处道兄、思文兄,真是折煞韩冈了。”

一等走过来,王厚、慕容武就异口同声地问道:“考得怎么样?”

韩冈回头看了一看:“只等发榜了”

