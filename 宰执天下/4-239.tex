\section{第39章 遥观方城青霞举(四)}

【第四更。睡了一觉,清早起来赶,终于赶出来了。果然是有压力才有动力。】

“修道,不就是为了长生吗?不能练就金丹,对百姓再有用,在他们眼中,也只是无用的废料而已。”

“所以对百姓来说,他们就是无用之人啊。”

对修道之人的评价,出韩冈之口,入王旖之耳,说说就算了。跟韩冈不同,他的几名妻妾虽然跟在韩冈身边耳濡目染,已经不是很相信一些骗人入彀的鬼话,但对佛道依然还是保留着几分敬畏,连王旖都不例外。

襄汉漕运的通道已经打通了,也就在这两天,消息就该传到了京城。虽然勾连襄阳和京城的渠道依然被方城垭口一分为二,但作为替代工程的方城轨道的完工,产生的影响应当也不会比渠道正式修成时差得太远。

等到秋税收过之后,需要大量民夫的堰坝、船闸,以及开挖方城渠道的工程就要开始,但对于韩冈来说,他来京西的几个主要目标,已经完成了其中一项,其后续的进度,他已经不怎么放在心上了。

只要方兴能够将六十万石纲粮在四十天内运抵京城,这就是对轨道最好的广告。接下来,推动轨道的全面发展就不是什么难事,更是顺理成章。通过轨道加强整个国家的物流能力,好处不仅存在于商业上,对于军事,也能有极大的裨益。

韩冈对轨道的计划很多,陆陆续续也整理了起来,形成了一个完整的规划方案。

他甚至打算将轨道的建设和管理向民间开放。国家打造干线,控制作为命脉的主干道,而放开来让民间修造支线,成为整个物流体系的补充。而全国性物流体系成型,使用的牲畜也将会是一个天文数字。马匹的保有量是一个国家实力的象征,大宋之前的水平让人叹息,但轨道网络形成后,至少能追近辽国了。

不过现在想这些也太多了,等到这个最终目标成型,还不知道要到猴年马月,需要多少年的时间去培养和发展。

韩冈盘算着轨道未来的发展,王旖则拿着丈夫的笔记原稿,继续往下看。看到不解的时候,就做个记号,等韩冈有空的时候再问。

这是门外的廊道上响起了脚步声。

“是素心。”王旖抬头喜道,她肚子正好饿了。

“不是她一个人。”韩冈摇摇头。

书房门响了两下,就被人从外面推开了。周南、韩云娘和严素心,不知怎么凑在一起过来了。三人手上都拿着一块巴掌大的水银镜,上面的红绳穿过镜纽,下面则垂着一条长长流苏,这是韩冈前些日子带回来的礼物。

“怎么了?”韩冈问道,“晚上还拿着镜子?”

周南将镜子递过来:“官人你上次给奴家的水银镜,怎么变得模糊了?”

“是啊,全都模糊了。官人你拿镜子回来的时候说过,有什么变化立刻就跟你说。”严素心和韩云娘一起点头附和着,一起将镜子递过来,很是疑惑的样子。

“模糊……怎么又模糊了。”

韩冈一手接过拿着如同多了几处霉斑的镜子。这应该是他一个月前,从工匠们那里拿到手的第一批水银镜。就在前两天,这块镜子还是闪亮如银,而今天,就已经是只能看到一部分清晰一部分模糊的人像了。

韩冈对此心中有底,同样的情况,在他听到的汇报上,出现过不少次,本以为这一次是成功了,没想到还是不能投入实际使用。

几个工匠为了试制水银镜,试验了汞和锡不同的配比。发现在汞锡齐中,水银用得越多,模糊得就越快,最后将水银降到了最小的限度,终于不会变得模糊。谁成想才一个多月,又复归原状了。

看起来是镜面镀层中的水银挥发得很厉害,也有可能是水银渗透到铜镜里面去了。

但不管是怎么回事,都证明用金属材料作为基材,同时无法密封隔绝的水银镜完全没有保有价值。

看来还是得用玻璃,韩冈想着。鎏银其实也是一种办法,但镀上去的银层容易发黑,这个问题不论怎么调整实验配比都没办法改变。

“官人这是怎么回事?”王旖还没有回屋看,但她的镜子想必也出了这个问题。

“自然是技术还不到家的缘故。”韩冈冲着四位位妻妾自嘲得笑了一下:“胜负兵家常事,水银镜就再让人去想办法好了,总有成功的时候。”

从周南、云娘手上接过另外两面镜子,几面一起递给身边的小婢:“去外院叫个人,送到城外柳树营去。夫人那里的,一会儿也让人带过去。”

水银有剧毒,既然会挥发,就最好丢得远远地,韩冈可不敢让水银来祸害自家的妻儿。

只是这样一来,显微镜的反光镜又成问题了,韩冈咂了下嘴,做个科学家还真是不容易,得不断直面失败,难道当真要他实验六百六十六次才能看到成功?

“官人。”王旖眼中多了些忧色,她拍拍桌上的书稿,“这个会不会……”

可能是因为忌讳的缘故,她没有把失败两个字说出来。

韩冈领会了,摇头笑道:“不用担心,伏龙山那里已经有好消息了。等李德新回来就可以了进行下一步了。”

曾经的金明寨寨主之子,陇右名医仇一闻的弟子,早在几年前就被从关西找来,投入了韩冈的门下。蕃人的身份让他很难与周围人交流,只能依附韩冈,而韩冈也正是因为这一点,有许多事可以放心的交托于他。

小事失败没什么,韩冈也不是很放在心上,但大事他可是慎之又慎,内外都做好了准备,也多次做过了验证,想失败也难。

……………………

“襄汉漕运打通了?”吕惠卿听到这个消息后,也是十分惊讶。

韩冈没将方城轨道的通车太放在心上,但无数道盯着襄汉漕运的视线却不会等闲视之。

方城垭口六十里的轨道的成功,就像一块巨石投入,顿时掀起了轩然大波。

“不能算是打通,只是通过方城山的那一段铺好了轨道。”

“看来韩冈还是急了点。”吕惠卿沉稳下来,“到底能不能成事,还得看今年秋冬。他能将京西南路和荆湖的秋粮,运多少过来。”

来跟吕惠卿报信的幕僚点头附和。

“不过以韩冈之材,运上六十万石,也并非难事。也可以说他已经成功了。”

从襄州坐船上溯至方城县,坐有轨马车走上六十里,到汝州再换船去京城。有通畅的道路运输,从这一件事上,吕惠卿知道韩冈是成功了大半。

轨道跟水道和普通的官道不一样,水道和官道上跑的车马船只,可以是私人的,也可以是官府的。但轨道上的有轨马车,只可能是一家独占。

只是这一点上,吕惠卿的眼前就仿佛出现了一个个铜板,叮当作响的落下来,洒了满地,闪烁着金色的光芒。

在汴河上,有能逃税的民船,但在轨道上,如何逃税?而且还有运费进账。

这样好的项目,如果自己这位参知政事插手进来,至少能让东南西北四座京城用轨道联系彼此。

“不对。”吕惠卿摇头。

韩冈是不可能留下这么大的破绽,他肯定会在确认成功之后,上书天子。就算自己第一个向天子建议在方城山以外的地方铺设轨道,但日后当真有了成果,自己也没脸跟韩冈争首倡之功。

一般来说,如果分润不到功劳,吕惠卿也就没了太多的兴趣。但轨道的作用,吕惠卿却是难以割舍。就算功劳不归他,但一旦成功,多了几条勾连南北的通道,做什么事都能多一分助力。

看来就必须等十一月的结果了。韩冈若是成功的将理因运送到扬州的粮食,通过襄汉漕运运抵京城,那么接下来,天子自然会有意通过轨道将无法用水路联络的州县,通过轨道联系在一起。

不过轨道有个坏处,就是必须在平坦的地方才能使用。如果有些斜坡,不是运力大幅下降,就是对挽马的要求直线上升。并不是皇帝金口玉言一开,就能让轨道面临的问题,就此烟消云散。

但吕惠卿不是很在乎,能将四座京城联系起来——大名府要过河——继而延伸到边地……比如定州、真定、沧州。

‘也只能在河北。’吕惠卿有着恍然大悟的感觉。就是不清楚韩冈究竟是轨道将官军的重心移到东面,还是利用轨道,给契丹人以压力,让他们不敢肆无忌惮的支援党项人。反正多一种手段,就是多一个选择,也是多一份保险。

种谔的奏章已经递到宫中几天了,天子是什么想法,现在还没有人能探明。开战是肯定的,到底是什么时候开战,还没人能说得准。不过有了轨道,有了新的漕运通道,天子可能又多了一份两份的信心。

不过韩冈究竟能得到什么样的封赏还说不准,进京任官几乎不可能,下一步究竟是回关西,还是去河北?

吕惠卿忽然发现,能决定韩冈到底去哪里并不是他,也不是两府,宰辅们甚至连影响都做不到,只有天子,只有天子才能决定韩冈的未来。

