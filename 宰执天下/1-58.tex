\section{第27章 夙怨难解杀机隐(下)}

已经离城十里,城东热闹的草市,早已成了极远处的一缕暗影。韩冈静静的站在官道边的凉亭中,眼望着东面。他仍是一身略显单薄的青布襕衫,高峻挺拔的身子似是感觉不到周围的清寒。呼吸凝成的水汽,在眼前结成白雾,寒冷的冬日清晨,大地寂静无声。王厚、王舜臣两人也似乎被这静谧的气氛所感染,只敢搓手哈气,许久没有说话。

不知过了多久,东面远方满目的雪白中,突然多出了一个黑色的小点。黑色小点越来越近,在众人的视线中已经分离成两骑一车。前面的骑手身材如公牛一般雄壮,一身厚实地冬衣遮不住身上块垒横生的肌肉,他身下的老马几乎被压垮了腰,一步拖着一步的在走,隔几步就是一声哀鸣,似是在叫着好累好累。在骑手身后,则是一辆由两匹马拉着的青布蓬两轮马车,赶车的应该是个熟悉道路的老把式,稳稳地将马车赶在官道正中。而在车子后面,又紧紧跟着一骑,亦步亦趋。

一见他们,韩冈便脸现喜色,连忙从亭中下去,站在路边候着,王厚和王舜臣如释重负,也跟着来到路边。

看到韩冈出现,前面的骑手突然加速,身后溅起的积雪如碎玉横飞,转眼奔到近前。在韩冈身侧,他一扯缰绳,飞身下马。老马重负得脱,正想奋蹄嘶叫一番,却被一只大手猛的强压住,动弹不得,四蹄直刨得雪地里多出了四个坑来。那名骑手豪放的定住坐骑,回身在韩冈面前单膝跪倒,“韩官人,赵隆幸不辱命。老爷,夫人,还有小云娘子,都已经给俺请了回来,还有官人舅家的二舍【注1】,也跟着一起来了。”

听说舅舅家的二表哥李信也来了,韩冈小吃一惊,抬眼看了看紧跟在车后的一骑,应该就是李信。不过自己就要做官了,亲戚来投也在情理之中。他急忙将赵隆扶起,温言谢道:“有劳赵兄弟了。”

“不敢称劳!不敢称劳!”赵隆连声逊谢。他视韩冈为贵人,发自内心的感激。自从结识了韩冈后,他便交上了好运。从城门守卫这个见鬼的差事上脱身不说,还被调入经略司听候使唤。跟在经略相公和机宜等大官身边虽是规矩太重,有些憋屈,但想到日后外放领兵的痛快,一些闷气的地方也不算什么了。故而当韩冈请他告假去凤翔府帮忙接父母回来,知恩图报的赵隆没有丝毫犹豫的便答应下来。

马车已到了近前,车把式将车停稳。一个小小的身影从车上跳下,扶着韩阿李从车厢中出来。韩千六跟在后面下车,韩冈的表哥李信也跟着下马。

相别再会不过一月,却恍若隔世。看着神色装束一如往昔,却已经成为官人的儿子。韩千六、韩阿李老泪纵横,韩云娘小手捂嘴,不让自己哭出声来,却也是泪水溢满了眼眶。

韩冈推金山、倒玉柱,在雪地中扑通跪倒:“爹爹,娘娘,孩儿不孝,让你们担心了!”

……………………

密室中,一灯如豆。

桌上幽暗的灯火,随着室中众人呼吸说话而闪烁不定。投在墙上的影子张牙舞爪的扭曲着,如同一头头凶戾的鬼怪,正欲择人而噬。

陈举的长子陈缉围桌而坐,继承了陈举慈眉顺目的一张脸如今狰狞扭曲,脸上的神情也与鬼怪无甚差别,“韩贼的父母回来了?……黄大!黄二!你们几个废物就干看着,一路追在后面?!”虽然声音里全是怒意,但音量还是被陈缉尽力压得很低。

黄德用的两个儿子脸色有些难看,陈举都要死了,陈家也完了,陈缉仍把他们两兄弟呼来喝去,当下人看待。要知道,他们的杀父仇人虽是韩冈没错,但直接逼死黄德用的,却还是不念旧情的陈举。只不过,如今都是一条绳拴的蚂蚱,同是被绘影海捕的通缉要犯,须得互相看顾,不好直接翻脸。

他为自己辩解着,“韩三派去接他父母的伴当可是城南纸马赵家的大哥!一身的好武艺!还没从军前,城南厢的地痞泼皮都给他打遍了,谁敢招惹他?”

“我难道不知赵隆那厮是谁?要你多口?他武艺再高,也不过就一个人!”

黄二帮着哥哥说话:“不止赵隆,还有一个,是韩家的亲戚。那厮警醒得很,不是个好招惹的。俺们跟了一路,都没找到机会,几次差点被他给看破。赵隆过去又跟俺们打过不少交道,一上前就会给他看出破绽。这两个人押着车子,夜里住的又是驿馆,急切间下不得手。”

黄大跟着道:“强行动手,俺们也怕打草惊蛇。失了风,让韩贼提防起来,以后怎么下手?”

“…………”陈缉沉默下去。

在座的都是陈举余党,在秦州也算是有头有脸的人物,谁想到转眼就成了逃犯。好不容易才逃过了缉捕,在秦州城外的找到了这个还算安全的落脚地。若说他们还有什么心愿未了,自然只有仍然活蹦乱跳的韩冈!

陈缉憋得胸闷,最后发着狠,“……等过两日过山风来了,一气灭了韩贼他满门!”

大宋天下自开国以来都不太平,王小波、李顺之辈,层出不穷。尽管大的反叛,自贝州王则之乱后,便再无一见。朝廷每逢灾荒便从灾民中收精壮为兵的政策,从根子上断绝了人数上千上万、席卷多州多路的叛乱。但自与西夏开战之后,疯狂增加的军费,以及大幅增长的官员数量,逼使官府收取更多的税赋。沉重的税赋负担让农民们无法承受,因而弃家逃亡的百姓、落草为寇的流民,二十多年里却变得越来越多。

七八人,十几人,小股的强贼按欧阳修奏章里的说法是‘一伙强如一伙’,甚至有的在光天化日下横行道左,劫掠民家,让地方州县焦头烂额。而那等挥起锄头种地,拿起刀来抢劫的业余强盗,更是数不胜数。天下各处路州,再无一日清净过。秦州尽管是军事重镇,但也没有例外。

狡兔三窟,陈举虽然明面上的家资尽没,但暗地里的积累还有一些。现在关西百姓的日子都不好过,找些亡命之徒也十分的容易。时近年终,强盗也要等钱过年,若能弄笔外快过个有酒有肉有新衣的肥年,没有人会说不愿意的。

过山风是一种毒蛇的名号,也是秦州附近的一伙有名的强人头领,手下有十几个小喽罗。陈缉拿着这些钱收买了他们。劫法场、救陈举,肯定没那个本事,但拿下韩冈的脑袋当个球踢,为自己出口鸟气,陈缉觉得还是没问题。

“四郎很快就会从凤翔押解过来一同受审,要不要先救了四郎出来再说?”黄家老大提出自己的意见,黄家老二也连连点头。

他们自黄德用畏罪自尽之后,便被陈举安排着去凤翔府投了四儿子陈络。凤翔府与秦州不是一路,秦凤路名字中的‘凤’字,来自于凤州,而不是凤翔府。黄家两子的海捕文书,虽然在凤翔府城门前贴着,但没两天就给新的公文盖了去。一人五贯的微薄悬赏,也引不动他人的贪念。而且老母妻儿很快又被陈举送了过来,两人在陈络庇护下,住得很是舒心惬意。

可舒心惬意的日子还不到一个月,便换作陈举倒台了。一封发自秦州的公文,让陈络直接在衙门里被绑下来,托庇陈络的黄家兄弟虽能幸运的逃脱,但家眷又给捉了去。只是这一个月时间,黄家兄弟跟陈络的交情深厚了许多,相对于陈缉,他们还是觉得跟着陈家老三更放心。

“先杀了韩冈,再反过去救四哥。”陈缉不想让韩冈警觉起来,“一月之间便毁了俺陈家几十年的基业,韩贼奸猾过人,再精明不过。若是先救了四哥,必惹得他警觉,到时再难下手!”

相对而言,诛杀韩冈也要比劫囚容易,不会造成多少伤亡,若是反过来就不一定了,伤亡惨重的队伍再想拉去杀人,可就难了。

说起韩冈,陈缉就恨得咬牙切齿。虽然仅是胥吏家的儿子,但陈缉自幼锦衣玉食,家宅虽然不敢造得过大,以防惹起官人们的嫉心,但内部的陈设却是秦州城中排得上的奢华。哪像现在他藏身的密室,安全虽是安全,但污浊的空气却让人窒息,陈缉何曾住过这等腌臜的房舍。

这一切都是因为韩冈!

陈举里通西夏一案,今天才正式开审,但结果早已预定,陈缉甚至都没心思去打听。他的老子陈举必死无疑,斩首都是轻的,多半还是被活剐,若是聪明点,现在就会自杀。

陈家的数十万贯家产,少不得被瓜分,连仆佣婢女,也会被发卖一空。而陈缉他的浑家和两个心爱的小妾,再过两日就要送进教坊司接客。陈缉不用照镜子,也知道他头上戴的幞头已化作了深绿色,苍翠欲滴。

陈缉紧咬着牙,牙龈上滋滋迸出血来:“韩冈那狗贼,不灭他满门,我誓不为人!”

注1:舍是舍人的简称。二舍,就是二公子,二少爷的意思,是对官宦子弟的尊称。

ps:陈举虽然就擒,但还有个儿子逃在外面,这是陈举势力最后一点余波。

第一更。本周强推,还望兄弟们的红票再给力一点,争取把宰执天下推到红票榜的第一页。

