\section{第32章 吴钩终用笑冯唐(14)}

吴逵生死不明,光靠一具焦尸完全无法证实身份。韩绛命人大索城中,掘地三尺也要把人给挖出来。在重兵围城的情况下,又有围墙壕沟,没人会认为吴逵能逃得出去。

不同于韩绛、种谔他们的心浮气躁,对于吴逵的下落,韩冈报着无所谓的态度。他只要三千名苦力就够了。虽然他事情已经隐隐的有些感觉,但还是装作什么都不知道。

如果吴逵是真的化身潜逃,他死中求活的手段也算是大胆了。而且正如韩绛此前所说,吴逵足够聪明。他前面的义薄云天的表现,使得跟随着他的叛将们没有在被围城时主动出卖他。如今潜逃,韩冈也不指望这些叛将能提供多少有用的消息——以吴逵之智,不可能让他们知道自己的去向。甚至其中还有几人,始终认为吴逵是不想让他们难做,而主动自焚。

但自韩绛以下,宣抚司没人会这么想,所以城中还在搜寻着。

只是吴逵刚刚从军时,曾在咸阳住了几年,地理算是熟悉的,搜寻起来不是那么容易。

另外吴逵进了城后,把麾下叛军管束得极为森严,还斩了两名犯事的士卒,以作警示,城中的口碑不恶。反倒是进城搜寻的官军,很有几个犯了点事。让燕达咬着牙,在城门口好生用军棍抽了他们一顿。

听说了此事,游师雄私下里对韩冈道:“吴逵做得聪明,这样就能让人明白他叛乱是出于无奈了。”

把治军严明的优秀将领逼反,要不是王文谅战死疆场,让韩绛洗脱了关系,光是这一条,就能把他堵得慌。

在搜索到吴逵之前,韩绛也不敢贸然住在咸阳城中,还是回返城外的营寨。韩冈等宣抚司僚属也跟他一起回返。而出降的叛军则被安排在城下,也就是两道城墙的之间的空旷地带,防着他们做反。

战事消弭,为了给前面做得准备清理后续,游师雄忙得脚不沾地。实在忙不过来,便拉了看起来很闲的韩冈帮手。韩冈在衙门里老做事的,一个能当五人用。两名能吏一起动手,很快就把事情理顺了下来,

手上的工作变得轻松起来,游师雄便跟韩冈扯着闲话,“吴逵此事,让我想起了一个人。”

“谁啊?”

“侬智高!”

“啊!是他!”韩冈一听,顿时恍然。对了,当年被狄青剿灭的广西侬人叛乱,也是如吴逵一样,叛乱的主角侬智高便是被火烧得认不出身份来,“吴逵也的确是像侬智高!逃跑的方式也是一样。这过往战例他记得不少啊……”

“玉昆,难道你就没想过吴逵当真死了?”游师雄却皱眉反问着韩冈,“如果他不想让人得到这份功劳,自焚是最好的手段。别忘了狄武襄,捉杀侬智高的功劳他最后没拿到手,是因为不能确认身份。不能确定谁敢报上去,万一突然冒出来,那就是欺君之罪。”

“景叔兄,难道你不觉得吴逵身边证明身份的铁枪有些说不过去吗?”韩冈同样反问着。

“可是当时侬人连伪作的平天冠和玉玺都有,就在侬智高尸体身上。”

韩冈被游师雄说得一时糊涂起来,但回忆起昨日见到吴逵的情形,却是怎么都不能相信吴逵会自焚。不过让人当成这样也不错,左右与他无关。看着燕达指挥着麾下将士,闹哄哄的把城里的每块砖翻过来,也蛮有趣的。

但韩绛很遗憾,对韩冈道:“可惜了玉昆你的功劳。”

“叛军出降,实与下官关系不大,而是慑于城外的官军……若是说下官薄有微功,那前面的陆都监也有功劳的。”

世上的事,不患寡而患不均。既然没能在第一时间把城中的叛军诓出城来,还要等过上一夜才出降,这个功劳虽然可以算在韩冈头上,但总有让人商榷和攻讦的地方,陆渊也肯定会出来争抢。正好韩冈本无意于此事,干脆就不要了。反正韩绛肯定要报上去,自己推辞一下,在天子面前留个好印象,日后的结果反而会更好。

另外韩绛也是没有功劳的,他为韩冈遗憾,也不过是移情而已。逼堵叛军,筑墙围城,功劳都是别人的。只要吴逵没捉到,韩绛都没脸去为自己去讨上一块蛋糕。见到韩冈推让,虽是纳闷,但以他现在的心情,也无意多问了。

掀帘而出,夜中的风微凉,清新的空气沁人心脾,让被帐内的油烟熏得头昏的韩冈,一下神清气爽。

已是深夜,城中还是在乱哄哄的搜寻吴逵的下落,城头上一片灯火通明。但找到也好,找不到也好,除了对韩绛等人有关,却影响不了大局了。在叛军出降的时候,陕西宣抚司的使命已经告一段落,

大帐边上,仍亮着灯火的小帐,是赵瞻所居。天子使臣现在多半是在兴高采烈的准备攻击韩绛。在韩绛到来之前,把叛军围堵在咸阳城中是他所指挥。而韩绛到来后,只是捡了他的便宜,却还是没有捉到吴逵。两相对比,赵瞻当然有理由嘲笑韩绛,想来他也会顺便敲打一下韩冈。

选择与赵瞻为敌,韩冈并不后悔。尽管他一开始并无意站在新党一边,但眼下的朝局,是非此即彼,没有站旁边看热闹的权利。

旧党以维护祖宗规矩为己任,讲究着循序而进,连吕惠卿、章惇等一干才能卓异的能臣,都被说是幸进之辈,又哪有他韩冈立足的地方?也只有新党一侧,才有新人涉足的空间。为了自己能顺利升迁上去,也只有选择王安石和他的新党。

至于赵瞻,韩冈完全不在意。同为天子使臣的可是有一个在罗兀城走到最后的王中正,这位王都知会怎么评价叛军和罗兀城呢?

十日一晃而过。

燕达终究还是没有找到吴逵,有狄青的先例在,韩绛也不敢把那具焦尸说成是吴逵本人的遗骸。罪魁未获,剿平叛军的功劳也便大打折扣。

而陕西宣抚司的处理结果也从京中出来了。

同中书门下平章事、昭文馆大学士韩绛,改观文殿大学士,出判许州。横山攻略功败垂成,其去职乃是情理中事。但韩绛能如宰相卸任的旧例,依然改授大观文,可见并非是降罪,只是普通的宰相出外而已——许州【许昌】离着汴京也近,更不能算是贬职。

陕西宣抚司,由知京兆府的郭逵暂时接任。只是韩绛所拥有的便宜行事的权力,郭逵向朝廷申请,却是没有被应允。在韩冈看来,郭逵的任务多半只是为结束陕西宣抚司的使命收尾而已。

至于赵瞻和王中正,他们都被召回了京中。

“最近几年,关中当是要镇之以静。”

这些天以来,韩冈跟游师雄的交情越发得好了起来,在等着郭逵来接手的时候,聚在一起评论着朝旨的用意。

“朝廷和天子的心意已经很明显了,短时间内,关中腹地再经不起第二次变乱。”

“现在就等朝廷对叛军的处置了,”韩冈叹了口气,“希望能有个好结果。”

“这一点玉昆你现在不用担心。”游师雄对惊讶的韩冈笑道,“朝廷最近有消息,秦凤路要从陕西路分出来了。”

……………………

天已将晚,这些天心情一直不好的赵顼,越发的显得很不耐烦,可枢密使文彦博却还是坚持着在宫外求见。

“跟文彦博说,朕累了,让他有话明日上朝后再说!”

听见赵顼不客气的言辞,李舜举犹豫了一下,但还是转身走了。

只是他没走几步,赵顼突又在身后喊了一声,“回来!”

他对转回身的李舜举道:“让文彦博去崇政殿候着,朕一会儿就过去。”

赵顼现在不想把这位枢密使给气得辞官。现如今,韩琦、富弼、曾公亮这等前朝宰辅一个个去职,如果文彦博再走了,朝堂上就再没有一位元老重臣。王安石等人虽然年纪都过了五十,但在朝堂上的资历依然浅薄,若是朝中真出了大事,少不了元老重臣的参与和压阵。而且赵顼也是需要一个不同的声音留在朝堂上。异论相搅,祖宗留下的话,许多也是有道理的,并不需要每一条都抛弃。

只是赵顼虽然答应召见文彦博,但他心里还是不想见着这位枢密使。

如果不是因为文彦博的强硬反对,他不得不多派了赵瞻出马,如今的关西也许会是另外一番局面。王中正能亲身入罗兀,而且是在断后的队伍中,直面西贼的追击。而赵瞻虽占了一点将叛军围困咸阳的功绩,可他的几番插手军事,也坏了不少事情。尤其是逼迫罗兀撤军,更是让赵顼心痛不已。

两千三百余斩首,加上都枢密、还有一名党项宗室,而且是正面击败了党项大军。现在越想,赵顼就越是后悔,如果当初换一个选择,也许横山之事就已经定下来了。

年轻的皇帝按耐不住这样的想法,总忍不住要去后悔。

一想到能彻底解决西贼的机会,跟他擦肩而过,悔恨如同毒蛇,在赵顼心中噬咬着。

