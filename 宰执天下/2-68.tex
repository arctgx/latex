\section{第15章 前路多坎无须虑(四)}

【第二更,第三更肯定能赶出来。求红票,收藏。】

目送着仇一闻师徒出门,韩冈转身走回厅内。严素心已经在客厅中。点汤送客的官场习俗她也知道,看着韩冈向厨房要汤水,自然明白客人要走了。

“还以为官人要留饭呢。”严素心手脚麻利的将几个青瓷茶盏收拾起来,一边很自在跟韩冈搭着话。

“他们是来道谢,可不是来蹭饭的。”韩冈说着又坐了下来,把自己杯里的酸梅汤喝光。严素心走过来,接过杯子,连着放在几案上盖子一起拿起来。只是她一弯腰,胸前一抹玉色从垂开的衣襟中透了出来,在韩冈眼前闪过。

韩冈一下怔住了,而严素心却毫无所觉的再次弯下腰擦着几案,那一抹动人的白腻又在韩冈眼前晃着。

“今天跟着来的是仇老郎中的那个坐监的徒弟吧?前些天就听说有个李郎中因为没治好窦总管的重孙子,被关进了大狱里。弄得城里的郎中们人心惶惶,都怕去官人家看诊。”

比起在陈家时,严素心在韩家要忙上许多,但她的心境却比在陈家时要舒畅许多。没有了日夜都在噬咬心灵的血海深仇,又没了在仇人面前还要强作欢笑的痛苦,严素心在无人时,总是不自觉的开心的笑出声来。而且韩家都是好人,老爷、夫人从不打骂,反而嘘寒问暖,而她的恩人也是和和气气,没事还能说说话,而且还是个守礼君子……

‘就是太守礼了!’

带着点莫名的嗔意,严素心往韩冈这边瞟了一眼。正正对上的眼神却一点也不守礼,反而仿佛有两团火焰在里面熊熊燃烧,包含着侵略性。

严素心被吓了一跳,啊的一声轻叫,连退了两步,双手捂着胸口,娇躯不禁轻轻发抖。

看到严素心如被逼到绝境的小兽一般的胆怯模样,韩冈虽然从让人沉醉的美景中惊醒,但一点恶作剧的心思又起来了,眼神更加肆无忌惮,看得严素心的如玉一般的小脸鲜红如血。

此时天气热,严素心穿得单薄。外罩一条银红色的薄纱褙子,褙子是对襟而开,与穿在里面右衽的长袍不同,就像后世的大衣,不过没有袖子,没有扣子。褙子底下是月白色的凉衫和鹅黄色的罗裙,都是轻薄得一阵风就能吹起来。。

韩冈自忖这些天来实在是浪费了不少时间,正想着是不是今天晚上一偿夙愿,严素心却是一咬银牙,红着脸捧着收拾好的杯盘茶盏,逃跑一般的急匆匆地往外走去。

透过毫无遮挡的薄纱褙子,可以见到一条蓝色宽幅绸带正紧紧扎在腰间,纤细柔韧的腰肢被勾勒出让人窒息的绝美曲线,而本还稍嫌青涩的双.臀,在纤纤小腰的对比下,却是显得丰盛圆润。少女步履匆匆,纤细的腰肢款摆,摇晃出让人迷醉的旋律。

韩冈眼睛眯了起来,视线追逐着动人的韵律,一直到消失在门外,再也挪不开去。心里想着,当真是浪费了太多时间了。不过既然已经醒觉,今天夜里的时间就不会再浪费了。

为入夜后做好了盘算,韩冈往内进走去还没走到正堂门口,就听见一个陌生的妇人声音从父母的房中穿了出来。

韩冈脚步随之一停,一转身,转往书房去了。这些三姑六婆来自己家,肯定没有好事。

书房里,韩云娘也在打扫着卫生,正拿了块布擦着书架。比起年初的时候,她个头没长多少,但胸前的起伏更加明显了,从侧面看去,月白色的绸衫下隐约透着里面的红色肚兜被看得分明。她掂着脚,够着去擦书架的高处,胸前的隆起就是一阵让人口干舌燥的微微颤动。

只看了两眼,心头又是一片火热。韩冈深吸了一口气,摇了摇头,感觉自己压抑得实在太久了,火头一被点起,就怎么也压不下去。果然太过压抑自己,对身体健康实在不好。

云娘不知道韩冈已经走了进来,还一蹦一跳的努力够着最高处的书架。娇小的个子,让她擦不到书架的最高一层。但她这么一跳,已经成长起来的酥胸,却是晃动得让韩冈的心火更旺。

不能再这么看了!韩冈竭力让自己清醒了一点,再这么看下去,真的要做出事来。小丫头可不是跟他年岁相当的严素心,过早接触男女之事只会伤了她。

从后面将抹布抢过来,在韩云娘叫着‘三哥哥’的惊讶声音中,韩冈抬手将书架最上面的一层给擦干净了。把抹布还回去,小丫头还嘟着嘴很不高兴的样子,直说着‘这些家务事三哥哥你怎么能做。’

韩冈不理小丫头的抱怨,坐下来,冲着父母的屋子呶呶嘴:“又是哪家的媒人上门了?”

韩云娘摇了摇头,“就知道前天来的是前街的李大姑,昨天两个都不认识,今天的也不认识。”

韩冈哼了一声:“一家一家的,还真不嫌麻烦。”

虽然这些日子,他清闲得紧。除了王厚等人,也没人来打扰他读书。但从后门进来的媒人却是络绎不绝,每天不断。

韩冈虽然刚得官时,很是风光了一阵。但后来因为他属于王韶一派的中坚人物,接连得罪了李师中、窦舜卿和向宝这三位大佬,让他的行情在秦州城中有待嫁女儿的家庭中下跌了不少。而接下来两派之间虽不见刀光血影,却依然惨烈的厮杀,更是让他落到了无人问津的地步。

可谁也没能料到,王韶区区一个机宜文字,竟然在与李、窦、向三人的争斗中获得了最后的胜利。秦州最高位的三名重臣,无不是在大败亏输后被赶出秦州。前日天子降下诏令,将韩冈本官晋了一阶,普通选人哪有这般幸运,都是流内铨发个公文过来就了事。且眼看着古渭大捷的封赏又要跟着下来,使得韩冈炙手可热,重新变成了众人争抢的香饽饽。

但韩冈却对这些把他当成肥肉的恶狗毫无兴致。王韶已经在江西帮他找了一门亲事。前些日子已经听王厚说过了,是王韶病故的前妻的内侄女,也就是王厚嫡亲舅舅家的女儿,如果真的结了这门亲,韩冈与王家就是姻亲了。

不过王厚的表妹才十三岁,离世间女子出嫁的底限十四岁,还差一年。按王韶的说法,先把生辰八字换了,把聘礼送过去,到明年那边就可以把人送到秦州来了。但由于紧接着郭逵要来秦州的消息太过让人震惊,王韶、王厚现在都忙得没地方站,早把此事放到了一边去。连韩冈自己都因为读书忘了,现在才想起来。

人生大事,既然想起来,就少不得要跟父母说一声。韩冈等着正堂那边再没了声音,便走过去。进了房,只看到韩阿李一人坐着,手上正对比着两块鞋样,却不见韩千六的踪影。

“娘,爹爹他人呢?”韩冈便问着。

“还能去哪?”韩阿李抬头白了儿子一眼,“又去普修寺了。天天往和尚庙里跑,回来都带着一身的烟味。这两天老是念着阿弥陀佛,烦都让人烦死!”

韩阿李好一通抱怨,韩冈听了,也不知话该怎么说。自家的老子种田是把好手,但除了农事以外,他却没有别的擅长。自从进了城之后,韩千六在家无事可做,又不像韩阿李那样经常又三姑六婆上门跟她闲扯,他在秦州城里根本找不到个伴,也只能每天往普修寺去找住持和尚聊上几句。

韩冈叹了口气,不管怎么说,烧香拜佛总比欺压良善要好。

韩阿李放下了手中的鞋样,沉着声对他道:“照俺说,家里要是还有块地就好了。让你爹他去料理一下,也省得他天天闲得慌。就算现在做了封翁,不好下地。租佃出去,闲时让他去绕几圈也是好的。”

韩阿李这是想要家里买些田产,但韩冈觉得不能这么浪费自家老子的种田技术。在过去,靠着韩千六的指点,下龙湾村田里的出产硬是比周围村子高了一两成去。

他想了一想,觉得趁机将藏在心底的一些打算先说出来一点,“这样吧,最近古渭寨就要开始屯田了,那里的荒地有几千顷,上好的河滩地也为数不少。机宜现在要从秦凤路上招募弓箭手来开垦。到时候孩儿在靠着寨边上的地方,划下几顷田来,让爹爹去照管也就是了。”

等屯垦开始后,韩冈就准备请王韶和高遵裕一起上书天子,在古渭寨边划出一部分宜垦荒地,作为奖励,赠给主管屯田的官吏们。

一般情况下,这等提议是犯忌讳的。由官府组织征发民伕、士卒开辟出来的土地,比如淤田所得,比如河滩新田,又或是得到新辟沟渠浇灌的荒地,一部分要归属参与工程的民伕和士卒,剩下的则是收入官府。而官府通常会将这些田地发卖出去,换成现钱。从律条上说,严禁官员从中渔利。

但韩冈借口也想得好,连主管的官员都不敢在古渭置办田产,百姓能相信古渭一带的安全吗?这不是为了私利,是为了稳定民心。只要提前把事情公开了,得到天子的同意,就不用忌讳日后有人说他假公济私。而且这么做,在实际上,也肯定是有效果的。

