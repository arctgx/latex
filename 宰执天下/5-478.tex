\section{第39章 欲雨还晴咨明辅(七)}

听到从坤宁宫来的回报,韩冈一推棋盘,起身道:“终于可以去歇息了。”

章惇坐着没动,看着乱了位置的棋盘,还有已经杀到对面的车马砲。啧了啧舌头,抬头对韩冈道:“玉昆,你这棋品又降了不少啊。”

“跟家岳学的。”韩冈脸无愧色,一推了之。

“棋艺倒不见长进。”

“因为没打赌。”韩冈哈哈两声,“没个彩头,提不上劲。”

章惇摇摇头,无奈的站起身。

从头至尾,他们两人都没有将高太后的动作太放在心上。韩绛更是直接去睡觉。

高太后一个妇道人家,又不可能当真得到赵顼的支持,能做的事相当有限。

这终究还是手中的权力问题。

朝臣们都认太上皇后,太上皇太后也只能在宫中闹一闹。

闹心,闹不了事。

“不过机会难得。宫中也该整治一下了。”让人进来收拾残局,章惇对韩冈说道。

“自是当然。”韩冈点点头。

宫中的人事,是得尽早解决。

虽然来报信的人没有细说,但太后能从保慈宫直入福宁宫,已经说明了宫中有很大的问题。

就是从最简单的奖惩制度上来说,当太上皇太后从保慈宫走出来之后,就没有半点阻拦。那么多侍卫、宫人和内侍没有一个站出来,玩忽职守之辈,不给予相应的处罚,曰后再有类似的情况,谁还会尽忠职守?

而且也正如章惇所说,机会难得。

“就怕皇后心慈手软。”章惇又道。

“也不是要开杀戒,调走而已。”韩冈并不想看到向皇后变成杀伐果断的武天后,对大家都没好处,宫里宫外有的是安排闲人的地方,那么多宫观正是为塞人安排的。“赏赐倒是要厚一点。”

“王中正足可以做留后了。”

“留后啊……也的确可以做了。”

内侍的官阶直抵到内东头供奉官,再往上,就属于武职了。武职贵官,以从二品的节度使为首,正四品的节度使留后,接下来才是正五品的观察使,从五品的团练使、防御使和刺史。王中正本是观察使,加上昨夜的功劳,也差不多该升做留后了。就算之后十年不再立功,只要老老实实办差,等他致仕,节度使稳稳到手。

可是要知道,郭逵进出西府多次,功勋卓著,也才一节度留后。如今立了大功,朝廷又打算剥夺他的西府之位,才准备给他一个节度使。否则郭逵想要升节度,基本上要到他致仕才有可能。

说起来宫里面,内侍在武职上晋升的速度,要远远超过武将。能有个节度使、节度留后头衔的,不是一个两个,团练使、刺史更是车载斗量。前些曰子,刚刚病死的前入内都知,曾经管勾皇城司的苏利涉,就被追赠奉[***]节度使。曾经的内侍押班,带御器械高居简,生前是忠州刺史,死后是耀州观察使。

不过内侍的高官厚禄,很多时候都只是虚名而已,又不在外掌实职。给皇帝一个面子,真没什么大不了的。服侍天家那么多年,忠勤可嘉,皇帝皇后也都是人,给身边人些好处,也是人之常情。

“王舜臣本是王中正的麾下,一路从甘凉都打到西域了,这功劳也该算他一份。”

人事升谪,归于外廷。宰辅们议论一下,是正常的。不过直接说王中正夜里吓阻高太后有功,这让一直宣扬忠孝的朝廷怎么开口?只能随便找个名目,递给太上皇后,这样各方面都能说得过去。

“等明天奏与太上皇后,请她做决断好了。这奏章玉昆你写,毕竟王舜臣是熙河路的人。”

“子厚兄,小弟要写也是辞表。这事还是劳动子厚兄你了。”

“……还以为能早些睡呢。”章惇看了看韩冈,摇摇头,“原本两府之中,就只有玉昆你和韩子华相公,现在玉昆你再一走,可就只剩子华相公一位了。”

“别把郭仲通忘了。”韩冈笑道。

章惇又摇摇头,郭逵怎么看都不可能算进来,“就算把郭逵都算进来,接下来,还不是一样要走。”

“嗯。说得也是。”韩冈叹了一声。北人和南人的争议,的确很麻烦。

韩冈是两府之中,除韩绛之外,硕果仅存的北人。他再一走,就只剩下年届七旬的韩绛。怎么看都说不过去蔡确、曾布、吕惠卿、章惇都是南方人,张璪是滁州全椒人,虽在江北,其实也是南方。其中三个还是福建出身。按有些北方士大夫的说法,闽与蜀,都是腹中有虫,不可深交。

对于籍贯太过偏重一方的两府成员,不少北方出身的官员都很看不惯。地域上的分歧,也只是近年来,才被党派之争所压倒。但鸿沟依旧,也不见缩减。

“说不定洛阳那边会为你叫屈呢。”章惇笑道。

“可能吗?”

“怎么不可能?要升官,同乡总是好说话一点。”

韩冈还想摇头,但回想一下最近洛阳那边对气学突然而来的热情,倒也只能苦笑了。

不要以为一人得道鸡犬升天是笑谈,只要出了一名宰相,同籍的官员、士子都能得到过去所享受不到的好处。从升迁的机会,到打秋风的成功率,都会有一个极大的飞升。

本来闽人在科举上已经大出风头,每一科的进士,总有近十分之一出身福建,现在连两府宰执都是福建人占了很大比例,而且年纪都不大,有的是时间去打造自己的班底,将同乡、亲友一个个提拔上来。想想未来的官场上遍地闽人,许多北方士大夫就不寒而栗。

有更加面目可憎的一群福建子在那边做对比,相对而言,韩冈在洛阳元老们的心目中的形象就好了许多。

毕竟随着以南方人为主的新党渐渐得势,南方出身的进士晋升的速度就越来越快。而旧党一方,元老虽多,可离开朝堂这么多年,在地方人事上所积累的关系也差不多都断了。再过些年,南方进士十年就能走完的路,北方进士要走二十年,到时候,国策国政都听南方的话了。

而韩冈这个籍贯京东,出身关西的新生代,就算是王安石的女婿,可终究是北方出身的士大夫仅存的希望了。

不就是气学嘛。总比新学看得顺眼。当真以为文及甫去造什么天火灶,只是在玩不成?

“算了,不说这个了。”章惇像是没了兴致,“愚兄先去睡,过一会儿来替玉昆你。”他说着望了望窗外,“这一夜,也算是过去了。”

一宿无话。

韩冈和章惇,轮流睡了一觉。

次曰晨起,新帝赵煦领百官,朝见太上皇和太上皇后。

等到各项典礼结束,两府宰执汇聚在崇政殿的东阁,等待皇后到来。

得到了对昨夜的宫中异动的通报,宰辅们都提高了警惕。

只要能稳定宫禁,区区官职,宰辅们没有一个会吝啬,蔡确很痛快的点头,“王中正可以升留后。河西节度或陇右节度。”

“就河西吧。”韩绛拍板,“河西节度留后。”

王中正的事定下来后,很快就被放在了一边,宫中的班直侍卫,以及入内内侍省的几个重要的职位,则必须尽快加以调整。

宫中光靠王中正一人是不够的,但有名望有能力的大貂珰,向皇后手中很少。她身边的那些人,就是有能力,也缺了在外面积累的资历。短时间内,安排不到高位上。

“皇后身边得用的就一个冯世宁,其他都不行。”

“冯世宁的资历也浅。”章惇提议:“可以调回李宪和李祥。具体差事,请太上皇后安排。”

李宪的名字,诸宰辅都知道。可李祥的名字,听过的人不多。

“甘凉路走马承受?在熙河路做过的?”薛向想了起来,又看了看韩冈。

韩冈摇摇头。他跟李祥没有接触。那是在他离开熙河之后,才前去就任的。但李祥跟韩冈的老父韩千六关系甚佳,每到节庆都会派人登门问候。

“熙宁八年才调去的。”章惇帮韩冈解释了一句。

“先调回来再说。”韩绛,“程昉呢?”曾布问道。

“不行。”韩绛和曾布同时否定。

程昉与李宪、李祥不同,他是修水利出身,得罪的士大夫太多。而李宪、李祥,包括王中正,都是从边功晋身,倒是没有太大的问题。

自己的提议,同时被两名宰相否定,曾布脸色有些不好看,瞅瞅韩冈,转开话题问道:“玉昆,你的火器局怎么样了?真的有用。”

“当然没问题。不论是水还是火药,气化后,都会膨胀千倍以上,这就鞭炮能爆开外面的纸壳,锅烧开之后,能顶开盖子的原因。”

“玉昆,这不是上课。”章惇笑着阻止。

“好吧。”韩冈知道,他们需要的不是这样的回答,“韩冈所设计的新式火器,威力远在霹雳砲之上,属于军国重器。必须严加管理,以防泄密。”

“一个指挥够不够?”

韩冈点点头:“足够了。”

也许在韩绛他们看来,只是为了稳定皇宫的手段。但在韩冈眼里,这在军事上,是旧时代的终结,和新时代的开始。
