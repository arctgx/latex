\section{第33章 女儿心思可知否(下)}

“啊!”在小丫头的挣扎中,韩冈突然低低的叫了一声,嘴里咝咝抽着凉气。

韩云娘立刻不赌气了,回首看着韩冈紧皱起的眉头,还有脑门上冒起的汗水,她一脸紧张的问着:“三哥哥,怎么了?”

韩冈没回答,他右手按着腰部,脸上的表情有着说不出的痛楚。

“三哥哥,你没事吧?”韩冈的反应,让韩云娘的声音里都带了哭音。

“前两天从马上摔下来,扭了筋……”韩冈说起谎从来都不带眨眼,一颗芳心都系在自己身上的小女孩更是好哄骗,他眯起眼,很享受的任由韩云娘柔嫩的小手在自己的腰上揉着。只是渐渐的,从小丫头身上传来的淡淡香气,将韩冈藏在心底的火焰渐渐引起,呼吸不由得粗重了起来。

“好一点了吗?”韩云娘抬起头,关切的看着韩冈的神色,浑然不知自己的动作,有多大的吸引力。

韩冈如今是个身强体壮的青年,正常的生理需要也是有的。可是小丫头的年纪摆在这里。韩冈并非道学先生,但虚岁才十三的小女孩子,怎么也难下得了手。而且也要担心着没有安全措施,万一让小丫头有了身子,身子还没发育完全的她,根本不可能平平安安的把孩子生下来,一尸两命是板上钉钉的。

韩冈能舍得吗?想都不用想。

身边连个出火的地方都没有,韩冈现在想着是不是真的要去惠民桥后的私窠子里走走?但万一得了病怎么办?虽然不会有据说是由猩猩传给人类的绝症,但其他病症应该不会缺。而韩冈,一向很爱惜自己的健康。

当然喽,千年之后世间流传的花样繁多,即便不走正途也有许多旁门手段,韩冈于此,理论和实践都不缺。只是他看着韩云娘犹带着稚气的小脸,还有认真的为自己按摩伤处的专注,便下不去手。韩冈欲哭无泪,太亲近了其实也不好,他都想不到自己竟然还有变成‘禽兽不如’的一天。

韩冈暗叹了一口气,自我安慰着,美味要慢慢吃下肚,猪八戒吃人参果那般可不行。他用力捶了下自己的脑袋,引得怀中的少女不解的抬起头来。算了,算了,还是多洗两遍冷水澡吧!

他抬起头,望着被火光映红的房梁,明天就是立春,比起正月初一的元旦,这才是真正的一年之始,也是很重要的一个节日。后天便要上路东行,往东京城报到去了,明天正好有空,去参观一下这个时代的节日祭典也是件乐事。

……………………

烛花爆了又爆,晕黄的火苗仿佛在跳着拓枝舞,在半截红烛上闪动的厉害。

严素心用力闭紧酸涩的双眼,眼珠子胀痛得厉害。在晃动的烛光下,要盯着手上正在绣着的鞋面,实在很耗眼力。不用等到明天早上,她现在眼皮下缘上的青黑色,都已经是用粉也遮不住了。

放下手上绷着缎面的花箍,将针线别在了绸子的一角。宝蓝色的缎面上,一朵缠丝夹黄的牡丹花已经绣到了底下的两片叶子,洛阳重瓣牡丹中最为有名的金带围,好似就生长在这块手掌大小的绸缎之上。

再有一天工夫,这双寿鞋就该绣完了,可家里取暖用的炭薪今天却已经烧完。严素心苦恼着,手指揉着眉心,她现在身无余财,只能靠着刺绣的手艺养活自己和招儿,但吃饱肚子已经不容易,哪里还能找出钱来再去买炭。

“六姐姐?”身后床榻上,一个粉雕玉镯的小女孩儿从被褥中撑起身,坐在床上很困的揉着眼睛。

听到声音,严素心忙转过身,又把她塞回到被子中去,“招儿,你继续睡吧……别起来。”

“六姐姐不睡吗?”抓着被角,招儿的一对大眼睛忽闪忽闪的。

“六姐姐一会儿就睡。招儿乖,听六姐姐的话,快点睡。”

小女孩儿很老实的点了点头,乖乖的闭上眼睛。才七岁的招儿跟严素心没有任何血缘关系,但她的娘亲同样是陈家的婢女,一直都很照顾严素心。前两年招儿的娘亲病死后,严素心便把她留在身边照看。

招儿应该是陈家的女儿,却不知是陈家的哪一位留得种,并没有被承认身份。今次陈家覆灭也就幸运的逃脱了落入教坊司的境地。同样幸运的还有严素心,她只是陈举的侍婢,而不是在宗谱上录了名的妾室。也便没有与陈举的几房妻妾一样,被送进教坊司中接客。

当陈举阖族覆灭之后,参与盛宴的一众官吏只留了一小部分陈举和其党羽的家产归入官中,剩下总计价值五六十多万贯的资财,便坐下来各自分赃。

其中田宅地产最受欢迎,尤其是陈举家的产业,更是人人争夺。陈家在秦州扎根近百年,拥有的田地多是良田,宅邸店铺也是位置优越。百年的积累,家世单薄一点的官宦家庭都比不上陈家这样深深扎根于地方上的土豪。

太平宰相晏殊在世时家中显贵无比,一曲‘梨花院落溶溶月,柳絮池塘淡淡风’,从骨子里透着富贵气派。但到了他儿子晏几道这一辈,尽管还有富弼这位宰相姐夫在,晏家就已经有了几分衰败的气象。富弼如今已年过六十,再得几年,等他过世,晏家定然会破落下去——晏几道那等富贵公子,小词写得是好,却没有保守家业的本事。

太宗朝的宰相向敏中,他在世时权势煊赫无比,但在他儿子的那一辈就已经败落了,孙子被更是可怜,若不是幸运的出了个当上了太子妃、如今又成了皇后的曾孙女,家势哪有重振的机会?

隋唐时的崔家、裴家那样代代高官显宦的山东世家,在晚唐五代的藩镇内乱中,早已灰飞烟灭。宋代的官宦家族,富贵容易,败落也容易。田宅地产流转不定,俗语道‘千年田换八百主’,说的便是此时的世情。真正能长久富贵的,反倒是稳守家乡的地方土豪,才能长保家族百年平安富贵。

陈家便是这样的百年家族,故而在陈举家中奔走的仆役婢女,兴高采烈的分享着陈家家产的秦州众官便没人愿意收下他们。他们都会是陈家的家产,而且是很值钱的一部分,但就是没人肯去要。

因为这些陈家的仆役婢女大部分都是家生子,服侍陈家几代人,谁也说不准里面有没有想为陈举报仇雪恨的。要找忠心可靠的仆佣,世上有的是,任用乡里不比把仇人放在身边安心?最后全都遣散了了事。

严素心也趁机带着招儿逃出生天。自陈家出来后,她就在城南租了间屋子。事前小心藏起的一点积蓄,再加上她出色的针线活,让她们度过了年关。

就在这段时间里,陈举在菜市口挨上了千刀万剐,当年祸害了她全家的仇人就这么被片成了一堆碎肉。而陈举的帮凶们,也不是被斩首,就是被流放。

严素心其实很开心,不共戴天的仇人受了世上最惨毒的刑罚而死,她不可能不开心。但当李师中掷下一根令牌,刽子手举起了手中的短刀,开始碎割着陈举,从菜市口传来的看客们的欢呼声不断传入耳中时,严素心一时间变得茫然失措起来。

她犹记得十年前,同样是在冬日。娘亲一边哭着,一边用力掐住自己的脖子。泪水不住滴在脸上,滚烫滚烫。出身世家的娘亲,自幼娇生惯养,比锅铲重的东西都没拿过。但那一天,娘亲的手力气很大,大到她怎么也挣脱不开,大到她很快昏死了过去。当她再醒来时,娘亲已经变成了挂在房梁上的一具尸体。而在此前一天,她爹爹的死讯正从南方传了回来。

严素心本以为要用上十几年时间,才能收集到足够的证据,为父母报仇,让陈家与自家一样家破人亡。但没想到才十年的功夫,好不容易取得了陈举的信任,就有人帮自己完成了夙愿。失去了宁愿以生命为代价也要实现的目标,她的心中仿佛突然间多了一个洞,空空落落,走起路来都如同幽魂。但又轻松了许多,连呼吸也轻快了,仿佛沉甸甸的一块巨石被撬掉了一般。

截然不同的两种感觉,在心中纠缠不清,几乎让严素心疯掉。她感激着王韶、韩冈这些把陈家一举毁灭的恩人,但同时,她又恨着自己不能亲手为父母报仇雪恨。

如果是由自己把陈举送入地狱,那该有多好?

烛花闪烁,火焰轻轻摇晃。严素心用剪刀剪去多余的烛芯,烛火重新稳定的燃烧起来。就着烛光,她又拿起缎面,接着飞针走线起来。

又不知过了多久,烛泪已经流满了烛台,严素心也终于将最后一片叶子绣好。放下花箍,神思从针线中脱身出来,感到了一丝放松。可这时,原本因为聚精会神而忽略掉的声音传入耳中。

身后的招儿略显急促的呼吸声,把严素心吓了一跳。她连忙用手背试了一下招儿额头,微微的有些发热。果然是生病的缘故。严素心轻轻抚着招儿的额头,心情被这场突如起来的病闹得胆战心惊。

‘这病,明天能好吗?’

ps:晏殊与人论富贵,看不起那等把金玉之词堆砌起来的作品,说是那种是暴发户,真正的富贵要从平淡中来,如他的‘梨花院落溶溶月,柳絮池塘淡淡风’,这才是真富贵。

今天第一更,红票。收藏。

