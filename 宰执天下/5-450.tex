\section{第38章 何与君王分重轻(八)}

商人喜欢银绢和钞引,官府收税也喜欢白银和布匹丝绢。

不论是商贸,还是税收,轻便而价值贵重的物品,总是更受欢迎。

这主要还是钱币的价值太小,无论是铁钱还是铜钱,都是重且贱,长途运输的话消耗在钱币上的运费,往往比起其本身价值都要高。

对于轻便且易于携带的货币或代替品,商人和朝廷方面都有很大的需要。

韩冈在报上发表文章,又对皇后提议继续铸造大钱,正是打算满足这个需要,但冯从义的心意呢?身价亿万的冯大官人,可不会无聊的有说闲话的时间。

“有话就直说吧。”韩冈说道。

冯从义干笑了两声:“哥哥知我,其实小弟也没打算做什么,安安稳稳赚钱最好。只是平安号仅仅是做飞钱,实在是太浪费了。所以小弟有个想法,就不知当不当说。”

“都让你直说了。”韩冈摇摇头,自家兄弟还绕着圈子说话。

“朝廷的钞引可否集中在长安兑换。这样也可以给朝廷节省一点。”

韩冈摇了摇头,笑问道:“客人多了不一定是好事吧?”

冯从义的笑容维持不住,韩冈一眼就看透了他的用心,亏他还准备好怎么去说服韩冈。

“的确如此。”他无奈的点点头。

朝廷颁发的钞引,其实际价值往往要比甘陕之地的粮价要高得多,这是为了吸引商人运输粮食去边境军寨。毕竟运费太贵,而朝廷运粮的话,损公肥私老鼠又太多。交给商人,免了运费和损耗,得到的好处比起朝廷在钞引上的付出多上好几倍。

不过那些拿到钞引的商人并不是亲自组织人员去运粮,而是就近买粮,甚至不买,而是动员边民/运输粮草,从边境军寨中交换到钞引之后,那些商人才出来收购。给出价格当然不高,但边民不可能去京城兑换钞引,卖给商人并不亏本,还能有些收获。而那些商人走南闯北,往往在京城也有背景和势力,拿着钞引去兑换,不用担心被克扣和拖延。这就是入中商人们赚钱的地方。

可入中商人也有个问题,他们想要将生意做长久,就必须将他们在京城得到的盐或钱或其他实物,统统换成钱,然后一年年的运去关西,否则手上没钱,怎么使唤得动那些边民?对很多商人们来说,最亏本的就是这一段运输,大大的影响到了他们的收成。

而现在就有了平安号。

平安号的主要业务就是京城和关西之间的飞钱。商人在平安号的京城分号中将钱存进自己的账户内之后,便得到了一张凭证。拿着这张凭证便可以轻松地回到关西,在长安、秦州或巩州这样有分号的地方将钱取出来——不过还要些手续费就是了——拿着钱,商人们可以再购买当地特产,然后再运去京城贩卖。接下来就是一个新的循环。

在这其中,只要相互间拥有了更进一步都信任关系,就再不要在关西分号内将钱兑换出来,直接在平安号的账户内进行交易。购买棉布为主的商品时,直接转账就行了,而后去拿货。毕竟平安号的股东都是雍秦商会的成员,他们手上的商铺有很多就在经营特产。

如果飞钱业务仅仅如此,就是双赢的好事。可现在平安号中逐渐出现了入中商人的身影。他们都是在京城存钱,然后在关西取回,最后带着钱去边地,购买百姓手中的钞引。也就是说,京师分号是硬通货净流入,而关西的几家分号则是净流出。想要改变这一切,就必须从京师运钱到关西。可这偏偏是朝廷都不愿意去做的折本买卖。

冯从义张开口,韩冈拦住了他,摇头道:“不是为兄不支持你,钞引对朝廷财计大有裨益。但私家只要沾手,必为众矢之的……就像是你曾经跟我说过的将一体折钱缴纳的那件事,都是不可能现在答应的。”

“有哥哥你的这一句,小弟回去就好交差了。毕竟不是小弟一人说的算。”冯从义笑着说道,他也只是说一说罢了,“不过税赋归一,对国家、对百姓都有好处的啊。并非小弟纠缠不清,去苛捐杂税,将税赋归并为一,朝廷和百姓都能因此得利。”

“税收是国家命脉,影响到万里幅员、亿万子民,不可不慎。可还记得免役法?”韩冈摇头叹息,“将税赋归并为一,对朝廷的好处不言而喻。可对百姓呢。”

韩冈记得张居正推行过类似的改革。具体细节他记不清楚了,不过他的记忆中也的确有一条鞭法的名称。从名字上来猜测,应该跟冯从义的提议相差不远。面对同样的困境,能够选择的手段总不会差太多。

这一税制改革,并不是独创,突然间从某个人的脑子里跳出来。而是与之前历次税制改革一脉相承。都是简化变得复杂和混乱起来的税制,同时让朝廷能够从中得到好处。不过是有识之士,所见略同罢了。

但这么做的问题也有,而且问题很大。

韩冈做过一定的了解:“唐德宗时宰相杨炎从租庸调改两税法,户税、丁税都改钱征收。为了交税,农民就要贱卖绢帛、谷物或其他产品以交纳税钱。”

“那是市面上钱币不足的缘故,只要能够有足够的钱币流通于世,必然不会有这样的结果。”冯从义立刻反驳,可见其下了不少功夫。

“但丰年时又会如何?丰年谷必贱,如果只交粮食,贱一点也就不卖而已,可要折钱缴纳又如何?届时就算多收了三五斗,在百姓而言也不是好事了。”

“可折变呢?粮折钱,钱折物,折到最后,要缴纳的赋税就翻了几倍。”

折变是如今的恶政。就是官府将所征实物以等价改征他物,‘因一时所需,则变而取之,使其直轻重相当,谓之折变’。可实际上,对百姓来说却是‘纳租税数至或倍其本数’,翻番了!其同样是逼着百姓将手上的粮食换成钱,甚至其他官府需要的实物。比起一条鞭法更恶劣。

但韩冈不以为然:“禁折变可就容易多了。不能因为长了疮就把好肉都割掉。”

冯从义的‘一条鞭法’,不是现在可以推行的,更不应当由自己来提出。

韩冈即便有一天能够主持朝政,他的执政方针也将是开源,而不是节流,更不是改变分配方式。

“这终究是大忌。暂时不要想为好。为兄现在还不打算成为众矢之的。”

冯从义皱着眉,他虽然仅是商人,可年轻人的胸中终究还是有着一颗不甘平淡的心。

韩冈正想再说几句,下人突然来报,说是章枢密来访。

冯从义一叹,不再争辩,起身先行入内。刚离开,章惇就到了——章惇与韩冈交情非常,有通家之好,他到韩冈这边,都是直接进门引至书房。

章惇来得虽快,却只看到了冯从义的背影。瞥了眼桌上还没来得及收起的杯盏,饶有兴致的问道:“可是令表弟,关西有名的冯四官人?在说什么呢?当不是家常吧。”

“正说如何富国富民呢。”韩冈半开玩笑半认真。

“富国富民?”章惇先是愣了一下,然后就笑了起来,“何须如此,只要能够让四民各安其业,内不困于病馁,外不害于贼寇……”

“然后老有所终、壮有所用、幼有所长,鳏、寡、孤、独、废疾者各有所养,男有分,女有归。”韩冈笑着接下去。

“行了,不要背了。”章惇摇摇头,“待天下大同曰,或有斯时。如今,只是空言。”

“不去做当然是空言,但去做呢?终归能更进一点。”

章惇不以为然,“怎么做?说说倒容易。”

“夫子所论,不过仁、礼、中三个字。拿来教化百姓,使得人人可以读书明理。”

“不是仁和礼吗?”

‘克己复礼为仁。一曰克己复礼,天下归仁焉’。孔子笔削春秋,字寓褒贬,其目的也不过以此为手段,对诸侯的行为进行点评,由此传达他的观点。孔子一辈子所想的,就是天下归仁,通过克己复礼来达到目的。这是在论语中就阐明的关节。

而韩冈则加了个‘中’,中庸之道的中。

当年初入京,韩冈就在程颢面前大放厥词。那个时候,他对儒学的理解的确是太粗浅,失之浅薄。现在虽仍旧比不上程颢、程颐和王安石这样的大儒,可好歹都读通了五经,以及十倍于此的传注,可以用儒学来包装来自后世的学问,在于大儒们的交往中,可以不再落于下风。

不过若是为了教化万民,精深了反而无用。书、易难解,诗、礼难明,孔圣之道并不是那么容易学得通透的。

韩冈并不妄自菲薄,论才智最差也在中上以上,又在书籍和交流对象上有着他人难及的优势。他都用了十年方才能做到糊弄人,普通人要是想把儒学学通,穷十年之功的结果也不过是小成而已。这个时代,书籍就是一个大问题,而出色的老师更是凤毛麟角,要不然张载、程颢、程颐也不能聚起那么多学生来。

而这些事在韩冈看来,是纯粹的浪费时间。有那份时间和精力,好歹也能精通一门实用的学问。比如水利,比如财计,比如军事,比如刑名,都是经世济用的学问。

韩冈的想法就是将儒学简单的归纳为‘仁为本,礼为用,中为行’。做人做事以仁为本心,而礼则是规则,法律、道德甚至三纲五常之类的都可以当框子装进去。而中,就是做人做事要秉持中庸之道。知道这些就够了。

教授于人道理越简单越好。剩下的时间,就可以去学那些经世济用的实学。这样也能吸引更多的人来学,对于推广教育有着极大的好处。

章惇很早就知道了韩冈的想法,但他一直不以为然,现在也不过是重复过去的对话。

“浮屠有大乘、小乘之别。小乘者,重自度。大乘者,自度之外,还要度无量众生。自度已难得,何况度人?更何论度亿万众生?”章惇叹道,“修身齐家治国平天下。玉昆,你的目标是在天下之上啊!”

“既然路在脚下,又明知走下去能达到,那为什么不走呢?不过是难一点而已。”

“不仅仅是难一点吧。”

