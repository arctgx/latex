\section{第六章 见说崇山放四凶(三)}

“正如韩冈所言,此事当交由开封府根究。”

吕嘉问毫不犹豫的附和了韩冈的意见。

他看得出来,韩冈是有几分进退失据,否则何须这般着力强调?

听韩冈的口气,自是坚持要将蔡京论之以重罪,来个一了百了,可好不容易有这一个机会,能让他如愿以偿吗?

代为通报的内侍,当庭禀说了开封府的奏报。从沈括奉旨出宫,指挥开封府下部众,配合李信、王厚搜检城中,安抚黎庶;到章辟光和王厚捉回了蔡京、蔡渭,都一一作了说明。尤其是蔡京、蔡渭两人,来自开封府的奏报中,很清楚的提到了蔡京准备械送蔡渭入官这一件事。

没有权知开封府的沈括首肯,来自开封府的奏报中绝不会有这一条。

韩冈多次相助沈括,非韩冈之力,在新旧二党中皆受人厌憎的沈括,如何还能回到朝堂上?

当年沈括见王安石罢官归乡,便打算转投吴充,谁知吴充厌恶其为人,拒而不纳,还如实奏禀,让天子为之震怒。原本沈括就要贬去南方,是韩冈一力相助,让他得以去京西立功。

而就在前些天,韩冈就在着崇政殿中,先是为其求取三司使,与吕嘉问交恶,后又推举其任翰林学士,让其重归两制行列。

韩冈对沈括可谓是恩同再造,可唯一的问题,就是沈括根本就不是会感恩的人。

果然是赤胆忠心、坚贞如一的沈存中……

李定玩味着在韩冈脸上一闪而过的表情,不知韩冈现在是不是在后悔。

沈括此人人品本就堪忧,此时的表现更是明证。若沈括当真有心帮助韩冈,他就不会连蔡京绑了蔡渭这一条都禀报上来。

直接将蔡京下狱,在朝廷派人下来之前,将他弄死弄残,不让他有说话的机会,这对韩冈是最为有利的选择。

以韩冈在开封府吏员中的声威和人望,又有沈括主持,府判章辟光看模样也是要投效韩冈,上有人遮掩,下有人施行,弄死区区一个蔡京,根本不是难事。事后报称畏罪自裁,或是病死,怎么查?

但沈括没有这么做。他选择了对自己本人最为有利的做法。

看起来每个人都清楚呢,在这场叛乱中,韩冈得到了什么,又失去了什么。

“臣意亦如此。蔡京是否曾械送蔡渭,可着开封府审问明白,并由御史台择人监审。”

李定配合着吕嘉问,却又加了一条,打算以防万一。

毕竟在开封府中,就算没有了沈括,也还有章辟光在。想不到这个投机的小人,又把宝压在了韩冈的身上。

尽管禀报上来的只是冠冕堂皇的消息,不可能会将沈括和章辟光各自的私心披露,可在列的大臣们都在官场中不知打了多少滚,从中看透两人真正的想法,以及做了什么,做冇不到的才是例外。

“蔡京小人,其自诉岂可采信?”张璪说道。

吕嘉问立刻回应,“所以要审问明白。”

“蔡京为人奸狡,事前与蔡确共谋,事败便立刻反噬,依其过往品性,当是能做得出来。”李定转头看了眼韩冈,“殿下可问韩冈,以蔡京为人是否能做出此事。”

韩冈秉笏拱手一礼:“蔡京的为人,臣事涉干连,不宜有所臧否。既然交由开封府审问,其后自能得知真伪。”

声音平静得仿佛没有感觉李定在玩以子之矛攻子之盾的把戏。不论蔡京人品,只要他反戈一击被确定,那他就在赦免的范畴之内。

‘故作镇静也济不得事。’吕嘉问暗自冷笑。

朝堂之上,早已将沈括看成是韩冈的人。但他今天的所作所为,却哪有半分相似。沈括本就是有名的见风使舵。明知此事却还是招纳了此人,活该被扯了后腿。

“为何蔡京不一刀杀了蔡渭?”吕嘉问问道。

“杀人灭口,其罪昭彰。”

杀人灭口。

李定强调时,两只眼睛也在瞥着韩冈。

这是警告,不要指望杀了蔡京便能就此高枕无忧。

韩冈就是要杀蔡京,也不可能亲自动手,只能让手下的人去做。事涉多人,只要想要审问,肯定能查出来。

吕嘉问和李定的作派,让王安石不禁皱眉。才空出几个位置,怎么就跟饿狗抢食一般?

攻击韩冈、反对宰辅,难道就能让太后选择他们继任?为了让太后能够理解,他们做的已经太直白了。

王安石瞥了一眼屏风之后。

李定、吕嘉问,甚至还有沈括和章辟光,突然间围绕起蔡京做文章。

向太后就算比不了一众朝臣们个顶个的精明,但也不可能不了解他们的用意。

李定和吕嘉问,其态度本就十分明显的在针对韩冈。他们或许真的能够成功,但这也不是王安石喜欢看到的。

“蔡京有罪与否,可由开封府审问明白,勿须再多言。”

不过王安石并不想韩冈头上少个笼头。从各个方面来说韩冈都太过危险,尤其是对早间才在近距离看过他一锤击毙蔡确的一众重臣来说,更是如此。

“若开封府析断有不尽人意之处,自有诸法司复核。”

同样的话出自不同人的口,用意就截然不同。

王安石并不完全放心自己的女婿,朝堂上谁都知道这一点,但谁也没料到,他会如此直接的表现出来。

“这么多案子压在沈括一人身上,开封府怎么办?”

屏风后,原本明确的态度忽又变得暧昧起来。

“有判官在,有推官在。”韩冈即时回答。

“……即如诸卿所言,都交由开封府吧。”

无人再反对。对叛逆党羽如何处置的争论,此时暂告一段落,直到开封府那边有了结果。

结果如韩冈所请,却没人认为这是韩冈的胜利。

但韩冈面对众人的双瞳中,是毫不动摇的坚定。

借重沈括是一回事,将希望放在沈括身上却是另一回事。

沈括的问题得之后再说,他就算当真做了墙头草,拧回来也好、拔掉也好,韩冈都能做得到。

而章辟光会倒过来更是意外之喜。看来之后要与章辟光多亲近亲近了。有他在开封府盯着,沈括想要做出些事来,也会受到牵制。

另外还有件事,韩冈双眼一扫身周的同僚,可能是自家的态度让人误会了,使得李定、吕嘉问他们弄错了一件事。

蔡京被一了百了自是最好,也是韩冈所期待的。故而方才也的确有些疏口,让李、吕之辈,以为找到了可供利用的破绽。

可蔡渭还活着,蔡硕也还活着,应该参与到叛乱策划中的刑恕也还活着,他们都还要被押往开封府狱中等候审判。

这样还不够吗?

“一众逆贼从党将发送开封府,那曾布、薛向该如何处置?韩卿,你怎么说?”

太后明显的已经很疲惫了,待前事一了,便重提曾、薛二人之事。有了之前的缓冲,她相信应该能快一点解决争论了。

韩冈站了出来:“在这之前,臣有一事当问?……敢问殿下,赵颢当如何处置?”

“不是赐死吗?送其一丈白绫,吾明天不想看见他!”太后的回复极为决绝,她当真是对亡夫的二弟厌恶透顶。

而韩冈紧接着又问:“其子孝骞呢?冇”

“……毁其玉牒,族谱上除名,找个地方养着吧。韩卿,你看如何?”

“殿下所判,臣无所改易,亦无可改易。正当如此。”韩冈点头,又道,“首恶、从党既然皆已有定论。曾布、薛向如何处置,便可以以此为参照。”

赵颢一死,四名首恶便一个不剩。而罪行更轻的从党,虽然还没有审问,但两边争论到最后,也就是一封赦诏了事。

对他们的处置,就是判决曾布、薛向的界限。

韩冈带着众人生生绕了一个圈子,最后定下了断案的范畴。

“赵颢既被赐死,为了京中安定,还请殿下对曾布、薛向稍作宽待。依律,从犯亦当减主犯一等论处。”

“赵颢是先帝二弟,英宗与太皇次子,否则何能逃脱凌迟极刑?”

“曾布、薛向皆是士大夫,国朝故事,何曾有士大夫以凌迟死?”

“无论如何,首恶已轻纵,曾布、薛向自不宜论之于死。”

“还是韩卿之言有理。”

太后的话,让吕嘉问一时襟口。从口气上,听得出她明显已经对持续不休的争论感到厌烦。

曾布、薛向两人并不足论,万一恶了太后就得不偿失了。不能再拧着太后的心意来。

“但区区远流,不足以为惩戒。”

“臣以为当举族流放交南或西域。三千里或不足,万里便可。”

“交南瘴疠遍地,多蛮夷,少人烟。而西域虽苦寒少水,生活却不甚艰难。”

西域还叫生活不甚艰难,那可就真是一个笑话了。除了几处绿洲,那里的生活,可不是中冇国之人能够想象。更不是以宰辅侍制之尊,在京冇城中过着养尊处优的生活的一干人,能够想象的。

但比起交南的气候和疫病来,却的确要轻上许多。交州的极南之地,比起岭南诸州更为可怕。

“西域缺乏人口。”

“西域的确缺乏人口,但西域诸族交杂,又有敌寇,万一中冇国之密泄露出去,就又是中冇国之害。当以交南为是。”

