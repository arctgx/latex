\section{第34章 为慕升平拟休兵(26)}

夜风渐渐停了,而大小王庄附近的骚动却是始终如一。

如果从高处向下俯望,可以清晰的看见几条光的河流从大小王庄流淌而出。那是辽军撤离的路线。

两万辽师,并不是沿着通向代州的官道一条道路撤离。而是分散开来,有三个方向,一条沿着官道,一条则是偏北的小路,另一条是向南趟过了滹沱河,看起来是准备从滹沱河南岸返回代州城。

以辽军的兵力和战马数量,如果只选择从官道走的话,就是仗着战马的优势,到了第二天中午也照样没办法全数撤离。现在分头离开,倒是快了许多——就像前一天从代州赶来支援的时候,除了最早出发的几波援军,剩下的也都是齐头并进。这样才能在一日之内,让连人带马总数数万的援军以最快速度抵达前线。

撤退十分顺利,但萧十三的心中却充满了失落。

‘失败了。’

一个翻盘的机会,萧十三企盼了半夜,可始终没有等到。

纵然心中还有所不甘,但他还是承认了现实。宋军的主帅韩冈看来宁可放弃一举在野地里击败敌军的诱惑,也要保证全军上下的安全,不肯出来冒上一点风险。

之前一众部族军打着火炬离开,不仅仅是为了照亮前路,更重要的是想要让萧十三和他的本部能够悄然离营,在黑暗中潜伏起来——刻意想让人注意,很多时候就代表有什么想要遮掩的。就像是很多人说谎时,声音就突然变大一样。

萧十三和他的本部兵马悄无声息的离开了营地,就潜伏在荒野中。身侧是只剩河心有水的滹沱河,在等待宋军主力出营追击的同时,倒是要分一半心去提防宋军绕道河对岸杀过来。

此时已经过了三更,寨中守军走了一半,可宋军的主力始终没有从营地中出来。

被派遣出来的宋军骑兵乍离营时声势浩大,差点就让本应依次离营的几个部族争先恐后的抢着逃离,要不是萧十三及时派人压阵,全军就要崩溃了。

不过在他稳定住了军心之后,双方骑兵快要交锋的时候,宋军那边却又草草收场,转头返回,然后在外围游走,一看便知并没有交战的打算。

做了几个月的对手,韩冈除非被逼无奈,否则绝不会冒险用兵的性格,萧十三已经完全了解了。

对面的那个同为枢密使的对手真要耗下去,萧十三也无可奈何,纵然避免正面作战的想法双方是共通的,但韩冈能选择的其他手段远比他萧十三要多上许多。

聪明人很少会放弃自己的优势,而选择冒险。至于韩冈聪明与否,那是不要讨论的一件事。

啃着又冷又硬的干肉,就着藏在怀里还有些温热的烈酒,萧十三下定了决心。

‘该走了。不能等到天亮。’

什么时候进攻才是最好的时机,这是兵家的常识。设身处地的站在宋人方向去思考,萧十三只会选择在黎明前出击。一旦天边有了一丝微光,他在滹沱河边的埋伏就再也瞒不过宋人的眼睛。

或许韩冈可能会更加保守,但萧十三不愿意再冒险了。

他作为主帅,对麾下的各家部族已经仁至义尽,不可能再让他的本部,以及耶律道宁手下的那两千多宫分军去阻截可能就要出来追击的宋军。

野外的远处,那充斥于视线中那满山遍野的火光,数量虽多如天上的繁星,可火光下的军队,或许都没超过两千,还有四五千甚至更多的骑兵等待着出动的时机。

不论从士气还是体力上,萧十三都没有太大的把握在击退宋军的同时,还能保证伤亡不会太多。

万一宋人的骑兵采取的是拖延迟滞的战术,那么当宋军的步军主力赶上来,就是末日来临的时候了。

从马扎上站起身,萧十三挺直了腰,回望了一下几条向着代州方向前进的光流,招来了身边的亲兵,“去跟耶律道宁说,差不多是时候了。”

……………………

乌鲁骑在马上,走在队伍的最前方。

上一刻抬头望着眼前的道路,下一刻就已经回头看着身后。

夜中的行军比起白天当然要难得多,而且人数一多,走起来就是磕磕绊绊。

不过他的心情远比在大小王庄的时候更加轻松。

身边用柴草和树枝捆起的火把噼啪作响,已经快要燃烧到了尽头。不过已经走了近一半,剩下的路程就算没了火光照明,走得慢些也不会有太多的危险。

乌鲁和他的部族由于已经在大小王庄驻守很久,所以运气很好的成为了撤退时排在最前面的几个部族之一。

不过他这一部走的并非是官道,而是位置靠北一点的另一条道路。关闭



之前因为宋军骑兵的出阵,让身后很是乱了一阵。乌鲁当即就带着族中子弟快跑了一阵,连头都不回。但之后看到宋军没有主动进攻的打算,提起的心好歹放下了一半,速度也稳了一些下来。

虽然道路比不上官道,一大半人马都在荒芜了的麦田里跋涉,但速度不算很慢,乌鲁计算了一下时间,赶在天亮之前,差不多就能抵达代州城的城墙脚下了。

回到了代州,下面当然就是设法到雁门关驻守,如果宋军攻来,代州城肯定是保不住。既然迟早要撤离,还不如早走一步。

乌鲁心中盘算着,回到代州后是不是就跟几个亲近的部族走动一下,好先一步被退往雁门关去。

仗他已经打够了,是该好生休息、享受一下收获的时候了。

……………………

随着前方的火龙逐渐接近,韩中信心情也越发的紧张起来。

敌军的人数众多,而且会不断增加,而他手下的兵力为数寥寥,如果被看破虚实,只要半刻钟就会被打得灰飞烟灭。

手心的汗水也越来越多,在盔甲的外袍上擦了半天,但始终都擦不干净。

“怎么,怕了。”

身边来自同伴的轻笑让韩中信不服输起来,“没有!正等着他们呢!”

“接下来可要小心些哦。”秦琬提醒道,“别被自己人砍到。”

“放心。”

韩中信应声说着,却又低头看了看身上的盔甲,并不是很放心的样子。

若是劫营倒也罢了,出战的兵力少,营中也有篝火可以照明。但野地夜战不同,难以列阵,无法指挥,就算一开始有些火光,到了一接战,没有人还会把火炬拿在手上。

暗夜中的厮杀,聪明人都在第一时间丢掉了火炬,而剩下也都会在混乱中熄灭。

伸手不见五指,可以说完全没有什么技术含量,单纯的在混乱中厮杀,打到最后,甚至会变成不分敌我,见到活动的东西就下手,谁也不敢慢上一点。

韩冈自不会让他的人在夜色下去攻击行进中的辽军,那样的胜利得不偿失。只是秦琬和韩中信等了许久的机会,却又怎么甘心放下?

管他什么危险,乱了辽贼队形就好!

他与秦琬自从领军进驻土墱寨后,几乎就成了被遗忘的角落。

宋辽双方的主力在官道一线上纠缠,他这边贴近雁门山,就变得十分清闲,辽军的探马出现的数量都不算多。

他们虽不能算是一步闲棋,驻扎在代州通往神武县的四条通道最为靠近州城的土墱寨,更多的原因还是为了保护神武县的侧翼,可一开始也还是有一部分是为了分散辽军的注意力才会被派遣出去。

当辽军开始被五台山方向上的西军,以及大小王庄方向上的对垒吸引了注意力,秦琬和韩中信这一部偏师就被抛了脑后。

直到前几日接到了来自置制使司的命令。

土墱寨距离雁门有七十里,当轨道建成,忻口寨主力前移,基本就处于后方了。不过为了保证战线的,在轨道开通的同时,这一支代州兵的驻扎地就开始向前推进。

秦琬和韩中信留了一个指挥驻守土墱寨,剩下的人马都带了出来,继续沿着山脚向代州方向进发。由于兵力不多,且倚靠山麓,这让他们并不畏惧辽军的进剿——大不了上山躲一阵——一路潜伏,然后在发现辽军出援的道路后,便潜到了近处。

他们的人数并不多,而辽军的注意力并不在这里。

本以为要等上几日,谁知仅仅一天就等到了他们的目标。

当看见最前面的敌军已至眼前,秦琬身边的亲兵吹响了手中号角。

……………………

一声号角从前方猝然响起,惊起了最前面的一批战马。

“是谁在乱吹号?!”

一扯缰绳,稳住了战马,乌鲁怒声高喝。但他随即就想起来,这是回程,他根本就没向前派出探马。

是宋人?!还是代州出来迎接的自家人?

他心下惊疑不定,正要派人去前面查探,远远近近的呼应便是一声接着一声此起彼伏。

被埋伏了!

多年的征战经验让乌鲁立刻反应过来,可还没等他发布命令。正对面的道路上,最早的号角传来的地方,已经隐隐约约能看到一排黑影打横拦住。

乌鲁瞪大了眼睛,想要看清楚。

下一刻,眼前的原野,点点火光逐一浮现。

背后的星火,将那一支军队的身影从黑暗中勾勒出来。

看着阵势似乎很是单薄。

但他们就在拦在了道路正中!

周围一片号角声,这数量可能会少吗?

而且最重要的……这一支敌人是出现在代州的方向上。

代州城怎么了?

