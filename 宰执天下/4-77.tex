\section{第15章 焰上云霄思逐寇(八)}

[不好意思,迟了一点]

尽管有许多交趾兵在撤退时,没忘记带走身边正在惨呼痛叫的同伴,但更多的死伤者则被人遗忘在归仁铺的大营边。

等到交趾兵稍稍退远,营寨大门敞开,两百多战士受命从寨中出来,来到被抛弃的伤兵身边,用刀枪给他们最后一击。

另外还有些交趾兵没有了来得及逃跑,就俯身躲在壕沟中,当宋军出寨来清理战场,无处藏身的他们只好出来主动投降。

一百多名交趾兵跪下来苦苦哀求,在营外的士兵们,一时不知道该不该下手。

“全都斩了。这次第,哪有多余的粮食养贼?!”李信知道韩冈的脾性,也不等待他的命令,直接让下面的士兵放手杀降。

李信的命令传出,刀枪就毫不犹豫的重新举了起来,寨中的守军也将弓弩对准了这批降兵。

“等一等。”韩冈派了亲兵过来,“运使说了,如果愿意斩了脚上的两根大脚趾,就放他们回去。”

没了大脚趾,能走路但负重和跑步都不行了,就根本不能再上阵。他们回到对面,对交趾军毫无用处,反而是浪费粮食和药物,而且还会降低士气。不像杀俘,能让人升起同仇敌忾的心情。

“李常杰见之必怒。”苏子元对韩冈道。

“就是要羞辱李常杰。”

自己这里越是表现得毫无顾忌,交趾人那边反而会更加犹疑。浪费一下交趾人的粮食也是好的,就看李常杰到底会怎么做了。

“倒是营栅毁损不小。得赶快修好。”

就在韩冈和苏子元视线所及的范围内,有好几处栅栏被砍出了缺口。虽然不大,但也暴露了这座营寨最大的缺点。

交趾军前面攻击时,将这一圈脆弱的栅栏作为第一目标的确是一个正确的选择。为了能给交趾军造成最大的伤亡,宋军一直等他们开始毁坏寨墙才开始射击。没有足够的木料,要修补还是很麻烦。如果再来上几次,这座寨子的栅栏可就保不住了。

“不过今天贼军应该不会再来了,”苏子元抬头看了看天,阴云密布,风也变得冷了起来,“很快就要下雨了。”

……………………

李常杰知道这是个硬骨头,也做好了暂时攻不下来的心理准备,但还是没想到这一次的攻击,受到的挫折如此之大,还没来得及给宋人造成多少损伤,就被当头打了回来,

自己的队伍不过刚刚撤退,宋人就直接敞开了寨门。这个行动,明显的就是在嘲笑自己。

骑兵还在逗留在战场上。如果他们冲得快的话,能赶在打扫战场的宋军回到营中之前,冲进宋军大营去。这么想着,李常杰摇了摇头,根本没有用处,交趾骑兵的水平如何,他很清楚。

抬头看了一下天色,看起来很快就要下雨了。加上方才的失败,今天的进攻也只能到此为止。

“太尉。”李常杰麾下的一名将领站出来提着建议,“今天天色晦暗,正好可以夜袭。”

在看不到星月的夜晚偷袭敌营,听起来倒是不错。如果宋人不防备的话,的确有几分可能成功。

但宋人可能不防备吗?

李常杰很快就拒绝了在夜中的冒险。要是攻打宋军大营时,有人从背后掩杀过来,直接就会崩溃。而且说不定还会下雨,下雨天进攻,对面的弓弩的确会威力大减,可凭着眼下军中士气,想在在雨水中走夜路都是问题,何摸黑攻打营寨。还不如继续派少数兵力去骚扰,让寨中的守军无法休息。

但李常杰现在的心情总是像堵上一块石头一样,在心中沉甸甸的。宋军到底有多少人,没有弄清这个问题,他始终难以释怀。

如果能确定宋人有更多的援军,李常杰肯定不会再选择进攻,只要等宗亶他们全数渡过左江,自己就可以向南方撤离。但要是只有八百人,被宋人唬住的屈辱,李常杰怎么都不能咽下。

方才围攻宋人营垒的时候,他已经派了一队骑兵绕过归仁铺,往昆仑关方向过去查探。如果宋军在昆仑关中有人,至少会出来驱逐。如果没有这么做,就证明眼前的敌军,就是来援的全部兵力。

“太尉!”一名部将匆匆进帐来禀报,“宋人把俘虏都放了回来。”

“怎么这么大方?”李常杰狐疑的问着。

“不是大方……”

“那是什么?”

“他们的脚趾都被宋人砍了!”

李常杰霍然而起,双眼圆瞪:“什么!?宋人竟敢如此辣手!”

一百多名交趾士兵,身子摇摇晃晃的,从归仁铺大营一路回来,走上几步就会摔上一跤,最后是互相搀扶着,走完了所有的路程。

北上犯境,又在钦州、廉州、邕州城中大肆杀戮。这样的敌人,宋人不但饶了他们的性命,砍下了脚趾后,不忘包扎止血。可这看似宽容的行为,却处处透着残忍狠辣,已经是一辈子的废物了,除了一条命以外,什么都不剩下。这比直接砍头还要狠毒。不但浪费军粮,帐下的士卒看到他们现在的模样,哪里还有什么战意。

李常杰看得目眦欲裂,脸色铁青,宋人下手太狠毒了。而跪伏在自己的面前,痛哭流涕的这群被释放的俘虏,也让李常杰感到愤怒,他们若是拼将一死,又何至于此?

“太尉,他们该怎么处置。”李常杰被人问着。

闭起眼睛考虑了一下,旋又睁开,“送到后方去,让宗太尉好生照料。”

李常杰作出了一个宽仁大量的决定,但这并不代表他当真放宽了心。捏得紧紧的拳头泄露了他心中的愤怒。他现在就在等着一个消息,如果当真能确认,不管是谁下了这个命令,他势必要其碎尸万段。

……………………

傍晚的时候,雨开始下了。

淅淅沥沥的并不大,但广西的雨季已经到来。

依照苏子元的的,疾疫很快就要多了起来,韩冈很担心军中的医疗卫生问题,但这不是眼下的急务。

“还要进一步逼迫李常杰。”有人看到了一队交趾骑兵往昆仑关的方向去了,这个消息一级级传达到韩冈的耳中,“让他来攻打寨子,而不是动其他的心思。”

守卫昆仑关到归仁铺的道路,这件事韩冈交给了黄金满来处理。且就在长山驿左近,韩冈还让黄金满派了五百兵,用于封锁垭口通道,骑兵想过去不是那么容易。反倒是步兵可以从小道上直接绕整条防线。只是在风雨中情况就不一样了,在弓弩发挥不了作用的时间里,要冲过一点阻碍,对于骑兵来说,并不要耗费太多的气力。

“要不要主动进攻?昆阳之战,汉光武以三千破伪新.四十二万。而合肥城下,张辽正是以八百军大破十万吴军。”这几日的接连胜利,让苏子元对官军的战斗力有了很高的评价。“如今我军也有八百精锐,可以一战。”

“那是两回事!”韩冈和李信同时摇头。

在昆阳城下,光武率精锐攻王邑所率领的新朝大军。这一战说是三千对四十二万,但袭营的时候,光武只需要面对伪新中军的那万余人,剩下的四十万根本来不及赶来救援,甚至不知道该不该出营。加上王邑因为身边有着四十万大军环绕,自以为可以高枕无忧,受到攻击时应对失措,这就注定了他们的失败。当王邑的本阵被刘秀击破,四十万大军直接就垮了。

而逍遥津,主要是孙权做了蠢事。若说当头一棒,韩冈此前也不是没打算给李常杰一下,只是没能守到机会。何况并不到冒险一击的时候,就算李常杰挥军来攻,他也能将寨子稳稳守住。

“对面的李常杰吃了好几次亏,又担心我们还有援军,小心提防还来不及,不会犯蠢,他不会给我们这么好的机会。”

韩冈也没有打算在这里守太久,他已经派了得力人手潜入邕州城,联系城中的百姓以及残留的守军。再有两天的时间,便能动员起他们中的大部分逃出城,藏进山中去。接下来,他就可以找个时机,撤回昆仑关。

不过韩冈的心中,总隐隐有些忧虑,知道他所率援军只有八百的人实在太多了,李常杰听到一次两次,自己的表现可以让他半信半疑,但次数多了,怎么可能再怀疑。到时候,他可就要面对交趾军的全力进攻。

就算是下雨,双方夜中对对方的骚扰依然没有停歇的意思。但两边对此都有所防备,所谓的骚扰也不过是普通的扰人清梦罢了。到了第二天,依然是下着雨,交趾军的攻击只是应付差事一般,连昨日一半的魄力都没有表现出来。

但到了傍晚的时候,一名斥候从雨雾中冲进营地,脸上满是惶急,“启禀运使,交趾军又来了增援,兵力至少有一万,很快就能抵达对面的大营。”

李信、苏子元和黄金满,脸色都变得极为难看,李常杰肯定是从哪里确认了己方的虚实,所以才敢将更多的兵力从后方调来,而不担心刘纪等人反叛。

韩冈沉吟了片刻,抬头笑问道:“你们说,现在交趾兵更恨谁?”

