\section{第36章 可能与世作津梁(四)}

【祝各位龙年快乐】

“韩冈已经启程去襄阳了?”

王廓点了点头,算是对他父亲问题的回答。

京西路转运使的行动并不是秘密,尤其是前些天,洛阳的府漕之争受到了京城众多官员的关注,那件事的结果关系到眼下的新旧党争,甚至有可能影响到未来的政局发展。不过韩冈退让之后,却把文彦博推进了坑里。越来越多的人不值文彦博的所做所为,转而支持韩冈。

“这一招以退为进做得好。”王韶一拍交椅扶手,忍不住赞道:“韩玉昆的手段果然厉害。”

王廓望着自己的父亲,心中有着无奈,低头再次提醒,“大人,行礼已经收好了。”

王韶狠狠的瞥了长子一眼,“急什么?天子派人来催了不成?”

“…………”王廓沉默的了下去。

王韶去职,王家南下,天子的确没有派人来催。但王韶算是因罪而离,眼下当然只有越快离开京城越好,这样才能向天子表明自己知错、诚心诚意接受处罚,愿意就此改过的态度。

王韶瞥了眼欲言又止的长子,甚是不满的冷哼了一声。但他随即又苦笑的摇起头来:‘实在太大意了。’

王韶想不到为了一个参知政事的位置,章惇竟然联络了蔡确——也许还有吕惠卿。蔡确也的确不负他刚刚闯下的名,一封弹章,便让王韶不得不引罪请辞。

别的罪名还好说,以天子现在打算稳定朝局的想法,西府应该在短时间内不会变动。可换成是引用乡里私人的罪名,天子却难以忍受这样的枢密副使。如果将自己再留下去,天子恐怕也会担心自己会将更多的国家公器,当成是私人授受的工具。

没有在第一时间对弹劾加以驳斥,也没有在第一时间以退为进的请辞,耽搁了时间,自己眼下黯然出京的结局,便已经注定。

“便宜了章惇!”站在后院的两层小楼楼上,王韶冷眼望着不远处的章府,心中愤恨不已,‘便宜此辈小人!’

……………………

“蔡确那厮决不饶他!”

挡在前头的拦路石,被御史中丞像树上的鸟儿一般,一记准确有效地射击,便给击落了下来,可章惇却没有任何欢喜之情。

章惇的二弟章恺当然知道他的兄长为何而愤怒,蔡确的这一封弹章根本不像是外界所说,是得自章惇的授意,而是他的独断独行。

眼下政事堂中只有一相一参,章惇当然也想能转任东府。但他并没有想过通过陷害王韶而将机会抢到手,甚至都没想过这一次能有机会转任政事堂中。

吕惠卿在政事堂里做着参知政事,自己想与他做同僚,第一个出来反对的很可能就这一位吕吉甫。而且新党之中,也需要一个合适的人选来控制枢密院,章惇一时间根本离不开西府。

可蔡确的行动,就让他成为了众矢之的,让王韶变得怨恨自己,要不然以他跟韩冈的交情,与王韶维持着良好以上的关系,是章惇的不二选择。

“会不会是吕吉甫授意?”章恺问道。他刚刚送了父亲章俞上京,哪里能想到转眼就在他的眼皮底下发先这一幕。作为章惇之弟,章恺还是比较清楚如今朝堂上的局面,还有他的兄长所站立的位置。

“不至于。虽说蔡确往往受人之命,但吕惠卿还不至于用这等手法来陷害为兄。”章惇摇摇头,沉声道:“他没空!”

吕惠卿如今正紧锣密鼓的筹备手实法,打算仿效当年的王安石,通过推行新的政策,从而乘势扩大自己的权力范围。这样的情况下,吕惠卿可不会节外生枝,暗地里来黑章惇一手。章惇能进政事堂的几率太小,而用这等策略,也只是王韶倒霉,章惇不过是坏了些名声罢了。而眼下要是与章惇再闹翻了,吕惠卿还有几个人能作为他的助手?他现在的目标可是王珪。

面对吕惠卿这些日子来的咄咄逼人,作为东府之长的王珪则什么也没做,每天上朝都是对天子的吩咐唯唯诺诺,不断的重复着‘取圣旨’、‘领圣旨’、‘已得圣旨’这三句话,完全没有自己的想法。不过当了一个月的宰相,就已经落下了一个‘三旨相公’的雅号。

在政事堂中只剩这样的一个宰相的时候,不论换作是谁来做参知政事,都会忍不住设法取得更大的权力,吕惠卿自然也是无暇分心于他事。

不过吕惠卿准备使用的手段却让章惇觉得并不合适。只是眼下的政局,让章惇无法将自己的想法说出来,他还没做好在新党中另立山头的准备。加上蔡确的背后一刀,使得章惇眼下只能保持着沉默,远离政事堂中的一池浑水。

章惇头疼得要命,眼下的局势越来越让人看不懂了。推开窗户,初夏的夜风便涌了进来。章惇从崇仁坊中望向皇城的位置,夜色下的皇城城墙,映衬着墙头上的一排暗弱的灯光,显得份外幽暗迷茫。

做了十一年的皇帝,赵顼的心思越发得幽深起来,越来越让人看不透了。就像是夜色下的皇城,明明是看得见,但仔细瞧过去,细节之处却是一片模糊,

天子到底想要做什么?

没人能明白。

……………………

洛阳现在风平浪静。

韩冈和文彦博之前的一番纷争,在韩冈的退让之下,似乎是平息了下来。加上韩冈的南下退避,看起来京西路是没有问题了,更不会影响到赵顼始终记挂在心上的襄汉漕渠。

只是赵顼依然觉得文彦博心胸狭隘是整件事的祸源,但也有为韩冈的心机而感到惊讶。赵顼也不是蠢人,童贯回来之后,尽管赵顼只听了他复述的一干细节,但赵顼却能看得出来,韩冈在受辱之后的举动,是这位年轻的都转运使不见血却依然狠辣的报复。

‘真不简单啊。’

赵顼想着。文彦博三朝元老,一点错处就给韩冈抓到。而韩冈的反击,完全是借势压人,让文彦博空有权势和人脉,却拿他无能为力。

且除了声名之外,眼下文彦博也已经为儿子文及甫事涉干请一事,上书请罪。不论文及甫最后怎么判,文彦博也只有请辞一条路可走。而韩冈,则是带着洛阳城中博来的好名声,施施然南下襄阳。看他的样子,似乎就是想在襄阳将京西路漕司的治所给定下,而在他远离了洛阳之后,文彦博乃至他的继承者,更是得配合韩冈的工作,否则就是名声上的大问题。

“韩冈心术难测,还是放在京城之外十几年,好好看清楚他这个人。”方才就是曹氏提起了有关韩冈的话题,眼下则是更进一步的提醒赵顼,“不要太早让他入京,更不能让他过早晋身两府。”

韩冈并非纯臣。对于这一点,赵顼并不感到讶异。

能在千万人中成为为数寥寥的进士,能在几万名官员组成的官场中熬出头的大臣,当然不会有只知忠心事上、不知阴谋诡计的愚直臣子。而韩冈更是朝臣中的佼佼者,哪有输人的道理。

“娘娘放心,儿臣明白。”赵顼恭声说着。心里却在想着,韩冈若是在襄汉漕渠上立了功,日后再安排他去河北,将黄河大堤给休整一番。无论如何,这一次,赵顼都不会讲韩冈太早纳入京中。

韩冈一直以来,都是才干智术而著称于世,他当然不简单。如果放在朝堂之中,恐怕不论是什么位置,都能交出一份让人不得不惊叹的答卷。只要与朝中的其他臣子做一下对比,就能逼得赵顼不得不提拔于他。甚至短时间内,就连通向两府的道路都能为他而打开。以韩冈的年纪和才干来说,一旦身登两府,日后权倾朝野,甚至能远胜韩琦。

不过将他放在京城之外,让他不断地在各路各州间调动,既不会浪费他的才能,也能压得下他的声望。让他攒个一二十年名望再入京城,日后如同其岳父一般为国出力,流传到后世,说起来也算是一桩美谈。

曹氏抬头,已经昏花的双眼,看着成熟起来的赵顼。要不是因为这名孙儿的居中转圜,当年真的想与那个不孝子拼个鱼死网破。不过当年些许纷争早已成为陈年旧事,曹氏也都抛到了脑后,现在她所关心的,只有大宋的江山。

“也是老身多说了,看来官家已是早有定见。”

赵顼腰弯了下来:“儿臣年轻识浅,还有许多地方需要娘娘的提点。”

曹氏笑了一笑,眼中却没有多少笑意。她的这位孙子,坐在大庆殿中的御榻上时,已经越来越老练成熟,不复当年的青稚。就像他过去坚持新法一样,如今那他对朝政的处置依然是全凭本心。眼下也就是跟自己一个想法才会如此畅快的点头受教,换作是其他事,不是阳奉阴违,就是婉言拒绝,甚至是忿而争辩。

在王安石之后,再也没有能左右天子的朝臣,并不是一件坏事,曹氏也算是能放下一点心,在宫中好好的休息一番了。

