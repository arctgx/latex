\section{第五章 圣贤需承传人荐(下)}

皇子有恙,病势沉重,在朝中也掀起了不小的波澜,有人忙着找药,有人忙着求医,甚至还有求幸进的,献上了刺了舌血写的金刚经来保平安。而宫中也是延医问药,求神拜佛。至于其中情真与否,各自心里都有数,绝大多数只是面上功夫。

至于吕惠卿,他根本就不去担心皇三子赵俊的健康问题,甚至是生死问题也不关心。

天子不过二十五六,身子骨虽然弱了些,但在后宫中还能施展得开,儿女也是一个接着一个的生,只是养不大而已。还不到需要关心的时候,过了三十后如果还没有子嗣,再急也不迟。

仁宗嘉佑时御史中丞张昪,为人清介,不与同僚结交,仁宗曾戏言其‘孤寒’。而张昪则直接反驳说仁宗才是孤寒,说:‘臣家有妻孥,外有亲戚,陛下惟昭阳【注1】二人而已,岂非孤寒’,据称仁宗立储的心思就这么定下了。但那个时候,仁宗都四十岁了,身体也多病,肯定是生不出子嗣。而说如今的天子‘孤寒’,未免嫌早了一点。。

现在吕惠卿倒可以算是孤家寡人了。新党这边都是盼着主心骨入京的架势,真正与自己马首是瞻的也就那么两三个。邓绾那棵墙头草,在拜相诏书出来前,已经贴着自己,现在又往回倒了。

原本吕惠卿在王安石离去后,一举升任参知政事,正是意气风发,要一展长才。但上则受制天子,下则人心难定,左右又被政事堂中的同僚钳制,雄心壮志无处施为。本来还有一分解脱的机会,谁能想到韩冈竟然突施冷箭。

王安石即将回京的消息一确定,身边刚刚聚集起来的猢狲全都散了去。都说是树倒猢狲散,可他吕吉甫还没倒呢!到了这个时候,吕惠卿才发现,王安石等的三十年不是白等的,三十年积攒下来声望已经转化为根基深厚的撑天之木,而自己只是缠在树上的藤蔓而已。

可笑自不量啊!

吕升卿这两天看着兄长心情不好,跟着在身边说些话来转移吕惠卿心头的烦躁,对于始作俑者的韩冈并没有好话,“可笑那韩玉昆,先是设法将王介甫请回京来,现在又张罗起让张载入京的事,难道不知道这两位虽然地位差距极大,但在儒门中都算是一脉宗师,大道根源则是南辕北辙,冰炭不能同炉。”

吕惠卿眼皮子动了动,其实他是不服气的,张载跟他分属同年,怎么张载就是宗师?他吕惠卿也同样在经义上成就非凡,不过是被王安石的光芒所掩盖了。

“而且韩冈直接举荐张载判国子监,这根本是狮子大开口,根本不可能成事。国子监祭酒、司业谁都不能指望,依照故事,国子监长贰之位极少授人。就算再大的名望,也只能做判监。”吕升卿说得兴起,“不过判监也不是这么好做的,当年的名儒胡瑗,被范仲淹举荐到国子监中之后,只是担任国子监直讲的一职。”

“韩冈为人多智,吕大防是个沉稳如山的人,王珪则是滑不留手,他们三个哪里会犯这等错?明明白白的是要明修栈道、暗度陈仓。”刚刚在韩冈手上吃了大亏,前面还有板甲、飞船之事,韩冈的心术手段,吕惠卿早就领教过了,倒也不会认为是他糊涂。

吕升卿闻言发了一阵楞,然后叹道:“……那以天子的心性,张载还真是入京定了。”接着有勉强笑起来,“想来王介甫入京后,听说自己女婿的作为,脸色必然很好看。”

吕惠卿没有笑,他怎么可能为这等事开心。

要不要直接阻止张载上京?

刚升起这个念头,吕惠卿就摇摇头,他这时候还表那个忠心做什么!?由着他们翁婿两个斗好了。自己若是越俎代庖,当真整下了韩冈,说不定还让王安石看不过眼,疏不间亲啊!但他吕惠卿也决不是任人欺辱的,反正王安石今年五十五,而他吕惠卿才四十四,迟早能等到王安石保不住韩冈的那一天。

等到入夜之后,吕和卿也回家来了。

吕和卿新近转任开封府推官,正巧摊到了陪同监斩的差事。今天就是在街市上,给赵世居、李逢谋反案收尾。凌迟三人,腰斩三人,开封府外的市口上很久没有那么热闹。

吕和卿也不是没见过世面,但一天之中,连着看了三场钝刀片肉的戏码,接着又是三轮生切活人的场面,回来后连吃饭的胃口都没有了。

他在吕惠卿和吕升卿面前连连摇头,脸上尽是不忍:“都是些无妄之灾,不过是素行不谨,结交错了人,哪个当真会有反叛的心思?一个个看着那真是叫惨啊,一直都在喊冤。”

“走错了路,看错了人,怨不得别人的。”吕惠卿颜色一沉。

吕和卿还是在叹气:“朱唐授了内殿崇班,赏钱五百贯。首告一人,得赐即如此之丰,恐日后年年可见人谋反了。”

同在书房中的吕升卿,则是听出来吕惠卿不是为了赵世居案在感慨,“大哥说得是谁?”

吕惠卿满腹心事,却也不想就此多说。他虽然一向城府甚深,喜怒难形于色,但这一次实在跌得太重,心理落差太大,有些失衡。勉强克制着心中的烦躁,转头问着吕和卿:“蹇周辅今天也一同监斩吧?”

“蹇周辅穿着新赐的紫章服就坐在我旁边,他也才一个推官啊!”吕和卿说到将原本定下来的诬告案子翻成如今谋反大案的同僚,更是愤愤不平。“害了多少人,竟然换了一身三品服色!”他吕和卿现在还穿着绿袍。

吕惠卿冷笑一声:“朱紫又不是多贵重的,熬着资历就行了。二十年历任无过便能赐紫,去年给太皇太后治病有功的翰林医官,记得也是早早的就赐了紫。你说他敢在为兄面前坐下来吗?”

“天道循环、报应不爽,蹇周辅迟早没好结果。”吕和卿难以释怀的诅咒了一句。想想,又凑近了,神神秘秘的低声问道:“永国公最近重病,该不会就是此案有冤的缘故吧?”

“别乱说话!给我藏在肚子里。”吕惠卿突然厉声喝道:“你亲眼看着李逢他们的下场,还不知道要谨言慎行吗?!”

长兄如父,吕惠卿一怒,吕和卿连忙站起来请罪,半句也不敢为自己辩驳。

等到再说些闲话,吕升卿和吕和卿一同就一同告辞离开兄长的书房。

“今天到底是怎么了?”出门后,吕和卿觉得自己今天被冲得有些冤,

“大哥最近的心情不好你也不是不知道。”吕升卿瞪了一眼,“不过今天听说王珪、吕大防和韩冈三人同荐张载,多半也有这方面的事。”

吕和卿惊问道:“韩冈什么时候跟王珪搭上关系了?”

“谁知道……”

……………………

正在揣测着韩冈和王珪之间关系的人,如今绝不在少数。

但韩冈很清楚,王珪只是做了一个顺水人情罢了,也是想看着翁婿对阵的好戏而已。张载值得推荐,所以王珪就推荐了一下——与王安石过不去,也符合他对自己的定位——哪里有那么多深层的含义。

不过韩冈也是得感谢王珪,他直接上书推荐张载没什么大不了,关键是请动了王珪王禹玉这位老牌的执政,这一点让许多潜藏的反对者为之束手。王珪再怎么样也是参知政事,反对他的推荐,阻止在士林中名望极高的张载入京,想博取名望的御史们也不会做这等蠢事。

韩冈正等着宫里传来最后确认的消息,毕竟王安石的消息现在已经到了。

“官人。”王旖脚步匆匆的迎了出来,就在院子里急问着刚刚走进家门的韩冈,“听说有爹爹的消息了?”

韩冈点了点头,他今天在监中就听到消息,派人回来通知过了。宰相等重臣快入京的时候,沿途的驿站都会派快马向京中通报,以便让人迎接,“岳父岳母昨日已经过了应天府,还有三天的路程,就能抵达京城了。”

“爹爹娘娘还有大哥大嫂他们身子可还好?”王旖又追问道。

“这怎么为夫可能知道?来通报的又没有说。”韩冈摊了摊手,好笑着,“不过岳父岳母是没问题的,不然传来的消息中肯定会提。而且从行程上看,中途并没有耽搁时日,元泽多半也没有问题。”

听了韩冈的分析,王旖提了一个多月的心终于放了下来。陪着韩冈往里屋走,偷眼看着脸色,小心的问道:“那官人的荐书批下来了没有?”

韩冈摇了摇头,“还没有,不过也应该就在这两天了。”

王旖欲言又止,其实她有些担心,韩冈与她父亲不合的地方就是在学术上。万一王安石先至京师,阻止了张载上京,到时候又要起纷争了。

究竟哪边能先一步?

王旖担忧着,随着韩冈走进后院的步子渐渐变得沉重下来。

注1:汉代有昭阳殿,赵飞燕姐妹曾居住。这里指代曹皇后。

