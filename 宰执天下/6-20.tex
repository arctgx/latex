\section{第四章 力可回天安禁钟(中)}

“来得不慢!”

尽管对面的人数是这边的班直人数两倍,郭逵没有半分惧色,反倒是犹有余裕的赞了一句。

若说这是军营,凭这反应速度,石得一绝对是数得着的出色将领。就是性命交关的政变,如果不是韩冈下手太快,又让宰相们立誓争取到了班直在身边,石得一在听到消息赶来,也还有翻盘的机会。

可是现在大庆殿内外都已经在宰辅们的控制之下,就是太皇太后祖孙三人,也被控制住了。石得一就算还有太后与天子在手,但他一样再难挽回。

大庆殿上,还有一群南班官——这些可都是宗室,宣祖的血裔。太后、天子若有不测,直接从里面挑一名出来做皇帝。比如那位三大王,或是他家的几个儿子都是上佳的人选。就像是汉初灭诸吕后,陈平、周勃不立有大功的齐王刘襄,却立了代王刘恒,也就是文皇帝,这便是大臣们齐心合力之后,所能拥有的力量。

除非石得一能学韩冈将领头的重臣都解决掉。可当时只有蔡确、曾布和薛向三人,韩冈针对蔡确一人、镇住曾布、薛向就够了。但现在,是王安石、韩绛、章敦这些更有份量的宰辅,郭逵、张守约也都站在平叛的一方。

大庆殿前广场面积巨大无比,排成队列,放下十万人亦是等闲。当年狄青在广西平叛,立功回朝。仁宗皇帝就曾经让有功将士在广场上重现击败贼军的战斗。而校阅立功的将士,每一次大战得胜之后,都会在大庆殿前的广场上举行。可皇城司的叛军穿过广场,只用了很短的时间,尤其是最前面的几十人,更是一路狂奔而来,将后面的同伴甩到了几十步外。

不过当他们看清楚是郭逵和张守约站在人群中,速度不由得慢了下来。

在连韩冈都以为他们产生畏惧的时候,这一队皇城司叛军中突然飞出了两支长箭。

也不知是不是没有事先分配,两支箭并非郭逵和张守约一人一支,都是奔着张守约过去了。

韩冈正在人群后,救援不及,只能大声喊:“太尉小心!”

张守约则一直都在防备着,他经验又是十足,一见箭起,便向旁边倒。但毕竟年纪已老,反应去比常人要慢了一步。冲面门而去的长箭看着躲过了,可冲胸口去的却怎么也让不开了。

旁边郭逵的反应则比张守约快得多,不暇细想,就用力一推,他身边的班直便被他推到了张守约的前面,硬是用身子将那支箭给挡了下来。

当当的两声响,两只长箭撞在了张守约一前一后的两名班直身上,只在厚重的铁甲上留下了一个印记。

张守约用脚来感谢那名救了他一命的班直。不耐烦的一脚踹开前面的人,老将直面叛军:“叛贼皆已伏诛,各位相公……和太后有令,除石得一外,余人皆无罪。杀石得一者可封节度使!”

郭逵随即大喝:“太后有旨,今次只诛首恶,胁从不问!蔡确已死,尔等还要负隅顽抗到何时?!”

皇城司那边的汹汹气势顿时就散了,奔行脚步也都迟疑起来。

原本是同伴的班直都成了敌人,太皇太后和新皇帝在内的大庆殿都被敌人控制住,若太后也被救出来了,那就当真败了。

“太后在哪里?!”石得一在人群中喊着,“现在是太皇太后在听政。保扶太皇,人人做官!”

“都是断头买卖,不想死的就上啊!”皇城司叛军中,又一人也跟着煽动人心。

张守约认识那人:“王忠!你回头给石得一一刀,就能跟老夫平起平坐了!是节度使!是太尉!”

“张太尉,你让太后出来啊!太后说什么我们就听什么!”石得一哈哈大笑,回头再一喝,“还等什么?错过这个村,就没这个店了!”

“你看着!太皇太后马上就押出来了!”郭逵喝令左右,“给我上,杀一贼,便入流,多杀一贼,便加官一级。谁能杀了石得一,谁就是节度使!”

皇城司叛军攻了上来,班直禁卫则在张守约的指挥下,于大庆殿前的台阶上排下了阵势,居高临下的反杀回去。

“怎么殿里面的都没这般聪明?”

韩冈遥遥望着石得一。已有心理准备是一回事,但应对也的确比里面的二大王强得多。

“那是他们没看到玉昆你捶杀了蔡相公。”

韩冈闻声回头,瞧了一眼从殿中匆匆出来的章敦,问道:“子厚。你怎么出来了?”

“太皇太后和齐王已就擒,你们还想为蔡确陪葬不成?!”章敦先是冲着下面大声吼了两句,然后才冷笑着对韩冈道,“殿内的哪个见过血?这群班直,打起来鹅都不如!鹅见了血还能叫唤两声。”

章敦瞥了眼韩冈还拿在右手中的铁骨朵——韩冈在穿衣抻袖时都只是将骨朵换个手,都没放下来——走到了韩冈的左手边。章敦早就见惯了死人,但看见蔡确的死状也不禁心中发凉,更休提从来都没见过血的班直了。

在台阶顶端一站,章敦就盯着旁边的两名御龙骨朵子直禁卫,“你们还站在这里做什么?放着眼前功劳不要,御龙骨朵子直的人都是这么没出息?!”

只吼了一声,两名禁卫便慌忙一前一后冲下了台阶,加入了张守约的麾下。

韩冈微微苦笑,他还指望他们给自己挡箭呢,章敦却将人给赶了下去。若是有人瞄准上面射击,可就要靠反应速度了。不过这话可不方便在现在说出来,更不能将人给叫回来。

章敦说得不错,方才他在殿上能压制住班直,宰相伏尸当场的场面至少占了三分。领着滴着脑浆的骨朵,韩冈本人毫不在意,解剖的尸体他看了不知多少,可从来没见过这等血腥场面的班直,回去后连隔夜饭也能吐出来的人不会没有。大部分也的确都给吓住了,当然更加畏惧韩冈。

这就是拿着天下最多俸禄的班直,日常收入比起战功卓著的西军还要多出许多,更不用说那些苦哈哈的厢兵了。可真要打起来,厢兵中不缺亡命之徒,西军更是好勇斗狠,自幼见惯了血雨腥风,根本不怕什么皇城司亲从官,而班直禁卫,能不手忙脚乱就不错了。

当年太宗皇帝就做错了。他攻晋阳城不下,班直请战,他却舍不得让他们上战场。班直禁卫无不是勇武之士,不当值时便操练武艺、阵法。只要见了血,必然是天下有数的强军。如今却圈禁在皇城里,狼都给养成狗。

肯定要改一改了。韩冈脑中转着和现在完全不相干的念头。

“怎么样了?!”章敦仔细观察着下面的战局,一边顺口问着韩冈。

方才皇城司的叛军前后脱节,这边没有趁机迎上去是个失误,不过形势依然有利。大庆殿这边只要守住上殿的台阶就行了,而石得一则必须攻破防线将太皇太后给救出来。目标的难易度天然就有巨大的差距。

“都一样没见过血,就看指挥和训练了。”韩冈说道。

班直中的成员,各个身强体健、孔武有力。没经历过战阵是块短板,但对面的情况也一样。至少力气上不会吃亏。

在张守约的指挥下,几十名班直在台阶上已经分出了前后次序。整支队伍进退有序,在看到敌军攻上来时,还能好整以暇的做出合适的应对。

“杀了石得一,就是节度使!”

班直们虽是仪卫,但正好是仪卫,才会人人穿上重甲,皇城司根本就没甲胄。而且皇城司亲从官从来没有过战阵的训练,班直却是日日操演,比三两日才得一次操练的京营都强得太多。

在张守约和郭逵指挥下,这些在章敦口中连鹅都不如的班直,表现都还算得上不错。将头盔的面罩一放,拎着兵器就上去了。面对韩冈这个级别的宰辅,他们心中没底,但换作是日日都能见到,从来都看不起的皇城司亲从官,那就另一回事了。

石得一大呼小叫,但他麾下的皇城司亲从官们却渐渐不支。装备不如人,指挥不如人,训练不如人,地势也不如人,本来就是必败之局。

而且班直心中都有些底,背后是满朝文武助阵,纵然没有太后、皇帝,重立一人也简单。加上有郭逵、张守约指挥,韩冈、章敦压阵,这个阵容实在是太过奢侈,就是辽国当年入寇,耶律乙辛都没享受到。

至于站在亲从官们背后的是谁?区区一阉竖!

韩冈眼睛突然眯了起来,连话也没说,直接振臂一挥,将手中的铁骨朵用力的投了出去。

还沾着宰相鲜血和脑浆的铁骨朵脱手而出,飞向下面的皇城司叛军中。

“玉昆,怎么了!?”章敦茫然不解,急声问。

“去!”

韩冈盯着铁骨朵的落点,见没有击中目标,不满的啧了一声。手脚却很麻利的将章敦向后拉。

铁骨朵只击中了一名叛军的胸口,砸得他向后一个倒仰。不过他倒下去的时候,背后的一人暴露了出来。那人双手拿着一件兵器,却是让人极为熟悉的。

“神臂弓!”几声尖叫同时响起。

