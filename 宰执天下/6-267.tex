\section{第36章 骎骎载骤探寒温(三)}

“夫子虽是说过有教无类,不过此辈……当是无可救药了。”

宗泽在润州南门的城头上,与景诚并肩而立,望着城外宛如星海的火光。

夜色已深,但润州城中无人入眠。城外星火如海,城内风声鹤唳。

润州内外对明教教众紧锣密鼓的搜捕,捕获了为数众多的信众,还有十数个传教的妖人。可是这番大动干戈,也让诸多信徒因恐惧而被煽动了起来。

两天前,丹徒县一甲长走报州中,说是他庄上有大户卫康正密谋造反。

这大户卫康,知名于县中,时常救人于困顿,有仗义疏财的美誉。平日里全家吃素,据称还善符箓,能用符水治人。

从他日常行迹来看,可算是半公开的明教信徒,而且是渠首一类的人物。

这几日,州中到处搜捕明教教众,卫康家中就多了许多生面孔出入。隐隐有只言片语传出,却尽是些大逆不道的言辞。甲长情知不妙,便连夜赶来州城中首告。

景诚听到消息,先是去找知州杨绘,杨绘托病不出。没奈何,他与宗泽和州中其他官员商议之后,便先命丹徒县尉带了一百多土兵去将卫康锁拿。

也不知是消息走漏,还是卫康事先派了人侦查,这一支人马在半路上受到了伏击,丹徒县尉当场战死,百多人死伤大半,只有寥寥数人逃回城中,连领路的那位甲长都被砍了脑袋。

从逃奔而回的残兵败将口中得知,明教这一回竟然拿出了十几副铁甲,由教中蓄养的一批护法穿上,冲在最前面。

这些护法,就是明教的打手。可以用来保护教产,也可以用来惩罚那些背叛者,更重要的是防止其他地方的渠首捞过界。

就是这批护法,之前从路旁一冲而上,将丹徒县尉率领的一干土兵、弓手打得哭爹喊娘,转眼就崩溃了。

卫康一战而胜,接下来的一日一夜,贼众席卷丹徒各乡,到了此时,一片片火把围定了润州城。

景诚心情沉重,代掌州务不过半月,就出了这么大的乱子,他这个通判责无旁贷。

幸好润州城还能守。城中的驻军虽多为厢军并不堪用,而且空额甚多,但景诚拣选城中青壮,轻易便拉起了两三千人守在城头上。润州城中刚刚搜检过,也不用担心这些人里面有多少明教教众。不用忧惧里应外合,即使贼军攻城,一时半会儿也打不进来。

“除了北面无贼之外,其他三面都有贼众。”他闷闷的说着。

“但贼众的人数不多。”宗泽眼睛贴在冰冷的玻璃片上,“那些火是虚张声势。”

透过望远镜,能看见城外的贼众人手两支火炬,还有许多就把火炬插在地上。乍看上去,就是人山人海。

“也多亏了有这些火光映着,否则还真分辨不出贼人的多寡来。”宗泽冷笑着。

景诚皱着眉:“南面的贼人不多,东面西面的贼众也都不多,那他们会在哪里?”

“杨知州还不肯出来理事?”宗泽忽然问道。

景诚摇摇头,懒得说那位知州。拿着引罪避位的名义,将州中公事全都丢到一边,现在都火烧房了,还躲在州衙后面的佛堂中,也不知是在念经还是在看笑话。

宗泽也不屑的哼了一声。杨绘那么大把年纪,却还是不知轻重,以私怨误公事。等此番事了,秋后算帐少不了他一个。不过这样也好,以杨绘的水平,他出来只会添乱。

不提杨绘,宗泽对景诚道,“卫康作乱州中两日,裹挟百姓不在少数,眼下三面皆是虚张声势,人数不多,想来他在北面或许设了伏兵。,”

“伏兵!”景诚惊道:“他想伏击京口的援军?”

“也有可能是想要攻打京口。”

京口那边从午后开始就断了消息,而城中派出去的信使也不知道到了没有。宗泽往坏处想,也不是无的放矢。

“京口,卫康这厮能有这番见识?”

景诚难以置信。区区一介乡民,不过是学了点惑乱百姓的妖法,还能把兵法都贯通了。

“卫康若没见识,那让进士出身的曹景明如何自处?”宗泽语带嘲讽。

想起那位兴冲冲的出门,却丢掉了性命的曹县尉,景诚也不知该说什么才好。前途无量的新科进士,却被信了邪教的乡农砍死,死得未免太不值了。

“京口不是卫康能攻下来的。”景诚平复了心情,对宗泽道,“而且京口也不会来援。若是卫康打了这个主意,那他可就要失望了。”

润州城的北面便是长江,长江之滨乃有京口。

作为天下有数的要道渡口,京口港城的防备远比润州城还要森严。驻泊在润州的禁军,驻扎地就在京口,而非是润州城中。只要京口不失,江对岸的援军随时可以南下。而且京口的驻军,也随时可以出动,攻击围困润州的贼人。

而且尽管长江上的渡口为数众多,可适合大军渡江,且道路适合运兵的渡口,也就那么几个。反贼们若是夺占了京口,官军就只能绕道南下,这么一耽搁,至少能给卫康争取到十天的时间。

十天之中,反乱的明教教众能将两浙路的局势彻底败坏,就卫康本人而言,十天时间,也足以让他跑到两浙东部,夺船入海也不是不可能。

从兵法上说,拿下京口远比润州更有战略意义。如果卫康不想才痛快几天就撞上南下的禁军,拿下京口才是他要做的。

只不过,屯有重兵的京口不是乌合之众能够拿下,而京口的禁军,也并不是景诚可以调动的。

江宁府、扬州、杭州、太平州、润州,都有禁军驻扎。真的想要剿灭城外的这些贼人,从京口调来禁军便可以轻易解决。

可如果严格的依照法度,各路各州的驻泊禁军,即使是路中监司都无权擅自调动。即使贼人火烧润州,兼任杭州知州的两浙西路安抚使,他也不便擅专。

不过儒家有经权之说,打着事急从权的名义,不说帅司,知州调动本州的驻军也是可以的,只是事后要承受后果。但再怎么从权,也得是知州下令,而不能是出自通判的命令。

杨绘不说话,景诚能搜检城中青壮助守城池已经是极限,想要调动润州辖下的驻泊禁军,那就是梦呓了。

润州不会去求援,京口也不会出兵援助,两边虽都只是坐守,但只要润州和京口都守住,三五日后,贼人就得要走了。

宗泽摇头,“现在卫康阻隔了润州和京口的消息。他完全可以遣人扮作京口援军来诈开城门。”

“怎么可能会上当?”景诚在内外交困的情况下,也忍不住失声而笑,“若是京口官军当真来援,也不会选在夜间,青天白日下,贼人再怎么装扮也扮不像官军。”

“也有可能遣人伪作传信,若是卫康把京口的驻军骗出来,又当如何?”

“贼人当真能有如此狡诈?”

“料敌从宽。”宗泽道。

卫康既然能伏击抓捕他的队伍,可见他有十分灵通的消息来源,以及胆大包天的决断,这样地头蛇一般的大户,高看几眼并不为过。而且,轻易拿下县中派去的队伍,想必也给了他更多的信心,以及更高的威信。

景诚道:“但卫康当真狡诈的话,就该去攻取其他县城,而不是来攻润州。”

宗泽踩了踩脚下的城墙,“论城防,润州城比得了哪家县城?”

北方的城池,即使是县城,都修得又高又厚。有的村庄的围墙都能有两丈高。

但江南的州县,很多都没有城垣,即使有,也都是低矮单薄,而且很多城墙都是多年未有重修,崩塌损坏的地方不知有多少,润州城也不例外。

除了规模之外,润州州城与县城的城墙墙体的规制都都差不多,修成后再也没有整修过。好几处都塌了。垮塌的地方,城墙顶端仅能立足,连拉弓都没空间。要是全塌了还好,那就得立刻维修,偏偏都是只垮了一半,既然从城墙顶上能走过去,也就得过且过了。

让宗泽来看,有个一千人马,调度好的话,拿下润州城当真不是难事。

景诚嘴唇动了动,似乎是难以苟同,想说话,却又忍住了。

宗泽看出景诚的心思,道:“不过也不必担心。即使是落到最坏的境地,旬日内亦可平定。”

“旬日?”景诚沉下了脸,这意味着宗泽或者说朝廷对润州的变乱早有准备。

景诚先前已经隐隐猜到了,搜捕明教突生变乱之后,宗泽并不怎么担心,可见他那边早有后手。由此而来的的安心感,远不及被隐瞒的愤怒。

“在泗州,有龙卫一个指挥,神机两个指挥,随时可以南下。”宗泽言辞平静,并没有为之前的隐瞒而愧疚。

果不其然!景诚收拾心情,问道,“是跟着汝霖你一起南下的?”

“是因为演习到了泗州,也没想到当真会派上用场。”

“那汝霖你已经派人回泗州了?”

景诚没多问宗泽怎么有权调兵。政事堂想给宗泽调兵之权,总是有办法的。只要调动的是京营,而不是地方上的驻泊禁军,帅司管不到,州衙也管不到。

“昨日便派出去了。”宗泽坦然道。

景诚正要说话,眼角的余光中却见几点星火正向城门这边扑来。
