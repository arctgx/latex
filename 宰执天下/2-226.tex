\section{第34章 山云迢递若有闻(七)}

千骑奔驰,风云随之鼓动。烈声撼地,让观者心旌动摇。

王中正一脸的紧张,脸色一点点的白了下去。他即便是在危险的罗兀大撤军中,也是被护翼在千军万马之间,不论是设伏,还是反击,宋军都占据着主动权。只是眼下,薄弱的兵力却要面对近三倍的敌军的攻击。虽然历经战事,可从未有见识过逆境的御药院都知,只一眼,便被凶猛如群狼的吐蕃蕃骑,吓得魂飞天外。

“都知不必忧心……贼人杀不进来。”

韩冈的语气同蕴含的自信又多了几分,王中正狐疑的看着他那份自信心过了头的微笑,心思却当真是安定了几分。

韩冈心如山岳之稳。在他看来,领军的蕃将心思过于急躁,犯了最大的错误。骑兵朝着营寨冲锋,这比直接冲击已经排下阵形的宋军箭阵还要鲁莽。

虽然不像有着城墙的堡垒那么保险,其实一道木栅要用来抵御骑兵,已经绰绰有余。而堡中的民伕和守军都已经上来了,一个个手持重弩,身上的披甲都是韩冈临时分派下去的。渭源堡是关键的转运点,堡中弓弩刀剑等军器堆积如山,皆是战时备用的。尤其是神臂弓,虽说威力强大,但过强的力道也容易损坏弓臂,故而备用品数量最多。

堡中的民伕和士兵,的确被突如其来的敌军惊到,可在在一众前广锐将校的奋战之下,军心随之振奋。前面在听到吐蕃骑兵来袭的时候,韩冈就已经发号施令,将这些军国重器不但补发给士卒,而且还分发给民伕们。他们都是曾经的广锐军成员,配合起他们旧日的官长,却是完美无缺,顺畅无比。

蕃骑如潮水一般涌来。刘源等骑上马的三十几名将校,并不蠢到直膺其锋,却也不回营中,而是远远的偏向侧翼。如毒蛇一般,在外围狠狠咬上一口,用娴熟的弓马技巧射落了七八名贼人。引得敌阵中分出了两百多蕃骑来追击他们。

刘源等人先顺着营栅而逃,蕃骑紧追不舍,不意却将侧翼暴露在守于这一段栅栏后的射手眼中,一片弓弦过后,便是二十多骑落地。

见着被宋人阴了一招,追兵更为愤怒,死死咬着不放。刘源等人见状,一拨马首,离开营垒转向西面逃去。一追一逃,转眼就绕得远去。

刘源引走了一部分敌军,等于是帮了堡中守军一个大忙,韩冈在城头上看着满意点头,接下来又将视线投回到敌阵中。

蕃人的旗号他认不太明白,可超过两千的骑兵,又有三分之一带甲,那么领军的不是木征的亲信大将,便是瞎吴叱,或是木征的另一个弟弟结吴延征。至于禹臧花麻,韩冈不认为他会为木征兄弟冒这么大的风险。

抹邦山绕过来的道路,脱离了大宋现有军力的护翼范围,所以缘边安抚使便把控制这条道路的计划放在日后。在兵力不足的情况下,保住一条通路都是勉强,缘边安抚司在两条道路之间,便选择了比较崎岖、但路程短了近一半、走鸟鼠山的北线,而不是走抹邦山的南线。

现在渭源堡外的这一部蕃军,他们走上渭源通往临洮的南线,挥兵偷袭渭源,这意味着他们的退路,随时可能被已经占据临洮的宋军给堵上。禹臧花麻发了疯才会为木征兄弟火中取栗,能劫掠一下粮道已经是尽了人情了。

在临洮城随时可能拈选精锐堵截后路的情况下,这次来袭的吐蕃人算是自作聪明,如果有机会,韩冈有心将其解决在渭源堡下。现在他反而担心蕃贼们向东去骚扰渭河谷地中的屯田诸堡。堡中精壮都给抽走,老弱妇孺可抵挡不了蕃贼的攻击。

“围着渭源堡反而不用担心了。”韩冈对王中正说着,“就怕他们分出兵力向东杀过去。”

“现在真的没事?”

“我们还有霹雳砲!”

韩冈手上的兵力虽是稀少,可他所在的这座营垒的防御构筑,是以面对万人的侵袭而准备的,各色装具一一备齐。而存放在营垒中,亟待转运的粮草和兵器,各种守城、攻城器具也是一应俱全,重型的有八牛弩,近处的神臂弓。当然,不论是安置在营垒中,还是准备运到临洮前线,都少不了最近声名鹊起的霹雳砲。

为了攻打临洮城,缘边安抚司事先做得准备无所不至。攻城用的器械,也都是事先准备好的,连工匠们都征调了三十多人。但他们却没有派上用场。如果攻打临洮时,王韶顿兵城下,攻而不克,这些工匠便将会带着霹雳砲和八牛弩的核心构件,前往王韶的军中听候指挥。可临洮城出乎意料的脆弱,让他们没了上阵表现的机会。

也因如此,现在他们却正好就在韩冈这边。工匠人数不多,仅有三十余人。韩冈用不着他们的工匠技术,却用得到他们的双手。亲手打造的霹雳砲,工匠们使用起来,自然不会逊色于从士兵们中挑选的砲手。现如今,堡中缺乏人力,韩冈便调来他们这群工匠,让他们来操作霹雳砲,而把原本的几十名砲手解放出来,穿上盔甲,端起神臂弓,到前面去作战。

两座高约两丈的重型霹雳砲,宛如一对拥有修长手臂的巨人,矗立在营寨大门处不远的两座台地之上——三座营门左近,都架设了两架霹雳砲,以作护卫——霹雳砲依然笨重,可比起旧时的行砲车,现在的霹雳砲需要的人员,还是少了许多。

工匠们分作两队,有条不紊的进行着准备。在来袭蕃人抵达之前的小半个时辰之间,韩冈已经让人把此战需要使用的砲弹全都搬运了出来,安放在霹雳砲旁。

不仅仅是滚圆的石弹,还有泥弹、碎石弹,以及近似于化学武器的毒烟球。如果是城墙的话,重达三五十斤的巨石砲弹的确管用,但遇上了奋勇的敌群,碎石、泥块反而比巨石更加有效。

工匠们对霹雳砲的操作十分的熟练,搬运砲弹、计算距离、调整配重。转眼之间,两声呼哨若有若无的滑进耳中,安放好砲弹的投石车就挥舞起长臂。呼的声响,两点黑影飞向空中。

划着完美的抛物线,两枚泥弹从天而落,在营栅外人群中猛然溅开。

吐蕃蕃骑蜂拥在营寨之外,外围的射击着宋军的神臂弓手,而内侧的蕃人,则正设法砍开栅栏,冲进营中。混战中,无人有暇抬头向天空看上一眼。直到砰砰的两声闷响,无数硬邦邦的泥石碎片劈头盖脸的砸来,骑手痛叫,战马嘶鸣,他们才惊觉宋人还有更胜神臂弓一筹的神兵利器。

今次所使用的泥弹有二十五斤上下,确切的重量使得砲组在计算落点时,只要调整配重就能大略的接近目标。泥弹的威力并不算大,两发砲弹,只有一个不幸站在落点位置上的蕃骑,被连人带马砸得筋骨节节而断。可四溅的泥片,也在敌群中造成了一片混乱。而从空中飞来的重物,在精神上更是给了蕃人不小的震撼。在冲营之前,还要抬头看一看,这给他们一往无前的决心,压上了几分踌躇。

泥弹仅是试射,紧随其后的第二发便是换成了碎石弹。外面用绳索编织起来的罗网,罗网之中里面则塞满了碎石,鼓鼓囊囊的,就是一个球状的包裹。

新型的投石车,发射速度快的惊人,接近于单兵使用的神臂弓,比起八牛弩要快,比起旧时的行砲车更是快了许多。两枚泥弹在敌群中砸出来的两片空地,还没有给重新整队冲锋上来的蕃骑所掩盖,下一轮炮击便已经到来。

砲车之下,两声呼哨一前一后。修长的七稍弓臂像是弹起的柳枝,倏尔一扬,两具霹雳砲同时发射。碎石网兜飞舞在空中。砸向了营栅前的敌军。比前一次的落点略近,却仍是准确的落在了拥挤的蕃骑之中。

被一条条绳索紧紧绑扎起来的包裹,在落地时猛然炸开。碎石飞溅,不同于之前的泥片,杀伤力强了十倍有余。坚硬的石子,比起干硬后的泥片更为致命,砸得头破血流,而战马也同样被砸伤了许多。连蹦带跳,将背上的骑手都抛下来许多。

碎石弹的作用不仅仅是杀伤,同时也打乱了吐蕃人攻击的节奏。以落点为圆心,大约五六丈的范围中,一片乱象。而更远处,被干扰到的骑手和战马,也都在一时间失去了攻击的能力。守军射手们的欢呼声随之腾起,乘机上弦射击,将已经混乱不堪的敌军,射得更为混乱。

兵败如山倒,阵脚一乱,想在敌前整顿起来,除非精锐方能为之。吐蕃人并没有这个能耐,变得像没头苍蝇一样盲动着。

“差不多了。”韩冈突然出声,没等王中正追问,像是在呼应一般,撤退的号角此时从敌军后方响起。

蕃骑如潮水般退去,领军的蕃将也终于放弃了一举破城的奢望,将己方骑兵回收,似是要整顿后再行出击。

“蕃贼会不会就这么退了?”王中正满怀希望的问着。

“吐蕃人并不愚蠢,在营外撤退时,还不忘把尸体拖走,可见士气仍在。不过他们咬着渭源不放……这是好事!”

