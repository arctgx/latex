\section{第30章 肘腋萧墙暮色凉(15)}

【上班时候写作果然不方便,断断续续的。今后只能一章下午发,一章在晚上写,夜里发。】

三百骑的铁鹞子逐渐逼近罗兀城,速度并不快,就像是信马由缰的碎步,但暗蕴在其间的张力却是越来越紧绷。犹如捕食中的猛兽,在最后一击前,都是徐步而行,一点点的捱到爆发的那一刻。

三百骑于缓慢的行进中,不断调整着各自的步调,渐渐的统合在一起。近于齐步走的骑兵队伍,一步步的逼近城池,带给城头上守军的压力,丝毫不逊于千骑纵横奔驰给人的震撼,而这种姿态,更是显得他们信心十足。

这等在临战前的气定神闲,让韩冈也不禁惊叹,能做前锋的果然都是精锐。比起他当初在渭源那里见识过的吐蕃骑兵,又是另一番气象了。

当敌军越来越近,他们的旗帜也越发的明晰起来。而配属于这些骑兵,无论是坚实精良的甲胄,还是高壮雄峻的战马,都是让罗兀城里的宋人骑兵都要相形见绌。

“不是铁鹞子!”身旁的种朴,突然带着惊讶的低喝道:“不是铁鹞子,是环卫铁骑!是护翼夏主的环卫铁骑!”

“……难怪!”

听说到眼前的不是铁鹞子,而是护翼夏主的环卫铁骑,韩冈先是一惊,随后也为之释然,要是数万铁鹞子都是眼前敌军的这般水准,莫说罗兀,关中都能打下来了。

韩冈对西夏军制也稍有了解,铁鹞子和步跋子,相当于大宋的马步禁军。而与大宋一样,在兴庆府,也有护卫天子的班直,总计六班,不离夏主左右。另外,在西夏国主身边,又有一支精锐的铁骑环卫,当夏主出行时护翼在侧。总计三千骑,分作十部。这一部,便正是三百人!

“今次若不是夏国母梁氏领兵亲征,就是梁乙埋来了。除了他们以外,环卫铁骑不会出动。”种朴继续说着。

‘不管是谁来,都应该派人出战了。’韩冈心里想着,不论是铁鹞子还是环卫铁骑,再让这群党项骑兵继续耀武扬威下去,城中的军心士气就要打着滚往下跌。有过几次战阵经历,韩冈对冷兵器时代的战斗也算有所了解,也明白士气和军心是胜利的关键。

也正如韩冈所想,主帅高永能也不会眼睁睁的看着敌军的骑兵像斗鸡一样在战前炫耀着自己勇武。

立于城头上,猎猎作响的高字将旗下的主帅高永能,已经连番号令,几个传令兵拿着令旗四散而去。片刻之后,战鼓在敌楼上响了起来。城门支呀呀的打开,近八百人骑兵从城门中奔驰而出,激烈的蹄声与缓步行进的环卫铁骑形成了鲜明的对比。自出城后,转瞬就在城外排了开来。

两个指挥的骑兵当先出战,看起来高永能是打算用更多的兵力尽快解决敌人的这支精锐部队。身处在城头上,韩冈还可以看到南门处,也开了半边城门,第三支骑兵指挥,悄无生息的离开了城中。

‘这是要包抄啊……’

“守城最忌闷守,倚城而战才是正道。让敌军杀到城下,城内的军心都要出乱子了。”种朴大概以为韩冈对战之事并不了解,向他悉心的解说着。

第一次交锋并不会决定战局,但足以影响城中的士气。高永能一口气派出了三个指挥的骑兵,不仅仅是为了把西夏人驱逐出罗兀城的周边,而分明是希望通过歼灭、至少是痛击这一支精锐铁骑,从而树立起城中守军的信心。

即将于党项军中最为精锐的一支骑兵部队交手,出城骑兵奔烈如雷的蹄声中听不到半丝犹豫。面对两倍于己的宋人骑兵,这一队环卫铁骑也是充满自信,见到城中守军出战,便立刻纵马前冲,从徐缓的山间溪流一转而变成了高坝泄水,并不退让半分。

两支骑兵飞快的接近,奔驰的战马在身后卷起了漫天烟尘,遮蔽了外界的视线。一往无前的气势,就像是龙门处的黄河激流。不过双方并不是缠战,两军在城头上的战鼓声中交错而过,便各自远离。只在交汇处,两边都倒下了十几骑。

一次近乎平手的对冲,让两军战意熊熊燃起。同时掉转头来。掌旗官高高举起的旗帜猛然向前斜倾,双方都又再次冲杀上前。

厮杀中的呐喊声传遍战场,城头上的战鼓激荡。种朴看得热血沸腾。看到己方的骑兵落马,他就握紧拳头。而见到铁鹞子栽下马去,他又连连叫好。还不时的望向高永能,脸上一派跃跃欲试的神色,大概是想向高永能申请出战。

几次交锋过后,双方犹不见疲态。虽然个人战力不及对手,但出战的八百骑兵,仗着人数的优势依然维持着均势。而环卫铁骑也没有退让的意思,重新调整队列,准备再一次的冲锋。

就在这时,一队宋人骑兵突然从西北侧冲进战场,赫然是方才悄悄从南门离城的那一个指挥的骑兵。他们从罗兀城南面的小道绕到了环卫铁骑的侧后方,意图进行夹击。

突如其来的骑兵,彻底扭转了战场上的僵局。四比一的比例,加之又是包抄,让环卫铁骑顿时失去了战意。见到宋人的又一支骑兵出现在侧翼,已经看不到取胜机会的环卫铁骑,终于开始退却。

南北两侧的宋军骑兵自然不会任由他们如此离开,歼灭对手的机会不会放过。但环卫铁骑的撤退行动出奇的娴熟,轻轻松松就从纠缠上来的敌人手中脱离。而他们在临走时,还不忘带走倒下的同袍。除了十来具尸体由于坠马位置的关系,而被宋军的抢先一步夺占过来,其他尸体都一起被架上了马,一起带走了。

一次算得上是激烈的交锋,斩首就只有十五具。这平手时的斩首功果然不容易,要想拿个大的功劳,就必须是围歼或是伏击的情况。但出战的骑兵终究还是逼退了西贼先锋,让他们不复来时的气焰,城上城下便是响起一片欢呼声。

种朴也尽是喜色,环卫铁骑是党项人手中最精锐的骑兵部队,除了西夏国主手上的三千人之外,也只有各大豪族中还个有两三百与之相当的私兵。前面八百骑兵与之交手而不落下风,虽然是在人数上占有优势,但也足以让大宋孱弱的骑兵部队因此而感到自豪——要知道,大宋让契丹骑兵也要绕道的步兵,到现在还没有出战。

带着斩获的敌军尸骸,出战的骑兵胜利回返,高永能极大方的撒下了让所有骑兵欢呼的丰厚赏赐,又下令将十几具敌军尸体,剥光了倒掉在城头上。

看到如同风鸡风鸭一样被挂在城墙上的西贼尸骸,种朴低声对着韩冈说着:“士气已振,接下来面对西贼主力,就不用担心了。”

就在环卫铁骑离开的大约一个时辰后,大地开始震颤,北方远处的尘头大起。滚滚的尘烟如同潮水一般向罗兀城方向扑来。从这个阵势上看,绝不再是三百人的小打小闹。而是少说也有三五千人的数目。

西夏人的前军主力终于到了。

而赶在他们抵达前,城门再次打开,三千步卒披甲持戈,腰携弓弩,自城中鱼贯而出,汇入了城墙前的空地,转眼已是阵列俨然如山,其凛凛之威,与之前的骑兵出战不可同日而语。

序幕方才已结束,正篇即将开场,但韩冈没有再继续看下去的意思,而是调转身,准备下城。

“玉昆?!”种朴瞪大眼睛,惊讶的叫道。

“不看了。”韩冈回身摇头,“我还有正事要做。在城头上看着,却是什么用处都没有。”

两人现在都可算是闲人,高永能基本是就不会让种谔的儿子上阵前,而韩冈他的任务则是救死扶伤。自从开战后,高永能就没有向韩冈和种朴他们这边瞟一眼,看样子就知道,无意让他们插足指挥之事。

但已经看到了高永能和罗兀城中的军队,敢于出战与敌正面厮杀的胆气,韩冈暂时也没有什么好担心的了,这对他来说也是足够了。

……………………

白天在罗兀城下的初次交锋,给了轻松击破宋军前沿小寨后,正得意洋洋的党项人当头一棒。一开始准备给宋人下马威的环卫铁骑无功而返,而接下来的更大规模的交手,再一次让党项人体会到,宋人越发优良的弓弩水准。

杀伤范围超过百步的数千硬弩,于同一时刻一齐发射,铺天盖地如飞蝗一般的景象,让跟随前军,在后压阵的梁乙埋也为之胆寒。而战后数以百计的伤员,更是让梁乙埋头痛不已。仅仅是野战就伤了这么多,到了攻城的时候,占据地利优势的守军手上的弓弩,威力必然更强。

“浪讹迂移。”梁乙埋在中军大帐中左右列队的将领中叫起了一人,“你觉得今次罗兀城中守军战力如何?”

浪讹迂移是统领环卫铁骑第二部的将官,今天的第一战,就是他带着手下的骑兵跟罗兀城中的守军对冲了一番。

听到梁乙埋想问,他当即摇头:“不好打!”

“宋人果然善于守城,”梁乙埋对帐中众将说道:“还是照计划去抚宁堡。夺下了那里,罗兀不攻自破!”

