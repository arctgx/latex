\section{第43章 修陈固列秋不远(七)}

“哈,人还真是多。”

蔡渭手搭在额头上,望着几十步外的蔡京家门。

人头涌涌,从蔡京家的门口,到他所在的巷口处,全都给堵上了。

巷口处,几个官员在跺脚大骂。看起来应该是跟蔡京一条巷子的住户,都进不了家门。前门给堵上了,后门也给堵上了。

“这是池鱼之殃啊。”蔡渭看着很乐。

之前冯从义遣人到负责旧城右厢治安的公事所报了案,管勾官和巡检不敢拖延,很快就派了一队兵马来保护蔡家的人身安全,但一群保护士兵都是软弱无力,只要不砸门翻.墙,剩下的都当做没有看到。

围在连邻居都受了害,一起遭了池鱼之殃。石头、砖块、瓦砾,甚至还有用荷叶包了牛粪、马粪丢进院子中。有的在半空就散开了,洒了院内院外一地。

蔡渭知道,蔡京和蔡卞同居一间官宅中。这下子蔡卞也同样倒了大霉。蔡元度是王相公的得意门生,但这一回,就是王相公都不会帮他,求到王安石的门上也没用——如果他现在能出门的话。

因为蔡京的私心,女儿做不成宰相夫人,王相公家里的枕头风吹起来,王安石都得绕着走。

蔡渭远远的下了马,让伴当都在外面候着,自己慢悠悠的晃过去,

他年纪还轻,喜欢凑热闹,听到蔡京在殿上做下的一切,便跑过来想看看蔡京这个竖子怎么败了事。见到被围得里三重外三重的蔡京家,兴致更是高了起来。人群中钻来钻去,从哪些小民嘴里听着骂蔡京的话,就像是三伏天里痛饮冰水那般痛快。

随便找了个看着就是事多的老汉,蔡渭就问道,“老丈,这是怎么了?是哪家欠了帐没给?”

见蔡渭是个读书人的装扮,老汉不敢失礼,点头哈腰的回道:“秀才你是不知道,要是欠了帐没给,哪里会有这么多人来?这家可是做御史的官人!”

“那可不得了。”蔡渭脸上又添了几分惊诧,“怎么有人敢招惹御史?!官家都能骂的!”

“御史骂人,要骂得在理,骂得在理,官家都能骂,可韩相公是什么人,谁有资格骂?!”

“韩相公?”蔡渭脸上的表情,似乎是在说着不能相信,“哪个韩相公?”

“就是宣徽相公啊。还能有哪个相公?过去的那个相州韩相公,早就回天上做神仙了。”

听到韩绛都没人知道了,蔡渭暗笑于心,“哦,他是怎么陷害的宣徽相公?”

“还能怎么陷害的?就是说韩相公名气太大了,立的功劳也太大,叫做那个功什么的”

“功高不赏。”

“对对对,就是这个,就是功高不赏。”老汉叫了起来,“所以蔡贼要斩了韩相公,以防万一!”

虽然左一个蔡贼、右一个蔡贼听得有些扎耳朵,但蔡渭还是兴致盎然的问着,“这事可是确实?”

“那还有假?!不然好端端的会有这么多人来砸他家的门?”老汉阿弥陀佛的两声,又道:“幸好有皇后明察秋毫,才让宣徽相公没被那个歼人给害了。”

“就是!就是!”

见到这边有人聊了起来,还是个看着有些身份的读书人在问,一群男女就拥了过来,七嘴八舌的数落起蔡京的罪状。

贪墨、受赇不用说,徇私枉法也是少不了的。还有蔡家的家丁去买东西,不是不肯给钱,就是往狠里杀价,反正是无奇不有。

这边一个装束挺精神的老头子说蔡京跟韩宣徽有旧怨,恨韩宣徽没有提拔他判厚生司;

那边一个手上抓着佛珠串的老婆子说蔡京是天狗转世,上辈子在天上咬过韩宣徽,结下了因果;

蔡渭听得眉飞色舞,只是半曰功夫,流言就变得稀奇古怪起来,虽然不值一哂,却是有趣得紧。

愉快的心情,一直保持到蔡渭回到家中,踏进蔡确的书房。

“蔡元长那边很热闹吗?”蔡确坐在桌前没动,只是脸稍稍偏了过来一点。

蔡渭本来准备瞒着蔡确,哪里想到一进书房就被揭破了,小声道:“儿子只是路过。”

“从东水关路过到旧城右厢?”

见老子连走了哪里都知道,蔡渭不敢再搪塞,低头认错,“孩儿知错了。”

“算了,以后言行要注意。”蔡确没心思在这时候教训儿子,要烦心的事太多。

“大人,可还有什么事要吩咐的?”蔡渭小心的问着。他一见到蔡确,就像老鼠见了猫,巴不得能早点离开。

“有,想怎么还账呢!你能办?”蔡确很不耐烦,摆手让一头雾水的儿子退下去。

儿子离开,蔡确这才长叹了一口气。

因为御史台的事,他欠了章惇和韩冈一个大人情。章惇还好说,他自己有问题才平添了这么多枝节。

但韩冈就不一样了,他在蔡京身上损失太多,而蔡京背后是蔡确——在表面上的确如此,不管是韩冈的损失,还是蔡京的后台!蔡确也不能当做完全跟自己没关系,必要的补偿还是要给的。

更何况韩冈还帮着清理了门户,又帮自己达成预定的目标。

蔡确又叹了一声,换做是别人赖了就赖了。但韩冈、章惇那个级别就不一样了。只有地位对等,才有资格作交换。也只有地位对等,实力相当,才会让他选择联合,而不是对抗。

韩冈这一回可不是王安石当年被旧党群起而攻时,故意辞官逼天子二选一,而是一脚将蔡京踹倒,然后用力踩进烂泥地里。他从来都不喜欢去借助他人成事。这样的人,蔡确只敢结交,不敢再轻易得罪。

胆大妄为,手段强硬,而且极为狡猾。

只是在殿上时,受气氛影响,蔡确当真以为韩冈压上了多大的赌注,硬是要保住气学的安稳。在官位和学术之间,韩冈选择了学术,在蔡确看来,的确是大损失。

不过等到回政事堂之后,听殿中传出来韩冈与太上皇后的一番问对,蔡确就全都想明白了。

韩冈是什么代价都没出啊,他本来就没有打算在近期重返两府,现在倒好,世人都以为他失去了很多,反而让韩冈在名声上得到了偌大的好处,更省去了以后的麻烦。等到曰后韩冈想进两府了,废除誓言的理由一堆一堆的。看看那两句赌约,韩冈留下了多少漏洞可钻!

蔡确用笔搔了搔头,还这等人的人情,区区一个殿中侍御史就提不上筷子,必须要更高更多的好处,才能抵消得了人情债。

若是当初苏颂还没有进西府,那还好办一点。可韩冈已经推了苏颂进西府。这人情债就不好着落在苏颂身上了。而且章惇、韩冈、苏颂,关系极为紧密,曰后如果苏颂或是章惇对东府有兴趣的时候,韩冈在背后肯定会出一把子力。

更有可能章惇进东府为相,苏颂接受西府之长的职位。如果三人同时发力,得到这个结果不是不可能。

这也是蔡确所不想看到的事。

几番思量,最后蔡确自言自语,“还是要给韩三送个拖后腿的过去才是!”

人选的问题,已经不用多费神去想了。之前韩冈才为他闹过的,三司使吕嘉问因此而灰头土脸,现在将他拉过来最是方便,也能让韩冈提不出异议。

——做不了三司使,而是改做一个翰林学士,想必沈括不至于有什么意见。

蔡确盘算着,玉堂那边正好有个空缺,提议让沈括来做,太上皇后那边也不会有异议,应该会很干脆的答应下来。

蔡确很清楚,沈括就是个墙头草,将他召回来,说不定还会背后捅韩冈一刀。就是他从此痛改前非,老老实实的站在韩冈一边,等他翰林做得生厌,想要往两府里钻的时候,有的韩冈苦头吃。

蔡确靠上椅背,心中有些得意,这人情债还回去后,也算是了了一笔账。而且还省了自己多少麻烦。接下来,韩冈的有些提议就可以公事公办了。

尽管事情的发展出乎意料,但还是得到了预想中的结果,而且可以说,比预计的还要好。如此一来,正好可以利用现在的形势,将自己的班底更加增厚,等韩绛致仕,独掌朝纲的曰子可终究要到了。

除去了成本,蔡确开始盘点这一回拿到手的净利。

一个、两个、三个、四个。

光是一个御史台,就至少有四个位置他能够确定抓在手中。而具体任官的人选,蔡确也早有了腹案。

但在这之前,蔡确决定好生的调教一下准备提拔的这几位,让他们不至于变成蔡京那种会反噬主人的劣狗。不过太忠顺的狗,不一定有姓格坏的劣狗能派上用场。所以还得找一个用处更大,但也稍微危险一点的家伙。

“去请刑和叔来。”蔡确吩咐着外面的亲随。

刑恕在蔡确门下奔走已经有不短的时间了,又在程颢门下很有些声望,在洛阳更是受都到一些老臣的看重。

左右逢源的地方,当世真没几个能比得上他。也不知道在吕公著、司马光那边,他是怎么遮瞒过去的。

但蔡确相信,刑恕现阶段绝不会背叛自己。一条好狗应该知道谁才是主人,谁才能给它们肉吃。

而等到刑恕有了读力找肉的能力后,蔡确会很干脆的将他给处理掉。

这一点,很重要。

