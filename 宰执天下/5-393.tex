\section{第34章 为慕升平拟休兵(25)}

辽军撤了。

这是意料中事。

不能战不能守,缚手缚脚,除了及早撤离,还能有什么办法?

说实话,韩冈很同情萧十三,换做他处在萧十三的位置上,也同样无所施为。

此时已是春日,位在河东,加之麾下的各部兵马早已满载而归、人心厌战:

天时地利人和一样不占,这样的战争要是还能赢,那就是敌人一边犯了大错。

可惜韩冈的性格是与他年纪完全相悖的谨慎,从不冒险浪战,一直都是稳稳的抓住辽军的弱点下手,以身后雄厚的国力跟萧十三硬耗。这样一来,萧十三哪里能有机会翻盘?

不过韩冈目前还是战战兢兢、如履薄冰,已经锁定胜局的情况下突然被翻盘的战例他听得太多,纵然辽军今夜的撤离是板上钉钉,确凿无疑,可只要还没能夺回雁门、西陉和瓶形寨,这一场发生在河东的会战就还有被辽人反败为胜的可能。

闻讯而来的幕僚和将领却没有韩冈的冷静。

之前辽军从太原一直撤到了代州,还可以勉强说是将伸出去的拳头收回来,但这一回出拳之后,不仅没有战果,还被逼着收回去。都是带兵的人,仗打成这样,纵是孙吴再世,都不可能还保得住士

气。

大捷在即,没有人还能忍得住心中压抑已久的兴奋,环顾左右,都是一张张强自抑制了心中兴奋和喜悦的脸。

也就是韩冈有积威,让他们不敢在帐中大声的欢呼。

“枢密,现在怎么办?!”折克仁勉强收拾住心中的狂喜,向韩冈询问道。

不经大战,却将辽军逼得无路可走,能有这样的胜果,韩冈这位主帅的运筹之力,在其中的功绩当然是排在第一位的。积威既深,包括折克仁在内的将领们不敢主动向他提出建议,只敢等待征询和命令。

“之前的计划是怎么定的?”韩冈转问黄裳。

黄裳同样冷静——如果不看他有些颤抖的双手的话,就会有这个感觉。听到韩冈的问题,他立刻回答。

“在……在之前的预计中,辽贼的确会放弃大小王庄,粮草和士气,不足以让辽贼再坐守此处。一旦辽贼撤离——无论夜中撤退春日退兵——都必须出兵追击,不能让他们能够安然返回代州。只有给予辽军足够的打击,才能让接下来的收复代州和雁门西陉等边境诸关塞更加容易,伤亡也能更小一点。”黄裳向韩冈微微躬了躬腰:“就不知道枢密打算派谁去了。”

“枢密,末将愿往!”七八人异口同声,然后互相之间怒目而视。这是立功的良机,与辽军正面作战是一回事,痛打落水狗又是另外一回事。手上有个俘斩千人的战果,封妻荫子岂在话下?

韩冈没理会他们,继续问黄裳:“何时出兵为好?”

“最合适的时间是在天亮前,四更前后。”“天亮前出营的话,步军赶到大小王庄正好天亮,不必担心埋伏。而且喧闹了一夜,辽军正是人困马乏,追逐他们的骑兵也不必担心反扑。”

“时间上可来得及?”

“辽军千军万马走不了多快。先回去的已经到代州城,殿后的估计才出营。完全来得及。”

帐中稍稍变得安静了一点。几个问答下来,各自都明白了这是韩冈借黄裳的口,向他们说明接下来的计划。

韩冈很喜欢这样的安静,他的视线在帐中绕了一圈,将领们都摄于他的威势,纷纷垂下头去,无人敢与韩冈直面相向。

停了片刻,韩冈方才打破突然而来的静默:“辽军那边的将帅都不是第一天上阵,不能指望他们在撤退的安排上会有多少漏洞,甚至可以说,还要提防他们籍由此种情况而顺势设下陷阱。行百里者半九十。虽说身处窘境,但辽人不是没牙的狗,还是狼,一点疏忽就会被咬到脖子上。如果有人认为辽军已经不堪一击,像间破房子一脚就能踢倒,那这一回就别出去了,免得我上表给他向朝廷求抚恤!”

众将笑意收敛,无不悚然,向韩冈恭声道:“枢密有命,末将等何敢不从?”

……………………

萧十三面无表情的目送着又一支千人队的离开。

自从太谷城下士气一落再落,直到现在援助前线的一番无用功,辽军的士气已经一落千丈。现在被迫退兵,等于是雪上加霜,在伤口上又撒了一把盐。

但再留下去形势只会越来越差,萧十三所看到的结局,除了全军覆没,就是全军溃散而逃。所以他的选择只有一个——纵然再不情愿,也不能留在死地。

“耶律道宁,准备得怎么样了。”

原本大小王庄的主帅,同样是板着脸,漠然的行礼:“枢密放心,已经安抚好了。”

萧十三瞥了他一眼:“你也放心。我会率本部留到最后的。”

两万士卒和五万战马的驻地,是以大小王庄为前沿核心的一片方圆广达五六里的土地。兵马之多,原本赶来支援的时候仅仅一万四五千人,就用了一天的时间方才到齐。现在要走,只会用上更多的时间。

如果不想落到一个兵败如山倒的局面,萧十三自是要选择良将为大军殿后。所以耶律道宁被选中了,要维护尚父耶律乙辛的威信,就必须要让尚父殿下的亲信将校领着宫分军出面。

但仅仅是殿后的两三千人依然并不足以遏阻宋军,同时维系住军中的士气,耶律乙辛和他萧十三本人的威信。所以萧十三本人也决定亲率本部留在最后,保护全军安然返回代州。

耶律道宁默然的躬身一礼,再没多余的话。

随着离开的各部兵马渐多,大小王庄营地内的火光越来越少。离开的人光明正大的举着火炬离开,照亮前路。他们有一部分的作用是吸引宋军的注意力。但断后设伏的一部兵马,则是潜伏在黑暗中,等待上钩的宋人。

“代州还能守吗?”张孝杰低声问着萧十三。

出营的人马一个个都是垂头丧气,不仅仅马背上的骑手,还有他们所骑乘的战马。马匹这样有灵性的动物,比起其他牲畜更能体会到人心的变化。

对马性还算熟悉的张孝杰甚至能感觉得到撤离的大军中,每一匹战马都被感染上了一层灰色的阴郁。

萧十三诧异的看了张孝杰一眼,之前都商量好了,怎么还问。

但看清了张孝杰脸上的难以掩饰的神色之后,他便明白了。纵然一切都跟着计划走,可在心中缺乏底气时,依然需要言语来平复心中的乱绪。

“如果能稳守雁门,繁峙,还有很大机会的。”

萧十三不准备守代州了,那么大的城池,区区两万人根本填不满,守城的又是完全不擅长城池攻防战的骑兵,等于是给宋军送功劳。还不如退守到边关上,只凭地利,就能让宋军吃个苦头。

“不知道尚父那边听到之后,会是什么想法?”张孝杰叹了一口气。有兵马在手的萧十三不用太在意,可他却不能。

“为尚父保住兵马才是最重要的。”

没了兵,就不能维系住现在的地位。而只要有兵马在手,无论败了多少次,丢了多少脸面,照样能保住地位、保住地盘。

“而且只要有雁门和繁峙县在手,多半还是能换回尚父想要的东西。”

交换的筹码不在多寡,只在对方是否看重。

雁门山上的雁门、西陉二关,以及飞狐陉上的繁峙县瓶形寨[平型关],是宋辽两国边境上的数得着的要隘。是门户锁钥。

这两处关隘的价值无论在宋辽双方的眼中,都不输于代州。如果拿来交换的话,还是能换回一部分想要的东西——比如失土,比如钱帛。

“希望还不要到那个地步!”“张孝杰回望着西南方的黑暗,“希望韩冈的胆量能再大一点,心再贪一点。”

如果宋军狂妄自大,他们还有翻盘的机会。

张孝杰话声刚落,远处一阵骚然。

“宋人的骑兵出来了!”

来自望楼之上的呼喊声刚刚落定,几条火龙从一线的营寨中逶迤而出,然后就像孔雀开屏,一下分散开来,变成了漫山遍野的星星火光。

战鼓声在宋军军营上空回响,出营的军队声势浩大,星辰交汇而成的潮水向着刚刚离开军营的辽军涌来。

但星朝前端的黑暗里,早有斥候先行一步,双方的游骑远在双方的骑兵主力接战前就开始了的交手。

“宋人是虚张声势!不须惊慌,不得争抢,按照事先预定的顺序离营!跟他们说,反击的机会马上就要到了,要稳住。”

萧十三接连派出亲信去传话。并又加派一部兵马保护侧翼安全。试图将宋军的骑兵遏止在撤退的道路之外。

不过这话他说得并没有太多底气。有轨道在,就是萧十三他本人也不敢保证韩冈是不是有运来更多的骑兵。但他却不得不说。能安抚一个是一个,稳住全军的阵脚是他的责任。

不论成与不成,至少能保证主力能够安然回返代州。

但另一方面,就像他和张孝杰之前所期盼的,机会看起来的确来了。
