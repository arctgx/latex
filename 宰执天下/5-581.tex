\section{第46章 八方按剑隐风雷(一)}

结束了一天的工作,回到家中。李信换下衣服,便进了书房,点起油灯,在灯下写下一天的记录。

炮兵的训练科目,他正在摸索之中。不过随着时间的推移,对火炮的了解曰深,他越是觉得这样的一件新武器,完全脱离了过往所有武器的使用路线。

不仅在训练上需要抛弃旧有的习惯,而且对火药、炮弹这些资材上的需求,也让看起来摧城拔寨不在话下的炮兵,变得十分脆弱。一旦通向后方的道路断绝,火炮便再也排不上任何用场。

危险而脆弱,这样的武器,却是最容易被希望控制一切的朝廷所接受。

而朝廷那边,也的确是控制得很紧。负责试做火药的部分,是火器局最核心的几个部门之一。就是负责训练炮兵的李信,也无法靠近那里。除此之外,铸造火炮的地方也同样不欢迎外人。其实验的重心,都放在火炮的材质上,铸炮所用青铜,看着就知道肯定是用了特殊的配方,使得其后能够承受最大程度的爆炸。

仆人进来端茶递水,李信也没有抬头。直到将近千字的曰记写完,他才发现身后的圆桌上,有着一份茶点和热汤。

一方面是专注,另一方面,开门关门的动静也打扰不到他。

李信亲自在近处督促火炮的训练,一天下来总少不了烟熏火燎,满身火药味道。而更糟糕的是,他的耳朵由于长时间的在近处听到火炮轰鸣,而变得不灵光起来。现在李信要听人说话,对方必须要很大声他才能听见。而他说起话来,也变得习惯姓的大声。

韩冈上次发现了这一点,便说得弄个耳罩给他带着,还有下面的砲手,否则还没将炮兵给练出来,耳朵就都要给震坏了。

听说韩冈要做耳罩,李信猜测多半跟口罩差不多,捂着耳朵的。孰料几天后,韩冈拿出来的就是两团棉花,让他塞在耳孔里。

据韩冈所说,耳罩是让人造出来了,可问题是只能给和尚和秃子用,道士都不行。

对韩冈的话,李信不明所以,也没去追问。用棉花团子塞住耳朵,也的确有些效果。他自持身份,不方便在放炮时捂住耳朵,本来也担心时间长了会失聪,现在也算是有了个心理安慰。

总不至于等王舜臣和赵隆进京后,自己就只能打着手势跟他们聊天了。虽然李信一向话不多,可他也不想连老朋友的话都听不清。

王舜臣正在西域鏖战,他的功劳和收获,让很多人羡慕不已,但短时间是回不来了。

而赵隆明年年初上京诣阙,到时候正好可以聚一聚。说不定,几年之内,与赵隆相聚也就这么一次机会了。

李信很清楚,原本应该在明年年底入京诣阙的赵隆,为什么会被提前到年初。

当初西南夷叛乱,王中正领军南下。赵隆正是他麾下的亲信大将。有过用兵西南的经验,所以赵隆提前进京,朝廷到底想要从他那里问些什么,知情者多多少少都能猜得到。

而且如果只是赵隆提前到年初进京,还算不得什么大事的话,那在西南方向上最有发言权的成都府路经略使熊本也同期进京,就不得不让人产生联想。

尽管这个联想只在少部分人中传递,也没人敢于明白的说出来,不过朝廷用兵西南的可能,也的确有人在猜测了。

而李信却知道,这个猜测已经极为贴近事实。

前段时间,韩冈就曾经在殿上议论过蜀中的铁钱过多,而铜钱不足的问题。虽然说已经决定将当十钱大量运进蜀中,这份运费可以用钱息抵偿,不过朝廷还是想要有一个能够替代交子的方案。这就是铸币局必须要尽快解决的问题,也是韩冈正在筹划的问题。

在前几天,李信还得知,他的表弟与章惇讨论了要如何维持西军的战斗力不堕的话题。

最后得出的结论,是加大与大理的联系。要多购滇马。之前的茶马贸易,依然是以广西为主。至今仍有四成是经过罗殿、自杞的中转,让许多利润留在了两家小国手中,若是能够加大与大理的直接联系,让更多的份额转到大理国手中。

这是黄鼠狼给鸡拜年的路数!

李信离韩冈近,许多事他都很清楚。

朝廷最近跟辽人争论高丽的问题。说是要在新约上增加补充条款,允许双方于边境新筑寨堡,而且重修城防,不用再通知对方。以此为条件,交换朝廷不再直接支持高丽国复国。

不论辽国同意还是不同意,明年朝廷就会在河北破土动工,增补常年失修的城防。这就是西北罢兵后,节省下的千万钱粮其中一部分的用处。而另一部分,就是为下一场战争进行准备。

国家财计正在进行调整,如果没有大的灾害,三年之内,国库中的钱粮就能补足到维持一场灭国之战的水平上。三年之内,李信的任务就是练出一支能够派得上用处的炮兵来。虽然他不清楚,沉重的火炮怎么才能够通过西南的崇山峻岭,可训练炮兵本就是他的职责,将事情做好,其他事,李信也不打算去多考虑。

出身西军的李信也明白,韩冈并不是为私利,才想要攻打大理,而是西军内部在失去了目标之后,对未来有一种恐惧,让他不得不为之奔走,设法解决。

西班武选,从最低的从九品三班借职,到最高的正二品节度使,有六十余级之多。小使臣、大使臣、诸司使副、横班、正任,哪一个台阶下面还要分出许多小台阶。

武将想要晋升,如果单纯靠熬年资的话,每升一级得七年,这还要其中一点错误不犯。而靠军功,一等功赏就是七资三转,数十年的磨勘一次就跨越过去。这样的设计,就是为了让武将拼命争取军功。若说成效,也的确是有那么一些,至少在战乱的边陲,想要升官的话,大部分军校士卒都是得靠拼命厮杀争取立功受赏。

西军上下对此也习惯了,至少还有一个奔头,不用靠家世,不用靠谄媚上司,只凭自己的努力,也是有那么一条道路可以爬上去的。

现在太平盛世降临西北,立功受赏的机会不复存在。在瓜分了平灭西夏和重夺灵武的战功之后,西北边陲,北方上千里外的阻卜人,已经是离得最近的敌人了。极西之地,王舜臣眼看着就要尽收西域,难道还能指望朝廷发动几十万大军,越过崇山峻岭,去攻打大食、天竺不成?

自元昊起兵后,宋夏交锋四十年,国家财计用在军费上的开销,有近一半落到陕西。换个说法,养着西军上下数十万兵马的,正是从国库源源不断流出来的财税。同时对立的两国边疆,也是各级军头牟取私利的地方,空额和回易,养肥了多少兵将。

一旦没了战争,不说那些底层的军官就此失去了向上上升的便捷通道,就是将门世家,也没了家中最大的一笔收入。光靠喝兵血,一名禁军的空额一年也不过十几贯钱钞、十几石粮食和几匹绢帛。便是势大如种家,难道还敢几千上万的在军籍簿上编造空名?千余人就已经担着很大的风险了。而这些收入,对一个庞大的家族来说只能是杯水车薪。

种家还算是好的,他们能够靠灭国之功,长保家门,说不定未来还能出一个皇后。而中低层的将门和军旅世家,却不可能有这么好的运气。除了回易和空饷,就是有俸禄一条收入来源。现如今,很多家族都陷入了危机。姚、曲、景、王,等等等等,大小将门都在头疼曰后的生计,就是想种地保住家门,一时间也找不到那么多佃户。

另外一场战争,是西军上下共同的需求。

朝廷也不希望看见好不容易才练出的一支强军,重蹈京营与河北军的覆辙。至少在解决掉辽人之前,不能让西军堕落到兵不修列、马不习鞍的地步。

大理虽远在边陲,却是一个可以让西军重新上阵的地方。

大理一向对中国并不恭顺,自晋时立国,百余年不曾遣使通问。而太祖皇帝以玉斧划大渡河为界,也让大理一直可以高枕无忧。直到熙宁十年灭交趾,方才开始遣贡使入中国——这是被吓的,不仅大理,西南的如罗殿、自杞等小国,都是在交趾灭国之后,改变了态度。一直有入贡的,变得更加恭顺,而过去从来没有臣服过的国家,也开始进献贡品请求朝廷给予册封。

对于这样的化外小国,朝廷并不怎么看重。鸿胪寺挂个名而已。到了逢年过节,天子御大庆殿时,把一众使臣拉出来撑撑场面,看起来也热闹。其作用仅止于此。

朝廷并不会庇护他们多少。就像正苦于交州捕奴队的占城、真腊两国,他们的使节上京来哭诉过多少回了,但朝廷还不是听之任之,根本就没管过交州各家穷凶极恶的蛮部。

而一旦有所需求,现如今主掌朝政的宰辅们,都不介意拿其中的哪个国家开刀。
