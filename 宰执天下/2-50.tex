\section{第12章 平生心曲谁为伸(四)}

【迟了一点,奉上第二更。关于下个月的更新问题,在书评区已经有朋友问了,俺在这里解释一下。十二月和一月,俺都是一日三更,保证两个月六十万字,是因为俺去年十月在龙空论坛说要发书,承诺一天五千字。但因故拖了两个月,直到十二月才开始发文。所以俺最后承诺,在十二月和一月把前两个月的欠账补上。而现在欠账终于全部还清,俺也可以恢复到正常的更新状态。从下个月起,一天两更六千字,中间可能会视情况加更一下。对此,请各位兄弟予以谅解。】

当天子和宰臣们在崇政殿中为文彦博的健康担心的时候,另一个人则已经不再需要被人担心健康问题了。

“窦副总管下手还真够狠的。”秦州州衙的后门处,王舜臣看着眼前被两名差役抬着的一卷芦席,啧啧着嘴,发着事不关己的感叹。

卷起的芦席合抱粗,五尺长。上面给遮得严严实实,下面却露出了两只脚。一只脚尚穿着黑色靴子,另一只脚却是光着,连袜子也不在了。

韩冈探手将席子的一角掀起,一张扭曲的脸露了出来。眼睛瞪得大大的,瞳孔散了,涣然无神,嘴巴和鼻子都因痛楚而歪斜着,看上去已经与生前的相貌有了很大的区别,这是在剧痛中被杖子打掉了小命的缘故。不过尸体只是口鼻处有血渍,但脸还是干净的,窦舜卿没打脸。

“抬出去吧。”

韩冈放下席子,直起腰退到一边。站在州衙后门口,把抬尸的拦住,也不是桩吉利的事。尸体堵着门,守门的门房都急着搓手。

王舜臣目送着一卷芦席被抬远,回头对韩冈说着:“王启年的运气还真是不好。”

“这不是运气。”韩冈摇摇头,“天作孽,犹可恕,自作孽,不可活。本官与他宿无旧怨,他为窦舜卿设计害我,才会落到如今的地步。”

王启年被杖死了,这也是意料中事。窦舜卿怎么可能不杀他灭口?先是出了个馊主意,却又被要谋算的对象看破,被硬逼着上门送信。奸谋被看破没什么,但闹出来就不好了。窦舜卿想把此事一推三五六,当然要把王启年灭口。

今天早间,窦副总管就是随便找了个借口,比如天气太热,早饭没吃好,树上的知了为何还在叫之类的罪名,把王启年叫到官厅去,扑翻了拿大杖敲了一顿。下手的都是窦舜卿身边那几个身强力壮的护卫,一个比一个手重,一二十棒下去就收了王启年的小命。

抬着王启年尸体的差役已经转过来街角,韩冈收回视线,又叹了口气。虽然王启年的死早有预料,亦有腹案,但看着已经投靠自己的人就这么不明不白的死了,心中当真是很不痛快。想来王韶眼睁睁地看着纳芝临占等七部被董裕打得族帐尽毁,也是这样的心情。

回过身,韩冈往衙门里走,不过不是回他的官厅,也不是去找王韶。王舜臣看韩冈走的路,却是径直往副总管和钤辖两家官厅所在的三进东院去的。

“三哥,你去哪里?!”王舜臣追在后面惊道。

“窦副总管那里啊。”韩冈轻飘飘的说着,像是吃过晚饭跟家里打个招呼,说要去邻居家串门一般,“王启年怎么说都是我勾当公事厅里的人,他被杖死了,总得跟窦副总管辩上几句,讨个说法。省得有人说我们不顾手下人死活。”

“三哥!你……”王舜臣先是急了一下,但立刻又反应过来,前面的是谁?那可是他的韩三哥啊,一肚子计谋的韩玉昆!别看他一直鲠着脖子大步往前走,但任是哪位高官显贵撞上他,可都是无一例外的跌得灰头土脸。王舜臣凑上前,压低声音问道:“三哥,你是不是在打什么主意?”

“你说呢?”韩冈笑着反问他,毫不犹豫地跨进了窦舜卿官厅所在地院落。

刚刚亲眼监督着把背主作窃的王启年杖死,看着他被打得血肉横飞,从厉声惨叫到无声无息,窦舜卿的心情终于好上了那么一点。

但他根本没有想到,转眼间,韩冈竟然直接杀上门来。而韩冈跨进院门那副气势汹汹的模样,也顿时引来一群人在外面探头探脑。

韩冈向着窦舜卿行过礼,指着脚边还残留着的血渍,毫不客气的质问着:“敢问观察,不知鄙厅吏员王启年究竟犯了哪条律法,为何要将其杖责致死?!”

窦舜卿闭目不理韩冈,仿佛开口说句话就会丢了他的身份。他的一个幕僚代窦舜卿回答:“办事不利,欺瞒上官。”

韩冈看了那幕僚一眼,也是窦舜卿身边的有名人物。名叫林文景,经常为窦舜卿做些私下里的买卖,仗着副都总管的威势,跟窦七衙内一样,在秦州城中横着走,平素里最是趾高气扬。

听到他代窦舜卿回话,韩冈便追问着:“不知所谓的办事,究竟是办得什么事?”

林文景哼哼冷笑了两声,扬起下巴,阴阳怪气的说着:“这也是你这个勾当公事够资格问的?!”

“难道我不够资格问?王启年可是勾当公事厅中的人!”韩冈抬手一指林文景,提声喝道:“还有!本官向观察请教事务,要说话也是观察来说,轮不到你这个白身插嘴!你给我闭嘴,站一边去!”

韩冈毫不客气的指着林文景的鼻子训斥,官厅外,又一下传来压得很低的哄笑。林文景的脸顿时涨得通红。他在秦州城中还没受过如此羞辱,自来到秦州的这段时间里,哪个不是对他毕恭毕敬,就算是李师中、向宝见了他,也是客客气气的。林文景紧紧的咬着牙齿,格格作响,恨不得冲上前,一刀劈了面前这个猖狂的灌园小儿。

窦舜卿这时终于睁开眼,抬手拍了下交椅的扶手,声音沉沉,“韩冈!你敢乱我公堂!”

兵马副总管的威势不是等闲,外面的窃笑声没了,厅内厅外都在等着韩冈的反应。

“不敢!”韩冈拱了一下手,腰背挺得更直,“下官只是来请教观察为何将鄙厅公人杖死之事。王启年自有家人,他被观察下令杖死,究竟是个什么罪名,又是因何事而死,本官总得跟他的家人交代一番。”

韩冈的口气稍稍软了一点,后面解释了几句像是在给窦舜卿台阶下。

“王启年办事不利,所以杖责于他,也是给人一个提醒。至于什么事,事关机密,不是你该问的。”窦舜卿没有说出杖责王启年的理由,但这也算是个回答了。他堂堂兵马副总管向个勾当公事开口解释,给足了面子,在窦舜卿想来,韩冈也该知趣的退了。

韩冈却正等着窦舜卿如此说话,立刻又追问道:“既如此,观察何不将王启年械送正厅,交由都总管处置。机密之事下官不得与闻,但都总管总该是能听的吧?王启年是经略安抚司中公人,观察代都总管定罪,未免是越俎代庖了。”

窦舜卿脸色木然起来,右手紧紧地捏着交椅扶手。李师中是秦州知州,秦凤经略安抚使兼兵马都总管,这三个差遣,韩冈却只把都总管这个身份提出来说,一句句的不就是在说自己只是副都总管吗?!

他看了看左右,恨不得立刻下令将韩冈一样杖死在厅中。只是他能这么做吗?外面有这么多旁证,以下犯上的罪名也栽不到韩冈头上,何况韩冈还是文官!该死的文官,窦舜卿心中发恨,‘这武夫真的不能做!’

“韩冈……”窦舜卿慢慢的念着韩冈的名字。

韩冈拱了下手,作出静候上命的样子来:“下官在!”

“你且下去,此事我自会跟李右司说。”被韩冈拉出李师中这张虎皮,窦舜卿其实也难再说什么。杀也不能杀,打也不能打,只能暂且退让,日后再前账后账一起算。但他却还是在话中争上了一口气。

韩冈一听,就在心中暗笑。虽然差遣不如人,但窦舜卿的本官观察使是正五品,而李师中的本官右司郎中则是正六品,论官品,却是窦舜卿在上。窦舜卿拿着本官称呼李师中,这是争着个名分高下,也不知李师中听了会不会高兴。

“此事下官也会禀报个都总管,请他给个公道!人命关天,不是想杀就杀的。”韩冈依然板着脸,义正辞严的说了最后一句。他行礼后告辞离开,丢下身后被他气得直抖的窦舜卿。

韩冈走出副总管官厅所在的院落,却见王韶和高遵裕就站在了院外,等着他出来。

韩冈向两位顶头上司拱手行礼,却没有半点讶异。州衙就这么大,他在窦舜卿这里大闹一通,两人怎么可能收不到消息。若是方才窦舜卿真的敢发作,王韶和高遵裕肯定会进来救人。

三人一路走回高遵裕的公厅,在房中分宾主坐下,高遵裕便问道:“玉昆,怎么今天发了这么一通邪火?只为了个王启年?”

“前几天王启年被下官逼着投了过来。本意是想让他送个投名状的,但没想到窦舜卿如此手辣。”韩冈摇头叹着,“今天看到王启年被抬出去,心情有些不好,干脆找着借口去闹上一通。”

“气出了没?”高遵裕笑问着,心道这韩玉昆真是年轻气盛,平日里精明厉害,但火气起来当真是什么都不管不顾了。

“当然没有。窦舜卿不走,下官日夜都睡不好觉,就感觉有条毒蛇在背后。”韩冈神色深沉起来,“窦副总管早早就把下官视为眼中钉,阴谋诡计一桩接着一桩,下官总得想个办法自保才是。”

