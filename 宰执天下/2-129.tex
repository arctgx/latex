\section{第26章 西山齐云古今长(中)}

【因为私人原因昨夜没有更新,在这里说声抱歉,这是补昨天的份,晚上还有两更。】

“球赛?是玉昆你明天下午在疗养院里办的那场?”

关于韩冈明天的计划,高遵裕已经听说了。古渭城不大,在城墙上绕一圈半个时辰都不要,夫妻吵架之类的小事传播开来,也只要半天功夫。只是他没想到韩冈会来邀请他。

“这也算是敦亲睦邻了,谁输谁赢倒无所谓,只望他们能把打架的力气放在球赛上。”

“玉昆你操心的事还真多……也罢,明天去一趟就是了。”

疗养院是韩冈的地盘,只要不犯王法,他想做什么都没问题,高遵裕不会干涉。不过韩冈还过来邀请他亲去观看比赛,让现在正主持安抚司运作的高遵裕很不以为然。

韩冈在疗养院中举行球赛,高遵裕觉得根本就是不务正业。要是踢场球就能解决蕃人和汉人之间的矛盾,大唐跟吐蕃斗了那么多年,又该怎么说?

高遵裕并不是多喜欢看热闹的性子,在他眼中蹴鞠不过是百戏而已,每年节庆祭典,都能看到宫中养得一群踢球的兵士上场表演脚法。而且那些兵士的水平,都是跟鱼鳔胶一般,几乎能把球黏在身上,指哪儿踢哪儿。天下间水平最高的比赛都看过了,高遵裕怎么会对低水平的较量感兴趣,但韩冈的面子不能不给,却也是没二话的就答应了下来。

韩冈谢过高遵裕,便告辞离开。一直在旁听着的一名亲信便对高遵裕道:“吐蕃人又不踢球,韩玉昆让他们上场,怕是会闹笑话。”

“笑话也无妨,要丢脸也是韩冈他丢脸。明天就去捧个场好了,闲着也是闲着。”

……………………

熙宁三年的冬至,对鲁平来说是个很寻常的日子。都长到二十多岁了,每年的冬至都是一个花样,换身新衣裳、吃吃喝喝一番,也就如此而已。又不是小孩子,早已对节日失去了无谓的期待。即便是要在今天参加一场蹴鞠比赛,也是一样。

对于曾经在秦州参加过齐云社【注1】的鲁平来说,踢一场球也算不了什么,自他十五岁开始上场,哪年过节没有一场比赛。即便今次的规则跟他所习惯的完全不同,可只要还是用脚来踢,做过三年齐云社球头的鲁平,就绝不会输给任何人。

鲁平他原本是内科的病人,是因为吃了不干净的羊肉,前些日子跟同一队的几个袍泽兄弟一起被送进了疗养院。调养了几天后,食物中毒的这群人陆陆续续的都出院了,就是鲁平因为当初吃得最多,便给落在了最后。

本来前两天也该出院了,却不合跟院中的吐蕃人斗了起来。事情的起因已经没人能记得了,但鲁平从内伤转外伤却是实打实的,在如同漩涡般,将一点小口角变成了一场席卷全院的群架中,他被一棒子敲破了脑袋,刚出了内科,就又送进了外科。

因为头上受伤的缘故,鲁平的头发都剃得干干净净,长条的细麻布带盖着合伤的膏药,在他的头上缠了一圈又一圈。摸着被光溜溜的脑袋,青茬茬的头皮发出沙沙的声响。鲁平近七尺的身高,外表又是恶形恶状,左眼眼角还有一条刀疤拖下来,狰狞骇人,乍看上去就是一个不知吃斋念佛、只爱杀人放火的假和尚。

换了球衣球鞋,鲁平跟今天的队友们站在了一起,高高低低总共十人,半是蕃人,半是汉人。只是穿着同样的红色衣袍,便模糊了不同民族之间差别。

标准的一支蹴鞠队是十六人的编制,一名唤作‘球头’的队长领队,下设跷球、正挟、头挟、竿网等位置。不过这样的编制是针对单球门的比赛,而今次组织的比赛,是唐时比较盛行的双球门——这里球门唤作鞠室——也因此,编制也好、规则也好,都与鲁平所习惯的完全不同。

各家球队都是依照不同花样的衣服区分队别,往往在衣服上还要绣花刺字,打扮得花团锦簇。只是今天出战的两队因为都是赶鸭子上架,来不及准备合适的队服。仅仅是分作红褐两色,内科队穿褐衣,鲁平所在的外科则是红衣。穿黑衣的也有,却只有一个人,嘴里叼着根竹管,仔细看过去,却是根木笛。

鲁平探脚踩了踩球场的地面,脚上的靴子是他参加比赛时的专用球鞋。古渭疗养院本就是军营改造,外面附送一块小校场,平整一下就是一块上好的球场。他昨天从朱中那里听过了关于规则的介绍,今天看了球场,的确与他过去的球场完全不一样。用石灰线描出来的场地,长三十余丈,宽十五六丈,两边各设一木框的球门。

‘只要往门框里踢是吧……’鲁平望着不远处的球门,心里满是自信。以他的脚法,比起把球踢进只有两尺见方的风流眼,六尺多高,近两丈宽的球门实在太大了。

离球赛开场还有一段时间,但球场周围的空地上已经陆陆续续的进驻了不少观众。比赛的消息早已传了出去,从一大清早,就来有人在院门前守着。等到开放门禁时间到了,大门敞开,今次来观众的观众便络绎不绝的涌了进来,竟有上千人之多。虽然无法与东京春时金明池争标,动辄十几万人来观战,但在古渭已经是难得一见的盛大场面。

鲁平为人四海,人面广,人头熟,其中有许多都跟他或多或少的都有些交情。场边一个大嗓门在喊着鲁平的名字:“鲁七!上去了别再拉稀,俺可是押了你的注!”

鲁平抬头骂过去,“拉你个鸟,爷爷就是只剩一条腿,三十贯的花红也落不到他人头上!”

“七哥,俺也压了你的注。赢了请你喝酒!”

“差的酒洒家可不要,至少得上锦堂春。”

“鲁七哥,才两天不见,怎么出家做和尚了。”

“等给你念经送终过后,爷爷会还俗的。”

鲁平人缘不错,名气也不小,跟他搭话的人不少。只是当他回过头,瞥见站在附近、同样穿着一身红袍的一个矮个子的蕃人,眼神一下危险起来,头上的伤口也开始隐隐作痛。

这个名叫乌克博的蕃人就是前两天跟他厮打起来的对手。虽然拿棒子在他身后下阴招的不是乌克博,但鲁平已经把乌克博给狠上了。他可是脑壳上被打了补丁,那条裂开来的伤口据说来回缝了十几道。虽然到现在也不清楚下手的究竟是谁,但只要知道是吐蕃人就足够了。

鲁平走到个矮体壮的乌克博身边,有三十贯的花红悬着,他只有今天并不想跟这蕃人翻脸。鲁平也不正眼看人,平视着前面:“喂,今天别拖爷爷后腿!”

他知道这些蕃人都会说官话,能住进疗养院的蕃人,无不是各家蕃部中的头面人物,学懂官话是他们必需的技能,与只知道跟牛和羊说话的普通蕃人完全不同。但乌克博没理会鲁平,双手合十,喃喃的念着佛经。

鲁平脸色难看起来,双手有意无意的握着拳头。过了一阵,他才松开手,一口痰便吐在乌克博的脚前,转身走开。

……………………

“怎么这么多人?”

还没进门,就已经听到嘈杂噪耳的喧闹声,等到正式走进校场,高遵裕也不免吃惊于观众的人数之多。球场周围人山人海,少说也有两三千人之多,几乎半座古渭寨的人都到了。

这还是古渭疗养院第一次举办比赛,消息又是两天前才传出来的,竟然一下子聚集了这么多人来观战,实在出乎高遵裕意料之外。

陪行在侧的韩冈脸上的微笑仿佛在说一切尽如所料:“都是闲得没事闹的。地里没活了,商人也要回家过年,蕃人更是老实,现在就是路上有人吵嘴,也能围上一群人,何况是球赛?”

古渭地处偏远,娱乐活动几乎为零。喝酒听曲的地儿都没有,虽然有两个妓寨,但都是面向普罗大众,里面的水准基本上是不堪入目的。所以尽管今次只是疗养院的内部比赛,又是事发仓促,还是吸引了大批的观众。

韩冈只打算先在疗养院中开个头,把观看球赛的风气带起来后,便能在城中推广更为正式的比赛。就算是在边境领军屯田,韩冈也不认为他的任务仅仅是耕战,文化娱乐也是很重要的方面。弓弦不能一直紧绷,总得有放松的时候。

而且蕃汉之间的矛盾尖锐,对日后缘边安抚司的发展也没有好处。要化解矛盾和纷争,光是上层压制和拉拢并不够,下层也要联络感情,这一方面没有什么比文化的交流更适合了。

注1:齐云社,也称圆社、天下圆。起源于北宋,盛起于南宋。在南宋时以杭州为主,全国各地都有分布,是全国性的蹴鞠运动的社团组织。由于齐云社的起始年代无法确定,书中就当作熙宁时已经出现。

