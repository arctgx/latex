\section{第36章 万众袭远似火焚(十)}

浓浓的药味弥漫在空气中,给清洗好的绷带高温消毒的炉灶烟火不绝,一个个身披蓝袍的护工们奔走忙碌。

前方连续克城,出兵以来的一场场胜利光辉夺目。但胜利的背后,是躺满了伤员的狄道城疗养院。

当韩冈走进病房的时候,刘源正裸着上身,让一名护工给他换药。在胸口几处箭疮都已经缝合,不复被送到战地医院时的惨状。现在又被护工拿着烈酒擦洗得发白,涂上了养伤的金疮药。刘源虽是觉得伤口处被烈酒刺激得一阵阵抽痛,但也知道这是最稳妥的医治,见到韩冈过来,照样是一动也不敢动。

看着护工小心翼翼地用干净的细麻布将刘源的伤口重新给裹上,韩冈走上前:“换好药了?”

护工起身点头:“换好了。”病院中事多,不拘常礼,说完他便知情识趣的抱着换下来的旧绷带走开了。

韩冈走到床边,刘源忙着要站起身:“多谢韩机宜救命之恩。”

“躺着吧……”

刘源勉强的抱拳行礼后,才依言躺下。躺下的过程中毫无半点滞碍,可见他背上一处伤口也没有。伤处尽在身前,他战场上的武勇让韩冈也为之敬服。

韩冈不顾血污的坐到榻边,沉默着,最后化为一声长叹:“我对不住你们啊。”

“……早就有准备了。”刘源又挣扎着向韩冈抱拳行礼,低声道:“也得多谢韩机宜。若不是机宜将我等的伤亡报了些虚头,我们这群叛将怕还是要被指使上前去。”

刘源笑着,笑容中毫无暖意:“再折腾几次,王经略可以放心了,天子也可以放心了。”

“倒也不至于会如此。”韩冈摇摇头,抬眼望了一下病房中,被占满的几十张床位,“就算照实数报上去,王经略也不定会再催逼着你们上阵。”

“或许吧。”

三天前,广锐军将校在刘源的率领下,付出了伤亡过半的代价下,将珂诺堡南侧山中的据点一个个的被拔出。广锐将校们的牺牲很大。阵亡五十三,刘源之下,几乎人人带伤,其中一时无法重返战场的有三十二人,韩冈报上去的数字虽然又夸大了些许,但丝毫没有改变这一战的惨烈。

可对于王韶、高遵裕来说,这样的交换再合算不过。这一战,全靠广锐将校拼死杀敌。否则以刘源他们攻克的几座寨堡所据有的地势,普通的宋军战士至少要多填进去千八百人,兵力并不雄厚的熙河军承受不了这样的消耗。而王韶更不想让秦凤和泾原的骄兵悍将成为他手下的主力。

只是躺在病床上的广锐将校,却不喜欢这样的算法,韩冈在病房中走了一圈。他们对每多寰护的韩冈感恩戴德,但言语间对王韶却是压抑着心头的怒意。如果王韶或高遵裕现在走进病房,多半就会起心让他们在阵上死个一干二净。

韩冈走出病房,回头望望,王韶和广锐的这个仇算是结下了。

可换作是韩冈来,他的选择,也当跟王韶一样。最多,就是多上一点辅助,并说些好听话罢了。爱兵如子,没有比吴起做得更好,甚至去.吮吸重病士兵的毒疮,但他的目的,却是为了让他们去死!

韩冈稳步走下台阶。

出兵的第四天,就夺下了珂诺堡,河州决战的第一阶段算是圆满完成,接下来是打通并固守河谷道,于此同时还得看看河州、以及周边诸多势力反应了。

康乐、当川,二寨堡,丢得无话可说。但位置关键的珂诺堡快速陷落,怕是大出木征的意料之外。

刘源一众攻下了外围寨堡,在失去了外围据点的护翼后,当大宋官军踏着晨曦,出现在珂诺堡下的时候,堡中的蕃军已再无抵抗之力。苗授亲自擂鼓助战,一通鼓后,宋军便已攻上城头,两通鼓未落,城门就告失守。等苗授丢下鼓槌,驭马冲进堡中的时候,堡内的守军几乎都逃光了。

而河州的木征尚未来得及援救,“或者说,他无意援救。吐蕃人放弃珂诺堡,放弃得太干脆了。接近五百步的堡垒,两通鼓就陷落,还没动用霹雳砲,怎么想都不对劲,而且还没有缴获到粮食。”

一天之后,韩冈出现在珂诺堡中,在王韶面前,说着自己的疑问,苗授正好不在,他并不用避讳。为了确定兵站的位置,以及接下来的战略,他也需要跟王韶见上一面。

“珂诺堡的城防,在木征手下的这一片地,据说仅次于河州城。貌似连香子城都远远不如。”韩冈看向智缘。

深悉地理的老和尚识趣的接话,“香子城的规模其实比珂诺堡要大,但珂诺堡地处要地,两道合流之处,比起香子城更为关键。所以香子城的寨墙只有一丈出头,而珂诺堡却有两丈之高。”

智缘的话接得漂亮,韩冈感谢的点点头,转头对王韶、高遵裕道:“木征不守珂诺堡,以香子城的城防水准,他大概也不会去防守。也许是准备在河州城下决战!”

“就怕他胆大到连河州都不守,跑到山里去。”高遵裕低头瞅着沙盘,“我们还能一个山头一个山头的去追着他跑?”

“但木征敢放弃河州城吗?”韩冈反问道,“一旦木征放弃了,他在河州周围蕃部中的威信还能剩多少?”

其实这是两头都怕。

王韶、高遵裕怕木征跟他们打起游击,让今次的攻势难以顺利结束。但木征定然也不敢放弃河州。

在民族主义的思潮尚未出现在这个时代的时候——至少韩冈在吐蕃人中并没有看到多少——木征对河州诸蕃部的凝聚力,绝不会有后世的民族国家那般稳固。

一旦作为核心的河州城失陷,宋军就能乘势横扫周边蕃部。可以逼迫原本聚拢在木征身边的诸多蕃部,离开他们的原主,撤回他们应募在木征军中的族人。

如果再能在河州久留,宋军甚至可以驱使刚刚归附的蕃军,去夺取木征核心部众的田地和牧场。没了这些,木征光靠一个赞普血脉的头衔,哪还有现在的号召力。

“而且木征跑进山中后,他又能坚持多久?”

熙州、河州两战,分别选在秋天和春天出兵,并不是没有来由。两战下来,有着稳定后方的宋军还能支撑,但河州的蕃部,就等着饿吧。而且宋军的战马有草料可以补充,但吐蕃人在春天出战的战马却都是瘦骨伶仃。木征组织不起来堪用的骑兵大队,也是今次出兵后,能这般顺利的缘故之一。

“但我们后路怎么办?”高遵裕问道。如果木征决战河州,抄截官军后路是必然,关键就要看能不能守住交通线。

“那就要看景思立和二姚的了。”王韶转向韩冈,微笑道,“还有玉昆。”

……………………

蔡延庆抵达陇西的时候,就从王厚嘴里听说景思立已经率部北上。准备在经过香子城、珂诺堡的支流汇入洮水的北面一点的地方筑堡了。

“令尊呢?”蔡延庆急问道。

“家严正在准备攻打香子城,只是现在正在珂诺堡囤积兵粮,以备万一。”王厚在蔡延庆面前,有一答一,他指着远处一队正准备西去的车队,“这已是第三批了。”

“步步为营,也算是做得稳妥的。”蔡延庆还算满意王韶的行动,指着身后的沈括,他和王厚互相介绍了,又道:“你与存中将事务交割明白,”

王厚点头应诺,目光一转,就落到了沈括身后的一辆碧油小车上。

蔡延庆看到了,代沈括说道:“处道,存中有女眷要安置,你且要安排好,不要惊扰到。”

“女眷?”

沈括竟然带着家眷随行?!王厚心如电转,这是准备在熙河久任了?

如果河州功成,照理来说王韶当要进京,不会在熙河久留,而自己肯定也要随着一起走。下面的官员,别的不说,韩冈早几年就准备考进士的,自然要锁厅。单是三人一去,缘边安抚司的主要官员,就少了近三分之一。他们空下的位置,肯定有人朝思暮想。沈括连家眷都带来,也许他在熙河的位置已经确定了。

只不过这也有些说不过去。除了韩冈这等本地出身的官员,熙河路的文官武将,基本上都是孤身上任,最多在本地纳个妾来服侍,不会将家眷带来,不论是王韶、还是高遵裕都是如此——王厚更多的像是一个得力的助手。

想不通的王厚,直接问着沈括:“熙河战事正急,又无风物可观。为何不将令眷留在秦州,也可安全一些?”

沈括脸色突然变得有些难堪,吞吞吐吐道:“拙荆一向随着在下。”

王厚哦了一声,又问沈括:“敢问中允,令眷,还有令郎、令嫒可有什么要求,下官好吩咐下面的人去措办?”

沈括愣了一下,道:“只有拙荆,沈括今次并未将犬子携来,都留在乡中读书。”

不带儿子,却带浑家,这是什么规矩?王厚弄不清沈括这么做是什么缘故,但看起来有些私人的因素。但他也无意细打听,哈哈笑了两声,遣了得力人手去安排,就此揭过不提。

