\section{第33章 枕惯蹄声梦不惊(九)}

早春的时节。从石岭关往忻州去的谷道中,有草木葱葱,有雪水淙淙,更有山花烂漫,开遍了山涧两岸。

但秀丽静逸的风光,一队接着一队的骑兵却都视而不见。马蹄声踏碎了山间的宁静,肃杀之气充盈在山谷中。

这一段的谷道并不狭窄,甚至可以说宽阔。西面山势高耸是没错,但东面却是宽阔的缓坡。再往前一点,谷地最宽阔的地方本还有一座集镇,以供南来北往的商旅行人在此落脚。而且位置也已经在石岭关后,照常理,自然不可能是有敌军在这一地段进行埋伏。

但这一队三四百人,战马多至上千的辽国骑兵,在行进时,仍有许多人左右环顾,紧张不安甚至让他们胯下的战马都在不安的摇着耳朵。

这一番小心提防,并不是白费功夫。当从西侧的陡坡上猝然而发的几记冷硬短促的弦鸣传来,这群在马背上左右顾盼的骑兵们便及时的反应了过来。

是神臂弓!!

俯身,缩头,夹.紧的双臂保护着腰肋的要害。契丹骑兵们的动作可谓是整齐划一,比他们的队列都要严整得多。

一支弩矢只以刹那之差,在领头的军官背上划空而过,落到了道路的另一侧。如果那名军官不动,箭矢或许就会命中颈项要害,可惜还是差了一步。

但另外两支落下的箭矢却没有瞄准任何人,而是对准了没有骑兵在背上的战马,毫无偏移的没入了队伍中的两匹战马的体内。

两匹战马凄惨的嘶鸣了起来,乱蹦乱跳着打乱了队列的中段。

几名契丹骑兵停了下来,张弓搭箭便向弩矢飞来的方向回射过去。另有两人则很是麻利的扯定受了伤的战马下了官道,远远的避到了一边。其他辽军则不管不顾,费了些许功夫,安抚了受惊的马匹,然后便继续低着头径自往前驱马疾行。

“呸!”远远地看见两名契丹人,手起刀落,给了那两匹战马一个痛快,一名矮壮的汉子在灌木丛中用力吐了一口痰,放下了手中的神臂弓,“直娘贼的,都学乖了。”

“那就再来一下。”在那矮壮汉子旁边,一名更为健硕的汉子坐在地上,脚套着神臂弓最前面的铁环,准备给手中的神臂弓重新上弦,“好歹再多饶几匹战马。”

“算了,用不着。”秦琬咬着根草茎,咧嘴笑着。他护送韩信南下时,尚没收到辽军败退太谷的消息,但在半路上就发现大批的辽军北上。还没等到来自制置使司的命令,就直接从俘虏嘴里得到了最新的军情。从那一刻开始,就一个个变得士气高昂起来。

秦琬此时早把发射过的神臂弓收了起来:“已经耽搁了这一支辽贼一时半刻,不算白费功夫,早些回去才是。”

“这才多一阵?!”矮壮汉子抱怨了一句,但还是依言起了身。

三人都是披挂了一身的蓑衣,放在草木横生的山林中一点也不起眼。起身后,便在山中疾行,很快就赶到了他们拴马的地方。上了马,便绕上了一条细窄的小路。没有惊动任何人的便悄然消失在山林深处。

只是冷箭射敌,就是这段时间以来,秦琬主张的战法。他和韩信将手上的那一支弃暗投明的代州兵领到了忻州西侧山中一处不起眼的废寨中,只从其中挑选出三百有武艺有胆力的精锐,让他们两三人一组穿着黄褐色蓑衣,穿梭在山里。

原本是因为不方便携带铁甲,而不得不用蓑衣补足,但潜入山林后,却与还没有完全发芽生长的草木融为一体。一支两支冷箭虽不起眼,但总能让辽军行军的速度耽搁上片刻。

这一群人潜伏在暗影之处,每每冷不丁的一刺,都让忻州到石岭关,乃至从石岭关到百井寨的辽人日夜难安。

一段时间过后,这一支只以骚扰为能的宋军,都敢潜到辽军的营地附近去射击。要么是绑着油布的火箭,要么是带着鸣镝的响箭,不是去烧粮囤、草垛,就是去惊军营、马圈。

辽军的应对很是乏力,漫长而崎岖的谷地,使得他们追不上,也守不住。尽管还有人打算看着风向点把火,将山烧起来省事。然而不幸的是,春天到了,前几日断断续续、淅淅沥沥的雨水虽然没有大到影响行程,却也使得山上的草木比起湿柴禾还要难以引燃。

说起来辽军的损失并不是很大。补给全靠劫掠,并不怎么需要后方运送粮草,之前又不曾担心过石岭关路的安危,根本就没有经由道路来运送粮草。但日夜难以安寝的折磨,还是在消耗着契丹战士、乃至战马仅存不多的精力。

敌驻我扰。虽然御寇备要上的四条只有这一条做得最好,可也让辽贼吃够了苦头。想必当韩枢密帅大军北上时,就能顺顺当当的取胜了。

秦琬带着些许兴奋和自豪的这么想着。

之前秦琬和韩信策反了降顺辽人的代州军,带着他们逃入了忻州西面的山中。之后又会合了一部分被打散的官军,一部分被辽人毁了家园的乡勇,总数接近五千。他们的形迹几乎可以称得上是流寇,基本上是靠韩信以枢密副使亲信的身份才团结了起来。

秦琬和韩信都不打算动用这一支乌合之众,除了挑选出来的那些精锐外,剩余兵马的作用就只是威慑,只要能安然的在山中留到官军北上的那一刻,必然能对代州的光复起到最大的帮助。

半日之后,秦琬便回到了临时驻留的废寨中。

从寨门处昂然而入,秦琬直接骑马来到正衙庭前。翻身下马,跨步进厅,他却只见一个三十不到的年轻人正大模大样的坐在正厅的最上首,几名指挥使如众星捧月围坐在周围。

秦琬神色一变,手就按到了腰间的刀柄上,他亲自送韩信绕过石岭关南下去见韩冈,离开这里也不过四五日,岂料就有人鸠占鹊巢了。

“你是何人?!”他厉声问道。

那个年轻人却安坐不动。上下打量了秦琬一下:“我是折十六。你就是秦二的儿子?”

听到是府州折家的折克仁,秦琬气势顿时一弱,但又立刻挺直了腰背,沉声道:“是折家的十六将军?”

“正是折克仁。”折克仁反客为主,仍是安坐着。只指了指下首近处的座位,几个指挥使便连忙给秦琬让开了这个位置。

秦琬踌躇了一下,还是坐了下来。折十六已是正八品的大使臣,地位已高,不是他这个未入流品的衙内可比。尤其是折克仁背后还有着麟府折家,更代表着援军,这时候没必要有和意气之争。更何况韩信已经安然回去了,也不怕折家敢吞没自己的功劳。

“你们歇在这个破寨子里面,旁边的山上连个望哨都不见安排。我这个外人报个身份就能进来上座。万一我是辽贼的奸细假冒的怎么办?!还有这位秦二的儿子,竟然就让他这么进了营地,连个通报都没有。”折克仁丝毫不留口德,他带着刻薄的笑意问道,“喂,你们是怎么活到现在的?!”

几个指挥使的脸色全都变了,在秦琬进来前,折克仁尽管极为冷淡,但也还没说过这么刻薄的话。

难道是要争功争权?!

一想到这里,他们便把身子缩了起来。这是绝对不能搅和进去的。折克仁这位折家的嫡系不用说,在河东军中名气已经很大了,但秦琬也跟韩冈的亲信交好,哪边都不好惹。

“十六将军是从西面来,所以没看到望哨。从这里往东去,还有三座营寨,皆临要地,各有数百人据守。辽贼若来,我处必先得报。”秦琬脱下靴子,盘膝而坐,“至于无人通报,却是秦琬进来得急了。”

这不过是撑场面的借口而已,尤其是后一点。蛇无头不行,营中没有一个主心骨,当然是漏洞处处。但这也是秦琬和韩信刻意安排的,不然多了一个头领,万一心怀不轨又该如何?还不如让这几位指挥使互相牵制比较安全。反正这样的情况下,纵使辽军来攻,也不过输得漫山遍野的逃跑,跟现在也没什么区别。

“既然这里已经有四座营寨,数千兵马,秦衙内你却为何坐视辽贼围攻忻州?”

“至少秦琬这里还没人发疯,总数虽有五千人,看似遍布山中,占据了四座军寨。可真正上阵起来,只有被辽人当瓜菜来砍的份。但也不是什么都没做。辽贼来得多了就跑,来得少了就攻,不让其有安生的时候,一切都按着韩枢密吩咐行事。”

“说的也是。蚊子、苍蝇虽然小,却是扰人得很。纵使有信布之勇,也是很难奈何得了它们的。”折克仁尽说着一些不中听的话,投过来的眼神冷淡如冰。

秦琬心中狐疑。若要争功争权,不应当着几个指挥使的面这么说话。不仅是不给自己脸面,也不给那几位指挥使脸面。难道折克仁知道了这几位都是先降了贼,然后才又投了回来的?!所以丝毫不给人脸面。只是秦琬却想不通,折克仁究竟是从哪里听说的,又是为何这么做。

