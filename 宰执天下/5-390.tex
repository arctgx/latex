\section{第34章 为慕升平拟休兵(22)}

战争的关键是什么?

是足兵,足食,民信之。

这是孔老夫子说的话。

放在国家层面也好,放在军队这一级也好,都是颠扑不破的真理。

“现在辽贼和我们都不缺精兵强将,也都不缺精炼的兵器。无论从什么角度来讲,兵力、装备、训练,辽贼都不输给我们。但有一点他们却差了我们很远……他们的粮草运不过来。”

就在辽人因为轨道的运输能力而惊骇莫名的时候,韩冈正在向麾下的一众将校分析敌我双方的优势和劣势。

“大小王庄的存粮,肯定是不够的,之前就是依靠代州的输送。可在加上援军之后,辽贼数量又增加了数倍。如果不能加快从代州运输,对面的辽贼和他们的战马很快就会饿起肚子。”

“而辽贼现在的兵力……”韩冈看了章楶一眼。

章楶会意接口:“辽贼在大小王庄的总兵力,估算是两万到两万五千,至多不会超过三万。至于战马,应是辽贼兵力的两倍以上。”

“就按两万算吧。”韩冈又将话接回来,“那就是之前大小王庄驻军的三倍。而且战马数量也是三倍。也就是说要从代州将粮食运来,就要投入之前三倍份量的供给。战马的食量,连草带粮一并算起来是人的十倍,往少里算也有五六倍。辽贼是两万人,四五万匹马。而我军在此处有近四万兵,万匹军马。辽贼的战马远比我们要多,在争战是个很大的优势,可一旦驻扎一地,需要消耗的粮草就要超过我方。”

之前辽军很好的保证了前线据点的粮草补给——毕竟代州和大小王庄只有四十里的补给线。但能够支持六千骑兵的补给线,在面对两万兵马时,肯定是要大打折扣。就算本来还有些存量,剩余的支撑时间只有之前的三分之一。

“而且最重要的是,从代州到大小王庄的官道是没有修补过的。辽贼不会这门手艺!”韩冈的话惹起一阵轻笑。

从代州出来的官道毁损得越来越严重,可整修道路的人力和技术,辽军手中完全不具备。没有整修过的翻浆道路,无法通行大型的马车。骑兵可以自由的穿越田地和原野,最多也只是因为田垄、沟渠而使得速度稍慢一点,越崎岖的道路越能带来攻击的突然性,但后勤补给却是对道路的要求要尽可能的好,否则就会极大地影响运载量和转运速度。

为前线的军队提供补给,从来都是以千万石来计算。这么大的数量,需要的是海量的人力和畜力。韩冈能凭借大宋的国力,直接打造一条延伸至前线的轨道出来,但辽军却搜罗不到民夫,也不可能学着宋人修筑轨道。

纵然代州和大小王庄之间的距离只有四十里,但经由这段路程为前线的连人带马七八万张嘴来提供粮草,一天两天或许没问题,可再来几天,对战马等畜力的消耗,将会严重削弱辽军的战斗力。这样涸泽而渔的运输,不说辎重队中的牲口能不能吃得消,就是道路也支撑不了。

“而我们有轨道,运力是马车和驮马的十倍,现在一天运送万石石粮草不在话下。”

韩冈再看看黄裳,黄裳跟着说了下去:“其实一天两千石束的粮草已经足够了。”

韩冈笑着道:“料敌从宽,饿肚子也算敌人。往多里算,我们人马合在一起,一天大概可以吃掉一千五六百石米麦,再加上一千多束草料……其实远不如辽贼吃得多。换算成车次,两只手弯弯手指就数过来了。”

韩冈的轻松感染了每一名将校。虽然其中有不少人是一辈子都没打过仗,但粮草补给的问题有多重要,他们还是比外行人要明白得多。

再能打的壮汉饿上几天肚子,也就跟滩烂泥差不多了。辽军一旦断粮,这一仗不用打就赢定了。

但韩冈的心中绝没有他表现出来的这么轻松。简易型的轨道维持性并不是那么的好。就在刚才,众将校还没进来的时候,黄裳正好向他报告过轨道那边又出了点问题。

不过对于此事,韩冈早就有了心理准备,而维护上的准备他做得更加充分。总计八百人的维护队伍,最多半日的时间,中断的交通就会重新恢复。

轨道通过枕木将车辆的重量给分散开来,这远比只有狭窄接触面的车轮更能保证路面的稳定。就像后世出现的履带车,总会比轮式车要更能适应不良的地面。在这样的道路上,纵然还是会有各种各样的问题,比如轨道沉降,比如轨道断裂,甚至由此带来的翻车、出轨,但这一条轨道所维持的补给线,永远都比辽人的后勤保障更加出色。

后勤上的优势,纵然只强过一点,也是对胜利的保障。

“就不知辽贼到底知不知道我们比他们更能耗。要是知道了,辽贼说不定就会立刻逃回代州去了。”

“应该会知道吧。连飞船都能学过去,轨道的名气也不小啊。”

“这可说不准。或许他们还以为我们把道路给修好了。之前也不是没有辽贼的探马在修轨道时探头探脑的,可要是不知道轨道,怎么看都是在修补路面。”

维修翻浆的道路上,木板是少不了的。在没有钢板的时代,在一时间来不及重新夯土垫道的时候,先得用厚木板垫着坑和沟,以便让车马通行。所以在修造轨道时垫上枕木,让不知情的外人来看,也的确只是在修复路面而已。

留着将校们在下面争论辽军进退。韩冈身边的几个幕僚也小声的议论着。

折克仁虽然刚过而立不久,但早在弱冠之前就已经穿越过战场,但他对现在的局面也看不明白,河东这里根本就没有什么大战,胜利便已在眼前。

“真没打过这样的仗。根本还没打就赢了。”

章楶道:“这可是不战而屈人之兵。也可说是善战者无赫赫之功。”

“昨天枢密还说了,因为我们有个好帮手。”

折克仁追问:“谁?”

“自然是大辽的尚父殿下。以辽军的习惯,一旦局势不利,又没有更多的收获,肯定就会打道回府。萧十三不是蠢人,他不会看不出现在的代州形势对他不利,但他留在代州而不退,是为耶律乙辛所迫。”

河东的战局掺杂了太多政治上的因素,使得代州的辽军等于是自缚手脚。一切决策只能以保住代州为前提。

如果双方都能放手施展,韩冈可是半点没把握带着这一批以京营禁军为主力的队伍打赢辽军,最多也只能做到‘礼送出境’。

可惜耶律乙辛太贪心了。韩冈想着,如果能收敛一点,而不是想要依靠从大宋身上割肉来喂饱手下的一群狼,河东的辽军不会陷入这样的窘境。

看起来是熙宁八年的那一次成功,以及乘势瓜分西夏让尚父殿下尝到了甜头,已经不知道什么叫做适可而止。

不知道他能不能及时的醒悟过来,或者说,萧十三和张孝杰有胆子违逆他的心意。

不过困兽已经落进坑里,又怎么能让他们离开。

韩冈抬眼看着帐中的文官武将,在这里的每一个人恐怕都不会答应呢。

……………………

粮草补给不上,最后的结果只会的是全军崩溃。

这是是萧十三以下任何人都不愿看到的局面。可他和张孝杰在帐中对坐相叹,却都没有一个解决的办法。

但就此撤退,也同样很难下定决心。一旦退到了代州,就再没有后退的余地了。而且宋军那时候就可以去攻打繁峙县,夺取飞狐陉的西侧入口,断掉代州与南京道的联系。那样可就真的只剩西京道一路了。

怎么会变成这样的局面?

在一开始,萧十三并不担心宋军会大举来攻。依照之前的计划,代州的援军将会在宋军攻打大小王庄时赶来援救,内外夹击,一举击溃宋人。更可在宋军增兵时,全力出击,击溃甚至消灭半道的援军——摆出这样的架势,来威胁宋军不要选择进攻。

拥有大规模骑兵的辽军可以选择的战术很多,但宋军还是来了,而且是一日一夜之间增兵数万。

如果是正常的增兵,就算没能在半道将援军击溃,就算宋人像现在一样守在营寨内。萧十三在接下来的几天,还是完全可以将主力逐步退回代州,依然让耶律道宁守在大小王庄,以保证大军不会断粮。等到宋军当真出击攻打营地,他再带兵赶回来击溃宋军的攻势——那时候,他有这样的信心。

只是现在,他已经失去了只凭借六千正军守住大小王庄这个前线据点的信心,就是留下一万兵马他都没把握能守住。

宋军能在一日一夜间,通过轨道将数万兵马运抵前线,可以跋涉百里之后一下车就立刻展开攻势,与自己对峙终日,谁能保证宋军不能用一个时辰解决大小王庄的守军?谁能保证宋军在攻打大小王庄时,不能分兵出来阻截甚至伏击援军。一个不好,内外夹击就会变成围点打援。

兵无常势,水无常形,事先谋划得再好,形势一变,一切都将变成无用功。

是决死反击,还是退回代州去?这一来一回,能保证士气吗?他还能控制得住下面的部族军吗?

问题太沉重,萧十三无法回答。

“还是退吧。”不知过了多久,雕像般坐了许久的张孝杰有了声息:“在这里也不是办法。攻宋人营寨肯定是攻不动,留在这里又肯定会断粮,终究还是守不住,还是早些回代州,再想办法。”

萧十三没有反应,垂着眼帘不知在想什么。但手背上的青筋涨了起来,像是

张孝杰盯着萧十三的表情变化,突的想起了一事,脸色陡然变了,跳了起来:“你该不会想着直奔忻口寨吧?!”

萧十三坐着没动,抬了抬眼:“找死吗?”

张孝杰放心了来,擦着冷汗又坐下了。那的确是找死。

谁也不能保证眼前的几万人就是韩冈手上的全班人马。宋人能用一天一夜将数万兵马运到前线,也能用几天的时间,从开封、太原运兵到忻口寨。只要忻口寨留有三五千兵马,守个两三天,等待他们的唯一可能就是全军覆没!

“那该怎么办?”张孝杰问。

萧十三苦笑着摇头,“走吧,迟早要退的。韩冈摆明了就是要等着我们粮尽退兵。也幸好如此,要是他们出战了再走就难了。”

现在士气还在,并非粮草吃尽,一旦宋军匆匆忙忙的追上来,翻盘的机会也就来了。

但两人正在议论,隐隐的从帐外传来了战鼓声。

一名亲将奔进帐中:“枢密、相公,宋军出战了。”
