\section{第36章 万众袭远似火焚(八)}

【今晚就这一更。下一更改在明天下午。】

箭如流星,弦如霹雳,一点寒光自风中掠过,胡千里满意的收下了一声断气前的长声嘶嚎。

“胡四,射得好!”

身后奋战中的袍泽,抽空传来几声叫好。力道超过两石的硬弓,通常能隔着四十步的距离,将敌人射落马下。而胡千里站在营垒的寨墙上,近在咫尺射出的利箭,将一名名吐蕃士兵钉死在地上。

胡千里紧绷着脸,额头上的汗水流得像三伏天的太阳照过,平时的嬉笑就像过了冬天的绵衣,被收藏进了橱柜里。一支支长箭飞过身旁,哪一支都能给只穿着皮甲的他带来重创。但胡千里仍不闪不避的张弓搭箭,稳定的双手将墙内的又一个蕃兵收进箭尖。

瞄准的目标明显的是吐蕃人中极出色的勇士,带着寨中的吐蕃士兵,与翻过寨墙、往寨门冲过去的广锐将校厮杀在一处。七八名广锐将校立抗三倍的敌人,虽然不见下风,但已经被围着难以移动。

胡千里双臂的力道注入长弓之中,喳喳的一阵响,绷得硬挺的弓弦被扯了开来。

下面混战中,那名领队的吐蕃勇士用藤牌硬抗着一记铁鞭,在木屑横飞的当儿,用力挥出了长刀。当的一声金铁交鸣,他面前的宋军被劈得连退了几步,几名宋人聚成的防御圈在这一退中出现了一个破绽。他正待抢上前去,忽地心头一阵发紧,让他猛得抬起头来。

对上目标的双眼,胡千里扣着长箭的右手立刻一松。振颤的弦声尚未停歇,离弦的长箭便没入了那名吐蕃勇士的喉间。

一箭中的,胡千里放松下来。下面被围困的兄弟被这一箭的战果振奋,挥出着刀枪,一下冲散了围困。

胡千里安心的笑了笑,可他身后传来的不是喝彩,而是一声急叫,“胡四,小心!”

急抬眼,两名蕃兵不知何时,竟然已经快冲到了他的身边。

胡千里连忙拉弓搭箭,可不知何时掌心已被汗水打湿,手指一滑,竟然没能勾起弓弦。

‘糟了!’

胡千里心中一声不好,两幅狰狞的笑容已经充满了他视野。

过往的记忆如同走马灯一般,在眼前历历而过,过去的广锐军都头,仿佛听到了勾死人的锁链声。

死亡前的一刻,他的心神却放松下来,几十年军中的生涯,在三千兄弟一起举起叛旗时宣告终结,现在又因故重新上了战场,死在兵戈之中,也不枉这一生的征战了了。

‘早该死了。’

一道雪亮的闪光,如电光般突然飞起。如同浮光掠影,从两名蕃兵的腰间一划而过。轰的一声响,冲到胡千里眼前的两名蕃兵被劈出老远。

而刘源,手提重斧,一脚踩在血泊中,出现在他们原先的位置上。

“小心点!”刘源的脸上带着一点一点喷溅出来的血渍,肋下还插着只剩前半的长箭。

胡千里眨了眨眼睛,重新站直身子,道了声:“谢了!”

简短的对话,在喊杀声中转瞬即逝,可几十年的袍泽兄弟,情谊更加沉淀了下去。

一斧之下,被分为四段的两人,拖着只剩半截的身体,哭喊着翻下城头。青紫色的肠子拖在半空中,断口却在刘源的脚下。

自从开始动手之后,血腥味充斥在鼻中,掩盖了其他的气味。初始时,嗅到血气就直欲呕吐,但现在一看到血红色的液体,刘源就像是莫名刺激到的公牛,兴奋得难以自抑。

刘源抬起脚,坚韧的肠子像绳索一样落地,而在之前片刻,一开始还在尖声嚎叫的蕃人就已经完全失去了生命。

回头对胡千里又道了声小心,内侧也有近一丈高的寨墙,刘源毫不在意的就跳了下去,又稳稳的落在地上。

刚刚站定,周围的吐蕃守军便立刻围了上来。

刘源一声暴喝,重斧带着风声抡圆了一挥,刚刚围上来的吐蕃士兵,便倒飞了出去。在身边清出一片空地,眼睛随之一扫,转了个方向,提着重斧便往营门处杀过去。

踏着血肉,一步步的前进。冲上来的敌军一斧砍断,没有什么能阻挡睁着一双血色双眼的刘源。

“刘疯子!”

墙头上,胡千里不知是骂,还是赞的念了一句,抬手一箭,将追在刘源身后,准备偷袭的蕃贼给钉在了地上。

用着简陋的梯子,攻打城寨的广锐将校一个接着一个翻上寨墙。随着越来越多的将士越墙入内,营地的反击如烟云般消散。

吐蕃人的弓箭并未停歇,但冲上来的敌军武艺强到难以想象,射往要害处的箭矢用兵器给拨开,而不重要的位置干脆用皮甲硬挡下来。

胡千里在城头上长弓连发,心头还在想着,要是有神臂弓就方便了。但以他们如今的乡兵身份,是不得配备军用重弩,尤其是神臂弓,更不可能发到禁军以外的士兵手中,只能靠着手上的硬弓。

刘源终于控制了寨门,吱呀声中,大门中开。被堵在营垒外的宋军全数冲进了营中。吐蕃人最后的顽抗瞬间被瓦解。片刻之后,宋人的大旗已经在城头上飘扬。

“这是第三座了!……下面还有五座,吐蕃人在这里布置的堡子还真他娘的多。”一名相熟的兄弟摊着手脚躺在胡千里身边,直着喘气,许久也不肯站起身来。

胡千里也在女墙上坐了下来,“谁让珂诺堡位置好!”

珂诺堡地扼两路,不论是河谷道,还是山道,想从熙州的狄道城往河州去,都必须经过珂诺堡。比起位于洮西的康乐寨和当川堡,从城防还是驻军,都强出了十倍。

刘源他们攻打的是珂诺堡外围的一处据点,占据了山势的地利,两百多守军压制着准备出山,进入河谷的宋军。珂诺堡近在眼前,但如果不能攻下珂诺堡周边的七八处据点,离着目标的最后两里,就如同行走在死亡线上。

天色渐渐的暗了下去,身边的兄弟不知何时都睡过去了,胡千里则还做在城头上,低头保养着他惯用的爱弓。肩膀突的重重地被拍了一下,抬头看时,却是刘源。

刘源在胡千里身边坐下,手上的大斧不知放到了哪里去了。他看看胡千里手上的硬弓,笑道:“听说王舜臣那个毛头小子,传说他的连珠箭术能压着一片墙的贼人。名头都快盖过刘昌祚那个神箭了。什么时候跟他比比看?”

“算咯。”胡千里摇摇头,紧了紧弓弦,“就算箭术胜了又如何,人家的官运没得比啊!说是毛头小子,可也是一路都巡检了。刘指挥你当年的官运都比不上他。”

“谁的官运能比得上,才二十出头吧……爷爷这个年纪,连去买笑的粉头钱都没有。”刘源骂了一声,朝着营外用力啐了一口,“跟了好人家了。”

“那是沾了河湟开边的光。韩机宜不也是才二十,就成了朝官吗?没军功,熬上一辈子,能熬上一个都巡检那都是祖宗牌位上烧高香了。”

胡千里叹了口气,收起弓。看着刘源的脸色,觉得有些不对。想想,问道::“走了几个兄弟?”

“十七个,还有两个怕是等不到回狄道了。”说起自家兄弟的伤亡,刘源脸色郁郁,“其他二十四个已经给包扎了伤口,等到了狄道基本上就能救回来。”

“攻城拔寨,损伤在所难免。”胡千里早看开了,没死是好事,死了也就罢了,“反正性命都是白捡回来的,也别想太多了。歇着去吧。”

“歇什么?要我们一鼓作气啊!”

“还要打?!”胡千里平平淡淡的口气维持不住了,“都天黑了!”

“夜战。”刘源叹了口气,“韩机宜为我们争辩了两句,就被赶回了狄道。王经略、高总管可是盼着我们跟吐蕃人两败俱亡啊!”

胡千里呵呵冷笑起来,“那我们就把珂诺堡也打下来,总不能让他们如愿!”

“珂诺堡我们没份,那是官军的。只要最后的一座山口营垒攻下,就没我们的事了。”

“还有五座吧?”

“只剩一座了!”

香子城是河州城的门户,而珂诺堡是河州的门户。在连续丢掉了三座驻防高地的营寨,吐蕃人一下放弃了接下来的四座城寨。将里面的兵力都集中到紧守山口出路的那一座营垒中。

只要过去了这座营垒,就只剩河谷中孤零零的一座珂诺堡。

“援军怎么办?”胡千里问着,“吐蕃人不会眼睁睁的看着我们攻打营寨!”

“王舜臣会带人堵着珂诺堡过来的援军,山口的营垒归我们管。”

胡千里正要说话,忽然闻到一股浓烈的酒香,一眼就看到刘源手中拿着一个黄皮的小葫芦,“是酒?!”他惊喜的问道到。

“烧刀子!上次韩机宜赏的。是疗养院中用的药酒,外面根本弄不到。”刘源晃着葫芦,酒香四溢,转眼间,就引过来一群流着馋涎的酒徒。

“拿碗来,”刘源一拍葫芦,“兑着喝吧。”

将一小葫芦的烈酒,分给了众多的兄弟。出动的命令也到了。只剩两百人的广锐将校聚在山道上,望着远处的山口。

“胡四!”

胡千里闻声回头,刘源指了指自己的左臂,上面绑着一圈白色的布带,在月色下很是醒目,

胡千里侧头看看自己的左臂,一声失笑:“啊,忘了。”

掏出发下来的白布条,在上臂处牢牢的缠上几圈。他一提长弓,对刘源道:“久等了!”

刘源抬眼望去月色下的山口,那一座只有百步上下的堡垒莹莹的反射着月光,他冷笑:“对,别让主人久候!”

