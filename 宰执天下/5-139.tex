\section{第15章 自是功成藏剑履(一)}

来犯的辽军败退,当捷报传到韩冈这里的时候,已经是第二天的黄昏。

王安石的新书早就收了起来,韩冈对于训诂学的兴趣,远远比不上近在眼前的胜利。

尽管在预备方案中也做了与辽人决战的准备,但不论从哪个角度来说,韩冈都没有在此时挑起宋辽大战的念头。将战斗的规模局限在略略扩大的边境冲突上,最为符合韩冈的利益。

“总算是了结了。”

纵然在外一点也没有表现出来,但为了这一场稳定河东的胜利,韩冈一个多月来殚思竭虑,付出的辛劳远远超过任何人,只是不足为外人道。一声长叹,却是放松下来的感慨。

“这可是澶渊之盟后,对契丹战果最多的一次。一战斩首辽军千级,几十年来何曾有过?!”黄裳却兴奋得坐不住,“要是那些萧十三派来的内应也能算,可就是两千人了!”

“也算不得什么。”韩冈笑着摇头,“正剧早完场了,这收尾的杂扮却拖到现在,无论放在京城的哪个瓦子里,早就一片声的喝倒彩了。”

“也不能说是杂扮吧。”黄裳倒是叫着屈。

他在京城中,也曾在各个瓦子里的勾栏看过百戏。一场戏,正杂剧是正篇,到了尾声则是上演一出插科打诨的杂扮。无论如何,他都不觉得韩冈率领河东军立下的功劳,会比种谔、王中正等人稍差。这可是对上辽人,依然稳稳的占着上风!换做是其他武将文帅,有谁能拍着自家的胸脯说,做得能比韩冈更好?

韩冈悠悠然的重又看起折克仁和折可大的捷报,从龙飞凤舞的字体中可以看得出来,撰写捷报的人,恐怕在动笔的时候,也是兴奋得难以克制了。

“经此一败,辽人不会再来了吧?”黄裳稍稍冷静下来后,不放心的又问韩冈。

“谁知道呢?要是萧十三发疯那可就难办了。”韩冈轻松淡定的笑容,让他的话像是在开玩笑。

黄裳陪着笑了两声,又道:“是不是派人去通知李都知回来?他在浊轮砦,当是还没有收到消息。”

李宪是在开战后,赶去浊轮川边的浊轮砦的。坐在丰州城,他始终做不到韩冈一般的闲适,最后向韩冈请命,去浊轮砦镇守,以防暖泉峰有失。

东胜州的辽军南下两条路,暖泉峰、浊轮砦一路,是李宪的河东军,柳发川、唐龙镇,是折家的麟府军。不过才他走一天,捷报就传回来了。计算脚程,这时候他应该才进寨。

“是得派人去。”韩冈点点头,“不过让他再在浊轮砦待上一阵吧,不让他守着暖泉峰的后路,恐怕也不能安心。勉仲,你帮我起草关报,经此一败,辽人或许还会反扑,左右四邻都要通知到,以防万一。”

黄裳兴高采烈的点头应声,韩冈要将柳发川的大捷传出去,哪有不愿意的道理。提起了笔,边写边问:“朝廷那边呢?”

“等派人确定了斩首数再说。这一道手续不能少。”韩冈看着黄裳提笔就写,不愧是福建才俊,文采可比陕西的士子强得多。

“辽人奸细煽惑,黑山余孽作乱……”韩冈抿着双唇,笑得意味深长,“这一下当不会有人再说嘴了。”

“啊?”黄裳疑惑停下笔。

“没什么。”韩冈笑道,“只是之前的两万斩首的确太惹眼了。”

……………………

“两万三千斩首!亏韩冈敢向朝廷说!就是种谔趁西贼内乱之际,灭了西夏的最后一部兵马,斩首也不过两万两千余!”

“河东军所斩党项,尽为黑山河间地的逃人,意欲归附中国。所以河东军能不伤分毫,便有数万首级。那是乘人不备。黑山逃人如何能想到,本因收留他们的官军会痛下杀手?”

“杀良冒功之罪,岂可轻恕?百禄昨日就听说,御史台那边要上本弹劾韩冈擅兴和杀良之罪。枢密备位西府,岂可默然不言?”

吕公著知道,在朝堂上有许多人都与在自己面前慷慨陈词的范百禄一样,认为韩冈在河东的所作所为,其实是在挑起宋辽两国之间的纷争。是以边疆的安定为赌注,为自己的官位鸣锣开道。就像是徐禧一般,以私心坏国事。

不过吕公著并不认为韩冈是这样的人。了解多了,旧时的偏见也少了一些。在他的了解中,至少韩冈之前的表现,一向是以国事为重。而重夺旧丰州,也是缩短疆界防线,以瀚海天堑为界,保全内地的良策,并非是韩冈好大喜功之故。

“……听说子功旧年随熊本平泸蛮,夷酋领众归降,有裨将欲杀之,是子功劝阻下来的?”

“些许小事,不足当枢密垂问。”范百禄是当年在王安石刚刚秉政时,便痛骂其十项大罪的范镇的侄儿。他曾随熊本平定西南夷,一向主张招抚、缓攻,用文臣治边,善待夷人。他对吕公著道:“杀降不祥,活千人者封子孙。韩冈如今屠戮归降蕃人以为己功,满手血腥,不知日后说起圣人仁恕之道,他愧与不愧?”

“韩冈不是贪功的人。他要是想贪那份功劳,当年就不会拒了撤离罗兀城和平叛广锐军两次功赏了。广锐军的性命,也是他保下来的。”吕公著猜测着韩冈如此上报的原因,“他是被下面的那群赤佬给裹挟了。李宪、折克行岂是那等会放过功劳不要,以国事为重的纯臣?”

“下面的骄兵悍将就该杀两个以儆效尤,哪有任其摆布的道理?!”范百禄厉声道:“若如枢密所言,韩冈更是有负圣恩。擅兴好杀犹不失一方名臣,可若是为僚属裹挟,那可就是无能至极。”

“莫要求全责备。韩冈尚不及而立,弹压不住也不足为奇。药王弟子的名声虽响亮,可德望还远没有养成。治政尚可,但统领一路兵马还是差了一筹。”吕公著叹道,“说起来镇守河东,还是韩冈第一次统领一路,掌管一方边事。之前有章惇,再之前有王韶,在广西和熙河,有他们两人掌控大局,韩冈的性子才没有闹出大错来。这一次独领一路,的确是做得错了。”

听到说起章惇,范百禄冷哼道:“章惇一向好兴兵,故与韩冈亲厚。韩冈的奏章肯定也看到了,这一次,看他如何为韩冈辩解!”

……………………

章惇正在看着韩冈的奏报,脑仁也是一阵阵的抽痛。

河东军的两万斩首实在是太过火了。前两天,河东奏闻说有了一万斩首,他就已经觉得不对劲了,只是认为韩冈会见好就收,也就没有去写信,谁想到到了今天,就变成了两万三千。这未免太骇人听闻,竟把党项人当成南面的交趾人一般。

种谔也是两万,可当时是西夏军内乱,又没有投诚大宋,种谔领军乘机掩杀,尚能说的过去。可这一批南下的党项人可都是意欲归附的逃人,好生抚慰安置还来不及,怎么就能让人杀了换功劳?

章惇也觉得韩冈做得过头了。他知道韩冈的手段和为人,要说他镇不住下面骄兵悍将,那是笑话。韩冈对异族的杀性,章惇可是在南征时,便了解甚深了。

出身陕西的韩冈,对党项人有着根深蒂固的不信任,乃是人之常情。西夏惯于背盟,大宋不知吃过多少亏。西夏的孑遗,死光了天子还能多睡个好觉。

韩冈将杀了泰半黑山党项,对于河东的长治久安是最佳的策略。但没必要将自己也陷进去吧,一旦黑山党项中有人聪明到入京敲登闻鼓,一切可都是韩冈的责任了。

不过现在章惇找不到人商量。

徐禧之事,一因其殉国,一因盐州城破后,西夏随即生变,使得他之前的守盐州策略不算过错。但吕惠卿依然自身难保,毕竟京营禁军在盐州死伤太重,总得有人出来负责,以安人心。

京营诸军在东京城中驻扎了百余年,许多偏裨将佐,都能与宗室、皇亲扯起千丝万缕的关系。盐州兵败所掀起的动荡,光是将逃离盐州的曲珍下狱,可是远远不够抵偿京营禁军家属们的愤怒。王珪和吕惠卿两人中,肯定要有一人被牺牲,甚至两人。

御史台要挑头攻击韩冈的消息,章惇已经可以确定。毕竟他这个罪名不容易洗。最关键的,只要让天子留下韩冈无法镇服麾下将领的印象,韩冈想要晋身两府,少说得往后拖上十年。

三十不到便望执政,成为众矢之的也不足为奇。

章惇正为韩冈担心,想着该怎么在天子面前为其缓颊,就听见门外有人喊:“枢密,枢密,太皇太后上仙了。”

太皇太后上仙?章惇愣了一下后才反应过来。苟延残喘到了今天,曹太皇终于走了?

这可不是好消息,没有太多的感慨,只有利弊的估量。

章惇只觉得心头又给压上了一块巨石,从今往后,宫中最尊贵的便是那个左心牛性的高太后了。同是反对新法,曹太皇可比高太后要知道轻重。

当真是祸不单行!

