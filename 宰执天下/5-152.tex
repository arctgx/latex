\section{第16章 晚来谁复鸣鞭梢(中)}

太原的暮春初夏,算是一年中最好的时节。风和日丽,气候不冷不热,雨水又不缺,将西面吹来的沙尘洗去。

却也意味着一年中最忙碌也最重要的时间就快要到了。

城外的麦子开始灌浆,再过半月就能收割了。而且这几日,连着几天都是白天放晴,夜中落雨,太原府中,从官员到农民都是欢欣鼓舞。灌浆期的天候如此之好,基本上可以确定,今年必然是个大丰收。

韩冈早早的就提前着手,将充实常平仓的钱物准备好,准备收购民间存粮。并联络路中仓司,让他们也早一步做好准备。

去年的一场大战,将河东常平仓中的多年积存吃掉了大半,使得今年的青苗贷发放数量都只有前两年的四成。正如他之前跟韩缜所说,就算不去计较谷贱伤农的问题,光是为了

韩缜已经回京一个多月。成为新任的参知政事,也有一个月了。政事堂中的人事,在这个春天变化得飞快。元绛出外,韩缜入内,此外,就是吕惠卿也终于离开了朝堂。

吕惠卿的去职,可以说是御史台的雪耻之战,吕惠卿的弟弟吕和卿在外置田时让当地县官为其做保,被御史捉到了把柄,称其借势欺压良善。说起来,这可能是陷阱,但给人抓住了,吕惠卿也只能是百口莫辩。

自然,这个罪名并不算大,天子如果想保的话,吕惠卿本人不会受到影响,吕和卿最多也只是罚俸赎铜。但若是天子无意留人,小事也能变成大罪名。

折可适送来了最新的朝报,上面最为重要的一条便是吕惠卿出任京兆府知府兼永兴军路经略使。

“吕吉甫出知京兆府?”韩冈本以为吕惠卿就是卸职,也会在东面或南面安身,没想到会给打发到西北来,“这一回算是做邻居了。”

“就不知吕大参愿不愿意做邻居?”折可适道,“福建人可难吃得陕西的苦。”

“引罪出外,可没有愿不愿意的一说。”韩冈抬头对折可适笑道,“后面一句可别在勉仲的面前说。”说着,韩冈又转头看看外厅,却不见黄裳,诧异的问道:“勉仲人呢?刚才就没看到他了。”

“方才今科太原府的几个新进士跨马从前街过,就见勉仲出去看了。”

“哦。”韩冈的眉头略略皱了起来。因为去年对夏、对辽的战争的缘故,黄裳无缘科举,看到太原府的新科进士回来游街夸耀,心情应该不会太好。

韩冈叹了口气:“勉仲的这一科是我耽搁他的,以他的才学,只要时运到了,一甲不好说,二甲前列绝对没有问题。下一科又要三年后,勉仲可不能再耽搁。”

“龙图何出此言?”黄裳正好跨步进门,听到韩冈的话,“学生一向水星不利,即便今年上京应考,也不一定有金榜题名的运气,更比不上跟着龙图,增长了学识,开阔了眼界,又有了用兵的经验,而且还得了官。这如何是科举能比得上?就是一榜进士,十年时间,也不见得能五削圆满。而学生附龙图骥尾,一年便已是京官,这些可都是龙图给学生的。”

黄裳的话发自肺腑。去年他辅佐韩冈主持大小战事,解试的时候都在胜州前线度过,连个贡生的资格都没拿到,当然不可能上京考试,只能准备三年后下一科的科举。不过可能是出于对于韩冈的补偿,他举荐的幕僚,朝廷都没有吝于封赏。黄裳在葭芦川大捷之后,因功入官。而在胜州大捷后,又因功加赠,眼下已经脱离选海,成了一名京官。只要三年后,能到了一个进士的资格,那么摆在黄裳面前的,便是一条金光灿烂的通衢大道。

韩冈摇摇头,“我为国荐才。因为勉仲你有其功,有其才,非是论人情。”

黄裳躬了躬身,谢过韩冈的赞许。坐下来又道:“方才学生在查对上个月的各处驿站报上来的账籍,发现来自代州的马递比前几个月多了许多,翻了一番。似乎有些不对劲。”

折可适误传了黄裳的行踪,正有些脸红,但听到黄裳的话,神色郑重起来,“代州的崔象先是两个月前上任的吧?是不是出来前奉了什么密诏,一个月时间,上下都掌握住了,就跟京城通起了消息。”

韩冈点了点头。边境军州的知州,本就有权直通京城。刘舜卿已经给调走了,新任知州来自京城。自他上任后,驿马使用如此之多,想来代州那里就有些秉承天子密旨的小动作。

“看起来,天子没打算耽搁太多时间。”黄裳在经略司中有了一年多的经验,很轻易的便看了出来。

“西夏都灭了,下一步当然是辽国了。”折可适道,“据说天子念兹在兹的,便是收复燕云。澶渊之盟,说不准几年后就会给废了。”

“灭辽?”韩冈闻言就笑了一声,“哪有那么容易!”

天子赵顼前段时间曾经遣使征询过韩冈的意见,该如何对付如今权臣当道的辽国。而且同样的问题,天子应该询问过了不少朝臣,乃至元老。至少上个月王安石写来的信中,便提及了此事。

韩冈在给赵顼的回复中,并没有说多少对辽国的战略规划。而是说了一些厚植国力的建议。

“自种谔取绥德,韩绛攻横山,西夏自此由攻转守,开始衰弱。但到其灭亡,中间经过了十年的时间。而且西夏覆亡之前,其国中母子相争,本就是自取其败。同时宋夏两国国力相差极大,这是大宋能将之不断消磨,乃至耗尽元气的主因。如果换作是辽国作为敌人,以眼下的国力、军力和人心,能做到几分?之前或许小胜过几次,但改以灭国为目的,那可就是两回事了。”

韩冈之前给天子回复,是亲自动笔,并没有让黄裳、折可适等幕僚知晓此事,不过他这番话中的观点,则不只一次提起过。

“越王勾践败退于会稽之上,十年生聚十年教训,国力复振。吴王夫差北上黄池之会,勾践发兵偷袭吴国,就是这样,也没有一举灭吴。逼夫差自尽,尚在十年后。昔有高句丽立文法,隋唐接连征伐,隋炀帝、唐太宗领军亲征皆不果而还。直至高宗时,耗尽其国力,方灭其国。”韩冈靠着椅背,“要想灭掉一个文法已立、根基深厚的国家,可不是打上一仗两仗就能做到的。”

一旦边境上的蛮族从部落联盟转变成一个制度井然的国家。为什么隋唐打高句丽比打突厥还难?地理是一方面,更重要的是,对手的类型不同。

当年王韶能推动朝廷同意攻取河湟,其中一条,就是上报说董毡、木征叔侄已经准备立文法,有立国的打算。

韩冈旧年曾经在赵顼那里听他说起辽夏,‘二敌之势所以难制者,有城国,有行国。自古外裔能行而已,今兼中国之所有,比之汉、唐尤强盛也。’

“要灭辽国,需有耐心,得做好前后几十战,绵延十数年的准备。眼下面对西夏都没能见全功,何论辽人?”

折可适沉吟道:“也就是说,要积蓄国力,等候时机了?”

“没错,先为不可胜,以待敌之可胜。慢慢来。”这就是韩冈对天子的回答。

先把国家、军队都打理好,然后再去想辽国的事。

赵顼论见识,当然是不差的。但这个皇帝想来性急,谁也说不准他什么时候会犯了老毛病。

韩冈之前的建议是军队在将兵法的基础上加强训练,武学的学生应该大力收录在军中有过功勋的底层将校,就像进士资格是通往朝堂最高层的通信证。在武学学习过,得到武进士的资格,也应该成为晋身之阶。强兵是打出来的,但一定水准的军队,可以通过训练得来。

真正到了国战之中,拼的就是消耗,人命、财富,如何能对拼得起这样的消耗,就是未来胜利的关键。有一定素质的军官,经过训练的士兵,历练一下就是一支强兵,比起通过在战场上优胜劣汰,要节省许多成本。

对于国政,韩冈则认为应该大力推广铁制农具,以略高于成本价的价格向农民推广,甚至可以作为青苗贷的一个组成部分,加入其中。

一方面,由于禁军已经全数换装完毕,因铁甲、钢刀等兵器,对钢铁的需求因而下降了许多,需要有一个新的途径来保证钢铁业不会萎缩——这是韩冈一直以来坚持的观点,民用比军用更重要——另一方面,铁制农具对农业生产的促进,有着极大的意义。

使用木质农具的农家,不论在广西还是在河东,占得比例极大。从效率上说,木质的农具远不如铁器。由此在农田中浪费的时间和人力,让韩冈觉得十分惋惜。如果省下这些时间,可以让农民打些零工,或是做些能赚钱的营生,对普通农户的家计有着很大的好处。

