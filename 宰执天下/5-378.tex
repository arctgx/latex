\section{第34章 为慕升平拟休兵(十)}

让韩冈及他的幕府头疼的消息,就是让萧十三和张孝杰欣喜的捷报。

之前可用的战力大部分由于消耗太大不得不休整,剩下的还要去朔州防备占据武州的宋军抄截后路,不得不让宋军的骑兵很是嚣张了一阵。尽管宋军若敢直扑代州城下,萧十三依然有足够的力量反制,可面对拿着神臂弓的宋国骑兵,一时间他还是缺乏有效的应对手段。

不过自从将麾下恢复战斗力的六千多精锐派出去,驻扎在代州西南四十里外要道旁的大王庄和小王庄之后,嚣张的宋军探马便在代州城外消失得无影无踪。

这很像是宋军的故伎。并不求交战克敌,只要将威胁性足够高的军队驻扎在关键的要点上,对敌军就是巨大的威胁。现在辽宋两军,在代州城西南的大王庄到崞县县城这一线上相互钳制住,甚至难解难分。

随着萧十三更进一步将神臂弓分派下去,双方的斥候之间的战斗也越发的惨烈起来。远探拦子马的伤亡报告让萧十三有几天甚至吃不下饭,而宋人那边,虽然并不清楚具体的数字,但萧十三可以确定,宋军的损失只会比己方更多。直如象戏中的兑子,既然双方的伤亡数字都是居高不下,相对而言,当然是骑兵更多的一方更经受得起损失。

一旦宋军没有足够的骑兵来保护行进中的步兵,那么他们想要向代州前进一步,都要冒着之前十倍的风险。

“西军怎么办?算算时间差不多也该来了。”张孝杰问着萧十三,解决了眼前的问题,未来的隐忧也就浮上台面。

韩冈的经历,在辽国上层不是秘密。

这名年轻的枢密副使必然是大辽未来数十年的大患,但对许多王公贵族来说,却又是救人性命的神佛一流的人物,轻易开罪不得。之前萧十三去攻韩冈所在的太谷县,也是大着胆子去的。灰头土脸的结果,更是让很多参加奔袭作战的将领们,都在暗地里抱怨萧十三把人往火坑里领。

以韩冈在陕西军中的威望,西军在他手下能发挥出十成十的本事。那样的敌人,不是赢不了,而是赢了也是得不偿失,百分百的折本买卖。

“别自己吓自己。等到了再说。只凭现在从开封出来的那群废物,也没什么好担心的。”萧十三说着自己都难以相信的话。

萧十三说京营禁军是废物,但他也不敢当真把忻口寨的那几万人当废物看待。

对于韩冈麾下军队的成分,交手了近两个月,捉到的活口也有百八十,萧十三差不多已经可以自称是了如指掌。

到现在为止,韩冈手上的主力却是立功最少的,甚至连出头的机会都没有。夺取武州的是河外的折家。现在造成正军骑兵最大损失的对手,也是出自河东军。

但这并不代表那几万京营禁军当真是百无一用的废物。即便是废物,拿起神臂弓,手持斩马刀,身披铁板甲,也是会让人崩了牙齿的废物。更别说率领那一支军队的主帅,不是那么简单就能击败的对象。有他领兵,狗也能变成狼。

萧十三不敢冒风险,让自家的精兵去往宋军的箭阵上撞,万一那是一块硬骨头,崩掉了牙后怎么办?不提韩冈,神武县可就是有头大虫守在那里呢!尚父殿下更难以容忍麾下的精锐受到无谓的损失。要是由此得罪了麾下的将领和贵胄,说不定尚父都要牺牲他来化解人们的愤怒。

何况就像张孝杰说的,韩冈最熟悉的关西禁军,这时候应该到了。虽然不知现在何处,但只要出现,韩冈能玩弄的花样就能多出三五倍。带来的压力,能让人喘不过气来。

不过萧十三还是有办法应对的。最好的解决手段,就是打破眼下僵局。如果现在代州的僵持局面能够打破,对还没有眉目的和谈,将会有着决定性的意义。

“占了武州的党项蛮子已经放得太久了,该让他们明白契丹精兵的威名不是凭空而来。”

……………………

“终于可以那些家伙送走了。”

收到韩冈亲笔签字画押的公文后,折克行长舒了一口气。多达两万余的阻卜人,只要他们存在于神武县,给折克行的压力就是如山一般巨大。

“枢密使怎么说的?”他问着与公文一起被派回来的儿子,“是打算要那些阻卜人上阵吗?”

折可大当然不会向他的父亲隐瞒什么:“这些日子损失的斥候数量不在少数,急需一批新人来补充。”

“所以就看上了那群阻卜人。”折克行摇摇头,却不再就此事发表意见,而是问起损失的骑兵情况:“怎么样?他们是不是都是代州出身?”

“也有太原的。”折可大回道:“所以现在京营的那几个指挥也被拉上去了。”

折克行的神色立刻难看起来,“金明池上那些耍百戏的能派几分用场?”

京营禁军,说是耍百戏,那还真是没错。比起马背上的杂耍来,这世上还真没什么人能比得过他。折可大听说过,若不是禁军都受了身份限制,否则赛马大会上的表演,也用不向远在沧州的在胜州的黑山党项中挖人了。

并不是说这些阻卜人存在没有好处,他们的反叛让萧十三根本不敢再调动异族来作战,反而还得分兵盯着几家不听话的。

可是两万多阻卜人停在武州,这让只有八千儿郎的折克行夜不能寐。而且粮草也是到了多少就吃多少,积攒不下来。万一粮道被断,就只有立刻突围一条路了。

“为父几次三番的要把这些阻卜人都调走,都被枢密将申请给压了下来。”折克行叹了一口气。

折可大立刻道:“但之前的情况的确不能让外人来。”

折克行也知道忻口寨当时的确很不稳定,有百姓、有败兵,还有随时可能出现的辽贼,再多了一群阻卜人,一点小冲突都能变成难以收拾的大乱子。

不过现在一山之隔的忻口寨,在韩冈的布置下,早已经稳定了下来。并将周围的城镇乡村都控制住了,同时一干乱源也被摘除。十几家阻卜部族人数虽众,可南下之后即便想要发难,也立刻就会被镇压下来。

十几家部族的族长被招到了折克行的帐下,躬身静听着这位主帅的训话。

折克行的训话出乎族长们的意料,“韩枢密有意攻打代州,只是帐下缺人听命。这可是难得机会,尔等究竟能不能在大宋站稳脚跟,就看这一回能不能让韩枢密满意了。”

“可是能亲眼看到韩枢密?!”

“当然。那还能有假?而且不仅仅是韩枢密,立了功后甚至还能去拜见天子。那可不是辽国的皇帝,只收岁贡,没有回赐。大宋的皇帝不一样,只要你朝贡入境,送上贡品之后,就能得到一份丰厚的回赐,作为酬奖。”

这些消息让族长们都开始感到兴奋,身子也忍不住的颤抖。

“朝廷的银绢够打发他们这些人吗?”折可大问着堂兄弟折可适。

他可见过不少部族,把朝廷的恩典当成了做买卖的好处。献上去的土产,很快就有了回报,回赐很是丰厚,远比看到的更多。国库也由此而损失不小。但就在此时,折克行竟然用这样的办法来吸引阻卜人更加卖力一点。

折可适哈哈大笑:“足够了。必要时,朝廷也会稍稍吝啬一点的。吃一堑长一智,总要让他们牢记朝廷的恩德不是他们敛财的工具。”

将一干阻卜族长打发了出去,折克行转回来对儿子和侄儿解释道,“此辈畏威不怀德,若是松上一星半点,说不定还让他们以为我等好欺。”

“不过韩枢密的名号在辽国也是传遍东西南北,想必他们也是听说过的,定然不敢违命。”

“这几人中之前就有两个拜见过枢密吧?好像当时就向枢密请求给他们族中子弟种痘。”

种痘法在阻卜人那里,契丹人为了收拢异族人心,也把种痘法当成了恩赐,这几年在各家部族的族酋带着子女上京后,都会为他们种痘。

虽说契丹人绝不会去宣扬韩冈,但各个部族又不是聋子瞎子,韩冈这位药师王佛座下弟子的名号早就传遍了草原。所以拜见韩冈时,几位族长顶礼膜拜不说,提出的请求中就有一条是为族中子弟种痘。

“枢密怎么回他们的?”折克行问道。

“只要听话,立下功勋后,朝廷自不会吝惜赏赐。”

折克行双手一拍:“说得好。这样才对!哪有平白无故就把好处先送了人。”

“嗯。”折可大点头,“骨头只有在狗叼回猎物后才能赏出去,从来就没有先给赏、后使唤的道理。”

听到儿子的比喻,折克行也忍不住微微一笑。

有种痘在前面吊着,再加上韩冈的赫赫声威,相信这一干阻卜人绝不敢对韩冈的吩咐阳奉阴违。

而自己手中有八千兵马,足以守住神武县。周边的几处要点都占了下来,以此为基础,北上攻打朔州,进而攻击太原,都是可行性很高的方案。
