\section{第36章 沧浪歌罢濯尘缨(30)}

冷面的份量不少,但几人久在军旅,都吃得很快,转眼就见了碗底。

韩冈没有打算点第二份,吃完冷淘,开店的老汉又端上了热茶水来供几人消食。

茶水的味道比刷锅水好点,或者说根本就是添了点碎茶末的刷锅水。韩冈出身低,不怎么讲究。但他喝得下所谓的消食茶,折可大却几乎咽不下。

折可大自幼锦衣玉食,哪里喝过刷锅水?只是看到韩冈毫不在意,也只能小口小口的抿着喝。

不过他抿了两口就放了下来,对韩冈道:“之前枢密写给家严的信已经收到了,只是地方军政,家严不敢妄言,所以才回了那封信。但家严在信上也说了,此事有枢密主持,定然能顺利完成。”

“令尊就是太谨慎。”韩冈摇摇头。他之前写信给折克行,说的是神武军的事。

他希望此番来援的西军,能全数移镇神武军。至少朝廷也要给出能让西军将士中的大多数愿意留在神武军的待遇。这样才能将神武军这个战略要地给支撑起来。所以他才会推荐白玉出知神武军。

不过神武军是折克行打下来的。虽然之后的政事安排依然与折家无关,但从人情上说,韩冈也需要事先向折克行提一句。这是韩冈有别于其他文臣的地方。

“这不是有位子……”

“哥哥,这里能坐得下。”

几个大嗓门忽然在门口处响起。

正说着话,却给人打断,韩冈心中略感不快,抬眼望过去。

只见几名士兵正招呼着往店内进来,其中两人手上各提着一个酒坛,看起来是要到这里借桌子坐。

刚刚光复不久的雁门县,店铺少,连桌位都少,本来就不够用。可韩冈出来逛街,他身边的亲卫将几张桌椅全都占了。不仅占了这一间,对面和旁边的店面都占了下来。几个士兵看到这一家客满,桌子却不满。韩冈三人倒也罢了,另外三人却占了两张桌,想借张桌子坐一下也正常。

“兄弟,让一让。借张桌子喝酒。”

“你这几个汉子好不晓事,才三个人就占了两张桌。让让!让让!”

几名士兵进来也没理会摆摊卖冷淘的老汉,直接就撵起了人。

闲下来的士兵就是祸害。一个个都是血气方刚的汉子,平常给拘在营里不得自由,一出来当然是满街撒欢。幸亏街面上也没多少百姓,也没给他们祸害的地方,闹也只是军中闹,让韩冈省了不少的心。只是没想到今天给他自己撞上了。

普通的平民男子被称为汉子,其中还带着些许贬义。就如好汉一词,虽不是后世的匪气,可也没有多少好意。真要赞人,只会称呼一句好男儿。

韩冈身边的亲兵吃的是朝廷俸禄,平常就是有品级的官员见了,也得带着笑脸。今天又是护卫着微服出游的韩冈,容不得有半点意外。神经一直都紧绷着,现在见有人上来的挑衅,哪里忍得住气,登时站了起来。人没离桌,手就按到了桌上的腰刀上了。眼睛反瞪回去,一副一言不合就要抽刀的架势。而坐在对面店铺的亲兵,也丢下筷子过来了。

“怎么要动手?”

几个士兵全然不怕,直接捋起了袖子。韩冈的亲兵今天没穿那身极为醒目的猩红锦袍,都是换了身普通士兵的军袍,在雁门县的大街上,一点也不起眼。

见闹得不像样,秦玑起了身,“枢密如今就在城中,想把事情闹大了给枢密看吗?要喝酒,跑这小店里来作甚?前面向东走到小石桥转向南行百步,进去巷子里就有人陪你们喝酒了!”

韩冈三人坐得靠里面,这也是为了保护韩冈着想。秦玑没说话时,那几个士兵都没注意里面的三人,这时听到秦玑出头架梁,他们回头看了看,上下扫了秦玑一眼,“你这厮是哪里来的鸟……”

“好胆!!”“好狗胆!!”

秦玑、折可大同时怒喝。

秦玑这是代韩冈出面。他受辱,不但他和折可大发怒,亲兵们的怒气更盛,眼看着就要抽刀,只听得笃笃两声,是韩冈用筷子敲了敲桌面。

声音不大,可传入耳中之后,亲兵们却像是冰水浇头,立刻乖乖的收起了脾气,让开了自己的位置,跟另一桌的同伴并做一桌。折可大和秦玑也都坐下来了。

“早让不就了事了。”领头的军官哼了一声,眼睛一转,看到与秦玑同在一桌的两人,却当即打了个哆嗦,调头就拽着同伴出去了。这个变化突如其来,折可大等人都还没有反应过来。

“小石桥向南行百步?还真够快的。”韩冈摇头笑笑,不愧是最古老的行业之一,从上古兴盛到千年后。目送几名官兵飞快地走了,他回头对折可大道:“多亏了小乙你了。不然也吓不走这几个酒鬼。”

折可大低头看看,他今天穿着一身军袍。军官和士卒的军袍区别很明显,有品级的和没品级的也有差别。但军袍吓不走人。秦玑是个没品级的军官,方才他出面教训人,立刻就给骂回来了。

对士兵们来说,不是直属的上司,根本就没什么好怕的。争个座位的小事,没死伤就行了。打架不要紧,赢了还能给顶头上司涨面子,打输了回去才麻烦。

折可大清楚,他们真正畏惧的还是韩冈:“是怕枢密啊。”

“他们没认出我。”韩冈摇摇头。底层的士兵,有几个能近距离接触自己的?哪里能认得。何况要真的辨认出他的身份,肯定是不敢走了。

“看到枢密的装束就够了。”

韩冈没有前呼后拥,没有官袍加身。除非是车船店脚衙一流,寻常人也看不出地位高下。可就算那等没眼力的,至少还能辨认得出韩冈的穿戴,读书人的打扮。要是精通布料质地的,还能看得出韩冈的身上的衣服是棉布所裁。

士人和军汉坐一起,若是在平时,肯定少见,偏偏就是这里多。现在的雁门县,基本上就跟军城没两样。士卒比百姓还多。军中的士人,至少都是将领一级的幕僚,背地里喊一声措大没关系,但当面还是得恭恭敬敬。不用靠各色曲里拐弯的关系,只凭幕僚的身份,都比普通士卒更容易接触到上层。要是不小心冲撞到了一个后台硬点的,不死也得脱层皮。

“好了。吃完就走吧,别挡了人家的生意。”韩冈坏了心境,起身让秦玑去会钞。老汉方才闹起来躲在一边,兵凶如匪,普通百姓不敢搀和,现在才敢出来收钱。

“枢密,可要去查一查他们?”出来后,折可大问着,

“用不着。一点小事。”韩冈无意去追究。

其实韩冈并不觉得那几名士兵有什么错。话是粗了点,但要求是合情合理,只是见到读书人却没来由的怯了三分。换做是只有折可大在,他们最多收敛一点,照样敢坐下来吃喝。

他对折可大道:“这就是人比位子多的坏处。少不了要挤一挤,要是挤不进去,就得等。三班院中常年两三百小使臣在候阙。”

“就是枢密之前在瓶形寨对辽国宰相说的话?”

韩冈没有对他的那番话保密,也保密不了,跟外国使臣的会谈记录都要报上朝廷。而且还不是你好我好的废话,而是关于两国未来的对话。

“嗯。田土。官位。还有方才店里的桌位。资源总是稀少的。你想占多一点,别人就要少一些。世间的纷争,无外乎如此。”

折可大苦恼起来。如果韩冈跟在京的宰辅斗起来,折家都可能会受到池鱼之殃。折克行想要折可大确认的,本质上就只是这一桩而已。至于出兵越境的问题,只要韩冈地位稳定,无论是否入京,便完全不需要多在意。只是他看韩冈的态度,似乎是并不准备息事宁人的。

行走在雁门县的街市上,熙熙攘攘的官兵和街道两侧零星的商铺相映成趣。

亲卫们隐隐的形成了一个圈子,将韩冈护在中央。尽管穿戴朴素,但这样的保护,还是引来了许多人的注意。

越来越多的人侧目而视,韩冈心知,闲来无事的逛街看来快要到此为止了。

有闲暇、有闲心,偏偏没有空间。换做是京城倒是会好一点。

只不过回京城并留在京城这一事,对于在外的重臣们来说,都不是一件容易的事。

太宗登基,赵普为了能再度为相,弄出了个金匮之盟——说自己亲眼见证了杜太后临终前遗诏要太祖传位太宗,还把诏书封存在一只金匣子中。但太宗皇帝这个儿子作为当事人不在场,偏偏赵普这个外人在场;太宗皇帝登基时赵普不说,偏偏被晾几年后才说。此事真伪由此可知。

丁谓被流放海南,当着朝廷使臣的面故意把家信托人转交,让表白忠心的书信得以送到真宗皇帝眼前。

以寇准之明,二度罢相后为了回京,还奉承真宗出面进呈所谓‘天书’,致使晚节不保。

这三位都是本朝初年有名的宰相。有开国元勋,有世所公认的奸佞,也有千古留名的贤臣,但为了重新回到权力的中枢,无论贤愚不肖,就只有四个字——不择手段。

谁敢阻拦,便是死敌。

所以韩冈现在只想知道,吕惠卿为了能回京,他这段时间做了些什么。

不过他想看的只是吕惠卿的热闹,而对于自己能否于近日回到京城,却没有半点怀疑。

脚步稍稍变得轻快了点。

应该来得及回家过生日呢。
