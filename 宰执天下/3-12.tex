\section{第五章 月满完旧诺(下)}

韩冈进来的时候,轻手轻脚,他的女儿就在内间由乳母带着,他可不想给闹醒。

周南此时正披着一件褙子,凑在烛台边。手上拿着针线和布料,专心致志的在缝着什么。

“怎么还没有睡?不是让你先睡了吗?”坐到她身边,他轻声问着。

生产后早过了一个多月,月子也已经过去了。这几天,韩冈都在素心和周南两边轮换着度过。今天轮到周南,但因为赵隆、王舜臣来访的缘故,就吩咐让周南先睡。大半夜还做着针线活,对于刚刚坐过月子的产妇来说,还是很伤.精神的。

“前面才给大姐儿闹醒。”周南把手上的布料展开来,正是一件小衣服的模样,“现在又睡不着了。想着这件衣服还没做好,就拿了起来。”

韩冈的一对儿女,到现在都还没有起名。年纪太幼了,最好起个好养活的小名,比如像王安石的小名獾郎。到了七岁之后,再起大号不迟。韩冈也有个小名,但他不想再听到。

韩冈探手过去,拿走周南手上的针线活,“不要缝了,灯下做女红,容易伤了眼睛。”

周南轻盈的起身,对韩冈屈膝一福,娇声笑道:“是,官人!”

服侍韩冈在后间的浴房换洗更衣,周南为了方便动作,也脱下了外袍。

一头青丝只用一条丝带系住,由棉布缝制亵衣,不如丝绸轻薄,却柔软而贴身。在月子中又养了一阵,整个人变得珠圆玉润的同时,身材比例却渐渐恢复了旧观。因为溅了水的缘故,亵衣紧紧贴在周南的身上。峰峦起伏的身材,在微光下显得分外诱人。

回到卧房,搂着换了身干爽亵衣的佳人,双手贪恋着怀中娇躯的丰腴。探手胸前,腻滑如脂。向上托了一托,却是沉甸甸的,原本就是一手难以掌握的大小,现在又大了许多。轻轻一握,五指就整个陷了下去。

只是稍稍一揉.搓,怀中的美人便娇.喘起来。近十个月的久旷之身,现在一点也受不得刺激。而韩冈的掌心,都变得湿漉漉的。

韩冈支持由生母喂养,周南也的确是自己哺着女儿。但可能是丰盈远胜常人的缘故,她一向量多,多余的地方还帮着严素心喂着韩冈的长子。前面刚刚喂过女儿,现在又有了一些。乳汁的味道,甜味中带着点腥气,这跟饮食有关。在韩家待了半个月才走的徐老稳婆,在养育儿女的方面,给了很多的指点。

从丰软的触感中抬起头,韩冈心火大旺。前几次同房,他都顾虑着周南的身子,不能尽兴。现在算算时间,也还得再等一个月左右。不过火气上来了,真的很难压下去。

周南虽然头脑沉沉,但还有着一分理智,推拒着:“官人,不行!”

“我知道。”韩冈也没有昏头,知道还不到时候。可他的手指却抚上佳人丰润的双唇,轻笑道:“其实还有别的手段。”

看不到脸上的红晕,但低垂下去的头,却述说着她心中的羞赧。教坊司是以曲乐事人的地方,对于人伦之道上的奉迎之术,明面上是不会刻意教授。但私下里,还是有着教习。该知道的,周南也都知道。

虽然周南心中也不反对,但还是羞涩的说道:“等过几日云娘妹妹进来,就能让她好生服侍官人了。”

韩冈的动作停了下来:“总是苦了你了。”

周南轻轻叹起,知道他为何突然这么说。她回身摸着韩冈的脸,动情的说着:“换作是其他人,谁会顾虑着我们女儿家的心思。能让官人挂在心上……也足够了。”

前面在她们怀孕的时候,照常理就可以收了云娘入房。但韩冈还是等她们生了孩子之后,心中有了依托,才有了动作。这份心意,周南和素心都能感觉得到。

小门小户的夫妻相伴厮守当然好,但既然上天没有给她这个命数,终身也已经托付给眼前良人,周南也不会再去争什么。能得一知心的爱侣,又有个女儿,日后当还能再生几个儿子,周南已经很满足了。比起在教坊司中,每每让她从噩梦中惊醒的‘一双玉臂万人枕’的未来,眼下的生活才是真正的幸福。

韩冈也暗自庆幸,幸好两女都是和婉的性子,一颗心也都在自己身上,并没有闹出不愉快的事。不过他不会就此而放心,许多事要未雨绸缪,在危机出现前就该化解掉。家庭也是一桩事业,需要用心去经营。

他纳了周南和素心时,家里连个酒席都没办,现在收云娘,消息都已经在外面传开了。不好生安抚一下受委屈的周南和素心,日后云娘在家中也是难做人。

“官人……”

周南在韩冈耳边轻声诉说,打断了他的思绪。白皙娇软的身躯渐渐滑了下去,随即,一股温热如水的感觉包围了自己。

灯台上的蜡烛已经烧到了最后,闪了几闪之后,便熄灭了。黑暗随即涌了过来,掩去了床上的春色。

……………………

九月十五,是韩冈纳妾的日子。

陇西城中,现在都知道韩冈新近要纳的妾室,本来是韩家的童养媳,几乎是当女儿在养。她侍奉了韩家父母近十年,最后被韩冈纳为妾室。凭着这份苦劳,今次操办一下也不为过。

韩冈还没有娶妻,就纳了三个妾室,而且还有了儿女。从礼法上说,当然不合规矩。只是一般的大户人家的子弟,基本上都是如此。十三四岁时,就跟通房丫头,十四十五就有了子女也有很多。世风如此,都是当成了寻常之事。没人向韩冈提出不对,韩冈也不觉得不对。

若是有人因为此事,而收起了结亲的念头,对韩冈而言,也不是多让人遗憾的事。

贺礼堆满了韩家的堂屋。官家钱明亮,带着两个识字会算的下人,将礼物一件件的登记造册,并对照着礼单,看看有无差错。

钱明亮已经写到了手软,冯从义在旁饶了一圈回来,庆幸自己不用再像过去,为姨妈家来抄写礼单。他对韩冈笑道:“今次送来给哥哥的礼,可要比别人家娶妻都要多多了。”

韩冈一笑。这是在说苗履。苗授的儿子苗履前些日刚刚娶妻,韩冈还送上了份厚礼。但从他眼下收到的礼物来看,的确要比苗履多上许多。

“只是成、刘、李那几家有些难过了。他们开始听到三哥纳妾,就眼巴巴的跑过来送礼的。现在连份席面都不能给他们……”

“平常那些个秦州商人的礼都不收得,今次收下了,已经是给面子了。怎么还想上席?!”韩阿李不快的反问着,冯从义不敢再多说话了。

送礼也不是想送就能送的,还得看资格。韩冈置办家业的本事过人,在熙河路不过三载,就已经是十万贯的身家。有产业,有田宅,不是那等看到钱就挪不开眼的穷措大。身份不够,无意结交的,直接就把礼单递还,在司阍处就给拒了。

其实这也是韩冈为人正直,他一直秉持着人情往来的道理,收下礼,就等于欠下人情,总得还回去,所以不想乱收礼。不像有些官员,收礼肆无忌惮,甚至是不送还要去敲诈勒索一番,想让他办事,还要再送钱。

如秦州的商人们,想在韩冈面前混个脸熟都难。除了几家准备在巩州种植棉田的秦州豪族的代表,基本上都只能跟冯从义打交道。即便是年节的时候,直接送到韩家门上的礼物,也从来都是不收的。

现在终于有了个巴结的机会,当然个个趋之若鹜——韩冈虽然要离开熙河,但韩家的根已经扎在了这里——只是当他们来到陇西城,送上了礼物之后,却听说韩家根本没有大肆操办的意思。

原本韩冈也是想着要操办一下的。想在自己离开之前,通过这场纳妾之礼,展示一下自己在巩州的地位和声望,镇住一些蠢蠢欲动的家伙。

但韩云娘本人却反对了。她不清楚韩冈的私心,但她知道韩冈收周南和素心的时候是什么样的情况。轮到自己,举行仪式已经是很风光了,但若是操办过甚,总觉得对不起两个姐姐。

云娘的反对,韩冈考虑再三,也放弃了之前的想法,还是以家庭和睦为重。至于要镇服一些藏在暗地里的小人,也不是没有别的手段。所以最后他就是请了几个亲近戚里入席,并没有开门宴客,有点雷神大雨点小的感觉。

纳妾的仪式没有婚礼的繁琐,也没有正式的规则。韩云娘向韩父韩母行过礼,由韩冈——而不是子女双全的妇女——挑开盖头,再敬过几个赴宴的亲近戚里的酒,仪式也算是结束了。

韩阿李看着云娘在自己身前行礼,笑容中,有了几许安慰。她一直都等着儿子给云娘一个归宿,现在也算是完成了一个心愿。孙子、孙女都有了,只要再看到儿子娶了正室,那就真的没有什么牵挂了。

外厅的酒宴继续着,王舜臣哈哈大笑的声音,隔着几重屋都传了进来。

云娘一身桃红色的婚衣坐在床边,双手不安绞着手巾。不知等了多久,终于听到熟悉的脚步声。身子微微颤抖了起来,她都等了好几年了,但临到头上,却是有些害怕起来。

但当更为熟悉的面容出现在眼前,砰砰乱跳的心脏却渐渐平复,一点点的安心下来,化作了一个绝美的笑容:“三哥哥,你来了……”

