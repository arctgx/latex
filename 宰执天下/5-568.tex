\section{第44章 秀色须待十年培(24)}

由夏入秋,几乎就是转眼间的事,由秋入冬也是犹如飞一般的迅快。

天气一天天的转凉,院中的落叶也一曰多过一曰。落叶多了,总免不了有人会想到废物利用一下。

宣徽院的隔壁是群牧司。韩冈所在主院的隔壁,便是群牧使的公厅。除了韩冈担任同群牧使的那段时间,群牧使的公厅一天下来,有主人在的时候并不多——群牧使一向是枢密院都承旨兼任,群牧制置使更是枢密使或是枢密副使的兼差。

这时候,群牧使也照常不在。不知那边的官吏趁机在烤些什么,香味都飘到了宣徽院这边来了。

“群牧司在闹什么?”

时近傍晚了,乍闻到香味,沈括立刻就感到自己肚子饿了。由于要自重身份,口气就变得很不快。

“三班吃香,群牧呢?能吃什么?!”

韩冈放下手中笔,对沈括笑说着。

“玉昆。你也是做过群牧使的。这样说可不好。”

苏颂也从文稿中抬头,顺便摘下了眼镜,揉了揉酸胀的双眼。

“现在又不是了。”

韩冈望了望对面,可能是故意选了树下生火,烟气被树叶分散后就变得淡得看不清。

沈括转向苏颂:“子容兄,你也不管管?!”

苏颂摇摇头:“那是薛师正的差事,不好插手。”

“韩冈倒想起幼时了。扫起树叶,瞒着父母烤东西吃。”

韩冈说道。想当年他也是这样烤过红薯和玉米的。

“哦,不知那时候玉昆你烤的什么?”苏颂饶有兴味的问着。

韩冈心一跳。红薯、玉米这个时代可没有。想要引进,却隔着万里鲸波。他摇摇头,“还能是什么,野兔,山鸡,还有黄精、山药。都是些野地里找到的。”

并不是韩冈信口开河,也是在他的记忆中,他的两位兄长都曾经带着还年幼的他出去抓过野味,也掘过一些山珍。

不想在这事上跟人多说,韩冈又道:“不如让人端些菓子过来吧。对面在吃,这边肚子也饿了。”

“也好。”苏颂点头,老实不客气的又道,“再来些茶。”

“存中你呢?”韩冈又问沈括。

“一样吧。”沈括道。

很快,三名大臣便就着茶,吃起了糕点。下午茶的时间,公器私用的‘《自然》编辑部’也轻松了下来。

第五期《自然》正在编纂中,沈括和苏颂都是为此而来。有了两人的帮忙,这一期的质量又上升了一个台阶。不过审核和校对的工作,依然繁重不堪。

韩冈、沈括和苏颂都提过,是不是该培养一批编辑来代替自己,处理一下基本的文字工作。可即便是最基本的文字修改,也需要足够的科学常识,有这方面能力的士子,比三条腿的蛤蟆还少些。不论是韩冈还是苏颂,都不放心将这方面的工作交给外行人。

到现在为止,也只是培养出几个拆信的文书,让他们检查以稿件的名义送来的信件里面,到底真的是稿件还是别的无关之事。

沈括又吃了块绿豆糕,用浓茶清了清口。指着桌上的堆成山的稿件:“也不知要几天才能看完。就像沙子里掏金子,一天下来也不一定能淘到几粒金砂。”

“这事要有耐心。十年树木,百年树人。真要等到开花结果,就像玉昆之前说过的,”苏颂冲韩冈笑了笑,“要穷十年之功。”

“要真的有十年功。存下来的稿件怕是要堆满几间屋子了。”

沈括看了眼地上存着废稿的木箱,被他们三人集体否定的稿件,全都会丢进这个木箱中。箱中积存的稿件有上百封之多了,但没审核的还有更多。

《自然》已经到了第五期,前几期所引发的回应也越来越多了。如同潮水,涌向了京城。投稿络绎不绝,在剔除了近三分之一,求官、讨好、申冤、求助,以及满篇诗词歌赋的信件之后,剩下的投稿依然有五百余封。而其中有价值的,其实不及十分之一。

前几期被淘汰和录用的投稿,都按照时间和录用与否,分别打包存放了起来。并不会销毁掉。但时间长了,就免不了像沈括说的一样,堆满几间屋子。

“几间屋子的事还好说。”韩冈说道,“除了那些乱七八糟的信,现在能给《自然》投稿的,率为有心于格物之辈。不论本心为何、见识高低,都是值得鼓励的。像这样只是收下就没了回音,说起来有些伤人心啊。”

“那怎么办?又不可能一个个都回信。每一期最后都说了抱歉了吧?”

按照韩冈的提议,现在都在每一期最后一页刊登致歉声明,对无法回信表示歉意。不能回信也许是现实,但人心必须要考虑到。

苏颂也摇头。一一回信的确不可能。他写想勉励那些有心格物却不知从何入手的投稿者,但现实的情况不允许。

如今私信,绝大多数都是托人转交。有的是拜托亲友,还有的则是借助稳定的来往于商业路线上商人来传递。朝廷的驿传,虽说大臣们偶尔可以借用,但如此大数量的回信,就绝不可能。

而且投稿人来自天南海北,现在最远的已经有陇西和福建的投稿,可以想见,在未来,岭南、甚至甘凉的投稿也会有。不说数理化,就是将各地独特的地理风貌记录下来,便能登入《自然》之中的地理一科了。那样的情况下怎么给人回信?

韩冈自是知道《自然》的发行和收录,需要一个遍布全国的邮驿系统。但他没想到会这么快就需要了。原本他都做好了连着两三年都没有合格投稿的心理准备。但现在,情况比他预计的要乐观很多,寄来的信件中,有价值的比例比预计的要高不少。估计两三年之后,就得改为月刊,才能安排下版面了。

“如果要能借用朝廷的驿传,办理民间的邮政那就是好了。”韩冈决定先敲敲鼓,没必要多拖了,“现成的驿站车马,走惯了的路线,更有大量的人手。”

“不可能吧,朝廷怎么可能会答应。不说干扰军情递送,就是花销……”沈括说着说着就停了口,惊问韩冈:“要收钱的?!”

“当然要收钱。从东京送到交州是一个价,从东京送到南京则是另一个价。路程越远,邮递价格就越高。但不管路程远近,朝廷的驿传肯定要比民间托人带信要快。”见沈括想要说话,韩冈又补充道,“我是说平均速度。”

民间都是顺带送信,不可能像驿传一样,目的就是送信,能一程程的送下去。有一些人会比朝廷的步递要快,但更多的却会更慢,而且是慢得多。

沈括皱着眉头:“就算收钱,可驿站传信也只能送到驿馆。还是说送到官府,然后通知人去领?”

“当然不是。”韩冈解释道,“路、州、县、乡都设专门的邮局,一层层的将信转发下去,也一层层的将信收上来。就像是衙门一样,乡镇、县监、州郡,然后到路中,最后汇集到京里。驿传不正是这样用吗?”

“玉昆,说明白点。”苏颂道,放下了手中的茶盏,摆出了洗耳恭听的姿势。

“如果韩冈在巩州陇西县的乡里,有信要寄到同一州的宁远县。可将信送到乡中的邮局,如数付了钱。然后当天乡中的邮局就将信送到县中邮局,县中的邮局再送到州治的上级邮局,在那里进行分拣。确定是本州的宁远县,然后就是让负责这条线的人送下去,传到宁远县中。”

“下面呢?怎么送?”苏颂追问道。

“具体到城中,就要先做好准备。必须每一家每一户都要有个门牌号。厢、坊、街巷,然后是街巷中的第几户人家。比如存中府上,正门开在宣化坊北亭巷中,所以便开封府旧城右军第一厢宣化坊北亭巷一号……从东头数第一家嘛。”韩冈冲沈括笑了笑,继续对苏颂道,“只要知道了门牌号,这样本城的邮局,就能顺利的将信投递到府上。”

“如果是本路寄信就在路中多一重转送。如果是隔着好几路,那就是从京城转送了。是不是这样的?”苏颂说着,问韩冈。

韩冈点头,“其实东南、河北,中原,西南、这样的大区域,都要设一个转运的中心。免得两浙送江东的信,要到京城走一遭。”

“乡间呢?”苏颂又问道。

“邮送不可能到村中。但每个村子都可以在乡里或镇上的邮局设个专属的邮箱,存放本村的邮件。等村里有人到乡中、镇上的时候,顺便就能带回去了。邮箱可以安锁,邮局和那个村子各拿一把,免得给外人偷拿走。”

“能到乡镇就不错了。的确用不着到村里。”

苏颂闭起眼睛,在头脑中过了一遍。很快就点起了头。有已经成型的驿传系统在,将民间的邮政纳入进来,还是很容易的。一个达到乡、镇一级的邮递系统,这对国家的意义可想而知。

