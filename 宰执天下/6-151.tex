\section{第13章 晨奎错落天日近(16)}

离开垂拱殿的路上,李定与章敦渐渐落在了后面。

望着前面走得不徐不疾的韩冈,李定低声:“此必是预谋已久,绝非仓促而为!”

章敦也看着前面,王安石早走得不见影子了,韩三相公和韩三参政走在前面,张璪稍后一步。

章敦大感无奈。

王安石性子急,平日里走得就快,现在肚子里压了一团火,出来后连招呼都不打,就走得飞快。

“当然不可能是仓促行事。”章敦淡然道,“韩玉昆几曾做过意气之举?那次不是谋定而后动?”

方才殿上,太后答应了韩冈的请求,同意让重臣们共商国是,一如廷推之例。

这并不能完全说是因为她对韩冈的信任贯彻始终,章敦也清楚,在上一次宋辽大战之后,太后一直都很希望能够在她手上完成收龘复河北、河东故地的夙愿。青史留名的诱惑,即便是女流之辈,也难以抵挡。这就是她为什么之前的一段时间,一直都没有对朝廷上的争论表态的缘故。

但她最信任的臣子始终反对出兵攻辽,今日殿上与王安石、吕嘉问争辩时的语气,也不像之前那般和缓,这肯定会让太后担心起万一失败了怎么办?丢掉了军心士气,让天下臣民失望,过去积累下来的威信也会荡然无存。

这样的情况下,让臣子们来共同议定大政方略,自己则只要点头就够了。事后即便证明有错,也能归咎于臣子,不至于让自己也陷进去。

不得不说,韩冈的确抓准了太后首鼠两端的心理,这一套伎俩,也让章敦感到十分的眼熟——臣子操控君上,或是吏员操纵上官,其实都是一脉相承,道理相通的。

而太后对韩冈提议的首肯,便让王安石怒气勃发。但王安石偏偏不能发作,明明心里强烈反对,却一句话也没有说出口,硬是给憋得涨红了脸。

章敦暗暗叹了一声,每次都是这样,韩冈总是拉着一帮人公然来瓜分两府的权力。不论哪位宰辅想要反对,都要顾忌朝臣们的反应。尽管身居高位,可宰辅们始终都要有足够多的支持者,才能在朝堂中保证自己的权力和影响。人心离散,这个宰辅就做得一点味道就没有了。

而韩冈双手将权力送上,哪个朝臣不乐意?就是他的反对者,一干新党中坚的朝臣,都很乐意在国是一事上,说几句有分量的话。

韩冈的行事作风,本来就是不把其他朝臣看重的东西当做一回事,而是追求千载留名、万世师表,这也是他跟王安石翁婿决裂的主因。否则都是权欲不重的人,怎么可能势同水火?

章敦还记得韩冈曾经对自己打了个比方,同样的一幅白纸,各有各的画法,两个画师都想在画纸上呈现出自己想要的东西和风格,理所当然就不可能合得来。相反地,如自己和吕惠卿这样的人,却只在意有没有执笔的机会,至于画出来的是什么,他们并不在意。

“这下不好办了。”章敦感触甚深的低声说道。

“的确。”李定点头,他有着同样的感触。

当初韩冈反对出兵,甚至有手段在一夜之间让局面扭转,不过当李定接受了王安石的请求,一起说服了章敦,韩冈便登时身处窘境。

当时李定并没有想过能让韩冈束手无策,过去的经验让他不会这般幼稚,但他能为韩冈想到的对策,依然是认为只能通过说服太后来压制自己这一方。

可是以新党的实力,以及王安石的威望,足够将太后的决定给顶回去。甚至还不需要硬顶太后,只要拖上两日,等北面打起来,结果也就注定了。

李定事前曾经猜测过韩冈会借重朝臣的力量,反过来进行压制,甚至有可能会拉着一众有着推举之权的重臣共同来讨论是否应该出兵。对此李定也做了一点准备。可是他只猜对了一半,韩冈所切入的方向让任何人都始料未及。

韩冈将目标对准国是,李定没猜到,王安石、章敦、甚至韩绛、张璪、苏颂,应该也都没能想到。而利用有宰辅荐举之权的重臣们,拉着他们一起共商国是,即便有一半,但正题上自不在预料范围之内,也就没办法在事前做好应对的准备。

“他会怎么做?”李定轻声问着章敦。

“还能怎么样,肯定又是多者为胜!”

“既然如此,还是有可能赢过他的。”李定道。怎么说,在人数上还是新党一方更占优势一点。

章敦摇了摇头:“韩玉昆的想法不可能那么简单,既然他能提出来,那么他肯定还有后手。当年所定国是,平章那边可是一点都不想改。”

李定沉默的走了几步,徐徐叹道,“的确如此……可这对他又有什么好处?”

自己不能独占的东西,也不让别人独占,宁可分给所有人?韩冈的想法,很早开始,就让李定感到难以理解。

“资深,你可知道,韩玉昆的目标是什么?”章敦问道。

李定反问:“是什么?”

“天子垂拱而治,士大夫共治天下。”

李定愣了一下,然后悚然而惊。

《易·系辞》中有‘垂衣裳而天下治’一句,自此之后,历代儒生都将此一事视为圣君的标准,也把此事当做了自己的目标。现今无论儒门的哪一派,都赞赏天子垂拱而治的治国方式,认为符合三代之治,使他们所要追求的最高目标,至少是目标之一。

而‘士大夫共治天下’,文彦博曾经说过的那句名言中就有这么几个字。旧党中的那位元老,他的这一句,明面上虽时常为人驳斥,但私下里,绝大多数朝臣都对对此赞赏有加。可是,文彦博说的是‘为与士大夫共治天下’,而章敦转述韩冈的话,却把‘与’字给删掉了,少了最关键的那个‘与’字,意义自是变得完全不同。

这两句话,一句源自经典,一句是切合现实,现在两句话给章敦修改拼凑起来,却让李定不寒而栗。

‘天子垂拱而治,士大夫共治天下’,章敦说这是韩冈的目标,这岂不是说,韩冈打算将皇帝放到供桌上去做个土偶木雕,而由臣子们共同治理国家?

看了看前后左右,李定更加小声的问:“子厚,记得当初你与韩冈关系亲睦,可之后……”

李定说到一半,章敦便点头,“的确有一点这方面的原因。”

“原来如此。”

权柄操于臣子之手,天子不能与之争,这岂不是太阿倒持?而想要做到这一点,天子绝不会坐视,做臣子的可就是要把性命赌上去。朝中党争就已能掀起狂风暴雨,而天子与臣子争,那可就只能用腥风血雨来形容了。

想到这里,李定猛然一震,惊骇的看着章敦:“那群臣共议,不就是……不就是……”

“恐怕正是在他的计划之中。”

“太后怎么就能信了他?”

“不信他还会信谁?”

章敦叹道,换做自己在太后的角度上,也只会信任屡屡救其于危亡的韩冈。

“可他一番辛苦,就是为给人作嫁衣裳?”

韩冈给自己弄好处,做一个权臣,甚至谋朝篡国,那还不难以理解,北面正有一个最新鲜出炉的例子。但韩冈辛苦一番,却是将权柄分于同列,这是到底为了什么大费周章?

“本来觉得想通了,后来又发觉自己没有想通,只是后来不好问了。”章敦很洒然的摇头道,“要不是这话是韩玉昆亲口所说,根本想都不会往那里去想。”

李定狐疑的看着章敦,会不会是章敦当时听错了,或是自己和章敦错误理解。韩冈已是儒门宗师一级的人物,或许他的话只是些白日梦般、说给弟子听的想法。就像儒生们追求三代之治,可实际上绝大多数只是当成了一句挂在嘴边的话,谁也不会当真让皇帝去仿效尧舜——皇帝要学尧舜禅让,哪个臣

子敢应的?

感受到了李定脸上透现出来的狐疑,章敦暗暗苦笑。很长一段时间,他其实也怀疑过自己是不是听错了、想错了,不过当日韩冈说出来的一句一字,至今仍鲜明的刻在头脑里,又怎么会弄错?

“此事暂且不提。”章敦叹道,“传出去也没人信。还是想想怎么去应对吧。”

“此事平章亦难为,如何应对?”李定同样叹道,三日后垂拱殿中龘共议,理应去多争取几个人支持,可这件事说起来简单,做起来就难了。

‘国是’不是军事、政事,而是国家大政。

当年王安石辅佐熙宗皇帝,决定了延续至今的大政方略——推行新法,富国强兵,先复灵武、再收燕云。

想要反对这几条、或是违背这几条的朝臣,无论地位有多高,都被赶出了朝堂。如今所有朝臣,不论是否心甘情愿,都是按照这几条所决定的方向走。

现在韩冈给了他们一个机会,可以让未来的国家大政依从他们的心愿而变,这是每个朝臣都难以抵挡的诱惑。

韩冈的每件事都正正打在要害上,不争夺一城一地,而是想要拔根。

李定唉声叹气,“这件事,不好办呐……”
