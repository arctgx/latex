\section{第34章 为慕升平拟休兵(27)}

宋军的出现完全出乎于辽军的意料之外。

代州城是不是被宋军攻占,乌鲁现在已无暇去多猜想,甚至还来得及有多少反应的时候,千百箭矢便离弦而出,直扑而来。

身在最前方的乌鲁完全看不清宋人的动作,但随着弦鸣随着风声和号角声一并传入耳中,他立刻翻身下马,顺手摘下鞍后的小盾,护住要害整个人挡在了自己坐骑的前面。整套.动作如同行云流水,没有丝毫的迟误。

头和胸藏在轻巧的小盾后,不过暴露在外腿脚还是感受到了几下触击,一开始的感觉很是轻微,可转眼之间便是一阵剧痛闪过全身。

乌鲁却松了一口气,身后的战马没有中箭,正呼吸安稳的将热气喷在自己脖子后。

备用的战马在后面,这时候可来不及换乘。必须保住现在的坐骑不受重创,这样就算自己中上两箭也照样能跑掉。要是反过来,就是把小命送给了宋人。

绝大多数辽军士兵都跟乌鲁一样,及时做出了反应,纷纷用自己的身体保护住了自己的战马。

在箭雨过后,很多人都庆幸不已,箭矢的力量比预计得要小,使得他们都没有受到太重的伤害。

不过没有人敢就此放心下来,依然提着盾严阵以待,他们已经习惯宋军一次拿出三四张神臂弓,然后在极短的时间内连续射出三四波箭雨。

可是他们没有等到更多的箭雨,而是密集的脚步声和疯狂的呐喊。

乌鲁脸色煞白,回头望着堵在身后的自家人,正是一片乱象中,这时候根本撤退不了。甚至连收拾混乱都来不及。

相隔一百三四十步,纵使是神臂弓也只能在辽军的前军中造成混乱,而无法给予更大的杀伤。但这一点混乱,完全扼杀了辽军反击的可能,也让宋军的战士趁此良机杀了过来。

韩中信脚步沉沉,提着长刀冲在最前,秦琬指挥全军,他这个副手自是要为全军之表率。百余步一晃而过,转眼间,他就已经领着数百将士杀入了敌阵之中。

冲向最前面的敌人,蓄势已久的长刀划着一道弧光重重地劈了下来。千钧之力破风而至,势道锰恶,似乎能将人马一分为二,但这一击却被一面木盾稳稳地给挡了下来。

砍在木盾上的一刀汇聚了韩中信身上所有的力量,没有劈开木盾,却让木盾后的敌人腿脚一软,半跪在地上。

韩中信反应极快,又是一刀劈了敌人的脖子,可再次被盾牌给格挡住。反震的力道很大,让韩中信都不由得退了半步。

是个强手!

念头一闪而过,韩中信更加兴奋起来。正要再次挥刀上前,跟在他身后的两名亲信已经左右抢上前去,左右挥刀合击,只一记就将那名敌人砍倒在地。

韩中信毫不犹豫,一脚踢开挡路的敌人,立刻深入敌军之中。

辽军在遭受箭雨的那一刻,早就将火炬全都丢在了地上,随着火炬渐次熄灭,战斗就在黑暗中展开。周围远处的火光只能更进一步的加深光暗的对比,并不足以照亮脚下一寸土地。

周边光线全无,可韩中信的双眼已然渐渐适应这样的黑暗。随着他在敌军中的深入,身后的同伴越来越少。

在黑暗中的厮杀完全没有分辨敌我的时间。可大部分人一开始并不知道这一点,看到眼前晃动的身影总是会先喊上两声,不过血的教训很快他们就学会了如何保护自己。

当的一声响,刀上传来的反震震得手上一阵发麻,一闪即逝的火星,并不是砍到了铁甲上的感觉,而是双刀交锋后才有的现象。

手腕一转,长刀立刻向下划了过去,惨叫从身前响起,刀锋入肉的感觉让韩中信松了一口气,迸射到脸上的血水也似乎比甘露更加清甜。

叫喊声已经分不清是官话还是契丹话,反正双耳中充斥的混乱根本无人能够分辨,也没有人还能冷静的去分析。一刹那的分神,很可能就是生死的分界线。

只是由于是在撤退,基本上所有的辽军士兵都没有穿戴盔甲。也就靠了这一点,才分辨得出面前之人到底是属于哪一方。

黑暗中的一通乱杀,也没有持续太久。

战斗结束得很快。

大部分位于后方的辽军,早就驭马分头逃窜,如同被惊散的蚊蝇,散入了夜幕下的荒野中。剩下的来不及走的,其实也不剩多少。

到口的好肉跑了大半,这让韩中信扼腕叹息。仅仅是放置在两翼的疑兵,实在来不及堵截四面奔逃的敌军。喘息着持刀立于道上,周围只剩下穿着盔甲的宋军将士,而辽军,除了躺在地上的,已经一个不剩了。

一点火光从后映来,有节奏的摇晃着,秦琬的声音随之响起,“守德,辛苦了。”

“想不到就这么结束了,俺还没杀够呢。”

韩中信的盔甲上,浓稠的血浆正向下流淌,脸上也是如同恶鬼一般的溅满了血渍,但他咧起嘴来一口白牙就在火光中闪闪发亮。

“辽贼的这一路算是给吓跑了,斩获虽不算多,却也说得过去了。”秦琬笑着说道,“辽军已经是惊弓之鸟,在方才的交战中,根本没有组织起像样的反击,而是选择了逃跑。可见其胆魄已丧,不足为虑了。”

“辽贼不该选择夜里撤退的。要是白天退兵,绝不至于被我们这点人给吓到。”

“前天一天一夜就运来了三五万大军,今天又是一天,现在肯定有七八万了。怎么能不逃?就是骑兵都肯定有一万多了,不比辽贼少。”

身边围过来的几位指挥使正兴奋地议论着现在的军情。

韩中信身为韩冈的亲信,当然知道这些都是胡扯。下面的官兵都以为只要有了轨道,要步兵有步兵,要骑兵就有骑兵,但韩中信知道,朝廷给予的支持不可能再多了。辽人的想法肯定是跟他差不多的。

一天三万,五六天后,就是十五二十万了。列起队来能从大小王庄一直排到代州城下。如果真有那么多兵,倒是能运得来。可惜河东这边接收的粮秣,很大一部分都用做了以工代赈的酬劳,这也是轨道能快速修筑的保证,也是稳定后方的需要。

除去了民夫的食用后,能养得起现在的五六万大军已经是奇迹了。不要说三五万大军,再增加个一万步军,就只能饿肚子了。

要不然这边也不会只有这么点人马去追击逃散的辽贼。

韩中信并不知道,就在数十里外,已经攻下了大小王庄的主力,正在展开了更大规模的追击。

夜色渐明,一夜没有合眼的辽军,完全暴露在了养足了精神的宋军的视线中。

攻下毫无防守的敌军主营,并没有增添多少战果,却让宋军上下对功劳的渴求更加旺盛。

辽军分三路撤兵,就属从官道撤离的中路军被追击得最为惨烈。

北路的各部兵马给秦琬、韩中信给吓坏了,分散逃窜后不敢再向前,除了一小部分散逸无踪之外,剩下的都是改向中军靠拢。

逃回来的不仅是败兵,同时也将混乱带到了中路撤退的己方兵马之中。

除了最早离开的几部兵马,剩下的全都被堵在了半道上。给宋军的骑兵给咬住,眼睁睁的看着宋军步军缓缓地逼了上来。

一路上,残尸遍地,到处都能看到失去了主人的战马孤零零的站在原野上。

耶律道宁回头望着河对岸惨烈的战况,心中庆幸不已。

之前他被萧十三千叮咛万嘱咐,一定要及时过河,走南线回去,现在他终于知道萧十三并不是那么愚蠢的统帅,只是对手太强了一点。

从大小王庄回代州城其实就四十里,但夜路难行,速度也快不起来。现在天亮了,可以用更快的速度赶回去了。

随着日头渐渐升高,宋军在兵力上的优势更加体现了出来。完全失去了组织,一盘散沙的辽军成了餐盘上的糕点,一个个的被吞下了肚子。

“到底有多少逃进了代州城?”

“应该有很大一部分不会选择回到代州,而是应该往繁峙县那边去。那几个落队的俘虏不是说吗,萧十三和耶律道宁”

“可惜骑兵太少了。再多一点,就能将他们抱怨了。”

繁峙县就是飞狐陉,通往南京道。代州城既然肯定是宋军的主攻方向,那么避往繁峙县就是最聪明的选择。至少还能得到南京道的援军相助,也方便撤退。但代州、雁门就不一样了,不仅要死守,甚至还有可能因为神武县的麟府军而腹背受敌。

可是很多人都猜错了。通过俘虏和斥候的确认,绝大多数的辽军都选择了返回代州城和雁门,而不是去繁峙县。

“也许正是因为耶律乙辛在南京道吧。”

“怕撞到尚父的气头上吗?”

“应该是挂念着战利品吧。怕被南京道的兵马抢过去。”

“枢密怎么看?”

几名幕僚语带调侃的猜测着缘由,又来问韩冈的看法。

“或许都有吧。”韩冈语气恬淡。

不管是什么原因,现在代州城已经就在眼前。

也就在同一天,曾经被派来军前的中使姜荣又来到了韩冈的面前。

“双方罢兵……”韩冈抬头看着代州城头那面在风中拂动得有气无力的大旗,“朝廷是要河东休兵吗?”

“是的。”姜荣低头不敢看韩冈,“休兵。”
