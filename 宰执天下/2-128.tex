\section{第26章 西山齐云古今长(上)}

【昨天的章节忘了修改就发上来了,有些地方不通,现在已经改好了。】

清晨的时候,韩云娘从睡梦中醒来。

睁开迷迷糊糊的双眼,从窗外透进来的,没有光,只有一记记低沉的钟声震动着耳中。

暮鼓晨钟,从城中心的谯楼上每日依时响起的悠扬钟声,固定在寅时三刻,把这座边塞小城从沉睡中唤醒。

手捂着小嘴打了个哈欠,云娘揉着眼睛,坐了起来。有些凌乱的秀发披散在白色的小衣外,在胸口处被顶了起来,峰峦起伏,已经不复青涩。虽然胸前的曲线已经初具规模,可沉睡初醒的困倦,仍显得一张小脸稚气未脱。

身体从温暖的被窝中离开,刺骨的冰寒便透过一层单薄绸布渗了进去,细嫩的肌肤上顿时激起一片寒栗。少女抱着膀子,向下看了看,房中的火盆不知什么时候熄灭了。

“李家的炭真是不经烧,下次不买他家的了。”

云娘嘟着嘴抱怨了一声,快手快脚的换好衣服。新制的夹袄紧紧裹着身子,再将襦群和褙子穿上,感觉方好了一点。将被子叠好,对着刚磨过的铜镜把头发理顺,就着火盆上一壶已经变温的开水洗漱好,内院中这时已经有了人声。

云娘推开门,更加浓重的寒气扑面而来,少女却笑颜如花,清脆的声音叫着院中高大的身影:“三哥哥,你起来了。”

韩冈点了点头,没有答话。一个箭步,一拳带着呼呼风声向前击出。他一向起得很早,坚持锻炼身体,冬练三九,夏练三伏,筋骨打熬得不输武将。现在他打的一套拳法是从赵隆那里学来的,并不是传说中的太祖长拳——太祖皇帝杆棒了得,但拳法在此时却没听说有何流传——而是五禽戏。

赵隆向韩冈传授时,信誓旦旦的说这套五禽戏是陈抟老祖所创,华佗就这么被欺师灭祖的弟子抹去原创权。不过这套五禽戏,刚猛有余,柔韧不足,韩冈怎么看都不像是健身用的拳法,曾给王舜臣、李信看过,都摇头说不是。不过这套拳法打起来便能出一身热汗,感觉十分痛快,便一直练着下来。

这时候,一缕炊烟已经从烟囱上升了起来,严素心正在厨房里忙着,两个打下手的粗使丫鬟在她的指挥下,也是在炉灶前忙个不停。

韩冈地位日高,在外面跟着他四处奔走走的亲卫姑且不提,光是分配到他门下服侍的老兵就有四人,现在都在外院住着。而且以韩冈的官职,虽然比不上宰相能向朝廷报销百名随从的月俸,但李小六也是每个月能从衙门里领到百来文钱,换季时也有做衣服的布料丝棉发下。

而在后院,丫鬟也多了三个。一个是在疗养院中病死士兵的孤女,自幼亡母,而后父亲又病殁,唯一的一个叔叔还是个泼皮,都想要把她卖给青楼,韩冈听说后就把她收留下来,让她服侍自家父母。而现在在素心手下的两个粗使丫鬟,则是瞎药送来的,都能听懂汉话。

“云娘,起来了?”严素心忙碌之余,一眼瞥见韩云娘身上的衣服还是有些单薄,有些心疼起来,“天气冷了,再多添点衣服才是。”

说着便给韩云娘端了碗热汤来。在冬天,厨房里热水一直都有,炉灶都不熄的。对官宦人家来说,木柴、木炭的消耗算不上什么。

少女安静的坐在厨房一角,小口喝着热汤,听着锅里咕嘟咕嘟的热水沸腾的声音,暖意传遍全身。

“好了!”韩家的美人厨娘把锅盖揭开,一股鲜美的羊肉香气便随着热气传了出来,里面是韩家今天的早饭。

从严素心手中接过两份早餐,韩云娘便小心端着向后走去。

“秋香,开门。”韩云娘轻声叫着门。门立刻开了,一个比云娘还要小一点的丫鬟走出来,把她迎了进去。

新来的丫鬟秋香长得很朴素,但人聪明,又勤快,把韩家二老服侍得很顺心,跟云娘、素心关系也很好。但韩云娘就不知道为什么韩冈听说了这个名字后,先是愣了一下,接下来便说她日后配姓唐的比较好。

韩千六和韩阿李起得一向早,毕竟刚从庄稼人的身份脱离不久,还是保持着鸡鸣即起的习惯。进门后放下食盘,云娘便向二老请安问好。冬天房间中有些冷,韩云娘先惯性的看了看火盆,却是将熄未熄的样子。

“李家的炭不能买了,烧得快,烟气还重。”见到云娘看了火盆,韩阿李便抱怨了起来,“不是说三哥儿在疗养院弄的火炕很好吗?就在床底下生火,屋里也不见烟,比起用火盆好得多。”

“三哥儿前些天说了,用火炕要把房子大修才行,现在天寒地冻的,也不好换个宅子住。再说这房子还不知能住几年,修了也不一定能用上。”

夫妻两人说着闲话,云娘服侍着两人吃饭。吃到一半,韩阿李像是想起来什么,放下筷子,“云娘,你等会儿去把小六找来。再有两天就是冬至了,得让他去外面的榷场跟义哥儿说一声,后天记得要回来吃饭。”

“知道了。”少女答应了一声,继续服侍着二老。吃过饭,说了一阵闲话,看看天色已经大亮,韩云娘便收拾好碗筷。先去厨房,再去书房。

今天是韩冈的休沐之日,虽然忙的时候根本没有休沐这一说,但到了冬天,公事简省,衙门里也清闲了下来。韩冈也没有必要再克扣自己的休息日。

锻炼过后,擦洗更衣,韩冈就照惯例窝在书房中读书,云娘知道她的三哥哥还是想考个进士出来。不便打扰他读书。远远的小声叫过李小六,照着手让他过来说话。

清朗的读书声一直持续到中午的时候,当韩云娘准备去找韩冈,一个熟悉的身影出现在他的面前。少女脚步一停,惊讶道:“朱郎中?”

“小云娘子,小人有礼了。”朱中知道云娘迟早是韩冈的房内人,不敢怠慢,礼数恭敬的问道:“机宜在里面吗?”

“三哥哥就在书房里面。”

韩冈听到了外面声音,放下了书。朱中进来,他就直截了当地问道:“是不是又打起来了?”

“伤了四个。”朱中忧心忡忡点着头。他也不奇怪韩冈为何能未卜先知。古渭疗养院有三栋病房,根据伤病的种类而区分,里面有汉人,也有蕃人。因为风俗、习惯、语言等方面的差异造成的分歧,两边总是针锋相对,吵架、打架都是很寻常的事情,朱中没少骂过他们,但还是没有用处,很有几次快要从内科病房出院的病人,转眼就送进了外科去住了。

也幸好单是跌打损伤这一项,疗养院的水平是外界的骨伤郎中所不能比,等韩冈招安了一批骨科郎中,加之石膏、夹板的运用,疗养院已经超越了这个时代。才不会因为内部的冲突,给世间添上一群残疾。

韩冈无奈的摇了摇头:“就说是我说的,打人的自己出来认罚,还要照数赔偿人员损伤的诊金和药费。”

朱中本就是为此而来,得了韩冈的命令,又聊了两句,便立刻告辞离开了。不知是因为在意疗养院的事,他是小跑着出了门。等到午后,王厚找了过来。听韩冈提起此事,他也是摇头失笑:

“玉昆你的伤病营里,都是年轻力壮的居多,不能让他们闲下来,闲下来就打架。人一闲,骨头就会发痒,肯定要给他们找点事做。还有那些有力气打架的,病好了就踢出去,留在疗养院里给他们养老不成?!”

“军中伤病的诊费药费还有食宿都由上面拨钱下来,但毕竟不算多,能住进疗养院里的蕃人都是各部里面的头面人物,付账从来不小气。疗养院靠着他们贴补呢,”韩冈无奈的摊了摊手。接着又道,“不过处道你说得也是,的确得给他们找些事来做。”

他想了又想,最后用着有些兴奋的语调说着,“当年在子厚先生门下,演射投壶时常有之,天气好时便登山游观。我想可以从这方面着手。”

“怎么个着手法?”

“内科和外科用蹴鞠来比赛,把怨气在比赛中消除,这是让两边的蕃人汉人都学懂体谅对方的道理。”

“……玉昆,古渭寨里脚法好的不多。风流眼在场中那么一竖,十脚里能踢进一脚的,一个巴掌就能数出来了。”

蹴鞠比赛,现在多是一个球门,就是在球场中央立一根一张高的杆子,上面竖一块木板,木板中的孔洞就球门。真要韩冈来说的话,现在的这种比赛可以说是花式足球,表演的成分居多。所以他看不顺眼:“设什么风流眼?!直接两边安球门就是了。”

能把足球往篮球筐里踢的的确是高手,但这样的比赛对抗不激烈,没有多少刺激性,韩冈看过一次,就失去了兴趣。要知道,在汉代蹴鞠可是正儿八经的军中练兵之术。就是在唐朝,也是激烈得紧,哪里是如今这般软绵绵的运动。

韩冈打算将规则改造成对抗性更强的现代足球,有关足球的规章制度本就有蓝本,韩冈毫不费力就能整理出来。简单、直接,让吐蕃人也能很快的适应规则。不过韩冈向王厚解释的时候,却说自己遵照的是古法,是复古,毕竟在唐时,蹴鞠运动还是以为双球门为主。

