\section{第15章 自是功成藏剑履(五)}

福宁宫中,赵顼正在听取勾当皇城司公事石得一的报告。

京城之中的流言蜚语,赵顼都能通过皇城司辖下的探事司在最短的时间内得到汇报。而对于他所关心的话题,专司京城伺察的探事司逻卒——也就是俗称中的察子——也能在数日内给予回复。

为了不给朝臣们所欺瞒,以及了解民心动向,对于直抵京城民间的耳目,赵顼一向看得很重。石得一这名看起来并不起眼的内侍,也就是靠了他在探听消息上的长才,成了赵顼所看重的宦官之一。

虽然权位远远比不上正在殿外统领班直宿卫个宫掖的王中正,或是犹在河东的李宪,但能贴近天子,差事又是查人隐私,还有密奏之权,在朝堂上,被文臣提名道姓叱骂的次数在内侍中可是数一数二。

听过了石得一禀报有哪些朝臣在国丧之期依然在私底下饮宴的报告,赵顼漫不经意的问起另外一桩他更关心的事:“韩冈在民间声名甚广,这一次他受弹劾,京城军民是怎么看的?”

石得一心中无奈,终究还是提到了这个问题。斟酌了一下言辞:“回官家的话,外面多说御史台的不是。韩冈献种痘法,救了天下百万幼子,不过杀了两万党项贼而已,就要受人弹劾。都觉得御史台只帮党项人说话。”

“就这些?”赵顼的脸上看不出有何变化,平平淡淡的追问着。

石得一犹豫了一下,又道:“……更有甚者,还讥讽御史台拿了西夏的俸禄,为西夏尽忠……这是国子监中传出来的说法。”

赵顼哈的一声笑:“韩冈还真得人心啊。”

就如监察御史们想踩韩冈上位,多是年轻气盛之辈的太学生们,其实也看不起那些御史。对四夷毫不留情的韩冈,最合年轻士子的脾胃。国子监的舆论对为党项人说话的御史,当然不会有好话。可如果将双方换个位置,御史要杀党项,而韩冈阻止,那么太学生们同样会反过来大批韩冈。

石得一知道以当今天子的才智肯定能了解这一点,可天子的诛心之言,还是让他心惊胆跳。

他偷眼瞟了一眼赵顼,依然不得要领不过伴君日久,还是知道怎么说话。石得一话锋一转,却道:“不过官家以密诏责问韩冈,而不是明正其罪。城中军民听闻之后皆赞官家处置有方,正如当年太祖皇帝处置李汉超一般。”

“哦?”赵顼的表情终于有了些变化,扬了扬眉,“当真有人这么说?”

“回官家,千真万确。臣不敢欺瞒陛下,改易一字。”

太祖皇帝和李汉超之间的事,赵顼知之甚详。听到外面拿太祖待李汉超事来比拟他对韩冈的处置,心中甚喜。

李汉超乃国初名将,为太祖皇帝所重用。以关南兵马都监之职镇守河北北疆,抵御辽人入寇。在军事上,李汉超做得很好,但他私下里却做了许多犯法之事。甚至强索一富户四千贯,不肯偿还,并劫掠其女为妾,逼得那富户来敲登闻鼓,将状子递到了御前。

太祖皇帝得知此事,亲自召见了这名富户。先以酒饭好生招待了,之后问他:“你的女儿原本要嫁给什么人?”富户答道:“庄户人家。”太祖又问:“李汉超未来关南的时候,契丹对你们怎么样?”答曰:“岁岁苦其入寇。”再问:“现在还是那样吗?”富户则摇头道:“不是了。”太祖由此便质问道:“李汉超乃朕之贵臣,你女儿能嫁给他做妾,岂不强于做农妇吗?假使李汉超不守关南,你还能保有家人财产吗?”将富户问得哑口无言。

不过太祖皇帝要是这么偏袒守臣,也不会这么让人敬佩他的手段。

等他将告御状的富户责遣之后,转回来,赵匡胤又遣使去质问李汉超:“家用不足,为什么不告诉朕,而向平民百姓告贷?这一次朕且宽贷你,以后这样的糊涂事决不能再犯!”并赐给李汉超银数千贯的财物;“速将借贷和人家的女儿还回去,日后如有所阙,可向朕来要。”

因为太祖的这番回护,李汉超感激涕零,并誓死报之。镇守关南十七年,军政皆有所成就,得士民敬服。

这就是太祖皇帝御下的手段,赵顼一向是极为佩服的。之所以同意蔡确的提议,也正是想到了太祖皇帝的先例。

如今御史台的弹劾如同狂风暴雨,遇上这样的情况,就是当朝宰辅也支撑不住,只能避位待罪。现在赵顼将弹劾都拦住,只下密诏责问,做臣子的只有感恩戴德的份。而之后论功行赏,谅韩冈也不敢奢望侧身西府。

这是最好的手段。赵顼所欣赏的祖宗之法,是包括异论相搅在内的御下之术,比起已经陈腐不堪的法度,控制朝堂的御下手段,才是万世不磨,值得承袭的宝贵遗产。

正想着,却听见外面通传说是干管通进银台司的宋用臣求见。

依例只有军情才得连夜送入寝殿,赵顼一面猜度着不知又是哪里的军情,一面招了宋用臣进来。

“宋用臣,是哪里的军情?”赵顼问道。

“官家,是河东捷报。”宋用臣双手托着一封实封状,一个字也不多说。

“捷报?”赵顼从鼻子里发出一声笑,“这一回又是多少斩首?两万五还是三万?黑山党项怕是都给他杀光了来换功劳。”

他边笑着,边接过用火漆和河东路经略司印封缄的捷报。

展开来,赵顼只看了几行字,呼吸便是一滞,表情也顿时变了。

用眼角的余光发现天子展着捷报的一双手轻轻颤着,双眼死死盯着奏章,脸色一阵红一阵青。石得一心中疑云大起,瞥了宋用臣一眼,却只见他垂头看着脚尖,身子如同枯木一动不动,连呼吸都轻了。尽量减少自己的存在感,这是宫中为防迁怒时最标准的做法。显然宋用臣已经知道天子看到了河东奏报后定然会由此反应。

韩冈到底报上的是什么捷报啊?!石得一疑惑难捱的心中大叫,随即学着宋用臣的样子,做起了木雕土偶。

随着时间的变化,殿中原本还算轻松的气氛,一点点的僵硬起来。越来越多的内侍感受到了天子心中正在酝酿积蓄的怒火。无一例外,他们都学着石得一和宋用臣的样子,一点动静都不敢发出来。

不知过了多久,在静如子夜的大殿中,忽然出现了一声压抑到极致的低语:“下去……”

石得一愣了一下,“官家?”但宋用臣跪下来的一声‘奴婢遵旨’,立刻让他后悔不迭。

“下去!!”赵顼随即一下提高了嗓门,厉声道,“你们两个都下去!”

石得一如释重负,同样跪下来磕了几个头,飞快的小碎步,与宋用臣一同倒退出了殿。

赵顼坐在御榻上,心中羞怒交加。来自河东的这份捷报,不仅是韩冈回击御史台弹劾的最佳武器,也让他这个天子在万民面前丢尽了颜面。

辽人突袭胜州,归附的黑山党项在契丹奸细引领下起兵呼应,幸而河东军早有所备,将计就计大败辽师。

这一战,韩冈是眼光长远,深谋远虑,洞悉了辽人的奸谋,让胜州得以保全。可御史们便成了在定国安邦的贤臣背后捅刀子的小人,让亲者痛仇者快。他这个皇帝,也是不辨是非的昏庸之君

以赵顼对臣子们的了解,御史之中肯定有得知这份捷报也不肯服输的人。到时候,改为弹劾韩冈挑起边衅,那更是在天下人面前坐实了奸臣陷害忠良的判断。

“王中正!王中正!”赵顼提声唤了两句,这才想起来王中正今夜是在殿外领班直宿卫。便命殿门处的黄门,“童贯,去招王中正来。”

童贯听了吩咐,连忙转身出外,片刻之后,王中正就奉旨匆匆入殿。

赵顼没有多言,只是让人将河东捷报交给王中正。

王中正一看,才知道为什么方才在外面见到石得一和宋用臣时,正在交头接耳的两人的表情会那么古怪。

的确是皇帝做得岔了,脸皮都给刮下来了。而且天子为什么要招自己过来,也能猜个八九不离十。

“官家。”王中正没有蠢到恭喜赵顼胜州大捷,而是小心的问道,“辽人在胜州输了一阵,是不是要河北加强防备?”

“有郭逵在,担心什么?!”赵顼怒道。他哪能看不出来,王中正这是在试探自己的态度,会不会以挑起边衅的罪名去责罚韩冈。

他怎么可能那么做,还要不要脸了!?赵顼现在想的是怎么挽回局面。

王中正放下心来,沉声道:“奴婢也读书。亦知君为父,臣为子的道理。三纲五常,父训子过,就是说岔了一两句,难道做儿子的还能记恨父亲不成?韩冈是当世名儒,纲常上当不会错的。且由草莽简拔韩冈入官,不正是陛下?下密诏叱责韩冈,却也是怒其不争的一时之误。换作是寻常臣子,陛下如何会为此激怒?直接交由有司依律处置便是。正是因为看重韩冈,才故而分外见不得他行差踏错。俗语中说的恨铁不成钢便是这个道理”

王中正的一番宽慰,让赵顼的心情稍稍好了一些,叹了一声,“王中正,你素知兵事。看这事该如何处置才不伤军心?”

王中正哪里敢多掺和,那是嫌死得不够快:“朝事非奴婢敢言……不过陛下的密诏,是不是先派人去追回?”

赵顼点点头,却又担心起来,已经出发两天一夜,还不知能不能赶得及。

