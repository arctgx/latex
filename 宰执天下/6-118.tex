\section{第11章 飞雷喧野传声教(15)}

小厅中热气蒸腾,两个暖锅中咕嘟咕嘟的响着水声,冒出的水汽,让房内变得烟雾缭绕。

韩冈拿了柄小巧的银火钳,从紫铜打制的暖锅锅底夹了两块木炭出来,让火头小了些。

水汽淡了一点,不过弥漫在房间中香气,依然没有散去。

这是一股很特别的味道,如今大概只有老饕,才能比较容易的分辨出这是海货特有的鲜香味。

天寒地冻的时候,弄了个热汤锅,与朋友一起吃喝,上至王公,下至庶民,都是寻常之举。韩冈自也不能例外,今日休沐,正好王厚不当直,韩冈便请了他过府,弄个海鲜汤锅,再热点水酒,再惬意不过。

从锅里夹起一块海参,韩冈对王厚道:“这东西终于能入口了。”

王厚从自己的锅里也夹了块出来,也不怕烫的直接放进嘴里,嚼了几口,眼睛就眯了起来,“这口味可比过去吃的好多了。”

“葱烧海参更是上品,只是得要会料理。”

“上次的瓦罐红烧肉还是玉昆你的介绍,在家里吃得连羊肉都不想碰了。”王厚咂着嘴,“既然玉昆你说葱烧海参好,回头我让家里的厨娘过来再学学。”

“这好说。”韩冈简单的就应下了。

“玉昆。”又夹了一块海参吃了,王厚突然压低了声音,表情也变得有些诡异,“这海参当真能够……那个……你知道的……”

“知道什么?”

“咳,玉昆!”王厚提声,有几分恼羞成怒的样子。

韩冈摇摇头,无奈的笑道:“处道,须知饮食有常,起居有规,良好的生活习惯,比什么补药都好。至于药物食材,的确能有一时的效果,可是火烧旺了,柴也会没了,还是当普通的菜来吃。”

韩冈不通医术,却精通医理,这是世所共知,见韩冈正经说话,王厚悚然恭听。

见王厚神色严肃,韩冈微微笑了起来,虽说说得都是正理,可是能让人如此认真记下,还是要靠自己的名声加成。

不仅是韩冈说的医理让人不敢轻忽视之,就是韩家的菜单,放到外面去也是多少人家争先仿效。

就像今天这一餐,要是传出去今天韩冈请人吃了海鲜,包括海参在内,东京冇城中所有海货都会涨价。

天知道,要不是处理海参的手法终于进步了,韩冈决不会再动一筷子。

前两年,韩冈第一次在这个时代吃海参,结果很糟。可以说,他从来没吃过这么难吃的海参,好端端的材料全给糟蹋了。

倒不是说严素心的手艺不佳,而是渔民在捕捞之后,对海参的初步处理出了问题。

干制海货的技术,在这个时代仅仅是最简单的晒干烤干而已,还没有更进一步的炮制手段。甚至海参这个名词,都是出自韩冈——毕竟现在的人参,在冇此时,还没有几百年后那般的地位,仅仅是《神农本草经》中几十种上品草药中的一味。更没有人将这个名字赋予给海里的奇怪生物。

韩冈并不知道这一点,将海参写进《桂窗丛谈》时也没有多注意。

为了填充字数,韩冈所出版的笔记里面,不仅仅有医疗卫生、天文地理、物理数算等内容,还有各地的风物,山珍海味也包括在其中。这也是为了吸引读者而考虑。但有些时候,韩冈也不免有些疏忽,将只在后世流传的名词,提前搬到了这个时代。海参也只是其中一例。

不过这也没什么,反正这个时代信息流传的速度和广度皆远不如千年之后,也没人看出韩冈的失误。相反的,因为韩冈的权威性,反而让海参就此定名。

海参列名有种痘法出现的《桂窗丛谈》中,便登时成了受到追捧的对象,而且很快就又流言传出,说是此物对男性某方面的机能有让人惊喜的小国——但就像韩冈告诫王厚的那样,海参的这种特别功效,并非出自于他口。

自从市面上能见到海参,收到的礼物中也能看到海参,海参便上了韩家的餐桌。只可惜渔民对海参的处理与处理海鱼一样,晒干了事,而严素心第一次料理海参,是直接像咸鱼一般的烧。

这当然让养尊处优的韩冈完全动不了筷子。

两年了,京东的渔民终于学会如何处理海参。先清理内脏,再用海水煮熟晒干,就跟南方用红盐法、白晒法处理荔枝等水果一样,虽说肯定比不过后世的处理手法,但好歹能让内地尝到远方特产独有的味道了。

现如今海中的虾蟹贝甚至还有鱼,都开始这样处理。这样的处理手段很耗柴薪,可比起单纯的腌制和晾晒,在口味上超出了不知多少。

海鲜锅汤鲜味美,吃一口菜,抿一口热酒,韩冈貌似随意的问王厚:“方才说的,都亭驿那边的情况怎么样了。”

“已经抓到几个了,这两天正在拷问呢。”

“手脚倒快。”韩冈笑道,“过去盯着内城各家宅院,如今换了个地方,看来也不差啊。”

王厚正要喝酒,听了韩冈的话,便停下酒杯,冷笑着:“皇城司的旧人哪有一个能派上用场?”

“是从家里调来的人?”韩冈扬了扬双眉,“他们怎么样?在京里习不习惯?”

“都是会抓老鼠的好猫,在陇西能抓,在汴梁一样能抓。”

王厚沉稳的笑着,这是一名得胜归来的将军,在为他手下屡立功勋的将士而感到骄傲。

王厚受命统掌皇城司,皇城内外皆是他的职权范围。

皇城的安全,由他手下数千亲从官负责。而作为天子的耳目所寄,皇城司的另外一项任务,也是有专人负责。

但这个耳目,也是皇城司最为朝臣所厌的地方。

日常交游说不准什么时候就被这些藏在暗地里的眼睛给报了上去。有些话说的时候不在意,偶有犯忌也是很寻常的,可这个‘寻常’传到了宫里面,就算天子不可能由此降罪,但在心里记上一笔,自己的前途可就黯淡无光了。

说起皇城司下面的探事司,还有探事司下面名为四十,实则数倍于此的察子,哪个朝臣不是恨不得哪天将这个衙门给取缔掉。

石得一当初提举皇城司的时候,便为朝臣所忌。王厚坐上同样的位置后,也是忙不迭的将这方面的事权给丢了出去,只抓着皇城司的亲从官。

不过有一件事是韩冈所托付,亦得太后钦命,王厚却推辞不掉,就是军器监中的机密保卫。

一开始仅仅是防止有人窃取图纸、数据,打探监中消息,渐渐的就变成扫荡京冇城内外的细作、密谍。

这一次辽使进京,加上皇城中那几具巨炮,就像是一块巨石投进了池塘中,连塘底的淤泥都给翻了出来。一时之间,皇城司大获丰收。

不过,现在抓到的,绝不会是全部。韩冈很清楚枢密院和北方缘边各路及边州的官员们派了多少细作去辽国国中。

“鼻子也要好才行,肯定还有漏网之鱼。”韩冈说道。

王厚更加自得的笑道:“玉昆放心,都是鼻子灵的好狗。有两个还是开边时的老人,玉昆你应该还记得。”

韩冈回想起过去曾经在自己手下听命的旧人:“张孝祖?冇封江?还是胡睿?”

河湟开边时,韩冈的工作偏向钱粮军械医疗卫生等后勤事务,而负责内务和对外谍报的便是王厚。不过也没分那么清楚,随军转运的工作,熙河路几次大战中,王厚都分担了一份。而谍报和反谍报的工作,韩冈也多次替王厚掌管,人事上了解很深。他所说的,都是当初王厚手下最为得力的几个人。

“调了封三来。钱云会也来了。”

“钱云会?”韩冈微微皱起眉头。

钱云会是王韶的亲兵,不是王厚的下属,是极阴狠的性子。有一回高遵裕的一个族亲,被自己人砍了脑袋,又被另外的一拨人捡了来冒功,钱云会奉了王韶的命,亲自动手,将杀自家人的几个士兵给碎剐了,事前事后,都是面不改色。

“怎么了?”王厚看韩冈的表情有些不对,也不知道韩冈是不是对钱云会有什么成见。

“不没什么。”韩冈摇头笑道,在王厚和他面前,钱云会倒是十分听话。不管什么事吩咐下去,都是没有二话,“有这两人在,我也就放心,相信辽人派在京冇城的一干细作,都能给他们挖出来。”

“玉昆你放心,已经圈出几个最可疑的了,现在都有人在盯着,吃什么、做什么、与谁联络,都会一点不漏的记下来。”王厚很骄傲的说着,“那些老察子可做不到这一点,他们也就会盯着宰辅和宗室的家门,然后在茶馆里竖着耳朵坐上一天。”

“还有报纸。”

“对!”得了韩冈提醒,王厚立刻就应道,“他们还会再抄抄报纸。”

韩冈笑着点头,而从千万人中,将来自敌国的间谍挖出来,京冇城的察子做不好,而从陇西调来的人做着最顺手。

这就是经验上的差距。
