\section{第21章 欲寻佳木归圣众(20)}

“陛下是不是累了?”

听到侍讲田腴带着提醒的询问,赵煦坐直了身子,开始变声的嗓音有些尖利,“没有!”

用教鞭指着挂在黑板上的地图,田腴忍不住暗暗一叹。

这已经不是第一次了,每当上到自然地理等有关气学的课程的时候,赵煦的精神就经常性的神飞天外。

这其实是一件很稀罕的一件事,为什么天子会那么喜欢经筵上的课程,而对自然地理和算学不那么感兴趣——虽说喜欢也只是相对,但至少要认真得多。

田腴过去曾经接触过很多开蒙不久的小孩子——包括他自己家的,也包括宰相家的——没有一个不对自然地理感兴趣。而经书,则几乎人人视为苦差。至于算术,普通孩子都一样觉得头疼,不喜欢这门课没什么值得奇怪的了。

若说这其中没有缘由,田腴怎么会信?多想一想,就知道到底是什么原因。

讲学在睿思殿的偏殿中,入秋后气候宜人,殿中门窗大开,穿堂风吹过整间殿宇。但御桌左右,都隔着一扇屏风,不让屋外的凉风直接吹到赵煦的身上。

小皇帝的脸庞遗传了他的父母,清秀,但下巴略尖,并非世人称道的方面大耳的福相,身子也比同年人削瘦,看着肩膀就窄,旁边陪读的小内侍差不多年纪,但比皇帝就要强得多了。内侍的身体通常要比常人虚弱,可皇帝却连内侍都不如。

小皇帝先天胎里就体弱,每日补药不断。同时又遵照医嘱,每天多走路,习练拳脚。但就只是这样,每到换季,都免不了伤风感冒,让御药院和太医局,上上下下折腾上一番。

田腴对屏风遮住了凉风没什么怨言,减少皇帝生病的次数是好事,也能挡着屋外的风景,免得这位学生分心。

讲解沧海桑田变化的地理课程很快就结束了,接下来的算学课程也是由田腴来教授,尽管在数学上,田腴的水平并不算很高,远不如司天监中的一干人,不过给天子教学已经足够了。但皇帝依然兴趣聊聊,田腴所出的一张考卷,竟然没有达到六十分的及格线。

田腴紧锁眉头,从考卷上几道错题来看,皇帝回去后完全没有复习,以前做错的题依然错了,而上一次教授的内容,也同样没有理解。

“陛下,君子六艺,不可偏废。数算乃六艺之一,纵圣人亦要用心。”

田腴的批评很直率。纵熟读经书,也不过是个冬烘罢了,想要经世济用,求实之学必不可少。皇帝若成了冬烘先生,国家日后不知要受多少折腾。

赵煦仰起头:“朕曾听人说,赵普以半部论语治天下,可有此事?”

皇帝的言外之意,自是半部《论语》便能治天下,《九章》能吗?

田腴心一沉,有些后悔自己之前的话说重了,但随即心思又坚定起来。一点逆耳的话都听不进去,竟然还反驳,这样下去如何了得?

智足以拒谏,辩足以饰非,这样的皇帝,对天下最为危险。

“韩王【赵普】虽不读书,却是太祖谋主,精于谋略,有管、乐之才。为相后读经书,不过是将过往辅佐太祖的阅历,与圣人之言相印证。圣人之言,用于人,教以仁,不明事理,纵能倒背如流,也不能说贯通的。”

赵煦脸上更加阴沉,默不作声,殿中一时山雨欲来。

旁边的内侍忽然扯了一下赵煦的衣角,赵煦回头看了一眼,便咬了咬下唇,低声对田腴道:“侍讲之意,朕知道了。”

田腴暗暗摇了摇头。并非他要用这种语气说话,只是没办法。

当今天子心中对气学的抵触,尽管他自以为藏得很好,但田腴怎么可能会看不出来?

这与头脑无关,再聪明的小孩子,也还是小孩子。只要一个有阅历的成年人有心去观察,态度上的差别,很简单的就能察觉得到。

就算是在人心如鬼蜮的皇宫中,赵煦自出生就有着独一无二的地位,可不是那些兄弟众多的皇帝,从小就要提防着别人,小小年纪就历练出过人的城府来。

既然皇帝对宰相倡导的学派不抱好感,那么他对宰相的态度,自然就可以推断的出来。

田腴清楚,没有哪个皇帝会喜欢辅政大臣的,只希望他日后能知恩图报,没有韩冈,他现在要么在地底下陪先帝,要么就被圈禁在高墙之内,哪里还能有现在的地位?

不过以现在的情况,不得不让韩冈介入进来。

结束了今日的教学,隔了一天,田腴趁着夜色悄然来到韩冈的府邸。

“官家的功课怎么样?”

稍许寒暄之后,韩冈开门见山的问道。

田腴道:“若与下官过去见过的那些学生比,自然地理,可算中上,数算,则是中等。”

“也算不错了。”韩冈笑道。

“经义一科,陛下已经可以通读五经,几可算得上是神童了。”

田腴冷着脸,他已经说得够直白了。他一直很感激韩冈对自己的提拔,但身为天子之师,明着泄露太多有天子的消息,未免有失忠孝之道。不过现在是不得不说。

“还差一点。”韩冈道。

以赵煦的年纪,十三经中至少要在不加句读标点的情况下畅诵十经,才能算得上出类拔萃。

前两年,韩冈听说有个叫朱天申的孩子,十岁上下,就能通读十经,被当地官府以神童之名推荐入朝,不过到了路中就被挡下了。天子正年幼,却送一个神童上京,把皇帝给比下去。当事的孩子没好处,推荐的官员也同样没好处。

“不过也没差多远了,比我家的几个孩子都强。”韩冈又道,“以天子的才智,自然地理和数算,成绩的确不当如此。”

“下官教授无方,不能让天子乐于其中。”

“诚伯,莫要妄自菲薄,否则可就是我见人不明了。”韩冈微笑着。

虽说好的学者不一定是好的老师,但田腴也给韩冈的儿子当过一阵老师的,同时也为了确定《三字经》和《幼学琼林》是否适合开蒙,而专门去教了一阵子的书,教学水平远在合格之上。更重要的,田腴还有从军的资历,又做过边地的亲民官,有阅历,有见识,是气学的中坚人物,要不然,他也不会他推荐田腴去当侍讲。

“这个侍讲,当真难做。”田腴苦笑道,他久在韩冈幕中,说话没有那么多顾忌,“真羡慕黄勉仲【黄裳】,邵彦明【邵清】了,不在朝中,不须烦心。”

“黄勉仲要担负几万条人命,邵彦明也得奔走陕西各州县上,可都不轻松。”

田腴、邵清是在《三字经》上列名的作者,其中田腴还是《幼学琼林》经义部的作者,其名气之大,还在周邦彦、黄庭坚这等才子之上。

不过两人都不是进士出身,为官时间又不算长,一路学政自然没有那个资格,不过一路之下,专责蒙学的职位,却是没人能争得过他们。有为天下幼子开蒙的《三字经》在前,又有《幼学琼林》在后,在开蒙教学上,自然有着别人难以比拟的资历。

邵清就在陕西,以提点学政副使名义,负责蒙学方面的工作。陕西是韩冈的根据地,蒙学的基础很好,气学的根基也深厚,而且比起富庶的江南、京畿,陕西的百姓对丁税减免看得更重,富户也更加重视自身的地位而不是财富,邵清去陕西,工作更好展开,立功也会容易许多。

而田腴在教学上更出色一点,加之他编修过的著作,比邵清还多了一个《幼学琼林》,所以被韩冈推荐到天子身边,负责教授自然科学方面的课程。

一个已经成了路一级的监司官副职,一个是天子身边人,不论哪个都是让成千上万官员羡慕不已的好差事。不过这些差事,也都不是轻松的活。

“腴为侍讲,万一失职,免不了要为天下人斥骂。”田腴叹道。

“诚伯安心,天子如今年幼,再大一点,会有所改正的。”

“……但天子已经不小了。”田腴沉默了一下,忽然说道。

韩冈本还在微笑,只是看见田腴的神情,忽然感觉不对,神色一下郑重起来,沉声道:“皇帝还小,还得过几年才是。”

田腴摇了摇头:“女子二七而天葵至,但并不是人人都是在十四五来,有十一二,也有十七八。”

“皇帝的身边,太后都安排了老成的人服侍。”

韩冈觉得,如果当真有这方面的事发生,怎么也该有些消息传出来。

“这宫里免不了有坏了心的人,不顾皇帝的身体,想要得到些好处。离天子大婚的时间,也没有多久了。”

韩冈忽然抬起眼,盯着田腴,田腴的视线没有避让,与韩冈对视着。

皇家嫁娶,基本上都是在十四到十七岁左右。

仁宗大婚是在天圣二年十一月二十一,论周岁是其十四岁半。若以其为例,赵煦大婚差不多还有三年的时间。就算拖长一点,也不可能超过十七岁。

照常理,三五年内,天子就要大婚。之后,到底是如章献明肃皇后故事,一直垂帘到她撑不住为止,还是请其撤帘,让天子亲政,都是一个问题。

所以连天子大婚的相关事宜,现在还没人敢说出口。太后的权威是一条,天子的身体状况也是一条,让人必须细加思量,否则站错了队,性命或许无忧,但前途就不用再费心了。

田腴现在提到大婚,也不免让韩冈多想一层。

不过他也了解田腴的为人,想了一想,道:“等我与太后说一说吧,如果当真有事,尽量悄悄解决。天子年幼失怙,又出了那等人伦惨事,不能再出什么事了。”

田腴沉沉的点了点头,的确是这样。

一个人的经历,会影响到他的性格。没有哪位朝臣不担心小皇帝的性子,田腴如此忧心忡忡,也是这道理。

等田腴走后,韩冈没有再见客,在空无他人的书房中,静默了良久,忽而洒然一笑。

当天下成百万的幼子开始学习自然科学,一个小孩子的叛逆又能做得了什么?

大势既起,总是贵为天子,想要阻拦,也不过是螳臂当车。

抬手将桌上《幼学琼林》的手稿拿了来。

有这个空,他还不如多校点一下修订稿,再多考虑一下大理的战事。
