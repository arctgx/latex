\section{第32章 江上水平潜波涛(中)}

穿过老旧的侧门,回到太医局的徐玑,就跟往常一样平静从容。

与医师打招呼,与医学生打招呼,甚至看见连洒扫庭院的老兵都会点头致意,总是谦和有礼的徐玑有着一个好人缘,没人知道他刚刚把天的健康状况透露给外人。

日历、起居注都是属于国史的一部分,是严禁外泄。而天、太后的病历,同样是机密的机密。身为御前医师,这件事徐玑当然明白,一旦事发,他会是什么样的下场。

半年前,徐玑还会为泄露了天吃鱼致腹泻的详细病情,而整日整日睡不着觉,幸好身边人都以为他是因为怕丢了差事,所以才会如此忧心。

但现在,他早就习以为常,甚至就在门前完成交易,都不觉得有什么好担心的。

根本就没有人去在意。

天小恙,登时就是多人会诊,什么消息能保密?

天的病历,虽说进了太医局就会锁进架阁库,但只要想知道内详情,收买一个架阁库的小吏就够了。

要不是因为在病历上,不能将天的健康状况全部说明,徐玑甚至都不会被人花钱收买,那些小吏的价码更便宜。

仔细考虑过之后,徐玑也不会去担心有人逼着自己去给天下毒。

每一张开给天的药方,至少要经过三位御医的眼睛,而天服用的汤药,不仅仅在熬制时,就有多人监视,熬好之后还有人专门尝药。

任何一帖药,医官开出药方之后,便已经与他无关了。区区一介医官,想要给天下毒,从制度上根本做不到。

既然如此,他还有什么可以担心的呢?再烫手的钱他也敢拿。

至于外界对天身体情况了若指掌,又与他何干?太医局,本来就没有禁止谈论的说法。

比如做太医局正的雷简,每次徐玑从宫回来,他都会问上一问。

“官家的情况怎么样?”

“只有些没精神,可能是累着了。”

徐玑坦然回答。雷简是宰相的心腹人,当年那位相公还没发迹的时候,雷简就跟他认识了。徐玑曾经几次听过雷简吹嘘,当初替父服役的那位相公,要在伤兵营开展卫生护理的时候,是他眼光独到,力排众议,又大力支持的。

雷简对天健康问题的关心,保不准就有宰相在背后指派。

“不会是病兆吧?”雷简不知想到什么不好的事,皱眉问道。

“应该不是。”一名刚回来的医官插话进来,“上个月我去给太妃看病,看见官家,比去年胖了一点了。”

“李三,宁德县君怎么样了?”雷简问着那位医官。

“只在拖时间了,不是这个月,就是下个月。”那医官一脸的浑不在意,又将话题转了回去,“我觉得官家真不会有什么病,身体养得不错了。”

雷简摇头:“去年官家长身体,那时候显瘦,如今也就腮帮有了点肉。胎里就弱,能好到哪里去?”

“那官家今年明年还能大婚吗?”

“应该没问题。”

徐玑回道,正想再问雷简,李三先插嘴进来:

“当然没问题,十二岁就能开了荤,现在当然更不用愁。”

雷简道:“身骨哪里吃得住?没看个头才几尺?徐五,你说是不是?”

“依官家的相貌,其实已是难得的标致后生,一点不尽如人意的地方,也是天道。”

徐玑说话小心,雷简和那李三同时一声冷笑。

“官家是心思太重,所以长不高。”雷简说道。

“心思能不重吗?”李三冷笑道,“内有太后,外有权臣。”

雷简脸色顿时有些难看,“胡说什么!哪有什么权臣。”

李三连忙道,“我说的是章相公。韩相公如今嫌事多,一心放在气学上。苏平章也跟韩相公一样,都不怎么管事了。朝堂事,全都是章相公发落,要不是知道太后对韩相公一直看重,还以为韩相公要倒台了呢。”

过去是两府是两党并立,东西对峙,但自从章惇执掌政事堂之后,新党势力大张,韩冈又不爱争权夺利,只管着自己的一亩三分地,铨曹四选现在就流内铨和三班院在他手里面了,却还只顾着蒸汽机,让章惇在朝堂上越发的一言鼎起来。

“韩相公只是不喜庶务。”雷简为韩冈辩护,“百年树人大计,韩相公可从来没让章相公沾了边。”

“百年树人?照我看这是耽搁人家年。”李三道:“三年蒙学,三年小学,然后还有三年学,还没研习经典,就先是要上年学。除了富贵人家,哪家能让自家弟不事生产的上年学,还不能去考解试?就是富贵人家,让弟读书,也是为了让他们光宗耀祖,不是平白耗上年。”

“岁上学,年课上下来,也算是打好基础了。十五岁之后,不论是钻研经术,还是研习实务,甚至去学兵法战策,都能优而为之,比翻来覆去读上几年经书都要强。”

“有这年研习经典,就能倒背如流了。再有两三年游学,解试也不为难事。若是有个夙慧,进士都能拿到手了。”

“考上进士的,世间能有多少人?”

让徐玑自己来看,还是韩冈的方法更好一点。

许多士人之所以每每饮恨科场,不是才智不足,都是根基上没扎牢,若是从小能打好基础,不说进士,普通点的诸科不会有多少难度,那可是韩冈特意留给自己门徒的自留地。

早前从关西传来消息说,韩冈在关西试行三年蒙学、三年小学、三年学的新学制,如果实证有效,便会如蒙学制度一样,推广到全国。

三年蒙学的教科书,都是韩冈这位大儒带领气学门生精心编订,又在关西推行有年。

课程安排得十分严密,更是面面俱到。一年三学期,年假、暑假、春假。每个学期多少课时都写得分明。语、算术、自然、地理、体育、历史,占满了除节假日外,所有的白天时间。

课本上分纲列目,哪一章讲几节课,又该布置多少习题,最后如何考核,什么样的成绩才能算是合格,都详细的编列出来。

对于平民百姓们来说,他们家的孩在蒙学三年出师,认识千字,又通数算,天地理自然博物的东西也装了一肚,连算盘都能打得噼啪响,身体更是在体育课上练过了。做什么都可以,哪家店铺不喜欢这样的伙计?能识断字,与人议亲也多了一份能说到的地方。

而对于士大夫们来说,见识广博从来不是坏事,三年蒙学培养出来的学习习惯,比起私塾、家学、族学里那种一个老师拿着戒尺灌输,也要强出很多。

这样的教学体系,徐玑此前闻所未闻。而成果,只看关西的读书人一下多了几倍,就知道有多大的好处。福建因为印书坊多如牛毛,书价贱如草纸,所以有了这么多的福建进士,而关西,西夏灭亡后百姓自此安居,又有宰相精心栽培,数十年后,进士、诸科的科之人,又会有多少出自关西?

因为关于宰相的议论让人心悸,又扯了几句闲话,三人各自散去。

但太医局,类似的闲聊却十分常见。

或是如今天一般东拉西扯的争论,或是简短的一两句话,京师里的大人物的身体情况,各个翰林医官心都能有点谱。

虽然大人物们都有自己用惯了的医官,可哪天有个意外,突然被拉上阵,不用单靠不一定靠谱的病历,也能做到心有数,不至于唱歪了调。

而更多地,还有那些家长里短。说道消息灵通,御医们可也不输他人。

公厅安静了下来,雷简又翻起眼前的公。身为太医局正,除了治疗任务之外,还意味着要处理局的政务,相对于其他医术高明的医官,雷简的工作就多是这些日常政务了。

江宁发来的急件刚刚送到他手,楚国公王安石的病情总算是稳定下来了,连夜赶去江宁的宰相家的妻儿也总算不是去见上最后一面。但一些成药需要太医局这边支应,所以发了过来。

第一次风,只要救治得及时一点,一般不用担心性命安危

王安石身边就有两名翰林医官,一内科、一外科,带着出自京师的全套人马,在江宁设了一座医院。医院就开在半山园旁,金陵书院之侧。

那位担任院长的内科医官,手下的医师和医学生们,除了日常门诊,以王安石这位元老为首的江宁官员,就是他们关注的重点。而王安石本人,更是有这位翰林医官亲自负责。

所以王安石一发病,立刻就得到了最好的救治。

命保住了,行动有些困难,说话也含糊了一些,不过意识很清醒,可以说是不幸的万幸。

保住了王安石,江宁医院上下必然会受到褒奖,但风这病症,一次比一次凶险,下一次可就不一定了。

提笔批复了这份申请,雷简心想,不知那一位现在会是什么样的心情。

韩冈这时候正安心的舒了一口气。

早几天,在自家妻儿还没抵达江宁的时候,他就从信报得到了王安石病情平稳的消息,而现在,确认了王安石脱离危险期,更是让他彻底安心了下来。

在学术上虽是敌人,可韩冈也从不会希望王安石有什么不测,倒是一直希望他能长命百岁来着。

自家的年纪也不小了,再过三十年,自己也多半会步此后尘。韩冈想着。

洛阳的那位相公,八十多岁还鹤发童颜、行动如风。年初进京来,在朝堂上声如洪钟,让太后都羡慕不已。看他的模样,说不定真能活到一百岁。

彦博以耄耋之年,还能如此精神,韩冈看了也只能表示羡慕,而王安石突然发病,则让韩冈想起了先帝赵顼,让他再一次警醒。

红烧肉是不能吃了……肥肠也是。
