\section{第40章 岁物皆新期时英(五)}

刚刚送走最后一名来拜访的官员,蔡确今天的工作这才算是一个结束。

不过并不是说之后没客人了,接下来还有私客。

“去请刑和叔来。”他吩咐着亲随,转身跨进门中。

脚下蹭过门槛,感觉有些一样。低头看过去,花厅门槛正中央的一段,不知何时,已经凹了下去,在灯笼下还闪着光,竟是给磨得光滑锃亮。

“这里也该换个新的。”蔡确指了指门槛,对另一位亲随吩咐道。

这名亲随跟着蔡确久了,点头答应之后,又凑了两句趣话:“正门那边的小门,最近才换过呢。都是想要来拜见相公的。”

蔡确喝着茶,随口道,“找个硬一点的木头。”

“木头恐不堪用,非得上等精钢才够呢。”

蔡确摇摇头,宰相府上门庭若市,换作是刻薄一点的天子,不会容忍太多。不过现在是太上皇后秉政,也就不需要担心什么。

门槛被磨下去越多,就代表着主人的地位越高。蔡家进出客人的门槛,可是半年就要换一次。等到韩绛离开,若能独相朝中,恐怕更是要三个月就换一次了。

但想要成为独相,不是那么容易。等到韩绛离开,剩下的执政中,章惇的资历还不够,张璪、曾布和苏颂更不可能。可是并不是说不能从外调选老臣回京就任宰相,吕惠卿也不是没有可能。

按正常的想法,是不会出现这样的情况。可女人的心思都是跳着走,蔡确是清清楚楚的记得司马光是怎么完蛋的,这辈子都别想再起复了。

若是同列,就算城府深沉如吕惠卿、章惇之辈,他们的心思也不难揣摩。韩冈就更容易了,不管他有多少奇思妙想,建立了多少功勋,但本质上都是跟王安石是一类人。只要掌握了他心中最根本的夙愿,一切就好办了。可是太上皇后的心思,想要真的猜透,真的是蔡确力所不能及。

刑恕很快就到了,蔡确笑着站起身,迎接刑恕进来,“和叔,可是久等了。”

刑恕是蔡确很看好的年轻后辈,这段时间,越来越觉得他可信可用。过几曰清理过御史台后,正好将他安排进去。蔡确觉得,刑恕是个聪明人。难得的人才,又知情识趣,而且以他北方人和旧党的身份,只要得到自家的重用,就能够让很大一批渺无前路的人才投到自己门下。

刑恕笑道:“方才刑恕正与博士说话,言笑甚欢,不知时间易过。”

蔡确弟弟蔡硕是武学博士,刑恕被唤过来之前,正是由蔡硕作陪。

“哦,说了些什么?”

“正说今科的举试呢。武举的考官人选已经定了,只是博士说今科没什么人才,比不得文试。但开封府、国子监的解试,八月之前就要把考官的人选给定下来,也不知会是谁来主考。”

“当是由礼房检正举荐,还没有报上来。”

蔡确不管这样的小事。但明年的礼部试,考官的人选安排,他肯定是要参与进来。

蔡确有些人想要提拔,从他这边,想要照顾还是能照顾得上的。只要在文章约定好的位置留下约定好的文字,很容易就让考官知道所要照顾的考生的身份。

不过必须要有真材实料的学问才行,另外,不要贪图高名。一甲二甲都是犯忌讳的,没那个能力,强要往里面挤,事后不甘心的考生,甚至已经考中的进士,都不会轻易的放过。文人能有多阴毒,本身就是文人的蔡确最清楚。

当年太上皇亲自点了叶祖洽为状元,之后照样多少人不服。要不是因为这是天子御笔,考官可都要连皮都给剥了,但之后的叶祖洽,因为得状元那一篇策问中奉承天子太过,在士林和官场中声誉并不好,晋升的速度与状元的身份不相匹配。

而之后的熙宁六年,太上皇将韩冈和叶涛这两个王家兄弟的女婿给安排在了第九第十,又是一场风波。还好韩冈本身实在太强,之后在琼林宴上差点逼得翰林学士杨绘从华觜崖上跳下去,压得一众人等没了声息。

“和叔。”蔡确问刑恕,“你在朝野内外人面都广得很,可曾听说今科有哪些有望一甲的考生?”

刑恕皱起眉头:“各地举人要到秋后方会陆续抵京,能夺一甲二甲的才子,到了考前方能见分晓。现在评定出来的,也就在京的一些才子。”

“哪些?说说。”蔡确饶有兴致的问道。

“若说有名气,京中眼下最显眼的就是黄勉仲和宗泽。其他人都差了一筹。”

蔡确之父名为黄裳,黄裳元吉这个词在人名上用得很普遍,刑恕在蔡确面前,很小心的用表字而不是名讳来提到黄裳。

蔡确很满意他的小心,点头道:“黄勉仲和宗泽宗汝霖,他们两人,我是闻名已久啊。”

“黄勉仲跟韩宣徽差不多,都是立了军功得官,然后回头来考进士,只是年纪大了点。又因为河东战事耽搁了学业。今科能不能中,还真说不准。”

“他是运气不好。”蔡确说道,“十几年前在福建士林就已经很有名气了,我都听说过他。在南剑州拿过乡荐第一,在历次解试中从来没落出前十。只是时运不济。这一回在韩玉昆幕中立下了大功,若是考不中,韩冈递上奏本,太上皇后怎么会驳他的面子?三十多快四十了,多少都会有些著作,献上去,一个进士出身朝廷不会吝啬的。”

“相公说的是。”刑恕低头道。

“宗泽长于兵事,在报上的点评都是真知灼见。如果能上殿,说不定也能得一个好名次。”蔡确点评了一下宗泽,又问道,“除了他们两人,还有谁有些名气?”

“还有刘燍,上一科本是省元,但犯了庙讳藩邸名,不得已被黜落,不过被国子监录为学录,今科卷土重来,也是争夺一甲二甲的人选。”刑恕想了想,“此外若再说有才学的,开封府内的刘槩、冯解也都可争夺一下一甲进士。”

“国子监呢?没人吗?”蔡确问道。

“国子监中有才的早就是公考、校定皆优等,直接进士及第了。余子碌碌,不过争一个进士,一甲是不用想了。”

蔡确点点头,刑恕算是说得有理。真有才学的学生,在国子监三年,早就一路升到内舍,然后通过考试直接出来做官了。

有才学和没才学的差距很大。只要不是运气问题,比如黄裳,或是自己犯糊涂,比如刑恕方才所提的那个犯了庙讳的刘燍,考中进士几乎是必然。争的只是名次高下。

蔡确中进士是在嘉佑四年。在之前的两年,嘉佑二年也曾参加开封府解试,不过未能拔贡。

嘉佑二年那一科的苏轼兄弟、曾巩曾肇兄弟,章惇章衡叔侄,早早的就知名于众考生中,没人怀疑他们能不能中进士。

就是吕惠卿,也因为家世的缘故名气很大。吕家的这一辈,最长的吕夏卿跟王安石同科,进士第九,之后吕家进士频出,到现在快有十人了,这只是同辈,皆以卿为后字。

蔡家也是如此,蔡确参加科举前,进士已经出了好几个。而他本人,嘉佑三年拔贡,嘉佑四年的时候,早在考前,也成了夺一甲呼声很高的考生之一,另一个是弃了前一科功名再来参加考试的章惇,此外名在高第的安焘、刘挚同样早早闻名在外。最后不出意料,几人名次都在前列。

“对了。”刑恕忽然道,“相公或许不知,现在民间已经有赌谁是今科状元。”

“什么?!”蔡确本是聊天的口气,一下就变得坚硬起来,但脸上很快又浮起笑容:“五千人,怎么赌?”

“赌籍贯。国子监是一赔一分五,押十文钱,如果中了,就返回十五文,开封府的赔率与国子监相同。”

蔡确再次收敛了笑容,冷然问道:“福建呢?”

“福建一赔一分二,这是最低的。至于赔率最高的,就是广东、广西和夔州三路了。尤其是夔州路,是一赔五十。”

“秦凤路呢?”

尽管从经略安抚使司来计算,秦凤、熙河、甘凉都是读力的路份,但从与科举发解试有关的转运使司来计算,却都是一个秦凤路辖下。

“因为出了一个韩宣徽,永兴军和秦凤两路的赔率都低了不少,一个是一赔十八,一个是一赔三十五。”

“斯文扫地。把国家的抡才大典当成什么了?!”蔡确咬着牙痛斥道。尽管他许多事都不在乎,但关系到士人地位,却不能当做等闲。

在过去,对所有能考中进士的士人,百姓们都是心生敬畏,目为天上星宿。可时至如今,科举却变成了赌博的工具。这样的变化让蔡确不寒而栗,什么时候京城军民对文人的敬畏淡薄到了这样的地步?

“所以只是私下里在开赌。”

“御史台是做什么的,耳朵长哪里去了!?”蔡确仍是怒气冲冲,清理御史台的心思更加坚定了。

“御史台肯定知道了。很多地方都在传,皇城司和御史台不可能不知道。”刑恕想了想,“说不定是打算将两大联赛一起给牵扯进来。”

“……动得了吗?”蔡确冷哼了一声,两大联赛背后的靠山,岂是御史台能撼动得了的?撞上去只是找死。

“当然动不了,如果乌台聪明一点,只让去抓开赌今科状元的贼子,这倒是不会有事。”

“不等御史台了。”蔡确站了起来,在厅中来回走着,“明天就去让开封府严查。还有韩玉昆,两大联赛与他脱不开关系,得让他让两家总社找出人来。这件事,必须查个水落石出!”

