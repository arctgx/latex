\section{第28章 临乱心难齐(四)}

【不好意思,有事耽搁了,待会儿还有一更】

韩冈上岸的地方往下游五里就是白马渡,一行人骑马过去只用了一顿饭的功夫。

白马渡,也称白马津,位于白马县城北郊二十里。

作为中原通往河北的一处战略要地,已有千多年历史的白马渡,在战国策、史记,都有提及。而三国时,官渡一战中,白马渡也是极为重要的一个侧翼。围绕着这座黄河上的要津,千多年来,无数战火硝烟,不时掠过这座古老的渡口。

不过如今承平百年,白马渡早就不见了金戈铁马,反而一座人烟辐辏的商贸胜地。即便是在隆冬时节,也能看到来往不绝的商旅。

隔着萎缩的黄河,可以看到对岸的黎阳津。大凡渡口,基本上都会建在河流水势平缓,两岸地势也平缓的地方。白马渡这边也不例外,平缓的水势,使得渡船来往安全。而到了冬天,往往这边当先上冻。等到隆冬时节,冻得如同钢铁一般的河面上,铺上长条的木板,上面再加铺了一层麦秸编成的草席,不仅仅是行人可以在此踏冰而过,就连太平马车也可以碾着木板渡过河去。

今年的天气也冷,韩冈觉得甚至比前两年在关西时,还要冷上一些。只是现在空气干燥,冷一点也不至于让人太过难熬。白马渡这一段的河面早已冻起来了,比方才韩冈去看过的那一段河水冻得还要结实,韩冈沿着大堤骑马过来时,已经可以看到有人就在冰面上铺设着木板。

这是一年一度的例行公事,也不过也要经过知县批准。前两天就由监镇递到韩冈案头上,韩冈看了后就签字画押,照旧例拨了秸秆和木板还有一百贯钱,用来铺设冰上的道路。

韩冈在大堤驻足,下面的一片鳞次栉比的屋舍,就是他的目的地。

白马渡这个镇子,由于是在百年间自发的形成起来,内部规划很糟。从上往下的俯视,可以看清楚,除了纵横两条主干道外,其他的街巷太过狭窄,完全起不到隔火的作用。韩冈翻看旧档,知道白马渡镇基本上每隔三五年就要烧一次。现在看过来,镇中的房屋也是有新有旧,有好几片屋舍明显是近年整体重建过的。

从堤坝上下来,听到消息的白马渡监镇带着人早迎了过来。镇内管勾烟火事的监镇官唤作王阳名,当初乍一听到这个名字,韩冈还以为跟后世有名的儒门宗师同名同姓,问清楚后,才知道差了一个字。

王阳名有着朴实的相貌,看着像是乡农,穿着锦罗绸缎也遮不住一身的乡土气。但韩冈知道,这一位也是天家的女婿——离着东京城太近,一颗石头砸出去,能砸出一堆皇亲国戚来——不过身为皇室偏远支系家的女婿,浑家也不过是个偏房生的宗女,荫补官也只是荫了最底层的一个小使臣。王阳名自不敢在韩冈这位进士及第面前拿大。

隔着远远的就向着韩冈开始行礼,等韩冈到了近前,上来陪着笑问道:“不知正言今日来镇上,可是下官有什么地方做得不妥当的?”

“想来看看冰上的道路铺得怎么样了。另外也是因为最近天气干燥,有些担心镇中的情况。”韩冈知道此时的人在言语上有忌讳,便刻意不提那‘火’字。

王阳名则听得明白,点头哈腰:“正言放心,年年都要防着,今年下官早就安排好了。水缸唧筒、斧锯绳索,都准备得妥妥当当,绝对是万无一失。”

“那就好。”韩冈没多质疑,就算两年前的一场火将镇子刚刚烧过四分之一,王阳名的预备要先去看过后再说。

王阳名小心的在前面引路,带着韩冈一行进了镇中。已经不是韩冈第一次来到白马渡镇,认识他的人不少。见到知县到了,纷纷退到路面上去,看着这位用心于公事、兢兢业业的年轻官人,没少了发自内心的一番夸赞。

“差不到也到饭点了,下官已经让人去准备了酒饭,正言不如先去吃了饭后再去看镇里潜火铺的情况。”

王阳名提议着,韩冈回头看了一下自己的随行人员,也的确都累了,“也好,但要简单一点。”

“下官明白,下官明白。”

上一次招待韩冈,王阳名使劲浑身解数的安排了一番盛宴,可韩冈就着开头的两道菜,吃了两碗饭后,就让人全撤下去了,滴酒不沾。到了乡中,他也都是如此。

王阳名不敢再触霉头,而现在白马县的百姓也都知道韩冈的脾气。不喜欢奢侈,也不怎么扰民,出巡时很少带着旗牌官,不会喊着肃静、避道什么的。此前韩冈轻而易举就将三十年陈案给结定,全县老少都知道如今的小韩知县明察秋毫,没人敢于因为韩冈的轻车简从,而小觑他这个年轻的知县。不扰民,为人又简朴的知县,对于百姓们来说,怎么说都是件好事。

刚向镇中走了几步,却听着路边上的小酒馆中传出一阵丁玲桄榔的声音,还有一阵叫骂声。

韩冈脚步一停,转头望着这家酒馆,向着里面呶呶嘴:“去看看在闹什么?”

一名随行的弓手立刻挺着胸大步走了进去,可一声惨叫之后,便捂着眼睛跌跌撞撞的跑了出来:“正言!是几个军汉,喝了酒不给钱!是宣翼军的!”

就在白马渡不远处,驻扎了宣翼军的两个指挥,归于白马县驻泊都监管辖,用来保护白马渡这个津梁要地。而再向东远上一点,还有一座千人厢军的军营,本属于滑州,用来护卫黄河大堤的,现在受开封府直接调派。

发着酒疯的声音从酒馆中紧追了出来:“什么知县,爷爷还是知州呢!”

韩冈一听,脸色沉了下来,点起两名从军中退出来的家丁:“去将人捉出来!”

王阳名在后面听了,看样子就知道韩冈要籍故来办人了。他跟白马县的禁军驻泊都监郑铎交情不恶,而且王阳名知道,郑铎本人就在镇子中的外室那里。趁其他人的注意力都在酒馆中,悄悄的招了从人过来,“快去找郑都监!”

韩冈身上没有军职,管不到这些赤佬头上,此地的驻泊都监也不受他管辖。但前两次来参见韩冈这位知县时,都监郑铎都是战战兢兢,不敢有任何桀骜不驯的神态。这也没什么好奇怪的,在大宋,武将从来都是要让文官三分。尤其是韩冈这等背景深厚的官员,随便一封弹章,就能让一名都监去琼崖钓鱼。

都监如此,都监手下的士兵当也是如此。四个穿着宣翼军军袍的军士垂头丧气的跪在韩冈面前,方才韩冈的两名家丁进去后,一拳一个,将他们打翻了给拎了出来。鼻青脸肿的,半点气焰都没有。而酒馆的老板捋着袖子气哼哼的站在一边,嘴角破了个血口子,显然是方才被这几个军汉打的。

韩冈低头看着几个军汉,冷着脸问道:“知道本官为什么要捉你们过来?”

军汉哪里敢说别的,只知连连叩头:“小人知错,小人知错。”

“吃白食也不算是大罪,只是本官问你们,吃饭给钱是不是应当的。朝廷若是不发俸禄,你们可愿意吗?”韩冈质问着他们:“朝廷的钱粮养着你们,是为了让你们保境安民的,但你们呢,扰民的功劳多一点!”

韩冈声色俱厉,四人脸色惨白,低头着贴在地上,不敢回嘴。

‘这就是京营禁军?’韩冈暗自摇摇头。换作是西军,却没有这等软蛋,一干骄兵悍将,逼起来直接顶嘴都有的。

韩冈捉了人在这里审,转眼就围了一圈人。看着一群吃白食的军汉跪在地上,镇子里的商户都低声的叫好。而另外十几个同在镇中的禁军士兵,闻讯也都跑了过来。

“店家。”韩冈不理围观群众,问着当事人,“吃白食并非重罪,小过而已。但旧时也有军士拿了民家一顶草帽,而被直接行了军法枭首示众的例子。不知你觉得这样处置如何?。”

周围禁军士兵闻言一阵骚动,但被韩冈凌厉的双目一扫,便一下就痿了下去。

而酒馆老板听了韩冈说要杀人,同样吓了一跳:“这个……这个……太、太重了一点。也不过打坏了几个碗碟,军爷给了钱就好。砍头就……就……”

“听到没有!”韩冈一下转过来,对着面色煞白、已经浑身瘫软的四个士兵,“看看人家的好心,想想你们自己做的事!愧还是不愧!?”

韩冈松了口,死里逃生的几个士兵痛哭流涕,冲着酒馆老板叩头不止,连声称谢。那老板则是手足无措,不知该做什么好。

“本官也知不教而诛的道理,但可一不可再。今日之前,本官未下禁令,那是本官的疏忽。但现在本官已经说了,从今而后,如果再有军士敢于横行街市、欺压良善、怙恶不悛,那本官就不能轻饶了。犯过轻者,少不了一顿好打;重者流放远恶军州;若有想试一试底线的,三尺快刀,本官也有预备!”

韩冈的眼神和口气比起今天地气温还要低,周围的一群禁军士兵听得冷汗涔涔,不敢有半个不字。再看了他们一眼,韩冈转头对着匆匆而来的一个胖子,“郑都监,你说呢?”

