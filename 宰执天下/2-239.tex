\section{第35章 重峦千障望余雪(四)}

终于跨过了选人和京朝官之间的门槛,这让韩冈心中欣喜。只是表露在外面的,依然是宠辱不惊的模样。以他的功劳,早就该升朝官了,现在才晋升,已经是很委屈了。

王韶的嘱托,让韩冈连声自谦:“有经略在,韩冈也只是拾遗补缺而已。”

王韶笑着摇摇头,韩冈能一下跳过了两任知县的资序,成为权发遣的通判,对他来说,更是个莫大的惊喜。以第一任通判的资序,加上经略司机宜文字的差遣,日后担任数万大军的随军转运使,虽然勉强,可也说得过去了。

韩冈治才难得,这是王韶早就知道的事。经世济用的手腕,当然要好好派上用场。

王韶是熙河路经略安抚使,他的治所按理说应该在熙州狄道,也就是过去的武胜军临洮城。但他却又兼任着巩州知州,也就是说他必须熙州、巩州两边来回跑。那么当他不在的时候,巩州的大小政事,也只能交由通判韩冈处理。且在王韶心中,他更为看重的是经略使的工作,至于政务,韩冈就该多担待一点。

“可下官也是经略司的机宜文字,同样也要两边跑。”

“那时就再说好了。”王韶早打定主意,不容韩冈推拒。

厅中的小吏端上了热茶来,韩冈亲手向王韶奉了茶,问道:“不知处道什么时候能回来?”

“当要到明年了。”王韶啜了口茶汤,叹道,“希望他在京中不要犯什么错,丢人现眼。”

“处道为人稳重,历事亦多,只有争光添彩的份,哪会有丢人现眼的事?”

“要是玉昆你一起去,我就不用担什么心了。”王韶看看韩冈,放下茶杯,问道,“没能诣阙面圣,不知玉昆你有没有什么想法?”

“韩冈自十六岁出外游学时起,就没有一次在家过过年节,能在家中陪伴二老,尽一份孝心,也算是韩冈多年的心愿了。”

王韶、韩冈加官进爵,王厚的官职也水涨船高,虽然还没有转官的资格,但靠着王韶这个老子,让他捞到了献俘京中的差事,连着苗授的儿子苗履,两个衙内带着瞎吴叱和一众战俘去了京城,想来也少不了赏赐。

而韩冈今次晋升朝官,照例必须得进京一次,但诏书上,韩冈却没有听到招他诣阙的词句。只是之后王中正从宣诏的中使嘴里探出口风,让韩冈明白了这是天子保全他的用意。

没能上京面圣,韩冈在微感遗憾之中,也觉得这也算是件好事。成为众矢之的的感觉的确不好,而且去年、前年过年时,他都在外面跑着,更早两年,他的前身又在外求学。算起来已经有四五年,没能在家与家人团聚了,而今年总算可以留在家中享享清福。

“玉昆你能这么看得开,也是一桩好事。”王韶对韩冈的洒脱很是欣赏,笑道:“有你在巩州守着,我去了熙州也能放得下心来。”

“有王舜臣在狄道【临洮】盯着,熙州那里当不会有大碍,经略大可放下心来。”

武胜军,也就是熙州那边,包约和禹臧花麻正针锋相对。熙州北部的山岭中的蕃部,都因为他们两家的缘故而祸从天降,估计再过半年,熙州北部蕃部的人丁,能有现在的一半就不错了。

而占据了洮西的木征,则是由狄道的驻军盯着,领军的将领就是王舜臣。

时至今日,王舜臣终于能独立领军,镇守着狄道城。而且他今次因为与苗授一同担任前锋的功绩,顺利的升任正八品的大使臣,与韩冈一样都成了能上殿参加朝会的官员。

说起来,不仅仅是韩冈,整个熙河路的官员将领的晋升速度,都是快得让人目瞪口呆。

王韶从正八品升到正六品,韩冈从布衣晋朝官,都是转眼间事。一个只做了一任县尉便辞官游历边地的小官,三四年后,便已是一方帅臣。而一个穷困潦倒得要服衙前役的措大,不过两年,也已经成了立于庙堂之上的朝臣之一。

武将立了战功后,升官速度一向比文官要快,但如王舜臣入官才一年多的时间,就已是大使臣,也是同样的不可思议。

有了这么些让人叹为观止的前例,到了明年的决战之日,蜂拥而来的官员,怕是能把熙河经略司的衙门大门给挤破。

韩冈想想那时会发生的情况,心中就有些发毛。王韶拼了命的要把他拱上京朝官的位置,也是看透了官场上,追逐功劳就跟苍蝇逐臭一般的凶猛。

希望不要闹得太厉害,来几个能听人话的,韩冈企盼着。

……………………

一只枯瘦刚劲的手,将手中的笔放下。

片刻之前,心神都沉浸在文字间。直到放下笔,一阵疲惫便立刻涌了上来。

张载用手用力揉着额头,而侍立在一旁的吕大临——蓝田吕氏四兄弟的老幺,吕大忠的弟弟——将墨迹淋漓的一页纸,轻手轻脚的手了起来。

“乾称父,坤称母;予兹藐焉,乃混然中处。故天地之塞,吾其体;天地之帅,吾其性。民,吾同胞;物,吾与也……【注1】”

读着读着,吕大临就激动起来。这一段文字虽然只有聊聊两三百字,但分明就是张载所创学说的总纲!将人道纲常与天道自然联系起来,真正的说通了天人合一的道理。

“大君者,吾父母宗子;其大臣,宗子之家相也。尊高年,所以长其长;慈孤弱,所以幼其幼;圣,其合德;贤,其秀也。凡天下疲癃、残疾、惸独、鳏寡,皆吾兄弟之颠连而无告者也。”

这一段是把君臣相处之道与家事相勾连,欲使三纲为一,又融合了孟子所说的‘仁义’。

而到了最后一句,‘存,吾顺事;没,吾宁也。’直接否定了佛老两家的来世、长生的观点,是儒学对生死的看法最简洁的归纳。

活著,顺天应人;死时,无所挂碍,安宁而去。

简简单单的一篇文字,将儒家内外之事全数包容,吕大临手都在抖着:“先生!这是……”

“这是《正蒙》中的一篇。”张载闭着眼睛,声音中满是疲累,这篇文字是他几十年的心血结晶,写出不费多少时间,却很是伤神,“另外还有一篇,等写好之后,我打算刻在书院正堂中的东西双牅上。”

张载正在说着,忽然惊道:“与叔,你什么时候来的?!”

“已经来了一阵了,见先生正在写文,不敢惊扰。”

“可有何事?”

“韩玉昆最近又升了官,想来跟先生说一说的。”吕大临犹盯着纸面上的一个个端正的小楷,随口回话,“不过比起先生的这一篇经义,韩玉昆的事就算不得什么了。”

“玉昆怎么了?”张载很在乎韩冈这个弟子,听到之后,便立刻询问。

吕大临回过神来,见张载很是关心韩冈的样子,便恭谨的放下这一篇价值千金的文字,垂手答话,“学生刚刚听到消息,说河湟那边接连设立巩州、熙州,又设立熙河路经略安抚司,王韶任经略使,而韩玉昆则是担任机宜文字,并兼任巩州通判一职。”

张载闻言便是有些惊讶,问道:“经略司机宜,还有下州的通判,这已是转朝官了吧?!”

吕大临点点头,张载的惊讶其实就跟他前面听说这个消息时一模一样:“韩玉昆已经是太子中允了,有天子特旨,而不是靠了五削圆满。”

“玉昆进用之速的确是个异数。”张载微微有着一点感慨,他当初转为朝官,可是在中进士后的十二年,也就是两三年前的事。韩冈这个弟子,在官场上的作为,的确比他出色得多。

但张载还是很欣赏这个弟子,吕大忠、游师雄,还有表侄程颢、弟弟张戬,都推重于韩冈,也不是因为他升官快的缘故。

“要找五份荐书,玉昆也是能找得到的。他的功劳比起现在的官职,更是远远超出。年初广锐之乱,不是玉昆孤身进城说降,也不会这么容易就平定。横渠镇离咸阳不远,能安然无恙,也有玉昆的一份……”

正说着话,张载突然猛烈的咳嗽起来,手用力按着胸口,一时间咳得喉中气息嘶哑,吕大临见状,连忙上来拍着后背。好半天,张载才回过气来。

“先生,要不要去长安找几个名医来看看?”

张载轻轻挡开犹在捶背的吕大临,“算了,也是老毛病了,与叔你也该知道的。”他笑了笑,“玉昆也不知从哪里听说了我这毛病,前日寄的信中便有说道,咳嗽多,要多吃梨等润肺之物,日常食补胜于药补。”

“韩玉昆是药王弟子,他说的当不会有错。”

“怪力乱神,儒者自当远避之。乡野中的这些传言,玉昆本人是从来不认的,这点他做得很对。”

张载说得郑重,吕大临点头受教。

“说起玉昆的信,其实里面还说了些其他的事,是关于格物上的一些原理,有关力的方面的”

注1:这一篇文字,是关学的总纲,而后被理学继承过去,世称《西铭》,是儒学的经典之一。。

