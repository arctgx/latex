\section{第五章 冥冥冬云幸开霁(四)}

“就按照韩卿家说的办。”

听到向太后的话,韩冈欠身行了一礼,“礼记有云:‘故上之好恶,是民之表也’。太后宽厚仁孝,正堪为万民之表率。”

“吾是宽厚,宽厚得这一回差点连命都送了。先帝交托吾的基业,也要落到那些贼子的手里。日后还会晨昏定省,谁还能说吾不宽厚?”

向太后的声音平稳得让人不由得心生寒意,话语中的的恨意深如渊海。韩冈心头都是一跳,怀疑起自己是不是又做出了一个错误的选择。

不过不论换做那个大臣来说,都不可能同意向太后处置她的姑姑。站在儒臣的角度,他们必须要保住高太皇的地位和待遇。

韩冈规劝道:“若父母慈爱,子女孝顺世所常见。如虞舜,父不慈,弟不悌,犹能孝于父,友于弟,其德方能光耀千古。”

章敦眼皮跳了一下,韩冈还真是会抓时机。

当年英宗皇帝与曹太后关系恶化,韩琦入内劝说英宗,英宗对韩琦抱怨说‘太后待朕无恩’,而韩琦便拿着虞舜为例子,规劝英宗要孝顺当时的曹太后。

可韩冈趁机提起宋英宗的旧事做什么?这是怕向太后想着换皇帝,故意提醒她英宗当年是个什么德性吗?

向太后久在宫中,自是知道当年旧事,“韩忠献劝谏英宗皇帝话,吾也还记得。太皇太后那边,吾是无话可说了。二叔那里,史书上有很多,更不用提了。倒是蔡确,他堂堂宰相,已是位极人臣。上追父祖,下荫子孙,吾什么时候慢待过他,他还有什么不满意的?”

韩冈没说话,蔡确到底为什么会反叛,向太后是肯定知道原因的,现在不可能明着说出来。没有韩冈坚持要保赵煦,没有向皇后坚持依从韩冈,蔡确不至于会走极端。

“蔡确昔年为臣所荐,可转眼又弹劾臣。云为国事,实乃私心。其本性如此,今日不过故态复萌。”

王安石当年重用蔡确,却被蔡确背后捅了一刀,此事尽人皆知。但王安石这么说,一看便知是要将问题归咎在蔡确的本性身上,而不是去追究是什么样的外在原因,造成了蔡确的叛乱。

“虎狼之心,岂是人能体会?奸佞之辈,其所思所想,自与正人君子迥然而异。王安石其言有理,不过是故态复萌罢了。”韩绛也这般说着。

剩余的两府宰执,至少章敦的态度也可以确定,是息事宁人,不穷治此案。张璪、苏颂的态度也大类如此,否则以蔡确、曾布和薛向在外的人脉关系,不知会有多少人被牵扯进来。

南丰曾家进士十余人,薛向家里也是数代为官。而蔡确,正在跟韩琦家议亲,本身又与冯京是姻亲,福建蔡氏亦是望族。

宰执班中,谁敢放言穷究不舍?若是株连起来,冇一两道弯后就能牵扯到他们或是他们亲友身上。

只有底下的官员,恨不得上面多空出些位置,才会有人想着将事情闹得越大越好。

“不过蔡确党羽,必是多为名利所诱,以至于利令智昏。”张璪忽然说道,“又自诩才高,以朝廷不能用,故而多怨。如苏轼,如刑恕,如韦四清等人,皆如此。再如蔡京,由台端沉沦下僚,久闻其对外多有怨言。又是蔡确亲族,其嫌疑亦远重于他人。”

如果张璪不是于在宫中当着其他宰辅的面公开宣言,而私下里与太后说,韩冈肯定会举双手赞成。不管蔡京真有罪假有罪,只要以叛贼党羽为名给他定了罪,他这个枷锁就别想再钳制住韩冈。

向太后又转问韩冈:“韩卿家,你看蔡京是否与蔡确有牵连?”

“臣与蔡京有旧怨,是与非,臣不便多言,请有司查证便是。”

一入法司,想要什么结果都容易。蔡京没了地位,没了后台。谁会为了他,放弃讨好韩冈的机会?

韩冈完全没有留着这道枷锁的想法。时过境迁,过去为了自证心迹,刻意竖起的障碍,现在已经没有存在的必要了。

不过韩冈的回答也是堂堂正正、在情在理,有嫌疑自然要问,难道还要他保蔡京无嫌疑不成?他能明说与蔡京有旧怨,却并不落井下石,而是让有司去查证,这已经算得上是正直了。

“殿下。”王安石这时上前一步,“臣以为,如今当务之急,不在皇城之外,而在皇城之内。”在他看来,向太后问了太多可以放在日后去审问的事,“没有宋用臣、石得一为内应,御龙四直与皇城司不会叛乱,而蔡确纵有叛心,也无能为力。”

宋用臣、石得一联络蔡确的可能性,要远远超过蔡确联络宋用臣、石得一的可能性。

天子近臣想要联络外臣叛乱,总能找到合适的人选。但让外臣去说动天子近臣做反,这风险冒得不知要超过多少倍了。

同样的理由也能用在宋用臣、石得一身上。

处理太后及天子身边事的宋用臣,掀动在外围执掌实务的石得一,自是远比石得一说动宋用臣要容易。

但宋用臣是先帝赵顼自李舜举后,最为亲信的内侍,他为什么会投向太皇太后?

这件事不是站在这里猜测就能想得到的。光是赵煦弑父,向太后拒绝另立,应该还不至于激烈到如此的程度。在这二十多天里,必然还有些事让宋用臣对向太后和小皇帝彻底失去了忠心。

“没有从贼的,就是忠臣,刘惟简、王中正都被关押起来了。宫里面的事,让他们来处置。”

不论王安石是不是想要乘机插手宫中的人事,但向太后的回覆,一开始便否定了这个可能。

常言到文章憎命达,现在韩冈也有类似的感觉。

他不是在想苏轼的事。也许千年之后,流传于世的名篇会多了岭南或西域大漠的篇章,不过现在,韩冈只是觉得吃过苦头,人真的会成长。才学,心性,都会有些变化,脱胎换骨一般也不是不可能。

向太后的冷静,远远出乎于韩冈的预料。如果她偏激的要大开株连,这还在预想之中,可刚刚经过了一场叛乱,还能想到不给外臣机会,在叛乱之前,她也许还没有这样的水平。

“朝廷里面,有谁是蔡确党羽,一体交付御史台和大理寺去审问。有功者,也当重赏,赏格由两府共议。”

“殿下。”王安石躬身道,“请殿下恕臣等擅专之罪,之前在大庆殿上,因从贼者甚多,不得不擅作主张,赦免了他们的罪行。”

王安石将当时大庆殿的一切,简明扼要的说了一遍。

向太后这时才知道,这场本已是十拿九稳的叛乱,究竟是怎么被翻了盘。韩冈说的杀了蔡确,竟是他亲手用骨朵给捶死的。

熟视韩冈良久,向太后轻声道:“多亏了韩卿家。”

“不敢。这是臣的本分。”

“张守约一定要救回来!”

“已有御医在为他诊治。”

向太后点了点头,望着面前一众宰辅:“多亏了诸卿。”

王安石率众人谢过。

向太后又道:“事急从权。既然相公们都说了要赦从犯之罪,那就这么办吧。”

她回头向后,“官家,非诸位卿家之力,你我母子几不能保。日后当冇时时念着今日。”

“儿臣知道了。”一直静静的站在后面的赵煦低声回答着。

“陛下可安好?”

赵煦与向太后被囚禁在一处。但群臣进来后,有意无意间把小皇帝给忽视掉了。但诸事已了,赵煦就在眼前,已经不能当做没有看到这位大宋的皇帝。

赵煦只披着一件小袄,不是出外视朝时的装扮,但神情态度却还是一如往日。

听到群臣的问候,他也只是简单的说了三个字:“朕无事。”

赵煦的冷静得莫说不像一个六七岁的孩童,就是成年人处在他的情况下,也不至于如此平静。

是天生的心性,还是没有意识到最后的结果有多严重?

赞叹赵煦早熟老成的话,世间已不知说了多少。如果没有炭毒一案,看到赵煦现在的表现,群臣必然要赞叹皇宋又出一英主。

可现在赵煦表现得越好,朝臣们心中的戒惧就又深上一层。

一想到十年之后,一名冷静早慧、却又弑父之罪的君主将要掌控朝政,在列的朝臣们,有几个不是暗自心惊?

赵煦在刑律上当然无罪,六七岁的小儿做下什么错事,都不会有人认为他是故犯,也不可能论于刑律。有董仲舒春秋决狱的例证在前,就是成年人误杀父母,也不会论死。但从纲常上,赵煦却绝逃不脱一个弑父的罪名,谁让孔夫子在春秋上写明了是‘弑’。

韩冈从赵煦脸上收回视线,落到王安石的身上。

众人之中,当只有一个王安石跟他是一般心思。

王安石之所以还要保赵煦,也仅仅是看在刚刚驾崩的赵顼份上,心中顾念着旧情,否则也会成为劝说向太后另立新君的一员。

恐怕有不少人再想怎么不给惊吓到。要是当真惊悸发病,也就能顺水推舟的换一个新君了。

