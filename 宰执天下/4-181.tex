\section{第29章 坐感岁时歌慷慨(上)}

‘吕吉甫、章子厚这玩得是哪一出啊?’

离着京城还有三天的路,但在韩冈下榻的驿馆中,就已经在到处疯传当朝宰相请辞去职的消息。

韩冈一开始还纳闷,他的岳父回江宁都快要一个月了,这条旧闻怎么还在传播。等他派人去一打听,才知道,原来说的不是王安石,而是冯京。是冯京冯当世辞相了。

这才几天?宰相和枢密使都换了人。韩冈望向东京城所在的方向,近晚的黄昏下,东北的天空是一片灰黑色的混沌,阴云遮蔽了大半天空。

王安石辞相的消息是和调令一起过来,接着在韩冈抵达襄阳的时候,吴充接任相位的消息传了过来。今天韩冈就在汝州,听说了首相冯京因御史弹劾而辞去了相位。从动机上看,幕后的指使者当是吕惠卿和章惇二人。

“张商英还真是好本事。”

韩冈难得佩服人,人家寻常做御史的,再敢言也不至于只挑大个儿的打。可今次领头弹劾冯京的张商英,却是一门心思就盯着当朝的宰执官。

张商英是章惇在荆南时推荐给王安石的人才,韩冈没见过他,但听章惇提起过,几年前他所引发的东西二府之争,也是很有些名气。

张商英被章惇推荐给王安石后,先是担任中书刑房公事,很快又转到了监察御史的位置上——这算是年轻官员晋升的快车道,只要好好做个几年,闯下了一些声望,就是日后飞黄腾达的基础。王安石挺欣赏张商英,为他安排的就是这条快车道。

但张商英坏就坏在他做事太过卖力,起手就找上了枢密院,最后闹得西府几位枢密使一齐封了印信,闹起了罢工。天子当然不会为了一个监察御史,而将当时枢密使吴充、蔡挺和王韶一齐罢去,因而张商英就被贬去监酒税了。

做了几年收酒税的官儿,任谁都会认为张商英会改一改他的脾气,但谁能想到几年后回返京师,当即就又找上了宰相冯京,而且还当真给他办成了。

一举扳倒了当朝宰相,这一下子,张商英这个名号,可就遍传天下,日后也就有了飞速蹿升的基础。从他的行事上看,当是个敢于冒险、喜欢以小搏大的人物。这与稳扎稳打,靠着军政两事上的功绩往上走的韩冈,并不是一条路数。

“如今朝堂上正逢一场大变局,张商英只是适逢其会而已。换作是王相公还在的时候,他根本就不可能成功。”

坐在韩冈下首,是他曾经的幕僚方兴。

两人在路上遇上是个巧合。曾经辅佐韩冈安置河北流民的方兴,如今正好要去京中守阙。而韩冈也要入京,便是无巧不巧的在半道撞上了。

做了一任县尉,没有功名在身的方兴,离着改官还有一段漫长的距离,他当然想要振作一番,而韩冈正好身边缺人——幕僚倒好说,虽然之前的李复四人全都因为交趾之功而得官,可他韩冈只要入了京城,想要投到他门下求个出身的官员当不知凡几——但衙门中韩冈还需要一两个助手,这对正巧任满候阙的方兴来说,便是天上掉下来了馅饼。

虽然方兴本人没有明说,但他的话隐隐约约是在暗指当今天子是造成如今朝局动荡的元凶,没有赵顼的袖手旁观、甚至是推波助澜,朝堂上怎么可能会有什么大变局?——当今的这一位皇帝,可是已经在御榻上坐了十年了。

“确是如此。”韩冈点头表示赞同。方兴的猜测不能算是有错,几年未有变更的两府名单,已经成了一滩死水,赵顼肯定不希望接下来的几年,这潭死水还会继续下去。

所以政事堂中的宰相换了人,王安石和冯京前后脚离开,枢密使吴充成了宰相。而枢密院中,蔡挺早已请辞,王韶地位还不够稳,章惇更是资历浅薄,接手枢密使一职的,赫然是前段时间上京后就没有离开的吕公著,而郭逵则是在十几年之后,再一次坐上了同签书枢密院事的位置。

“全都乱了。”韩冈感叹一声。

才两个月功夫,朝局和风向都乱了。而且吴充和吕公著分别执掌东西二府,这其中的政治意味很重。天子赵顼的心中,似乎有缓和新旧两派的矛盾,改变过去近乎一面倒的情况,希望两边能同心同德的治理天下。

但这乱象,不仅仅是赵顼的功劳,自然也不可能如他所希望的看到同心同德的场面。

“这几年的朝堂就像是一口下面烧着旺火的大锅,里面的水都已经烧开了了。之前锅上的盖子,由于死死压了个几千斤重的巨石,热气热水能从缝隙中冒出来,却掀不开锅盖。可现在千斤巨石不在了,加之管烧锅的放纵,被压在锅底下的乌七八糟的东西自然全都给迸出来了。”

方兴冷笑着,他说的话正是韩冈心中所想。

王安石虽然强势,但他稳定朝堂的能力却是没话说的,如同定海神针一般。这两年朝堂上基本上保持着稳定,其实都是他的功劳。

现在王安石辞去相位,去江宁府担任知府,被留下的人有可能和衷共济吗?……当然不会!恐怕等几天后,到了京城,就能看到吴充和吕公著的动作了。

不过现下身在襄城驿馆后的小楼上,讨论什么都是空的,东京开封还在几百里外,而自己也不过是个都转运使而已,距离宰执之位还远得很,不必操那份心。

只是眼下风暴还在继续,也不知道三天后,抵达京师的时候,会出什么问题,这场风暴又会将多少人的官位一次打得粉碎。

韩冈推开窗户,一阵广西见不到的冰寒扑面而来,的确是个真正的冬天。将对朝堂动荡的担忧放在一边,韩冈很快就想起了他刚刚病逝不就的老师。

张载籍贯是汴梁,只是缺钱才不得不寓居横渠,但这些年来,张载的父母和亲弟弟张戬都是葬在横渠镇。所以他到底是留在京师,还是归葬横渠,韩冈猜不出来。若是在京城,还能去见上一面,若是回了横渠,短时间内可就没办法将主动提高。

不过关学一脉,少了张载这个核心之后,又该由谁撑起关学的大局?韩冈知道自己还差上一筹,但诸多弟子中,能有这个能力的似乎也没有。

韩冈摇了摇头,合上了窗户。被寒风吹散了房中暖意,很快就又恢复了过来。

韩冈的贴身亲卫提着个食盒上来了,驿馆中的驿卒将做好的饭菜送到门口,就由他送了上来,里面有着韩冈和方兴今天的晚餐。

“听说隔壁住着一个从京城出来的官人,”亲卫一边摆着碗筷,一边对韩冈说道。

驿馆里不住着官,还会哪里住着?韩冈信口问道,“可曾问了他的名讳和身份?”

“姓舒,听说是个御史,来京西查案的。”

“舒……御史……”韩冈念了两遍,随即恍然,想起来了究竟是谁。姓舒的官员多得是,但姓舒又是御史的眼下可就一个。

“舒亶怎么往京西这边跑来了?”韩冈纳闷的自言自语,这个时候他不是应该跟着张商英一起痛打落水狗吗?

舒亶这个人,韩冈听说过。

在韩冈刚刚做官时,因为他曾经亲手杀人的缘故,曾有人拿他比作张乖崖。不过在韩冈之前,还有一个被比作张乖崖的年轻官员,就是英宗治平九年礼部试第一的舒亶。

舒亶考中进士后,第一任是台州临海县尉。台州当地的民风彪悍,一向难以管束。一次一名胥吏酒后发狂,追逐其叔母。被抓到县衙中后,又趁醉使泼,不服判罚,舒亶便直接就亲手拿刀将他给杀了。下手果决之处,与张乖崖如出一辙。

非刑而杀,算是一个罪过。但诛杀此名胥吏,也是情有可原,所以舒亶也就是接下来两年被停职,之后又因父丧而回乡守制,很久之后才被张商英推荐给王安石。

不过韩冈知道舒亶不是因为他与自己一起被人称作是张乖崖,而是因为他几年前在熙河路做过一阵营田司的勾当公事,也就是跟韩冈的父亲算是同事。尽管不可能深交,但也有着一份交情在。

“等吃过饭,他多半会来拜访龙图。”方兴笑着说道。

“或许吧。”韩冈对此一点也不惊讶。他的身份不一样了,就算是炙手可热的御史,想要见自己,也必须是他自己主动过来。

等韩冈吃过饭,就开始有人来拜谒了。不过都是住在驿馆前面的低阶选人,襄城不算大镇,人数并不多,韩冈不想多事,很快就打发了他们。等这边的稍稍安静了下来,就有一封拜帖送到了韩冈的面前。

韩冈将拜帖看了,就立刻派了人下去,过了片刻,小楼上的脚步声响起,先是在前领路的亲兵,接着就是一个身材高大魁梧的绿袍官员来到韩冈的面前,双手一合,一揖到底:

“舒亶拜见龙图。”

