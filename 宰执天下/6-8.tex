\section{第二章 天危欲倾何敬恭(五)}

“那不是二大王?”

“正是齐王。”

“真的病好了。”

“什么时候病过的。”

“头上怎么包着绷带?”

“好象是撞墙撞的。”

“病好了还撞墙?”

“谁知道这位友悌的二大王是在想什么呢?”

议论声纷纷而起,在议论声中,宣德门的侧门缓缓打开,齐王赵颢车驾排开所有等待入宫的官员,第一个驶入皇城。

在赵煦登基后不久,也就在赵顼驾崩之前,向太后依照过去的惯例,将小皇帝的两位亲叔叔都进拜大国,一为齐王,一为鲁王。

不管赵颢是否是曾经准备与赵煦争夺皇位,但只要朝廷没有正式追究他的罪行,那么他依然是大宋的亲王殿下。

当赵顼猝然暴毙的消息从宫中传出来,已经疯了一年多的赵颢突然之间就清醒过来,哭着喊着要来祭拜他最尊敬的兄长,甚至在监督他的内侍拒绝他的要求的时候,一头撞向墙壁,以示坚决。

如此友悌的亲王,朝廷怎么能拒绝他的要求?而且重病痊愈也正是可喜可贺的一桩事,可以顺便慰藉正逢丧子之痛的太皇太后。

“好个高洋。”

宣德门内,一名身着素服的官员毫无顾忌的评论着,惹得附近几名官员侧目而视。

二大王的心疾,适时而生、适时而退,可不正是与南北朝时,那位装疯卖傻、成功的扮猪吃老虎、最后顺利地谋反得手的北齐皇帝,有那么几分相似?外在表现的疯病是一桩,内里的野心也同样是一桩,而且更相似上几分。

不过这话如果是文官说出来倒也罢了,恐怕会有人拍手叫好。但这一位看丧服的式样,明显是武班那一边,而且一干配饰,还证明他有资格穿戴五品服色。一名武夫都敢这么说话,还是让周围一干文臣拉下脸来。

苏轼偏过脸,看向他去。

三十出头的年纪,容貌很是英武,素色的丧服下,修长的身躯如劲松般挺直,但这张面孔苏轼并不熟悉,对于有过目不忘之才的苏轼来说,对方显然不是在京的朝臣。只是若他不是保养有方,三十多岁便在穿戴比同五品,要么靠山很硬,要么就是军功显赫。

“是王襄敏的儿子,王hòu王处道。”

“兰州知州,熙河路钤辖。”

“就是他。”

身边官员的窃窃私语,化解了苏轼的疑问。这一位显然是个名人,认识他的人多,听说过他的也不少——苏轼也听说过他的名讳——不仅是靠山硬,军功也不小。也难怪没人出来呵斥他。

今年轮到他上京诣阙的吗?

苏轼有几分好奇的打量着这位名气不小的年轻将帅。

虽比不上其父一举打开西北僵局的开创之力,也没有听说他有其父能识人用人的出众眼光,但在陇右坐典要郡,镇冇压诸蕃,他冇的能力已经表现得很突出了。

王韶在陇右素有威望,王hòu在熙河路也是一言九鼎,极得吐蕃人敬畏。王hòu镇守兰州好些年了。但凡有蕃部闹事,他一封信过去,就能让蕃人们全都老实下来。

今年是他上京诣阙的时间。本来在天子晏驾的时候,但凡边臣守将,都必须坚守在岗位上,不得离任半步,但王hòu是正好撞上来的,

“十几年前,他就在熙河路。十多年过去了,他还在熙河路。再这么下去,不是藩镇也是藩镇了。”

苏轼还听到有人在背后低声私语,似乎很不服气的样子。名义上是担心朝廷,暗地里却是攻击居多。

但陇西的地方势力盘根错节,有归顺的蕃部首领,也有迁移到那里的大户,还有依靠军功在当地扎根的军汉。而这三等人,全都与当年开拓河湟的王韶有着牵扯不清的关系。没有王hòu这样的身冇份,想要镇住陇西,不知要费多少力气。

而且王hòu的背后是韩冈,以及如今在禁军中势力最大的西军这个山头,真要想给他点难看,就等着他背后靠山和势力咬上来吧。这样hòu实的根基,不知有多少武官羡慕他呢。

正想着,身边窃窃私语声突然没了,苏轼心有所感,回头望去。

却是靠山到了。

没有执政一级的数十元随相伴左右,今日在韩冈身边跟随的只有寥寥数人。看他的样子,似乎已经丢掉了宣徽北院使和资政殿学士两个职位,完全将自己当成了罪臣犯官。

韩冈律己之严,倒不愧他一向的名声。其视官位如敝履,遵循朝廷法度,倒让许多习惯于僭越仪制的官员为之惊异。

韩冈在离着宣德门还有很远的地方便扯住了缰绳,停了下来。视线扫过门前等待入宫的官员们,苏轼甚至感觉到他遥遥的冲着这边点了点头,好象是在打招呼。

苏轼皱着眉,回头看,韩冈打招呼的对象果然不是自己,而是王hòu。听说两人之间还有姻亲,不知王hòu此番进京,入住京中驿馆时,是不是先去韩家拜访了。

不过当许多官员,纷纷凑过去问候韩冈,王hòu却留在了这里,没有一并凑上去。也许自别人眼中是双方生分的表现,可在苏轼的感觉中,两人之间显然早有了默契,不需要无谓的礼节。王hòu背后的靠山,依然牢靠坚挺。

韩冈引罪辞官,但朝廷还没有批准,即便批准了,地位也不一定会下降。朝廷也不可能让他离开京城,去地方担任州官。只要还在京城中,韩冈即便是布衣,也能与执政分庭抗礼。

钟声和炮声先后响起,文武百官汇入宫中。他们今日的任务是劝说皇太后听政,不要再陷入悲恸之中。尽管谁都知道,这几日向太后还在处理朝政,并没有耽搁政事,但该走的仪式一点也不能缺少。无论如何,这是传承下来的法度,不是随随便便可以否定的。

听政,小祥,大祥,之后又三月禫除,再后,便是梓宫入葬和神主入庙。一整套流程都要走完,才算是个结束。如今只是一个开头。

皇太后和小皇帝此时皆在梓宫前,守着大行皇帝的灵位。文武百官所要做的,即是来请两位移驾,完成请听政的任务。

皇太后没有戴上凤冠,头发也披散下来,没有梳成发髻。妆容不整,以示心中的悲戚。而小皇帝也没有带着幞头,还能看见剃青的头皮。

太皇太后也同样在大行皇帝的梓宫前,守在灵堂中。天地之内,如今最尊贵的两位女性之间,完全看不出有什么芥蒂,相互配合着完成今天应有的程序。

紧紧盯着太皇太后的几位宰辅,终于是松了一口气,这是认命了,还是懂得顾全大局?但不管怎么样,能将请听政的仪式顺利进行下来,真的还要多亏了高太皇太后的配合。

三请方允,臣子们的努力终于打动了太后,而沉默的天子也被向太后牵起手,入内整理妆容。

片刻之后,文德殿中,文武百官依序排班,等待着太后与天子的出场。

王安石和韩冈都上表请辞,虽然还没有批准,但两人皆无意站在请求辞去的官职位置上。只是依照仍旧保留的本官官阶。

依照合班之制,王安石依然能够与宰辅并立,冇但寄禄官仅为礼部侍郎的韩冈就只能立于翰林学士为首的诸殿阁学士之下。

原本被很多人认为韩冈绝不会甘居诸学士之下,可实际上却与他们的猜测大相径庭。散官阶无干排班之序,手持竹杖的韩冈在一群空着手的官员中显得十分的特别。

韩冈站得很平静,对不是飘过来的视线,完全没有理会的意思。他这份宠辱不惊的气度为人击节,不过没有人注视他太久。宰相班下首位置的赵颢,比任何人更加惹人注目。

二大王站在朝臣中,毫不旁顾。很多人都难以理解他的想法,都已经发过疯了,现在就算再回复,群臣也不可能拥立他,难道他还真的指望做高洋不成?

真要说起来,同样是伪装的发疯,第一个是能忍人所不能忍的枭雄,第二个可就是东施效颦的蠢货了。对于这位又开始上蹿下跳的二大王,朝臣们还以为太后会将其直接圈禁了,想不到还是容忍了他。联想起方才十分配合的高太后,这是在私下里达成了什么协议不成?

当然,现在稳定的局面,也是两府宰执所乐于见到的。在明面上,哪位宰辅都不想看到朝廷典礼被破坏,让朝臣和天下士民怀疑其他们的能力来。

视朝听政的仪式没有半点波折的顺利完成,率群臣拜礼之后,韩绛恭送皇太后与天子退朝离开。文武百官也依序退出了文德殿中。

出了殿门,仰头望着阴云四合的天空,韩绛一阵恍惚,就这么结束了?亏他还准备了许多应对的手段。

他回头望了望两府中的同僚,王安石,以及赵颢和那位没什么存在感的三大王。

‘这样也好。’他想着。没有横生枝节是最好的。

