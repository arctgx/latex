\section{第14章 落落词话映浮光(中)}

半寸厚的书卷,拿在一只苍老的左手中。

手又宽又厚,五指粗短,指节凸起,看得出其中蕴藏着很强的力量,这只手抓住的东西,就不会被人抢走。

手背上已经能看见褐色的老人斑,掌心上有磨出老茧,却保养得很好,肌肤细腻,看不到有多少皱纹,不过手背上有一道三寸多长的疤痕,鲜红的。

拿在手中的书,封面的边缘已经磨毛了,但书页很干净,保养得比手都好。

这本书就是一本普通的书卷,唯一有所差异的地方,就是封面封底是一张纸,将书脊也保护了起来,而且书脊上还印了书名。

九域游记。

“陛下……”

听到声音,拿着书的手没有动,手的主人低沉的应了一声。

“什么事?”

“太子殿下回来了,今天猎了一头虎,三头熊,二十多只鹿,说要将虎皮献给陛下。”

“让他先去梳洗了再过来,累了一天,汗也多,梳洗更衣免得着凉生病。”

“奴婢知道了。”

侍卫应声而退,离开时有着松了一口气的释然。

耶律乙辛静静的将书合了起来。

他不喜欢吵闹,尤其是在批阅奏章和读书的时候,更不喜欢有人打扰,亲近人都知道。不过太子率众游猎回来,肯定得禀报给他,这就要冒些风险了。

耶律乙辛并不是依靠游猎夺取了天下,所以即使他按照常例,巡狩四方的时候,也不会当真带着宿卫去狩猎,而是交给了年轻的皇太子。

这种四成是娱乐,四成是惯例,只有剩下两成有着军事意义的活动,对耶律乙辛来说,可有可无。

不过他很喜欢捺钵,尤其是坐在御帐中,接见四方臣子的滋味。

如果想千秋万代的统治下去,一个合格的继承人必不可少。他的长子,做一个能够守成的皇帝,已经可算是合格了。

但南面的那一位,会给他守成的机会吗。

宋人始终不肯承认自己的身份。耶律乙辛不知道宋人的坚持有多少是因为不肯对篡逆之人妥协,又有多少是为了省下那五十万银绢的岁币。不过耶律乙辛现在并不是很在意

来自于《自然》期刊上的一篇论文,让他知道了日本有多富庶,地里面埋了多少金银。

渡海灭倭得到的好处,有一多半靠了这篇文章。

但正是因为如此,耶律乙辛对宋国的参知政事越发的忌惮起来。

耶律乙辛低头看着放在虎皮毡子上的书。

这是《九域游记》的第一版,被送到耶律乙辛的手中,已有一段不短的时间了。

听说在南朝国内,这部书在初版之后,很快便再版、三版,乃至更多版本。

据耶律乙辛所知,韩冈的著作,以及气学一脉的著作,每一次的内容都会有所修订。

所以从藏书家的角度来看,每一个版本都有收藏的价值,其中自然是以第一版最有价值。

耶律乙辛不是藏书家,但他也的确更喜欢老书拿在手里的感觉。

重新拿起书,他随手翻开。熟悉的文字,看个开头,就知道说得是哪一段。

‘贤周通连夜报信,勇鲁达三枪败贼’

第八回。

经过了两个不同人物的故事,述说的对象终于转回到周通身上。周通自离乡后,坐着蒸汽车,沿着铁路一路南下,在扬州,来到秀州换了汽轮船,却因故卷入了一场叛乱。

这鲁达本是州中提辖兵甲盗贼公事,少时做了沙弥,法号智深,但长大后,便还俗投军,后因功被推荐进了武学,还得了官身,后因酒后错手杀人,被发配岭南,之后又因缘际会,弄到了一张度牒,做了和尚。

‘平生不修善果,最爱杀人放火。’

虽是平直的一句赞,却让人看得煞是痛快。

‘鲁达点着了火绳,一扣扳机,砰的一枪,打得那贼头前胸通后背,透风透亮。后面的两个贼头提着刀赶上来,只看那鲁达不慌不忙,上弹点火,又是砰砰两枪,将那桃花山的二大王、三大王,一一轰碎了脑壳,红的白的,流了一地。’

几句话的打斗,放在说书人的口中,能铺陈出洋洋几千言、惊心动魄的场面,不过在耶律乙辛眼中,更重要的是——

到底什么是火绳枪?

一支可以拿在手中的兵器,竟能把人胸口打穿,把最硬的脑壳崩碎?

在第一次看到这话本的时候,耶律乙辛立刻就让人去打探了。

他当时已经听说了火枪,一种可以随身携带的火器,一种为了取代弓弩而设计出来的武器,但他不知道什么是火绳枪。

付出了四名细作的性命,辽国天子得到火绳枪的图样,其相对于火枪,就是大黄弩相对于弩弓,属于下面的一类。接下来,他就对书中所说的淘汰了火绳枪的燧发枪更加感兴趣了,因为那可能就是火枪中的神臂弓。

普通人有了兴趣,他会开始对此用心努力;而皇帝有了兴趣,却是千百人一齐拼命满足他的要求。

没用多久,耶律乙辛就知道了,宋人正在制造燧发火枪,而且已经渡过了实验制造的阶段,只是暂时还不能批量制造。

耶律乙辛已经考虑过会有这样的可能,但蒸汽车、汽轮船,以及两者的核心——蒸汽机,都只是出现在《自然》上的一种只存在于想象中的机器,燧发枪却已经出现在宋军手中,而书里被燧发枪淘汰的火绳枪,却根本就没有在宋国禁军中装备。

得到了有关火枪的信息,耶律乙辛就像他对火炮的兴趣一样,立刻就遣人去进行研究和制造,不过远比火炮要困难,近一年的时间过去了,火器局那边,始终没有给他一个满意的回复。

除了火器之外,《九域游记》给了耶律乙辛还有很多信息。

这部作者不详的话本,书中的气学痕迹太深了,或许是韩冈写的,或许不是,但成书之际必然得到了韩冈的指点。

韩冈推出这部话本的用意,应当是减少朝野内外阻力,同时也是要告诉世人,他想要的是什么样的宋国。韩冈的用心,实在是太明显不过了。

不过最让耶律乙辛心中发寒的,是太过光明正大,根本不怕泄露军机。

耶律乙辛知道,这是因为韩冈充满了信心。

‘历史的车轮’。

这个词出自于第九十一回。两个很普通的词汇,很无稽的拼凑,让人心头火发,却又不寒而栗。

难道大辽注定会成为在车轮前面挥舞双臂的螳螂不成?

可不管韩冈为了什么原因传下了这部书,他至少给了耶律乙辛一个明确的方向。

韩冈打算做什么,将要做什么,从这本书里完全可以了解得到。

比如视城墙如无物的重型榴弹炮,比如让弩弓成为玩具的火枪,比如成列而战、以排枪毙敌的神机营,比如不再需要挽马、只要有水有煤就能拖动万石列车、日夜不停的蒸汽车,比如能无视风向、载万石货物、两三日间横渡千里冥波的汽轮船,再比如三五百人拥枪炮坚守,数十倍大军亦难攻破的棱堡。

这其中,有的还只是幻想,有的已经出现却还未普及,有的则是抵在了眼皮下。

一年前,吕惠卿被调离河北,两国之间已经恢复和平。

不过河北边境上,宋人正在拼命的修造寨堡,改建城寨,而且修成后的外形特异,不再是将马面加密,而是将城寨四角修成外凸的五边形,然后在里面架上火炮,不论想攻击哪一边的城墙和城门,就会受到来自两侧的炮火夹击。

也是在这部话本中,耶律乙辛知道了这种堡垒的名字——棱堡。

出自第十六回。

建筑在辽东北方的旷野中,是边地的一座指挥所,驻守一个指挥三百多的戍卒,被两万东虏重重围困,却让其丢下了一千多条命狼狈而走。

第一次看到这里,耶律乙辛除了为韩冈的野心而冷笑之外,便是暗暗称赞他不愧是带过兵的。

若是说书人的话本,那就是一名守将大展神威,用神枪接连挑下敌方大将三十六员,杀得贼人丢盔弃甲,但这本书里面,几场战事,却都十分符合实际。没有让说书人口沫横飞的斗将,只有切合实际的攻守。

所以在书中看到棱堡,又得知宋人在河北开始修筑同样类型的堡垒,耶律乙辛就开始让人修筑同样的堡垒来试验。

修成后的几次演戏,耶律乙辛发现,韩冈完全没有夸大棱堡的效果,只要在墙中枪炮的弹药不绝,堡中不缺食水,并不算坚实的堡垒,就能变成了险关要隘一般。

南京道南方边境对面的寨堡,正在迅速改成的棱堡形制,而宋国河北两路境内的州城县城,除了以砖石包墙,也都在城门和四角加筑了炮垒。一旦河北的寨防完工,再想入侵宋国,难度不啻十倍。

不过耶律乙辛早没心情攻宋了,依靠日本的金银,依靠辽宋边境恢复起来的榷场,根本没有攻打宋国的必要。

与其期盼在战场上取胜,不如决胜于庙堂。

只要宋军不来攻,耶律乙辛也不想打过去。至于韩冈,耶律乙辛并不担心。

处在臣子的位置上,却有超过天子几十倍的人望和民心,这样的臣子怎么不该死?皇帝怎么会留着他?

即便他能活到四十岁,也别指望能活到五十岁,这样的人根本没必要去防备。

最多十年,就能见分晓了。

现在,还不如安安心心的看书。

