\section{第39章 欲雨还晴咨明辅(41)}

方兴已经在一旁等了很久。

他是新任命的军器监丞兼管勾火器局。同为军器监丞,铁船等事归臧樟管,火器就轮到方兴出面。

听到蔡确的问话,又收到韩冈递过来的眼色,方兴上前一步:“相公。方兴这边已经准备好了。”

“好。”蔡确带头迈开步子,笑顾韩冈:“就看看玉昆你这一回还能带来什么惊喜了。”

韩冈点点头,举步跟上。

蔡确跟着方兴走,回头看了一下跟在后面的人群,判军器监黄履依然不在人群中:“黄安中还没到?”

方兴立刻回道:“之前已派了两拨人去通知大监了,现在应该正在回来的路上。”

蔡确转回来对韩冈笑道:“难得黄安中如此勤谨。过去可没这么勤快。”

“这几曰见黄安中,倒是让韩冈深愧自己太悠闲了。”

韩冈这几天为火器局、铸币局两件事,与军器监多有联络,判军器监的黄履那边也见过了两次面。

只是今天,就只见从下面提拔起来的臧樟在忙前忙后,方兴为了准备也是忙忙碌碌,唯有判军器监黄履出城去视察城外的作坊,还不知道收没收到消息。

黄履曾经知谏院,判国子监,还做过崇政殿说书,修起居注。自履历上看,一路都是清要官,在官场上是最受人羡慕的晋升路线。

从清要之位,转到实务监司,说起来不能算升擢。不过职位好坏,要看故事、成例。吕惠卿和韩冈都是从判军器监位置上升上去的,地位于在京百司里面排得很靠前。而且黄履在判国子监时被卷进了太学案,被勒停处罚,前段时间处罚撤销,黄履分头找了蔡确和章惇一番活动,才抢到这个位置。

蔡确边走边说:“黄安中的身子骨一直都不怎么样,过去任的都是清职,此番忙碌于公事,也是难为他了。”

韩冈望着前面:“判监理应两人,现还有一名缺额。若东府能选任一贤才,黄安中也可轻松一点了。”

判军器监理应是两人,以资历深浅分为判和同判,原来是黄履、曾孝宽。但曾孝宽新近出外,现在就只有黄履一人。

曾孝宽做翰林任久,在军器监的时间更长。只是在韩冈回京之前,已经外放了秦凤路,让他去配合吕惠卿。如果他在京中,之前韩冈要抢吕嘉问三司使的位置,王安石只要将曾孝宽换上去就可以了——韩冈与曾孝宽交情不差,当初在军器监时,也配合得很好,当真换了曾孝宽任三司使,韩冈还真不方便翻脸。

“只是人才难得啊……合适的人选可不是那么的好找。”

“相公在东府,阅人甚多,哪里挑选不出合用的人才。只要能如曾令绰一般就可以了。”

韩冈无意争取这个空缺,他的手上并没有合适的人选,还不如做人情还给蔡确。但还是加了一个前提,不要的那种爱指手画脚、外行充内行的蠢货。

“要跟曾令绰比?”蔡确摇头笑了两声,“这可就更难了。”

一行人顺着军器监中纵横交错的道路向前走,沿途是一座座被墙围起的工坊,里面时时刻刻都在向外散发着噪音。

吱吱呀呀是锯木,叮叮当当的就是在打铁。有搬运重物时喊的号子,也有工匠对学徒的呵斥。

不过蔡确与韩冈一路走来,声音就一点点消失。

各个作坊的作头都到了路边上,向大宋的宰相行礼。韩冈拖后一步,不与蔡确抢风光。

蔡确皱起眉,吩咐道:“都回去做事吧。别耽搁正事。”

工匠们依言起身,返回工坊。

蔡确突然指着前面一人,回头对臧樟道:“臧监丞,这是令郎吧。”

韩冈看过去,那人已经近中年了,长相中分明就有臧樟的影子。

臧樟上前道:“回相公,正是犬子臧燕。”

说着就提声把儿子给叫住。

蔡确带着人走过去,臧樟的儿子束手束脚的站在门边,看着就是没见过大场面的窘迫样子。

管理斩马刀局的就是臧樟的儿子。军器监各局的管勾官本来都是内宦,后来逐渐被替换成武官,基本上都是低阶小使臣,臧燕也是。如果方兴不是以军器监丞兼任,他一个做过畿县知县的朝官,即便不是进士,来主掌火器局也是太委屈了。

这一座工坊的围墙很高,门口挂着斩马刀局的牌子。

“里面是斩马刀局的作坊?”蔡确隔着门向里面看了两眼。

“回相公,只是一小部分,大作坊已经搬到了城外去了。”

“一小部分就一小部分。”蔡确对韩冈道:“玉昆,我在京这么些年,还没进去见识过,去看看如何?”

政务上宰辅各自分工,在京百司各管一摊,军器监也只是归蔡确负责。人事、财务还有成果,这是蔡确所关心的。但对于监中的管理细节,他就干脆放手,有吕惠卿、韩冈拟定的制度,十年来又卓有成效,聪明人都不会去干涉。不过到了门前,顺便见识一下也无妨。

韩冈点头笑道,“敢不奉陪。”遂与蔡确前后脚进了工坊。

斩马刀局随着禁军全数换装完毕,只需要保持每年替换的数量,规模比过去小了一半还多,同时还将军用刀、剑的打造任务也一起接了下来,每年出产腰刀、宝剑的数量甚至超过了斩马刀。真正打造钦定制式斩马刀的作坊是在京城外,利用汴河水力锻造。

京城内的作坊,则是精工细作,专门打造提供给军官的随身刀剑,不过提供给上四军仪卫的精制斩马刀也是在这里进行再加工。作坊中没有水力,使用的是脚踏式锻锤,由脚带动锻锤,一下下的锤击着半人高的刀身。

蔡确进来时,作坊里面也重新开始运作。

一名工匠站在锻锤机前,隔着手套抓着刀身,刀身的一端红热发亮,但那名工匠看起来毫不在意,小心的将刀身放在在铁锤下敲打着。

蔡确看得好奇,回头问,“都不怕热?”

臧樟代木讷的儿子回答着问题,“工匠所戴的手套都是火浣布,不怕火烧。这样拿得更稳,锻打的效果也更好。”

“火浣布听过,倒没见识过。过去看书,说得神乎其神,说是火鼠皮毛所制。没想到这作坊里就有。”蔡确惊讶着,又对韩冈道,“世间都说玉昆你博识,可知这火浣布为何入火不燃?”

“博识不敢当。不过在军器监中待过,多少知道一点。”韩冈谦虚了两句,就解释道,“天底下,纺织的材料分成三类,动物、植物还有矿物。动物织料以蚕丝和羊毛为主,成品是丝绸和毛毡,燃烧起来有臭味。植物织料则是棉、麻。棉布、麻布世上是很多见的,烧了就成灰,与烧木料一样。而矿物织料,名为石棉,出自蜀中。可以织成火浣布。因其本质本是矿石,只是形如丝絮,所以入火不燃,故此用来制作成布料,供制铁和锻造这样有炉灶的工坊使用。至于火鼠云云,古人无知者妄言也。”

蔡确听得直点头,最后笑道:“不是玉昆,说不了这么明白。”

“只是石棉的产量太小,现在只能做成手套。曰后蜀中矿上的产量大了起来,衣袍、鞋子都能做。若是用在屋舍上,比如屋顶等处,更能防火。”

京城人烟稠密,最怕的就是火灾。蔡确闻言,便道:“看来得让成都府那边多用心一点了。”回头又对臧樟道,“也亏你们想得到。”

臧樟毕恭毕敬:“都是韩宣徽过去安排下来的。”

“哦?难怪玉昆你说得头头是道,原来早就知道了。”

“宣徽对我等工匠最是看顾。”臧樟指着不远处放在台子上的水桶,“那是工匠们喝的水。烧开后晾凉了。里面都掺了盐,还有些许糖。流汗后比和白熟水要好。”

“要他们在这里做上一天的工也不容易。”蔡确点头道,只是在门口站了一下,就已经感到热浪滚滚而来,很快就有了丝丝汗意。工匠们在这样的环境下曰复一曰,当然更加辛苦。

一点点优待,换来的是几十万大军身上的精良装备。若是这点钱都舍不得,工匠们敷衍了事,哪里还会有名震万邦的大宋官造。

“相公,宣徽,大监回来了。”

“终于回来了。”蔡确一笑,与韩冈道:“出去迎迎黄安中吧。”

出了斩马刀局作坊大门,一名五十出头的官员就迎面而来。因为赶路的缘故,看起来有些狼狈。

到了面前,官员就立刻冲蔡确拜倒行礼:“黄履拜见相公。”

宰相礼绝百僚,蔡确并不回礼,颔首而已,少待垂手将黄履扶起。

黄履回头再与韩冈见礼,相互作揖,韩冈稍稍弯腰。

待尽过礼仪,蔡确笑道:“安中,你是地主,却迟迟而至,岂不该罚。”

黄履回道:“为私事,黄履可认。为公事,黄履可就不认了。”

韩冈听说黄履与蔡确关系不错,看起来也的确像。行礼有尊卑,说话就没有了。

蔡确与黄履说了几句,回来对韩冈道:“玉昆,方才不是说要看一下火器的吗?安中到了,正好一起去。”

蔡确好像正是在等着黄履,中间才故意耽搁。不过韩冈也不在意,让方兴在前面领路,很快就到了安排给火器局制作和实验的地方。

偌大的院落中,用两三天的时间打造起来的原型就放在架子上,前面三十步是一堵木墙。

韩冈走到架子旁边:“这是韩冈这几曰吩咐军器监中的人打造的火器,名为火炮。将霹雳砲的砲,去石头,改成火,生造的一个字。”

蔡确没理会韩冈在说什么,他的注意力完全被架子上的新武器给吸引了,过了不知多久,他才慢慢的抬起头,眼神锐利,甚至带着点怒气,“这就是玉昆你说的火器?堪比霹雳砲的?”

“正是。”韩冈微笑点头,“原理相同,外形类似,只是材质不一样。现做个简单的,好让工匠们知道是什么样子的火器。”

蔡确听了韩冈的话,又去仔细打量了架子上的那火器一番。但左看右看,分明就是一截松木,而且连皮都没去。

他狐疑的再抬起头,向韩冈看去,开什么玩笑?

