\section{第48章 梦尽乾坤覆残杯(四)}

薛向的话,让向皇后腾起一丝希望,但随即被章惇打破了。

“《春秋》有载,许世子止弑其君买。”

知枢密院事冷冰冰的述说着。不论是故杀还是误杀,在孔子那里,都是一个弑字。

“这样啊……”

向皇后没了声息。纵使是没读过多少书的太上皇后,也知道弑这个字有多么沉重。圣人的文章,一字都难以更易,既然说误杀也是弑君、弑亲,那就是不可饶如的重罪。

宰辅们也寂寂无声。几个月前,他们才拥立上台的天子犯下了如此大错,也让他们进退两难。

弑父之罪,历数过往中国君王,隋炀帝算是比较有名的。

没名气的还有一些。南北朝的南宋刘劭、北魏拓跋嗣,五代梁朝的朱友珪,以及一些外国、番邦。

不过也就隋炀帝多坐了几年江山,其他几位事后都没有活过一年半载。

弑父之罪,天地不容。弑君之罪,同样难容于天地。

同时犯了两条滔天大罪,哪里还有容身之处?

赵煦虽然是无心之过,可是有圣人的如椽铁笔在前,任何理由和借口都难以帮他洗脱。

只是赵煦才六岁,什么都不懂的孩童。他吩咐宫人密封帐幕,纯属一片赤子之心,真要归罪于他,也极难说得过去。

这跟已经成年的许止不同。许止进药害死了其父,还会有人怀疑其中有什么情弊,一个六岁的孩子,又哪里会有那么复杂的心思。就是怀疑,也会怀疑到向皇后的身上。

各自的头脑中都是一团浆糊,如果这是一件发生在普通人家的案子,都不是那么容易能析断明白,何况还是发生在天子身上?

春秋决狱说君子原心,不当以罪诛,可不代表无过。弑亲之人,到底有没有资格再继承家业,谁能判得让人心服口服?

现在将这件事放在赵煦身上,就是他这个皇帝,到底还能不能做下去的问题。

就是向皇后也很清楚现在的局面有多么的糟糕,“众位卿家,现在该如何是好?”

叹了一声,韩冈出班,脱下官帽,拜倒于地:“天子有过。臣忝为帝师,教导无方,实难辞其咎。”

自确认了赵顼的死因后,王安石头脑一直都是昏昏沉沉,对他有知遇之恩的皇帝却死在了他的学生手中,本来就因赵顼之亡而伤心的时候,却又撞上了这桩人伦惨剧。

老年人最忌大喜大悲,今天的事,放在其他宰辅身上,只会让他们思前想后、考虑得失,只有王安石心痛如绞,反应也变得迟钝了。直到看到女婿出来请罪,这才稍稍清醒过来。同样是免冠伏地:“臣亦有罪。”

“宣徽!相公!”向皇后急声道,“你们这是做什么?!”

还能做什么?韩绛想着。

当年商鞅变法,太子犯法,商鞅不是找太子的麻烦,而是将太子的两位老师处置了,一个脸上刺了字,另一个则将脚剁了。

韩冈和王安石从赵煦还是太子的时候就做了他的老师,如今小皇帝犯下了弑父之罪,他们怎么能

置身事外?

就算这是因孝心而起的意外,两人,包括还不在场的程颢,至少都得辞官去职才能抵得过。

不过韩绛作为首相也不能干看着,“介甫、玉昆,现在首要之务是该怎么对天下臣民说这件事,不是引罪请辞的时候。”

韩冈随即起身,又搀扶了王安石一把。

请罪是必要的表态,既然已经表明了,就没必要再跪着了。

整理好衣冠,韩冈对向皇后道:“殿下。这件事不可能保密了。”

“没有别的办法了?”

如果仅仅是赵顼病死,谁都不会认为哪里有问题。在中风后,而且是后遗症极为严重的情况下,能拖这么长时间,已经可以算是奇迹了。

可是偏偏又有包括一名御医在内的三人与赵顼同时死亡,这就不能不让人产生联想。

到底是什么原因会造成太上皇和御医、宫人一起丧命?

会有人认为这是正常的病故吗?还是一个不幸的意外?

都不可能,外界的猜测只会往谋杀的方向偏过去。甚至有些有心人,还会故意将事情往那个方向扭转。

蔡确叹道:“殿下。三条人命在,堵不住天下悠悠之口。”

真相可以掩盖,但三条人命掩盖不了。皇权虽重,控制力却如筛子一般,越是强要封锁消息,就越是会传得满城风雨。

而且在列的诸位宰执,也没人会愿意为赵煦掩盖事实。

这对他们有什么好处?

除了引火烧身,让世人怀疑起自己也参与到弑君的罪行中,赵煦成年之后,更是会想方设法的杀人灭口。

在列的哪一个不是熟读史书,就是进门后一句话没说过的郭逵也都将春秋和诸史翻了一遍又一遍。

看多了史书,有哪一个会相信皇帝的人品?即使君臣相得如李世民、魏征,到最后还不是以悔婚毁碑为结局?

帮小皇帝瞒下太上皇驾崩的真相,最后得到的绝不会是感激和三代富贵,而是满门抄斩。

蔡确心中哀叹,这一回,定策、拥立的功劳是彻底作废了,当初的辛苦到底是为了什么?真还不如王珪那般直接离开朝堂来的省心。

他视线掠过一众同列,这里面,有多少会为才登基的小皇帝赴汤蹈火的?

恩未施,信未立,威权还不知在哪里,对未来的收益更无法期待,现在还有谁会忠心于他?

恐怕只要想通了之后,即便是向皇后也不愿意不明不白的将这一次的意外瞒过去。否则外界都会怀疑到她身上,而赵煦曰后也肯定会设法将罪名推给她,然后以为先帝复仇的名义,将向皇后和向家打落深渊,来个死无对证。

可一旦公开的话,赵煦就很难再坐在天子之位上。年纪再小,也得为他做的事负责。

换一个皇帝,这话说得简单,可事情却哪里能那么容易就做得出的。废立天子,

蔡确犹豫不定,无法有一个决断。

不仅是他,就是王安石、韩绛,不敢也不愿说出有关废立的字眼。

只有姓格勇毅,胆大包天的大臣才能领头做出决断。

章惇、韩冈一时为众人所注目。

章惇率先站了出来,“殿下,以臣之见,此事必须向百官公开。毂辇下一同事主,官阶有尊卑,国事难共商。但事关天子、社稷,此事却不可隐瞒。”

没人反对,这个真相实在太过沉重,谁也不愿意压在自己身上。

韩冈、蔡确之前也表态过了,这件事既然无法隐瞒,当然就得尽快公开。至少要将主动权抓在自己手中,也免得事情泄露后变得被动。

只是章惇还是没有说到其他宰辅所关心的话题。

“官家那边怎么办?”向皇后问道。

“……”章惇张开口,却没有声音,这个决定可不好下。

如果要废帝另立,不可能拥立两位亲王的儿子,只要赵顼的两位弟弟还活着,就不可能让他们的儿子当皇帝。另外也不可能刻意再立幼主。为防年幼夭折,至少得十岁出头。这样的话,几年后就到了亲政的年纪。多半还是要在濮王一系中再做甄选。

但废掉皇帝的话,岂是这么容易能说出来的?首倡废立,可不是什么好事。

不过章惇终究果决,不顾仪态的舔了舔嘴唇,正要说话,却被韩冈打断了。

“殿下,此事不是区区十数人能做决断,还请招在京的侍制以上官共议。”

韩冈的提议似乎是顺理成章,但却让人匪夷所思。顿时,十几道含怒夹忿的眼神就像标枪般投射过来。

这等于是将宰辅们好不容易抓在手中的权柄,分给所有侍制以上的重臣。

韩冈这是疯了吗?张璪在想。韩冈虽然是宣徽使,可参政议政的地位却绝不下于枢密使和参知政事。

蔡确惊讶得说不出话来,韩冈到了现在还要保着小皇帝?

谋不可以决于众人。这是谁都明白的道理。

人越多,就越难做出极端的选择。除非有人引导,否则必然分作数派相互攻击收场,最后商议和妥协的结果只会是保持现状。

章惇也面露怒色,瞪着韩冈。

虽然说只要事情公开了,灭口就毫无意义。

不过背着弑父之罪的皇帝,谁敢让他留在天子之位上?不怕他自暴自弃,干脆做一个隋炀帝?

就像是参与过屠杀的军队,谁也不敢将他们召回国中。纵使再善战,也不能让他们戍卫京城。

不要指望疯子能念着旧恩啊!

可是章惇几次想开口,却都没有说出话来——他终究不是霍光。

胆子最大的章惇不站出来,谁敢于出面反对韩冈的意见?不说别的,只要反对的态度传出去后,文武百官那边可都要得罪了。

向皇后犹豫了一阵,终于点头,“就依宣徽的意思。”

“臣还有话说。”韩冈却又说道。

“宣徽请说。”

“天子是上皇唯一的血脉,无论如何都必须保全。”

韩冈望着向皇后,想必她不会愿意重蹈曹太后的覆辙。
