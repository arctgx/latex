\section{第20章 土中骨石千载迷(11)}

自从西京的元老重臣们接连本,几天时间的过去后,天子、请求发掘殷墟的朝臣也越来越多,住在南京的几名老臣言辞恳切的向天子请求。但天子始终没有一个回应,一直保持着令人玩味的沉默,这让朝廷之中的气氛变得有些诡异起来。

谁都不知道天子是不是又要强行将这件事压下去,许多人都等着想看看韩冈还能有什么招数来应变。

不过处在漩涡之中的韩冈,则是安安稳稳的在编修局中主持《本草纲目》的编修工作,顺便甲骨文的整理和拓印让兴趣浓厚的黄裳负责了——反正这件事可以慢慢来,并不耽搁他迎考的复习——另一便则继续整顿厚生司和太医局的工作,只是心里面藏着怨愤越发的浓重起来。

韩冈尊敬王安石、张载、程颢还有许多儒者的品行和为人,但对儒生们皓首穷经的行为,很难给予更多的认同,这世可由有着更多的正经事该做。可身在这个时代,却不得不披一层羊皮,得小心谨慎的将有益于天下的知识一点点的放出来。

虽说这也是为了尽量不浪费这些知识所能给自己带来的利益,可这般小心翼翼如同做贼的行事作风,加之时时提防被人拆穿,十年时间不得不苦读儒典经籍,要说韩冈不觉得憋屈,那是绝对是谎话。眼下皇帝一直做着绊脚石,韩冈可是越发的看天子不顺眼,只是这些心思只能藏在心底。

每天照常在太常寺中处理三个衙门的一应公事,到了日暮放衙后,便照常回家,并不去酒楼去招妓饮宴,也极少接受他人的邀请。

但这一日到了午后,一封署名韩缜的请帖送到了韩冈的案头,考虑了片刻之后,韩冈在请贴给了一个肯定的回复。

参知政事韩缜请客,由于过去曾在群牧司**事,加在胜州划界谈判韩冈帮了不少忙,韩冈与韩缜之间有几分交情在,他设宴请客,韩冈也的确不便拒绝。

苏颂就在旁边看着韩冈将韩缜的家人打发出去,便随口问道:“韩玉汝无缘无故请客,打得到底是什么主意?”

韩冈摇着头:“说不清楚,反正不会是吃饭喝酒。”韩缜的宴请是打探消息,还是代替某人传话,韩冈一时间也没办法猜得透,但要说请客只为吃饭聊天,韩冈和韩缜的交情还不到这一步。

“说不定是请玉昆你赏花的,秋菊再不赏,就只能等着赏梅花了。”

“那也要韩冈会做诗才行……不过倒是有一件事可以确定。”

“什么?”苏颂问道。

“绝对不会是请韩冈去联谱联宗的。”韩冈笑着说道。

苏颂闻言,当即一阵大笑,笑罢却又道:“那可说不准,有玉昆你在,别说灵寿韩,就是安阳韩,也照样愿意交你这门亲啊。”

韩冈的嘴角向下扯了一下,“寒门素户,可是不敢高攀。”

相州安阳出身的韩琦家就不说了,相三帝立二主,天子都要承他的情,乃是外臣之中,最为显贵的一门。灵寿韩家,韩亿是参知政事,韩绛是宰相,韩缜现如今也是参知政事,再往前,也是历代为显宦,标准的簪缨世家。虽说比不从东汉到隋唐延续几百年的山东士族,但也是当世数得着的豪门。

而韩冈人人都视其为宰相之备,不出意外,日后必然能入居东府。如此一来,韩姓在这天水一朝,可是数得着的显赫。只是三韩并非一族,一句八百年前是一家也勉强得很。

不过今世间同姓联宗攀亲的多,尤其是门第不显的寒门士人,都愿意攀个贵胄同姓,是不是同族也没人在意。就是曾经垂帘听政的章献明肃太后刘娥,也因为自家的寒微出身,想与一刘姓重臣联宗,认下一门亲,只是给人毫不犹豫的拒绝了。

韩冈也是没兴趣随便跟人攀亲,早年还是寒微小臣时,连两个韩姓显贵家的大门都没有进过一次,现在就更不需要了。而且在清议中,这样的事终究还是会惹人非议,在天子那里,更是对前途有碍。

韩冈和苏颂说笑了几句,这件事也就放在了一边去了。待到放衙之后,韩缜派来的家丁便已在太常寺门前等候韩冈,在前面引路,一路将韩冈领到了参知政事的宅邸前。

参知政事的府,求见的官员数以百计,如同当年王安石任职东府,王韶担任枢密副使时那般车马盈巷。但韩缜的儿子就在巷口迎接韩冈的到来,让堵在巷中的人马全都避让了开去,径直入了韩缜府。

韩缜设下的是私宴,请的只是韩冈一人,也知道韩冈好清静的性子,并没有将家里养的伎乐搬出来表演,但累世簪缨,世家的底蕴远不是寒门可比,器皿和食材都是第一流的。

坐在在池畔小轩中,凭栏而亡,月色下,庭院中假山和水塘的景致尽收眼底,却因为生得极旺的炉火,而一点感受不到深秋之夜的寒意。

与韩冈对饮了一杯烫过的烧刀子,韩缜叹着满口的酒气;“眼见着就要入冬了,今年又是南郊之年,下下都是忙得脚不沾地,要不是开封府今天终于将圜丘和青城行宫给修好了,也没有个空闲。”


“尚幸太常寺中倒是清闲。”

见韩缜不忙着进入正题,韩冈也不急,笑着饮酒吃菜,韩缜家菜肴的口味还当真不错。严素心和家里的厨子虽然也不差,但还是比不豪门家宅里面的名厨。

“太常寺不涉礼制,也就本朝如此。县令不在县,刺史不在州,六部九寺没一个实职。这官制乱得跟一团麻似的。”

“不是有传言说朝廷要改制吗?”韩冈道,“若真能正本清源,倒也是不错。”

“那样的话,玉昆你这个太常寺可就要忙起来了。”

“那还是不要改的好。”韩冈哈哈笑道,“清贵的差事可是难找的很。”

韩缜也笑了起来,斟满酒又与韩冈对饮了一杯。

韩冈放下酒杯:“对了,听说这一次南郊,家岳要改国转封了?”

