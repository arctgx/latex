\section{第28章 大梁软红骤雨狂(五)}

【今天晚上赶不下去了,就只有一章。缺的明天补上。】

新磨的铜镜光可鉴人,镜中的一张如花俏脸有着倾城之姿,却是略显憔悴。修长的双眉微蹙,眼波流光,笼罩着愁云。镜面明晃晃的,照出了镜子的主人这一年来所受到的相思之苦。

周南只穿着一身素白的亵衣,坐于镜前。对着镜中的自己,慢慢梳理着如墨染过的秀发。青楼之中的生活,向来都是晨昏颠倒,西窗外泛着亮色的红光,而她才刚刚起床。

玉色的纤手捏着牛角梳,从丝缎般的长发中滑过,早间出去买胭脂水粉的墨文,正站在她的身后。

周南百无聊赖的梳着头,神色间透着麻木,日复一日在欢场上重复着的生活,早已耗尽了她的心力。但随着身后小女使的几句话,脸上的呆滞转瞬消失不见,先是惊讶,而后转为狂喜:“什么!你见到韩郎了!”

墨文被周南的一声惊呼吓了一跳,身子一震,不禁退后了半步。

周南已经转身跳过来,两只手像捉小鸡一样,一下抓住了墨文的双臂。双眼闪亮如含着星光,追问着:“你见到韩郎了!?”

墨文直点着头,“看到了,看到了,就是在胭脂铺的时候看见韩官人骑马过去的。”

“不会看错吧……怎么不叫住他的……应该是他……还不到一年时间……”

周南一时间陷入混乱之中,说了好一通,都不知自己在说什么。反而是墨文比较清醒,“小婢看着韩官人往城南驿去了,应该是刚刚进京。”

“快让人备车,我要去城南……”周南突然说不下去了,患得患失的神情出现在脸上,万一那冤家已经忘了自己呢?前次有个赵隆送信来,后来又有个王舜臣带了私信从秦州来,但今次韩冈的恩主王韶率归顺朝廷的蕃人入京,声势浩大,天子连续数次招他进宫。周南一直都期待,可就是没有等到半封信。

“墨文,还是你……不,还是我……”教坊司的花中魁首犹犹豫豫,始终拿不定主意。

她当然想早一点见到情郎,但又怕见到心中的那人后,听到的话语会让她绝望。其实周南几乎都快要绝望了,因为最近一直纠缠她的那一人,让周南不敢去确认,她的心上人到底有没有勇气为了她去对抗。

“周姐姐。”门外这时有人唤着周南,“秦二官人又来了,请姐姐快点过去。”

“啊,二大王来了!……姐姐,怎么办?!”

墨文慌张了起来。秦二官人就是先皇英宗的次子,当今天子的二弟。如今他的封国为雍,是为雍王,而雍州乃秦地,所以便以秦为化名。毕竟身为皇弟,总不能光明正大的出来逛窑子。

“周姐姐……”门外的人见房内没有回音,又催促的喊着。

“这就来!”墨文代周南应了一声,又对周南问道:“姐姐,你看现在怎么办?”

“真烦人。”周南的一张俏脸这是已经挂了下来。若是普通的客人,只要推说一句‘倦了’,就能搪塞过去。但雍王身份不同,哪里能怠慢?

眼下虽然赵颢都是从后门进来,只听一曲,喝两杯酒就匆匆而去,从没有留夜的意思,但谁也说不准他什么时候就会得寸进尺。要是雍王殿下用强,难道还能真的捅他一刀不成。现在管着周南的许大娘,甚至把屋里的剪刀都收起来了。雍王要是真的有意,只要露点口风,许大娘肯定会把周南现在随身带的匕首给悄悄收走。

周南从枕下拿起一块叠好的丝巾,白色的绢绸上绣着一对活灵活现的鸳鸯,是他几个月来的的心血。递给自己的小女使:“墨文,你待会儿代我去城南驿,悄悄的把这手巾交给韩官人,不要给人看到。”

墨文接过丝巾,收在怀里。又问:“只把丝巾给韩官人就行?”

“……够了,应该够了。”周南有点艰难的点着头,她的心中也没有底。

墨文应下了,便帮着周南更衣上妆,片刻之后,艳冠群芳的花魁便仪态万方的出现在雍王殿下所在的小厅中。

坐在厅中正位的年轻人,相貌还算俊秀。穿着士子襕衫,装束都是再朴素不过,乍看上去,就是一个普通的穷书生。但世上哪有能隔三差五就逛窑子的穷书生,而且还是达官贵人才能光顾的地方。何况教坊司中人去宫中的次数不少,颇有几个见过当今雍王殿下的。而赵颢带出来的伴当,竟然还是一个阉宦。

雍王殿下的身份,其实在一开始就被人揭穿。但一国亲王做这等掩耳盗铃之事,教坊司中上下,也只能当作认不出,看不到。

赵颢见着周南进来,如果是普通的妓女,看一眼也就过去了,就算长得貌如天仙,对于天子亲弟来说也是等闲。他现今尚居于宫中,见过的绝色甚多,并不比周南差到哪里。只是听说了周南执匕吓走了一个宗室,是风月班中难得的刚烈女子,他才有了兴趣。

“秦二官人万福。”周南盈盈下拜。

“一日不见,如隔三秋。我这数日,对周小娘子的绝妙歌舞可是日思夜想,辗转反侧啊……”赵颢则是装着一副花丛老手的模样,只是在周南眼里,却是全然无趣。

用着虚伪的笑容陪着喝了两杯酒,周南站到厅中,曲乐声起,随着乐声歌舞翩翩。伴着欢快的曲乐,载歌载舞的女子,颜如牡丹,色如芍药,回身旋舞时,衣袂飘然有如百花绽放,而神色间又有着拒人千里外的凛然。

正是这种不可轻辱的凛然,和她作为歌妓花魁身份之间的错位,吸引了赵颢的目光。他眯起眼,双手打着节拍,享受着难得轻松的时刻。

虽然已经娶妻生子,但赵颢如今还住在宫内,因为谁也不能犟得过他的那位贵为太后的娘亲。只是赵颢虽然在兄弟中最受疼爱,但身处在大内之中,身心照样都受到压抑。他跟自家的王妃又是合不来,现在也只能在安仁坊这边寻一个放松的机会。

看着周南柔美动人的舞姿,赵颢想着自己的王妃。虽是国初历任太祖、太宗、真、仁四朝的名相冯拯的曾孙女,却是个让人感到乏味,却又善妒的女人。两女的身份天差地远,但给他的感觉则是有着完全相反的差距。

要是她知道自己出来逛窑子,不知会不会向娘娘哭诉。

想起自己亲生母亲,赵颢心中突然一阵虚怯,忙喝了一口酒压惊。他心中明白,自家的亲娘纵然再疼爱自己,也不会喜欢他私下里出宫来逛窑子的这些事。就是因为害怕如今的太后,赵颢连度夜也不敢,只能稍坐片刻就离去。

就在过去也没几年的治平年间,当时赵颢的父亲,也就是先皇英宗赵曙,即位后不久便发病,不能理事,如今的太皇太后出来垂帘听政。等到父皇病愈,太皇归征,赵颢的母亲仍不许赵曙亲近嫔妃。

曹太皇当时让人传话劝诫:“官家即位已久,今圣躬又痊平,岂得左右无一侍御者。”

而身为曹太皇的亲侄女,又是自幼被抚养在宫中,关系如同母女一般亲近,但赵颢的母后还是硬邦邦的回话道:“奏知娘娘,新妇只嫁得十三团练,即不曾嫁他官家。”

这件事在京城穿得沸沸扬扬,隐隐的,还有人拿隋文的独孤皇后来比较。曹太皇当年被仁宗立为皇后,从来不干涉仁宗在后宫中宠信谁人,故而人人称其盛德。但现今换作了评价高太后,世人不便说其悍妒,便用严肃两个字来形容。

也因此,先帝英宗虽然有嫔妃,但赵顼、赵颢他们排在前头的兄妹几个,可都是一母同胞。

对于如今大宋国的皇太后,太皇太后压不住,先皇也压不住,而皇兄当然也凹不过。她想日日见到儿子,赵颢、赵頵两兄弟便都留在了宫中。

在前两年,有个姓章的小臣说赵颢他和他的四弟赵頵已经成年还留在宫中,于礼不合,当赐邸于外。当时赞同此事的人不少,如今的宰相王安石,也上书表示同意。但当太后一通火后,那个小臣就被赶出京去,连王安石都不敢再说什么,几年过去了,也没人再提这茬事。

赵颢本身也有一份心思在,所以也没有离开宫中的打算。不过最近宫中喜信频传,而自家则是闱内生乱,他心中就有些烦闷,才会出来散散心,否则,他肯定是在宫中做一个老实听话的乖儿子。

这是赵颢的秘密,从未对外人道。当然,雍王殿下并不知道市井传言的威力,他自以为隐秘的举动,早就传遍了京城,而监察京中内外的皇城司那边,自然也收到了报告。要不是顾忌着高太后,早就给御史和皇城司捅上去了。

一杯酒喝下去,摇了摇头,雍王殿下不去再想那些让他烦心的琐事,很快沉醉于眼前的歌舞之中。

