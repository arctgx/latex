\section{第九章 拄剑握槊意未销(三)}

大约五百多人的骑兵停驻在灵州川的荒滩边,红裳锦袍,是典型的大宋马军。

战马一群群的在河边上喝水,正常的情况下,它们的主人在喂马、饮马之后,都会顺便就着河水洗刷一下,这样骑着才算精神。但现在几乎所有的骑兵士卒,却是连照料的念头都没有,而是横七竖八的带着战马找了树荫躺了下来。

河边的五百骑兵,已经完全失去了一支军队应有的秩序。蓬头垢面,衣衫不整,旗号尽失,就连盔甲,也不见几人还带在身边。身上有伤的用布条胡乱裹了一下,没带伤的也跟乞丐没有多少区别。

有人闭着眼睛休息;有人在伤口的创痛中呻吟;有人则是发着呆,双眼死鱼一般瞪着;还有些人,眼睛滴溜溜的转来转去,却不知在想些什么。但每一个人的手中都紧紧的攥着坐骑的缰绳,就是睡着了,都不见松手。

种诂半闭着眼睛,坐在一块石头上。对于麾下士卒的颓丧和军纪的混乱,他已经能做到视而不见。

一场败仗之后紧跟着连续数日的追杀,全军上下现在惶惶然如同夜里发现黄鼠狼进了窝的母鸡,彻底乱了阵脚。十万大军在西贼的追击下散了鸭子。被追杀得别说脸面了,就是底子都丢光了。

就是现在回想起当日,种诂也觉得败得实在是太突然了。十年来的累累胜绩,在这一战中化为乌有。

种诂还能记得当日城破在即,从战场那边产来的战鼓声都洋溢着得意。谁能想到西贼竟然能决堤放水,一下就让攻城大军近乎崩溃。

之后灵州城中杀出来的骑兵,加上兴庆府方向的伏兵同时来袭,外围的泾原军被水势分割,无法会合,加上慌乱,一下子就崩溃了,接着就是包括种诂在内的两路骑兵被数倍于己的铁鹞子击败,接下来就是身在灵州城下的环庆军,也同样是在一片混乱中全军溃散。

水势漫过膝盖,对步兵的影响很大,但对骑兵而已则仅仅是小有阻碍,种诂当时不在正面战场,没看到中军主力如何失败,但之后但他率部撤向中军方向时,就看到全军跑得漫山遍野。从时间上看,环庆军的抵挡连一时半刻都没有。

之后两军残部会合,高遵裕强令苗授殿后,而苗授又把这个任务交给了运气不好的种诂。最后的结果就是只剩下三分之一的人马——这可是骑兵啊,有那么多步兵逃散的情况下,根本就不该有这么大的伤亡。

“皇城。”亲兵提着水袋小跑着过来,毕恭毕敬的递给种诂。他两眼红通通的,灰尘密布的脸上还有两道明显的泪痕,

种诂伸出左手接过水袋,用牙齿拔掉塞子,大口喝起亲兵刚刚打来的河水。他右臂则是直直的垂下来,不见动弹。

泾原路为环庆路殿后,而种诂以第三将的骑兵为整个泾原路殿后,一路连番大战,损兵折将的同时,种诂本人也难得幸免,暂时只剩一条胳膊能用了。

前天最危险的时候,身边的亲兵都给杀散,他一人被七八名铁鹞子围住。

种诂从来都不是以武艺著称的将领,其少年时曾以叔祖隐君种放为榜样,号为小隐君,心思放在文事上,在兄弟中枪棒、弓弩都是倒着数,也只比普通的军官略强那么一点。现在年纪大了,武技也在不断退步中。

就在前天的混战中,种诂拼了命才用铁枪扎翻了两个武艺最强的西贼,肩膀上却挨了一铁锏,幸好仅仅是废了肩甲,事后一看,整块铁板都扭曲了。不过好歹把下面的肩胛骨给保住了,只是伤了筋,得修养好一阵子……但运气不好时,说不定一辈子都得与这个伤处打交道。

种诂对此倒没什么好在意了,他都往六十岁走的人,说一辈子,其实也就几年十几年而已。以自家先人的寿数,种诂也不指望自己能活到八十岁。

冰凉的河水压住了心中的焦躁,种诂放下只剩一半的水囊,正看见亲兵脸上的两道泪痕,问道,“怎么了,哭什么?”

“皇城。”亲兵低着头,抽噎的道:“二哥、八哥他们……”

“哭个屁,要嚎丧回去再说!上阵你见过不死人的!?”种诂呵斥了一声,寒着脸站了起来。

“皇城,这就要走了?”亲兵急道,“要不要再等一下,十一哥说不定还能赶上来。”

“等什么?怎么等!”种诂下面的双手紧紧握着拳头,并不是他心中不痛,只是不愿表露出来,“十一有那个命,自己就能逃回来,没那个命,等也没用!”

就在两天前,他麾下的骑兵虽然败阵,至少还有个军队的模样。但连续数日的殿后阻敌,不喜欢读书、只顾着练武的次子战死;笑起来憨厚得很的八侄儿战死;关系一向不错的三个指挥使战死;跟在自己身边多年的亲兵们有一半战死;听命敢战的精锐一个个战死疆场,活下来的全都是滑头。

整整四天的断后,种诂手上三个指挥的骑兵,只剩下眼前的一群惯看风色、双脚麻利的老兵油子。想让他们拼命杀敌,纯属做梦,就是天王老子来了都没用。

种诂向着北面张望了一下,虽说能逃出来的都逃出来了,但其实还有许多人并没有被确认阵亡。比如十一,也就是自己的第四个儿子;比如好些个副指挥使和都头,只是在战场上的混乱中失去了踪影,并不是说他们一定就不会再回来。

只是现在不可能回头去找他们,也不可能在这里久留,下面的士兵哪一个都不可能老实听话的留在西贼随时都会追上来的地方,都想及早赶到韦州。

种诂并不清楚他的顶头上司究竟是在韦州,还是逃往更南面的地方,甚至一直逃回横山南侧,但之前说好的就是在韦州会合。再说有城墙的地方总比荒郊野地更能睡个安心觉,只希望西贼没有绕道前方,抢先夺下韦州。

单手一撑马背,种诂跳上马,抬起马鞭,指着前方:“前面就是韦州,早前感到城中,今晚可以好生歇一歇。”

败兵们看到他的动作,也一个个都起身上马。但有十数人的坐骑,刚刚骑上去,就一声惨嘶,轰然倒地。

没人关心他们,几天的追逐战,倒毙于途的战马见得多了。只是握紧了手上的兵器,防着他们过来抢夺自己的战马。但那十几人脸上先是绝望、继而又转为凶戾。

种诂懒得为此说话,麾下的这一干奸猾之辈,多一个少一个都无所谓。打马前行,根本都不管身后的事。

半日之后,韦州城遥遥在望。看到了城上官军的旗号犹存,种诂终于放心下来。

进城时费了一番周折,城中守军如同惊弓之鸟,多番查验身份,才将种诂一众放进了韦州城中。

被上百柄神臂弓指了半日,种诂的脸色越发的难看。被引去参见主帅时,还是一样的板着脸。

在州衙中,种诂见到了高遵裕。苗授不在,据说是受了重伤,在随军的疗养院中躺着。

惨败之下,高遵裕变得反应迟钝,神思恍惚。他的腰甚至都是驼着,往日根本看不到太后亲叔这幅模样。

种诂心知高遵裕是给失败打懵了。他好歹还经历过三十年前的三次惨败中的两次,也亲眼见证过之后十几年党项人肆无忌惮的杀入国中劫掠,顺便还毫不脸红的将朝廷的岁赐搬回去的情形。眼下的败阵,还不至于让他变得灰心丧气,但高遵裕就没这份被磨练出来的坚韧了。

主帅都这般模样,下面的士卒就不用提了。不管韦州城中还剩多少兵力,看起来都不像还能支撑得住的模样。

“高总管。”种诂拱了拱手,行了个礼。

换做往日,高遵裕好歹还能记得安抚一下在后方拼死阻敌的种诂,但现在没有那个心思,“贼军还有多远?”他问道。

种诂没心思计较这等小事,“之前四日,末将与西贼接战数十次,发现是三支铁鹞子轮番追击。不过昨夜他们都没追上来,多半是为了将息马力,算时间差了有半天的路程。”

种诂自知,要不是党项人不想战马在追逐战中劳累过度,死得太多,他根本就逃不回来。逃命的宋军可以不顾战马的生死,但党项人却不能不顾。

“半天啊……”高遵裕紧皱着眉。

“不知接下来该怎么办?”种诂问道:“西贼休整之后,肯定还会追上来,是要坚守韦州吗?”

高遵裕犹豫了一阵,抬眼问种诂:“大质【种诂字】意下如何?”

种诂他没有背黑锅的打算,抱拳道:“还请总管示下。”

高遵裕凝神注视种诂好一阵,最后一摆手,“你先下去歇着吧,这几日辛苦你了。”

种诂行礼之后,转身出厅。

种诂不看好接下来的战局,追击自己的三支铁鹞子加起来都没超过一万五,可见其主力有更为重要的工作要完成——王中正那一路危险了。如果王中正再败,这一战就没法儿打了。

不知道朝堂上能不能看清着一点。

种诂叹了一声,这要看京城中的反应了。以军情传递的速度,金牌急脚递将战败的消息传到京城,也就在这两天。

