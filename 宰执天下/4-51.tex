\section{第11章 安得良策援南土(一)}

“射得好!”

苏缄大声的夸奖着出城袭敌的勇士们。虽然离得远了,不知道战果到底如何,但还是能看得清有几人是被抬着走了。射杀几人,苏缄不在乎,但在李常杰的将旗在城下升起之后,狠狠打压了交趾军的气焰,却是他看得最为开心的事。

就在州衙前,苏缄亲自端起斟满酒浆的银杯,将两艘船中从弩手到桨手,一个一个全都敬过一遍。

看见高高在上的苏皇城亲自给他们这群脸上刺字的军汉敬酒,人人激动不已,都是跪下来磕过一个头,再接了酒一口饮尽。

等一轮酒敬过,苏缄再一指身侧。

他今天开了府库,将库中积存的财物全都搬了出来。一串串铜钱,一匹匹锦缎,还有铸成小锭的金银,全装在箱子中,摆在了州衙门前的空地上。炫花了围观的数千军民的双眼。

苏缄高声喝着:“出战前本官已经许诺过,只要敢出城杀敌,每人都是二十贯大钱、二十匹彩绢。本官言出必行。”

京城中的上位禁军的俸禄,一个月才一贯钱,四匹素绢。而在广西这边的下位禁军,甚至连一半都不到。更别说厢军和溪洞土丁。二十贯铜钱,二十匹彩绢,除非三五年不吃不喝才能攒下来。

知州的敬酒,再加上丰厚的赏赐,不仅受赏的士卒兴奋得脸红,连周围围观的军民也看得眼红了。

“各位将士杀敌,本官也不吝重赏。如今只是财帛,等到杀退贼人,更会将诸位的姓名上报朝廷,让天子亲授封赠!”

三十余名官兵一齐拜倒于地,齐声欢呼:“多谢皇城恩典!”站起来后,更是兴奋的无以名状。一旦报上去,没官的少说也会有个一官半职,而有官的,更是加官晋爵没得跑,这让他们怎么不兴奋?

苏缄也一样心情舒畅,这是在赏功,但也是为了鼓动士气、战意的手段。只是他的手段还不止于此。

“把军器都搬出来!”

待到欢呼声稍歇,苏缄提气喝了一声,登时就有一群士兵抱着一具具神臂弓穿过人群,走到苏缄的面前。

“排开了!”

苏缄须发颤动,再一声大喝。

神臂弓一架架的被平放在地上,排得整整齐齐。这些神臂弓如果直接去数,其实数量并不多。只是在州衙门前一张张平铺开来,却是涨满了视野。除此之外,还有其他的刀枪弓弩,也都一起摆了出来,以壮声势。

琳琅满目的军械,让人望之心安。至少可以知道,对于贼人的来犯,城中不是没有准备。

苏缄弯腰拿起一张重弩,举起来对着周围的军民道:“神臂弓的威力,各位都看到了!这乃是军国利器,杀贼犹如割草一般。就算是契丹、党项,也不敢直撄其锋,何况区区南蛮?我城中有此物在手,试问贼人何能破城?!”

苏缄高声宣扬着神臂弓的威力,但他心中藏着深深的遗憾。

要是没有为了防止城中居民开城逃离而将城门用砖石填起,前日用神臂弓将贼军射得狼狈而逃的时候,就可以趁机出城追杀一番。即便只能派出千人,也能大败贼军,给交趾人一个教训——这实在是太可惜了。

交趾从来没有受过教训。

太宗时的南征也是以失败而告终。自从五代分裂出去之后,交趾一直以中国自居。欺压四邻,其国主甚至在国中称帝。对此等悖逆不道之举,朝廷却一直是采取着视而不见的态度,不想在南方生事。姑息养奸的策略,如今终于见到恶果。

苏缄反对对交趾姑息养奸,但沈起、刘彝调来广西之后的举动,他同样反对。尤其是刘彝的那种将所有的侬人蛮部,全都推到交趾那一边的愚蠢之举,更是让他从来没少上书过。禁绝市易,最吃亏的不是交趾,而是两国之间的蛮部。而且之前对侵占广源州不闻不问,其实也是将出产黄金、兵员的边州送给交趾人的愚行。

交趾不过旧唐的数州边地,合起来也难跟广西一路相比,但朝廷几十年来的行事方略,却让交趾人一年年的变得贪婪、骄横,不过一嘬尔小国,竟然敢兵临中国,完全不将朝廷放下心上。而国中之人,也视交趾如虎,钦州、廉州、太平寨、永平寨,交趾人北返的一路上,几多城寨都没有坚守。在他的邕州城中,竟然也有人要临阵脱逃。

赏过出战的勇士,炫耀过城中的守备,下面就该是惩戒的时间了。

让人将排开来的军械和金帛财物都收了起来,苏缄的脸色沉了下来,语调阴森的喝道:“带翟绩!”

苏缄的声音将场中的气氛降到了冰点,一时静了下来。

两名近卫装束的士兵,拖着个披头散发、穿着军服的汉子,从衙门中一路拖出来。到了苏缄面前,将汉子狠狠的掼在了地上。那汉子被五花大绑,掼在地上,像条虫子一样不能动弹手脚,就只能勉强抬起头来看着苏缄。

苏缄踏前一步,指着那汉子:“大校翟绩,身为朝廷命官,食天子俸禄,临敌之时,不思报国,竟欲弃城而逃。军法在上,此罪难饶……”

“苏缄,你别说爷爷,你还不是让你的儿子先逃了!”翟绩愤怒的大吼着,他已经放开了一切,临阵脱逃肯定是死罪,但他死前也要给苏缄一个难看,“爷爷就守着门,眼睛好使得很,看得一清二楚。你身边一直跟着的陈先生呢?难道不是护着你的儿子逃了吗!?”

翟绩咧着嘴大声的吼着,人群中响起一阵嗡嗡的议论声,看着苏缄的眼神都有些不对了。

苏缄冷眼看着翟绩最后的疯狂,他的确已经将身边最得力的幕僚派了出去,如果只是派个急脚递当信使,苏缄也不放心。另外一方面,也是想保着离开邕州、回返桂州的长子。他的那位幕僚也是剑术大家,有他同行,当能让自己的儿子苏子元安然的返回桂州。

可惜只能让长子一人返回。

“出来吧。”

在人们的低声议论中,苏缄回头喊了一声。就在他的身后,高高矮矮有着数十人出现在衙门大门处,男女老少皆有,最小的竟然是个五六岁的小女孩儿,被一个身穿绫罗的贵妇人抱在怀里。

苏缄回身指着他们:“本官长子苏子元,是桂州军事判官,奉王命有守土之责,本官所以让他回去了。但本官的其他子孙,全都在此处!区区交趾,决破不了邕州城,不过若有一个万一,本官一家三十六口,自当与邕州城偕亡!”

苏家一门三十六人,被几千道视线盯着,平平静静的纹丝不动。如果是在交趾军登陆钦州的消息传来之前,苏子元可以带着妻儿一起离开。但在交趾军至的消息传到后,再将妻儿一起带走,邕州城就没办法守了。

苏缄转又等着临阵脱逃的军校,“翟绩,你呢?!你的职位在哪里!?”

翟绩脸色灰败,无言以对。苏缄嫌恶的看了他一眼,一挥手,“拖到十字街口,斩首示众!”

翟绩被拖走了,苏缄提声问着所有人,“本官阖家欲与邕州同生死,不知尔等是否愿与本官共存亡?!”

须发花白的老人,已是老态龙钟,但他拿着忠义之心质问着在场的每一个人的时候,他的身形在人们的眼中变得高大无比。

“愿与皇城共存亡!”

这不是苏缄安插在人群中的亲信在喊话,那几个还没来得及说话,方才出战的几名士兵就抢先一步喊了出来。

“愿与皇城共存亡!”

更多的人吼了起来。

“愿与皇城共存亡!!”

这是在场所有人的呼声。

“愿与皇城他共存亡!!!”

一股股声浪引动了整个邕州城,这时已经是全城数万军民在同声呼喝。

城中的高呼传到了城外,李常杰和宗亶脸色微变,一下难看了许多。

鼓动过全城的士气,苏缄与通判唐子正开始巡视城中。

唐子正随着苏缄的步子,低声说道:“邕州城禁军、厢军、枪杖手在册者,总计六千两百一十四人,实际则有两千八百余,精壮只占其半。如果要凭这些兵来守城,还是太难了一点。”

这个刚刚清点出来的数字,与苏缄所掌握的数据并没有太大的差别。南方军队空饷吃到一半,领军的将校已经算是很清廉了。苏缄笑了一笑,反问道:“怕了?”

唐子正冷哼一声:“不过一死而已。”

苏缄回望与自己的次子恰巧同名的副手,笑道:“邕州是侬智高之乱后重新增修过的,城垣高峻厚重,哪有这么容易被攻破?”

侬智高起兵叛乱,攻下邕州城立国,旋而亡于狄青之手。短短时间,两次被攻克,旧邕州城城垣残破。所以王师光复之后,重新加固增修。高墙深垒,不比桂州、广州稍差。

苏缄望了望城外:“李常杰虽号宿将,也不过是欺负一下占城而已,有多少攻打坚城的经验?城中军心民心如今皆可用,竖起招兵旗,少说也能再招募两千愿意吃兵粮的。且现下邕州城内军民大概不到十万,其中应该会有两万丁壮,到时候都可以上城。”

唐子正放松点的笑了一声,“只要守到桂州的援军来就行了。”

“嗯。”苏缄点了点头,“下面就等援军来了。”

只是他脸上,却是隐隐有着忧虑,援军……当真能来吗?

