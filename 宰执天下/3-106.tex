\section{第29章 百虑救灾伤(11)}

退朝之后,只用了一个时辰,吕惠卿在朝中的发言,以及得到天子允许的结果,就已经传到了粮行会所之中。

听到这个消息,大行首金平的脸色全都变了,其他几个行首也几乎都陷入了恐慌之中。

既然朝廷将售粮的权力转交给自己,又给了每斗十文的差价作为补贴,他们就再没有高价卖粮的权力。如果还想坚持着一斗一百三十五文的价格,那就叫做敬酒不吃吃罚酒,天子和朝堂绝不会容忍。

但关键的问题还不在这里,而是潜藏在背后的王安石的真实用心。

金平手脚冰寒,从没想过王安石下手竟然这般狠辣,过去一百多年,什么时候将刀子挥到宗亲们的头上?就算过去王安石强行推行宗室法,也只是砍俸禄,砍亲缘,没说要砍人头的,所以自家才会有恃无恐。但王安石指使吕惠卿在朝会上出此提议,分明是要他们这群粮商的小命。

脑中晕眩不已,金平眼前一阵发黑。无穷无尽的悔恨涌上心头。本来看着还有十天就到年底,成功就在眼前,只想着再拖上两日,并不会有什么大碍,拖不起的是王安石才对。却完全没有想到这么一拖,竟然就要将自己的小命给拖没了。

金平能推断出来的,大部分行首都能推断出来,一个个便如丧考妣,失魂落魄。但还是有人没有看明白王安石的险恶心意:“将王相公给的米麦卖完便关门就是了,怕个什么?”

“哪有那般简单?!”金平噗的一口血竟然真的给吐了出来了,唇齿间鲜红一片,面色狰狞。颤抖的手指犹然指着那名蠢货,“你说卖完了就卖完了,到时候挤在门前的百姓谁会相信?闹出事来,你说王安石敢不敢将所有的罪名栽到我们身上?!到时候,谁还能保着自己的项上人头?!”

这一下,每一个人都明白了王安石的心狠手辣——变法的拗相公如何会按着旧时的规矩来?

“那……那该怎么办?”

“放开所有的仓库……”金平心头火烧火燎的直喘气,勉力的说着,“有多少就卖多少,身家性命要紧!”

从诏令公布的当天开始,东京城中的每一家粮店前,都排起了长长的队伍。官府运来的粮食被一扫而空,而刚刚买到米的百姓,将之送回家后,转而又排到了队列最后。许多人排了一次又一次,眼下的旱灾人们都看在眼中,就算家里只有两口子,也恨不得囤上七八石够吃一年的粮食。这一份需求,即便是为官府代售的粮食和店中的库存都加起来也供给不了。

很快,大大小小的粮店门前的队伍就停止了移动,前两日还傲气逼人,用眼角瞥人的粮店掌柜和伙计们却不敢挂出了售罄的水牌,纷纷出来,陪着笑脸劝告正在排队的客人:“各位,小店的米面现在都已经卖光了,还请少待片刻,要不过一阵子再来也行。”

可是有人不买账,尤其是在队伍中排到快到自己的时候,竟然被告知已经卖光了的人们更是火冒三丈:“这两个月,你们也赚够钱了。现在王相公为了让你们讲点良心,又贴了多少买路钱,你们还想怎么样?!囤着粮不卖,当真要俺们身上的钱都刮光吗?!”

王安石跟宗室那是死对头,东京城里有谁不知?京城百姓说起政治秘闻来,比起外地的官员都要门清。在无法降下东京粮价的情况下,王安石将粮食交给东京粮行来转售,人们都道这是宰相为了不动用常平仓而向粮商们认输了。粮价由此而降,但降下来的米面依然难以买到。原本对王安石的怨恨,这下全都转移到粮商们的身上。

“只是一时还来不及运,”米店的掌柜尽力分辨着,“还请各位少待一阵,运粮的车子一会儿就到了。”

“拖延时间谁不会做?哪个又会信你们?!等你们一次十几石,一次十几石的将粮运来,俺们要买到过年的米,都要等到明年上元节了!”

没有哪家粮店的存货能完全满足百姓们的需求,而百姓的耐心却在这两个月的物价腾飞中给消磨得一干二净。想要将足够的粮食运到城中,粮商们已经发动手上所有的运力,但对于所有在粮店前排队的百姓们来说,却全然是杯水车薪。

也便如此,同样的争吵就出现在每一间粮店前,甚至有几间粮店还发生了民众冲入店中打砸的情况。

不管是粮店里的存粮是真的卖光,还是假的卖光,只要百姓有所不满,即便仅仅是在粮店之前喧哗,落到有心人手中,也足以钉死粮商们的罪名。而百姓们的不满,却是怎么也无法避免的。

先是灾情引得粮价高涨,等到南方粮至,粮价却还是下不来。先给个期待,然后又是一盆冷水,一次、二次,这怨气就是越积越重。由于王雱、韩冈的策略,民众的怨气已经成功转嫁到粮商们身上,不像针对朝廷那般让人会觉得心里有忌讳。百姓将心中的不满宣泄出来,这件事岂能避免?

“依仗裙带之势,恣意取财,以至于民怨沸腾,如鼎中汤滚,难以遏抑。”在天子面前,王安石厉声说道:“京师不稳,天下难安。金平等一干在官粮商以一己之利,致使京中民乱。当追夺其人出身以来文字,重治其罪,以儆效尤!”

粮商们哪还有什么可以辩解的?

物价高涨致使百姓不安那是实打实的,他们高价卖粮也是实打实的,罪名洗都洗不掉。当他们没有在纲粮抵京后的第一时间将粮价降下来,他们的命运就已经决定了。

此案一出,连续两月物价高涨的罪过,便由粮商们全盘承受。王安石身上背负的民怨则散去了不少。

面对东京粮商这一个堵在路前的绊脚石,王安石只有两条路可以走。一条就是用海一般多的粮食淹过去,另一条路就很简单,直接将绊脚石给挖掉。

王安石变不出粮食。直接开常平仓卖粮那是不可能的——韩冈也知道,后世曾经发生过的那场没有硝烟的战争中,胜利的一方是靠着极端充沛的资源才做到的。

能选择的当然只有第二条路。这个方案,早在开始准备利用雪橇车从南方运粮进京时就已经决定了下来。由王雱起头,韩冈则进行修改和完善——王雱,乃至如今朝中所有的官员,都有一个很大的缺点,或者说历史局限性,就是不敢发动群众,而韩冈则完全没有这方面的困扰。

另一方面,由于年龄以及性格的因素,不论韩冈,还是王雱,对于官场上的规则都没有多少忌讳。都喜欢将敌人一棒子打死,而不是你来我往的纠缠。

原本的情况,直接处置粮商是不可行的。看着百姓身处物价飞涨的困境,宰相却不开常平仓平抑粮价,反而逼着粮商低价贩卖,道理上怎么都说不过去!

自身不正,如何能服众?此事如何又能做到名正言顺?——在过去的百年里,都是先由朝廷大举放粮,然后再严令粮商降价,哪有硬来的先例——粮商们的后台都不会心服口服,必然有的闹腾。而且这等粗暴的做法就算粮商们不能硬顶,也能软着将之拖延。

但当南面的粮食入京后就不一样了。此前所有的人都是用民生、民心为借口来攻击王安石,百姓们的怨恨都由不肯开仓放粮的宰相承担。可纲粮抵京后,粮商还不立刻降价,背离民心的已经变成了他们。所以王安石要做的,就是彻底的将身上的怨恨丢给粮商,将自己给摘出去。

使怨有所归,这一次争得就是大义的名分!

轻易的说服了天子——赵顼其实也对不断挖着大宋根基的亲戚们厌烦透了,有了能搪塞祖母和母亲的借口,当然只会点头——朝廷对于粮商们的处理速度便是极快。

腊月二十三,天子下诏,根究东京粮行囤积居奇、戕害生民的不法之举。

腊月二十四,东京粮行自大行首金平以下总计三十七家粮商就同时抄家,查抄并没入官库的粮食不计其数,有传言说甚至接近百万石。

腊月二十五,开封府、审刑院、御史台在天子严令下,放弃休假,展开三堂会审。

腊月二十六,在京诸仓敞开卖粮,以七十八文一斗的价格一次投放市场超过百万石,并且不再限制购粮数量,东京百姓聚集宣德门前山呼万岁。

同一时刻,韩冈踏进县衙前庭:“开封势力最大的行会完了。”

昨夜东京城那边传过来的消息,粮行行首们被羁押后,他们的县主夫人曾想到宫中哭诉,却被曹太皇和高太后拒之门外,据说连她们也在株连之列,一个都别想逃过。

“不知会怎么判了,可不能轻了!”游醇对商人们全无好感,对于囤积居奇的粮商们的下狱治罪拍手叫好。

“大概明年才会有判决,不过领头的几个当是绞刑无疑,其他则是流放,是否罪及全家那就要看天子的心情了。”

韩冈说着,脚步突的一顿,诸立竟然就跪在屏门前。

