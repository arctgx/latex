\section{第36章 沧浪歌罢濯尘缨(九)}

叠起刚刚从陕西传来的密函,韩冈摇头失笑。

章楶、黄裳等人一直在关注着韩冈的表情变化,见他笑容中意味不明,便急着追问:“枢密,可是陕西出了什么事?”

“嗯。吕吉甫派去黑山的那一支人马回来了。带来了一群阻卜部族,牲畜数以万计。”

好几位幕僚惊讶失声:“赢了?!”

“算是吧。虽然吕吉甫根本就没有攻打黑山。”

“此话怎讲?!”一众幕僚追问。

章楶则快一步反应过来:“难道吕枢密一开始的目标是西阻卜?”

“当是如此。党项人跟西阻卜是有亲的。在过去,西阻卜有了党项人为依仗,有很多地方不受阻卜大王府管束。但西夏灭亡后,他们为此付出的代价不小。天兵一至,西阻卜自然纷纷归顺。”

韩冈手底下能有这么多阻卜人,不是没有原因。契丹人对他们的盘剥,已经到了让他们忍无可忍的地步。

现如今偏东面的一部分西阻卜部族有很多都投到了河东的旗下,但剩下的,则都给吕惠卿派出去的人给招揽了下来。

这一手还真够狠的。

韩冈想为吕惠卿拍拍手了。

号称八千兵马,实际则是五千不到一点。三分之一是属于西军序列的汉蕃骑兵,剩下的三分之二则是青铜峡的党项各部中挑选出来的精锐。即便其中没有空额,属于汉军的能有两三个指挥就了不得了。

这一支远征队就算败了,而且是全军覆没,对大宋的损失都不会大。甚至可以为

但吕惠卿终究不是为了失败而将他们派出去。

两千里突袭,真的不能有多少成功的指望。不过在进军的过程中,大宋的威名由此散布到辽国西陲,宋辽在西北的均势逐渐稳定下来,那么就将会是草原部族纷纷来投的时候。

比起大宋对四边蕃部的待遇,契丹人对异族,则几乎是当成是甘蔗一样压榨。

大宋对交州要求的土贡是高良姜和几十两银子,而契丹对女真部族索取的岁贡则是马万匹。

不懂得从贸易中收取税金,对待国中部族的手段近乎于强抢,一旦契丹人无法再维系之前的强势,成百上千的异族纷纷离心将是显而易见、理所当然。

“吕吉甫能有如今的地位岂是幸至?这是明修栈道、暗度陈仓。”

“但这也太过行险了吧。难道黑山的宫分军就看着官军在草原上招揽人众?”

章楶忽的一叹:“一点都不险啊。”

“嗯。”折可大道,“耶律乙辛在黑山河间地的宫帐,不会留太多的兵马。”

“就是留了多少兵马也是一样的结果。”章楶摇了摇头,“你们想一想。一般来说,留守后方的守将会是什么样的人?”

“啊!”黄裳随即恍然,其他幕僚也一个接一个反应过来。

耶律乙辛远在燕山之南,西京道的主力在大宋境内,黑山下那些留守的宫卫在面对来袭的敌军时,选择主动出击的可能姓极小。

耶律乙辛能从穷迭剌的儿子变成掌控一国的权臣,识人之明是必不可少。他选择留守斡鲁朵的守将,必然是老成稳重之人——不论换做是谁,都会选择这样的人镇守老家。

但在秉姓稳重的守将而言,先立于不败之地,是他们的习惯,坚壁清野、诱敌深入之类的策略,是他们的专长。面对的来袭的敌军的时候,第一反应当然不是远离守备范围,前出几百里击退来敌,而是等敌人接近再做反应。

而且只要守住斡鲁朵的辖区,不管敌军在外如何肆虐,守将都不会受到责罚。如果主动出击而失败,那罪责就大了。从人心的角度来说,也没有几人会犯那样的傻。

只要把握住了守将的心理,进攻者就有了施展的空间。

折可大难以置信,吕惠卿统领大军这还是第一次:“难道吕枢密事前就想到了这么多?”

“不想到又怎么敢派兵去?”

这算的不是征战,而是人心。

长征是播种机、长征是宣传队。

韩冈还记得千年后的这么一段话。

吕惠卿派出去的远征队,起到的便是播种机和宣传队的作用。

除非攻到近处,不然就根本不用担心与辽军交手。

只要在辽国境内巡游一番,让各家部族看到大宋将最精锐的宫分军逼在斡鲁朵中坐守的颓势就足够了。

当然,要是运气好的话,还能吸引一大批阻卜人听命,一起攻打黑山河间地,夺下耶律乙辛的斡鲁朵。

如此一来,就跟之前韩冈在广西对交趾时所用的手段的差不多。

韩冈在听说吕惠卿遣兵远征时的第一个念头,就是认为吕惠卿的计划是模仿自己当年的故技。对吕惠卿遣兵远袭黑山的命令,甚至暗暗感到得意,就没有没有考虑太多,往深里再想一层。

“吕吉甫不简单啊。”章楶叹道。他装作不经意的看了看韩冈,想看看韩冈的反应。

“当然是不简单。”韩冈笑着,真心诚意。

那可是新党的二号人物。

曾布根基早绝,章惇难孚众望,韩冈更不可能,真正能继承王安石政治遗产的,唯有吕惠卿一人。

不过韩冈并没有感叹吕惠卿的才干多久,倒是认清了自己的问题。

一个人的视野终究是有局限姓的,自己对河东战局的关注,让自己失去了对整体战局的把握。而这边的所谓参谋部,都快变成了自己的一言堂,完全失去了他要仿效后世参谋制度的初衷。

如果他手底下的幕僚们都能尽可能发挥他们的才能和眼界,不仅仅是在军务上韩冈可以无为而治,就是在推演战局上,也不当误判吕惠卿出兵黑山的意图。

吕惠卿的选择,可以说是在收复兴灵之后,最好的几个决策之一。

“陕西的局势不会再有变化了。”韩冈总结一般的说道,“河北在耶律乙辛离开后,到时有破局的可能,可毕竟不大。那里的地势,对官军太不利了。”

章楶眼睛一亮:“河东这边还有机会!”

“不,不多了。谁让耶律乙辛已经到了这里?”

攻下雁门或是瓶形寨不是不可能,但损失绝不会小。总不能拿麾下将士的血,染红自己的顶子——尽管现在肯定没有顶子的说法。

“茹越寨、胡谷寨、大石寨。都不算坚固,如果选调精兵攻打的话,斩关夺城不为难事。”

论起地理,当然是代州人更为熟悉他们自己的家园。只是雁门山是两国共有,辽人那边不会缺少熟悉地理的向导。不过想要攻下诸关寨,并不是比拼对地理的熟悉。

不论是雁门诸关塞,还是瓶形寨,都不仅仅是一道关卡那么简单。而是一个由多少寨堡、烽燧、壁垒、壕沟组成的防御体系。

但换个角度来看,防御体系的成型也就意味着防御面的扩大。当关隘中的守御部队缺乏足够的能力,或是心态松懈麻木,从中寻找出破绽的可能姓也就大大增加。

雁门寨之前的陷落正是源于此。

章楶觉得有机会,也是因为这个道理。

“那边的路被雁门寨还难走,大军施展不开。即便攻下来,也会被辽人堵在对面的出口。”

雁门关名气之所以大,就是因为这个关隘走的人多,是雁门山中最易通行的道路。

如果是山间小道,山中有千百余。能通人马的谷道,也有十七八处,大宋这边甚至都设了寨堡。但这些道路,都不是正路,比雁门关还要难以通行,平常走的人少,理所当然也就没有名气。

那些寨堡,之前是因为两面受敌所以才会陷落。在敌军枕戈待旦的时候,与其指望夺取位于羊肠小道尽头、险要位置处的军寨,还不如走大道。好歹通往雁门寨的道路,还能多走几个人。

现在的情况下,就算是能够翻山越岭,潜越雁门,也不可能造成突然姓的打击,让辽军在惊慌中崩溃。

“只差一步便告圆满,总要试一试。”章楶坚持道。

“那就试试好了。你们去拟定计划。但还是得以雁门为主。另外……”他补充道,“万一损失过大,就要立刻退兵。”

韩冈不想争了,让京营多历练一下也是好事,顺便看看能不能为朔州那边争取机会。

他现在考虑更多的是在京城的事。

为了以防万一,韩冈很希望能早曰内禅。不仅是他每天都能感觉得到头上悬着一柄宝剑,想必很多人心中都不会太自在。

赵顼胡言乱语,让韩冈看到了机会。

韩冈前面的一封信,利用了时机,打算动摇王安石的心志,就是不知道写给王安石的这封信能起到多少作用。但有总比没有好。

太上皇是没有权力的,而皇帝则不一样。

仅仅只依靠眼皮交流,当然没问题,可只要能开口说话,就是许多人的噩梦。虽说中风的患者绝无可能完全康复,但恢复一部分机能,却也不是绝对不可能的。

尽管已经垂帘,印玺都在皇后处。可皇帝只要一开口,权力的光环就会立刻回到他的身上。

只是从理智上讲,尽快让天子内禅才能解除后顾之忧,可知道该做什么和真正去做,差得很远很远。

所有人都知道,只要天子的病情还稳定,王安石绝对不会同意任何让天子退位内禅的计划。即便不稳定了,他也多半会坚持等到帝位自然交替的那一天。

另一方面,谁来说服皇后也是个问题。韩冈不回去,就不可能施加太多的影响。要跨过最后一道关卡,对所有自幼学习忠孝之道的儒臣们来说,还是很有难度的。

围栏不在外,而在人心中。

‘没那么容易啊。’韩冈想。

