\section{第35章 把盏相辞东行去(四)}

“痘疮!”王舜臣一声惊叫,赵隆和李信当即倒退了几步,远远的避开。北宋的痘疮,其实就是天花。这个时代,从皇室到民间,婴幼儿死亡率都是高达五成,其主要罪魁便是名为痘疮的天花。赵隆和李信都没得过天花,自是有多远就躲多远。

“痘疮?……是水痘啦!”王厚上前查验了一下,他小时就得过天花,运气好撑了过去,耳鬓、额角等不显眼的地方,还有当时留下的疤痕。眼前的小孩子身上的浆疱,并不是天花的样子。他抬头问着专家的意见,“玉昆,你怎么看?”

“不是痘疮。”韩冈这个身体没得过天花,更不知道水痘和天花的区别,但药铺里的专业人士轰人出来时并没有避讳,想来也不是会要人命的烈性传染病。

严素心低下头看着招儿已经满是水疱的小脸,“是水痘,郎中都开了药方,就是没钱抓药。”

韩冈掏了一下怀中,钱袋里只剩下百十文,他问着王厚,“处道,还有钱没有……”

王厚向外掏着钱,“玉昆你倒是一片仁心。”

韩冈正色道:“当初若救我的孙道长少了一份仁心,小弟早已是一堆白骨了。”

“说的也是,也算是件阴德吧。”王厚把一串铜钱递给韩冈,韩冈装进自己的钱袋,转手一起交给严素心,又问着:“还够不够?”

看着韩冈温文尔雅的微笑,严素心抿着嘴,不想让自己哭出来。她哽咽着低下身去道谢,但抬起头时,韩冈已经带着人走远了。

王厚走在韩冈身边,沉默了一阵突然说道:“玉昆,方才你做得岔了,不该扶她的。你虽是好心,可街上人多眼杂,传出去对玉昆你的名声不好。”

韩冈哈哈笑着,浑不在意:“方才本有,心中却无。如今虽无,心中却有。处道,你着相了!”

王厚愣了,想了一想,便摇头自嘲而笑:“愚兄的确是着相了。……不过玉昆你在普修寺里倒真是住得久了,说话也越来越有禅味。”

韩冈停步抬头,看着普修寺的匾额,“除了香火塑像,这庙里,哪还有半分禅意?”

……………………

寺中的住持和尚道安,这时正陪着几人说话。看着韩冈等人进来,便急忙站起。

他们都是不够资格出席韩冈的饯行宴,而特地在普修寺中等候韩冈。王五、王九,还有周宁,在周宁身边,又站着一个让韩冈看着眼熟的黑瘦青年。

当初的德贤坊军器库中的两名库兵——王五和王九,在陈举一党被清理之后,已经改在成纪县衙中做事——这是韩冈的安排。

陈举在成纪县只手遮天,县中的衙役胥吏都在他的指挥之下,他一倒台,几十个在县衙中奔走的吏员,没有一个不受到牵连。及时找到新后台的,留任原职,而有些牵扯过深又找不到后台,便落职回家。空缺出来的职位,给多方瓜分干净,韩冈也趁机塞了几人进去。王五、王九便是其中的两人,其中年长的王九还是个班头。

韩冈籍此向外界证明:“跟过我的,我都不会忘记。”

德贤坊军器库一案,王九和王五在历次审问中咬定牙关,帮着韩冈把罪名坐实在黄德用身上。不管怎么说,刘三尸身的要害处,都有他们留下的刀伤,秦州和成纪县的仵作可分不清死前伤和死后伤的差别。王五、王九一想到投名状都交了,哪里还能有改口的胆子。

不过这样一来,韩冈便欠下他们的一笔人情。理所当然的,韩冈帮着他们洗清了一切罪名,还在成纪县中安排了两个有油水的位置——虽然是衙前,却是在衙门中长期服役的长名衙前,比起韩冈当时服的衙前役是天壤之别。

“你们是玉昆保下来。在衙门中好生做事,等玉昆回来,如果愿意的话,就让你们跟着他去办事。”王厚教训着两位王衙前,看着他们唯唯诺诺。

另一边,韩冈又与陪他从秦州一直走到甘谷城的民伕中的一员——周宁搭起话来。

看到周宁,韩冈便想起他在甘谷城创立的甘谷疗养院,以及在疗养院中做事的一众成纪县民伕。甘谷城的防御体系早已整修完毕,韩冈当日带去甘谷城的民伕,已经跟被留在甘谷修城的那一批人一起被放了回来。

只是领头的朱中却是被征召入军中,成了一位军医,负责外科——这是韩冈临走时的意见。有了这重身份,想来朱中应该很快就能娶上媳妇了。

至于周宁,则是因为韩冈看在他能写会算的条件上,把他安排到了户曹书办的位置上,这是刘显原本的职位,如今刘显已经成了刀下之鬼,周宁名正言顺的夺下了户曹书办的位置,油水自然丰厚。才几日功夫,周宁身上的穿戴已然不同。

周宁先向韩冈道过喜,祝他一路平安,这才把身边的黑瘦青年拖了出来。向韩冈道:“小人的这位族兄,一样姓周,单名一个‘凤’字。”

韩冈看着眼熟,听得耳熟,再一细问周宁。才知道他的这位姓周名凤的族兄弟,正是当日被韩冈顶了德贤坊军器库差事的那一位,而后韩冈又在被派了去甘谷押运军资的那一天,在县衙里见了他,听陈举说他的老子上了吊,让周凤成了家中唯一的男丁——单丁户,自此便免了衙前苦役。

“只是小人的这位族兄,因为从军器库中调离得太巧,被怀疑是陈举一党。前些日又牵连到官司中,剩下的一点家财也都全没了。现在想寻口饭吃,还请官人成全。”周宁在韩冈面前说着好话。

而木讷的周凤则上前一步,跪倒在韩冈面前:“小人周凤多谢韩官人救命之恩!”

说罢便砰砰砰的连磕了三个响头。这三个响头他下了狠劲,头抬起来是,脑门上已是一片鲜红。

韩冈神色微动,的确,周凤可算是被他救了性命。若不是韩冈横空出世,让刘显将管库的职司从周凤的手上夺了去,他少不得要在火海中化为焦尸,还得落个罪名,老子和家产一样保不住。陈举的盘算,如今也不是秘密,周凤又是当事人,知道这件事的内情并不出奇。

韩冈抬手示意周凤站起,“你与我都受过衙前之苦,也算是同病相怜,举手之劳,帮一下也无妨。王九……”

王九会意的上前一步,低头抱拳:“请官人吩咐。”

“你看看县衙里什么地方还有阙,给周凤一个位置。”

“官人放心,小人明白!”王九低头应是。

周凤则连连磕头:“多谢韩官人!多谢韩官人!”

“起来吧!”韩冈端坐着,双眼犀利如电,他经历得多了,便越来越有人上人的气势,“别的我就不提醒了。只望你能以己心体他心,当初受过的苦,不要再害到别人身上……否则我决不饶你!”

“官人放心,小人决计不敢。”周凤点头哈腰的应承下来。

……………………

次日清晨。

天空东侧有了点微光,而西半边的天空却还是一片墨蓝。凌晨的寒意如刀似剑,宽阔的道路上,只有寥寥数人。

韩冈从下龙湾村出来,父母和韩云娘的眼泪和嘱咐还沉甸甸的压在心头。王厚、王舜臣等十几人,就已经守在了南门处等候。

韩冈远远的向王厚他们拱手道:“韩冈累各位久候了。”

王厚也远远的在门洞下行礼,带着众人迎了过来。但走到了近前,所有人视线却齐刷刷的望向韩冈的身后。他们指着紧跟着韩冈的一名十二三岁的小童,惊问道:“这是谁?”

韩冈道:“今次上京,身边没个得力的伴当实在不方便,所以带了这个小子。你们应该都见过的,是李家的小六。当初来报信的那一位。”

没人能想得到,韩冈带在身边的伴当,竟然是李癞子的小儿子。王厚对他有点印象,正是前日在下龙湾村中守株待兔时,赶来通风报信的那个小子。韩冈能将陈家余孽一网打尽,李癞子的倒戈一击不无功劳。为了酬谢这份功绩,韩冈便收了李家的小六在身边坐了个伴当,连嫁给黄家做媳妇的李八娘,也平平安安的回到了娘家。

王厚上下打量了李小六一阵,皱眉摇头,“玉昆。如今道路不平,贼人众多,还是再多带个老成干练的的伴当上路才是。”

“三哥,还是找个可靠点的帮手。要是实在不行,俺跟你去。”王舜臣也劝着韩冈,“如今路上可不太平。”

“处道你们都放心,”韩冈豪爽的拍了拍挂在马背上的一弓一刀,“有弓刀在此,韩某还怕那些剪径小贼不成?”

韩冈说得豪气干云,而实际上他也不认为路上会碰上什么贼子。陈家余孽已经荡清,还有什么好担心的?

书生仗剑游学天下,他三年前就已经孤身做过,如今就算身边带个累赘也没什么大不了的。更何况他走得都是直通京城的官道,按后世的分类算是国道,路上人来人往,车水马龙,没哪家贼人会这般不开眼。

住的是驿站,走得是通衢,要是这样还能碰上贼人,韩冈可以去买彩票了——虽然这时代没有彩票。

拗不过韩冈,王厚他们也只能作罢。跟着韩冈一起,几人一起往东门走去。南门是接人,东门才是送人。王厚边走边说:“大人和吴节判今天都要来,酒菜也提前派人在十里铺那里备下,就等着玉昆你上场了。”

“又要劳动机宜和节判两位了。不知到时还有什么吩咐。”

“吴节判那里愚兄不知道,大人却是要有一封私信想托玉昆你带给王相公。”

韩冈听着一震,说是带信,实际上这是面会王安石的机会,一个从九品的选人想见到宰执官可不是那么容易的一件事,是王韶特意为他安排的?他看了看王厚,脸上果然有些笑意。“当是要多谢机宜苦心!”

“说起来,吴节判怕是也要有些信件托玉昆你带去京城。”

“这是当然的。”韩冈点点头,北宋又没有邮局,驿传系统又不送私人信件,要想送信给远方的亲友,只有转托给相熟的友人。

ps:辛苦了许多,韩三终于有了点地位,小弟也越发的多了。兲之气逐步晋级中。

第二更,求红票,收藏

