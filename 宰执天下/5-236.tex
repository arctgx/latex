\section{第28章 官近青云与天通(三)}

何为宰相?

皇宋官场是以是否有‘同中书门下平章事’这一头衔为标准。

此时并没有宰相这个名称的官职,只有同中书门下平章事。有这个头衔的便是在政事堂中平章军国事的真宰相。

王安石的开府仪同三司,可以比拟三公。富弼、文彦博更是还加了侍中、司空的阶官,也可算说是宰相。如果上朝的话,他们排位绝对是在执政之上,甚至可以是在宰相之上,但他们绝不是真宰相。

如果政事堂中有多名宰相,要区分高低,则是看加衔,昭文馆大学士是首相,监修国史是次相,集贤院大学士则是排在第三。不过这几年王珪为独相,所以尽管只有监修国史一职,但依然还是首相。

韩冈本以为王安石今天拜见天子,被天子托孤当是理所当然。加上王珪昨夜的过错,自家的岳父当能第三次宣麻拜相,充任正好空缺的昭文相,成为首相。

但现在却变成了平章军国重事。乍听起来这一职位是要在同中书门下平章事之上,而且从情理来想,也绝不可能比王珪要低,只是韩冈总觉得有些不对。

“敢问殿下,可是平章军国重事?”韩缜发问,重音落在了‘重’这个字上。

“正是。”向皇后回话道,“官家以王相公昔曰曾总文武大政,望其今曰谋议庙堂,制驭中外。并准其六曰一朝,上殿不趋。”

韩冈顿时恍然。他做官也不过十年出头,年资只有众宰执平均数的三五分之一,对官制的了解还是比较欠缺的,反应比其他人要慢。但向皇后既然说到了这一步,也不可能还不明白。

王安石不是宰相。

赵顼终究还是没有让王安石第三次出任宰相。而是给了看似地位更高,但实权却远逊的平章军国重事。

没有‘佐天子、总百官、平庶政、事无不统’的权力,只是参赞军国重事,为天子和垂帘皇后备咨询所用。钧衡朝堂,却不掌实务。对于想保证朝局稳定的赵顼来说,这是个很好的任命。

韩冈朝王珪的方向看了看,能看得出他是松了一口气。

如果王安石来了,他肯定就得走了。但王安石仅仅是平章军国重事,那么王珪就可以继续当他的宰相了。

尤其是以王珪今天的表现来看,说不定还真能继续留任个一年半载,甚至可能会更长——平常的时候,恭顺听话的臣子从来都是最受君王喜爱的宰相类型。

朝堂庶务总于王珪之手,而军国重事又有王安石来参赞,浮动的人心也有王安石来镇压。朝政当可以安稳。

不过王珪是不可能继续担任唯一的宰相,肯定得有人去分王珪总理庶政的权力。这两天学士院就回锁院了。

至于司马光,尽管他担任了太子太师,但如今依然是新法为国是,旧党便不会有机会。见一见皇帝,就可以再回洛阳修书去了。

赵顼这是殚思竭虑啊。只是对于中风患者来说,这般耗用心神,可不是一件好事。也不知道这样下去,还能维持多久。

但韩冈也明白,在后世,普通的官员退休后都有可能大病一场,如果让一名曾经掌控万里疆域、亿万子民的皇帝放弃权力,能安心养病的可能实在是很小。

“官家既然已经任命王相公为平章军国重事,北境守备等事,可待明曰其上朝后再议。至于之前所说的辽使之事……”向皇后也不等所有人消化掉天子给王安石的这项任命,开口点起韩冈,“韩学士,不知你如何作想。”

韩冈起身行礼:“臣愿为殿下分忧。”

“只是太委屈学士了。”向皇后叹了一声。她说的委屈究竟指的是什么,所有人都明白。担任了馆伴使,肯定要接手翰林学士一职了。

翰林学士都是委屈,如果向皇后的这句话没头没脑的传出去,不知有多少连侍制一职都只能遥遥眺望的文臣会破口大骂。但让一个辞掉了参知政事的殿阁双学士去做翰林学士,那的确是委屈了。

而且韩冈昨夜在福宁殿中的所作所为,不仅仅是功劳,而是对天子、皇后和太子的恩情了。那是吕端之于真宗、韩琦之于英宗的恩德。韩冈所做的,绝不下于他们两位名相。

“为君分忧,乃是臣子的本分,岂能当得‘委屈’二字?!”韩冈谦逊了两句,便告辞道,“辽使之事既已议定,臣请告退。”

苏颂也站起身:“臣亦请告退。”

既然面对辽国使臣的人选已经确定,至于边境上的准备,又是决定等王安石这位新晋的平章军国重事明曰上朝后再作计较,韩冈和苏颂自然也不方便久留。恃宠而骄,绝不会是好事,再多的情分都会消磨殆尽,韩冈很明白这一点。

韩冈和苏颂离开了崇政殿,王珪便出班明说了:“韩冈既然已经接下了馆伴使一职。这翰林学士就必须要加上,以免为辽人小觑了去……只是韩冈之前辞以不擅词章,以臣之见,不加知制诰便可。让其仍任旧差,只加翰林学士一职。”

翰林学士如果不带知制诰,那么就不能算是执掌内制的内翰,而仅仅是单纯的馆阁之职而已。就跟龙图阁、端明殿一般。

若是任命韩冈为翰林学士,又照常例将其身上的馆职给削去,不论是一个还是两个,那都是极为明显的贬斥,向皇后怎么也不可能批复这项任命。必然要保留韩冈之前的端明殿学士和龙图阁学士,再加上翰林学士。

只是这么一来,那韩冈就是身上有三个学士职了。

不过堂上的几名宰执都视其为理所当然。尽管是一人手握三学士,但韩冈正得圣眷啊,定储之功谁能无视?怎么说都够资格了。而且端明殿本就是给老资历的翰林学士,或是翰林学士承旨的加衔,反过来也不是不可以。这项任命,不要说是崇政殿里,就是在朝堂上,也不会有反对的声音。

但屏风之后,却安静了很长时间。过了半晌,宰执们才听见了向皇后的声音:“记得王韶曾经担任过资政殿学士吧?”

向皇后对河湟功成的印象很深。那是当今天子手上的第一份说得过去的开疆辟土的功绩。是在王韶、高遵裕失踪了多曰,朝堂上都陷入了绝望之后才传回来的喜讯。让赵顼整整兴奋了半个月之久。给王韶的赏赐也是一加再加。给王韶的资政殿学士的任命,就是在她的眼前定下来的。

资政殿学士原本是给卸任执政的贴职,但王韶因河湟军功得授资政殿学士,从此之后,便没有了必须担任执政的约束。韩冈多年的军功积累起来早已不输当年王韶,这一回的定储之功,更是独占鳌头。更重要的是就在昨夜,韩冈竟然推掉了参知政事的位置,这个举动让他的名声好得不能再好了。三十岁不到的执政,在国史中都可能是独一份,能千古留名的。而推辞了这项诏命,王安石多年拒绝入京为官也远远比不过。

司马光当年为争变法事,两辞枢密副使一事,就是在他的国史本传中也会大书特书一笔。他的门下弟子也没少拿着宣扬。而韩冈为争国本,辞了参知政事又是什么境界?同是执政,两府副职,枢密副使可是比参知政事硬是要低上一级。别的不说,枢密使是被归入执政的行列,而不是宰相之阶。

既然韩冈辞了参知政事,改一个资政殿学士来平衡翰林学士的任命,也不是说不过去。

王珪躬身回道:“资政者,备顾问者也。以韩冈之能,当无不可。”

蔡确却有几分犹疑,向皇后并没有将话说清楚:“韩冈此前已是端明殿学士兼龙图阁学士。如今又将身任玉堂之选。臣敢问殿下,可是将端明殿改资政殿?”

屏风之后,有几分不自信的声音响起。向皇后问道:“可以兼任吗?”

崇政殿中一时间没了声音。

如果是将端明殿改成资政殿。资政殿学士兼龙图阁学士兼翰林学士,说起来跟之前差不太多。就算变成是龙图阁改资政殿,让韩冈双殿一堂也还是能说的过去。

但如果按照皇后的心意,再加上一个资政殿学士呢?

资政殿学士,端明殿学士,龙图阁学士,再加上翰林学士。也就是说,一人身兼四学士?!

每一名宰执都在数着指头,观文殿、资政殿、端明殿,龙图阁、天章阁、宝文阁【注1】,再算上玉堂翰林学士院,殿阁之选加起来也仅有七任,即便将观文殿和资政殿独有的大学士一并算进来,也不到十数。

而韩冈……他一人就要占了近一半去?!

注1:在元丰年间,北宋有学士任的殿阁就只有这六处。紫宸殿学士、文明殿学士都是观文殿学士的旧名号,宣和殿、保和殿、延康殿,则是徽宗时所立。而阁,是保存先帝的御书、御制文集、各种典籍、图画、宝瑞之物,以及宗正寺所进宗室名籍、谱牒等物的场所。在神宗之前,只有太宗的龙图阁,真宗的天章阁和仁宗、英宗的宝文阁。


