\section{第31章 九重自是进退地(五)}

王家上下急得跳脚,王韶却硬是拉着韩冈下棋。

不过这一局棋,韩冈靠着王韶的两个缓手,吃掉了他的一条大龙,逼得王韶投子告负。

王韶要拉着韩冈再下一盘,韩冈则是笑着谢绝,说是难得赢上一回,还是见好就收。

王韶也不强留,王厚代替王韶将韩冈送出门。跨出门后,韩冈的脸色就严肃起来,回头对王厚道:“方才枢密之言,只为安定人心,还是要早点报与开封府……处道你应该已经派人去了吧?”

最后一盘棋,韩冈在布局时就输了一筹,以他和王韶的棋艺对比,是应该输的。但重新开始下之后,王韶接连几次昏招一出,韩冈登时就翻了盘。王韶正常情况下,不会这样的。

王厚点点头,知子莫如父,知父也莫如子,王韶方才的动摇他也不是没看出来,“愚兄明白,方才其实已经派人去了开封府,家里面的人也都派出去找了。”他的表情沉重,“就是上元节时,这样的事少不了,天知道能不能给查得出来。前些年,英国公家的县主可是出了事,最后只能嫁个开生药铺的,听说还是天阉。下手的那群贼人,到现在都没能捉到。”

“苏子容新尹开封。十三被拐,开封府难辞其咎,好歹要着落到他身上。”枢密副使的儿子当街被拐走,苏颂即便上任才两天,也照样得担上一份责任,韩冈虽与苏颂有一份交情在,但也不会昧着良心帮他开脱,“今日上元,苏子容当还在衙署中,我现在就去见他,请他派人用心追查,好歹让十三平平安安的回来。”

韩冈拍拍忧色难掩的王厚的肩膀,“十三有福相,又是聪明伶俐,纵有厄难,也能化险为夷,枢密方才说的也不是没有道理。处道你尽管就安心好了。”

“多劳玉昆了。”王厚叹了口气,看起来放松了一点,“我跟你一起去。”

韩冈笑道:“你我兄弟,十三也自是我兄弟,自然是要放在心上的。”

其实韩冈也后怕,这一次是王韶家的十三出事,下一次说不定就是他家的几个儿女被人拐走了。不过现在想想,王家的十三郎会出事,也是家里给他穿戴得太好。前面看到的时候,珠光宝气的,把韩冈的几个儿女都比下去了。

倒不是韩家没钱。而是韩冈一家,一向都不喜欢将自己和子女打扮的太过奢华。这也是南方人和北方人的区别,北方的财主通常喜欢将钱挖个坑藏起来,穿戴则是朴素得很。而一般的南方人,就算是吃了上顿快没下顿了,也照样愿意往家里的器皿、装饰这些无用的东西上砸钱。

不过王家的十三郎在上元夜被拐走,韩冈却总觉得莫名的有种让人放心的感觉,却是说不出来为什么,难道是直觉不成?上的阵多了,当真给历练出来了,韩冈有些疑惑。

韩冈翻身上马,等了片刻,王厚进去一趟后就带着两个伴当和三匹马出来,“走,去开封府!”

……………………

赵顼这个上元节过得并不痛快,站在宣德门城楼上俯视属于他的芸芸众生,也没给当今的大宋天子带来多少快感。

主要就是在于被他招到城楼上一同饮宴的两个弟弟,无论是被封做雍王的二弟赵颢,还是做着嘉王的四弟赵頵,都是子女俱全。

赵顼到现在为止,就两个分别刚过周岁和不到周岁的六皇子、七皇子,加上略大一点的三公主。他马上都要三十了,生下的儿女都超过十人了,就保住了三个。

小孩子在养到七岁之前是完全不保险的,而要真正能让人放心下来,好歹也得到十五六岁之后。

只有两个周岁上下的就跟没有一样,根本不保险。赵顼此前的五个儿子,可都是还在襁褓中就夭折了,最大的也才在这世上活了三年,他如何不清楚这一点。

赵颢有一子一女,长子七岁;赵頵远比赵顼要小,才二十二岁,却是已经有了三个儿子,长子都五岁了。两人的子嗣,至今为止并无一人夭折。而且两个弟弟的儿女一个个精神得很,不似赵顼如今膝下最长的第六子,从出身就是体弱,比这个年纪该有的体重要轻许多。

从两个弟弟身上看过来,接连夭折了五个儿子、三个女儿的赵顼当真是如同被巫蛊诅咒了一般。

看到两个弟弟儿女平平安安的生长,赵顼便是暗自神伤,甚至还有一丝嫉妒。朝臣中,也从不见人像他一般子嗣艰难。

韩亿八个儿子,从韩维、韩绛、韩缜开始,哪一个不是高官。韩琦、富弼、文彦博,没听说哪个绝嗣。年纪相当的臣子中,韩冈更是五六个儿女,全都安安稳稳,皇后前两天还跟他提起过,想要问问到底是怎么照顾的。

正在赵顼暗自神伤的时候,宋用臣上殿来了,脸上的喜色让赵顼看不顺眼。

却听宋用臣叩拜之后道,“奴婢方才赏灯回来,在东华门外拾得一个失落的孩子,领进宫来,此乃官家得子之兆,奴婢等不胜喜欢。不知是谁家之子,未请圣旨,不敢擅便。特此启奏。”

赵顼听了之后,脸色就好看了一点,“宣来让朕见一见。”

宋用臣捡到的小孩子很快就被带上来了,长得粉雕玉琢,眉清目秀的甚是讨喜,身上的穿戴也是上等的丝缎上缀着珠宝,一看便知不是普通人家的出身。

虽然小孩儿才五六岁的样子,却是如同大人一般在天子面前叩拜行礼,口呼万岁,礼数竟然一点也不见错。

在天子面前,即便是也是得提心吊胆,俗谚说的伴君如伴虎,赵顼本人都不否认。不是战战惶惶、汗出如浆,就是战战兢兢、汗不敢出,能比得上这个小孩子的着实不多。

赵顼看着心中好奇,“你是谁家的孩儿?可还记得姓名?”

“臣姓王,乃枢密副使臣韶幼子,排行十三。”

竟是王韶家的儿子。赵顼一听,这还了得?!枢密副使家的儿子竟然也敢拐走,东京城里的治安都成了什么样了,“可曾记得是如何被贼人拐的?”

只听王寀将整件事,从头到尾说了一通。他本是被家人抗在肩上出来观灯,却给贼人趁着家人贪看花灯的时候,将他转到了自己的肩上。当王寀发现之后,知道事情不好,却没有哭闹,只当做不知道,一直到了东华门,恰与宋用臣一行相遇,这才高呼有贼。贼人猝不及防,连忙丢下王寀跑了。

王寀说话口齿伶俐,这个年纪难得的有条有理,而且为人聪慧无比,知道如何自救,换作是普通的小儿,恐怕就是哭闹着被人拐走了,赵顼越看越是喜欢,

而且还是王韶的第十三个儿子,王韶的子嗣一向多,赵顼也是知道的——这些年来,又是积功,又是郊天,另外还靠着升任执政,总共得了十来个荫补,到了上个月的祭天大典,一看王韶递上来的荫补名单,还是他的儿子。如此巧合,在赵顼想来,当然是个祥瑞,乃是得子之兆。

“今夜就在宫中歇上一夜,等明天就送你回家。”赵顼打算留王寀在宫中一夜,也好讨个吉利,吩咐着殿中的小黄门:“好生带到后面去,给皇后说一声。”

转过来又对宋用臣道:“宋用臣,你且去开封府,将今天的事与苏颂说了。上元之夜,贼子猖獗,这是开封府治事不力,命其搜检城中。”他顿了一顿,“朕知此事难为,不过……”

“陛下。”王寀被个小黄门抱着要往后宫去,就在怀里转身过来道,“要想捉到贼人其实不难。”

“为何如此说?”赵顼笑着问道。

“臣出门时,娘亲在帽上别了绣针彩线,以压不祥。臣被贼人所掳时,密在他的衣领上缝了一道。只要去查一查衣领,便知贼人。”

赵顼大感惊奇,啧啧称叹不已,不但知道能如何自救,甚至还不忘留一条捉贼的线索,“常听人说夙慧,今日方才亲眼得见。”提声道:“宋用臣,可曾听明白了?!”

宋用臣恭声答道:“奴婢明白。”

赶在上元节前就任开封知府的苏颂精神抖擞,尽管宋用臣说得不明不白,但韩冈和王厚方才来过一趟,两边一对照,当然也就知道了事情的真相了。

他也不耽搁时间,当即升堂,将衙门中专管捕盗的四名都巡检给提了过来,基本上辖区内的贼人,他们这些地头蛇都可以说是了如指掌。

苏颂将整件事一说,又道:“这番天子亲下旨意,本府便以三日为限,若逾期捉不到人,莫怪本府不讲人情。”

“大府放心,即有证据,哪有捉不到的。小人便以三日为限,若不能按时捉贼归案,甘领责罚。”

也不用苏颂下狠手去催逼,衙中的一众衙役、快手和弓手都知道这件案子是天子督办,哪里还敢推诿拖延,纷纷出去搜寻贼人。

一日之间,整个东京城都翻了过来。这一番全城搜检,惊得城中的地痞泼皮鸡飞狗跳,连带他们也为了求个安生出来帮忙搜检可疑之人。

四名知道贼人特征的都巡检将分头查验捉来的嫌犯,也就在当日,便将一名衣领上绣了彩线的贼人,连同他所在的团伙一并捉入开封府中。

注:这一段故事真伪难以考据。出自于岳飞之孙岳珂的《桯史》,自称是从王寀的孙子那里听来的。到了明代,以此改编的《襄敏公元宵失子,十三郎五岁朝天》又出现在二刻拍案惊奇之中。

