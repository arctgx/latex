\section{第34章 为慕升平拟休兵(12)}

代州就是一条东南、西北向的狭长盆地,不论是南面的五台山,还是北面的雁门山都是高耸入云。当宋辽双方几近十万人马填塞进这个小小盆地之中,每一处山口、道路以及可供通行的平地、树林都有了监视之后,任何包抄后路,截断粮道的伎俩都是一个笑话。

随着双方开始在前线堆积兵力,使得两家的兵马只剩下硬碰硬的较量一条路可以走。而这样较量的结果,也使得韩冈面前计算伤亡图表上的曲线,总是处在居高不下的位置上。

将那份让他越来越无奈的图表丢了下来,韩冈轻声自语:“是不是少一点比较好?”

黄裳没听清楚,立刻问道:“什么?”

“没什么?”韩冈摇了摇头。阻卜人马上就要到了,这时候收手只会让之前的付出变成了无用功,也会让骑兵们的士气大落,所以韩冈很快就找到了一个借口:“我在想阻卜人到了之后,该不该给他们配发神臂弓。”

“连甲胄都不该给!一片都不行。”黄裳厉声叫道:“将军国之器都送给那些蛮夷,万一他们恃此生乱,造成的损害不会比辽人少到哪里。”

“嗯,的确如此。神臂弓和甲胄都不能给辽人。”韩冈其实也不相信那些蛮族会循规蹈矩。即便他们的当真会循规蹈矩,神臂弓和板甲这样的兵器也不可能白白的分发给他们。就连折家都用了八九年才给族中子弟凑齐了神臂弓,而盔甲更是没有配齐。凭什么这些阻卜人一到就能有好处?

神臂弓的损失太大了,这本也不算什么大事。但这个损失和战果之间的差距越来越悬殊,就很难让人认同对神臂弓消耗性的使用手段。

“不过神臂弓的作用差不多已经到头了,辽人越来越不好对付。不知道京营的那几位还能想出什么招数。”

神臂弓装备骑兵就是所谓马上射弩,其实就是京营禁军在天子面前表演的技巧。也是得到了京营禁军的将领的提议,之前才会有装备上弩弓的探马。

每逢三月金明池开,天子驾临这座皇家园林,上四军都会出来表演一些高难度的技巧,或者叫杂耍,可以与民间的百戏【马戏】同场竞技。这些技巧并不是那么实用,但很好看,总能惹起全场士女百姓的欢呼。可韩冈每次看到这样的表演,都会不由自主想起千年之后,那个南方大国的阅兵式时如孔雀开屏般的车上神技。

从成本和实用性上说,骑兵携带神臂弓是很快就会被淘汰的战术,韩冈很清楚这一点。但他没想到会这么快。可惜马刀啃不动胸甲骑兵,而马枪之类的火器,现在连设计图都没有。

不过现在的当务之急——不论是辽军,还是宋军——都是粮草的问题,而不是什么火器或是神臂弓什么的。

韩冈手上的粮食很紧张,就是在忻口寨中也很少有放开肚皮吃饭的时候。而辽军光是在代州吃了两个月存粮,差不多已经将搜集来的粮食都消耗一空。连自家的装备都快吃不消的时候,韩冈不信辽军的粮食能供给得上。

“粮草对我们也好,对辽人也好,都是一个大问题。我们就不说了,之前的储备并不是以长期的大规模作战为预期,能撑上半年已经很了不得了。而辽贼……”韩冈回头笑了一下。

黄裳会意点头,“他们抢不到就得饿肚子。南京道的积储基本上已经空了。”

如果是要跟辽贼长期作战,浅攻进筑的战术是最好、同时也是最适合现有条件的战术。利用代州盆地中星罗棋布拥有坚实围墙的村镇,来作为出击部队的出发点和落脚点。不断将被辽军毁弃的乡村和镇子,全都修复起来。然后一点点的向前。

这样的方略,其实是大宋过去在攻击西夏时,行之有效的战术总结。凭借庞大人力物力和财力资源,一步步的勒紧契丹人脖子上的绳索。这是极难破解的战术,除非内部自己放弃。

韩冈对皇后的使节说只要能保证辎重补给不缺,在入秋之前,他能将代州的辽贼清扫一空,收复失地,并保证武州不失。但实际上,他更清楚,东京城那边能稳定维持的供给最多能到六月。

三年储方有一年积,这一仗打下来,京畿一带的仓储几乎是消耗一空,剩下的一点还得保证东京城的供给。

河东如果没有被辽军肆虐,不会落到这样的窘境。甚至只要保住太原,韩冈前两年在太原任上开沟渠兴水利,推广更好耕作技术和机械,存粮是不会少的。可惜现在连同太原一起,朝廷要做到难民直到明年夏收的口粮不缺。

韩冈这边很难,而辽人那边也同样的难。粮食只能种出来,却变不出来。

韩冈交叉起双手的手指,眯着眼晴看着最新的军报。正是因为大家都难,所以才有了办法。

……………………

“尚父那里还是没消息吗?”

“没有。希望能尽快来。”

张孝杰和萧十三此时正等着耶律乙辛的消息,但已经比约定好的迟了半天,也不见有动静。而且却连辨别耶律乙辛到底有没投项都还没有确定。

“有谁敢闹事,直接杀了就是了。”萧十三冷哼着,“尚父这两年脾气好太多了。若是换在过去,早就下手了,哪里会忍到现在?”

张孝杰动了动嘴唇,想说些什么,但转念一想,又阖上了嘴。没什么好说的,萧十三的话是正理。

宣宗皇帝祖孙三代都杀了。两位天子、一位太子,一位皇后和一位太子妃,天家的血沾满了耶律乙辛的手,还想吃斋念佛不成?!

但耶律乙辛这两年,尤其是这一回与宋人的战争,其实是违逆了他的本心。但为了自身的威望,也是为了日后坐上那个位置少上一份阻力,十分被动的投入了战争之中。

只是现在看来,耶律乙辛做出的选择是极端的错误。如果他能够更加积极一点,现在的局面应当早已经逆转。

可惜现在说什么都已经太晚了。

在河东这里,宋军正以忻口寨为核心,辽军以代州为出发地,不断将手中的兵力向前线调遣。在这其中,各家的底牌也不断被翻上桌面。

“宋人在挖掘沟渠?”萧十三听到这个消息时,他正在苦思如何应对宋军越来越强的自信,和越来越大的动作。

“是的。有好几处都在开挖,最远的就是大小王庄。。”

萧十三面前的是从代州州衙后的白虎节堂中找到的地形沙盘,同时还有详尽的地图。不过代州的舆图只是少数,更多的其实是代州正当面的朔州、应州,以及飞狐陉另一侧的蔚州。宋人的狼子野心从中可以看得一清二楚。

待萧十三把沙盘和地图详详细细看了一遍之后,把西京道辖下的知州、知县都招来大骂了一通。储存在代州白虎节堂中的舆图、沙盘,竟然比掌握在他手中的地理资料,要详尽和准确十倍。

这不是三五个月就能完成的工作,宋人这些年来到底派了多少细作来测绘西京道的山川地理,只看这些详尽到一条山溪、一座山村的地图,就知道必然是个极其庞大的数字,甚至让萧十三有被扒光了衣服的感觉。

不过现在,对照沙盘和地图来查看宋军的行动,却又能让人觉得一目了然。任何计策和方略都能一眼看透。

“宋人这是在准备防守啊。”张孝杰一看到战报,再对应到沙盘上的位置,韩冈的企图也就猜到了个八九不离十,

“宋军最前沿的据点离大小王庄只有八里,那里也是宋人开掘沟渠的重点地区。一旦给他修好这一条的防线,大小王庄的骑兵就派不上用场了,甚至有覆灭之忧。身后有高墙深垒,宋军若踞此出击,大小王庄的兵马就算赢了也必然会损失大半的兵力。”

“能守方后能攻。宋人肯定是想着先保证能够稳守,接下来才会考虑进攻事宜。”

宋军一步步逼近大小王庄,很快就会将两座大村给攻下来。等到驻守两村的骑兵败退后,宋军就可以继续向代州前进。一步接着一步,速度虽慢,但肯定是沉稳一如其主人的性格。

这一切的过程,最多也只消两三个月。到时候,宋军就能把代州围得水泄不通。再接下来,代州可就保不住了。

“韩冈有三个月时间吗?”萧十三嗤笑了一声,“……可惜他没有啊!”

张孝杰神色变得凝重起来,萧十三不是无的放矢之人:“这是为何?”

萧十三没有直接回答,而是把一份被镇纸压住、倒扣在桌上的文件递给了张孝杰。

张孝杰狐疑的接过来,低头仔细看。这并不是最新的战报,或是南京道那边的军令,却是一份细作传回来的情报。

前后看了三遍,张孝杰如释重负的重重靠上了椅背,在交椅的吱呀声响中,他长长的舒了一口气:“嗯,看来是没有。”
