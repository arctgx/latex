\section{第30章 随阳雁飞各西东(18)}

大公鼎低头看着刚刚从城下拖回来的伤兵。

脸上血肉模糊,让人看了就心中发怵。石子、铁屑一粒粒的嵌在血肉中,如同胡麻饼一般,还能看到烧焦的痕迹,是烤过头的胡麻饼——这一点,在他的衣甲上更为明显——但叫痛声却是中气十足,显然只是皮肉伤。

直起腰,大公鼎给随军医院的医工们让开了位置,让他们将这位新到的伤兵送进病房中。

“又是宋人那种能喷火的竹枪?”

“看他脸上的伤不就知道了。”

大公鼎的两个儿子大昌龄和大昌嗣在他的背后小声议论着。

“伤而不死,论威力远不如神臂弓,怎么南人还用?”大昌龄低声说着。

“就是伤而不死才麻烦。”大昌嗣比他的兄长多了份见识,“劈面挨了神臂弓一箭,一死百了,埋了烧了都方便。给竹枪烧一下,虽说死不了,却别想再上阵。南人的心肠可是歹毒得紧!”

大公鼎在前面不觉皱了下眉。长子没见识,次子虽有见识,就是爱卖弄,说话不看场合,都是不省心。

只不过二儿子说得也没错,士卒只伤不死的确是很麻烦。尽管不会像大昌嗣那样明说出来,但大公鼎同样觉得伤兵们还是一死百了比较好。人死了,拖到营地外远远的埋了就是。但换成是受伤,却要好生照料。

一声来自身后病房的凄厉惨叫否定了大公鼎的想法——军中的伤病,并没有得到所谓的‘好生照料’,甚至不能叫做照料。

不是大公鼎他们这些高层将领忽视,而是实在缺乏合格的医疗人才,使得病房不远处,总能燃起焚烧尸体的火堆——幸好党项人死了之后就地埋了就可以,需要将骨灰带回家去的,只有大辽子民。

一声声嘶哑的叫声如同杀猪一般凄惨,大概是因为清洗伤口时的疼痛,大公鼎看看身边,连亲兵们都是一幅不忍卒听的表情。

“应该将那些巫医丢进火堆烧掉才对。”大昌龄愤怒着,“士气全完了。”

一声声的惨叫仿佛是在印证大昌龄的正确性,幸而病房内的医工们做了些补救,惨叫声戛然而止,一下就变得安静了起来。

大公鼎父子自然是知道医工们是怎么做的,大昌龄冷哼着:“早用柳树皮塞住嘴不就没这么闹了!”

“塞嘴的是柳树枝,”大昌嗣更正道,“裹伤口才用柳树皮。”

大昌龄悻然道,“还不都一样。”

由于宋人种痘法的流行,宋军中的医疗制度,如今也被辽人仿效了起来,学着宋人设了随军医院和疗养院,连里面的章程,都是跟宋人一模一样。

但跟宋人军中的那些翰林医官不同,溥乐城外的随军医院中充斥着旧日的巫医。当一名伤员被抬进医院的病房后,巫医们会先用柳树根烧成的灰来止血,再抹上柳树叶炼出的药汁,然后用柳树皮过好伤口,最后再往伤兵们嘴里塞一截柳树枝好让他们闭嘴。如果不管用,他们还会绕着火堆跳一段大神。

这就是全套的医治流程和医疗手段。

并不是说巫医们在国中时都是用柳树来医人,他们也会用其他药草,只是到了兴灵后,一时间还能找到的药材好像就剩柳树了。

而且在这么做之前,他们会先确认伤兵到底有救没救,以免浪费经过精心炮制的柳树皮。所有看起来快不行的士卒,不论是真的没救,还是看起来救不了,都会**脆利落的放弃,除非这些伤兵有个奢遮的好后台。

这样的医工,当真是丢进火堆里烧了最好。

已经是入夜,不远处的溥乐城头上,灯火将城墙的轮廓在沉黑的夜色中勾勒了出来。

而围城的营地内,一堆堆柴堆也在熊熊燃烧着,热浪驱散了寒流。士兵们围在火堆边小声说着些什么。只看他们时不时回头望着充作病房的营帐,就知道多半是又在议论宋人这几天所用的新兵器。

大公鼎知道,由于八牛弩、神臂弓、板甲和飞船的关系,大辽军中其实十分忌惮宋人的各色新式兵器。从上到下,莫不如此。三个南人士兵才能抵得上一员辽兵,南朝之所以能跟大辽分庭抗礼,一个是每年按时送到的岁币,另一个,就是仗着手艺精巧,打造出来的各色兵器。

对宋人神兵利器的畏惧,澶渊之盟后,便有了八牛弩。宣宗驾崩后,多了飞船。到了兴灵,亲眼见证了板甲和神臂弓的作用。今天则又加上了火器。

宋军的火器绝不止竹火枪——这是前几日从城下回来的士兵起的名字——前些天党项人攻城的时候,大公鼎已经看见过城中守军使用了不少。

毒烟火球烧起的毒烟逼退了两次进攻,而猛火油柜更是给党项人带来了不小的损失。只是在大公鼎看来,都不算实用,远比不上神臂弓的威力。只是将汉人的手艺又表现了一番而已。

就如现在将人喷得满脸开花的竹枪,其实说起来也没多少用,随便拿面盾牌就能挡住了,隔得远了更不用担心。而神臂弓在近处的射击,不是厚重的橹盾根本防不住。

说起管用,还是前两日从城头上飞起来的火箭。这两天那种刺耳的尖啸声好歹是没了,顺带的,飘在天上的飞船也没了。两天前,绑着火药的长箭不停地从城头上飞起,一个劲的瞄准飞船的气囊,一日之内连射了三五十次,终于是给宋人射中了。

由丝麻织物缝制的飞船在火箭射穿之后,如果只是破个口子,还能修补一下,只是烧起来后就真的没办法了。虽说这飞船为了不间断的监视城中,本就是两具轮流上天,可烧了一具后,剩下的一具也不敢用了。这自然也让士兵们更加畏惧宋军的兵器。

这里的冬天冷得很啊。

大公鼎深深的吐了一口气,淡淡的白雾弥散开来。望着磐石一般屹立在灵州川和瀚海之间的溥乐城,清晰的感受到了夜色中那深重的寒意。比起辽东和大定府,也差不太多了。

大公鼎。大姓,名公鼎,是渤海国王大祚.荣传下来的血裔,世代居于辽阳。大辽定鼎,渤海国灭,他这一支依然在辽国做着高官显宦。到了统和年间,其祖以充实中京道的名义,被迁移至中京大定府。如今因为之前支持耶律乙辛,又被赏赐了兴灵的土地,一族上千口,一同移居到党项人的故土上。

如今大公鼎是安化州怀远军节度使——这个安化州,就是兴庆府。西夏灭亡,不再是国都,自然不方便保持兴庆府的名号。本来改回原名兴州也不差,可惜东京道那边已经有了一个兴州,所以便成了安化州——西夏旧都的军政之事皆由其掌握。也因此,他才会率部随大军南下。

虽说大公鼎执掌一州军政,且是西夏旧都,不过在他的头上,还有一个统掌西平府【灵州】、安化州、怀州、顺州、静州、定州等六州府军政之事的西平六州都管——耶律余里。

这个西平六州都管司管辖的范围正是贺兰山下的兴灵地区。

由于大漠阻隔,兴灵其实可以算是孤悬在外的飞地,能发来援军的只有黑山下尚父的斡鲁朵。而两地之间的距离,沿着黄河走,差不多有一千多里。说起来草原上的阻卜人其实离得还更近一点,但无论如何,奚族、渤海、契丹,甚至汉人,都不会去信任他们。

大公鼎心中隐隐忧虑着,这几日来感觉越来越不妙,但都管耶律余里一直都是有恃无恐,一门心思围着溥乐城。尽管耶律余里说是尊奉尚父之命,打压一下南人的气焰,顺便将兴灵的党项人清理干净,但谁都知道,耶律乙辛不过是为了给出使南朝的萧禧一壮行色。

“也许尚父并不在乎兴灵的得失,胜也好,败也好,都能逼宋人拿出好处来。”

大公鼎的低声自语,却被大昌嗣给听到了。

“父亲!区区南人,若不是据守坚城,我大辽精兵早就将他们踏平了!哪里会败?没看为了溥乐城这么多天,韦州没来援救,连银夏的种谔也没敢来!溥乐城里的可是他的亲生儿子!”

大公鼎闻言,脸色更加阴沉了几分。

如果是宫分军或是皮室军的详稳来说这番话倒也罢了,头下军说什么必胜?

大公鼎精通汉人之学,甚至能做汉诗。他清楚,耶律乙辛将西平六州分割授受,其实就是分封建制。能被分派到此处得到一片领地的,全是在宣宗皇帝出事之后,选择支持耶律乙辛的部族。但他们并不能算是耶律乙辛的嫡系,所以才会被迁移到兴灵来。这里是个田土肥沃,水草丰茂的好地方,只是跟宋人靠得太近,离本土太远。

“你要小瞧种谔,先挣下跟他差不多的功劳再说!”大公鼎厉声呵斥着儿子。

大昌嗣闻言不敢再辩,只是还小声的咕哝着:“不过是对党项人有些功劳,算得了什么?”

“奴瓜你第一次与人见面的时候,是先用鼻孔看人的吗?!”大公鼎叫着儿子的小名,显然已是怒极。

