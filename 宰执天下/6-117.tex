\section{第11章 飞雷喧野传声教(14)}

清风楼上,左禹单人独坐。

临窗的小桌前,再无一人陪酒。

楼下大街,车马川流,行人如织,好一派热闹的景象,而楼阁之上,也只有左禹的这一桌是形单影只,冷清无比。

左禹在京城的商界,算是小有身份的行商。尽管没有加入哪家行会,但他走的是河北到京城的商路,主要是贩运来自北国的药材和毛皮,真材实料,价格合理,所以与几家相关行会的行首关系都不错。每年到了开封城中,总能得到各方宴请,没了宴请时,就出面请客,除了早饭之外,难得能有一顿自己一个人对着酒杯。

对于一名一年只有几十天在京城的行商来说,每一顿饭都是与人结交或加深关系的机会,浪费这样的机会,就是在浪费金钱。如左禹这样的行商,便是去小甜水巷消火,也会多招呼几个朋友同去,以期能够加深彼此的情谊。

有时候,从京城传来的一句话,就能让一家商号化险为夷,起死回生——朋友是从来不会嫌多的。

只是这时候,左禹完全不想跟任何熟人照面,连随行的伴当都没带,找了个不熟悉的酒店,坐下来临窗独酌。

但就算拿起酒杯,左禹不想听到的东西,依然往他耳朵里钻。

“听说了没有,辽国的国使看到皇城中的大将军炮,吓得连魂都没了,在炮座前面怔了有小半刻钟,让太后多等了好一阵。”

“不是怔了半刻钟,是吓得屁滚尿流,不得不换了一身衣服才上殿去见太后,”

“这些北虏,弄得京城里一股骚气不说,还把皇城都污秽了。”

“该不是耶律太师看打不过了,求了个法师想要做法,故意的吧。”

“怎么是故意?”

“肯定是鞑子没见识,觉得火炮是小韩参政弄出来的法器,所以才有白日放雷。想要破术法,带不了黑狗血进皇城,就只能用粪尿了。”

辽国国使刚到京城就被吓得屁滚尿流,这当然大涨宋人的士气。可相应的,所有辽人自是愤恨不已。

周围的酒话传进耳朵里越多,左禹捏着酒杯的手便收得越紧。

雕花银杯虽然好看,可绝对算不上结实,当邻近的两桌酒客因为说起同一话题,开始大笑着一起开始祝酒,银杯终于喀嚓一下,被捏得扁了。

一直都对外自称乡贯保州的行商左禹,实际上却是出身于辽国的南京道析津府。

尽管通过不同途径了解到的细节都告诉左禹,辽国国使被火炮惊得魂飞魄散完全是以讹传讹的谣言。可当他听到辽国的国使在传闻中如此丢人现眼,依然就像自己被侮辱了一般,羞恼之情充斥胸臆。

从石敬瑭将幽燕诸州献给辽国那一年开始,左禹家就一直是辽国的子民,言行举止风俗习惯依然是汉人的模样,但对于任何加之于辽国的侮辱却还是感同身受。

就算一个人喝酒的时候,都免不了要受气,左禹重重的一顿坏掉的酒杯,“店家,结账!”

丢下才动了几筷子的酒菜,在跑堂小二惊讶的目光中,左禹会了钞,赔了酒杯的钱,就跨出门去。

走到大街上,车来车往,左禹一时却不知往何处去。

国使丢人现眼,让左禹愤恨不已。

不过更让他心烦的不是耶律迪在皇城中的失态,而是今天收到的命令,要他尽快打探到有关火炮的实情,乃至得到火炮的具体图纸,可以供国中进行仿造。

东京城中无人知晓,商界小有名气的河北行商,不过是一名细作,手上让人羡慕的货源,其实则来自辽国国内的支持。

虽说细作并非本来的营生,可父母兄弟乃至长子都在国内,左禹也不敢不尽心。何况人在异国他乡,左禹日夜提心吊胆,从来不敢与人深交,也没什么知心好友,毫无归属感的国度,让他宁可选则自己生长的地方。

不过来自上面愚蠢的命令,使得左禹对谣言的恼怒,化为了对国中高官显宦们的愤恨。

若是在往日,左禹总是会选择在开封府外和内城十字大街处的酒楼请客,不仅档次高,能落足人情,而且官吏出现的最多。听到小道消息的几率也是最高。

但他现在根本就不想去请人。

有关巨型火炮的具体消息,早半个月前,就由另外一拨安插在京城中的细作就传了回去,只是这一次的使节运气不好,没有收到。而左禹到了京城之后,也从几个相熟的生意伙伴那里听说了一点。

不仅有关巨型火炮的底细都探听到了,还有来龙去脉,在酒桌上都披露了一干二净。

那几件上万斤青铜铸成的火炮,也只是听个响而已。

从不同渠道总结出来的结论,左禹已经不怎么怀疑了,可他有办法让国内相信吗?

当然不可能。

左禹很清楚这一点,他没那个能力让国内上层相信自己的话。

在那些高高在上的贵人们眼中,所谓新铸的万斤火炮不能发射炮弹的报告,肯定是宋人已经无法避免这门火炮的消息泄露,故意散布出来的谣言。而京城中的细作们为谣言所惑,更重要的是不敢去查探。

而且同一时间,在京城中传播的消息,也有很多是驳斥这些说法,认为几门特意放在皇城内的火炮,是远远超过之前所有火炮的神兵利器,可以一炮糜烂百里。

所以之前上报的情报,完全给国内当成了搪塞。

左禹暗暗叹了一声,要是自己也这么报上去,恐怕不会有好日子过。

下意识的,就往落脚的会馆走去,虽说回去后就少不了客来客往,可现在也没处去,而且抛下仆人单独离开太久,也会惹人疑窦。

不过走了没多久,左禹突然感觉到有人盯上了自己,已经转了两个路口,却还是被盯着,已经不能说是误会了。

脚步随着心猛地一沉,但又立刻恢复正常。

皇城司的人,全都盯着宋国的朝臣,哪里有空管一个行商,即便是契丹细作也不管他们什么事,真要说起来,自己被开封城里的小贼盯上的可能性都更大一点。

左禹摸了摸怀里,襟袋里的钱囊还在。

若是有伴当在身边,沉重的钱袋可以放在他的身上。有了大钱后,随便带个几贯铜铁钱在身边也不费事。但左禹单独出来,就带了半贯不到。

在酒楼中坐了一坐,就剩下十几枚十文大钱,还有一些零碎的钱币。算不上多,可再加上两个随时可以到金银铺去兑换的小银锞子,要是给哪个小贼摸了去,自己就亏大了。

再转过前面的街道,迎面而来是一支人数众多的队伍,旗牌俱全,还有一张极为显眼的清凉伞,左禹一见,便和周围所有行都谦卑的退到了路旁。

也许空口白话的流言无法取信于人,但要是能得到火炮的图样,想必国内就不会太过催逼了。

大宋宰执的队列从面前走过,左禹视而不见,开始认真的思考起来。

……………………

从王安石的府上出来。

韩冈还在回忆着之前与自家岳父的争执。

有关辽国使者受辱于皇城的谣言传得满天飞,王安石如今虽不理事,也不免开始关注。

无论见到火炮之后有多么震惊,也不至于让辽国的国使连自己的身体都失控。

据当时守着宣德门内的神机军将校回报,耶律迪的确大吃一惊,但等他走到御前,冲太后行礼的时候,已经完全没有异样了。

相形之下,反倒是太后改变过往的成例,辽国国使甫进京便招其陛见的举动,在朝堂上更惹人非议一点。

“太后想瞧个热闹,也不是什么大事。”

韩冈此前还对自家岳父如此说道。

“对辽邦交岂有小事?!”

王安石立刻反斥回来。

外交的确无小事。韩冈承认这一点,否则朝廷也不会要求出使的使节,记录好自己的一举一动,并监察身边的使团成员。

可眼下宋辽两国之间的关系,又岂是正常的外交?不过是放几门礼炮,结果虽然是让辽使稍稍吃了一惊,但那也算不上是有多违反外交礼节。

在韩冈看来,这绝不是什么大事。

有太常礼院前后盯着,还有一沓子惯例、故事要遵从,太后接见辽国国使,在礼数上不会行差步错。而不动声色的给敌国使节一个下马威,此事无伤大雅,至于京城中的流言,那就是百姓们喜闻乐见,故而才会有这样的情况。

现如今还不是与辽国交恶的时间,朝臣不会同意太后出来观兵耀武,只不过放上两炮,两府中人都睁一只眼闭一只眼,也就王安石感觉不合适。

在韩冈有着明确意见的公务上,王安石争不过韩冈,确切的说,是争不过韩冈和他背后的太后。太后总是会选择倾向于韩冈,使得隔一段时间便与韩冈唱反调的王安石的意见,最后免不了为人忽视。

只要韩冈不推动党争,抢先挑起事端,在他过来的时候抱怨一下,也已经是王安石现在所能做到的全部了。
