\section{第13章 晨奎错落天日近(20)}

蒲宗孟清晨起来的时候,离上朝正好还有半个时辰。

尽管从学士府往皇城去,都要近两刻钟,但蒲宗孟一睁开眼,就有七八名使女,端着银盆、银镜、手巾、漱口水、早餐、衣冠、饰物,依次上来服侍。

先用盐水,再用浓茶,先后漱口两次,最后才拿着马尾制的牙刷,沾了牙粉来刷牙。牙粉中掺了薄荷,漱口后依然清凉,不像之前用松脂和茯苓制成的牙粉的怪味让人习惯不来。

用力鼓动着腮帮子,蒲宗孟冲洗掉了嘴里的牙粉残余,杯子被取走,洗脸的银盆就段到了面前。

银盆里面装了半盆洗脸水,还冒着热气,里面掺了一点香精,清清淡淡,清雅怡人。

领头的使女嗅了一下摇头,吩咐道:“还要再加两滴桂花精露。”

一名使女听命,忙拿出了一个浅绿色的玻璃瓶,拔下银质的塞子,向盆中滴了两滴新鲜的香精,盆中的温水散发出来的气息,越发的香气馥郁。

用掺了香精的洗脸水洗过脸,略嫌清简的早餐就端了上来,年纪大了,蒲宗孟再怎么好奢侈,为了养生也只能越吃越清淡。

匆匆解决了早餐,先冠冕,再衣袍,然后是零碎的饰品、腰带。一名使女举起半尺大小的银镜,对着蒲宗孟。蒲宗孟戴上水晶眼镜,在银镜前左照右照。

“学士今日好讲究。”昨夜侍寝的姬妾在旁笑道。

蒲宗孟调了调襟口,“今天朝会非同以往,岂能不慎重?”

“奴婢也听说了,满朝朱紫,同聚文德殿上,共商国是,乃是小韩相公的提议。”

蒲宗孟的这姬妾不过十七八,提起小韩相公,便不禁悠然神往。蒲宗孟眼中一冷,身前镜中,白发红颜,对比分外强烈。

“想不到都传到尔等耳中。”蒲宗孟神色平淡的说道。

姬妾听出了话语中潜藏的怒意,连忙笑道:“只是闲言碎语罢了,闲来无事听来说说。这等国家大事,我等奴婢议论得再多,也比不上学士殿上的一句话有用。”

蒲宗孟眼神稍稍和缓了一点。

韩冈的任何言辞,总能让京城士民奔走相告,口耳相传。

这是他历年来积累下来的声望所带来的,也是蒲宗孟愿意将赌注压在他身上的原因。

蒲宗孟的妹妹是周敦颐的继室,因而从渊源上,他与周敦颐的弟子二程也有些关系。

当然,这种关系除了登门造访时写在帖子上有点用,基本上都不会被人放在心上。蒲宗孟的政治倾向,从来都不在旧党那一边。尽管他入朝甚早,不说富韩之辈,与苏轼那逆贼都有些交情,可他之前站在新党一边,现在又选择了韩冈。

蒲宗孟扫了眼床榻前,小桌上有新学的书,也有气学的,主要还是气学的;而一旁的书架上,程学的书也有,不过放在最下面,很长时间都没有动了。尽管看不到灰尘,可上面连个折痕都没有,新得就像是刚买来的。

自己都这般,还能怪无知妇人?

蒲宗孟自嘲的笑了笑,又整了整衣襟,然后举步出门。

蒲宗孟在朝臣中,被称为是最为奢侈的一个,什么一日必屠羊十只,什么一夜必燃烛三百支,什么‘常日盥洁,有小洗面、大洗面、小濯足、大濯足、小大澡浴之别。每用婢子数人,一浴至汤五斛’,为此御史盯上他也不是一天两天了。

可一日十羊,并不是他一家吃,还有亲友、门客要分赡。每夜燃烛三百,则是过去的事了,现在用的是玻璃油灯,与其他官宦和富贵人家家里一般。

至于说起爱洁,以医道闻名的韩冈同样不差,听说他在家中也是天天洗澡,只不过韩冈找了个清洁厚生做名目,被人群起仿效。而他蒲宗孟天天洗澡,早晚洗脸、洗脚,就是奢侈的代名词了。洗一次澡,要五斛热水算得了什么?多少官宦家中,都打造了只用来烧水的锅炉,专门用来洗澡,每天烧得热水绝不会比五斛更少,很多的就是在家里砌了泡澡的浴池,木质的,石质的,还有贴了瓷片的,即使是将最小的浴池给灌满都至少五斛滚水。

可有着这等名声,就是御史手中的把柄。即使依然站在新党一边,也做不了王安石的心腹,新党中也收不到人缘,总会有人想把自己给拱下去,那时候,章敦、吕嘉问,哪个能靠得住?何况这样做,还会恶了太后,只有站在韩冈一边,才能得到太后的青睐。

出了内院院门,上朝的随行人马都已经准备好了,狨猴毛皮制成的狨座,在火光下仍能反射着金芒。

蒲宗孟翻身上马,一行人点起灯笼,打起旗牌,簇拥着他,自府中鱼贯而出,还有两刻钟,有足够的时间抵达不远处的皇城城下。

前往皇城的道路上,官员越来越多,人虽众,但气氛却与往日迥然有别。招呼声稀稀落落,大多数三五成群,并辔而行,相互交流着什么。

蒲宗孟一时没有遇到熟人,但前面的队伍突然慢了下来,一人转身迎了过来。

比起蒲宗孟身边的十几随从,那一支队伍的成员足足有数十近百之多。显而易见是宰辅一级的队列。

“可是玉堂承旨蒲学士?”

“正是。”

“小人乃张参政府中家仆,奉参政吩咐,请学士上前叙话。”

‘张璪?’

蒲宗孟皱了皱眉头,想了一下,然后依言上前。

快要抵达皇城城下,蒲宗孟和张璪分了开来。

蒲宗孟前行了几步,然后下马。而张璪则往更前方去了,没什么人敢拦在参知政事的前面。

蒲宗孟望着张璪,眼神沉凝。

方才几句话,两人都是在说着今日的会议。而言辞之下,更是在试探着对方的选择。

几句话过后,蒲宗孟知道了张璪的选择,他相信,张璪也知道了他的选择。

因为他的决定早就做出来了。

两日前,太后与韩冈的问对,从宫中传出来的记录很详细,可偏偏最关键的内容没有出来。

当时蒲宗孟在学士院中笑言,‘这下王介甫和章七得傻眼了。’

尽管当时只有几个吏员在场,但估计这话现在已经传到了王安石与章敦那边去了,不过更重要的是传到韩冈的耳朵里。

韩冈就像胜利者一样,对太后说了那么一通话。

他的自信心,到底是从哪里来?难道不是从已经被说服的太后身上?!

有其果,怎么可能无其因?

以韩冈的为人,他怎么会没有把握就出手?

蒲宗孟可以肯定,从宫中传出来的肯定不是全部的对话,而仅仅是一部分。

他遥遥看见韩冈,而韩冈正好也将视线投射过来。

两人相互点头致意,接着便各自将头扭了开去。就像交情一般的同僚,尽过礼数没有多余话可说。

可一切都心照不宣。

蒲宗孟给韩冈的感觉是修饰过度。每次见他,上下衣袍都是新制的。

据说蒲宗孟的公服是一个月换一套,月月常新,韩冈知道这不确切,而是半月换新,根本就不下水去洗。

这个时代的染料,染到布上,很容易脱色,洗一次就会变淡一次,而且掉色还掉得不均匀,一次两次还好,洗个三五浇,就可以看见穿衣服的人变成梅花鹿了。

不论是衣冠朱紫的达官贵人,还是皂、青两色衣袍居多的寻常百姓,他们染过色的衣服都是一样不耐洗涤。王安石经常穿一件洗脱色的公服上殿,一点也不在乎,在京城,也经常可以看见一身退色朝服的穷苦官员。韩冈则会稍稍注意的一点,洗过两三次后,便会换掉退色比较严重的公服,衣服积得多了就拿去染坊重新染色。而蒲宗孟则从来不会出现穿旧衣的情况。

这样性喜奢侈的官员,虽然不是自己的基本盘,但他也是会支持自己的一份子。

国是从来不会直接在诏书上出现,而是从一条条的法令中体现。王安石拿着国是压人十几年,甚至没有落于文字。今日与一众重臣共商国是的协商会议,只是决定是否要改便未来的施政方针的朝会,但这已经足够韩冈施展了。

两天前,韩冈朝后留对;一天前,也就是昨日,太后下诏,东府签书,对共商国是的协商会议的制度进行了初步的规定。

王安石对此没有表示异议,默认了。东府之中,位居前列的韩绛和张璪都支持韩冈,有了他们的签名,诏书就有了合法性,这也是除了王安石不想寒了人心之外,默认韩冈把重臣拉出来选举的另一个原因。

两府宰辅拥有提案权,如果有平章军国重事,同样有着提案权。但这一份诏书,排除了宣徽使等一系列能立足于宰执班中的重臣的提案权,也就是说,吕惠卿此时回京,也只有投票的权力。

确定之后,五年内禁止在举行同样的会议,这五年间,敢于沮坏国是者必远窜,只有五年后,才允许宰辅再次提议。而这五年内的治政方针,需要达到什么目标,将会使用什么手段,都在协商会议上给定下来。

成败在此一举,可韩冈的脸上,完全找不到患得患失的不安。

“玉昆。”章敦不知什么时候走了上来,“今日胸有成竹?”

他低声问,抬头望着在城垛上探出炮口的火炮。

“太后垂帘有多少日子了?”韩冈反问。