\section{第11章 城下马鸣谁与守(四)}

韩冈给折可适的意见拍了板,这一次的合议就算是定了案。

与会众人向韩冈行礼后离开。韩冈的幕僚全都是气学弟子,相互之间交情颇深,出门后皆是结伴而行,只有折可适是单独一人,与他人都隔了好几步。

韩冈在后面看着,有些头疼。难道说文武之别差距就那么大?但种建中在同窗学友之中,人缘却并不差。过去求学时代的交往不谈,就是近年来,种建中由文职转武职,但去拜访他的同学依然不少。韩冈甚至听说过他与同学游山玩水,最后作诗唱和的传闻。

或许还是门户之见的问题,也有可能是自己的看重让他受到众人的敌视。韩冈对此微微皱起眉。不说折家的地位,就是折可适本人在军事上的才华,也足以对得起郭逵和韩冈对他的看重了。气学的弟子们若是都这般小家子气,未免就太让人失望了。

不过等折可适走到回廊转角的地方,就见到黄裳和几个同学一起走了上去搭话。虽然接下来,几人就转过了回廊,看不见了身影,但就在那一瞥之间,韩冈倒是看清了他们之间的气氛并不是针锋相对的僵硬。

这是好事,如果双方的交情能继续进展下去那就好了。

放下了心头事,韩冈转回厅中,处理还剩下许多的公务,并等待着妻儿的到来。

“龙图,夫人她们到了。”入夜前,韩冈的家眷终于抵达了太原城。

一别数月只有书信相通,远在京城的妻妾儿女,还真是让韩冈惦念不已。他就是心如铁石,也不免要思念家人。不能在父母身边尽孝,也不能在妻儿身边陪伴,韩冈作为儿子、丈夫和父亲很是失职。深闺中,王旖她们也不免有悔教夫婿觅封侯的怨怼,不过看到瘦削了几分的丈夫,却把满心的幽怨抛诸脑后,却又抱怨起韩冈不会照顾自己,也不知道注意身体。

而小儿女们见到父亲,则是雀跃不已。韩冈对儿女虽然在检查功课时严格,但做完功课就管得没那么严了。尤其是在京里的同群牧司任上,韩冈清闲无比,指着自然万物说着有趣的故事,闲来教习弓马,带着儿女一起出外游玩,比起经常抡戒尺打掌心、把三个已经入学开蒙的儿子女儿打得眼泪汪汪的王旖可是宽和多了。

一路车马劳顿,耗尽了小孩子们的体力。吃饭前还精力旺盛的闹腾着,吃过饭后,就一个个睁不开眼睛。王旖看了,就让他们的乳母和贴身侍婢领着去房中睡觉。

家里的孩子回了他们的房,正屋中可就只剩韩冈和妻妾几人。

没了外人和小孩子看着,韩冈也不用装得道貌盎然的样子,探手揽过王旖的腰,笑道:“这几月,让娘子辛苦了。”

王旖白了丈夫一眼,虽说是赞许的好话,可韩冈以现在的动作说出来,总是带着调笑的味道。

“官人每次回来都这么说,可一旦出外,就把奴家和孩子们都丢到脑后,没有个半年,就想不到派人来接。”

“官人该不会乐不思蜀了?”周南轻笑着问道。

严素心佯作帮韩冈辩解:“官人孤身在外,可是一向洁身自好。”

其实给韩冈送美女的有不少,孤身在外的年轻官员,最容易为女色所吸引,不过韩冈没空,而且他胃口也被养刁了,庸脂俗粉哪里能看得上,半开玩笑道:“洁身自好那是为夫忙得都没心思,什么时候清闲下来会考虑的。”

王旖听了,转过头就不理会韩冈。韩冈哭笑不得,“为夫说笑呢,这是吃哪门子的飞醋?”

周南笑道:“姐姐的醋味,不如官人席上放得醋多。”

“三哥哥让人做的菜是够酸的。”

山西多面食,因水土原因,又甚为嗜酸。西军出阵,除了定规的粮秣,以及随身的糗糒【北宋的行军干粮】之外,盐和酱是少不了的。而河东军出阵,除了盐和酱,还要更加一份老醋。为了方便携带,军中以一尺粗布浸在一升老醋中,吸饱了醋之后拿出来晾晒,制成了古代式的方便调料。吃饭时,丢一片醋布到汤里,就是一碗酸汤。

韩冈在河东几个月,口味在不知不觉间,慢慢的也改了,今晚接风洗尘的酒菜醋味便颇重。

“吃醋?为夫的确醋吃得多了。”韩冈说着慢悠悠的站起身,踱了两步,就堵在门口。笑容一转就变得得意:“为夫曾听闻,醋吃多了生女儿,碱吃多了生儿子。自上任以来每日喝醋,将养了这么多天,正好要试一试管不管用。”

闺房间的秘事不能宣之于外,不过第二天起来,人人都觉得韩冈荣光焕发,精神大振,仿佛变了一个人。

韩冈在河东忙碌于国事,几个月下来,如同一根被张起的弓弦,被越拉越紧。而王旖、云娘、素心、周南他她们的到来,让韩冈在生活上也得到充分的照应。一夜过去,不仅仅是身体上的欢愉,精神上也得到了放松,他几个月来不断紧绷的神经,也因此而稍稍松弛下来。

但荒于政事是不可能的。就像京城里,每天早上天子和宰执们都免不了崇政殿上的议事。在地方州县中,每天一大清早,勤快的州官县官都会召集僚属,安排下一天的任务。

只是并不是所有的官员都会在例会中坚持太久,像韩冈这样只要人在治所中就会按时召集僚属一点也不耽搁的官员,其实算得上是凤毛麟角,大部分都会沉湎于湖光山色,柳岸杨花,呼朋唤友,饮酒赋诗之中。十年寒窗,可不是为了日夜操劳。

第二天的上午,韩冈便召集了经略司下面的官员,统一了观点。基本上,韩冈的声望已经极高,掌握河东经略司没费他多少力气,几位官员皆无胆量顶撞他的决定。尽管他们口虽服,心还不一定服,但在韩冈捅出大篓子之前,他们还只能老实听话。

当韩冈定了基调,不用多少口舌就让上下一致同意调遣三千骑兵去协助种谔,而剩下骑兵则协助步卒谨守寨防。

调动骑兵的命令和为此上禀朝堂的奏折刚刚发出去之后不久,韩冈又收到了京城传来的一则新闻。不动声色的浏览了一遍,他就直接找了个火头,将小小的纸片给烧了,另外照常用密语做了记录。

天子近况欠佳的消息没能让韩冈动容。不是他不担心赵顼的健康问题,也不是他不在意赵颢成了大宋天子的后果,只是信上写明了天子御文德殿主持朝会,既然他还能坐在朝堂上,履行他的职责,那么情况就并不是那么需要担心。或许天子身体的确不好,但怎么也能拖到几年之后,就是盐州全军覆没,也不至于当场倒毙。

“……或许吧。”韩冈没太多把握的低声自语。又不是能掐会算的半仙,寿数之事还真是不能一口咬定。

赵顼身体不好也不是一天两天了,所以一发病,就城里城外的鸡飞狗跳。换做是自己,有几人会认为药王弟子的韩玉昆发一阵晕眩,就会命不久长?

韩冈的心中其实一直都有这样的隐忧。不过这不是他能担心得来的。眼下的局面,分明是赵顼接连犯错的结果。要不是他急功近利,连累了前方的将士,不会有如今的窘境。

就不知道徐禧本人有没有回天之力,盐州的安危现在只得看他的能耐了。在所有人都不看好的情况下逆转成功,这样的战例在历史上并不鲜见。韩冈一直都认为固守盐州绝不是上佳的选择,但从来没有认为这是必败的结局。

不是韩冈自负,如果换做他自己或是郭逵去,倒有七八成的把握能将盐州守下来。换作是其他宿将领军,也绝不会稍逊。若是徐禧能将盐州的守御之责交给高永能或曲珍其中任何一位,同样是绝不会像现在一般令人担心。

但守住银、夏二州的把握,则是百分之百,顺便还能让攻的西贼和西贼的援军大半人马有来无回。上阵厮杀那是要死人的,风险越大,死得就越多。韩冈从来都不认为在有其他选择的情况下,有必要去冒风险,即便那个风险只有两三成。从面对灵州城开始,他的态度一直如此。

可令人遗憾的,皇帝总是做出错误的选择。眼下不论盐州成与败,韩冈本人远在太原,实是鞭长莫及。不过辽国若是想得寸进尺,只要西军还没有被伤到根基,那就只能是白日做梦,韩冈并不介意一脚将他们给踹醒。

不由自主的,脑中就闪过脚踹一群契丹贵人的画面。半是作呕,半是自嘲,韩冈一时忍不住就呵呵笑了几声。

“官人笑什么?”黑暗中,熟悉的声音在耳畔响起。

韩冈笑声一收,左手就向美貌的厨娘身上探去:“还没睡?”

“嗯……”严素心不自在的扭了一下,将韩冈的手拍开,“官人不也还没睡?”

“为夫在想怎么治一治契丹人呢,眼下就属他们得意!”

“官人有办法?!”严素心手肘支起身子,带着惊喜的问道。就算对外面的局势没有太多了解,她也是知道,哪一方才是大宋真正的威胁。

韩冈轻声喟叹:“就因为现在还没法儿去做,才只能在夜里想想。真的要做的时候,可就是要在白虎节堂里面发号施令了。”

严素心安静了,没有再多问,娇躯却紧靠向了韩冈。似是要用自己身上的暖意,来安慰失意的良人。

韩冈搂着主动贴上来的美妾,一瞬间就明了了严素心的心绪变化。虽然自己后面还有没来得及说出口的一句可以解释误会,但这时候却没有必要说了。

“不说这些不合时宜的话了。”韩冈翻了个身,在严素心耳边轻笑,“春宵苦短,还是做些合时宜的事。为夫就不信,不能给金娘生出个妹妹来。”

“官人……不要……”

来自胸口上一对小手无力的推拒,伴随着呢喃低语,韩冈发觉身下的娇躯一下火热起来。夜半时分的欢愉,一时散尽了他心中的积郁。

可惜太原府衙中的旖旎春光,只在夜半无人私语时。来自西北两个方向上的凛冬之时的肃杀,则已如洪流一般向孤悬在盐州的三万宋军将士侵袭而来。

当横山以北尽被深秋的萧瑟所笼罩,阻卜对盐州周边的骚扰也到达了最高峰。而就在同时,第一面西夏战旗,出现在盐州城下。

徐禧怀着胸有成竹的微笑:“终于来了。”

