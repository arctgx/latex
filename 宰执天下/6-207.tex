\section{第24章 夜雨更觉春风酣(上)}

听到前院人马喧嚣,严素心立刻站起身,对着玻璃窗整理了一下衣服,便匆匆走出房门。

来到二门处时,韩冈已经先进了门来。

“官人今儿回来怎么迟了?”

严素心说着,上前去为韩冈脱下外袍。

但让她感到意外,韩冈并没有穿早上出门时的那件斗篷,而是换了条素色的,而连腰上的金带和鱼袋都卸下了。

“怎么了?”她疑惑的问着。

韩冈将斗篷交给爱妾:“太皇太后上仙了。”

严素心闻言一怔,“太皇太后上仙了?”

仿佛是为了给韩冈的话作证明,就在此时,从天际中远远的传来了悠悠钟声。

第一下钟声来自于东北方,那是开宝寺的方向,不过紧接着,整个东京城,所有的钟都响了起来,

一声接着一声,将太皇太后的死讯传遍整个京城。

“怎么了?出了什么事?”

周南、云娘这时候才带着孩子们出来,两女方才带着孩子在后园中,出来便迟了一步。

“太皇太后上仙了。”韩冈又解释了一下。待孩子们行过礼,他问道:“你们姐姐还没回来?”

“给六婶留下来吃饭了。”

“这样啊。”

王安石的六弟王安上,如今在三司办差,担任盐铁副使。亲戚间平日里也有些来往,今日王旖便被请了过去。

不过被请去王安上家做客,对王旖来说其实苦不堪言。王安上本人倒好,但王旖的六婶婶眼孔小,总想借韩冈这个做宰相的侄女婿的光,对王旖的态度也。每次到了王安上家,说话都带着巴结,让王旖感觉十分不自在,但对方又是长辈,更不好翻脸,来请时也不能一直推脱,只能硬受着。

想起自家妻子如坐针毡的样子,韩冈就忍不住想笑,“这顿饭可不好吃。”

严素心和周南却没笑,周南紧张地问:“官人,真的不要紧?”

由不得严素心、周南不担心,对太皇太后去世的消息,韩冈的反应实在是太平淡了。太皇太后上仙,宰相却直接回家了,这怎么看也说不过去吧。

尽管当年太皇太后差点害死韩冈,韩家诸女都恨不得其早死,但不管太皇太后过去做了什么,宰相这个态度,不免为人诟病。再怎么说,周南和严素心都不想看到自家丈夫为士论攻击的情况。

看见两位姐姐脸上严肃的神情,云娘也张大眼睛,一起看着韩冈。

“不妨事的。”韩冈语气平静。

宫变失败之后,太皇太后其实就已经死了,政治生命宣告结束,在宫中也是被严加看管着,几年下来,在宫中的势力烟消云散,在朝野也是形同隐身。逢年过节的典礼仪式,都是以身体不适为借口,没让她参加。直到真正重病垂死,才再次惊动朝堂。现在她死了,只会让人松口气,

而且如果是当年宫变后不久,高太皇就去世,少不得会有谣言说是子妇弑姑——大宋以孝治天下,父母不论做了什么,做子女也不能报复。但这么些年过来了,去年太皇太后就病重待死,撑到今年才去世,这就不用担心什么谣言了。

宫变之后,连高家的人都高官显爵的养起来,只是不能任实差。之前太皇太后病重,太后不仅辍朝,还命宰辅去大相国寺祈福,做得已经是仁至义尽,没人能说她不是。

这样的情况下,还怎么有乱子?

进了厅门,韩冈坐下来大模大样的翘起脚,云娘上来帮着脱下了鞋袜。周南、严素心也从身后使女手中拿来了更换的衣服。

这些琐事,一直以来全都是妻妾们来做,尽管韩家的婢女上百,但王旖四女从不假手他人。

“明天开始就要忙了,今晚权且先歇一歇。”韩冈边换衣服,边说着,“太常礼院可是从今晚开始就要忙了。”

他说话中带着点幸灾乐祸的意思,这也算是苦中作乐了。

待韩冈穿好家常的袍服,周南拿起一条束带,来给他围上,顺便问道,“政事堂今晚就没事?”

韩冈抬起手,“都交给熊本了。”

将带扣扣好,调整了一下束带位置,周南仰起头,“熊参政?原来是他今天当值。”

“不是他……不过大哥去了横渠书院,还不是要先抹几天桌子?苦活累活,本都是新来的差事。”

云娘一下捂着嘴,想笑不敢笑,后面的使女也有差点笑出声的。

周南却没笑,她依然不能安心。当初宫变,可就是因为前夜是两个谋逆的宰执值守,才差点造成了不可挽回的后果。

让下人们先退了出去,她低声问,“官人,当真不要紧?”

“有王君万在,中书那边还有宗泽值守,怕什么?”

韩冈终于把自己的底气给说了出来,他也不想自家的妻妾都惶惶不安。

王君万是张守约的老部下,也是韩冈的老熟人。前些日子,王厚升任了三衙管军中排在最后的龙神卫四厢都指挥使,在北庭立了些功劳的王君万便接替了他的位置,担任东上阁门使。

这两年,皇城内的差事,至少有一名会是来自于西军的将领担任,王君万便是最新的一个。

王厚、李信先后升任或出征,但在宫内,韩冈不缺执掌兵权的门人。不说西军,就是京营禁军,当年河东御寇,也有多人在韩冈麾下听命,在韩冈手中,升官发财的为数众多,只靠旧日的威信,他想做点什么都有人听他的吩咐。

更不用说,中书门下今日值日的还有宗泽,更有多少想讨好韩冈的堂后官,真要出了什么事,韩冈必然第一个得到消息。

在内院换了衣服,若是往日,韩冈稍事休息,就会去外院面客。但今日,太皇太后去世,一应应酬也就要歇上一歇了。要不是王旖出去了,家里也可以难得一次的轻松一个晚上。

换完衣服,韩冈先去了一下书房,出来时,手里拿着几封信。

“是大哥的信?”

“就是大哥的。”韩冈扬了扬手中的信,“素心,看过了吗?”

严素心摇了摇头,韩冈没允许她看,她怎么可能拆信先自己看。

“大哥怎么样?”

韩冈子女众多,但家中的老大,是从她身上掉下来的肉。纵然对待子女,他都会说两句狠话,但亲生的儿子,怎么能不挂念?

“大哥一切还好,成绩也不差。”韩冈看着信,“倒是瑞麟了得,上一会射猎,硬是射杀了一匹狼。”

“狼?!”严素心的心一下就提到了嗓子眼。

“不要担心。”韩冈摇手笑道,“只是一匹而已,僧多粥少,能抢到这一匹,是瑞麟的本事,”

种家、姚家,七八家将门的子弟都在书院中,横渠书院中属于军事的科目比重,并不比其他科目要少。寻常的射柳不说,田猎也都按季举行。书院之中多少学子,都要参加射猎。突然间发现了一匹狼,几百人一起打狼,王祥说是射中还不如说捡到更合适些。

但听见韩冈夸奖王厚的儿子,周南就不禁脸上带了笑意。

严素心偷眼看这周南,见她笑起,也跟着笑道:“瑞麟越好,南娘就越高兴。这么着紧女婿啊?”

“我们做父母的没办法陪着他一辈子,只能靠她的夫婿了。南娘,待会儿春天的衣服,可要让人给他们带去。”

“哪一件?”

韩冈在中间插了进来,让人去取两人的新春装。

很快衣服就拿了来,韩冈的手指在衣角捻了一下,“手感这么细……是陇右的细棉布?”

韩冈也分不清棉布是不是自家的,但他知道,这个手感很细,不是普通市面上能买得到,感觉上就是陇右的。

“是不是陇右的不清楚,但肯定比江南的好。”

尽管大量采用机器辅助,陇右棉布有着巨大的生产成本优势。但运输成本上的差距,使得陇右棉布的最终成本,只比江南棉布的成本略低一成而已。不过陇右棉布,在市面上,就是卖的比江南棉布更贵一点。

早前江南棉布的售价因为京泗铁路贯通,价格下降了一成有余,甚至还有继续降价的余地。而最普通的陇右棉布,其每匹的价格经过不断调整,如今要比等级相当的江南棉布高出三五十文的样子。这个差价,没有大到影响到世人购买时的选择,同时还体现了陇右棉布的品牌价值。毕竟最高档的棉布,甚至能与蜀锦相当。

陇右棉布如今早就成了一块闪亮亮的招牌。同样的质量,一匹只差三五十文的话,世人只会去买陇右棉布。而且市面上还有一种专供军中的三层锦,以其厚度为名,虽不如民间传说的结实得可以做盔甲,但做内甲却是不差。没人不喜欢结实耐用的衣衫。这三层锦从来不出现在市面上,只有军中发下。在市井中只有偶然得见,却已经能够抵得上普通的三匹棉布的价格。

而江南出产的棉布,供给军中时,却是愈见轻薄,军中士卒,得陇右布则喜,得江南布则怨。尤其是京营禁军,一见江南布,便怨声载道,纵使被强行弹压下去,也还是记恨于心。
