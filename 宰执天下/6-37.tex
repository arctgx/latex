\section{第六章 见说崇山放四凶(四)}

夜色越发的浓重。

小殿中已经许久没有任何动静。

赵颢坐在正中冇央,死气沉沉,仿佛雕塑,很长的时间内都不见动作。

四周被明亮的灯火照着,不留一丝死角,灯后几十对警惕的眼睛正看着他。

赵颢先是从大庆殿被转移出来,先交给郭逵看守,到了午后又被转到内西门小殿中。

最后一个残存的失败者,正等待着胜利者对他进行的裁决。

赵颢知道自己的结果,他的大嫂绝不会再给他任何机会。

所以他没有重施故技,再装疯卖傻徒惹人嗤笑。他心中坚定,就算是死,也要像一个太宗皇帝的后人。

紧闭着双眼,拒绝灯火的光亮。赵颢在黑暗中沉浸在幻想里。

坐在大庆殿上,成为了真正的皇帝。

王安石、韩绛、章敦一个个被赶出了朝堂,在岭南的荒郊野地眼睁睁的看着子孙病死。

最是桀骜不驯的韩冈,也跪在自己的脚前,舔着靴子然后献上妻女祈求免死。那个让自己蒙受了多少侮辱的歌伎,更是要当着他的面,好好的整治到死为止。

干涸的笑声在寂静的殿中响起,又旋即收止。

即使在再美好的幻想中,理智也在不断警告赵颢,一切只是幻想而已。

韩冈就是他的天敌。像猫对老鼠,蛇对青蛙。

无穷无尽的悔恨噬咬着赵颢的心灵。

他失败在没有阻止天敌来到大庆殿上,他失败在没有阻止韩冈说话。

如果他让宋用臣在前夜就率班直去韩冈家中直接命其自裁;

如果他命石得一直接在宫城门口就将韩冈斩杀;

如果他在韩冈站出来后,就命韦四清将其乱刀砍死;

如果他在韩冈夺取了武器后,立刻命班直保护蔡确;

如果……如果……如果……

每一个如果之后,赵颢都会幻想起成功后的未来,然后就是越发深沉的后悔,后悔没早杀了那个担粪的小儿。

一串或轻或重的脚步声由远及近,听声音大约一队十来人的样子。

赵颢忽的一抖,悔恨也罢、幻想也罢,一切都烟消云散,原本凝固的姿态彻底瓦解。

他缓缓的抬起头,望着门口。

脚步声停在了殿门外,赵颢的身子再次僵硬住了。

只听得外面低声交流了几句话,大殿正门随即被推开。

门外一片黑,从明亮的大殿中看不清站在门前的究竟是谁。

是来送饭的吧……肯定是来送饭的!

赵颢对自己说着。

可几步之后,领头的一人走进了光线照耀的区域。

“王中正!”

赵颢惊声尖叫。瞬息间已是面如死灰,无论他怎么幻想,都想不到太后会让王中正来给自己送晚饭的可能。

来到赵颢面前,王中正行了一礼。

“正是在下。中正见过二大王。”

这个称呼让赵颢的眼中立刻闪起了希望冇的光芒。

若朝廷已经议定了他的罪名,肯定要夺去他一切官爵,废为庶人、族中除名。

可王中正现在却称呼他二大王!

还没有定罪,还能拖上几日!

但王中正接下来的动作打破了赵颢的幻想。

他向旁边让开一步,被亮出来的,是一名随从双手上捧着的一卷诏书,以及另一名随从捧着的一段白绫。

一见白绫,便犹如被巨锤击中,赵颢脑中一阵嗡嗡直响。

“想必大王已经明白了,也不必中正再多费唇舌了。”看着赵颢脸上的表情,王中正不紧不慢的说着,“谋逆重罪,朝中共议是凌迟,不过太后仁心,不想让大王见血。”

他深深的看了赵颢一眼,“想必大王不愿就此谢恩,所以中正也就不勉强了。”

赵颢直直的盯住那段白绫,仿佛什么也没听到。

王中正并不介意赵颢的沉默,弯了弯腰,“请大王上路。”

“悔不事先杀了韩冈!若是孤先命人杀了韩冈,你这阉人也敢在孤面前无礼?!”

王中正微微一笑。

很多时候都在最近处看着韩冈如何翻手为云覆手为雨,赵颢这番话听在王中正耳中,实在是自不量力的笑话。

为朝廷、为太后和天子立此殊勋,韩冈的未来已经在无人可以阻挡,能够在其微时便留下一段交情,实在是最大的运气。

他和声道:“若大王当真能够指使得动宋用臣、石得一、韦四清,韩东莱方才在殿上,就不会杀蔡确,而是大王了。”

世所公认的名将、内侍兵法第一、总是能够站在胜利者一方的王中正对赵颢微笑:

没人将你放在眼里。

赵颢须发怒张,尖声骂道:“阉货!”

王中正轻咳了一声,不急不怒,“大王,请体面点。”

他使了一个眼色,向太后赐下白绫便被搭在殿门边的支梁上,垂下的两端打成了一个硕大的结,变成了一个环。那名内侍还向下扯了一扯,足够结实。

王中正无视赵颢的愤怒,饶有兴致的看着,待一切准备就绪,他才悠然回头,对赵颢道:“大王,还请快一点。”

赵颢身子一抖,污言秽语再也骂不出口,他绝望的望着白色绞索,恐惧充满了他的心中。

王中正的催促,击破了赵颢本已经脆弱不堪的外壳。

“孤……孤……孤不想死……孤不想死!”赵颢颤着声,涕泪横流,“王中正,王留后,你求求太后。我是英宗的儿子,我是太皇太后的儿子,我是熙宗的弟弟,是官家的叔叔,她不能杀我,她不能杀我啊!”

王中正皱着眉,一脸的无奈。

赵颢方才满口市井俚语的污言秽语已经很难看了,可现在这个态度,却更加的难看了。

他叹了一口气,回头对左右随从道:“去帮一帮大王。”

两名班直立刻上前,一左一右,从腋下就将赵颢给架上。

赵颢拼死挣扎,使出的力气甚至让两名看起来有千斤气力的武夫都差点抓不住:“放开孤,放开……呜呜、呜呜。”

正在尖叫着的赵颢嘴里,给塞进去一团布,嘴被顶得老大,却说不出一句话来。

宫里面的内侍都给人塞惯了抹布。

宫中但凡杖责,都会怕哭闹声惊扰到贵人,总会先堵上嘴巴。王中正少年时也没少做过,乃是行家里手。

两名班直将赵颢牢牢夹住后便停下了动作,将视线投向王中正:接下来要怎么办?

“直接点吧。”王中正说道。

“呜呜呜呜。”

赵颢满是血丝的双眼里充满恐惧,像砧板上的活鱼一样一阵乱扭。

“别动。”王中正声音轻轻。

但他身后随即风声响起,又有两人猛扑上来,八只手如同铁钳将赵颢死死卡住。

站在挣扎得涨红了脸的赵颢面前,王中正叹息着:“这是何苦呢,大王要是能自重一点,都还能存一点体面。”

死到临头,谁还能有什么体面可言?但王中正说得仿佛理所当然的一件事。

随侍在侧的几名内侍一阵景仰。所谓生死,哪比得上声誉?也难怪王中正能走到这一步,这是只有功成名就的当世名将,才能够看到的世界。

王中正转头看着为赵颢准备下的刑具。

但赵颢被紧紧的拘束着,可白绫绞索则又被套在在高处,想要将他弄上去,恐怕会很麻烦。

叹了一口气,王中冇正又转回来看着赵颢。

没奈何,只能在多费点手脚了。

上前一步,王中正从随从手中接过根本没有被宣读的诏书,小心翼翼的插进赵颢的衣襟中。又亲自解下了那条丈许长的白绫,在赵颢脖子上松松的绕了一圈。

做完这一切,王中正耗了许多功夫,不过所有人都在看着他,甚至连一丁点的声息都没有。

最后,王中正拍了拍赵颢皱起的衣角,整理了一下襟口,笑容平和,举止从容,仿佛是给亲友在送行。

内侍兵法第一的王大珰,见惯了尸山血海,连亲手送亲王上路,都这般从容舒缓。

“拿好了。”王中正指着白绫两端。

两名孔武有力的班直听命而行,一人拽住了白绫的一头。

“送大王上路。”王中正的声音有着不容拒绝的坚定。

绷直了的白绫在赵颢的脖子上一下收紧,赵颢的身子也随之绷直,眼珠子瞪着几乎要掉下来,表情狰狞仿佛恶鬼一般。

但齐王殿下挣扎了几下之后,很快就松弛了下来,然后一阵恶臭在殿中弥漫开来。

王中正冷静的看着,不过他没有下令停手,直到近一刻钟之后,他才冲几位班直点了点头。

“去看一下,死了没有。”王中正吩咐着。

他这一次一并带了一名御医过来,正是为了确认二大王的死讯。

御医仔细的检查了心跳、脉搏和瞳孔,回头对王中正,“二大王已经过世了。”

“吊上去吧。”王中正说道,“对外就说是二大王谢恩后自尽。”

赵颢的尸身被挂在小殿中,离开了殿中,走了很远之后,王中正回头望过去,依然明亮的灯火下,殿中摇晃的影子,就像是被挂在屋檐下待风干的腌鸡。

‘结束了。’

王中正看了看自己的手,干干净净。

‘其实也没什么。’

他想着。

