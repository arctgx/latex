\section{第四章 惊云纷纷掠短篷(六)}

【第二更。】

叶涛的脚步也快了几分。

这几天,由于国子监中有人首告监中教授、直讲为人不正,收受学生馈赠,并以贿赂升不合格的学子入上舍——三舍法已经在国子监中推行,两千多名学生分为外舍、内舍和上舍,要想升舍都必须参加考试,而升到上舍之后,就有机会直接授官,差一点的也能直接参加殿侍或是省试。

已经是相当于进士科举的太学三舍升迁考试出了贿案,结果当然是天子震怒,名御史台彻查。御史台的穷究到底让每一位在国子监中讲学的官员都变得战战兢兢,他们无一例外,都是收过学生礼物的。

但叶涛觉得很委屈,他边走还边抱怨:“不过收点瓷器、竹簟和茶纸而已,就是夫子也是要收束脩。”

“正如致远所言。即有师徒之份,往来便是人情。怎么能以赃论处!?”沈铢咬着牙,“这哪里是不通人情,实在是御史台想弹劾人想疯了。”

……………………

王舜臣好不容易摆脱了几个同僚。

尽管他现在被夺了官身,但人人都知道他的靠山了得,只要这一阵风声过去,随时都能够起复。

而且跟他兄弟相称的韩冈,日后前途不可限量,过两年随口在天子面前提一句,官复原职也是等闲。

所以王舜臣,人人都想来个‘雪中送炭’,以便跟王舜臣背后的韩冈、王韶和章惇拉上关系。抱着这几条大腿,日后飞黄腾达也不再是梦想。

可有谁知道,王舜臣都怕往韩冈家里去。自己做下的蠢事,到了三哥面前,少不了要劈头盖脸的挨上一顿训斥。

犹豫不定的一步步挪出了宫门,王舜臣带在身边的伴当身旁,还站着两名身穿赤色元随袍服的汉子,都是韩冈家的人。

昨天还在东京城西的八角镇上,韩冈就已经派了人在等。一直陪着自己到了宫城,眼下想不去拜见韩冈都不可能。

见到等候的目标终究还是出来了,几人牵着马一起迎了上来。

王舜臣暗叹了一声,知道肯定跑不了,干脆就认了命。一咬牙,凶悍之气充斥胸中。难道还能砍头不成,不过是一顿训斥而已,怕个什么!

上了马,跟着韩家的家丁一路来到韩府。

从门口的司阍到院中奔走的家仆,见到王舜臣,都上前行礼问安。韩冈和王舜臣以兄弟相称,在韩家,王舜臣也能当半个主人。

王舜臣却也不敢多耽搁,穿过还在整修之中的几进院落,被领着一路来到位于后花园中的书房里。

韩冈正在书房中,读着手中的一封信,双眉紧锁,眼中也有几分凄然。

“小弟拜见三哥。”王舜臣进了书房,就跪下来磕头,砰砰的就磕了几个响头。

韩冈没让王舜臣起来,将手上的信扬了一扬,“你可知道这封信上说了什么?”

王舜臣有些楞,莫名其妙怎么能猜得到。摇摇头,“不知道。”

韩冈眼中戚色更浓,声音低沉:“王资政病得重了。秋天的时候也不知在哪里染了疫气,肚腹上生了毒疮。冬天好了些,但过年时却又一下转重了,这个春天不知道能不能撑得过去!”

王舜臣闻言一下跳了起来,惊叫道:“王枢密快不行了!?”

韩冈闭了一下眼睛,旋又睁开,叹道:“应当能吉人天相吧。”叹了几声,他的眼神转而锐利起来,“你我二十岁不到就得了官,都是借了王资政的光。你我年纪轻轻便身居高位,嫉妒者有之,憎恨者有之,如何能糊涂得做下此等蠢事!”

“俺也知道错了。”王舜臣并不争辩,低着头,“幸好三哥你比俺聪明,没有做了错事出来。”

“木秀于林,风必摧之;堆出于岸,流必湍之;行高于人,众必非之。我也是在风尖浪口之上,一举一动还不是被多少人盯着!”

“可惜他们都奈何不了三哥你!”王舜臣摸摸脑袋,“也就是俺太蠢了,学着三哥你做事做人,就没这一次的事了。三哥你放心,吃过这一次的亏,以后再也不会犯同样的错了!”

见王舜臣态度诚恳,韩冈也算是满意,放了他过关:“记得这句话就好。”又问道,“早上就进了皇城,中午也没吃吧?”

总算是过了关,王舜臣这一下子就放了心下来,笑道:“一天两顿也能过活,中午一顿少了也无所谓。”

韩冈起身,“先去吃饭,酒也帮你准备好了。”

王舜臣搓着手,紧随在后,喜道:“还是三哥了解俺。”

在小厅中,韩冈先招了妻妾儿女来拜见了叔叔,等一通礼节过后。韩冈和王舜臣才坐在一起,围着一桌酒菜,一边吃喝,一边说话:“这一次伐夏之役,不知道你是怎么想的?”

王舜臣一口将杯中酒喝光,抹了一下挂在胡须上的残酒,目光灼灼:“三哥知道俺的性子。总不能看着将功赎罪的机会在眼前飘过去。就是去做个阵前冲锋的小卒,也是甘心。”

不出意料的答案,韩冈叹了一声:“我已经给王中正写信去了,让他把你要过去。”

王舜臣眼睛一亮,惊叫道:“当真!?”

韩冈提着酒壶给王舜臣倒酒:“王中正的脾气你也知道,好名好利,只要你能帮他挣名挣利,许多事他还都能帮你担着。”

“多谢三哥。”王舜臣郑重其事的端起酒杯,向韩冈敬了。然后问道:“是要小弟去秦凤还是熙河?”

“王中正此次领两路兵马,但人不可能分成两个。只会是坐镇秦凤。赵隆跟着他在茂州立功,最是亲近。还有个刘昌祚,在秦凤做副总管,你去秦凤,没多少机会。”

“熙河路啊……”王舜臣脸上的喜色有些淡了,论起离兴灵远近,自然是秦凤近,熙河远,但他旋即又振奋起来,“熙河就熙河,不会比在秦风的赵隆走得慢。”

韩冈深深的看了王舜臣一眼:“环庆、鄜延、泾原、秦凤,此四路设立百年,各家势力盘根错节,就是一名十将、都头,都有可能跟总管、钤辖牵扯上关联。即便是种子正,也不能说在鄜延一手掌控全局,他也要妥协、退让,也有许多人要照顾,不会给你多少机会。但熙河是新辟,真正得用的将校就那么几人,你又是熟门熟路,只要调动熙河路兵马,少不得你的机会。”他说着又摇摇头,“当初就不该让你调去鄜延。”

王舜臣脸有惭色,其实那本是他看着熙河没有仗可打,所以才跟种谔一拍即合。他振奋起来笑道:“去熙河路也成,先把兰州拿下来。”

“兰州有禹臧花麻里应外合,当能一鼓即下。”韩冈完全不担心禹臧花麻会反复,一年几十万贯的利益,早就让禹臧家和熙河路各家人马紧密联系了起来。他瞧着王舜臣:“挣些功劳,官复原职不难。”

王舜臣纵声大笑道:“只是官复原职可对不起三哥你帮得俺这么多忙。这一回,俺拼了这条命,把兴庆府给打下来!”

韩冈脸色冷了下来,没有什么情绪波动的眼神盯着王舜臣,让他的笑声戛然而止。

“论远近,秦凤、泾原两路离兴庆府最近,一直向北就行了。论资望,种谔、高遵裕,哪一个王中正都压不下。”韩冈声音清冷,“你能比秦凤、泾原的兵马更快赶到灵州,种谔、高遵裕要熙河兵马做偏师的时候,王中正敢出面挡着?!”

他质问的声音一句比一句更严厉:“几十万兵马都往兴灵赶,你确定你能抢先吃到最大的一份?!”

王舜臣不是笨蛋,相反地,只要没被利益冲昏头脑,他绝对能算是个聪明人:“三哥的意思是……”

“往西去,将河西的凉州占下来。”

“凉州?……”王舜臣皱起眉,“能打下凉州,省得去灵州跟人争抢,倒也是好事。可朝廷肯答应吗?王都知也不一定能答应。”

“一块大饼六家相争,你力气不大,脚程不快,真要去抢,落到你手上就剩了饼渣。而河西的这块饼虽小,却是一家独吞。王中正该知道饼得拿到手才是自己的。”韩冈指着桌上的碗碟,权当作关西的地理,“兴灵危急,河西的党项兵马全都要往回调,正好是空虚的时候,只要一支偏师打下洪池岭,河西就在掌中,王中正当能舍得这一部人马。”

“三哥你已经写信跟王都知说过了?”

“你也知道,我是反对出兵兴灵,太过冒险了。现在让熙河路向西夺占河西,说出去是我的意见,最后结果就难说了……我不方便留人口实,由你传话,王中正能知道这是我的意思。”韩冈看了王舜臣一眼,“你若是不愿,三哥我就推荐其他人去了。你就在京城里多留个一年半载,我们兄弟也好多聚一聚。”

“三哥你也别吓唬俺,俺怎么可能不愿意?!”王舜臣提声道,他哪里甘心会在京城待罪,“三哥都为俺铺好路了,哪里还有二话。”

“其实你也可以放心。”韩冈笑道,“王处道那里我写过信了,他会帮着说话的。相信王中正不会误会这是谁的想法。”

王中正当初能在蜀中立功,平定茂州之乱,打先锋的赵隆当为头功,加上熙河路的蕃部,自己也能使唤得动,他等闲不会得罪自己。何况王中正过去的功劳,绝大多数都是从自己这里捡过去的,有过去的经验在,当会对自己的判断多信任几分。

王舜臣点点头,举杯敬向韩冈,心中的块垒俱去,他也就能放得下心来痛饮一番。

王舜臣在京中留了下来,等着王中正给天子的奏报抵京,而数日后,王中正的奏报还没到,苏轼则已经被押解抵京了。

