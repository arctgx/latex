\section{第34章 山云迢递若有闻(一)}

大军西去,送行的韩冈在城门口遥遥而望。

长长的洪流远上云山之间。灰黄色的烟尘滚滚如龙,渐渐消失在同样颜色的山中。

这是最后一批出战的队伍,他们将前往渭源堡与本阵汇合,听候王韶和高遵裕的命令。

计算脚程,最早出发的苗授和王舜臣此时也应该快到渭源堡了。最多休整一天,就会先一步翻越鸟鼠山。不知大来谷要道,吐蕃人有没有堵上。如果木征和瞎吴叱弃其不守,接下来苗授他们就会照计划直扑临洮。

‘希望一切顺利吧。’韩冈盼望着,转身返回城中。

王韶领军出征,高遵裕也同样随军出征,苗授、刘昌祚和姚兕等将领都向西去了。蔡延庆昨天走了,回了秦州。现在的陇西县城【古渭寨】中,除了韩冈,就只有秦凤转运判官蔡曚则留了下来。

高挂在澄清天空中的太阳,虽无夏日的炽烈,但照到人身上也是暖意盎然。前天的一场小雪仿佛并不存在,连土皮都没打湿,转眼就云破日出,消失无踪。只是通远这里的气温已经很明显的下降了,一旦站在背阴处,就能感受到一股股寒气透体而来。

韩冈进城后,先是去仓库看了收下来有一阵子的棉花。前段时间他的心思都放在粮食和草料上,并没有注意军中保暖防寒上的问题。现在稍有空闲,感觉还是先去看一看比较好。

在仓库中看到了收获下来的棉花,韩冈终于发觉自己对农业的认识实在太少了,对农产品的加工也不甚了了,所以对于棉花收获重量的理解,跟韩千六完全不同,两人说的根本不是一回事。

所谓亩产七八十斤,那都是连着棉籽和棉桃外皮的份量。去了这两样无用的累赘后,得到的棉絮就可怜得很了——好像加工后的产品叫做皮棉,加工前的称作籽棉,韩冈也不清楚自己的记忆是否准确——一亩地出产的皮棉,也就几床厚被的份量。就算一点不差的织成布匹,也就五六匹棉布的样子。

韩冈可以确定,第一次种植棉花的成果,用最温和的评价也只能说是初步成功。看着堆成了草垛一般的肮脏不堪的棉绒,韩冈心里暗愁自己对自己提议的这项经济作物实在太过忽视了。

尚幸干净的棉绒用来骗骗商人还是没问题的——棉布的价格此时并不便宜,就算是低档的吉贝布,也能卖个三四贯。在西北,一亩地的出产能值五六贯就已经是很丰厚了。但现在的棉花产量,即便只算纯利都能有五六贯——只要能把棉花纺织成布就行。

韩冈叫来了仓库的主管,让他找人把这些棉花都清洗干净。韩冈脸色不渝,便没人敢推脱。只是一声令下,立刻有人把脏兮兮的棉绒一批批的拿了出去。

韩千六这时不知从哪里听到消息,也赶了过来,这本是他的差事。韩冈却是不敢责怪自己的父亲,遂详细问起了棉花的事。

韩千六絮絮叨叨的说了一通,便抱怨起来:“这棉花什么都好,就是去籽麻烦。前面刚收下来时,不知费了多少人工,才让人找到了办法,但还是耗费人力。”

“这事孩儿会想办法的。”

韩冈依稀记得有种叫轧花机还是轧棉机的机器,能够直接把棉花中的棉籽给去掉,他在老电影里看过,还是用脚踩的。机器好办,这个时代的能工巧匠不少,提供大概的构思,给出悬赏,很快就能得到回话——这是他在让人打造霹雳砲时得来的经验。

通远军也有一个匠作营,原本的用处是修理兵器。韩冈前日参观过匠作营后,就有心用水力代替人力的捶打。已经请了王韶向上申请,从几大瓷窑产地选派一名为瓷土坊制作水力冲锤的工匠来。此时瓷器的原料瓷土,基本上都是将瓷土石用水力冲锤粉碎后,加以漂洗沉淀而得来。

能将石头砸碎,用来锻打铁器就不会有问题,水力冲锤叠层锻打出来的兵器也绝不会比那些名工锻造的器物要差。说不定现在京城里卖得死贵的倭刀,这里也能山寨几把出来。

随着棉绒一点点的被搬运出去,放在后面的一筐筐棉籽也露了出来。韩冈走过去,拈起了几颗棉籽来看着。

只是看到儿子拿起棉籽,韩千六却连忙叫起:“三哥小心点,这棉籽好像有毒!前两天有个小子偷吃了,上吐下泻,肚子疼了一夜。最后没法子,把他送到疗养院里去了。”

“有这事?”韩冈惊讶得,回头问道,“现在人呢?”

“不是三哥你让他回家休养去了?就罚了半个月的俸。”韩千六疑惑的说着,“他爹娘都来叩头了,说三哥你人好,救了命还减了责罚。”

韩千六这么一说,韩冈仔细回想,却像不来是不是有这么一回事。不过他一天要批阅的公文得按斤来计算,大事禀报给王韶、高遵裕,而琐碎小事都是他和王厚来处理,哪里还能记得一个月前的批文。当然,这点小事他也不会放在心上就是了。

“偷吃种子是自己做死,怨不得他人,没有死是命好而已。至于救他,那是孩儿的本份,谢不谢由他。只盼他日后能循规蹈矩,不要再做蠢事。”韩冈又想了一下,“得把棉籽有毒的事宣传一下,不能让人再犯蠢。”

棉花的事一时说不清楚,蔡曚的一个贴身亲信却找了过来,“韩机宜,运判现在正在衙门等着,命你赶快过去。”

“‘命’我过去?”韩冈反问着,对第一个字加强了重音。

蔡曚的随从似无所知的点了点头,催促着,“还望机宜不要耽搁了。”

韩冈心头一下火起,可转又按捺了下来,现在不是闹脾气的时候,“你回去与蔡运判说,我即刻便到。”

韩冈现在的身份是随军转运使,这个临时差遣是为了让人管理出战大军后勤补给的任务而设立。

如今以粮草为主的各项军用物资正源源不断的从秦州运来,接着就要从陇西县城运往前线的集结地渭源,再从渭源运抵真正的前线。随军转运使的职责就是把运来通远军,抵达了陇西县城的物资送到前线将士们的手中。

韩冈希望能把囤积在城中的粮草尽快运往渭源。之前在主力还未到达的时候,王韶和韩冈都不敢将辎重堆积在前线。若是变成了党项人守罗兀时的情况,被吐蕃人偷袭下渭源堡,要掉一批人脑袋的。

所谓三军未动,粮草先行,可不是说把辎重队伍放到大军前面打头阵。仅仅在军队行动之间,要提前准备好粮草。而王厚之前押送粮秣去渭源堡,其数量也仅占今次总量的十分之一不到,只是为了大军抵达渭源后不饿肚子而准备的。

等到王韶率领的主力开始翻越鸟鼠山,韩冈就要前往渭源堡,同时也要把随军转运衙门转移过去,而不是了留在后方的城中。至于陇西城中事,则是交给另外一人处置。

韩冈赶到衙门的时候,正冷着脸等着他的秦凤转运判官蔡曚,他的临时差遣也是随军转运使,与韩冈司掌同一职位,也就是计划中当韩冈去渭源后,接下陇西事务的那人。

两人同掌一职,不论是过去还是现在都是很正常的情况。一场出动上万战力的会战,各方被征调的人力数量更是数倍于此,不可能只让一名选人来管理后勤。一般都至少是朝官,以韩冈的身份能被任命为随军转运使,已经是个异数。

蔡曚很明显不喜欢韩冈这个异数。“韩机宜”,他的口气还是一向的冷淡,“不知为何耽搁了?”

“方才去仓库检查冬料了,这天说冷就冷,还是先预备着。”韩冈不喜欢有人跟他分权,尤其是很不友善的蔡曚。但他还是保持着礼貌,他并不想给前线添乱。“不知运判有何指教?”他和声问道。

见韩冈似是低头,蔡曚微微冷笑。回身坐了下去,态度高慢的问着韩冈,“第一批向渭源运送粮秣的车队准备好了没有?”

蔡曚不知好歹,韩冈眼神彻底冷了下来,硬邦邦的回道:“此是韩冈份内事,运判勿须操心。”面对暴怒而起的蔡曚,韩冈微扬起头,仗着身量,居高临下,“运判如果疑虑,还请去看看今次的诏书。我俩的姓名孰前孰后?!”

话不投机半句多,韩冈拂袖而去,改去检查明天清早就要出发的辎重队。

没过片刻,已经被韩冈荐到衙门里做事的李小六匆匆跑来,气急败坏:“机宜,蔡运判又在闹了,说是要查过去一年的旧档!”

韩冈正在检查要去渭源的车马,信口道:“别去理他就是。”

“可……可蔡运判他……”李小六吞吞吐吐。蔡曚在衙门里蹦得正欢,以他的身份衙门里的胥吏谁敢不从?

“我不是说了吗……”韩冈冷如寒水的眼神和口吻,明明白白的向李小六传递出他真实的心意,“别去理他!”

李小六毕竟跟着韩冈日久,一下恍然大悟。对韩冈的吩咐心领神会,低头答诺:“小的明白,我们……不去理他。”

