\section{第24章 缭垣斜压紫云低(五)}

“知道了。”韩冈打发来送信的家人出去,“回去跟夫人说,让她带着大哥儿大姐儿他们先去城南驿见岳父。等放衙之后,我就直接过去。”

家丁领了命,就匆匆出去了。

虽说是之前猜错了,不过韩冈也懒得再多想。王安石赶在祭天大典之前抵达京城,究竟是因为没有考虑太多,只是按着预定的行程走,还是因为还想在政坛上有一番作为,见到人之后就能知道了。

这时候,皇帝陛下应该已经在斋戒沐浴了,但他绝不会将王安石丢在城南驿,明天必然要召其越次入对——这是必须给老臣的体面。但到底要不要让王安石参加郊祀,可是能让赵顼头疼死。

韩冈暗暗笑了笑,倒是有些幸灾乐祸的味道。

“王介甫终于是又回来了。”苏颂的声音中多了几分感慨。

此时在编修局内,苏颂就坐在旁边,还有几名编修同在厅中。王安石入京反正也不是什么需要保密的事,直接就是在正厅里说了。

正是王安石所推动的变法,大宋才有了如今的气象,当年苏颂一力反对新法,如今看来,已经不可能再坚持过去的观点了。如今王安石出外数年后回返京城,到底会给朝廷带来什么样的变局,是留在朝中,还是依从诏令去相州,都是让人无法不去关心在意的。

“本以为还有几天功夫呢。”韩冈笑说道,“没想到会这么快。”

“王相公没有先一步遣人入京?”一名编修惊讶的问道。

韩冈微微一愣,这倒是个好问题。过去家人尚在京城,他每次回京都会先行遣人通知,按道理王安石也该遣下人知会自己这个女婿一声,也好做些准备,出城迎接才是。

难道当真是想打朝廷一个措手不及?韩冈不免有着这样的猜测。

……………………

“人是派了,谁料到在路上出了事啊。”王旁笑着向韩冈解释道。

在放衙之后,韩冈便依言赶往城南驿。本以为此时的城南驿应该不会太热闹,绝大多数官员应该等到天子作出决定后才会赶上门来请安。可出乎意料的,今夜的城南驿却是人满为患,不知多少官员和士子想见上王安石一面。

王安石为此高挂免战牌,声称旅途劳顿,不便见客,将所有人都拒之门外。也就韩冈,仗着自己的女婿的身份,还算轻松的穿过了人群,进到了内院的一座独门小院。

许久不见的王安石精神矍铄,但的确是老了许多,头发白得更多,皱纹也更深了几分。只顾着跟外孙和外孙女们说话,笑得很是开怀。王旁和王旖则就在旁边说着话,见到韩冈便连忙迎上来。

一家人见过礼,畅叙了一番离情,韩冈便半带抱怨的笑问着为什么不事先通报一声,也好有所准备。王旁精神旺健,也富态了不少,看起来这几年管粮料院的日子过得不算坏,几句解释,倒是结开了韩冈的困惑。

按王旁的说法,他们一行人到了南京应天府【商丘】之后,就派了人先行赶来京城,孰料那人在快到陈留的时候出了意外,受了不轻的伤——王安石和王旁他们还是经过陈留的驿站时才知道此事——受伤的那名家丁因伤势的关系,不便继续上路,所以现在还留在陈留县中。依靠王安石的面子,被安置在新开的陈留医院中接受医治。

正逗着外孙们说话的王安石这时抬头来,“多亏了玉昆你的医院,什么病都能治,要不然也只能就在当地去找擅长跌打的杏林高手了。”

“也幸好是在陈留。”韩冈说道,“如今的医院,除了东京城中的两家外,开封府内只有陈留、管城和白马三县建了医院。等到了明年,才会轮到北京、南京和西京。”

“稳定一点也好。”王安石点头道,“玉昆,你接下来是不是准备在全国各地设立医院?”

“不,小婿最多也只打算每一路设一座医院。毕竟是官办的医院人数有限,替代不了民间的医馆。而且一旦全数转成官办,恐怕就成了官宦子弟除荫补外另一个求官的出路了。”韩冈笑容冷冽,官僚们的德行古今中外从不会变,“伎术官转正官总比其他手段要容易一点,未免就有失钻研医术的初衷。在小婿看来,官民两方都不能少。”

韩冈只打算建立数目不多的医院。更多的还是维持现在负责一片的家庭医生,为区域内的普通人家提供日常的诊疗服务,此外再有专科诊所则负责一些专项的病症——比如牙医,稳婆什么的。在韩冈的构想中,这个时代的医院,其存在的主要意义,应该是以培养医师,进行医学研究,负责灾害时的紧急救治,而不是垄断医学。

在王安石的示意下,王旖领着几个小孩子去后面翻看礼物去了。可惜这一次王安石上京,并非留任京城,韩冈的岳母还有王旁的妻儿都仍是留在了金陵城中,否则也能有个作陪的。

等他们离开,王安石笑容微微收敛,眼神也变得犀利起来:“玉昆,有件事我一直都想问一问。”

“请岳父明示。”王安石要问什么,韩冈心知肚明。

“我在金陵听传闻,殷墟甲骨是你编纂药典时才碰巧发现的。这个传闻,应该不是真的吧。”

“不敢瞒岳父,在收到岳父的《字说》后,小婿就立刻遣人打着采药的名义去了安阳。”韩冈微微笑道,有些事再坚持谎言可就要生分了,王安石也不是好骗的人,“对《字说》的看法,小婿已经在给岳父你的信中写明了。但若是没有证据,一切都是空谈。既然小婿说格物致知,当然就不能用嘴皮子来证明,或是打笔墨官司,这样是争不出个对错来,谁都不会服气……就是断案,也得讲究个人证物证俱全,这样才能让人犯伏法不是?”

“玉昆就这么有把握?”

以采药的名义去动手,绝不是动动嘴、派个人那么简单。土石矿物是药类的一大分支,譬如丹砂、雄黄,都是每家药铺都少不了的重要药材,但派人去殷墟刨坑,一旦被抓个正着,用采药做理由可没人会信。想也知道韩冈到底是冒了多大的风险。

而更重要的,土里寻宝这等事纯属运气,哪里能够心想事成。以韩冈的为人,怎么会将自己命运放在运气上?王安石很难相信这样的说辞。

“没有洹水之南的殷墟,还有岐山之下周原。只要有几件证物就足够了。”韩冈笑着说。

“周原?!”王旁都忍不住一声惊叫。

“正是周原。”韩冈说道。

王安石摇摇头,他都不知道该说什么好。周王朝起家的周原,周文王先祖古公檀父率领族人安居下来的地方,也真亏韩冈敢去挖。

“这也太冒险了。”王安石道。

“不过也算是运气,一开始小婿想要的是各式带铭文的礼器和冥器,只要花钱,总能在当地人手中买到,就是急切间没有现货,也能雇请当地人去想办法。”

王安石的询问,韩冈当然不可能说实话。也没有藏头去尾,掩去部分真相,说些让人误会的所谓‘实话’——尽管这是韩冈最常做的;而是直截了当的就说了谎。用谎言替代谎言。

不是他不信任王安石父子的人品,而是多一个知道底细,就多一分危险。只有一个人知道的才叫秘密,两人以上,谁知道什么时候会一不留心给暴露出去?

“只是对外得有个说得过去的名义。”韩冈继续说着,“丹砂、雄黄都是山中所产,矿坑里挖出来的。从平地里掘不出矿,唯有龙骨,所以让人打了收购龙骨的招牌。谁能想到这龙骨,偏偏就是关键。”韩冈微微一笑,“也算是运气了。除了时间上有参差,其余的事基本上都是事实。”

王安石脸色微沉,有关运气的说法,他在《字说》的序文中曾经提到过——‘天之将兴斯文也,而以余赞其始’。韩冈的话,听起来就是像是在针锋相对。

插不上嘴的王旁在旁边有点着急,左看看,右看看,不知怎么开口调解。

看着微笑中却眼神坚定的女婿,王安石心中叹了一口气,大道之争,本来就不是讲人情的地方。

气氛正尴尬的时候,一名家丁几乎是小跑着从前院窜了过来,脸上慌慌张张的,在跨进门中的时候,莫名其妙的就被门槛绊了一下,一头栽进了厅中。

“吴平,你这是什么样子?”王旁大感丢脸,厉声向拼命想要爬起来的家丁质问着。

吴平手忙脚乱的爬起来,捂着痛处结结巴巴的说着:“相公,二郎,宫……宫里面来人了,说是官……官……官……官家就要到了。”

官家就要到了?

这话听在耳中,却没人能立刻反应过来。就是韩冈也是先是一愣,等一下明白过来之后便立刻起身。

这跟韩冈当年入京的情况对比鲜明。赵顼对韩冈可以放在一边晾着三五日、七八日,半个月都可以的,但对王安石却决不能这么做——新法还在,慢待王安石,免不了会被人误会要改易新法了——今天遣使慰问,明日招入宫中,这是韩冈预先猜测的。但赵顼亲自出宫驾临驿馆,这份礼数,却是怎么也不可能猜得到。

转头发现王旁还愣着神,而王安石则已经是一脸激动的站起来。赵顼如此待他,传到后世可算是君臣知遇的典范了。

紧跟着那吴平,一名身穿紫服的内侍也进来了,却是大家都熟悉的石得一。没人敢拦的这位大貂珰亦是差点被门槛绊倒。进厅刚站稳,也慌急慌忙的冲着王安石道:“相公,官家就要到了。”转脸看见韩冈,也只是匆匆招呼了一句“端明也来啦”便径自扶着王安石就出了厅去迎驾。

韩冈扯了一下还愣着的王旁,对听到外面动静赶出来的王旖吩咐了一句,便一同出门,在城南驿的正门前,与驿馆中的近百官员一同向着已经御驾亲临的赵顼行礼问安。

如此宠遇,让王安石复相的流言在一夜之间传遍了京城。

