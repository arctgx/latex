\section{第11章 立雪程门外(上)}

接近腊月的时候,洛阳城断断续续的下了七八天的雪,至今未有停歇的意思。

雪一直不算大,但聚沙成塔,不知不觉间也积了有近两尺厚。雪花还在飘落,天地皆白,将洛阳城中的老屋古庙都妆点一新。

程家院中的几株腊梅这时也开了花,淡雅香气沉浮于素洁的冰天雪地之中。浅黄色的花朵,褐色的树枝,被细雪染成纯白,玉树琼花一般。

程颢虽然任的算是闲职,但西京竹木务在大雪之后,还是有些事务要处理,大清早便除了门去。程颐则照着往常的时间起床,先去问候了父亲,然后也如平日一般,回到书房中去读书。从微敞的窗户外,飘进来一丝半缕的腊梅清香,却省了焚香这一事。

只是程颐沉浸在书中没有多久,家中的一名老仆便送了一封信来,后面还附带着一份门状。

程颐先拿过信。信封的抬头上写着伯温表兄并伯淳、正叔二侄,是张载的亲笔。

一封信厚厚的,从开口处看进去,塞在里面的信函竟然有十几页。程颐一见到这封信的厚度,知道里面肯定是有着张载最近的研究成果。也不顾其他,抽出信便看了起来。

不知过了多久,桌上的一杯热茶已经都不冒热气,程颐才摇着头,将张载的信放了下来。

这封信中,除了问候之外,的确说了很多关于格物方面的见解。有形而上的道,也有形而下的器。张载在格物致知的方面的确走得远了,虽然信中说的以实证道的做法不算错,但终有难以验证的时候。而且关学之中天道与人道之间的割裂现象,也越发的严重了起来。

‘终究还是难近大道。’

放下信,程颐这才拿起门状。题头是末学晚生,后面缀的名字则让程颐也不由得一怔,竟然是韩冈。

不过想想也是,韩冈要上京应考,以自家的兄长对他的看重,依礼数,现在经过洛阳时,也该来拜会一下。不书官职,只道晚生,这一项让程颐很是舒服。拿过纸张,提笔写了几句,便折了起来递给一直等着一旁的老仆,

“拿出去,让来人回复其主,早有通家之好,直接上门来便是。”

老仆犹豫了一下,并没有接下来。

“怎么了?”程颐手一顿。

老仆低头,“送信来的秀才就在门外等着!”

“什么?”程颐面现讶色,一下便站起了身。

以官位来说,韩冈已经在程颢之上。程家的老父做了几十年官,磨勘多少任,才一个正五品,也只有去世后,才有资格一触四品的门径。而韩冈这样的官品,不但亲自上门送信,甚至就候在门外等回音,这个礼数就重了。

士大夫之间的正常拜会,除非已是通家之好,要不然都是先派人送上一份名帖来。如果主人愿意相见,便落书约好时间。如果不见,也会在回书上找个理由。但这一段文字往返,基本上都是仆人奔走,这也是让双方之间有个转圜的余地。

而现在韩冈的做法,却是晚辈拜见长辈,下官拜会长官时的礼数,容不得程颐不惊讶。

“快请门外的韩官人进来。”

‘官人?’程家老仆得了命,便转身往外走,心中有着几分疑惑:‘穿着秀才的衣服,又站在门外等着消息,怎么可能会是官人?’

但他知道自家的主人用词一向精当,有官身的人才会叫做官人。而不是像市井中那般,就是个普通富户,都能道他一声员外。天知道,朝中能混到正七品员外郎的有多难。

让一名官人在下雪天候在门外,想到这里,老仆心中益发不安,连忙快了两步。

……………………

天上的雪一直不停,雪花不住的累积,系马桩下守着的伴当不耐烦的来回走着,而韩冈仍是心平气和的等在程府门外。

自在雪中辞别了身负皇命的程昉之后,韩冈和种建中继续前往京城。

雪地里走得虽是艰难,但还算是顺顺当当的到了洛阳。在驿馆中落了脚,种建中要去拜访洛阳城的亲友。而韩冈则带着张载给表兄表侄的家书,在洛阳找上了程家的门。

离开横渠镇前,张载给了韩冈几封信。第一封是给在周至县监竹木务的弟弟张戬。第二封,便是给在洛阳任着跟张戬一样的职务,同样跟竹子脱不清干系的程颢,以及其父程珦和程颐。

自从在京城中在程颢那里聆听教诲之后,韩冈也会给程颢写信,只是不及给张载的那般频繁。他前两年几次经过洛阳,但程颢在外任官,而程颐则跟着在蜀地治事的程家老父程珦,登门拜访也见不到人。直到今年,程珦致仕归乡,程颐跟着回来。而程颢也上书在洛阳要了一个清闲一点的差遣。

既然程颢已经回来了,旧日多承其情,韩冈路过洛阳时,总是要拜见的,何况张载还托付了顺道送信的任务。

只是程家这看门的老仆一进去,就没个回音,韩冈默默地等着,头上肩上都落了满雪。路边经过的行人车马,看着程家门前的韩冈,指指点点,惊讶万分。伴当来劝过几次,韩冈却始终无意离开。既然已经在等了,就该等到底,半途而废才是要不得的。

一匹马踩着雪行了过来,在程家门前停下。骑手翻身下马,也惊疑不定的望了韩冈好几眼。

就在此时,程家的偏门给打开了。骑手一见门开,就两步上前,笑道:“正是巧了,还想敲门呢。六丈,小子今日奉我家主人命,送请帖来了。”

“是尧夫先生的请帖?”程家老仆问了一句,就急急的对骑手道:“你且稍等。”

丢下送请帖的熟人,老头子忙跑到韩冈这边。看着头上肩上全是积雪的韩冈,诚惶诚恐的致歉:“官人勿怪,官人勿怪,小人多有得罪,让官人久候了。”

韩冈笑了笑,身子一动,积雪纷纷而落:“伯淳先生与我有半师之谊,在门外候着也是礼数。”

程家老仆让出了路,“我家主人有请官人,还请入内一叙。”

韩冈被领着走进程家家门,他的伴当便捧着礼物跟了上来,与那名骑手擦肩而过。

洛阳城中的尧夫先生,自然只有一个。邵雍邵尧夫,也是如今的当世大儒,学术兼及儒道,太极之说,更是上承陈抟老祖。不过他更为有名的是算命点穴的本事。邵雍在洛阳城中的宅邸‘安乐窝’,便是靠着帮仁宗朝的状元王拱辰的父母点了吉穴挣来了。而前两年,司马光和富弼更是将安乐窝原属于官产的地皮给买下来,赠与了邵雍。

韩冈对于邵雍的了解也就这些了,除此之外,就是那句流传很广,听起来别有深意的‘天根月窟闲来往,三十六宫都是春’了。

韩冈进来后,邵家的仆人也被领进门来。不算大的府第,四人前后走着。

程颢今日不在,一开始送信时就已经知道了。即将面对二程中的另外一位,韩冈也有些期待。张载对程颢的评价是在程颐之上的,但好歹也是程朱中的一人,理学的开创者之一。何况,还在后世也鼎鼎有名的那一句。

‘饿死事小,失节事大’。

如果这句话不是针对妇女,而是说着士人,那倒真是很有道理,也值得敬佩。不过韩冈现在并没有听说程颐有说过这句话,在张载和张戬的面前,也不便去打听。

走进了程家的客厅,终于见到程颐。

与温文尔雅,交谈起来让人如沐春风的程颢截然不同,韩冈面前的程颐,神态沉严肃重,动作也是一板一眼,不打半分折扣。并没有因为韩冈在门外雪中等候了一个多时辰,而让他外在的态度有半分变动,只是眼中的欣赏却是没有掩饰。

这位当世大儒,日后先是流芳青史,继而又遗臭百年的颍川先生。给韩冈的第一印象,就仿佛是一部《礼经》变成了活人,在他面前教演着什么才是正确的见客礼仪。站定,回礼,问候,甚至连点头弓腰的角度,都是恰如其分的符合了他与韩冈之间的关系。

与韩冈见礼后,程颐又依着标准的礼节向他告了罪,然后才从邵家仆人手上接过请帖。

邵雍使人送贴来,但言安乐窝中腊梅花开,拟与三日后设宴,邀请二程前来赴会。

“且去回复贵主:承蒙尧夫不弃,乃至书相邀。程颐感念盛情,自当与会。不过家兄今日往城北本司公干,且等家兄回返,再遣人回复贵主。”

这一过程中,程颐对邵家的仆人并不假以辞色,而邵家的仆人进门时没看到程颢,也是神态明显的变得拘束起来。

韩冈一切都看在眼里。

看来张戬果然说得没错。说起二程与邵雍这位洛阳城的另一位大儒的关系,的确是有些微妙。程颢还好,对什么人都能平和相待,就算政见不合的王安石,也没有闹到翻脸的地步,与邵雍更是能相互和诗。但程颐,就跟邵雍邵尧夫不怎么亲近了。

