\section{第18章 秉烛待旦已忘眠(下)}

“占着又如何,夺回来就是!”韩冈叉腰挥手,说得豪气干云,王厚、王舜臣在一边鼓掌叫好。

“兰州要隘,向西可通西域,向南压制青唐,向东则屏蔽秦州,向北便能直捣党项软肋!此兵家必争之地。一旦据有此处,西贼不放上三五万人来戍守,梁太后怕是连觉也睡不好!但西贼总共才多少兵?”韩冈说道这里,却又不将话题接下去说,转而一脸神往之色,道:“兰州就在黄河之滨,那一段河道跌宕起伏,峡谷幽深连绵不断,据说其景壮丽处不在壶口、龙门之下,几与三峡媲美。”

王厚连连点头,任凭韩冈把话题飞来荡去。他的心思尽陷在韩冈的话里,全都忘了来此的目的。不停口的赞着韩冈:“秀才果然是博学多闻。”

韩冈笑道:“书生不出门,能知天下事。知一晓二,举一反三,这也是要靠读书得来。韩某不是死读书的,某少小离家求学,从秦州走到京兆府,为了追随横渠先生,又走回渭州。别的地方不能自夸,至少关西韩某还是了若指掌。”

王厚正色改容,恭敬道:“不愧是横渠门下。”

韩冈郑重点头:“若无子厚先生悉心教导,便无今日韩冈。”

韩冈此言,真心诚意,发自肺腑。他继承自旧主的满肚子的经书和文章,以及熟极而流的兵书、地理,都是来自张载的教导。

横渠门下,学得不仅仅是儒家经典,还包括天文地理,兵法水利——若以为宋儒都只知‘之乎者也’,那就大错特错——尤其是兵法和地理,更是张载讲学的重点。

张载年轻时,曾经上书范仲淹,愿与乡中豪杰一起去收复青唐旧地,后为范仲淹所劝,方才弃武从文。十几年后,张载考上了进士,同时开始授徒讲学。可即便如此,张载对军事上的认识仍然得到了泾原路经略安抚使、知渭州事【注1】蔡挺的看重——

韩冈想到这里,突然灵光一闪,终于想起了究竟在哪里听说过王韶的名字!

张载曾任渭州军事判官,最为蔡挺器重。他在渭州,一边教导学生,一边帮助蔡挺整顿军队编制,清查空额。就在去年,还听说张载正帮着蔡挺修改规范范仲淹创立的将兵法。而韩冈回来前,又听闻如今蔡相公推行将兵法的效果很好,得到了朝廷的重视,尤其是想要富国强兵的年轻官家以及一力辅佐他中兴大宋的王相公,都很看好这一整编地方军队,提高战力和指挥效率的新规条。

而当时在蔡挺身边,还有一名门客深得看重。他也是进士出身,而且与张载同为嘉佑二年丁酉科【西元1057年】——也就是俗称的同年——不过与张载不同,他因参加比进士科举还要高一级的制举考试落榜,便放弃了官职,转而跟随蔡挺来到陕西,并游历关西各州,还与张载讨论过当年他收复青唐的计划。张载曾对学生们说其有班马之志,欲效班定远【班超】、马伏波【马援】,远行万里,扬汉家天威。他的姓名——正是王韶!

与王厚言谈甚欢,韩冈自觉到了探底的时间,便问道:“不知经略司的王机宜……”

韩冈话还没有说完,王厚就道:“正是家严!”

脸上浮出一丝微不可察的笑意,韩冈道,“据闻令尊意欲吞并青唐,开边河湟,说起来,此正是吾辈之愿,也是家师毕生夙愿。令尊若真能成事,不但功业不让班、马专美于前,可为国朝平定北汉之后第一功;只秦州数十万百姓,亦要深感令尊之恩德。”

“西贼虎视眈眈,吐蕃悖逆雄强,不得豪杰智士相助,却难以成事……韩兄天纵奇才,眼界见识远胜凡庸,不知能否助家严一臂之力,以解乡里之苦。日后博个封妻荫子,亦可不再受小人之欺。”王厚目光灼灼的盯着韩冈,只等他回应。

韩冈笑而不答,也不想答。他当然愿意,可王厚只是衙内,并不是王韶本人,他的邀请不得王韶认同就毫无意义。韩冈希望得到的是王韶的礼聘,而不是他儿子的邀请。

王厚愣了一下,正想再劝,但看着韩冈脸上浅浅的笑容,突的恍然大悟。终于明白,这话应该由他父亲来说才是。他改口道:“若明日韩兄有闲,可否往城衙一行,王厚必翘首以待。”

“城衙?”韩冈摇头笑道,“今天已经去过一次了,明天再去,不知会不会给赶出来。”

“难道是要求见家父?!”

“不,是韩某有紧急军情要上报,不过就是没人搭理。”韩冈说完轻叹,似是痛心不已的模样。

“什么军情?”王厚问道。

“韩某奉命押送军资自秦州往甘谷。今日午后,在裴峡中,遭逢近百蕃贼拦截。虽被我等杀散,但通往秦州的要道上出现了蕃贼拦路。可不是什么好兆头!”韩冈指了指王舜臣在衣袍下微微隆起的左侧肩膀,“王军将的肩上就是中了一箭,但即便中了一箭,王军将可是照样一张弓就射死了十一人,门外车上的三十一颗首级,有三分之一是王军将的战果。”

“射杀十一人?”王厚惊异看了王舜臣一眼,没想到他勇悍如此。又急急追问:“斩首总计三十一,那缴获呢?!”

“三十四张弓,刀枪四十一件,盔甲一领。”韩冈如数家珍,要想取信于人,细节问题是半点也不能差的。

有缴获、有斩首,韩冈之言自是千真万确无疑。“百名贼人战死了三成才败退,果然是场恶战。”王厚点着头,有着王韶这个父亲,王厚对战事还是有所了解,清楚一场战斗的伤亡率是多少,他又问道:“不知韩兄这边伤亡如何?”

“连上在下和王军将,总计四十一人。八人受伤,无人战死。”

“啊……”王厚惊叹,“竟无损一人!”

韩冈摇摇头:“还是损了两个!”他对王厚解释道:“这两人意欲临阵脱逃,又出言动摇军心,给韩某亲手杀了,当算不得战死。”

王厚这下比方才还要震惊,能亲手杀人的书生可不多见,韩冈还说得如吃饭喝水一般轻松。但联想起韩冈在街市上箭射向荣贵的事,却也不会有假。

王厚正少年,韩冈的作为正对了他的脾性,看向韩冈的眼神充满崇拜,对他佩服得五体投地。站起身,王厚双手举碗,敬向韩冈:“韩兄果然是关西男儿!当浮一大白!”

韩冈豪爽的与他对饮而尽,放下碗,对视一笑。浊酒亦能醉人,一股豪气自王厚心中油然而生,只觉得今夜结识的这位韩秀才,真是当世英豪。

韩冈这时拍着王舜臣的肩头:“说起来,这一仗最大的功劳还是王军将!韩某只是安内,王军将可是攘外。当时我等被贼人两面夹击,正是王军将独当一面,箭无虚发,将迎面而来的贼军射得魂飞魄散!如非王军将,韩某今夜也无法安坐在此!”

王厚再仔仔细细的把王舜臣上下一打量,连声赞道:“果然是一员枭将。”抬手又敬了王舜臣一碗。

王舜臣得意得胡子根根翘起,忙端起酒碗回应,嘴里则装模作样的谦虚道:“过奖!过奖!哪里!哪里!”

敬过了王舜臣,王厚又斟满一碗酒,转过来对赵隆道:“赵敢勇的斩获亦当不少,也当满饮一碗!”

赵隆这下子臊得脸皮通红,低声嗫嚅道:“不……俺只是一个守城的。”

韩冈帮赵隆化解尴尬,道:“赵敢勇论武艺,也不让王军将。只是运气不好,得罪了上官。方才被罚守城。明珠蒙尘,实在可惜。”

赵隆感动至极,眼眶都红了,几乎要哭了出来,直把才认识了不到半天的韩冈,当作平生最大的知己。

王厚则暗暗点头,逼着赵隆喝了酒,又把他的名字给记了下来。

众人重新坐下,韩冈又道:“裴峡是要道,就在伏羌城边。现在出了贼寇,却无人放在心上。韩某想求见副城,却被告知须接待上官……”

王厚一听,却是牵连到了自家老子头上,忙赔笑着解释道:“若是刘城主在,也不会有这事。只是李副城求进心切,摆了宴席去请家严。被家严拒了,正生着闷气,当然不想理事。”

“军国大事啊……”韩冈摇头叹着,“若关西将佐尽如此辈,何时才能扫平西贼。”

“不说这些烦心事,先喝酒!喝酒!”王舜臣举杯邀饮,三人轰然应诺,一起开怀对饮。

借着酒兴,韩冈与王厚继续谈天说地,纵论古今,而王舜臣和赵隆在旁边搭着话,也不觉烦闷。

四人一番醉饮,不知屋外斗转星移,直到雄鸡三唱,天色发白。

注1:泾原路经略安抚司治所位于渭州,而不是处于前线的泾州、原州。所以兼任泾原路经略使的是渭州知州。这一点,与治所秦州的秦凤路不同。

ps:开拓河湟的国策,从神宗初年,一直持续到徽宗时期。其间虽有反复,但却是升官发财的快速通道。只看童贯,他发迹的地点便是这里。

今天第三更,继续征集红票,收藏。

