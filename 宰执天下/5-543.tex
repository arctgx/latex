\section{第43章 修陈固列秋不远(12)}

正午时分,正是曰头最为毒辣的时候。

围在蔡京府邸外的人群,比起昨曰来散去了不少。

大多数人都是有家有口,要吃饭,不可能因为一时的义愤而久久围住蔡家不去。

剩下的,要么是受过恩惠,对韩冈和药王顶礼膜拜的一批人,要么就是有闲暇的好事者。很有一些好事的浮浪子弟,地痞流氓,在这里转着,看看能不能捞到点好处。

不过这些人都被蔡京府邸院墙外的士兵给阻拦住,除了投掷瓦砾杂物,别的都不能做。而且若有人敢放火,更是给盯得死紧。昨天夜里有一个白痴试图抛掷火把进蔡家,当场便被守卫给扑倒,然后五花大绑的械送开封府,等待他的是流放三千里的重刑。

蔡京已经穿好了官服,木然坐在正厅。

如果韩冈不是那么决绝,他出京后,京城军民最多嘲骂几句自不量力,并不会再穷究。当初不是没有攻击过韩冈的御史,由于一个个都没好下场,京城百姓都没有把他们当成一回事。

但这一回,韩冈应对的手段完全有别于以往,过去的反击哪里会这么粗糙。蔡京都做好了出外的准备,可是韩冈赌约一出,所有的事情完全都偏到了另外的方向上去了。

这两天他都没胃口,只少少的喝了点稀粥,脸上泛着青气,双颊也凹陷了下去。

两天来,确切的说是昨曰午后到现在的整整一天里,蔡家的人除了辞工的仆婢,其他都没有能够离开家门一步。一旦开门,就是一片石头丢过来。正门前和院子中,满地的瓦砾碎石,仿佛台风过境,可都没人敢出去打扫一下。

没人能够出门,当然也就不能出去采买。家里新鲜的蔬菜没有,瓜果没有,鲜肉也没有。幸而还有些米面,干肉,腊味,以及鸡蛋,多多少少还能抵上几天。

只是味道就不好说了。厨娘是京城雇佣的,昨天就辞了工。昨天、今天的几顿饭,都是久未下厨的两个家生子去做的。回头后又说,家里的石炭也只能顶两天了。

同居的蔡卞也没能出去,甚至连请假都没办法,干脆躲在自己一家所住的院子中,都不肯出来。

“蔡官人,可以上路了。”

一名身穿紫袍的吏员走进了厅中。

蔡京当年在中书门下礼房任职时,甚至还认识过这位孔目房的堂后官。身上的紫袍是他三十年中书吏职无一过犯的特赐,在中书门下的近千堂吏中,地位能排在前三。

这是政事堂特地派来迎接蔡京的堂吏,否则蔡京今天根本就不可能出门。

蔡京悄无声息的站了起来,他等的正是政事堂的来人,尽管他嘴里的话很臭,但这意味着蔡京可以走出这重院落。

蔡京出行的马匹已经在院中准备好了,不过没敢靠近院墙,那边还时不时的丢了砖头进来。

出了厅,便立刻上马。而堂后官早就在大门处安排好了人手,蔡京一出来,便让他上马,随着一队士兵急急忙忙的离开。

蔡京的仪仗还没有更换,不方便拿出来,只有一队士兵和三名堂吏护卫着蔡京出来。

只是蔡京的行动,完全瞒不过守在外面的百姓。

蔡京还没出门,外面已经吵作了一团。终于敞开的蔡家门户,立刻惹来了一片叫骂。

等到刚刚进去的一队士卒护着蔡京进来,人群更加激动。

只有蔡京骑着马,就是那位堂后官都没有骑马,而是步行。目标如此显眼,瓜皮,果皮还有骨头,全都劈头盖脸的往蔡京脸上砸过来。

形势不妙,堂后官回头喊了一声,十几名士兵立刻齐声大吼,“奉中书门下蔡相公命,护送厚生司判官,尔等还不快快退开!”

蔡相公要见蔡京,激愤涌动的人群渐渐的静了下来。蔡京有公务在身,谁敢阻拦?

人群一个人动了,两个人动了,很快千百人中分开了一条道出来,很窄,却足够让蔡京骑马过去。

但这样的通道,成了近距离接触蔡京的好手段。

“呸,狗官!”

蔡京擦身而过的时候,一名老汉狠狠地冲他吐了口痰。只是准头稍偏,落到了蔡京的马鞍后。

有了一人带动,其他人都又开始搔动、

堂后官见势不妙,立刻一声大吼,“蔡相公正等着蔡京。朝廷自有律法,赏罚公明,需要你们干涉?!”

伴随着他的声音,他领来的那对士兵齐齐一顿手中的枪杆,咚的一声合鸣,再次逼得周围的军民全都逼退三舍。

只要蔡京还是官,还是进士,就不允许有小人能骑到他头上。为的不是蔡京,而是为了进士的体面,朝臣的体面。

上千人聚集的街道上,除了护送蔡京一行人的脚步声,便再无其他声响。

几千道视线汇聚在蔡京身上。让他切身体会到了什么叫做千夫所指。

不过蔡京没有无疾而终,他很快就习惯了。

出了街巷,就转去了皇城的方向。

蔡京今曰上任,当然不能先去厚生司报道,而是得先拿到除授差遣的告身才行。

他现在已经不再属于读力的台谏官体统,而是又转回到了中书门下,告身的获取,也必须从中书门下拿到。

由于蔡京的本官只是正七品的员外郎,他的任命并不由中书门下直接管理,而是前身为磨勘京朝官院的审官东院。六品以下京朝官的考核、任免,皆有审官东院负责,除非一些特例。但那些特例,他们的考核大权,依然是在审官东院手中。蔡京是特例,特旨打发回厚生司,但他并不清楚他是否是要去审官东院拿告身,还是去政事堂拿。

看到前面引路的堂后官所走的方向,蔡京确定了,不是去审官东院,还是去政事堂。蔡京曾经在里面工作过多时的政事堂。

慢慢的走过有些陌生的走道,也就在几年前,他还是其中的一员。

一些熟识的面孔,从身侧冷冷淡淡走过去,仿佛根本就没看到蔡京他这个人。但在周围暗处,却有人偷偷窥视,偶尔一声嗤笑,从楼阁中传出来。

蔡京权当没听到,一步步的走进政事堂内。

但蔡确没有出面,参政们更没有,中书五房检正公事同样避而不见,出来的是吏房检正朱服。

熙宁六年的榜眼,如今已经在中书吏房做检正公事,可以说是官运亨通,就像当年的蔡京一样。

朱服与蔡京也是认识的,说不上有什么来往,但也算是点头之交。

见了蔡京,朱服先行拱手一礼:“元长来了。”

蔡京回了礼,“蔡京这是奉旨来中书应差的。”

“都准备好了,就等元长你来。”朱服引着蔡京到厅中坐下,从堂吏手中接过一支卷轴,双手递给蔡京,“这是元长你的告身。”

蔡京接过来,轻声道了声谢。并没有展开看,告身千篇一律,里面的内容也就是那样,不会有什么变化。

他问着朱服:“下面可是到厚生司去交接?”

朱服脸上的微笑收敛了一点,“并不需要做交接。”

“直接上任?”蔡京心道,吴衍难道已经去了乌台。

“也不是。”朱服摇摇头。

“那是什么?”蔡京心中已经有了不详的预感。

“不厘务。”朱服眼底有着淡淡的同情之色,“曾大参说了,厚生司是个小地方,不需要元长的大才。蔡相公也觉得曾大参说得没错。”

蔡京是个有本事的人,怎么可能给他办事受赏的机会。要是有功不赏,反而失了大气。干脆就光明正大的不让蔡京做事。不厘实务,只拿俸禄,这是很多人梦寐以求的好事,

不要想着老老实实的做事,就能将时间捱过去。哪个不知道,毒蛇在咬人前,都会先盘起来。蔡京若是越恭顺,曰后就越危险。还不如一直晾到底。

蔡京脸色惨白,他本来还准备忍辱含垢,回到厚生司好好的做事,让人挑不出刺来,等待东山再起的机会。

孰料蔡确、曾布根本就不给他任何机会。完完全全就是当成猪来养。

这样的生活,难道要这样持续十几年吗?

……………………

“蔡持正提防着蔡京,玉昆你也同样提防着他,名声也毁了,他这辈子算是完了。”

晚间放衙之后,章惇和韩冈出宫门时,正巧遇上,同行了一段路。

并辔而行时,随口聊着闲话。

韩冈摇头,他可不这么觉得:“蔡京还有四十吗?他现在才过了半辈子。”

“半辈子?不让他做事,只是用官俸养着,再有才的人没几年就会废掉。”

“那可不一定。文章憎命达,或许过几年能出个才子也说不定。”韩冈眯了眯眼睛,盯着章惇,“就像苏子瞻,这几年文章越发的精进了。”

“嗯嗯,说得也是。”章惇随口一句,又顾左右而言他,“不过倒是想不通曾子宣掺和进来做什么?蔡京还说了他的好话。”

章惇这是要引开话题乱开腔,韩冈看了他一眼:“就是因为说了好话才要插一脚吧。”

“也不见张大参说话。”章惇道,“难道是张大参还准备用蔡京这只死老虎?”

章惇的态度还是说笑居多。

韩冈倒是没那么乐观,蔡京在后世也是鼎鼎大名,不大可能一脚踩死。但也没有太放在心上。蔡京就算曰后能翻身,也不过是依附在皇权上的寄生物。这样的对手能带来麻烦,却不可能引来祸患。

吴衍就任了,召回沈括的诏书也在今天发出去了。游师雄,还有王舜臣,有关他们的任命也同样通过驿站送了出去。

只要气学的人心稳定,韩冈就没有任何可以担心。

听到了蔡确准备怎么处置蔡京,韩冈就已经将他抛到了脑后,他还要做正事呢。

给蔡京这么一打岔,好多事都耽搁了,但现在终于可以心无旁骛的去做了。

