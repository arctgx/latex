\section{第九章 拄剑握槊意未销(13)}

【今天兄弟结婚,要去帮忙,中午一更先欠着。对不住各位了。】

自侦查到契丹摆出了南侵的姿态,急脚递沿途一路疾奔,三天时间就从代州赶到了京城。

‘这个速度还真不得了。’韩冈想着,顺便用眼角余光观察着崇政殿上诸位宰辅的表情。

他眼下的地位很特殊,并非两府中人,却在崇政殿中有着足够分量的发言权。韩冈并没有因为这个机会为自己争取什么,只要不问到自己的头上,就不会多说一句。

“耶律乙辛知道了官军兵败灵州的消息!”王珪说了句废话,可即便是吕公著都没心情送他一个嘲讽的微笑。

辽人抵达西京大同的兵力被确定的只有两万,但没人认为会只有两万。如果辽人当真南下,二三十万铁骑就是转眼间事。

“河东险关重重,雁门诸关更是一夫当关万夫莫开。辽人一向不擅攻城,旧年其承天太后携辽主举师南下,车驾已至澶州,而边关诸城仍自保得全。河北城池如此,何论河东险关,陛下勿须忧虑。”

朕担心的是这个吗?赵顼恨不得拿起桌上的镇纸向元绛砸过去,他不是刚登基的黄口孺子,不用这等好听话来哄!他要能解决问题的实在话。

吕公著出班道:“辽人以骑兵优胜,河北方是其用武之地。攻打河东,其得不偿失,必不至如此。现太行八陉有三陉在辽人手中,军都、蒲阴、飞狐。契丹选兵南下大同,不过是分进合击的打算。”

吕公著算是说了实话,但一直对出兵西夏不以为然的枢密使,不会在这时候让天子舒心,“河北虽有郭逵坐镇,等闲匪类的确不须担忧。但如今边关虎狼环伺,辽人聚兵数十万,非郭逵所能当。旧年王超亦是名将,平戎万全阵的十五万人马在其手中,可契丹人依然攻到了黄河边,逼得真宗皇帝亲征澶州。”

“现在可是夏天!”王珪厉声驳斥。

吕公著反问:“离入秋还有几日?”

枢密使这一回成功的让赵顼心情沉重起来。

防秋,与秋收、秋税、秋粮一样,都是属于秋天时风物。大宋的北方边界,到了秋冬都是一年中最为紧张的时候,守军无不枕戈待旦,以防万一。也就是这两年,国中军事实力上涨才稍稍安定了一点。

河北的边界由于都是平原,无险可守,又跟辽国签有协约,不得私自增筑边关,乃是边州的城墙,故而一直以来,宋人都是在边界上植柳榆为边墙,决河水硬生生的造出了塘泊河曲八百里,另外还种植不合水土的水稻——收获许多时候还没有撒下去的种子多——用以阻挡辽国铁骑。

在夏天水丰的时候,这一套防守体系还是很管用的,但到了冬天,却因为水面封冻,而变得毫无意义。而且有一点更为讽刺,就是辽国或西夏的入寇,基本上都是在秋冬战马膘肥体壮的时候。春季夏季,那是要将养马力的,强行出兵的话,体力不足的战马,倒毙于途的情况会十分严重。二虏南侵率为财货,没有为了还没有抢到的财货,而把自家战马累死的道理。

“难道契丹人当真会撕破澶渊之盟,而大举南侵吗?”一直在等待时机的吕惠卿终于开口。

吕公著怫然不悦:“岂有将生死置之敌手之理?!弑主谋君之事都做了,耶律乙辛还有什么不敢做的?”

吕惠卿反问:“世所言无利不起早,南下攻中国,与耶律乙辛有何益?”

十几道目光转投向韩冈,殿上君臣皆记得韩冈早前曾经分析过耶律乙辛不会领军南侵的道理。韩冈却默不作声,没人开口问他就不会说话。

“局势已改。”吕公著有所准备,不过他没想到是吕惠卿而不是韩冈出来质问,“三个月前高遵裕和苗授还没有惨败灵州。”

“仅仅是两路驻军,相对于官军总数,损失微乎其微。”

韩冈惊异的望了元绛一眼,他到底是在帮谁?

只见吕公著声线陡然拔高:“两路兵将十万余,七成是禁军,已经是天下禁军的八分之一,而且还是最堪战的西军!”

“尤过于真宗仁宗之时!”吕惠卿针锋相对:“当年没有板甲、斩马刀和神臂弓,亦挡住了国势正盛的辽人。”

“难道泾原、环庆两军就没有?”

“灵州战败,乃是攻之败,非守之败。攻守之间,难易自是不同。公即为枢密,不该不知!”吕惠卿不等吕公著反驳,“不知耶律乙辛为何要南侵?能为大奸大恶,心术亦当过于常人。其人虽为权奸,辽国朝野皆从其意,但贸然南侵,一旦兵败,他可就是死无葬身之地。”

“参政想要为耶律乙辛做保人?”吕公著讽刺道。

吕惠卿怎么会帮耶律乙辛作保,暗骂了一声:“不,如果中国势弱,其必会立刻举兵南侵。耶律乙辛是权臣,把持朝政,名不正言不顺,必须卖好国中重臣和一众部族。到时候,他将身不由己。”

他看了赵顼一眼:“契丹先帝死因故暴卒,耶律乙辛嫌疑颇深,尽管其挟天子以令众臣,但国中隐忍不发者尤多。南侵也好,坐视也好,无论耶律乙辛做什么,他的目的都不会是大宋的财物,而是维持他现在的地位。以臣观之,只要西贼还不能彻底击败官军,耶律乙辛就不会立刻下注。”

“尽是臆测。”吕公著给了吕惠卿的分析一个高评价。

“是否臆测,自有公论。”吕惠卿不跟吕公著纠缠了。

“西夏的粮食还能吃多久?”元绛突然问道。

吕公著眼神闪动了一下,这是个好问题,不过他也挺意外,元绛什么时候转了风。

疑惑归疑惑,顺水推舟的回道:“去岁是十年以来最好的年景,各路州县基本上都是丰收。而辽国和西夏,却也一样是十几年未遇的丰年……加之西夏自从罗兀之役之后,便开始备战备荒,兴庆府中的粮食储备,当不在少数。纵然开战后消耗极大,应当还是能吃到年底。若是料敌从宽的话,明年夏收也不是没可能。”

吕惠卿没有再站出来,而是看了一眼侧前方。

王珪自知自己必须说话了:“粮草只是一方面,钱物呢?人丁呢?牲畜呢?为了这一场平夏之战,朝廷动用了陕西乃至全国的军力、物力。西夏国中已经被打烂了,一旦战争延续下去,来不及的秋播,明年的口粮从哪里来?夏天更是战马养膘的时候,可党项的铁鹞子却要连续奔走数千里,连番与高苗、王中正以及种谔李宪所领诸军交锋,到了秋天还能剩多少兵马?”

关于这一点,是朝堂上早就讨论过的,当时就是作为攻伐西夏的依据之一。

坚持下去,西夏迟早要完蛋。就是嵬名氏、梁氏打算拮抗到底,其他部族,不会跟着他们一条路走到黑。

从失去横山开始,西夏就已经开始了衰亡的进程。没有了南方的屏障,宋军可以任意进出。没有了步跋子的来源,光凭党项骑兵组成的铁鹞子,仅仅是一支瘸腿的军队。

“所以有耶律乙辛出面配合。”吕公著道:“眼下的局面不正是明证?”

赵顼心头堵了一口气,异论相搅的确是钧衡朝堂的好办法,但外患在的时候,内忧却始终解决不了,如何不让他头疼欲裂。

“韩卿……”赵顼将希望放在韩冈身上。

“臣亦以吕参政之见为是。”韩冈躬了躬身,“不过正如吕枢密所言,中国安危不可寄望敌手。河东、河北当加强防备……幸而辽人不到秋后不会轻动,以河北塘泊,亦南来不得。至少有两个月的时间去安排。”

基本上什么也没说。

赵顼沉默着,紧抿着嘴。对了!他想起来了,这一位也是不省心的。

韩冈暗自叹了口气。

他不是跳大神的,也不是耶律乙辛肚子里的蛔虫,怎么可能知道辽国权相在怎么想。

但韩冈同意吕惠卿的观点,这与他几个月前的判断一脉相承,现在也一样没有改变。辽人南下的可能性不大,眼下的情况依然还是讹诈的手段。只要添个十万贯岁币,让耶律乙辛能用来收买国中部族,又能大涨他的声威,肯定乐于就此收手。

韩冈又扫了眼几位宰辅。他就不信,这群狐狸,哪个会算不出耶律乙辛的盘算。

虽然一个都没往这个方向说,但用钱解决问题,从来都是澶渊之盟以来的第一选择。眼下避而不谈,不正是此地无银三百两吗?

以朝廷的财力论,十万贯并不多。

一名普通的禁军士兵,朝廷花在他身上的钱粮,一年少说也要三十贯,甚至五十贯,十万贯岁币,不过两三千人,五六个指挥的——而且这还是步兵。

可当今天子辛辛苦苦十几年,到最后还要增加岁币,天子的脸可就丢尽了。韩冈相信,赵顼能生吞了提议之人。所以宰辅们都不提这茬,让赵顼自己做出选择。韩冈同意不愿意去丢这个人,因为根本没有必要。

不过说不定真的会走到这一步,韩冈想着,还是先将自己摘出去比较好。

外界都传说他在危急的时候,很有可能会被派出去镇守边关。

朝堂上虽说很缺乏通晓兵事的重臣,郭逵镇守河北,蔡挺则已经病死,章惇擅长的领域在南方,但招王韶入京的诏书已经发出去了,等到王韶上京,韩冈有很大几率会被安排去河东。

不过韩冈了解得更清楚,王韶的病情很成问题。韩冈与王韶基本上保持一个月一封信的频率,过去王韶的信全都是亲笔所写,但他这两个月收到的,除了签名,都不是王韶的字迹。

因为王韶的事,这段时间,韩冈的心情一直很糟糕。如果王韶不能入京,自己就很难离开朝堂。

只是眼下的局势,还是能利用一下的。

