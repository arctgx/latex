\section{第27章 片言断积案(上)}

韩冈直截了当的退堂,将疑问和混乱留在了大堂外。

新来的韩知县,将在三天后,在何双垣墓前重审此案的消息,很快就从旁听的围观者那里,传遍了白马县中。同时也从诸立口中,传到了他的两个弟弟的耳朵里。

“到墓前审案?”诸霖脑筋转得飞快,“……这是要挖坟呐!”

“挖坟就有用了?”诸家老三嘲笑着韩冈的一厢情愿,“要挖坟开棺找证据,这么些年都几次了?都没一个新招数!哪次何阗、何允文他们肯答应下来?不是亲孙的怕棺材里有证据,是亲孙的也怕会被指着脊梁骨说不孝。”

“的确是老套了。”诸霖冷笑着,“记得一开始的李知县,后来是王知县,再后来的那个叫什么……长得一对鼓眼泡的那个提刑,他下来后也是要开棺,哪一次都没成!”

“说不定韩正言死人能让说话呢……”诸立沉吟着,突然冒出来一句。

“让死人说话?”诸家的老二老三以为诸立说了个好笑话,一起哈哈大笑起来,但笑了几声,看着诸立的表情不像是在讽刺,诸霖收起了笑容,试探的问着:“大哥是怕他有什么手段?”

诸立摇头不语,微沉着脸,皱眉在想着些什么。诸家老二老三对视一眼,心中都觉得有些不对劲了。这时,留在县衙中打探后续的亲信这时回来报信。

“有消息了,韩正言回去后就吩咐了,接下来要斋戒三日。”

“斋戒三日?!”诸霖一听之下,心头大惊。先命亲信离开,回头便急问着诸立:“大哥,他该不会在孙真人那里学到什么法术吧?”

“谁知道呢?”诸立摇了摇头,鬼神之说,他一向是半信不信。可韩冈的一系列传说从心中划过,就算孙思邈没有传他法术,但他也曾用什么格物之说压服了翰林学士,说不定,韩冈真的有那等鬼神莫测的手段,“能下令三日后于墓前审案,若是断不下来,脸皮都要丢尽。能这么做,多半是有些把握的。”

“那……那我们怎么办?”

“不能让他挖坟!”

诸立绝不想让韩冈成功。要是三十年的陈案真给他断了个明白,立下声威的韩正言在白马县可就是说一不二,他们诸家兄弟除了奉承听命,什么都做不了,那样的生活过上一两年,想想也是够憋屈。

“必须让那两人一起反对!”诸立吩咐着他的两个兄弟,想了想,又提醒了一句,“小心点,不要让人看破了。”

……………………

“难道是要开棺验尸?”

与此同时,韩冈的三名幕僚也在猜度着究竟是怎么一回事。诸立兄弟能想到的,他们当然不会想不到,而且想得更深。

“怎么个验法?”魏平真抬起眼,饶有兴致的问着方兴,“韩正言从来都不肯承认他是药王弟子。不论用什么方法来验证,如果没有药王孙真人在后面撑腰,什么结果都是不能让人信服的。可一旦拖出了孙真人,那正言此前所有的否认可都是谎话了。不论是在天子陛前,还是在相府,又或是洛阳、横渠等处说过的话,都要被他自己否定掉。以正言之智,至于为一桩争产的旧案这么做吗?”

方兴则道:“也不一定要真的开棺,只是要看一看两人的反应而已。真的肯定愿意,只是答应的会勉强些。而假的则绝对是不肯干的。”

“此等不孝之行,就算是真孙子,怕是也不会愿意。”游醇摇头表示反对。

惊动先人灵柩,使祖宗在地下不得安寝,那是大不孝。许多时候,就算尊长被人谋害,为了遵从孝道,都不会允许官府验尸,以验明凶手之罪。而是私下里去去找仇人报仇。

魏平真也笑道:“想来过去那几位打算开棺验尸的知县、提刑,也是这般想的。”

方兴立刻反驳:“正言岂是那等庸官可比?身份不一样,传说中的药王弟子,足以让人相信正言的手段。过去何阗、何允文二人不肯开棺,那是开棺也没用。墓碑上都没有证据,棺材里当然也不会有。可现在不同了,至少在外面看来,正言肯定是能将此案断出来的的。”

见着游醇不以为然,方兴质问道,“要不然节夫你来说,正言今次是有何用意?”

“可能真的能有什么办法吧……格物致知的道理,正言最为精深,也许才此事,也有所创建。自是小弟才智浅薄,学问不精,却是想不到。”

游醇很坦然的自承不知,并没有因方兴的态度而生怒。只不过,他也是想破脑袋也想不明白,韩冈的葫芦里究竟是在卖什么药。斋戒三日,那是行大礼、举大事之前的仪式。韩冈信心十足,又为此而特地斋沐三日。从韩冈到这一套行动中,不论游醇怎么去思考,都会往药王弟子的身份上偏去。

“要断成铁案,必须要让原告被告都心服口服,或是全县的百姓都认为断得有理,否则必有反复。日后牵扯不清,肯定会有人籍此来攻击正言。”魏平真说着,摊开了手,摇着头很是无奈,“正言肯定是有办法,我们也只能看着了。”

韩冈不肯说究竟要怎么做。他们也只能在这里胡乱猜测,到时候,说不定就会有个惊喜或者惊吓等着他们。

外界对三日后的断案同样众说纷纭,尤其是当韩冈要斋戒三日的消息传出去后,各种各样的猜测一下都泛滥起来。当然,都不会少了药王弟子这个身份。

至于韩冈本人,则是一切如常,斋戒的确是在斋戒,毫不在意的吃着粗茶淡饭,白菜烩萝卜的吃了整三天。这三天里,韩冈也没有耽搁下公事,前前后后跑了好几个乡,视察当地的灾情。而在乡中被父老请着吃饭时,都是再三吩咐只上最简单的蔬饭,一点酒肉都不要。每天回衙后,还都要吩咐人烧水,沐浴一番方才睡觉。

韩冈三天来的一番举动,则是助长了另外一桩传言在县中快速的散布开来。

“肯定是滴血认亲。不然为什么要到坟墓前审案?这下要开棺验尸了。”

“何双垣死了都几十年了,骨头翻出来都能用来敲鼓,哪儿来的血?认什么亲?”

“这你就不知道了吧?县尊可是药王孙真人的弟子,什么手段没有?听说以孙真人的医术,别说没有血,就是骨头和肉都没了,只需要一根头发,就照样能验出是不是亲生的。虽然韩县尊不是孙真人,但好歹学了一点。”

“这事我也听说了,据说只取出一根骨头磨碎了,然后让子孙的血滴上去,能融进去的就是真货,融不进去那就是假货!”

“胡扯,上次我家的狗抢骨头,被咬出的血照样染到骨头上去了。狗是猪孙子吗?”

“肯定还有法术在。要不然县尊为何要斋戒三日?不就是为了要施法吗?”

“损毁先人骨殖,也亏那两老夯货愿意。”

“有什么不愿意的。为了两顷地,怎么都要答应下来。亲祖父如何?戳脊梁骨又如何?哪有田地实在?!”

“世风日下,人心不古啊!”传言的最后,一干老措大摇头叹气。对比着眼下的现实,只能遥想着千百年前那个重礼守孝的神话时代。

……………………

预定开审的日子终于到了。

比起前一次开审,有了三天时间的酝酿,关注此案的人数翻了好几番。可以说,全县男女老幼,连同经过白马的路人,都听说了这桩闹了三十年旧案。加上一番神神怪怪的传言,使得涌来要一看究竟的,成千上万。大半都是先去了清水沟,去抢一个好位置,小半则是在县衙前候着,准备跟韩冈一起出发。两边的人数粗粗一数,加起来,差不多白马县的百姓都到齐了。

但就在韩冈要领众前往审案地,此案的原告和被告却一齐拜在韩冈的脚边,“县尊,这个官司小人不打了。”

“县尊,学生要撤诉。”

韩冈脚步一停:“不打?这是为何?”

何允文重重的磕了一个头,“如果要毁损先祖遗骸,这场官司小人只能不打了。”

“小人不孝,不能守先人庐田,致使为奸人所玷。”跪在地上的何阗痛心疾首。“一争三十年,也只是想争回来奉与香火血食。可要是毁伤遗蜕才能验证,小人今日也只能撤诉了。”

“开棺验尸?不知尔等从何听来?本官有说过什么吗?!”韩冈眼神一下凌厉起来。虽是年轻,可历经风雨而磨砺起来的气势,高居云端的地位,双眉只微微一皱,如刀似剑的眉眼凝起的威严,就压得两人张口结舌。

何允文从压迫感中勉强挣扎出来,战战兢兢的问着:“当真不会伤到家祖遗骸?”

韩冈冷哼一声,根本不理会何允文的问题,提气高声,让声音传遍周围群众:“经过这三日,本官已知此案真相。今日到何双垣墓前审案,也只是让白马父老做个见证!是非黑白,转眼即知,你们究竟怕个什么?!”

说罢一甩袖袍,不再理会何阗与何允文两人,他俐落的翻身上马,马鞭遥遥一指城北,“去清水沟!”

