\section{第17章 桃李繁华心未阑(上)}

敇建横渠书院。

曾贤仰头望着汉白玉牌坊上的几个大字。

横渠书院的山长苏昞,因为去年以横渠书院的名义向朝廷献上了《正蒙新注》,太后一时兴起,给了横渠书院这么两个字。

敇建……

敇建横渠书院。

当然,从小就在横渠镇上长大,幼时便在横渠书院附属的蒙学读书,年长一些,正式成为横渠书院的成员,曾奇知道这两个字附带的东西没有这么简单。

太后给了两个字,朝廷便为此拨款三百万钱,为横渠书院修建山门,同时赐地千亩,供学生饮食。

三百万钱,足足三千贯,至少能装十台大车,但曾贤没有看到钱,只看到了这面高大的牌坊。另外官府划来的田地,则有一片直接跟他家在镇西的十几亩田连在了一起。

而自从有着太后亲笔题字的牌坊立起来之后,不过半年时间,来到书院的学生又多了一倍,家里在镇上新盖的两间屋子也全都租了出去。兴旺发达是好事,可是两年后的明算、明工两科,小韩相公为气学门人量身定做的科目,竞争者可就更多了许多。

“曾小乙。”一名同学喊着他的名字,“还不回家?”

“这就走。”

曾贤放下心头事,与同学一起沿着水泥铺就的道路回镇上去。

自牌坊立起之后,从牌坊到正门,只许步行。上元节后,知县过来,便是在牌坊处下马。

一队车马这是沿路而来,也在牌坊前停下。进出书院的学生们,都停下了脚步。

车队一行人,纷纷下马下车,最后从第二辆车上下来的一个中年人,明显是众人之首的样子。

“啊。”

看到那个中年人,曾贤不禁惊讶出声。

“谁啊?”他的同学不认识,“是小乙你认识的?”

“是当世陶朱!”

才说完,曾贤立刻听到一声冷笑。

“陶朱公?……于今安有范少伯?”

“朝廷赐的三百万钱,在他眼里就是区区三千贯;千亩地,也只有百亩能入眼。”

“陶朱公可不光是富就算数的。”

“顺丰行的大东家,韩相公的亲表弟。这座书院,有一半是他捐的。”

“照样还是当不起!”

看着一脸傲然的同学,曾贤放弃的摇摇头。

这几年,被冯从义推荐到王舜臣麾下,由此得到官身的气学门人,已经有七个了。此事在书院中尽人皆知。

从熙河路开始,一直向西去,甘凉、安西、北庭等地底层的流官位置,能给气学门人占去了大半,正是靠了包括冯从义在内,多少有力之人的举荐,光靠韩冈一人,怎么可能让气学一脉好处尽占?

只说经义,冯从义肯定连刚入学的学生都比不上。可论眼界、论见识,书院中又有几个能与他相比?

曾贤可不会因为冯从义是商人而觉得可以鄙视一下他身上的铜臭味。铜臭到了极致,那就是香了。就像龙涎香,《自然》中可是说了,就是鲸鱼的粪便,因为里面有鱼骨的残渣。

但曾贤没兴趣教育他的同学,费尽口舌也不一定有效果,反而平白无故的招人鄙视。

牌坊内,这时有一群人从正门方向快步走来,曾贤远远的看清了走在前面的第一人,“山长来了。”

……………………

敇建横渠书院。

上次冯从义过来时,还没有这座牌坊。

太后颁了诏、提了字,又赏赐了田地和钱钞,让书院扩建了规模,也让敇建二字可以堂堂正正的戴在头上。

冯从义的身旁,学生来来往往。

小的十四五,大的,二十五六也不足为奇。

年纪小的学生,对他这个带着七八伴当、明显不是士人的陌生人,投来几许好奇的目光,而年长的学生,则是目不斜视,见怪不怪的径直擦肩而过。

“人更多了。”冯从义轻声说道。

“那是。”

“听说多了一倍。”

“两千多人,跟国子监一样多了。”

“镇子上都住满了。”

身边的伴当一阵附和。

教授的学问与官学截然不同的横渠书院有了朝廷的册封,这一下子让关西一地还在观望的士人,彻底站在了气学的一边。

但冯从义知道,韩冈虽然为横渠书院躬谢天恩,但他并不是很喜欢让书院染上太多官方的色彩。

“陶朱公来了!”

牌坊后的阶梯上,远远地就一阵大笑声。

人随笑声而至,冯从义才到牌坊下,就等来了前来迎客的主人。

周围的学生则纷纷侧目,然后恭敬的向那人行礼,齐声道:“见过山长。”

冯从义向来人一揖到底,“冯四见过山长。”

苏昞向学生回了礼,又迎上前与冯从义见礼,拉着冯从义的手,展颜笑道:“去岁冯兄未至,让人好生想念。”

冯从义也大声笑道:“去年没能来书院染身书香回去,冯四这一身俗臭味越发的不能近人了。本来是想来的,只可惜奉了我那表兄的命,去了西域一趟,一去来一回八个月,剩下的四个月就只能在家里将养了。”

与客人并肩前行,苏昞问着:“冯兄去往西域,想必是有所见闻。”

“大漠风光,在下做不得诗赋,不知该如何描画。不过,玉门关那里,每天出关去西域屯垦的汉人,每天络绎不绝。想来十年之后,天山南北必定皆汉腔唱歌。”

“风物岂得与人物比。”苏昞笑道:“得闻此事,尤胜百篇天山、大漠。”

“苏山长说的好。”

苏昞一声长叹:“千载之前,班定远与博望侯相继西域,自那时起,便有汉人屯垦,回鹘也好,突厥也好,还不知在何处。自大唐中衰,北庭、安西为胡人所有,不再见汉人踪迹。昔年读史,不免为一叹再叹,岂料有今日,西域终于重归汉家。”

“西域水土最好的地方,还要数伊丽河谷,七河汇聚之处,水土丰美远胜安西、北庭两地。家兄曾说,只有攻下那里,再移民百万,才能安心下来。”

“安西、北庭两大都护府这两年平静得很,难道就是为了此事做准备?”

“军国大事,山长你问了我也不敢说啊。”冯从义摇头道,“东黑汗在疏勒死了快有三万兵马,受伤的更多,还要提防西黑汗,若官军兵发伊丽河谷,东黑汗说不定就要给西黑汗吞并了。”

“西域那边还没装备火炮吧。”跟在苏昞身后一人问道。

“要不是担心被西夷给偷学去,早就把火炮拿去西域用了。王景圣上次回京见识过火炮后就说了,给他五百火炮,他能打到大食西边去。”

“辽人不是也把火炮学了去?怎么不怕辽人偷学,倒怕西夷偷学。”那人抱怨着。

“打辽人也没几年了,可打西域还不知要多少年。辽国的情况能打探得到,西域那边可就打探不明白。万一给西夷偷学了去,过个二十年后,朝廷打算西征,却发现大食城头上全都是一门门火炮,比官军带过去的都多,那样还怎么打?”冯从义笑着道,“什么时候朝廷决定大举西征,一路打到极西之地去,那时候,才会动用火炮。现在对付一下黑汗人,只用神臂弓、斩马刀和板甲就够了。”

“听人说王都护是个急性子?”

又有一人开口,问冯从义,苏昞见状,接过话来:“正任的团练使,除了国姓的王孙,就属他最年轻。北庭都护、安西并受其节制,他也不必急于一时。”转过来,他对冯从义笑道:“冯兄新近从西域回来,不免想多问几句。”

冯从义呵呵笑:“这也是寻常。说起来北庭那边,当真是兵甲堆积如山,也不知运了多少过去。若是按照南方的情况,铁器易锈坏,理应多准备些。不过西域天干,一年下不了几场透雨,铁甲放在外面几年都不带有锈斑。可朝廷还是送了那么多去。现在北庭军中踢球时,都是穿着甲胄,根本就不怕坏。”

“穿着甲胄怎么踢球?”一人好奇地问。

“也不是踢了,就是抱着球往球门冲,想拦住就直接撞上去,咚的一声响,一指厚的胸甲能撞弯过来。一场球赛下来,撞坏的铁甲能有一半多,血流满面的场场都有,比起蹴鞠痛快得多!”

冯从义的话在树荫遮掩的石板路上传了开来,有人皱眉,有人向往。

说话间,已经抵达书院正门。冯从义与苏昞相让着走进大门。

“一年不来,屋舍更多了,人也更多了,这书香味更浓,倒映得我这俗人更加俗了。有山长在,书院日渐兴旺啊。”

“还多亏了冯兄。”

“不,没有横渠,就没有家兄。没有山长,书院不会有今日。”

看着今日的书院,冯从义感触颇深,当年修起横渠书院的那一笔钱,有很大一部分,还是自己奉了韩冈之命送过来的。

当时横渠书院草创,还是在山前的一座庙宇中开课,之后第二次经过横渠镇,也就大大小小十来间房,给学生们住的房舍还是茅草屋顶。倒是一干学田开垦得很好,也开辟了引水渠,改成了上乘的水浇地。风车、水车都修了,还附建了磨坊,给书院赚些菜钱。之后每一次经过横渠镇,冯从义都能发现书院有了变化。

在张载去世之后,苏昞一人坚持守在横渠书院中,拒绝了朝廷的征辟,拒绝了同学的举荐,固守在这里,看着书院一步步扩大,成为关西士人人人向往的圣地。

\section{第17章 桃李繁华心未阑(中)}

曾贤回到家里的时候,他父亲刚刚从乡下回来。正脱下外袍,交给家里的小养娘拿到院子中去抖干净。

曾贤在进房前,也拍了拍衣服上,几天没下雨,风一吹身上都是灰。

曾贤父亲端着凉茶喝了两大口,“韩相公的表弟来了,大哥你在书院那边看到了没有?”

曾贤有些惊讶,“阿爹怎么知道的?”

“顺丰行的冯大官人到了镇上,横街的那几家,哪个还能在店里坐着?”

“顺风行的大东家见他们了?”

“见个屁!”曾贤父亲冲院子吐了口口水,“卖斤屎还要先撒尿加二两份量的,冯大官人会搭理他们?!李麻子脸上的黑字不是官家的墨宝,李黑狗腰上的金带也不是官家赐的,凭他们也能见到韩相公家的表弟?”

曾贤拿起茶壶,给自己父亲喝空的茶杯满上:“阿爹说得是。”

谁让卖米面的李麻子和贩南货的李黑狗与自家支持的不是同一队?

曾家住在镇东,横街那边属于镇西,两边各有一支球队,每个月都要踢几场。长年累月下来,两支球队的球迷就成了冤家,尽管只隔了一条镇子正中央的大街,每天低头不见抬头见,照样是冤家对头。

“冯大官人这一回来,也不知书院里谁要倒霉了。上回来,那个王账房就全家去了西域。再上上回的老王账房,他倒是自个儿吊死了痛快,可惜他家眷照样给送去了西域,温明府说得好,既然贪来的钱都一起用了,那当然得一体治罪,还敢以自尽对抗王法,更是丧心病狂,不能不从重处置。”

曾贤嗯嗯啊啊的应着,顺手整理自己今天上课的笔记,他知道,自家父亲絮絮叨叨起来,就没完没了了。不过他更清楚,冯从义时隔一年来到横渠书院,书院中与账目有关的管事们,可都要提心吊胆睡不着觉了。

不知要送多少人去西域,曾贤想着,这可是很重要的。

……………………

一群人战战兢兢的站在冯从义的面前。

冯从义一反方才与苏昞的谈笑风生,脸沉了下来。

想讨好京中那位韩相公的人很多,所以给书院捐款的人很多。雍秦商会中的成员,或是成员的后台,每一个都不小气,捐款数量少的几百,多的上千。这不是小数目了,几百上千亩地一年的出产。

冯从义是书院的财神爷,又是韩冈十分亲近的表弟,所以尽管他就是一个商人,但苏昞还是对他有着足够的尊重。自然,这也是因为他性格不错,又善于与人结交的缘故。

这些捐款都被用来购买土地,书院的地产,超过了横渠镇土地的一半还多。日常开支,都是从出产中获得。

书院之中,为了方便日常运作所有教学之外的杂务,都是由外聘人员处理,从日常饮食,到院中清洁,还有田地收账。此外,财务也有专门的账房来管,老师和学生都不沾手。

每个月,会在书院照壁墙上公开账本,同时无论是师长还是学生,或者是捐款人,都有权利随时查账。

这其中,绝大多数捐款人从来都不会查账本——他们捐钱,就是为了结交,捐了之后再查账,那就是得罪人了——许多学生和老师,也不会去关心账目,觉得一身铜臭。但冯从义每次来,都会让手下人细细检查一番,因为他代表的是韩冈,因为韩冈希望他捐出的钱,能用在该用的地方。

现在一干管事就在冯从义面前,战战惶惶。

至今为止,即便仅仅是在采买时收受回扣,等待他们的都是开革的处分。名声坏了,一辈子都别想再寻到好差事了。更严重就会直接报官,被冯从义送去西域的账房有两个,连同他们的家眷,全数流放异域。就算贪污不算过分,不至于株连亲族,犯案的本人,也会被送去西域。

近十年来,横渠镇所属的郿县,连着三任知县都是横渠书院出来的学生。犯到他们手中,结果当然是注定的。尤其是现如今,为了能更好的控制西域,即便是窃盗小罪,只要是累犯,立刻就是发配北庭或安西军前。任何想从横渠书院师生们的牙齿缝里刮钱的人,在伸手之前,都要好好考量考量。

等了半个时辰,苏昞等到冯从义回来了。

“怎么样?”

“这一回还算好,都学聪明了。”冯从义淡然道,“不过管采买的周冲还是辞了。”

“要不要解官?”苏昞问道。

周冲在苏昞的印象中,是个很老实的一个人,不然也不会让他去管采买。但苏昞更信任冯从义的审计,顺丰行中的账房,天底下没有比他们眼睛更利的了。

“还不到那种程度,去年冬天,书院下发冬衣,周冲引来的裁缝用剩下的布料,给他家里的孩子做了两套衣服。”

两套衣服就要撵人,按平常的标准,是严格的过了头。别说是书院中的雇员,就是签了卖身契的家奴给主人家出外采买,拿个一两成回扣都是天经地义的,主人知道就不会说什么。

过去第一次用这样的标准来开革书院雇员的时候,冯从义回答苏昞的质问,说事情要防微杜渐。还反问,箕子为什么见到纣王收了一双象牙筷子,立刻就跑了?

现在苏昞不再多问,已经习惯了。

但冯从义总是会向苏昞多解释几句:“书院给出的工钱,比其他地方相同的佣工要多两成,四季和年节的衣料、节赏都比其他地方要多,这样还手脚不干净,是人心坏了,绝不能留下来。”

“不过这件事是怎么知道的?”

苏昞挺纳闷的,很隐秘的一件事,冯从义一来就知道了。若不是知道韩冈的表弟有颗七窍玲珑心,保不准就会以为他在书院里安插了耳目。

“是有人出首。”

苏昞脸变了,“此人也不能留。”

收受好处一事,若是正直之人,应该当面指正。若是忠心之人,也会及时上报。当面不说,又不及时上报,而是隐瞒下来等待时机告发,这样的人人品卑劣,甚至比收受回扣还恶劣,书院中不能要。

“调来顺丰行。这样的人,的确不适合留在书院里,不过我们这些做买卖的,还是要有几个耳目。有番周折,也能让他知道日后怎么做事。”

“也好。”

苏昞不想在这些俗世上多纠缠,定下了开革名单,便直接放下了。有冯从义盯着,什么人也别想泛起坏心思。

只是免不了又要感慨一番,“书院是教化之地,却连离得最近的雇工都教化不了,有负圣人之教。”

冯从义全然没在意,苏昞从来都不是书呆子,现在的话,也不过是发发牢骚罢了。

……………………

“只有两个。”

曾贤次日回到书院,一名同学就凑了上来,低声通报最新的消息。

“发配?”

“开革!”

“西域难道不缺人了?!”曾贤反应很大,这可关系到半贯制钱的赌金。

韩冈看重西域得失,此事人尽皆知。所以只要有机会,许多官员就会将人发落去西域。不管是不是罪囚,只要有汉人在那里占着土地就可以了——非我族类,其心必异。即使是罪囚,也远比蛮夷更可靠。

曾贤本以为赌这一票不会输,没想到这一回却变了样子。

“缺得多了。”压中冷门的同学嘻嘻笑道,“但总不能‘弃灰于道者弃市’。就拿了两件衣服。”

“怎么说?”曾贤问道。

从同学处得到了详细,曾贤苦了脸,许久方叹道:“道之以德,齐之以礼,有耻且格。”

“曾小乙,输便输,不要输不起啊。”赌赢了的同学笑着说道,“说真的,被开革还不如去西域,不过是换个地方种田,朝廷其实已经很宽大了。”

“西出阳关无故人。”

“无故人总比自己不能做人要好。饿肚子,可是要变鬼的。”

曾贤抿了抿嘴,却也不再强辩。

书院里都在这么教。衣食足而知荣辱,仓廪实而知礼节。

气学一脉,从不空谈仁义。在他们的心中,百姓吃饱穿暖,才有知礼知耻的基础。

一箪食,一瓢饮,在陋巷,不改其乐的是复圣颜回,不能拿圣人的标准要求普通人。

所以士人想要实践横渠四句教,就必须先从实事做起。

求实,务本。

乃是气学一脉治学的宗旨。

“更别说你我若去西域,立马一个官身,再来几年,说不定就能入流了。”

书院中的消息很灵通,图书馆中,连朝廷下发到县中的塘报都有。

曾贤当然也清楚,如果自己愿意去西域,即使不能立刻做官,可历练一段时间后,还是有很大可能成为有俸禄的官员。

可是这个决心不是那么好下的。去了西域任官,这辈子还能不能回中土可就难说了。天下人人向往中原,四荒的官都没人愿意做,所以官吏一旦任职岭南,这辈子就要蹉跎在海天之外,就是进士也难保能够重返中原任职。西域现在的情况,说不定就会跟那岭南一样。

不到万不得已,曾贤还不像将自己的未来给赌进去。

“好了。小乙。”一只手伸到了曾贤面前,“愿赌服输。”

曾贤叹了一口气,然后认命的开始往怀里掏钱囊。刚摸出几个金灿灿的大钱,就看见一人徐步走来。

看见那个衣着寒素的年轻士人,曾贤连忙将钱重新揣进怀里,拱手行礼,而他身边,已早有人弯腰躬身。

“曾贤见过助教。”“赵菏见过助教。”

那人微笑着一一还礼,寒暄了两句,然后告辞离开。

望着他的背影,赵菏茫然若失,“一箪食,一瓢饮,回也不改其乐。”

“文诚先师的儿子,只要去东京城,哪个门子敢拦着他?颜子,张助教想做就做,不想做就不做。”

横渠先生张载张文成的儿子张因。

张因在书院中是属于比较特别的学生。在学习的同时,还辅助教学,是为助教。

张因是张载唯一的儿子,张载过世时,他尚未成年,因张载遗爱,故而备受张门弟子的照料。一众弟子,以韩冈为首,纷纷赠金赠地,使得张因成为横渠镇上除了书院之外最大的地主。

而张因成年后,就将自家的土地捐了大半出来,大部分做了书院的学田,小部分则是留作族里的祭田。只给自己留了百亩,供养老母,供己读书。

书院中,寻常学生要么学义理,要么学治事,张因是两者并重,一面苦读张载的著作,一面则学习自然数理方面的知识,对科举则毫无兴趣。

前两年大考,张因位在前列,山长苏昞曾兴奋的对人说,‘释迦不以罗睺传,老聃不以子宗传,孔子不以伯鱼传。气学一脉,子宗可传。’

所以在书院中,张因不仅仅是因为他的父亲而受到师生们的尊重。

“听说顺丰行的冯东家这一回来,还准备请了张助教一同上京,但张助教又拒绝了。”赵菏轻声说,满是羡慕。

“也不是所有人都想上京的。”

曾贤拍拍手,背后有靠山,不愁吃穿,不愁前途,安安心心的做自己想做的事,放着这样的日子不过,上京做什么?

要是自己有张因的条件,也肯定会留在书院中,去打造那些机器。看着巨大的机械转动起来的样子,远比读书更有趣。

只可惜啊,曾贤想着,自己永远也不可能有张因的条件,未来依然模糊。

