\section{第17章 桃李繁华心未阑(下)}

“这天越来越热了。”

方兴手中的折扇,有一下没一下的摇着,额头不见半点汗水。

“这听雨小筑真是难得。”游醇带着好奇,上下张望。

十尺见方的小屋,只摆了一张桌。陈设极为朴素,以夯土为墙,以青砖为地,头顶上能看到未上色的房梁和椽子。没有上漆的桌椅,不见华饰的陈设,唯有两个摆满书的书架,给简陋的小屋增添了几许书香。

如果只看房内,任谁都很难想象,这是东京城中近两年最有特色一家新店,只有单独的包厢,每天接待客人有定数。即使是预定,通常也要等到七八天后。

但最难得的是日头火辣辣的时候,屋前却有雨水垂帘,只听着水落声,心中便是一片清凉。

透过门上的竹帘,可以看见外面的水车一角。竹木水道从远处引来的流水,被水车扬到屋顶,顺着瓦片流淌下来,

水车无声无息的转动,木斗带起的井水浇在屋顶上,一阵一阵,极有节奏的响着。

屋顶流水哗哗作响,窗前的水帘打在屋檐前的青石板上,水落石出,有如空谷清音一般。

方兴轻摇着折扇,听雨小筑,这名字乍听来俗不可耐,只有亲眼看见才知道有多难得。

春赏花、秋赏月,夏日听雨,冬日观雪,四个院落依时开闭,不管哪个节令,都只有四分之一的地盘接待客人。而且不论那个院落,每间厢房在修造的时候都很注重隐秘性,或是竹篱,或是树墙,或是池畔假山,将包厢遮掩,除非刻意去寻找,否则即使是走出包厢,也很难看到其他客人。

“好了,先喝酒再看。东西在这儿,也跑不了。”

方兴放下扇子,邀请许久不见的老友入座。摆在桌上的都是一些清淡的菜肴,连酒水都是清冽的果子酒。

提起没有花纹的素色瓷壶,给游醇倒酒,方兴笑道:“夏天只有听雨小筑。到了秋天再来,就是望月居了。”

“望月居是有玻璃屋顶的那个?”

“节夫你也听说了?”

“今天在馆里问了一下,便被人拉着说了好半天,颇受人羡慕啊。”

方兴哈哈笑道:“就是那一个!与宫里的那间新修的温室用了同样的玻璃屋顶。中秋之日,月上晴空,在屋中仰头望月,诗兴什么我是不知道了,不过想着千古以来,唯有今人能享受到这样的乐趣,心里痛快得很呐。”

游醇笑了笑,没说话。安于逸乐,这时候说,未免不合时宜。但心思太多放在享受上,

“其实望月居最有意思的还是下雨的时候,能清楚的看到头顶上的雨水,还能安然坐着饮酒,此间乐,古人不知。”方兴举起酒杯。

游醇举杯应和:“都说今不如古。其实也有古不如今的地方。”

“因为人心不古嘛。”

放下酒杯,游醇问道:“最近京中有什么新闻?”

“征大理算不算?”

游醇摇了摇头:“听了很多了,可一日朝廷不决定主帅人选,便一日是空谈。”

“不过报纸上说得挺多。”

西南方面的主帅人选,还没有诏书出来。朝廷的塘报和外面的报纸,都在连篇累牍的抨击高氏为逆。

“名不正则言不顺啊。”游醇轻叹了一声。

朝廷要名分,当然只能这么做。

其实如果排除掉掌握国政这一条,高智升、高升泰其实可算是大理拨乱反正的忠臣。元丰三年,逆臣杨义贞杀国主段廉义,自立为君,高智升便立段寿辉为国主,命子高升泰杀杨义贞。只要他一日不篡位,一日便是拨乱反正的忠臣。

不把他们的名声毁了,朝廷可没脸直接派兵上阵。太祖皇帝的卧榻之侧虽好,可玉斧划界都丢一边去了,再借用太祖的原话,说了徒惹人笑。

“……那大气压铜球实验呢?”

“是相公在去年九月的《自然》上写的那个实验?”游醇沉吟道,“上京的半路上,已经听说有人验证成功了。”

“的确是成功了,而且是在国子监的大门前。”

这是韩冈在自然杂志上提出的,用来验证大气压的存在的实验。横渠书院第一个进行验证,然后一帮好事者在国子监的正门前又重复了一次,

两个一样大小半球形的黄铜碗,合起来就是一个严丝合缝的铜球,只有一个抽气的小孔。用真空泵抽出铜球中的空气,用了八匹马,都没能将铜球给拉开。

“当初相公用水银柱确认大气压存在,却还有人不承认。且以国子监中谬论最多,说是若大气压当真存在,小小飞蛾都要背着几倍的重量,怎么活得下来的?还有监生在监中说,自己手不能提肩不能担,自不如农家子能担重担。”

游醇摇头,这是自己作死,话说得婉转点,日后还能为自己辩护。说得这么明白,不是生生的把自己打包送给人去讨好宰相?

“我在西京,也听闻人说,天将降大任于斯人也,不做宰辅,不得重荷。”

方兴笑了起来:“这可还算会说话的。”

“其实是成了习惯,反而感觉不到了。”

“节夫这话说得对。现在铜球实验出来了,国子监又丢了一次脸,多来几次也就习惯了。”

游醇暗暗摇头,国子监是新学巢穴,尽管大部分学生只是为了进士,但死硬的新党成员还是有不少的。在方兴这种韩冈的心腹眼中,便是死敌的老巢了。不过在外人看来,自己也是韩冈的亲党,不能当做没事人一样站旁边看热闹。

“横渠书院现在也越发的厉害了,天下间的书院,当数其第一了。”

“有太后青目,韩相公照拂,金陵、嵩阳两处如何比得上?”

金陵书院和嵩阳书院,两家书院政治色彩与横渠书院一样浓厚。王安石致仕后每隔两天就去一趟金陵书院讲学。而嵩阳书院,一直以来就是旧党的巢穴。

这样一来,横渠书院便与金陵书院、嵩阳书院一起,成为士林中有口皆碑的三大书院。

相较而言,老字号的白鹿洞、岳麓等书院都没落了。近一些的应天书院,仁宗时改府学,变为应天府书院,之后应天府升南京,又改为南京国子监,在成为官学同时,也同样失去了在学术上的地位。

游醇从洛阳来,嵩阳书院的情况他很清楚。

有了横渠书院在前,嵩阳书院早前便献书朝廷,向太后要求得到同样的待遇。而金陵书院,好像也不甘心居于人后。

“但不是差敇建二字那么简单……”游醇心中不免感慨,嵩阳书院之中,浮躁之气越发得重了。大程、小程两位,也无法强行管束住书院中的学生。

“差得地方多了。不说别的,钱财上就差得远。”

方兴意气风发,但游醇不太喜欢书院参杂了铜臭味。

随口应付了两句,便扭开话题:“说起来,那个真空泵到底是什么?真空好明白,可泵做何解?”

游醇一直很佩服韩冈。在他看来,韩冈才思无所不包,自然之道在韩冈那里,能牵连万物,无一事可脱。唯独不好古,想着以今胜古,连字都能生造,泵这个字,古来未有,怎么也想不明白。

“节夫也想不通?……其实都一样。泵与火炮的炮不同,同时是相公生造,炮字易解,可泵字难明。明明是水落石出,也不知为什么成了抽水抽气的机器。却不如火‘炮’说得明白。”

“还问过相公吗?”

“哪里敢用这等小事麻烦相公?”方兴摇头,他当年给韩冈做幕僚,只是宾客与知县的距离,而现在却是普通朝臣与宰相的差距,纵有情分,见的次数少了,哪里有时间浪费。想了想,又笑道,“其实还有点让人不明白,为什么火枪还是那个‘枪’,没有改成火旁!”

游醇还是只能摇头,同样不明白。

喝了几杯酒,方兴用手指沾了酒,在桌上画了几笔:“说道生造,这个‘砼’,节夫可还知道。”

“水泥吧。”

“是水泥弄出来的石头。人工之石,又是诸物混同,所以叫做砼。”

游醇点头。仝同相通,砼这个字,可算是生造字中起得最好。

不知从何时开始,水泥渐渐多了起来。原来据说只是江南富人害怕墓墙中的砖石被盗,改用水泥砌墙以代替砖石。可现在。从窑烧出来的水泥、拌合黄沙、石子,浇模凝固后,就成了石头一般坚固的东西。

“要不是水泥太贵,完全可以直接拿来筑城墙了。”

“可谁出那份钱呢?”方兴大笑道,“水泥可比黄土贵多了。”

“筑桥基的话,这笔钱就省不得了。”

“自然。”

夯土墙,就是两块夹板中间加黄土,用锤子夯实。而水泥筑墙,同样是几块夹板,然后在中间灌上搅拌后的水泥,凝固后就成型了,比起夯土墙更结实。若是全用水泥筑成城墙,那就是浑然一体,等于是一块巨型的石头。就是火炮,能砸坏夯土和包砖的城墙,但怎么击毁已经成了一整块、厚达数丈的石头墙?

但水泥的价格太贵,现在的水泥,最大的用处依然是用来刷墙和抹地。还有种用法,就是在墙头上,用水泥黏上一堆碎瓷片,甚至铁钉。而砼,仅仅是用来造桥墩和台基,水泥最大的好处是,遇水反而更容易凝固,石拱桥架在两岸,承接石拱的桥墩、台基,用上水泥最让人放心。

两人喝着、说着,数年未见的生疏在觥筹交错中渐渐弥合。

等到月上柳梢,方兴和游醇才踏足屋外。

出来抬头看见巨大水车,与屋前的水帘,游醇叹道,“当真日新月异啊。”

“且等十年后再回头看今日,或许亦已变得寻常了。”

“不消十年,两三年便是一大变了。”

……………………

“我是不是看错了?”

“应该没有。”

“但那是韩相公吧?”

“还有章枢密。”

“他们进去了?”

“进去了!”

宣德楼下,待漏院前,数以百计的朝官们发出的声音,如同几十群黄蜂聚在一起振翅。

在王安石离任之后,朝堂上变得十分和平。没有激烈权力斗争,除了争夺进入两府的新席位,有了一些龃龉之外,其他时候,都各自相安。

新党官员,该擢升的时候,依然擢升,政事堂并未因为他们身份和倾向而进行干预。

几年下来,新党之中对当初王安石力推吕惠卿,以至于与韩冈决裂便颇有怨言,章惇在新党中的地位也更加稳固。

不过东府、西府的两位大佬坐在一起说话的场面,这两年几乎看不见。除非是在内东门小殿或是崇政殿等议事之处,否则两人之间根本没有什么交流。

但今天韩冈和章惇赶在早朝前,一先一后进了待漏院中。让众多朝官跌掉了他们的眼镜。

不过韩冈和章惇的理由,也不过是早上太过闷热,而宰辅们的待漏院中有冰降温罢了。

稍稍的寒暄之后,两人一时间没有了话题。厅中静了下来。韩冈安静的喝茶,章惇也同样低头喝着茶水。如果有人此时进来,看见这个场面,传出去,朝中又会是一阵鸡飞狗跳。

片刻之后,章惇咳嗽了一声,打破了尴尬。

“听说玉昆你有打算改动科举?”

韩冈点点头:“是有这个想法。”

“打算怎么改?”

“如果是别人问,我会以为是为了家中子弟。子厚兄来问,倒是不会有个误会。不过,子厚兄当真想要知道?”

章惇的两个儿子章持、章援,下一科就要参加科举了。以他们的才学,一甲二甲虽不容易,三甲还是有希望的。而以章惇的身份,想要事先得到部分考题的内容,同样不是难事,不过章惇的性格,绝不会为了儿子去伸手。

“是要废三经新义吗?”

韩冈摇头:“行事勇决上,韩冈比不得家岳,此事得日后再说。”

“难道是科目有变?”

“朝令夕改是朝廷大忌,礼部试和殿试已经改过了。至于诸科,条贯早已议定,又何须改?”

“那又有什么听不得?”

“是解试!”韩冈道。

“改成百分制吗?”章惇也是笑着问的。

“是。”韩冈点头承认。

“这不算什么。”章惇道。

礼部试改百分制,这是韩冈的创举。

也就是说,到了最关键的礼部试时,即便经义部分的错漏较多,也不会刷落考生。只要之后的策论写得好,照样能够得到高分,获得成为进士的机会。

这就给所有不属于新学的士子一个机会,不去学习新学,也能够成为进士。

对此,国子监中诟病很多,但不仅仅是其他学派的门徒,就是其他路州的贡生,却大多举双手欢迎。

比起国子监中长年累月的进行新学的熏陶,地方上的士子,却极度缺乏优秀的老师,很多人对新学的释义一知半解,这让他们很多直接就在经义部分中,便被刷落。若是经义折算成一部分的分数,有信心在策论上将分数追回来的贡生,数量可是不少。

最关键的一点,百分制后,题目分数比例成了关键,若是经义部分只折算成二十分,而策问部分八十分,学《三经新义》还有什么用,考官的倾向决定一切。若是各占其半,那没说的,经义谁也不敢放下。

不过韩冈没有这么做,而是采用了六十对三五。经义三十五分,策论六十分,之外还有一个卷面评分,字体和整洁度算五分。新学对此反弹的不是太厉害,而其他学派的士人,也感觉比之前进步一点。

礼部试和殿试都改过了,再改解试,其实不算什么。

考试内容和纲目不变,考试办法采取百分制。就算不再局限于进士科三次大考中的某一次,而是从地方的解试开始,也不是什么惊人的消息。自从礼部试和殿试,都采用百分制来评判高下之后,士人们也都知道会有这么一天。

“如果只有这一点。”章惇眼神深沉了起来,“那没什么。”

“下一科解试,我打算在经义和策论之外,再加考一项常识。”

“什么常识?!”章惇沉声问。

“《幼学琼林》里的常识。”

