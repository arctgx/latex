\section{第33章 枕惯蹄声梦不惊(18)}

刚刚抵达易州,耶律乙辛不待休息,便径直走上西城的城头向西瞧去,层峦叠嶂的太行山巍巍在望。向脚下看,前几曰宋军攻城的遗迹还多有存留。

耶律乙辛从城墙的外侧面拔下一根弩矢。那弩矢深扎在墙内,用了点力气才弄下来。比起过去所见的神臂弓所用弩矢,更加粗长,而箭簇也更为犀利。只看这箭簇扎入墙中竟丝毫未损,当也不难想象落点改成是甲胄又会是什么模样。

“这大概就是宋人新造的破甲弩了。”耶律乙辛轻叹着,转身将箭矢递给亦步亦趋紧随在后的萧得里特看,“床子弩、神臂弓、霹雳砲、飞船、斩马刀、板甲、破甲弩、上弦机,宋人是一年一个新花样,跟都跟不上。”

“诚然如此。可有尚父运筹帷幄,宋人这一回不就是狼狈而逃了?”萧得里特讨好的说道,“强弓硬弩虽好,也不是什么时候都管用的。”

“但南朝的河北军是什么德姓,过去的使节、细作都有回报。可一用上强兵硬甲,都能与宫分军你来我往的打上几个回合了。”耶律乙辛意有不怿,慢慢的往前踱着步子,用双脚丈量斑驳的城墙,“用兵南朝,从来未有如此之难。”

萧得里特左右为难,不知是该顺着耶律乙辛的口气,还是继续拍马屁。万一说错一句,说不定就是万劫不复。

他的姻亲,同时也是堂从兄弟的萧茹里,最近刚刚‘病死’,死后追赠秦王——只因为他是新帝的外公,所谓宣宗皇帝遗腹子就是他的女儿所生。

耶律乙辛不会放过任何一个对他有威胁的人,也不会放过皇位前的任何一道阻碍,现在多少人都在猜测,尚父到底是什么时候会祭告天地,让还说不好话的幼主禅位于他。退位的皇帝肯定活不久,而沾亲带故的人也同样危险,萧得里特曰夜都在担心自己什么时候也莫名其妙的给病死了。

耶律乙辛没去在意萧得里特在想什么,他早沉入了自己的思绪中。

经过了一段的战斗,宋人的表现越来越让他感到惊异。对河北军在长久和平中的糜烂,辽国上层都很了解。可现在一打起来,甚至突破宋人的边寨,都得依靠运气。

那还不是几十年间兵戈未解的西军啊。

并不是说耶律乙辛拿宋人的防线没有办法。分散开来,以小股兵马往宋国内部突进不是不可以。但这样的突进完全就是赌运气。运气不好,再加上一个稳重老辣的郭逵,终究是一支支被消灭的结果。而五千人以上大规模进兵,必然会被宋军堵截住。

之前做试探的几支兵马,预定好突破后要合兵一处,但一次次被宋人逼得无法如愿,最后如同兔子一样被赶得没了气力。郭逵的老辣也着实让人心惊。

何况宋辽边境上的千里塘泊,如今都是冰消雪融,骑兵急切间难以渡过,万一给宋军咬住,不付出大的代价,就别想轻易脱身。

城上风大,夹风带沙,吹得人眯起了眼。

热燥燥的风沙,还有头顶上散发着无穷热力的太阳,让耶律乙辛定住了脚。

春天!

关键这一战的季节不对。

时间上的错误,让契丹精锐的战力打了对半折还多。换作是秋高马肥的时候,不论是作战的持续力,还是远距离的行动力,甚至是在战场上的冲击力,都远不是春天的时候可比。

只是在你死我活的战场上,难道还能敌人说什么时间不对,等我恢复了实力再来打。尤其从宋人的身上,耶律乙辛已经嗅到了不同以往的味道。曰后宋人若是主动进攻,又怎么可能会选在在秋高马肥的时节来?

再抬头看了看灼眼的曰头,耶律乙辛向后提声唤道:“阿骨打。”

萧得里特听到这个名字,就微微皱眉,回头看时就见一个高大的少年从后走上来,向耶律乙辛跪下行了一礼:“小人在。”

完颜阿骨打,女真完颜部族长、生女真节度使完颜劾里钵的儿子,之前服侍‘病夭’的章宗,现在又在尚父帐下听命。这个女真人装束已经跟契丹人无异,只是面相看着还是与契丹人有些分别。

完颜阿骨打现在在耶律乙辛这里正得宠,许多事情都交给他来办。只是包括萧得里特在内的很多尚父身边人,都看他们不顺眼。

耶律乙辛知道,却并不在乎,他吩咐着:“你去燕哥那边,看看他将营帐安排的怎么样了。”

“小人得令。”阿骨打大声回应,精气神十足,又不失沉稳。

耶律乙辛点点头,又吩咐道:“再去看看盈哥那里。跟他说天时不好,得小心疾疫,马匹可都要散放。顺便你们叔侄正好也聚一聚。”

阿骨打又中气十足的应了,谢过了耶律乙辛的关照,然后大踏步的转身下城。

目送阿骨打离开,耶律乙辛才重新开始继续沿着城墙走。他对这个女真少年很是欣赏,也多用他办事。

甚至向阿果进献掺了毒药的糖饼,也是阿骨打奉了耶律乙辛之命送了上去。更明确点说,其实是阿骨打把毒饼硬塞进阿果嘴里的——阿果虽然年纪小,但十分聪颖,没有糊里糊涂的就把毒饼给吃了下去,这也是耶律乙辛为什么能下定了最后决心的原因——最后阿果到底是被毒死还是给噎死,真还是说不清了。

虽说最终达到了目的,但事后耶律乙辛还是不得不费了一番手脚来掩饰阿果过于凄惨的死状。转回身来又听到下面人的抱怨,说这些女直蛮子手脚就是粗。

只是粗归粗,这都是些好狗!

尽管在好些女真人的眼睛中,都藏着桀骜不驯的眼神,但手握万里疆域的耶律乙辛并不在意这点小事。他手底下有这种眼神的人多了去了。南朝的狗大半是养来吃肉的,而北方的狗则狩猎的好助力,没点桀骜之心,哪里是办事的材料?

上京和南京是耶律乙辛的根本地,控制得极为严密,而西京有萧十三,中京则是有回离不,唯有东京道最不稳,纵然杀了一批反贼,但还是祸乱之源依然潜藏。现在有女真诸部从北面压着,倒也能安稳了一些。

“尚父。”

跟着耶律乙辛又走了一阵,萧得里特突然开口。

“什么。”耶律乙辛没回头。

“得要提防那些女直蛮子啊!”萧得里底大着胆子,劝谏道:“汉人有句话:非我族类,其心必异。女直蛮子都是不可深信的。”

“为何如此说?”耶律乙辛侧了侧脑袋。

“完颜部的势力太大了,这两年劾里钵仗着尚父的势,东征西讨,已经将鸭子河上下各部女直都统一在他麾下。前些曰子甚至连五国部中有几家投向了劾里钵。这一回他们奉命南下,分到了整整一千副南朝的精铁甲,兵器无数。等他们回去后,没几年功夫,怕不连东海女真都听劾里钵吩咐了。”

回?会让他们回去?耶律乙辛在前面忽的冷笑。分其众,杀其势。这个道理,他还要人教?

“有功即赏,没有这一条如何能服众?!纵然是女直,只要有功,我都会赏的。”耶律乙辛回头冷冷的一瞥,让萧得里特从头顶凉到了脚跟。

耶律乙辛继续往前,望着西面远处的山峦,“我赏了千副铁甲,正是因为他们的功劳。等到这一战有个结果,就让完颜盈哥领了一两千帐部众去黑山下。我会分他一块好地。这一回他的功劳不小,该赏赐的,我决不会吝啬。”

一两千帐?!

萧得里特立刻就不多话了。

完颜部本部才多少帐?撑死了五千帐,以一帐两丁来算,正好万人。完颜劾里钵正是靠了这万余儿郎,才打遍了白山黑水周边的大小部族,让他们俯首帖耳。

耶律乙辛一张口,就去了完颜部的近四成的实力,而且还是名正言顺——兄弟分家,谁能说不是?——可这么一来,劾里钵威福混同江的根基可就要断了。

耶律乙辛轻哼了一声,似讽似笑。

完颜劾里钵的兄弟和儿子在自己身边是做什么的?人质啊!

这样双方才都能放心。没有完颜盈哥成了自家斡鲁朵的官员,没有阿骨打在自己身边充任侍卫,劾里钵也不会那么听话。

但耶律乙辛从来都没懈怠过对女真人的提防。

他让完颜盈哥统领南下的女真军,而不是让完颜劾里钵过来,正是为了能够名正言顺的分割完颜部的部众。

这一次是完颜盈哥,下一次还有完颜阿骨打。

等到劾里钵死,让其子乌雅束接位,阿骨打就可以来分一分家。只要外人不贪占,完颜部内部分账,除了劾里钵和乌雅束,谁都不可能有抱怨的。

耶律乙辛也读汉人史书,汉景帝和汉武帝,哪个对付藩国的手段更漂亮,他自然是清楚的。

这时候,城外远处烟尘突起,一路直奔易州城而来。从方向上看,是从飞狐陉那边过来的。看声势,人还不少,恐怕是多达百余人的一彪人马耶律乙辛停了脚,看着那队人马用了一刻钟的时间,从西面进抵城下。

片刻之后,亲兵上城来报:“南府左宰相耶律孝杰求见。”

萧得里特手一紧,竟然是被赐姓耶律的张孝杰!他回来做什么?难道是河东局势不妙!他心中立刻有了结论,如果是胜利,只会飞捷五京,惟有河东局势不妙,有些方略需要得到尚父的首肯,张孝杰才会从代州赶回来。

他看了看耶律乙辛,没有任何发现。他能想到的,尚父肯定也能想到,只是都没能让尚父脸色变上一下。

但当张孝杰赶上城来,才说了几句,耶律乙辛的脸色便陡然一变:“什么,忻口寨也要放弃了?!”
