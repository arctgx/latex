\section{第34章 为慕升平拟休兵(三)}

经过了连续敌烈度的实战,韩冈麾下马军的战斗力有了明显的提升。纵然距离契丹的精锐骑兵还有不小的差距,不过配上了神臂弓后,基本上在交锋时不会太吃亏了——除了损耗太大以外,倒也没有更多的缺点了。

韩冈翻了翻就在桌上的报告,那是主管军械的幕僚呈上来的。自他率军进驻忻口寨后,因各种原因损坏并交还武库的神臂弓已经超过一千具,而无法收回的更是另有四百具之多。在还没有大规模交战的情况下,如此之大的武器损耗,不是韩冈,换做他人是完全背不起的罪过。

也许有了火药武器后,会比神臂弓更节省一点。钢铁制品怎么也当比木制的弩弓更结实。

之前韩冈便收到了信报,关西那边已经有了利用火药兵器来守城的战例。除了装了火药的喷火竹筒外,还有了在箭矢上绑了火药来加大射程的成功记录。

据韩冈所知,宣抚陕西的吕慧卿对此表现出了很大的兴趣,甚至都向朝廷具表上奏,要军器监对此进行研究和开发。

论起对军器监的影响力,宰辅中只有吕慧卿才能跟韩冈相抗衡。绝大多数军器监中执行至今的法度,几乎都是吕慧卿和韩冈遗留下来的。而军器监——同时也包括性质相类似,人员交流频繁的将作监——的官吏和匠师,也基本上是两人提拔和重用过的。

现在吕慧卿除了上奏以外,还以枢密使的身份,让军器监的亲信去设法组织人手来进行火药武器的研制工作,想必很快就会有些成果。比如克制飞船的火箭、守城时竹火枪,都可以更进一步的加以改进,并通过朝廷的批准,而大规模制造。

等待已久的时机,可以说已经到了。尽管还有吕慧卿这个意外存在,可韩冈并不觉得他能在这方面能争得过自己,且吕慧卿也绝不可能有自己这样十足的信心。而在韩冈看来,保持一点内部竞争,对热兵器的推广也是大有好处的。

韩冈放下幕僚的报告。又打量了一下折可大,不过短短的十数日,他整个人就变得又黑又瘦,感觉都快要脱形了。

“下去好生歇上几天吧。”韩冈温言说道,“这些日子你累得也够呛,”

就算他这段时间把折家的继承人当做包身工来使唤,也知道压榨人不能太过分,要有张有弛才行。过于疲劳的情况下,人很容易犯错,韩冈可不希望折可大因此而有个三长两短——不论是从他的身份还是他的能力,韩冈都损失不起——同样的,骑兵军官中的大部分,体力和精力都到了极限,再逼着他们出动,反而达不到预定的目标。而现在休息一阵,之后也能够更好的表现。

折可大一楞,忙问:“那末将的差事谁来接手?”

“还有人啊。虽然不如你,但也够用了。”韩冈对折可大道,“你回营后只管好生休息,不要多操心。”

韩冈利用麟府军中骑兵军官来防止敌军的侵袭,并藉此练兵。折可大这段时间的确是累得如同死狗一般,心中一直想着能向韩冈求个恩典。甚至偶尔还会闪过一些阴暗的念头,觉得韩冈是不是想要对付他们折家,不然没道理让他这位府州折家的下任家主来冲锋陷阵。还连带着将一群来自府州,已经成为麟府军中中坚力量的军校,都拖进了危险的泥潭中。

不过随之而来的是功劳簿上逐渐累积的战绩倒也让他觉得还算是值得的,血汗流得有回报。眼下他甚至感觉工作已经渐入佳境。眼下韩冈不用人求,主动给他放了假,折可大一下就感觉自己是被抛弃了。

“枢密!”折可大忍不住声音大了起来,“末将虽不才,可营中能代替末将的人选,却不是那么多。”

“诚然,营中的其余将校最多也只有你的六成七成。不过秦琬那边已经将他手底下的代州兵训练得差不多了,虽说不能追着辽贼跑,但守住几个寨子是没问题。”韩冈耐着性子向折可大解释着。

折可大认识满心狐疑:“可怎么这么快?!才多少时间就练出来了?”

“代州兵本身就有底子,现在也只是重申号令而已。”韩冈对折可大这样的将才,一般情况下都是宽和得很:“他们许多都是投了辽贼的叛逆,如果不能证明自己已经痛改前非,这场大战之后,纵然我能保住他们的性命,可他们日后也别想活得多好。”

折可大不情不愿,可韩冈的决定他也不敢反对,低头答应了下来。

秦琬被韩冈所看重。但一名武官如果不能领军上阵,不立下让人信服的军功,终究是还是没前途的。

之前秦琬和韩信一同策反了代州降敌的官军,并率领这群反复不定的残兵败将,骚扰和威胁攻打忻州的辽军。甚至可以说,忻州城的保全他起了很大的作用,所以秦琬由此挣回了韩冈的一份荐书。等朝廷的回覆到了之后,就是正经的官人了。

只是再要往上升,还需要实打实的战阵上的功劳。那些说降、扰敌、临难不屈的功绩和行动,总不能吃上一辈子。就算韩冈不说,本身就出身在军营中的秦琬也知道要怎么做才能让他自己能够在军中立下根基。

秦琬的本部就是原本跟随他骚扰辽军的那不到三百人的队伍,韩冈这些天又从手下的代州军民中,选拔了一批合格的士兵,凑足了三个指挥一千两百余人,配属到秦琬的麾下。

这段时间以来,秦琬就在不停地操练着他麾下的士兵。配合他的副手,也是制置使司安排下来了,正是秦琬的熟人,同时也是韩冈亲信的韩信。

在折可大离开之后,韩信便奉命到来。

韩冈向他询问了一阵营中操练的进度,以及进驻废弃寨堡的准备,韩信都给出了肯定的回答。这让韩冈心情更好了几分,不过他立刻就想起了一件事来。

“对了,韩信。等朝廷的批复下来之后,你也该起个正式的性命了,你总不能一直用现在这个姓名字号。”

“没有枢密的栽培,就没有韩信的今天。小人的名号是枢密所起,当然也得用。”韩信语气诚挚,“何况能与淮阴同姓名,是小人的光彩。”

韩冈笑着摇了摇头。

韩信这一回立下了汗马功劳,韩冈于情于理都不能再让他做自家的奴仆,既然荐书都写了,当然得将他从韩家脱了奴籍。

仆从从主家脱籍出来,没改姓名的倒也罢了,改了的正常都是要回复旧姓名。不过韩信旧姓恰好姓韩,本名也只是个排行,不改其实也是无所谓。但韩冈对此很坚持。

这个时代的风俗习惯依然承袭旧唐,纵然在律法上,仆婢的人身安全已经得到了最基本的保障,仁宗时更是已经被编户齐民,视同庶民。可是在世人眼中,依然非是良民的身份,依然是贱籍。一名仆役入家中,你不给他起名更姓,他甚至就有可能会觉得你不把他当作贴心人看,也就很难得到他们的忠心。

不过这也是针对家中仆婢,当这些仆婢离开了主家,甚至得到了官身,当然就不能再维持现在的名讳。否则御史台那边肯定会兴高采烈的欢呼找到了韩冈犯蠢的机会。

“那是过去的事了。”韩冈坚持说道,“现如今既然已经是同朝为官,怎么还能用旧时做仆役时的名号,肯定是要改。”

韩信都跪了下来,缓缓摇头,“要是没有枢密的恩德,哪有韩信的今日。怎么能刚一得志,就忘了旧恩德。小人要是,回家后,爹娘也饶不过。”

韩冈无奈,叹道:“也罢。信字可以留着,不过还是得加个字才行。”

韩信神色一喜,高声道:“敢请枢密赐名。”

韩冈沉吟了一下,“守信二字本是最好的,不过自威武郡王【石守信】之后,太多人起了这个名字,反而就不能用了。”

韩信不知道威武郡王是谁,但他知道点头。

瞧着韩信老实等待的模样,韩冈笑了一笑,“老子有言:‘多言数穷,不如守中。’所守者,只是一个‘中’字而已。而我儒门,也说守中:‘中庸之为德,其至矣乎。’不便用‘守信’,不如就叫中信吧。‘信’字不变,加一‘中’字。韩中信!”

“可是中间的中?”

“嗯,正是。”韩冈颔首。

韩信大喜起立,端端正正的在韩冈面前拜倒:“多谢枢密赐名,从今以后,小人就是韩中信。”

韩信只拜了一拜,韩冈就拦住了她,“尊长赐名,一拜一起就够了。”

但韩信又坚持多拜了两拜,涎着脸笑道,“中信只是想请枢密赐下表字,一并凑全了好。”

韩冈指着韩信的鼻子,笑骂道:“你这狗头,倒是越学越惫懒了。”

“中信不敢,”韩冈对家人和气,韩信……应该是韩中信,面对韩冈时,说话也不是那么恭谨严肃:“只是秦小乙都能得枢密赐字。中信不才,自问不会输给他。”

秦琬的琬,是一种浑圆而无棱角的圭,所谓琬圭无锋芒,有藏锋含光之意,故而表字含之。

这是韩冈为秦琬所请而赠与的,故而让韩中信看了眼热。

不过在韩冈本人看来,含之这个表字都还是过于秀气了。只是他本来也没有起名的才华,这还是左思右想才灵光一闪的。不过,含之也有谦逊内守的意思在,秦琬是有些傲气的,韩冈赠以此字,也是希望他言行上能稍稍注意一点。

现在韩中信也要一个表字,韩冈皱起眉,头有点痛。想了一阵后方说道:“中,儒之守,信,将之德。你觉得这个表字如何?”

“守德?”

“嗯。”韩冈点了点头,正想更进一步解说一下,外面却传来了紧急通报。

韩冈稍一打听,就发现这是从太原传到了此处的。而太原的消息,则来自太行山以东、位于南京道的辽军。

“耶律乙辛遣使请和?”

信使刚刚点头,几名军官便鼓噪了起来,“辽贼果然请和了!这一战是不是就这么结束了。”

韩冈摇头,事情没那么容易完!

如果掌控辽国的不是耶律乙辛这个权臣,而是地位稳固的皇帝,根本就不会有这么。可惜耶律乙辛为了稳固自己的地位,维护自己的威信,让事情的发展一步步走到了最坏的一条路上。这是最大的错误。权臣掌控朝政本来就不是名正言顺,不论做得多好总会有人反对他。

现在看来,萧十三终于是明白过来了,不趁此时定下和约,日后可就有的苦头吃了。

“朝廷多半不会答应下来!”

“为什么?”韩中信奇怪的问道。

韩冈冷然怒哼一声:“兴灵方面的损失如果不算的话,辽国在河东、河北两地的伤亡,加起来还不一定超过一万。而光一个河东路,代州、忻州、太原三地,军民死伤就是数以十万,财产损失更是难以计数,二三十年都不一定能恢复元气。就这么连声抱歉都没有就完事了?哪有那么便宜的事!”

“那我们该怎么做?”韩中信沉声问。

韩冈安安稳稳的喝着他的茶,这是刚刚随着春衣一起从京城送来的。

放下茶盏,他慢条斯理,话声中杀机隐现:“他们谈他们的,我们打我们的。”
但韩信又坚持多拜了两拜,涎着脸笑道,“中信只是想请枢密赐下表字,一并凑全了好。”
