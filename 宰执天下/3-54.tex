\section{第20章 廷对展玉华(下)}

王安石是昭文馆大学士;王韶是资政殿学士;章惇是直学士院;吕惠卿因为是新近起复,也是担着集贤校理一职。

大宋左武右文,受天子看重的朝官,甚至京官,身上都会带上一个文学职位。韩冈现在得了一个集贤校理,也总算是向外确认了他受到看重的程度。

不过韩冈自锁厅后,现在还没有一个差遣。赵顼并没有明说集贤校理究竟是虚衔,还是正式的职司,必须要确认一下。他躬身谢道:“陛下所赐,臣感激涕零。惟臣不擅文学,实不敢当……”

“此是贴职,非是馆职。”

贴职是兼任,而馆职则是正任。韩冈自知才学深浅,他需要一个文学职衔的名头,却不方便去崇文院整理文章、卷宗,而赵顼也明白这一点,才点明了这是贴职。

韩冈放心下来,恭声谢过天子的恩赐。此时天色已晚。赵顼说了一个下午,看起来也有了几分倦意。韩冈看得明白,就打算先行告退。

但赵顼却,“在殿试上的卷子,这两日朕又看了几遍。将熙河、秦凤军政之事说得鞭辟入里,也可见韩卿你在西事上用心至深。”

“陛下求直言,臣不敢有所隐瞒,自是尽所知而言。”

赵顼悠悠的点点头:“即是如此,还望韩卿能‘尽所知而言’。”

韩冈略低下头,知道说了一个下午,终于到戏肉了。将简单的觐见,变成了廷对,看起来今天自己给天子的印象还不错:“……敢请陛下垂询,臣当知无不言,言无不尽。”

“新法如今已经推行了五六年,成果是有,但反对声也从未断过。不知韩卿是如何看待?”

‘果然还是此事。’

韩冈无意在新法上多言,皇帝不是蠢人,倾向太过明显,免不了会被怀疑他是在‘亲亲相隐’。日后想要帮王安石说话,在天子的心目中,也站不在公正的立场上。必须要将赵顼关注的焦点,转移到自己可以说、方便说的议题上。但天子既然问了,就必须给出一个确定的回答。

稍稍组织了一下语言,韩冈道:“商鞅变法,步过六尺者有罚,弃灰于道者被刑,秦人岂不怨?!”

他一开口,便说着变法的不是。步过六尺、弃灰于道,此等小事都施以刑罚,都是被历代儒家批烂掉的苛政。

但赵顼想的透,韩冈的这一句,不过是上承苏、张的纵横术而已。顺着话头下来:“但秦因此而兴。”

“陛下说得正是!”赵顼接得恰到好处,让韩冈也方便往下去说,“秦人之所以能并吞六国,一统天下,便是靠着商君之法。而商鞅立法严苛,无事不至,又岂是会为了让道路上保持洁净?那是为了让秦人自日常时,便惯于依从号令,上阵后对军令不敢有所依违而设立。”

他见着赵顼点头深思,进一步的又道:“其实就在这宫掖之中,也有如商鞅立法之严苛者。”

赵顼听了一惊,立刻追问:“此人在何处?!”

韩冈一拱手:“臣曾听闻近年来,宫中夏日无蝉鸣,不知可有其事?”

赵顼恍然,放松了下来,改容而笑:“此是殿帅宋守约之功。”

宋守约,他自熙宁二年担任殿帅后,便对守卫京城和宫室的殿前司诸军大加整顿,号令森严。甚至下令军中,到了夏天,必须将宫中的知了全都赶出去。若让他听到一声知了叫,就是一顿军棍大杖伺候。京城之中多有传言,说宋守约厌恶蝉鸣,所以有此号令。

“以臣之愚见,宋殿帅岂是恶蝉鸣?直是为了教训士卒,使诸军不敢违抗军令。”韩冈加重了语气,“宋殿帅行事之道,与商君立法一脉相承。”

赵顼点头:“当日朕也问过宋守约,他道‘军中以号令为先。臣承平总兵殿陛,无所信其号令,故寓以捕蝉尔’。”

“蝉鸣难禁,但宋殿帅能去之。若日后陛下有命,诸军又何敢不从?!”韩冈高声断言。

“果然是‘天下智谋之士,所见略同耳。’宋守约亦是如此说。”赵顼笑道:“他若听到,当引韩卿为知己。”

“宋殿帅总领天下禁军,岂是微臣可比。”韩冈谦虚了一句,前面一段话造势已成,下面就该说正题了:“商君禁弃灰,殿帅止蝉鸣,此二事岂不严苛。可秦因此而兴,而今之禁军,陛下亦能如臂使指,此即是二法之功。故此可知,法无分善恶,须相其时,待其势而用之。”

“……可时势如何能定?”赵顼皱起眉头,仔细想了一阵,抬头问道。

有此一问,韩冈知道天子已经被说服了大半。他的论述其实有些牵强,但援引赵顼身边的实例为证,说服力因此而大增。

“商君之术,争于六国时,为善法。抵定天下后,为恶法。宋殿帅之令,若于战时,军心不定之时,必当会引起兵变;而放在如今的太平之时,却是教训士卒之良策。法之善恶,是否依循时势,是要从目的和结果来评价。如新法例,都是权衡利弊,乃可施行。”

“那以韩卿观之,如今新定诸法是否依循时势?”

韩冈当然不能直截了当地说是或是说否,必须从他最为熟悉的领域着手:“均输、市易二法,施行于京师、东南,臣无从知晓,不敢妄言。但在秦凤、熙河,保甲、将兵二法,使军民堪战;便民、免役二法,使黔首安居;农田水利在巩州淤灌良田千顷,此诸事,都是韩冈亲眼所见……”

韩冈将自己在关西的见闻娓娓道来,内容当然要比两千字不到的殿试策问要丰富得多。这一席谈,虽然免不了偏着新法,但说的有理,以可算是持平之论,让赵顼十分赞赏。至少对新法,在西北地区的推行多了许多信心。

赵顼很是看重韩冈,能给他带来如此多收获的年轻官员,现在也就韩冈一人。三年来,韩冈的种种功绩,却只付出了一个太常博士和集贤校理就打发了。就像家里招的佃户,只留其他佃户一半的收成,却能提供五倍、十倍的租子,有哪一个佃主会不喜欢?要是国中朝臣都如韩冈一般,使得四夷宾服当非难事。

不过这样的回报也的确微薄了点。学成龙虎艺,卖与帝王家,幸好如今天下贤才,也就一家可以卖。若是放在战国、乱世,这样的付出可留不下人才。

做了五六年的皇帝,赵顼早就明白不会有人无缘无故的忠心于自己。要想臣子继续为国效力,必须给予恰当的回报。这是维护家国稳定必须要遵守的基本规则。而将有才能的臣子放到合适的位置上,也必然可以得到最好的回报。

只是要给韩冈合适的回报却很难。

到了朝官这一级,本官的品级高低已经不是很重要了。就如王安石,现在才是正三品的礼部侍郎,远不如在外面任州官的文彦博、韩琦等人,可谁能说他不是礼绝百僚的宰相?

重要的是资序!

而韩冈的资序实在太浅。做官的时间,满打满算才三年。想在朝中用以要职,冠以‘权发遣’的头衔,都还差了一两级。

当初王安石设立三司条例司,吕惠卿、曾布、章惇等人遽升高位。可他们被人称为新进的时候,其实已经做了十来年的官,进士的资格都熬老了。想把才三年资历的韩冈安排在高位上,在河湟很容易,在其他地区就难了,而在朝中更是难上加难。

有功不赏,当然是有失公正。可将资序不到者提至高位,日后却必然会有人援此为例。到时候,功劳什么的就不会有人提了,只会看到入官三年就可以晋为朝堂中的高官显宦。

但这件事没必要跟韩冈本人说,等琼林苑结束之后,跟王安石商量一下,再提也不迟。

“卿家可多多请对,朕也欲常见卿家。”廷对终于还是到了结束的时候,赵顼最后对韩冈嘱咐着,颇有依依不舍的样子,“当初张载在京城的时候,朕曾对他这么说过,可惜他很快就请辞了。”

“家师根究天人之道,无意于宦途之上。不过教书传道,亦是为国作育英才。”韩冈本人站在这里,当然就是最好的证据。“近闻经义局编修经书,直追经义本源,一改汉唐旧释。韩冈不才,愿以身保家师入经义局,无论删定修纂,注疏释义,当不辱于朝廷,不愧于陛下。”

机缘巧合,赵顼提到了张载。韩冈便不会放过这个机会。他要推举张载入朝,就算不能为官,也要在最近就要成立的经义局中占个位置。

虽然韩冈马上就要与王安石的女儿成亲,而方才说话,也是尽可能地帮着新党。不过在学术上,他就不可能站在王安石那一边。一道德虽然好,但要是让气学无法登于朝堂之上,韩冈就不能认同。在这一点上,他绝不会妥协!

赵顼则是沉吟着,一时竟无法决断。

