\section{第12章 兵蹙何能祓鬼傩(上)}

在邕州城外的高高挑起的旗杆上,一颗扣着头盔的头颅正在广西冬日的暖风中腐烂。

苏缄在正对面的城墙上停了脚步,好用早已经变得老花的眼睛,看着已经变形发黑的那个东西:“桂州那里如果再派援军来,最好是个有胆识有才智的。”

跟随苏缄一起巡视城中的三子苏子正忙坐了下来,靠着墙,整理起套在身上的札甲。甲身里面用来绑扎的皮带断了一根,还没拿去修,松垮垮的,穿在身上走几步就会歪掉。‘也不知道州中军器局的几个修补匠有没有空。’苏子正想着。在他父亲立下的规矩中,工匠们首先要紧着城头上的守军。

“援军不会来了。”通判唐子正低声的说道。其实也不用压低声音,现在邕州城中谁都知道,桂州的援军不会来了。

旗杆之下,四五十个拢做一堆的头颅,堆了三四十堆,一群群苍蝇嗡嗡的绕着乱飞,乌鸦也围了一圈。七天前,桂州派来的三千援军,连同主帅张守节,在昆仑关处全军覆没,只剩下脑袋到了邕州城,也能算是来了。不过全都便宜了乌鸦和苍蝇。

“真是过了一个肥年。整整三千人呐!”唐子正这几天脚都跺得痛了。就算都是废物,但也都是拿着刀枪弓弩的废物。城中现在人不缺——城中丁壮分作三班上城,就在身后城墙根下的屋子里,都躺着刚刚下城的守军——就缺乏足够的军器。

“最多两千……”苏缄从文职转武职,由进士作武将,很清楚广西军是什么样的情况。只是想起来军中那些弊病,手就要压着心口,话都不想多说。如果广西路中军州兵籍都是足额,贼军的攻势也不至于这么顺利。

唐子正倒不在意空额不空额。“刘经略派他们出来,不是为了防着有人查他们空额的。”

“这些天交趾人死得更多。”苏子正终于将他的盔甲调整好了,走上前来,城外的贼军看起来有了些动静,他专神的看了一阵,发现一里外的前营营寨中,似乎是有些异样,只是远了点,看不清楚,“只是不能出城斩了他们的首级下来,可惜了那么多的功劳。”

苏缄没有儿子的遗憾,功劳是战后才计算的,先要保住城池才能去谈功劳,连升三五级、七八级、九十级,也不是什么好事。“都是神臂弓的功劳。”

“毕竟是神兵利器。”唐子正对苏缄的未雨绸缪感佩不已,“多亏了皇城从京城里催着发了下来。”

苏缄现在只恨自己没有厚着脸皮多要一点,这东西虽然好,但遇上激烈的战事,损耗也未免太大了一点。‘这南方的天啊’,苏缄忍不住要叹气,钢刀、重弩、铁甲,再好的军器到了广西,都存不了多久:“神臂弓还剩六百二十四具,箭矢的数量也不多了,得省着些用。”

“神臂弓用的木羽箭,不是不需要翎羽,要省很多材料吗?”

“木羽箭是便宜,用着薄木片做翎尾,也就一文钱一支,比起白羽箭都要便宜许多。就是邕州城里面没那么多造箭矢的工匠,也没法儿铸范。”唐子正认为这场战事结束后,自己去军器监也够格了,一支箭矢几文钱,他张口就能报出来,“铁料用民家的铁锅也可以、库中也不缺牛皮来制胶,就是缺匠人,缺炭火。不能铸范,箭镞都造不了。”

“库中还有八万多支,一张弩也就一百多支。只能等贼军射进来时,在地上捡着用了。”城外的贼军也有神臂弓,就在他们竖起旗杆的同时,也拿出了两百多具,估计是一个指挥的数目。“真不知道是谁家的援军。”虽然领着援军的张守节已经自食恶果,可苏子正还是忍不住心头的气。

“桂州城中的神臂弓也没多少,最多一千五,那还是经略司的治所。”苏缄上京时与韩冈的结交,让朝中配发下来给邕州神臂弓没有在周转中耽搁,可桂州就没有这么好运气了。

“被射杀的贼军也有两三千人了,他们撑不了多久。如果援军还在,说不定都已经回师了。”唐子正回头再看了一眼张守节的头颅,一只乌鸦在乱蓬蓬的头发上跳着,嘴里不知叼着个什么东西,仔细看看,忽然发现眼眶空掉了,“去年见到张守节,看着也是个一幅豪杰作派。没想到竟是个胆小如鼠的人物,如今又落个此等下场。”

“此人外强中干。”苏缄不想再看,转身就往前走。守城的军士纷纷行礼,对这位老人礼敬有加,是发自内心的崇敬。围城的这些日子来,苏缄的表现,邕州百姓都看在眼里。

“如果他没有在昆仑关处逗留不进,及早来援,李常杰反而腾不出手。只要在附近扎下城寨,掎角之势即成,交趾人早就退兵了。”苏子正还再回头望着。

说起让邕州城彻底孤立的罪魁祸首,唐子正与苏子正就有了共同语言:“的确是自作自受。刚开始的时候,哪一家贼人愿意离开富庶的邕州,去与援军对拼,丢了当先入城的机会,不知损失会有多少。那时候李常杰最多派点人去看着昆仑关。”

可是到了屡攻不下的时候,李常杰反而要用那三千援军来提振士气。

自从邕州被围,刘彝派出了广西都监张守节率领三千兵马,敢来援救。但张守节是个胆怯无能的将领,在路上磨磨蹭蹭。苏缄等不下去,派人带着包有求救信的蜡丸,连夜潜出城去,去找广西提点刑狱使宋球——经略刘彝,苏缄是不敢信了。而五天后,援军的消息就传来了。

从后来的城外喊话,苏缄他们用了一番功夫将整件事拼凑了起来。张守节逡巡不进,害死了随他出战的将士,也毁了邕州等待外援的希望,让人对他都没办法同情一星半点。

用竹牛弯角号角声被吹响了,伴随着战鼓,抬着长梯,又是数千交趾军涌了上来。护城河的水被引走了。只要将木板一搭,就可以直抵城下。啄食着腐肉的乌鸦乌压压的飞起,“又是一批来送死的!”苏子正盼着这样的进攻多来几次,死得多了,贼人自然就要退了,“交趾人毕竟还是不擅长攻城。”

这一次的进攻瞄准了东南角。城头上也响起了锣鼓,就在城下休息的守军收到了被攻击的信号,争先恐后的冲上城头。苏缄心中更安稳了几分,“军心尚在,邕州城当能稳守。”

“皇城,下官先过去指挥了。”唐子正告罪之后,匆匆忙忙的往那边赶过去。走得快了,能看得出他的左脚有些跛,前些日子被一枝流箭射到了左腿上,到现在也没有完全痊愈。

“多亏了有他。”这些天来,苏缄的副手表现出来了足够的军事才华,而且临阵更是奋勇,哪里还是文官,根本是最出色的武将。

有了唐子正的指挥,加上英勇奋战的士卒,邕州城东南的攻防战优势明显的就在守军一方,两轮神臂弓齐射,就让交趾人的攻势立刻被压制了下去。

苏缄放心的转身要下城,今天城池可保无恙。

尖利的号角声从另一个方向上传来,苏缄和苏子正的脚步停了。苏子正两步跨到外墙边,只见交趾人从后方的前营营地中正推着一辆辆的车子出来,缓缓的逼近了邕州城。

交趾军推上来的车辆有四个轮子,一条长梯斜斜的从车上架起。这样的车子,只要靠上城墙,就是一道登城的阶梯。比起在倚在城头上的竹制长梯,强了不啻千倍,而这样车子竟有十五架之多。而在云梯车之后,是一辆辆仿佛移动房屋的四轮车,车上顶棚是厚厚的牛皮。交趾的士兵就藏在牛皮下面避箭。

“那是云梯车!还有攻濠洞子!”苏子正一向想学着他父亲的稳重,但看到交趾人推出来的攻城器械之后的反应,还是差了苏缄一筹。拳头用力捶着城墙,“什么时候贼人会打造攻城车了?!”

“将油抬上来。”苏缄不慌不忙指派着,这么大的岁数不是白活的,世上已经没多少事能让他惊讶了,“桂州的援军里面,只怕有人投贼了。”

论起攻城守城,只有宋人最精。四方蛮夷连提鞋的资格都没有。不是得到了宋军俘虏的襄助,只凭交趾的技术,怎么都不会知道该怎么打造适用攻城器械。

一桶桶从城中搜集来的油,被提上了城头,堆放在云梯车可能会靠上城墙的地方。邕州城头上这些天来都用着烧滚的金汁向下泼,而苏缄刻意扣下油料,就是为了预防出现眼下的情况。

手持神臂弓的精锐部队,也一同上了城头。

咚、咚、咚的几声闷响,随着云梯车一辆辆靠上城墙,交趾士兵就从后面的攻濠洞子里冲了出来。窜上云梯,就要往城头上冲。

“倒油!”

苏缄一声令下,一桶桶油就立刻浇了下去,沿着云梯向下流淌。黑瘦矮小的交趾兵,还没有反应过来,宋军下一步的反击就到了。不需要苏缄再下令,谁都知道下一步该做什么。

一支支点燃的火把丢了下去,更大上千倍的火炬升了起来。就靠在城墙边上,十五架火炬的火焰升得比城墙还高,刚刚跳上云梯的交趾士兵在火焰中打着滚,凄厉的惨叫让乌鸦们都吓得远远的飞走。火势蔓延,连着攻濠洞子一并都陷入了火海。

守在城头上的弩弓手都不放弃这个机会,用着神臂弓或是其他弓弩,点杀着纷纷逃窜的背影。移到城墙边,“贼人技止此耳。”苏子正哈哈大笑,贼人的惨叫让他心怀大畅,没有比这个更好听的音乐了。。

只是苏缄眯起了眼睛,却是怎么也笑不出来。

