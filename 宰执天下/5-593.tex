\section{第46章 八方按剑隐风雷(13)}

就像是暴雨前的雷鸣,由远及近,一双硬底箭靴重重的踏在才铺好了地板的廊道上。

京东东路兵马钤辖杨从先,正阴沉着脸,大步向前走着。

充满血丝的双眼,剔起的双眉,还有紧紧按着腰间宝剑的右手,无不在向人说明他已经怒到了极处。

新任高丽国相金悌追在杨从先身后,大呼小叫,“杨将军!杨将军!请息怒,还请息怒!”

杨从先充耳不闻,前面有宫女、侍卫,但一看到他直欲择人而噬的模样,无不吓得手颤脚颤,纷纷闪到一边,竟没有一个敢于站出来拦住他。

只有金悌奋力跑着,好不容易方才追了上来。他喘着气,说不出话来,探手扯住杨从先的袖子。等气息稍稍平复,他忙急急的开口:“将军,将军,有话好说,有话好说!”

杨从先低头看着扯住自己衣袖的手,沉默不言。

金悌忙不迭的将手抽回,陪着笑脸,“将军,这实在不关鄙国国主的事,都是下面办事不利。再两天……五天,呃,十天……半个月,半个月之内定能有回报!”

杨从先的回答是抬起脚,然后狠狠的踹了出去。

轰的一声巨响,通向居住高丽新王行宫的大门被一下踹开。

杨从先横了呆若木鸡的金悌一眼,大步跨了进去。

这座供给高丽新王的行宫,本来是耽罗国的星主借给来此避难的高丽君臣。多年的宗藩关系,让一群破落户也得到了礼遇,而这一位新近加冕登基的高丽王,住进来的第一件事,就是改造宫室,要有个上国的体面。

耽罗岛乃是蛮荒小国,是与高丽一水之遥,不得不对其称藩称臣。其国主号曰星主,已在岛上传承了数百年。早在王氏高丽立国之前,耽罗国对半岛上的朝贡就开始了。如今每代星主接位,都会去高丽觐见高丽国王。

这一回一群丧家之犬加上千余名残兵败将来到,陆陆续续又有得到消息的高丽旧臣,纷纷渡海南下,在耽罗国借来的港口和寨堡中,拼凑起了高丽国的新朝廷。

被他们所拥立的高丽新王王勋,不是因为他的血缘与国主有多亲近,而是他叫做王勋,与已经死在辽营中的世子同名。正好可以用来凝聚高丽国中的民心。他们远在大海对岸,难辨真伪,有此误解,也能让更多的义士投奔而来。

杨从先没有去掺和高丽逃亡朝廷的内政,他一直在整修港口,打算将耽罗国,变成大宋水师在海外的基地。而且还要防止辽军会紧随而来,也必须要做好军事上的防备。

当他还沉浸在不断进展的土木工程中的时候,一个噩耗传来,占据了高丽的契丹人欲壑难填,竟然渡海东进,又派兵去攻打曰本。

而他们出兵的数目……十万!

杨从先清楚地记得在韩冈接见自己的时候,曾经提到过占据了高丽的辽国有攻击曰本的可能。

既然韩冈之前都当面提过了,现在事实又印证了,不管之后韩冈是不是又自己否定掉这个可能,是不是只是随口一说,事后就忘了个干干净净,杨从先明白,他都得把契丹入侵曰本,当成韩冈事前已经叮嘱过的事,自己现在没办好,那全是自己的过错!

混迹在官场几十年,杨从先即便是武将,也不会不知道如何对待上司。

高高在上的宰辅,他们说出来的话,他这个跑腿的只有当圣旨捧着。不对的得忘掉,说中的就得吹捧,有点擦边的更是得拉到先见之明上。决不能明白的指出错误,那样比老实认罪的结果要凄惨十倍。

只是现在就算是想要老实认罪也不会有好下场,不说提前侦察到辽军过海侵略倭国的动静,就连辽国出兵的数量都没打探到。朝廷还有可能原谅他这个无能的水师大将?

第一次是那两个昏了头的倭国僧人说的,可以当成被吓疯了之后昏话。可这一回,高丽王王勋依然告诉他,辽人的确出动了十万兵马渡海,还以此为理由,声称辽国在高丽国中的兵力空虚,要求朝廷派出大军为他复国!

杨从先恨不得掐死这位无能又愚蠢的高丽王,辽人要是有这本事,又有那么多人马、船只,早就大闹江南了,去曰本那个鸟不拉屎的地方做什么?石头里攥油,不嫌硌手吗?

当时他硬是咽下这口鸟气,出来后遣人一打探,方才知道,派出去的探子根本就没回来,完完全全是王勋随口说的数目。

要是他将这个数字报上去,那他就死定了。在官场上,不会有再有任何前途可言。

章惇也好,韩冈也好,杨从先接触过的这两位宰辅,都是最看重下属的能力,眼里从来不揉沙子。完全没道理的消息,哪里可能骗得过去?就算自己想要编造一个数字蒙混过关,可他军中,还有朝廷派出来的走马承受,他可不会跟着一起发疯。

这个差事已经办砸了,那两位还能给自己几次机会?

杨从先不敢想象,要是他把这个十万再送去京城,自己的两座靠山会怎么处置自己。

现在落到耽罗岛上苟延残喘,可以说是非战之罪——他抵达高丽的时候,离开京略近一点的外岛,多半被辽人攻占了,而那些没有被攻下来的,都是小而贫瘠,且无险阻,根本不能作为根据地。只能一路南下,来到高丽唯一的藩属国落脚。

而之前没有探查到辽军过海,报给朝廷的辽人过海数目,也同样情有可原。从高丽去曰本,并不用经过耽罗岛。而那个十万,毕竟是来自于从九州岛上刚刚逃出来的僧人,而不是自己的臆测和谎言。可纵然如此,他也注明了,那是倭国僧人所说,真实数目他正在着力打探。

可现在呢,寄希望于高丽人在对面半岛上的耳目,希望他们能打探出真实的数量,以弥补前过,偏偏愚蠢的王勋想要骗朝廷出兵,还是咬定十万。没有完成朝廷交托的任务,甚至连最基本的军情都不能掌握,他杨从先,还能将责任推到其他人的身上吗?还能说这不是我的过错,都是王勋的错?

朝廷中宰辅们不会管那么多的,留给他的评价,只会是扎扎实实的无能二字。

留给他杨从先的时间已经不剩多久,他和他麾下的兵马都是外人,不可能潜入高丽,去打探辽国到底有多少兵马渡海,也不可能分兵去曰本,去人生地不熟的九州岛打探详情,只能借重高丽流亡朝廷在其本国中的势力。

但即位的高丽国王王勋却是个废物。他会排斥异己,会残杀兄弟,就是不会卧薪尝胆,破釜沉舟。想要依靠这个暗杀了几个族亲后,便安稳的躺在后院中去享受他的后妃的无能国主,等于是将自己的未来抛进茅坑中。

不能再这么下去了,杨从先在打听清楚之后,便立刻领军来到了高丽小朝廷的行宫中。

跟在他身后,有上百人的卫队,只有金悌敢于追上来,而其他臣子,都不敢上前半步。

杨从先心意已决,却又不是金悌可以改变的。

高丽国王的行宫寝宫只是一个不算大的院落。

杨从先闯进了院中,将脸皮彻底撕破。

听到消息的王勋匆匆忙忙出来,衣裳不整,脸上还有没擦尽的胭脂痕迹。

“杨将军,怎么过来了?……相公,还不替孤招待一下杨将军!孤先进去更衣,稍待就出来。”

见杨从先气势汹汹,王勋立刻软了脚,说着就想将责任推到金悌身上,自己转身就想走。却被杨从先的护卫一把给揪住,一左一右,牢牢的架了起来,押到了杨从先的面前。

王勋挣扎着:“杨将军,这是做什么?”

“谢谢大王你啊。多谢大王通报军情。”杨从先语气阴森,就像外面的积雪,丝毫没有一点暖意,“张口就是十万兵马,要朝廷趁机出兵。要是本将将大王你的话传到京城,你让章枢密、韩宣徽怎么看我杨从先?!”

“孤知道错了!孤知道错了!孤这就再派人去打探。……前几天,孤还跟相公说了,要封将军你做郡王。这耽罗国就封给你了。要是不够,五道两界,君可任选。杨将军……杨将军!”

杨从先看到他这个样子,更是心头火起。高丽群臣拥立这个王勋之后,亏他还在同时递上去的密奏中说此人堪用。

明明只是同名而已,只是个宗室,却要摆出正牌子的高丽世子即位的作派。王勋要大修宫室,杨从先便要求耽罗星主去征发民夫,为其改建府邸。王勋要仪仗,杨从先也分了些盔甲、兵器给他。当时还是觉得,这是正当的要求。没有一点的威严,怎么让来投的忠臣义士相信有复国的可能这些事,杨从先都容忍了。

杨从先恨得想撞墙,为了这个无能的废物,他生生的浪费了两个月的时间。之前来到岛上的还有另外几名宗室,却都在一夜之间死了个干净。为了大局,杨从先当做没看到。要是他们还活着,谅王勋也不敢放肆到这般地步。

“金大使。”杨从先没再理会王勋,转身面对金悌。

金悌自从扶了王勋登基之后,便被封为宰相。但杨从先等宋人,从来没称呼他过一次相公。

“想必大使你也明白,朝廷派本将来此究竟是为了何事!如果做不到,本将也只能回京请罪了。”杨从先寒声说着。跟随在后的亲兵,各个眼露寒光,更是杀气腾腾。

“难道不是匡扶正统,救我高丽危亡?!”

“朝廷要的,是想要复国的高丽国主,而不是窝在房中玩女人的国君。不复国倒也罢了。就连打探消息都做不到。朝廷还有这样的废物做什么?还不如没有?!”

“杨将军,你到底想要如何?”

“换人。”

杨从先的要求就这么简单。

