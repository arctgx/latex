\section{第二章 天危欲倾何敬恭(六)}

素净惨淡的正旦已经过去了。

大宋及其诸多藩属国的亿万子民,终于辞别了动荡不安的元丰四年,迎来元佑元年的新春。

经过了半个多月的时间,纵然依然还是在丧期之内,但赵顼的死已经在民间沉淀下来。提起来叹几声,不提的也没人去挂念,该过日子的还是过日子。排除掉离奇的死因,那仿佛杂剧中的故事,剩下的,也不过是十几年来,又一个驾崩的皇帝。

对宗泽来说,虽然身边还时时有人提到,但也不是需要特别关心的事了。

今年的进士科举将照常进行,迫在眉睫的礼部试,更让如他这般的贡生感到紧张,并无余暇去考虑无关紧要的问题。

除了吃饭睡觉,宗泽恨不得将所有时间都投入到学习中去。

“汝霖,狗肉炖好了,要吃吗?”

门外传来邻居周文璞的声音。同为考生,周文璞也没时间出门,不过他有个好伴当,能在寺庙里借锅烧狗肉。

“狗肉啊。”宗泽摸了摸嘴边的燎泡,钻心的疼,推门而出,向周文璞带着歉意道:“多谢宗琳兄美意。不过今天是不行了。不知是不是这段时间常吃狗肉吃的,火气太盛,燎泡总不见好,得上吃几天清淡的。”

“你也不吃,伯才兄方才也说吃不了了。难道让小弟一个人吃不成?”

“还有多少?”

“做了实验的有几十只。说是毒死的,都不肯吃,你看看,多浪费?!”

“是浪费。”宗泽附和着,他是不在乎,出身江南人,河豚都吃过,还怕被毒死的狗肉吗?

“说实话。小弟乡里多山,山中多猛兽。野猪常见,大虫也不少。那些被猎人用药箭射杀的野猪、大虫,拿出来有谁会不吃。药箭上涂抹的可都是见血封喉的剧毒。一点烟气算什么?烟熏的肉吃得多了。”

周文璞扯住宗泽絮絮说了一通,方才告辞离开,多吃、多话,这样算是他放松考前心情的一种手段。

宗泽也是一样,所以能体会周文璞的心情。他自己能调节,但有时候与朋友一起大吃一顿效果会更好一点。尤其是最近,与周文璞一起把狗肉吃得太多,几乎都吃伤了。

之所以近日常有狗肉入肚,倒是多亏了《自然》和韩冈。

太上皇之死,离奇古怪。照常理,京城中多会为此产生谣言,而且肯定会不利于太后,或说她暗害上皇,或说是她护持不利,让别人暗害了上皇,然后将事情推到才六岁的小皇帝身上,甚至可能会与太祖皇帝之死联系起来。

但韩冈的出面,却让谣言没有了传播的土壤,给太后和宰辅们减轻了无谓的压力。

那并非是空口白话的辩解,而是一个有关炭毒的实验。

在新一期《自然》期刊中设计出来的验证炭毒的对比实验,是拿着最常见的狗冇来做试验品,一次就要同时做三组。

同样箱子,同样放在箱子中的暖炉,同样大小的狗。其中两个箱子被密封,一个箱中暖炉的烟气能通到外面,一个烟气则排在内部,而三个箱子中的另外一个,则是箱子本身不加密封,有洞来透气。点燃暖炉后,将这样的三个箱子放置一个晚上,到了第二天早上打开来查看结果。

据宗泽所知,这一对比试验,很多看到这个实验的人都饶有兴致的做了。狗不值钱,去肉摊上买几只很方便。箱子、暖炉等实验器材,同样很容易就能弄到手。与传说中太上皇的死因相关,对此有兴趣的人很多。

通过事后的交流,实验的结果也出来了。

没有密封的箱子中,狗都活着。而密封的箱子中,暖炉也不通气。而暖炉对外排气的密封箱中,狗有的活着,有的则死了。

另外有一点很多人都注意到了,全然密封的箱子中,实验动物死时基本上都是无声无息,动静很小,而暖炉对外排气的密封箱里面,死掉的狗,全都是在临死前经过了一番剧烈的挣扎。

在这一期《自然》中,在公布实验的同时,也向世人征询其原因和原理。

尽管暂时还没有人能够阐明其中的原理,可不管怎么说,有了这个实验打底,很多人对朝廷所公布的赵顼死因,也就没有太多的疑心。即便朝堂中还有人认为上皇之死是被谋害而不是意外,但在市井上,基本上已经没人再怀疑了。

实验总是很有意思。可之后收拾残局却很麻烦。被毒死的狗肉没多少人愿意吃,偶尔有个周文璞这般的老饕愿意承接,却也是凤毛麟角。被丢弃的试验品实在是让人觉得可惜。

订购全年的《自然》期刊,能成为当年的皇宋自然学会的通讯会员,而想成为正式成员,必须有超过三篇论文在期刊上发表才行。一旦成为正式成员,便能够得到一枚徽章和一份证书,同时不用再订购期刊,直接由学会免费寄送。

吃了点清淡的饭菜,宗泽准备午后去不远处的图书馆一趟。上京时能带在身边的书籍并不多,但省试在即,许多典故、文章都要在考前重温一边,免得到了考试时都给生疏了,去图书馆看书就是最好的方法。而且还能见到不少士人,相互切磋一下,学艺也能有所进益。

整理了一下行头,宗泽正准备出门,就听见外面一片噪杂。

走出小院,只见一群人从前殿走了过来。主持和尚在人群中点头哈腰,不知在说些什么。

宗泽在人群外远远的看过去,突然惊讶的瞪大眼睛,人群正中冇央竟然是韩冈。

这一位怎么来寺庙里了,难道是来烧香的吗?可传言中不是说他对浮屠一向没好感?

一个个问题从脑中泛起,宗泽一时间都忘了要去图书馆。

韩冈慢悠悠的走着,一边与身边的王厚说着话,一边里里外外打量着这间寺院。

他可不是来烧香,而是打算来拆迁的。

正谄媚的说着奉承话的胖和尚脸上堆满了笑,韩冈带着恶意的想着,如果他将内情告诉这位一身狗肉味的主持,不知还能不能笑得出来?

皇宋大图书馆理所当然的要设在城中,但京城之内,寸土寸金,每一片可以利用的土地都是有主的。想要修建一栋新建筑,就得推倒另一栋旧建筑。

如果要修皇宋大图书馆,至少要十数亩的土地,要征用大片的地皮。如果是民间的土地,朝廷也不能强夺,就得花钱买下来。

图书馆以教化黎庶、普及文事为由而设立。其建筑结构,依韩冈的想法,从木结构改成砖石建筑,以防火灾。里面的藏书更是得以十万计。要花的钱很多,可现如今,朝廷穷得叮当作响,从铸币局出来的钱币,转天就能给用出去,肯定是拿不出钱来买地。不过幸好开封府名下就有土地,比如这一间寺庙,是开封府中选出来的几处适合的地点之一,这前后数进左右皆有偏院的寺庙,其地皮不是庙中的产业,而是开封府名下的土地。

“玉昆,你觉得如何?”

王厚上了三柱香,为亡父祈求冥福,然后问韩冈。

韩冈摇摇头,“离开封图书馆太近了。”

他更希望两家图书馆能离得稍微远一点,这样才冇能能普惠更多人。

“近又如何?”

“我这个馆长难道不要担心买卖好坏?人气多了,才能聚才啊。”

韩冈好象是生意人的口气,但王厚知道他想聚的是什么‘才’。

“说起来这馆长的称呼真的不怎么样,怎么不起个好点的官名?”

“可以了。难道还能是提举皇宋大图书馆?”

“不带使职,不加学士,就是一个光秃秃的馆长。朝廷也真拉得下脸皮的。”

“朝廷能同意设立大图书馆就可以了。没听过善财难舍四个字吗?要是跟韩、蔡二位相公说一下,用建图书馆的钱钞,换个大图书馆使的虚名,你看他们愿不愿意。”

朝廷前日已经下令要在东京城中成立皇宋大图书馆,面向普罗大众,由刚刚卸任的前宣徽北院使、资政殿学士韩冈出任馆长。不仅仅是王厚,很多人对图书馆没有起一个更有蕴意的名字颇有微词,但韩冈的态度则是越直白越好。

前后走了一遍,韩冈也没注意到寓居在其中的考生们,跟王厚道:“换个地方吧。再走走。”

王厚哪里会有反对的意见?与韩冈一起出来,上了马,向另一处要查看的地点赶过去。

“对了玉昆,方才就想问了。”王厚在马背上问道,“早间在驿馆里面听人说,日冇本败了,向辽国割地称臣。每年要缴岁币白银百万两,黄金十万两。这事是真是假?”

“日冇本败了是真的。”

辽国这段时间在日冇本高歌猛进。来去自如的骑兵战术,让数百年只在岛屿中上进行小规模战争的日冇本守军吃足了苦头,两次会战据说皆以全军覆没告终,被攻下平安京或许只是时间问题。

“不过缴纳岁币百万两银,十万两金就纯属胡说八道,把现在的日冇本朝廷卖了都没那么多钱。”

王厚啧了一下嘴,“我就说嘛,果然是胡扯。”

“若辽人当真夺取了日冇本,当地的金银矿藏便能为他们所利用,若是全都能挖出来,说不定还真能有那么多,至少白银不会少。”韩冈又补充道,“只是那些矿藏还都埋在地里呢。”

在前一期的《自然》中,因为辽国入寇日冇本,所以里面有几篇地理文章都与日冇本有关。其中韩冈还化名写了一篇,以铁、铜、银、金为例,说金属分轻重,越轻的在地表越多,越重的地表就越少,都沉在地底深处。只有火山从地底喷发出岩浆,能将地底的矿藏给喷出来。日冇本多火山,理应金银矿居多。

这是明摆着下套,让辽国尽量多的输送兵力去日冇本,反过来逼迫朝廷向海军加大投入。也因此,日冇本多金银的谣言也传了出去。

