\section{第20章 冥冥鬼神有也无(七)}

【下一章在十二点。】

“……先是两队铁鹞子左右来攻,只是微臣提前一步列阵,有着神臂弓在手,一旦齐射,贼军即便十倍于我,也不能近前……”

“战到中途,皮室军自背后袭来。微臣幸有所备,立刻收紧阵势,背依山林……”

“……就在微臣列阵的周围,是上百辆翻倒在路上的粮车。皮室军和铁鹞子,想要越过粮车直攻微臣本阵,都冲刺不起来。”

“不过他们还是人多,为了不将这一队皮室军吓跑,当他们冲出来之后,前阵就改用陌刀自护,后队的神臂弓还是多往党项人那里射过去,只用几十人压着皮室军的马弓。”

折可适的口才远远比不上桑家瓦子里说书的丁三四、刘合万,可赵顼还是听得津津有味。朴素的言辞,仍能让大宋天子有着身临其境的感觉。

折可适带在身边就一个步卒指挥,要抵挡三路骑兵,而且其中还有一支是兵力相当的皮室军。这样的压力,就算有了一点布置,也是危机四伏,赵顼听了都为折可适他们感到心惊。干咽了口唾沫,攥紧了的手心都冒出了汗来。

“皮室军果然还是上了当。”折可适的声音高亢了起来,“一见箭矢射得稀疏,就一批批的冲了上来。等到他们越过阵前,藏进山中的弩弓手,就开始齐射火箭焚烧粮车。”

对了。赵顼想了起来,还有一开始逃进山中的弩弓手,他们伪装成民夫,就是为了让皮室军落入陷阱中,

“两百多步长的道路上,一下子全都是火焰和毒烟。虽然烟被风刮散了一点,但马匹身处烟火中根本都呆不住,皮室军一下全都乱了。微臣也趁机率部退到背后的山上,与之前的弩弓手配合,封住皮室军从烟火中冲出来的道路。”

“山爬上乱箭齐发,而臣又领队绕道了大路上堵着,等到援军赶来,皮室军已是瓮中之鳖。最后这一部来攻的皮室军,就只有三十余骑逃了出去。连同主将萧药师奴,共计四百一十二骑尽数授首。”

其实在这其中有许多疑点。就算烟火再大,也不可能完全封堵住道路。以之前皮室军一贯的表现,开始时最乐观的预计也只是能给皮室军三四成的伤亡,功绩的大头还是在铁鹞子身上。但这一次却是铁鹞子先逃出来,他们出来后,皮室军却是多拖了有半刻钟,这让援军得以先行一步赶到战场。

而且铁鹞子逃出来后,根本就没有援救皮室军的意思,否则至少能救出一半。另外只看两队铁鹞子的伤亡人数和皮室军完全不成比例,就知道他们多半早做好了准备,说不定在一开始给皮室军腾出进攻的位置时,就已经看破了官军所用的计策,可他们并没有提示皮室军。

但这话就不能对外说了,根本没有证据的事,只是猜测而已,说出来只会亏了一众拼了性命的袍泽兄弟和自己。折可适瞒下了最后一段党项军的异动,其他则是依照着事实而说来。

“中国有精兵强将在,皮室军看来也不过如此而已。”赵顼碍于宰辅们都在场,不便放声大笑,但他心中是得意非凡。

虽然是用了计策才打赢的,但折可适以区区五百步卒,力敌三路精兵而不露败相,本身已经说明了官军的战斗力。而且既然郭逵和折克行敢使用这个计策,也证明了官军的表现决不是偶然,而是将帅们公认的事实。

现下在河东、陕西的整体局面上,都是官军彻底压倒党项人。种谔、王舜臣在葭芦川大捷,稳固了朝廷对横山的控制;郭逵夺回了丰州,让府州保住了屏障。

“皮室军虽强,遇我精锐却如土鸡瓦狗。臣为陛下贺!”

王珪踏前一步,手持笏板一揖到地,向着赵顼高声恭贺,让大宋天子眯起眼睛不住的点头微笑。

王珪上前讨好天子,其他几名宰辅却都有些冷然。

此战的确是大捷,此前王安石也带了群臣一起恭贺过天子。可小觑契丹、党项却是还早得很,得给天子泼盆冷水。不仅吴充这么在想,王安石、吕惠卿等人也都在这么想。

“如今官军气势如虹,与党项军交战直如摧枯拉朽一般。只要再等数载,等国中禁军全数配上铁甲、陌刀、神臂弓这样的神兵利器,而陕西、河东的粮秣又加以备足,便可以收复银夏,夺回兴灵!”

王韶冷水泼得委婉,赵顼就只听到了后面的两句,开怀笑道:“对,就要收复银夏,夺回兴灵!”

“陛下!”吴充冷水泼得激烈了一点,“皮室军有十万之众。另又有宫分军以十万计。一战斩首四百,也只是伤及皮毛罢了。且皮室军受创,以北朝睚眦之性,如何会干咽下此事。河东、河北、陕西要早作防备,以防契丹兴兵来攻。”

赵顼笑容渐渐的收了起来,点头道:“此事不可不虑。”

王安石对吴充的话不以为然,辽国绝不可能就此撕毁澶渊之盟,但眼下却是要泼天子冷水,也不便出言驳斥。而且公开撕毁澶渊之盟不可能,但私下里绝不会少做手脚。

皮室军近乎全军覆没的消息,辽人应该收到了,就不知道他们会做出什么样的应对了。

……………………

听说萧药师奴在丰州全军覆没的消息,皮室军左部详稳耶律兀纳一阵头晕目眩,一口血就喷了出来。

他悉心挑选的四百多精锐,竟然无一得还!

“药师奴那个废物!”他在房中抱头痛叫。

四百六十多皮室精骑啊!

太祖皇帝创皮室军时为数三万,到了太宗时,皮室军大加扩充,号称三十万。但如今皮室军的宿卫之职被宫卫军取代之后,分屯地方,五京道加起来也不过五六万,西京道这里更是只有不到一万。一下损失了四百余,而且是最让人无法容忍的全军覆没,一个都没跑出来。

耶律兀纳心火直上,几乎要五脏六腑烧成焦炭。猛咳了一声,鲜血咳得到处都是,“药师奴那个废物!”

“来人呐!”耶律兀纳也不管嘴边、胸前的斑斑血迹,站起来大喝着,“传令各部计点兵马,且听本帅号令!”

虽然这一次惨败完全是自找,但插手宋夏之战的对错与否对耶律兀纳来说并不重要,重要的是如何安抚下皮室军所部,以及该如何对宋人进行报复。

不得上命而直入宋境深处,那是绝对不行,但在边境打个草谷,发泄一下火气,却是一点问题都没有。这点权限,作为左部详稳,耶律兀纳却是有的。

“慢着!”耶律兀纳的副手萧引吉从门外走进来,“此事还是急报魏王,说不定魏王正等着这个消息。”

听到萧引吉提起耶律乙辛,燃烧在心头的火焰猛然更加旺盛,耶律兀纳现在可不在乎耶律乙辛,但萧引吉话中透着的隐义,却让他头脑冷静下来,“此话怎讲?”

萧引吉坐下来:“自从南朝天子任用王安石秉政以来,南朝一日强过一日。即使是皮室军这样的精锐,也挡不住两三倍有着飞船、板甲、陌刀和神臂弓的禁军。南朝的户口要比大辽多得多,也更为富庶,打造得起装备。一旦两国交战,大辽虽不说必败,但要获得一胜,也不知要投进去多少条性命。”

耶律兀纳脸色一点点的黑了下去,他可不想听这些屁话。

萧引吉坐得却是更加安稳:“相对于我大辽,而西夏更弱了。不论是铁鹞子还是步跋子,现如今一对一也胜不了南朝的铁甲禁军,这几年更是都没有胜过一场,败得一次比一次惨。此前夺占丰州,也是运气居多。南朝一旦回过神来,立刻就能夺回去。”

耶律兀纳的脸黑得更厉害,眼神也更加危险。党项人的胜负,又与他何干,一气陷了四百皮室精兵才是大事。

萧引吉仍在说着:“如果大辽不去支撑西夏,几年后,上京道在黑山就能看到南朝的巡卒了。”

耶律兀纳的神色变了,萧引吉说到了这个地步,他也不可能听不明白,“难道魏王一开始就是打算日后全力支持西夏,才要我点起一队兵马去丰州。”

萧引吉点点头:“想必魏王早早的就看到了南朝日渐强盛这一点,所以才派了药师奴去。如今经过丰州的一战,国中的每一个人都该看清楚了……也许这就是魏王的想法。”

“也就是说,为了这个理由,魏王拿着我麾下的四百六十多名儿郎去送死。”耶律兀纳声音低缓了下来,隐隐的蕴藏着巨大的愤怒。

“并不是让他们去送死,而是试探,是想确认猜测是对是错。”萧引吉解释着:“只是不幸证明了魏王的猜测。这也是让国中早一步醒悟才不得不去做的!至少为大辽多挣了两年的时间。要是等到宋人开始攻打西夏,我们这里还没定好是旁观还是插手,那时候,就当真只能眼睁睁看着南朝的巡卒杀到上京道的边境来了。再过几年,说不定就能用钱将阻卜诸部都买了过去,送钱送兵器,西北路招讨司可镇得住他们?”

面对默然无言的左部详稳,萧引吉进一步道:“阻卜诸部现在用着骨箭都已经很麻烦了,当他们有了铁箭之后,上京道……不,大辽的国中局面又会变得如何?”

