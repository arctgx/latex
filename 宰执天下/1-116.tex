\section{第44章 文庙论文亦堂皇(一)}

【凌晨第一更,求红票,收藏】

自王府出来,李舜举回宫缴旨。来回跑了十几趟的苦活,终于有了个还算圆满的结果,他总算可以松下一口气。

从左掖门入宫,又穿过了两重门,回到崇政殿前。李舜举这时脚步一停,吃惊的看着御史中丞吕公著从殿中退了出来。

御史中丞的地位不是一个小小内侍可比,李舜举连忙避到一旁,躬身行礼。吕公著则眼睛也不瞥一下,视若无睹的径直走过去。

直起腰,李舜举回头看看走下台阶的御史中丞,心底一点疑惑升起。能让御史台的长官在入夜前赶入宫中,难道说出了什么大事不成?还是说要弹劾谁?

想到这里李舜举便摇摇头,暗骂自己糊涂了。以如今的朝局,吕中丞要弹劾人,除了王安石还会有谁?!

……只是从官家的态度上可以看出,即使要牺牲对两代天子皆有殊勋的元老重臣,他也要把王安石给留下来。连韩琦都没能做到的事,吕公著恐怕更不成。如今王安石的地位,并不是御史中丞能动摇得了的。

‘大概是豁出去了。’李舜举猜测着。

吕公弼、吕公著兄弟俩,一个是枢密使、一个是御史中丞,同居高位已经有半年了,朝中年前便有传言,最多一个月,两人中的一人就要出外,甚至可能是两人一起外放。既然出外已成定局,也没什么好顾忌的,不趁最后时机弹劾王安石,还要等到何时?!

可惜现在都是无用功!李舜举暗暗摇头,虽然他不看好变法派的日后,但眼下,王安石的确是稳如泰山。

得了通传,李舜举进了崇政殿,跪下叩头行礼,将王安石终于领旨的结果回禀。可他说完,却发现赵顼并无因此而露出欣慰之情。皇帝的脸色很阴沉,一如当日刚刚看到韩琦奏章时的模样。

李舜举在赵顼身边服侍了不短的时间,所谓御药院,名义上说是管理宫中药方、药品,其实则是天子最为贴身的侍臣。赵顼露出了这样的神色,李舜举心知,多半又是哪里出了什么事。

“李舜举。”

“臣在。”

叫了声名字后,赵顼陷入沉默。李舜举低头跪着,静静的等待。好半天,赵顼才又开口,“近日京师内,可有什么传闻?”

李舜举偷眼看了看赵顼的脸色,比方才还要阴云密布,一如夏日午后即将爆发的雷霆雨暴。他心里一颤。若在平日,说些圣君明皇的马屁,再找两个市井趣闻说一说,引赵顼一笑也就过去了。但今天,怕是不会这么容易就能过关。

赵顼想听到的传闻,李舜举明白。即便他不愿意,他也不得不搅和进如今两派相争的朝局中:“多是关于王参政请郡之事。”

“……除此之外呢?”

“……”李舜举不知赵顼想问什么,想听什么,也就不清楚该说些什么,脑袋有些发懵。他是勾当御药院,在天子身边听候使唤,跑跑腿而已,并不管皇城司下面的探事司。京城内的流言蜚语,该问勾当皇城司的王保宁才是。

“关于青苗法、均输法,京中有没有什么怨言?”赵顼见李舜举张口结舌,不快的追问了一句。

“这……微臣近日虽是多出宫城,但皆是去王安石邸宣诏,并不敢在外多耽搁。”李舜举斟词酌句,力图使自己撇清一切干系,“关于青苗、均输二事,也只是稍稍听到一点议论,若说怨言却是称不上。”

李舜举知道分寸,有一说一。又不是有资格风闻奏事的御史,怎么敢乱说话?在内侍省中,他本就是以谨言慎行而被提拔起来的。但他自幼入宫,朝堂之事了解甚深。以过往的经验,李舜举并不看好王安石和变法的结果。

王安石得罪的人实在太多了,外臣姑且不论,宫里面,曹太皇、高太后可都对他没好感,宫外面,宗室们也是骂声不绝。

世间都说王安石是开源而不节流,因为他说过天子在自己身上多花点钱没什么。但李舜举知道,王安石实际上对冗官、冗兵、冗费的三冗下手从来不软。改革荫补制度的任子法和改革军制的将兵法都在筹备中,而针对占去朝廷财计差不多一成的宗室开销,现在也因为新的宗室任官法,而缩减了许多。

在仁宗朝,权相吕夷简为了与范仲淹相争,刻意拉拢宗室子弟,不论亲疏都封做环卫官,领着一份俸禄,使得本来就已经捉襟见肘的财计,更加入不敷出。宗室们的大饼,不论后续的哪一任宰相都不敢轻动。但王安石上台后,第一刀就斩在宗室子弟身上。他修订了宗室任官法,使得五服之外,便不再归入皇亲,不列宗谱玉牒,纯粹的外人了,当然就不用再给他们发俸禄和赏赐。

这对朝廷和主管财计的三司来说是求之不得的美事,但对于那些挨到王安石那柄名为缩减三冗的砍刀的人们,却恨得咬牙切齿。每天进宫向太皇太后和太后哭诉的宗室,从来没少过。

只是赵顼这次第突然又问了起来,却不可能是哪家王公又跑来哭诉。天子心意已定,连韩琦韩相公的奏章也没有效果,谁来哭都没用。

那就是吕公著说了些什么了——但李舜举想不出,吕公著还能拿出哪桩事,比起韩琦的奏章还要引起天子的愤怒……和惊惧?

赵顼无意识的把玩着御桌上的墨玉镇纸,眼神也是漫无目标的在桌上晃着,李舜举的回话也不知听没听到。又是半天的沉默过去,他才慢慢吞吞的问着,犹豫不决的轻声细语中所吐出的词句,却是石破天惊:“有没有传言说……韩琦欲行尹霍之事?!”

李舜举差点惊得都要跳起来,一颗心脏先是骤然一停,继而就像重鼓咚咚咚的在胸腔中用力捶响,清晰的传进耳朵里。冷汗也是刹那间冒了出来,全身都被汗水湿透。平日还算灵活的舌头僵住了,声音带着颤:“尹……尹霍?!”

尹霍就是伊尹和霍光。伊尹是商初贤相,因即位为王的商汤嫡孙太甲昏庸暴虐,便把他放逐到桐宫三年,待其悔改后,才又迎回;霍光是汉武帝任命的辅政大臣,亦曾废立天子。两人都是权臣中的权臣,虽然在历史上,他们的名声都很好。可是,有哪个皇帝会希望自己的朝堂中有伊尹、霍光这样的臣子?

‘这是要让韩琦灭门吗?!……吕公著方才该不会说得就是这事吧?’李舜举心惊胆颤,吕公著之父吕夷简早年与韩琦算是政敌,但也没闹到要让人家破人亡的地步,不过是吵吵嘴,拿着弹章互相丢着,怎么会在这时候……

‘不!’李舜举突然间灵光一闪。一点传闻动不了韩琦,三朝元老的韩琦从来没少被骂过事君不恭,心怀悖逆。富弼也被人说过欲行尹霍之事。两人不都是平平安安的做着他们的元老重臣?应该还是为了王安石和新法吧?

李舜举心中揣测着,一时忘了回话。他的沉默让赵顼不耐烦起来,声音陡然拔高:“李舜举!!”

勾当御药院、入内内侍省都知被吼得浑身又是一颤,心道回去肯定要在御药房中找些惊风散、平气药什么的吃上几斤,小命都快吓没了。他忙高声回道,“此事必是无稽之谈,微臣委实没有听说。韩相公事君以忠,为三朝元老,陛下切不可以对传闻信以为真!”

“你也没听说啊……”赵顼像是放松了一点,只是神色依然阴郁。

就在刚才,他下诏慰留王安石,并命政事堂和三司条例司逐条批驳韩琦的奏章后,御史中丞吕公著便赶入宫中,上奏道:韩琦三朝元老,朝中军中皆是威信甚著。如今其不满新法,奏章又被批驳,难免有尹霍之事。京中近日亦有传闻,恳请天子下旨穷究。

表面上看起来这是吕公著在尽自己风闻奏事的权力。可想深一层呢?以韩琦的身份,这种传闻跟本撼动不了他,而且也听得多了。但却是在引导赵顼去思考传闻出现的原因,是不是因为百姓心中有怨,才有了这样的期盼——目的依然直指王安石。

吕公著是在危言耸听,这一点,赵顼知道。但他却还是因此而忧心忡忡,不是因为担心变法是否祸国殃民,而是担心起自己的皇位来。

太皇太后、太后都不支持变法,两个弟弟又都住在宫中,前朝宰辅也是众口齐声的反对,万一他们真有个心思,他还能坐在崇政殿里吗?

在御榻上坐得久了,虽然日夜辛劳,但这掌控天下的权力的滋味一旦尝过,便没人肯再放下。赵顼也不可能例外。

因为这件事,他连王韶的万顷荒田变成了窦舜卿口中的一顷四十七亩都没心思去计较了。若是自己被废了,天下千万顷良田都不再是他的了,西北边境上的万顷荒田又算得了什么?

一名小黄门这时进殿通报:“官家,王安石在外求见,言说入宫谢恩!”

“快请他进……”赵顼犹豫了一下,改口道:“就说朕已安歇了。让他明日照常上朝便是。”

