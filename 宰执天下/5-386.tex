\section{第34章 为慕升平拟休兵(18)}

敌军将至的号角声在大王庄和小王庄之间此起彼伏。

敌军来袭的通报惊起了许多尚在酣睡中的辽军官兵。

最少的也在大小王庄睡了半个月,基本上所有的士兵早就知道相隔不远的宋人迟早会来攻打营寨。

大小王庄的驻兵只有六千。但通过这六千兵马,辽军便牢牢封锁住了通向代州的道路。任何想要进攻代州的军队,不解决这批骑兵就无法顺利穿越过去,而且随时都有可能受到寨中守军的突袭。

而辽军以大小王庄为中心,向外四散出击的游骑,不仅将之前很是猖狂的宋军骑兵给打回了原形,同时进一步加强了此处守军对道路的封锁能力。

这就是一颗钉子。让宋人想要攻打代州就不得不拔除的钉子。

久候方至的敌军也没能让他们多吃惊一点,甚至有不少胆大的还有打哈欠的余暇,然后不紧不慢的穿戴好衣袍和甲胄。不过这样的镇静,只在他们看到了宋军的阵列之前。

乌鲁被吵醒的时候,虽有几分担心,但还能记得自己的责任,不让下面的儿郎看出他这个主心骨心中的动摇。可是当他出帐一抬头,就看到了直上云天的浓黑烟柱,心中便是咯噔一下,知道事情不对。再等他登上望楼,向着北方一张望,立刻就将之前的小心翼翼抛到了九霄云外,愤怒的大吼了起来:“什么不到一万啊。那些探马的眼睛都瞎了吗?!”

南面列阵而待的宋军如山如海,一座座方阵横亘在两座隔道并列的营垒前。阵势两端不断有新的兵力汇入战场,并向大小王庄包拢过来,在他的位置上只能依稀看得到头尾,那已是远在里许之外。

宋军的营垒和辽军一样,基本上都是用被毁弃的村寨重新整修而成。其最为靠前的一座营地,距离大小王庄只有七里,其他营垒相互之间的距离也差不多,都是一般的乡村间隔,或多或少,并不是整齐划一。相互之间鼓角相闻,一鼓声落,一鼓便起,远近声闻,震天动地。

又有数以千计的骑兵,一部分守在军阵的两翼,为大军护翼,还有一部分如恰分散开来的群狼,恣意游走在广阔的战场中,以遏制从大小王庄出击的辽军骑兵。

完成了布阵的宋军阵势有着宽大的正面,且阵型又厚实无比,怎么看也不会与大小王庄的驻军兵力相当,是更多!而且是多得多!

放眼望去,少说也有两万多,若是几座营中还有足量的守军的话,那就是三万人了。

三万啊。

整整三万啊!

三万只苍蝇飞起来都是惊天动地,这三万人难道是从地里钻出来的?

“眼睛都瞎了啊!”乌鲁愤怒的冲着大王庄的方向大声喊。

……………………

耶律道宁并没有听见那个被风卷走的呐喊,但他有着相同的想法。

他摊开了手掌,掌中尽是汗水。

并不是耶律道宁惧怕已经近在眼前的宋军主力,而是这样规模的军队竟然在自己眼皮子底下出现,自己竟然毫无所觉,这样诡异的情况让他感到心惊胆寒。

莫说是连人带马,就是两万只老鼠、蛤蟆,这一路跑过来,也不会这般无声无息。他派出去探马,虽然大部分被拦截在外,可渗入宋军防线的那一小部分,不论是出身何处,年长年幼,有无经验,都视异口同声的确认宋军大营中的兵力也就最多一万而已——至少在他们探查时如此。这样的情况下,如何需要怀疑?耶律道宁也的确是确信不疑的。

但现实就在眼前。

不论是这几天宋人暗中调兵,瞒过了所有自己派出去的探马;还是在一夜之间就把这些士兵都调来前线,都证明了宋军的作战潜力绝不是表面上的那一点,很多地方的实力超乎了想象。

在这样的情况下宋军开始进攻,那他当然要做好最坏的准备。放弃一切耽搁时间的通信手段,甚至不敢再停下来多观察一阵,使用速度最快的烽火,用最快的速度通知到代州。否则一个不好,就有

可能全军尽墨在此处——谁不也能保证宋军会不会再玩出什么花样来。

耶律道宁心丧若死,他的任务是阻截并清理宋军的探马,同时探查宋军的动向。现在他没能做到,让宋人的大军像是从地底钻出来一般。不论他有多么委屈,但这个罪责必须要由他来背。

只是耶律道宁更清楚,他现在绝不能乱,否则援军赶到也来不及了。

“慌什么?”他吼着手下还在震惊中的人们:“不是早就知道宋人会来攻打营寨了吗?只要守住一个时辰,一个时辰后,援军就能到了!”

……………………

仓皇失措的号角声一声接着一声,自大小王庄惊动了整座代州城。

四十里外点燃的烽火,由三座烽火台接力,半刻钟之内便传到了代州。

晨光下,代州城的西门吱呀呀的打开。仅仅推开了一半,便有一名骑兵从门中一跃而出,紧跟着又是一骑。一名接着一名,一队接着一队,最后多达五百人的骑兵如飞一般冲出了代州城,直奔前线的大小王庄而去。

那一队一人三马,区区五百人,却有着千军万马的气势。正是萧十三特意挑选出来的精兵,作为全军的前锋。

也只稍稍迟了一刻钟,为数更多、几近千人的一支骑兵也冲出了敞开的城门,紧随前一支队伍的脚步,直往南方奔去。

之前为了应对宋军可能会有的攻势,就是连赶去援救的前后次序都已经做好了安排,只是萧十三没想到会这么快就给用上。

萧十三正身在城中的校场内,目送了第三支援军的出发,而在他身前则聚集了为数更多,总计三千兵马,接下来就是他亲率面前的这一支主力出发。

接下来代州城中的兵马都会分部分批的陆续赶往前线,在四个时辰之内,两万骑兵就可以全数抵达战场。可以在短时间内在局部战场聚集更多的兵力,大辽傲视宋人的地方,不仅仅是单兵的武勇。

只不过萧十三此时也很迷惑。

按照道理,宋军在前线上的兵力与己方相差不大,如果要进攻的话,很难在短时间内攻破大小王庄。甚至宋军在离开他们自己的营地后,就要时刻警惕一支兵力相当的精锐骑兵的突击,很难放手施展。

所以才会有五台山的那支奇兵。细作传回来的情报,正好证明了韩冈对大小王庄的驻军无可奈何,只能另寻他途。

按道理耶律道宁不应该这么慌张,甚至等不及派人回返传信——四十里路,不惜马力的急速飞驰其实只要半个时辰——连代表最危急情况的烽火都放了出来。这意味着什么,萧十三有诸多猜想,只是都觉得说不通。

“就算韩冈想要暗渡陈仓,但这个栈道修得未免太好了一点吧。只凭那区区一万兵马就把烽火逼出来了?”张孝杰同样百思不得其解,在大军毕集的校场上还是在考虑这个问题,“难道是宋军突然增兵?”

萧十三摇了摇头:“忻口寨离大王庄可有一百里啊。”

张孝杰沉默了下去。

从忻口寨到宋军前线的几座军寨差不多在一百里上下,这一段路程,正常行军要走两到三天。骑兵纵然可以一天跑下来,但第二天战马也很难恢复气力再行上阵。何况大举增兵的行动,怎么可能瞒得过前线的耶律道宁?宋军还在修那条破烂的道路啊!

萧十三也想不通,但他懂得放弃,现在没时间让他玩猜迷。

“想那么多也没用,去看了就知道了。”

该做的应对早已事先都约定好了,直接赶去前线就能一清二楚。不论宋军是怎么逼得耶律道宁放起烽火,他都有信心将之击溃。

就在烽火点起的一个时辰后,萧十三及其亲领的三千精骑,紧随在先一步出发的两批援军之后,赶到了大小王庄附近。

迅疾的驰援让萧十三和他麾下骑兵的坐骑耗尽了气力,甚至有口吐白沫,直接倒毙于地。但此时已经无人顾及。

就在他们的眼前,宽达近十里的战场上,旌旗如林,阵列如海。宋军的战鼓敲得一声声雷鸣。以万计数的宋军战士填满了眼前的空间,几十辆床弩和霹雳炮从营地中推了出来,轻易的封锁了庄中守军出击的通道。

骑兵给堵在寨中难以脱身,而之前的两批援军,却被宋人的骑兵拦在了庄外。其中有一半不类宋人装束,却冲锋在最前,萧十三只用了千里镜一张望,脸色顿时变得更阴沉。

“是阻卜那群贱种!”

阻卜人只是小事,真正让萧十三惊诧莫名的是宋军几乎是在一夜之间就变出来的大军。什么时候前线多了这么多宋军,他之前根本就没有听到任何消息,也没有得到任何报告,根本毫无征兆的出现在他的眼前。

“这不可能!”