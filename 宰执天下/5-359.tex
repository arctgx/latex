\section{第33章 枕惯蹄声梦不惊(16)}

“这不可能!”留光宇叫了起来,“郭逵宿将,怎会败得如此轻易?!”

“元章……”韩冈深深的看了太原府通判一眼,留光宇回过神来,连忙请罪,只是苍白的脸色一时间变不回来。

韩冈回顾信使,沉稳的态度像是根本没听到噩耗,“正如留通判所言,郭仲通乃是宿将,用兵稳重,岂会轻易败阵?即便是败,也只会是攻易州不克,河北败不了的。”

“枢密神机妙算,的确是在易州那里败了。现在官军又退回了保州,由郭枢密亲自领兵镇守。”

韩冈闻言,就对帐中的几位幕僚笑道:“你们看,我没说错吧?”

但没人跟着笑。韩冈门下客哪一个不知道,攻打易州的主将可是他的表兄李信啊!

韩冈收起笑容,又问信使,“究竟是怎么败的。我那表兄虽然官卑资浅,但行事最为稳重,怎么会为辽贼所乘?”

“辽贼的兵力比预计的多得多,甚至多了一倍。耶律乙辛那狗贼东京道的兵马都给他调入了榆关,甚至都将女真人调来了南京道,”

女真?韩冈神色一凛。

这一战兵发仓促,无论辽宋在事前都没做好开战的准备。只看萧十三的麾下,基本上都是西京道的兵马,否则一旦动用了上京道和中京道的部族军,韩冈这里的压力将会大得多。

而从时间上看,女真兵马的调动几乎不可能是在开战后才开始的,必然是在这之前,甚至有可能并不是为了这一战而调动——可召集一群强盗同去抢.劫;和控制一群强盗,让他们俯首帖耳,完全是两个概念。难道耶律乙辛很早以前已经收服了女真诸部中的哪家大部族?不会是完颜部吧。

“还有,从飞狐陉来援河东辽寇的仅有八千部族军,而不是之前所侦知的两万。我们被耶律乙辛骗了!”

“我说呢,这边被打得缩在城里,那边就把辽贼的动向了解得一清二楚。原来是给骗了!那群狗才,眼珠子都长哪儿去了!?”折可大大声怒道。

韩冈也想骂人了,两个数目之间未免差得太多。

“也就是说,自始至终,耶律乙辛都是将心思放在河北,而不是河东?”黄裳皱眉‘不愧是耶律乙辛。’韩冈暗生感慨。

这位大辽尚父很明智,或者说,看得很透。

不论河东河北的兵备差距,只从地势上看,河北的平原对擅长纵马驰突的契丹人永远都是最安全的地形。

地势上的欠缺,就需要大量的人力来补足,而大宋这边无论怎么在河北加强军备,都不可能比得上河东的层恋叠嶂。

由人所造成的防线很好解决,总是会有办法突破。可自然生成的山川,辽人再有本事也不可能长出翅膀飞过去。

‘幸好耶律乙辛用错了萧十三。’韩冈暗自庆幸。萧十三如果在一开始没有因为贪心南下太原。甚至更贪心想来抓自己…………韩冈忽的心中一动,似乎有些不对劲,那好像并不是萧十三的主意。

至少耶律乙辛曾让人给萧十三带了一句‘韩冈在哪里’,韩冈本人已经从一个属于五院部的契丹士兵那里听说了。

没有耶律乙辛的这个错误,让萧十三领军南下太原,继而太谷,到了现在,韩冈就只有硬攻石岭关和赤塘关一条路可走了。

他左右看看,帐中的气氛有些压抑,就连章楶都紧锁着眉头,默然不语。

毕竟敢于杀了天子一家老小的权歼,在任何人眼里都是一个深沉、阴险、足智多谋的角色,甚至在这个人人普遍相信神佛存在的时代,这个如魔王一般的人物还有着神秘色彩的加成——比如辽穆宗耶律璟转世什么的:

大辽诸帝,只有穆宗皇帝耶律璟,是太祖耶律阿保机之次子太宗德光的传承,而其余如世宗耶律阮、景宗耶律贤、圣宗耶律隆绪、兴宗耶律宗真、宣宗耶律洪基,以及现在刚刚死掉的小皇帝章宗阿果,都是阿保机长子人皇王耶律倍的嫡系血脉。且穆宗耶律璟正好是为人所弑、死于非命,非是寿终正寝。这都让耶律乙辛的弑君之举笼上了一层因果报应的轻纱。

“别把耶律乙辛想得太厉害,他也只不过是个人而已。”韩冈说道,“他要真有运筹千里之外的能力,就不会让萧十三南下太原了。更不会没算到青铜峡的党项人会将弓刀对准了他。”

韩冈的这几句,好歹让气氛缓和了一点。

“打仗这件事,就是看谁犯错更少而已。”他又继续说道,“这一回宋辽大战,大宋这边犯了很多错,辽人那里也是犯了很多错。只是到现在为止,他们比我们更少一点……不过,这一战还没结束,接下来,只要我们少犯错,获得最终胜利的必然是我们。”

“枢密说得是,获胜的必然是我大宋。”留光宇第一个附和韩冈。接下来,其余幕僚也纷纷表示同意。

韩冈微微一笑,带着些许自嘲,他不过是空口说白话而已。只不过他的声望和过往战绩能够给人以信心。

“李信呢。”

到了这时,韩冈方才问起他表兄的情况。

“李刺史在攻打易州的时候,一直都在防备着辽贼的援军。只是来的太多,最后方才不支而退。在退兵时,李刺史领军殿后,最后苦战得脱,受了一些伤,尚算无恙。”

胜负兵家常事,保住姓命就好。韩冈放下了心来。或许这一回刚刚得到的遥郡刺史可能会被剥夺,但只要人还在,可就有卷土重来的一天。

可李信之败,也让韩冈更加警惕。

辽国毕竟是与大宋平起平坐的大国,现在在河东一点优势,那也是因为面对的辽军无心作战,而并非宋军有多么强势,如果都在最佳状态上,胜负尚未可知。

不过战略规划,以此时的话叫做‘庙算’,本来就是在战争开始之前,便千方百计削弱敌人的实力,加强自己的力量,让敌人在错误的时间、错误的地点,与错误的对手打一场错误的战争。韩冈自觉在这方面做得还算不错。

现在看来,河北的局势现在又偏向辽人了,但河东这里,终究是他韩冈占优势。

“枢密,接下来该怎么办?”章楶领头问道。

幕僚们望着韩冈,在眼下河北兵败的时候,韩冈的判断甚至就直接决定了整个北方的战局。

“去拿下忻口寨。”韩冈说道,“只有拿下忻口寨,河东的局面才能真正的打开。”

代州和忻州之间的要隘忻口寨,只有收复此处,才能稳固忻州,继续攻打代州。

两天后,宋军收复了忻口寨,但更确切一点的表达,不过是你丢我捡,仅仅是几次斥候间的小规模冲突之后,辽军又退了。

“这也太明显了吧。”韩冈的身边,黄裳低声骂。

幕僚们都很清醒的看到了这一点,而且辽贼本就不擅长演戏,他们的特长都在急进倏退,不在诱敌深入上。

但下面的好些将领却不是那么清醒,一个个叫嚣着追击辽军,夺回代州。

“你们钓过鱼吗?只是吃了两口鱼饵,就恨不得将自己的脖子送到钩子下面去?!”

韩冈难得的训斥人。换作是普通一点武将出身的统帅,说不定这些个来自京营的将领敢阳奉阴违,自行出兵追击。但韩冈自然不同,将领们登时就偃旗息鼓,甚至连腹诽都不敢。

“不要管辽贼怎么想,做好我们的准备。”

稳一点,必须要稳一点。韩冈提醒着自己。

耶律乙辛动用了东北渔猎部族的兵马,当然也能调动起来草原上的游牧民族。原本抢得囊橐皆满、连马都背不动的契丹骑兵,将会被一群红了眼的新强盗所替代。

“勉仲。我让你去查的那位在忻州城下为辽贼所害的殿值叫什么名字?”

坐镇在忻州城中,训斥过众将的韩冈问着黄裳。

在忻州城下被辽军残害的那名斥候,用他的生命鼓舞了城中军民的士气。也是压倒骆驼的最后一根稻草,彻底打消了萧十三固守两关,然后解决忻州城守军,以稳固代、忻两地统治的如意算盘。此等忠勇豪杰,在辽军撤退后,他的遗骸便为城中的官民收拾妥当,停放在忻州城中的一处庙宇内。当然,事迹也传到了韩冈的耳朵里。

“此殿值姓王名宣庆,是府州军中虞侯,在折府州帐下奔走,这一次是奉命在折克仁身边随行。年三十三,从军十七载,其妻折氏,家有二子一女,皆在幼年。”

黄裳大概是从折克仁那里打听了消息,回答得很是详尽。

“原来是贵家之婿。”韩冈对折可大道。

折可大低声,“王宣庆平常只是普通,看不出竟有如此忠勇。”

“疾风知劲草,板荡识诚臣,忠信之德,岂是平时能看得出来的?”韩冈叹了一声,“安排人手去为他设个祠吧,过几曰我去祭拜他。”他又对黄裳道,“勉仲,诔文就麻烦你了。”

黄裳拱手行礼:“旌表忠臣之德,乃是黄裳的光荣,岂敢视为麻烦?枢密放心,这篇诔文,黄裳必用心去作。”

韩冈满意的点点头。

稍歇几曰,接下来,当然就是代州!
