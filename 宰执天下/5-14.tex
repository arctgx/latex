\section{第二章 牲牢郊祀可有穷(下)}

“发明了飞船的韩冈?!”萧十三差点就要跳了起来。

张孝杰眼神转利:“发明板甲的韩冈?!”

“是造雪橇车的?”萧得里特脸阴沉起来了。

“不仅仅是那些东西。青唐羌还有南方的那个什么交趾,他占了很大一份的功劳。”耶律乙辛沉声,“其用兵远在同姓的韩琦之上。”

“萧禧几年前为云中边地多次出使南朝。据他所说,在白马津浮桥过黄河的时候,都能听到当地人称赞只做了一年知县的韩冈,连水井都称为韩令井,说是救了百万流民,河北甚至有为其竖长生牌位的。”张孝杰还记得当时萧禧说话时的神情,明明只是说着南方的风土人情,提到韩冈之后却郑重其事得如同在朝堂上宣读国书,“南朝都拿他比富弼。当年富弼在地方任州官时,也曾经救了数十万流民的性命。”

比起南朝的名相韩琦,多次使辽的富弼,在辽国留下的名声更高。

“富弼不如他。”耶律乙辛没有当面与富弼打过交道,但朝中有关富弼的传言,却听过不少。

萧得里特并不为耶律乙辛对韩冈的髙评价而吃惊,南朝的那位年轻的重臣,他的名字早已经在辽国同样等级官员的议论中,给许多朝臣留下了极为深刻的影响,

自从韩冈发明板甲之后,听说南朝原本造一具铁甲的时间和花费,现在能造十具。才几年功夫,河北禁军已经人人有铁甲,听说有些战马也开始挂甲。

他看看萧十三,前些日子还听他抱怨五京和蒲速斡鲁朵的甲胄工匠,全都是废物,学着宋人造板甲,却比不上宋人的速度。再这样下去,恐怕南朝连征用的驴子都能穿上甲胄,还怎么跟宋人打仗?!

还有雪橇车,那是在积雪上载货运输的好工具,辽国乃是苦寒之地,用到雪橇车的地方比宋国更多,自南朝引进才两年而已,就已经传遍了五京道中。这也是韩冈的发明!

而且最关键的是,飞船是韩冈发明的。

如果没有飞船,他们根本奈何不了大辽天子。宿卫的控制权一直在耶律洪基本人的手上,不论是挑选刺客,还是想在饮食或是车马上做什么手脚,都没有任何成功的机会。

精通兵法,长于政事,发明众多,而且还不到三十岁。如果他能活得长久,将会是未来三十年大辽的噩梦。

“就是他发明了种痘?”三人齐声问。

“种痘法据说是他得仙人传授后又加以改进才得到。韩冈本来在南朝民间被传说是药王弟子,有他在军中坐镇,便不生疾疫。去瘴疠遍地的南方攻打交趾,南朝天子特意调他去担任副使,也的确安安稳稳的将交趾打下来了。现如今,南朝人人都认定他的师傅就是唐时的医仙。也有说法他是药师王佛座下护法金刚。”耶律乙辛饶有深意的看了看三人,微笑道,“现在看看,许是药师王佛转世也说不定。”

辽人几乎都是虔信浮屠的佛门弟子,韩冈被传说成佛陀转世,萧得里特三人脸色就有些发青发白。

张孝杰突然想到耶律乙辛想要说什么:“该不会这个飞船也是韩冈故意……”

耶律乙辛道:“先帝驾崩,虽然说是意外,但也可以说是他韩冈下的手。他可是跟佛陀扯不开关系的。”

药王弟子、药师王佛座下金刚,或是干脆就是药师王佛转世……想到如今的局面也有韩冈的一份功劳,张孝杰、萧得里特和萧十三心中都开始冒起寒气。眼下焉知此非韩冈之谋?

“有了种痘,就多了人口,得了民心,有了板甲、雪橇车,宋军战力大增。一个飞船,不说能当做巢车使用,甚至让大辽无力援助西夏……”萧得里特声调阴沉。

“此子切不可留!”

张孝杰和萧十三难得的有志一同。信佛归信佛,但要是佛祖敢坏他们的富贵,照样敢拆寺庙。

耶律乙辛摇摇头,“现在的当务之急不是韩冈。”

要是自己能坐稳位置,日后有的是时间去下手,要是自己坐不稳这个位子,只有死路一条,到时候大辽的存续又与他何干,被灭了还能让自己出一口气。

“太傅的意思是……”张孝杰小心的问着。

耶律乙辛说道:“飞船很有用,雪橇车也是一样,板甲的打造速度虽比不上宋人,但也比过去强了……”

“可是要从南朝弄回种痘之术?!”萧得里特终于是听明白了。

“国主年方幼冲,侄儿向叔叔要一个防痘疮的方子,总不能不给吧?”耶律乙辛笑道:“只要南朝开始推广种痘法,怎么也能弄到手。”

张孝杰抚掌而笑:“明面上,暗地里,两边同时下手。双管齐下,必能将种痘之术弄到手!”

就是萧十三也明白了耶律乙辛的用心:“只要在国中推广种痘法,人心也就来了。”

在国势动荡的情况下,人望是不下于军力的关键因素。现在耶律乙辛权势赫赫,但先帝、皇后、太子、太子妃的死在传言中都与他脱不开关系,看起来这些议论没什么大不了的,但当他压不住阵脚,就是全局崩溃的结果,没有什么人会跟着他走到底。

可一旦推广了种痘法,只要说一声是太傅所赐,那么人心也就有了。一个喜欢大修佛寺的皇帝,能跟万家生佛相比?

萧得里特算是明白了:“下官这就回去挑选得力之人去南朝。”

耶律乙辛摇头:“去南朝刺探机密,你不擅长,我自有安排。至于派遣使臣……”声音一顿,看向张孝杰。

张孝杰会意,“太傅放心,下官会做得妥当。”

萧得里特脸色有些难看。萧十三盯着张孝杰得意的微笑,眼中闪过一抹阴狠。

耶律乙辛抬头看了看萧得里特:“虬邻,临潢府就交给你了……过了上元节,我和天子就往东京道去。”

“太傅,你要去东京道!”萧得里特吓了一跳。

张孝杰也惊道:“太傅,难道是要去鸭子河?”

萧得里特连声劝道:“太傅,万万不可,上京道可离不了太傅你坐镇!”

“今年的头鱼宴还要照样进行,若是春捺钵不去鸭子河,那些女真人恐怕又要有不轨之心了。万一他们给人收买了去,就不是一天两天能解决的了。”大辽天子本来就是该巡狩四方,耶律乙辛不打算改变,只要手上还有兵,不怕有人敢作祟。他冷然一笑:“正好可以看看撒班敢不敢动手!”

……………………

远在千里之外的大辽东京辽阳府,也有一群人在关心着春捺钵的问题。

“今年的春捺钵应该不会来了。”

“耶律乙辛肯定不敢来,缩在临潢府中。”

“漆水郡王怎么说?”

“大王说了,还要等谢家奴那边的回话。”

“就不能东京道这里先举义旗?西面有西南招讨司的挞不也在,中京有六部大王谢家奴,只要漆水郡王首举义旗,西京、中京必然举兵响应,剿灭逆贼,指日可待。”

没人回话。

合围是合围了,可首举义旗却不是好差事。第一个起兵清君侧,就是资历和人望,同时也代表着危机。相对而言,危机的可能性更大一点,相比起西京道来,临潢府离东京道并不算远。

厅堂中,一个个与会之人都守着沉默是金的格言。

因为废太子之事,辽国的朝堂上早就被清洗了一遍。耶律洪基从天上掉下来的时候,也正是耶律乙辛权势正盛的时候。反耶律乙辛的势力现在根本是一团散沙,想推翻耶律乙辛、做一做皇帝的宗室很多,但有那个实力的却没有一个。

在皇太叔耶律重元叛乱之后,成了惊弓之鸟的耶律洪基,一直利用耶律乙辛打压所有的宗室,有能力的、有威望的、有实力的,都被利用各种各样的理由被贬斥、被削弱、甚至被处刑,太子耶律浚之死,就是耶律洪基这份恐惧心态发挥到最高潮的结果。使得眼下没有一家能有足够兵力和威望来推翻手握重兵的耶律乙辛。既然只能合作,那么当然是让别人先出头,自己再出来占便宜。

“胡睹衮老贼已经将忠心的朝臣全都给囚禁起来了!”一个年轻人终于忍不住拍案而起,“再耽搁下去,他的位置就一天比一天更稳!”

“引吉的儿子,我们都知道你父差点就给耶律乙辛害死,但总不能贸然去攻打临潢府吧?粮草兵力都要准备好才行。耶律乙辛手上有十万精兵,得好好的筹划一番。”

“也要顾着天子啊,这可是先帝唯一的后嗣了,贸然攻击,可是会被耶律乙辛下毒手的。”另一人也在推脱着。

“你以为阿果能养过十岁?”年轻人声音尖利起来:“他可是太子的儿子,胡睹衮会留他到成人?!等两年看看,少不得会冒出个宣宗遗腹子来。你们以为萧茹里的两个女儿进宫是做什么的?!”

其余几人都不接口,他们就是要等着耶律乙辛下杀手。眼下耶律乙辛还占据着大义的名分,可以挟天子以讨不臣,等到耶律乙辛害死了小皇帝耶律延禧,可就没有这层光环了。

小皇帝绝对活不长,这是辽国国内所有人的共识。

父母和祖父母都是耶律乙辛害死,若是活到十五六岁能秉政的时候,得到国人的拥护,耶律乙辛一党哪里还会有好下场。

而且眼下对小皇帝父母的追赠也是个大问题。

耶律浚是废太子,以庶人的身份而死。但他的儿子偏偏继了位,依常理,当追赠其帝号,以全孝道。可追封一个以谋反之罪而被废黜的庶人为皇帝,那么订立谋反罪名的官员,自然全都有罪。

现在耶律乙辛他们只能拖,所以有的是时间。

“耶律乙辛是不会等你们的!”那年轻人站起身,狠狠的丢下一句话,拂袖而去。

“年轻人啊……”一群人在后面摇头。

