\section{第11章 飞雷喧野传声教(13)}

经过了开封东门外最大的仓储地利国仓,就看见一队人马守在路边。

“是馆伴使到了。”接伴使如解脱一般的叹着。

自过了黄河后,前来通知并确定行程的信使便一波接一波,但只要没有看到人,接伴使就始终得提心吊胆。

不过到了这一刻,接伴使的工作到此便算是彻底结束,下面的接待,就是馆伴使的职责。

如果这一次过来的正旦使是宋人的老朋友萧禧,他肯定会认出人群中的蒲宗孟。

但耶律迪却不认识这位老牌子的翰林学士,没什么名气的宋官,根本就没必要记住。换作是萧禧曾经遇到过的那一位来迎倒是要小心对待,可惜人家现在已经是参知政事了。

耶律迪很散漫的用契丹礼节向对面正作揖问候的蒲宗孟行礼,“劳烦蒲学士久候。”

蒲宗孟在东门外显然等了有一阵了,他的随从们一个个冻得缩头缩脑,就是他还挺胸叠肚,看着有几分气派。不过转头看看路边,竟放了两个暖炉,中间一张交椅还没收起。

南朝上下若皆是这等人,还真没什么好怕的。

辽国正旦使的行程,一直都在蒲宗孟的掌控中,但来往于途的信使并没有告诉他,这位国使是个不通礼数的蛮子。

耶律迪的无礼,让蒲宗孟的脸色稍稍变了一下才恢复如常,径直转向接伴使,向熙宁六年的榜眼点了点头:“朱校理一路辛苦了。”

“为国事,不敢称劳。”朱服连忙躬身回礼。

“朱校理是小韩参政的同年,可惜知道得迟了,前日才听说,否则当更亲近一点。”

耶律迪从旁插话,他还是前两天才知道这位总是愁眉苦脸的接伴使,是跟韩冈同时考中的进士,而且名次还在韩冈之上。

听了即时的翻译,蒲宗孟不知道这是不是辽国国使已经了解到了朝中的现状,故意如此刺激自己。

但耶律迪的视线此时已经在追逐着不远处城墙上的人群。

开封正在整修城墙。

城东面的工地上,能看到数百上千的民夫,沿着墙上的架子奔走着。只砌到中段的砖石,让城墙上下两端有了极为明显的分野。

东京城的城墙并非一条直线,而是弯弯曲曲宛如水波。尽管耶律迪对守城的战法不了解,可多看几眼之后,就能明白这样布置城墙有着什么样的好处。

“最近开封的新城城墙因故加筑,弄得地上也是一片泥泞。换做平日,这城墙之侧,水波粼粼,杨柳依依,也是一番景致。”

见到耶律迪关注城墙,蒲宗孟很快便收拾了心情,指着城墙上下,微笑的向耶律迪介绍着。

“这城墙怕不有五丈高吧?”

“或许还要高一点。”蒲宗孟扬声道,“开封周围五十里,光是为了给外城城墙包上城砖,就从天下各路调运砖石达三万万块!”

三万万?

换算成钱不知要有多少。

耶律迪感觉到蒲宗孟和他从人的视线都落在了自己的脸上

大概是要等着看自己咬指吐舌的惊讶表情。

“有五丈高的城墙,已经算的上是坚固,而贵国还要在上面加筑,不知是要防备谁?”耶律迪悠悠然问道。

南人到底有多害怕大辽?这里距离边境可是有一千里。

越是仔细的观察,便越能发现南朝的虚怯。

耶律迪现在越发的肯定,之前不能击败宋国,不过是个意外,因为不敢尽全力而缚手缚脚才会产生的意外。

一年的时间,不费吹灰之力便灭掉了两个百万丁口的国家,耶律迪毫不怀疑大辽的国势正处于最鼎盛的阶段,镇压东西南北,远及万里之外,要不然宋人为什么会不惜巨资来给都城包上城砖?

“说不上是防备,毕竟现在也没有外敌能入我中原半步。不过是为了修造放置火炮的炮台,顺便加增少许,算不得什么。”蒲宗孟远比接伴使朱服要大方许多,十分坦然,“倒也不是不想在边寨上修,但火炮毕竟才出来,炮台到底怎么修才好,谁也说不明白。在京城先把各式炮台都修一下,评出优劣高下,就可以推广下去了。”

‘火炮!’

听到蒲宗孟嘴里吐出这个词,耶律迪淡淡微笑就浮现在脸上。

不用弩箭,一个契丹勇士能打三个汉兵。

这是过去在契丹国内流传的豪言壮语,不过在这豪言壮语背后,就是对宋人弓弩深深的戒惧。

不过现如今,就是宋人用了弩箭也不怕了。因为大辽这边,也有了威力更大的远程武器。

就算没有从行商嘴里听到那些传闻,就算没有去辽阳府亲眼看一看,耶律迪都清楚,火炮究竟是多么危险的一种武器。

那毕竟是出自韩冈的手笔。即使在上京道的草原之上,韩相公的名气都是如雷贯耳,尽人皆知。

草原之上,既缺乏富足的生活,也缺乏治病的良医。而天花,就是诸多让草原之民畏惧的病症中最为恐怖的一种。他们可以不知道谁是太师,可以不知道当今天子是谁,但不会不知道发明了种痘法的药师王佛座下弟子。

也是依靠种痘法,尚父殿下拉拢了一大批异族的人心。只要顺服听命,按时进贡,就能得到朝廷的回赐。

对于任何一个草原部族,人命最为金贵,在争夺草场的时候,男丁稀少的部族只能被挤到水草最稀薄的驻地,甚至还有被吞并的危险。能在天花下多保存下一个男丁,就意味着几年十几年后之后能骑马挥刀射箭的汉子,来自朝廷的赏赐,是任何一个部族的族长所不能拒绝的。

但耶律迪相信,尚父殿下不会忘记是谁带来了这一切,到底是谁让他可以有今日的风光。那一位带来天赐良机的南国参政,说不定,也可以让他失去一切。

火炮既然出自韩冈之手,又能得到南朝如此看中,大辽上下谁能视而不见?

就在辽东,辽阳府的铁场,同样是日夜火焰不熄。那里不但能炼铁,同时还能够铸造火炮。

这一年来,南京道上的铜匠,还有铸钟匠,全都给集中去了辽阳。依照细作传来的图纸,来仿制火炮。

火炮不过是外形特异的铜钟而已,而且不用考虑音色,有了图纸,甚至还有了具体的数据,对铸钟匠来说没有任何难点。只用了半年不到,火炮便铸造成功,而且尺寸还比宋人的火炮更大一点。

尽管没有商人口中那么夸张的威力,但发射起来惊天动地,的确不负韩冈之名。

相比起重弩,火炮更适合大辽的军队,用来克制宋人的军阵,没有比火炮更优秀的武器了,而且南京道上的城池,也有了最有力的守护者。从高高的城头发射出来的炮弹,放在地面的火炮难道还能与之比较射程吗?

“啊,或许林牙还没听说过火炮。”

蒲宗孟的试探拙劣得让人感觉很可笑,耶律迪笑问道,“听说过了。听说又是小韩参政的发明,只是了解不多,想来又是一件利器?”

直询军情,蒲宗孟却回答得坦然,“的确是利器,今年就造了八千门火炮直接配发军中。”

他自知辽国有多少奸细在国中,岂会不知火炮底细,耶律迪装痴卖傻,反倒惹其暗笑。

‘是三千门,而且是虎蹲炮。’

这又是个想要靠吹嘘来吓唬人的蛤蟆。

耶律迪心下冷笑。

殊不知肚子鼓得再大,也依然还是蛤蟆。

辽阳府那边也铸了虎蹲炮。射程比马弓还短,速度比重弩还慢,说是适合防守军阵不被骑兵冲击,但实际上,有多少效果还得上了战阵再说。

宋人在装备大军前,肯定也有试用过,可宋人哪里知道骑兵的应用之妙,在看过了虎蹲炮的效果之后,大辽这边早有了多种的应对手段,当真上了战场,足以给宋人一个惊喜。

“八千门?!”耶律迪不介意在宋人面前多皱几下眉头,让南朝多得意一下也无妨,“那可要不少铁。”

“不过是几千万斤铁,几百万斤铜,再加上几千万斤石炭,不算什么的。”

我们就是财大气粗。

从南朝的翰林学士的口中,传出的是暴发户的口吻。

这就是南朝最让人害怕的地方。

前一次,南朝造铁甲,不想累及国计,所以才将百万铁甲,用了三年的时间打造出来。但在造铁甲的时候,斩马刀、神臂弓之类的南朝利器,完全没有耽搁。

这一回,就是三千门虎蹲炮,同样不会耽搁其他兵器的生产。

幸好大辽这边数以十万计的骑兵,同样能够装备上甲胄,对阵时,绝不会输给宋人多少。而且骑兵行速飞快,不想与宋人交战,直接就能绕过去。断粮道,掠乡村,难道运粮种田的农民,还能装备上铁甲不成?

耶律迪继续与蒲宗孟交换着辞锋,但注意力却一直放在左右,直至进入城门之后也没有改变。

进入东京这座富丽甲天下的煌煌巨城,辽国使团的成员只要是第一次南来,都不免为城中的繁华盛景而目瞪口呆。可耶律迪脑袋里却在想着,要是率领一千骑兵冲进来,该哪边放火,该哪边纵马。很快他就有了计算,在城市里巷战是骑兵的难点,但放火从来不难。

进入都亭驿歇下,午后时分,耶律迪便被宣诏到皇城中。

正常都该有两三日的休息,这一次却有违常例,耶律迪心中狐疑,却没有拒绝的道理。

在宣德门外没有等候,直接走进了深长的门洞,然后耶律迪便听见了震耳欲聋的轰鸣声,与甬道的回音一并回响在耳畔。

耳中嗡嗡鸣响,走出来后,耶律迪仍是一阵头晕,正想向蒲宗孟发作,双眼突然瞪大了。

城门两侧的石台上,放置着四门巨大的火炮,只看那斜指天际的炮管,甚至比耶律迪他的个头都要高,比他身子都粗,巨大的车轮都有五尺径圆,这是什么样的火炮?!怕不有数万斤的重量,辽阳府那十几门辛辛苦苦造出来的火炮与之相比,是侏儒和巨人的差距。

“这是太后敇命的左右金吾卫大将军炮。”

蒲宗孟得意的声音,完全传不到耶律迪的耳中,大辽正旦使已经完全呆住了。
