\section{第三章 陋室岂减书剑意(上)}

“富与贵,是人之所欲也,不以其道得之,不处也。贫与贱,是人之所恶也,不以其道得之,不去也。君子去仁,恶乎成名?君子无终食之间违仁,造次必于是,颠沛必于是。”

日头一点点的升起,驱散了秋日清晨的寒意。已经到了秋后翻耕麦田的时节,自麦收后修养了一阵的下龙湾村的村民们便又扛起锄头,出村下田。村口的土路上村民络绎不绝,而朗朗的读书声此时正从村口边不远处的一间破旧草庐中传了出来。路过的的人们纷纷停步惊讶的循声望去,虽然屋舍已经不同,可熟悉的读书声,仍让他们觉得仿佛一下回到了几年前,韩家三子日夜用功苦读的时候。

“韩家的三秀才病好了?!”

“应是大好了!这几天晌午后都看见他家的养娘扶着出来走动。”

“俺昨天也看到了,是能下地了,就是瘦脱了形。啧,原来多壮实的一个后生啊,跟他家大哥、二哥一个模子出来的,牛一般啊……现在风吹吹就会倒。”

“怎么三秀才比过去还要用功了点?病才好啊!”

“他一病大半年,现在肯定是想将功课补回来。”

“真该让俺家的两个小子来看看,这才是能中进士的样子。韩家三哥在外面两年,不是白饶……”

“好像三秀才也比以前和气了,昨天还跟俺笑着打招呼来着。”

“没错,没错!的确是和气了不少。”

韩家老三在小村中的地位不低,此时的读书人都是很受人尊敬。记忆中的韩冈都是埋头于诗书,是个很淡漠的性子,对村人礼数周到,但笑容就欠奉了。不过贺方这两天本着敦亲睦邻的心思,要改变村民心中自己前身留下的恶劣印象,不想竟让他们受宠若惊。

“也幸亏大好了。韩菜园这半年为了儿子,家产都败光了。如果再不好也没得钱来买药……”

“一顷多地如今一点不剩,两进的宅子也卖了。韩菜园夫妻两个还得没日没夜的去山里挖山菜,也不顾大虫、花熊。这年岁啊,真的生不起病!”

“倒让李癞子那厮捡了大便宜,他想韩家的三亩菜园多少年了,现在终于让他完了愿……”

“哪里完愿了?他哭还差不多。那三亩菜园是典卖,不是断卖\footnote{宋代的田宅买卖分为两种形式,一种称为典卖,即田宅卖出后,卖主有赎回的权力,而买家无权拒绝,相当于使用权同时转移的抵押贷款。一种是断卖,也称绝卖,卖家无权赎回。理所当然的,典卖的价格和断卖的价格有不小的差距。},能赎回来的。菜园子才典过去,三秀才病就好了,李癞子现在怕是镇日都要担心韩菜园将田赎回去。”

还带着一点橘红色的旭日光辉,从支起的窗棱缝隙投射进来,映在夯土筑起的墙壁上,而窗外村民的话也随着阳光一起透了进来。站在村口议论韩家的都是些乡里乡亲,多有几分替韩家庆幸。可他们的议论传入入耳,贺方的读书声却是低沉了下去,甚至有些不易觉察的哽咽。

这个时代的秦岭可比后世荒凉得多,老虎满山乱窜,在韩冈留下来的记忆中,还有老虎夜里冲进村中叼了羊走的例子。贺方没想到父母为了给他筹集医药费,竟然连性命都不顾了。还有河湾边的三亩菜田,那是从祖父辈留下来的,只看韩冈的父亲都是人称韩菜园,便可知那块菜田实是韩家的命根子。

韩冈就算已经魂飞魄散,仍能影响着贺方占据的身体,去反对卖出这块田地。可惜他到底还是迟了一步,等他意识清醒,菜田已经被咬着牙典了出去。幸好还能赎回,不然韩家真的成了彻彻底底的无产者——以此时的说法,叫做客户\footnote{宋代的主客户与唐时不同。不再是按照本地土著和外来移民来区分,而是根据有无常产,也就是田地和房宅来划分。家有田宅者是主户,没有的便是客户。}。

“韩家这两年也不知遭了什么灾,恶了哪路神灵。今次兵灾,一下没了老大老二,好不容易养大的三个儿子,两个拔了短筹,就剩个措大\footnote{措大,古代对读书人的贬称,也有称穷措大,村措大。}老幺!”

“是不是前两年祭李将军,韩菜园那次碰跌了香炉,遭了祟?不然怎么连丢了两个儿子,韩三秀才也是一病小半年,差点又丢了命。韩菜园和阿李嫂前日去了庙里许愿,就一下就好起来了!”

“去,小心夜里李将军老大箭来射你个对穿!李将军可是个会作祟的?!”

“……俺也只是说说罢了!”

“韩三秀才得病是受了风寒又赶了紧路,关李将军何事?现下病能好,这才是李将军福佑。”

耳中不断被聒噪着,心中也躁得厉害,贺方没心思继续再读下去。咬人耳朵背后议论人的事,无论时代和地点,都是少不了的。但自己成了他人嘴里咀嚼的谈资,贺方总觉得心中有些不舒服。

贺方住了声,轻轻合上了捧在手上的《论语》,放到了书桌上。论语一卷完全由人手抄写而成。纸面上的列列小楷,方正光洁,一丝不苟,近于欧体,工整得如同铅字印刷出来一般。这是从欧体字脱胎而来的馆阁体,贺方早年曾经被他的祖父逼着习字,学得也是欧阳询,看着韩冈一笔一画尽着心力抄写出来的的方正小楷,只觉得十分的亲切。

不过馆阁体是满清时代的说法,在贺方如今身处的这个时代则是称作三馆楷书——所谓三馆,是昭文馆、史馆、集贤院的统称,也称崇文院。其地位在朝堂诸多馆阁中最为尊崇,此时的宰相都是兼着三馆大学士的馆职\footnote{北宋前期——也即是宋神宗元丰改制之前——但凡宰相都会兼任三馆大学士。一般来说,宰相班次满员为三人,首相为昭文馆大学士,次相为监修国史,而末相为集贤院大学士。通称为昭文相、史馆相和集贤相。}——只是不论是何等称谓,要想进学参加举试,写在试卷上的字体最好是这一种,否则让负责誊抄试卷、以防考生考官串通作弊的书吏错认了几个字,那可就真是欲哭无泪了。

书卷中的文字虽是工整,但所用的纸页却甚为粗糙,书页边缘裁剪得也不平齐。很明显韩冈制书的手艺并不过关。而一摞摞堆积书桌和书架上的书卷,不仅仅是贺方方才所读那本《论语》才制作得如此粗糙,其中大约有一多半都是书写整齐、制作粗糙的韩记出品。

贺方并不怀疑这些手抄本的出处,一个十六七岁的少年离家远行,寄寓在城外的破败庙观中。白天入城求学,夜中则就着残烛月光,奋笔抄写从同窗学友处借来的珍贵书籍,无分寒暑,不知节庆。这一幕幕的辛苦笔耕的记忆仍清晰至今存留在韩冈的脑海内,而为贺方所继承。

韩冈的毅力和耐性,贺方有点惊讶,但算不上佩服。大概跟自己高中时的努力程度差不多。都是夏练三伏,冬练三九,没有一日辍笔。

‘十年寒窗已过,可惜没能等到金榜题名的时候。……但就算苦读十年,能中进士的机会,也不过像千军万马过独木桥,比还没扩招的大学还难考千百倍,这笔投资还真的不合算。’

承平了百多年,拥有两千余万户口,贺方估计差不多应该有一亿子民的大国,如今是每三年才录取三百余名进士,平均一年只有一百。

而且进士科取士向来是东南多,西北少。福建、两浙的军州,一科出十几个进士都不稀奇,甚至一个世家大族,一科出了五六个进士的事也是实实在在发生过的。

而陕西一路二十多军州,哪一科进士加起来能超过五个,都算是大丰收。连续十几科都没一个进士出头,在西北的军州更是常见。至少在韩冈留给贺方的记忆中,好像从没有听说这二三十年来秦州有哪位士子得中进士\footnote{北宋一朝一百六十余年,平均每年的进士数量大约不足一百,总体计算一万五六千有余。其中开封、两浙、福建和江东诸路的州府就占到了八成以上,如福建建州八百多,福州五百五,常州近五百。而北方几路则是寥寥无几,常常是个位数。如文中所说的秦州,据地方志记载,北宋时期中进士的只有两人,而秦凤路近十个军州,加起来也仅有十一人——以上数据皆出自贾志扬的《宋代科举》。}。

五六百万人口的陕西路,每科进士都是个位数,平均到一年中,不到百万分之一的比例让人想想就感到绝望。

读书、进学、参科举、中进士,是贺方的这具躯壳原主人十年来的唯一追求。但希望如此渺茫,投入回报如此之低,让贺方对科举完全没有任何兴趣。他现在心中都在转着该怎么利用自己拥有的知识——就像造烈酒、肥皂、玻璃之类——在这个世界攫取地位和财富的念头。

ps:在宋代中进士很难,尤其是在陕西,更是难上加难。舍难取易,去弄玻璃、烧酒,看起来就容易得多,但实际情况真的是这样?