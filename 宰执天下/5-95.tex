\section{第11章 城下马鸣谁与守(七)}

“终于回来了……”

前军主营的东侧,李清从主营方向上收回视线。叶孛麻果然还是赶在入夜前回到,不知才扎下营盘,就被招去白池堡,在太后和国相那里究竟得到了什么样的吩咐。

不过不论是什么吩咐,当不会跟之前离开中军时梁乙埋说的话。等到明天,自家这一队多半要打头阵。

再看了更远处,盐州城在夜幕下的阴影,李清掉头回营。

他麾下四千多人马所驻扎的营地,离着主营有近一里的距离。位于一座矮坡上,附近有一条河沟,位置更靠近盐州城一点。

安营扎寨,从来都没有聚成一团的说法。只要兵力足够,都是分成数个部分,控制相邻的几处要点,在敌军来袭时占据足够的地理优势,同时出入方便,吃喝拉撒等方面也不会互相干扰。

李清倒是很庆幸这一点,他不喜欢跟叶孛麻走得太近,也不方便走得太近。作为西夏国中汉人的代表,李清由于始终全力支持梁氏兄妹,在朝堂上的排位越来越靠前,如今更是统领汉军,逐渐成了梁家一派的核心,与外姓部族太亲近了,必然会引来猜忌。

在把守营门的士兵行礼中,李清走进了自己的营地。

熊熊的火光中,一个巨大的黑影冉冉升起。李清脚步一停,目送其一直升到二十丈的高空,然后被一根结实的绳索紧紧拽住。吊篮中一名瘦削的士兵双眼锐利如电,警惕的监视着盐州城内城外的一切动向。

而在另一边,一艘同样巨大的飞船,正一点点的向下降低高度。吊篮上气囊已经瘪了不少,比起刚刚升上去的那一艘有着十分明显的区别。

一名匠师打扮的男子,就在下落中的飞船下方大呼小叫,指挥两个士兵将飞船上抛下来的绳子一圈圈的绕在桩子上。

“一个时辰。”李清低声自语,他一直都在算着飞船浮空的时间。

一艘飞船能浮空的时间,一般得看季节和天气,晴朗无风的夏日,两个时辰都有可能,换作是冬天刮风,勉强上去,没两刻钟就得下来。深秋的夜晚,大约是一个时辰不到一点——不过这是宋军官造的飞船才能达到的时间,换作是西夏工匠们的出品,能飞上天的都不多。飞船上面的气囊,要缝制得一点不漏气,这份手艺不是随随便便就能找到人手。幸好之前在灵州缴获了七八艘飞船。

灵州城下的胜利,得到的不仅仅是苟延残喘的时间,以及数千数万的铁甲、长枪和马刀。还有数以千计的战俘,只可惜其中的工匠为数寥寥。汉家的工匠天下万邦都是有名的,来自军中的工匠更是稀有,哪一个不想要?

往年每一次大败宋军,俘获的工匠都是争夺的焦点,为此大打出手的情况都不鲜见。幸好这一次没有铁匠,否则说不定也会打起来。

朝堂上经过一番争夺和利益交换,能打造攻城器械的几名大工匠几家瓜分,并约定好开战时将他们一起带出来打造器械,而下面的小工也照样有人抢着要。只是最后还剩下一个修补飞船的匠人,单独一个人不便瓜分,就给梁家拿了,出阵前又转派到李清这里。

李清看着飞船的降落,心中暗道,若是俘虏中有一个知道怎么打造板甲、会制造专用锻锤的工匠,就是在紫宸殿上,多半也会闹到拳脚相争的地步。

李清想象了一下那个场景,嘴角就翘了起来,那样还真是有趣。

降落中的飞船已经到了三丈的位置,匠师在下面仰着头,大呼小叫。

李清抬头看着一高一低依然漂在半空中的两艘飞船,城里的宋人随时可能出来偷袭,为防万一,就是夜里也不能将飞船随意的放下来,留下监视上的漏洞。

两艘飞船都是在灵州之战中缴获的战利品。原本西夏国中也不是没有造过飞船,但那种粗制滥造的货色,如何比得上出自宋国军器监的上品。缴获之后,就直接拿来用了。对于飞船居高望远,探查城内的能力,每一位西夏将帅都很是期待。

不过换作李清本人,宁可骑马跑到盐州城的城墙下查探,也不愿意站在那玩意儿的吊篮里,要离得远远的才安心。

从天上摔下来的大辽天子,让许多人对飞船望而生畏。甚至有传言说,如此飞天神物,宋人却不敝帚自珍,就是因为他们有本事给乘上飞船的敌人下咒,辽主耶律洪基就是最好了证据。

以李清的见识,自然知道这是无稽之谈,但理智和感情是两回事。他曾经登上过飞船,有过一次随船上天的经验。登高望远让人很是期待,但双脚之下,隔着一层藤条就是三十丈的虚空,那样的感觉让李清完全忘了向远处眺望,总是不自觉的看着脚底下。偏偏那一次飞船突然受到了狂风的袭击,吊篮中的李清命悬一线。最后还是运气好,加上几十人在下方扯着缰绳,才没有让李清成为在西夏国中坠落的第一人。

但从那一次之后他的心中就对飞船产生了隐隐的畏惧,让李清不愿意太过于接近这等能载人悬在半空中的器物。

下落中的飞船尤在两丈多近三丈高的位置上,飞船升起来快,但下降就是要等气囊中的热气自己冷却下来,速度很是缓慢。

李清不着急,很有耐心的在下面等着飞船一点点的向下蹭着。不过吊篮中的人似乎没有耐心等待,一翻身就从吊篮中翻出来,在吊篮边沿和绳索各搭了一次手,眨眨眼的功夫,就稳稳的落在了地面上,片尘不惊。

从吊篮中跳下来的是个中年的汉子。骨架子很大,身材也高,但露出来的手脚都是筋骨凸出,显然体重不会太重,要不然也不适合坐进飞船中。胡须蓬乱,都把相貌遮掩住了。

看到李清就在旁边,他抱拳行礼:“武贵拜见太尉。”

“无须多礼。”李清说道,急着追问,“盐州城里的动静如何?”

“似乎有些乱,城头上的人太多。”

若是一千多人马出城,前后左后最少都要几十丈才能排得开,就算隔了两三里地,在飞船上也立刻就能发现。可城墙顶上才多大的地,两里开外看城墙,早就跟一条线也差不多了。李清疑惑的问:“何以见得?”

“城头上的火炬时不时就被遮着,只会是人多。上了城墙的官……宋军,至少要有三四千。”武贵皱眉,“还没有去攻城,不应该有这么多人上城……”

李清也觉得纳闷,大半夜的,往城头上派那么多人有什么意义。城里也有飞船飘在空中,只要看守好城门,城墙顶上派个十几队士兵,分段巡视巡视就够了,“你看是怎么回事?”

武贵想了一下后道:“……或许传说是真的,盐州城中主管防务的不是高永能和曲珍,而是徐禧,或者是他从京中带来京营禁军中的主将。”

“所以才出了这等笑话?”李清往城墙的方向看了看,“倒当真有道理。”

“传言中还有盐州城粮秣不济的消息。既然前面都说准了,这一条也当是有几分可能。此来推断,盐州城中存粮数目肯定是不足的。”

“说得也是!”李清点点头,转头又笑道:“我们随身携带的粮秣,加上后方运上来的,配合起来能支撑上半个月。再杀了牲畜,又是半个月。盐州城中的宋人能支撑一个月吗?”

“按说是撑不了的。”武贵摇摇头,“这段时间盐州城的消耗和输送来的数目都是能计算到的,不可能坚持太久,十天都很勉强……刚刚夯筑好的城墙并不算牢靠,如果有个两三年时间,或许能坚硬的如同石头。但十天之内,不可能。”

李清闻言哈哈大笑:“要不是事先得到了这个结果,这一战还真没人敢打。”

“谁让天子不会用人!”

武贵眼中的恨意一闪而逝,却被李清捕捉到了。不过这也算不得什么,西夏国中,对大宋的赵官家抱着恨意的人车载斗量,不独他一个,但李清觉得要感谢他,“也多亏了是这样的天子,否则西夏早就被灭亡了。”

武贵古板,对李清的话没有附和的笑意。李清不以为意,“若城中军情当真能如武贵你说得那般,这份功劳就立定了。过两日,武贵你当能人如其名了——以武而贵啊!”

“贵……”武贵咧开的嘴笑得有些惨淡,“赤佬什么时候贵过?”

武贵没有感激涕零的跪下来谢恩,这样的不识好歹,李清却不以为意。有能力,脾气大点也正常。武贵之前对,总是自取其辱

“明天我这边就要打头阵,不知武贵你……”李清试探着武贵的态度,也没有将话挑明了。

“在下不想杀到自家兄弟,京营倒也罢了,西军中熟人不少,厮杀起来太难看了。”武贵干脆了当的拒绝,“等这一仗结束,要整治一下阻卜人的时候再上阵吧。”

“想惩治阻卜人,也得等到将宋军全都赶回横山以南。否则可没这个胆子。”李清继而又笑道,“既然武贵你不愿意去跟宋人打交道,也就罢了,都由你。不过营中之事,还要你多盯着点。”

