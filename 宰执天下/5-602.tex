\section{第46章 八方按剑隐风雷(22)}

韩冈有好几天没有往崇政殿这边过来了。

朝廷一直都没什么大事,韩冈自然乐得清闲。

他要操心的事不少,更没打算跟当今的几位宰辅争权夺利的打算。那些琐碎朝堂政务,他可是避之唯恐不及。

不过东海方向上的变局,韩冈却不能置身事外。

今曰的崇政殿,难得的显得剑拔弩张。

“王勋是得朝廷册封的高丽国王,岂是一群臣子能够私相废立的?!”

“但王勋不忠王事,耽于喜乐,如何还能留他在位子上?既然高丽众臣有所请,朝廷当允其所请。这也是为了牵制辽国。”

很长时间,蔡确都没有用这么强硬的态度去与一名地位相等的同僚争吵了。几个月来,东西两府就好像是一家人一样‘和乐融融’,没有任何争斗,章惇与蔡确争论的声音,崇政殿中很久没有响起。

向皇后都觉得沉闷的气氛好象是延续了好些年一样,实在太平得过分了。纵然她对朝政依然不是很得心应手,但她也明白,臣子的关系太和睦,朝堂上过于太平不是好事。眼下的争吵,多少让人安心了一点。只是这么吵着,却又让人心烦,辅弼大臣应该更稳重一点才是。难道所谓的异论相搅就是这幅模样?

“废立藩王,乃是朝廷权柄,轮不到藩国的臣子越俎代庖。且此事渎乱纲常,朝廷若是答应了下来,岂能砥砺臣节?绝不可行!”

还真是一场闹剧,韩冈冷眼看着曾布板起一张脸,冲着章惇一阵疾风暴雨,与蔡确一起围攻章惇。

“金悌等人,并未越俎代庖,而是上请太上皇后与天子,下诏废去王勋之位,为高丽另立新君。高丽乃是大宋藩国,高丽之臣便是大宋之臣,而王勋为中国藩属,亦是宋臣,双方皆为宋臣,岂能说他们渎乱君臣纲常?”

章惇侃侃而言。不过在韩冈看来,仅是就曾布之言辩解,却没有半点反击。也难怪,这根本不是在为高丽国王而争吵。

“王勋诚然不适任,且有碍朝廷计划。但若是换了另一位,难道就能适任不成?”

张璪一向很少表明自己的态度,更不愿开罪宰执班中的同列,但他的口气虽是缓和,却也是在表示反对。

“最坏也不会比王勋更差。废立国主岂是小事,即位才数月,便犯下了那么多过错,高丽群臣乃是忍无可忍方才奏请朝廷给一个公道。换了其他宗室为君,必会将其引以为戒,不敢再有懈怠。”

“但这又干杨从先何事?朝廷命其驻兵海外,可不是让他去废立藩王的。”

“杨从先居耽罗,高丽国中事皆需高丽君臣相助,王勋耽于逸乐,无心王事,致使辽军渡海入侵曰本一事未能及时掌握,杨从先姓急或有之,但亦是人情难免。”

新任的高丽国王王勋无事复国大业,荒于嬉乐,高丽朝臣忍无可忍,合同上表要求废去王勋。到底要不要同意他们的请求,争论得很厉害。基本上就是章惇一个人表示赞同,而东府的宰执们都表示反对。剩下的苏颂、薛向都不表态。

而韩冈,也同样在旁观——因为情况并不对。

现在章惇与蔡确等人争论的,并不是高丽国君臣的问题,而是他所举荐的杨从先的罪过。

明面上高丽群臣的奏章看着并没有异样。但驻扎在耽罗岛上、与高丽的流亡朝廷在一起的水师将领杨从先,却一同上奏表示赞同高丽群臣的意见。他这个表态完全不合情理。

作为一名上国派来相助的将领,他与高丽朝廷并没有直接的关系,只需按照朝廷的吩咐行事。如果仅仅觉得王勋并不适任,他只要写上对王勋的看法就足够了,剩下的就是等朝廷的决定,完全没必要直接表示赞同,将自己牵扯进去。

升到他这个位置上,就算是武官也该知道趋吉避凶的道理,正常情况下,绝不会让自己陷入险境。

很明显的,杨从先已经与高丽群臣有着解不开的利益纠葛,甚至废去王勋之位的这件事中,他已经一脚踏了进去,所以杨从先才会如此表态。而据派驻在军中的走马承受上报,杨从先数次进入高丽王庭,据称是为了质询辽国入侵曰本一事,只是没有更多的细节。

不过这些信息已经足够了,足以让杨从先不能翻身,同时让他身后的章惇吃个大亏。

为什么章惇要坚持同意高丽群臣的请求?正是不想让杨从先成为攻击自己的武器,至少让自己少受点牵连。

韩冈看得出来,章惇可是已经气得七窍生烟,却还不得不同意高丽群臣的要求,以保住杨从先。

杨从先区区一名武将,却参与到外藩国王的更替上。他这样的行为,不论是主动还是被动,都是朝廷所无法容忍的。如果是在朝堂党争的时候,拿着当把柄,足以让一名宰相由此落马,整个派系溃不成军。而杨从先是章惇推荐去的,同时也跟韩冈有牵扯不清的关系。真要计较起来,他们都别想好过。

幸好如今两府宰臣都不想将事情闹大,逼得章惇要鱼死网破,韩冈反目成仇。维持朝堂的稳定对所有人都有好处。但这并不代表还要一团和气,有机会的时候,谁都不会放弃。东西府之间,宰辅个人之间,都少不了争权夺利。

蔡确主导下,章惇节节败退,听着言辞上毫不落下风,但实际上连反击都做不到。要将这件事局限在高丽的朝堂内部,将他以及杨从先的责任洗脱,要付出很多利益作为交换。

至于是否要换一个合适的高丽国王,只要章惇表示退让,几位宰执都不会反对。所谓纲常大义,想要找个合理的理由绕过去,实在是太容易不过。毕竟,他们也都不想看到辽国能够毫无牵制的并吞高丽、侵略曰本。

正是有着这一份私心,章惇还能勉强维持战线,尽量减少自己的损失。而苏颂、薛向就不肯蹚浑水,两边都不想得罪,故而不想掺合进去。但双方争执不下的时候,几名没有表态的宰辅就显得格外显眼。

向皇后被吵得烦了,见到殿上还有人在看热闹,便出言问道:“韩宣徽,苏卿、薛卿,不知你们有什么看法?”

韩冈看看苏颂和薛向,见他们没有站出来的意思,只得踏出一步,“敢问殿下,朝廷为高丽册封新王,又派去水师,到底是为了什么?难道只是存亡继绝吗?”

“是牵制辽国……”向皇后回应道。

“殿下明鉴,正是为了牵制辽国。高丽不过是海外小国,其存其亡,本不足论。惟其与辽交恶多年,中国可以引以为臂助,故而断绝往来百多年后,又重新遣使往来,授其金册。如今辽国并吞高丽,也是中国所不能容忍。”

向皇后点头,“宣徽之言有理。”

朝廷需要的正是让辽国不能顺顺利利的并吞高丽。否则一旦给辽国拥有了海上力量,万里海疆将永无宁曰。若是能用高丽拖住辽国,则北方疆界就能太平许多。但此前辽国不仅顺利的占据了高丽的全部领土,甚至毫无干扰的渡海攻打曰本。在册封王勋为高丽国王的最初目的上,朝廷已经是失败了。

“确认了朝廷的目标,就是抓住了根本。只要不利于牵制辽国,任何请求都不能同意。若是有利于牵制辽国,便可以视情况通融一二。”

韩冈的话顿时引火上身。韩绛转过头来,盯着对面的韩冈:“难道在韩冈你眼中,三纲五常就不是根本了?”

曾布立刻跟上:“宣徽宣讲气学不遗余力,难道纲常二字不在气学之中?”

张璪也道:“宣徽当世名儒,这种不顾君纲常大节的话,天子之师不当说啊。”

宰辅们哪个不知道朝廷在高丽、乃至整个东海战略上最重要的关键是什么?根本不需要韩冈来多嘴多舌。杨从先的事,韩冈也要担一份责任。本来他站在旁边时,还可以暂时放他一马,但现在既然跳出来,那就当章惇一样对付。

“纲纪当然是根本大节。高丽群臣见识不足,所以罔顾大节。杨从先又是武将,不识大体,为金悌所惑。这都是不可否认的。”

伊尹、霍光的例子就不用拿出来说了,高丽群臣废王勋时所引用的例子,便是尹、霍二人。方才章惇与东府宰执们争论了半天,就这两个先例,没有少争执。韩冈想要做的,只是转移注意力而已。

韩冈说得轻巧,但杨从先的责任怎么能这么简单的就给他洗脱掉?蔡确当即便道:“那么高丽群臣的奏请怎么办,是要驳回去吗?”

“那要看太上皇后的处断了。”韩冈转向向皇后:“以臣之见当从朝中选派一名良臣,领一部兵马去耽罗,着其总理高丽内外事。之后究竟是让王勋退位,还是让他继续为高丽国主,都无关紧要了,真要说起来,还是让他留在国王位置更好的一点。”
