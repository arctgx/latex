\section{第二章 天危欲倾何敬恭(七)}

辽国在大宋的耳目众多。

因为辽宣宗耶律洪基的缘故,以及板甲、霹雳炮的功劳,有韩冈主持的《自然》杂志,当然是探子们关注的重点。

《自然》虽不涉及军事技术,但里面许多内容,只要有足够的认识,都能引用到军事上。

虽然根本无法统计,但韩冈确信,每一期的《自然》,以及他、苏颂和沈括这等精通自然之学的学者历年来的著作,都大量的流传到了辽国。

据不同方面的回报,讲究实证的气学,由于耶律乙辛的提倡,在辽国的儒生中已经蔚然成风。相较在大宋,气学左突右支,还是难以压下其他学派的境遇,在辽国国中,气学快要将那些老派的儒生赶尽杀绝了。

虽不能宣之于口,可韩冈还是很乐于见到这样的局面。

科学的发展,不可能局限于一个国家,知识的传播,也远比商品更容易。辽国人口虽少,文化程度亦低下,但亦有千万人口,其中人才不在少数——在另一个世界,数百年后的西方,又才有多少人口——那么多学者在一起切磋砥砺,悉心问学,不是出不了大家。

有压力才有动力。气学想要在大宋国内发展,辽国对气学的看重,必然是最有分量的砝码之一。

正是了解到了这一点,当辽国入侵日冇本,趁着在士林中被引发的的风潮,韩冈便毫不犹豫的将日冇本的矿藏卖给了辽国。

以韩冈在辽国的信用,加上日冇本现在已经开采出来的一些矿藏,辽国必然会将小心思放在山里地里。日后让日冇本拥有黄金之国别称的那些金矿银矿,说不定很快就能被翻找出来。

一旦日冇本的矿藏真的开发出来,得利的并不完全会是辽国。

穷人乍富,不一定是好事。

大宋的丝绸、布匹、瓷器等日用品、奢侈品,将会源源不断流入辽国境内,然后辽国手中的金银也将会源源不断的流入大宋。国内欠缺的硬通货,也能从这样的贸易中得到充分的补充。

当然,宋辽之间也不会光是有本钱的买卖。在两国的边境上,总是少不了抢劫商队的马贼,那么到了茫茫无际的大海之上,又怎么会例外?

也许十几二十年后,一艘满载着黄金白银的宝藏船自日冇本的港口驶出,行向辽国本土,当十数日过后,港口在望,船员们欢呼鼓舞的时候,几艘挂着骷髅旗的战舰从晨雾中缓缓穿出,让欢呼声戛然而止。

想想,还是很有意思的。

韩冈没有将自己的谋算告知于人打算。对王厚,也只是将《自然》上刊载的内容,在寻找图书馆馆址的闲空中,当做闲聊的话题说了一遍。

王厚也只是当做奇闻异事,不清楚气学在辽国受到重视程度,就不能顺冇便联想到辽国。那不过是学术上的推测而已,而且还不见得是正确的理论。

接下来,王厚和韩冈又去了两处拟定的馆址,韩冈都看不上眼。

一处是临河岸,位于城东,汴河畔。虽有风致,可地势卑下,湿气也大,对藏书不利。同时万一京中暴雨成灾,那个地方必然要淹水。

另一个在城北,地势还算是高了,可是地皮太小,周围屋舍又多,隔不出有效的防火带,若是被牵连得一股脑给烧了,那可才是冤枉。尤其是在石炭场大火之后,对于火灾的预防,人人都绷紧了神经。火灾隐患太大的地方,韩冈不敢选。

摇着头从第三处宅院出来,王厚就感叹着:“不是水,就是火,选一个好地方这么难。”

“在京冇城买房建宅,有人能为了选址而跑上一年。”

“一年?!他都不嫌累?”

“今天才一天,处道你怎么就累了?”

“累?玉昆,要说每天骑马的时间,你可远远赶不上我。别说骑术了,就是弓冇弩枪棒,如今你也不一定能赢了。”

“我骑术本也没多强,弓冇弩枪棒也都是野路子,处道你赢了我也算不上是本事。”

王厚在西北,手下皆是桀骜不驯之辈,光是靠王韶和韩冈如何能让人心服口服,日夜操练武艺,水平大涨。韩冈可不会跟他比。

王厚轻笑了一声,“今天就当是逛一逛东京冇城了。”他看着川流不息的街道,“兰州可没这么好的景致。”

韩冈今天一个下午都是拉着王厚东奔西走,这根本就不像是当真打算找一个好地址,的确像是在带着王厚游览东京风物。

他若真要为大图书馆选一个合适的位址,只要将要求一条条列出来,让手下人去操办就够了,自己根本没必要浪费一个下午的时间。

王厚倒了乐得多于韩冈联络感情,又是难得上京一趟,兰州在西北虽可算得上是繁华,但与京师一比较,那就连乡下的村庄也不如了。

“真要喜欢京冇城的景致,处道你愿不愿意回京任官?”

王厚与韩冈是生死之交,又有姻亲,如果韩冈在宰执位置上,当然并不方便将王厚调回来。

但现在韩冈已经卸职了,既不是宣徽使,又不是资政殿学士,担任了与宫观使相当的大图书馆馆长,私下里已经有人称他是柱下史——这是老聃曾经担任过的职位。不过实际上应该是征藏史,柱下史则是御史的前身——不过连衙门都没有。

没有韩冈这个干扰因素在,王厚调回京冇城不是什么难事。

王厚皱起眉头,沉吟起来。

“这事不急。”韩冈见王厚的样子,就笑道,“处道你可以慢慢考虑。”

“玉昆。”王厚转向韩冈,沉声道:“如果你有事需要王厚出力,只管说,调哪里都没问题。”

韩冈听得出王厚的话中之意,“处道你还是想留在陇右?”

王厚追忆起过往:“当年先君让我在陇西任官,就是希望王家这一支能世镇西北,两三代下来,也能出一个将门世家了。”

“但现在吐蕃臣服,西夏灭亡,王舜臣又打到了西域去。就剩个辽国,会打起来的地方还在河北、河东。”

“是啊。”王厚微微苦笑,“十年前那是想也想不到会有今天的局面。现在在兰州,教训兵马、巡视寨堡都比不上劝农劝工来的事多了。”

王舜臣开辟了西域,又有甘凉路在西北,西夏本路也变成了宁夏路,兰州已经不能算是边地,而是西北中枢要郡之一,控扼通往西域的要道。在往来通商上的任务,比起军事来,还要重上许多。

“西北已经太平了。这不会天上掉下来的,是从襄敏公开拓河湟开始的。当年襄敏公在古渭寨中,对着地图殚思竭虑,不正是为了今日?”

“可惜先君没能看到今天啊。”王厚轻声一叹,感觉到气氛有些沉了,随手指着不远处的巷中,两间围墙看不到头的宅子,笑问韩冈,“玉昆,哪里是哪家皇亲国戚的府邸?”

韩冈也顺着改变了话题,望了过去,“哦,那是二王邸。”

“二王邸?”

“二大王,三大王的宅子。原本是马军教坊,后来改建的。”

“疯病才好的二大王?”王厚冷笑了一声,“朝廷对他还真是宽待。”

说着,他往冇那边又望了几眼。就发现有人就守在两家王府门前不愿,看着像是做买卖,可落在王厚眼中,却有着说不出的异样。

“细作?!”王厚话出口才发觉不对,“……什么人?”

“官家的人。皇城司的。”

“一直都盯着?”

“当然。”

王厚撇了撇嘴,也不知是冲谁了。

“先帝的丧期已经过了大半。小祥过了,再过几日就是大祥。那时候,盯着二大王、三大王的人还会多。”

“都快二十天了,过得还真快。可惜回来得不巧,樊楼盛景是没法儿见识了。”

“除非处道你肯留到百日后。”韩冈笑道。

天子之丧,以日易月,所以十二日的小祥,等于就是周年祭。而二十四日的大祥,便算是两周年,再过三天,就算是服完丧了——一般来说,三年丧是连头带尾,也就是两年出头便算是三年。曾经有服丧二十五个月的说法了,但如今通行的还是二十七个月——不过以日记月之后,天下禁乐的时间,还是多达百日,这点是不会变的。

“那还就真要在京里做官了。”王厚也笑了一笑,双腿一夹马腹,往前行去。

韩冈也驱马前行,却又回头望了一眼两间王邸,心中带着疑惑和提防。

如今情况顺利得过分。怎么想,太皇太后和二大王都不是息事宁人的性格。尤其是二大王赵颢,都装了疯子。如今又看到了机会,怎么会一点不折腾?他选在这时候病好,不正是想争一争的打算?

但只要向太后那边能稳得住,怎么折腾都没用的。尤其是二大王,他的名声都臭了,怎么还能去争?

就算真要有什么动作,自己也不是没办法应对。

不去多想,韩冈掉头而去。

……………………

“东莱郡公、王厚……”

半日之后,石得一念着下属送上来最新情报,陷入了沉思。

