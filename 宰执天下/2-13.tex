\section{第五章 平蛮克戎指掌上(一)}

【第一更,求红票,收藏】

赵顼今天的兴致很高,自昨夜收到两份急报后,他的心情便一下转好。因为上个月,又夭折了一个儿子的痛苦,到今天已经烟消云散。

一个是绥德城那边的消息。绥德城是横山中无定河的枢纽要地,自从两年前种谔设计攻占绥德之后,西夏为了夺回此城,连连派大军攻打。前段时间,西夏权相梁乙埋甚至学着宋人的样,在绥德城北,一口气建立了八座连环寨,试图用一个寨堡群来抵消宋军占据绥德城后,逐渐在横山确立的战略优势。

梁乙埋的策略看似很有效,因为自八座连环堡建立之后,绥德城的守军便杜门不出,任凭党项骑兵在城下耀武扬威。但就在七天前,鄜延路主帅郭逵在忍耐许久之后终于出手,遣大将燕达自绥德城中攻出,西夏人猝不及防,一日间八堡尽毁,守军狼狈逃离。此一战,宋军败敌愈万,斩首数百,实为绥德立城以来第一功。

一个则是来自秦州的奏章,另外还附带了几份弹劾,都是说了一件事。就是秦凤经略司机宜文字王韶,于前日集七家蕃部之力,一举击败近日颇为不顺的托硕部,俘其族长以下首酋百余人。

无论是绥德还是河湟,这两件事,都是赵顼近年来最为关心的事务之一,同时也是朝廷在关西确定的主要战略。两地同时来了捷报,赵顼当然心中难掩喜意。

虽然王韶那边还是被弹劾,说他不守经略司之命,私自联络蕃人。但这个指责很无稽,因为王韶的职司就是提举秦州西路蕃部,他能召集到七家蕃部,反而是他为人忠勤职守,行事卓有成效的明证。

故而今日赵顼在崇政殿中,便命他的宰执们一起商议该给王韶和燕达什么样的赏赐——至于郭逵,他的官职已经升得太高,都已是节度留后,总不能因为一场小胜就封他做节度使。那可是从二品的官位,而现在的两位宰相都还没有从二品,郭逵升得太高,对宰执们来说也是不想见到的,所以仅是加封他的食邑。

燕达的赏赐很快定下了,虽然文彦博还是酸酸的说了几句怪话,批评赵顼妄开边衅:“鄜延自绥德立城以来,日日烽烟不断。郭逵虽遣燕达破西贼围城八堡,但西贼败而不损,不久之后,必然再起大军。”

但文彦博如今势单力孤,原本与他一起拖人后腿的吕公弼最近终于离开朝堂。尽管吕公弼一走,文彦博在枢密院是一人独大,但到了崇政殿上,形只影单的他,就被王安石压得喘不过气来:“西贼连番攻打绥德,又不惜人财物,连设八堡围城,由此可知绥德之重,实甲于横山。西贼即重绥德,我又何能弃之?”

“燕达之赏不必多言,依功赏之制照常赏赉便可。”赵顼很干脆的加以处断。燕达的功劳明明白白,没有什么可说的。

天子下了决断,文彦博摇了摇头便不再多说什么了,垂下眼帘,退入班中,仿佛入定了一般,他这么快就宣告放弃争执,让赵顼都觉得很不习惯。但少了文彦博的反对,赵顼也觉得轻松了不少。接下来,他又问道:“王韶之功又该如何封赏?”

“此事王韶无功而有罪!”文彦博又站了出来,六十多岁的老臣,依然声如洪钟,冲杀在反对变法的第一线上。

方才在绥德和燕达方面的退缩,本就是为了在王韶和河湟这件事上蓄力。文彦博在朝几十年,早就是老狐狸褪白了毛成了精。若是每件事都硬顶到底,天子听听就会厌了,下面的话便听不进去。有些事可以说几句就放下,这样其他更重要的事情,就可以重点攻击了。事分主次,时分前后,文彦博很清楚今天哪件事可以作为突破口。

“王韶不尊将令,以诈术取功。向宝一路钤辖,为其所诓,以至阵前中风。此人此事如何可以论功?!”

赵顼倒觉得无所谓,在他看来,王韶拿大张旗鼓的向宝做幌子,自己却潜渡古渭调集蕃部兵马,打了个托硕部措手不及,这是古之名将才有的智术,近人罕有一见,是难得的人才。他笑呵呵的说着:“自来兵不厌诈……”

“向宝可不是兵!”文彦博厉声说着,“王韶为人诡谲,心怀狡诈。军议中,王韶亲举向宝为主帅,事后却连夜入古渭,召集七家蕃部。向宝忠于王事,却受此奇耻大辱,再以此事厚赏王韶,非是朝廷优待重臣之道。”

的确,向宝在赵顼面前也是露过脸的,听说他被王韶气得中风,赵顼也觉得王韶做得过分了一点,要是能在事先透露给向宝两句……赵顼这么想着,突然自己都觉得好笑。这怎么可能?!两边早就跟仇人一样了,王韶怎么可能透露自己的计划,向宝也不会为王韶守秘。

王安石出面为王韶辩解:“托硕部被王韶以七家蕃部合攻,不费朝廷一兵一卒,便俘其族主,汉之班超也不外如是。向宝之事,是其气量太小,也算不得王韶的错。”

“越是得胜轻易,越是得谨慎小心。今次得胜轻易,下次得胜轻易,终有轻易不来的时候。唐明皇便是因为西域屡屡大胜,而忘记了虚外守中之理,将朝中精锐尽数付与胡人,最后至于有安史之乱,马嵬坡之厄!”

文彦博说得声色俱厉,他还记得赵顼刚登基时,就穿着一身甲胄跑到曹太皇和高太后面前,问着自己这身盔甲穿得怎么样。虽然给曹太皇训了一顿,问他天子须着甲的时候,国事又会如何?但这皇帝就是不吃教训,总是想着观兵四方。

难道‘兵者,凶器也,圣人不得已而为之’这句话没人教过?不知道一场仗打下来要死多少人,朝廷又要付出多少粮饷?

“兵甲不休,士卒不练,且空饷之多,骇人听闻。如此弱兵,如何堪用?”文彦博摇着头,他是枢密使,军中情弊他看得比谁都清楚。

“所以冗兵要加以编练,汰其老弱,择其可用者而留之。正如蔡挺近年来在渭州所创将兵法,便是编练士卒、加强战力的良策。”

事情哪有这么简单?!文彦博亲身经历过战争,可不相信世上会有一道命令就让士兵变成精锐的策略。他对战争的了解,比在列的十几名重臣,和坐在上面的天子都要多。

仁宗时的贝州王则之乱就是文彦博带兵平定的。王则是弥勒教信徒,他以‘释迦佛衰谢,弥勒佛当持世’的名义在庆历七年起兵,占据贝州,乱了整个河北。朝廷几次用兵不果,最后不得已,时任参知政事的文彦博自请领军。

当年文彦博出征时,仁宗皇帝很高兴的对侍臣说,此战必胜。以文彦博的‘文’,加上贝州的‘贝’,合起来就是‘败’,王则必败啊。但打仗可不是靠一个好意头就能获胜,当日为了围堵王则,文彦博和副帅明镐可是把贝州城用围墙围了一圈出来,挖掘地道,又声东击西,费尽了气力才打进去的。

在文彦博看来,赵顼高坐在宫廷里,却指点着边疆战事,实在是不知军中疾苦,跟何不食肉糜的晋惠帝也差不离:

“楚王好细腰,宫中多饿殍。陛下重武好战,一闻兵戈便欣喜不已,如此日久,边臣必有投陛下所好者,边衅再无一日而绝!”文彦博诉说着赵顼重兵事会带来的后果,他不是在危言耸听,这是他的经验之谈。

王安石为官多年,心知文彦博说的也并不算错。人都是有私心的,一旦看到王韶、郭逵、燕达、种谔等人因军功而封赏连连,总会有人见猎心喜,想着学他们一样,通过边功来加官进爵。但喝水会呛死,吃饭会噎死,总不能因此而不吃饭不喝水吧?

王安石再次出头反驳。说起来这也算是王安石的悲哀,司马光不在,朝堂诸公就他和文彦博针锋相对,其他人都是做了锯嘴葫芦。而王安石的几个助手,地位都够不上站到崇政殿上,即便吕惠卿的崇文院校书一职,也只够让他多见天子两面。

就听着王安石接着文彦博的话头,反过去质问着:“御西贼为边衅否?破逆羌为边衅否?郭逵、王韶皆是秉王命而行威福于边地,岂是妄开边衅者?至于他路边臣妄开边衅,朝中自有律例在,当会依律处置。”

“王卿所言甚是。”赵顼一等王安石说完,便立刻点头表示同意。不想再继续这番争执。

但文彦博却不肯消停下来,他转移话题:“王韶前次欺君罔上。秦州并无一亩荒田,他却敢妄言良田万顷。前罪尚未治之于法,岂可赏其微末之功?”

王安石道:“李若愚曾在广西帅司与李师中交好,王【和谐】克臣又宥于流俗之论,皆不能秉公而言。还请陛下再选派良臣,前去秦州查验。”

赵顼想了想,王韶刚立了这么大的功劳,也不便就因妄奏之事深罪于他,既然王韶坚持秦州有万顷荒田,就还是再派人去查证一番,“荒田垦殖,向来是转运司份内事。就让沈起再去一趟秦州,他是陕西都转运使,去秦州正好名正言顺。”

