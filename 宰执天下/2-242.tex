\section{第35章 重峦千障望余雪(七)}

天色将晚的时候,王安石方才离开崇政殿。

在崇政殿中,天子问了王厚不少问题,王安石都听在耳中。

赵顼最为关心的是明年河湟决战的情况。一开始就问韩冈是否有私酿酒水,也怕熙河经略司人人私心,不肯用心于国事。而后在王厚口中,听到王韶、韩冈的一番筹划,赵顼的心情也是变得难得一见的欢畅。

当王厚趁机向他请求转为武资,声称要为大宋封狼居胥,赵顼便是一口就答允下来,还很高兴的亲口许诺王厚在转武官时,可以援例提升一级。

天子的心思都放在河湟决战上。对这个明年开春决战的计划,王安石也并不反对。熙河经略司的指挥水平,以及西军的战斗力,早已经在这几年中,通过一次次大捷而得到了验证。

王安石只是觉得时间看上去有点紧,如果能在攻下熙州后,再停上一年用来安置移民和开辟周围良田,有了足够的钱粮补给再行开战,可能会更为稳妥一点。

不过王安石他也明白,自己需要这份功劳,赵顼也很想看到这份功劳。天子变法,是为了内圣外王。对内,使百姓安居乐业,对外,让四夷宾服恭顺。

如果把三司条例司当作新法正式开始推行的标志,那到如今已有三年之久了。不过现在地方上推行各项新法条令的阻力依然还是很大,各项条令带了的回报虽多,但怨言也不见不少。王安石迫切需要一个军事上的胜利——一个决定性的胜利——来向天子证明推行新法的效果。

前面横山攻略以失败而告终,河湟就再容不得半点失败,而且必须尽快见到成效。

幸好河湟那里情况很不错,至少要比韩绛当初的陕西宣抚司要好。

王韶本人是难得的帅才,在他指挥下,河湟捷报频传,两三年内,便将熙州、巩州收归大宋,官军兵锋离着河州就只有一步之遥。而经略司内,高遵裕、王中正之辈又能与之和衷共济,人和这一项上,完全不用让人忧心。

加之一众属吏、将校都是少有的干练之才。尤其是韩冈,不论是从军事还是政事,哪一个方面都是极为出色的年轻人,有他主持后方诸务,可以让前线奋战的将士毫无后顾之忧。

“韩冈?!……就是前年和去年来过家中的那个韩冈?”

浑家吴氏的声音传入耳中,王安石猛然惊醒。不知什么时候,他竟然已经身处于家中,老妻吴氏正坐在对面。一考虑事情,就忘了周围的事,这毛病他到现在都没能改掉。

“什么事?”王安石疑惑的问着。

“还能是什么?二姐的事啊!”吴氏只当王安石犯了迷糊,但前面丈夫说出的那个名字,让她沉吟起来,“韩冈的确是不错,家世虽说差一点,但二姐若是嫁过去,反而是件好事。就是有些风流了些,这点不好。”

前两年韩冈两次入京,吴氏都见过那个上门来拜访的年轻人。能两次进相府,当然是得到了自家夫君的看重。以自家夫君的眼界,人品那是不会差得。而韩冈留给吴氏的印象很深,也很不错,相貌、气度、前途、才学都很出色,与二哥的关系也很好。

而且韩家小门小户,没有太多的牵累。如果二女儿当真嫁过去后,不会像嫁到吴家的长女那般天天受气。就是韩冈前次为了个名妓,跟天子的弟弟闹得满城风雨,最后让官家出头收拾残局,这一点终归有些让人感到犹豫。

“蓄养歌妓的事也听多了。韩冈才一个,也算不上什么。这两天,就得找人做个媒,你看看谁人合适?”

吴氏一头热的说着,王安石有些恼火:“胡说什么?我什么时候说要把二姐嫁给韩冈了?二姐的事急不来的!再说,还不知道韩冈那边有没有定下亲事,小心落了空。”

“急不来?那还要等几年?”吴氏一下变得满腹怨气,直冲着王安石嚷嚷。为着二女儿的事,她日日心急如焚,只是见着丈夫忙碌,不想去打扰。但今天终于忍不住了,“天天想着治国平天下,这修身齐家,你做到哪一样了?!二姐转年就十八了,你这做爹的坐得稳如泰山,我这做娘再不多想想,二姐就要成嫁不出去的老姑婆了!”

“也不一定要韩冈。”王安石见着吴氏听到一个年轻人的名字,就盯着人不放,就好像自家的女儿嫁不出去一样。他王家的门户、家教也不差啊,至于这么急切吗?

“就算不是韩冈,其他人家也行……你总得找个好人家来吧?”吴氏还是急着。

王安石皱起眉:“如今找上门来的,都是些趋炎附势之辈,哪有几个正经人家?!”

“那就去找!”吴氏提声叫道。

“爹、娘!”一人适时的推门进来,打断了书房中的争执。

“大哥!”

见着是儿子王雱进来,吴氏讪讪的停了口,在儿女面前吵架,不论是王安石还是她都是有些难堪。

王安石咳嗽了两声,问道:“大哥,有什么事?”

“厨中已经把晚上的饭菜做好了,正等着爹娘来呢……”王雱回头望望门外,“本是二姐来的。但见着她久不回来,儿子就过来看看。”他笑了笑,“也难怪她不好意思进来。”

“二姐在外面?”吴氏闻言,狠狠地瞪了王安石一眼,忙着出去追女儿了。

王雱躬身目送吴氏离开,这才走近前,对王安石劝道:“爹爹,二姐的事也的确得加紧操办了,总不能再拖了。”

“你也觉得韩冈好?”

“韩冈儿子是没见过。但从传闻中听来,人品并不差。文学上虽是稍逊,可其才干已是名传朝中。如今不过是弱冠之年,已积功为朝官。观他过往行事,对变法每多援护,当是有心于国事的人才。”

同样名满天下的年轻俊杰,心高气傲的王雱并不会认为自己比韩冈稍差。本官同为太子中允,但多了一个进士头衔,还是崇政殿说书,有着天天面见天子的资格。评价起韩冈便是很客观,没有半点嫉心。

“这为父也知道……”

王雱在王安石身边坐下来:“韩冈第一次上京时,给爹爹出的三条策略,无一不是扭转乾坤的上上良策,可见韩冈对新法的一片至诚。他又几次拒留京中,更足见其并非趋炎附势之辈。”

“就是太过头了。”王安石摇着头,“青苗法改名、胥吏重禄,这两条都还好,但第三条……”

“比起舜去四凶的征诛之术,韩冈定得的条策,已经是很温和了。新法诸多条令,哪一条不是卓有成效,大人如今何须再顾忌着那些愚顽之辈。找孩儿说,就得征诛今之‘四凶’,将之远窜四荒!”

王安石看着侃侃而谈的长子,暗自叹息着。年轻人都是这般无所畏惧,牵挂少、顾忌也少。就像韩冈,随口几句话就要挑起党争。而他的大儿子,也是年轻气盛的不把党争后果放在眼里。只有在官场上多待上几年,才知道不是事事都能强着来的。

那些被他打压下去的旧党中人,都叫他拗相公。说他王安石是一意孤行,不听人劝。可若他真是这般行事,这些年来的诸多新法,早就全数推行下去了。何须一条条的在一路或几路中先试行,查看结果后,进行相应的修改,才会推行全国?——王安石只是不理那些旧党胡言乱语的掣肘之词而已。

“大哥,你真的觉得韩冈好?”

“是不是韩冈,孩儿不便多说。但总得找个与爹爹你同心同德的人家。”王雱停了一下,语气沉重叹道:“总不能让二姐也‘和泪看黄花’吧?”

王安石默然不语。

‘和泪看黄花’是他长女写的诗句。嫁到吴家的大女儿是王安石全家心头上的一桩恨事。她自小聪明灵慧,又工于诗词,极得疼爱。王安石左挑右挑,特意挑了好友吴充的儿子。偏偏因为变法之事,两家生分了,让大女儿在吴家过得很不舒心。

秋天的时候还寄了封信来,上面写了一首七绝:‘西风不入小窗纱,秋意应怜我忆家。极目江山千万恨,依然和泪看黄花。’

“让为父再考虑一下……总得先问问韩冈到底有没有定下亲事。”王安石叹着,国事、家事,事事让人烦心。

他问着王雱:“你做着崇政殿说书哦,明天就要上殿宣讲,可准备好了没有?”

崇政殿说书的位子不好坐,不但要像天子讲解经史要义,同时也是天子身边的顾问。必须见闻广博,又精通经史,少点才学就会被天子问得张口结舌。而且说出的话,多少只耳朵听着,仁宗朝被任命为崇政殿说书的贾昌朝、杨安国,他们两人旧日的文名,便是因为说错了几句话,被人引为笑谈而一落千丈。

王雱虽然不是第一次上殿宣讲,但王安石作为父亲,总是要担着一分心。

王雱自信的笑起:“以孩儿的才学,爹爹何须担心。这么多次下来,何曾出过丑的?”

