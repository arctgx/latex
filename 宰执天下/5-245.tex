\section{第28章 官近青云与天通(12)}

虽然御史们如同吃了五石散一般兴奋,让向皇后为之头痛不已,可韩冈还是照旧在他的太常寺中编修药典。

不在其位,不谋其政,许多事就算知道,他也不会插手。而且眼下他还有更重要的事要做——《自然》杂志的第一期,每一篇文章都要修改再修改,以求无懈可击。这个开门红,他是一定要打响的。

拿着笔,韩冈仔细抠着文章中的一字一词。偶尔改上一两个字,以求用词更加精确。以实为证的思想贯穿始终,可要跟以新学为首的一干儒门学派打擂台,文字上也不能留下太多破绽。

苏颂进来时,就看见韩冈紧皱着眉头在稿纸上一点一画,咬文嚼字的模样,跟贾岛苦吟作诗也差不多。完全没有在朝堂上挥洒自如的豪快。

“玉昆。”

苏颂的声音打断了韩冈的思绪。抬头看见苏颂,他连忙起身相迎。

“子容兄今天来得早啊。”

“不早了。”苏颂径直就坐了下来,“玉昆,相州的事你可知道了。”

苏颂的脸上有着掩藏不住的急色,韩冈有些惊异,“当然是知道了,相州献上了一件两千斤重的方鼎……难道还有什么变化?”

“还要什么变化?!”苏颂真的有些急。在他看来,相对于高喊着祥瑞祥瑞的相州众官,直接将殷墟带入人们的视野的韩冈,则更为冷静——冷静得过了头。这件事,可不是那么简单,“殷墟如果只是甲骨还好,发现礼器也没什么,但现在不是天子重病吗?!时间上可是太不巧了!”

“这话怎么说的?难道还是因为这件方鼎才让天子发病的不成?”

“难道不会有人这么想?!”苏颂反问韩冈,他叹道:“旧日新法鬻河渡坊场,以至于司农、祠庙皆在买扑之列。南京阏伯、微子庙亦在其中。张方平谏阻道,‘宋王业所基,阏伯封于商丘,以主大火;微子为始封之君,是二祠者,亦不得免乎?’天子由此震怒,批语‘慢神辱国,无甚于斯!’天下祠庙由此皆得保留!……玉昆你将两件事连起来想想?!”

苏颂说的事情,韩冈旧时也曾听说过——韩家在秦州旧居旁的李将军庙,也曾经传说过要承包出去,不过这小道消息传传就没了下文——南京应天府,故名宋州,军额归德军,而太祖皇帝旧为归德军节度使,所以国号便来自于此。微子是殷商遗民,为周室分封于此,立宋国。阏伯更是殷商之祖,高辛氏之子,一名契【注1】,葬于商地,商丘之名便来自于此。

“那又如何?难道子容兄你还担心能治罪于你我不成?”

“玉昆,话不是这么说的。”苏颂对韩冈满不在乎的心态都快没话可说了,“愚兄倒也罢了,你现在可是众矢之的啊!”

韩冈摇摇头,依然不放在心上:“盗墓者刑,毁人坟茔者罪,谁挖的墓,谁当然就有罪。可若仅仅是从地里掘出来的古物,又何须担心?殷墟那是殷人故都,却不是殷人坟墓啊。那件方鼎难道是出自殷人墓中吗?奏表上可没有这么写。”

韩冈真的不在意。要不要翻一翻朝中百官的家底,金石古玩里面有多少是从土里掘出来的古董?又有多少根本就是随葬品?就是宫中,随葬品也是极多见的。更不要说隔三差五献上来的祥瑞了。在石上发灵芝、稻麦生多穗,鸟兽翔舞云云之外,从土里发掘出来的上古礼器也都是有资格进献天子的祥瑞。

不过看见苏颂越来越难看的脸色,韩冈摇摇头,干脆的明说了:“子容兄,你实在是多虑了。宋之国号的确来自于微子,但他是周室所封殷之遗民。阏伯虽为高辛氏之后,可上灵高道九天司命保生天尊上帝还在,阏伯又哪里有立脚的地方?”

上灵高道九天司命保生天尊上帝就是真宗皇帝在天书闹剧中编出来的赵氏之祖,名为赵玄朗,曾为人皇,又曾转世为轩辕皇帝,如今号为圣祖。

 “当年张方平的谏言,不过是因为鬻售天下祠庙损了朝廷体面,天子故而震怒,如今可不一样。君子之泽,五世而斩。即便是天家,七世之后,翼祖亦要祧迁——这就是几年前的事——子姓宋氏,可能与翼祖相提并论?更别说皇宋国号,是来自于周时封土,阏伯、微子不过是沾光而已。”韩冈轻笑道:“玉清昭应宫一把火给烧光了,里面供奉的圣祖神位都一把火烧了,难道重建了吗?”

苏颂默然不语,可脸色依然沉重。

“方两丈、高五尺、台陛四、壝墙一重。”韩冈对抬眼看过来的苏颂笑笑,做了几个月判太常寺,下过一番苦功,坛庙的礼仪制度如今也算精通,“这是大火之坛的规模。主位是大火,陪祀乃是阏伯。左传云:‘陶唐氏之火正阏伯居商丘,祀大火,而火纪时焉。相土因之,故商主大火。’州县官为太祝奉礼。”

苏颂看着韩冈,等他继续说下去。

韩冈喝了口茶,继续道:“高禖以青帝为神主,高辛【就是殷商之祖】陪祀,坛宽四丈,高八尺,皇后亲祷之。”

韩冈想说什么,苏颂已经不用在听他说下去了。

高禖即是句芒,婚育之神,上巳日祭祀句芒求子乃是几千年来的惯例。排位远在大火之前,仪制更是远在大火之上,更不用说在大火旁边陪着吃冷猪肉的阏伯——阏伯的老爹高辛还在高禖旁边做陪祀呢。

求子和护佑幼子的高禖既然有如此高的神格,那么难道皇后还能因为区区殷商去跟韩冈过不去?现钟不打去打铸钟?

而且因为韩冈的缘故,慈济医灵显圣守道妙应真君,也就是孙思邈,已经有了朝廷亲封的神职,亦有保护幼子之力。真要较量起来,胜败当可知。

苏颂长叹一口气,语重心长:“玉昆,小心小人!”

当天晚些时候,从宫里传出来的消息,正好映证了苏颂的话。竟然还真有御史上书,说天子之病,乃是相州发掘殷帝陵寝的缘故,甚至直指韩冈,是肇始之因。

向皇后沉着脸走出了福宁殿。

她刚刚与服侍赵顼的宫人一起给她的丈夫擦洗过,换好了一身干净的内衣。看到丈夫如今模样,向皇后心中也是一阵酸楚。

在过去,赵顼的身体纵然不是康健壮实如同文彦博、王安石、韩冈这样的牛高马大的臣子,但也是好端端的能走能笑。可现在却变成了一个瘫在床上的残废。

尽管无法主动进食,幸好还能吞咽——这也是为什么赵顼虽然不能说话,还能发出点声音的缘故——所以食物皆是流质。主要还是酥酪,羊奶之类,再配些菜粥肉粥,就跟快断奶的婴儿差不多的食谱。

这样的生活,对于赵顼来说,理所当然是个极大的折磨。尤其他在病发前还是坐拥万里河山亿兆子民的皇帝,落差实在是太大了。可以很明显的看得出赵顼正在一天天的变得憔悴起来。

坐在崇政殿中,向皇后看着面前的奏章,思虑良久,最后招来宋用臣,“去请韩学士来。”

看过了皇后让人拿过来的弹章,韩冈却是平平淡淡,并无怒色,更无惶急,“所谓殷墟,乃是古都而已。长安、洛阳,自周、汉至唐,建都于此者不知凡几。也不见因修城而坏国家根基。”

甲骨乃是殷人占卜之物,体现的是殷人敬鬼事神的作风,并非是随葬品。这一点,有无数先秦古书可以证明。韩冈完全不担心。

而且更重要的,是他对自己的信心。

韩冈是亲眼看见王安石是怎么从荷天下三十年重望,变成了旧友皆背离的境遇。有王安石前车在前,韩冈从来就没想过将自己的根基放在士大夫身上。王安石的声望既然来自于士大夫,自然也会因为士大夫的背离而声望大损。

但韩冈声望的根基是来自亿万百姓。在子嗣艰难的皇帝面前,药王弟子的光环是韩冈的不坏金身。而就士大夫而言,一直都否认这个光环存在的韩冈依然是同辈中人,而且正好可以站在高处鄙视一下愚民。就算想攻击他,也只能从构陷上着手:面对儿子生一个死一个的皇帝,去构陷拥有保护幼子的光环的韩冈——这当然是笑话。

韩冈朗声道:“关键还是得确认此鼎是否来自于殷商诸王的陵寝之中。如此方可定案。臣请殿下,遣使至相州,以查探详情。”

韩冈的请求,向皇后考虑了一阵,点了点头。肯定是要查一查的。

只是韩冈知道,这根本是查不出来的。

所有进献上来古物祥瑞,不是房屋改建发现,就是种地时意外刨出,就是司母戊方鼎,也是说寻找龙骨时,发现了殷人祀神的坛庙。没人会傻到说是来自古墓之中,这关系到来自朝廷的是官爵之赏,还是枭首一刀的原则问题。

此外,还有一点让韩冈确认派出去的使者将会劳而无功——因为现任相州知州名叫李珣:真宗宸妃李氏的亲弟李用和的次子,更准确点说,也即是仁宗皇帝生母章懿李皇后的嫡亲外甥,是仁宗的表兄弟!

既然李珣在奏章上签了名画了押,而且还是排在第一,一旦查明那件方鼎来自于墓葬,那么他是绝对脱不开身的。可是以李珣的身份,当真能治罪于他吗?

而且韩冈还知道一点,李珣与韩琦家是有亲的,否则就不会让他去相州接韩正彦的班。没有得到韩家人的同意,这份奏章,更不可能递上来。

查得出来吗?

更别说甲骨一片,如今都是三贯往上,而外形完整的更是高达十贯。殷墟的发掘在相州已经是一门产业了,在庞大的利益面前,没可能查出真相。

而就在当日,司马光抵京。

注1:契和阏伯究竟是同一人还是两人,如今尚有争议,但从张方平的话中,是当做同一人看待。
