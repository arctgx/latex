\section{第二章 天危欲倾何敬恭(上)}

章敦是到了放衙后才过来的。

韩冈回来后就让家里的人准备了酒菜。准备等章敦来了之后,两人就在内厅里坐下来,一边喝着烫好的热酒,一边吃着精美的菜肴,这样也能好好的议论一下现在的局势。

只是章敦进门时,脸色就像是韩冈拖着一大笔钱不还一般,挂着一张脸,实在不是喝酒聊天的气氛。

韩冈见状,便先行将他引到了书房中,还是坐下来慢慢聊。

又在衙中忙了一下午,章敦已经感觉自己快吃不消了,不过头脑还是陷在了兴奋中。这种明明困得很,却偏偏睡不着的感觉。让他的心情更加恶劣起来。

瞥着言笑自若的韩冈,这还像是一个引罪辞位的官员吗?神清气爽得让人嫉妒。

韩绛担任山陵使,其他宰辅也都是忙得滴溜溜的乱转,没一个能清闲下来的。章敦之前还奉了太后旨意前去王安石府上探问,放衙后又要韩冈这边问询,前几日更是恨不得一天有二十四个时辰。

连续几天的连轴转,让章敦累得像条狗,看见因辞官而变得清闲无比的韩冈,抱怨脱口而出,“玉昆,你真够自在的。”

“人不患寡而患不均,子厚兄,你可是嫉妒了?”

“怎么能不嫉妒?”章敦坐下来,“早知道就不放玉昆你脱身了。”

辞官卸职,这叫脱身?朝中至少九成九的官员不会这么认为。不过在韩冈而言,的确是从天大的麻烦中脱身了。

“现在后悔也来不及了。”韩冈哈的笑了一声,“子厚兄,这几日可是辛苦了。”

哪壶不开提哪壶,只是看见韩冈诚挚的神情,章敦涌上一阵无力感,他不就是这样的性子吗?

放下了茶盏,章敦道:“玉昆,可知这一回三军那边还是要赏赐?”

“没听说,不过多少也能猜得到。”

大宋没有太上皇的先例,过去都是老皇帝死了、新皇帝即位,然后大把的撒钱作为犒赏,也有收买人心的意思在。所以现在,登基和驾崩相隔甚远,在朝廷而言,当然是不想多给一份冤枉钱。可三军将士,却又有哪个会嫌好处多的?

章敦叹着气:“早知道会如此,当时就不那么急了。”

要是早知道赵煦才登基几个月,赵顼这位太上皇就驾崩了。当初就没必要那么急着内禅了,现在也不用纠结皇帝被杀的问题。但话说回来,要是早知道赵顼会因为事故而亡,做个预防也没现在的事了。

韩冈的嘴皮子动了动,将这话咽了下去。经过了之前章敦和王安石的提醒,他有了几分自省。回想起来也真是有问题,性格上好像是有些变化了,越来越喜欢与人争执了,如果只是学术上那是没什么,但跟家人和朋友相处时也如此,可就不是什么好事了。

“最后是怎么办的?”韩冈改问道。

“拿出了甘凉路上的土地作为赏赐。想要的自己报名就是了。还敢闹的,自有军法等着他们。”

韩冈想了一下,明白了什么叫做自己报名,“流放甘凉?这倒是好事。不过……不可能一点好处不给吧?”

赵顼枉死,在世人眼中必然是阴云重重。烛影斧声传了多少年,熙宗皇帝之死会怎么被编排也不难想象。这时候更是必须要以厚葬厚礼来向外展示,免得给人更多的借口。

“得靠铸币局了。今年的税赋早就有了去处,一文都动不得。大行皇帝的这笔开支,只能靠铸币局。中间有个差错,连弥补的手段都没有。”

这一年,花钱的事一桩接着一桩,内库外库里面的积蓄像破堤的洪水一样往外涌,看着帐籍上的数字,就让人怵目惊心。章敦算是知道当年吕夷简、范仲淹看着西北军费泛着跟头往上涨的时候是什么样的心情了。要不是韩冈开辟了新的财源,用大量铸造新币来填补亏空,还不知道怎么将局面支撑下去。

“钱币的质量不下降,就不会有事。天下钱荒有多严重,不必韩冈多说。只要百姓还愿意使用,不论铸出多少,民间都能容纳下来。”

“就是玉昆你不在才让人担心。”

“只要审查上没有疏失,换回还不是一样?中书门下和宫中派出人都要加强监督,若有过犯,直接夺官发配,谅也没人敢一试王法。”

韩冈不担心查不出来。各处钱监使用的铸币模板都有细微的差别,每年又会有一个变化,质量有问题的新钱,立刻就能查出源头来。要担心的只是执法的问题。

“玉昆,你写的钱源论,自己都忘掉了?钱币有价值是因为信用,论起信用,天下数十钱监的提领加起来也比不上玉昆你一个人。非是愚兄妄自菲薄,论起信用,朝廷中没人能比得上你,愚兄也远不如。”

“子厚兄太高看韩冈了。纵是如此,也总得习惯过来。”

“这话应该将根基扎好再说的。玉昆你在铸币局的时间太短了,有个三五年,才能将信用建立起来。现在猝然放手,天下军民都有疑虑啊。”

韩冈皱起眉头,盯着章敦看了一阵,“子厚兄,你今天过来,该不是做说客的吧?”

“这是一件事。铸币局和火器局不是什么好差事,但朝廷还是希望你现在为太后和朝廷分忧。”

铸币局虽然是新设,但谁都知道,其前身是三司盐铁司下面的衙门,让韩冈这种入过两府的前任执政,专责任职,这可是比罢官夺职还要严重的羞辱。韩冈此前虽是任职,但那也只是兼任,正职还是在宣徽院。

章敦相信以韩冈的为人应该能体谅朝廷的难处。但他也明白,这个要求过分了。

“没问题啊。”韩冈一口答应下来,干脆无比,想了想,又道:“顺便把皇宋大图书馆的馆长一职给小弟好了。这样文武财俱全,也算是圆满了。”

“皇宋大图书馆……馆长?”章敦略一思忖,点头道,“这事好说,可比照宫观使。玉昆你看如何?”

“这该是朝廷决定的,怎么定就看朝廷了。”

宫观祠禄官算是朝廷用来养老、养闲的地方,在里面任职的官员都是拿俸禄不管事。宰辅去职,若不是出外为官,多是会出任宫观使,比如景灵宫使、太一宫使,而地位低一点的官员,就是提举、管勾,比如提举洞霄宫、管勾崇禧观之类。地位太低了还去不了,至少得是知州以上资序的官员。若是所谓馆长能比照宫观使,等于是将还在纸面上的皇宋大图书馆提到了宰辅一阶。

但这么做,也不会有太多非议。天子储才的崇文院,也就是三馆秘阁,性质同样类似于图书馆,只是不对外公开。而宰相,都是要兼任三馆之职——昭文馆大学士、监修国史以及集贤院大学士。纵然公开的大图书馆比不上天子的私家图书馆,就如外制中书舍人比不上内制翰林学士,却也差不了多少了。说不定日后就有内馆外馆之分了。

“好了。明面上的事应该就这件吧……”

韩冈坐得正了一点,闲话说了一通,但应该只是章敦明面上过来的理由,真正的原因,不可能是这么点小事。

“嗯。”章敦点了点头,神色更严肃了一点,“玉昆,你可知二大王做了什么?”

“二大王不疯了?”

“今天来拜祭梓宫了,你说还疯吗?”

韩冈冷笑了一声,“……太心急,不足为虑。”

“自是当然。他要敢闹,一丈白绫,二两牵机,朝廷也不会吝啬这点赏赐。但这只是个开头。”章敦脸色阴沉如晦,“玉昆。太皇太后今天也过来,你可知道?”

母亲来拜祭儿子,顺乎天理人情,没有阻拦的道理。可是这位高太皇,又哪里是会息事宁人的?

“可是出了什么事?”

章敦慢慢的摇了摇头,沉着声:“一点都没有。”

韩冈不由的皱起了眉头,事有反常必为妖,这可一点也不正常。

“现如今只是群臣轮班守灵,等过了二十七日,天子领群臣祭奠梓宫,当着文武百官的面,玉昆,高太皇闹起来,又会是什么样的情况?”

图穷匕见,韩冈明白,章敦这是来算账了。

当日宰辅们之所以没有强行要求向皇后行废立之事,并不是觉得现在时间还早,赵煦的身子骨又不好,可以熬着等他死。只要赵煦还在殿上坐着,一想到日后亲政会怎么对付,哪个还能睡得好觉?

可是当时情况太过突然,有心废立天子的宰辅没时间相互串联。加之韩冈力保赵煦,向太后也被他说服了,唯一能压下韩冈的王安石又绝不会同意将小皇帝废掉,所以才没有立刻发动。

而韩冈的提议,更是撒手锏。

‘只要太上皇的血脉还在,就不容其他人坐上去。’

一番话似明非明,的确可以有另外一番解释。如果要变通的话,就是让向太后长久控制朝堂,等到赵煦生出继承人,便让他去做太上皇,弑父之人不可王天下,剩下的就是顺天应人了。

但现在冷静下来,宰辅们的心思又是一变,总得十年后考虑。

无论如何,韩冈都必须给个交代。

