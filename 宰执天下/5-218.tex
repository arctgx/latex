\section{第24章 缭垣斜压紫云低(十)}

和几名嫔妃惊喜的叫声,惊动了厢房内外所有正关心着赵顼安危的人们。

赵颢瞥了一眼过来,神色中带着惊疑甚至是一丝惧意,但没等韩冈仔细分辨过赵颢的表情,他就又换上了欣喜欲狂的面具,凑近了盖着黄绫被褥的御榻。

韩冈轻轻摇头,挡回了紧跟着投射过来的几道惊奇的视线。这真的是巧合,绝对跟他没有一点关系。不过韩冈进来前,几位御医又是扎针,又是灌药,只要赵顼不是快要断气,这么一番折腾,怎么也该醒了。

但除了韩冈本人以外,其他的人都不约而同的向韩冈望过来。没人会以为这是自然而然的结果,韩冈作为药王弟子,不管怎么说也当有一份特殊的能力存在。就算韩冈不通医术是肯定的,但特异的能力,也能让他表现得面面俱到。

病榻上的赵顼睁开了眼睛,围在榻边的人们,不论真情假意,浮在脸上的都是惊喜关切的笑容。

韩冈的眼中有着淡淡的同情,昨日还是一言九鼎的天子,今天就成了病榻上的残废,这样剧烈的转变,不知道赵顼能不能接受得了?

但韩冈很快就发现他多虑了,这个发现让他的心沉了下去,沉浸在了最深的海沟的最底层。

赵顼的视线漫无焦点,从他睁开眼后,就让人感觉他眼神中充满了茫然。在皇后嫔妃还有儿子呼唤下,也看不出有多少变化。而他脸上的表情因为中风的缘故,变得很是怪异,更是让人捉摸不透。

过了片刻,在众人越来越失望的时候,赵顼发出了声音。他张开口,可并不是过去听惯了的金口玉言,仅仅是从喉间发出一阵荷荷的怪声。

看着这样的丈夫,向皇后一下变得失魂落魄。朱贤妃也用力搂紧了儿子。只是在过于年幼的赵佣的眼瞳中,依然透着茫然。

韩冈的脸色微微泛白,掩在袖中的手攥成了拳头。赵顼的病情远比想象得要重得多。这一次的中风,从时间到结果,都是往最坏的方向发展。

通往外殿的通道上响起了急促的脚步声,随即章敦便出现在门外,又毫不犹豫的跨步进房。

“陛下醒了?”

“天子醒过来了?”

“已经没事了?”

几人同时出声,随即章敦领头,蔡确紧随,然后是王珪、薛向等几名宰执鱼贯而入,吕公著虽是拖在最后,但磨磨蹭蹭的也走了进来。几名内侍追在宰辅们的身后,却都没有敢拦着他们。

在听到寝宫内殿传话说官家醒了,第一个不顾一切就往内宫闯的便是敢作敢当的章子厚。这原本应该宰相作出决定,章敦却自顾自的行动,逼得其他宰执不得不跟着一起走。

擅闯内宫自然是失礼,而且有罪,但落在天子眼里,却是不顾个人安危,更加忠心的表现。如果都没有做,那倒是无话可说。但有一人或是几人做了,那么没有动弹的,自然会被记上一笔。眼下罚不责众,最后也不会追究。

可是,但章敦等人见到了赵顼的现状,在一瞬间的惊喜之后,却又都陷入了深深的恐慌之中。

“陛下!陛下!臣是王珪啊……”王珪充满感情的呼唤着,但赵顼手指也没动一下。

经过御医们一番检查——其实也不用御医开口,检查的过程所有人都看在眼里——大宋的第六任天子,在今日的中风之后,不但失去行动的能力,甚至连开口的力量也失去了。

韩冈以沉思的表情应对所有试探的目光,始终保持着沉默。

他的心中很有些疑惑。在他看来,中风虽然身体反应迟钝,但意识却不一定会受到太大的影响。就算伤到了头脑,也不是一下变得痴呆般的老糊涂。韩冈前世今身也是见过几位中风的患者,口齿不清,嘴歪眼斜,行动不便,甚至瘫痪,可韩冈却没在其中见过一位真正中风发病。就此变成痴呆的病例。

只是此时韩冈没有多余的精力去考虑医学上的问题。对他来说,这是最糟糕的结果。自然,就是赵颢最想看见的局面。

也正是因为有了先入为主的看法,韩冈才能在赵颢的眼神中找出他暗藏的欣喜。‘果然。’韩冈心道,这位二大王从来都不是让人心服口服的人选。

如果赵顼能够清楚明白的表达自己的心意,就算是瘫痪了也不打紧。一个依然掌握着权力的皇帝——尽管比之前肯定要损失一点——有足够的能力来为自己的儿子铺平通往大庆殿御座的道路。

或者干脆是赵顼一病不醒,就此驾崩。以他留下的朝堂和余威,朝臣们也完全可以推赵佣上位,而不用担心任何阻挠。

只要当下的几位宰执能坚定支持赵佣,尽管还没有出阁,但皇嗣的身份还是让赵佣能够顺理成章的继承大统。

王珪一直都是赵顼的心腹,十年来其他重臣在两府中进进出出,甚至将天下都搅得天翻地覆的王安石都已经两进两出,可王珪一直都被留在政事堂中。他就算不敢旗帜鲜明的领头用力赵佣,也决然没有胆量转头就背叛赵顼。蔡确最会看风色,一般来说不可能将身家性命压在赵颢身上。

已经完全失去存在感的吕公著,在朝堂上代表着旧党的势力。可只要还有皇嗣在,决定谁继承帝位的时候,任何一位以君子自诩的旧党臣子都很难抛弃自己的名声,去支持赵颢——尽管他一直都表现得反对新法。而且作为世家子弟父亲是前代权相,本人又经年执掌西府,吕公著根本不需要表态,他只要等待结果就够了。就算是赵颢上台,也不能动他吕家分毫,甚至还要优加宠礼。真正会在帝位传承上搏一把的,反倒是那些地位不高、名声不显的卑官小臣。

担任参知政事的韩缜,他的情况也跟吕公著类似,绝不会为了日后可能到手的宰相之位,而为赵颢搏命。同样的理由,薛向也不会差得太多。剩下的章敦,他肯定是两府之中,最为坚定反对赵颢登基的一人,都不用多想。

文官如此,武将也是一般。三衙中的几位太尉也都是在赵顼手中提拔起来的。赵顼的发病突如其来,若是在数日之间进行帝位更迭,上四军也好,班直也好,开封府中管军的将领们都不可能在短时间内与其他有资格争夺帝位的宗室——其实也就一个赵颢——搭上关系,完全不需要担心有人能动用武力来争夺帝位。

最关键的还是赵顼在位日久,而赵颢又没机会建立自己的势力,仓促之间并没有发力放手一搏的能力。

但眼下的情况就不一样了。赵顼活着,却跟死了没有两样。政务、军事、礼仪,还有继承人,一切天子该尽的义务,以他眼下的状况都没有办法去完成。依照旧例,必然是太后出来垂帘听政。在赵顼中风的情况下,一个小小的宫廷政变,就能让赵颢坐上大庆殿中的御座。

而且理由更是冠冕堂皇。为了大宋的基业着想。不能让太后垂帘太久,但让过于年幼的皇子来继承打通,同样也是不合理的。

赵顼也许只是不能将自己心中的想法化作语言表达出来,但对于一个致力于掌控天下的人来说,这样已经可以判定他不适合再坐在现在的位置上了。或许对赵顼来说,这样才是最大的悲哀,比死还痛苦。

赵佣被朱贤妃抱在怀里动弹不得,乌溜溜的眼睛往着赵顼,半点也不关心现实中发生的事。

“阿弥陀佛,真是上天保佑啊。”赵颢长长的叹了一口气,“皇兄能醒过来实在太好了。”

赵顼完全没有动静,让赵颢继续上演他那滑稽的独角戏,“外面现在肯定是人心惶惶,。”他看看王珪,“从今天开始,东西两府应该就得轮流宿卫宫城,那可就是要辛苦了。”

王珪嘴里发苦,这就是要逼宫了?虽然作为宰相,可以严词厉色的直接驳斥赵颢,王珪也的确张开了嘴,可突然间变得干涸的喉咙发不出一点声音。

章敦眯起的双眼变得危险起来。

这不是说的对还是错的事,而是有没有资格说的问题。宰辅们接下来该怎么做,天子可以发话,皇储可以建言,太后、皇后在皇帝不能也有资格说话。区区一介宗室,纵然贵为亲王,也是决然没有资格插嘴——不论说多少,也不论说什么,没资格就是没资格。

宰相这时候应该直叱其非,换作是韩琦或是王安石为相,能当场让赵颢下不了台来,根本就不会在乎高太后就坐在旁边。可惜眼前的宰相是王珪。他只顾着关切的看着赵顼,虽没有附和,可也没有叱责,浑然没有听见的样子。

王珪、蔡确不顶事,就连自命君子的吕公著都当了哑巴。帝统更迭中事,臣子没有做好觉悟,又岂敢妄自发言?

赵颢一切都看在眼里,嘴角浮起一抹浅浅的笑意,又骤然收敛,变得庄重严肃起来。

