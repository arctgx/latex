\section{第47章 节礼千钧重(下)}

“韩冈好狠的手段!”

上元节已经过去了半个月,在政事堂归属宰相的公廨中服侍冯京的亲近小吏,不意又听到冯相公又以愤然的口吻,提到了如今正春风得意的军器监韩舍人。

作为一名衙中小吏,他知道什么时候该听,什么时候又该装聋子。连大气也不敢喘,眼观鼻、鼻观口、口观心,立于桌案边,除了磨着墨的双手,全身纹丝不动,做了个泥胎塑像。

冯京浑没将身边的小吏当做人,不过他平常也不至于自言自语。只怪韩冈的递上来荐章却是满纸的杀机,让他看得心也是抽了一下——竟然是以造灯有功的名义,将军器监中的两名官员调到广南东路去任职。

这是明明白白的要人命!

冯京倒没想到韩冈下手这般狠厉,故意用着如此荒谬的理由将人往死里打。

今天是他冯京轮值掌印,韩绛休沐在家,想必韩冈就是看准了这个时候将荐章递上来。

现在如果将荐章驳回去,再回来时,肯定就不会是掐在韩绛休息的时候了。若给韩绛看见,必定会转到天子面前。而官司一旦打到御前,此前做的一些小手段都要曝光,这对冯京来说可不是好事。

不过这份荐章递上来的时机,也代表了一件事。看起来韩冈已经查出来,是谁站在铁船华灯的背后了,所以才会拿着荐章来挑衅。

算人不成还被人看破,一想到这件事,冯京心里就堵得慌。铁船明明白白的造不出来,所以冯京才会下手,谁能想到韩冈竟然能拿出个板甲来?!怎么都算计不到啊!现在想来,韩冈写了《浮力追源》,分明是就是挖了陷阱引人往下跳。

做了往陷阱里跳的蠢事,冯京连着半个月,心里的郁闷都没有消散,到了眼下郁闷的感觉则更为强烈。那灌园小儿未免欺人太甚了!

狠狠的咬了咬牙,冯京又冷静了下来,他能做到宰相这一级,绝不是那等没有自控能力的人。其实想想也不是什么大事,只要准了这份荐章,韩冈杀鸡儆猴的手段,于他冯当世就没有什么影响。

吕惠卿吃亏很大。韩冈已经在利用设立板甲局的机会,准备将监中有关锻造和甲胄方面的能工巧匠全都掌握在手中。而白彰再一去,只要一两个月,军器监就会逐渐从吕惠卿掌中脱手了。

至于他冯京,不过是丢个小卒子而已,无关紧要。军器监他本来就插不进去手,从吕惠卿换了韩冈也没什么。人死了冯京反而能安心,他门下也不缺听话有用的走马狗。

想到这里,他就提笔在荐章上圈了一下,批了个可。

‘小卒而已。’

说起卒子,冯京就想起象戏【象棋古称】,他可是好久没下了。朝中喜欢象戏听说司马光最近无事,将象戏由两国变成七国,弄出来个了战国七雄的混战来,真不知道他是不是太过清闲了。

“下没下过象戏?”冯京问着身边的小吏。

“回相公的话,小人下过,只是下得不精。”

“象戏通兵法,可以练一练!”冯京抬眼望着厅外的天空,不知在看着何方,“只要棋盘还在,胜负还未可知。无论如何,这边可是车马俱全啊……”

……………………

韩冈一家人正坐在房中,火盆生得很旺,屋外虽然冰雪厚积,可室内温暖如春。

韩云娘给韩冈捶着肩膀,周南、素心看护着熟睡中的儿女,王旖则盯着桌上的一幅棋盘,‘楚河汉界’四个字绘在棋盘中央。两边车马炮、将士象,加上一边五个兵卒,井井有条的排在各自的位置上。

“大都博奕皆戏剧,象戏翻能学用兵。车马尚存周戏法,偏裨兼备汉官名。中军八面将军重,河外尖斜步卒轻,却凭纹愁聊自笑,雄如刘项亦闲争。”

韩冈拿着檀木折扇,轻轻敲着桌面,吟着诗句。

听着丈夫曼声而吟,正专注在棋盘上的王旖抬起头来:“这是官人作的诗?”

“这是伯淳先生所作,前日写了信来。顺便还有这首咏象戏,又附送张‘九九象戏’棋谱,是跟邵康节【邵雍】一起下的。”韩冈指了指王旖面前的棋盘,“这就是为夫以‘九九象戏’为本,改了规则后的新象戏。”

“可都不一样啊。”王旖看着韩冈拿出来的棋盘,小鼻子都皱了起来,“偏将、裨将都没有,反倒多了两个士?还有,官人不是说这规则是本自‘九九象戏’吗,为什么要放两头‘象’?而且还加了砲,这不跟大小象戏一样吗?”

“象戏、象戏,没有象算什么?且都是甲士护将帅,哪有偏将裨将守在中军帐的道理。”韩冈哈哈笑着,“霹雳砲更是为夫发明,又怎么能不加上去?”

不过韩冈拿出来的棋盘上,有‘象’无‘相’——让宰相来护卫将帅,这等于是颠倒贵贱、轻重失伦。火‘炮’也不会有,两边都还是石‘砲’。

“那为什么棋子不放在格子里?”

韩冈更是理直气壮:“象戏即是用兵法,哪有大军在格子里跳的道理?全得在道路上走啊。还是大象戏的规则有理。”

“这副棋盘横九道,纵十道,是‘九十象戏’,已经不是九九了。”

“没看到伯淳先生的诗句里有汉高楚霸吗?楚河汉界当然得画上,这么一画,当然就变成十道了。”

象棋至今尚未定型。虽然上至帝王将相,下至贩夫走卒,喜欢下象棋的极多,但外界的规则亦有多种,大象戏、小象戏,程颢下的‘九九象戏’,从唐时流传到现在的‘八八象戏’,甚至还有以战国七雄为本的七国象戏——程颢在信中说是司马光别出心裁,但设计出来后,却找不到人来玩。

这些种类繁复的规则之间甚至连棋子都不一样,更是与后世不同。就如程颢寄来棋谱上的‘九九象戏’,也是与韩冈来自千年后的记忆有很大的区别。

楚河汉界算是有了,车、马、卒、将也都有了。不过尚没有士,反而代之以偏、裨二将,另外砲和象都给去除了,过了河的卒子还是斜行的。最关键的区别,棋子竟然走在格子中,跟国际象棋一样。这一点也跟纵横皆为八路的‘八八象戏’相同——唐代的‘八八象戏’,不但在格子中走棋类似于国际象棋,甚至连棋子都是立体的,车、马、将、卒都将形象雕刻了出来——反倒是民间的大象戏、小象戏是如围棋一般,在线上走着。

韩冈因为不习惯这里的规则,下棋老是输。输得急了,便将象棋规则重新按照自己的习惯改了一改,今天便拿了出来。反正如今世间的规则全都是乱的,自己定了在家里玩,谁也管不着——韩冈也没有对外推广的想法。

不过他的小心思瞒不过枕边人。

“官人……”王旖促狭的问着韩冈,大眼睛忽闪忽闪的,“今天忽然做了一副新象戏,是不是因为前天输太多彩头了?”

王旖这么一问,旁边的周南立刻用手绢捂住了嘴,而素心和云娘也背过脸去笑了起来。

“胡说,输就是输,赢就是赢。我怎么会不服气?……为夫偶尔下一次,又比不得你们天天在家练习,当然赢不了。”

韩冈强调着。不过他悻悻然的口吻,却惹得周南她们笑得更厉害。

王旖忍住笑:“官人棋品就跟爹爹一样呢。”

“说什么呢?”韩冈绝口不认他的棋品会跟王安石一个等级,“为夫下棋何曾浑赖过?!去年最后一次跟岳父下棋,他快输了的时候,可是直接把棋局给搅了。还说什么‘莫将戏事扰真情,且可随缘道我赢。’为夫可是眼看着就要赢了!”

“好!好!”王旖举着一只手,虚虚拍了拍,像是哄小孩一样哄着韩冈,“那官人就教教我们怎么下这韩氏象戏了。”

韩冈瞪了王旖一眼,撑不住自己也笑了起来。

论起棋艺,周南是个名手。围棋方面在教坊司难逢敌手,有说法是不输翰林院中那几位棋待诏,而象棋方面也是一流水准。王旖家学渊源,韩冈的岳母吴氏便是棋道高手,但碰上周南,却难有胜绩。

不过王旖除了输给周南以外,在家中却是坐二望一。在周南和王旖的熏陶下,严素心和韩云娘在围棋、象棋上的技艺大涨。韩冈闲暇时也跟妻妾下过几次,事先说好不许留手,然后就是连败。不论围棋、象棋都是没怎么赢过。

新规则一来,王旖便连输两盘。换了素心替位,韩冈更是轻而易举的开盘二十几步就胜了。回头看看云娘,韩云娘摇摇头,她可下不赢。韩冈再得意看了一眼家里的大国手,周南则抿嘴一笑,盈盈而起,接替了素心。

“很有信心嘛……今次可是要在棋盘上杀个落花流水。”

韩冈说得自信,只是开局的十几步一过,他的形势便急转直下。居着守势再走了三十多步,一支马天外飞来,竟然再有一步就会被将死。韩冈苦思冥想,但始终想不出渡过难关的一着。抬眼看看周南,一双玉手正轻轻的敲着棋子,天香国色的玉容上满是成竹在胸的悠然。

正是窘迫的时候,门外突然来了救兵,说是有人求见。韩冈如释重负,长身而起:“待为夫去去就来。”

随着他的离开,房中便是一阵清脆的笑声传了出来。

片刻之后,韩冈笑着回来了。不再是只有家人们才能看到的不带任何心机的笑容,而满是官场中的深沉。

“官人?”王旖声音轻轻。

“一份重礼,”韩冈意味深长的笑着,“就快要准备好了。”

