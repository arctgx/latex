\section{第38章 何与君王分重轻(19)}

物竞天择,适者生存。

殿上的每一个人都在咀嚼着这简单的八个字。

宋用臣觉得这两句意思不是很难理解,可看似浅近的文字里不知为何却让他有一种十分沉重的感觉。

不止他一个,其他人也都沉默着,思考着。

或许大部分都是在寻找其中的错处,但他们的确都还是在认真思索着。

“天择……何为天?”王安石第一个打破了沉默,质问韩冈。

“自然也。”

韩冈当然不会去指望自己点出了八字真言,王安石、程颢能纳头便拜。挑刺是正常的,但在后世流传百年,被称为信达雅典范的译文,怎么可能在仓促间便找出毛病来。如果有问题,也只会是韩冈自己身上——准备不足,学问不精。但韩冈为了今天,已经准备了很久了,而且他只要列举事实,就能证明自己的正确。

“何为竞?”

“逐也。”韩冈停了一下,换了一个更合适的词:“挣命!”

宋用臣轻噫了一声,没有比这个词更贴切的了。挣扎姓命在旦夕之间,难怪让人感觉沉重。

本想跟着出来驳斥的蔡卞也一时失语,没想到韩冈会有这个回答。

“自然之中,天生万物,若不能适应环境,那就只有被淘汰,死路一条。”韩冈抬手指外,朗声道,“庭外之槐,每年结槐子以万计,能发芽成树的也许一颗都没有。山上的大虫,健康时纵然能雄踞山岭,可一旦老病,追逐不上猎物,就只有饿死。山中麋鹿,往往为狼群追逐,逃得慢的就会落入狼口,跑得快的方能活下去。自然万物,芸芸众生,无不在挣命。有多少能像华夏之人,老有养,病有医,安居可至寿终?”

能活到寿终正寝的人就是在大宋也不多,这个时代人均寿命能有四十就不错了,皇帝正瘫着呢,就算能活过今年,但明年呢?只是韩冈的话,又难以辩驳。

以孝治国的大宋,若有不赡养父母的逆子,亲民官都要付教化不力的责任,数量的确极少。有病可以求医,本也是大宋值得骄傲的地方,辽国都要派使团来请求医疗援助。这的确不是禽兽可比。

自从董仲舒创天人感应说,使武皇帝独钟如数,儒门正道也渐渐式微。白虎观会议,更是将谶纬与儒学挂钩,使得儒门的道路完全走偏了。但今天的会议,却让人有正本清源的感觉。

但蔡卞如何甘心?他胸中憋了一口郁气,却无法发泄出来。韩冈将话题引入了他所擅长的领域,现在就很难再找到下手的空间。

趁着对手犹豫和组织话语,韩冈话锋一转,转到了夷狄身上,“自然之道,禽兽虫豸无不遵从。而夷狄之所以类禽兽,就是他们遵行的是自然之道,而非人之道。”

父子聚麀是特例,是表征。物竞天择、适者生存则是普遍的真理,方是本质。

“为何多少蛮部之中,父死子继?占据更多的财富、资源,源自于禽兽的本能。只有占据了更多,才能活得更好。‘一箪食,一瓢饮,在陋巷,回也不改其乐’,安贫乐道者,市井中多有其人。但蛮夷之中,出不了颜回。”

“何谓岁币?乃是饲虎之肉,盼他吃饱了就不在噬人。可辽国虎狼之心,得赐岁币之后,几曾安分过?”

“蛮夷畏威而不怀德,何也?中国之威,决其生死,不得不惧,不得不从。天壤之中,万物各自挣命,弱肉强食。强制弱,弱从强。也许狼群少有人能见,不过苑囿中圈养的猴群见过的人当不在少数。最有力的雄猴可为一群之主,其余弱者无不听命。”

“而中国之德,则会被认为是软弱可欺。自然之中,只有弱者才会让出私物,狗埋骨头是为了什么,怕被抢!蛮夷心如禽兽,视人亦如禽兽,面对弱者,他们唯一会做的,就是欺上门来,不会有半点同情。就算中国再以德化待之,也只会被认为是畏惧,馈赐再为丰厚,也会被认为是理所当然。”

韩冈滔滔不绝的一番话,将进化论的一个变种送给了四夷。反正他们用得上。而无论如何,韩冈都不会将之用在自家身上,赤裸裸的宣扬弱肉强食,在儒家为根本的社会里,会没有任何立足的余地。但当今民间,弱肉强食却不胜枚举,任凭哪一个都能随随便便数出几十条见闻来。豪右世家兼并田宅,这不是弱肉强食还是什么?前几年京城粮商趁灾荒囤粮不售,打算牟取暴利,这不是弱肉强食是什么?

韩冈说得太透了,尽管他说得是自然之道,说得是蛮夷为禽兽,但联系到现实中。心有戚戚焉的,宋用臣觉得不止他一个。也难怪王安石和程颢会保守起来。

只是在皇帝皇后面前,有谁能说,当今大宋同样是弱肉强食?有谁有这个单子韩冈的论述浅显易明,就是流传到民间,看过猫捉老鼠,螳螂捕蝉、黄雀吃虫的人,都能体会到其中的意义。

不过还是有人站出来试图阻拦韩冈:“‘天地不仁,以万物为刍狗’。枢密说来说去,似乎就只是这一句。”

天地不仁,以万物为刍狗。乃是天无私亲,一视同仁的意思。但吕大临想说的,却是这一句的根源,出自于老子道德经中。士林之中,许多人都很熟悉这一句。只是韩冈,拿着,当然会惹人议论。

“内圣外王也只是一句。”韩冈反驳。

内圣外王之说,出自于《庄子》,并非儒门教条。汉时论儒,是礼乐刑政。那时的儒家,虽然有天人感应、有谶纬图说。其根本,还是落在了实际的事务上,‘礼乐刑政四达而不悖,则王道备矣’。就是到了现在,也只是因为儒门开始融合佛老,原本属于佛道两家的名词,才渐渐流行起来。

现在无论哪一家都有直接从佛道两家借鉴的东西。就是最为排佛的盱江李觏和欧阳修,在文章中也大量使用佛家的辞藻。邵雍糅合儒道,多有谈及内圣外王的道理,故而被程颢视为邵雍象数之学的核心理论——‘尧夫,内圣外王之学也。’

当年邵雍与程颢说‘道’。指着桌子,从[***]之内,推到[***]之外。韩冈是不懂邵雍是怎么推的,反正邵雍之后说要教程颢,程颢赞了两句后,就推说自己没时间学,‘某兄弟那得工夫要学?须是二十年工夫。’还不让学生刑恕去学:‘刑七二十年里头待做多少事,岂肯学这的’。

邵雍和二程远不如表面上那般和睦。或许当真和睦,但在道统之上,程颢、程颐不会给老朋友面子。

吕大临措置着话语,他要反驳韩冈,又不能显得自己是在为邵雍招魂。

一犹豫间,韩冈又抢先一步:“物竞天择,适者生存,乃是自然之道,非人之道。人之异于禽兽者,也正是道不同的缘故。鹦鹉能言,不离飞禽,猩猩能言,不离走兽。纵有一丝像人的地方,本质是不变的。”

韩冈无意去宣扬进化论或者说天演论,他还没疯。超越时代一步是天才,超越太多可就是疯子了。而且韩冈一直在说以实为证,想要证明进化论这个观点,现有的知识储备完全不够,没有足够的证据,怎么支撑起进化论?何况比起单纯的一个理论,树立实证主义的大旗,对韩冈计划中的未来更加重要。以现在的情况,韩冈只是要将进化论用在社会上,并将其归属于敌人。

“人禽之分,只在礼也。有礼者人,无礼者禽兽。华夏之所以异于夷狄者,就在于这个‘礼’——凡人之所以贵于禽兽者,以有礼也。礼,说小了,是事神致福,人情往来。但往大了说,是文法,是规范,是纲常。有了纲常、规范、文法,华夏生民就能各安其分,各司其职,上下有序,四民皆安。只不过这礼必须禀仁心而行,否则便是与率兽食人无异。人心不能安。”

整间大殿之中,只有韩冈一人的声音,在他分清了自然之道和人道的差别之后,几乎所有人都放弃了在这一问题上跟韩冈一较短长。不仅仅是说得太通透的缘故,也是韩冈的总结已经足以让人深加思考。

这个有什么意义?

蔡卞心情逐渐安定下来。王安石和程颢都不就韩冈的论点发话,可见他们并不是很重视。

明华夷之辨,除了让人听着解气,又有什么作用?

《春秋》三传,《左传》叙史,《谷梁》论义,《公羊》说复仇,仅仅兼及华夷。真正将尊王攘夷、华夷之辨拿出来当做《春秋》总纲的,还是从孙复《春秋尊王发微》开始。远比不上《三传》的好处。

只要大家都明白过来,韩冈肯定会受到围攻。以气学最近咄咄逼人的样子,心中反感的不再少数。

蔡卞正想着韩冈会不会千夫所指,韩冈正按照自己的节拍继续着,“人居乾坤之中,天生万物予人。禽兽亦在其列。”
