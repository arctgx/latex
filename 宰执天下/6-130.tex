\section{第12章 锋芒早现意已彰(11)}

“不是已经七个了吗?”

当种师中从种谔口中得到了愿意投效的辽臣人数,很惊讶为什么朝堂诸公还没有改弦更张。

“才七个!”

种谔拿起茶盏,杯中的茶水,泛着莹莹碧色,是炒青的茶叶。

炒青的茶叶如今在市面上越来越多,制作简单,饮用也简单,不过包装和运输比较麻烦。不像茶砖、茶饼那样可以一个摞一个,一般装在竹筒里,价格也不算低。来自秦岭中的炒青山茶,已经可比福建的龙团了。

可种谔全不在意那么多,摆了一阵的茶水正温热,扬起脖子咕嘟咕嘟的喝了几口,然后丢到了桌上。叮铃哐啷几声脆响,全是让人烦躁的声音。

见到种谔这幅模样,种师中就知道他这位五叔,来到京城没多久,又已经对这每天按部就班的太平日子感到不耐烦了。

“是不是那几人官位太卑?”种师中猜测着,但一对上种谔投来的眼神,立刻改口,“也是,能送到太后御前,不会低了。”

种谔盯了侄子一眼,沉着脸坐了下来。

对耶律乙辛篡位心怀不满,愿意成为内应,现在已经将消息送过来的是七人。

摆在太后案头上的密信已经有十余封——有几个性子急切的接连送了几封投诚信——分别来自河东与河北北面的西京道和南京道。

发信人在信中皆希望朝廷能尽快派兵北上,他们将会在官军抵达的时候,出来为大宋效力。

这七人的地位放在辽国国中,并不算很高,可也都是各自所在地区有头有脸的人物,影响一城一池,动摇当地人心,并非难事。

这几人的承诺的价值,并不比跟随捺钵巡游四方的契丹重臣的投效稍逊。七人分散在辽国各处边州,只要他们的投效有一处能够成功,就等于是拉开了序幕,无数辽人将会争先恐后的蜂拥来投。

河东、河北当面的敌人如此,灵武地区北面的阻卜人更不会为辽人守节。那群鞑子虽还没有进化到会写字的地步,但他们已经聪明到可以分辨出一手拿刀一手拿钱的大宋,与两手都拿着刀的辽国,哪个是更值得跟随的主人。

在种谔看来,没有比现在更好的时机了。要是等到耶律乙辛坐稳了御座,宋辽两国南北并峙的局面又将会继续下去。

“有这么多内应,朝廷里面至少也得改个口才是。”

“才七人啊。”种谔长叹息,“若有个十几二十人,再有几名北虏重臣,就不用什么议论了。”

若辽国有一二宰辅级或只是地位稍逊的重臣明确表态,就是反对最力的韩冈,也很难再坚持自己的意见。

至于准备不足的问题,只要有那份心,对于如今的大宋来说,根本不是问题。

种师中自不会质疑种谔的判断,如果他的兄长种建中在这里,多半会多问几句,但种师中可不会。在过去的十年中,他没能挣到多少军功。

“要不是王平章抱着私心,也不至于变成如此局面。”

有着一位做太尉的叔叔,种师中对朝堂局势比一般的朝臣都了解。

王安石同样支持出兵,可这位平章军国重事的本心,却不是要灭辽,至少大部分不是。

平章军国重事在两府不配合的情况下,只有建议之权,无法掌控朝政。而东府三人中,韩绛、韩冈、张璪各自分管一块,相处融洽,哪个也不会愿意王安石的手伸进来。

显而易见的,王安石在朝堂平静了一段时间后,准备利用这一次辽国的危机,将吕惠卿从河北拉回来。

“王平章怎么想不重要,重要的是吕宣徽怎么想。”

以吕惠卿的脾气,他是愿意靠闹事被招进京师,还是愿意靠军功让朝廷无话可说?

种谔确信吕惠卿绝不会跟王安石一个想法,过些天,河北边境上的局势肯定会有一个变化。

种师中点着头,“要是朝廷肯出兵,肯定要以五叔为帅。”

种谔摇摇头:“也要郭二不去争才行。”

“郭枢密不是反对出兵吗?”

种谔哼了一声,不满溢于言表,“他什么时候将话给说死了?”

种师中愣了一下,“是哦!”

“那个老狐狸,比郭遵难缠多了。”种谔撇着嘴。

他与郭逵的兄长郭遵也打过交道,郭遵当年是名震军中的猛将,一对铁锏,一支铁枪,在三川口之战中,几进几出,横扫西贼,直到最后被绊马索弄下马,方才战死于阵上。比起勇冠三军的郭遵来,郭逵就更偏向谋略和用兵,武艺不算太出众,可断事用人,军中无人可及。

郭逵在朝堂上一向很低调,表面上他跟着其他宰辅一起反对出兵,但他的反对,只是放在枝节上。只要风色有一点变化,他肯定很乐意站到支持出战的一边。

也许有的将领如楚将昭阳,在功成名就之后,便不敢再画蛇添足,做到了武将所能走到的最高位之后,便不愿再冒险。但郭逵绝对不是这样的人。

种谔在郭逵手下受过不少气,视其为毕生要超越的目标,故而对郭逵,也是十分了解。

郭逵还没老。攻辽,灭辽,名垂千古的良机,这种诱惑,种谔经不住,他相信,郭逵也一样经受不住。

“要是哥哥能早点回来,就能知道辽国的虚实了……”种师中话到一半,忽然神情一凛,紧张地问道,“那逆贼不会将使团给扣下吧?”

种谔哂道:“耶律乙辛留着十九他们做什么?平白还要多出一份口粮。”

如果出使辽国的使团能够带回辽国的内情,是否出兵,将会一言而定。三位使臣中,王存是位文人,向英是个外戚,真正知兵的,只有自家的侄儿。有些话从他嘴里说出来,将是无可驳斥的铁证。

只不过……

种谔不自觉拿起茶杯,却发现方才已经给他喝光了,将茶杯放在桌上,这位闻名天下的宿将紧皱着眉,

若是辽国内乱,种建中可就难回来了,如果他能够太太平平的回来,就证明辽国风平浪静,同时耶律乙辛有足够的信心控制辽国。

这么看来,还是不回来的好?

……………………

“七人!?”

“等到耶律乙辛篡位之后还会更多。”韩绛悠悠说着。

韩维已经有很长一段时间没有回京了,可王韩之争还是很了解的。

将党争化为国是之争。远的有牛党李党,近的有新旧两党。王安石和韩冈翁婿两人如今的情况,实在有太多先例。

皱着眉想了一会儿,韩维问道,“王介甫是不是打算退了?”

他跟王安石是老朋友,王安石这一次的行事风格,与他记忆中的王介甫有着很大的差别。如果说仅仅是为了朝堂上多一个助手,也没必要这么做。

“或许吧。可能是心里累了。招了韩冈做女婿,恐怕比与吴冲卿做亲家还让王介甫后悔。”韩绛笑了起来,“招吕惠卿回来跟韩冈打擂台。”

韩维微微摇头:“吕惠卿可不是任人摆布的傀儡。”

“他能使得动河北的禁军?!”韩绛冷笑着。

作为首相,可不会干看着边臣逾越朝廷给予的权力。

“若是边境上辽军挑衅,难道官军能含辱忍垢不成?要是官军进行反击,朝廷能说什么?韩冈当年可这么做过!”

韩冈当年在河东做过的事,吕惠卿当然也能做。

边境上一来二去,冲突也会变成战争了。一旦引动了朝廷清议,当士林议论皆以雪耻为上,朝廷怎么去责难一名一心恢复故土的重臣。

“韩冈既然这么做过,他就知道该怎么应付。吕惠卿也没韩冈在河东的威望。”

又不是药王弟子,没有朝廷诏令,吕惠卿能使动几个边州的将校?韩绛可不信吕惠卿有这能耐。

“韩冈要能应付早就动手了。他太顾惜名声。”韩维说着,摇了摇头。

若是吕夷简这种心灵强大到对士林言论嗤之以鼻的宰相,还可以不顾士林言论。可韩冈自成为宰辅之后,做事便畏首畏尾,既想做一个控制朝政的权臣,又想做一个教化众生的圣人,这样下去,两边都不可能成功。

“逐二兔者不得食。韩冈最不该的就是太贪心!”

韩维自问换做他在韩冈的位置上,又有太后全心全意的信任,早就寻事针对新党了,当年王安石怎么做的,照着学就是了。以新党在朝堂中的人数,破绽满身都是。可惜韩冈不愿弄脏手,将早该解决的事情拖到了现在。

韩绛嘴皮子动了动,可他什么话都没说。

他与韩维并非同母兄弟,虽是血脉至亲,却也有必须小心翼翼的时候。

而且他的看法也有几分与韩维相同,原本韩绛以为韩冈是个有决断的人,做事十分明智,绝不会婆婆妈妈,没想到在政争上,总是手软。而在这一次的事上,他表现得更为首鼠两端。

太过注重名声,就是诸事不顺。否则以韩冈当年在横山的表现,何至于跟新党纠缠这么久?
