\section{第八章 战鼓尤酣忽已终(下)}

【第二更】

石弹的落点集中在城门附近,两刻钟的时间,已经有近两百枚石弹击中城墙。墙体上弹坑密布,仿佛一张翻过来的石榴皮。

一块块破碎的土石,随着石弹一起掉落,城墙上的裂痕越来越深。又一枚石弹呼啸着,划过一条完美的抛物线,重重的撞了上来。城墙立刻颤抖了一下,似是无法忍受重创所带来的痛苦。

轰鸣声中,数千上万斤重的黄土墙体缓缓垮塌了下来。崩碎的土石上,一阵灰黄色的烟尘腾起至十丈高处。大约三丈长的墙体垮塌了外侧的一半。从墙顶直至中腰,微微泛红的城墙内芯暴露在外。

城外的宋人军阵中立刻响起了一片震耳欲聋的欢呼声,欢声如雷,人群如海。高遵裕轻抚长剑,拈须而笑。等城墙的墙体再毁损一段,彻底打掉了西贼坚守城池的决心,就可以正式攻城了。

他没时间和粮草在城下磨蹭,高遵裕他要一举破城!

“看来用不到地道了。”一人在姚麟耳边兴奋的吼道。

“不,还是需要的。”姚麟抬眼看了一下不停的抛射出石弹的霹雳砲。

在这些天来的射击中,军中的神臂弓已经大量损毁,而粗制滥造的霹雳砲则毁损得更快。无奈之下拼凑起来的攻城器具,不可能如标准件一般支撑太久,眼下就已经毁了近四成。

“只有加上地道才是最稳妥地,如果今天攻城不果,再想来攻可就没有现在的士气了。必须要一举破城。”

环庆军的前进营地中,略偏西北的地方,有一顶与高遵裕、苗授聚将军议时一个等级的大帐。不过里面传出来的悉悉索索的声音,以及从中进进出出的灰头土脸的士兵,无论如何都不会让人误会这是将领们共议军事、运筹战策的帷幄。

新鲜泥土的味道从离城一里的前进营地的帐篷中飘散出来。地道的出口就在帐篷中,每天挖出来的泥土,到了夜间从里面运出,然后堆到营地角落里。

灵州城壕三丈宽,深浅不知,但从灵州周围渠道的平均深度来推算,不会浅过五尺。要越过濠河,不受渗水影响,地道至少要挖到两丈深才行。

斩马刀如今都能用上夹钢。锨、镐等应用在营垒城防上的工具,虽然舍不得用钢,但铁是管够的,不像过去,竟然还有木头的。

有了更为优良的工具,地道又深入地下,而且就算在夜中,高遵裕也是一刻不停地用鼓声和奔马来遮掩地底的声音,在短时间内将地道开挖成功便也不足为奇,料西贼也想不到官军有这个本事。

对于地道的开凿,高遵裕十分放在心上。特地选派亲信督促,两天前地道就已经穿过了城壕,如今更是挖穿了城墙,只差一步就能将地道贯通。等到在城墙外侧再开个入口,杀到城下的官军就能直接钻过城墙。

到时候,城墙上有云梯送上去的精锐,城墙下也有善战的敢死之士,灵州城如何不破?

……………………

十数里外的厮杀声依稀可辨,苗授负责的是外围防御,随着远方的欢呼声一阵接着一阵,他的眉头越皱越紧,“太顺利了!”

苗授身边的将校都是一脸羡慕嫉妒的望着战场的方向,听到苗授的话也就几个亲兵。

“总管有什么吩咐?”一名亲兵凑上来问道。

“我是说实在太顺利了。”苗授心中一团疑云,只想将心中的疑惑说出来,“灵州一失,兴庆府就守不住了,西贼怎么会不拼命来救?城中也该有兵出来反击才是,哪有这么抱着头让人放手痛打的道理。”

“有总管坐镇,西贼应当是怕了总管的赫赫声威。”

亲兵的马屁,苗授没有理会,充耳不闻。

狗急跳墙、兔子急了还能蹬鹰,生死存亡之际,党项人怎么可能会没有拼命的的勇气?如仁多零丁、梁乙埋这样的文武宰臣这时候好歹出来一个,让嵬名阿吴在灵州城中顶着,根本不合常理。

危机感越来越浓,一阵阵的心悸让苗授坐立不安。他领军堵在通往兴庆府的道路上,以防西贼偷袭;附近的几条主要的河渠全都派了重兵去防着有人决堤。

西贼反击的途径只有那么几条,不论有什么花招都别想瞒过他去,可为什么到现在都没有反应?

不对劲,实在是很不对劲。多年来上阵所积累下来的直觉不断警告着苗授。

可苗授还是想不出究竟是哪个环节会出问题。

一名骑兵从远方狂奔而来,到了苗授近前被亲兵拦了一下,随即又被放行。他在苗授身前跪倒,匆匆说道:“总管,七级渠的河水涨起来了,比起昨日涨了五尺有余。小将军命小人急速来报,请总管早做安排。”

“五尺?!”苗授差点从马背上摔下来,“你们都是瞎子吗!?”他怒吼,“不是五寸,是五尺!眼睛都瞎了!?”

那个小校脸色发白,竭力镇静下来为自己辩解着:“一开始都没注意,早前河水涨得也不快,只以为是上游下雨才会涨了水。谁知道方才一个时辰就一下涨了两尺多。”他抬起头,惶惶然的说道:“总管,还请速做决断,再过一阵,可能就要漫过堤坝了!”

七级渠的下游是兴庆府方向,西贼在那里堵着河水,他们的主力必然也在那里,也许在二十里外,也许在三十里外,反正肯定是斥候游骑过不去的地方。

苗授横目扫试过他麾下的士卒,骑兵给高遵裕调了去,剩下的基本上都是步兵。跑过去差不多要半天,对手还是以逸待劳,根本没办法打。而且这段时间中,河水必然漫过堤坝,冲向灵州城。

苗授暗叹一声,招过一名亲兵:“将此事通知高总管,我们必须要撤军了。”

……………………

已经不是漫过堤坝的问题了。

七级渠的堤坝眼下破开了一段六丈多长的缺口。堤坝近百里长,六丈只是微不足道的数字,但缺了六丈,却让百里长堤完全失去了作用。

从另一段堤防赶过来,看着眼前根本无法填补的缺口,苗履手脚冰冷,脑中一阵晕眩。

西贼的准备的确做得太过充分了。这一段河堤肯定早已给掘松,只是外表上看不出来而已。可只要水位涨上来,却会一冲就垮。

浑浊的黄色河水从缺口处奔涌而出,激流上泛着白沫,直奔向灵州城的方向。浪奔,浪流,水花甚至飞溅到了苗履的脸上。

冰凉的触感让苗履回过神来,眼下不是发呆的时候,他立刻抓过一名亲兵,“快放狼烟,灵州城没法儿攻了,我们得立刻退军。”

……………………

“七级渠决堤了?是否确凿无疑?”

终于等到期盼已久的消息,仁多零丁霍然而起,进一步确认着消息的真伪。

“回老太尉的话,小人亲眼看到堤坝上开了个口子。水冲得堤内的石头都立不住脚,在水里滚着,宋军的人马只能站在堤坝上干看,一点办法都没有。”

报信是自家的亲信,仁多零丁没了怀疑。他先是放松的长叹了一声,回头对叶孛麻笑道,“幸好七级渠及时破了堤,不用我们辛苦去挖土了。”

叶孛麻点了点头,双眉间的皱褶松弛了下来,眼中满是轻松的笑意,“想必宋人没想到七级渠会破堤。”

“既然定下了放水的策略,自然是早就做过了准备,难道还要临时破堤不成,那也未免太小瞧人了。”

“对于兴灵地理,宋人了解得太少了,只想防着我们破堤放水,不想想直接将水渠从下游堵起来有多方便?”

“还有十几条渠道,虽说水量比不上七级渠,合起来也差不了多少了。兴灵沟渠千八百,宋人怎么能守得过来?”

“该去灵州了。”

“嗯,是该去灵州了!”

心中的得意不得不靠言语诉说出来,仁多零丁和叶孛麻一阵大笑,而后齐齐上马,统领麾下众军向灵州城的方向奔驰而去。

……………………

咚咚的一声声巨响,云梯重重的撞上城墙,竖在顶端的防箭挡板倒下,挡板后手持刀盾的宋军战士立刻跳上了灵州城头,举盾挡住迎面而来的枪刺,而后一刀劈开了试图阻拦的守军。紧紧跟随着他们,一群选锋精锐沿着云梯也冲了上去,血雨腥风的惨烈搏杀才城头上展开。

官军终于冲上了城头,又是一阵欢呼在城下响起,想到即将到来的盛宴,城下的宋军将士更是

“让地道下面做准备。”高遵裕握紧了手中的剑柄,胜利就在眼前,让他连呼吸都变得粗重了许多。

“太尉,水!水!”一人这时突然疯狂的扯着高遵裕的衣袖。

高遵裕怒瞪了他一眼,然后向他手指的方向望过去,立刻就瞪大了双眼。

防守在西面的骑兵已经变得混乱,正向着中军这边退过来,再定睛一看,追逐在他们身后的,一道暗色的痕迹,那是河水正在淹没大地。

破堤的洪水远比战马的脚步更要迅捷,只用了小半个时辰,便已经涌到了灵州城下。

流到灵州城下的水势已经变得不再湍急,并不是如缺口处的山崩地裂,而是渐渐的漫了上来,一点点升高水位,从淹过鞋底,到没过脚踝,然后再往膝盖处涨上去。

压制城头守军的射击戛然而止,而城头上一片呼喊,士气大振的守军绝地反击,不但将攻至城头的选锋逼下了城墙,还顺便用油罐将云梯车一辆一辆的给点燃。而地道……已经被水所淹没,里面的精锐大半未能逃生。

已经不可能再攻城了。

“只差一步啊!”高遵裕撕心裂肺。

一只秃鹫在高空盘旋着,半个多月来的经历告诉它,今天依然会有一顿丰盛的晚餐。锐利的鹰眼扫过大地,追寻着一个个依然鲜活的食物。

大地之上,是已显混乱的数万战士,失败突如其来,这同样让他们接受不了,难以相信眼前的现实。但冰凉的河水在提醒他们,这并不是做梦。

浑浊的河水,让宋军官军惊慌失措。水会涨到哪一步?边上就是黄河,是不是黄河破了堤?听多了黄河水患的传闻,人人心中惊惧。

人心一片混乱,心中皆是明白,这一战已经再难挽回。

