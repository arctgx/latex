\section{第11章 城下马鸣谁与守(20)}

高永能平躺在床上,蜡黄的脸上不见一丝血色,嘴唇都是惨白的。呼吸声细不可闻,胸膛不见起伏,仿佛一个死人。

随军疗养院中捆扎伤口专用的细麻布条在头上绕了一圈又一圈,黑糊糊的药膏就在抹在麻布下的伤口上,但血水还是从包扎处不断的渗出来。只有从这里,才能看得出高永能他还有一口气吊着。

营中的医官对这样的伤势束手无策,和几个护工站在一边,不知该说什么,也不知该做什么。高家的几个在军中的子侄都跪在榻前,一个个哭红了眼。

“君举……高君举!”

曲珍俯下身子,在高永能耳畔连着唤了几声,见他始终没有反应,无奈的摇了摇头。

虽然常言说瓦罐不离井边破,将军难免阵上亡,但兔死狐悲、物伤其类,看到高永能现在的惨状,曲珍连安慰人的话都没心情说了。直起了身,吩咐了医官好生照看,就大步的离开临时安置伤员的这间小庙。随军疗养院中的气氛让人感到十分的压抑,曲珍一刻都不愿意在其中多加停留。

高永能是一个时辰前,在城头上被一枚十几斤重的石弹击中了头盔,一句话也没有的就这么倒下去了。再坚固的头盔,也经受不起霹雳砲抛射出的石弹,就算是从敌楼的墙壁上反弹过来的也是一个结果。那是用来摧毁城墙的武器,血肉之躯挨了一下,砸中的还是头颅,没有当场阵亡,已经让人很是惊讶了。但高永能的脑袋还是跟着头盔一起陷了个坑下去,按照医官们的说法,这叫做颅骨骨折,无药可医,包扎一下,仅仅是尽人事而已,能不能活下来,得看老天爷的心情。

城外霹雳砲的目的不是伤人,造成的伤亡虽多,也只是附带。党项人平均每天都能新造出三架霹雳砲,以替换旧有霹雳砲损坏后的缺口。用霹雳砲来摧毁城墙,只要盯着一个点来轰击,刚刚修筑成功没有多久的墙体,根本支持不了多久。

而事实也正是如此。在经受了数日积累的伤害之后,盐州城的墙体,尤其是西壁的城墙,有很多地段的外侧都坍塌了下去。原本能供四马并行的城墙,只剩下一半的宽度。有几处更为严重的地方,都出现了从内到外的裂痕。

走出随军疗养院,石弹撞击城墙的轰鸣声重又在耳畔响起。都快入夜了,红霞已经映着半幅天空,可党项人的攻势还是没有停息,轰轰的震动,让人不由得忧心起那道已经千疮百孔的垒土墙。

曲珍停下脚步,怔怔地望了一阵声音传来的方向,猛不丁的出声唤道:“十四。”

“太尉有何吩咐。”

紧随在曲珍身后的一名十四五岁的少年闻声便上前一步,他有着一对跟曲珍相似的招风耳,这也是大部分陇干曲家族人的特征。

曲珍侧头看了一眼。族内排行十四的曲涣这个孩子,最让曲珍欣赏的就是他从来不拿自己的身份炫耀。在营中都是跟其他小校一般,叫着自家的官称,而不是喊着叔祖。

“你去找你三叔,让他准备好几条长一点的绳子。”曲珍吩咐着。

曲涣有点发楞,他年纪虽小,却聪明得很,否则曲珍也不会将他带在身边做侍从。他没想到曲珍竟转着离城而逃的想法。

“食君之禄,忠君之事。但跟着那个蠢货一同下黄泉,死都不能瞑目。”在侄孙单纯的目光注视下,曲珍没有半点羞愧之意,为了守住这座盐州城,他尽了心尽了力,守不住城池不是他的责任。

“城破之前,我会坚守到底。但城破之后,那就是各安天命了。”就算是在侄孙面前,曲珍都是问心无愧。

盐州城已经山穷水尽。

战前最担心的粮草问题,只因为有越来越多的人不用再吃饭,消耗的数量远少于预期,到现在还有不少剩下的。

从鄜延军中精挑细选出来的精锐,全都消耗在了城头上。这是应该用在关键时候的尖刀,如今却是在一点点崩坏了刃口。

高遵裕败了,就在昨日,城外还有人挑着首级、旗帜和头盔之类的战利品在城墙下炫耀,试图动摇城中军心。

灵州之战后,已经被打断骨头的环庆军还没有经过彻底的休整,便又被强迫上阵。精气神全都完蛋的队伍,还有胆子跟西贼交上手,高遵裕的胆量让曲珍吃惊非小。

种谔还不知道在哪里,信使倒是派来了两次,都是要他们再支撑几日,援军不日即到。

可鬼才会相信他的话。

“恐怕种谔现在的打算就是想等我们死后再过来捡便宜。”曲珍边说边笑,曲涣看得心中直发毛。

收敛起笑容,曲珍又回头冷淡的看了侄孙一眼:“还耽搁什么?”

曲涣收摄心神,不再犹疑:“末将明白了。”

曲涣小跑着走远了。曲珍转身望着城墙又冷哼了一声。党项人布置在城外的包围圈,跟一面渔网差不多,捉的是能被网眼拦住的大鱼。大股的人马是跑不出去的,但人数少点,想走却并不难。

正要往西城的敌楼去指挥作战,却听到轰然一声巨响,前方尘头大起,紧接着就是一片声的在喊:城破了,城破了!

曲珍脸色一变,“怎么这么快!?”

徐禧已经没有了一个月前的意气风发。纷乱的须发很久没有打理,灰烟满面的一张脸,完全看不出重臣的气派,这是与士兵们同饮食同起居的结果,却也没有换来多少士兵们的信服——不能带来胜利的主帅,纵然爱兵如子,却永远也不可能得到军心。

就在他面前,一枚石弹砸在了已经垮塌了一半的墙体上。当所有人还以为不过是跟之前一样,半毁的墙体还能支持一阵,整整六丈的城墙便全数垮塌了下来。待腾起的烟尘落定,变露出了只剩半丈髙的残余。垮下来的黄土,则变为攻入城中的缓坡。巨大的缺口成了放在狼群面前的鲜肉,西贼蜂拥如潮水,瞬息间就淹没了试图堵住缺口的十几名士兵。

若是能立刻组织起守军中的精锐反击,或是设法调集几百名弩手用神臂弓封住缺口,还算有撑过去的希望。但城墙的垮塌,就如同弓弦的崩断,人心一下子就散了。当最后一根稻草压下来的时候,驻守在城内的官军就再也没有继续坚守城池的意志。

徐禧亲眼看见区区二十多名铁鹞子在缺口前下马,然后踏着浮土冲入城中。试图封死缺口的一队士卒,接战不过片刻,就被这群党项精兵斩尽杀绝。而那队党项人紧接着就转往城门口杀过去,没费吹灰之力就逐走了守军,趁势夺占了盐州的西门。

盐州城并不大,城墙边的混乱已经传到了城中的每一个角落。

从上到下,几乎每一个人都知道盐州城已经守不住了。

李舜举的手颤抖着。他用一柄匕首从衣袍的内衬上割下一块白绸。右手的食指在刀刃抹了一下,用着指尖在白绸上匆匆留下十几个字,权当作遗表交给护卫他来盐州的班直侍卫,“快带着遗表走吧,上京去,迟了就来不及了……”

班直不肯走:“都知。要逃一起逃!”

李舜举笑着,泪痕满面:“即受之王命,自当忠于王事。死便死尔,但恨不能为君分忧。”

“都知!”那班直眼圈也红了,抽着鼻子叫着。

“走吧,快走吧!”李舜举催促着,将班直推出了屋子,转回身,将门关上,“臣死不恨,惟愿官家勿轻此贼。”

班直侍卫亲眼看着门被关上。纵然心情苦涩,但他还是他跪下来磕了几个头,然后起身飞奔而出。

徐禧还站在城头上,身上早已是甲胄完全。站在一群护卫中间,举着刀向前与攻上城头的党项人拼杀着。护卫人越来越少,越来越多的人抛下武器,只有徐禧还精神十足,病态一般的奋力战斗。

没有像样的武艺,只知道挥刀乱砍,但在亲兵们的护卫下,徐禧成了这一段的城墙上最后一名还站着的宋人。

毫无怯色的向着围过来的党项战士挥砍过去,但肚子突然一凉,迈出去的脚步突然就没了力气。徐禧疑惑的低下头,一根锋利的长枪不知何时突破了腹部的板甲,深深的刺进了小腹之中。

将长枪捅上去的党项兵放开手,同样在疑惑着:“看他身上的穿戴,怎么这般不济事?……他是大将吧?”

徐禧不懂党项语,他只感到全身的力气随着腹部的伤口向外流失。

不该是这样啊!

徐禧捂着肚子上的创口,只觉得这完全不合道理。

他还要领军攻克兴灵,他还要收复燕云。他还要晋身两府,他还想被人称为相公。满腔的雄心怎么能就在这里化为泡影?!

紧紧攥着枪杆,徐禧咬牙瞠目的模样,竟把几名党项士兵吓得连连后退。

但他的脑后突然一痛,一片晕眩的黑暗中,就听见一个百般不屑的声音:“装神弄鬼!”

‘不该是这个结果!’

直到最后,也不甘相信这个结局。抱着深深的疑惑,徐禧的气息渐渐消失不见。

夜幕降临,盐州城终于完全被攻克。四座城门一个接一个的被打开。火光映红了天空,听到城中的喊杀声,城外的党项人全都在向四座城门冲去。

曲珍用根绳子从城墙上槌了下来,回首看了眼城头,便毫不犹豫转回身,带着寥寥数人,悄然向南,消失于黑暗之中。

