\section{第11章 飞雷喧野传声教(十)}

垂拱殿的常朝,太后根本就没到,由韩绛押班,率领不厘务的朝臣进拜。

而在垂拱殿的常起居上,太后就带着小皇帝赵煦端严正坐,看着下面的群臣参拜。

常起居是内朝,都是有实务在身的文武大臣,议论的也是朝廷内外的事务。

军国重事,不能谋于众人。真正的国家大垩事,还是在崇政殿中讨论决定。

但今天不论是在垂拱殿上,还是在崇政殿上,都没有什么大垩事拿出来议论。

冬至后,年节前,也没什么事可以奏报。

之前东府三位宰执所议论的六路发运司的事,在政事堂内部就能解决了,用不着惊动到太后。而昨天韩冈奏报于太后的大小事务,也不需要在朝堂上再说一遍。就算是重要如天子赵煦的教育问题,也可以等到明年再说不迟。看了一眼比正常七岁儿童要小上一圈的皇帝,韩冈觉得最好还是不要那么早提经筵的事。

倒是吕嘉问上来说了一通西北盐税。

盐是朝廷专利,各路吃哪里的盐,朝廷都有规定。河南及河北一部,还有关西吃的是解盐,巴蜀四路,也就是益、利、梓、夔,是井盐的行销区。河东是土盐,剩下的地方,则全都是海盐——其中也分了广盐、福盐、淮浙盐、东北盐等分区。

不属于当地行销区的盐,决不允许在当地发卖。而且有的地方为了将官盐卖出去,各家各户在缴税时,甚至得将自家在盐上的份额给买回去,也就是强行抑配。

陇右路和永兴军路上盐池众多,解州盐池之外,还有银夏的青白盐池,旧熙河路上的众多盐池,在过去,私盐十分泛滥,价格高昂的官盐很难卖出去。现在官府采取了薄利多销的手段,也开始讲究质量,官盐价格比私盐还贱,朝廷的收入却没有减少。这是公私两便的好事,只是关西两路的盐价低了,弄得河东、京西的私盐贩子都是从陕西买了官盐来卖。

吕嘉问便是为此上奏,要么就是陕西抬高盐价,要么就看着京西、河东两地盐税大减。

来自陕西的大臣对此极力反对,都不用韩冈出面,吕嘉问的意见就被顶了回去。然后朝堂上定下来的方略,就是在出关中的道路上加派人手进行检查,捉到盐枭一律重惩。只不过按照早上在宣德门外三名东府宰执商议的结果,对盐枭的重惩最后只会以流放来处置,而不是过去的砍掉脑袋。

韩冈对当今的盐法早有不满,正考虑着该如何改,直接抽了三司的老底,所以对吕嘉问归班时投来的眼神根本就不加以理会。

自竞选失败后,吕嘉问颇受到了几次弹劾,但他硬是坐在三司的位置上不肯挪窝,王安石虽不理事,也始终保着他,所以一众御史也奈何不了他。

韩冈也懒得理会他。三司是为了分割宰相手中的财权才设立的,但现在政事堂的堂库中,有免役法、市易法等新法收入,加上来自内库的借贷,三司卡不了政事堂的脖子。

年初的时候,国库穷得叮当响,顺带将内库都刮了一遍。可在今年的夏秋两税入库后,加上新法收入与铸币局的铸币税,朝廷财计也就宽裕了许多。

尤其是铸币,铁钱看似价廉,可架不住国中多铁,今年各路一共铸了五百万贯铁钱,光是京中,就有两百万贯,这还是害怕铁钱贬值特意收敛的结果,否则再翻一倍都可以。铁钱五百万贯,五文的青铜钱和十文的黄铜钱,从面值上来计算,也有五百万贯了。除去原材料和人工,纯利超过三成。

而且这样的买垩卖,不用担心会做不长久,除了铁钱得稍稍收敛一点,铜钱想铸多少都没问题。这一年来所发行的青铜钱、黄铜钱,少说有四分之一被埋进了地里,市面上只会嫌钱少,不会嫌钱多。

铸币的量大了,也稳定了,铸币税也就能够旱涝保收了。其收入归入内库,政事堂开一张借据,就拿了六十万贯现钱到了手中。这就是国债。如果有需要,还可以再给内库开单子,不过就是宰辅们签字画押嘛,动动笔就有钱,韩绛、张璪、韩冈,哪个会嫌写字累?

而太后那边,一边是新铸钱和新织的丝绢大批的送进内库中,一边则是给付百官、三军的赏赐,以及政事堂递过来的借据。再多的钱绢只能过过眼。不过政事堂拿了钱,至少还有借据,加上政事堂也不会将钱都借走,给太后留了不少。看到半满的库房,好歹心中不慌了——一年就半满,两三年后就要想着加修库房了。

既然政事堂与内库之间的交流更多一点,三司使在太后心中的重要性也就更低了几分,吕嘉问今日的质问,连一个泡都没冒就沉入了水底。

结束了崇政殿的议事,太后并没有留人说话。回到政事堂中,韩冈就与韩绛、张璪等人收到一份加急奏报。

这是一份来自雄州的密报。雄州知州探查到了辽国正旦使手中国书中的内容,以及另外负有使命,故而早一步派人将消息送到京师,好先行做好准备。

“明年的岁币将十万匹绸缎改成棉布……”张璪冷笑着道,“耶律乙辛是穿腻了绢绸的衣服了吗?”

“玉昆,你怎么看?”韩绛问韩冈。

“如果耶律乙辛愿意将二十万两白银都改为相应的银币,那倒是没问题。”

白银兑钱的比价,今年因为铸币的缘故,变得高了一点——为了供给日后的金币、银币,国库在慢慢囤积金银——基本上达到了一两兑三贯的水平上。不仅仅是白银,黄金兑换的价格也是一样上涨。韩冈估计,等到什么时候银币铸造成功,即将发行的消息传出去,银价和金价还会有一个跃升。甚至只要有一点苗头,就可以看到市面上的金银大量的减少,界身巷中各家金银交引铺挂出来的水牌上面的数字,打着滚儿的往上涨。

现下用银七铜三的银币代替岁币中的白银,肯定能省上一笔。

“铸币局已经能造银币了?”韩绛惊讶道。

他可记得,韩冈当年说铸币局事,除了已有的一文、五文、十文三色钱,还提到了要造大面额的钱币,百文的,一贯的,十贯的。但依照韩冈的说法,为了防伪,需要通过锻压来保证钱币无法仿造。所以一年过去了,依旧法铸造的钱币,填满了国库,而新法铸币却没有一点消息。

“要是锻轧造币的机器造出来了,铸币局就可以改名造币局了。”韩冈摇头,“还早得很。”

“那玉昆你怎么那么说?”

“只要耶律乙辛肯要,开炉铸钱给辽国也没什么,造个母钱也不费什么事。”

“这样啊。”韩绛摇摇头,“玉昆你真是让老夫空欢喜一场。”

“只是元佑重宝,怕是耶律乙辛不想要。”张璪说道。

“只要是真金白银,就是印上大康的年号,他都会要。”

听了韩冈的冷嘲,韩绛、张璪哈哈笑了起来,大康可是给耶律乙辛害死的那位宣宗皇帝最后留下的年号。

陪着笑了几声,韩冈收敛了笑意,说笑到此为止。他正色道,“辽人的要求甚为无理,还要挡回去吧?”

“当然。”

韩绛肃容点头,前面的话自是笑话,要是辽人想要什么,这边不论是一口答应,还是讨价还价,都是丢脸。清议一起,宰相的脸面往哪里搁?

张璪也冷声道:“耶律乙辛若是以为拿下两个小国,就能恐吓中国,那就未免太蠢了。”

已经定下的和约,要是能够这么容易改动,当初几番大战又是为了什么?

辽国想要将丝绢换棉布,的确不是什么大垩事,以岁币中的丝绢质量,也抵得过数量相当的棉布。

可哪个宰辅都不会答应下来,只因为那是辽国提出来的条件。

大宋这边怎么动脑筋钻空子都没问题,宰辅们也不在意,但辽人想要改,先打过一场再说。

将耶律乙辛的疯人疯语丢到脑后,韩绛旧话重提,“发运司的事怎么办?光靠严刑峻法非是治本之法。”

“因为薛向,发运司损失了很多有才干的官员,这不是短时间内能够弥补得上的。”

“难道邃明想要将那些人调回去?”

“自然不是。”

“那邃明兄是何想法?”

“发运司中官吏如此猖狂,只是因为汴水独一无二,无可替代。就算再有一薛向,能整治好发运事,但十年后呢,百年后呢?那时又会如何?”张璪看看韩绛、又看了看韩冈,“吾等备位辅弼,当为百世计。”

“原来如此。”韩绛点头,视线转向韩冈,“不过此事得问计玉昆才是。”

韩冈心中一震,双眼微微眯起,从韩绛、张璪的脸上看过去,都是目光灼灼的望着自己。

真是图穷匕见了。韩冈想着。两只老狐狸这一搭一唱的,就是打着轨道的主意。
