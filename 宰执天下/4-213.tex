\section{第33章 物外自闲人自忙(九)}

在经过了一番友好坦诚的交流之后,韩冈终于觉得不能再耽搁文相公宝贵的休息时间,而斜对面的文家六衙内看样子也是脸色不对,不知是不是内急。

作为一名好客人,当然要为主人家着想,在当前的话题告一段落之后,韩冈遂起身告辞。

大概是觉得天色太晚的缘故,文彦博并没有挽留韩冈。将韩冈送出了待客的小厅,在台阶上与韩冈拱手而别。

而文及甫则一直将韩冈送出了府衙大门,在府门外的众目睽睽之下,作揖行礼,目送韩冈一行远去。

等到转回身来,文及甫已是一幅如释重负的模样。看看头上的太阳,韩冈在文家竟然待了差不多一个半时辰,

“总算是走了。”走进门时,文及甫低声咕哝着。

这一个半时辰,对他来说就是放在架子上烤,背上留下来的汗水,就跟烤架上的全羊滋滋作响的油花,不停地冒出来。就是跟房里的侍妾闹上一夜,也没有这么累过。

而且自己还不是主力,只是偶尔才跟韩冈搭上两句,自家的父亲则是一直不停的跟韩冈说着言不由衷的话,一个半时辰啊,七十多岁的人!

想起自家的父亲,文及甫就立刻加快了脚步,这么大的年纪竟然熬了一个半时辰,就是寻常见了亲朋好友,也不会坐上这么长的时间,光是加了消风散的清茶,都喝了有四五杯。

回到方才见客的花厅,文彦博已经不在了。扯过一名正在厅中收拾的小厮,文及甫开口一问,就听得小厮回答说:“相公已经回了房休息去了。”

文及甫心中顿时有了几分不祥的预感,匆匆又往文彦博的日常起居的房中去。

一进门,就看见文彦博正在一张软榻上闭目养神。

文及甫的心一下就提了起来,担心的发问。“大人,可是累着了?”

“累什么?真当为父老了不成!?”文彦博双眼一睁,一对眸子湛然有神。他推开儿子,霍然起立,又将上来搀扶的侍婢的手甩开,大步在房中走着。

前任宰相高大的身材如牛一般壮实,就算年过七旬了,腰背也是挺直的,肩宽腰圆,并不输给刚刚离开的韩冈,声如洪钟:“为父这身子骨活到一百岁都可以,要亲眼看着那灌园小儿怎么败的!”

“大人……”文及甫提心吊胆,这个时候,实在不适合再跟韩冈斗下去了。

自家的老爹已经不是宰相或是枢密使了,必须要加个‘前’字。所谓人走茶凉,也许旧时的关系还在,平时也会讲个人情,但再想如过去领有东西二府时一般,一呼百应,走马狗云集门下的情形,已经是不可能再重现。

如果韩冈步步紧逼,就像当年李中师对待富家一样,那样当然会有人抱不平,但一旦反过来,自家主动出手跟韩冈这个炙手可热的新进顶上,又有几人会赴汤蹈火、在所不辞?

韩冈不论私心如何,如今在外人看来,都是做到了仁至义尽。如若父亲再硬着要与他为敌,帮忙的不会有,上门劝谏的朋友倒是会多起来。

文及甫很悲观,他想劝,也不知该怎么劝。

文彦博眼神则凌厉了起来:“你怕个什么,为父有的是耐心。”

文及甫放心下来,但他仍忍不住想苦笑,尽管如此想来有些不孝,自家已过古稀、将及耄耋之年的老父与二十多岁的韩冈比耐心,还是有些难度的。

……………………

韩冈回到家中,从人们各自散去了。与几名幕僚聊了两句,便返回后宅过来。

素心和周南正在读着《千字文》,长子、长女就在旁边跟着念。

尽管还未聘请蒙师,但自家的儿女开蒙,在学问上也用不着求诸于外,只是要从小多与同龄人交流,这才是适合成长的方式。韩冈在外面听着房中儿子女儿正高声重复着‘天地玄黄、宇宙洪荒’,心想是不是建个蒙学比较好。

王旖一直都在内间等着韩冈,与云娘一起绣着花。见到丈夫正好赶在预定的时间回来,王旖就带着一家人将韩冈迎进了房中,服侍着更衣奉茶,关切的问着,“官人,今天谈的怎么样?”

韩冈笑了笑:“还能怎么样?有了今天的这番话,潞国公那里算是就此揭过。只要不会干扰漕渠之役,这一次的事,我也不与他计较了。”

韩冈的口气很大,王旖的笑容就变得有些勉强起来,“官人……”

韩冈笑了,问道:“这次的事是为夫的错吗?”

王旖摇摇头:“是潞国公的错。”

韩冈一拍手,“说的正是啊,是文潞公的错。天下之事,道理最大。潞国公既然不占理,那还有什么可说的?总不能因为他乃是显宦元老,就能倚老卖老吧?何况为夫从头到尾都没跟他计较过,换作是他人,早就上表弹劾了,还上门帮他洗刷外界的误会,落井下石还来不及。”

周南在旁掩口而笑:“官人的不计较可比弹劾厉害多了。如果官人只是上表弹劾,潞国公还不至于坏了名声。”

“这就叫做不战而屈人之兵,用道理大势压人,可比赤膊上阵有用得多。”韩冈哈哈笑了一声,“不说这些扰人的俗事了,今天大哥儿、大姐儿的功课念得怎么样了?”

“又多学了四个字。《千字文》前四句的十六个字,大哥大姐现在都会写了。”素心拿着两张纸出来炫耀,“算数嘛,看今天答得二十道题,十以内的加减都不见再有错。”

“还是慢了一些。”周南看了看王旖,似有一份羡慕,“二哥儿还没开蒙呢,就已经认得百十个字了。”

“这样就好,快慢无所谓,日有所得那是最好的。”韩冈倒是挺满意,人的资质有高有低,这点是勉强不来的。

韩冈的五子一女,最小的三个才刚刚开始学说话,略过不提,老大韩钟有些好玩闹,耐不下性子,比他妹妹稍逊一点,而金娘因为是女孩子,所以能认真读书,论头脑其实都不差,至少都可算是中上。但王旖所生的次子韩钲,则可以说是聪慧了。五岁不到就已经认得百十个字,尽管比不过如白居易那些个生有夙慧的才子,也算是十分出色,韩冈自己都显得逊色许多。不过因为年纪还小,就没让他跟着哥哥姐姐一起开蒙。

夸了已经开蒙的长子长女两句。很不顾形象的将儿子女儿一起抱在腿上,“这千字文别的倒无所谓,唯独‘金生丽水、玉出昆冈’这两句就一定要记住。”

金娘疑惑的张着黑白分明的大眼睛看着韩冈,而老大韩钟则点头念了起来:“孩儿知道了。金生丽水,玉出昆冈。”

严素心脸沉了下来,啪的一声拍在桌上,吓得韩家大哥儿肩膀一缩:“怎么能乱说,要避讳的!”

王旖也不高兴了,指责着韩冈:“做爹的怎么能乱教?!”

就跟天子的名讳不能随意乱说乱写一样,父母的名讳也都要避开,不能直接写,也不能直接念。写的时候要少上一笔或多上一笔,而念的时候都要换个发音。

“其实也没什么大不了的,临文不讳嘛。”韩冈摸摸被吓到的儿子的脑袋,又看看不以为然的妻子,“不过世风如此,若是不遵从,被人骂做不孝也不好。以后记住就行了。”

韩冈的一对儿女一齐点点头,“孩儿知道了。”

韩冈将儿女放了下来,“好了,下去玩吧,都闷了一天了。”

但两个小家伙却没动,抬头看着王旖。只见王旖点了点头,方才行了礼后跑了出去。

韩冈冲着王旖笑了一笑:“贤妻治家有方啊。”

王旖不高兴的沉着脸,抱怨着:“人家都说是严父慈母,你这个做爹的还真是的,有哪里严过!害得大哥二哥和大姐都怕我了!”

“是姐姐管教得好。”

“小孩子不管教得严一点,才会坏事,是官人当甩手掌柜的错。”

周南和严素心连忙就说着。

韩冈对儿女的管教一向比较疏松,也是他这些年多在外任职,对儿女有份愧疚,平时多有宠溺。不过他敢这么放纵,也是因为王旖对子女一向教训严格。

对于怀胎十月生下的亲骨肉被王旖严厉管教,素心和周南私底下却都很高兴,这样才证明王旖对不是自己亲生的儿女也一并放在心上,另一方面也是因为王旖对亲生的韩钲也同样严格。

韩冈笑了一笑,家里面事有王旖主张,自己可以轻松一点,不过打算编纂蒙学教材也是得加紧了,别自己辛辛苦苦编写来,自家的儿女享受不上。

只是眼下要处理的公事也是一桩接一桩,洛阳这里人多、官多、事也多,许多时候,都身不由己,韩冈也觉得有些烦了,心道还是早点南下比较好。抬头正色对妻妾道:

“这两天,洛阳的事情也差不多了,等过了郑国公的寿诞,再将几位老臣都拜访过,也就可以南下了,倒时候,也没有这么多的烦心事了。”

