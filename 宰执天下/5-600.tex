\section{第46章 八方按剑隐风雷(20)}

韩冈带着笑意说着。

一开始他的确为了方便起见没有给细菌和病菌给出一个明确的定义,而是直接成为病菌。虽然之后他还是逐渐修改之前的说法,不过没有流传开,仅仅是苏颂、沈括等寥寥数人知道。

在公开的信息中,微生物依然是病毒。尽管这其中是有些问题,只是由此也让饮用开水的习惯逐渐在民间普及,故而韩冈也就听之任之。

但现在,终于有人站出来,说韩冈弄错了。不管细节上的对错,能发现韩冈之前的错误,并推翻他此前的结论,这便是里程碑,是韩冈期待多年的收获。

气学决不能像其他学派,树立起一个圣人之像,然后不敢对圣人的言论越雷池一步。这是披着儒学外衣的科学,是一门不断推翻权威的学问,必须要踩在先人肩上向上走。

质疑,才是气学的根本。

章惇眼中的韩冈,语气中有着一份很明显的得意。就像看到了自家的子侄有了出息,自得的对外人说上一句终于成器了。

章惇暗暗感叹,这就是器量。一人能否成大器,还是要看他的气度。

韩冈的姓格素来强硬,将横渠传承看得比天还重,为了气学与王安石斗了不是一次两次,最近更是把蔡京吊起来当靶子,让世人看到胆敢攻劾气学的下场。可是现在直接有人登门说韩冈错了,韩冈却高兴得很。如果一切争端止于学术,恐怕也不会有那么多争执了——当然,这是不可能的。章惇很清楚,几家学派的交锋绝不可能局限于学术,早就跟政治脱不开干系了。

苏轼放下酒杯。

韩冈认错,这可是难得一见。不是诗词歌赋,而是在他最擅长的领域承认失败,真的是前所未有的一件事。这会是气学从内部崩溃的第一块砖吗?会不会由此事开始,让人觉得韩冈的观点尽是谬论。就像他用腐草化萤和螟蛉有子二事,直接打翻了诗、礼两部的历代传注一样。

只是从韩冈的态度上看不出来,能够很自然的在外人面前说出来,就证明他根本没有放在心上。不会是在设陷阱吧?骗得人跳进去后,就拔出刀来。

苏轼不擅长考虑这些勾心斗角的问题,想了想就觉得烦了,直接就问道:“也就是说,那水中的八万四千虫就不是病毒,而是细菌喽?”

韩冈微微皱眉:“一钵水中到底有多少细菌,得看水质才行,要是蒸馏出的熟水,可没那么多。若是从河塘底舀出来的池水,千万倍亦不止。”

“八万四千,言其数多耳。宣徽不必如此执着于数字。”

韩冈当然知道在典籍中,八万四千、三千之类的数字,并不是具体的数目,而仅仅是表明很多而已,但他不喜欢对数字如此粗略的态度。他一直想纠正的恶习中,这是很关键的一条。

“精研医术就需要精确。什么样的水能用来冲洗伤口,多少比例的酒精能够拿来消毒,都要计算事前事后的细菌数量。错一个数字,就是多少条姓命。人命关天,岂能不执著?”

“世尊之言,非关医术,只是让人敬畏,明白自己的罪孽……水中细菌无数,九成九无害于人。也难怪佛祖戒令喝水前要持咒一番。”

“就算九成九无害于人,但还有一分是病菌,该烧水还是得烧水。尤其是灾异之后,难民聚集,要防止疫病传播,干净的饮食是最重要的一条。”

如今儒门诸派,气学、道学皆排斥佛家,新学也坚持着儒门正统,唯有蜀学,却有将佛道两家与儒门熔聚一炉的打算。这当然为韩冈所不能忍。

“烧水便是杀生,杀生救己,少不了要持咒一番。”苏轼扬着双眉,“苏轼听闻宣徽平素指斥浮屠乱道,所言皆非,不知如何看待水中八万四千虫这一段?”

韩冈的嘴角抽搐了一下。他最是不喜欢在这种说法。平白无故的占自己的便宜,让他很是不爽。

苏轼这算是挑衅了。但也是事实。自从韩冈推出了病毒论之后,这段时间以来,可是有越来越多的人跑去供奉佛祖。

这便是韩冈素来反感佛老,佛法却能够借其名而行的原因。为什么韩冈成为了药师王菩萨座下弟子?就是因为佛法中的一钵水中有八万四千虫,人身上有八万四千虫,这些本是空泛的论点,却因韩冈得到了事实的验证。

现如今的佛门传法,许多时候都会拉上韩冈的名字。一想到自己辛辛苦苦的宣讲气学的成果,让浮屠教众窃走占据,这就让韩冈心头压了不少的火气。

“既然子瞻相问,韩冈就明说了……”他沉吟了一下,然后道:“这么说吧。如果现在有一人,明知水中有致病的细菌,却不向世人透露,反而以此为名,让世人念咒忏经,信他的教义,聚敛财货土地,还不交税赋于官府。若有今人如此行事,敢问子瞻,此人依律当如何判?”

韩冈端起酒杯,喝了一口,润了一下喉咙。他并不是针对苏轼,而是针对所有脸皮老厚的佛门弟子,戳破他的谎言,让他们明白,自己是绝对不会转向他们的一方。

“咳……咳咳……玉昆,你这话……真是……咳咳……”

章惇差点没给酒呛死,满满一杯酒,一半洒到了外面,剩下的一半也没能顺利地灌进肚子里。不过他根本没在乎这么多,韩冈用来做比喻的说法实在是太过匪夷所思了。

按韩冈这样的说法,佛祖就是明摆着的妖人惑世。这一罪,地方官当视情节轻重给予不同的判罚,最甚者,可以引用十恶不赦中的不道一条,那时就只有四个字:决不待时——先砍了脑袋再说。

苏轼更是一时结舌,他完全想不到佛祖说的一钵水中有八万四千虫,还能从这个角度来解释。

“释迦摩尼几千年前降世,刑统也管不到他头上。韩冈乃是今人,不敢仿效,所以稍有所得公诸于世,现在看看还是不够完备,但还算有开创之功,后人以此为发端,迟早能够解决其他因病菌而染上的病症。释迦摩尼能创立佛教,传承千载,天下万邦,信众无数。其论才智论见识,肯定是远在韩冈之上。如果他不是宣扬教义,而是将他的才智用在了钻研医术上,又会是什么情况?千载光阴,种痘法当早已问世,数以千万计的幼子能得其救助,不至夭亡。甚至其他病症,伤寒、疟疾、痨病、疽痈,这些疾病都能有治疗的手段。”

韩冈滔滔不绝,苏轼愣了好一阵,才反应过来:“佛祖既已传法度苦厄众生,也无须再留医术以救人。”

“那曰后子瞻生病,去相国寺找个和尚来念上一卷佛经就行了?不知治头疼的是法华经还是华严经?”

“饿则吃,病则医,等死了,就找和尚念经。天地自有其理,当顺天应人。”

“既然天地自有其理,我等只需顺天应人,又何须求神拜佛。有他没他,不都一样?”

韩冈不信鬼神,纵然他无法解释自己为何会来到这里,但他相信,必然会有一个合理的原因,只是现阶段还没有总结和研究的条件,绝不会托付于无法探明的神秘。

科学,本就是承认自己的无知,然后不断追求对未知世界的认识,而不是心安理得的把世界万物的根本,安置在超自然的东西上,从此不再去根究。

章惇苦心举办的私宴,在不断的争论和调解之中勉强进行着,最后终于到了结束的时候。章惇心身俱疲,没有了挽留客人的力气。

韩冈、苏轼先后告辞,章惇靠在书房中躺椅上,只能苦笑。今天酒没喝多少,菜没吃多少,口水则费了许多。

‘韩三舌辩过人,识见广博,暴得重名非是无因。不过,我可不会跟他喝第二次酒。’苏轼离开时这样对章惇说道。

苏轼喜欢谈天说地,而韩冈又以渊博著称,只要坐在一起,应该能够谈得来。

苏轼虽然疏狂,却不是看不懂人情的人,不会当着韩冈的面,议论诗赋。韩冈器量恢廓,些许冒犯也会一笑了之。谁知韩冈器量虽大,可就是太过较真了,把苏轼都带得只顾争辩,全然忘了喝酒。这一回,算是做了白工。

道不同,不相为谋。这句话说得真是好,韩冈和苏轼这辈子是不能合得来了。论事论人都差得太大,隔阂之深,如海一般了。

不过这一回自己算是尽了力,心中再无愧疚。曰后两边再有什么龃龉,也不管他的事了。

章惇再一叹,免得麻烦……免得麻烦。

韩冈先一步从章家告辞,很快便回到了家中。

周南伺候着更衣,又端了茶上来,笑问道:“官人在章枢密家跟苏舍人说了什么?可有做诗词?”

韩冈摇摇头,放弃一般的长舒一口气,“道不同,难共语……幸好不会有第二次了。”
