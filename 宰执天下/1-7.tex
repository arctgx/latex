\section{第五章 心念亲恩思全孝(上)}

在河岸边徘徊了一阵,下定决心的韩冈要回到家中继续读书,韩云娘也要跟着回去收拾家务,她便扶着韩冈向村中走去。

两人刚刚走到村口,这时从下游的渡口处过来一人,看到他,韩冈的脚步不由得停住,小丫头则不知为何忽然胆怯的躲到了他的身后。

那人脸皮上尽是疙瘩,双眼外鼓,大嘴前凸,褐色隐花的绸布直裰盖不住高高挺起的肚腩。乍一看去,活脱脱一只秋后将要冬眠的胖蛤蟆。人能长出这幅模样也是难得。韩冈通过前身的记忆认得他,正是不断撺掇着韩家卖田的李癞子。

李癞子是村里排第一的大户,脸上疙疙瘩瘩如同翻转过来的石榴皮,像个癞蛤蟆一般,所以有了这个雅号,多少年叫下来,连本名都没几人知道了。其人在村里名声并不好,却跟县衙里的班头——外号黄大瘤的黄德用结了亲家,又通过黄德用结识了在成纪县衙中、祖孙相继传承了三代的押司陈举!

这陈举可是关西江湖上有名的奢遮人物,有着仗义疏财的美名——尽管他疏的财全是从成纪县百姓身上盘剥得来。

陈举继承父祖之业,把持成纪县衙政事三十年,曾经让两任知县、七八个主簿、县尉灰头土脸的从成纪县因罪罢任,其中一个背时的知县,还被夺了官身,‘追毁出身以来文字’——也就是说,这位倒霉知县身上的官皮给剥了,从官诰院和审官院被除了名,这比夺官去职还让官员们畏惧,毕竟夺官还有起复的机会。另一个更倒运的主簿,则参加了琼州【今海南海口】终生游,再也没能渡海而回。

自此之后,后任的知县、主簿等成纪县官员再没一个敢招惹陈举的。而陈举也识作,只要头上的官人老老实实,他便不会太过欺凌上官,如此两下相安。

李癞子攀上了陈举这尊大神,从四年前开始便当上了下龙湾村里的里正。他依仗了陈举和亲家,将许多差役赋税都转嫁到别人的头上,祸害了村中不少人家。不过若不是因为韩家老三重病急需钱,以韩家的家底,本也不会被李癞子欺。

也许是受到身体原主的影响,也许还有这几天来了解到内情的原因,韩冈对李癞子全无半点好感。为了一块土地,恨不得杀人放火,不论前生后世的哪一个时代,总是有这样的人。如果不是落到自己头上,韩冈对此本不会在意。可李癞子通过近乎于诈欺的手段,将韩家的田宅一点点的搜刮到自己手中。韩冈已经在心底立誓,日后肯定是要一报还一报的。

在仇人面前,韩冈却更加斯文有礼,他冲李癞子拱了拱手,行礼问好:“李里正,多日不见,一向可好?”

“韩……韩家三哥啊!好,好,都好。”李癞子有些狼狈的应答道。他的声音如公鸭一般沙哑难听,投过来的眼神不知为何却甚是怨毒。

李癞子的表情,韩冈看在眼底。他有些纳闷,李癞子已经如愿以偿将家里的田宅都刮了去,自家恨他理所当然,但他恨自己,却是从何说起?……难道真的是因为担心他家将田地赎回?

韩冈冲着李癞子又正正经经的一拱手,摆出一副真心诚意的模样:“小侄一病半年,其间家中多蒙里正照拂。等他日有闲,必摆酒致谢。还望届时里正不要推辞。”

“好说,好说!”李癞子眉头一皱,韩家的老三原本就是个能文能武的英才,只是有些傲气,不太爱搭理人。没想到在外游学两载,现在却变得伶牙俐齿起来。

在他眼中,韩家老三有着久病后的消瘦,一袭青色素布、圆领大袖的襕衫下空空荡荡,弱不胜衣。但其宽大的骨架子仍在,六尺高的个头仍给李癞子很大的压抑感。肤色是久未见光的苍白,脸颊几乎都被病痛消磨尽了,凸出的颧骨在脸颊上投下极深的阴影,唯独一双凹陷下去的眼睛被浓黑如墨、修长如刀的双眉衬着,愈发显得幽深难测,让李癞子浑身都不自在。

李癞子不耐烦的样子韩冈看得分明,能让仇家不痛快的事他一向很乐意去做,而且还有件事他也想要弄清楚。

“里正,河湾上的那块菜田……”韩冈开门见山的刚提了个头,就看到李癞子眼中的凶光顿时狠了三分,他心里有了数,分明是戳到了症结上。

“这个过几日再说!”下龙湾的里正爆发般的吼了一句,扭过头,转身就往村中走去。他心中暗恨,这措大病好得这么快作甚?再病个半月,让韩家把典地的钱花光,他哪还会需要担心什么。

盯着李癞子远去的背影,韩冈冷哼一声,李癞子眼中的凶光他也看见了,但自己已经病好,不论李癞子能玩出什么花样,他都有能力去应对。

……………………

到了傍晚,韩冈的父母韩千六和韩阿李\footnote{中国古代的习俗,正经人家的妇人闺名向不公开,外人相称多是用娘家姓。前面加个阿,或是后面跟个氏,出嫁后再冠上夫姓。一般来说民家用前一种称呼,而官户人家则是用后一种。如文中韩冈之母,娘家姓李,夫家姓韩,便唤作韩阿李,等韩冈有了官职,可以封赠父母的时候,就成了韩李氏。再如八仙传说中的何仙姑,正是北宋时人。当时有一道奏章曾提到她,其中便称她为‘永州民女阿何’。}也挑着空箩筐一身疲惫的回来了。韩千六手上提着个坛子,闻着有酒味,但里面装的却是酒糟;韩阿李的箩筐里则放着半截羊腿,用荷叶包着,进门后就递给了迎上来的小丫头下厨料理。听着从儿子房内穿出来的琅琅书声,夫妻两人相视一笑,都觉得再苦再累也是值得的。

韩云娘晚饭准备得很快,很麻利的处理好羊腿,肉切下来熬粥,骨头剔出来熬汤。把碗筷一摆,进去叫了韩冈出来,一家人便围坐到桌边。

韩千六和韩阿李都是四十多岁的年纪。可能是常年劳作的缘故,两人看着都有些苍老,比实际年龄要大上一些。韩千六跟韩冈身高差不多,都是有着六尺上下,在关西也算是高个,相貌轮廓也很是相似,浓眉大眼,方脸刚劲,称得上相貌堂堂。

相对于韩千六的高大,韩冈的母亲就矮了些,相貌并不出众,不过里里外外都是一把好手,也是韩家的主心骨。因为韩冈的外公曾经做到都头,他的舅舅如今在百多里外的凤翔府也做着都头,斩过几十个贼人的大斧常年在家中墙上挂着,武家出身的韩阿李的脾气,远比总是笑呵呵的韩千六要硬上许多。她将手中的擀面杖一举,下龙湾村没人敢大喘气。

韩千六东头坐着,韩阿李坐对面,韩冈位子在下首,而小丫头就只能站在一边服侍,等到大家都吃完后再去厨房填饱肚子。韩家虽是寒门,但一样守世间的规矩,若是有外人来做客,连韩阿李都得躲到厨房去吃饭。

三人围坐在大桌旁,显得空空落落,冷冷清清。本来连着韩冈的大嫂,这是一个是七口之家。在韩冈没有出外游学,而他二哥也还在家里的时候。韩家三子连同父母总共五人挤在一张桌边,大嫂和韩云娘则在旁服侍着,一顿饭吃得倒也热热闹闹。

但自韩冈的大哥、二哥同时战殁之后,仅仅过了三个月,他的大嫂就被娘家叫了回去,还一起带走了二十亩的嫁妆田\footnote{在宋代,妇女的财产权受到法律保护,出嫁的嫁妆在离开夫家的时候也能随身带走。}。依礼制,夫死后当有三年孝期,可在西北边陲也没那么多臭规矩。韩冈只从云娘那里听说,原任大嫂过了年就要再嫁人了。

如果没有融入原主的记忆,韩冈也许会对此很惊讶,但既然已经把记忆融会贯通,他便只觉得理所当然。理学如今还是提不上台面的学派,世间更没有饿死事小、失节事大的说法。丈夫死后,还在生育年龄的寡妇再嫁极为常见,就算本人不愿,娘家也会逼着走。

若是哪位寡妇能带着大笔家财出嫁,那追求者甚至能踏破门槛。真宗朝曾有张贤齐和向敏中两位宰相,为了争娶一个有十万贯嫁资的寡妇,将官司打到了天子面前,闹得朝堂鸡飞狗跳。世风如此,矢志守节那是没影的事。

韩冈拿起筷子,低头吃着自己的病号餐,一如往日的羊肉粥和小菜。每天早中晚三餐,花样都是不变,韩冈也没有怨言。他知道父母的辛苦,更知道这些来得有多么不容易。

韩千六、韩阿李吃得比儿子简单得多。与这个时代的普通农民们一样,韩家平日里的菜谱很是朴素单调,满满一碗看不到几滴油腥的素汤饼——其实就是面条,只不过宋时凡是跟面食有关的食物都要缀个‘饼’字——再加上几个炊饼。

PS:注释中出现的何仙姑恐怕没人不知道,上洞八仙中的曹国舅和何仙姑,正是出现在这个时代。而吕洞宾,铁拐李也同样在此时流名远布。有没有兄弟想看到他们出现在本书里的?想看的话,投红票支持一下啊!
