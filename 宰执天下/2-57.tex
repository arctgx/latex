\section{第13章 不由愚公山亦去(二)}

【今天第一更,晚上还有一更。求红票,收藏。】

“王家有贼?!”傅勍闻言便咧开嘴笑了,猩红的舌头舔着上唇,如一头嗅到血腥味的饿狼,毫不掩饰的把内心的饥渴展露出来,“今天倒真是事多。想不到还真有这等趁火打劫、趁乱行窃的贼人!”

照空甩了一记响鞭,驭马转向,浑忘了跟刘希奭打声招呼,傅勍就带着一队跟在身边押阵的巡城直奔王家而去。

到了王家门前,他收缰止步,看了一眼紧闭的大门,大声吼道,“院里的贼人听着,本官领兵在此,尔等插翅难飞。还不快快开门,自缚出降!”

王家的院门没有丝毫动静,傅勍怒气勃发,抬手便是一指:“来人!去把门给本官撞开!”

三五条壮汉领命上前,哐哐的撞门声随即响起,傅勍再伸手指了指王家的邻院,“来人,把院墙给本官封上,里面的贼人一个也不得放过!”

跟着傅勍的巡城甲骑中,又是奔出了几个手提弓箭的汉子,径直进了王家的邻院中,替换了守在里面的百姓,不让贼人逾墙出逃。

院门一下接着一下的被猛.撞,而细长的门闩看起来随时都会在下一次撞击中折断,钱五忙叫了几人顶在门后,却也不知能守着多久。

咣咣的撞门声让窦解心惊肉跳,每一声入耳,他身子就要抖上一下。

“李铁臂!钱五!现在怎么办!?”窦解在院中急得发昏。前面他又换了两面墙想翻出去,都看到一群人守在墙底下,现如今几面都给围定了,当真是插翅难飞。

“不管了!”李铁臂一咬牙,等门外的人冲进来再想走可就来不及了,只能拼上一下了,“快,护着七衙内翻墙出去!拼一拼,墙对面的那些鸟货挡不住我们!”

一个伴当打头阵跳上了院墙,但他还没翻过去,就啊的一声惨叫,重重地摔了下来。看着插在他肩头处,摇摇晃晃如同风中蓑草的长箭,院中众人自窦解以下,脸色全都跟死了爹娘一般,这真是把他们当作贼来看了。

外面的傅勍看着院门始终撞不开,心头火气则是噌噌而起,大骂出声:“一群废物,还不拿斧子过来!”

潜火铺的铺兵手上就有斧子,绳、锯、斧这些都是防止火势蔓延的必备工具。几名巡城被傅勍一句喝骂,忙从潜火铺借来斧子,喝叱连声,用力砍起王家的大门。

雪亮的利斧破风而下,重重的劈在门扇上,轰然一声,木屑横飞,顿时就在门上开了个半尺长的口子,而门后也传来一声尖叫。

“好!别停手,把这门给我劈成柴禾!”

傅勍兴奋的等着大门被砍开,却听到后面一片喊声。回头一看,只见着净慧庵火势突然转急,火焰又腾起了有半天高。他权衡了一番,觉得还是救火要紧。

“都小心一点,进去后贼人若有反抗,一律格杀勿论。”说罢他就拨转马头,赶回去指挥救火。

就算没了傅勍压阵,劈在王家院门上的斧头,依然一下快过一下。一块块木材碎片纷纷从门上被砍了下来,门板上的缺口也是越来越大,渐渐将门闩露了出来。

门前,一个身材粗壮的大汉将手上的利斧对准了暴露出来的门闩,使足气力向下一挥,就听到一声脆响,细长的门闩被一分为二。大汉收回斧头,猛力一脚,院门晃了一晃却没有开,被里面的什么东西给挡住来。但再一脚之后,已是伤痕累累的半扇木门竟被他踢崩了下来。

木门支离破碎的倒在地上,堵在门后的一个窦七衙内的伴当连滚带爬的退了老远。那大汉随即提着斧头当先而入。跨过门槛,转头一看,剩下半扇木门后,也靠着一个贼人。大汉也不多话,抬手一斧,照脑门来了一下。半边天灵盖被削飞,红的白的顿时哗啦啦的淌了满地。

提着刃口上不断滴着脑浆和血液的板斧,大汉如同饿虎的双眼一扫院中,再没一个人敢动弹一下。紧跟着他,后面一队巡城也手持刀斧带着绳索一拥而上,将院内众人一个个捆绑起来,而后又踢门进屋去搜查。

前面有大汉杀鸡儆猴,又见到巡城们手中明晃晃的利刃,钱五、李铁臂都聪明的没有反抗,他们的希望最终还是放在了窦解的身上。

“我爷爷是窦观察!我爷爷是窦观察!”窦解在被绑起来的时候,还连声喊着。

只是领头的巡城大汉抬手就给了窦解一巴掌,打得他满口是血,半边牙都松了,让他就此没了声息:“你这贼人是窦副总管的孙子,爷爷还是韩相公的儿子呢!”

他再一声吼:“把他们都给绑牢了,押到刘、傅两位官人面前请功。”

立马于熊熊烈火之前,傅勍意气风发。今夜即已救火,又将擒贼,被酒精搅得昏昏沉沉的脑中,只剩下功后受赏这一事。而从刘希奭的角度看过去,傅勍映在火光中的剪影,从里到外,都透着志得意满四个字。

由于傅勍的有效指挥,火势渐渐小了下去。这时候,王家的贼人也被押了过来。傅勍得意洋洋的居高临下,俯视起被押解到他脚边的俘虏。

可他第一眼就看到了被捆成了一枚粽子,半边脸肿得跟馒头似的窦解。傅勍浑身的酒意顿时化作冷汗涔涔的冒了出来,窦家的七衙内他认得。

虽然傅勍才回到秦州没有几天,但窦七衙内的赫赫威名早已是如雷贯耳,也亲眼见证过窦解在城中横行霸道的样子。窦舜卿的权势,哪里是他一个小使臣抗得下来。

傅勍心底叫苦不迭,‘今天是犯了哪路太岁,怎么给撞上了这一位?!’

该怎么办?是押回去还是就地释放,他心中纠结着,但对上窦解充满恨意的双眼,傅勍猛然醒悟过来,‘不,不能让窦七衙内的身份暴露。’

可这时不知是谁在人丛中冒出了一句,“这不是窦七衙内吗?”

被叫出了身份,窦解顿时爆发出来,面容狰狞的大吼着:“我爷爷就是窦观察!我也有官诰在身,尔等将我这朝廷命官绑起,是想造反不成?!”

‘完了!’傅勍悲叹着,‘怎么摊了这蠢货。’他将求援的眼神投向刘希奭,却见秦凤路走马承受却也是目瞪口呆的愣在当场。

“啊!这不是王家大嫂吗?!”

“快来人呐,王家大嫂被打得快不行了!”

“啊也!那些贼人把王押衙的儿子女儿都丢到井里去了!”

一连串吊高嗓门的喊声适时的从王启年家的院中传了出来,将窦解的罪行当众叫破。一传十,十传百,在场几百人都听到了,火场中的空气仿佛凝固,连救火的人也停了手。不用眼看,直接就能感知到,燃烧在周围百姓心中的怒焰,甚至比还要炽烈。

刘希奭终于从极度的震惊中警醒过来,环视着怒意沸腾的人群,他干咽了口唾沫,怕是不用等到明天天亮,窦解今夜做的事就能传遍整个秦州。

傅勍这时靠过来,脸上的神色比哭还难看,“走马,你说该怎么办?窦七衙内还有官身啊……”

‘还能怎么办?!’刘希奭在肚子里从傅勍开始一直骂到傅家的祖宗十八代,若不是这个醉鬼,他如何会落到眼下这般进退两难的境地。

“傅勍!你领兵巡检城中,难道不是为了捕盗?今夜你既然捉到了贼人,不送去衙门见官,难道还想放了他们不成?!”刘希奭从牙缝里挤出声来,却是破釜沉舟。眼下的情况与窦舜卿结下死仇已是板上钉钉,既然如此,不如在窦舜卿的身上再踩几脚,踩得他不能翻身,这样才能保全下自己。

在数百围观百姓面前,秦凤走马展示着自己铮铮铁骨,“不管是不是窦观察家的衙内,也不管他是不是有官身,即犯律条,伤人害命,决没有轻饶的道理!傅勍,将这些贼人押去州衙,请李大府给个公道!”

他再指着仍在燃烧着的火场,对着欢呼出声的数百人众,放声喝道:“火势尚未熄灭,尔等如何能放手,还不快去救火!”

方才一番话,刘希奭已经树立起了些许威望,他如此一说,众人便纷纷应是,灭火的工作重又紧张的展开。留下傅勍继续指挥救火,刘希奭便亲自押了窦解一众回衙,跟在后面百姓又有五六十人,都是些老弱妇孺,不用参与救火,却能去跟着看热闹。

窦解双手被一根绳子绑了,绳头则扯在刘希奭的随从手中,走得踉踉跄跄。刘希奭丝毫不顾忌他的身份,让窦七衙内恨不得把这名阉人身上的肉一口口的咬下来。他瞪着刘希奭的背影,嘴里不停的念叨,“等我爷爷来了,就把你千刀万剐。”

听着后面传来的声音,刘希奭心中愈发的坚定。既然已经得罪窦舜卿,那就得罪到底好了。他是中官,是天子近臣,在天子心中留下一个刚正不阿的名声,比拍好窦舜卿的马屁对他更有利。

“走快一点!”刘希奭沉声喝道,“早点让窦副总管看看他孙子做得好事!”

