\section{第37章 朱台相望京关道(05)}

耶律乙辛正等着前线的消息。

贡事不谨,为臣不恭。私通他国,心怀悖逆。

这是耶律乙辛给高丽国王王徽定下的罪名。

从宋辽边境退兵后,他便准备从高丽那边找补回来,很快就在东京道组织起了两万多兵马,大部是留守东京的部队,并准备好了渡江的船只。当张孝杰带回了韩冈的话,耶律乙辛再也没有半点犹豫,立刻发兵南向。

王徽几年前就有了风痹之疾,他通好宋国,除了联宋抗辽、以求自保外,也有一部分是从宋人那里求医问药的想法。近来病势加重,诸子争位,不趁这个时候下手,迟了后悔都来不及。

高丽将国土分为五道两界,其中的东界、西界便是高丽的北部区划,首当其冲。

出兵十日,辽军前锋抵达高丽国境。东西皆至海滨的千里长城,就是除了沿途的几处山城寨堡之外的第一个目标。

于四十年前修筑的千里长城,给了高丽人超过实际的安全感,国库中的钱粮都用在了修筑寺庙之上,加之近年来又有了日渐强盛的宋国撑腰,高丽国中差不多已经是马放南山。

这一次当听说辽国侵宋不支而退,甚至不得不割地求和。从国王到朝臣,各个弹冠相庆,除了一小撮倾向于辽国的臣子外,其他人无不为宋国的胜利而庆贺不已。有宋国在南方虎视眈眈,最危险的邻国便再也不能为患高丽了。甚至有人还想打过鸭绿江去,夺取高句丽和渤海的旧地。

此时遇袭,高丽军完全没有准备。甚至是长城外的烽火台燃起了烽火,都仅仅认为是山野间的生女真又来骚扰了——这是事后,从长城三关守将口中拷问出来的结果——没人认为辽国在此时还有攻击高丽的余力。

全数铁甲的辽军,虽无法胜过装备更加优良的宋军,但面对周边小国时,却有着压倒性的优势。前锋的一次试探性的突袭,却轻易的击破了高丽苦心经营多年、横贯半岛的千里长城,夺取了定州等关隘。这让与宋军鏖战多日而不得一胜的大辽精兵,终于顺了一口堵在心口的郁气。

突破长城后,接下来便是西京平壤。

耶律乙辛没有耽搁时日,攻破长城的第二天,便点派精骑南下,希望能打高丽一个措手不及。

“尚父!大喜!大喜啊!”张孝杰奔走入帐,兴奋之色溢于言表,“平壤守军出战惨败,其西京留守开城出降。这一回八百铁林军立了大功啊,一个冲锋便把出战的高丽人给冲散了!”

“哦!这么快?!”耶律乙辛霍然而起,进军如此顺利,让他都忍不住喜上眉梢,“铁林军伤亡如何?”

“三骑而已,皆是坠马。”

“好!好!!”耶律乙辛不停点头,显是极为满意。

刚刚经过重新整编,得赐军号为铁林的两千具装甲骑,是他的心头肉。这一回派出去的八百骑,从辽东过鸭绿江后一路南下,几百里的道路皆是崎岖难行。抵达目的地后又要立刻上阵,耶律乙辛纵然对铁林军充满信心,可也担心因为太过劳累,而使得他们遭受过重的伤亡。想不到战阵之上就仅有三骑伤亡。

现在想想,还是两国在装备上的差距实在太大了。契丹骑兵虽大多数都不在状态,但影响不了铁甲、重弩等神兵利器所带来的优势。耶律乙辛觉得自己是太糊涂了,平白放着路边树上挂的肥肉不要,却去啃南面的硬骨头。

“其余伤亡呢。”他又关切的问道。

“总数不过百余。连同于途得病的,也仅只两百。”

“好!好!好!!”耶律乙辛一声比一声高。他终于放下心来:“下面可就是开城了,希望高丽国都能名副其实。”

张孝杰笑道:“尚父放心,天兵一到,自是旗开得胜。高丽王徽,除了开城投降,便无第二条路可走。他的开城就只有开城。”

“一切顺利的话,打下了开城。把王徽带回上京,换个听话点的高丽王……那个出了家的就不错。”耶律乙辛依稀还记得王徽有个儿子出家做了和尚,“只要他献上岁币,就让他做高丽王。不要多,二三十万贯就可以了。”

高丽王室能在海贸中收到的税也就这么多,耶律乙辛一口就要包圆了。

“尚父还打算留着高丽?”张孝杰轻声问着。

“先留着吧,前日不是说过吗,他们还有用!”

灭亡高丽、控制海上商路的想法,耶律乙辛的确有,可他清楚现在做不到。海贸不是那么容易就能上手的,而且辽国船只去宋国的港口只会被当成敌寇,换成是高丽船只就会好些。

等再过两年准备妥当,差不多就到时机了。耶律乙辛默默的想着。

……………………

“竟然是高丽!”

接到了河北的通报,韩冈都不禁吃了一惊。

不仅是他,他的幕僚们也都惊讶莫名。他们都以为耶律乙辛现在最该做的事是稳定国内,保住他摇摇欲坠的地位。想不到他会以攻为守,选择继续侵略他国。

“那位尚父殿下,终究还是进取的性格啊。”韩冈有些遗憾。

如果耶律乙辛能像西夏梁氏那般,败了一次就回国杀上一回,两次折腾下来,辽国就跟西夏一样完蛋了。可惜的是,耶律乙辛打着找补的念头,想从高丽把亏掉的本钱给赚回来。

黄裳疑惑道:“高丽小而贫,纵使攻下来,又能有多少好处?!何况眼下又是最坏的时机。”

“大宋和倭国之间的贸易完全是由高丽商人控制,其中的利润,每年当在百万贯以上。还有高丽的人口,多达数百万之众,对辽国来说也是一个大补剂。不过……”

“不过什么?”黄裳追问。

“高丽北方多山,契丹骑兵难有施展的余地。”韩冈对半岛上的地理印象很淡薄,不过多多少少还是有一点。

章楶点头表示赞同:“枢密说得是。要不是有河山之险在,只凭高丽人的本事,隋炀如何会三征而不得?辽国也早就把高丽给灭了。”

韩冈轻咳了一声:“高句丽和高丽不是一家,只是王氏攀附。”

除非汉唐是一家,高句丽灭亡多年后才建立起来的高丽,连国姓都不一样,如何能混为一谈?

章楶摇摇头:“此辈皆是蛮夷,攀附本是常理。不过我中国也不需要为其修谱牒,他乱认祖宗,又与我何干?”

“……说的也是。”韩冈笑着说道。

高丽的问题迟早要解决的,名称和传承什么的也不用太在意了,他也并不打算把问题留给后世。

半岛南方的土地至少能安置几百万人口,占领下来后,对推动海洋产业的发展也有着极大的促进作用。在韩冈的计划中,那是必不可少的一片土地。只是他没想到耶律乙辛下手会那么快,这边才结束,那边就开始动手了。抢生意抢到自家头上来了。

算是意外之喜吧。韩冈心中也不知是该喜还是该恼。虽然这个结果很可能与他跟张孝杰说的话脱不开干系。

“既然辽国正用心高丽,想必萧十三不敢有所异动。”黄裳欣慰的对韩冈道,“枢密的计划现在是更加圆满了。”

“只算是运气。”

一切都是因势利导。征伐高丽是意外,但耶律乙辛无力旁顾则是早已确定的事实。

要不是确认这一点,韩冈怎么会放任折家出兵复仇?

北面的邻国比过往的任何时候都更要渴求和平。

此外,麟府军针对的黑山党项和阻卜部,对契丹人来说,本也是需要提防的对象。

折克行虽然过了界,但在辽人而言,一个是无力反击,不得不镇之以静,另一个则是死的都是心怀异心的墙头草,无伤西京道根本。纵然免不了要有反应,可绝不会再起大军。区区外交照会,正好可以证明辽人外强中干的底细。

而一切正如事前所料,依然坐镇西京道的萧十三加强了边境上的防备,却没有更多的反应。而是采取了先礼后兵的态度,谨慎节制之处,完全颠覆了宋人过去对辽国固有的印象。

“他们的粮食不够吃了。”韩冈很确定。

萧十三提供不了足够的粮草来集结大军,更不可能组织起人力来打草谷。六月的牧草虽丰美,可出征的战马可不能只吃草。天平完全的倒向了韩冈这一边。

黄裳点头道:“肯定是不够了。那么大的一仗打下来,哪边都需要口粮啊。”

章楶看着韩冈和黄裳两人,忍不住暗暗担心,提醒道:“只不过这一回枢密你放纵折家越界,恰巧碰上了辽人攻高丽,或许京中会有人以为抓到了枢密你的把柄。”

“交通辽国吗?证据呢?”韩冈呵呵笑道:“他们若是找辽人作证,那可真是再好不过了。”

“枢密,可千万要小心,真要有心于此,什么样的证据都是能找得到的。”章楶很清楚,构陷二字有多好写。三人成虎这句成语的出处又是哪一桩。

“不妨事的。”韩冈有心理准备,这些年同样的事情他见得多了。

领头跟自己过不去的是王安石,韩冈还是比较相信他岳父的人品。不过他不会把自己的命运放在别人的人品上,纵然是王安石。

‘我可是专家啊。’他低声道。
