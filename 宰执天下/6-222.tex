\section{第28章 夜钟初闻已生潮(四)}

轰的一声巨响,震动了整座横渠书院。

“又来了。”

王祥咕哝了一声,翻了个身,把被子盖到了头上。

但他的午睡立刻就没了下文,韩钟推开门,大声嚷嚷,“瑞麟,蒸汽机又爆一架!”

“关我屁事。”王祥在被子下面闷声闷气的说道。

韩钟一屁股在床前的坐墩上坐下,拍着被子,“瑞麟,你家我家都出了钱,还叫不关你的事?”

王祥闹的没法儿再继续睡了,一翻身,将被子掀起来,不快的说道,“作死的人,管我何事?!蒸汽机事关天下,一旦推行于世,便是千年不遇的变化,岂是三两月之内就能做出来的?!”

被王祥的起床气冲到,来人依然笑嘻嘻的,丝毫不动怒,“瑞麟这话说得倒是没错。”

在他看来,这段时间,许多有心蒸汽机的人们,也的确是太过急功近利了。

耶律乙辛给出的王爵悬赏已经让人目瞪口呆,而雍秦商会为了蒸汽机,拿出了总计二十万贯的财货,更是让整个关中为之沸腾。

辽国的郡王,远在万里之外。可雍秦商会的十万贯,可是实打实的现钱。而且雍秦商会的声明中还说实用化的蒸汽机不可能一蹴而就,只要能完成一步,就有一步的赏金。

最基本的蒸汽机就只要求有抽水的功能,只要能代替风车,从深井中抽出水来,便算是成功。

而作为第一步悬赏金的三万贯,已经放在了横渠书院中,只要通过了横渠书院的验证,便立刻发出去。

本来有人觉得,横渠书院太过于参与到商人们的活动中,实在是有失斯文,但苏昞一阵发作,当着所有师生的面,发表了一篇演讲,

‘都说儒家诸经是大道,而工匠之事,是技,是术。术与道比,自是等而下之!但什么才是道?孔曰成仁,孟曰取义,这就是道。圣人所说纲常、礼法,其目的都是为了一个仁字。秉仁心,施仁术,最后实现仁道。不是拿着经书空谈仁义。

何为仁?由温饱至小康,由小康至大同,让天下万民一步步得到这样的生活,这就是仁。一切有违于此,皆是违背圣道。蒸汽机虽只是器物,却能致民安康。若是此等仁器,不失大道,却有失斯文,那就是斯文错了,该摒弃之!’

有在关西德高望重的山长苏昞作背书,又是在横渠书院中,有数千士人作见证,关中上下,又有谁不信雍秦商会的诚意?就看谁来拿了悬赏去。

太祖皇帝曾经说‘措大眼孔小,赐与十万贯,则塞破屋子矣。’真是一点不错。

这份悬赏,即使对于有心进士的士人,也纷纷忍不住心动了。

一般的文官,一年的正当收入,连同年节赐物在内,也不过两三百贯而已。三万贯,要赚足一百年。

更何况,能晋身朝官的官员,十中无一。选人阶段的文官,俸禄也就一百贯上下,最多两百贯。加上一些不能见人的收入,能有五百贯就是天大的喜讯了。

相比起三万贯的初步悬赏,总计二十万贯的好处,诸多措大,的确撑破了眼孔和屋子。

横渠书院所教授出来的学生,无一不深深明了蒸汽机的原理,甚至前人设计出来的有用部件,在过去的《自然》中也能找得到,勤走图书馆,没有翻不到的。

所以一时间,研究蒸汽机在书院中蔚然成风,有一个人闭门造车,也有多人联手,签下了协议,共同去博取那三万贯的悬赏。

但相应的,横渠书院内部的试验场,以及学生们的住处,都经常出现轰隆的爆炸声。

各家的锅炉、气缸炸了一遍又一遍,上一次,韩钟和王祥所住的小院里,还飞来一根铁制的曲轴,砸到了院子中,尚幸没有伤到人。

王祥起身,一边打着哈欠换衣服,一边说,“眼下只有人伤,再过一段时间,可就是要死人了。”

韩钟则道,“也不是全然都是坏处,即是没能发明可用的蒸汽机,说不定在这中间,能发明别的东西。”

“这倒是,若没人去研究锅炉,也不会顺手将高压锅给造出来。”

高压锅是如今在关中开始流行的新玩意儿。是纯用铸铁制成,不论是锅身,还是锅盖,都是铸铁的。锅身和锅盖上下设计好卡口,盖上后只要转动一个小角度,锅盖便被牢牢卡死在锅身上。在锅盖内缘,还垫有一圈石棉,一旦合上,上下的缝隙便被牢牢封住,在锅盖的正中央,有一根不到一寸长的小短杆,中有小孔,连通内外,另外还有个活动的塞子,能够盖在小短杆上。

如果锅中有水,煮开后,水汽无法散发,就会让锅身中的压力越来越大,直至将那个小塞子给推起来。

用这种高压锅极省柴薪,煮饭上面的盖子冒了气,就可以从炉子上拿下来了,放上一阵,自己就熟了。而肉类,也很快就能炖烂,若是放在炉子上忘了时间,连骨头都能煮得入口即化。

这种压力锅,贵虽贵,可这么厚的铸铁,一看就能用上几十年,又省柴薪,如果配合市面上的小煤炉的话,日常的饮食能省下一半左右的薪炭钱。

所以在市面上出现才两个月,就立刻在关中传开了。

“就是太重了,妇道人家哪个提得起?”韩钟说道。

王祥此时已经收拾好了,起身与韩钟一起往书院走去。

“提不起?”王祥向着一名同学遥遥拱了拱手,偏头对韩钟道,“别小瞧妇道人家。挑着上百斤的担子,走上几十里的山路,健步如飞,关西这边多的是。还有……你不记得食堂的哪一位了。那可是膀大腰圆,肚子里能行船,两条胳膊上能跑马。”

“……是在京兆府耍过两年女相扑的那人?”

“还有别人吗,那块头,你我加起来才抵得上。那可是号称赛张飞的!”

仁宗的时候,女相扑正流行,各大瓦子里面天天都有女相扑的比赛上演,靠相扑吃饭的女子有数百个,后来仁宗皇帝听说了这个热闹,便召了几个女相扑进来在御前比赛。可司马光听说了之后,一封谏章把仁宗皇帝的脸皮打得噼啪响,不仅天子不能在皇城里看把戏了,就连京师瓦子里的女相扑,也一股脑的给关了门。

不过京师的女相扑给禁了,地方上还没有。关西年年征战,武风甚炽,最受欢迎的就是相扑。当年种世衡守清涧城,城外修庙休到一半,突然发现大梁太重抗不上去。种世衡眼珠一转,就说要赛相扑,登时满城百姓都涌来了,正好落入他的计中。当种世衡开口说架上房梁就开演,老老少少便一拥而上,一文钱不要就平白帮他将一根大梁扛上了山顶。

所以京兆府中,在蹴鞠联赛出现之前,相扑就是最受欢迎的比赛。而女相扑,受欢迎程度丝毫不比男子相扑差到哪里。赤裸上身或是只穿一件什么都遮不住的短裳,下面与男子同样只穿一条兜裆布,哪里能不受欢迎?

不过又是司马光坏事,他在京师被王安石赶出京师,来到京兆府担任知府之后,第一件事,就是继续跟新党过不去,新法在京兆府中完全推行不下去,第二件事,就是拖横山攻略的后腿,弄得广锐军叛乱的时候,直面其锋的邠州,就连城防战具都没有准备好,差点就给攻破了,第三件事,就是清理市面,像女相扑这个他亲自写了谏章的赛事,当然是第一时间被取缔。

横渠书院厨房中做菜的寡妇赛张飞,就是其中一名受害者。

王祥啧啧称叹,“上次三年级的几位师兄喝酒闹事,她直接上来,一巴掌就把一位师兄拍飞了。第二天好不容易爬起来,又被拎到了训导那边去,要多惨有多惨!”

韩钟想象了一下那个场面,不寒而栗的抖了一下身子,的确是可怕,莫说区区一个高压锅,就是磨盘也能一手掀起了。

两人说着话,徐步走到半山腰上。山门在望,王祥望着不远处冒起的腾腾白气,一声叹,“终归是太乱了。”

“凡事有利有弊,总体上还是好事,总比读经读到傻了要好。所以山长只是疏导,却没有反对。”

“只是约束不严,换作是岳父来,就会好些了。”

韩钟不同意的摇摇头,“家严的约束何曾严过。”

过年的时候,韩钟去了一趟巩州乡里,拜见了祖父母。经过一次长距离的旅行,他又成长了许多。

表面上看,韩冈施政苛刻,就是行乞也被严禁,弄出一个养济法,将乞丐都发配到边疆去。但韩冈知道,他的父亲在朝堂上的手段其实并不狠厉。

世人只看到了他一锤砸死蔡确的果决,却没看到曾、薛、苏等逆贼,十恶之罪却连死刑都没有判,仅仅是发配岭南。又有人以为他们很快就会病死——因为各种原因——但他们到现在为止,依然活得好好的。

不过若是有韩冈亲自指点,蒸汽机应该很快就会出现,虽然这个想法没有来由,但韩钟就是如此确定。
