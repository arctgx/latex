\section{第37章 朱台相望京关道(02)}

宰辅们的窃窃私语,监察御史们看得很清楚。只是都聪明的视而不见。

快要到朝会时间了,东西阁门使和御史都已经出来整顿官员队列。

强渊明低声对蔡京道:“恐怕又是为了河东的事吧。”

“还能有别的?”

“其实放宽心一点,让他们回来又如何?”

‘回来又如何?’

蔡京轻笑了一声。他觉得自己若是处在宰辅们的位置上,怕是很难底气十足的说上这么一句。一旦吕惠卿回来,两府诸公恐怕谁都斗不过他,他可不像王安石,主持大政之后,还能轻易放手。

若只比个人才能,蔡京不觉得大部分宰辅在朝堂政务上会输吕惠卿多少,可要是比起势力,却肯定跟吕惠卿没得比了。

“王平章的姓子执拗归执拗,可看他现如今的样子,怕早就是无心朝政,心里只在乎新学存续。一旦吕吉甫回来,他多半就会把大权交托过去,然后安心下来去教太子读书。王平章年过六旬,韩玉昆才交三十……他也只能靠吕吉甫。”

强渊明点了点头,“的确如此。”

无论从学术上,还是在朝堂上,吕惠卿都是王安石认定的继承人,这也是新党内部公认的唯一人选。

王安石一旦放弃现在主导朝堂的权柄,将吕惠卿推上来。新党人心必然归附。当年吕惠卿就已经代替罢相的王安石久任朝堂,维系新法、新学、新党,到时候只是恢复正常而已。

幸而让吕惠卿顶替王安石执掌朝纲,韩绛、蔡确都不会甘心,下面的两位参知政事——曾布、张缲也都不想有如此强势的同僚。否则,他们这些绳纠百官的监察御史都要被吕惠卿压在头上。

吕惠卿也不是没在政事堂待过,他的为人和行事作风,亲身感受过的人不少。否则这一回,也不至于被那么多人愿意为了让他久留陕西而出上一把力。阻击韩冈的是王安石,而趁机拦下吕惠卿的,可就是他历年来开罪过的一群人了。

“不过河东那边朝廷到底是怎么想的?总不能装没听到,至少总得说句话吧。”在强渊明看来,以韩冈的身份在张孝杰面前说的那番话,至少一个‘非所宜言’的评语,朝廷最轻也该给个罚铜的处分。

“下诏斥责?”蔡京摇摇头,据他所知这个方案两府早就考虑过了,“且不说能不能说动皇后,一旦当真为此下诏,韩冈多半会趁势辞官,谁来收场?”

如果换做天子身体还好的话,韩冈这么做就是自寻苦吃。可现如今的情况,两府只会陷入被动,皇后那边就更不方便说话了。

召回京来质询更不可能。那正好让韩冈如愿以偿。

只能放一边。

只是树欲静而风不止。宰辅们的想法,下面的官吏们可不一定要配合。之前朝堂如一潭死水,给在京宰辅们牢牢镇住,那是没办法,可现在既然韩冈已经掀起了轩然大波,岂会少得了兴风作浪的人?

蔡京轻轻的捏了捏笏板,他其实也不介意趁风破浪一回啊!

机会快要来了。

……………………

文德殿后,向皇后已经是全副穿戴,凤冠朝服,翻着永远也看不完的奏章,只等着上朝的时间。

赵佣乖乖的坐在她的旁边,一声不吭。六岁的皇太子,完全不见同龄孩子的好动,稳重得像是成年人。大清早起来,可也看不出贪睡的困倦。

放下了一份来自江州的奏章,向皇后看儿子坐着一动不动,关心的问道:“六哥,要不要吃点果子。”

“娘娘,孩儿不饿。”赵佣先站起身,然后端端正正的行礼回话,“娘娘可是累了?”

“娘娘不累。”向皇后笑了,“坐着吧。”

赵佣又是行了礼,然后才坐了下来。

开蒙就学才不过半年,就有了很大的长进。说话、举止更加稳重。宫里宫外见了,都觉得有这样的一个好学守礼又聪慧的太子,大宋的未来是不用担心太多了。

小学生的学业不求他能作诗作赋,跟白居易那样六个月能识‘之、无’的天才比,最重要的是礼节的学习。但凡儒者,礼这一项都是必修的科目。

东宫的师傅保的数量不少,资善堂内的老师更多,不过最主要的还是王、韩、程三人。其中韩冈远在河东,王安石多忙于政务,其实还是程颢给赵佣更多的教导。向皇后一向对程颢看不顺眼,但也不得不承认程颢在教书育人上的确是水平很高。

想起赵佣的老师,向皇后就不免联想起东宫名义上的师傅们。

之前因为冬至曰的剧变,东宫三师给王安石、司马光和吕公著占去了,但太子少师,太子少傅,太子少保这东宫三少并没有任命。这一回,倒是要封出去了。虽说是东宫,可是此番跟赵佣没半点关系,只是酬奖功臣,给予吕惠卿、韩冈和郭逵三位主帅的奖励。

政事堂的想法是给郭逵一个节度使加太子少保的头衔,然后让他养老去。除此之外,给吕惠卿一个太子少傅,算是筹奖功劳。韩冈则是太子少师。

基本上都是虚衔,无非是加官进爵的那一套,而且还是很简吝的那一种。

十九级的检校官,十二转的勋阶,好听而已,东府毫不吝啬。可到了有点实质的封爵立刻就小气了起来。

吕惠卿还能晋封郡公,但韩冈和郭逵爵位却没变。

韩冈、郭逵两人历年来积攒军功,爵位都已经是开国郡公,再升就是国公。眼下朝堂上,除了王安石,就连韩绛都没有得封国公——尽管只要担任过宰相,终究都会得授国公,且最近有说法要给他晋封,但毕竟现在还没有。

郭逵且不说。韩冈的军功虽高,晋封国公则未免过早,且吕惠卿也都没升到郡公。论功劳,两者相当,论资历、差遣,则吕惠卿还在韩冈之上,总不能厚此薄彼。

在向皇后看来,政事堂为此找了一通理由实在是煞费苦心了。

而在增添食邑上,东府也表现得很吝啬。

世间都说万户侯,但三人功绩如此,都没一个食邑过万户。

东府给出的理由是依故事食邑万户则封国公,三人既然不是国公,当然不能受万户食邑。原本韩冈是食邑四千户,食实封一千两百户,现在只是加赠四千食邑,食实封一千八百户。总计八千食邑,三千实封。

至于韩冈、吕惠卿是否回京,那是一如既往。边关人心未定,需要两人继续坐镇,倒是郭逵,则是越快调回越好。陕西种谔,也跟郭逵差不多,擢升节度留后已经定下来了,待兴灵稍定,便将他召回京中就任三衙管军。

这些天来,向皇后对在京宰辅的感观越来越差。

这一回与辽国大战的结果,不说要强过两次惨败的太宗,比起真宗皇帝也要强了不知多少。纵然岁币依然要给,可夺回了多少土地,这都是太祖皇帝用钱买不回来的。

此番功成,实可往太庙夸耀一番。可宰辅们有志一同,怕庆贺的声势大了,会惊动到福宁宫中的官家,硬是不让照常例来。

不想谎言被拆穿,向皇后也无法反对宰辅们的意见。

一场本该是太宗之后对辽国第一场扬眉吐气的辉煌大捷,就这么在朝堂上无声无息的给压了下去,弄得好像是输了一般。

还有京营,两府的吝啬闹起了多大事,要是能省下来倒也罢了,可到了最后还是给了多少钱才压下来。

难怪官家总是要换人来管两府,这些宰辅留在朝中久了就是祸害!

“圣人。圣人。”

宋用臣的唤声,惊醒了沉思中的皇后。

‘时候到了吗?’

向皇后站起身,牵着赵佣的手走向前殿。面对重复又重复毫无变化的朝会,心中再无波动。

朝会一如既往的乏善可陈,有野心的官员们依然还在观望。之后的崇政殿再坐,与会的重臣们又刻意避过了对韩冈奏疏的议论,没有给向皇后半点开口的机会。

帘幕之后的皇后对宰辅们的行径都麻木了。

这些人能把皇帝逼得只能躲在宫里生气,她一妇道人家如何是其对手?

当王安石领着众宰辅准备离开崇政殿,去往福宁殿问安的时候,一个声音打破了殿庭的死气沉沉。

“臣有本奏于殿下,请留对。”曾布冲着帘幕后的身影躬身行礼。

向皇后猛的直起身,王安石脸色陡然一变,与会重臣神色各异,但都是以惊异为多。

曾布!

幽沉的殿阁下,曾布矮小枯瘦的身影,却仿佛平坦如水的通衢大道路面上凸起的石块,让人觉得分外的扎眼。

曾布竟然自请留对……

自从仁宗初年的权相丁谓,被同僚王曾用此法请去了琼州之后,这样的行为已经成了宰辅中最为忌讳的行为。而反过来想,一名宰辅会请留独对也绝非小事,必然是要开罪一大批同僚的动议,否则没有必要选择如此激烈的手段。

他要做第一个吗?!

曾布脸上看不出有半点紧张。

当初他上表声言市易法之弊,被王安石赶出京城,曾公亮曾以书简一封相送——塞翁失马,今未足悲,楚相断蛇【注一】,后必有福。

如今正是验证的时候。

注一:贾谊《新书·春秋》载,春秋楚相孙叔敖,幼时遇两头蛇,恐他人又见,埋之,惧,谓其母曰:“吾闻见两头蛇者死。”“母曰:‘无忧,汝不死。吾闻之,有阴德者,天报以福。’人闻之,皆谕其能仁也。及为令尹,未治而国人信之。”

