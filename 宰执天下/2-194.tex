\section{第32章 吴钩终用笑冯唐(一)}

【今天事多,第一更迟了一点,下一更会在十二点前完成。明天应该能照正常情况,下午更新第一章。夜里第二章。】

两千多枚首级,一层层、一摞摞的叠放在绥德城衙的广场上。一面面党项人的战旗、鼓号、兵甲等战利品,也都整整齐齐的摆放在旁边。

出战大军凯旋回师后,炫兵耀武都是常例。将战利品摆出来,让所有人都看看自己的功劳,没有哪家得胜回返的大军会不愿意。

张玉和高永能就得意地站在城衙的正门前,身后簇拥着一群跟着他们从罗兀城回返的将校。

他们脚前的战利品,有龇牙咧嘴,有闭眼闭口的,还有只剩半边天灵盖、红的白的混合着凝在脸上变成紫色的。这些足球大小的玩意儿,都被盐码过,防着腐烂。但还是有股恶臭,引来了一只只苍蝇嗡嗡的围着绕着。

其实早在昨日,就在宋夏两军还在北方激战的时候,从城外的无定河中,冲到岸边上的西贼尸首连着战马,便是一具接着一具,单是收拾起来的就是四五十了。想想浸在河里面飘下去的,三五百总是有的。张玉、高永能派出的报捷信使还没到绥德,城中军民就已经知道,前线肯定又是一次大捷。

这是大宋与党项人作战的历史上,几十年都见不到的战果。绝非此前的一系列所谓的大捷,都仅是几百几十的数量。眼下的两千多,单是十几个一堆的放在一起,就是黑压压的一片。

引来参观的人们成百上千,一时间人山人海,观者如堵。城中的军民但有得闲的,便都赶着在这些战功送往延州之前,来看上一眼。

围观的军民中鼓噪声不绝于耳,不时的有人高声喝彩。也有人消息灵通一些,知道这只是战术上的胜利,辛辛苦苦建起来的罗兀城终究还是被放弃了,从今以后,绥德又将是前线。

“怎么就退军了呢?杀了这么多西贼啊?两千多斩首,算起来西贼少说也要伤亡上万,哪里还能再围着罗兀城?!”

“广锐军作反,不得不回来啊。”

“左不过是三千贼人,把罗兀城放弃做什么?俺辛辛苦苦的担了一个月的土,现在全成了白干了。”

“就是朝廷派来的赵郎中乱来,又不懂兵事,还乱发令箭。要不然,西贼都被打得屁滚尿流,怎么还要放弃罗兀城?”

“韩相公不是宰相吗?怎么就任着一个郎中乱来?”

“赵郎中可是奉了官家的命,韩相公难道还能为抗圣旨不成?。”

“蒙蔽圣聪,天下的事都是这些奸臣坏的。”

下面细细碎碎的讨论声传入耳中,种谔恨不得提刀杀人的眼神,狠狠地盯着一堆堆叠在自己眼前的人头,眼中看到的却是赵瞻那张盛气凌人的脸。

要不赵瞻那厮逼迫,今次罗兀之战当是能飞捷京中,哪里会闹到鸡飞蛋打的地步。

张玉和高永能今次的表现,足以证明他们能稳稳地将罗兀城守住,而这么多的斩首,也证明西贼无力与大宋拮抗。只要能坚持着把剿灭下去,横山就已经是大宋的囊中之物了。哪像现在,降官肯定少不了,撤职编管也不是不可能。而最让他愤恨的,就是多年的心血一朝尽丧。

相对于种谔,张玉和高永能他们的心情就轻松了许多。

斩获的两千三百多枚西贼首级全都亮了出来,还有以嵬名济为首的十几名身份更高的将校,加之都罗马尾的死信已经得到了确认,今次罗兀攻防战虽以宋军撤离而告终,可板子怎么都打不到罗兀城众将的身上,而功劳也绝不会少。

细浮图城那里也有了消息,折继世在收到张、高二人的通知后,于西贼可能利用其来抄截的道路上,设下了伏兵,又很顺利的等到了奉梁乙埋之命赶往抚宁废堡的结明爱和旺莽额两军。被称为将种的折可适领军冲杀于阵前,亲手用长枪跳下了敌军大将结明爱的长子,立下了大功。

只有种谔一人失落不已,毫无功勋可言。当他依言遣兵北上接应,见到的却是得胜而归的大军,并没有赶上激战。因为环庆副总管张玉亦在罗兀城中、又参与了全程战事的缘故,同为一路副总管的种谔甚至连借着部属高永能的光,从中分润一笔功劳的机会都没有——枢密院要评判此战的指挥之功,只会算到张玉的头上,而不是给远在绥德的种谔。

张玉回头看了看身后众将,突然发现少了那个让他很欣赏的高个子年轻人,“韩玉昆呢?”他问道。

“韩玉昆去绥德城里的疗养院,安顿今次的伤病了。”一名幕僚回答着张玉的疑问。

兵凶战危,这一次撤离罗兀城的行动,虽是宋军在战场上一直保持着优势,但照样还是有了四五百人的伤亡。幸好直接战死的并不算多,而受伤的又得到了及时的救治,绝大多数都能保住性命。

张玉听了,笑赞了一句:“韩玉昆做事还这么勤快!”

高永能也道:“罗兀之事也多亏了有他。”

在旁听到的众将一并点头。正常情况下,最为艰难的撤军行动竟然如此顺利,而且还能一举击败追兵。这其中韩冈功不可没,众人都看在眼里。

炫耀过了今次的战绩,摆放在衙门外的战利品就被收拾了起来。这些东西过两日还要送去延州,让宣抚司来点验。接下来,照常理就该是庆功宴了,但今次一战,明胜实败,唯一值得庆幸的是军心未损。种谔亦是无心于庆功宴。而自罗兀城回来的这些将领也没有当着种谔的面庆贺的意思。

家室在绥德的便归家团聚,来自外地的,则各自找地方私下里庆祝,韩冈从疗养院回来报个到,也回自己的住处去了。

周南早早的就得到了消息,一直坐立不安的在等着。终于等到变得黑瘦许多的韩冈站到眼前,她差点就要哭出来。在韩冈面前虽然是在笑着,但几次背转身,用手背擦着眼角。

韩冈把周南拉到身边抱着,觉得她本就是轻盈的身子,现在变得更加弱不胜衣。虽然怀里的绝色佳人,就算是在最憔悴的时候,依然有种病恹恹的媚态。可韩冈还是心疼不已,不意周南用情如此之深,才半个多月不见,就已经快熬得病倒了。

不像第一次去罗兀城,那时的罗兀虽是前线,可只是在筑城而已,并没有多少危险。不过这一次韩冈去罗兀,那是当真被围了城。八万西贼大军在国相梁乙埋的带领下,甚至还打下了抚宁堡,断掉了罗兀城的退路。

有那么几天,绥德城中到处在传着罗兀城已经陷落、城中诸将皆尽殉国的消息,周南差点都疯了,拼命的让钱明亮出去打探,三天三夜都是茶饭不思,直到阵斩西夏都枢密都罗马尾、并斩首千人的捷报传来。可是在韩冈回来之前,她都是恍恍惚惚的,要不是墨文逼着,都不记得要吃饭。

热恋中的少女尽心的服侍着久违的情郎,饭才吃到一半,便给韩冈拉到了床上。墨文端着刚做好的酒菜进来,就看到纠缠在一起的两具身躯和抛了一地的衣物,红着脸忙跑了出去。

云收雨歇。周南被折腾得再无半丝气力,汗湿的发丝弯弯曲曲的如蛇一般的贴在雪白的背上,整个人也是软绵绵的趴在韩冈的胸口,娇声喘息着。而韩冈方才也是一番辛苦,加之多少天来的奔波劳累,却也是一时间没有多少气力再来一次。

抚摸着周南腻滑得一如最上等的瓷器一般的肌肤,韩冈享受着难得的宁静时光。半晌之后,他才吞吞吐吐的开口:“明天还要去延州……实在是不想去,但罗兀城的事是完了,可在宣抚司中的事还没有个了局……只要……就带着你回古渭。”

韩冈说得有些絮絮叨叨的,因为罗兀城已经弃守,细浮图城那里也不再需要囤积上万大军,很快就将恢复到正常的驻军数量。不过种谔没法儿在绥德等折继世回来,他君命在身,高永能和张玉一至绥德,第二天他就要亲领一军前往延州报道。而韩冈也被通知,明天要一同出发。

周南沉默着,就像是睡着了一般。但过了一阵,她突然出声:“……官人……”

“什么?!”

“给奴奴一个孩子吧!”周南将脸贴在韩冈的心口上,呢喃的说着。

只要在韩冈身边,她的心中就是充盈的。但韩冈一旦不在,心头又会变得空落落的一片,总是在发呆,要么就是听到一点谣言便惊慌不已。如果普通人家,也许能常伴左右。但韩冈是官人,一封诏书、一份官诰,就会离家远去万里之外任官。如果这时身边能有个孩子,心也许就能安稳了下来了吧?!

怀中佳人微微的颤抖直接通过紧紧相贴的肌肤,传到了韩冈的身上。

“傻丫头!一个哪里够?五个六个都不嫌多!”韩冈一下子翻过身来,又把这具让他沉迷不已的动人娇躯压到身下。双手抓着纤细的脚踝,将她修长柔韧的双腿向上一直推到紧紧压住胸前的两团丰盈。他用力杵了下去,笑着:“就让官人现在帮你完愿好了!”

周南双手向上搂住韩冈的脖子,不顾一切的逢迎着,“官人要奴奴生几个就生几个。”

一夜绻缱之后,韩冈在周南的婆娑泪眼中离开了临时的住所,随军前往延州。等待他在陕西宣抚司的使命,彻底的有个了结。

