\section{第36章 万众袭远似火焚(三)}

河州城中,也已经有春风吹过。

从门外吹进来的风带着雪化时的湿寒,但比不上站在木征面前的这位年轻的吐蕃贵族,带给周围人的寒冷。

青谊结鬼章。

鹰钩鼻子,略细的眼睛,败坏了他端正的相貌。一眼看过去,就是一个危险的人物。

青谊结鬼章是鬼章部的新任族长,只有三十岁不到。看到他,木征就想起了同样年轻的禹臧花麻。不过禹臧花麻给人的感觉更为狡诈一些,他借着木征给他的许可,把武胜军【熙州】北部抢掠一空,直接回到兰州去。虽然有着共同出兵的承诺,木征并不知道他能履行多少。

鬼章部位于木征的河州和青唐王城之间,黄河的南岸。算是个大部族,只尊奉青唐王城的命令,而无视更近一点的木征。今次青谊结鬼章带来的援军,也并不完全是他本族的士兵,有一半——而且是装备更为完善的一半——是由董毡交给他的。

木征没想到董毡派来的援军主帅,会是鬼章部的族长。年轻不是问题,气焰太盛才是让木征头疼不已的一桩事。

“河州山高林密,宋人肯定走不惯。等他们从临洮一路走到河州城,早就没有力气了。”无论是木征还是青谊结鬼章,都是坚持叫着武胜军和临洮,而不是宋人改名后的熙州、狄道,这是他们的一点自尊心,虽然于事无补,“我们坚壁清野在河州城下等着宋人过来,趁他们疲惫不堪的时候,就全军出动,杀光这群宋人,还可以一举收复武胜军!”

‘要是有这么简单就好了。’木征想着,只看臣服宋人的青唐部在武胜军烧杀抢掠的手段,坚壁清野的策略根本就不可能管用。

可他并无意提醒青谊结鬼章,年轻人就该摔打摔打。如果青谊结鬼章的失败,能换来董毡对宋军的重视,木征很乐意把青谊结鬼章的队伍,送到王韶手上。

无视掉青谊结鬼章狂妄自大的意见,木征对即将面临的决战,有着自己的一番考量。

正面难以相抗的情况下,除了抄截粮道,别无他法。如果能让青谊结鬼章在前面吸引宋人的注意力,他就可以率领主力绕道宋军背后。

引得宋军深入河州,然后出兵断绝他们后路,这是最为简单易行的策略——最重要的是有效。

不需要地图、沙盘,河州、洮西的山山水水都准确的映在木征的头脑中。他熟悉河州的一山一水,熟悉河州的一草一木,山中的部族都遵从他的分派,占着地利与人和,他绝不会像偷袭渭源堡的两个兄弟那般失败。

从宋人占据的武胜军【熙州】通往河州的道路上,适合成为宋军葬身之地的地方,木征想来想去,就只有两处,

“是香子城,还是珂诺堡?”

……………………

简单的接风宴后,景思立被王韶的儿子领进了白虎节堂之中。

熙河路的帅府中枢,不如秦凤路的高大,但也是一般的肃杀。与秦凤经略司的白虎节堂另一个相同之处,就是在正堂中,同样摆着一幅巨大的沙盘。

沙盘周围,是同样参加了接风宴的王韶、高遵裕、韩冈等经略司中的高官。只是多了一个景思立没有见过的和尚,高而瘦,有着风吹日晒而出的粗糙黝黑的肌肤,像是一个托钵的苦行僧。但他竟然是身穿紫衣,这一点就不是任何一个苦行僧所能拥有。

“这位是智缘上师。”韩冈为景思立介绍道。

“阿弥陀佛,贫僧见过景都监。”比两年前,黑瘦了许多的智缘口宣佛号,向景思立合十行礼

“原来真的是上师。弟子失礼了。”景思立连忙还礼。

老和尚穿着的御赐紫衣,秦凤一带的独一份。景思立曾听说过智缘的传闻,韩冈还没介绍,其实就已经隐隐约约的猜测了出来。

号称诊脉便能断人休咎,在东京城中都是让王公大臣趋之若鹜的高僧。到了关西这偏僻之地,得到的尊敬自然更多。对于佛教,景思立说不上信与不信,该烧香时烧香,该拜佛时拜佛,却不会把阿弥陀佛挂在嘴边。但智缘这两年的一番作为,证明了他的能力,也证明了他的名声不是平白得来,让景思立对他保持着一定的敬意。

只听韩冈继续说着:“这副以河州、熙州为主的沙盘,也多亏了智缘上师这两年来的一番辛劳,探查各处蕃部虚实。”

智缘又念了一声佛号,“宣讲佛法,普渡众生,并不算劳苦。”

智缘自从前年来到王韶帐下,便被他派出去宣扬佛法。拥有佛陀护持,智缘走遍河湟都不用担心自己的安全。就算落到木征、董毡的手中,他们能做的也不过是软禁而已。吐蕃人对浮屠的信仰可以说是沉迷,智缘靠着他的口才和医术,以及宋僧远超蕃僧的佛学水准,在河湟蕃部,结下的善果甚多。他的名声也已经是不逊于王韶、韩冈的响亮。

当然智缘还是有敌人,那些蕃僧肯定是恨不得杀掉让他们出乖露丑的对手。王韶之所以会向天子要求一名高僧大德,就是因为要与蕃僧打擂台的缘故。

智缘是见过天子的人,英宗皇帝重病时,作为京城中有数的名医曾被召入宫中,还因此被司马光指名道姓的在奏章中抨击过。正经儒臣对僧人的厌恶世人皆知,司马光的奏章等于是助长了智缘的名气。僧人就跟名妓一样,名气越大,人望越高,司马光帮了他的大忙。

但智缘他来到关西后,历经千辛万苦,走遍千山万水,不仅仅是为了一点名气,而是希望能更进一步的留名青史。他兼通儒释,在儒学上,水平并不比一般的贡生秀才要差。普渡众生的要旨,智缘看得很淡,他的性格更近于儒者,对流芳百世的渴求,远超普通的僧人。

与智缘见礼过后,景思立便专注于沙盘之上。通过智缘携回的地图,以及这几年所搜集的地理情报,所制作而成的这具沙盘,虽说不上多完备,也比不上巩州、熙州的沙盘精确,可用来确定进军路线,也勉强够了。

“从狄道往河州去。近三百里路,途径关隘、寨堡多处。上上之策是一鼓作气的将之拔取。一旦中间有所阻碍,耽搁上一天,就是上千石的粮秣消耗。而攻城拔寨并不难,难得是如何铲除木征的势力。木征是赞普血裔,在河州根深蒂固。不论是将之收服,还是将之击灭,都不是一桩容易的事。”

韩冈的话,引来了景思立提议:“最好能设法引得他出来决战。”

“就算决战都难以将他留下来。”

除了智缘之外,在列的都是上多了战场,皆知任何一场会战中,就算能取得再大的胜利,要想除掉敌方的主帅,都是千难万难。除非木征不跑,头脑发昏的准备硬拼到底,又或是官军打得他无处立足,上天无路、入地无门,逼不得已而投降。否则,都很难把他彻底解决。

“……瞎吴叱、结吴延征也算是个例子吧?”景思立又道。

“那是运气,不足为例。”这话别人说不得,只有韩冈自己说才没问题。

“那就得看木征会不会自己主动来攻。”景思立已经看出了这番对话,是王韶来测试自己的水平,也便抖擞精神,说着自己的看法,“攻打我军的后路。”

高遵裕不屑的冷哼道:“坚壁清野,诱敌深入,然后断敌归路。木征能用的手段也只剩这一条了。”

这是熙河经略司上下共同的认识,但这个认识是取决于正面战场上的官军,能否让木征不敢面对面的全力交战。如果决战的兵力不足,木征可以从容的吃掉出战的官军,然后再向后阵扑来。

今次出战总共有三万兵马,还有一干自带干粮的蕃军,加上成千上万的民伕。人数虽众,排得上用场的却很少。可后方的守备却是少不了,不论是熙州还是巩州,可能受到兰州的攻击——而且不一定会是禹臧家,党项人这时候很有可能会出手——太过绵长的战线,需要足够的兵力来保护。

兵站制度在去年的临洮会战中,有着显著的功效,当然会沿用下去。只是其中要占用的兵力,却绝不会少。而北面的禹臧花麻还要加紧防备,以防不测。

真正能上阵作战的主力,最多也只有两万人马。

可无论是给两万还是三万人马准备粮秣,带给后勤体系的压力一样很大。必然需要可靠的官员来主持随军转运之事。韩冈可以确定自己的必然是随军转运使之一,另外一个又会是谁?

韩冈希望是蔡曚,那个蠢货之所以还能坐在转运判官的位置上,就是因为王韶和韩冈都不想换个更聪明的过来,而在临洮会战结束后,没有向朝廷汇报蔡曚在拖后腿。

还有,又有谁能阻止想要前来分功的官员们?别说官员,王韶和高遵裕的府中,现在都挤满了不知从哪里来的亲朋好友,都是想在军中挂个名号,在军功簿上分上一杯羹,让他们不胜其扰。

不过现在也没必要考虑这么多了。

“兵械皆备,粮草已足,差不多已经可以出兵了。”韩冈看了看景思立,“景都监已经到了,就不用担心巩州、熙州的安全。”

景思立惊道:“泾原路的军队还没到啊?!”

“兵贵出奇。早就准备好了,何须等待全师齐集?”王韶的意见就是经略司的命令,“……宥之,你军远来,兵困马疲。先在陇西歇息两日后,再全军前往狄道。”

