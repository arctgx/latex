\section{第四章 岂料虎啸返山陵(二)}

【还有一更下午补上】

将前任宰相以事涉谋反的名义——说难听点就是待罪之囚——召回来询问。就算是所有对王安石恨之入骨的旧党大臣,他们在最疯狂的梦里都没敢这么想过。

他们穷究李士宁涉及谋反一事,的确是准备敲着边鼓,迂回前进,将王安石的两个儿子拖进来后,最后逼王安石接下一个教子无方的罪名。但怎么会有人敢将王安石直接牵扯进来,而且说这话的还是韩冈——王安石的亲女婿!

即便王安石已经离开相位,但他一手推动并主持的变法事业,直到现在也还是国家施政方略的主流。对于当今的天子,王安石是如师如友。尽管现如今,王安石的圣眷已远不如当初,可是尊重和信赖还是有的。隔三差五,赵顼也还会派亲近的内侍带着礼物去江宁探问,可见他对王安石的宠信不衰。

赵顼疑惑的盯着韩冈,这是他的激愤之语吗?

如果韩冈现在的表情是愤怒,方才的说话是咆哮,正在殿中的几名御史,就可以送他一个君前失仪的罪名。可韩冈平静得如同井中水月,一点情绪波动都没有。

“招王安石入京询问?”赵顼心头疑云丛生,一个字一个字的问着韩冈。

韩冈即刻回道:“此案久拖不决,牵连甚广,又事涉宰相之子,自当将其父招入京来。”

天子与韩冈的一问一答,顿时就让所有人都明白过来了。韩冈说得这几句话,用意只在最后一句——

他是要请王安石回来!

他是要请去位出外的王安石回来!

……这如何使得!!!

同样的反应出现在不同的阵营中。

一旦王安石时隔半年多重新见到天子,那情况会变得怎么样,在殿上的一干重臣,都能推测得出来。最近这段时间,他们的表现实在太差了,天子不可能不怀念的过去王安石为相之时的朝局。

旧党中人当然恨不得王安石在江宁养老,从现在的五十三四,一直养到死为止。以王安石的声望,一回来就必然轻而易举的控制住朝局,到时候他们就又要过上在泰山底下生活的日子了。

不过新党中人应该是欢喜的。吕惠卿用了半年的时间没能解决掉旧党,如今事事受到朝堂上的掣肘,已经让底层的新党官员感到不耐烦了。身居高位的几个核心,身上的压力都很大,王安石能回来,对他们都是好事——只不过,应当将一人给排出。

王珪脚动了动,但他还是忍了下来。冯京今天不在,他没打算强出头。瞅着对面的吴充,盼着枢密使能站出来。可这时一道人影从王珪眼角闪过,定睛看过去时,竟是吕惠卿跨出班列。

来自福建的参知政事立于大殿中央,对着天子朗声道:“陛下!臣以身家性命作保,王旁必无涉此案!”他要保着王安石这面旗帜,却不想这面旗帜重新在政事堂中飘扬起来,新党的中军大纛只能有一面,就是他吕惠卿。

韩冈惊异的看了吕惠卿一眼,他没想到吕惠卿竟是第一个跳出来的。以他的判断,吕吉甫再怎么不想看到王安石回京,至少也该稍稍犹豫一下,排在第三、第四号出场才是。

吕惠卿义正辞严,从他的表面上根本看不出来他是为了阻止王安石回京,才如此卖力的为王旁争辩,“王旁自少承袭父兄之教,行事谨严,虽与李士宁相往来,但只是泛泛之交,绝不至涉及奸谋!”

吕惠卿站出来说话,但章惇、曾孝宽却是犹疑着,一时不知道该不该站出来。他们当然希望王安石这面旗帜回京,但现在不站在吕惠卿一边,可就是明摆着要分裂了。

章惇正犹豫间,韩冈冷澈的眼神已经瞥了过来。除了面朝天子的几位,站在殿尾中央的韩冈可以将殿上任何人的神态看在眼里,当然也包括章惇的。

章惇知道,这是选择站队的时候了。今天殿上的争议不可能隐瞒起来,吕惠卿的私心也瞒不了明眼人,若是自己选择错误,就是彻底的开罪了王安石。而且韩冈的这番话,究竟是不是秉持了王安石的心意,章惇他也无法确定。

在王安石和吕惠卿之间的犹豫只有一瞬,章惇也同样走出班列,转身对着天子:“臣亦愿以阖族性命作保,王旁与谋反一案绝无瓜葛。但李士宁即涉谋反,就必须就此查个水落石出,还王安石父子一个清白。”

站在前面的吕惠卿闻言身子猛然一震,背后传来的声音,让他只觉得双脚站立之处仿佛是虚悬在万丈深渊之上,空空荡荡,让他无处可以着力。

章惇的背影映在韩冈的眼中,在唇角边得到了一丝欣慰的笑容,看来他的这位友人已经明白了过来。得到章惇的支持,京城中的新党成员,就不再只能听着吕惠卿的命令,而是有了更为恰当的选择。

吕惠卿心中焦躁无比,邓绾该出来说话了,但御史中丞所在的方向却是没有任何声音传来。

出来的是蔡确,“陛下,此案事关重大,确当根究。”

吕惠卿的心冷了下去,现在他都只能盼着吴充出来。

吴充他当然也要阻止王安石上京,只是他一时之间不知该如何是好。若说追究下去,王安石肯定是要进京了。若说不追究王旁涉案,他前面的话还摆在那里,站在近殿门出的那个灌园小儿,可正等他反口呢。

韩冈冷眼看着殿上的一团乱象,差点忍不住要大笑出声,实在是太可笑了。王安石还没回来,就让殿上乱成了受了惊的猴山一般,要是当真回来了,又会是什么样的情况?

都是韩冈的一句话造成的。王安石能不能上京,韩冈不敢保证。但他直接掀桌的行为,却能让殿上的所有人无法应对。从今天开始,朝堂上的政争就可以歇一歇了。而且有一件事,所有人应该都明白,外任的臣子是可以上书自请入京诣阙的。

殿中的臣子各自上台表演,可就是天子的态度耐人寻味。

不论臣子们在说什么,赵顼都是一言不发,始终不肯给个回音。一直到了退朝的时候,他都没有为今日殿上的争辩做出评判。

净鞭响起,内侍尖着嗓门唤着退朝,但韩绛却没有动。本应领着群臣恭送天子的首相,直截了当对天子道:“陛下,臣有一事需奏禀,今日请留对。”

一直以来在政事堂中存在感稀薄的韩绛,被冯京和吕惠卿逼得成了泥胎塑像的韩绛,竟然在这个时候开口,连韩冈都愣住了。

“准!”赵顼开了金口,只吐出了一个字。但这个字却如黄钟大吕,在每一位臣子的心中掀起惊涛骇浪。

韩绛再怎么势力单薄,他都是宰相!说话的份量,怎么都不会太轻。尤其他现在自请留对,绝不会是为了说些家常话!

天子在议事后留臣子下来很是常见,王安石甚至曾经有很长一段时间,每次议事之后都能被天子留下来说话。但臣子自请留对不同,自丁谓之后,很少有人这么做了。

丁谓是真宗末年的权相。其人把持朝政,铲除异己,甚至陷害曾经举荐过自己的寇准,世人目之为奸相,却无人能奈何得了。但就是这名机敏多智,奸狡过人的宰相,却被人给用计谋害了。与丁谓同时为相的王曾,以过继儿子为借口,征得了丁谓的同意,自请留对。趁此良机与不满丁谓已久的章献刘太后联合起来,将其赶出了京城,贬去了琼州。

王曾若不能自请留对,章献太后刘娥就不能知道政事堂中哪位是丁谓的反对者,投鼠忌器下,也不能下手对付丁谓。而当王曾主动留了下来,两人便是一拍即合,让丁谓败得不明不白。

自此之后,重臣自请留对就少不了会有构陷同列的嫌疑,成为了朝堂上的一大忌讳。沉默已久的韩绛选在今天自请留对,让许多人心中都腾起不好的预感。

韩冈望着吕惠卿和吴充煞白的脸,在猜测韩绛用心的同时,也不免暗暗一笑,恐怕外面没人能想到,这两位还有心灵相通的时候。

出宫后,韩冈回了军器监处理日常事务。等到下午,看看时候差不多了,就派人去与章惇联络。既然章惇已经表明了支持王安石回京的态度,就有必要跟他通个气。不论王安石能不能回京,章惇今天的表态,已经证明了他并没有彻底的站在吕惠卿的一方,而是依然拥护王安石。

只是等到韩冈回到家中,派去找章惇的家人才回来对他说:“章学士刚刚回家,就被招入了宫中去了。”

正在旁边帮着韩冈更衣的王旖,惊讶的停了手:“今天出了何事,要锁院了?!”

又不是逢年过节,翰林学士连夜入宫,除了封锁学士院,为天子撰写重要的诏书,也不会有别的事了。就算是妇道人家的王旖,也有这个见识。

韩冈一拍桌:“是韩绛!韩绛果然荐了岳父。”

“官人?”韩冈没头没脑的话,让王旖听得一头雾水。

“韩子华今日自请留对,当是为了推岳父复相。”韩冈说着,猛然间哈哈大笑起来,他本来只求震慑一干政敌,顺便引动天子重新启用王安石的心思。哪里想到竟引来了韩绛的动作,让局势一下激变。这一个结果,却比他设想得还要有意思,“山中无老虎,猴子称霸王。这下老虎要回山了,看看又有那只猴子还能跳得欢?!”

