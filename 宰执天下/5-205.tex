\section{第23章 弭患销祸知何补(13)}

“玉昆。”身后传来苏颂的声音。

韩冈停步回身,准备打招呼的时候却被苏颂满是血丝的双眼吓了一跳,“子容兄,该不会一夜没阖眼?”

苏颂倦容满面,却还是在微笑:“荧惑大冲,十五六七年才能碰上一次,怎么能放过?以我这年纪,很难有下一回了。”

“就算不是大冲,用千里镜看火星,都远比过去要清晰得多,何必如此劳神?”韩冈摇了摇头,难以认同。正如苏颂自己所说,他的年纪可也是不小了,撑不住这样忙碌的熬夜生活。

“等过了这几日再说。火星大冲能多看一日便是一日。”苏颂笑说道,“再过些日子,等到了岁星、镇星冲日\footnote{大冲和冲日的定义是配合现代天文学的理论,应该不是古有的词汇,或为古词重新定义。眼下查到的资料,是出自于清代的《历象考成》,而这本的本源,却是丹麦天文学家第谷。}的时候,还要多看一看。”

“这样未免也太辛苦了。有些事可以让下面的人去做。”韩冈一贯是将手上的事尽量安排给下面的人,自己掌住舵就可以了。

“这般辛苦也是没办法。”苏颂无奈,“玉昆你也是知道的,司天监中人浮于事,勾心斗角的本事一个胜过一个,提起历算来却无一人可用。光将步天歌背熟了又能如何?”

“不如上请天子,另设天文历算局好了。”

“我也早有此意,正在寻找人选。气学门下贤人甚多,玉昆你不妨多推荐几个。”苏颂看了韩冈一眼,“玉昆你在天文星象自出一格,更胜世人,其实玉昆你才是最好的人选。”

“我对天文也是不甚了了,步天歌可都没背熟。”韩冈轻声笑了笑,这不是自谦,他的确是彻头彻尾的外行人,“但也是如此,才能从窠臼中跳得出来。有时候,往往外行人看得更清楚一点。不识庐山真面目啊,过去在外用兵的时候,倒是有不少次体会到了……不过若子容兄有何差遣,只管吩咐就是了,韩冈俯首恭听。”

苏颂闻言点了点头,“那我可就不客气了。”

苏颂本来也是当世数一数二的天文学大家,韩冈的理论却是正好印证了他往年观测天象后产生的疑问。他观察天文几十年,对天体运转的观测结果和古上的差异一直抱着深深的疑惑。而月绕地、地绕日,五行星与地球并列绕日而行的理论简单直接,却远比通行于世的理论更加贴近事实,加上对一些天文现象的重新定义、定名,等于是捅破了一层窗户纸,让他的眼前陡然间一片光明。

只是剩下的几分疑惑,依然需要大量的观测来释疑,让苏颂没有立刻将韩冈的理论全盘接受下来。与韩冈的讨论之后,连着多日苏颂都埋首于钦天监历年的观测资料中,甚至面禀天子,要制作性能更加优良的望远镜,并改进旧时仪象。

韩冈对此倒也不介意。是非与否,一切取决于观测,理论只有被现实所映证,才能证明其正确性。不唯上,不唯,只唯实。

这就是格物致知,是气学的根本,如今苏颂正在做的,正是韩冈所希望看到的。

当然,这个观测的结果很可能将会是颠覆性的,乃至于浑天仪、浑象仪,都得重新设计,恐怕只有日晷才能留存下来。

韩冈和苏颂并肩走着说了一段话,又有人上来与两人打招呼,却是章惇。

枢密副使主动问候,韩冈和苏颂都立刻回了一礼,不过苏颂和章惇算不上有交情,冲韩冈点头示意了一下,便走开去问候其他人。

文德门就在前方,门前也有御史和阁门使检视入朝的文武百官。

“真真是好笑。”章惇眯着眼晴盯着几名御史,低声说着,“明明都已经丢人现眼了,也亏他们还能厚着脸皮站在文德门前。”

韩冈也低声冷笑:“早就说了,如今的御史台是一代不如一代。”

蹴鞠联赛上的惨剧被当成意外放过之后,反应最为诡异的就是御史台。随着赵顼作出决定,他们立刻就偃旗息鼓了。这让不少人都有着跟章惇和韩冈一样的感觉。若是几十年前,御史台中人只会死咬到底,越不给皇帝面子,就是越有面子,哪里会退得这么干脆?

在过去,多少重臣都是因为御史们穷追猛打,让天子烦不胜烦,最后不得不饮恨出外避一避风头。弹章交加而上——这‘交加’二字,用得最多的就是在御史们的身上。

韩冈本以为台谏官们还会再闹腾个半个月,让天子将御史台中再清洗一遍,谁想到就这么了无声息了,还真是让人始料不及,“朝廷选拔御史不问资望,甚至可以选拔资历浅薄之人,本就是看在他们为官未久,未为世俗所染,希望此辈能不惧天威、不畏权势,放胆直谏。但现在的御史台,离朝廷用人的初衷是越来越远了。”

“玉昆你是希望他们多弹劾你几次?”章惇轻笑。

想要沽取直名,也得看看后果。韩冈已经被确定是未来的帝师,御史们可以与他划清界限,有事没事弹劾几次,但当真与他结下深仇大怨,一二十年后新帝登基,可能有好果子吃?

随即他又收敛了笑容,“不过日后若是天子有过,想来他们也不敢站出来谏阻。”

章惇顺着韩冈的话头说着,但走了两步,他却突地一愣,脚步也缓了一缓。

惊异的望着韩冈的背影,章惇皱起了双眉,究竟是从什么时候开始,韩冈居高临下的评论朝臣,就这么让人视为理所当然。

御史们再年轻,那也是跟同品阶的官场中人相比。至少都是三十岁往后,甚至年过不惑——在官场中,这依然是年轻。而韩冈这个到明年才交而立的后生晚辈,有什么资格在这里装老成?

感觉到章惇没有跟上来,韩冈停步回头向后看,“子厚兄……?”

章惇快走了两步,笑笑,恍若无事。

章惇用眼尾瞅着韩冈的神态,他似乎没有一点觉得自己的话有哪里不对。再细想想,刚才与韩冈说话的时候,自己也的确没有感到任何违和的地方。

章惇忽然觉得,这番话韩冈若是当众说出来,恐怕也不会有人觉得韩冈不够资格,甚至也不会有人感到异样。

章惇一时间不禁有了几分感慨。以韩冈的年纪所达到的地位,如果将宗室一并算进来,其实也算不得什么。但身份为世人所认同,视为理所当然,却依然是独一份。

韩冈并不知道章惇在想些什么,与其并行走到文德门前。锋锐内敛的视线扫过了门前的一众御史,隐带一丝不屑。只是当他站到正门口,与几名御史对上眼,却感觉他们的神色中藏着几分狠厉,并不像是认输的模样。

韩冈顿时心中一凛,难道他们还想再纠缠不成?

韩冈心念电转,却保持着面色平和,徐步走进门中。

该不会要学唐炯弹劾王安石,在殿上给自己好看?

可在大朝会上跳出来当堂弹劾他韩冈,和普通的上本弹劾截然不同,这事就是一翻两瞪眼,连一点转圜的机会都没有。天子也肯定不会愿意看到有人破坏朝会的秩序。

唐炯仗着兴头出来弹劾王安石,现在还不知贬到了哪里去,估计这一辈子都不可能回京城来了。若是有人想学唐炯,韩冈倒是并不介意。

而且在眼下的局势中,赵顼是绝对不会支持将韩冈赶出朝堂。若有人胆敢这么做,当会惹怒赵顼。他让韩冈佑护皇嗣的心意世人皆知,若是继续攻击韩冈,到底是意在韩冈还是意在皇嗣,可就得让人多想一想了。

只要有几分理智,应该不会做蠢事,只是……韩冈又回想了一下方才的几名御史的表情,在利益面前,拼命往悬崖下跳的,可都是所谓的聪明人啊。

文德殿前,朝臣们排好班列,在东西阁门处相对而立。今日定好了日程,要觐见、陛辞和谢恩的朝臣,也在正衙前排好顺序。

群臣毕集,接下来就该是天子升座。净鞭响过,乐班开始奏起升座的配乐。黄钟大吕,直叩心腑,群臣肃立恭候,但直到奏乐结束,却也不见天子上殿。

殿中一片静默,人人心中疑惑,甚至有许多人腾起了不祥的预感。当年仁宗皇帝曾经在朝会上发病,当着辽国使臣的面,说皇后、宰相要造反,这一回该不会是又撞上了。但没人敢动弹一下。

不过时间若再拖长一点,宰相就得进内殿去通问了。过了半刻钟,在对面西班的章惇等人的眼神催促下,王珪直了直腰,便要动身入后殿。幸而通向后殿的侧门处人声响起,当今天子终于出现了。

迟了半刻钟,虽然不算很长时间,但在规矩森严的朝会仪式上,是根本不该出现的场面。朔望朝会,完全是礼仪性质的朝会,君臣进殿的时间,都是早早就规定好的。乐班奏乐的时间,也是固定的,乐声一停,净鞭响过,天子就该出现了。

肯定是有事发生。

耽搁了半个钟的仪式重新开始,群臣向着御座上天子依礼揖拜。韩冈用眼角的余光看赵顼,虽然脸青唇白,依旧是体虚气短,但也没有突发疾病的模样,难道是后宫里面出了什么大事不成?

文德殿的朔望朝会并不是奏事的场合,垂拱殿和崇政殿才是。拜礼之后,就该是入京的外臣觐见天子。但一直留心的韩冈却看到班列后,台谏官的那一拨人中,有人整理衣冠。

‘要动手了?’韩冈眼神变得阴冷起来。

可是谁也没想到,坐在御榻上的赵顼却早一步开口:“王珪。”

王珪愣了一下神,禀笏出班:“……臣在。”

“辽国遣使告哀,云其幼主病夭。依例……该如何措置?”

