\section{第33章 女儿心思可知否(上)}

已是熙宁三年正月初八。

厢房中,一灯如豆。韩云娘趴在桌前,小巧的下巴压在手臂上,呆呆的发着怔。

‘三哥哥怎么还不回来……’

她侧着头,灯火映红了小脸,一根一根的扳起手指算着。三哥哥是腊月二十二被拉去的古渭。当时娘娘还抱怨说‘皇帝不差饿兵,打仗不赶年节。就是西贼也要过年,都快年底了,还要拖着人往外跑。’

而三哥哥那时就说,肯定能赶在除夕前回来。可如今除夕过了,年节过了,都已经是正月初八了,早早就该回来的三哥哥却始终不见踪影。

“大骗子!”

韩云娘百无聊赖的在桌面上划着手指。老旧的方桌上,每一道痕、每一条沟,都数了一遍再一遍。今天该做的针线活都摊在一边,好久都没动过。明天说不定又要挨娘娘骂了,但小丫头总提不起精神来做事。

烧干了灯油的火头忽明忽暗的闪了几下,终于熄灭了,房中顿时陷入黑暗之中,一股浓浓的油烟味散了开来。

小丫头仍没精打采的靠在桌前,既不想起来给灯添上油,也不想就此去睡觉,就这么软绵绵的趴在在桌面上,手指一圈圈地划着。

远远的传来一声狗叫,划破长夜中的寂静。很快,全村的看门狗都狂吠了起来。连刚刚抱来,养在院外的一条刚断奶的小黑狗也跟着一起尖叫着。

小丫头这下终于坐直了身子。是狼进村了?还是来了大虫?

下龙湾近着秦岭,围着村的篱笆又不算结实。野兽夜中入村都是常事,每个月都有个两三次。不过很少能造成什么损失,往往都会被村中各家各户养的看门狗给吠走。

韩云娘推开厢房的门,而韩千六和韩阿李也披着衣服从正屋中走了出来。三人互相看看,韩千六便上前去查看大门是否拴好。这时一阵马蹄声由远至近,逐渐压倒了狗群的吠声,在门前嘎然而止。

“是三哥哥!”小丫头惊喜的叫了起来。

韩冈和李信在家门口翻身下马,一条模模糊糊的黑色暗影便窜到了脚边,两眼绿油油的泛着光,一阵乱吠。韩冈猛不丁的被吓了一跳,定神一看,却是条通体黑毛的小狗,难怪在夜中看不清楚。

正月初三,韩冈随着王韶自古渭寨踏雪而归。用了五天时间,方回抵秦州。他们午后便抵达州城,送了王韶回府。韩冈考虑了一下,还是决定早点赶回来,向家里报个平安。过年不能在家中陪伴二老和小丫头,他心里也觉得有所亏欠。

从秦州城往下龙湾来,若是春夏秋三季,入夜时河上的渡船早已停摆,往往过了申时以后便回不来了。幸好现下是寒冬,朔风凛冽,藉水上的冰层早冻透了底,骑着马踏冰而过,也用不着渡船。

在路上奔波劳累了多日,韩冈的骨头都要散架,不过他还年轻,又早从病中恢复了元气,身体上并没有大碍。只是他倒是没想到,好不容易回了家,先出来出来迎接自己的,竟然是这么一条小黑狗。才半个月功夫,不意连墙上的狗洞都挖好了。

细碎的木底靴踏地声从院中响到门口,院门吱呀一声开了。月色下,久违的一张宜嗔宜喜的俏脸出现在韩冈眼前。只是一与他对上眼,韩云娘脸上的欣喜之色立刻就褪去了,嘟起小嘴,刷的扭过头去。

韩冈看得一笑,小丫头也会闹别扭了。

“三哥儿!”

韩阿李和韩千六也跟了出来,围着韩冈和李信,三人又惊又喜。此时不是后世,隔着几十里,便是消息难通。韩冈一去古渭,深入蕃部之中,拖过了预定的回程时间,家里谁不担心?

“爹,娘,孩儿回来了……”韩冈对着父母就要照规矩跪下行礼。

“跪什么跪!读书都读呆了!”看着儿子、侄子的唇边、头发还有衣物上都凝着一层薄霜,韩阿李心疼得要命,拉起韩冈连声催促着:“快进屋!赶快进屋去!”

老娘发话,韩冈和李信依命牵着马走进自家院中。小黑狗追在两人的脚边,一路叫了进来。韩冈弯下腰,捏着后颈上的皮,把直冲着自己乱叫的小黑狗揪了起来。小黑狗大概只有一两个月大,被韩冈两根手指拎着,呜呜的不敢再高声,有些可怜兮兮的样子。

韩冈的家里两年前本养了一条看门狗,早前赶回家中为两位兄长奔丧的时候还看到过。但等韩冈病好后便没再瞧见。不过这也不是不能理解,韩冈病得时候家里穷得人都养不活,更别提狗了。现在家里境况好了,也该养上一两条来看家护院。

韩冈问着:“这玩意儿哪儿来的?”

韩千六道:“你刘叔家的来福刚生的,前几天来拜年的时候送过来。还没起名字,三哥儿你给想个口彩好的。”

“狗名字要什么口彩?”韩冈信口道:“现在叫小黑,以后叫大黑。”

“这叫什么名字?”

“小黑狗,又不是小白狼?不叫小黑叫什么?旺财、来福之类的太俗了,我也不喜欢。”韩冈笑道,把刚刚有了名字的小黑狗放在地上,它刺溜一下便钻到了院子中的磨盘后,又探出头来冲着韩冈龇牙咧嘴的叫唤。

“别说那么多了,快点进屋暖和暖和。”

韩冈和李信身上都是裹紧披风,浑身上下包裹的严严实实,可脸色仍在夜风中冻得发青,韩阿李一个劲的催着两人赶快进屋去,而韩冈则是先从石磨上挖起一捧雪,用力搓着冻得有些发僵的脸颊和双手。

冬天最忌讳的就是冻伤。若是耳朵像王厚那样得了冻疮后发脓流水,第二年基本上就会再复发,一年一年都不会间断,而贸贸然从冷地里走进暖和的地方,肯定会生疮。李信也学着韩冈的样儿,两人用雪直搓得脸上手上的皮肤滚热发烫,才跨过门槛走进温暖的屋内。

掀开帘子一进门,一股暖意顿时传遍了全身,韩冈舒服的叹了口气。这个时代还没有出现温度计,他只估计着这几日的气温应该是在零下十度上下,虽说比起腊月初一阵寒流后的天寒地冻要好上许多,可这个温度下在野地里跑上三天,也是件很要命的事。

不知是不是没有工业革命的缘故,还是自然气候演变的因素,北宋的气温比千年之后要冷得多,据说广州冬天都会下雪;有些年份的冬天,太湖上都能行人。在秦州城中,逢着冬天,路边倒毙的尸体并不鲜见,往往一场寒流之后,城北的化人场就能连续两三天的生意兴隆。韩冈也是靠着预防措施得力,才没有生了冻疮。

吩咐了韩云娘去厨房烧热汤为韩冈、李信驱寒,韩阿李把火盆拨旺,招呼着两人快点坐下来烤火。

韩千六也在火盆边坐下:“三哥儿,不是说除夕前就能回来吗?怎么拖到今天,俺去城里问都问不出个所以然。究竟出了什么大事?”

“倒没什么大事!就是被雪阻着回不来。隔了两百多里几重山,古渭的雪比秦州大多了。在古渭,腊月底的那场雪下了都有一尺多厚,等回来时过了伏羌城,马才能放开蹄子跑。”

韩冈轻描淡写的说着,仿佛当真大一点的事也没有。但实际上,古渭的事情已经不能算小了。虽然当日隆博和硕托两部在古渭寨中的纷争,被刘昌祚强行镇压下去。不过连刘昌祚都没想到,在古渭寨被杀的竟然是隆博部族长的三子。隆博部的族长死了一个心爱的儿子,肯定是不会善罢甘休。而硕托部身后则站着河州木征,势力更强,木征的弟弟董裕还娶了硕托部的女儿,如果真的打起来,自不会作壁上观。

两部有着几十载的积年旧怨,大打出手那是不消说的。王韶已经命刘昌祚详加查探,戴罪立功。事发的当天,又发了急脚递,不顾艰险的送信回秦州,名正言顺的请李师中整顿兵马。一旦两部纷争,便可趁机出兵,着手打击木征在古渭和渭源一带的影响力。

王韶此次借机主动出招,使得李师中再一次陷入两难境地。一旦两部厮杀起来,动手还是不动手,便成了困扰秦凤经略使的新问题。

而且身在古渭却让两个蕃部在古渭寨中厮杀起来的这件事,对王韶来说虽也是个过错,但如果李师中真要追究起来,身为寨主的刘昌祚却要首当其冲,王韶身上摊不到多少罪名。到那时候,届时秦凤军中排位前十的西路都巡检,免不了也要给逼到王韶这边来了。追究还是不追究,对李师中来说,又是个问题。

王韶是幸运的,在另一段历史里,他会因为没有及时发现隆博、硕托二部间的战事,而被李师中和向宝领头群起而攻,陷入更深的困境之中。

帮助王韶避免了落入如此窘境的功臣,并不知道自己立下的功劳。他此时已经和表哥李信一起坐在融融暖意的屋中,喝着热面汤,有些无奈的听着爹娘的抱怨。

ps:日后兵发河湟的线头埋下了,韩冈也可以回秦州了。接下来,就是上京了。二十多万字了,连个从九品的官衔还没正式到手,不知俺是不是第一个。

今天第二更,照例求红票,收藏。

