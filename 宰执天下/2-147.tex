\section{第28章 大梁软红骤雨狂(八)}

【第一更,连续熬夜头昏的厉害,今天写不下去了,下一更明天肯定补上。前面欠的一章,也会抽空补上】

章俞已经回乡去了,现在在京城中的宅子,只有章惇和他的妻儿住着。当章惇回到家时,已经是三更天了。

而章惇的两个儿子,章持、章援,一个哈欠接着一个哈欠,但就是不敢去睡觉,而是在书房中等着父亲回来。

章惇推门进了书房,开口便问:“大哥,四哥,功课做得如何?”

章持和章援一个十岁,一个八岁,年幼易困,等到半夜,已经是迷迷糊糊的了,但听到章惇的声音,便立刻跳起来老老实实的行礼站好。如果章俞此时在场,多半就要笑说这么老实的孩子,根本不像他的子孙。

少年时的章惇,行事荒唐,胆大妄为,甚至还被人告到衙门里去过。偷香窃玉的本事跟章俞是一个模子铸出来的,一个偷岳母,一个偷族叔的小妾,算是婶母,毫无士行可言。

如此品行,加之出身方面的因素,在族中章惇是被人当作另类看待。可是他能有如今的成就,也是因为赌上一口旧年怨气的缘故。在嘉佑二年第一次中进士时,章惇才十九岁,比他中状元的侄儿章衡整整小了十岁。但就是由于在族中受到歧视缘故,便不肯屈居章衡之下,弃了进士头衔,下一科又考了个进士出来。

不论是自信,还是才学,章惇都是第一流的,仅仅是品行上有些暇疵,所以愁困于人才稀缺的王安石,还是将他加以重用。而这样的章惇,对两个儿子的管束却是很严格,章持、章援每天的功课他都要亲眼看过才放心。

从两个儿子今天学的经文中,抽了两句出来,询问其大义。见他们都能回答得上来,章惇忍不住绽开了一丝笑容,很爽快的放了两个小子回去睡觉。

夜深人静,灯火幽幽。外面的更鼓咚咚的响着,可章惇仍是毫无睡意。他随手翻着摆在桌案上的一摞名帖。如今章惇官位虽然还不甚高,但受伤的权柄却是煊赫一时,接了曾布的班,做了检正中书五房公事,掌管所有发往政事堂的文字,赶着上来巴结他的官员并不少,摆在书桌上的名帖也从不见少。

他每天都要随手翻一翻,权当作消遣,会从中挑出几个来见一见面。不过今天章惇并没有什么兴致,随便看了看就准备让人拿去收起,但其中一张正好在这时跳入他的眼帘,章惇的手一下便停了。

将吸引了他注意力的名帖和附带的信件拿起来细看,章惇提声叫来昏昏沉沉的仆人。他把名帖一摊,“秦州韩官人的帖子是什么时候来的?”

那个仆人是听说过韩冈的,章府的家人,一听说秦州韩官人就知道指的是谁。方才韩冈派人来送信时,他也留心记下,“回官人的话,是打初更的时候,韩官人的贴身伴当奉了韩官人命,送了帖子过来。”

‘韩玉昆倒还记得要找谁帮忙。’章惇笑了一下,对仆人道:“去把明德请来。”

路明在睡梦中被人叫醒,头昏脑胀的就要骂人。但一听说是章惇请他,便忙把满腹的怨声收起。住在别人家里,当然只能客随主便。

路明自从决定从商之后,便跟章惇拉上了关系。虽然韩冈曾经说过有事可以去秦州找他帮忙,不过远在秦州边境的韩玉昆,怎么能比得上京城中宰相心腹的章子厚,而且要做买卖,在京中也比秦州更能大张手脚,投靠谁对路明来说当然不是问题。

路明只是没有读书的本事,但他胆大心细,见识甚广,又善于探听消息,所以虽然他在商人中还算是新人,人脉也还没有建立起来,但不到一年的时间里,跑了三趟京城之后,就已经有了点身家,不复当日的寒酸。而且要不是京城中大行会坐地分赃,身为行首的豪商们把持了贩卖的渠道,路明现在当已是腰缠万贯了。

章惇没等多久,路明便装束整齐的来到了他的书房。行过礼,路明坐下来便问道:“检正唤在下前来,不知有何要事?”

“韩玉昆今天入京了,不知明德是否已经知晓?”

路明点着头:“在下已经知道了。事情还真是巧,方才韩玉昆的伴当李小六来送名刺,在下正好见到。还让他带了话回去。”他笑了一声“本还准备明年开春后,去古渭拜访一下韩玉昆,没想到今次就已经上京来了。”

“既然明德已经知道,就不必我多说了。明天就请明德你去见一见韩玉昆,说我在樊楼定下位子,好好聚上一下。”章惇想了一想,“顺便把教坊司的周小娘子请来,最近她的名气可是越来越大了,中书里面都有人提过她。”

路明犹豫了一下,道:“他事检正尽管放心,路明必然办得妥当。只是教坊司的周南,还请检正不要请她来献艺。”

章惇心中生疑:“这是为何?”

“周南对韩玉昆一往情深,她吓走高密侯的匕首还是韩冈当日所赠,的确是教坊中难得的贞烈女子。若是仅仅如此,她日后能归于韩玉昆,也算是一桩美事。可是如今二大王正倾心于周南……”

“雍王!?”

“正是雍王!”路明点头,“只是化了名字,但市井中已经流传开来。韩玉昆年纪轻轻便已经立下了这么多的功劳,前途不可限量,若是因为一个妓女就恶了雍王,毁了前程,就实在太可惜了。”

路明弃儒从商,换作是普通的士大夫,肯定是鄙视加疏远。不过章惇并不在意这些。他是福建人,家乡山多地少,工商之人不比农民更受人歧视。倒是北方出身的士大夫,惯于土里刨食,都看福建人、乃至整个南方的士人不顺眼,国初时有南人不为相的说法,而司马光也说过‘闽人狡险,楚人轻易’,地域之间的歧视可见一斑。

章惇对路明的态度则很明确,‘即便是鸡鸣狗盗之辈,也还是可以一用。’

不同于王安石的观点,认为孟尝君只重鸡鸣狗盗、因而国士不至,治国要找的是那种得一即可‘南面而制秦’的贤才。章惇一直都是抱着物尽其用的原则,只要有一点长处,总有用得上的时候或地方。

路明虽然无甚才学,但做生意还是有点水平,而包打听的本事,则更是让人惋惜他为什么不是皇城司中的成员。今夜的表现,也更证明了这一点。

不过章惇跟路明的想法不一样,“这件事得韩玉昆自己来处置,你我越俎代庖反为不美。以韩玉昆的才智,他定然会有所取舍。”

……………………

夜半时分,大内武英殿中仍是灯火通明。

赵顼俯身望着群山中的无定河,眼神定定,许久也不眨一下眼睛。半天后,他才出声问道:“宋卿,你是殿帅。你说说今次兵发罗兀,还有哪处有疏漏?”

步军副都指挥使宋守约没有动弹,只是皱起了眉头。虽然从官职上,副都指挥使上面还有都点检、都指挥使等职位,但实际上,都点检自赵匡胤做过后,开国后就不再授予臣子,只是空名而已。而都指挥使,也常常空缺。三衙管军之一的侍卫亲军司步军副都指挥使已经是当今武臣中屈指可数的高位。

宋守约形貌严重,平日里总是挂着一张脸,盯着人时,一对眼睛就如冰山一样没有半点情绪蕴含,冷冰冰的,让三衙的兵将望而生畏。而且他更是有名的御下苛刻,宿卫宫掖时,嫌夏天的蝉鸣躁耳,便下令将树上的蝉虫全都赶走。

宋守约自在三衙任职的这几年来,每到夏日,进入宫中的官员,都能看到一群士兵,汗流浃背的举着竹竿往树上扑打着,守卫宫中的每一颗树不受蝉虫的侵扰——安安静静的夏日深宫,也就成了东京城中的一大特色。

但宋守约在这时候却是没有板着惯常的棺材脸,反而是一副忧心忡忡的样子。已经是三更天了,可天子仍未入眠。自己年岁大了,睡眠少点无所谓,但赵顼的身体本就不算好,再熬夜下去,说不定就要病倒。

他没有理会赵顼的询问,反而劝谏道,“官家,横山那里,韩相公已经筹划妥当,兵精粮足,领军的种谔亦是老于兵事,已是万全之备,官家勿须忧心。还是早点歇息去吧,明日还要上朝。”

赵顼嗯了一声,却还是没抬头。

能否控制罗兀,将决定横山的归属。即将开始的一战,也便决定了西夏的国运。此前的历次小规模的战斗,都是以大宋一方获胜而告终。一次次的胜利,如同吹气球一般把赵顼对军队的信心给膨胀起来,一战定乾坤,这样的诱惑,是赵顼所无法抵抗的。

方方面面都考虑到,赵顼自问已经做到了最好。鄜延那里,拥有最为精锐的将领和军队,拥有足够的粮草储备,而韩绛并不以此自得,对每一方面都要求做到最高,基本的兵粮不提,对军中医疗也是极端的重视……

“对了。”赵顼像是想起了什么,“李舜举,今日是谁在中书值守?”

一直随侍在天子身边,如幽魂一般站在殿中一角的李舜举站了出来,“回官家的话,是冯参政。”

“你去问问冯京,韩冈何时能到。一旦韩冈抵京,就让他越次觐见。”

