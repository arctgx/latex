\section{第十章 千秋邈矣变新腔(25)}

元佑元年的科举已经结束了。

包括明法科和特奏名的考试,也都有了结果。

新科进士在宣德门前拜谢君恩,然后去参加琼林宴,接着各自回乡炫耀去了,而没有被取中的士子,也大多早早返乡。一时之间,京城中诸多专供士人的寓所变得人去楼空。

而制科的御试则在此时按时开始。

制科御试的考题出自韩冈手笔,除了硬是加上了以百分制为核心的评分标准之外,就没有别的更动了。

这可以说是多此一举。通过百分制来评定名次高下的确很简单,可这一回御试只有两人通过阁试,御试上的题目也只有一题,根本没有必要。

不过因为过关的两人分别是韩绛和张璪所推荐,韩冈要改动考题的评分方法,将考题从文辞、道理等方面详细的加以评定,只要征得他们的谅解,就是王安石反对也没用。

最后的结果,一个三十五分,在第四等,一个十五分,只有第五等。

若是按照六十分为及格,这样的分数实在是惨不忍睹。就是太后也低声对宰辅们说这个实在不成话。原来的等级制度,一二三四五分等级看起来并不直观,可现在换算成分数,让人看来就觉得一百分中只能拿到三分之一,这样的表现实在是不及格。

但既然朝廷旧例是第四等为合格,那一位也就顺利的获得了制科出身,同状元待遇;第五等则黜落,同样是依照旧例,仍给官加以勉励。

此番事了,朝堂中一时恢复了平静。

没有了廷推宰辅,也没有了抡才大典,王安石与韩冈这对翁婿一时间也没了争执的必要。

而随着春日的到来,陆上道路畅通,海路也变得稳定,来自于国境之外的消息也就多了起来。

葱岭之西,黑汗的军队据闻已经开始集结,甘凉路上,正加紧给安西都护府输送物资。

粮草可以就地征集,但军器就必须从后方运去。雪化之后给安西都护府的第一次运输,便是多达一万张的马步弓和两万套的神臂弓,以及一百余具大小不一的床子弩,当然,箭矢弩矢都不缺。甲胄、刀枪、骨朵、铁板之类的铁制军器,不易损坏,只需要进行少量的替换,但也各送去了两三千件作为预备,而最重要的军器工匠,总计五十余人也一并前往。

于此同时,新一批多达九个指挥,三千四百余人的援军,从凉州出发开始向西域都护府前进,护送重要的兵器,同时更是为了稳固刚刚收复的新疆土。有过多次拓张的经验,朝中上下都清楚,这等过去没有见识过皇宋天威的新领地,不经过两次三次反复,不会老老实实的降顺。

按照朝堂中业已议定的结论,安西都护府辖下的汉军数量,在两年之内要达到一万五千左右,而蕃军的数量则以两万为限,再加上降顺各部的私军,如此方能保证天山南北两侧的安全。尤其是天山南北两麓适宜耕种的土地,能够安排下数以十万计的移民,这就需要更多的官军去西域以保护来自内地的移民。

也就在半个月前,交州之南,占城和真腊两国再次遣使来哭诉,也一如既往的再次被朝廷所无视。前一次,两国使节直接在两广就被打发回去,这一回,占城、真腊国使准备绕道泉州,不过结果依然不变。

放在世间的道德中,驱使奴隶,不顾其生死,绝对是作孽。但作孽也是交州蛮部作孽,大宋的子民照旧心安理得享受着蛮部所提供的大米、香料、木材,以及他们提供原材料所制成的白糖、蜜酒、果脯等各色特产。

这两年,交州的种植园不断扩大,交趾奴工的数量已经不敷使用,亟需稳定而可靠的奴工的来源。大批要被报废的甲胄和兵器因而‘流失’到交州蛮部手中——尽管这些兵器在大宋军中看来,已到了必须更换时候,可放在南方蛮部手中,依然是克敌制胜的法宝。

捕奴队在南方两国的奸细的引领下,每个月都能弄回几千人,而在这几千人背后,是大量的村庄被毁灭。占城、真腊几次调集大军来进剿,总是无功而返。即便偶尔能逼退捕奴队,追击到国境线时,未免引来穷凶极恶的宋人,又不得不止步。私下里韩冈写信给冯从义,什么时候占城、真腊想通了,改行做人口输出贸易,自己去外国捕奴转卖给交州蛮部,他们也就能够解脱了。

巩固新疆很重要,安抚旧域同样重要。

灵武故地在三月、四月又迎来了大量的移民。旧日位于山中的屯兵和民户,大都移居到灵州附近。在这个冬天,当地衙门组织人力,将被毁坏的灌溉河渠给修复了,其余渠道也全数修整了一遍,在闸门、堤坝、分流水路上,更进行了改进,用了最新式的工程设计,远比党项人在唐人的基础上进行的发展要强得多。关西有数的产粮之地,加上近处的盐池,又没了贪婪的党项贵族,只要官员治理得宜,原本在党项人手中就以出产丰富闻名塞上的灵武之地,日后会更加的繁荣富庶。

而一干党项余部,尽管成功的从青铜峡中杀了出来,可是在种谊、赵隆等名将的监视下,接受了朝廷对土地的划分,没有半点异动。以叶家与仁多家为首的党项余部都明白,如果他们再敢起异心,下一战,就是党项灭族的一战。

春天的到来,也意味着因冬寒而停止的农、工两事的开始。

不过在河东,即使是在冬日,恢复生产的工作也没有停止。战乱之后人民流离,空出了许多土地,原本因为田主阻挠无法进行的水利和道路建设,如今就少了许多阻碍。代州、忻州、太原的水利及道路工程,正通过以工代赈的形式顺利的运转着,无数回到家乡的难民,也靠着出卖劳力,渡过了这个艰难的冬天。

至于最重要的铁路并代线的轨道铺设,中间虽有反复,但也进入了最后的攻坚阶段。按照主持工程的李诫的禀报,不出意外的话,在六月之前,就能彻底结束工役。但这只是个开始。

这条目前国中最长距离的轨道运输线,能否稳定有效的运行,将是日后与汴河平行的京泗线,连接北疆的京保线,以及向西延伸的开封至长安,乃至秦州的最重要的参照对象。大量铁路专业的官员和匠师,也需要通过这条线路进行培养。

相对于河东,河北在战争中受到的伤害更小,恢复得也更快一点。只要今年夏天能够正常收获,河北的局面就用不着担心太多。此外由于重新订立了和约,加之辽国的重心正放在东面,河北国境线上的寨堡正紧锣密鼓的增修着,主要是为了配合火炮来修筑炮台。强盗大赚了一笔之后,总要消停一会儿的。但要是就此不加防备,那就未免太愚蠢了。辽国对此也只能默认,即便表示抗议,朝廷也不会理会。

于今说起辽国,就不能不说日本。

依照最新的消息,如今日本已经没有什么天皇了,高丽国王也在耽罗岛上苟延残喘。但契丹势力下的日本国和高丽国的确还存在。

就像耶律阿保机曾经在渤海国的土地上册封了他的长子耶律倍为东丹国王,让其独立领国。这一回耶律乙辛的次子成为了新任高丽国王,而日本国王也是耶律乙辛的一个儿子,不过下面还分了好几个郡王、国公,各自占了一片地,相当于分封了。如此好处均沾,耶律乙辛的声望再上新高,无论南北,都在等着看他什么时候篡位了。

“高丽、日本败得太快了,看看这才几天功夫?”向太后现在忧虑的就是这一件事。

宋辽大战刚刚结束也没多久,大宋这边还在努力恢复元气,辽国那里就已经灭掉了两个千乘之国了。

按说辽军南侵时也没占到多少便宜,怎么面对高丽、日本这样的国家就如此摧枯拉朽。当年官军攻打交趾,还在广西准备了一年才开战。

高丽和日本的大捷,对契丹军心士气的恢复,有着显而易见的作用,也足见辽国的战斗力依然强大。

韩冈费了一番口舌才向太后解释明白。

军事技术上的跨越发展,尤其是普及到普通士兵的铁甲,让高丽、日本的军队,在正面战场上完全无法与辽军相抗衡。

不论两个小国拼凑出来的是几万,还是十几万的大军,其核心都是区区一两千,甚至为数仅只几百的精锐。

维系战场上士气不堕和打开战局,都是核心精锐的工作。正常开战时,这些精锐都会被攥在统帅和将领们手中,成为坚持战线的中坚,以及关键时扭转乾坤的胜负手。

但在面对纯粹以强兵组成的敌人,以征发起来的农民为主力的军队,完全不能与之抗衡。

辽军在日本的几场大战,漂洋过海而来的信息并不算多,可通过前后情报上的只言片语,就足以拼凑出了高丽和日本惨败的原因。

辽人的兵力的确居于劣势,但计较起战斗力,还是更为优胜。辽军只要直接冲向敌阵阵势的薄弱处,击溃了当面敌人,接下来就像是牧羊犬赶羊一样,让败兵去冲击其他敌军,这样一层卷一层,转眼就能让敌军彻底崩溃。

以契丹铁骑的实力,只需发挥出正常的水准,区区一千甲兵,便能轻而易举的就压倒了数万农兵。

而且辽军作战的方式也随着装备的提升而开始改变。在过去,契丹骑兵中,重骑兵的数量并不多,而且其冲阵从来都是从侧翼,或阵列的缝隙中冲过,遇到坚阵就避开绕路,绝不会硬冲硬打,不过在高丽和日本的战法就变了个模样,敢于正面突破了。

“即便是最精锐的倭兵与高丽兵,也抵挡不了具装甲骑,何论一群农夫?”

“只有流寇才会是聚农为兵,那等流寇,纵使十倍与官军,又何足论?北虏虽胜,胜之不武。”

“北虏的疆域同样横贯东西万里,契丹骑兵同样能探及西域。大食良驹,中国能够得到,北虏同样能够得到。但中国有中国的战法。具装甲骑的确勇不可挡,可面对神臂弓和火炮,却又能有何施为?即便是厚如城墙,也经受不了几炮,何况血肉之躯?”

“所以还要看火炮?”太后只记住了韩冈的最后一句。

“火炮,以及围绕在火炮周围的钢铁、火药、轨道、马匹等一切的促进火炮运用的事与物,都应是发展的重点。”韩冈语调平稳徐缓,“要想破辽,当以己之长攻敌之短。只有更精强的武器,和经过严格训练的军队,才是克敌制胜的关键。”

