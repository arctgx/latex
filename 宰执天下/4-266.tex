\section{第43章 庙堂垂衣天宇泰(13)}

张载身前没有置产,绝大部分的财产都送进了横渠书院这个无底洞。在他去世之后,张家的遗孀遗孤甚至连他的后事都办不起。还是几个学生,加上冯从义代表远在广西的韩冈,一起料理的后事,在横渠镇外张载父母的坟地边安葬。

在这之后,韩冈又指派冯从义帮张家置办了五百亩地,以保证自己的师母师弟的生活。在他的领头下,其他学生也零零碎碎凑了三百亩出来。只是这八百亩地,分布在凤翔府、京兆府、邠州、渭州的四个州府包括郿县在内的八个县里,收起帐来不仅仅是麻烦,而是让人痛苦了。

不是因为凑地皮的人太多,导致土地零碎——韩冈送的五百亩,也分散在三个县中——而是因为在大宋任何一个内地州县,想要买到能有正常产出的大片田地的几率几乎为零,根本做不到。

做地主的没有不想将自己的土地连成一片——韩冈当年之所以会被人陷害,也不过是李癞子想要韩家的三亩菜田,好将家中的河滩田连成一片——但心想事成的寥寥无几,除非有几代人的经营,否则家里的田产都是东边一片,西边一片,能有十几亩整地都算大片了。

而且这个时代,土地交易频繁,有说法是‘千年田易八百主’,许多土地几年就换一个主,不仅仅是在商业气氛浓厚的南方如此,就是视土地为生命的北方地主,也会因为各种各样的原因买地卖地,而且整地零卖的情况很多,一百亩地,今天送礼卖三亩,明天还债再卖五亩,再完整的田地,几年下来就成了。想要将分散的土地重新拼凑在一块儿,那目标田地的主人怎么可能还不趁着良机,抬高价码?

但为了解决八百亩地收账困难的问题,后来依然是冯从义和张载的学生们出手,靠着在关西大户中的人脉,帮着卖地置地换地,最后得到的田地还是八百亩,不过集中在凤翔府内的郿县、盩厔、扶风、岐山四个县中,同在一府,收账也方便多了。

听到今年租税收缴得差不多了,苏昞点着头,一幅很是满意的样子,“子厚先生为人宽和,,现在师母师弟也是一般,再不帮忙照看着,迟早会被下人所其欺。”

“季明兄说得是,小弟一定会注意照看的,这段时间可没有少跑乱。”拍胸脯保证过后,他试探的问道:“不知最近有什么喜事?怎么季明兄一下变得春风满面。”

苏昞也不瞒人:“是玉昆那边的事。”

“襄汉漕运成功了?”慕容复惊讶的叫道,“六十万石纲粮这么快就运到了?!”

“那下面是不是河北轨道该拿上台面了?”游师雄在战略上眼光,张门弟子中能排前三。一条马车速度的运输线,对国家战略的影响不言而喻,他看出其中门道的时候不比任何人晚。

“都不是,你们说的还没有登上台面,”苏昞摇摇头,从袖子里摸出一卷书来。“你先看这一卷的内容再说,”他对着游师雄说道。

游师雄疑惑不解,接过来一看,脸色全都变了。抬头惊问:“这是真的?!”

慕容武好奇的探着头,就在游师雄的手中看到了让苏昞变得神神秘秘、而游师雄本人差点跳起来的报告。

竟然是种痘。

慕容武不怀疑韩冈的能力,但韩冈放出种痘法的时机很成问题,在官场上稍有眼色的,都能看得出其中有点不对劲。藏了十年了,再藏个两年也没什么关系,等到有关轨道的功劳先拿到手再说。选择当下放出来,理由当然只有一个,“这是何苦呢,再等两年也没有关系。”

但苏昞不这样看,“玉昆所学讲究以实为凭,玉昆精研格物之道,格出了其中的道理。又有谁能说一句不对?”

世人是现实的,韩冈通过板甲、飞船、轨道,再加上如今的牛痘,一步步的树立起了无人能动摇的权威,他在学术上的观点,自然也就如同天子的金口玉言一般,对其他学派拥有了压倒性的优势。

可以说,程颐入关中后的多日辛苦,韩冈只用了区区四个字,就将他打回了原形。

苏昞兴奋无比,但游师雄和慕容武则是面面相觑。以两人的政治智慧,哪里看不出韩冈为此付出了什么样的代价?

看苏昞的样子,只在乎能不能维系气学道统,但游师雄和慕容武却要为韩冈担心他日后的前途——这同样事关所有张门弟子。

“前五名皇子接连夭折,还有三位公主也是一样,其中多半就有因痘疮而夭折的。”游师雄声音干涩无比,“玉昆不愿有伤圣德,故而隐匿至今。但天子那里还不知会怎么想,万一有个奸佞进谗言……。”

“玉昆既然将事情做出来了,肯定是考虑过了后果,你我也不必为他担心。”苏昞让游师雄和慕容武不必操心太多。

可游师雄和慕容武又哪里能不担心,当韩冈的奏章送到预案前,惹怒天子几乎是必然的,而且还少不了会升起猜忌之心。韩冈还不能抱怨,他所玩的就是这样的游戏。

苏昞现在的心情最平和:“有了玉昆的种痘免疫法,所有人的心都能安定下来。”

苏昞虽然没有明说出来,但前日看到韩冈让人送来的《桂窗丛谈》时,早已被韩冈的选择给触动了,还激动了很久。他在官场混迹多年,韩冈付出的代价苏昞难道能不知道,如此胸襟和见识的人物,的确是是世所罕有。

韩冈虽然人在京西,但心还在气学上。为了维护气学一脉的根基,宁可放弃光明灿烂的前程,也要坚持心中的信念,这才是真正的儒者。韩冈都能做到,他为何不能做到?

“愚兄准备辞官了。”苏昞说得真诚,他此前只是辞了差遣,不去候阙,但在收到韩冈的来信后,就准备离开官场了,“虽然苏昞所学有限,不及子厚先生之十一,但同列张学门墙,总不能眼睁睁的看着玉昆一人苦撑大局。”

游师雄和慕容武对视一眼,皆在对方的眼中看到了无奈和叹息。苏昞这样的知名前辈,都给韩冈带下了水。

还有韩冈,怎么就那样的死心眼?韩冈平时看着多精明厉害的一个人,当初跟他岳父因学派不同而闹得差点翻脸,都有可能是做给天子看的,但这一次的表现实在是有辱过去立下的赫赫名望,难道他认为使天子的子嗣不再受痘疮困扰的功劳,能让他继续高歌猛进下去?

但两人无奈归无奈,韩冈的选择让他们也无法指摘,最后只能干脆了当的让韩冈继续守着下去。

……………………

襄汉漕运的启动,使得襄州不仅成为物资集散中心,同时也成了信息情报的集散中心。

汴水上的两个转运中枢扬州和泗州,京城大商号至少会在其中一处设立分号。如今的襄州,也有成为另一处商行聚集的中心城市的趋势。

就在伏龙山中、黄庸还没有登门造访韩冈的时候,种痘免疫的传说就已经以襄州为中心,向四面八方散布开来。这个消息沿着沟通南北的通道,北上京畿,南下荆湖,不数日就传遍了沿线的各大州府,最后随着不断运抵京城的纲粮抵达了京师地界。

为了推行手实法,吕惠卿前些日子将吕升卿外放做了提点开封府界诸县镇公事,俗称的府界提点。吕惠卿是打算在京城做个样板做出来,让朝堂上下都看一看手实法的成果。

吕升卿要为兄分忧,现在是忙里忙外。由于府界提点衙门因为韩冈的缘故,在熙宁七年搬到了白马县。吕升卿就不得不前往白马县,隔上好一阵才能回一趟京城。

不过吕升卿更多的时候,还是在开封府辖下的十八个县中来回跑。接触的人多,走的道路也多,听到的消息自然同样的多。种痘法出现在京西的消息,很快就传入吕升卿的耳中。

刚开始仅仅是一两句话,说有这回事而已。吕升卿哪里会当真,只当做笑话跟自己的幕僚说。但等到个中细节随着时间的推移一步步的补充完整,他心中顿时就火烧火燎。急忙丢下手上所有的事,找了个借口跑回了京城。

回到家中,吕惠卿还没有回来。他在书房坐立不安的等了大半天,好不容易才等到吕惠卿回来。吕升卿草草的行礼问好,就急着道:“大哥,你听说了吧?外面都在传韩冈发明了种痘之术,能防痘疮了!”

吕惠卿仿佛没听到吕升卿的话,坐下来,抬起眼,慢悠悠的叹道:“建国公昨夜病卒。”

“啊?”吕升卿一时没反应过来。

“韩冈的奏章是今天早上到的。”吕惠卿语气平和的就像是寒暄时聊着天气:“而皇第七子建国公在昨夜夭折了……”停了一停,露出了一丝若有若无的笑容,“是痘疮。”

