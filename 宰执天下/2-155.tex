\section{第29章 顿尘回首望天阙(七)}

【第二更,求红票,收藏】

夜已深。

微晕的烛光照耀下,王安石的心情有着难得的平静。

新法推广越来越顺利,无论青苗贷还是均输法,还有仅在京畿诸路推行的免役法,都给空荡荡的国库带来了丰厚的收入。也因为这些收入,让文彦博之辈的攻击,在天子心中毫无份量。

针对吏人的重禄法,尽管只是刚开了头,要完全推行开必须等到两年后,但也让胥吏们感恩戴德,也更加用心做事。而胥吏中的不法之徒,如今要处置他们起来,也便更加名正言顺。

农田水利法虽然短期内难见其功,但在侯叔献、程师孟,还有内侍程昉等提举官的监督下,兴修水利、淤田开荒的工作都在稳步进行中,开辟出来的新田,还有改造好的下田,都会给国库带来更多的收入,而百姓也能从中受到恩惠。

新年将至,眼下朝中无甚要紧的小事都要暂时停下了,不过新法的推广还是在持续着,曾布在司农寺,章惇在政事堂,曾孝宽在开封府,还有其他为了新法而奔走的官员,一直都很用心卖力。到了明年,眼下已经颁行的逐项新法,都要更加深入的推广下去。只是新法的重心,将要转到军事上来。

筹划已久的将兵法要施行,吞吃掉八成财税的各营冗兵,也都要一一加以整顿,不能作为有效战力的禁军士兵,他们的军额都要降等取用。至于厢兵,也要汰撤很大的一部分——这主要还是挤掉空饷的用意。为了弥补人员上的损失,保甲法便要同时加以推行,让各地百姓一起组织起乡兵,用来保护自己的家乡不受盗贼侵扰。

另外还有保马法。皇宋百万大军,军马总数才十余万,这一点,莫说契丹、党项比不了,就连吐蕃、大理都可以站在一边笑话一下。自他在群牧司任判官的时候,就已经知道几大牧监已经完全烂透了,可以不用抱着什么期待。现在唯一的选择,就是把马寄养在百姓那里,至少可以得到更多的照顾。

王安石叹了口气,作为一国宰相,他要考虑的事情很多、很多,不仅仅是新法的推行……还有刚刚空出来的开封知府的人选。

外界都宣称天子对他王安石言听计从,但实际上,在人事方面,赵顼一直把握得很好,对如何控制权力有着天然的认识。单看文彦博尽管始终不能再任宰相,但他仍能稳坐钓鱼台,把枢密使一职牢牢地攒在手中,便可知端地。

现今天子看好的新任知府,是前河东都转运使刘庠。刘庠进士中得早,很早就入朝为官。因为他执掌河东地区的钱粮,前些时候在与陕西合作时,被韩绛看不顺眼,所以卸了任。正好韩维现在要离任请郡,刘庠资历功绩都够得上,韩维一走,他就要紧接着上任去了。

刘庠家世不高,但身份不差。是仁宗时宰相蔡齐的女婿。而蔡齐则是真宗钦点的状元,如今进士跨马游街的体例,便是从蔡齐开始。

‘又是一个宰相女婿。’

其实被顶替的韩维虽然岳父不是宰相,不过其父韩亿却是宰相家的娇客——太宗、真宗时的名相王旦的女婿。

宰相的女婿往往能升任宰相……王安石突然想起了自己,自家也是宰相了,是不是日后也能找个宰相女婿。大女儿的夫婿没有这个本事,而二女儿,就要看她的运气如何了。

自家的两个女儿,德言容功,哪一条都不差。但长女就是因为王安石跟吴充的翻脸,便受到了夫家的责难。多年的交情都靠不住,想要给二女儿找个更合适的人选,让王安石大费思量。

原本他的弟子蔡卞就是个很好的人选,可是因为事情耽搁了,不然有了这个进士女婿,王安石就再没有什么要为儿女担心的了。但现在,王安石还不得不为此而头疼。

宰相家的女儿,不是唐时的那些公主,不会没人要。找人上门提亲的为数不少,不过都让王安石和他的夫人给否决了。朝堂上的年轻人中,能让王安石看上眼的可不多。想来想去,韩冈都能算一个,而且章惇前日还真的透露过一点代韩冈做媒的打算。

王安石不会因为韩冈出身菜园而小瞧他半分,门户低一点,嫁过去就不会受婆家欺负。王安石希望看到的女婿,要有前途,肯上进,人品要好,才学也要出色,最好能中个进士。为了女儿的未来着想,不能找个体弱多病的夫婿,若是文武双全那就更好了。还有相貌,至少能看得过去。至于身家多寡,王安石倒不介意,大不了嫁妆给多一点,当不会让女儿吃苦。

王安石难得的有闲空为女儿的未来细细考量,但他所知道的那些个青年才俊的本事,想起来就让人心浮气躁。他一条条的数过来,猛然发现,韩冈好象是这群人中最为合适的一位。

就是脾气倔了点……

王安石突然想到,自己是不是待韩冈过于刻薄了一点。立下了那么多的功劳,只是因为不合己意,就连天子都见不到一面——虽然是冯京在其中作梗,而他前面说的气话本也没打算当真,但韩冈今次的确是没法儿面见天子——的确很吃亏。

还有女色方面瓜葛太多,王安石开始算着韩冈的缺点。就算高居庙堂之上,有些消息还是能传进他王安石的耳朵里的。

韩冈前次上京才几天,就诳得一位名妓要死要活的跟着他。这也算是本事!但王安石在女色上从来都是严谨自守,本身看得也很淡,身边除了一位结发老妻,便再无一名侍妾,对于韩冈这等沾花惹草的行事便有些看不惯。

不过王安石不会在他列出的条款中,加上专一这一条。士大夫娶妾是世风如此,王安石也没有打破风潮的愿望,只是希望自家女儿的夫婿人选,不会太过沉湎于醇酒美色。

这些要求在王安石看来也并不高,而且也不是逐条都要满足。王安石想要再等等,他觉得他的女儿应该能找到更好的人选。

“爹爹还没睡吗。”王旖在外面望着书房,从窗外看进去,她的父亲正坐在书桌前。

“还有一阵吧。”王旁也在看着书房。“爹爹考虑的都是国家大事,现在还不到睡觉的时候。”

“什么国家大事,也不必着急成这样,早点睡不好?”王旖嘟了下嘴,又问道:“大哥大嫂这两天就该到京城了。”

王旖问着王雱行程。王旁明年二月就要成亲,他为了能及时赶上弟弟的婚礼,就拼了命的赶在年节前上京来。

“不出意外的话,应该就是这样。正好可以让大哥见一见韩玉昆的本事。”

‘韩玉昆吗?’王旖悠然神往。她现在只是有些好奇而已,为什么家人中常常提到他的名字,就算是吕惠卿、章惇这样的心腹,也很少有在二哥王旁的嘴里被提及。

‘究竟是个什么样的人啊?’王旖浮想联翩。

……………………

听到外面的车马声,许大娘问着刚刚进门来的干瘦汉子,“甘穆,是那小贱人回来了?”

甘穆点着头:“的确是南姐儿回来了。”

回想起今夜雍王过来,听说周南被人请了去后的神色,许大娘心头就直冒寒气。都说雍王好学勤谨纯孝知礼,就算经常来找周南陪酒,也没有强逼着她侍寝,看起来也的确是个好脾气。但今天,只被雍王的眼神扫了一下,许大娘就浑身发起颤来,那等惯于杀人放火之辈才养出来的眼神,哪里是好好先生能有的。

许大娘很想把周南赶快献出去,省得日后纠缠起来,倒运的都是他们自己。但周南又是个烈性子,若是受了辱,说不定就会自尽,但在自尽前,她会做什么,那就说不准了。

“姓韩的关西措大底细,你打听到了多少?”

甘穆陪着小心的把自己打听到的情报都说了出来:“听说那措大在关西很有些名气,还传说他是什么孙真人的弟子,救治伤病无数,又得王相公和韩相公的看重,就要到鄜延赚军功去了。而以他的年岁,他现在的官职已经很大了。”

“大!?”许大娘不屑的一声哼笑,“能比雍王还大?”

“官家的弟弟,连王相公都比不上,何谈韩措大!?”

虽然人人皆知,到了大庆殿上,亲王的班次位于宰相之下,赵颢还要站在王安石的后面。可宰相经常换人,如今的官家登基后已经换了四五个相公了,却没人听说过官家的弟弟还能换人的。

“那就是了!”许大娘拍了拍手,“你就去跟那个措大说,周南是雍王看上的。若不想开罪雍王,早点回他的关西去!”

她阴狠笑着,洗去了所有妆容的下面,是一张皱纹横生的老脸,“那小贱人不是要为着措大守节吗?等措大逃了,看着她还能守个什么?!”

