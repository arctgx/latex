\section{第34章 山云迢递若有闻(16)}

说起古钱,韩冈在前生,只会想起圆形方孔、黄灿灿的铜钱。

但黄铜钱,主要出自于明代之后。在宋代,青铜钱才是主流。而且因为如今铜料稀少,铁钱在市面上也是大行其道。比如缺铜的川中,外路的铜钱内运不易,便是只通行铁钱,与外界隔绝了币制。朝廷为了能攫取四川钱息之利,甚至规定了运铜钱进川都是犯法之举。

——也因此,蜀地才率先有了交子这种纸币的出现。铁钱实在太重,而且铜钱和铁钱的交换比通常是一比三到一比五之间。同样购买一件商品,用铜钱和用铁钱,能相差五六倍的重量。蜀中商人为了能便于携带钱钞,才会开始使用交子来代替铁钱。

而陕西,因为跟川中接壤,同时又是耗费钱税的大户,铜钱不敷使用,也便算是半个铁钱区。如今是铜钱铁钱同时通行,许多时候,还是以铁钱为主。

当年在元昊起兵叛乱的时候,为了补充军费,陕西甚至还发行了当十大钱。不过因为这摆明了是在剥削民财,只比铁质小平钱重不了多少的当十大钱,当然在市面无人使用,反倒引来许多伪造当十大钱,这自是让当十大钱更难通行于世。

有鉴于此,朝廷便不得不下令将之贬值,先转为当三大钱,见仍是无法流通,又不得不转为当二大钱。至此大钱回归本值,用小平钱改铸也失去了足够的利润,方才开始流通。

相对于后方能提供的刀枪剑戟,缘边安抚司更渴求足够的钱粮补充,尤其是能在当地直接出产,而不是因为后方的转运而消耗大半——这种期盼,朝廷和天子都是一般。要不然,屯田和市易就不会这么受到看重。韩冈的父亲韩千六也不会因为屯田有功,而得到了赠官。

高遵裕想得明白,若河湟之地真的有了钱监,这对平戎一事有着难以估量的帮助。

只是如果将缘边安抚司的关注焦点放到岷州,这就意味着战略方向的暂时转移。

要分兵攻打岷州,并且还要在州中设立钱监,那就意味着道路、寨堡、驻军、矿场、工坊等一系列需要耗费大量钱粮的先期投入,以及配属的工匠、矿工和军队,都要消耗大量的资源。而且就算能满足这一系列的条件,等到正式出产铁钱,多半就要一两年后了。

这就有些太过耽搁时间了,还不如用着后方送来的钱粮,解决河州木征,进而慑服后面的董毡,顺便再将禹臧花麻的爪子给剁了去。

高遵裕方才听了韩冈和王韶的话,一下激动得有些头晕,但现在冷静下来,心中默算着:“铁钱大钱一贯十五斤,小钱一贯十二斤。如果以百万斤生铁计算,即便不连火耗,岷州一年也只能出产七八万贯铁钱。相当于两万贯左右的铜钱。这是不是少了点?”

韩冈嗯了一声,点了点头,“百万斤的年产量这只是预计而已,实际如何,下官并不能太确定。有可能多,有可能少……可终究还是一项财源。说不定运气好的时候,一年三五十万贯也有可能。”

高遵裕先是有些发楞,可当他看着韩冈脸上浮浅的笑容,一下明白过来,“……这是说给朝廷听的?!”

韩冈笑着不答,王韶却没什么忌讳,道:“如果朝廷听说在岷州设立钱监,一年能产四五十万贯铁钱,天子岂有不乐之理。而我们便可以有足够的理由,向朝廷申请更多的钱粮,朝堂上反对声应该也会小上许多。”

画上一块漂亮的大饼,而让人追加投资。这样的做法,后世很常见,已经近乎于骗术。但偏偏很是管用,只要描绘的蓝图足够吸引人,那就能成功骗取更多的投资。

这个时代也是一般常见,比如王韶的平戎策,比如种谔的横山攻略,哪一桩不是向天子画出了一个美丽的未来。王安石的新法,也何尝不是先给赵顼看到了让他心动的前景,才得到了他的鼎力支持。

至于最后的结果如何,并不是现在苦于钱粮不足的缘边安抚司需要担心的——赵顼就算想给河湟下拨更多的补给,也得征求三司和秦凤转运司的意见,如果两边反对,就算内库都不一定能动得了。韩冈和王韶这是给赵顼和政事堂一个充分的理由,加大对河湟的投资——只要日后真的有铁钱产出,少上一点都没关系,或是用战功来代替,如果能顺利的解决河州木征,岷州的事更不会有人提了。

“玉昆,怎么想起了这个主意?”王韶笑着问韩冈。

“早上正好看了一下帐册,当真花钱如流水,满篇红字看得触目惊心。恰巧又从瞎吴叱的一个亲信那里听说了此事,在瞎吴叱把岷州让给结吴延征前,他正管岷州的铁器。”

韩冈的话正好是王韶方才说过了,王韶跟高遵裕对视一眼,摇头而笑,道:“倒是个会效顺朝廷的人。这也算是个功劳,到时给他报上去就是。”

“下官转头就把他的姓名年甲要过来。”

高遵裕忽又问道:“木征在洮水对面几里的地方也在增修一座寨堡,玉昆你过来的时候有没有看到?”

“下官是从东面来……怎么可能看到。”韩冈摊了摊手,又奇怪的问道:“木征是怎么想的?感觉有些莫名其妙。”

“谁说不是!”高遵裕心有戚戚焉,“五六具霹雳砲齐发,什么堡子破不了?”

“安抚是准备占下那座寨堡?”

“攻下好说,就是派兵驻守麻烦。现今光是守住临洮就至少要有五千兵马,哪有多余的兵力。”王韶插话进来,一笔一笔的算着,“为了守住临洮城,城中就要驻扎进三千兵马,才能算安稳。庆平堡和野人关——现在改名做通谷堡了——这两座兵站,扼守着临洮向东联通渭源的要道,得保证各有一个满编的马军指挥。还没修筑的南关堡、北关堡,是临洮南北门户,同样要保证各有一个指挥的兵力。

单是这几处,就要五千兵马。如果再去控制,以那里与洮水的距离,不放上一千兵,怎么都不能让人安心。还不如在临洮城对面,直接贴着洮水西岸设堡,只要两百人就足够了。”

高遵裕方才没跟王韶商量好,听着就有些皱眉,“难道就放着不成?”

“下官也是觉得还是拔掉得好。等临洮城完工后,正好洮水冻透,那时就直接杀过去。木征就算有多少盘算怕也是没有办法了。”韩冈笑道:“不管木征他们在想些什么,剩下的就让包约【瞎药】自己去处理。杀也好、抢也好,都是青唐部的事。为了这片地,相信他会拼命。”

缘边安抚司从一开始就没有分兵控制整个武胜军蕃部的意图,而是将这里的蕃部都转交给包约管理。只看宋军如何修筑临洮周边的寨堡群,就知道王韶他们的心思,就仅仅是放在保住临洮城和洮水的控制权上。

为了明天夏收前后攻取河州的行动,要事先在临洮积存粮秣军资。之后就是向西攻打河州,只要保住临洮这一小段的稳定,守住征战大军的后路,武胜北部靠着兰州的那一片地盘,就让包约跟禹臧花麻争夺去。

“那到底要不要打岷州?”高遵裕转过头来,又问起岷州的事。

“下官觉得,此时正好结吴延征败亡,瞎吴叱又在我们手中,攻取岷州不须太大气力。甚至只要留着铁矿,好用来设立钱监,其他地方,都可以暂时不加理会。”

“玉昆你的意思是先占着再说?”

“也省得河州的木征,派兵从岷州绕道,来骚扰渭源或是武胜军南部。”

其实韩冈现在有个想法,为什么一定要攻取河州?

眼下西夏受挫严重,短时间内没有重启战端的能力,若是能在这段时间中,乘隙攻取兰州,对宋夏两国之间的战略形势,能有更进一步的改善。

若是能与木征暗中达成协议,以攻打河州为幌子,把明年的战略目标改为北上攻取兰州,应该能打禹臧花麻一个措手不及。

仔细想想,这个方案很有可能会实现,禹臧花麻根本支持不住官军和木征的同时进攻。只是接下来就要面对西夏人的反扑的,木征甚至董毡会不会在身后插上一刀,韩冈都没把握。

韩冈摇头失笑。

如果能控制河州,大宋在河湟势力稳固,加上屯田市易,即便是攻打兰州受挫,也不会损伤根基。但若是换成是冒险失败,整个河湟大局,都会像横山攻略一般,十年八年都缓不过气来。

何况王韶是靠平戎策上台,突然间改变策略,这不是让他难看吗?

他又摇了摇头,冒进还是要不得的。

他在这里想着,王韶和高遵裕正看着沙盘。

瞎吴叱、结吴延征一死一擒,临洮城已经即将完工。木征又无意东进,禹臧花麻甚至把精力放在了武胜军北部,今年冬天的这一场武胜之战其实已经到了尾声。

在算得上顺利的今次作战中,如何为明年的决战做好准备,就是他们现在要考虑到事情了。

谁来守临洮?

