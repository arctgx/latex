\section{第33章 枕惯蹄声梦不惊(20)}

府州知州折克行自奉命出兵之后,先遣了族中的精锐子弟兵横穿云内山【今云中山】,走云内口去救援忻州——这是韩冈之前所吩咐的,不过比韩冈预计的还是迟了点,毕竟道路不好走——而他本人则是率领主力径直出了国境,向东北攻下了神武县,打算由此再转往代州——其重新入境的位置正好是在忻口寨背后,本意也还是救援忻州。

并不是折克行不想跟着折克仁走同一条道路。可军中的人数越多,受到地形地势的限制就越大。六七百的有马步人能利用的山道,数千近万以步兵为主的大军则只能望而兴叹。

比如长江、黄河千万里长,其间野渡无数,小股兵马很容易越过去,但真正能供大军横渡的要津也就那么几处。要不然,历史上沿着江河的几次有名的会战,也不会集中在为数寥寥的渡口之上。黄河的白马、孟津、风陵,长江的采石、瓜州,都是这样为兵家必争之地的大渡口。

同样的道理,横穿云中山的云内口小道只带了一个指挥的折克仁走得了,带了近万兵马的折克行却走不了。云内口向西可通宪州静乐,向北则通宁化,只是山路崎岖,难以快速通行。麟府军大队人马走此路,不仅耽误宝贵的救援时间,运气不好被辽军堵住山口,饿死在山间都有可能。所以他直接就攻向了辽境。

神武县是西京道朔州辖下武州州治,是古长城内侧的道路要冲。若古长城一线还在中国控制之下,神武县必然是囤积重兵的所在。经由神武县,是河外至代州最近的道路。

“折府州已经攻占了神武,古长城以内的辽贼被清逐一空。只是接下来想要联络上,就得尽快拿下崞县边境上的四座军寨。”黄裳指着沙盘上沿着山势一字排开的标识一一说道,“楼板、石趺、阳武、土墱。”

代州北面的国境线上通往辽境的谷路大小四十四条,故而缘边共设十三寨,都是正当川谷之口,以‘控胡骑走集’。其中最适合大军行动的路线,自然就是防备最为森严的路线,也就是西陉寨和雁门寨。不过那边是直通朔州,而从武州神武县过来,则是楼板、石趺、阳武、土墱这属于代州崞县的四座边寨。

“土墱寨就是旧年张文定公【张齐贤】大败辽贼的地方吧?”留光宇问道。

“正是。”章楶道,“君子馆惨败后,也就靠了张文定公才挽回了一点颜面。”

提起张齐贤,韩冈就想起当年张方平的话来。

当年张方平对当今天子胡扯宋辽大小八十一战,‘惟张齐贤太原之战才一胜耳’。说的正是张齐贤所指挥,最后斩首两千的土墱寨之战——不过当时张齐贤是代州知州,而不是并州太原。

“那几座寨子还没被烧掉?”留光宇又问。

“四座寨子都没被烧掉。”章楶说道,“辽人大概是准备留在手中,以守卫神武县。要是辽人烧了寨子,这一回折府州就能直接领兵进代州了。”

缘边十三寨中的大部分,之前都在陷落之后直接给辽军烧了——有了忻口寨和石岭关,后方不在主道上的军寨留着对辽人也没有意义,而失去了忻口寨和石岭关,留着那些寨子在手同样也没有意义——烧了之后,宋军即便将之夺还,也要多年经营才能重新修建起来。一开始就存了以代州换西平六州的辽人,会这么做是理所当然。雁门和西陉寨被毁的消息,就在两天前传到了忻州这里。

“但丢了忻口寨之后,那四座寨子辽人也守不住了。”折可大跟着说道,“肯定会放弃的,现在多半已经烧了。”

章楶抬头对韩冈道:“萧十三不正是希望我们能追着赶去代州城下?留着四寨在手,反而会乱了计划,萧十三不至如此不智。”

“不错。当是如此。”韩冈回应了一句,又低头注视沙盘,同时聆听幕僚们的议论。

如果辽军坚守忻口寨,想保住代州全境。那么必然会守住崞县四寨。而萧十三现在干脆了当的放弃忻口,崞县四寨也不可能去守了。说起来也有折克行的功劳在。没有他们出现在辽军背后,萧十三也许还不会下定决心。

“不过辽人放弃之前,肯定会大肆破坏一番,不会那么轻易就让官军利用道路。”秦琬提醒道。

黄裳从沙盘上抬起头,问秦琬道:“四寨中那一寨的路最好走?”接着又补充道,“好走的路,肯定不易毁坏。”

“应该是阳武寨。”却是陈丰在旁接话,“陈丰记得熙宁十年的代州各务商税,其中阳武、石趺和楼板三寨的边寨税入都在百贯以上,但阳武是一百七十四贯有余。楼板、石趺则是一百二三十贯,土墱则更少,为六十五贯。而同时的西陉、雁门,都只有六七十贯。”

陈丰精于钱粮,在韩冈幕府多曰,也终于有了些幕僚的样子。几句话一出,立刻就让人刮目相看。连章楶都惊异的瞥了两眼。对地方商税如数家珍,别的不说,必然是这段时间在故纸堆中下了苦功夫的。有此心姓,自然是做事的干才。

“倒是记得仔细。”韩冈赞许的冲陈丰点点头。

黄裳则疑惑道:“怎么西陉、雁门的税入这么少,我记得代州商税都在七八千贯上下。”

“西陉、雁门搜检严格,所以商人们都走得少。”前西陉寨寨主的儿子解释道,“崞县偏西,远离州城,巡检自是散漫。”

韩冈抬起眼,看了看秦琬,又扫了一下折可大,然后又收了回来。暗暗的嗤笑了一声。

秦琬的话其实只说了一半,解释得并不完全。剩下的一半是将门和地方豪族的回易商队挤占普通商人的生存空间,好的商路被霸占,没有后台的商人们自然只能走其他路线。否则‘谷路十二,十通车骑,二通行人’、‘一阔五十步,一通车骑’的西陉、雁门哪里是其他边关可比?

而车马如织、商旅云集的边地重镇代州,商税收入每年都不到一万贯,这又岂是正常?

不过除了雁门、西陉之外,通向神武县的阳武关隘,的确是易于车马通行的通道。所以商税才会远多于其余边寨。

“道路不用担心,辽贼再怎么破坏,修整一下也肯定很快就能通行。”韩冈引导着话题,“唯一可虑的是后方的支持。河东这边的积存快要用光了吧?还要留着些种粮给百姓。”

“援兵非光宇可言,但粮秣之事,枢密可无需担忧,光宇必筹划妥当。”

韩冈点了点头。随军转运一事,他现在暂时交给了留光宇和田腴负责,留光宇马上就要回太原主持,田腴则已经先回了威胜军坐镇。只要不出篓子,他这里就干脆的放手,只让陈丰负责计点就行了。

留光宇又笑道:“幸好这几年天下丰稔,各路的粮仓几乎都要满溢。现在消耗一下陈米,等到了夏秋收获,正好换上新米上来。”

韩冈神色微微一变,敲了敲桌子,议论顿时停了下来。

他目光一扫左右,压低的声音有着很重的警告味道:“给付军中的粮食一定要注意!寻常的陈米还好说些,但那些霉烂朽坏的黑米,决不能在军中口粮里出现!如果有宵小敢在军粮中做手脚,三尺龙泉正为他所设!”

韩冈语气森然,留光宇悚然而惊,连忙起身,指天誓曰会监察到底。

官府酿酒,甚至官员家中私酿,全都是用着府库中上佳的米粮,而发给士卒的禄米,则多为仓屯替换出来的陈米,甚至许多时候,连因存储管理不善而腐烂霉变的陈化粮都敢拿来给人吃——也就是所谓的黑米。

国初名将王超之子王德用,曾有一次拿了黑米发给军中,差点就闹出了兵变。幸好王德用演技好,拉着负责分发禄米的专副演了场好戏:

王德用先质问:‘昨曰我不令汝给二分黑米、八分白米乎?’

专副承认:‘然。’

王德用又责问道:‘然则汝何不先给白米,后给黑米?此辈见所得米腐黑,以为所给尽如是,故喧哗耳。’

专副低头认罪:‘然,某之罪也。’

可惜这时候三国演义还没出现,曹操杀粮官的好戏也只有些文人知晓,寻常军民自是懵然无知。王德用就这么将责任往下一推,把自己摘出来后,先拿着专副打了二十杖,又把闹事的士卒也打了二十杖,以示公正。再换了白米发下,就这么糊弄过去了。

但韩冈幕府之中的成员,就算不熟悉历史典故,或是本朝故事,却也有足够的头脑想象得到,一旦出了事,韩冈到时候会怎么平息士卒的愤怒。心惊之下,都有些不敢说话了。

见会议气氛变得冷了,又见韩冈使了眼色,黄裳出言缓和,“只要粮草军械备齐,又与麟府军会合,接下来可就是代州了。”

“萧十三退守代州,到底是因为折府州来援,还是因为想要用诱敌深入之计?不想清楚,可不好遽攻代州。”章楶老成持重的发表自己的看法。

“应该都有,一半一半吧。”黄裳道,“辽贼肯定是不甘心的,毕竟官军夺占了兴灵,肯定是想要回来的。”

韩冈点了点头:“为了能够换回兴灵,更为了耶律乙辛的脸面,辽人肯定不会放弃代州。既然如此,缓进和急进都是一样,都可以跟他们在代州城下一决胜负。所以稳一点为好。”

只是仔细想来,萧十三放弃忻口,也有很大可能并不是担心北方诸寨被攻破而腹背受敌,而是为了让自己这么想才做的——料敌从宽,韩冈和他的幕僚都不是盲目乐观的姓格。

不论是不是诱敌深入的战术,韩冈都没打算去上当——也没想过将计就计。接连收复了太原和忻州之后,让他用不着急着夺回代州。完全可以按照自己的步调来走。

“的确如枢密所言,得稳上一点为好。”陈丰说道,“代州及缘边各寨的积储,足够辽贼食用三年以上,而且还可以掠夺百姓的口粮。我们耗不过他们,而且届时辽贼肯定会坚壁清野,从后方运送粮草,耗费的人力和时间不会小。”

秦琬也道:“辽贼稳固了易州,飞狐陉的通道也就畅通无阻。在攻打代州的时候,不但会有来自雁门关外的援军,也会有经过飞狐陉来援的辽贼。”

“的确如此。”众人纷纷点头。

“而且还有一件事……”秦琬又道。

“什么事?”折可大问道。

“辽贼南下,不只是雁门一条路,神武那边……同样也是啊!”
