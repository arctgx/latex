\section{第44章 文庙论文亦堂皇(四)}

【第一更,求红票,收藏】

“我的那两个同伴呢?”韩冈问着,虽然他已经可以确定刘仲武和路明的去向。

果不其然,驿丞回道:“刘官人和路学究,方才被章老员外一股脑儿请了去。章老员外还留下话,请官人回来后,往状元楼去,他已备下薄酒数杯,正翘首以待。而王大参也使人留了话,请官人今晚去他府中一叙。”

想不到自己一下变得炙手可热起来。韩冈自嘲的笑笑,低头看着手上的两份名帖。今晚要去哪里并不用想,虽然章俞儿子章惇的名声,韩冈在东京的这些天已经听了不少,可王安石的亲信比起王安石本人来,还是差了太多了。

王安石称病期间,为了表明自己强硬的态度,杜门不出,完全不见外客,据说连吕惠卿、曾布这几个得力助手也不例外。王安石现在请自己过去,肯定是已经接下了诏书,准备复出理事了。

这是好事啊,韩冈暗暗欣喜。有王安石出来支持,至少王韶那里的压力可以减小不少。

韩冈回房很快的换了身衣服,放好了章俞的名帖。同时把王安石的名帖收在袖中,准备到王府上时退回去——参知政事的名帖不是随随便便就能收下,地位不够,拿到手上就要退回。如韩冈这样的从九品选人,根本不够资格拿,照礼节肯定是要退还的。

在众人羡慕的目光中出了驿馆,韩冈当先遣了李小六去状元楼,对盛情相邀的章俞说上一声抱歉。这小子生性伶俐,状元楼又离城南驿不远,韩冈也不怕他走丢。看着李小六走远,韩冈转身在街口找了一名租马人:“去左军第一厢的太平坊。”

租马人看到韩冈,当即陪上笑脸:“官人是去王大参的府上吧?”

“你怎么知道的?”韩冈微感惊讶,内城的太平坊是达官显贵们的聚居地,有好几十户人家,他怎么知道自己是去找王安石?京城出租车司机的头脑聪明到这等地步?

租马人则笑道:“小的就在城南驿边上做买卖,虽然没运气让官人照顾到生意,还是听到了不少关于官人的消息。”

“原来如此。”韩冈点了点头,自感好笑,凡事说破就一点不出奇了。他跳上马,便挥鞭向王安石府赶去。

……………………

兴冲冲地入宫谢恩,却被赵顼拒之门外,王安石此时的心情当然好不了。但他并无空闲发怒,赵顼会做如此转变,理由不问可知——御史中丞吕公著午后赶着入宫奏事并不是个秘密。但他到底跟赵顼说了什么话,却让人颇费思量。

吕公著入宫后到底说了什么?为什么天子心情变得这么快?聚在王安石书房中的吕惠卿、曾布、章惇三人都有些心不在焉,想着同一个问题。

吕惠卿想了一阵,便不去再猜测,放弃似的自嘲的哼了一声。他虽然还是有些在意,不过并不是如曾布那样紧锁眉头的忧心。富国强兵的规划才开始,天子离不开王安石,这一点吕惠卿看得很清楚。而且他的举主如今也只能见招拆招,不可能再称病逼着皇帝表态。

章惇也是很快就放弃了去想那两个让人头痛的问题。皇城里面从来都是有谣言没秘密,明天就能知道的事,何必赶在今晚苦思冥想?

只有曾布眉头紧皱。王安石刚刚称过病,用离职来要挟天子,这一招短时间内不可能再用,到了明天,也只能照常上朝理事。但他被拒之于宫门外的模样,怕是已经传遍了东京,曾布不难想象,明天去中书,政事堂中的几位宰执,会是什么样的眼神。

“别想那么多!说说最近有什么事?”

王安石敲了敲桌案,把三名助手的注意力集中过来。他不是那种能在短时间内转换心情,变得气定神闲的人。但执拗的脾气,却让王安石越受压迫便会越发的强硬。坚定的意志和自信,是每一个政治家和改革者都必须的性格,王安石也是从不缺乏这两点。

王安石相问,章惇先开口:“三司条例司是众矢之的,在参政称病的这些天里,陈旸叔【陈升之】多次上奏要废去三司条例司。同时还反对设立中书条例司,但言两司无故事、无先例,以撤去为宜。”

曾布一声冷笑:“若不是当初陈旸叔一力支持参政和新法,又怎会让他先登上相位。想不到他当了宰相,反过身来就变了一张脸。”

章惇也笑了一下,笑容中夹着讽刺:“得鱼而忘荃。陈相公可谓是荃相。”

‘荃者所以在鱼,得鱼而忘荃。’荃就是竹笼,用竹笼捕鱼,捕到鱼后却忘了竹笼的功劳。章惇引用出自《庄子》的这句话,就是在讽刺陈升之过河拆桥,王安石听得也是一笑,心道,这章子厚还是口舌不饶人。

“三司条例司是众矢之的,日后也免不了受到最多的攻击。青苗贷和农田利害条约皆是与农有关,可不可以将两事归入司农寺?”吕惠卿提议道,又笑着加了一句,“陈旸叔总不能说把司农寺也撤去吧?”

“……吉甫这个建议很好。”王安石考虑了一下,便点头赞许,“六部九寺如今都是空有名头,却无实职。所有的事务,全都给中书门下管了。但只要名头在,重新运作起来也没人能说二话。就这么办……”王安石突然笑了笑,“只要我还在这个位子上!”

变法派的四名核心人物就这么一个问题接一个问题的讨论着,王安石闭门不出,耽误下来的政事实在不少。时间不知不觉的过去,灯油已经添过了两次。

王安石继续问着章惇关于三司条例司的事情,曾布则是专心致志的凑过去听着。吕惠卿比章惇还要了解三司条例司,也没心思听他说。坐了许久,他也累了,直了下腰,松松已经僵硬的腰骨,不经意间,却见到王安石家的一个老家人在书房外探头探脑。

吕惠卿看着暗叹,王安石御下太宽,哪有这么不懂规矩的。回头看看听得聚精会神地王安石,吕惠卿招招手,把王家的老家人唤过来轻声问道:“有什么事?”

老仆知道吕惠卿在王安石心中的地位,也不瞒他,回道:“相公找的韩官人来了,三郎正在偏厅陪着他。”

“韩官人……是韩冈?”说起‘韩’姓,吕惠卿第一个想起的是韩琦,接下来是韩绛、韩维、韩缜三兄弟。但会被王安石赶在夜中找来,又只够资格被王旁陪的,最近就只有一个从秦州来的韩冈。

老仆点了点头:“的确是叫这个名字。”

“让他再等一等。”吕惠卿吩咐道。秦州之事虽然重要,但也重要不过皇城内外的争斗。比起韩琦、文彦博、司马光、吕公著这些老奸巨猾的对手,能报出一顷四十七亩这个数字的窦舜卿,实在蠢得可爱了。王韶若是连他也斗不过,还是干脆收拾行装回乡去养老好了。

听到王家老仆转述的话,韩冈便坐下来静心等着。王安石府的偏厅空荡荡的,还有不知从何处来的诡异风声呼呼作响,火盆和油灯发出来的光跳得厉害,幸好身边有人作陪,才不显得鬼气森森。

韩冈与王旁隔着一张几案,同坐在一张长榻上。王家的下人端了茶水进来,韩冈看了他一眼,却发现还是方才的老仆。难得王家就没其他仆役了?想想方才进来的时候,韩冈也的确发现王安石府的宅院不小,但府中人气不足,许多地方都没有打理,看起来有些破败。

若是王韶那样离家在外为官的八品官倒也罢了,王安石这样的一国参政竟然只养了几个家仆,这简朴实在是难得一见,比之有名的包青天,世称的阎罗包老,也差不多。

韩冈一向尊敬清正廉洁的官员。王安石不尚奢华,不纳妾室,不好钱财,再加上他本身的才学,每一条都让韩冈肃然起敬。但这不代表他乐于与清官打交道。

但凡清官,都是些极度自信的人物,把自己的信念和原则视比天高,而强求他人与他一般遵守,说难听点,就是所谓的偏执狂。律己严,待人也一样严,韩冈了解到的包拯便是这样的人物,后世传说的海瑞也是一般,而王安石又是有名的执拗,所以他心中免不了有些忐忑,与王旁寒暄起来,就有了些顾忌。

不同于他父亲那张著名的黑脸,王旁长得并不黑,反而是皮肤白皙,而且看上去少了点血色,大概身体不太好,有些瘦弱。相对于王韶家的二郎,王安石家的二公子乍看起来并不讨人喜欢,显得很阴沉,没有少年人的神采。而且论名气,王旁也远远比不上他那位早慧的兄长。

王雱的獐旁是鹿,鹿旁是獐的轶事,与司马光砸缸,还有文彦博树洞捞球,同样是韩冈在童年时就听过的历史故事,在此时也是广为流传。而且韩冈还从王厚那里听说过,王雱十三岁时,听到一名老兵提及河湟之事,当即便说‘此可抚而有也。使西夏得之,则吾敌强而边患博矣。’论见识,王雱也是一等一的,他的弟弟肯定比不了。

说了一阵久仰大名天气真好之类的套话,王旁喝了两口茶,问道:“听韩兄口音来自关西,不知是哪一路州县?”

韩冈一听,心中生疑,‘怎么王安石一点公事都不与儿子讨论?’同时顺口答着:“在下来自秦州。蒙相公青眼,得任秦凤经略司勾当公事。今次入京,便是往流内铨递家状的。”

“秦凤?是熙河?!王韶?!”王旁声音冷不丁的尖锐了起来。

韩冈觉得王旁的口气有些不对,再想起王雱少年时便倡导熙河之役,心中便有了点猜测。他故意笑着:“还要多谢尊兄。若无尊兄首倡开拓熙河,此事也难得到相公的支持。”

不出所料,韩冈就看着王旁的脸色一路阴沉下去。韩冈暗地里为之叹息,有个太过出色的兄长,做弟弟的也免不了辛苦。

“家兄旧日也不过随口一说,早就忘了。家严用事,皆自有主张,亲族从不得预。不论是支持开拓河湟,还是提拔韩兄,都是家严自己的想法。”

“不管怎么说,韩冈都要多谢相公的支持和提拔,才能一展胸中抱负。”

“也是韩兄才华卓异,家严才会另眼相看。”

