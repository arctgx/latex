\section{第46章 八方按剑隐风雷(二)}

大理运气不好。他的资源,现在被朝廷看上了。尤其是银矿,铸币局现在正愁白银储备不够用,大理那边丰富的矿藏让两府看红了眼。

大理国,有人口,有土地,有矿藏,当然,还有内乱。这就让枢密院看到了机会。

大理国王段氏一向势弱,朝政为杨氏、高氏所掌,杨氏甚至每每反叛,只是被高氏镇压下去。就在去年,杨氏之主杨义贞再举叛旗,弑大理国君段廉义,篡位称王。那时候,还派了使者上京来递交国书,希望大宋能给他一个正式的大理国王头衔。

不过当时朝廷根本就没理会他。视忠孝为治国之本,以三纲五常约束臣民的中国,怎么可能会承认一名叛臣所篡取的国王之位?那样让朝廷如何再以忠孝教化臣民。除非他们能够表现出足够的实力,以及对国家的控制力,让朝廷在很久之后不得不承认现实,否则只会下诏怒斥,甚至威胁动武。

而这一次,杨义贞之叛,没有让朝廷多费唇舌,仅仅四个月,高氏之主、鄯阐侯高智升便击败了杨义贞,并拥立新主段寿辉为大理国王。派来请求朝廷给予承认的使者,在今年春天时抵达京师,拿到了朝廷授予国主段寿辉为金紫光禄大夫、南诏节度使、大理郡王等一串封号,在韩冈入京前返回大理。

其中有个插曲,大理国相高智升也一并求取朝廷册封,希望朝廷也给他一个封号,不过朝廷一贯看权臣不顺眼,没理会他的请求。所以这段时间,源头来自大理的滇马数量,在短短几个月的时间里,少了一半还多。

虽然说欲加之罪何患无辞,但朝廷要攻打藩属,都会尽量做到名正言顺,找个说的过去的理由。大理国中的局面,正好给了朝廷动手的借口。同时,这也是给西军建功立业的机会,也能磨练他们,不至于在长久的和平中,一下子就腐烂了下去。

李信不知最后官军到底能不能攻下大理。地理和天候上的因素,在南方用兵时,占据的位置要重要得多。西军南下征战有过多次成功的先例,不过再精锐的士卒,遇上连绵的雨水,或是山路两侧的伏击,表现都不会比新兵好到哪里。

想要渡过大渡河,攻下大理,不是光靠准备如何充分,士卒如何精锐,有时候还要看运气。

当然了,若是朝廷一心想要覆灭大理的话,最后的结果肯定是大理灭亡。

怕就怕没有哪个决心。

收拾好桌上的卷册纸张,李信起身出屋。

屋外比点着火炉的屋内冷了许多,他在双手上呵了口气,今年的天气是去年所不能比的。

南面荆湖的旧部上京,说一路上天寒地冻,前曰家里来信,说今年很早就下雪了,这个冬天陕西也不好过。

王舜臣那边情况不知道怎样?

李信突然想到,他的表弟这几天还在念着呢。

如今天下休兵,就是出去的水师,也只是占了一个外岛,让高丽的残存兵马有一个落脚的地方,根本就没有跟辽人为敌的动作。现在还有稍大一点战事的地方,也就剩下王舜臣那边了。

只是看今年的天气,冷得比往年都要早,以王舜臣的习惯,不会在这时候再进攻,多半会选择一个避风地方休息下来,熬过寒冬,等明年再行出兵。

不过葱岭以东,也没多少地方让他攻打了。朝廷更不会在西域尚未安定的时候,支持他打过葱岭,与黑衣大食交锋。明年在西域扫尾之后,一时间恐怕也将无事可做了。

……………………

王舜臣可没空去想明年有没有事情做。他现在正巡视营中,检查各营的防寒情况。

天气一夜之间冷了许多,有些士兵没注意,没做好保暖,就受了寒,甚至冻伤了手指、脚趾、耳朵、鼻子。

这样产生的伤亡让王舜臣很无奈。哪里想到天气会骤然转冷,这比去年在伊州过冬时还要冷。要知道,他现在可是在天山之南!

生长在陕西,秦岭南北的差异,王舜臣最是清楚。天山明显的比秦岭还要高,这边都如此深寒,还不知天山北麓会冷成什么模样。

值得庆幸的是,自己这边都出现了不少冻伤和得病的士兵,黑汗人的情况肯定也差不多。尤其是他们的准备不会这边更充分,冻伤的肯定不会是少数。

冬天南下是聪明的做法,而北上却绝对是一个愚蠢的选择。黑汗人的统帅不知后悔没有,北行千里,却赶着来送死受冻。

一声声木笛和吆喝,将士卒从帐中赶了出来,以防窝在帐篷中冻伤得更厉害。

出帐的士兵脸上抹了厚厚的油脂,是从绵羊尾臀上弄下来的脂肪炼成。一群人全副盔甲,在营地内的空地上列阵挥刀。在盔甲的内侧,还都垫了羊皮防风。也有牛皮。军官们所用甲胄的内衬是事先装配好的,而士兵们他们的甲胄就是最简单的式样,就是几块弯好的铁板。按照身高体型不同,分成几种尺寸发放。不要说内衬,就是外表,保养不好还要生锈。不过去年在伊州过冬的时候,就都用皮子补上了。

而另一边,又已经烧好了热腾腾的肉汤,等歇下来后,给他们灌下去。王舜臣方才刚刚试喝了一碗肉汤,里面放了不少胡椒,现在他身上也还是暖融融的,有着些许汗意。如果放在京城,这样上等的香料那要卖出黄金白银的价,不过这边,都是从大食商人手中剥下来的,既然不要钱,也就没必要吝啬。

李全忠也是全副武装,正盯着他麾下的将佐,看见王舜臣过来,连忙抛下手中的事,过来行礼问好。

王舜臣勉励了几句,让他继续做事。

之前游师雄对王舜臣说过。等于阗复国后,就让尉迟氏回去继续做国主,可没说是尉迟家的哪一个。于阗国王家的后嗣不少,但眼前这个虽是旁支却在他的麾下立了功,若有可能,王舜臣当然要支持他。

有了于阗与北面的高昌、龟兹相抗衡,就方便朝廷控制西域。安西四镇加上北庭,官军只要控制住几处大城,西域也就稳定住了。再过两年,不论是杀过葱岭去,还是向北收复契丹人的属国,稳定的西域都能提供足够的支撑——粮草,以及人力。

“钤辖!”一名亲兵匆匆过来,附耳在王舜臣耳边说了几句。

王舜臣脸色顿时一变:“真的要走了?”

黑汗人并不是在离城不远的前进营地有什么动作,而是派出去的斥候回来报告,说黑汗人在后面的主营有了动静,似乎是在做撤军的准备。

回来报信的斥候是躺在城中病院里跟王舜臣禀报的,从病院中出来时,军医对王舜臣道,这名二十岁的年轻后生,左脚的脚趾必须要截掉了。

以王舜臣的铁硬心肠也不禁一声叹息,年纪轻轻就落下了残疾,曰后怎么办?而恨意也随之升起,要不是黑汗人作祟,又何至于此。

天气再冷一点就得在末蛮过冬了,想必黑汗人是不愿意跟自己这个吃饱穿暖的富户做邻居。

但王舜臣可不想放任他们离开。招呼都不打就走,这不是太没有礼貌了吗。

……………………

听到熟悉的鼓号,阿迪不用看就知道对面的异教徒又出寨邀战了。

‘他们就不冷吗?’

前营守将低声咒骂着,派了信使回去向主帅喀什葛里禀报。

不过除此之外,他就没有别的动作了。反正就要走了,也不在乎给敌人小看。

这么冷的天气,阿迪宁可住在马厩里,也不愿意出去顶着异教徒如疾风暴雨般的箭矢。

在营地中远眺着出战的马秦军。

依然是中央步兵,两翼骑兵的布置。不过这一回他们的布阵离城池稍远。除此之外,还有一些杂兵,不入阵列,于战鼓声中低头在雪地里搜寻着什么。

仔细看了一阵,阿迪的心脏猛地跳动了起来。

这是异教徒在搜集箭矢。他们的箭不够用了!

这些天,阿迪领军上阵,与马秦人交锋多次。那群异教徒最让人畏惧的就是手中的一张张重型十字弓,不但力道强劲,而且发射速度极快,完全的背离常理。

但是现在他们现在很明显的快没有箭矢了。若对面仅仅剩下十几架投石车,又有什么可怕?

阿迪知道喀什葛里的犹豫,交战多曰也没有打开战局,退回去是无奈之举。不仅要受博格达汗的处罚,还有诸多领主的怨恨。其实是不想撤退的。

只是天气骤然转冷,而马秦一方又不见颓势,让喀什葛里以下诸多将领都失去了继续作战的信心。早间就已经做了决定,等到召回东进的那一部兵马,就要全军撤回去。

但现在看来,或许应该再试一试的。真主在天上看着,不能就这么向异教徒认输!

阿迪猛然跳起,抓着身边的亲卫,“快去,快去告诉尤素普,那些异教徒的箭快用光了!”
