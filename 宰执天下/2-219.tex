\section{第33章 旌旗西指聚虎贲(六)}

“爹爹。”看着纷纷落下的细雪,韩冈叫住了韩千六,“今年棉田的收成怎么样?”

“总共才一顷地,一亩产棉不过七八十斤。收上来后,又要去籽,又要梳理,比起缫丝要麻烦许多。丝棉三四两就能填满衣服,棉花至少一斤。”

“……慢慢来吧。”韩冈摇了摇头,果然还不到时候。

棉花种植在通远军还是第一年,自从去年韩冈让来往西域河西的商人们搜集棉种,转过年来便是一包包棉籽堆满了半间仓库。韩冈没想到自己轻飘飘的一句话,能让那些商人们忙不迭赶来奉承。

既然种籽足够,韩冈本意是先种个几亩做实验的打算便放弃了,一口气种了百亩之多。他这也是担着风险,幸好韩千六有本事,也有了。韩冈不知收成多少算是合格,城中也没人知道,七八十斤的产量是多是少,只能让商人们再去打听。

不过通远军这里不适合养鸭养鹅,不然大规模的制作羽绒服也省事。韩冈自己就有一件,里面用的是雁绒。如今市面上也有用大雁腹部绒毛做的斗篷,数量不少,但价格很高。来源不稳定,并不适合普及。真正合用的,还是能够规模化养殖的蚕和棉花。如果局限于河西,就只有棉花。

“慢慢来吧……”又叹了一声,韩冈与父亲抵达了衙门前。

进了衙门,韩冈去正厅听候命令,韩千六则是自去自家的官厅。

蔡延庆正在正厅中,王韶、高遵裕打横陪话,转运判官蔡曚也在。

“玉昆,你来得正好!”见到韩冈进来,蔡延庆连忙叫着他。

韩冈先躬身向他们行礼,然后不紧不慢的问道,“不知运使有何指教?”

“防寒的衣料,还有行军用的雨具,准备得如何?”蔡延庆急急问着,竟也是问着关于下雪后的应对。一场雪后,天气只会越来越冷,若是没有预备,军中就会多上许多无谓的损伤。

“挡雨的斗笠和蓑衣,韩冈已经事先预备好了。”韩冈回答着,气定神闲,“配发给将校军官的油布斗篷,也都在好端端的放在仓库里,前几日韩冈是再三的检查过,都没有问题。无论是隶属于通远的军队,还是来自于外面的援军,就算赶过来时没有带上这两样装具,下官也能为他们配齐——只要领头的军校签字画押,能让下官报账就行。”

韩冈的回答体现了他做事的周全,蔡延庆点点头,而王韶、高遵裕也都笑了一笑,韩冈的最后一句,算是在半开玩笑。

“那冬衣呢?”蔡曚却是冷着脸问着。

“冬衣的问题不好办!”韩冈先摇着头,他感觉着蔡曚的态度有些不对劲。心中有些疑惑,不过回答时没有一点拖延,“照旧年规矩,孟冬十月才下发丝棉。现在才九月中,今年的冬料还要半个月才能到。韩冈这里想要问一下运使和运判,能不能把参战的外路援军的配发丝棉和冬衣,不送到他们原本的驻泊之地,而是直接发到通远来?”

“不可能!”蔡延庆尚在考虑,蔡曚就已经一口否决,“漕司行事自有轨范,若是事事从权,事情就要乱了套!”

“既然如此,那也就罢了。”韩冈轻描淡写的口吻,就像是看到学生写错了一个字的先生,很是不在意。他冲着蔡曚微微一笑:“其实在征调各路援军时,诏书中已是通知了他们携带冬衣。据韩冈所知,绝大多数都携带了冬衣。只是韩冈觉得,若是能再有一两套冬衣,或是更多的丝棉,参战的将士过得更好一点。”

这几天,两路援军到来时,韩冈并不仅仅是点算人数,以便计点粮草。同时还小心的检查着十几支队伍的兵械和装具情况。他是缘边安抚司机宜,不仅仅是出谋划策,处理庶务,也有义务要为王韶判断出各军的强弱和堪用与否。韩冈和王厚辛苦了几天,基本上心中都有了底,比如冬衣、雨具,合格的将领不可能不带。

韩冈方才的提议,只不过想试探一下蔡延庆和蔡曚两人的态度。现在一看,至少有一半清楚了。韩冈看了看,王韶没什么反应,而高遵裕则冲他露出了一个赞许的微笑。

蔡延庆的私德很好。当蔡延庆来秦凤路任职时,韩冈就已经从高遵裕那里听说过。

蔡延庆是前朝宰相蔡齐的侄子,因为蔡齐一开始没有儿子,他便被过继到蔡齐的膝下。后来过了十几二十年,蔡齐终于晚年得子,蔡延庆便主动回到了自己原来的家里,并把自己的家产全数留给了他的那个年幼的堂弟,不论是自己挣得,还是蔡齐曾经给的,一点都没有留下。他这等不爱财帛的义举,在莱州乡中颇受好评。

只是蔡齐的女婿刘庠,就是前些日子跟蔡确争庭参礼的开封知府。刘庠是铁打的旧党,韩冈不知道蔡延庆的政治偏向,但好歹跟刘庠也算是亲戚,可能也差之不远。即便蔡延庆对自己看起来有结交的意思,但许多话韩冈也不敢多说。总要提个心眼,有机会便要出言试探。

但这番试探,由于蔡曚抢着出头,蔡延庆的态度仍无法确定。反倒是蔡曚的这番举动,则让韩冈确认了他的派别——又是一个旧党!要不然,说话至少也会宛转一点,‘不可能’三个字,未免强硬过头了,也不符合官场上正常的处事习惯。也只有有人想表明自己的立场,才会有如此激烈的言辞。

由于蔡曚和韩冈隐晦的交锋,使得气氛有些冷场。

蔡延庆出头缓和气氛,他问着韩冈,“玉昆,今次的随军转运由你负责,不知你有何想法?”

韩冈想了想,答道:“今次出战,不能指望因粮于敌。通远军的动静这么大,木征只要稍有头脑,都不会正面拮抗。反而要担心他命其弟瞎吴叱坚壁清野,然后绕道我军背后,威胁粮道的安全。”

“也就是说,你没把握运粮到军中?”蔡曚冷淡的问着。

韩冈权当没听出蔡曚话中的刺,答道:“从陇西到渭源的这条路并不需要担心。青唐部、纳芝临占部,还有沿途村寨中的保甲,都能护住。就是过了鸟鼠山后,直至临洮,那一段行在山谷间,很是危险。”

韩冈如此说,王韶便借口道:“到时会安排人手护卫,别的都不怕,就是粮道一定要保护好。”

只是在正厅中稍作商谈,衙门外的钟鼓楼上,鼓声响起,出兵的时间也已经到了。

苗授、王舜臣先到,一身介胄结束整齐,头盔上的红缨鲜亮如血,在王韶的案前,他们单膝拜倒。苗授双手上举,接过了王韶掷下来的令箭。

而后又有刘昌祚和姚兕领头,二十几名将校分左右罗列,整齐的站着,听候王韶的指派。

此外,包顺【俞龙珂】、包约【瞎药】还有张香儿也来了。今次王韶并没有下令让他们出兵,可欲擒故纵的态度,反而让他们主动送上门来。这也是逼不得已,王韶手上也没有多余的钱粮,既然他们主动上门,正好可以让他们自备干粮。

从这一天开始,先是苗授、王舜臣誓师出军,紧接着中军、后军便是次第而行。

赵隆领着先行挑选出的选锋,跟着王韶居于中军。刘昌祚于后军坐镇。来援的诸路兵马都安排妥当。用了一头黑牛恤鼓祭旗,王韶的帅旗扬起,浩荡大军一路向西,向着洮水两岸,直扑而去。

……………………

兰州。

夜深了,禹臧家族长的主帐中的灯火,依然亮着。瞎吴叱的信使砰砰的磕过头,卑躬屈膝的出去了,脸上带着如释重负的笑意。

禹臧花麻看着他倒退着出帐,年轻的脸上带着一丝玩味的笑意。寒风从掀开的帐帘处卷进来,带着帐内火光一阵跳动。

“花麻你要出兵!?”忍耐了半天的一个长老终于叫了起来,“上次……”

“出什么兵?”禹臧花麻的反问打断了长老的叫喊。

老头愣了神,被禹臧花麻瞪大眼睛望着。脑中糊涂起来,吃吃道:“援救瞎吴叱啊……”

“为什么?”禹臧花麻又半眯起眼,把瞎吴叱送来的一段锦绸举起来对着光看着,说话漫不经心。

长老完全糊涂了,“……花麻你不是收了瞎吴叱的礼物,答应要出兵吗?!”

“说归说,做归做。”背信弃义的话,禹臧花麻说得理直气壮,毫无半点愧色。他把锦绸丢到一边,又拿起一只银酒壶,又对着灯光照着。良工打造的纯银酒壶,在灯火下,反射着温润的光辉。“瞎吴叱还真是大方。”他赞叹着。

“那就不去救瞎吴叱?”长老问着,虽然禹臧花麻收钱不办事有些说不过去,但这个选择,让他放心许多。

禹臧花麻拍了拍手,叫来了几个亲卫:“去,传令各部,让他们整顿兵马、做好准备。”

“花麻?!”

长老惊叫着站起,他根本弄不清年轻的族长到底在想些什么,反反复复的脑中都成了一团浆糊。

禹臧花麻把送来的礼物丢到一边,悠悠然的看着帐外,“不是没机会的。”

