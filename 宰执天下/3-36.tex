\section{第13章 上元惊闻变(上)}

距离二月的礼部试越来越近,韩冈日夜攻读诗书,将几年来逐步掌握的经义典故,一点点的融会贯通,对于儒家经典的掌握,又更加精深了一层。

于此同时,针对礼部试上可能会出的题目,他也是一日一篇的做着模拟的卷子。锻炼文章别无他法,靠着手熟而已。一个月下来,韩冈行文的速度,也同样是更加得心应手,更上了一层楼。

在这段时间中,朝堂上也是有了一点变化。

前任宰相陈升之,因为王安石的建议,外放一任任满回朝后,并没有回任宰相,但却去了西府,担任枢密使一职。其与吴充同掌枢密,靠曾经担任过宰相的资历,却硬是压了吴充一头。可以想见,这个新年,吴充应该过得很是郁闷。

但另一方面,被中书预定为同判司农寺的吕惠卿,却给天子改为了检正中书五房公事。王安石有意让曾布留任在中书之中,而将司农寺另派他人执掌,但赵顼却否决了他的提议——‘翰林学士位高,不当为宰相属官’。从这一点改变来看,天子当是在向外界表明他对朝堂人事的控制力——尽管王安石能提议陈升之坐上枢密使的位置,但他决定好的任命,天子只想要改变,那就能改变。

现在没人会对此觉得奇怪了。从治平四年的年初开始,天子到如今已经做了六年的皇帝,不可能再像最开始的一两年对王安石言听计从。王安石的地位尽管依然牢固,但有心人仍可以看得出,天子越来越明显的掌控朝堂的倾向。

找这个情况下去,韩冈估计着,也许再过了一两年,天上有个异象,地上有点灾变,或者是王家的亲眷犯点错,王安石就该出外了。但这对韩冈来说并没关系,潮涨潮落乃是常理,就算是开国功臣的赵普,也同样是在政事堂进进出出好几次,王安石何能例外?

韩冈娶的王安石家的女儿——通过交换生辰八字和婚书,韩冈已经知道他未婚妻的闺名是王旖——而不是她的父亲。韩冈从来都没有过攀附王安石的想法,未来岳父的权力可以借助,却决不能依靠,这是最基本的做人原则。

而在腊月中旬的时候,慕容武听到到消息,上门来拜访韩冈。靠着他跟韩冈的关系,有着几分运气的进了王韶府邸。

聊了一阵即将到来的礼部试,慕容武也免不了要提到,最近在外界传得沸沸扬扬的韩冈与王家女儿的婚事。

从韩冈这边的到了确认,慕容武连忙站起来向韩冈贺喜。一番礼节往来之后,慕容武重新落座:“想不到传言尽是真的,现在外面嫉妒玉昆你的可有不少……”

“都是看到小弟风光的一面,没有看到小弟吃苦的时候。西北边陲,满目胡尘,小弟有多少次濒临绝境?有多少次死里逃生?如果重新回到三年前,小弟倒是想着换条轻松点的路来走。”

韩冈如今的收获,是付出来代价后的应有回报,他当然不会有任何心理负担。

慕容武叹了口气:“可外面谁又会去考虑玉昆你的辛苦呢?”

“他人想法又何必放在心上!难道思文兄你这个锁厅举人都没有人嫉妒吗?是否要一直挂在心上?”

锁厅的贡生一向在贡生中被视为另类,能在科举前就有了官身,基本上都是靠着父荫而来。获得贡生资格又远远比普通士子要轻松,当然让人心中嫉恨。而韩冈,虽然他不是靠着父荫,但一任朝官参加科举,那更是人人侧目。韩冈本人并没有多好可供攻击的地方,功劳历历在目,所以他灌园子的出身,便成了受到嘲讽的焦点。

但韩冈不在乎……那等又羡又妒的眼神,还有只能在嘴皮子上图快活的郁闷,是让他最为开心的一件事。

时间过得飞快。

鞭炮声噼噼叭叭的响着,硝烟味弥漫在东京城内城外的大街小巷之中。除夕夜,王韶领着了家中妻妾子女,在后园中祭祖上香。韩冈遥祝过父母之后,跟着王家上下一起守岁听着开宝寺塔上熙宁六年的钟声敲响。

元旦之日,韩冈依然放弃了参加正旦大朝会的机会,留在房中读书。随着上元夜的临近,天上的月亮从一弯如钩,渐渐变得丰满了起来。

年节锁印。除了中书、密院之类的重要机构需要轮班值守,让王韶难以在家休养,如王厚所在的三班院等衙门,都已经放了长假。

韩冈埋头苦读,准备着最后的冲刺,而王厚就带着弟弟妹妹们,去东京城繁华热闹的街市上四处游逛。几乎每一天回来,都要抱怨两句此时的物价,“比上个月又涨了一些。”

韩冈不理他,眼睛对着书本,随口回道:“到了腊月、正月,物价当然要涨,不涨价才奇怪。”

“外面可都是在传言是市易法施行的缘故。”

韩冈眼睛依然看着书:“比去岁时究竟高上了多少?”

“当然没有多少,市易务不是吃干饭的。但多少人又会去回忆旧时的情况?还是相信耳边的传言,归怨于王相公和市易法比较简单吧?”看到韩冈终于放开书本,投来惊异的眼神,王厚扬了扬下巴,似是有些得意,“我自己想出来的。”

韩冈抿嘴微笑。士别三日,当刮目相看,王厚的见识和判断的确是越来越出色了。他说的一点都没错,群众就是这么好煽动。物价上涨使得民间怨气升腾,只要给他们一个目标,怨气就会朝着目标蜂拥而去。

这可不是因为教化不足的缘故。就算是千年之后还不是有过因为无稽的传言,成千上万人蜂拥去买盐的笑话——那时可是普及教育已经超过几十年了。作为个体,人类可以很明智很冷静,拥有出色的判断力。可一旦处于群体之中,还能保持着独立思考能力的就很少了。

“从一开始,我就没看好市易法。阻力实在太大了,强行推行,得不偿失。”韩冈为王安石和新党的行事手段而摇头,“不知处道你听没听过狗急跳墙的这个说法?狗善奔,而不善跳,但被逼到绝境,就算是狗也还是能够越过七八尺高的院墙。

其实京城豪商们也是如此,先是均输法夺走了他们对汴河运力的控制,便民贷夺去了他们放贷取息的收入。但因为他们还有赚钱的门路,靠着盘剥外地行商,把持京中商贸,他们至少还有条活路,当时还不敢起来闹事。可市易法一出,京城豪商们都已经被逼到了悬崖边,狗急跳墙下,闹得鱼死网破也不是不可能。”

“是啊,就是这个道理。”王厚有会于心,点了点头。转而又笑问道,“玉昆,你怎么不提醒你的岳父?!”

“太迟了。市易法公布已近一年,市易务设立了也有半年的时间。该得罪的都得罪了,几十万贯的现钱也已经送到了国库中。到了这个时候,哪还有反悔的可能?只能咬牙支撑下去。也许日后市易法可以修改,却不会是现在。”

韩冈没有说下去,但想必王厚也明白,新党决不会在这个时候变更法度,否则其余法令都会受到连锁冲击。就像一条大坝,就算再单薄,在洪水来临时,也有抵挡之力。但只要有了一道缝隙,就会在洪流的冲击下一溃千里。

“你这个做女婿的还真是……”王厚摇着头,“怎么看都不跟王相公是一条心。”

“支持该支持的,反对该反对的。若小弟是个阿谀奉承之辈,王相公会招小弟为婿吗?君子和而不同,就算亲如家人也是一样。”

即便是父子之亲,也有能说和不能说的,何况他还是只是个刚刚定了亲的女婿?除非王安石主动询问,否则韩冈他何必多费唇舌。再说了,就算狗急跳墙,豪商们和他们的靠山也没有招数。

赵顼做了几年皇帝,位置早就稳了。王安石本人掌控朝局,也不是轻易就能撼动的。难不成他们还敢闹兵变?京营的士卒要有这个胆子,母猪都能上墙。豪商和他们背后的那群人,恐怕还是要到了开春之后才会闹腾起来。

只是韩冈想得简单了点。

正月十四,乃是上元前夜,正是一年一度最为热闹的时节。韩冈为了读书,没去凑那个热闹。但王家上下几乎都出去了。京城的灯会之绚丽,为天下之最。各个衙门都会聘请名匠打造灯山,互比高下。天子也会在今夜出宫观灯,与民同乐。王韶作为朝中宰辅,当然得随驾而行。

王家府邸所在的崇仁坊陡然安静了下来,远离闹市的官员府第聚集之所,现在成了东京城中,最为安宁的地方。韩冈坐在灯下,静心静气的读书。可到了后半夜,一条惊人的传闻就在东京城内外传递,也随着回到家中的王厚,传到了他的耳朵里

——当今宰相王安石,在宣德门处,竟被守门兵士给掀下了马来。

