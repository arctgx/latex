\section{第11章 五月鸣蜩闻羌曲(二)}

【第三更,求红票,收藏】

这是一场王韶日夜期盼的胜利,但首开胜利的人却是出乎他意料之外。

董裕所领万余人马劫掠七部后,沿渭水回返。于是日黄昏时,在荒石谷西六里处,被青唐部瞎药率军偷袭得手,而后青唐部族长俞龙珂主力齐至,全灭董裕本部,斩首一千一百余级,溺死于渭水者无数,而罪魁董裕、结吴叱腊亦已授首。

“这是大捷啊!”高遵裕仰天长笑,把几天来的郁气一股脑的笑出了心底。虽然他立刻想起这样实在有失形象,竭力恢复平静,但嘴角仍忍不住翘了起来,连声对王韶说道:“韩玉昆做得好,韩玉昆做得好!”

高遵裕这两天在古渭寨亲眼看到了董裕的炎炎凶焰,早就不再幻想今次能把他怎么样,只想着韩冈能撺掇着青唐部至少跟董裕打一仗,弄几个斩首回来,让他和王韶挽回一下颜面。谁能想到,韩冈和青唐部最后竟然给了他这么大的一个惊喜。

“韩冈果然是个人才!”高遵裕现在对韩冈是赞不绝口。

“嗯,玉昆他做得是不错。”王韶点点头,附和得有些言不由衷。

能说动青唐部的俞龙珂,让他抄截董裕后路,最后竟然还让他成功了。除非这个胜利是个假消息,不然当然得说韩冈做得不错。而这份韩冈让亲卫连夜带回古渭的捷报,听起来一点问题都没有。有一千一百级斩首,还拿到了两个罪魁的首级,这事做不出假来——韩冈都让报信的亲卫带回了董裕的头盔以及他麾下两个有名首酋的脑袋。

可是王韶还是发现了这份捷报中的问题。

对于今次董裕敢于率大军深入青渭,而丝毫不顾忌青唐部的颜面,王韶也曾想过其中的问题。要么是俞龙珂默认了他的行动,要么就是董裕在青唐部有个实力并不比俞龙珂逊色多少的支持者——除了瞎药不会有别人。

俞龙珂是不会出卖青唐部在青渭的利益,这对他一点好处都没有,他已经是青渭排名第一的蕃部的主人,让董裕在青渭肆意妄为,只会有损他的声望,从情理上说,俞龙珂不可能与董裕达成协议,只有始终觊觎兄长之位的瞎药有着铤而走险的理由。

瞎药的野心在青渭已经是司马昭之心,路人皆知。王韶不止一次考虑过利用瞎药和俞龙珂的矛盾去收服他们中的一个,只是韩冈却说没必要去用什么计策,直接压服他们就可以了,不拿他们作伐,其他地方的蕃人不易心服。

既然如此,捷报中说瞎药先打,俞龙珂后至的战报就耐人寻味了。凭借对于蕃部事务的了解,王韶很容易就看出了些许不对劲的地方,也大略的推断出真实的情况——大概这场战功是给瞎药抢在头里得去了,而董裕和俞龙珂都被他玩弄于股掌之上。

“这是今上即位以来的第一功!”高遵裕依然兴奋的说着,“报上去后,天子定然欣喜。”

王韶摇摇头:“青唐部并不是宋臣,这个功劳真的要计较起来,也算不得是我们的。不像七部,已经纳土归顺了,他们的战功,就是我们的战功。”

“让俞龙珂上表归附不就成了。”高遵裕说得很轻松,“厚加封赏,他怎么会不愿意?朝廷从来不会亏待人。”

“封赏太重可不好,只是斩了董裕这只小虾,后面还有木征那条大鱼。现在赏得重了,日后再拿什么给他们。”

高遵裕心有不快:“难道这次大捷不能报上去,为他们请功?”

“报,当然要报。”突然醒悟过来的王韶立刻说道。

这事谁会知道?!

王韶看了看虽然脸色怏怏,却犹沉浸在狂喜之中的高遵裕,连他这个同提举秦州西路蕃部,也不清楚青唐部中的内情,又有几人能看破。

反正秦州上下,除了像自己这样深悉古渭蕃部内情的人物,也不会有几个官员能知道俞龙珂和瞎药几乎势不两立的情况。

外人只会如高遵裕一样,把这场大捷,当作是王韶、高遵裕决断,韩冈领命而行,被说服的青唐部族长尽起族中大军,将来犯之贼悉数斩于马下的胜利。在给朝廷的捷报上,王韶也会这么去写——也许向宝清楚,也许还会说出来。但他一个中过风的武将,现在担心自家事还来不及。攻击把他气成中风的仇家,他的话,又有谁会相信?

而就算不纳土献籍,青唐部把董裕斩了却是事实。趁着古渭寨兵力微薄的机会来犯,毁了附宋七部的罪魁都没能逃脱,谁也不能说他王韶失败了。而且青唐部出战时,他派出去的韩冈一直跟着青唐部族长身边,这件事,谁也无法否认。

同时王韶也不信,以朝廷对战功的慷慨,俞龙珂和瞎药能对此毫不动心。蕃人不知忠义孝悌,却是看重财帛利益得紧,既然如此,诱之以利,自然是无往而不利。

“你好生去休息吧,这几天都辛苦了。”王韶对着半跪在下面的亲卫,示意他回去休息。亲卫谢过恩,磕了一个头,领命出去了。

转过来,王韶笑道:“既然要为之报功,就得好好想想该怎么写,才能让天子看得出我们在蕃部中打滚的这群人的辛苦,也好给我们多一点支持。”

“说得是!说得是!”高遵裕现在乐得都不会说不,笑得见牙不见眼,才到秦州没几天,就分了这么大的一份功劳,他哪能不欣喜如狂。

又商议了一阵这请功奏章该如何写,高遵裕连连打起哈欠。被董裕折磨了三四天,现在终于听到捷报,心情放松之下,体内的疲累便如潮水般涌了上来。向王韶告了罪,他便回房休息去了。

高遵裕出去了,王韶独坐在官厅中。此时捷报已经通传寨内,只听着欢呼声从南传到北,又自东传到西。压抑许久的心情,终于彻底迸发了出来。董裕的军队在他们面前耀武扬威了几日,现在听到他被砍了脑袋,自是要宣泄一下。

听着外面欢呼雀跃的声音,王韶突然想着,万一韩冈传回来的捷报,是个假消息,不知寨内的士兵又会如何。只是这个念头闪了一下,就给他笑着摁下去了——董裕死了当是事实,韩冈行事虽精进勇决,却不是信口开河之辈,逢上大事尤其沉稳,他说董裕死了,自然不当有假。

如今王韶是喜忧难分。

附宋七部被灭,等于打断了他在青渭的左膀右臂,日后想在青渭把话说大声一点,又得费心费力了。尤其是纳芝临占部,他们对朝廷忠心耿耿,又早早的归附,就是如宋人一般。今次遭受灭族之厄,连吹莽城都被焚毁,让王韶也是深感愧疚。

但今次青唐部斩了董裕,又斩首一千一百级,正如高遵裕所说,是当今天子登基以来边功第一。只要俞龙珂肯对宋廷献籍纳土,甚至只要装装样子,这个功劳就能算在他王韶和高遵裕的头上。在天子面前,他的地位将水涨船高,而河湟之事也自然能得到更多的支持。

‘至少,得把屯田和市易的本金给我拨下来,’王韶恨恨地想着。他到秦州都两年了,从一开始就说着要屯田,要市易,要开榷场,要茶马互市,但到现在,连天子和王安石都是空点头,一点实际都没有。让他在秦州打饥荒,也得看李师中肯不肯给!现在好了,有了前次和今次两份大功摆在御前,政事堂也该大方一点。

说起来,关于古渭立军的奏章也应该能从政事堂被翻出来了。当初为了跟李师中争胜,他把古渭立军的建议呈了上去。而后却因为秦州荒田之争,当初他和韩冈一起商定的计划,连他们自己都忘掉了。如今重新提起,反对的声音肯定还在,但自己说话的声音却已经大了许多。

天子当是还想继续看到河湟开边之事上的节节胜利,想来也不会再让人阻挠自己行事,解开李、窦之辈给自己的束缚,让自己可以放手施为,一展胸中抱负。

而一旦古渭建军,他就真正拥有了军政两方面的权力,财权也不再受到秦州的束缚。所有准备已久的计划、措施、手段,都可以施展出来。这让已经缚手缚脚多年的王韶心动不已。

多亏了这两场连续的胜利。

王韶突然又想起,这两场大战的胜利,很大一部分的功劳都要算到韩冈的头上。没有韩冈的建议,他就不会连夜赶去古渭,团聚七部攻打托硕。而没有韩玉昆连夜入青唐部,也不会有如今的胜利。

现在想来,韩冈的确是个人才,这个灌园之子到底让他惊讶了多少次,王韶自己都数不清了。连王安石给他的信中都赞许有加,只是信中王安石又隐隐约约的提醒他要对韩冈稍加注意。

连一国参政都对他有了几分顾忌,可以想见,韩冈在京城中不知又做了什么大事。王韶自认不如王安石远矣,王大参都顾忌的人物,自家难道能稳稳地控制?

而韩冈出的主意,又将向宝气成了中风,这也不知是多少人因他而坏了身家性命和前程。故而自踏平托硕部之后,王韶一直都在忧心着自己到底还能不能驾驭得了破家灭门的韩玉昆。

‘先用着再说吧……’王韶心神不宁的想着,却又自嘲笑起,‘器量毕竟还是不够啊。’

