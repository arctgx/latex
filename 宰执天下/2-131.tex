\section{第27章 京师望远只千里(一)}

【第三更。求红票。】

如火如荼的气氛,从观众席一直燃烧到球场上。

一次争抢之后,收拾了伤口,重新上场的鲁平越发的急躁心情让他失去了原本娴熟的技巧,很快就有被人撞翻在地。从地上翻起身起来,鲁平便握紧拳头,正要上前讨个说法,乌克博已经冲了上前。一拳便瞄准撞翻了鲁平的对手砸了过去。

鲁平惊讶得瞪大了眼睛,他怎么也没想到乌克博会为他出头。只是当鲁平看到乌克博被人还手打翻回来时,他便大吼一声,握起拳头冲了过去。转眼之间,小小的冲突就变成了一场群架。观众们一下激烈起来的助威声中,裁判嘴里的木笛滴滴的尖叫着,冲上前把扭打在一起的一群人硬是给分了开来。

看到这一切,高遵裕扭头对韩冈笑道:“难怪玉昆你要设个裁判……是叫这个名字吧……没人上去拦着,打起来就停不了手了。”

韩冈摇了摇头,对高遵裕无奈的笑道:“火气太盛了也不好啊……”

群殴结束了,而比赛继续进行。欢呼声依然炽烈,如同酷暑时的户外,热力的确是一浪接着一浪。

对于韩冈做法,高遵裕已经看出了端倪,所谓化解蕃汉矛盾的打算,恐怕都是假的。本质上还是打算用蹴鞠锻炼其看好的下属。所以韩冈越严厉,高遵裕就越开心,韩冈的手下,可就是他的手下,而且分布面越广越好:“玉昆,这场比赛的确是还了蹴鞠练兵之法的真面目。但如果只是局限于疗养院中,是不是太可惜了一点?”

对于高遵裕的疑问,韩冈早有定计,“现今古渭城外每月逢五有集市,逢十五则是大集。如果今次安抚能同意连蕃部都组织齐云社,一起参加比赛。下官打算就把球场设在在榷场旁边的空地上。逢五的日子举行球赛,可以让每一个球员与来赶集的民众们打好关系。”

韩冈打算把附近所有的蕃部部族一网打尽,都让他们设立蹴鞠球队,到时候就可以举办蹴鞠联赛,当比赛有了利益,理所当然的便会带来足够充分的人际交往。

“蕃人可以带队参赛?”第五丰摆脱了沉默。问着韩冈。

“蕃部、汉军一视同仁。从今天的情况来看,正常的一场比赛,少说也会有三四千人观众,都比起普通的集市都要热闹,如果以一张门票十文钱的价格卖票入场,就已经是不小的一份收入。而且另外再加上让观众们吃喝玩乐的收入,也不会少到哪里去,至少能做到收支平衡。”

韩冈回着第五丰的话,顺便将后世的一些营销手段向高遵裕做了初步的解说。高遵裕不由得感叹:“玉昆……你去不做生意实在太可惜了。”

“入则为将相,出则做陶朱。范蠡助勾践复国灭吴。最后功成身隐,携美泛舟五湖之上,千年之后,追忆古今,范大夫的眼光行动的确让人钦慕不已。”韩冈不是口中说说,而是真心的感到范蠡值得他去佩服。

“可千年前后,也就出了一个范蠡。”

比赛已经渐渐接近尾声,因为没有守门员的缘故,比赛的分数两边都是上了两位数。最后的结果应该也不会大的改变。韩冈已经把三十贯花红准备好了,胜利者能分到其中的六份之五,而剩下的人却只有六分之一。为了争夺着高额的花红,球场上的局面更加火爆起来。无论是观众还是球员,都是用尽了气力为自己喜欢的球队去拼命、去助威。

王家的老仆这时突然不知从哪里冒出来,把王厚叫了下去。片刻之后王厚回来时已经变得脸色沉重,不知为何眼眶也红了。他扯过韩冈,避开众人的耳目,头低了半天,这才说道:“……我那表妹命乖福薄,不能与君……齐眉举案……”

韩冈有了点不妙的预感:“难道……”

“三个月前……染了时疫……连着舅父一同……”王厚说着说着一下哽咽起来,俗谚道见舅如见娘,他亲娘早亡,舅舅就是娘家最亲的人,但现在连亲舅舅都病死了。到时候王厚的娘家恐怕就是再没有足够的人才,来维护他们族中的关系。

韩冈一时不知该如何反应是好,聘妻和未来的岳父因病故世,他理因恸哭几声。但两人都是他从来没有见过面的陌生人,又没有正式成婚,还不到哭丧的地步,到最后,也只能五味杂成的说一声:“是吗……”就此了事。

但很快,又是一桩突如其来的大事向韩冈冲击过来。

一名胥吏匆匆跑进校场,在点将台下被护卫拦了下来。一番争执之后,胥吏递上了一卷文书,红色丝带扎起,加之鲜红的蜡印封记,代表这是政事堂下发的公文。高遵裕打开了一看,神色变得很古怪。韩冈被他叫过来:“中书门下移文,召玉昆你即日入京。”

……………………

世所常言,中年三大乐事是升官发财死老婆。

但韩冈过了年才二十岁,心境虽然有着中年人的沧桑,也绝不可能因为未过门的妻子往生而感到欣喜,而是分外感到人命的脆弱。在医药技术发达的千年之后,在有着完整的医疗体系的国度,不论是哪种爆发性的传染病,都不可能就这么轻易的夺取人的性命。

三个月……时疫……

夏天的时疫,多半是在洪水后爆发。只要拥有洁净的饮食,干净的住所,这时疫其实完全可以得到预防。但就是有人没有撑过去。

王厚望着窗外的因冬天的到来而变得稀薄起来的阳光,追忆着过去在家乡度过的岁月:“我那表妹比我小了七岁,其实只是在小时候见过。她自幼懂事,知书达理,是个难得的女子。”

韩冈随口应着,他现在还不知该怎么把这个消息,知会自己的父母。还有王韶那边,不知是派人加急去京城通知,还是等他回来再说。而且韩冈和王家的关系原本已是姻亲,但现在却又倒退回去,没有多少关系保证两家日后的紧密联系。

如果是妻子先过世,丈夫要为之守丧一年或是半年。而韩冈这边根本是毫无瓜葛,要去服丧就实在是太过了。韩冈不会去做,但他现在也的确没有了跟人定下婚约的打算。“等上一年再说,此事小弟不想太急。”

而王厚这边,他的确没有放弃用婚姻把韩冈与王家联系起来的打算。只是先死了一个,不可能立刻再送一个过来,和亲都没这么勤快。总得等些日子,双方都要留些脸面下来。

而韩冈既然承诺会等上一年,王厚就不是很担心他会背叛自己的父亲。王厚了解韩冈,他虽然智计百出,心狠手辣起来也是百无避忌,但本质上还是重情义的那种人。韩冈受教于张载,当听说张横渠辞官归乡,要修书院、设井田,便立刻把受到的赏赐分了一半给他送过去。以韩冈的为人,就算宰相来做媒,怕也是会给他顶掉。

不再去想伤心事、烦心事,王厚问着韩冈,今次去京师是好事还是坏事。

韩冈笑道:“小弟这一年来忝附骥尾,略有微功。今次见招于东府,想必不会是坏事。又不是割据藩镇的节度使,如果小弟犯了事,直接移文秦州或是提点刑狱,根本不需大费周折,调小弟入京。”

“……说的也是。”王厚木楞楞的点着头,不知他到底听明白了几句。

其实王厚的才智虽然略逊于韩冈,但对于朝中内情、故事都了如指掌,应该很容易就想得到这一点,而且应该比韩冈还快才是。看他眼下的模样,今天的消息给他的打击,肯定不小。

韩冈拍了拍王厚的肩膀,他的心情虽然不可能像失去了亲人那样悲痛,但总之也不是很好。而对于来自京中的莫名其妙的召唤,他倒没有去想太多。虽然王韶如今就在东京城中,这份堂扎应该也跟他脱不了干系,但韩冈没指望他能派人回来通风报信。

政事堂的公文皆是用马递发来,从京城到古渭,也就是七八天的时间。而王韶要想把消息传回古渭,最快也至少要半个多月,不比中原、东南等交通便利之地,民间的消息传递,有时候比普通的官方驿传还要快上几分。

看到得到京中后才能见到王韶问明情况,韩冈不再去多想,只想着今次能不能就此转官?……韩冈如探自嘲的笑了起来,这是不可能的。一个合格的领导者,再怎么欣赏下属,除非能看到足够的好处,否则都不该为了一个人而破坏已经运转良好的规则。韩冈不认为自己能够让天子和政事堂为自己破例。

问明白了韩冈的态度,王厚告辞离开,他还要赶回去写信通知自己的父亲。而严素心进来收拾书房,随着她的动作,从她身上传来的淡淡香气,让韩冈略显烦躁的心情,渐次平复。探手拉过少女,缭绕在鼻端的动人香气也一下变得浓郁了起来。

