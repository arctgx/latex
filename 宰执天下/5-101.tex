\section{第11章 城下马鸣谁与守(13)}

“郭惟忠所部已进抵小红崖,今夜之前,当能将营寨修好。”

“玛克密娘山的斥候回报,发现一股贼军正向神泉寨方向行进,大约四百骑。不过其中有两三成是党项的装束。”

“韩相公、李太尉,乌龙寨急报,精移岭下的两个村子正受贼军攻击。贼军人数当在千人上下。”

“贼军一部正在宁甲村休整,其兵力在千人以下。”

自从前两日,数百阻卜贼穿过葭芦川防线,攻入晋宁军内陆劫掠。从那一天开始,从远方赶来的贼寇越来越多。

葭芦川流域诸寨堡所组成的防线在阻卜人的铁蹄下,如同薄薄的一张纸片,转眼就被冲破,并踩在脚下。

晋宁城中,一时兵荒马乱,谣言纷起。最夸张的甚至谣传到夏州的种谔已经战败深思,有三万党项和阻卜联军,正往晋宁军杀来。不论信与不信,从晋宁军东渡黄河往对岸的克胡寨去的百姓,这几天翻了好几倍。

一直到昨天午后,经略使韩冈领五百骑兵渡河而来,才将浮动的人心稍稍安定下来了一点。不过该渡河的还是渡河,并不因为韩冈的到来而有所改变。

不过眼下的晋宁城衙中,则是乱中有序。在公厅中进进出出的将校、官吏,都是脚步匆匆,却没有一丝惶急。

围在厅中沙盘边的韩冈、李宪,以及十几名将领,现在皆是聚精会神,甚至都有一股隐隐的兴奋。看着小吏依照军报,将一面面不同花色的小旗,插在蜜蜡制成的地形沙盘上。

韩冈低头盯着沙盘,头也不抬:“张世宗,进入神木寨【今陕西神木】暂避的百姓,可曾安顿好了?”

刚刚进了厅来的中年将校忙不迭的躬身回话:“禀经略,末将来晋宁之前,已经将得胜沟、禹庄两个村子的百姓安顿下来了,食宿皆已安排妥当。等这一仗结束,就可以让他们回去重修村寨。经略尽管放心。”

“这就好。”韩冈点点头,直起身,锐利的视线扫过厅中众将,“眼下贼寇乱我河东,晋宁、麟州的百姓备受其扰。本帅正要出兵进剿,这时候,决不允许有借机牟利、劫掠百姓之举。官军为民剿贼,可不要在百姓中也成了强盗。”

立刻就是一片声的应诺。来自经略使的吩咐,没有哪个将领敢于当面不加理会。

“李瑛,你的第九将左部明天之前能不能分出一个指挥进驻到窟薛岭下?这条路上只有一个指挥驻守,实在有些不放心。”

一名瘦高个的将领在韩冈面前恭声道:“经略明察,末将的宁河堡中现在就只有三个指挥,分出一个指挥后,就是剩两个指挥了……”

李宪在旁插话道::“宁河堡地势险要,驻屯三个指挥的兵马,能在万名敌军的攻势下,守住七八日。但阻卜人从不会攻城,就是只有七八百人守堡,也不用担心会被攻破。”

韩冈和李宪表达了相同意见,李瑛自然不敢再说什么:“末将这就派人回去通知。不过……”

“不过什么?”韩冈追问道。

“窟薛岭中的粮食不够,宁河堡中也缺粮。移防调兵不难,但人马的口粮如何是好?”

韩冈转过去看李宪:“都知……”

“在下现在便着人去安排。”李宪说着,从身后站在墙边的一众官僚中,点了一人处理。

待李宪吩咐过下面的将校,韩冈放心的长舒一口气。李宪陪笑道:“这一下,阻卜人当是插翅难飞了。”

韩冈摇摇头:“现在只是将几条主要的通路给堵上了,还有许多的山间小道,没办法全部落锁。”

“给阻卜贼人捡了便宜去。”李宪骂了一声:“要不是官军为了连同弥陀洞,一心守着葭芦川北,也不会让贼子捡了这个便宜去。”

韩冈看这地图,也在同时叹了一口气。河东军的主力之前就驻扎在弥陀洞到晋宁城一线,最北到麟州的神木寨为止,几处寨堡全都在葭芦川北侧。而阻卜人恣意劫掠的位置,绝大部分都是在葭芦川南侧。的确是钻了空子,捡了便宜。

不过在阻卜人钻了进来之后,韩冈考虑再三,最后还是没有下令葭芦、乌龙、神泉、宁河沿河四寨堡扎紧篱笆,而是密令前方寨堡,将口子再放大一点。同时还遣人昭告葭芦川内部各乡村、部族,提防阻卜贼寇,若是无法抵挡,可以退入各寨暂避锋芒。

之前的几十天里,来自北方草原的阻卜骑兵,如同一群蝗虫,将银夏之地,能够劫掠的部族一扫而空。正在饥渴的寻找下一头肥美的羔羊。这时候,听说了葭芦川这边还有可以劫掠的村子,就像嗅到了蜜糖气息的蚂蚁,全都涌了过来。

葭芦川南岸,一直到黄河边,就那么大的一块地。东西不足百里,南北只有五六十里,一下子涌进来三四千乃至五六千的阻卜人,一下就是处处烽烟,村村告急。但对于想要将贼人一网打尽的韩冈来说,强盗聚集起来,正好是最方便下手的时机。

跟在韩冈身边,对于他的计划了解甚深。黄裳瞅着沙盘上,葭芦川南岸星罗棋布的代表阻卜人的黄色小旗,冷笑道:“多得跟苍蝇一样,一队队的,也不知合力而行的道理。”

“勉仲兄可是糊涂了,这不就是贼虏一贯的战法?别把他们当成令行禁止的军队,只是一群结了盟约的强盗。不事生产、以劫掠营生,到了冬天,家里没存粮了,就找个富户抢一把。若是只抢一处,还不够分赃的,当然要分头去抢。抢到的财货绝不会分给其他人。既然看到有同行的部族发了财,剩下的如何能遏制自己的贪心?就是有哪个首领是聪明人,可下面的部众呢?”折可适在黄裳身旁笑道。

两人的交情这些天来渐渐的好了起来。对于折可适稍显冒犯的话,一点也没去在意,只是叹道:“举目皆贼,北虏西寇,又有交贼、阻卜贼,堂堂大宋,却没有一个堂堂正正的对手。”

“难道要学着徐禧的话,王师不鼓不成列吗?”折可适笑道,“宋襄公之后就不可能有了。”

黄裳一声感慨:“《孙子兵法》出来后就没有了。春秋无义战,孟轲之言岂是无因?”

折可适虽读过孟子,却早记不得其中的细节,“我们是官,他们是贼。官兵剿贼,这是天经地义。难道不是义?”

仅仅是韩冈的幕僚,两人在这个场合下,都是站在墙边上,低声私语并没有人在意。

在沙盘边,韩冈和李宪已经将最后一点计划敲定。

李宪回身望着众将:“做了那么多准备,现在阻卜贼寇已经被关在了葭芦川南岸的这一小片地方。官军将他们团团围住,这就叫关门打狗,可别让狗给跑了!”

韩冈笑了笑,更正道:“是关起门窗打苍蝇,一个也别给我漏掉!”

“诺!”张世宗、李瑛、苗昌等部将一抱拳,高声应答。

折可适和黄裳面面相觑,难道他们的议论给韩冈听到了?

韩冈根本就不知道自己两名幕僚的窃窃私语,转头对李宪道:“光靠大军还是不足以将贼人全数消灭,等到贼人被打散之后,清剿残寇,乡里的保甲才是最合适的人选。一个也不漏掉,没有百姓的支持,可做不到。”

“韩经略所言甚是。贼寇猖獗,这些天来有许多村子被迫离乡逃入山中。不过河东百姓皆负勇力,一旦贼寇败阵,必然会出来追杀。晋宁军的弓箭社,在河东也是有名的。”

韩冈回头看了看,晋宁军知军贾逵就在身后竖着耳朵听着,“贾逵,这件事就交给你了。要把百姓组织好,赏赐也不得克扣。”

贾逵忙不迭的回复:“下面明白,经略尽管放心。”

李宪重又低头看着沙盘,视线的落点从旗帜密布的葭芦川一路北上,麟州、府州、丰州,一直到了沙盘的另一边。看着沙盘边缘的白木框,仿佛在看着辽国上京道的广阔原野:“若是一切顺利,西阻卜当是元气大伤。如此一来北阻卜从此就有了机会。辽国的西北路招讨司、阻卜大王府,能不能挡得住北阻卜的大军?”李宪忽的嗤笑,“就是不能赢过契丹人,好歹也能给耶律乙辛添上一点麻烦。后路一乱,看他怎么平叛?”

这一次要一举将西阻卜的贼寇吃掉大半,方便北阻卜乘势而起。在辽国上京道中给耶律乙辛钉上一根钉子,韩冈前几天与冯从义议定的计划,之前也跟李宪商议过了。遣人加急送了信给李宪,争得了他的认同。否则要让李宪同意进一步放开葭芦川防线,绝不会这么容易。

不过在韩冈看来,计划归计划,实际运作的情况可不一定会有那么顺利,“这件事成与不成还得看运气,能有那就最好了。不过这不是主要目的,消灭这群阻卜强盗,让他们血债血偿!”

