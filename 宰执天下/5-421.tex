\section{第36章 沧浪歌罢濯尘缨(23)}

吕嘉问到了入夜时分才匆匆过来。

连同元随一行二十多人,一路提着灯笼,进了章敦府中。

也幸好吕嘉问还只是三司使,虽有计相的别号,但终究不入两府。来往宰辅之门,便没那么多顾忌。

将吕嘉问迎进内厅,章敦问道:“怎么这个时候才放衙,可是衙中有急务?”

“还不是得多谢子厚你!你今天不请我过府,我自己都要找上门来。”吕嘉问没好气的哼了一声,他这两年与章敦走得很近,熟不拘礼,没坐下来便开始抱怨,“枢密院好大方啊,三十万钱绢大笔一挥就送出去了……也罢,左右掏钱的不是你西府,睡不着觉的也不是你章子厚!”

章敦早就知道会有这一出。

这两天从河东回来的京营禁军因为不满意朝廷的赏赐微薄而怨声载道,虽说暂时还没出乱子,可谁都知道那些赤佬不会乖乖的偃旗息鼓。所以枢密院重新考订了赏额,给每个兵卒又加了两匹绢四贯钱。只是这么一来,便换成了已经为之前的赏格而焦头烂额的三司衙门怨声载道了。

章敦叹了一声:“也没办法,京营不安抚,京畿也安稳不了。现在只是抱怨,难道还要等兵变闹起来不成?这次赤佬,韩玉昆只敢拿他们充门面,都不敢用他们上阵。混到一个大捷,回来还有脸邀功!过些日子慢慢收拾!领头的一个都别想跑!”先把赤佬们安抚下去,然后再秋后算账,这是遇上军心动荡时一贯的流程,章敦性子再强硬,也不会自寻麻烦:“……你看韩玉昆多聪明,仗刚打完就把人给打发回来了。闹事也不会闹在他的地盘上。”

章敦祸水东引,吕嘉问却不上当:“打完仗了,当然就没他的事了。怎么定赏格,还不是你们枢密院的事。原本就已经不少了,现在一下又添了一半……韩玉昆在河东修轨道,论用兵那是没话说。稳一点总比贸然出阵败了好。要是赵王有先见之明,肯定不会拿赵括换廉颇。但这钱花得如流水啊!”

“只是暂定。政事堂那边还没说话呢。”

吕嘉问气得反而笑了起来,“暂定?暂定的事会发到三司来?韩冈能把事情推到你章子厚头上,你又能把事情推到政事堂身上,难道政事堂就不会把事情往三司推?”

“终究只是几十万贯的事,前面上百万贯都给了,现在何苦省这么一点。”

“民脂民膏是能乱花的吗?!”

吕嘉问是世家子弟,口袋从来没缺过钱。就算与家中翻了脸,也从没愁过钱财不够用的。可自从临危受命接任了三司使,他就恨不得找条河跳下去,免得再为钱烦心。

靠着老天帮忙,好不容易才有了点积蓄的国库,又变成了个无底洞。窟窿深得让吕嘉问夜里睡不着觉。在他看来,朝廷迫切需要一个能够恢复收支平衡的手段,否则接下来的几年,少不了要盘剥百姓了。现在既然还没找到,就得能节省就节省,免得日后罪名落到自己头上。

他的声音突的低了下来:“仁宗大行后四年紧接着英宗大行,国库中连犒赏群臣、三军的钱都拿不出来。没有此事,哪来的新法?”

悖逆的话吕嘉问不好说,但他言下之意章敦也能明白。

当今天子差不多也就剩一口气了,虽然仗着祖宗庇佑,这口气一直还吊着,可说不准哪天就断了。万一到时候不能让葬礼办得风风光光,被压下去的那群人可就有的话说了:

——变法十数载,什么都变了,唯独天子念兹在兹的国库没变!

这评语传出去,新党执政的根基都会因此动摇。

其实如何封赏经历战事的大军,大宋朝廷经验丰富得很。无论胜败,都会给予赏赐。先把士卒和底层军校安抚好,就是上面的将帅因封赏不足而有怨心,也不会闹出大事来。

只是现在的问题是京营禁军,战斗力不如陕西河东的同僚,可说起精明厉害会算计却是远远胜出。

河东战事从太原府一路打到辽国境内,真正与辽军奋战厮杀的主力,依然是河东本镇的兵马。京营禁军自从到河东后,一直被韩冈捏在手里面,直到最后,才与辽军有了短暂的交锋。

平心而论,京营禁军的存在,成功的逼迫辽军不敢放手一搏,时时刻刻都要提防他们的行动,也算是起到了一定的作用。但战后记功,却不会把这种功劳都计入在内。论战果、论俘获远远不如实际作战的河东兵马,斩首数目甚至还不如剿匪平乱的七千西军多。

朝廷论功行赏,京营禁军理所当然就只能拿到最基本的数目,比不上有战斗、有战果的西军和河东军。

将心比心,他们不甘心也是正常的。

没有功劳也有苦劳啊。何况他们是跟全副武装而且又凶悍无比的辽人打,不是跟那些连甲胄都装备不起的蕃人夷人战斗,是要搏命的,一个不好就会全军覆没。提着脑袋上战场,最后只拿到了些打发叫花子的钱,京营禁军一贯有闹事的传统,当然不会安分下来。

拿着章家的婢女送上的湿手巾擦了擦脸,又喝了两口绿豆百合汤,吕嘉问火气也消了点,“不说这件事了。皇后也应允了,这笔钱会从内藏库中支取,不从左藏库走。”

章敦微微一愣,昨天还没消息,今天就说通了:“不是说好了这一回发给三军的犒赏,内藏只出两百万贯,剩下的都由三司筹措,怎么又要从内藏支取了?”

左藏三库,储存的是天下州府的贡赋。钱库、金银丝绵库、生熟匹帛库,三库之中基本上就是国家储备的主体,由三司主掌。群臣、三军的俸禄,以及朝廷的日常开支皆从此处支取。

而内藏库原本则是太祖时存来准备夺回幽云诸州的封桩库,后来变成了皇帝的私房钱。但只有少部分是用于天家的开销,绝大部分的用处,是给群臣、三军的赏赐,或是灾荒时救济百姓,代表天恩,而战争时的军费很大一部分也是从内藏库支出——‘军旅、饥馑当预为之备,不可临事厚敛于民’,这便是立内藏库的目的。

当今天子近两年设立的元丰新库,就属于内藏库的范畴,‘五季失图,猃狁孔炽。艺祖造邦,思有惩艾。爰设内府,基以募士。曾孙保之,敢忘厥志?’以这首四言诗的每个字为库名的三十二间元丰库,就是为了准备日后伐辽的军费。

此外朝廷因为没钱,也时常向内藏库伸手借钱。从本质上讲,内藏库也属于国库的一部分,两府和三司想内藏库伸手要钱时,一贯是理直气壮。

对此皇帝是心有不甘的。所以内藏库都是由内侍来掌管,不许外廷插手。真宗皇帝甚至还明明白白的下诏,不许打探天子私囊里有多少钱,也严禁内部泄露——‘诏内藏库专、副以下,不得将库管钱帛数供保及与外传说,违者处斩’。

这一回宋辽开战,内藏库也是照常例出钱。半年不到,支出了近四百万贯,大半是军费,小半是给三司的借贷。打起仗来花钱如流水,一下就空了。

当然,这也跟自今天子登基之后,没有几天太平日子有关。熙宁四年的拓土横山、熙宁五年的河湟开边、熙宁七年、八年的天下旱蝗,熙宁九年、十年的南征之役,都没消停过。而进入元丰之后,又是平夏之役,以及刚刚结束的这一场与辽国的交锋。

这般折腾,存不下钱是理所当然的。国家财计能维持到现在,还是多亏了大宋的底子厚实,另一方面,也是新法的功劳。

如今和议已定,宋辽恢复旧盟,要犒赏出征三军的时候,主管内藏库的宋用臣拿出了两百万银绢后摊手说只剩下给后宫的脂粉钱了。

在北方开战的时候,两府连哄带骗,从皇后那边将内藏库的帐簿给弄到手了——尽管只是副本,可也不再像过去,只听管勾内藏库的内臣每月一报,实际情况一头雾水,连借钱都不知道可以借多少——这时候看看账簿,宋用臣说的也不是谎话。

宰辅们聚在一起商量了两天,决定不足的功赏从左藏库中支取,在账面上冲抵之前向内藏库的借款。

可是吕嘉问拆东墙补西墙,好不容易才从必不可少的各项日常开支中挤出了给予三军的封赏,正准备歇口气,却听到还要三十万钱绢,将绢也兑换成钱,就是总计近五十万贯额外开支了。

试问吕嘉问如何不跳脚?也幸好有向皇后帮了他大忙,“皇后说了,内宫可以节省一点,给三军将士的犒赏不能节省。这一回,多亏了皇后圣明……”

章敦摇摇头。

还说什么场面话啊。皇帝还在世,但也差不多等于不在了。欺负孤儿寡妇,得逞了也实在没什么好说的。
