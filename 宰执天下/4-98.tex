\section{第18章 青云为履难知足(四)}

【不好意思,迟了一点。】

王安石心力交瘁,须发在短短数月见,花白了大半。

政事上的,家事上的,还有自己身体上的,到处都是问题,一个个重担压在肩头。虽然腰背依然挺直,但面容中总是显得几分疲惫。

他的长子王雱,已经病入膏肓。

王雱即是他的儿子,也是他的学生,更是他事业上的助手。

不像背叛的曾布;也不像渐渐疏离的吕惠卿;更不像韩冈那样,自己趟着一条路,只是接近,但永远不会向着一个目标同行。

王雱是自己的继承人,不论是在学术上、事业上还是本身的身份上,都是如此。可是眼见着就要白发人送黑发人,王安石即是再豁达,也难以在此事上超脱,心情总是沉浸在伤痛之中。

“王卿。”

“臣在。”

只是在朝堂上,王安石将自己的心绪深深的掩藏起来,表露在外的,还是一名言不苟合、行不苟容的拗相公的模样。

对王安石的态度,赵顼先叹了口气,也不知道该怎么开慰。将此事放过一边,他说道:“剿平交趾,势在必行。今天又有十几份奏章,说着平交的策略。不知王卿看了没有?”

在打进广西的时候,李常杰曾经四处散发檄文,说他们的进攻是吊民伐罪,拯救为新法所苦的大宋百姓。当檄文传到京城中来的时候,还在朝野内外闹出了小小的乱子,不少人私下里说怪话。不过当李常杰丢盔弃甲的被王安石的女婿打回去后,这些闲言碎语一下就不见了,讨伐交趾一下成了主流。

王安石道:“平交之策,当征询当地守臣。苏子元即将入京,陛下可仔细询问。至于如何用兵,此前已有胜绩,沿袭即可。”

李舜举私下里已经确认了韩冈的战绩,虽然还没有到京城,但已经发了急报回来。一万多俘虏和首级是寻常关西一次大捷的十倍。只用了一千五百兵就获得了如此辉煌的成果,所以韩冈用兵的方略,也就成了可以仿效、依从的良策。

赵顼点着头:“以官军为主,蛮军为辅,过去的做法不是没用。但比起来,还是韩冈在邕州的做法更好一些。”

这也是见到了在邕州之战中,官军与蛮军互相配合的好处。以数量虽少却精锐无匹的官军为刀刃,而以数量充足、战力则逊色许多的蛮军充作刀身,这样不但能保证军队依然有着足够的战斗力,也能省下不少钱钞。

但这么搭配的前提条件,是官军必须要有远超敌军和蛮部的战斗力。否则外不能击败敌军,内不能镇服同列,只会落下一个笑料。关于这一点,王安石比谁都清楚,韩冈给他的私信中,十分清楚的指明了这一点。

“不过调集南下的大军必须要精锐。大胜交贼的四个指挥,都是从荆南军中挑选出来的精兵,如此方能以一当十,为中流砥柱。若是就近调集,必有滥竽充数者,至是则不见其功,反受其害。”

“当然是从关西调集精锐南下。”赵顼说着又皱起眉头,“就不知道茂州、横山两处,能不能及时结束。”

北方与辽、夏已是一触即发;南方与交趾则是大战即将再启;江南、淮南的灾区,正处在青黄不接的最困难的时刻。于此同时,蜀中的茂州也出了事。

成都府南面的茂州,是西南夷的地盘。本来茂州没有城墙,只有一条篱笆分割内外,城周的蛮夷经常入城劫掠财帛人口。当地的百姓饱受其苦,年年请求州中修筑城墙。所以到了去年时任知州的李琪为此上书朝廷,最后得到了准许。

只是当奏折批复下来时,李琪已经调任了,不过新任知州范百常接任后,还是照旧要筑城。但茂州建城,给周围的蛮部带来了很大的压力,不但发财的目标没了,日后还要担心官军以茂州城为基地,横扫周围蛮部以复旧日之仇。有着这个担心,当地蛮部掀起叛乱也就在情理之中,没人会怪罪范百常。可没能及时镇压下去,范百常却脱不了责任。

“茂州之事,是范百常措置失当,即知蛮贼反对,当事先做好准备,而不至于等到叛乱之后措手不及。”

“范百常当治罪,不过茂州至今未失,倒也不无微功。等王中正到了,就能里应外合。”

赵顼派了内宫中的名将王中正去协同处置,有他配合个性稳重的蔡延庆,应当能将茂州给平定下来。而王中正离开时,请旨将熙河路的兵马调去了一部分,由赵隆、苗履两人统领,此外还有八百吐蕃骑兵,都是能在山地跑的良骑。

赵顼对他们很有信心,“有王中正领军,所部又为精兵良将,区区蛮贼,指日可平。”

王安石抿紧了双唇,前面他告假养病,赵顼派出内侍同领平蛮之事都没能阻止,现在一提起来,心里就是一阵不痛快。但他也没有办法,两府已经通过了这一项任命。

一场病下来,整个人就老了许多,长子的病情是一桩,另外他对朝堂的控制力也在下降中。与他对立的照旧对立,原本亲附的则在逐渐疏离。成了执政后,吕惠卿虽说是避嫌,但他的确是与王安石可以保持着距离,已经渐渐可以算是新党中独立的一支了。

而韩绛两个月前,因为一桩小事已经出外了,说起来也是他自己在政事堂中待得没有意思,没有力争的缘故。否则他挣扎一下,还真不能拿他这位宰相怎么办。

如果将两府宰执看做一个整体,自熙宁六年之后,已经几年没有大的变化。基本上都是熟面孔,只多了一个吕惠卿。另外也就是王安石走了又来,韩绛来了又走,仅此而已。

尽管两府中两派分立,赵顼将异论相搅的一套把戏掌控得恰到好处,但看来看去,所谓流水不腐、户枢不蠹。如果此番事了,天南地北的乱事能安定下来,东西两府照常理就会有个大变动了,至少也会有两三个空缺下来。

能想到着一点的有很多,有资格在其中踏上一脚也不算少。

最有可能的几人,本人的能力、资历,在朝中的人望,以及天子的信任,互有高下,到了最后,能比的也就是功劳。

……………………

“辽国短时间内打不过来。本身内部乱事不绝,加之执掌朝政的魏王耶律乙辛,刚刚逼死在辽主耳边乱吹风的皇后萧观音,去掉了这道障碍,接下里就是正主儿太子要对付,他没胆子分心南顾。”

傍晚时分,不用当值的吕惠卿回到了家中,与他对谈如今天下时局的,仍是他的弟弟吕升卿。

“耶律乙辛撺掇着辽主与西夏联姻,本来就是为了给党项人撑腰,让他们继续南侵。倒不是安了好心。”

耶律乙辛的态度,只是刚一得知辽后萧观音的死因,朝堂上上下下就知道耶律乙辛的目标是辽国太子耶律浚。不看看耶律乙辛栽给萧皇后的罪名是什么?——与伶人通奸!亲生母亲有了这样的罪名,耶律浚的位置就很危险了。

“耶律乙辛虽然权倾朝野,但辽主可只有一个儿子。不论耶律乙辛和耶律浚孰胜孰败,两边之争,绝不会简简单单就结束。”

“能立功的地方一个是西夏,一个是交趾,茂州那边倒是不用指望了。不是王中正有本事,而是熙河军的战力远过西南夷的蛮部。”

“说来也好笑,现在连太常礼院里面的人都在讨论如何平定交趾了。真不知道他们能想出什么办法。”

“章子厚和韩玉昆的配合已见其功,李舜举又证明了他们并没有谎报战功。只要派下去的兵力不多,他们有足够的地位来统领。天子也不会冒着风险临阵换帅。想夺他们的位置,只会是痴心妄想。”吕惠卿冷笑一声,“没看王韶都不说话了嘛?之前他可是有着去交趾的打算,想着跟韩冈再配合一次。”

“大哥打算怎么办?就看着章惇和韩冈两人立功吗?”吕升卿心中有着些许忧虑。等章惇立功回来后,肯定是要进枢密院了,过个一两年就能进政事堂,到时候,就是吕惠卿最大的一个竞争者。

吕惠卿摇摇头,他不会离开东京城,但他可以举荐他人去。只要自己没走错棋,最后他还照样是接手王安石留下的遗产的第一人选。

“光有章子厚和韩玉昆是不够的,他们手下还需要领兵的大将。帅与将是两回事。运筹帷幄、统观大局,这是帅。临敌指挥,阵上厮杀,这是将。韩冈和章惇都是帅才而不是将才,之前的邕州大捷,也多是靠了李信和黄金满两人的指挥。”吕惠卿胸有成竹的笑道,“如果朝廷决定要遣军南下,不论兵力是一万还是两万,都必须要有一名地位更高的大将来统领。李信入官才几年?他的资格远远不够。”

吕升卿听明白了吕惠卿的打算:“大哥准备推荐谁?”

吕惠卿没有即时回答:“吴充本有意郭逵,不过这已经不可能了,郭逵可是远在章惇、韩冈之上,他若是统领南征行营,必然是要做主帅,而不是大将!。”

“那赵禼也不可能了。”吕升卿沉吟着,“又是文官,还是帅臣。”

对于派谁来统领南下大军,一开始的时候争论很多,统帅援军南下的章惇韩冈只是打前站而已。郭逵、王韶、赵禼等人都是榜上的热门人选。只是在邕州大捷之后,这么想的人已经寥寥无几,现在闹腾的,都是想着占些便宜而已。

“赵公才本来就不可能,不想想他守着哪一路,横山边上的位置,哪里能轻动?!”吕惠卿摇摇头,“要么是在三衙管军中选资历浅的,要么就是近年来战功煊赫的宿将。”

“燕达还是苗授?”

“主要就是燕达、苗授二人。另外,曲珍、种诂勉强也能去。”

“究竟是谁?”

“那要看天子如何定夺了。”

