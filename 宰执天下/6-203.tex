\section{第22章 鼓角连声彩云南(下)}

开启的城门,并非是出战的大军,而是请降的使者。

权相高智升之子高升泰代表其父、代表大理,出城请降。

“第三次了。”熊本对赵隆道。

“第三次了。”赵隆点头,狞笑了起来,“高升泰!”

这已经是大理国开战以来的第三次请降了。

第一次被派出来请降的,是大理国中的清平官,相当于翰林,那时候官军还没有渡过若水,所以清平官还有几分傲气,给熊本赶了回去。

第二次,也就是官军逼近洱海之后,过来的是大理朝中的九爽之一,其位相当于九卿,是高氏族人,但要求颇多,熊本还是没理会他。

除了这两次,走小路直接去向朝廷请降的使臣,光是在半道上被拦住的,就有六批,加上没拦住的,只怕有十几波。不过有韩冈、章惇把持朝政,自是不用担心有人在朝中扯后腿。

第三次,也就是眼前的这一次,不为大理,只为高氏。高智升第嫡子,高升泰终于出城来了。

大宋以讨逆为名,为段氏举兵南下,不管这个理由多么可笑,其明面上的目标就是当权的高氏无疑。

段氏或许不能保住王位,但至少能保住性命,在东京汴梁城中,少不了给他的一座府邸。可高氏又能有什么?既然宋人以权臣乱国为名来攻打大理,不族诛高氏,怎么名正言顺的结束这场战争?

官军已经兵临城下,最后一战就在眼前,大理国的命运已经注定,高氏父子已是笼中困兽,但他们如何会甘心就此走向覆灭。

这样的情况,再不挣扎一下,可就当真会身死族灭。

“大帅要去见他吗?”赵隆问道。

熊本道:“你觉得他会有什么要求?”

赵隆想了想:“保命吧。应该不会再蠢了。”

熊本呵的一声笑:“自来恩自上出,他们要做的,是等着朝廷的发落,不是讨价还价!我们是来卖菜的吗?我不见他,赵隆你自去做攻城准备,秦升,你带他去看看蕃部!”

说罢,熊本转身回帐。

赵隆和熊本点名的那位幕僚交换了一个眼神,各自依命离开。

营门处,高升泰焦躁不安的等待着。

但他的脸上不敢表现出来,依然一幅谦卑的表情。不论再如何屈辱,只要能保住性命,日后就有翻身的机会。

宋人南下时,的确是震动大理朝野。但那时,高氏父子还有几分把握,山高水长,万里路遥,这就最好的防御。但宋人一路南下,什么险关要隘都没能挡住宋军的脚步,高山湍流,宋人全都如履平地。

最早的时候,高智升和高升泰派出使者请降,是打算敷衍一下,诓得宋人退兵。之后,就是能保住权位,继而是能回乡自守,到如今,只要能保住性命,甚至只要保住家族血脉,剩下的都可以不要了。

焦躁的等待中,一名中年文官走出营地,打量了高升泰两眼,道:“可是高侯?”

高升泰一揖到地,“小人就是高升泰。”

秦升回了半礼,“本官是总管帐下机宜文字秦升,奉总管命,特来迎接。”

“小人见过秦机宜。”高升泰连忙又行了一礼。

“请高侯与本官来。”

秦升说着,却没有进营,而是转头向东,那边有着西南夷人的营帐。

“呃……”高升泰张口想说些什么,但没有敢说出口。

秦升回头看了高升泰一眼,“总管说了,要下官带高侯你看一看石门蕃部的营地。”

高升泰拳头上的青筋都暴了起来,但最终还是不敢多说,小心翼翼的跟在秦升的身后。

他的随从,想跟着上去,却被拦住了。高升泰回头摆摆手,没上他们跟上去。

石门蕃的营地,就在官军营地旁边,直接就占了大理子民的房屋。

熊本秉承圣旨,对汉人加以关照,这一回大宋南征,时至今日已经有七八百户汉人,得到了官军的保护。熊本前日特意将触犯汉人的蕃部施以重惩,就是要让西南夷上下都明白,即使是奴隶,只要他是汉人,就比手握上万男丁的洞主、族长都要尊贵。而行刑之后,他又公布另一路的官军已经快要会合,则是彻底的将所有异心都给压了下去。

蛮夷们对熊本派出的幕僚俯首帖耳,对高升泰则是怒目而视,更有少年人在旁提着长刀,一脸的跃跃欲试,让高升泰的心脏都提到了嗓子眼,生怕哪个毛头小子一时头脑发热,让他这个相国之子,实质上的大理太子,死的不明不白。

不过他的提心吊胆,在深入石门蕃部的营垒后就再无暇去顾忌了。

铁甲。

铁盔。

钢枪。

站在营门口迎接的一干蛮夷,尽管他们身上都穿着盔甲、还住着长枪,但并没有让高升泰太过惊讶。毕竟是精锐,跟在宋人身后,有点好东西很正常。但入营之后,几乎每一个蛮兵,都有一个黝黑的铁质头盔,枪刃上闪着精光的长枪,也是人人都拄着一支。

这几个月来,不少奉命清剿入寇蛮夷的败将逃回大理,都在说蛮夷的甲坚兵利。之前高升泰还觉得是战败后的脱罪之词,甲坚兵利这个词用在宋人身上无可厚非,用在蛮夷身上,岂不是个笑话?可现在看来,还是说得少了。有如此装备的军队,即使是在大理国中,也不过数千而已。

中原的兵器,高升泰也不是没有见识过,流入大理的宋国刀枪,只要他想要,自然能拿到手。高升泰曾拿着自己的配剑与宋人佩刀对砍,刀剑交击之后,锋刃上都迸出了缺口。但自己的佩剑是国中最好的匠人打造的,只能与宋军小卒手中的武器相当,宋人的武器到底有多精良,就可想而知了。

看着高升泰脸上的表情变化,笑容在秦升脸上一闪而过。带着高升泰过来,让他看的不是大理国子民的痛苦,而是石门蕃部的装备。

“高侯,此间蕃部手上的装备,都是朝廷所赐。每个部族按照出兵人数,十比一的比例给予铁甲,而点钢长枪和精铁头盔,则是人手一把,小头领还能得到一把腰刀,刀刃夹了钢,用对力量,能一刀砍断碗口粗的树。”

高升泰脸色泛着青色,满口苦涩,仿佛嘴里被塞了一个青青的生柿子。

山中的蛮夷,连衣服都没有,就在营地中,高升泰看见很多人都裹着破破烂烂的粗布衣服,但他们手上的武器、盔甲,却都是锃亮的。不是宋人给的,还能是天上掉下来的吗?

高升泰明白,不是大宋求着蕃部出兵,才给了这么多好处——如果当真是这样,聪明人都会选择给丝绢、瓷器,而不是给兵器作为酬劳——只不过是根本不在乎,就算那些蕃人想凭着这些武器反叛,也不过是给宋军的将帅多送一份功劳而已。

“我中国别无长处,唯有富庶二字。这一场南征之役,朝廷分三路出兵,总计马步军七千八百人。”秦升回头看了一眼高升泰,“不及大理十一。”

高升泰黑着脸,没回话。大理国若真的点集兵马,的确能拼凑出十万大军来。就是国中常备军,也有五六万。若是籍民为兵,二三十万也是有可能的——只要各部都听话就成。

 
 但宋军来攻时,身边还带着几万北面山中的蛮兵,那些蛮兵横行于国境内,烧杀抢掠,无恶不作。许多刚刚听命出兵的部族,一看到自家要受到攻击,立刻将兵马
召回。而大理国的常备兵,几次与宋军交战,每次都是惨败,不论是伏击、还是正攻,不论是野战,还是守城,几乎都是被少数精锐的宋军给击败。

真要说起来,几次会战,大理一方几乎没有占据过的人数上的优势。而且与宋军交战时,更是连近战的机会都没有找到。全都是被宋人的弓弩,以及神秘的火枪火炮,在半路上就给打垮了。

“可只为了这一场南征之役,朝廷在开战前,就准备了铁甲一万六千领,军袍三万五千套,鞋四万双,帐篷四千顶,神臂弓八万张,弩箭三百万支,枪六万三千杆,战马一万一千匹,大小车两千六百辆。”

秦升仿佛成了说书人,一连串的数字排比着,将大宋的富庶,渲染得让人眼晕目眩。

“不算粮秣、饷钱、犒赏,仅仅是军资一项,便合计一千七百万贯,以京师金银铺兑换的价格,大约是六百万两官银……本官知道大理盛产金银,不知一年能出产多少?”

 
 熊本每说上一个数字,高升泰的脸色就白上一分,出兵还不到八千,就准备了。绝不是熊本胡诌,都打了这么久的仗,高升泰当然知道宋军的数量还不过万,可就
是一千两千,都能轻易击败,不是靠钱堆上来,还能是什么?宋军小兵的装备,都赶得上国中的大将,装备上差得太远,这仗输得一点也不冤。

秦升冷冷笑着,“蕃部随同官军出战,朝廷一点赏赐都没有给。但你们的人,你们的地,你们的子女妻妾,都是他们的战利品。如果不听话,他们就得在黄泉路上给你们做个先锋官,所以都是奋勇争先……如果尔等再顽抗到底,不顺天兵,来日破城,官军就不先进城了。”

或许秦升的话不尽不实,但一想到户户飞花、街街流水的大理城,有可能变成了人间地狱,高升泰就不寒而栗。

高升泰拜倒在地,“上官容禀,小人奉旨而来,正是为了请降。”

“你们的降顺,不是朝廷要的降顺。”秦升冷着脸,犹如冰山,“大理朝中,自段正明以下,必须于明日前自缚出降,不得再抗拒天兵。只要听命,朝廷自会有恩泽。至于高氏……如果朝廷能得大理,又有什么罪过不能赦?”

高升泰抬头正想说话,忽然只听见满营的欢呼声,如山崩海啸,直扑而来。

高升泰惊疑不定,秦升也是脸色微变。

两人不再多话,匆匆离开石门蕃营地。还没回到官军营门处,就见一人奔来,喜笑颜开,“机宜,大理国王起兵,尽屠高氏一门。如今已经开城投降,乞求朝廷宽宥!”

秦升大喜过望,高升泰如五雷轰顶,身子晃了一晃,要不是身后的随从扶着,差点一头栽倒在地上。

“护送高侯下去休息吧。”秦升吩咐道。

高升泰被左右架着走了。朝廷因高氏篡权而起兵,人人都知道是借口,但不影响高氏成为大理人心目中的罪魁祸首。如今大理国中生民死伤惨重,怨恨大宋只能很在心里,而实力衰弱的高氏一族,就成了发泄的出口。

若是能够以高氏为代价,能让宋军就此退兵,那就更好了。过去只有一两个那么想,到了兵临城下,怕是所有人都想抓住这根救命稻草,高氏焉能不败?

不过这个消息来得实在是太晚了,应该再早一点才是。

秦升想着,整了整衣冠,喜气洋洋的往营中去向熊本恭喜道贺了。

这一场战争,已经不用再打了。

……………………

韩冈的耳朵终于清净了。

几天前,大理的请降使臣第四次来到京师,韩冈再一次主张将其拒之门外。

次日清晨,大理使者在宣德门外痛哭流涕,若不是有人拦着,就一头撞死在城门前。

朝野内外,有许多人想结束这场战争,大理使者能跑出守备森严的馆舍,来到宣德门处,自然有人在背后支招。

朝野中如今正在争论,要不是采取了韩冈的计划,以近乎于灭族的威胁来清洗对方子民,大理国早就屈膝请降了。可是大理诸部都被随宋军南下的诸多蕃部逼得团结一处,逼得继续作战。大理战事始终不休,将士伤亡惨重,都是韩冈的错。

许多人为出征士兵的安危而义正辞严的时候,仿佛都忘了他们平素里是怎么对待赤佬们的。这对韩冈虽没有什么影响,但也是吵得他头昏脑涨。

但随着捷报的传来,原本对韩冈的谋略甚嚣尘上的攻击,一下子消失无踪。

逆臣被斩杀于宫中,尸体被城中军民分食殆尽,大理国君臣自缚出降,赶在冬日降临之前,大理国,灭亡了。

再大的牺牲,在胜利结束的战争面前,都变得那么的不起眼,而这一次的战争,几乎没有一场像样的大战,官军的伤亡多为疾病和各种各样的意外,只要不计算参战的西南夷,真正属于战殁之人,最终也只有两百多。

灭千乘之国,只死了几百人,这是一场史无前例的战争。科技和仆从军的作用,在这场战争中发挥得淋漓尽致。

他们的表现,就是韩冈想要的结果。

大宋的武器装备彻底转向火器,再也没有什么议论了。

当朝中的风向转为,韩冈已经开始与下属商议起在大理设立蒙学,并招收当地人进入蕃学学xi的计划。

宗泽不反对扩大京师蕃学的招生名额,却反对开办蒙学。不为他事,只因为蕃学和普通蒙学的学xi科目截然不同。

几个拿着五经教授忠孝之道的蕃学,所教出来的学生,和一个学xi了自然格物之道的学生,哪个对大宋的统治更不利,这是一目了然的事。

“放心。”韩冈道,“一个在汉家的城市真正接受了几年教育的蕃人,回到依然野蛮的部族中去时,就跟鱼离了水一般,别说衣食住行,就是呼吸都会不舒服。”

没有能够交流的对象,反而会被视为异类,一群白羊中的黑羊,那种孤独感,怎么不让人窒息?

为什么士人爱好逛妓院,还是因为面对家中的妻妾无话可说?女子无才便是德,依然是大部分富贵人家的圭臬。比如高太皇,比如向太后,都很少读书,仅仅是识字而已。

能如王安石的女儿那样能吟诗作对的大家闺秀实在是少数,能如曾布之妻魏玩,诗词做得让男子敬服的就更少了。相比起以《女诫》、《女则》、《女论语》培养起来的名门闺秀,从小就被训练各色技能的妓女,尤其是琴棋书画样样皆精的花魁,才是士人眼中适合交流的对象。

即使有着天大的气运,让这位学成归来的蕃人掌握到了族中权力,让他的蕃部开始学xi汉人的文明,可区区一人又能成什么气候?到了工业化的时代,工业人口才是重点;科学技术的发展,也是要靠大量的从业者才能支撑,就是辽国都学不来,又何况偏鄙小邦?

反倒是一旦接受了汉家文明,绝大多数人很容易就会被融化在文明之光中,最后不顾自己的身份,而为汉家奔走出力——这样的例子过去很多,日后也有很多。

且过去汉人教化蕃人是以儒学为,那还有挣扎反抗的可能,儒学的根基很难扎进西南蕃部之中。但随着气学格物之道的大发展,再想顽抗就没那么简单了,愿意的会被融合,不愿意的也同样会被融合——主动、被动的区别而已。

宗泽不再争辩。

韩冈做得宰相久了,独断独行的情况也多了起来,为小事与其争论,不是智者所为。区区蒙学毕业,想成才也没难么容易。

“相公,当如何安排大理?”宗泽岔开去问道。

“已经议定了,设一路掌控局面,设十州分而治之。只要滇池、洱海用来耕作、放牧就够了。”

“什么路?”宗泽问。

“彩云之南。”韩冈道,“云南路。”
