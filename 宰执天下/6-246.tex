\section{第33章 为日觅月议乾坤(四)}

云娘紧张起来,“三哥哥,应该不要紧吧。”

“放心。”周南搂着云娘,“有老平章和官人在,越娘怎么可能入宫?”

把王安石家归入到候选名单之列,韩冈的内侄女似乎便有可能成为皇后的候选。

尽管这只是编列名单,之后还得发信去询问是否有适龄未字之女,等得到肯定的回复之后,再从宫中派人出来查看本人进行审核。过关之后,还要入宫进行一段时间的教养和观察,再从中选取十名以内的合格者,最后推荐到太后、太妃与皇帝的面前。

这其中的流程,如果走完的话至少要半年以上的时间,中间会淘汰绝大多数入选者。

王旁被列入名单之中,而且也的确有一个适龄的女儿,但只要不愿意,他直接就可以拒绝,难道朝廷强买强卖,还能强迫到王安石这位顶级元老头上?

“老平章还在世,官人更是现任的宰相,谁敢让老平章的亲孙女,官人的内侄女当上皇后?以老平章和官人的身份,更不可能做嫔妃。不用担心,不用担心。”

“那三哥哥怎么……”云娘担心的看着韩冈,依然面沉如水。

“官人,是不是信已经寄出去了?”周南小声的问道

韩冈点头,应了一声。

周南仰起头,“既然这样,章相公那边肯定得给官人一个交代!”

韩冈手中的名单是副本,负责此事的章惇既然没有察觉——周南不觉得章惇会玩这种没意义的小伎俩,只会是下面的执行者出了问题——也意味着已经将信函寄了出去,询问其家中是否有未许人的适龄女儿。章惇作为负责人,理所当然的要为此事担上一份责任。

韩冈摇摇头,脸色缓和了一些,有了点笑容,不过却是苦笑。

严素心握上他的手,“官人,可是担心章相公那边是故意如此。”

“不是。”韩冈摇头,“我是在想,我那岳父病得真不是时候。”

周南惊得差点跳了起来:“官人,你是说……”

韩冈笑了起来,“我那岳父虽不至于视我为死敌,但拆台的事,肯定会乐意做一做。”

“但老平章可是被唤作拗相公的!”

严素心摇头,感觉难以置信。普通官员贪求做外戚的好处,但王安石这样的元老重臣,又何必去贪图富贵?难道不是名声更重要?

“如今他身体不好,有些事过去不会做,不代表现在不会做。”

云娘终于是想明白的了,“三哥哥,你是说老平章想要让越娘做皇后?!!”

“是啊。”

“那怎么办?!”云娘急道。

周南也问,“官人,要不要派人去江宁?”

“来不及了。”韩冈摇头,重又舒舒服服的靠回躺椅:“还是等章七派人来吧!”

韩冈不信章惇会拖到明天才得到名单出错的消息。

近千候选文武官的名单,事前出了差错,章惇没有察觉情有可原,但事后还懵然无知,章惇这位宰相就未免太失败了。

而除非是想跟自己翻脸,否则章惇今天晚上,肯定要派人来自己这边给一个说法。

“相公。”门外传来唤门声。

严素心起身出去,转回来后,带着几许惊讶,对韩冈道:“官人,章相公遣人来了。”

片刻之后,换了一身见客的衣服,韩冈来到外书房,

就见一个熟悉的年轻人向他躬身行礼,“章缜拜见相公。”

“玉成不必多礼。”

韩冈让年轻人起来,分宾主坐下。

章楶排行第七的小儿子章缜,表字玉成,如今就跟在章惇的身边。

章惇当天晚上就遣人过来道歉,这个态度足够好了。

又遣了最亲近的族侄,而且还是韩冈旧部、与其交情颇佳的章楶的儿子,肯定是知道事情不妙。还是有些私密事要通过章缜来交流。

“小子奉家叔之命来此,是为了今日给相公的那一份名单。”

章缜知道事情紧急,没有寒暄,直接开口。

“我已经看到了。”

“家叔说,他已遣人连夜去江宁,追回发去楚公府上的信函。”

韩冈不置可否,“令叔打算怎么处置?”

“家叔说了,在中书门下办差,办岔了差事,不究原心,只问结果。既然这件事办得大错特错,该抓的抓,该判的判,绝不能轻饶。”

不论是什么原因,不论是一时疏忽还是故意如此,韩冈都没打算去了解。

上千人的名单的确多得让人头疼,章惇没仔细看情有可原。但太后、宰相交代出去的事,下面的人却没放在心上,这就该死了。

韩冈轻敲着桌面:“一时疏忽,岂是罪过?”

章缜立刻回道:“在中书门下办差,岂有没罪过的时候?”

韩冈笑了起来,这位小章七的反应倒快。

想想章楶,的确让人羡慕。

七个儿子各个成才,次子章综是韩冈的同年。熙宁九年,更有长子、四子、五子三人同时考中进士。三子章綡虽没中进士,可前段时间在国子监中拿到了一个头名。老六、老七同样是聪颖过人。

就是元老之家,有这么些成才的儿孙,也足以感到骄傲了。

“玉成你回去与子厚相公说。还是先问一问,是自己申请去安西都护府拓边,还是选择问罪发配。”

章缜点头称是。心中也不免感叹,韩冈的不留情面。

问罪发配,多半就是一个死字,但去西域拓边,不可能是一个人上路,多半是要全家西行。无论哪位宰相,需要心狠手辣的时候,绝不会给人留下半分余地。

韩冈的这个要求,章惇绝不会反对。

无论如何,当朝宰相的内侄女一旦做了皇后,最吃亏的就是那位宰相。不让韩冈有个出气口,谁知道他心里的邪火会烧到谁人身上?

解决了怎么处置犯错之人,章缜又道:“小子还有出门前,家叔还吩咐了两个问题,想要请教相公。”

“什么问题?”

“第一,太后得知此事后,会赞同还是会反对?”

“太妃和天子必然是愿意的。太后的想法,则难测度。”

王安石病重垂危,离死不远,而且他的家族下一代缺乏出色的人才,两位还在世的弟弟,王安上,王安礼都缺乏晋身两府的才能和机缘,行事都会被人盯着,不用担心外戚窃国权柄。

但王安石又是元老重臣,由他开创的派系占据了朝堂的半壁江山,聘其孙女为后,与天子好处多多。

如果从这个角度来看,他的孙女儿的确是最合适的人选。

但这桩姻缘对韩冈绝无好处。

即使赵煦做了韩冈的内侄女婿,结果还是跟现在一样,甚至还会让韩冈的处境更加艰难。

要是赵煦娶了王家越娘,那韩冈的相位可就不那么稳固了,与天家的亲戚关系,对宰辅、对重臣,都只会是绊脚石,而不是助力。

韩冈更不用去幻想赵煦会这一桩婚姻改变对自己的看法。

事关天下权柄,连父子兄弟都会反目,就不用说姻亲了。看看曹操怎么对待他的亲女婿的?反过来,汉宣帝又是怎么对待他的老丈人家的?

朱太妃、赵官家,都会乐意看到韩冈陷入这样的境地。至于太后,韩冈真没把握她到底会不会答应。

“第二,家叔想让缜来请教一下相公,楚国公会不会答应遣孙入宫?”

“不知。”韩冈摇头,这件事他同样说不准。

王安石没多少日子可活了,病情或许比想象的要轻,但这一点却是无可质疑的。

当寿数只能以日来计算,像王安石这样的人,不会畏死,会去考虑毕生的功业是否能够得到保全。如果想保住家门长兴不堕的话,最好的作法就是让孙女去做皇后。

而且孙女做了皇后,成为下任天子的曾外祖,便可抵消韩冈在年龄上的优势,新学就能保住了。

至于韩冈这位二女婿,大宋开国以来又没有杀过宰相。韩冈直到现在为止,又都是标准的忠臣,有大功于国,有大恩于天子,即便在民间的名声好得超凡入圣,即便已是功高难赏,只要日后远离权位,哪个皇帝都只会用高官厚禄将他给养起来,

一举数得,不过是损些声名,中风后的王安石说不定做得出来。而王旁,还有在江宁的王安石的一干子侄,包括韩冈岳母的娘家人,怕都是会推动此事。

韩冈两个不知,让章缜的眉头都皱了起来。

“你叔父怎么说?”韩冈问道。

“如果相公能确定,便回来。如果相公不确定,就再问相公两件事?”

“什么事?”韩冈饶有兴致的问道。

“第一件,是相公的内侄女品貌如何?”

“只幼时见过数面,不过听闻如今品行相貌都可算得上是出色。第二件又是什么?”

“第二件,相公家的衙内可有年岁未婚配的?”

韩冈脸上终于绽开了笑容,猛然间哈哈大笑起来,章惇的心性真的是让人佩服,必要的时候,竟能如此决绝。

“当然是有的。”

章缜也笑了,有此一招,一切都不是问题了。

“不过。”韩冈收了笑,“我不愿意。”
