\section{第32章 金城可在汉图中(11)}

折可大一阵感慨,然后恢复过来,“方才机宜说了有事吩咐,不知为了何事?”

“是留通判的吩咐。”张俭拉着说折可大往北门走,“你可知道韩枢副的那个亲信,就是前两天往北去的那个。”

“嗯。说是要传话忻州,并联络代州被打散的官兵。”对于韩冈派出去的这位亲信,折可大颇有期待,隐隐的也有些佩服:“希望他能做到。”

“若他能做到,这一回攻入河东的辽贼多半就回不去了。”张俭低头看着脚下,忽然沉声,“阁门,你可知道,这一战在朝廷看来,其实是在所难免的。”

“此话怎讲?”

“澶渊之盟已有七八十年,人心懈怠,对当年宋辽连年鏖战以至两败俱伤的旧事都忘了差不多了。这一回与辽国之战在所难免,不在今年,就在明年后年,逃不掉的。不过这一战也是机会,至少要打出三五十年的太平时日来。若能再出一个澶渊之盟,对两国百姓也未尝不是一桩幸事。”

“这话是谁说的?”折可大立刻变色追问。这话张俭说不出来,就是张俭代为传话的留光宇,区区一个通判,也不可能有这么大的口气!

张俭瞥了折可大一眼:“是韩相公的信上说的。”

“是枢密相公?!”折可大惊问。韩冈写信给留光宇了?

“不。”张俭摇头,“是韩子华韩相公。”

‘原来是这个韩相公。’折可大恍然。

文武高官在民间不是相公就太尉,但在官场中想得人唤一声相公,至少得是两府中人,而真要计较起来,却只有宰相才能称相公。韩绛这个相公可比韩冈的枢密相公成色要高多了。

不过河东、太原面临危局,现在一说韩相公,城内的军民官吏十个倒有九个半会认为是韩冈。至于韩绛,绝大多数百姓根本就不知道有这号人物。

但韩绛终究是身居云中的大人物,折可大不敢不敬,“韩相公怎么会给留通判写信?”

“你不知道?留通判跟韩子华相公有亲啊!”张俭一脸惊讶,“而且这留通判还是韩枢副的同年。”

“同年的事我知道,跟韩相公有亲还是第一次听说。”不过折可大现在并不关心什么亲戚关系,王.克臣还是外戚呢,英宗皇帝的亲家,又怎么样?只是他又感叹起来,“想不到朝廷是这个想法。”

“这是两府内外共同的判断。事有缓急利弊,但并不是固定不变的。因势利导,缓可急,急可缓,而坏事也可变成好事。”

折可大深吸了一口气,又缓缓吐了出来,终究是宰相。纵然上有王安石,下有吕惠卿、韩冈,让身为首相的韩绛在政府中不是那么起眼,可这份见识不愧是宰相之才。

他神态恭谨,虚心问道:“韩相公在信里有说些别的吗?”京中相公私信上的内容都能知道,明摆着就是留光宇本人透露出来,要张俭转告的。

“倒没有别的话了。只是让留通判一切听另一位韩相公的。”

折可大沉默了一瞬间,“……留通判可是有什么吩咐?”

“没别的,留通判说了,今晚若无事,请折将军你过府一叙。”

折可大没犹豫,立刻点头:“只要无事,下官必至。”

话声刚落,却猛地听到前方一片声的再喊,“辽贼!辽贼!”

“城外来了辽贼!!”

声音凄厉,如夜枭惨嚎,让混乱的市井顿时安静了下来。

‘只要无事……’折可大一声暗叹,看了看愣住了的张俭,‘怎么可能无事!’

寂静仅仅维持了一瞬间,前方随即涌起一片人浪,街道上鸡鸣犬吠,骡马相嘶,哭声喊声一片沸腾。

人们你推我搡,纵然北面还有高高的城墙,城门也早已紧闭,但街上的行人还是像是没头苍蝇一般乱冲乱撞,最后变成了向南逃窜的浪潮,直冲正走到街口的折可大、张俭而来。

折可大的亲随见势不妙,猛地拔出了腰刀,三五人杀气凝聚,却像是中流砥柱一般,让人流一分为二,从身旁涌过。

张俭看着两边的混乱,双唇都失去了血色,怎么乱成这幅模样了。要是辽军现在攻城,一天半日就能给破了城去。

折可大脸色阴郁,左右看看,然后转身跳上马。

“阁门,哪里去!”张俭惊声大叫。

“去府衙!”折可大一声怒吼,挥空一抽马鞭,分开人众,泼剌剌的蹄声便往府衙方向奔去。

虽然他不想管事,也不当管事,但自家的性命,折可大并不打算放在那位王经略手里。凭自家的家世声望,再借一下韩冈的虎威,折可大相信自己在这个时候当能压得住阵脚。

“二十天!”

一声暴喝声震府衙,让如旋风般冲进府衙的折可大惊得停了步。就在院中,望着大堂内,那是自从来都不会高声大气的王经略嘴里发出来了。

“二十天,不对,这信是昨天从铜鞮县发出来的,到今天就只剩十九天!十九天!十九天后援兵便能到了!”

河东经略使的脸色一改半个月来的苍白,满面红光。

他晃着手中的一页信纸,如同抓着一根救命稻草,中气十足的大喊着,“只要守住十九天!十九天!援兵就能到了!到时候,将北虏尽灭在太原城下!”

王.克臣双眼神光湛然,环扫四周,“这是制置使韩枢相说的!!!”

……………………

“萧安素应该快到太原了吧?”

烽火山城上的萧十三意气风发,攻下雁门的是他的人,拿下代州的也是他的人,现在第一个攻去太原的也是他的人。如此武功,一时无人可以匹敌。

“只有先攻下榆次县,防住河北军,才能算是安心。”就在萧十三身边的张孝杰轻叹一声,“更别说我们在河东人生地不熟,不比河北,万一走错了路,可就麻烦了。”

一名北院枢密使,一名南院宰相,两位大辽的重臣站在宋国国中的险关之上,远眺群山峻岭,一时气象迫人。

“有人带路,何须担心。”萧十三仰头哈的一声笑,“这些商人,吊死他们的绳子他们都敢卖过来。”

“听说前几日,枢密杀了一户商人?”张孝杰忽然问道。

“又不是大辽子民,杀几个抢几个又能有什么大不了的。”萧十三哈哈大笑,“他家的几房妻妾和女儿都不错,若相公有兴致,送你一对如何?”

“他们我大辽做事,好歹留他们一条狗命才是。”张孝杰无奈的摇了摇头,“汉人有个说法,叫做千金市马骨。留着他们,重用他们,能引来更多南人投效。”

千金市马骨的故事,萧十三也听过,不需要张孝杰多解释,“听说韩冈之前是准备用商人来跟尚父谈判?”

“没错。”张孝杰冷笑道,“堂堂韩学士,药师王佛弟子,都是菩萨了,想不到也会犯蠢啊。真不知道他是怎么想的,竟然敢相信那些商人。”

“不说是市马骨吗?”

“这哪里是市马骨?这是放贼入库。”

“反正那是尚父的事,让尚父去操心好了。”萧十三摇摇头,“其实说起来将事情闹大的还是陕西宣抚使吕惠卿。南朝的朝廷当还没有毁约背盟的想法。”

张孝杰冷声道:“南朝的两府并不是铁板一块,肯定是各有各的盘算。但吕惠卿是谁任命的?这一回宋人的朝廷纵然给吕惠卿给坑苦了,却也是他们自作自受。”

萧十三点点头,笑了一下。心中在想,天底下的乌鸦一般黑,张孝杰说宋国朝廷的宰辅们各有各的盘算,难道在大辽朝中就不是这样了?张孝杰下的判词,还是以己推人的成分居多。

“雁门关打下来了,石岭关打下来了,代州也打下来了,也就忻州未下。若能攻下太原城,忻州自不在话下。局面之好,攻势之顺,就是当年承天太后领兵攻到澶州城下也是比不上的。”萧十三眼神闪烁不定,“以相公的看法,接下来当如何做?”

张孝杰不假思索的便说道:“以打促和。见好就收。”

萧十三惊讶张大了双眼,问道:“你跟尚父说过了?”

“这正是尚父的打算!”

张孝杰这个汉人越来越受耶律乙辛的信重了,这让萧十三很不舒服。不过对眼下局势的判断,这一点他是能认同的。

越是深入河东,辽军面临的危险就大。河东的土地对喜欢纵马追风的契丹铁骑来说,实在太过狭促。进退只有几条路,就像钻进风箱里的耗子,指不定就给人活擒了。

见好就收,这是萧十三等几位主帅的共识。退回来守住赤塘关、石岭关,

一名骑手自南而来,穿过了南下的队伍,一直冲到了关城脚下。

片刻之后,一名亲兵走了上来,递上了一封信。萧十三展信一看,脸色便是一变。

“怎么了?”张孝杰在旁诧异的问,萧十三去年亲手给幼主灌了药,都没有变过一下脸色。

黝黑的脸庞如阴如晦,萧十三阴沉沉的说道:“韩冈将至太原!”
