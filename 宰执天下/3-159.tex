\section{第41章 礼天祈民康(六)}

比起预计的时间早了一天,韩冈抵达了东京城。

大礼在即,城内城外戒备森严。韩冈与童贯一起从白马县赶回来,一路上,不过一百多里的道路,竟然遇上了十几队巡检马队。等到了城门口,城门守兵的搜检比起韩冈前日离京时,则又严密了三分。

因为搜检耽搁了太多时间,城内城外都排起了长龙,队伍中的人们只能一步步向前蹭着,怨声不绝于耳。如果韩冈不是穿着官袍,童贯又亮明了身份,恐怕也要城门处等上一两个时辰才得入城。

“韩提点,官家正在宫中等候,还请快一点!”

进了城,童贯急着催促着韩冈。看着现在天色,已经是申时初。再不赶紧入宫,可就要等到明天。而到了明日,天子就要开始在大庆殿斋沐七日,静心礼天,等待郊祀大典的开始。

这段时间中,天子一般也就会接见一干宰辅重臣,而韩冈想要觐见,虽然也不是不行,但未免会有些议论,耽搁了天子斋沐的时间。在官家心中,他童贯当是少不了一个办事不利的罪名。

但这位小黄门与韩冈已经算是很熟悉了,也有巴结交好的想法,昨日奉天子口谕到了白马县,便将赵顼的一番话倾囊相告——这并不算违背天子的诏令,因为本来传递的就是口谕,但已经足以让韩冈了解到赵顼的心情和想法,同时也有所准备。

沿着城中的街道,韩冈和童贯很快便抵达了皇城前。

从左掖门进宫,童贯领着韩冈往崇政殿走去,沿途的官员看到韩冈,惊讶之余,也有着不少人羡慕,这个时候并不是天子接见朝臣的时间,除了一干重臣能在黄昏之前直上崇政殿,其余小臣一年也不见得有几次机会,而且看韩冈风尘仆仆的样子,还是刚刚抵京,这份圣眷朝中少有一见。

天子委以重任,韩冈却连番辞官不就,这一番作为,日后多半就又是一个王安石!

一道道又羡又妒的视线,韩冈全然没有放在心上,他现在正在暗自措辞该怎么将中书都检正这项任命给顶回去。

走进殿中,韩冈一瞥之下,在殿内竟然只看到了冯京,而其他几位宰辅却都不在。不便再多想究竟是怎么回事,韩冈于大殿中央再拜起身,垂手等着天子的发落。

“韩卿,你可终于来了!”赵顼微微笑着,可说出的话却一点也不和气

赵顼今天很有几分不快,本就因为大典将至而心浮气躁,现在对他任命拒绝得很干脆、让他难以省心的韩冈到了面前,免不了要更添火气。

听到天子说话的口吻腔调,韩冈心中有了一丝明悟,他终于知道,赵顼的火气是哪里来的。

孙永任了桥道顿递使,拉着韩冈一起忙得焦头烂额。开封府界如今风声鹤唳,一点小事都能引得从县中到府里一起鸡飞狗跳。那么赵顼这位当事人为了大典而心浮气躁,也是在情理之中。

这个时候,韩冈顶了赵顼的诏令,做了不给天子面子的事,当然不会有好结果。换做平常,也许根本不算什么,赵顼也不会强逼着韩冈来,但正好撞到了这个时间段中,韩冈就少不得要看到天子的难看脸色了。

运气还真是糟,韩冈心中一叹,道:“臣不敢。陛下即是有招,臣自当兼程而来。”

“不知朕所任命的中书都检正一职,韩卿是否还要推辞。”赵顼平平和和的问道,却是紧咬着不放,“以韩卿之功、之材,也当得起这个职位。”

韩冈说着惯例的回话:“微臣微末之功,难报陛下恩德之万一。只是中书之事,中书检正乃是军国之重。臣虽小有才干,忝有微劳,但素未习其事,便不敢贸然奉召。臣若不能胜任,不仅败坏国事,也有伤陛下识人之明。”

韩冈一番话,就是说赵顼实在太看得起自己了,自家当不起。安抚流民,使之不至为乱,韩冈过去有经验,同时也是从白马县花了几个月时间做准备的。但在朝堂之中任事,担任的还是中书五房检正公事这个职位,难度可是天差地远。

韩冈的回答,赵顼也算是不出意外。自承他难以做到,所以不敢接受。但这也是惯常的回答,但凡有哪个臣子被任命了让他们不愿接受的职位,有很多都会加以拒绝。而自称不能胜任,便是最为常用的一条,朝廷一般也就不会再强迫他们接受。

“韩冈,当年同知起居注一职,王安石连辞八九次,难道你要学着来不成?”冯京微笑着,似乎是漫不经意的插话道。

韩冈的脸色倏然变了。

韩冈无意担任中书五房检正公事一职,此前已经将心意由孙永传到天子那里,想来宫中派出来的天使,总不至于把他追到厕所里去。像当年捧着诏令的宦官,追着王安石一直追到厕所外,只为了求他接受朝廷的任命一般,如今应该是不可能了

可韩冈万万没想到,冯京竟然在天子面前说他是在仿效他的岳父,虽没有明言他心怀诡诈,但赵顼哪里可能听不明白。这个指责甚至是诛心刻骨,让韩冈都不愿承受。

冯京这是要毁了他的名声。传到外面去,他在士林中也会沦为邯郸学步的丑角。虽然眼下辞官不就的官员很多,但并不代表他们能体谅韩冈。

王安石屡次拒绝清要之职,都是在说京中为官给的俸禄太少,所以求着要外放一个州郡,好多挣点钱来奉养长辈以及家里的一堆弟妹。这是出于王安石本心,不想在朝中任官,而想在州县里来推行自己的治政方略,因此而来的名望乃是附带,并非王安石孜孜以求,所以赵顼相信王安石的人品,故而才会任用他主导大政数载。

但韩冈如今的行为若是在仿效王安石,就不是东施效颦四个字可以让人一笑而过了,那是心怀诡谲。可以博取人望的手段,如果是刻意做出来,他暗藏的目的当然就要惹得人深思。

赵顼脸色也变得难看了起来,原本仅仅是心中有一番怒气,此事却是变得狐疑和猜忌起来。

他如今求的是朝堂的稳定,异论相搅虽是祖训,却也没有哪个皇帝是希望朝中乱哄哄的,臣子们每天我攻击你、你攻击我,你弹劾来、我弹劾去。所以他在留了冯京、王珪在朝堂上的同时,却大力支持韩绛和吕惠卿。

但韩绛、吕惠卿并不和睦——赵顼看得很清楚——很有可能在他们之间,会大打出手,将朝局弄得乱成一团。所以在中书内部,他需要一个合适的人选来总括诸房庶务,并弥合韩绛和吕惠卿的关系。

在赵顼看来,韩冈正是这个合适的人选。可是韩冈却对这项任命连番推辞。

若是畏难,这就让赵顼很失望,想不到他看重的臣子,竟然也是拈轻怕重的懦夫;若是如同冯京所奏,是为了学着王安石的先例,而在养望,则更是让赵顼不快。只要忠心事君,日后自有他的好处。现在却怀着诡谲之心,试问哪一个天子如何敢对其加以大用?

换作是在朝中的其他官员,换作是普通的臣子,赵顼也不会这般心中不快。但赵顼对韩冈的确是很是看重,所以对于那些对韩冈的弹劾和指责,赵顼从来也不相信。可相应的,韩冈若是让他失望,赵顼心头的怒火,便也只会更多。

终见天子变色,冯京暗喜于心,蔡确的确看准了韩冈的弱点。

但他也知道,不过一句话而已,当然不可能一下子将韩冈打死。但只要在天子心中种下一枚猜忌的种子,韩冈日后想要再往上爬,也要多上许多坎坷。

天子任命韩冈为中书都检正,冯京当然知道天子是在打着什么注意。韩冈被韩绛所看重,同时也是王安石的女婿。在天子看来,理所当然的,他就有着弥补韩绛和吕惠卿之间矛盾的能力,让新党不至于内部分裂。

从冯京的角度来说,新党内部一团和气,就是他的梦魇。那时候,他当真只能做个反对者,对着韩绛、吕惠卿的治政空喊着异议。所以他必须要针对韩冈下手——韩冈有那个本事,他的确有能力或者说有机会,调和如今已经显露在外的韩吕之争。

但冯京从蔡确那里得到的手段,并不是让韩冈不去接手中书都检正一职,因为韩冈有回心转意的可能。而是让他即是接手也无法改变局面——从根子上直接动摇天子对韩冈的信任!

这才是上佳的手段!

冯京垂下眼帘,看着手中的笏板,暗暗自得不已。

乍惊乍怒之后,韩冈的心情却平复下来,化作微微一笑。

冯京的手段是不见血的阴狠,的确是入骨三分,就算是否认,也不可能改变天子的猜疑。猜疑之心虽然微小,但一旦种下,就像杂草一样再难拔出。

只不过,冯京弄错了一件事。他能站到现在的位置,主要靠的是自己。要真的依靠着所谓的圣眷,凭着他所立下的这么多功劳,岂止是一个七品右正言?!

河湟开边的事就不用说了,就是罗兀撤军、咸阳平叛,他有多少功劳没有受赏?再加上还没有完全收尾的流民安置,他韩冈这些年立下的功劳,按部就班的做到宰相的冯京得闪一边去,他在外地任官的几十年积攒下来的功绩,根本不配与之相比。

