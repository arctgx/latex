\section{第34章 为慕升平拟休兵(14)}

半曰前,还是像雨后的蘑菇一样被营帐的填满的营垒,现在就只剩下了一片空荡荡的白地。

曹博收拾好了自家的随身家当,跟在同一队的兄弟身边排好了队列,听从指挥使的号令,从营地中鱼贯而出,走上了出营的道路。

这几天来,随着一个个禁军指挥的离开,原本被兵马填满的营寨,渐渐的变得空空荡荡起来。今天终于轮到了曹博他所在的这个属于宣翼军的步军指挥。

从孟州一路走到了忻口寨,还没有过上阵厮杀一次。但再过几天,抵达了前线,那时可就真的要见血了。年纪不过十七八的曹博,心中有着一点对未知的恐惧,但更多的还是对功勋的渴望。他热切的盼望着能够在这一回的战争中,立下绝大的功劳。就算不能做官,也能拿回一大笔赏钱。

作为今天第一个被派往前线的指挥,这个宣翼军的指挥紧随着军旗,走在大道的正中央。道路两边,是一群明显是异族装扮的鞑子。

大军出营,入营的就要赶快把道路让开,不能耽搁出营大军的时间。但过去出营时,避道相让的只是官军和民夫,可没见过异族。

“是阻卜人。”曹博的同伴中耳目灵通的看了两眼就不屑的吐了一口痰,“被麟府军打败之后,就投靠过来的北方鞑子。”

阻卜人?曹博好奇的向道边的人群张望着。

并不是如同传说中那样长得奇形怪状,除了装束以外,与汉人也没有多大的区别。

其中的一名阻卜人吸引了曹博的注意,不是因为阻卜人他本人,而是因为他身后的马匹。那匹马比周围的战马都要高出半个头,显得极为神骏。

地位高的有好马,地位低的骑劣马,这一点就算在大宋的马军中也是一样。曹博望着那匹器宇轩昂的高头大马,心里估摸着这匹马的主人应该是阻卜部中有名有姓的大人物。

不过再大的人物都是鞑子,曹博唯一在意的是那匹毛色发赤的骏马。他最想成为的是一名参加赛马联赛的骑手,夺得甲级赛事头名,拿回数都数不清的奖金。若是能骑上那样的马,应该就能参加赛马联赛了吧。

不知不觉,双眼注视着那匹好马的曹博离开了队列,也停下了脚步。

啪的一声响。一只大手从后重重的刷了过来,差点将曹博拍做了一个滚地葫芦。曹博踉跄了两步,缩头回头一看,他这一都的将虞侯正恶狠狠的瞪着他。

“发什么呆?走了!”大嗓门的将虞侯站在队列外冲年纪小小的曹博吼着。

曹博慌慌张张的钻回队列,还没走两步,

又是一声大喝在耳边炸起,“走直了。别丢人现眼,让那些阻卜鞑子好好看一看官军的威风来!”

……………………

达楞与出城的宋军擦身而过。

或整洁或肮脏,但那从他眼前划过的一件件军袍的料子似乎都是价格能抵上两三匹丝绢的厚棉布,而那名手忙脚乱的宋军小卒,分明地位不高,可他身上的衣带,竟然也是丝绸的制品。

宋国果然富庶。达楞不由得心中泛起一阵感叹。光凭传闻绝不会有亲眼看到时这么震撼。

不过如果仅仅是富庶的话,当然只是一只待人分食的肥羊。可是宋国不仅仅是富庶,一辆辆大车的篷布下,那闪着银光的铁甲,虽然已经不像在神武县时那般让他惊骇,但也不禁让他的目光流连不去。还有那同样放在车上的一柄柄巨大的斩马刀,那能打造多少支箭头啊。而一车车满载着箭矢的大车已经早一步上路,早早的就落在了达楞的双眼中。

若是有宋国的帮助,赶走契丹人绝不是梦想。

达楞他并不是这一回投靠宋人的西阻卜的成员,而是属于北阻卜。属于阻卜诸部族的盟主,阻卜大部族长磨古斯的辖下。

当宋辽交战的消息传到了阻卜大部族长的耳中,磨古斯的亲信部众中唯一会说汉话的达楞便被派到了西南的西阻卜来。瞅一瞅风色,看一看到底有没有机会。

若辽军占上风,他们就跟着乘火打劫一把,这个便宜不占白不占。但若是辽人吃了亏,被宋人压着打,他们之中有很多人并不介意重新换个好主子。而达楞更清楚,在他的主子心中,还有着借助宋人之力,将契丹人赶走,成为草原的主人的想法。

在这里的并不止阻卜一家。大黄室韦和梅里急等阻卜以外的草原部族,都派了人在这里探风色。这些天达楞在神武县的阻卜营地中绕了一圈又一圈,竟然发现了不少曾经在临潢府见过的老面孔。

聋子和瞎子是不可能在草原上生存,就是露出一点颓像,也会立刻被分食个干干净净。

什么时候蒙古和克烈的人来了,倒真是齐活了。蒙古、克烈两部的位置比阻卜诸部更要靠北,达楞也只是有偶尔入贡时才能看到他们。但达楞相信,如果蒙古、克烈听说了辽军惨败,肯定会连夜派人来看看情况。

契丹人常年欺压草原诸部,搜刮起来不遗余力。在契丹人的强迫下,阻卜诸部岁贡马两万。纵然阻卜部族多如繁星,可平均下来,分派到其中一个部族的贡马数量依然是一个大数目。

每年都要上贡,而且还不能是劣马,必须从部落中挑选最好的马匹送过去。一年两年,十年八年,到现在已经上百年了。这等于是一个人身上的伤口始终都在失血,永远没有愈合的可能。

换成宋人来了,绝不会更差。毕竟契丹人是放牧的,上好的草场被抢了许多。而南边的汉人是种地的,不会跟他们抢牧场。

阻卜诸部不可能现在就投靠宋国,达楞也没办法代替他的主子做主。但这一回只要能与宋人勾搭上,对他对阻卜诸部肯定有说不尽的好处。达楞不信,宋国不想在死敌背后扶持一个盟友来。但在达楞表露了身份,并表露了想面见宋人的主帅之后,得到的回应却是毫无挽回余地的拒绝。

“枢密不会见你!”

过来向达楞通报这个回绝口信的是一个三十出头、十分英挺的宋国武官。帐外通名是折阁门,而守在外面的则恭敬的口称将军。虽然年纪不算大,且达楞也不知道阁门到底是什么官,但既然姓折,又被称作将军,地位肯定不低。

只是这个消息让达楞又惊又怒,忘了去揣测这个折将军的身份,“为什么?!”

“枢密说了,他不相信只有一张嘴皮子的人。”面对粗鄙不文的阻卜鞑子,折可大也不拽文了,直截了当的表明了韩冈的心意。说话太过文绉绉的,这些蛮子也听不懂,“空口白话,谁知道你是真是假?”

达楞怒形于色,大叫了起来:“草原之上,没人敢冒充磨古斯的使者!”

哗的一阵响,守在帐外的卫士听到声音,慌慌张张的冲了进来,提着枪指着达楞。折可大不耐烦的摆手示意让他们出去,转回来对达楞道:“磨古斯与耶律乙辛比起来如何?阻卜比之大辽又如何?就是奉了耶律乙辛的辽国使者到了我大宋,也要有国书、有牒文、有关防、有印玺、有签押,有随行的人众,有事先派来的前导,这样才能让我国打开国门,放他们进来觐见天子和宰相。你光凭一张嘴,凭什么让我们相信你的身份?!”

折可大瞥了发起愣的达楞,蛮子就是蛮子。

但达楞很快就昂首挺胸起来:“这里的各家部族族长,没有不认识我达楞!”

折可大哼了一声:“活人作保,不如死人作保。那些族长也没资格为谁作保。枢密说了,去拿二十个辽贼的首级来作证据。这样枢密才会相信你,你也自然可以见到枢密。”

“二十个?!”

“没错,少一个都不行。”身材高大的折可大略弯着腰,俯视着达楞,“不论你用什么办法,只要能拿到二十个辽贼首级,枢密就会见你。”

……………………

“二十个是不是太多了一点?”

折可大得到了韩冈的吩咐,去打发阻卜大部族长的使者。刚刚得到消息的黄裳却惊讶于韩冈的命令,这是毫无必要的折辱。不论是真是假和见与不见,都没有必要这让这个可能是大宋、阻卜之间纽带的使者,无缘无故的送命。

韩冈笑了起来,“如果他的身份真的能在磨古斯面前说得上话,二十个首级应该很容易弄到手。用不着他亲自上阵。”

“……从各部手中借首级?”黄裳立刻就反应过来。

“不过一千贯而已。”

制置使司开出来的一枚辽军首级的赏格是五十贯,二十个不过一千贯。韩冈当然可以看不起这点小钱,但对于阻卜各部来说,却是一笔巨款。黄裳很难相信那个磨古斯的亲信真的能弄到这些首级来。

韩冈强硬的态度,已经不能让黄裳出乎意料了。他很难理解韩冈对待这名使节的态度。韩冈手上缺乏对草原部族有着充分了解的幕僚,达楞的到来对韩冈和制置使司如同天降甘霖。

草原上部族虽多,可现阶段,在拖辽人后腿一事上,真正能派得上用场的,还是阻卜诸部——如果从历史上来看,女真诸部肯定更能派得上用场,可惜东京道堵在中间,陆路绕不过去,水路还没有开发出来,缺少足够的水文资料,当然更不可行,而且最近女真诸部好像已经完全投靠了耶律乙辛的样子。

投机的作风遍布整个草原。或者说学不会见风倒的部族是不可能在草原上生存太久。只是学会了,还是一样活得艰难。对每年都要从他们手上剥削贡物的辽人,草原诸部有着根深蒂固的敬畏,可长年累月积攒下来的愤怒,只要爆发出来,便能让他们忘掉这份敬畏。

在这样的情况,又何必无故开罪有很大可能姓成为臂助的磨古斯的使者。

看出了黄裳心中的疑惑,韩冈笑道:“勉仲,要记住一件事。我们想恢复汉唐故地,这靠我们自己是能做到的,只是早晚而已。而阻卜部想摆脱契丹的控制,靠他们自己却做不到,没有外力相助,永远都是幻想。不是我们求他,是他求我们。明白吗?我们要的不是盟友,而是听话的藩属。没有必要委屈自己,也没有必要太宽纵他们。此辈畏威而不怀德,让他们看见大宋有破辽的实力,踢他踹他,他们还是会趋之若鹜,若是没有,再善待也是养着一匹狼。”

韩冈并没有在北阻卜人身上分什么心,他现在所看重的是新近归降的西阻卜。

随着阻卜人的南下,虽然说连人带马多了几千张嘴,但向北方神武县的补给量则减少得更多,营中积储反而增加得更快了一点。与此同时,手上多了两千轻骑兵,也给了韩冈更多的战术选择。

尽管不能给这些阻卜人配备甲胄,也不能给他们装备神臂弓,精良的兵器同样不便下发。但一些从辽人手中夺取的兵器给他们倒是不算很麻烦。

自居中国的汉人一直都视契丹为蛮夷,从骨子里戒惧,但也从骨子里看不起——人怕老虎,不代表人要与老虎平起平坐,禽兽而已——同样的道理,契丹人从来没看得起阻卜这群脏猴子,只是要贡品贡马的时候才会正眼看上一回。

给从来都看不起的猴子在背后捅上一刀,这份屈辱,韩冈想让辽军也好好的尝上一尝。

虽说萧十三肯定有了防备,但不管这一刀能不能捅上去,对大宋来说都不是坏事。投名状嘛,不是拿着敌人的血证明,就是用自己的血证明。
