\section{第44章 秀色须待十年培(22)}

黄裳现在正在赶文章。

黄裳面前的草稿,已经看不清本来的面目了。

宽大的桌案上,叠放着一摞摞书卷和文稿。随手抽取,又随手放置,高高摞起来的书堆,看着就是摇摇欲坠。

在地上,已经横七竖八落了好几本书,但黄裳只顾着不断的从书堆中抽出书来翻找,印证自己文章中引用的那些典故的出处是否有所错讹,却没空腾出手来整理一下。

一旁的油灯已经添了两次油,火光依然稳定,但灯盏中的清油已经见了底。而茶壶中也见了底。一名侍婢进来看了看,轻手轻脚的将落到地上的书籍和手稿整理好,添了油,续了水,便又悄步退了出去。

侍婢进来又出去。黄裳头也没有抬,而是专注的看着面前的手稿,不为任何事而分心。

明年四月,紧随在进士科之后,制科便将开考,留给黄裳的时间已经不多了。

策、论各二十五篇,述说大宋周边的地理军事,并评论过往战事和历史。说着简单,要写好可不是那么容易。只有呈上去的文章数量和质量都达标了,通过了两制官的审核,才有资格参加制举考试。这还多亏了黄裳他考得是冷门,否则要写的文章会更让他头疼。

除此之外,第二关阁试六论,都是往冷门中出题。想要过关,九经、兼经、兵书、诸史,便都要贯通。很多考生,都是连题目的出处都记不得,由此饮恨。要将这些经史传注更加深入的钻研,半年时间几乎就是一瞬间。

不过黄裳依然是有信心,只要能抵达御前就算是赢了。

制举诸科中,有官身者能参加的考试有六个科目:贤良方正能直言极谏科,博通坟典明于教化科,才识兼茂明于体用科,详明吏理可使从政科,识洞韬略运筹帷幄科,军谋宏远材任边寄科。

其中第一个直言极谏是最多人考的,只要文采好,再骂狠一点,多半就能中制科了。苏辙就是这么考上的,骂得最狠,让仁宗皇燕京不敢把他黜落。才识兼茂明于体用科,也多是说说。而其他几个则是要会做实事,空有一张嘴不行,就是博通坟典明于教化,也必须是授徒一方、教化百千的当世大儒才有资格来考。

所以设立这么些年,诸科都是空置。不说没人来考,就是有人考,也通不过。做了实事,直接就能升上去了,不做实事,哪里会有实务经验,让人通过考试?同时做多了实事,又哪有时间攻读,两制、秘阁的考试都不可能通得过。

也只有如今的黄裳算是一个异数。有才学,有经历。得到了朝廷赐下进士出身后,还要考一个制科出身出来。这也是为了通向宰辅的道路更加通畅,让人无可置喙。

黄裳准备考的是军谋宏远材任边寄一科。这一科,就需要考生对边境上的人文地理有极为深入的认识,同时对国家战略更得有一个长远准确的看法,若是考中了,就是放出去做边臣的路数。这等于是为黄裳量身定做的位置。

当然,考官也很重要。有资格担任军谋宏远材任边寄这一科考官的,朝中也没几人。韩冈是举荐人,这是得排除,但剩下的呢?吕惠卿在外,剩下的也就章惇、薛向而已。

而且既然是韩冈举荐,太上皇后那边肯定得给一个面子。只要没有犯讳之类的大问题,考官们也没有异议,她当也不会反对给黄裳一个制科出身的身份。

这就叫做朝中有人好做官。

但话说回来,再好做官,也不可能随便交上些文章就可以通过的。制科的地位,既然犹在进士科之上,难度自然远甚。至今百多年,通过制科的还不到五十。

每一科大比之后,当科排名前列的进士试卷都会公诸于众,好坏自有世人评判。黄裳若是能通过考试,他的文章自是都会公诸于众。如果丢人现眼,他考这个制科做什么?已经被赐予进士出身了,若不是为了求一个圆满,根本不必这般辛苦。

仅有一次的机会,黄裳绝不容许自己有分毫懈怠,眨了眨酸涩的眼睛,对着茶壶喝了口发涩的茶水,转瞬又投入到书稿之中。

……………………

还真是用功。

听到从别院传来的回报,韩冈就想起了当年上京赶考,在王韶家苦读的自己。

当年几个月的辛苦,换来了在官场上的通行证。没有一个进士资格,哪里可能升得这么快?可惜那时候自己根本就没有黄裳的文才,否则写些策论编辑成册呈上去,说不定也能被赐一个同进士出身,也就免去了考前几个月的紧张冲刺。

不过韩冈从来没有后悔过那一段时间的苦读。重新拷问自己对经史掌握,也让将儒学偷梁换柱成为可能。否则连基本的引经据典都做不到,靠什么说服士林?靠什么与那些大儒辩论?

“勉仲正是关键的时候,让别院那边小心服侍。让谭运过来。”韩冈吩咐了一句,便让管家退下。

黄裳如果通过了制科,自己在朝堂上的助力就多了一分。就算他考的是材任边寄而不是直言极谏,可只要是制科出身就行了,就有足够的资格进入御史台了。

在苏颂进入西府,沈括就任翰林之后,缺乏在台谏中的控制力,便是韩冈一系在朝堂上的最大漏洞了。若是台谏中有了一个可以信重的黄裳,韩冈就可以将重心在学术上,不用太担心朝堂上的问题了。

就像拼图一般,一块块的将手上的短板补足,韩冈畅想起自己对朝堂的布局,也免不了有些成就感。只是现如今气学一系在朝堂上只是有了雏形,离新旧两党一呼百应的声势差了不知多远,一切都还早得很。

谭运很快就过来了。他曾经是军器监小炉作的作头,又曾经兼领过斩马刀局,在韩冈手下办事得力,算是韩冈在军器监的亲信之一。现在就被韩冈提到了铸币局过来,与另外一名从京中钱监提拔的官员,同勾当铸币局公事。加上名义上提举铸币局的苏利涉,以及下面的几个入流和未入流的小官,共同组成了铸币局在中央的管理层。而地方上,还有几十名管理着各地钱监的低品官员,同样属于新组建的铸币局。

从地位来说,铸币局自从属于三司盐铁衙门之下读力出来,并改为现在的名字,其在官场上的序列并没有改变多少。依然比军器监、将作监这等政事堂之下第一级的衙门,要低上一等。只是因为有苏利涉这样的大貂珰来主掌,比火器局、板甲局要略高半级。

但局中的流内官之多,却是要在一干寺监之上。而且很不好管理——各地钱监的监当官要么是宗亲、国戚,要么是哪家高门显宦的荫补子弟。犯了错要打板子,立刻就有一堆亲朋故交过来帮忙说情。

幸而谭运的工作是在实务上,人事上的问题都由另外一位勾当官负责,用不着他劳心费神。而且局中人不会不知道,真正统掌铸币局的是韩冈,若是惹到他翻脸,太上皇后都不会保。这些钱监官员,聪明人不多,可会看风色的不少,没人敢犯韩冈的虎威。到现在为止,新官上任的三把火,还都没有烧出去。

铁质小平钱,青铜折五钱,黄铜当十钱,这三种币值最小、但发行量却必然最大的铸造钱币,也很快就在技术上成功了。经过小范围的试铸后,得到了太上皇后和两府认可,已经将模具发了下去,正在大量铸造之中。现在仍是存在库房内,等到冬至之后,便立刻会通行天下。

不过人事和生产上的顺利,不代表技术上的顺利。最新一次的模锻实验今天又失败了,谭运过来也是带着请罪的姓质。

试图制造锻机来压制铜币、银币、金币,在现阶段看来,依然是一个遥不可及的目标。

银币、铜币的原胚可以先铸造出来,连纹路花样都不需要,仅仅是光面的金属圆板,实在不用费太多事。可模锻冲压的机器,却是很让人伤脑筋。动力源好说,水力、风力、畜力都能用,可怎么将这些动力转化成能够连续将钱币冲制成型的机器,还没有哪个工匠能给出让人满意的答案。

另一方面,冲压还需要模具。就像铸币需要的母范一样。但能够使用几百次上千次、最好是数万次的冲压模具,需要坚硬耐磨的材质。可在这个时代,不论现有的哪一种合金,都做不到要求中的机械姓能。

不论是冷锻还是热锻,这两个问题同样都有着难以解决的问题。

不过铸币局中划拨的研发资金足够使用。模具和机器各有三组人马在不同地方刻苦攻关,谁先成功,谁就能拿到一个大使臣的悬赏。武官从小使臣升大使臣,不亚于文官自选人转京官的难度,这么大的胡萝卜吊在面前,韩冈不愁那些工匠不拼命。

论起用心刻苦,那二十多位工匠,并不比黄裳差到哪里。

韩冈所要做的,现在也只是等待了。

