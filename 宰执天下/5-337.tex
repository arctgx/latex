\section{第32章 金城可在汉图中(16)}

站在太谷县的城头上,似乎能看到极远处一道道烟火,或许是辽人劫掠后的暴.行,又或许是正在受到攻击的村庄,里面的村民在焚烧自家囤积的草料和粮食。

想到这一点时,陈丰忽然间感觉有些喘不过起来,压抑得厉害。陈家是耕读传家,农人被逼到烧掉一年的心血,他很容易想象那是被逼到如何绝望的境地了。

辽军已经来了。

自从石岭关陷落,辽军进入了太原府之后,数以万计的强盗如同蝗虫一般扫荡了乡村、城镇。综合各方传来的情报上,敌人的数目是在四万到八万之间。此外从京城那边传来的情报,自从代州陷落后,河北方向至少有超过两万兵马通过飞狐陉抵达河东。

当然,对辽军总兵力的这些预估,都是些极为模糊的数字。甚至韩冈看过了几份报告之后,都是直接就丢到了一边,一句评语都没有。

不过之后的军事计划,倒是以上限八万和预估的五万来分别规划应对的方略。只是陈丰和其他幕僚也从韩冈那里听到一点口风。辽军应该不可能将这个数目的骑兵堆到小小的太原盆地之中,这完全是浪费了骑兵的特长,蠢到了极点。

“除非他们完全把自己视为了强盗,而忘了应有的军事常识。”韩冈如是说道。

韩冈的用词听起来总有些怪,不像黄裳等人早已习惯,陈丰总觉得听得别扭,但也不到听不懂的地步。

也许那些一身羊骚.味的部族族长的头脑和想法跟强盗没有两样,但辽国的高层,从耶律乙辛到其余重臣,哪一个都不至于被抢.劫来的赃物冲昏了头脑,会合理的利用手上兵力,以期更多的好处,不会有太多浪费的。

且对于在太谷县周围的预设战场来说,骑兵的数量一旦超过五万,就没有多少区别了。韩冈甚至还说,要是辽人过来个十万八万反而更好了。

陈丰在军事上并没有多少水平,他自己也清楚这一点。能成为韩冈的幕僚,完全是一个误会。所以到了韩冈身边后,极其珍稀这来之不易的机会,努力的多听多看多学。对于韩冈的计划,虽然迟了一点,但总算是了解了。

辽军来得越多,就能更容易让他们选择南下决战,而不是抢一把就走。因为一旦胜了之后,就能抢得更多。而反过来说,堆在一起的骑兵,可比同样数量的步兵,容易对付太多——虽然陈丰对这一点有着很深的疑问,不过他还是选择相信韩冈和其他几位擅长军事的同僚的判断。

“公满,看到了什么?”章楶的声音从身后传来。

“没有。”陈丰摇摇头,回头向章楶行礼。

因为年资和官位,让制置使司中自黄裳以下的所有幕僚,没人能与章楶争一高下。不过在幕府中,陈丰和他都是新人,从心理上比其他人更接近一点。

虽然韩冈曾经让章楶跟来援的官军暂时留在铜鞮县,但他却主动带了一个指挥的骑兵,连夜赶进了太谷县城。虽然这是不守军令的行为,章楶的这个态度还是让人赞赏,韩冈也只是给了一个警告,记了一笔便抬手放过了——自然,也只有文官,而且是制置使司中的幕僚才能这么做,韩冈手下的武将是绝对不敢的。

“不要急啊。”章楶可能是误会了,来到陈丰的身边,“北虏很快就会到了。”

现阶段来到太谷附近的辽兵,还仅仅是远探拦子马,最多的一股也没超过一百骑,主力并没有南下。不过陈丰也知道,只要韩冈的计划成功,站在城头上看着辽人在城外旗帜如海,的确没有几天了。到时候,究竟是胜是败,也就在三五日之内就能见分晓了。

见陈丰的神色并没有松懈下来,章楶笑道:“担心太谷县守不住?”

“如果不是枢副坐镇城中,陈丰是不觉得六千兵马能守得住太谷城太久。”

来自开封的援军已经陆续抵达在太谷县的南面。依照韩冈的命令并没有北上,而是暂时停下了脚步,驻扎在太古城南四十里开外。表现上看,他们离得太谷县甚远,如果辽军围攻太谷,短时间内是无法赶来救援。不过实际上,还有两千兵马趁夜悄然进入了城中。

原本太谷县就剩下禁军厢军各一个指挥总计六百出头的兵力,韩冈进驻又带来了威胜军的半个将两千两百人。加上新近进入城中的两千京营兵马。就是近五千了。除此之外,还有千余在太谷县招募的兵源,虽然一时间不能形成战力,但真正打起来,都是还不错的补充兵。

“其实足够了,并不要守太久的。”章楶说道,“只要收到门关上就足够了。”

陈丰笑了一下。在幕府之中,章楶对韩冈计划是毫无保留的支持,甚至不输给黄裳这样久随韩冈的门人。现在问他,说什么都不会有别的答案。

“其实陈丰是在担心太谷周围的百姓。”陈丰指着北方的天际线,“虽然也有游骑在外解救,但还是杯水车薪。接下来受的苦只会更重。”

“所以才必须要有这一战,也为了日后能一劳永逸。何况有了枢副的《御寇备要》,百姓也知道该怎么做了。小股的北虏不用担心,人多了,那也担心不来。”

由于地处北地,又临近太行山,太谷县周边的村庄基本上都有寨墙,以防盗贼。当然,那样的围墙肯定是访不了辽军的进攻。大多数村寨,即便仅仅是百余契丹骑兵,也能很轻易的攻破。不过在韩冈的《御寇备要》公开下发之后,至少比之前的情况好得多了。同样数目的粮草,逼得辽人必须出动更多的兵力。

出外打草谷的兵力被迫增加,就等于了辽军能用来上阵的大军的总体实力在下降。春天战马本来状态就不好,长时骑乘奔驰,倒毙的数目就不会小,即便没有脱力而死,上阵后也没办法有更好的表现。

陈丰点了点头,却换了话题:“怎么枢副还没来?”

章楶也疑惑起来,韩冈今天是要巡视城防,所以他们两人才提前过来做准备。但从时间上看,韩冈现在也应该到了。

“大概是有事耽搁了。”章楶说道,望着东面,“方才东城那边好像开了城门。可能有消息到了。”

“北虏?”陈丰眼皮一跳。

章楶点点头:“多半吧。”

……………………

韩冈的确被一条新到的消息耽搁了行程。

不过并不是辽军,而是太行山中的强盗。太谷县东,靠近太行山的几间村子,这两天突然受到了下山的强盗攻击。虽然没有多少人员伤亡,也没有太多的损失,不过这个势头可不算好。

站在普慈寺的大雄宝殿中,站在河东地形沙盘前,韩冈沉吟不语。

看着将太原城东侧,一座小小的城池模型上的红色的角旗,换成了黑色的旗帜。今天最新得到的报告。榆次县陷落。井陉通道被封死。而在太原北方,一座座城镇、关隘,都已经打上了黑色的标记。只是现在,太谷县东也得用上红色和黑色以外的另一种颜色了。

“太行山中多盗贼,这几年编练保甲才好一点,不过也只是稍好一点,还是有许多盗贼出没太行东西两侧。”黄裳曾经跟随韩冈在河东任职,对于太行山中的情况,多少还是知道一点。

太行山中贫瘠无比,许多山头连树木都看不到。生活在太行山深处的山民,拿起锄头就是农民,换上弓刀那就是贼寇。根本抓不胜抓。

“枢副的《备要》的确是御寇良策,但给那些强盗学去了,日后官军围剿可就要头疼了。”太谷县的知县也在说着。

“你们总是看到小问题,须知现在最大的问题是辽人南侵,而不是太行山中强盗。”韩冈摇头,心中对敢于直言的太谷知县有几分惊奇,不过对他的话,却没多少认同,“要学会抓大放小,先解决最大的问题。”

这其实是主要矛盾和次要矛盾的问题。韩冈只是遗憾,《矛盾论》并不是他现在能写得好的。

“何况强盗终归是小时,没有问题的良民永远都是绝大多数。”韩冈又对太谷知县道:“你想想,河东户口众多,人口甚至几近千万,太行山中的盗贼又有多少,有百分之一吗?”

百分之一就是近十万。黄裳和太谷知县都摇头。太行山中真有那么多盗贼那还了得?皇帝在福宁殿中都别想睡安稳了。

“只是枢副将一些秘策都教给了河东百姓。日后若有人心怀不轨,亦是祸患啊。”

“若有贤君良臣,名将强兵,就是有人欲为寇,也只会是拿颈血一试王法的结果。”

韩冈随之一笑。士大夫,或者说统治阶层,都会对数目远远超过他们的百姓有着深深的戒惧,对失去控制畏之如虎。所以希望百姓是牛羊,不需要也不该有任何。而放开他们枷锁的,便是灾难。

但韩冈不同。一直以来,他没打算在这个时代超前太多,也努力想融入这个时代,但基本观念却是根深蒂固,更改不来的。

对于黄裳和太谷知县的担心,他只会一笑了之。
