\section{第20章 冥冥鬼神有也无(五)}

【下一更还是十二点。】

远离道路上的喧嚣,在一处狭小的谷地中,数百名骑兵或坐或卧,屈身在树荫下,互相之间的交谈都是尽量压低了声音。有的是检查着随身携带的弓箭,还有些则给腰刀打磨上油,还有一些更是闭目养神,静静的不发一言。

数以百计的战马辔头都是带得好好的,又一匹匹的十分驯服的被拴在树上,没有太大的动静。如果望向后半段略为宽阔的谷底,那里还有为数更多的马匹。

萧药师奴坐在一张小交椅上,他正等着派在前方的耳目送消息回来。

摊到这个差事,虽然心中是有些怨言,但该做的事还是要做好,毕竟是魏王亲自吩咐下来的差事。萧药师奴就是胆子再大,也不敢有半分违逆。

而且现在萧药师奴也觉得魏王殿下是高瞻远瞩,宋人的确已经不同于以往,若不是亲自来看一眼,怎么也不可能相信。

刚刚进入丰州的时候,萧药师奴都没有想过,眼下的对手会让自己这般头疼。

河东禁军的确是难得一见的精锐,而郭逵也不愧是南朝第一名将。

尽管在大辽国中,南朝的所谓名将都不过是笑话罢了,没有跟契丹铁骑面对面交过手,哪人会将他们放在眼里。但这些天来的几次交锋,不但让萧药师奴了解到宋军的强悍,也让他看到宋国将领的精明强干——郭逵的一番布置,完全不辱名将之称。

利用逼近到丰州城的主力,压着党项人不便轻举妄动。接着又分派兵员看守沿途的寨堡和要地。看似是分兵的愚行,但宋军不仅战斗力已经彻底压制住了党项人,连进入丰州的军力也远远胜出,几天来的规模虽小但次数频繁的交锋,都是以党项人的失败而告终,就算萧药师奴领军几次助阵,也不过挽回了两次面子。

在郭逵的率领下,攻入丰州的宋军的每一步都是稳稳当当,不露多少破绽。如同一只刺猬,想要咬上一口,就要做好嘴被扎穿的准备。

眼下宋军不断加强对道路两侧多处要地的控制,并逐渐将丰州城孤立出来,这就像一根拴在脖子上的绞索,一点点的被收紧,留给萧药师奴活动的空间越来越小。

‘该走了。’萧药师奴这两天一直都在盘算着。

在受命领兵来到丰州之前,萧药师奴一直以为,他可以轻松击败三倍以上的宋军,即便对手列阵而战,也能在党项人的配合下,轻易击溃战阵;若是到了宋夏两军决战的战场上,只要抓准出击的时机,他手上的皮室军铁骑完全可以在瞬间扭转战局。

可是现在,他绝对不会再这么想了。只要遇上带着神臂弓、身上披甲的宋军精锐,就算只有一个指挥,萧药师奴都会掉头就走,他手上的这点兵力损伤不起。最近的两次得手,打的都是护卫辎重的队伍。

萧药师奴奉命来此,并不是一定要帮着党项人保住丰州——党项人自己都没想过能将丰州城保下来——而是尽量帮着党项人消灭南朝的精锐。打压下南朝的气势,让他们永远畏惧大辽。如果不成,也要试一试宋军的深浅。一旦确认党项人无法独力抵抗宋军的攻势,日后大辽便会尽全力支持西夏,以防唇亡齿寒,但这前提是萧药师奴本人的名声要受点损伤。

‘实在是很吃亏。’

正在想着该走还是该留的萧药师奴,突然眼神一动,如刀锋一般锐利的视线,落在了草木森森的小路上。

两人一前一后从掩映在林深之处的小路走来,前面是萧药师奴排在外围的哨兵,后面是一名党项人,萧药师奴曾在嵬名阿吴身边打过照面的亲信,并不是他所等待着的斥候。

嵬名阿吴的亲信快步走到萧药师奴的面前,行礼后也不站起来,低头跪着说道:“启禀萧将军。我家太尉昨日听说将军又大败宋人之后,就在城中杀羊置酒,等着将军回来庆贺,不想将军去了保宁寨歇息。今天我家太尉知道将军必然少不了有捷报传回,又洒扫庭院、设下宴席,等着将军奏凯歌而归。”

在来到丰州之后,唯一让萧药师奴感到满意的就是党项人足够恭顺。昨天一战根本不算成功,只是冲了一下,看着对面的神臂弓犀利难当,萧药师奴直接就撤退了,当时被抛弃在战场上的西夏军当然都看到了,想不到嵬名阿吴的态度还是这般恭谨。

萧药师奴自问嵬名阿吴为何依然如此恭恭敬敬的,带着看透了一切的笑容:“听说前两天,你家的神勇军司吃了个败仗……”

那名亲信立刻抬头抗声:“将军误信宋人谣言。左厢神勇军司只是略有损伤罢了。而且此战也拦住了鄜延路的宋军,让他们无法来支援郭逵。若说谁胜谁负,反是宋人更吃亏点。”

“倒是会说嘴。”萧药师奴冷笑了两声,“你去回跟你家太尉说,尽管放心,今夜本将就回丰州城去,让他盯好道路,别让宋人埋伏上。”

萧药师奴说得不客气,亲信神色不变,额头向地上贴去:“小人明白,这就将将军的话传回去。”

“等等。”看着嵬名阿吴的亲信就要走的样子,萧药师奴叫住他,问道:“你家太尉何日与宋人决战,难道要等到丰州城中的粮食吃光不成?”

“此事事关重大,也只有将军和我家太尉来决定,小人哪里够资格说上一句半句。”

看着又跪下来的党项人,萧药师奴厌烦的挥了挥手,“你先回去吧。”

嵬名阿吴派来的信使传了消息后就走了,萧药师奴则摸着下巴沉吟了起来。

看起来,似乎是要与宋军决战的样子,或者反过来,宋军已经准备攻城了,所以才眼巴巴的派人来找自己回丰州去。

但萧药师奴不愿回丰州去。此前对宋军的几次试探,由于他的小心谨慎,麾下伤亡的人数很少。可如果他再回丰州,接下来的决战肯定就不会那么轻松的混过去了。

而且几天的游击下来,战马的脚力已经消耗得差不多了。连续作战,损耗最大的就是战马。往往一场奔袭下来,每一匹战马都能掉了十几二十斤的膘。现在更是逐日上阵,每日奔波不停,即便是一人三马轮换着来,也已经快到极限了。

这些天下来,宋军的战斗力萧药师奴已经看得明明白白,如果是在丰州城下决战,西夏根本没有多少胜算。想要封堵宋军的粮道也是一条计策,但成功率太低,而且首先断粮的肯定是丰州城。

手中的马鞭无意识的拍着大腿,萧药师奴的眼神逐渐凝聚。霍然而起,他已经下定了决心,今天就是最后一次,打过宋人辎重队,给魏王一个交代后就退往北面的保宁寨,那里的粮食还吃上几天,然后看丰州城下的决战情况再定行止。无论如何,他都绝不会为为了党项人拼命。

一声唿哨忽然响起,通往官道的小路上也有两人匆匆赶来。前面的是萧药师奴的斥候,后面的则是附近的铁鹞子来联络的信使。看来是他等的宋军辎重队终于到了。

不需要萧药师奴催促,他麾下的儿郎全都站了起来,收好弓刀,来到各自的战马边,解下缰绳,等着主将的命令。

萧药师奴行云流水一般的翻身上马,拔出腰刀,向前用力一挥。锋利的刀尖为麾下的将士指明方向。

随即一群最为精锐的骑兵穿林踏叶,直奔萧药师奴所指的方向而去。

……………………

折可适高踞马上,回头望望身后的队伍,辆辆独轮车正紧随他的步伐。

从府州到丰州的道路路况恶劣,两轮的马车并不好走,只能使用独轮车。一辆车要两人服侍,前面的人拉,后面的人推。就这样才能艰难的将军中亟需的辎重粮草送到前线去。

折可适这几日随军押运,最常见到的袭击,就是贼军的哨探在道边树林或草丛中时不时的射上几箭,然后就先转头逃跑。一旦押送粮秣的护卫,开始适应这样的骚扰,对一系列的异动变得漠视起来,接下来就是正菜上桌。铁鹞子和皮室军前后夹击,此前两支被击溃的辎重队,都是如此而失败的。

针对敌军的战法,宋军采用了最笨但也最稳妥的办法,监视他们可能出现的地点,然后全程都提高警惕。另外派出精干的小队,搜索道路周边,甚至看准风向,直接放火烧山。不过真正的杀手锏,还是在折可适这里。

折可适领军在补给线上巡视已经是第三天了,与铁鹞子和皮室军交手过多次,每一次都是铁鹞子被留下来殿后,而皮室军都是见着风色不对便转头离开,一点也不耽搁。皮室军临阵脱逃的行为连着几次下来,他麾下的将士倒是将契丹骑兵看得低了,并不像一开始时心怀忌惮。

但要怎么将这一队滑不留手的皮室军留下来,让折可适费尽了心神,总不能一直戒备着这支毒蛇一般狡猾的敌人。

主帅郭逵已经率领中军陆续抵达丰州城外的大营。现如今他们正在清理丰州城周边的山林,与埋伏在其中的铁鹞子和步跋子激烈交锋。当官军击败了他们,能够稳定的控制住最后的十里地后,就可以进一步向前,在丰州城下设立攻城营地。

郭逵用兵以稳为主,虽然有着压倒性的兵力,但他依然并不急进,而是稳稳的一点点将丰州的土地给夺回。

眼下丰州的诸多寨堡已经夺回了大半,只剩北面的几座,而补给线上最容易受到攻击的十几处要地,也都派了人给监视住,这样一来,就缩小了敌军可能出现的区域。所以直到现在,也没见到他所等待的敌军出现。

‘难道这一次要无功而返了。’折可适刚在这么想,忽然就心生警兆,莫名的感到一阵心悸。他连忙向周围望去,一声唿哨就在同时传入耳中。

折可适精神一震:“来了!”

