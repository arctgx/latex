\section{第七章 烟霞随步正登览(八)}

将第一份选票放在另一边,王中正随即拿起了第二份。

“天章阁直学士、右正言、同修国史、权知太常礼院王存——曾孝宽。”

在王中正拖长声调的唱名之后,曾孝宽的名下也多了一横。王存不是新党,是中立派,一贯以来都是两边都不得罪,多年都待在三馆之中,一步步升上来,显得人畜无害。但他曾受曾公亮举荐,站在曾公亮之子曾孝宽一边是理所当然。

“宝文阁侍制、右司谏、判都水监杨汲——吕嘉问。”

“龙图阁侍制、右谏议大夫、知广州陈绎——吕嘉问。”

“龙图阁侍制、起居舍人、知谏院黄履——吕嘉问。”

陈绎是老资历,年过六旬,还做过权知开封府、翰林学士,因故被贬,而后方才起复,连他也支持吕嘉问,算是出人意料。

包括他在内,连着三位都选择是是吕嘉问,一个正字很快就要填满。

选举时以正字为计,是两大联赛总社选举会首时的做法。有正大光明之意,一五一十的也方便计数,眼下也同样借鉴了过来。

看见吕嘉问一时间独领风骚,很多朝臣的心中就翻腾起来。一开始就拿到了四五张选票,吕嘉问入选三人之列,已经是板上钉钉。不可能有变化了。若其他人与他差距太远,向太后若是想跳过他,改选他人——比如韩冈——为枢密副使,也不是那么方便了。

“翰林学士承旨、右谏议大夫曾孝宽——李定。”

王中正亲手打开一张张选票,继续念着,很快人数已经有十人了,其中吕嘉问有四票,李定和曾孝宽,都是三票,而韩冈的名下却是连一横都没有。

韩冈觉得自己似乎是成了新党增强凝聚力的工具,否则吕嘉问和李定绝对不会有这么多的支持者。

但这也是他的目的。新党越是抱成团,对立的一方当然也会相应的联合起来。眼下一票都没有只是排列顺序的问题,韩冈确信自己不会给剃成光头。还有好些个旧党成员在后面。

不过韩冈可从来没打算为旧党的主张张目,更不会认同文彦博等人的政治观点。他举起的旗帜,是气学。以气学为核心,形成了一个利益集团没问题,但旧党若想要鸠占鹊巢,那就要有另一番说法。

相对于韩冈心中的笃定,苏颂则是在皱着眉。

韩冈不会就这么输了吧?

虽然说现在念出来的,都是新党的成员,而且还有吕嘉问推举曾孝宽,曾孝宽推举李定这样的选票。但韩冈到现在一票都没有,还是让人忍不住为他担心。

如果韩冈不是所谓的推举宰辅的倡议者,他遇上这样的情况,应该先行选择退出,免得结果太过难看。可这一次是实验性质的推举,韩冈作为提倡者不可能临阵脱逃,必须第一个参与进来。若是韩冈说自己不参与其中,让别人闲来做实验,这如何说服其他人?而若是到了一半就退出说自己不玩了,那更是等于是戏弄了所有人,就是太后那边,也保不住他,必须降诏责罚。

从一开始,韩冈的就只能选择坚持到底。

苏颂远远的望着韩冈一眼,至少此时,韩冈的神态依然沉稳。

过去多次直面敌人的千军万马,谁也没听说过韩冈有过失态的情况。眼下文德殿中的场面,对韩冈来说好像还是小了一点。

这样就好!

苏颂稍稍安心下来。

他早已是气学中人,格物致知的道理经由他手也多有阐发,更别说每一期的《自然》上,他的文章都不比韩冈要少。万一韩冈于此失败,威望大跌,新学气焰大盛,气学恐怕又要蹉跎一段时间了。

只是得尽快打破这片空白。

苏颂看着韩冈名下的一片雪白,又想着。

“枢密院直学士、吏部郎中、权群牧使韩宗道……韩冈。”

看到前任参知政事韩亿之孙,韩综之子,韩绛的侄儿选择了韩冈,让韩冈终于有了第一票,苏颂虽是有些惊讶,但终于真正觉得安心了。

虽说韩绛的两个弟弟韩缜、韩维都没有站在新党一边,可韩宗道却是一贯与韩绛同步,他的位置也是依靠韩家的声势而来。但现在他却是选择了支持韩冈。这让人不得不深思,他是不是在代表韩绛表明态度?

如果当真是这样,南北之争可就当真不是单纯的市井流言了。毕竟在新党之中,唯一一个代表北人的韩绛,没有选择支持自己人。

不仅仅是吕嘉问、李定和曾孝宽,就是王安石、章敦都向韩绛那边望过去,只是首相韩绛低眉顺眼,只看着手中的笏板,却不跟任何人交流视线。

得到韩宗道的一票之后,接下来又是吕嘉问和李定各得一票。

前五票就得了四张,但之后接连被跳过,吕嘉问名下的‘正’字终于可以补上最后一笔,可之前的优势已经不复存在。不过韩冈还排在第四,却是让吕嘉问在紧张之余,心中稍稍舒畅了一点。又更加盼望这样的优势能够维持到最后。

被吕嘉问远远抛在后方,韩冈接下来终于又得到了一票:“权知开封府、右司郎中、翰林学士沈括:韩冈。”

通过这一票,沈括的立场终于可以确定。更多的人想通了之前沈括跳出来对一众参选者和选举者弹劾的原因。不过这样的行事作风太过于反常,一点也不像习惯直接了当的韩冈,所以疑惑也随之而生,无法确定下来。只是当人们将视线转移到韩冈脸上的时候,依然看不见他有什么反应。

心如山川之险,胸有城府之俨。这不过是朝堂上的基本功,但到了此时还能做到完美无缺,韩冈的定力也开始让人恨起来。

尤其是下面的青绿小臣,由于事不关己,反而更想看到一些让人出乎意料的场面。譬如方才的沈括弹劾,又或者更早一点的内禅或太上皇驾崩那样场面。当然了,如蔡确和二大王的叛乱,连皇城司亲从官和班直禁卫都鼓动起来的情况就未免太危险了,能少一点就尽量少一点。

“资政殿学士、礼部侍郎、提举大图书馆韩冈:弃票。”

“天章阁学士、右谏议大夫、御史中丞李定:吕嘉问。”

韩冈终究还是没有脸毛遂自荐,自己推荐自己。不过到现在为止,除了韩冈之外,没有一张票弃权。这是侍制们的权力,若是不能好生行使,现在对不起自己,日后也难以见后辈。

李定那一边则没有任何顾忌,他推荐了吕嘉问。现在吕嘉问、李定和曾孝宽就是成了一个循环,转着圈一个推荐一个。感觉有些可笑,但站在韩冈一边的却没有人能笑出来。

啊,不对。

韩冈一直古井不波的脸上终于有了的一点变化,唇角略略向上勾起。仿佛在笑。只不过看其表情,有着很明显的讽刺味道。

范纯仁的注意力一直集中韩冈的身上。他脸上的表情变化,也没有逃过范仲淹次子的双眼。

从吕嘉问、李定和曾孝宽三人身上,能看得出是很明显的党争。

韩冈想让太后看见的就是这一个场面吗?

范纯仁想着,然后就听到了自己的名字,

“宝文阁侍制、右司郎中、知齐州范纯仁——韩冈。”

接受了富弼的劝告,范纯仁也不知道对与不对。至少从现在情况来看,还是有些希望的。

范纯仁之后,又隔了曾孝宽的一票,韩冈再一次得到了推举:

“龙图阁侍制,右谏议大夫,知应天府孙觉——韩冈。”

又是旧党。

吕嘉问都想要笑出来。

孙觉好《易》,喜读《春秋》,著作甚多,他的学术体系自成一家,与王安石、韩冈都截然不同。不去支持以新学为根基的新党好理解,却选择了讲究格物致知、对所有经义都抱着疑问的韩冈,旧党彻底倒向韩冈的苗头,也越来越明显了。

或许下一面旧党赤帜,就是姓韩名冈。

旧党越多的支持韩冈,新党成员就越不可能投他的票。韩冈这一回失败后,下一次推举参知政事,也会跟今天的情况一样。

难道要让太后将所有侍制以上官都评定了新党、旧党之后再来举行廷推?那也要王安石答应才是。

到了如今,王安石是绝对不可能允许他毕生的功业受到半点威胁。韩冈若是利令智昏,投效了旧党,一直都犹豫不定的他的岳父,也就能有一个决断

了。

宣读过的选票接近二十张了,韩冈名下的正字还未圆满,与李定和曾孝宽相同,只有吕嘉问最多。但也没将其余三人抛开太远。四人现在的票数还是成焦灼状态。

“宝文阁侍制、右谏议大夫、河北都转运使李承之——韩冈!”

王中正的一个重音,让韩冈得到了第五票,也让吕嘉问的脸色为之一变,曾孝宽和李定一时间都是难以置信。

李承之竟然选择支持韩冈!

李承之可是新党的一份子,而且是绝对的中坚。免役法的提出者之一,变法之初就是参与者。在新党中,与章敦的交情极好,当初将章敦荐到王安石面前的,他是其中一人。要不是在三司使任上犯了错,被降官出外,眼下必然也是宰辅之位争夺者之一。

曾孝宽知道韩冈与李承之有交情在,可新党的几位核心,谁与李承之没有交情?

难道就因为他们都是北人?!

但接下来又是一票支持韩冈,却是标准的南人,而且是旧党,“宝文阁待制,右司郎中李常——韩冈。”

李常是王安石的老朋友,却因为变法之事,与王安石分道扬镳。多年在外任官,最近刚从淮西提点刑狱任上任满回京。本是要转任,却因病留在京城,一时间没有出去,眼下正好在殿上,也拿到了一张选票。

南北分立,新旧党争,并不可能截然两分。北人可以投向新党,南人也可以投向旧党。不过投向吕嘉问、李定和曾孝宽三人的选票,其投票人的籍贯,却微妙的显得过于偏靠南方。

“福建,福建、江东、开封、福建、江西……吾今日终于知道了什么叫做党同伐异。”

向太后的笑声让身边的杨戬毛骨悚然。在太后的手中,拿着一份名单,在每个人的姓名之后,是那位官员的籍贯。一眼望过去,基本上全都是南方人,北人寥寥可数。

新党、旧党有时候难以分辨,除了那些旗帜分明的两党核心人物,剩下的很多人都是谁在台上便支持谁,不过籍贯就无法作伪了,能看得一清二楚。

而新党多南人,旧党多北人,这是世所共知。韩冈的支持者虽少却遍及南北,而其余三人的支持者加起来却都集中在南方,只有一二例外。

就算看个几十遍资治通鉴上的牛李党争,也没有眼前的活剧更加直观。何况向太后根本就不去看司马光的《资治通鉴》,尽管最近刚刚又呈上一部分新写好的篇章。

“不过,既然吾已经答应了,就这么办下去吧。看多了,也就知道了。”向太后冷笑着。

