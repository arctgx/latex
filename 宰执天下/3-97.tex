\section{第29章 百虑救灾伤(二)}

听说挖出自流深井就能,井十六兴奋的满脸涨红“富顺监的盐井,往往深有百丈,非此不得出卤出气。不过若只是深水井,那就只要二三十丈就够了。用着开盐井的方法,日夜不息的话,最多一个月便可见成效。”

“开盐井的方法。”韩冈听了有了些兴趣,问道:“跟普通开井有什么区别?你之前没有用吗?”

井十六磕了一个头:“县尊明鉴。小人所说的开盐井法,乃是富顺监中独一份,外地从没有人见。小人怕随便用出来,会给人认出身份,所以都是用着寻常的掏井法。”

“是用钻……”韩冈刚开口,就自嘲的摇头,这时代哪有钻机。

“钻?不是!”井十六也摇头,“是用石头砸,还有唧筒……”

井十六想为韩冈解释一下富顺监盐井到底是哪方面的独特,但他比划了半天也解释不清楚,反都让人听着糊涂。最后急得满头是汗,在韩冈面前嘣嘣的磕头谢罪,“小人嘴笨。这活计也是祖传下来,自小看着父祖怎么做才学会的。空口白牙,一时说不明白。”

韩冈摇摇头:“也罢,到时候本官再过来看好了。不过你要记住,过去你敝帚自珍,那倒也罢了。但如今你想要本官荐你为官,这一套钻井法可都是要献于朝廷,传于天下。日后就不是你家的祖传秘诀了,这一点你要好好考虑清楚。”

“不要考虑,不要考虑。”井十六却把韩冈的话,当成责怪自己没有将钻井的手段说出来的,心中更是着急。脸上的汗都收了,脸色一下都变得发白,变成了一只磕头虫:“小人愿意将开井密法原原本本的献出来!”

韩冈弯下腰亲手将他扶起来,笑着安慰道:“这些先等打出深水井再说。若没有个例证,什么都是空谈。至于人手,我会安排人听你指派。只要这件事办得好,你以后也不用姓井,完全可以恢复旧姓!”

井十六惊讶的张开了嘴,完全没想到自己不说,韩冈就已经知道自己现在所用的姓氏是假的。

挥手示意仍在愣着的井师离开,韩冈回头问着身后的幕僚:“觉得怎么样?”

方兴摇头道:“总觉得不靠谱啊……”

“他不是说了吗?富顺监的盐井能深达百丈,深水井只要二三十丈,也不算离谱。”

富顺监应该是后世的自贡,韩冈虽没去过自贡,但当地的盐井名气甚大。能流传到千年之后,可想而知,其中的技术也不会太过于落伍。

方兴皱着眉:“可谁能保证一定会出水。能不能碰上水脉,都是要看运气。这井十六前面挖的水井,也只是比其他人出水要高而已,并不是说十成十出水的。一直挖到石头还没有什么水的枯井,似乎也有好几口。”

“都这个时候,什么招数都要用上。撞上一个是一个吧!”韩冈的叹气声说着自己心中的无奈,“何况本来就没指望过他。”

在井十六出头之前,韩冈本就是准备以打造各种器械来提水。要不然他张榜悬赏,将唧筒、提水滑轮等一系列现有的器械列出来又为何事?

比如唧筒,利用其原理可以开发出后世农村常见的手压式提水机,再如提水滑轮,可以由此改进成畜力水车。韩冈所期盼的一开始就是能在普通水井中通用的机械,而不是少见的自流井。

但韩冈对井十六的看法,其实就跟王安石之前要用破冰运粮的情况一样,如今的旱情看起来还会延续下去。先不管能不能成功,看着这些似乎能派上用场的招数,总得试上一试才甘心。

以王雱都免不了要连夜奔波。此等危急存亡之秋,哪还余暇考虑能不能成功?

“世上本来就没有百分百成功的事,就是开一眼普通的水井,也不一定能见水。秋来的大旱,让许多水井都干了。换作正常的年份,那还会有开十眼井才一眼井有水的?”韩冈说道:“即便是第一口不出水也没有关系。不一定要见水,只要知道怎么凿井,有了足够的人力之后,可以普遍撒网,终究还是能撞上几个的。等流民多了,还怕没有人力可用吗?”

一口自流井,如果是在工业发达的后世,一下就能给抽干掉,但在如今,仅仅是用来饮水和灌溉,情况会好上不少。深层地下水比表层的要干净,即便不能自流,日常饮用也不错。洁净的井水能大大降低疾病的发生率。

瘟疫是个比较宽泛的名词,其中有各种疾病,完全不能归纳到一处,唯一的共同点就是它们都是烈性传染病。而在这些病疫中,与水源、饮食有着千丝万缕联系的痢疾占了很大比例。至于其他烈性传染病,也是可以用洁净的饮食和整洁的生活环境来降低发病率。

韩冈起身走在流民营中,视察着新近搭建起来的窝棚,方兴连忙追在他身后。

整整齐齐排列在营中路边流民营的窝棚,都是半地下式,对着路面开得大门,要下去几个台阶,才能进去。窝棚陷在地下有近一米深,从地下挖出来的泥土又当作外墙垒起,为此节省了不少建筑材料。

不过这不是韩冈自己的主意,乃是此时北方经常能见到的穷人家的住宅。住在这样的窝棚中,保暖的情况要比全地上式的好上一些,可是不能防雨,只要大一点雨水,就能灌进窝棚中。但是如今的情况,要是下了雨,恐怕这里的流民还是欢喜的为多。

韩冈看过几家窝棚,甚至进屋看了一下,但污浊的空气让他心头多了一点忧虑。发现他现在要考虑的,不仅仅是饮用水的问题:“石灰窑也得赶紧建起来,预防疾疫都得靠石灰,还有室内的通风和卫生,都要向流民加以宣讲。”

石灰水是最为简单易行的消毒手段。依照韩冈订立的制度所建立的任何一个疗养院,都是将石灰作为一项最为重要的药物而采办。甚至在秦州、陇西两处的疗养院,都有自己的石灰窑。到了白马县,没有不用的道理,何况还能用作简易水泥,可用的地方有许多。

方兴点头记下。而韩冈也从怀里掏出个小本子,用着小小的碳笔条在上面,草草的写了几个字。

立德、立言、立功。对于儒者来说,那是毕生所求。韩冈每做一件事,也都会一一记录下来,然后总结归纳。不论是疗养院的制度,还是后来主持的后勤运输,韩冈都有规章制度问世,被赵顼赞许后,已在军中开始推行。

如现在的流民安置,韩冈也准备写点东西出来。救灾救民只是短时间而已,过去了也就过去了。不及时加以总结归纳,日后就没有仿效和改进的目标。

“还有蝗虫!”在韩冈的本子上,前后分成了三个部分,流民一桩事,抗旱也是一桩事,另外还有蝗虫:“还要养鸡养鸭来对付明年的飞蝗。”

方兴一听,忙着摇头:“鸡鸭之物,可不一定有用。”

“此事我又哪能不知?”韩冈叹道:“养鸡养鸭只是辅助而已,就跟井十六的深水自流井一样,不会作为主要手段。到时候,还是要以组织民力灭蝗,花钱来买蝗虫为主。一斤蝗虫给个十文八文,没有说不愿意的,也可以让小孩子出来挣点零花。”

“正言想得周全。”方兴轻轻赞了一句。做事最怕就是不管不顾的一条道走到黑,事先将方方面面的都想到,并留下改正的余地,这才是做事的正确方法。

“旱情一桩。流民一桩。蝗虫又是另外一桩。”

此外还有从宿州运粮的事,虽说对外要保密,也不用自己来督管,但怎么说都是与自家的发明有关,还是要挂在心思上。一根根屈着手指,韩冈发现自己除了正经知县要做的工作以外,身上担的责任未免太多了一点。

方兴听了也在叹气:“蝗、旱、流民,这都是天灾人祸,各地的知县知州,无不是直接推到上面去,要些赈济下来就够了。只要能吃到朝廷施舍的稀粥,灾民们也会跪下来磕头,叩谢恩德,没人能说这样做有什么错。”

韩冈笑了:“说的也是,现在的辛苦,纯粹都是我自找的。”不过走了两步,他却又道:“只是这些事,家岳自找过,富彦国自找过,韩稚圭也自找过。有贤者表率于前,韩冈也不敢后人呐!”

方兴低头,向韩冈拱了拱手。不避繁剧,视民如伤,这是如今官员中难得一见的美德,遇事就趋吉避凶、没有担待的官员反而多见,当然值得敬佩。

韩冈这番话,也完全没有掩饰他的野心。可这又是理所当然,二十二岁就做到了右正言,若还没有一望公辅的胆量,那就不是谦虚,而是怯弱了。

而方兴他现在所辅佐的韩冈,在胆量上所得到的评价,从来只有胆色过人四个字。

