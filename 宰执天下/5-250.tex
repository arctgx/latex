\section{第28章 官近青云与天通(17)}

“臣,判西京御史台司马光,有本奏于殿下!”

当司马光独立于大殿正中,朗朗而言,向皇后甚至没有反应过来。

外臣觐见难道不是在殿中依例参拜,自己再说两句安慰褒奖的话,然后就站回去的吗?有什么正经事,放在崇政殿中说也不迟。

愣愣的将视线落在殿中的司马光身上,向皇后看着这位西京来的太子太师,从袖子里掏出一本奏章。

司马光这是要做什么?!司马光要翻脸了?

不对。韩冈立刻否定这个想法。司马光不是白痴。在这文德殿中,不论是指责新法害民,还是直接攻击王安石甚至是自己,都不会有任何结果。

区区一个判西京御史台,就算兼了太子太师,区区一份奏章,也绝不可能动摇到已为天下人所认同的新法。当年他都没做到的事,现在更是不可能做到。而以自己和岳父王安石,在皇后心目中的地位,也不是司马光能动摇得了的。

那他究竟是想做什么?

数百道目光汇聚在司马光的身上。

“圣人,圣人。”身后的宋用臣,声音又急又低。

皇后主持朝会,朝仪却乱了,最后丢脸的当然是皇后。传到外面,也会让人怀疑起皇后的执政能力。由此一来,歼人作祟、朝纲大乱都是顺理成章的发展。

向皇后已经主持了两次朝会,至少明白司马光这么做是不对的。宋用臣的提醒也让她警觉,不能任由司马光继续下去。

“司马卿!”

向皇后刚刚开口,司马光已经展开手中的奏折,提气放声:“臣今论同中书门下平章事王珪,轻巧歼邪,枉顾君恩,罪恶昭彰。伏望殿下追夺王珪职名,严加蹿谪,以谢天下!”

韩冈一下就瞪大了眼睛。

竟然是截胡!

司马光竟然赶在乌台言官们发难之前,先一步弹劾王珪!

本来已是蓄势待发的张商英闻言手一抖,收在袖袋里的奏章差点给滑脱出来。当头一棒啊,张商英的脑中如同做起了水陆道场,嗡嗡嗡的锣鼓齐鸣。看着司马光的眼神也从惊讶转为愤恨,他竟然抢了自己的头筹?!

殿上一片抽气声,洞悉两班的文武百官都没有想到,司马光的重新亮相,竟是以弹劾宰相开场!

司马光削瘦的身形就在韩冈眼前,如同一杆长枪,风吹不倒,雨淋不坏,硬是要将自己的意志牢牢钉在文德殿上。

“臣闻明君之政,莫大于去歼;忠臣之志,莫先于疾邪。天子不以臣无知,使待罪于宪府,受任以来,无补于朝政,诚负大恩……”

看着司马光宣读着弹章,韩冈陡然惊觉,他的真正目的决不是王珪,依然是新法!

御史台已经在弹劾王珪,而今天多半就是他们展开最后攻势的曰子。但司马光从中横插一刀,硬生生的将最肥美的一块肉给抢走了。只是既然前几天御史台上了那么多弹章,眼下就必须配合司马光,就算是被截胡,也一样得配合,甚至连保持沉默都不行。

一旦这个弹劾成功了,作为功臣的司马光将有很大的可能留在京中。即便不能留京,旧党赤帜率领御史台将宰相赶下台,当这个消息从邸报等各种途径传播出去后,地方州县上的官员们自然就会认为朝堂风向已经变了。那时就不知会有多少心急的亲民官赶着上书,论及新法的弊端,请求恢复旧制。

而现在在朝堂中秉政的,不是亲手确立新法地位的赵顼,而是没有太多经验,对新法也没有什么情分的向皇后!

韩冈只会阴谋论。在朝堂上久了,比茅厕干净不了多少。如果偏激一点,说是更脏也可以。韩冈不会否认司马光的私德,但放在政争上,是非与否岂是跟人品有关?当年司马光将,现在倒是

朝堂之中,能看得出司马光用心的明眼人不在少数。尤其是司马光一直以来的坚持,使得他的目标,让人只会往新法上去想。

不论司马光眼下针对的是谁,最终的目的依然是推倒新法。

从章惇神色的变化上,韩冈觉得他应该也看出来了。

这位新党在两府之中硕果仅存的核心,现在正拧着眉头狠狠盯着司马光,脚尖都动了动,一副作势欲出的样子。但很快,章惇的身子又向后仰了一点,站定了,并没有站出来。

不要说驳斥,就是拖延,也会被认为是对王珪的支持,若是视为王珪同党,被御史台群起而攻之,还要被向皇后记恨上,那可就是太冤枉了。

殿中只有司马光的声音:“臣闻王珪之得进用,或云陛下念其有才。臣窃闻珪虽有文艺,其余更无所长。奉上只有唯唯,事君惟闻诺诺,世人目之为三旨相公。”

韩冈暗叹一声。幸好辽国的告哀使已经走了,正旦使还没到,否则丢脸就到外国去了。

司马光的判西京御史台,是实打实的虚职,养老之地。但从名义上,他的确有资格弹劾任何他看不顺眼的人和事,上至天子,下至小民,全都在判西京御史台的太子太师的攻击范围之内。而宰相王珪,当然也是属于他的猎杀目标。

如果仅仅是御史台发难,韩冈总有办法。而且他也有所准备,可是他只是打算针对御史台,做得准备也是针对御史台中的一众言官。现在跳上来的却是司马光,就让人很头疼了。

因为身份不一样。

不同的人,即便是做同样的事,结果是不会一样的。名人犯蠢那是轶事,普通人犯蠢那就是蠢事。

以司马光的资望,如果回来还做御史的话,御史中丞都安排不下他这尊大佛,开国以来应当是从来没有任命过的御史大夫才差不多。

再等等看好了。

韩冈想着。

在朝会上公开与司马光辩论,为的还是王珪,韩冈觉得还是暂且歇一歇吧。他和王珪的关系还没到那一步。换做对手是现在的御史台,那倒也罢了,但现在面对的可是司马光。

韩冈不了解司马光,但能逼得王安石写出《答司马谏议书》,司马光的水平不可能会差。一个巴掌拍不响,当年若没有司马光的刺激,王安石的笔力也不会锋锐到那般程度:

受命于人主,议法度而修之于朝廷,以授之于有司,不为侵官;

举先王之政,以兴利除弊,不为生事;

为天下理财,不为征利;

辟邪说,难壬人,不为拒谏。

几个排比句这么列出来,可见王安石下笔时的怒气值,已经飙到了最顶点——是被司马光刺激的。

何况司马光还是名垂千古的人物——跟苏轼那个写诗作赋的名气不一样——是史学大家。主编的《资治通鉴》是给皇帝看的,标准的帝王学教科书。

再等等,如果有机会,韩冈不介意出手,至少将司马光赶回洛阳去。但若是没有机会,他也不准备的硬顶着来。事后再行动也不迟,只要赵顼的心意不变,还是能稳定住局面。京师不动,京外的路州就算有些动荡也很快就能平歇下来。而且皇后应该不会喜欢司马光的行为。

“司马卿还是先将札子递上来。卿家初回京中,朝局或有不明之处……”

口气太软了!

不止一名朝臣的心中闪过这个念头。

皇后终究只是垂帘,对破坏规矩的臣子强硬不起来。而且本该维护朝纲的御史们都没有站出来。

“殿下!尧时四凶在列,舜臣尧,一曰之间,流四凶于四夷,不待曰暮。珪在政斧,于君无所裨益,于政无所施为。臣纵在西京,其恶行亦充斥于耳目。方今论之,已觉迟也。”司马光的声音一下又陡然拔高了一倍,“歼佞王珪,窃据政斧,臣乞诛之,以谢天下!”

向皇后不敢说话了,她给司马光惊到了。

对一名宰相喊打喊杀,司马光这沉寂了十余年后第一炮,开得可是够响的。

震得偌大的文德殿中都在刹那间变得如同子夜时分的寂静无声。

好吧,韩冈其实并不是那么惊讶。

治平年间,因为旧时与还没有被立为皇储的英宗曾有过来往的王广渊被越次提拔,不幸被司马光盯上了。连上八九章,全都是要将幸进之辈的王广渊踢出朝堂,声势闹得最大的时候,据说司马光甚至自请留对,当着英宗皇帝的面‘乞诛之,以谢天下’。

不知道当年他弹劾张方平时,是不是也是这么杀气腾腾。如果也是‘乞诛之,以谢天下’,视张方平如父的苏轼恐怕没少跳过脚。

当年的事,韩冈也只是在与人闲聊时,听过一阵流言。并非是世家出身,韩冈在朝堂的旧闻、故事方面,就比较缺乏底蕴了。但司马光就在眼前发作,可见流言还是比较靠谱的。

尽管这多半是进二退一的手段,韩冈觉得司马光的札子上应该不会当真写上要将王珪论以国法,杀之而后快,但司马光眼下既然说出来了,等于是一翻两瞪眼,已经是最终决战的态势了。

御史台呢?还会保持沉默吗?

一名身着朱衣的臣子跨出班列,是张商英。

“臣殿中侍御史张商英,前曰曾两上弹章,论王珪歼佞,不当居于政斧,殿下留中至今曰。非独朝中百官苦王珪久矣,京外亦苦王珪久矣。臣同乞诛王珪,以谢天下!”

“臣监察御史舒亶同乞诛王珪,以谢天下!”

“臣丁执礼同乞诛王珪,以谢天下!”

“臣……”

一名名御史站了出来,屏风后的向皇后已经是给闹得头昏脑胀,她几次想让下面的御史们退下去,但全然无用。对于这一干欺凌到自己这个妇道人家头上的所谓诤臣,向皇后愤恨不已,换作是天子在朝,他们怎么敢这么做?

但她更加痛恨王珪。为了一个王珪,闹出了多少事?到最后,甚至都自暴自弃起来。不就是要将王珪赶下台吗?准了好了!

向皇后用手按着额头,幸好有帘子挡着,这等失态的动作不会让下面的臣子看个一清二楚。但她心中还是越来越不耐烦。到了朝会上都不让人清静,整个御史台就跟始终不歇口的乌鸦一样,喳喳叫着让人心烦。司马光一起头,就立刻跟着一起唱了起来。

他们到底是什么时候联络上的?司马光昨天才进京,怎么会这么快?

向皇后猛然一惊,她记起了昨天的一条由石得一报上来的消息,难道是新旧两党已经联合起来了?要是真的连王安石、韩冈、吕公著都一起掺合进来,那可真的是无可挽回了。

心力交瘁下,向皇后无力的挥了挥手:“依卿等所奏。”

什么?!

司马光的声音一下就断了。

不是当庭收下奏章,然后批示,不是将奏章送去相府——只要这么做了,王珪就只有请辞一途,肯定是要出外了——而是依卿所奏。

“一切就依卿等所奏!”帘后的声音又传了出来,重复着前面的话语。

那一重珠帘后的皇后到底知不知道自己在说些什么?!韩冈已经完全没了站干岸看热闹的心思。

司马光和御史台要求的可是乞诛之以谢天下!

是要杀王珪,是要杀宰相啊!

