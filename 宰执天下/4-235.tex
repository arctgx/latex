\section{第38章 岂与群蚁争毫芒(八)}

韩冈回到下榻的住处,没过多久沈括就过来了,带着他的长子沈博毅一同来拜访韩冈。

沈博毅眉眼与沈括有几分相像,还没考上进士,本来是在家里读书,却被继母看着碍眼,被赶出了家门。也就二十六七岁的样子,跟韩冈差不多年纪。

只是年龄两人虽然差不多,但气质就差得很远。韩冈久居高位,说话时,幽黑的双眸盯着对方的眼睛,慢条斯理的语调,随时随地都给人无形的压力。寻常官吏被他一盯,早就汗流浃背,没几个还能想着他的年纪。

而沈博毅历事不多,就差得很远。在韩冈面前有些束手束脚,没法放开来。

韩冈随意跟他聊了几句,见他问一句才答一句,很是紧张的样子,知道自己的身份给人的压力太大了,也就不跟他多说,勉励了几句,转过来与沈括闲聊。

“不知存中兄听没听说过蛊胀?”

“这个病只在南方有。”沈括点点头,这是很可怕的一种疫症,他当然不会不知道,“在两浙见识过,疫症所及,连着几个村子都不见男丁。江南各路都有发病,加起来少说也有数十万人之多……难道玉昆知道该如何医治?”

“其实蛊胀广西也有,不过患者不多,远比不上襄州。”韩冈摇摇头,答非所问。

“襄州?”沈括闻言脸色便是为之一变。

韩冈点头:“正是襄州。”

襄汉漕渠的难关并不仅仅在方城垭口一处。其他地段其实也是有些难以解决的问题。比如襄州的港口,肯定是要扩建,但那里一直都有名为蛊胀的疫症,传播范围很广,如果预防治疗不当的话,会大大拖延襄州未来的发展。

韩冈看过疫区的好几家村子,里面的村民大半都是重病缠身,各个面黄肌瘦,有很多人病入膏肓,整个人干枯瘦小,肚子因腹水高高挺起——蛊胀之名由此而来。而得了疫症的儿童,则是如同侏儒一般矮小。

以韩冈见识,所谓的蛊胀,根本就是血吸虫病。南方河流湖泊为数众多,得了此等病症的自然不会少。这个时代缺乏后世一服见效的特效药,而一吃多少服才能有效果的药方,又不是村民能承担得起,一旦得病,就很难再治愈,只能等死。

“此病一发,便是下痢便血,三人之中便要死一人,等到急性病症减退,就会转成慢性,肚腹便会因此鼓胀起来。我曾询问过多人,主要是急性发病的,都是下过水后不久便发作,由此观之,此病应是缘水传播。”

韩冈的推测让沈括觉得有些牵强,“荆襄多河湖,有水的地方很多。河边一走,哪有不湿鞋的?”

“的确。”韩冈点着头,“并不是有水的地方就有疫症。蛊胀发病只看地区,出了疫区便少有人得病,只可能是水土有别与他处。所以我就让人从疫区和非疫区的河滩上取土,在不同地点各取了五十份。一番对照后,发现这些土壤的成分都差不多,只有一样是前者明显多过后者,远远多出许多。”

“是什么?”沈括立刻追问。韩冈这番推理倒是有了几分道理,沈括也专注起来。

“螺蛳。”韩冈说道,“当然不是普通的螺蛳。只有两三分长,一分宽,像根极小的钉子,所以我也为其顺便起名叫钉螺。而这钉螺,正是蛊虫的源头。”

“蛊虫的源头?如何得知?”

“放到显微镜下,便能从钉螺中能看到散出的无数蛊虫,有头有尾,能游于水中。”

听韩冈这么一说,沈括便心急难耐,只想要找个钉螺来看一看。他和韩冈共同语言很多,不只是开通在即的襄汉漕运。

“旧显微镜不成,太模糊,必须是加了水银镜反光的。”韩冈进一步补充道。

沈括放下了急性子,笑道:“玉昆你前日让人送来的水银镜我已经装在了显微镜上。有了水银镜反光,用显微镜时,看东西就清楚了许多。原本看不清的东西——即便是细如发丝,落到显微镜中,便是纤毫毕露。”

“小弟本也是苦恼着怎么才能够让显微镜更加有效。后来突然有了一个想法,所以一等水银镜造出来后,小弟就立刻装在了显微镜上。”

韩冈让人打造水银镜可不是为了给家里的妻妾当镜子用,铜镜也能凑合,而且还耐用。小小的银色镜片,其实是装在显微镜上的配件。

“玉昆你之前,谁都没想到汞锡齐【注1】还能派上这种用场。”沈括说起显微镜就很兴奋,“前两天拿来看着树叶,到处都是脉络,细密如网一般。本以为已经很难得了,没想到还能看到蛊虫。”

一直沉默着的沈博毅突然插嘴:“其实孩儿也拿着显微镜到处去照,干树叶,干葱的皮,透过显微镜,能看见里面一格格如同蜂巢一般。”

“那一物,我称之为细胞。”韩冈吃惊于沈括的儿子竟然也对显微镜感兴趣,而沈括父子则对韩冈几乎全知的能力感到惊讶,“所谓聚沙成塔,百丈之塔,起于沙砾砖石。动物、植物,皆是一般无二,全是由无数细胞构成。”

“……与元素论很有几分相像。只是一个是原子形成万物,一个是细胞合成生物。”

韩冈并没有将元素说和原子论公开发表,但在沈括面前已经提起过,并多次探讨过。

太虚即气,这四个字就是气学的根基。万物皆由气所化,也就是所谓的‘天地之塞,吾其体’。但天地本源之气——或者用‘炁’更为合适——究竟是如何化为万物,张载并没有说,自然也无法说。五行过于虚幻,难以实证。

但韩冈能说得明白。炁凝为不同元素,元素化为万物。天地万物,都能归结为某一个元素,或是元素集合。

韩冈已经将原子、分子论和元素说向沈括和盘托出,最后也得到了沈括的认同。作为一名专注于自然之道的学者,在沈括看来,韩冈提出的理论,至少是最符合实际的,同时也最有用的。

“原子不可分割,不会变化,细胞可不能。所以说点石成金是梦幻泡影……是痴人说梦!”

“点石成金还是能做起来的。只要能先将石块还归本源之炁,再重凝为黄金。”

“还归本源之炁?那可得有重开天地之力。”韩冈笑道,“天地初辟,太虚中分,清炁为天,浊炁为地,元素凝于清浊之间,化为万物。溯本追源容易,但复散为本源之炁就难了。”

这是韩冈的世界观,糅合了气学和科学的成果。

十年的时间,韩冈在工作之余,精力大多都投入了来,已经有了自己的一套从气学发展而来的学术理论。将后世的科学体系镀上一层儒学的光辉,这一份初衷,如今也算是有了小成。

化学必须建立在元素、原子、分子论上,否则就缺少了最重要的根基,虽说要推翻五行论有些难,但不经过这道关口,是无法成事的。

不过这个文明的特性,让韩冈宣讲起科学理论来,比起西方的贤者要轻松许多。基本上都是实用主义者,只要能带来现实中的成果,任何道理都能让人信服。

就像韩冈随便拿了汞锡齐作为镜子,便能以此证明自己的观点。

“这便是明体达用之功。天生万物,各有其用。明其根本,便能用之于实处。”韩冈冷笑了一声,“过去的那些个道士,拿着炉鼎烧来烧去,只是为了炼个外丹,道书上一提丹法,全都是云山雾绕的玄虚之谈,就没个实用的,最后就是拿来骗那等贪心的乡绅。”

水银镜,本质上就是用水银融化了锡之后形成的化合物,也就是汞锡齐。很早以前,炼丹术士们就发现了这一个闪闪发亮的化合物的制备方法,但一直没投入实用,如今则是成了如今世间常见的炼金骗术。

只要在黄金外镀上一层汞锡齐,看起来就像是一块白银。而将这个白银往火里一送,镀层被烧化,里面的黄金便显露出来。多少骗子就靠这一手来招摇撞骗,骗了一家又一家,没有见识的土豪这世上是在太多了。

沈括摇头叹道:“那等骗徒,若当真有玉昆你的半分眼界,只靠着制镜,数年间便能为一巨富,何必辛辛苦苦的去骗人。就是用来照人,也比铜镜强的太多。”

韩冈则不同意沈括的说法,“水银镜是好,就是太容易磨花了,不是很实用,除非上面能有一层透亮如冰的东西做遮挡。”

“一片水晶是最方便的。”沈括忽然冲着一笑,“这也算是明体达用吧?”

韩冈点点头,“先格物致知,进而明体达用。”

格物致知和明体达用是相辅相成,通过研究事物的本质,明了自然之道,也就是规律。而在了解到自然规律之后,便能投入到实际应用之中——‘明体’,而后‘达用’。

……这就是韩冈的道。

注1:中国古代称化合物为齐。

