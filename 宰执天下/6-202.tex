\section{第22章 鼓角连声彩云南(中)}

天上的云纯白,衬在更为澄澈的蓝色天幕上,白得极为鲜明,仿佛用笔勾勒出了界线,蓝白分明。

赵隆很久没有见过如此让人神清气爽的天空了。

他在延州待了很长一段时间,延州的天始终就是灰蒙蒙的,一半是因为横山对面刮过来的沙土,一半是因为延州城内百姓每日不可或缺的石炭。一年四季,天空总是蒙了一层灰色薄纱,天空是灰蓝的,云是灰白的,身上的衣服即使是新裁的,只要在外面走上半日,再鲜亮的颜色也会变成灰蒙蒙的。

京师的情况,也跟延州差不多。这还没算北方常有的沙尘。一旦风沙掠起,戴着口罩也不免满嘴的沙土。

更别提大理的青山绿水,根本不能相提并论。

皇帝修行宫,世家大族造别墅,都是为了远离污浊,享受山清水秀,而这里水土气候,足以让长安城外的终南山相形见绌。如此偏南的地方竟然四季如春——按军中幕僚的说法,其纬度与两广相同,就在两广的正西面——生活起居自然更是宜人。

对比眼前的这片四季如春丰饶喜人的土地,赵隆感觉他自幼生长的关西,仿佛就是西域的荒漠一般。只要不去周围的崇山峻岭之中,不去那些瘴疠之地,洱海之滨的生活,来自北地的汉人完全能够适应下来。

“真是好地方啊。”

“怎么,喜欢这里?”

大军扎营之后,熊本巡视营中。走到营门前,便看见他麾下的第一战将,正望着远山近水。缓步走近,迎着风,听到赵隆一声低语。

“当然喜欢。”赵隆回身向熊本行礼:“大帅。”

熊本走到赵隆身边,远眺远处的大理城。

峰峦起伏的苍山下映在洱海万顷碧波上,深色的城墙静静匍匐在苍山下,阳光柔柔的洒了下来,青的山、绿的水、黑的城,完美的交融在了一起,说不出的赏心悦目。

“说起山清水秀,江南、蜀中都不输给这里。但江南夏日暑热、冬日湿寒,蜀中则多阴,只有大理这里,方有四季长春,百花不谢。”

“大帅说得是。”

“既然喜欢,在这里多留两年如何?”熊本抬眼问道。

大理灭国在即,战后官军回师,则必须要有大将留守在这里,等到局面稳定再离开。赵隆明白,这个人选,除了自己就只有还在鄯善府那一边的李信了。

“只要朝廷有命,赵隆不敢推辞。”赵隆正色道,又笑着说,“不过这里的饭菜,末将是吃不惯,不管怎么做都没关西的味道。要是能白天在关西吃饭,晚上回这里住就好了。”

熊本盯了赵隆两眼,笑了起来,“子渐,老夫曾经听过一个笑话。京东那里有家人,他家有个正当年的女儿,被两户人家求亲。东边一户人家的儿子,貌丑,但他家里富,西面那家穷,但他家儿子长得标致。东西两家,各有各的好处,也各有各的坏处,有女儿的那家父母决定不了,就让女儿自己选。选东家,露左手,选西家,露右手,在东家吃饭,西家睡觉。”

听了熊本故事的开头,赵隆就已经在苦笑了。

“鱼和熊掌哪个都想要,就是要不到,才要有所……”熊本终于注意到了赵隆的表情,“……怎么,听说过了?”

“这个故事,末将当年听襄敏公说过,大帅一说,正好想起来。”

“哦,难怪……”熊本有了兴趣,以王韶的性格,当不会无缘无故讲古,“所为何事?”

听到熊本这样发问,赵隆却支支吾吾起来,“其……其实也没什么,就是寻常闲聊时的话。”

熊本笑了一笑,不再多问。赵隆一向敢说话,能让他不好说出口,多半就是跟韩冈有关,“子渐,鱼和熊掌到底该选哪一个,全凭你自己,朝廷不会逼迫人。你只要记住一点……不论选哪个都不要犹豫太久。”

赵隆拱手道,“大帅放心。末将明白。”

“大理城就在眼前,”熊本转头遥遥指着远处那卧于山下的阴影,“拿下那里,鱼和熊掌,想吃哪个,都随你。李信已至鄯善府,转眼就能攻下来,等到他来了才能攻打大理城,你我的面子可就要丢尽了。”

前方有熊本坐镇,黄裳在后方主持粮秣,加上参加过大宋历年战争的士兵军官以及幕僚,充斥于南征军中,这让整个帅府行辕,以及其麾下大军,像轨道上的马车,顺滑而又一刻不停的运作着。

但在大军分路并进的情况下,各路主帅免不了各具私心。遇强敌,则是畏难不进,希望祸水他引;遇弱敌,则争先恐后,给了人设伏并各个击破的机会。

当年熙宗皇帝以举国之兵征讨西夏,差一点就功败垂成,就是因为将帅争功之故,所以这一次,领军将帅的尊卑高下分得很清楚,谁为主、谁为辅,看了官位就一清二楚。

所以熊本从来没有担心过广西的李信能快他一步。

李信手中的军力不足,出兵时间也晚,只是偏师而已,但在熊本抵达大理城下的时候,李信已经攻到了鄯善府附近,这是从大理俘虏那里得到的消息,而且是十天前的事。若一切顺利的话,李信此时已经拿下了鄯善府。

但熊本只凭他手上得到太后、两府所授予的权力,李信过来之后,要是不听命令,熊本完全可以将其推出帐外斩首的。就是看在韩冈的面子上,也能让李信栽个大跟头。

李信一过来,就必须听命行事,即使其攻下大理城,一样少不了他熊本的统帅之功。要与李信争功的,只会是赵隆。

赵隆抿了抿嘴,“末将这就去查探敌情,等明日便去城下扎营。营地一成,三日之内,不能拿下大理城,绑着大理君臣来拜见大帅,末将这个官就不做了!”

通过杀人立威,整顿了麾下兵马,熊本的行辕已经来到了太和城中。原本还有些许放纵的西南诸部,被熊本的手段吓得魂飞魄散,已经没人敢于去触动他的逆鳞。

城池的西段位于山峰之上,东段则一直绵延到湖畔,西有山峦、东有湖泊,不论是西段还是东段都不需要另外大费周章的去修城防,故而南诏前期的都城太和城,其实就只有一南一北两条城墙。

不过后来,南诏国迁都到了如今的大理城——当时的名字还是羊苴咩城,太和城便荒废了下来,尽管还有人居住,只是都城就在十几里外,这座城池也没有了太多用处,只有原本是城中内城的金刚城,还保留至今。

宋军的营垒,正是傍着金刚城而修起。一座座帐篷,如同雨后破土而出的蘑菇,陡然间出现在南诏旧都。而跟随在官军身后,是如同蝗虫一般涌进了洱海周边的土地的西南夷诸部。

此地距离大理城可谓是近在咫尺,但终究是还有十几里地的距离,并不是适合大军出击攻城的营地。想要攻打大理城,就必须向前驻扎营垒,距离大理城下不宜超过三里。

赵隆的话让熊本摇头,“子渐你何须如此?你身负一军之重,切不可莽撞行事。”

“让下面的儿郎查探,总不如自己亲眼看一看。”赵隆道,“何况末将不过一武夫,只凭勇力立足军中,若是连胆子都没了,日后也没法儿再领兵了。不过还请大帅放心,末将还没活够,可不想这么早去见阎王。”

赵隆在大理城外纵马飞驰,来自西域的良驹,让大理特产的滇马相形见绌。不过用来行走山路,运送货物,没有比滇马更合适了。现在赵隆麾下大军的日常耗用,粮草的部分,是靠了就地征收,以及得到的战利品,而其余军资正是靠着滇马的千里转运。

赵隆的身后,紧跟着他的一群护卫,人人高头大马。本来看见宋军逼近城下,不得不赶出来的大理骑兵,看见这边的壮盛军容,隔得很远便停下脚步,根本不敢再靠近一点。

赵隆仔仔细细的看着周围的环境,而他手下的亲卫之中,还有一人拿着纸笔,在一块木板上飞速的记录着。

滇池、洱海,两湖之滨,是难得水草丰美的上等,不论是放牧还是种田,都堪比江南。至少能养下十万户,这样的土地,屯下来,就没人愿意还了。

现在赵隆正在命人将这里的土地给绘制地图,以备后用。远离主力的他,肆无忌惮留在大理城的城下

看见自家的兵马梭巡不前,城头上鼓号响了几次,一次比一次更急促,他们终于动了。

大理人一动,赵隆也终于动了,引着敌人的骑兵在城下绕了几个圈子,赵隆忽然缓下马来,张弓搭箭,一箭便射落了追得最前的敌人。紧接着调转马身,领着麾下的战士一个反冲,标准的骑兵战术让赵隆收获了又一场胜利。

“该上大阵仗了。”

下面的儿郎去割首级,赵隆则看着那两座大营,等着营中开营出战。

大理城外有两座大营,驻兵在三万以上,城中还有一万兵马——这是精锐——若是在城中籍民为兵,还能更多一倍。这是大理最后的一点财产了。把持了宰相之位的高智升,他老家已经给官军抄了,正恨不得能报仇雪恨。

相对于城中的守军,城外的宋军,才不过五千之数。如此薄弱的兵力,正常的将领看过来,都会觉得有个两万人马,足以踏平城外的营地。赵隆预计,自己这里的兵马都用上,指挥上不出大问题,这一战可就赢定了。一战稍定,便可以建营,只不过这几天的夜里,得小心大理人夜袭。

但大理人始终没有动静,连夜袭都没有,直到第二天,大理城中终于有了动静,北门的城门缓缓地开启了。
