\section{第28章 大梁软红骤雨狂(九)}

【第一更】

西北大战即起,东府政事堂和西府枢密院为了能及时处理紧急军情,依故事都会留下一人值守。

今日东府中有参知政事冯京值守,李舜举奉口谕匆匆而来。中书的馆舍中,也有着让人睡觉的房间,李舜举本以为冯京会在内间小睡,却没想到这个时候,他还坐在灯下读书。

冯京起身拜礼之后,肃立在李舜举身前,听着天子的近侍把上谕传达。但接下来的情况,却不是冯京下拜接旨的惯常戏码,而是一动不动的站着,双眉向危险的角度上挑,眼中怒火隐隐燃起。

李舜举心中咯噔一下,知道事情不对了。

“不知韩冈究竟是何方人氏?”冯京慢吞吞的开口。缓慢的语调中,明显的参杂着大量愤怒的成份,“是哪一路的监司,还是缘边要郡的守臣,又或是有紧急军情需待他面禀天子?!”

这下轮到李舜举低头:“……是秦州缘边安抚司机宜。”

“秦州缘边安抚司的王韶不是来过了吗?天子难道没见他,又要见韩冈作甚?韩冈区区一个选人,非是因功进京,只是调任而已,连觐见都不够资格,何谈越次?究竟有何先例故事?!”冯京一个问题一个问题的砸向李舜举,不经意的却透露了他对韩冈的了解。

天无二日,民无二主,天子这大宋只有一个,每天能接见的人数也是有限,而想要见到天子的臣僚,却数不胜数。所以面圣的机会,是人人争抢。为了平息这样的纷争,便有一份顺序表排了下来,哪一天,该谁人入对,都有定数。但天下间总有突发之事,总会有人有实际需要,必须要尽快见到天子,所有就便有了越次入对这一说法。

不过大家都在排队,你想插队总得有个让人信服的说法。故而有资格打破次序的,要么是要有足够的身份地位——普通的监司官和州官还不够资格,必须是要郡、要路的守臣——另一个,就是身负紧急军情,备天子询问,而韩冈,两个都不是。

“……”李舜举沉默着,就算想说话也不敢开口,他只有传话的资格,公事上没他插嘴的份。

冯京居高临下的瞥了李舜举一眼,重重怒哼一声,显是怒气仍在,但口气已经和缓了下来,“如今依序等待面圣的尚有百人之多,皆是身荷军国重任。韩冈不过一偏鄙小臣,却能跃居众人之上,有乖常理,必会惹来议论,对韩冈本人也非是好事。他连京朝官都不是,仅仅是个选人。虽是小有才智,薄有微功,但越次觐见,奖誉过甚,岂是周全之道。你回去回复官家,朝堂之事,祖宗自有成法在,当依此而行,陛下谕旨,臣不敢奉领!”

冯京冷冷的拒绝了天子的谕旨,说是为了维护朝廷惯例,但更重要的是因为他对王安石的反感。冯京并不是新党一派,他升任参知政事,本就是赵顼秉持历代宋帝处理朝堂政局时,所惯用的‘异论相搅’手法的结果。

韩冈来自秦州王韶门下,很明显就是王安石一派。今次天子连夜点名要让他越次入对,冯京怎么想都是有人为了他在天子面前说了话。在冯京看来,韩冈这等新进逢迎之辈,如同见缝就钻的苍蝇,实在让人很难对他们升起好感。

冯京不想看到韩冈坏了朝堂上的规矩,没有理由为了一个选人,而改变维护朝廷秩序的成规。又非地方主帅,又非军情在身,这样的地位,实在让人看不到他越次入对,在天子面前能有什么作用。将天子的口谕丢在脑后,冯京决意维护朝廷秩序。

阻了天子无视朝规的口谕,冯京也有了点淡淡的自得,‘幸好是我,若是王禹玉【王珪】听了,肯定不敢有所推搪。’

冯京不把圣谕当回事,李舜举也不敢多话,这样的事多了去了。莫说口谕,就是天子亲笔写的手诏,被宰执、两制打回来的情况也是常见。

宰执们处理的决议,天子若是反对,便会被一通拒谏的指责给淹没。反倒是天子诏令,宰执们看不顺眼就可以不加理会,皇帝也没辙——几乎所有的诏书前面,题头都是‘门下’二字【注1】,其含义就是诏书必须经过门下省的审核,才拥有颁行天下的权威。这一条例从唐时,一直传到宋朝。如今中书、门下两省合一,并称中书门下,也就是政事堂。

故而冯京拒绝赵顼的口谕,他是理直气壮。

李舜举哪有说不的资格。本朝的内宦,大约是历朝历代以来最没有地位的。完全没有汉时十常侍把持内外朝政,更不似唐时的神策中尉,想换皇帝就换皇帝。朝中内外事,宰相无处不可干预。他们这些内侍,如果恶了宰执或是那里让言官看不顺眼,一封弹章上去,就算天子也不一定能保得住他们。

躬身应承下来,李舜举就要回去覆命。可冯京忽而又叫了一声,“等等……”

李舜举连忙转了回来,听候冯京发落。

“李舜举,你此时奉天子口谕过来,难道官家现在还没有就寝?”

李舜举一呆,心道冯京怎么说起这事,但还是得老实回答:“官家的确还没就寝。”

冯京双眼重又泛起怒意,厉声喝骂:“如今已是三更天后,官家却尚未安寝。你身为天子近侍,如何不加以劝诫!?”

李舜举低声回道:“官家在武英殿中,与宋殿帅商议军事,下官不敢打扰?”

“天子行事不当,难道你们就不能规劝?就看着官家中夜不眠?传到宫外,外人不知天子勤政,反倒以为官家耽于嬉乐……在这样下去,太皇太后和太后还能看得过去?是不是得换一个敢说话的跟着官家!”

冯京疾言厉色,李舜举吓得不敢抬头,连声请罪。

而拿着李舜举发作了一番,冯京瞪了一下眼,把他赶了出去。

李舜举如逃命一般急匆匆的走了,冯京犹有余怒,端起杯中冷茶一饮而尽,又重重一声把茶盏顿在了桌上,‘这王介甫,前日任用新进之辈,好歹还是进士出身的京朝官。现在韩冈不过一选人,素无重名,又无出身,竟然还让他越次入对。真是越来越过分了!’

……………………

韩冈并不知道自己倒霉的被误伤了,兀自安然入睡。

抵京后的第二天,是冬日里最受人欢迎的无风的晴天。当清晨的阳光透过窗棱,射入室中的时候,韩冈已经醒来。离开温暖而让人留恋的被褥,起床后,他匆匆梳洗了一番,吃过早饭,跟王韶说了几句,便起身前往中书等候发落。

韩冈是奉了中书的命令,从秦州赶到京城的。他现在已经知道,这是因为王安石是想把他调去鄜延帮着韩绛。但昨天跟王安石闹了一点不快,韩冈便想着要怎么拒绝这个让人麻烦的任务。

韩冈并非朝官,也不用赶在上朝时去宫中。他要去的中书门下,只有朝会之后,才会正式开始办公。慢悠悠的骑着马抵达宣德门前,偌大的广场是空空荡荡。拿着中书发到手上的文字,顺利的从右掖门进宫,韩冈直往中书省的馆阁行去。

通过中书省的一名公吏呈了名进去,跟一群同样等待宰执召见的官员们一起,韩冈在门厅处坐起了冷板凳。他在这些官员中显得很年轻,不少人都多看了他几眼。

等了许久,韩冈只见门厅中的官员越来越多,却就是不见有人被召进去。

“今天怎么这么慢的?”有人低声抱怨起来。

有人消息灵通:“政事堂里现在人手少,王相公今天又被留中,如今政事堂中只有冯大参一人。”

众人恍然。如今政事堂中,名义上有两名宰相,一名参政,但眼下韩绛在关西,王安石今天朝会后又被天子留在崇政殿中,只有冯京一人处置公务,当然快不起来。

“王禹玉不是已经任参政了吗?怎么他没来?”

“王禹玉前日才上了第一份辞表,至少还要七八天才能成事。”

王禹玉就是擅长以金玉为诗、人称至宝丹的王珪。他已经被内定为参知政事,现在正处于辞让名爵的阶段。等到辞个几次后,才可以正式的担任一国副相。

韩冈就一旁静静倾听这群官员闲极无聊的谈论着朝廷上的各种传闻,时间一点点的过去,不知过了多久,一名中年公吏走进门厅。

厅中顿时静了下来,几十双眼睛看着这名公吏。

“秦州韩冈。”公吏叫着韩冈的名字。

韩冈应声而起。

“请韩官人跟小人来。”公吏的声音平静得毫无起伏,转身便要往里去。

韩冈微微一愣,周围突然尖锐起来的视线仿佛如针一样刺着皮肤。正常情况下,普通官员都没有单独谒见宰执的资格,必须跟着七八个官员一起去拜谒。他刚刚得罪了王安石,现在却还单独叫进去,难道他还如此看重自己?

“只有我一人?”韩冈追上去问道。

中年公吏没有回答,只是重复道:“韩官人,请跟小人来。”

注1:无论唐宋,诏书的开头都不是‘奉天承运,皇帝诏曰’,而是‘门下’。若是看到唐宋时的历史剧,有哪人读诏书读出了‘奉天承运’,就可以笑一笑了。

