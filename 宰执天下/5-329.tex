\section{第32章 金城可在汉图中(八)}

韩冈的要求,威胜军知军自无不允之理。

说起来,就算韩冈不问他,他也没有干涉的权力。名义上军判官是知军的幕佐官,实际上军州属官的任命,什么时候需要经过知军知州的许可了?那是流内铨、审官东院和政事堂的权力。

拥有便宜行事之权的韩冈自然也有,唯独知军没有。

现在制置使给面子,知军又岂能给脸不要脸。

“公满你呢?”

陈丰又怎会不愿?机会有多难得,用脚趾想都知道了,谦虚了两句,便点头应下,“陈丰愿效犬马之劳。”

“很好。”韩冈点头,“我忝为本路制置使,不能在威胜军久留,明日肯定是要往太原去,公满你且收拾一下,明日一早随我北上。”

陈丰楞然,嘴张了开来,却合不上去。

正待恭喜陈丰受到重用、而韩冈得一得力幕僚的威胜军官吏们,也都愣住了。

韩冈眼神发冷,就连田腴都带着冷笑。

权力与义务相对应,制置使有便宜行事之权,要负担的义务自然也就远比普通的经略使更重。成为制置使幕中一员,相对于好处,自然也有沉甸甸的义务。官职和好处不是天上掉下来了,要拿命来换的。

这算是给新人的第一课。

“万万不可!”出人意料的,却是知军跳起来了,“北虏既然已破是石岭关,现在肯定是已经打到太原城下了。枢副身边只有十几人,遇上北虏怎么办?一旦枢副在去太原的半道上有何意外,连累到河东局势败坏,以致于河北、陕西的战局逆转,这个责任可不是枢副一身承担得起的!”

‘倒是会说话。’韩冈的心道。以大义相责,是下台的好台阶。如果自己没有一定要去太原的打算,就可以趁机下台了。

劝告人人会说。很明显的冒险活动,当然要劝诫。出自知军之口,却是出乎意料。韩冈本以为陈丰会开口。要么阻止,要么赞成,刚刚成为自己幕僚,陈丰理应要表现一番,可却是知军来出头。

韩冈看了陈丰几眼,却见他仍有些呆愣,暗暗地摇了摇头。金无足赤,人无完人,虽有些眼界见识,但心性胆略还是差得远。不过也不能求全责备,就当是千金买骨好了。

后世理所当然的自然地理常识,这个时代却是极为稀罕而无人教授的知识。不是千年之后,能对山川地理有一定认识的在官员中都是少数,同时还要再了解一点军事,就更加稀少了。陈丰能一眼看破辽军应有的动向,这份见识还是很难得的。

“枢副何不留在铜鞮县?等到京城的援军,正好可以北上逐寇?”通判也在附和。

“既然已经受命退贼,岂能藏身后方?这样如何激励军心民心?”韩冈笑了一笑,“进太原府城估计是来不及了,即便北虏会先攻榆次县,也会在太原城外放上一部兵马,但赶到太谷县应该没问题。”

太谷县在太原府中,除了州治阳曲,以及井陉道上的寿阳外,城防是最坚固的一座,驻军亦多。虽然从地理上,去祁县或是干脆到汾州坐镇,更能接近关西,引来西军支援,只是汾州已经出了太原,而祁县的城池并不坚固——韩冈虽然要维系河东的军心民心不堕,也不会拿自己的性命开玩笑。

韩冈又道:“先前往太谷县,便能先一步挡住辽军南下的道路。当然,榆次乃至寿阳也要加强防备,需要有人赶去榆次传话。”

知军立刻道:“传话之事事关重大,不可耽搁,下官这就安排人手。只要枢副写好公函,今晚便出城送去榆次县。”

知军行事干练,也会说话,今天的表现也不差。已经有好几人悄悄将视线投向方才刚刚被韩冈调走的陈丰。这位刚刚得志的制置使幕僚的表现方才可是有些逊色了。

陈丰亦有知觉,仓促的说道:“除了榆次之外,还有西面的汾州也需要提醒加强防备。那里由汾水直通关中,只有控制了榆次和汾州,北虏才敢放胆南下。”

陈丰的话一出口,韩冈的脸色一瞬间变了一下,挑选陈丰入幕府,是不是太过于草率了?千金市马骨,买的好歹也是千里马的骨头,不是劣等马的。

西军主力皆在兴灵,银夏两处,短时间内调不过来。且刚刚打完一仗,也必须要休整一段时间。这一点,辽人不会不知道。在河东所属兵力不足以两路并进、同时攻城掠地的情况下,自然会掂量明白榆次和汾州的轻重,分出先后来。想要在官军反应过来之前,提前占领榆次县,再多的兵马也会嫌少,除去压制太原城的一部分外,剩下的州县一时间就不会理会。

“至于太原,城高墙厚,辽人当只会封出城中守军的出路,不会强行攻打。”陈丰继续说道,“所以只要让城中知道有援兵将至,便可稳守住太原城。”

“……说的也是,汾州那边也的确需要提醒加强防备。”田腴倒是会维护新同僚的脸面,因为这关系到韩冈的脸面。

韩冈轻轻的摇了摇头,汾州也罢了,通知一下也无伤大雅,可有些错误是必须立刻指正,这比个人的面子更重要。

田腴看到了韩冈的反应,立刻改口,“不过太原就难说了……城中兵力不足的内情当会为北虏所探明,北虏多半会用重兵试图攻下太原。”

论起军事,虽然不是田腴的长项,但在韩冈身边久了,也有一定程度的认识,话说得条条入理。不过这番话一说,就更让人担心起太原的安危来。威胜军的官员们的神色又变得更加凝重起来,纷纷将望向韩冈,希望他能给出一个否定的答案。

“河东表里山河,非骑兵用武之地。一直以来,河北才是北虏南下的首选之地。但如今局势变易,河东这边险关接连被突破,而在河北却没有打破边界僵局,故此北虏必然会将主力转移到河东来……或者说,已经转移了,如此一来,北虏必然会有攻下太原的想法。”

辽人并没有同时在两个战略方向上展开兵力进行大战的能力。当他们在河东取得突破性的进展,河北那边自然就会就降格成牵制性的战场。太原城为河东一路重心,北接代州,南邻中原,东有井陉通河北,西有汾河入关中,乃是四通之地,辽人怎么可能放着不理?

“太原兵马本来就不多,又调了半数去河北,剩下的已经不敷使用。可调去河北的兵马又不能调回来,否则河北的战局也有糜烂的可能……不管怎么说郭仲通是不会同意从河北抽调兵力,若是通过朝廷公文往来,等争出一个结果来,差不多就要到夏天了。”

韩冈之前曾建议不要调回河东派往河北的兵马,这极有可能成为王.克臣推卸责任的借口。说这番话的时候,韩冈在心中也不禁感叹时局变化得太快,让人意想不到。

他的一番话近乎危言耸听,官员们更加不安起来。知军和通判肚子里都在咕哝,还是少说两句吧,下面的人胆都要给吓破了。

“不过河东局势表面上看虽是危在旦夕,但依然有着反败为胜的机会。所谓祸兮福之所倚,福兮祸之所伏。辽人越是深入河东,他们就越是危险。”韩冈的道理很浅显,在座的个人很容易就想明白了,只是他们依然竖着耳朵,听着韩冈的接下来的话,“河东与河北不同,千里平原上,骑兵能纵横驰突,想走就走,想留就留,但在河东,只要卡住几处险关要道,就是瓮中捉鳖,关门打狗。”

“关门打狗?”韩冈的话很有趣,这让原本严肃的威胜军官吏们有好几个都轻笑出声。

不过关键的还是韩冈表现出来的心态。自收到消息后,依然能保持不急不恼,气定神闲的表象,让在场的所有官吏为之心折之余,也放下了心来。

不过也有人腹诽,哪里会有这么好的机会?能守住太原就很了不得了,最多也不过是扩大一下收复的范围,夺回一部分失土。

河东的将兵法推行的并不彻底,韩冈之前只在缘边各军州和太原府中设立了七个将。当他准备整顿河东南部军州的禁军时,就被调回了京城。而他设立的七个将时所提拔任用的将校,正好就成了新任边将的针对目标。没被重用的被提拔,被重用的则打压下去,都不需要去费心调查,直接看一看每个人前后的官职就可以知道了。

时至今日,韩冈为了提高河东军的整体战斗力而团聚组建的七个将,可以说是给废掉了大半。已经排不上用场。西军一时间来不了、河北军也回不来,只剩京城和河东本地的兵马,又都是一时间派不上用场。

很多人都明白这一点,韩冈却笑道:“只有一心想要把入寇的北虏全歼,才有希望将失土夺回。要是仅仅抱着保全太原的心思去打,结果肯定是将代州、忻州一齐丢在北虏手里。”

田腴反应很快:“取乎其上,得乎其中?”

韩冈微笑点头。

他的意思正是《论语》之中的‘取乎其上,得乎其中;取乎其中,得乎其下;取乎其下,则无所得矣。’这一段所表达的道理。

说起来《论语》的确是本值得深思钻研的经典,只是被历代儒者的过度解读了,在后世才会弄得让人有逆反的心理。

“话说回来,若真的有机会全歼北虏,我也不会放过!”韩冈咧开嘴,白森森的牙齿衬得温温和和的笑容一下变得狰狞起来,“既然来了,那就不要走了!”
