\section{第26章 任官古渡西(五)}

凌庄失望了。

他送给韩冈两位幕僚的赠礼,没有起到一点作用。那个姓魏的查验账簿时,还是一点情面都不讲,而姓方的出去找人做冬衣,可以笑眯眯的跟自己的人打招呼,却没有帮着自己说好话的意思。

随着时间一天两天的过去,凌知县不敢再拖了。不及时交割官印,开封府中必然会有人下来查问,到时候韩冈岂为自己隐瞒,那可是会有麻烦缠身。

对身外之物,不能再纠结多久。凌庄咬着牙将亏空补上,重新将帐册整理好,让魏平真和韩冈先后验过,画押签字。最后交割了印信,走过了万民伞、脱靴礼这一干程序之后,带着一大家子车马,一路往京城去了。

离开的时候,凌庄还是得陪着笑脸,韩冈的地位和未来都是他不敢得罪的。更别说他要去京城守阙,免不了要经过中书和审官东院,韩冈这位宰相之婿虽不能帮自己挑个肥差,但要坏事却很容易,歪歪嘴就可以。

随着白马县的一众父老,走过场的送走了前任知县。看着凌庄垂头丧气的离开,诸立冷笑着转回来。这就是官员和胥吏的区别。

官员离任都少不了这一番苦头,后任不可能接下前任的烂摊子,让自己陷入困境,两三年的时间,要想将帐册和库存做得严丝合缝,诸立可没见过几任知县有着能耐。

而胥吏不同。他们在库房中作手脚,只要串通好,比起官员来要容易许多,而且更为稳妥。有着几十年的经验,诸立所造出来账本、库存,都能一一对上,不会有半点差池。而且许多时候,在白马县这样的津梁要冲,诸立在外面收受的好处,并不比入帐的正税要少,没必要去贪库中的钱。

在自家中聚起了县衙内的诸多吏员,诸立提声道:“这一位的性格,想必各位都明白了吧?”

胡老二也是赫赫冷笑着:“韩正言眼里还真是揉不得沙子啊……那点小错处,州里来人,哪次不都是一眼带过?竟然一点情面都不讲。要不是看着脸不像,还以为包侍制来白马做知县了。”

“账本上的那几个错处,如果有人有心去根究,还是能查得出来。到时候,他免不了会因此而受罚。”

“所以说他应该是很在乎名声,一点会给人抓把柄的地方都不留。”

“这样不是最好?韩正言的名声,我们也可以帮他在乎着。”

诸立摇头:“别说浑话了,看看他接下来做什么。是等着磨勘过去,还是想要有所动作。确定了之后,我们就好做出应对了。”

白马县的胥吏聚在一处说话,韩冈不可能知情。可他也不会在乎那些胥吏在讨论什么,更没兴趣知道。

他可不再是旧年要服衙前役的穷措大了,如果是想讨论着如何对付自己,那就是老鼠给猫戴铃铛。不过想来白马县的胥吏们也不会那般不智,就算换作是陈举,面对着身为朝官和宰相之婿的知县,必然是低声下气的好生服侍着,除非到了万不得已,否则绝不会呲一呲牙。

他要想解决县中的某个胥吏,就算那名胥吏的地位跟当年的陈举差不多,也不会花费他太大的气力。只要将自己的心意透露出去,连借口都不用,多少人会抢上来要来帮忙。

当然,新官初上任,不熟悉情况,随便放火可是会烧着自己。韩冈也不会随随便便找个看不顺眼的来杀鸡给猴儿看。

先要熟悉白马县。从风土人情,到地理历史,都得心中有数。而且还有田土、人口、税收等重要数据需要去了解。新法的推行情况,那也是不能少。而且最为重要的,还是为了明年可能的灾情做准备。

到了白马已经有七天,头顶上依然是无云的大晴天。

白马县靠着黄河边上,韩冈在衙门中坐了两日,今天上午处理完一些琐事,就带着三名幕僚,随从,以及一队弓手,出城往着黄河而去。

远远的就听到了水声,高达数丈的黄河大堤如同一条长龙,从西横贯,一直往东而去。立于大堤之下,仰头上望,高耸的堤坝让人惊叹不已。不过如今秋冬水枯,又是旱了几个月,站在几丈高的黄河大堤上,离着黄河河水,竟然还有上百步的距离,而黄河对岸的大坝,更在几里外。

韩冈看了一阵风景,就从大堤走下去一点,众人连忙跟上。只看着韩冈突然向后招来一名随从,吩咐了一句,那个随从就掏出匕首,就在河滩上掘起土来。

一团泥土托在韩冈随从的手上,而混在土中,有好几个长条状的东西。

“这是什么?”游醇不解的问着。

方兴难得的收起笑容,板着脸:“蝗虫。”

“蝗虫?!”游醇惊道。

魏平真一指脚下的这一片河滩,干涸开裂的土地上,密密麻麻的全是小洞,“这里全都是蝗虫卵。”

游醇的脸色转瞬就白了下去,他不似方兴和魏平真见多识广,过去都是钻在诗赋经籍中,根本不知道蝗虫卵是个什么模样。在福建,也难以见到遮天蔽日的蝗虫。今日只是看见着河滩上数都数不清的小洞,一个洞就是一枚卵,“这该有多少蝗虫?!”

魏平真阴沉着脸:“这里算是少的,河北只会更多。今年河北可是连续三次蝗灾,不可能没留下种来。”

韩冈拿手拨了拨土,将一条虫卵捏在手中,“这一个卵鞘中能孵出几十只蝗虫,单是我们周围的这一小片河滩,明年开春数以百万计了。而白马县这一段河滩,怕是有亿万了。”

“一个能孵出几十只来?!”这下子,不仅是游醇,连魏、方二人,脸色都发白了。他们可没机会看过《昆虫记》,当然也不会了解蝗虫的一生。

韩冈将虫卵丢开,回头望着左右:“蝗虫畏水喜干,如果此处淹水,那就都孵不出来。”

方兴抬头望着无所阻拦的太阳,咬着牙:“这鬼天,哪来的水?!”

“也只能盼着今年冬天多下雨雪,否则明天开春后,河北、京畿都要出大乱子了。”韩冈声音沉沉,夹杂在滚滚的黄河水中,仿佛是丧钟声中传出来的悼词。

就在韩冈等人在黄河滩上,为明年而忧心忡忡的时候,白马县的胥吏们则是在阴暗之处,有着一番盘算。

韩冈接任的这三天来,除了今日午后出门去黄河边,其他几天,都是再看旧档。让人打开架阁库,搬了不少档案回去。五等丁产簿、田籍等簿册,都先后察看了一遍。从他的这番行动中,白马县的胥吏们,也终于知道这位从七品的右正言兼集贤校理,并不是来此熬资历的,而是想要有所作为。

如此勤勉的知县,胥吏们并不是没有碰上过。该怎么应对,心中都有数。不过诸立却是有另外一份心思在,韩冈怎么说都是宰相的女婿,这条大腿到了面前怎么能不抱?

不过大腿也不是随便能抱的,总的有一番方略。“先得放出风去,如今知县事的韩正言,是天子、宰相都看重的少年才俊,连翰林学士都比不了,蕃人看着他都要低三下四。能明断是非,清正廉洁,日后少不得也是个阎罗包老。让人把争产的案子都拿过来,请韩青天仔细的去审!”

诸立一向相信自己看人的眼光。既然看透了韩冈的为人,那么就要顺势而为,以便让自己从中渔利。白马县是紧邻开封的要地,他能在安安稳稳的立足生根,靠得就是进退自如、能软能硬的手段,绝不是好勇斗狠。

“争产的案子,从来都是最麻烦的官司。传唤人证、打听消息,翻检旧档,都有使唤到我们的地方。”诸立教训着两个弟弟,“好好侍候着他,帮韩正言断上几个大案出来,他有了光彩,我们这番辛苦当然也会有回报!”

“原来如此,我们知道,我们知道。”诸霖和他同样是赵家女婿的三弟连连点头,一副心领神会的模样。

“当然喽,我们也得先让韩正言明白,没有我们,他什么事都做不好。”诸立脸上的微笑,在诸家老二和老三的眼中显得高深莫测,“这样才能体现我们的能耐……你们说是不是?!”

诸立的弟弟们,也只有点头的份,满口的夸赞:“大哥真是好算计!”

韩冈一行人,从黄河边回来,已经是傍晚。但却有一份诉状在县衙中等着他。

这是一桩争祖坟的案子。原告、被告都姓何,但不是同族。他们从三十年前就开始争夺一座坟茔,都说是他们的先祖。每一任知县到任,他们必定要来的争上一争。

“争祖坟。”韩冈看了两眼,就问着值守的胥吏胡老二,“祭田有多少?”

没提防韩冈一下问道关键的地方,胡老二老老实实的回道:“……两顷又十五亩。”

为了两百一十五亩地,竟是打了三十年的老官司。

