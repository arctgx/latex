\section{第34章 为慕升平拟休兵(20)}

不过李诫和黄怀信眼下并不在这里,结束了这一条简易轨道的修筑工作后,他们各自带着手下的匠人和民夫,转移到了南方,分别主持忻州和太原两个盆地中的简易轨道的修筑工作

“此役若能远逐北寇,李、黄二人功不可没。”虽说黄怀信已经转迁武职,官位远在李诫之上,但在章楶这样的文臣口中,永远都是把李诫放在前面,“战后将此铁路留与代州,可就又是一条方城轨道了。”

“不,忻代铁路还差得远。”韩冈笑着摇头,他为了自己方便,在铁轨替代了木轨之后,便将这一条新的轨道称为忻代铁路中段,南段是忻州到忻口寨,而北段自然是往代州去了,“只是比方城轨道一开始的时候要强些。那时候当真是两眼一摸黑,从我开始,哪一个不是摸着石头过河?哪里像现在,几年下来,方城山那边上上下下都是老手了。”

已经更换上了铁轨的方城轨道,可以持续的运输能力远在这条忻代铁路中段之上。这不仅仅是质量和结构的问题,更有管理维护人员的技术经验等软件上的差距。

“不过这条铁路,”他又说道,“修筑和运营都是实行军法,且营中军律森严。少了不少无谓的损耗上,这一点是要胜过方城轨道的。”

长距离轨道的运营,技术有难点,但运营上的难度更大。就是只有六七十里的方城山那边,磨合到现在也还不能说是已经完善妥当了,只能说说得过去。

刚建成时的严格管理,只持续了一年不到。在初任的官员一个个因功调离之后,方城轨道便因为管理不善,造成了不小的内耗,幸而之后几年的时间,增速惊人的运输量掩盖了这个问题,同时其内部终于是找到了一个平衡——私人和公家共同利益所在的平衡点——所以才能有一年百万石的粮纲,以及倍于此数的商货由此进入京畿。同时还有相当数量的北方特产经由此处离去。

如果不是军用轨道,而是新建成的普通民用轨道,肯定要相当长时间的磨合——除非能像当初一样有韩冈这般威望的重臣盯着,又能有对未来的许诺来稳住中低层的管理者

但军用轨道有个好处,就是可以不计伤亡,不用在乎损失,同时也不用担心在这么短的时间内还有人能从中渔利。

车辆准备好后只管往前送,省去了民用轨道许多事务上的烦扰,一刻钟不到便能发出一列,截止到卯正,忻口寨中少了三十多个指挥,而前线则多了一万四千余人。虽说中间有几起事故,也造成了一定的损失和伤亡,可是在运送兵员这个大目标之前,一切可以忽略不计——另一方面其伤亡数字在运送人数的中的比例,其实也可以忽略不计。

章楶一声长叹:“万军夜行百里而兵将不损,就算是再jīng锐的骑兵也难得见上一回。有轨道输送兵员,自此以后,辽贼能猖狂的手段可就又少了一半。”

回望旌旗如林,骁勇如海,每时每刻都能看见己方的兵力在增长,巢车上的每一个人不免心旌动摇。

数万大军严阵以待,但又有谁知道,他们刚刚经过了百里路途的跋涉?若是同样的速度放在河北,三rì三夜,就能从白马津抵达宋辽边境。不论是作为援兵,还是北上突袭的奇兵,都能让辽军猝不及防,乃至一败涂地。

当大宋给他们引以为傲的步军安上了铁质的轮轨,辽人依靠战马所维持的战略优势,已经越来越小,甚至在有些情况下都能被压制和抵消。

其纵横离合、惊扰中原的rì子,真的已经不长了。

……………………

相对于韩冈能轻松地跟着他的幕僚们展望未来,萧十三和他的下属们甚至连眼前的事都弄不明白。

就在他们的眼前,宋军的兵力不断地增多几乎是每过一刻钟半刻钟,就有一面指挥旗自寨中移出,原本收缩起来的阵形,又开始向两端扩张。

宋军那边近乎于变戏法一般的从口袋里变出一支支队伍,情形诡异得难以想象。那座看起来规模不小却藏不下这么多人马的营地,像是藏了一口泉眼,能咕嘟嘟的连人带马从泉眼中喷出来。

“肯定是假的!肯定是伪装!”一名将领大叫起来,“汉人多狡诈,张起旗帜,就想吓退我们!枢密,让我带儿郎冲一下,到了近前皮就能剥掉了!”

“……不是伪装!”萧十三咬牙切齿,且很是艰难的吐出了四个字。

纵然从千里镜中不可能将一名名士兵给区分出来,但士兵身上的甲胄反光是冒充不了的,列队行进时的队形和速度也伪装不了的。

近在眼前的事实,让萧十三无法自欺欺人。只是他也没有办法解释,为什么宋军的兵力还能以这么快的速度增加。

又不是蝗虫,孵化时能从地里成片成片的钻出来!

在飞船上的千里镜中,那一队队离开营垒,汇入大阵之中的新至宋军,就像是一块块不断在背上累加的石块,压得萧十三以下一众辽将喘不过气来。

抬手扯了下领口,呼吸感到稍稍顺畅了一点,萧十三放下心头重压的长舒了一口气之后,终于放弃了猜测韩冈究竟是怎么将这几万大军变到前线来,同时还在继续变出来的手段。

不论是昼伏夜出的逐rì潜行,直至全军潜入对面的大营;还是先将大军移动到后方近处的营垒,然后一夜奔行数十里,或是其他可能——这些猜测各有各的道理,可在细节上,却又都有着许多说不通的地方——但其中任何一种都代表了韩冈对军队有着如臂使指的控制。对于这一行动严密有效的消息封锁,也同样证明了他麾下大军的战斗能力。

至于之前所猜测的一夜之间从忻口寨突进百里,冷静下来就知道不可能,萧十三也不会去吓自己。

现在要考虑的是如何应对。

宋军列阵在前

战?还是不战?

号角声在沉思中响起,略显高扬的变调,表明又有一支援军从代州赶来了。

这让萧十三又松了一口气,虽然说韩冈已经将他手中的兵力送到了前线,但代州和大小王庄的区区四十里距离,以及人人得乘的战马,使得援军赶来的速度绝不会让韩冈有机会利用兵力上的短暂优势来改变战局。

随着时间的过去,宋军或许会越来越多,但自己手中的兵力也会同时增长,绝不会输给韩冈。

心中有了底气,再看对面的军阵,原本因为紧张而被忽视和忘却的问题,这时一个接着一个从脑中浮了起来。

为什么夜行进入前线的宋军今rì不在rì出前出动,而是在天亮后才出营摆开阵势?为什么宋军在骑兵已经挡下了前几波援军后没有继续攻打大小王庄?为什么宋军现在收缩得这么紧?到底是在提防谁?

这些问题可以有不同的回答,每一个都能说得通。但如果要在其中选择一条能适合所有情况的,那就不多了,甚至可以归纳成一条:

韩冈不想让他麾下大军出战。

其原因或许是为了避免无谓的伤亡。或许是不相信自己的部属。或许是不想损失太大以至于回京后受到责难,或许是想要将所有的辽军都吸引过来,然后通过一场决战来一劳永逸。

不同的理由,只有一个结果——韩冈选择了亮出拳头、虚张声势。以此来吸引所有人的目光,让他们甚至于忽视了该注意的地方。

韩冈在眼前摆出的阵势终究还是吓唬人的。否则一开始就该将手上的兵力全体出动,将大小王庄一举攻破。既然没有,那么他的心思肯定还是放在五台山。

萧十三自觉是看破了宋军的伎俩,双眉飞扬起来,“守住大小王庄。以不变应万变。看看宋人有什么花样!”

宋军的确没有什么花样,他们继续让一支支援军赶来前线,但速度比起辽军一方也没快到哪里。双方就这么对峙着,直至暮sè将临,张孝杰仅仅带了几十名亲卫从代州赶来,

大辽的南院宰相好像是从水里土里打过滚一般,不仅仅是灰头土脸,而是满身的污泥,一张脸都是花的。张孝杰一贯注重仪态容姿,如此的狼狈多少年也难得见到一次。

“落了马?”萧十三惊讶的问道。张孝杰虽是汉人,可马术也非等闲,寻常岂会落马?再看张孝杰骑乘的马匹,并不是他惯常所乘的那匹枣红sè的赤熘,队伍中也不见赤熘的踪影,“把赤熘都累倒了?!”

“可知宋人是怎么运兵的?!”张孝杰都没功夫理会萧十三的关心,等气息稍稍顺过来,他直起腰,指着远处已经开始收兵回营的宋军,“可知道宋人的那几万人马是怎么过来的吗?!”

“怎么来的?”萧十三笑道,“难道不是宋人暗施诡计?或者说是昼伏夜出的潜行到这里?还是这里的士兵早已在南面稍远的地方休整了数rì,等待出动的良机。”

“那是轨道!”张孝杰大叫了起来,“没有什么诡计,没有什么潜行,也没有什么半道藏兵,只是轨道。能昼夜运输粮草兵员的轨道!能一月运送六十万石粮草的轨道!能一夜走出一百里的轨道!”

张孝杰的声音在萧十三又开始泛白发青的脸sè中低沉了下来,带着几许颤抖,“那是韩冈的轨道!”
