\section{第34章 道近途远治乱根(上)}

大辽的冬天是残酷的,同时也是荒凉的。

在耶律乙辛手上的千里镜镜头中,只有大块大块的白色,以及零零星星的灰黑,看不到半点活动的生灵。

但不论如何残酷,如何荒凉,这都是他的领地。

天地寥廓无极,大辽的国土也一眼望不到边际。

儿郎们在此游猎,附庸们则纷纷弯着腰走进属于他耶律乙辛的御帐。

辛劳一生,农夫运气好能攒下百十亩地,牧民最多有个几百头羊,做工匠的得到一间工坊,做官人做贵人,大概也就能得到一个头下军州,以及皇帝面前的一点情分。

如自己一般,以一生时间,得到一个幅员万里的国家,还有什么样的人生更有成就感?

耶律乙辛想不出来,也不觉得会有。

即使这片土地远不及南方的邻居富庶。

耶律乙辛很清楚,如果是在南方,他这辈子都不可能有如今的成就。就像南朝的那位年轻的宰相,纵然有天纵之资,又深得军民之心,可他这辈子都别想弑君篡位,等小皇帝亲政之后,想有个好下场都难。

可惜了那样的才干。当年耶律乙辛还听说,南面的那位宰相还打算生聚十年,等自己死后,辽国内乱,然后趁机北上。现在看看,期以十年的究竟是哪一边?

耶律乙辛这两年对南方的担忧越来越少,宋国主弱臣强,这内乱的局面本就是明摆着,耶律乙辛当年就经历过这样的局面,双方必须有一个倒下,才会有一个安定的结局。

若是日后南朝的那位宰相输了,是不是在这边给他留一片地?送他十个八个头下军州都是值得的——只要他不嫌这边太荒凉。

耶律乙辛知道对方会怎么想,对久居东京的南朝人来说,即使是最繁华的析津府都是荒凉的,更不用说鸭子河畔或是临潢府旁的山林和草原。

没有亲眼看过,只是听人描述,耶律乙辛实在很难想象,连同宫城和皇城在内,有着五重城墙,最外围的一重城墙甚至有上百里长的巨城,究竟是什么的一副模样。但有一点可以确定,宋国的东京城,绝不是大辽国中的任何一座城市可以媲美的。

根本就不用指望南朝人会像自己一样,欣赏这独属于自己的一望无际和渺无人烟。

不过太荒凉也非是好事,至少对围猎不是个好消息。

“这里还能捕到猎物吗?”

耶律乙辛放下手中的千里镜,侧过身,问着身后的完颜部之主。

“回避下的话,自入秋后,小人就把这一片山林给封起来了,不让人进去狩猎采药。养了半年,要獐子有獐子,要野猪有野猪。就是虎熊,也是有的。”

完颜劾里钵毕恭毕敬的回答着。

完颜部之主,在白山黑水之间,人人皆敬称太师而不名的完颜劾里钵,站在大辽皇帝的面前,眉目间所流露出来的谦卑和恭顺,是他的部众在背后完全想象不到的。

但即使他们看见了,也不会觉得哪里的有问题。

大辽的皇帝,受到怎样的尊重都不为过。尤其是耶律乙辛这样对女直人颇多照顾和信任的皇帝,在女直各部中,更是受到普遍的崇敬。

耶律乙辛对完颜劾里钵道:“春夏秋冬四时捺钵,也就你们这边最让人省心。换做其他几处,总是闹得让人待不住。”

“陛下,是不是捺钵的地方不太好?小人听人说过,靠海太近,被风吹得多了,容易骨头疼。”

完颜劾里钵话说得鲁直,却透着浓浓的关心。

“平州是个好地方,冬天歇着其实不差,就是南北两边吵得慌。”

“上次阿骨打回来也说闹得厉害,他从早上一直守到夜里,也不知道怎么有那么多话说的。他自己都累得不行,就担心陛下会不会累到。”

契丹乃以游牧为生,立国之后,亦未改游牧之法,辽国国主每年皆按季巡游四方,四时行在之所号为捺钵。

在过去,四时捺钵的位置,大体固定,延续了百多年。但耶律乙辛自登基之后,很快便改动了捺钵的位置,以适应国内的变化。

夏捺钵,在鸳鸯泺,维持对宋人的压力;秋捺钵在临潢府外;冬捺钵,放在了靠海的平州,尽可能的靠近他的财税中枢。

只有春捺钵的位置保持不变,为了更好的控制住女直,在鸭子河畔举行的头鱼宴,耶律乙辛怎么也不可能放弃。

其中夏冬两季的捺钵,是辽国南北两部,也就是契丹官和汉官两个不同官僚体系的重臣,聚在一起共同议定国家大政的日子。

每到这时候,耶律乙辛都要为调解两边的口角官司头疼很久,实在是吵得慌。到那个时候,他就开始庆幸,幸好宋人将岁币给停了,不然惦记着这些好处,吵得时候会更久。

等到转到了鸭子河这边,情况就好了许多,女直人吵虽吵,但不闹腾,说话也让人省心省力。

像完颜劾里钵这样粗莽之辈,连阿谀奉承的话都说不好,跟这样的人打交道,不知要省了多少心神。

“累是累,可也不能让别人累去。”耶律乙辛捏着千里镜,笑着说。

劾里钵一阵点头,“是,是,陛下说得当然对。”

“好了,先回去吧。”耶律乙辛转身往回走。

下了这片山坡,再远一点就是捺钵所在。那里一改北方远处山林的荒凉,显得喧闹无比。中央处的金色的御帐,在阳光下熠熠生辉,反射着绚烂的光芒。

出来走了一走,耶律乙辛的心情很好,很大方的对完颜劾里钵道,“这围猎的准备,劾里钵你办得好。想要什么赏赐,只管直说。”

“别的不敢向陛下讨要,原本冬天族中粮食有些不足,杀了些老马也撑过来了,现在过了头鱼宴,可以捕鱼了,多撒几次网,也就能填饱肚子了,倒是没什么可担心的。就是盼着陛下多派几个医官来,自从开始种痘之后,族中的小崽子越来越多,阿骨打说是排老二,其实把死了的算上,得排第五了,现在能有这么多小崽子,都是陛下的恩德。只是这么一来,小人部众老弱太多,得病的不少,就盼着有医官能给诊治一下。”

完颜部需要更多的医生,也需要更多的药材。但有着一整座长白山的特产,需要什么样的药材,完颜部都能用自家的特产交换过来。唯一急缺的,就是医生了。

耶律乙辛皱着眉头,回想着之前的记忆,“朕好像听谁说了,你前两个月,把族中的大巫杀了五六个。是不是因为这个原因才缺医官的?”

完颜劾里钵立刻气哼哼的说道,“那些个萨满,平时就会跳跳唱唱,摆弄点树皮草根,也不见救了几个人。说什么种痘的医官犯了忌,要斋戒敬神半个月。小人嫌他冒犯了陛下派来的医官,就砍了。没有陛下每年派来的医官,族里不知要死多少小崽子。”

即使是汉人,也有所谓的祝由科,以巫术来医人。大辽国中的其他部族,更是不缺能沟通鬼神的巫人。这些巫人,在过去,都兼职着医疗上的工作。直到宋国的先进医学传来,种痘法和卫生制度的效果在辽国国中得到有效验证,才让巫术退出了医学界。

不过如女直这般偏远的部族,巫师还是占据着医师的职位,同时对族中事务还有着巨大的发言权。只是劾里钵的弟弟和儿子与大辽宫廷联系紧密,对装神弄鬼的把戏不再畏惧和相信,才会这般干脆的砍了五六个大巫的是脑袋。

“也亏你能下得了手。好歹还有些用处。”

即使仿效宋人设立医学、医院和钦天监,耶律乙辛也没说把巫人都给砍了。只要不造谣惑众,留着他们也能起到一点拾遗补缺的作用。大凡巫人,多半有一两个秘方,说不准什么时候就能用得到。

完颜劾里钵回答得极为干脆,“冒犯了陛下的人,就该死。”

“好了,好了。朕知道你是一片忠心了。”耶律乙辛笑得很是开心,“要多少医官?”

完颜劾里钵犹豫了一下后说道,“……以小人的心意,当然是越多越好。但愿意来北地的医官当真不多,小人也不敢勉强,能有六七人就心满意足了。”

“六七个?抵那些大巫的数?”耶律乙辛又一次笑了,笑得开怀,“朕给你十人,还有一些南朝来的药材。”

“多谢陛下,多谢陛下。”

完颜劾里钵大喜过望,立刻拜倒谢恩。

“当然,规矩你知道的。”耶律乙辛提醒道。

“族中的尸体都要提供给医官们解剖后再下葬。小人明白,陛下放心。”

“你记得就好。”耶律乙辛点头,只有这样,才能让国中医官们的医术赶上南方的同行。至少是在外科上。

汉人重尸骸,契丹人、女直人也同样看重,但在尸骸和活人之间做选择,被汉人视为蛮夷的契丹、女直,都没有那么多的忌讳。

在山坡下上马,一路回到御帐。

完颜劾里钵再拜而退,等在帐前的张孝杰紧皱着眉头,盯着完颜部之长退了出去。

不待帐帘垂落,张孝杰便立刻回头上前,“陛下。”

耶律乙辛将笑容收敛,从闲散悠然的老者,变成了手握万里疆土的君王,“怎么,又要跟我说完颜部的势力太大了?”
