\section{第八章 朔吹号寒欲争锋(六)}

结束了与韩绛、张璪的会面,韩冈返回他的公厅。

在看见多得让人绝望的公文前,被触动心事的他,还在想着日后攻辽的主帅安排。

郭逵已经做到了武臣的最高位置,不可能出典大军,出镇于外的机会。

文臣中,章敦和自己都有资格,但几乎都已是位极人臣,实际领军的可能也很小了。

如果是防守那还好说。若是举大军,北上收复失地的主帅位置,不论是谁,都有各种各样的顾忌。倒是架着天子、太后,来个亲征是一条可行的道路。

这样的遐想只占用了韩冈几步路的时间,当他站在公厅门口,所有多余的心思全都飞掉了。

韩冈用了整整一个时辰来处理公务,让堆在桌案上的小山消去了几堆。韩冈之前让人拿出来的傅尧俞的履历,终于重见天日。

刚才与韩绛、张璪议论过堂除的人事安排,边州是不方便了,近期在京西只有一个缺额——唐州知州。

韩冈觉得,他已经为傅尧俞找到了适合他的位置。

先将傅尧俞往京城附近调,做上两三个月的知州,等朝中有了适合的空缺,便可以调他回京。这是十分常见的升迁和回京的办法。

想必这样的安排,也能让西京中的几位满意,也不会让王安石觉得有太多的问题。

让人将傅尧俞的履历送回架阁库,韩冈喝着热茶,稍事休息。

闲暇下来,韩冈就想起王安石今天并没有上朝问政。

王安石作为平章军国重事,是六日一朝,不来是正常的。但韩冈还以为他会过来盯着。

中书门下都是新党,可想也知道他们奈何不了自己,不过王安石看起来倒是大方。还是说,经过了这么多事后,自己对自家的岳父已经有了偏见?

平章军国重事不常置,其权力范围很模糊,并不像宰相、枢密使一样,有着明确的制度规定。依照字面意义上的理解,是处置军国重事。至于何为军国重事,那就要太后来决定。

只是王安石作为新党领袖,在新党成员遍及朝堂的每一个角落的时候,只要是他想干涉的事务,都可以轻易的做到——同样的位置,权柄是大是小,其实还是要看人才是。

不过到了快黄昏的时候,韩冈终于知道王安石为什么今天没赶着过来了。

一名韩冈很熟悉的王家家丁被人领了进来,“平章命小人传口信与参政,这几日若有空,请参政过府一叙……还有老夫人也说,好些天没见几位哥儿姐儿,请一起过来。”

这是摊牌吗

韩冈并不意外。

以王安石的性格,不会尸位素餐。他既然是新党之首,当然就保护新党的利益。六十出头的年纪,正是政治家大展宏图之时,王安石因为心力交瘁,倦怠于政务,但新党一旦遇上敌人,他还是会披挂上阵。

这一番去岳父家,韩冈认为王安石至少会逼着自己确定想要什么,就像今天韩绛、张璪所做的那样——虽然说朝堂权力不可能分割清楚,但划定大致的势力范围,却是必不可少的。

不过岳母吴氏的话,倒是冲淡了这一回鸿门宴的气氛。

韩冈没有耽搁时间,当即派人回去通知王旖,让她带着儿女先去平章府。而他等到放衙,也直接去见王安石。

坐在王安石的书房中,韩冈与书房的主人聊着天。

“玉昆,今天第一次以参知政事上朝,可有何想法?”

“之前很长时间,小婿上朝后都是站在西班中看着对面的同僚。今日却终于可以站在东面看人了。”

王安石摇头苦笑,他这个女婿有时候实在让人无奈。

他将话挑明了问:“玉昆做了参政,在治政上可有什么想做的?”

韩冈想了想,“政事之先,理财为急。”

王安石当年对赵顼说的话,现在韩冈还给王安石。

“这是太后今日询问时,小婿的回答。”韩冈笑着道。

“哦……不知玉昆打算如何理财?”

“敢问岳父,今年的军费几何?”

王安石道:“最多只有之前的八成。”

铁甲的制造量,已经超过了禁军的数量。斩马刀、腰刀、骨朵、枪尖、箭簇之类的钢铁军器制品,更是数以百万计。

现如今在军器上,除非进行全军换装,否则短时间内,不再需要大规模的制造,仅仅是就足够了。将刀枪剑戟,弓弩、甲胄、霹雳炮、床子弩、战船、战车等所有军器计算在内,每年装备更换的费用都不会超过三百万贯。

而没有了战争的消耗,军队的维持费用其实与过去比起来,也不算很多。

“没错。”韩冈点头,“因为终于天下太平了。西贼覆灭,王师进抵葱岭。北虏也转头向东,却攻高丽、日本了。现在连西军也要削减兵数。”

“玉昆可是在担心?”

“当然担心。”韩冈立刻道,“澶渊之盟后,三十年太平时光,使得举国上下找不到一位可用之将,一支堪战之师,任由西贼肆虐。这样的局面,不能再重复。”

“但西军也不是就此马放南山。”王安石道。

“的确,并不是解散了事。而且百姓也能得到好处。”

经过了辽人入寇之后,河东军损失惨重,需要大量生力军来补充。所以西军中至少有八十个指挥要转调河东。剩下的也是汰弱留强,让老弱屯垦,废去的只是山中的无数寨堡。横山深处,消耗了大宋的太多资源,没了这一笔开支,关西诸州的百姓,能够轻松很多。

“玉昆,你尽说军事,可是要做枢密使?”

韩冈可不想做。

东府的权力比起西府要大得多,韩冈就算做了枢密使,手中的人事权和财权,也比不上参知政事。

东西两府并称,不过是自古以来文武并称。更重要的是自开国以来,外敌对国家的威胁太大。自仁宗之后,军事开支常年保持国家财政支出的近八成,而军事及外交在政治上的地位,这让同时握有军政及对辽外交之权的枢密院,在朝堂上便有着与政事堂相当的份量。

如果军费大幅下滑,军事在国家政治上的地位下降,那么枢密院也很难保证现在的地位。

“军事亦是国事,不是枢密使,也可以议论。不过西军的调整,小婿也参议过,暂时没有更多的意见。但如此大规模的削减军费,节省下来的开支,并不是存起来就行了。小婿的本意,并非增加朝廷的收入,而是让朝廷开支调整得更为合理,用到该用的地方。”

“哪里?”

“很多。比如小婿正准备提议加大民生投入,各州各县都要设立医院、药局,并设局让鳏寡孤独得以安养。”

“玉昆,这可不容易。。”

“先做起来了。不做永远成功不了。老有所终,壮有所用,幼有所长,矜、寡、孤、独、废疾者皆有所养。这不是天上掉下来的,是必须有人先开始去做的。”

“若能如此,的确可以追及三代了。”王安石没有太激动,正与韩冈的话相似,在他看来,事情必须是做出来,而不是说出来,“不过……就这些?”

“同时小婿还打算给官吏加俸,相应的,也会加强对犯法官吏的惩处,边疆缺人,犯官总是拖家带口,正好用来充实边疆。”

王安石摇了摇头,后面说的惩处是附带,一开始的一句才是正题。这是收买官吏,树声威,以利诱和威胁相参辅。

对韩冈的几条政见没有什么新鲜感,王安石道:“还以为玉昆你会在军器监中大刀阔斧一番。”

韩冈摇摇头,他可没有打算对军器监大刀阔斧。

军器监的权力范围很大,在韩冈做了参知政事后,火器局也会重归军器监。理应将生产和研发两个系统分割开来。但由于人事制度的关系,很难做到。这么多年都将就下来了,韩冈也不打算强行更动,以免误事。

“小婿正想说,近日小婿打算将王居卿调到判军器监。”

韩冈并不隐瞒,王安石也不惊讶,谁都知道登上参知政事之位的韩冈绝不会让军器监落在他人手中。

“现任的两位判监,一位是慈圣的从子,另一位是黄夷仲。”王安石说道,“玉昆打算替换哪一个?”

慈圣就是慈圣光献曹后。判军器监便是她的亲侄儿曹诵。另外已经去世的那一位太皇太后,还有几个侄儿,其中曹评知审官西院,曹志勾当皇城司,曹诱提举醴泉观。至于亲弟弟曹国舅,等他死了,至少一个郡王要追封。

曹家、高家都是因为是外戚,故而连子侄都得到重用。即便出了赵颢、蔡确的叛乱,但皇宋以孝治天下,只要高滔滔还能做她的太皇太后,只要她还是先帝的亲生母亲,高家的待遇就不会降低。现在向家也一下子飞黄腾达起来。现在王厚之外的另一位提举皇城司,就是姓向。

在军器监中,另一位与曹诵配合的是黄廉。黄廉很早就投靠新党,王安石欲改役法,他便是马前卒。上一回炮打太尉府,炮弹上的判军器监黄正是他。

“黄夷仲。”韩冈毫不犹豫。

“玉昆。曹诵比得上黄廉?”

“比不比得上,那要怎看了。诸事无扰,黄廉不如曹诵。”

王安石脸色一变。韩冈的话太直接了,另一位判军器监,他只需要一个干拿钱不做事的。

“玉昆,你这是道统之争,还是党争?”

“岳父,小婿一向认为道统之争,不是在嘴皮子上,是在做事上。谁做到了圣人之言,谁就是道统所在。气学讲究以实为证,只在这一点上,小婿不会担心输给谁!”

“你这是做事?”

“日后看结果!”
