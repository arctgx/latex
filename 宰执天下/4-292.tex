\section{第48章 辰星惊兆夷王戡(上)}

年节前的东京城,应当是一年中最热闹的时候。

置办年货的行动在祭灶的这一天,也终于到了最高潮。熙熙攘攘的人群,淹没了城中各条主要商业街的街道。

前两日的一场大雪,在路面上已经看不见多少痕迹,只是隔着一段距离,就有一个半人多高的雪堆堆在路边。而远处近处的屋顶和树枝上,则依然是一片纯白。

只用了一天多的时间,就将城中主要的道路全都清理了出来,开封府的市政管理能力并没有受到知府更替的干扰。韩冈对这一点并不惊讶,曾经担任过府界提点的他很清楚官府在其中起的作用并不大,完全是行会和市民们的自觉自愿。除了几条御道是动用了厢军,其他的街巷全都是当地居民和商家的、清理出来的。

顺丰行送来的一辆四轮马车,沿着西大街正向城西的新郑门驶去。京城中通用的四轮马车,就是四个轮子上架底盘,没有什么转向装置,比起双轮车来要笨重榔槺得多,基本上都是用来载货,而不是乘用。不过四轮马车的车厢足够宽敞,对韩冈来说正合他的心意。

一大清早,韩冈上朝后就去了群牧司绕了一圈,做完了橡皮图章的工作,便找个借口从衙门里出来。反正天子对他的要求并不是辛勤工作做出一番成就来,干脆就让赵顼如愿以偿好了。

韩冈斜倚在马车中,全身上下两百零六块骨头中都透着懒散。听着车厢下的车轮声,百无聊赖的打了个哈欠,过两个月再稍微认真点做事也不迟。

他的身上已经换了普通士子的衣服,只要出了城,不出意外不会暴露身份。跟在马车旁的几个伴当也是衣着朴素,看着并不起眼。但他们在马背上背挺腰直的气派,一路上还是吸引了不少人的目光。如果给路人知道他们的身份,就不仅仅是吸引路人目光的问题了——若不是因为出行总是逃不开各色麻烦,韩冈也不会选择坐车,放弃骑马。

“韩信。”韩冈叫着车窗外的亲信伴当,“离新郑门还有多远?”

紧随在马车旁,韩信抬头看了看西面的城墙,低头对韩冈道:“回龙图的话,只有两里地了,不过前面路上人多,最少也要一刻钟。”停了一下,又问,“是不是要走快点?”

韩冈的这名家丁,与古时同名同姓,论理起这个名字并不太合适。不过韩冈没打算改。智信仁勇严,为将五德,信之一字是不能缺的。何况先圣孔子也说过,可以没有粮,可以没有兵,但不能没有信——人无信而不立。

“我知道了。不要急,走稳点。”韩冈又躺回到软榻上。

对于这辆四轮马车,顺丰行下了很多功夫。

内外的装饰看着很是朴素,没有如今流行的金银装饰,但实际上十分用心。车轮车毂,都是从军器监买来的精品,底盘、车厢都是名工打造,选用的木料也全都是上品,用大漆层层抹过,色泽深褐近黑。

车厢内的软榻上,铺着上等的羊皮,软软的,很是舒服。松木内壁上有几个暗格,可以存放散碎的物品。小小的香炉摆在车厢一角,正好卡在预先留出来的凹槽中。韩冈嫌烟气重,没有动用,但过去残留下来的余香冲洗得并不干净,车厢中依然有着一股沉香的味道。

这样的车厢虽然坐着舒服,但韩冈没有太大的兴趣。他现在只是盼望能早点赶到城外,昨天就得到了消息,他的妻儿今天就该到京城了。

但马车慢慢悠悠的在人流中行进了没多久,随着前面的一片锣鼓声,一下变得又更慢了,接着又停了下来。

韩冈的双眉皱了起来:“前面出了何事!?”

听着车厢中传出的声音似乎带上了一点怒意,韩信连忙道:“昨天城中各厢联赛全都结束了,肯定是在路上游街庆贺。这里是左二厢,当是鸭儿街的那一队。小门小户凑起来的球队,竟然连太后家的队伍都赢了,最后还夺了头名,不知多少人赔了老本。”他笑道,“等联赛过后就要举行季后赛了,要是鸭儿街队争到头名,那就更有趣了。”

韩冈没太关心具体的比赛结果,谁赢谁输他都不是很在意,但一个冷门球队,而且是出身平凡的球队,能一路过关斩将打进季后赛,对掀起普通人对蹴鞠联赛的爱好,可是有着推波助澜的作用。

“从明天开始,蹴鞠小报送一份到我书房。”韩冈在车中吩咐着。

“小人身上就有。”韩信立刻回道:“今天早上才买的,龙图要不要看,顺便打发一下时间?”

“嗯。拿进来。”韩冈应了声,伸手接过韩信从窗户中递进来的小报。

韩冈拿着报纸先没看,却对外面说:“赌球要省着点。小赌怡情,大赌可就要伤身了。”

“小的明白。”韩信连忙说道,他很清楚,韩冈绝不会放个烂赌鬼在身边:“每场也就十几钱,不敢赌得大了,还要存着娶浑家呢。……其实赌得大的也有,弄得倾家荡产、卖儿典女的看到好几个。有他们在前面,小的哪里敢踏进去。”

“赌只是玩而已,若当成发财的途径,那可就大错特错。”韩冈在车中说了句废话,又自嘲的摇了摇头。

这个时代赌博是禁不了的。什么都可以赌,大到房子、车马,小到水果、鱼虾,都能拿来当彩头。韩冈能确定赌马必然成功,就是因为世人对赌博的爱好来的。

展开手上的报纸,密密麻麻的一片黑字。上面有参加季后赛的球队名单,有球队的联赛成绩,有详细的赛况,有对名球员的评价,还有名家对各支参加季后赛的冠军球队的点评。由于是分区联赛结束之后,季后赛开始之前,这一期的报纸,内容份量很足,两尺见方的报纸全都填满了。当然,这份报纸上也不缺广告。陈家剪刀,刘家画扇,张家新衣,林林总总几十个广告在上面,广告词还很别致。

这并不是韩冈的发明。给官员看的朝报,唐时就有了,如今更多。每隔三五日,中书门下就会下发一份新报,让官员了解朝廷的政策。而传播朝廷中小道消息和秘闻的小报也多,不过不是固定发行,被禁的也快,但始终无法禁绝。

当蹴鞠联赛在各地展开之后,为了能让更多的球迷能看到比赛结果,以便下注,刊载比赛结果的小报也就随之诞生。首先是在京城,继而传播到每一座城市。刚开始时仅仅是简单的比赛结果,但随着时间的发展,仅仅是一两年的功夫,就变得跟后世没有太大的区别。

在韩冈看开,报纸能流行多半还是城市中识字的人多的缘故,农村能有个半成就不错了,但城内,多多少少认识些文字的男子少说也有四成。

韩冈低头看着报纸,过了一阵,锣鼓声渐渐远去了,马车重新启动,大约一刻钟之后,车速又减了下来。车帘被掀了一下,一个士兵向车中张望,不过被韩冈一瞪,连忙低头告罪,将车帘又放下了。

这是过城门时的检查,当马车再一次启动,终于是出了城。

过了金明池和琼林苑,韩冈的马车停在了路边。下了车,找了家干净点的,又能看到官道的小酒馆中坐下,点了热酒热菜,在火盆边慢慢的喝着。韩信等人也都叫了两碗热腾腾的羊肉汤,泡着炊饼吃。几个伴当都细嚼慢咽,只有韩信几口吃光,骑着马就向前赶去了,他得先一步去迎上主母和小主人。

大概一个时辰之后,韩信带着另一人回来了。

一见来人,几个伴当就跳了起来,“伍四。”“是伍四哥。”是被留在襄州的家丁。

“龙图。”伍四远远地就喊,“夫人和三位娘子,还有哥儿姐儿,就在后面,马上就到!”

韩冈暗骂一声,不过幸好伍四是浓重的西北口音,他叫破自己的身份,听清楚的没几个。只是原本就看着韩冈一行人觉得不对劲的店主,一下就瞅了过来,一脸惊容。

韩冈摇摇头,先一步赶出门去,留着伴当在里面结账。

王旖一行的确没多久便到了,慢慢的停在了守在路旁的韩冈身边。

“爹爹!”当先跳下来的是韩冈的两个儿子,才一个月不见,却感觉又长高了一点,都精神得很。

然后王旖带着素心、云娘一齐下了车,女儿金娘,还有三个抱着孩子的乳母也都下了车来。

一家之主出来迎接妻儿,自然是不合规矩。尤其是重臣,都要自重身份,但韩冈可不在乎。

官道人来人往,车马随即移到了路边上,韩冈对着眼泪汪汪的妻妾,“不要再耽搁了,从襄州道开封,路途遥遥,回到的府中都要好好休息几天……”看看周南没有下车来,问道,“南娘怎么了?”

“南娘妹妹身体有些不适,躺了有两天了。”

