\section{第33章 为日觅月议乾坤(六)}

西安到洛阳八百里,洛阳到京师四百里。

三天前从长安出发,到了今天已经看见了熟悉的汴河。

新造的铁路与老旧的运河在京畿大地上齐头并行,直通向那一座繁华富丽甲于天下的雄伟巨城。

一座座高高拱起的虹桥从一侧窗口掠过,由于黄河水带进来的泥沙堆积,虹桥之外,就只有高起的堤坝。但另一侧,是一望无际的田野,间或点缀着大大小小的村庄。

专供重臣的专车车厢上,装设了玻璃车窗,而不是普通列车的木栅车窗。内部的装饰,也是与吕惠卿的身份完全匹配。

离开了长安京兆府,拖家带口的上路。为官三十载的吕惠卿,还是第一次觉得千里跋涉的旅程是如此的轻松。

八节车厢,行礼、仆从,家眷,各有安排。甚至还有专门一节用来见客、起居的车厢。

车厢宽阔,站起来甚至可以走上几圈做散步。吕惠卿的卧室之中,甚至摆了一张兴起不及十年的拔步床来,除了上下都固定以外,与富贵人家所用床榻别无二致,甚至比吕惠卿在长安用的床铺都好。

吕惠卿现在所在的书房,除了桌椅书架皆固定,一切与正常的书房无异。

若说有区别,就是面前的这一张独运匠心的书桌。

只看桌面,与寻常书桌别无二致,但书桌下方,却是带了抽屉。

官造的笔、墨、纸,便整齐摆放在抽屉中。一方澄泥砚则是直接镶在桌面一角,砚台边框上有着波浪状的起复,这是精心设置的笔架。

笔洗也同样嵌在桌面上一角,不过不是惯常的瓷器,而是新出的铁胎琉璃器,以铁为胎,熔石化液,搪制而成。琉璃盆色如白瓷,盆中的嬉水双鱼则是鲜红欲滴。

这还仅仅是一张书桌,车中其他各处,无不可见设计者的用心之处。

骑马风大,马车局促,真要说起出行舒适,自是以行驶在铁路上的有轨马车为最。当年奔波于一座座驿之间的时候,吕惠卿从来没想过出行还能有如此享受。

尽管吕惠卿不想承认,但他也不得不承认,韩冈主持国政的这几年,大宋的万里江山简直变了一个模样。

大工大役,劳民伤财。即使新党在位的时候,也绝不敢在区区数年中,兴起长达数千里的工役。

只是因为畏惧辽国,又看见了铁路运兵运粮的好处,朝廷才开始决定大修铁路。

关中通了铁路之后,纵使西夏复起,亦是反手可灭。但更重要的,是这一条条以军事为名兴修的铁路,反倒带来了更多的税入,让国势蒸蒸日上。

当从东而来的铁路贯通潼关直抵长安,当从北而来的铁路自太原直抵黄河北岸,吕惠卿不需要出门去看长安城中日渐增多的南北时新货,只要翻翻府中的帐籍,看看商税增加的数额,就知道两条铁路所带来的好处到底有多大。

吕惠卿启程前,正听说京兆府的蹴鞠总社正准备与北地各大州府携手,将各地蹴鞠联赛的冠军球队,于年节期间齐聚京师,共争竞标,号为天下大会。

没有铁路,没有三日千里的高速,这样的提议,只会被视为疯人呓语。

“铁路虽好,日常维护就不是小数。”

“光是节省下来的驿馆开支,就足以弥补上维护费用。”

“天下铁路才几条,能省下多少?”

“铁路是不多,但全都是修在交通要道上,这也是驿站开支最多的地方。”

两个儿子在前面的争论,透过车门传了过来。

吕惠卿摇头,这种事有什么好争论的?这两个儿子比起他们的兄长来差了不少,正经事却不见他们争。

之前为了几家越长安西去的中书官吏,还问到自己面前,是否有唆使他们在名单上做手脚。

吕惠卿当时就把两个混蛋给赶了出去。

什么当问,什么不当问,活了这么大还不明白吗?

意外也罢,故意也罢,当事的七人都去了西域,那就是章、韩二人打算继续维持朝堂上的局面。

不过王安石既然已经选择了破釜沉舟,要保住新学的未来,想要将局面再维持下去,可就越发的难了。

吕惠卿就在此时,转迁他职。

章惇可以让他无法觐见太后、天子,韩冈可以让他在北地的几个大州府来回调动,但总不能不让他路过京师?

“看看热闹也好。”吕惠卿自言自语。

能搅搅浑水更好。他如是做想。

……………………

“介甫平章近况如何?”

“几乎快要复原了,前几日登高,上覆舟山时都没让人扶。”

“中风好得这么快!在江宁的翰林医官是哪几位?”

“还是家岳的底子好。”

“记得当年初变法,介甫平章连着几夜不睡,第二天还能上殿与富、文之流打嘴仗,这身子骨,自不是寻常人能比。”

“好身体是练出来的的。家岳退隐之后,每天去书院之前,都要先去蒋山【紫金山】走一圈。不是如此,如何扛得住病?”

两位宰相走在殿宇间的廊道中,低声的交流,让所有人都不敢靠近。

几步路的沉默后,章惇的话题跳回了朝堂,“……吕吉甫要入京了。”

“乱不了阵脚了。”

派系不同,韩冈也就不会像章惇一样,担心吕惠卿入京的危险。

“阵脚不乱,水会乱。”

“能乱哪家的水?王家?”

“狄家。”

韩冈脚步的节奏稍稍变了一点,随即笑了起来,“狄家如今风头占尽。家严前日还给我写了信,问跟亲家孙女争后位的,究竟是狄谘家的还是狄詠家的?还是说狄家有两个女儿都想要荐入宫中。”

章惇也笑了,“幸好只是一个。”

“所以回信去,就说是狄詠所生,却是庶出,且嫡母悍妒,三岁便逐出家门,养在其伯父狄谘家中。”

“若非身世曲折,岂得诸多口舌?前两日,陆佃登门,也说起狄家事,说‘今士大夫家娶妇,亦必求嫡,况于天子’?”

“这就是睁着眼睛说瞎话了。”韩冈立刻道。

“的确,如今有几个嫁娶是在乎嫡庶的?”

“门风比嫡庶重要。”

“玉昆。”章惇摇摇头,韩冈的糊涂装得太过了,“男是看前程,女是看嫁妆。有了进士才,朝臣家的嫡女可任选。有个千亩田,纵使外室所生,亦能招个进士女婿。”

如今士大夫家招婿,若非亲戚故交,便是要看前途,少问嫡庶。

一个才识驽钝的嫡子,哪里比得过一位进士在望的庶子?

章惇还是奸生子,照样娶得是名家之女。章惇是什么身份?

韩琦都是庶子,他又是什么身份?

男子前途与嫡庶毫无瓜葛。

男子如此,女儿家也一样。

妯娌之间,比的也是嫁妆多寡,而不是嫡庶。

嫁妆少了,你就是嫡出又如何?嫁妆多了,别说是滕妾所生庶女,就是外室所生,乃至奸生,还不是照样大把人去争?

韩冈笑而不语。

世风如此,何必多言。

章惇也知韩冈脾气,摇头又道,“只是庶出还好说,真的并不讲究那么多。慈圣再蘸,章献寒微,哪个讲究了?但狄家这个女儿身份委实太曲折了点。”

慈圣光献曹后,当初是先嫁了人,只是新婚之夜出了意外,又被送回了娘家。传言说是新郎官在洞房花烛夜被金甲神痛击额头,头疼欲裂,心知曹后贵不可言,自家高攀不上,甘愿送回,任其再嫁。真实情况如今已是无从得知,可不论从什么角度来看,曹后是二婚无疑。

至于章献明肃刘皇后的出身,说寒微已经是太温和了。根本就是蜀中银匠龚美之妻,之后被卖给还是太子的真宗。这位前夫龚美后又改姓刘,与章献皇后认为兄妹,还编了一个好身世出来。

狄家女再差,也是枢密使家的亲孙女,婚姻又清白。可她的父母实在不好定。

到底是以所生为父母,还是以所养为父母?朝堂上为此头疼了不止一日了。

“前几天太常礼院里面还吵了一回。”韩冈笑道,他已经听人说了当时礼院中争论的内容。

‘亲生父母俱在,女儿又不像男子,有过继之说,自当尊其亲生父母。’

‘狄詠夫妻弃其所生,狄谘收养,恩同再造,十几年养育之恩,以春秋大义,当以其为父母。’

‘国朝以孝治天下。万一其选为皇后,难道亲生之母不须加恩,难道嫡母不须加恩,难道养母不须加恩。’

只为了这件事,礼院中便大吵了一番。

“两父三母,当真做了皇后,日后朝廷有得头疼。”章惇叹着气。

“听说太后很喜欢狄氏女。”

“太妃似乎更喜欢玉昆你的内侄女。”

“她喜欢的不是我那内侄女,而是想借助家岳的身份。”

两位宰相于言辞间,对太妃颇有不满。如果给外人听见了,必然会惹来一场乱,

“玉昆你,你看怎么办?”

“相机行事。”

韩冈越发的看得开。不过狄家女,的确不适合母仪天下。

不过吕惠卿就要到了,他到底会做出什么样的选择?
