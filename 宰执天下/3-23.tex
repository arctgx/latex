\section{第十章 大河雪色渺(下)}

对于程昉的事迹,韩冈知道的,在京城待了有一年的种建中,了解得更清楚。

所以他很纳闷,程昉既然在河北管着几千上万民伕和厢军,用了几年的时间在漳河、黄河边修堤淤田,为什么还弹压不住,

不是东风压倒西风,就是西风压倒东风。刚才几个士兵的举动,分明就是在试探程昉。而程昉一时不查,弱了气势,便让肆无忌惮了起来。如果凭借着身份、地位,都震慑不住下任,为人所凌逼,也是。

程昉被韩冈帮了一手,压制住了手下兵丁,心情大好之下,便拿出钱钞向被伤到的几个百姓赔礼,然后让驿丞想办法腾出一个房间来。

做完了这些杂事,程昉这才跟韩冈、种建中正式叙了礼。

三人坐下来后,程昉便挑起话头,问着韩冈:“韩博士今次是准备去京城赶考的吧?”

韩冈上京赶考的事,京中知道的不少。毕竟河州大捷之后,王韶带着木征等一干俘虏上京,在其中起了关键作用的韩冈却没有到场,基本上都会多问上一句,韩冈做了朝官后,还要考进士的事情,很快就传开了。

只是程昉不知道韩冈怎么会跟种谔的侄儿走了一路。问话的同时,便下意识的瞥了种建中一眼。

“正是。”韩冈点头,“正好彝叔与韩冈分属同门,也要上京赶考,便一同出来。”

“原来如此。想不到种衙内竟然也是横渠先生的门下,今次一同上京赶考,当能同簪金花。”

“仅是明法科而已。”

高中之后,能簪御赐金花的,也只有进士一科。种建中用了七个字来更正程昉的错误认知。除此以外,他对程昉就没有别的话可以多说了

种建中的态度,韩冈已是见怪不怪。王中正、李宪这些在宫中呼风唤雨的大貂珰到了地方上,当地的官员中,除了一些意图钻营的没廉耻的货色,也都是不怎么跟他们亲近。

士大夫与内臣之间的交往,肯定都会受到士林的诟病。外臣跟宫中走得近了,连天子都不会乐于见到——家奴与外人亲近,哪家主人都不可能乐意,而且对于主人来说,自身也会有危险。

文彦博当年第一次被罢相,就是因为他跟宫中走得太近,不但结交宦官,还给宫里的贵妃送了许多珍物,最后惹起了仁宗皇帝的不快——论起人品,文宽夫其实是完全没有资格嘲笑他人。

韩冈尽管对宦官们没有多少的歧视,可也不愿意跟内侍走得太近。王中正那是没办法,见得多了,熟人间总得讲些人情。板着脸,把宦官当贼盯着,那是包拯、唐介一流的名御史的工作。保持正常的往来,才能让工作顺利的进行。

至于萍水相逢的程昉,就也不必刻意去亲近,尽点人情,一起吃顿饭就告辞拉倒。

只是韩冈善于为人处世,照着礼节邀请程昉一起吃饭,一杯酒下去,几句话一说,却便是宾主尽欢,轻易的拉近了与程昉的关系。

摸着酒杯,韩冈问着程昉:“不知都丞西来,可是有何急务?”

韩冈问话没有稍作曲言,问得很是直接。程昉并不觉得有必要藏着掖着,到了华州之中,自己的任务自然要公诸于众。而且前面几个骄横的士兵,已经说出了口,就更不需要隐瞒了,“程昉是奉了天子命,来关西察访河州灾伤。”

‘果然如此。’韩冈道:“这做驿馆里面,便有不少是河州来的流民。若是都丞能让他们安然返家,可谓是善莫大焉。

“程昉西来,正为此事。”来自宫中的都水丞摇头苦笑:“不意在道上御下不严,差点坏了大事,倒让两位见笑了。”

“京营禁军嘛……”种建中语带不屑的摇头,心有所感的他终于插了口,“家叔这两年也没少因他们而置气。”

程昉与种建中一同叹起气来。

韩冈基本上能知道种谔为什么会被开封禁军给气到,也能理解程昉和种建中两人为什么要叹气。

京营禁军传承自后周,太祖皇帝奉周世宗之命统领,周世宗驾崩后,赵匡胤便是仗着兵权而黄袍加身。而河北、西军中的禁军,又有好些军额都是来自于京营。对于这样的一支近在京中的队伍,历任天子都看得很紧。

其实这京营禁军说烂也不能算烂,至少弓术表演还是很有些水准。王舜臣当年去三班院报到回来后,曾说遇上过一个箭术只比他稍逊的开封人——以王舜臣的性格,那名与他同时参加考试的京营军官,箭术当不会在他之下。

不过真正到了战场上,这些平日里水平看似很高的将校士卒,就会露了本相,现了原形。刘平、任福、葛怀敏这三个丧师辱国的大将,无不证明了这一点。

程昉、种建中心头郁闷,一壶酒转眼就被他们喝光。

韩冈让驿丞再送一壶酒,转头却是一名班直护卫提着酒壶上来。他陪着笑脸:“都丞、博士、衙内请尽管喝,小人为三位倒酒。”

韩冈抬头就了瞪了那班直一眼,吓得他连忙放下了酒壶。

“你们是班直吧!?低三下四的服低做小,天子的脸面何在?!”韩冈厉声叱问着,眉心处的川字纹路,表明了他心头的火气有多大,“天子近卫是给人斟茶倒酒的?!做你们该做的事去!”

韩冈一甩袖袍,那位班直便讪讪的退了开去,与另一位同伴闪到了大厅一角去,不敢来触韩冈的霉头。连着神卫军的士兵都被吓到了,远远闪在角落里的身形皆缩了起来。

程昉在旁看到了这一幕,一边暗赞韩冈的谨慎——正如韩冈所言,天子近卫岂是能为人臣端茶递水?宰相都不能如此妄为。韩冈年纪轻轻,却是老成稳重得紧。不论那班直是真的想着过来讨好,还是另有图谋,韩冈都没给他半点机会。

另外,他更是叹着韩冈的威严。历经多次生死,在千军万马杀出来的气势,京中升上来的文官武将果然是远有不及。莫说是手上积攒了几千近万斩首的韩冈,瞪一眼,班直护卫都要闪一边去。就是方才种建中压着几名神卫军的士卒,可不是光靠着他叔父的名号,本身经过了多次战事后的气势,就已经先声夺人了。

外面的风雪越发的大了起来,吹得门扇哗哗直响,不过厅中的火盆更旺,透进来的寒风也吹不散听众的暖意。

种师中年纪小,需要顾忌的地方少,便被他的兄长唤过来倒酒。

韩冈接着种师中的斟好的酒,与种建中、程昉对饮而尽。

他敢于如此斥责班直,也是自有分寸。他占到了正理,并不是仗势乱压人,而且韩冈也知道自己能震得住这两名班直。文官,尤其是领过军的文官,基本上军中士卒们见了都是要怕的。

皇宋重文,文臣行事向来少受约束,若是哪一个武臣敢学着文官的行事,‘肆无忌惮’这四字考评当场就能贴上去。而文官一旦领过军,杀人放火的事便也见多了,心狠手辣起来,再凶狠的将领都要瞠乎其后。

文彦博因为守夜的士兵拆了他的凉亭取暖,能一口气将几十人远窜蛮荒。韩琦为了能镇住狄青,硬是找了小借口,就杀了狄青的爱将焦用。

广锐军叛乱,环庆路经略使王广渊什么都没做的,便被吴逵赶出城去——他就是一个废物。但当广锐军南下,附近几支有些看似不稳的队伍,就被王广渊诳到峡谷中,一气杀了两千多。

而王韶在熙河,砍那些杀良冒功的士兵时,也从来不眨眼睛。韩冈还记得高遵裕的一名远亲,人称高学究的。被高遵裕放到斥候游骑中挣功劳,不知怎么就给一队的同袍给杀了,剥光了丢在草丛里。而后一个糊涂鬼出战没斩获,回来时正好见到了路边横尸,大喜之下,砍了首级就回来报功。

但首级的真伪向来要检验,吐蕃和汉人之间,光是发型容貌就有很大的差别,更有许多细微的地方,能够让人确认真伪。这一验,就验出了真伪,甚至在韩冈主持的复验中,给查明了身份。正好杀了高学究的凶手们回来报称高学究失足落下山崖,这样事情便被爆了出来。

冒功的糊涂鬼被杖八十,而六名凶手,全都给王韶下令在寨门前给碎剐了。最后悬在寨门边的六个首级下,就是高高的一堆碎肉。

这样的狠手,武将很少能下得了,只有文官能做的出来。

甩开几个不听话的士卒,三人喝酒聊天,一座皆欢。

韩冈并没有在程昉的任务上插话,虽然背后迷雾重重,但这不关他的事。只要知道此事,并加以小心的不去涉足,便是足矣。而程昉也没有更深的与韩冈等人结交的意思。

萍水相逢,结个人缘,日后也许有用到对方的时候,只是现在,却是各自睡去。到了第二天,程昉冒雪西行,而韩冈也同样东去。

