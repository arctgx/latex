\section{第31章 停云静听曲中意(24)}

雁门陷落!

殿上死一般的寂静。

这是明修栈道暗度陈仓啊!若不是身在殿上,韩冈都会为之击节叫好。谁也想不到辽人会弃河北、取河东。

而这条从河东用金牌急脚送回京中的消息,让向皇后一阵头晕目眩,差点就晕倒在屏风后。

宋用臣赶紧上来扶着,心里直发毛,“圣人!不要紧吧?不要紧吧?”要是皇后也跟着中风,这可就全完了。

下面的宰辅们也慌了神,向皇后虽然不管事,但终究还是代天子听政,万一出了意外,谁也当不起这个责任。

在屏风后,向皇后重新坐好。这两天尚因为河北的局势没有恶化且逐渐好转而兴奋着,谁曾想那根本是辽人故意上演的好戏。她恍恍惚惚了好半天,最后方才开口问道:“此事确认了吗?”

“军国重事,没人敢作假。”王安石冷着脸说道。

“是吗?”向皇后的声音渐渐低了,过了片刻,而又重心提振起精神,“好歹河东的城防不差,兵马将校都是历练过的。”向皇后自我宽慰的说着,又问韩冈,“韩卿,可是如此?”

再怎么说,韩冈之前在河东待了两年,又很是打了几个胜仗,路中应该是有不少精兵强将。虽说代州败了,可不是还有太原吗?

她望着殿中的臣子,希望得到韩冈和宰辅们的认同。

韩冈没有回答。章敦、蔡确和薛向却都阴沉着脸,不见半点释怀。两府之中,只有他们是之前留任下来的成员。对当初胜州之战引发的一系列事端了解得很清楚,赵顼为了打压韩冈,究竟做了什么,他们都是见证者甚至是参与者。

“殿下,河东北界的知州、知军都是这两年才换上的。”蔡确叹着气说道,“旧日并无实绩。”

韩冈在河东,因为黑山党项的两万多斩首功,让赵顼弄得好生没面子。皇帝将韩冈调回京城,任了闲差,而跟着韩冈的河东将领们,也成了迁怒的对象。有功的升了闲差,没功劳则更是调到了数星星的地方。

尤其是代州的刘舜卿,西陉寨的秦怀信等七八名韩冈举荐过的将领,无一例外全都被调到了南方——荆湖、蜀中,秦怀信去年秋天的时候甚至病死在了夔州路【今重庆、贵州】任上。而之后接任的将领,要么是河北、要么是开封,还有一个来自于江南。

“怎么……”向皇后正想问个究竟,随即便醒悟了过来。

这当然不是打仗的路数!

就算是向皇后也明白了过来,惊讶的说不出话来。依蔡确说法,今日河东危局,其源头不是别人,乃是当今天子。

韩冈有几分不满的扫了蔡确一眼。都到了这时候,还玩心眼。这话一说,皇后怎么也不可能将河东战败的消息透露给皇帝了。只是单纯的失败,提上一句两句,说不定还能从天子手上得到些帮助。但眼下的失败,肇因却是皇帝本人,刚刚有了点气色的赵顼,如何听得来这个噩耗?恐怕又要堕入深渊了。

不过韩冈心里的确也有怨气,不论是西陉寨,还是雁门寨,或者是代州州治所在的雁门县,只要有一名良将坐镇,无论如何都不会变成现在的局面。

“雁门要塞,西陉寨是前关,而雁门寨是后关,代州城更是屯有重兵。官军据险要之地,却还让辽人连破三关,这是将帅之责。”

赵顼将韩冈举荐和重用过的将领们纷纷调离前线,换来了一批‘老成稳重’之人。他们很好地完成了赵顼的任务,没有挑起边界事端,更清除了韩冈对河东边界的影响力。变成了今日的结果,韩冈自然不会将这个责任担在自己身上。

“现在岂是争论是谁的责任的时候?!”王安石焦躁道,“太原府的兵马呢?!”

薛向立刻回答:“太原府的主力刚刚去了河北,现在应该到了真定府。如果要召回,他们从井陉赶回来,至少要五天。”

太行八陉,从北至南,分别是军都陉、蒲阴陉、飞狐陉、井陉、滏口陉、白陉、太行陉和轵关陉。

其中最北面的军都陉、蒲阴陉皆属于辽国,也就是后世有名的居庸、紫荆二关皆在辽人手中。

南面一点的飞狐陉,则宋辽两家各居其半。辽人据其东,为应州、蔚州,大宋据其西,即为代州。以瓶形寨——也就是后世的平型关——为界。

至于轵关陉、太行陉和白陉,那是河东连接中原的要道。作为联通河北、河东的战略通道,其实只有井陉和滏口陉一北一南的两条路。

从太原府至真定府的井陉,眼下是最为重要的道路——这井陉即是娘子关的所在,不过现在的名字叫做承天军寨。河东派往河北的援兵,就是走的这条路。

“麟府的兵马如何?”

“不用想河外之地的兵马了。雁门关既然被攻,北虏的目标又是太原,麟府军和胜州的兵马肯定会受到北虏的牵制,不可能腾出太多的人马来救援。”

向皇后慌了,没了麟府的援军,仗可怎么打:“这……这如何是好!?”

“太原不失,河东无忧。”韩冈立刻说道,皇后六神无主可不是一个好消息。

薛向也接口:“河东形胜之地,丢了代州,也不是灭顶之灾。先行保住太原,北虏便南下不得。出援的兵马解决了河北之敌,回师夺回代州也不在话下。”

向皇后吃惊不小:“去河北的兵马不用调回来?!”

“来回奔波几百里,其间劳而无功,那两万多兵马就是调回来也没力气和士气打仗了。”章敦强调道:“解决入寇河北的北虏才是重中之重!”

“那太原能不能守得住?”

关键的节点便是在太原。对照着沙盘,就是对军事一窍不通的向皇后,也知道太原到底有多重要。太原西南连接关中,向南便是中原,东面有井陉道通河北,北面便是代州。

河东地理,就是从南到北一连串的盆地所组成。即便雁门关失陷的现在,只要太原能守得住,那么局面就不会落到最坏的境地。可万一失陷了,那么

“殿下勿须担忧。太原乃是当世坚城,不会那么容易被攻破。”韩冈说着违心的话,安慰着皇后。

现如今的太原城,可不是唐时的仅次于长安、洛阳的北方名城,北汉灭亡后,太宗皇帝烧了那座方圆二十四里,人口数十万的雄城,迁址重建。如今的太原城,里面是一条条断头路,几乎没有易于通行的十字通衢,整座城池的修建目标是防止后世叛贼据城叛乱,防御的目的是防内而不防外。虽然由于地理位置的关系,太原城依然是天下有数的大城,可惜其防御能力还不一定能比得上一座普通的县城。

向皇后听出了韩冈话中的言不由衷,无力的一声叹,“王.克臣可能守得住太原?”

众皆沉默。如果是空泛的问一句太原是否可守,还能学韩冈那般敷衍一下,但具体到人,谁也不敢打包票。

现任知太原府是王.克臣,同时也是河东路经略使。一个六十多岁的老臣子,仅仅是身份不低,并没有多少值得让人期待的地方。

王.克臣的儿子王师约是驸马都尉,尚了徐国公主,乃是当今天子的亲妹夫——徐国公主并非高太后所生,并不是太受看重。不过终究还是天家儿女,王.克臣的地位自是非同一般。

能与皇家结亲,必然是勋贵世家出身。王.克臣的曾祖是国初名将王审琦,对太祖有翊戴之功。其祖王承衍更是太祖皇帝的女婿,作为太祖皇帝曾外孙的王.克臣当然是标准的世家子弟。而且他还考中了进士,在勋贵中算是很出色的人才了。

他的能力并不算差,在京东任职的时候,曾经在一次洪灾中惊醒的提前整修河防,拯救了满城百姓。可说多出众也算不上,比起一干名臣来,还是有不小的距离。只是在皇帝而言,他比韩冈要听话得多,而且祖父和儿子都是驸马,与天家关系极深,作为自家人也更得信任。

但要谈到王.克臣能不能应对眼前的局面,殿中君臣十几人没一人看好他。朝野上下,也不会对他有一分半点的信心。而且知太原府的责任并不仅仅是太原。

没真正领过兵,手下又没有良将,河外援军很可能会被牵制住,麾下精兵更是刚刚被派去了河北,一时间赶不回来——按章敦的说法,甚至是不能调回——河东的局势已是坏得无以复加,想要王.克臣在这样的情况下有回天之力,未免太过苛求人了。王.克臣他本人,也在方才送到的八百里加急中,附带了求援的奏表,半点也不能让人放心的将河东的安危交托于他。

想要挽回现在的败局,数遍朝中,有这等能力的,其实并没有几位。眼下在殿上,最合适的也唯有一人。

向皇后嘴唇动了动,却没说什么,只是同宰辅们一起望着殿中的那名深得她信任的年轻臣子。

在她而言,药王弟子坐镇京城,儿子的健康就能有所保证。只是儿子很重要,但国家安危更重要。而且这个坏消息还得瞒着天子,以免病情加重,这就决定了必须尽快解决入寇河东的北虏。

成了众望所归的焦点,韩冈暗暗一叹,说不得,这一回又要去河东走一遭了。

只是再过两天,就是太子正式出阁读书,连仪式都已经准备妥当。要是自个儿去河东这么一耽搁,王安石和程颢那边可就要抢个先手了。

这番心思一闪而过,韩冈仍没有多作犹豫,孰轻孰重他还是拎得清。上前一步,韩冈对着向皇后拱手道:“臣韩冈,愿为殿下分忧!”
