\section{第17章 往来城府志不移(二)}

赵顼透过窗户抬头看着天上的阴云,皱起的眉头,将心中的担忧表露在了脸上。

正是夏收时节,一场不适时的大雨,往往能将一年的收获拉下来一成半成。看似不多,但以一路几千万石的收获为基数,那可就是个让人无法承受的数字了。

尽管京城这边已经收获了,不需要有太多的担心,但河北、河东开镰的时间,往往要比京畿迟上几日。不知道会不会会不会遇上下雨的日子。若是在已经确定丰收的情况下,撞上一场破坏收成的暴雨,这一喜一悲,对身体可不好。

农为国本。平日里,各地的气候状况都会及时传到崇政殿中。今年上半年,天下各路都是风调雨顺,就是有点水旱蝗等灾害,也仅仅是局限于一州数县,并没有泛滥开来。

前些天,南方两淮开镰,报称丰收的喜讯几天之后便在赵顼面前高高垒起。

而这几天,京畿的麦田也开始收割。不见是各州各县,就是赵顼在南郊种得几亩田,报上来的也是大丰收,亩产四百余斤。甚至还献上了两支麦秆,就在赵顼的御案上摆着。麦芒一根根的有些扎手,金黄色的麦粒虽不算饱满,但其中一支上面长了三条穗,另一支的麦穗则是两条,乃是难得的祥瑞之兆。

虽说是种,其实也不过是籍田之礼,在春耕时执犁推了三推而已。但毕竟是赵顼亲自动手过,看到自己犁开的田地大获丰收,心中总是多上一份欢喜。

在御田中的新麦已经送进了宫中,明天,或许今晚,就能吃到用新麦磨出的面粉所做的炊饼、馒头或是汤饼了。

不过赵顼也做了十多年的皇帝,知道所谓的四石亩产,至少要打个七折,能有三石就不错了。而且还是天子田的缘故,水肥没人敢省,杂草、害虫时时清理,不敢有所懈怠。京畿田地的地力,耗用得很厉害。换作是普通的田地,一亩旱田,在丰收的年景,差不多也就两石上下。

北方的麦田,也只有关中的白渠周边诸县,河北的大名府,京东的郓济等寥寥数地,才会有亩产三石的时候。

赵顼轻叹了一声。田亩的产量是苛求不来的。北方的收成肯定是不能与南方相比。越是向南,产量就是越高,而且还是一年双熟。到了五岭以南,三熟都不是问题,只看田主勤不勤快,会不会照料庄稼。

在过去,广西诸州的稻米产量,这两年都多了起来。每年总有五六十万石的稻米,沿着左右江一路入海,与交州的稻米、木材和白糖等特产,一起运到福建,两浙和江东。市面上多了一百数十万石的广西大米,加之本身也是丰收的缘故,江南的粮价这几年都被牢牢压制在一贯一石半。连带的让京城的米价,也稳定在六十余钱一斗的水平上。

虽然赵顼不情愿承认,但他也清楚,这是韩冈在广西担任转运使和邕州知州后,给朝廷和国家带来的回报。

出身农家的韩冈,比起邕州过去的任何一名知州更为关心农事。而他在当地巨大的声望也能支持他的一切主张。更重要的,是交趾人在邕州的屠戮,以及大批的交趾俘虏,使得韩冈可以大刀阔斧的在广西推广更好的耕作技术,开辟沟渠来浇灌田地,而不用担心受到当地大户的阻挠,以及百姓对工役的反对。但换作其他官员处在他当时的位置上,恐怕都不会将精力分散在农事上。

只看韩冈在广西农业上的作为,便是宰相之任,何况这还是他诸多功绩中,并不怎么起眼的一项。就是太过出色了,要是他的才干,是由几个人分别拥有那就好了。

在浅灰色的天空中,几道笔直向天的黑色烟云十分的显眼。城西、城东都有几束浓烟直上云头,赵顼知道,那是两座铁场旺盛的炉火带来的结果。

京城的铁场,为了释放炼铁后的余烟,都修了高达近十丈的烟囱,城西、城东加起来有四五根。从烟囱中散布出来的烟尘,时不时的就飘到城中。使得润肺止咳的川贝母的销量,比过去还没有建立铁场的时候多了十倍。

眼下光是东京城周围的铁场,一年就有四千多万斤的生铁,百万斤的钢。而徐州钢铁产量,比京城还要多。如今大宋禁军能做到甲坚兵利,靠的就是一年上万万斤的铁,和数百万斤的钢。

被开采出来的大量石炭,并不仅仅应用在炼铁上,东京城中的千家万户平日里的生活已经都离不开石炭。一艘艘石炭船,充斥在在五丈河中。城外的河南河北十二炭场,一座座黑色的山峰拔地而起,囤积的石炭以千万计。前些日子,河南第八炭场的一座煤山突然无火自燃,闹得京城中连着几日都浸没在烟雾中。

而且石炭多了,烧砖也便宜了许多。河东路经略使韩冈前些天上书,利用麟州神木寨附近的石炭来烧制砖石。因为澶渊之盟,边境的寨堡不便轻易改建、增筑,但想到用砖石砌起外墙给城防带来的助力,辽人的反对声可以丢到一边去。若是再将内线的一干军寨的城墙包起来,前后两重壁垒,就是辽人来了,也只能望而兴叹。

被韩冈的这项提议所引发,朝堂上已经有了将东京城城墙用青砖全数包裹起来的动议,如今五十里东京城墙,有砖石防护的地段,只局限在十几道城门附近,以及城墙顶端。至于墙体,夯土还是暴露在外。若是能用青石砖包起来,就是面对霹雳炮,也能安心。

看到这些奏章,赵顼为此下定了决心,还是早点将韩冈调回来,让他留在河东,还不知会有什么事。可对于是否要增筑东京城,却还没有定下来。

东京城墙是四年前才修建完成,这时候又兴工役,京畿百姓和在京的厢军的怨言恐怕不会小。而且钱粮还是要多留一点才能让人安心。要想收复燕云,再多的钱粮储备都只会嫌少。

收复燕云,赵顼不会急于一时,但他也没打算拖到十年之后。赵顼绝不愿意守着如今用岁币买来的和平。收复幽燕、云中,是他的毕生所愿,付出再大的代价也心甘情愿。

要将新近得到的甘凉、银夏的州县初步安定下来,需要大量的移民和屯垦,就算动作再快,选用的官员举措得当,也差不多要五六年的时间。等到五六年后,西北稍定,那时候,差不多可以筹划对辽国开战。

赵顼回头看着张挂在殿中一角的舆图。从河北进攻,最大的问题的就是兵力移动不及辽人,粮草又转运不济,但对于这样的困局,眼下也有了应对的手段。

不过赵顼现在最想的并不是在河北铺设轨道,而是平行于汴河,从亳州将轨道铺到开封。

尽管这些年来,借用雪橇车,冬日依然可以利用汴河运输。但冬天在汴河上使用雪橇车的成本太高,且限制很多。如果天气不够冷,汴河水不封冻,雪又不足的话,就只能坐等老天赏脸。从时间上看,从十一月汴河封口,到来年二月重新启用。漫长的一百多天里,真正供雪橇车安稳运行的也就一个多月到两个月的时间。

如果能换成轨道,那么情况就两样了。至少不用再靠天吃饭,不论冬夏,都能派上用场。而且比起谁都能利用、无法稳定控制的水路,改成轨道之后,不但抽税查税方便,还能多饶一笔转运的费用。汴河北段这些年因为黄河泥沙涌入的缘故,河床越来越高,在汴河中的船只,比堤外的房顶都高,

就如方城垭口处的轨道,双线加起来也不到百里,但平均每个月的收入,稳定在三万贯上下,就算维持这条轨道的费用,一个月也要近五千贯,但利润是成本的五倍,这样的买卖,可以说是一本万利。以眼下的收入水平,只要再有两年的时间,就能彻底的收回了当初修造江汉漕渠所有的投入。

何况这还没有将多了一条联系南方的命脉,给京城带来的好处算进来。且有了江汉漕渠,开发荆湖可是更加方便了。

由谁来主持兴修亳州到开封的轨道,从沈括开始,赵顼已经在纸上写了七八个人名出来,但自始至终,赵顼都没有将韩冈的名字书于其上。

天色将晚,赵顼方从崇政殿中出来。站在殿门后恭送的内侍不是过去的那个身材高大,形似武夫的童贯。童贯办事不力,已经给发派到江西的洪州做走马承受了。但新来的黄门,却比不上童贯心思灵动,除了勤勉,也没有别的能力。

这几日,唯一的儿子赵佣又生了病,赵顼从崇政殿出来后,没直接会福宁殿,而是先去探望儿子的病情。

专职照顾赵佣的老宫女,在宫中被人唤作国婆婆。见到圣驾亲临,连忙带着宫人跪拜迎驾。

儿子在房里面病着,赵顼没耐心顾这些俗礼,急着问道:“六哥怎么样了?”

“回官家的话,喝了钱太医的药,已经能睡得安稳了。”

赵顼稍稍放心了下来,钱乙是小儿科圣手,他开的药不会有什么问题。

“今天有谁来过?”赵顼又问道。

“就朱娘子亲自来过。太后、圣人、大刘娘子、小刘娘子、刑娘子,都遣了人来探视。送来的药也都造册后收了起来。”

赵顼听着点了点头,唯一的皇嗣病重,除了生母能来,其他人都不得不避嫌。

‘也是六哥儿体质一向虚弱的缘故。’大宋天子暗叹。

从胎里出来,就没少病过,不论是穿多了或是穿少了都少不了生上一场病。一个痘疮,身体壮的小儿能撑过去,如赵佣这样的体质,只有夭折的份。种了痘之后,至少放了三成的心。只是会造成小儿夭折的疾病可不止痘疮一种。

韩冈提出了免疫法的理论。依据这个理论,所有得过之后就不会再犯的疾病,都有可能跟牛痘一样来事先预防。为此,赵顼给厚生司和太医局的钱从来都没有节省过。还不知什么时候能看到成效。
