\section{第30章 众论何曾一(二)}

以雷霆手段一举铲掉了绊脚石,同时将民怨转嫁给一干粮商,王安石在京城和朝堂重新确立了地位和声望。他的相位,一时间不会再动摇。原本想看着他笑话,准备携起手来将其请出东京城的一干人等,也都偃旗息鼓,一个个都安分了起来——反正河北京畿的旱灾还在继续,今年的肯定是要绝收,到时候再出手也不迟。

只是被王安石所击败的粮商,却都不是让人省心的货色,差不多各个都能与赵顼攀上亲。虽然卷着民意一股脑的鼓动天子将他们给捉了起来,但如今事情稍定,麻烦也便来了。

宗室也分远近。绝大部分的粮商,他们娶的县主、宗女,与天子的关系都不算很近,只是在大宗正寺有个名字罢了。可是其中一人的身份,却让赵顼听说之后,都会感到棘手,更别说王安石、吕惠卿他们。

“粮行行首高扬的儿子娶得竟是临汝侯的女儿!”

说话时,吕嘉问面色严峻。王雱听着却有些纳闷。临汝侯又怎么样?郡公的女婿也在大狱中坐着呢。再说京中几千宗室,公侯遍地,他哪知道临汝侯是谁?

吕惠卿也奇怪吕嘉问的一惊一乍,很少见他如此模样:“一个宗女而已……”

“是县主!”吕嘉问立刻更正,神情更加沉重。

“县侯的女儿怎么封县主……?”王雱脸色一变,急问道:“是哪一房的?!”

看到王雱终于明白,吕嘉问叹道:“是濮安懿王的曾孙女!”

厅中的诸人同时吃了一惊,王安石都免不了脸色一变。王雱惊问道:“怎么可能,濮阳郡王是什么身份,怎么会答应将侄孙女儿嫁给商户?”

英宗皇帝赵曙是濮安懿王赵允让的第十三子,只是自幼被没有子嗣的仁宗皇帝养在宫中。他登基后的濮议之争,就是是否要追赠其父为帝,还是只称皇伯,从而引发的朝堂之争。虽然英宗没有成功,赵允让只是被称亲。

可不管怎么说,濮王一系在如今的宗室中,地位十分特别,就算是天子也要让他们三分。赵允让的次子,也就是英宗皇帝二哥,如今袭封的赵宗朴最是要面子,怎会可能会答应这么一桩婚事?

吕惠卿叹道:“高扬之母是魏王家第八房纪国公德存家的山阳县主,其妻亦是县主。本来就是皇亲国戚,为儿子与濮王家结亲,大宗正寺怎么会管?”

王雱听得更为惊讶,母、妻皆为县主,高扬本人至少也一个地位不低的环卫官。忍不住问道:“高扬此人怎么自甘下流!?”

“商人出身,还能怎么样?用钱买来的亲戚,能洗多干净?米商又是祖传的行当,他又如何甘心放弃?”吕嘉问长叹着:“说实在的,当是临汝侯那边贪了那几万贯的彩礼,还有四时八节都不会少的礼金。临汝侯所在的那一房早年去了南京定居,与京城的兄弟们来往得也少,一个庶出的女儿出嫁,哪一个会在意?”

吕惠卿对此也稍有了解:“在南京应天府的那一批宗室,不在天子脚下,他们做出的事是向来出格。”

吕嘉问摇着头,叹气一声接着一声:“高扬也是聪明,被捉起来后根本就没细说,硬是在狱中坐着,也不让自己家里面来闹。等过了年,开封府开始查玉牒,这才给发现了。现在消息也到了南京,年前事情在风头上不好闹,现在风声稍定,到了太皇太后面前去求情,说不定还真能脱身。”

“那就诏令与高扬之子和离,将女儿领回去就是了。”王雱很不在意的说着,“反正都是为了钱。”

曾布摇摇头:“这不合法度。”

依律夫妻是可以离婚的。丈夫因故单方面遣出妻子,叫做休妻。而夫妻两人都同意离婚,则称作和离。但丈夫犯了法之后,妻子单方面要求离婚,从法律上说,是不会得到允许的,更不合纲常。

“而且还有儿女在。”曾布接着反问,“骨肉连心,总不能把他们都和离掉吧。”

“不然还能怎样?总不能就此放人吧?”王雱狠声说道,“这可是天子亲自下的诏令!”

“但天子必有悔意,怎么说都是濮王家的人。”吕惠卿作为天子近臣,很了解赵顼的为人。如今的皇帝就是这般,心思和想法都容易波动。当日因粮商们盘剥民财而勃然一怒,将之尽下大狱治罪,谁求情也不理会。可是等到这年节一过,怒气稍收,想法也会随之改变。

宗室们的反扑乃是预料中事,但濮王一脉的身份太过于棘手,天子很难加以重惩。可一旦这一个被放过,所有人便都能籍此脱身。

吕惠卿和曾布都望向王安石,他们都知道该怎么做,但这句话还得王安石来说。

一直沉默着的王安石,不出意外的保持着刚硬,一点也不在乎得罪濮王一脉的后果,“祖宗亲尽,亦须祧迁。更别说此辈贪于私利,动摇国本。从饥民身上渔利时,可曾想过会造成多少百姓成为路边饿殍,可曾想过会因此而造成民变?!即是如此,如何还能宽宥?当依律加以严惩!”

吕惠卿、曾布都知道王安石会这么说。他们更清楚,这番表态,对于王安石却不会有好结果。吕、曾二人都是熟知文史,几乎在同时想起两个人来——商鞅、晁错。

商鞅变法,触犯了以太子为首的秦国贵族。晁错则是鼓动景帝削藩,开罪了所有的藩王。两人最后都没有能落个全尸。

不过对于新党和新法,并不用太过担心。就像商鞅被车裂之后,秦国依然坚持他所订立的法度,而晁错被朝服腰斩于市后,汉景帝、汉武帝照样还是要削藩。

可是从王家的角度来说,后事堪忧啊!王安石眼下这个态度,当真是为国无暇谋身了。身受天子知遇之殊恩,欲鞠躬尽瘁以报之。虽然让人敬佩,但家族都不顾了,他们怎么都学不来。

粮商一案,是由开封府、御史台、审刑院三堂会审,不过最终的结果还要秉承天子之意。在赵顼的态度表明之前,王安石暂时还不能插手其中。

暂且丢下这件烦心事,王安石问道:“方今京中的粮价如何?”

身为三司使的曾布立刻答道:“前面动用了一百一十万石常平仓存粮,京畿粮价都恢复到七十文一斗。”

“不是七十八文?”王安石惊讶的问道,心头微微生怒。官府卖粮可都是一陌一斗,七十八文的价格是他亲手批准,怎么没人跟他说,就私自将粮价降到七十文去了。

“官府散出的米价还是七十八文。”吕嘉问接口道:“给出七十文的是京畿残存的粮商。金平等大粮商皆被捉了起来,这一干没被捉起来的中小粮商全都被吓到了,哪里还敢再卖高价。”

王安石略略皱眉,有些担心的问道:“他们不会亏本吧?”

“只说米价。粮商们在田间收购稻谷,基本上都是二十文一斗。加上运费、人工,还有碾制的损耗,成本也不过五十文。”吕嘉问掌控市易务一年多,浸淫日久,商务上的事情也便越发的熟悉起来,“金平等大粮商,前段时间以超过正常一倍的价格高价购粮……”

听到这里,王雱冷哼一声,“此辈心怀叵测,”

吕嘉问附和的点着头:“谁说不是,虽说成本贵了二三十文,但真的给他们得逞,明年……不,是今年。今年仓中多一斗,他们就能多赚六七十文甚至一百文。不过中小粮商就没有这份财力,没有在这上面花钱。放到现在,就是他们的运气了。”

停了一下,吕嘉问问道:“相公,要不要将官中售粮的价格也降下来?”

王安石摇头,“不,用不着。常平仓卖粮是为了降粮价,不是赚钱。仓里的粮食还要用来赈济灾民,能少卖出一斗就是一斗。”

常平仓的确不是用来的赚钱的,现在仓中的粮食因为价格标得高而卖不出去,可到了流民来了的时候,就都要免费送出去了。

吕嘉问起身向王安石行礼以示敬意:“相公仁德爱人,嘉问感佩。”

曾布在一边冷眼看着吕嘉问奉承着王安石,他这个三司使做得很没有意思。吕嘉问是他的下属,却从来不听他的话,有事从来都是找到王安石这一边来,或是去找吕惠卿,而两人也没有对此破坏朝规之举加以指正。就如今日之事,吕嘉问不先通报自己,直接到了王安石这边才说出来。几个月下来,曾布的心中已经积攒了一团火。

唇角保持着温文尔雅的浅淡笑意,收在袖中的拳头捏紧又放松。

权力的争夺要未雨绸缪,只看在宣德门之变上横插了王安石一刀的蔡确,他现在侍御史知杂事的身份,就知道天子的态度了。如今也只消仔细看着赵顼怎么处置这一次的案子了,若是天子还是想要保着几家亲戚,那自己该怎么做,也就可以确定了。

