\section{第44章 本无全缺又何惭(下)}

同群牧使。

韩冈再拜起身接过诏书的时候,发现宣诏的石得一,脸色有点尴尬,上来道喜时,说话也不是很流畅。

他的反应,自是有所缘由。

群牧使和同群牧使,仅仅是加个同字,意义则大不相同。这是进士和同进士的区别,也是夫人和如夫人的差距。

本来听宫中传出的消息,韩冈还以为自己能替代担任现任群牧使的韩缜,谁能想到却是与韩缜做类似于伴奏、回声、影子的同僚,当真只需要签字画押了。

想想也是,群牧使一向兼任枢密都承旨,眼下天子就打算将自己晾起来,难道还能让自己在枢密院占个位置吗?

群牧使还有余事可决,又在枢密院兼差,但没有兼职的同群牧使,基本上就是跟宫观使一样的闲职,看来天子是一点事都不要自己做了。

抬眼看看石得一,天子近侍慌慌忙忙说了两句恭喜的话,便拿着喜钱向韩冈告辞。

石得一换作给其他官员宣诏,绝不会有现在局促不安。可韩冈不一样,实实在在的仙佛传人,怎么看都像是星君转世,对于他们这些身体不全的中官来说,很是有几分类似于面对宰相时的畏惧。

原本议定的韩冈职位真的是群牧使,可今天制书一下,就变成了同群牧使。根本就不让韩冈有再立功勋的机会。

不到三十就已经是当世有数名臣,不论在什么职位上都能做出一番成就。眼下还有献上种痘法的功劳没有赏赐,同时朝中又有在河北兴修轨道的动议,到时候,还要将首倡之功追到韩冈身上。

天子不是不想看到韩冈立功,但要是他再立功劳怎么解决?

韩冈的官位不能再升了,不做过一任宰执,本官升到谏议大夫基本上就已经到了顶。就是用爵位、食邑、散官、检校等空衔来赏赐,也是很麻烦。韩冈不及而立,就比得上那些辛苦了几十年的老臣,就等于给日后留下一个先例,让后人可以钻空子——到时候,想钻空子的人不会提到韩冈的功劳,只会看到韩冈的年纪。

石得一其实也为天子而感到头疼,韩冈留在京中怎么都不好安排,可为了六哥,还不能将他往外地打发,宫中都有说法,过一阵子,就让韩冈去资善堂兼差。

石得一匆匆忙忙的走了,韩冈毫无介怀的与驿站中前来观礼的其他官员打了招呼,笑着回去修改写好的谢表——原本是为了感谢天子赠以群牧使的奏表,现在得改成群牧副使才行。

接下了同群牧使的任命之后,韩冈不是急着上任,而是派人去襄州接家眷,又派人去开封府找房子。就跟宰执和三司使这样的高官一样,群牧使也有固定的宅邸,任职之后就可以搬进去,而同群牧使只能可怜的去租房子。

官员上任一般都有期限限制,需要在时限内就任。对于身在京城的韩冈来说,他十五天内到任就可以了。既然天子赵顼安排了这个职位,要他不要做事,自然就赶在最后一天去衙署——在需要体察上情的时候,韩冈一向会变得很知情识趣。

在城南驿住了两天,开封府那里批了一间带着七八亩后花园的宅子来,说是太祖一脉由秦康惠王的三子高平侯传下的宅邸。因为王安石改定宗室法,出了五服的宗室不再享受官职爵禄,这件宅子就被开封府找借口给收回了。空了一年,正好安排给王安石的女婿住,比起旧时韩冈担任判军器监时的三进宅子要大得多。

房子很旧,又是空关了一年,草木丛生,灰尘边地,不仅仅要打扫起来,许多地方还要修理。按照惯例,开封府安排了二十四名兵卒,来府上听候使唤。韩冈就让他们先将主屋打扫出来,自己好住进来。

房子有了,韩冈接下来忙的倒还不仅仅是自己的事。闭门谢客的章惇那里,韩冈让人稍了封信,说了些安慰的话,尽一份心意。接着韩冈没去见他的顶头上司吕公著,而是去登门拜访王珪,他有事要求当今首相帮忙。

以韩冈如今的声望地位,纵然是礼绝百僚的王珪也不能慢待,很快就将韩冈迎进了会见亲友的花厅中。

听了韩冈的来意,王珪问道:“这个吴衍就是当年举荐玉昆你的人吧?”

“正是。韩冈为布衣时,是三人同荐,王资政、张团练,接下来就是吴通判,当时他正在雄武军节度判官任上。”

韩冈拜访王珪,正是为了当年帮了他大忙的吴衍。

吴衍算是很落魄了。如果当年他没有站错队,没有疏远王韶,能辅佐王韶攻取河湟,光是那一份军功,如今好歹一个上州知州。

加上王韶之后升任枢密副使,作为王韶的旧班底,不论是在熙河路任官,还是跟着王韶入京,甚至只需凭借参加过西北战事的资本,就能在官场中飞黄腾达——天子有开疆拓土之心,拥有军事才能和经验的文官,永远都比普通文官升得更快。

但吴衍做错了选择题,应该属于他的机会给了别人。韩冈姑且不提,就是当年听从他指派的王舜臣,只是一介兵卒的赵隆、李信,乃至仅为一衙内的王厚,如今都是高官显宦,镇守一方。

而吴衍,近十年过去了,他现如今还仅仅是个下州通判,想要再往上升,得一年年的熬着时间,这叫磨勘。

这才是底层官员的现状。选人转官很难,而京官升朝官也不是那么容易。金字塔状的官场构成,每向上一步,几乎都要经过一场不见血却同样惨烈的厮杀,要与无数同列相竞争。如果没有一个底蕴深厚的靠山,想多走一步都不可能。以吴衍的情况,他这辈子做到知军就到顶了。

不过韩冈对吴衍旧年的帮助感激甚深。若非有他指派了王舜臣护送,韩冈极有可能躲不过陈举接踵而来的攻击。

之前吴衍远在淮左任官,韩冈无从相助。正好吴衍如今上京,虽然他本人太重脸面没有登门,但韩冈从城南驿那里听说之后,在情在理也得帮他一把。

“玉昆当真是念旧情。”王珪笑赞了一句,接着就爽快给了韩冈一个肯定的答复,“厚生司中光是一个判官的确不够。”

“多谢相公。”韩冈起身拱手行礼,他是真心感谢王珪。

“如今天子有心振作,朝中要仰仗玉昆你的地方还很多。”

王珪是三旨相公,对天子来说,这样听话明事理的宰相的确很好使用,但到了面临危局和战乱的时候,能力就要考虑在前面了。

天子有意对西夏开战,王珪必须主动参与其中,掌握足够的资源,否则天子肯定是要在开战前换上一名到两名更为合适的宰相。

从熙宁三年开始,王珪就进入了政事堂,九年宰执,这个时间长得令人惊叹,但也代表着天子随时有将他换掉的可能——能做满十年宰辅,在立国以来的一百多年里,屈指可数。

王珪需要表现出自己的能力,取得让人信服的功绩,不仅是为了将眼下首相的位置坐稳。而且更是为了准备在离开相位后,能够更顺利的卷土重来。

韩冈和王珪都是明白人,互相皆有所求,坐在一起,反而不用云山雾绕的兜圈子了。

“天子欲用兵于西夏,以如今大宋军力,当能轻易取之。但辽国动向不明,一旦开战,河北、河东或会变生肘腋。玉昆可有以教我?”

“料敌从宽,要用兵西夏,必须得将辽国一起算进来。纵然辽国一时不敢南侵,但在河北边境上囤积二三十万大军,不是不可能,甚至可以说是肯定会这么做。”

“所以要修轨道?”

韩冈点头:“河北军力主要在三关布防,大名府和开德府【澶州】也有重兵。不过几处兵力分散,辽军一旦南下,避实就虚,当为其各个击破。不过一旦有了贯通河北南北的轨道,大名府和开德府乃至京城,都可以轻易遣兵支援。若河北能以更胜辽人的速度在三关聚集起大军,想必辽主和魏王也得重新考虑一下火中取栗的后果。”

王珪想要的是胜利,不是辽国入侵后的替罪羊,轨道的好处显而易见:“这修路的人选,还得玉昆你来推荐。”

“李诫在营造之事上是奇才,下面的一干工匠也都听他的吩咐,相公可以将实务交予他来处理,另外遣人居上统管便是。”韩冈停了一下,“不过有件事,韩冈当先禀于相公。七百里和六十里完全是两回事。铁、木、石子这些建材要及时运到。七百里河北轨道的勘察、营造、转运、修理,更是前所未有的难题,一点意外,很可能要用上比预计的时间更长的工期。”

王珪咋舌:“要不是玉昆新授群牧司,要我就将你绑到河北去了。”

韩冈发现王珪有个好处,他能放下宰相的架子,该开玩笑就开玩笑,该恭维就恭维。当然,这也是韩冈的身份地位到了。

尽管王珪礼绝百僚,但韩冈已经差之不远,怎么说他都已是金字塔上最顶端的一群人中的一员。见面时都要给份脸面。

韩冈已是实打实的重臣。

