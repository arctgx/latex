\section{第13章 晨奎错落天日近(五)}

“有日子没来这里了。”

韩冈打量着今天落脚的地方,轻声的喟叹着。

“是很久了。”

童贯声音更轻,生怕打扰到感怀旧日的韩冈。

福宁殿原本是内宫的中心,但前一任主人不在人世,而现任的主人,则跟着太后住在慈寿宫中。

整座福宁宫空荡荡的没有人气,从正殿到偏殿,包括宰辅们将要宿卫的地方,人员数量,远远比不上要小上一倍还多的慈寿宫。

正殿的方向上,能看见点点光亮,但绝没有天子还在那里时的灯火辉煌。

上一回韩冈在此处休息,还是熙宗皇帝刚刚去世的时候。

韩冈慢慢走在离宫墙不远的廊道间,感慨着时间。

一转眼一年就过去了。

“参政。”童贯方才离开了一下,现在又回来了,低声对韩冈道:“参政今晚的住处方才小人已经安排好了,只是今天有些仓促,才派人去收拾,得稍待片刻,还请参政宽宥。”

“嗯,没关系。”

“多谢参政。”童贯腰身弯了弯,又道:“王平章和郭枢密还没到,不过他们的屋子也备下了。总共三间房舍,参政打算选哪间?”

韩冈想了想,“……就迎风的那间吧。平章和郭枢密年纪都大了,晚上受不得风。”他侧头瞥了一眼惊讶的童贯,“还是你想说,那几间房之前给修过了,不再漏风了?”

“是,啊……不!”童贯连忙道,“回参政的话,那几间屋子的确都有些漏风。不过小人已经让人去找毡子贴着墙张挂起来,这样风就进不来了,炉子也安排人生火了。”

韩冈站定脚,看着童贯道,“去了海外一趟,办事倒比以前要强了。”

童贯笑得更加谦卑,“不敢当参政的赞,只是多了些历练,也开了眼界。不过还是在京城的时候,能多得几次参政的教诲,进益会更多一点。”

韩冈笑了一笑,重又慢慢的向前走。

慢慢走到今晚的住处,里面还在忙着铺陈摆设。

就像童贯方才对他说的,福宁殿中供宰辅宿直的几间屋子都是漏风的,不过挂上厚厚的羊毛毡,又点起了炉火,情况就好了许多。

福宁殿的正殿后殿在赵顼驾崩后修缮过一次,不过偏殿没有整修过,跟皇城里的大部分建筑一样,都是四面漏风,不过总比回到政事堂好一点。

仁宗之前,宰辅宿直都是在政事堂或枢密院中,即便是现在,若有军国事需要留宰辅在皇城内,基本上,还是住在政事堂中,留宿于禁中的次数寥寥可数。

中书门下的一干建筑,不知多久没修过了,漏风漏水,冬天冷,夏天热,春天秋天也没多舒坦。

今天七月里的时候,有一次午后暴雨,韩冈当时恰好入宫去了一趟。没了主人在,吏员们都各自照管自己的一摊事,完全没注意到里面在漏雨。等到韩冈回来,才发现摆在桌上,

当值的堂吏韩冈没开除他,将其降职调任了。有过当罚,韩冈也没打算用这点事体现自己的宽宏大量。

每到留宿政事堂的时候,韩冈总是在想,等天亮了就去提议将皇城里面的各个衙门都修一下,可当真等到天亮,韩冈立刻就把这个想法跑到九霄云外去了。

开什么玩笑,光是提议,就不知要在朝堂上挨多少骂,受多少白眼。御史台的房子破落得其实更厉害,要是政事堂中有人打算把自己工作场所弄得更舒服一点的想法传出去,让那些每个冬天都在房子里冻得瑟瑟发抖的御史们怎么想?

即便撑过了御史的弹劾,等到皇帝同意了,朝堂上也通过了,可钱从哪边来?

就是找到钱了,从工程款中也弄不到多少好处,跟后世的情况截然不同。更不用说等一两年后修好了,也轮不到自己享受,这是何苦呢?

让童贯叫来的内侍们继续整理房舍,韩冈从偏殿出来,就见到几盏玻璃灯笼引着人向这边过来。

远远的已经看清了身份,是同在今日宿直的王安石和郭逵。

韩冈迎上前打了个招呼,“岳父,郭枢密。”

“玉昆。”“参政。”

离开慈寿宫后,三人再度见面。郭逵退到一边,韩冈则落后王安石半步,回头走上偏殿的台基。

今天是第一天,也是最重要的一天,若有何变乱,基本上只会在今天发生。等到明天,宫中人员安排该调整的就调整好了,想乱中取利便没有那么好的机会了。

决定宿直人选的时候,韩冈和王安石都是主动要求担任今日宿卫的职责。

说起来还是两边都互不放心,担心今天留在宫中的时候,会借机做出事来。

苏颂离开的时候,对着韩冈摇头苦笑,从今天这件事上,连外人都能看得很明显,王安石和韩冈翁婿两人的嫌隙已经很深了。

站在偏殿的门口,王安石停住了,转身遥望正殿,韩冈陪在他身边,郭逵则找了借口,先进了殿中,只留下翁婿二人在门口吹风。

韩冈不说话,王安石也不说话,两人之间仿佛冻结了一般。一众内宦和禁卫,都是能躲得多远就多远,深怕一阵狂风,被风脚给扫到。

不知沉默了多久,王安石突兀的开口,“开封府今天要忙些了。”

韩冈微微一笑,今天白天的时候,沈括就要忙着宵禁的事了,当然忙。不过,比起历任权知开封府,这点辛苦也算不上什么。京师百万军民,每天都有千百桩事等着开封知府来做,什么时候不辛苦?

所以他反问:“开封府哪有不忙的时候?”

韩冈存心给王安石添堵。又是一阵静默,才听到王安石道:“……将吕吉甫召回来吧。”

韩冈笑了起来,“回来权知开封府?沈存中会很乐意。”

太后重病,你还想举兵北向,到底是想做什么?真正的目的,到底是在北,还是,在南?

很遗憾,韩冈不能这么责问吕惠卿。吕惠卿完全可以明面上偃旗息鼓,私底下让人挑起边衅,将罪过推到辽人身上。以辽人的脾姓,想要拆穿都难。

今日朝堂,没有太后相助,韩冈根本拦不住吕惠卿。

可韩冈完全不在意,一个玩笑之后,迎着王安石含怒的目光,又道:“太后只是小恙,不日便可痊愈。太后康复之前,我等一如往日便可,没必要改变任何事。”

“京城中会乱的,太后的病情在民间,只会越传越离谱,人心也会越来越乱。”

不论向太后的病情轻与重,都不是可以对外随意透露的消息。而且即使是透露了,也不一定会有多少人信来自朝廷的辟谣。绝大多数的时候,总是小道消息和谣言更能让百姓们相信。朝廷的信用,本来就是这么回事。

即使是太后痊愈了,朝臣、宗室、外戚,甚至包括宫中的内侍、宫女,看待太后心态也会有所变化。

有了一次,就会有第二次。这一次晕倒,下一次就有可能昏迷不醒。

人心一旦有所动摇,一切鬼蜮心思就有了冒头的机会了。

“御史台会乱吗?”韩冈再次反问,“章子厚会忘掉提醒李资深吗?”

王安石沉下了脸,当韩冈开始反问的时候,总是那么的尖酸刻薄。

……………………

“得看好御史台。”

“子厚放心。”

章惇直至入夜才从宫中出来,同行的正是御史中丞李定。

听了李定的保证,章惇张了张口,却没有话说出来,只有一声叹息。

他不是什么都不懂的新人,他在官场上都几十年了,遇上今日的形势,御史台能玩出什么花样,怎么可能不明白?

太后暂时不能理事,这就是一个机会。

御史们的弹章,纵然不可能让韩冈直接出外,也能让他灰头土脸的在家蹲一阵。

可章惇也好,王安石也好,都不想动用这柄能割伤敌人,却也会让自己被割伤的利器。

御史台几经清洗,如今万马齐喑。

绝大多数御史为宰辅们所控制。太后不想破坏朝堂中的稳定局面,宰辅们跟她用一个心思,所以御史们的野心都被压得死死的。而一干金紫重臣,由于在国事上发言的机会比过去更多,也很少通过关系去煽动御史,针对心中目标。

只有一两个看不清时势的愣头青,不过雨水淋漓的南方,会让他们冷静一点。

旧党推荐给韩冈的人选有不少,可韩冈只会将人安排到地方上,或是京中一干实务差遣,如御史这样的清要之职,韩冈完全不去理会旧党的要求。

自始至终,韩冈只推荐了一个游醇进入御史台,那是他的幕僚。而且那也不是韩冈亲自所推荐,而是苏颂出手。

即便这段时间以来,与吕惠卿屡屡相争,韩冈也没有动用他能影响到的几位言官的力量,去弹劾吕惠卿,以图干扰他对辽开战的调门。

吕家是福建大族,子弟众多,自有贤与不肖之分。吕惠卿的几个亲弟里面,吕升卿和吕和卿都不是那么干净。

按照过去政争时各方惯用的手段,想要将吕惠卿弄下来,直接从他的兄弟们身上入手,连番弹劾,一步步的将吕惠卿牵连进去。

而韩冈这边,想找错处也不难。

正是因为两边都有顾忌,也不想彻底撕破脸,才保证了很长一段时间以来,朝堂斗而不破的局面。

可一旦没有将那群饿狗好好的拴在牢笼里,让他们出来见了血,又会是一场大风波不说,新党与韩冈之间的关系也会彻底破裂,接下来的发展,就又是牛李党争和新旧党争的局面了。

李定在御史中丞的位置上坐了不短的时间,章惇的担心他也明白,而且韩冈那边他也不想招惹,但有件事他还是想问清楚。

“吉甫打算怎么做?他当真有把握?”

“当然!”章惇斩钉截铁。

等吕惠卿回来,新学有他为中坚,比起年纪老迈的王安石,他其实更合适成为新党旗帜。而且从这一年的情况来看,新党也的确到了该新老交替,让生气勃勃的吕惠卿取代王安石,这样才有希望压制住韩冈。

“好的,我明白了。”

李定再一次点头,比之前更加镇重,只是他还没有想通,为什么吕惠卿会对打赢辽国那么有把握?

章惇暗暗叹了一声。

吕吉甫的盘算,也许其他不明军事的朝臣不明白,但韩冈不可能想不明白。为何一直坐视不理,让自己陷入被动的局面?

…………………………

下榻的房间已经收拾好了,童贯也过来禀报了,可王安石、韩冈两翁婿还没有进去的打算。

“吕吉甫实在是太有把握了,不是吗?”韩冈依然在反问。

“……他在边地的时间不短了。”

“小婿可没他那个把握。”韩冈抬头看了看夜空,转眼就是年节,能看见银河,却看不见月亮,“就是让小婿来。”他顿了一下,“最多……也就是能让辽人再拿不到压岁钱罢了。”

王安石身子轻轻一震,然后仿佛什么事也没有的恢复了平静。

“是吗?”他说道。

韩冈微微笑了起来,“也就这点想法,岳父以为能瞒多久?”

王安石的声音低沉下来:“已经足够久了!”
