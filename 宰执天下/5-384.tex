\section{第34章 为慕升平拟休兵(16)}

乌鲁最近很轻松。

虽然他和他的部族在太谷城下伤亡惨重,但自从太原府返回代州之后,便没有了什么需要拼命的地方了。

应该是为了公平起见,所有参加了太谷之战,并且伤亡惨重的队伍,全都被安排在了代州城中驻扎。当宋军探马拿着神臂弓让许多头下军中的勇士有来无回,他和他的部族那时候正安安稳稳的守在代州城中的营地里。只是之后要选派精锐驻扎大小王庄的时候,乌鲁和他的部族也就因此不幸被选上了。

不过乌鲁也由此分到了五十领精铁甲,一五十具神臂弓,以及马弓、腰刀各五十。当他麾下的儿郎穿戴上这些武器甲胄之后,立刻改头换面,焕然一新,精气神都不一样了。

出征的儿郎都换上了新装备,仗着甲坚兵利而猖狂一时的宋人,也不得不放弃了骚扰攻势,开始回到他们惯用的手段上——修筑营垒,开挖沟渠,然后躲在营垒中,守在沟渠后。

几步跨上望楼,正守着这座木制望楼的两名族中子弟立刻向乌鲁行礼。他的部族军和几个小部族驻扎在小王庄营寨的东南角上,这里的方位,也归于他所掌握。

站在望楼之上,能清晰的看见寨墙外宽达四丈的道路。那是宋人修建的官道,不过现在奔驰着大辽的骑兵。

在道路对面就是大王庄。从庄子的规模上,大王庄比小王庄大了两倍有余,所以出战的主力也正是驻扎在其中。两座庄子靠着官道的一侧上,有着一排店铺。如果没有战争,这里应该是客流不绝的上好店铺,但现在只剩下火烧过后的断壁残垣。

而沿着道路向南方去,便是宋军在前线的营地。不过那座营地在三丈高的望楼上其实是看不见的,离了足有八里路,再好的眼睛也很难在这个距离看见宋人的营垒。只有高高在上的飞船加上千里镜,才能看得分明。

不过正是近午时分,一道道稀薄的炊烟出现在南方的远处,只是依稀能见,却也把敌军所在的位置给表示了出来。

宋军前几天都不断有军队抵达大小王庄对面的营寨。炊烟的数目在几天中增长了三五倍,但这两天就没有什么变化了。似乎对前线兵力的补充已经告一段落。

虽然为了节省马力,也是为了减少不必要的伤亡,除了阻截宋军骑兵的侵彻,辽军的探马并不会深入宋军的前沿防线之内。但在高高飘起的飞船上,很容易就将这段时间抵达前线的宋军数量都计算得七七八八。

至今为止,大小王庄对面的宋军总数不会超过一万。大辽的六千骑兵对垒宋军的近万兵马,看起来差不多是势均力敌的样子。不过只要有需要,主力驻扎在代州的辽军,远比忻口寨的宋军更容易扭转前线的实力对比。

这些宋军的驻地其实并不集中,而是分散在几个相邻的由废弃村庄改建的营垒中。几座营垒相距各有远近,只是都不超过五里。

不过据探马回报,其中只有大小王庄正当面、同样位于官道边的营垒,在村庄的旧址外围与大小王庄相反的位置上,扩建了一圈,使得占地面积比其余营垒大了数倍。这些村庄聚合在一起,其实就像是一座大营。居中是主营,而周边的村寨则是作为护卫的小营。

但这样的营地规模,看起来能安排下五六万兵马,而不是现在的万余人。可见宋军这是为之后主力进驻而准备。除了最近处的主营地,其他小寨在这个时候都显得空虚无比,好像都只有四五人驻扎其中。依照宋人的兵制,一个指挥而已。

“乌鲁。”老胡里改从下面爬了上来,看起来紧张的很,“你都回来了,大营那边怎么说?真的要攻打宋人的营寨吗?”

早上的时候,从大营那边传话说要讨论一下攻打宋军营地的事,乌鲁得令后也是匆匆赶了过去。传话的人说得语焉不详,让很多人担心了半天。

“没错。”乌鲁仍在低头看着下面的骑兵,是刚刚从宋人营地那边回来的,“耶律道宁说宋人的主力虽然还没到,但肯定是要来的。干脆趁这个机会攻破边上的一两个小寨,给宋人个下马威。也免得他们到了之后,能立刻出兵。”

“还好,还好。”老胡里改松了口气,脊背也弯了下来:“总比攻打宋人的主寨要好。什么时候出兵?我们打哪边?”他又抬起头来问道。

“不会出兵,谁也不打。”乌鲁转过身来,冲着老胡里改笑着摇头:“耶律道宁他这么说,可谁也没答腔,那可是宋人守的营地!当初宋军修寨子的时候没能阻止,现在再去打不就是自找苦吃吗?我们图鲁部的儿郎可不剩多少了,我可想着将他们一个不少的带回去。”

要想尽快攻下四五宋军把守的营垒,至少需要三倍到四倍的兵力。与此同时还要防备宋人的援军。想要在这样的情况下攻下宋军的营寨,哪里能有那么容易?要向代州请求援军,又未免太小题大作了。最重要的,在太谷县,大多数人已经吃足了攻城的苦头了。

只是听到根本就不用去攻击宋军的营寨,老胡里改却仍是忧心忡忡的样子,眉心的皱纹比额头上的还要深两分。

“老胡里改,你的担心是白担心啊。”乌鲁哈哈笑着用力拍打老胡里改单薄的肩膀,“你都没看到耶律道宁的脸色,变得跟烤猪一般。谁都不是笨蛋。难道宋人不知道兵力太少会引来我们的攻击吗?都只派了几人守寨子,分明就是陷阱。也不想想宋人的援军有多近?这就跟我们这里和代州一样啊。”

大小王庄最大的优势,就是与代州之间的距离。

大小王庄距离代州城只有四十里,快马一个时辰不到就能赶来。任何攻打大小王庄的行动,都等于是同时要击败尚在代州城的两万兵马,那里面可是还包括超过五千皮室军及宫分军的一等精锐。

何况驻扎在大小王庄之中近七千骑兵,有两千多具装甲骑,那是尚父殿下放在西京道的家底。其余的四千多兵马中,也有一千是属于萧十三麾下的西京皮室军。

这一部进驻前沿的兵马,不仅是战斗力远超同侪,身份也极为特殊。万一被宋军围困,代州那边必然会拼命来援。

同样的道理,宋人设立的那几处外围小寨,可都是距离主营最多也只有五六里,援军随时能赶过来。虽然说,可硬碰硬的战斗,至今也是在大小王庄驻扎的六千多人要尽量避免的。

“我们只要守在这里。如果宋军当真来攻,萧枢密就会率援军赶来,直接抄了宋军的后路。当年宋人的皇帝就是这么被我们打败的,如今也会一样。你还有什么好担心的呢?”

老胡里改看着口沫横飞的乌鲁,深知他并不是信任萧十三,只是这样的作战方略,就根本不需要乌鲁和图鲁部的部族军去冒险,当然能得到他全心全意的支持。

‘什么时候,我们连正当面的敌人都不敢去面对了?’老胡里改低声说道,却只有他自己能听到。
……………………

“什么时候我契丹儿郎连近在眼前的敌人都不敢面对了!”

隔着一条官道,数个时辰之后,在大王庄一角的望楼上,四十多岁的老将耶律道宁正满口酒气的长声叹息。

没人能回答他的问题,同在楼台上的两名亲信都低垂着脑袋。任凭耶律道宁拿着个酒葫芦喝得不知白天黑夜。

作为此处的主帅,耶律道宁他有义务去清理宋人的探马,并阻止宋军越过大小王庄去攻向代州。但他没有权力率领去冲击可能造成重大伤亡的宋军营寨。

除非能得到大部分下属的支持,否则他所下的命令连大帐都出不去。之前的失败,已经彻底毁掉了耶律道宁的威望,现在的他,在全军上下,是所有人嘲笑的对象。也只能躲到望台上喝着闷酒。

“畏敌如虎,这样怎么打得赢宋人?”耶律道宁惨笑着,一仰脖子又灌下了两大口烈酒。

“统领,你这话就错了!为什么要攻城拔寨?我们契丹骑兵从来都是阵列不战吧。连列阵的宋军都不打了,何况明知有强兵坚守的营垒?”

耶律道宁闻声回头,醉眼中只见一个熟悉的面孔从楼台下探了上来。是萧十三身边的亲信,北枢密院敞史萧思温。

“这不是萧敞史吗?怎么来了这里?”

“枢密和耶律相公听闻了早间之事,所以就让我来了。”

张孝杰早年得赐姓为耶律,实际上应该是耶律孝杰,不过在四帐、五院、六院的所有皇族们心中,可都只知道一个张孝杰。不过在公开场合的称呼中,还是没多少人会叫错。

“枢密的耳目还真是灵通。”耶律道宁悻悻然的咕哝了一句,然后道,“说吧,枢密打算怎么处置我?”
