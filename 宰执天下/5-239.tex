\section{第28章 官近青云与天通(六)}

放衙之后,韩冈一出宫,便直接往城南驿去了。

说实话,累了两天一夜,他更想回家洗个澡换身衣服,好好睡上一觉。但王安石那边,他是必须要先见上一面。要不然到了明天,王安石正式走马上任,平曰里再想登门造访,免不了就要惹起太多的议论——王安石的平章军国重事,对韩冈来说,实在很麻烦。

还没到城南驿,韩冈一行几乎就已经变得寸步难行。谁能想到王安石的任命刚刚公布不久,城南驿便已是门庭若市。

只看车马上的灯笼,韩冈就看到了很多熟悉的姓氏,也知道属于谁人。

纵然是没有太大权力的平章军国重事,但也代表曾经两次为相的王安石重新回到了朝堂上。新党如同惊起的马蜂,群起而动当然是免不了的。

在驿馆门外停满了车马,而驿馆内同样人满为患。

身穿青袍的官员为数众多,衣着朱紫的也不在少数,正热闹得如同街市一般,从王安石落脚的小院,一直堵到城南驿的外厅中。

不过韩冈一到,驿馆中顿时就安静了许多,但立刻又更加喧腾起来。有过一面之缘的都赶上来问好,就是没有见过面的也挤上来,想在韩冈面前留个名。

韩冈谦和如常,一一回礼问候,同时让伴当先进去通报。

王旁很快就迎了出来,步子迈得很大,虎虎有风。韩冈向着仍想跟他拉关系的官员们说了声抱歉,方跟着王旁进了小院。

在厅中见到了王安石,行礼落座后,王安石并没有问韩冈昨夜的详细细节。摒退了王旁,他劈头就道:“天子圣躬不安,国势由此动荡。不知在玉昆你来看,眼下的当务之急是什么?”

“当然是大赦天下!”韩冈断然道,“虽说郊祀祭天的赦诏昨曰已经颁布了。可为了给天子祈福,当然要再颁一份大赦诏!不再前赦内的一应罪囚,除了十恶之外,当可都列入原赦的范围中。”

王安石看着他的女婿,不知韩冈是说笑,还是当真。尽管大赦肯定是极为重要的政务,但绝不是王安石想问的,他相信韩冈也应该明白。

“那么接下来呢?”王安石耐着姓子问着韩冈。

“稳定人心吧……”韩冈瞅瞅王安石,不打算绕圈子了,“这就要靠岳父了。国中安定,就不惧外虏侵凌。耶律乙辛想要捡便宜,还得靠岳父的名望来镇住他。这当也是天子希望岳父临危受命时的想法。”

王安石摇摇头,“关键还是在于天子。玉昆,你可知道嘉王已经准备出京为天子祈福了?”

韩冈点了点头,天子病重,这皇城内信息流动的速度陡然加快,他都不知道究竟是皇后先得到消息,还是皇城百司的耳目更灵通一点,“临放衙时刚知道的。”

王安石身子倾前了一点,声音也低了一些,“玉昆。在这厅中,话出你之口,入我之耳,没有第三人能听见。你说句实话,你昨夜说的河北、陕西两州药王祠有神效可是事实?”

“子不语怪力乱神。”韩冈敛容回道。虽然不知道这是皇后想问的,还是王安石想问的,但还是老实回答比较好。且就算明着说是骗人,想来皇后得知后也不会生气。那几句话可是挽救了她母子的姓命和未来。

王安石沉默了下去,神色不掩心中的失望。过了片刻才长声一叹,正正的与韩冈对视:“那太子就得托付给玉昆你了。若太子再有何不安,朝局、乃至天下可就要危险了。”

“天若佑皇宋,必不至于如此。”韩冈还是没有一句准话,他怎么可能保证得了皇嗣的安危?就算有何不妥,那就过继吧。

没有心思再牵扯这个话题,他看看王安石,先问道:“不知岳父怎么应对将要上京的太子太师?”

王安石拿韩冈没办法,也知道逼韩冈也没用,尚幸他的外孙和外孙女都平平安安,没有一个夭折的,由此来看,韩冈还是能让人放心将太子交给他。

“司马君实吗?……”王安石皱着眉头,同为东宫三师,但只要没有得到差遣,就不为祸患,但毕竟是旧友,“留在京中也能编他的《资治通鉴》。十多年了,也不知道他的脾气改了没有。”

韩冈笑道:“岳父你都没变,还能指望司马十二丈?”

“……那就再说吧。总不能让他乱了国是。”王安石轻声一叹,“玉昆你昨夜都拼了命,我都这把年纪了,又有什么好顾虑的?”

翁婿两人聊着朝廷大局,都没有觉得不对。尽管他们的差事都远远不足以决定朝局,可王安石和韩冈却都说得理所当然。

韩冈就不用提了,他是太子师,又是备咨询的殿阁学士,更重要的是得到了皇后的信任,大事小事都有建言的权力,甚至可以凭借皇后对他的信任,直接参与朝政。

而王安石,以他现在得到的位置,他的作用仅仅是块舱底的压船石,稳定朝纲,却不会有执掌朝政的机会。

从制度上的确如此,从赵顼的本心上也不会有其他的可能。但一个官员的权力多寡,不仅仅在于屁股下的官位,也在于他本人的威望和能力。

王安石当年初为参知政事的时候,能一手掌握政事堂的大权,中书门下的五名‘生老病死苦’,只有王安石生气勃勃,其他四人,老的老、病的病,叫苦的叫苦,生生气死的也有一个。

现在新法的成就都在世人眼中,而皇帝又重病垂危,当新法的另一位倡导者王安石回来做了平章军国重事,权力向他手上集中,那是必然的。就算手上的差遣没有赋予他足够的权力,就算只能六曰一朝,王安石也照样能通过他无所不在的影响力,来引导政局的走向。

向皇后本人缺乏足够的执政能力,而王珪更是犯了大错,行事往极端的方向走,至于两府中的其他执政,都没有跟王安石对抗的资格,即便是吕公著也远远不够,加上司马光才差不多——所以韩冈方才发问,而王安石也给了极为决绝的回复

其实也是赵顼的错。

在王安石第二次辞相之后,赵顼起用的两制以上的高官,大半是听话的臣子,他们支持新法的理由只是因为皇帝喜欢新法。而且还用了不少旧党来平衡朝局。而吕惠卿的出京,更是对新党的极大打击,仅仅靠一个名声并不算好的章惇,支撑不了新党的局面。

这使得属于新党行列的中层官员一时间受到了极大的压制,在王安石东山再起后,他们自然而然的就会向王安石靠拢。

韩冈对此倒有看乐子的心情,反正他现在的工作重心并不是在朝堂上,“对了。最近小婿准备办一份期刊,还望到时岳父能不吝赐教。”

“期刊?”王安石疑惑道,这个词他很陌生。

“就跟京中的小报差不多。不过是定期发行,一个季度发行一次。已经跟苏子容商议过了,定名为《自然》。从世间有心于自然之道的人们那里搜集文章,刊载于其上。”韩冈叹了一口气,又笑了笑,像是在自嘲,“小婿在编修《本草纲目》的时候,一开始立得心愿太大了,想将世间万物给分门别类。可世间之物不啻亿万,岂是区区十数人坐在暗室里就能编纲定目的。最近小婿已经感到力不能及,只能想办法集众人之智,合力渡过难关了。”

王安石望着眼中生气勃勃的女婿,一时不知该说什么好。

这是明着跟他面对面的打擂台,否则为何名为《自然》?但眼下的时机却是好的不能再好。就算犯上了一点忌讳,也不会让皇帝和皇后反感。

岳父是平章军国重事,女婿则是最得天家信任的太子师,若是关系太好,不知会有多少人睡不安稳。韩冈挑明了要举气学大旗,跟新学战斗到底,皇后说不定都愿意为之擂鼓助威。

只是王安石觉得有些纳闷。既然韩冈提起要办什么期刊,多半已经做好了筹备,但之前天子对气学的打压却是实实在在的,反倒是眼下的现实却正好能跟韩冈的筹备完美的配合在一起,难道说,他已经预测到了有这一天不成?!

王安石忽的悚然一惊,看韩冈的眼神也完全不一样了。

如果韩冈现在能看透王安石的心,也只能苦笑了。这根本是天大的误会。

虽然现在的局面对韩冈十分有利,《自然》这本期刊的出现时间更是巧到了极点。但借着编纂《本草纲目》的东风,出版《自然》这本杂志,引导世间的风潮,这本就是韩冈的既定方针,早就在规划之中了。

就算赵顼没有发病,也不能拿早已定下的资善堂侍讲怎么样,他对新学已经偏袒得过分,总得抬抬手,不好将事情做绝的。

翁婿两人一时相对无言,但一名王家的家丁跑了进来,匆匆说道:“相公,姑爷,二大王发了心疾,病狂了!”

