\section{第34章 雨泽何日及(一)}

清晨的时候,韩冈已经活动过筋骨,浑身热气蒸腾,身上穿的一件短褂都被热汗湿透。紧贴身体的衣裳,将他棱角分明的身躯勾勒出来,越发的显得身强体壮。

这时候,太阳已经出来了,淡金色的晨辉洒遍白马县衙的后院。接过云娘递上来的汗巾,韩冈擦着汗,往院中特别辟出来的浴室去。不经意间,眼角的余光忽然发现院中的两株已经长出了叶子的腊梅上,有着星星点点的细小反光。他的脚步一停,转身走过去,定睛一看,就发现在两株腊梅花的枝叶上,有着一颗颗晶莹剔透的露珠。

在叶面上发现如同宝石一般的露珠,让韩冈大喜过往。天气干燥了八九个月,终于有点湿气了,前些天可都没有发现。再想想,这两天天上的确是多云偏阴。看起来旱情已经开始扭转,说不定过个几天就快要下雨了。

韩冈今天要去京中,看到了下雨的希望,出发前的心情也好了许多。

冲洗过身子,回到房中。昨晚云娘和周南就帮他整理好了行装,还有换洗的衣服,现在周南又将包裹打开,坐在床沿上,看看有没有什么缺漏。

素心领着两个小丫鬟端着今天的早餐进房来,一边张罗着,一边笑道:“可能快要下雨了。院中腊梅的叶子上今天可都是露水。”

周南低着头,拿着件内袍犹豫着该不该放进去,随口答道:“前些日子没在意,都忘了照顾院子里的花木。昨天才想起来,就让墨文去浇的水,多半是得了水后挂出来的。”

严素心哦了一声,韩冈也微微皱起了眉头,心中不免失望,可能是因为浇了水才有露的。

不过再想想,天气有变化倒是真的,虽然今天还是晴天,但天上还是有云层在,下雨的日子应该离着不算很远了。一旦下了雨,所有的指责就都可以丢到脑后去。

王旖也早早的起来了,后面的两个乳母抱着韩冈的一对儿女,一起走进来。一家人聚集一堂。

孕期进入第四月,王旖害喜的情况终于在某一天之后突然就停了下来,丰满起来的腰身上,能看出来有孕的迹象,行动也变得有点吃力起来。

“官人,现在已经已经转了任,是不是要从这里搬出去?”王旖坐下来,问着韩冈。

“不用,”韩冈摇头笑笑,捏了一下正在酣睡中儿子的小脸,“安心的住着就好了。过两天将外面的牌匾改了,这里就是府界提点的衙门。”

现在的白马县衙原来是滑州州衙,而旧日的白马县衙被封存着,原本有着改为寺庙或是道观的计划,韩冈也曾有将之改为文庙,将县学安置于内的想法。只是都没有来得及实施,现在正好可以让新任的白马知县搬回去。总不能让他这个府界提点住小房子,而知县住大院。

另外,衙门的搬迁千头万绪,另一位府界提点,确切点说,应该是叫做同提点——因为是武职的缘故,所以要加一个同字,以示要比文职低上半筹——暂时应该也不会搬到白马县来。而且武职出身的同僚,没有与自己相争的资格。只要他韩冈还在白马县中,这个院子完全可以安心的住下去。

陪着家人吃过饭,安顿下白马县中事务,韩冈便乘上驿马,与七八名随从直奔京城而去。

……………………

韩冈借着驿马一路飞奔,区区一百多里地,一个白天就走完。一行人抵达京城时,正好赶在城门关闭前。

入了城,韩冈并没有去相府拜见王安石,而是先去了宣德门登了记,等待入对,接着则是去城南驿馆安顿下来——进京等待入觐的官员,不方便访亲探友。如果是奉旨出外察访的使臣,回京后更是连家都不能回,必须等缴了旨之后才能回去。

不过韩冈不能去王安石府上,并不代表王安石那边不能派人来见他。遣了一名随从去相府通报,顺便在驿馆附近的一间清静酒楼定了一个包间。到了初更的时候,换了一身便服的王雱就走了进来。

久不相见,王雱很是热情。一进门,就上前拱手行礼,笑道:“恭喜玉昆了。”

韩冈摇头失笑:“若是清要之职,还当得起恭喜二字。如今的这个府界提点,却是吃苦受累的活计,小弟可不知喜从何来。”

王雱深深的看了韩冈两眼,不知他是真心话,还是在说笑。试探的说道:“现在开封府中,除了孙府尹,可就是轮到玉昆你了。他人都是先吃苦受累,才能步步高升,而玉昆你却是反过来了。”

“当初天子有意让司马君实提举二股河工役,不知吕公著是怎么说的?”

王雱笑容终于收敛了起来。

黄河自仁宗庆历四年后,多次决口,下游一段分出东流、北流分别入海,故而被称为二股河。到了熙宁元年,黄河再次决堤,天子赵顼有意将北流填塞,导水东流。司马光此前受命视察二股河情,回来后也发了不少议论,所以天子让其担任‘都大提举修二股工役’,自然是顺理成章。

但御史中丞吕公著却说,‘朝廷遣光相视董役,非所以褒崇近职、待遇儒臣也。’——让司马光去主持工役,这不是对待近职儒臣的道理。以吕公著的说法,儒臣有说话的权力,没有做实事的义务。

韩冈似乎是在抱怨,只是王雱口中绝不输人:“玉昆若是能为近职儒臣,即可远离此等繁事俗务。如今晋升府界提点,岂不是离着司马十二当年的职位更近了一步?”

韩冈哈哈一笑,“玩笑而已,元泽不必当真。”

“能者多劳。”王雱说着好听话,“现在也只有玉昆你能安抚下河北流民。”

“谬赞了,小弟可不敢当。”韩冈拱手一礼,并不当真。

王雱则定了定神,问韩冈道,“玉昆,不知现今白马县中的流民人数究竟有多少?”

“流民人数我这边不是天天上报吗?其中可没掺一点假。”韩冈说道,“到昨日,是六万四千四百余口。现在估计快要到六万八了。流民超过十万之前,小弟之前的准备尚能支撑。但若是过了十万,以白马一县之力,就无能为力了。”他神色转而变得严肃起来:“时间不多,所以小弟准备在七八天之内,将府界提点一职接手过来。”

………………

第二天清早,韩冈换了朝服,进宫参加朝会。不过他参加的并非每隔五日的百官大起居,只是由普通朝官日赴的常朝而已,天子并不露面,仅由宰相押班。对着空无一人的御榻行过礼,各自散去。

但韩冈没有离开,他已经得到通知,今天可以越次上殿。与其他同样等候入觐的朝官一起,守在阁门内,等着内殿重臣议事结束。

但等许久,不见宫中有传。一直等到快中午的时候,才有人来找他,不过并不是天子遣来的班直,而是王雱。

“出了什么事?”韩冈看着王雱的脸色不对,从阁门中出来后就立刻问着。

王雱双眉紧锁:“有人昨夜上书弹劾,今天天子就拿着那份弹章来质问家严。说方今大旱,民情忧惶,十九惧死,逃移南北。并说外敌轻肆,敢侮君国,皆由中外之臣,辅佐陛下不以道……”

这等口水弹章过去从来不少,韩冈惊讶于王雱的紧张,“上书是为谁人。韩稚圭?富彦国?还是文宽夫?”

王雱发狠道:“是监安上门的郑侠!他在奏章中还说白马县流民几近十万,为玉昆你承宰相之命而阻之,不得抵京以沐皇恩。”

韩冈听着倒没生气。御史们道听途说的事多了,文臣只凭谣传就写奏章的事也多,一个监门官说白马县流民如何如何,根本不算什么特别。但有一件事却让人很奇怪:“区区一介监门官,选人而已,他怎么将奏章直接递到天子案头上的?”

除了天子的特别要求,否则就算是朝官的奏章,也都是得由中书或是枢密院中转,更别说是选人这等偏鄙小官。若非有此定规,崇政殿早就给雪片般飞来的奏章给埋起来了。所以韩冈有点纳闷,郑侠的奏章是怎么给赵顼看到的,还是有黑手在后面。

“是马递!”韩冈闻声看过去,吕惠卿竟然也沉着脸走过来。

大宋皇宫在消息方面就是如同一座四面开洞的破房子,王安石还在殿上受着天子质问,而吕惠卿就已经打探到了消息:“郑侠日前上书中书无果,他便将奏章伪作边地急报,通过马递,从通进银台司直接发进了宫中。”

“就算如此,也不至于让天子深责,一个小小的监门官,他说的话又怎么让天子相信。”韩冈沉吟了一下,“安上门是南门,仲元上次回来还说,蔡河边的流民不过两千,现在应该已经在安置了吧?”

吕惠卿叹了一口气,“不仅仅是奏章,还有一幅流民图。”

