\section{第23章 谁言金疮必枉死(上)}

到得早,不如到得巧!

周宁并不知道韩冈在踏入库管衙门前,为什么会突然冒出这么莫名其妙的一句话来。他只知道从秦州到甘谷的为期四天旅程的最后一关,终于就在眼前。

周宁曾听说押运粮秣军资中最为头疼的,不是艰险曲折的道路,而是抵达目的地后接收资材的官吏。如果说这一路杀机四伏的行程,是死后黄泉路的话,那甘谷城的管库衙门就是黄泉底下的阎王殿,而监理库帐的管勾官——齐独眼便是坐在殿中的阎罗王。

扒皮抽筋齐独眼的凶名,秦州道上服差役的衙前无人不知,周宁相信韩三秀才肯定也听说过,那位王军将也是一样。要不然王军将也不会入城后就扯着韩三秀才走到一边说了好一阵,从两人那里模模糊糊传来的话,周宁听着,好像也是莫名其貌的——“到得早,不如到得巧。”——这一句。

在三十多名民伕中,只有周宁才在少年时开过蒙、读过书。他一向自视高人一等,头脑自认比其他民伕要高出一筹,可周宁还是想不通韩冈说的这句话究竟是个什么意思。

韩三秀才带着自己走入齐独眼的公厅时,没有半点犹豫,看起来比走亲戚还自然。但周宁跟在韩冈身后,想起齐独眼扒皮抽筋的名号,却是心惊胆颤,‘若是王军将在就好了。’

可惜王舜臣并不在。他在入城后跟韩冈说了几句,便与车队分道扬镳,往城中心去了。虽然是借口,但王舜臣身上的确有吴衍签发的公文要送去城衙。故而韩冈是独自则领着车队,抵达了城南的库区。

艰难的穿过了因捷报而变得拥挤不堪的街道,车队抵达库区之中。民伕们在衙门外看着车子,韩冈只点了周宁跟在身后,一起进了衙门里。周宁肚子里的一点墨水,被韩冈所看重,村塾的塾师并不是只教着学生们去读千字文和论语,算学也是开蒙时必学的科目。周宁能写会算,韩冈找他做个伴当,也有日后提拔任用的心意在。

位于库区边的库管衙门就是普通的一进院落,一座单独的公厅。于深夜中入城,照常理应该等到第二天才会被招进去。不过因为捷报的缘故,公厅中灯火通明,不知多少胥吏跑进跑出,忙个不停。一场恶战下来,赏赐肯定少不了,虽然大头要等到朝廷发下,但提前预支一部分,让参战的将士们快活一下,更是多少年来的惯例。只是这赏赐的多少,还得看着库中充裕与否。

甘谷城的军库管勾官齐独眼的大名,但凡来过甘谷或是即将抵达甘谷的民伕和衙前,无不是如雷贯耳。可韩冈和周宁见到齐隽的第一面,却正碰上了他与人打擂台的一场好戏。

一名三十上下的军官就跟齐隽面对面的对峙着,在灯火下,他左颊上杯盏大小的伤疤十分的显眼,而身上还有着血与火的味道。疤脸军官看起来很是心燥,一副火烧火燎的模样:“齐管勾,都监要的酒水不是五坛,是五十坛!总共两千弟兄,你就给个五坛,想让大伙儿喝掺酒的凉水不成?!”

齐隽叫着撞天屈,看他委屈的样子,完全没有半点扒皮抽筋的狠戾:“徐殿直,不是本官不给啊,库房你也看了,空荡荡得能跑死耗子,哪还有多的酒水。这些天,因着西贼攻甘谷,预定中的辎重车队一家都没到。巧妇难为无米炊,本官也没辙啊!四十五坛酒,谁能变得出来?!”

“这话你跟两千弟兄们说去!看他们答应不答应!”

疤脸军官瞪目怒骂,齐隽则苦笑摊手,他敢对衙前扒皮抽筋,却还不够资格在赤佬们身上吃肉喝血。碰着刚刚大胜归来的队伍,若不是真的没辙,他怎敢触这个霉头。

站在门外,韩冈和周宁一切看得尽在眼中。

韩冈低下头去,掩去唇边眼角绽出的笑意,他手上可是有着足以让得胜归来的两千将士满意的东西。他低声自言自语,“到得早,不如到得巧。”

周宁听到了,惊得瞪大了眼睛,难道韩三秀才早就料到了会有现在的这一幕?这未免也太……太……周宁不知该用什么词来形容韩冈洞烛内外的先见之明。他惊叹的看着韩冈的背影,‘难怪有人说他日后肯定少不了一个进士……’

韩冈轻轻咳嗽了一声,上前两步,不待通报便跨进了房中:“两位官人,在下有事容禀。”

“滚!这有你说话的份!?”疤脸军官旋风般的回头怒骂,心情正烦,竟然还有人敢燎他的眉毛。这一声惊雷般的暴喝让门外的周宁吓得连退了三五步,差点一屁股坐跌在地上,而离得更近的韩冈,却眼皮都没动上一下。

韩冈微笑着继续说了下去:“在下奉命押送犒军之酒水银绢,刚刚到得甘谷。总计酒水六十坛,银五百五十两,绢八百匹。还请齐管勾查验。”

“酒水?!”疤脸军官脸色变了,顿时转怒为喜,一把扯住韩冈,急叫道:“在哪里?在那里?快带俺去看看!”

韩冈歉然一笑:“还请殿直稍候,等齐管勾点验后自当交给殿直!”

“你是哪个县的?文书在何处?要点验的军资又在哪里?”韩冈的出现解了齐隽之困,可他不改平日声口,拖长声调便要在韩冈身上扒层皮下来。

韩冈还没回话,疤脸军官心中火烧火燎,一拳捶在了齐隽的桌案上,震散了一地的文书,破口大骂:“鸟你的‘县’!鸟你的‘文书’!鸟你的‘点验’!谁不知道你这贼鸟尽吃着衙前的肉,少扒点皮会死啊?!都监正等着发赏,你再拖着试试?!”

齐隽被溅了一脸口水,脸色阴沉得可怕。他是从九品的文官,拍着他桌子的徐疤脸却只是个正九品的右班殿直,是武臣!但在徐疤脸面前,他却硬不起来。很简单,齐隽他是进纳官,用钱买来的官身,虽然从官职上属于文资,但不会有一个士大夫出身的文官会将他视为同僚。莫说是一个正九品的武官,就是还没入品,只要占着一点理,便完全可以不给他半点面子,即便他齐隽在经略司有后台,也不会因着一点明显不占理的小事为他出头。

一阵微风卷入房中,灯火闪烁,映得房中忽明忽暗。房中三人的心情也如灯火一般,有明有暗。

韩冈谦恭着的站在一边,只有眼神中透着喜色。他挑起了头就已经足够了,不需要再煽风点火。大势如此,齐隽纵然有着将衙前扒皮抽筋一般的凶悍,却也不得不低头。

阴着脸,暗自发狠了一阵,齐隽在徐疤脸不耐烦的催促中,一把抢过韩冈手上的文书,看也不看就在最后面签名画押。又随手写了一张回执,盖上印,递给了徐疤脸:“短了少了,也别来找本官。”

他眼睛一转,又冷冷的盯了韩冈一眼。独眼中传出来的信息,韩冈确实收到了——走着瞧!——这是齐隽现在心里最想说的话。

韩冈对着齐隽抱拳行礼,姿态像是在道谢,挺秀的眉眼中却凝集着满不在乎的笑意。齐独眼怎么想他可不在乎,既然齐独眼已经怄一肚皮的怨气,那让他肚皮的怨气再多一点也无妨。

韩冈如今最不怕的就是得罪人,甘谷立城不过一载,齐独眼扒皮抽筋的大名已经遍传秦州。据韩冈在出发前打听到的传言,齐独眼跟陈举好得能穿一条裤子。既然跟陈举已是你死我活的关系,跟齐隽翻脸,也不会让自己的境况更为艰难。

他是押运的衙前,既然齐独眼已经签了回执,那就再管不到他韩冈的身上。何况陈举已经没几天好蹦跶了,韩冈不认为王韶会放过他。即是如此,作为同一条线上的蚂蚱,齐隽如何能独善其身?唯一可虑的是张守约会保着他,但看张守约派人过来催赏赐的态度,齐独眼很明显是经略司掺进来的沙子。得罪了他,张守约怕是乐见其成。

徐疤脸接过回执,转手递给韩冈,笑道:“张都监没了消息,这两日南面便没一队人马敢来甘谷。伏羌城的刘安到了安远就不肯再挪一步,反倒是你们这队转运银绢酒水的先来了。下次见到他,洒家要好好问问他,看他臊不臊。”

韩冈接过回执,小心的折起收好。他辛苦了这么些时日,也就是为了这薄薄的一张纸。

徐疤脸又拿起桌上的过关文书,看了一眼标注的时间,当即又惊叹道:“四天!四天就从秦州到了甘谷城,竟然一点都没耽搁!”

‘秦州!’齐隽正盘算着怎么把眼前这名走了大运的衙前煎皮拆骨,这时听着一惊,身子一下绷直了。泛着凶光的独眼死盯住韩冈的脸,这难道是陈举要对付的人?!

韩冈谦虚的笑了一笑,道:“将士们正等着这批军资,韩某自奉命北来,只恐走得慢,就压根没想过要拖延时间。至于打下甘谷……凭一万西贼也配?!”

“说得好!”徐疤脸大笑着拍了拍手,越看韩冈越是顺眼,口气也温和了许多,“对了,还没问过衙前的名讳?”

“韩冈!!!”

回答的不是韩冈本人,陈举派来甘谷联络齐隽的黎清,正站在门外。他张大了嘴,难以置信的看着在房中笑意盈盈的韩三秀才。

ps:陈举留下的最后一招也没用了,只要有胆子向前走,前面总是会有路的。

今天第二更,征集红票和收藏。

