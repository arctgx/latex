\section{第42章 潮至东崂触山回(上)}

赵煦张大了眼睛,惊讶的望着台陛下的蔡京。

竟然还有人敢于在殿上指斥那位韩学士?不知道母后最尊重和感激的就是他吗?!

赵煦还清楚地记得半年多前的夜晚,他的祖母指着那位韩学士大骂,却完全奈何不了他。事后,就有人说,多亏了韩学士,否则二叔便要得意了。

赵煦还记得他的父皇,也就是不久之前的事,在韩学士回京第一次上课后,就突然决定要上殿一起听课,还将王先生、程先生一并请了来,然后就又在殿上发病。再然后,自己就成了皇帝。

但……父皇真的有病吗?

王先生、程先生、韩学士,三人都是父皇礼聘来的,可见他们三人都是最受父皇看重的饱学之士。但其中王先生和程先生的年纪都不小了,只有韩学士最年轻。

很多人都说,韩学士是父皇留给自己的宰相。

可自己天天都要去拜见父皇,为什么父皇从来没说过?

“不用韩宣徽,难道用你不成?!”

赵煦只听到母后阴沉沉的声音从背后传来,顿时浑身一个激灵,他从来没见识过母后如此愤怒。赵煦强忍住回头看个究竟的念头,小小的身板在御座上重又坐得端端正正。

“不是不用。韩冈之才,是臣十倍,百倍,如何可以弃之不用,只是不能重用。这是为保全韩冈,也更是为了皇宋安危。”

“又是在说韩宣徽太年轻,善始难善终,可这不都已经辞了枢密副使!?”帘后的向皇后更加激愤,急促的喘了几口气,愤愤然的说道:“……吾算是明白了,你们这些措大,就是见不得人好!”

蔡京纹丝不动,毫无惧色,好像被皇后痛斥的并不是自己,“韩冈虽退,却继以苏颂。且韩冈又时常上殿,与闻朝政,建言军机,虽无枢密之名,却有枢使之实。”

“韩宣徽那是有累累大功在,说到兵事,朝堂上有几个人能比?不然问谁?蔡京你们这些只有嘴皮子最能耐的措大?!”

蔡确叹了口气,向皇后这是将殿上的文臣都骂了进去。但她骂得越凶,朝臣们对韩冈的意见就越大。这不是帮忙,这是扯后腿。看看对面,韩冈虽然低头看着笏板,但也能看到他是一脸的无奈。

半刻钟前,因赵挺之、蔡京抢了先手,而气得七窍生烟的的蔡确,现在看见韩冈的样子突然很想笑。

风水轮流转,韩玉昆你也有今天。

蔡确苦中作乐,而章惇都要气急败坏了。

蔡京居心叵测,临难上书的姿态摆出来,传出去,谁都要高看他一眼。

而且几句话的功夫,整件事都给他扯得偏了方向,已经不是什么关说法司、私酿酒水的闲事了。

之前章惇还有些迷糊,到了这时候,他哪里还能看不出来?虽不可能像韩冈一样,因先见而对蔡京极为重视,可蔡元长现在既然自己都跳了出来,他如何还会不清楚整件事的来龙去脉?

蔡京不是要将韩冈一锤子钉死,而是要踩韩冈上位,纵然接下来会一时失意。但出京时,免不了会有人出面相送:蔡君此出,极为光耀!

自家也成了踏脚石,还被泼了一身粪,章惇恨得磨牙,这鸟货,是要做范仲淹啊!

薛向用眼角的余光观察着身边的韩冈。韩冈虽是宣徽使,但他在朝堂上班次,是比照知枢密院事而来,也就是站在章惇之下,苏颂和薛向之上。就在身边,也觑得亲切。

韩冈看起来并没有被蔡京激怒,神色也很沉稳,可却是在轻声叹气,当不是为了蔡京,而是皇后。

皇后应该让韩冈自己出面与蔡京辩驳的,她本人只要最后做出评判就够了。不能处在公正的位置上,任何决断都会被人质疑。

为什么要异论相搅?就是要让臣子们在下面,皇帝才能维持高高在上的地位,哪有卷起袖子,赤膊上阵的道理?想到这里,薛向暗骂了自己一句,上面的不是皇帝,是皇后。

蔡京心中正得意。

皇后脱口而出的气话,看着是帮韩冈,却将韩冈反击的机会完全抹杀。

韩冈在殿上时,辞锋有多犀利,朝廷内外谁人不知,其实比他用兵还更擅长一点。蔡京最后选择韩冈为对手,也是冒着风险的,完全是赌上去了,为了在未来收获宰相之位,把自己现有的一切都赌上去了。

最坏的结果,就得还任故官,回到进入御史台之前的职位上!

这比受责出外监酒税还狠,监酒税只是让被弹劾的重臣看的,说明这位失败的御史已经受到了处分,但不用多久就会起复,而且能升得更快,因为他尽了本分。

而还任故官的惩罚,却是最悲惨的。依故事,台谏罪黜,皆有叙法。若还故官,即永不叙。台谏升迁罢免后的派遣例归中书直接注拟、取旨除授的。即使监酒税都一样。而还故官,便意味着该台谏官将不会再受到叙复重用。曾经进入台谏的资历,等于被注销了。从此只是一个普通的朝官,甚至还不如,泯泯众人也。

若是受到这样的处罚,除非曰后还有人一力相助,挽回局面,否则连就任知州的资格都没有,熬资历要熬到死。而曾经两次被踢出台谏、去监酒税的张商英,现在却已经做到知州了。

至于追毁出身以来文字,却绝不可能落到自己的头上。弹劾大臣,是御史的本分。即便太上皇后想要发落自己,韩冈、蔡确、章惇他们,都要拼命拦着。

幸而有了皇后现在的这几句,韩冈之后再有什么回击,都掀不起波澜来。不可能会有什么坏结果,甚至交州都不用去,可以直接去南方监酒税去了。

不用去瘴疠地冒险,又顺顺利利的走上了预定的路线,他哪里能不得意。

只是从表面上,完全看不出蔡京内心的愉悦,只见他依旧心平气和的对暴怒的太上皇后道,“臣旧年曾任官厚生司,亲眼看见百姓对韩宣徽如何顶礼膜拜。天下的药王祠,自种痘法出世后,香火大盛,庙中都有韩宣徽的金身。之后,臣亦曾使辽。亦亲眼得见辽人对韩宣徽敬畏。这一回,辽国枢密使萧十三与韩宣徽一番话后,辽国便立刻入寇高丽。非是韩宣徽与辽人勾结,而是辽人对韩宣徽敬畏如神,不敢违逆。”

韩冈的名声,曰后当然会危及天子。

“敢问蔡殿院如此之高,这就是韩冈的罪责?韩冈是不该将种痘法公诸于世?”

蔡京话声刚落,韩冈紧接着就问道。他不敢再等了,再迟一步,帘幕之后的那一位,就又要坑队友了。

“非是宣徽之罪,而是世人多愚,连累了宣徽。”蔡京言辞恳切,仿佛是真心为韩冈叹息,“但既然世情如此,为了皇宋基业,也不得不委屈一下宣徽。”

“原来如此。”韩冈点点头,表示了解。

“不仅仅如此。”蔡京温和而从容,再次面对向皇后:“韩冈文韬武略,世所罕有。格物之论,名震士林,天下士子,闻其言,无不悚然静听。领军在外,所向披靡,更能安抚卒伍,稳定军心。三军一知韩学士至,便皆尽安心。而在朝中,一遇军国大事,朝廷必急招其以备咨询。虽在两府之外,亦犹如两府中人。”

蔡京看了一眼蔡确,继续道:“且蔡确居相位,韩冈阴助之,章惇与韩冈更是相交莫逆。三人互为表里,同操朝政,曾布、张璪只能画押应诺,吕惠卿立有殊勋,却仍得留居外路。长此以往,陛下可得安?”

蔡确终于不能笑看韩冈的窘境了。蔡京这分明是在拉拢曾布、张璪,并示好并不在场的吕惠卿。

“心达而险、言伪而辩,危言耸听,无过于斯!”蔡确狞声说道。

蔡京反问:“相公以蔡京为少正卯乎?”

《荀子》中所载,孔子曾言:人之五恶,胜于盗窃者:一曰心达而险,二曰行辟而坚,三曰言伪而辩,四曰记丑而博,五曰顺非而泽。少正卯五项皆备,所以被孔子诛杀。蔡确指蔡京现在正是占了其中的第一和第三条,聪明而用心险恶,言辞虚伪却说得有理有据。

蔡确森然说道:“子产不毁乡校,但子产也曾诛邓析!”

蔡京毫不示弱的回应道,“相公可诛蔡京,可能诛尽天下正人?”

蔡确的威胁并没有意义。实际上,朝廷都不可能杀了蔡京。向皇后就是有这个念头,包括韩冈在内的朝臣们,都要出面阻止。但蔡确是要带出下面的话。

“欲陷君于不义,何可谓正人?”章惇一下就配合上了,冷然说道,“夫伤忠臣者有似于此也。夫无功不得民,则以其无功不得民伤之;有功得民,则又以其有功得民伤之。”

他知道向皇后听不懂这段话,更不会知道出处,又详细的解释,“忠臣往往为小人所间。若忠臣行事无功绩且不为民所赞,小人便会诋毁其人。若忠臣有功劳有人望,同样会被小人诋毁。人主若为其所惑,那就真的是让。比干、苌弘都是因此而死;箕子、商容只能出奔;周公、召公更是无辜受疑;而范蠡、伍子胥,也不得不背井离乡。”

这是《吕氏春秋》里面的一段话。说得是子产和邓析。

子产是春秋时郑国名相,执政二十余年,在中原四战之地,维系郑国不至为强邻所欺,而邓析同样是郑国的大夫,精于律讼,而且被后人视为名家的鼻祖。

有一次洧水涨水,郑国一个富人被淹死了,有人得到了他的尸体。富人的家人请求买回这具尸体。得到尸体的那个人索要高价。富人家里把这件事告诉邓析,请他拿个主意。邓析便道:“勿须担心,其他人不会买。”得到尸体的那人没办法了,也去找邓析,邓析则对他说:“放心,除你之外,他们在其他人那里买不到。”

一件事能够正说反说,都能说出道理,但用心却非是正道。当子产每次出台一项新的法令,邓析就会教人怎么从中钻空子。一次次的重复,最后子产无法容忍,直接诛杀了邓析。

所以《吕氏春秋》中就评价说,那等诽毁忠臣的小人跟邓析很相似。如果忠臣没有功绩、得不到世人的拥护,那些小人就会以此为理由来诋毁忠臣;如果忠臣能够顺利的建功立业,还得到了世人的赞许和尊敬,而小人们,就又会以这些功业和世人的尊敬来中伤忠臣,说他们威胁到了君主。

尽管在《左传》中,是另一位郑国国相诛杀了邓析,但原因和理由是一样的。

蔡京很了解《吕氏春秋》,更了解《荀子》《春秋》,但他并不在意,因为他知道,当他光明正大的将韩冈的威胁说出来后,朝臣们曰后都会开始利用这个理由来攻击韩冈。

所有人都知道韩冈太过年轻,地位与他的年纪不匹配,能力、名望更是让人忌惮,朝廷有充分的理由压制他。但每一个人都在等别人站出来,他们都畏惧韩冈的反扑,可蔡京不在意,他敢说,敢做,敢赌。

“韩冈才高、名重、望隆,得民心、军心、士心,以及,”蔡京抬眼看了一下上方,“……圣心!若其垂垂已老,蔡京绝不会多言。但韩冈如今刚过而立,已是进出两府,这让人如何不担心曰后其身登相位,把持朝政数十载?若如此,天子将居于何处?”

蔡京的质问震动朝堂,面对诛心之论,韩冈比蔡京还要平静。

“殿院弹劾韩冈,可是秉承他人之意?”

蔡京正气凛然:“此乃蔡京本心,忠于事,忠于君。忠于国。无非一个‘忠’字。”

“既然如此,那就很好解决了。”韩冈心平气和的对蔡京道:“只要殿院肯终身在京外为官,韩冈终生不入两府亦可。”

