\section{第21章 欲寻佳木归圣众(四)}

夏日的宫廷,若无人经过,便是寂静无声。

一只夏蝉刚刚飞到一株梧桐树上,才叫了两声,便有两名禁卫拿着杆子扑打上去,顿时就没了声息。

这是从宋守约做殿帅开始留下的习惯,近二十年下来,已成为宫中依循传承的故事。新太尉换了好些个,没有哪个改掉了这个惯例。

而这两年苗授担任殿前副都指挥使之后,又增派人去给宫中的大小树木下面铺上鹅卵石或是水泥。

因为在《自然》中,曾经刊载过有关蝉虫一生的论文:蝉虫在树枝上产卵,幼虫孵化后掉落地面,钻进地里吮吸树根汁液,长成之后便从地里爬上来,蜕壳羽化成虫,这已经成了很多人的常识。苗授这么做,正是防止地里的蝉虫爬出来吵闹宫廷。

王中正从宫外自家的府邸来到宫中,顿时就觉得耳畔清静了许多。

这就是故事。

宫中的故事每每可笑,以枢密使之尊,每当宫宴只能下去做陪客,这本是因为枢密使刚出现时,本为天子近侍所任,非是朝臣,更不可能与宰相东西并立,晚唐、五代,枢密院地位渐渐上升,立国之后,又逐渐为士大夫所控制,但这个陈规陋矩,直到熙宗登基之后才改过来。

王中正有时也在想,要是换作唐时,枢密使这个位子自家也能坐一坐。若是朝廷能按功劳授官,自家照样能进西府。可惜如今士大夫势大,即使在梦中,王中正都不敢想象自己能站在西班最前面的位置。如今的皇宋,阉人想要再做回枢密使,除非出一个商纣王、隋炀帝那样的昏君才有可能。

不过更多的感慨,因带御器械的身份而在太后身旁值守的王中正就没有了,一天下来,他心中更多的还是小心谨慎。

在家中,他是一家之主,妻妾儿孙们都要在他面前恭恭敬敬。但在宫里,在太后面前,他就是必须要守着规矩的家奴,即使节度使已近在眼前,也没有狂妄自大的本钱。

但王中正尽管年纪大了,身子骨也不怎么利索了,却也从来没有请假过一次。在太后面前卖力,得到的是情分,在太后看不见的地方卖力,不过是功劳、苦劳而已。

“王中正。”

一听到太后的声音,王中正立刻弯下了腰:“臣在。”

心中却忍不住在想,这是不是《自然》中所说的条件反射,虽然这个词怪异了点,但道理是一点不错。这两年,天下成千上万条胃穿孔的狗,都验证了这一点,‘可怜的狗。’

“韩相公今天的话都听到了。”

王中正闻言一惊,满脑子的胡思乱想顿时一扫而空,“……是,臣都听到了。”

“你是怎么想的?”

太后的问话,让王中正头疼起来。

韩冈早已经走了,现在太后正在回宫的路上。

接近一个时辰的觐见中,太后和韩冈聊了不少话——之所以用聊,是王中正根本不觉得这是君臣问对。

一开始的确说了一些有关沈括的任命,但随后话题便转到了沈括家中悍妻的身上。再之后,话题就更偏得离题万里了。

在整个觐见的过程中,王中正见证了韩冈是如何想方设法将跑偏的话题给带回去,话题又是怎么屡次被太后给带歪的。

现在太后问对着方才的一番‘问对’,到底有什么看法,王中正一时真不知道如何回答,少不得一推干净,“太后和相公的一番问对,干系天下,岂是微臣区区内侍能够妄作评判?”

“就是吾与命妇说话,也会说几句朝事,你更是拿朝廷俸禄,这些话有什么不能说的?……算了,你也为难。这个不好说,那你对韩相公怎么看?”

王中正顿时放心下来,不要回答那个问题就好了。议论太后与宰相的问对内容,这是明摆着的干政,是最要命的。

只是评价官员贤与不肖就简单了,这是天子近臣的本分。

“韩相公治学为贤人,治国为能臣,世所罕匹。”

不过议论在位的宰相短长,终究是不妥,王中正还是用了一个世间流传的比较保守的称赞。太后听政多年,问出这种话来,岂能没有用意?保守点总不会有坏处。

“那跟之前的韩相公比起来如何?”

“是安阳的韩相公,还是灵寿的韩相公?”

王中正一边用问题来拖延时间,一边想着要如何避免开罪韩冈,又能让太后满意。

“两个都有。”

朝堂上,韩姓的相公一直不缺。总是去一韩相公,又来一韩相公。王中正在宫中服侍多年,几位韩相公都打过交道。

韩绛总是对宦官不假辞色,王中正每次见到他,总能感觉到平添几分寒意。当初韩绛领军要收复横山,他王中正奉手诏去延州体量军事,刚到延州便被打发去了前线,要不是韩冈恰好在罗兀城中,保不准这条小命就要交代在党项人刀下。

但其他宰执,无论是同姓的韩琦,还是富弼、文彦博,对他们这些内侍就算见面带着笑,也是为了探听宫中消息,讨好天子罢了。

而韩冈,王中正没见过他刻意勾结宫中得宠的内侍,同时韩冈也并没有将阉人视同异类——有太多重臣可以用来作对比,态度上的差异十分明显。只是相交快有二十年,王中正却还是看不明白韩冈这个人。

韩冈本身就是当世有数的学者、大儒,任官多年更是将天南地北都走遍,见过的人和事无数,见识远非困居宫中的妇人能比,但韩冈对太后的选择一直保持尊重,基本上没有独断独行的情况。人事安排,如果自己的想法,都会尽量说服太后,如果不成功,便干脆放弃。如果有必要,过一阵子,他会再来劝谏。比如当初想让李南公担任三司使,太后最后没有同意,韩冈便没有再坚持。

太后对韩冈持之以恒的信任,是一直以来韩冈的态度所造成的。若韩冈靠着当初的功劳便骄横跋扈,一点情分早就消磨殆尽了。但韩冈能一直这么做下来,完全不像是对太后的敬畏,反倒像是自个儿定出一个规矩后,便按照规矩行事。

有韩冈在这边,三位韩相公的高下其实也不必多说了,可王中正总不能就这么简单的得罪人。

“安阳与灵寿的两位相公,皆是治世能臣,朝中若有变,皆能以天下相托付。小韩相公亦是堪为国家柱石,不让两位相公专美于前,更是出将入相,文武皆能,仿佛古之贤臣,今人不能及。”

韩琦是两朝顾命、定策元勋,而韩绛也是主持先帝内禅时的首相,同样是有定策之功。以天下相托付的评语,两人都当得起。不过王中正只提治世,避而不谈武勋,当然是有所褒贬。

王中正的评语,没有得到太后的反应。跟在太后身后,看不见她的表情,王中正也只能安静的等着。

走了几步,他才听见前面传来的声音:“多亏了有三位韩相公先后秉政,英宗、先帝还有官家才能安居宫中。”

“国有贤良,是祖宗的福佑。”王中正立刻恭声附和。

“最近也是事情多,要是朝中多有几个能如三位韩相公的臣子,吾也能轻松一些了。”

王中正低低的应了一声。

太后当然累,近年来,天下太平无事,朝中又有贤相主持,向太后也渐渐变得怠政。隔三差五就要辍朝,每天只到内东门小殿坐上一坐。猛然间宫中最近一下子不太平起来,多少事压身,习惯了每天处置几桩事的太后,肯定习惯不了。

太皇太后继半年前一次重病,尚未康复,近日又再次垂危,十几名太医会诊,都说太皇太后没多少时间了。而齐鲁大长公主,因为忧伤过度,侍亲劳累,突发恶疾,短短几日内就重病不起。皇帝的祖母和姑母病重垂危,或许再过几日,这一家子除了皇帝之外,就剩下一个老三和他的几个儿女了。

大长公主就算了,向太后对太皇太后只恨其不能早死,只是事到临头,该有的礼数一点也不能缺。太后每日照样得去探问,然后板着脸回来。等到太皇太后薨逝,更是要平添多少事。

“天下安危,全在太后身上。近日太后劳累过甚,当好生调养。”

“还要怎么调养?太医局中,能跟华佗一般开膛破肚来治病救人的医官都有好几个,有他们的药方子,还要怎么调养?”

王中正道:“臣最近在服用蜂王浆,半年下来,只觉得精神旺健,身上的一些老病也没了。《神农本草经》中说蜂蜜安五脏,益气补中,止痛解毒,除百病,和百药,久服则轻身延年。但蜜蜂幼虫用蜂蜜喂养只能变成工蜂,而吃了蜂王浆,才能变成蜂王。即可知蜂王浆的滋养之力远甚于蜂蜜。”

“这个吾日常也在吃。”向太后点点头,问:“卿家也看《九域游记》?”

随着《自然》和《九域游记》的流传,曾经在两本书中出现的如蜂王浆、羊初乳、冬虫夏草之类的补品就成了世人眼中的滋补佳品,如今甚至成了贡品,向太后也常年服用,只是蜂王浆的数量少,又不耐存储,无法赏赐臣子,只能在宫中分享。

“九域虽然是小说家言,但里面都是格物致知的道理,无论天文、地理,还是兵法,都是正论。医药也同样如此,所以臣时常翻看。”

“韩相公说话一向是有道理的。”

王中正轻笑道,“只是韩相公不肯认。”

“一国宰辅,分心去写小说家言的确不合适。也怪那些拗性子的,不然何至于如此。”

王中正唯唯,总不能附和太后去骂王安石。

“王中正!”太后的声音忽然严肃起来。

王中正早有心理准备,躬身道:“臣在。”

“韩相公一心想要修轨道,觉得沈括这件差事办得好,便想让他继续办下去。只是沈括一向没什么好名声,家里又是有名的不靖,总有人要说话。你就去看看沈括修的轨道如何,如果当真差事办得不错,只要他能过廷推,吾就准了。”

“臣——遵旨。”
