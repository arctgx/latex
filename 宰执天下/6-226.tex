\section{第29章 雏龙初成觅花信(中)}

再是早熟,赵煦也没脱离小孩的水平,他对心情的掩饰,在韩冈眼就跟笑话一样。

韩冈觉得赵煦的确是把话听进去了,而且肯定会铭记在心。

只不过到底是记恨还是记仇,就得另说了。反正不会是作为指导日后行事的箴言,从而谨记在心。

身为臣,在面对犯错的皇帝时,不是诚惶诚恐的劝诫,而是当成小孩一般的训斥,落在皇帝耳朵里,当然不听。小皇帝又是处在叛逆的年纪,能听得进去那才叫有鬼。

但韩冈并不觉得自己的话有哪里不对,听不进去,就是赵煦自己的问题。

要是自己的儿,可就不是讲道理这么简单了。韩冈虽没体罚过自己的孩,但王旖却不会手软。此外韩冈也会罚孩写上十张大字,抄上一卷书,或是做上一百道应用题之类的惩罚,韩冈的儿女们,除了最小的几个,其他可都被罚过。

多半也是看出了小皇帝心实际的态度,向太后在旁告诫道,“官家,相公说的话当谨记在心。”

赵煦一幅老实听话的模样,“孩儿明白,娘娘放心。”

向太后叹了一声,走了过来。手轻撑在床褥上,坐了下来,“哥,你这个年纪,还不到近女色的年纪。相公方才也说了,官家你年纪太小,还不到时候。娘娘也罢,相公也罢,包括天下臣民,其实都盼着官家能早日为天家开枝散,但要是现在就弄坏了身,日后怎么生儿育女,难道你想让你父皇绝嗣不成?!”

赵煦的年纪的确小,熙宁十年出生的他,如今勉强可算是十二岁。这个年纪就开了荤,在富贵人家都不算什么稀罕事,多少富贵人家的弟,很多都是在这个岁数前后,从贴身侍女身上长大成人的。可说出去还是难免人言,尤其是赵煦的身骨还不好。

向太后说得自己情绪激荡,最后眼圈都红了,带着明显的鼻音。

赵煦的眼睛也红了,哽咽道:“娘娘,孩儿知错了。”

向太后拿着手巾擦着眼角,摸着赵煦的头,“官家知错就好。”

不,没有认错。

赵煦的神色可没有半点认错,伪装出来的表情,瞒不过冷眼旁观的韩冈,里面透着太多的不耐烦。

偏见也好,经验也好,反正韩冈都没看出赵煦有认错的想法。尤其是在向太后说她正盼着赵煦能开枝散,更是一个显而易见的怒意从他脸上闪过,正好被韩冈捕捉到。

难不成赵煦已经听说了那个太后和朝臣在等他生下皇,便让其退位为太上皇的谣言?

这倒不能算是谣言,考虑过这么做的人很多,也包括韩冈一个。但若是在明知会造成自己退位的情况下,赵煦还敢亲近女色,这可真是一点自控力都没有了,还是说,压力大到只有用这种事来发泄?

不过是压力的问题,韩冈可没有多余的同情心给皇帝。

“陛下。”

韩冈的称呼,让两位至尊同时转过头来。韩冈冲太后轻轻一颔首,然后对赵煦道,“有过能改,善莫大焉。陛下能自知其失,臣等不胜欣慰。但太皇太后上仙不久,齐缞之期未结,陛下虽是天,不受此事约束,可终究难免人言。”

赵煦脸色顿时为之一变,整个人都抖了一下,早熟的他,自是不会误解韩冈这番话的用意。后门处的珠帘,也突兀的晃动了一下,

韩冈的言辞直如威胁,正如他所说,这件事最重要的一点,是时间不对。

在阻挠了英宗皇帝亲近嫔妃之后,高滔滔这一次又出手了。

在高太皇去世尚不满一年的情况下,作为嫡长孙的赵煦,其实并不方便亲近女色。齐缞之期,孝贤孙们不把自己弄得形销骨立,反而弄女人弄得发了病,若是有人告不孝,绝对一告一个准。

尽管皇帝守父母之丧,都是以日计月,一个月不到就除服。之后日日欢歌、生儿育女,也不会算不孝,最多会有些闲言碎语而已。只是小皇帝之前曾有过大不孝的行径,现在又来了一次,即便可以通过天的身份避过罪名,可在道理上,还是避免不了不孝之讥。

韩冈倒是无意拿什么不孝的罪名去痛责赵煦,这件事在他看来实在是不值得一提,毕竟从名义上,赵煦不必去守上一年孝,本就是皇帝的特权。更何况,韩冈也没听说过熙宗皇帝当年登基之后,为他的父亲英宗憋上三年。既然有先例在,韩冈自不会多说。

此外,他也没在太后那边,看到她对赵煦的不满,只是恨其不成材。

但等到这件事传到外界,可就没有几个像韩冈一样好说话了。赵煦过去做出的那些怀念先父的举动,立刻就会被批评是惺惺作态——就算祖母再怎么不慈不仁,做孙的还在丧期之内,便沉浸在女色之,可就违背了儒家大义,纲常人伦。

赵煦显而易见的乱了阵脚,过了一阵,艰难的抬起头,咬着下唇,“相公,这不是朕要做的,只是……只是一时受奸人蒙蔽。”

向太后到底是阅历差些,被赵煦的小伎俩诓得信以为真,“吾也知道这不是官家你的错,若不是郝随这一等人坏了心肠,官家如何会病成这幅模样?”

看着赵煦拙劣的表演,韩冈一幅欣慰的神色——好歹认错了,表面上的回应还是要给的。

只是这么简单就把身边的人给抛弃了,缺乏足够的担当,虽然还是小孩,情有可原,但既然坐在皇帝的这个位置上,一切的评价可就跟年龄无关了。

不过赵煦的这番推托之词,还是给了他一个机会。

“太后陛下说得正是,若非皇帝身边无人匡正,反而诱使陛下纵情欢娱、不惜己身,绝不会有今日之病。身为近侍,却不能匡正陛下,身为宫人,却致使御体违和。此二等人,行迹昭彰,当如何处置,臣请陛下决断。”

赵煦惊得差点就从床榻起来,慌忙对太后道:“还请娘娘处断。”

“不。”向太后摇头,“官家你身边的人,还是你自己来处置最好……”

赵煦猛抬头,先看太后,又盯着韩冈,然后在韩冈平静的眼神,移开了视线。

赵煦的容貌还是如孩一般,泛着青白,在灯光下,显得很不健康。在赵煦唇角,则已经可以看见绒绒的胡须,喉结也有了点形状,已经开始脱离了小孩的身份。

“逐出宫外……”赵煦嗫嗫嚅嚅,偷眼看韩冈,看见韩冈面无表情,又连忙改口,“不,赐死,尽数赐死!”

“官家!”

向太后忍不住一下叫出声来。

就是旁边的一些个宫人、内侍,也都被吓到了。自真宗仁宗开始,宋室对宫人从无如此苛刻,几十条人命,说杀就杀了。

向太后没想到赵煦会冒出这一句,“陛下,可是真心如此处置?”

赵煦偷眼看了看韩冈,点头道,“是!”

向太后无奈的抬头看韩冈,“相公?”

她治政一向宽和,当年宫变的一干主角,纵使是韩冈等宰辅有意宽纵,没有她的首肯,也不可能让曾布、薛向和苏轼逃出生天。

熙宗年间,每年天下大辟【死刑】人数时多时少,多时超过三千,少的时候、除去几次因南郊大赦而只有三五百的特例,其他也都在千人以上,但自从向太后垂帘之后,大辟人数陡降为一百两百,都没有超过三百的,除了十恶和谋杀重罪,几乎都没有人犯被处死的例,全都发配边境去实边了。

前两年,在韩冈的主持下,元佑编敕新成——相对于宽泛且多不合时宜的刑统,编敕才是断案时更多采用的法律条——其对刑罚条款进行了大幅修改,大辟条减二十一,流放的刑条则增加一百一十,死、徒、杖、笞诸刑减少的条款,全都加到流放上去了。

编敕一出,世人皆赞太后之仁。而太后,也更加清楚的了解到韩冈对死刑的慎重。

赵煦的决绝,当然让向太后很不喜欢,可之前已经说了让赵煦自己决定,又不方便改口,她也只能求助于韩冈。

韩冈皱起眉,“陛下,用法不正则失人心,臣请陛下慎思之!”

赵煦猛抬头,青白的脸上泛起两团红晕,是愤怒造成的结果。

“那就请相公说该如何判!”赵顼冷着脸,硬邦邦的说道。

“陛下依法决断便可。”韩冈道。

赵煦却强硬的坚持,“请相公定夺!”

向太后在旁看得眉头直皱,赵煦对宰相缺乏足够的尊重,韩冈的一片苦心都给他浪费了。

韩冈没搭腔,平静的看着赵煦。

赵煦终究是心虚,一开始还能仗着怒气反瞪回来,可被韩冈淡然的眼神盯了一阵后,满腔的怒火被冷水泼得干干净净,再抵不过这压力,扭开了头去,气势也弱了下来,“请相公为朕解惑。”

韩冈叹了一口气,“郝随诸内侍,不任其职,可发配安西军前听用。至于宫人,未得陛下恩宠者,亦发配安西,配与有功将士为妻。至于曾得陛下恩宠者,臣请陛下依仁宗时故事,先出宫别居以养,若有喜讯,也方便将其召回宫待产。”

“相公所言……”向太后本欲点头,但转念一想,便转对赵煦道:“官家,你看如何?”
