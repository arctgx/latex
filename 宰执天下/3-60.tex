\section{第22章 明道华觜崖(三)}

“这下闹得大了。”曾布领旨之后,不禁叹起。他知道,琼林苑上的发生的事,肯定会传进天子的耳目中。但绝没想到会如此之快,而皇帝的反应也是让人出乎意料。

“不仅仅天子要来,你看对面。”吕惠卿板着脸指着北方,“消息好像已经传出去了。”

华觜冈北面,隔着小湖,就是从新郑门出来的通衢大道。本来这片小湖就是阻隔,所以外面并没有围墙遮拦。从道路上,就可以看到琼林苑的内部。现在在对面的湖岸上,不知挤了多少百姓,粗粗一看,竟然数以千计。

“怎么会有这么多人?!”曾布惊问着。

“因为琼林宴啊!”

东京百万军民都在关注的琼林宴,一篇好诗出来,都能惊动全城,何况这一次的赌赛?转眼就被传出去了,现在几千上万的人都在湖对面,等着看华觜崖上丢石头。

“不论结果如何,输的人这辈子都要成笑柄了。”曾布眼神深沉,盯着韩冈的侧脸。

高挺的鼻梁下,略薄的双唇紧抿着,显得坚毅无比。韩冈大概也是明白,他这是以自己的一生为赌注。

‘……就看他能不能赌赢了。’曾布想着。

……………………

在等着天子的时候,楼下、湖边的进士们早就开赌了。

众进士中自然是以押杨绘的为多,只有寥寥数人押着韩冈这个冷门。前十名中,有几个自重的没参赌,但参赌的都是押着杨绘,唯有第八名的留光宇与众不同。

留光宇的友人劝着他:“元章,你看看慕容武,他是韩冈同窗兼知交,你看他脸色,有半分胜券在握的样子吗?同时张横渠的学生,韩冈知道的,他怎么会不知道?”

留光宇却是坚持己见,“小弟倒是觉得韩玉昆说的有理,这冷门押着也不错。”

心中却是冷笑,这里都是些蠢人,还看人脸色做什么?有这个时间,自己拿着轻一点、重一点的东西试一试就知道了。

属于官中的筷子和碗不便乱丢,有宴上失仪之忧——就像韩冈和杨绘争辩,自始至终也都是笑眯眯的,谁也没有争得脸红脖子粗。更别说捶桌子砸碗,以增气势。

但留光宇用着腰带带钩和铜钱试过了,试了几次,都是同时落地。

关于这一点,他可没有与人分享的意思,说不得自己一人独赢,这脸面就涨起来了。到时候,看看那些赌杨绘的还有什么好说的?

练亨甫,今科的第六名。他瞥了洋洋自得的留光宇一眼,也是在冷笑,‘这胖子,难道以为就他一人做了验证?’

下落的高度只有一人高,轻重也差不了太远的情况下,就算同时落下,也不能证明什么,何况练亨甫几次试过之后,还是觉得有一点微妙的区别。

这种情况,观人是最不会有问题的。韩冈的说法本于张载的理论,但慕容武却半点不知,既然如此,怎么可能让人能相信韩冈?

“等着丢脸吧……”练亨甫不知是对谁在说。

……………………

天子的车驾很快就到了,几百名班直、内侍随行。只听着一片山呼万岁之声,从新郑门传了过来。

曾布、吕惠卿、杨绘还有韩冈等上百名琼林宴中人,都一股脑的去了琼林苑大门处相迎。

赵顼的车驾一直驶到华觜冈下。

等韩冈等人受诏到了楼台的最高层,赵顼从外廊处回过头来:

“朕只是想看上一看这赌赛的结果如何,不必耽搁时间了,先试了再说。”

韩冈和杨绘在琼林苑上闹出的这一通,赵顼听了之后先是有些恼怒。琼林宴上的赌赛,从来都是赌酒、赌诗、射覆、投壶,现在竟然比起了丢石头。

但听了来龙去脉之后,他立时就明白,今日一事虽是杨绘先行挑起,但却中韩冈下怀,他是借势要将张载推入经义局中。

但赵顼怎么想,都觉得一斤的铁球怎么会跟十斤的同时落地,怎么想都不可能。只是他将砚台和笔一丢,好像是差不多同时落地。

从道理上想不通,从实验上却是能证明,究竟结果如何,赵顼起了几分兴趣,干脆就来琼林苑走上一遭。

赵顼的命令干脆无比。在天子的注视下,两名小吏战战兢兢的一个捧起石头,一个拿起秤砣,然后将秤砣放在堵门石上。

“这是为何?”赵顼奇怪的问道,难道不是两个分开来一起放手吗?

“这是为了防止放手的时机前后有差别,最后影响到结果。”韩冈向天子解释着,“就算是一个人来丢,时机上还是会有些微差别。只有现在这般,才能免除。”

“这样不会有何影响?”

“不会。”韩冈摇头,“秤砣并没有和堵门石绑起来,是分来开的。如果秤砣比堵门石落下要慢,当然在后面会拉开距离,一前一后入水——就像一马一人前后靠在一起站着,可一旦跑起来,距离就会渐渐拉开。如果一样快,更不会有问题,可以看到石头和秤砣始终贴在一起。除非是一斤重的秤砣,坠速比三十斤的堵门石要快,在后面推着石块,这样才会有影响。”

赵顼摇摇头,这当然不可能。

这就是韩冈对实验的设计,要不然出问题的可能性就太大了。五十米左右的高度,差不多四秒。上面松手的时间有了零点几秒的延误,到落水时就是七八米的差距。那可就是自己给自己打脸,韩冈如何会去做?!

而且这个方法另外还有一个优点,就是将秤砣受到的风阻给挡住了,受力小的秤砣会一直黏在石块上,不会出半点意外。

韩冈如此提议实验的步骤,解释了两句后,连杨绘都没法再反对。

若是反过来,秤砣在下,堵门石在上,杨绘肯定要反对到底。但现在秤砣放在堵门石之上,既没有绑着,也没有粘着,杨绘若是反对,反而会让他显得心虚,也难以说出个道理来。

总不能当着天子和这么多同僚,以及新科进士面前,说什么孙思邈嫡传的法术。那他可就要成为朝中的笑柄了。何况也不用怕什么,方才上楼来的时候,林深河可是对他低声说了句一切放心。

所有人都注意力重新集中在托着石头和秤砣的小吏身上。只见他将手颤颤巍巍的探出栏杆,双手一放,石头和秤砣嗖的就直落而下。

赵顼立刻伏在栏杆上向下望去。

咚的一声响,就见着水花掀起了老高。

先是下面一片喧哗,嘈嘈的听不清楚。然后留在下面的童贯跑了上来,对着天子和众官道:“同时!同时!的确是同时落下来!”

结果一出,所有人的眼睛都看向杨绘。

不得不说,杨绘有着国之重臣的表现,都已经输了,但脸色只是略白而已。

而韩冈也知道,这个实验真要较真起来,却还有些说道,总有人嘴硬着。

只听杨绘道:“此必是韩冈以术法蒙蔽圣聪,不然为何方才一定要将秤砣放在石块上?还请陛下下旨,将两物分开来重新再试一次。”

韩冈冷笑,他就知道有人输不起。转头对两名小吏道:“请两位将手摊开。”

杨绘和林深河脸色大变,但在天子面前,他们也只能看着两名小吏摊开手,上面还沾着血迹。

赵顼瞳孔一缩,沉声问道:“这是什么?!”

林深河脸色苍白,两个琼林苑提举也过来厉声追问,“这是什么!?”

“不知是黑狗血还是公鸡血?都抹了血,还有什么术法?!”韩冈哈哈笑了,他不理杨绘,转身对赵顼道:“其实这个实验,臣从来没有做过,也根本不需要做,只需从道理上想一下就够了!”

“此话怎讲?!”赵顼惊讶的问道,一众官员也是骚然。

既然如此,何必多费手段?!

杨绘绝然不信,但韩冈胸有成竹,微笑中充满了自信。

既然前面说是这是理,自然有通过逻辑方法进行证明的手段。初中物理中的内容,韩冈又怎么可能会忘记?

现在杨绘反应过来,要换一种实验方法,韩冈是绝不可能答应。不管用什么实验,都会有误差。理想化的实验,也只会出在理想中。真的将堵门石和秤砣分开来丢下去,各种因素造成的误差肯定少不了,几乎不可能同时落地。

必须用理论来给杨绘最后一击!

“有一辆快车,一天能从东京驶到洛阳。还有一辆慢车,要三天才能从东京走到洛阳。”韩冈双眼一扫,所有人都在聚精会神的听着,“试问如果将快车和慢车用绳子绑在一起的话,情况如何?!”

“当然是快车拉着慢车走!”赵顼立刻道。

“陛下圣明!”韩冈赞了一句,道:“不管怎么说,肯定都是比快车要慢,比慢车要快!”

赵顼点头:“自是这个道理!”

天子点头首肯,杨绘想了一阵,也是点头。周围人众都没有反对声,这个道理哪还有错的?

韩冈笑了:“同样的道理。依照杨学士的说法,越重的下落速度越快,越轻的则越慢。那么,如果将轻物重物绑起来,就是将堵门石和秤砣用绳子绑起来丢下去,那落速就应该是比堵门石要慢,比秤砣要快。是不是这个道理?”

韩冈问着杨绘,吕惠卿则在一边轻轻一击掌,恍然自语:“原来如此。”

而杨绘则迟迟不敢答,他知道韩冈的话中必有陷阱,但他左想右想却想不出陷阱在哪里。等不及的赵顼帮他回答了:“正是如此……但这又如何?”

韩冈一下提高了声调,厉声质问:“所以臣只是想问一下。既然杨学士说重物要比轻物落得快,那么堵门石和秤砣绑起来后有三十一斤重,为什么会比三十斤的堵门石还要慢?……应该是快呀!”

寂静无声。

的确,应该是快啊……

看着杨绘的脸色惨白了下去,韩冈冷笑不已。

先以实据为验,再以推理证之。试问,谁能驳得了?!

