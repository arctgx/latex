\section{第十章 千秋邈矣变新腔(22)}

“谁是状元?”

文彦博停下了脚步。.

“宗泽。是太后钦点。”

文及甫看了看手中厚厚一叠信纸,然后抬头说道。

文彦博沉默了片刻,才又说着:“……似乎听说过此人。”

“去岁他在京师两家快报上,化名评论河东战事,很是出名。”

“哦。是哪里人?”

“他是浙人,婺州义乌的。”

“义乌……考卷呢?七哥有抄来吗?”

文彦博八子,只有文及甫在家侍奉老父,其余皆在外任官,光是在京中的就有两人,只是地位都不高,也没有什么实权。

“七哥附在信上发回来了。”

脚下是一座两尺来宽的小桥,文彦博看过宗泽的文章之后,就沉默的低头看着桥下淙淙溪水。

溪水清澈,溪底的白石青藻清晰可辨,一尾红鲤打了个水花,追着几只小虾从桥下游了过去。

观鱼半晌,待鱼儿游远,文彦博方抬起头,“义乌虽在江左,但多山多矿,民风悍健,又淳朴至孝,近于北风,与南方之人大不相同。”

“大人说得是。”

难得文彦博赞人,文及甫连连点头,等着老父的下文。

但文彦博却又走了起来,文及甫连忙赶上去搀扶。

已是暮春,自邙山中流淌下来的溪水越发的多了。

位于邙山下的文家别业,向以山林秀美著称西京。

文家别业之后,有山坡,有溪流,更有芳草萋萋、篁竹丛丛。春夏秋冬,揽胜访幽,皆会感到惊喜。

父子两人沿着青石板铺就的小路,一路向上。穿过一片竹林,文彦博方才幽幽说道:“就知道此子不会甘居人下。”

不用文彦博说明,文及甫也知道他父亲到底说的是谁。自不会是宗泽,只会是出题的韩冈。他的七弟将宗泽的试卷一并抄来,重点还是在题目及评判标准上,而不是状元郎的答案。也许宗泽的回答十分出色,但在真正的宰辅眼中,没有实绩为凭的答案,也仅仅是一篇好文章。

文及甫单手艰难的翻出了长信中的某一页,随着文彦博的脚步,扶着他边走边说:“七哥在信里也说了,这一次殿试考题的改变,完全是韩冈的独断,韩绛、张璪皆不得参与。”

“不是说朝堂上,”文彦博偏过头,“是儒门之中。各家之争,如今愈演愈烈。王安石、韩冈翁婿二人之间更是。韩冈此子或许可以不在乎一时的官位高低,但他绝不会甘心让新学压在他的头上。”

“但韩冈这么做,气学就成了众矢之的了。”文及甫争辩道。

“那些新进士出来后怎么说?”

“当然是骂韩冈。”

“你觉得有用吗?”

文及甫摇起了头,“没用。”

“对,没用。欧九因文体黜落多少贡生,也没见能奈何得了他,天下文风都为之一改。眼下仅是在殿试上,又是名次高下,谁敢轻易开罪韩冈?赶去找张载、韩冈的著述都来不及。”

“这么看来宗泽当是气学门人。儿子记得他是以评论河东战局而出名,想必韩冈那次去河东,当已经投入其门下了。”

文彦博不置可否,抚摸着路边一支将及一人高的竹笋,“才一天,都这么高了。”他回头对儿子,“别看刚出头,转眼就不一样了。看现在,想得到昨天才一尺多高吗?”

文及甫会意,点头道:“儿子也听说他曾去听过程伯淳的课。”

“博采众家,方是治学之道。宗泽的文章不差,光靠读新学、气学两家的著述肯定不够。”

不管有多少侥幸,不管太后多么偏袒,宗泽这位偏向如此明显的考生,王安石和章惇都没能拦住他成为状元,本身必须要有足够的才华,可不是像那位叶状元一样。

以叶祖洽状元之位,十余年方得为河南府通判。要知道状元释褐授官,一开始就是京官,通判资序。与三五名之后的进士,需要从选人阶段开始苦捱完全不同。洛阳河南府是四京之一,地位高于他处,府中通判也有知州的资序,可同科的韩冈都两入两府,其他同年也有做到知州的。

这与叶祖洽本身的才干有关,能被挑选为熙宁三年庚戌科的状元,只是因为一句‘祖宗多因循苟简之政,陛下即位,革而新之’投合了先帝之意,王安石又因为要变法,而把这种溜须逢迎之辞当成是号角,才让叶祖洽捡了便宜去——眼下党争归党争,但还没有到只论派系、不顾事实的地步,真没有水平,绝难在诸宰辅那边逃得了好去。

文及甫也有同样的感慨,“能将这样的文章置入榜末,王存之辈,可谓是有眼无珠。”

宗泽的名字被放在了最靠后的位置,倒数十名之列。从礼部试的前百,降到倒数十名之内,如此巨大的落差在历年的考试中也不多见。

文彦博回头,有几分不快的瞪着儿子:“你看了宗泽的卷子没有?!”

“……看了。”

“看了还不知道他为什么会被排在最后?”

文及甫干咽了口唾沫,小声道:“因为在策问中太过尖刻。”

文彦博重重哼了一声:“知道还说!”

今科殿试策问一题,是很多人事前都猜测到的询问阙政。

正常当然是要多说几句太后的丰功伟绩,然后批评宰辅;若想赌一把的话,就可以拿,批评太后对二大王姑息过甚,宰辅不能事先防备,如今的情况,太后不可无责——就像对郑庄公一样的批评,然后再赞一通太后的治政,来一句瑕不掩瑜。

而宗泽文章中的批评,比起后一种的手法更为犀利,尤其是批评太后与朝廷。对河东、河北的灾民用心不够,颂扬太后执政的篇幅远远少于其他人。试问那位考官敢于将这样的试卷放在前面?

现在太后的一句话,将位居倒数的考生一下提拔成状元,考官们哪一个能逃过识人不明、判卷无术的罪责?太后没有介意宗泽的直言,反而大加褒奖,王存之辈却将他放在最后,以此来讨好太后,如此作为,在士林中怕不要被视之为歼,事后也会为御史所论,以罚铜论处。

被训了一句,文及甫扶着文彦博,不敢多说话。

下了小坡,那条溪流又出现在眼前,沿着溪边小路走着,文彦博问道:“王存等人只是罚铜,其他处罚有没有?”

“没有,有人帮着说了话。”

“是韩冈?!”

拔高的尾音让文彦博的问题充满了嘲讽的味道。

“是章惇。说王存等人诚有过,然猝不及防下,也难免错讹,不宜重惩。韩冈没有反对。”

文彦博沉默了几步,回以重重的一声冷哼。

文彦博的心思,文及甫这个做儿子的多多少少能猜到一点。从对考官和状元两件事上可以看出来,韩冈还没有与王安石、章惇等人真正撕破了脸,互相之间还极力维持着关系。这种斗而不破的局面,肯定不是文彦博想看到的。

父子两人默默在小路上走着,贴身的仆婢前后都在十步之外,不敢打扰到文彦博和文及甫。

年岁越大,文彦博的身体却越发的康健。每曰晨起和午后,文彦博都会从别业后的竹林走上一圈,不是养尊处优,少有运动的文及甫能比。文及甫这个第六子是文彦博中年之后才生,论年岁也不过四十出头,可随着文彦博在山上竹林中走了一圈,老宰相仅是微有薄汗,文六衙内却已经是呼哧带喘。

在山下水池畔的小亭中坐定,看着呼吸粗重的儿子,文彦博摇摇头:“真是没用。”

不再理会儿子,文彦博低头仔细地看起这一次殿试的考题来。

许久,文彦博抬头道:“这一题申论,当是韩冈准备在制科御试上出给黄裳的题目。”

若是其他考题,不论是策问,还是论。不论黄裳写得多少,都会有异议。只有这种新体例,才会让人无法置喙。

文及甫此时已经缓过气来:“大人说的是,儿子也是这么想的。”

“如今韩冈将这制科考题放到了殿试上,若仅仅是加了一题,其实不足论。评卷的考官,可以只看策问,不顾申论。韩冈要是拿申论做文章,反而落了下乘。”文彦博眯着眼睛,“过去也曾有诗、赋、论三题并举,但最后评定高下还是看赋文的水平,诗与论,有个中上水准就可以了。但韩冈将两题明确为三七之分,尽管申论只居其三,但也没人敢放弃这一题了。”

少了申论,就是少了三十分。在四百多新科进士水平相差不大的情况下,一分都代表上下十名的变化,何况三十分?

听了文彦博的话,文及甫就想起了信中那位只做了一刻钟的头名贡生。

原本他为考官们排在了第一——其申论一题在第三等,也是唯一一名在第三等的考生。在用上了百分制之后,原本第一题很难做到出类拔萃的考卷,因为第二题的高评价,比起其他考生至少多了七分半,一下就拉开了差距。不过在王安石、韩冈等宰辅看过之后,给共同黜落为第五等,总分一下就少了十五分,不仅没了第一,连前三、前五、前十都没能保住。

“但宗泽被取中,也是靠了太后钦点的结果。韩冈的谋划,也是无用。”

太后的钦点就是一切,既然说宗泽是状元,那他就是状元。真要说起分数,他绝不会有其他人高。即便第一题能够得到上等的评价,第二题也不会让宗泽与其他考生拉开差距。信中将这一次殿试之事说的很详细,事后有人问韩冈,对宗泽,韩冈的评价是第四等上、第三等下。以殿试评卷应有的苛刻,自是要取下限。依然是第四等。

“能别出心裁,又能使之顺理成章,这是韩冈的本事。就算这一回不是宗泽被取中,也不会是将国子监中将经义倒背如流的‘人才’。”文彦博在最后两字上加了重语气,满是讽刺,“诗赋选拔不出人才,经义一样也不行。苏轼的当年这么反对更改进士科的体例。申论也不能,可至少能知道那些新进士有多少见识。”

“也只是纸上谈兵。”文及甫道。

“好歹能谈了,而不是吹嘘。所以王安石才能容得了他如此行事。”

“王安石的脾气好像变了不少。”文及甫想到了之前第一次推举,韩冈能够入两府,还是他的父亲遣人去帮的忙,要是韩冈与王安石继续维持下去,岂不是白费功夫?

“是韩冈懂得收敛,也是才开始的缘故。”文彦博不急不躁。

韩冈迟早会明白,宰辅和儒宗之间,绝不可能维持一致的行事作风。

或许韩冈已经明白了。
