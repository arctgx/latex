\section{第34章 山云迢递若有闻(九)}

“想走?哪有那么容易!”

目送送上门的兔子跑掉,韩冈可从没有这么大方,“刘源,缀上他们,拖住贼人的行程……只要拖住一天时间,临洮城收到消息后,必然会有所应对。”

“韩机宜,穷寇勿追!”王中正连忙阻止,堡中才多点人手,哪还能分出去追敌。

“都知放心,只是拖延而已,不是让他们追击。”

“可战马不够……才六十余骑!”

渭源堡中的骑兵一早就被韩冈派了出去,现在的几十匹都是靠着刘源等将校的奋勇拼杀抢来的。吐蕃人虽然败退,可要刘源的这点兵力过去,还想拖住敌军的脚步?只是给人送点心!

“可以用挽马、驿马!追敌不是上阵,拖延也不是厮杀,是不是战马无甚大碍!都知大可放心。”韩冈仍然坚持己见,无视王中正的意见。

挽马、驿马都是无法上阵杀敌的军马,或是因为体格,或是因为脾性,在军队挑选战马时落选了,但在韩冈看来,挽马也好、驿马也好,用来载人是没问题的,到地头再下马作战就可以。

京中有一军号为龙骑,全军都是有马步人,也就是行军骑乘,而作战时下马。虽然龙骑兵现在已经是名不副实,连代步的骑乘马都不剩几匹,但有马步人的作战方法,刘源等人也是能理解的。

“刘源!你有没有问题?”韩冈厉声问着。三百精锐将校,就算有大半骑着挽马、驿马,足以抵得过一个指挥的骑兵选锋,拖延一下贼人,他们应该能完成。

“没有!”刘源单膝跪倒,抱拳的动作一如往昔般刚毅,“末将接令!”

刘源一众身为被流放到通远来的贼囚,他们对于官衙的命令,根本没有讨价还价的余地。韩冈能问上一句,已经算是宽厚了。

韩冈反身对心头有了几分火气的王中正解释道:“只要能拖住一天就够了。以派去临洮报信的速度,明天早间,王安抚就能得到消息。从抹邦山返回洮水西侧的渡口,就那么两处,只要临洮及时出动骑兵,去抄截后路。那两千吐蕃骑兵,至少也要留下大半。”

王中正脸色忧一阵,喜一阵,难以作出决定,最后仍是摇了摇头,“……还是太过冒险了。”

如若是刘源他们出了点意外,让吐蕃人再杀回来,一下少了三百精锐的渭源堡如何抵挡得了?退敌之功难道还不够,偏偏还要再锦上添花!

王中正心中很不情愿,可他偏偏又压不住胸有成竹的韩冈。

“关于此事,都知更是不用担心。这附近是青唐部的地盘,包顺包约也不是带着所有部众去了临洮……韩冈前面已经派了人去左近的蕃部征调人马,很快就会有回应。”

刘源带兵出去了,堡中的马匹也都给广锐军将校带走了。幸好渭源这里堆积了大量辎重,其中并不缺乏马鞍。人人跃马扬鞭,飞驰而出。

第一批附送蕃部的援军赶来了,他们是渭源堡附近的一个青唐部的分支部落派来的,只有二十多人,有老又少。

“韩机宜,他们未免弱了一点。”王中正指着几名白头凝霜的蕃人,“都老成这样了,如何拉得开弓。”

“还请都知少安毋躁,再等一等。”

正如韩冈所说,很快第二批援军也赶来了。这下人数稍多了一点,大约百来人。手上都是提着制作精良的弓刀,制作的水平不差

有了两队开头,后续的一批接着一批的人赶来,快到堡中灯火点起的时候,附近的蕃部已经来了快有一千人了。

“王、高二安抚在河湟用心多年,恩信深著,眼下只需缘边安抚司的一道命令,便能让通远诸部闻风景从。”

韩冈如此说着,可王中正见着这些蕃人,在向韩冈跪倒行礼时,韩机宜三个字说得更是字正腔圆。很显然,韩冈的名望在蕃人中并不必王韶、高遵裕稍差。

有了上千名蕃骑环绕,王中正的底气一下壮了起来,“韩机宜,可是要出兵?!”

“暂时有刘源在前面拖着就够了。”

夜色对行军当然不便,而对于骚扰却是最好时间段——韩冈让刘源去拖延蕃人都行军速度,并不是随意下令,更不是让他们去送死。要不是看到成功的几率很大,他自然会像方才王中正所提议,选择见好就收,而不是全歼来敌。

“夜中行军不便,而且这些蕃人需要整顿一下。等到明天早上开始追击。”尽管人多了,但大部分来援的蕃人并不是训练有素的士兵,在战场上十分的脆弱,韩冈不会选择让一盘散沙上阵,“过一阵,王君万也该回来了。”

知渭源堡的王君万,在渭源此地任官近两年,因功官位涨了两级,但差遣始终未变,最近加了个巡检的兼职。今天被韩冈派去巡查鸟鼠山道,一大早就带了几十个骑兵出去了,要不然以渭源堡的地位重要,也不至于一开始连个骑兵都拉不出来。

韩冈手边没人领军,若是让这些蕃人一窝蜂地出去,被打得大败而逃,他韩冈也少不了罪责,总得给他们一个主心骨。但刘源叛将,所以只能等王君万回来。

“究竟能如何,还要看刘源的本事了。”

……………………

从洮西渡过洮水,再绕过抹邦山,杀奔渭源堡。一日之间奔袭一百多里,结果却是大败而归。虽说瞎吴叱和结吴延征两人所率领的都是本部精锐,但这一趟下来,也是狠狠的伤了一番元气。

行在洮水的支流边,骑手和战马都是垂头丧气,而且吐蕃军上上下下都是跑了一天一夜,如果有个胜利,还能振奋一下士气。但现在,根本都没了前进的动力。

看着撤退的速度越来越慢,瞎吴叱对结吴延征道,“还是先歇一歇吧。”

结吴延征觉得这里离着渭源堡太近,离撤回洮西的渡口又太远,“小心退路被堵上,说不定后面渭源堡也会有追兵。”

瞎吴叱怀着一点侥幸的心理,“渭源堡中的兵力不多,歇上一个时辰也没问题。”他又叹了口气,“现在都没了气力,歇一下,才能走得快。”

说着,他就让人传令了下去,不过为了防备追兵,也还是派出了几十名哨探。

奔驰了百里,人和马都累得不轻,终于得到了瞎吴叱的命令,吐蕃士兵立刻横七竖八的躺了下去,转眼间便躺满了山谷,甚至很快就有了鼾声。

结吴延征看了这一幕,同样叹了口气,摇摇头,也坐了下来。

‘才一个时辰,应该没有关系吧。’

夜色很快降临,不知不觉,一个时辰就已经过去了。

结吴延征和瞎吴叱跳了起来,踢着下面的族兵,催促他们起来上路。

已经睡得迷迷糊糊的蕃兵被叫起,依然有些晕头转向。虽说不敢违背两名主人的命令,但动作磨磨蹭蹭,场面乱作一团。

而在这个时候,刘源纵马出现在山谷中,身后紧随着近三百余名骑手。夜影中,黑压压的一片。数百点小小的灯笼在黑暗中闪烁着亮光,这是他们坐骑的眼睛。

“吐蕃人太大意了。”刘源耳边,有人轻声地说着,很是兴奋。

虽然吐蕃人放出了哨兵,但他们因为大军即将动身,而靠近了本阵,已经失去了预警来敌的作用。

刘源默默的点了点头,“差不多是时候了。”

大军启程上路的时候,就是最为脆弱的时候。歇息过后的吐蕃人即将动身,这是偷袭他们最好的机会。

刘源握紧了长枪,胯下的战马感染到了刘源压抑在心底的兴奋,轻快的打着响鼻。刘源顺了顺马鬃,西军中的战马多是从吐蕃人手中买来,多年的经验,让他调教起刚刚抢来的这匹战马,也是很容易就上了手。

韩冈让他去阻滞逃敌,但他却是远远的吊着吐蕃军,根本不上去厮杀。刘源带着他的兵,像一头饿虎一般,遣行在茂密的树丛中,静静的尾行着猎物,等待一击噬喉的机会。

在他看来,韩冈无论兵事、政事、胆色、才智,都是他所见识过的文官之中,最为出色的几人之一。可今次未免太高看了眼前这些蕃人。

文官用事一向谨慎,刘源是清楚的,只是今天来袭渭源的吐蕃人,既然已经从城下败退,那就根本没必要惊动临洮城的主力——出击和败退的两种不同的情况下,蕃人的战斗有着天壤之别。

为了劫掠而杀出来的时候,士气高涨的蕃部骑兵,绝对是一个强敌,以蕃人军纪的松散,甚至都能做到令行禁止。可他们一旦失败撤退,就再无严整的军纪可言。就像眼皮下的这个场面,根本是一盘散沙。

长枪遥指前方,刘源深吸了一口气,一声暴喝脱口而出,“杀!”

蹄声踏碎了夜色,六十余骑领头冲出了山口,直接冲进了混乱的敌军之中。而紧随在后的两百名战士只是稍慢一步,也随即嚎叫着冲进了敌阵,下马冲杀起来。

让人猝不及防的偷袭,搅乱了拥有两千战士的吐蕃军。瞅准了中军位置而冲杀过去的刘源等人,在没头苍蝇一般混乱的敌阵中,像切菜砍瓜一般的轻松。

在黑暗中,吐蕃人不知道有多少宋人冲进了自己的队列内,只听得左右前后,一片声在喊叫。晕头转向之下,更多的人选择了逃窜而不是反抗。

瞎吴叱和结吴延征重新组织队伍的努力,在这番冲击下,化为了泡影。各自被人流冲散,混乱中,结吴延征突然听见身后乱声大噪,急回头,只看见一道弧光映月,朝着头面直劈了下来。

两个时辰后,一名骑兵冲进了渭源堡中。

“赢了?!”

正准备出战的王君万当先跳了起来,又惊又怒。

“是大捷!”

被派回来的信使根本不把王君万放在心上,向韩冈和王中正重复道,“阵斩敌将,是大捷!”

王中正的惊喜,王君万的隐怒,韩冈都看在眼里,暗自一叹,这还真是出乎意料的结果。

