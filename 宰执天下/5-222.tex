\section{第24章 缭垣斜压紫云低(14)}

方才在御前的时候,两府已经安排了宿直的人选。

当韩冈与蓝元震一齐抵达今天的住宿的地方——与福宁殿只有一墙之隔的小殿——就发现王珪和薛向已经坐在里面了。

今天的政事堂是王珪宿直,而枢密院留守的便是薛向,只是不见三衙管军中应该在今天宿卫宫掖的张守约。

“玉昆。”王珪看到韩冈便立刻起身问道,“天子的情况如何了?”

“若不出意外,不久当能开口说话。”韩冈还是方才的说法,依然加上了不出意外这个前提。让王珪和薛向不得不多想深一层。

“可能康复?”王珪追问着。

韩冈默然不语,只反看回去。

王珪闭了闭眼睛,睁开后便坐下来颓然一叹。药王弟子既然也认为天子无法恢复,那么当真就没办法了。

其实王珪也自知是问得多了,过去也不是没见过中风之人,何曾有过没有后遗症的病人?以赵顼的病情,既然连话都没办法说了,再站起来的确是很难了。

蓝元震将韩冈送来后,便立刻转出去安排三人的食宿,还有服侍的人选。几位重臣贴身的仆役不能入宫来,这就必须将他们给招呼好了,半点也慢待不得。

“张太尉呢?”韩冈坐下来后,问起了今天三衙管军中当值的马圌军副指挥使张守约。

“张守约带着一队班直去巡视宫中了。”薛向说道。

王珪也跟着道:“今夜人心浮动,张守约去走一圈也是好的。”

韩冈点头道:“张太尉是宿将,有他巡视宫中,倒是能让人放心不少。”

知道张守约对韩冈有举荐之德的官员为数不少,两府之中全都知道此事,所以听韩冈在背地里还对张守约尊称太尉,王珪和薛向并不以为异。

当年王韶举荐,张守约也搭了一把手。据传闻说,若不是王韶抢了一步让韩冈入了文资,张守约就会荐韩冈入武班。如果韩冈走上的是那条路的话,说不定当今的大宋就多了一位将种了。

王珪投向韩冈的视线中带着些许嫉妒。此子文武兼备,医道更不用说,又精擅农工之事,好像什么都能精通。据称也就不会下棋罢了。

想起韩冈不擅下棋的传言,王珪不由得暗暗一笑,冲淡了一点阴郁的心情。记得当年的林和靖【林逋】自称百艺皆能,惟独不会担粪和下棋。过去几年,世间的一些个刻薄之人,就说韩冈要比林和靖强上一筹。不过自种痘法出来后,倒是没人再敢乱说了。

韩冈、王珪和薛向有一句没一句说着话。

王珪是宰相,韩冈则是轨道的创立者。若是平常有这个机会,薛向肯定会趁机提到两句铺设铁轨的事。之前,薛向不仅跟韩冈敲定了合作的协议,也跟王珪谈妥了,并互相之间做了利益交换。就薛向而言,他本就准备三人找机会坐在一起,将整件事给敲定,但现在实在不是时候。

‘又不知要拖多久了。’薛向不无恼火的想着,但又不知道该去抱怨谁。

三人都没有太多闲聊的心思,往往是某人说上一句后,过上半天才有人另接上一句。心中都是沉甸甸的,仿佛有块巨石压在头顶

赵顼变成了如今这副模样,太后垂帘已经成了定局。最多也就等两三天,一旦确认天子的身体无法恢复,那么宰辅就得率领一众朝臣请求太后垂帘听政。

王珪现在很是担心,一直以来他都知道太后对自己可不是那么欣赏。要讨太后的好,并不是做三旨相公就能解决。

“今天晚上,还不知有多少人睡不着觉呢。”

正说着,就听见宫里报时的云板敲着二更的点,而蓝元震正好跨过门槛进门来。

蓝元震给王珪、薛向、韩冈他们安排好了食宿,便转了回来。韩冈等三人都不关心这些小事,他们宿卫宫中,也不是为了吃喝睡,无可无不可的听了蓝元震的一番禀报,便让他回去向太后和向皇后复命。

