\section{第八章 朔吹号寒欲争锋(二)}

“的确是有渊源。”

韩冈欠了欠身子,对太后道,“昔年元昊猖乱西国,文正公受命帅陕。先师文诚先生其时年方弱冠。也曾拜望辕台,上书请文正公出奇兵,攻陇右,以为偏师。而文正公则勉励先师向学,并以论语相赠。先师由此钻研经义,并教书育人。所以臣能得先师授学,正是文正公的功劳。”

“令师的建议,不正是参政旧日跟随王韶在陇西做的那些事?若范文正当年依照令师的建议,是否当时就能平灭西贼?”

范纯仁闻言微皱眉,太后听了韩冈的话,怎么就会想到那里去。

只听韩冈回复道:“几年前故世的王襄敏,其当年所上的《平戎策》,其根本正与先师不谋而合。但这一谋略,放在三十年前,却很难成功。当时兵不习战事,将不通兵法,故而才容元昊坐大。其实若是换作十余年前的西军,在元昊刚刚举兵时,一路兵马便可将其剿灭,可是三十年前的西军却一败再败,几无还手之力。当其时,可守不可攻,文正公为陕西主帅,对手下的兵将看得要比先师更清楚一点。”

韩冈的持平之论,没有偏袒张载,这让孝顺的范纯仁对他立刻平添了几分好感。

之后的问对,十分的顺利。

向太后明显对范纯仁很是看重,问了很多地方上的问题。不过也时不时的向韩冈询问,而韩冈也尽力回答,对范纯仁多有维护,不过在说道新法时,韩冈就十分坚持,丝毫没有偏向旧党的意思。

比寻常的问对多了一些时间,半个时辰左右,范纯仁才从崇政殿中告退出来。

走了没两步,又听见背后有人叫,“尧夫。”

回头看时,却是韩冈追了上来。

尽是被人从背后叫住。范纯仁不由得想到这是不是今天哪里不顺。

而韩冈这么快就出来,也让他挺惊讶。

“敢问大参有何指教?”

“不敢。之前听孙莘老提起傅尧俞此人,并说其人与尧夫相友善,不知是否确实?”

孙觉已经向韩冈举荐了人?范纯仁略感惊讶,昨天他都没听孙觉提起,难道是回去之后又有了什么变化?

不过孙觉向韩冈推荐的傅尧俞,范纯仁却十分的熟悉。

“傅钦之?……大参没有听说过他?”

“听说过此人,不过并不熟悉。此人如何?”

傅尧俞与韩冈之间没有交集。韩冈也只知道他似乎是在英宗时比较受到看重。

“庆历二年的进士,治平年间的知谏院,离朝也有十余年了,难怪大参不熟。”

范纯仁的感慨,听起来有几分讽刺韩冈年资浅薄的味道,不过韩冈也没在意,“不知傅钦之的为人如何?”

“傅钦之性亮直。英宗时,为尊濮王一事上表谏阻,天子不受,便坚辞出外。既至州郡,绝口不言出外缘由。有人问及,方才说:‘前日言职也,岂得已哉?今日为郡守,当宣朝廷美意,而反呫呫追言前日之阙政,与诽谤何异?’”

“若如尧夫所言,诚乃佳士。”韩冈点头说道。

这跟自己倒是挺像。一码事是一码事,在言官的位置上尽力劝谏,到了地方,就安心做事,不会拿着劝谏天子被贬一事,宣扬自己的刚正。对御史来说,这种贬官是扬名立万,换做他人,会宣扬一辈子的。傅尧俞绝口不提,可见其人的正直。

“大参可知他现居何职?”

“孙莘老说了,监黎阳县仓草场。”韩冈抿了抿嘴,“的确非是待贤之地。”

…………………………

在宫中无法细说详情,范纯仁很快便与韩冈道别而去。

回到政事堂自己的公厅中,一群堂吏上来拜见韩冈。

又捧上了一叠亟需他批阅的公文。

韩冈拿起公文,开始批阅,顺便让人去找傅尧俞的资料。

中书门下的架阁库中,有着每一位官员的履历。想要查找任何一名官员,基本上都能找得到。

傅尧俞的档案很快便被拿来了。

在上面,韩冈看到了少年得志的新进,二十不到就中了进士,三十出头便就任御史。四十岁成了殿中侍御史、起居舍人,同知谏院,为英宗皇帝所重用。不过到了先帝赵顼在位之后,便因反对变法,官途一落千丈。

先是出外,然后一年六迁,让他一整年的时间都在道路上奔波。

这基本上就是除了远窜岭南之外,整治政敌最狠厉的手段了。就算是贬去监盐茶酒税,虽是穷困,实际上也能得一个安稳,可若是不断迁官,在路上说不定什么时候就得了急症。可见其开罪王安石不浅,下面的人为了迎合王安石,下了狠手。

不过韩冈又看了傅尧俞的历年考绩,却皆在优等,完全不像是旧党的样子。

比如富弼、文彦博,他们因反对新法出外之后,到了地方完全不会去推行新法,而是消极怠工,甚至干扰新法的推行,然后上表说百姓反对新法,所以无法执行。

傅尧俞不一样,他对新法反对归反对,但到了地方上却还是尽力去执行,这一点,只看他历年的考绩就可以明白。而且他还不是被贬后的改弦更张,要不然现在就不会接连做了好些年的监仓草场。

之所以会成为监仓,倒不是新党的打压。是因为他在徐州时,有人告发一人‘谈天文休咎’,也就是拿着天象说吉凶祸福,傅尧俞以没有实证,不加受理。但朝廷对此罪一向十分重视,由路中提刑进行审判。当受到告发之人被处刑之后,傅尧俞也因为没有及时将人捉拿归案,而被削去了官职。过了半年多,重新被启用为监黎阳县仓草场。

难怪孙觉会推荐他。

如果不是极擅作伪的人,那么就是个标准的正人君子。

看起来是旧党元老们害怕损害了与自己的关系,选择了一步步来。

韩冈觉得这的确不错。

现阶段,双方都在试探中,洛阳那边能保持着这样的心态,对双方的合作很有好处。

傅尧俞的品行,韩冈很满意,剩下来,就是他的能力问题了。

在韩冈看来,品行也许很重要,但更重要的是才干。不过旧党元老悉心挑选出来的这一位,能力应该也不会差到哪里。否则也不必孙觉,和他背后的那几位如此费心。

以傅尧俞的资历,至少一个卿监才能安排。

而想要拿到这个等级的职位,对已是参知政事的韩冈,却并非难事。

就像王居卿,韩冈便准备安排他去做判军器监。

王居卿的档案现在也在韩冈手中。

王居卿也近六十岁了,升任侍从官却是最近的事。

他过去在地方上的功绩,履历中记得很清楚,跟韩冈之前了解过的没有区别——朝中的重臣,韩冈都是会尽量去搜集他们的资料,这也是为弥补家世底蕴不足的缺陷——的确是个能做实事的人才。

不过更让韩冈感兴趣的,是有关酿酒酿醋的一件事。

酿酒酿醋的连灶法据履历中的记录,是王居卿所献。但在韩冈的记忆中,这似乎是吕嘉问的功劳。

虽说名字似乎有些区别。一个是连醦法,一个是连灶法,但看功用,都是省下了柴草费,应该是一回事。

据韩冈所知,吕嘉问便是因为行连灶法每年为朝廷省了十六万贯,故而得到了褒奖,之后才在王安石那边留了名。

如果推理一下,可能就是吕嘉问抢了王居卿的功劳,或是吕嘉问因为推行而独揽其功,而王居卿虽然是献法之人,却被遗忘了。

这是多少年的仇怨了?

如果这旧档中记录得没有问题,也难怪在廷推上,王居卿会背后捅上一刀。

既然有这份旧怨在,韩冈倒是可以安心的使用王居卿了。

安排王居卿、傅尧俞很容易。

比起在西府做枢密副使的时候,参知政事手中的人事权要大得多。

主管低阶和中高阶文武官的铨曹四选——三班院、流内铨、审官东院,审官西院——其前三个衙门,很长时间以来,都在政事堂掌握中。

而自从王安石主持设立审官西院,把本属枢密院的中高阶武官考课选任之权,也转到了政事堂手中。在人事权上,枢密院更是一落千丈。虽然说枢密使们对武将的提名,审官西院一般不会驳回,但若是遇到两府相争时,枢密院只能吃瘪。

而且政事堂中的宰辅,并不是只能通过铨曹四选来间接影响人事安排,还有所谓的堂除,也就是归属于政事堂直接注授差遣的职位。从地方,到中央,堂除的范围无所不包,而且还在不断扩大中。

现在韩冈只要一句话,上百个军州,近千县监,都可以拿出来让人挑选。

在韩冈而言,困难的不是位置问题,而是他手里缺乏中低级的官员。

政事堂之中,韩冈缺乏足够的青绿小臣,去占据中低层的位置。而新党一方,每隔三年就有数百个选择,在国子监中更有两千余人等待挑选。

如果韩冈有足够的人手,现在就会设法将他们安插进中书五房之中。中书五房检正公事,以及各房的检正公事、习学检正公事,都是宰辅们必须控制的位置。

这些个变法后才设立的官职,曾布、李清臣、李承之都是从这里起家的,韩冈也曾经有机会就任最高的中书五房检正公事,只不过给他推了。

但进入中书就等于搭上了飞黄腾达的直通车,必须要升朝官才能就任,现在这些位置上,理所当然的都是新党成员。

看着这几个职位,手中无人的韩冈也只得干瞪眼,所以他就任参知政事,便只能依照惯例,安排了几个亲信的家人,进政事堂做掌管文字的吏员。

