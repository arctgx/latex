\section{第二章 天危欲倾何敬恭(11)}

京师子民天然的就带着一种优越感,不仅是因为居住的位置,也是因为见识。皇城根下,车夫都能指点江山。外人所不知道的宫闱秘闻、朝中议论,随随便便都能从他们嘴里砸出来,让土包子晕头转向。

而县中、乡里的居民,相对京城军民而言,更是有一种自卑感。许多乡民,一辈子都不会走出百里地,对那些见识过京城繁华,敢于离开乡里的邻人,往往都会掺有一份羡慕和敬重。羡慕他们的经历,敬重他们的见识和胆量。

而现在有一份报纸,区区十几文钱,就能得到京师最近的大小新闻,这就不仅仅是买消息了,更是在买优越感。

“都没想到获利能有这么多。”蔡确感慨着。作为一名宰相,不可能避讳财利的话题。言谈中口不离财,却没有半点不好意思。

韩冈奇怪的道:“虽然卖报利钱不少,可邮政局至少要分去一半,最后还不是要归政事堂?”

“说的就是邮政局。京城之外,一份报平均能收两文钱邮费,积攒下来可就不得了了。报纸说多不多,说重不重,上万份也不过一车拉了。”

蔡确大赞着韩冈。韩冈很多时候都能出人意表,总有些让人惊艳的想法。只看将事关国家命脉的驿传投入民用,就是到他根本就不是朝廷旧规能约束的开创之才。

其实现在除了信件、报纸和期刊外,已经有更多人利用起邮政递送来。过年送拜帖,这是京城年节时的风俗,都是帖子到就算是拜过年了。而在这一个正旦,已经有人用邮政来递送拜帖。

韩冈则觉得有些纳闷,蔡确前面的话和现在的似乎对不上。前面还在叹着朝廷花销多,打听报社赚了多少,现在又为邮政大唱赞歌。

“不过也要谨防涸泽而渔。”韩冈提醒道。

蔡确摇头笑道:“玉昆你终是要为那两家说话。”

“其实最好的办法还是尽量降低运输开支,如此同样的收入,邮政局的净入也就可以更高。”

“……怎么降低?玉昆可有良策?”

“之前就想跟相公说了。过去是朝廷拿不出钱来,可现在不一样了。修建轨道干线,正是其时。”

“轨道……”蔡确笑了一声,道,“还以为玉昆你忘了。”

“轨道将会是国家命脉,与汴河一般沟通东西南北。军国重事,韩冈岂敢或忘?相公不也是没忘?”

“玉昆说的是,这可是忘不了的。”蔡确点着头。这么重要的事,他怎么可能忘得掉?尤其还有比几万亩良田还要重要的产业在里面。

“先修好轨道的纵横干线,再从干线中分出支线,就如同树木茎干,将根吸上来的水肥输送到每一片叶子上。轨道的速度和运载量都远超现在在官道上的车马。通过轨道来运送邮件,邮政驿传的开支就能减少许多。官员都坐有轨马车行动,就能节省驿站的开支。而各地商货,更是能由此流通。国家财计自然会更为宽裕,反过来也就能更加促进轨道交通的发展。这是一而二,二而一的事情。”

“也不求那么多。”蔡确听着,却没有什么反应,喟叹道,“现在也只求能公私两便就好。”

韩冈应声道:“公私两便,本是应有之理,做事但求两全,岂能一偏?”

又说了几句闲话,韩冈起身告辞,却没有再提什么。

蔡确欲言又止,还是送了他出去,临别时,他对韩冈道,“本来还以为玉昆你今日来是为沈括做说客的。”

“有相公在,何须韩冈多言。”韩冈笑道,躬身行礼,然后辞别出门。

沈括从政事堂拿不到钱,转求到韩冈门前,知道韩冈与沈括关系的,不用亲眼看到都能猜得到他会这么做。但蔡确只是拿捏沈括而已,终究还是要给

钱的。谁敢让京城乱起来?宰相也不行。所以韩冈没必要多费唇舌,过来一趟,不管提没提这件事,之后沈括都会如愿以偿。

韩冈并不像这么早来拜访蔡确,不过既然有个由头,那也就顺水推舟了。

从枢密副使退到宣徽北院使,再退到大图书馆馆长,再继续退下去,真的就有些麻烦了。

当然,就像韩冈常常说的,也只是有些麻烦。

皇帝有生杀予夺之权,的确是很危险。

可天子什么时候能做过快意事?有可能不经法司,就将一名重臣拉出去处斩?只要他有个动作,所有的大臣都会警惕起来。群臣联手,皇帝又能有什么办法?

臣子们习惯了对天子的冒犯,日后也不会将手中的权力放下。十年之中,这样的胆量能不能培养出来?难说得很。但还是有时间去尝试。就算不行,事到临头,也轮不到他们再犹豫了。那时候只要有人出来领个头,还是都会跟着一起走的。

送了韩冈离开,蔡确的脸便沉了下来。

韩冈到底在想什么?真的这么有恃无恐?

蔡确发觉自己真的越来越难猜度韩冈心中的想法。

他知道章敦去过韩家,应该也跟韩冈谈论过皇帝的事。可转天过来,蔡确去问章敦,那位枢密使却支支吾吾,语焉不详。

这样子让人如何有信心?

韩冈今天过来,蔡确本以为韩冈会交个底,可韩冈却只是东拉西扯,将过去的事都说了一通。

日本的金银,如果真的有《自然》中说的那么多,辽国几十年内都不会再为患中国。而国家财计,有轨道配合驿传,还能通过铸币来补足。财计充裕,商路畅通,加上对外以土地为目的进行开拓,保证天下不至为乱。

这些都可算是韩冈的谋划。

蔡确觉得,韩冈今天过来说了那么多,其实归纳起来就一句话:听他的不会有错。

蔡确知道韩冈这是为了他安心才来。沈括不去韩冈家走一趟,韩冈都会找个由头登门。但蔡确想听的可不是这些话,而是如何度过眼下困局的手段。而韩冈一个字也没有说,除非还有什么给他忽略了。

蔡确想着,重新开始梳理起他与韩冈的对话,分析着里面是不是藏了些了什么。

茶叶、辽国、日本、金银、西军……

西军!

蔡确脚步猛地一沉。

难道是要聚兵为乱,让皇帝拱手画诺不成?

这种没脸没皮的事要做出来,韩冈还有脸去教徒子徒孙吗?

完全没有。有的只是韩冈近乎自吹自擂的话语。

难道是说,他知道小皇帝活不长,根本不用担心。

‘或许有那个可能啊。’蔡确漠然的望着前面的道路,‘是啊,可能……’

蔡确脚步沉沉的回到厅中,就见有两人等在里面,一个是蔡渭,一个是刑恕。

“什么时候回来的?”蔡确停住了脚,问蔡渭。

“儿子和和叔到了有一会儿。知道大人在见客,就没敢进来。”

“都听到了?”蔡确沉声问道。

刑恕却笑道:“轨道通天下商货,其利百倍,就是刑恕,也不免心动。”

蔡确板着脸,冷哼了一声。

刑恕的挑拨粗浅得很,但却正中蔡确的心思。挑拨离间本就是看人下菜碟,精妙粗浅与否只是末节。

刑恕见蔡确的模样,嘴角微微一翘,“公私财利都给他一手抓了。朝廷、私人但凡有点好处,都是借他的光,要承他的情,还要赞他远见卓识。”

韩冈今天过来,从辽国说到轨道,从西军说到邮递,这分明还是在维护他的势力范围,要蔡确做一个表态。

在宰相面前都如此跋扈,刑恕不觉得蔡确有那么好的涵养。

“轨道一事,有政事堂居中主持,用得力之人,聚州县之力,也用不着闲官插手。韩馆长特特提起,却是笑话了,不关他的事啊。”

刑恕冷冷笑着。

韩冈此前在朝中的地位,本身的能力是一条,但更重要的是依靠种痘法和冬至夜定储之功,皇太后对他的信重也来自于此。

如今他在民间的地位没有变化,但在朝中的影响力随着他离开朝堂,而为之大落。

一个名号可笑的宫观使,莫说影响朝政,就是上朝也是笑话。有见过集禧观使、太一宫使隔三差五的入崇政殿的吗?

皇宋大图书馆,看其心思,是想弄成三馆秘阁那样的储才之地。尽管韩冈设法掩盖他的心思,他作为先导设立的开封图书馆中,没有编修、修撰,只有管理。但这样的刻意远避,反而彰显了他的用心。

或许到时候,他会将图书馆变成向上晋升的跳板,将能搜罗来的人才都塞进去,其中的一名管理,都能有经天纬地之才,改天换日之志。

但那又如何?

过去朝廷还念着他的药王弟子身份,还需要他来保住小皇帝的福寿安康。就是当年先帝在位时,对他心生嫌隙,也没敢去动他分毫。只能忍着、看着。

但如今,就是原先最想要保住小皇帝性命的一群人中,大半都盼着幼主早日驾崩的,太后都少不了有这样一份心。韩冈还有用武之地?

不能顺应时势,却想着逆流而行,这是韩冈最大的错误!

