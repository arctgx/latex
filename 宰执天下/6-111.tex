\section{第11章 飞雷喧野传声教(八)}

韩冈给向太后推荐了去黔州的将领人选,这件事只要枢密院那边不反对,基本上就成了定局。

但枢密院那边肯定还要争上一争,这跟交情无关,章敦也好、苏颂也好,在他们的那个位置上是不能不争。

不过现在这件事算是可以放下了,向太后也在说,“这件事就先这么办吧,等明日跟章、苏二枢密说一下。”跳过了依然在任的曾孝宽,太后又道“不过明天此事传开,御史台恐会说黄裳生事了。”

“西南夷虽为蛮夷,亦是宋臣,其治下百姓,自然是皇宋子民。若其流离他乡,罪在本乡父母,岂在收留他们的州县官身上。若御史台以此为由弹劾黄裳,臣不知是何道理?”

“都能像参政这么有见识就好了,朝中糊涂得太多。”太后叹了一声,又问韩冈,“参政,运去关西的虎蹲炮没问题吧?”

“已经如数发出。连同配发的药包与炮弹,自入秋后,已经发去了十批,共计三百五十门,枢密院前日也说,皆已配发各部,目前正在日夜教练之中。”

“其他火炮呢?”

“都按照预定的计划在生产,年前的生产数量足以让神机营再增加两个指挥,陛下可以放心。”

“军器监有参政看着,就让人放心多了。”

太后赞了一句,韩冈欠身一礼,紧接着就等来了老问题,“那万斤的重炮还是不行吗?”

“还是只能做礼炮。”韩冈回复道,“现在能做礼炮,因为不要发射炮弹,填充的火药减了许多,真要上了战场,太容易炸膛了。看起来得等局中的工匠再历练一阵,技术再上一层楼才行。”

不需要发射炮弹的火炮,炮管壁就不需要太厚,只要口径够大就可以了。想要铸造出能将百斤炮弹发射冇出两三里外的重炮,现在暂时还做不到。可仅仅是样子货,这就很简单了。又不像铸钟一样,还要在模具上刻上经文、纹上图样,外表的装饰一切从简,连结构也简单到一根圆筒。以青铜的韧性,加上减少发射药的装药量,两门专用的礼炮,代替了之前的小口径火炮,成为重要的礼器。尽管没有造出个实际能用在战场上的重型榴弹炮,能弄出个礼器来,也不算是浪费朝廷的钱粮。

向太后叹了一口气,她对火炮十分看重,对于威力更加强大的新型火炮,实在是迫不及待了。但韩冈都说要等,那就的确是没办法。

“对了。今天沈括过来说了,各路州县的砖石虽然还没有运到,不过京西那边的青条石已经开始向京冇城运送,应该可以开始先动工了,这个冬天能修一点是一点,没必要浪费时间。”

“要说修筑之事,臣不如沈括。既然沈括说可以先修起来,当可以先开工了冇。”韩冈道,“等这一次修补完成,增加了炮台,还有包墙的砖石,东京冇城当可不惧任何外敌。”

“还要两年时间,真是够久的。”太后又是一声叹。

“是陛下仁心,不愿扰民,否则征用百姓,就会快上许多。”

由于要在外城增添四十五处大小炮台,整条外城城墙都要大改,京师的护城河也同样要进行开挖和增补,沈括身上的任务很重。不过沈括想要修筑的环城轨道,则是给否决了,技术上的难度可以克服,但运行上的问题实在太麻烦,也就是因为少了这一桩工程,才能够在两年内完工,否则还不知道要等到何时。

“加筑城墙并非急务,辽人也攻不过来,只是为日后考虑才加筑的。还是不要征发太多民夫。”

“陛下仁心。”韩冈行了一礼,衷心的称赞着。

受了韩冈的赞许,等韩冈回到座位上,太后说道,“其实还有一件事想要征求一下参政的意见。”

太后的话声有些吞吐,似乎是还在犹豫。

韩冈道:“请陛下训示。”

“今年就算了,都已经过了冬至。但明年春天,开春之后,该怎么办?今日有人上书要重开经筵了,不知参政如何看。”

韩冈听得出太后声音中的为难,而他听到此事后,也同样陷入了为难之中。

自从出了那次意外,加上之后的宫变,没有人再关心这位小皇帝的教育问题。所以这一年就这么拖了下来。可是从道理上,又不得不给赵煦安排一个甚至几个老师,既然有人上书要重开经筵,太后也好,宰辅也好,都不方便将之否决。

王安石、韩冈与程颢都因为天子弑父的公案而辞职。现在王安石、韩冈官复原职,是因为之后宫变中的功劳——尽管韩冈还多了一重手续。

而程颢授了崇文院校书一职,仍旧留在京冇城讲学。这是韩冈推荐,倒不是为了让新党多一个靶子,而是看在过去的情分上,也是因为有足够的自信。

如果要重开经筵,到底是韩冈三人重归原职,还是另选贤能,这都是需要考虑的。

但最关键的一点,还是皇帝本人身上的问题不能避开。

太后万一不豫,又有谁能阻止赵煦出面听政?

这个问题,很早就困扰着韩冈。

不管怎么说,赵煦的皇位是他保下来的。可指望皇帝这种生物会感恩,那就是太过愚蠢了。韩冈从来不相信身居高位者的人品,他们总能找到各种各样的理由来做他们想做的事。

向太后的身体情况要重点关注。万一太后身体不豫,她手上的权柄自然会旁落。但韩冈不可能容许天子的生母朱太妃听政,而赵煦出面听政,更是危机重重。

等到赵煦再大一点,向太后的身体不再如今日这般安稳。王舜臣和李信两人里面,至少得有一人留在京冇城中——就是王厚和赵隆,韩冈都不是那么有把握。

不过从另一角度来看,害怕赵煦亲政的人,宫里宫外都有一大批,不独韩冈一个。而认为赵煦没有资格做皇帝的,世间更是多。

赵煦的情况早就向天下公开,赵煦之所以不被废掉,还是看在被他误杀的先帝的面子上。要不是他是先帝赵顼唯一的血脉,早就废掉了。到时候他若是,韩冈手上有了大义的名分,有些事还是能做一做。

所以现在,还是做圆满了再说,让小皇帝学一学儒家经典,也不会有什么问题。

在内东门小殿中,一问一答,被留了近一个时辰,韩冈出来时抬头看了看天色,都已经是黄昏了。

他叹了一口气,这还算是休沐吗?

当然不能算是休沐,等到他回到家中,看到一封急件,立刻就愤怒了起来。

亲自过来唤韩冈吃饭的王旖,看到丈夫脸上的表情,就暗暗的叹了一声,合上书房的门扉,悄无声息的退了出去。只留下了韩冈,脸色铁青的盯着六路发运司送来的奏报。

——今年年前最后一批纲运,损耗量超过了一成。有鉴于此,六路发运司请求朝廷明年在南方六路加征,以保证纲运输送足额抵达京师。

在薛向倒台之前,六路发运司的工作一向做得很好,纲粮损失率已经很多年保持在百分之三到五的水平,但从今天三月开始,纲运的损失率一路上涨,直至今天的百冇分之十一。

韩冈其实从一开始,就不是很看好今年六路发运司的工作。

就算正常的升迁都无法避免‘人亡政息’,而薛向更是因为叛乱而得罪,他在六服发运司中留下的种种制度,如何能保留的下来。

韩冈可是听说,薛向曾经留下的碑文、匾额都给清除了——这不是一处传来的消息,关西、河北等薛向任职过的地方都有。

而前些日子,韩冈在做白马知县和开封府界提点时的幕僚魏平真,三年扬州军事判官任满回京,路上借道汴水,坐上了官船。因为年纪大了,又不担心走慢了没有好缺,一路便是走两日,歇一日,顺道看看风景。

据他所说,泗州的六路发运司衙门如今正张罗着要搬家,只因为里面薛向留下的痕迹实在太深了——这是魏平真从当地驿馆里听说的消息。

而新任六路发运使请求将衙门从泗州迁至扬州的章疏,前几天还在韩冈的案头上放过,上面列出的种种理由,倒是说得头头是道。不过韩冈、韩绛、张璪三人商议了一下,然后驳了回去。劳民伤财不说,泗州在汴水纲运中的地位也不是扬州能比的。

在这样的情况下,还指望他们能维持薛向留下的善政?

当年因为船工与押运的士兵联手干没纲运物资,纲船时常报损,就是在只有六尺深的汴水中,都能屡屡上报大风倾覆船只,薛向便受命主持汴河水运。在他的主持下,将官船和民船同时编为一纲,进行发解输送,抵达京冇城后,会对比官船和民船的损失率,如果多于民船,押送纲粮的官兵与船夫就要受责。

由于纲船在汴水中有着航行优先权,不论是载人还是载货的商船都愿意被编入进来。这就让那些奸猾贼子不敢有所动作,使得纲粮的损失率大幅下跌。

所谓靠山吃山,靠水吃水,六路发运司中的老鼠一个比一个滑溜,只有薛向那等深悉情弊的老人,才能一眼看破他们的伎俩,并时常都在京中盯着他们的一举一动。

现在没了薛向,一切就都回到了十几年前。

韩冈摇摇头,拍了拍奏章。

明天,他要就此事与韩绛、张璪好好的商量一下。
