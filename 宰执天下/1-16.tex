\section{第九章 闹市纷纷人不宁(下)}

韩冈义正辞严,声音也大得足以让整条街都听见。当着街上百多人的面,被人揭了老底,黄德用的那颗大瘤由红变青,又由青变红。发狠了半天,终究还是不敢让跟班上前把站在眼前大放厥词的村措大打个臭死。身为县衙班头,当街殴打士子,这等横行霸道之举,其实是犯忌讳的。光天化日之下,这等干犯律条的事黄德用却也是不敢做。除非能找到一个说得过去的借口,那时才是想怎么处置,就怎么处置。

“好!好!好!算你韩三有胆色!……就看你能硬到什么时候!”

黄德用也不知道横渠为何物,只是被韩冈激得怒极反笑,也不再多说,一把推开围观的众人,转身便走。

“黄班头好走,韩某不送了!”韩冈对着黄德用的背影,遥遥的把话送了过去。

刘三见主子走了,也急急忙忙的跟了上去,走时还不忘丢下一句狠话:“韩三,你记着!”

韩冈哈哈大笑:“韩某记性虽好,但小喽罗我可记不住!”

韩冈俏皮话伴着刘三狼狈而走,引得四周观众又是一阵哄堂大笑。在秦州城中,黄大瘤的人缘显然不好,看到他和他的跟班受窘,开心的占了绝大多数,却没一个出来为他们说话的。

听见身后的笑声,黄德用面色越发的狰狞。他本打算先困住韩家来应付差役,让韩千六不得不卖儿卖地,最终将人和田产自个儿献上来,而不是下死手去硬抢。毕竟用这等绝户计去谋夺他人田产家眷,也不是什么光彩事。韩冈好歹也是个读书人,若是真的闹到衙门大堂上去,强压下去虽然不难,但少不得要麻烦到陈举陈押司。

不管怎么说,黄德用是不想惊动到陈举这尊大神的。今天听说韩冈老老实实的来服役,本以为几句话把没见过世面的少年人给吓住,不闹出大动静就把人和田弄到手。但现下给韩冈在街头上一阵耍闹,陈举又怎么可能不知道。黄班头脖子上的大瘤红得发紫,显是气急败坏。他面目狞恶,发狠道:“区区一个村措大也敢在俺面前抬着头说话,也不看看俺黄德用是什么人物!到了这秦州城里,是条龙得给我盘着,是只虎也得给我卧着!”

目送着黄德用一班人走远,韩冈向着周围叫好声不绝的闲人们拱拱手,转过身进了普修寺中。

跨入寺内,韩冈脸上笑容难掩,尽管方才在街上只有百多人见识到,但至少他的名字应该能在两三天内传遍整个秦州城。

只是普修寺的住持和尚却一脸忧心,“韩檀越,你怎么硬顶那黄大瘤。”道安和尚快七十了,乃是胆小怕事的性子,“他是陈押司的亲信。陈押司在秦州城可是一手遮天的,任谁也开罪不起!”

“惊扰师傅了。”韩冈冲道安作了个揖,道:“只是这等小人须让他不得。否则他得寸进尺,却是更为难制!”

老和尚摇头叹气,韩家老三别的都好,就是性子太烈了。小时候狂傲一点那是没见过世面的夜郎自大,听说这两年在外游学,怎么还是这个脾气,“年轻人的脾气太刚烈不是好事,忍他、让他、不要理他,这才是长远之计。如今闹起来,事情怕是会难以收拾啊。”

韩冈低头唯唯逊谢,心下冷笑:‘我只怕事情闹不大!’

他当着街上近百人的面跟黄大瘤撕破脸皮,此事怕是到了今夜就能传遍城中。而他韩冈身为横渠弟子的消息,也同样会传入有心人的耳中。黄大瘤见识少,不清楚韩冈口中的横渠先生究竟为何方神圣,但秦州城中总会有人知道的。

韩冈师从张载两年,见过的官宦子弟为数众多,很清楚他的老师在关西拥有什么样的人望。与张载弟子比起,黄大瘤又算得上什么东西!?韩冈方才其实根本不需要刻意激怒黄大瘤,只要设法把他自己的身份传出去,多半就会有一两个官员看在张载的面上,帮他脱离现在的困境。

可最大的问题还是在这个‘多半’上!韩冈最不喜欢的就是将希望寄托在他人身上。万一没人帮忙怎么办?万一帮忙的人出手迟了一步,韩家已经被逼得卖地卖女又怎么办?所以韩冈只能选择把事情闹大。声势闹得越猛,他横渠弟子的身份传播得也就越快、越广。黄大瘤毕竟只是小人物,事情真的闹大了,怕是他自己都要退缩。说不定他背后的陈举也会投鼠忌器,反过来整治黄大瘤和李癞子。

想到这里,韩冈不禁暗叹,也就是在举目无依的秦州,若是在长安,根本就不会有这么多麻烦。哪个士子会眼睁睁的看着自己的同学受小人之辱?就算关系生疏,但同窗就是同窗!且少年人容易激动,只要几句话就能挑拨起来打抱不平,对付起黄大瘤、李癞子之辈,实在太容易不过。

又转回厢房中,韩冈有些疲累的躺了下来。前面已经把事情做了,就等着看看效果究竟如何。

……………………

“想不到这书呆子倒是硬气。照我说,不如把他安排到德贤坊的军器库里去好了。”

“刘显!监德贤坊军器库是什么样的差事,给了韩三那措大?你是帮俺还是气俺?!

成纪县衙的一间偏院中,本是两人相对而坐。只是黄德用现在大怒跳起,几乎要指着对面的户曹书办刘显破口大骂。刘显也不理他,只端起茶盏慢慢喝茶,韩冈早间去户曹缴还征发文书时,是一副只知道之乎者也的书呆子模样,黄大瘤竟然对这等穷措大气急败坏,让刘显觉得很好笑。

见刘显气定神闲,黄德用慢慢冷静下来。他眼前的这位四十出头的清癯书生可是陈押司的谋主,不动声色便能致人于死地,不然自家也不会找他来商量。“究竟是怎么一回事?”

刘显放下茶盏,凑了过去,压低的声音透着诡秘:“你可知道,经略司的王机宜提议要重新检查秦凤路各军州军备的事?”

“王机宜?李相公不可能会答应吧?”黄德用并不知道越俎代庖四个字怎么写,但他能看得出王机宜如此提议,可是有着侵犯经略使权力范围的嫌疑。

“不,李相公已经点头同意了。”

黄德用闻言一奇,问道:“不是听说李相公跟王机宜合不来吗,怎么又同意了王机宜的提议?”

刘显笑道:“新官上任三把火,李相公来了秦州已有半年,这也是应有之理。何况李相公是秦州知州,有机会对另外的四州一军指手画脚,他怎会不愿意?再说了,就算有怨声,也是王机宜的提议,须怨不到李相公的头上。”

秦州知州按惯例是兼任着秦凤路经略安抚使一职,在军事上有权对秦凤路辖下包括凤州在内的几个军州进行指挥,所以秦州知州的本官品级往往比普通知州要高上几级,也时常被人尊称为经略相公——相公一词在宋代最为贵重,官场上的正式场合,只有宰相才能如此称呼,但在地方上,路一级的最高长官有时也能享受到——不过平日里,秦凤路下面的另外那四州一军,对秦州知州李师中的话,却是爱答不理。能有机会找几个不听话的同僚的麻烦,李师中岂会不愿?

刘显继续道:“既然是李相公下令,秦州自是要排第一个。再过几天,等李相公从东面回来,州里各县各寨便都要开始检查,你以为成纪县会排在第几个?”

黄德用遽然站起,神色甚至有些张皇。他先探头出去看看门外,而后才返身回来,压低声音问道:“还是用七年前的那一招?”

刘显笑得风清云淡,低头啜了口茶汤,方慢悠悠的点头道:“这样最是干净利落。押司也是这般想的。”

黄德用有些担心:“县中不会有事,但州里会不会查下去?李相公可是个精细人。”

刘显笑着摇头,道:“经略相公去了陇城县,陈通判也刚刚罢任,其阙无人补。现在州衙里是节判【节度判官】掌兵事,节推【节度推官】掌刑名,知录【知录事参军】掌大小庶务,其权三分,你说他们哪个能管到成纪县中来?等到李相公回来,该死的死了,该烧的烧了,人证物证又早已备齐,他能做的,也只剩定案了!”

说完,刘显端起茶盏又啜了一口,一举一动都摆足了士大夫的派头。轻易的完成了陈举交给她的任务,顺带又能从黄大瘤这里捞上一笔,刘显心情很放松。只是他得意之余,却忘了再细问一下黄德用在普救寺前,韩冈到底说了些什么。如果让他知道韩冈的老师是横渠先生张载,恐怕就笑不出来了。

“好!”黄德用啪的一声重重拍了下大腿,狞笑着:“今晚俺就让刘三带上两个人去德贤坊,帮押司把事办了。顺便给韩三点教训。看他明日是杀到州衙里,还是到州衙里被杀!”

