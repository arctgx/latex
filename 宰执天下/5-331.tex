\section{第32章 金城可在汉图中(十)}

刚刚闪过一辆载着一对小儿女的独轮车,身后就窜了上来几头驴子。

拿着马鞭抽开身前的混乱,折可大望着前面更加漫长的一条街巷,怀疑起自己到底能不能及时赶到府衙去。

现在街道上尽是托儿挈女逃进城来的百姓。纵然幸运的逃过了辽军的劫掠和杀戮,带着仅存的家当逃进了太原城,但偌大的城市却没有他们安身的地方,绝大多数只能在别人的屋檐下生活。

这天还冷得很,一夜过去就是几十条人命。折可大这几日亲眼看见几辆马车走街串巷,将无人收拾的尸骸一条条的捡起来送去化人场。化人场就在太原城的西南角,现在还正冒着烟。这就是现在的太原城。

纵然经过了多次战争,可折可大还没有看过如今这幅凄惨的场面。

“直娘贼的。真是兵荒马乱啊!”就在折可大前面,一名身穿青色官袍的青年正冲着几名挡道的百姓发着怒,手中的马鞭举得老高,前面牵马的家丁也是连推带搡。

只看背影,折可大便认了出来。那是经略司衙门中的机宜文字张俭,他的同僚,虽是文官,却是武将的脾气。不过荫补出身,也真没人把他当成正经的文官看待。

“火气别这么大。这辰光,发火也没用。”折可大在后面突然开口。

张俭冷不丁的被吓了一跳,回头见是折可大却是转惊为喜:“嗨哟,折阁门原来你在这儿啊。”

“有事找在下?”

“是有些事。”正说话,一名骑兵擦着鼻子从张俭身前窜过去,猛地就将下文给他砸进了肚子里。

张俭脸色先是发白,继而又由白转红。被吓得气得不成话,红得发紫的一张脸:“直娘贼的,是赶着投胎去啊!”

指天骂地的咆哮了几句,在周围的百姓看热闹的眼神中,张俭狠狠的咬牙:“王经略再不出来弹压,太原城不要辽军来攻就能破了。”

‘王.克臣有这个能耐吗?’折可大冷哼了一声,从鼻子中哼出的满是不屑。太原守不住,第一原因就是在王大经略身上。

只是折可大现在也只能腹诽,根本改变不了什么。

一年前,折可大被调到了太原府,在经略司中担任闲官至今。而原本在韩冈幕府中的堂兄弟折可适,则是安然在丰州做官。

丰州建立之后,立功甚多的折家在河外势力大涨,也更加根深蒂固。下任家主的折可大转调太原,名义上是以立有殊勋为由,实际上还是人质的成分居多。

折家对此没有表现出任何不情愿的态度,这是为了让朝廷安心。不过折可大本以为自己能被调入京城,被安排在太原倒是出乎意料。不知道是不是朝廷不想做得太过明显,还是觉得太原就足够了。

不管怎么说,人质就是人质,不用指望说出来的话有人听,反正不受重用就是了。

看得出藏在折可大眼底的不屑,张俭收起怒气,改怒为叹:“王经略现在就盼着令尊能早日领兵前来,若有五千麟府精兵坐镇,不用上城墙,城中军民就能安心了。”

“在下都写了亲笔信了,还能怎么样?”折可大摇头,“王经略要调动麟府军,我折家世代忠良,岂有不应之理?我写信本是多余,不过是让经略相公放心罢了。”

“折阁门,依你之见,太原城能守得住吗?”张俭见老于战阵的折可大都不看好所谓的太原城防,心中就像是十五个吊桶打水,七上八下。

“我就怕他们不来打太原。辽贼真要跑来围着太原城,只要城中不乱,还是能守一阵子的。怕就怕辽贼跑去攻打榆次县,堵了井陉道,河北的兵马几百里路赶回来,迎头对上,结果可不好说。”让过一队匆匆跑过街道的士兵,折可大又冷着脸哼了一声,“都不知道乱跑个什么?!”

“阁门若要回家去,还是绕道北门的好。”张俭陪着折可大往边上让,又道,“可就算井陉道被堵上了,好歹还有西军和京营啊。”

“京营烂得边境上的乡兵都比不上,西军更是全都在贺兰山下,哪里能来得及。”又躲开一辆马车,折可大烦躁的扯着襟口,却没忘了说自家逗留此处的原因,“在下正在等人,等人到了就走。”

等人?还没等张俭想出一个眉目,就见一人穿过街道往这边过来。张俭认识,是折家的门客,在太原辅佐折可大左右,好像还是折家的亲戚。

折可大丢下了张俭,上前半步:“打探清楚了吗?”

“清楚了,清楚了。”那门客连连点头,遗憾不已,“石岭关丢得冤枉啊,吴都监着实该死。”

“到底是怎么回事?!”

“吴都监领兵到了石岭关后,便将忻州的人马赶到了关城上,自己则坐在烽火山城。大郎你说,他该不该死?!”

折可大愣了一下后摇头:“麟府军在河东还不是一样被人视为另类?太原兵马跟代州的合不来也不是一天两天了。”

“石岭关都丢了,哪是一句合不来?!”

“吴宝当然该死。”张俭凑了上来,他比折可大要早知道缘由,亦是发恨,“吴越同舟都能互助互济,吴宝这把自家人当仇人看了!他若是逃回来,当天王经略就能斩了他祭旗。”

“杀了又能如何?石岭关都丢了。”折可大摇头,这还不是朝廷自家弄出来的事?

天下禁军六十万,绝大多数分布在京畿、河北和陕西三处,时日久长后,军中便出现了三个不同系统的山头。但大山头之下还有一个个小山头。

除了陕西因为年年战乱,内部调动频繁,其他地方,禁军更戍之法早就停了,各地的禁军基本上都是驻泊禁军,比如河东各部禁军,太原核心之地有一部,北部边境诸军州有一部,以及河外的麟府丰,多有在当地驻扎了五六十年的情况,军中士卒基本上全都是本地出身,底层军官也都是在近处调动。

加上河东在五代接连出过开国天子,朝廷也是有意无意的破坏河东军中凝聚力,尤其是太原,毁城分兵,被提防得极严——否则就不会有石岭关和赤塘关这两座相距十余里的太原门户,却分属两州的安排——自然而然的就出现了太原、代州和麟府三个不同的系统。

石岭关本身隶属忻州,驻扎在关内的兵马自然全都是忻州兵,真要说起来,忻州军是属于河东军中的代州一脉,与太原一系似近实远。

这点龌龊事,河东军中谁不知?

石岭关前关后城,北面是旧关城,而关南的烽火山城,是仁宗年间修建。地势之险要不逊关城。可就是因为有了新城,旧关城便没有再修葺,那可是一条破烂的防线。而且关城守军的家泰半都安在新城内外,旧关城若破,看到家中的父兄子弟被太原来的都监害死,石岭关的军属如何会不恨坐在城中的太原军?

“韩学士若在,断不至于如此!”门客毫不客气。

“是韩枢副了……”折可大感慨连连,当初在韩冈麾下杀辽人、杀党项,那杀得才叫一个痛快,哪里是现在可比。

门客看着拥堵混乱的街巷,人在叹气:“枢密相公若能早来,何至于满城兵荒马乱?”

“那也要到了才好。”

“可好歹是来了。等到韩枢副到了太原,就不用听王经略乱指派了。”张俭不是王.克臣的人,说话毫无顾忌。

折可大摇头:“能等得到再说吧。”

辽军已经进了太原府界,韩冈若是轻车简从,运气不好就能成了辽人的战利品。若是要带大军北上,那就还要等上半个月才能收拢各地的驻军。

三天前,韩冈的亲信和朝廷的诏令同时到了太原府。韩冈擢枢密副使,出任河东制置使,统辖河东军事,这一任命在太原城中引起了极大的震动。

对于太原军民,自然是欢欣鼓舞。韩冈离开河东不过一年,在太原任职的时间也不到两年。作为河东帅,他军事上虽有成就,可他作为太原守,在治政上其实并没有太用心。但太原上下偏偏对他极为怀念,多有人将其视为归宋以来最出色的太原郡守。更别说论起治军克敌,方今朝中也没几人与他相提并论。

另一方面,对于看得更深的官员们来说,在韩冈走马河东后,大半个枢密院都在外面的现状,绝不是一件好消息。陕西、河东、河北,一位枢使统领一路,立国之后,就没看到这样的局面。显而易见,现下的局势虽不能说开国以来未有,但也绝对是澶渊之盟以来最危险的一次了。

不过放在眼下,有一个军功煊赫的名帅坐镇,决不是区区王.克臣能比。

折可大以军中望气法观府衙,帅府行辕上的五色云气所聚,怎么看都是一头猪!

“唉。”

他唉声叹气,即盼着韩冈早至,也怕他在路上出了意外。矛盾的心思,此起彼伏,难以平定下来。
