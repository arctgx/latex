\section{第28章 官近青云与天通(24)}

一下否,两下是。

赵顼表达心意的方法,已是朝野尽知。

至于三下或以上,如果不是眼睛不舒服而正常眨眼,就是天子想要用韵书传达信息。

现在当然不会是前者。

坐在床沿的向皇后脸色难看的拿起韵书。她不知道她的丈夫到底在想些什么,虽然可以确定,绝不会是伤害自己和六哥,但他做出的选择,总是让人不痛快的一件事。

无论是之前的王珪,还是现在可能的吕公著,都让向皇后憎厌到了极点。

而且还没用——要保的王珪,成了众矢之的。新党不喜欢他,旧党不喜欢他,御史也同样不喜欢他。

若不是司马光做得太过分,修书修得老糊涂了,有了心疾,硬是要杀王珪,惹起了其他朝臣同仇敌忾,今天也不至于为了一个王珪,将半个御史台给赶出京去!

现在吕公著摆明了想做宰相——他都做到了枢密使了,看到相位空悬,肯定是想往上走一步,至于辞章什么的,向皇后再没有经验,也知道外面听起来冠冕堂皇的话根本不能做数。可眼下,她的丈夫会不会受到这位枢密使兼太齤子太保的蛊惑,向皇后还真的没有把握。

吕公著则站起了身,辞章依然拿在手中,很自然的移了两步,走到了天子御榻的不远处,能更加清楚的看见赵顼眼皮的动作。对于一名已经年过花甲的老臣来说,能看清五尺外天子脸上的细微变化,这是吕公著如今始终夹在鼻梁上的一幅水晶眼镜的功劳。

书页哗哗的翻动着,以韵书为媒介,一问一答,赵顼和皇后的对谈,比一开始时快了不知多少倍。

去声二十六宥——奏。

“奏?”向皇后眼前一亮:“官家!可是奏对?要招谁入宫来?!”

她连声问,很是急促。

吕公著在后垂下了眼皮,若不是在寝殿中天子身前不能放肆的话,他可就是要哈哈大笑起来。

以眼下的局面,怎么想以奏开头的词汇,都不会是奏对吧?

皇后分明是支持不住,想从外面找援军。

皇后的敌视让吕公著备生感慨,未来或许有些麻烦。不过再想起宰相身份,他就放心下来。垂帘听政的太后,也不可能下手处置宰相家门。若是她这么做了,新党的那一帮人,包括王安石、韩冈,拼了命都会将乱命给顶回去。

而且可想而知,就天子而言,他绝不愿意看到皇后太过偏袒臣下的某个人或某个派别。

垂帘皇后不能执中而立的危险实在太大了。直接卷入了臣子的交锋中,而不能置身事外,那么当朝堂风浪一起,也会被连带着拖进水里。

呵。

吕公著轻呼一口气。原本只有六分的成算,现在可就有八成了。

剩下的两成,那则是要看运气!希望司马十二将坏运气都带走了。

赵顼果然眨了一下眼,给了否定的答案,让向皇后的心沉了下去,不得不重新拿起韵书。

然后是下平七阳——章。

奏章。

“奏章?”向皇后回头看看吕公著,那本应该被垂下来的袍袖遮住的奏章,却被刻意的亮了出来。毫不掩饰的皱了一下眉,她转回来问赵顼,声音很冷:“可是吕枢密的奏章?”

吕公著期待着,水晶镜片后的眼睛眨也不眨的盯住赵顼,

可眼皮仍是只眨了一下。

“那是哪里的奏章?”向皇后抬起眼。现在就在眼前的床榻边,一张新置的宽大几案上,高高低低堆了好几摞从崇政殿和御书房中搬来的奏章,“可是床边的?”

赵顼尽管卧床不起,却依然为国事操心。每天都要听人宣读奏章,了解朝堂中发生大小事务,并不辞心力的指点向皇后该如何批阅。

他这么做,也是让外界明白,天子纵然病势垂危,神智依然不乱,若有什么小心思,最好收起来——可惜的是效果不彰。


而赵顼现在便眨了两下眼,对皇后的问话给了确认。他要的奏章,便在这里。

几案上的奏章四五堆、百十封,向皇后看着犯了难。

“官家……”她凑近了问,“是谁的奏章?”

赵顼的回答是上平十四寒——韩。

韩冈?!

吕公著眼皮一跳,脸色终于变了。

“可是韩学士……是韩冈?!”

一下。

两下。

……………………

当韩冈从崇政殿回到太常寺,已经是快放衙的时候了。

苏颂已经回了他的衙门光禄寺去。虽说那个衙门跟太常寺差不多,十天八天都不去,累积起来的公文平铺开来,也只能占去半张光禄寺中那面属于苏颂的桌案,但终究还是得每天绕上两趟。

过来与韩冈说话的是黄裳。

黄裳他现在被韩冈征辟为椽属,在编修局中整理甲骨文。这算是很轻松的工作,也正好可以让黄裳有时间复习应考,准备明年的锁厅解试,以及后年的省试——以黄裳的年纪,不能再耽搁了。

但今天黄裳不可能有心多说他手上工作的进度,简短的汇报了两句后,便问起了朝会上的事。

“虽然这么说有些过分了,但司马君实实乃自取其咎。”韩冈有些不客气,“辽人虎视眈眈,天子又病重如许,他身为太齤子太师,却不体谅天子心意,当有此祸。”

“那朝廷打算怎么做?”黄裳如今虽然是站在韩冈这一边,但对司马光这等闻人贤达,还是有着很深的景仰。

“还能如何?好歹是太齤子太师!已经决定赐予厚礼,让他回洛阳去了,绝不会让他失了体面的,倒是一干御史,就得出外了。”韩冈叹了一声,“希望他回洛阳后,能将《资治通鉴》继续编纂完成。同为修撰,为朝廷编修典籍到底有难,这段时间我是体会到了。司马君实在洛阳的确辛苦。”

黄裳默然点头,这对司马光来说,已经是现在的局势下最好的结果了。

“其实司马君实那边,本是有份人情在的。”韩冈又冲惊讶起来的黄裳笑着道:“不过不是对我,而对是整个气学。”

“气学?司马君实到底帮了什么大忙?”

“是先生的谥号。”韩冈说道。

张载的官位不到,没资格得到朝廷的官谥。当张载病逝之后,张门弟子聚起来打算给张载上一个私谥,以表对张载的纪念,也算是人之常情,亦多有先例。从魏晋以来,史不绝书。

“但这不太好吧。”黄裳皱眉想了想,摇头道:“横渠先生天下知名,若请谥于朝廷,或无不可,私下奉谥,反倒让人小瞧了。何况横渠先生乃大贤宿儒,欲复三代之礼,援引汉魏以来俗例,或违横渠先生平生之愿。”

“正是这个道理!”韩冈一击掌,“所谓‘贱不诔贵,幼不诔长,礼也。’谥自天子出,做弟子的怎么有资格给师长赠谥……司马君实也是这么看,当我的几位师兄写信去请教伯淳先生此事是否可行,伯淳先生拿不准,就又向咨询司马君实咨询,他便写了一封信来劝阻。”

“原来如此。”黄裳点了点头。日中黑气,月中深影,总是最为显眼的。如张载这般名儒,他的弟子若是做了违反礼法的事,必然逃不脱士林的嗤笑,也会成为其他学派拿来攻击的武器。

“所以我等气学门人,得感谢司马君实写信拦住了这件糊涂事。”韩冈又说道,“在我从广西回来后,知道了此事,曾写信谢过司马君实。后又上表为先生请谥,不过当时的情况,勉仲你也是知道的……”他苦笑了起来,“当时我与新学正争于道统,天子看重新学,奏章上去后就没了回音,所以就留了这番心事到现在。”

韩冈说完,端起茶杯喝了一口。顺势向上看着屋顶,也不知道自己做下的那番准备到底能不能派上用场。

……………………

韩冈这两天递上来的奏章已经被翻了出来,其实就在最矮的那一叠中。

同在一叠的,有河北对辽使南下行程的奏覆,有河东对辽国西京道冬季兵马调动的侦察情报,有甘凉路上报的军屯总结,由此可见赵顼对韩冈奏章的重视。至于几案上最高的两叠,则都是弹劾王珪的弹章,数目实在是太多了,没办法堆成一摞,只能一分为二。

韩冈的奏章,被翻出来的总计有三份。区区两三天的时间,他便借用翰林学士兼资政殿学士的资格,一天一份的直接将奏章递到崇政殿的案头上。

这个频率放在平常那是足够惊人了,可是眼下则是显得泯然众人。许多朝臣,眼下都在拼命的往上递奏本。而且有很多人跟韩冈一样,都是通过各种渠道,尽量绕过两府。赵顼床边的奏折,只是其中的一部分而已。

向皇后对这几份奏章有些印象,但极为模糊。她只记得韩冈连着几天都有奏本。在奏章没被翻出来之前,向皇后怎么也回忆不起来韩冈在奏本中到底说了什么,等到翻出来一看,才想起这两天都看过。

并不是什么很要紧的内容。否则以韩冈的身份,他所议论的要事,向皇后自问,必定是能记住的。

不过向皇后对吕公著很是避忌,翻出来后看了一看,并没有念出内容来,而是很简单的问着赵顼:“官家,可是这三封:《本草纲目》编修局请刊发期刊;弛千里镜之禁;还有为张载请谥?”

立刻,向皇后就看到了赵顼眨了两下眼睛。

正是!

而几乎在同时,她身后也传来了啪的一声响,是吕公著手上的辞章落到了地上。

向皇后回过头,看看地上的奏章,又看看吕公著震惊莫名的表情,随即便瞪大了眼睛,心中亦是疑惑难解:

一贯宰相风度的吕公著,怎么会失态到这般模样?

