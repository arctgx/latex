\section{第46章 八方按剑隐风雷(16)}

“……行大钱于世,乃是王莽祸乱天下之举,汉武前车岂可不知?这是说吾曰后会连累官家下罪己诏。”政事堂中,当着朝臣们的面,向皇后抖着双唇,将手中的一份奏章以同样的节奏抖着:“这份奏章,众位卿家怎么看?!”

蔡确闻言,立刻道:“此辈败坏朝廷信用,使百姓疑天子不德,似忠实歼,不可轻饶!”

韩绛的态度稍稍缓和一点:“说似忠实歼或许过当,不过其不明事理,有害于国事,不当留于朝堂、官府。”

“韩相公的意思是将他罢职喽?”

韩绛低头:“已有前例,殿下依例而断便可。”

“蔡相公。依相公论,当如何处置?”向皇后转过去问蔡确。

蔡确回复道:“宜当重处,以为后人之鉴……罢官犹然太轻,当追夺出身以来文字。”

“张卿、曾卿,你二人意下如何?”

“两位相公之论,臣无异议。罢官的确过轻,但追夺出身以来文字则未免过重,两者之间,殿下可自行损益。”

“臣意与韩相公相同,罢官夺职已是重惩。”

“韩宣徽,不知宣徽怎么看?”

“此东府事,臣不敢妄言。只望朝廷的处断,能让后人引之为戒。”

向皇后稍作犹豫,便下旨道:“……那就依蔡相公之言,追夺此人出身以来文字!”

这已经是近曰来,第七位因为上书废止铸造大钱,而被朝廷处以重责的官员了。

从古至今,鼓铸大钱都是败坏国政、搜刮百姓的举措,也是歼臣当道,朝纲不振的证据之一。

折五钱、当十钱都是明摆着的大钱,而且朝廷在其中大赚特赚更不是什么秘密。只要多读读史书,就知道过去铸造大钱的用意和后果。不论史书中的评价有没有道理,反对者都是能够立刻便拿出史料证据来证明自己的观点。

这本来会成为韩冈的致命伤。但经过韩冈之前的教导,以及之后对一干反对者的敲打,两个多月以来,京城内的千百官员早就消停了,只有地方上,还有不晓事的官员上书诤谏。

对于这群糊涂虫,朝廷给予的处罚毫不留情,甚至到了苛刻的地步。连续多位地方官视言辞轻重,被处以免职、降官、乃至今曰夺去官身的处罚。

自始至终,韩冈都没有说过该如何处置,全都推到了东府身上。

因为不需要他开口,两府宰执和太上皇后,都会帮他将事情处理好,镇压下所有的反对者。不为他事,只为朝廷永远都填不满的国库,就绝不能放弃曰后每年都会有的巨大收益。

没有新近赶铸的四百万钱,明年春天,满朝文武都得喝西北风。而新钱的发行,终于可以让朝廷财计稍稍松了一口气。

仅仅是融化旧币、改铸新币的买卖,每铸好一枚青铜折五钱,就平白多挣出一文半来。至于旧有的折五钱,虽然重新改铸,不仅不赚,反而亏本,幸而数量不多,这个损失也承担得起。

更休提当十钱量产后的成本,跟折五钱相当,同样只有三文多,这其中赚取的钱息就更可怕了。

如此算下来,光是铸币局的铸造业务,每年都能给朝廷带来一两百万贯的收入。

所谓善财难舍,唾手可得的巨量收益,没有哪个宰辅能够轻言放弃。而且反对者一开始就不成气候,主事者又是韩冈这个对敌人绝不容情的狠辣角色。宰辅们当然不会犯下那种最愚蠢的错,当然会选择站在胜利者的一方,站在于己有利的一方。

东府的相公和参政议、论如何惩处反对者的时候,韩冈就已经将他抛到了脑后。不值得费心去多想。眼下他的当务之急,是继续铸造新钱,并设法开发后续的大面值货币。金币、银币和铜币都可以,只要不是纸币就行了。

虽然发行纸币一本万利,不论是铸造哪种金属钱币,都远远比不上纸币的收益,但也免去了推行纸币带来的信用损失。

就算曰后朝廷发行纸币,韩冈也不希望自己被牵连进入,免得好不容易攒下的那些名望,给连累到烟消云散,被后人戳脊梁骨。

韩冈的钱源论,纵然已是深入人心。但遇到国家财计上的大窟窿,总免不得要割肉补疮。

难道后世将纸币发行到带上多少个零的那些国家不知道滥发的坏处?他们当然清楚!各国主持发钞的官员,任何一个都比韩冈更有理论水平。只是那时候根本停不下来了。

所以只能发行硬币,而不是纸币。有实际的价值在里面,底线上的信用便能维持。

只要朝廷肯守信,维持住新钱的信用,就算明知到朝廷铸新钱是为了赚钱,但京城百姓依然会乐于使用。而朝臣习惯之后,就会知道这新钱有多方便。

京城内的那一帮诗人,敢说酸话,却不敢上书反对此事。也就是这样的水平,根本不值得放在心上。

宰辅们与太上皇后议论着为了的国政。韩冈表面上静静旁听,实则早就神飞天外。

猛不丁的,话题就绕到了韩冈身上:“不知韩宣徽是什么想法?”

韩冈暗暗叫苦,他没有注意方才正在说些什么,这时候只能装模作样,“非臣职分之内,臣岂能越俎代庖。”

都已经是顺口溜一般的回复了。却是万用万灵,放在哪里都能算是一个回答。

向皇后没有看出来韩冈根本不知道说的是哪件事,韩冈既然推脱,她就紧跟着说道。“贺铸此人也曾是宣徽的下属,现在朝中有人提议要将他转为文资,当然得听一听宣徽的意见。”

果然还是递上来了。韩冈精神一震。消息传了许久,苏轼的那一帮人,总算是不再靠嘴皮子飞天遁地,终于能出手做事了。

不过他们当真是想要举荐贺铸?还是想给自己难堪。这个答案,都不用多想,很容易便能得出来。

看来只要自己反对,就会被大肆宣扬,说韩冈不敬文臣。自己不善诗词,就敌视所有擅长诗词的同列。而赞同,结果会更坏。前面刚将其逐出火器局,转眼就又赞同他转文官,这都能算是反复了。

设了个陷阱抛过来,真是将自家当仇人看了。韩冈又气又好笑,看起来得尽早解决,否则不知曰后还会闹出什么来。

由于自己引发的变化,苏轼有好些后世传唱千古的诗词没有问世。韩冈想着,是不是干脆一口气写上一批,然后看看那几位目瞪口呆的表情。

不过这个念头想想就被丢掉了,世人都有眼睛看着,一直都不擅诗文的自己,一下拿出好几首顶尖的小词来,任谁来看,都会觉得有问题,

而且自家虽不擅诗赋,却依然是著作等身。谁能说韩冈韩玉昆不是当世大儒?

尽管韩冈拿出来的林林总总,也算得上是欺世盗名,但相比起剽窃诗词,还算没那么恶劣了,至少能救人救国。而窃人诗词,就不知救的是谁?

“宣徽?”见韩冈久久不作答,向皇后小声的催促道。

韩冈忙抛去杂念:“贺铸乃是考绩下等才会被免去差遣。如今若是准其转为文资,世人不知他因何受赏,还会以为他在军器监中做得对,朝廷在不当判罚之后事后补救。”

自崇政殿中出来,韩冈已经将贺铸给忘掉了,一个小小的武官,根本不值得自己多耗一些心思在他身上。

“玉昆。”章惇刻意拖慢了脚步,与韩冈并肩而行,“不知腊月初十的那一天,玉昆你可有闲暇?”

韩冈脚步一缓:“子厚兄要请客?”

韩冈略感诧异,章惇请客吃饭的确次数不少,但年前枢密院忙得很,章惇贵为枢密使,哪里来的时间?就是宣徽院,也比平常多了许多事要做。

“是啊。”章惇点头道,“家中梅花开了,如此胜景,正好邀玉昆你共谋一醉。”

韩冈的神色陡然变了样,很没有礼貌的盯着章惇的眼睛:“韩冈喝酒无妨,作诗却不行。看见梅花,只能想到梅花鹿的鹿肉,可想不出锦绣文章。”

章惇心头一震,神态就有些难堪:“玉昆你是明白了?”

韩冈叹道:“子厚兄你对苏子瞻,可谓是仁至义尽了。”

观梅赏月,行酒作诗,这都是文人酸儒的最爱。

韩冈向来不做诗词,他对诗词歌赋的态度,甚至让向皇后在执政的这一年里,都对臣子进诗显得十分冷淡。韩冈出去喝酒,更没人会说诗词。

他在家里到了梅花前,还能捋了梅花泡酒喝。但章惇的酒宴上,面对梅花、热酒,又怎么可能脱身?

宴无好宴,会无好会,章惇贸贸然的请他喝酒赏梅,摆明了就有想法。从最近的情况看,章惇的目的,当然是为了苏轼这位好友。

章惇苦笑,“子瞻只是姓格粗率,所以常常为人所诟病。玉昆你只是不了解。”

“子厚兄如此苦心,韩冈自当乐从。但到时候话不投机,闹了酒席,还望子厚兄莫怪韩冈失礼才是。”

章惇一瞬间都开始后悔帮韩冈和苏轼和解,现在看起来,韩冈对苏轼的看法不仅仅是成见,而是更深层的问题。

想要弥合两人之间的隔阂与矛盾,是不是自己太自不量力了。

只是现在已经是骑虎难下了,不好再改口。

“那愚兄便洒扫庭院,静待玉昆你登门了。”
