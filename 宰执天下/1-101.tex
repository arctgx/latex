\section{第40章 中原神京覆九州(上)}

【第一更,求红票,收藏】

夜色沉沉。

王安石此时早已无心于诗词,虽然元日所写的诗句已经传遍了东京内外,但当日踌躇满志的心情,如今已经不复存在。

他静坐在书房中,没有点灯,无星无月的夜晚,大宋参知政事的书房里,是一团不见一丝光亮的深黯。所有来拜访他的属官都给他拒之门外,吕惠卿、曾布、章惇、谢景温这些在变法上得力的助手都一样被拒之门外。

王安石只想静静的好好想一想,以求能想出一个对策。

就在今天,来自大名府的一封奏章,乱了天子赵顼的心,也让刚刚展开的变法大业的根基彻底动摇。

判大名府,河北安抚使,魏国公。

韩琦。

相三帝扶二主的韩琦韩稚圭上书天子,奏言地方推行青苗贷不守条令,有故意调高利息的,也有把青苗贷贷给城中的坊廓户的,种种不端,累及百姓,而且青苗贷本说是赈济百姓而为,现在却收取利息,是与当初抑兼并、赈贫困的初衷相悖,且官府逐利有失朝廷脸面,请求废弃青苗法。至于朝堂入不敷出,就请天子‘躬行节俭以先天下,自然国用不乏’。

英宗朝留下来的宰执官中,富弼反对变法、文彦博反对变法,张方平反对变法,欧阳修反对变法,到如今地位最高,声望最隆的韩琦终于明确的表明了自己的态度。

韩琦的反对,让赵顼犹豫了。他起用王安石变法,是为了平定西北二虏,是为了一扫百年积弊,不是为了与朝臣为敌,更不是为了祸害百姓。

王安石很无奈。

青苗贷的本质难道他没跟赵顼说清楚?早早的便说明白了!

就是为了充实国库,以便整顿军备。摧抑兼并的口号只是对外说的。但解生民困厄,‘不使兼并者乘其急以邀倍利’,却也是实实在在的效果。比起民间高利贷百分之百的年利,官府的青苗贷一期才两成,一年不过四分的利息,算是很低很低了。

若说地方官员在推行青苗贷时不守法令,该惩治的惩治,该斥责的斥责,又有哪里难做?若是青苗法本身有什么考虑不周全的地方,在施行中加以修正,难道还做不到?至于给坊廓户贷钱,只要有保人,只要能还得起,借给他又何妨?青苗只是个名字,不是说只能借给农人,城市里的坊廓户照样是大宋子民,让他们不受高利贷之苦,不也是理所当然的吗?

可韩琦就是反对!

韩琦什么想法?王安石不知道,但韩家在相州的事,王安石却是知道的。

韩家在相州世代豪族,权势熏天。相州的土地一多半都姓韩,相州百姓又有多少家不欠韩家的高利贷?韩家家业大,要用钱的地方多,每年的收入,田地的租佃是一块,而高利贷的利钱也是一块。但青苗贷一施行,每年十几二十万贯的高利贷利钱都会被官府取了去。韩家难道要喝西北风不成?

韩琦说青苗贷是为了扶贫济困,抑制兼并,不该收取利息,这样才能让百姓受惠。而与韩琦一样,执这样说法的反对者有很多。他们其实都是揣着明白装糊涂,看起来是为百姓说话,但实际上对朝廷毫无收益的法令怎么可能持续下去,真的按照他们说的来,怕是又有人会跳出来说是虚耗财税,恳请罢去。多少与国有益的法令就是这么被阻止的。

但这事王安石不能明白的指出来,韩琦的地位不同。英宗皇帝是他扶植上去的,就凭英宗不肯出席仁宗大奠之大不孝,若没有韩琦居中调解,如今的曹太皇说不定已经把英宗给废掉了。而今上登基时,韩琦又是以宰相身份,依遗诏辅赵顼坐上御榻。

相三帝扶二主,韩琦的功劳,不比前朝的郭子仪稍小,实实在在的定策元勋。韩稚圭在天子心目中的地位,朝野内外无人可比。王安石也自知不能相提并论,单是资历、人望和权威就差得太多。尽管就是因为这些功绩、人望、权威,使得韩琦不得不避忌出外,但只要他远远的说一句,东京城照样得抖上几抖。

如今在天子周围,还有谁不反对新法的?好不容易安排了吕惠卿为崇文院校书,在天子近前以备咨询。但据说吕惠卿的父亲最近身体并不好,可能过段时间他的第一号助手,便要丁忧归乡。

均输法得罪了京城里的豪商们,因为他们通常与宗室联姻最多,所以一并得罪了宗室。青苗法得罪了以高利贷为生的地方上的世家大族。农田利害条约还好一点,不过是鼓励地方修造水利,多多开辟荒田,可说不定在实行过程中,地方官员会摊派劳役和费用,还是会惹到一批地方世族。

太急了!王安石视线漫无目标在黑暗中游走,心中叹着,实在是太急了!一次过便捅了几个马蜂窝,如何不会朝野骚动。

可若不是年轻的皇帝心急,他又何必接二连三推出各项变法条令?一年颁布一条,有个缓冲的余地,方才是正理。

变法之要,首在得人。他王介甫仕宦三十年,沉浮官场,纵然不愿同流合污,却如何不知循序渐进的道理?让提拔起来的人才在历练中分出高下,辨明贤愚,这才是正道。但天子等不得,国库等不得,均输法、青苗法,农田利害条约,一桩桩法案颁行得如此仓促,不都是因为赵顼想快点看到成果,所以要尽速充实国库吗?

可现在好了,因为韩琦的一封奏章,赵顼便变了颜色。

王安石悠悠长叹,若天子不能坚持,他入朝两年来一番心血又是何苦?

如此下去,一切都要打回原形,就像仁宗庆历年间的那次新政,起得轰轰烈烈,去的悄无声息。范文正当时的人望并不在自己之下,意欲革新的意志尤其坚定,他一笔一勾的划去不合格的官员,连‘一家哭何如一路哭?’的话都说出来,欧阳永叔又抛出了《朋党论》,以对抗吕文靖【吕夷简】一派的指责,为了推行新政,他们得罪多少人?但最后,仁宗皇帝退缩了,还是一切成灰,出京的出京,贬职的贬职,烟消云散,仿佛一场噩梦。

说起来,如今变法的危局,其实就是庆历新政的翻版。如果不能度过这道难关,二十年前范仲淹的失败和落寞,便是日后他王安石和他的一众助手的下场。

王安石绝不甘心!

他等了几十年,好不容易才等到一个实现心中抱负的机会,哪能就这么化为泡影?

但局势危急如此,以韩琦为主的反变法派已经磨刀霍霍,要想斗败他们,只有破釜沉舟一途!

抬手从书架上抽来一片纸,王安石提起了笔,开始草拟起自己的请郡出外的辞章。

他要辞去参知政事之位,到地方上去——如果赵顼不能给他一个满意的交待。这是以退为进,也算是给天子的最后通牒。

没有犹豫不绝的余地,王安石必须让皇帝从他和韩琦之间作出一个选择。就让天子自己衡量一下好了,究竟是继续推行变法,以求富国强兵,还是按照韩琦这些老臣的想法,狗苟蝇营的拖下去。

这就是王安石的性格,言不苟志,行不苟合。一如他早年在写给友人的一封信中所言——‘时然而然,众人也;己然而然,君子也’。

世人说他是集天下人望三十年。这不过是因为他屡次拒绝入京担任天子近前的侍从官,而留在地方上的缘故。不爱名位,性格清介,儒生们都在夸赞这样做的王安石。

不爱名位?

错了,他王安石爱名位!只有拥有了名位才能实现自己的抱负,实现自己的理想。他不爱名位的种种表现,只是过去的三十年一直没有得到一个一展才华的机会。只有天子支持,他才会坚持。

辛辛苦苦写了万言书,天子也不给个回复。所以当王安石看到仁宗皇帝无法坚持变革朝政,无法实现自己的愿望,自担任过度支判官后,他便拒绝再担任修起居注一职。

修起居注的任命,是记录天子的言行,天天都能面圣,是晋身的快车道。平常官员照规矩推辞个两三次便会接任,司马光也只辞了五次。可他王安石硬是辞了九次,甚至为了躲避传诏的内臣而避身到厕所里,这不是待价而沽,不是欲擒故纵,因为他实实在在的不想做。虽然最后还是接了下来,却是因为可以转任知制诰的缘故。跟在天子身边记录言行,王安石实无兴趣,但能够成为为天子草诏的知制诰,可以封还词头,拒绝草拟错误的诏令,直接参与朝政,这样的职位王安石不会拒绝。

但无论是接下来的知制诰,还是后来再次转任的纠察在京刑狱,他都没有作出什么建树。仁宗末年官场上的死气沉沉,让王安石觉得窒息。不能实现自己的理想,高官厚禄又有什么意义?趁了母丧离开京师。寻常官员回乡守制,都盼着能夺情起复,没几个甘愿守满三年。而他硬是在金陵住了四年还多,其间授徒讲学,就是不出来复任。

可在内心里,王安石始终还是想着一展抱负,希望能在更大的舞台施展才华。

所以当新天子登基后,表现出富国强兵的心愿后,他便不再拒绝任用。赵顼用他为知江宁府,继而找他入京为翰林学士,他王安石便一次也没有拒绝过,并没有按照官场上的惯常规矩,推拒几次,表示自己的清高和不爱权势。

不能实现心中所愿,百辞而不应,若能有一展才华的空间,他王安石便能一招即至。

对于此,有人失望,有人冷笑,但王安石的本心如一。

始终不变!

