\section{第13章 不由愚公山亦去(六)}

【磨人的年关终于过去了,计算一下,从二月二日到八日的这七天里,俺只更了七章,欠了七章。从今天开始,下面的一个星期,俺一日三更,将欠下的帐都补回来。】

将圣旨一一宣读完毕,王中正剩下的工作是去验证古渭大捷的真伪,不过这事并不用着急,也急不来。俞龙珂和瞎药在秦州住过几日后,将臣服大宋的姿态做足,就已经回到他们的老巢静等封赏了。

王中正要数人头很容易,都用盐腌过后堆在库房里,就等着朝廷来点验斩首数真实与否。但要跟俞龙珂和瞎药面对面的做个确认,却是要费上十几天的功夫。

窦舜卿、向宝接了圣旨后,都是面无表情站到一边去。王中正不去触他们的霉头,上前向王韶、高遵裕和韩冈一一道喜。两边一冷一热,一忧一喜,正是对比分明。

但大堂中中最得意的并不是王韶他们,秦州知州李师中这时笑眯眯的从堂后小门走了进来。

王中正一见一名身穿紫袍的官员走出来,连忙丢下王韶过去行礼。大堂中的所有文官武官,也都一起向着一府之尊躬身示意。

李师中回了半礼,笑道:“都知奉旨西来秦州,师中有失迎迓,多有怠慢,还望都知恕罪则个。”

“大府所言,中正绝不敢当,何有恕罪一说。”王中正随口敷衍了几句,心中疑惑丛生。他进州衙宣诏,却不通知秦州州衙的现任主人,他的这番举动其实就是表明了天子对李师中的态度。如果正常情况下,李师中该是惶惶不安才是,但眼前的这张深深透着得意的笑脸,却哪有半分惶恐。

为了给王中正这位天使接风洗尘,李师中就在大堂处传下宴席,并邀请秦州所有官员一齐参加。正日的宫宴能摆上大庆殿,在衙门大堂上摆宴也是一年都要有上几次。

宴席筹备要有一段时间,主宾王中正去他刚刚被安排下来的住所去沐浴更衣,顺便休息一下。而大堂中的窦、向、王、高等人也四散而去,等着宴会的开始。

王厚和赵隆跟着王韶和韩冈一起回官厅,高遵裕则另有事,并没有跟过去。

一别经月,再见面时,两人都穿上了官袍,这让王舜臣看得眼热不已,一路都直勾勾的盯着赵隆身上的一片青色。

不过他和杨英也得了官身,前几天,擢两人为官的公文已经发到了秦州——他们还不够资格收一道圣旨——但他们的官诰,要上京去三班院报道才能拿到,不比王厚、赵隆直接在京中就收到手那么简单。

王韶在前走着,王厚在后面跟韩冈说着入觐天子时的见闻:“今次愚兄越次入觐,侥幸得睹天颜。不意在崇政殿的屏风上,看到玉昆你的名讳!”

韩冈笑道:“确定是韩冈两个字吗?还是说天下就小弟一人叫这个名字的?”

“玉昆别自谦了,天子可是几次提到你。”天子对韩冈的关注让王厚羡慕不已,即便时隔近月,也是一样的心情。

回到官厅中,王韶也不问自家儿子在京里的经历,也不看他带回来的私信,坐下来便劈头问道:“玉昆,这次算不算作茧自缚?”

韩冈略感无奈的点了点头,“李经略今次可能是要代替窦副总管留在秦州了。”

韩冈回答得直接,让王韶叹了一口气:“早知如此,就留下窦舜卿了。等李师中走后再对付他,也是一样。”

在鱼和熊掌之间挑一个出来,已经是让人大费思量。而要在臭肉和烂虾之间挑一个,更是让人头疼,韩冈两个都不想要。可回想起方才李师中脸上得意的笑意,就能知道他对代替窦舜卿被留任秦州充满了信心。

方才在大堂上,王韶跟李师中一样都在笑着,但他笑得有些发僵,尽管外人看不出来,但韩冈跟他处得久了,却是一眼就看了个透底。李师中得意了,王韶要能开心的笑着那才叫有鬼。

韩冈轻轻咳嗽了一声,双眉紧锁的王韶又看了过来。韩冈说正事先清嗓子的毛病,他们也习惯了。而王厚虽然听得不明不白,但见到父亲神色严肃,知道说得是见大事,也不插嘴,在旁静静的听着。

就听见韩冈说道:“记得在下前次去京城,正是二月初的时候。那时正巧碰上韩相公上书天子,反对青苗法,备言新法扰民乱国……”

韩冈说到这里,便是一顿。他的话自是有的放矢,让王韶脑筋飞速转了起来,嘴里问道:“就是让王相公告病求去的那一次?”

韩冈点了点头:“王相公此举,当然不是真的要求去。其实就是在跟天子说有我没他,逼着官家在变法和不变法中间二选一。”

王韶闻言心中一动,这番话韩冈从京城回来后就跟他说过,但现在这种情况下提起,当然另有深意。王韶的眼睛眯了起来:“玉昆,你是要我学着王相公?”

韩冈微微一笑:“王相公的招数学不来,但将其本意学来也就够了。”

“有我没他吗?”王韶双眼眯缝得更厉害,将目光压缩得更为锐利。

韩冈又点点头,却没有说话。

窦舜卿今次赴阙必然是一去不回。天子要维护秦州内部稳定,不可能让一个在秦州声名狼藉的官员坐上知州兼一路安抚使的位置。而向宝的座位也给张守约顶了。当窦、向二人尽去,秦州军内地位最高的三人中,硕果仅存的李师中,自然能稳守他的位置。看透了天子心思的秦州知州,所以才能笑得那么得意。

李师中、窦舜卿还有向宝这三人,就是河湟开边一事上的三块绊脚石。王韶在秦州枯守两年,费尽心力,抓住了时机,才有了托硕、古渭两次大捷。而平戎策中用屯田、市易二策,以根本陇右的计划,至今未能施行。

韩冈早已下定决心要助王韶早日功成凯旋,就绝不会容许他们中的任何一人还留在秦州。今次是难得的机会,连续两次大捷让王韶和河湟拓边之事在天子心目中的地位直线攀升,如果不趁此良机尽快逐走李师中三人,谁也说不准日后局势还会有什么样的变化——说不定过几日王韶连续惨败个几场,也不是没有可能的。

原定的计划是将留任机率最大的窦舜卿跟着李师中和向宝一起赶走,现在虽然算是有点弄巧成拙的味道,但也不过是把目标由窦舜卿改为李师中罢了。

韩冈的提议,就是要让天子明白,最后留在秦州的李师中与王韶水火不容,逼得天子在两人中选择一个。而最后究竟会选择谁,他有着足够的把握。王韶也同样有把握,不再向韩冈做确认,而是问起儿子这一趟去京中有何见闻。

官宴准备得很快,王韶只问了儿子几句话,来通知赴宴的小吏已经走到了门口。

大堂中,李师中和王中正在上首分宾主坐下。坐在左右两排席位上的,则是秦州城中的所有官员,皆是分着官位高低坐下。韩冈刚刚晋了一阶,位置则向上提升了几位。而王厚和赵隆两人,也够资格参加,只是坐在了最后面。

秦州城的官员陆陆续续都来了。窦舜卿和向宝也坐到了他们的位置上。很快,张守约也到了。在通传声中,新任的秦凤路兵马钤辖大步走进厅内。先与已经坐定的向宝对视一眼,各自把视线挪开,然后跟迎上来的李师中互相见礼。

张守约须发皆是花白,是关西军中有名的宿将。他从军四十载,在军中打滚的时间跟向宝的年纪差不多大。可他却直到今天,才能与向宝平起平坐。而且若不是向宝中风,他要等着接班恐怕还要熬上几年。想到这里,他望向王韶和韩冈的眼神中,便多了一分感激。

各自坐定,李师中起身祝酒。一番正式宴会前的繁琐礼仪之后,这时,宴会才真正开始。饮酒行令,互相敬酒,也有歌妓被找来表演陪酒,气氛逐渐热闹了起来。

一直喝着闷酒的窦舜卿,在敬过王中正之后,又向李师中举杯,叹道:“家门不幸,下官治家无方,管束不严,才让那些地痞无赖蛊惑了下官那不成器的孙子。事已至此,下官也不敢求大府徇情枉法,只求大府能根究那些个诱良作恶的贼人之罪,让他们不能再害了其他家良家子弟。”说着,老眼里就流下了两行浊泪。

终于来了!一直暗中观察着的韩冈随之眼神一凛。李师中坚持将窦解下狱,并主持审理此案。是因为猜到窦舜卿将顶替他的职位,为了要在天子心中博一个直名,以便早日起复,才如此不留情面。但眼下前提已经不成立了,窦舜卿求上门来,以李师中的为人应该做不到铁面无私。

窦舜卿低声下气的求着李师中,请他把罪名都推到窦解的狐朋狗友身上。而他当着王中正的面把话说出来,也有着让王中正将他这番话传到天子耳中的意思。希望能让天子看在他的一张老脸上,放他孙子一条性命。

窦舜卿自称下官,给足了李师中脸面。秦州知州扶着窦舜卿坐回座位,摇头叹道:“师中已是五日京兆,当谨守本分,却无暇他顾。”说的冠冕堂皇,实际上却是在向窦舜卿承诺不会在任上追究窦解之罪,早前的芥蒂,似是一扫而空。

见着李师中眼中难以隐藏的得意,韩冈转眼望了一下上首处的王韶。却见他正转着酒杯,有点犹豫不决的模样。

韩冈心中微怒,如果王韶不肯上,他可就要上了。王厚方才都说了,他的名字已经被天子记在心中,既然如此,韩冈就没什么好顾忌的。官位高低的差距是可以被皇帝的关注所抹去,现在在天子心中,他对李师中的看重,并不一定能高过自己。

韩冈腰杆一挺,正待说话,王韶终于有了动静。他放下酒杯,对李师中正色道:“大府却是说错了。虽为五日京兆,仍是一府之尊。既有待审之案,却无不断之理。是非自在人心,想来以大府之明睿,当能还秦州百姓一个公道!”

王韶还算有担当,也有很大一部分是被李师中压制久了,心中积蓄的旧怨让他毫不避讳。

王韶此言一出,全场酒酣耳热的气氛顿时冷了下来,静得一根针落下都听见。窦舜卿咬牙切齿,李师中脸上阴云密布,而王中正的眼神也深沉了下去,两眼转动,在三人身上来回跳着。

韩冈微微一笑,当着王中正的面与李师中过不去,这就叫‘有我没他’。就让天子衡量一下,秦州城中该留下谁为好?究竟是李师中还是王韶。

李师中抿着嘴盯着王韶一阵,视线便向下首移去。他的幕僚姚飞说得不错,每个人的行事习惯都是不一样的,王韶的性子从来不是这般直接,反倒跟坐在下首处的某人很像。李师中揣摩着王韶的这几句话,分明就写着韩记出品。

瞪着韩冈唇角边似有似无的微笑,李师中的眼睛被扎得生疼,脸色犹如九月重霜,狠狠低声骂着,

“灌园小儿!”

