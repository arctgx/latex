\section{第七章 烟霞随步正登览(11)}

“奉世。”

王安石出了文德门后便停下了脚步,等待着后面的李承之出宫来。

见王安石叫住了叛投韩冈、改变了推举局势的李承之,后面的官员纷纷绕了开去。不过脚步全都慢了下来,偏着头,竖起了耳朵。

“何以如此?”见到了李承之,王安石很直接的就问道。

李承之仿佛早有预料,回答得很快,“因为韩冈为变法做得更多,因为他才会将变法坚持下去。”

只要能秉持公心,吕嘉问、李定、曾孝宽等人与韩冈之间的差距一目了然。

真宗、仁宗时的名相李迪的侄子,一名世家子却支持新法,虽然说有庶支子弟对主脉的心结——犹如当初的吕嘉问——但没有对新法一定程度上的认同,不会如此投入。

“孙觉当初如何说新法?李常又是何人?还有范纯仁,还记得当初他在谏垣,是怎么奏论青苗的?”

“在承之看来,是西京那边有求于令婿。”

王安石摇着头:“玉昆与我等并非同道。”

“介甫,道统之争,诚然事关百世千秋。但新学如果必须要有人扶持,才能压制住气学。等到介甫你百年之后呢?又该如何?”

王安石的脸上出现了世人所惯见的倔强,“等我闭了眼,便随他去!”

但在他闭眼之前,却绝不会容忍。

“望之、令绰皆有才干,可他们拿得回幽燕吗?”

“有子厚,有吉甫。”王安石的倔强让他绝不退让半步。

李承之笑了,“再加上令婿岂不是更稳妥?”

“未闻中枢乱而边郡能安。”

“子厚、吉甫和令婿皆为人杰,相信他们能够和衷共济。”祖籍幽州的李承之与王安石对视着,“介甫,你现在是为国,还是为己?”

……………………

王安石和李承之在文德门前的争论,在过往的官员们的心中,掀起一阵惊涛骇浪。

王安石性格峻急,尽人皆知。但他刚出文德门,便拦下了李承之,还是让人感到惊讶莫名。

看到这一幕,不少人都在担心王安石会不会在回去之后,便组织人手上表弹劾李承之,就像他当年对付曾布一样。

而看到李承之这样的新党核心成员也改投韩冈,更有人担心起新党会不会就此树倒猢狲散?

现在细细想来,韩冈在两府中势力已经不弱了。

东府中一相两参,韩绛、张璪、韩冈。而西府也只有三位,章敦、苏颂,郭逵,其中郭逵很多时候并不计算在内。

宰相韩绛会选择支持新法,说不定只是因为其他兄弟都站在旧党一边,未免去孤注一掷的风险。韩宗道今日的选择可以代表韩绛的态度。张璪的性格软弱,能力又有欠缺,韩冈虽是资历最浅的参知政事,可他把持东府大政,并不会让人觉得惊讶。

而西府冇之中,又有苏颂为其张目。章敦声势虽强,可苏颂的资历亦老,再加上韩冈在军事上的影响力,虽在东府,却依然可以助苏颂一臂之力。

韩冈先一步出来了,看见他的岳父堵在门口,但他没有在旁边等着看王安石到底是在等谁,行过礼后便先行离开。

今日的朝会因廷推之事,耽搁了太多时间,崇政殿再坐只能改到下午。韩冈打算先去东府熟悉一下公事,再等着宫中的通知。

沈括也早早便退出来了,向韩冈遥遥颔首致意,然后出宫回衙。

看到这一幕的章敦暗暗感叹,这株墙头草,总算是选对了一次。

沈括的一票虽只占了七分之一的份量,可他在推举之前对吕嘉问等人的攻击,保证了韩冈日后在两府之中,不再受当初誓言的干扰。

……………………

“大参。”

黄裳快步追上了韩冈。

这一回选举之后直接在殿上宣诏,就没有来回辞让的辛苦。韩冈从文德殿出来之后,就已经是参知政事的身冇份。

“这边事了,勉仲你之后可要辛苦一点了。”

“黄裳明白。”黄裳的笑容显得他游刃有余,“不妨事。”

韩冈重新站上朝堂的最高点,黄裳地位也就水涨船高。等到他通过了制科,他的面前就将是一帆顺水。唯一的难关,就是制科的考试。新党的考官在其中占了大部分,在韩冈今日另举大旗之后,黄裳想要通过考试,难度比之前高了许多。

不过黄裳觉得这一次的推举还有一个好处。

从此之后,再也没人能说韩冈人望太高,需要加强提防。

重臣之中都有大半并不付和韩冈。难道说这些重臣,代表不了千百朝臣的态度,代表不了亿兆万姓?

他们当然不会承认这一点。每一名朝臣上表劝诫天子、太后时,都是自诩是代表天下军民说话。

既然如此,只有少部分支持者的韩冈,当然也做不了权臣。

另一方面,太后也肯定看到了新党在朝堂上的势力,随随便便举出的一个人就能与韩冈获得相当的支持,而且可以同时推举出三人。从此之后,太后不仅从心情上愿意支持韩冈,更可以理直气壮的说,这是异论相搅。

……………………

韩冈!

张璪远远的瞥了正在与黄裳说话的韩冈一眼。

在蔡确作法自毙之后,张璪好不容易有了些轻松的感觉,连呼吸都舒畅了不少,但这样的日子才过去了几日,就要回到过去了。

在张璪看来,韩冈若是来到政事堂,在韩绛的默许下,绝大部分的权柄多半都会落到他的手中。

说不等到时候,自己能够决定的事务,也许就只有中书堂后官每月二百二十贯的午餐费的分配了。

而且韩冈是不是亏本了?原本他做宣徽使的时候,每月的给餐钱是比照宰相和枢密使,都是五十贯,现在做了参知政事,可就只有三十五贯了。

张璪无声的笑了一笑,然后又肃容向前。

笑话可以放一边,韩冈刚刚就任参知政事,太后那边可就给他出了一个大难题。

半个月之后,可又是要为了枢密副使一职进行一次廷推。

刚刚失败的三人,脸皮不知够不够厚,如果他们能厚着脸皮参选,那情况会难说,可是他们三人如果不参选,很多人的心怕是就要像春天的猫狗一样乱跑乱跳了。

蒲宗孟、陈绎等人肯定都会动一动心思。新党的票数只要能争取到一半,就必然能够入选。

而韩冈一方,投票给他的几人中,同样有人有机会叩开两府的大门,只要韩冈支持。

对有些人来说,事前不论如何许愿,只要能够选上,事后翻脸不认也没什么关系。但在韩冈而言,他的身冇份、他想要维持的形象,都不能让他食言而肥。

也许韩冈事前并没有许诺什么好处,但投票给他,难道就毫无回报?那样的话,他的那一面刚刚竖起的旗帜也立不了多久。

李承之曾为三司使,只不过是犯错被贬,之前比不上吕嘉问、曾孝宽和李定更有机会,但如果没有三人竞争,有韩冈代为安排,跻身三人行列不是什么难事。

还有沈括,权知开封府的他本就是新党推出来要与韩冈打擂台的人选,以他今天的表现,难道不值得韩冈推他一把?

而韩冈一方中,冇属于旧党的几人,范纯仁、孙觉、李常没有就任过两制官,不能参与宰辅推举,但韩冈难道不需要酬谢他们?

范纯仁是任满回京诣阙,准备转迁;李常则是因病留京;只有孙觉是将要从知应天府调回,正常情况下,范纯仁和李常两人接下来都要离开京冇城。

可谁不喜欢繁华甲天下的开封府?不喜欢留在朝堂之中?他们究竟是出外,还是留京,韩冈必须要做一些表示。

这个表示,不仅仅是针对他们,而且也要包括他们背后的一干元老。

范纯仁是富弼的代表,孙觉、李常则都曾是吕公著门下士,而且还有文彦博在背后活动。

到底该怎么做才能不失人心?

这是韩冈作为领袖需要面对的第一个考验。

作为一党领袖,韩冈要头疼的事还多得很,路也长得很。尤其他得到支持的原因,并不是因为他们支持气学,而仅仅是为了反对新党。

也只有今日在支持韩冈一事上,韩宗道、沈括、李承之、孙觉、范纯仁、李常、王居卿这些观点各异,性格相悖,完全拉不到一块儿的官员才会联合了起来。

一群连政治观点都不能统一的支持者,比起山头林立的新党来,恐怕更像乌合之众一点。

韩冈要领导这样的一批人,比起当年王安石筚路蓝缕的从旧党的反对中杀出一条路来,并不会容易多少。

而且那个时候,朝中对变法改制的迫切需要,远远不是现在所能比。韩冈又如何去改变朝堂上的已经成型的法度,去迎合旧党的需要?

孤臣可以自清,可一旦结党,就不得不受到各种各样的牵累。韩冈身边的人,他们的言辞都会影响到韩冈本人身上。如果他举荐旧党,进而干扰到国是,就是太后也保不住他。

所以说,一切才刚刚开始。
