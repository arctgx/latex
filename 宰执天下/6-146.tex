\section{第13章 晨奎错落天日近(11)}

“出事了,韩三要推荐章七宣抚河北河东了!”

“他们什么时候又勾结上了?!这不可能啊。”

“谣传吧,他们不是已经……”

李格非放下手中的卷册,用力的咳嗽了几声,隔邻的声音顿时低了下去。

在乌台中的时间,零零碎碎的加起来已经近三年了,李格非已可以自诩为乌台中的老人。可以凭借资历和地位,来压一压才进来的新人。

在宫变一案中,尽管关系算得上亲近的蔡京成了附逆罪臣,曾经请教过文学的苏轼同样被视为逆贼,可李格非身后的相州韩氏的背景,让他不像强渊明和赵挺之一般,被视为逆臣党羽而遭到清洗。同时乌台中的大清洗,也让他少了许多竞争者。

没用多久,李格非便从监察御史里行,成为正式的监察御史,又从监察御史升任殿中侍御史里行。虽然因为资历浅薄而加了一个里行,可比之正任的殿中侍御史,李格非手中的权柄一点也不差。

不过近几年,御史台几经灾劫,旧年敢于大言的风气被一扫而空,台中的御史越来越循规蹈矩,全都是他人手中的悬丝傀儡。

之前快一年的时间,朝堂中平静如水,宰辅们有志一同的保持着朝堂的稳定,御史们的弹章基本上是瞄准了官品低微的官员下手,事后统计一下,选人占了一多半。对重臣的攻击也不是没有,可全是些鸡毛蒜皮的小事,要么就是仪容不整,要么就是举止不肃,基本上连金紫重臣们的头发稍都动不了。

这样的状态,也让御史台成了冬眠的熊,缩在窝里人畜无害。直到近日朝堂再起波澜,御史台才终于有了一点活力。

北地烽烟将起未起,京垩城朝野为之喧腾,乌台本非清静之地,自不能置身事外。

隔邻的几位谏官,是新党一系,与外界争论辽军是否南下不同,御史们议论的话题与朝堂的关联更加紧密。但这几位新人的耳目消息,还是不如老人灵通。

李格非咳嗽几下,嫌吵占了六七分,而剩下的三四分,倒是不想这几位后辈丢人现眼。

在御史而言,敢言只是其中的一个条件,耳目是否灵通,同样是关键性的条件。风闻奏事,风闻的风从何而来,才是重点。

与那几位仅仅知道韩冈与章敦开始勾结起来的同僚相比,李格非得到的消息更新一点——而其他资历稍长的一干御史,也几乎是在同时,通过各自渠道,得到了最新的消息。

“什么,章七……章枢密去了平章府上?!”

随着一道急促沉重的脚步声进入隔邻的房间,惊叫声随之而起。

整整迟了三个时辰。

李格非算了一下时间,摇了摇头。

几位新人的根基浅薄,在这件事上有了明证。

李定与章敦、吕惠卿都有交情,王安石在邀请章敦的同时,一并邀请了李定。连一台之长的去向都不知道,消息未免太闭塞了一点。

有了平章府的一行,章敦的动向一下子变得扑朔迷离起来,

韩冈与章敦一贯交好,曾有说法韩冈还是章敦之父的救命恩人。两人的交恶似乎是在宫变之后。而最近两人再次走近,只是为了吕惠卿和对辽的主帅之位,以及主帅之位所带来的宰相身份。

韩冈图谋吕惠卿,必须要依靠章敦。

在章敦背离王安石之前,韩冈只能反对战争,无法去跟吕惠卿争一争对辽主帅的位置。

任谁都知道,气学一脉,全部维系在韩冈一人身上。

苏颂只能算是外围,不可能为气学全心全意。而且当韩冈出外,苏颂一人在枢密院也独木难支。

现在已经不是两年前宋辽大战的时候了,那时候,朝廷上下一心,只想着将辽人赶出去。而这一回,韩冈一旦出外,朝中的新党保不准后面怎么扯他的后腿。

自古未有政敌居于中枢,将帅还能立功于外的例子。韩冈若是自请出外抵御,只会落得英名尽丧的结局,跟当年范仲淹去陕西时一样的结果。

所以一开始,王安石和吕惠卿要北伐,韩冈便联合韩绛、张璪一起反对,绝不去考虑到河北展示自己的才华——若当时他毛遂自荐,要自己代替吕惠卿,朝野内外、无论敌我的都会更信任他,而不是在河北已有一段时间的吕惠卿。

直到他成功拉拢了章敦,这才有了传言中要举荐章敦为两路宣抚,统领河北河东兵马,抵御辽寇入侵。

从反对对辽开战,到大力支持设立宣抚司,并没有经过多久。只是御史们还不能说韩冈是前后反复,一来韩冈还没有上表,二来他推荐章敦为宣抚使是为了防御辽军入寇,不是吕惠卿的出兵辽境——尽管成了宣抚使后,拿到便宜行事的许可,越界北上绝不是问题。

只是王安石的反击,让韩冈的计划落了空。没有了章敦,他只能继续阻止朝廷出兵。但辽人一旦南下,吕惠卿将之抵挡住,就有很大机会拿到中书门下那个尚缺人的位置。

延续了十余年,直至太后垂帘才宣告结束的新旧党争,以及庆历时吕范两党的政争,李格非都了解得很深,如今朝中的局面,正在向势不两立的方向发展。

御史台的性质,决定了接下来的日子里,必然要处在风尖浪口上。

李格非捻着下巴上的胡须,心思犹豫不定。有的人视乱局为进身之阶,可他还是比较喜欢平稳点的生活。

这汪浑水,到底该不该继续蹚下去?

…………………………

“章子厚这一回是改姓沈了?”

苏颂几十年的养气功夫,也掩不住话中的讽刺味道。

韩冈的心思多是放在气学上,对权柄不会争执太多——韩绛、张璪会支持韩冈,也正是因为这一点。若是韩冈什么都想要争一争,政事堂中怎么可能一团和气——可对于学术,韩冈从来不会让上半步。

章敦无心学术,如果是他代表新党居于宰相之位,朝廷还能稳当一点。可若是吕惠卿这样的人回到朝堂上,新学气学再起争端,那就是鸡犬不宁了。

原本听说韩冈已经说服了章敦,苏颂以为大局已定,可没想到会再起波折。

韩冈心情本也有点阴郁,可听到苏颂的话,却不禁微笑了起来,沈括的名声当真烂透了。

“还不知道是什么情况,等他见过家岳回来再说吧。”

苏颂在韩冈的脸上看不出他有任何愤怒的迹象:“玉昆你这么相信章子厚?”

“说不清,只能先看看再说。”

都已经是宰辅之尊,信任两个字未免太奢侈了。也就是章敦的性格高傲,让他不屑于做一些鬼鬼祟祟的事。要说信任,韩冈信任的是章敦的性格,而不是为人。

接受王安石的邀请,光明正大的去王安石府上拜会,这可一点也不违背章敦的脾气。之后章敦会怎么决定,韩冈也只能先看看再说。

苏颂问:“要是章子厚当真回头去怎么办?”

“天要下雨,娘要改嫁。还能怎么办?只能随他去喽。”

没有了章敦这个盟友,韩冈之前的计划自然只能作废,但总不能哭着喊着求他吧?

韩冈可以充分信任苏颂,可苏颂在朝中的根基不垩厚,根本抵挡不住新党,而太后方面对他的信任也远远不够。在这样的情况下,韩冈无法放心的出京。

苏颂也知道自己的问题,能进入枢密院还是韩冈推动的结果,叹了一声:“当真是没办法了?郭仲通呢?他愿不愿去河北?”

“他当然想去河北与辽人见个真章,到了孙辈,说不定还能出个郭皇后。可他现在哪里敢掺合进来?”

苏颂问的,韩冈都考虑过,可惜都不行。

种谔倒是天不怕地不怕,可韩冈倒是不敢让种谔去河北。他在河北军中素无威信,没有从父辈开始打下的基础,以种谔的性格,很难掌握好陌生的河北禁军,保不准就给他闹出事来。

“玉昆,你是不是有把握……之前吕惠卿给玉昆你写了信来吧?”

苏颂看着韩冈,过去有过不少相同的例子,韩冈做事总会留上一个后手。现在他看韩冈的口气,似乎也是一样有所预备。

“如果这一回杀的不是皮室军,仅仅是普通的巡卒,吕吉甫肯定会低头。可皮室军的背景太深了,不是普通的辽军。”

要是派入大宋境内被歼灭,他们的后台无话可说,打仗哪能不死人?之前宋辽交战,被消灭的皮室军数量也不少了。可这一回是在辽国境内被杀,性质完全不同,而且时间上,也让那位刚刚登基的伪帝下不了台来,

不得不说刘绍能的能耐真的大了,吕惠卿的运气也好,怎么就能给他在边境上撞上一队皮室军来。

“区区三人,就挑起两国之乱……”苏颂摇头感叹。

“不过是一拍即合罢了。”

碎掉的盘子,用胶粘不起来。澶渊之盟破裂之后,新约不过是习惯性的订立。有识之士皆知,过去七十余年的和平时光,已经一去不复返了。

战争即便不是现在,也会是在不久的将来。
