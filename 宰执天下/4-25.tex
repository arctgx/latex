\section{第四章 岂料虎啸返山陵(四)}

初夏的天气说变就变,出城时还是好端端的,可没过多久,就已是阴云四合。一声霹雳接着一声霹雳,待到王安石避到道边凉亭中的时候,一场暴雨就倾盆而下。

王安石身上的衣裳有些脏,这几天他出城游山玩水,擦了碰了,也忘了换一身干净的。骑着的那头老马被伴当拴在了亭外,另外一个伴当抖着王安石刚刚脱下来的一件雨衣。

将斗笠倚在墙角边,王安石凭栏望着外面的雨水。青袍芒鞋,木簪别着花白的头发,完全看不出是现任的江宁知府,前任的中书门下平章事。

“扶栏观雨,相公可有诗作否?”

一个五十多岁的老道,就在王安石身边卸下了蓑衣。捻着花白的胡子,笑着问王安石。他是寓居在钟山定林寺的道士,唤作李叔时。王安石常常往钟山去,一来二往的就熟悉起来了。

“今天倒是没有诗兴。”王安石,“不过昨夜倒是和了一首咏雪诗——‘若木昏昏末有鸦,冻雷深闭阿香车。抟云忽散簁为屑,翦水如分缀作花。拥帚尚怜南北巷,持杯能喜两三家。戏挼弄掬输儿女,羔袖龙锺手独叉。’”

“以叉字为韵……”李叔时皱眉一想,立刻恍然,“相公可是在和苏子瞻的《雪后书北台壁》?”

“正是!昨夜翻了《眉山集》,一时有了兴致。”

苏轼的《眉山集》,熙宁七年才成的书。可如今已遍传于世。这本诗词集,尤其以其中的两首以‘尖、叉’两个险韵的七律为人推重。

李叔时一时感慨:“一诗既出,天下传诵。苏子瞻如今已不下当年的柳屯田。”

“这比喻可不好,苏子瞻要强过柳耆卿不少。”王安石望着亭外如瀑暴雨,蔽日阴云,“苏子瞻出外数载,诗风为之大变。新读《眉山集》,仿佛脱胎换骨一般。”

苏轼旧年一时迷糊,批错一封判词,不得离京不出外。这一桩公案,世间早已传得沸沸扬扬。李叔时虽说只是一个道士,但能与王安石往来,见识自然不差。苏轼因何出外,他当然是知道的。但在另一位当事人的岳父面前,那一句‘此皆是令婿的功劳’却不好说出来。

王安石偏头看看李叔时,倒看出来几分内情,笑道:“苏子瞻为人疏阔,所学也不合我意,但诗文却是极好的,这一点,可比我那女婿要强。”

李叔时不便做答,转而笑道:“夏日和雪诗,相公也是雅兴。”

“雅兴吗?”王安石一声长叹,“‘放归就食情虽适,络首犹存亦可哀’,哪里来的雅兴!”

正常的宰相外放,基本上都不会处理实务。能三五日一坐堂,就可以称为勤快辛劳了。如文彦博在大名府那般万事不理,被来巡视的转运判官告发上去,反倒是尽忠职守的转运判官吃了挂落。

王安石也不给下面的人添麻烦,也是隔三差五才出来坐堂,不过当他出来视事,积累下来的公务,也不用太多时间就能处理完毕。王安石的才干,在大宋历任宰相之中,也是排在最前面的,以宰相之才用于一郡之地,自是轻而易举。

平日里则是读书读史,或是考订已经用心撰写了二十年的《字说》一书,闲暇时还携朋唤友,一同去城外游览金陵山水。王安石如今交友往来,只是随性而为,身份地位根本不放在心上,李叔时这个住在佛寺中的道士就是其中一人。

一场暴雨下了小半个时辰就结束了,王安石趁着天色放晴,就在钟山脚下的前湖边走了一圈。到了入夜之后,他方才骑着老马,辞别了李叔时,慢悠悠的回到了江宁城中,回到府衙后院的家中。

低头看见王安石袍子的下摆沾满了泥,靴子也都湿透了,正在做着女红的吴氏,就半是心疼半是责怪的念叨着:“怎么就不知道雇一架肩舆?谁跟你出去的,下次不要带着他们了!”

王安石摇了摇头:“岂能以人为畜……”他从来都不乘肩舆,就是上山过河,骑不了马的时候,也是只凭自己的双脚,“前湖那边也没得地方雇。”

“又是跟李道士……”吴氏阴沉下脸来,“仔细看看你的靴子,别污了家里的地。”

王安石知道如今妻子听不得姓李的道士,让两名婢女将黏在脚上的靴子用力的扒下来,一边笑道:“李叔时又不是李士宁。”

“李士宁那个道士说起话来嘴跟涂了蜜一般,听了他说话就知道不是好人,你还偏偏让他住在家里。”吴氏停了手上针线,回忆了一下,又立刻狠狠的补充了一句,“还给他写诗!”

“‘行歌过我非无谓,唯恨贫家酒盏空’。为夫何曾信过李士宁的神神怪怪的疯话,只是见他难得会写诗,赠了一首诗而已。何况结交宗室也不是他的错,王珪还跟宗室有亲。”王安石这时黯然一叹:“不是他连累我,是我连累了他啊。”

王安石如何不明白,李士宁涉及谋反案,不过是有人借题发挥罢了。在官宦人家行走的佛道之流,从来都不曾少过。李士宁不过是跟赵世居走得近了,如何算是罪名?只因他跟王安石也亲近啊,所以被盯上了。

就手换了一身干爽的衣服,王安石又问道:“今日东京那里可有书信来?”

吴氏回了他一个后背:“做宰相时,忙着朝政倒也罢了。现在都回江宁了,还为谁辛苦?”

王安石上前对老妻陪起了笑脸:“等致仕后,为夫在城外买座宅子,悠闲过日子……就在江宁城和钟山之间的谢公墩上,离城七里,离山七里。名字为夫都起好了,离山半程远,就叫半山园。”

吴氏叹了口气,“还不知道要到哪年呢……”

人回来江宁了,心还在东京城。游山玩水是悠闲,可回来后心思就不在山水里了。不仅仅是丈夫是这样,儿子也是一般模样。一想起刚刚病愈不久,就坐到书桌旁的大儿子,吴氏就心疼得不得了:“你这个做爹的也不劝劝大哥,少辛苦,少熬夜,累得身子骨都毁了。”

王安石点头,也为儿子担心得皱起眉来:“等大哥儿过来,就跟他说说。”想想又笑了,“二哥最近倒不错,在府界提点司里越来越有长进了。让他跟着玉昆学着做事,的确是做得对。”

“二姐儿的信你也看了,玉昆待她又多好?你过去还跟他斗气。”吴氏说了王安石一句,又叹着,“可怜大姐儿就没那个福气了。”

老夫妻俩正聊着天,府上的司阍在外面禀报:“相公,官家又派中使来了。”

吴氏很是有些纳闷:“都这时候了,怎么还有中使上门?”

“可能是入城迟了……”王安石提声吩咐,“让他进来好了。”

可进来通传的司阍却道:“中使在外,要相公出去接旨。”

“什么?!”吴氏一声惊叫。

江宁府衙,出自东京的中使们是常来常往。探望元老重臣,是朝廷的恩典,也是收买人心的手段。但王安石受到的恩泽在出外的重臣之中数一数二,跟韩琦相仿佛。基本上隔上几天,就过来一队带着礼物和口谕的宦官。不过这些中使只是携礼探问,并不是宣诏,并不需要摆出香案、洒扫庭院,更不可能要王安石这位重臣跪领。可今日的这一位中使刚来,便直接就要王安石出外接旨。

吴氏一把攥住王安石的手腕,紧张得手都在发颤:“莫不会是李士宁的事!”

“母亲放心,此事绝不至于。”王雱从内间慢慢的走了出来,一场大病让他削瘦了不少,双颊凹陷了下去,穿着袍子空空荡荡,仿佛里面就只有一个衣架撑着,就是一对眼睛更为幽深,“当是天子想到父亲大人了。”

王安石点点头,他这位宰相还不至于被不相干的谋反案牵连到。

换了朝服,摆了香案,王安石出门恭迎圣旨。阖府上下,连同外面府衙里的官吏齐聚大堂,听着来传诏的蓝元震抑扬顿挫的将拜相大诏念了出来。

蓝元震念完诏书,有些紧张的等着王安石的反应。他手上还一封招王安石入京的谕旨,如果王安石要推辞拜相的诏令,就将这道谕旨拿出来,先把人召回京中,再来完成三辞三让的的手续。省得让内侍背着拜相的圣旨,东京、江宁两边来回跑。

但王安石没有推辞,叩拜之后,恭声领旨。他从来都不喜欢做那些虚文,想接就接,不想接就不接,他推辞诏命从来都不是给别人看的。

拿着诏书,王安石对王雱叹道:“‘遽周岁历,殊拂师瞻’。只为了这八个字,也得去京城啊!”

原本辞相时的怨气,半年多来也渐渐的散去了,王安石心中不再是耿耿于怀。听到诏书中的这八个字,回想起熙宁初年,赵顼敬他如师长,而他待赵顼也如弟子一般的时候,王安石的心也软了。已经转了一个年头,哪还有过去的怨艾,而赵顼也在这两句话中透着对王安石的孺慕之情。

就再去京城一趟好了,变法大业也只走到一半,还有一半更为艰巨的路还没走完。

不管怎么说,王安石还是舍不得他一生所寄的功业。

