\section{第32章 忧勤自惕砺(上)}

延和殿。

赵顼坐在御榻上,虽然自幼传习的礼节,让他腰背还是挺得笔直,但看着就是有些病恹恹的,没什么精神。

大宋天子原本体质就不算好,这段时间灾情遍及天下,忧心过度,饮食不安使得他如今的脸色更是白中透青,腮帮子也凹了下去。

“王卿,”在重臣奏事结束后,又是照例的王安石一人留对殿中,赵顼望着他一直倚为朝中支柱的宰相:“明日祈雨之事,就要劳烦王卿了。”

王安石持笏躬身一礼:“陛下忧悯旱灾,损膳避殿,诚垂意于此,臣敢不尽力?”

赵顼叹了一口气,还是这等寻常的套话,他早就听厌了,也说厌了。昨天,赵顼诏令两浙、淮南、京东、京西、陕西各路灾伤州县长官祈雨。今日,辅臣应诏祈雨。再过两日,赵顼也要亲自出马。

两个多月来,他减膳食,居偏殿,日夜祝祷,不可谓不诚心。但天下受灾的区域却是日渐扩大。而这几天为了祈雨,他又斋戒沐浴,每餐只有两盘时蔬,就是单纯的清粥小菜而已。荤腥之物全都给免了,酒水当然更不可能有。但他苦心如此,殿外的阳光还是那么刺眼。

赵顼望着殿外反射着阳光而变得发白刺眼的地面,双眼不由得眯起来:“王卿,如今诸路大旱,迁延弥月,百姓流离失所。此当是朕德政不修,失爱于上天之故。朕欲大赦天下,不知可否?”

王安石回道:“正月乙卯,陛下已然赦天下;去岁冬月明堂时,陛下亦曾颁赦诏。今日若再赦,便是一岁之中三赦天下。商汤旱时以六事自责,首曰‘政不节欤’。一岁三赦,即是‘政不节’,非所以弭灾也。”

王安石论事时,总是能引经据典。赵顼沉吟了一下,点头称是,“……王卿说得是。”

不过赵顼的心中却难以释怀,旱情影响的可并不仅仅是民生问题。

经过了两年的休养生息,西夏已经缓过气来,但陕西有诸多名将坐镇,加之熙河路蕃军整饬得力,梁氏兄妹决不敢轻动。但契丹人近来却在河东有了动作。年初的时候,契丹来贺正旦的使节更曾暗示,辽主有意索取关南及代北之地,重定地界。

“今日雄州来报,契丹遣北院林牙萧禧为使,携国书已至边境。其人南来,必是索要关南、代北二地。如今河北大旱,京畿大旱,道上不免流民。萧禧一路南下,以目中所见,必有轻中国之心……”赵顼说着,愁眉不展。

“岂有拥万里而畏人者?!”王安石厉声反问,“陛下坐拥万里,国中甲兵百万。一时灾伤,何惧外人知晓。河北大旱,难道契丹国中就无灾?!”

“如若契丹来使坚要关南、代北两地当如何处置?”

王安石言出决绝:“若如此,决不可许。”

“若萧禧强求之……”

“遣使徐以道理与之辩而已。”王安石毫不在意,过去应付契丹人都是这么来的。

赵顼紧锁眉头:“若契丹出兵奈何?”

王安石耐着性子,“契丹亦人也,其以中国自诩,必不至于此。”

相比起反复不定的党项人,仅仅是喜欢趁火打劫的契丹人,还算是遵守信诺。自订立澶渊之盟的几十年来,也不过在庆历年间,趁着西夏多敲了一笔岁币去,并没有动过刀兵。而且契丹人惯会虚言恫吓,眼下的情况还不如庆历时危急,根本不需要怕的。

接着王安石又道,“昨日冯京亦有言,‘我理未尝不直’。”

赵顼摇头,两国相争此事何曾有理可言:“江南李氏何尝理屈,亦为太祖所灭。”

王安石心中同样在摇头,他的主君乃是太平天子,没有经过风浪,经不起挫折和坎坷。压力一大,身子骨就软了。换作是任何一个在官场上几经起伏的臣僚,必不致于如此惶惶不安:

“今地非不广,人非不众,财谷非少,当与周世宗、太宗同论,即何至为南唐李氏?若独与李氏同忧,即必是计议国事犹有未尽。不然,即以今日之土地、人民、财力,断无畏惧契丹之理!”

赵顼怎么可能不畏惧,西夏人从来都不用太担心,但契丹人可不一样了。自唐末之后,多少次入侵,将契丹铁骑的恐怖写进了宋人的噩梦里。虽然太宗之后,契丹人再也没有在两国交锋中占过便宜,后来还被逼着签下盟约,但赵顼就是担心,丝毫没有道理可讲,“如今河北大旱,三关陂塘干涸,难御契丹人马!”

作为宋辽交界的河北三关——淤口、益津、瓦桥【位于今河北霸州、雄县】——说是关,其实无险关,无要隘,本无险可据,就是三座建于平原上的城寨。是唐末在燕山失守之后,为防止契丹铁骑入侵而修筑。不过三关很快就被契丹人夺取,直到周世宗柴荣出兵收复。

但三关的位置不过是一片因黄河泛滥而造成的盐碱地,故而大宋开国后,纵屯有大军,契丹骑兵依然能随意深入宋境。后来到了真宗的时候,驻守高阳关的主帅何承矩便趁机于此塞河潴水,形成了一道长约四百里,宽五六十里的河网湖泊地带。自此除了冬天要担心以外,其余季节,都可以高枕无忧。就算澶渊之盟两国罢兵,对于三关陂塘的整修也从来没有停过。甚至利用此地积水,而耕种水稻。积水的稻田,同样能用来阻挡契丹战马。

只是眼下的旱灾,却直接导致三关外围的陂塘湖泊已经干涸大半,形势并不比冬天水道冰结时要安全。赵顼的担心也不是没有一点道理。

可在王安石看来,这一点道理,也不过是赵顼的杞人忧天罢了,“契丹若欲南来,当以秋冬马肥之时,岂有春来发兵之理?”

“说得也是。”赵顼头慢悠悠的点了一阵,突然又冒出来一句:“……可否将郭逵调往定州。”

王安石额头上的青筋突突的跳了起来,前面的话都白说!

郭逵是什么身份,随随便便的就调往定州,这让天下士民怎么看?一旦与契丹遣使索要土地的消息联系起来,宋辽开战的谣言必定甚嚣尘上,河北军民如何能安心——还嫌流民不够多吗?更何况,王安石从来就不喜欢郭逵。

“如今西夏蠢蠢欲动,少不得郭逵坐镇关中。”

“不知王卿有何提议?以如今之势,必得一晓畅军事之能臣御守北地。”

“待臣与密院退更审计,明日奏禀陛下。”王安石手头没有合适的将领或是通晓军事的文臣,唯一能想到的就是薛向。只是他现在管着六路发运司,汴河水运中的事务他暂时还脱不开手。

赵顼不想与王安石争了,宰相坚持不同意的任命,那就争不出个结果来,除非他免去王安石的相位,否则没有宰相签署的诏令就是不合法的中旨,“此事就交由卿家与枢密院相度,明日再做商量。”

王安石一躬身:“臣遵旨。”

方才一番的话,赵顼也说累了,换了个话题,“昨日白马县韩冈上书。但言逗留黎阳的河北流民不可胜计,恳请免去流民渡资,让流民不至于强行渡河而枉送性命。此事可有之?”

“此事诚有之。”王安石点头,这事瞒不了的。他回道,“春日和暖,黄河解冻,河上渡口重启也就在这两日。黎阳县也上报有流民聚集渡口。韩冈此亦是未雨绸缪,否则流民没于河中,有伤陛下圣德。”

“韩冈的一番布置,是他到了白马县后就开始。”赵顼沉吟了一下,问道:“说是未雨绸缪,难道他早在去岁就知道灾情会延续到今年?”

王安石不知天子到底是怎么看韩冈在白马县的一番作为,韩冈在奏章中半点也没有隐瞒白马县的情况,以及他对于流民的安置之法。现在又请求免去流民的渡资,等于是邀请流民南下。

但他还是要为着女婿辩解,“韩冈所行诸事,皆是有备无患。若旱情持续,便有所预备,不至于临事生乱。若旱情不至,深井、风车、沟渠、医馆、石窑,日后亦有所用。”

赵顼点了点头,他并没有怪罪韩冈的意思,而且很是赞赏。他方才忧心政事军事,直到现在心情方才稍微好了一点。

开封府界内的传言琐事,赵顼通过遍及京城之中的皇城司亲事官都能探听得到,加上派驻于当地的耳目,韩冈在白马县中所作所为,他都了解得一清二楚。

赈济灾民必然要花钱,而韩冈花得都是在刀刃上。开井、补种、灭蝗,加上安置流民的准备,每一件事都筹办得游刃有余,所耗钱粮更没有半点浪费。如深井、风车、水渠,大半皆是乡民自出人力物力,官府连给付流民的工钱都省了许多。等旱情解除之后,京畿之地就又多了上千顷不虞干旱的水浇地。

这才叫作能吏!

所以韩冈在县中预设流民营,又上书申请免去渡口渡资,赵顼也没有生气。他如此行事,换作别人,必然少不了一个贪功的评价。但赵顼对韩冈一向看重,而且韩冈又做得出色,所以在他眼中,这就叫做勇于任事、为君分忧——不同的人,做同样的事,得到的评价是远远不同。就像名人做的蠢事,能被称为轶事,而普通人犯傻,得到的只会是嘲笑。

“韩冈所上诸条,皆许之。白马县中所耗钱粮,皆由开封府库补足。”赵顼想了想,道:“至于流民,先让他安排着。过几日,看情况,再让他名正言顺的主持。”

