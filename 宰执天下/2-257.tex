\section{第36章 万众袭远似火焚(13)}

王韶已领军前往香子城。六千熙河军,以赵隆领选锋为先导开路,苗授、王舜臣为前军,开始向河州进发。

一日之间,珂诺堡中就只剩王韶留下来的五百守军。不过今天午后,景思立就该到了,倒也不用担心珂诺堡的安全问题。

变得空空荡荡的珂诺堡安静了下来,但韩冈还是没有空闲。

为了能加快狄道至珂诺的转运速度,走康乐寨、当川堡的山道,还有河谷道,都被利用了起来。运送粮秣物资的队伍经过两条道路,一队接着一队抵达珂诺堡,三天之内,就到了五支车队、马队。

点算数目的工作,韩冈有亲信的吏员来完成。但他也有必要时不时的去抽查一下这个工作的完成情况。现在珂诺堡要支援的人数是六千,接下来的一段时间,将会增加到两万。两万将士,加上四千上下的战马,都将要依靠珂诺堡来供给他们的需用。

也许今次作战的难度不及横山面对的党项人,但换作是随军后勤方面的工作,可是十倍于当初的罗兀城了。

韩冈走过在仓囤区。看着下面的小吏拿着铁钎插进粮袋中,抽出中心部位的,检查袋中的粮食是否完好,有无霉变。

“损耗了多少?”韩冈拿着随车而来的出库单问道。总计十六车,出库有两百四十石,比起在通途大道上能装五六千斤的太平车来,一车只能装一千多斤的份量,实在是少了一点。但蕃区的道路不可能跟正式的官道相比,能超过千斤的运载量,还是韩冈让人将车辆都换了宽幅的车轮后的结果。

“这是从洮水边过来的,袋数都是清点过了,没有什么损耗。”小吏回答着,将清点后的单据签名画押后递给韩冈,“如今人马食用与押运的粮秣分装,也没人敢像过去那般在路上犯事。”

过去运送军粮,都是送一路,吃一路,根本不分。甚至有的民伕会为了运送时轻松一点,故意在路上倾倒一批,然后说路上给牲畜吃掉了。现在将两边分开来,又是一程转过一程的运送,道路上的无谓消耗减少了许多。

只是另一方面,由于兵站设立后,要驻防道路的兵力增加,他们的消耗则多于过往,一进一出,其实又抵消了不少。但毕竟兵站制度对后勤方面的帮助是显而易见的,不论是蔡延庆,还是沈括,到现在为止都没有对此要有改动的想法。

而韩冈在实际主持后勤转运的事务后,其实也对之前的随军转运使的工作方式感到很惊讶。过去的随军转运使,是字面意义上的随军,大半时间跟着主帅走。从后方组织押运粮秣的工作,都是交由民伕出发地的州县官。然后到了军中后,再由随军转运使分派。如果路途过于遥远,那粮草就会先送到大军出阵前的驻屯地,再由随军转运使亲自领人去接应。有许多时候,他们甚至就相当于一个押粮官。

韩冈即便对军中后勤再不了解,也不会认为后勤主管的工作会是押粮官这么简单,何况他在几个经略司、安抚司中也经历了许多,后勤上的弊端也看得很清楚——后世物流发达,在运输路线上设立转运点的必要,韩冈多多少少心中还有点数。

随车而来的不仅仅是沉重压车的粮草,还有后方带来的消息。

蔡延庆坐镇陇西城,王厚被他点名过来打下手。有王厚配合,秦凤转运司对熙河的支援工作,也变得井井有条。另一方面,沈括在接手了随军转运之职后,并没有立刻烧上几把火,仅仅是提拔了两名办事得力的小吏。

该怎么说呢……这应该算是很聪明的手段。在韩冈本人还在同任一职的情况下,若是沈括恣妄威福,来什么下马威,韩冈是绝不会坐视。但提拔两个办事得力的小吏,却没人能干涉。这样的一步步的扎稳根基,也就是为日后在熙河的久任打下基础。即便战事结束后他不能在熙河任职,但沈括的这一番表现,也照样能算是中规中矩。这边的战事功成,必然不会少得了他的一份功劳。

会做人的聪明人,而且还识时务。韩冈放下了点心来。就算还有点小动作,也是睁一只眼、闭一只眼。

一个可以合作的对象,韩冈很有兴趣跟沈括见个面,只是他现在还是无暇分身。

王韶已经出兵,后面的粮秣运输也要及时跟上去,从珂诺堡到香子城依然还是是河谷道,可以走马车。而从香子城【今甘肃和政】到河州城【今甘肃临夏】,中间还要翻越一座山,尚幸并不高峻,独轮车也照样能够送粮过去。

‘看起来要在香子城多放上两营转运粮草的民伕了。’韩冈想着究竟从哪里抽调人手比较合适。

到了午后,景思立在预定时间中率部抵达珂诺堡,刚刚清静下来的寨堡中,一下又多了五千兵马。

听说王韶昨日就已经出发,景思立似是讽刺的说道:“王经略动作好快!追都追不上。”

“木征就在河州等着。经略当然走得要快一点。”

“还有粮食上的问题吧?”

“都监大可放心。庙算不胜,如何敢出阵。粮草供给,可是庙算中第一个要计算的。”

韩冈对后勤压力并不担心。经过一个冬天的抢运,熙河路的存粮大概能给三万大军使用到四月底。而五月初,就是麦收的时节。只要蔡延庆能在四月底之前,给补充陇西城补充上三五万石的粮草,便不会造成断粮。王韶虽说提前出兵,但其实消耗并没有增加多少。有整整一个半月的时间,补充二三十万石都不会有什么问题。且以今年熙河路预期的收获,更足以将眼下的三万大军,维持到九月以后。

“都监在珂诺堡中休整一夜,就可以上路追上去了。不待都监的主力抵达,经略也不会贸然跟蕃人决战。不过经略留信下来,还请都监在珂诺堡中,留下步骑各一个指挥,以助守后路粮道。”

………………

王舜臣驭马领军,在山道上疾速前行。

流经香子城、珂诺堡的河流是洮水支流的大夏山水,而河州城则是位于离水【今大夏河】之滨,洮水和离水都是黄河的支流,王舜臣所部正要翻越就是两条黄河支流之间的分水岭。

这道分水岭南坡陡峭,上山的道路要曲曲折折的绕上两个回弯。可北坡平缓得很,翻过这座山口,下面就是一路缓坡。

就在山道边,几处吐蕃人设立的堡垒上,有着火燎过后的黑色痕迹。夯土的墙体上还能看到历历箭痕,尤其是靠着路边的几面墙,被神臂弓射上去的箭矢密密麻麻的排满了一片,如同毛虫身上的毒毛一般密集。

‘赵隆看来是拼了命了。’

一夜将道路上的据点给拔出,熙河选锋表现出来的实力,并不比前广锐将校稍差。

刚刚骑马上了山口,尚未来得及远眺北方风土,一名骑兵就赶过来向王舜臣汇报,“都巡,吐蕃人正在攻打选锋军,还请都监速去支援。”

不用这名哨探多说,王舜臣就已经看见了千多名吐蕃骑兵,正攻击只有他们一半不到的大宋官军队伍。

“立起我的将旗,击鼓,准备作战!”

王舜臣不怕对手多,只怕对手不来。几句话下了命令,他所率领的前军就已经在坡道上列阵。

鼓声传递去了下方的战场,交战中的宋军和蕃人一时缓下了手脚。

同时抬头仰望,一彪宋军战士,正从坡道上下来。对手的援军突至,令吐蕃人士气沮丧,而下山的缓坡也帮了宋军的一个大忙。

王舜臣让麾下的士卒横队站在坡顶,一边向下射击,一边缓步向下。山坡多树,吐蕃人骑着马不便进入林中。只是留在坡道上,就会被宋军用弩箭尽数攒射。

见势不妙,在号角声中,千余名吐蕃骑兵徐徐而退。重新恢复平静的坡道上,受伤的士兵和战马被留在了战场上,就像退潮后的沙滩上的贝类,孤零零的好不可怜。

让人收拾战场上的伤员,王舜臣见到了赵隆。王韶身边率领选锋的亲信将领,正坐在路边喘气。他身上的甲胄都有着各处箭创,胸口的护心镜也凹下去一块。胯下的坐骑,不再是昨天看到的那匹黑马。

“怎么打的这么厉害?”

“蕃人连吃奶的气力都拿出来了,我们扎个营都难。”

赵隆这一天也是拼了命,连他这个主将一天来,也就现在才有机会坐下来喘口气。他们这些先头部队,自下了山之后,仗着各个战力精悍,一味猛冲硬打,击破敌军的攻击有三四次,但自军伤亡率也变得很高。

“早知道就换个手段了。”赵隆为自己人的伤亡叹着。

“不要说废话了。一开始不就知道木征会给我们来个下马威吗?”

站在山口上,王舜臣远眺极远处的吐蕃人。青葱嫩绿的山岭下,帐幕连天接地,填满了离水两岸。

“有五万吗?”赵隆低声问着。

“看起来只多不少!”

“看来木征有足够的兵力来抄截我们的后路了。喂”他叫着王舜臣,“你说木征会抄那条道?”

“多半是珂诺堡吧……不是说蕃人挖了好几条暗道吗?他们肯定不知道我们已经都给找出来了。”这时候,又是一堆蕃军骑兵攻来,看着吐蕃人在山坡下啊耀武扬威,王舜臣冷眼看着他们一下,“等王经略到了就会有他们的好看了,那才是真正的决战。”

在赵隆所率的选锋军中,已经找不到是谁对河州蕃军射出的第一箭。但当王韶和高遵裕率主力抵达离水之后,河州决战的最终戏码,也终于上演到正章。

