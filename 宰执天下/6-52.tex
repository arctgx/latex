\section{第七章 烟霞随步正登览(四)}

“说是二十六,没想到昨天又多了一支手。”

黄裳走进宣德门的时候就听见身边的人在说。

“三日五百,六日一千,这是范妙才。”

“是宝文阁典军。”

黄裳并不认识这两个说促狭话的,但他们说的是谁,他还是知道的。

昨日兼程入京,并及时在宣德门登记,让有资格参与选举的重臣数量,从二十六增加到了二十七。

但黄裳没有在前后左右的人群中看见范纯仁。虽然他并没见过这一位,但范纯粹是见过的,不知道两人长得像不像,见面后能不能认出来。不过就算范纯仁与其弟并不相像,但等他出现在宣德门时,肯定会引起围观,不认识的也会认识了。

“有资格当选的就那么几个,这叫人怎么选?”

“京城外的哪个不受牵连?冯当世的儿子都要被女婿给拉进去了。偏偏苏轼、刑恕又都是交游广阔的。开封府这一回为自己清扫开路,可是不遗余力。”

“呸,凭他也配!”

“怎么不配?手上攥着多少人的把柄,谁不要畏其三分?早点将他送上去,也可让开封府给空出来。要不然,就等着被传进开封府二堂吧。”

权知开封府沈括,虽说的人望低得可怜,不论新党、旧党都不待见他,可是他能得到的选票,在预计中至少能排进前三。他就像掌握御史台的李定一样,名声虽然不好,但让他们留在原位上实在太危险了。若他们落选,说不定会拿着手中所掌握的阴私来报复所有不投票给他的大臣。为官这么多年,谁屁股后面没有些没擦干净的东西。

黄裳觉得这件事恐怕也是自家的恩主事前所没想到过,若不是由沈括来主审所有叛党,那他在这一回的殿上推举中,根本赢不了任何人。而决定将这一主审权交给沈括时,韩冈好像正在殿上。

不过沈括的人望之低,并不仅仅是他的反复无常,也包括他的籍贯。

“大宋治下四百军州,难道都只在南方?”

“如今是南风大盛,就是韩三,不也照样是王平章的女婿。”

“王平章对女婿还不如对仇人好。曾布做参政的时候,王平章可没拉他下来。”

南北对立的传统源远流长,这两年因为变法,使得南方大胜,北方纵是心有不甘,但还是给强压下去了。黄裳自己就是南方人,而且是南方人中名声最不好的福建子,闽人。被称为腹中有虫,视为奸猾的代表。对南北之争,黄裳终归不可能去支持北人,最多也只是因为韩冈的缘故,而选择中立。不过这样态度,光是在韩冈那边就过不了关。门人首鼠两端,放在谁身上都不会高兴。

“就是今天了。”

“等了半个月。终于等到了今天。”

“可是有好戏看了。

“不只是好戏吧。哪家瓦子里能看着这场面?”

“不知要是没被选上,会是什么模样?”

“那还真要好好看看了。”

稍稍走慢了一点,充斥在黄裳耳边的窃窃私语,就变成了幸灾乐祸的内容。

也有可能是自己的幻听,黄裳想着,毕竟在宣德门内说要看乐子也未免太猖狂了。

不过当他回头,就发现了几名低品的朝官。方才说得很开心,但对上黄裳的双眼后,就立刻噤口不言,这让黄裳一下就确定了方才到底谁在说话。

这几位都是年纪老大,却只有一身青色官袍。这个年纪还没有一身朱紫,没有后台是肯定的,同时应该也是没有多少才能,否则朝堂上能做事的官员数量绝少,真正有能力的早就升上去了,或是贬出去了,而不是靠熬资历熬到了这一步。

对他们来说,高层的变化,的确只是些茶余饭后的谈资。至少在可以看见的未来,他们的生活不会因为两府中的人员变动而有任何变化。

黄裳暗暗记忆这几人的形象,很快就穿过了宣德门,当他重新沐浴到头顶的阳光,周围一线就安静了,没人会在皇城中的高声喧哗。

黄裳随即举步,随着人群,往文德殿过去,然后他看见了韩冈。

……………………

韩冈走得不快,但周围都空出了一片,比起人流中朝官们,速度反而更快一点。

他看似沉稳的走着,矩步方规,行动举止与他的身份相匹配,可他的心中却在想着一些不相干的事。

惟俭可以助廉,惟恕可以成德。

范文正公这一句说得很好,可惜能做得到的就寥寥可数。

韩冈不觉得自己能够做到这两句上的要求。

韩冈并不算节俭,比起范纯仁在招待客人的饭菜上加上两撮肉末就算是超越父辈的奢侈,韩冈家中的日常开支可算是石崇、王恺一流了。不过他的清廉,不会比任何清官差,而在百姓们的口碑中,亦是以清廉著称。

他对人也不够宽容,饶恕两个字在他的字典里,定义肯定与范仲淹完全不同。但德行,当今世上谁也不敢自称能与他相提并论。

不过前一句倒也罢了,后一句就是做到了,恐怕在朝堂上也没有太大的用处。

韩冈瞥着不远处的范纯仁,却并不在意自己成为众人瞩目的焦点。

范文正公一次又一次的被赶出朝堂,而遏制了他整个官场生涯的政敌,绝大多数时间却都能够安稳的坐在相位之上。欧阳修说两人最后还是和解,戮力同心,共御西北二寇。可就算欧阳修所言为实,在和解时,范仲淹的心中恐怕也是苦涩的。

如果这一回的选举放在庆历年间,尽管当时范仲淹扬名天下,光芒四射,但朝堂上,尤其是高官之中,会选择他的依然是寥寥可数。一切都要看实际利益,而朝臣们一贯又是最为现实的一批人。

终于要开始了。

站定在文德殿外,韩冈收回了飞出去的思绪。虽然还有朝会,但流程早已确定,朝会一结束,可就要等着开场了。

而开场之后,这一场大戏,不知会变成什么模样。

但有一点可以确定,韩冈从未觉得自己会输。

……………………

章敦没想过会遇上这样的场面。

虽然经过吕嘉问和李定的努力,拥有投票权的重臣们绝大多数皆已明确的表态。但南北之争的暗流,却不知何时在朝堂上蔓延开来。

或许这就是韩冈为何如此平静的原因。

朝堂之中,章敦自问没有人比他了解韩冈,韩冈的平静,不仅仅是因为他的胸有成竹,更是因为他已经做好的决定。

看来要分道扬镳了。

章敦不无感慨。

在同心协力了十余年之后,韩冈终于要与一直若即若离的新党划清界限,打算用地域之争来争取自己的支持者。

范纯仁的及时出现,让韩冈的谋划看起来已经成功了。只是侍制中的北人,还不足以让韩冈能够确定无疑的入选。

这样当真好吗?

章敦摇摇头,以地域划分众人,按韩冈的心胸,不该如此。而且文彦博、富弼那批人的胃口不是那么容易填满的。

或许富弼、文彦博他们并不是一定要让韩冈进入两府,而是要在太后面前将南北之争给明白的展示出来。

一旦太后看清楚了南方人已经占据了朝堂,非其同道便难以在两府立足,就是积功最多,才干亦强的韩冈也比不上那几位,那么接下来,自新党大兴之后,一直被压制的北人,就有了出头的机会。至于韩冈之后到底能不能进入两府,恐怕并不在他们关心的行列中。

以韩冈的才智,不可能想不到这一点,除非……章敦目光森冷如冰,这又是韩冈为了他那个目标,而使用出的手段。

……………………

钧容直正演奏着朝会上的韶乐。

王中正跟随在牵着小皇帝的向太后的身后,走上了台陛。

立于帘后,居高临下的王中正能很清楚殿中的一切动静。

下面的朝臣们,很明显的并没有多少心思放在眼下正在进行的朝会上。

朝会之后,便是东京城中数千官员翘首以待的选举。最终得以参与选举之中的二十七名侍制,他们将会选出三名枢密副使的候选者。

而王中正心中有着一丝不安,有第一次就有第二次,随着时间日久,这样推举就会成为朝堂上的惯例,天子只能选定朝臣们推举上来的大臣,不像过去有着绝对的取舍权。

而且推举一事也不会局限在两府宰臣身上,日后肯定还会扩大。说不定日后连选拔两制、侍制,罢免宰辅等官,也要通过重臣们的同意。届时罢免一名宰辅,又需要多少人来进行选举?

而且随着廷推制度的发展,党争会日趋激烈化。通过什么样的途径得到的位置,就会向何处何人负责,唯一不需要的,就是向天子负责。

那时候,就是标准的垂衣裳而天下治。

只要各个位置上,都放上贤能的官员,那么天子什么事都不需要去费心,做个太平天子就好。

或许天子从此以后就是摆设了。

王中正的心中挣扎着,到底是要做个忠臣向太后说明,还是干脆保持沉默,反正不论说什么,只要是攻击韩冈,太后都不会听。

就在王中正挣扎的时候,朝会开始了,又结束了。

正常情况下,王中正应该扶着太后起身、退朝,接下来就是崇政殿中议事。但现在不同,太后心血来潮,说是要在文德殿中决定一切,并允许侍制以下的官员能够旁观。

嗡嗡的议论声响了起来,与方才的寂静截然有别。

太热闹的也不好,王中正想着,这对韩冈可不是好消息。

……………………

吕嘉问自信满满,李定也似乎是胸有成竹。从他们脸上的表情,很容易看出这一点。

韩冈必败。吕嘉问对此充满了信心。

他已经确定过了每一张选票的去向,除了范纯仁之外,其他人都不会选择韩冈。

纵然韩冈名垂当世,但区区二十七张选票中,他拿不到其他人。这些选票,被三人瓜分,包括沈括在内,已经没有其他人涉足的空间了。

韩冈能争入前三名吗?

吕嘉问想放声大笑,只是眼角余光处人影一闪,一人站了出来。

‘沈括?!’

