\section{第三章 墙成垣隳猿得意(中)}

王厚叩拜之后,告退而出。

边臣五天两入对,不但证明了王厚所在位置的重要性,同时也是赵顼想多听一听他带来的好消息。

比如茶马交易的丰厚成果,比如陇西户口的急速增长,比如熙河田亩的稳步扩大,比如四方蛮部的争相投奔,甚至是熙河路各州,如今逢四奉九就要举行的蹴鞠联赛的赛况,都让赵顼听得津津有味。

一想到在京城中踢得温文尔雅,已经软绵绵得只胜脚法花样有些看头的蹴鞠,到了陇西,竟然变成了每每致人重伤、鲜血四溅的运动,赵顼就忍不住热血沸腾起来,恨不得亲眼看上一回。

憧憬着金戈铁马的蹴鞠比赛,好半天过去,赵顼都没有看一眼桌上奏章的心思。他就是不想去想这些日子来,困扰着他,并朝堂上下直接分裂的两桩案子。

但他的臣子从来都不会让他如愿以偿。今天轮班随侍的李舜举在赵顼耳边通报着:“官家,范百禄求对。”

“宣!”赵顼很是无奈。

关于赵世居、李逢谋反案,性格软弱的沈括已经被踢到了一边去,都是范百禄独自上殿奏对。走上殿来,范镇的侄儿朗声道:“陛下,李士宁今日已经招供,熙宁五年六月乙卯,他曾以钑龙刀一柄赠与世居。并尝见世居母康,以仁宗御制诗取其中四句赠之。且曰:‘非公不可当此。’”

好吧,按照范百禄所说李士宁送赵世居钑龙刀的那段时间,他正好就住在王安石府上。赵顼倒不在意将一两名道士依律碎剐了,送去见三十三天见太上老君,但他绝不想将王安石也牵扯进来。

苦恼的让范百禄将供状留下,赵顼示意他退下去。只是片刻之后,御史中丞邓绾又上来了。

“官家,邓绾求见。”

“宣……”赵顼的声音中透着疲惫。

邓绾上来了,同样是案情有了进展,不过这一次是牵连到了冯京身上。赵顼也是同样的处理,留下供状,将人请了出去。

两件案子最新的口供就摆在面前,赵顼却也无意去多看上一眼。

两桩案子,其实只要他下一道不再深究的旨意,便能就此了结,可是赵顼就是不愿。大宋天子从两件案子的口供上,看到了不少悖逆之言,让他的心头堵得慌——供状中还记录说,曾有人称赵世居貌似太祖。作为太宗一脉的赵顼,可不喜欢听到这一句话——已是羞刀,如何能轻易入鞘?他决不想就此罢手。

可赵顼也不想将两桩案子扩大化,大狱一兴,少不了要牵扯上几百户人家,几十名官宦,对于朝堂的稳定决不是好事。

就不能就事论事吗!?

因为这两桩案子,让朝中的新旧二党彻底的对立起来,弄得朝政不知耽搁了多少。赵顼这几日心中又急又恼,嘴角都生了两颗血燎泡,不碰都疼得厉害。

还有韩冈,吕惠卿拿着厢军聚众的案子,想要有着一番图谋。韩冈这位当事人反倒置身事外,一句都不多说,似乎是对吕惠卿的想法,并不怎么认同,只是不去反对。赵顼也无意管这么多。但军器监在韩冈手上,就成了漏勺一般,不论造出什么东西,转眼就能传遍东京城,最后又会掀起一场风波,这就让赵顼有些难以接受了。

赵顼隐隐约约的也能揣摩出韩冈的心思:他是为了尽快推广格物之说,宁可让自己犯点错处、受些污名。对于一名皇帝来说,只要下面的臣子将事情办好就行了,私底下的想法他并不在乎,韩冈好歹还能算是一名能臣。且一个在民间与神神鬼鬼牵扯不清的官员,身上干干净净,反而不是好事。韩冈沾染一些污名,赵顼却是乐于见到。

不过要是神臂弓、板甲、霹雳炮、床子弩这样的军国之器泄露出去,赵顼就不能容忍了。在过去,按照军器监中的制度——甚至在军器监成立之前——各个作坊中打造的各项器物,尤其是有关攻城器具的二十一作,其制作之法,都是不立文字的,只让工匠们口耳相传,且严禁打听其他作坊的情况。就是编纂《武经总要》,其中的一些数据都是刻意加以模糊。

“官家……官家……官家!”李舜举唤着天子的声音,一声高过一声。

赵顼身子一震,从思虑中被人惊醒,有点茫茫然的问着,“出了何事?”

“官家。”李舜举轻声道,“翰林学士、馆伴使刘庠有要事求对。”

刘庠是被派出去陪着辽国使臣的官员。按照澶渊之盟,真宗皇帝和辽圣宗结为兄弟。自此之后,宋辽两国成了亲家,赵顼照辈份还要喊如今的辽国天子为叔叔。两国之间敌意犹存,但在场面上,都不会有所欠缺。所以到了年节或是太后、天子的生辰,两国都会互遣使节去对方那里拜贺。使节来了,也要安排人去接待。同住一驿,趋朝,见辞,游宴,都要陪伴左右,担下这项差事的就是馆伴使。

馆伴辽使的官员,通常都不会是低阶的官员,常常能见到翰林学士来接客。刘庠最近刚刚回京来,做了翰林学士,转头便被任命为馆伴使,陪着辽使。他的急事,当然不会跟西夏有关。

一听事涉辽国,赵顼便紧张起来,立刻道:“宣!”

刘庠很快就进了殿来,他是个直接的性子,行过礼之后,看门见山的对赵顼道:“陛下,辽使今日向臣打听了轨道之事,并详加细问飞船、铁船等物,还遣人往印书坊购买《浮力追源》。观其行事,必有所图谋。”

赵顼一听,心头就是一惊。如果韩冈的一干发明泄露出去后只在国中流传,他根本就不会去担心。可事情一旦牵扯到契丹人身上,赵顼却是怎么都不能安心下来。

“速传韩冈进宫!”

韩冈又被加急召入宫中,两三天就能见一回天子,但这可不能算是宠幸。

上了殿,拜了天子,听了刘庠将整件事重新说了一通,韩冈直接就问道:“不知学士在担心什么?”

刘庠正气凛然,对着赵顼行礼:“陛下明鉴,秦人修郑国渠,十年不能东窥,可一旦水到渠成,便是席卷天下。轨道、飞船二物,世人趋之若鹜,当可见其效用,契丹人未必不会有用之于国中的打算,臣请陛下要对此事早作提防。”

刘庠口口声声请赵顼要多家方便,言下之意就是在说韩冈正在打造的轨道、飞船,没有才是好事,造出来就是个祸害。这样的想法,其实已经在朝堂中成了一股潜流,吴充说过、冯京说过,许多人都认为,这等容易仿造的发明造出来,其实是在帮着契丹和党项。这番言论,其中有多少公心,有多少又是私心,根本就不用多想。

“秦人修郑国渠是为了灌溉,能让其国中多出产百万石粮秣,其功用犹在当今白渠之上。不知辽人修轨道又能有什么用处?”韩冈冷笑着反问,“如果辽国整修轨道,绝不会是始皇修郑国渠,而是隋炀帝修大运河了。正是因为中国缺马,能节省畜力的轨道才有用处。若是如契丹那般,战马至以千万计,他们辛辛苦苦的去修轨道又了做什么?”

“飞船呢?”

韩冈都不想辩驳了,这些天他心头已经很烦了,虽然还不至于生气像赵顼的嘴角上那两个很明显的燎泡,但频频浪费口水,当然还是难以忍受。但在赵顼询问的眼神面前,还是得耐下性子询问:“飞船是攻城、守城时用的器物,只能固定在地上以防飘走,又不能用来作战,学士何须作杞人之忧?”

赵顼还要依靠韩冈主持军器监,也不能同意刘庠对韩冈的攻击,而且还是韩冈辩驳过多次的陈词老调,也没有什么新意可言。说了几句,就让刘庠退下去继续配契丹使臣了。

只是事情牵涉了到辽人,说不定党项人也已经在伸手了。若是过了一两年,在辽、夏两国的城头上,腾起一艘艘飞船;在码头甚至是大道之上,又有了有轨马车来运送粮秣军资,这对赵顼、对大宋来说,绝对是一个灾难。

“韩卿你还是要在监中多加防备,严守监中机密。朕可不想看到过两年,铁鹞子当真全身上下都穿上整套铁甲。”赵顼的话已经说得有些重了,但他还是要提醒韩冈。

“臣遵旨。”韩冈恭声行礼,抬起头后又道,“陛下放心,臣回去后必严加督训,让监中机密不至于泄露到西北二虏的那里。”

“哦,那就好!”赵顼有些累了,抬起手揉了揉额头。

“陛下身荷天下之重,还要多保重御体。”韩冈看着赵顼的动作,关切的说着。

‘保重?怎么保重?天天都不让朕得个清静!’赵顼想骂出声,但作为天子的自制力让他忍了下去。矜持的点着头,“韩卿有心了。”

韩冈低下了头去,‘差不多就在这一两天了。’

