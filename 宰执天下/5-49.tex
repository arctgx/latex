\section{第七章 苍原军锋薄战垒(五)}

肆虐了两天的狂风已经停了,灵州城南门五里外的宋军营地,终于可以见到天光。

漫天的星辰从地平线上一直闪耀到天顶,璀璨的银河横贯苍穹,纯黑的天幕上看不见一丝云翳,明天应该是个好天气,苗授想着。

就是狂风大作的时候,高遵裕也命环庆军列阵于城下,用神臂弓清扫城头守军。虽然不无战果,但只要没有足够攻城器械,光是压制城上的弓箭手,根本毫无意义——除非敌军打开城门,出城反击。

高遵裕的本意的确如此,可他这种试图用无谋的举动,引诱城中守军出击的计策,并没有能够成功。党项人只从其他没有官军封堵城门出来。而在风沙中列阵的官军,看起来像是块十分好下口的肥肉,但藏在里面的骨头没能瞒住党项人。他们只从其他几处城门出入,然后跑到外围骚扰官军。

按理说这样的情况下,最好能干脆将四座城门都赌起来,可苗授很清楚,官军不能分兵堵住灵州四门。灵州后面还有兴庆府。以环庆、泾原两路的兵力,一旦分兵围城,很可能就是当年高粱河之败的翻版。

太宗皇帝领着刚刚灭掉北汉的禁军围着辽国南京析津府打得正高兴,背后就被耶律休哥捅了一刀,几乎送了性命不说,周、宋两世经营了多年的大梁精兵也被打断了脊梁骨。无论是高遵裕还是苗授,都没有向太宗皇帝学习的打算。

苗授抬起头,头顶上的群星闪烁,明月皎皎。不知为何,他眼中的天幕却似乎隐隐弥漫着赤气。

观星望气乃是兵家秘传要旨,苗授虽算不上精通,也是有所了解。

大军已出,兵凌敌境。苗授不观五星,不观星宿,只观诸星。

羽林四十五星,三三而聚散,在垒璧之南,主天军营阵翊卫之象。今五星入羽林,乃是关梁不通,兵起之兆。

北落师门主候兵垒,色白带赤,营垒或变生肘腋,变则带血。

天垒城十三星,形如贯索,主候北夷,其星芒角变动,难道是的契丹哪里又有什么动作?

苗授仰望星空,心中的不祥之感怎么也无法抹去。

“父……总管。该安歇了。”苗履的声音在身后响起,“明日卯时还要军议。”

苗授从星辰间收回视线,看了看儿子,回身向大帐走去。

苗履忙跟在后面,犹豫了一下,问道:“是不是有什么不对的地方?”

苗授自嘲的笑了笑,摇摇头:“是为父想太多了。”他抬头再看了眼天空,‘应当是吧。’他在心中说道。

这是泾原军进抵灵州城下的第四日,对环庆军而言,则是第三天。

粮秣的补给依然紧张,今天从南方运抵的粮草有两千石束,一半粮、一半草。这还是没有受到大的骚扰的缘故。但从侦骑那里得知,更多的铁鹞子已经从贺兰山脚下绕过了灵州南下。接下来无论是去抵御王中正的秦凤、熙河联军,还是骚扰泾原、环庆两路粮道,又或是赶去瀚海东侧,堵住种谔、李宪西来的道路,对官军来说,情况都很不妙。

很有可能,苗授和高遵裕两军接下来必须独力解决灵州守军,而不能再指望援军。

这意味着两军必须通力合作。

苗授之前为了向高遵裕示好,特意将他在鸣沙城得到的那点存粮,分了一半给环庆军。但依然没有能买来一个‘好’字

进帐门前,苗授远远的向环庆军的营地望了一眼,那里还在为姚麟今天的大捷在庆祝着,营中灯火通明,也不知道还有多少酒水可以供他们消耗。

苗履也随着父亲向同样的方向望了一眼,忍不住冷笑了一声:“斩首一百七十级,也好意思摆酒庆贺。”

“地方不一样。”

苗授完全没有贬低姚麟功绩的意思。

如果是在横山的崇山峻岭之间,一百七十这个数字的确算不上什么。但眼下是在骑兵可以纵横驰突的平原之上。四条腿的骑兵冲击严阵以待的步兵军阵也许很难,可遇上战事不利,却能转身就走,步兵想拦都拦不住,就是骑兵也只能比比谁的马快。能有十分之一的伤亡已经可以说是惨败。

姚麟今天击败三千多铁鹞子,顺手还斩下来一百七十个首级。从斩首数上看,西贼的伤亡必然超过一成。在开战以来,已经可算是排在前面的大捷了,从难度上,更是首屈一指。

“但八百破三千,这个数目也不对劲。环庆军什么时候有那个本事了?”苗履说着,亲手为父亲掀起帐帘。

苗授走近大帐,道:“姚麟好运气,占到了天时地利。没听他说是顺风破贼吗?白天那么大的风,换作是契丹宫分军处在铁鹞子的位置上,也只有转身跑。追杀敌骑,追上了就是一个首级。”

苗履跟着进来,帐帘在身后放下,“要换做是儿子有三千骑兵,当时就能分成两部,一部两千人,用以抵挡敌军攻势。另一支千人队就绕道敌军后方,前后夹击,便能反败为胜。”

“这话别对外面说,省得被人笑话。”苗授在自己的座位上坐下来,给了儿子一个蒲团,让他坐下来说话,“你在被人偷袭时,能一下子数清贼军的数目?而且还是沙尘漫天的时候?不清楚敌军有多少,你敢分兵?你老子我都不敢!”

苗履被堵得不敢说话,苗授摇摇头,叹道:“有时间还不如多想想怎么打下灵州城。”

苗履冷笑道:“让高总管去想,他不是说有万人足矣吗?反正儿子是想不出来只用万人怎么攻下灵州城。这一回好好看看髙总管的本事。”

高遵裕将泾原军排除在外,只让环庆军参与攻城,这让苗履乃至整个泾原军上下都感到愤怒和羞辱。论起抵达灵州城下的前后,泾原军比环庆军还要早上一天。

有人是做不得高官。官位低的时候,才智、品性都不缺,官位一高,整个人就变了样。只知道争功诿过,这样的人并不鲜见。苗授对自己‘幸运’的撞上一个,也只能高叹无可奈何。

“早点歇着吧。”他心情有些郁闷的赶儿子去休息。

次日清晨,点卯和军议结束后,苗授领军出外巡视。

苗授要监视兴庆府的反应,要清理投靠党项人的奸贼,要堵住所有党项骑兵越过灵州城下的守军到后方骚扰的打算。

苗授手上的兵力就那么多。没办法面面俱到。幸好飞船终于能够上天了,从天上俯视大地,灵州城内的动作没有什么能瞒过飞船上的人。

正如他昨夜所预测,今天的天蓝得分外高远,天气好得让人不禁觉得延续了好几天的沙暴其实就是以一场梦。没有了如同帘幕一般的沙尘阻挡,灵州城外的远山近水尽数落入苗授的眼中。

这是一条夹河延伸的狭长绿洲,东面是荒漠,西面是高山。从贺兰山上流淌下来的雪水浇灌了大地,使得这里的土地如江南一般丰沃。

在兴庆府和灵州周围,是沟渠纵横、以万亩计的水浇地。水稻、小麦等五谷在田地中顺利生长,每年的收获,足以养活上百万军民。

而这上百万亩的田地,借用的是黄河水和高山雪水,通过百千条大小沟渠留到农户家中的田地里。

党项人自从占据了这片土地之后,在这些灌溉水渠上下的功夫不小,但对于围城的大军来说,却也十分的危险。

不论泾原军还是环庆军,在扎营时都是特意挑了几处地势略高的地方。其实也就是城外的村落,里面的村民基本上在战前就被强迫移进灵州城中,留下的房屋,全都给烧了去,营帐就设立在废墟上,稍稍清除干净,就将营寨搭建了起来。

可一旦党项人使用水攻的话,只要在合适的位置掘开几条水渠,便能让灵州城外成为水乡泽国。到时候就算只能守住他的营地,对灵州城也只能望而兴叹。

“河堤上要小心了。”

苗授也不知道是对谁在说着。但他很快就派了人去河堤上巡视,可见他对水攻的畏惧。

眼下正是雨水多的季节,黄河中流水湍急,水位又髙,堤坝给掘开,灵州城下可以划船进出城中。

但掘开黄河的结果,是同归于尽。正常人还不至于选择这一条。多半是挖开兴灵之地的千百条渠道,只要从其中挑选出一条或是几条干渠来,最后得到的就是一样的结果。

苗授回头望着接近到地平线上灵州城,小小的仿佛抬抬脚就能走上去。不过四丈髙的城墙能让所有没有准备好攻城器械的士兵感到绝望。

到底从哪里找来足够的木料,这让两路总管都陷入巨大的困境中。近处缺乏木料,而远方则运送困难。从道理上说,这件事跟粮秣转运上遇到困境都差不多。

是不是换个办法,将城中西贼引诱出来?苗授想着。

现在官军还有足够的实力,若是再拖下了去,情况可就不一定了。

