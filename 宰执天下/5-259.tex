\section{第28章 官近青云与天通(26)}

蒲宗孟跪坐下来开始起草王珪的罢相诏。

这笔得罪人的买卖被张璪丢给了他,不过蒲宗孟不在意。王珪完蛋了,这辈子都翻不了身,在诏书中踩上两脚反而能让皇后看着喜欢。

当初不小心触怒了龙颜,战战兢兢了多日,靠着运气才过关,眼下他可不想再犯糊涂。

打叠起精神,亲自磨好了浓墨,蒲宗孟打算用他堂堂玉堂华选的文笔,让王珪看得吐血,让皇后听得心花怒放。

罢去了最不招人喜欢的王珪,宰相之位已然空悬。将会是谁接手东府之长的位置?

张璪和孙洙都在偷眼看肃然而立的吕公著,不见喜愠,深沉难测。两名翰林暗暗称赞,只是这副宠辱不惊的气度,就是标准的宰相。

张璪尤为欣喜,王珪罢相,吕公著进用,两府人事大变,他这位站对位置的翰林学士承旨晋身两府的几率实在很大。

还在殿中的吕公著却被张璪、孙洙看得极不自在。两位翰林学士在想什么他很清楚。他其实恨不得就此离开,但开口求退的时机却不好把握,现在天子开始发布谕旨,他也只能先在寝殿中做个石雕。

赵顼继续眨着眼睛,下一个字是‘去声十卦——拜’。

张璪、孙洙立刻精神抖擞起来,神情专注的观察着天子。

“官家可是要拜相?”在得到了丈夫的肯定之后,向皇后接着问道,“官家想拜谁做宰相?”

去声九泰——蔡!

正在动笔起草罢相诏书的蒲宗孟手一颤,长长的一道墨痕从下划到上,这张草稿是废掉了。

干咽了一口唾沫,张璪强忍着回头看吕公著的念头,而孙洙则没忍住,飞快的瞥了吕公著一眼。一瞥之间,就见吕公著神色依然如故,完全看不到什么异样的地方,但孙洙总觉得太子太保的脸色很白很白。

竟然不是吕公著,而是蔡确!

以资历论,如果要蔡确和吕公著同时拜相,吕公著必然在前,而且吕公著本人就在这里,天子不可能在蔡确之后才提他的名字。

吕公著完蛋了。

三名翰林学士皆看到了结果,却都想不通缘由。自请留对的吕枢密,怎么变成了引火烧身?

而蔡确的运气更让他们羡慕,蔡确升朝官才十年,就已经升到宰相之位了。而且还是从御史一直升上来,连出外都没有过一次。

羡慕到让人恨呐!

翰林学士们五味杂陈,而天子,并没有停止他和皇后的交流——入声三觉——确!

赵顼亲自确定了宰相的姓名,向皇后稍稍安心了一点,至少蔡确的立场她今天已经确认了。

张璪领了旨,与蒲宗孟并排跪坐,开始起草蔡确的拜相诏。

但赵顼的眼皮仍没有停,又是‘去声十卦——拜’。

难道还要一名宰相?!

张璪和蒲宗孟同时停笔,等着赵顼的谕旨。

上平七虞——枢。

入声四质——密。

“是拜枢密使?”向皇后得到了赵顼的确认。

上声六语——吕。

三名翰林学士的呼吸都停滞了,西府中已经有一个吕了。再来一个,难道会是……

去声八霁——惠。

下平八庚——卿。

宰相蔡确。

枢密使吕惠卿。

……………………

“新法大兴啊。”韩冈冲苏颂举起了酒杯。

已是入夜时分,学士院依然锁院,翰林学士们依然留于宫中,但皇城在日落后便落了锁,将等结果的朝臣们全都赶了出来。

谁也不甘心回家去等消息,留到明天再看结果,更是没人有这个耐心。

所以御街左近的酒店茶肆,在这一个冬夜里便人满为患,甚至州桥边的夜市中也坐满了衣着青绿的官员,间中还点缀着一两件朱袍,都在等宣德门处贴出来的诏书榜文。

韩冈和苏颂也到了前些天他和章敦一同饮酒的西十字大街横巷中的小酒店里,坐下来等消息——章敦今日宿卫宫中,倒是没能一起来。

黄裳也没作陪,前面韩冈和苏颂的对话让他一头雾水,有了些自卑感,听着也是没意思,回住处读书去了。早点中了进士,才有参与韩、苏议论的资格。

此处离着宣德门并不算远,在吓走了几名小官后,接下来倒是清净了。

坐下来不到一个时辰,王珪罢相,蔡确拜相,吕惠卿回京任枢密使的三条重磅新闻,便由留在宣德门处的元随,送到了他们这里。可想而知,整个京城都要沸腾了。

“吕与叔自取其辱。”苏颂叹道,天子当着枢密使的面又任命了另一名枢密使,而且还是对立的派系,那么这名枢密使就只有一个选择,“旧党在朝中已没有立足之地了。”

韩冈笑而不言,举杯饮酒。

蔡确是新党,吕惠卿是新党核心,王安石更不用说——唯一的精神领袖,两府之长加一个平章重事都由新党担任,那么理由就只有一个,赵顼已经不打算继续使用旧党维持朝堂平衡了。

“是不是要恭喜玉昆?”苏颂举起酒杯,笑着回敬韩冈。新党大兴,为了朝廷稳定,势必需要一个反对者。提前做了准备的韩冈,自然是最佳人选。

韩冈却摇摇头,叹息道:“如果天子不是当着吕宫保的面任命的吕吉甫,这恭喜小弟倒是可以觍颜受了。”

吕公著辞位,东西两府全在新党手中。韩冈的资格还不足,势力又薄弱,完全替代不了旧党的位置。

吕公著失势,但留在西府中做靶子,韩冈所代表的气学成为钧衡朝堂的新生力量,那么朝堂上将会出现一个稳定的三角形。这是韩冈预计的,但现在的情况完全不是这样。

苏颂一点便通,皱眉想了一阵,道:“……如果有第二位宰相倒是好办了。”

韩冈笑了:“若是天子还要提拔一名相公,怎么会放在吕吉甫的后面?”

宰相的位置可要在枢密使之上,拜枢密使的诏书都出来了,韩冈不觉得今天天子还会任命第二名宰相。

“说得也是啊。”苏颂一声叹。天心难测,皇帝的想法实在是很难琢磨明白。

拿起酒壶,苏颂随兴的给自己和韩冈倒酒。可突然间他整个人都怔住,酒壶倾斜着,只见壶中的烈酒,溢满了银杯,流到了韩冈的手上。

苏颂应该是想到了什么,韩冈没有吭声,让酒水继续流淌,静静的等着苏颂自行清醒过来。

“我明白了!”当银壶中的酒液将将倾尽,苏颂终于回过神来,一声断喝,但一看到看着满桌的酒,他就吓了一跳。

韩冈却哪里会在乎桌子,立刻抓着苏颂问道,“怎么回事?”

“新党大兴啊,玉昆!”苏颂重复着韩冈的话,浅淡的微笑里,自有深意在其中。

韩冈闭了闭眼,顺着苏颂的话意去思考,灵光随即闪过,这不正是郊祀之夜的翻版!

“原来如此!”他点着头,这下如何不明白,“好个官家!好个官家!好一个盛极则衰!”

“的确是盛极则衰。”苏颂招呼韩冈换到另一张桌子上,“新党大兴……那接下来呢?”

“自然是四分五裂。”韩冈冷笑着,“烈火烹油,鲜花着锦的日子如何过得长久?!”

只会是这个原因了。

韩冈对赵顼的决断力不无佩服。冬至之夜的时候,就已经有这个感觉了,现在则更为深刻。那是为了儿子能顺利即位成人,他极为决绝的抛弃了新法。而今天,则又决绝的抛弃了旧党。

一切的关键,还是因为皇后这几日对旧党的看法变得极为恶劣的缘故。今天在朝会上,不少人都看出来了。所以天子才会放弃旧党。要不然留在朝堂中打擂台难道不好吗?

当是天子确定了即便留着旧党,皇后主政时,也会在新党的撺掇下将之全数逐出京城。那么也只能干脆一点,与其等着日后朝局混乱,还不如自己还能控制局面的时候,将一切都给皇后安排妥当了。

当初赵顼能干脆了当的抛弃新法,抛弃新党,如今也能以近似的理由,抛弃旧党。吕公著的算盘,终究是还是从自己的角度来考虑问题,而不是从皇帝的角度。

还是那句老话:屁股下的位置不同,对事情的看法也同样不同。

韩冈屈指敲着桌面,苦笑着,其实自己也有这个倾向,否则应该能猜到赵顼会怎么做,而不必现在这般惊讶。

所谓盛极而衰啊!

当朝堂上只剩新党后,仅仅是精神领袖的王安石决然压制不住内部分裂的倾向,吕惠卿绝不是甘居人下之辈,而蔡确只会更加贴近皇后。如此一来,新党必然会分裂。

尤其是吕惠卿,赵顼调他回来,一方面加强新法、新学,另一方面,可就是让他自立门户。

外有韩冈与新学争道统,内里则因权柄而自相攻伐。就算没有了旧党,依然是个异论相搅的局面。平章军国重事的王安石可以将政争压制在合理的范围内,却弥合不了人心。

这就是赵顼的计算。

韩冈在想明白后,便不再放在心上。赵顼不过看着眼前,最多也就三五年后,而韩冈的目光所及,却是数以十年计,乃至数百年后的未来。

换了一桌新菜,苏颂拿着筷子夹着,一边与韩冈道:“蔡子正宰相,吕吉甫枢密,接下来会是谁?”
