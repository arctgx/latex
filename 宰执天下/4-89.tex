\section{第16章 夜凉如水无人酌(上)}

遍地是尸骸。

刚刚还是震天动地的战场,此时已经一片死寂。只有手持大斧的宋军士兵静悄悄的走在被血水染红的地面上,挥动手中的大斧将他们的战果一个个收割下来。

左江、右江,珠江的两条支流在邕州汇合后,就称为郁水。郁水之滨,就是这一片战场。

郁水南岸,数以千百计的交趾兵站在岸边,久久不肯离去。而韩冈与他们对视着,嘴角翘起的纹路中之有冷笑——侵略者的下场,就在他的身后。

刚刚结束的一场战斗,应该是这一次邕州大战的最后一场战斗。

一方是拥挤在江边的、争抢着渡船的交趾兵,另一方,则是挟胜势,追袭而来的宋军。

孰胜孰败,在开战之前就已经确定。

能驱动士卒,背水一战的韩信没有出现在交趾军中。统领交趾后军的将领最后就带着四五百人反冲过来,不过他们背水一战的勇气,在士气高昂的宋军一个冲锋下,就被杀得烟消云散。

前日在昆仑关前的一场大战,李常杰溃不成军的败退下来。被追杀得丢下了近半人马,方才脱离了昆仑关的群山。直到他退到了平地之后,总领后方的宗亶领军来援。

面对兵力远超自己的对手,又是身在平原之上,韩冈手上兵力微薄,担心敌军反扑,没敢强逼上去。只能看着李常杰、宗亶两军会合,整顿兵马往南面江水方向撤离。自己就只能远远地吊在后面,等待着再咬伤一口的良机。

不过等到李信和黄金满的两部兵马带着捷报追上来,加上驻守在宾州的一千人也终于不用提防交趾兵再穿山而来,一齐汇聚到韩冈的麾下,情况就有发生了变化。

一时兵强马壮,连番大捷士气正盛,韩冈便又领军继续紧追了下去。就在郁水的渡口上,没能来得及过江的四千交趾军,终于被宋军咬上。轻松的战斗之后,一半交趾兵跳了江,一半则成了战利品。

“熙宁以来的战事,斩获应该是以此战为首!”李信的脸上掩不去心中的喜色

“只不过是交趾兵,比起党项人还差得太远。十个八个才能抵一个。”虽然如此谦虚着,但韩冈的脸上也带着浓浓的笑意。

这的确是一场破天荒的大捷,加上之前几战的斩获,连同一众俘虏,韩冈领军南下后,大小五六战,俘斩总数过万,而他手上的官军最多时也不过一千五百人。

“经此一战,交趾人的狂妄也该收敛一下了。”黄金满冷哼着,“他们过去可是逼着广源州年年去升龙府进贡!”

交趾军的核心主力在这一场战事中损伤甚多,额头上刺着天子兵的首级,已经数出了一千多。就算交趾国一年两熟三熟,粮食产量极高,但他们的常规军估计也就在五万上下,精锐的‘天子军’更是只有十分之一,而临时动员的兵马,不会超过十万。

只有攻到交趾国境内,其国中全民皆兵,兵力也许就会再多上几倍。不过那样拼凑起来的军队,只是来送功劳的,只要稍稍注意一下就足够了。西夏号称可用之兵七十万,基本上就是这样的情况。

而李常杰这一次带出来的七八万人,除去了三分之一广源州四大首领的盟军,剩下的五万兵,已经占了其全国能动用的总兵力的三分之一。

邕州一战,俘虏、斩首加起来超过一万,而受伤的只会更多,韩冈给交趾的打击不可谓不大,当是刻骨铭心的损失。如果算上此战对于军心士气的损害,还有国中部族因为此败而产生的异心,甚至十年内都补救不回来。

“这一战交趾国中精兵损失良多,国势必因此而不稳,李常杰回去之后,他身上的麻烦绝不会少。”韩冈没有幸灾乐祸,他还希望李常杰能镇压下交趾国中的反对势力。如果让交趾国中的反叛者成功了,占了李常杰的首级送过来,对他的愿望来说可就麻烦了。

黄金满没有韩冈想得这么多,他望着江水,“可惜没了渡船,要不然就能追过去了。如果能再追击百里,李常杰和宗亶别想稳定军中。”

韩冈摇头笑了笑,“就算有渡船,要安稳的过河也不是那么容易,说不定会乐极生悲。”

转回身来,韩冈看着立有大功的蛮帅,“黄金满!”

“末将在!”抱拳行礼的广源州大首领此时已经换了称呼。

韩冈手上的空头宣札只能封赠正从九品的武官,像黄金满这样能聚拢四五千兵力的大首领,根本没有任命的资格。不过经略使章惇此前已经确认了黄金满的身份,要推荐其为广源州刺史、并广西诸蕃都巡检。他的两个儿子,也一并得到了举荐——这还是章惇还不知道韩冈退守昆仑关后几场大战的战果,如果将他和李信一起击败两只偏师八千人、击败昆仑关下李常杰的两万兵,大战、小战一起算进来,一个节度使绝对少不了。

“今次邕州大战,若无你率军反正,决不会有今天将交趾贼军追击到郁水之滨的结果……”

“末将微末之功,何足挂齿。都是运使指挥有方,末将听命而已。”

“你也莫要自谦,你的功劳朝廷都会记着的。不过你投效了大宋,也许刘纪等人会就此报复……”

“多谢运使挂心,不过给刘纪十个胆子,他们也不敢再对小人的部族下手。反而要来求着运使恩典。”

“是吗,那样就好!”韩冈点点头,熟悉广源州形势的黄金满既然这么说了,当不会有错。而自己只要提过了,就算有事,也不是他的问题了。

“下面该怎么办?”李信问着韩冈。

“该去邕州了。……不过还有一些残兵留在山林间,要把他们都搜出来。还有这些俘虏!”韩冈英挺的脸庞有些扭曲,声音中满是寒意。

苏子元此时已经先去了邕州。从俘虏的口中,韩冈两天前就知道了苏缄殉国的消息,甚至连阖门死难的事都了解到了。苏子元不亲眼看见怎么都不愿相信,从山里出来之后,韩冈就给了他五百人,让他和黄全先去邕州恢复城中秩序。

交趾人在邕州所造成的杀孽,滔滔珠江江水也难以洗清。不仅是苏缄阖门死难,邕州城内的百姓死于战火的据说也是十中五六。

这些日子韩冈也不禁会想,如果自己再早来个一两天,事情会不会是另外一个结果。只是他南下的过程中,没有一时半刻的耽搁,他和章惇,以及下面的一千五百荆南军,已经是尽了最大的努力,用着最快的速度。没能来得及救下邕州,任谁也无法责难。所有的怨恨都归结于李常杰,还有他率领的交趾军!

“三哥儿,杀俘不祥!”李信连忙提醒着韩冈。不是他不怨恨交趾,杀光了对他来说也轻松,但他表弟的名声更为重要,“若是传扬出去,两府中的相公们不会体谅。”

“我不会杀他们的。”韩冈又换上心平气和的微笑,“还是之前的方法。愿意砍掉大脚趾的,饶他们一条性命,等朝廷来了恩旨,说不定就能放他们回去。如果不愿意,那就没办法了,我也没有多余粮食去养日后战场上的敌人!”

即便砍掉两脚的大脚趾,走路、种地、做工都不成问题,只是不能负重、不能奔跑,当然就不能再上阵。对于这些满手血腥的屠夫,韩冈自认为砍掉脚趾,已经足够宽宏大量。

“三哥儿,不是说钦州、廉州和邕州有几万人口被掠走,现在还在交趾人手中?说不定朝廷会要用俘虏交换他们回来。若是看了俘虏的脚趾,交趾人万一报复回来,朝堂哪里会体谅你的苦心,只会说三哥儿你生事!”

交换两国逃人的情况,宋辽、宋夏之间经常会有——如果是在面对蛮夷的边州,有时候在蛮部中,不堪奴役的奴隶逃过来,当地的守臣也会将他们还回去,也算是多少年来的规矩了——有时候也会交换一下俘虏,尤其是这些年的宋夏两国之间最为频繁。

不虐待俘虏,同样是大宋约定俗成的守则,若是有所触犯,少不了会受到弹劾。李信可是很清楚,章惇在荆南到底是怎么做的,过去在关西,又是怎么做的。韩冈若是犯了这个规矩,他在朝堂上的敌人,可不会放过这个机会。

“交换?”韩冈却是哈哈笑了起来,他并不在乎这么多,:“说什么笑话呐。失陷在贼人手里的百姓,我们要打到升龙府将他们救回来,谁耐烦跟交趾人磨嘴皮子?!”他瞥了李信一眼,“还是说表哥你只是将贼人赶走就心满意足?灭族屠国的功劳就放着不要了?”

“怎么会!”李信一下急了,“若是打到升龙府,领军的当然得有俺一个!”

“好!这才像表哥你平日的样子!”韩冈哈哈的笑着,“既然要平交趾,这脚趾当然也得剁掉!古有黥【刺字】、劓【割鼻】、刖【断足】、宫【阉割】和大辟【死刑】五肉刑,以这些交趾兵所犯罪行,只剁了脚趾已经是恩典了。”

