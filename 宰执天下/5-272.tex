\section{第30章 随阳雁飞各西东(三)}

午后的福宁殿,已经没有阳光能照进来了。

一支支玻璃灯罩中的蜡烛闪动着并不明亮的光芒,与尤笼罩在阳光下的殿外对比起来,殿内显得分外晦暗。

蔡确进内殿探视过天子,随即便出来。皇帝情况看起来跟前些时候相比,并没有太大的变化,但只依靠稀粥汤水来进食,却也不知能撑上多久。

向皇后正等在御书房中,屏风依然拦在书房中央,将内外隔开。

向着屏风另一面的皇后行过礼,蔡确便被赐了座。

“蔡卿今日请入对,是为了何事?”皇后的声音冰冷,似乎满是怒意。不过蔡确知道是为何如此,倒是一点也不担心。

“臣是为了今日枢密使吕惠卿的奏章。”

“是吕惠卿要朝廷速遣良臣知京兆府的那一份?”向皇后问道。

“正是。”

“蔡卿是何意?”

“殿下,如今枢密院仅有两名副使,已是勉强支应。一旦边境事起,自难顾首尾。吕惠卿当速速招还,命其主掌西府!”蔡确顿了一顿,“不过吕惠卿的奏章更是老成谋国之言,不能须臾拖延。故而臣今日请入对,恳请殿下速选良臣,命其直接就任京兆府,以稳定关中局势!”

向皇后想了一下,明白了蔡确的意思:“蔡卿欲举荐何人?”

“臣举知成都府蔡延庆。蔡延庆曾为秦凤转运使,王韶开拓河湟,其主管军中转运事。更是在王韶离任后接掌熙河路。这几年他在成都府路,先是配合王中正平定了茂州之乱,之后更是安抚了各个羁縻州的夷人。军政两事上,俱有才华,在陕西又不乏人望,以臣之见,如今能替代吕惠卿的人选,以蔡延庆最佳。”

“蔡延庆?”向皇后念着这个只是最近才常看到、还显得很陌生的姓名。她知道蔡延庆已经在成都做了好几年知府,同时又是成都府路的经略使,镇压西南夷甚为得力。只是他偏偏又姓蔡。透过屏风上的薄纱,看向蔡确的眼神中就不免有了几分狐疑。

蔡确好像什么都没感觉到,“蔡延庆旧年为王韶转运粮秣兵器,韩冈为其属,据闻甚为相得。究其因,多半是因为两人乃是京东同乡的缘故。”

世人都道韩冈乃是关西人,但实际看过韩冈家状的向皇后却知他的本贯乃是京东,不过是从祖父辈迁移到秦州。

“蔡延庆乡贯在京东?”向皇后神色稍缓。

“正是。”蔡确道,“京东莱州,乃是仁宗时参知政事蔡齐蔡文忠之侄。”

“蔡齐?可是大中祥符八年的状元郎?”向皇后问道。任何一科的状元郎的名字都是名扬天下,纵然是几十年前的人物也是一样。

“正是大中祥符八年的进士第一。蔡齐为状元,大得太宗皇帝赞,‘诏金吾给七驺,出两节传呼’,如今进士跨马游街便由此始。不过蔡齐子嗣艰难,曾以蔡延庆为嗣,后蔡齐病殁,得一遗腹子延嗣,蔡延庆随即归宗而去,不携一物。莱州官民,无不叹服其人义行。”

蔡确知道怎么说,才能让向皇后欣赏起蔡延庆。

向皇后听了之后,果然就点起了头。蔡延庆在继嗣承嗣上所表现出来的品行,跟某人成了鲜明的对比,“此人的确合适。让翰林学士院草诏,明天就发出去。”

“殿下。”蔡确连忙提醒,“蔡延庆现在成都,而新任之地乃是长安,若是照常例在就任前上京诣阙,一来一往就不知要耽搁多少时日了。当命蔡延庆奉诏后先行上任,待西事稍安,再招其入京不迟。”

“此事当然。”向皇后点点头,觉得蔡确说得很有道理。

“既蒙殿下应允,西事当可稍安。”蔡确离座起身,今天的第一桩事算是圆满完成,“臣已无他事,权请告退。”

“蔡卿稍待。”向皇后叫住了蔡确,“今日来自洛阳的一干士人的联名奏书,不知蔡卿看到了没有?”她的声音又冷了下来,甚至满满的皆是怒意,“内外勾结,任用奸佞,囚禁天子、圣母,真是好大的罪名啊!当吾是贾南风,还是武瞾?!”

隔了一层屏风,都能感受得到皇后那边传来熊熊怒火,让不大的御书房恍若盛夏。墙角处的内侍,各个冷汗淋漓。

蔡确眼神却闪过一层喜色,心道‘果然来了’。

“此辈狂生,心怀叵测,辱及天家,自当惩处之!”他立刻回道,但语气又随之一转,“只是若严加处置,反倒遂了他们的心意。”

“是吗?!”向皇后气得浑身颤抖,话声一下变得尖利起来,“难道政事堂觉得他们说得有理,打算听之任之!?”

“殿下有所不知。”蔡确轻叹,“旧年苏轼苏辙兄弟同赴制科,苏辙文中论仁宗自奉过奢,喜好声色,致使国用不足,而宰相不敢谏,司会不敢争。执政皆论其策不对问,当黜落,而仁宗则道此乃贤良方正能直言极谏科,‘求直言而以直弃之,天下其谓我何!?’故而将苏辙列为第四等。”

“这是伪作鲁直,以后世之名要挟仁宗!”向皇后怒喝着。仁宗的话中可都把苏辙的小心思给点出来了。

而且别以为她看不出来。儒生上书乱骂人,然后博个名气,这样的人多得是,也见多了。她的丈夫每年总要有个三五次被气上一回。明明是皇帝,偏偏还回嘴不得,更不方便降责,就是明升暗降,也会被人拆穿,最后惹起一片叫屈声。只能放着不理,然后躲在宫里生闷气。倒是那些重臣,反而不会乱说话,也容易处置。

她突然间明白过来,“蔡卿的意思是今天的这一封奏书,也是一样的心思?”

“殿下明见。”蔡确道,“制科之难,远过进士一科。一二等向不授人,能入三等者,几十年来亦不过三两人,四等便已中格。王平章旧日亦曾说,苏家父子之学,乃是战国纵横家一流。伪作鲁直挟圣君,却是纵横家惯用的手段。而今日上书之人,更是无才无德,除了伪作直言以邀名,别无进用之法。”

“这样的人还想得用?!”

“纵不能用,也不能加罪。在世人眼中,是是为国无暇谋身,纵使说错了,也是好意。”蔡确叹道,心中却是大乐,“若是将之责罚,反是为其扬名。之前洛阳就有回报,说是嵩阳书院的一干学子意欲为流言叩阙,不知怎么就改成了上书。”

“这件事之前政事堂怎么没有上报?!”

“一开始只是有所传闻,不敢遽然相信,直到今天终于看到章疏,方才知道竟有人大胆如此!”

“……可能查得出是谁在煽惑人心?”

蔡确摇摇头:“流言蜚语,如同浮灰飞絮,如何查得出来路?”

他不打算将旧党再踩上几脚,只要在皇后心里再留上一根刺就足够了。查出了明确的犯人,就会怨有所归,而查不出来,恶感日积月累,皇后对旧党的压制,将不再会局限于吕公著、司马光那区区数人。

向皇后的心口上像是给堵了一块石头,怎么都顺不过气来。她临危受命,一心想将这个国家平平安安的治理好,对得起丈夫,对得起儿子,谁能想到,那些深受重恩的臣子,一个接着一个想要翻天。先是明着欺上门来,幸好朝中还有忠臣。等到被天子一股脑的打发干净,又立刻在洛阳传递谣言。

“上书的人确认出自是嵩阳书院!?”虽然这篇满是道听途说捕风捉影的文章中并没有写,但方才蔡确一提,向皇后就已经牢牢记下了。

蔡确微微一愣,“……是。”但问题并不是在这里,嵩阳书院只是表征,重点是其背后的旧党。

“新任的资善堂说书,程颢是在嵩阳书院里教书吧?”对于丈夫给儿子找的两位新老师,王安石不必多论,名不见经传的程颢,向皇后怎么可能会不去打听他的底细。

“确实如此!程颢、程颐兄弟于嵩阳书院授徒多年,司马光亦曾讲学其间,吕公著也曾多次造访。”

“程颢他也是韩学士的老师吧?”

蔡确更正道,“仅是半师之谊。”

“半师之谊……”向皇后念了一句,像是在咀嚼这个词的含义,继而又道,“听说韩学士曾经立雪程门,站了半日之久。”

“确有其事。”

“当日韩学士都已经是功臣了,雪地里站半日,官家都不能这么做。”皇后的语气变得危险起来。

蔡确越听越是觉得不对劲,话题怎么越扯越偏了。

“相公意思,吾已经知道了。但程颢乃是陛下亲任的资善堂说书,一时也不能拿他如何。”向皇后,满腹怨气的说着,乃至咬牙切齿,“一面说吾勾结外臣,囚禁天子、太后,一面又干干脆脆的接下诏命。这事倒是做得漂亮啊!”

蔡确张开了口,想说话,却不知该怎么说。

怎么误会到了这一步?!旧党都成了死耗子,让皇后继续保持对他们的恶感,将嵩阳书院视为旧党的巢窟也就足够了,却不是要皇后对程颢有何动作。

程颢虽然在嵩阳书院里面教书,而上书的也是嵩阳书院中的士子。但以韩冈对师长一直以来的尊重,纵然日后,也不会容许无缘无故的加罪于他。

但听现在的口气,恐怕稍待时日,就要拿程颢开刀了。

这事真的麻烦了!

