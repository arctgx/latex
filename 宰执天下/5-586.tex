\section{第46章 八方按剑隐风雷(六)}

 姆百热克抱着头,瑟瑟发抖。

周围是嚎叫、是厮杀、是刺鼻的血腥味、是刀砍过骨头的让人牙酸的声音。

无尽的混乱中,这位年轻的回鹘贵族,已经失去了所有的勇气,忘光了战前的豪言,与黑汗人同样是本家血仇的汉人,在姆百热克心中是救苦救难的菩萨,默默祷告着,期盼他们早点到来。

除此之外,他就只想忘掉眼前的一切,躲进一个没人能够找到他的地方。

可是混乱的营地中,哪里有半寸净土?

黑汗军攻入城中的人数不断增加,守军占据的空间越发的狭小,一片片营帐被控制,守军已经被压制到了几个角落处。

如姆百热克这样被冲散的回鹘士兵,躲藏在营地之中,被同样分散下来的一批黑汗人仔细搜寻着。

从来没有躲藏的经验,他的踪迹很快便被发现。一名黑汗人正拿着弯刀逼过来,脚尖都已经到了眼前,已能听见呼吸,可是姆百热克除了将自己缩得更紧一点,什么也没有做。

远方的城门处,手持长兵的汉军,从城中杀了出来。

但忽然间,那人突地抽身而出,紧张的向四周张望了一下之后,便轰然倒在了地上,弯刀叮当落地。

“你就躲在这里?”头顶上传来低沉的声音。

听到声音的姆百热克浑身一颤,小心抬起头来,先是见到一支白色翎尾的长箭正深深的扎在那名黑汗人的眼眶中。在那黑汗人的身边,是穿着一双尺寸极大的脚,穿着汉人大官才会拥有的箭靴。

姆百热克视线上移,出现在他眼前,是几千汉人中他最熟悉的面孔。

“小人……小人……”

操着并不熟练的汉语,姆百热克趴在地上语不成句。

一击毙命的神射,在万军之中,当然只属于一个人。他不知多少次从同伴的嘴里,听到过那让人畏惧又憎恨的神技。

被王舜臣救了一命,但姆百热克的心中更为惶恐。他胆怯畏战的模样完全落在了王舜臣的眼中,这在军中是完全不能饶恕的罪行。又是被杀人只看心情的恶鬼看见,姆百热克在那一刹那已经有了的丢掉姓命的觉悟。

哼了一声,王舜臣都没有再多说一句,换了个方向,向前走去。

跟在他身后一批近卫,也都紧紧的追随在他周围,不留下一点空隙,更没有多关注姆百热克一眼。

姆百热克这是才发现,跟着王舜臣的,也就只有着这数十人。其他汉军,都在更远处的寨墙边,从他这里可以看到,手持五尺长刀的汉人正与黑汗人厮杀在一起。

看看寨墙处的激烈战斗,又看看一边前行一边射击的王舜臣。

姆百热克脸上表情将他心中的挣扎展示了出来。看着王舜臣渐走渐远,他僵硬的身体终于有了动作,提起自己兵器,然后跨步追了上去。

越过那名倒霉的黑汗人的时候,他脚步顿了顿,探出手去,用力的拔下了那支长箭,也不顾上面的污血,就插在自己腰上的箭囊中。这是吐蕃人和回鹘人共有的习俗,拿到一名神射手射出的长箭,收在箭囊中,以求能得到一身的好箭术,更能辟除被射中的危险。

姆百热克飞快的追上了王舜臣,几十名近卫都用警惕的目光看着他,手上长刀随时准备落下。

“跟上来了?”王舜臣却回头看了他一眼,又不知对谁说:“将箭囊给他。”

不明所以,姆百热克楞在了当场。

“拎好了!”

十几件装满长箭的箭囊递到他的手中,王舜臣的亲兵不假辞色的吩咐着,姆百热克终于明白过来。

王舜臣转身前行,遥遥丢下话,“跟上来。”

“还不快跟上!”亲兵轻踢了姆百热克一脚,催促道。

姆百热克慌慌张张的抄起箭囊,连忙追在王舜臣的身后,心中却如同有一块巨石被推开,终于安了心下来。

营地中的战斗,此时越发的炽烈起来。虽然攻击城门失败了,但攻入营内的黑汗士兵人数越来越多,规模也越来越大,几乎所有的守军都被分割包围。他们缺乏组织,只能任人鱼肉。

至于汉军,不论是杀入营地内的汉军,还是城头上的汉军,又或是守在寨门前的那一批,都将注意力放在了寨墙上,希望能够堵上黑汗人进入营地的通道。正与他们一寸寸的奋力争夺着。

可作为全军主帅的王舜臣,却在此事放弃了对汉军的指挥,反而带着几十名亲卫穿行在营中。

不过他看似随意的走着,但总是能够踏在关键的位置上。

走到营地一角,前方数百名守军被压缩到角落中,围着他们的黑汗人军人数甚至没有超过他们,可能还更少一点,但由吐蕃与回鹘两家参杂的守军,却完全没有组织起来防御甚至反击的动作,而是在黑汗人的攻击下挤得越来越近……

看到这一幕,王舜臣只是一箭射过去,接着又是一箭,然后三箭、四箭,转眼间便将黑汗军中的几名军官全数射杀当场。失去了指挥官,这一支黑汗军顿时陷入了混乱,而王舜臣的亲兵更是分出了一半,冲入了混乱的敌军之中,大肆砍杀起来。

终于盼来了援军的出现,那些守军的士气陡然上升,勇气重新回到了他们的心中,让他们纷纷逆冲而上,挥刀将方才几乎把他们逼入绝境的黑汗军,砍成了一块块碎片。

解决了一方敌军,王舜臣又是转身便走,而那群士兵看见了主帅行动,聪明的都纷纷跟了上去。

王舜臣就这样救了一处又一处,转眼间就汇聚了人数近千的大队人马。不用他出言指挥,只要他箭矢所向,他身边的勇士就会群起而攻,将那人给砍成肉酱。

‘终于能派上用场了。’

王舜臣回头看了看追随着他的吐蕃人与回鹘人。之前因为担心士气而不敢使用它们,现在却可以尽情的使唤了。

他带过来的援军正攻向寨墙,截住黑汗人后方的支援。但寨中不靖,便没有机会稳定营地内部,更不用说反击敌军。

王舜臣只要看见敌军,便会拉动弓弦。在姆百热克的眼中,他手上的动作快得能拉出虚影。

不过这样的箭术,也最为消耗箭矢。跨在王舜臣腰间的箭囊,转瞬就稀疏了下来。

丢下的空掉的箭囊,王舜臣回头看了姆百热克一眼,领会心意的回鹘贵胄,立刻双手递上了一份新箭囊。

一队黑汗人杀了过来。他们刚刚整顿了队形,足有两百多人,在混乱的营地中,显得人多势众。

他们并不知道王舜臣真正带领的只有亲卫,其他都是惊魂甫定的杂兵。看到这边人马众多,黑汗军立刻心中便有了胆怯之意。正挣扎于走还是留,王舜臣的神射让他们不用再为难了。

当一名、两名的军官被射倒在地,还有人叫嚣着要给他们报仇雪恨,但随着弓箭的消耗,投降的士兵便越来越多。

最后甚至不用王舜臣射倒首领,只用弓稍一指,他麾下的守军便会将他面前敌人一举扫空。

被王舜臣解围救出的守军越来越多,这些人有了王舜臣撑腰,又见援军正在奋战,更是没有半点畏惧,直接冲上去你一刀我一刀的交换起来。

营垒的大门这时候仍在是激战中。

黑汗人进入营垒后的两大攻击目标,一个就是城门,另一个便是营寨之门。对城门的攻击彻底失败,而争夺营寨大门的行动,则陷入了泥潭之中。

战事胶着,王舜臣的到来却让天平大大的倒向了大宋一方。

被王舜臣紧紧盯着,上千名回鹘士兵奋勇杀敌,内外夹击的后果,便是这一部黑汗军全军覆没,再也没有一个人能够逃出生天来。

营地内的厮杀在飞速的减少,原本震耳欲聋的嘶嚎怒喝,短短时间中几乎再无听闻。

只用了一刻钟便将被分割的西营守军重新组织了起来,并顺利的夺回了营寨,就连寨墙上的争夺,也随着城内外战斗的平静,而变得稀少起来。

黑汗士兵逃走了一部分,又被杀了一部分,剩下的俘虏,多达上千人。

敌人尚未肃清,也没人有心思在这时候处置俘虏。不过聚集起来的将校们,还是恭喜并感谢王舜臣的奋战,让他们得以逃脱黑汗人的弯刀。

“没什么。”

王舜臣的口气,这仿佛微不足道的小事一样。

他本来可以直接指挥最亲信的汉军,领着他们整治城中。但谁也不敢保证,黑汗人会不会将古拉姆那样的精锐,混在这几处攻击点上,又或是藏在人群内。

为了避免仆从军大量阵亡,王舜臣就吩咐了汉军去攻击城墙,而他本人则亲自带着人赶去营救回鹘人和吐蕃人。只有王舜臣才有着让众军俯首听命的权力,也只有他能在将人救下来后,让他们一起去解救更多的人。

这个选择,让王舜臣以最快的速度恢复了营寨中的秩序,也让他可以策划接下来的反击。
