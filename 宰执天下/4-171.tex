\section{第26章 鸿信飞报犹觉迟(三)}

【昨天有事,欠的一更,今天补上】

“‘吾何罪而至是’?”耶律乙辛讶异的睁大了眼,显然是难以置信。

“‘吾何罪而至是’。”萧得里特沉沉的点头,表示自己并没有说错。

耶律乙辛摇摇头,又撇撇嘴,不知该说什么好的啧着舌头。过了一阵,才破口失声而笑。先是呵呵的轻声,而后笑声越来越放纵,最后竟是难以遏制的纵声狂笑起来。

亲自押送废太子去上京流放地的萧得里特,还有直接参与太子谋逆一案的萧十三也陪着大笑了起来。

三名辽国重臣,笑得恣意狂放,一直压在他们的心头上的巨石,终于是挪了开去。用了近三年的时间,总算是看到了触手可及的结局。

这般幼稚的话,不到绝望到神智昏聩的时候,怎么可能会问得出口?!

纵是父子之亲,轮到帝位谁属的问题上,就没有什么亲情可言了。南朝在史书中涂脂抹粉,实际上还不是如此。而北方寒地,父慈子孝也不是没有,但到了争权夺位的当口,也别指望会留情。

加上十几年前的皇太叔耶律重元谋反一案,至今犹在朝堂上和大辽天子心中留下深深的阴影,对谋反一事最是紧张和提防。

耶律乙辛只是指派手下,去告发东宫近臣和几名宿卫正密谋拥立太子登位,准备好了证人证据,接下来就是等着皇帝的雷霆之怒,落到太子身上。

在萧皇后因通奸案被赐死之后,作为她亲生儿子的太子殿下,怎么可能还在天子的心中保着原有的地位?

大辽的魏王殿下冷冷笑着。耶律浚几年前兼领南北枢密院时,开始针对自己下手,自家为此都将身家性命全压上去作赌注了,他却没有这个觉悟,最后落得废为庶人,囚于上京的下场,又有什么好抱怨的?

萧得里底向着他的主人为自己表功,“末将已经让人围着房子建了一圈高墙,只有鸟能飞过去,墙上只留一道小门通饮食,又有一队人马紧紧看守着,任谁也别想跟里面多说上一句话。”

“终究不能关上一辈子。”萧十三向上指了指,“上面可就这么一个儿子!”

这是不说废话吗?瞥了表情中带着狠决的萧十三一眼,耶律乙辛哪里还需要人提醒,更不需要人催促他不要留手。

“有皇孙就足够了,”耶律乙辛重复着,“有皇孙就够了!”

再一次在耶律乙辛这里得到确认,萧得里底和萧十三终于放了心下来。毕竟他们做的这等事,若是爆发出来,抄家灭族都是轻的。不能绝了后患,夜中也不能安寝。

只要有着如今正当幼龄的皇孙作为继承人,耶律浚这个人就不必存在了。而如今的天子正当壮年,等到皇孙即位,还有很久很久,中间出个什么意外都不足为奇。

放下了心头事,萧得里底忽然侧起耳朵,有些纳闷的问着:“都这时候,怎么没听到出猎的号声?”

“这一个月来,上上下下可都没有出去打猎了。”

萧得里底先是愣了一下,随即恍然,“怪不得昨日回来时没看到飞船呢……这倒也难怪了,毕竟这一次闹出来的不是皇太叔。”

亲生儿子要造自己的反,耶律洪基哪可能有个好心情。秋天是一年中最适合打猎的时候,但耶律洪基已经有一个多月没有出去游猎了。这在过去是不可想象的。

虽说游猎四方,是大辽天子用来威慑并安抚边地部族的必要手段,但喜欢打猎,到了当今大辽之主这个份上,已经可以说是本末倒置了。无论是春夏秋冬,只要是合适打猎的时候,他都会跳上马直奔猎场而去。到了猎场,又是从早到晚都在拉弓射箭,甚至于不眠不休的时候都有过。

话说回来,若不是耶律洪基对游猎的爱好大过处理朝政,也不会有如今耶律乙辛把持朝堂。从平定皇太叔之乱后,耶律洪基对于政务处理的琐碎事务,越发的感到不耐烦起来,只想着沉湎于轻松的游猎生活中,而不是为了国政耗费太多的个人精力以及时间。

但耶律洪基绝非蠢人,他只是嫌处置政务太麻烦而已。诗词做得好的很,与臣子相唱和,诗作集结而成《君臣同志华夷同风诗集》,年轻时也是勤勉,满脑子的励精图志的想法。但现在,却是抛下了所有的事务,将自己的爱好发扬光大。

耶律乙辛如今虽说是把持朝政,但日常也是战战兢兢如履薄冰。他的根基太过于脆弱,甚至可以说是如同一根手指粗细的树枝,只要外力稍微强上那么一点,就能让耶律乙辛如今的权势和地位墙倒屋塌。他的背后,可不会有全力支持的自家部族。

万一有一天圣眷不再,当即就能让他辛辛苦苦搭建起来的势力在一瞬间土崩瓦解。就算在当今天子治下,一直能得宠下去,到了下一任天子继位,也很难再保证如今的地位。

不过那还是日后的事了,可以慢慢的考虑对策。眼下解决了最棘手的问题,当然是得庆贺一下。

耶律乙辛拍拍手,诏来在外守卫的士兵,让他去让人做些准备。专门摆下了一桌酒宴,用来招待两名为与此事奔走的得力手下。

酒过三巡,萧得里底拿着酒杯开口询问他北上的这个月里,南朝是否又有什么动作。

“南朝最近又调了一批河北的军队去关西。多半是用来练兵的,”萧十三说着,“为了平灭交趾,南朝只出动了一万人,其中西军只有五千,一番大战下来虽然辛苦了,得到的回报却是确实没有一点折扣。西军之强,也是有口皆杯。”

“萧药师奴那个废物。”萧得里底至今犹对丰州的那一次败阵耿耿于怀,对于吃亏,他和耶律乙辛都由心理准备,但全军覆没决然想象不到的,“一个人都没跑出来,总觉得其中有鬼。”

“放在后面看守马匹的至少二三十人,上了战阵的死光了也就罢了,怎么一群留守的也没回来?有鬼是肯定的。”萧十三早就看出党项人在其中作祟,全没有安着好心,“梁氏兄妹,当真是胆大妄为。”

“他们有恃无恐。”耶律乙辛叹着气,“宋人都开始磨刀霍霍。西夏一去,接下来就是大辽的了,难道当真把党项人丢给南朝不成?肯定要在他们投降宋人之前,给他们一个保证。”

其实回头想想,萧药师奴全军覆没的疑点还是不少的,宋人多是步卒,就算是设下陷阱,也不该落到这般凄惨的田地。但耶律乙辛并没有去深究,真相并不重要,重要的是皮室军当真败了,败得毫无悬念。面对拥有百万强军的南朝,党项人已经是苟延残喘,若是大辽不在背后支援他们,结果只会是一个。

不过当时朝堂上叫嚣着的出兵报复,那是绝对不可能的,大辽并不占理,本是师出无名,在丰州更是打着党项人的旗号,现在怎么也不方便起兵。

当然,道理大义这等玩意儿,是从汉人那里学来的东西。草原之上,那是兵强马壮者称王,契丹人是不讲究这一套,需要时拿来妆点门面,不需要时,就用来擦靴子。但是,没有压倒性的实力,贸然动手就是损兵折将的结果。只好拿这番道理聊以自.慰,搪塞舆情。

自始至终,耶律乙辛都没有与宋国正面交战的想法。胜利对他没有好处,而失败,就是他的末日。目下的支持西夏,即便是宋国,也只会加大对西夏的支持力度,不会选择战争。

萧十三看得出来耶律乙辛对南朝的忌惮,而他自己也对宋国的禁军忧心不已,“用钱砸出来的百万大军啊……”虽然是讽刺的口气,但其中更多的还是有着七八分的羡慕。

“若仅仅是钱倒好办了,偏偏宋人如今的事却是用最少的钱,换来了最精良的装备。没听说吗,一套板甲和头盔加起来还不及过去的十分之一,百万铁甲都不成问题了。”

天下诸国,不论哪一家,若是跟宋人比谁钱多,那当真是傻了。家里只有两口羊一个帐篷的穷鬼,跟有着十几群羊的富户比家产。

而旧时宋人的钱没用对地方,不过如今已经不一样了。百万铁甲兵,加上神臂弓、斩马刀,又有飞船充作耳目来防备偷袭,甚至战马,也因为收复了吐蕃,也有了一个稳定的来源。这样的军队,只要不怯弱,拿得稳刀枪,上阵后至少能与同样数目的契丹骑兵相抗衡。其中战斗力最强的西军,更是不会输给皮室军和宫卫军。

摆在耶律乙辛面前的是两难的选择。每过一天,宋人的战斗力就会强上一分。但撕破盟约的代价,耶律乙辛不想付也付不起。幸好宋人的第一目标还是在西夏,在收复兴灵之前,宋人也无意挑起一场针对大辽的战争。

到底该如何解决和应对,这是一个必须尽快下决断的问题

