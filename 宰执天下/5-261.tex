\section{第29章 浮生迫岁期行旅(一)}

凌晨时弥漫在城中的寒雾,被腾起在半空中的太阳驱散了大半,可阳光照上身时,暖意还是没有感受到多少。

已经是腊月初,天寒地冻,比半个月前更冷了三分。晴空万里的日子,河中的冰层却又更厚了三分。城内的许多水井都冻住了,提不出水来。只有少许用蜀中凿井法凿出的深井,靠着地下深处地气尚暖,还有净水提供。

街边卖洗脸水的摊子上,一名小贩正吆喝着,身周热气蒸腾,水雾弥漫。生意倒是很好。五六个妇人、小子正提着桶在摊前排队。冬天的时候,洗脸水的生意总是最好。市井中许多人家懒得升灶化冰,干脆买水洗脸,然后出外吃早饭。

今天早上,新任的侍御史租的官宅里的水井同样被冻上了,出来时赶得急,也是不得不向外面买了洗脸水来洗脸。

在福建时,甚至是开封以外的其他地方,都不会有人能想到洗脸水也能拿出来卖,只有民风习逸成惰的京城,才能看到这样的行当。

“元长!”

来自身后的呼唤让蔡京从街边的摊贩上收回目光,回头看时,一名身着青袍的官员正骑着马过来,向自己招着手,惹得周围市民纷纷看了过来。

蔡京冷起脸,待那人勒住马,便冲他喝道:“强渊明,喧哗市井,惊扰百姓,今日你犯在我手上,等着被参劾吧。少不了你的罚铜治罪!”

强渊明被吓得不轻,连连拱手,“小的知罪,甘愿受罚。只是敢问,一天的俸禄有找吗?”

“哪有那么便宜的事?!”蔡京板着脸喝斥着:“好歹清风楼的一张席面!”

“请客可以,小弟也当请客。莫说清风楼,状元楼也成啊!不过元度来不来,小弟可是要好好谢谢他。”强渊明嘻嘻笑道,“当然,还要巴结一下元长你这位台端,小弟俸禄微薄,可要少罚几个大钱。”

“这台端做得殊无味。”蔡京却不开玩笑了,苦笑起来:“昨天你没看到,李邦直一来便给人下马威,还不知日后怎么说呢?”

“且不看,元长你在王相公和韩三资政那边都能说得上话,何惧他李清臣?韩魏王的侄女婿,要不是天子钦点,哪里能坐得上台长之位?”

蔡京笑笑,摇头不言。只是他私心里还是在叹息自己的资历,否则这一回就该是侍御史知杂事了。若是能做到御史中丞的副手,过两年去知谏院,再过几年升御史中丞,都是有先例在的。

可惜他现在只是别称台端的侍御史,主掌台院。虽然是乌台三院台院、殿院、察院中最高位的台院,终究还是比不上御史中丞的副手,有一条巨大的鸿沟,需要三五年的时间去跨越的鸿沟。如今次般连跳两级的运气,很难再有第二回。

监察御史的人选,照例是由翰林学士、御史中丞和侍御史知杂事三方举荐,然后让天子从中挑选,两府插手不得。不过宰执们要想在御史台里安插人手还是很容易,翰林学士和乌台长、副,都不可能是油盐不进的人。

台谏官可以指斥两府,两府宰执谁控制了台谏,谁就立于不败之地。现如今台谏空了大半,赶在韩绛、吕惠卿、曾布进京之前,这些缺额便被剩余宰执早早瓜分殆尽。

除了李清臣是天子钦点——他这位判太常礼院在郊祀前后的表现还算不错——其余人选,背后都有两府宰执身影。

蔡卞是王安石的学生,又在国子监中宣讲新学多年,如果没有蔡京的话,他进御史台不会有任何阻力。可是现在必须避亲嫌,所以蔡卞向王安石推荐了关系甚好的强渊明——其实蔡确和蔡京也有亲,蔡京的曾祖父和蔡确的曾祖父是兄弟,正好是五服中亲缘最远的缌麻亲。蔡京之前为御史时,曾在天子面前供述,赵顼没当回事,诏不问。所以到了这一次蔡确升宰相、蔡京晋侍御史时,倒是方便了,直接过关。

正在前面街口等着两人的赵挺之,他也被人推荐入乌台。不过私下里走的门路不是王安石,而是蔡确。

不过蔡京和强渊明过去的时候,赵挺之却在望着别的地方,并没有看着两人。

蔡京骑马过去:“正夫,在看什么?”

赵挺之回头一看,见是蔡京和强渊明,先打了个招呼,然后冲南门方向努努嘴。蔡京和强渊明转头看过去,只看见一票人马往南门去,浩浩荡荡的队伍有上百人之多,里面还多是朱衣的元随。

“蔡相公?还是王相公?”强渊明立刻问道。

只有宰相和枢密使才有如此规模的元随队伍。吕公著和王珪都已请辞,尽管还没批准,但他们出门后也不会再张着旗牌,带着元随。现如今的京城,也就新上任的蔡确和王安石,能有这般人数的随行人员。

“当是王相公吧。”蔡京道,“蔡子正今日文德殿上押班,初上任不可能告假。”

“是王相公,还有韩三资政。”赵挺之尤望着远处的队伍,目光中满是欣羡之色,“也不知是为了何事?”

“元长你知道吗?”强渊明问着蔡京。

“是来送人吧。”蔡京的确知道,“直舍人院的王安礼避嫌出外,前几天堂除他去江宁府任知府。”

“王相公自清得过分了。”赵挺之闻言摇摇头,“平章重事又不理庶务。”

“京师嫌疑地啊!”蔡京轻声一叹,又道:“而且王安礼又是跟苏子瞻一般行事不谨的性子,留在京城中徒惹人议论,早点出外也免得为人攻劾。”

“行事不谨?”强渊明道,“小弟只闻说他治衙有政声。之前曾有言或会代钱大府为开封知府。”

“不是传言,是真事。”蔡京道,“前几天翰林学士蒲宗孟论钱藻青城行宫郊祀前毁损之罪——这是恨钱藻不死——然后皇后就有意让王安礼接任,不过给王相公拒绝了,之后又以亲嫌奏请让其出外。”

“就因为他行事不谨?”

“可不是这么简单。前些日子……也就是冬至前,台中就有要弹劾他的说法,不过给耽搁了。现在还不让他出外,过些日子,小弟说不定都要上本了。”蔡京对两名同年好友笑了笑,“大臣狎妓,王安礼他做的是最肆无忌惮的,甜水巷中依红偎翠、放.荡形骸都少不了他。这还算不上大事,真正能拿出来论事的,一个是他知润州时,曾私致仕官刁约家侍婢,刁约死后又以主丧为名,诱略其婢女二人,另一个就是王平甫【王安国】刚满丧期新满,他便招妓饮宴。只为这两件事,王相公那边就饶不了他。”

强渊明吃惊道:“元长连这些都知道?”

“御史风闻奏事,若是耳目不灵,问题可就大了。”

“……多谢元长提点。”赵挺之向蔡京拱了拱手。

“也是小弟多嘴,进了乌台时间长了,自然会有有心人私下走报的。不必太过担心。”蔡京笑笑,又向南望过去,“不过韩三资政怎么也出来给王安礼送行了,两边来往听说可不多。他不是王相公,五日一上朝,庶务全不理。”

“怕是避白麻吧?”赵挺之笑道。

“张横渠的谥号交给太常礼院议了,《自然》期刊批了,千里镜的禁令也改了条文,可以说是弛禁了。可这韩资政还是看不起区区一个枢密副使啊!”强渊明的话中有着浓浓的酸味。

酸味是当然的,韩冈的行为让蔡京心里也是犯堵。

韩冈辞枢密副使的章疏,已经上到了第四本。谁也不知道天子会不会发下第五份诏书。这辞章的数目可比当年司马光辞枢密副使时还要多。而且之前韩冈已经辞过一次参知政事。在士林中的名声好得不能再好,就快赶上在民间的评价了。现如今,世人只盼他入两府,却不会有多少异论了。

不过蔡京的脸上却看不出来,一边催动马匹和赵挺之、强渊明往西门走,一边笑道:“韩三聪明得很,两府之中危机四伏,他哪里会掺合进去。只看郊祀之夜的定储之功,清凉伞在他而言乃是唾手可得,何必在乎迟早?”

这一点就不需要蔡京来解释了,如今半个京城都在议论天子对两府的人事安排。除了一开始时对两府尽数新党的惊讶,之后便很快就了解到了天子的用心。

韩绛和吕惠卿的恩怨,吕惠卿和曾布的恩怨,王安石和曾布的恩怨,蔡确这个见风使舵的新党和其他人的恩怨,两府中的恩恩怨怨都传遍了京城。

“元长说得是。”赵挺之大笑,“现在的两府是天子圣心独运,虽说皆是旧日同道,可东西两府不可能合得来,王平章也绝不愿看见曾子宣入政府。只为他,王相公连着两天请对入宫,好不容易才被安抚下来。等韩子华、吕吉甫和曾子宣入京后,照样好戏连台,比黄河龙门处的漩涡还险三分,韩玉昆如何会往漩涡里跳。”

“说反了吧,韩三进西府,害怕的该是吕吉甫和曾子宣。没看二大王、司马十二和吕枢密是什么下场?三大王现在多半已经到了地头,他是一刻都没敢在京里多留啊!”强渊明哈哈笑着,又一下收敛起笑容,“元长,说实在的,你这个殿中侍御史可是惹到他才得来的,可是险得很啊。”

蔡京知道韩冈肯定不会记恨,但能不去招惹韩冈,他是绝不会去招惹。就算再嫉妒,也是知道强弱之别,“韩资政器量宽宏,岂会在意这些小事。”他扬起鞭,“别说了,时间不早了。别李中丞到了,我们还没到。”

“说得也是。”

三人都是给解职出外的李定送行的。李定是受牵累而出外,有王安石在上面,很快就会回来,给他送行并不犯忌。大半个御史台都会到,当然不能耽搁时间。

三人挥鞭驱马,加速往西门行去。

疾行间,蔡京不经意回头,自韩冈以三章呈于天子,据说王安石和他没有再见过面,若说恩怨纠葛,王安石和韩冈这对翁婿,他们之间的矛盾可是更难调和。

今天以给王安礼送行的名义同行,也不知会说些什么。


