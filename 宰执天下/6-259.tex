\section{第34章 道近途远治乱根(中)}

类似的谏言,张孝杰在耶律乙辛的面前提过不知几回。完颜部每并吞一次相邻的部众,都会引起张孝杰的警惕。对于这个如同野狼一样难以驯服的部族,他天然的就有着极深的不信任感。

按照宋人的说法,只要不立文法,那就根本不用担心哪家蛮部能够坐大。但一旦有哪家蛮部立了文法,那就必须要出兵剿杀了。

完颜部本是野人,并无制度,但大辽给了他们文法。自从坐上了生女直节度使的位置,有了官署,也就有了属僚。

节度使兼理军政,椽属自置,本就是一个小朝廷。在完颜部的部族长而言,生女直节度使本是空衔,可他们从中知道了节度使辖下有哪些官职,并以此安置之后,这文法也就建立了起来。

随着完颜部的势力渐增,他们的危险性也就一天比一天更高了起来。

“完颜部的人口日渐增多,仅仅数载而已,五国部便有好几个小部落已经投靠到完颜部的旗下。东海女直四分五裂,年年相互征伐。待完颜部并吞五国部后,不用十年,东海女直也将尽数归于完颜部。到那时,东京道上生女直,还有多少不属于完颜部?”

生女真诸部分布在辽阳之北的广袤大地上,是东京道北部的主人,也是辽国天子每年春日都要来到鸭子河畔的主因。

“你就是担心太多了。”大辽皇帝很不耐烦,端起金杯,将里面的热酒一饮而尽,“劾里钵的年纪也老了,而他的儿子不少,有什么好担心的?”

推恩令的作用是有过太多验证的。耶律乙辛自觉有必要的时候,完全能够将完颜部四分五裂。

让完颜劾里钵的弟弟、儿子均分其部众,其中最勇武也最得耶律乙辛喜爱的阿骨打可以多得一份。

耶律乙辛重重的放下杯子:“难道朕要劾里钵将部众均分给盈哥和阿骨打他们,劾里钵难道还能说不干?之前朕已经分了他的身家,也没见他起来做反。”

在他的御帐三百步之内,随时都有一支八百人的女直宫卫在守护,从东京道各部女直招募来的部族勇士充斥其中,其中不乏各部贵人的子弟,耶律乙辛只要拿出一块闲地来,就能帮他们从自家的部族那边分出几百家。到时候,哪个不对他感恩戴德?

早前为了削减完颜部的实力,还有一部分部众被迁往了黑山,与宋人遥遥相对。

尽管这是在削弱完颜部的实力,但耶律信乙辛做得光明正大,分给完颜部的都是好地。以渔猎为生的完颜部,其实并不适应游牧的生活,可耶律乙辛给予他们的赏赐之珍贵,也是包括完颜劾里钵在内,任何一名女真人都无法否认的。

远隔几千里,音信难通。白山黑水下的完颜部,与黄河畔的完颜部,实际上已经分立为两个不同的部族。时日一长,谁还认识谁?

“如果劾里钵起兵,待其势大,劾者难道会不起兵呼应?”张孝杰反问。

完颜劾者是前任完颜部之长完颜乌古乃长子,完颜劾里钵则是次子。但乌古乃觉得长子性格柔顺,不宜为部族之长,故而将族长之位交给了次子劾里钵,让完颜劾者守着家门,甚至都没让他分家出去。

等到耶律乙辛攻夺的西夏故地为宋人所占,为了固守仅存的黑山之地,便迁移了大批女直人过去,最后还从完颜部中分割了上千帐,交给了完颜劾者,让他带去了黄河之滨。

尽管相距甚远,可兄弟就是兄弟。若当真女直有变,张孝杰可不觉得完颜劾者会袖手旁观,或是站在朝廷的一边。

“莫说劾里钵不会反叛,即使他反叛了又如何?劾里钵、劾者、盈哥,他们有谁能胜过朕的神机军?”

张孝杰说的这些话,耶律乙辛早听得厌了,完颜部越来越强盛他当然知道,可是作为大辽的皇帝,他有必要担心连铁器都不能自产的女直人吗?

在过去,契丹人就像宋人畏惧他们一样畏惧野蛮的女直人。契丹人对宋人来说是强盗,但女直人对于契丹人来说,也同样是强盗。

可如今,大辽的钢铁产量能将契丹人的战马都以铁甲覆盖,精铜铸就的火炮,就是铁石所砌的城墙也能砸成碎片。再精悍凶蛮的女直人,遇到人马贯甲的具装甲骑,面对黑洞洞的炮口,可有半分活路?

耶律乙辛最看重的神机军,是以契丹和奚族为主。火枪、火炮、战马、甲胄,从上到下,有着国中最为精良的装备,又在成军的数年中,分批出征,作战经验是南朝对应的队伍所不能及。拥有着一支多达五千人的精锐,耶律乙辛有信心剿灭任何部族的叛乱。

“事有万一。神机军固然勇不可当,但其过于依靠辎重,万一后路被断,弹药不济,可就危险了。”

耶律乙辛的脸当即挂了下来,神机军是他的心头肉。可被张孝杰一说,却成了纸糊的老虎,仿佛一戳就能破。

“张卿,朕知你不喜女直,尤其不喜完颜。但你总要想想,完颜部才多少人口。女直才多少人口,北疆那么大的一片地,那点点人口撒下去比饼上的芝麻还少,臣服于完颜部的部族虽多,可完颜部想要管起来,也没那么容易。”

完颜部的人口再多,本族的户口也没有超过一万户。以大辽北疆的苦寒荒凉,百里方圆的土地,也就能养活万把人,两三百里之外,甚至连控制都难。而那些附庸,都是有利则来无利则去,耶律乙辛为何要担心他们?

“汉人说得好,万般皆重,惟户口最重。只要女直人的户口赶不上国族,永远都别想有机会叛乱成功!”

自登基之后,最为耶律乙辛重视的政策,不是炼铁炼钢、大造火器,不是开疆拓土、攻伐小国,而是推进医学、鼓励生育。

备受耶律乙辛看重的新的医疗体系,彻底排除了旧时巫婆神汉的干扰,在正确的道路上越走越远。

大批的医学生在不断的实践中飞速成长。

医术再拙劣,也比部落的巫医要强;医患关系再紧张,也还有种痘法兜底。这就为医学生们准备了大批即使医死也不用担心的病人。

没有如汉人一般太过坚固的不可毁损先人尸骸的先天桎梏,又不受儒学门徒那种虚伪的仁义的束缚。医学生们也就有了大量可供解剖的尸体和活体。

他们的医术又有什么理由不飞速进步?

两所医学院,十五所医院,十七家巡回医馆,两百位医师,近三千名医学生,不仅为耶律乙辛带来了大辽疆域上数以千百计的各部异族的忠心,也带来了飞速增长的辽国人口。

其中自然是以医疗水平最高的契丹和奚族增长最快。女真部的人口膨胀虽快,但契丹、库莫奚这两个支撑起辽国的两大族,人口增加的速度则更快。

耶律乙辛没有办法对户口进行精确的计算,但通过保赤局反馈的数字,每年国中的新生儿数量,都是以百分之十以上的数字在飞速增长。

耶律乙辛的儿子有八人,孙子都超过五十了,重孙也有三个,如果他的帝位能维持下去,他的后裔将会成千上万。

如何胜人一筹?要靠人多势众啊!

张孝杰低头,不再反驳。他清楚,心有定见的耶律乙辛不是这么一次就能说服的。但只要在他的心里扎下一根刺,然后时不时的摇一摇晃一晃,迟早会溃烂,最后烂个干净。

见张孝杰被他说服,耶律乙辛颇为自得。用皇帝权势压人,哪里有用才智压服臣子来的让人欣喜?

但他旋又叹起:“大辽对南人所不及的,其实还是人口。若是有胜兵百万,又何愁不能饮马长江?”

火器出现之后,个人勇武上的差距被大大缩小,而优势人口的作用,则越发的明显起来,要不是南朝内部不靖,耶律乙辛早就寝食难安了。

张孝杰沉声道:“可辽宋之战,或许就在十年之内。”

“或许……但南朝之患不在我而在彼,等他国中君臣内讧,都不一定能出兵。即便议定出兵,等大军出征,不是黄袍加身,就是回军剿灭权臣。有的好戏可看!”

耶律乙辛哈哈笑起,同样的事,发生在自己身上,那叫一个惊心动魄,发生在别人身上,那可就是喜闻乐见了。

张孝杰没有跟着笑。

南朝不靖,难道北朝内部就安靖了吗?

如果纯以治政的才能来评价,他的这位皇帝绝对是大辽开国以来排行第一的明君。即便是以武功来评定,攻取了高丽和日本的成就,也是太祖之后其他皇帝所不能比。

更别说耶律乙辛这些年给国人们带来的多少好处,即使是最底层百姓家的儿女,也能享受到种痘和读书上的好处。而贵人们,房中多了皮肤白皙、脸型圆润的高丽女,手下则多了听话、忠心的倭奴,还有来自宋国的绫罗绸缎、玻璃器皿,

要不是谋国篡位四个金色大字,明晃晃的在耶律乙辛的头顶上亮着,大辽的万里封疆之中,又有谁人敢于有不顺之心?

不是说耶律乙辛他坐不稳皇位,而是说他应该坐得更稳,理应是作为一代圣君受到万人敬仰,但如今在国中,提起当今的皇帝,却在比较过过去的几位先帝之后,不论怎么称赞如今的皇帝,最后都少不了摇摇头,叹息一声。

有些事,做过之后就是要背负一生的罪名。

笑声在耳,但张孝杰依然觉得,想要高枕无忧,想要幸灾乐祸,现在还远远不是时候。
