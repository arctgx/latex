\section{第20章 千山红遍好凭栏(中)}

从蒸汽机跳到了粮食,韩冈的话题跳得厉害,但内容却让李诫悚然而惊。

县令嘴里的一团乱麻,最多也是只是一县动荡,但宰相嘴里的一团乱麻,可就事关天下了。

他斜了一眼宗泽,中书兵礼房的检正公事容色不动,显然是早已知道内情。

李诫整理了一下思路,试探的问道,“相公,可是今年的收成……”

李诫反应极快,韩冈赞赏的点头:“江淮之地,这几年弃粮改棉的农户愈见增多,预计今年至少了百分之五。”

百分之五,乍听起来不是个值得宰相动容的比例,寻常官员听了,只会觉得韩冈是小题大做,可李诫精于数算,更了解国计,知道这个比例意味着多大的一个数字。

“会不会算错?据李诫所知,江左的粮价近年没多少变化。”李诫怀疑的问道。

“这是相公命人从棉布产量上推算出来的。”宗泽道。

李诫顿时无言。

韩冈家里种了几万亩棉田,天下棉布大兴也是韩冈开的头,他从这个角度来计算,绝对比看朝廷的账簿要准。

“江左纲粮事关天下,应当诏禁才……”李诫一句没说完,便停了。

根本禁不了的,想想就明白,这么赚钱的买卖,朝廷怎么禁?若是朝廷当真下诏,更会让西北棉商的后台韩冈成为众矢之的。

其实这件事也简单,只要江南的粮价涨上来,种粮比种棉赚钱,自然会有人弃粮改棉。

但他可不敢开这这个口,李诫更清楚,江南粮价上涨到底意味着什么。

宗泽道:“棉与粮食争地,而丝绢不占良田,江南棉田日多,朝中其实也多有议论。但棉布、棉絮保暖耐用,不是丝绵可比。”

李诫皱眉道,“若是两广出产能够再多一点,江淮的棉田的亏空也就能补上了。近年江淮粮价稳定,也”

“明仲这话说得好!”宗泽道:“江南的亏空,只能靠广东广西,还有荆湖两路。两湖、两广地多人少,虽多疾疫,但水土肥美,若将之开垦出来,。现在是苏湖熟,天下足。什么时候变成湖广熟,天下足,就不用担心江南农户尽种棉花了。旧日两湖、两广的疾疫,多是天花、伤寒和腹蛊,如今有了相公的牛痘,天花不用怕了;若遵循相公的厚生制度,伤寒也难以传播;加上如何杀灭血吸虫,更不用担心腹蛊。假以时日,两湖、两广的出产,绝不会在江南之下。”

“禹贡中的扬州,土惟涂泥,田唯下下。千载之后,却变成了上上之地,非先人胼手砥足,岂得如此?”韩冈叹着,“只是千年时间,让人等不及啊。自章子厚平荆南,荆湖移民也推行了十几年,两路的出产要补足江南的缺口,也不是那么容易。”

江南现在改种棉花的田地大约只有百分之五,但随着棉布的普及,改种棉花的农户只会越来越多。人性趋利,即使韩冈贵为宰相,想要扭转这个趋势,也不过是螳臂当车。

“记得相公曾经说过,”宗泽回忆道,“新疆增产,无外乎移民,良种,还有改进耕作方法三条。”

“还记得啊。”韩冈笑了起来,这东拉西扯的几条,他自己都快忘了:“当年去广西的时候,邕州的田地,即使就在江边上,也几乎都是旱田,,当地农户也不修水渠,甚至连施肥都不会,漫种漫收,亩产不及江南的三分之一。”

“相公广西一任,平灭交趾只是小功,使岭南为乐土才是无人可及的大功。”

韩冈笑着摇头,难得见宗泽拍马屁。

“不过这几条知易行难。”韩冈道:“当年熙宗皇帝问家岳,变法难在何处?家岳的回答是乏人。君子六艺,射、御皆为武事。三代士人出将入相,文武皆能,如今的士人,却视武事为粗鄙下贱之举,也就近两年,国势大振,方才改了那么一点。武事如此,就不要说农工之事了。”

随着韩冈就任宰相,投靠韩冈的官宦、士人一日多过一日,但合用的人才依然少得可怜。会做官的太多,会做文章的同样的多,但会做事的就太少了。

韩冈很早就打算设立农学,可惜相应的人才难得。能全篇背下《齐民要术》的士人车载斗量,可是能够写下《齐民要术》这个水准的士人却一个都难找,总不能找老农来教书。

相对的,不需要教书育人,只要在农业上下功夫,那就简单多了。韩冈家里就有专人来进行农业研究,改良棉花、小麦等作物的种子,改进农具,改进耕作技术。同时改造田地,韩家的庄子十几个,三万多亩地,能照应得过来,一个靠轮种,三年一休耕,一个便是靠不断改进的耕作技术。

韩冈叹息着:“同为搜粟都尉,知桑弘羊者多矣,可又有几人知赵过之名?”

李诫要多想一想,才记起赵过此人的来历,点头道:“以殊勋而无缘青史,诚可悲也。”

《齐民要术》中说‘神农、仓颉,圣人者也,其于事也,有所不能矣。故赵过始为牛耕,实胜耒耜之利’,而贾公彦在《周礼注疏》亦说:‘周时未有牛耦耕,至汉时搜粟都尉赵过,始教民牛耕。’

耕田的手段,由双手挥动的锄头,变成牛拉的耕犁、耒耜,史书中所记录的功臣是汉武帝时的搜粟都尉赵过,不管这个记录是否是事实,赵过的名声不显是确凿无疑的。

李诫是世家子弟,从小得到最好的教育,见识广博,手边的书也是汗牛充栋,若不是性格与科举不合,也就去考进士了。他能看到的书籍,他能学习的知识,都不是寻常士人能够相提并论。他都要多想一想,才能想得起来的人物,寻常士子有几个能记在心上?

即使是《周礼注疏》中提到了赵过这个名字,可《三经新义》早已成为经义圭臬,同为周礼注释,世人当然更愿意去诵读进士科中必考的《周礼义》,而不是被替换掉的《周礼注疏》。

但李诫的感慨与韩冈的叹息并不在一条线上。

李诫感慨的仅仅是赵过这个人而已。

但在韩冈看来,如赵过这等功绩无可计量,在史书中,连一篇列传都吝啬不与,那些让生产力不断进步的人民,更是卑微得在史书中不得一见。这才是韩冈叹息的地方。

朝中绝大多数的官员,纵使其中有人才高八斗,也有人颇得清名,更有人累世簪缨,但在韩冈眼中,他们依然远不如自己父亲对这个国家的功劳,西疆的稳定,是建立在驻军军粮能够自给自足上的,做到这一点,老农韩千六的作用,比多少知州知县都要大。

生产力的发展,才是最核心的问题。

男耕女织的田园生活,是无法与工业化的大生产相抗衡的。

如果是用手摇纺机,一次只能处理一个纱锭,而现在最好的水力纺机,已经能够做到将近一百个纱锭。

但水力的局限性太大,蒸汽机的作用无可替代。缺乏足够的工艺水准,同时自身也没有太多记忆,韩冈并不指望去造内燃机和电动机。现阶段的工业化的动力源,除了蒸汽机,韩冈想不出还有别的机器可以代替。

尽管韩冈很早就在《自然》上公布了蒸汽机的原理,《九域游记》中更是将原理和作用都说了个透,但能够实际投入使用的蒸汽机还是遥不可及。

不知道当年瓦特是怎么发明——好象是改造——具体细节韩冈已经记不太清楚了,但他相信,他给出的蒸汽机的原理和结构,应该是走在正确的道路上的。有飞轮、有连杆、有锅炉,当然还有装着来回移动的活塞的双向气缸。

只要工艺技术达到标准,数以千计的工匠、士人付出努力,蒸汽机就能出现在世人的面前。当然,即是蒸汽机现在就发明,距离蒸汽船和蒸汽机车的出现,还有颇长的一段距离。

即使是再有二三十年的发展,蒸汽机想要拿来驱动船只和车辆,说不定还是达不到要求,但困于水力不足的纺织机械,却肯定可以摆脱河流的拘束。

可是要实现这一切,第一个是工艺上的问题,第二是材料上的问题,第三条最为重要,就是人的问题。

即使是气学内部,对设计和制造也依然有着偏见,所以韩冈之前才大发感叹。

宗泽比李诫看得更清楚,“相公一片苦心,世上又有几人能看见。”

韩冈利用小说话本来宣扬,又以利诱之,他做的一切,都不是寻常宰相会去做的。

“差得远了,朝廷为蒸汽机给出的悬赏不过一个小使臣,而辽人那边的悬赏,则是高官厚禄,全都齐了。”

韩冈贵为宰相,但受到的牵制依然很多,不可能凭着自己的心意,拿出朝官等级的文武官职来悬赏。而辽人那边,倒屐相迎的活剧,据说耶律乙辛已经演过好几次了。

“但中国技艺,岂是北虏能比?南京道的工匠,也远远比不上军器监和将作监的大工。”

“呵,的确如此。”韩冈淡然笑道。牢骚归牢骚,其实他并不是很在意耶律乙辛的举动。

辽国挣扎的越厉害,局面只会越好。

王安石变法中的最重要的一步,就是要一道德,统一思想和意识形态。

但王安石没有做到,包括韩冈在内,一群人在跟他大唱反调,而韩冈也不认为自己一个人能够做得到。

不过生产力的发展没有人能够阻挡,当工业化进程的大车开始启动,那些绊脚石也只会在车轮下被碾进泥地里。

韩冈要做的,只是保证起步阶段的安全罢了,之后,那就是一个自然的历史进程了。

“明仲。”韩冈问李诫,“知道我想要借重你何事了吧?”

他满意的看见李诫点头称是,心中确信,这将是自然历史进程最新的一步。
