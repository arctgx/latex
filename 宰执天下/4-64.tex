\section{第14章 飞度关山望云箔(三)}

‘拿下昆仑关?!’李信摇头。他虽然没走过昆仑关,不知险峻如何。但昆仑关如此大的名气,也绝不会是八百人一攻就破的关隘。这不是说句话就能结局,今天要不是广元蛮贼自己犯蠢,赢是能赢,但伤亡绝对不会这么小,“我们只有八百疲兵。”

韩冈要攻昆仑关救邕州,苏子元惊喜的几至感激涕零。但他冷静下来,就知道以手上的兵力根本不可能:“是否是抄小道至昆仑关背后?”他只想到这一个可能。

拦在邕州和宾州之间的山区,只是一片连绵起伏的矮山而已,能绕过昆仑关的小道不少。但山中草木丰茂,蛇虫为数众多,雨后往往有所谓的瘴疠之气,其道路一向难行。不过眼下是少雨的冬日,比起其他季节,算是好了不少。

“狄武襄攻昆仑关,就是以奇兵自小道绕过关城,前后夹击。这一次李常杰来攻,听说也是以遣奇兵走的小道。”韩冈道,“要绕过昆仑关去还是很容易的。

“但贼军难道会不防备?只要派人监视道路,想偷袭根本不可能。而且无论狄青还是李常杰都是奇正相合,没有说只用奇兵。”苏子元不是要驳斥韩冈,他更希望韩冈能驳回他的疑问,“毕竟我们只有八百人,哪里能分得出奇兵、正兵两路来?”

“我几曾说要绕过昆仑关。邕州危在旦夕,我们没有那个时间。而且手上兵微将寡,走小道往邕州绕过去,这是自蹈死路。”

韩冈否定了之前的猜测,李信和苏子元都糊涂起来,“那要如何攻下昆仑关?”

“靠朝廷!”韩冈正欲深入解释,却见到前面来了一队人,领头的穿着官袍,“宾州知州来了。”便不再继续说下去。

赵明骥在宾州知州任上只做了五个月,并不是正式的知州位置,而是以桂州教授的身份暂摄宾州州事。这在两广很常见,不足为奇,琼崖岛上除了琼州以外的三个军,甚至都有过吏员权摄州事的例子。不过落到个人头上,仍可算是一桩美差。但交趾入侵,尤其是昆仑关失守后,赵明骥就恨不得将这个烫手的位置丢出去,早早跑回桂州。

尽管赵明骥穿着一身官袍,但在韩冈等人眼中,他不像是官员,就是个穷人乍富的村学究的气象。赵明骥带着城中的官吏走过来,一路上小心翼翼的不去看堆在一边的首级,艰难跋涉才到了韩冈的面前。

“下官拜见运使。明骥见过苏军判、李都监。”

赵明骥在韩冈三人面前将自己的位置摆得很低,韩冈和苏子元就不用说了,都是正经的京朝官,而李信这一等级的武将,也不是他敢得罪的,尤其亲眼见识过李信的武功之后。

少说也有千人以上的贼军,而且还用了计策,在背后藏两百伏兵。但这些贼人,荆南军竟然不费吹灰之力就给解决了,一场战斗轻轻松松,就如同切菜砍瓜一般简单。被他们驱赶的宾州百姓群起而攻,外面又有二十多骑兵,而偌大的一片原野,连着遮蔽的地方都没有。到最后,来犯的蛮贼一个都没能逃出去。尤其是那些中了掷矛的伏兵,有许多甚至被都插在身上的掷矛牢牢的撑住,尸身斜倚着,就是不倒地。让赵明骥看得心中发毛。

“下官身处道中要地,望着北方日盼夜盼,早早就盼着荆南来救援。今日终于让下官等到了……”赵明骥歌功颂德的说着废话。

苏子元听得脚板磨着地,很不耐烦,他还急着想要知道韩冈究竟要怎么夺回昆仑关。而李信尽管仍是默不做声,但他也是不耐烦的望着正在打扫战场的麾下将士。

战场之上,到处都是血淋淋的痕迹。参战的士兵这时正在用刀斧将他们的功劳从尸身上一个个的斩下来,韩冈的几个亲兵在一边做着记录。蛮贼不论轻伤重伤,一律一斧头解决。一千多斩首,光是堆起来就是一座小丘。

被拯救下来的百姓,则坐在尸堆上抱头痛哭,他们之中,有许多都是跟贼人同归于尽。官军能有这么大的战果,也是靠了他们的奋力反抗。战斗结束后的第一件事,韩冈就是命人赶紧将受伤百姓抬到干净的地方包扎急救。

韩冈和李信手下的亲兵几乎都派出去了,可以说是韩冈的影响,如今西军将领们的亲兵,基本上都是受过全套的战场急救训练,这是无法普及医护制度下的权宜之举,因为能让将领收服军心而流传开来。

一枚枚首级被交过来点验,脸上尽是刺青的蛮贼头颅,就算死后,依然狰狞得如同鬼怪。李信念了一声佛:“天道循环,报应不爽。”

“南方虔信浮屠的极多,李佛玛和李日尊也是一座寺庙接着一座寺庙的建。”苏子元冷笑着,“一点慈悲心也无,光想着建寺庙、塑金身就能成佛,哪有这般容易。”

看见苏子元和李信分心说着他事,赵明骥也没有少说哪怕一句奉承话,他是真心实意的感激韩冈和李信。他见识过打得敌军全军覆没的战绩——就在二十多天前。接下来的这些日子,他连着多少个晚上都是夜不能寐,生怕一觉醒来,城外就是一片交趾的旗帜,好不容易才盼到了今天官军的大胜,

“运使、都监、军判。”韩冈派去计点伤亡的亲兵回来了,“军中伤亡已经计点出来,四人战死,二十七人受伤。”

韩冈点点头,整场战斗他都看在眼里,差不多也就这个数字。受伤的短时间内不能重新参加战斗,不过八百多人的队伍,这下又少了三十名战力。

赵明骥却是在惊叫着,“以千人之军攻千人之军,敌尽授首,而官军只亡四人。此乃当世奇功。韩运使指挥若定,李都监武勇盖世。”

“是蛮贼弃其所长,用其所短,乃是作法自毙。如果对阵厮杀,伤亡差距不至于如此悬殊。要不是他们押着百姓随行,也总能逃出一批,也不会全被绊在战场上,一个都没逃掉。”

“运使文武双全,名传当世,区区南交蛮夷,哪里能及得万一。”

“也是多亏赵知州力保宾州不失,若是让蛮贼得了宾州城,我等倍道而来,必定会顿兵城下,被打个措手不及。”

听到韩冈的话,赵明骥一张圆脸顿时红得发亮,韩冈简简单单的一句话,就坐实了他的守城之功。他的权摄州事,就可以将‘摄’字改成‘知’了——权知宾州。

他鞠躬哈腰:“下官已经在州中备下屋舍和酒食,还请运使、军判、都监,带着忠勇将士入城歇息。”

“也好。”韩冈回头对苏子元和李信说道。“还是先进城休息,再说其他事。”战场上的士兵,在路上走了一整天,紧接着又是一场大战,现在虽然因为兴奋于胜利而忘记了疲累,但很快就会撑不住的。

如今已经确认的事实只有一个——邕州城破只在旦夕之间。为了援救邕州城,光是救援宾州根本不够,至少要拿下昆仑关。这样才能逼得交趾贼军不敢再围攻邕州

如果邕州城已破,韩冈绝不会冒进。可眼下偏偏是这种暧昧不明的情况,就算只是派人去打探消息,都是耽搁时间。“进城后,就商议一下如何拿下昆仑关……韩廉,你将那个何学究带回来,他应该吃过苦头了。”

“对了。”韩冈又吩咐着赵明骥,“赵知州,这一战的战果要及早传回桂州,以安广西民心。”

赵明骥听了忙不迭的说道:“下官这就去调派马递。”

宾州知州走远了一点,点起负责传递消息的属吏,让他立刻让递铺中的人做好准备,等人头点算完毕,就立刻带着战报出发。

大约一百多在战场上受了上的百姓,正在官道边的亭子里面急救。韩冈带着赵明骥过去探视,看到几个官人过来,被解救的百姓忙着跪下来冲着韩冈磕头,却没有一个感激赵明骥的。群众的眼睛是雪亮的,谁帮了他们,他们就感激谁。

韩冈看了一下救治的情况,已经有四五十人被包扎好了伤口,但另外还有二十多人躺在亭子外,旁边有亲友在哭着,都是来不及挽救的伤员,也不知能有多少可以救回来。不过他还是跟赵明骥在城里要了一间干净的营房,要将随军医院先建立起来,这些救下来的伤员,先行搬去城中养病。

将城外要处理的事一一分派完毕。战果也清点出来了。阵斩广源州大首领刘纪之弟刘永以下、大小蛮将十九人,斩首一千一百二十四,只是俘虏少了点,只有一个——只要不是汉人,全都给杀了。另外还有百姓们被抢去的财物,韩冈让赵明骥负责清点归还。

确认了战果,韩冈等三人骑上马,带着满载着战利品的大军,与赵明骥一起往宾州城中去。宾州城上城下,尽是震耳欲聋的欢呼声。宾州城民为着大胜而归的王师而欢呼。

穿过城门,苏子元再也等不下去,“运使,到底准备如何攻下昆仑关?”

“前面没听到吗?镇守昆仑关的贼将是广源州洞主的黄金满。”

