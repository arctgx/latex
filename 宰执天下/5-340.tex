\section{第32章 金城可在汉图中(19)}

结束了城头上的宣讲,在高昂的士气中,韩冈走上了北城的敌楼,幕僚们纷纷跟在身后。

在楼中望着北方,站在高处,可以看得更见全面。也看到了比之前更多的辽军,无边无岸。

在狭窄的山谷中与宋军决战,吃亏只会是以骑兵为主的辽军。他们绝不可能直接去攻打任何一处通道,只会在盆地之中扫荡。要不是太原军在石岭关表现的太差,辽军甚至可能不敢南下。

如果韩冈坐镇在南面的山谷中,辽人绝不会冲动。可现在韩冈在平坦的盆地上做了鱼饵,却不愁萧十三他们不上当。

韩冈低声问着背后的黄裳,“出去的游骑都撤回来了?!”

黄裳摇摇头,“有两队还没有回来。”他看了看韩冈,“不过也不用担心。枢相事先不是都叮嘱过了?如果来不及撤回城中,就先向南方退过去。”

韩冈微微的摇了摇头,不再多提。

哪一队被安排在哪一个方向,事先都是确定好的,现在黄裳避而不提究竟是哪一个方向上的游骑没有回来,多半是最北面的几队,可能已经陷落在敌阵中了。

太谷县外围的宋军骑兵其实只是作为耳目才派出去的,最大的作用也只是拦截辽军的探马。面对汹涌如潮水的辽师主力,不要说正面迎战,就是迟滞阻截,都缺乏足够的本钱,翻不起半点浪花。

招了太谷知县上来,韩冈吩咐他道:“城中最要紧的是稳定,撤入城中的百姓一定要照顾好。”

太谷知县一口应承,“此乃下官分内事,还请枢相放心。”

韩冈点点头,也不多说废话。真有了乱子,凭他的声望也能压得下来,最多借个人头就可以完事的。韩冈对此很有自信。

现在太谷县十里之内,所有的村庄都被放弃了,大批的百姓撤入城中,可不只是寺院中的僧尼。除了太谷县这个避难所,还有更多向更南方撤离的百姓,都让人担心。

但除了最为靠近太谷县的几十条村落,更外围的村寨都还在固囘守中。除了北面的已经受到了攻击以外,在今天之前,东西两个方向上的村庄都还算安全。不过到了明天,那就不一定了。

眺望了一阵漫山遍野的辽军骑兵,韩冈又回头,带在自信的笑容:“该做的布置已经都做了,这一回辽贼肯定会分散出去,没粮没水,他们在城下待不住!”

黄裳立刻道:“这么一来,多半就难攻城了。要是在城下抢不到粮食,还能去远一点的地方抢了再回来。但水就没办法了,还能跑出十几里去喝水?”

“就怕他劫掠一番就会北返。”

“太原太谷相距百五十里,两三日间往返三百里,人吃得住。战马吃得住?”

“辽人不可能想不到吧?”

“当然预料得到。也肯定会有准备。不过不论有多少准备,也变不出食水来!”

韩冈屈指敲了敲窗台,下属幕僚们的议论停了。

看看左右,韩冈的语气平静中隐藏着一份激昂,“无论如何,这一次,得让北虏有来无回!!”

……………………

冲在最前的乌鲁拉住了缰绳,浩浩荡荡的千人队就在太谷水边停了下来。

骑在被汗水打湿了毛皮的爱马北上,乌鲁眺望着北方。稍远一点就是这一回的目标太谷县城,看着远远比不上太原城的雄伟,从平坦的地面上升起的墙体,似乎只比这几日攻打过的几个寨子稍高一点。

看看天色,再耽搁些时间,可就来不及扎营了。而且自从进入南朝境内,基本上就没怎么歇过,打草谷也好,打仗也好,都是纵马奔波。虽然乌鲁带了三匹马南下,可战马的体力消耗不少,需要稍事休整。

喝水休息,接下来就是一鼓作气,将这座县城给攻下来。乌鲁望着城墙,心里想着。接着又遗憾起自己不是前锋,排在后面的结果,就是先行出动攻城的绝不是自己。

不过也不错。乌鲁宽慰着自己。万一前面的几部没能攻下来,轮到自己时,说不定正好能碰上被消耗了太多的守军支持不住的情况。那时候可就是要发财了。

虽然他还有些可惜没能攻进富庶的太原城,但一个县城之中,所拥有的财富也绝对能让数以千计的来自于北方的契丹儿郎感到满意。

出没在西京道最北面的草原之上,他们这一个来自国舅横帐的千人队,让草原各族闻风丧胆,乌鲁正是其中数一数二的勇士。

翻身下马,乌鲁就解下了坐骑的鞍鞯,准备牵着马先下河去饮马,自己也顺便喝点水。

“乌鲁!别喝,这水不干净!”一声大叫停止了乌鲁的动作。

他回头看了看,是个五十六十的老家伙,同族的老人,比乌鲁高一辈,不过地位不算高。正火烧火燎一般的冲自己叫着。

乌鲁哼了一声:“老胡里改,你乱说个什么?”

“这水不干净。”老胡里改已经抢了上来,指着河水:“你自己看看清楚。”

乌鲁早看清楚了,河水的确并不清澈,甚至还带着若有若无的臭味。

“不就是有点脏吗?哪条河水会干净得一点污糟都没有?”乌鲁对此依然毫不在乎,饮马的时候,上游人撒尿马拉囘屎,下游不照样洗澡喝水?说着他就脱了靴子,牵马跨进水里。手上还提着个羊皮水囊,准备到河心弄些干净点的水。

但老胡里改却一把扯住了乌鲁,一巴掌把他的脸给抽了起来。丘壑纵横的老脸已是七窍生烟:“乌鲁,你要把婆娘孩儿还有打草谷来的这么多好货都留给你那个弟弟,老头子我不拦着你!但你找死前,你先瞪大眼睛看看城上的那是谁的旗号?死了也好做个明白鬼!”

乌鲁半边脸都肿起来了,一向凶悍的他不知有多少年没吃过这样的亏了,就是同族的长辈平常面对他这个有名的勇士说话时,也得和和气气。

但乌鲁来不及生气,就算是个浑人,他也感觉到了老胡里改话中的关切。他依言顺着胡里改手指的方向往城头上上望过去。一面绣了汉字的大旗用飞船悬起,高高的飘在半空中。

乌鲁当然不识字,但那面大旗实在是大,看起来比下面载人的篮子都大一圈,拖下来能做顶帐篷,绝对不是普通的宋将。

“那是谁的将旗?”乌鲁回头问道。

“还能是谁?韩菩萨啊!”胡里改声色俱厉冲着乌鲁的耳朵大叫。

乌鲁眨了眨眼睛,然后才反应过来,黝囘黑的脸竟然变得白了:“韩菩萨?治了痘疮的那个韩菩萨?!”

“菩萨奴三五千都有,韩菩萨还有第二个吗?!”胡里改气急败坏的踹了乌鲁一脚,扯着他的胳膊指着周围:“你睁大眼睛看看,有几个人敢喝水的?!”

乌鲁左右一转,这才发现在蜿蜒绵长的河滩上,的确就零零星星的十来人牵马下了河堤。而且是有人刚下水,就立刻跟自己一样被叫住了。

干咽了一口唾沫,乌鲁战战兢兢的发问,“真的是韩菩萨?”

“你这就只知道抢抢抢,杀杀杀,就不知道睁大眼睛、竖起耳朵。”老胡里改恨铁不成钢的戳着乌鲁的脑门,瞪着昏黄的老眼,“现在有几个不知道坐在太谷县中的是韩菩萨?!”

“真的是韩菩萨~~?”乌鲁心慌得厉害,那可是将痘疮给根除的神佛一般的人物。天下病症无数,医生也是无数,可只听说过给人囘治病,可有谁听说过把病给剿灭掉的?!

“乌鲁你难道都没感觉到吗?进了太原府之后,打起草谷可比代州要难多了。”老胡里改语重心长,“你前两天打草谷回来,不也在说南人自己烧了房子、烧了粮囤,害你白跑了一趟?这都是韩菩萨做的。他签名画押的公文散得到处都是,上面全都是怎么教南人怎么为难我们的。”

“真的是韩菩萨……”乌鲁现在的脚开始发软了,声音也在发抖。看看脚下,潺囘潺的河水正绕着脚踝,他脸色陡然一变,仿佛被炭火烫了一般,往后一蹦两蹦,直接蹦到岸边上。

老胡里改忙跟着一起上岸来,直接丢了湿透的靴子,扯了块布擦着脚上的水,边擦边叹:“那个韩菩萨能救人,但也能杀人啊。别的倒也罢了,这太谷县周围的食水,想要要命就别碰一星半点!”

乌鲁用力点着头,打了个唿哨把被丢在河中的爱马叫上来,从岸上的包袱里掏出半幅棉布,也跟着一起擦起脚来。

“老叔。”乌鲁郑重其事的向老胡里改道谢:“这一回多亏了你就了俺的命啊。真要喝水染了病,能回家的可就只剩灰了。”

乌鲁说着,就浑身不自觉的直打哆嗦。想起了过去族里怎么处理疫症病人。有人染了能传人的疫症,可就是连人带帐一把火给烧个干净,而且是不论死活,染了就烧。

老胡里改摇摇头,“不是救你啊!要是水里只是有毒倒也罢了,我怕水里面是有病啊。万一你得了疫症,保不准就能连累了阖族老小!连我这条老命也被送进去。”

乌鲁的声音低了,扯住老胡里改,指着中军的方向含含糊糊的问:“……怎么就敢来攻韩菩萨坐镇的城池的?”

“转世投胎后不还是凡人吗?更别说皇帝也是神佛转世,谁也不怕谁!但我们上辈子还不知是什么马啊狗的,却怎么不怕?!”长叹了一声,胡里改又凑到点头如啄米的乌鲁耳边,低声道:“不过真要攻城,还是别手软,命可是自己的……只要记着箭别往韩菩萨身上射就是了。”

“真的还要打?”乌鲁惊问。

“谁知道呢?”胡里改鼻中哼着,“听命行囘事吧。”
