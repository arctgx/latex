\section{第18章 青云为履难知足(15)}

‘果然还是在这里出问题了。’听到赵顼的问题,韩冈想着。

只要对禁军稍有了解,都应该知道,大宋禁军之中,唯有西军常年历战,作战经验丰富。每逢大战,最适合上阵的当然惟有西军。眼下要讨伐交趾,不论是从实力上,还是战绩上,甚至在算进当初狄青率西军南下平叛的旧事,都必须动用西军。

这时候还要问自己究竟是打算用哪一路兵马,明显的就是朝堂上还在犹豫着要不要调动西军,甚至连赵顼本身都在犹豫中。

吴充的嘴角抽动了一下,似笑非笑。北方的重要性远在南方之上,天子和朝堂都不会答应削减关西的防守能力,再从西军中调兵南下。他倒要看看韩冈会如何开口,说服天子,压制宰辅中的反对意见。

“丰州落入贼手几近一载,州中生民罹难,渴望官军如久旱盼甘霖,不可再拖延须臾。臣亦闻郭逵在太原已经准备妥当,自当速速发兵收复丰州。”

韩冈的话一出口,连赵顼都有些发楞,韩冈怎么帮起河东说话了?甚至连发言都跟之前宰执们争执时,吴充、王珪等人的发言没有多少区别。

“于此同时,为防西贼以丰州为饵,趁机攻打横山或是其余要地,缘边诸路都必须做好防备,甚至在必要时抽调一部兵马,攻入西夏境内,作为牵制。所以关西必须保持眼下的数量。另外熙河路为了平定茂州之叛,已调兵南下,现如今也不能再减少熙河路的兵力。”

韩冈一边说着,一边用眼角余光审视着在场的宰辅们的表情,果然都有一丝难以掩饰的惊讶。他们似乎忘了一件事,在关西战事上的发言权,自己也绝不输于任何一名臣僚,哪有不利用的道理。

不打无备之仗,这是他一直以来遵行不悖的信条。

“难道韩卿不打算使用西军?”赵顼终于忍不住心中的惊讶,照韩冈这么说来,关西的兵根本就不能动。

“交趾远在南荒,粮饷转运不易,是故用兵不可多。南方有多瘴疠,待雨季一至,则疾疫大起,故而用兵不可久。兵即不可多,又不可久,便必得拈选精锐,以期速胜。”韩冈的态度一直很明确,没有关西军中的精锐,他绝不会参加平定交趾的战争,“他处臣不甚了了,惟西军久历战阵,良将辈出。陛下如从关西调遣强兵良将,交趾当能一战而定。”

赵顼现在是一头雾水,韩冈的话算是自相矛盾,先是说关西的兵力不能减少,现在又说想要平定交趾,就必须使用西军。在天子面前说话自相反复,一旦传出去,御史们基本上就会争相上书,质疑韩冈担任现在职位的资格。

不过在场的君臣不会去怀疑韩冈的智商,只会是认为韩冈是话中藏话,别有一番谋划。

“可是打算调用在茂州平叛的兵马?”蔡挺以为自己想到了,“从蜀中去邕州,基本上都能走水路,如果从茂州调兵,倒是方便。”

几个宰辅都暗自摇头,吴充更是没掩饰脸上的讪笑,茂州的兵马岂是现在能调动的?韩冈若是如此打算,少不得要批得他灰头土脸。

韩冈看了看副枢密使,蔡挺这番话看起来就像是在给自己设陷阱——调兵方便不代表茂州方便——不过他并不会踩上去,“并非如此。茂州初定,但当地蛮部仍未彻底降顺,贸然调兵离开,恐怕蛮部就会揭竿复叛。”

“除此之外,哪里还有关西军?”吴充摇了摇头,不论韩冈是不是糊涂,但西军肯定是没多余的兵力调给广西,“陛下!河北禁军兵甲俱足,校阅训练也是逐日而行,虽不如西军精锐,但交趾更是远远不及西北二虏。若以河北禁军攻打交趾,只要指挥得当,粮秣备足,不贪功冒进,当可一举平复。且有韩冈在,当不用担心北人不服水土。”

王韶觉得吴充似乎是铁了心要派河北军上阵,但章惇韩冈在河北军中素无威信,统领大军的时候,如何能让下面的将校士卒俯首听命?如果败了,先被追究的可是领军的韩冈章惇,吴充这个推荐人却不会有太大的问题。

韩冈当然也不会去要河北兵:“河北禁军兵甲俱足,日常校阅亦多,然久不习战,贸然上阵,恐多有折损。即便交趾再弱,河北军与其对阵时,数以千记的伤亡也是免不了的。”

吴充立刻闭口不言,韩冈的话本来就是兵家正论。不过韩冈东否定西否定,西军调不了,河北军又不肯要,难道要用京营,那样可就真的是个笑话了。天子都不可能答应的。

赵顼此时正在想着韩冈的话。精兵只出于战阵之上,这一点就算是他都明白。几十年不上阵的河北军,如何能与关西禁军相提并论?

可旧年曾让契丹铁骑也得绕道而行的河朔精兵,现如今却落到了要上阵的时候,统军的将帅连要都不想要的地步。眼前的韩冈,还有之前要领军去茂州的王中正,都是如此。

韩冈出自关西,只相信西军的实力,而王中正也是在鄜延路和熙河路亲眼见证过西军的战斗力的,他们的态度也许不足为奇,但在场的宰辅们,都没有一个出来质疑韩冈的言辞,甚至连反对最力的吴充也一样。

也就是说,河北军不堪一战,已经成了朝堂上的共识。

赵顼的心中可谓是五味杂陈。如今在南方横扫蛮夷的荆南军,说起来也是以西军为核心,算得上是西军的一支。难道日后南征北战,都要靠西军不成?

“兵不习战便不可大用,韩冈所言甚是。”冯京这时站了出来,“且用兵贵在严号令,要做到能如臂使指。邕州大捷,是因为荆南军常在章惇麾下,而李信又是韩冈的表兄,无论兵将,都能掌握得住,不会自行其事。如果换作是河北、京营,章、韩二人,又能让宿将骄卒们信服?不如少待时日,等到鄜延、熙河两路可以调动,王舜臣、赵隆等良将从阵前抽身,再进兵交趾。”

冯京一番话,王韶在旁听得都是脸色大变,李信、赵隆、王舜臣如今赫赫有名的少壮将领,都是与韩冈交好,从年龄和战绩上,日后都是要坐镇一方的主帅,甚至有望晋身三衙。但现在特意点出来,根本是居心叵测,更是让韩冈和他王韶一起陷入困境。

韩冈冷淡的瞥了冯京一眼,哂笑一声,“冯相公所言韩冈不敢苟同,既然明知河北军不堪使用,为何加以习练?如今可以避战,日后难道还继续避战?”

韩冈的话毫不客气,冯京很有风度笑了一笑,反问:“西军不成,河北亦不成,不知韩冈你打算请调哪路兵马?”

“韩卿!”赵顼望着韩冈,说了半天,他的确是一直没有给出一个明确的回答。

韩冈抬眼面对天子:“陛下,攻与守,自然是进攻更难,损伤也会更大。不过如果是守御,就算是以邕州几千从未上阵不堪一战的老弱,不也挡住了交趾的十万大军?”

吕惠卿闻言便是眉眼一挑,王韶也一拍沙盘的边缘,‘竟还有这一招!’

在场的宰辅们都明白了,赵顼也明白了,“韩卿的意思是……”

“调河北军入河东、关西助守!而西军南下!”韩冈朗声说着,“西军如今天南地北皆是战绩赫赫,这是从皇佑之后,用了三十年的守御之功,慢慢历练出来的。河北军底蕴深厚,想必只要多历烽烟,循序渐进,数载之后,必然不下西军。不仅河北军,京营禁军也当上阵历练。日后也能派上大用!”

韩冈双目一扫殿中诸宰辅,想将西军拖在关西?他韩冈还打算将河北、京营两部一起都拖下水!

冯京、吴充等人一个劲的说关西不能再减少兵力,又建议调动河北军,那就让他们如愿以偿好了!

韩冈心底冷笑不已,这样的说法的确会拖住西军南下的脚步,但与此同时,难道不是也会让天子担忧起其他地方禁军的战斗力来?

禁军五十六万,陕西也不过是占了其中三分之一多一点。难道剩下三十多万就让他们继续烂下去?!

“祖宗之时讲究着内外相制,禁军更是要逐年更戍。边军常年作战,而拱卫京师的京营亦是战功赫赫之军,使京中不至受之于外。只是现如今,禁军多已驻泊,更戍之法多年不再施行。京营、河北久不习战,而西军偏重一方,何谈相制!?”韩冈厉声质问,轻轻一转,就将冯京栽过来的罪名全都卸到一边。

赵顼沉吟着,点着头,“韩卿所言甚是!”

韩冈瞥眼看了吴充一下:“方才吴枢密欲以河北军南下,以练兵论,并不为错。不过正如欲起沉疴,先得以温补的方子来滋养元气,哪里能遽然施以虎狼之药?以西军南下攻打交趾,而河北军填充过来,守御缘边寨堡。而京营也当同样拣选精锐,调动去前线临战待敌。”

武英殿中现在只有韩冈一人的声音,为官数年的他,不想殿中宰辅们这般会受到思维定势的无形约束,能跳到圈外来思考:“相比起军制之重,南北贼虏,不过是癣癞之疾。臣请陛下再行更戍之法,使诸军得历战火!如此,五十六万禁军方能名副其实,而不是仅仅一西军而已!”

