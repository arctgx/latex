\section{第13章 不由愚公山亦去(三)}

【可能这些天酒喝多了,思路有些慢,到现在才把昨晚的一章码出来。今天初五,不出意外的话,还有两章,希望思路能顺畅一点了。求红票,收藏。】

一刻钟后,魏楼上的韩冈和杨英,已经从由净慧庵火场赶来禀报的王九口中,听到了窦解在王家被擒,又被走马承受刘希奭亲自押往州衙的消息。

“这么说,窦解现在应该已经在州衙里面了?”一听完,杨英就紧张的追问。

“不出意外的话,当是快到州衙了。”王九肯定点点头:“为防万一,刘走马押着窦七衙内走后,老五就在后面跟着去了衙门查探,还招起了几十个男女在后面跟着。周家两兄弟则还在净慧庵那里救火,等火灭了就会脱身回来。”

杨英回过头来,已是喜上眉梢:“韩官人,这算是大功告成了吧?!”

韩冈抿着嘴,想了一阵,最后偏偏头,对杨英笑道:“本以为傅勍不敢把窦七绑回衙门,没想到刘走马会横插一杠。唉……”他叹了一口气,“这才叫人算不如天算,后面的计划全都得变了。”

杨英和王九顿时紧张起来。杨英迟疑的问着:“韩官人,难道窦解被押到衙门里,反而是坏了事?”

“不,结果只会更好!”韩冈笑道,“比预计得好得多!我在定计时,从来都是做着最坏的打算,不成想今天突然冒出个刘走马,这丢铜板还能丢出个浑纯来!”

赌博掷铜钱,掷成全字或全背便唤作浑纯,即是赢家通吃,可几率如此之小,很少有人能成功。韩冈事先也绝不敢去幻想着会有这么好的结果。

在他想来,傅勍肯定不敢把窦解械送有司,只能拿着窦解身边的跟班作数。可如此徇私枉法,秦州城内必然会掀起轩然大波,高遵裕便可以光明正大的出面上书天子,顺便再明着送王启年的寡妇去京中告御状。那时无论窦舜卿会不会派人来阻截,韩冈都是赢定了——他只怕事情闹不大!

而现在,横地里冒出来的刘希奭把窦解押去州衙,不必请动高遵裕出头,事情便已经闹大,却正如了韩冈之愿。

“今次之事,你们做得很好,比我想得还要好。”韩冈夸着王九,并不吝啬赞许之词。整个行动中,除了王启年遗孀遭了罪,一对儿女受了点惊吓,再没有其他伤亡。为了让净慧庵中人能及时逃出,王九可是亲自花钱在里面睡了半晚。

“不过你们在中间掺和了这么久,下面就该站到旁边看热闹了,也防着窦舜卿狗急跳墙被误伤掉。”韩冈拿起酒壶,找了个干净的酒杯斟满了,郑重的递给王九:“王九,这一次多亏了你们,事情才如此顺利,且满饮此杯,权且代表本官的谢意。”

韩冈看着受宠若惊的王九接过酒杯,脸上泛起了微笑。一直悬在心头上的巨石,终于被放了下来。他提心吊胆了多日,总算是安全了——窦舜卿无法再在秦州为官,而焦头烂额的窦副总管在秦州剩下的短暂时间里,也不会再有精力来跟他过不去了。

……………………

此时,窦舜卿结束了一场宴会,刚刚回到家中。

换了衣服,在房中坐下。喝着端上来的滋补药汤,他问道:“七哥儿人呢,怎么我都回来了,他还不来请安?去找他过来。”

一个仆人领命去窦解院子转了一圈,回来禀报道:“七衙内好像出去了,不在房中。”

听着仆人回来说窦解不在自己的房中,窦舜卿就把手上茶盏在桌案上重重一顿,怒道:“这个小畜生!又不知逛哪家青楼去了!”

前些日子,窦舜卿一直都将窦解禁足,禁止他出外。不过在关了他几天后,窦舜卿还是放了孙子出来。窦家的这个长门嫡孙,至少在窦舜卿面前,一直都是摆出听话受教的模样,故而也最受他宠纵。当窦舜卿的几个儿子受了荫补后在外为官,他唯独把窦解这个冢孙留在身边。只是窦舜卿没想到,他的这个长孙,越来越不成样。

‘回来后要好好治治他。’窦舜卿发着狠,‘他那些狐朋狗友全都刺配了事。’

“出事了!七衙内出事了!”林文景急匆匆的走了进来,打断了窦舜卿的盘算。

窦舜卿悚然一惊,他的这位幕宾不是还大惊小怪的性格。“七哥出了何事?!”他急问道。

“七衙内犯了事,被押到州衙里去了!”

“押?!”窦舜卿花白的眉毛一挑,阴声道:“是谁押了老夫的孙子!?”

“是刘走马!”

“刘希奭吃了雄心豹子胆,敢动老夫孙子!”窦舜卿狠狠一拍桌子,大发雷霆,“这阉货倒是有胆,前面跟王韶勾勾搭搭,老夫都不理会了,现在竟然为个灌园小儿出头,跟老夫过不去!说,他栽的七哥是什么罪名?”

林文景也是听到风声就匆匆而来,说不出个所以然:“小人听到七衙内出了事,就急着赶过来禀报,没来得及细问。”他突见窦舜卿脸色一下变得难看起来,忙为其出谋划策作为补救:“不过不管什么事,都是跟在七衙内身边的那群狐朋狗友给撺掇的,与七衙内本心无关。”

窦舜卿满意的点头,林文景的意思就是要把所有的罪名都栽给窦解的那帮子狐朋狗友。他对林文景道,“你给我带话给李师中,老夫那孙儿一向被管得严,作奸犯科的事是不敢做的,只怕是有人打着他的名号作恶。他又有官身,还望不要失了朝廷体面。”

林文景点着头:“小人明白!”

……………………

目送着林文景怒气冲冲出了庭院,李师中冷笑着对坐在一侧的姚飞说道:“窦舜卿是老糊涂了,竟然以为让人说上两句就能把这事给瞒下来,也不打听一下这案子闹得有多大!就让窦解在大狱中住上一晚。等明早再好好审一审他。”

姚飞也是冷笑:“杀其夫于前,欲灭其满门于后。前面窦舜卿杖死王启年的案子都要翻了,窦解的官身肯定保不住。连窦舜卿自己都脱不了干系。”

两人都在冷笑着,并没有半点同情窦舜卿的意思。虽然对付王韶时,李窦二人是同仇敌忾,但现在窦舜卿翻了船,李师中却不会为他趟浑水,“刘希奭既然插了手,那这案子就是通了天,窦舜卿手再长也都挽回不了。”

“这一下,窦舜卿也不可能留在秦州了。”姚飞阴阴笑着。

“王韶屡立新功,这些天子都看在眼里,免不了要大加封赏。既然王韶用功无过,那我是不可能再在秦州待了。而不出意外的话,张守约从京中回来,也会顶替向宝的钤辖一职。至于窦舜卿,若不是有今日之事,他肯定会被留任的。”

自从古渭大捷之后,李师中除了没有去迎接王、高二人带回来的凯旋大军,以表明自己的立场,并没有再与王韶他们为难半分。现任的秦州知州很清楚,他在秦州的时间已经寥寥无几,很快即将外任,说不定还会被挑出个罪名被降官处置。

王韶在一片反对声中连续两次大捷,斩首数百上千。换作他是赵顼,也不免会想,如果王韶能得到秦州上下的全力支持,立下的功劳定然十倍百倍于前。既然如此,但凡之前明着跟王韶过不去的官吏,都别想再在秦州待了。比如窦舜卿、比如向宝……再比如他李师中。

当然,秦州是边地要郡,直面党项、吐蕃,天子和政事堂为了秦州军政两方面的稳定,绝不可能同时调换这么多官员。他李师中算是罪魁祸首,肯定要走第一个;向宝重病在身,无法执掌军务,又挡了张守约的路,同样会被尽速调走。那么,秦州军方排在前三的最后一人窦舜卿,京中就不会再轻易动他,相反地,他说不定还可以再进上一步——

“窦舜卿、向宝还有经略你,都是反对王韶的拓边之策。如今经略和向宝若是被调职,为了稳定秦州军务,窦舜卿甚至可能会进上一步——顶替经略你的职位,来权知秦州!”

若是在前两日,说起此事时,姚飞的声音中肯定会带着几许不忿,连带着李师中的脸也会板起来。

秦州局势变化的方向,无论是李师中,还是姚飞,他们都是有着同样的判断,最占便宜的不是王韶和高遵裕,而是窦舜卿。鹬蚌相争渔翁得利,倒也罢了,只能说人家眼光好、手段高。但窦舜卿明明是与王韶为敌的急先锋,其他人都倒了霉,偏偏就是他把最大的桃子摘到手中,这当然会让李师中和姚飞愤愤不平。

但现在不同了,姚飞是笑着说的,

“不过现在是不可能了。”

“相信这一事,王韶和高遵裕能看得出来,韩冈……应该也能看得出。”李师中赞叹着,“韩冈他们挖下了这个陷阱,让窦解那傻子自己跳了进去,顺便把窦舜卿一起扯落下去。这灌园小儿,倒是越来越会用计了。”

姚飞点点头,犹疑了一下,却又皱着眉摇起了头:“总觉得不像韩冈的手笔。”

因为吃过韩冈几次大亏的缘故,姚飞承李师中的命令,曾仔细研究过韩冈的过往行事,发现他的性格向来是宁从直中取、不向曲中求。遇上艰难险阻,往往都是直截了当的一剑斩过去,虽然劈下去的角度通常出人意表,但无一例外都是正面的对决。而今次挖陷阱诱窦解上钩,虽然大获成功,但姚飞却觉得这个计策太过于阴险,不似韩冈的本性。

李师中洒然笑道:“不管是谁的手笔,都是针对着窦舜卿。他来秦州时,私下里应是奉了韩稚圭的意思与王韶为难,现在又因王启年之事,跟韩冈是水火不容。王韶他们他们当然要把窦舜卿赶走,省得他任了知州后,会变本加厉。”

无论是李师中,还是姚飞,两人的对话中都是透着浓浓的幸灾乐祸的味道。

窦舜卿完蛋了!窦解也完蛋了!

若是秦州处断不公,莫说当事的刘希奭要利用他身为走马承受能动用马递的权利,直接奏报天子,高遵裕说不得也会将此事捅到天上去。而且以王韶和韩冈的行事手段,他们说不定会把王启年的遗孀直接送到京里去,去敲那登闻鼓,窦舜卿如何遮拦得住?!

李师中长身而起:“不管怎么说,这一案,我会秉公而断!”

