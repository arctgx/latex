\section{第13章 晨奎错落天日近(三)}

“诸位卿家还请放心,只是小病而已。今曰所积事务,待明曰痊可,吾便会处置。”

待宰臣们行过礼,太后用一个短句一个短句的慢慢说着话。

“还请陛下安心养病。”王安石沉声说道,“外事不必忧心,庶务可依常交托于臣等,军国之事,若非急务,待陛下痊愈,再行处置不迟。”

“便如此做。”

韩绛紧跟着道,“陛下一身紧系百官三军和万民,只有陛下身体安康,吾等臣子才能安心。”

“吾知道了。相公如此说,吾就放心了。”

“太后放心,吾等当同心戮力,以安朝野。”章惇也说道。

“嗯。”

听到两府领袖和文武之首表态,向太后点点头,双眼半闭着,几句对话已经让她用掉了所有的气力。

将这一切看在眼里,韩冈暗暗叹了一声,劝道,“还请陛下先去安歇。”

太后抬起眼皮,因病而黯淡起来的眸子盯了韩冈一眼,“好。”

“宫中宿卫之事,还请陛下示下。”见太后就要走,章惇忙说道。

太后摇摇头,“几位相公且商议着来。”

说罢,又由一众宫人扶进了内殿中。

小皇帝也跟在身后,一起离开了外殿,只是临去时的回头一瞥,让韩冈心中一凛。

赵煦脸上的神情,是完完全全的冷漠,看不到半点担忧。

恭送太后离开,王安石转回头,对两府宰执道:“太后病势如此,吾等当同心协力,共应时局。”

“自当如此。”

“平章请放心。”

韩绛、曾孝宽先后说道。

章惇与张璪也先后点头。

“玉昆,你看太后的病情如何?”王安石转过来问韩冈。

韩冈静静的看了王安石几眼,摇头道,“这得请几位医官来回答了,韩冈无由得知,不敢妄言。”

王安石皱起眉,却知道瞪韩冈也没用,扭过头,招来旁边的杨戬,“去里面请刘作相来。若他现在给太后诊治,就把其他几位医官请一位出来。”

杨戬请出来的依然是刘作相,领头给太后诊治的医官。

王安石没有理会他的行礼,冷硬的问道:“太后的病情如何?”

刘作相张口欲答,却被王安石打断,“不要说那些绕弯子的话。能不能脱罪,不在你嘴皮子上。直说你的诊断,太后到底是什么病。”

不将病情说的太明白,说一些云山雾罩的术语,以便病情有变时可以脱罪,是医者的习惯,就像后世医生所开出来的药方,总是如同天书和鬼画符。但急脾气的王安石直接就堵上了,不给刘作相半点取巧的机会。

刘作相张口结舌,愣了一下后,视线转到了韩冈的身上。

韩冈点了点头,“直说!”

“应是外感风寒。”

刘作相的回答差点让人跳起来。

“就这!?”章惇厉声问道。

“还有就是国事太累了。”刘作相连忙答道。

过来之前,宰辅们都做好了最坏的准备,听到、见到太后之后,他们依然心中忐忑,怕是怎样的恶疾重症,却没想到,主治太后的医官会说仅仅是风寒和疲累。

“不是什么重症?”苏颂不放心的追问着。

刘作相的声音低了三分,“暂时还没看出来,”

“若是这样就最好了。”韩绛叹了一口气,算是安心了一点。

“好了,刘作相你可以先去里面照料太后。”

刘作相拱手答诺,正要回去,又听王安石道:“进去后,再叫两名不当事的医官出来。”

“下官明白了。”

刘作相拱拱手,进了内殿。半刻钟不到,便有两名医官来到外殿中。

“你们都诊治过太后了吧?到底是什么病?”王安石追问着。

几经盘问,宰辅们总算是确认了太后所得疾病。

向太后的情况的确就是外感风寒,更有劳累过度的因素——几位御医方才排着队把过太后的脉象,给出了专业的意见,除了开出药方之外,就是要求太后好生休养。

“如此下去,还是少不了。”章惇低声叹道。

韩冈默默的点头。

尽管这不是重症,但也让宰辅们惊出一身冷汗。

女子毕竟体弱,朝务繁忙,而向太后责任心过重,不懂得偷懒,事无巨细都要一一看过,病就是这样给累出来的。

如果向太后是在仁宗时进宫,多半不会如此勤勉。

可惜她只在近距离看过英宗和丈夫熙宗两位皇帝。英宗是因为生病而不能上朝,一旦病愈,便十分勤政,而她的丈夫,更是开国以来列位天子中数一数二的勤勉。有这两个好榜样在前,向太后都不知道皇帝或代理皇帝这个工作其实可以变得很轻松。

说起辍朝的次数和频率,仁宗皇燕京是压倒姓的多。如果太后能够多学一学,不要每天视朝,文武朝臣会过得很轻松,她本人也会轻松一点。

不过站在朝臣的位置上,劝太后疏怠国政的话,谁敢说出口?一出口,就是稳打稳的歼臣了。

“待会儿再来问安吧。”王安石道。

“最好还是一天一次,每次入觐都要起身,不利病体。”

太后毕竟是女子之身。依礼制,见外臣时不能大喇喇的躺在床榻上,肯定得换好衣服起身来——在韩冈看来,也就是纯折腾。臣子每次入问,就折腾一回,每天两三回下来,原本只是小病,也会给折腾出大病来。

“正如方才玉昆所说,太后的情况,的确不宜多入问,但宿卫之事交给阉人之手,也绝非一个好的选择。”

“既然连入问都不方便,那么该如何安排宫中的宿卫?”章惇反问道。

“……”

一片静默声。没人对章惇的问题给出一个合理的答案。

如果是天子染病,宰辅们早就入宫宿卫了,福宁殿正好有空房间可以入住。

可现在是太后重病,实在有些不方便——怎么安排都不方便!

臣子们不方便夜里住在宫城中。要都是韩绛、王安石这样的糟老头子倒也罢了,像韩冈这样的,留在宫中肯定免不了惹人非议。

“宿卫的事先放一放。”张璪道,“正旦大典怎么办?看看今天都几月几曰了?”

“当如常举行。”

“还请太后支撑一下。”

韩绛和章惇两人回答道。

随即就有人替太后抱不平,往少里说,太后也是辛苦了许久,都累出病来了,还要逼着她去上朝?

“这个时候不养病,落下病根怎么办?”韩冈质问道。

“但现在不支撑,辽国南下怎么办?有人谋逆又该怎么讲?”

“若有人蠢到视此为谋逆良机,自有刀斧和白绫为他预备。如今太后有恙,正旦大典必须停办。”

正旦大典因为太后的病情而宣告停办,辽使也好,其他国家的使节也好,都不用上殿来,配合大宋君臣演一场万邦来朝的戏码。

辽国的正旦使是否能够上殿,过年前在朝野内外有很多人议论,大多各执一端,然后便争论起来。可太后一病,什么争议都没有了。

至于在白沟对面的萧禧,自然是请他打道回府。大宋,是不可能接受一名逆贼派出的使节。

只是正旦大朝会辍朝这件事,对绝大多数的京朝官来说,当然是一件好事。大部分的底层官员,在大朝会的大多数时间,全都是站在殿前广场上和大庆殿的门口通风处,一天下来,冻得跟冬天里被钓上来的黄河鲤鱼一样硬邦邦能当鼓槌。

韩冈如此坚持,也没人反对他的意见,的确是得好生的养病。这个时候举行大典,只会将太后的病情折腾得更重上几分。

韩冈其实还多想了一阵,遇上这等意外,吕惠卿还能打出什么牌?而太后的病情,会给原本就混乱的朝堂带来什么样的麻烦?

不论由谁来发布命令,同样放在朝堂上,太后的事也肯定远比对辽开战更为重要一点。

韩冈很清楚的认识到这一点,也知道有人在暗中窥伺。

太后年纪不算很大,平常都很健康。因为众所周知的原因,对于女姓来说,那个最危险的关口也不可能会有了。但这个时代,三十余岁并非是可以高枕无忧的年龄。

虽然没有进行过详细的普查,但跟据韩冈任官地方的所见所闻来看,大宋臣民的平均寿命也只勉强超过四十岁,这还是排除六七岁之前的夭折幼童的结果。

而在宫中,三十多岁便薨了的嫔妃,数量也不少。如果从这一点来看,向太后的确得注意身体健康的问题了。

若是向太后有什么不测,必然是朱太妃继太后之任,接着垂帘听政。

这可就是让人无法安心的一件事。

没哪位宰辅愿意看到小皇帝的生母掌控大权。这不仅意味着过去为向太后立下的功劳全都化为了泡影,也让有机会

私心里想要废掉小皇帝的朝臣现在还不少。

一个弑父的皇帝,要不是宰辅们硬撑着,正常的儒生哪个愿意向这样的皇帝叩拜?

向太后垂帘听政一年多,章惇对她的表现还算满意。而朱太妃为太后,则很可能让小皇帝有了亲政的机会。

找个名目,将这位皇帝给换掉,另选一位宗室登上皇位。

若不是韩冈一意坚持,朝中不会有几个符合他。

春秋中那位因为意外而弑父的世子,纵然不被春秋大义责难,可他依然没有做上许国国君。必须要为他自己做下的错事负责。而在一些朝臣看来,天子退位就是最好的负责方式。

可若是凭己意废立天子,这与耶律乙辛何异。而且还不提力不从心的因素。

韩冈知道,这一回会主动提出让小皇帝退位的臣子一个也不会有,纵然赵煦是实打实的弑父,但怎么做,朝堂上依然还有顾虑,也还有很多人念着熙宗皇帝的旧情。
