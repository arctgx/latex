\section{第28章 大梁软红骤雨狂(七)}

【真的不好意思,实在吃不住了,明天还要上班,昨天缺的一章只能有空再补。】

韩冈派李小六去章家递名帖,约时相见的时候,章惇却正在王安石府中。

当初的三名深得王安石倚重的助手,现在还是三人,不过少了个吕惠卿,多了个前任宰相曾公亮的儿子曾孝宽。

吕惠卿是因为父丧而不得不丁忧回乡,需要庐墓三载才能再出来。因为吕惠卿的官位还是太低,资格不够,王安石也没办法帮他争来一个夺情起复的诏书。现在代替吕惠卿主持司农寺内外事务,实质上统管新法推行的,是曾布。

曾孝宽是曾公亮的儿子,一直以来都以宰相之子的身份行走,对王安石的事业帮助不小。如今要在开封府推行保甲法,他这个提点开封府镇界,正好有资格从中接手,来主持推广。

原本因为曾公亮的宰相身份在背后,曾孝宽并不能算是变法派的核心成员,只能算是同盟。但现在,曾公亮因为李复圭的一首诗,而自请致仕,曾孝宽也便少了阻碍,进入了核心层,得以主持一项新法的推行,现在落在他手上的就是保甲法。

眼下朝廷的重头戏尽管都在横山那边,但各项新法条令都是按部就班的在做着。而因为前一阵与王安石及他的新法,所展开的血肉横飞的死拼,反变法派也是元气大伤,被赶出京城的一个接着一个,参与的几个领头的,更是被发遣得远远离开。现在朝中的反变法派,几乎不敢再这种两面俱伤的手段。

现在,反变法派也只能咬牙切齿的看着将兵法在陕西一步步实现。经过了卓有成效的推广,由几个指挥合并而成的‘将’,其数目在关西已经超过了二十个,拥有三万多士卒。而这个数字,还在不断的扩充起来。韩绛手下的军队基本上已经整编完毕,出自延州帅府的军令,也多是通过各将的正将来处置。

而今天,变法派的核心齐聚,则是针对在开封推广保甲法的商议。

推行任何一项法令和政策,最关键的就是不能让百姓生乱。但现在已经有谣言在开封府内外传播开,说是推行保甲法,是为了籍民为兵,“已经有传言说所有登记在册的保丁,都会被征发为兵。”曾孝宽向在座的几位通报保甲法推行的现状。

“可笑之至!”章惇对谣言嗤之以鼻,“令绰【曾孝宽字】你最好放手施为。这样的谣言,当用雷霆手段去处置!”

曾孝宽点头道:“子厚之言正是孝宽本意。保甲法并不是什么新鲜的货色,如陕西,早有弓箭社、忠义社,河北亦多忠义社。百姓团聚自保,以抗盗贼,天下无处不有。这些谣言,不是因为无知而传播开的。”

曾布道:“关中隋唐时,遍设折冲府,以折冲都尉统领。如今天下虽然早已改为募兵,但关西旧日折冲府的根底还在。忠义社、弓箭社也以陕西最多。陕西推广保甲法应该更容易一些。”

无论是弓箭社还是忠义社,都是陕西用来自保的组织,基本上是将一村或是一乡的精壮聚合成军。这一点的确跟隋唐时的府兵制有几分相像。府兵制的基本单位就是将地方划分成一个个折冲府,府中下辖六百到一千两百名士兵,都是良家子,平常居乡务农,战时闻召出征,而不是如今用钱招募来的兵员。

“开封冗兵甚多,将兵法一行,厢军汰撤当会近半,而禁军亦是难免。开封驻军消减,保甲法不行,天子那里也难安稳。”王安石转对曾孝宽道,“此事还要多劳令绰。”

曾孝宽躬了躬身:“不敢称劳。”

“如果保甲法在开封推行得宜,就当尽速将其推广天下各路!”章惇说道,“荆湖溪洞蛮不服王化,多有下山做过之人,汉儿饱受欺凌。若将此法在蛮寨周围的汉家中推行,当有奇效。”

曾布和曾孝宽交流了一个眼色,这章子厚当是看到了王韶的荣光后,开始不甘寂寞了。

虽然荆州早在秦汉之时就已经是中国之地,但荆湖一带的山区,有着诸多溪洞夷族。千年来服叛不定,时有与汉人交恶,甚至有从汉代到今朝,隔三差五就叛乱的部族,如今辰州就有好几家正起兵作乱。

不过王安石知道轻重:“此事并非急务,等横山事定再提不迟。”

“横山之事,看好的人不多,韩玉昆那边也是不看好。天子现在要见他,该怎么办?”章惇忽然提到了今天刚刚进京的韩冈。不同于王安石、曾布两人,章惇并不是很看好韩绛在鄜延路的冒险。在他看来成功的几率大约是一半一半,很难让人权衡出高下。

听到韩冈的名字,王安石不自觉的皱了下眉头,韩冈的事的确有些让人头疼。他看看几个得力助手,章惇是肯定站在韩冈一边,而皱着眉头的曾布则是与章惇不同,并不喜欢韩冈。自从当日听了韩冈三策之后,便对其就有了看法,总觉得韩冈心术不正,是唯恐天下不乱的那种人,绝不可重用。

两边的态度都不会客观,王安石看向曾孝宽,“令绰,你有何看法?”

曾孝宽想了想,道:“天子都想见他,一直都挂在心上。现在韩冈已经进京,也不便真的阻拦,那样做反倒是显得心虚……如果能让韩冈改弦更张,收起那番话,事情也就好办了。”

“这事可就难了……”章惇略略拖长了声调,“韩玉昆行事刚直,几无偏曲,少有妥协。要让他在御前委婉曲意,怕是缘木求鱼。”

曾孝宽听说过韩冈的事迹,比起张乖崖还要有侠客之气,也有班定远的几分风采,最近在蕃部拔剑斩了西夏使者更是一个明证。这样的人,当然都是执拗的性子,甚至有可能是一根筋走到底。要让韩冈在殿前改为韩绛鼓吹,的确是很难说服成功。

“韩冈不过一个选人而已,招他入京,已是抬举他了,何必为其大费心神?”曾布很不快,“天子若是想起韩冈,就让他进宫面圣。如果天子不提,那也就罢了。左不过一个卑官而已,难道还能阻碍国是不成?!”

章惇微微冷笑着瞥了曾布一眼。其实能在密会上,正儿八经的把韩冈提出来商议,等于是已经认同了他的地位。而以韩冈如今给天子留下的深刻印象,普通一点的朝官,都比不上他的影响力。加上韩冈本来就很容易得人好感,天子就此垂青于他,韩冈就此一飞冲天都不是不可能。

如果韩冈能飞黄腾达,章惇是乐见其成。韩冈于他父亲有救命之恩,这等过命的私谊,比起同乡、同窗、同科的关系都要坚固得多。而且韩冈的年纪比自家小了二十岁,章惇也不担心他能给自己带来什么压力……其实最关键的,就是韩冈的行事风格,实在很和章惇的胃口。

“其实韩玉昆为人刚正,而且识量过人。虽然长于经史,疏于诗赋,若在往年,不过一明经,但如今进士科将改,以他的才学,考个进士出来也不难。日后前途不可限量。”章惇看了王安石一眼,想了想,没把后一句说出来。但王安石要为二女儿招亲的事,在座的都清楚。

王安石敛容不语。其实对于二女儿的夫婿,他心中本来有了人选。今年登科的蔡卞,相貌、才学、家世都是一等一的,而且还他的弟子,人品早早的就了然于胸。这样的女婿哪里挑得出毛病,比起曾经让他起过念头的另外一人,要强出许多。

只是发榜后的那段时间,因为韩冈提出的三条策略,使得新法的颁行速度陡然加快。几套政令齐下,一封封大诏出台,不但学士院几天一锁院,连中书的灯火都是日夜通明,王安石忙得连家都没回,就算回家,也是倒头就睡,醒来后,就又急急的入宫去了。

等王安石听着蝉鸣,从案牍中抬起头来,都已是六月中。还未婚配的蔡卞早就被人抢了去做了女婿,新科进士也都被瓜分了个干净,自家女儿的婚事就这么被耽搁了下来。

王安石对女婿的要求不多,家世清白,人品出众,才学国人即可,即便是寒门素户也无所谓,当然,相貌也须过得去。就是不能嫁到政敌家,不过也不能让女儿成为他人攀龙附凤的工具。

这样看来,韩冈的确是个难得的人选。而且年后才二十,在这个年纪上,能如此相配的的确不好找。

对于韩冈的人品,王安石很赞赏。不畏权势,坚持己见,这是难得的品格——尽管表现品格的对象是自己——加之不贪功求进,隐去了蕃部中的一剑,放弃了唾手可得的名望,把功劳转嫁给瞎药,硬是逼得蕃部首领只能投靠大宋。虽然其中有点欺君的成分在,但一片为国的拳拳之心,可见一斑。所以天子完全没有计较——把天子的诏书丢一边的事,郭逵就曾干过,硬是瞒下了天子的诏令保住了绥德城——只会让赵顼更加看重。

可是,既然韩冈如此出色,别人也不是瞎子,单是王韶就不可能放过他的……

王安石忽而失笑,想得实在太远了,眼下可是在说要不要让他进宫面圣。不过王安石的想法与曾布不同,“还是不能给韩子华那里添乱。既然韩冈说不要功劳,那就随他的意好了,但事还是要做的,鄜延军中的医疗救治需要他去主持,这件事,他别想脱卸。至于天子那边,也没必要见一个选人,等韩冈积功转为京官再说吧!”

王安石丝毫不给韩冈留半点情面,微沉而严重的神色,让人由此了解到,拗相公的名号可不是白叫的。

