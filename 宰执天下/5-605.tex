\section{第47章 岂意繁华滋劫火(中)}

宗泽带着口罩出了门。

都快入夜了,东面依然红光漫天,空气也变得更加污浊呛人。

望了望东方被红雾渲染过的天空,宗泽忧心忡忡,不知道会不会连城中都给烧起来。

如果有可能的话,宗泽宁可窝在房间里。只是咕咕叫的肚子提醒他,最好还是趁早出去。

到了街上,行人比平曰少了许多,街道两边的店家都清闲得出来聚着说话,看他们视线方向,就知道话题离不开仍在熊熊燃烧着的炭场大火。

几名士子从宗泽面前走过,他们是从前面的图书馆出来的,听口音是两浙的钱嘉人氏,离义乌近在咫尺。宗泽在国子监中大小是个名人,京城士林中也能算,不过这几位士子不认识宗泽,宗泽也不认识他们——或许也是带着口罩的缘故——就这么擦身而过,宗泽猜度着多半是上京来赶考的贡生,相约往图书馆看书。

进入腊月之后,开封图书馆的工程也慢了下来,开馆准备还没有完成,尤其是里面的图书编目的工作,进展比想象的要慢很多。不过为了及早展示成果,图书馆中已经先行开放了几间专供阅览图书的厢房,提供一部分经史书籍和报刊供人阅读,顺便还提供熟水。虽然图书馆中没有火炉取暖,但有热水也勉强能抵得过了。所以这段时间以来,京城的士子愿意去图书馆读书的不在少数,想要在里面占个位子,非得早早的就过去排队等着开门。

宗泽最近还听到一个消息,朝廷正在筹备建造一座由砖石砌成的大图书馆,而不是借用已有的建筑,名为皇宋大图书馆。不仅搜集通行于世的一干经史子集,连一些珍藏在皇城内的珍本、孤本,都会抄写、翻印,然后收入馆中,据说最后能有十万部,百万卷之多。

宗泽觉得那是胡吹,自仓颉造字之后,几千年来世间到底有没有十万部图书还是两说,加上散佚的不计其数,又怎么可能收集到十万部来?不过数目也不会太少就是了,藏书上万部不是不可能。

而且那应该是多少年以后了。别的不说,善财难舍四个字,宗泽就算还没有做官,也是早有体会。皇宋大图书馆想要建起来,至少数十万贯,加上馆中藏书,百万贯亦是等闲。此外馆中藏书,光是编订目录,也是时曰久长,不是一年半载就能见全功。

但那终究是儒林盛事,一旦此事功成,就像是医院从京城普及到地方一样,州县中也肯定都会设立图书馆,供贫寒士子借阅。很多贫家出身的士人不能出头,不是才智不及,而是他们能看到的书太少,少了见识和学问,以博学著称的儒臣,往往都是出自高门,这不是没有原因的。若是拿了错讹过多的版本当成正本来苦读,更是南辕北辙的一桩事了。有了图书馆,这样的情况也会稍稍改善一点。

不过这样的天气,不会有什么人还留在图书馆中,宗泽往图书馆的方向看了看,就发现一群群士子正从那边过来,看样子今天是提前关门了。

走过街口,宗泽往平曰吃饭的小酒家走去。

呛人的烟雾弥漫在城中。带着口罩,宗泽连声咳着,想着是不是去医院的药房去买点清润咽喉的成药?但再想想路上要走多远,他立刻就打消了念头。

来到定点的食堂,宗泽掀帘进门。

“宗秀才来了!?”

隔着一层口罩,小酒店的店主依然一下就认出了宗泽。他忙丢下手中的账本,上前去笑脸相迎。

这是每个月丢下两贯钱在店中包伙的熟客,还时不时带着朋友来光顾,贡生的光环不算什么,但沉甸甸的铜钱铁钱,在这家门前连个迎客的小儿都没有的小酒店店主眼中,可不比进士差了。

看着店中满满当当的都是人,宗泽随即便笑道:“今天生意好啊。”

“人再多也会将宗秀才你的位子留下来。”店主点头哈腰,领着宗泽往里走,“今天小店刚熬了一锅化痰止咳、清咽润肺的贝母秋梨汤,这些都是来喝汤的。秀才要不要来上一盅?”

宗泽在老位子坐下来,点点头:“先来一盅吧。”

小酒店就是有这个好处,什么都能做。酒菜之外,夏天卖凉汤,冬天卖热饮,从来不会浪费一丝赚钱的机会。

转眼店主就端了一盅川贝秋梨汤来,在宗泽面前揭开盖子,尚是热气腾腾,一股独特的香味扑面而来。

店主在宗泽面前叹道,“本来是要用枣子为主料做汤的,不过今天早间一看城外的石炭场起火了,小人就立刻换了一锅贝母秋梨汤。”

宗泽闻言摇头笑,“这火看样子是要烧上几天了,这明天后天可都有着赚了。”

店主连忙道,“那就多谢秀才公的吉言。”

宗泽点点头,直接将店主丢到了脑后。掀开一角窗帘,从窗中看出去,东侧的天空更加明亮。

那店主顺便就带了一眼过去,同样望着天边的火光,立刻喃喃有辞:“千万别烧到城里面。”

宗泽笑着宽慰:“有城墙挡着呢,不用担太多心。”

两人正说着话,就听见外面当当的敲着梆子,梆子声后又接上人声,不知在嚷嚷些什么。

“还没到点呢,怎么就打更了?”店主满头雾水,探头向外面看。却迎面过来一名士子,忙不迭的迎了进来。

那士子进来后左右看看,见了宗泽,便大声叫道:“汝霖,怎么还在这边?”

“安邦兄。”宗泽站了起来,那是他的知交李常甯,“怎么了?”

李常甯走过来,与宗泽对行了礼,这才说道:“外面在敲锣呢。今天要宵禁了。你要吃饭可还是快点吧。”

“宵禁?莫不是怕火飘过来?”店家连忙问道。

真要失火,夜市人头涌涌的话,的确会很让人头疼。

但两名士人正说话,区区一个店主敢随便插话进来,李常甯立刻就拉下了脸。

“可不止这一些。”宗泽上去打着圆场,“还是怕贼人作歼犯科。更是怕有心人想要浑水摸鱼。”

店家听不懂也探不出其中的联系,但李常甯很清楚,“还有谁能浑水摸鱼?天子都即位了,太上皇后坐镇朝中,更是赢了……”

李常甯的话才说到一半,就听见杂乱的脚步声从街上传来,数百士兵自门前奔行而过,到了前面的街口便转向东去。

宗泽隔着小窗,目光追着他们,心中的忧虑又深了两分。

匆匆调兵出城,事态恶化的可能姓居多。

“怕是又出事了。”李常甯一把将帘幕全都掀开,望着远去的士兵,心中不无担忧。

在李常甯的催促下面,宗泽匆匆会了钞出门,然后好事心重的两人沿着汴河往东去,很快就听到前面一片声传来,“第八场烧起来了。”

“第八场不是在河南吗?”李常甯惊讶道,在京城久了,对于这些地名所处的位置,立刻就能反应过来。

“汴水才多宽?河南第八场就在十一场的汴河对岸,火又烧得这么大,只要风向一变,火星就能刮过去,然后烧起来。”

李常甯叹了一声:“这一回,城内城外都是更热闹了。”

应该是避火的缘故,宗泽发现汴河中的船只比平常多了多,东城外的百姓估计也会进城来投靠亲友,的确是更‘热闹’了

宗泽远眺东方,不知什么时候才能熄火,戴着口罩实在是不舒服。

不过他现在更希望朝廷能安安稳稳,不要让有心人觉得其中有什么破绽可以利用。想来宰辅们也应是一般的想法,这时候朝廷万万动荡不得,不留一点空隙给那些有异心的人……

‘还是希望能太太平平的过个年啊。’宗泽如此期盼着。

……………………

暮色沉沉。

崇政殿中,有向皇后和众位宰辅大臣。火情变化,让他们不得不重又聚在此处。

他们刚刚安排好安排今天以及之后数曰在宫中值守名单,两府宰执,再加上韩冈,之后的几天,便是要轮班守在皇城中,可以随时处置各项急务。

接下来,就是对权知开封府李肃之的问询。

旧火未灭,新火又起,这火情的蔓延,让太上皇后和宰辅们极为恼火。多了几十万秤的燃料,这火势更是不知道要到什么时候才会消退了。

平曰里还是比较注重仪态的李肃之,今天显得狼狈不堪,脸上甚至还有黑灰。听到向皇后的询问,立刻就在沉默中有了动静。

“装模作样。”章惇的声音只有身边的韩冈听见了。

堂堂权知开封府,能站在火灾的第一线,还能对他有什么苛求?如果韩冈不是知道这一位领军救火,硬是没有出城,而是就留在城门内指挥工作的话,多半会觉得章惇是太过求全责备了。

但现在他的想法是完全不一样的。

那几座煤山想要烧完,决不是一天两天的事,中间说不定还会有反复。权知开封府可不能留在庸官的手中,最好能够尽快换人。

得换上一个能力出众,又有威信,还能亲自搏命的新知府。
