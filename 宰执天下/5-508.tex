\section{第39章 欲雨还晴咨明辅(37)}

“臣不敢居功。”

韩冈和蔡确异口同声。

蔡确说道,“铁船本是韩宣徽倡言,无臣一事,何能居功,”

韩冈也道:“殿下,铁船如今亦只是小成,且待事成再论。”

蔡确跟着又说,“不过眼下虽是小成,已可见曰后的成果。当赐各匠师、佣工以金银,以激励人心。”

“相公此言甚是。”向皇后一口答应,并不是多耗费的事,这点开支,大宋还给得起。

“不过水师之事,不仅仅是船的问题。”韩冈又开口道。

向皇后精神一震,终于是等到韩冈主动说话,连忙道:“宣徽请说。”

“高丽与京东东路隔海相望,自登莱出海一直向东,四百里外便是高丽,顺风只需三曰。走明州,路程增加数倍,遇到危险的机会也大了许多。但去往高丽的海船,为何却多是自明州出发?”

“之所以不走京东,只是为了提防辽人。”韩绛很奇怪韩冈为什么开始胡说八道,他老家的事,他本人应该很清楚才对,“登、莱因近辽境,自祖宗时便诏禁海商,也就两年前稍稍释禁。这两地可以不论,但若说往高丽去,现在多有走密州胶西板桥,不独明州。”

“相公有所不知。海商提防辽人作甚?多一天航程,可就花多一天的本钱。即是不说登、莱,但去明州的海商依然要比密州更多。只从税入看,便可知韩冈所言不虚。”【注1】

韩冈说到这里,稍稍停顿了一下,双眼在殿中环视一圈后,重又开口,“除了因为在商贸往来上,京东远不及两浙繁盛,也有海路不熟的缘故。”

一方面,高丽海船多选择明州为进入大宋的港口,是商业上的问题。在京东东路和两浙路之间,自然是带来的土产能卖上高价,且收购的特产又便宜的两浙路更为合适。登州和胶西的板桥港加起来,现在也远比不上明州。韩冈就算全力涉足进去,亦改变不了客观现实。

但另一方面,也是船工、水手们自身的因素。否则大宋使臣出使高丽,为什么都是从明州出发?而不是从更近的登州、莱州、密州出发?他们可不需要关心商业,只会希望海上的行程越短越好。

要说登州的港口条件不佳,但密州胶西的板桥港也并不输给明州港。很大程度上,都是因为去往高丽的水手们多是两浙出身,走惯了明州北上的航线,不习惯走其他的海路,只会一次一次沿着熟悉的旧路走。

其实肯定还有洋流、风向之类的问题。但韩冈不甚了了,便干脆避而不谈,反正在列的也没几个人明白。说服人的时候,也最好盯着一条重点来说,不要杂七杂八,反而让人听乱了。

“臣去过交趾,也多少也对海运了解一二。现在的海运航路都不会远离海岸,全都是近海航线,绝大部分的时候,都是在能看到海岸的地方航行。一有不测,便立刻靠岸。远离海岸的航路,比如从广州经过南海往西方天竺、大食去的航路,走那条航路的几乎都是蕃商,就算用的船只是广州、泉州所造,但船主是汉人的并不多。虽然每年都有一些船只在南海上沉没,可绝大部分还是能安然过海。真要计算起来,其实同样距离的海程,毁损数并不比航行于近海更多。”

蔡确轻声的哼了一下。

作为宰相,他听惯了人说话,一旦有人在他面前弯弯绕绕的说话,肯定就是藏着些什么。韩冈的话其实很有些意思,‘同样距离的海程,毁损数并不比航行于近海更多。’,既然话说得这么绕口,也就是说实际上远洋航行还是要比近海多死人——靠近海岸翻船和海中央翻船,终究是是两回事。

韩冈蒙得了上面的太上皇后、前面的韩绛,却别想瞒过他蔡确去。就算很早就随父亲离开家乡,可蔡确他也是福建人……泉州!

隔着垂拱殿正中央的通道,蔡确聆听着对面韩冈还没有结束的议论,心中揣测着,他到底想要什么呢?

“之所以只循旧路,不走远洋,不是不能,只是不敢。只见海天,不见陆地,目光所及唯有一叶孤舟,让人不能不惧。”

薛向闻言点头。南方的情况他不清楚,不过从长江口往北走,黄水洋【注2】、青水洋,船只都走得多,但愿意走海中更深处的黑水洋的船只就很少了。说起来他曾经考虑过出长江入济水,从济水经梁山泊,走五丈河入京城的运输路线,但一听到海运,下面的官吏一个个脸就白了。

“宣徽是想要那些水手敢于走深海航路吗?”薛向问道。

“并非是韩冈所想,而是非得让他们去走。要想守住海防,就必须一批有胆色,有能力的水师兵将。大宋万里海疆,岂能让一群不敢出海的士卒守护?”

韩绛说道:“人心不可不虑。深入海中,举目不见陆地,人心浮动可是难免的。”

“人心易安抚,厚给饩廪,常加褒赏,再换上更大、更稳、更为安全的船只。如此人心再乱,自有军律处置。”韩冈道,“天时、地理才最是需要顾虑的。”

“天时不必宣徽说,出海不看天候,就是船毁人亡的结果。海上遇难,绝大多数也都是遇上狂风巨浪。至于地理……可是说的航路?”蔡确问道。他帮韩冈引下话题,心中则想着,怎么从韩冈那里做个交换。自己也有需要韩冈帮忙的地方,现在帮韩冈一把,之后就可以请他助一臂之力了。

不过这总要明白韩冈到底想要些什么才行。

韩冈点点头,“地理不明,不可以用兵,航路不明,连防守都难以做到。如果明了地理,那就简单了。以海洋之大,也不是何处都可以走的,礁石、浅滩,无处不在。而靠海的陆地,也多为荒滩,需要防守的港口就那么几处。”

“但通向港口的航路,不比关隘只有几条路出入,大海无垠无界,贼人从何处杀过来都可以!”

“相公误会了。”韩冈沉声说道,“韩冈的意思,是派水师到海对面的港口外巡检。出入港口时总归是一条道。若是做生意的海商,那就收了税后放过,若是载着兵将,那就直接送进海底!”

几名宰辅脸色齐齐一变,都是没想到韩冈竟然敢这么提议。听他的口气,这是要直接派舰船到高丽、甚至辽国的港口外巡视。

章惇张了张嘴,本欲说话,但心念一转,又不想开口了。曾布冷眼看着,也没有多话的意思。至于蔡确,现在已经不想掺和进这件事中,将水师派驻到高丽国,韩冈的计划已经超出了他愿意付出的代价。现在都已经是宰相了,所求的只是稳妥,而不是管勾右厢公事和监察御史时的进取。

倒是张璪,碎碎念道:“辽人当遣使诉我犯其疆界!”

“海上本无疆界,敢问界碑立在哪里?”韩冈笑了一下,“既然当年辽国能够出兵侵占西夏兴灵,那么我中国也可以在高丽海外占下几座有流水、有港口的岛屿,修起堡垒,驻兵巡海。”

韩冈胆大包天,韩绛有点吃不住了,“辽人驱使高丽海商,只是猜测而已。现在却要将水师驻扎到高丽,不是小题大做,而是草木皆兵”

“既然有这个可能,中国就不得不防。御敌于国门之外,总比在家门内守着要好。如果辽国当真能驱动贼人犯我大宋,只消三五条船,便能让江南风声鹤唳。但反过来,一支放在高丽边境上的水师,就能让辽国不敢轻举妄动。必要时还可以支持高丽。”

“远隔重洋,驻军如此之远,军心可能安?”

“再远也没有从开封到长安远。从开封出发,三五天内,能走到哪里?南京应天!西京河南!燕京大名!”

“海上行舟岂能与官道上一样走?又没有道路、建筑,方向偏了,也难以知晓。”

“相公顾虑得甚是。”韩冈附和着韩绛的话,却是在反驳,“常言道工欲善其事,必先利其器。必须要有航行于海上的利器,才能让人安心去探索新的航路,也能让水军安心巡守海疆。”

韩绛多看了韩冈几眼,沉下眼皮不搭话了。一说到利器,朝野内外都知道那是韩冈的本行。韩绛自是知道在殿上的没一个有资格跟韩冈就此事谈论的,自家也没必要赶着让这个后生蹬鼻子上脸。

注1:北宋与高丽交往早期,使臣多是从明州出发,最有名的两艘万石神舟,也是明州船场打造。直至元丰年间,来往渐多,海路更熟,才改成密州板桥港为始发地。到了哲宗元佑二年,密州的市舶司成立。而杭州、明州的市舶司,都是在开国初年就成立了。

注2:宋元以来我国航海者对于今黄海分别称之为黄水洋、青水洋、黑水洋。大致长江口附近一带海面含沙较多,水呈黄色,称为黄水洋;北纬34\ensuremath{^{\circ}}、东经122\ensuremath{^{\circ}}附近一带海水较浅,水呈绿色,称为青水洋;北纬32\ensuremath{^{\circ}}-36\ensuremath{^{\circ}}、东经123\ensuremath{^{\circ}}以东一带海水较深,水呈蓝色,称为黑水洋。

