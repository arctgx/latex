\section{第20章 土中骨石千载迷(15)}

“要王介甫管殷墟?”

“给均国公封王?!”

当王珪从崇政殿中带着天子的诏命回来,不论是蔡确还是韩缜,都大惊失色的异口同声发问。

不仅是因为天子想要将已经退隐的王安石从金陵拉出来,更重要的是天子另外还想要给均国公赵佣晋封王爵。

“官家怎么这么急,是不是均国公有什么不适?!”

蔡确尽可能的想维持宰辅重臣的沉稳,但话一出口,才发现自己已经紧张得变了腔调。

王安石远在金陵,暂时放一放再说,反正心急上火的应该是韩冈,还有外面那一干老臣。但均国公赵佣就不同了,突然提前封王,肯定是有哪里出问题了。

世间就有所谓的冲喜之法.皇帝这般心急,谁知道是不是这目下唯一的皇嗣又生了重病。就在夏天的时候,宫中和朝中可是为了皇子的病情,很是乱了一阵。

万一这一回病情更加严重,乃至于有什么不幸,继承皇嗣的可就是那位二大王了。

韩缜没有跟着追问,但也很是紧张的盯着王珪,纵然是世家子弟,哪一位做皇帝,对他的影响远小于蔡确,但这一份关切,终究是的免不了的。

不仅是两位参知政事,就是厅中的十几名官吏,也都屏住了呼吸。

“天子有意重开资善堂。”王珪稳当当的坐着,答非所问。

蔡确和韩缜闻言,一下都放松了下来。

两人能够身居高位,自然有足够的头脑来领会王珪的话中之意和皇帝做出这两项决定的用心。

“原来如此。”

“侍讲资善堂的人选倒是要好生挑选了。”

蔡确和韩缜各自点头说着。

天子一心袒护新学,彻头彻尾的要贯彻一道德的初衷。不过为了补偿韩冈,所以让他成为东宫官,直接与下一代的皇帝打交道,当然,更有让韩冈保护皇嗣安康的用意在。

虽然能够理解皇帝的打算,但蔡确和韩缜都明白,这件事情不会这么简单就落幕。毕竟天子动作太大了一点,而且对于普通的官员来说,要完全明了天子的用心,他们所了解的情报和信息还是太少了一些。

天子的吩咐,只要崇政殿中传扬出来,就算还没有转化成正式的诏令,也肯定能在京城中掀起了一场轩然大波。

不论是调动王安石去负责殷墟发掘,还是封均国公赵佣为郡王,任何一事都能让朝野内外瞬息间就变得动荡不安。而两件事所代表的意义,之后的影响和后果,都是让人担心不已。

均国公的事可以放心一点,但天子意欲让王安石重新出山的一桩事,又重新压在蔡确的心头。

蔡确深吸一口气,却发现胸口依然憋得厉害。

别的不说,先看看王安石的份量有多重?在西夏灭亡,变法.功效显而易见的现在,他一个人抵得过所有的元老重臣。

天子就是因为王安石的权柄和声望是在太重了,早早的就将他打发了出去,省得在军国之事上受其掣肘。

蔡确很清楚自己到底是为什么才能够上位。在熙宁四年的时候,在开封府中连个推官的位置都坐不上,五年的时候也不过一个新进的御史,但如今已经是东府中仅次于王珪的第二号人物,这晋升之速,也只比吕惠卿稍差一点,韩冈等人都得瞠乎其后。

说好听点是厚积薄发,在外任官十几年后,终于得到了一个机会,但自家的事自家最清楚,是因为自己顺了天子的心,才有了这一番的境遇。

王安石眼下是负责殷墟,但之后呢,谁敢保证他不会再一次卷土重来?王安石今年刚过花甲,对于一名为天子治理国家的宰相而言,这个年纪可以说是正当年,远远不到该退场的时候。

天子的两条诏命——如果将重开资善堂一并给算进来,那就是三条——就其本心而言,乃是针对目下的道统之争罢了,但对于蔡确来说,王安石重新出山的意义,却比什么都要重要。

他看了眼韩缜,又瞥了一下王珪,蔡确知道,天子打算请动王安石重新出山,这一条消息必然能牵扯不知多少人的心,能够让消息以更快的速度传播出去。

……………………

太常寺也在皇城之中,政事堂中的消息传到太常寺,只用了半个时辰而已。

判光禄寺的苏颂,每天都是到太常寺这边来报到。相对于几乎没有实际工作的光禄寺,《本草纲目》才是重中之重。苏颂本打算将今天要将之前的有关禾本科的几个条目着重修改一下,但被这一桩消息给打乱了计划,进度却是慢的可以。

一个积年的老吏在苏颂面前小心翼翼的将消息禀报,但苏颂依然能够维持住脸上的微笑。

“玉昆,这一回天子可算是苦心积虑了。”挥手让人退下去后,他冲着一脸事不关己的韩冈说道。

只要听到天子有意重启资善堂这一句,前面两条让人惊疑不定的新闻,便能够勾连联系起来。

皇帝的心思终究是还是在新学和皇嗣上,通过对皇子晋爵的安排,以及殷墟发掘的主持人选的安排,极其清晰明了的表达了心意。

“资善堂?”韩冈摇摇头,“还的确是出人意料。”

资善堂是皇子读书的地方,天子赵顼登基前,和他的两个兄弟就在资善堂中读书。在这之前,真宗也为仁宗开过资善堂。

不过英宗不是仁宗的亲儿子,直到最后一刻才被确立为嗣君,似乎就没有进过资善堂——韩冈不是很确定。对于朝廷故事,由于资历和家世的欠缺,韩冈在这方面算是一块短板。

他又说道,“但以均国公的年纪,未免太早了一点,揠苗助长可是不好。”

也许在皇帝眼中,这还真是妥协了。对一心想要用殷墟撼动新学地位的韩冈,先打了一个巴掌,然后不得不给颗甜枣来安抚。同时,给韩冈一个潜邸重臣的好处,就像挂在驴子前面的一束草,也可以让他考虑着日后在新朝秉政,眼下就少一点折腾。只是就一般的情况来说,以当今天子的年纪,这应该是二三十年后的事了:除了英宗之外,之前列位的大宋天子,都是活到了五十以后——尽管没有一个过花甲的。

能维持住新学的地位,能安抚韩冈,还能让他安心等待二三十年,这可算是一石数鸟的好主意了。

韩冈真还没想到赵顼敢将资善堂当做好处来安抚自己。这怎么能算是补偿?

赵顼终究是要他韩冈来保着如今唯一的皇子。留在未来的皇帝身边,在韩冈身为药王弟子的传言流播天下,并献上种痘法之后,就已经成为既定的事实。让韩冈去资善堂中为皇子授课,不是奖赏,反而应是一桩顺理成章的任命。

“天子的目的是要均国公就学吗?”苏颂笑着,声音更低了几分,“这一回,可是难得天子愿意退让一步。”

“只要天子没有明说,诏令还没有下来,这一事就不能确定,说不定到时候又会有什么变化。”

“变化应该不至于,宫里面也是希望看到玉昆你侍讲资善堂。”苏颂停了一下,又道,“可不只是资善堂,天子让令岳去主掌殷墟发掘,不是玉昆你最想看到的吗?”

韩冈点了点头,算是默认了。

不管皇帝到底是否是想拿资善堂当做补偿,仅仅是让王安石去负责发掘殷墟,韩冈都是乐见其成。这么做,便是格物致知。不穷于经,而本于实,这就是韩冈加诸在气学上的根本理念。王安石为了证明新学,却不得不按照气学的规则来做事,这自然是韩冈所乐见的。

苏颂轻叹一声,“而且玉昆你能为皇子授书,这一回,千里镜的禁令也可以收一收了。”

韩冈微微一笑,带着点得意。

尽管天子的算盘打得劈啪作响,让人很不舒服,对韩冈个人,也算不上是补偿。可不管怎么说,这一事对于气学而言的确是个好消息。

能成为皇子的老师,就代表他的学问得到天子的认同。而千里镜与气学紧密相连,之前气学因为禁令而受到的打击,这一回终于可以缓过气来了。

苏颂也算是能安心回去观察星象了。虽然上缴的那一具千里镜让人可惜,但钱财乃身外之物,再造一具也就是了。只要能夜观天象,身外之物丢了就丢了,也没什么大不了的。

“不过能看到这一步人不多。”苏颂对韩冈说道,“过两日市井中不是传言天子御体欠佳,就是均国公病重需要冲喜,所以这么心急的给均国公晋封郡王,顺便准备重新起用令岳来稳定朝局。”

赵顼准备让王安石重新出山,而且掌管殷墟发掘,当然,之后的解读也肯定是归于王安石管辖。天子对新学的支持,在这件事上,一点也没有掩饰的展示给世人。不过此事,乃是让普通人摸不着头脑的道统之争,仅在士林和官场起波澜,意识形态的问题,眼下还不至于波及到普通百姓。

但帝位的传承,却是关系到天下的每一个人。以苏颂对世间人心的了解,他很清楚,谣言必然会因第二件事而泛滥。

突然之间,将郡王之位赐给才及五岁的皇子赵佣,到底是意味着什么?会想到这是安抚韩冈的手段,世间又能有几人?

“谣言就谣言吧。”韩冈端起茶盏,很不在意的说着,“市井中的谣言哪一天都不曾缺,只要不去理它,终究会不攻自破。”

赵顼在六皇子身上下功夫到底是为什么,大部分朝臣多多少少的能猜到一点。不过对韩冈和气学的意义,也就寥寥数人能看得透。

韩冈没打算为此说些什么,还没有确认呢,就露了口风,未免太不稳重了,就是确认了,妄加评价,也是平添口舌。至于苏颂这边,完全不需要他的多嘴,自然会保持沉默。

轻抿了口茶水,韩冈的心情很好。

王安石主掌殷墟发掘也好,得以往资善堂侍讲也好,意义都是深远非常,气学及格物致知的理念,从此正式得到了官方的认可,也不枉费他这一段时间来的辛苦。

