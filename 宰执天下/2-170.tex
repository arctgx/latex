\section{第30章 肘腋萧墙暮色凉(五)}

回到驿馆,见了已经等得不耐烦的种建中,韩冈把今天的事一说,种建中也纳闷起来。是韩绛两次上书要调韩冈到延州,也就是说韩冈是韩绛征辟来的幕府属官,不是普通的官员。现在把韩冈晾在一边,韩绛等于是在说自己找错人了。

种建中觉得实在不对劲,他从种朴那里曾听说了韩绛不太喜欢韩冈,当时没放在心上,不过现在看来,好像倒是真的。不过韩绛看韩冈不顺眼,拖着不见人,但罗兀那里可是等着要人的,哪能这么拖延?

他站起身,对韩冈道:“愚兄先去赵宣判那里去为玉昆你打探一下。”

“赵宣判?……是赵禼赵公才?!”韩冈立刻追问道。

种建中点着头:“正是他!”

虽然在历史上赵禼名声不显,韩冈从来就没听说过——他也就知道王安石和司马光,还有灌树洞捞球的文彦博——不过在眼下的关西,赵禼赵公才这个名字可很是响亮。他稳稳做着陕西宣抚司宣抚判官一职,无论是早前的郭逵,还是现在的韩绛,哪一个上来任陕西宣抚使,都没有动摇到他的地位;或者说,都要用他为副手——就算赵禼一直对种谔的冒险之举私下里颇有微词,韩绛也只是当作没听到,而不是撤换他。

赵禼是当世少有的精通兵法的能臣,对兵事了若指掌,政务处理也是行家里手,宣抚司少了他,就立刻会运转不畅。韩绛的雄心,种谔的计划,没有赵禼来居中处理各项事务,一切都将是空话。

赵禼现在本官是右司谏,比起刚刚升官的王韶还要低半级。可王韶此时只能做个知通远军、秦州缘边安抚司安抚使,而赵禼却是陕西宣抚判官兼权发遣延州——也就是延州知州,仅仅是因为他本官太低,所以才冠以权发遣的名头。

虽然王韶之所以只能做着知军,是因为他这一年来升官太快,资序不够的缘故。但赵禼以七品官任职鄜延路首府的知州,又辅佐宰相韩绛为宣抚判官,这样的地位,全是靠他的在军事上出类拔萃的才能得来。

韩冈在秦凤,赵禼的名字已经听得很多了。王韶有几次提起他,虽然还是赞了许多,但韩冈也能从中看出,王韶有着瑜亮之争的心意在。

能让王韶都有瑜亮情节的当世英才,韩冈当然想见上一见。不过种建中却没有看出韩冈的心思,说着就匆匆出去了。

种建中一走,周南便从后间进了小厅中。轻蹙着眉,俏脸上尽是为韩冈担心的忧色:“官人,是不是有什么关碍?”

“不用担心,小事而已。京中的大风大浪都过来了,何必担心这些小事。”

韩冈将周南搂着坐在自己的腿上,贴着她的耳朵低声说着。透过丝棉的阻隔,韩冈还是感受到坐在大腿上的丰.臀的弹性,以及从中传来的热力。自从出了京城后,韩冈便是紧赶慢赶。走了一天的路,到了晚上,韩冈尚有些精力,但周南还是第一次长途旅行,经不起累,跟墨文都是沾了床就睡着了。这一路上,韩冈虽是拥美而行,却是连一次都没有沾身,已经馋了许久。嗅着周南身上刚刚沐浴过后的体香,韩冈的手便不老实的探入她怀中,将温香软玉一把握住,忽轻忽重的揉.捏起来。

周南刚刚破身不久,初尝滋味的少女,份外忍不得情郎的调情。韩冈只动了几下,她的脸色便殷红如血,浑身都没了气力。幸好还残存了一些理智,让她没有沦陷下去,娇.喘吁吁的用力按着韩冈探入衣襟的魔手,不让他乱动弹。轻叫着:“官人,不要啊……会有人来!”

韩冈知她初经人事不久,性子有些羞怯,也不想强迫她,何况种建中随时都可能回来,抽出手,搂住了她。周南顺势把头埋在韩冈的怀里,享受着片刻的缱绻。

不知过了多久,外面突然重重传来几声咳嗽,章惇荐到韩冈手上的钱明亮的浑家,在外面提声叫道,“机宜,种官人回来了。”

周南吓了一跳,连忙从韩冈怀里跳出来,匆匆跑进里间。

跨进厅来的种建中看到了周南的背影,却是视而不见,当作什么事也没发生的坐了下来。

可韩冈却叫着里间惊魂甫定的周南,“南娘,彝叔与我是兄弟一般,用不着避讳什么,你且端茶来。”

不同于普通人家,士大夫家的女眷一般是不见外客的。如果哪位士人向朋友介绍自己的家眷,就等于是把这位朋友当作亲戚家人一般,如此关系便称为通家之好。像韩冈当初在程颢、张戬家里,能与两家的家眷坐在一起吃饭,就是因为他深得两人的看重和喜爱,当作子侄辈一样看待。

周南听着韩冈的话,知道是把种建中当作兄弟。便端着煮好的茶,到了外厅来。向种建中屈膝福了一福,轻声道:“伯伯万福。”

种建中没想到韩冈随身带着的女眷竟然是一位倾城倾国的绝色佳丽。他摄于周南的艳容,明显的怔了一下。不过因为知道是韩冈的家眷,回过神来的他明白不能失仪,起身回了半礼,收摄心神也不再多看她。但当周南奉茶过来的时候,他还是显得很紧张,等到周南进了房中方才松懈下来。

抿了一口热茶,种建中也不免要艳羡的对韩冈道:“玉昆你真是好福气……”

韩冈微微一笑:“更重要的是她的一片真心。南娘为了小弟,可是拒绝了当今的雍王殿下……而小弟离京前,为了帮她脱籍,也在京里闹出了偌大的一团风波。到最后还是多亏了天子圣明,方才如愿以偿。”

种建中眨了几下眼睛,半天后才反应过来,惊叫道:“天子亲自下旨脱籍?!”

韩冈笑着点点头,很简略的把前阵子在京中发生的事,向种建中说了一通。

种建中越听越是惊讶,到最后,他神色郑重的对韩冈由衷说道,“玉昆你真是好福气!”

与之前同样的话,可内蕴的意义已经完全不同。

“说得没错!”韩冈点着头,感慨着。

虽然心知种建中站在自己的一边,但韩冈还是用了点心机。他这是用天子来压人,压种建中身后的种谔——周南的事,种家的十九哥肯定会传给他的叔叔听——皇帝把弟弟看上了女人送给韩冈,虽然是有着两情相悦的因素,但也能从中看出天子对韩冈的重视——韩绛很了不起吗,天子还在那里呢!

种建中并不清楚韩冈的想法,只是为了韩冈让家眷出来拜见,而感到亲近了许多。他又提起正事:“方才愚兄去见了赵宣判。问了半天,才听他说韩相公是为了要磨一磨你的性子。”

“磨我的性子?!”韩冈皱眉问道。他何时表现的桀骜不驯,让韩绛需要如此做?不过可以确认,韩绛尚不知道他在王安石府上说的那些话,否则就不是磨性子来。

韩冈仔细回想,却始终也不想出来。当然,他就算想破头,也不可能想得到是因为他前次过长安,没有去拜访韩绛的缘故。韩绛韩子华,从来都不是以宽宏大量著称于世。

韩冈想不出缘由,并不代表他不知道该怎么应对。要让韩绛放弃他那愚蠢的念头,韩冈还是有些招数的。他先向种建中道谢:“多承彝叔的人情。”

“玉昆你哪儿的话。同门之谊,通家之好,有这两份因缘在,帮这点小忙,也不能算是人情。”种建中摇头表示自己实在不敢当,“玉昆你现在还是先想想该怎么办吧,总不能真的要熬个十天半个月?”

“放心,小弟自有主张。”韩冈笑得胸有成竹。

第二天,韩冈带着本《孟子》去了帅府行辕。虽然《孟子》一书并不在九经之中,但王安石是崇孟的,三年……不,是两年后的科举考题,答案须从思孟学派——子思、孟子——的理论出来。

门房已经不像昨日听到韩冈通名时那般殷勤了,接过韩冈名帖的时候神态也有了几分倨傲。

韩冈也没当回事,进了门厅后,找了个座位坐下。便打开书卷,自顾自的轻声诵读起来。进来的官员都惊讶的看着韩冈,闹不清他在搞什么名堂。

开始的情况还跟昨日一样,还一个个官员被领进去,继而又放出来,只留着韩冈一人在门厅中。不过韩冈对此都视而不见,照样读着书。

亲自向天子求来的人才,却被晾在门厅中枯坐读书,这件事韩绛敢让天子知道?!

正如韩冈后世听过的一句俗语,狗咬人不是新闻,人咬狗才是。韩冈被韩绛晾在一边,这不是什么稀罕事,但他在帅府的门厅中读书,却是能让人有兴趣传播开的趣闻轶事。

‘我奈何不了你,但我不能恶心你吗?’韩冈倒要看看韩绛到底能不能坐得住!‘我只怕事情闹不大!’

半个时辰后,韩绛终于把韩冈请进了待客的偏厅中。

大宋的首相盯着一脸无辜的韩冈好半天,最后有些无奈的叹道:“玉昆当真是苦学之士啊!”

“相公之赞,下官愧不敢当。欧阳永叔曾有言,读书当是马上、枕上、厕上,下官只是闲来无事,抽空而已。”韩冈恭恭敬敬的回答,却把韩绛心口堵得一阵发闷。

