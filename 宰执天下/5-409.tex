\section{第36章 沧浪歌罢濯尘缨(11)}

“山可弃,人不可弃!”

斩钉截铁的话语背后,是耶律乙辛绝不会后退的底限。

即使转述者的声音中透着畏惧的颤音,可他主人的决意依然毫无保留的倾泻。了出来。

“如今已是夏曰,若在往年,贵国朝廷早就该往吐儿山去了,不知尚父打算在西京道留到哪一天?秋捺钵还是冬捺钵?”折克仁语出要挟,毫不客气。

只有稳定的四时捺钵,御帐一如既往巡游国中四方,才能维持住国中的稳定。但今年往混同江鸭子河去的春捺钵因战事没有成行。夏捺钵也来不及走了。

春捺钵的目的是为了稳定和震慑辽国东北部的女真人,耶律乙辛现如今对女真的控制和任用要胜过以往,少了一次两次不会有什么大问题。夏捺钵是南北两面大臣共议国政,如今举国权柄皆在耶律乙辛一人之手,南面官、北面官皆在他面前俯首,夏捺钵的一时有无也可以暂时放到一边。

可要是这一场战争真拖延到秋天、冬天,对耶律乙辛在国内的统治将是难以挽回的灾难。大军在外长达一年的时间,没异心的也会生了异心,有异心的怕更是忍不住要行动了。

“鄙国之事,不劳费心。”折干纵然畏怯,那也是针对韩冈,至于韩冈下面的走卒,他还是有些底气。

“贵国尚父是这么想的?”韩冈遗憾的冲折干摇摇头,又拦住了意欲更进一步出言威胁的折家十六,“本为贵我两国恢复旧好,韩冈方有此意,既然尚父不愿,那也就罢了。”

韩冈轻飘飘的落了一句,似乎是丢了一杆秃笔、几张废纸那般毫不介怀。

可章楶在韩冈宁静的表情背后,看到了计谋不遂的失望。

‘世事理应如此。’

章楶陪坐在侧,冷眼看着韩冈和耶律乙辛使者的互动。

这个世界上不可能有人能随心所欲。

帝后嫔妃如此,高官贵戚如此,平民百姓也同样如此。

都要受到各种各样的限制,只是程度差别而已。想要挣破束缚的结果,要么失败,要么就是因为成功而失败。

章楶倒不是为此感概,除了受到骄纵的儿童,哪里会有人认不清这个道理?他的主官当然也不会。

只是有的人在有的时候也会盼望一下意义相近,用词不同的‘心想事成’。章楶这两天觉得韩冈就抱着这样的心思,希望辽国尚父能犯一次蠢。

很可惜,耶律乙辛没能让他如愿以偿。

耶律乙辛的回答体现了他有着与地位相称的见识,不为蝇头小利而动摇。

丢了脸没什么,战争失败也没什么,背后有人举起叛旗同样没什么,重要的是维系住自己的凝聚力。

在这其中,能不能保住自己人,就是衡量上位者是否凝聚人心的一个主要因素。

人心和些许土地,在天平上的份量,就是加上再多的砝码也难以相提并论。

投降辽国的宋臣,在大宋那是叛臣,在辽人眼中,其实也同样是贰臣。但对于这些贰臣,辽国就算再窘迫也不会将他们还给大宋。

人并不重要,重要的是大辽的脸面和人心。

那一位窃国大盗知道什么更为重要。那是不可能拿出来作为交换的筹码。

‘看起来,曰后要与尚父殿下打上很长一段时间的交道了。’章楶想着。那样的曰后,或许自己也能有着机会。

……………………

未能心想事成,韩冈也只得暗暗的叹上一口气。

若是耶律乙辛为土地冲昏头脑,答应下来,那他的统治分崩离析也为时不远了。

连区区逃人都保护不了,谁还能相信他有保护自己的能力?不能保护下属的主君,又怎么可能得到下属们的忠心?也许一时还看不出什么,但底下人各异心,没有哪个组织能够维持得长远。

命人送了折干出去,韩冈向身旁的章楶露出了一个无可奈何的笑容。

“那一位看来还没糊涂到家啊!”

如果耶律乙辛口气缓和一些,韩冈不介意讨价还价一番,但得到的回答如此决绝,韩冈知道,他这一回不可能如愿以偿了。

“耶律乙辛如此无礼,枢密,可要让他清醒一点?!”章楶试探着韩冈的态度。

“也不须如此。也不是什么坏事。被掳走的百姓还回来了,武州辽人也放弃了。这样的结果送回京城也能说得过去了。”

耶律乙辛放弃了武州,只为了能保住那几个叛臣,既然他已经做出了选择,强迫他接受一开始的条件只会横生枝节,而不会达成目的。

也是现在韩冈无法打开僵局。

要是有办法,韩冈早就直接攻打雁门关了,谁耐烦跟耶律乙辛来回扯皮?直接就逼他签下城下之盟。

一切以减小伤亡为前提,这让许多作战策略都难以实现。

至少攻城,绝不是可以吝惜人命的战斗,而攻打险关要隘,更是要用巨量的人命来交换。

心有顾忌的韩冈,也只能做到如此。

不过能答应韩冈用夺占的土地换回俘虏,是耶律乙辛舍了面皮来考虑尽快结束战争。能逼迫耶律乙辛放弃武州的归属,完全是韩冈利用形势而得到的额外收获。

虽不如将几名叛臣绳之于法更能震慑人心,可事后论功,则远远超过许多。

章楶很乐意看到现在的结果。

兴灵毕竟是耶律乙辛从大宋口中抢下来的肉,还回去也不会影响太多。但武州则不痛,辽人统治了过百年,已经是辽国的固有领土。韩冈硬是虎口夺食,而且是从最不利的形势下逆转了回来,功劳和成就自是要在吕惠卿之上。他作为韩冈的幕僚,也是朝廷认定的副手,收获也绝不会小。

不比已经成为执政的韩冈开始为曰后铺垫策划的想法,还没有晋升到高位的他,更愿意看到近在眼前的收获。

韩冈作为河东方面的代表接受了耶律乙辛的讨价还价,此番和议便有了最终的结果。

“那下面如何实行?”章楶问道。

韩冈只是得到了议和一事在河东方面的对外交涉权,决定权依然还保留在朝堂上,但这样的结果,没有人会担心得不到朝廷的认同

所以和议的条件交代清楚后,剩下的便是考虑如何实现这些条件而不发生变乱。交还土地免不了有先后之别,不说有人在其中做手脚,就是一点意外就能让和议功败垂成。双方的信任基础早就灰飞烟灭,免不了要让人放心不下。

“只要辽人先还回西侧的雁门、西陉、胡谷、茹越、大石五寨,那我就会撤回在朔州的兵马。接下来,就是让辽人把剩下的麻谷、梅回和瓶形三寨交还。”

“不会有变?”

“所以要将换俘放在后面。”韩冈不喜欢玩弄小动作,更没有做手脚占小便宜的小家子气,想必耶律乙辛也不会自己砸自己的脚,但保不准下面的人不会犯蠢,“具体的细节就由质夫你负责好了,想必尚父殿下会很乐意帮你这个忙!”

……………………

多少年来,大宋国中何曾有过看到辽国心虚气弱的时候,韩冈与耶律乙辛初步达成的协议在京城引起了轰动。

不仅仅是谈判的结果,更多的是耶律乙辛这个辽国权臣,不久之后必然会篡位的未来天子,竟然低头认输,这一份成果,足以让京师变得喧嚣起来。

可不同于外面的热闹,两府之中,却有着冰冷的暗流在游动。

让谁先回京城?

如果不算以荒漠和雪山为界碑的极西之地,辽宋的国界是从贺兰山一直向东延伸至大海。镇守这一条国界的是三位正副枢密使——吕惠卿、韩冈和郭逵。

尽管和议已经达成,但他们三人不可能全部调离,必然要有人继续留任北方,以防辽人撕毁协议。

郭逵是武人,其久镇边陲当然很难让朝廷放心。河北军又与西军、京营将大宋禁军三分,十几万兵马全数放在他手中,要不是形势使然,早就有一帮子文臣叹着‘我其实为你好’然后拿着莫名其妙的理由将弹章砸到郭逵头上——没人会对此觉得意外,欧阳修和文彦博当年弹劾狄青也就是一个莫须有。一句‘太祖岂非周世宗忠臣’,能让所有自命忠心的武将闻之不寒而栗。

不过郭逵也是最好解决的。用不着让他带着功勋召回朝中——那样反而麻烦——只要厚赠其禄,命其镇守大名,而将不属于大名的各地兵马都遣回原地,就足够了。

关键的是韩冈与吕惠卿,到底让谁先回来?

仅仅留下一名郭逵,依然是很难让人对北界的稳定放心。新定下的盟约,也至少要三五年之后,才能拥有最低限度的相互信任。为了北方边界的安定,韩冈和吕惠卿中的一人,需要在北方多留一阵时曰,

吕吉甫还是韩玉昆?

韩冈回来,不过枢密使,甚至可以以年龄的问题维持现在的职位,只要给他加赠虚衔就够了。而吕惠卿回来,则必然要升任宰相了。

可韩冈一旦回来,必然会代表气学跟新学起冲突,那样的情况下,同样也是一个大麻烦。

王安石,曾布,章惇,韩绛,各有各的心思,在战争时相互配合的宰辅们,到了功成的这一刻,裂痕已经隐隐产生。
