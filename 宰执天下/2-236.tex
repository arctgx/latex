\section{第35章 重峦千障望余雪(一)}

天气一天比一天更冷,洮水已经完全冻结,厚达尺许的冰面,只要不是奔马而过,基本上就不会有落水的危险。

但王韶还是没有立刻下令麾下大军立刻跨过洮水,临洮城还有最后一点才能完全修筑完毕,扼守南北通路的南关堡和北关堡,也得再过七八天方能竣工。

如果全军出动,攻打洮水西岸的那座同时在修筑的城寨,万一北方来敌,纵使攻不下完工在即的临洮,万一伤到了民伕也是不好向上交代的一桩麻烦事。

这一日,王韶暂且将临洮之事放在一边,带着韩冈,沿路往南面行去。在他们的身边,有着赵隆率领四百通远选锋护卫,在已经被如狼似虎的大宋官军清理了一遍的道路上,根本不需要担心太多的安全问题。

经过抹邦山,道路渐渐宽阔起来,左近的竹牛岭山势虽然高峻,但并不影响只在山下河边经过的道路。这条路直通渭源,除了少数几处外,地势也都算得上平缓,远非北线经过鸟鼠山的那条道路可比。

王韶悠闲的坐在马背上,抬头望着竹牛岭被积雪覆盖的峰峦,又低头看看前方的坦途,对韩冈道:“若不是今次兵雄将勇,钱粮充裕,当自此路缓进,引瞎吴叱、木征等辈越抹邦山来此对阵。”

“而后再遣一军由鸟鼠山直取临洮?”韩冈问道。

“呵呵。”王韶笑了两声,道,“若不能以势压人,也只有依仗计策了。”

韩冈道:“还是正面制敌更稳妥点。”

“计策伤神,而且太险,不如泰山压顶来得痛快。”王韶也同意韩冈的说法,“一个不好,就是瞎吴叱兄弟在渭源堡的结果。”他又问韩冈,“玉昆,你觉得这条路如何?”

抹邦山向南便是竹牛岭,绕过竹牛岭向东,可通往渭源堡,也即是前日瞎吴叱、结吴延征两兄弟偷袭渭源堡的那条路——之所以临洮—渭源的南线要绕个马蹄形的大弯,就是因为竹牛岭的阻碍——而在竹牛岭西侧向南,就是直通岷州的道路。

“的确比鸟鼠山好走,就是绕得圈子大了点。”

王韶提醒着:“但此地还通岷州。”

“若欲定岷州,竹牛岭下必得设立一处寨堡。最好就在刚才经过的那个地方。”韩冈回头指着了过来的道路上,变得狭窄崎岖的那一段,“光靠北关堡驻军来扼守此路,实在有些吃力。”

“由谁来守?”王韶反问道。

“招募蕃军弓箭手一个指挥如何?”韩冈知道岷州的钱监在明年之前不会开张,没必要在此分心太多,对于不太重要的寨堡,使用可以信任的蕃人,比驻屯官军更方便,“护翼寨堡可以直接用包约的人,那样只要堡中放上一百官军就够了。毕竟不是主道,而且北面还有北关堡的驻军,随时可以支援。”

“……还是两个指挥比较好。竹牛岭东西两侧都要设立一个寨堡,省得有人再偷袭渭源。”王韶说着。

行了几步,忽然又问道:“玉昆,如果我推荐你来镇守武胜军,你愿不愿意?”

……………………

崇政殿中每日惯例的议事,不同于朝会时的按部就班。军国大事,都是由此而发。国事争论,基本上都是在崇政殿,而不是文德殿中发生。

文彦博正在喘气,毕竟年纪大了,吵起架来,毕竟不如殿中的其他年轻人。幸好王珪、吴充、冯京这些新进执政,都跟王安石不是一条心,这让文彦博终于有了喘气的机会。

但前一番争议,他终究还是输了。

判司农寺曾布,日前奉旨巡视京畿诸路免役法和农田水利的推行情况,不想他却带回来一封郑州的百姓联名上请的奏文。请求废州为县,也就是把郑州给废掉,只剩县治。

去掉了州府,对百姓们来说,就少了一个剥皮的衙门——一年省去几十万贯的税赋,省州官十余员,郑州州役省四百余人——而且,郑州紧邻京畿,一旦废州改县,必然归入开封府管辖。相对于郑州这等工役频繁、赋税繁重的小州,开封府连免役钱都会减少许多,州中吏民得享的便利为数甚多。

只是郑州紧邻开封,旧党势力盘根错节,州中官员多为旧党党羽,新法施行不便的奏章,郑州州衙没有少递过。一旦郑州被废,对于旧党不啻又是一个巨大的打击。

今天先是文彦博站出来横加反对,然后便是王安石跟参知政事的吴充争论了一通,两个亲家在朝堂很是斗了几句嘴,吴充连脖子下的瘤子都涨红了。

不过,因为同在京畿附近的滑州的吏民,在听到了郑州要废州改县的消息后,也上书申请同样的待遇。当王安石拿出这封奏章后,赵顼便下了决心,也宣告了文彦博和吴充的失败。

郑州被废置,以管城、新郑二县隶开封府——降原武县为镇,并入阳武;降荥阳、荥泽二县为镇,并入管城——同时废滑州,以白马、韦城、胙城三县并隶开封府。

开封府地界整整大了一圈,而郑州和滑州两州官衙中,少了二十多名官员的编制。大约十名左右旧党中坚必须开始等待新的官阙,这也难怪让文彦博气得直喘气。

当然,要把废置二州说成是政治.斗争就未免太小瞧王安石的心胸了。他的目的是撤并天下州县,裁减冗官,节省民力和费用,郑州和滑州仅仅是个开始而已。大宋天下四百军州,两千余县,要合并裁撤的地方还得很。

有人说他王安石只懂开源,可王安石用事实证明,他节流的本事更大。再过几日,他就准备把手伸到文彦博的地盘上,提议裁撤整编厢军。

王安石的变法计划不仅仅局限于财计,军事和政治区划,而是涉及到国政的方方面面——也包括教育。方才商议的议题,便是变革旧日的教育之制。昨日他上书天子,改建国子监旧舍,扩大国子监的招生范围,在天下州县,设立州学、县学。并将国子监分为三级,外舍、内舍和上舍。

在县学、州学学习后的士子们,通过推荐考试,进入国子监学习。一步步的从外舍升到内舍,再从内舍升到上舍。在王安石的计划中,到了日后,就是如今的进士科举也要废除,而是改用通过国子监学习升入上舍的学生为进士。

正如他旧日所言,治国之要,便是‘一道德’,让朝中官员。若处江湖之远,那就任你非毁指斥,身居庙堂之上,就必须遵循朝廷国是。最近他正在整理过往文稿,要把他毕生的学术做个总结,对儒家经传重新释义,希望能成为国子监教学的依据。

‘不过还得慢慢来。’王安石想着,‘至少还得两年到三年的时间。’

王安石神思一阵恍惚,惊醒过来时,便发现崇政殿上的议题,现在已经讨论到王韶和高遵裕刚刚送到的一封奏报上。

半个月前,临洮和渭源两边接连传回捷报,让赵顼兴奋不已,而昨日,王韶和高遵裕联名上奏,声称岷州多铁,若朝廷设立钱监,一年出产当有四十万贯,请朝廷速调派工匠五百,设监铸钱,以佐河湟之用。

“但凡工匠起屋,事前皆是信誓旦旦,说工省价廉。等到桩基建起,无不坐地起价。”文彦博大概是歇好了,养足气,再次站了出来,“王韶此举,不过工匠故技。”

以文彦博的老辣,怎么会给王韶和高遵裕骗过?直接把他们的小心思给捅出来了。虽然没有明着要钱要粮,只是要人而已,但实际上,能不给钱粮吗?等人派过去,准备设立钱监,立刻就会伸手要钱。

可缘边安抚司的用意,赵顼和王安石他们何尝不清楚。只要王韶不是无中生有的欺君,设法挤出一点钱粮拨给他,也无关大碍。韩绛在宣抚陕西的时候,也没少用各种借口,从赵顼的口袋里掏钱,还不是照样给了。

“比起横山的六百万,河湟的几十万不为多。”赵顼说着。王韶一出手就有回报,当然要多投些费用进去。比起横山让他郁闷数月的情况,还是河湟更能带给他好心情。

要是广锐军不是给自家添乱,能像他们在渭源堡表现得那般出色,罗兀城如何会得而复失?

赵顼这些天来,越想越是恼火。已经成了实边流犯的广锐军士卒,他们的表现实在让赵顼听着窝心。

那个刘源,名不见经传,旧时只是一个指挥使而已,偏偏敢带着三百战马都配不齐的士卒,夜袭数倍于己的敌军。这份胆色,与三国时,百骑劫营的甘宁也不差多少。怎么就能让他成了叛贼呢?!

韩绛的确坏事!

一开始他还认为是韩绛运气不好,可现在,觉得韩绛坏事的想法却是渐渐坚定。战死的王文谅是忠臣,造反的吴逵则是逼不得已,既然两人都情有可原,那真正有过错的,就是御下不谨的韩绛。

‘唉,一国宰相,用人的手段竟然连一个选人都比不上。’

赵顼觉得自己真的使用错人了。

