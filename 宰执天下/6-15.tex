\section{第二章 天危欲倾何敬恭(12)}

大祥。

名义上天子驾崩的两周年纪念,以日代月的祭礼之日。

在京文武百官,全都在持续了不知多久的仪式上,耗去了大半气力,冬日的寒风又顺便带去了身上的大半热量。

就是身处停灵的大殿中,韩冈依然感到寒气逼人。

他同情的望了一眼站在殿门外的那群低品官员,殿中空间有限,就只能委屈他们了。

王厚和李信两人都是正从七品的诸司使,倒是能站在殿中,不过几乎就在门口,而且是靠后一排,都快要贴上在殿中的班直了。那个位置,有前面的人挡着风,反倒是应当比韩冈这边通着殿门更暖和一点。

幸好在所有人都被冻僵之前,大祥终于是结束了。

在朝臣们的脸上,都能看得出隐藏不住的如释重负。真要说起对大行皇帝的悼念,还真的没有几人。其实也跟太皇太后差不多。

不过在群臣祭奠结束之后,就轮到命妇们出场,就是王旖也得入宫祭奠,如此才能算是大祥的结束。

不知道这一回,高太皇太后会不会来。

前些天的小祥,王旖也入宫了。

回来后对韩冈说,太皇太后只出来了一次。

被软禁在宫中多日,在亲生儿子的灵位前,太皇太后就是连一滴眼泪都没有。

韩冈听王旖回来说,她当时离得太皇太后的位置稍远,并非是能看见她脸上的泪水。只是命妇们要表现自己挂念赵顼,对前代天子的依依不舍之情,都至少会在手上拿着一面汗巾用来擦眼泪,皇太后手中也拿着的,但太皇太后的手上却什么都没有,自然更不会抚棺痛哭之类的表演。

尽管是亲生儿子死了,却连表面文章都不做,可见她对赵顼的心结了。

离得近的命妇们全都当没看到,离得远的,要么看不清楚,看得清楚的也一样会当做什么都没有。

高太皇太后在民间的口碑,还不如走街串户的尼姑。她现在这个态度,也不会让她的名声更坏一点了。

是不是当初的那一句皇后权同听政,让母子之情烟消云散?还是为最喜欢的次子抱不平?

韩冈这几日闲极无聊的猜测着。虽然他没有了解太多,但实际上赵顼母子之间的感情淡薄,早在赵顼发病前就已经不是秘密。反倒是前两年去世的曹太皇,赵顼跟她的感情很深。

不过无论如何,牛心左性、性情刚硬的高太皇,终究还是太上皇的亲生母亲。她就是如此态度,也不可能加罪于她。

不知道赵颢又是什么样的心情?

看到压在他头上的兄长,才三十多就亡故,是不是暗喜在心。活得长久,就可以在对方的坟头上大笑了。如果他是抱着这样的想法,那倒是好了。

不过想来赵颢不至于有这么浅薄。能疯上一年多,不会这么自欺欺人。只是他还能有什么手段改变现在的一切?向太后一日临朝,他就一日没有翻身的机会。

难道高太后还能翻身?

真要说起来,宫里面的气氛是有些不对。韩冈心中也有数,总有些人想要改变,机会难得啊,但他们能做的很有限。

向太后控制宫中已经一年多了,该换的人都换了,太皇太后成事的几率可不大。

至于宫外。

动武是笑话,聪明的武将都不会插手皇家之事,就是有拥立之功,也会被文官铲除,当然,也不能指望他们会出面反对,只会保持中立。但过年的这段时间,李信和他手下的炮兵们都在城内的火器局内,表兄弟之间,韩冈还是能够信任他的。

而文官那边,只要没有宰相和枢密使出马,参知政事和西府副职们就算做事来了,他们也能给翻过来。

韩绛那边有王安石压着。章敦为人果决,但他真的想要做什么,应该还会再来通一下气,之前自己可没把话说死,韩冈对章敦还是比较了解的。

就是蔡确的心思不定,之前去他家拜访过,可韩冈对这位宰相还是没有把握。拖过这几日,马上就能稳住了。

有一个共同的敌人才是好事,维持一定的危机感,才能让大臣们齐心合力将皇帝变成垂拱而治的‘圣君’。祸福之间,是没有定数的。

只要再有几天。

接下来便能除服,算是天子的丧期过去了,百官也不用再持丧。脱下了素色丧服,换上了淡色的惨服,虽然这也是丧服的一种,不过至少不是满眼白了。

不过在宫内,太后、小皇帝还要为熙宗皇帝持心丧三年,禁绝宴乐。见外臣时,一切如常,宫宴照样要开。可在内宫里,则就必须是做出一个守孝的姿态,得等正式的丧期结束才行。

朝臣们依序离殿,下了台陛便散了开来。

韩冈与苏颂一路。

“玉昆,”苏颂走着,问道,“这一期《自然》的稿子好了没有。”

“这边才三篇能看的,其他都不行。不过有一篇不错,说钱塘潮的原理的。是日、月的引力所致,还有钱塘江口的地势的缘故。”

“玉昆你觉得他说得对?”

“没去过两浙,更没看过钱塘潮,那边的地势一点也不清楚。”韩冈其实去过,甚至还亲眼见识过八月十八钱塘潮,“不过海潮是日月所引,这点倒是没错,地势的原因也能说得通,看起来是有些道理。就算有错也没关系,大胆假设,小心求证,要允许犯错误的。”韩冈笑着,“子容兄你那边呢?”

苏颂点点头,“也有两篇挺不错。一个是说北辰的角度不正,并不是正北。”

“沈存中已经说过了吧?”

发现北极星角度不正,在这个时代,不止沈括一个,很多人都有这个认识。

“但这一篇说得更清楚一点。”

“哦。另一篇呢?”韩冈又问道。

“另一篇是议论金星、水星哪一颗更靠内。”

“哪一颗?”

“当然是水星。金星容易看到,水星却难得多。”

“真够简单的。”

“文章中没那么简单。对了,通讯会员……”提起韩冈生造出来的新词,苏颂还是觉得拗口,顿了顿,“通讯会员他们定的份要一本本的发出去,送到的时候也不能比送去书坊要迟,这是要提早发啊。”

“这些杂务就让下面的人去操心吧。”韩冈笑道,“子容兄你别太操心了。”

“倒也是。”苏颂笑了笑。

自然书社虽然是韩冈、苏颂,还有沈括担任审稿,但下面还是雇了编辑、书办、杂役,拢拢总总十几人,琐碎的杂务还是交给那些人去做。

比如文印,制版,发卖,现在又包括了通讯会员的登记。

所谓通讯会员,是新设立的自然学会的成员。而《自然》,就是自然学会的会刊。订购全年的《自然》,便能成为当年自然学会的通讯会员。想成为正式成员,则必须有超过三篇论文在期刊上发表才行。

一旦成为正式成员,便能够得到一枚徽章和一份证书,同时不用再订购期刊,直接由学会免费寄送。等到正式成员多了之后,就开始选举会首,将自然学会正规化,以便传承下去。

本来苏颂是想将会员的标识做成是腰牌的外形,不过韩冈觉得还是别在襟口更为显眼,也更别致一点。苏颂对韩冈这种奇怪的审美观无话可说,他也没有争执的兴趣,系在腰带上,还是别在胸前,他都是无所谓的。

苏颂抬头望着天空,干净得没有一丝云翳,“今天天气好,得早点回去。”

他那具当做宝贝的望远镜,刚刚更换了反射镜片,这两天正在调试。昨日轮值,宿卫宫中,念着家里的望远镜,苏颂的心里如猫儿挠着。

京城的冬天,清明的天空不多见。这段时间夜中,而石炭的消耗也节省了不少,让天空也变得更干净了一点。正是观星的好时候。天上的星辰移动从来都是不等人的,错过一日,可就要耽误不少时间。

“的确得早点回去。”韩冈也抬头看了看天,转头对苏颂道,“犯了宵禁也不好。”

这段时间,开封城中一直都在宵禁中。丧期禁乐,管制也严格,现在丧期算是结束了,可禁令要三个月出头,才过去了十分之一。不过严禁闲人夜行的宵禁,则没几天了。再拖些天,京师中不知有多人要饿死了。

“子容,玉昆。”

听到身后有人唤,韩冈和苏颂回头,却见是曾布。

“子宣兄。”

苏颂则惊讶道:“今日不是子宣和薛师正宿直吗?怎么要回去了。”

“不,方才在殿上冷得够呛,得多走两步,绕回去。”曾布有些惊讶的样子,看韩冈,“是玉昆说过的吧,受冻了不能立刻烤火,必须将血脉活动开才好。”

“啊……是有这回事。”韩冈点点头。

曾布又道:“薛师正找了王厚过去。王厚那个新任的副都承旨兼西上阁门使,可能枢密院有事要先交代给他。过一会儿才会出来,玉昆你今天要请他喝酒,得拖一阵子了。”

“现在可不敢请喝酒,只能一杯清茶为贺了。”韩冈笑着。

王厚的职位刚刚定了,他将会留在京城,担任枢密院副都承旨,兼西上阁门使。

枢密院都承旨是西府的大管家,上承诸位枢密使,下接枢密院二十四房,地位极高。当初韩冈任同群牧使的时候,韩缜便是都承旨兼群牧使。纵然都承旨的副职远比正职的地位要低,可终究是有实权的职位。

这是个很不错的差事,甚至可以说很好。不说任官西府的多少好处,能进入中枢,就代表他日后的任官方向也将包括中枢,不会局限于边疆。多了发展的空间,自是值得庆祝的好事。

不过更重要的是阁门使,这是在皇城中插上一根钉子。

“好了,不耽搁两位了。”曾布告辞。

“那今天晚上就要劳烦子宣了。”苏颂道。

“算不上。”曾布笑道,点了点头,先行离开。

苏颂也往前走,走了两步,却不见韩冈跟上来,回过头:“玉昆?”

“啊,没事。”

韩冈摇摇头,压下心中的一股异样感,快步追上,与苏颂并肩出了皇城。

……………………

夜色渐浓。

苏轼睡得正沉。

若是在过去,才二更天过一点,正是兴致最高的时候,不过现在他好些日子没有去饮宴取乐,每日都是早睡早起,虽然说没了玩乐,精神反倒旺健了起来。

“舍人!舍人!”

身旁的侍婢推着苏轼沉重的身子,将他从梦乡中唤醒。

“还没天亮吧。”苏轼缓缓张开眼皮,眼前只有黑沉沉的床帐。

“舍人,是宫里面来人了!”朝云急促的说着。

外间同时传来了王闰之焦急的声音,“官人,宫里面来人了,要官人速速入宫!”

“怎么一点动静都没有?”

上一回苏轼被换入宫中,敲门如拆屋,将宅院中上上下下都给惊动了,可这一回动静却好像小了许多。

苏轼坐了起来,让朝云帮着整理穿戴,笑着说:“旧日曾问包孝肃日审阳、夜审阴,夜里唤人,这是哪里要我去写文章?”

“官人!”

王闰之在外面焦急的催促着,等到苏轼不紧不慢出来,又催着他往前面。

这一回来通知苏轼的宫人,不是上一次的那个,很陌生的一名小黄门,还带了四名班直护卫,见了苏轼,就急匆匆的催促着:“苏舍人。请速速入宫。”

苏轼不慌不忙:“宫里出了何事?太后可有何吩咐?”

小黄门闭口不答,只是在说:“请舍人速速入宫。”

“果然如此。”

苏轼的声音不大,却正好让周围人听到。

一切尽如所料。

废立天子?这肯定是废立天子!

就跟上一次通报太上皇死因一样,提前通知在京重臣入宫,以防生变。否则又有什么事才需要他这个中书舍人连夜入宫。

皇帝弑父,不论从哪一条上,都不应该再继续坐在天子的位置上。

弑父之君,岂可为天下主?

也就是韩冈这样有私心的大臣,才会硬是帮他遮掩。王安石、程颢、韩冈,都号为大儒,却罔顾大义,做了太子师,就把圣人传下来的道理给忘了,日后看他们怎么还有脸拿着《春秋》教徒弟?

也别说日后了,现在都已经是挡不住。也不知是两府中哪一位挑头出面的?

苏轼没多耽搁,等到下人将马匹备好,便飞快的上马出门。

离开了家门,很快就转上了大道。

比起上一回,因为火灾而萧条的街道,天子丧期中的禁令让街道更为冷清,除了值夜的巡城,就看不到其他人。

等上了御街,两百步宽的大街上,依然冷冷清清,看不到其他入宫的官员队列。

苏轼这时候却纳闷起来,怎么不见其他人?

……………………

王旖醒过来的时候,发现枕边人不在床上。

在床上坐起来,才发现韩冈正站在窗前,窗帘被他拉开了,沉默的望着屋外的夜色。

“官人?怎么了?”王旖拥被而起。

“不。没什么?”

韩冈摇摇头,依然静静的望着外面,“没事的……没事。”

