\section{第26章 当潮立马夜弯弓(上)}

……………………

张璪摆脱了失落,正在为韩冈出任参知政事的诏书奋笔疾书。

赵顼静静的等待着韩冈的回答。

“臣不敢奉诏!”

清朗却又决绝的声音,打碎了寝殿内的寂静。

韩冈在说什么?!这时候还玩欲拒还迎的把戏!?

连赵颢都瞪大了眼。三辞三让的旧例,难道韩冈当真准备一丝不苟的按流程做完?

韩冈却不在乎别人怎么看,他退后一步,一字一顿的重复着极为简洁的五个字:“臣……不敢奉诏!”

不是故作姿态,不是欲拒还迎,更不是墨守旧规,韩冈的眼神坚定如钢,清晰明了到不让任何人误会的表态,他不想在这个时间,这个地点,这个局面下,接下这个参知政事。

赵顼病得不能说话;司马光被召回;又与吕公著同为师保;同时留在宫中宿直的韩冈又出任了参知政事。

这几桩事发生在一夜之中,是人都会怀疑韩冈在其中动了手脚。还能靠王珪、薛向帮他解释不成?也要人信啊。

新党必然会与他决裂,可韩冈他还没打算跟自己的岳父翻脸。而旧党那边,韩冈从来就没讨过好。众矢之的的他,一个孤家寡人的参知政事,能保得住气学?那可不会是再局限于学术领域的争锋了!

纵然成为帝师能保证十年后复兴的希望,可这又要耽搁多少时间?

时至今日,官位只是韩冈达成目的的工具。韩冈当然想更进一步,可他并不打算拿自己的心血去做交换。

韩冈前世曾经在旅途中翻过不少闲书,《舌华录》之类的古文笔记也曾翻看过,其中有一条给韩冈留下一份似模糊却又清晰的记忆:

禄饵可以钓天下之中才,而不可以啖尝天下之豪杰;名航可以载天下之猥士,而不可以陆沉天下之英雄。

不要太小瞧人啊!

“韩学士……”向皇后开口想要劝。

但换来的是韩冈的再一次重复:“臣不敢奉诏。”

赵顼闭上了眼睛,眼皮沉沉的,让人清晰的感觉到他心头的疲惫,竟有一股穷途末路的气息。

要是拖到最后,逼得赵顼自己明说要册立太子,那么今夜没有开口的王珪、薛向和韩冈,还怎么能忠心于六皇子——做了,不一定会记得,但没做,却会被记一辈子。官场上,拜年送礼是这个道理,册立太子同样是这个道理——赵顼现在又岂能逼着他们离心离德?

赵颢看着他的皇兄,不知为何,一股兔死狐悲物伤其类的悲凉窜上心间。赵顼刚刚发病不过一天,宫中宿直的三位重臣,竟全都跟他离心背德。换作是一天之前,又有哪位重臣敢如此违逆天子?

向皇后正瞪着韩冈,她的眼神中充盈着愤怒……以及哀求。

只是韩冈依然毫不动摇。

如果是牺牲了十多年的心血,只为了一个参知政事,这个交换他绝不会做。

赵顼今夜的几封诏令,已经触到了韩冈的逆鳞。他不在乎钱财,不在乎官职,但他不能不在乎他的心血。

不仅仅是气学,还有新法所带来的一切——自从熙宁二年,他接受王韶的举荐之后,新法就已经跟他脱不开关系。

这不是皇帝一人的东西。赵顼没有权力毁掉。

王安石的,吕惠卿的,王韶的,章敦的,还有他韩冈的。这是数千上万参与到新法进程中的人们的心血。这关系到无数受益于新法的百姓们的生活。

纵然今天的赵顼自觉是逼不得已,但韩冈却绝不会认同。

如今的大宋,之所以能从仁宗、英宗遗留下来的财政黑洞和军事惨败中爬上来,是建立在新法顺利推行的基础上的。

新法不仅仅旧党口诛笔伐的聚敛之术,更是‘国是’,是行之有效的国家战略。

被开拓的话河湟可以作证!被灭亡的交趾可以作证!被瓜分的西夏可以作证!戒备森严的辽国边寨同样可以作证!

一旦旧党粉墨登场,主导朝局,那么之前十几年新党所建立的一切,便会成为沙土垒砌的大坝,在洪流中被冲垮毁坏。就算十几年后重新修起,造成的伤害也注定留存,不可能恢复原状了。而攀附在新法成就上,由气学格物所造就的一切,也将会是连锁性的崩塌。

军器监、将作监,交州的蛮部分封,河湟的诸部羁縻,许多制度都是韩冈与王安石、章敦、吕惠卿这一干新党中人交流之后制定的。韩冈看不到在旧党上台后能有幸免于难的可能,即便衙门会留下来——这是肯定的,几十个实职差遣就算司马光、吕公著也不敢随意废除——但其中的制度却留不下来。

或许在天子的眼里,相比起皇嗣的传承还是小事,可在韩冈这边,却绝不是可以轻言放弃。

当然,韩冈不会蠢到只拒绝自己头上的那一份升任参知政事的圣旨。赵顼的那三份诏书,毕竟已经写好了。

赶在重新睁开眼帘,双瞳中透着决绝之色的赵顼眨眼之前,韩冈再一次开口。

“参政之职,臣不能奉诏。”这一回,韩冈改了用词,不再是‘不敢’,而是‘不能’,同时,还明确了仅仅是针对参知政事一职,而不是侍讲资善堂。他跪倒在地,拜了一拜,抬起头,视线扫过太后、皇后、宰相、亲王,最后落在赵顼的脸上,与已成废人的皇帝对视着:“臣不辞万死,恳请陛下册立太子!”

王珪不提,薛向不提,那么他韩冈来提。

虽然以药王弟子的身份,第一个而不是跟着其他人之后来请立皇太子,等于是在明说赵顼活不长了。以韩冈在医学领域中的份量,他现在做的事一旦传到宫外,便是给京城中正在疯传、连夜色也决然掩不住的谣言,敲上了千真万确的印章。

不会没人明白这个后果。王珪、薛向、韩冈三人中,绝对不能领头请立太子的,只有韩冈。这一点,王珪、薛向肯定清楚,瘫痪在床的赵顼同样应该明白,甚至赵颢都能想得通。

可王珪做了哑巴,而薛向也随之仿效。所以赵顼无奈之下给了韩冈参知政事一职,并不是要任用他的才干,也不是让他代替王珪提议,而是更加直白的表明了保护赵佣的心意——依然是在催促王珪。

其中最多也只有一小半的打算,是希望韩冈在王珪仍然退缩的时候,开口请立太子。只因为韩冈开口的代价实在太大了。

韩冈却不能等待下去,辞了诏命带来的损失,必须立刻弥补。混乱不可避免,但这正是韩冈想看到的。他现在需要争取时间。

众目环伺下,端明殿学士低下头去,静待赵顼和王珪的反应。

但出人意料的,紧接着韩冈跪下来的是谁也没有想到的张璪,“臣张璪,请陛下册立太子。”

几乎在同时,薛向也跪了下来:“臣,枢密副使薛向,恳请陛下册立延安郡王为皇太子。”

薛向比韩冈更加明确的点出了太子的人选,更是自报官名来助长声势,这是在弥补他之前的过错。

王珪已经站不住了,扑通一声跪倒。只是犹豫了一下,便毁了他的未来。他今夜的错误,让他的家族日后很难再享受到宰相之后的优遇。朝堂中唯一的宰相脸色灰败,颤声道:“臣王珪,请陛下册立太子。”

三名重臣联名请立太子,包括了东西两府的宰执,以及名声广布的贤臣,赵顼和他的后妃们终于可以稍稍放心下来。

当向皇后再去看韩冈时,眼神便只剩下了感激。

上声二十哿——可。

终于等到了这一句,赵顼忙不迭的眨眼认可。

才起草了三分之一的第四封诏书草稿被撤下,换上了新的一张稿纸。张璪册立大诏。

赵颢冷眼看着韩冈。

之前韩冈不能晋升两府,都是以他年资浅薄为理由。如今既然开了头,日后也就没办法再以此为借口。原来是只差一步,现在则是隔了一层窗户纸,随时都能捅破。这一回如果韩冈接下任命,必然会有许多反对的声音,但换作是下一次,恐怕就为数寥寥了。

此人太过聪明。赵颢想着。也许在自己登上皇位的道路上,这个灌园小儿就是最大的阻碍。

韩冈冷静的感受着蕴含了不同心情的眼神。或许在他们眼中,自己辞去诏命,只是不想被人看成是用支持延安郡王为太子来交换参知政事这个职位,是自清之举。但韩冈很清楚,这完全是为了维护现在的大好局面不被破坏。

赵顼要废除新法为代价换取赵佣即位,并平安成人。对此薛向认命了,王珪也是当做理所当然,但韩冈绝不会接受。

普天之下,莫非王土,率土之滨,莫非王臣。天底下的一切都是皇帝的。但有识之人都明白,这其实只是说说而已。

就算是皇帝,也不能说臣民私家的所有物,就是他的东西。官是官,民是民,皇帝是皇帝。天子不能随意动用国库,更不用说百姓们的私人财物。即便是内库中的财货,也必须时不时拿出来赏赐百官、军队,或是补贴国用,连账本也得在三司里面放一个副册。

皇帝手上所有的权力——财权、人事权、行政权,以及制定国策的权力——全都收到士大夫阶层的强力制约,更需要士大夫们的配合。韩冈哪里能眼睁睁看着赵顼毁掉包括他自己在内的无数人的心血。

你可以主导开始,但你无权选择结束。

赵顼眼下因为中风而瘫痪失语,做出启用旧党的决定也是被逼无奈。但韩冈认为他其实还有一个更好的选择。

韩冈用眼尾余光瞥了脸色木然的高太后一眼,看来还有机会。他深吸了一口气,定了定心神。自己手上的力量太小,成与不成,这一回可是要搏一搏了。

“陛下。”就在所有人都在等着张璪的草稿的时候,韩冈说话了,“臣曾听闻河北祁州,陕西耀州、各有一药王祠,甚为灵验。若以至亲去祈福,或有奇效。”

两个亲王,两座庙。

