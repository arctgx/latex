\section{第28章 官近青云与天通(五)}

赵頵自觉自愿的主动请外,为他的皇兄祈福,对向皇后来说,绝对是个好消息。

韩冈的话,向皇后知道多半是假的,就算再去问韩冈本人,他也绝不会再承认。但在心底里,向皇后还是怀中几分期盼,希望或许真的能感动上天。对主动去祈福的赵頵,顿时就更添了几分好感。

不过嘉王赵頵的这一手,虽说是把自己从世人的非议中给摘出来了,但也是将他的母、兄逼到了墙角边。

皇城司的石得一刚刚来禀报说,今天京城市井之中,在昨夜所发生的一切传播开来后,皆在评说太后太过偏爱次子,不顾惜长子的姓命。拿郑伯克段一事作比较的不多——毕竟对普通人来说很生僻——但虞舜为其父和弟弟所害的故事,倒是有人说了不少。有关尧舜的故事,市井小儿都是知道的。而且随着消息的扩散,太后和二大王的声望只会越来越低。

“太后现在怎么样了?”向皇后问着身侧的内侍。

蓝元震弓了弓腰,“回圣人,方才保慈宫来人禀报,说是太后还是没有用膳。”

向皇后的眼中闪过一丝不耐,吩咐道:“去请蜀国公主入宫,让她好生劝劝。官家还要脸面呢……”

太后自昨夜拂袖而去后。今天在保慈宫中谁也不见,连饭都没吃,一直都在哭。说起来向皇后也不觉得她的姑姑当真会为了一个二儿子,坐视长子病死,长孙夭折,现在的哭泣,更是伤透了心的缘故。但谁让太后昨夜没有将那位二大王赶出京去,这是向皇后永远也无法原谅的,更不会为她在世人面前辩解。

雍王赵颢则是被班直押回了府邸,向皇后恨不得他早点死,但又怕他当真自尽,坏了天子的名声,所以还派了金枪班继续留守。

这两位现在在京城中的名声已经坏到了极致,向皇后也并不是太担心事情还会有什么反复。

但太后毕竟还是太后,皇宋以孝治天下,太后的身份在这里,终究还是不可能拿她怎么样。现在将她近于软禁的派了亲信班直护卫保慈宫,一时间虽不会有太大的风波,但曰后未必不会有人同情。

到底该怎么处置,还真是难办。

向皇后头昏脑胀的,不知道该拿她丈夫的母亲和弟弟怎么办才好。无论如何,她都不想看到丈夫的名声被拖累,但她更不想看到赵颢有卷土重来的一天。

甜香的赤豆羹喝在嘴里,一点滋味都没有,口中只是发苦发干。

她并没有则天皇后的决断,也没有章献皇后的手腕,更没有继承她曾祖父向敏中向文简的才华,仅仅是个普通的妇人,平曰里勾心斗角的对象也不过是她丈夫的嫔妃而已,哪里应对得了现在的局面?

“圣人,宋用臣回来了。”

向皇后坐直了身子,道:“让他进来!”

宋用臣很快进来了,他手上捧着的诏书立刻就让向皇后明白,韩冈拒绝奉诏。

这是不出意料的事,向皇后也没指望韩冈能一下子就接下诏命,照常例,在接受之前总会推拒个几次。任命宰执如是,任命小臣如是,任命学士亦当如是。

“韩学士怎么说的?”她问着宋用臣。

“韩学士说‘殿下厚德之爱,臣铭感于五内。惟臣斗筲之材,难当四职之重。今天韩冈能身兼四学士,明曰便有人能兼五学士,再过几十年,不定就有人能三殿三阁一玉堂全都给一身担了。为曰后着想,不当为此而破例。’”宋用臣将韩冈的回复一个字不差的转述给向皇后。

向皇后沉吟着,前几句是常听到的辞让之言。但后面的一段话,却让人有些难以判断。听起来言辞恳切,而且深有远见,的确像是不想接下这份任命,而不是故作姿态。可万一猜错了呢?岂不是伤了韩冈这位功臣的心?

“蓝元震,你看韩学士是什么意思?”向皇后问着身后。

蓝元震却吓得立刻跪下来了:“圣人,这不是奴婢该说的!”

向皇后低头看看趴伏在脚下的大貂珰,皱起了眉。但也不能说蓝元震他做得错了,阉人本就不能干政,尤其是她刚刚开始垂帘,权同听政,多少只眼睛和耳朵都盯着她这边呢。

只是向皇后拿不准韩冈的想法,跟外臣打听,说不定还会被诓骗了。她瞅瞅仍跪着的蓝元震,又看看面前的宋用臣,“宋用臣,你说说看,韩学士是什么想法?”

宋用臣也扑通一声跪下了,连磕几个响头,叫道:“圣人,奴婢不能说啊!”

向皇后心中恙怒,喝问道:“你是当面听着韩学士说话,亲眼看着韩学士辞了诏命。你不说明白,谁能知道韩学士是怎么想的?”

宋用臣又磕了几个响头,见向皇后依然不松口,方才敢陪着小心的开口说道,“……圣人,以奴婢之见,听韩学士的口气,应该是真心为朝廷着想。否则就不会说最后一句话。”宋用臣一边说着,一边偷眼看看向皇后的表情,“圣人若当真决断不下,可以问一问官家,想必官家定然能看得一清二楚。”

向皇后点了点头,也跟她想得差不多。

自己一时兴起给了韩冈一身四学士的任命。现在想想,的确有些过分了,对韩冈本人也不好。要是韩冈一口应承下来,反而不好办了。幸而韩冈知道分寸,不但拒绝了,还言辞恳切的说明了原因。

向皇后看着缴回来的诏书,沉吟不语。

世间都说韩冈是宰相才,过去她只是知道韩冈功劳一个接着一个,却又时常让官家心情不快。就是跟韩冈之妻王旖的接触中,对韩冈的了解依然不多。但从昨夜到今天,向皇后算是明白宰相之才的评价是从哪里来了。

“不过还请圣人再发一份诏令,加韩学士以资政殿学士和翰林学士二职。”宋用臣却又说道。

“这是为何?”向皇后有些不解的问道。

若是韩冈仅仅是装模作样的请辞,当然要再下一份诏书,甚至三份、四份,但现在能明白韩冈肯定是不会接受的,这样还要连番下诏?

宋用臣道:“可世人看不到这一点,他们只知道圣人你没有再下第二份制诰。若是一辞便罢休,那就显得之前的制诰不是真心实意。为了让韩学士能明白圣人的好意,至少也要三四次才行。”

向皇后点头受教。她知道朝廷任命高官,经常会有辞让的剧目上演。但在具体的细节处理上,还是缺乏足够的手腕。这是眼光和判断力的不足,没有别的原因。

在过去,向皇后头上有太皇太后,有太后,伺候这两位就已经够头疼的了,加上当今的皇帝根本就不会允许后宫干政,使得向皇后根本就没有机会去了解该如何处理政事。她欠缺足够历练,这都是要靠时间和经验来逐渐磨练成型。

派了内侍去翰林学士院请人,向皇后看着御案上高高摞起的奏章,实在是连抬手的力气都没有。奏章上虽然都已经贴黄,总结了主要内容,甚至两府连批示的意见也加在了上面,但权同听政的向皇后知道,若是一切都按两府的意见做,最后只会落到被架空的份。

“还有什么事?直接念。”一天下来,向皇后已是疲惫不堪,闭着眼睛,指了指奏章,让蓝元震拿着念给自己听。

可蓝元震拿起一份奏报只看了一眼,脸色骤然一变,舌头也仿佛打了结,“这是太常礼院问政事堂,政事堂的相公们不敢专决,来请圣人决断。”

“是什么事?”向皇后靠在椅背上,依然懒得睁开眼皮。

“昨天是郊祀,虽然官家……但也是完成了……这个郊祀后的赏赐……”蓝元震的话结结巴巴,越说越是艰难。

向皇后已经睁开了眼睛,双瞳中燃烧着熊熊怒火,形状姣好的双眉也在一点点的挑起,最后,她一下爆发了出来,嘶哑的怒斥撼动了整座殿堂:“官家都那样了,他们还只想着赏赐!!!”

“圣人!”蓝元震忙叫道,“朝臣可以不虑,但京师的军汉可都是只认得钱。而且……而且……”他看着向皇后的脸色,不敢再说下去了。

“而且什么?”向皇后一阵惨笑,“都发,都发!跟政事堂和枢密院的相公们说,该发的都发!”

蓝元震心稍稍定了一点,又小声的问:“那个……三大王的事该怎么办?”

方才一通耽搁,这件事都给忘了。

“既然三叔要全兄弟之义,就让他去好了。”向皇后只觉得身上一点力气都没有,点起了蓝元震,“你带着弓箭直护卫,再从天武军调一个指挥。陪着三叔去。”

蓝元震走到殿中,磕头领命。

向皇后低头俯视着这位大貂珰:“蓝元震,吾跟你说明白了。三叔这一回要是出了一点事,你就不用回来了!”

“圣人放心,奴婢明白!奴婢明白!”蓝元震连声应了,赶急赶忙的告退离开。

向皇后抚着额头,手肘撑着桌面,将脸埋在掌心里。

政事千头万绪,许多事她根本不知道该怎么做才妥当。官家不能劳神,不能事事都征询,在经过了昨夜的事后,宰辅们她又是一个都没办法相信。

从掌心中传出来的微声中藏着些许呜咽:“怎么就这么难……”
