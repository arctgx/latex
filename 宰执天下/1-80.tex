\section{第35章 把盏相辞东行去(一)}

惠丰楼中,韩冈本以为除了王厚之外,就只有王舜臣、赵隆等几个相熟的友人。惯例的十里相送,要到明天他启程才是时候,到时王韶、吴衍说不定都会到场,而今天,应是王厚找个借口来喝酒。

他没有想错,王舜臣跟着来了,李信也到了,还有杨英——王韶自德安带来的乡里,也是最贴身的亲信——同样到了,连赵隆也辞过王韶,匆匆的赶来赴宴,几个相熟的同伴的确都来为韩冈饯行。

但他又料错了,由王厚主持的饯行酒他并没喝到。刚刚走上惠丰楼的三楼,一个坐着位置最好的一桌的客人,便派了个仆役来跟韩冈打招呼。

抬眼看去,王厚和韩冈两人都吃了一惊。虽然是韩冈很陌生的相貌,从来没有打过交道。但韩冈知道他是谁,王厚也知道他是谁。

秦凤路走马承受——刘希奭。

一个阉人。

出自宫中,按规矩不得结交地方官吏,担任着走马承受之职的阉人,不知为何没有参加鞭牛后的春宴,却身在惠丰楼上,还派人过来跟韩冈打招呼。

“可是韩玉昆?”刘希奭远远的招呼着。

韩冈略一犹豫,便主动上前,向刘希奭行礼道:“韩冈见过刘走马。”

刘希奭起身还了半礼,笑道:“久闻韩玉昆大名,却总是错过。今日得见,方知名下故无虚士。”

大概以为韩冈第一次亲眼见到阉人,王厚有些紧张的注视着韩冈的神色。他知道但凡士人都不会对阉宦有任何好感,生怕韩冈在见面时有什么失礼的举动。但韩冈老实本分的行礼,让王厚松了一口气的同时,还有了点淡淡的的失望。

与王厚猜想的不同,韩冈并不歧视阉人,不过少了二两肉而已。只要不是自己下面少,他并不在乎别人有没有那二两肉。韩冈也不会把历史和小说混在一起,很清楚北宋的宦官们不会葵花宝典,也不会有避邪剑法。只是想法虽然很不现实,他还是期待着能见着一位能说出‘要圣旨,来人那,咱们给他写一张’这句台词的奢遮公公来。

可出现在韩冈面前的阉宦刘希奭,没有想象中的阴阳怪气,站在人群中就是一个很普通的男子,只是没胡子罢了。他的声音略显高亢,但下体健全的男人中,也不是没有声音尖细似女子的。如果不是明着介绍出来,韩冈也做不到在第一时间便发现他与常人不同。

走马承受,全称是‘诸路经略安抚总管司走马承受并体量公事’,这么长的名头,写起来不方便,说起来更饶舌,一般都简称走马承受,或直接称为走马,就跟韩冈的经略安抚司管勾公事的简称抚勾一样。

刘希奭拉着韩冈的手往自己的桌边走,显得亲热无比,“玉昆果真是大贤,甘谷疗养院刘某近日刚刚去过,里面诸多伤病对玉昆你可是交口称赞,感恩戴德。”

“走马过奖了。韩冈只是适逢其会罢了。”韩冈有些纳闷着刘希奭的示好,被阉人拉着手,鸡皮疙瘩都冒了起来。只是他掩饰得极好,看不出半点异样。

刘希奭豪爽的笑道:“适逢其会便能帮一城的将士解除后顾之忧,到了玉昆真的领下提举伤病事的差遣,路中各寨还有多少将士会畏敌如虎?日后西贼再犯秦州,总少不了玉昆的一份功劳。来来来,明天玉昆你就要上京,趁着今日尚在秦州,刘某权且以水酒一杯一助行色。”

秦凤走马拉着韩冈在自己桌上坐下,又招呼着王厚过来。王舜臣等三人地位不够,在旁边的一桌坐了,由刘希奭的伴当招待。

刘希奭在秦凤地位特殊,人人敬他三分,就连李师中等闲也不想得罪他,而惠丰楼又是官产,刘走马要请客,谁敢慢待?

不移时,美酒佳肴便摆满了两张桌子,再过片刻,惠丰楼里两名头牌歌妓也走了上来——惠丰楼是秦州最大官营酒店,里面的歌妓也是教坊司中精挑细选——玉手将琵琶轻拢慢捻,便在桌边婉转而歌。虽然是最常听到的柳永词,但并非是‘寒蝉凄切,对长亭晚,骤雨初歇’那般扫人兴的歌调,而是‘变韶景、都门十二,元宵三五,银蟾光满’,唱着东京的元宵胜景,正好韩冈在年节时入京,即应时,又应事,取一个好意头。

‘他想做甚?’王厚的脸上写满了疑问,如今的秦州官场上,王韶并不受待见。而韩冈作为王韶手下第一得力的谋主,也当然是一个待遇。现在刘希奭宴请韩冈,摆明了是要帮着王韶一手。他为何在这么做?

王厚的疑虑刘希奭看在眼中,但韩冈脸上清浅自如的笑容,却毫无半点异样。但以韩冈的才智,会看不出走马承受的宴请在秦凤官场中的意义?怕是已经看透了才是。刘希奭自此才在心底里真心诚意的叹了句:‘果然是名不虚传。’

刘希奭的任务就是在秦凤作天子的‘耳目之寄’,实司按察之职。他负责监察秦凤文武众官,有风闻奏事之权【注1】。

不过,并非是不论大事小事都要上报,也是有选择的。像陈家的覆灭,裴峡谷中的战斗,韩冈察举西贼奸细的功劳,都会报奏朝中。而伏羌城中韩冈与向宝家奴的冲突,便不会上报——一是因为向安事后处理的好,二是刘希奭并不觉得为了这等小事有必要得罪向宝。

从走马承受接受的差遣来看,他们只是兼任了监视任务的一个情报搜集官,不会也不该偏向地方上任何一位官僚,更不能有派别倾向。就算到各处寨堡视察,也不允许接受寨主堡主们的宴请。

但是人就有立场,而且走马承受与天子之间的联系并不是单向的,天子的心意有时候也会透过走马承受来传达。王韶是赵顼亲自拔擢出来,放到秦凤。支持他的行动,也是会受到天子的赞许。

同时,建功立业的心思,刘希奭也有。所以他会找韩冈搭话——如果直接找王韶,那是结交地方官吏。但韩冈是即将上任的新人,先打个照面,顺便一起坐坐,了解一下性格为人以及才学能力,即便官司打到天子面前,都不能说他有错。

韩冈不可能看得透刘希奭的所有盘算,但刘希奭设宴为他饯行代表的意义,以及可能引发的官场变局,总是能推断得出。这是雪中送炭啊………

这阉人当真是帮了大忙,韩冈举杯敬向刘希奭。而韩冈这一举杯,便让王厚放下心来,‘看来对大人并不是坏事’。心情一松,原本充耳不闻的歌声,也在耳中清晰起来。

惠丰楼的两个台柱子,都是不到二十的佳丽,自幼在教坊司中得人教导,琵琶铮铮,歌喉悠扬,端的是色艺俱全。从桌的王舜臣等人已为声色所迷,看得如痴如醉,王厚家教严谨,只偷眼看了两眼,便不敢再看。只有韩冈,他与刘希奭推杯换盏,谈笑正欢,半点也没有把两位歌妓的表演放在心上,眼神投过去也只当是山石流水,连眼皮都不带动弹一下。

蹬蹬蹬,又是一阵楼梯响。

“我说惠丰楼的两个台柱子去了哪里?原来是在这里给人唱曲儿。”随着一句有些做作的声音,从楼下呼啦啦的上来了七八个人。打头的是个油头粉面的年轻人,面皮粉白,双唇鲜红,仔细看去,他脸上当真是涂脂抹粉,好生打扮了一番。

韩冈的眼皮子终于跳了一下,刘希奭这个没下面的阉人,看起来还是个再正常不过的男子,但眼前的这位,却是不折不扣的人妖。男人涂脂抹粉不知是哪里的风俗,至少韩冈在秦凤可没见过。

刘希奭站起身来。韩冈停了一下,也跟着站了起来。能让秦凤走马起身相迎,来人必然是有官身的。但看来人的模样,不是正经官员,而应该是荫补。

‘是窦家的哪一位?’

李师中的家庭情况,韩冈已经清楚,没有这等货色。而秦州城里,够资格荫补子孙的官员,除了李师中,就只有窦舜卿。韩冈正想着,刘希奭已经给了他答案:“原来是窦七衙内。”

“窦解。”王厚在韩冈耳边轻声道。秦州官场内的消息,他一向打听得一清二楚,“窦舜卿的亲孙,出自长房,家中排行第七。但窦舜卿的前六个孙子都夭折了,所以算起来,他还是长房嫡孙,荫补了个正九品的右侍禁。”

王厚说到荫补,不经意的哼了一声,声音很轻,但落在了韩冈的耳中,却不禁了然一笑。

王厚当然不喜欢荫补这两个字,因为他不是王韶的长子。王韶可以推荐韩冈,却不能推荐自己的儿子,而王厚又不是读书的材料,正常情况下肯定是要等荫补入官。不过论荫补顺位,王厚比他的大哥王廓来得要低。自来荫补子孙,都是长子长孙居前。虽然王廓在家乡悠闲度日,而王厚却是在西北边陲风吹雨淋,但规矩就是规矩,礼法纲常不容违逆,而王厚,就只有等待另外的机会。

注1:看过水浒的朋友都知道,花和尚鲁智深在出家之前,做到了关西五路廉访使。所谓廉访使,其实就是走马承受,只不过是在徽宗时改了名字而已。

ps:第一个太监出场了——虽然北宋的太监并不是指的阉人。拓边河湟,阉人出场很多,最有名便是的童贯。

今天第二更,求红票,收藏。

