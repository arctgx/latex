\section{第15章 经济四方属真宰(中)}

送了王居卿离开,韩冈回到后院。

跨进门,就看到四位妻妾向往常一样,正坐在一起,一边做着女红,一边聊天。

“官人。”王旖四女放下了手中的织物,站起了身,“前面的事都处理好了?”

“都好了。”

韩冈活动着肩膀,一天下来,身心疲惫。

做了宰相之后,手上的事多了许多。苏颂年纪大了,也不喜揽权,加之现在政事堂中还缺参知政事,韩冈等于是一人处理所有政事。

“官人这个宰相做得太辛苦……”

“参知政事什么时候选出来?”

家里的妻妾,都知道韩冈不好揽权,更注重自己的理念能否施行,故而都希望他能早点找到合适的人,将一干不重要的庶务都交托出去,免得再这么下去,变成五丈原的诸葛亮了。

韩冈叹了口气:“廷推都拖了两次了,下个月的廷推,估计还会再拖。李定根本就选不上,这边的沈存中也一样。李清臣没根基,李奉世只比沈存中差一点,新党、旧党都不会选他。曾令绰还不是进士,在西府还好说,入东府就难了。”

“不至于没人可用?”

“天下不乏人才,只是一时之间不凑手。”

“官人这话说得就像是穷措大上街,买不起东西不是缺钱,只是一时不凑手。”

周南戏谑的说着,韩冈苦笑着摇摇头。

韩冈不希望新党插手进东府,而新党那边也不希望让韩冈继续控制政事堂,按照编订成型的廷推条例,有资格参加廷推的重臣中,放弃投票的人数超过一定比例,廷推就得不出结果。自韩绛求退后,两次给两府增加人手的廷推都因为各方的拆台而失败了,接下来的第三次,韩冈也没把握能通过。

反正僵局持续下去,权力会更加集中在他韩冈的手中,对于新党来说,同样不是一件好事。也许这第三次,他们该学聪明一点了。双方都退让一步,应该能得到一个双方都能认同的结果。

“好了。”王旖道:“现在就别说公事了,官人好好歇歇。”

周南、素心端来了茶汤、菓子,而王旖和云娘也过来帮韩冈换下见客的衣服。

过去,但这两年王安石先是在江宁任知府,之后卸职就任宫观,一年前致仕。没了政治因素造成的隔阂,夫妻之间的关系又恢复了和睦。

家里面平静,韩冈也能安心去处理国事。修身、齐家、治国、平天下,这一层层递进,的确很有道理。

“对了,官人,大哥今天来信了。”严素心说道。

韩冈喝了一口茶:“大哥在书院还好吗?”

韩冈一向觉得读万卷书,不如走万里路。在家里的老大年满十五之后,便被韩冈打发去了横渠书院。

“大哥还好,现在跟祥哥住在一起,偶尔上街逛一逛。”

“今天收到信,二哥也闹着要去读书。”

“二哥得再有两年,他还小。”依照韩冈的想法,至少要到十五岁之后,才能出去游学,“跟他说,等他年纪到了,就让他去横渠书院读书。”

王祥是韩家的准女婿,王厚几年前入京后,在韩家常来常往,很得韩冈夫妻的喜爱。这一回韩冈将韩钟打发去读书,王祥也一起跟了过去。

按照韩钟的信中所说,他现在在横渠镇上租了一座小院,与王祥住在一起。

从出发,到租屋,都是两个孩子自己自己去处理,他们身边都只有一个伴当,绝大多数时候都得靠自己。韩冈暗中让人照料,却没有出面,尽量让他们能自行完成一切。

“瑞麟那孩子性格稳重,又好上进,有他在身边,大哥的学业就不用担心了。”

通过一干护卫暗中回报,韩冈对王祥的表现更加满意。性格稳重,做事沉稳,待人处事也谨严守礼,但也不是冬烘先生,也有少年人的活力。有这样的人在自家儿子身边相扶持,做父亲的哪有不放心的道理。

“官人这么说,奴家就放心了,再过几年,就能还金娘一个进士夫婿了。”

“那要看他们努不努力。若是用心的话,进士不好说,诸科是轻而易举。”

王安石还在,新党也依然遍布朝堂,韩冈这两年没有对新法大动干戈,只是做修补和调整。

给七十岁以上老者鸠杖,许其入府不拜——此乃汉制。国初时,朝廷也有规定,但一直没有注重施行,韩冈让人专门上表奏请朝廷为此下旨,要依照法令施行。

这是惠而不费之举。朝廷之中,也没有反对声。不过如果是要花钱的项目,比如设立照料鳏寡孤独的养老之处,即使这是儒家先贤的理想,一样会被人反对,而且也绝不现实。

这些细微之处的变化,对朝堂和士林的影响并不大。这几年,气学秉政,朝堂上最为重要的变化,就是科举的科目又增添了两项。

韩冈曾经报与韩绛的明算科和明工科,堂而皇之的成为国家择士的一部分。两年半前公诸于众,下一科,便是第一次开考。

“不过诸科总没有进士好。大哥和祥哥在横渠书院读几年书,是不是让他们回国子监来?”

横渠书院是气学的本山,而国子监,至今仍旧是用三经新义来教书育人。进士考试的内容,也同样是新学圭臬的三经新义。

王旖四女都希望家里的儿子能够高中进士,而不是诸科。但这样一来,就意味着家里的孩子必须去学习新学。气学宗师的儿子,却通过学习新学而榜上提名,这不啻是一个巨大的讽刺。

老大是严素心的儿子,她不好说话,王旖也同样不方便说。所以严素心开口试探,想问一问韩冈到底打算怎么做。

“等他们在横渠书院中学得差不多了再说其他事。”韩冈不快的说着,但看了看妻妾,语气缓和了一点,“至于国子监,也不必那么急。他们还小,想考进士,过两科再说不迟。”

进士科的内容,还有国子监中的科目,韩冈早就想变动了,但遽然改变并不合适。以王安石的急脾气,都用了三年的时间,韩冈并不打算太过仓促——要捅马蜂窝之前,还是显得做好周全的准备,免得被蜂蜇。

现在韩冈正是在设法动摇新学的地位,同时给他的根基,也就是陕西的士子们,更为畅通的入仕渠道。

下一次大比之年,照常例,能有四百名新科进士进入官场。除此之外,便是诸科。

诸科之中明算科一百二十人,明法科八十人,明工科八十人,此外每年还有二十人,通过太医局与厚生司的联合考核,成为翰林医官。

这几门诸科考试中,不仅仅新设立的明算科和明工科是百分制,便是明法科,也会改成百分制。还有医官考试,同样是百分制。

加之殿试考试,上一科就已经是百分制了,韩冈并不担心将之延续下来,能有多少反对的意见。

等到人们都习惯了百分制的考试,下面就会是最为重要的解试与礼部试了。

听了韩冈的话,最为关切的严素心就笑道:“有官人的话,奴家就放心了。大哥还没什么,总不能金娘没有一个进士夫婿。”

继承了母亲的容貌,又是韩家唯一的女儿,金娘虽为庶女,在京城的内外命妇中也算小有名气。不过她早早的订了亲事,让许多人扼腕叹息。

韩冈亲生的儿子已经有九人。不过并非嫡出的老三、老四,被过继给了韩冈两位亡故的兄长,在名义上,算是他的侄子。却只有这么一个女儿,当然备受疼爱。

“三哥哥,你说祥哥要两科之后再考,那金娘的婚事怎么办?”

韩家的长子、长女是前后脚出生,年纪相当,等到两科之后,早过了二十岁了。

“再过两年就差不多了,也不用等到功成名就的时候。”

一晃眼就十几年了,韩冈这个岁数,放在后世,甚至可能刚结婚,但在这个时代,已经到了儿女要成婚的年纪了。有些地方成婚更早,三十出头就有孙辈了。

“苏家的金娘呢?”

追谥忠勇的苏缄长子苏子元,韩冈当年率军南下,与苏子元交好,为自家的长子向苏子元的女儿求了亲,那也是邕州陷落时苏家唯一的生还者。

“苏伯绪今年任满,要上京了。”韩冈道,“到时候,金娘也会一起回京。会让伯绪在京师留任几年,到时候正好让大哥和金娘完婚。”

“金娘一直都在岭南,只有书信往来,这一回终于能见到人了。”

“应该不会差,你们也用不着担心。”韩冈道,“大难不死,必有后福。金娘是个有福相的孩子。”

随着儿女越来越大,儿女们的事,渐渐的就成了家庭议论的中心,看着孩子们一天天长大,韩冈也越来越觉得自己的年纪在增长。

一晃十几年,不管自己还能在朝堂中多少年,时间总是不够用的。

时不我待啊,韩冈想着。

