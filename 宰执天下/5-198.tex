\section{第23章 弭患销祸知何补(六)}

“这分明是欺君!!”

从御史台的正厅中传出来的声音阴沉无比,如同窗外的天空。

不止一个台官将士大夫的风仪丢到了脑后,对开封府上下咬牙切齿。

本来御史台中倒有三分之一的御史,准备要借蹴鞠球赛后的那一桩惨剧,好生将京城中的风气整治一番,顺便将那个几次三番都轻易从乌台口中脱身的家伙也一并拉下来,

可开封府干脆了当的就推了个替罪羊出来,将罪责都推到了死掉的南顺侯李乾德身上。那放在桌上的厚厚的一摞口供抄本,连同奏报的抄本,在一众御史们眼中,白纸黑字的全都是嘲讽。

什么时候御史台已经被人小瞧到这般地步了!御史台都已经盯上来了,竟然还敢在这么一桩大案上做手脚?这未免是欺人太甚了。

十几名御史一个个怒气难遏,唯有上首处的李定如同老僧入定,端坐着不发一言。他这个御史中丞一直都没有说话,看着手底下的人仿佛是正在吃饭时被人一脚踩了尾巴的猫一般蹦的老高,心中暗叹,御史台的成员真是一蟹不如一蟹了。就算拿着欺君的罪名出来,又能吓得了谁?

欺君从来只是喊出来吓人,定罪时才可怕,可在朝中做官,没人少做过,谁会老老实实的什么话都跟皇帝说?

“二十一家大行会中,有十三家的行会或是行首养了球队。开封府中的官吏也有许多人以两项赛事为财源。如此势力,只要想找,能找来一千个证人为这桩案子寻找证据。”

李定越是深入的去了解两家总社的实力,便越是发觉这件案子的棘手。

总不能将证人都拘入台狱审问。这不是笑话吗?

开封府选择的人选,的确是让人无从措手。

谁让李乾德是降臣,而且他的军队曾经兵犯中国,这个罪孽是曾经的交趾王永远也洗不脱的。眼下罪名落在他的身上,那么未来的几十年内,决没有机会翻身,何况这一次连苦主都没了。谁会为这样的人去争辩?

“就算是天子也想大齤事化小,小事化了。”李定说着,他这几天见过赵顼几次,对天子的态度有所了解,“李乾德已亡,也不可能活过来为自己。死人是不会说话的。”

李定的话中透露了几分皇帝对此案的态度,只是没人理会他。下面的御史各自小声的交头接耳,议论着破局的办法。

李定皱着眉听了一阵,神色中的不耐烦的成分也越来越浓。。

御史中丞已经做了许久,按旧例差不多也该离任了。只是他在天子那边远没有蔡确得宠,不用指望能升到东府或是西府去,在台中说话的声音也便一天比一天弱。御史本来就只需对天子负责,即便最低一级的监察御史里行,也可以弹劾宰相,从不需要以御史中丞马首是瞻。何况李定的名声从来就没好过,在以清流自诩的御史中,根本就无法服众。这两年下来,说话没人理会的的情况他早就习惯了。

不过这一次的情况特殊,越是将核心遮着掩着,就越是代表他们在此事上有所图谋。天子的耳目众多,不会看不到这一点。失去了天子的信任,对御史们来说,那就是一个灾难。

正想最后一次,尽一下御史中丞的职责,难得迟到的张商英终于抵达了会场,只是脸上的表情,比起方才会议上的几名御史更加阴郁十倍。

“怎么回事?”李定问着。

张商英没有多话,直接将手中叠起来的一页纸打开,递到了李定的手中。

粗糙单薄的纸面,以及纸上并不整齐工整的文字,让人一眼就看出来,这是如今市井中十分常见的小报。

张商英要让人看得内容就在头版上,李定看了两眼,脸色木然的转手交给了下面的人。一份小报就这么在御史们的手中转了一圈,最后又转回了张商英那里。而厅中的气氛,也就在小报的传递过程中,变得跟张商英的表情一般森森如晦。

阴鸷的眼神左右横扫了一番,张商英恶狠狠的说道:“看到了没有,这是步步紧逼啊,要将罪名彻底坐实在南顺侯的身上!”

御史台中的官吏们见多了这样的小报,李定平日里可没少看到乌台中人拿着薄薄的一张纸在私下里仔细研读。这其中不仅有吏员,还有言官。

自从齐云总社在几年前开始五日一次的发售刊载了球赛赛况的《蹴鞠快报》,京城之中的各色小报便越来越多。很多小报,都是在上面刊载了一些商家打招牌的广告,拼凑几个荒诞不经的古今故事,再加上几篇佛经道藏的片段,然后夹杂着近日的新闻,敷衍成文。

小报上用的全是简笔的俗体字,而且还是歪歪扭扭的活字印刷。看完后就可以用来做包裹,肉铺上时常能看到有人拿着小报而不是荷叶将买来的肉裹好离开。

外地也许要差一点,但在京城中,文风荟萃,百万军民中倒有一多半的男丁能识文断字,女子也有三成在幼年时学过《女则》、《女戒》、《女论语》,虽说绝大多数人——甚至可以说其中的九成九——学问并不精深,只是认得三五百字,背得《论语》和《千字文》,但连正正经经的家信都写不好,可是看懂小报上的文章,连估带猜的,倒是不会有什么问题。而且小报读得多了,认字的本事也能有所长进。

数以十万计的识字之人,让这些小报在京中活得有滋有味,甚至于京城周边的造纸作坊,也变得一日多过一日。些年来,要不是一众小报还没有变成传播谣言的揭帖,早就给朝廷禁了。

但小报发行的之多之广,也颇让人为之忌惮。在京城流行的诸多小报中,《蹴鞠快报》的发行量是最高的,据说这份三日一期的小报,每一期都能卖出两万份。当这样的一份纯粹以联赛赛报为主打的小报,在头版的位置上刊载一场断案的新闻,在京城之中所能引发的风浪,可想而知。李定就算只看这份《蹴鞠快报》,也知道南顺侯府眼下究竟会是一个什么样的情况。

齐云总社是想要在结果出来之前,在京城百姓们的心目中先一步造成既成事实。

但事情不是这么简单。由此带来的风波,只会愈演愈烈,最后形成一场让御史台无法扭转的风暴,

这是谁的声音更大的问题,这是谁能代表更多人说话的问题。

就像肇事者的身份被确定为南顺侯李乾德一样,当《蹴鞠快报》开始在报纸上刊载审案的新闻,整件事的性质变了。

如果事情继续发展下去,引领民间舆论、甚至士林风气的不再是他们这些言官清流。

过往多少重臣败在掌控士林的清流的言谈之中,控制了士林,就是让天下言论只能发出自己想让人听到的话。可是一旦这柄刀子给外人抢走了,那么清流和士林的地位也将会一落千丈。

“要不要干脆遣人将齐云总社给封了?……朝廷还没定案的事也是他们有资格说的!?”一名御史提议道。

李定从张商英手上拿过报纸,将头版上的那条新闻从头到尾的又仔细读了一遍,最后无可奈何的摇了摇头。报纸上的报道本就没留下一点破绽。

这快报上一点也没提定罪两个字,只是说开封府昨日初审,其结果已报请朝廷复核。完全看不到扭曲夸大的成分。一看就是公平公正的报道,甚至没有因为李乾德的身份而大做文章——在李定眼中,这是极聪明的做法,与其灌输,还不如让其自己去想,这样得出来的结论才会根深蒂固,让他人无法动摇。而齐云总社,当然也会在人们的心目中,拥有更加权威的份量

别的不说,在这份快报背后可是齐云总社,而齐云总社背后,则是数以百十计的皇亲国戚和一干豪商,这背后的势力则惊世骇俗。如果没有足够的证据,就去将齐云总社给封了,等于是给了齐云总社后面的诸多豪门一个结结实实的把柄。到时候,在御史台中,提议的、执行的,全都要给人捏在掌心里面。

既然他们敢公然公布案情,肯定也必然是做好了一切的应对准备。从天子的角度来说,他是不会介意多一个了解民间民生的通道,御史台的攻击,只会被皇帝毫不在意的丢到一边去。

李定沉吟了好一阵,最后点起了一名吏员,“去南顺侯府,看看有什么动静。”

“只是去看看?”张商英不满的问道。这个时候,可是需要切实的行动。

“只能先去看看,哪里还能做些什么?”李定反问着,让张商英等台官哑口无言。众目睽睽之下,多少双眼睛盯着,御史台也不敢去作威作福。

“不能就这么了结!”一名御史大声叫道。

“不会就这么结束。”李定给了很肯定的回答。他可不想看到御史台的职权在自己手上被削弱,“日子还长得很!”

