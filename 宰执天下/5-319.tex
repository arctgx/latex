\section{第31章 停云静听曲中意(26)}

“官家那边能不能瞒得住?”章敦转了话题,低声问着。

“皇后肯定不会露口风,宫里面都是聪明人,也肯定不会乱说。除非一时错口,否则官家就只能靠猜测了。可就算猜,也只会猜测是我们这些臣子欺负皇后不通兵事。终究是夫妻,不会误会太多。”

皇帝迟早会了解到真相,到时候会有什么反应,谁都不能肯定,不过有皇后在中间隔了一层,他们这些做臣子的倒也不用太过担心。

“玉昆,”薛向则再一次提起之前皇后问过的问题,“这一回你有多少把握?”

“该说的方才在崇政殿上都说了。”韩冈正容回道,“太原城不需要担心……除非在我抵任之前就丢了。除此之外,就算辽军已经攻到了太原城下,我也有把握将其赶北面回去。至于更进一步收复代州……则就要费些时间和力气了。”

韩冈有着充分的信心,辽军能攻破雁门关不过是运气而已。而运气是不可能一直拥有的。

只听他说道:“其实雁门关被破也有好的一面。从河北一路变成了两面进兵,辽军现有的兵力是不够用的,必然要侧重其中一路,或河东,或河北。但不论以那一路为重,另外的一路都很容易解决。”

……………………

福宁殿一片死寂。

天子赵顼半躺半靠在床榻上,正冷冷的瞪着皇后。下面的内侍、宫女连大气也不敢喘。

任命韩冈为参知政事,宣抚河北、河东。

这是向皇后刚刚对赵顼说的话,也由此引发了天子的愤怒。

以枢密副使宣抚河东,这就是方才在殿上和重臣们议定的结果。韩冈本人,也没有推辞这项任命。

不过为了让赵顼答应这项任命,向皇后不得不用些进二退一的小手段。宣抚河北、河东只是准备跟皇帝讨价还价的。关键只在河东。参知政事也是一样,退一步一个枢密副使也就够了。

毕竟河东的局势不能告知天子,那完全是皇帝的私心造成的恶果。为了丈夫的安危,向皇后只能选择隐瞒下来,只是还要忍受丈夫的误会和愤怒。

要不是韩冈日日都要进殿问疾,不可能瞒着赵顼出京,向皇后根本都不想将这项任命告诉丈夫。否则就算再怎么隐瞒,他也有可能会猜得到北方的局势有变。

‘河北军情如何?’冷冷瞪了皇后一阵,赵顼开始在沙盘上划着。这项突如其来的任命过于荒谬,冷静下来后,他自然不会察觉不了其中有问题,但心情不会变,。

“还算好,南下的北虏还没有突破定州和保州。东面在三关也挡住了。就是北虏进攻越发得猛烈。不知道能支撑多久。郭逵已经几次发急报了。”

‘河东呢?’

“王.克臣老迈无用,向河北派去的援兵出发时就迟了整整两天,现在胜州又有兵患,雁门关那里也在报急,这都不是他能应付得了的。”

赵顼张着眼睛,视线定在皇后的脸上,熟视良久,方才写字道:‘让赵禼去河东。’

赵禼?

环庆路经略使赵禼?

皇后姣好的双眉立刻就拧了起来。

赵禼有这个能耐吗?现在可是辽军打破了雁门关!但她没办法将真相说出口。

或许赵禼也有这份能力,但他在河东的声望远远不足以安定民心,凝聚军心,乃至震慑入寇的北虏。要是赵禼有韩冈的才干,以及在河东的声望,向皇后早就让他去了。她根本就不愿意将韩冈派出京去!但凡有人在能力、声望和经历上,有个韩冈六七成的水平,不论是谁,向皇后都会改派他去河东。

太子过两天就要出阁读书了。二月二,龙抬头,正是风俗中小儿往塾中窗户里扔葱,开聪明的日子。韩冈若是这时候出去了,太子给程颢教坏了怎么办?生了病又怎么办?王安石虽然是资善堂翊善,可这位元老重臣,终究是不可能给一个六岁孩童开蒙,教其识字的。王安石和程颢在保全幼子上,又哪里比得上药王弟子?何况满朝文武,也只有韩冈最能让她信得过。

压着烦躁的情绪,向皇后辩说道:“陕西局势未定,赵禼一时也离不开。吕惠卿贪功,有稳重的赵禼在旁辅佐,这才让人放得下心来。河北局面虽好,但郭逵毕竟是武臣,难以使动河东派去的兵马,也需要韩冈压阵。”

向皇后担心赵顼听到真相后病情加重,不能跟丈夫说明河东现在的局势。但不将话说明白了,赵顼又如何会答应让韩冈重领河东一路,而且还是以参知政事领河北、河东宣抚使一职!

当年为了韩绛有资格宣抚陕西河东两路,可是让他做了首相,昭文馆大学士。韩冈一任两路宣抚下来,宰相的资格都有了。所以就算是现在做不了宰相,也是要给他一个参知政事。

赵顼不愿让韩冈去。但他在妻子的话里听出了一点不祥的味道,虽然猜不到雁门关都破了,不过北方的局势应当没那么简单,否则也不会让韩冈走。

吕惠卿其实也是一个选择,不过他宣抚陕西,闹出了这么大的乱子,事后肯定要治罪。可若是让他再兼个宣抚河东,那么回师后就只能任命他为宰相了。留下这个成例,对国家的未来不好。但任命韩冈为宣抚,那绝对不行!

‘韩冈安抚河东。’赵顼以指画字,打算让韩冈恢复旧职,看了看皇后的脸色,犹豫着又添了一笔,‘安抚大使。’

加了一个‘大’字。

在今日的官制上,并没有安抚大使这个说法。但这就跟当年真宗皇帝为了宠褒王钦若,特意在资政殿学士前面加了个‘大’字,变成了资政殿大学士一样,赵顼是打算给皇后一个面子,反正也只是好听一点,至少比得授节钺、宣布威灵的宣抚使要好些,那可是真正意义上的代天巡狩。

只是皇后却不答应。

宣抚是临时的差遣,经略安抚使则是要在河东久任。向皇后只想让韩冈挽回了河东危局就回朝,哪里可能会答应他去顶替王.克臣的位置。就算加了一个‘大’字,但本质上看起来也没有什么区别。

将良将悉数调离边境,用了一群废物守边,现在辽人都打进国中了,还这么猜忌贤臣。何况太子怎么办?就这么让程颢去教?虽然前几日,程颢上殿的时候表现得也很好,很会教书的模样,但向皇后就是对他有成见,根本就不想儿子身边有这个人。

强耐着不说出心中的怨声,向皇后好言好语的规劝。

半个时辰之后,一从寝宫内殿跨出来,皇后的脸色转眼就黑了。

方才劝了半天,赵顼终于是给了韩冈一个河东制置大使的差遣,可以统领河东兵马,经略安抚使也从其号令。虽不是统摄军政、监司尽为其属的宣抚使,可也差不太多,且又是一个临时差遣。

至于参知政事的任命,赵顼并没答应,而是以员额已满为借口,换成了枢密副使——政事堂满员是宰相三人、参政两人,现如今宰相之位还空一个,而参政的位子则已经满了——却正是意料中事。

虽然不能说是称心如意,不过终究也能算是达成了目的,同时也瞒下了河东的败局,不让丈夫担心。但一番近乎买菜一般的讨价还价,丈夫心中那股隐隐的猜忌,向皇后感觉得很明显。不是针对韩冈,而是连她这个皇后也一并猜忌上来了。这让向皇后心里闷得发慌,更是倍觉委屈,她到底为了谁才这么委曲求全的?!

有这个感觉的并不只是她一人,殿中上上下下都察觉到了。

宋用臣小心翼翼的跟在向皇后的身后,大气也不敢喘。等到王中正、蓝元震、石得一等几名大貂珰纷纷赶来,服侍着她在崇政殿的内殿坐下来。待皇后喝了几口热饮子,神色稍稍缓和,宋用臣轻声劝道:“官家会明白圣人的苦心的……”

“明白什么!?”向皇后顿时又恼了,啪的一声就将茶盏重重的顿在了小几上,“要是都明白了,官家病情有个反复怎么办?!宋用臣,蓝元震,石得一,吾就在这里跟你们说明白了,河东现在的局势若有一字半句传到天子耳朵里,你们就可以去死了!!”

向皇后俏脸含煞,眼尾上吊,气呼呼的拿着三人泄愤。三名大貂珰连忙跪下,指天誓日,绝不会让官家听到半点风声。

三人其实都是宫中顶儿尖的大宦,都有了武职,属于两府的管辖范围。实际上就算是皇后,也不可能一句话就将他们给处刑。但万一天子真出了事,他们的一辈子也就完了。小命不一定会丢,可被降罪后下半辈子去守庙,那比死都可怕。

“王中正。”向皇后又转向另一位大貂珰。

王中正俯首帖耳:“臣在。”

“你也戒令班直,不可妄传一句,违者严惩不贷。”

王中正的地位与其他三人不同。以他的过往战绩,换作不是阉人,枢密院都能进的,向皇后自然不能对他喊打喊杀,口气也是缓和了许多。

“臣遵旨!”王中正行礼,“还请圣人放心。”

“放心……”回头看着福宁殿的方向,向皇后苦笑中满载着伤感,“如何放得了心?”
