\section{第44章 一言镇关月燎辉(上)}

“王子纯他们已经走了多久了?”

“三十二天。”

“想不到都一个月还多……唉……再过些天就是五月,田里可都要开镰了。”

“田地还是小事,有人料理,总不会放着不管。倒是临洮堡那里,到现在韩玉昆也没能攻进堡去。王经略他们若是不能回来,河州、熙州不知还能不能保得住……”

沈括和王中正有一句没一句的聊着,脸上都是忧思难解。窗外刮进来的风,多了一丝温热,已经没了春天的花草味道。

两人同在狄道城中,几个月下来也算是有些交情了。沈括虽然对阉人的态度跟所有的士大夫一样,一句‘敬而远之’只取后面的两个字。但如今狄道城中能说话的就只有王中正一人,闲得无事或是心里发慌的时候,也只有聊一聊天才能开解一下。

当然,他们聊天的范围也脱不出眼下的局势,却不可能在深入或发散了。

王韶、高遵裕追击木征,至今音讯全无。景思立被诱出兵,以至全军覆没。韩冈领兵救援临洮堡,却被阻拦在离着目标还剩五里的地方,始终不能寸进。

河州方向倒是顺利,苗授和姚兕姚麟也算是有些本事,没给如今烧得熙河路焦头烂额的火势上再添把柴。只是他们要钱要粮要军械的本事也同样不小,狄道作为转运的枢纽,沈括的工作一直都是让他忙忙碌碌,能歇下来的时候并不多。

王中正却是没什么事可做,熙河经略司上下早被王韶打造得如铁桶一般。而韩冈接手后,就算远在临洮堡外,照样让外人插不进手去。看到沈括每天只有区区一个时辰的闲暇,能坐下来说话,王中正都有些羡慕。若是每天能忙得没时间吃饭,至少不用因为有空胡思乱想,而听着衙门大院外的马蹄声就心惊肉跳。这才一个月的功夫,害得他鬓角都白了一半。

王中正现在想想,当初他跟李宪争个什么呢……有着罗兀城的功劳难道还不够吗?何苦贪心不足,硬是要到这河湟来!现在后悔都来不及了。

如果王韶有个不测,天子几年来放在河湟之地的心血,跟着横山攻略一样鸡飞蛋打。熙河经略司肯定完蛋,而他王中正王都知,也肯定都要被踢到荆湖以南的那个地方去。而跟到一定时间或是逢上大赦,就会被重新启用的外臣不同。他们这些宦官,如果不能经常让自己名字传到天子耳中,那么很快就会被人们所遗忘。而跟在天子身边的其他内侍,也根本不会在天子面前提到被贬黜的背时货的名字。

“如果王经略、高总管再没有消息了,京城就要有消息了。”

王中正叹着。他都在想着是不是要赶快给李舜举送点东西过去,也好在自己走霉运的时候,能有个人帮忙拉扯一把——如今天子身边的亲近内侍,也只有李舜举这个老实人可以让人相信。李宪、石得一之辈,那都是上边笑哈哈,下面捅刀子的主。

“景思立败亡的消息早就该到京中了,王经略和高总管失了音信的事,应该更早一步呈递上去。韩玉昆顿兵不进,肯定也会有人上报,沈秦帅、蔡运使,都要撇清责任,下面有递密折的也有好几个。收到这么多不利的军情,朝堂上要做决定也就在这几日了。”沈括好歹断断续续的也在京城待了几年时间,对朝堂决定边事处理方案的流程和时间也有所了解,“就不知道天子会有什么应对了……”

王中正舔了舔嘴唇,稍稍犹豫了一下,还是对沈括说了,“……罗兀城的事,当初天子后悔了很长一段时间。如果不是赵瞻硬是逼着退军,其实还是能保下来的。今次熙河的情况也类似。一天听不到被确认的噩耗,天子一天不会下决心放弃河州。”

“只要没有更坏的消息……?”沈括问着。

“只要没有更坏的消息!”王中正点头。

“……报…………”

一声拖长声调的急报传入耳中,一名身佩金牌的急脚在卫兵的带领下来到王、沈二人面前。

“秦州急报,十万西贼齐集柔狼山,预备攻打德顺军。领军者已经打听明白——是仁多零丁!”

听到这个消息的一瞬间,沈括的王中正的脸上同时失去了血色。

“糟了!”

“完了!”

也正如王中正和沈括大惊失色,当十万西贼寇德顺的紧急军情传到东京城后,两班宰执们齐齐被招进崇政殿中,朱漆的大门紧闭。但噩耗已经难以阻止的在东京城传播开了。

“那个都监本是德顺军的知军,如果不是他被调去熙河,跟着王韶糊里糊涂的出了事。党项人也不敢直逼德顺!去年他们在无定河边吃得亏可不小。”

“是啊,夺下河州又如何,老家都给党项人抄了。”

“河州肯定要撤军了。”

“要不是王相公硬撑着,熙河早就该撤军了。惨败啊……经略、总管都生死不明,还死了一个都监,折了上万兵马。真不知拖了这么久是为了什么!”

“还不是王相公不甘心,前两日,跟冯当世【冯京】,王禹玉【王珪】还有吴冲卿【吴充】在殿上吵了个地覆天翻,硬是说不会熙河不会有事。天子本都听着几位执政谏言就要下旨了,却硬是给王相公堵了回去。可现在呢……”

“都是好大喜功闹的!穷寇莫追的道理都不懂,竟然追到了雪山里面去了,把一路军事让个才二十岁的幸进之徒管着。不过弱冠的黄口孺子能有什么能耐,名气都是吹出来的……”

“不是虎口夺食吗……不对,那一位可是龙子龙孙。是龙口夺食!”

“也就一张嘴皮子和下三路的本事。现在好了,出了事那就原形毕露。”

“都是王相公闹出来的,尽是任用新进之辈。吕惠卿、曾布,还有现在吕嘉问,哪一个上来不是弄得天下鸡飞狗跳。换个老成一点的,根本就不会有今次的大败。”

外界的言论一面倒,但宫中始终没有消息传出来。一直到殿顶上的琉璃瓦开始反射着银月的辉光,紧闭的崇政殿大门终于打开了。

不论是东府还是西府,从殿中出来的宰执们的神色都是阴沉着。就算最为沉稳,这些日子以来,一直都在为王韶、韩冈辩护的王安石,也都是紧锁着双眉。

两名内侍跟着匆匆而出。大步走在前面的是李宪,在宫中以知兵闻名,后面的小黄门只有十七八岁,一幅包裹就在他身后背着,里面是个长条状的东西。只要对宫中之事稍稍熟悉一点,看到他们的模样,就能立刻知道,这是出外颁诏的使臣。

就在宫门口,李宪两人跳上马,带着一队班直护卫,就一片蹄声的往西去了。

“看来退兵定了!”

这一夜的东京城,不知多少人在弹冠相庆,也不知有多少人在忧心忡忡的望着西北。

听到德顺被西贼攻打消息已经数日了,蔡延庆都带队赶回了秦州去。陇西城那边靠着王厚的分派,才能保证着供给前线的粮草不至于匮乏。

但沈括和王中正都知道,秦州那边很快就不会有粮草运来了。而在预定的计划中,接下来的两三个月,也当是靠着今年河湟之地的夏粮来支撑。

巩州的屯田点马上就要开始收割,但熙河经略司和巩州的主要官员们都不在任上,王中正和沈括都不知道就靠着王厚一人,到底能不能忙得过来。

两人正忧虑的时候,却见到一人大步随风的走进官厅中。

一见来人,沈括惊得跳起:“玉昆,你怎么回来了?!”

“临洮堡那里怎样了?”王中正也急急追问。

“不必担心,西贼那边已经快断粮了,临洮堡更是稳如泰山。”

“所以让王舜臣在临洮堡下守着……玉昆,你也真放心得下!”

韩冈当然放心得下,临洮堡的局面已经稳定下来,不论是西夏人,还是宋军这边,在无法得到大量援军的前提下,都没有改变眼下战局的能力。有着刘源辅佐,被千叮咛万嘱咐的王舜臣不会闹出什么乱子来。

而熙河路本身,就像一座正在酝酿之中的火山,随时都有喷发的危险,韩冈是不得不回来。

“西贼寇德顺。这是项庄舞剑,意在沛公。西贼至今元气未复,现在只是要抱着不能让我大宋控制河湟的心思,才出兵攻打德顺。”

“而经略司在攻打河州之前,早就考虑过西贼会攻打秦凤、泾原两路的情况,也事先上报给天子要早作预防。调集到熙河来的两万军,都是在确认不会影响两路防御军力的基础上,才调动过来的。”

“现在秦凤、泾原两路,早就做好了防御准备,西贼根本破不了德顺军,就像他们攻不下临洮堡一样。”

在听说了仁多零丁领军攻打德顺后,韩冈就已经确定退兵的诏书很快就要到来。现在他必须要说服王中正和沈括,只有他们与自己站在同一条船上,才能将王韶留下来的局面给维持下去。

就算因此而开罪了天子,他也在所不惜。

