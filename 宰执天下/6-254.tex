\section{第33章 为日觅月议乾坤(12)}

吕惠卿站在半人高的穿衣镜前,年近六旬的形容正映照在清澈的银镜中。

价值千金的大幅玻璃银镜,即使是吕惠卿家里,也只有两年前给二女儿置办嫁妆时,才买了两面。

一面放进了二女儿的嫁妆中,一面则补给了早一步出嫁的长女。之后尽管几名宠妾曾经闹了两次,吕惠卿也没舍得再买——商家只收了进货本钱还要千贯出头的单价,可不是一笔小数目,在吕惠卿看来,没必要如此招摇。

想不到在城南驿中倒是随便摆着。即使这里是宰辅入京才能入住的院落,也未免太过奢侈了。

还是说这样的穿衣镜又降了价?

眼镜的价格降得飞快,每年就要打个对折。吕惠卿在长安,曾不经意的发现,连衙中的小吏都带着一只单片眼镜。现如今,水晶眼镜依然存在,可更多的还是玻璃制品。

只是玻璃这一门产业,朝廷每年的收益便是数以十万计。

从眼镜到镜子,从器皿到窗户,玻璃越来越常见,从天家和高门显第,到富贵人家,再到寻常百姓家,一步步的走进千家万户。现在城市里面,有几户人家没有一个小镜子?

吕惠卿至今也没想明白,韩冈为什么要将丝织的技术扩散出去。为了收买人心,为了网罗人众,这的确能说得过去,可怎么看,也觉得韩冈做得太大方了一点,那可不是铁路。

但是如果韩冈要公开其他赚钱的技术,或是提议改进已有的技术,吕惠卿是肯定要支持的。绝不会因为门户之见,而不让气学的成员去做他们最擅长的事。

对着镜子那个苍老熟悉的面孔,吕惠卿忽的一哼——外儒内匠,耶律乙辛的说法其实没那么荒谬。

没有人服侍穿戴,吕惠卿的手显得有些笨拙,扯了下襟口,腰带又给带歪了。

耐下性子将朝服的衣襟一点点整理好,镜中之人,眼圈青黑,一脸倦容,那是半夜没睡的结果。

双手捧着长脚幞头,端端正正的戴到了头上。再对着镜子,薄薄的双唇微微抿着,就算昨夜惊闻噩耗,也没能动摇到他的心志。

昨夜连夜进入城南驿拜访吕惠卿的官员,总共有三人。

相比起新党在京城的实力,依然站在吕惠卿一方的人数,已是微乎其微。只是有三个人,已经足够让吕惠卿了解到这段时间朝堂上的变化,甚至昨日宰辅们和太后的一番言谈。

探手拿起桌上的笏板,吕惠卿随即踏出门去。不论要面对什么样的局面,他都有了足够的准备。

轻车简从前往皇城,吕惠卿区区数人的队伍,撑不起宣徽使的凛凛之威。无人知晓,这区区数人的队伍,便是堂堂宣徽使的仪仗。

抵达皇城时,城下已经聚满了文武朝臣。大臣们三五成群,人群中议论纷纷。

毕竟不是所有人都知道昨天发生了什么,也不是所有人都已经下定了决心。

朱太妃回到了圣瑞宫之后,便再无消息传出。天子那边的反应也是毫无消息。太后的想法更是难以捉摸。

这些未知,已经让人觉得此刻安静的皇城,山雨欲来,狂风满楼。

而宰辅们议论的内容,同样掀起了轩然之波。似乎是刻意宣扬,两府辅弼在密室中的议论,变成了拿着铁皮话筒对全城在说话。

请求太后继续垂帘听政,宰辅们其实根本不必多此一举。

对绝大多数朝臣们来说,反对也好,赞成也好,都不如什么都不说。万言万当,不如一默。

只要没人不识趣的提起天子亲政,垂帘听政将会顺理成章的延续下去。

这本是应该是朝臣们心照不宣的一件事,可章惇、韩冈却带着两府一起上请太后继续垂帘。

不但让太妃的心迹昭彰于世,同时也曝光天子之过,最重要的,这就逼得朝臣必须选边站了。

如果是为日后计,当然不宜开罪天子,以年纪来看,太后总归活不过皇帝。

太后在世时有多么春风得意,皇帝亲政后,就有多么伤心失意。

眼下霸占两府多年的宰执们,皇帝一旦亲政,怕是一个都不会留下来。

可是韩冈为什么不担心天子亲政后的报复?

难道他会愚蠢到认为自己有定策救亡之功,可以让天子不敢动他分毫?

答案当然是否定的。熟读史书的臣子们,都知道桀骜不驯的功臣是皇帝最优先的处置对象。

那么问题来了——

皇帝还能活多久?

“官家近况如何?”

吕惠卿就听到身边有人在问。

身处人群之中,披着防寒斗篷,将朝服罩住的吕惠卿显得并不起眼。

不过当他看过去的时候,三人视线交错,那边的两人连齐齐脸色一变,匆匆散开。

吕惠卿倒不觉得他们认出了自己——看不见朝服,又是多年未上京,哪可能认出自己就是当年意气风发的吕惠卿——而是怕自己认出他们。

匆匆一瞥,若非熟识,怎么都不可能将人分辨,可吕惠卿却是当真认得其中一位。

那一人是位京师闻人,地位虽不算高,却人脉靠山都不缺,名声也不差。厚生司一坐多年,从判官做到判司,韩冈旧年的举主,判厚生司吴衍。

厚生司与太医局本是一体,如果是他,皇帝和太后的近况,的确是瞒不过的。

但要是皇帝身体不好,大婚是为了冲喜,消息早就会传遍天下了,又何须多问?

既然什么消息都没传出来,皇帝还有精神去看他未来的皇后嫔妃,那么担心皇帝寿数不永,眼下依然是多余。

吕惠卿并不觉得韩冈有本事去算太后和皇帝的命,不过是世人以讹传讹。

儒门子弟,原本就该是敬鬼神而远之。这一点上,吕惠卿与韩冈有着共同的语言。

但吕惠卿,终究是不可能跟韩冈走在一起的。

眼下最重要的,是告诉天子外面还有忠臣。

是郁郁而终,还是决不放弃,端得看是否看得到希望。

吕惠卿知道自己现在能给小皇帝的,也只有希望了。

……………………

垂拱殿中。

重臣们向太后拜礼已毕,各自归班。

这是常起居。

原本是宰臣枢密使以下要近职事者并武班,每日朝会的地点,号为常起居,又号内朝。相对于由不厘实务的朝臣参加、连太后、皇帝都不极少露面的外朝,内朝的重要性当然不言而喻。

而如今,内朝基本上已经变成了议政重臣共论朝政的场所,武班成员成了摆设。

韩冈曾经向章惇提议把三衙管军也归入议政之列,不过给章惇拒绝了,枢密院有发兵之权,而无统兵之重,而三衙有统兵之重,无发兵之权,将三衙管军纳入议政之列,枢密院将如何自处?

如何自处?

韩冈还想着将枢密院归入政事堂的掌握中,宰相兼任枢密使是有先例的。而将一干掌兵的太尉拉入伙,实际上等于是将这些武将纳入到政事堂的管辖范围之中。

三司使的任命,如今已经需要经过廷推,实质上已经操纵在政事堂。除了内库之外,大部分的财权都掌握在了政事堂手中。等到军权也同样在握,相权便可以与皇权抗衡了。

这个道理章惇当然明白,就是因为太明白了,所以才反对韩冈的提议,他缺乏韩冈的肆无忌惮,觉得应该在稳妥一点。

可惜,如果有着三衙管军的支持,韩冈可以更加轻松的面对皇帝,还有想要搅风搅雨的那一班人。

韩冈看着对面,那一班人中,现在还敢跳出来的,也就是一个吕惠卿了。

吕惠卿老了。

这是今天看到吕惠卿之后,窜过韩冈脑海中的第一个印象。

的确老了,相比起当年初见时的意气风发,几经沉浮,又在边疆蹉跎十年之久的吕惠卿,完全是一幅六旬老人应有的模样。

须发花白,面容甚至有些枯瘦,只是黯淡昏黄的双眼扫过来的时候,还是让韩冈的肌肤一阵发紧。

“老而弥坚啊。”

一旁章惇带着调笑的低语,却不是夸奖。

只差三岁的章惇,看起来比吕惠卿小了几近十岁。身为首相的辛劳,却没有带来多少风霜,相貌反而愈发温润起来。

或许是遗传,章惇的老父,耄耋之年鹤发童颜,前些日子还学了张三影一把,来了个一树梨花压海棠。被韩冈几位宰辅拿着开玩笑的时候,章惇的脸色可是有趣得紧。

现在看章吕二人的相貌,可没人能说他们是一辈人。

“吕卿在京兆数载,可是辛苦了。”

就在韩冈在想章惇他家那位真正老而弥坚的老夫的时候,吕惠卿已经上前陛见。

向太后照常例慰劳了他几句。但不提功劳,只说辛苦,太后对吕惠卿的成见当真深到了骨头里。

“关西一向难治,事务繁剧,臣以驽钝之才,只得勉强应付。每每想疏怠一些,一想到先帝和二圣的恩德,不知如何报偿,只能加倍用心。今日上殿,又得睹圣颜,实在……实在是……”

吕惠卿的话,说着说着忽的就哽咽起来。

韩冈顿觉不对,只听见吕惠卿带着哭腔:“前次见陛下,陛下还是孩童模样,时隔数载,今日再见,不意已是英俊少年。先帝若还在,看见陛下如此英姿焕发,可不知会有多欢喜!”
