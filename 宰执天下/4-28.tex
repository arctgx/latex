\section{第五章 圣贤需承传人荐(中)}

“……精穷坟典,倡行礼义,见在凤翔府横渠镇教授,聚徒百余人……”

赵顼摸着上唇处的髭须,低头看着御桌上的一封推荐张载入国子监担任判监的奏章。

“……其学尊礼贵德、体天明道,以《易》为宗,以《中庸》为体,以《孔》、《孟》为法,黜怪妄,辨鬼神……”

这个评价可是高得很啊。赵顼心中想着。

如果这是韩冈的奏疏,那一点也不会让人惊奇。可在这份奏疏的落款之处,赫然是王珪的名字。当然,在赵顼的案头上,也有韩冈推荐张载的奏疏,还有吕大防推荐张载的奏疏。

韩冈一心要举荐张载入经义局,吕大防的三个兄弟都在张载门下,他们举荐张载是在情理之中,怎么连王珪也一同来凑起了热闹?

赵顼一时间有些想不通,韩冈到底是什么时候走通了王珪的门路。但从赵顼对朝局的了解中,推荐张载一事,韩冈在政事堂中,恐怕也只能找到王珪这一个助力——王安石是绝不可能让其他学派的宗师,来干扰到新学在京中的统治地位。

一杆朱笔拿起,放下;放下,再拿起。犹豫再三,赵顼也没有决定到底要不要让张载入京任职。

这并不是要顾及王安石的问题,还关系到朝廷推广教化的根本大计。

以张载如今的声望,也的确当得起国子监中的任何一个职位。韩冈这位弟子的表现,更是让人期待起张载会如何教导那些个心高气傲、桀骜不驯的太学生们。而且早在韩冈之前,张载就已经名满关中,陕西士子闻风影从,这点赵顼也是知道的。

要不是因为他打算以王安石的新学为朝廷论学之本,这两年早就要招张载入京了,根本不需要他人来荐。

可就是因为新学已经成为朝廷,《三经新义》对经传典籍的新注释,也已经是国子监中考试时唯一的正确答案。赵顼不能不考虑到张载入京后,进国子监任职,会对此事造成多少反作用。

只是张载的学说,赵顼却是很有几分欣赏。

‘为天地立心,为生民立命,为往圣继绝学,为万世开太平’

如今流传出来的四句,赵顼当日一听之后,便为之激赏终日。虽然听说这几句出自张载的学生们——其中最后一句还是韩冈说的——但赵顼更清楚,这四句其实是对张载所传学术的总结。来自于张载,发自于横渠,并不是凭空而出。乃是张载几十年的悉心传授、谆谆诱导后,在关学门下弟子的心中得出的结论。真正说出这四句话的是张载,而不是他的弟子。

此四句气魄宏大,眼界深远,不宥于章句,而是直追本心大道。赵顼很喜欢这四句话,若是他的臣子们能以这四句为圭臬而行之不移,那他这位天子,也就当真能‘垂衣裳而天下治’了。

又考虑了片刻,朱笔再一次被拿了起来。判国子监不能给张载,但还有其他的职位。张载曾为崇文院校书,在三馆之中,不是没有位置安排他。

“官家!”一名小黄门让人在外面通报后,匆匆进了崇政殿中。

“什么事?”赵顼在奏章上振笔疾书,也不抬头,方才一阵犹豫耽搁了太多时间,他的御桌上还有厚厚的一摞奏章等待他批复。

“三皇子……”

“三哥怎么了?!”赵顼没等他说完,就厉声急问。手在奏折上一抖,顿时就是一滩朱红如血的印记。

小黄门偷眼看了一看赵顼的脸色,心里发着毛,换了个听起来稍微缓和一点的措辞:“三皇子身体有恙,圣人和宋娘子【注1】已经急传太医来问诊。”

赵顼眼睛都急红了,什么张载、什么举荐全都丢到了一边去,他的儿子生病了!而且事情既然会报到他的面前,就绝不是小病!心慌意乱的丢开摊了一桌的奏折,忙着站起身,匆匆的就往后宫去了。

……………………

将难题交了出去,韩冈和吕大防现在就等着天子的回音。

参知政事王珪、新任龙图阁侍制吕大防,加上判军器监韩冈,三人同荐张载判国子监。以他们三人的身份,这份举荐在正常的情况下,多半就能得到一个肯定的答复。只是以眼下的局面,在韩冈看来,赵顼应该不会让张载去国子监。

太学生们都是未来的官员,他们的教科书只会是科举中的标准答案,而如今科举,都是改以三经新义为蓝本。这就是王安石要‘一道德’的结果——想入朝为官,当然可以。但必须守规矩,从学术观点到治政策略,必须与朝廷的大方向保持一致。

“可是以子厚先生的性格,绝不会按照三经新义来教授学生。一旦子厚先生以气学大道来授徒,就会与天子和朝廷的本意相违背,必然会出乱子,太学生们肯定也会反对——不入科举的学问,国子监中又有几人会愿意花费时间去学?!一心想道的在国子监中不会有几人。”

吕大防默默点头,韩冈的分析其实也是他的判断。“但天子不会直接给否决。”他又开口道。

“微仲兄说得正是。”韩冈也同意吕大防的判断,“天子不肯批复荐章,在意料之中。只是以天子的性子,多半会改以其他官职做为补偿。”

不仅仅是因为张载的声望,还有以王珪为首的三名臣子的面子要顾及,赵顼肯定会做出一定程度的补偿,而韩冈所想要的就是这个补偿。

“旧年子厚先生是崇文院校书,如今重回崇文馆也是在情理之中。”

“不能去国子监传道,去三馆任职也是好的。三馆是清要之地,庶务极少。空暇下来的时间,也可以用来授业传道。比起国子监中,一门心思”

吕大防入朝为官的时间是韩冈的数倍,对官制的熟悉程度也不是韩冈能比,他的判断,韩冈也能信个七八分。

其实判监的这个职位也不好。司天监、将作监、少府监、秘书监、国子监、都水监,以及设立未久的军器监。这七个监司中,国子监名义上的主官为祭酒,都水监的主官是都水使者,其余的都是监——如秘书监监、将作监监——通常被尊称一声大监。

只是古来大、太相通,如果不是自己本官的品阶压在判军器监的差遣之上,少不了就会被人称作韩大监,悲剧一点就是韩太监了。韩冈也不希望自己的老师沾到这个让他不舒服的称呼。

一般来说,以当今天子的勤政,递上去的奏章只要两三天就能得到回复。可是接下里的两天,宫中却是毫无动静,而不久便传出来的消息,也解释了韩冈心头的疑问。

“永国公重病?”吕大防没想到韩冈会给他带了这个消息,“可有大碍?”

韩冈摇了摇头,脸色也是沉重,“永国公一向体弱,这一次病得不轻,不知道能不能撑过去。就算撑过去,日后一切也是难说。”

皇三子赵俊被封为永国公兼彰信军节度使,如今刚满两周岁,不过他是赵顼眼下唯一活下来的儿子,是整个皇室的中心,如果他能不出意外,基本上就是未来的大宋天子。

只不过,这不出意外的几率未免小了一点。

这个世代新生儿死亡率其实高得可怕,寻常人家基本上是一半一半,到了官宦人家比例会低上一点,比如王旖,她就有姐姐在王安石去鄞县任职时夭折,后来又夭折了一个儿子,相对于长大成人的两子两女,夭折率正好是三分之一。只是到了宫中,这个死亡率却是陡然增长。

英宗在登基前就有三个儿子了,谁都没有为传承帝业的储君担心过。但在英宗之前,因为仁宗皇帝无子一事,可是在朝堂上闹了近二十年的风风雨雨。再前面的真宗皇帝,也只有仁宗这一个年过四旬才有的独子。

也许是哪里犯冲,坐上皇位之后,大宋的皇帝无不是子嗣艰难。太祖、太宗的儿子,绝大部分都是在登基之前所生。而真宗只有一个,仁宗没有,英宗的三个儿子也是在登基之前生的。到了如今的天子这里,已经生有四个儿子,现在却只有皇三子一独存。这就不免让人怀疑,是不是这个皇宫哪里有问题。为什么不论是坐在御榻上的是哪一位,生了儿子都很难养活?

韩冈现在的心情有点压抑,如果赵俊当真出了什么意外,这就代表帝位的第一继承人又转移到了雍王赵颢的身上。

尽管历史上赵颢没有机会登上大宝——韩冈至少知道宋徽宗是谁的儿子、谁的弟弟——但韩冈所了解的历史如今已经改变了。

至少王安石这一次复相是在夏天,而不是春天。‘春风又绿江南岸’这一名句,很有可能不会再出现。说不准赵顼可能一辈子都没办法养活一个儿子也说不定。

赵顼现在还年轻,还有机会努力,但韩冈觉得,他已经是要往这方面多多考虑的时候了。

注1:北宋宫中多称呼皇后为圣人,而嫔妃则是娘子。

