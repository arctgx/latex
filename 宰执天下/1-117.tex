\section{第44章 文庙论文亦堂皇(二)}

【第二更,求红票,收藏。】

再一次被留了饭,张戬和程颢的热情让韩冈心中感到很温暖。今次能通过铨试,也是靠着他们的提点和教导,并没有因为韩冈是王韶所荐,而冷漠上半分。

几天下来,韩冈几乎像世交子侄辈一般被张、程二人关心着。张戬和程颢甚至把韩冈介绍给自己的家眷——这在古代,是极亲近的表现。两人的儿女都只有十岁上下,但诗书传家的出色教育,让几个小孩子的学问已不比普通乡儒稍差,礼节上更是过人。

在饭桌上,张戬和程颢不再提及有关一顷四十七亩的话题,说过了便说过了,答应了也答应了,纠结于此事不是他们的性格,而是转到了韩冈今次铨试的考题,以及刘易、程禹这两名在考试过程中使坏的令丞身上。

听了韩冈对今次考题的复述,张戬和程颢同时皱起眉头。“这题不算难吧?”张戬奇怪的问道。

“若真的要与玉昆为难,不会出这么简单的题目。”程颢也跟张戬一个想法。

“可学生听陈判铨话中之意,却是在暗指刘、程两位令丞的确是盘算着与学生为难。”韩冈不认为自己会看错听错,这是他的优势所在。

张戬又回想了一下韩冈方才说的题目,又与程颢对视了一眼,一齐摇头道:“太简单。”

韩冈也觉得纳闷,可他转而一想,面前两人皆是饱学之士,程颢更是有着宗师水平,对于经义考题的难度把握不住也不奇怪,这跟正常的初中数学题让数学系的博士生来评价难度是一个道理。不过这么想来,韩冈突然发觉自己的经义水准好像也变得不错的样子,自己不是也没发觉被人刁难了吗?还以为刘易、程禹故意把题目往简单里出。

张戬和程颢还在讨论着,也不知怎么的,他们从铨试的考试难度太低的这个问题上,开始怀疑起明经科的考题难度来。不过张戬是进士出身,程颢也是进士出身,纵然他们的经学水平远高于诗赋,但他们考得还是进士科,对明经科的考题并不了解。

张戬道:“过几日找一下近来几科的明经考题,看看出得究竟是什么题目。”

“是应该找一下。”程颢表示同意:“若是考题太过简单,朝廷的抡才大典也就失了选拔贤才的作用。”

“最好找九经科的,若是五经,三传,这些科目就太容易了。”

“若是九经科都不成,下面的各科就更不用提。”

明经科不同于进士科,依照考试所用经书范围,细分为五经、三传等好几个科目。三传是指春秋三传——《左氏》、《公羊》、《谷梁》,考题不会超出三本书的范围。五经则是指《周易》、《尚书》、《诗经》、《礼记》、《春秋》这五本儒家经典,考试范围自然就在其中。除此之外的开元礼、三礼、三史也皆是如此。而在这些科目中,以九经的考试范围最广,包括以上所有的各科要考的经典,自然难度也就最高。

听着他们的对话,看着越说越兴奋的两位师长,韩冈开始为下一科的明经科贡生们担心了。有两位鸿儒御史盯着,而且都是有资格成为主考官来主持明经科举试,明经贡生将要面对的考试怕是前所未有的难度。要是听到日后的明经比进士还难考,落榜的考生跑去叩阙喊冤的消息,韩冈一点都不会觉得奇怪。

“对了!玉昆,”张戬比程颢早一步从对明经科考题的讨论中回过神来,毕竟这里不是讨论事情的书房。想起还有客人在,他补救似的问着韩冈,“最后一道断案,你方才说过判的是阿云案吧?”

韩冈点点头:“正是。”

“登州的?”张戬又追问了一句。

“的确是出自登州。”

听韩冈如此说,张戬和程颢的脸色有了些变化,一齐问道:“玉昆你是怎么判的?是流刑?还是绞刑?”

韩冈不知张、程二人对阿云案的看法,但想来应该不会跟王安石一条路——也许为人温和的程颢有些难说,但以张戬的性子,和他对纲常的维护,他肯定是支持大理寺的判断,判阿云绞刑。

韩冈与王韶王厚讨论阿云案时,是从司法程序上,来阐述自己的观点——阿云与韦高是丧期为聘,未婚夫妇的关系是非法的,不当以此为前提来决狱。

但在儒门弟子程颢和张载前面,他不好这么说,因为此番言论已经近于法家了,而是最好要表现出自己的儒学水平。同时自己早早的看过有关阿云案的朝报,这件事形同作弊,韩冈也不想承认。心思一转,便不理法律条文,只往儒家大义上领:

“圣人之言,皆是以仁为本。阿云未伤人命,罪不至死,故而学生判的是流刑。”

“以仁为本?”

韩冈为之解说:“仁为本心,礼为纲常法纪,而中庸为行事之道。仁、礼、中,这三个字,是学生近来读书的一点体会。”

“仁、礼、中?”张戬轻声念着,韩冈的观点并不出奇,可单独把仁礼中三个字提出来的说法,却也不多。

“圣人之说本心是仁,一部《论语》,涉及仁之一字几达百处。而礼之一事,夫子说得更多。仁和礼是名教之根本,也是圣人在兹念兹的两个字。”

“那‘中’呢?”

“‘中也者,天下之大本也’。中乃行事之法,临事不偏、执两用中,此为中庸之道。”

虽然韩冈说得很简洁,甚至有些偏驳,但中庸的思想向来被程颢所看重,韩冈能看到这一点,并着重提出来,程颢听着有些欣慰,不禁点头微笑,不枉他这些时日的一番教诲。

韩冈的底子程颢看得很清楚,张载的这位弟子才智过人,善于为人处世,治事上亦有长才,但学问上却有所不及,对经义只是囫囵吞枣,并没有深入的钻研。无有大道守本心,程颢便担心这韩冈的才智会用到歪处去,故而他才不避嫌疑的悉心教导,希望让韩冈日后不会走偏了路。

韩冈的论断不算严谨,而且太过简单,圣人之道,岂是三个字就能概括的?但韩冈在求学中,能有所思、有所感、有所发,在程颢看来,已是难能可贵的一件事情。韩冈的心性虽难以继承张载或自己的衣钵道统,但若他能秉持‘仁礼中’这三条行动处事,却已不失为一君子。

韩冈见程颢点头而笑,心中亦是一喜。这代表他对儒学理论简单直接的归纳得到了儒学宗师的认同。

所谓‘我注六经’,将经典往繁琐里解释,一个‘若曰稽古’,就能扯出十几万字的注释,这是汉儒唐儒的习惯。而抛弃这些琐碎的注疏,而直接取用儒家经典的原文来证明自己的观点,以‘我’为主,而不是以‘经’为主,即‘六经注我’,这是宋儒的做法。

在此时,重新注释以《论语》为首的儒家诸经并不稀奇。泰山先生孙复便倡导舍传而求经,著《春秋尊王发微》,弃《左氏》等春秋三传于不顾;安定先生胡瑗,著《论语说》,徂徕先生石介有《易解》,公是先生刘敞有《七经小传》《春秋权衡》,亦是别出机杼,不惑传注。气学张载、理学二程,他们也莫不如此,皆是对儒家诸经有着不同于汉唐注疏、属于自己的见解。

韩冈也是一样,虽然他如今对九经的各部主要注疏,都能深悉大意,说个八九不离十。可他对这些扣着经典文字,一字一句加以注释,比经书繁琐了千百倍的注疏,却没有多高的评价。

韩冈一直认为,要想传播思想,理论是越简单越好。所以他就把儒学根本归纳成简单的三个字——仁、礼、中,而直截了当放弃了对经文的注释。只观大略,不暇细务,以这八个字为自己辩解,韩冈自认站在儒学大家面前也不会露怯。

“以冈之愚见,儒者之行不外乎守仁心,尊礼法,执中道。仁为礼本,以阿云案论,若韦高被杀,阿云自当斩,若韦高重伤不起,也是当处以绞刑,但韦高不过是轻伤,为些许微伤害一命,却有违仁恕之道。弟子观阿云之罪,杖遣过轻,杀之过重。杀人偿命,伤人服刑,所以学生便判了流三千里编管。”

仁为礼本,如果按照韩冈的想法,后世所谓吃人的礼教,便是只有礼而无仁,走入了邪道,并不是真正的儒家。如‘君要臣死,臣不得不死,父要子亡,子不得不亡’,这样的违反仁道的说法,便是对儒学最无耻的扭曲。

儒家的根本是什么?是仁。礼仅仅是纲常,是外在的规条。后世吃人的礼教,只顾维系礼法,完全背离了儒家仁的本心,这样根本不能算是儒了,而是彻头彻尾的邪教。就算给孔子多少封号都改变不了这个事实。

程颢认同韩冈秉持仁心的判决,不妄杀一人,比什么都重要。而张戬则有所不满,“律贵诛心,韦高虽未见杀,但阿云确有杀心。韦高虽是轻伤,阿云杀人未遂的罪名却不能宽贷。”

“先生说的是!”韩冈低头受教,并不与张戬争论。张戬愣了一下,随即便摇头失笑。若仅是杀人未遂,苦主轻伤,凶手也只会是流配而已。阿云会被大理寺判绞刑,则是因为她和韦高的关系。前面韩冈对此根本不提,想来也是不承认阿云和韦高丧期纳聘的未婚夫妻关系。

不过张戬也不想争了,还在吃饭呢,为一桩已经有定论的案件争论根本毫无意义。

