\section{第40章 帝乡尘云迷(六)}

“的确是有这个原因在。”韩冈点了点头。

不过更重要的是韩冈无意在门下给人做走马狗。先晾一下吕惠卿,日后说话才能硬气。

王安石在的时候,他都没有在王安石面前伏低做小,现在政事堂中的几位哪个够资格让他低头奔走于门下?从韩冈他一开始任官,就连推举他的王韶,都只会把他当做同路的盟友,从没有将他当成门客来看待。

荫庇门客和举荐贤才差别可是太大了。

王韶举荐韩冈,那是为国举贤,甚至是有求于韩冈的才能。说得偏激一点,得了官后,韩冈都不用去道谢。但他的几个门客,如魏平真和方兴,韩冈举荐了他们为官,日后见到了他正在外面疯着的儿子,都是要行礼的。除非他们日后能考上进士,成了天子门生。否则这个主仆关系一辈子都脱不了【注1】。

这就是差别!

虽然现在韩冈已经是官员,不可能再有什么主仆之分。可如果他轻易投效政事堂中的任何一位,不论是韩绛、还是吕惠卿,只要他靠着两人升了官,日后如果翻脸,那世间舆论不会关心是非,只会抨击他背叛。

而且韩冈更清楚,赵顼对自己的信任,是因为他从来都是与新党若即若离。要不然,他如何能说服因为流民图而震怒的天子?因为赵顼觉得他可信!

现钟不打去打铸钟,韩冈还没那么蠢!

不过这番想法韩冈虽然没有说出来,但王旁与他已经很熟悉了,哪能看不出来。犹有疑虑:“若是玉昆你谁人都不亲附,在朝中恐怕会成为众矢之的。”

“放心。”韩冈满不在意的笑着,“政事堂中的二相两参,内斗还来不及,哪有余暇来对付我?”

只要他韩冈没有正式插足进那汪浑水中,无论政事堂中哪一位都不可能做得太绝。即便四人同心,要将韩冈提出朝堂,还要过天子那一关。而就算过了天子那一关,也不过是外放一任州郡罢了,还能将他贬斥不成?!

而他韩冈再熬过一任资历,就能去次府一级的州府担任知府知州了。

秦州、渭州这等要兼任一路经略安抚使的大州,也许还要差一点,可绝对够资格担任带着钤辖、都监这等武职的要郡边臣。而再过几年,到了自家三十岁的时候,即便是担任路中监司主官的资序都算熬满了,那时谁还能压住他韩玉昆,不让他入朝?!

………………

“就算能压着提点三年五年,难道还能压着提点三十年五十年?”方兴坐在窗边,望了一眼酒楼下滑行而过的雪橇车,叹了一声。回过头来向坐在对面的,“真的要跟提点结下死仇,最好先给子孙在找条退路。”

魏平真深有感触的点了点头,方兴说出了他心中所想的——韩冈年纪上的优势实在太大了,以至于到了现在,已经大到没人敢于无缘无故的与他结下死仇的地步了。

就算在洛阳、大名和相州几位重臣,也不见他们专门针对过韩冈说话——虽然这可以解释成他们并不将韩冈放在眼中,但韩冈这一年来在开封府安置数十万的流民,可以说是一手稳定了新党的根基。要不是他的一番努力,王安石根本拖不到秋后,就要离任出外。这样的情况下,韩、富、文这几位还没有挑了韩冈出来,将他给整下台去,完全不见当年揪着吕惠卿、曾布、章惇大骂出口的样子。

如今当真的敢与韩冈过不去的,也就剩些茅坑里的石头,还有在御史台中将挑刺当成是为国为民的言官们。可那些奏章也只敢有事说事,并不见他们将话题推演开来,即便指责韩冈的人品道德上的问题,言辞中也有所保留,从没有将韩冈往死里得罪。就像当年吕诲弹劾王安石,不管有理没理,先列下十条大罪再说的情况,韩冈收到过的弹劾中一次也没有出现过。

这就是年龄带来的优势。

“还有提点的才干功绩。二十多岁的朝官朝中也不是没有,可谁也不可能去担心得罪他们的后果。”

韩冈日后进入政事堂的可能,比起现在学士院的几位翰林学士都要大,甚至大得多。

如果不能将韩冈一帮子给打死,现在跟他过不去,就等于给子孙留一个身居高位的死敌,保不准就破家绝嗣了。除非有着准备作着名垂青史的诤臣,将自己和儿孙都置之脑后的,他们才有可能跟韩冈过不去。

他们两人,还要加上仍在县学中督促着学生功课的游醇,昨日京中消息传来,他们三人已经确定可以任官。虽然都仅仅是从九品的判司簿尉,但官身就是官身。

为着一个流内官,两人努力了多少年,就算跟着宰相和枢密副使,都没能拿到手,争抢的人实在太多了。可跟着韩冈,却轻轻松松——不,回想起一年来的辛苦,他们的工作决不能叫做轻松,可付出的代价能有所回报,对于方兴和魏平真来说,就已经足够了。

端起酒杯,两人对饮而尽,相视一笑,平生夙愿得偿,哪里能不为之欣喜欲狂?

……………………

崇政殿的大门缓缓合上,从殿外刮进来的寒风被挡在了殿门外。

摇晃的火光安定了下来,但赵顼揉着额头的手却没有定下来。

几位宰辅刚刚离去,说了一通,基本上都是关于人事上的安排,让他很是头疼。

为了一个中书户房检正的位置,四人争得有些激烈。尤其是吕惠卿和冯京,互相攻击对方提名的人选,也就韩绛昨日刚上任,话少一点。

赵顼最终还是选择了支持吕惠卿。他要保持新法和朝政的稳定,所以他基本上都会支持吕惠卿。冯京、王珪如果不能理解到这一点,赵顼也不介意换一个更为合适的反对者。

不过赵顼相信他们能将调整过来,毕竟与王安石在朝堂上共事了五六年,应该已经习惯了。

“蓝元震。”赵顼叫着今日轮值随侍的内臣,“现如今京中流民情况如何?”

蓝元震正管着皇城司,不仅仅是京城之中,皇城司的探子,已经将耳目伸到了京府各县,只是不敢踏出开封府的地界。

蓝元震知道赵顼想听什么,立刻回道:“回官家的话,白马县虽然还要靠着朝廷的赈济,但县中的情况却是很好,百姓安足,人心稳定,县中的几个流民营也都平静无事。”

“白马县的事情就不用说了,韩冈非是百里之才,做得好不奇怪。”

韩冈虽然已经不是白马知县,他还是管着白马县中之事。这半年多来,赵顼担心会干扰到韩冈安置流民,甚至没有派一个知县过去,硬是让一个京畿大县的邑宰之位空悬。虽然这也是为了安置流民,但他赵顼为此事破例,也是顶着议论的,他待韩冈可谓是不薄。

蓝元震很少听到天子如此明白的称赞一名官员,不过放到韩冈身上,也不至于让他感到惊讶:“除了白马县外,开封其余诸县镇,流民总数也不过五六千人,皆已得到安置,不至于为乱。”

赵顼点了点头,神色也放松了一点,他可不想在郊祀大典前闹出事来。“前日朕下旨,招募在京流民去熙河、荆湖屯田,现在有多少人报名了?”

“仍逗留在京的流民报名者为数众多,不论是去熙河路的,还是去荆湖的。三日之中,都已经超过一千户了。”蓝元震知道他说的这些,天子肯定已经都从开封府界提点司的奏章中知道了,紧接着下去说道,“这两千户河北流民,皆是自愿,并无一人被逼迫。”

赵顼抬眼问道:“背井离乡,他们就这么放心?”

“流民们都说诏书上有着官家的鲜红大印,而且小韩提点也不会骗他们。”

赵顼微微一笑。他做了多少年的皇帝了,近臣们说的话,他一定程度上还是能分辨出其中真伪。蓝元震的前一句,是说着让他开心罢了,后一句才是实话,且也有怕他对韩冈心生不满的想法在。

但赵顼可不是会嫉妒臣下得人心的天子,韩冈文臣,岂足为患:“朕亦曾听闻,包拯任开封府,闻其上任,开封百姓人人喜乐,皆称包侍制即至,一城百姓可以安居无忧。看来韩冈并不差他多少了。”

“那是陛下慧眼识人。”蓝元震说话,不改内侍阿谀奉承的声口。

但这也是赵顼喜欢听到的,点了点头:“韩冈这一年来的确是辛苦了,换作是别人,朕恐怕没有那么多好觉能睡!”

他从御桌上拿起一本奏章,随手翻了一下,叹了一口气:“冯京、吕惠卿还有王珪,都在开封城中坐着,想不到还不如在黄河北面的韩绛会看人。”

注1:主仆关系,一旦定下,在古代社会真的是一辈子都洗脱不掉。比如后来的岳飞,他早年曾经在相州韩家做过庄客,也就是佃户,到了他成为统军主帅之后,见到韩家的人,也都是恭恭敬敬。

