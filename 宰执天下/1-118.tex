\section{第44章 文庙论文亦堂皇(三)}

【第三更,求红票,收藏。今天才发现,这一卷的卷名竟然弄错了。初六之卷写成了六二之卷,真是糊涂了。】

因为这一番议论,这顿饭吃了不短的时间。饭后,韩冈自张戬家告辞出来。正巧听着更鼓咚咚咚响了几下,敲了初更二刻的点。按后世的算法,应是过了九点的样子。若是在秦州,不论是城里城外,此时早就是一片黑了,看着星月光,听着野猫叫,除了更夫和巡城,再无一点人气。但在不夜的东京城,现在才是刚刚开始热闹的时候。

甜水巷一带是开封城东的闹市区,别的不说,单是小甜水巷的近百妓馆,每天夜中都能招来数千名寻芳之客。更别提附近林立的酒楼、店铺。

街市上灯火通明,亮如白昼,人如潮涌,声如鼎沸。悠悠乐声自小甜水巷中飘出,丝竹如缕,不绝于耳。转头向巷内看了一看,就见着一盏盏灯笼高挂,门头下,人影憧憧。就在这一瞥之间,就不断有人擦身而过,急急的走进巷中。

不少嫖客们都是租了马赶过来的,而初更时分,总是来的人多,去的人少,这让韩冈租马变得方便了许多。

骑在马上,有一搭没一搭与租马人说着闲话,一边看着周围热闹非凡的街市。吃饭的,逛街的,做小买卖的,满眼皆是人群。

即便这些天来天天晚上都能看到,但每一次看到东京丰富多彩的夜生活,韩冈心中总忍不住一阵感慨。即便是千年之后,夜色能比得上东京城的,也不过是一些一线的大城市,或是普通城市市中心最为繁华的几条街道。

抬起头。天顶上,已经看厌了的天狼星还在闪烁着,只是被周围的灯火压得若隐若现。而其他的星辰,自然比天狼星还不如,完全消失无踪。

天文地理都是连在一起说的,依照此时的理论,天上星辰的分野都对应着地上的九州。想学习天文,必须了解地理。可韩冈地理学的水平极为出色,但天文学却是连星星的名字都说不清。

这主要还是韩冈受到后世的影响太深了,看到天狼星就想到大犬座,看到边上的猎户座,却想不起来那颗红色的亮星究竟是参宿二还是参宿四。仅仅是隐约记得,猎户座中央三颗星组成的腰带,被称为福禄寿三星而已。

若是能把中国的星图传到西方,用三垣二十八宿取代古希腊四十八星座就好了。韩冈抬头望着被灯火遮掩住的无尽苍穹,这样想着。

低下头来,韩冈又回到现实中。自己的官身已经确定,但王韶那边又出了问题,他现在要面对的是两千里外的秦凤经略和兵马副总管。

不过这事倒不难!

窦舜卿、李师中是疯了,韩冈现在脑子里只有这个想法。

对于秦凤经略司对河湟战略下的绊子,韩冈虽早有所料,但也没想到理由会如此荒谬。窦舜卿的做法实在太不聪明。三百里河道上只丈量出一顷四十七亩的荒地,这不是疯了不是?!

王韶口中的万顷荒田其实只有一顷,李师中的无耻和窦舜卿的愚蠢所编就的谎言,危言耸听,骇人听闻,欺君欺到这份上,王韶实在是万死难辞其咎!但这样的谎言根本骗不过明眼人,其实很容易戳穿,韩冈乐得看他们发疯。

可韩冈也明白,谎言重复千遍也许成不了真理,但重复个三五遍就能给人洗脑了,关键是看谁在说。他这可是经验之谈,无论前世今生,皆是有过。若是赵顼身边的人异口同声都这么说,就别想大宋天子能洞烛千里,明察秋毫。一旦赵顼真的信了,王韶决没有好下场,自己也要跟着倒霉。

不过只要赵顼耳边的大合唱中有了一点杂音,那就完全不同了。王韶是赵顼亲自提拔起来的,他的《平戎策》也是先递到赵顼面前,赵顼看好此策,才交给王安石的。赵顼本身,也是期待着王韶能够成功。

从人性来讲,皇帝不可能喜欢听到有人说开拓河湟这项战略的坏话。人总是听到自己想听的,相信自己想相信的。如果在一面倒的攻击王韶的声音中,有一个不同的声音出现,那么赵顼就会犹豫,便不会立刻作出决断,肯定会再派亲信去秦州确认。

这样一来王韶便有了缓冲的时间,对于窦舜卿和李师中的谎言,他就可以从容的上章自辩。身为天子耳目,秦凤走马承受刘希奭必然被征询意见,不出意外应该也会为王韶说句话。一旦两方打起嘴仗,就不是短时间内便能吵出个结果。一旦拖到王安石出来视事,此番风波必然迎刃而解。

所以就要看程颢和张戬了,不知道他们能不能超越派系之争,为王韶争取一下时间。韩冈轻轻敲着马鞍,指尖弹在皮革上,发出哒哒的声响。租马人识趣的住了嘴,知道租他马的小官人正在想事情。轻抖马缰,走到前面去领路。

韩冈对程颢和张戬的人品还算放心。以他这些天来对两人性格的了解,相信他们不会昧着良心去附和窦舜卿的说法。即便他们不会支持王韶,但秉着公心、执中而论却没有问题,而王韶也只需要朝廷派人去秦州公正的测量田地,让事实可以说话。

说起来,反变法派虽然对均输、青苗都是众口一词的反对。但实际上王安石的反对者们却是分作两类,一类是利益之争,一类则是理念之争,并不能混而一谈。

利益之争,来自于身家利益被侵害的阶层,主要是拥有大量产业的士大夫、宗室还有京中豪商。青苗贷伤了他们放贷的收入,又影响到他们兼并土地,均输法让京城豪商——主要是各家行会的行首——无法再通过垄断入京商路来谋利,所以他们对青苗法和均输法皆深恶痛绝。

而理念之争,就是那些真心认为与民争利是不对的儒生们。他们认为与民争利有失朝廷体面,青苗贷应该贷,可不该收取利息,至少也得少收利息。这类人人数不多,但各个都有甚有名望。张戬和程颢都是其中一分子,甚至包括张载也是这般想的。

对于此,韩冈并不惊讶。张载是儒学宗师,又精通兵事,天文地理并有涉猎,但不代表他精于财计和治国。当年张载和众弟子们还正儿八经的讨论要如何恢复周时的井田制,以抑制如今愈演愈烈的土地兼并,韩冈的前身当时也在场,还听得眉飞色舞。而程颢程颐虽然与张载学派有别,观点相异,但也是一般的把周制顶礼膜拜,同样想着要恢复井田。

韩冈几乎想笑,居然是井田制!

也不看看现在什么时代了.虽然复古制、从周礼,是每一个真正的儒门子弟毕生的心愿——所谓‘郁郁乎文哉,吾从周’。但时代毕竟不同了,上古时一里之地九百亩,是如‘井’字一般分割土地,按照公田有无,平均分给八户或九户人家。而以如今的形势,哪里有那么多地皮再划分给平民充作井田,能做到清查隐田,平均赋税已经很不错了。

两个派别虽然反对变法的理由不同,但针对的目标却是一样,故而同气连枝,一起唱响反变法的大合唱。如张戬、程颢这般的理想主义者,看不透潜藏在暗流下的利益纷争,只知道为了自己的理念而冲杀在前。像他们这样的人物,往往名望甚高,又为人甚正,没人会怀疑他们是为自己的利益争斗,很容易就相信了他们的话。而利益阶层则是乘势而为,站在后面掀起冲击变法的一波波巨浪。

对韩冈来说,利益之争是没法调和的,他不可能指望文彦博、吕公弼他们会为王安石所赞赏的河湟拓边说好话,因为这件事不可能给他们任何利益,反而会让王安石的地位更加稳固。相反地,张戬、程颢却能用道理加以说服。

君子喻于义,小人喻于利。

韩冈轻笑了起来,这个道理,圣人说得还真没错。

没在路上耽搁,韩冈和李小六主仆二人很快就回到驿馆。

刚进门,驿丞迎了上来,一阵点头哈腰,堆成一朵花的讨好笑容:“韩官人回来啦?可吃过了没有?要不要小人吩咐厨房一声?”

韩冈讶异地看了他一眼,这一位城南驿中的主事,几天来对自己虽然是恭谨没错,但从无今夜这般卑躬屈膝。前面他从流内铨回来,正式得了官身,也不见他有何异样。而看看周围,坐在厅中的一众官人们投过来的眼神,也是又羡又妒。

“可有人来访?”韩冈只想到这个理由。

驿丞点点头,递过两张名帖,“一个是王大参的,一个则是一位章老员外亲自送来的。”

王大参?!韩冈心中一动,接过名帖一看,头一张的书款果然是王安石。参知政事的名帖拿在手中,也难怪城南驿的驿丞一脸的恭敬,左右赔着小心。

另一张则是章俞,看来他的那支慢吞吞的车队终于到了东京。进京的官员多是住在城南驿,章俞能找过来也是理所当然。

