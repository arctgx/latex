\section{第34章 为慕升平拟休兵(13)}

萧十三和张孝杰收到的情报很简单,加起来还不到一百字,但比起万字以上的公文更为重要。

辽军是靠了河东商人的帮助才打进的河东,虽然说其中有很多只是看在钱的份上帮了一点小忙,实际上并不会真心希望大辽夺取河东,但一人一点帮助,集合起来就让萧十三见到了暌违百多年的太原。而且自打进河东之后,这样的帮助就变得更加主动,许多县城乡镇都是他们帮着打开,许多不便携带的产业和财物都是他们收购的,许多道路都有他们来引导,这使得一时之间有更多的曾经有过往来的商人赶上门来涎着脸奉承。

可是自从太谷一败,萧十三一路退守到代州城下后,绝大多数的大辽之友一下子就不见了。就像他们出现一样突然。

不过宋辽两家和平了近百年,边境地区的关系如同渔网一样密不可分。就像宋人能画出朔州、蔚州的详尽地图一样,萧十三退守代州后还是能收到简略粗率却准确的情报——那是他自己的耳目。

尽管这些情报简略只剩下一句话甚至几个字,比如最新的这一份,不过是说忻州有近一万陕西禁军在剿除匪患,领军将领是名叫白玉的陕西都监,如此而已。

可是对萧十三和张孝杰来说,这简单的只用了一张一指宽、半尺长的白纸所写的情报,已经足够让他们对宋军的行动计划作出了判断。

能在陕西军中升任都监,白玉必然是一个能力在水准之上的将领。别的地方,萧十三不敢这么说,但唯独在陕西,萧十三却可以如此认定。四十年战火不断,十二年开疆拓土,每一名西军军官都经历过战争,每一名士兵都嗅过烽火的味道,那是一个庸人无法生存的世界。

白玉名声不广,萧十三没听说过他,他手下的两个来自西夏的党项幕僚也说不出他的战绩,由此可见,这位西军将领多半会对军功有着超乎想象的饥渴。

可这样的将领,被韩冈安排去了剿匪!

作为一个纯正的契丹人,萧十三很清楚他的军队最大的弱点就是无法承受大规模伤亡。他麾下的军队,全都是将帅的私产。部族军、头下军是部落族长和贵人们的,宫分军和皮室军很大一部分是尚父的,萧十三本人也有两千多只听从他吩咐的皮室正军。

在辽国国内,绝大部分显贵的地位都是跟他们能掌握的军队数量多寡有关。如果麾下的战力有何损伤,那么地位也免不了会随着下降。

就如这一回对宋作战,部族军只是少抢到一点,就能闹到尚父哪里,若是他们伤亡大了一点呢?归属尚父的宫分军、皮室军一旦损失惨重,整个西京道就会动荡不安起来,还可能更进一步的影响到大辽的安危,更不用说会尚父的心情会变得多糟糕;就是萧十三本人,也舍不得他的军队有所损伤。所以萧十三才会尽量避免与宋军的正面交战,尽可能的拖延时间,拖到宋人愿意开始停战谈判。

幸好宋国那边也是同样的情况。韩冈也一直在避免决战。也许是因为对帐下军队的不信任,但更多的当是因为一旦来自东京城的军队伤亡太大,让他不好回京后向朝廷交代——同为宰辅,又面临相同的困境,这点心思萧十三还是很容易看透。

在这样的情况下,韩冈的安排就显得很奇怪了。为什么他身边的会是京营禁军,而不是更为精悍也更为善战的西军?

不管从什么角度来看,剿匪的差事绝对要比在前线对垒安全太多,除非宋国的山贼能比大辽的十万甲兵要强,不然韩冈为了区区盗匪,把陕西派来的援军丢到了战场外,这不仅仅是牛刀杀鸡的问题,而是应该质疑主帅本人的能力。

萧十三和张孝杰不会去质疑韩冈的能力,也不觉得韩冈自大到会认为用不着那近万西军兵马出场,更不相信韩冈会不在乎由此而造成的大量不应该有的伤亡。

既然如此,那就只有一个答案,在忻州参与所谓剿匪的关西禁军的处境,比驻扎在大王庄小王庄正对面更加危险。

这危险究竟从何处而来,还需要多想吗?代州腹地和忻州只隔了一重五台山!

放弃了更为稳妥却比较耗时的修筑堡垒、徐徐而进的战术,而改用出其不意的策略,都是证明了韩冈心中的急躁,需要尽快结束代州的战事。

“当真是可笑,宋人是没招数了!”

当萧十三和张孝杰将这一次收到的情报向亲信将领公布,并透露了他们的判断之后,无论文官武将都是一幅嘲讽的笑容。

“这叫黔驴技穷。”

“想想看我们是怎么打下的雁门和代州!”

萧十三和张孝杰在僚属们的兴奋中交换了一个眼色,同时摇了摇头。

宋军采用的计策,并不是看着那么简单可笑。当韩冈并不吝惜人命的安排探马,当宋军开始修筑营垒、开掘壕沟,摆明了要正面对垒,代州城内城外都当敌人将会一步步的攻过来,如同绳索勒紧自己的脖子。

在这样的误会下,一旦尽遣主力与宋军对峙,那时候一支奇兵从五台山中突然钻出来,与前线的友军前后夹击,能让与宋军对峙的自家军队全线崩溃。要是一个不好,被歼灭了大半。那样的情况下,朔州根本守不住,宋军甚至能直抵大同城下。而一旦大同失手,西京道崩溃,南京道还如何保得住?

只是天可怜见,这一回老天爷终于保佑了他们大辽一回。韩冈的布置给他们先知道了。

兵贵出奇。但所谓的奇,就是冒险。当将要成为目标的被袭击方做好了准备,奇兵不再成为奇兵的时候,要面对的不是能让人撞得头破血流的磐石,就是能将一切吞没的陷阱。

“繁峙县最好再加强兵力,四千兵马还是少了。”

“说不定山中还有小道,得多遣人巡视。”

“瓶形寨[平型关]也得加强驻守的军力。”

“要不要在几个出口留出一点空当来?可以来个关门打狗。”

“不要冒险,只需要让他们出不了山。领军穿越深山,这是破釜沉舟之举,一旦失败,就是返回原路少说也要丢掉一半兵马。”

萧十三和张孝杰在噪声中交换计划时一点都不担心这是韩冈的狡计。比如剿匪的西军,说不定根本就不存在,说不定下一刻就在忻口寨中出现。萧十三和张孝杰对此完全不担心。

这段时间以来,俘虏的宋军有近百人之多。他们的口供中都没有提到忻口寨中有关西禁军存在。其中有不少人抱怨韩冈将京营禁军的主力留在营寨里,而让他们这些出身河东为主的骑兵出来拼命。虽然其中没有流内品官,但这么多只耳朵都没听说西军到了忻口寨,那么西军潜伏在此的可能性就很小了。

如果不是有这些口供印证在前,萧十三和张孝杰又怎么会轻易的就相信了那份情报中的内容?

在麟府军夺占了武州之后,萧十三不得不分兵返回朔州。不过由于从河北那边又加强了一部分兵力,这使得萧十三手中可动用的军队数量还算充裕。为了保证这支援军能顺利的通过狭窄绵长的飞狐陉,光是往沿途的几个军寨运送粮草,就费了大功夫,但现在看来这份辛苦没有白费。完全可以调出一部分,防备五台山方向的来敌。

当下面的将领、官员和幕僚不再喧闹,萧十三声音轻得如同对人诉说:“就让我们——给韩枢密一个惊喜!”

……………………

“太行匪患是困扰河东路多年的痼疾。”

“山河险要,地形崎岖,陡峭而复杂的山势,河东历任的多少文武官员想要进剿山中匪寨,最后都只能望而兴叹。”

黄裳的通告中,折可大看看折克仁,折克仁也瞅瞅折可大,叔侄俩交换了一个你明白我也明白的眼神,一同默不作声。

“枢密在河东经略任上的时候,曾经有过进剿匪患的念头,不过当时首要的目标是辽、夏两国,区区匪患根本不上台面。只能排到后面再后面。等到一切尘埃落定,终于能腾出手来,枢密又被调回了京城,连顺手打扫一下庭院的时间也没有。”

章楶摇摇头。那是因为天子不放心,又为了不影响到正在进行的宋辽谈判——生怕大规模的军事行动引起辽人的误会。

“等到这一回辽军入寇,韩冈仁心,想过收复这些积年老匪,让他们能与河东军民共赴国难。可惜这些贼寇没有几个懂得大义。”

盗匪的确不知大义。黄裳自己说得都想笑了。在这个时代,民族大义虽然没有后世说得那么明白,但由于五代时沙陀祸乱中原,立国后契丹、党项接连成为国家大患,对异族的敌视和不信任其实已经浸透到了大宋的普通百姓身上,在士人阶层就是《春秋》中的华夷之辨成为仁宗时代各家学派的中心议题。

不过河东匪患之所以没有赶着来受招安,完全是由于韩冈的‘吝啬’——主要是韩冈很反感杀人放火受招安这一套强贼做官的传统流程——既然韩冈不肯拿出实质性的好处来,那么绝大多数有名有姓的太行山贼哪个肯甘心响应他的号召,与同血脉同出身的河东百姓共体时艰,共抗辽军?反而是乘机劫掠地方,在辽人造成的灾难之后,又给当地百姓带来了第二次的灾劫。

“所以枢密说了,匪患刻不容缓。太原已定,现在为了忻州的稳定,必须要剿灭太行、五台的匪患。”

所有人都知道,忻州和代州之间隔了一座五台山,只剩最西侧忻口寨所在的那十七里山口有着坦坦大道勾连二州。

韩冈驻军在忻口寨,忻州就是他最可靠的后防。只要忻口寨不失,忻州就能维持稳定。更南面的太原府更不会有什么乱子。

“白玉是关西宿将,手握精锐万余。名将强兵,由他主持忻州剿匪,忻口寨当可无忧。”

黄裳的总结陈词,有着韩冈的称道,人人点头称是,似乎没人想得到忻州和代州之间只有一座五台山。

是的,只有一座五台山。
