\section{第32章 金城可在汉图中(七)}

韩冈看看左右。

厅室之中,人心惶惶,甚至还有好几个面色如土的。厅中人虽多,又已是春日,但还是让人感觉到了冬天的刺骨深寒。

只是议论了一下石岭关为何被破,好像倒是把人都给吓到了。

如果仅仅是太原知府的私心那还好说。但若是河东军内部的问题,导致了战斗力的急剧下降,可就完全不同了。只要明确了这一点,当然会让每一位河东文武官员都变得提心吊胆起来。

尤其是威胜军,从太原南下,可就是这里了。而直攻东京开封,占据世间最为富庶繁华的城市,对任何北虏都是难以抗拒的诱惑。

“枢副,当速向朝廷请求援军!”威胜军知军急声说道。

“这是自然的。石岭关到底怎么破的,为此寻根究底也挽回不了,先挡住辽人才是正经事。不过援军之事诸位可以放心,我出发时,枢密院已经在调遣人马,此时第一批应该已经启程了。”

韩冈的话让众位官员稍稍放心了一点,但依然是悬在喉咙里。援兵能不能赶得及,这还真不好说。毕竟太原近而开封远。

“那河外的兵马也得调回来,还有西军……”威胜军知军正说着,却不意发现韩冈瞥过来的眼神冷厉如刀,心脏猛地抽了一下,慌忙停了口,战战兢兢:“下官逾越了。”

韩冈笑着摇摇头,示意自己没放在心上,不必在意。可一众官吏哪个会当真,全都不敢再开口了。

十余人环坐厅中,却悄然无声。只有韩冈意态自如,喝了一口茶,只听他悠然叹道,“这一回总算是明白为什么后唐庄宗莫名其妙就败了。”

韩冈的话说得让人莫名其妙,这不紧不慢的态度也让人听着上火,威胜军的官员惶惑的眨着眼,都不知该如何接口。

“难道不是‘祸患常积于忽微,而智勇多困于所溺’?”田腴念着欧阳修的评语。河东的败局在开战前就已经决定,至少前一句正好能印证得上。

“那是国败,而不是军败。后唐庄宗之胜,胜在上下一心,其败,是败在失了军心。若军中将校皆忠于庄宗,仅仅是李嗣源、石敬瑭有反意又能作何为?北面的几个知府知州是否私心大过公心且不论,仅仅是将帅无能,难道其他人就没有问题?”

韩冈从身边的一名班直的身上抽出一柄腰刀,刀身上黑白纹路交织如花,识刀之人一看便知这是柄镔铁宝刀。

拿着刀,对着众人:“若是以刀来比官军。提刀的手是朝廷,将帅是刀柄,朝廷控制着将帅,而决定刀砍向何处。而士卒是刀锋,士卒越勇猛,刀刃也就越是锋利。一把名刀在手,便能万军辟易。而刀身呢?那是什么?”

将帅和卒伍之间是些什么人,答案人人知道,但厅中却没人开口。

“是杂阶的军校?”威胜军知军好歹给了韩冈一个面子。

“正是!刀身脆弱不堪,即便锋刃再利,也一样排不上用场。若是刀身坚固,则锋锐差一点也无妨。”韩冈笑了一笑,“你们也许不知道,马刀和斩马刀并不算锋利,甚至还有些钝。但刀身坚实,一刀下去,就算碗口粗的木头一样能砍断。若是杂阶的军校得力,这一仗不会打得那么难看。只是换个将帅,一两年的功夫,能跟皮室军打得有声有色的河东军,哪至于就变成了废物?!”

韩冈想说的是从队正、十将、将虞候,到都头、指挥使这一干不入流品的杂阶军校,也就是后世的士官。士官不得力,使得将帅们对战斗力的影响过于深重,而大宋为了防备五代之患弄了个‘将不私军’出来,这就使得一支军队的战斗力随着将领的变动,而急剧波动。

韩冈一直都认为自己很重视参谋和士官制度,但现在回想起来,他在河东的时候,在这方面做得实在太少了。

将为一军之胆,也是首脑,而下面的军校则是骨骼、血脉。当有能力有声望的将校纷纷被夺职、左迁,剩下的士兵即便依然精锐,也不过是一堆排不上用场的死肉罢了,何况换上了一群废物,士兵们的战斗力如何可能保持下去?人心都散了。

这是世间的通论,之前韩冈和黄裳、田腴讨论代州之失时,也正是这么想的。但现在回想起来,却是犯了大错误。一个合格的组织体系,至少不能将希望放在一人的身上。

封建军队在组织力上与近代军队有着天差地远的距离,参谋体系以及士官培训制度便是其中原因之一。

“明明士兵们都有不弱的战力,甲胄兵械更是当世无双,但仅仅是将帅无能,却变得人见人欺,连险关要隘都没守住,这并不能视为理所当然。”“在幕中,是幕僚不得力,在军中,则是军校不得力!”

韩冈发了一番议论,打了一个岔,看似不着调的抱怨,使得厅中的官员们还是莫名其妙的面面相觑,但各人的脸色却都缓和了许多。

见威胜军的官员们不再为辽人而惶恐,韩冈也不乱开话题了,轻咳了一声:“说回正事。辽人这一回南下,攻下河东对他们来说也是一个意外。否则,就不会在攻下代州后,有近十天的时间没有南下的动静,仅仅是劫掠代州乡间。还有河北,竟然被阻挡在保州、霸州而不能深入,对北虏来说,就是一个失败。这也是证据。可见其内部并没有做好开战的准备,只是为了配合兴灵之争,而不得不仓促发兵。”

韩冈一扫厅中,见人们都在用心聆听,喝茶润了润喉咙后便继续说道:“不过对我皇宋来说,雁门关的失陷也同样猝不及防。朝廷整顿兵马需要时间,基本上要到半个月之后才能有大批的援军抵达。在援军抵达之前,还要靠河东上下一心。”

“枢副放心,威胜军上下一切遵从枢副之命。”知军起身,带着众僚属向韩冈表态。

韩冈满意的点点头,示意他们都坐下:“再说眼前事。石岭关沦陷日前沦陷。从石岭关至太原百里有余,又是谷道,即便占据石岭关后立刻发兵南下,现在最多也只是前锋攻到了太原。至于辽军主力想要进入太原腹地,至少还有两三天的时间,甚至更久。”

两三天时间其实也很短暂,只是没人敢说出口。

“诸位可知辽军进入太原府后,会先往那边走?”韩冈用考官的眼光审视着威胜军的官员们,“太原四通之地,北虏会先攻何处?”

“是榆次!”其他人还在思考,一名官员却已经应声,“北虏入太原后,第一步当是榆次县!”

这名官员坐在下首处,只比县丞、县尉高一点,四十多岁,穿着八九品的青袍,听口音不是北人。

“你说说为什么辽人的第一目标为何绝不是威胜军,而是榆次?”

“榆次是井陉的出口,一旦北虏占据了榆次县,便能断了河北和河东的联系。倘若北虏直接南下,而不顾井陉,却有被河北军截断后路的危险。”

辽军的动向很容易猜,如果是在沙盘上一眼就能看得出来。不过现在没地图、没沙盘,想要做判断,就要靠平日的积累了,需要在军事和地理上有一定的认识。

韩冈甚至有几分意外,从这个官员的外表上可一点都看不出来,“你是本军的陈判官吧?”

之前韩冈刚刚住进驿馆的时候,赶来求见的威胜军和铜鞮县的十余位官员一个个都自报家门,但韩冈哪里可能全都记下来。只有知军、通判和知县记住了。其他人只认了脸和官职,加上一个姓氏。

那官员站了起来,恭恭敬敬的回话道:“下官陈丰,今忝为军判。”

“不是进士出身?”

官员若是有进士资格,自报家门时不会不提,东华门外簪花游街那是一辈子的荣耀,但这位却没有提,应该不是进士。

陈丰话声顿时磕绊了一下,羞恼的涨红了脸,进士出身的官员看不起非进士出身,在官场上也是常例,但当着众人的面,由枢密副使问出来,却还是伤人的很。很有些勉强的说道:“下官是治平四年明经科出身。”

“十四年了。”明经科虽比不上进士,但在官场上也算是有出身,为官十四年只为一下军判官,算得上是升的慢了,“应该怎么做?”

陈丰闻言精神一振,又是喜色上面。韩冈正盯着他看,这一下子对陈丰的看法却顿时低了三分。忽喜忽忧,这城府心性上差了。

不过陈丰心中的激动远比脸上表现出来的更胜一筹。他都四十多岁了,还只是一个下军幕职官,多半一辈子升不上去,至于转官那根本是个不切实际的幻想。

现在一个机会落在了眼前,陈丰怎么可能不激动。韩冈的幕府对他这一等小官来说,等于是龙门了。只要能跳过去,让韩冈满意了,虽不至于立刻就能飞黄腾达,但至少京官朝官还是有很大希望的。

“先要保住太原不失。只要太原城还在,北虏就不能肆无忌惮。”

“如何保住太原?”

“当速招援军。只要有大军在外,北虏便不敢随意攻城。”

“援军远来,一时间可排不上用场。”

“……也不必援军力战,只要让辽军和太原城中知道有援军就行了。”

陈丰虽不能说是对答如流,但每个问题,他也只是想了一想便能做答,这让韩冈很满意,可说是惊喜。

“表字呢?”

“呃。”陈丰一时没领会过来。

“我是在问军判你的表字。”

一阵狂喜涌上心头,陈丰连忙道:“不敢当枢副的垂问,下官草字公满。至公至正的公。满招损、谦受益的满。”

韩冈转头对威胜军知军道:“我来此仓促,制置使司连架子都没搭起来。现在身边缺一个掌机宜文字的,打算将公满暂借到制置使司中来,不知可否割爱?”

