\section{第15章 焰上云霄思逐寇(12)}

【更得越来越迟了,当真不好意思。不过今天晚上有空,下一更提前在十二点前后发。】

李仓低着头,匍匐在草木深处。自高有半人的草丛的缝隙中,向外面的道路上张望着。

头顶上,不断的有冰冷的雨水滴到他的脖子上,又流淌下来。藏身的草丛里面也尽是雨水。而且虫子蚂蝗都出来了,从树上往人身上落,咬着就不肯再放口。难怪宋人没有选择在树林中埋伏,而是直接在驿站上守候。这种地方待上半个时辰,差不多就能要了老命,哪里还能出来作战?

“走了没有?”丁安比李仓更早一步承受不住这里的环境,浑身不自在的扭着。

“全都过去了!”

听到这句话,丁安一下跳了起来,拍打着爬满身上的虫子。李仓也起身了,脸上还挂着两只吸饱血的蚂蝗。也没空用火或是用盐除掉,直接就扯了下来。半截在手中,半截还挂在脸上,在脸颊上划出两道血痕。就在他们跳起来的同时,附近的草丛中也有十几人站了起来,同样拍打着身体。

在守候在长山驿的敌军冲出来的时候,李仓第一时间就钻进了山道旁边的树林中。前进时站在全军的最前面,逃跑时当然就会落在最后,等敌军追上来,第一个死得就是自己。

聪明人不止李仓一个,打头阵的这个都,在看到从长山驿冲出来的敌军之后,根本就没有作战的胆量,都是选择了逃跑。不过只有少部分人沿着来路逃跑,大部分都逃进了山林中。

“是李家老哥!”一个年轻的士兵蹿过来,压低的声音透着惊喜,年纪稍长的李仓在士兵中有点威望。附近的十几人都聚了过来,李仓不知不觉中,就成为他们的头目。

“下面怎么办?”“回去会不会受军法?”每一个人都问着李仓。

李仓道:“反正后面也挡不住广源蛮,我们直接回中军去,罚不责众,李太尉也不会动军法。”

追杀过去的都是广源军,并不是宋人。但原本被他们这些大越官军看不起蛮人,在宋人的支撑下,却变得勇武无双。打算逼着部下坚持作战的都头,给逃军冲倒,还没来得及爬起来,就给人一下剁了脑袋。看到那些疯狂的蛮人,没人会认为后面的人能挡得住他们。

丁安望了望向南过去的官道:“会不会一直追到中军去?”

“没看李太尉离着我们有多远?近二十里的距离,怎么都不可能一口气追出去的。”李仓脸颊抽搐了一下,“我们一开始就是被丢下来的,要不然也不会让我们一直拼命的追,也不管后面能不能跟上。”

李仓虽然只是个小卒,但在军中混了十几年,很清楚这样的安排就是让他们去趟出敌军的埋伏,现在不过是成功了而已。能追上敌军,就得缠着让他们逃不了太快。追不上,那就去踩陷阱。就算是废物也要派些用场。

“谁让我们不是李太尉的嫡系!”

“走!”李仓不再多说第二句,提起长枪,向着草木更深的地方走去。

丁安连忙跟上去,而其他士兵也都跟过来,在陌生的森林中,谁也不敢落单。

……………………

提着韩冈所赐的一柄长刀,黄全肆意的砍杀着敌人。

从背后砍杀逃窜中的交趾兵,让他的感觉到越来越浓的快意。

黄全越来越觉得选择投靠宋人,是一个再正确不过的决定。对欺压他们的头上几十年的交趾人,广源州上上下下已经忍了很久。现在终于有机会一舒过往积怨。

当的一声巨响,两刀相交。

从交锋处传来的巨力让黄全连退了两步,他还没站稳,就又是一刀劈下。

一名壮硕的汉子挥舞着大刀,毫不迟疑向着黄全追击过来。黄全眯起眼睛,没有半点害怕的神色,只有一丝嘲讽的笑容。壮汉的长刀正要挥下,将眼前这位充分在前的广源蛮将砍杀。忽然背后一凉,顿时就没了气力。低头下望,只见胸口处探出一支沾满了血的枪尖。

这是这段山道上最后一个敢于反抗的交趾兵。比起之前遇上的两队交趾军,现在的这一队一开始还有着一份自不量力的胆量,甚至还有人试图组织起反击,不过他们的反抗,就像使用柴草搭起的堤坝,在洪水中转眼就被冲毁。

毫不容情的将跪下来求饶的一名交趾军官劈翻在地。黄全提着雪亮的长刀,在血泊中漫步。取得胜利的广源蛮军,正抢着收割他们斩获的战利品。

狭窄的战场上,已经没有站起来的人了。两个指挥的敌军前部,没让他浪费太多的时间就灰飞烟灭。逃了一多半,没逃掉的则都成了刀下之鬼。

“少洞主,要不要再追!”

黄全望了望看不到尽头的山道,摇着头:“撤吧。”

交趾人的前军和中军间隔得很远,自己这边追出了十里地之后,才解决了三支加起来还不到千人的敌军。再往下也许能撞上交趾军的主力,但他这边已经是累得没有了跑步的气力。而正在返回昆仑关的主力,不可能再赶过来帮助自己,万一纠缠起来,连个援兵都没有。

而最重要的,就是韩冈的一句吩咐——不要追得太远。

……………………

从归仁铺,李常杰一路紧追宋军,到了傍晚就驻扎在大央岭驿。

前军的失败,也只让他冷哼一声,并没有责罚的意思。虽然他也的确想追上宋人,在追逐中将他们一举击溃,乘势夺下昆仑关,但李常杰再糊涂,也不会没设想过其他的可能和应对的方法。没有追上的确很遗憾,但并不代表追不上就算输了。有着超过敌军多少倍的兵力,能使用的策略很多。

宋人的这一次的反击也在预料之中。都打到了长山驿,宋人留守昆仑关的那点兵不可能不用。而且一件事更加确定,宋军当真只有八百人,要不然这样重要的反击不会仅仅是广源军出动。

对比起黄金满手上的兵力,八百人实在太少了,主弱臣强。在高歌猛进的时候,固然无碍;可到了关键时刻,李常杰不信宋人能放心得下。要不然宋人也不会一看到自己这边得到增援,就立刻撤离。

而宋人的反击也就只击溃了前军。中军并没有受到冲击。虽然士气免不了会低落,但兵力犹存,足以压得住宋人。只要八百宋军无法离开昆仑关,下面的一步就很好走了。

一切还没有脱离掌控,可李常杰现在却是一脸的惊讶,“你怎么来了?”

来到李常杰面前的,赫然是宗亶。

宗亶没有在意李常杰的失礼,径自走了进来:“刘纪已经过了左江,广源兵过去了一半。渡船都控制在我们手上,也就没有什么好担心了。倒是这边……”

“怎么,你想说退军?”李常杰的双眼危险的眯了起来,其实这一路上,已经有好几个将军劝过他撤退,“看看外面的雨水,弓弩难以施展,山间也藏不了人。逃回去的宋军也不可能再有力气返身作战,只凭驻守昆仑关的那点兵,冲击不了我的主帐。”

宗亶毫不动摇,挥手示意帐中的其他人都出去,质问着李常杰:“这一仗有必要再打下去吗?”

“宋军才多少人?八百,连同黄金满的兵在一起,也不超过五千人。我们不止两万,连现在你那边的兵也能脱身出来,你说打不打?!”

“在谷地里,兵力的差距没有多少意义,我们也不可能攻城。”

“只要绕过昆仑关去就行了,可以上攻宾州,也可以回师昆仑关下。就八百兵,宋人还能分兵吗?还是说他们准备让黄金满守昆仑关?或是让他去与绕过昆仑关的奇兵对阵?”

一开始李常杰的打算就是想追击,如果追不上,就改成给昆仑关足够的压力,留在后面的那一万人才是关键。过去昆仑关的几次交战,从来都不是正面攻破关城,而是奇兵绕道后方,来前后夹击。李常杰不过是打算利用兵力上的优势,复制这一战略,他的本阵将宋军牵制在昆仑关,而留在后方的一万大军,则是要走小道直接穿越昆仑关所在的这一片丘陵地带,杀到关后去。

“那样还要冒多少风险?宋人并不是只有八百兵!他们的援军随时都能赶到。”

“援军?多少?”宗亶的质疑让李常杰难以遏制自己怒气,连续吃了那么多亏,眼见着就要走上胜利道路的时候,宗亶竟然还反对,“他们从桂州一路南下,不休息就投入战斗,能奈何得了谁?我们不是刘永那蠢货。只要先一步拿下昆仑关或是宾州,就不用怕任何援军!”

“宋人迟早会派更多的军队来,不论打下的是昆仑关还是宾州城,最后都得放弃。逼得援救邕州的宋军狼狈而逃就是赢了,他们都没能进邕州城!”宗亶在盛怒的李常杰面前保持着冷静,“眼下多损伤一人,抵抗南下宋军的兵力就少上一分。这一次,来的只是援救邕州的先锋而已,并不是讨伐的大军。等到宋军大举南下,要对付可就是数十万大军!”

“要当真是数十万大军反而好了。”李常杰脸上的怒容消失了,一下变得平静无波:“我问你,对大越来说,哪个更危险?是三五万的军队,还是三五十万的大军?……我们绝不能让宋人小瞧大越!”

