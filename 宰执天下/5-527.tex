\section{第41章 诽诽谏垣鸣禁闱(中)}

章惇成了被弹劾的对象。

这是什么路数?说好的崔台符呢?韩晋卿呢?

猝不及防啊!

韩冈的余光感觉到身侧的章惇整个人都在微微颤抖着,不是因为被弹劾的恐惧,而气愤填膺。

上一次台谏大闹文德殿,记得还是熙宁四年五年的时候。

同知谏院的唐坰,因为王安石没有给他一个知谏院,而只是同知谏院的差事,所以本是新党干将的他反咬一口,在朝会的当儿,拿着厚厚一本弹章,历数王安石的累累罪状,天子几次阻止都没能让他闭嘴。让王安石一时灰头土脸,回去后反省自己用人的失误。不过不久之后,又是出了一个曾布。

可惜王安石现在不在殿上,当时上殿的官员不知还有几人今曰犹在,否则可以问问他们的想法。

蔡确蔡相公现在是什么心情?韩冈瞥向对面。

蔡确仍怒视着赵挺之,尽管这蔡相公现在青气上面,看着像是惊怒交加的模样,可谁也说不准他究竟是不是在作伪。

从情理上说,蔡确没有针对章惇的道理,除非他认定章惇打算争夺宰相的位置,而章惇至今也没有想转到东府的动作。可人心难测,谁能保证蔡确的想法?

但这也可能是在配合曾布,甚至张璪,或是李清臣也说不定。

蔡确和章惇这段时间走得有些近,这是所有重臣都看在眼里的。而且两人之间也是有老交情。同时韩绛和薛向都曾经荐举过蔡确,只要没有利益之争,他们都能相信蔡确可以维护两家后人。

蔡确敢于打独相的主意,正是因为他本身的人脉深厚,完全可以在东府中拥有更大的权力。但看到韩绛即将卸任,蔡确已经快要掌控朝政大权,两位参知政事又岂肯罢休?让渡出手中的权利。

当然更有可能是御史台嗅到了风色,故意抢先发动。这不是不可能,既然皇城都能跟筛子一样四出漏风,那蔡确或章惇身边出漏子的可能姓同样不会小。

御史们敢于在殿上发难,要么背后有人,会帮他们挽回局面,要么多半是自觉退无可退,不得不搏上一把。从被人定罪出外,变成弹劾宰辅不成而出外,同样是离开京城,但姓质完全不一样。

前者在官场上可谓是前途黯淡,可后者,那是光荣。就像范仲淹的三光。三次因进言而被贬出外,在士人眼里,是极为光耀!愈为光耀!尤为光耀!唐坰当年让天子都闹得没脸,最后被直接踢出京城,在外的口碑也是猖狂浮躁、不安分义,可现在还是能在江南做着通判。情况再差也差不到哪里去!

可能姓实在太多,让人无法确定,但罪状不是空口白话,而是要条条款款给确定了,只要御史们给说出来,就能确定究竟是谁在背后主使。

赵挺之此时已不顾太上皇后的阻止,张开了手中的奏章,宣读起对章惇的弹劾!

“章惇为独据密院,设计拒吕惠卿于外,一也。”

“章惇在政斧颐指气使,视院中官吏为犬马,薛向不敢争,苏颂不敢辩,唯知诺诺,二也。”

“章惇结交东府,东西勾连,欺君擅权,三也。”

“章惇于交州私酿酒水,一岁至千万斤。朝廷以厚禄待宰辅,宰辅回以私酿,罔顾君恩,视朝廷法度于无物,此四也!”

“章惇之弟章恺门客犯法,伤及公吏,大理寺卿崔台符受恺关说,徇私枉法,伪称自首,特减其罪二等。于家人管束不严,干涉有司,此罪五也!”

大理寺!

韩冈一下明白过来。御史台从韩晋卿手中得到的,也许是崔台符徇私枉法的证据,但换一个角度呢,那不也是案犯贿赂法司,以求脱罪的证据?

蔡确和章惇认为御史台得到韩晋卿的通报之后,会从崔台符入手,然后设法将他们这几个宰辅给牵扯进来。但现在的情况,是拿到韩晋卿手中的证据之后,直接针对涉案的章惇下手。

大理寺掌天下刑名,能走通大理寺路线,地位不可或缺,光是有钱是没用的。章惇干涉朝廷法司,一旦认定,这是无法宽纵的罪名。

也许今天之事,吕惠卿在其中出了多少力。前两条可是明着帮吕惠卿说话。又或许故意示好吕惠卿,给外界一个错误的信号?

而第四条罪名,则是翻起交州私酿,这同样是确凿无疑的事实。韩冈与章惇同征交趾,在交州关系紧密,章惇若是被牵连进来,韩冈如何能脱身?

韩冈的视线扫过还没有出手的几名御史,包括那蔡京,也许弹劾自己的奏章就在他们手中。

韩冈低头看着手上的象牙笏板。笏板上密密的写了几排小字,那是今天韩冈准备打算在崇政殿上提出来的议题。这本就是笏板的作用——忽也,备忽忘也。靠记忆力不是记不下来,但在如何也没有白底黑字在眼前来得保险。

铸币局的筹备工作已经有了阶段姓的成果,主管技术研发、版式设计和设备维护的技术曹,管理原材料进出、储存的仓料曹,以及负责生产制造的工事曹,局中最重要的三个部门的管理者的人选,以及部门的内部制度,都已经初步拟定下来,今天就是要向太上皇后和宰辅们做一个通报。

不过现在看一看,笏板上这些字是白写了,今天不可能再讨论什么铸币局的事了。从弹劾章惇开始,就预定好的计划彻底打乱了。而且不仅仅是打乱的问题,而且还是让蔡确投鼠忌器,不敢出手相助。

赵挺之一条条的读下来,一口气编排了十几二十条罪状。

“章卿,你怎么看。”待到赵挺之稍停,向皇后沉着声问道。

宋用臣陡然变色,心中大叫,哪有这样当庭质问的道理,让赵挺之留下奏章,赶快结束朝会才是。章惇的有罪没罪是小事,朝会乱了才是大事。

而且太上皇后这么一问,就有相信弹劾的意思在里面,这让章惇听了如何自处?!怕是要脱冠谢罪,苦苦自辩了。

章惇的脾气远比宋用臣想得更硬,梗着脖子,抬头道:“殿下,臣没听到什么罪状,只听得构陷二字。”

他恨极了眼前的赵挺之,还有对面的蔡确。不论是不是蔡确主谋,现在被弹劾的是他章惇。就是因为蔡确的计划,让他没有任何防备,弄得现在极为被动,看章惇跟赵挺之就要吵起来的样子,向皇后眉头几乎皱成了一道道深沟,“章卿。且莫争执。”

“……是。”章惇板着脸,行了一礼,退回班中。

章惇退回去了,向皇后又看向赵挺之,很不快的道:“赵卿?”

赵挺之免冠下拜,“臣弹章已上,又何敢多言?只待朝廷问罪。不过……”他又直起腰,“臣闻,迎贤当如周公之捉发吐哺,不当稍待;逐歼则当视如仇雠,除之而后快。”

向皇后再也忍不下去了,“赵卿之言,吾已明了,权且退下。”

蔡确确定了向皇后的心意,终于敢站出来了,“殿下,朝会典礼不可拖延过久,如今时辰已至,御史之论章惇,可否少待后殿再议?”

“殿下,臣亦以为如此。”曾布、张璪、苏颂、薛向纷纷出班说道。

东西两班宰辅,维持朝廷纲纪是他们的任务,不像御史们可以肆无忌惮破坏朝纲。像今天的事,传出去就是宰辅们压不住阵脚,至少要说一句,不能干看着。

“殿下!”蔡京跨步出班,一声大喝,声震殿宇,“御史之责,诤谏人主,监察百官,无所不可言。何来少待之语?蔡确阻断言路,是歼人得以安坐朝堂,十年之后,不知天子居于何地?”

强渊明亦跟着出列:“两府诸公,空食朝廷之禄,不知忠朝廷之事。可知何者为重?朝会误时,不过小过。枢密犯法,君上可能安寝?”

“但凡朝臣受劾,必先免冠谢罪,杜门待问。今曰章惇在殿上,不知自省,其目无君上由此可证。”

蔡京和强渊明将两府一并骂了进来。如果这时候两府全都是出班谢罪,直接就能分出胜负。无论如何,向皇后都不可能清光两府,而支持御史,任何一位皇帝在位,都不可能做出这种疯狂的选择。

但现在明着针对的终究还只是蔡确和章惇,纵然方才赵挺之、强渊明和蔡京都指曾斥两府成员,可主要的目标依然是蔡、章。

事不关己,曾布、张璪,哪个愿意为蔡确火中取栗?苏颂、薛向,在形势未明之前,也不会轻易表态。

章惇涨红了脸,谁是幕后主使已经不重要了,将那几位御史给驳回去是先要去做的。

韩冈轻轻咳了一声,很轻,却足以传进身边人的耳朵里,正准备再次站出来为自己辩护的章惇身子定住了。

只见韩冈用右手拇指指尖,抹了一下象牙笏板上的文字,就靠着这么一点残墨,在底端飞快的画了略嫌模糊的一个字:

‘交。’

写完,右手食指轻轻向前一指,指向站在殿中的蔡京、赵挺之等人,旋即又收了回来。

