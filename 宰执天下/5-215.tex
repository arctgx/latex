\section{第24章 缭垣斜压紫云低(七)}

冬至日的前一天,韩冈坐在太常寺的花厅暖阁中,屋外雪落无声,下了一夜的暴雪已经渐渐收止,面前则站着一名青袍小官,刚从屋外进来,被冻得脸青唇白。

还没到冬至郊祀的日子,韩冈为了以防万一而做的准备,倒是出乎意料的提前一天派上了用场。

不过在他事先安排下的急救队派上用场之前,韩冈先是在瞪人:“青城行宫的马厩被雪压塌,去找群牧司,怎么找到厚生司这边来了?”

正常传话报急,应该派个会说话、有条理的积年老吏来。寻常递送公文的差事从来也不会让官人来做。可韩冈面前的这位枢密院派来的传话者,明显是荫补出身,二十上下的黄口孺子。被韩冈眯起眼睛盯住,就像是被鹰隼盯上的老鼠,舌头都开始打结。

结结巴巴的费了好一番功夫,才让韩冈听明白,南门外的青城行宫,不止是马厩,连兵营也塌了,人和马都伤了不少。

韩冈两条修长浓黑的眉毛皱得越来越紧。

祭天时的圜丘就在青城行宫中。临时驻扎在行宫外围的军队,是为祭天大典而准备,总共集结了上万兵马。

记得六年前,韩冈还在开封府中担任府界提点的时候,也曾经觉得行宫外的军营有点破败,需要用点功夫翻修一下。但那一片军营,不过并不属于青城行宫的内部建筑。开封府中也没有多余的经费,最后只是草草的将屋顶给修了一下,只求不漏雨雪。

眼下一场只持续了一天的雪——尽管是暴雪——就让军营墙倒屋塌。该不会从熙宁七年之后,就没有再整修过吧?

韩冈摇摇头,不是否定,而是无奈。就他所知,开封府里面的官员做得出来,毕竟坐在那几个职位上的人换得太快了。

不过疑惑也好,感慨也好,眼下人命关天,没时间给韩冈多耽搁。当着枢密院来人的面,韩冈提笔签下了手令,直接调了一队人马去南门外的青城行宫抢救伤员,之后才派人去政事堂通报。

带着手令的小吏和枢密院的官员都出去了,韩冈揉着眉头。不是因为灾情,而是为了见鬼的官僚主义。

厚生司辖下的医疗人员如何应对大灾之后的紧急救治,一切都有预案在。依照韩冈组织人手与开封府一同编定的条例,开封府辖下的每一座医院都组织了一支紧急救难队。每一支队伍,都有至少一名翰林医官主持。开封城内的两支急救队,更是各有三名翰林医官分任正副职。

依照预案,今天的这场雪后,如果有房屋大面积倒塌,造成大量的伤亡以至于来不及送往医院,只能在现场进行急救的话,开封府直接联系东城、西城两所医院就可以了。只需事后到厚生司这边报个备,补个手续归档,并不需要韩冈的手令或是厚生司的正式文书。

但事情一旦牵涉到军队,这手续就麻烦了。先是受灾的那一支队伍上报三衙,三衙转呈枢密院,枢密院论理还要跟政事堂联系,然后让政事堂给隶属于中书门下的厚生司下命令,出动救难队。今天好歹是绕过了政事堂,总算节省了一点时间——所以韩冈出的是手令,而不是以正规的格式盖印签押的公文。

只是灾情告急的消息在三衙和枢密院中绕了一圈,耽搁的时间依然不止一两个时辰,很可能就多了几十人枉死。如果能绕过两个衙门,至少绕过枢密院,情况可能会好一些。不过绕过枢密院跟军中联系,绝对是文臣的大忌。就是小小的急救预案,韩冈也不方便与三衙直接交流。可若是事情要经过枢密院,那么结果还是落得今天这样。

如果有单独的军医体系,倒是能省下一份心来。但大宋的军医,从来都属于太医局管辖,而不是军队。如今军医的人事权,也是由厚生司掌控的。

很长一段时间以来,太医局的医学生都要轮班去军营中坐诊,或是派到外路担任医官,医治受伤或生病的官兵,眼下则又多了一份在医院里实习的工作。

在厚生司成立后,原本的疗养院也转到了厚生司的旗下,需要住院甚至隔离的士兵,都会转到两所医院辖下的疗养院中安置,并不存在单独的医疗体系。

但现在看来,还是组建一套军中医疗和急救体系比较好。一方面是避免再出现今天的情况,另一方面,一直都进展缓慢、让韩冈心烦不已的人体解剖学,或许可以抛开旧有的束缚,能有一个大发展也说不定。

想到这里,韩冈倒有些坐不住了。找来纸笔,开始在纸上打起草稿来。

韩冈动笔写字,下面的官吏不敢打扰。他们也不知道韩冈在写什么,只是知道了青城行宫出了事,就已经让他们人人大惊失色。

临到郊祀之前,参加大典的士兵出了意外。虽然不知伤亡的具体数字,但枢密院都派人来了,数目应当不会太少。

南郊祭天——吉礼、凶礼、军礼、宾礼、嘉礼这五礼中吉礼的头一条大礼——却是以死人为开场,终究不是吉利的事。听到这个消息,太常寺中的官吏们没有哪一个能掩去各自脸上那一分或多或少的忧色。

韩冈倒是不在意什么预兆,正是思绪泉涌的时候,只用了小半个时辰,就将一份奏章的底稿给起草完成。不过具体的细节还要再斟酌一番,得与人商讨过后,再上书天子和政事堂。

放下笔,看着纸上涂抹修改后的文字,韩冈抿起的双唇有着一丝自嘲的笑意。主动放弃一部分权力,对于一个衙门的主官来说,不能算是称职,传出去,下面的人说不定要骂娘。但韩冈的心思,并不是局限在小小的一个衙门里。

接下来就该与人商量一下细节,好好推敲一番。只是韩冈抬起头,看看左右,这才想起来,今天苏颂并不在这里。

苏颂今天不仅没有到太常寺,光禄寺那边也没去。而是告了病,请假在家,没有来上工。也不知他是真生病了,还是干脆想偷懒。韩冈估计多半是后者,所以就随便派人去苏家探问了,尽一份人情。

而太常寺这边的官员中,也有六人赶在今天请病假,正好占了总数三成。如果加上胥吏,那人数就更多了。

遇上大雨大雪或是大冷大热的极端天气,请病假的人就特别的多,韩冈也是见怪不怪了。反正太常寺是清水衙门,人多人少都不会耽搁正事,即便是在郊祀之前也一样。

对于这一点,身为太常寺的主官,韩冈不知是该庆幸还是该悲哀。不过厚生司倒没这般惫懒的模样,对于灾害天气,必须要安排专人值班,以便及时做出反应。在韩冈还掌管着大宋的医疗机构的时候,他手下没人敢违反他的命令。

站起身,推开紧闭的房门,一股寒流便立刻冲入温暖的厅中。

屋外白茫茫一片,雪虽然没有昨夜那般大,但还在下着,天也是阴阴的,完全看不出到底是什么时辰了。反倒是地面的积雪映着光,倒是更亮一点。

说起来这雪下得还真不是时候。

昨天白天的时候还只是天阴而已,但到了入夜后,就开始下雪了。鹅毛般的雪片铺天盖地,尽管只是一夜而已,可街道上积雪就有一两尺深。

如果这个天气再持续半日,明天的郊祀就不得不停止了,只能改为明堂之礼。仅就此事来说,对韩冈倒是不错的消息。他并不是很在意那点参加郊祀的赏赐,能免了那等在寒风里受冻的活计,却是一桩好事。

“瑞雪兆丰年啊,如果不是积雪压塌了房屋,这时节下场大雪还是件好事啊。”韩冈在廊下感叹着,真心希望明天也可以不用太劳累,正好可以去城南驿拜访一下王安石。

之前曾让人头疼的班列问题,因为王安石的谦让而没有翻起大浪。赵顼是想让王安石参加郊祀,甚至还亲自将他的位置安排在王珪之上,可王安石在崇政殿上坚辞不受。但他也没有打算站在王珪之下,自称身体不适,不能参加郊祀。让王珪松了一口气……韩冈另外还觉得赵顼也应该松了一口气才是。

可这话被他身边的官吏们听到后,得到的却是一幅幅苦脸。

对于太常寺的一众官吏来说,郊祀的意义,可不仅仅是依例分到手的那几块冷猪肉。

无数在自己的职位上拿不出突出的成绩,又没有后台,只能依靠磨勘来按部就班升级的京官,都对冬至郊祀期盼不已。

参加郊祀,以功劳论,绝不下于普通的军功,官阶少说也能升上一级。因犯法而受处分的罪臣,也能被赦免旧过。至于公卿重臣,他们的酬劳就不仅仅是官阶的晋升了,还有恩荫。比如担任大礼使的王珪,两个荫补的名额轻松到手。

韩冈的几个儿子都依靠他们老子的军功,早早的就得到了官职。完全可以不在乎。但其他官员,可没有韩冈的豁达了。

一天下来,雪灾后的救治,厚生司的救难队表现得很不错,但也不可能将死人给救活。青城行宫那边,报上来三十五名死者。而整个东京城,也不过死了六十二人而已。

到了黄昏的时候,雪停了,甚至连天上的阴云也开始消散,开封府派出大量人手清扫御街上的积雪。变得晴朗起来的天气,让韩冈的期望落了空。

中夜,半轮明月高悬,洒下清冷的辉光,千百颗星子镶嵌在夜空中,熠熠生辉。清朗的夜空,让昨夜的暴雪仿佛一场梦,但韩冈只有一个感觉:

“冷得够呛啊。”

不管怎么说,元丰三年的南郊大典终究还是到了。

