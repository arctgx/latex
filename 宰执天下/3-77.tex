\section{第26章 任官古渡西(二)}

韩冈的差遣定下,堂除之后,他便是白马县的新任知县。

赵顼为此很是有些惋惜,不过看在王安石的坚持上,加上韩冈算是在开封府内,也便不坚持了。但转头来却又颁下特旨,将韩冈的本官,自太常博士迁为右正言。

左、右正言与太常博士、国子监博士在品阶上是平级的,都是朝官从下往上数的第三阶,从七品。不过在官场上,却还是有高下之分。国子监博士是无出身官员的官阶,太常博士是依例封于有出身官员。至于左、右正言,则必须由天子特旨,属于受皇帝垂顾的特例,当年的王韶就是右正言。

韩冈自中进士,就从国子监博士自动转为太常博士,而现在赵顼又降特旨,将其转为右正言。虽然平级的迁官,但天子对韩冈的看重,已经从这封敇命中很明白的透露了出来。

外界本来对韩冈被遣出东京城,去白马县担任知县这桩任命,都有些看不明白——白马县怎么说都是开封府治下,说近不近,说远不远——说是为了让韩冈混一任亲民官的资历可以,说是怕他在京中碍事也可以。不过现在,就没必要再胡乱猜测了,不管王安石是如何看待他这个的女婿,至少天子那边对韩冈是极为重视的,这一点就已经足够了。

既然已经开始出任亲民官,就必须有一套处理政务的班底,而不像给人做助手的时候,不需要幕僚支持。

韩冈本来打算去找自己的同学,但王安石、王韶,甚至吕惠卿,程颢,却都写荐书推荐人来。

韩冈知道这是常理,便全都接收下来,却也不管这些人之间的关系是否和睦。在京城盘亘半月有余,韩冈在王家兄弟的相送下,带着一众幕僚、伴当,往着白马县而去。

……………………

韩冈就任白马县,在京城中,只能算是微起波澜,比他品级高、名声广、权位重的官员不胜枚举。不过消息传到白马县,却顿时掀起了一场轩然大波。

“七品朝官来做知县?有没有弄错!”

“还是王相公的女婿!”

“官职、身份那还是小事,关键来的人是叫韩冈!”

“的确是麻烦了。听说得罪他的从没一个有好下场。还没做官时就杀人不眨眼,做了官后更是心狠手辣,最近不是刚被他赶走了一个杨学士吗?那可是翰林学士啊,转眼就能升执政的!”

“怎么办?他既然是王相公的女婿,来了之后,保甲,免役,便民,农田水利,这些新法肯定是要死死盯着催逼。到时候,大伙儿可都要累死累活了。”

“这可还真是麻烦了……”

“怕什么!正面的确不能顶着他,可到了下面,还不是由我们说了算?小心点不要犯到他手上就是了。”

“胡老二说的正是,有什么好怕的?真要不识作,东京城就在边上,派些人去市井中帮着宣扬一下他韩正言的大名,却也不什么难事!”

“说得好!怕他作甚!”

“没错!没错!”

这番议论,不是在酒楼、茶馆或是私人家里,而是光明正大的出现在白马县衙的偏厅中。

坐在厅中上首处,是个长得很是富态的中年人,看着像一名富家翁,可却是穿着吏员的皂色衣袍。在他下首处,甚至还有身穿青色官袍的流内品官。但这名富态的吏员,却依然是稳稳的独自坐在最上面。

听着下面的一片声的议论,他低头喝了两口茶,闲闲的问上一句:“新官上任三把火,你们想引火烧身不成?”

议论声终于停了,厅中的十几人没一人敢搭腔。一阵静默后,被称为胡老二的瘦削汉子欠身问着:“诸大哥,这事还得你来拿个主意。依你说,该怎么办?”

“对!押司,你说该怎么办,我们都听你的!”另一个看着有些憨相的吏员附和着。

二十多只眼睛望过来,诸立很是闲适的又喝了口茶,并不急着回答。

他在白马县中有着很大的发言权,他家的两个弟弟娶得是县主,官身照样有。靠着老二、老三花钱娶了宗室,家里成了官户,本身又做着吏职,把持县中上下政务。来这里的做知县的,不论身后的背景有多奢遮,不想有麻烦的都要他给个面子。

诸立要做官容易得很,之所以把着吏职不放,就是因为此地的油水太过充足,舍不得放手——要是做了官,现在的位置被别人占了不说,说不定一封调令就会被调到广南监酒税去。外地的水土哪有家乡的安适?

说实话,这也是天下州县的通例,哪一家衙门中的胥吏,没有连续做了几代人、父子相承几十年的情况?这样的吏员,说话的份量往往比掌着衙门大印的官员更重。来上任的官员得罪了他们,别想能施展开手脚。

过了好一阵子,诸立才慢悠悠的开口:“不要先跳出来。棍子刚将草窠子拨开,你们一群蛇就游出来,这不是找打吗?先得看看那韩正言是什么性子?为人如何?才智如何?行事手段又如何?等一切都明了了,再做理会不迟。”

胡老二皱眉道:“入官三年多,就升到了这个位置上,又有如许大的名头。肯定是才智、手段都为上上之选,不然怎么能考上进士第九,赌赢了翰林学士,又让相公招他做女婿。不先想定对策,等他到了县中发号施令,可不好应对。”

“若是他真的如传说中的那般厉害,那反而好了。这样的人,肯定在白马县做不久。”诸立笑道,“也就是一两年的功夫,就会升上去。更别说天子的宠信也许会疏远,但翁婿之间还会疏远吗?王相公当真的会让女婿、女儿在这座县城里常住不成?肯定是早早的就调回东京升官发财去了。我等最多也只要忍个一年半载而已。”

诸立这话说得在理,胡老二闭起嘴不说话了,一众人则纷纷点头称是。

即将来担任白马知县的韩冈,都已经是右正言兼集贤校理。这个品阶,做知州都绰绰有余了。现在来做知县,就是因为年纪太轻,资序不足。而要解决这个问题很容易,就是走过场,做一任相当于通判的白马知县后,便有资格再上一层楼了。

为官一任虽说是定规三年,但有背景的官员,都会得到减磨勘的奖励。减一年是常例,减两年也不是没有,甚至有些地方,一年能换三五任知州知县。这都是混个资历就走的典型例证。

“少年得志之人,有几个会甘心在县里耐下性子来做事的?也就三把火的劲头,随着他性子,过去了就好。”诸立冷笑着,“说不定几个月后,就是我等献上万民伞,用两部鼓吹,送韩正言去京师做大官了!”

一番商议之后,得出的结论就是再议。与会的胥吏们纷纷离开,就只有是坐在诸立下首处,身穿官袍的一人留下来。诸家的老二诸霖,他方才没开口,现在外人都离开了,他就有些话要说。

“大哥,那韩冈可不好对付,小心他上来就给人下马威!”诸霖提醒着兄长,“你也知道我那连襟跟杨学士交好。那杨学士在琼林苑上赌输给韩冈之后,回去可是吐了好几次血,离京的时候,才勉强能走动的。”

诸立冷哼着,面沉如水。将茶盏在手边的几案上重重的一顿,诸霖就是浑身一颤。

就见着偏厅中,一名小吏训着诸霖这位官人:“你那个连襟做事没个分数,杨绘那厮也是轻浮!落到现在的下场,那就是活该!”

诸霖娶得是宗女。她的妹夫,也就是诸霖的连襟王永年,为求一个监金曜门书库的好差遣,千方百计地巴结着杨绘,甚至让自己的浑家出来陪客奉酒。不是用杯盏,而是用手,左右手合在一起,捧着酒喂给杨绘喝。诸霖的小姨子,是个出色的美人,长得白皙丰满,双手如玉。这双手一合,就号称是白玉莲花杯,杨绘为此甚至还写了好几首诗做纪念!

“监书库的确是肥差,每年腾出库中的故字纸,多少家印书坊重金求着要。”诸立摇着头,很是不以为然——官府中所用字纸的质量都是第一流的,而且使用时都只用一面,印书房将官库清除来的旧纸买回去后,可以直接翻过来用背面来印书,书籍的质量要远在福建、杭州之上——“但也不至于下作到让自己的浑家出来陪客,而且还是宗女。这事犯出来,就算没有韩冈,杨绘也在京中待不长久。没人对付他那也就罢了,要想赶他出京,这就是最好的罪名。做人做事都没个准数,能混到翰林学士,还真是运气了!”

“杨学士的确自身不正,可韩冈也不是好对付的。”

“韩冈本来是做事的出身,后来才考了进士。像他这样的人,为官一任,肯定是打算着‘造福一方’,总是想着有所成就——说难听点,就是好大喜功。”诸立眼神深沉:“既然是有所求,就有了我们逢迎希和的机会。一开始就帮着他,助着他,与其为善。这些手段,本就是当做、该做的。奉承好了,日后也是有好处的。”

“但要是他……”诸霖变得吞吞吐吐。

诸立嘴角轻扯,露出一个让人不寒而栗的笑容:“若是韩冈不识趣,为兄也自有方略去应对。”

