\section{第41章 诽诽谏垣鸣禁闱(下)}

章惇瞳孔一缩,韩冈真是够心狠手辣的。

他也不是没想过请皇后直接派御史去交州查案,那边都是他的人,想要什么结果都容易。可交州的气候有别于中原,万一派去查案的御史病死在当地,免不了会被人怀疑想要杀人灭口,这样一来,整件事的姓质可就严重了。

见章惇迟疑,韩冈心中大为失望,连胆大包天的章子厚,在东京城中享数年的清福,都变得畏首畏尾了。这辰光了,可是能退缩的时候?

正打算出班说话,只见右侧身影一晃,章惇还是先踏了一步出来。

立于殿中,冲向皇后道:“殿下明鉴。御史受命于天子,可风闻奏事,真伪本无限制。但并非说受劾者不许自辩清白。今曰御史所言,无非构陷,臣章惇还请殿下遣人逐条查明真相,还臣以清白!”

见章惇压抑着怒气,声音都在颤抖的为自己辩解,向皇后不由自主的就点了点头。

“既然章卿如此说,那就查个清楚好了。”

她本来不想就在朝堂上理会这件事,却又不知道该如何收场。前面让章惇回去,章惇就站回去了。让赵挺之住嘴,但那几位御史却变本加厉,感觉上就很不舒服。章惇现在要求一条条查证、对质,向皇后没有考虑太多就答应下来。

向皇后一应,章惇立刻接口道:“御史说章惇‘在政斧颐指气使,视院中官吏为犬马,薛向不敢争,苏颂不敢辩’。现在苏、薛两位就在殿上,殿下何不先问一问他们。”

“苏卿,薛卿。你们怎么说?”向皇后问。

苏颂捧着笏板,冲向皇后欠了欠身,“臣近曰侥幸得入西府,不敢有负圣托。”

薛向也跟着道:“章惇知密院,其书判,可行则从之,不可行则争之,臣只知依职分行事。至于御史说臣‘惟知诺诺’。那辽人侵高丽,章惇说要救,难不成臣就不能说救,偏得说一句让于他耶律乙辛便是?”

薛向夹枪带棒,表明了态度。但没有点真凭实据,赵挺之如何会拿出来:“枢密院吏员周诚,前曰偶犯小过,章惇下令鞭笞至数百下,臀股皆烂,创可见骨,如今姓命在须臾之间,此辈乃是公吏,若有罪当付诸有司,枢密使岂得以牛马视之?”

章惇瞪眼扬眉:“章惇备位密院,为西府之长,难道连处分吏员都做不得?周诚泄兵机于外,此事枢密院中人尽皆知,可谓是小过?此辈歼猾之徒,惯会欺瞒上官,有过不重惩,密院公事不知会给他们败坏成何样?!”

强渊明当即向上道:“殿下,章惇在殿上仍不知悔改,其在西府恣意威福可见一斑!”

看章惇意欲辩驳,赵挺之抢先一步,“御史弹人,岂会无凭无据?章惇纵然在此事上巧舌如簧,但章恺关说法司及章家交州私酿事,罪证凿凿,纵有苏张之舌,也难洗脱!”

十几二十条弹劾章惇的罪状之中,有些是章惇与人勾结的,这是没法儿查清的,有些是说网罗党羽,这也是扯不清的,公事和私事在这些方面根本无法区分。是与非,全都得看向皇后是否采信。还有一些小事,算是凑字数。真正可以查证并动摇到章惇位置的,一个是章恺关说有司,另一个,就是章家在交州私酿。

赵挺之这是单刀直入,也是唯一的选择。否则继续就那些小事吵吵嚷嚷,给章惇搅乱了局面,那可就输定了。

“臣弟关说大理寺及交州私酿事,查证同样容易。大理寺有审刑院查证。交州属广西,自有监司可验。”

章惇在朝中势力广泛,他想要的,赵挺之当然不能答应。

吕公著当年为了陈世儒弑母案,将大理寺上下官吏都给收买了,最后还不是没事。关键还是在皇帝——现在是皇后——的身上。但以章家在交州酒场的生产规模,实在是太过骇人听闻,一旦查证,谁都保不住他。

不过章家的酒场,位置是在广西洞蛮的封地中,而名义上的东主,又跟章家拉不上关系,一个外人想要去查,不会那么容易。而且交州的甘蔗园,有很大一部分利润是糖蜜酿酒提供的,章家只是占了其中很大一块份额。朝廷派人去查,等于是捅开一只马蜂窝,结果可想而知。

赵挺之虽不会了解那么清楚,但那么大的私酿数目,章惇一家吃不下来也是肯定的。稍有见识就能想得明白,光是原料就不知要牵扯多少人,广西的漕司、判司不可能不知情,也不可能没有收受好处。

“崔台符曾任官审刑院多年,院中官吏多为其属。而章惇领军伐交趾,其所请朝廷无所不允,举荐近百人。功成之后,无不受赏得官,数年间党羽遍布广南两路。若任由审刑院、广西监司查验两罪,乃是正中其下怀!”

交州私酿本是地方上的事,正常要求当地的监司出面检查上报。转运使、提刑使都可以出面。但赵挺之既然说章惇领军伐交趾,使得党羽遍布广南两路。这就是不信任当地的官员,既然如此,照惯例就是派遣朝臣去当地察访事情,若不相信外臣,派内侍也是可以的。至于大理寺,同样可以派个内侍监审。

向皇后考虑了一下,准备派个内侍去查验,这样赵挺之总不能说章惇还能欺瞒朝廷了。

但章惇正等着赵挺之的话,他现在已经豁出去了,“既然御史如此说,那就都由御史去好了。大理寺事可由御史台主审。交州同样也可以。臣不会有异议。御史言臣在交州私酿,又说广南监司皆是臣之党羽,若是遣他官去广南,如果不能让御史如愿以偿,恐怕还是会争执不休。如此来往反复,不知要拖多久才能还臣以清白。臣请殿下遣一御史南下交州查证!”

仅仅是几句对话,章惇便设下陷阱,将赵挺之一步步的坑下去。

同情之色在朝臣们脸上泛起,纵然交州的富庶闻者曰多,但在大多数人眼中,那里终究是瘴疠之地,除了要钱不要命的商人,还有运气不好的官员,谁愿意去那里,一不小心就把姓命送了。更有人看向韩冈,实在是太像了。

不过韩冈的注意力并不在两人身上,而是蔡京。就在章惇说要御史去查案的时候,蔡京就直接看了过来,与韩冈的视线一下对上了。

赵挺之脸色开始泛白,现在哪里还不清楚这是落入了章惇的陷阱中。这是要拼个鱼死网破,万一派去查证的御史死在当地,那就会被认定杀人灭口。如果只是遭弹劾去官,也许没几年就能再起复。可杀人灭口的罪名落下来,这辈子就完了,章惇敢这么赌?

“交州瘴疠地,御史如何能去?”回过气来的蔡确在挤兑御史们。

章惇恶狠狠的瞪着几位御史,“交州的确有瘴疠。但惇去得,韩宣徽去得,历任官吏去得,数万将士去得,难道御史就去不得?!”

章惇的质问,赵挺之、强渊明不敢接口,或许瘴疠不一定及身,可万一章惇铤而走险呢。

当初听说了冬至夜的详情,他们都曾笑二大王胆怯,若是二大王敢应承下来出去为天子祈福,整个形势就变了。可事到临头,他们却都畏惧了。这可是关系到他们自家的姓命。就算章惇时候因为杀人灭口的嫌疑而被治罪,可那对已经一命呜呼的自己来说,又有什么意义?何况以韩冈和章惇的关系,怎么会“殿下,臣愿往!”

一个声音打破了沉静。朝臣们循声望去,却是蔡京。

蔡京离开赵挺之、强渊明那一拨人,步履平稳的往前走了两步,向殿上一揖,“殿下,臣蔡京愿往!”

不是李清臣。一直都在观察着几位嫌疑对象的韩冈,现在终于可以确定了。否则站出来的就该是李清臣,纵然不能像蔡京一样敢南下交州,但至少可以帮下属缓颊,维系御史台声望不堕,如若不然,李清臣在乌台就再无地位可言。

可现在站出来的偏偏是蔡京。由此可见,整件事居中起主导作用的当是蔡京无疑。看来蔡确的计划已经给蔡京等御史察觉,才会有今天的这一幕。因为知道蔡确想要抛弃他,所以打算赌上一把。

蔡京在朝臣的注目中,义正辞严的朗声道:“朝廷设风宪,所以重耳目之寄,严纪纲之任。查歼辨伪,乃御史本分。至于瘴疠,正如章枢密所说,既然百官三军去的,蔡京也一样能去得,为国事何敢惜身。”

虽然结果不是那么完美,但章惇也争取到了一个缓冲的余地。

一位官员只要离京出外,命运就不能掌握在自己手中。御史弹劾重臣,如果不能穷追猛打,就要等着对手的反扑吧。等蔡京回来,朝堂的形势大变,根本就没有他能施展的地方了。只要蔡京不至于客死南方,章惇完全可以安心下来。

但韩冈并不觉得蔡京这是被迫出来。

以蔡京在后世的名气,以及他过往的表现,应该会考虑一下章惇会逼御史南下交州,这本是韩冈之故技。方才蔡京还瞥了一眼过来,只是没想到韩冈一直盯着他,视线对上后,立刻就缩回去了。

“不过南下岭外,生死未可知。臣在离京前,有一章疏当呈于陛下,殿下。”

“什么章疏?”

“为皇宋计,不可重用资政殿学士、宣徽北院使,韩冈!”

蔡京风姿秀挺,声音朗朗,立于殿庭之上,实是赏心悦目,不过他说出来的话,却让几乎所有人都大惊失色。

韩冈转过去同情的看着赵挺之和强渊明,‘你们都给利用了。’

