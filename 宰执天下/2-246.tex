\section{第36章 万众袭远似火焚(二)}

景思立心中有了那么一瞬间的动摇。现在还没见到正主,让人看破了自己的心思,就算以他的老辣,也是一阵惊慌失措。

景思立想要留在熙河路博取军功,以他现在的身份,少不得也要一个都监、甚至钤辖才能安排得下。而钤辖、都监,都有资格独立领军,景思立一旦到了熙河,等于是抢了眼下熙河诸将的领军机会。不论河州决战后,王韶还能不能留在熙河,但他所一手组建起来的势力,却肯定是一个不愿让外人插足的团体。

韩冈仿佛没有看到景思立脸上一闪而逝的惊容,继续说道,“听说朝廷汰撤厢军的目标是二十万。不过真正要动起手来,也不会当真如此狠手,多半还是能留下二十四五万的样子。”

景思立收摄心神,他不敢肯定韩冈现在说的话,是不是王韶本人授意,也不清楚这是不是一个考验,但他知道,他的回答肯定会影响到王韶对自己的看法,“思立听说,在陕西最后只会剩四万到五万厢军,多数还要集中在永兴军经略司辖下。日后的边寨防务,大的城寨有禁军,小的寨堡,就是靠乡兵弓箭手。”

在熙宁之前,戍守边寨的多有厢军,但到了熙宁五年的现在,大多数边地寨堡,都变成了乡兵弓箭手来驻防,实行的是半兵半农的制度。免去了乡兵们全额税赋或半额的税赋,但不用发给薪俸,抚恤也不用多给,对于朝廷来说,绝对是一桩美事。

而且他们所拥有的保护乡土的意识,让乡兵们的战斗力远胜于厢军,甚至接近于装备齐全的禁军。故而几年的功夫,戍守边地的厢军几乎都是被乡兵弓箭手所替代,尤其是保甲法在陕西各路推行之后,结成保甲的乡兵们的作用更是让人无法忽视了。

韩冈叹道:“就不知今次陕西汰撤下来的数万厢军,朝廷会怎么处理了。若是不能小心安置,也许会出些乱子。”

“以思立之见,最好能上书天子,从其中拈选精锐,派到边地去实边屯田。”他看了看韩冈,“熙河路其实就不错。”

大约一个时辰后,秦凤军终于抵达了陇西城。城外的几处营地,早已经安排妥当。

驻马营门边。亲眼看着手下的队伍,在十几名经略司属吏的指挥下,顺顺当当被安顿下来,并没有发生过往大军移防时必然会出现的混乱。景思立对熙河经略司的理事手段,暗暗的有了几丝敬畏。

“真是让人吃惊。”景思立赞叹着熙河经略司,虽是借机示好,但语气也是由衷的,“整整一万人马,换作是移防他路,没有两三个时辰的功夫,根本不可能安顿下来。”

“多谢都监的夸赞,”韩冈一笑拱手,“韩冈愧不敢当。”

“是机宜你的安排?!”景思立心中说着果然如此,韩冈处事手段闻名关中,秦凤军的安置工作说是他的事先筹划,能如此稳妥就是理所当然,并不值得惊讶了。

“王经略有命,韩冈哪有不尽心尽力的道理。”韩冈不多说废话,单是安排秦凤军入营,就又是耽搁了一个时辰。他拱手延请景思立入城,“经略已经在衙中等候,还请都监速速入城。”

听闻韩冈如此说,景思立更不多耽搁,带着一队亲兵,急忙打马进城。

一行人飞驰而行,转眼就到了陇西城的东门前。在城门处,好几列满载着一袋袋货物的车队一溜摆开,城中的车斗堵住了并不算宽阔的大门。

他们本是一辆一辆的要接受检查入城,现在韩冈和景思立到了,守城的士兵忙着让他们把车子赶到一边去。从袋口漏下来的麦粒,可以看得出里面装的都是粮食。

见着这些运粮车队的领队之人,都不是军汉或是吏员的模样。景思立转头问着韩冈,“这是去折博务入中的吗?”

“折博务还是刚刚成立,这些入中的商队算是第一批了。”韩冈回答着,不出意外的在景思立脸上发现了一丝忧虑。他笑道:“都监大可放心,今次一战,真正军中需用的大头,已经都在仓囤中了。他们这些商人只不过是拾遗补缺而已——春时不便征发民力,只能用他们代替。不过若是效果好的话,日后补充熙河路粮草的任务,说不定就要靠这些商人了。”

景思立点了点头,但并没有说出自己是赞同还是不赞同。

就在二月初的时候,朝廷同意在巩州设置折博务,以商人入中的变通手法,向熙河路加速输送粮草。

所谓入中,就是招募商人把粮草运到边寨指定地点,兑换钞引,而后商人再凭钞引,去京中或是其他地方去领取报酬。最早的时候,付给商人们的报酬是现钱和金银,后来转为实物,如香药、茶叶,而现在更为普遍的便是盐。

原本以秦凤转运司的运力,支撑起万人左右的大军,保证正常的补给没有任何问题。但换成是三万兵马,对于陕西民力几乎就是涸泽而渔了。能有别的手段做个补充,不论是蔡延庆,还是赵顼、王安石,都不会介意使用。若是早有明证且卓有成效的手段,更是不会有一点反对之声了。

但陕西缘边各路入中,商人们兑换钞引时,发给的都是解州的池盐。作为北方最为上乘的食盐,解州池盐的价格要远在井盐、海盐之上,所以商人们趋之若鹜。

入中的政策,在缘边各路其实一直都在施行着,尤其以靠近解州的鄜延和环庆两路为多。这两路的入中,占去了大半的解盐份额,也因此,能分配给熙河路的食盐数量,就显得微不足道——这就是为何之前韩冈和王韶都没有把注意打上入中纳粟上——可是如今运力不足的情况实在难解,设立折博务纯属无奈。为了解决给付解盐不足的问题,韩冈给王韶出的主意,是用河湟荒地,以及官田出产的棉花来抵数。

当时王韶犹有疑虑,担心这空口说白话的荒地地契和根本还没下种的棉花,根本吸引不了商人们的眼睛,因而为防万一,还把盐钞都放了进来,希望能用巩州的井盐,来代替解州的池盐——王韶本还想过用茶做报酬,但如今茶园都给官府给包了,尤其是靠近陕西的蜀中,那里的茶园有大半出产被运到熙河路这里向吐蕃人交换战马,吐蕃人不再缺茶,换成茶叶,就没有多少利润可言。所以这一个方案被放弃了。

但商人们最后的选择,却证明了韩冈的正确。不仅仅是因为巩州的井盐过于咸苦,难以入口。更为关键的,还是利润的关系。对于愿意入中输送粮草的商人们来说,棉花如果纺成棉布,带给他们的利润绝不止百分之三百,比起三成五成的盐利,用着最简单的算术算一下,那要强出十倍八倍——只是要稍等一段时间而已。

“旧时商旅入中,拿到钞引后,换来的官盐其实并不够补偿运送粮秣的费用。官盐只是个幌子,有了这个幌子就可以名正言顺的对外出售盐末,从党项人的青白盐池那里回易来的私盐,也就可以光明正大的掺进去卖掉了。”韩冈当时是这么向王韶解释的。有个擅长经商的表弟,让韩冈对于商人们的奸猾手段,多有了解,“拥有一斤官盐的量,奸商们往往都能卖出十斤去。可这般卖盐终究是犯忌的一件事,利润也只有三五成,哪比得上棉布的三倍五倍呢?”

王韶和高遵裕虽然没听说过那段著名的、对商人追求利润的行为的评价,但也算得清三倍和三成的区别。毕竟这些奸商的手段,也是他们或多或少都了解的。

而对于边地的商人们,以及他们背后的豪门来说,三倍和三成他们也一样算得很清楚。虽说荒地尚未开垦,棉花只刚刚栽种,但以这些豪门所拥有的影响力,难道还怕朝廷转过脸来会赖帐不成?而且,天子和朝堂也盼着他们能出手,让熙河诸军州的出产更为丰厚,根本不可能会翻脸不认人。

——只是这一切的前提,是熙河经略司,用过去两年里的一个接着一个的胜利,向所有人证明了他们的能够保护大宋臣民在河湟地区的利益,否则,又有谁会到熙河路来冒险?

顺利的进了城,韩冈将景思立送到衙门中,王韶和高遵裕都在正厅中降阶迎候。王韶、高遵裕与景思立说话,韩冈还有事要处理,抽个空就起身告退。只是他一出厅门,就被王厚给拉住。

王厚性急的问着:“厢军的事,景二是怎么说的?”

韩冈回头看了看正厅,把王厚拉得更远了一点,“看起来他并不反对熙河要这批退下来的厢军。”

“这就太好了!”王厚很兴奋的一锤掌心,“只要他这个知德顺军能帮我们说话,秦凤路争不过我们。若能多了三五千户,秦凤转运司的钱粮,几年内必然还要向熙河倾斜。”

“这事就再说吧,先准备着就是了。”

关于厢军的事,韩冈和王韶只是为了未来筹划,至少并不是亟待处理的事务,真正要对付的还是远在河州的木征。

