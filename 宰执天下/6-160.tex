\section{第13章 晨奎错落天日近(25)}

太后的反应让章惇心沉了下去。

吕嘉问的话,不过才开了个头,就被太后给打断了。

太后的倾向激烈得又是一个出乎意料。

表面上是让吕嘉问不要打岔,干扰正常的会议流程,但吕嘉问连话都没能说完,朝臣们看在眼里,还会怎么想。

被太后当庭一驳,吕嘉问的脸色红了又白,一时说不出话来。

当初他主持市易司,成为旧党攻击新党的靶子,而曾布也趁机叛离,那时候的吕嘉问,慌得不像样子,有失大臣体面。

正常情况下,吕嘉问口舌如簧,又能胆大妄为;但重压之下,却缺乏随机应变的捷才。

‘这个时候,可不能发怔啊。’

章惇叹了一口气,若是吕嘉问敢拿出自己的提案来,大概就会给太后直接骂回来了——只有宰辅才有资格拿出自己的提案。

举步出班,章惇道,“陛下。十余年来国势蒸蒸日上,新法之功也。一应法度确有不尽人意之处,但行之有效,当继续施行,只视人情稍作修改便可。如今北虏虎视眈眈,岂能视而不见?且耶律乙辛篡逆之辈,中国不可与之媾和。当拒使者、绝外交、断岁币,河北、河东,更当加强武备。”他提声放言,“陛下,北虏,腹心之疾;南蛮,癣癞之患,臣以为疗伤医病,当以腹心之疾为重。”

两边较量的中心,已经偏离到了争夺变法主导权上。

章惇没有例举王安石的功劳,没有去述说新法的作用有多大,更没有攻击韩冈的提议,既然韩冈要进一步变法,那么他所能做的,就是顺水推舟。

李定的心提了起来,章惇这是迫不得已,否则该由自己来出面来提出新党自己的提案。

他知道章惇的话多半不能将太后打动,但他更清楚只要在朝堂上胜利了,太后只能认同殿上的决议,否则事有反复,韩冈的有关国政会商的动议,就成了笑话了。届时,韩冈比单纯的输了投票还要丢脸。

但现在这个胜利,已经从一开始是十拿九稳,变得十分渺茫了。

章惇一番话说得含含糊糊,太后听了皱眉,“章卿可明说国是当如何更易。”

章惇朗声道:“断绝岁币、修筑轨道、加强武备、以御北虏,余事如旧。”

“是御寇,不是讨贼?”

太后敏锐的把握到了章惇用词中的关键,问话的同时,向王安石望过去。

十余年前,旧党是绊脚石,十余年后的今日,王安石是绊脚石。被人当做绊脚石,他该如何反应?

但王安石不知何时低下头去,看着笏板,没有任何反应。

“是。”章惇平静的说道。

殿中响起了一阵低低的喧哗,没人能想到章惇在这个时候选择抛弃了王安石。

李定一下要紧了牙关,这与之前在王安石府上议定的提案截然不同。

当局势不利的时候,在提案的陈词中,可以有些妥协,可以有点退让,但绝不该是投降。

当时议定的用词,应该是‘相时而动’,但章惇的‘以御北虏’是彻底的否定了进兵辽国的可能。

李定的双眼瞪向章惇,这是要另立山头吗?!还是看到势头不好,准备过河拆桥?

章惇不觉得自己有回应李定视线的必要。

彻底放弃了王安石和吕惠卿之前主张的攻辽战略,王安石还好说,主张攻辽的吕惠卿不可能短期内回不了朝堂了。

说起来还是章惇的私心。但好端端局面,因为王安石和吕惠卿,让韩冈有了搅乱国是的机会,新党内部自然有着异声。

人心思惰,已经成了重臣,多半还是不希望朝堂上再起动荡,太后、韩冈的组合,的确让人畏惧。可一份正常提案,还是会有一定的效果。

章惇的提案基本上都不变动,但名义上还是加强了对辽的防御,而最大的变化,就是要修筑轨道。

说起来跟韩冈的提案没有太多区别。

除去没有开拓新疆的内容,也就比韩冈少了一句继续变法。其他几乎完全相同。

这样的情况下,该怎么投票才合适?

不过连章惇都仿效上了韩冈的提案,已经没有多少人还觉得新党能够取得胜利了。

还有人期盼王安石能够坚持到底,交上自己的提案,不让章惇代表整个新党。但无论是谁,王安石、曾孝宽、吕嘉问,都不敢在这时候,出面分薄新党的选票

“好了,若没有其他人另有提案,”向太后看了看两府,急匆匆的说道:“就请诸卿从韩参政与章枢密的提议中选出一个最为合适的。”

不要再耽搁时间了,该结束了。

不止她一人这么想着。

……………………

不多的箱笼被龙门吊直接吊进了船舱中,王安石一家在京城中的时间,也只剩下最后的几个时辰。

王旖在船上与吴氏说话,王旁在后面的一条船上安排人手整理行李,王安石和韩冈站在栈桥边,很长一段时间以来,他们已经很久没有能够如此心平气和的对话了。

汴水中的倒影,因渠中流淌的黄河水而显得浑浊而模糊。

王安石低头望水,过了不知多久,他低声问:“玉昆,你到底计划多久了?”

他的问题没头没脑,但他清楚,韩冈知道自己问什么。

“不敢欺瞒岳父。”韩冈的回话恭敬一如既往,可内容完全没有半点谦退,“如何治国平天下,小婿心中自有一篇文章,写成也有不短的时间。但小婿从来没有想过这么快就能接手朝政。其实本来打算以十年为期。毕竟……我能等得起。”

王安石沉默着。船只在晃动,水中的倒影越发得模糊起来,更加让人觉得晃眼。

的确,唯有时间,唯有在时间上,朝堂之中没人能与韩冈相争。

十余年前入京,自己已是‘欲寻陈迹都迷’,而韩冈,即使是今日,也可算是青春年少。

“那辽人呢,玉昆到底怎么安抚下来的?”

这是王安石百思不得其解的原因。这三个月来,朝堂上波涛不断,但河北边境上,仿佛被杀的不是皮室军的人,辽国方向更是平静得让人难以置信。

“是太后的堂兄。”韩冈毫不讳言。

向家在河北一路,利益关系可是不浅。王安石当然知道这一点,可他想问的并不是表面上的东西,而韩冈始终避而不谈。

现在表面上,辽人之所以偃旗息鼓,默认岁币被裁,完全是因为边境重开榷场。但王安石总觉得,其中还有更深层次的原因不为人知。但三个月来,他始终没有找到。

三个月的时间不算短了,四季已经从东走到春,都快要到夏季了,北方也在这个时间内安定了下来,朝堂更是如此。

当日共商国是的会议,也就是韩冈口中的皇宋第一次政治协商会议,以八票之差,让韩冈获得了胜利。

新党惨败,王安石终于发现自己已经无法掌握新党的人心。

用了三个月的时间,王安石终于卸去了平章军国重事的差事,现在他的身上,只有一个判江宁府的差遣。

而在这三个月的时间里,朝堂上的动荡也渐渐平复。不过巨浪过后平静下来的水面,已不可能恢复到浪起之前的模样。

章惇依然盘踞在枢密院中,尽管有一批人视其为不下与韩冈的罪魁祸首,但也有一批成员还是认为,王安石举止失措、偏听偏信是这一次重挫的主因——二者的分野,只在是否能够留在朝堂之中。

政事堂中,多了一名宰相。不过就任中书门下平章事兼集贤院大学士的,是苏颂,而不是众望所归的韩冈。苏颂对自己在垂老之年,却因人成事的在两府中混日子,除了苦笑,只有摇头。倒是苏家的子弟,对此兴奋不已,让人望之叹息。

韩冈依然在参知政事的位置上,官阶职衔上,一点变化都没有,仍旧是东府三人中的最后一位成员。

至于原来的那一位参知政事张璪张邃明,则是至枢密院接替苏颂的位置——知枢密院事。尽管不能直接成为宰相,可也算是进了半级,本官也同时进阶。而且他从韩冈对宰相之位的态度上,也看到了一线希望。

除此之外,两府之中,就没有别的变化了,曾孝宽还是签书枢密院事,郭逵也照旧是同签书。

气学一脉控制政事堂,新学一脉控制枢密院,双方对掌权柄,维持着朝堂上的平衡。

两府之下,三司使吕嘉问卸任出外,出知扬州,权知开封府沈括接任。时隔多年,沈括再一次出判三司,但已是物是人非,曾经意气风发,想要在两府中有所成就,现在只剩下混一张清凉伞,好拿回去应付家中河东狮的念头。

而新任开封知府,是相州韩家的韩忠彦,韩琦的长子。只看在韩琦的面子上,开封府一职就不能算高。

引发这一次朝堂大动荡的罪魁祸首——判大名府吕惠卿两个月前被调任许州,河北转运使李常接手大名府和河北防务。

御史中丞李定,也在同时离开了京师,但接替他的不是韩冈的人,也不是旧党,而是新党另一位干将,曾任御史中丞,昔年在台谏任职多时的邓润甫。

新党重镇或出外,或调职,一时之间,新党中已经不存在能与章惇相抗衡的对象。至于同在西府的曾孝宽,缺乏进士头衔,想要再进一步的希望十分渺茫。

韩冈一方,游师雄就任三班院,他初来乍到,不便遽然高位,但加上审官西院的李承之,中低阶武官的人事之权,已稳稳的控制在韩冈手中。

新党退让,韩冈党羽与之对掌朝堂,至于旧党,相州韩家在其中分润到了一点好处,不过旧党之中,得益最多的还是富弼。

尽管年岁尚幼,但熙宗皇帝唯一的女儿曹国长公主已经有了婚约,长大成人后将会成为富弼的长孙媳。

富弼本人从中无从取利,年届八旬的他已危在旦夕。这个婚约,也的确暗藏了冲喜之意,不过更重要的还是安抚旧党人心。富弼家中无贤才,得以尚公主,至少能保三代富贵,这一件事上,至少表明了朝堂不会过河拆桥,也代表了朝廷对旧党的优容。

船将行,护卫航船南下的都头,已经在招呼着还没有上船的乘客。

“好了。”王安石早看腻了浑浊的河水,回身向船上走去,“该走了,该让世人忘掉我这等老朽了。”

韩冈陪着王安石:“不管怎么说,岳父你留下的功业,不会被人忘记。”

“何谈功业?”王安石叹了一声,十几年来,一桩桩、一件件,都在他的心目里流过,“不过日后是否能更进一步,就看玉昆你了。”

“岳父,即使只是为自己,我也会尽力让大宋变得更好!”

王安石听得觉得扎耳朵,只是正想说话,舱中人语响,王旖走上了甲板。王安石瞟了韩冈一眼,不再多话。

王旖下船后,轻声细语:“爹爹,孩儿带了一部新的闲书来,已放在舱中,爹爹闲暇时可以多看一看。”

“书吗?谁的手笔。”

王旖回头看了丈夫一眼,道:“小说家言,佚名之物。”

