\section{第45章 成事百千扰(上)}

【真的是迟了,实在抱歉】

过了年,就是熙宁八年。

噼里啪啦的鞭炮声,一阵接着一阵,即便以司马光的心性之沉稳,也难以安心的继续将书读下去。

‘起昭阳作噩,尽阏逢阉茂’,南北朝时的一卷,读到宇文泰鸩杀废帝一节,他终于难以忍耐耳边的嘈杂。

离开东京,算算也快有六年了。距他修起独乐园,也已有数载,而掘了地窖写书读书,差不多亦有两年了。小园虽云独乐,但墙垣卑小,占地不广,外界的喜乐照样随着鞭炮声传进了独乐园中的读书堂来。

读书堂的书桌上堆了一大摞名帖。如今的习俗,就是过年时送名帖门状。过去讲究着过年时上门拜贺,但在官场上,来往的人情甚多,哪有一一拜访的精力和时间?逐渐的就变成了新年派仆人上门送名帖,只将心意送到。司马光不能免俗,元日送了十几张出去,却得到了几百张回来。

司马光放下资治通鉴的手稿,带着嫌恶的眼神撇着桌上高高堆起的名帖一眼,觉得还是去地窖里读书比较好。

读书堂的这间书房他平素都不使用,而是在地窖里著书,偶尔用一次却吵着这般厉害。站起身,就要带着书下地窖。

“君实。”司马光的贴身老仆敲了门后,走了进来,指着书桌上的名帖,问道:“是不是都收拾了?”

司马光回头看了摞在桌上的名帖。世风日下,人情如纸,一张门状就算是登门拜访了,司马光还是有些看不惯,“都收拾了。”

老仆麻利的收拾起书桌,司马光又要下地窖,儿子司马康却也进了书房来。手上拿了一封信:“大人,刑和叔【刑恕】又写信来了。”

“刑和叔?”司马光接过信,严肃的一张脸上多了点欢喜。

刑恕是程颢的弟子,也曾投奔于他和吕公著的门下,考上进士也早,不过因论新法不便而被王安石出知于外。这些年来,信也来得甚勤,司马光倒是挺想着他的这位门人。

看到刑恕的信,司马光突然想起一事:“前日刘贡父【刘攽】的信还没有回,今天得先写好。”他对司马康道,“前日刘贡父写信来,说蔡确是倒悬蛤蜊。想着回信提醒他勿要再谐谑侮人,不意却给忘了。”

听到了刘攽如此拿蔡确的名字开心,司马康想笑,又不敢在父亲面前随便笑,紧抿着嘴,脸也给憋红了。

蛤蜊又名壳菜,反过来就是蔡确【注1】。而蔡确身为御史台中人,就像是蛤蜊一样。风闻奏事如同张开的蛤蜊嘴,大得没有边。而一旦合起来,也跟蛤蜊闭壳一般,咬谁都是一嘴血。对于在御史台中为虎作伥的蔡确,这个绰号再确切不过。想必只要流传出去,转眼就能从京城、洛阳,散布到天下各处。

“刘贡父平生多为口舌所累,至今不改。”司马光又叹了口气。

他与刘攽交情匪浅,编修资治通鉴并非司马光一人之力,而是由司马光提举整个修书局的功劳,刘攽便是其中的主要成员。其人乃是当今的史学名家,尤其精于汉史,如今通行于世的《汉官仪》和《汉书刊误》便是其所著。被司马光推荐负责资治通鉴中的以汉史为主的部分篇章。

“刘贡父若是能改,何至于做了员外郎,才得馆阁校勘一职?”

刘攽最爱拿人名讳开玩笑。曾有名叫马默的御史弹劾他玩侮无度。有人私下里告诉刘攽,他立刻就道:“既称马默,何用驴鸣?”又写下一篇《马默驴鸣赋》作为报复。

王汾的名字与‘坟’同音。而刘攽的‘攽’与‘班’同音。一次,王汾拿刘攽的名字说笑,道“紫宸殿下频呼汝。”——上朝时,唤班吏都会拖长声调叫着‘班班’。刘攽则回道:“寒食原头屡见君。”——寒食节都是要上坟的。

据说,去年曾布和吕嘉问之争,王安石袒护吕嘉问【字望之】,使得曾布出外。当时在官场中流传,出自于论语,岂意‘曾子避席,望之俨然’的玩笑,就是刘攽所说。

甚至他还拿如今声名正盛的韩冈来取乐过。‘扶摇万里倒飞回’,这就是拿韩冈的表字在开玩笑。

江山易改,本性难移,司马康可不觉得刘攽能改了他这个多嘴多舌、爱拿人姓名开玩笑的毛病。

正说着刘攽,方才那位老仆此时又走进来,向着司马光父子行了一礼,递上一封拜帖,“君实,程家两位官人在外求见。”

在洛阳说到二程,自然是程颢、程颐到了。

司马光低头看了一下身上所穿的家居常服,对儿子道:“你且出去陪伯淳、正叔叙话,待为父更衣。”

等过了半晌,司马光换了一身见客的衣服出来,就听着程颢、程颐,在与儿子说着话。

程颢道:“正心诚意。诚意在致知,致知在格物。格物则在于穷究物理。”

“凡眼前无不是物,物物皆有理也。火之所以热,水之所以寒,以至君臣父子之间,穷其理方能致知。”这是程颐的话。

司马光听了,淡然一笑。他素闻二程对格物致知有着别出心裁的释义,只是如今被人抢了先去。而司马光本人,却是对二程或张载的新解不以为然,虽然不至于仍遵循郑玄、孔颖达的注疏,但自有一番见解。

与来访的客人见过礼,坐下来后,司马光问道:“不知方才在说着什么?”

司马康连忙道:“正在说韩冈的浮力追源之论。”

洛阳离得开封甚近,韩冈在京城中传播来开的新论,没有两天也便传到了洛阳来。二程也好,司马光父子也好,耳目都不闭塞,在年节之前,便已了解到了大概。

“韩冈吗?”司马光又是一笑,笑容中透着深沉,让人看不出心中所想,“不知伯淳、正叔如何看?”

程颢点点头:“只觉得甚有道理。能将船浮水上的道理,说得透了,也只有韩玉昆。”

司马康立刻道:“只是韩冈一番论调,多是说着自然之道,不见涉及半分纲常,未免偏驳——横渠张子厚的砭愚【即西铭】一文可没他这么偏。”

程颐道:“韩玉昆的确少言纲常,有失轻重。不过以他的年纪,能穷自然之理,已是难得。”

程颢也道:“记得韩冈曾说过,欲以旁艺近大道,的确是有点跛脚了。不过纲常一事,重在施行,韩冈在白马县断何家争坟案,可是依着纲常来判的。”

程颢程颐一力回护着韩冈。其中缘由,司马光怎会不知?

王安石的那个女婿素来在二程面前执弟子礼,两年前过洛阳,又曾经在雪地里占了一个多时辰。尊师重道之举,世间罕有人能及。二程因此而看重韩冈,也在情理之中。

不过司马光对韩冈,也是不明白他到底是站在哪里。

韩冈娶了王安石的女儿,却并不能说他是铁杆新党。韩冈对新党若即若离的态度很是明显。他的确帮了王安石的大忙,但也曾与王安石为了举荐张载和二程入经义局而相争。如今更是不理政事堂中的变局,弃了要职,只求管着军器监。

“韩玉昆所倡导的束水攻沙之策,是否可行姑且不论。但他在开封主持修堤,造福万民亦深受流民所礼,则是明明白白。”

“黄河金堤如何能不修?一旦要修,都少不了要驱动民力。而为政之上善者,就在于不扰民——韩冈可是做到了。去岁从洛阳逃回去的流民,都是求着要韩玉昆在主持。是洛阳此地的主持之人有过,若有韩玉昆主持,当能皆大欢喜。”

司马光点着头,二程的话说得的确没错。

黄河从洛阳境内穿过,虽然有北邙山挡着,不惧黄河水患。但修堤毕竟是事关百万生民的大事,司马光当然时刻挂心。当河北流民逃离洛阳工役,而跑回开封求着韩冈来主持,其所作的一切,换作是谁来评述,评价再低也得给一个‘能吏’二字的评语,想找茬都不容易。

前年去年的连绵大灾,其中的粮商一案和郑侠一案,都跟韩冈脱不了关系。

但司马光和二程都不可能回护囤集居奇的粮商,轮到他们来主持,此辈奸商必然也是要严加惩处。而安置流民数十万,不使其致乱,放在谁人眼里,都是天大的功劳。

熙宁六年七年的天灾,那是王安石的错,与韩冈无涉。至于仗义执言的郑侠会因为韩冈而被贬恩州,也是郑侠他本人有错在先。谁让他攻击韩冈,如果不涉白马县事,只论京师之事,韩冈又怎么可能有理由上殿驳斥?

从这一件件事看来,韩冈绝不是攀附新党而求高位的奸佞,甚至可以算是有为的能臣。但他坐视新法残民就是有过,司马光怎么都做不到对他没有看法。

而且韩冈造铁船,无论如何想都是无用于国、浪费民脂民膏的行为,是为了宣扬浮力之说,而特意造出来作为证明的。

“私心重了点!”看着犹在辩说的二程,司马光得出了结论。却不只是在说韩冈。

注1:古音壳、确同音,参见平水韵。

