\section{第11章 城下马鸣谁与守(12)}

盐州城被困,连同京营禁军出城反击失败的消息,早早的就传到了鄜延军主力驻扎的夏州。

对于盐州城的困局,以及盐州派出来求援的信使,种谔没有半点的推诿,当即传令全军整顿兵马,救援盐州。而就在当天,他就领着千名骑兵先期赶往宥州。

宥州在夏州之西,也在无定河边,正好是位于夏州和盐州的中段。虽然种谔一力主张驻守银州夏州,但宥州城中还是有着两千多兵力,同时粮草数量也很多,充当着保护道路的骑兵队伍巡逻时的中继点。

不过到了宥州之后,前面挡路的西贼数以万计。种谔当然也不可能带着几千人马就往敌阵冲过去,不得不先在宥州停下来,等待后面人马赶来。

千军万马不是张张口就能立刻出动的,需要调遣的时间,也是理所当然的。直到盐州被围的第六天,种谔依然还在宥州,而且因为阻卜人的活跃,甚至有一支杀到了黄河边,让河东路狼狈不堪,使得种谔不得不分兵保护道路,使得聚集兵力的速度又慢了两分。

救援行动的缓慢,鄜延路这边有着充分的理由,任何人都不能从种谔的行动中挑出错来。局势正在依照着种谔事前的期望而变化着。

对于种谔的盘算,了解最深的当然是他的几位子侄。

“最好的时机是城破前后的一两天,这时候攻到盐州城下,兵困马乏的党项人必当难以抵挡。不过环庆路那边当也会出兵援救,五叔能在宥州一坐五六日,耐性可比过去强多了。”种师中在他的嫡亲兄长面前直接拆穿了种谔的心思,丝毫没有讳言,“听说西贼的太后也来了,秉常和梁乙埋也都到了,若能将他们一网打尽,可比攻下兴庆府的功劳还要大上许多。”

在种谔幕中主管机宜文字的种建中,刚刚处理完十几份公函,又为种谔起草了两份军令,好不容易才休息下来,亲弟弟就过来聒噪。

种建中用倒满热茶的茶杯暖着手,对种师中的兴奋不以为然:“西贼围了盐州城,周围要道、据点都遣了重兵把守。想要如愿以偿,左村泽、柳泊岭以及铁门关都要尽快攻破。否则不是西贼就此重新控制盐州,就功劳给别人抢了去。”

“哥哥说得是,想要掌握好救援的时间,就得先将道路重新打通。”种师中嘿嘿笑着,就在种建中身边坐了下来,“不知道十七哥能不能做到抢先一步成功。高总管想必跟五叔有着一样的盘算,西贼一边要攻打盐州城,一边还要分兵拒我王师,其实也够辛苦的。”

种建中放下茶盏,冷声道:“这时候不搏一把,给官军占据了银夏之地,之后他们还能机会吗?”

“哥哥说得是,就是这个道理!”种师中一拍手,“所以眼下的环庆路肯定是养足了精神,就等着摘桃子呢。”

“高遵裕要将功赎罪,肯定不会放过这个机会。”种建中冷笑,“要不是吕大防托辞辞了任命,庆州知州就该他做了,环庆军也由他来主持。高遵裕只能靠边站。”

在过去,对蓝田吕氏兄弟,种建中都是很尊敬的。即便是在背后提起,肯定也是称呼表字,或是称呼官职,但现在却在人后直呼其名。

种师中也不是聋子瞎子,因为吕家兄弟投向程门的缘故,加上吕大临在张载行状上做的手脚,关学如今正是一分为二的局面,一部分人站在吕家兄弟那一边,但更多的则是在韩冈的支持下,坚持着张载留下来的道统。

种师中疑惑道:“吕大防不是因要为兄弟守丧吗?吕判官【吕大钧】是当真病死了,怎么叫托辞?”

“什么时候听说过边臣要为兄弟服丧了?就是遇上父母之丧,边臣都要夺情,何论兄弟。这分明是畏战!”

种建中一顿火发过,焦躁的心情也收敛了一点,“眼下官军驻扎在宥州寸步不移,而盐州那边还不知道还能拖多久。得尽快冲破过去,否则盐州城破,反倒是我们成了送上门的肥肉。若是给环庆军抢了先,情况就更糟了,五叔多半又是几天几天的睡不好觉。”

……………………

小院中的两株紫薇在秋风中凋零。

昨日夜中的一股寒流,不仅给燕京析津府带来了第一场降雪,也让紫薇树的落叶洒满了庭院。

晨光洒满小院的时候,两名汉家的婢女拿着竹耙,清理着满地的落叶。耶律乙辛透过半开的窗户,看着这两名正值花信韶龄的美婢打扫庭院,神色间是难得的轻松自在。

紫薇是从南方移植来的花木,光滑的枝桠都快长到了耶律乙辛的窗外。这一种长不高的花木,也许在南方算不上珍贵,但在北方,却成了耶律乙辛居所中的装饰。

繁花落尽,树叶凋零,紫薇树只剩下光光的树皮。这样光溜溜的树干,猴子上去也得滑下来,也不知是谁起的诨名流传在民间。猴脱刺的名号,也是给捆绑上了。

耶律乙辛听人说过,紫薇在百花中花期最长,从夏至秋,百日常开。但如今已经是深秋,花期过了,却无人在意。

已经是深秋,秋捺钵的日子早就过了。在往年,这时候都该启程往举行冬捺钵的广平淀去了。但耶律乙辛甚至放弃了城外的御帐,而带着年方幼冲的天子,住进了析津府城中。天子起居在宫室中,耶律乙辛则是在宫室边找到了适合自己居住的宅院。

身居宅院中,耶律乙辛大权在握的发号施令。前来求见的官员数以百计,让人们完全无视了天子。

不过就是权柄独揽的耶律乙辛,也是不能将兵力随意调动的。尤其是宫分、皮室这样的精锐,全都是一人三马,胃口比得上十几名步兵。对他们的调动一次,都是几十万担的粮草成为泡影。

耶律乙辛也是不想浪费宝贵的粮食资源,现如果不能从宋人得到补足,那么消耗的必然都是南京道为数不多的存粮。

“打仗嘛,就跟做生意一样,必须要有赚头。”张孝杰早早的就来到耶律乙辛下榻的住处,跟着他入宫向天子请安,“不论是直接抢掠,还是设法从宋人那里取得,都得避免折本的结果。”

“说得没错。本钱的确就那么多,想要靠着本钱不断的赚钱,就得多想想办法。”

耶律乙辛在寝殿外等候着天子传唤。对于如何压榨宋人这个关键性的问题,他也一直在考虑,并根据局势的变换而改变其中的某些细节。

一名十二三岁的年轻人从殿中出来,在耶律乙辛面前磕了头,“天子已经在殿中等候,还请尚父、相公进殿。”

跟在耶律乙辛的身后亦步亦趋,张孝杰抬眼望着前面的同伴,“那是阿骨打吧?变得还真快。不过出去几个月,回来一看,都变得像模像样了,看不出来是才来”

“是个聪明人,知道怎么向人学。”耶律乙辛话声沉稳。

张孝杰轻笑了起来:“日后劾里钵当有得头疼了。长子乌束雅和次子阿骨打之间,哪一个统领部众都不会给完颜部丢脸,能有一个就该喜出望外了,可完颜部偏偏有两人。再过几年可是有得乐子看了。”

“那就最好。”耶律乙辛的话很简单,但他一样想看到完颜部的四分五裂。

耶律乙辛将完颜部部族长的次子放在小皇帝的身边做宿卫,这段时间以来,多少人都在暗中观察着他的一举一动,而所有人对他的评价都很不错。

张孝杰向耶律乙辛请示:“宋人已经在给下面的仆佣开始种痘,据说效果很好,种过痘的都没有的天花。是不是早日给天子种上牛痘?省得他们再继续收买人心。”

“暂时还不用。”耶律乙辛有着足够的耐心,“等过几天再说。天子种痘事关重大,不能草率从事。在外面找人练手就练手好了。”

“下官明白。”张孝杰应了一声,却又道:“不过南朝副使蔡京不好打发,他的眼睛太毒,许多事都瞒不过他。”

宋人派来的使节,正使沈括,副使蔡京、夏元象,上百号人马,吃喝拉撒都是靠大辽来支持。消耗之大,绝不在战争之下。

“沈括不好应付。而他下面的蔡京也不见简单。南朝人才济济。”耶律乙辛叹着气,他手上可就是缺能做事的官员,韩冈、郭逵、王韶这样的名帅他不指望了,沈括渊博、蔡京智高,纵然没有军略上的成就,也是难得的人才。可惜在他麾下,就是张孝杰、萧十三都算是人才了。

“不过就是人才再多,也比不上尚父的超凡绝伦。”

耶律乙辛微微一笑,却不顺着话接口,而是转过话题:“郭逵现如今坐镇定州,韩冈在河东。两边相互照应。想在他们身上打出缺口,并不是那么容易的。”

“尚父是胸有成竹了。”张孝杰用着陈述句的语气。

“不过是另辟蹊径而已!”

