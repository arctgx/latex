\section{第13章 羽檄飞符遥相系(二)}

【昨天有事出去一天,今天早上起来赶,早上还有两更。】

与叶孛麻几句话定下来盟约和后续的计划,仁多零丁停下脚步,回头看着跟在身后的外姓诸将,“各位,如今大夏是亡了,要打要救,那就是拓跋族嵬名家的事了。我们党项羌自己得想自己的出路。我和仁多公已经定下盟约,从今而后,共同进退。不知诸位可愿意与在下二人一同去商议一下对策?”

终于等到两位领头人的发话,一群平日里都是威风凛凛的将军,现在则是一个个争先恐后。

“愿从两位枢密的号令。”

“惟以枢密之命是从。”

一群不知前路在何方的党项部众,对于投降大宋之后的结果还是有几分畏惧,若能与仁多家和叶家抱成团,这样也能安心一点。

身为外姓诸将的一员,仁多零丁对这种心理洞悉甚明,笑道:“从今往后,也没什么枢密不枢密了,就像从前一样,各部坐在一起商量商量,合计一下。喝点酒、吃点肉,就把事情谈好了。总得我们党项人寻条最稳妥的出路。”

叶孛麻也道:“到时候大家伙儿同进共退,也免得被人欺负。”

一群人大点其头,哪里还有二话。

“李太尉。”叶孛麻扬声招呼着默不吭声的李清,“一起去合计一下,到时候,也能互相照应。”

李清拱手笑道:“我汉军自当与诸位同进退。不过在下营中尚有急务,待安排妥当,便来与诸位共襄盛举。”

冷眼看着仁多零丁和叶孛麻私议之后,大摇大摆毫无顾忌的召集了外姓将领一同去仁多部的营地议事。李清却是砌词推拒了叶孛麻的邀请,上了马,直奔辕门之外。

他是汉人,在投了大宋之后,没必要、也不应该跟党项人走得亲近。依大宋过去的惯例,日后党项各部掳走的汉人,少不得要都赎回来,到时候,说不准会乱上一阵,跟他们同进退只会坏事。

而且现在最重要的是掌握军队。党项人不用担心,用血缘和婚姻联系起来的关系,让仁多零丁、叶孛麻等人可以稳稳的掌握好本族的军队。但自家的兵权来自于西夏的官职,如今西夏败亡在即,有野心的人这时候肯定会忍不住要动手脚了。

李清不知道自己出来前布置下去的安排,能不能稳定到自己赶回大营,心急如焚下,连连挥鞭,带着上百亲兵,直奔汉军大营而去。

而仁多零丁和叶孛麻也紧跟在李清之后离开了中军主营,与十多名党项将领们一起,转向仁多家的大营。

……………………

“全都走了……”

嵬名济咬着牙,环视帐中。空空落落的大帐中,不是姓嵬名的,要么就是跟嵬名有姻亲的。已经不见一名外姓的将领,就是李清那个汉人,也趁机跑了。

“太尉。”梁乙埋叹了口气,“树倒猢狲散,人心散了,也是没办法的事,还是想想怎么应对现在的局面吧。”

嵬名济的眼神压了过去:“太后垂帘,国相秉政。大白髙国的国政全都在你们两人手中。如今的局面,是谁的责任?”

梁氏兄妹能掌握国家大权,一个是他们控制了国家的统治体系,地方上文臣的任免权就在他们一言之间,同时御园六班直和三千环卫铁骑的兵权,也在他们手中。

除此之外,另一个重要原因就是嵬名家对他们的支持。尤其是在秉常亲政之后,其倒行逆施,残杀重臣,并盘剥了大量的牲畜送与辽人,惹起了宗室们对他的极度反感,从而导致了梁氏囚禁了秉常,重新垂帘听政。

但梁氏兄妹秉政带回来的是国家破灭的结局,嵬名家纵然还有着西夏国内三分之一以上的实力,但以当年太祖所做出的那一番事业,辽人和宋人,对他们只会有压制和清洗,以防再出一个李继迁,绝不可能会有什么的优待。

从高高在上的宗室,一下沦落到丧家之犬,这样的落差,换作是任何人都是难以忍受。

“太祖、太宗和景宗打下的江山,都是毁在你们梁家人的手里!”嵬名济高声控诉着。

梁太后作势便欲发作,梁乙埋连忙抢前一步,苦着脸叹道,“国势至此,乙埋难辞其咎。可若不是辽人背信弃义,局势不至于如此。灵州也赢了,盐州也打下来了,若非辽人暗施冷箭,还是能撑下去的。”

“同意跟辽人结亲的又是谁?!”嵬名济有厉声叫道。

梁乙埋好声好气的跟嵬名济辩解着:“就是不跟辽人结亲,大夏也抵挡不了宋辽两国同时进攻。到时候宋人西征,辽人趁机南下,还是一样的抵挡不住。”

嵬名济恶狠狠的吼道:“没有一年三万向辽国进贡的马驼,国势会衰落的这般厉害?!”

虽说这是秉常那孩子亲政时定下来的贡物的数额,可当时朝堂上,最后也没有几人反对。怎么现在就把罪责算到了他梁家的头上。梁氏虎着脸,在她眼里,嵬名济这人已经不可理喻了。

她忍不住驳斥道;“国势衰落有怎么样?之前我们还不是打得宋人狼狈而逃吗?还不是攻下了盐州吗?就是辽国,不是我们被宋人给牵制着,他敢进我国中半步?!”

“若太后当真有胆,那就回师攻辽!”嵬名济的话,让梁太后更加确认了自己的判断。

梁乙埋耐下性子苦口婆心的劝说着嵬名济:“瀚海就在身后。想退回去的话,拼尽了马力,不顾惜战马的损伤,也不是不可能。但失了战马,铁鹞子还能跟辽人一较高下吗?就算手中拿着钢刀和身上穿着铁甲,没有马那就什么也不是。”

“谁说挡不住的?进入兴灵之前,还有克夷门【今宁夏乌海市南】一关!三山夹持的险要之地,辽人想要突破,还没那么容易!”

“兴灵的空虚,所有人都是知道的。辽军突破黑山的北疆防线之后,到兴灵除了克夷门一处关隘,就没有别的阻碍了。而克夷门的右厢朝顺军司仅存的千余老弱残兵,能阻挡至少为数上万的辽军吗?这是不可能的!现在能派上用场的,也就盐州这里的兵了,若是再拼光了这些子弟兵,可就连落脚的地都没有了!”

辽人占了兴灵、黑山,宋人占了银夏、河西,大白髙国的国土除了荒漠和山峦之外,就只剩一个盐州。要想收回其中任何一处,就要有另外一方的支持。现在辽国明摆着要与宋人平分西夏,宋人纵有不满,但终究还是会答应下来。这样的现状,梁乙埋升不起半点战意。

而他如此态度,却让嵬名济的眼神越发的阴郁起来,“若不敢决死一战,就退位让贤吧,大白髙国的国运,还是该掌握在我嵬名家的手里。”

梁太后听得越来越怒,最终忍不住叫道:“嵬名济,你是不是要做反了?!”

‘糟!’梁乙埋大惊失色,这句话不能说的!

还没等他出言缓和,嵬名济已经阴狠的笑了起来,“做反?……早就该反了!”

梁乙埋一下就免得面无人色,梁太后则指着嵬名济,对在场的其他将领怒声问道:“你们就看着他发疯?!”

无人回话,嵬名家的将领们都沉默着,眼中闪着沉沉的光,让人看着心头发寒。

见他们没有半点反应,梁氏终于慌了神,提声对外大叫着:“来人,快来人!”

嵬名济面无表情的从腰间抽出腰刀。今天的军议是仓促中召集众将,又是在军营中举行。将领们佩带的武器都没有在入帐前收取。这一下的疏忽,却是派上了用场。

梁太后嘶声力竭的叫喊着外面的人进来,嗓子都破了声,却是没有半点回音。而梁乙埋则是苦苦哀求,以至于涕泪俱下。

嵬名济的心没有半点动摇,手持钢刀,杀意凝聚:“还请两位上路!”

……………………

从中军大营中出来,李清用最快的速度回到了他的营地中。

背叛西夏对他来说没有半点犹豫,下面的人都会支持他的决定汉人在西夏从来都是被压榨的对象,税赋、差役,都是排在最前面的第一等。没什么人会对西夏有所留恋。即便是犯了法逃到地下,只要一纸赦书,就能让他们立刻丢下西夏这张皮。尤其是现在的形势下,他麾下的汉军更不会支持继续与宋人为敌。投靠大宋,才是汉军上下共同的选择。

现在唯一要担心的,就是有人自作主张,先带了人跑去投诚。四千多汉军,是李清在大宋立足的根本,也是他为自己争取利益的本钱。兵力多寡,决定了投降之后官位的高下。若是少得多了,不仅受封的官位要低上几级,也会被人看成不会收拢军心的庸将,日后再想出头,可就难上加难了。

李清一路快马,只是快到营地门口方才稍稍收敛了速度。就在大营门口,高高低低站着十几名士兵,其中领头的一个,就是李清这几年最为看重的武贵。

