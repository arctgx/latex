\section{第20章 心念不改意难平(九)}

【还是三更,昨天……现在算起该是前天了。前天欠的两更还是没能补上,试试看明天能不能还上一章。凌晨更新,求红票,收藏。】

在王韶手上占了个便宜,高遵裕也不把心中的得意亮出来。温言道:“苗授为人胆识过人,又读过书,不是那些粗鄙不文的庸夫可比,子纯你见了他后必然喜欢。”

王韶也没有多少失意的感觉。他前面会犹豫,是因为高遵裕在今次的封赏中,得以晋为秦凤路钤辖——也就是说现在秦凤路上有三名钤辖,比起正常的情况要多上一名——如果都巡一职再给高遵裕的人抓到手上,缘边安抚司的兵权等于就是被他控制了。

不过毕竟高遵裕现在还是自己人,而王韶也自信他还是能控制得住场面,笑道:“即是胡翼之的弟子,想来是不会差的。”

安定先生胡瑗,与徂徕先生石介、泰山先生孙复并称于世。著作等身,是前朝有名的贤者大儒,更是时所公认的‘真先生’。曾统管国子监,为一代学宗。

虽然胡瑗时运不佳,没能考上一个进士。但他凭着对儒家经典的阐发,为周易、论语、春秋做注疏,又有《武学规矩》传世。他在苏州湖州教书育人,名声日振,前来投奔他门下的士子数不胜数,就连范仲淹的儿子范纯佑、范纯仁亦是出自他们下。

最终他在四十四岁的时候,被范仲淹举荐入朝,一出仕便得了秘书省校书郎的官衔,虽然是从九品,但却是个京官。

胡瑗在苏湖两地办学,将学生分为经义、治事两斋,对弟子因材施教。治事斋的弟子,学习诸经要义,而治事斋下,又分为治民、讲武、堰水、历算诸科,斋中弟子都是主选其中一科,再辅修另外一科,学成后便是经世济用的人才。‘明达体用’这四个字的座右铭,在胡瑗的学校中,得到了很好的体现。

胡瑗的弟子‘皆循循雅饬’,‘衣冠容止,往往相类’,苗授当是治事斋讲武科出来的学生,王韶希望他能不辱其师之名。

两人把西路都巡检的推荐定下,看看时间已经到了午时。普通百姓是一日两餐,午时对他们来说并不是饭点,但王韶、高遵裕都是高官显贵,却都是一日三顿少不了的。

“王惟新。”王韶提声叫着门外亲卫的名字。

一名二十上下的黑瘦汉子立刻走了进来。王惟新是王韶新近从他的随扈中刚刚提拔起来的亲卫,在王韶原来的几个亲卫各自为官的时候,不得不重新又找人来统领他身边的随扈。虽然王惟新武艺算不得高明,但为人认真朴实,对命令从不打折扣,这是王韶抬举他的主因。

可他不是听到王韶的声音才进来,而是进厅来禀报的,“安抚,钤辖,张香儿求见。”

“让他进来。”

王惟新领命出去唤纳芝临占部的族长进来,王韶则转头对高遵裕苦笑,“都是自找啊,世人都说当官好,看到我这模样,不知他们还会不会这么想。”

“忙过这一阵就好了,最多再一个月……”

高遵裕正说着,张香儿慌慌张张的跑进来,扑通一声跪倒在地,“王安抚,高钤辖,大事不好了!大事不好了!今天西面有消息传来了,康遵星罗结要起兵了!”

“是星罗结部的康遵星罗结?!”高遵裕惊问道。

“是!是!”张香儿直点着头,偷眼上望,只见高遵裕面有讶色,但王韶却没什么反应,深沉的眼神罩着自己,让张香儿心底有些发寒。

王韶是在猜着张香儿的慌张模样到底有几分是真。在这个看似胆小如鼠的族长带领下,纳芝临占部历经两次战事,在附宋七部中吃得亏最小,占得便宜最大,如今七部合一,尽数归于纳芝临占。张香儿手上的实力,甚至已经超过了战前,在青渭一带,跟俞龙珂、瞎药鼎足而三。

而且在今次李宪带来的封赏中,他也是跟青唐部的两兄弟一起,得到了蕃部巡检一职,占尽了便宜。这样的人物,却是遇事一惊一乍,王韶怎么想都觉得张香儿的狼狈和怯弱,至少有一半是装出来的。

高遵裕却没想那么多,只催着张香儿让他把事情的详细快点说出来。

“小人也没听到多少,就是从西面传来消息说,康遵星罗结如今受了木征的支持,正在联络当初跟随董裕的各家部族,说是渭源堡扩建后,朝廷就会拿他们祭旗,要先下手为强!”

张香儿的话,王韶只信一半。但康遵星罗结投靠木征,联络诸部的消息应该不会有假。

虽然董裕死了,结吴叱腊也被砍了脑袋,但当初与董裕一齐来攻打附宋七部的星罗结部却依然逍遥。当日,俞龙珂和瞎药兵少,只能盯着董裕本部打。却放跑了康遵星罗结。让他带着战利品轻轻松松的回到了族中。

从康遵星罗结在古渭之战中的作为上看,他也是条会看风色的狐狸。不过他的部族就在渭源堡不远处,一旦渭源堡增筑,星罗结部就要直面朝廷官军。以他在古渭之战中结下的仇怨,也难怪他要投靠木征,来抵抗朝廷。

高遵裕摇头叹气,:“渭水边的尸首还没被乌鸦吃光呢,想不到又有不怕死的来了。”

……………………

李德新陪着韩冈在各间病房中巡视着。每一间病房过去都是一栋营房。几天过去了,秦州内外的军中伤病,都已经转移了过来,人数有百多人。送来的伤病员按照病症不同,被分派到不同的病房中。

这些伤病看到韩冈,只要能起身的,便是纷纷起来向韩冈行礼,有的甚至是跪下来叩拜。韩冈看这架势,再看他们脸上的虔诚,心中也不知是该喜还是该忧,他药王弟子的身份在民间当真是被坐实了。

被人当着庙里土木偶像拜着,韩冈只觉得麻烦,绕了一圈后就匆匆回去了。不过回去之前,还找了仇一闻商量了一下,如何用最短的时间培养出合格的军中急救人才。

——郭逵已经同意了韩冈建议。打算在秦州军中选拔卫生员,不过因为郭逵听着不顺耳,却把名字改了,改称医工。在郭逵报请朝廷批准的奏文中,声明要在每一个百人都,都置拯危急医工一员,专司战地急救,俸禄比照队正。

郭逵在巡视疗养院的第二天,便上书朝中。不论是秦州疗养院上,还是在随军医工之事上,他比韩冈都显得还要急切。这绝对不会是拉拢韩冈的手段,以郭逵的身份,真要拉拢人,绝不至于做到这般地步。

只是郭逵的目的虽然不是为了拉拢韩冈,却不代表他没有一石二鸟的想法。他做的事,都是对韩冈的支持,确信韩冈会对此感激万分。

不过韩冈见到郭逵时,却向他辞行:“秦州事已毕,疗养院中下官已经安排好了,有仇一闻主管,李德新辅佐,院中诸事可保无忧。古渭那边的事下官已经耽搁了太久了,不便再拖延,过两天下官就想去古渭。”

韩冈在渐渐变得冰冷起来的眼神中,保持着谦虚恭谨的微笑。而他将郭逵的好意三番两次的拒绝,对于可能招致的愤怒,韩冈早有了心理准备。拒绝上位者的好意,带来的可不是洒脱一笑,往往就是毫不留情的打压,正所谓敬酒不吃吃罚酒。

郭逵如冰刀一般的视线渐渐缓和下来,在他脸上已经看不到半点怒气。他微笑着:“该去的,当以公事为重……不知玉昆你什么时候回来?你是管勾秦凤路伤病事,路中有五州一军,寨堡数百,可不只是秦州一地。”

韩冈明白郭逵已经有了让他无暇在古渭寨久留的想法,只是他自有主张,“有秦州、甘谷、古渭三个样板在,各地依样画葫芦即可……只是这事还要劳烦太尉说上一句。”

“本帅说一句就够了吗?”

“秦州有太尉坐镇,是秦州上下的福气……非太尉威名,不足以震慑众军。”韩冈说着最后一句,声音有点意味深长,似有隐义。郭逵听了,脸色渐渐有了变化。

“大哥儿,你怎么看?”韩冈离开后,郭逵问着自己儿子对韩冈的看法。

郭忠孝道:“韩冈为朝廷效力,非与大人为敌。合则来,不合则去,没有大不了的。”

郭逵暗叹着,自家的儿子是有些书呆子气,在程颢程颐那里都学傻了。不过话说回来,儿子性格宽厚,总比因睚眦之怨便记恨一辈子的小人要强。

郭逵也没心思跟韩冈过不去,韩冈的话儿子听不出来,但他是听得分明,道:“托硕、古渭两役,皆是蕃人出力厮杀,王韶即未厮杀阵上,又未运筹帷幄,不过是说动了蕃部,让他们出战,自己在城中等结果罢了。但木征不同,手绾十万大军,光靠蕃人根本无力与其拮抗,不出动官军是不可能的。王韶要掌着他的缘边安抚司,就由他去好了。但河州不可能不打,只要动手,这统领全军的帅位,可不是区区一个缘边安抚司能接得下来。”

“大人意思是?”

“战事展开的越大,为父领军的机会就越大。若是一次出动个三五万兵,除了为父,谁能镇压得住?我也是盼着王韶能在古渭早日功成,打好根基……”停了一下,他叹道:“韩玉昆可真是个聪明人!”

