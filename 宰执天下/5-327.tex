\section{第32章 金城可在汉图中(六)}

自酸枣重新启程,韩冈丢下了大队,带着十几名伴当便向着太原紧赶慢赶。

一路穿州过县,兼程并道,日出即行,日落而不止。在出发六日后,便抵达了威胜军的铜鞮县【今沁县】境内,离开太原就只剩下两天的行程了。

只是离行程的目的地太原越来越近,韩冈一行人的心情却越来越沉重。从北面过来的信使越来越多,从他们身上得到的情报也越来越危急。

尤其是太原府的王.克臣,他是一天一封求援信,雪片一般的飞往京城。由于军情用了太多的铺兵,马匹更是都被换了,路上走得再急,抵达威胜军的时间也硬是比韩冈计划中的多消耗了一天。

而且据最新的军情,辽军已经占据了忻州,州中到底还有几处城寨没有陷落,被堵在石岭关上的官军斥候那边并没有确切的消息送回来。只是从情理上说,忻州州治所在的秀荣县,应该不会那么简单就陷落。

计算时间,韩信正在赶往忻州的路上,若是遇上了辽军,恐怕会难以脱身。这让韩冈很是为他担心。河东的军事形势恶化得太快,让韩冈始料未及。

辽军现在的动向对太原的威胁性越来越大,石岭关下据报已经集结了辽军重兵,虽说石岭关是险关要隘,但也不是没有被攻破的可能。

“枢副。可是在担心太原?”

黄裳被韩冈留在后队,现在跟在韩冈身边的幕僚,是田腴,也是气学三字经的作者之一。田腴察言观色也有了些许经验,看见韩冈脸色沉重,就知道肯定又是在想河东的事了。

“我是担心啊,这一仗打下来,河东百姓今年的口粮还不知在哪里?”韩冈轻声叹,“今年明年,河东可能会出大事。”

虽然方才并不是考虑的这一事,但将帅为一军之胆,韩冈不想在下属面前表现出对战局的担忧,宁可说些民生方面的闲话。

已经是春时,野外花开遍地,官道下的田间地头都能看到农人忙碌的身影。暮色下,多有扛着锄头、赶着牛羊回家的农民,依然安详如素。可想想太原以北,已经沦陷的国土,那里的百姓逃难还来不及,如何还有心思去料理自家的田地。

“只要能及时收复代州,还有可能补种,而且朝廷肯定会发给赈济的。今明两年的钱粮,只要枢副上表申请,朝廷自无不允之理。”

看到田腴考虑起如何避免河东缺粮的损失,韩冈微微一笑,但立刻又沉下脸来。其实现在最让他担心的并不是河东百姓下半年的口粮,也不是石岭关,而是赤塘关。

当年宋太宗在夺占了太原之后,又将目标放在了北方。当他会兵准备北上,却被石岭关给挡住了去路。官军几次攻城不果,宋太宗便明修栈道暗渡陈仓,明着攻打石岭关,精锐却潜行向西,一举攻下了赤塘关。

同样勾连南北,赤塘关和石岭关在这一点上没有任何区别,攻下了其中任何一座关隘,就意味着已经打通了南下或北上的道路。但作为关隘,赤塘关的防御水平远比不上石岭关。从自然生成的地理条件,到人工修建起来的城防,都差了不止一筹。

若是辽人学太宗赵光义的故智,夺占了赤塘关,形势将变得极为恶劣,甚至有可能会难以挽回。尚幸辽人毕竟不擅攻城,想来事情还不至于会向最坏的局面发展。

将担心放一边,韩冈打马直行。当一行抵达威胜军郡治的铜鞮县城时,天色已晚,城门也已经关闭。

一名伴当上去喊话。看到韩家的伴当只是在城下喊了两声,又将身份凭证送了上去,城门就被打开了。田腴就有些咕哝:“防卫不算严密啊。”

“那是之前已经遣使传过话了。沿途州县早就知道我会经过此处。”

之前经过的几座城市,都是不巧在白天经过,进城时没有受到阻碍。但换成了宵禁中的城市,可能就不一样了。不过实际情况是两回事。田腴本以为会有些波折,但看到城门官都从城中跑了出来,倒是吓了一跳。

韩冈出京时走得太急,身边的人手不足,整个制置使司的架子都没搭起来。黄裳带着大队在后,身边也就一个田腴,已经跟了很久了。

因为韩冈的举荐,田腴得到了一份官职。不过并非流内品官,而是一个上州的助教,有俸禄,无品级。虽然田腴只是文学好,可换作到了制置使司的幕中,做个文书,打理一下手边的杂食,田腴也不至于这些小事都做不到。

而且到了太原情况就不一样了,韩冈在太原的门生故吏甚多。且王,克臣终究不可能将整个衙门全数清洗,借调几个熟悉的下属也不成问题。

“知军和知县都吩咐下来了,只要相公一到,就立刻通知他们。”监门官殷勤且恭敬,弯腰低头,为韩冈牵着马。

韩冈并没与监门官多闲话,直接放马进入了城中。

比起还算缓和的乡间,城中气氛则多了许多了临战前的紧张。方才在官道上,一路看到的客商不及往年的三成,而在城中,犹在街面上的普通百姓,也是一副强颜欢笑的模样。街道两边的酒楼茶肆里,灯火照耀下,多有人聚在一起窃窃私语,却少了应有的丝竹弹唱。

天色已晚,韩冈很快就在城中的驿馆安歇下来,得到消息的知军、通判和知县等大小官员纷纷赶来拜见。韩冈出去见了他们一面,多说了几句话,让他们着意安抚人心,并要准备好钱粮。威胜军驻军并不多,加上之前河东出兵援助,韩冈也只要求做好粮草和军械的准备。当然,城防更是重点。

虽然明日要还要早起,没有时间多耽搁,可韩冈为制置使,必须得到州县的支持,不能不为此多花上一点时间。

寒暄过、叮嘱过,送走了一众官员,韩冈赶紧上床休息。只是他睡下了还没半个时辰,外面就是一片的嘈杂声。韩冈脾气一向不错,但这也并不代表他没有起床气。板着脸在客厅坐下,便领着客人们鱼贯而入。

州里的知州、通判、判官、推官、参军,县中的知县、县尉、主簿,全都挤进了小小的厅室中。

看到他们脸上的神色,韩冈彻底的清醒了,情况不对。威胜军的知军现在是一脸恍恍惚惚的神色,仿佛是陷入噩梦之中而无法清醒,进门时甚至还在门槛上绊了一下,差点儿就一头栽倒。

“怎么了?出了何事。”韩冈心中有了不好的预感。

威胜知军却好像忘了词,张口结舌,一个字也倒不出来。

“枢副!”田腴声音嘶哑,在旁替他说话,“辽人攻破了石岭关!”

厅中一片寂静,就连韩冈的一行伴当都停了动作,仿佛凝固了。这怎么可能?那可是太原的门户啊!

韩冈没有发怒,甚至有一瞬间,他感觉这件消息当真是荒谬绝伦,“辽人怎么做到的?!”

韩冈当年遍巡河东各军州,太原的北面门户他也经过了好几次。韩冈可以百分百的确定,两座关隘中,石岭关的城防体系要比赤塘关更为完备,也更加难以攻破。没道理石岭关破了而赤塘关安然无恙。

没人能回答这个问题。过了半天,威胜知军小声的说着:“石岭关隶属忻州,赤塘关则是太原的。”

韩冈不敢相信王.克臣会因为石岭关属于忻州而不放在心上,他低喝道:“王.克臣是河东经略!”

威胜知军连忙改口:“下官也只是猜测,是猜测。”

“可能是军队败坏了。”通判倒是敢说,“自枢副返京后,新来王经略将学士留下的将校一个个投闲置散,只用了一批无能之辈。这一下子全都露了底。”

王.克臣在经略安抚使任上丢了河东半壁,即便他是英宗皇帝的亲家,也逃不掉罪责。这个罪他能背上一辈子。

韩冈在河东,空闲时练兵不辍,赏赐也没停过,也提拔了好几个有才干的年轻军校,河东北方的军队战斗一流,完全不需要韩冈操太多有的心,一切都是按部就班。但换了王.克臣就任后之后,情况就变了。

平心而论,论起在军事上的用心,王.克臣并不下于韩冈,甚至犹有过之。韩冈在京城也有听闻。每逢校阅必大开府库,从将校赏到参加校阅的保甲乡兵,可他在军事经验和常识上的匮乏,让练兵成了表演,越是会演戏的,得到的馈赏越多。加上王.克臣本身的倾向性,也让那些在韩冈时代得到重用的将校受到排挤,很难再有施展才华的机会。

或许这些问题,随着时间的推移,会有新的平衡重新建立起来。到时候,军队之中的混乱也能够平息,可惜辽人没有给他们这个时间。

韩冈现在距离太原还有两日的路程,但在辽军破关的情况下,他很难跨过这最后的两百里地。

是兵行险招?还是稳扎稳打。

这必须要有个决定。
