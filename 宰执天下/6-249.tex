\section{第33章 为日觅月议乾坤(七)}

“看得见后苑?”

童贯的声音猝然在韩中信的背后响起。

正拿着千里镜往宫城中窥视的韩中信,随即放下手上的千里镜,转过身,笑着上前行礼:“末将见过皇城。”

童贯两只眼睛瞅着他,韩中信却面无异色的呵斥身后的亲兵,“还不快端张椅子来!”

童贯摇摇头。

自己上城来,踏上阶梯,韩中信的手下定然便已报给他听了,可这惫懒东西,却大喇喇的等到自己上城来。

只是他是韩冈的亲将,在河东立功为官,如今在神机营中地位也高,童贯也不想跟他一般见识。

在椅子上坐了下来,顺手就捶了捶腿。

前几年去广西的时候,不小心坠马伤了腿。如今就留下了些后遗症,寻常走路无事,但上下阶梯就有些吃力了。除此之外,还有些阴雨天酸疼的小毛病。

但依靠在广西立下的军功,童贯回到京师之后,就开始掌管皇城司兵马。

而李信在回返京师,继续掌握扩大了的神机营。但神机营如今人数多达万人,他这位都指挥使只有三分之一的时间在京城中,大部分时间,还是在外廓城那边的几座军营里住着。

为韩冈守着宫中的神机营将领,现在就是这位韩中信。

“看得见后苑?”

童贯捶了捶腿之后,朝韩中信手上的千里镜努了努嘴。

“怎么可能?宫城城墙比皇城高,只能看见对面的张温那厮。他今天守宫城吧?”

韩中信说着,随手就将千里镜递给童贯,一点也不在乎这件事已经违反了宫中的禁令。

在皇城城墙四角的敌台上,除了火炮之外,也固定了千里镜,用来观察城中动静。这些都是宫中名匠制造的精品,远眺观察,甚至能看清十数里外城垣低矮的外廓城边墙。

千里镜能视远如近,所以镇守皇城、宫城城墙的兵将严禁以千里镜窥视宫闱。

童贯把玩着韩中信递过来的千里镜,黄铜镜身上有个小小的丁字,却没有军器监产品应有,当是大匠丁满亲手所制,多半是从韩冈手中得到的赏赐。

“官家前两天在后苑拿着千里镜登高,怎么,想学官家?”

“哪里敢啊。”韩中信呵呵笑着,挤眉弄眼的低声道,“不过官家要是娶了狄家的小娘子,怕也就只有现在能这么玩了。”

童贯摇头道:“狄小娘子听闻脾气甚好,不似她嫡母。”

韩中信遂将话题一转:“狄郎君也是,平白长了个好相貌,却那般惧内。”

“这些日子外面也多有笑话说此事。”

“请夫人阅兵?”韩中信没去理会童贯话中深意,“其实这个笑话当初是相公用来笑话定西侯的。”

“哦,是吗?”童贯这回倒有些惊讶了。

王舜臣平定西域,又镇守西陲多年,被封为定西县开国侯,遂人称定西侯。

不过世上人人知道王定西,却没几个知道他惧内的。

韩中信来了精神,说得是口沫横飞,“那时候,定西侯刚刚在襄敏公面前出了头,得了一个官职,种家便把女儿嫁给了他。原本定西侯是种太尉侄儿的伴当,定西侯夫人还在娘家时,定西侯见了还得弯腰。成了亲,这腰杆子也没能直起来。指了东,他不敢去西,叫去抓狗,他不敢撵鸡。所以相公就看不下去了,他把定西侯当兄弟看,便把定西侯找了过来骂了一顿。王定西脾气暴,当场说说回头就带人去给那婆娘点好看。相公就拿手指戳着他脑门,有胆子就自己回去啊!还要带人,就是带了三五百人到了家中,到了你浑家面前,怕不是给她点好看,而是请夫人阅兵了。这件事在陇西,谁知道这笑话现在给安到了狄郎君的头上。”

“狄郎君、王定西一般的怕老婆,这笑话安在谁头上都一样。要不是沈枢密不领兵,怕也给人编排上。”

高官家的阴私事,一向是朝野内外传言的重点。

沈括怕老婆,所以两个儿子都被赶出家门。狄詠要不是怕老婆,女儿怎么会给赶出门去?其实都是一路货。

“相公最是念旧,要不是当初在陇西有一份人情在,如何会帮着照顾沈枢密家的两个儿子,现在一人一个进士,要是没相公照顾,哪里能看到这一天?”

韩中信这话说得过于露骨,可童贯也就点点头,恍若理所当然。

当朝宰相寒微之时便与之结交,童贯如今的地位,少不了韩冈在背后襄助。否则以童贯的资历,哪里来的那么多军功给他?又不是王太尉那一系的,跟着出门就能平白得军功的好事,宫里面哪个内侍不愿插上一脚?可就是没那个命。童贯自知要感谢谁。

“听说圣瑞宫那边已经定下了皇后的人选。”韩中信忽然道。

童贯身子一震。

朱太妃原本是属意王安石的孙女,但这么多日子了,韩冈和他的派系都没有出面反对王安石的孙女,所以朱太妃如今对狄家的女儿热情了许多。

而韩中信提起此事,怕就是韩冈在后面说话了。

当真要让内侄女做皇后?童贯不明白韩冈的用意,但他明白自己该怎么做。

“太妃这半年上的确操劳了许多。不过这件事,太妃说的不算,太后和相公们说得才算。”童贯举起千里镜,“其实这边也能看见内东门小殿……相公们已经到了。”

……………………

“官家大婚,当在明年。这皇后的人选,虽说还要几个月的教养,但其实已经可以定下来了。

待群臣行礼毕,向太后便开门见山。今天她招众宰辅入内详议,便是为了此事。

韩冈却狐疑的用余光看着屏风后,似乎不止一人在屏风后落座。

‘是朱太妃?’

韩冈犹疑,章惇随即接口,“可是王旁、狄谘二人之女?”

朱太妃都没隐瞒过自己的喜好。还没到最后阶段,再过两月,将候选淘汰到十人之内,那时候才是真正的挑选。但她早早的便把自己的倾向展露了出来,使得外界已经确定了皇后的候选人名单。

“正是。王、狄二女子,品貌性格都是上上之选,他人所不能比。王老平章元勋故旧,狄青亦是勋臣,皆是好门户。”

向太后说得很郑重。

王、狄二女,她也觉得很不错。尽管不是她本人的选择,但朱太妃推荐到她面前,也是经过一番挑选的。而且选择她们,总比自己选一个自己觉得合适的,日后却被废了的皇后要好。现在挑选皇后的是皇帝生母,日后即使跟自己斗气,也不会祸害了人家女孩儿。那样简直是造孽。

章惇道:“陛下,礼须夫妇所生。狄氏女嫡母悍妒,女生三岁而逐其所生,今鞠于伯氏,将以所生为父母?将以所养为父母?”

这个问题,向太后与朱太妃商议过。听了,她便说道:“三岁上已过房。”

章惇立刻回道:“女子无过房之说。”

屏风后稍稍静了片刻,似乎有人在对向太后说了什么,过了片刻,太后的声音方才响起:“……若做狄詠女,以狄谘主婚如何?”

章惇没回话,却是邓润甫出面:“故无此礼!天家事,当循礼,不可如小民。”

向太后道:“不得已,则无奈何。”

邓润甫随即反驳,“以国家之盛,岂宜作不得已事?”

“韩相公,你看如何?”

被两位宰辅接连反驳,向太后开始避而不谈,另找他人。

韩冈从屏风上收回自己的思虑,出班行了一礼,“臣敢问太后,若以狄氏女为后,不知当尊礼何人?”

既然朱太妃在这里,韩冈便没去提狄氏女两父三母的问题。

朱太妃就有三个父亲,说狄氏女两父三母、头项太多,却正好有一个成例在。

朱太妃之母先嫁崔杰,之后嫁朱士安,因为不便携女再嫁,故而将女儿托付给了亲戚任家养大,故而是有三父。

另一种说法,则是李氏先嫁给崔杰,崔杰病死,后为任氏妾,再之后才嫁给朱士安。更恶毒一点的谣言,就是朱太妃乃是其母私通所生,要不然为何放在任家养大。

不论哪一种说法,朱太妃都可说是有三位父亲。故而当朱德妃成为朱太妃之后,崔、任、朱三人,皆封师保。

既然有朱太妃追赠三父的例子在前,那狄氏之女的两父三母,也不是什么问题。

而韩冈的问题却是刻薄了。前后几人侧目,看出朱太妃也在屏风后的,不止韩冈一人。

“自当尊礼嫡母!”

朱太妃的声音在屏风后响起。看来她至少知道,在太后面前该如何说话。

“即非其生,又非其养,生而逐之,十余年来并无寸功。尊礼其人,只因礼法所在,故不可违。其人悍妒如此,欲以其女为后,可不虑将来?”

韩冈的话说得很明白了,强势的外戚对身处弱势的天子有好处,但强势的丈母娘可就一点好处都没有了。

“相公也是觉得王氏女更合适?”向太后的声音中不掩惊讶。

“王氏女与臣家有亲,臣须避,不当议。”韩冈道,“臣只知狄氏女虽为上选,其家不甚佳。”

王安石的孙女做皇后,对韩冈的确有所不便,但武将的外戚更加危险。王旁领不得军,狄家可是将门。

枪杆子里出政权,韩冈日常进出皇城,可不会将皇城的控制权,交到赵官家的手上。

韩冈如此反对狄谘之女,却让朱太妃觉得这个人选太对了。

宰辅们虽不同意,变通的办法她也有,“若狄氏女为妃如何?”

向太后想了想,也跟着道:“以王氏女为后,狄氏女为妃,也算是两全其美之策。”

韩冈不以为然,“岂有宰辅之后为人滕妾的道理?!”

韩冈此话一出,屏风后朱太妃的声音立时阴沉了许多:“狄氏庶出,嫡母不贤,难为正宫,做嫔妃岂不正合适?!”

章惇勃然作色:“太妃可是忘了,那是枢密使家的孙女!”

文武固然殊途,可狄青终究是枢密使。

自来嫔妃多出自小门小户,让枢密使家的女子做嫔妃,朝堂诸公哪个能看得过眼?

宰辅们可是连皇后之位都不想要,连公主都嫌碍事。如何会看得起嫔妃?

几位宰辅一顶再顶,太妃怒气上涌:“难道天子还纳不了一个武夫的孙女做妃嫔?”

张璪也坐不住了,“狄青勋臣,又曾为枢密使,岂可纯以武夫视之。”

一个个臣子皆是贱视嫔妃,将朱太妃的火气越逗越高,“吾曾闻,普天之下,莫非王土,率土之滨,莫非王臣。官家纳个嫔妃都要被说三道四。天下的,难道不是官家的?!”

“天下是天下人的天下,非是一家一姓的天下。欲以天下奉己身,非是天子,乃是独夫!”
