\section{第十章 却惭横刀问戎昭(二)}

“‘既载壶口,治滩及岐。既修太原,至于岳阳。’能在《书》【尚书】中留名的古城,到如今也没剩几座了。”

“晋地,山之西,河之东,表里山河,晋文恃之为霸,唐高倚之立国。五季天下争雄,亦只在大梁和晋阳之间。”

“毕竟太原是以龙城为名,王气蕴藉嘛……”

“龙城之名犯忌讳啊,所以我们脚下的不是千年晋阳城,而只是唐明镇。”

“晋阳一如金陵,王气多而寡淡,故而立都于此,从无长久,毁了倒也罢了。”

“多而寡淡,彦直是在说昨天席上的酒吗?”

太原城的城墙虽说是城防重地,但书生们上城眺望远山近水、论史作诗,也就几个大钱的事,能买斤酒,就足够让守城的兵丁放人上城了,天下城池多半皆是如此。

不过眼下兵兴在即,太原城内城外气氛如同绷紧的一根绳索,闲杂人等想要上城,先下狱问一个窥探城防军机的罪名。

但城头上下的官兵,都知道今天上城来的这一干措大,乃是新上任的小韩相公幕中的门客,有份参议军事。监门官都鞍前马后的小心服侍,士卒们当然不敢冒犯半点。几名书生当真在城上犯了什么差错,他们也权当什么都没看到。

不过韩冈的几名幕僚都是通礼法、明事理的正人,能上城看一下城防,差不多也如愿以偿了。没有什么乱七八糟的要求。只是迎着暮色下的清风,望着远处的山岭,长叹古今变迁。

“李存勖立后唐。石敬瑭立后晋。刘知远立后汉,此三人皆是在河东路节度使上起家。‘兴晋阳之甲’,自赵鞅始,叛臣踞此起兵者尤多,风水着实不好。太宗皇帝拆了晋阳旧城,也不为过当。”

“后唐庄宗、后晋高祖、后汉高祖,三人皆成帝业,岂能直呼其名。”

“五季之时,龙蛇起陆,霸业成于兵甲,英雄出于草泽。如朱温、存勖、敬瑭、知远、郭威诸人,此辈不过乘势而起,得意一时。率为蛇蛟之流,并无真龙之相,故而身即死,国遂灭。有什么不敢说的?!”

“柴家尚是国宾。”

“奈何不姓郭!”

几个年纪不甚大的幕僚争论着对五代诸帝的称呼,几个年长的则望着远处的山水,“山势崔嵬,水势奔流,可惜今天没有丹青妙手。否则一幅画,就能让龙图的心情好转起来。”

“王子纯可是龙图的故人和恩主,哪有那么容易就换了好心情上来。”

“王子纯走得实在不对时辰,偏偏是在这个正需要他的时候。”

“就指望龙图早点换了好心情,这样大家都能放心得下。”

韩冈是到了太原才收到王韶过世的消息。

尽管之前已经有了一点不祥的预感,但当真听说时,还是连着几日心情低沉。

一府之尊心绪不佳,太原府衙上,便仿佛有阴云密布。衙中的胥吏,摸不清新来府尊的脾气,就连说话都小声了许多。

直到昨日,韩冈在衙中后院置酒遥祭王韶,才算是勉强从坏心情中解脱出来。

他还有许多正经事要做,至少要抽时间去代州一趟。边防重地的战备情况到底准备得如何,不亲眼看一看,根本放心不下。

“过两天龙图就要去代州体量边寨防务,一路上的行程还不知怎么安排的。”

“知道个大概就行了。过于详尽的细节,反而会有问题,泄露出去就不得了了,要是居中来个劫杀,情况就会不可收拾。”

“李宪已经在太原府中,就不知道他会不会跟龙图一起悲伤。”

李宪领军镇守在弥陀洞,前面有种谔顶着,其实河东军完全可以回返本镇。但为了保证河东、鄜延两路的联系,还是需要在弥陀洞和葭芦川放上一支兵马。而且弥陀洞、葭芦川的位置处于两路之间,不论那一路出了事,都方便援救,所以河东军就没有退回来。但韩冈到任之后,便发文让李宪先把手上的事情放一放,回太原来共议军事。

李宪已经在太原府衙门之中,韩冈坐了主位,李宪为客,而黄裳被韩冈唤来做个陪客。

诸多门客之中,黄裳算是年纪排在前列的一位,比起韩冈都要大不少,学问亦是精深,在韩冈的十几名门客中,很是得人尊重。

黄裳连年落榜,从二十不到,考到了如今的三十五六,对自己的学问几乎都要丧失了信心,所以才会受了韩冈的邀请,入了他的幕府。想走一下韩冈先立功得官,再去考进士的旧路。

“观察可知辽国已经派了使臣出来,说是要调解大宋和西夏的纷争?”一番寒暄之后,韩冈看门见山的挑起话头。

“辽国使臣?怎么个调解法?”李宪不信积年老贼能金盆洗手,“大辽尚父给出了什么章程?”

“一条是撤军,另一条是增加二十万银绢的岁币。”

李宪仰头哈哈一笑,眼中却一点笑意都没有:“好如意的算盘。”

“因为他有二十万铁骑,而且也是漫天要价,让我们落地还钱。”韩冈自问换作自己处在的耶律乙辛的位置上,多半也会这么做,“能实现一半,对耶律乙辛来说就已经足够了。”

李宪点点头,韩冈说的没错,“不论应允了哪一条,都能让耶律乙辛坐稳他的位置。”

“可惜这两条,不论哪一条天子都不可能答应。倒是第三条可以给大辽尚父一个面子。”

“还有第三条?!”李宪惊讶的问道。

“耶律乙辛想要种痘法,这倒是小事,给他也无妨……反正想偷学也不难,还不如大方点。”韩冈哈哈笑道,“承蒙大辽尚父看得起我那点江湖小术,韩冈受宠若惊啊。”

陪客的黄裳摇着扇子笑道:“当是体会到龙图的发明一向管用的缘故。”

韩冈和李宪纵声大笑,辽国天子都从飞船掉下来了,韩冈的发明有多有用,耶律乙辛自然是深有体会。

一番谈笑,外面的云板已经敲起了初更的点。韩冈听到钟点,就吩咐下人去准备酒饭。

李宪却起身,“时辰已晚,不敢再叨扰经略。今日先行返家,明日复来衙中听候经略吩咐。”

李宪既然这么说,韩冈便没有出言挽留:“观察领军在外多日,家中定然想念。既如此,就不耽搁观察了。”

李宪虽然是宦官,也是有家室的,有妻有子。其妻姓王,之前还得了诰命。其子出身寒微,也是有荫补在身。现在就住在太原城中,离州衙并不远。

内侍的官阶,升到从八品的内东头供奉官就到顶了,再往上就要转为武职。转为武职后的宦官,都会出宫置办家业,自然少不了要娶个浑家,来管理家中大小事务。也有的大貂珰,出于某种补偿心理,甚至一个妾接一个妾的买回家。

据韩冈所知,王中正家里的姬妾就有七八人,据说只有一个浑家的李宪,算是洁身自好的那一类。

李宪谢过了韩冈的关心,然后起身告辞出门,回家探视。

待到李宪走后,黄裳拧起眉头:“此阉竖好生无礼!”

韩冈对此付之一笑:“走马承受总不方便与监察的目标坐在一起谈笑,我留饭也只是尽人情而已,原也不指望他能留下来吃。他失礼不失礼无所谓,只要我这边不失礼就行了。”

宦官出外,身上少不了有监察当地官员的差事。在外的兼职走马和一路帅臣走得太近,天子肯定是不愿意看到的。李宪刻意保持距离,其实也不足为奇。

李宪谢绝了酒宴,告辞离开,韩冈也算是松了口气。他可没多少空闲吃喝玩乐,在酒宴上耽搁的时间,可是都要用减少睡眠来补回。随便吃了点东西填饱肚子,便又埋首于公案文牍之中。

河东如今北面有辽人虎视眈眈,西面还有没有结束的伐夏之役,东面又有在必要时援助河北的义务,军务上千头万绪是不消说的。同时在政务上,由于太原府是河东的核心,辖下九县二监,其繁忙也是必然。

在五代时,太原——或者叫晋阳——曾经是与汴梁平起平坐的天下重镇。五朝中有三朝出自太原,而大宋立国后,对十国的统一战争,抵抗到最后的北汉,也是盘踞在太原。

当宋太宗终于灭亡北汉之后,对北汉都城的处理,就是干脆了当的毁弃了千百年来不断翻修的晋阳城。纵火将周围四十里,城门二十四座,以汾水为内河,城外套城,城中有城,为旧唐北都的雄城烧毁,又掘汾河放水彻底毁去根基,并在行政编制上把太原府降格为并州,撤销太原县,将榆次县改为州治,之后又移至唐明监。

同时赵光义还把平定县、乐平县分割出去,设立平定军,将太原东部的井陉这条战略通道隔离出来。失去了东部险隘娘子关,若再有人想凭借太原作乱,必须先收服平级的平定军。否则河北军可以轻易通过井陉天险,直接杀奔过来。

