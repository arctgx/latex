\section{第48章 梦尽乾坤覆残杯(三)}

王中正闻言一愣,他已经很长一段时间没被人用这种口气支派,尤其是关系一向不错的韩冈,更是都没见过他这样的态度。

韩冈却很不耐烦的样子,双眉登时竖了起来,“你当真想要旁听?!”

王中正慌慌张张的摇头,他当然不想听。

宋用臣、刘惟简同样不想听。在场的没一个内侍、宫女想听。

掌握在手中的秘密不一定是把柄,有时候更是催命符。

他们这等天子家奴,听到不该听的话,知道不该知道的事,不论地位多高,也只有死路一条。

尤其方才韩冈翻来覆去问了大半天,在提到小皇帝的时候,总是有些突兀的打岔过去,反应快一点的都知道有问题了。

一想到那个让人不愿意去想的可能,曾经手握十万大军、内侍地位第一的王中正都恨不得逃出殿去,更不用说其他宫人。

韩冈此时的态度虽差,却等于是放了他们一条生路。

“这就出去。这就出去。”王中正连连点头,“会看好他们的。”

瞅着还没得到太上皇后的应允,便一窝蜂冲出福宁殿的内侍、宫女,蔡确也想跟着出去了。

他现在都在恨自己为什么这么晚了还要过来;为什么方才在御街上不一头栽下来,受伤回府;为什么不早点得个伤风感冒,告病十天半个月的。

韩冈的强硬十分反常,越过太上皇后去指使宫人,更是不应该。

当年不过二十出头,就被人视为未来宰相,韩冈一向是以沉稳著称,千军万马都没能让他动摇,今天晚上却出奇失态了。

看到现在的韩冈,任谁都知道这一回事情严重了,而且是绝非一般的严重。

蔡确做官只想着福泽绵长,可不愿沾上这等断头买卖。

明知韩冈现在多半是抱着要死大家一起死的心态,但蔡确现在找不到任何理由来脱身。

没有了数量近百的内侍,宫女。只剩向皇后,王安石,韩冈和八名两府宰执,总共十一人在殿内,偌大的福宁殿顿时显得空旷无比,分外清冷。纵使两侧的暖炉正炽,也驱散不了众人心头的凛凛寒意。

向皇后和宰辅们都在等着韩冈的发言,但韩冈立于殿中,许久都没有一句话。

“宣徽。”向皇后忍不住催促着。

“这是一个意外!”

韩冈的开场白否定了赵顼被谋杀的可能,不过同时也坐实了太上皇龙驭宾天非是顺理成章的病卒,而是出自事故。

有事故,就有原因。

“宣徽可以明说,到底是怎么回事?”向皇后紧张的问着。

“回殿下,是烟气之毒。”

“不可能!”

王安石和韩绛同时叫了起来。

韩绛气急败坏:“烟气臭秽【注1】,寝宫内那么多人,谁会不注意到?韩冈你没看见上皇所用的暖炉,烟气是通了水的吗?臭秽之气,通过水洗之后,可就干净了啊!现在哪家的玻璃烛台、玻璃油灯,不是这样的设计?要是还有毒气,还能有几人活着。”

“的确是这样没错。但只有火气之毒才会造成尸身脸上有血色,如同入睡。而且帐中三人同时暴毙,遗骸之状与上皇一般无二。还能有其他原因吗?这必然是放在帐中的那支暖炉造成的,否则哪里来的毒气?”

炭火燃烧后的气体有毒,就是这个时代也不是什么秘闻。尤其燃烧不充分时,气中多烟,也就是所谓的臭秽,会置人于死地。这样的案子虽然出现的并不多,但也不是没有官员在任职地方时遇到过。

但世人对一氧化碳中毒的认识,都离不开燃烧不充分而一并产生的烟。从水中通过后,烟气消失,毒姓也洗脱了,赵顼帐中的暖炉就是以这个认识而设计的。

韩冈说得纵是有理,但也不是没有其他的可能,将问题推到暖炉上,还是很难让人信服。

苏颂出班,为韩冈助阵,“殿下明鉴。中炭气之毒的死者,肌肤红润,犹如生前。与普通病卒或是中毒而死的尸体,完全不同。臣旧年在开封府,就遇上过两件中了炭气毒的案子。臣虽没有亲自查验,但据当曰推官和仵作的回报,死者都是同样的特征。两件案子的卷宗在开封府中皆有留存,殿下可以遣人查验。”

苏颂原本就有经验,他任职开封的时候,处理过两次有关的案子,其中一次还是灭门案,而章惇则听韩冈闲聊时提起过。所以他们之前看到赵顼的尸身,再听到赵煦挪动了暖炉,才会那么震惊。

向皇后半信半疑,“为什么过去没有听说有多少人死于炭气之毒?石炭在开封府用得久了,暖炉则是新造的,说起来,这几十年,宫中为什么没有出事?”

“只有小门小户才会出事。寻常的富贵人家,屋舍高大,毒气很容易飘散。贫户则根本烧不起燃料取暖。以寝宫之大,上皇本不会有事,偏偏换了床,毒气聚在帐中没有散发出去。”

曾布还是不信:“难道是暖炉坏了?但暖炉坏了就会有烟气,殿中这么多人都没有一个发现的?”

韩冈没有说话,章惇指了指东面,“石炭场。”

所有人都恍然大悟。

因为气味被盖住了。

平常暖炉坏了,绝不会发现不了,偏偏今天石炭场大火,烟雾无孔不入。就是天子的御榻,那一张如同房间的大床被放下帐帘,里面也早有了石炭场产生的烟气,所以没有注意到暖炉漏了气。

难怪韩冈说是意外!

看着宰辅们恍然的模样,韩冈放弃了向众人说明无色无臭,才是炭气之毒——也就是一氧化碳——最可怕的地方。

韩冈现在对气学的态度是希望别人来指出自己的错误,超越自己,继续往前走。

虽然不会故意留下破绽,但对于一些错误的认识,都没有在特意去加以更正,他更希望有人能够通过格物自己去发现。这其中,就包括了一氧化碳中毒。

所以他才没有将这一常识主动公布,而是希望有人能够发现其中的问题,能够给出合理的解释。就是跟苏轼和章惇聊起来,也没有清楚的说明过。

而且韩冈在叩问上皇圣安时,就看到过那只暖炉,也看到了大大的八步床,但他没想过会发现一氧化碳中毒,寝宫人进人出,有事不可能察觉不到。福宁殿里这么多人呢。

但现在当真出事了,韩冈总不能对外说他注意到了,却大意了。

所以韩冈才会说是意外,否则麻烦缠身。

对于赵顼的死因,没有人再有疑问。

了解了死因,对于案子来说,已经算是告破了。

但剩下的问题,却更加恐怖。

因为凶手……说轻点就是肇事者。

是当今的皇帝,太上皇的亲骨肉。

是弑君。也是弑父。

并不是他本身的意愿,但结果如此,动机也改变不了可悲的事实。

“宣徽……当如何处置?”向皇后颤声向韩冈问着。

她的丈夫暴毙,致死的原因找到了,但不可能没人去猜测其中的问题,要么归罪赵煦,要么就归罪于向皇后。

虎毒不食子,只要不是则天皇帝一样的女人,很多时候会为子女担下罪责。但赵煦不是向皇后亲生,要让她在自己的名誉和小皇帝的名声之间做个选择,何其之难?

而且一旦有这些罪名缠身,到了赵煦亲政,肯定会忙不迭的将罪名坐实,别说向皇后本人,就是向家恐怕都逃不过一劫。说不定还没到那个时候,虎视眈眈的朱氏就会在她儿子帮助下,以此为借口夺下太后之位。

只是,她能将责任推到才六岁的赵煦身上吗?

韩冈摇头,“臣一时拿不定主意,殿下何不先问问殿上诸公?”

没有一个开口,就连王安石都不知道该怎么说。

只有薛向大着胆子道:“殿下。旧有故事,此事不为罪。”

向皇后精神一振:“薛卿家请明言!”

“春秋时,许国国君悼公重病,太子止进汤药于悼公,悼公饮药随即而亡。此事究其本心,本为其父病情,所以董子说,君子原心,赦而不诛。”

这一件事,与今曰小皇帝的过失几乎没有两样。

许止进汤药,自己没有先尝便给其君父喝下去。而赵煦没有征求专家的意见,便下令移动暖炉,密闭帐幕。

这都是犯了大错,造成了他们的生父和国君的死亡。初衷虽为好意,却造成了最坏的结果。

西汉大儒董仲舒以春秋决狱之法论许止之罪——许止父病,进药于其父丽卒,君子原心,赦而不诛——可以赦免,不当论其罪。

这是根据《春秋·公羊传》而定义的判决。

殿中的哪位进士出身的宰辅不知道这个典故?但他们为什么不说?却让薛向抢了先?因为在《春秋》原文之中,对于许止的做法只有两个字——弑君。

杀了就是杀了。

无论如何,赵煦弑父是铁案,无法洗脱。

注1:南宋的宋慈在《洗冤集录》中有记载一氧化碳中毒的症状和原因:‘土坑漏火气而臭秽者,人受熏蒸,不觉自毙,而尸软无损。’这应是中国历史上有关一氧化碳中毒最早的记录。按照北宋煤炭的使用情况,也应该会有这方面的认识。

