\section{第五章 九州聚铁误错铸(二)}

【有活动,回来晚了。稍迟一点还有一更。】

无定河是贯穿横山山脉的一条重要河流。

源出横山北麓,上游由南向北,过了夏州之后又转向东行,横穿银夏之地后,到了银州方才转头向南。浑浊的无定河水切过横山,在鄜延路境内一路南行,最后注入黄河。

因为贯穿了横山,无定河河谷便成勾连宋夏两国的一条重要通道。发生在无定河畔的战争,自古以来就没有停歇过。同时也是因为有着丰沛的水源,无定河河谷不仅仅浇灌了银夏之地的万顷农田,同时也是鄜延路的粮仓所在。

故而宋夏两国,都在这条河上建立了大大小小数以百十计的寨堡。仅有两三百步周长的寨堡姑且不论,光是八九百步乃至一千步以上的大城,在宋境,有绥德城、罗兀城;而在西夏国中,则是有银州、石州、夏州、宥州和洪州。

一旦夺占了整条无定河谷,便意味着官军控制了银夏之地,将西夏两大核心地区拿到了其中一半。在过瀚海直取兴灵之前,唯一还要费些力气去攻打的城池,就只剩青白盐池所在的盐州城——这是整个银夏地区,唯一一个不在无定河河谷中的大型据点。

盐州就在环庆路的正北,紧邻鄜延路西侧的环庆军只要攻下横山中位于青岗峡上的要隘蛤蟆寨,穿过横山后,便是盐州。

不过环庆路兵马副总管的高遵裕是不可能放弃灵州这个第一目标,环庆军只会偏向西北,攻打清远军城,一旦翻过横山,直通灵州的灵州川就在环庆军的眼前。

“沿着灵州川过瀚海,可比我们容易得多。要抢在环庆军前面攻下灵州,不是那么容易。”

“我们既然已经提前出兵,环庆军肯定忍不住。此时的灵州川水量正丰,春天雪化时候的河水,足以供应环庆军的饮用需要……我们就只有绿洲。”

“瀚海中的两处绿洲,得及早派人去夺下来,至少也得追在西贼的身后,不让他们有机会破坏水源。一旦投了粪尿进去,几年内绿洲就别想用了。”

“不到绝望的时候,西贼不会毁了绿洲里的水源。否则就算他们能逃脱一劫,十年内也过不了瀚海了。何况为了防着出意外,太尉还调了十名井匠随军,大不了用个两天开井,人手是不缺的。”

摆在种朴、种师中等几名年轻将校面前的,不是延州总管府的白虎节堂中的精细沙盘,仅仅是一张粗陋的地图而已。但围着这张地图,几名年轻的将才却议论得热火朝天。

鄜延路的提前出兵,不但让东京城中措手不及,同时也给了党项人当头一棒。遍及缘边五路的细作,让西夏高层把握到了宋军预定出兵的时机,却也因此被种谔阴了一着。

顺利进兵,使得鄜延路出征的每一位将校,现在都是很轻松的模样。

出兵半月,鄜延路的精锐跟随着种谔的将旗一路过关斩将,沿着无定河河谷杀奔过去。

在预先安插的内奸的帮助下,鄜延军只付出两百人的微小代价,便突破横山后的第一个关口,夺下了银州城。

攻破了银州之后,种谔并没有急着沿着无定河向西攻打石州、夏州,而是反攻向东北方的弥陀洞。那里是西夏左厢神勇军司的治所所在,驻扎了一万多西贼。不能将左厢神勇军司给打掉,想再向西,就会有被截断后路的危险。

而且左厢神勇军司就像一个楔子,钉在鄜延路和河东之间,堵住了勾连两路的北线道路,使得双方沟通不便,如果要传递消息,至少得向南绕行百里。想要河东和鄜延合兵一处,这根楔子就必须拔掉。

攻击弥陀洞是鄜延军出征后的第一场难关。得到银州陷落的消息之后,弥陀洞的守将立刻提高了防御等级。等到鄜延军抵达弥陀洞城下,出现在他们面前的是一座拥有近万名守军的坚固城垒。

不过宋军提前来袭,使得弥陀洞中守军的士气正是低落的时候。只要稍通军事就能知道,想要攻下这座城市,这是最好的时机。一旦攻城不克,顿兵城下,守军的士气就会回升。

为了用最快的速度攻下这座城池,种谔派出了从鄜延路五万多禁军中精挑细选出来的仅有一个指挥的选锋军作为攻城的尖刀,又亲自站到了城下击鼓助威。

一方士气正盛,一方士气低落,加上种谔早已准备好了长梯,又用数千张神臂弓为选锋军清洗城头守军。一座让河东、鄜延两路兴叹数十年的坚城竟然一鼓而破。当守将领军逃离时,送给鄜延军的斩首功已经多达两千,加上伤兵,纵然种谔无法分兵去追击,左厢神勇军司的这一支精锐骑兵,已经丧失了绝大多数的战斗能力。

稳定了后路,确保了和河东路的联系,接下来沿无定河向西的行动便是顺理成章。到了今天,种谔已经站在了石州城头,而为数八千人的前锋已经攻下了夏州城外的铁冶务,是一个拥兵数百的小据点,即是出产铁器的工坊,也是夏州的外围防线之一。

“石州城中挖出了多少粮草?”在石州城最好的一座府邸中,种谔向分司军中粮秣的侄儿问着。

种建中应声答道:“大约七千石,如果再能挖出几个粮窖,当能有八千石。足够全军十天食用。”

种谔放声大笑:“想不到还留下了这么多!这下攻下夏州城没有问题了!”

种建中点头:“绥德的粮秣应该很快就能上来了,正好来得及补上。”

因为种谔选择了先行攻打弥陀洞,给了石州守军足够多的时间。让他们能将城中人和资源全都运往更为坚固的夏州城。而对粮草的清理是重中之重,任何一名合格的将领,都不会给敌人留下一粒米,一根草。

只是坚壁清野的盘算打得再好,也得下面的人没有私心才行。

石州守将想要坚壁清野,不给宋人一粒粮食。可惜的是,石州城中有许多人一旦来不及将家里的存粮全都带走,绝不会烧掉,而是会设法将存放粮食的地窖给遮掩起来,赌一把运气。

而运气是站在宋军的一方。种谔手底下各式各样的人才都不缺,又是惯抢粮的。要从地窖中翻出来足够的粮食,在别人看来千难万难,可种谔随便找了几个人,在城里城外绕了两圈,便将石州藏起来的存粮起出来大半。

七八千石的存粮,足够全军吃上十天。

接下来虽说不一定有那么好的运气,攻打夏州城也不一定能有之前那么顺利,但只要统管粮秣转运的李稷能早点将粮秣运上来,足以让种谔攻下夏州城。

“为什么我不在乎出兵的时间被传出去?还不就是要瞒过党项人。夏天的瀚海根本不是大军能走过去的,党项人都难做到。就是知道官军要到近五月的时候才会出阵,兴庆府那边以为可以拖到秋天,所以才会不急着出战,也没有提前调集各地兵马。”

种谔很是自负的笑道:“调用得早了,粮草就会供应不上,党项人现在穷得很,仓底都快空了,养不起征发起来的大军。他们要想跟官军拼命,必须要卡准时机,只有在合适的时间点上召集全国兵马,才会不白白浪费仅有的粮草。”

种建中拱手恭维道:“五叔的谋算,可谓是看透了西贼的五脏六腑。”

种谔捻着胡须,很是得意。这一回,头功必然是他的。虽说冒犯了天子,但已经有了既成事实,难道还能让他退回去吗?

“中军就在石州休整一日,到了明天,兵进夏州城!只要打下了夏州城,银夏之地就是大宋的了。”

到了快入夜的时候,亲信进来禀报,说是绥德城派人来了。

“哦,绥德城那边终于来人了!快让他进来。”

种谔一下午都在跟部将们议论战情,正是有些倦了,但听说转运司驻扎的绥德城来了人,却立刻振奋起精神来。

很快,绥德城派来的人被领了进来,矮矮胖胖,一对小眼睛中透着精明。是李稷手下第一得力的亲信,也是粮官。

种谔一看到这个胖子,一贯威严的脸上就多了一分笑意:“李运使果然了得,种谔还以为要,押送了多少粮食来?”

胖粮官没有答话,他脸色有些难看,苦着脸从怀里掏出一封信来,“太尉,这是运使让小人连夜送来的……天使很快就要到了,运使想要太尉心里先有个底。”

“天使……”种谔的心中涌起了一阵强烈的不安,命种建中接过信,又问道:“什么事?”

“太尉,天子命你回军。”胖粮官坦言相告。

种谔脸上闪过一阵潮红,身子晃了一晃,差点从座椅上翻倒。

“五叔!”种建中连忙扶上去。

种谔却没理会,从侄儿手中抢过信笺,撕开来就看,越看手抖得越厉害,“这是乱命啊,这是乱命。”他最后抖着信叫道,声色俱厉。

