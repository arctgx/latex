\section{第34章 为慕升平拟休兵(23)}

萧十三和张孝杰匆匆赶上寨墙上的望楼。

千里镜中,极远处敌军军营的方向,宋军骑兵率先出营,地平线上一条扭曲的黑线,就如潮水一般,向大小王庄的方向涌来。

在骑兵的保护下,宋军的各个营寨都在向外出动兵马。号角和鼓声远近相闻,一队队高举旗帜沿着寨前的道路开始向大小王庄进发。

看到这阵势,完全可以确定宋军大举进攻在即。

萧十三左手紧紧地攥住了拳头,宋人的嚣张气焰让他心中犹如一团火在烧。右手随即拔剑出鞘,点兵出营。

宋人既然来攻,就给他们一个痛快好了!

“不对!”聚将鼓刚落,张孝杰身边的一个亲兵赶来拉住了准备出阵的萧十三,“宋人好像在扎营。”

萧十三此时刚刚将他的帅旗打起,听闻此事直接就从马上翻下来,三步并作两步,一下就抢上望楼,拿着千里镜望着宋军的方向。

离开营寨的几万宋军,就在出寨后两里多的地方开始列阵,中军本阵从官道一直延伸到两端的田地中。就是先一步出来的宋军骑兵,也没有跨过两边营垒的中间线,而是在本阵外围巡游护持。

而就在本阵之后,大约有上千人的样子,围出了一个很大的区域。在五六里的距离外,具体的细节依然看不太分明,可那么宋军聚在一起打算做什么,老于兵事的萧十三和张孝杰就算是模糊的一瞥,也是立刻就看明白了。

萧十三的脸顿时紫胀起来,血涌上头,“韩冈那个措大欺人太甚!”

宋军这是摆明了欺负辽军不敢硬拼、不善守御的弱点,就打算列寨以待,一步步将营垒修到大小王庄的鼻子底下,而不是将营地一鼓作气的攻下。

这样手法其实跟攻城差不多。寻常攻城,如果不能一鼓作气登城的话,就会在离城池稍远的地方设立一座主营,然后再于城池近处设立一座小营,也就是前进营地。这样才能保证攻城时不会在前往城下的过程中受到攻击,同时也能保证大军在休息时,不会被守军的出城偷袭所惊扰。除非是实力相差悬殊,否则不会直接将兵力放在城墙底下。

宋人这就是采取了这种手段,不断的压缩双方营垒之前的空间,然后逐个修筑前进营地。今天前进两里半修一个营寨。明天再往前两里半修一个。第三天就不用走了,出营就到了大小王庄的寨墙底下。

这对萧十三当然是不可接受,宋军要是在眼皮下扎营,那时就是想走都来不及了。可如果辽军要破坏,那么就要从正面去冲击宋军的军阵。

宋军看起来是完全有自信抵挡住他的攻势,并将阵后的营寨给建立起来。如果换在过去,宋人如此自大,萧十三肯定会嗤之以鼻。可宋人既然能修出一条百里长的轨道,一夜运兵数万。他们现在在战场上修筑营垒,必然是有其依仗,又怎么可能是自大之举?

萧十三为此咬牙切齿。

宋军纵然是在立营,可兵马就在近前。要是现在当面撤退,宋军的骑兵立刻就能追过来,绝不会让他顺顺当当的回代州。

进攻不能,撤退也不能,进退两难之下,萧十三一时之间不知该做什么的好。

他手上别看有几万人马,但能与他共患难的一个也没有。

如果宋军换个不那么靠谱的主帅,萧十三当然有信心能率领着一群各自异心的队伍获取胜利。可如今面对的是南朝中为数不多的几个擅长兵法的统帅,萧十三缺乏获胜的自信。即便能取胜,也得付出惨重的伤亡。

这样一来,他根本就没办法向耶律乙辛交代,伤了部族军,他肯定要负责,损伤了尚父派来的宫分军,压制下面的部族,就更难交代。至于拿着自己的根基去换取胜利……他萧十三还没疯!

就算此番败了,只要兵还在,只要他的一众族人还掌握着三千精兵,他就肯定能保住身家性命。可要是没了兵马,保不准就会给牺牲掉。

他扭头看了张孝杰一眼,可张孝杰却将千里镜压在眼睛上,只顾看着远方的敌阵。

麾下最为精锐的两千具装甲骑都已整装待发,可萧十三现在,已经没了出寨的念头了。

…………………………

萧十三当不会选择硬攻。

韩冈早在赶来河东的时候,他就确定了如今辽军最大的弱点。

如果坐在那张位置上的名正言顺的大辽天子,即便是辽宣宗耶律洪基那样的皇帝,也能使动下面的部族去为他出生入死。

可耶律乙辛不行!

他要维护自己的权威,就要不停地打击跳出来的反对者,然后收买和安抚隐藏着的反对者。

辽军中一个个山头就类似于军阀,当缺了大义名分的耶律乙辛掌控国家,现在剩下的就只是利益的计算了。

韩冈对这样的事见识多了。当年在对付西夏的时候,正好是梁氏秉国,多少次西夏军的战败,都是因为各个部族不能协同为国。

都不想为人火中取栗,最后就没一个肯卖命的,当然只会是失败。

营地修筑得很快,作为营栅的大木桩子钉下了,外围又挖了一片比马蹄略大的陷马坑,再埋进一片碗口粗细的短木桩,营垒最重要的一步就算是完成了。

他对京营禁军的战斗力没有太多的信心,但修筑营垒却不会差到哪里,这是宋军的基本功。而且

阵后立寨本也不为难事。

想当年三川口之战,宋军在延州城外只有五里的地方被元昊打了个伏击,主帅刘平还能领着残部到左近的山头上设寨立营,守足了一个晚上。难道现在的情形还会比三川口时还要差吗?

当营栅全数完工,而外侧的陷阱一一齐备,辽军的主力还没有出营,所有人都知道,辽军今天又败了一阵。

“辽贼败了!”

章楶松了口气,却又有着不可思议的感觉。

辽军已经怯弱到这般的程度了吗?

‘慢慢来,不要急。要稳扎稳打。辽贼本就拼不过我们。’

自从到了忻口寨后,一直都听韩冈这么说,虽说乍听起来十分被动,更耗时间,可如今看起来,韩冈的战略却正好卡在辽贼的脖子上。

“辽贼缺乏运粮的手段,现在又损了士气,不是今天夜里,就是明日清晨,萧十三是肯定要退了。”韩冈说道。

章楶随之一笑:“没地方打草谷,不退奈何?”

八百人的维护队伍,以及一百多熟练的车夫,就足以保证四万战士在前线的粮秣供给。而在大小王庄的辽军想要保证他们的口粮补给不缺,少说也要数倍的人马,这当然是很难做到的。

在没有后勤补给线的情况下,要想保证军队的日常食用,就必须分散开来搜寻粮草。但眼前的辽人肯定没办法。当辽军侵入毁掉了代州盆地中的每一座村庄,相当于辽军主动帮助大宋使用了坚壁清野的战术。除了代州和繁峙县之外,基本上就没有打草谷补充粮草的地方。他们在代州的种种兽行,现在看来等于是在自己脖子上拉绳索,只是死得不够快。

“辽贼会不会抄截我们的粮道。”章楶忽然问。

“有六千骑兵就足够了。他们要是敢冲进来,肯定就回不去了。”

骑兵的作用就是保护。保护中军能列阵,保护辎重兵能安稳运粮。韩冈手下的骑兵虽然从人数到战斗力都远不如辽军,可本职工作还是能完成得很好。

“早知今日胜得如此轻易,就不用讲白玉派去五台山了。”章楶笑叹道,“要是萧十三没有分兵出去,这里再多上一万马军,今天晚上辽贼就肯定要跑了。”

“白玉已经将忻州的盗匪清扫得差不多了,稳定后方的功劳不小。”

韩冈摇摇头,虽然这种误会是他故意没有澄清才维持到现在,但幕僚之中只有黄裳一人是没有怀疑他的想法。

他让白玉去清扫山贼,白玉就只能老老实实的去清扫山贼。要是胆敢自作主张,就是立了功,韩冈也能让白玉和他一门子弟一辈子都出不了头。

但章楶等人不清楚韩冈的禀性,以为他是故意对外这么宣传,以隐瞒真相。

翻越险山峻岭的军事行动不是不可能。国外的有汉尼拔越阿尔卑斯山,国内的有高仙芝穿西域葱岭,远的有邓艾走阴平小道灭蜀,近的就是王韶、高遵裕领军入露骨山追击木征。

地势是决定大军行止的重要因素。可比起抓住敌人的心理漏洞,给出决定性的一击,崎岖的山峦已经算不了什么了。

只是这一切成功的关键,是敌军不要防备。

可回想一下,辽人是怎么拿下雁门、打进了代州的,萧十三如何会不防备五台山?

当然,五台山北麓大小山口不在少数,为了提防随时可能到来的宋军,辽人就必须放置两倍到三倍的守军来防守五台一线,而且还不一定能挡得住如狼似虎的西军。

少了三分之一的兵马,萧十三肯定更不敢选择即刻开战。这为韩冈直趋代州,省下了多少事。

这一回是在大小王庄,下一回,可就是在代州了。

那时候,萧十三还能退吗?
