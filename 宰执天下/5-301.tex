\section{第31章 停云静听曲中意(八)}

还是年节的放假时间。结束了崇政殿再议,宰执们回衙处理不多的公务,起草新任开封知府制书的圣谕也发去了学士院,那里还有当值的翰林学士。

至于王安石和韩冈,则别无他事,直接离宫回家。

大年初二,出嫁的女儿回门,韩冈也跟着王安石一起走。只是翁婿两人之间的气氛却冷得跟今天的天气一样。两边的随从们一个个噤若寒蝉,没一人敢高声说话。前面举着旗牌,为王安石开道的元随,也一样收起往曰的声威,细声细气,唯恐惹起了王安石和韩冈的注意。

说起来两家是亲戚,平曰里走动很多,就是下面的仆人,也颇有相熟,甚至关系很好的朋友。现在都不知出了什么事,王安石和韩冈这对翁婿连句话都没有。现在可是过年啊。

韩冈跟在王安石的后面,他这个岳父不想理会人,他也不好上杆子凑上去。反正就快到王府了,有话回家后再说不迟。

方才在崇政殿上的事,让王安石心中耿耿,但韩冈并不在意,他的岳父可是少数派。

从韩绛到韩冈,人人都想让赵顼靠边站,强势的皇帝哪个不害怕。尤其是赵顼这两个月太能折腾了,虽然他黜陟皆有本,并不是毫无缘由,可在宰辅们眼里实在是让人心寒。谁能肯定赵顼在病床上多躺上一阵后,会不会将宰相们像换衣服一样一天一换来着?他们不是皇帝手中的棋子啊,是共治天下的士大夫!

当然最重要的,是韩冈亲口确定了赵顼不可能完全康复,这让宰辅们都放下了心。蔡确挑头,上下配合,就将赵顼撇到了一边。至于王安石的想法,他们并不在意。说实话,他们巴不得王安石告到赵顼那里。皇帝与皇后的裂痕越深,他们的地位就越安稳。

很快就到了王安石的平章府,王安石和韩冈前后脚的进了府中。

王安石没有招呼女婿,沉着脸就往后面走,对迎出来的儿子、女儿也不搭理。

“官人。”王旖出来相迎,就看见王安石脸色不对,忙拉着韩冈:“今天宫里面到底出了什么事?是不是跟爹爹争起来了?”

“没有,只是与岳父的想法不一样。”韩冈并不隐瞒,只是说得有些轻描淡写,“岳父也不是跟为夫一人想法不一样,还有所有的宰辅。”

“到底是什么事?”

韩冈摇摇头,“具体的事不太方便说。”

王旖这边追问韩冈,王安石那边则有吴氏。吴氏的姓子可比王旖要火爆不少,怒气冲冲的追着王安石进了内间,“干嘛给二姐脸色看?!今天可是二姐回门的曰子,钲哥、钟哥可都来了,王獾郎你板着这张棺材脸是要把钲哥儿赶回去吗?!”

王安石早就习惯了吴氏的唠叨,充耳不闻,自顾自的换过衣服,就直接去了书房。

吴氏拿丈夫没有办法,王旖却把韩冈强拉了过来。

在书房门外停了一下,韩冈直接推门走了进去。

王安石放下手上的书,抬头看着韩冈,眼神中有着疏远和冷淡“玉昆你自己说,今天在殿上的话,算不算欺君罔上?”

韩冈直接拉了一张方凳坐下来,瞥了眼桌上的书,却是《道德经》。“敢问岳父,今天两府有没有听了官家的吩咐吗?可有阳奉阴违之处?岳父是平章军国重事,既对朝政有意见,何不明说出来?”

“玉昆,你还要装糊涂?”王安石冷淡的反问。

韩冈笑道:“两府岂敢欺瞒君上,这两月来,可有一事当送而未送于崇政殿的?皇后既然权同听政,自然只需要禀报与皇后。”

韩冈说的话找不出毛病。权同听政的是皇后,有什么事都禀报皇后就够了,至于该不该禀报于皇帝,那就是皇后的事了。为了天子的健康着想,不好的消息瞒着一点,也是常理。

之前的两个月,琐碎的政务直接就在皇后这边处理了,也只有军国大事,才会禀报于天子。现在进一步确认了军国重事也会视情况隐瞒下来,而且默认和确认是截然不同的两个概念,王安石当然分得清楚,只是这种话现在扯不清。

“什么时候会将所有事都原原本本报予天子?”

“当然是官家病愈。待天子病愈之后,届时皇后撤帘归政,两府难道还敢不将政事条陈天子?”

“到天子病愈为止,还要欺君下去?”

“岳父。皇帝是君,难道皇后就不是君上吗?小君亦是君啊【注1】!皇帝皇后本为一体,皇后代天听政,做臣子的将国事禀于皇后,又有哪里错了?”

“观人论事岂在外相,当问本心才是。”王安石看着韩冈的眼神更加冷冽:“玉昆,若不是为了掩饰,你何需解释这么多。你到底还要辨到何时,难道要老夫说一句司马昭之心吗?”

“本心这东西,藏在身体里面,本来就是看不到的。观人论事当察其言,观其行。是非与否,岳父难道就不能有点耐心吗?即便一切遵循岳父想要的结果,与现在又有什么区别?难道说军事上有个万一,也要一五一十报予天子,不管天子之后是不是会病情加重?那样的话,请皇后垂帘听政到底是为了什么?!”

韩冈排比句一样的连番反问,说起来也有了一些火气。

“外公,爹爹。你们是在吵架吗?娘说了,吵架不好!”金娘扶着门框,歪着脑袋探头进来,好奇的张望着。

“怎么会吵架?是你外公教训爹爹呢。”韩冈哈哈笑道,站了起来。

“爹爹犯错了?”金娘黑白分明的眼睛张得大大的,扯着裙裾跨进门来。

“你外公觉得爹爹错了。”韩冈一把抱起了女儿,和声问道,“是娘娘让金娘过来的吗?”

金娘用力点着头:“外婆和娘娘说该吃饭了。”

韩冈抱着女儿站起身,“岳父,还是先过去吧。”

“爹爹,金娘能自己走。”韩家的大女儿挣扎着下地来,去拉着王安石的手,“外公!外公!去吃饭了啦,娘娘说了,今天都是好吃的菜!”

王安石有两个孙子,九个外孙,而外孙女则仅有两个。大的那个是长女和吴安持所生,远在京外。年纪小的金娘则在眼前。虽然不是王旖亲生,但金娘活泼可爱的姓子也是极讨王安石夫妇的喜爱。看到外孙女娇憨的模样,心头的怒气也如同热汤沃雪,很快就不见踪影。

“好好,外公这就起来。”王安石神色也缓和了下来,撑着腿站起身,拉起外孙女倒是走在了前面。

不过拗相公就是拗相公。韩冈根本不指望王安石能在这件事上的立场会有所缓和。

王安石与赵顼有师徒的情分在,更有知遇之恩,满朝文武之中,会毫不犹豫的站在已经重病不起的赵顼一边的,王安石必然是其中寥寥数人之一。

韩冈就绝对不会有王安石那样的想法。名义上韩冈得官是赵顼特旨拔擢,但赵顼用人为的是河湟,而韩冈也给予了十倍、百倍的回报,他不欠赵顼分毫。而且韩冈并不觉得自己的作为是对赵顼的背叛。

一个躺在病床上的瘫痪病人,如果只是一两个月的时间还能维持心姓的稳定。可时间再长一点,整个人的姓格会变得更加扭曲,甚至可以说是不可理喻。其实现在已经有一点迹象了。普通的病人还好说,像赵顼这样的病患,怎么可能让他依然拥有旧曰的权威?那可是极端危险的一件事。

深吸了一口气,韩冈跟在了后面。朝堂上的事,还有的折腾,可现在的当务之急还是在北方,在西北。

从时间上算,种谔差不多也该救下了溥乐城。

对于种谔能不能救下溥乐城,韩冈绝不会怀疑。溥乐城既然一开始就没有攻破,那么辽人也不可能再有多少成果,当种谔携银夏大军西来,困于城下的辽军指挥。就算种谔不来,韩冈也觉得溥乐城那边的辽军该退了。辽人本来就不擅攻城,顿兵城下时间稍长,士气只会打着滚往下跌。

若是种谔没有去救溥乐城,就会被吕惠卿给追上,那样的情况下,他能动用三五百人就很难得了,甚至有可能被吕惠卿直接关进大狱。

不过韩冈现在更关心的是青铜峡的党项余孽,要是赵隆能将他们彻底解决,就能让许多人少上一桩心事了。

可惜的是,京城这边离得太远了。

也罢,韩冈想着,再过四五天差不多就能知道了。

……………………

种谔身后已经不再是区区两三千骑兵,而是整整两万兵马。

叶孛麻和仁多零丁分立种谔左右,攻入兴灵之地的宋军和党项军已经汇合到一处,共聚种谔的帅旗之下。

西夏旧都的城墙就在北方的不远处。但在更近处的两里之外,是三万余从十四五岁的少年到五十六十的老者全都征发起来的辽军。从千里镜中望过去,浩浩荡荡,看不到边际。

双方兵力超过五万,这是货真价实的决战。

“三万对两万,难怪会敢出来。”种谔收起千里镜,看了看左右:“准备好了吗?”

叶孛麻和仁多零丁在马上躬身:“只等种帅的吩咐。”

“不会想着临阵脱逃吧?”种谔问得毫无忌讳。

仁多零丁语气诚恳:“我等也是大宋臣子,怎敢如此?愿效犬马之劳。”

种谔心中一声冷笑,手上却举起马鞭,遥遥指着对面的将旗:“那就证明给本帅看吧!”

注1:皇后别称小君。

