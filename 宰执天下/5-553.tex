\section{第44章 秀色须待十年培(九)}

郭逵到了快黄昏的时候都没有回去的打算。

一个下午都耗在了军器监火器局这边,看着两门火炮轮流试射,甚至还兴起起来,上去亲自点火试射。

普通的铁质炮弹,铁砂、铅子,还有韩冈特地命人用纸壳做了外包装的霰弹——也就是用纸筒里面塞满了铅子,都试了个遍。

郭逵尤其对定装的火药和霰弹赞赏不已。

打仗之前,有经验的士兵都清楚,刀剑盔甲要及时上油,弓弩也要保养,箭矢若有空也会用小刀修一下。多做一份准备,战时就多一分安全。

相反地,那些经验不足的士兵,上阵后就会手忙脚乱,拿着弓箭,手抖着上不了弓弦的士兵,每次上阵,总能一抓一把。如果火药、铅子不是定装的话,装填手们小手一抖,不知会洒多少到外面,也不知会有多少掉到里面。到时候,弄不清装药量,也弄不清装弹量,原本就是生手的士兵,只会变得更慌。

而定装的霰弹、火药就不会出现这样的情况。就算一时脱手,没坏就捡起来,坏了就换新的,根本不用操心太多。

郭逵打了那么多年的仗,却还是不知道士兵们的下限到底在哪里,在河北,京中,他见多了各色奇葩。与其寄希望于他们能临阵超常发挥,还不如将他们都当成白痴比较好。这样有些好表现,也能算是惊喜了。

在郭逵的要求下,霰弹接连射了好几次,挡在炮口前面的靶子,不说活羊,牛都牵了一头过来,一炮打成了筛子,半边的皮都没法儿用了。

看着一炮把上千斤的公牛都打得倒在地上吐血沫子,郭逵兴奋无比,以牛皮的防御力,完全可以跟普通的马铠相媲美了,穿着铁甲挨一下,也要看运气。

‘可惜了。’韩冈看着地上的公牛,为那几百斤的牛肉惋惜不已。如果是铁弹的话,还能废物利用,现在可就不行了。一副皮子坏了倒是没什么。

这些年,由于铁甲普及,牛皮甲的使用率一路下跌,反倒不及猪皮羊皮更受军中重视,做甲胄内衬,便宜一点的猪皮、羊皮都能凑合。多出来的牛皮,多是拿去做靴子了。等到曰后火炮推广,制作弓弩的牛角、牛筋也肯定会用得少了。

只是像今天这么浪费,也还是不多见。小小的院中,血肉横飞,血水淙淙的流成了小溪,如果不知情的人过来,还会以为军器监什么时候改行做屠场了。

韩冈对郭逵喜欢霰弹并不惊讶,大宋从不缺乏远程的射击武器,但缺乏近距离一击致命的,就是斩马刀,也要胆大敢战的勇士才能面对敌军斩杀出去。而且当辽军在阵前横掠而过,为了不破坏阵型,拿着斩马刀也不能随意离开队列,光凭神臂弓或是破甲弩的射击,在辽军都装备了铁甲马铠的时候,不足以带来更大伤害。相形之下,一记时机恰到好处的霰弹,足以打乱辽军的冲击,然后就是步兵们的事了。

一声声的轰鸣,在军器监中几乎就没有停过,监中的官吏、工匠和士兵,都忍不住停下了手上的工作,去打听到底是出了什么事。

又是一盆井水泼上去,热得烫手的炮管顿时滋滋冒出水汽来。

臧樟转过来冲韩冈等人摇摇头,这门火炮暂时是不能发射了,得换另一门来用。

郭逵却还是点着头。

就算如此,从耐用姓上看,也照样比八牛弩要好很多。角筋之物制作的弩弓,尤其是力道极强的床子弩,使用寿命也就是几十次,连续使用太多,更是容易损坏。而火炮,只看到不停的泼水冷却,然后就顺顺当当的一发接着一发。一个下午,将韩冈储备五十多枚的定装火药都消耗一空。

郭逵意犹未尽,看多了各色兵器,他很难得有这样的心情了。看见火炮,就不由得联想到未来燕山脚下的金戈铁马,成千上万的契丹骑兵,被一蓬蓬铅云撕成粉碎。

原本他觉得最近达成的和约也差不多了,想要灭掉辽国,至少得在十年之后,那时候,他也上不了阵了。可现在看到火炮,已经沉寂下来的心情又开始躁动。

以朝廷打造铁甲的速度,给北方堪战的三十万禁军装备上火炮,就是那种最简单最轻便的步人炮,只要一两年,加上青铜的野战炮,一两千门的话,三年就差不多了。

如果朝廷要攻辽,以他郭逵的能力和资格,至少能当上一路主帅才对。

但他的两个儿子却找了过来,打断了郭逵的遐想。还把刘惟简给带过来了。说是太上皇后有召,招郭逵入宫觐见。

郭逵刚住进去,房子就塌了。这当然不是件小事,那是新进雄武军节度使,依然是签书枢密院事的郭逵的新赐宅邸,这么一下子就塌了,那还了得?早早的就有人报上去了。

韩冈和章惇对这件事,本来准备报个意外就了事,大事化小,小事化了得了。又没有打到皇城,也不是故意要往郭逵家炮击。现在跟郭逵又说和了,苦主不闹,那还有什么官司好打?

当然,如果郭逵觉得心有不甘,韩冈和章惇可以配合他一下,向开封府讨个公道。到底是怎么修的房子,十几斤重的东西掉下来,就把房子给砸塌了。偷工减料没见过有这么肆无忌惮的。

韩冈不是说笑,若郭逵真有这方面的需求,他愿意帮郭逵一个忙。当初他修成方城轨道,因种痘法回京后,被赐宅邸。开封府不说帮忙修一下,连物料、人工都要自己掏钱,这像什么话?早就在整顿一下了。

谁成想还是给捅上去了。

今天的事,对郭逵固然是无妄之灾。但对韩冈和章惇来说,也是运气不好。

章惇叹了口气,韩冈摇摇头,既然宫里面遣人来召见,也不得能当做没这回事。这下子肯定是要上殿请罪,逃不过去了。

太上皇后召见郭逵,肯定并不清楚郭府正堂坍塌的真相,只是准备安抚一下这位劳苦功高的名帅,免得外面说朝廷苛待功臣。他们这两个罪魁祸首,不能让太上皇后无辜的道歉。

郭逵虽然对火炮还有些依依不舍,但君命难违,只能先起身往宫中去。

章惇、韩冈也紧随其后,跟着他一起,同往皇城那边过去。

向皇后对于韩冈、章惇与郭逵一起进来很是吃惊,不过出去传诏的刘惟简已经先了解到了一些内情,上去报给了向皇后。

“火器局也是,怎么就这么不小心。不知太尉家中,有人受伤没有?”

“托太上皇后福,家人都安然无恙。只是房屋小损。”

“没有就好,没有伤到人那就最好。”向皇后点着头,又道,“吾方才已经责令开封府帮太尉重修正堂。”

开封府!

郭逵迟疑了一下才上前。韩冈和章惇面面相觑,郭逵真真是走了霉运了。

向皇后当然不知道开封府负责管理官宅的那一帮子有多不靠谱,见郭逵谢恩,也就不再多想。问郭逵道,“听说太尉是从军器监那边过来,可是看到了火炮?”

“看到了。的确是军国利器!隔了五百步也能击垮臣家正堂的大梁,霹雳砲、八牛弩都比不上!”郭逵毫不掩饰自己的赞赏,“方才臣在军器监中也看过试射,铁甲就跟纸一样被打穿,千斤公牛也是一炮毙命,的确是前所未有的军国利器!”

“当真?!”向皇后惊喜的对韩冈道:“这才几天,宣徽又立功劳了。”

韩冈道:“乃是火器局上下用命,臣不敢居功。”

“火器局成立不久,便造出了火炮,当重赏才是!宣徽明曰将有功人等具名奏来,吾必不吝封赏。”

韩冈向向皇后行了一礼:“臣代火器局中上下谢殿下厚恩。”

“如今只是火炮主体的炮管实验成功,安装炮管的炮车,还有配合发射的火药,都还到收获的时候。正式装备军中的正品,其实还远远未能完成。现在表现出来的威力,远未到预计中的水平。待到一切圆满完成,殿下再赏臣也不迟。”

“那还要多久?”

“至少还要两三个月。”

皇后有些失望,“这样啊。”

韩冈见状,补充道,“等到火炮正式定型,就可像板甲一样快速制造。只要材料能跟得上,就不会有任何耽搁。”

“那吾就等着宣徽的好消息了。”

新式兵器第一次试射,就将郭逵的新宅给砸了,闹到韩冈和章惇齐上殿请罪,可以说是让朝野震惊的大笑话。

但一炮轰掉了郭逵家的房子的确很是好笑,可这个威力让东京军民再一次了解到了韩冈说话算话的特点。

这还是刚刚造出来的未完工的产品。等到经过多方改进,火炮的威力,将会变得更大,而且肯定会大上许多——韩冈在御前的解释,也流传到外面。

又有了一件能够镇压敌国的利器,这对东京军民来说,是个可喜可贺的大好消息。

两家快报飞快的在报上公布了这个消息,并且将火炮的威力吹嘘得神乎其神。一时人心激荡,郭逵家也重新宾客盈门,都想看看火炮的功绩,差点踩坏了郭家的门槛。

辽国的使者即将抵京,名义上是恭贺新天子登基、再叙两国友好的国信使,但实际上,就是外面的百姓们都知道,这是挟着三旬灭高丽的声势来做示威的。

高丽好歹是向大宋称臣的藩国,尤其是在前些年,因为将高丽从辽国那里挖过来,作为万邦来朝的功绩,朝廷可是好生的一番宣扬。

京城之中,没几个百姓会知道真腊、三佛齐这样的南方小邦,尽管他们隔三差五就派使者过来,向朝廷哭诉交州羁縻的蛮部在他们国中掳掠子女云云。但没人有会没听说过高丽国的存在。

现在朝廷才派了人去援救高丽,就传来给辽国侵吞的消息,哪个心里痛快了?那些酒桌上的宰辅们一个个都叫嚣着要膺惩暴辽。又喊着要给辽国使者一点颜色看看。

但谁都知道,这些话只是醉鬼的胡言乱语,真实的情况还是让人堵得慌。

幸好有了火炮,看看辽国的使者如何还能耀武扬威!

