\section{第40章 帝乡尘云迷(一)}

王安石已经离去,而韩绛尚未抵京。

东府中书门下,便以新就任的次相集贤院大学士冯京为首。

朝会之后,宰辅们回到政事堂中,共议今日要处置几项重要的政事。

“‘交趾蠢蠢欲动,似有所图’。桂州沈起的这份奏章,两位都看过了吧?”冯京高坐于中厅正位,将从广南西路首府桂州【今桂林】的知州发来的奏章,当先拿在了手中,“这沈起,妄图开边衅、谋私利、邀功图赏,此辈败坏国事,使天子难以安寝。不知两位参政有何看法?”

王珪先啜了一口药汤,漫不经意的道:“将他调离便是。”

这些天来,王珪看着神色没有什么异样,但话语不多,明显的心情不好。他是老资格的翰林学士,升了参政也有四年了,本以为拜韩绛为相之后,天子会过上一段时间再任命第二名宰相。可没想到天子的动作那么快,还没等自己发力,就已经为冯京锁院宣麻了。他进入政事堂只比冯京迟了三个月,没想到区区三个月的时差,竟然让天子都不加考虑自己的资格。

冯京也知道王珪是怎么回事,瞟了他一眼,就转到吕惠卿的身上:“吉甫,如今朝廷正忧于北事,无暇南顾。禹玉也说了,沈起还是调离为上,不知你意下如何?”

“相公所言甚是。不过交趾那边不能不防。不如换一个稳重有韬略的去替他。也防着万一有事,广西措手不及。”

吕惠卿没反对,只是多提了一句自己的意见。沈起不是他的人,也与新党瓜葛不深,没必要护着他。

更何况吕惠卿现在也不想多事。他晋升过速,熙宁五年回来时才一个品阶最低的正八品朝官,仅仅两年时间就进了政事堂。虽然吕惠卿一直都很确信,凭着自己的才干,迟早能问鼎相位。不过这两年的际遇,也的确出乎他的意料。

也多亏了曾布,要不是他忽然之间闹出了那一场,在背后捅了王安石一刀,现在进入政事堂的本来应该是他才对。只可惜曾布其人胆略和能力都不缺,就是缺乏看人的眼光,和分析时机局面的判断力,如今落到江南西路一知州,也是他自找的。

吕惠卿明白他现在要做的是扎好根基,将新党牢牢控制在手中,培植出自己的势力,如此才会有钧衡朝堂的可能。

至于冯京,吕惠卿根本不放在眼里。他的存在,只是天子要在政事堂中留下一个不同的声音罢了。王安石是熙宁三年年底方才正式成为宰相,可之前做参知政事时,就已经把持了朝政。熙宁初年的政事堂中两相三参,曾公亮老迈、富弼称病、唐介暴卒、赵抃叫苦,只有王安石生气勃勃,这生老病死苦的笑话至今也有流传。就算没有韩绛,等自己用上一两个月时间,将新党重新整合起来。国家大事,冯京也就只有说说话的机会。

可冯京眼神冷冽,吕惠卿明着是在附和自己,但他的提议,其实等于是承认了沈起奏疏的真实性:“如今南平郡王不过七八岁,去年才刚刚登基。主少国疑,安定国中尚且不及,岂有北犯之理?”

交趾国一直以来都向大宋称臣,上百年来,国主从丁姓变为黎姓,又从黎姓变成李姓,但作为大宋臣属的从来没有改变过。交趾国王登基后,都要遣使东京,上表称臣。而朝廷给他们封爵则都是南平郡王、静海军节度使。去年交趾国王李日尊病死,朝廷追封他为南平王,李日尊的儿子李乾德不过六岁而已,如今是交趾王太后在垂帘听政。

他再冷冷的看了一眼吕惠卿一眼:“沈起在桂州一番兴作,擅令疆吏入溪洞,点集土丁为保伍,授以阵图,使岁时肄习。继命指使因督餫盐之海滨,集舟师寓教水战。广西走马报上来的这一些,枢密院、政事堂何时下过命令?现在忽然上表,明着是在欺瞒朝廷,以逞私欲,哪有半分实话?吉甫你太多虑了。要找人替他,也要找个能安心理民的,将沈起所兴诸事一概废弃,以释交人之疑。否则交趾人哭到大庆殿上,岂不是要让契丹、西夏看笑话?!”

吕惠卿反驳道:“辽之承天,不也曾领军南犯?还有西夏,女主当政之时,寇边的次数也不减少。”

逼着真宗皇帝签下澶渊之盟的辽国皇太后萧燕燕,当年就是亲自领军。而熙宁初年,不断南犯的西夏,控制朝政的也是太后。

冯京则哈哈笑了两声:“交趾蕞尔小国,如何比得上西北二虏?吉甫你想的也太多了。”

吕惠卿皱起眉,正要再反驳回去,王珪则插言道:“刘彝此人如何。他在虔州【赣州】做的不错,正好也已经任满。”

冯京依稀听过这一个名字,只是一下子想不起来。他是管理大宋亿万兆民的宰相,普通的州官很难在心中留下什么印象。疑问的视线投向王珪,王珪则很配合的说道:“刘彝曾为制置三司条例司官属,后因言称新法不便而被罢去。不过他精擅水利,曾任都水丞,后又在虔州兴沟渠,制水患,惠民甚多。有他去桂州,当可无虑。”

听到王珪之言,冯京嘴角向后拉出了微不可察的弧度。得到提醒,他也记起了刘彝这个人物。比他心中的人选还要好。转头又瞧着吕惠卿:“吉甫,你意下如何?”

吕惠卿并没有不同的意见。并不是他畏惧冯京、王珪两人合力,而是他乐见刘彝去桂州。

制置三司条例司是最早设立的新法制定机构,不论是青苗法、还是均输法,都是来自于其中。如今虽已经被撤销,但司农寺已经全盘接手条例司的工作。当时侧身其间的官员,有成为新党中坚的吕惠卿、曾布、章惇,也有后来转头旧党的苏辙、程颢、刘彝,而他吕惠卿,当初跟刘彝可没少争执过。

桂州在哪里?

岭南!

桂州的位置的确重要,是南方重镇,冯京和王珪都希望有个新党的反对者坐上去。但吕惠卿不在乎,反正他手上没人能争这个位子,而诋毁新法的都去了岭南,他才高兴呢……为什么要反对?

从岭南任官一趟回来,依例会加上一官,或是多减几年磨勘,这是太宗时就制定的规矩,至今未变。王珪可能看上了这一点,不过就此病死岭南的也不是没有,否则太宗何必定下这项奖励。

“就依相公、参政之言,让刘彝去桂州替沈起回来。”

确定广南西路的主帅人选,毕竟是小事。冯京第一个将其抽出来,只是因为这一桩公案,没有多少争执的余地。以此事开头,成功的压制吕惠卿,便可顺势而下,将接下来的几桩公事一气呵成的按照自己的心意处置下来。

冯京也是心急,天子的心意,全东京城都明白,他冯京当然也同样清楚。不趁韩绛抵京前的这段时间,稳固了在相位上的发言权,等首相抵京之后,哪里还有自己说话的地方。

好不容易升任了宰相,冯京怎肯甘愿做壁上观?

他是当朝宰相,不是给人做陪衬的饰物!天子需要政事堂中有一个反对的声音,但他冯当世绝不会甘心只做着一个反对者。

……………………

河阳孟州【今巩县】,离着京城并不遥远,马递只有两日的行程。

不过孟州在黄河北岸——山南为阳,山北为阴。水南为阴,水北为阳——所以河阴在黄河南岸,而河阳则在北岸。

此时正是黄河上冻的时节,河面上的冰层已经能挤碎渡船的船底、船帮,只是还不到让车马在冰面上通行的厚度。

来送诏书的使臣前两天拼了命的过了河,来到孟州州衙时,脸色都是白的。但韩绛不能拼命,更不愿拼命,只能在黄河北岸,等着什么时候天气突寒,将大河冻上,那时才能顺利渡河。

不过即便韩绛还没有回到京城,但他已经是宰相了,而且是首相。

韩绛过去曾经坐过一任首相。不过那是王安石让给他的,而且也是为了能名正言顺的指挥攻略横山的大军,统率河东、陕西二路兵马。

但那一次,他在相位上只坐了短短几个月,就因为轻弃罗兀城,而不得不黯然告退。

此事非战之罪,而是天子意志不坚,加上庆州兵变的缘故。但韩绛也明白,其实他也有机会的,将天子的诏令顶住,将西夏人给拖垮。这几年来一直都在后悔,如果当初他坚持下来,也许西夏现在就亡了。

不过世事无常,绕了个圈子,现在又绕了回来。时隔三载,他现在又是宰相了。

从天子公布他和冯京的任命时间上,韩绛清楚,皇城中的那一位仍然还在维护新法。

一直以来,他韩子华都是新法的支持者,从来没有变过。自己能接手王安石留下来的职位,天子肯定是考虑到了这一点。

在房中一声轻叹,韩绛闭上眼睛假寐起来,现在就等着黄河上冻,好回到阔别已久的东京城。

