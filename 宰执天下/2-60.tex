\section{第13章 不由愚公山亦去(五)}

【不知道这一更该算是昨天的份,还是今天的份。春节只剩最后一天,继续征集红票和收藏】

韩冈跟在王韶、高遵裕疾步走进州衙大堂。

无论是州衙大堂,还是县衙大堂,除非节庆大典,或是中使持圣旨驾临,否则都是将正门紧闭,只开两侧的旁门供人同行。东侧旁门号为生门,寻常人等皆由此进出,而西侧号为死门,只有待决死囚才从此门拖走。

今日来得是宣诏使臣,秦州州衙大堂正门自然中开。炎炎夏日炽热的阳光从敞开的大门处照了进来,一名头戴软脚幞头,身着绯罗袍的宦官就站在大堂正中央,在他旁边是一名小黄门用朱漆托盘托着明黄绸缎盖起的几卷圣旨。

而在大堂门外的围观者中,韩冈惊讶的发现了穿着官服的王厚和赵隆的身影。视线对上,他们两人便微笑着不出声的打了个招呼。

高遵裕明显认识今次来宣诏的天使,他进堂后,就上前拱手行礼:“原来是王都知。”

王中正慌忙回礼,脸上堆起的笑容甚至带着谄媚,“高提举今次为朝廷立了大功,听到古渭大捷的消息,连天子都惊呆了。直说高提举和王机宜办事得力。”

高遵裕笑着与王中正一通寒暄,宣诏使臣在天子舅公面前,也不得不卑躬屈膝。不同于士大夫可以不把高遵裕的外戚身份放在眼里,甚至还可以时不时的还能拿着这个身份敲打一下高遵裕,在宫中做事的宦官,对太后的叔叔是畏之如虎。

韩冈随着王韶上前跟王中正见了礼,从这个阉宦的嘴里得到了‘年少有为’的四字评价。他随口谢过,与王韶、高遵裕一起等着王中正宣诏。

王中正却还在等人,可并不是韩冈预料中的李师中。秦州知州现在正在二堂那边继续审讯,虽然可以肯定他必然得到了消息,但既然王中正没有通知他,李师中也不会放下案件,自己贸然走出来。等王中正宣诏完毕,他才会出来迎接,为王都知洗尘。现在替代李师中出现的,是窦舜卿和向宝两人。

向宝跟王韶、韩冈之间仇深似海,到现在他中风的后遗症依然存在。他步履维艰的走进大堂,正眼也不瞧王韶和韩冈,走过去跟王中正不冷不热的行了礼,便沉默的站到了一旁。原本是意气风发的军中少壮派的领衔人物,现在已经是暮气沉沉。只有在视线掠过王韶和韩冈时,才会在眼底出现一闪而逝的杀机。

韩冈看了看形容憔悴的向宝,中过风的他在官场上已经是死老虎一只,就算对自己恨之入骨,他也是什么都做不了了。

收回视线,却又瞥见大堂外的王厚,用手正指着向宝,嘴唇无声的念着,看上去像是在念着张守约三个字。韩冈会意的轻轻点头。果然是张守约顶替了向宝,看来今次向钤辖调离秦州的消息已是板上钉钉了。

在向宝进来后不久,窦舜卿也走进了大厅。老迈的都副总管容色同样有些憔悴,而看向韩冈这边时,眼中的杀意也是不禁流露出来。虽然韩冈并没有留下什么破绽,但并不影响窦舜卿怀疑到王韶和韩冈头上。

窦舜卿带着恨意的眼神,韩冈若无所觉,眉头挤出的纹路也不是因为已是焦头烂额的窦副总管,而是为了李师中。

秦州知州没有被宣诏使臣请出来,而是请了窦舜卿,这让韩冈大惑不解。天子和王安石不可能不调走李师中。王李两家打的笔墨官司在崇政殿的案头能叠起两尺高,几乎是水火不容。李师中在秦州一日,王韶的手脚就要被枷上一日。有两场大捷为王韶的才能作证,赵顼怎么还会留着李师中在秦州做河湟拓边的绊脚石?

今次张守约诣阙回来直接顶替向宝,是韩冈意料中事。在他的预计中,窦舜卿应该会被留任做个过渡,而李师中则是肯定要先被调出秦州——这也是王韶和高遵裕共有的看法。而且在官场上资历比王韶、高遵裕和韩冈加起来都多,两场大捷会给秦州官场带来什么样的影响,想必李师中自己都清楚。

韩冈这些日子费尽心力的设计将窦解弄进大狱受审,就是想着先下手为强,不然窦舜卿顺顺利利的接替李师中当上了秦州知州,即便是个过渡,他韩冈也少不了被扒层皮。

韩冈头痛着,而王中正已经开始宣读诏书,第一份诏书的内容就解释他的疑惑。

宣诏的顺序由官阶高低决定。等他请来的官员都到齐,王中正回头掀开漆盘上的明黄绸缎,取下摆在最上面的一卷诏书,“窦舜卿听诏。”

窦舜卿上前跪倒。

王中正用着尖细的嗓音念着诏书。这份诏书中并没有提到半点窦舜卿将万顷荒地说成一顷的欺君之言,而是赞许了他在秦州的苦劳,并让他回京城诣阙。

‘果然还是要调走李师中。’韩冈听着听着,便恍然大悟。

边地要郡守臣在上任前,一般来说都要面圣陛见,述说自己对即将担任的职位的看法,以及上任后要施行何种。窦舜卿被召去京中,便是为了接替李师中而做准备。

但现在可不是一般情况,离秋季只剩两个月了,届时关西缘边各路就会迎来一年中规模最大的西贼攻势。防秋的一桩桩繁琐的事务如今已经要开始进行准备,在韩冈王韶他们的预想中,将是窦舜卿直接替代李师中,以防耽搁了防秋。可没想到,天子还要让窦舜卿去京中走个过场。

“还真是稳重……”王韶压低了声音说了一句,听口气却是在抱怨。

朝廷的这种稳重之举不仅让王韶抱怨,也让韩冈觉得不痛快。如今他的孙子犯了事,窦舜卿少不了干系。他入京诣阙的同时。窦解的罪行也会递到天子案头。他也不可能再接任秦州知州一职,甚至不可能留在秦州。既然向宝走了,窦舜卿也走了,为了秦州内部的稳定,有极大的机率到最后是李师中被留任下来。

这算是弄巧成拙吧?看着侧前方王韶变冷的表情,韩冈能猜出他的想法。

‘算了,还是有办法的。’见过了李师中最近的表现,韩冈却还是有些把握。

紧接着窦舜卿,接旨的是向宝。一番抚慰之词之后,向宝被免去了他的都钤辖之职,调入京中。因为阻挠河湟开边之事,他本是要被降罪,但一场中风让他博得了不少同情,升了半级,改去养老了。

窦舜卿入京诣阙,向宝职位被免,秦州官场的一场大震动,就在一盏茶的功夫中,被王中正画上了句号。

接下来,王中正一改方才拒人千里之外的冰冷,变得笑容可掬起来——轮到王韶、高遵裕和韩冈领旨受赏。

王中正并不是一开始就被派来秦州宣诏的。因为托硕大捷,给王韶等人的封赏其实早早的就跟张守约一起出发。但当古渭大捷的捷报传到京城后,与张守约同行的宣诏使臣便被金牌加急召回京中,改由地位更高的入内内侍省副都知王中正带着改动后圣旨来秦州。

虽然王中正带来圣旨中,并没有将尚未经过验功这道手续的古渭大捷之功一起计入,但给王韶等人的新封赏,却比一开始时优厚了不少。

冲着跪在地上的王韶,将前面一段奖誉其屡立功勋的开场白念完,王中正说到了关键。

王韶本官升任从七品左正言,散官恩受正七品上的朝请郎,勋职为六转的上骑都尉。这三项与早前的封赏并无区别。但天子还另赐了他五品服加银鱼袋,让王韶可以提前穿上象征五品以上官位的绯红色官袍,佩上侍制以上重臣才有的银鱼袋,而作为文学备选的贴职,也换做了直集贤院这个职位。

换上绯红官袍,佩上银鱼袋,在王中正面前再一次跪倒谢恩,此时的王韶终于有了个边疆重臣的模样。

高太后的叔叔虽然在古渭大捷中什么都没做,只是凑数而已,但功劳本就是见者有份。不过他这个功劳要等到几个月后,现在给他的诏书,只是说他忠勤有加,谨事王命。靠着外戚的身份而得到开国男这个爵位的高遵裕,他的食邑就因为这八个字而被加封了两百户。

过了王韶、过了高遵裕,接下来便是韩冈,比起给王韶长篇累牍的赞许,韩冈得到的只有寥寥数句。

韩冈跪在地上,听着头顶上传下来的声音,“褒功录善,邦有常法。尔以才行,自昭于时。比见推称,当增位序。当迁一等,其往懋哉。”

一段废话,韩冈只注意到了‘当迁一等’四个字。他的本官要升官了,才四个月本官就晋升一级,即所谓的未成考而迁官,这在官场上算是很难得了,更难得的是韩冈还没有进士出身。而且这还没有将古渭大捷的功劳算进来的结果。

选人没有正九品,自从九品的判司簿尉上加升一级,便是从八品的试衔令录。王中正读着制书后面的段落,韩冈的本官由原来的密县县尉,叙迁为试衔知莱州录事参军事。

韩冈领旨谢恩,淡然的表情上看不出多少欣喜。迁官一等的这个奖赏,对他的功劳来说实在太微薄了。而他心中还在算着,到底还要积累多少功劳才能从选人转为京官。品级对寄禄官并无意义,选人七阶,除了最底层的判司簿尉,其他六阶都是从八品。而京官还有从九品,但从八品的选人却远远不及从九品的京官。

不过好歹是升官了,凡事都得一步步来,不用着急。韩冈这么想着。

