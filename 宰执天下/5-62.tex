\section{第九章 拄剑握槊意未销(十)}

凉州城已经挂上了大宋的旗号,王舜臣的将旗也在城头上高悬。

自兰州分兵以来,困扰王舜臣所部六千人的最大问题,只是地理而已。翻越洪池岭【乌鞘岭】造成的伤病超过五百,几乎都是冻伤,而攻打凉州和沿途寨堡,也不过两百多伤亡。

这一路过来,王舜臣所部斩杀的敌军也不过千多人,但以六千兵力,就攻下了河西重镇凉州,终归是一件值得庆贺的喜事。

能有这样的战绩,在非战斗减员的损耗如此之多的情况下,也多亏了王舜臣的名望。

王舜臣是戴罪立功,根本就是个白身,但他往军前一站,有哪个敢跳出来炸刺?后台硬得跟铁铸的一般,犯了那么大的事,还能回来领军,秦凤、熙河两路,哪一个不让他三分。

而且王舜臣本人的箭术高超,又有过往功绩,军中威望甚高,没有谁敢违逆他的命令。等到顺利的攻下凉州城后,更是说一不二。

而此时的王舜臣,正从军营回到自己占下作为落脚点的宅院。一名中年的幕僚陪侍在侧,貌不惊人,但脸上一团和气,很容易让人感到亲近。

“吐蕃十四部,汉人的六大家,族长族酋们都答应了,只要将军还想往西去,他们都愿共襄盛举。”

“几乎是凉州所有大家族的合力了,归义军当年也不外如是。”王舜臣喃喃自语。

旧唐主导河西东归的张义潮在大宋境内名气不大,但在河西、陇西的民间,则是如雷贯耳一般。

安史之乱,大唐势力中衰,吐蕃借机一举夺占河西。吐蕃在河西的暴政持续了将近百年,到最后,终于出了一个张义潮。

张义潮麾下的势力,是历经艰险方才一一收归汉土,如今给木征等人占据的岷州、河州,都被张义潮光复。之后更是打了河西周围大州一周,只可惜好景不长,张义潮死后,其婿索勋夺位,归义军势力大衰,直至论现在甘州回鹘和吐蕃人手中。

张义潮的为人,王舜臣听过他的故事就是钦慕不已,身陷虏境,却能杀虏归汉,非大丈夫不可为之。

但王舜臣对张义潮的赞叹已经够多了,没必要时时挂在嘴边,他回头看了眼幕僚,“难怪听冯四说,冯远你的绰号是左右逢源,到哪里都能混个脸熟出来。”

“乃是姓名所累。”冯远苦笑了一声,“其实小人的人缘,不过占了和气生财四个字,其他掌柜也不会比小人差,只是他们不叫冯远。”

冯远并不是跟冯从义有亲,也不是冯从义收的家人,只是恰巧姓冯而已。在顺丰行中,是专门负责开疆辟土的大掌事之一。他会跟随王舜臣西行,正是奉了冯从义之命,开辟河西这条新线路。

半个月下来,王舜臣觉得这一位很好用,比起为他写奏折的酸丁来,头脑、见识、胆略都是一等一的,只可惜他不便挖韩冈和冯从义的墙角。而且冯远这一级的掌事,每年都有少则一两千、多则三五千贯的股红,比宰相、学士的俸禄都高,不可能跟着自己吃糠。

“好了,你也别谦虚了。这些天可是帮了俺大忙。接下来借重你的地方还有不少。等这一仗打完,就在报功的捷报中加上你的名字。”王舜臣赞了冯远两句,又毫不犹豫的给了一个好处。

冯远微微一笑,恬淡平和的向王舜臣表示谢意,却并不将他所许诺的官职放在心上。

王舜臣也没打算挖墙脚,提上一句也就代表他的心意,没有必要多说什么。他很兴奋的说道:“还没说说到底是什么礼物?”

冯远没有回答,而是当他走进庭院后,就突然停住脚,将手向前方一指:“将军请看!”

王舜臣漫不经意的瞥了一眼,当他看到院中的那一个礼物之后,就再也挪不开视线

他的眼睛在一瞬间就亮了起来,呼吸也变得粗重。喉咙很干,如同烧了起来,又像是被吊上岸的鱼,双唇一张一合,却不知能说什么。

他受到的震惊,甚至比看到绝色佳人还要更强烈三分。

出现在王舜臣面前的仅仅是一匹马。

但这匹马有着五尺有余的肩高,快跟身量不高的王舜臣平齐。双目莹润,显得十分聪慧而又灵性。四蹄修长,背部曲线优美,臀部结实有力,淡金色的皮毛如同锦缎一般闪闪发亮。

站立在庭院中的这匹马,如同一颗宝石,散发着诱惑的光芒。

王舜臣看得目眩神迷,如此神骏的龙驹,直如绝色佳丽,万金亦难买,须得量珠而归!相比起来,他一直视如珍宝的那匹四尺七寸的河西乌骓,就是私窠子里十文钱一次的便宜货色。

王舜臣小心的靠近这匹宝马,尽量不让它感到威胁,小声的问冯远:“这是什么马?大食马还是大宛马?!”

“是大宛马。汉武帝曾经用黄金马交换亦未能得的汗血宝马,也就是大宛马中的一种。”

王舜臣双眼亮起,灼灼如晨星:“当真是汗如血色?!”转头就想伸手抚摸那锦缎般的皮毛。

手还没伸上去,那匹大宛马就打了个响鼻,一股热气冲着王舜臣的脸喷了一下,然后抬头扬蹄,对王舜臣的接近很不喜欢。

冯远就看见王舜臣立刻收回手,小心翼翼的样子,看起来是生怕吓到它。他会心一笑,“这匹似乎是没有。不过看模样就知道绝不是凡种,汗血宝马也不外如是。”

“的确。”王舜臣低头向下看了一下,“是母的,一匹牝马……好烈的性子。可惜是牝马!”

王舜臣不无遗憾,要是公马就好了。单匹母马是无法留下良品后代的,一两代之后,就会泯然众人。

冯远也同意王舜臣的看法:“这样的上等龙驹,就是用河西马来配种都嫌太过浪费,比牛粪上插花更让人心痛。不过要是没有阉割过的牡马,价格可就是天价了”

“管他要加多少,倾家荡产也值得。不论是牝马还是牡马。”王舜臣放声长笑,“这一匹多少钱?!”他已经做好了心理准备。

“不用一文钱。”冯远摇着头。

王舜臣脸上的兴奋和急切一点点褪了下去,眼瞳中只剩下冷静精明的光芒在闪烁:“哦,是谁这么大方?”

“献上这匹马的是潘罗征,在凉州城中算是大户,在城外也领有一个部族。随时都能拉出两百骑兵。”

“这匹马是他养得起的?”王舜臣不信,小小的吐蕃蕃部,保不住这样的宝马。

“原主自然不是他,”冯远的笑容意味深长:“是住在他家里的大食商人所有。不过那个大食商人前几天官军攻城的时候,不幸中了流矢……”

“流矢?……好个流矢!”王舜臣唇角勾起,露出一个了然于心的微笑,“继续……”

“因为这匹马已经是无主之物,所以自然就任潘罗征处置。”

“等等!”王舜臣发现这里面有个很大问题:“大食商人不会单身出来行商,他的商队呢?,”

“都是流矢。”冯远脸板得十分正经,“关于这一点,潘罗征没说,小人也没细问,也就是顺手查了一下,倒是不难。”

“办事倒也利索。”王舜臣的评语不知是给谁的。

冯远也不多去猜:“因为种种不幸的意外,所以这匹马就落到了潘罗征的手中。他想献给将军,又怕一层层报上来,在中间就给人贪墨了,然后也让他中了流矢。所以就托到了小人这里,也是将军抬举小人,才让他看中了。”

王舜臣不用细想也知道潘罗征必有所求:“他想要什么?”

“将军应该听说过潘罗支吧?”

王舜臣眼神陡然凌厉起来:“他是潘罗支的后人?难道是准备重立六谷联盟?!六谷联盟不是给党项人杀得差不多了?”

接连三问体现了王舜臣对河西的了解。也让冯远不用解释太多来龙去脉:“董毡麾下有不少六谷部出身的,都是旧时凉州被元昊领军攻克后,逃亡过去的。而且元昊当年在甘州也松了松手,没有像在凉州一样下狠手,让六谷部保住了不少元气。”

六谷部或是叫六谷联盟从来不是什么恭顺的角色,归义军的覆灭,六谷部的前身也出了一份力。西夏太祖李继迁就是死在对六谷部的征伐中。之后其子李德明几次攻打亦是无功而返,只是后来被盟友甘州回鹘反戈一击,大伤元气,最后让李元昊捡了个便宜去。

“马,我代天子收下了,他养了这些马多少天,将草料钱算给他。”王舜臣清楚什么样的原则必须坚持到底,“跟潘罗征说,老老实实的做大宋顺民,自有他们的好处。”

“只是一个六谷联盟。”

“有联盟,就是有异心!就是一百人都嫌多。看看董毡,他堂堂赞普,现在还不是老老实实听朝廷之命?!六谷联盟有什么必要重建?归义军才是最该重建的!”

“归义军已经损失很多了,凉州汉人绝少,反倒是沙州、甘州汉人多些。”冯远道,“其实河西的汉人,大半都改了吐蕃人的习俗,所以旧年曾经孤悬西陲、犹一心维持汉统的归义军也早已星散。”

“俺记得三哥曾说过,入华夏则为华夏,入夷狄则为夷狄。原本是汉人,忘了祖宗,现在就是夷狄。”

王舜臣可是想着镇守河西的位子。从地域上看,河西与熙河有着很明显的地理分隔,联系并不能算紧密。等到战局平定之后,说不定这里就要另设一路。如果自己能全取河西之地,运气好一个副总管,差一点一任钤辖也是少不了的。但如果只取了凉州,那么在攻打西夏的那群将领们的压制下,说不定一个都监就打发了——那边,这边打的是凉州,怎么都比不上的。

