\section{第四章 岂料虎啸返山陵(一)}

第二天便是月亮正圆的望日,也即是朔望大朝会的日子。

就在圆月西斜,仍是星光漫天的四更时候,聚集在宣德门外,等待上朝的官员们之中,气氛便已经很不对了。

多少人在交头接耳,在人群中传递着一桩石破惊天的消息。只用了片刻的时间,差不多所有人都知道王安石的儿子已经被牵连进了李逢、赵世居一案之中。

韩冈并不奇怪这一点,官场就是这么一回事,消息不可能被隐瞒,只会被或有意或无意的扭曲。

昨天赶在午后,蔡确送来了,尽管只是提前了几个时辰——到了入夜之后,章惇、吕惠卿还有王韶都派人来送信了——但这几个时辰,就是一桩大人情,能让韩冈早做应对,也能让韩冈有时间去通知王旁。虽然他没有去做,可人情就是人情,是恩怨分明的韩冈必须要还的债务。

所以当他见到蔡确板着脸,站在宣德门下履行着御史的职责,用鹰隼一般锐利的视线,盯着官员们的一举一动是,韩冈也得在脸上浮起一个笑容,作为感激这份人情的表示,点着头送给蔡确。

礼仪性质的朔望大朝会上,气氛虽然紧绷着,却并没出什么意外,很顺利的就结束了。除了冯京这位宰相没有上朝之外,也就王厚上朝,稍稍引人注意了那么一点。如当年的唐炯一般,在大朝会上将新旧两派一齐骂遍的疯子,十几年也不一定能出一个。

但朝会之后的‘崇政殿再坐’,也就是一如既往地崇政殿重臣议事,却不可能再如朝会时一般风平浪静。

照常例,与朔望朝会同样扩大化的议事,让韩冈等身居要职的两制以下的臣子,也得以与宰执们一齐站在了崇政殿中。

照常例,此时的议事,应该说一些需要协调各个监司之间关系,共同来应对的重要议题。

照常例,应该是宰执们保持着重臣的风仪,在天子面前,与监司主官们一同商议军国重事。

但今天却没有什么常例了,赵顼眉头越结越深,他在这座殿上坐下来已经有半个时辰,但正经事一件也没有开始议论。

李逢、赵世居一案的主审范百禄正唇齿翻飞:“世居自受人言貌类太祖,便结纳匪人,议论军事、怀挟谶语、搜检星图,所谋非小,所交非类。李士宁收其所赠钑龙刀,与其共饮,岂能置身事外?李士宁其人出入睦亲宅【注1】,王旁与其深交,又岂云不知?”

吕惠卿苦恼无比,他不想帮助王安石辩解,他要坐上新党真正的领袖,就必须削减王安石的威望。但现在他却必须为了新党,而保护住王安石这面旗帜,“杜甫赠汉中王瑀诗云‘齨须似太宗’,与此何异?李士宁交游甚广,收受赠礼甚多,何止一刀。此事又与王旁何干?厢军聚众为乱,千百人得见。今日能动用厢军,日后难道就不会动用禁军。”

“厢军为乱,乃大臣行事不谨,致使军变。军士当深责,大臣又岂无罪责?近闻有力工放火于汴河之滨,此事难道只是力工之过?”

韩冈抿着嘴,并没去在意自己正被人攻击着。吕惠卿为了控制局面的走向,肯定要帮自己说话的。

他只是看着一个个正口沫横飞的国之重镇,他们用言语当做刀枪,向着对手砍去的时候,到底有没有考虑过天子的心情?

……应该是考虑过了。韩冈转动着眼珠,看看吕惠卿,又看看吴充,对自己的判断加以确认。

已经经过了一番深思熟虑,但他们最后的选择,还是一定要就此分个胜负出来!

韩冈抬起眼皮,望着高高在上的天子。面无表情,端坐如木偶石像,可眼下的这个局面,应该是赵顼不想看到的。

作为一名领导者,不论是他统领的是一个亿万人口的国家,还是仅仅十来人的小队,都不会希望下属是铁板一块,将自己架空起来,让自己的存在变得毫无意义。但也不会希望手下人势均力敌的对立起来,让该做的正事无法顺利的施行。

正方反方的一个合适的比例,应该是四六开,或是三七开。让主导事务的一派,有着足够的权力去做事,但也不至于让他们太过张狂,而忽略了领袖。必要时,只要偏向反对派,凭着手上的权力,就能将正方反方交换一个位置。

但正反两派的比例如果是二八,情况就会变成一面倒,偏袒反方也改变不了结果。而若是对半开,就可以见到正在崇政殿殿上,上演这一幕扯皮和互相攻击的场面。

异论相搅,是赵顼的选择,也是大宋几代天子经验的集合。就算王安石当政的时候,朝堂上新旧两党的比例,也是保持在一个正确的水平线上。新法的确是在顺利的推行,但王安石也不能不仰仗天子的权威才能行事。

可眼下的情况,很显然异论相搅的手段已经让朝局走上了歧途。无论新党、旧党,都没能占据上风。赵顼尽管在政事上继续偏向吕惠卿,但天子既然要保持着朝堂上的两派对立,吕惠卿也就无法像王安石一样,控制住朝堂大局。

并不是吕惠卿能力不足,而是他的威望不够,不足以如王安石一般,借助一点点的皇权,就能顺利的压制住对手。冯京、吴充、王珪都是根基深厚,不输当年的富弼、韩琦、文彦博多少,可吕惠卿却没有王安石用三十年时间,积累下来威望,而仅仅是一个新进而已。

可是这样的僵局不会保持太久,天子不会容忍朝堂分裂的局面继续下去。对此,无论是哪一边都很清楚。只不过,在双方的想法中,两边既然肯定要分个胜负出来,与其等着天子自己下判断,还不如先行动手,自行将结果得出,最后再让赵顼对这个结果来加以认同。

吕惠卿一直就在准备这样去做,只是他缺乏一个恰当的借口,让他将几个绊脚石赶出朝堂。

他从不认为自己没有能力,也不认为自己会输给冯京、吴充,仅仅是一时之间没有找到合适的借口而已。所以厢军聚众生乱一案,吕惠卿立刻就紧紧抓住不放。虽然只能说是很勉强的借口,但只要能将对手逼入势不两立的境地,还想坚持新法的天子就不能不认同他吕惠卿想要的结果。

同样的想法,也同样存在与冯京、吴充等人的心中。相对于汴河边上的官营水力磨坊里的一干厢军,犯下的那点小事,赵世居、李逢谋反案就严重得多。也让难以受到天子偏袒的旧党,有机会彻底清除新党。

双方争辩的焦点从谋反案到厢兵作乱案,继而又将厢兵作乱案丢到一边,却把对方施政上的错误一个个的揪出来,将崇政殿吵得如同菜市口一般,当然最近行事不谨、出了不少纰漏的韩冈也成了靶子。新党旧党的臣僚围绕着韩冈的功过争论了起来,他们并不在乎对错,只在乎能不能压倒对方,

“厢兵作乱,力工纵火,皆是韩冈行事不谨之故!”

“纵然军器监要代水磨坊,但其中厢军的给俸,何曾会少?既然俸禄不减,此辈若无人指使,如何会与京中作乱?”章惇立刻出班,帮着韩冈说话,“指使厢军攻击大臣府邸,岂能轻赦!?”

“不知辽使在外搜购《浮力追源》,又是谁人指使?”

王韶为韩冈辩驳:“辽使年年来买书,不见有人查。如今不过一本流传世间的寻常书卷,又何须大惊小怪。”

“有此书,即可造飞船、板甲,契丹骑兵百万,得此二物,乃是如虎添翼。”吴充音调低沉,似乎是在为大宋一片黑暗的未来而痛心无比。

章惇一声嗤笑:“不知之前是谁说铁船乃是无用之物,《浮力追源》尽是无稽之谈?”

赵顼听得烦了,心头如同火烧,嘴上的燎泡越发的疼了起来。一眼看到韩冈仿佛无事人一般,站在班列的后面始终没有发话,就点了他出来。不论吕惠卿是准备将韩冈当成是杀手锏,还是根本没有将韩冈当成战力给算进来,赵顼都想听听他的意见。

“韩卿,这两件事你怎么看?”

韩冈依言站了出来,一锤定音的话本来是准备留在最后才说的,但天子相询,也就只能提前了。

“厢军聚众为乱一案,其事涉及微臣,臣不当言。至于王旁事涉李士宁案,子不教,父之过。既然范百禄言王旁与李士宁往来密切,那就召其父入京来询问便是。”韩冈此言一出,所有人的脸色都变了——基于不同的理由,“想必能教陛下自此安心!”

韩冈用重音着意强调了最后一句,在静默下来的崇政殿上,他谦卑的低下头,双眼盯着脚下的一块块被烧制得闪着金属光泽的方砖。双手持笏,等着天子的回覆。

他不怎么擅长下棋,但他擅长掀棋盘。

注1:睦亲宅是宋代为宗室们所建的宅邸群,其实就像是个笼子,便于看管而已。

