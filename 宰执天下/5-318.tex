\section{第31章 停云静听曲中意(25)}

韩冈自请缨,殿上君臣都松了口气。

世所共知,韩冈在河东声隆望重,功勋累累。甚至不用等到他抵达河东就任,只要将他将至河东的消息传过去,人心立刻就能给他安定下来。而京城之中的民心,也同样能安定下来。

——有良臣守边,国中还能有什么不放心的?

向皇后急声吩咐身边人:“宋用臣,你去玉堂传诏,命蒲宗孟速速起草韩学士的任命,安抚河东并总管河东兵马。”

“不可!韩冈为安抚则误大齤事!”蔡确突的跳了出来,大声反对,“安抚河东必为太原知府,如今军情紧急,岂有余暇顾及政事?且任命一下,王.克臣恨不能立刻交接,如何会安心署理军政?韩冈决不可代为安抚。”

蔡确的话说得在理,韩冈重为安抚使的消息一到,王.克臣就可能会立刻整理行装,太原军民看到之后会怎么想?或许王.克臣有名臣风范,但谁敢冒这个风险?而蔡确话中的另一层用意,殿中更是都听明白了,向皇后也不例外。

“……宣抚河东如何?”向皇后缓声问道。

“正该如此!”蔡确肯定的点头,“非此不足以稳定河东,统御一路将帅兵马。”

“只是宣抚使应该要两府中人吧?”向皇后问得明确了。

韩绛十年前宣抚陕西、河东两路时,他是东府第一的昭文相,吕惠卿现如今宣抚陕西一路,则为西府之长。

想要担任扶绥边境、宣布威灵、统兵征伐、安内攘外的宣抚使,无论是翰林学士,还是单纯的资政殿学士,都是不够资格的。

“韩冈当为枢密副使。”蔡确说道。

“相公说得有理!”向皇后点点头,肯定蔡确的意见后,方才征求其他宰辅的看法:“诸卿可有何意见?”

两府之中自是无人反对。

韩冈早就该入两府了。参知政事辞了,枢密副使推了,日日参与崇政殿之会,世人都是视其为不挂名的宰辅,两府之中也视若平常,到了这时候,哪里还会有反对的意见。

“平章呢?”她又问向王安石。

王安石默不吭声。他是韩冈的岳父,不便点头,但他也不会反对。都这时候了,就没必要再自清,反正人人都知道,他的女婿跟他不是一条心。

章敦则多看了蔡确两眼,也不知这位宰相是在示恩韩冈,还是在讨好皇后,或者兼而有之?

最后皇后方才回顾韩冈:“学士,且为国家计,这一回可不能再推托了。”

“为君分忧,臣不敢辞。”韩冈躬身行礼。

危急关头,临危受命,倒也没什么好计较的。不过执政之位,还要皇后求着才肯做,说起来这排场也大得惊人了。

“那就好!那就好!”向皇后点头连声,声音中也多了几许喜意。

“不过天子那边……”韩绛的话没说下去,但谁都知道他想说什么。

“为臣者当以忠直敢言为上,只是天子玉体违和……”蔡确同样欲言又止。

“吾知道,河东的事自当先瞒着官家。等学士赶走了辽兵再提不迟……一切以官家身体为重。”

“殿下所言极是,当以官家御体为重。”韩绛领头奉承。

虽然这么一来,日后肯定会让皇帝和皇后生了嫌隙,不过宰辅之中没安好心的可是大多数,对他们来说,皇帝和皇后生了嫌隙并不是多坏的结果。而且阏塞天子耳目的手段,有一就有二,现在的确是为了天子的身体着想,可日后渐渐就会变成另一种情况。所谓防微杜渐,怕的就是从小事渐渐发展。皇后若能压制住总是不肯安心养病的皇帝,绝不会一件坏事。

确定了援救河东的人选,向皇后便问韩冈:“韩学士,不知这一回到底有几分成算?”

“只要臣到了太原,敢以阖家老小担保太原不失。”韩冈轻松的笑了一笑,“而且眼下是春天,一冬天战马能掉了几十斤膘,不养一养就赶着上阵,能死一大半去。殿下其实不用太担心。”

韩冈的自信感染了皇后,让她放心下来。

“那么学士还有什么要求?钱粮、兵械和兵马都尽管提。”想了想,她又说道:“出战不能没有兵马。西军正好歇下来,学士可在其中拈选精锐,调其北上。”

“西军调来无用。刚刚才打过一仗,缓急间派不上用场。”韩冈缓慢而坚定的摇头否决,“大宋承受不起再一个高粱河之败!”

“兴灵一仗打过,耗尽了西军的气力,必须要有一个大的休整期才能恢复如初,这不是一封诏书就能把兵马调到千里之外的。而且战功的赏赐还没发……确切的说,兴灵之役到底是功是罪还没下定论,如何调兵遣将?”

太宗赵光义惨败于高粱河,有很大一部分原因是因为平灭北汉的功赏没有及时发下,使得军心不济;剩下一部分原因,就是刚刚结束了北汉之战,就调兵东行,攻打幽州。战略上有突然性,可就没考虑到军队的承受能力。

这一惨痛的教训尽人皆知,两府中人不需要韩冈多做解释。只是向皇后还有些懵懵懂懂,她对旧日战例。韩冈不得不费了一番口舌来解释,并顺势将自己的想法和计划从头到尾的说了一遍。

……………………

“魏泽该死!”

在殿上章敦还要压制自己的心情,回到了枢密院,便没了那么多估计。破口大骂着顶替了刘舜卿的州官,“调走了一条大虫,本以为换上的好歹是条狗,谁想到竟然是头猪!”

章敦的话,让韩冈觉得莫名耳熟。他是跟着章敦、薛向一起回来的,方才在殿上已经定下了以枢密副使宣抚河东,虽还没有宣麻,可也算是西府中人了。

“这不是很正常吗?一朝君,一朝臣。府君换了,下面的将校还能安稳做着事?人心散了啊!”韩冈倒不在乎在章敦、薛向面前,说几句悖逆的话。其实这话也没什么,反而是拉近关系的手段。

“玉昆。”章敦觉得韩冈话中有话,“你可是事先知道了河东的内情?”

“好歹韩冈也曾安抚河东,旧属不在少数。天子派去河东北界的人选倒行逆施,自然有人会求到我门下,也少不了抱怨。”韩冈摊摊手,“但我又能说什么?”

以代州知州魏泽为首那几位调去河东时,必然是得了赵顼的嘱咐。既然是秉承天子之意,又有什么不敢做的?前任有功却左迁,他们到任后自然一切都会反着来。为了讨好天子,去找刘舜卿和秦怀信的差错也不在话下,甚至两人留下来一众亲信,也都成了打压的对象。

正如韩冈所说,不是没人求到他门下,可韩冈又有什么办法?本来就是受了他的拖累。也只有辗转托人照顾他们的家人。至于那些被找出差错的军校本人,韩冈则只能干看着。

“刘舜卿被证回易,秦怀信被查冒功,‘无才无德无能,所谓战功莫非杀良,便是编造’,这些弹章难道就没报上给枢密院?刘、秦二人在雁门关中的亲信和重用的将校被一网打尽,一个个被追究罪责。”韩冈一声冷笑。

练了《葵花宝典》的东方不败都能荒淫好色,何况几个让皇帝不开心的臣子?要不是韩冈本身底蕴强,同样少不了被秋后算帐。一边是皇帝,一边只是被打压的臣子,他们难道会怕得罪韩冈,而不去奉承天子?可能吗?

‘这几年可是好一通折腾啊,雁门关的军心早都散了。但凡军心士气还有个一星半点,就不会这么连丢三关!西陉寨能被夺,雁门寨能被破,但代州城怎么会丢?!’韩冈满肚子的怨言岂是一句两句能说得清的,但终究还是没有再多言出口。

“此并非天子本意。”薛向干巴巴的为皇帝辩解着,也只是顺口而已,做臣子的习惯罢了。

谁都明白,代州乱到能让辽军攻破雁门关,这当然不是赵顼的本意!但造成这一局面的却是赵顼无疑。

上面说一,到了下面就变成十,这样的情况太多太多了。或许赵顼是想清除韩冈在河东军中的势力,甚至可能只是因为在韩冈这边下不了台,进而迁怒到几个将领身上,但他调去的接任者,却做得变本加厉,十倍、百倍,唯恐让皇帝失望。

韩冈叹了一声,“天子本是圣明,但也架不住奸佞想要奉承讨好。”

章敦则狐疑的多看了韩冈几眼,却是弄不清他说的到底是讽刺还是真心话。

说是讽刺吧,但韩冈的神情不像。可要说是真心话,都到了他们这个地位,哪里还会相信什么圣君为奸臣蒙蔽?这种情况的确有,但放在现在说的这件事上则绝对不是。赵顼是什么样的人,下面的小官、百姓不知道,他们这等日日面君的重臣怎么可能不清楚?

不过想想也就罢了,追究责任到底在谁身上,也不该在火烧眉毛的时候。
