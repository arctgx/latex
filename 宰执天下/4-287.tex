\section{第46章 了无旧客伴清谈(九)}

“这……三哥,”冯从义瞟上来的眼神似乎是在问韩冈是不是在说胡话,“小弟不过一个小使臣而已,哪有资格上书的?!”

“哪里是要你上书?你要真写了奏本,还不在枢密院就给人挡下来了。”韩冈冷笑着,中枢两府官吏的德行,他再熟悉不过,“这么好的主意,就是到了横行、侍制那一级,都少不了会动心,见到了就会打主意给贪掉。抽掉一个小使臣的奏章,对西府官吏来说根本不算什么;抹去记录,也就跟通进银台司再打个招呼的事——没了证据,我说话都不管用——做得绝的还能先栽你个罪名,这种事不是没有过。你这边一解决,过两日,枢密院就能换个人报上去了……愚兄是要你回熙河路操办。”

“回熙河路?那多耽搁时间!”冯从义说着,“直接在京城做起来,也就说句话的事。”

“……什么时候口气变得这么大了?”韩冈扬眉笑问。

“三哥你可别不信,小弟可不是在吹。只要三哥你点头,十天之内,赶在年节前,小弟就能在京中将赛马争标给操办起来。就是马球联赛,两个月,弄个六队八队出来不成问题。”冯从义昂头挺胸,“上个月,何仁美——就是邠国大长公主的驸马的亲娘舅——还带话给小弟,问蹴鞠联赛是不是一年再多踢一个循环,踢半年、歇半年未免太浪费了。他可是帮他外甥和外甥媳妇在问!”

才几年功夫,冯从义拉着整个陇右商人的势力,以棉布为敲门砖,在京城商界站稳了脚跟,眼下已经是举足轻重的地位了。牢牢控制着棉行,又利用蹴鞠联赛上下沟通,上至王公勋贵,下至地痞泼皮,他都能说得上话。真要细论起来,他在京中的人脉关系比韩冈还要深厚。

有关的传言,韩冈也听得很多。虽说熙河、广西乃至京城局面都是韩冈开创的,但能做大做强,还是靠了冯从义本人的能耐。对自家表弟的经营之术,韩冈也是很有几分佩服。不过冯从义现在急冲冲表功的样子,倒是没了名震京城商界的冯大掌柜的气派。

“怎么就急了,小孩子似的。愚兄怎么会不信?陇右冯四在京城的名号,我这做哥哥的可是如雷贯耳,大名鼎鼎的冯财神啊。”韩冈笑了两声。神情也郑重起来,“京城太过惹眼,愚兄身侧也多挂碍,如今做事都不方便。如果你在京城将赛马的事做起来,只因愚兄的缘故,最后的结果可能会跟预想的反着来。”

韩冈看了冯从义一眼,发现他专注的听着,满意的继续下去,“现在熙河和京城关系紧密,有什么新奇的活动一两个月就能传到另一边。只要你能在熙河将赛马争标的声势给做大了——马球队练起来不是一天两天的事——必定会有人主动上门来询问究竟。”

这是下饵钓鱼,要人主动上钩。冯从义点头:“小弟明白了,回去后就办。”

韩冈怔了一下,他还没有细加解释呢,怎么这般爽快就同意了,“就不怕万一京城中的人直接将赛马操办起来,甚至捅给天子?”

“三哥怎么可能眼睁睁看着小弟吃亏?”冯从义嬉笑了一声:“其实只要他们想将此事做成,免不了要将熙河路的事拿出来做证明,否则谁会跟着他们走?何况只要熙河路一做好准备,小弟就会让下面的人在京城将此事同时传扬开。传到天子耳中,也只是一两天而已。”

“正是这个道理。”韩冈听得更满意了,“到时候,愚兄也就可以顺理成章的将你荐到天子面前。”

冯从义犹豫了一下,推脱道:“其实现在有个官身就够了,在天子面前露脸反而麻烦,到时候成了众矢之的,反而会拖累三哥。”

韩冈盯着表弟:“当真是这么想?别想是否拖累我,先想想自己。不用担心别的,当今的这位皇帝,一心要想振作。只要是能有补于朝廷,天子必定不会吝惜爵禄。”

“狗肉上不得台面,小弟也不是够资格进文德殿,崇政殿的人。若是以赌赛之事上殿,反而会给三哥你脸上抹黑。”冯从义突然又呵呵一笑,“而且也要怪三哥你,小弟刚到洛阳就听说了,御史台上下都被你得罪光了,但他们奈何三哥你不得。可现在小弟要是跳上去,这不是自找苦吃吗?”

冯从义想得很清楚,顺丰行眼下的兴旺,是靠着韩冈撑起来的。若是韩冈倒下了,顺丰行转眼就能败落掉。韩家的根底太浅薄,这与那些根基深厚的世家大族是不一样的。所以冯从义很明白,韩冈在朝堂上的地位就是一切,无论如何他都必须维护韩冈的形象。

他会匆匆赶来京城,就是担心韩冈会出事,眼下既然知道皇帝不放在心上,也就能放心下来。靠了种痘法,韩冈的恩泽即将遍及天下。冲外面说一句自己是韩龙图的表弟,寻常百姓不必说,就是一干心高气傲的士人,也得给几分面子。

这样的情况下,上不上殿又有什么关系?看了天子又不能让荷包里多几两金子、银子,说不定还会给刮上一笔。要是天子褒奖太重,还会引来御史台的那群乌鸦,躲着还来不及。

见冯从义神色不似作伪,韩冈点点头:“既是如此,愚兄就静候佳音了。到时候,该拦着的肯定会拦着。”他笑了一声,“既然不想担这个虚名,至少要将实利拿到手!”

冯从义起身,抱拳一礼:“那京城里面就托付给三哥了。”

韩冈皱起眉抬起手,示意冯从义坐下:“什么叫‘托付’?这其实是愚兄的差事!”

冯从义笑道:“赌马一事,三哥你看到的是千军万马,小弟看到的却是真金白银。只为金银,就不会是三哥你一个人的差事。眼下这叫各取所需,公私两便嘛!”

韩冈指着表弟,无奈的摇头笑着:“你啊……这张嘴真不愧是财神爷的水平,难怪界身巷金银交引铺的宁大的名号会给你顶掉。”

韩冈的心情很好。

赌马自然会引发天下富户养马的兴趣。也许能够最终上场、参加各级赛事的赛马为数寥寥,也许只有一两千而已,但作为基数,养在富户家中的马匹,必然是几十倍上百倍于正式上场的赛马,而且都是经过训练的马匹,只是在训练的过程中被淘汰了而已。

也因此,也就有了培育马种的好处。

后世的纯血马,不就是从几匹阿拉伯马繁衍下来的?当然,纯血马是特化的短距离竞赛用马,屡屡近亲回交,最后脆弱得没人照顾就活不长,那样的马并不适合作为军用马。不过要培养出纯血马,不知要多少年,不用担心,更不用指望。

眼下的情况,肯定是有心参赛的富户豪门四处去搜罗上等良驹。河西马都是普通,说不定,印度、阿拉伯,或者是俗称汗血宝马的阿尔捷金马,都有可能到手。只要能设立种马配种收费制度,以名次排定收费高低,想必良驹的血统会一代代的在中原流传下去。

冯从义的心情也很好,这个主意可是他给韩冈出的,“想必假以时日,中原富户家家养马,中国的良驹当不输给契丹。就西夏每年上贡契丹三万匹马,两国加在一起还是赢不了。”

“大宋国力岂是契丹、西夏能比?”韩冈道,“不过西夏上贡点的三万匹,不仅仅是马,也有骆驼。贡品光是马,西夏也吃不消。”这也是为什么群牧司下面的官吏有信心能让天子施行户马法,都是西夏和辽国闹的。

“都是骆驼一样也吃不消啊,西夏本身国力就不算雄厚,在这么给辽国吸血,迟早完蛋。真想不通辽国怎么这么贪?不是说辽国魏王是权臣吗?能掌大权怎么还会如此糊涂。当真比不上当年的韩大王。”

“能比得上那位晋王的的确不多,就是将大宋历代宰相算进来,也寥寥无几。如今的魏王自然不如。”韩冈对那位做了辽圣宗便宜老子的韩德让很是佩服,身为权臣,生前生后荣宠不衰,也可见他的能耐,“不过耶律乙辛是个聪明人,他肯定是不想这般盘剥西夏,不但削弱了西夏的实力,还会让党项人离心离德,动摇秉常的地位,但不从西夏那里弄来足够的回报,他也无法唆使得动国中各部势力全力支持西夏。”

“三哥的才智果然是没什么人能比得上呢。”正说着,外面传话说饭菜已经准备好了。冯从义立刻停了口,摸摸肚子:“吃过饭,睡一觉,明天小弟就起程回陇西,会漂漂亮亮的将赌马的事办好。”

“这么急做什么?”韩冈不高兴板起脸,“过了年后再说。钟哥儿、钲哥儿还有金娘都想你这个表舅呢。”

冯从义摇头:“小弟也想留下来,可事先都跟家里说好了,今年在家过年,也答应过姨父、姨母了。”

韩冈闻言神色一黯,叹了口气,他这个儿子不孝,不能侍奉父母身边,要是再留着被当做儿子看待的冯从义,也的确过分了。

“好歹也歇上两天,回去有半个月就足够了。”

“谁知道路上会不会下雪,这一次过来是运气好。一路晴天,路上的雪又不厚。但回程就不一定了,早点走,也能防着路上有事耽搁。”

