\section{第31章 停云静听曲中意(七)}

从福宁殿中出来的时候,向皇后面色如挂重霜。

厚底宫靴沉沉的踏着黝黯的金砖。她脚步落下时,一记记的重音让周围的内侍、宫女们身子抖得厉害。

在宰臣们离开之后,向皇后还是觉得吕嘉问做开封知府不合适,而韩绛等宰辅连点异议都没有,也实在不对劲。为此她便没有立刻前往崇政殿,特意多留了一阵,规劝了赵顼几句,却没有劝动丈夫。赵顼坚持要任命吕家的那一位家贼为开封知府。

重病垂危的天子,如果有了宰辅们的支持,纵然她是垂帘听政的皇后,也照样轻而易举的就被架空掉。就在昨天,口出成宪,说话即为圣旨的她,现在却只能含忿夹怨的走出福宁殿。

刚出殿门,前面的廊道拐角处便转过来两位前呼后拥的宫装美人。当她们看到皇后銮驾做先导的宫女,立刻就退避到道边,屈膝行礼道着万福金安。

“是来给官家请安的?”向皇后居高临下的问着。

“回圣人,妾身正是来给官家请安。”邢妃和宋美人两位嫔妃低声回话。

自从天子中风垂危,太后也不再管理后宫,垂帘听政的向皇后主掌宫内宫外事,除了朱妃以外,其他的嫔妃都很少再到福宁殿来,只是早晚来问个安,日常都是在自己所居的院阁中老老实实消磨时光。可现在天子病情稍稍有了一点起色,就又都纷纷拥到了福宁殿来。

“官家就在里面,正醒着呢,你们进去吧。”

向皇后没有多理会她的姐妹,依照礼节说了两句话后,便径直往崇政殿过去。銮驾从眼前过去,邢妃和宋美人如释重负,年轻一点的宋美人甚至都长舒了一口气。方才的短短几句对话,来自皇后身上的巨大压力让她们差点喘不过气来。仅能管理后宫和拥有生杀予夺之权之间,可是有着天壤之别,权同听政的皇后是她们这些嫔妃决不敢触怒的对象,只能祈天祷地的期盼天子早一点康复。幸好在这两天,终于看到一线曙光。

宋用臣紧随在向皇后身后,从侧后方偷眼瞥了她一眼,心尖顿时颤了几颤,连忙脸色发白的低下头跟着走路。

宰辅们已经在崇政殿中等了好一阵。崇政殿再坐不同于文德殿上的朝会,一般情况下,宰辅们都是有座位,甚至能得赐茶汤。只是这必须有天子或权同听政的皇后、太后来赐坐,否则就得站着等。

章敦、蔡确等几人正交换着眼神,他们并不知道向皇后留下来的确切原因,但他们能想象得到。七八成的可能是着落在之前开封知府的新任人选上。

韩绛自然也能猜想得到皇后迟迟不至的原因,殿上有好几人在观察着他的反应,只是他眼观鼻、鼻观口,表面上不露一丝破绽。

殿后一片脚步声,不过其中那种笃笃的踏步声让人听着耳熟,可走路的节奏却很是奇怪,心思剔透的人自然能听得出皇后殿下的心情应是极为烦躁。

向皇后在屏风后坐下,透过单薄的素屏看着首相韩绛的眼神很是不善。王珪给她留下的印象实在是太恶劣了,连带到恭顺听话的宰相都让她愤怒异常。

忠臣往往多有劝谏之行,而绝不会凡事皆顺从天子意。旧日魏征违逆唐太宗,让李世民回到后宫还念着要杀这个‘田舍翁’泄愤,长孙皇后却换上朝服向唐太宗拜贺得一贤臣。历代贤后都知道要劝谏皇帝勿为小人所惑,要听从诤臣的反对意见。她心里只有一个朴素的认识,凡事都依从上命的必然是奸臣。经由王珪之事后,这个认识便更为根深蒂固。向皇后为什么那么信任韩冈,就是因为韩冈他硬顶着皇帝临危乱命,保下了她母子二人,这才是忠臣之举。

“官家任命吕嘉问为开封知府,诸卿可有何意见?”

一开口,向皇后便提起吕嘉问的任命。蔡确眼睛顿时一亮,韩绛紧绷起的肩膀也一下松弛了下来。皇后对天子独断独行的任命耿耿于心,对他们绝不是坏事。本来还以为要一阵子皇后才会将自己心中的主张一表露出来呢,谁知道会这么快?

“臣无异议。”

“臣亦无异议。”

“吕嘉问资历虽浅,但也是上州知州,与一权发遣可也。”

东府的几名宰执接二连三的投出了赞成票,这让几个月来见多了朝臣们互唱反调的向皇后心中怒意更炽。

“以资望论,吕嘉问自是远远不足。不过其人饶有才干,这几年历任地方考绩多有课最,否则王平章旧日也不会倚之为臂助。可为适任。”蔡确最后总结道。

这是将王安石架了起来,王安石也不可能说蔡确说得不对。而且他也的确需要吕嘉问回来。赵顼选取吕嘉问为开封知府,明显的就是为王安石助势,也让王安石手中的权力能确实的转化为对朝局的控制。如果没有得力心腹在外为臂助,同样可以被两府架空掉。

王安石脸色一瞬间变得很不好看,只是黑面皮的王平章脸皮变得更黑一点也没人能看出来。

“吕嘉问确为人才。”他干巴巴的说了一句,却不再多言。

“吾曾记得吕嘉问旧年曾经掌管东京市易务。那两年,京城里物价贵了不知多少,民怨也多,甚至隔上几日就有宗室入宫哭诉!”

向皇后对吕嘉问这个人记得很清楚,熙宁六年、七年的时候,王安石手底下的大将,被骂得最凶的几人中就有一个吕嘉问。食货一事最关民生,京城中物价涨起来后,纵然罪责最后因为粮商一案落在了一干宗亲们的身上,但吕嘉问这个市易司的主事者却肯定是被骂得最狠。

“难道朝廷里面就没有更合适的人才?!天下各路,那么多转运使、安抚使,各地州府又有那么多知府知州,偏要巴巴的选一个民怨多的!官家这是乱命,尔等皆食君禄,难道就不知道要出来劝谏吗?”

向皇后怒声质问,双手紧紧抓着椅背,白皙的脸皮涨得鲜红。

宰辅们无人接口,皇后心中对天子的任命有了芥蒂的确是好事,但也不能闹得太厉害,那反而会坏事。只是现在谁上去都会被皇后骂回来。

隔着屏风,向皇后看不到宰辅们视线的落点。但成了众矢之的的韩冈知道,这话必须他来接。

他踏前一步,“殿下。论及朝堂,比吕嘉问更为适任的确大有人才,可以吕嘉问的才干,也确能适任权知开封府一职。”韩冈言辞恳切,“殿下,这一任命毕竟是天子所拟。天子病体初愈,当以顺之心意为是。万一天子有思虑不周的地方,有殿下和两府诸公拾遗补缺,当也能弥补得了。”

韩冈站了出来,向皇后的火气便稍稍收起了一点,无论如何,她也不方便冲韩冈发火。只是她狐疑的看了看几名宰臣,韩冈是不是给他们挟持了。

“凡事都奉承官家的心意,那岂不是奸臣所为?与王珪何异。”

王珪真的完蛋了。韩冈闻言心里便冒起了这个想法,在皇后心中都是铁铸的奸臣了,日后恐再难翻身。

不过他的话没耽搁,语气更加诚挚:“并不是说一切都依从天子命。只是非是事关军国之重,这等无伤大雅之命,也就从之心意便是了。”

韩冈说的话可算是医嘱。病人心情好,病才能好得快,本来就是放诸四海而皆准的道理。向皇后沉思良久,最后还是点了头。

“且令其权发遣开封府事。”

在向皇后撤帘归政之前,皇帝手中的权力在程序上是掌握在她手里的。如果向皇后硬犟着不在吕嘉问的制书上盖印签押,宰辅们除了逼其奉还大政外,其实也没别的办法——总不能学韩琦,将皇后灌醉了糊弄她签名画押。现在向皇后点头认可,吕嘉问担任开封知府的最后一道槛也终于被跨过去了。

皇后松了口,宰辅们也算是松了一口气。蔡确的眼神在韩冈身上转了一转,能在皇后面前最能说得上话的,果然也只有他了。这还真是……蔡确的心中也不免升起一股深深的嫉妒和忌惮。

向皇后勉强同意了这个任命,但她立刻又想起了一事,“现如今西北战局不明,若是边关军情有何不妥,那该如何是好?”

“若有捷报,自当飞报于天子。”意在言外,若是什么不好的消息,就不能贸贸然说给皇帝听了,一切当以赵顼的身体为重,不过韩冈还是把话挑明说了出来,“至于凶信,自当有殿下先行处断,待时机再报予天子,以免忧急伤身。”

韩冈的话,深究起来,近于悖逆,但宰辅们无一人站住来反对。不论有没有跟章敦、蔡确合谋,他们的心思却都是一模一样的。皇帝能将他们随意送入两府,自然也能一句话将他们从两府踢出去。而刚刚接手大政的皇后就必须要他们的支持,来维持对朝堂和国家的控制。

向皇后想了半天,觉得韩冈说的还是在理。自己与官家有夫妇之亲,本是一体。如今权同听政,代行君权,有自己在这边听着,臣子们只要将朝事都报与自己听,就不能算是欺瞒君上。至于官家,现在当然是以安心养病为上,等到病好了,就没有这么多忌讳了。

韩冈则瞅了瞅沉默如石的王安石,不知他的岳父会不会将方才的一番话泄露给赵顼呢?恐怕很难决断吧!
