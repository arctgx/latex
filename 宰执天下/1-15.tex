\section{第九章 闹市纷纷人不宁(上)}

正在吃饭的时候,一条令黄德用惊讶不已的消息,让他放下了手上碗筷:“不是韩千六来,反是他儿子到了?!”

站在黄德用面前通风报信的人个头只及五尺,瘦得像根竹竿,脸颊上看不到肉,倒显得两只眼睛如牛眼一般老大,像只饿久了的猴子多过像人,乃是黄班头手下的衙役,姓刘行三。刘三他腿脚利索,又是个包打听,是黄德用手下第一个惯得使唤的。韩冈入城不到半日,刘三便已经把韩冈的行动打听得清清楚楚:

“的确是韩三秀才,而不是韩千六。韩三入城后就径直到了县衙,在户曹刘书办那里缴了文书,已经把名登了。现在是往东门口的普修寺去了,许是想借间厢房住下来。小的看着他进了普修寺的门,便赶着回来报信!”

“代父应役?这措大倒是有孝心!”黄德用赞了一句。世风日下,如今有孝心的小子倒也不多见了,自家的两个小子还不如他。

“韩三一入城就直奔县衙,俺以为会是去敲冤鼓呢。哪想到他会服软,老老实实的去户曹缴了文书。俺们兄弟几个倒是白在鸣冤鼓下面守了一天。”

“肯服软就好。”黄德用笑了起来。韩家若不服,虽是早有定计,却总归有些麻烦。现在这么一服软,也省了他许多事。

韩冈即已入彀,韩家的田和人肯定是要换主了——衙前两个月,没个三五十贯别想有好日子过——河湾的菜田归亲家李癞子,但人可是就要送进黄德用的房里了。

一想起韩家的小养娘,黄德用的心头、胯下便是两团热火在烧着,那相貌,那身段,他做梦都在想。前次去下龙湾村探亲家,看到擦身而过的韩云娘,黄德用差点就走不动路。这等带着胡人风情的小美人,实在太合他的口味。

伸出舌头舔了舔被烧得发干的嘴唇,黄德用兴奋的站起来,“走。去见见韩三秀才去!”

……………………

普修寺中,韩冈此时已经把自己的房间收拾整齐,连随身携带的书卷,也在床头处稳妥的收好。就算不在家中,若有空余时间,他还是照样想多读读书。要想在此时混出个名堂,肚子里没货,根本难以实现。

普修寺是秦州城中的一个小庙,只有三个和尚,两重院落,供着佛祖的大殿还没有两丈高,香火当然也不旺盛。大的寺院,自家就有田,可以雇佃农来种粮种菜。如普修寺这等小庙,便只能靠着香火钱来买了吃。

和尚要守戒不吃荤,菜可是要吃的。普修寺的蔬菜供应有三成是韩家负责。韩千六信佛,不敢多赚寺庙里的钱,每次卖菜给普修寺,总会把价钱算得便宜一点。多少年下来,普修寺的几个和尚也算是跟韩家有些交情,跟韩冈也很熟。当韩冈今天说是要借个空厢房落脚,主持和尚道安没二话就借给了他。

韩冈不是没考虑过去州衙击鼓鸣冤。但前世留给他的经验,让他明白贸然上访从来不会有好结果,被拦着还是小事,若是给人乘机找个借口弄进大狱里吃牢饭那就惨了。韩冈从不信什么青天大老爷,尽管按他的盘算的确是要借助秦州官员的力量去对付成纪县的胥吏,但他绝不会把希望寄托在那些官员的人品上。

“韩檀越,县里的黄班头来了,要你快点出去拜见!”

道安老和尚在外一声唤,韩冈在内听到声音,心底杀意顿起,快刀一般的双眉一挑,直欲飞起斩人。

韩冈早已想通了李癞子大费周章的原因。李癞子不想让韩家赎回河湾菜田,只有两条路可选。一个办法是对存放在县衙里的田契做手脚,让韩家赎无可赎。但这里有个问题,因为韩家与李癞子定的典卖契约,为了省去契约税并没有去县衙登记,仅是只有指模和签名的‘白契’,而不是加盖了红泥官印的‘红契’。此种避税方式虽是世所常见,但最后使得存放在县衙架阁库中的田契上,还是韩千六的名字。这种情况下要改动契约,不是十几贯就能解决的问题。

另一个办法,就是设法让韩家把手上的一点钱都用掉,无法再赎回田地。这世上还有什么比支应差役还要费钱的差事?只要请黄大瘤说动户曹的吏员,发一张征调衙前的公文,几天工夫就足以让韩家沦入赤贫境地。而黄大瘤……韩冈突然冷笑,前几日韩阿李不是说过了吗,黄大瘤可是对小丫头垂涎三尺。借用韩家的钱和人来让韩家万劫不复,李癞子……不!应该是他背后的黄大瘤当真是用得好计!区区一个李癞子,还想不出借用衙前害人的计策。

韩冈恨透了趁火打劫的黄德用,他自行送上门,韩冈求之不得。他准备的几套剧本中正有这么一段。只是黄大瘤来得太急,这里还没安顿好,就已经杀了过来,当真是步步紧逼。

‘也好,先把事情闹起来再说!’

韩冈眉目生寒,大步出了厢房门。从院落外转过去,就见着三个随从如众星捧月围着黄德用站在正殿中央。黄大瘤的一张圆脸扬得高,瘤子挺得更高,仿佛一枚倒转的葫芦,得意洋洋的正等着韩家的三秀才低头叩首。

“韩三还不过来拜见黄班头!”作为跟班,刘三帮主子催促着。他一见到韩冈,便心中生厌。高大的身材让刘三嫉妒不已,而读书人自有的风仪,也是混迹下流的刘三远远难以企及。一身宽袍大袖的韩冈从殿后转出,步履从容、举止自若的姿态,猴子怎么也学不来。

“韩冈见过黄班头。”韩冈走过去,只对着黄德用随意的拱了拱手,连腰也不弯一下,“韩某还要到街上置办点什物,顺便再去县衙里问问安排给韩某的究竟是什么差事。黄班头若有事差遣韩某,还请边走边说!”

说完,也没等黄德用有何反应,便自顾自的往庙门外走。韩冈此举根本就没把人放在眼里,可谓是无礼之极。成纪县的黄班头脸上霎时阴云密布,瘤子涨得血红,这几年除了头顶上面的那些个官人、衙内,还有谁敢如此落他面子?

“韩冈!你站着!”一见主子发怒,刘三忙追着韩冈一声大喝。

韩冈充耳不闻,只快步走到普修寺门外,方停下来转身回头。黄德用虎着脸带着三人跟了出来。韩冈脸上似笑非笑。黄大瘤四人怒容满面。几人对峙在普修寺门前,顿时引起了街上众人的注意。

韩冈久病,身子骨弱了许多,可读书人的气度还在,青色的襕衫穿在他身上,更是透着遮掩不住的文翰之气。他笑得冲和恬淡,连原本给人感觉显得太过锐利,仿佛要被刺伤的如刀眉眼也在笑容下柔和了许多。而跟韩冈比起,黄大瘤四人形象各异,却没一个好的,倒显得是妖魔鬼怪一般。

“韩冈,你好胆!”刘三直指韩冈的鼻子叫骂,只是五尺出头瘦如麻杆的他,在身高六尺的韩冈面前,明显气势不够,就是一只气急败坏的瘦皮猴子。

韩冈无视掉吱吱乱叫的瘦猴子,对上黄德用,冷然问道:“不知黄班头有何指教?!”

黄德用上下打量了韩冈一阵,阴险的眼神似是盯上了猎物的毒蛇,他慢吞吞的道:“……韩秀才,你倒是有胆色。”

“韩某自幼受圣人学,多读诗书,胸中自有天地浩然之气,纵有些魑魅魍魉扰人清净,某又岂会惧之?”

“你就尽管耍嘴皮子好了。”黄德用凑上前,在韩冈耳边阴恻恻的低声说道:“看你这张利嘴能不能保住你家的养娘!”

韩冈闻言,双眼眯起,眼神一下转利,‘原来真的是你。’

猜测终于得到证实,找到了想打自家女子主意的祸首,韩冈突的温文尔雅的笑起来。他退了半步弯腰拱手,语重心长地规劝道:“韩某观黄班头项上赘疣多生,体内气血必亏,若不戒绝女色,怕是难过耳顺之龄。韩某一番肺腑之言,还望班头深思之!”

韩冈的刻薄话说得文绉绉的,黄大瘤愣了一阵,方才反应过来。而围观的众人中早有不少听明白的,顿时引起一阵哄堂大笑。

黄大瘤脸色铁青,瘤子血红,他几乎一辈子都没受过这样的羞辱,瞪着韩冈咬牙切齿,“你好胆!”

韩冈如愿激怒了黄大瘤,脸色便是一变,声音突然大了起来:“不如班头胆子大!你为了图谋我家的田地,篡改了官府文书逼着我这单丁户出衙前差役。不过为国不敢惜身,此事韩某我认了!现在你又得寸进尺,将主意打到韩某家人身上!有胆量的,把我韩家赶尽杀绝,看韩某敢不敢杀到州衙里去,呈血书敲冤鼓!韩某在横渠门下数载,同窗好友甚多,若我有个什么三长两短,别以为没有为韩某抱冤雪恨的!”

