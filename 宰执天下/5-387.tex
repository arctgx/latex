\section{第34章 为慕升平拟休兵(19)}

代州的辽军在烽火点起的一个时辰之后,终于陆续赶到了战场。

对此期盼不已的不仅仅是大小王庄中的守军。以韩冈为首的宋军高层,也同样是期待已久。
当接连三波辽军援兵抵达战场后不久,第四波援兵——也是人数最多的一波——在大小王庄后方的一处坡地上打起了萧十三的帅旗,从庄中传出来的欢呼声隔了两里地,传进来在巢车上远观敌阵的韩冈等人耳中。

“辽贼的主力到了。是萧十三!”
来自更高处的飞船上的通报,让章楶、黄裳几位幕僚都松了一口气,等了许久终于是把大鱼给等出来了。
韩冈点了点头:“那就撤回来一点,不要挡着萧枢密的路。”他的语调十分轻快,完全听不出有因为没能及时攻下大小王庄而功亏一篑感到遗憾的意思。

韩冈简短的语句,很快就通过他的幕僚团转化为一条条具体的军令,从飞船上垂下来的一串旗帜发生了变化,而佩戴着小旗的传令兵也立刻上马,向着各自的目标狂奔而去——在广达数里的战场上,金鼓和号角声并不足以将军令传达到各个角落,必须要通过飞船旗语和传令兵相互印证来保证准确无误的传递。

守在本阵周围的骑兵先一步出动了,他们要会合之前就在外围处阻截代州援军的友军,保护正在前方的步军阵列安然退回。
大地在奔驰的战马脚下震颤,扬起的灰土一时间都干扰了高达数丈的巢车顶端的视野。辽军的主力刚刚跟随萧十三赶到,人和马的体力都还没有恢复,养精蓄锐的这一支骑兵足以挡住他们。

前方的鼓声渐渐停了,围攻辽军营垒的几支队伍停止了他们的进攻,从最前线开始后退。
远眺着前方的两座村庄,辽国的旗帜在千里镜中历历可见,这两天方才押送粮草赶到忻口寨的留光宇不无遗憾:“要是辽贼手上没有神臂弓,那两座庄子一个时辰不要就能打下来了。”
“要是没有神臂弓,辽贼根本就不会守在此处。”韩冈笑着道,“有弩有马,能打能走,就是仗着在代州得到的军械,萧十三才有胆子把这几千人放在我们眼皮下面。”

大小王庄里面的辽军手中,拥有太多的神臂弓,在韩冈并不打算付出太大伤亡的情况下,很难再短时间攻克。神臂弓威力很强,是大宋官军用以克制四方蛮夷的神兵利器。可现在掌握在辽军的手中,攻打大小王庄的难度便陡然提高。如果有足够的时间尚能剥丝抽茧,但要赶在代州辽军大批赶到之前攻下营垒,难度就太高了一点。所以在一开始,韩冈就没打算攻下这座营垒。

在韩冈平和的话声中,最前沿的几支部队陆续脱离了战斗,甚至连堵在辽营门口的一系列攻城车也都拉了回来。而在外阻截代州援军前锋的骑兵——包括阻卜人——也随之陆续回返。
失去了金鼓与厮杀齐鸣的喧腾,一时间,战场上陡然安静了起来。

……………………

“宋狗退了!”
萧十三松了一口气。
刚刚抵达战场的萧十三正领着他的本部,在战场边缘处恢复体力。但他的帅旗不仅让大小王庄中的守军倍增信心,看起来同时也让宋军失去了继续作战的勇气。
宋军终究是不敢在战场上拼上一把。这让他放心了不少。
或许是因为五台山那边还有一支奇兵,让韩冈不愿意付出太大的代价。也或许是对营中的上万张神臂弓无可奈何。

虽然萧十三还是极为惊讶宋军突然间冒出来的这么多兵力,但不管是什么原因,看到宋军没能破掉自己的布置,他心中平添了几分自负。

抚摸着被汗水打湿皮毛的坐骑,感受到腿股处的酸胀,萧十三想着进庄子后定要先歇一歇才是。
只是直到萧十三他领军前行,率军开始进驻了大小王庄,依然没有发现宋军的中军本阵有丝毫后退的迹象。

宋军不是撤退,是收缩。并没有撤回营垒中,仅仅是围绕着韩冈的本阵把阵型收缩得更加紧密。

‘有用吗?’萧十三冷笑。当退不退,那可是庸将所为。
他所率的兵力,并不是援军的全部,后续的还有更多的军队将会陆续赶到,到那时候,刚刚遭受了一场挫折之后的宋人,可能挡住兵力相当、且士气正盛的大辽骑兵?

只是当他收到了同样高悬于天际的飞船的通报,脸色顿时变得发青发黑。
“这怎么可能……”不知是第几次开始呻吟同样的话语,来自探察自宋军方向上的最新情报,让萧十三难以抑制心中的慌乱。
……………………
就像一批批辽军骑兵不停地从代州赶来,宋军也在增兵。

昨夜前线的兵力已经提高了一倍,而今天,前线兵力的增长更是没有停歇。

就在敌军面前的,轨道的威力,在方城山和天下各处港口、矿山发挥了多年之后,第一次在战场上体现。
一个指挥接着一个指挥抵达前线,一面旗帜接着一面旗帜加入到营垒前方的阵列之中。
每过一刻,前线上的阵列就厚实一分。

虽然很多人对此惊诧莫名,甚至一开始时折克仁直喊着难以置信,不过见识过千年之后的运输力量的韩冈,并没有觉得这有多了不起。
从忻口寨到前线,总计七十里长的简易轨道,由于不需要长时间使用,所以仅仅是将生铁轨道固定在枕木上,铺设在草草修复好的地面。但这比起在官道上行驶的马车,依然是强了不知多少倍出去。

一列由十二匹挽马拉动的轨道马车,拖着四百人上下的一个步军指挥,包括他们的个人物品和军械,走完全程只要两个时辰。而在整条轨道上,有整整二十列轨道马车在运行。至于配属的骑兵,早就在这些天与辽人的对垒中,逐渐转移到了最前沿。
相对于驻扎在代州、只需要一个时辰就能赶来支援前线的辽军骑兵,忻口寨以步卒为主的宋军,想要靠他们的双脚赶去百里之外的前线,则需要至少两天的时间。所以大小王庄上下才会十分安心,宋军若真的大规模进驻前沿战线,必然能早早的就侦察到,到时候便能相时而动,无论是分兵骚扰,还是全力出击,都能压得宋军失去了继续投入兵力的信心。
但宋军利用紧急修筑起来的轨道,经过沿途的一座座村寨,昼夜不息,仅仅一夜之间,便将前线的军队数量,提高了一倍有余。

纵然是亲眼目睹了整个过程,章楶到现在为止依然觉得匪夷所思:“想不到轨道的运力一至于斯,难怪薛师正一直想要在河北、京东和淮南修筑新的轨道。”

留光宇也回首几面正从官道上移动下来的军旗:“旧日曾闻轨道堪比汴水,当时尚犹疑,今日见之,信然。”

“此皆是黄怀信和俞正主持修造之功。”
主持修筑这一条简易轨道并不是韩冈曾经在方城山时任用过的幕僚,而是一名内侍和将作监中由匠师升上来的伎术官。
宫中以精于机械巧器闻名的内侍黄怀信,以及国初闻名天下的大匠俞皓的曾孙俞正。可惜两位功臣并不在这里,在此处轨道修筑成功后,他们又赶去了后方,督促忻州和太原府的轨道。
两个人都是跟韩冈打过交道。

韩冈当年任职开封府界提点,在修筑黄河金堤时,曾得到了黄怀信所打造的修城飞土梯、运土车。这本是用来帮助修筑东京城墙的机器,能很轻松的将筑城的黄土从城墙之下运送到城墙顶端。后来在河堤时又经过改装,加装了滑轮组,效率加倍提升。开封一段的河堤,能在半年之内修筑完成,飞土梯起了很大的作用。

而俞正也是在韩冈担任白马知县时认识的。在韩冈为流民营的饮用水卫生安全伤脑筋的时候,家学渊源的俞正配合深水井打造出了性能优良的提水风车,并设计了干净耐用的竹制饮水道。保证了流民营中的饮水安全。

他的曾祖俞皓技艺闻名天下,曾号称是鲁班再世。开宝寺旧木塔是其代表作。刚修起时向西北倾斜,当受到质问时,他回答说京师多西北风,现在倾斜,百年之后就会被吹正。后来一如其所言。当京城的人们开始担心开宝寺木塔接下来会因风倾斜的时候,木塔便被雷火烧毁,这让早已作古的俞皓又多了一层神秘的色彩。此外他的祖母俞氏也很有名气,作为俞皓的女儿撰写了通行于世的《木经》,为天下木制建筑和家具的圭臬,之后招赘了女婿上门,所以俞正才姓俞。

不过在那之后,韩冈很快就调任他处,与他们两人没打过什么交道了。但两人的名号还是不时传入耳中。两人在营造上的才干,让他们在这之后都参与过各大工矿和码头的轨道修筑,并不比沈括和李诫经验少多少。
