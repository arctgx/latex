\section{第46章 龙泉新硎试锋芒(五)}

【第二更,红票,收藏】

一番话说得投机,韩冈被王安石留下吃饭,吕惠卿、曾布和章惇也照惯例留了下来,加上王旁,总共六人。

王安石向以清廉著称,参知政事家的饭菜也没有什么特别,甚至不比张家、程家好到哪里。不过韩冈还是见识到了传说中王安石吃饭时的心不在焉,他的确只盯着面前的一盘菜在吃。而且王安石不拘小节,有些菜从筷子上落下,掉在衣服上,他也是拈起来就放进嘴里,在座的几人都见怪不怪,倒是韩冈吃惊不小。

一顿饭吃完,韩冈又重新坐到了王安石的书房外厅中。厅内已经点起了七八支蜡烛,大概是御赐之物,每一支蜡烛都有儿臂粗细,燃起来后,空气中还带着淡淡的香气。

比起饭前,厅中现在多了一个王旁,暂时不是说正事,王安石也不介意让自己的儿子一起过来聊聊天。说起来他的这位二儿子性格上有些阴沉,王安石还是希望王旁能多参加一些士人间的聚会,增长阅历,结交朋友的同时也可以改改性子。

坐下来,闲聊了几句。王安石问着:“王子纯的确有眼光,运气也不错,能在伏羌城遇到玉昆。只是王子纯他信来的不少,说得却不清不楚,不知是玉昆为何会摊上衙前役?又是为何会被人陷害?”

“……说起来也不算什么,”听见王安石问起自己的经历,早有准备的韩冈便沉声说着,“韩冈的经历,天下千百州县,每天都会发生。能如在下这样遇上贵人的却没几人……”

在王安石书房的外厅中,韩冈将自己从病愈后的遭遇和经历,一桩桩、一件件的娓娓道来。没有什么遗漏,但也无须夸张,平铺直叙的词句,已足以让在座诸人叹为观止。

其实,韩冈的这几个月来的遭遇,已经完全可以算是一个传奇。是个极精彩的故事,又是摆在眼前的事实。除了王旁,四名听众都是见多识广,但生长在和平安宁的皇宋腹地的士子们,即便是王安石、吕惠卿这样少年时便走遍四方寻师访友的读书人,也绝没有这般波澜起伏、危机处处,却又每每绝处逢生的人生经历。

王安石也不免为之惊叹。韩冈他被陷害,他被压迫,他被谋算,但最后,却是他站在数千人的尸体上放声大笑。如果只看韩冈背后的三份荐书,以及王韶所写的几封私信,任谁也不会知道他这一路走来有多少艰难险阻,又是怎样被他一步步的跨越过去!

难怪能得王韶如此看重!也难怪他能一下得到三份荐书!

韩冈不出意料的在王安石他们的眼中看到欣赏和赞叹。

塑造个人形象讲究技巧,韩冈在张戬、程颢面前温良恭俭,做出一副勤学好问的好学生模样,虽然他的确好学,但他所表现出来的性格,却与他的本心背道而驰。之所以这样做,因为韩冈明白,要接近程颢、张戬这些道学家,不把自己打扮成同类是不成的。

所以他把一身的锋芒收起,将果决的手段敛藏,最后出现在在张程二人面前,是一个好学、勤谨、肯上进、同时还有些才华,最重要的是为人正直守礼的韩玉昆。

但在王安石面前,那就不一样了。韩冈需要给王安石留下一个深刻的印象,张戬程颢面前的那种好孩子的形象是不成的。

他不介意说出在德惠坊军械库中杀人反栽的盘算,也不介意说明他在裴峡谷杀了两名陈举内应的决断,因为王韶每每拿来比拟韩冈的张乖崖,他杀人放火,灭了道左黑店一家老小的轶事,也是到处流传。

“若非是玉昆,换作是他人,即便是我处在玉昆的位置上,怕是会凶多吉少。”曾布叹着说道:“倒是子厚,应该能杀出一条路来。”

章惇摇摇头:“难说,我可没有玉昆的好身手。”

吕惠卿觉得两人都没说到点子上:“武艺倒是其次,智计亦是末节,关键是玉昆能下决断。在伏羌城,对向宝家奴的那一箭,射得的确好。”

“其实这些算不得什么,因为在下清楚,阴谋诡计从来是见不得光的,只要自己行得正站得直,理直气壮,便是鬼神难侵。”

韩冈说到这里,犹豫了一下。但立刻,眼神坚定起来,把准备已久的一番话,缓缓说了出口:“话说回来,也是同样的道理,青苗贷一事其实有个更简单的解决方法。不需添支俸禄,只要把事情摊开来说就可以了。韩相公、文相公,他们不是说青苗贷伤民吗?那就把他们家里放贷收息、残害百姓的事都曝出来。放在光天化日之下,让天下人看清他们的用心,好做个评判!”

韩冈轻轻笑着,微微眯起的双眼寒芒四射。入京后压抑许久的如剑如刀的锋锐性子,此时终于扬眉出鞘。

王安石前日称病不朝,请郡出外,那是无可奈何下的防守,像个女人一样对着三心二意的情郎说着有我没她。但韩冈的建议却是彻头彻尾、犀利果断的进攻。

依照朝堂惯例,玩着一些阴谋诡计,韩冈没这个本事,即便是前面加薪的计策,也不过是拾人牙慧。但他可以挥起大锤,照脑门直接来上一下。

简单,直接,而且有效。

龙泉三尺新磨,正要一试剑锋。

厅中一时静了下来,谁会想到韩冈突然间出了这个主意。王安石盯着韩冈的那对犀利锋锐的眉眼,突然发觉他对这名关西来的年轻人,了解得实在太肤浅了。想不到韩冈在谋算深沉的外衣下,藏着的竟然是锋锐如剑的性子。

章惇不掩激赏之色,曾布打了个哈哈,“这田籍户产可是不好查的。”

“何必要查田籍户产?!窦舜卿说一顷四十七亩时,可曾查过田籍户产?可有半分真凭实据?当然,窦舜卿是信口胡言,睁着眼睛说瞎话。但我们说得都是实话,文家、韩家,他们两家难道没有放贷收利之事?!只是数目多少的问题,差个一点,又有什么关系。只要激得他们上章自辩,那就足够了。”

韩冈一直以来其实都对变法派的畏首畏尾有些不以为然,既然已经得罪那么多人,何不干脆得罪到底?!看看商鞅是怎么做的,只是城门立木吗,他可没少杀人,顺便把太子的师傅都治了罪。如今还把对手留在朝中,这不是给自己添乱?富弼、韩琦是走了没错,但他们离开朝堂的原因,是因为他们在政事堂太久。新帝登基,他们这些元老重臣本就是要先出外的。

在韩冈看来,王安石实在太克制自己了【注1】。

如今都是看着反变法派向王安石身上一盆盆的泼着脏水,而王安石他们只是招架,为自己辩解,却少有对进行人身攻击的。当年庆历新政时,吕夷简是怎么对付范仲淹一党的?从欧阳修闺幕不修,到苏舜钦卖故纸公钱,再到攻击范仲淹结党,几桩事一起发动,便把范党一网打尽!

“再说韩稚圭的弹章。他说青苗贷不该贷给城里的坊廓户。凡事须正名,以青苗贷这个名字,贷给坊廓户是不对。可改个名字不就行了吗?把青苗贷改成利民低息贷款,韩琦之辈还能说什么?名正方能言顺,只听这个名字,就知道是为了救民水火的,而且没了青苗的局限,贷给城里的坊廓户也没了问题。同时明白指出天下的利息太高,朝廷是不得已而为之。”

“接下来韩、文、吕诸公还会有什么手段,在下不知道。但有一点可以确定,只要把他们私底下的一些心思暴露出来,他们不可能再去迷惑天子和世人!”

韩冈说得毫无顾忌,完全不在意自己的地位与他所攻击的韩琦、吕公著等人有多大的差距。

按道理说,韩冈一个微不足道的从九品选人,在朝中,不过是升载斗量之辈。煌煌神京,天下中心,这里并不是适合他的舞台,完全不够资格上去参与演出。上面的主角,是王安石、是司马光、是文彦博、是吕公著,也有身居千里之外,也能动摇京城舞台的,有富弼,有韩琦。即便是配角,也是吕惠卿、曾布、章惇、张戬、程颢之辈。如果一个最底层的官员自不量力的跳上去,被踢下来,跌个粉身碎骨,是最有可能的结局。

但是……韩冈就是不愿意在旁边看着热闹。他以一介布衣撬动秦州官场变局,如今已经能在王安石面前说上话,如何不能让朝堂为之动摇。那座光鲜亮丽的舞台,他暂时还不能站上去,但在幕后推波助澜,也不失一桩快事。所以他方才出谋划策,所以他现在兴风作浪。而且既然已经决定站在变法派这一边,韩冈自然不会再想看到王安石犹豫不决,最后走向记忆中的变法失败的命运!

可是王安石他们如今做得最多的就是辩解,因为王安石不愿意用上与自己的反对者同样的手段——他深知如此去做的后患。

一旦他们这么做了,牛李党争可是最好的前车之鉴。一旦变法派不再局限于就事论事,开始攻击反变法派的人品、策略、用心,那样……就是党争的开始。不再是因政策才划分出来的派别的争斗,而是党同伐异,不论对错,只论党籍。王安石暂时还不敢这么做。

但在韩冈看来,韩、文、司马等人可没这样的觉悟。他们不断攻击变法派的人品,攻击变法派的政策,攻击变法派的用心,好吧……只要跟新法挂上钩,没有一件事他们不攻击的。

党同伐异,不论是非,这不是党争是什么?

既然反变法派已经跟疯狗一样疯狂乱咬,宁可自己一身膻,也要把新法拉下马,那就该反咬回去。谁的身上都不干净,韩琦、文彦博都不是清白纯洁得跟刚出身的婴儿那样屁股干干净净的人物,韩琦在相州没少夺人田产,文彦博在仁宗朝勾结内宫的事也还没洗干净呢,在老家也是一样一身是冤债。

党争并非好事——这是对天子来说的。因为一旦党争开始,就必须分出个胜负,就像唐时的牛李党争,又或是庆历年间的吕范之争,非得将对手一网打尽不可。即便是天子,也无法置身事外,更不能像过去的一年里那样和着稀泥,玩什么祖传的‘异论相搅’,必须旗帜鲜明的选择一边。最后的结果,就是得到天子支持的一党,把所有的敌对党人,赶出京城,赶出朝堂——自然,在现阶段,只会是新党。

这些道理,王安石他们岂会不明白,在座的几位都是对历史比韩冈精通百倍的俊杰才士,何事不能看得通通透透。只是他们在朝中站得太久,牵连太多,投鼠忌器,不敢下手而已。

王安石他们即便是家中窜进一只老鼠,也会因为顾忌着周围全是易碎的瓷器,而任其啃着米缸里的存粮,但韩冈却不介意拿起官窑的雨过天青去砸蟑螂。

因为他是初来乍到,因为他关系全在秦州,因为他根本不在乎京城掀起多大的风浪——除了在座的五个人外,没人会相信是一个从九品拉开了党争大戏的戏幕,即便是日后传扬开来,韩冈只需一声冷笑,就能为自己洗个白白净净。

‘我只怕事情闹不大!’韩冈没说出口,但王安石他们都听明白了。

王安石轻轻摇头,曾布低头沉思,章惇面露微笑,王旁目瞪口呆,而吕惠卿则在心中暗骂着王韶不会带眼看人,

‘他哪里是张乖崖?……

……分明是贾文和!’

注1:翻看熙宁二年到熙宁五年这一段时期的史料,就能发现新党实在太好人了。史书上满篇都是旧党的攻击和弹劾,把附和变法的大臣说成是猪狗不如,主持变法的说成是奸佞小人,连王安石这样道德和人品都挑不出错来的人物,也有十条大罪和辩奸论等着他。而新党一派却少有如此激烈的弹劾,连攻击对手人品的情况都很少见,直到熙宁五年后,变法有了成果,才彻底的把旧党势力从东京城清除出去。

