\section{第28章 官近青云与天通(18)}

“万万不可。”

声震殿堂,不知多少人在同时开口,但其中并不包括韩冈。他根本就没动弹,因为跳出来的人太多了。

吕公著、蔡确、韩缜、薛向、章敦,几名执政全都出班,而下面的朝臣,侍制以上的也有三分之一站了出来。

开什么玩笑,这可是杀士大夫!若是偏鄙小臣倒也罢了,过去也不是没动过刀子,但侍制以上的重臣,开国以来,还找不到被诛杀的例子。

他们可是真正与天子共治天下的士大夫!王珪死不足论,但先例一开,日后谁能保证不杀到自己的头上?!

‘玩脱了吧……’

韩冈悠悠然的看着犹发着愣的司马光、张商英等人。

但话说回来,这个回答谁能想到?过去不管怎么喊打喊杀,最后也不过一个落职出外了事。谁能想到会来一个‘依卿所奏’?韩冈都不免被吓了一跳。

吕公著站出来后便领头开炮:“殿下,祖宗以来,慎刑慎杀。立国以来,未曾杀一士大夫!”

‘前面怎么不早说!’向皇后看都不看吕公著。

“殿下,王珪有罪当罚。司马光乱朝仪,御史台不能正,俱当治罪!”这是韩缜出来和稀泥。

‘罚两个月俸禄吗?’向皇后在帘后冷哼一声。

“王珪诚有罪,罪不容诛。但王珪乃天子素日所重,如今圣躬不安,遽然论死,或会惊动圣躬!”蔡确动之以情。

‘早念着官家的病,今天就不会有这一幕了。’皇后心中的火气渐渐上来了。

“殿下!”太常礼院的李清臣则维护法度:“杀宰相,岂可如杀一鸡犬?王珪有罪,但朝廷自有律条在,即便要论罪,亦当付有司详断!”

“那就是我的错了!”向皇后前面自知失言,所以只是腹诽,现在终于忍不住开口了,还好声音不大,但还是将宋用臣吓得魂不附体。

“王珪前日已上辞章,可见其已服罪。勿须再施以重刑。”御史中丞李定方才没有站出来支援下属,监察御史独立性很强,也不是他这个主官能控制得住的,现在倒方便他站出来。

向皇后差点咬碎了银牙,恙忿积于胸臆:“这是怪罪吾将弹章留中吗?!”

宋用臣直冒冷汗,幸好声音还是不算太大,要是和朝臣吵起来,那就更麻烦了。

“司马光凌迫君上,当付有司论罪!”章敦也说话了,只有他的目标是司马光,这让向皇后的心情平复了一点。

章敦才不会跟人争王珪有罪没罪。让地方稳定?那也简单得很,挑两个不长眼的发配去监盐茶酒税就是了。至于王珪和司马光,两边都赶出去就是了!

开罪了御史台,章敦一点都不在乎,他举荐起来的张商英、吕惠卿提拔的舒亶,都越来越不听话,甚至有反噬的迹象,走了才好。而让王珪安稳出外,也正好可以腾位子出来。王珪一走,肯定要提拔新的宰相,而且至少要有两名宰相来平衡局势。到时候空位子出来,自己向上走一步,去东府做参知政事是很有可能的。

形势一面倒的要保王珪。向皇后知道,如果再依照臣子之言改口的话,肯定是要让人笑了。但不管怎么说,现在肯定是杀不下去。

“司马卿,你说如何?!”向皇后勉强压下了心头的怒意,问着司马光。只要司马光和御史台给个台阶下,今天的事也就算了。

“当诛之!”

司马光硬邦邦的回道,毫不犹豫。现在他已经不可能改口。坚持到底还能说是嫉恶如仇的表现——反正王珪终究也不可能真的被杀,朝臣们也都明白究竟是怎么一回事,不至于引来士大夫们的仇怨——但若是临阵退缩,毕生积攒下来的声望可就要付之东流了。

“王珪当诛之!!”几名御史也是骑虎难下,只能硬挺司马光。从他们的角度来说,宁可被罚出朝堂,也要保住一个能够卷土重来的名声。

“是吗?”向皇后声音阴冷下来,手也紧紧攥着袖袍。

“圣人!圣人!千万不能啊!”宋用臣慌得汗水直流,急着在她耳边低声叫着。向皇后要是使起小性子,麻烦真的就大了。难道要入内通报天子来救人?!

“殿下,臣韩冈有言。”

旁观良久的韩冈,终于施施然站了出来。也让成了菜市口的文德殿,平静了下来。虽然他还不是宰执,但江湖地位已经到了。

王珪肯定是杀不了的,向皇后的话最终会被士大夫们给堵回去。但再吵下去,局面只会越来越坏,甚至能让司马光和御史台博个好名声。若是朝会成了刷声望的地点,没脸的肯定是向皇后。

韩冈当然要向皇后收回她前面的话。虽然会影响到她的声望,但之于向皇后,却是损伤不大。难道垂帘听政的人选还有别的选择吗?既然没有,又有什么好担心的?!

不过这个责任肯定要有人来承担。司马光和御史台必然要为他们的行为负责。至于王珪,算他运气了。

看到韩冈站了出来,向皇后的心情也稍稍平静了一点:“学士请讲!”

“今日之事,事在张商英、舒亶诸御史。臣工有罪,罪在御史台。”

韩冈的眼睛长到哪儿去了?!

向皇后当即被噎得气息一滞。当头跳出来的明明是司马光。领头搅乱朝会的难道不是司马十二?!众目睽睽之下,难道韩冈还想帮司马光把罪名推到御史们的头上?

过了半晌,向皇后方才压着心头气,开口问道:“御史台何罪?”

“奏劾无状!”韩冈一字一顿:“乌台劾王珪,弹章百十计,悉已传之朝野。臣只闻其中夺职、远窜、毁废等语,不闻一字涉及大辟!”

司马光的奏折并没有读完,到底有没有诛杀王珪这一条,韩冈不敢百分百的确定。但张商英等御史的弹章上,却可以肯定没有‘诛王珪、谢天下’这一条。

韩冈双目一扫张商英、舒亶等人,“御史论事自有规制。若是奏报民情,或可风闻。但弹劾臣僚,总得依法度行事。前日章疏言贬,今日殿上论诛,前后不一,奏劾无状!”

“话不是司马光先说的?!”向皇后觉得委屈,司马光是始作俑者,张商英、舒亶等人只是击鼓摇旗罢了。

皇后的抱怨,韩冈也愣了一下,立刻道:“臣闻朝廷选萃,必得清正而有风望者为御史。而张商英、舒亶今为御史,却闻风改辞,不闻清正在何处?司马宫师居洛阳,穴地修书,让人闻之不免惊骇。今日之言,未必无因。而张商英、舒亶等人又有何缘由?”

好了,韩冈的打算,这下全都明白了。

虽然是在说御史台,但谁都看得出来,他的刀子更多的还是落在了司马光的身上。给司马光安的罪名是泄愤——记得司马君实在洛阳待了多少年吗?他今天哪里是恨王珪,他是恨王安石啊!

事君惟忠,而司马光却在国事中掺入私心,这是品性问题。而且让司马光在洛阳修书的,可是还在福宁殿中的天子……这分明是怨望!

怨望!做臣子的,哪个敢让这两个字挨身?

韩冈的攻击不可谓不狠毒,殿中大部分人都这么想着,皇后的心情也一下好了起来。

司马光则被怒火烧红了双眼:“雷霆雨露,皆是天恩,此理臣岂不知?臣劾王珪,只为天下、朝堂,何为怨望?!”

只有深悉医理又了解韩冈的苏颂却皱眉看着韩冈,他觉得韩冈的话似乎还有一层深意。

苏颂方才同样是站出来阻止皇后乱命的一个,但他也只用不可杀士大夫来谏阻皇后,并不像韩冈和章敦直接指责司马光和御史们。

韩冈是《本草纲目》的主编者,他说的话从医理的角度来理解则更为确切。司马光有病,而御史们无病。司马光是犯糊涂,而御史们是心怀叵测。

而韩冈接下来的话,也证明了苏颂的猜测。

“韩冈非是在说宫师怨望。”

韩冈语气平和,心中却是叹息,有些事他不想做得太过分,可既然入了朝堂,就别想干净得起来。面面俱到既不可能,那就得党同伐异。纵使面对的是《资治通鉴》的主编司马光,只要他还想毁掉新法和气学共有的根基,那就没有人情可讲。

“学士此言又是何意?”向皇后在帘后听得更加糊涂。

“须知阴淫寒疾,阳淫热疾。此乃是疾作之故,非是宫师的本心。”

殿上顿时一片哗然。纵使没听明白的向皇后,也在管勾御药院的宋用臣匆匆解释下,明了了韩冈话中之意。

‘阴淫寒疾,阳淫热疾’出自秦医和的六气六疾论——气有阴阳风雨晦明,疾有寒热末腹惑心,六气淫则六疾生。

韩冈说‘阴淫寒疾,阳淫热疾’,但任谁都知道,韩冈决不是在说司马光有寒热之症。医者说话,不可能太直白。在‘阴淫寒疾,阳淫热疾’之后,六气六疾论的剩下四句是‘风淫末疾,雨淫腹疾,晦淫惑疾,明淫.心疾。’

韩冈的本意自然是秦医和的六气六疾之论中的‘晦淫惑疾,明淫.心疾’这两句。二大王是心疾,而司马光不是惑疾就是心疾——反正心疾、惑疾都是神智有毛病,是在指责司马光的神智有问题——因新法不得不在洛阳修书十余年,郁愤在心,以至疾作。

虽然这在性质上,比怨望要好一点,但只要韩冈的话被人采信了,一个神智有问题的太子太师,便不可能再立足于朝堂!

司马光眼中一片血红,不意昨日还在席上端茶倒酒的后生晚辈阴狠至此!

但无论如何,司马光掘了地窖在地下修书之事,殿上人人知晓。行事有悖于常理,若不是怨望,那就是有病,最轻的说法,也是人老悖晦!

总得认一个吧!

