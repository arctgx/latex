\section{第46章 正言意堂堂(上)}

上元节时,万户悬灯。

一盏盏灯笼,悬于大街小巷之中,仿佛将天上的群星拉到了地面。

大内之前的御街上,一座座造型各异的灯山一字排开。展示在宣德门之前。

而属于各府院监司、皇亲贵胄的彩棚幕次,也同样搭在御街之上。帐篷和彩棚上,同样挂满了灯笼。

正所谓‘天碧银河欲下来.月华如水照楼台’,御街和东西大街,却像是两条银河纵横交织在一起。

如果从高处下望,整座东京城就是一座灯的海洋。

宣德门城楼上,赵顼穿着红衣小帽,受过群臣拜贺之后,带着后宫嫔妃坐于一处,饮酒观灯。而宰执和翰林学士们也在城楼上,同享天子钦赐的恩泽。

受了天子甘霖沐泽,做臣子的便要为此而作诗作赋,以谢天恩,并记今日之事。

喝过天子赐下的御酒,重臣们便分韵即席赋诗。好坏不拘,只要应个景就行。当然也不是所有人都能立刻作出诗来,回到家中苦思冥想出来再呈上也不是不可以,只是会被人笑罢了。

王珪才思敏捷,很快就将御制诗做了出来。金玉满堂、符合节日气氛的富贵诗正是他的擅长,虽然备受人笑,甚至他的兄长都戏称他的诗作是‘至宝丹’,但毕竟应时应景,在宫中很受欢迎。

吕惠卿运气不佳,拈了险僻的韵字。不过他的才气在重臣中算是第一流的,只是少费思量,也敷衍了一篇出来。只是他心中有事,写出来后,只确定了有没有犯讳,便没有再多修改。

他跟韩冈之间肯定是闹翻了。

吕惠卿听了曾孝宽说,韩冈在看到灯船的时候是笑着,但他心头怒火有多旺,吕惠卿也能猜得出来。

都是白彰做的好事啊!虽然他直到站在了曾孝宽的面前,得到提醒后,好像才反应过来,叫起了撞天屈。不过其中真伪如何,却说不清楚。曾孝宽回头就说了,“白彰不能用了。”

主持灯山打造的白彰究竟是真糊涂,还是假糊涂,到底有没有到下面的蒙蔽,吕惠卿无从分辨。只是有一点可以肯定,韩冈肯定是恨透了自己——白彰怎么算都是他吕惠卿的人——如果互相交换位置,吕惠卿肯定也会这么判断。

究竟是谁!吕惠卿眯起眼睛,扫着在座的同僚,到底是谁下了黑手?将他和韩冈都给害了!

就在吕惠卿观察着十几位宰辅和学士,他们也都各自完成了今天例行的应制诗。几个宦官将诗篇一张张的贴到了壁上,用灯笼照着。赵顼走过去,一首首看了一遍,随手圈出了头名——又是王珪第一。

赏了今年的上元诗赋,喝了一巡酒,赵顼在嫔妃们的陪伴下,又向下看着满城的灯火。

“官家,那是铁船吧?”

附在天子耳畔的绝色佳丽,遥遥指着城下的一座灯山的正是最近新得宠的朱才人。除了一开始在宣德门上接受百姓拜礼时,向皇后伴在赵顼身侧近处。其余时候,反倒是朱才人靠得天子近些。

顺着春葱一般的纤纤玉指,赵顼望着斜下方、略远处的那艘灯船,很有些惊讶,那的确是军器监灯山的位置所在。他没有想到韩冈竟然这般有底气,在上元节的时候,拿着铁船当作了灯山式样摆了出来,

看着这艘周身流光溢彩的铁船,对韩冈甚为了解的赵顼,知道多半很快就能看到真正的铁船在汴水上航行了。只是赵顼觉得有一点让他纳闷,“灯山不是冬至之后就开始打造吗?为何军器监的灯山会是铁船?”

天子身后的几个高品内侍互相看了看,提举皇城司的石得一便上前一步,“军器监的灯山原本是并不是船型,不过在年节时垮塌了下来,难以修复。而后军器监才不得不用了六天的时间,将新灯山给赶制出来。”

“难怪!”赵顼笑了一声。看来不是韩冈为了彰显自己,而故意弃了原先的灯山,而又重新打造的这座灯船,“这样就好。这样就好。”

冯京笑着,略略提高了音量:“陛下,韩冈既然能把铁船亮出来,肯定是有把握了,想必很快就能看到实物。”

赵顼兴致高昂的点着头:“朕也是这么想的。”

吕惠卿终于知道到底是谁下的黑手了,几乎要咬碎牙齿,冯京这是将他和韩冈都害了进去。

看到军器监的灯山,王珪是紧皱眉头,韩绛是眉头紧皱,会有这种表情,全都是因为他们对军器监的内情并不了解,以为韩冈对打造铁船已经有了把握。

吕惠卿回头再看看枢密院的正使、副使三人。吴充的表情与韩绛、王珪相似。而置身事外的蔡挺,与韩冈关系紧密的王韶,两人无一例外都在欣喜中透着深深的疑惑。他们的神色中,都能看得出来他们也不了解今次的真相。

既然其他几位宰辅都以为铁船即将功成,那么唯一一位笑意盈盈的冯京,自然就是仅有的可能。

就在吕惠卿推断着真凶是何人的时候,走到天子身后的冯京说道:“其余各家的灯山,不过是好看而已,别无他用。可军器监的这艘灯船,代表的却是军国之器,今夜评灯,军器监的灯船当是魁首。”

上元节时摆出的灯山数十近百,这么多的彩灯,肯定都要分个高下,免不了要排个座次。赵顼略一沉吟,笑得更为开怀“……的确是这样。今年灯山的头名,也不用等到正月十八了,今天就可以定下。”

吕惠卿暗叹了一声,冯京这是在给韩冈的棺材上钉钉子!

官场上的规矩就是这般。

不论是要做什么事,只要没有上报,最后即便没有成功,也没有什么关系。可一旦正式报与上知,在文牍档案上留下了文字,那就再难改易。若是没有成功,就必然会受到惩罚。

之前,铁船一事尽管在东京城中——甚至可能在北京大名、西京洛阳、南京应天——都已经传得沸沸扬扬。只要韩冈没开口——没在公开场合、没在正式场合开口,那都不算数。只要他不主动出手去做,任谁都催不了,也逼不了。

可现下军器监已经将铁船搬了出来,等于就是对东京城的百万军民正式宣布:我们判军器监的韩舍人,要打造铁船了。

只要来观灯的人——无论天子、群臣、还是百姓,都从中听到了这条宣言。

一座红褐色的船型灯山,就将韩冈摆在架子上烤!

“陛下。”韩绛忽然出声,叫住了被冯京煽动得正在兴头上的赵顼,“韩冈不请于上命,便以铁船饰为灯山。此行未免有失轻佻,也太好大喜功了一点!”

“不然,区区一座竹木为骨的彩灯灯山,何须请于上命?”冯京状似不屑的反驳着,“下面的灯山,有卧佛、有罗汉、还有麒麟、彩凤,难道各家也曾奏请陛下不成?”

韩绛眉头一皱,又欲强辩,但赵顼已经很不痛快的板下了脸。

明明是节庆,还说这些败人兴的话。不就是韩冈顶了中书都检正的推荐吗,还记挂在心上,宰相气度一点都没有——天子的表情说明了一切。

冯京微微翘了唇角,似乎很欣赏天子对韩绛的态度。

在旁瞧见冯京得意的眼神,吕惠卿更是恨得咬牙切齿。他不惧与韩冈翻脸,但被人陷害,那他无法忍受了。因为一座灯山与韩冈交恶,更是无妄之灾。韩冈的手段心术,吕惠卿都要暗暗提防,更不用说他背后的王安石——韩冈再怎么不驯,也是一直帮着王安石的好女婿。

“陛下,既是如此,不如诏韩冈上来询问,看看他到底只是造灯山,还是要打算给铁船张声势!”

王珪似乎是在敲太平鼓,但他话中的意思却是附和着韩绛,‘张声势’三个字可不是好评价。

赵顼想了想,就准备点头。韩冈没有伴驾的资格,但如果天子特旨,却是无妨。

王韶已经看出不对劲,他耳朵不聋,眼睛不瞎,不论韩绛、冯京和王珪,都没有安着好心。同时更是嗅到一丝让人感觉不妙的味道。

现在灯船已经亮了出来,东京城上下都在盼着看到真正的铁船,韩冈怎么说都难以洗脱,之后若是难以成事,不但名望大损,还要因为妄报欺君而受到惩罚。

他站了出来:“陛下,不过一座灯船而已,就将一小臣找来询问,未免有失轻重。此事待韩冈自请上表再议不迟。”

赵顼脸色阴沉了下来。他知道韩冈跟王韶的关系,王韶不可能跟韩冈过不去。既然如此,他的宰相和参政的话中必然有什么问题。

赵顼无意多想其中缘由,只是觉得他的宰辅们上元节时还在勾心斗角,不让他得个清静,做事未免也太过火了一点。这异论相搅,搅得朝堂上鸡犬不宁,可不是好事。

他的视线移转,转到了一直没有吭声的另一位参知政事身上:“吕卿,你看如何?”

吕惠卿略作犹豫,“……臣以为,陛下现在招韩冈觐见也无妨。臣也很想早点知道到底铁船能不能成事!”

