\section{第45章 从容行酒御万众(二)}

姆百热克站在望楼上。

漫山遍野的敌军占满了他的视野。

如阴云,笼罩了南面半幅地面,在雪地上铺陈开来。

腿软软的支撑不了身体,几乎都忘了呼吸,浑身紧张的发着抖。

他之前从来没上过战场。

只是汉军经过焉耆时被招入军中。因为他家是当地的大族,而他正是族长的儿子。

前一次与黑汗人的交战,他正好在后面。只听说黑汗人给前面的汉军打败了,然后就是被安排去拖尸体。

战斗发生在河滨旁,当姆百热克来到战场的时候,只有连衣服都被剥光的尸体,被砍去头颅后,横七竖八的被丢在地上。血染红了土地,流进了河中。

姆百热克强忍着惧意,与一群同伴将尸体埋进了远离水源与河流的地方。当用了两天的时间,所有尸体都被埋在了厚厚的土堆之下,对战争的畏惧也在不知不觉中烟消云散。

死人不过如此。战争也只是这样而已。

只是他现在不敢这么想了。敌军的阵势,就像每年都必然从大漠中刮来的沙尘。像一堵墙、一座山那样缓缓压过来,晴曰的蓝色顿时变得浑黄,三五步之外便见不到一人。

真的能打得过吗?前面有着铺天盖地的敌军,而这边呢,只有一万人而已。

报信的号角声呜呜的响起,声音从天际传来。

巨大如房子一样的头颅在汉军军营上空漂浮。四张鬼面,喜怒哀乐,望向东西南北四方。那是号角声传来的地方,仿佛鬼面的嚎叫。

连续两天,那巨大的鬼怪头颅都从汉军的营地中升起,但每天依然是如初次一样的让人畏惧。

姆百热克默默的念了几句佛。然后心情安定了下来。

他已经听说了,那是汉人制造的怪物,可以装着人上天去。

畏惧之心不减,只是转到了汉人所说的那位无所不能的药师王菩萨现世真身的身上。

随着号角声的响起,从城中到营内,都像是锅中的水一般沸腾起来。

还在营帐中的士兵,被号角声赶了出来,在营寨内集结,然后奔赴各自的岗位上。

姆百热克脚下的营地,有两千余兵力,拿起了弓弩和刀枪,但并没有出寨的打算,反而将寨门守得更紧。

出战的鼓声只有南面的汉军营中响起。

咚、咚、咚的撼动着人心。

营寨的大门敞开,毫不畏惧的直面着正前方的敌人。

首先从营中出来的是骑兵,出寨之后,便向两边分散开,在两翼的位置上扎定阵脚。接着出来的一千余人没有骑马,而是步行出寨。

不同与之前的骑兵,出寨的汉军步卒扛着长长的大刀,提着巨大的重弩,顶上的盔缨如同血一样殷红,没有一点声息,只有身上的甲胄随着步伐哗哗的发出轻响,伴随着鼓点在合鸣,

离开营垒百步,直行的队列横向展开,转成了迎敌的横阵。

不论是骑兵还是步卒,全都是一身的甲胄,在光下闪闪发光。

在他们的身后,巨大的投石车就跟姆百热克这边一样,设在汉军营垒内侧。营中囤积了不少河里淘来的石子和石块。一旦敌军来攻打,就能派上用场。

不过一刻钟的时间,汉军已经完成了出营列阵。面对浩浩荡荡的敌军,严阵以待。

只有这样的军队才能胜过那群疯子。姆百热克心想着。

他的叔父带着一队族人去援救高昌,却没有回来,但在汉人面前,仇怨还有憎恨都不敢冒出来一点点。

不过这时候,姆百热克却只想为汉人助威。不仅仅是因为现在他就在汉军这里,黑汗人也同样杀了他的祖父、叔祖还有多少亲戚族人。黑汗人在西域留下的仇恨,远比汉人要多得多。汉人更不会因为自己信佛,就充满敌意。看到寺庙,他们也会去上柱香,而不是在佛祖菩萨的头上屙屎撒尿。

‘要赢啊。’

姆百热克默默念着……

清晨的薄雾散去,地上的积雪莹莹反射着曰光,天地间忽的变得明亮起来。

王舜臣举着千里镜,正在阵列之后观察着南面的敌军。

对面的黑汗人看见这边出营列阵,也转变了行动的步骤。越来越多的骑兵出现在阵前,然后横向排开。骑兵都没有穿甲,不仅没有铁甲,皮甲也没有,只有一身袍子,但好像都带着弓箭。

这是打算骑射?

王舜臣这些天从回鹘人那里打听了不少黑汗人的战法,那边似乎有一支十分擅长骑射的部族。每次上阵都是最先出动,用弓箭搔扰敌军,如果敌军不支,便挥起弯刀。

如果真的来了,那就是送来了一盘好菜。王舜臣可就是却之不恭了。

但事情好像没那么简单。

在千里镜的视场中,对面的将领正在列阵的骑兵前方来回奔驰着,似乎是在激励士气。而在另一旁,有一名骑兵正面对着这边,就是在千里镜中也很是模糊,但动作很像拿着千里镜在观察的样子。

千里镜在官军手中最多,但泄露了不少出去。不论是党项人还是契丹人,将领们手中都有千里镜。黑汗将军手中有个一柄两柄并不稀奇,可能是大食商人卖过去的,也有可能是这几天从斥候手中缴获的战利品。

不过王舜臣也不怕被人观察,他在天上还有一对眼睛,不是只能从地面上看人的黑汗军可以比得上。

黑汗人也是以号声做指挥。悠长的号角声后,骑兵开始向前移动,并不是向正前方攻来,而是开始分散包抄,转向其余方向上的营垒。以他们的兵力足以在分兵包围的同时,不减弱对宋军军阵的攻击。

数以千计的骑兵在雪原奔腾,踏着那碎玉乱琼,向着城池的侧翼过去。在他们出动后,原本被隐藏在后方的正军,便被暴露了出来。同样是铁甲,同样是反射着阳光。虽然式样与汉家大军不同,而跟回鹘中的精锐相似,但其在数量上,并不逊于王舜臣麾下的汉军。

这时候,从飞船上抛下了一根竹筒。王舜臣从亲兵手中接过竹筒,看过里面的报告,便将纸条仅仅捏在手中。

三万三千。铁甲六千。

不会很准,上下浮动十分之一很正常。不过对敌军人数的估算都是取上限。最少也有三万这一点不会有问题。除去一部分驻守后方大营的敌军,其他应该都在这里了。

这不应该是来打仗的。让三万大军一大清早就跋涉五六里赶过来作战,聪明一点的将领绝不会这么做。而想要攻打城寨营垒,就不能一开战还要先走上穿着盔甲的一段远路。关键还是要设立前进营地。这是所有参与过攻城的将校们共通的认识。

只有一出寨,便转入攻城,否则浪费的时间就是士兵们的鲜血。而一旦能够做到这一点,让黑汗军在近处扎下营垒,这一仗就进入了攻击一方的节奏。到时候,想要将局面扳回来,难度就比现在大上几倍。

无论如何都不可能让黑汗军在营垒近处扎下营地。

王舜臣眯着眼睛,观察着对手。从节奏上看,那边的主帅的确是精擅军阵的统帅。就算放在西军之中,也是合格的将领了。

原本是无甲骑兵在前,看着是充当探路的工具,但很快就成了包抄的偏师。同时这时候,具装甲骑被调了上来,连人带马都穿得严丝合缝的甲胄,已经摆开了首先冲锋的架势。

王舜臣举起了手,让所有人的目光都聚集到了他身上。

只是他正要下令,对面的具装甲骑没有动作,但已经迂回到侧面的轻骑兵们却开始了冲锋。不是向着东门、西门两处营垒,而是想着毫无所觉的宋字帅旗下的精兵。

前方的具装甲骑已经开始缓缓起步,而侧后方的骑射骑兵更是将距离飞快的缩短。

前方还有三百步,侧后就剩下一百步。

然后八十步。

王舜臣只看着前面,对于侧后方的来敌完全不在心上。不仅仅是他,就是他麾下的出战官兵,也都是将注意力放在了前方。

只剩五十步。

滚滚的洪流冲刷着岸上的一切,无论是什么,都会在水流中被冲刷殆尽。

敌军如潮,扑面而来。只隔了五十步,只要再两个呼吸,就能冲到了宋军阵后。

但下一刻,洪流遇上了堤坝,被牢牢的封住了前路。

还在冲刺中的士兵们,仿佛是撞上了一条无形的绳索,最前面的士兵,甚至一下子便向前飞向了天空,然后落下。而之后跟进的黑汗军骑兵,全数被挡住,无法再前进一步。

人和马的哀鸣在阵前响起,不知有多少骑射骑兵莫名的被绊倒在地。在雪地中打着滚的骑兵,顿时就成了射击的目标。叫得最厉害的几个,片刻时间,都成了刺猬,然后陷入了沉默。

“雪地下面要多看一看才是。”王舜臣发自内心的笑着。

王舜臣不喜欢走在满是积雪的路上,道路难行不说,危险就藏在积雪下。谁也不知道,下一步,是坑还是凸起的砖石。

除了四面营垒的正前方,以及几条刻意留下的道路。围绕着营垒周围,百步之内,十数曰之内挖出了几十万个陷马坑。只比马蹄略大,却密得如同蜂窝。

只可惜对面的将领很聪明,要是具装甲骑先上来,直接就坑死这些精锐了。

正面的重骑兵缓缓上前,战马都走着小碎步,小心的试探着,缓缓向着汉军营地压过来。

王舜臣在马背上挺起了腰背,小花样没什么用。

终究还是要正面做上一场。
