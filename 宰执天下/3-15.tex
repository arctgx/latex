\section{第七章 儒统渊源远(上)}

十一月的洛阳已是草木凋零。前日的薄雪已经化了,但气温便因此而又冷了三分。

清晨的时候,程府门外,行人往来之声不绝于耳。程家非富户,安身在普通人家混居的厢坊中,不比城北富弼等重臣所在的厢坊清净。

程颢此时早已起来,向父母问安之后,就在院中慢慢踱着步子,作为日常养身的功课。他的儿女,也一个个过来,先向父亲行礼,而后,又进了里屋,跟祖父母请安——程家是大儒之家,礼法上的规矩一向恪守,子弟们也是不敢有任何疏忽。

去年程颢尚为镇宁军判官。但今年年初,父亲程珦从四川任官回乡,自请致仕。老父年岁已高,又常年在外任官。弟弟程颐放弃进学,一直随着老父四处迁移。现在父亲回来了,他这个做长子的,也该尽一尽孝道。请了一个近乡的差遣,以便归乡奉养父母,究研天地道理,教书育人,官职高低倒也不放在心上了。

只是担任了西京监竹木务这个差事,就让爱吃竹笋的女儿受了委屈:“阿爹监竹木务,什么都好,就是家里没笋子吃了。”

女儿娇憨的说话,让程颢呵呵笑着,“等明日让阿娘卖给你。”

程鄂娘摇摇头:“不要了……等阿爹卸了任,再买来好了。”

“说的对……行事自当如此。总不能像那些贪官污吏,一分归了公府,两分入了家门!”

程颐从里屋里出来。他就算在家中,也是衣装俨然,气貌严重。跟程颢有七八分相似的相貌,就是因为他这种始终严肃的表情,而不会让人错认是永远带着温和笑容的程颢。

程鄂娘见到叔父出来,也立刻上前请安问好。

程颐对这个侄女很疼爱。十三岁的女孩子,相貌无可挑剔,礼数比那些士子还要出色。小小年纪就甚有见识,性格也温婉。在家中见亲戚,不论贫富,都能一体待之。在他看来,在女子的德行上已是无可挑剔。但程颐点头作为回礼时,仍是不假言笑。

程家的女儿一向受祖父祖母疼爱,行了礼后进了正屋。

程颢则是照着习惯在院子中走着圈子,走了两圈之后,忽然问着弟弟道:“对了,前日横渠表叔的信函可曾看了?”

“看了。”程颐点了点头,笔直的双眉却是皱了起来。

程颢微微而笑:“表叔一向说着太虚无形、气之本体,想不到今日也说起了格物致知的道理……”

程颐心头纳闷的就是这一点,格物致知可是他一向提倡的观点,什么时候张载也转向了,而且转得有些让人摸不着头脑:“表叔的《订顽》一篇做得是极好的。明理一而分殊,发前圣之所未发。可与孟子性善养气之论同功,孟子千载以下,未曾见也。可格物致知之说,为何《钉顽》《砭愚》两篇中未曾多言?这一变,虽然其理可究,其源可寻,但总是觉得有些突兀。难道真的是如表叔所说,受到学生的启发不成?”

“‘未济,男之穷也’,这一条释义又是从何而来?”程颢反问着。

程颐为之哑然。

两年前,他随父亲程珦转任至成都。街边偶逢一正读着易经的桶匠,不知怎么就聊了起来。别的倒也罢了,唯独‘未济,男之穷也’这一条,桶匠却解说得发人深省,一句‘三阳皆失位’让程颐茅塞顿开。后来他给亲友写信,每每提及此事,皆叹世间隐士多有,只是不得人知。后来他撰写《易传》,关于这一条的注释,就是桶匠的原话。

程颢看着辩倒了弟弟,也没有得意的心思。他慢慢的在院中踱着步子:“道理说到难通处,往往会归于虚玄。魏晋耽于清玄,唐人崇于释老,莫不如此。但清玄释老之说,最畏的就是以实证之。若真能如表叔信中所言,格尽万物之理,释老之说,当溃不成军……二哥,这难道不是你我的本意吗?”

韩冈与张载书信往来,在信上所说的,只是韩冈想要阐述的观点的冰山一角而已,但张载已经由此阐发而开。程颢、程颐再一看张载的书信,就已经能推究出这套理论的作用。他们都是当世大儒,这样的理论如果能达到圆融通达的完美境界,将对儒学起到什么样的作用,那是一眼就能看得出来。

同说天理,两家学派各有不同,在亲戚的交流中,不免互相吸取对方的见解。‘但吾学虽有受,但天理二字却是自家体贴出来。’程颢对自家的学说有着充分的自信,对正确观点旁引博证,倒也没有门户之见,反而更赞起了韩冈,

“这两年,玉昆因着边功,已是名动关中。想不到他在学问上,却也一点也没耽搁。”

当年韩冈上京时,程颢就在韩冈那里听到了几句以数达理的说辞,只是当年韩冈自己都没有成型的理论,程颢想了几日后,也只能将之当成年轻人别处一格的见解。但现在看来,韩冈已经在他自创的道路上行走了。

韩冈名气的确是越来越大,洛阳这边,都经常能听到他的一些事迹。可韩冈身为儒门弟子,却跟早死了几百年的孙思邈扯不清关系,以鬼神之说愚弄世人,岂是正人所为?还有他曾在程颢面前明言支持新法,又跟京中名妓牵扯不清,这一桩桩一件件,都让程颐很不喜欢,他摇着头:“此子非是我辈中人!”

“也不尽然。”

程颢倒是很欣赏韩冈。

当年韩冈上京,也曾逐日上门聆听教诲,算是他的半个弟子。如今声名更盛,除了些少年人的风流韵事外,却也没听说还有什么恶行。关西军中人人感其恩德,疗养院之事,绝对当得起一个仁字。至于药王弟子,世间流言而已,韩冈当年都当笑话跟自己提起过。程颢知道,世间愚夫愚妇,往往都喜欢这样的奇闻异事,就算全力去辟谣,都不会有结果。他怎么会放在心上?

而且韩冈的人品,让程颢为之激赏。“韩冈这两年立功甚多,其得到的恩赏,大半都奉予表叔。横渠书院,还有横渠镇上的井田,多得其力。为人饮水思源,其本心可知。”

听着程颢所言,程颐不知不觉的点起了头。能有韩冈这样的弟子,其实他也有些羡慕张载。自家的门下,现在还没有一个能光大门楣的弟子出现,而张载门下,已经出现好几个了。

程颐挺直了腰背:“表叔在横渠教书育人,如今已见其功。时不我待,等明年开春,我就去嵩阳书院长住。虽非门派之争,但儒门道统正流,不能轻易与人!”

程颢默默点头。非是他也有着争强好胜之心,他可以借鉴和吸取其他学派的观点和长处,但儒门道统,却正如程颐所说,不能轻易与人。

如今各家学派如百花齐放,世人难以穷尽。

王安石旧年以《淮南杂说》名世,英宗年间又在金陵教书育人,世人目之为淮南学派。随着王安石成为宰相,变举试,修庠序,一整套举措下来,他的学说已经遍传天下。等到天下的州学、县学都以王学为课本,淮南学派必然会在士林之中成为主流。

盱江李觏,虽然已经去世十多年,但他的学说依然在江南一带流传。‘治国之实,必本于财用’,王安石新法之本源,便来自于此。不论是王安石,还是张载,又或是二程本人,对他的观点都有借鉴和引用。

在横渠镇中教学的张载,有别于中原各家,文武之道从不偏废。随着几个弟子逐步崭露头角,他的名望渐渐也起来了。如蓝田吕家的三兄弟,如在平定广锐之乱上立了殊勋的游师雄,再如名满关中的韩冈,都是其中的佼佼者。听说如今在京中为龙神卫四厢都指挥使的种谔,他家中也有子弟拜在张载门下。

至于近在身边的司马光,自到了洛阳,司掌西京御史台后,就不再过问朝事,给人写信,落款都是‘迂叟’。前日还听说他明年准备买地置园,连名字都事先起好了——唤作‘独乐园’。也不知是不是园成之后,就闭门不问外事,一心修他的《资治通鉴》。

同样也在洛阳的邵雍,近来正忙着在他的安乐窝中,编纂《皇极经世书》。皇极经世,以易为宗,以象数为本,推究天人演化之道。二程本就是深通易学,释《易》为义理,而邵雍则是偏于象数之学,再偏下去,那就是往卜算之道上走了。在二程看来,已经走入了歧途。

王安石,李觏,张载,程颢程颐,都是推崇韩愈的关键,崇奉孟子,自承道统依此而来。而扬弃了此前流行于世的荀况、扬雄两人的学说。可各家继承自思孟学派的源头,阐发出的道理却是各不相同。

究竟是哪一家谁能更近大道一步?

程颢在院中慢慢的踱着步子,程颐端坐于石墩之上,一时之间,两人都失去了言语。

