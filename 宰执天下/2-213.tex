\section{第32章 吴钩终用笑冯唐(20)}

已是麦熟时节,田间麦浪翻腾,眼见着丰收在即。在田间从事农活的人们,正掘开阡陌,为麦地交上最后一遍水。

在陇西县城外新近开辟出的几条渠道,引得是左近山间汇入渭河的支流,灌溉起城外上百顷田地。这是韩冈离开前与王韶、高遵裕一起定下的规划,没想到已经成了现实

王韶见着韩冈注意着流过道边的水渠,便道:“自从古渭升军之后,有了人力,开辟渠道就方便多了,才一个月功夫,就开了总计三十里长的河渠。现在人手更多,今年一年还能开辟出更多的灌溉渠道。令尊在其间,给了不少的指点,等收获后,安抚司会向上为令尊请功。”

韩冈恭声谢过王韶。但在前段时间,收到的章惇写给他的私信中,已经提到了赠官的消息。韩冈进城拜见父母时,并没有将此事说出来,准备给韩千六留一个惊喜。

战战兢兢的周南,在眉开眼笑的韩阿李面前,终于放下了心来。而韩冈看着严素心和韩云娘泛红的眼圈和幽怨的眼神,心道接下来的一段时间,夜里都要辛苦了。

重新上手的政事,比起宣抚司中的庶务简单了许多,让韩冈处理起来轻松愉快。

等到田间开镰的时候,陕西诸路高层的人事安排的最终结果终于传来了。

首先是秦凤路,沈起正式被任命为秦州知州,而不是此前的暂代;经略使、都总管两个兼职,理所当然的也同时转正。王韶和高遵裕对此都不是很关心,如今已经不是一年前的情况,秦州知州现在也压不倒缘边安抚司的声音。

同样暂代要职的张守约,也终于升任了他梦寐以求的秦凤路副都总管。在军中熬了几十年,如今成了高阶将领中的一员,韩冈也为曾经举荐过他的张守约而感到高兴。

被替换的郭逵自然还是留在京兆府稳定关中,而副总管燕达,则在结束了招捉使的临时差遣后,被调回到鄜延路,接替种谔留下来的空缺——鄜延路兵马副都总管。

卸职后的种谔去了京中,担任起龙神卫四厢都指挥使,统领上四军中的龙卫、神卫二军。虽说这是三衙管军中最低的一个职位,可毕竟还是统帅天下百万大军的主将之一,非功臣宿将不可任职。种谔得此一官,可谓是超迁。从此以后,他就是可以光明正大的被称为太尉的大帅了。而不是像他的父亲种世衡,只是在民间有个太尉的称呼。

横山攻略,本就是由种谔倡导并实际主持。虽然以失败而告终,但朝堂上都认为他只是运气不好,非战之罪。在今次参战的诸将之中,种谔是唯一没有晋升本官官阶、得到赏赐的一人,不过在横山攻略失败后,依然还要让他去京中镀一层金,可见天子对他的期望还是很高。

直接领兵参与了横山战事的两名副都总管中的另外一人——环庆路副都总管张玉,功勋亦著,尤其是在罗兀城退军的过程中,表现尤为出色,因此本官被升为正任官中的团练使,已经武臣中顶尖的贵官中的一员。

只是张玉并没有像种谔那样被调入京中,而是顶替了在广锐军叛乱时,颟邗无能、措置不当的庆帅王广渊,担任庆州知州、环庆路经略安抚使兼兵马都总管。以他已经是宿将的身份,成为一路统帅,可以看得出天子和朝堂已经把他和郭逵一般,当作了边地的定海神针来对待。

五个经略安抚使路,现在已经有两个是由武将来担任主帅。郭逵在永兴军路,张玉在环庆路,虽然这是庆州兵变后,不得已而为之的举措,但这也是真宗朝以来极少有的情况,想来也是长久不了。过个一年半载,多半朝廷就会忍不住了,改让文官来取代他们。只是在眼下,却还是他们春风得意的时候。

此外,高永能去了泾原,折继世回了河东,但凡在横山一役中有上佳表现的将领,无一例外的都厚赠封赏,有了各自的去处。

相对于一个个加官进爵的将校,宣抚司的文官当真吃亏大了。韩冈回头看看,连种建中都成了小使臣最高一级的东头供奉官;而亲身参加了罗兀城撤军,并献策伏击了嵬名济的种朴,更是一跃成为正八品的内殿崇班,进入了大使臣的行列——已是相当于文臣中的朝官了。

虽说武将只要有战功,晋升就是这般迅快,而犯了错,降级也很快,可种家兄弟的境遇,让王厚都为韩冈抱起不平来。

在自家的小院中,坐在荫凉的树下,韩冈为脸色愤愤的王厚倒着酒。不以为意的笑着:“连番大战,斩获无数,晋升起来当然快。以他们的功劳,受到今次的赏赐,并不算待之过厚。”

“但你可不是这样。”王厚尤是难以释然,“看看玉昆你,以你的功劳,不论是在河湟还是在横山,单独拿出来都能入朝上殿。可现在呢,种家的人反都抢在你前面了。”

韩冈轻笑着,给自己的倒了一杯自家酿的青梅酒,倒满微黄色酒浆的杯壁外侧,有着滴滴水汽凝成的露珠。天气暑热,传说中的青梅煮酒,绝没有连酒坛一起放在井水中冰镇过的酒水喝得舒爽。

他举杯向着王厚,笑容毫无挂碍:“各有各的缘法,各有各的际遇,强求不来的。”

比起一时的官场得意,天子的重视才是第一位的。章惇在给他的信中都说了,天子可是为不能依功封赏,苦恼了许久。种朴的名字,皇帝不一定能记住,而韩冈这两个字,就算没有写崇政殿的屏风上,想必赵顼也不会忘了。

种建中寄来的信笺,顺便还提起了赵瞻的结果,虽然在所有参与了关西战事的文官中,赵瞻在枢密院那里得到了最高的评价,多人联名为他请功,而天子也没有驳斥,本官都跳了两级。但已经三个多月过去了,原本的开封判官早被人占了去,但新差遣依然未至。作为朝官就只能在家中候着,这也算是赵顼对他不满的反应。韩冈对此,也只是一笑而已。

冰凉的酒水下肚,韩冈放下杯子,又拿起筷子,严素心做的下酒小菜可是一绝。吃了一块烟薰兔肉,他才又道:“横山攻略虽是败退,但西夏国势也因此削弱了不少。前日还听说,兴庆府那里生了点乱子,梁氏兄妹杀了不少人。几年之内,党项人那里就算再动刀兵,也不会到穷乡僻壤的河湟来,而是往环庆等上佳去处去劫掠,我们可以安安心心的收拾木征和董毡。”

王厚终于放开了,呵呵一笑:“家严近日也念叨着吴钩终用,因横山之事,河湟已是蹉跎许久。接下来……也该轮到我们了。”

……………………

横渠镇是勾连东西的要道,是渭水流入关中平原后,经过的第一个大镇。站在镇中,南面的太白山头上的皑皑白雪清晰可辨,只看着山头,便仿佛有一阵凉意冲散了夏日的暑热。

就在镇子外,是一片丰收在即的麦田,由青转黄的麦浪一眼望不到头。田地中阡陌纵横交错,将一块阔达数顷的地面,划分成一个个豆腐块似的方田。

顶着正午时分最为炽烈的阳光,有两名五十上下的老者,缓步走在狭窄的田间小道上。后面一人是在长安见过韩冈的吕大忠,而走在他身前一点,与他年岁相当的老者,带着斗笠,一身短打,装束看起来像个乡农,但他的步伐舒缓中而带着沉稳,自有规矩在足下。举手投足,都与土中刨食的农民在在不同。虽然貌不惊人,但神采内蕴的醇和气质,是饱学宿儒才有的气象。

吕大忠望着田间,脸上有着难以掩饰的喜色,“先生,这块地今年必是丰收无疑,井田当真有效。”

对着前面的老者,吕大忠的声音恭谨,并不因年岁相近,而有所怠慢。

“贫富不均,教养无法,虽然人人都说是要大治,实则不过是苟且而已。欲行仁政,首先便是得行井田之法,以均贫富。”斗笠老者语声徐缓,温和而诚挚,即便是语带责备,也会让人不会感到生气,而是虚心接受。“王介甫赞井田,正叔、伯淳【二程】也赞着井田,但并不是光说就可以的。”

老者温润的眼神中,有着少年一般追寻着理想的神采,“世人皆知井田之善,却拖延不行,不过是畏难而已。如果能缓缓图之,十年二十年,一代一代行之不移,终有成功的一天。虽然你我可能看不到,但总能遗泽于后人。”

“先生说的是……可惜玉昆没能来看一看。不论书院还是井田,都有他一份功劳。”

韩冈前日从长安回通远军,正好经过了横渠镇。但当时他还是押送着流放通远军的罪囚,为防他们给地方带来危害,每天的行止都是有着定数,就算韩冈本人也不能随便离队。甚至害怕惊扰百姓,在经过沿途城镇的时候,都必须加速通过,严禁耽搁。

所以韩冈还是无缘到新修好的书院中一行,也无缘看一看,由他资助而买下,作为关学一派进行井田实验,分给农民的田地。这让吕大防感到很遗憾,也为韩冈遗憾。

老者在田垄上慢慢的走着,正午的烈日也没能让他脚步多上一份急促。他一束束的看过沉甸甸的麦穗,“此事不用急。玉昆虽然困于俗务,但心性仍是吾辈中人。同是在大道行走,终有能见面的时候。”

