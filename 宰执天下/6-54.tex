\section{第七章 烟霞随步正登览(六)}

烧掉?

张璪现在确信韩冈和沈括之间没有联系了。

这话不应该由臣子来说的。

将刑恕、蔡京等人所有的信件一股脑的全烧掉,最是干净,从此人心可安。否则就没完没了,睡在家里也得担心夜里有人敲门。

只是在场的大臣们哪个不知道该怎么做?但他们有一个敢说出来的?

袁曹官渡之战后,曹操从袁绍大营中搜到大批属下私通袁绍的信件,下令将之尽数焚毁的是曹操本人,而非是帐下谋臣。处在当时曹营臣僚的位置上,首先是要自清,而不是为曹操着想来安定人心。

虽然说干掉了蔡确、又恨蔡京不死的韩冈,完全没有必要担心被误会与叛党有勾结,可是收买人心这一条,就无法洗脱了。得到群臣的感激,对臣子来说,并不是什么好事。

要么是心里有鬼,要么是收买人心,身处两难境地,缄默不言才是上佳之策。

就算韩冈早已是债多不愁,也没必要给自己在天子的心中,再添上一重恶感。何况沈括已经解决掉了韩冈的对手,又自己跳了出去,没有了对手的韩冈只要等着就够了,也完全没有必要再多话。

只是因为沈括突然冲吕嘉问等人下手,韩冈担心被人视为主使者,在权衡利弊之后,他才会冒上绝大的风险,去选择解决沈括的问题。

说起来韩冈还是看重名声,而不是未来自身的安危。

这种惜名不惜身的行事作风,张璪能够理解,却绝不会去仿效。

太后会答应吗?

但朝廷不可能去冒风险选择一名与叛逆纠缠不清的宰辅。

赶在之前结束,,吕嘉问等人就再无机会。

张璪期待着来自御座之后的回答。

只是当先出声的不是太后,而是另一人。

吕嘉问几乎是冲了出来,“不可!!决不可烧!”

吕嘉问几声大叫,让许多朝臣对他怒目而视,也让正准备同意韩冈意见的向太后改变了要说的话:“为何不可?”

吕嘉问急促的说着:“臣与逆党旧日或有往来,亦不乏文字。但从逆之事,却是无中生有!”他的声音尖利,一边对沈括怒目而视,一边为自己辩解:“臣之清白可昭日月,若今日焚去信件,臣将无法自辩于陛下面前。他人信件可以烧,但请陛下留下臣的信件,公诸于众,以示臣的清白!”

吕嘉问果然没有糊涂。

王安石略略放心下来。

不论吕嘉问本身有多少问题,他都是从一开始就站在新法一边,从未有过叛离。只是这一事,就让王安石绝不会答应有人将他跟叛逆联系在一起。

烧去已经被搜检入官的信件,有罪的当然可以趁机脱身,但无罪的官员,便无法再自辩。前面沈括刚刚攻击过吕嘉问,若太后当真听从韩冈的建议,将所有信件一起烧掉之后,吕嘉问要怎么辩解,才能让人觉得他没有与叛逆有勾连?这分明是坐实了吕嘉问身上的罪状。

已经被沈括点名的吕嘉问等人,都不能放任私家的信件被烧掉,至少得设法表明自己的清白和心胸坦荡,否则日后别说是参加推举,就是朝廷中的位置都坐不稳了。

“臣曾孝宽请陛下留下臣的信件。”

“臣黄履请陛下留下臣的信件。”

一干涉案朝臣,都被逼得站出来自辩。甚至包括没有牵涉到的李定,也出来了。身为御史中丞,李定这时候不站出来,就是不适任的表现。

“刑恕、蔡京等叛逆党羽为官日久,往来官宦都数以百千计,难道说他们都是叛逆不成?从其家中搜寻出来的信件,必然大多都是寻常问候。若不加检视便一起焚毁,是视诸臣皆为叛逆同党。请陛下另择贤能,加以检验,以还诸臣清白。”

沈括站在文德殿中央,连一句回话都没有,看起来茫然失措。

投机又失败了——虽然不知道他到底是投靠了谁。但沈括这一次,可是犯了众怒了,原本还有可能被选入三人之内,但现在已经没有任何机会。

这意外还真是一重接一重。难道这次推举,将会无疾而终?

王安石望着殿内。

若这一次的推举不能举行,韩冈的政治威信至少要打个大折扣。短期之内,肯定是难以挽回。

只是经过了这么多事,王安石对韩冈猜忌心的很重,只要是韩冈的建议,总要多想一想,因为总不会那么简单。

很多人都在看着韩冈,等他的反应。

信件是证据,不可能烧一半留一半,要么全烧,要么全留。若是太后同意了吕嘉问、李定、曾孝宽等人的意见,那信件就会都保留下来,让人从中寻找与叛逆勾结的证据,而作为提议者的韩冈本人也会坐实李定的攻劾。若太后选择了韩冈建议,却必须先为吕嘉问、李定等人开脱,只是在韩冈的立场上只能如此选择。

只见韩冈拱手一礼:“事涉内外千百臣僚,请陛下速下决断!”

事涉内外千百臣僚,听到这一句,很多人都放下了心来。韩冈依然坚持他的主张。

只是这话虽说十分直白,但还是有那么一点隐晦,太后能听得懂吗?还是有人担心着。

过了不知多久,屏风之后,向太后终于有了决定,“……吾知卿等必无与逆党私通之事。但那些信件留着徒乱人心,还是烧了吧。”

“陛下!”吕嘉问叫道。

向太后提高了音量:“就当做没有这回事。”

吕嘉问等人需要的就是太后的这一句,以后便再无人能够利用与逆党的通信来攻击他们,至于那些信件,烧了还是最省心。

“陛下圣明。”韩冈行了一礼。

李定、吕嘉问等人则是沉默弯腰行礼,然后返回班列之中。

沈括的攻击被化解了,韩冈明为相助实则栽赃的手段,也因为太后的一句话被化解了。

这样的一场骚动,让原本期待推举的众人,心头稍稍的冷静了一下。

唯一的问题是沈括退出了。

沈括退出的情况也考虑到了。谁敢将希望放在墙头草身上?巴不得他跟韩冈对杀,可若是他有了

“不要再耽搁了。”向太后对王安石道:“平章,该开始了。”

王安石点了点头,被一场骚动延误了片刻推举,终于开始了。

选票一份份的发了下去。

并不是立刻就写下要推举的名讳。

而是先行举荐出几人参选,然后才会在纸上写下他的名讳。否则若写上选票的人资历不够,或是早已经是宰辅之身,不能就任,那就是废票。

京城的赛马、蹴鞠两家联赛总会选举会首时,参加投票的成员一个一百多,一个则是近三百。这样的选举,多几张废票无关紧要。而宰辅推举,京内京外所有侍制在内,也没有超过五十个,其中还有很大一部分是在京外各路监司或是大州府任官,根本回不来。一张废票可就能够改变整个选举的局面。

必须先行确定几人参选,然后才会开始选举。

朝堂之中,够资格被推举的就那么几人。只是其中有几人会推荐韩冈?

苏颂是宰辅中的一员,他无法直接表态,只能等着看谁会先出来。

“臣举韩冈。”第一个站出来的是范纯仁,“两府之位,非是赏功,非是赏劳,而是在于对国事有所裨益。无论军国之事,韩冈可谓是有口皆碑,功绩累累。若能任职其位,当能裨补于朝廷。”

死硬的旧党分子出面,定下了韩冈的基本盘。

很多人心中大叫,传言果然不错,韩冈已经决定离开新党,与旧党联手了。之前还有人不信,毕竟韩冈和文彦博、吕公著、司马光的恩怨,朝堂内外无人不知,就算现在刑恕连累了吕公著、司马光,顺便将道统近于旧党的道学也连累了,可这么快转向,洛阳一应元老的腰骨未免柔软的跟他们年纪不相称。

好歹也要过个一年半载啊。

不过上面的重臣们对此没有什么反应,至少他们不会将惊讶表现在脸上,就看着一名内侍,在台陛上的一张空白的屏风处,写上韩冈的名字。

“臣举吕嘉问。”龙图阁侍制、知谏院黄履出面道:“吕嘉问掌市易务,使国库充盈,得天子倍加赞许。其升任三司使,严掌钱帛进出,使得国用不乏。”

作为新党中的一份子,他当然不会选择韩冈。韩冈不能算是新党,新党为主的朝廷,为什么要将韩冈推上去?

“臣举吕惠卿。”出声的并非王安石。招吕惠卿回来就任宰相,也是有着很多支持者。

“守护北门,非吕卿不可。”向太后决然道。

“陛下。”王安石回头对屏风后道,“现在只是推举,待群臣定下名单,陛下可以从中取舍。”转过来他又面对群臣,“今日乃选任枢密副使,吕惠卿向日已为参知政事、枢密使,如今岂能降用?不当列名”

作为主持者,王安石很直接的否决了太后的干扰。不过是第一次,已经有模有样。

王中正眼角的余光看见太后面现怒容,然后又平静了下去。她是给王安石的口气给气到了,不过也应该有一些是因为担心韩冈不能入选。

以王中正对太后的了解,半个月的时间,不足以调整侍制以上官的名单。否则太后肯定会进退部分重臣,以保证韩冈能够顺利的通过推举。

不过韩冈已经做过了枢密副使,这一次选不上,下面的参知政事也能参选,又或者将宰相的空缺让给章敦,枢密使的位置则留给韩冈。

机会还多的是。

只不过两府之中太热闹了也不好。王中正想着,就快变成菜市口了。

章敦一边看着侍制们推举候选者,一边又瞅着韩冈,心中总觉得有哪里不对。

韩冈的态度似乎有问题。虽然说不出,但怎么看怎么奇怪。

不对!

当章敦的眼角扫过沈括的时候,突然灵光一闪,

蔡京,蔡京怎么与刑恕并列,成了叛逆党羽了?

