\section{第20章 冥冥鬼神有也无(13)}

【对各位书友说声抱歉,周日白天有事要外出,中午的一更只能明天补上。】

李宪病后初愈,脚步还有些虚浮,往中军大帐走过去的时候,身子依然摇摇晃晃,但身后的护卫伸手想搀扶他,却都被他给推开。一病多日,连自己的差事都只能交托他人,这已经够失败了,要是被人搀着走路,他哪还有脸踏进章惇的营帐。

“李承受。”

“承受。”

“小人见过承受。”

尽管如此,李宪在安南行营中依然还是得到足够的尊重——尽管是官位使然。

一路上,见到他的将校士卒,都是立刻避让到一旁行礼问安。还有些没穿军袍的,却是老远就跪在地上,头也不敢抬。当然不是营中的禁军士卒,而是配军。禁军都是有俸禄、有兵籍,多少人抢都抢不到一个位置,而配军罪囚则是在军营中做着粗重的杂务。犯法流放,第一等是海上的沙门岛,第二等就是岭南了。岭南的军营别的不多,就是配军的罪囚多。

从位于邕州城外的军营西北角的天王堂,到章惇旌节所在的中军大帐,有近一里的距离。李宪就是从天王堂出发,去参加章惇主持的军议。

天王堂中供的是北方多闻天王,也就是毗沙门天。不过现在大营中的天王堂,除了主殿以外,都给占下来做了疗养院,但只安排生病了的将校,李宪就在里面躺了有七八天了。

李宪作为内侍,是安南经略招讨总管司走马承受并体量公事,冗长的名称代表着他的眼睛和双手,能接触到安南行营的每一个角落。可他比韩冈早一步从京城南下,可是在半路上就病倒了,在桂州修养了整整半个月。等到病情稍稍好了那么一点,就赶着来邕州,但刚刚到了地头,就又倒下了来。过了将尽十天才算好得差不多,也算是幸运的赶上了出兵交趾的最后时间。

参加军议,还有见到安南行营中所有的文武官员,李宪都还是第一次。李宪从挤满了偌大的中军大帐的一张张面容上看过去,其中有些人去天王堂中探过病,但更多的还是十分陌生的脸庞。尤其是脸上满是刺青的蛮部洞主,满满当当的竟有七八十人之多。

李宪暗自思忖着,看来章惇和韩冈并不准备在三十六峒蛮部中树立几个大首领来,而是打算不论大小一视同仁,否则就应该点选几个可靠或是势力大的部族来,而不是让他们挤满大帐。

章、韩两人的做法,李宪并不能断言好坏,两种手段各有利弊,就是帐篷里面人太多,看着倒像是菜市口。

监军的到来,代表这最后一次战前军议终于可以开始。

今天进行最后一次军议,明日便要誓师出征。

“这是广源州的黄团练。”韩冈伸着手,为李宪引荐着身有官职的蛮部首领。原本在邕州、桂州的行营将领和经略招讨司官员,基本上都在章惇、韩冈的许可下,去探视过李宪。

“末将拜见承受。”在战前赶到邕州的黄金满向着李宪行礼问候。

只是黄金满是正任团练使,论官阶甚至在章惇、韩冈之上,更别说李宪。就见他连忙回礼:“李宪见过黄团练。”只是心中充满疑惑,为什么黄金满敢在这时候到邕州来,不怕交趾或是刘纪等人抄他的老巢。

“广源州四大首领,申景贵、韦首安都已降顺,只剩刘纪还犹豫不决,不肯投降。”韩冈紧跟上来的话,解释了李宪心中的疑问,“现在韦首安本人已随黄团练到了邕州,而申景贵也派了儿子来表示诚意。等军议结束之后,就可以招他们进来问话。”

“难怪……”李宪点着头,很是感慨的章、韩两人的手段,“运筹于帷幄之中,决胜于千里之外,尚未出阵,亲附交趾的部族恐怕就不剩多少了。”

“依如今的估算,交趾能动用的兵员也就只剩五万,除非交趾能征发起国中所有丁壮,否则在人数上也根本比不上即将南下的大军。”韩冈正是要在所有人面前宣扬交趾如今的困局,他大声向李宪介绍一众蛮部首领:“这些都是愿从号令的各峒洞主,总共七十八家,拥兵共计六万,还有广源州两万兵马,他们将会在官军出战的同时,一起出发。”

十万!与现如今安南行营的官军兵力合起来就是十万。

就算打个折扣,都有五六万人之多。

当初李常杰是带着交趾、广源的洞蛮杀来广西,现在却是官军反过来利用洞蛮反杀回去。就是成色不够高,这一群强盗远比不上西北二虏的穷凶极恶。

“十万!”但李宪还是夸张的笑着,赞美着章惇、韩冈的手段:“加上官军,已经足足十万大军!有十万大军枕戈待旦,区区交趾又何足道哉?!”转过身来,他对着一众洞主教训道,“尔等宜当努力,若有功绩,朝廷必不吝赏赐!”

坐在主帅交椅上的章惇与韩冈飞快的交换了个眼色,嘴角都凝起一丝冷笑,李宪这两句训话,是趁机在帐中确立自己的地位,倒是会借势。

蛮部洞主们则哪里会想那么多,一起低头受教。冻州洞主龙礼合恭声道:“朝廷将交趾人口、土地,都赏赐了下来,小人等哪还敢贪求更多?定当效死,誓破交贼!”

“定当效死,誓破交贼!”在龙礼合的带领下,洞主们齐声发誓,“定当效死,誓破交贼!”

李宪似乎很满意,点着头笑着转身。只是登时就见到韩冈脸上的微笑带着寒意,心中先是一凛,然后就发现韩冈视线的焦点并不在自己身上,而是落在身后。

李宪顿时就明白了,这并不是针对自己,而是看起来打算要团聚蛮部的龙礼合。朝廷是不会允许左右江三十六峒中出现一个核心,如果有人有这份野心,肯定就会被盯上。

韩冈收回带着危险的眼神,继续向李宪介绍了几个蛮部的将领,连同侬智高的弟弟侬智会一并介绍过来。

李宪不断的点头微笑,将人名和相貌一一对应上,心中却是在赞着章惇和韩冈的手段。

能聚来这么多蛮部首领,基本上都是畏惧于官军的威势,当然他们也是想着在征南的时候,分上一杯羹。但朝廷是不可能让他们顺顺利利的跟在官军后面捡便宜,必然要有方略应对。

决不是让他们上阵与交趾军厮杀——若真想要数万蛮军上阵作战,就不会允许他们现在这般一盘散沙,肯定是要整合起来。如今既然是各自独立,那么作战的主力只会是官军,而一众蛮部,则只会是分散出去,劫掠地方。为防这些蛮部太占便宜,所谓的刖刑,应该就是很重要的一个手段。

其实李宪在听说韩冈威逼洞主们对掳来的人丁都施以刖刑,以便管束的时候,就是这么在想了,眼下看着满帐的蛮人更是得到了确认。

仅仅是从对交趾丁口施加刖刑这一点上,就不能算坏事,至少各家蛮部即便得到了交趾的人丁,也无法用来扩张自己的势力,只能拿这些废人来做农务工。

而且这些受了刑后的人丁,仇恨的主要目标不会是打完就撤离的宋军,而肯定是日后朝夕相见的左右江蛮部。蛮部一时得到了土地,可日后还会是一团乱,不是交趾人起兵反叛,就是他们将交趾人都杀光。且蛮部想要拿到土地、丁口,就得先跟他们预定中的奴隶拼命厮杀——想必这段时间,经略招讨司已经将刖刑一事给传扬出去了。

与交趾军正面抗衡的是官军,而在外围劫掠州县村庄的则是蛮部,苦活累活看似都是官军来做,而蛮部可以避重就轻的捡便宜,但交趾人的抵抗必然会十分激烈。不过支援他们的主力都在与官军对垒,失败是必然,可一旦打出了火气,下手就轻不了,到时候蛮部能得到多少收获还真难说。这实际上就是毁掉了日后几十年内,任何人借助交趾这块土地兴起的可能。

也难怪天子会认同经略司的意见,将辛苦得到的土地分给这群蛮人。毕竟迟早是自家物,要拿回来容易得很,而且朝廷还不用担骂名,也就是士林之中会有些书呆子乱说话。

认识了一众蛮将,站立在自己的位置上,望着大帐中央的章惇和韩冈的侧脸,李宪暗暗慨叹,‘好手段啊!’

章惇站起身,帐中静了下来。

章惇并没有换上战袍,显得一派儒雅。但锋锐又不失沉稳的眼神,则只有屡经磨练的重臣才会拥有:“时至今日,南下的官军已有五千之众,加上广西能动用的兵力,总共一万两千兵马。这是实数,也不忌讳对外传。”

章惇微微一笑,他并不介意说出手下兵马实际数目,也不打算号称几万十几万的虚头。因为有过去的战绩在,由前日耀武的威慑在,完全足够了。

“千五官军、配上黄团练的五千人马,打得李常杰的十万大军狼狈而逃。现如今,官军一万两千,又有左右江诸峒和广源州兵马总计八万,交趾人即便有百万大军,也不在话下。不过朝廷用兵一向以稳妥为主,还有两万来自陕西的精兵即将抵达广西,只是要到明年正月。但眼下正是用兵的好时节,本帅也不愿耽搁,打算先将通向升龙府的道路给打通。”

兵不厌诈。后期的援军不会再有,这一件事,是高度保密的,只有最高层的几人知晓。李宪望了望韩冈、燕达和李信,他们都是神色如常,将这个谎言当成是真实。

章惇抬手指了指最后面的蛮部洞主,继续说着:“交趾北疆的山路,由于各位洞主的奋战,如今已不再是阻碍。只要拔下门州、官军可以直逼平原。接下来的打算,想必诸位应该已经都知道了,本帅和韩副帅也多次说过。不过为防有人忘却,本帅如今再重复一遍。很简单,就是搜山检海、犁庭扫穴。”

“以一月为限,先以富良江以北的州县乡村为目标,设法将交趾军从升龙府中逼出来决战。”章惇笑容敛起,语调生寒:“如果李常杰打算困守升龙府,富良江以北,就别想留下一座村庄!”

