\section{第39章 太一宫深斜阳落(二)}

【今天第一更,求红票,收藏】

递过一串香火钱,转头看着在香案前虔诚叩拜,连磕了十几个响头的路明,韩冈等他站起身后,便问道:“太一天帝难道兼着文曲星君的职司?路兄拜得如此虔心?”

“见庙拜上一拜,求个心安,也不指望真的能管用。”路明也许是不想跟韩冈讨论这个话题,带着韩冈往偏院走,又道,“真要说起香火旺盛,入京贡生都去上香礼拜的,却是城南的二圣庙。”

“二圣庙?”韩冈只听过二郎神,被仁宗封做灵应侯的灌口二郎在蜀地很有些名气,而二圣他可是从没听说过,“不知供得是哪二圣?”

“子路,子夏。”

“子路?子夏?”韩冈听着一愣,“是圣人门下七十二贤人中的子路和子夏?”

“正是子路、子夏两位贤人。”

“他们不在文庙里供着,怎么分出来立庙?春秋时还没科举吧?连九品中正都不知在哪里,两位贤人怎么保佑贡生中进士?”韩冈想不明白,疑问一连串的问出来。

“谁说不是!”路明好像已经忘记了方才自己在东皇太一前叩的十几个响头,摇着脑袋说得痛心疾首,“身为圣教弟子,却拜那些土石木偶!‘敬鬼神而远之’,‘不语怪力乱神’,圣人之教全都忘了个干净。土石无知,岂能干系抡才大典?”

这位应该是没少拜过二圣庙,也没少捐香火钱,但每次都不灵验,一肚子气便发作在子路和子夏身上。几日下来,韩冈已经看透了路明的脾性,但戳穿了便没意思了。

他也笑着道:“若说起拜神求个心安,秦州也是一般。韩某乡居左近便是汉将军李广之庙。只要是进山行猎的猎户,有事无事都会拜一下飞将军。飞将神射,石头都能射进去。可出行远游,却决不能拜他。”

“为何?”

韩冈笑了,出行不拜李广的理由的确很有趣:“防着迷路失道啊。”

“迷路失道?”路明的头上转着问号,满是疑惑的样子。

“想想李广,他一辈子迷了多少次路!但凡只要他能识路,又怎会‘冯唐易老,李广难封’?”

“啊……啊!”路明啊了几声,突的一脸恍然,哈哈大笑着,“原来如此,原来如此!妙!妙!真妙!实在有趣啊!”

‘真的想明白了?’路明干笑的样子,韩冈看在眼里。暗地里摇头,看来路贡生今科又是没指望了。别的倒也罢了,怎么连《史记》都没记下来?!考试时,要写文章绝少不了引用经史。路明自己一个劲说可惜的嘉佑二年那一科,欧阳修出的题目不也是从中国最早的史书——《国语》——中节录下来的?

“京城之外,还有个梓潼庙!”大概觉得尴尬,路明转又说起贡生拜神求进士的话题,“庙就在利州路上,自金州出蜀的道路边。据说也是极灵验,蜀地出来的贡生没有一个不拜的,听说苏子瞻、苏子由也拜过。想不到以苏子瞻之豁达,也不能免俗。”

韩冈忽然发现,虽然路明无甚才学,而且又喜欢胡吹大言,但肚子确实有货。四方传闻,朝野典故,比王厚都门清。看来他这三十年来,在东京常来常往,又是混迹在士子之中,读书的时间多半用在包打听上了。

出了主殿,转过廊道,路明带着韩冈去看那几株据说是唐初名相褚遂良种下的老梅。只是梅院中早早的便给人占了下,七八个年岁不一的士子,正坐于雪上梅下,烤着火盆,喝着热酒。正在热火朝天的吟诗作对,行着酒令。韩冈看看那些士子,又瞥了路明一眼,想不到这里也有不把即将开始的省试放在心上的人物。

好风雅的儒生大冷天的坐在屋外聚会喝酒,除了吟诗作对、兼做扯淡,也不会有其他正事。韩冈并没兴趣上前凑个热闹,便顺着廊道继续徐步向前。庭院中的士子对庭院旁、廊道中,来来往往的游人习以为常,韩冈和路明的经过并没有打断他们的谈话。

一个二十出头的年轻人举杯喝了一杯酒后,操着南方口音,突然问道:“爆竹声中一岁除,春风送暖入屠苏。王大参这首新诗不知各位听过没有?”

他的声音很大,熟悉的诗句传了过来,韩冈一下便竖起了耳朵。。

“王大参的新诗?当然听过。”接话的同样年轻,就是黑瘦了一点,也是南方口音,不过是福建一带的腔调,与前一人明显不是同乡。

韩冈与他一起将后两句吟了出来,“千门万户瞳瞳日,总把新桃换旧符。”

韩冈的声音很低,并没有惊动到院中的士子们,只听着他们在说:“新年新气象,王大参这首诗明明白白是在说变法。均输法、青苗法、农田利害条约,王大参弄了这些还不知足,今年朝中怕是又有大动作了。”

京城不像秦州,把高官都叫做相公。皇城脚下,对名位的称呼是件很严谨的事情。王安石还是参知政事,不是宰相,参知政事的简称大参,自然说的就是王安石。

流传千古的诗句,就在身边近处完成,韩冈走进历史的感觉忽然间又深了一层。原来王安石的元日是在这个情况下做的。

新桃换旧符……新法易旧法……难怪。看起来王安石是在用此诗来表决心呢。

“大动作?王大参该不会是又要提变诗赋为经义策问吧?”

“怎么可能,都这时候了,还来省试改经义。城中数千贡生,到时候登闻击鼓,叩阙上书,谁做不出来?”

韩冈脚步不停,十来丈长的廊道转眼走尽,从侧门进了偏殿。隔着偏殿侧门,韩冈驻足停步,只听着院中那个大嗓门的士子又在说着:“王大参做得好诗,却偏偏跟诗赋过不去。若不是苏子瞻,今科进士都要改明经了!”

“自隋唐至圣朝,都几百年了,哪一次进士科不是用的诗赋?王相公自己都是靠着诗赋出来的,却过河拆桥,改什么经义策问!”

“苏子瞻说得好,‘自政事言之,则诗赋策论均为无用矣’。皆是‘以空言取天下之士’,用诗赋和经义策问又有什么区别?”

“若是出身陕西的司马君实提议倒也罢了,谁能想到会是江西人!”

几人操着南腔北调,一阵七嘴八舌。今科进士科举试,王安石欲变诗赋为经义策论,不过让苏轼给谏阻了,这是去年的事。韩冈从王厚那里听过,多少知道一点内情。不过他并不认为王安石会就此偃旗息鼓,去年的建议应该只是试探,王安石上表的时间,地方上的解试都要开始了,即便通过,当制敇传抵整个国家,通过解试的贡生早就选拔出来了——解试的考题只会是诗赋。既然拔贡用的是诗赋,那省试还能用别的吗?

王安石的提议必然是试探,想看看究竟有多少人会反对此事——也就一个苏子瞻。司马光都是同意的,王安石要想将提案通过,又有什么难度?试探而已!

就像后世的高考改革,从来不会跟在读的高中学生为难,都是提前个三年,变在即将入学的高中新生头上。否则哪家的家长和学生不会闹?王安石真要改变科举制度,只会在下一科推行。

“还抱怨个什么?今次照样还是诗赋。都已经定了王内翰知贡举,当日领了命便入贡院锁院了。还能再变不成?!”

内翰,就是两制官中的内制——翰林学士。制,乃是为天子草诏的意思。两制,分别是内制翰林学士,外制中书舍人,都是有资格为天子起草诏令的官员。翰林学士是天子近臣,所以是内制,而中书舍人,隶属中书省,所以是外制。故而翰林学士通称内翰。

据韩冈所知,如今的翰林学士中,姓王的只有一位,便是与王安石同年登科的王珪王禹玉。

“既然是王禹玉知贡举,不用说,当是以富丽堂皇为上。考场中当是要注意一点了。”

“至宝丹嘛……”另一人笑道。

王珪的诗文金匮满眼,所以世人称为至宝丹,这一点,韩冈也是听过说的。揣摩考官的心思,从中分析考题的范围,看来只要是考试,都是一个模样,时代的差异也没造成多大的区别。

只听那位福建举人又说道:“今年上元夜,王禹玉被招入宫应制诗文,可是收了嫔妃们多少笔润,满袖子的都装满了宫钗出来。”

言者羡慕,听者神往。如此恩荣,哪个士子不想是自己得到。

另一人则提醒道:“不要只看知贡举。同知贡举的吕中丞,苏掌诰还有孙直院可没一个喜欢金玉满堂的诗赋。”

韩冈今次又不参加科举,对考官的性格也不感兴趣。知贡举的王珪,他从王厚那里听说过,同知贡举的吕中丞,就是他老师的举主吕公著。但苏掌诰、孙直院,都是姓氏加个官位简称,却让韩冈完全摸不着头脑。他对朝堂了解得还是太少了。

但他也并不着急,已经有了官身,在官场上待久了,自然逐渐的会知道。

