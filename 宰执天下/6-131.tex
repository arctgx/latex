\section{第12章 锋芒早现意已彰(12)}

种建中早上醒来的时候,外面寂静无声。

穿好衣服,掀开帐帘,顿时一阵寒风扑面而来,空气比帐中清新许多,但一口气吸进去,从鼻中到胸中一阵干疼,差点连肺也给冻住。

天色依然是黑沉沉的,连着几日的阴天,遮蔽天日的浓云这一夜也没有散去,仰起头也看不见星星和月亮。

而远远近近的火光,却仿佛星海落在了地面上,铺满了整片草原。

一丛火光背后,基本上就是一个住着辽国士兵的帐篷,乍一看上去,就仿佛天上的繁星一般多。

‘该不会尚父把举国大军都调过来迎接我等了吧。’

刚到冬捺钵的时候,向英在夜里看到连天接地的星火,还开玩笑的说着——大概是为了向人证明自己的胆量。但近几日,那位与种建中同为副使的国戚,连个笑容都没有了,镇日躲在帐篷里面。

清水都是凿冰融化,厨子送了水来,匆匆洗漱过后,种建中跨上了马,开始在营地栅栏的内侧进行每日两次的惯例巡视。

“又少了一点了。”

种建中眺望着点点火光,突然说道。

“十九官人?”跟在身后的亲兵没听清楚。

“没什么。”种建中说道。

骑在马上,与几个熟面孔打了照面,种建中回来的时候,早饭已经准备好了。

士兵们吃饭的地方就是各自帐前的篝火便,打了饭菜回来吃,要是风大,就回帐中去。

今天没什么风,都在外面吃。一日三餐,王存和向英都在自己的帐篷里解决,只有种建中会跟士兵们一起吃饭。种建中走过去时,位置和饭菜都给准备好了。让人从锅里舀了一碗热水,喝到嘴里时,就已经没那么烫了。

看种建中喝着涮锅的开水,一名小校过来殷勤的问着,“副使,不喝酒?有热的。”

“算了。”种建中摇头,“你们分吧。”

昨天自己喝酒的时候,几十只眼睛都盯着,差点连口水都流出来,这让他怎么可能还喝得下去。而且那样的酒,种建中也不是太想喝。

耶律乙辛登基时,来自大宋的使团没有被召去观礼。

等到耶律乙辛正式坐殿,他们依然没有被召去觐见新君。

馆伴使受耶律乙辛之命,照常例来赐使团酒食,王存代表整个使团,辞而不受。

当时所有人的手心里走攥着一把汗,只盼着辽人还能顾念着过去的交情,还有南面的母国能让辽人投鼠忌器,但之后的结果,也不过要三位正副使节和几位医师,与其他使团成员享受同样的伙食标准。

主食是硬得能当盾牌的面饼,必须要泡在肉汤里面半天才能够入口。整个使团每天有两只羊,还有定量的盐和葱,没什么蔬菜,不过有豆豉。另外还有酒水供应整个使团。

对于主要是禁军士兵的使团底层成员来说,顿顿有酒有肉,已经算是优遇了。京城酒店中的酒菜虽好,也不是这些士兵能够经常吃的。

可在一众官员而言,没加香料的羊肉,味道不正的劣酒,根本难以入口。

作为使节,种建中自进入辽境之后,每天不是接受当地官员的宴请,就是在馆舍中享受精心准备的美食。现在的饭菜,这是他们之前从来没有受到过的待遇。

但如果这就是拒绝承认耶律乙辛皇帝身份所得到的惩罚,那位‘伪帝’当真可算得上是宽宏大量了。

只是种建中还是觉得辽人这么做,未免有些小家子气,大方就该大方到底才对。

不过这件事,也提醒了使团中的所有人,在耶律乙辛登基之后,有许多他们一开始所没有注意到的问题,是难以避免同时根本无法解决的。

不提觐见辽国皇帝,呈交国书等事,不去见一见耶律乙辛,他们怎么启程回国?

……………………

成了皇帝之后,耶律乙辛只是换了个住处,外面的守卫都没变。每天的日常起居,也没有发生什么变化。

就像是完成了一个任务,除了一点放松感和安心感之外,也没有太多的惊喜。

做尚父时处理的事务,做了皇帝后,一样要去处理。并没有因为地位的改变而变了什么。

越来越庞大的国家,越来越多的人口,让皇帝要处理的事务也多了起来。耶律乙辛已经代理天子职权好些年了,许多事在他来说,也不用太多时间去审阅,可每天要放在公务上的时间,也绝不会少到哪里。

处理完早间送来的奏章,耶律乙辛拿下了架在鼻梁上的水晶眼镜。问问时间,不知不觉间,竟已快到中午了,而张孝杰这时又带着一堆事过来奏禀。

耶律乙辛漠然的听着张孝杰的报告,然后一一作出指示,等到稍稍告一段落,才抬眼问道:“南朝的使节现在怎么样了?”

“昨日并无异动,今日情况尚无禀报。”

“快到该猎虎的时候了,可惜今年不能请宋使随行了。”

耶律乙辛看起来颇感遗憾。正常这个时候,一般会去邀请宋使参加猎虎等游猎活动,炫耀武力,同时也表示两国的通好之情。可惜宋人不肯服软,一切面会的仪式都耽搁了下来。

张孝杰连忙道:“是臣准备不周之故,稍待便去请。”

“用不着,不必强求。”耶律乙辛摇头,又问道:“去那边看病的还多不多?”

“之前已经完全没有了,听到陛下允许朝臣去求医问诊,这两天才有了点人,不过还是以种痘的居多。”

“其实种痘也不是什么难事,也没有必要一定要宋人的医官给种痘,难道医生就不能种?”

“陛下所言最是。种痘此事极易,国中也有圣手名医,他们来种痘也是一样的。”张孝杰附和一句,然后低声又道:“不过还请陛下放心,不论是谁去求医,都有人随行。”

耶律乙辛摇头:“把人都撤回来,没那个必要。要叛向南朝,也不会找被看管的使者。”

撤除陪送人员不是什么大事,只要使团外面还有军队围着,有眼睛盯着,什么花样都玩不出来。但这件事,除了耶律乙辛自己说,没人敢这么做。

“是臣考虑不周,臣回去就办!”张孝杰头点得跟小鸡啄米,上上下下,又下下上上。

“顺便把使团日常供应也恢复旧时,一切如故。”耶律乙辛瞥了张孝杰一眼,眼神中充满了深意,“没必要那么做。”

“陛下,宋使不识好歹,竟然拒绝陛下的善意,臣下岂有不怒之理?而且拒绝日常供给也是宋人自己的选择……”

“这些小事也不是该你这个宰相管的!”耶律乙辛对张孝杰道:“反正他们也回不去,交由他人处理好了。”

张孝杰立刻问:“陛下可是要将宋人都扣留下来。”

“扣留做什么,不扣留他们也回不去。”耶律乙辛笑着呵呵了两声:“要脸,就别想回去!”

说了两句,耶律乙辛就放弃了那些死板的宋人。宋国使团的命运,只在耶律乙辛散步时成为话题。区区使团,不值得他多浪费心力。

耶律乙辛道:“还是说说你的想法吧,该怎么让南朝早点攻过来?”

没有被火烧过,就不会畏惧火焰;没有赵光义因箭创烂掉坐臀,就没有五十万银绢的岁币。

做了大辽的皇帝之后,耶律乙辛已不再需要战争。但如果战争不可避免,他还是希望能来得更早一点。

然后一场战争打出几十年的和平来,有生之年,让宋人不敢北顾。

……………………

“又少了两个千人队。”

早上起来的时候,向英就听到了种建中的轻声细语。

在王存的耳边,种建中向他简单说明了这几天辽军军力的变化。

到底是为了平叛,还是正常的移防。这让人无法判断。不过这么下去,跟随在耶律乙辛左右的辽军也不会剩下多少了。

“但辽主身边的宫卫和皮室军,不可能少于三万人。其他军额的士兵,也有不少是精锐。”

几天的巡视,让种建中对辽军大营的变化有了更进一步的认识。他向着王存解释道。

“只是说说而已。”王存打了个哈哈,笑道:“不过若是当真有这样的结果,到时候,可就是想怎么走就怎么走了。”

“是啊。”种建中微笑的附和着,“若辽军都散尽了,我们也就可以回去了。”

“不过那时候,就要抱怨捺钵没有设在边境上了。从永州走到边境上,可是要很长的一段路程。”

“还是等着辽人的护送吧。”王存摇头,开玩笑说两三句就够了,怎么将使团带回去,才是重点。

“可朝廷绝不会同意耶律乙辛的反逆!”

一旦朝廷答应下来,那是动摇国本的危险之举。连如此恶劣的篡位都能容忍,日后还怎么让臣子效忠天子?君臣父子,三纲五常。这边是大宋维持稳定的基石。

“朝廷肯定有办法,内翰不用担心。”

包括身家性命在内的一切,都是要看朝廷接下来的行动了。

这个时候,朝廷应该得到耶律乙辛篡位的消息了。

不知两府诸公对此是怎么看待的?种建中很好奇。
