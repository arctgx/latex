\section{第36章 骎骎载骤探寒温(中)}

润州州衙正堂。

结束了持续数日的州中巡游,在驿馆好生睡了一觉的宗泽精神饱满。相较之下,润州本地的官员倒像是在青楼中辛苦操劳了三天三夜的模样。

宗泽前两年出京任官,就是在上任前先走了一圈,是临行前韩冈的建议,让他不带任何成见的先看一看自己将要任职的地方。

当时宗泽微服巡游了七日,在治下仔细的听了看了。一等上任,便抓了一桩积年的冤案,不仅抓了真凶,还将从徇私枉法的前任,到助纣为虐的吏员,一齐给办了。又将一处藏污纳垢的僧院给毁了,从中救出了三十多名女子,同时也为过去几十桩悬案找到了犯人。

两件案子总共斩了八人,流放了一百多,还有两位官员罢官夺职,六人受到从降官到罚俸不等的处分。在这之后,宗泽就彻底坐稳了位置,之后不论是催粮纳科,还是兴修工役,都是一言而决,无人敢于顶撞,所有的政策都顺利的施行。两年后,宗泽课最上等,顺利的回到了中枢。

这一回宗泽是钦命在身,不便微服,但他这么绕了润州一圈,尽管一句话都没说,润州上下,还有提点刑狱司,却都打起了十二分的精神,不敢有半点怠慢。

谁也不知道他看到了什么,也没人愿意去赌他看到了什么。与其靠运气过关,还不如先把人给奉承好了。该办的事,当然也要用心给办好。

宗泽能感觉得到这些官员心中的隐忧,也知道他们为什么而担心,而他很乐意让这些官儿多担惊受怕几日,维持这样的情绪,对他的任务很有帮助。

“在下出京前,章相公和韩相公只吩咐了两件事。”宗泽很罕见的拿出一幅高高在上的口吻,“第一,谁是主谋,第二,怎么防止同样的事再次发生。事情已经发生了十日以上,人也捉到了几个,想必这主谋,各位已经查明了吧?”

景诚的腰比上一回见转运使还多弯了几分:“是明教妖人蛊惑工人,纵火焚厂。”

“确定了?”

“人证物证确凿,已经确认了。”

果然是明教。

宗泽没有半点惊讶,煽动这么多人,怎么可能不露马脚?出京前,朝堂上都有了判断。现在连人都抓住了几个,口供早该拿到了。

“贼首在何处?”宗泽问道。

景诚回道,“妖人已逃匿,路中已经下了海捕文书。妖人党羽正在搜捕之中。这两日州中已拿获了多人下狱审问。”

“不会误捕良民?”宗泽再问。

“州县中派出弓手、土兵拿贼时,皆已耳提面命,绝不敢骚扰良善。且明教教众衣白茹素,极好分辨。”

明教在两浙、淮南、江东各地传播得很广。宗泽自幼见识过不少。在他所认识的人中,也颇有几位喜穿白衣,戒荤、戒酒的。名义上是礼佛,但实际上,两浙人氏多半都清楚,这样的人多半就是明教教徒。

现如今,连和尚都喝酒吃肉,一个个油光满面,持戒如此严谨,必然不是真信佛,而是明教教众。

“此事要尽快公布于众,免得民间不知因由,反而多生事端,或为妖人所乘。”

“州县中已贴出了告示,这两日还会在本地报纸上刊载。”

对宗泽的每一个问题,景诚都给出了合格的回答。

宗泽问的,景诚都准备了,宗泽没问的,他也准备。为了将这一位钦差应付过去,全州上下的官员都为之集思广益。

谁都知道,宗泽此番身负重任,这一次下江南,总不会就盯着润州一州。尽管润州这边损失最大,伤亡最重,但两浙路诸军州中,明教信徒人数最众的地方,可不是润州。

“润州虔信明教者甚众,其中必有不知情由的无辜之人,通判打算如何处置?”

“下官会依律处置。不会宽纵,也不会陷人入罪。”

被顶了一句,宗泽笑了一笑,没去在意。州中具体的差事,宗泽本就不打算插手。只是担心各州各县成了惊弓之鸟,将事情做得太过火,把两浙路闹得鸡犬不宁。

离京前,韩冈曾经对他说了句‘惩前毖后,治病救人’,这是不要妄杀的意思。而景诚的回话,也正符合了韩冈的要求,宗泽自不必再多说什么。

不过死罪可逃,活罪难饶,被捕的明教教众多半会抄没家产后发配边疆。而且这些邪教徒,与其留在天下的腹心之地,还不如丢到边荒去自生自灭。

只是确认身份很简单,想要解决却很是棘手。

韩冈的第二条要求,难度可高得很。

宗泽看了看景诚和厅中的其他官员,没再多提怎么一劳永逸的解决日后的问题,饭要一口口的吃,事情要一桩桩的做。

润州现在是有了推卸责任的对象,所以才能上下一心的去捉明教教徒,如果他宗汝霖再多说一句,开丝厂的大户也有一份罪责,那么下面的反弹就是他这位钦差也不一定能吃得消。

不过,话不能公开说,但私下说可就没问题。

屏退了其他官员,宗泽和景诚来到倅厅偏院的客厅中。宗泽先向景诚行礼,“诚甫兄,宗泽有礼了。”

景诚拜见过韩冈,也见过宗泽,虽只是两面,但也算有了交情。宗泽现在叙起私谊,他自然乐意回应。

两人重新见过礼,寒暄着分宾主落座,景诚问道,“汝霖方才言及,南来之前,韩、章二相曾吩咐二事。前一事,已可上覆朝廷。但这后一事,恕诚愚鲁,不知当如何去做,还请汝霖多多指点。”

宗泽笑了,“诚甫兄何须自谦,此番变乱的根由,不信诚甫兄不知。宗泽离京前,韩相公可是吩咐了,要多多请教诚甫兄。”

景诚眼皮跳了跳,也不再兜圈子,直说道,“没有了明教,还有暗教,不能放火,也还能劫道。只要工厂还在开,乱事就不会休止。”

“工厂必须开下去,这件事不容更改。”宗泽斩钉截铁,“但对工人,必须多给条路。官府得告诉他们,如果实在不想进工厂做工,又找不到其他差事,可以迁居他处,不论是云南,还是西域,都会有大片无主的土地。只要循规蹈矩,官府肯定会给他们一条活路。”

景诚叹道,“此事诚亦明白,只是难为啊……”

“此事当知难而行。教化百姓,这是官府之责,不教而诛,则是官长之过。但教后再犯,那就是犯事者自身之罪了。”

景诚摇头,宗泽高居庙堂之上,哪里看得见下面的情况,“汝霖,你可知丝厂建成之后,乡里还有几家能听见纺机响的吗?”

仅仅两年多的时间,两浙男耕女织的小农生活,便被工业化的机器碾得粉碎。

原本养蚕、缫丝、纺织,一家人就可完成了生产,现在就只剩下养蚕一件事可做了。

养蚕比工厂中缫丝还要辛苦。早在准备蚕室开始,全家老幼的生活都要为蚕虫让路。到了蚕虫五龄的时候,更是从早到晚桑叶不能断——一旦断顿,造成蚕不结茧,多日的辛劳便会化为流水——这个时候,养蚕的人家,连睡觉都没空,要不停地添桑叶,清蚕沙,只能抽空打个盹。也就在这个时候,市面上的桑叶往往会大涨价,逼得蚕户高价购买桑叶。

两浙的许多大户,有桑园,有丝厂,偏偏就是不养蚕。把最为繁重,也是最易出错的环节,交给普通百姓。而他们则是贵卖桑叶,贱收蚕茧,从中牟取暴利。

弃蚕、烧茧的情况,在两浙各地,已经不鲜见了。只是没有成规模,所以还没有被重视起来。但这并不意味着蚕农会一直忍耐下去。

景诚嗓音低沉,将路中州中的变化,没有任何夸大的告知于宗泽。

“长此以往,两浙必乱。依诚之见,与其让那些卑劣之徒盘剥百姓,不如由官府设立丝厂。”

“与民争利之事,朝廷不会做。朝廷刚刚收到润州大火消息的时候,就有人提出要官办丝厂。但章相公和韩相公都否决了。这不是铁路,也不是盐铁,是丝绢。朝廷管不来,也不能管。”

怨归工商,朝廷不承其责。若是怨归朝廷,那乱子可就大了。

“十株之内的桑树,不再计入家产。”

五等丁产簿,以家产计算户等。田地、房屋,还有农具,耕牛,都会折算进去,而桑树,只要数量超过标准,同样要计算在内,只有三五株的话,才会依律并不计算。

在宗泽南下时,章惇和韩冈都给了他一个承诺,承诺放宽计算户等的标准,用以安抚人心。

景城摇头道:“缓不济急。桑树成树要三年,等到三年后,不知多少百姓要倾家荡产。”

宗泽道:“终究是好事。桑树多了,可以多卖桑叶,也能贴补家用。”

“但眼下的事情怎么解决?”

“遇上洪水怎么办?”宗泽反问道,“依然只有一个办法。”

防民如防川,从来都是堵不如疏的。民生多艰,除了鼓励移民,宗泽想不到其他更有效的办法。

要富户、地主少盘剥一点,手段软了只会阳奉阴违,手段硬了反而会出更大乱子。相比起来,还是移民的手段最合适。

“只要能吃饱饭,就不会有民变。能吃得了做工苦的,那就去做工。不想做工的,那就移民。若是两样都不想做,只要他们能找到其他吃上饭的差事,朝廷自然乐见。流落街头,朝廷也会帮着他们移民。至于什么都不愿做,将罪责归咎于朝廷,受人蛊惑想要闹事的,朝廷也绝不会姑息。”

宗泽语气强硬。南下前,韩冈的赠言还有一句,宗泽没有说,但他相信景诚明白。这一回拿明教教众杀鸡儆猴,两浙至少能安定三年。三年后,桑树也长得差不多了。

“愚氓无知,视涉足他乡为畏途,终身不出乡者比比皆是。想要他们移民万里之外,还是太难了。”景诚说道。

“所以要教化。总不能因为他们愚昧无知,就放弃了教化。哪个读书人不是从一无所知开始的?白居易半岁之前认识字吗?孔门弟子,教化愚氓那是分内之事。”
