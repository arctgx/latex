\section{第18章 青云为履难知足(六)}

沿着左江往南,有多个军寨,还有一系列的镇子。古万、太平、永平,都是在左江江畔或是边境设立的寨子,陀陵、武黎、罗白,这些都是商旅富集的镇子,在这一次交趾北侵时,毁坏得最为严重。从韩冈派人查验的回报中,这些军寨和镇子,几乎都化为了白地。

几座军寨是邕州南方的缓冲区,是一道道防线,同时也是震慑周边溪峒蛮部的战略基点。重新设立寨堡,也是重新恢复大宋在左江两岸的统治。

韩冈召集左右江三十六溪峒,主要目的是先将邕州南方一直到边境的寨堡重新修起,同时整备好道路,并对交趾进行试探性的骚扰攻击,为大军南下做好准备。

可惜他的命令没有招来所有溪峒洞主。

回到州衙,韩冈接见了三位远道而来的溪峒洞主,说了几句勉励的话,便让吏员带着他们去安顿下里,好生招待。

空下来的官厅中,李信走了进来:“二十九家溪峒,总共到了二十七家。其中就算加上泗城州、思恩州还有忽恶峒,也只有十一家是峒主亲领。而没到的两个溪峒左州、忠州,到现在也没有一个解释。”

“不是还有十六家小洞主吗?”韩冈脸上笑着,只是眼睛里面一点笑意都没有。

李信也冷着脸。这十六家可都算是难得的聪明人,知道这时候贴上来少不了有回报。但他们派不上实际的用场:“那些都不能算数,加起来也比不上一家大溪峒。”

的确,虽然城中只有二十七个有官衔在身的溪峒蛮部,可此外没被召集的小溪峒则主动到了十六家,不过都是洞主带了十几人、七八人过来报到。

左右江地区说是溪峒三十六,但实际上的溪峒大大小小有数百家之多,不过列名于邕州籍簿之上、得受官衔的溪峒洞主,总共是二十九家。这些得朝廷官职的洞主,要么是把持着战略要道,要么就是麾下户口众多,皆是有名有姓的大部族。而那些小溪峒,则就是只有数百户口,甚至百来户口的小村落,几乎可以忽略不计。

“右江来的大多都是洞主亲领,而左江就不是这样,全是托他人代领,借口都还一样……称病!”韩冈冷笑连连,左江诸峒的洞主在想什么他哪能不清楚,“都是怕着被征召起来与交趾死拼,交趾贫瘠,拼死拼活也没有个赚头。”

右江是往百色云南方向去,左江才是往交趾、广源走,韩冈需要的是左江诸峒的全力支持,这样他才能重建被毁坏永平、太平、古万诸寨。

“这一干贼蛮,都是欠敲打!如果是在荆南,州中一封公文下来,哪一家敢推脱半分!?”李信恶狠狠的说着,“武陵也说是溪峒三十六,实际上有四十五家,这四十五家没一家敢违逆官府。”

“溪峒和溪峒不一样!谁让这百来年,官军的刀子没砍在他们身上?能来二十七家,到场十一个洞主,这还是看在我赢了李常杰的份上!”韩冈冷哼了一声,“换作是琼崖的三十六黎峒,琼州派人去召集洞主,恐怕一家都不会亲自到场。”

溪峒是南方蛮部的另外一种称呼,据韩冈所知,不仅是广西、荆南,就是海南岛上的黎族人,也是有溪峒之称,被称为黎峒,也有个三十六家的数目,似乎是惯例一般。

当年跟韩冈合作建立棉布买卖的秦州几大商户,曾遣家中亲信不远万里去找传说中善织吉贝布的黎人,到过海南岛。可去了十几个,只回来了一半。不过几个都是有本事的,将黎人所珍藏的纺车织机给瞧了个遍,只是回来后打造出来的机器,远远比不上如今已经经过几次改进、效率几十倍上升的纺纱机和织布机。

倒是琼崖当地的风土人情给传了出来。其中有一个还骗到了一个洞主的女儿,最后打通了琼崖海货的商路。

到了晚间,韩冈下令设宴招待已经抵达的洞主和洞主代表们。

“不等忠州、左州了?”韩冈命令一下,黄全就惊讶的问道。之前几日都没有设宴,今天宴席一开,可就代表客人到齐,正事要摆上台面了。

“他们还会来吗?!”韩廉反问着。

“倒是为什么忠州、左州两家会不来?”新上任的军事判官很是疑惑。靠着临阵反戈、一起说服黄金满的功劳,何缮捡了个大便宜。

忠州、左州两个大溪峒就在左江边上,离着邕州也不算太远,他们始终不到,本身就透着诡异。

韩冈击败了接近十万的交趾侵略军,让李常杰丢盔弃甲而走,他凭着这份战绩,以广西转运使的身份发话,按理说是左右江三十六峒洞主不敢驳韩冈的面子,好歹派个人来应卯,可左州、忠州两家偏偏还是没到。

“会不会跟去年古万寨被袭有关。当初父亲就怀疑他们是听到了风声,所以事先下手,省得给……”黄全顿了一下,跳过后面的话继续道,“说不定他们就是元凶,因为心虚,所以不敢应招前来。”

通判道:“也有可能是想看看风色再说。不过现在连泗城州的洞主都到了,他们应该很快……”

韩冈抬手打断了通判的话:“他们怎么想的我不管,我只知道他们没有到!”

藏在一对黑褐色眸子中的眼神森寒如冰水,让邕州通判连汗都冒不出来了。

“李信。”

李信一抱拳:“末将在!”

“就按事先说的办!”韩冈早有安排,不会眼睁睁的看着自己送去的命令被人当成手纸。

李信当然了解韩冈的计划,他也是参与者。终于等到了韩冈下令,沉声冷喝:“末将遵命!”

“……要小心一点。”

“是!”

当夜的宴席上,李信并没有出现,聚集在州衙厅中的洞主们也没人知道李信不至意味着什么,更不清楚城中少了一千兵马。他们只知道韩冈摆出来的酒菜美味无比,而年轻的广西转运使对他们很是和善,宴会上的气氛也让人兴致高涨。

醉饮一夜,第二天,就是说正事的时候了。

再一次回到昨夜酒宴上的大厅上,欢乐的气氛荡然无存,几十名洞主或是洞主代表沉默着,等待着韩冈的发话。

韩冈的发言开门见山:“交贼前番入侵,虽然被打得丢盔弃甲而退,但古万、太平诸寨已经被毁,寨中百姓流离失所,听说有许多被你等收留,对此,本官要代朝廷向尔等道谢……不过现在交趾已退,我想他们也该返回家园了。”

来自右江的洞主事不关己,各自安坐着。而左江的洞主代表们则互相交换了一阵眼色,最后势力最大的安平州站起来说话:“前些日子太平寨被攻破,的确是有汉儿避入我等峒中。现在交趾人败退,大半都已经离开了。也就还有几户留在峒中,等小人回去之后,便将他们劝回乡里。”

“很好!”韩冈点点头,并不管他们现在说的是真是假,有这个态度就好,“左江上的榷场,旧时就在古万诸寨,商事往来也都在寨中,想必尔等也盼着早日重建。”

“那是!那是!”一众连声应承。

“诸寨重建之事,本官自会安排人手,只是需要一批护卫,来保住修造寨子的丁壮,这需要各位洞主的义举了。”

说是保护修寨的民夫,但实际上就是用来攻打交趾的队伍,这一点,所有人都心知肚明。一阵沉默之后,又是安平州的代表打头站起来,“既然是转运相公的吩咐,我等哪有不听的道理。小人家的洞子虽不大,可也能出一百精壮的儿郎!”

拥有上万户口、随随便便就能出动两三千兵马的安平州,很是大方的给了韩冈一百人。

由安平州的洞主代表领头,其他溪峒当然也就有样学样,一百、五十的凑份子。看他们如同割肉一般挤出来的痛苦表情上,之后送来的壮丁,其中不知会有混有多少老弱病残。

“一百五十人。”这是下石西州的洞主阿含报的数目,已经是二十七家大溪峒中出兵最多的了。反倒是十六家小部族,则是纷纷挤出了一百多士兵。

邕州通判的脸色越来越难看,这根本是打发要饭的,这点可怜的兵力怎么骚扰交趾。

只是韩冈来脸色如常。

一直以来,大宋对左右江三十六峒管束极少,遍地都是羁縻州,只是名义上臣服而已。就像是广源州,在名义上也是大宋的领土,但刘纪等人却是向交趾上贡,大宋朝廷也不管不问。

即便韩冈打退了李常杰,可旧时的印象还留在这些溪峒蛮人的心中,大概是以为只是交趾和大宋之间的战争,与自家无关。虽然不敢违抗韩冈的命令,但敷衍过去也就能了事了。与交趾人打起来,死的都是自家儿郎,何苦为他人拼命。

“也是好事,现在这些个峒主拿出来的兵马,是我招他们来的,论理邕州还要为这些兵马出粮饷。”事后,韩冈笑着说着,“等过两天,再看看吧,说不定能把粮饷给省了。”

接下来的几天,韩冈接连留着洞主们设宴款待,从宾州一驮驮的运酒过来。只是交换来的承诺,也就是将总兵力从两千五六,涨到了三千人的数目。

到了第六天,韩冈再一次将洞主们召集起来,同样的地点,同样的人物,只多了几个摆设。

一排首级,摆在了大厅中央。

“忠州、左州,不从本官号令,竟敢迁延不至。”韩冈环目一扫静默下来的厅中,“从今往后,三十六溪峒,就没有这两个家了。”

