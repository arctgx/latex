\section{第39章 欲雨还晴咨明辅(15)}

赵挺之没有在宫门处看见韩冈和他的元随。

一名名宰执从他面前进入宫中,王安石和韩冈始终没有出现。

韩冈当真坚持辞官了。

之前的谣言得到了证实。宣德门前,不少官员都在用隐晦的言辞议论着。

但重要的是韩冈为什么辞官。韩冈的请辞到底是怎么回事?赵挺之听了多少消息,都没能得到一个合理的解释。

在拥立新帝登基之后,原本针对韩冈的攻击,完全都可以抹去。

也许会有人猜测韩冈只是故作姿态,等待皇后挽留。上表请辞,并不一定当真要辞官,绝大多数时候,只是想要表达自己的态度,要天子给一个说法。但气学求的是真,求的是实。道理从真实中来,行事也需真实不伪。韩冈若是那样做的话,他多年来积累的名声就完了。

更多的人则认为,韩冈有拥立之功,之所以依旧会辞官,那是他断错了皇帝的病。

那句皇后害我,也同样传得很开,只是市井中无人敢于公开议论,只是最开始流传一下,很快就消失了。但人们回到家里会不会说,那就是另外一回事了。

这到底是不是真相,世人说得信誓旦旦,可赵挺之却抱着深深的怀疑。

现如今,只有王安石、韩冈翁婿辞官是确凿无疑的事实,其余的真相,没有人会出来证实。

“朝廷怎么可能会明说太上皇有疾?”赵挺之昨天指着自己的心口对同僚李格非叹道,“遮掩还来不及。”

一个只能靠眼皮和手指与外界交流的病人,想确诊他到底疯了没有,什么名医都没用。只有身边人最清楚。太上皇后说太上皇疯了,那就真的疯了。哪个臣子还能上去为太上皇抱不平?

在这件事上,就算是可以风闻奏事的御史,也不敢涉足太深。帝位传承,事关身家姓命,可不是能图嘴上快活的事。

章惇的旗牌渐近皇城。

骑着一匹身高体健的河西良驹,知枢密院事正用目光梭巡着人群。

宣德门前,三五官员聚在一起,原本应该整顿秩序的御史对此都视而不见。

在自己接近的时候,还没进门的官员们都望过来,但看清楚了身份后,又都转了回去。

看到这一幕,章惇哪还会不知道究竟是怎么一回事?

毕竟是高下有别,能够了解到的消息差得很远。

御座上的变化,带起了政斧中的大变动,就是市井中的愚夫愚妇都能知道发生了什么,但想要知道到底是为什么,即便是朝堂上的官员不一定够资格。各种各样的传言,会将真相搅得让人无法分辨得清,没有可靠的消息来源,猜测就会变得远离真相。

民间只知道,韩冈是断错了太上皇的病,加上之前所受到的攻击,然后引咎辞官,王安石的辞职也是心怀愧疚的缘故;普通的朝官,则知道苏颂上位是让韩冈辞职所给出的交换条件,可见并非引咎,而是王安石和两府联手的结果;地位更高的重臣们,则更清楚韩冈在皇后心中的地位,在皇后的支持下,韩冈还会请辞,两府给出的压力可想而知,不会有商量的余地,至于苏颂,是两府主动示好,并非交换;唯有早就成了一条线上的蚂蚱的两府中人才清楚,韩冈根本就是在无人逼迫的情况下主动辞官的,王安石的辞官是自责,苏颂的位置则是韩冈拉下吕惠卿的回报;而章惇最为了解,韩冈辞官是为了他心中更重要的目标,官位并不放在他的眼中,这是其他宰辅所不能理解的地方,无不认为是借口,最多也只是认为是很小的一部分理由。

猜测终究只是猜测,外人的议论,无论如何都造成不了什么影响,也只是图个口舌痛快罢了,大多数人都与这云霄上的变动牵扯不上任何关系。

也只有到了章惇这个等级,方才是息息相关。王安石和韩冈相继辞官,短时间内不会再入朝堂。西府之中,薛向的发言权远小于章惇,进来一个不好权位的苏颂,总比喜欢争权夺利的同僚要强。

章惇想着,跨马进入了皇城之中。

………………赵煦正面安坐,向皇后在侧后垂下一道帘幕。

王安石不在,韩冈不在,崇政殿中的人数比之前变少了,不过马上就要多了一个苏颂。

对苏颂代替韩冈进入西府,向皇后很有些看法。

向皇后接触过苏颂,但次数不多,有些了解,却也不能算是深入。苏颂的才学,她是知道的。是朝中数得着的饱学之士,学问偏近气学,故而跟韩冈走得甚近。曾经担任过权知开封府,好象是因为断错了案子而不得不辞官。

两府宰执们能同意他入西府,肯定不会一进来就要大展拳脚的姓子,多是如同薛向一样,在许多事情上保持沉默,只管着自己的一摊。只是朝堂上不是养老的地方,苏颂能不能顶替得了韩冈,那还要看他进了西府后的表现。

不论是什么缘故,年纪轻轻就能身登两府,总是有能力的,不可能只靠机缘或关系。眼下的吕惠卿、章惇、韩冈就是最好的例子,再往前,韩琦、寇准同样都是明证。坐上宰执的时候越是年轻,能力当然就越强。苏颂的年纪太大了,现在才被韩冈推荐上来,纵有才学,但治术肯定就不行了。肯定是比不上韩冈。

但看在韩冈的面子上,也只能忍耐下来。

第一件事,永兴军路提刑使司奏论吕惠卿之弟升卿不法事。

向皇后将奏章拿出来向宰辅们征询意见,然后蔡确就出来开始发表看法。

向皇后冷淡的听着蔡确的‘意见’。心中无聊的甚至觉得可笑。

这是预定的计划。

两府中的职位并没有任期的说法,但凡离开的宰辅,都是主动请辞。如果不主动,就设法让他‘主动’。

郭逵好办,暗示两句,就会立刻递上辞表。而吕惠卿就有些难办,他在陕西是有功无罪。

所以就必须弹劾一下,否则就不方便将吕惠卿请出两府。除非吕惠卿像王安石和韩冈一样主动辞职,否则就只会是一封弹章开头,设法让其主动请辞。

没有哪位官员是干净的,就是本身没有问题,身边的人一样找得出来。看的只是需不需要。

如果有人事后为其叫屈,还有韩冈这个例子压着阵脚。军功、拥立、名望一个不少,照样辞官,吕惠卿至少还差一点。

向皇后不在乎吕惠卿心里怎么想,只要不回来就行了。现在在她看来,但凡没有经历过内禅的宰辅,都不可深信。苏颂是韩冈所荐,那还好一点,以精明厉害著称的吕惠卿就让人放不下心了。

就比如王安石,之前还做平章的时候,向皇后都不敢违逆他。他的脾气执拗起来,两府都得向他低头,韩冈几次三番的想要让张载入京,都是让王安石给压着。就连韩冈都赢不了他岳父,王安石所看重的吕惠卿又怎么可能是简单人物?

这个弹劾只是走过场,接下来就要看吕惠卿识趣不识趣了。

依照正常的程序,重臣辞官,不论是因为什么理由,就算是犯了大错,只要不是贬责,其进呈的辞表,天子都要驳回至少三次以上,才能批准。这是优待重臣的惯例。

最近一次例外的情况,还是英宗皇帝的时候,而对象是在仁宗立储问题上开罪了英宗的蔡襄——仁宗晚年病危无子,朝臣纷纷上表请求立英宗为储,只有蔡襄一人没有上书。

越是想要表现出对辞职臣子的看重,就越要多否决几次辞表。

吕惠卿真正要卸任,多半还要一两个月时间。

从吕惠卿再想到韩冈。眼下以韩冈的情况,至少要半个多月的功夫。而以王安石的地位,至少还要比韩冈多拖两天才是。

向皇后很想问一下能不能快一点。

国家大事有多少时间可以耽搁?那个什么火器局,如果是跟板甲一样有用,那肯定是国之重宝,当然要问韩冈。百官三军的封赏,官员还好说,那等军汉不是曾经统领过他们的韩冈,肯定压制不住。辽国攻打高丽,这更是要征询一下韩冈的意见。

只是她担心这么问了,会让韩冈成为众矢之的,好像离了他,两府就不做事了。就算关系最好的章惇心里都不会痛快。

而且韩冈不来,该面对的是还是得面对。

高丽使者就要到京城了,但怎么处置高丽,宰辅们最后还是决定等待高丽使者到了再做决定。而且还得看看高丽国能不能抵挡得住辽军的进攻。也就是说,依然是再议。现在所做的,只是发文让登州准备好船只,以便可以随时运送一批军械过去。

没人对高丽的安危太过在意。就算高丽被灭了,对大宋来说,也不过是远在海外的事。需要做的,也就是花点钱,花点时间,从军库的最底层拿出些刀枪弓弩和甲胄,去支持高丽复国。少说也有一两百万的高丽人,哪里可能那么快就被辽国征服?以契丹人的德姓,就是再老实的农户,都会给他们逼得走投无路。只要有人能揭竿而起,高丽肯定是处处烽火,有的是让耶律乙辛头疼的时候。

牵制辽国,这就是大宋对高丽的期待,谁管他是怎么牵制的?就这么简单。反正按照韩冈的话,那些都是禽兽,它们争地抢食,人只要在旁边看着就是了。有必要的话,干涉一下,没必要那就是另说。

结束了对高丽的处置,韩绛出班道:“殿下,陛下,年号之事,太常礼院亦已进呈,还请早曰做出决定。”

