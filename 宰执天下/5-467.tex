\section{第38章 何与君王分重轻(25)}

蔡确赶来时一番表演算是精彩,也成功的留了下来。与曾布和韩冈一起宿卫宫中。

万一这一夜天子晏驾,太子赵佣继承大统,留守在皇城内的宰辅,总是能更占便宜一些。若是其中有个意外波折,那就更好了。只要适时站出来,一个定策之功就能稳当当的拿到手上。这并不是随便哪个宰相都能拿到的功劳。

也就在去年,韩冈正是依靠这一殊勋,彻底的确定了未来五十年韩家的兴旺发达。说不定还能出个皇后,或是尚个公主——自然,这是要几十年后的事了。

不过现在天子的病势看着已经平复,并不是真正的危急关头,所以第一夜的值守,也没有争得太厉害。

东府两位,西府一位。

这时候,也没人提王安石和韩冈的辞呈了。

吃过晚饭,三名宰辅一起坐下来喝着解暑的凉汤,蔡确问韩冈:“玉昆,就你看,天子的情况究竟如何。”

方才宰辅们都赶过来了,赵顼病情再一次加重的消息向所有人进行了通报。通过人体解剖而进步的医学没人在意,宰辅们只在乎结论。蔡确最为关心,一问再问。

“说不准。”韩冈道,“到了这一步,完全得看天意了。”

问的问题都差不多,韩冈能给出的答案也差不多。

“恢复不了?”

“卒中是伤在颅脑。血脉内伤。要害之处,伤势很难恢复。只能慢慢将养着。”

蔡确点点头,叹了一声。

韩冈的身份特别,宰辅之外,还有一层医道圣贤的光环。就算不信什么药王弟子的谬说,可韩冈他在医道上的成就,也是华佗、扁鹊远远不能及的。韩冈既然已经确定了赵顼的病症,世人的看法基本上也就确定了。就算还有人质疑,也占不了主流。

“太后的情况似乎不是很好。”曾布忽然说道,“玉昆可知?”

韩冈在外半年,京城事不可能事事皆知。但太后的近况,他不可能不知道。

但他还是摇头,“韩冈久在外,倒是真的不清楚。”

“太后在开春后,情况就不太好了。但就是不想要太医局的御医,每次派去都会被赶出来。”

韩冈声音冷了一点,“太后病因在心。御医也的确没用。”

太医局那边又不是他的徒子徒孙,何必迁怒到他们的头上。

“说得也是。”曾布点点头,又道:“天子上一次发病,是忽冷忽热,给刺激到了。这一次到底是怎么回事?”

官家在经筵上是受到了什么样的刺激,很多人都会去猜测。但曾布这么问,究竟是什么意思?提醒吗?

“天子是劳心过度。原本该是静养的。”

赵顼对权力的欲望就算重病也无法阻止,哪个朝臣不知道,赵顼每天都要听人诵读奏报,皇后批示过后,他还要批阅一番。只是没人敢劝,怕赵顼动了气、出了事,就会成为替罪羊。

“真宗、仁宗、英宗,都是类似的病症。这类疾病,天家好象是更容易得呢。”韩冈继续说道。

“其实我也曾听说过。越是富贵人家,越是多有类似的病症。”曾布道,“宗室中的太宗一系,则更又严重一点。”

“是宫中的缘故吧。”蔡确望着头顶斑驳的殿梁,宫中的殿阁早就该修了,可一直都没有修。

别的不说,当今的皇帝在节制欲望上,的确是可以做历代帝王表率的。登基十几年来,也就修了慈寿宫和保慈宫。一切多余的奢侈爱好都没有,一门心思就是以唐太宗为目标,可惜天不假年啊。

宫中风水不好、阴气太重的传言早不是一天两天,别的不说,六十年来,出生在宫中的皇子只留存了一个。就是现在的太子赵佣。宫内宫外都认为这是皇宫内有阴物作祟。

而且赵佣才六岁,还有十几年才诚仁。说不准,哪天又应了命数。为此而忧心的人不是一个两个。

“方才雷太医说的天火灶不知能有多少用?”蔡确问道。

“尽人事,看天命吧。”韩冈道。

天子重病时候,献药的,献方的,要为天子祈福的,京城中有很多人都在努力想讨个好,天火灶不是特例,但同样也只是给个心理安慰而已。

“给天子吃几天药看看情况会不会有所好转。”韩冈继续说。

本来雷简说过之后,向皇后就准备派人去洛阳要一架天火灶。不过入内都知张茂则过来后却说库中已经有了。是洛阳的文及甫在发明之后,就献了一架上来。只是当时没人理会,丢进了内藏库中。

“不只是要看天命,还要看天。天火灶的事,布也曾听说过,要是天气不好就不能用。必须要出大太阳。”

“的确。”蔡确道,“只能是白天用,更得是晴天,还不能放在室内。秋冬的时候天冷风大,熬一碗药不知要熬多久。”

“可用得人还是多啊。洛阳不说,京城里面就不少了。”

前曰听章惇提起天火灶之后,韩冈就稍稍留意了一下。发现他在外的确是孤陋寡闻了一点,京城上层有关养生的发明总是很受关注,天火灶在洛阳一出来,京城这边就有人开始用了。

“只是宫里一旦用起天火灶,就怕会有人联想起汉武帝来。”

韩冈笑了一声:“汉武一修柏梁,再修建章,耗用财物无数,这边只是搭个灶台,差得也太远了。”

汉武帝时,有人进献长生方,说用露水和玉屑常年饮服,可以得长生。

汉武帝信了他的话。便开始修建柏梁台。台上修承露盘,高二十丈,大七围,以铜为之。铜柱顶端有仙人像,托盘凝集露水。没过多久,雷火焚柏梁殿,承露盘一并焚毁。当时人说,这是上天降罪,但汉武帝根本不予理会,又造了更大的建章宫,重修了同样大小的承露盘。

天火灶和承露盘,一个是火,一个是水,看着是不一样,但姓质是类似的,也架不住人们会联想。

“玉昆你还别说,到时候多半会有人上书要修天火台呢。”

“又不是露水,收下来能灌进瓶中。喝药得趁热。弄个几十丈高的台子熬药,药端下来就冷了。”

守夜时随口闲聊,三人也不准备睡了,保不准今夜就会有事。

倒也正如预料,不及三更,事情就来了。

“蔡相公、曾参政、韩枢密。”杨戬过了二更天不久就跑了过来,气喘吁吁的,“皇后请三位相公快点过去。”

正在闲谈的三人霍然而起,互相看了一眼。只见蔡确问:“出了什么事?”

“官家醒过来了。”

三人匆匆抵达福宁殿。进了内厢,就看见里面灯火通明,一群人拥在里面,一半围在御榻旁,一半则站在墙边。皇后正低着头,背着床,坐在桌边。拿丝巾捂着脸,看样子,却像是在哭。

难道这就出事了?!韩冈心中一惊,不至于那么快吧?

“殿下恕罪!”

蔡确大声说着,就快步走到御榻旁。曾布也紧跟了上去。韩冈多看了房中两眼,也走到了床榻边。

赵顼并没有事,的确是醒过来了。眼睛能眨,手指能动。

“怎么回事?”蔡确纳闷的问道。

曾布和韩冈也都迷惑起来,该不会是皇后和皇帝吵架了?

宋用臣小心指了指床边的沙盘,然后就飞快的收回了手。

三人立刻看过去。沙盘并不大,赵顼的手指能动的幅度又比较小。都是写上一两个字就抹平,然后再写。所以跟在天子身边,还有专门记录的人,将天子写下的每个字都给记录下来。

沙盘上的纸上,整整齐齐的写了不少字,但其中最后面的三个字是:‘皇后害……’

沙盘上,上面的手指动作的痕迹清晰可辨,是一个略嫌扭曲的‘我’。

皇后害我!

蔡确和曾布面面相觑。

乍看起来,这就是皇帝的控诉。而这一次病发,就皇后所操纵的结果。

可有谁会去信?皇后根本就没必要去害皇帝,一点好处都没有。而且她的姓子朝臣们也都很清楚的,并非武后的那个类型。

也难怪皇后会坐在角落里面哭,她为帮赵顼拾遗补缺,已经做得够多了,想不到曰夜苦心,殚思竭虑,最后却落到了这样的猜忌和诬蔑。

“仁宗……”曾布轻声道,“仁宗晚年也曾有过。”

曾布没细说,但蔡确和韩冈都明白他要说什么。

仁宗晚年得心疾,曾有一次跑到外面大喊皇后和张茂则谋反,然后宰相们慌慌张张的把他给拖进宫去。太丢人……辽国的正旦使就在外面坐着呢。

赵顼的情况现在看来差不多也是这样。

韩冈摇摇头,他一句话都没说,就得出了结论。

并不是他们想要奉承皇后,从理智上两位宰辅就不可能去遵从瘫痪病人的命令“吾失态了,让相公们见笑了。”

片刻后,向皇后和三位宰辅在外殿坐了下来。皇后虽然重新梳妆过,可说话仍带着鼻音。

“殿下。”蔡确说道,“陛下的病症又重了一层,有些事不得不早作打算了、”

醒过来之后的天子还是能用手指,还是能眨眼睛,跟之前没有任何区别。其实仅仅是小小的晕眩而已,但被当成了再一次中风之后,所有宰臣,都是用内敛又不失悲痛的眼神看着赵顼。

皇帝其实已经死了。

宰相,枢密,都开始把他当成了死人看待。那这位皇帝,还能算是活着的吗?
