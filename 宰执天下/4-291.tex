\section{第47章 天意分明启昌运(下)}

【写到一点熬不住了,码字的速度越来越慢,满篇的字都不成句了。睡了三个小时起来才写完。】

“这……”赵顼犹豫着,尽管有关厚生司保赤局在京城种痘的动向,一举一动都会传到赵顼这里,但他还是不敢冒险,“听说京西这段时间种痘,唐州又有一小儿,在种痘后暴毙……孙儿还是有些不放心。”

“也不是说种了痘,就不会得其他病,小儿暴病夭折也不全是因为痘疮。数万人里面才出几个,只能说他们命不好。六哥能托生在天家,是真有福分,不会有事的。”

赵顼点了点头,“孙儿知道了。”却还是没有肯定的答应下来。

又说了几句闲话,赵顼不敢让曹氏太过劳累,就起身告辞了。

赵顼走后,曹氏卸了装束,又躺了下来。问着身边的内侍陈醒:“官家这些天是不是还是去刑氏那里多一点?”

“官家一向心肠软,刑娘子痛失爱子,多去陪一陪也是常理。”陈醒突然压低声音:“不过刑娘子这些天时常对人说,如果韩冈能将种痘法早几天献上来,七哥就不一定会有事……”

曹氏摇摇头,“官家只是心疼她,但心中自有主张。”

宫中的人都是眼明心亮,皇帝在六皇子的种痘上又是怎样的犹豫不决,人人也看到了。就是韩冈将牛痘早些天献上来,肯定是先用上几个月在京城试行,哪里来得及赶得上给七皇子种痘。

陈醒低声:“刑娘子说的不是牛痘,是人痘。”

“大损阴德之事,天子如何能用。焉知人痘是不是上天的试探?韩冈这件事做得对!宫中本就六十年无皇子长成,再损了阴德,还想多少年没皇子?”曹氏又叹了口气,脸上多几分悲戚“皇嗣不保,又岂在一个痘疮,仁宗皇帝夭折的那么多子嗣,没一个是因为痘疮,是上天不留啊!”

……………………

吕惠卿收到消息时,元绛已经派了人去都亭西驿。

对于天子的任气之举,吕惠卿根本就没放在心上,反倒是对天子不找他,而找元绛有几分不满。不过与藩属入贡有关的公事,是元绛的职权范围,吕惠卿也知道,天子找元绛是情理中事。

驱逐西夏使臣,在过去并不鲜见。党项人一贯如此,一边抢钱抢粮,一边派人上京要钱要物,顺便还要增加岁赐。英宗和今上的脾气都不必仁宗,怎么样也忍不下这个口气,好几次将西夏使臣遣送回过。

不过在熙宁五年之后,西夏人就老实多了,天子也不为已甚,不与他们计较。只是如今西夏国主摆明了要投靠辽人,那么也没必要再与他们敷衍。

何况就是党项人再愤怒又能如何,釜中游鱼,灭国也是指日可待。秉常的心情根本就不重要。

步跋子和铁鹞子,是西夏步骑的两大主力。铁鹞子是党项族为主体的骑兵部队,而步跋子则是横山蕃组成的步兵。在横山南麓尽入宋人之手,北麓蕃部人心向宋的情况下,步跋子已经土崩瓦解。

而且横山蕃还是侵宋时粮秣的主要来源,没有了横山蕃部的支持,党项人过了瀚海之后,光凭银夏的出产,只有饿死的份。

不仅仅是横山,在宋夏两国接壤的地区,所有在那里生活起居的部族,都已经投向了大宋。

兰州的禹臧花麻从河湟开边、熙河路成立之后,多少年了,一直都与国中有联络,只是因为兰州城中有六千铁鹞子,暂时还不敢翻脸。可一旦朝中决定夺取兰州,兵发兰州城下,禹臧花麻会立刻倒戈一击。而且他最近写来的密信上面,说了许多有关西夏朝堂内乱的事,就差明说恭迎王师了。

“吉甫,你怎么看?”

王珪的讯问,让吕惠卿回过神来。王珪和元绛的视线都投了过来,政事堂的正衙中,三名宰执在座,这是每天的例行会议。

“还是报与天子圣裁比较好。”吕惠卿没注意正在议论的是什么话题,但说一句呈交圣裁是永远不会错的,尤其唯一的宰相还是王珪。

王珪狐疑的看了吕惠卿一眼,却也不反对:“那就呈交天子。”

两名同僚敲定,元绛更不能反对,“也好。襄汉发运使的人选就让天子来决定。”

‘原来说的是这件事。’吕惠卿这下才知道方才在讨论什么。不过他对沈括没好感,襄汉发运使到底安不安排沈括出任,吕惠卿并不在意。

“都亭驿那里的情况怎么样?”吕惠卿喝了一口茶,问道。

元绛刚想说都亭西驿已经派人去了,突然反应过来,“是都亭驿?”

王珪也是愣了一下神后才反问:“……枢密院那边什么时候会知会中书?”

掌北界国信诸务的是枢密院北面房,与辽国之间的外交事务,一切归于枢密院掌管。这是因为与辽国的交往,不属于朝贡体系的缘故,两国的地位相当,互称南朝北朝。

而西夏在立国后,虽然与大宋战争不断,但因其名义上向宋称臣,属于藩国之列,故而与其外交关系,一直在中书门下的辖下。

元绛也接口道:“且有陈绎这位翰林学士作陪,更不关中书的事。”

选派馆伴使是按照国家的份量来的。陪辽国使臣的通常是翰林学士,高丽和西夏平级,再后面,就是真腊、三佛齐、回纥之流。翰林学士是天子私人,掌管内制,中书门下管不到学士院,只能管着外制的中书舍人。

吕惠卿笑道:“惠卿只是想知道如今辽国的朝堂上到底怎么样了。耶律乙辛害死了故太子,辽主迟早会明白过来。如果辽主处置耶律乙辛一党,其朝堂必有乱局。攻打西夏,当是时也。”

“等到河北轨道建成,大名守军两三日内可达三关,辽人也不足为虑了。”元绛道,“眼下还是让翰林学士继续接待好了。”

“说得也是。”吕惠卿微微一笑。

各自低头喝茶,静了片刻,王珪忽而开口:“说起翰林学士,倒有件有趣的事不知你们发现了没有?”

“什么有趣的事?”元绛问。吕惠卿也放下了茶盏。

“最近几年的翰林学士,有不少名讳从糸的。韩持国名维、陈和叔名绎、韩玉汝是缜,之前有邓文约——绾。”王珪停下话来看看元绛,笑道:“厚之也是一个。还有杨元素,杨绘!可惜在韩玉昆身上栽了个跟头。”

“这还真没注意。”吕惠卿侧过脸对元绛道,“厚之,的确是如此啊。”

元绛看了看王珪,又看看吕惠卿,道:“其实此事,元绛惊异已久。”

“此话怎讲?”王珪和吕惠卿一齐追问。

“少年时,元绛曾梦人告之:‘异日当为翰林学士,须兄弟数人先后入禁林。’自思素无兄弟,疑此梦为不然。直到数年前,得除学士,同时相先后入学士院者,便是方才所说的几位。由此方悟弟兄之说。”

王珪和吕惠卿两人交换了一个眼神,心中皆是不信。但元绛话既然说出来了,也没必要指着说骗人。

王珪笑道:“看来厚之能入东府,乃是上天注定。”

“听到厚之的话,倒想起韩玉昆的事了。”吕惠卿与王珪有着一模一样的笑容,“他的遇仙说不定是梦中所授,要不然这些年来,那位孙道人早就该出来了。”

“还真说不准。”王珪也点头附和。

聊了一阵闲话,又该说正经事。

明天地方州县就要封印了,等过了年后,而一般的朝臣,也只是正旦大朝会才要上朝。但中枢两府就不肯能有这么好的运气了,得照常入崇政殿,而且夜中还要轮值。过年的一个月,总是事情最多的时候。

而开封府也是一般。

吕惠卿道:“许冲元刚刚接手开封府,接着就要过年了。今年年节时的城中巡夜,还不知他怎么安排。”

“苏子容之前已经安排好了吧?”王珪还记得苏颂的安排,“他前两天还上了奏本,依旧年故事,城中临时增加一百二十七个潜火铺。”

“希望能管用,今年的火灾能少一点就好了。”吕惠卿想起开封府每到冬天就紧张起来的样子,不禁心生感叹。

元绛经历过的火灾更多:“没有就最好了。”

王珪摇摇头:“开封府每逢过年,都少不了有火灾,不指望没有,只要能少一点就够了。”

说几件正事,跟着就又说两句闲话,过年前的议事,总归是有几分悠闲。用了一个时辰,对几件重要的公事进行了沟通,三名宰辅就准备分头回自己的官厅去。岂料外面通报,检详枢密院兵房文字薛昌朝带着名通进银台司的小吏在外求见,说是有要事通传。

三人心中起疑,一齐坐下来,招了薛昌朝进来。

薛昌朝进来时还是领着那名小吏。在三位宰执面前,小吏就有几分慌张,张开口要说话,却结结巴巴的不成语调。

“怎么了?”王珪皱眉问道。

“慌什么!”吕惠卿呵斥了一声,问薛昌朝,“出了何事?!”

小吏被两位宰执呵斥得舌头打结,惨白着脸半天也说不出话来。一同进来的薛昌朝,代他出来说话了。声音不大,却让整个政事堂正衙都安静了下来,“通进银台司消息:雄州急报,辽主驾崩!”

