\section{第四章 惊云纷纷掠短篷(一)}

“陛下发钱以本业贫民,则曰‘赢得得儿童语音好,一年强半在城中’;陛下明法以课试官吏,则曰‘读书万卷不读律,致君尧舜知无术’;陛下兴水利,则曰‘东海若知明主意,应教斥卤变桑田’;陛下谨盐禁,则曰‘岂是闻韶解忘味,尔來三月食无盐’。”

烛光下,吕惠卿读了几句抄来的舒亶弹章,屈指弹了一下这张不大的纸片,冷笑着:“李资深这是恨苏轼不死啊。”

“这不是舒亶写的吗?”吕升卿疑惑道。

吕惠卿冷眼的瞥了弟弟一下,话都懒得说一句。

吕升卿怔了一下,明白了过来。舒亶完全是在配合李定的奏章来写。

李定在弹章中说苏轼‘所为文辞,虽不中理,亦足以鼓动流俗,所谓言伪而辨’,舒亶就在自己的弹章中说苏轼‘讥切时事之言,流俗翕然争相传诵’。李定说苏轼‘腾沮毀之论,怙终不悔,其恶已著’,舒亶就将苏轼的诗文一句句的拿出来细细分析给天子看。

两人一唱一和,加上一干很快就要参合进来的御史,看着声势当是要置苏轼于死而甘心。

“今日听传闻,说李定之子年前曾过其门,苏轼依故事设宴,但在席上却冷嘲热讽,说‘好一个呆长汉’,李定之子是大惭而退。”

“……”吕惠卿沉默了好一阵,半晌之后摇摇头。都没什么好说的了,“此事若为真,李定衔苏轼入骨,倒也不为过了。李定之子乃是后生晚辈,纵是厌见其人,遣人代为主席便可,岂可如此行事。苏轼轻佻如此,实是有失体统。”

“李定遣其子过苏门,或许主动化解旧怨的打算。当年毕竟是苏轼攻李定,不得李定首肯,其子当也不敢赴苏轼之宴。”

“‘知其生不逢时,难以追陪新进;查其老不生事,或可牧养小民’。”吕惠卿叹了一句,“苏子瞻的文章的确不错。《知湖州谢上表》里面,这一句写得最妙……”顿了一顿,“这把好刀递到李定的手里,是给自己的棺材钉钉子呢。”

吕升卿叹道:“这一次苏轼的罪名肯定是小不了了。”

“文王拘而演《周易》;仲尼厄而作《春秋》;屈原放逐,乃赋《离骚》;左丘失明,厥有《国语》;孙子膑脚,《兵法》修列;不韦迁蜀,世传《吕览》;韩非囚秦,《说难》、《孤愤》;《诗》三百篇,大底圣贤发愤之所为作也。”

吕惠卿将司马迁的《报任安书》在这时候背出来,幸灾乐祸的味道就太浓了。不过他也是苏轼所说的新进,苏轼的文章传播得越广,自家的名声就被糟蹋得越厉害,只是幸灾乐祸,没有顺便落井下石就已经可以算是宽宏大量了。

“但以言辞罪人,御史台那里是不是做得太过了一点。”吕升卿并不是为苏轼叫屈,而是兔死狐悲,“一旦开了头,后人仿效之,谁还敢作诗?”

吕惠卿闻言,眉头突然皱了起来,很是有几分疑惑:“韩冈素来不做诗,是不是知道会有这一天?”

吕升卿也给带得疑惑起来,“……还真说不准,他的神仙弟子,肯定早就被叮嘱过了,不见他连医术都不学,省得被人找去治病,坏了神医的名头。就是孙真人,也不可能手上的病人一个都不死!”他越说越是肯定,“能中进士,又怎么可能连诗都不会做,那些村夫子还写诗呢,韩冈的才学好歹也比他们要强得多。就是不入第一第二流,三流总能挤进去的。”

“在章子厚家奔走的有个叫路明的,他当初跟韩冈一起进京……”

“西太一宫题壁?这小弟也听说了,路明也见过。他说整首诗都是韩冈所作。不过路明他还说了,韩冈后来自陈是在路旁废庙中所见。”

吕惠卿冷哼一声:“愚兄走得庙宇也多了,新的旧的,大的小的,市井中的,深山里的,怎么我没这个运气?好事偏偏给他遇上了!”

“韩冈不是都遇了仙嘛……神仙能碰上,撞上一个壁上有佳作的废庙,也不是不可能。”吕升卿回想了一下,道:“不过路明说他也曾问韩冈是在哪间庙里看到的,韩冈就没回话,说不定还是梦里撞进去的。”

“这一首,当是韩冈所作。”吕惠卿很肯定地说着,“当初与章子厚议论,他也是觉得韩冈写得出来。”

“可‘断肠人在天涯’,以韩冈当年的经历心境分明是写不出来的,他可是就要入京为官了!何况当时还是冬天,‘小桥流水人家’,在关中无论如何都看不到。”

吕惠卿哼了一声:“好好想想,韩冈当年从张子厚门下赶回乡里,到底是了为了什么!”

“啊……”吕升卿一下张大了嘴。

韩冈如今名震天下,遇仙的故事更是遍传海内。世人中十个里面倒有九个知道韩冈是两个兄长殁于王事之后,赶回家奔丧,然后病倒在路边的破庙里,遇到了孙真人。而韩冈说他看到那首题西太一宫壁,也是在破庙看到的……

“这下倒是能对上了。”吕升卿喃喃自语。

“两兄战殁,甚至是尸骨无存,仓皇间回乡奔丧。”吕惠卿慢慢的说道,“当时的心境难道还不是断肠人吗?”

吕升卿搓着下巴,缓缓的点头。

“此一篇《题西太一宫壁》,论文字,论格律,都不算高妙,但其意其境,却是动人心魄。甚至压倒了介甫相公。短短五句,不见华彩,却出乎意料的让人心生感触。要写出这样的诗作,并不要太好的文采,只关乎经历、心境,正好是韩冈这样的人能写的出来的。”

“大哥说得正是。”吕升卿连连点头,附和道:“并不是要有苏轼那样的才能写得好诗,就是韩冈这般文采平平的士人,心境到了,也能有一名篇传世。”

可吕惠卿忽得又皱起眉来,“怎么说到韩冈身上了。”

吕升卿眨了眨眼睛,也愣了。议论了半天韩冈的诗才,吕家兄弟才发现自己的话题莫名其妙的就偏掉了。

“苏轼之事大哥你觉得该怎么办?”吕升卿问道。

“现在还不是表态的时候,暂时由着李定他们闹去。”吕惠卿道,“御史台已经请了上命,遣人去湖州捉人了,有什么话等苏轼上京后再说吧。治他的罪,当能给州县中明里暗里反对新法的一干鬼祟之人一个警告,手实法推行也能更加顺利。但以言辞、诗文定重罪,这一点就万万不能了……不为苏轼,只为自己。”

“大哥说的正是。”吕升卿点头,“就是只为了自己,也肯定是要劝一劝天子。苏轼文才旷世,怎么也得保住他的性命。”

“……要真的这么说,苏轼多半就死定了。”吕惠卿声音低得很,没让他的弟弟听到。

……………………

韩冈刚刚赴了韩缜的邀约,在群牧使府上吃喝了一顿。前后十巡酒,二十道正菜,加上甜点、菓子,凉菜,对韩冈来说,实在是丰盛得过了头。灵寿韩家的豪富,也总算是领会到了。

在席面上,两人并没有说多少公事,只是天南海北的聊着天,说着不着调的闲话。

韩缜请韩冈,也只是联络感情的打算,都在一个衙门里面共事了,没坐在一起喝过酒,怎么都是一件奇怪的事。

韩缜早就想请韩冈一起饮宴,也正式出言邀请过。不过韩冈如今绝足欢场,对于一些脂粉味太重的酒楼敬谢不敏,韩缜等到过了年了,才邀请了韩冈过府一叙。

虽说在席上并没有论及正事,但一顿酒,喝得宾主尽欢。到了初更的时候,韩冈才带着几分醉意,告辞离开。

迎面吹来了一阵夜风,韩冈裹紧了斗篷之后,酒意也被冬夜的凛冽寒风给吹得不知踪影。

明天就是上元灯会的初日,街巷中到处都是各色彩灯。有挂在屋檐下的,有拴在树梢上的,还有直接摆到了大街上——通常有两三人髙,数丈长,这是灯山。只是大部分的花灯还没有点起来,在风中摇摇晃晃。不过少部分亮起来的花灯,已经足以用流光溢彩来形容,照得街上通通透透。

韩冈一行十余人,都是骑在马上,转过一道街口,前面便是南门大街,韩冈回他的宅邸,都要经过宽阔的南门大街,虽不比宽敞得如同广场一般的御街,但五十步的大街,也是可以当广场用了。

此时的南门大街两侧,摆满了灯山,不是之前看到的民间行会所造的灯山,而是在京百司的灯山——地位高的衙门能摆在御街之上,地位低的,就只能在南门大街,以及东十字大街,西角楼大街挤一挤了——这些拿着官中的钱扎起的巨型彩灯,外形各不相同,有的是描述了一个有名的历史故事,有的则是天南地北的飞禽走兽,看了就给人以争奇斗艳之感、

