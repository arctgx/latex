\section{第12章 平生心曲谁为伸(八)}

【祝各位书友新春快乐,大吉大利。】

韩冈刚退了两步,一匹河西骏马就一阵风似的几乎是擦着他的鼻尖掠过。他不由自主的身子后仰,抬起了头。视线却与一对一晃而过居高临下的眼睛对上,韩冈瞳孔随之猛然一缩。

窦解!

窦七衙内骑着马一掠而过,卷起的狂风吹乱了韩冈的衣袍。只对上了一眼,两人的视线便交错过去。窦解好像在马上有回头,向韩冈这边看过来,可跟着他却给坐骑连甩了两鞭,用着更快的速度跑了。

韩冈冷眼看着他跑远,积郁在心底的怒意越来越盛。

严素心蹲下身子紧紧抱着招儿,花容失色,被吓得不轻。韩云娘脸色更是煞白如纸,躲在韩冈怀中,身子仍止不住的颤抖着。她方才在看到一群烈马当面奔来的时候,被吓得怔住了,虽然知道该逃,脚却动不了,若不是韩冈用力扯了她过来,肯定就会被撞上。她在韩冈怀中仰起头,眼中带着泪花,带着浓重鼻音,“三哥哥,你没事吧?”

“韩官人,你没事吧?!”胖掌柜急着跑了过来,问着同样的话。方才他看到韩冈差点被奔马撞上,心差点跳出嗓子眼,若是韩冈在店前被撞了,不论死活,他都要被提溜进衙门里去熬上一次油。

韩冈脸色冷得如极北寒冰,眼神直如冰刀一般,瞪着窦解的背影。怒火熊熊,把心底的杀意锻炼得更加狠厉。

就让你再猖狂两天!

韩冈看了远处的窦解最后一眼,收回了目光,“我没事!”他沉声说着。

“那是窦七衙内吧?”胖掌柜也望着几骑远去的背影,恨恨有声:“窦副总管也不管着他这个孙子!整日在秦州城中弄得鸡飞狗跳。这两天他又迷上射猎,日日天黑后才从城外回来,在街上快马赶着回府去。”

韩冈哼了一声,不点名的说着窦舜卿:“自古道修身齐家。前一项都做不好,后一项如何能成?”

“这窦七衙内就该挨上几刀子!听说城北有家小娘子被他看上了。那小娘子因不肯相从,就被窦七硬是强上了。可怜那小娘子性子贞烈,受了辱,当夜便投了井。这个叫惨呐……”

胖掌柜声音突然压低了,神神秘秘的说着,“小人听说窦七衙内半年来在秦州作恶不止一桩,王押衙一直跟着他,全都看到了。前日他被窦副总管杖死,就是因为掺和进了这些事中,才被灭得口!不过王启年虽然死了,可据说他事先就知道会出事,留下了窦七衙内的罪证,现在还藏在他家里。”

胖掌柜说完,很得意的抬头看着韩冈,想看看他的一番话能给韩三官人带来什么反应。但韩冈神色淡然,却是毫不在意。

“啊,对了!”胖掌柜一拍脑门,恍然大悟的模样,“韩官人你前日还为着王押衙跟窦副总管吵了一架,肯定都知道了。”

韩冈轻轻的点了点头,眼底的阴寒在面上晕开,最后在唇角处凝出了若有若无的一丝冷笑:

‘王九果然办事得力。’

……………………

窦解一路纵马狂奔,毫不将息马力。他从南门进城,取道河西大街赶回城中心偏东的窦府,只用了不到半刻钟的时间。不过窦七衙内一行没有往窦府大门过去,而是绕道偏巷,在窦府的侧门处勒马停下。

窦解跳下马,将缰绳一丢,让伴当处理坐骑,甩着手就从捱着一条缝的侧门溜进了家中。他在偌大的府邸里小心翼翼地走着,看他前瞻后顾的样儿,全然没有在外面的横行跋扈。

窦解的禁足虽然已经解除,但最近窦舜卿心情很糟糕,若是让自己祖父知道自己镇日往城外去游猎,少不得一顿排头要吃。窦解不想触他的霉头,一回到家中便变得小心谨慎起来。

安全地回到自己的院中,窦解终于松了一口气。一路上碰见了几个仆役,不过他们都是视而不见,全当没看到窦解这个人——在官宦人家做事,少不得有几分眼色。

换去了外出射猎的短打武服,窦解在房中坐下,喝着侍婢端上来的解暑凉汤,他终于放下心来。就算被叫去前院,也不会暴露自己今天出城去射猎过的情况。

不用再担心祖父,窦解很快就想起了方才匆匆一面的韩冈。

前日窦解亲眼见着自家祖父被灌园小儿气得发昏,从衙门里回到家中后,抬手就砸了十几件名贵的器物,又连杖了七八个不开眼的仆役,恨恨地念叨了一夜要把韩冈碎尸万段。听说自家祖父已经上书朝中,向天子弹劾韩冈。

以正五品的观察使之尊,去弹劾一个从九品的选人,窦解确信韩冈也没几天好蹦达了。虽然眼下灌园小儿依然活蹦乱跳着,还能带着女眷出来逛街。但窦解已经可以去想象他被夺官去职,失魂落魄的样子。

一想到今天差点撞上了韩冈,窦解的心中便是自叹着好运。若是当时马头偏了一下,将他撞死,日后就看不到好戏了。

跟在韩冈身边的两个小娘子真是好货色,虽然只是匆匆一瞥,但她们的相貌身形已经让窦解一回想起来,就惊艳不已。

这灌园小儿哪里来的这般运气?!

不过等到韩冈落马,那两名小娘子肯定逃不出自己的手中。窦七衙内想到这里,就嘿嘿的笑出连声来。

“七衙内!”窦解的一个伴当这时在门外通报了一声,疾步走进院中。

这伴当今天并没有陪着窦解出城射猎,窦解一看到他,便向他炫耀起来,“李铁臂,今天你没去城外真是亏大了。我们今日可是满载而归,钱五还射到了一头……”

“七衙内,你现在还说这些?!”李铁臂脸色惶急的走到窦解身边,贴着他耳朵咕哝了一番。

窦解听着听着脸色就变了,惊声就叫了起来。“什么!这事怎么给传出来了?!”

李铁臂嘘了一声,紧张的回头张望了一下,见没有他人听着,他又贴在窦解的耳边,“七衙内,还是快点把王启年藏在家里的那些东西给拿回来处置掉,不然给跟窦副总管过不去的那些人先下手,可就麻烦了。”

“好你个王启年,竟然还敢给我留下这一手,活该你被打死。”窦解阴着脸发了一阵狠,站起来,“我去找爷爷去……”

“万万不可!”李铁臂连忙阻止,“让副总管知道了此事,七衙内你今年还能出门吗?!”

李铁臂可不能让窦解去找窦舜卿,甚至连跟在窦舜卿身边的人都不能找。只要这事传到窦副总管的耳中,眼前的这位乱了阵脚的废物七衙内最多被训上几句加上禁足半年也就没事了,但自己这帮帮闲,少不得要被愤怒的窦副总管找个由头刺配远恶军州,省得再勾引窦七衙内在外做混事。

窦解被李铁臂唬住了,当真不去找自家的祖父。不过一时之间他能找到的人手也不多,想了一想,窦解道:“你去把钱五他们几个找来,让他们跟我一起去王家,掘地三尺,也要把王启年藏起来的东西给翻出来。”

……………………

傅勍觉得自己的运气糟透了。他堂堂一个正九品的武臣,竟然沦落到要在夜里领兵巡视秦州城,而且还不是管理者全城的巡城甲骑,而仅仅是北城一地。

骑在马上,傅勍仰着脖子又灌了几口酒,放下半空的酒坛,他仰天骂着:“爷爷不过是多喝了两口酒,至于把爷爷弄来巡城吗?哪家的正九品官人要巡城?!就是天子脚下,巡夜的也不过是个大将【注1】罢了!”

一口口冷酒灌下肚中,微凉的夜风却吹得傅勍心中更为燥热。也不知哪里来的夜枭在叫,时不时的就是一声尖啸,更是让他心烦意躁。

傅勍从三阳寨寨主的位置上被捋下来也没几天,却已经看透了人情冷暖。过去还奉承着自己的人,现在已经对他不屑一顾。曾跟自己称兄道弟的,也是关紧了大门。使得他只能日日买醉。

就在傅勍醉晕晕的时候,却不曾想竟然碰上了刚刚自衙中出来,准备回家睡觉的秦凤路走马承受刘希奭。

这其实是件好事——巡城甲骑碰上官员夜归,有条不成文的规矩就是护送他们回府。

如果傅勍此时还清醒,肯定会去在刘希奭面前卑躬屈膝的说上两句奉承话,运气好些,把这位阉宦捧得开心了,请他在天子面前说些自己忠勤于事的评价也不是难事。

可傅勍偏偏醉了酒,浑身上下都散着浓浓的酒气。带着连累了胯下的一匹乌云马也是一副醉态,走上三五步,马蹄子就要打上两个晃。

刘希奭看着心中不快,一夹马腹,就要加速离开。

傅勍酒意还未清醒,不顾尊卑的追上去与刘希奭并辔而行,“刘走马!怎么走得这么快?!夜深了,还是让下官送你回去!”

一股酒臭直冲鼻子,刘希奭的心情由不快变成了恼火,他眼一瞪正要发作,这时却见前面突然跑来一人。

“傅官人!”是一个潜火铺的铺兵冲了过来,他跪在傅勍马前,心急如焚的禀报道:“前面的净慧庵起火了!还请傅官人带兵去救火!”

注1:这里的大将是无品级的武官官阶中的一级,并非统领大军的大将。

ps:回来后就急赶慢赶还是没能在除夕夜把这一章赶出来,真是很遗憾。不过欠下的章节,俺都会补回来,请各位书友放心。

