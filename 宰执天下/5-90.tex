\section{第11章 城下马鸣谁与守(二)}

在御苑荷池畔的猝然发病,赵顼虽然竭力表现出若无其事的样子,之后在太医局教授名医杨文蔚的诊断下,也确诊赵顼一时之间并无大碍,暂时只需小心保养,多加休息,但赵顼的一举一动,都受到宫内宫外无数双眼睛的监视,无论什么样的秘密,只要与天子有关,转眼就会传出宫去。

不过这消息传播出去时的扭曲程度,往往也是让人瞠目结舌。半日后,当消息散布到京城中时,就已经是天子因灵州兵败、辽人助夏而忧思过度,并因此得了风疾,如今正是重病垂危,旦夕难保。太医局中的一干御医都被传入了福宁殿。

听到这个消息,宰执们慌慌张张的入宫求见。当时宫门都已然落锁,王珪和吕惠卿硬是逼着守门的石得一将宫门打开。一闹就闹到了福宁殿上,直到赵顼亲自出来解释方才真相大白。而京城中的骚动,到了次日早朝,天子御文德殿,这才渐渐平息。

只是明面上的风波虽说平息了,可海面之下的人心,却越发得动荡起来。经此一事,天子的继嗣问题,重新升上了台面,成为了朝堂政治中一个迫在眉睫的关键议题。

有一就有二,今日天子只是晕眩而已,但下一次呢,再下一次呢,说不定就在几个月之后,天子就无法再安坐在大庆殿中的御榻上。且这次因军情紧急而忧思过度,当盐州兵败或是辽人南侵时又会如何?

做皇帝的一向难以长寿,赵家的历任天子都没有过六旬的例子,英宗的寿数更还不及四十,而当今的这一位,则已经三十有一,早年过而立了。以他的身体条件,什么时候出事都不足为奇。

曾经垂帘听政、能够稳定朝堂的太皇太后就在旦夕之间,而皇嗣只有排行第六的赵佣和排行第八的赵倜两人。且皇八子赵倜的身子骨很不好,虽说种过了痘,不用担心痘疮,但夏天时曾经惊厥过两次,谁都不敢确认他到底能不能保得住。

这样的情况下,最受高太后疼爱、排行又仅次于天子的雍王赵颢,他的行情也就水涨船高。这几日,还去了大相国寺一趟,说是为太皇太后和天子祈福。太皇太后倒也罢了,可天子这不是没病吗?

吕惠卿就着灯火,烧掉了刚刚写了一半的信函。

没必要再给韩冈写信了。在韩冈手中留下一个证据,等于是给了他一个把柄。就算上面的文字再隐晦,一旦捅出来,也是件麻烦事。既然天子的病情在京中传得沸沸扬扬,韩冈不会不知道。韩冈在京中自有耳目,吕惠卿相信他能收到这些消息。

在吕惠卿看来,就是只为自家盘算,韩冈也当设法维持盐州不失。

…………………………

秋天的气息越来越浓,一座座山头被染成金黄或是深红,碧蓝的晴空也越发的高广。

秋税的工作才进入尾声,冬播马上就要开始了,而许多地方还在麦收后种豆,收割和犁田都是麻烦事。太原入秋后的雨水有些偏少,这也很让人担心。尽管实际上负责这些工作的都是下面的知县,但韩冈每天要翻阅签押的文件,数目是越来越多,几乎到了倍增的地步。

韩冈真是烦了这样的差事,河东军中还有许多事等着他发落,但太原府政务上的事情却是比军务还多。忙了一个上午,桌案上的公文只见增多,不见减少,就算是长于政事的韩冈,也不免效率越来越低。

不过中午的时候,一名来客让韩冈重新提振起精神来。

“龙图,夫人和三位娘子一行大概天黑前便能抵达太原。”

韩冈听了心中狂喜,夫妻别离几个月,终于有空将她们接来太原府了。赏了提前赶来报信的家人,让人安排他下去休息,又派了人出城去迎接。

韩冈细细回想,自己自从进入官场,历次履新,从来都是匆匆忙忙的上任,紧赶慢赶的怕耽搁了时间,几乎每一次都是自家先期抵达任所,等到一切安排妥当,才派了人去接自己的家眷。还真没有像一般的官员,能够带着家人,悠悠然然的一路游山玩水,最后在领受任命的一两个月后方才上任。

人总是喜欢自己没有的东西,韩冈也难以免俗。在忙碌中,时常对此心生羡慕。真是同官不同命,什么时候自己也能这么轻松的做事就好了。

‘还是要多多培养助手,若下面的幕僚能多分担一点责任,自家也能轻松一点。’就在午后例行的军议上,累了半日的韩冈不由得分了心神。

黄裳并没有觉察韩冈的分心,犹在朗声对众人说着今天的议题:“如今阻卜人阻断夏州通盐州的道路,种谔肯定会顺水推舟,决不会全心全意的去救徐禧。”

这是黄裳对种谔是否会强行出兵救援盐州的预测,基本上厅中的幕僚们,都同意他的看法。

今天驻屯晋宁军的李宪遣人传书太原,并将种谔的一封信同时送来。在信上,种谔请求河东共同发兵,维护通向盐州的道路。

“种谔致书,请求龙图共同为此发兵,说好听点是应付故事,说难听点,就是祸水东引。”

“如果龙图不发兵相助,种谔便可趁势推卸责任,若龙图发兵相助,只要没能成功挡下阻卜人,龙图也要担上一分责任。”

“种谔的心思若有三分用在正经事上,恐怕官军早就打下灵州城了。”

“但现在,种谔在写信给龙图的时候,肯定也为此上奏章了。朝廷一旦下旨,到时候,也不得不从。”

“李宪所部分驻晋宁军各寨,是不是可以调用一部分,去协防夏州?就是之后种谔要推卸罪责,我们也算是说得过去。李宪虽然没有明说,但他既然让人将公函和种谔的书信一并携来,肯定是站在种谔的一方了。”

“没错!当是如此。否则他就应该分成两拨来送信,借以自清。”

“以我河东军的兵力,谨守葭芦川和弥陀洞,保住银州、夏州就已经是竭尽全力。如何还能分心于盐州?不要忘了北面的契丹人,他们可不会站在旁边看热闹。”

“月前辽人受挫于在西陉寨外,便偃旗息鼓。可从三日前起,代州重又急报军情,明摆着是在配合西贼的行动。”

“一旦北方兵火起,河东的兵力都得往北调,如何有多余的兵力却守护道路。”

“不如就此上报朝廷,报称辽军似有举兵南犯之意,请求加派援军。想必朝廷当不会主张种谔了。”

下面幕僚们的议论,韩冈全都听在耳中。他现在有些后悔,之前将自己对盐州的态度表明得太早,使得现在他的一众幕僚,都开始变着法儿的找借口推卸援助盐州的责任。

如果他们的身份仅仅是河东路经略使的门客,一切为他韩冈着想,那的确不算错。但他们更多的还是气学弟子,韩冈可不想看到一群只会争功诿过的官僚。

坐直了身子,正想说话,位于下首的折可适抢先一步开口,“若是如此行事,世人将如何看待龙图?天子又会如何看待?”

幕僚们的议论被打断了,十几道视线全都汇聚到了折可适的身上。

一人冷笑着反问:“上禀西夏内乱,请求出兵灭夏的是他种五,为争功而抢先出兵的是他种五,连瀚海也过不去的也是他种五,如今退守银州、夏州,声称贼军势大,请求河东同保道路的还是种五。却不知世人如何看他?天子又是如何看他?”

“龙图岂是种谔可比!”折可适向韩冈拱了一下手,“不论在河湟,还是在横山,龙图一直以来都能做到为君分忧。不以私心坏国事。尤其是当年在横山,龙图坚持认为罗兀必败,事先都说过纵有功亦不愿取,但仍兢兢业业保住了罗兀城的数万兵马,最后就连伤兵都带了回来,还夺了上千斩首。之后龙图又说降了广锐军叛卒。泼天的功劳,龙图却是言出如山,一分未取。龙图就是因为有这样的品行,才会备受世人景仰,才会受到天子看重。种谔有私心,那是他行事多偱诡道,不晓大义,但龙图岂会是这样的人?你们难道要龙图学种谔不成?!”

这个帽子可就够大的,给折可适扣在自己的头上,做得不合人意,就是不晓大义了。这可是以‘大义’相要挟,在座的,哪个看不出来。

黄裳偷眼望向韩冈,却没有在他的脸上发现一丝一毫的不快,相反的,却是面带微笑,显是心情很好。

一众幕僚心中咯噔一下,韩冈对折可适的言辞看起来毫不在意,那就代表他倾向于协助种谔和鄜延路。而评判者站在了对手一边,那么接下来不论怎么辩论,结果也很难改变。

想不到竟然给这个武夫得意起来了。

一众幕僚心中很是有几分不甘心,但其中还是有人沉思起来,他说得并不能算错,韩冈的形象对气学门人来说,十分的重要,不能有所损伤。

