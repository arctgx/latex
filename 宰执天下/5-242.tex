\section{第28章 官近青云与天通(九)}

还在睡梦间,脑后的醒枕一滑,司马光一头就撞在了床板上,突然间就惊醒了。

摸了摸头,司马光从床榻上坐起来。所谓醒枕,分明就是一段圆木,睡在上面,一不小心就会滑下来。

连着五六天,司马光都在地窖里写书。

为了修改《资治通鉴》,他在地下不知日夜,连睡眠和吃饭的时间,都不固定。写累了就去睡,枕着圆滚滚的醒枕,睡上片刻,落枕惊醒了之后,吃点东西,就再坐到桌前。

连着在地窖里多日,对司马光来说,近几年来是经常有的事。有了精巧的通风装置,在地下多日也不会觉得憋闷。

而身处地下,与外面的世界,就仿佛隔得很遥远,让人不至于心烦意乱。

看着桌前堆起的书稿,司马光皱着眉头,也不知在地下几天了,冬至应该过去了吧?

司马光想着,却往地窖的阶梯那里走过去。

推开地窖门,贴身的老仆惊喜的迎上来,“君实,你出来啦?!”

司马光点了点头,问:“什么时候了?”

“君实你在底下七天了。前天是冬至,现在都入夜了。”

“七天了?”司马光点着头,对他来说,也不算太久。

“这几天可有什么事?”

司马光正问着,司马康正好跨进外厅的门来,手上还拿着一封短笺,看见司马光,立刻惊喜道,“大人出来了!?”

“有什么事?”司马光问道。

司马康递上帖子:“是富相公府上遣人送信来了。”

老仆识趣的告退了:“小人去吩咐厨房给君实准备酒饭。”

司马光接过来,却是邀请他参加耆英会的请帖。富家在城北的后花园里的梅林全都开了,正好可以宴请耆英会中人。

耆英会中人,从文彦博和富弼开始,都是七十往上的老臣。也就司马光年纪小,才刚过花甲之年。本来是文彦博和富弼邀请他时,司马光是准备推却的,但还是被富弼强邀进了耆英会中。

司马光看了看帖子。短笺上已经有了文彦博和楚建中的签名。找来笔,司马光将自己的名字写上。这张短笺要传遍耆英会中人,如果要参加的话,写上名字就可以了。

方才刚刚出去的老仆小跑着进来,喘着气:“君实,有中使来宣诏了!”

“宣诏?”司马光皱了皱眉。

今年是郊祀之年,今天又是冬至后的第二天,难道是郊祀后的赏赐不成?这么想着,司马光漫不经意的就跨出门去。

中使立于庭中,展开手上的诏书,在香案和司马光父子面前高声道:“给事中、西京留守兼判西京御史台、端明殿学士司马光接旨。”

司马光拜倒于庭中,漫不经意的聆听着来自京城的诏令。就他而言,这种赏赐,有不如无。

但他立刻就瞪大了眼睛,身子都不由得颤抖了起来。

太子太师!

入京!

……………………

司马光到底会不会上京?韩冈和章敦也坐在一起议论着。

如今天子病重,做臣子的不便去酒家饮宴,观赏伎乐。但放衙之后,找的地方坐一坐,却也不犯忌讳。

天寒地冻,韩冈和章敦也懒得往远一点的正店里去,直接就近在西十字大街的巷子里一家清静的小脚店中坐了下来。

脚店虽然小,却打理得很干净,墙壁当是新近才粉刷过,干干净净,看不到一点烟熏火燎的痕迹。桌椅上也都看不到什么油腻和污渍,让韩冈和章敦进门后就不由的点着头。

而且这家店虽然门面不大,可建议章敦和韩冈来此的章家亲信却说这里的酒菜不输正店。

不过因为天冷的缘故,加之朝堂动荡,来此喝杯热酒,侃一侃朝堂上的八卦的客人倒是很有几个。只是当韩冈和章敦两人带着元随们一起进来,却是把正在高谈阔论的他们全都给吓跑了。

几碟还算精致的酒菜摆上了桌,店主和小二便躲到了厨房里。金紫重臣登门,就算不认识人,也能认识衣袍服饰,站得近了都是祸患,京城的店家一个比一个识趣。

没了外人,元随也隔了一张桌,韩冈和章敦说话也就不那么有顾虑。新近的朝政不能在外议论,这是最基本的原则,但京城以外的事,就没有了那么多忌讳。

进门时正好听见坐在角落里的几名吏员装束的酒客,议论着司马光到底会不会奉诏。坐下来后,酒菜一上,章敦和韩冈的话题,也就自然而然的从司马光的身上开始。

“太子太师,入京之招,两份诏书登门。前一份,司马光多半会接下来。但后一份就难说。”章敦笑说着。他对店家端上得来的热酒很满意,是点火就着的烧刀子,又烫得恰到好处。举起明显是店家珍藏,专供贵客使用的雕花银杯,也不待人劝酒,一口就干了下去。

“多半会来啊……”韩冈低声道,笑容发冷,“秋风起,蟹脚痒。螃蟹在河塘里生活经年,呆不住,便要随秋风入江。”

“玉昆你是不是太小瞧司马君实了?”章敦笑着摇了摇头,拿起热水中的酒壶,给自己倒了杯酒。韩冈的比喻有些过分了。

韩冈没见过司马光,但章敦见过。司马光的心术手段,他了解得很清楚。仁宗立英宗为皇太子,世间都说是韩琦的功劳,但实际上却是司马光推了最后一把。司马光在定储之事上所说的那几句话,比其他重臣连篇累牍的奏章都管用。

且在熙宁二年三年的时候,王安石因新法在外受韩琦、富弼、文彦博等元老重臣沮坏,在内天子又犹疑不定,不得不以进为退,告病在家,逼天子做个决断。时任翰林学士的司马光,在帮天子起草的一封慰留诏书时,却在文字中隐藏锋锐,将王安石气得连病都不装了。要不是旧党实在不成器,司马光也拿不出切实可行的救国危急的方略,王安石哪里能将司马光赶出京去?

章敦将手上银杯递到韩冈面前,杯中的烧刀子映着银光,清洌如水,“司马光为人,正如这烈酒,虽是狠辣在内里,但从外面看起来,却是清澈如水一般。”

“当年的事,韩冈也是知道的。那段时间正好是韩冈被王襄敏举荐,第一次上京的时候。”韩冈笑了一笑,却有几分感怀,“当时,韩冈可是每天都要登门造访‘王大参’府,在门房里坐上一两个时辰。”

韩冈一提,章敦倒是想起来了,“不意都过去了那么久了。”

“是啊。已经十一年了。”韩冈感叹道:“不过当年初次上京时,在岳父府上见到,大半还在京中。也就吕吉甫现在关中知京兆府,曾子宣还在江南做他的知州。”

如果不算上王旁,加上韩冈,在王安石府上会面就是五人。

王安石眼下成为了真正的元老重臣。章敦、吕惠卿都是出入两府,当年刚刚得到推荐、仅仅是从九品选人的韩冈,如今与两府的距离就是一层纸。只有曾布,王安石恨透了他的背叛,更是被新党所唾弃,这辈子很难再有机会了。

不过话题扯得远了,那是熙宁三年年初,现在是元丰三年年末,已经是差不多十一年前的事了。而司马光在洛阳,也已经快十一年了。

“司马光正值盛年,却被岳父逼得退隐洛阳十一载。在地窖里修书,怨意非同小可。就他而言,多半即便是上京来发泄一通怨气也是好的。”

一名心怀抱负的名臣,却在年富力强的时候无法施展自己的才华,不心怀怨望才有鬼,说归说,这世上谁能做到雷霆雨露皆是天恩,做臣子的甘之如饴?

章敦笑了笑,却不说什么了。

韩冈冷笑着:“若是这一次是太后垂帘,你看看司马光做不做得了横行霸道的螃蟹?吾日暮途远,故倒行而逆施之。”

韩冈说得刻毒入骨。这话是伍子胥携吴军破楚国,鞭尸楚平王后所说。同样的话,汉武帝时的名臣主父偃也说过,‘生不能就五鼎食,死亦要五鼎烹,吾日暮,故倒行而逆施。’

但章敦仔细想想,却也没办法驳他。章敦自问,换作是他本人,若是从今天开始十余年不得任实职,只能依靠修书打发时间,猛然间接到朝廷的召唤,就算是其中有些问题,也肯定是要上京一趟撞一撞运气的,即便不成功也能发泄一下怨气。

“诚可惜哉。”章敦漫声吟道。

‘当然。’韩冈点头微笑。

身为太子太师,纵然只是太子名义上的师傅,但与未来的皇帝就有了一份亲近之意。韩冈以己度人,只要司马光心还没死,肯定会奉诏上京而来。何况天子病危前都想到他,做臣子又岂能无动于衷?但王安石做上了平章军国重事,成为了货真价实的元老重臣。如今是不会有司马光的机会了。

“当年一众定策元老加上两宫合力都没能掀翻新法,区区司马光,再加一位不得圣心的吕公著,又能如何?”

对于旧党,韩冈丝毫不在意。重要的是道统之争,新学和气学的恩恩怨怨,终究还是要分一个高下。

…………………………

“司马光接旨了?”文彦博半阖着眼,貌似随意的问道。

“接下来了。”文及甫在老父的面前躬身而立:“大人,你看天子的病情会不会……”

“安心等就是了,不会拖太久的。”只有父子二人,文彦博也不怕说两句悖逆的话。

仅仅比骑着快马的使臣迟了一个时辰。身在西京的诸多元老,一个个都收到了来自京城的紧急传信。而司马光府上的消息,也传到了文彦博的耳中。

天子中风,延安郡王为皇太子监国,皇后权同听政。然后王安石、司马光、吕公著担任了太子三师。

一条条让世人惊骇的消息,并没能让三朝宰辅的情绪有太大的波动。亲眼见证了仁宗、英宗的驾崩,又重病了一名皇帝,在文彦博看来,也不过如此而已。

不过皇帝的病情无所谓,可此事所带来影响,文彦博却不可能当做河畔清风,绕身无碍。

虽然没有更进一步的消息,但朝局由此动荡却是完全可以肯定。

文彦博呵呵笑了起来。

皇帝中风,便不可能再理事。皇后纵然垂帘,但终究还是妇人。以他们这些元老的身份,加上遍及天下的门人故旧,不是没有机会的。

难道皇后就一定会支持新党?那可不一定啊!

就算支持,可以改变的。

文彦博耷拉下来的眼皮猛然一睁,却是威棱四射。

