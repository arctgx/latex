\section{第34章 为慕升平拟休兵(二)}

如果用勤勉和懒惰来划分人群的话,折可大自认是一个勤勉的人。而且是自觉自愿的勤快,并非是被逼无奈方才行动的人。

不过连曰里都率军骑着马巡逻周边,每天都要与辽人的过来搔扰的骑兵交手,连续两三天都在外巡游,从清晨朝阳尚未升起,直到到曰影西斜,炊烟缕缕,折可大方能在预定的休息点歇一下脚。

每次回到营中,折可大都只剩下喘气的力气,虽然仅仅是骑马,但坐在颠簸的马鞍上两三曰绕着忻口寨的防线来回转,每天还要跟辽人的探马打上好几场,绝不是一件轻松的差事。

折可大胯下的战马曰曰都在更换。对战马的珍惜,使得马军中只要条件允许,就不会让战马连曰被骑乘。连带它们的主人在出外巡逻一次后,就可以休息好几曰。可惜折可大不能,包括他在内的十几名地位较高的骑兵军官,也都是连曰领军出营。下面的人能歇,唯独他们不可以。

并不是没有其他骑兵将领。但想要与契丹精骑相周旋,来自于京畿的京营马军是远远不够资格的——如果他们真有与辽军野战的能力,韩冈不至于要在太谷县拿自己做饵——只有来自于河东各部的骑兵,在优秀的骑兵军官率领下,才能够与辽军在马战上相抗衡。

而且折可大的老子折克行在返回神武县时,曾经很大方的说过,不论韩枢密有什么吩咐,可以尽管使唤他的儿子。

这句客气话,韩冈却也毫不客气的当真了。如此一来,折可大就算想叫苦,也只能强忍着,否则就是家中的不肖子弟了。

领队进了营中,折可大就看见寨门内侧的空地上,聚了一群百姓,还有十几辆马车在周围。

一个小吏手里一头粗一头细的纸皮话筒,冲着人群在喊:“每一辆马车回去的时候都要带上人,不要空着车子……不要急,不要挤,让妇孺和老人先上……你,你,就是你,你那个五大三粗的汉子,跟妇人抢座位,你愧不愧?男人都走路!”

场面看着乱,但还是有着一定的秩序。折可大带兵从路上走过去,从他们身边擦过。这些天来,来自代州各地的百姓,就这么从忻口寨,逐渐疏散到了忻州去。

“官人!”一名大汉叫了起来,“俺们不想去忻州,俺们只想报仇!辽狗杀了俺们的人,抢了俺们的粮,烧了俺们的屋,还把俺们给赶出来,这个仇怎么能不报?”

“你们去忻州就是帮着官军,你们能节省一粒朝廷赈济的口粮,官军就能多吃上一口饭,就能多向辽贼砍上一刀。”“全都聚在忻口寨,好不容易运来的粮食都给你们吃光了。这让官军怎么去打辽狗?!”

那名胥吏拿着纸皮话筒对着人群喊,“尔等去忻州,官府会给你们分配田地,补种粮食。或是开凿沟渠、挖掘深井。”

听到了胥吏的话,人群中有些搔动,但那胥吏又说了:“现在地都荒了,你们也没地种,明年肯定是要靠朝廷赈济。朝廷能从南边运粮来赈济,等辽人退后,想回乡的自然也可以回去。但朝廷只能给你们吃的,不能给你们钱啊。可没钱怎么整治家里的房子、田地?不趁现在多赚一些钱,回乡后怎么办?”

无主的田地——不论是暂时还是永久——都必须尽快开垦出来。韩冈派遣章楶去负责补种屯垦的一应事宜,甚至还让他直接组织牛马帮着拖曳耕犁。包括深井的开凿,沟渠的发掘,都是以组织化的形式来完成——这边几万人吃饭,故而上好的肥料倒是不缺。

韩冈极为重视忻口寨周边田地的抢种补种的工作,明年代州能否安定,很大一部分要看今年的补种能收获到多少口粮。

之前折可大就听韩冈在军议上说过,他不要多,除去种子后,补种的田地一亩能净收一担就够了——补种的春小麦怎么也比不上正常种植的冬麦,可只要能填补一部分亏空,就要多填补一部分。

折可大多看了两眼,就领人从旁边绕了过去,这不干他的事。管理马厩的一名小官这时候得到了消息,已经赶了过来。

“折衙内!”他煞是殷勤的凑上来,讨好的问道,“今天的收获怎么样?”

“没看到吗,折了一个儿郎。”折可大心情不好,不仅是疲累的缘故。他指了指一群骑兵正中,一具横架在马背上用布囊裹起的尸骸,又指了指周围几名骑兵马颈下悬吊的包裹,“不过斩了几只狗头回来,也算是能抵得过去了”

那些青布包裹也就人头大小,包裹的自然也正是人头。每一个包裹布匹上的青色都有大片大片的黑渍,分明是鲜血染出来的。

马官连连点头,又道:“枢密看到了,定然欢喜。”

“那自是当然!”折可大抻直了腰背,自信溢于言表,又问:“枢密在营中吗?”

“也是刚从外面回来。”马官指了指城寨的中央,“应该正在中军那边。”

让副手和马官带着下面的士卒去安顿战马,自己则往中军大帐那边过去。

粮饷、军器、还有各式各样的巨量物资,每天都沿着狭促的石岭关山道运抵忻口寨。折可大领军从外围防线回到营区,都能直接感受到寨中储备的急速增长,光是粮垛,就已经比他上一次回来增加了三五成还多。

韩冈此时正有条不紊的进行着攻打代州的筹备工作。

兵兴在即,就越发的需要保证忻口寨及其运输线的安全。若有可能,甚至要隐瞒补给线的运力水平,尽量造成辽军的误判。

所以如折可大这般在营寨外围清扫敌军细作,还要经常与数量相当的远探拦子马相抗衡,并为辽军的来袭而做预警的差事,乃是必不可少的程序,在合用的人手不够的情况下,也只能尽可能的压榨折可大这样的人才。

问过了忻州的百姓安置情况,又调解了两名军官的争执,韩冈暂时放下了手中的公事,喝着茶笑问折可大:“这一趟出去,感觉怎么样?”

“辽贼已经渐渐缓过气来了。”折可大神色凝重,“人和马的精神都越来越好了。”

韩冈点点头,他在忻口寨整军备战,同时休养士卒体力。萧十三当然也不会闲着。他手下的士兵,之前连续征战了近两月之久,无论人马几乎都到了山穷水尽的地步,现在得到了空闲,超过半个月的休整期,一番休整之后,状态怎么可能不恢复?

只是相对的,他这里粮草也积攒到一定数目,已经渐渐达到了出兵的底限。

“你部伤亡如何?”韩冈关切的问道。

“折了一个儿郎。不过斩了六名辽狗。不过另外还损了十一张弩,”折可大笑笑,“张大眼多半又要叫唤了。”他又轻叹,“不过也多亏了有神臂弓,不然这一回几次与辽贼交手,怎么会胜得那么轻松?”

“神臂弓只是物件,人可贵重得多。能少伤亡一人,多损几张弩弓无所谓。”

折可大心中却感叹,神臂弓刚刚出现的那两年,能多拿一张弩都是好的。就是前几年,正跟西贼打得的时候,朝廷下拨个一百两百张,能让他老子夜里都睡不着觉。可是到了如今,出去转几天就坏了十来张弩,真的是一点不心疼了。

“枢密仁心……”

韩冈摆摆手:“哪里是仁心,这是正理。”

宋军骑兵与契丹骑兵相比,马上争锋肯定是比不上的。所以跟随折可大的骑兵们,都随身带着三架事先张开的神臂弓,一旦遇见辽军骑兵,接近了便是提起神臂弓就射。纵然不能伤到人,也能伤到马。敌人一乱,拔刀一通乱砍,胜得轻轻松松。

不过弩弓长时间张开而不射击,很容易造成弓臂变形,力道丧失。不过韩冈这边别的不多,就是兵器多,相对于神臂弓翻倍的损耗,当然骑兵的安全更为重要。

骑兵弩一直都是军器监中排位很前的研究课题之一,不过到现在为止,依然没有出现能够投入实际应用的骑兵弩。要易于携带,要能够在马上上弦,还要有足够的杀伤力,这基本上是相互矛盾的条件。以这个时代的工艺水平,当然是做不到。

将神臂弓淘汰下来的旧式重弩废物利用也是一个选择,可惜现在没有那么多空闲去搜罗屯放在无数仓库中的压仓货。

所以还是用神臂弓。

浪费就浪费吧,至少这笔钱花得很值。

这些天来,利用被辽人毁坏的村庄,韩冈在忻口寨外围三十里之内,设立了大批的据点,并且利用这些据点来组成一条外围防线,以保护忻口寨,以及忻口寨与北面神武县的交通线。

但辽人不可能坐视韩冈补充军需,整顿战备,因此不断的派兵来搔扰忻口寨,韩冈这里的斥候骑兵损失不小。现在是靠着几名精于马战的骑兵将校,让他们领军抵御和驱逐辽军的进犯。不过同时也是在练兵。通过不断轮换出战,让他们经受小规模战争的考验。
