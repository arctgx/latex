\section{第九章 鼙鼓声喧贯中国(七)}

嵬名阿埋是知道霹雳砲威力的,他曾经见识过国中偷学而来的霹雳砲,毁掉了一堵城墙有多么轻松,而宋人的霹雳砲比那些仿制品要高大了一倍有余,威力只会更加恐怖。困守城中只能坐看石弹一点点的将城墙寨墙给拆个干净,绝不会有好结果。如果拼一拼,就能毁去这些霹雳砲,嵬名阿埋肯定会立刻下令去赌一下运气。可是这厚达一尺多的积雪,让他打消了所有出战的念头,在雪地中跑不快的铁鹞子,只会成为神臂弓的靶子。

“尽管砸好了,雪这么厚,砸塌了城墙也别想攻上来。”嵬名阿埋喃喃自语。在种谔搬出了飞船和投石车后,在他的眼中,厚厚的积雪比起城墙更安全。他已经下令守军撤下城头,只把旗号留下。这道城墙,坏就让他坏好了。

匆匆下城的士兵们有点乱,嵬名阿埋转头过去呵斥了两声,回过头来吩咐道:“让芭良也将人撤离寨墙边,只要人不被石弹砸到,寨墙坏了也能再修起来。”

一名信使带着主帅的命令从城头垂了下去,跑去了不远处的营寨传令。

嵬名阿埋此时已经镇定下来,方才看到飞船后的惊慌,已经在他发现宋人的兽首飞船在山谷中的北风里向南大幅的偏着,得靠绳索系在地面上之后,便半点也不剩了。

不过是个巢车而已!不足为虑。

嵬名阿埋扶着墙头,望着银装素裹的远近山川。无定河一带千丘万壑,沿着河谷向前走一里,两边都能经过十几条沟,这样的地形要想防着偷袭很难,而偷袭就很简单了。

别以为他没有后手,嵬名阿埋捏着墙头的积雪发着狠。

自从被调到罗兀城后,他就知道自己最后很有可能被当做弃子来处理。为了保命,他可是什么招数都想过了。嵬名阿埋的才智虽不算高,但架不住他为了要保住小命而拼命。

嵬名阿埋的幕僚惯会察言观色,看着阿埋的神色就知道他在想些什么,讨着好笑说着:“宋人进抵城下已有时日,一开始的防备肯定也松懈下来了,昨夜更是送了一份大礼过去。只要今天宋人不能攻下此城,到了夜里可是有着好看了。”

“河谷狭窄,在谷地扎营,也就别想着防火了。”嵬名阿埋冷冷的说了一句。

“举烟!通知外面的孩儿!”他接着又是一声怒吼。

这时候,就让宋人多得意一下吧,腾起的狼烟多半会被他们当成告急的信号。回头再用力的盯了天上的飞船一眼,嵬名阿埋转身就下了城去。

罗兀城中的敌军从城头上撤了下去,城外寨中的西贼也让出了寨墙内的一段,这一变化立刻就从飞船上传到了种谔的手中。

种谔毫不在意的看了一眼,就丢到了一边去。西贼会怎么做,这根本就无关紧要。

“放狼烟了!”种朴的叫声隐隐带着兴奋。

几道黑色浓烟此时自城中腾升而起,高高的散入云霄。

“还没正式攻城呢…嵬名阿埋已经要向北面求救了。”种朴快活的大笑着:“这里的雪都一尺厚了,赏逋岭、马户川和立赏坪少说也有三尺,哪里能来得及赶来救援?!”

“就算来了也能早早的发现。”种建中将兴奋收于心底,冷静的说着。

“霹雳砲准备好了没有?”种谔不理自家的子侄在说些什么,在前面问着部将。

部将单膝跪倒,抱拳高声:“只等太尉令下。”

“开始吧。”种谔语气淡然。

平静的语调,立刻就引发了一场疾风暴雨。

六十余架霹雳砲几乎都是对着军寨而去。居中的霹雳砲当先发射,载满重物的前臂向下一落,后臂长长的稍杆便猛然一样,嗖的一声响,就将近五十斤重的石弹远远的抛掷了出去。划过一条完美的弧线,石弹的落处离着寨墙尚有一段距离。

控制这架霹雳砲的砲组,随即就在前臂处加装了配重的石块,第二次发射,石弹则正正的撞上了寨墙,卡擦一声脆响,木质的寨墙顿时就可撞开了一处豁口。

只用两发就确定了配重。得到的数据,立刻传给了每一个砲组。第一轮齐射,就有三分之一准确的飞进了党项人的军寨中或是砸在寨墙上。第二轮、第三轮,命中率都在三成以上。靠着配重和标准化的石弹来确定射程的霹雳砲,稳定而高效。

须臾之间,就如同风暴一般的弹雨攻击,带给的寨中守军的就是全然绝望,西夏人在此地不是没有守具,可要想跟宋人比起来未免差距太大了一点。藏于城中的霹雳砲,与城下的霹雳砲对比起来,却如刚通过了蒙学的小学生,与已经通过了贡举,正要入京参加进士考的贡生做比较。

面对南方的围墙已然尽灭。飞过来的石弹钻进寨中后,就在其中横冲直撞起来,一声声惨叫响彻云霄。原本为了避让石弹而腾出的空地,现在又扩大了许多。

“该雪橇车出动了。”

一辆辆载着宋军步卒的雪橇车就从营中鱼贯而出。原本是运粮的车辆,现在则成了运送士兵穿越雪地的工具。

种谔带来的军队中,除了几百匹挽马之外,骑兵用的战马也就五百多匹,原本是作为斥候和游骑来使用,现在下了雪,全都派不上用场。论起吃用,两万人马的宋军,比起驻扎在罗兀城内城外的西贼,耗用的粮食还要少上许多——对面可是至少有三四千匹马,而一匹战马,消耗的草料粮食,差不多能有常人的五到十倍。

不过毕竟是两万人,要想维持着补给线,也不是那么容易的一件事,幸好这时候有了雪橇车。

利用雪橇车在山区运粮,比起手推车或是马骡来运输要轻松得很多。一百辆雪橇车沿着冰封雪盖的无定河就这么一路上来了,也不用像过去那般劳师动众,发动数万民夫,很是轻松的就维持住了两万大军的粮秣补给。种谔敢以一路之力发动此番大战,也有不用征发百姓的缘故。

所以种谔要等到冬天,等到下雪之后再行动。

看着载着宋军的雪橇车向着寨墙冲去,被下属急匆匆的催上了城头的罗兀城主将手脚冰冷,脑中一阵晕眩。

比起飞船、霹雳砲、神臂弓、斩马刀和板甲,在宋人的军械中,雪橇车的名气一点也不大。嵬名阿埋哪里能想到,宋人还有这样的一个招数,竟然会有着在雪上轻松运送兵员的车辆。

车上所载的宋军步卒粗粗一数估计有四五百人之多,差不多一个指挥,如果让他们在寨外下车结阵,想歼灭他们,绝不是一时三刻能做到的。而这段时间里,宋人的雪橇车还能运上几个指挥冲过来。一旦让两三千名宋军步卒,开始结阵冲击已经没有了围墙的寨子,寨中守军根本就不可能守得住。

连忙点起自家的侄儿,“谕密,速领军出城堵截!”

嵬名谕密听命,立刻领了一千铁鹞子出城。

“来不及了。”

在嵬名谕密尚在城门内侧整顿兵马的时候,种谔就已经得到了消息。

积雪阻碍了党项骑兵的冲锋,如同在白纸上爬行的蚁蚕,一点点的挪动着。而宋军的战车同样是碾着积雪,却硬是要比党项骑兵还要快上一成。看似微小的速度差别,却使得宋军能够提前一步排下阵势,用劲弩将射得人仰马翻。

再雄峻的战马,背上连人带甲小两百斤的累赘,而且还是在雪地里跋涉,必然是举步维艰。而拉着雪橇车的挽马,则只是要拖动身后的车厢而已。尽管雪橇车在眼下,还不如平日里两条腿奔跑的速度,但来去如风的铁鹞子,更是只能落在后面吃灰。

一支支弩箭从下车结阵的宋军手上的神臂弓中离弦而出,组成的箭雨泼洒向避让转向的敌军。军寨中的守军打算展开反击,帮助出战的嵬名谕密,但霹雳砲的攻势一直没有停歇,而且还换成了能够飞溅伤人的泥弹和碎石弹,片刻之间,就在军寨中造成了数倍于之前的伤亡。

“射得好!”种朴右手握拳,一声大叫。

“西贼肯定要退了。”种建中说着。

也正如种建中所言,随着雪橇车第二次载人出寨,罗兀城头的撤军号角仓促的响了起来,嵬名谕密带出去的铁鹞子开始后撤,而营寨中的守军也从北门匆匆逃了出去,绕了个圈子转回罗兀城中。他们离开前还在营中放了一把火,只是没能烧起来。

远远望着官军冲入西贼留下来的军寨,在欢呼万胜中,种朴摇摇头:“没想到竟然这么容易,西贼还真是废了,与几十年前根本没法儿比啊!”

“是官军已经远胜过往。”种建中笑道。拥有神兵利器,手握绝对的优势,这一战实在是太轻松了,在过去怎么也不敢想的。

刚刚从后面上来的游师雄为人稳重,提醒道:“西贼惯使奸计,还是要小心为上。”

“的确是没到得意的时候,这些话等你们打到兴庆府再说吧。”种谔虎着脸教训了子侄两句,一指位于正前方的城池,“今日先破城,给我将罗兀城攻下来。……不管嵬名阿埋还有什么谋划,打下罗兀城就什么都没了。”

熙宁八年十月廿七,宋军收复罗兀城,并斩首三千余级,城中守将嵬名阿埋余众窜入山中。

但于此同时,三万西夏军出现在河东路的西北角、同时与西夏和辽国西京道交界的丰州。猝不及防之下,仅有不到两千守军的丰州州城,只坚持了两日便宣告陷落。麟府军救援不及,反而在半道遭遇埋伏,主将折克行苦战得脱,只能率军回镇本州、坚守待援。。

自此大宋君臣方才明了,西夏无论是对秦凤路的进攻,还是在罗兀城的坚守,都只是幌子,党项人的目的,从一开始就是将辽国给牵扯进来。

就在丰州陷落的消息抵达东京的同一天,相距万里的南方,一支满载着大军的船队驶离了交趾永安州【今越南广宁省芒街】的港口,进入了茫茫南海,船头的方向……是北方。

