\section{第20章 廷对展玉华(上)}

【不好意思,今天白天有事外出,迟了一点。但两更依然不变】

没有来得及让韩冈一展口才,便被不速之客给打断。

天子遣使传诏,找韩冈入宫觐见,让楼中的喧闹一下消失得无影无踪。看着这位上楼来的身高体壮,像武夫多过阉人的宦官,没人再敢说些什么。

士人多是看不起阉宦,但对于身负皇命的使节却不能有半点不敬。

在场的不会有人认为这是天子要降罪于韩冈,才特地招他入宫觐见,必然是有什么好处在等着他。一想到天子竟然眼巴巴的派人来找韩冈,更是惹得众人心头的嫉妒如同火上浇油一般。

‘终于来了。’

天子的召见,韩冈对此可是等了很久。将殿试时隔着几十步的距离的会面排除出去,他这个官做到了从七品,才第一次正式觐见天子,这与他成为朝官的年纪一样,在如今才朝堂中,可算是独一无二了。

在众人在愤怒中参杂了更多的嫉妒羡慕的眼神中躬身领旨,然后韩冈回身对着这一众儒生,一拱手:“诸位兄台,且恕韩冈要先行告辞。”

韩冈如同老友一般告退,众儒生一个个都愣着,不只是该回礼相送,还是昂起头不屑一顾。

不等他们决定过来,韩冈已是掉头不顾而去。而在离开前,韩冈没忘了让随行的伴当掏钱会钞,也没忘记拉一把叶涛,“致远兄,你前面不是说午后尚有要事?”

叶涛先是一愣,继而连着点头。他当然知道,韩冈一走,他便要成为众矢之的,哪还有留下来的意思。跟着韩冈下了楼来,在门口向韩冈告辞:“那小弟就先回住处去了,过两日再来联络玉昆兄。”

送了叶涛离开,清风楼的小二也牵了韩干的马来。这时,楼上一阵爆发式的喧哗猛然响起,传了下来。惹得们前的人们纷纷抬头上望,韩冈的嘴角也不免露出了一丝讥讽的微笑。

传诏宦官也向上看了一眼,回头便催促着:“还请韩博士上马,不要让天子久候。”

韩冈点头一笑:“自然,韩冈怎敢耽搁。”随即翻身上马。

宦官也跳上自己起来的马匹,比韩冈落后大半个马身,一起向着位于东京城北的宫城而去。

传诏宦官在前行中,与韩冈稳定保持着距离,提缰避让过路前的行人也是十分轻松,显得骑术很有些水准。一路走着,他奉承的对前面的韩冈说着:“当日韩博士在狄道城运筹帷幄,独守河州不失,保下了整个熙河路,小的跟着李都知,全都看在眼中。回来后,官家都是详详细细的问过。对于博士,官家一直记在心上,更是时常提及博士的名讳,几年来一直渴求一见。”

“韩冈久沐天恩,也何尝不想一睹清光,只是始终不得其便。”

韩冈说着惯例的场面话,却想着这宦官的话,在说他曾经跟着李宪到过熙河。

仔细回想了一下,韩冈也依稀记得这位被天子派来招他入宫的宦官。身材高大如武夫,没有多少阉人阴柔之气的宦官,的确不多见。当初李宪奉圣旨至狄道城传诏,命韩冈自河州退兵,便带着这人在身后,记得是由他背着退兵的敇令。不过当时韩冈硬顶着圣旨,连话都不便跟李宪多说,与这宦官也只是打过两三个照面。

不过韩冈发现这阉人蛮会说话的,‘小的跟着李都知,全都看在眼中。回来后,官家都是详详细细的问过’,听起来好像是他帮自己说过好话一般。可实际上的情况,应该是天子问李宪和王中正的才是。

升起了点兴趣,韩冈问道:“记得曾在李都知处见过黄门,不知怎么称呼?”

宦官听到韩冈相问,一下就兴奋起来。韩冈什么身份?宰相的女婿!冯京、富弼,那都是宰相的女婿。以韩冈如今的功绩、品阶,更重要的是天子的看重,日后保不准也是一任宰相。

而且韩冈在陕西,尤其是秦凤、熙河两路的事务上,有着很大的发言权。如果能得他说句好话,说不定就能去熙河或是秦凤作上一任走马承受也说不定。日后也好模仿着王中正和自己的恩主,还有多少前代大貂珰,出外掌兵。

连忙在马上弯下腰,恭声的回道:“不敢当博士垂问,小的姓童名贯,为祗侯高品,如今在崇政殿中听候使唤。”

韩冈的身子有一瞬间的僵硬,脸色也变了一下,不过他骑马在前面,没让后面的人看见。

“……童……贯!”

“正是小人。”

来自千年后的前世,对于历史不甚了了。使得韩冈对这个世界名人的认识,多是来自于前身残留下来的记忆,如张、程、邵、李等大贤名儒,哪一个的事迹不是前身才会知道的?曾经的贺方只听过一个名字而已。

不过来自于千年前的回忆里,宋神宗、王安石、苏轼、欧阳修、司马光这等千古名人之外,眼前的这位正冲着他谄媚不已的小黄门的姓名,却也一样的如雷贯耳,流传千年。

“呵呵……”韩冈失声而笑,千古名阉啊,在熙河时竟然错身而过,“童贯,一以贯之,这个名字起得好。”

他以一句随口而出的好话,掩盖住了自己的震惊。

而童贯只听到了韩冈的赞,喜笑颜开:“贱名有辱清听,贱名有辱清听,当不得韩博士的赞。”

童贯现在还没有一个官身,祗侯高品属于没有品级的小黄门,距离内侍官制中从九品的黄门还有一段距离。更别提跟王中正、李宪那等已经转为武职的大貂珰相提并论。所以韩冈一句赞,便让他如此兴奋。

不过韩冈知道,童贯日后可是能封王的——如果历史依然像他记忆中那般发展的话。只是他韩冈既然已经到了这个时代,自是不会让童贯有成为六贼的机会,未来的靖康之耻也绝不会再出现……只是可惜了水浒传。

不移时,已经到了宫城外。留了伴当在门外牵着马,韩冈和童贯下马后,验过腰牌,就从东掖门步行入宫。穿过了两重宫门,用了一刻钟的时间,终于走到了崇政殿前。

韩冈留在殿门外,童贯进殿回复。

很快,殿中就传出话来:“宣韩冈进殿。”

集英殿中殿试,只是一瞥而已,但已经给赵顼留下了很好的第一印象。虽然说不上很英俊,跟冯京那是没得比,但依然出众的外形,加之历经磨练出来的气质,在四百多名进士中,绝对是出类拔萃的。

而今天崇政殿中的正式召见,君臣之间的距离,远远短于集英殿,更是让赵顼看到了韩冈出色的地方。

但凡第一次觐见天子的臣子,多半是诚惶诚恐,而韩冈完全没有慌乱。行动致礼,都是依着应有的礼节而来,不见一星半点的错误。

赵顼知道韩冈是张载的弟子,而张载本人就是深悉礼法而在朝中闻名。韩冈得其传授,自不会不知面君觐见之仪。

可学以致用不是简单的事,殿上失仪的重臣从来不少。而韩冈非但礼节没有错处,他在御前的态度,与王安石那等经常在崇政殿中见面的重臣相比,根本也差不了多少。如果硬要说其区别,也只是略带拘谨一点而已。

沉稳的气质,出众的外表,正好符合了赵顼这些年来,通过韩冈一系列的发明和功劳,所猜度出来的形象。

赵顼满意的点着头,带着难得一见的笑容:“自从韩卿入官后,朕就始终都想见上韩卿一面。谁知道阴差阳错,一直拖到了今天。”

“臣以驽钝之才,竟蒙陛下记挂于心。臣感激涕零之余,也是愧不敢当。”

“渭源堡,香子城,珂诺堡,数次镇守后路,力抗贼军。非韩卿之力,河湟之事几是难保。”

“乃是陛下圣德庇佑。”

开场的都是惯例的套话,就算是说着感激涕零,也是将情绪收敛的只有稍稍的波动,不会痛哭流涕,以此来表现自己看到天子后有多么激动。

韩冈很清楚,越是在天子面前,越是要表现出庄重的姿态,否则就是轻佻——这个评语,对于以宰执天下为目标的臣子来说,就是个致命的词汇。

见着韩冈,不因自己的喜怒而动摇,赵顼又看重了他几分——这也是人之常情,看好一个人,看他做什么都是好的——仔细想一想,其实也只有如此沉稳坚忍的性格,才能在王韶和高遵裕前去追击木征的时候,稳定住内外交困的熙河路。

王韶和高遵裕都想拿到收复河湟最后的功劳,都不愿放弃追击木征,所以一起领兵翻越了露骨山。而他们之所以能安心离开,却是相信韩冈能将作为后方的熙河路,稳定的支撑起来。韩冈并没有辜负他们的期待,不但击退了西夏人,更是顶住了朝堂上的压力,一直将路中秩序维持到捷报的传来。

一年来,赵顼不知多少次庆幸韩冈的抗旨矫诏,也悔恨过自己当初向罗兀城派错了人,不然,西夏国此时已经是垂死待毙。用人之误,造成的后果一至于斯。

