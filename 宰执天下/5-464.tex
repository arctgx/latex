\section{第38章 何与君王分重轻(22)}

向皇后的话声刚落,殿中群臣同时抬起头,瞅着太子,看向赵顼,还不忘扫了眼屏风。不过他们很快便将视线同时收了回来,目光相互交错,又都避开去。

心中各自狐疑,太子看着没事,是天子情况不对?还是皇后有什么想法?

只有王安石和韩冈盯着宋用臣,宰辅的地位,给了他们更多的特权、胆量、以及经验。

宋用臣汗出如浆。

直视天子,是不敬之罪。殿上其他臣子都不敢冒渎,但王安石和韩冈却有足够的权力去不在意。但王安石和韩冈盯着的是自己,角度的差别虽然很细微,可那随着视线而传来的压力,却让宋用臣不会弄错。

如果仅仅是宰辅盯着自己,宋用臣还不至于出冷汗,只是若再加上皇后呢?

天子没事,是皇后有事。

皇后随意插话,完全不合常理。而且还是出言打断经筵,这更是让人不得不心生疑虑。

丈夫在场的时候,哪里有妻子随意说话的份?像经筵这样的场合,要不是皇后有垂帘听政的资格,根本就进不了集英殿。就是普通人家待客,也很难见到主人还没发话,做女主人就自作主张逐客的。母仪天下的皇后更不用说,规矩可要更重一点。可现在皇后却做出来了,逐的还是包括宰辅在内的一众清贵官。

当年郭皇后在跟受宠的嫔妃的争吵中误伤了仁宗,就被废了皇后之位。而现在向皇后所做的,真要穷究起来,肯定是要比家里面的小摩擦的姓质更严重。

莫不是忍不住想要夺权了?

身在宫闱之内,自幼在皇城内长大,宋用臣就像熟悉自己的手脚一样熟悉权力的争夺。

天子病瘫,归根到底最后还是得让皇后做主。

掌控大权半年多,国政上无所疏失,军事上更是让太祖太宗之后的几位官家都望尘莫及,可皇帝却总是不肯放开手,依然是每天都要听人读奏章,干涉国事。

但为了防止皇帝心忧过度,辽国入寇的消息都是瞒着他的。弄得现在的奏章,都要设法改过一遍,让上面没有任何会让皇帝心生疑虑的地方。只是撒的谎越来越多,也越来越难以弥补,为了圆上一个谎,就要撒上三个。小心翼翼的时间久了,的确是会让人心中感到不耐烦。

如果是当年的章献明肃皇后,肯定早就发作了。当今的圣人虽说按宋用臣心中的印象,与章献皇后差了太远,姓格强硬的地方,远不及曹太皇和高太后。可人也是会变的,能忍到现在,也算是有耐姓了。

可再怎么想,也不该在有王安石这位平章军国重事在的地方下手。

要是王安石站出来,天子再一配合,皇后肯定落不着好。

皇后这是弄错了时间地点,换做在后宫中,想怎么做就能怎么做。现在殿上至少有两位宰辅,虽说都递了辞表,可地位怎么也不可能因为一两封辞表被动摇。

就是韩枢密可能会站在皇后一边,可那边还有个王平章。

从他这边往向屏风后,就看见刘惟简的一张黑脸挂在,唇上青青白白,毫无血色,真是有些好笑的一副表情。但宋用臣现在笑不出来。他知道,自己脸上的表情只会是刘惟简的翻版。

成了众矢之的,宋用臣弯腰,瞥了眼沙盘,然后一咬牙,“皇帝有旨,今天经筵到此为止。”

王安石眉头一皱,想要有些动作,视线扫过殿上的其他人,却又有些犹豫不定。

韩冈迎上王安石的视线,稳定有力的点了点头。

“臣谨遵圣谕。”韩冈先一步转过身向上面行礼。

他可不是忠心耿耿的臣子,且弄不清情况,随便表态也不是好事。

王安石略一犹豫,也行礼恭送天子离开。

就像之前过来时一样,赵顼被抬着离开了集英殿。皇后拉着太子,跟随在后。

送走皇帝一家,臣子们才退出了集英殿。

“岳父。”韩冈低声叫着走在前面的王安石。

王安石比韩冈叫他还要早上一步停下了脚。

当然不可能那么简单就回去,走慢一点,如果有什么要招呼的,肯定会有人赶上了。

陆佃看见韩冈和王安石收住了脚步,停在集英殿外。但他这等小官却不敢走慢一点。天子不豫的时候,宰辅们是可以随意进寝宫,探视天子病情。其他人可没这个资格。

看着程颢,与王安石和韩冈打过招呼之后,就先行离开,陆佃也上前,草草向王安石与韩冈行过礼,跟着大队一起走。

天子到底是出了什么事?陆佃心中翻腾着。

就是方才没有多想,现在看了韩冈和王安石的反应,他也知道不对劲了。

天子肯定是想要让两家联手进剿韩冈,可这偏偏正中韩冈下怀。完全成了他的独角戏。刚才殿上皇后的插话,本以为是给天子一个下台的台阶,现在看来好像有别的原因。

陆佃暗自想着。皇后一直在经筵上拉偏架,要不然蔡卞也不会落得那么惨。

这根本就不能算是经筵。本应是讲授儒门经典,然后联系实际,以资天子治国。但韩冈是讲他的那一套东西,然后东拉西扯到经典上——今天是《春秋》。

可世人不会想太多,韩冈在经筵上的胜利,到了明天肯定就会传出去。

不过方才的一幕让他觉得,情况或许能有些变化。

……………………“玉昆,天子怎么了。”

待人都走远,王安石问韩冈,语气稍显急促。

韩冈摇摇头,“小婿不知。”

天子一直都是瘫着的。除非是病情好转,身子有更多的地方可以动弹。否则就算是眼皮和手指的情况再恶化,也很难分辨得出来。他方才在殿上隔了那么远,怎么可能看得清天子到底出了什么事?

“岳父也不用担心,到了如今,情况还能再坏吗?”他又反问着。

王安石没有因为韩冈的话而释怀,以赵顼的情况,再坏一点,就只有一个可能了。

不过他也没有太伤感,王安石其实跟很多人一样,现在都有些麻木了。年过花甲,送走的亲朋好友不知多少,早就看惯了生死。经历的皇帝已经有两个了,再多了一个,其实早已习惯。

天子苟延残喘到此时,朝臣们也都习以为常。每天上朝时,看到的是帘后的皇后,而不是正襟危坐的皇帝。且宰辅们经过了对天子的集体瞒骗,还能有多少忠义之心,其实也很难说了。希望京城内外大小寺庙的钟声早点敲响的,不会是少数。

王安石慢慢的走着,忽然开口:“物竞天择,适者生存。想不到玉昆你格物致知最后格出了这么一个道理。”

韩冈没想到王安石会将话题转移到方才的经辩上去,“怎么,岳父觉得不对?”

“不。”王安石摇头,很慢的道,“是太对了。”

他慢慢的向前走,让韩冈跟在后面,久久也不开口。

大宋的治国之术,其实是儒法兼有,王霸道杂辅之。熟读经史,就会明白这一点。

而韩冈的物竞天择,说是与仁术相对。可就是放在大宋,放眼一望,也到处都是例证,人与禽兽之道并存于世。拿来做华夷之辨的证据,明了人与禽兽的分别,其实还不如说是通行于世的法则。

无论是物理还是算数,包括给韩冈镀上金身的防疫之法,在王安石看来,依然只是杂术。研究的越深,就与张载所谈论的道相背离。这是王安石始终维持着信心的缘故,但今曰再一看,韩冈却当真往大道上走了。

现在新学的后辈一个都不成器,难道还能指望自己继续压着自家女婿?

那天韩冈过来拜访后,留下的话等于是下了最后通牒,王安石一下被刺激,早就决定要跟气学顶到底。

已经失去了压制气学的最好时机。或许说,机会从来都没有来过。

王安石明白,今曰同上经筵,虽然是草草收场,可最后得胜的,终究是韩冈,这样的结果,对王安石和程颢都是很大的打击。

都想在经义上给韩冈迎头一击,但谁也没能想到韩冈会从春秋中阐发出‘物竞天择、适者生存’的理论。

说句实话,王安石也给弄得措手不及。他一开始根本就没想到韩冈有在五经上与真正的儒者辩难的才能。

当韩冈提起春秋,程颢主动上去配合。可若是程颢不配合,王安石也会帮一把。

春秋讲的是尊王攘夷,董仲舒最为鼓吹的天人感应就是其中的重点。董仲舒的《天人三策》中有多少属于《春秋》成分?《春秋繁露》说得又是什么?还有春秋决狱,几乎将春秋提到了五经之首的位置上。

气学最大的破绽就在天人感应上!这是所有人共同的认识。

两人都与韩冈关系密切,韩冈的根底别人不知,王安石和程颢却是十分清楚的。都是儒学宗师,韩冈再怎么隐藏,几番书信往来,议论一下经典,他在儒学上的实际水平,早就被他们给看穿了。

不论韩冈如何改篡对《春秋》的注解,他最后都绕不过董仲舒去。韩冈纵然能别出蹊径,有信心将《春秋》要义另作阐发,可王安石和程颢哪个不是对自己的学问充满自信的?拉回到天人三策中又有何难?

可是韩冈给出的答案,远远超乎想象。

料敌不明,也无怪乎输得彻底。

王安石在想什么,韩冈多多少少也是能猜测得到。

不过他并没有太多自满的地方。

他所主张的学问,从目标,到纲领,再到实行的手段,以及由浅入深的学习路线,其完备程度都是其他学派所不具备的。

比起当世所有的学派,气学本身的根基不在那几本圣人编写的经书上,而是整个自然。

这就是时代的差距,这就是眼界高度的区别。是有着一个世界几百年千万人的积累作为底牌,差距之大,远远超乎王安石、程颢所能想象的极限。

王安石、程颢纵然再有能力,也缩短不了这样的距离。

胜利是理所当然,唯一的问题只是实行者的水平,会在这一过程中造成的波折多寡而已。

“枢密,请留步!”身后声音打断了翁婿两人的漫步,杨戬匆匆从后跑了过来,叫着韩冈,“枢密,皇后口谕,请枢密速至福宁殿。”

