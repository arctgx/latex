\section{第一章 庙堂纷纷策平戎(七)}

【第三更】

京城中,并没有因为夜色而告终。

郭忠孝回到家中的时候,已经三更天了。

郭忠孝本以为父亲已经安寝了,但去了正院,才知道父亲郭逵还在书房等候他带回来的消息。

在门外禀告一声,郭忠孝推门进屋,一股酒气冲进鼻中。向屋中一张望,郭逵喝了酒,正靠在书房里间的软榻上。一名小史拿着热手巾,给郭逵擦了脸后,又就手递上一盏醒酒汤。

听到儿子回来的动静,郭逵挥手示意房中的无关人等都出去,只留了父子二人在房中。问郭忠孝道:“韩冈怎么说?”

郭忠孝在郭逵面前站定:“韩冈没有明说,只是孩儿看他的样子,似乎是动心了。”

“哦,是吗?”郭逵端着醒酒汤,笑道:“看来韩玉昆还是不能免俗,免不了要任用私人。”

以此作为交换条件,其实更有问题吧。郭忠孝腹诽着,只是不敢明说出来。

郭逵眼神忽然变得剑一般锐利,深深的钉了儿子一眼:“腹诽就不必了,为父只是说笑罢了。韩冈要是这么简单的人,也走不到今天的这一步。”

郭忠孝张了张嘴,想要为自己辩解,但又不知怎么说,小心思根本瞒不过精明厉害的父亲,只能低头:“孩儿知道了。”

“朝廷爵禄,不是拿来跟人做交易的,不过以韩冈的为人品性,就是当真想要那几个位子,他推荐上来的人当也是有资格有能力只是没运气的,不会滥竽充数。而且如果他荐上来的人不够资格,为父也有办法挡回去。”郭逵顿了一下,示意儿子坐下来,然后缓缓说道:“三年前,王舜臣谎报战功,已经被人揭出来了。”

郭忠孝大惊失色,“谎报战功?鄜延路的王舜臣?怎么会有这种事!”

“自然是他。”郭逵冷笑道,“谎报军功本也不是大事,谁战后不吹嘘,杀良冒功都不鲜见,但传出来时间不对。大战在即,以王舜臣的身份,肯定要做鄜延路的先锋官。想要他这个位置的为数甚众,过去你知我知尽人皆知的事,现在就是把柄了。”

“大人是要保王舜臣?”郭忠孝道,“那这样一来,就不用那熙河路的官职交换了,韩冈和王舜臣听说是生死之交。对韩冈来说,一百个官职都比不上王舜臣的安危重要。”

郭逵暗自摇头。自家的这个儿子虽然是个读书种子,也算聪明,但跟韩冈比起来差得老远。要不然为何他不事先跟自家的儿子提起此事,那是因为他会在韩冈面前露出破绽的缘故。

“挟恩求报,可是会得罪人的。”郭逵笑道,“为父还想你能跟韩冈拉拉交情呢。”

利益交换是利益交换,人情是人情,郭逵在官场日久,自是分得清楚才是。有些事适合做交易的筹码,有些事则就适合做人情。

郭逵的回答让郭忠孝一时无语。片刻后,问道:“那大人准备怎么办?”

谎报战功可大可小,闹大了,论死都有可能。但如果大事化小,也就本官降一官而已,依旧任原职。

“保他一条命吧,不过要打回原形了。天子为了震慑众将,免得他们在战时有样学样,不会轻轻放过。”郭逵扯了一下嘴角,“听说他的年纪比韩冈还要小上一点,只是当年为了做官才改了年纪。河湟功成的时候,据说他才过二十。二十出头的都巡检,从七品的供备库副使!”

郭忠孝知道,自家父亲因兄长战死的荫补得官时,也正好是二十岁,却仅仅是个三班奉职。这个王舜臣,跟韩冈一样少年得志,之前不知有过多少人羡慕。

“保住他的性命难不难?”郭忠孝问道。

“此事一出,他在熙河路的军功肯定就会惹起怀疑了,但他箭术却是实实在在的,曾在天子面前演武。吃两年苦头,立点苦劳功劳,韩冈再求个情,多半就会升回去了……人才难得啊。”

“眼下用人在即,天子应该让他将功赎罪吧。”

“前面为父也说了吧,是有人看上了他的先锋官的位置。怎么还会给他将功赎罪的机会?”

“纵使王舜臣不能为先锋,也不能让那等小人得逞!”郭忠孝沉着脸,首告从来都不是值得鼓励的风气,尤其是为了官位和功劳,更是小人之为。

“种谔也不会。王舜臣虽然跟韩冈走得近,但毕竟也是种家的人,娶得还是种家的女儿。种谔有瓜田李下的嫌疑,保不住王舜臣,但不会给害自己的人占这个便宜去。”

将王舜臣谎报军功的旧事揭出来的那一方,其实也是对着种谔去的。麾下将佐谎报军功,没有查实的主帅本就难辞其咎,加上王舜臣和种家的关系,更会让天子怀疑起当初种谔的功劳有多少是虚构的。其实有很大机会将种谔一并拉下马。

但郭忠孝相信自己父亲的判断,种谔应该能保住自己。他本人也觉得,在开战之前,天子不会动一路主帅。最多也是拿着王舜臣敲打一下种谔,杀鸡儆猴,给所有人提个醒,不要有侥幸之心,但作为被杀给猴子看的鸡,王舜臣的结果就难说了。这时候郭逵的态度便很关键。

郭逵的打算,郭忠孝也算是明白了,的确是卖了韩冈一个大人情。但还有个疑问:“大人是什么时候知道此事的。”

这么重要的事,事前知道而不知会一声,韩冈之后心中肯定会留下芥蒂。若是刚刚知道不久,那还好说些。

郭逵微微一笑,“明天早上。”

郭忠孝没有话说了,姜还是老的辣。

王舜臣的事,可以放一边了。见到父亲谈兴正高,趁这个机会,郭忠孝有很多事想要问一问。

“大人要去河北,靠韩冈当真有用吗?”这个问题郭忠孝一直想问,韩冈一个同群牧使,怎么有资格插话执政的请郡的要求。

郭逵低头啜了一口已经变得温热起来的醒酒汤,一股酸气直冲囟门,双眼不由自主的就眯了起来,“知道章惇为什么去职吗?”

“……难道因为是韩冈?”郭忠孝疑惑道,听父亲的口气是这个意思,可他觉得应该不是这么一回事,“不是其弟强买民田的缘故吗?”

“二哥你以为强买民田能有多大的事?”郭逵冷笑,今天晚上可能真的是醉了,说话也没有了平日的顾忌,“重臣出外,岂有因为田地的缘故?只是表面的借口而已。”

“可是……”

“没什么可是不可是的。天子不会留太多新党中人在朝中,尤其是王安石的那几位得力部将,他们过去得罪的人太多,留在朝中平添乱事。但天子还在犹豫中,但等到韩冈上京,不想看到章惇与韩冈一唱一和,天子就动手了。”郭逵哼哼的冷笑两声,不知是在嘲笑谁,“别说章惇,就是吕惠卿,他在朝中时间不多了。若是国势艰难之时,吕惠卿这等能生财兴利的辅臣还有留用的必要。可现在国中形势看起来如同花团锦簇一般,留着他不闹心吗?天子要的是平稳,可偏偏吕惠卿想要有所作为。”

“手实法乃是残民之术,此等害民之臣,本就不该留在朝堂之上!”

“残民?你说哪个民啊?一等户二等户加起来,户口有后三等十分之一吗?”郭逵手扶着额头,“三等户以下,哪个要担心被人告发隐瞒财产?只有一二等户才要担心。”

“过了河,桥就该拆了。皇帝就是这样的人。国也富了,兵也强了,还留着王安石做什么?保着新法不变,王安石这个众矢之的去了对天子来说更好一点。现在章惇、吕惠卿不过是循着王安石的路罢了。”

郭忠孝终于开始冒冷汗了,“大人,还请慎言。”

“家里面说说有什么关系?”郭逵瞪着郭忠孝,几个儿子中以他最为出色,却还是太幼稚了:“你若是只想做个荫补官,为父就不会跟你说这些话,反正你也够不到这一级。但你如今想要考进士,为父就不能不说!朝堂之上,可不是你们平常挂在嘴边的东西。不聪明一点,给人卖了还帮着数钱。”

郭忠孝已经不知自己是该点头还是该摇头,郭逵的话完全不合他学到的圣贤教义,但郭忠孝更清楚,他父亲没必要骗自己。

“你看到的东西,和实际的情况,永远都不会是一回事。”郭逵仰天叹了一口气:“为父在外面有个贪于财货的名声,你以为这为父想要的吗?”

他站起身,居高临下的俯视着发愣中的儿子。

“都说辽国内乱是攻夏的良机,可辽主之亡绝非意外,耶律乙辛乃是有备而为……既然如此,辽国的内乱又能持续多久?不要小瞧耶律乙辛。”郭逵笑了一笑,透着浓浓的讽刺:“有件事为父从没跟人说过,我旧年曾跟耶律乙辛当面打过交道。”看了眼陷入呆滞中的儿子,他补充道,“两次!”

