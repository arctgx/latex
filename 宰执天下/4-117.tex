\section{第19章 萧萧马鸣乱真伪(三)}

【写起来就忘了时间,差点忘了更新,抱歉,抱歉。】

李育冷汗涔涔,背后都是湿淋淋的,他的几个同伴马竺、周毖、陈震也是同样汗流浃背。

虽然他们所在的酒楼包厢,并没有金碧辉煌的奢华,但清雅宜人的布置处处体现着匠心独运四个字,陈设壁挂无不是上品。论起格调,远在让人纸醉金迷的酒楼之上。

淡淡的檀香气从别致的,飘散出来,虽不浓烈,可搭配起房中的布置,却是恰到好处。就算没有点了妓女陪酒,歌舞助兴,仅仅是室内室外的环境,加上设宴主人的身份,绝对可以算是很出色的招待了。

但几位被推荐到韩冈面前,担任幕僚的关学弟子,却早就没了享受的心情——尽管韩冈给他们的,并不是声严色厉的质问,也不是针锋相对的考校,只不过是在闲聊中随口问几个问题而已。

广西、交趾的风土人情,他们问遍了京城能找到的广西人;旧年朝廷在广西的战事,他们找到了曾经去过关西的将领;韩冈过去的著述,他们也都找来详读;为了做好这些准备,他们全力动员了所有自己所能影响到的人脉关系。可是韩冈问的问题,与李育他们事前做的功课全然无关。轻描淡写的几个问题出口,就仿佛拉家常一般。

不论哪个回答,都只能看到韩冈再瞬间皱了一下眉头,然后很快的又舒展开来,让人只以为自己眼花。但李育等人不会这样认为,虽然自己的的确确是竭尽全力,可显而易见的他们并不能让韩冈满意。

几人心中都有些急。考校他们的是关学的衣钵传人之一,未来前途无量,必然会将关学一脉发扬光大。不能通过他的考核,可是会失去了眼前这个难得的机会。

曾与韩冈齐心合力,将从河北涌来京城的三个幕僚,如今都任官出外。魏平真、方兴都是无意科场,直接去了地方州县任官。而另一位叫游醇的,是二程的弟子。据说他本准备要考进士,靠着身上的官职,很顺利的通过了锁厅试。只不过今年还是落了榜,没有考上进士。现在好像已经回到了洛阳去,重新在嵩阳书院聆听二程教诲。

在韩冈身边不过一年,三名幕僚都得到了官职。关学一脉的弟子中,想在韩冈幕中混一个出身的为数不少,这等天上掉下来的好事总不能便宜了外人,可最后能得到张载推荐的也就他们寥寥数人而已,李复四人每一个愿意放弃。

“你们也知道,除了班直之外,军中的空额哪边都不会少。上阵厮杀并不是兵籍上说三千人,就有三千人。有的有两千,有的有两千五,若是运气差时,就只有一千五百、甚至一千人。所以上阵之前,最重要的就是要先将人数清点好,这样才能分派任务。只是要清点军中空额,则是会犯到下面忌讳,”韩冈笑道,“当初我为了点清来援熙河路的各家到底有多少兵马,可是很费了一番周折。”

韩冈顿了一下,环目一扫,几人却都是张口结舌。这是算不上太难的问题,可惜李育四人都接不上话来。

心中暗叹,还是少了历练。他的考题不是什么运筹帷幄,而是具体到行军打仗中的每一个细节,这是他的实际需要。

韩冈作为统安南经略招讨司中管后勤、参谋的第二号人物,必须要掌握第一手的准确数字,不论是己方还是敌方。这就需要幕僚们从得到的庞大情报中进行分析选择,同时也要针对内部,让章惇、韩冈几位主帅,能了解军中内部的实际——知己知彼方能百战百胜。

这是李复他们接手后要做的工作,可眼下,他们很明显还不够资格,竟被问得张口结舌。

韩冈并不是打算给他们当头一棒,这样太过明显的下马威用不着自己出手,真做了就反而显得自己器量太小。到了广西之后,他们自然会看到现实,现在只不过是出于好意的提前点醒。

李复等人都是读书求学中的新人,经验能力肯定是没办法与当初做了多年幕僚的魏平真、方兴相比。不过都是关学一脉,韩冈不介意拿出一点时间,来从头培养自己的班底。只是在这之前,必须要让他们有一个清醒的认识,他们的任务不是穿着鹤氅、摇着羽扇,而是繁重忙碌的一系列庶务工作。

至于谋主、策士什么的,韩冈当然需要,有人能为他在混乱的朝堂上站稳脚跟而进行指点,当然是件好事,可凭着眼下张口结舌的几人,想来也不会给他带来惊喜。

“行军打仗是两件事,阵上指挥不容易,而率军赶路也同样不容易。”放弃了前面的话题,韩冈亲自为几位同窗斟酒,换了个简单的问题:“你们可知道,寻常步卒一日能走多远?”

大部来自关西的几人,对于这个简单的问题还是能回答的。李复道,“五十里而争利,则蹶上将军,一般最多也只有三四十里。”

三句话不离兵法,这点让韩冈失笑。但他立刻掩去了笑容,点头道:“说的没错。一般来说,普通禁军在比较平坦的官道上行军,一天差不多三十里、四十里的样子,精锐一点的禁军则可以一天走出五六十里。如果是临战前的急行军,精兵一日百余里,连续两三天,都是可以的。不过换作是敌情不明的情况,则只有二三十里,再多就会有危险。”

马竺道:“听说龙学从桂州南下时走得更快。”

“抵达宾州前的那几天,的确是一日百余里。一个是因为是在境内,不虞遇到贼军埋伏,可以放开来走。还有一个原因,就是我带在身边的兵力少,才八百多人,易于管束。”韩冈叹道,“如果能让上万步卒在一条道路上做到一日百里,这份能耐就当得起名将二字了。”

“也就是要兵精将良?”一直以来都说话不多的周毖问着。

“的确,要想做到一日百里,兵是要精,将是要良。可不仅仅是兵精将良的问题。怎么让士兵在长途行军时,不至于士气低落,可以随时接战,不是有人望的将领说句话就够的——”

李复立刻接口道:“善抚士卒?”

韩冈笑了一笑,空口白话的四个字可以写在书上。可没有具体的条款,就没有任何实际意义,细节之中才能见到真章。

陈震沉声道:“与士卒最下者同衣食,卧不设席,行不骑乘,亲裹赢粮,与士卒分劳苦。”

韩冈的脸上还是保持着淡然的微笑,“这是吴起!”

依然是没有说到点子上。这个时代,想让军中将帅学习吴起的做法根本不可能。其实如果当真将医疗卫生制度上读得熟了,并加以理解——尽管里面没有说这方面的事——但如果有心,能做到触类旁通的。

李复抓抓耳朵,陈震则头歪了一点,与同学们一起用疑惑的视线看这韩冈,等他说出答案。

“很简单,到落脚的地方能用热水洗个脚,能吃到热饭热菜,喝到热汤,对听命打仗的士兵来说,没有比这个更好的待遇了。”韩冈神容严肃,“但这些事必须要快,不能等到卒伍们的身子冷下来。一旦人歇息下来,出过汗的身子很快就会变冷,再吃些冷食,内外交加,很容易就病倒一片。而没有热水洗脚,不能让走了一天的两脚血脉活络畅通,第二天就别想走得更远。”

这算是军中的常识,也是出门在外的旅行者的常识,就算是码头上抗包的苦力也都了解一些,但放在李复等读书人身上,只要没有经历过,可是一个都想不到,只能点头受教。

韩冈仔细盯着他们脸上的表情,倒是没有见到哪一个有不以为然的表情。

这就好。韩冈心中满意,继续道,“军中的士气不可能靠着干粮、咸菜来保持,有肉有汤的热饭菜比起还没到手的银绢恩赏更有效。洗脚的热水,能让士兵们第二天多走上十里。只是这些事说是很容易,要安排起来不容易了……”

……………………

来自秦凤、泾原两路的西军刚刚南下,罗兀城也只是刚刚捉到几个刺探军情的斥候,韩冈还在京城考校他的幕僚,而河东早早的就拉开了战争的序幕。

不同于战事还在酝酿之中的广西,也不同于尚不见动静的关西,河东的战鼓此时已经敲响。

代州、宁化、岢岚、火山,河东北方边境各军州面向辽国西京道的关隘,皆是戒备森严。

雁门关上,一双双警惕的眼睛盯着北方,防着西京道中的契丹铁骑在丰州战事正酣的时候,乘隙突袭河东。

大军在集结。

来自于河东各州的精锐力量在一个月的时间中,一批批的渡过了黄河,渐次聚集在河东路唯一一片位于黄河西岸的土地上。

而河东路经略安抚使、太原知府、雄武军节度留后郭逵,大宋军中排名第一的大将的将旗,眼下就在府州城上飘扬。

在郭逵的大旗下,汇聚在麟州、府州的六万马步禁军俯首听命,整装待发。而在他们的身后,有近二十万民夫,数万马骡牲畜,为大军的粮秣来回奔忙。

而在此之前,意欲收复失土的宋军,已经将耳目送进了山峦的另一面。

一名宋军斥候潜伏在丰州的山林中。在他身后不远处的战马马鞍前,挂着一个血迹未干的头颅,那是他在路上遇到的落单的西夏哨探。

阻止了西夏哨探带着搜集来的情报回到丰州,年轻的斥候想着更进一步的将党项人的情报搜集回去,两桩功劳一献,自己好歹也能再往上挪一阶。

时间一点点的过去,斥候的战马安安静静的啃食着地上的嫩草,斥候一动不动。在他下方的山道,是丰州驻军每日巡查必走的道路,只要有耐心,肯定能得到他所需要的情报。

一名骑兵忽然从被山壁遮挡的道路上冒了出来,斥候精神一振,一双眼睛也立刻睁大了。跟着这么骑兵之后,是更多的骑兵。一名名、一对对,最后竟是一队多达三百余骑的队伍。从人数上看,大约相当于宋军骑兵一个指挥。

那队骑兵越来越近,最前面的掌旗官已经走到了他的正前方,隔着四十多步,伏在山林中的斥候眯缝起了他那一对如同鹰隼一般锐利的眼睛:‘那是什么……’

尽管这一队三百多骑兵,高高举着西夏的旗帜,但他们的装束大半,怎么看都与党项骑兵有些区别……

不!斥候神色郑重的慢慢摇头,一双眼睛紧紧锁着在面前渐次远去的那一支骑兵队伍。

不是‘有些’,而是‘很大’

是很大的区别!!

“怎么可能?”压得低低的声音如同在噩梦中呻吟,“这怎么可能?!”

