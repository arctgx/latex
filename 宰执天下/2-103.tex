\section{第22章 瞒天过海暗遣兵(八)}

【第一更,求红票、收藏。】

情势急转直下,又一次大胜而归的喜悦还在心头,紧跟着就是意想不到的敌军来袭,两种心情的落差,宛如从天堂落入地狱。站在渭源堡的最高处,王厚低头望着已经把他推到地狱的敌人。

高高竖在半里外的敌军将旗上的名号,是由生造出来的党项文字书写。王厚并不认识这种同样是由横竖撇捺组成、却与汉字截然不同的文字,军中也无人能辨认。不过渭源堡内外数千军,还有不少人在战场上见过这面旗帜,也与这面旗帜下的军队在金鼓声中厮杀多年——旗帜的主人,是西夏国中首屈一指的吐蕃豪族,也是镇守大白高国西南边陲的大将,如果王厚没有猜错的话,当是禹臧家新近登位的族长禹臧花麻亲自领军来袭。

绣在白色旗帜上的禹臧二字,王厚多看了几眼后,眼睛就仿佛被灼痛了一般,不由自主的将视线转移了开去。除了稳定在渭源堡半里之外的大纛,被滚滚烟尘所遮挡的地方,还有着数以千计的敌军。模模糊糊的,让想计算出他们数量的王厚的眼睛盯得生疼。

军中多有人言:人马上万,无边无岸。虽然眼前的贼人决计不到万人,但数千大军汇聚一处,已是浩然如海。黑压压的一片从渭源堡西三里处的军营,一直延伸到堡下。另有数百名骑兵在堡外纵横奔驰,隆隆如雷的蹄声中,扬起的不仅仅是灰黄色的烟尘,还有浓浓的战意。

“为什么西贼的兵能在这里?!”

“这些事可以以后再去查证,先想想眼前……贼军有多少?”

同样站在城头上的王韶没有儿子那么紧张,用着平和淡定的声音询问着。当然,他询问的对象不是王厚,而是知渭源堡王君万、缘边安抚司准备差事赵隆、还有尚无官身、但自束发起就已经身在军中的苗履三人。

计点兵数,是兵学中最基本的科目。能力出色的斥候,或是老于兵事的将领,往往只要一眼,就能看得出眼前的敌军究竟有多少数目,进而推断出敌军的总兵力,并不需要他们排着队来等着数数。

同样的道理,只要有点军事头脑的将领,也都会为了不让自己手下的兵力被人看破,而通过各种手段进行掩饰和伪装。比如就在王韶等人眼前,敌军就用着奔马掀起的尘土,将自己的兵力数量模糊起来。不过有经验的将领还是能说出个大概:

赵隆的回答是:“四千上下。”

苗履则报出:“七千到八千。”

而王君万观察到的数目却是:“六千。”

从三名将领出得到三个不同的答案,王韶选择了中庸之道。

“六千兵……”他从鼻子中冷哼一声,“禹臧花麻未免也太小瞧人了!”

听着王韶的意思,王君万问道,“不用点烽火?”

王韶摇头:“用不着,派回的信使就足够了!”

王韶的自信自有其底气。现在他手中的兵力,就算不包括一千三百余蕃军,以及两千多民伕,再除去跟随王舜臣留在星罗结部主城处、扫荡残兵的三个指挥,依然保持着两千一百这个数目。虽然禹臧花麻带来攻打渭源堡差不多有六千骑,可真要在城下硬拼起来,不一定能在王韶的两千兵手上占下便宜,更别提还有蕃军和民伕随时可以补充上阵。

——无论是契丹还是党项,又或是吐蕃,只要是跟大宋有过战争的异族,都明白一个道理:布下箭阵的宋军阵列不能去冲,而守在城下的汉人更是不能去招惹。当汉人有城池可以依靠的时候,其战斗力往往是打着滚往上翻,尤其是西军,最擅长的就是倚城而战。要不然,大宋开国以来,也不会在山区中不停的大兴土木。

而王厚那边忧心难解,紧皱着眉:“就怕王舜臣那里会有麻烦。”

王韶放心地很:“不用担心他。以吐蕃人的攻城手段,星罗结城不是这么好打下来的。屯在城中的粮秣当还没烧,城池打下来时也没有大的损坏。王舜臣手上的三个指挥更都是精锐,才两成不到的空额,足足有一千三百人啊……”

一个指挥正常的兵数当是在五百人,不过由于军中普遍的吃空饷喝兵血的情况存在,足额满编这四个字往往只存在于兵籍簿上。一般来说越是精锐,空额的比例就越少,王韶留给王舜臣的三个指挥都是精兵强将,空额就只有一成多一点。能强过这个数字的只有东京城中的龙卫神卫捧日天武这上四军了。

就像自古渭寨今次出征的三千官军,在编制上的数字是四千。而渭源堡,在王君万上任后,堡中的驻军得到了加强。按编制是三百兵,而实际上,也达到了两百出头。少掉的一百兵便是空额。这些幽灵士兵的俸禄,就给各级军官们瓜分了。

只不过这个比例也只有常年与党项和吐蕃交战的西军才能达到。论起兵员空额,关西的军队算是大宋百万禁军厢军中最少的一路,一般都能保证实际编制的七成到八成。而最坏的情况,就是江南,能有五成就了不得了,而广南两路由于天高皇帝远,实际兵力往往只能达到编制的三成。

这也是为什么从天子到王安石,再到蔡挺、张载,都想推行将兵法的缘故。听说有两千敌军来袭,便点出四千兵马去迎战。从兵力上算是绰绰有余。可到了战场上,却发现只有两千兵,再去掉其中不堪战的,就只剩下一千出头。这样的笑话却是根本让人笑不出来。王舜臣手上是空额仅仅一成多的精锐,王韶相信他应该籍此能多守几日。

“那西面的营垒会不会有问题?”苗履以手加额,忧心忡忡的望着远处的营寨,领军驻扎在寨中的是苗授这位西路都巡检,更是他的父亲,“蕃军可是有一多半在那里,民伕也有一千,家严手上才一个指挥……”

“授之岂会压不住纳芝临占部的蕃人?你这做儿子的难道不知道你父亲有什么手段?”

王韶同样不担心苗授。那座营垒从一开始,就是为了保护筑堡民伕而设立的,造得坚固异常,并没有打丝毫折扣。而且其位置也是跟渭源堡一起,形成了最适合防守的犄角之势。以眼下禹臧花麻的兵力,并不足以分兵同时攻打渭源堡和营垒。如果选择一个主攻方向,那无论王韶还是苗授,都不会是保守的性格。

“若是木征投靠了禹臧花麻怎么办?不然禹臧花麻怎么能出现在渭源堡这里?中间还隔个武胜军啊!现在仅仅是禹臧家的兵,等到木征把他的军队调来……”

“木征绝不会投靠禹臧花麻!”王韶的判语斩钉截铁,“他……”

话音刚起,一只利箭就从城下蹿了上来,直奔王韶面门。王君万眼疾手快,手一张,一把就将长箭抄在手中。掌心兀自火辣辣的,可王君万却立刻从身边的卫士腰间抢过一张弓,搭上箭就要射回去。但城头下,一名骑兵正举着一张大弓,在蕃人的欢呼声中越奔越远,方才的那一箭竟然是驰射!

“好箭术啊……”王韶推开脸色发白的一群失职亲卫,毫不在乎的向下望去。嘴角露出一丝冰寒刺骨、让王君万和苗履都心惊胆战的笑容,“看起来禹臧花麻有些急了,这不是激我出战嘛!”他又回头,笑得更为阴冷,“……要是木征投效了禹臧花麻,可会这般着急?”

王君万和苗履都安心下来,只是王厚了解他的父亲。他在王韶的眼中,很清楚的看到了一丝焦急和紧张。

‘究竟是在担心哪里,渭源、西营、王舜臣,还是别的地方?’王厚看得出来,想不明白。

………………

“木征绝不会投靠禹臧花麻!”韩冈一口断言。略略高亢的声音,传达了他对智缘的担忧不屑一顾的心情。

但智缘一对花白的长眉仍然紧锁着。就在一刻钟以前,他都不会想到禹臧家的军队竟然会出现在渭源堡下。更不会想到会在去渭源的半路上碰到。从王韶派回来求援的信使。

通往渭源堡的官道边,韩冈、智缘以及护卫他们的一队骑兵停了下来,纷纷望着西面远处的群山。隔着四十多里地,再灵敏的耳朵也听不到远处的厮杀,但从信使王惟新口中已经打听清楚了这个紧急军情。

“木征真的不会投靠禹臧花麻?”王惟新显得比智缘还要焦急,趁贼军还没有合围,加急冲出渭源堡后,他的心思就七上八下的,惶惶失措。要是王韶出了意外,他这个亲卫哪里还会有好果子吃。

“木征是吐蕃王家血裔,而禹臧花麻只不过是西夏的看门狗,他就算要投西夏,也是直接投靠兴庆府,而不是兰州,凭禹臧花麻也配?”

韩冈的冷笑比他的话更有效,看到出现他脸上的不屑笑容,王惟新也安心下来。

