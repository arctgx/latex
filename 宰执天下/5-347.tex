\section{第33章 枕惯蹄声梦不惊(四)}

天渐渐的亮了。辽人的确趁着黎明前发动了最后一次攻势,不过在警惕的乱箭之中还是宣告败退。

城南城北的两片集市,皆是一片焦黑,只剩下残垣断壁上的缕缕青烟随风拂动。辽军最初的攻势就是从这里展开,不过现在却看不到几具尸骸,可能同样被烧成了黑炭,也有可能是火起后就顺利撤离了。后者的可能性远比前者要大,不然如院落和道路这样的空地上,应该会有为数众多窒息而死的尸体。

韩冈转身对黄裳笑了一笑,“看来勉仲你猜对了。”

当时黄裳和另外两名武将就猜测辽人利用城外的建筑潜近城池,只是声东击西的战术,现在看来似乎并没有错。

“猜没猜中都一样啊。”黄裳苦笑着摇头。猜对猜错都毫无意义。城中的防御措施本就是为了应对全线进攻而计划的,岂会为辽人的计策而影响?

但也在这一夜中,太谷城内储备的箭矢消耗超过三分之一,而弓弩损坏也将近一成。同样规模的守备力度,城中最多只能再支撑两天,接下来就要用人命来拼了。

不过辽人也不可能再来两次三次昨夜那样等级的进攻了。只要看看外面就很清楚,辽人死伤枕藉,数百近千之多。

这些人,绝大多数身着甲胄,在辽军尚未全数铁甲化的现在,必然都是萧十三麾下的精锐。相对于整体兵力虽少,但绝对是伤筋动骨的损失了。而且还有那些虽然受伤但还有爬回去气力和运气的,数目只会比躺在城下的更多。

城头上这时又有些乱声,很快就有人来报,说是从辽军的营地那边来了一队骑兵,过来想将尸体和重伤员都拖回去。

没人脸上能看到担心的神色,倒是人人带笑,这完全是犯浑嘛。

“萧枢密被气糊涂了吧?”

“若是发了疯才好。”

倒是田腴清醒:“萧十三再糊涂也不至于下这样的命令。多是部族军来救自家人的。”

片刻之后,城上再来报告,就说是城头上的一阵乱箭将他们又赶跑了,还顺带留下了十多人。然后就再不见声息。

到了中午,韩冈巡视过城池四壁守军,又去医院探望了在昨夜受了伤的伤兵——基本上都是意外,只有一人是中了流箭——终于城外又有了动静。辽军的骑兵开始接近城门,四座城门都有,总数差不多有七八千。

那些骑兵没有绕城而行,只是静静的停在离城一里多的地方。但那并不是辽军继续进攻的标志,而是撤退。

从城头上,甚至不用望远镜都能看见驻扎在城外的辽军,正大批的从背离城池的方向离开他们临时驻扎的村庄,一队队的向着地平线的远方行去。

随着辽人越走越多,越来越远,越来越多的人了解到了辽军的动向。欢呼声便渐次而起,不可遏制。传遍了城墙,传遍了城中。

“撤了,撤了!辽狗撤了!”

城上城下,官兵百姓,皆是欢呼雀跃。无论是达官贵人,还是市井小民,无论是僧道,还是平民,都是欣喜欲狂。

数万辽师围城,虽然仅有一日,但之前准备御敌时的压力却如同阴云一般笼罩在所有人的头上。如今云开雾散,又如何能不欣喜欲狂。

可是相对于全城军民越来越响亮的欢呼,韩冈的神色却没有任何变化,双眉反而渐渐的拧起,他周围的幕僚和下属,也因这位制置使沉静如初的表情而逐渐冷却下来。

“不要庆功得太早。只要辽贼还有一兵一卒留在河东,就不是欢呼胜利的时候。”韩冈声调低沉的一盆冰水浇到僚属们的头上,“辽贼究竟是向南还是向北,这是必须要先查清楚的!”

在韩冈的威压下,制置使司的成员们收起了喜乐之心,开始成一圈低声讨论起:

“萧十三以骑兵隔绝消息,多半是意在南来之兵!”

“但辽贼移动的方向似乎是向北走的。”

“万一是陷阱呢?”

韩冈听着幕僚们的讨论,知道很快就会有一份商议过后的文稿摆到他的面前,主要就是以之前针对辽人南下而制定的计划书为蓝本,加以修改。

不过韩冈其实并不是这么想,以他对辽军的了解,昨夜一战后的士气和兵力的损失,让萧十三很难再冒着巨大的风险进行一场大规模的决战。

而且辽贼个个抢得身家丰厚,谁还会再搏命?要不是以为太谷城能一鼓即破,城中又是金银无数,昨夜他们也不可能那么拼命。

但战阵上什么事都有可能发生,辽人不是不可能南下。

所以现在就得看章楶的了,希望他不会让自己失望。

……………………

辽军正在撤退。

在退回放养马匹的河畔绿地之后,便纷纷上马,准备启程北返。

萧十三的眼前这一片有些乱的场面,堂堂枢密使的脸色越发的阴郁起来,

“枢密,其实还是有机会的。”一名幕僚在仔细观察过萧十三的神色之后,终于有了决断。

萧十三沉着脸反问,“什么机会?”

“援军。宋人的援军!”

原本太谷县就是陷阱,韩冈拿自己当做鱼饵的陷阱。这一点,萧十三以下很多人都看到了。

但若不是鱼饵本身太过美味,而鱼钩看起来也很脆弱,萧十三也不会赌上这一把,可惜他失败了。不过既然失败了,他就不打算再去追加赌注,去试着翻盘,那样的结果只会越输越多,直至输光了本钱。

从代州、石岭关、榆次县来推断太谷县的城防,如今已经确定是个巨大的错误。但到现在为止,萧十三也想不通什么时候韩冈在太谷城中调入了那么多兵将。从几十支来源不同的探马那里,甚至包括不同时间抓来的俘虏,他得到的是几乎同样内容的回报——太谷县中的兵力不可能超过五千。若非如此,萧十三无论如何都不会选择去攻打太谷城。

“宋人的援军一天就走了八里路,你告诉我,他们是怎么想的?!”

早就知道是陷阱,跳过一次了,好不容易爬上来,难道还要向更深的地方跳第二次?!

如果是别人充任河东主帅,萧十三还有可能再去赌上一把,可现在坐镇城中的是韩冈,又刚刚表现了他的能力,萧十三又怎么可能还会犯傻。

“……或许是宋将胆小如鼠!”

“统领援军的宋将是嫌脑袋在脖子上呆的时间太久了,打算换一换地方吗?宋国的皇帝和皇后怎么可能容忍有武将将太子置入险境?”

韩冈现在的性命是跟宋国太子挂在一起的。虽然三十岁的宰辅日后可能会很危险,以韩冈现如今在天下万民中的声望,甚至有可能成为尚父一样的人物,但只要南朝的皇太子还需要他这位韩菩萨,南朝的皇帝皇后就绝不会想看到他有任何损伤。

萧十三相信,南朝的将领们都能明白这一点。可是太谷县被围后,援军却用着蜗牛一样的速度前进。要说这不是韩冈事先的吩咐,又怎么可能。围城打援的确是一个好招数,但萧十三已经不会去幻想这一次能够成功。

先退吧,趁损失变得更大之前先离开再说。如果天欲兴辽,就让宋军追上来吧,这样的话,野战中一举逆转,绝不是白纸做梦。

……………………

乌鲁紧盯着不远处的中军。

在旗帜下,有着这一战最主要的责任人。

连同自己在内总共三百另五人,回来的只有两百三十多,八成还不到,而且回来的也是人人带伤。就是乌鲁本人,脖子上也缠着一圈捆绑伤口的绷带——这也是从代州那里得来的。

明明太谷城外宋人早就做好了准备,但萧十三却还坚持去攻打太谷城。

这上千战死的同袍,是萧十三那贱种贪功害死的。要不然,早就该带着打草谷得来的收获,返回大同府了。何须现在满心失望和颓丧的返回北方。

乌鲁低头看着胯下的枣红马,马鞍之下,连脊梁骨几乎都能看得一清二楚。干瘪、瘦削,已经完全不见让举族上下都羡慕不已的良驹的形象了。这一匹还算好的,乌鲁总共带了三匹南下,其他两匹的情况只会更糟。

春来战马体弱,经过了一个冬天,战马身上积存的膘已经都消耗光了,春时不经将养却赶着南下,已经有大批体质稍弱的战马倒毙路旁。现在有用了几天时间,多走了几百里冤枉路,等于是又要多损失上一批战马。

就算是这样,只凭一路得来的收获,还能说是一桩胜利。但这样的胜利再来个几次,大辽也不剩多少好战马了。

乌鲁的手紧紧攥着刀鞘,投向萧十三的眼神中透着愤怒和桀骜。

“乌鲁!别犯浑!”

一声焦急的呵斥从身后传来,不过来得更早的是探过来的一只手,紧紧抓住了乌鲁坐骑的缰绳。

乌鲁拧过脑袋,铜铃似的一对圆眼凶光四射盯着胡里改。

老胡里改没松手,“低下头,低下头!”

乌鲁怕老胡里改声音太大,引起他人的注意,也不应低下头,

“越是这时候,萧十三越是要杀人。没能打下太谷城,现在他要立威,就只能从你们身上着落了。别给他机会!你没看其他人一个个乖的跟孙子一样?”

乌鲁几乎要翻脸,他哪里可能真的去杀萧十三,他还没疯!乌鲁压低声音发怒道:“你当我想不到?!我怎么可能做这等事!”

“我知道你当然能想到,但你多瞪他一眼就是把脖子往刀口上多凑一分。不是吗?若他真的拿你开刀,举族上下就都完了,这一分一毫的风险也不能冒啊!”

老胡里改将乌鲁的缰绳越攥越紧,眼角的余光还瞥着中军大纛的方向,生怕萧十三注意到乌鲁方才的不驯眼神,恼羞成怒后迁怒到头上来。

宫分、皮室两军,萧十三动不了;五院、六院、乙室等贵胄更不用说;出身大部落的就算有些冒犯,萧十三肯定也只会当没看到;而小部落就跟屁一样无足轻重。最危险的就是他们这等人数不多不少的中等部族。偏偏他们回去的这一路上,还跟着中军,这是踩在刀锋上走路啊。

趁损失不大提早离开,还算是做得不错了。真正错的,还是没查探明白太谷城中宋军兵力的数目,同时也是对手太强的缘故。想明白了这一点,老胡里改对萧十三的愤怒还不及乌鲁的十分之一。

乌鲁并不关心老胡里改现在想什么,他依然纠结于自家损失的儿郎,“死伤这么多人,等回去后,定要求尚父给我们所有人一个交待!”

老胡里改知道乌鲁这是在说气话,也不多劝。等他真的发了疯想要连同其余各部跟萧十三过不去,在想办法不迟,反正那时候,也有族中的长老能拦着他。

老胡里改回头望了已经被远远抛到身后的城池一眼,他可不信,宋国的那位神佛弟子,会高抬贵手的放人一马。心中暗叹,‘先能回去再说。’

