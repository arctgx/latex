\section{第34章 云庭降鹤宴华堂(下)}

文及甫如果得罪,那么文彦博少不了要请辞眼下的职位,相对的,韩冈这边也许能少个不确定的因素——即便之前已经经过了谈判,但韩冈也不可能全心全意的相信文彦博的人品。如果文彦博去职,只要不是再弄来一个曾经做过宰执的元老,不论换了谁来,韩冈自问依靠自己的名位都能让他影响不了襄汉漕渠的工程。而有了文彦博之事在前,天子应该也不会选个会找麻烦的人来洛阳。

不过到底河南知府兼西京留守的位置到底会不会换人,此事还是得走着瞧,说不准赵顼得给文彦博留个体面,不像政事堂中的变化,几乎已经可以确定。富弼方才说‘眼下政事堂中只有一相一参’,明显就是把吴充排除掉了。

这半年多来,政事堂中的成员,交替变换得就跟走马灯一样。无论冯京和吴充,都是上去后就下来了。

从表面上看,都是台谏官的功劳,但实际上,自然是他们不能让当今天子满意。要不然即便冯京、吴充犯了再大的错误,天子都可以帮他们挡下来——不过就是将台谏清洗一遍而已,王安石刚刚推行新法的时候,赵顼不是没干过。

可事实上的情况,冯京和吴充显然是不如王安石受到赵顼的信任,当然不可能为了他们而将弹劾他们的台谏官全都赶出京城。冯京已经去了河阳府,而吴充则说不准会去哪里了,他的儿子给他太大的拖累。

接替吴充的王珪,是政事堂中的老人,担任参知政事已经快十年了,可在朝堂上,别说王安石,就是相比起吴充来,王珪他在政事堂中的存在感就像清晨的雾气一般稀薄。

几年前王安石尚未第二次登上相位,吕大防曾经说动王珪联名上书推荐张载进京,这件事上韩冈欠了他一份人情。但接来下王安石进京后,王珪便再没有跟张载有正常以上的联系,也没有说趁机帮着宣扬关学一番,让好不容易卖出的人情,一下就淡了下来。

王珪在政事堂的一贯表现也是类似于此,不跟正当轴者为敌,不与天子的心意相违逆,除了做官的目标肯定是放在宰相之位上,要说王珪这位晋身东府并不比王安石晚多少的政事堂老人有何独特的政见,韩冈抓破脑袋也想不出来。

也许王珪做上宰相后,会摇身一变,变成另外一个强势的性格,但韩冈觉得,王珪多半还是会成为一个对天子唯命是从的宰相。这个性格应该会比较符合当今天子的都需要,且连续三名宰相去职,为了维护政事堂的权威,王珪也应该在相位上坐上比较长的时间。

而没有了倾向比较明显的吴充,以王珪的性格和能力,应该压制不住虎视眈眈的吕惠卿。同时从王珪的实际需要上看,他的确应该引荐合适的人选入政事堂;而从政事堂眼下的人数上看,也是亟需补充新血。

就不知道政事堂中的新血,是从枢密院转调,还是从三司、学士院或京城以外提拔合适的人选。

条件适合的人选,数量不少,猜都没法儿猜。其实除去年龄和资历不看,韩冈也是够资格的,直学士都能进枢密院,学士要入东府自然毫无问题。可是年龄和资历就是最大的问题,韩冈也不指望能有这个运气。反正结果两三个月后就能见分晓,也不用去多想。

韩冈心如电转,眨眨眼的功夫,倒是将这一次朝堂变局所造成的影响,推算得差不多了。相对于京城中的风起云涌、海浪滔天,洛阳这里只会是风波初兴、死水微澜。只要不会干扰到襄汉漕运,韩冈也不会去关注太多。

韩冈和富弼在还政堂中又说了有小半个时辰的闲话,外面的从人来禀报,说是有亲眷来访。富弼便告了个罪,让次子富绍京送了韩冈出来往正堂旁的偏厅小坐,等待寿宴开始。

寿诞之日,富弼自然十分忙碌。自己能在寿诞当日与他谈上小半个时辰,富弼交好自己的打算显而易见。一旦传出去,肯定有不少人羡慕,更会影响到许多人的决断。

韩冈坐在偏厅中,虽然只是富府正堂诸多偏厅中的一间,但韩冈是一人独踞,总比熙熙攘攘的其他厅室要强。至于陪客说话的,则是富弼的长孙富直柔,从这个人选上,也可见富弼的想法。

这一次能得邀参加富弼寿诞的的确人不多——这是以宰相的水平来说的——除去进了后院的女眷,也就两百多人的样子,不过送来的礼物堆满了庭院,富家的账房恐怕要辛苦一番。

这么多人,在寿宴开始前便陆续抵达富府。他们之中身份有高有低,关系也分远近,被各自领到不同的厅中招待。如何分配,就要看富家的安排了。但总归有一个原则,以互相之间的亲疏关系来分派。

富家当然不会把文彦博或他的儿子请进这座小厅,只是当韩冈与富直柔说了几句闲话之后,就听见外面有了的动静。再定睛看了眼前的来人,竟然是程珦和程颢,而另外一人不是程颐,而是韩冈正打算找他见上一面的吕大临。

韩冈一见之下,便连忙站起身来行礼。三人看到韩冈倒没有什么的惊讶,应当是被引过来时,先听人说了。

“玉昆来得倒早。”程珦先笑道。

“也只半个时辰而已。”韩冈让过自己的位子,请程珦坐上去。

程珦没有多客气,在韩冈的谦让下,坐上了韩冈方才所坐的上首客座。韩冈又让了程颢坐下,接下来的吕大临,韩冈辞让了一阵,还是让他坐了第三位。

韩冈外在的表现就是一贯的尊师重道,身为贵官,却坐在最下首的位置上,这让程珦和程颢甚至吕大临都很欣赏,而富直柔在惊讶之余后,更是连声称赞。

在富弼的寿辰,其长孙富直柔还在座,韩冈也不好询问吕大临他所撰写的张载行状到底写得怎么样了,也就跟程珦、程颢聊着天,吕大临偶尔插着几句嘴。也知道了程颐有急事去了嵩阳书院,所以没有过来贺寿。不过以程颐的性格,应当也不喜欢喧闹的宴会。

将近中午的时候,富府内热闹的气氛终于到了最高潮。富直柔也告了罪,匆匆离开偏厅。

富家的庭院中摆出了香案,富弼换上了一身朝服,带着富家上下数十口,向着来此宣诏的使臣拜倒。

而所有来客,都来到了庭院中观礼。韩冈看到了文彦博,没想到他竟然还是来了,当真是给富弼面子。当然,也可能有着证明自己没有因为此前的纷争而受到影响的想法。隔着几十步的距离两人对视了一眼,韩冈欠了欠身,而文彦博则是点头示意,观察着两人动作的人们,都是一阵讶异,都没想到两人之间看起来一点芥蒂都没有。

依照惯例,东京城中派了一名中使前来宣诏,以示对老臣的眷顾。而这一位正是韩冈的熟人。

当一名身材高大、面色黧黑、下颌处甚至有几根胡须,看着一点也不像是宦官的中使宣读着诏书,韩冈心中就不免泛起一股荒谬的感觉,日后他可是罪魁祸首之一,眼下他每一次进步,都有可能是给北宋的坟墓上培土。

才几年功夫,童贯就能成为来富弼府上宣诏贺寿的使臣,韩冈心中不无惊讶。身负皇命,出宫宣诏的中使,也分个三六九等,能给元老重臣,地位都不会低。以童贯在宫中的资历,机缘未免太好了一点。

历史也在改变,童贯应该是没机会再封王了,眼下甚至能导致他封王的那一位皇帝都没生出来。不过现在的童贯倒是显得意气风发,抑扬顿挫的将满是好话和套话的诏书高声念了一通。

领过诏书,富弼带着一家老小谢过了天子的恩德,照常例将诏书在堂上供了起来。

圣旨接过了,客人也到齐了,剩下的自然就是富弼的寿宴。宴席采取的是分席制,基本上是两人一席,不过富弼、文彦博、作为天使的童贯,则是一人独坐。

上席之后,席位的安排主要是按着身份地位而来,除此之外,还有名望的因素。为了安排席次,富家肯定是费了不少脑筋。人都争个名分,稍稍出点差错,就会得罪人。

不过韩冈则没有争,反而大加退让。他的地位在宾客中排在前几,只是他没有依从富家的安排,坐到上面去。程珦和程颢不上座,他也便拒绝了富家安排的位置。

“若是朝堂之上,自是当以故事、律法为主,以官品定席次。但此番宴席,并非朝堂,韩冈不敢居于长辈之上。”

韩冈摆出了尊师重道的态度,富弼自然要成全。韩冈的位置被程颢占了,富家又调整了一下,让本来就被请到前面的程珦与程颢坐在同一席上,韩冈则与吕大临联席。

经过这一回,韩冈对师长的尊重,也是变得更加有名。程门立雪的故事,曾经有人怀疑,但现在却是不会了。

富弼的寿宴并不奢侈,前前后后也就十几道菜,但上门来贺寿的人们则并非为了吃喝。上前被富弼敬了几杯寿酒,闹了一番之后,也就都散了去。

韩冈则是与吕大临回他的住处,方才在席上也说好了,要先看一看张载的行状。

不大的房间中,韩冈拿着吕大临写好的初稿,直接就读了起来。经过这么多年的锻炼,就算没有标点符号,一篇文章放在眼前,韩冈也能畅顺的通读。

吕大临的文章虽然缺乏苏轼华丽的文采,也没有王安石古朴中的韵味,但作为跟随张载数十年的学生,文字中自不缺真情实感。

韩冈一开始,本是边看边微笑点头,只是看到一半,却放了下来,板起的脸上没有一丝表情:“与叔……你写得还真好啊。”

