\section{第24章 缭垣斜压紫云低(四)}

韩冈和薛向在州桥夜市上公然对饮,只用了一天便哄传京中。

毕竟州桥连接御街和朱雀门,人来人往,每日里行人车马成千上万,乃是京城中最热闹的去处。那一夜,亲眼看到两人对坐饮酒的,怕不都有近千人了。

州桥夜市,名满京城,甚至可以说是闻名天下。过去也不是没有宰执一级的重臣来尝鲜。但人家都是派了家人来买,要么就是换了身衣服,不敢暴露自己的身份。身着官服在市井中公然吃喝,而且还是在距离祭天大典只有寥寥数日的时候,绳纠百官的御史台当然不能视而不见,上纲上线也是必然。

不过赵顼在看到弹章之后,更多还是在猜测薛向和韩冈之间的交情到底是哪里来的。在赵顼的记忆里,两人过去并没有共事的经历,也没有共同的爱好,或是姻亲的联系,不比韩冈和章敦、苏颂之间的关系。

但赵顼总觉得心里不痛快,做了皇帝这么多年,他是越来越憎恨撞上无法掌握或是一无所知的事情,总是想着要查个水落石出。

这件事,从前两天自皇城司那里收到消息,到明天就要开始斋戒了,几天下来,赵顼却一直都没有想明白,而皇城司也没有个给出一个让人满意的回答。

当今天御史台的弹章上来,让赵顼又多了一重苦恼——

朝廷并不会禁止臣子们的来往,只是对宰执以上官之间的往来会有所约束。而且很多时候,这种约束也只是空谈,说说而已。

绝大多数重臣们之间或多或少都有着千丝万缕的联系,或是姻亲,或是血亲,从无例外。就算是寒素出身,只要有着出色表现,也很快就能得到高官们的青睐。最典型的例子就是韩冈。商家出身的冯京也可以算一个。早早的就做了宰相家的女婿了。而重臣们之间互相联姻的例子则更多。

当年晏殊与富弼翁婿同列,能不让他们走亲戚吗?文彦博和吴充,吴充和王安石都是亲家,能不让他们书信往来吗?说起来韩冈跟文彦博、吴充乃至他赵顼都能七拐八绕的攀上亲,能将韩冈踢出去吗?

看着奏章上为了一顿夜市上的酒水而慷慨激昂的文字,赵顼就觉得头疼的厉害,脑袋蒙蒙的,发烫发胀的疼。

很有几分不痛快的将奏章丢到与桌上,赵顼却无法将整件事也一并丢到桌上,不再去考虑。

这件事虽然不大,但肯定是要给予惩罚,只是到底要给两人什么样的处分?却是赵顼不得不先行考虑清楚的。

南郊祭天在即,现在揪住韩冈和薛向的错处给个处分,过两天颁德音大赦天下,这两位到底是赦还是不赦?

赦——朝令夕改,朝廷丢脸。不赦——则于理不合,又不是犯了论死的重罪,赃罪都能赦免,小小的‘混迹市井,无人臣体’的罪名却不赦免,如何说得过去,难道要在赦诏上强调,祭天之前某几天犯的罪过不能赦免?

唯一合乎人情义理的办法,还是找个借口拖上几天,等到郊祀大典过后,再罚个俸了事。

但这只是明面上的处罚,暗地里,赵顼已经在考虑是不是在人事上也给与一定的处罚。

韩冈不能轻动,面子和儿子之间,是不需要考虑选择哪一项的。而薛向就不一样了,是不是看情况将薛向清出去,赵顼想着。

——如果能找到合适的人选顶替他的话。

枢密院中,薛向负责的仍是他最为擅长的财计,也就是军费的支出和收入。朝廷每年的开支有一多半用在百万大军上,在薛向上任之后,虽然军费并没有缩减,但使用的效率有显而易见的提高,许多莫名其妙就消失在账簿中的资金,至少能让赵顼知道到底花到了什么地方——尽管不是全部——这些事,不是靠御史监察就能做到的。

朝堂百官中能在财计这个方面比得上薛向的人才,不是没有,赵顼随随便便也能数出十七八个,三司里面有一堆够格的人才。

但性格为人还要敢作敢为,不能与贪渎的臣子沆瀣一气,也不能得过且过不敢出手革除旧弊,这么一来,立刻连十分之一就不到了。精通财计这个能力,可就是代表能在金钱上上下其手的手段比寻常朝臣要多得多,很少有人能忍得住这个诱惑。要不然在钱粮上上心的臣子也不会被‘君子’们所鄙视,谓其为小人。

另外还有一点更关键,地位也要够得上,能身入枢府镇齤压群小。没有足够的身份,就算性格能力都合乎要求,依然排不上用场。薛向之外,赵顼一时间却找不到第三个了——第二个是韩冈,这个人选赵顼无论如何都不会选。

‘暂且留中吧。’

赵顼在心中对自己叹着,将奏章丢到一边垒起的公文上。

拿起了下一份奏报,赵顼却又停住了动作。过了好半天,他才清醒过来,瞥瞥前一份奏章,想了想,却又探手拿起来……然后直接塞到了最底下。

眼不见,心不烦。

对于这等掩耳盗铃、自欺欺人之举,宋用臣眼观鼻鼻观心,木然肃立在赵顼凶狠侧,他什么也没看到,什么也没听到,更不会妄作猜测。这是宫中尽人皆知的自保之法,朝堂上的事,连边都不能占一下。

宋用臣能保持这样的标准,但其他人却不可能人人做到,天子将弹章留中的消息,全然没有耽搁,没过半日便传到了皇城中的两府百司之中。在这其中,自然不会少了韩冈的太常寺。

“多一事不如少一事,郊祀之前,朝廷就算有什么想法,都不会在这时候干扰到南郊的顺利进行。”韩冈闲适自在的与苏颂对饮热茶时如是说。

苏颂回之一笑,不赞同,也不否认。想必有不少人的想法都跟他一样的,但苏颂还是很稳重的没有做任何表态。或许这一回韩冈当真转到了关键点上,或许这件事就这么过去了。但秋后算账也不是不可能,一切都要看天子的心思来定了。

天朗气清的冬夜,州桥夜市便如往日一般的人满为患。而王家杂食铺子的生意,则更要比平时火爆上好几倍,连薛枢密和小韩学士这样的重臣都不顾御史弹劾,上门大快朵颐,听说了这个消息的东京城的百姓们,也不介意花些小钱,来尝一尝这种让两位重臣都忘了朝廷律法的旋炙猪皮肉。

韩冈放衙之后,又一次从州桥上过。王家杂食铺子依然在路边,不过韩冈没有再下马入店的想法。只是远远地看了一眼,只看见铺子中的店主和两个小二忙得团团转,外面竟然还有一群人在等着空出桌子来。真是热闹得让人想象不到。韩冈本想找个元随去排个队,然后给家里带上几份来——在家里吃,就没人能管得了了——但看到这般模样,也就只能将想法收起,先放在一边。

虽说打算将整件事抛到脑后,可回到家中,在换衣的时候,却听到王旖问起今日御史台的弹章。韩冈不得不为京城官宦人家内眷的情报网感到咋舌不已,才几个时辰功夫,就将连很多朝臣都不知内情的情报,传到了王旖的耳中。

对于妻子的疑问,韩冈付之一笑:“郊祀之前,不论有什么事,官家都会担待起来。还是多想想冬至怎么过吧。郊祀回宫后也就是午时的样子,到时候一场宫宴之后就没事了,时间拖也拖不到晚上。不从现在就开始准备,到时候别连州桥夜市上的食铺子都比不上。”

“官人以为奴家主持中馈过了几年冬至了,难道还要官人来提醒?”王旖轻哼了一声,拿着一领丝绵袍服侍韩冈穿上,脸上浮起一丝忧色,“爹爹到底什么时候能抵京?算时间也就该在这几天了。”

在韩冈担任了资善堂侍讲之后,王旖已经完全不担心韩冈还会在朝堂上受到什么样的处罚,只在想着自江宁北上的老父。就算不是在烈日炎炎的盛夏时节,但上京之路迢迢千里,路上染上疾病的可能性还是有不少,毕竟不是当年正当盛年的时候。许多时候,一点从窗户上透进来的冷风,就能让一名跟王安石年岁差不多的老者风邪侵体。

韩冈想了想:“说不定要等到冬至之后。”

“南郊之后?”王旖偏头想了一想,隐隐抓到了一点头绪,“大概是不想参与南郊大典吧?”

韩冈点点头。京城人重视冬至,甚至跟元旦年节之时也差不多。换新衣,喝热酒,祭拜先祖,一切都不下于年节。王安石也不可能免俗,但以他身上的官衔,这时候入京城,肯定要在南郊大典上站着。虽然很想早一步看到父亲,但王旖还是知道孰轻孰重。

他又笑道:“而且排班轮次也不好办。总不能让岳父和王禹玉并肩同列吧。如果站在王珪之前,难道还能让岳父来顶替王禹玉这名当朝宰相?”

王安石身上还有同中书门下平章事和侍中的两个虚衔。虽说是虚衔,但也能算是宰相,只是并非实职,只在俸禄和朝会排班次时管用。而宰相,在祭典之上,要参与主持的地方还是很不少的——不仅是王安石,文彦博、富弼都有几乎跟他差不多的虚衔穿戴在身上——可偏偏南郊等仪式之时,就能派上用场。

若是寻常老臣倒也罢了,但以王安石过往的成就,绝对是与普通宰辅不一样的,他到底是站在王珪之上还是之下,恐怕能让赵顼脑袋疼得变成两半。

幸而以王安石这些年在信件上表现出来的性格上的转变,多半不会去争这个口气。

“反正只是一场祭天的大典而已,不是吗?”韩冈笑道。

但到了次日,中午的时候,一名家丁气喘吁吁的跑进了太常寺。而在他之前,韩冈就已经知道究竟发生了什么事,脸上也没有了昨夜那般轻松的微笑。

他的岳父在一个时辰前抵京了。

