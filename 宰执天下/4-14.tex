\section{第二章 凡物偏能动世情(一)}

已是三月末,夏天的脚步越来越近。

暮春的微风越发的薰人,少了三月初的花香,却更添了几分暖意。

汴河之上,有人来,有人去。与亲友相见时,抱头痛哭;与家人分离时,洒泪而别。官船停靠的码头上,这一幕幕活剧天天都能看到。已是不足为奇。

韩冈也是来送人的。苏缄要走了,他在京中待了有半个月,两次入觐面圣,可见天子对南方局势的重视。而在苏颂的牵线下,在宫中匆匆一会之后,韩冈也与苏缄又见了两次,一起坐下来喝酒聊天,联络一下感情。

有了这一份酒桌上培养起来的交情,两边的关系也就密切了起来。在苏缄向赵顼要到了一批军器之后,韩冈便送了他一个顺水人情,答应将邕州的单子放在军器监出货的最前面——天子点了头,枢密院也已批复,军器监这边只要将单子上的军器生产出来,就不用再送去库中耽搁时间,只需将几份公文缴上去走流程,就能直接顺着汴河将这批军器派送出去。

韩冈这算是顺水人情,惠而不费。也就是因为已经归属三衙的军器,要转给地方州县在制度上需费更多的手续,而神臂弓这样的神兵利器总是紧缺的缘故,他才有得人情做。但已足以让苏缄感激三分,也给足了苏颂面子。

等到苏缄启程返回广西,苏颂便约了韩冈一起来相送。

三月的春风中,汴水畔拱手相别,当然不会有‘寒蝉凄切’;也不会是‘满天风雨下西楼’。但以苏缄、苏颂的豁达,分别时也免不了要感慨动情,说一句‘此情不可道,此别何时遇?’——两人皆已老迈,时日不多,再会面也许已是遥遥无期。

在苏颂家的子弟送过他们的叔祖之后。苏缄带上京来见世面的孙儿孙女,便一个个上前来拜别苏颂和韩冈。

韩冈虽然年轻,但名声之大,苏颂都难以比拟。面对苏颂,苏缄的两个孙子是恭恭敬敬,而在韩冈的面前,则多了几分崇慕。两个生长在广西的孩子,虽然不及京城子弟的能说会道,但胜在质朴,颇得韩冈好感,也出言勉励了他们几句。虽然两边的年纪相差不远,但外人看来,却是半点也不见违和。

另外还有苏缄的孙女,尚未长到需要避忌外人的年纪,也一起过来细声细气的向苏颂、韩冈道着万福辞行。小女孩儿乖巧知礼,长得也讨喜。看苏颂直接从袖子里掏出一个四时庆喜的小金牌来做饯行礼的样子,就知道他很是疼爱这个侄女儿。

韩冈也带了一份饯行礼来,但都已让人送上了苏缄的官船,现在则是两袖空空。

“这下可丢人了。”韩冈毫不介意的摊了摊手,半开玩笑的说着:“这样吧,小娘子可有什么想要的,金糖、菓子,还是泥人、塑像,我这就派人去买来。”

小女孩儿仰起了头,张着一对黑白分明的大眼睛:“这些七娘都不要。大爹爹连日愁眉不展,七娘只想要大爹爹能笑起来。”

韩冈被一个六七岁的小丫头惊到了,转头看着苏缄,见他脸上也是带着讶异。摇了摇头,顿时哈哈大笑起来,如此乖巧聪明的小女孩的确少见,不论是不是有人教的,能流利的说出来,已经很难得了。

“这礼要送倒是不难,皇城勿须再担心。给邕州的军器,今天早上就已经装船发出了去,要不然韩冈也没脸来相送。船走汴河入扬子江,从湘水再转灵渠下去,说不定会比皇城还要早一步到邕州。”

苏缄一听,顿时喜上眉梢。这一件事,半个月来一直存在心上。韩冈虽然信誓旦旦,可不看到实物,如何能放得下心来?

“七娘多谢舍人。”小丫头装着大人的模样,冲韩冈福了一福,等抬起头,却又不好意思的躲到了苏缄的身后去了。

“难得的孝顺孩儿啊。”韩冈对着苏缄夸着,“我家的女儿再过上两三年,能有七娘一半乖巧,我也能放心了。”

苏家的这个女孩儿的确很不错,韩冈看着也喜欢。要不是自家的儿子才三岁,说不定就要跟苏缄定下亲事了。

摸了摸孙女儿的头,让乳母带她先上了船。苏缄来到韩冈身前,正容行礼:“多谢玉昆。”

“不敢当啊!只是为国,何敢劳皇城谢。”韩冈还了一礼之后,不由得一叹,“不过其中神臂弓也只有五百架,几场大战下来,差不多就要报废光了。”

重弩保养不易。其力道往往都在三石以上,几百斤的力道就藏在弩身中,当然很难保证使用寿命。尤其是在战场上,集中在短短的时间里连续射几十箭下来,总会有一批重弩会报废。而不像战弓,其使用寿命要远远胜出。

不过神臂弓有个好处,就是筋角之物用得少。‘以檿为身,檀为弰,铁为登子枪头,铜为马面牙发,麻绳扎丝为弦。’弩身是山桑木,弩臂是檀木,遇水也不会对弩弓损伤太厉害。哪像普通的弓弩,到了湿润的南方,其中用着牛筋牛角的部分,很快就会因为吸水而失去弹性。

“能有五百架神臂弓就不错了,原本城里还有一百架。有六百神臂弓守城,十天半个月,邕州城决不至于有失,到时候桂州也就能派兵来支援了。”

韩冈脸色有着一分沉重,苏缄的口气似乎就是在确定战事已不可避免:“交趾人当真敢于来犯?”

并不熟悉历史的韩冈,自然也不清楚交趾人到底有没有在此时犯界。但他能确信,广西广东没有在北宋丢给交趾过,至少在他的记忆里并没有。要么就是这一仗干脆没有打,要么就是打了,但只是很小的战争而已。

“这一事也只能是未雨绸缪,谁也不能说交趾一定会出兵。但刘经略禁汉人与交趾互市,这等于是将边境的侬人部族全都推到了交趾一方。有了侬人部族的支持,就是多了两三万兵力。说不准什么时候,交趾就会动手了。”苏缄浑浊双眼眯了起来,叹着气道:“前几天不也跟玉昆你说了吗?广西军中皆已糜烂,实际兵员不及军籍簿上的三分之一。邕州以南,也就几个寨子还能抵挡一下,其余州县哪里还有兵来守?”

大宋南方的军队基本上可以当成是笑话,这一点是天下人的共识。要不然当年侬智高叛乱,也不会让狄青领着西军万里迢迢的赶赴昆仑关。而苏缄当时在广东征发当地兵员,就是在侬智高的蛮兵手上吃了一个大亏。

不过区区一个南方小国,若当真敢于侵犯大宋疆界,却也是自寻死路。如今不是太宗的时候,因为北方战乱未休,所以放了交趾一马。现在交趾若敢将动手的借口送来,天子肯定是要笑纳的,韩冈也百分百的支持:“交趾本是汉唐旧郡,如今却成为外藩。若交趾当真敢于凌犯中国,那就是大宋恢复前朝旧疆的时候了!”

韩冈少年锐气,苏缄、苏颂听着倒是一点也不奇怪,便是相视一笑,同声道:“若交人胆敢逆天犯顺,自当出兵重惩之!”

船上的船老大这时过来催促,“皇城,时候差不多了,再迟就来不及赶到雍丘了。”

行船多忌讳,尤其忌讳行不依时。

苏颂也是时常泛舟于江湖之上,自然知道这个规矩。轻声一叹,对苏颂、韩冈拱手相辞。他在岭南多年,在京中除了苏颂,更无亲友。这一趟上京,能多一个韩冈,却是难得至极。韩冈虽无赋诗以表离情,却还是跟苏颂一起,照习俗在河边折下一枝柳枝,赠给了苏缄。

接过柳枝,别过苏颂和韩冈,苏缄走上跳板,登船起航,并不回顾。一艘六七百料的官船,就随着水流,渐渐南去。

身在宦海,人送己、己送人都是常事,目送着苏缄的座船远去,韩冈心中的感慨很快也收了起来。不过他并没有立刻上马回京,而是和苏颂一起在河边慢慢的走着。

侧过脸,望着汴河中的潺潺流水,苏颂道:“新改制的水轮机,我心中也有了规划,图样也画出来了,过几日就去军器监里,看着如何与锻锤配合起来。”

“多谢学士!”韩冈低头谢过。

苏颂这是帮了大忙,换做是普通的士大夫,谁会愿意去做工匠的活计?韩冈可是听说了,这段时间,有人背地里在讽刺苏颂是贪了他韩玉昆在军器监贴出来的悬赏。此等言辞,韩冈嗤之以鼻,可不管怎么说,也足够恶心人了,相信也传到了苏颂的耳中。但苏颂他却没有半点动摇。

“不过我过两天就要去应天府上任了。若是不能成事,也只能让玉昆你再另想办法了。”

“能得学士相助,韩冈已是喜出望外,哪敢再得寸进尺?”韩冈笑道:“何况得了学士指点,此一事定然能顺利见功。”

