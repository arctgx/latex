\section{第40章 败敌逐远山林深(中)}

胜败已定,举着捷报的信使正向着京城狂奔而去。换马不进铺的急脚递,只需十日上下,就能将捷报传于京城。

而王中正也从熙州狄道城狂奔而来,在珂诺堡与奉命而行的韩冈汇合,一起抵达了王韶所在的河州大营。

见到韩冈,王韶和高遵裕都点头赞许的笑了笑,香子城在苗授赶到之前能守住,靠的是韩冈之前派去的王舜臣,之后解决袭击后路的五千敌军,也是韩冈亲自指挥。虽然因为翻山越岭的腿脚比不上吐蕃人,但珂诺堡下斩首千余,香子城下还点出了四百斩首,这份功劳基本上都是韩冈的,苗授只能靠边站——吓跑了敌军,怎么都不能算功劳。

被王韶、高遵裕一齐迎进营中,王中正并不休息,而是先去烧成一片白地的河州城绕了一通。大战之时,他身在安全的狄道城中,只是等到战事结束,不到河州城中走一圈,享受起王韶送给他的功劳来,未免有些难以心安。

“木征逃了,禹臧花麻也退了。一场血战,河州城也攻下来了。平戎一策到今日终于告一段落,”王中正想着王韶拱手致意,“恭喜王经略,多年夙愿得偿!”

“此是天子看顾,众将齐心,吾等叨天之幸,才有今日的成功。”

王中正和王韶不停的交换着恭维和谦让,携手走进主帐。

韩冈跟在他们后面,王中正说的话,他却有些不以为然。要说河湟之事告一段落,未免还早了一点

毕竟木征还在。曾经归顺在他帐下的诸多河州蕃部,也没有伤了元气。对于河湟战略来说,还远远没有到庆功的时候。

木征以一州之地,妄图抗衡拥有亿万人口,百万大军的大宋,的确是螳臂当车的自不量力之举。但距离化作的天堑却也是实实在在的问题。

以这个时代的道路条件,兵力投送是有其局限性的,而后勤上的供给,更是要让每一位将帅都不得不为之叹息的问题。凭借大宋的实力,眼下的确是支撑起了河州战事。可这是因为此时世界上最为富庶的中央王朝能承受得起大量的路途损耗,并不是说这个距离对汉人的辎重队没有影响。如果路程再远一点,再崎岖一点,那事情可就麻烦了。

木征逃窜的方向是露骨山,那个地方比起河州还要遥远,基本上可以算是青藏高原的边缘地带了。王韶的想法,这两天已经在给韩冈的命令中说得很明白了。韩冈对此是可是忧心不已,想来王中正恐怕也是一样,很快就笑不出来了。

正如韩冈所料,王中正进帐没多久,的确一下就把笑容给收了,“什么,要出兵露骨山?!”

“不获木征,河湟终究还是定不下来。”王韶音声平和的对王中正解释着,“若是哪天给木征瞅准了机会,说不定河州城又会被他给夺回去。”

只要站在大帐外,向南张望,就能看见那片顶端带着白帽的山峦。靠着沙盘的帮助,王中正对河州地理有所了解,怎么都不会认为翻越露骨山是个好主意,“木征的确逃过了露骨山去,但也用不着也跟着追过去吧!万一出个差池,好不容易的大捷可就要大打折扣了!”

“但木征值得我们冒这个风险。”

“不如改从岷州走。”韩冈的发言,引来了王韶、王中正他们的倾听,“露骨山对面,木征逃去的那个地方,在过去可是叫做洮州!是熙河经略司辖下五州——熙、河、巩、岷、洮中的洮州。是洮水的上游所在。有洮水在,要去洮州,何必走那么危险的露骨山,直接沿着洮水上溯不行吗?”

“缓不济急!返回熙州、岷州绕行,至少要多上一个月的时间。”王韶对此早有考量,“有这个时间,木征早已经将洮州的蕃部整合完毕,足以调兵封死岷州通洮州的几处关口。”

“可一旦翻越露骨山,粮草转运等事可就难以为继了。除了自带上一部分粮草,其余必须就地自筹。”韩冈对自己的后勤组织能力很有自信,可是遇上一座终年积雪的高山,他的自信心也不会变成自大,还是能够认清自己的能力极限所在。

王韶心中对此早有腹案:“出兵的数量不会超过三千人。战马也都将用来驮送粮秣,不会多用于骑乘。只要能降伏洮州蕃部,人马食用,不成问题。”

言下之意就是要学着契丹、党项和吐蕃这些蕃人,直接开抢了。韩冈对此没有什么心结,只是要考虑的困难还有很多。

“春天雪化的时候,山中土石中的冰层融化,很容易山崩滑坡。”韩冈盯着王韶。这个道理,在西北的山区只要多待几年,不可能不知道。

“玉昆,木征不除,河湟之事就没一个了局。依你之智,应当能看到才是。”王韶主意已定,即便是韩冈,也动摇不了他的决心。

“下官不是要阻止此事,只是不在战前多做预备,经略你要领军穿越崇山峻岭,这个风险未免太大了一点。”

“瞻前顾后,不知会延误多少良机。”王韶渐渐的有了点火气,“玉昆,你什么多好,就凡事想得太多。须知世事总有难以逆料的时候,你承袭子厚之教,尊奉思孟之学,难道没学到‘虽千万人吾往矣’的胆魄吗?”

‘这句话可不是放在用兵上的。’韩冈悻悻然的腹诽着。

而王中正去在旁边听着听着脸色越变越是震惊,“怎么说着说着就变成了经略来领军了?!”

韩冈、王韶和高遵裕都同时瞥了王中正一眼,当然是王韶亲自领军,怎么可能换他人领队去洮州!

不是王韶要争功,他是主帅,一路经略,下面将领的功劳也就是他的功劳。现在是要追击木征,光靠宋军并不一定能逮到他,必须要在同时征服洮州的蕃部——而换作是其他人领军去洮州,根本不够资格跟木征争夺洮州蕃部的归属。

只有王韶可以!

在河湟,韩冈的名气很大,知道他的身份的蕃人,见面时都是点头哈腰。但王韶的名气更大,他是整个河湟战略的倡议者和主持者,几年来连战连捷,诸多蕃部一一臣服于他的威名之下。到了王韶的面前,以包顺、包约为首的蕃部首领们,连大气都不敢喘。

只有王韶能够凭借他的威望,从木征手上抢来洮州蕃部。

韩冈冷哼一声,要是换作景思立、姚兕或是苗授领军,他操这份心做什么?

但王韶不同。即便不论私交。王韶本人乃是一路经略,而且是一手打造起熙河路的灵魂人物。若是出了什么意外,这样的损失,熙河路、乃至整个河湟开边的战略都承受不起。

所以韩冈会大力反对,这实在太过冒险。

王中正从王韶的脸上看到的是认真二字,要亲自领军翻越露骨山的确不是自己的误会。心惊之下,他更要全力阻止王韶去做傻事。

但王韶抢先一步了王中正的劝阻,下了决断。

“好了,此事不必再议!”长身而起,他朗声说道:“我将在军中点集三千人马,去追击木征。翻过露骨山,平复洮州!……公绰,你呢?”

王韶问着高遵裕。

“既然已经把河州城都夺了,将木征赶得如丧家犬一般,如何不追到最后!”高遵裕放声大笑,豪气干云,“自当与子纯你同进退!”

“好!”王韶和高遵裕之间其实早就已经商议妥当,不过是在王中正面前演一通而已。他说着两人商议好的决定,“授之,我领兵南行后,就由你来率领本部镇守河州。”

苗授应诺,王韶又对韩冈道:“玉昆,你回熙州去,经略司中的一应事务就由你来处置。若有要事,则请王都知过目。”

韩冈叹了口气,这等孤注一掷的做法,他宁可该从岷州沿河上溯。虽然有可能将战事拖上个一年半载,但终究要稳妥上不少。但王韶这个样子是劝不住的,只能躬身接令,“下官必尽心尽力。”

他虽是巩州通判,但也是经略司机宜文字,在军政两方面都有权力。以韩冈在河湟之地的威望,使唤包顺、包约都不成问题。王韶此言,等于是将经略司的工作交托给他照管,只要将王中正糊弄过去就行了。

王韶一一将事务分派下去,将景思立安排去替换姚麟驻守熙州北方;让姚兕姚麟两兄弟,清理河州诸蕃。聚集在河州的两万大军转眼就要星散。

韩冈看着王韶将事情一一安排妥当,最后向他提议道:“经略,最好能把智缘给带上。”

“……是对洮州蕃部还是对木征?”

“木征日暮途穷,待到天兵紧随其后,翻过露骨山,想必他也不会再有多少悖逆朝廷的心思了。”韩冈没有继续说下去。想来王韶也能明白,这时候正是智缘的那张嘴发挥功用的时候了。

王韶稍作考虑,点头道:“也罢,多一个不妨事,就带上他好了。”

