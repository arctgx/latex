\section{第二章 牲牢郊祀可有穷(上)}

还有两天就要过年了,上京临潢府中,却是城中连树都是白的,看不到一丝红绿。

虽然是个昏君,而且死因成了属国中的笑柄,但耶律洪基死后应该享受的礼仪,却是半点也没有俭省。

虽说契丹人有属于本族特有的丧仪,但汉人的礼节在辽国一样通行。天子服丧二十七日,心丧三年,都是少不了。如今已经是过了期限,可百日之内,辽国国中不得有吉礼喜乐,更不得游宴射猎,同样要执行。尽管私下里没人去,弄得今年的年节一点喜气都没有。

宫城内外禁卫森严。

辽国天子依四季游猎四方,作为行宫的捺钵都设在城外。一年之中,进入城池的时间,许多时候加起来也只有区区半月。

不过在登基、册封等大典时,天子还是要进入京城中,在宫殿内举行典礼。自然,天子的丧礼肯定不能例外。

“宋人的正旦使到了。”

“来看热闹吗?”

“想看我们的笑话呢。”

萧十三和萧得里特都在已经改成灵堂的正殿之中,站在一边,对应有的礼节毫不在意。人都死了,还有什么好怕的。

殿内烟雾弥漫,环绕着一具巨大的棺木。

在棺木内,曾经统御万邦的大辽天子,如今只是一滩支离破碎的烂肉。过去一个眼神的变化,就能让萧十三和萧得里特立刻战战兢兢魂不附体起来的皇帝,除了最后留下的遗容,再也吓不了任何人了。

耶律洪基从数十丈的高空坠落,骨头碎成了一片片,一只手和脚不知怎么回事,也与身子脱离了关系。而头颅像是一颗被砸烂的西瓜,留在惨事现场的痕迹至今也没有收拾干净。那样的尸体连衣服都穿不起来,而头颅更是只能用一个木雕来代替,以一层层绸缎保住。

双脚离地,本来就是把性命一并悬空。上得山多了,会遇到大虫;下得河多了,会遇到蛟龙;上得天多了,摔死也不足为奇。

当真是龙驭宾天了。

礼官和内侍关注着长明灯和火盆,不使烟火断绝。而耶律乙辛的两名得力助手则在殿门口窃窃私语:“胡都堇已经去了析津府,有达鲁古盯着,南京道可以不用担心。”

五京道中,钱粮来源的南京道,耶律乙辛控制得是最稳的。上京道也是同样很稳固。中京道就不好说了,而兵力强盛的西京道、东京道更是一地虎狼。

萧得里特的脸映在殿中跳动的火光下,鹰钩鼻在脸上的投影摇摇晃晃,显得十分的阴森,语气也是阴森森的:“监母、女姑两个斡鲁朵的人今天终于是到了,虽然。就只剩窝笃一个还没消息。如果再不来,可就必须下手了。”

“十一宫只有一家有反心,不足为虑。尤其是窝笃,在五国之乱是可是伤了元气,到现在还没恢复。”

“十二宫。文忠王府不算,十二宫中,有反心的就只有一家。”萧得里特更正道。在枝节上做文章,显得心情很是放松。如果是半个月前,他可是一天到晚的紧绷着脸。

辽国每位天子即位,都会设立自己的宫帐,称为斡鲁朵。有属于斡鲁朵的土地、户丁。裂州县,割户丁,以强干弱枝,诒谋嗣续,世建宫卫,入则居守,出则扈从,葬则因以守陵。

这是掌握在天子手边的最后一份力量,也是最可信的力量。

除了天子之外,太祖皇后述律平和承天皇太后萧绰也都设立了自己的斡鲁朵,分别称为蒲速和姑稳,汉名长宁宫、崇德宫。此外圣宗皇帝的弟弟耶律隆庆、文忠王韩德让也被特许建立,不过耶律隆庆是宫,而韩德让仅仅是府而已——并不算在斡鲁朵内。

当然,新帝耶律延禧也开始组建自己的宫帐宿卫,起名做阿鲁斡鲁朵——汉名永昌宫——意为辅佑,这是第十二宫。

十二宫,十二个斡鲁朵。除去新帝耶律延禧仅在纸面上留下名字的阿鲁斡鲁朵,其余十一宫的常备军加起来大约不到十万骑,但必要的时候,全力动员起来的大军能超过二十万。谁控制了宫帐,谁就能控制大辽三成以上的军力,而且是精锐。

在十一斡鲁朵中,已经明确向新帝表示顺服的,有八个,此外还有二个暂时也没有反叛的意思。唯有位于辽阳府附近的兴宗皇帝的窝笃斡鲁朵至今没有消息。

“如果真要讨伐叛贼的话,就不知道是由谁来领军了。”萧十三问着。

“不管是谁领军,必须速战速决。以雷霆之势,将叛军剿灭,这样才能一举稳住局势,省得刚刚归顺的又反了过去,也能不让宋人捡到便宜。”萧得里特冷笑,“如果国中真的内乱起来,南朝的岁币恐怕就会不见踪影了。”

“南朝现在一心想着灭夏,西夏国内现在的情况你也知道了,再看到大辽国内的情况。南朝皇帝肯定不会放过这么好的机会。”

一提起西夏,萧得里特就是满肚子的火气:“秉常就是一个蠢货!还没控制兵权,就跟梁氏争锋。现在被囚禁,闹得西夏国内人人离心。原本还能支撑一下的局面,全都给毁了。就看宋人什么时候攻过去!”

“梁氏囚禁秉常的事,太傅是怎么说的?”萧十三问道。

“太傅也是发了大脾气。”萧得里特摇头叹道,“消息送来的时候,差点拿了剑把信使给砍了。”

“这么大的脾气?”萧十三惊讶了一声,“……我倒是都没看出来!”

萧十三回溯前些日子见到耶律乙辛时的记忆,没发现有什么不正常的地方。待人处事甚至跟耶律洪基没有驾崩时都没有什么改变,没有因为彻底掌控朝局,而变得张狂放肆。甚至隐隐的,更为谨守臣子礼节。

“那是太傅的养气功夫好啊,转眼火气就收起来了。要不是当时我就在帐外,也不会知道太傅发过那么大的火。”

“那现在该怎么办?放着秉常不理?”萧十三问道,“好歹他也是尚了公主的驸马,总不能不闻不问。”

“秉常关着就关着吧,这个时候说话梁氏也不会听。”萧得里特叹了口气,辽国国势衰弱,本身又有内战的可能,梁氏不可能会老实听话。

而且秉常做事太蠢了,放了他出来,梁氏灭族都有可能,那还不如投降宋人,秉常要圈禁一辈子,但梁家说不定还能换个富贵终老,子孙万代。

“一年三万匹马驼啊,为了让国中同意支援西夏,秉常年年入贡。现在梁氏上台,可是不会老老实实的送了。”萧十三磨着牙。西夏的贡品是三万马驼,这是给皇帝,朝中重臣们也有礼物可收。作为耶律乙辛的亲信,他拿到手上的那可不是笔小数目。

“看看宋人会怎么做吧。只要宋军打过去,就可以让梁氏兄妹好好想想了,是成为宋人的阶下囚好,还是给大辽送马送骆驼的好。不过是些身外之物,倒不用担心将他们逼得投降宋人。”萧得里特又长叹一声,“不管怎么说,等宋人攻过去,还是得出兵援助。”

“哪里出兵?”萧十三皱眉问道:“耶律燕哥要镇着西京,他手下的皮室军不能动,上京道的兵力多半在临潢府这边,要镇着东京道,不能动。西北招讨司的三万人是用来压制阻卜人的,同样动不了。哪里有兵去支援党项人?”

“援兵有的是,就看梁氏舍不舍得花钱了。”

“哦。”萧十三恍然,他也听说耶律乙辛到底打算怎么做了,笑道:“就不知道想要买一个大捷,不知要花西夏多少钱。”

“不指望能赢,能多消耗宋人一分兵力,日后大辽与其对垒,也能顺心一点。”

萧十三和萧得里特同时摇摇头,他们都不看好西夏,整个大辽的朝堂上也都不看好西夏。

轮值的官员掀开帘子进来了,外面忏经的僧侣声音传了进来。轮值的礼官给萧得里特和萧十三行礼后,给长明灯里添油,又去了角落里站着了。

等殿中安静了一点后,萧十三问道:“奚王府那里情况怎么样?”

“谢家奴暂时还没有消息。”萧得里特说道,“那个老狐狸称病不至,总不能将他绑过来。”

萧十三愤怒起来:“他是打算最后谁赢就听谁的?还是说他打算等到我和双方打到精疲力竭,出来做个得利的渔翁?”

“应该选择后一条路吧。”萧得里特叹道。

有点野心的人,都会选择后面的一条路。而谁赢就听谁的话,是最蠢的做法。这样的庸人有,但指望谢家奴也是庸人,那就是在拿自己的性命在冒险。知奚六部大王事萧谢家奴,可是有名的精明厉害。

萧十三是暴烈的脾气,但以现在的局面下,再是浑人也清楚这时候决不能将中间派往对手那里推,“怎么办,就眼睁睁看着他做渔翁不成?”

“太傅说不用担心。”萧得里特道。

萧十三立刻反问:“怎么不用担心!?”

“的确不用担心。”从殿门处有人突然插话。

萧十三猛然回头,看到了来人,眼瞳顿时一缩:“张孝杰。”

“是耶律孝杰。”汉人打扮的张孝杰笑道。

