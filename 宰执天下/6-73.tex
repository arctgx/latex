\section{第八章 朔吹号寒欲争锋(13)}

不过李达随即又轻声叹了口气。

这又如何?

换作是大理寺的狱吏,的确决不会将刑恕的尸骸弄到这般破绽处处,便是衙门里的老斫轮,也应该一眼就能看出其中的问题。只怕是眼前的这位章辟光章府判,害怕人多嘴杂,泄露真相,没有安排一名老手来布置,只敢驱用亲信。殊不知这样做,反而是欲盖弥彰。

只是如今最炽手可热的韩参政,可是亲口认定了这具尸体是源自于自缢!

‘君子学道则爱人,小人学道则缢死’,才一天的功夫,就从朝堂传到了京城中。

可见是多么迫不及待。

李达好不容易才做到了大理寺少卿,又不是吃饱了撑着,费什么力气去证明他是被人先弄晕,然后才挂在房梁上的?

而且事涉大逆,作为逆贼同党的刑恕,死得也不冤。

无论新旧两党,现在都是有志一同,尽快将这一桩牵连太多的案子给压下去。

刑恕之死虽是蹊跷,但新党也不敢闹起来。蔡确不知与多少人有关联,此外还有曾布、薛向,若这边从刑恕身上开了头,之后就就别想结尾了。

真要是将真相捅出去,开罪的不只是一个韩参政。

作为法官,李达知道自己的职责是查明案件真相,将罪犯绳之于法,让受冤者得到昭雪。但身为朝臣,李达更明白,到了他这个等级,政治因素却已经是许多案子的唯一考量。

转了两圈,李达就结束了自己的检验工作,对章辟光道,“果然是自缢。”

章辟光点头叹道,“刑恕此贼行大逆不道之事,自绝于二圣与朝廷,本当明正典刑,千刀万剐以抵其罪,如今一根绳子吊死了自己,倒是太便宜他了。”

李达道:“说的是啊,的确是太便宜他了。”停了一下,又问,“……当时的狱卒呢?”

跟在后面的典狱立刻道:“就在外面关着。他实在是太不小心了。”

“也没必要太苛刻。犯了大逆之罪,这些贼子哪一个不是惶惶不可终日?畏惧朝廷天威,选择自尽也是常有的事。”

章辟光也道:“人要想死,实是防不胜防,真要咬了舌头,撞了墙,怎么救?”

典狱连点头:“下官这就让人将他放出来。”

这间牢房就不必李达再多费唇舌,再细加检验,开具的依然还是自缢的结论。

从牢房中出来时,李达瞥眼看见了外面的一群狱吏中个头最高的一个,五大三粗,手上裹着细麻布,“手怎么了?”

狱吏没提防,被李达吓得一个激灵:“禀……禀官人,是……小人是之前修家里屋顶给界刀伤了。”

李达笑得和蔼可亲:“早些去搽点药,狱中阴冷还好,若是热了起来,伤处容易烂掉。”

狱吏愣愣的看着李达,一幅没听明白的样子。

“明白了吗?”李达笑着问。

“明……明白……”狱吏点点头,又摇摇头。

章辟光脸色微微变了一下,转头喝问典狱:“可是明白了?”

典狱心领神会:“小人明白,小人明白!”

李达点头,转身向外。

早点烂掉,烂光了就没了物证。

不过刑恕死了,短时间内,韩冈就不可能杀蔡京。否则就太过明目张胆,而且也会让沈括、章辟光陷入被动。

可若是拖延时日,保不准会有什么变化。

李达真不知道韩冈到底是怎么想的。难道在他的心里,蔡京的威胁,还比不上刑恕?

当然,这不是李达能够考虑的事,他只要办好自己该办的,然后在韩冈那边留下份人情就好了。

……………………

刑恕被大理寺确认是自缢而亡。

得到这个消息之后,一些人最后的一点不安,也终于放了下来。

几名骑手连夜从新郑门出了京城,然后一路向西狂奔而去。

京内京外稍大一点的动静,现如今都在皇城司的监视下,那几位骑手的离开,也没能瞒过王厚的耳目。

次日一早,宣德门前,韩冈笑着对苏颂道,“西京的那几位终于可以睡个好觉了。”

“玉昆。”苏颂瞥了眼韩冈,“你之前好像也这么说过。”

“前几天睡觉,他们还得学司马十二,用个圆木做枕头,现在可以用个软和点的了。”

苏颂微微一笑,神色变得深沉起来:“司马君实啊……不知道《资治通鉴》什么时候能修好?”

“天知道。”

韩冈摇摇头,以司马光的写作速度,还不知道什么时候能写完这部史家名著。

“《本草纲目》呢?”

“……天知道。”韩冈又摇头,哈哈笑了两声:“太史公修《史记》,用时十三载。班固修《汉书》,二十年未成。本朝司马十二用了十多年也没将《资治通鉴》写好,所以我们也不必着急。”

“薛文惠修《五代史》,用时一年半。”

“可能与《史记》《汉书》放在一个书架上吗?”韩冈笑问道。

开国初年,薛居正受命修《五代史》,只用了不到两年就完成了。这个速度,不仅让后世史家诟病不已,就是同一时代的士人,也多有不满。所以才有了欧阳修的新五代史。

苏颂反问:“历朝历代,又有哪部史书能与《史记》、《汉书》并列?”

“若论文教,本朝不让汉唐。这修史比不上,有伤盛德啊。”

苏颂转过头来,深深的看了韩冈一眼:“玉昆……你想要把谁打发去修史?”

韩冈微微眯起了眼睛:“出外监盐茶酒税,居京中编纂类书,子容兄会怎么选?”

……………………

留在大城市中做官妓,还是去边州嫁给卒伍?

韩冈还记得王韶曾经这样问过自己。

一直以来,韩冈都对官妓制度极为反感,自纳了周南之后,更是绝足欢场,从不参加召官妓过来佐酒的聚会。

当初,他曾经与王韶、王厚议论过将犯人妻女收入教坊这样的处罚,实在是有违圣门大义。

儒门讲究气节,却将女子失节作为处罚。

韩冈当时都说,将她们们远嫁戍卒也行,一辈子都只能打光棍的士兵很多。

当时王韶问了韩冈两个问题:

第一,有人愿意嫁吗?对绝大多数官妓来说,去边疆过一辈子比死都可怕,何况还是嫁给卒伍,王韶让韩冈去教坊问问有几个愿意嫁给赤佬,而且是不知多少岁的赤佬。

第二,万一那个戍卒积功得官该怎么办?

在韩冈来看,前一件事,那是针对已经习惯了浮华的官妓,犯人的亲眷在还没有沦落时,至少其中大部分还不至于愿意将自己的姓名列入贱籍,要后悔,也是嫁过去后才会后悔。

后一件,就是想得太多,难道说一个罪犯的亲眷,还能唆使得动丈夫犯法?若是怕她做了官夫人,朝廷不好安排,直接让士兵娶妻后离开军队屯垦边疆就行了。

接下来王韶怎么说的,韩冈现在已经记不清了。他只记得自己没有说服王韶,而王韶也没能说服自己。另外还有讨论的起因——当时讨论的,其实是一桩本家户绝、只有出嫁女的遗产继承案。

出嫁的女儿,就不算这家的人,只要不是株连姻亲,便不会受到牵累。但未出嫁的在室女就不一样了,一并要受牵连,往往没入教坊。虽说可以不死,但由此沦入贱业,也不比丢掉性命好多少。

不过相对的,在继承权上,在室女就比出嫁女要大得多。若有兄弟,在室女至少能拥有三分之一的继承权,无兄弟就能继承全部家产。另外归宗女——也就是丧夫或是离异后回家的女儿——也拥有与在室女相类似的继承权,但继承权要稍低一等。

而出嫁女,即便是没有其他儿女继承门户的情况,也只能拿到家产的一部分,一般只有三分之一,其余没入官中,而且还有上限,不得超过两千贯“给出嫁诸女并至二千贯止’,除非遗产很多,超过两万贯,这才会请天子决定增加多少:‘若及二万贯以上,临时具数奏裁增给’。

当时韩冈和王韶、王厚,就这么莫名其妙的从出嫁女的遗产继承,扯到了教坊上面。那时候,熙河路还是天边的浮云,陇右最为富庶的巩州还只是一个边境的寨堡,未来的两位宰辅和一名横班,只能屈居在简陋的房间中,门外倒是还站了一名安西都护府都护,和一位功绩显赫的州将。

而之所以突然间会想起来这件事,当然与出嫁女的继承权无关,一方面是因为早上在宣德门外,问了苏颂一个有些类似的二选一的问题,另一方面也是因为现在正在于韩绛、张璪议论的话题。

这个话题并非是如何处置大逆案与案犯官们的家属,在整桩案子还没有结案之前,除了那几个为了安定人心而特旨处置的主犯,所有人犯不可能先于案件之前进行宣判,他们的妻女亲眷当然也不会例外。

而是一个韩冈前世曾经听闻多次的名字,而且总是与当今的大文豪联系在一起的名字。
