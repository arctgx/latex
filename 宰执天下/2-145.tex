\section{第28章 大梁软红骤雨狂(六)}

韩冈从王安石府回来时,李信也回来了,不过他看起来脸色并不好,大概是在三班院中受了点气。

安慰了他两句,韩冈不由得叹了口气,这就是机遇和机缘的差距了。

当初赵隆、王舜臣和李信三人都是几乎同时跟随起王韶,只是后来李信被张守约调了去,三人的道路便分了岔。跟着王韶的赵隆、王舜臣都是靠着军功直接得官,名字直接呈到天子面前,得官前的试射演武只是走过场,三班院也刁难不了他们。

但换作是李信,他是被推举来试射殿廷,通过后才能得官。没有过得硬的军功,在三班院受到刁难也不足为奇。

而当初跟韩冈一起上京的刘仲武,情况跟李信一样。他能够一切顺遂,那是因为他有着向宝的荐书。出自京营,当时而且还兼着的向宝在三班院颇有几分人缘,所以没人跟刘仲武过不去。

三班院和最近新近成立的审官西院,虽然要向枢密院负责,但实际上都是独立,不过韩冈的关系还延伸不到三班院中,真的要找人帮忙,也只能抓瞎。

韩冈很清楚,李信的才能的确出类拔萃,绝不输给西军中那些声名鹊起的年轻将校,但他沉默寡言的性子,让他很难一下子得到他人的看顾,只有日积月累的相处,才能看到李信出色的一面。

不过只要给李信上场演武的机会,一个‘绝伦’的评价肯定是少不了的。虽然韩冈有些担心,但试射殿廷就在眼前,应该不会再生枝节。

倒是韩冈自己这边让他有些烦。从他抵京,到现在才不过半日过去,递了名帖要拜访他的官员已然为数不少,大概是存着通过他跟王安石拉上关系的心思。韩冈望着堆满了桌上的名帖,头疼欲裂。不加理会是不可能的,但全部会面更不现实。可是如果要在其中挑挑拣拣,他也弄不清哪人可见、哪人不可见、哪人可见可不见。

韩冈今夜已经叹了好几次气了,官场上的应酬交接的确很麻烦,尤其是京城,不像秦州那么单纯。他探头望望隔邻,理应热闹非凡的王韶那边,这时候则是安安静静。

韩冈前面已经把王厚托他转交的信件给了王韶,里面的消息想必不是王韶想看到的。韩冈是刻意在明确了自己的站队之后,才让王韶知道他跟自己的姻亲关系已经不复存在。

王韶方才看了信后,虽然没有痛哭流涕,但也免不了伤心动情。平日总是坚定如花岗岩一般的眼神,今夜却是泛起了水光。他摇头叹息着:“想不到出了这等事。公庥也不过四十,竟然一病不起。还有……”

韩冈被王韶看了一眼,见他又是摇头一叹,没再说下去。

公庥是韩冈岳父的字,也是王韶的前任小舅子,与王韶交情匪浅。而今年发生在江州的一场夏季疫症,必然不会仅仅针对韩冈的聘妻和岳父,少说也要夺取上百人的性命才能够资格称为疫。王韶的亲朋好友中怕是还会有一些噩耗,只是没有传消息过来。

不过王韶并没有在悲伤中沉浸多久,很快就从伤感的情绪中拔出来,跟韩冈说起正事。尤其是王韶几次面圣时,天子多次提及韩冈的事,都跟韩冈本人说了。

听着王韶的意思,韩冈这才知道他这次入京应该是能够面圣的。也是天子有心要见他,所以才让他往京中走一遭,否则直接就从秦凤调任了——韩冈并不是京朝官,调职其实并不需要到京中走过场。

韩冈对此是有一些心理准备的,王厚都能见天子,自己被皇帝接见也是理所当然。只不过现在情况不一样了,王韶都在怀疑王安石那边会不会阻止天子招韩冈入觐。

堵塞天子言路是每个权臣都想做的事,而让天子只听自己说话,更是臣子们所梦寐以求。王安石虽然是正人君子,但并不代表他喜欢看到天子面前有人说他的不是、不断的给新法挑刺。

韩冈是支持新法的,还出了几个主意,对新法的推行有着不为人知的殊勋,而且他还是河湟拓边的中坚力量,怎么看都是变法派的干将。但是韩冈对眼下炙手可热的进军罗兀的计划,却完完全全的站到了反对派的那一边。

韩绛那边已经是箭在弦上,不得不发。大军齐集,钱粮皆备,从上到下都知道要打仗了,这样的情况下,没有可能突然收手,就是天子也很难阻止烽火燃起。韩绛又是宰相,他在外领军,枢密院管不到他头上,天子的诏令他也完全可以不加理会。韩冈在天子面前说什么没用,最多也只不过是证明一下自己的先见之明罢了。

只是不论是从眼下朝局的稳定上,还是从维持与韩绛的关系上,王安石都不便让韩冈去动摇天子对横山战局的信心。尽管韩绛一旦得胜,回来后王安石也得避他锋芒,但凭着王、韩两人的交情,以及共同的政治利益,王安石都会对开拓横山一事鼎力相助。否则让韩绛听说了王安石在战前放了韩冈在天子面前进了谗言,等于是把韩绛往政敌的方向推去。

而且对王安石来说,他也不想听到有人反对陕西的战事。司马光连上三本,先是反对整修长安城防,继而反对河湟开边,最后就是对横山的战事大加指责。旧党赤帜所反对的,正是新党要支持的,如果其中出现了一点动摇,就等于是在大堤上开了个口子,让反变法的一派乘虚而入,由此为切入口,重又开始攻击新法。

以己度人,韩冈自问处在王安石的位置上,也会想着把反对的声音都给赶出朝堂去。如果做不到全部驱逐,那就有选择的排除。越是思维清明、手腕出众的越不能留,只把那些仅会叫着大道理,实际上百无一用的废物,留下来让他们恶心人。

韩冈突然失声笑起,真是闲得没事做了,竟然帮着新党想着如何打击政敌,还把自己给绕进去。

见不到天子那就不见好了,反正迟早能见到的。如果今次的退让,换来的是远离鄜延路那个漩涡,这笔买卖就做得不算亏——他可不想自己的名字跟失败联系上。

韩冈笑声未落,一名驿卒在院外敲门,递进来一封信,说是送信之人要见韩冈。韩冈把信拆开一看,里面没有信纸,只有一块薄薄的绣了鸳鸯的丝巾。韩冈算不得风流人物,在京城中,会送这等女儿家信物的也只有一人,他忙唤了李小六,出去把人接进来。

果然是墨文,才一年不到的时间,周南身边的小女使相貌没有多大的变化,但个头已经蹿了两寸多高。

墨文来到韩冈面前,行礼过后,小女孩儿很大胆的抬头与韩冈的眼神对上,“小婢受我家姐姐的嘱托,要传话给官人,不知官人可曾记得当日的三年之约?”

“这不是你姐姐的原话。”韩冈摇头笑了笑,小女孩的脸上藏不住心事,她进来后老道的韩冈一眼就看出不对劲了,“你家娘子那里出了什么事?”

“没……没什么?小婢只是怕官人忘了当初的约定……”

韩冈的嘴唇不高兴的抿了起来,如刀如枪的眼神,盯得墨文越发的不自在,声音细了下去。

熟视良久,韩冈单刀直入的问道:“有谁在缠着你姐姐?……既然你姐姐已经托付终身于我,无事不可直言。何须相瞒,直说无妨。”

小女孩儿终于怕了韩冈仿佛能看透人心的眼神,低下头,吃吃的轻声道:“……是雍王殿下。”

“谁?”韩冈愣了一下。

“是雍王殿下!”

“当今的二大王?!”

“对!”墨文突然爆发一般的大声叫着,她又抬起了头,小脸上怒气冲冲:“就是官家的嫡亲弟弟!前些日子,有个侯强要姐姐陪夜,被姐姐拿着官人送的匕首给吓走了。但现在雍王殿下化名秦二,一直缠着姐姐……”声音中渐渐带起了哭腔,“官人,你不知道姐姐的性子,逼到最后,她真的会什么都不管不顾的!”

韩冈看着眼含珠泪,雨带梨花一般的墨文,平和的笑了起来:“前次相别时我也跟你姐姐说过,我韩冈骗人的时候不少,可从不欺心。回去让你姐姐放宽心,过两天就去看她。天无绝人之路,一切放在我身上。”

小女孩子很好哄,带着韩冈的承诺,墨文破涕为笑,放下心头大石一般的轻松走了。问明白她出来时有人随行,韩冈便也不派人护送。韩冈现在发现,他要头疼的事情变得更多了。

‘二大王啊,还真有些麻烦了。’

跟亲王争风吃醋,韩冈是始料未及,的确是个麻烦。不过天子做不得快意事,难道亲王就能做得了?雍王殿下是以秦二的名义出来的,他易姓更名,必然是有所顾忌。要是他堂堂正正的表明了身份,事情可就要比眼下还要麻烦十倍。

想了一阵,韩冈还是准备先试探一下能不能让周南脱籍,如果不行,大不了直接把人弄出京城去。天子没事都出不了东京城门,这管不了事的亲王的命令难道还能追出京城去不成?大宋户籍看似严密,但要做手脚也没想象中的那么难。

据韩冈所知,章惇有位好友现今正在开封府中任推官。要想让周南脱籍,还得靠他帮忙。找来纸笔,韩冈匆匆写了一封信,折好后交给李小六:“小六,你去拿了我的名帖,往章府走一遭。”

