\section{第37章 异乡犹牵故园梦(上)}

平一郎小心翼翼的走进了厅中。

厅中一个身高看着有八尺,腰围似乎也有八尺的巨汉,一身绫罗绸缎也压不下他身上的精悍,那是平一郎他的主人。

在他主人的对面,还有两位客人,一名中年男子,穿着棉布衣服,另一名则是个少年,站在那中年男子身后。看模样,不像是仆从,似乎是晚辈。

以他主人的体型,普通点的身材就会变得没有任何存在感。平一郎能立刻注意到两位客人,那是因为他主人站立的姿态,和脸上的表情。

“一郎,来,先见过冯大东家。”平一郎的主人向他招着手,把他介绍给身边那位身量中等的中年男子。

在反应过来之前,平一郎先是吓了一跳。

他的主人即使在所有大宋海商中,也是数一数二的人物。去见契丹蛮子的大官的时候,腿都不带弯,搂着肩膀称兄道弟。

可他在这位冯大东家面前,腰杆子仿佛变成了柳树枝,摇摇晃晃,软软绵绵,说话间还带着讨好。

如果是普通的下仆,也许还看不出主人表情中那点细微的奉承,但从小就在宫廷中长大的平一郎,却是看多了类似的表情。而且他的主人,坐在椅子上,却连椅背也不敢靠。这地位得差的有多远。

客人们就在眼前,平一郎不敢多想,依足了规矩,向两位客人行礼。

平一郎的主人向客人介绍着他的底细:“一郎本来是在下在倭国的伴当,会说官话,办事又麻利,在下在倭国,多少生意多亏了他。这一回丝厂要另雇人,便把他招揽了来。”

从倭国到中国,离开家乡千万里,平一郎被招揽来管理被辽人卖给大宋的妇孺。在外表上已看不出他与汉人有什么异样,连装束也换成了汉人模样,只是举手投足还分明是倭人的习惯。

平一郎起身,就看见冯大东家的眼睛瞥过来,稍稍打量了一下,便对主人说道,“看起来文弱了点。”

“还好,一起回来这么些天,也没水土不服。要是他病了,在下可真的要头疼了。”平一郎的主人陪着笑脸,又招呼起冯大东家身后的小客人,“令侄还是坐下吧,坐下来说话。”

这位十三四,最多十五岁的小孩子,却让平一郎的主人腰骨弯折得更厉害,脸上的笑容也愈加谄媚。

“没事,小孩子多站一站没坏处。他爹让我带他出来,就是要多历练历练,多见识见识。”冯大东家好像不在意。

但平一郎发现,自家的主人只要看见那位小公子在冯大东家身后站着,就变得十分不自在,整个人都心神不宁。

“姓平?太平的平?倭语是这么念的吧。”冯大东家眼睛里透出好奇的神色,用有几分怪异的日语发了一个‘平’姓的发音,在得到平一郎点头后,他用更加好奇的眼神打量着平一郎,“平姓可是倭国的大姓啊,该不会是哪一家的公子吧?”

“回贵人的话,小人就是普通人家,小人父亲是贩鱼的,过去没有姓,父母给起的名号就是平一郎,是来中国后,入乡随俗,便以平为姓。”

“这样啊,看来是我误会了。”

从表情的变化上,平一郎觉得冯大东家没有相信自己的话,但他却没有追问。

倒是跟在冯大东家身后的少年好奇的问道:“四叔,为什么误会了?”

冯大东家很有耐心的解释道:“倭国的风俗,只有贵人才有姓,平民百姓就只有个名号。他这平姓,就跟你的韩姓一样,出了好些宰相、重臣。”

平一郎的主人笑道:“小官人不知,要他真是平家人,也不会在这里了。”

“老爷说得是。”平一郎低头说道。

平一郎已经习惯了伪装,在中国人面前暴露自己的身世,没有半点好处——他们可是跟辽人做买卖的商人——还不如将自己的出身说得低一点,然后通过勤奋一点一滴学会了汉字汉话,这样反而会被看重。

又被稍稍问了几句家世和倭国的风土人情——平一郎觉得,似乎是在满足那位少年的好奇心——就听主人吩咐道,“一郎,你先下去。待会儿,一起去工地上。”

“诺!小人明白!”

平一郎轻手轻脚的退了出去,还听见冯大东家说,“站着不动倒看不出来,一说话,一走动,倒是立刻就能分辨了。”

“倭国与中国的礼数差得也多。”平一郎的主人说道。

出了厅,平一郎没敢走远。一会儿还要跟着主人去工地,看情况,那两位客人可能也要去。就走到院子的角落处,静静的站着。

离厅门稍远,已经听不见厅中的说话声,但隔着一堵院墙,对面的声音却传了过来,两个人,都是男人的声音。

平一郎听过着两个声音,是他主人蓄养的清客,读书不成,但依然是士人,他的主人对他们也很尊敬。

“……听说是因为天子要大婚,所以特特南下来买绢。”

“官家的婚期就没两个月了吧,怎么现在才来说要买绢的?”

“朝廷的库房里面不知有多少宫造的丝绢,江南历年的贡赋也都堆在内库。”

“好象是太妃说太简素了,不好看。也不是什么大事,太后也不想驳了她的面子。”

“江南百姓要受苦了,这还不是大事?”

“除了仁宗皇帝,本朝的天子,都没有即位后大婚的先例。而且还是头婚,比起官家来,”

“等着吧,别到时候买绢变和买,和买变加税。”

“就是太妃和皇帝要加税,相公们也会拦着。”

“太后年纪也大了,官家再有两年就得亲政,相公们再耿直,也要为家里考虑。万一让官家记恨上了,现在没什么,过些年后,报复到子孙身上,他们辛苦一辈子到底是为了什么?还不如让官家开开心心的把王老相公的孙女儿娶回去。”

对面的声音高亢了起来,平一郎想了想,换到了对面的角落站着。身为异邦人,他知道这间院子里面的大部分人,都在猜忌自己。所以他时时刻刻都提着小心,遇上现在的事,自然是尽量不要让人误会的好。

离开对面的杂音远了,平一郎便发现,在这院子中,能听见江涛阵阵。

涛声从极远处传来,像海涛,又多了几分柔和,仿佛扬子江上的雾霭。

扬子江的寥廓,不是亲眼目睹,就绝对无法想象。离开江口都还有一日的海程,就能看见海水的颜色已从深蓝变成了浑黄。即使越过了大海,但长江的壮阔,依然让平一郎心魄动摇。

就是这座院落旁的松江,仅仅是一条汇入扬子江的河流,也宽阔的堪比日本的任何一条河流。而松江的源头,幅员八百里的太湖,更比琵琶湖大了不知多少倍。

一想到,洞庭、鄱阳、洪泽,一座座湖泊,都不下于太湖的辽阔。中国之大,当真只有亲眼看见了才能感觉得到。

“小官人小心脚下。”

平一郎的主人和冯大东家没有让平一郎等候太久,很快一起出了门。在后门上船,艄公掌舵,船工摇橹,一路向工地赶过去。

下船的地点,是与松江相通的黄浦东岸的一处码头上。

平一郎听人说起过,这里原本是荒地,是官府刚刚划拨下来,交给他的主人和其他几位大东家——他不清楚这位冯大东家是不是其中之一——建造倭人坊,让过海而来的妇孺,居住在这里。

仅仅半个月,倭人坊的围墙还没有建起来,但从码头延伸出去直到倭人坊的铁路已经铺设好了,只有半里多长,但修建倭人坊的物料,都从码头上,通过铁路运到工地上。

工地的旁边,稍稍高出周围一点点的小丘陵上,有着一片草屋。也是刚刚修起来,供人居住。

不过快中午的时候,住在草屋里面的人,都在工地上帮忙。

一行人走过去时,她们纷纷都跪了下来。

来到中国的倭人,平一郎都问过她们的姓名和年龄,也是他一一登记起来。最小的十岁,最大的有四十多。太小,太老,他的主人和另外几位东家都不肯买。

尽管被当做奴仆买下,但平一郎还是为他的同胞感到庆幸。留在国中迟早要死,多少贵人被送进矿山里面挖矿,没两天就被拖出去丢了。

每天都能吃到这么好的食物,山里的狼和熊,一个个都是毛光水亮,要不是契丹蛮子一个个都喜欢射猎,闲来无事就拿着弓,带了鹰犬入山,山里的野兽早就下山来攻击村庄了。

平一郎没有时间感叹什么,不说日本与中国的差距,就是与契丹,也是天差地远。他的国家在自己的天地里称王称霸太久了,完全忘了这个世界有多么残酷。

当契丹蛮子跨海而来,天下升平的梦境便彻底破碎。

每一个契丹蛮子,都装备着比将军最好的甲胄都要坚固都是铁甲,拿着名匠打造的唐刀也比不过的钢刀。

在过去,日本的唐刀大批大批的被中国的海商买走,那时候,国内还嘲笑过中国的匠人,连柄好刀都打造不了,难怪被契丹蛮子欺压。

只有到了中国……其实还没有到中国,平一郎就见识到了中国的刀剑有多么犀利。

就在船上,水手们人手一把钢刀,全都是能将上等唐刀一刀砍断。

那些高贵家名的武士,在契丹入侵之后,有很多都下海做了海贼。他们平常都躲在濑户内海中,一看到商船过来,就一起冲上来。

这一艘大丰号从界镇满载着妇孺返回中国的时候,就遭到了海贼的攻击,一时间二十多条小舢板围攻上来。

七八个武士跳上了,一个个手持太刀,身手矫捷。

几名在本国招收的水手边逃边拿着木桨挥舞,但手腕粗的木棍都被一下砍断,下一刀,就被砍死在甲板上。

平一郎即使在宫廷中,也很少见到这般身手的武士。

可船上的汉人水手拿着自己的钢刀迎上去时,一刀就劈断了对面武士使用的太刀。

只是好勇斗狠的水手,就靠着手中犀利的钢刀,与剑术高超的武士斗了个不相上下。

等到船主让人搬出了藏在船上暗格中的虎蹲炮,武士们就彻底败了。

火光一闪,虎蹲炮就将船头上的数十名旧日的武士打成了齑粉。他们身上流出来的血,把甲板都染得通红。

就在那一刻,平一郎再一次确认了,想要复国,只有在大宋才能找到道路。
