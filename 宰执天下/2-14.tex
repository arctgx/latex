\section{第五章 平蛮克戎指掌上(二)}

【第二更,求红票,收藏】

四月中的秦州,已经有了炎暑的一点苗头。在家中还好,但到了外面,尤其是午时前后,日头火辣辣的,照得人皮肤发痛。

在正午时分,顶着烈日出城,王韶原本就是黝黑的一张脸,被太阳晒得黑里透红。韩冈也是热得受不了,要不是顾及着形象问题,都恨不得换上一身短打,而不是穿着宽袍大袖、厚重无比的公服。

通往西门的大街上,韩冈和王厚紧紧跟着王韶,外围是赵隆和杨英带着护卫们守着。他们没有骑马,反而是安步当车。虽然连韩冈都不知道王韶是搭错了哪门子的弦,但既然王韶有这份兴致,他和王厚这样的小辈,也只能奉陪到底。

王韶很悠闲的走着,左右看着大街两边的店铺,时不时的还走进铺子问问价钱,显得兴致很高。

“是不是为了市易在查货价?”韩冈在王厚耳边低声问着,王韶不是爱逛街的性子,何况大热天里逛街,本就是脑袋坏了才有的蠢事。

“谁知道。”王厚也摇摇头,他的老子心里在想什么,他这个做儿子的有时也不清楚。

韩冈看着在一间绸缎铺中,问着一匹碧纱价的王韶,心中越来越是疑惑。若他真的是为了市易做调查,应该把那个元瓘一起叫来才是,他才是王韶内定的主管市易事务的人选。

从绸缎铺出来,王韶又转进来一间兵器铺。在西北,为了抵御党项西贼,官府并不禁止平民百姓携带兵器,只要不是硬弩长枪,如长弓、腰刀这些并不犯忌讳。不像中原内地,平民出外远行,只许带着朴刀。

这样的政策,使得兵器铺也能光明正大在大街上营业。也就是平民购买弓刀,必须在簿子上加以登记,就像药铺卖砒霜等毒药一样,都是要登记的。

王韶走进的这一间兵器铺,在秦州城中算得上比较大了。三开间的门面内,在墙上高高低低挂了不少长弓腰刀。王韶在里面转了一圈,看上了一张弓。招手让掌柜把弓拿下来,冲着韩冈和王厚道:“玉昆,二哥,你们过来看。”

“是不是兴州的弓?”韩冈看了一眼,便问道。

“官人好眼力,的确是兴州造。”兵器铺的掌柜点头笑道:“三位官人,这可是小店的镇店之宝,足足两石一斗的力道,力气小一点的根本拉不开。”

虽然大宋是以弓弩为上。远程攻击,向来在军中被看得很重。上阵时,卒伍们无论拿着长枪还是刀盾,都少不了带上一张弓或是一架弩,

但党项人那边,也是一向重视弓弩。军中用弩,党项人由于技术原因,造不出力道出众的硬弩。但长弓的制造技术就是有名的出色,能造上等弓箭。尤其是兴庆府的官造,比起东京城弓弩院的出品,还要高上一等。

在西北,一张兴州良弓,往往能卖到十贯以上。韩冈常用的那张,由过世的二哥送给他的一石三斗的战弓,便也是出自于兴州。

“玉昆,你既然认出来了,就来试试。”王韶说着,就把长弓递给韩冈。

韩冈接过王韶递过来的长弓,用力拉了一下,缠了马鬃和人发的弓弦勒得他手指生疼。果然是张能杀人的硬弓,不是给墙上装饰用的玩具。

“有没有扳指?”韩冈问着。

“有!有!”店主立刻从店里的角落处,掏出一个牛角做的黑色扳指。

韩冈拿过来套在右手大拇指上。用扳指勾住弓弦,前后弓步站定。右手后扯,左手向外一推,两膀子一起用力,只见他吐气开声:“开!”

就听着弓身嘎嘎的响了两下,这张硬弓在韩冈手中被拉成满月。

“玉昆好神力。”王厚拍手笑赞着。

兵器铺的掌柜也在说着好话:“官人果然神力惊人。”

韩冈松开手,弓弦嗡的一声回复了原状。他放下长弓,摇了摇头:“哪有两石一,能有一石七八就不错了。”

被韩冈戳穿,掌柜仍是一脸笑容,“做生意嘛,这也是正常的。不吹上几句,本钱早折光了。何况真有两石的弓,也不是普通人就能拉开的。如官人这般两膀子有千百斤气力的人物,秦州城……不,秦凤路中也没有几个。”

韩冈把长弓递还回去,又道:“如果掌柜的你弄到两石二三的硬弓,我倒想要一张,若只是这一石七八,那就算了。”

王厚听着乍舌:“也只有玉昆才能用得好两石两斗的硬弓”

“是想拿来练练手罢了,如果是阵上使用,我的那张一石三就已经够用。但平日习练,力道强一点倒没坏处。”韩冈笑道,“不过,兴州的两石强弓,做出来的少,流出来的更少。不定能弄到。”

不知被韩冈的话触动了哪根心弦,王厚突然叹到:“现在西北说起弓,就是兴州弓,说起鞍,就是灵州鞍。如今的都作院、弓弩院,造出来的什物是越来越差了。”

王韶点点头,转身往外走,边走边说:“最近王相公有意更易军器监,设提举军器监一职,究其因,便是因为京城都作院里的弓弩兵甲越造越差。”

“我军向以弓弩为上,籍以与契丹、党项骑兵相拮抗的,也是以锋锐著称的箭阵、弩阵。可如今,弓弩一年不如一年,一批差过一批,再难上阵。”韩冈附和着,关于军中的弓弩兵器,的确是质量越来越差。

“玉昆你只是听说,我在可是亲眼见着。的确不堪……”王韶话说了一半,突然停住脚。向着斜对面拱手作揖。

大街斜对面,王韶行礼的方向,一个官员刚刚把腰直起来。韩冈认识他,是与王韶同为机宜文字的官员,复姓宇文。韩冈看他的模样,应该是先一步向王韶行礼。

就跟韩冈把陈举弄得族灭之后,秦州城中的胥吏少有人再敢招惹他一样;自王韶把向宝气得中风后,除了李师中、窦舜卿那几个高官,秦州城内的低品官员,还真的没几个敢在王韶面前拿大,这个宇文机宜先向平级的王韶行礼也是一桩事。

王韶和宇文机宜都没寒暄的意思,隔着老远行过礼后,宇文机宜转身离开。看着他背影,王韶叹着:“都是向宝的功劳啊……”

“不知向钤辖什么时候会被调走?”韩冈问着。

王厚道:“向宝最近不是听说已经能走了吗?说不定过几天就销假回来了。”

“向宝不可能再留在秦州。”王韶边走边说:“他肯定要走的。不管向宝最近恢复得有多好,但中风就是绝症!多少人盯着他的位子,现在有了这么好的一个借口,哪个肯放过?天子或许会看在他为朝廷丢了脸的份上,让他继续留在军中。但秦凤为军国之重,天子不会容许一个五尺残躯,执掌秦凤军事。”

韩冈点点头,王韶说得的确没错,在世人心中,中风就是绝症,再怎么都恢复不了。既然向宝因中风而病倒,没人会相信他能复原。即便他真的复原,官场上那些想顶他的班的,也会当作没看到。

大概张守约也是这么想。韩冈便问道:“不知张老都监能不能接任钤辖一职?”

张守约也是韩冈的举主,韩冈当然希望他能水涨船高,再晋升几步。别看都监和钤辖在一路将领中只差了一步,钤辖下来就是都监,但这一步几乎就是天壤之别。就像州官中,知州和通判的差距。张守约若能跨过去,日后他的面前便是海阔天空。

“张守约这个月就要回京奏复,就看他在天子面前的表现了。”王韶也挺希望张守约能更近一步,“若是张守约能为钤辖,在秦州城中,也能多个人说话。”

韩冈也道:“希望张老都监能在天子面前把万顷荒田之事为机宜分说清楚。”

“荒田……荒田!”王厚突然怒起,“把一万顷说成一顷,又从一顷说成一顷都没有,窦舜卿他们还弄不厌吗?!”

韩冈笑道:“除了荒田之事,他们还有什么能用来攻击机宜?”

“三百里的渭水河谷,窦舜卿、李若愚他们竟敢说一亩地都没有,朝中竟然还正经八百的派人来查验……”

“没办法。自来都是眼见为实,耳听为虚。京城和秦州隔着两千里路,天子亲眼看不见,还不是只能由着人随口乱说。”王韶悠悠叹着。这种事,谁也避免不了。天子不是圣人,不可能真的洞烛千里,只能通过文字作出判断。当来自秦州的两方奏报互相矛盾时,赵顼也只能听着他派出去调查的内臣的一面之词。

“其实也不是没有解决的办法。”韩冈沉吟着,突然说道,“就让天子亲眼看一看秦凤地理,自然能知道谁在说谎。”

“怎么看?”王厚奇怪的问着。

“看地图?”王韶的反应很快,他摇着头,韩冈的办法并不现实,“不可能的。地图谁都能画,而且即便看着地图,也照样分辨不清哪里是山,哪里是田。即便呈上御览,在天子那里也比不过内臣的一句话。”

“不是地图。”韩冈笑了一笑,又摇着头强调一遍:“不是地图。”

