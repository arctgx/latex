\section{第14章 霜蹄追风尝随骠(十)}

李宪终于到了。

韩冈等了他许久,近二十天的时间,就到了两批骑兵,两千五百暂时派不上用场的白食客。今天跟随李宪而来的,是第一批步兵,两个将八千人,另外还有两个指挥的骑兵。后续的另外两万多士兵,会在十天内

李宪一见到韩冈,便道歉道:“李宪来迟了,辽人那边差不多快腾出手来了吧?”

“还来得及。”韩冈与李宪往内厅走:“辽人也不是一身清闲。黑山那边打得正热闹呢。”

“还好。否则就是愧对龙图了。”李宪笑了一声,又问道,“屈野川的营垒修好了没有?”

“柳发川和暖泉峰的都已经完工了。不过也只是营垒而已,要修成城寨,必须要征发更多的民夫来。”

李宪苦着脸:“府州、麟州户口都不多,不知能征发起来多少民夫。黄河东面的军州人口不缺,又太远了,过河都不容易。”

“人员的确是个麻烦,不过也不是不能解决,调配得好的话,麟州、府州的人力还是足够的。”

“有龙图在,的确是不用担心,当年重修大河金堤,只有开封一段”

当年利用河北黄河,束水攻沙的方略,可到如今为止,内堤都没有完工,真不知道还要多少年。

“都知谬赞了,可当不起,只能是尽力而为啊。”

与李宪在厅中分宾主坐下,等帐前服侍的老兵奉上了茶汤,韩冈道:“不过眼下军情紧急,也没时间给都知接风洗尘了,还望勿怪。”

“也没心情喝酒了。”“预计需要几座城寨?”

韩冈一根根曲起手指:“丰州城重修是不用说的,浊轮砦、子河汊小寨、唐龙镇等七个寨子都要修补或扩建,也就是屈野川、浊轮川旧有的八个城寨全都要兴工役。另外柳发川和暖泉峰各修两个千步城,周围四座到六座寨堡环卫,如此,丰州之地可当十万辽师。”

“这不可能!”李宪差点就要叫起来,“没有三五十万民夫,哪里能修完这么多城寨?”

“这当然不是一个冬天能完工的份量。”韩冈不急不缓的解释道,“丰州、浊轮砦、子河汊小寨,这是亟需重修的。然后柳发川、暖泉峰两处,剩下的寨堡得等到日后慢慢来了。”

“原来如此。”李宪点头道。其实想一想就该知道这肯定不会是赶在这个冬天完成的巨大工程,他也是一时被惊到,“不过这钱粮是不得了了。”

“为了河东西陲安靖,该花的钱还是得花。而且等到大宋和辽国之间的风波平息之后,处在要道上的柳发川和唐龙镇可以设立榷场,利用榷场的收入回哺两川的寨防。穷十年之功,也就能将丰州的寨防给完成了。”

“若榷场能够建功,寨防甚至日后驻军的开支,朝廷至少可以放一半的心了。”李宪抬眼问韩冈,“龙图已经写好奏章了吧?”

“当然。”韩冈笑道“这一修造寨堡的计划,绵延十载,不是一任两任就能完工的,必须先从朝廷这边定下规划。”

李宪沉吟一下,道:“……李宪愿附龙图骥尾,不知龙图是否愿李宪占个便宜。”

韩冈哈哈笑道:“固所愿也,不敢请尔。若得都知襄助,这一方略当能更易得到天子的认可。”

李宪如今是天子最为亲信的几位领军宦官,在宫中的人脉深厚,顺水人情给了就给了。日后天子派遣中使巡边或探访,有李宪和王中正这两道关系,就方便许多了。而且他也知会过折克行了。

夺下丰州后,折家的势力必然延伸至屈野川,两个榷场就代表着稳定的财源,但折家与辽人回易通道的利润就会大幅下降,韩冈不打算太过开罪人,需要事先取得折家的谅解。

一名亲兵悄步走进厅中,“龙图,丰州那边派人来了。”

“好,我知道了。”韩冈点点头,对李宪道,“这个信使,都知要不要见一见。”

“谁?”

“当然挑起了整件事的那一位。”

李宪哦了一声,“被射掉耳朵的?”

“不只是射掉耳朵。”韩冈啧啧叹着,“他可是咬着自己掉下来的耳朵,追着贼人一直杀到辽境。虽然没有追到人,但也烧了两间巡铺。我可是想见见他呢,正好被折克行派来了。”

……………………

韩冈上下打量了被折可适带进来的人。跟折可适差不多的年纪,相貌有五分相似,说兄弟更合适。隔了十几天,折克仁耳朵上的伤口已经结疤,不需要包着伤口,少了半只的缺口便分外惹眼。

“你就是折克仁?”

“回经略的话,末将正是折克仁。”

折克行被派来送信只是表面理由,实际上是让他过来待罪的。眼看着边境上越闹越大,罪魁祸首之一的折克仁若不来韩冈这边来报个到,折家的态度可就成为问题了。

“古有拔矢啖睛,今有拔箭啖耳。”李宪嗤笑一声,脸随即一板,喝道:“折克仁你可知罪!”

折克仁闻言立刻跪伏在地,头也不敢抬,“折克仁知罪!”

“好了,都知。”韩冈看得出来,李宪是投桃报李,扮黑脸呢,“宋辽两国之间本有约定,若有贼人越境犯事,捕获后可各按本国律条处断。捕获之前手段的粗暴点也没什么关系。可惜每次都给那些贼人跑了。”

两国之间,如果是势均力敌、又不想撕破脸的话,有个搪塞的借口就够了。

“攻入辽境又是如何?!”李宪厉声问道。

“此乃被贼人所伤后一时义愤。若辽人肯处置犯界伤人的贼人,我们也可以赔他们的军巡铺啊。”韩冈道,“折克仁,到时候重修两间巡铺,你愿不愿认罚?”

折克仁闻言喜色上脸,伏地不动:“任凭经略处置!”

李宪动了一下嘴,他帮韩冈做桥,却不想这么轻轻放过折克仁。但看着韩冈的神色,终究还是没有说什么。

攻入辽境,烧毁巡铺,折克仁做下这么大的事,使得局势愈演愈烈,就算他是被射伤了耳朵,但由此重惩也不为过。可韩冈连板子都不打,罚点钱便就此放过——而且按照韩冈的说法,如果辽国不给折克仁的耳朵一个交待,他甚至连赔偿巡铺的钱都不用罚——李宪难道还能跟韩冈顶着来不成?只要局势不恶化到不可收拾,如何处置折克仁的权力就在经略使的韩冈手上。

“折克仁!你追击入辽境,这件事没什么好说的,算不得你的错。”折克仁躬了躬身,专心听韩冈的训斥,“但你追至辽境的时候,有没有想过,那是辽人的陷阱?是诱你追击,然后设伏。西军就不说了,河东这边与西贼交锋多年,没少吃过被诱伏的亏吧?”

折克仁连反驳都没有,低头向地,诚诚恳恳的道:“末将知错!”

放过折克仁,韩冈也没有想太多。只是小事而已,事情的起因不在他,而且冲突是必然,折克仁只是偶然撞上的。

韩冈又冷下脸:“这几日出去打柴的士兵,被人伤了两个,其中一个还死了。这笔账有机会还得算一算……好了,折克仁,你且起来吧。说说折府州有什么事要让你来麟州禀报?”

虽然是附带,但终究不会是无关紧要的小事。

折克仁依言起身,谢过韩冈后禀报道:“这几天,有三百多黑山党项蕃人南下避难,比起之前数量大幅增加。知州估计,接下来南下的蕃人可能会越来越多,知州想问一下经略,该怎么处置他们?”

韩冈闻言就皱起眉,与李宪对视一眼。自从辽人击破西夏的黑山军,直取兴灵,就有黑山党项南下,但数量一直不多,突然间大幅增加,肯定是契丹人为了控制黑山开始下重手了。

辽人势大,当地的党项人逃难是肯定的。西夏已经灭亡了,避往大宋是他们唯一的选择。不过这群蕃人能不能相信,会不会在国中生乱,则是一个问题。而且他们最终人数会达到多少,则更为关键的。问题的大小,跟人口关联很深,少了还好说,一旦人数多了,安置他们的地方就要费一番思量了。

韩冈扭头对李宪道:“黑山的乱事不可能迁延日久,辽人为了尽快解决黑山河间地,转向丰州来,肯定会加大在黑山的动作。南下逃难的蕃部当会大幅增加。而且辽人也很可能故意驱赶他们一起杀过来,甚至混在这些党项人之中,或是伪装成党项人。”

李宪沉吟道:“丰州的兵力看来不够啊。”

“看来要劳烦都知了。”韩冈道。不管怎么处置,手上有足够的兵力镇压局面,让南下的蕃人老实听话,这都是首要条件。

李宪闻言会意:“休整一日,李宪就领军北上。”

“多休整一两天也没关系。折府州那边一时还撑得住。”韩冈笑着对李宪说了一句。心道李宪既然北上,看来自家也得往丰州去坐镇了。

折克行和李宪之间,谁为尊长的问题很难处理。李宪的官位的确在折克行之上,但李宪的职司主要是在征西之役上,丰州之战他并没有受到明确的朝廷诏令。而且武将一贯大小相制,如果没有朝廷的明确诏令,就是一路副总管的号令,都巡检都可以直言拒绝。这就在两人之间留下争权的可能。

只有韩冈的身份可以镇压住两名实际统军的将帅,从他的角度来说,也不愿意指定李宪或是折克行代替自己指挥全军。这是统帅之权,随便交给阉人或武将,丢脸的将是他韩冈。而且边境的冲突,更多的是属于政治的范围,不需要武将上阵,这是经略使的责任。

