\section{第14章 霜蹄追风尝随骠(15)}

五百多里的路程,已经走了大半,距离屈野川也只剩下一两天的行程了。

耶律世良一路都提起来的心,正随着接近宋境,而一步步的升到了喉咙口。可想而知,直到通过宋人的防线,这颗心都不会落回原位。

身下的坐骑耷拉着脑袋有气无力的走着,几百里的长途跋涉,几乎耗尽了这匹草原良驹的体力。而连着数日都是冷食果腹的宫卫铁骑,也都是失去了应有的锐气。

一路上耶律世良都在苦思冥想,要用什么办法来突破宋人设在屈野川边的防线。但除了出发前所做的准备之外,剩下的也只是随机应变四个字。

覆盖了一层积雪的荒漠上,听不到清脆的蹄声。数百骑兵踏着早已被踩平变黑的雪地,向着目的地前进。

道路上的军队绵延逶迤,天色越发的晦暗。耶律世良正想着是不是该扎营,前方的数里之外突然传来激烈的喊杀声。

勒马停步,一行人望着视线不及的远方。同样的情况已经出现过多次,没有人感到惊讶,只是都不想蹚浑水。另外耶律世良也想知道这一回究竟又会耽搁多久。

很快,在前面探路的斥候返身回报,“是固密部不知与哪一家打起来了。”

“是固密部抢人,还是被抢?”耶律世良问道。

“是被抢。”那名斥候立时答道:“有一伙贼人要抢固密部的粮食和战马,人数在六百上下,跟固密部差不太多。”

“看来有好一阵耽搁了。”

耶律世良不想绕道。听声音,固密部和强盗打得越发得激烈,要绕过前面的战场也不知需要绕多远。还不如就此扎营,歇上一夜,等明天再动身启程。

在耶律世良的命令下,麾下战士立刻翻身下马,做着扎营的准备。而他本人则依然远眺着前方。在天色完全黑下来的时候,另一名斥候也转了回来,带回来了最新的军情,“固密部败了!”

这一次不用斥候回报,没过片刻,耶律世良就已经看见回窜的固密部残兵。

不过那些残兵败将显然畏惧耶律世良这一支看上去就是兵强马壮的队伍,完全没有接近,而是远远的绕了开去。

耶律世良身边只有四百骑兵,但各个精锐,精良的装备可以掩藏起来,可百战精兵的气势,却是在生死线上打滚的人们都不会忽视的。而且四百人的数目,放在南逃的黑山党项部族之中,也算是人数比较多的一支。

在南下的道路上,耶律世良的四百人马,多次吓退了试图打劫的南下部族。不过也因此而引来了数支小部族的投效。正常的情况下,没有哪支部族会拒绝扩大自家的人口。而从耶律世良的角度来讲,若是能多吸收党项部族听命于己,一同南下旧丰州。有他们掩护身份,肯定能顺利的通过宋人的防线。

可惜的是耶律世良不敢与其他的部族深入交流,再怎么伪装,也瞒不过真正的黑山党项。面对这几支试图靠上他这一株大树的几个部族,耶律世良都是拒之千里,根本不予接纳。若还是不识相,就干脆了当的开杀戒,杀光了事。

两次下来,队伍中多了三百多可以用来替换的马匹,但也少了二十多人的身影,以及同样数目、尚能跨马而行的轻伤员。对于这个交换,耶律世良本人并不乐意见到。所以他还是很担心的攻击固密部的强盗,转过来再来劫上一票。

但耶律世良是白担心了,一夜的提心吊胆,也没有等到强盗的出现。一彪人马带着困倦重新上路,经过昨日的战场时,只看见了一地被剥光了衣物的尸体。而死去的战马,也被割去了四肢以及躯干上最好的肉,只剩残骸。固密部的这些尸骸被冻在地面上,肢体和表情扭曲着,让人望之生畏。

没有多看失败者一眼,耶律世良所率领的这支骑兵,越过昨日的战场,继续向东南方向前进。尽管周围还有一些如同独狼的强盗耳目在外围游走,但耶律世良本人,还是很快将这一场战斗抛诸脑后。

强盗终归是好对付的,可想糊弄过宋人,却绝不会那么容易。对于自己身上担负的任务,随着离宋人越近,耶律世良就越是担心。

堆在屈野川一带的宋军,据探查接近五万。而西京道上可以动用的精锐,也不过这个数字。而宋人已经修好了诸多寨防。无论什么样的骚扰,都不可能将宋人逼退。西京道的实力,短时间内怎么看都不可能胜过严阵以待的宋人。想要搅浑水,让大辽有机可乘,几率微乎其微。

在西京道的主力逗留在黑山河间地的时候,宋军出兵占据了旧丰州。从那时起,屈野川、及其以南的土地,就已经没有办法再挽回了。就如同在宋夏两国大战盐州的时候,三万宫卫突袭兴庆府一般。兴灵之地,宋人和党项,都不可能再占回去了。

耶律世良这一次的任务,真正说起来,不过是找回点面子,顺便给宋人添些麻烦而已。实际上,并没有太多意义,想要让宋人放弃旧丰州,跟大辽放弃兴灵一样困难。不过在耶律世良而言,这也是他个人飞黄腾达的机会。

伪装身份,混迹在南下的黑山党项部族之中,设法混入宋境。找机会挑拨归附的黑山党项与宋人的关系。等到河东乱起,便寻机返回大辽。

在耶律世良的计划中,他将会向西横贯大漠,转去兴灵,再从兴灵沿着黄河回西京道——南逃的党项人充斥于途,又是厮杀不断,从地斤泽等几个绿洲走,反倒比原路返回更稳妥。

不管最后能不能在丰州闹出乱子来,但只要能成功回返,就是大功一件,耶律世良可是盼着自己能够有机会在捺钵中找到一个属于自己的位置。

翻过一重低矮的山岭,冰结的屈野川就在眼前。

周围的风景已经不再是微微起伏的荒漠沙丘,不过冬天的山林,也是单调的白色和灰色。得等到来年春天雪化之后,才能看到铺满大地的鲜艳色泽。

环绕在周围窥伺的骑兵,也不再是黑山党项部族的成员,而是一名名身着陌生装束的陌生面孔。

耶律世良随即提高了警惕,从他们出现的地点来看,身份是肯定的,目的大概也能猜得到。目光随着那些陌生的骑兵转动,耶律世良等人的心中则开始回忆自己接下来要冒充的身份。

“你们是那一部的?!!!”远远地,就有人提声打探着这边的底细。

早有准备的耶律世良立刻让手下一名会说党项话的亲随高声回复:“我们是锡丹部!”

那边寻究底细的声音过了片刻,又再次响起,“酋首达克博何在?”

耶律世良一颗心猛跳,这个名字还是他为了能顺利的冒充锡丹部,特意去打听的。怎么宋人都知道得这么清楚。

只是转念一想,宋人收编的黑山党项部族不在少数,从中挑几个在河间地的蕃部中,人面广、人头熟的黑山党项,也不是什么难事,不过对于这个问题也是有准备的。

“老族长被契丹人害死了!现在的族长是老族长的儿子斡得!”

随着回话声,一名骑手从耶律世良的队伍中出来,立马阵前,“我便是斡得,被契丹人害得家破人亡,如今来投奔大宋,愿一心一意,做大宋治下的忠臣!”

“哦,是吗?”折可大回头,“你们还认识达克博的儿子?”

几名来自黑山下的党项人互相看看,都是摇头,“没什么印象。只认识达克博一人。”

“那还真是让人遗憾。”折可大悲天悯人的叹了一口气,让人传话道,“你们这是请求归附的样子?”

耶律世良立刻明白了宋人的用意。怀着被羞辱的愤怒,他聪明的选择了下马。在他的传话下,所有人全都离开了他们赖以傲视同侪的坐骑。

正想着日后如何将这份羞辱回报今天的宋人,耶律世良便发现两侧山间的突然冒出了千百名手持重弩的宋军士兵,而挡在面前的栅栏后,也出现了上千名宋军战士。

是陷阱!

念头一闪而过。当耶律世良正想有所动作,胸腹和头面处便传来一阵剧痛。只来得及低头看上一眼已经没入胸口的箭矢,惊骇欲绝的神色便凝固在脸上。

千万箭矢如蝗如雨,一波接一波的攒射将小谷中的契丹人射得无处逃窜。

折可大瞄着在在箭雨中四处奔逃的契丹骑兵,毫不犹豫的让人传令不要停下来,要不停地射击。通知友军的号角声也响了起来,埋伏起来的两支人马转过去堵着契丹人的后路。

折可大冷眼望着下方,只见在箭雨组成的风暴中,一人毫不犹豫的迎着狂风暴雨,双手向天,似乎在高呼什么。但折可大没有心情去猜测,他从上面领到的命令只有一个,

“可疑之人,宁可错杀,不可放过!”

