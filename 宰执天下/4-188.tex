\section{第31章 九重自是进退地(二)}

元丰元年正月初一。

天气难得的晴好,月初的夜晚,没有什么能掩盖得住天上的星光。一个个发射上天空的烟火,也压不住天狼、南河三和参宿四的光芒。

虽然才是三更天,刚过了子时,但灯火映着雪光,倒不显得有多阴暗。前天的一场雪,让东京城变得银装素裹起来。

韩冈推开窗,噼里啪啦的爆竹声立刻就大了起来,一股寒风卷入房中。深深呼吸了一口冬夜冰寒的空气,守夜时变得昏沉的头脑,一下又变得清醒起来。

回头看看,方才闹着要守岁的儿女都被乳母抱回房去了,小孩子熬不了夜,放过鞭炮就困得睁不开眼了。房中就剩几名妻妾正帮着自己整理着上朝时的服饰,连使女都给打发出去了。

他笑道:“已经是元丰元年了。”

以熙宁为号的十年中,在军事上算是开国以来的一个高潮,南征北战皆有所得,西夏被打得毫无还手之力,而开疆拓土的功业,也是让当今天子走近太庙也能扬眉吐气、不愧先祖。

不过在政事上,朝局上的两党争端不说,就是天灾也是一个接着一个。市易法、免役法,还有易名为便民贷的青苗贷,本质上也是从民间刮钱以充国用。如果没有天灾,其实倒也无妨。但熙宁年间的后半段,也就是最近的这几年,整个国家的民间财富在连年灾异下,是在不断萎缩的。

就在去年,韩冈在广西还不觉的,但五岭以北,又是个全国性的大灾年,也幸好安南之役没有动用多少兵力,尽量俭省着来,否则还不知能不能支撑得了。

看着这个局面,其实是往汉武帝方向走了,士林和民间也是有所议论。赵顼本人当然是不喜欢的,王安石曾要他以尧舜为目标,而赵顼也是觉得至少也得是个唐太宗。被视为到最后要下罪己诏的汉武帝,那自是一个屈辱。

在这样的情况下,改元求个吉兆也是必然。

“听说是太常礼院给出了两个年号,让官家钦点。”王旖与韩冈说着话,周南和云娘则拿了韩冈的朝服过来。

今天是正旦大朝会,平常韩冈所穿的三品公服当然就不能穿了,必须穿上衣裳都为朱红色的绯罗袍、绯罗裙。衬里是白花罗的中单,韩冈在散官升到六品之前是没有的。不过他现在已经积功为从五品下的朝散大夫,又被赐了三品服色,早就能用上了

——散官阶与本官是两回事。本官决定俸禄,又名寄禄官;散官只决定服色,也就是朝服、公服的装束而已,远比不上决定俸禄多寡的本官重要【注1】。

韩冈张开手,让云娘拿着一条素罗大带帮忙将套上身的中单给系好,“是哪两个年号?”

“美成,丰亨。”王旖偏头看了看韩冈,指着告诉云娘,“腰间要系紧,罗带白头不能露出来。”

云娘立刻好一通忙活。这也是韩家的习惯,上朝时,帮着韩冈穿戴都是妻妾的工作,都不让使女插手。

韩冈啧着嘴品鉴着美成、丰亨这两个礼院制定的年号,看来朝堂上下的确是被连连大灾给吓怕了,都是祈求丰年的。“不过这两个的确不怎么样,听起来就不顺耳。美成是羊大带戈,不吉利啊。”

韩冈说得跟外面的拆字算民的瞎子一样,‘美’字拆成羊和大,而成则是包含一个‘戈’字,羊大了要杀,当然不吉利。

“官家也是这么说的。”王旖上下打量了丈夫一番,看着没有问题,便点了点头,周南忙将穿在外面的绯罗袍拿来。

韩冈又是张开双臂,让周南和云娘一起,将袍裙穿在身上,王旖依然在旁监督。

韩冈这是要参加大朝会,衣着、装束上有一点不对,就是不敬之罪,御史们可是不会嫌自己工作少。不过有王旖这位宰相的女儿盯着,周南又是出身教坊司,对服章之仪都是很了解,韩冈就能乐得轻松。

“丰亨只看字面倒是不错,丰亨豫大嘛。财多德大,故谓之为丰;德大则无所不容,财多则无所不济,无所拥碍,谓之为亨,故曰丰亨。”多年的勤学不辍,韩冈已经可是算是底蕴出众的儒者了,孔颖达的注疏也是信手拈来,“天子为什么不喜欢这个年号?”

“为子不成。”

韩冈一拍手,“难怪!”

“官人别动。”周南一声叫住韩冈,让他一下停止了动作,将下裳给韩冈套上,又拿了一条黑色的犀带出来,与云娘一起动手系紧在韩冈的腰间。

赵顼不喜欢丰亨中的‘亨’字,就是因为下面是‘了’,比‘子’少了一横,所以叫‘为子不成’,与父母不利。以此为你年号,当然是对高太后有影响。

“天子为人至孝,所以不喜欢这个‘亨’字。”王旖说道。

不过韩冈估计更多的还是怕‘子’少一笔的‘亨’,会绝了他赵顼的皇嗣,这也可以解释成是‘为子不成’。

所以赵顼将丰亨,去了亨字,前面加个元。元者,始也,又可做‘大’解,按颜师古的说法,是‘更受天之大命’。元丰便是受天之大命,始丰、大丰。

元丰年号出台的由来,也只有前宰相的女儿,才会如此了若指掌。

据韩冈所知,当初以熙宁为年号也是这个原因。治平四年,赵顼登基的第一年——年初英宗驾崩,当时还没有改元——也是灾异连连。

五月旱、六月涝,近七月的时候,河北流民在道,这都不算什么了,从八月开始,京城、福建接连地震。所以为了求一个平安,故而有了熙宁二字——‘熙’是繁盛,‘宁’自然是安宁。当时是希望老天爷能消停些。

“以元丰为年号,是九月初的事了。爹爹也知道的,十一月的时候,诏书都预备好了,是冯相公领头签押,不过是到了今年冬至郊祀的时候,才公诸天下。那时,却已经换成是吴相公了,便又忙着改诏书。”

韩冈听王旖说着他不知道的故事,忍不住哈哈一笑。

冯京也是倒霉,作为宰相,在郊祀之年竟然还能给赶下台去。可以说张商英的弹劾是卡准了时机,赶在郊祀之前下手,冯京作为宰相只能避位,为了保证朝廷三年一度的大典能顺利实行,只能换一个宰相了。

当然,天子其实也有将张商英踢出去,保住冯京的选择。但御史中丞邓润甫带着一众御史紧跟着开始弹劾冯京,这样一来,赵顼总不能为了冯京将御史台给清空掉——王安石能有这份量,冯京可不够资格。且在郊天大典上,御史们的工作很重要,所以也只能让冯京走人——换一个总比换许多要好。何况冯京被弹劾的罪名一时半会儿也辨析不清,赵顼没时间为他耽搁。不过换了吴充上来任职,恐怕天子也有对吕惠卿等人不顾大局的愤怒在。

苦恨年年压金线,为他人作嫁衣裳。吕惠卿实在是心急的过了头,须知欲速则不达。

韩冈暗叹了两声也就罢了,反正这不管他的事,他可从来没想过要与吕惠卿混一边。

只是在帮韩冈穿衣服而已,周南和云娘都忙得额头出汗,王旖也不跟韩冈说话了,眼睛上上下下打量着,仔仔细细一点也不放过,确定衣裳的穿戴没有问题。

系上了腰带,接下来就是各色配饰。韩冈散官是从五品下,但他被赐三品服色,装束上等同于三品官。公服是紫色袍服,而朝服也是三品一级。银剑、玉佩、银环一一配上,周南又拿了一条狮子纹的锦绶给韩冈系在腰侧,一直垂到膝边。

云娘捧着一顶.进贤冠,让韩冈坐下来后,给他带上。王旖也走上来,将冠冕挪得端端正正的,然后才插上了长长的犀角簪,将进贤冠和发髻给别上。

头冠、衣裳、配饰都穿戴好了,下面还有鞋袜。白色罗袜,黑色的木底皮靴,这也是朝服的一部分。

外间的房门被推开,去了小厨房的严素心带了两名粗使的丫鬟,碰了几盅冬日进补的药汤、还有给韩冈的早饭进来。

韩云娘和周南蹲着身子帮韩冈穿着鞋袜,而严素心捧着药汤先给了韩冈,接着又给了王旖一盅,两名使女也将早餐在桌上摆好。

两名美妾终于穿好了鞋袜,俏生生的站了起来。韩冈抿着滚热的汤水,如果他愿意,他的妻妾都可以帮着吹凉了,喂到他的嘴边。生活起居,任何一项都能有人服侍到最细微的环节。这样什么腐朽糜烂的生活,在后世能享受的是凤毛麟角,而在这个时代,却是十分平常。

换好了朝服,吃过了严素心精心烹制的早餐,韩冈就要启程去参加正旦大朝会。临行时,也不忘向妻妾道别,顺便还提醒了王旖早点去休息,“午后你也要入宫去,得养足精神。”

王旖也有诰命在身,靠着韩冈得了个郡君的封号。等到下午,她就得换上外命妇的服饰,去宫中拜见太皇太后、太后和皇后,这是免不了的繁文缛节。

“奴家知道了,官人就放心上朝去好了。”

王旖屈膝福了一福,与周南、素心和云娘将韩冈送了出去。

韩府的大门中开,一队骑手从院中鱼贯而出,向着宫城的方向过去。

