\section{第39章 苦心难成事(中)}

王雱请王安石听韩冈的劝告,王安石却是皱眉不语。他要是能这么容易就动摇,就不会被称作拗相公了。

韩冈心中叹了口气,这个时候,只能直截了当的将些不中听的话说出来了:“小婿敢问岳父,如今天子对岳父的信重,可比得上熙宁初年?”

王安石现在面临的问题,并不是放到台面上来的天子、宰相对辽态度的分歧,而是他能坐在宰相位置上的信任基础的正在瓦解。天子对宰相的谏言充耳不闻,其实并不是稀罕事。没有哪个皇帝会是宰相怎么说,他就怎么做。

可是如今这等事关宋辽两国国家关系的重要议题上,天子一意孤行,视宰相的意见而不顾。从王安石这边的角度来看,说的绝对一点,其实已经是在逼着他辞相了。

要不是看到了这一点苗头,那一干元老重臣,也不会如此肆无忌惮在奏章中胡说八道了。

王安石面沉如水,默不作声。灯花噼噼啵啵的一声声的爆着,韩冈和王雱静声等待他的回答。最后房中的静默化作颓然一叹:“只从得五分时也得也!”

熙宁初年做着宰相的曾公亮,曾被苏轼责备其‘不能救正朝廷’,他当时回道:‘上与安石如一人,天也。’

那个时候,天子对王安石差不多是言听计从,视王安石如师长。就算熙宁二年对新法的反对声到了最高潮,赵顼也因韩琦的奏章而犹豫不定的时候,王安石只用了一个告病不起,就立刻让天子明确了立场。

可是现在呢,别说五分了,赵顼对王安石的信任,能有过去的两三成,就不会出现如今的局面。

王安石过去做过的事,现在却无法再重复一遍。再想告病不起,以用来要挟天子回心转意。赵顼纵然会优加抚慰,但他心底里对王安石的成见,也只会更加深一层。

看着灯下王安石在疲惫的老态下依然紧抿的双唇,韩冈知道他的岳父绝对不甘心就此离开东京城。以他的脾气,那是非得要碰个头破血流不可。

可如今在相位上多留一日,日后复相的机会就会少上一分。趁早抽身离开,才有卷土重来的可能。

“已经不是熙宁初年了。”韩冈平静淡然的声音,仿佛有打碎幻想的魔力。比起王雱这个儿子,作为女婿的韩冈说话可以更为直接一点,更加不留余地。

此事木已成舟,很难再有挽回的余地。越是拖延下去,王安石的地位就越危险,说不定就有一天,连吕惠卿、章惇等人都要将他给抛弃。

新党作为一个政治集团,几年间已经逐渐成型。虽然在士林和朝堂高层中还比不上旧党的势力,可底层官员对新党的支持率却是不低。而且在天子不可能放弃新法的情况下,新党也不可能被赶下台。这时候,不再受到天子信重的王安石很有可能会被他的门生们给抛弃——只为了不影响新党本身的利益。

王安石的双手不由得攥紧,腰背不服气的挺得更加笔直,但他神态中透出来的颓唐却怎么掩饰不了。

离开相府的时候,已是深夜。虽然最终王安石也没能给个明确的回复,但韩冈相信他的岳父会好好考虑这件事的。

再怎么说,在郑侠上流民图的那段时间,若是处理不好,王安石就已经不得不辞相了。如今已经拖了半年的时间,新党因曾布造成的变乱也已经初步平复下来,这时候离开,没人能说他是因罪辞任,在新法的施行上,也不会留下后患。

……而且还能将在割地失土的罪过在天下人面前分说个明白,眼下的时机不好好掌握,接下来可就没有这么好的机会了。

王雱亲自送了韩冈出来。

相府中的石板小道上,两名家丁提着灯笼在前面引路,韩冈和王雱在黯淡的灯火下并肩走着。

“多谢玉昆了。”王雱开口轻声的说道。

韩冈摇摇头:“其实岳父心中应该已经有数了,小弟也只是挑明了而已。”

王雱脚步变得重了一点。

大宋开国以来,没有一位宰相能一直坐在相位之上,即便是有从龙殊勋的韩王赵普,也是几上几下。要说王安石父子没有想到会有这一天,那当然不可能。只是当年意气风发的时候,怎能想到天子的信任会这般快的烟消云散。只要有天子支持,就算有再多的人反对,王安石也能坚持着将新法推行下去。可若是失去了天子的支持,王安石绝对抵挡不了旧党的攻击。

“事已至此,只能徒唤奈何。”将韩冈送到相府门口,王雱最后叹道。

韩冈借着大门前的灯笼,看着大舅子的脸色。即便是在夜幕下,也掩不住王雱脸上的憔悴。在他的嘴角处,还有心急上火憋出来的燎泡。王雱的身体一向不好,一年总要生个几次病,韩冈有些担心,说着:“元泽,你最近的气色好像不太好啊。你也别太操心了。”

王雱笑了笑,神态忽然间变得洒脱起来:“京中事了,愚兄就陪大人出外。那时候,便可以游山玩水,忘却尘俗烦忧。再也不用为朝堂上的事情头疼了。”

韩冈笑着摇摇头。以王雱的性格,怎么可能安居在外。恐怕休息个两天,就要竖起耳朵听着朝堂上的动静,过个半年就要设法开始撺掇王安石复相了。

这并不是说王雱的利欲熏心,而是在朝堂上掌控政局的快感,是在京城之外的州郡里治理百姓远远比不上的。王雱从来都不是安于野逸之辈,这一点,韩冈如何能看不出来。

“对了,”韩冈突然想起了什么,“有件事还是要提一下。不知元泽能不能转告岳父。”

“什么事?”

“越是丑事,越不愿听人多提起,这是人之常情,还望元泽能多劝一劝岳父。既然木已成舟,在天子面前,还是不要多提弃土之事。否则恼羞成怒,反而会多上许多不应有的后患。”

“此事愚兄如何不明白。”王雱微微苦笑,他和韩冈都是能经常见到皇帝的近臣,知道所谓绝地天通的天子也只不过是个普通人而已,若是一个劲听到有人在耳边说起自己过去犯下的错事,一开始也许会悔过,但时间长了,次数多了,就绝对不会再有什么虚心纳谏的想法,而是会激起逆反心理,“只是父亲能不能做到,那就两说了。”

赵顼一直以来都是想着要做个比拟唐太宗李世民的明君,现在他却在契丹人的压力下,割让了河东的土地。不管割让的土地多寡,这都是仁宗朝都没有做过的事。以赵顼的性格,等他事后回过味来,必然要悔不当初。这时候若再有人一个劲说他犯下的蠢事,那事情反而会向期待之外的方向偏离。

既然在人家手底下做事,就不能不考虑赵顼本人的心理承受能力。没有换东家的可能,也有着日后重新来过的想法,王安石最好的做法,就是不再天子面前提及此事,而是告病离去。

离开了相府,韩冈第二天,就离京返回白马县。

在他的地盘上,韩冈一边处理着政务,一边竖起耳朵听着京中朝局的变化。也不出他的意料,王安石那个拗相公还是在苦劝不已。

且不仅是王安石,吴充、吕惠卿等一干身居朝堂之上的臣子都没有一个支持赵顼。理由很简单,一旦割地失土,毁了名声的只会是他们这群实际掌握朝政的臣僚,那些元老重臣绝不会受到半点牵连。

吴充作为枢密使,给赵顼鼓劲:“周世宗拥一旅之众,犹兴兵抗虏。”

可惜赵顼却说着:“五代之国,乃盜贼之大者,所以不惜其命。今日兴事,须是万全,岂可不畏?”

吕惠卿在旁帮腔:“陛下所言诚是。但譬如富者自爱其命,贫者不然。未必小国便不亡,为政须计较利害尔。为天下不可太怯弱!”

天子则回道:“契丹亦何足畏,但誰办得用兵?”

谁也不敢拍着胸脯说一定能将契丹铁骑阻挡于国门之外,即便有人拍着胸脯,也要赵顼肯信。

当赵顼对朝堂上的反对之声全然不顾,又亲下手诏给负责谈判的韩缜,威胁道:‘朝廷已許,而卿犹固执不可,万一北人生事,卿家族可保否?’王安石终于放弃了劝说,上表请辞相位,遂了许多人的心思。

辞章初上,赵顼便当即驳了回来。接下来的半个月,辞章开始在相府和崇政殿之间来回往返。但世人都很清楚,王安石此次辞相,已经再无挽回的余地。

从熙宁初年,新法逐步实施,到如今的熙宁七年将尽,六七年间,大宋的国力的确在一步步的强盛起来。换作是仁宗、英宗之时,绝无可能在西南、西北以及荆湖同时开战,并且卓有成效。即便算上熙宁七年的旱灾,王安石向赵顼交出的答卷也远在合格之上。

但终究会有曲终人散的一天,熙宁七年十月初五的这一日,王安石离开了政事堂,离开了宰相之位。

