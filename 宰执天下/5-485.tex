\section{第39章 欲雨还晴咨明辅(14)}

耶律乙辛知道自己的弱点,也知道怎么弥补。

尤其是在张孝杰带回了韩冈的传话之后,他就更清楚了。

刚刚结束了一场失败的战争,不待喘息,便掉头东向,去攻打海东大国高丽。

大辽的铁骑不费吹灰之力就击穿了高丽北界的千里长城,又飞快的攻破了西京东宁府,也即是平壤,然后继续南下,包围了高丽王都开京。

胜利得来的轻而易举,辽军的攻击势如破竹。

在包围开京的过程中,作为先锋的万余铁骑经历了两次野战,一次是正面遭遇高丽军,另一次是天亮前的突袭。依靠坚实的甲胄,精良的武器,以及优秀的战术,辽军以微小的代价,击败并歼灭了高丽王国最后一支进攻力量。

当高丽人不得不选择龟缩,从宋人手中得到的神臂弓、破甲弩,以及霹雳砲,就发挥了巨大的作用。

几次与宋人交战,尤其是突袭河东的会战,在河东武库中,缴获的弓弩刀枪的数量巨大。让耶律乙辛可以给主力大量装备神臂弓。但战场上的损耗也同样巨大,不懂得保养,不懂得使用,神臂弓的使用寿命在契丹人手中大幅降低。损耗量让耶律乙辛看了回报都心惊肉跳。

在大辽尚父的手上,并没有能够大规模制作神臂弓的作坊。而小批量的生产,制造成本又实在是太高了。

也不知道宋人是能够降低成本,还是根本不在意成本。总之只有一点,宋人造得起,用得起,而大辽却造不起,也消耗不起。

幸好在所有的弩弓都消耗一空之前,宫分军已经将高丽国都周围的据点全数拔出,彻底的包围了开京城。

高丽国王王徽临时禅位给世子王勋,并修书遣使向耶律乙辛请罪,同时献出了与宋国通好之后,由宋人那边送来的十几道国书。希望以此能够平息大辽尚父的怒火。

‘贡事不谨,为臣不恭。私通他国,心怀悖逆’,

这十六字的评语,是耶律乙辛所找的借口。王徽想用嘴皮子来熄灭契丹铁骑杀戮劫掠的欲望,这当然就成了辽军中最新的笑话。

完全没有理会城中百姓的安危,辽军用了五天制作霹雳砲,然后就用了半曰功夫打破了城墙,攻进了城去。

高丽国王带着百多名臣子逃到了江华岛上。那座小岛距离岸边虽近,却还是隔着宽约数里的海水,不擅舟楫的辽军也只能望洋兴叹。

不过逃掉的也只有新任的高丽国王王勋,自王徽以下,宗室、贵胄、朝臣的家中子女、下人,各色人等数以万计,尽数被俘。如今正在被押解北上——这是耶律乙辛在开战前的要求。

从开始南下,到攻破开京。攻打高丽的辽军只用了二十多天的时间,其中还包括在开京城下的五天等待。这样的破敌速度,就是号称知兵的张孝杰也为之瞠目结舌。

再怎么说,高丽都是此前百年,大辽几次三番没能征服的对象。

百多年来,辽丽交兵三次,每次攻打,高丽国王都是很快降顺,然后再锲而不舍的向北偷窃土地。一点点的蚕食,一直将手伸到了鸭绿江口附近。

这等狗盗鼠窃的国家,一直像烂泥一样不怕人踩,表面上恭顺,私下里就是个贼。只是这只贼很棘手,手中又不乏强兵,一直以来,辽国君臣都没办法下定决心去解决高丽的问题。

直到今天,耶律乙辛要化解军中的愤怒,并给予所有支持者足够的好处,在强攻南朝未果,甚至可以说是惨败的的情况下,不得不选择拿高丽开刀,证明自己实力,并让南下的士兵尽可能的获得更多的好处。

战争开始了,然后就结束了。

张孝杰也不知该怎么评价高丽的实力,因为他之间接触过的东京道守臣,都在说高丽是千里大国,不是那么容易就能征服,岂料事实完全不是这样。

“太弱了。”他也只能这么评价。战争的发展完完全全脱离了他的预期。

“不是太弱了,是大辽变强了。”耶律乙辛冷静的说道。

张孝杰明白了,也知道为什么耶律乙辛说这话时还绷着一张脸。

因为宋国更加强。

大辽的士兵与大宋交战始终不顺,甚至有丢盔弃甲,丧师辱国的例子,但那个原因,只是南朝变强得更多。

带甲百万。

过去一听,就知道是吹牛的数字。现在则是即完全可能,不过是耗费些手工和时间。

如果南人的皇帝当真想要,两百万、三百万套板甲,也不是不可能弄到手。

这是钢铁产量上的绝对优势,使得铁甲的原材料不再束缚于铁料不足。

耶律乙辛曾经听人说过,南朝已经不再按斤来计算钢铁产量,而是用石。

这一换用,具体到数字上,就是减了百倍。但南朝的钢铁量,还是轻而易举就在百万之上。

精铁百万石!

这是何等可怕的数字!

一副甲胄即便装具配件齐全,最多也不过三五十斤重,一石少说都能抵两件。南朝的朝廷只要拿出每年钢铁产量的一半来打造板甲,那就是一百万套了。

一年一百万,两年呢,三年呢?

就算这个数字不算精确,打个折扣,年产量也至少能达到五十万石。

无论如何不可能再少了。再往少里算,就是自欺欺人,耶律乙辛还没有那么幼稚和愚蠢。

耶律乙辛听出使南朝的使节回来禀报,开封城镇曰被黑烟笼罩,煅烧石炭的炉子,熬炼精铁的炉子,在东京城外一座接着一座,如同树林一般。

而如此规模的钢铁树林,在徐州据说还有一片。除了开封、徐州,南朝的其他路份,也都有大大小小的铁场。如果全力生产,到时候说不定能用钢铁将御道都给铺上。

越是知道自己国家的细节,越是没有信心。这样的国力,大辽卯足了力气也追赶不上。

除了另起炉灶,耶律乙辛也没有别的办法了。

只要能够掌握高丽的水师,只要能控制好他们。大宋东方海疆,就是他们出没的地方。或许高丽水师的真实本事不行,至少可以让宋人如此认为。反正宋人的海上水师,到现在还只存在于港口中。

而留给南朝君臣的选择,除了维持和议,想来也没有别的办法。海上的战场不是那么容易开启,而整顿现有的水师力量,也不是那么容易的一件事。就算宋人能够成功,也在几年之后了。到时候若是打不赢,直接就撤退好了。高丽的未来,可以再做议论。

不知道那个在河东击败了大辽的年轻宋臣,在得知高丽被灭亡,水师又落入大辽之手后,会有什么样的反应。

以韩冈的年纪,只要不出意外,他必定是南朝未来的主政者之一。而南朝向外扩行的国策,也必然会在他手中实现。到了那时候,以耶律乙辛所了解的韩冈为人,怎么可能会看着中国继续保持南北分立的局面?

皇后很信任他,太子也要依靠他,自从南朝的皇帝发病之后,韩冈的地位就变得更加重要。

这样一位地位重要,又有才干的敌人,他的存在就是一种危险。如果有可能,最好将之扼杀在最开始。可是现在已经迟了。

耶律乙辛曾经想过一咬牙,干脆派人去南朝境内播些谣言,好动摇韩冈的位置。但始终没有把握,最后的还是个耶律乙辛放过了。

谣言是个好手段,只是要看人、看时机。

对于韩冈这样的重臣来说,谣言有用,也没用。

对于有关下属的谣言,君王不是相信,而是想相信,需要相信。有那个需要,确信谣言有利与他,这样他才会去相信谣言。

就如那位从天上摔下来的宣宗皇帝耶律洪基。

因为皇后通歼一事,耶律洪基对太子耶律信本来心中就有了嫌隙,再加上早年皇太叔耶律重元的叛乱,使得耶律洪基完全无法信任自己的至亲。这样的情况下,只要一句谣言,就把耶律信从太子的位置上给掀了下来。

不是耶律洪基相信了谣言,而是他选择了谣言去相信。

回到南朝,不管再怎么散布有关韩冈的谣言,只要南朝太子还没有成年,皇后就不敢怀疑有护佑幼子之力的韩冈。而韩冈在南朝百姓中的声望,也不会有人去相信那些不实之词。

在皇太子成年前,韩冈的地位可谓是稳如泰山,任何谣言都动摇不了他的地位。若是让他得知了谣言的真相,对大辽来说实在是得不偿失。小动作若是做得太难看,惹怒了韩冈之后,结果可想而知。

耶律乙辛不喜欢无谓的战争,只想看到最后的好处,他的目标,无论宋辽,了解的人所在多有。他也不打算隐瞒什么。

只要能让他实现那个目标,坐上那个位置,没有什么不能商量的。

和议如此,国土如此,当然,现在刚刚攻下的高丽国同样如此。

想到这里,耶律乙辛长身而起,“传令开京,继续南下,追敌至天涯海角。高丽国……没有必要再存在下去。”

