\section{第33章 为日觅月议乾坤(十)}

“吕宣徽。”

东京车站的提举官在吕惠卿面前点头哈腰。

方才吕惠卿下车时,他就已经带着笑脸迎在车门口,现在脸上的笑容更盛,“驿馆已经准备好了,还请宣徽移步。”

吕惠卿向西南方向望过去,车站的围墙外不远,有着一片建筑群,雕栏画栋、飞檐斗拱,掩映在花木和围墙中,“是青城驿?”

如今铁路大多汇聚京师,过去走京师算绕远的路线,现在全都要经过这座车站。

来往官人的数量陡然增加,故而朝廷便决定如果官人只是过境,就不安排去城中馆驿居住。并直接在车站旁,修了一座专供官员及其家眷居住的新驿站,以南郊圜丘所在的青城为名,号为青城驿。

提举的笑容变了一变,转瞬又恢复,“宣徽若要入住城南驿,下官这就派人去通知他们准备。”

“不必了。”吕惠卿不出意料的在他的脸上找到一丝喜色,心下冷笑,“我久不入京,这番移镇,当觐见太后、天子再去赴任,不过家人就不必入城了。”他回头,对身后的儿子吩咐道,“你们就先在这边的驿馆住下吧,不要乱走动,为父带几个人进城就好了。”

虽然这一次回来,吕惠卿并不打算就此离开,但京师便是龙潭虎穴,多少豺狼虎豹在城中等着要咬上自己一口,等于是在独木桥上走,他可不打算给人留下半点把柄。

让两个儿子带着全家去安顿,转身找上笑容已经变得僵硬的提举,让他安排马车,吕惠卿就带了四个伴当,以及一点行装,就这么轻车简从的,径自往城中驶去。

几年未至,京师的变化已经让吕惠卿有了一种沧海桑田的感觉。透过马车车窗向街道两边望去,这种感觉越发的浓重起来。

城南这里原本一向拥堵,每天都有成千上万头猪穿过南薰门进入城中的大小酒楼,乃至千家万户千家万户。行人、车马和牲畜络绎不绝,加上是不是有官员仪仗出入,往往只是进出城门,就要费上一两刻钟。而如今城南更是有了铁路车站,进出的行人车马就更多了。但直到窗口陡然变黑,片刻后又忽而变亮,吕惠卿才陡然发现,他进出城门时竟然没有堵车。

是靠右行驶的功劳?

早两年,吕惠卿就知道京师颁布了一系列的新规矩。

在韩冈的指挥下,开封府利用将京师乞、盗之辈一网打尽而得来的威信,大力整顿京师秩序,甚至连行路都给管了起来。行人车马都是靠右行驶,如果要停车下马,必须靠边。

当初,官员路遇,有避道之仪。如果是遇上宰相在道路中间走,大小官员更是都得让道路边去。但依据新规,即便是宰相出行的仪仗,也是靠右行进。这样做,当然有违礼仪,有损官宦的威仪,可章惇、韩冈自己都主动如此,下面的人还有谁能说?

如此严令,实有潜移默化之功,可以让百姓循规蹈矩,与之前打击丐盗的行动可归为一体,吕惠卿本来也有仿效的打算,但想想还是没有去做,一个面子上过不去,朝廷也没有下令,二来,以长安的交通情况也没有必要这么做。此外,他并不觉得,当真会那么有效。

但今日看来,这样的法规推行下去之后,京师的交通的确从此变得不再拥堵。

从方才起,挫败感一直缭绕在心头,不过吕惠卿心中的斗志也是愈加旺盛。

韩冈秉政七八年,国虽大治,基础依然是建立在自己辅佐王安石所推行的新法之上。如果自家有机会秉政,在韩冈、章惇的基础上,他同样可以做得更好。

辽国……

似乎韩冈、章惇都忘掉了。

吕惠卿思绪起伏,但车窗外掠过的人影让他猛然惊醒,

“停车。”

吕惠卿大声喊道。

马车刚刚靠边停稳,他便推门下车,走近街旁,“仲元,你怎么在这里?”

……………………

“玉昆,你怎么对太后提起吕吉甫的事?”

韩冈在会议后被太后单独留了下来,章惇焦躁不安的等待着。一见韩冈回来,便急匆匆的上来询问。

“必须要提的,不是吗?”韩冈反问。

天子即将大婚,吕惠卿此番过境京师,必然要在殿上闹一闹。

这是章惇前日与韩冈议论吕惠卿上京事时对他说的,说得斩钉截铁,说得信誓旦旦。

对吕惠卿会做什么,韩冈可没有章惇的把握,当时就感觉,难怪说最了解你的只会是敌人。

耶律乙辛在辽国,兴工贬儒,声称有工无儒国亦大兴。又说韩冈之学,格物之要,便是弃儒重工。

尽管在明面上,韩冈将耶律乙辛的言论嗤之以鼻,甚至连驳斥都不屑去做。就是有人当面去质问气学门人,也只会得到不屑的一瞥。但实质上,不得不说,耶律乙辛看得很准。所谓气学,全然是挂羊头卖狗肉。

而章惇将吕惠卿看得如此深刻,自然是将之视如寇仇,绝不希望其有机会重新回到朝堂之上。

故而章惇希望与韩冈联手,若吕惠卿当真在觐见时闹起来,就趁此机会让他彻底断送回朝的前途。

当然,要是吕惠卿不在殿上闹起来,那就更好,继续让他在名城要郡之间来回任职。

至于写奏章什么的,那更是不用放在心上。便是写上一百封奏章,章惇、韩冈也能压得下去——天要冷了,政事堂的暖炉有得好柴烧。所谓宰相,当皇城司都俯首帖耳的时候,就是沟通内外的唯一通道。隔绝中外这种小事,做起来根本不费吹灰之力。

韩冈当时是答应了章惇。

他跟吕惠卿又没交情,这几年私底下也颇让人闹心,拿着他当人情,韩冈有什么不愿意的?

但韩冈给章惇的承诺,可不包括主动在太后面前先一步下眼药。这就不是帮忙,而是赤膊上阵了。

章惇不明白,韩冈这又是想做什么?

“玉昆,你是怎么对太后说的。”

“一旦吕惠卿在殿上要让陛下归政天子,陛下若是依然让其就任京外,不免有吕武之议,若是留其在京,必然会聚集起一批郁郁不得志之辈,大肆诽毁朝政。”

这是章惇之前对韩冈分析的话,竟被韩冈转述给了太后。

这不是韩冈要抢功劳,而是韩冈替章惇分担日后来自朝野的攻击,挺身为章惇作掩护。毕竟最不希望吕惠卿回京师的不是韩冈,而是章惇。

章惇看着韩冈那张平静的面容,真希望自己有个他心通的本事,能将他的五脏六腑给看个通透。

“那太后又是怎么说的?”

“‘相公想让吾怎么做?’”

“玉昆……”

章惇都没力气了,韩冈就不能一口气说完吗。

“此事陛下心知便可,免得届时猝不及防。吕惠卿才识过人,熙宁时便已入政事堂,如今久在外郡,自是心生不满,希望朝中有变,得以重回两府。”

此乃诛心之论。

吕惠卿如果当真提到撤帘归政之事,在太后的心目中,立刻就成了无法信重的小人。而此前,纵然比其他宰辅疏远,至少也是可以放心让其镇守要郡的重臣。

“但这一切的前提,就是吕吉甫会在觐见时,提起撤帘归政之事。”韩冈说道。

“玉昆,我之前也说过了。吕吉甫其意在天子,而非太后。又远离朝堂多年,急需声望。即使他明知我们会在太后面前说他是非,他也绝不会避让。这一次,是他唯一的机会。”

“那太后在一日,吕惠卿便得外任一日,再无机会返京。”

“等到天子亲政,他便是宰相第一人选。”

韩冈轻轻摇头,“那他有得等了。”

方才向太后还在殿上叹息,‘只要官家成才了,吾便撤帘归政。垂帘听政,说起好听,做起来有多累,又有谁知道?’

太后虽是叫苦,可官家若是不成材,他亲政之日依然遥遥无期。

成才与否,谁来评价?

只要太后垂帘下去,她和天子之间的裂痕将会越来越深。对那些支持天子亲政的官员,自是会越来越不待见。

当朝臣们明白了这一点,在吕惠卿成为天子一派的赤帜后,朝臣们就必须要在双方之间选边站了。在人心混乱的时候,统一思想——或者说整风——是必不可少的。

谁是敌人,谁是朋友,都要在这一次区分开来。

韩冈对此有着清楚的认识,而章惇同样明白这一点,方才在殿上请求太后继续垂帘的宰辅们,都明白这一点。

天子大婚在即,已经容不得人再暧昧下去了。

接下来,太后什么都不需要做,自然会有韩冈、章惇等宰辅冲杀在前。

“你我当先一步做好准备。”章惇说道,“朝野内外都得有所准备。”

未来的压力,将绝不仅止于朝堂。

“那就给他们添点乱子。”

韩冈命人拿来纸笔,开始在上面写字。

他并不在意士林中的评价,好也罢,坏也罢,都不会影响到他在民间的声望。但在士林中有个好评价,总比坏的要强。

“童生,秀才,举人,进士。”

韩冈写出来的八个字,章惇只看了一眼,便心下了然。

秀才是对读书人的尊称,相对于贬低的措大,而举人,自是贡生。加上之后的进士,前面的童生就很好理解。四个词联系起来,便是一条路,一条读书做官要走的路。

但韩冈特意写出来,自然用意更深。章惇抬头,“这四个有何用意?”

“阶级。”韩冈极为简短的回答道。

章惇脸色陡然一变,“玉昆,你可知道,你一旦这么做,可是要得罪所有北方读书人!”

“放心,”韩冈笑道,“这怎么可能会不考虑到?一是一,二是二。”

“……那还有用吗?”

“当然,只要有足够的好处,或许收买不了一个人,却肯定能收买许多人。”
