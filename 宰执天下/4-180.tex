\section{第28章 遥别八桂攀柳枝(下)}

韩冈在他担任广南西路转运使的两年日子里,于桂州城中逗留的时间很短,但这并不妨碍他在桂州得到全城百姓的人心。

韩冈回头望了望送他出城的千万生民,人潮如山如海,仿佛是上元夜的灯市。这些桂州百姓,并不是被官吏强迫着出城来,而是听说了韩冈离任之后,主动出来相送。

他作为转运使,不算是亲民官,并不直接接触百姓,而且两年来先是领军作战,之后又多是留在南方,本来是不可能得到万民相送的殊荣。

但他刚刚抵达广西后的胜利,不但将交趾侵略军打了回去,也让桂州内外的官民放下了一颗战战兢兢的心。之后又是与章惇一起,将交趾灭国,从今而后,广西不用再担心听到交贼入寇的号角。另外在李常杰领军入侵时,桂州派出去的援军全军覆没于昆仑关附近,韩冈为他们报了仇,他们留在桂州的家属,对韩冈自是感恩戴德。

桂州城中的大小官吏倾城而出,他们身后是人山人海的桂州百姓,而被推举出来的几名乡绅父老,住着拐杖来到韩冈面前。

万民伞的风俗还没有流传开来,但脱官靴以表离任官员遗爱一方的节目,这时候已经有了。几名父老跪在韩冈面前,让他将脚上的靴子给脱下来,留给桂州城。

韩冈将他们浮起来后,照规矩谦虚了几句,推脱了一番。一个老家伙髙声说起来,“韩龙图为官一任,造福一方。征讨交趾,使广西生民自此永享太平。又有德政遗爱一路,我八桂中人,无不感念在心。”

韩冈觉得这话说得很是中听。他在广西两年,主要的精力都是放在剿灭交趾国上,不过他在广西一路的德政也不少。

桂州、邕州、交州等几个路中上州,州学、疗养院,都建立了起来。还有负责埋葬无名尸的漏泽园,自邕州埋了数万尸骸之后,韩冈也顺势在邕州设立了一座,此外交州也有。同时,又有收养无儿无女的孤寡老人的福田院,旧时只有京城中有,但如今在邕州和交州都设立了。

这些公共设施,花销都不少,而且是要常年付出。韩冈也只有趁着邕州、交州人少地多的情况,能专门划拨出官田来为此提供资金。

如果是一般喜欢邀风赏月的官员,只要府库中有些闲钱,多半就会造些无谓的建筑,或建楼,或建亭,以供人游玩——自然,有闲情雅致的不会是家中无隔夜粮的普通百姓——倒是出过一些千古名篇,岳阳楼、醉翁亭,让后人传唱。

只是韩冈不擅诗文,对此也毫无兴趣,他治政的目标是德惠百姓,做得多是有关生老病死方面的事。

从百姓的角度来讲,这应该算是他留在广西的最大的德政了。

韩冈洗耳恭听,就见那老家伙说道,“龙图为救一路百姓,下令禁绝槟榔,这一事,德惠万千生民,善莫大焉。”

‘槟榔?!’

韩冈身子一颤,一股子啼笑皆非的感觉涌了上来。他的确是反对嚼食槟榔。自到了广西之后,看着人人口中殷红如血,地上一滩滩红色如同血痕,韩冈个人很是反感这样的习俗。

俗语说‘路上行人口似羊’,嘲笑的就是两广之民,说他们不停的咀嚼着槟榔蒌叶和蚬灰的样子,就像不停嚼食草叶的羊一般。

民间有传言,说是嚼槟榔能避瘴气,能驱虫、消食、化痰,但韩冈觉得,良好的生活习惯比槟榔要管用得多。多食槟榔会毁掉牙口,还容易上瘾,片刻不吃就会觉得口舌无味,另外随地乱吐汁水也会有着卫生方面的问题,对身体健康带来的害处远远超过好处。

而且更为重要的是吃槟榔吃成习惯后,一户人家每天都要有十几文乃至几十文的额外花销,对于普通百姓来说,这就让他们根本存不下钱来,对于灾害、意外和疾病缺乏足够的抵抗力,一遇灾年,就只能成为流民。这个问题,比口腔健康更严重。

所以韩冈自从到了广西,一见嚼槟榔的恶习猖獗,就严令禁止军中入乡随俗的嚼食槟榔,需要药用时,则煎水服用。甚至还找了几个因为常年吃槟榔,牙口全都坏掉的人,在全军面前展示,用以警告。

另外还有一次,就是刚刚赶走了李常杰,重建邕州的时候,他还将在军营外转悠的槟榔小贩抓起来的打了二十板子,然后分了土地给他们,让他们好生的种地过活。

韩冈是传说中的药王弟子,既然他说槟榔对人体有害,相信的人还当真不少。就这么一番软硬兼施的手段下来,至少明面上,广西诸州嚼食槟榔的现象大减。虽然不知道日后会不会复发,但放在眼下,的确可以算是一个德政。

只是为了这一件事对自己感激,特意在千万人前正经八百的说出来,韩冈却当真有种哭笑不得的感觉。

“不仅仅是槟榔。”另外一个心思活络的过来打着圆场,“龙图至广西后,收治百姓甚多,又推广避疫之法,让人知道该如何治病防病。两年来,广西未有一次稍大一点的瘟疫,此皆是龙图之功。”

桂州的父老代表恭恭敬敬的退了开去,手上托着韩冈刚刚脱下来的官靴。韩冈换上了一双新鞋子,又是一人端着一杯水酒上来,之后还有一人折了柳枝来送……

走完一套流程,将自己的官靴留在桂州,韩冈领众启程。

他毫不犹豫的上马动身,将数以万计的百姓留在身后。

过去两年在广西的生活让他难以忘怀,而即将到来的新生活,则是让韩冈心中期待不已。

……………………

“京西路都转运使……”吕升卿头靠上椅背,“想不到京西路一分为二才几年,现在又合并了。”

“那是因为天子要让他开凿襄汉漕渠。”

韩冈的新职位是将京西南路和京西北路合并而成的京西路都转运使。

方城山是京西南路、京西北路的界山。如果想要开凿襄汉漕渠,由汉水直通京城,为了方便起见,最好事权同归一人,故而天子将草拟的京西南路都转运使改为京西路都转运使。

京西南路、京西北路在五年前还是一路——京西路,不过就在熙宁五年便一分为二,如今重新合二为一,也不会让人觉得不习惯。

其实换一个角度,安排一个临时性的职务也可以。但临时性的职位,任务一旦完成,就可以回京了。到时候,想将立有大功的韩冈再踢出朝堂去,从情理上根本说不通,同时也会让人感到心寒,还不如就让,即便襄汉漕渠完工之后,他也可以一直留在京西。

“韩冈选了一个能讨巧的好题目。”吕升卿翻着兄长带回来的资料,突然间就冷笑了起来,“当初的沟渠都已经挖好了,也通了水,就是方城山那段实在太浅了而已。韩冈到了京西之后,只要将方城山那一段着重开挖,再掘深个几尺,差不多就能将河渠给开挖出来了。”

“若是当真这么容易,怎么会没人去考虑过?”吕惠卿方才已经将弟弟手中的资料看过了一遍,比起一目十行的吕升卿看到了更多的细节,“那沟渠中的水,是方城山上下来的溪水,不是用堰坝提高水位后的回水。根本浮不了船。”

吕升卿再仔细一看,果然是如此。

就听吕惠卿继续道:“韩冈是打算将荆襄到京城的交通线给打通。如果南方的纲运能从江汉之地直入开封,这等于又多了一条命脉,功劳比起平灭交趾,还要大上数分。”

在这之前,汴河的运力已经开发到了极致,雪橇车出来之后,连冰雪覆盖的冬季也可以运送货物。但东京的安危全都放在汴河上,这毕竟不保险,汴河也经常淤积,河中的泥沙已经让行驶在河上的船只,比起堤外的房屋还要高出许多。一个不好,就是京城内外变为泽国。如果能再有一条来分流,自然是能让人放心很多。

“每年六百万石粮纲。”吕惠卿屈着手指计算着,“只要这一条交通线运力能达到汴水的三成……不,两成,五分之一,就算成功了。”

“只要一百二十万石?”吕升卿惊讶道。

“一百二十万石,算多一点,一百五十万石。已经足以让天子满意,搪塞住悠悠众口。南阳的气温比开封稍暖,能保证三百天的通行时间。韩冈只要每天运送五千石纲粮入京,就算是他赢了。”

“五千啊。”听吕惠卿这样一算,还真是不算很多。一艘福建的中型海船,要装下五千石的货物,只要两趟而已。

“不,还有一点别忘了……”吕惠卿忽然又说道。

“什么?”

“运费,运费一定要便宜。若是价格太高,就失了纲粮的本意了!”

“大哥你觉得韩冈他到底能不能做到?”吕升卿问道。

“若无把握,韩冈不会说出来,这是好事。虽然有人不这么认为。”吕惠卿笑道,“韩冈之行事,无论是在关西还是在广西,都是尽量不动用民夫……”

“他在白马县可不是如此。”吕升卿插话道。

“那是以工代赈,赈济灾民用的,不能归于一类。”吕惠卿说着,“韩冈行事,一贯如此。但这一此开凿襄汉漕渠,沟通蔡河,直达京城,就不可能不征发民力。其中只要出上一点乱子,御史就能即刻上书。别忘了,那可是在京西啊。”

“是不是直接坐着看就好了?”吕升卿又问着。

吕惠卿不置可否,但吕升卿说的没错,这一次,只需坐视就可以了。他还有更重要的事要做。

自王安石辞相之后,并没有立刻任命新的宰相,而是让冯京一人坐在宰相之位上,“冯当世!”

