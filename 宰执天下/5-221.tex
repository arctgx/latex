\section{第24章 缭垣斜压紫云低(13)}

目送韩冈离开,三名御医也暂时退到了外间。寝殿内只剩赵家的祖孙三代这么一家人。

向皇后坐在床沿,低头整理着并不需要整理的被褥。朱贤妃搂着儿子,与其他几位嫔妃站在一边,大气也不敢出。

赵颢则站在高太后身后,扶着椅背,低垂着眼。

帝位之前,兄弟如仇雠。赵顼一病不起,赵颢和赵佣就跟死敌一般。两边如同划下了一条鸿沟,隐隐然如两军对垒。

一时间都没人开口,连视线的交流都没有,静得能听见玻璃灯盏中的烛花在爆。

“让韩端明走了这样好吗?”蜀国公主感觉到有些喘不过气来,想打破这样让人难熬的气氛“万一……皇兄这边……有个反复……”

“他是端明殿学士!”向皇后毅然决然的说道。

韩冈是端明殿学士,文臣中仅次于宰辅,不能将他当成御医来使唤。就向皇后本心,她当然想让韩冈就在外间守夜,随叫随到,以防万一。可这个时候,药王弟子是决然不能开罪的。

向皇后让宋用臣代赵佣送韩冈,也就是崇以师长之礼,确认了韩冈是赵佣的师傅——尽管韩冈还没有被任命为资善堂侍讲。那本是应该在赵佣在宫宴上亮相后才进行的流程,向皇后却硬是在此时加以确认。不论是对赵顼,还是对赵佣,保住韩冈的地位和他的忠心,是最为优先的事项。

“但皇兄都病成这样了,还是稳妥点的好。有韩冈在,比外间留十几个御医都安心。”站在高太后身后的赵颢说着,弯下腰,问他的母亲道:“娘娘怎么看?”

“二哥说得也有道理……”高太后也想盼着儿子安全,却不在意那些细节。

“也就几步路而已。何况天子寝宫,外臣怎么留宿?”向皇后抬起眼说道,紧跟着又低下头,也不顾高太后脸色突地沉了下来。

韩冈走得痛快,是向皇后能放心的主因,若当真有危险,当不会走得这么放心。不过向皇后对韩冈的话,还有着很深的疑虑。

‘不出意外的话’,在说起赵顼的病况时,特意加了这个前提条件。看似是跟御医们一个脾性,绝不会将话给说满,但其中是否隐含了深意,身在深宫中的向皇后却无法不去多想。不问个清楚,她怎么也不能放心得下?让宋用臣以送韩冈的名义和蓝元震一起出去,也有更进一步确认的想法。

探手理了理赵顼的头发,看着只能睁着眼睛,却无法做出其他动作的丈夫。向皇后满心凄楚,若真有个万一,这该让他们孤儿寡母的怎么办?

……………………

“端明,请随小人来。”

两名大貂珰陪着韩冈一起出来。韩冈默默一笑。虽然才名不显,但在宫中多年,向皇后看来也不是全无头脑。

韩冈是不是有话不能明说,除了他本人以外没人知道。可作为病家,肯定有许多事要向韩冈咨询,而且不仅仅是病情的问题。天子寝宫中人多嘴杂,耳朵更多,当然不方便问。韩冈离开的时候则比较安全。

但眼下的形势,谁也不能保证赵顼过去所信任的内侍高品们会不会转换门庭。有一件旧事,大宋皇宫中的人们永远都不会忘记。

太祖皇帝驾崩时,孝章宋后急令一名深得太祖信任的内侍去召四皇子赵德芳入宫,可这名内侍却跑去找晋王赵光义——他的名字叫王继恩。继恩之名是太祖所赐,其人曾过继为张姓,也是太祖皇帝让他归宗。如此深恩,换来的却是毫不犹豫的背叛。

有此前车之鉴,向皇后派出两个关系并不算好的内侍同出办事,其实也是理所应当。而且表面上两人各有分派,更是浑如天然。只是多半还是瞒不过人,当事人也好,旁观者也好,韩冈觉得他们应该都是明白的。

在蓝、宋二人的护送下,韩冈走上通往西门的回廊。离得寝殿稍远,宋用臣终于开口:“陛下得上天庇佑,终归无恙。但若非有端明在,那还真是不知会变成什么样子。”

“陛下乃是天子,登基后施政顺天应人,当然是福德过人。韩冈在其中倒真是没起到什么作用。”韩冈顿了顿,又道,“既然今天要留在宫中,还请两位遣人去寒家报个信,以免家中挂念……在宫外家人也不便进来。”

“那是自然,那是自然。”宋用臣和蓝元震一齐点着头,“既然端明吩咐了,小人这便派人去府上报信。”

“多谢了。”韩冈抬头望了望已然挂在飞檐上的半轮弯月,又笑了笑:“家岳尚在驿馆,得尽快知会一声。”他轻声叹,“当年家岳能一展胸中抱负,都是天子的重恩。若是听了传出去的只言片语,心里还不知怎么着急呢。”

蓝元震和宋用臣都是聪明人,要不然也升不到内侍高品。韩冈说的话也并不算隐晦,所以他们的脸色便一路惨白下去。

“端明的意思是?”宋用臣颤声发问。

“意思?什么意思?”韩冈回头反问。

两名大貂珰算是明白了韩冈到底是什么意思。

这位年轻的端明殿学士是绝不会给出一个明确的答复的。但只要往坏里去准备,那也不会有错。如果往坏处去想,韩冈话中的一个‘尽快’,已经说明了形势有多么危急。也就是说,方才药王弟子在寝宫中的一番话或许仅仅是在为延安郡王争取时间而已。

“王相公”

‘很好。’韩冈想着,能明白就好。

方才在寝殿中,自己并没有将话说死,不论最后会是什么样的情况,都会有合理的解释。而刚刚说的一番话在字面上也只是提到王安石,怎么理解那是向皇后她们的事,同样完全不用担心被戳破画皮。不过接下来,向皇后那边应该就会设法让王安石复相了。

希望她们能做到。韩冈边走边想,却并不抱太大的希望。这件事说说倒不难,但要做到,可就不是那么容易了。

学士院锁院,天子御内东门小殿,任命宰相的正规程序一个都不能少。更重要的还是要取得王珪等宰辅的谅解和同意——这点就很麻烦了。以王安石的声望,一旦复相,他们都得靠边站。怎么说服他们放弃一部分权力,其实是很考验人的一件事。如果事前的沟通没做好,就算复相的诏书签发出来,政事堂也可以轻而易举将这个决定给推翻。

而且还要赶在高太后正式垂帘之前。否则,以高太后对新法的反感,将王安石换成司马光也不是什么出奇的事。

能不惧高太后的压制,甚至反过来压制高太后,搜遍如今的京城,也只有一个王安石。韩冈现在还做不到的事,王安石能做到。会坚定不移的站在赵顼一边的朝臣,不管人数有多么少,其中必然会有王安石一个。相信这一点,向皇后应该是了解的,就不知道她能不能做到。

宋用臣送韩冈出了福宁殿的外门,便立刻告辞回去了,他必须尽快将韩冈话中隐含的消息通报给皇后,不管是不是他自己的误解,都必须让皇后知道。要安排几位宿直重臣的食宿的蓝元震则继续陪着韩冈,

蓝元震和石得一一起管勾皇城司多年,探查京城内外事,其间得罪的人太多,如果赵颢登位,下场就不只是在某间皇家寺庙里养老那么简单。

行走在深夜中的宫城中,廊下灯笼的晕晕光圈勉强照亮脚下的道路,前面引路的两个小黄门各自提着一盏玻璃提灯离得远远的,他们自一出来就保持着这样的距离。在深宫中,缺少自保的智慧通常都活不长。那两个看起来只有十一二岁的小宦官并不缺乏这样的智慧。

蓝元震脚步蹒跚,身为天子近臣,他的权力、身家全都建立在与皇帝的关系上,赵顼的病倒对他的打击不可谓不大。尤其是韩冈刚刚的一番话,更是将才升起的希望之火,给掐灭了。

“好端端的,怎么就能病倒呢。”他喃喃念叨着。

“是阴阳失谐。”韩冈并不清楚蓝元震这是试探还是感叹,随口道:“也就是寒热变化得太快了。骤寒骤暖啊……”

蓝元震的脸色陡然发青,“今天郊祀的时候,我们怕官家在圜丘上冻着,特意在大次中生了旺火……但过去也是一样这么做的,官家不一直都是好端端的,也没见发病!?”

他的声音大了起来,争辩着。前面的两个小黄门闻声回头看了一眼,立刻又快步往前多走远了一点,带着手上的玻璃提灯一路摇晃。

“病气如贼。库房的守卫就算再疏忽,也不一定日日被贼偷,但终有着贼的时候。一次可就足够了。”

韩冈一声喟叹,乍暖乍寒的时节,的确容易诱发中风。初春暮秋,这等气温剧烈波动的季节,便是中风的高发季。而人工制造的温度波动,诱发了赵顼的发病同样常见,当然也不是什么值得让人惊讶的事。

蓝元震安静了下去,脚步,。方才,声音中带起了哭腔:“为了怯寒,登台时,官家还多喝了几杯热过的杨梅酒,本以为能怯除寒气,谁想到还是这么不经熬。”

韩冈脚步一顿,随即又恢复正常。心中一时泛起波澜,难道赵顼中风还有喝酒喝多了的原因吗?那还真实罪莫大焉了。

