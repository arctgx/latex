\section{第14章 卧薪三载终逢春(上)}

【年后上班第一天,忙了一点,现在才赶出来。第二更,求红票,收藏。】

虽然李师中对韩冈瞪眼暗骂,但终究改变不了结果。他挨了王韶当头一棒,却不能就此事发作。王韶说的本就是正论,做一天和尚敲一天钟,既然案子在手上,就必须将之审下去。

当着王中正的面,李师中也只能哈哈干笑了几声,道一句王子纯说得有理,自当如此,举起杯来,敬王韶的酒。而酒宴上的气氛,被一桶冷水浇过,就再也没热起来。过了一阵,秦州知州推说头疼,向王中正告罪后,当先退场。

王韶的用心,李师中先前已经看破。他本奢望着眼前的局势可以让他留任秦州,他能对王中正这个阉宦笑脸相迎,也是因为有了一点自信。但王韶当面表明了他的态度,最终天子会怎么选择,结果又是为何,其实已经有了分晓。

一场宴席便随着李师中的离开不欢而散,而王韶的这次图穷匕见,已经在秦州官场中引起了轩然大波。

不知内情的外人,并不清楚王韶的本意是想着让王中正把他与李师中水火不容的情况报给天子。在他们眼中,王韶这是挟着因两次大捷而来的声势,明着要在官宴上与李师中分出个一二三来。

在外人看来,王韶发难的时机选得让人拍案叫绝。窦舜卿被他孙子连累,李师中也不受天子使臣待见,向宝的钤辖之位更是被王韶的盟友张守约所替代,秦凤路主管蕃部事务的机宜文字如今气势正盛,眼下正是重新划分秦州官场派别的良机。

要不是王韶的资历实在太浅,连个通判都没做过,而担任秦州这个节度要郡的知州,至少是得有侍制以上头衔,秦州知州的位置落不到他人头上去。而现在,如果李师中、窦舜卿尽去,现任的秦州通判也不够资格接任,只有从京中另外派人来。

以如今王韶的功绩,以及天子因两次大捷而被吊起来的胃口,派来的新任知州必然会全力支持河湟开边。在其他官员看来,王韶的底气就在这里。

对于外人的误会,王韶倒没管这么多,韩冈听了一点传闻,同样没放在心上,现在他们最重要的工作是把王中正给陪好。

尽管天子那边做出选择至少要到一个月之后,但王中正的选择已经出来了。在秦州点验过一千多颗首级,他就跟着王韶往边境上去。

在永宁寨见识过了马市榷场,在古渭接见了来前来拜见的俞龙珂和瞎药,最后王中正又随着王韶一起到了渭源堡。王中正对渭水之源很有兴趣,不过王韶要在堡中处理一些琐事,就安排了韩冈和王厚陪着他去渭水的发源地去走一走。

低头看着脚下的清澈见底的涓涓溪流,王中正怎么也看不出这跟浑浊汹涌的渭水有何关联。即便是因伏旱而水位低落,他所见到的渭水,依然涛声如雷。王中正抱着深深的疑问:“这就是渭源?”

“这正是渭源。”王厚点头答道,他指着不远处,流淌出眼前这条溪流的那座林木森森的山峦,“那里就是《书》中所载的鸟鼠同穴山。”

“‘导渭自鸟鼠同穴?’”王中正随口就将《尚书•禹贡》中的词句引用了出来,显然对儒家经典是了若指掌。

“正是这一句。《山海经》亦有载,‘渭水出鸟鼠同穴山,东注河,入华阴北。’不过鸟鼠同穴念着冗长,现在都唤作鸟鼠山。鸟鼠之名,可是有着几千年的历史了。”

韩冈点头说着,心中却在惊叹王中正竟然能把尚书中的文字信手拈来。暗叹着,能在宫廷中混出头来,果然不可能是个简单的人物。

从方才王中正露的一手来看,他对儒家九经的了解,也许比王厚还要强一点。而他的书法,韩冈这些天没少见识过,的确是上品无疑。

韩冈曾听说,宫中的那些个内侍高品,基本上都是自幼入宫,在宫中就学。经过多年教育熏陶,无论文才武艺,皆有可观之处。出外任官,往往胜过一些只会吟诗作对的士大夫。

想起真宗朝的宦官名将秦翰,再看看眼前的王中正,韩冈不禁感慨,所谓传闻流言,确是其来有自。

秦翰一生领兵南征北战,前后负伤几近五十次,北抗契丹入侵,南平益州叛乱,在关西又与李元昊的祖父李继迁对抗,死时三军恸哭,是开国以来有数的良将。

而王中正在不经意间表现出来的学问,已经可以让普通儒生自愧不如。而他现在身穿着青布襕衫,打扮得就像个文人,细长的眼眉也让他有着些斯文气。

不过王中正却有着贪财的毛病。前几日在秦州时,各家给他送的礼,他可都是毫不推辞的一股脑儿都笑纳了。王韶和高遵裕听说了此事,都皱眉不已。比起家无余财的秦翰,王中正的德行可是差了许多。

“时候已经不早,要到渭源的品字泉处,现在得走快一些了。”王厚在前催促着。

韩冈抬头看了看天色,的确已经近午。山中可没有后世那样正经的水泥路,走得慢了,黄昏时就来不及出山了。

“处道说得也是。”韩冈回头向王中正问询,“都知,我们是不是走快一点?”

“那就快一点好了。吾亦是想早一点见见,渭水源头究竟是什么模样。若是能再见识一下何为鸟鼠同穴那就更好了。”

“同居一穴的鸟鼠却是难见。”王厚笑道:“去岁在下随家严来过,只是见到蝙蝠乱飞。”

“原来已经来过了,难怪如此道熟。”王中正转过来问韩冈,“韩抚勾你呢?”

韩冈道:“在下尚是第一次来此。”

一行人快马加鞭,很快就进入了鸟鼠山中。从被烈日炙晒的野地里,走进草木葱郁的树林,一阵沁人心脾的清凉便降临到众人身上,让人神清气爽。

而一阵清脆的铃铛声这时从林木深处传来。王中正还没来得及询问,就看到前方道路转弯处,闪出一队蕃人马帮。二十多匹马背上都有两个大包裹,而赶着马队的则是六七个蕃人。

这几个蕃人一见到迎面过来四五十名骑兵,立刻紧张起来,用力勒停坐骑,手上也握住了刀柄弓臂。不过当他们看清了韩冈这一彪人马的装束,却放松了下来,驱赶马匹避让到路边。

韩冈等人骑着马昂然而过,不理会这些蕃人。经过老远,王中正却回头望着,问道:“此处为何有蕃商?”

韩冈向他解释:“鼠鸟山南,支流尽入渭水,鼠鸟山北,水脉尽入洮河。这座山实是渭水和洮水的分水岭,从河湟往秦州的要道便自山中过,故而商旅众多。此时还算少的,等到秋时马膘长上来,这条路上哪一天都能看到十几家马队经过。”

王中正看看脚下越来越狭窄曲折的道路,皱眉道:“难道去河湟,就没有其他路了?”

“当然有!”韩冈点头,“另外一条路走的是北面的露骨山。不过露骨山地势险阻,道路难行,轻装骑兵经过容易,但载着货物的商队就不好走了。”

“这条路还算好走!?”

韩冈笑道:“这条路是唐时修筑,已经几百年没有整修,所以看着破败狭窄,其实重修一下,就会好走得多。”

他停住马,叫过两名军汉吩咐了几句。就看见两人点头后,走下道路。拔出刀,在道边一片稀疏的草地上挖了一阵,掘出一个坑来。

韩冈指着坑里的黄土:“无论汉唐,皆于此修桥铺路。看这下面就是夯筑过的熟土,可见本是官道的一部分。而上面的土层是这两百多年来洪水泛滥后才淤积起来的。所以只能生草,长不了树木。”

他又指着眼前的山峦,“等日后攻下木征设在山背后的两处寨堡,就可以腾出手来重修鸟鼠山道。那时向河湟运输粮秣就会容易不少。不过若是能夺下河州,控制了洮水,大部分的粮秣军资又可以改由川中水路转运,费用比起走秦州还要节省。”

听着韩冈将鸟鼠山道的古今娓娓道来,王中正总算是明白了一点为何眼前的年轻人这么得人看重。识见渊博,谈吐出众,又加上设疗养院、制沙盘军棋的才能,的确是难得的人才。再想起韩冈自称是第一次探索渭源,竟然已经对此处如此了解,可见他在其中下过多少功夫。

一行人在树林中,顺着连接河湟和秦州的道路走来一里多地,又跟着王厚拐进了一条小山道。山道一路向上,前方不断的有垂下来的藤条和树枝拦路,韩冈不得不派出人手拿着刀去前面开道。

听着身侧林中传来的流水声,韩冈、王中正他们又走了大约一个时辰,树林中的山道终于到了尽头,眼前豁然开朗,原本被树林遮挡的渭源溪流重新出现,而一座苔痕处处的破庙出现在众人面前。

顺着水流,王中正看着破庙边一个碗口大的石穴中汩汩流出的清泉,摇头叹道:“想不到滔滔渭水,其源头水脉竟然如此细小。”

王厚小道:“无论江河,上溯至源头,也不过是一眼清泉而已。”

王中正转头向西,眼神似是透过了眼前的山峦,望着极远处的某个地方:“江源不敢望,却不知何日能见到大河之源。”

韩冈闻言,嘴角微微翘起。身边的这位阉宦,果然对拓边军功动了心思。

