\section{第34章 彩杖飞鞭度春牛(中)}

但韩冈却年轻得过分,让人不禁怀疑起这份规程的出处究竟是不是他本人。李师中幕中的一名清客看过之后,便当即摇头道:“此一篇,必是韩冈剽夺无疑!他绝然写不出来。”

正如写诗作赋,不可能跳出作者本人的经历,初出茅庐的韩冈如何能如积年老吏那般面面俱到?

如果只是靠着臆想作出的诗句,便完全无法与融入真情实感的作品相比。没有亲自走过蜀道,李白也写不出《蜀道难》,不是好酒狂纵的游侠性子,《将进酒》也不会出现。大漠孤烟直,长河落日圆,不是亲历大漠,如何写得出来?

李师中的那位在王素帐下同样做过幕宾的清客,当时也对他说,“范文正【范仲淹】帅府陕西之时,曾有《渔家傲》多首。皆是以‘塞下秋来’为首句,道尽了边镇劳苦。但欧阳六一嘲其为‘穷寨主’之词,也做了首《渔家傲》,送与要入关中的王尚书,自谓是‘真元帅之事’。

当日学生也在场,曾听着尚书家的几名家伎按曲而唱,但如今只记了‘战胜归来飞捷奏,倾贺酒,玉阶遥献南山寿’这一句,剩下的早忘得一干二净。而范文正的‘衡阳雁去无留意’,却遍传天下,至今犹唱。”

李师中来秦州有半年多了,对‘白发将军征夫泪’已深有体会。而欧阳修并未在关西任官过,他的‘玉阶遥贺南山寿’不过是凑趣敷衍之词,既乏实感,又缺真情,当然无法流传。

欧阳修再如何自吹自擂,他的这首《渔家傲》也是远远比不过范仲淹的‘塞下秋来风景异’,反倒是‘叶小未成荫’,‘笑问鸳鸯二字怎生书’这两首,由于是真情实景,却是引人之至。当然,正因为欧阳修将十四五岁的少女风情写着入骨三分,世间才有了他帷幕不修,私通侄媳的传闻。

李师中明白他的清客为何要提到欧阳修和范仲淹的《渔家傲》,就是想说完全没有实务经验的韩冈,不可能写出洋洋两万言的伤病营制度规程来。但李师中只用一句话就问得清客哑口无言:“不知韩冈抄袭是谁人?”

如果是一个少年写出了有悖于他生平经历的上佳词句,多半就可以确认他是剽窃,但有关军中医疗制度,历朝历代都没有先例,也没有章程可循,韩冈又是从何剽来?

除非他真的是孙思邈的私淑弟子!——可在李师中翻看过的史书中,孙思邈好像也从来没有这方面的著述和言论。

如果此份规程的确是韩冈自出机杼,再加上他一言灭尽土豪满门的手腕,韩冈的才智已足以让李师中感到心惊胆战。他仅有的缺点,也就是差一个进士出身,又早早的出仕,性子太过急切了一些。

李师中现在很后悔,早知道韩冈才干如此,他根本就不会同意让他来【和谐万岁】经略司任职,危险的苗子只该早点拔除。可如今天子已下特旨,想再改口就没那么容易。

远远望着风姿秀挺的韩冈,李师中心中火烧火燎的一阵烦躁。自从王韶把韩冈招致门下后,小动作也当真是越来越多,让他不胜其扰。而且同时举荐韩冈的还有吴衍和张守约,这让本来已经孤立无援的王韶,等于一下又多了两个得力的臂助。

‘至少得把他从王韶身边弄走!如果有机会,栽他一个赃罪那就更好……’

韩冈忽然间一阵毛骨悚然,方才他转身间无意中对上的眼神阴冷潮湿,让他只觉得有一条冰冷腻滑的毒蛇,在背后蜿蜒盘旋。他貌似不经意的四面张望,但那道眼神却再也没有出现,唯一能确定的,方才盯着自己的是聚集在春牛旁的秦州官员们。

韩冈向那里望去。李师中四平八稳的站定,只是眼皮半耷拉着,大概是在等着鞭牛仪式结束。紧跟在李师中身后的秦凤路兵马副总管却正好往他这里看来。

韩冈略略低头,避过那道审视的目光。

秦凤兵马副总管窦舜卿是个新面孔,就赶在腊月中,他受命来秦州上任,据说是为了顶替了颟邗无用的前任。可窦舜卿须眉花白,腰杆也微驼,看起来比张守约还要老上许多,也完全没有张守约身上百战功成的气势。乍看上去像个文官,而且是庸庸碌碌的文官。

正如窦舜卿的外表,韩冈也没听说新来的窦副总管有什么出众的战绩。好像就在京东【大体是山东】打过海盗,还有就是在荆湖北路【今湖北】剿过叛乱的蛮瑶。

韩冈祖籍京东,自他祖父那一辈才因故迁来关西,听到窦舜卿为老家剿灭贼寇的事迹,倒有几分亲切感。但如今的海盗,其实就跟前日死在韩冈手上的过山风差不多,一伙也就十几人、几十人的样子。若是剿灭海盗都能算是战功,那他韩冈手上的战绩,便已经不比窦副总管在京东差了。

窦舜卿是承继父荫而得官,其父好像升到了横班,是朝中总计不超过三十人的高层将领之一。而窦舜卿本人,甚至比他父亲还要官运亨通,竟是以殿前都虞侯、邕州观察使的身份,来领秦凤路马步军副总管一职!

驻扎在开封府界的十万京营禁军,分属两司三衙统领。两司是殿前司和侍卫亲军司,而侍卫亲军司又分为侍卫亲军马军司和侍卫亲军步军司,这两司与殿前司便合称三衙。其中殿前都虞侯便是殿前司排名第三的统兵官,仅次于殿前都指、副都指挥使,统领着京城内外拱卫天子的班直侍卫,以及捧日、天武等上位禁军。

不过放到窦舜卿这里,殿前都虞侯就不是实领的差遣,而是与向宝‘带御器械’的加衔一样,是一个荣誉性的头衔。比起天子身边的宿卫,殿前司统兵官当然要远远高出一大截。向宝能让前任副总管形同虚设,但在窦舜卿面前却根本抬不起头来。

在关西,名位能与窦舜卿相抗衡的武臣,也就只有宣徽南院使、静难军节度留后、判延州兼鄜延经略使——郭逵一人。

而观察使一职,同样是武臣中屈指可数的官位,世称为贵官,仅次于节度使和节度留后,排在武臣等级的第三级,其下是防御使,团练使和刺史。

通常这等贵官,不仅是给武将,更多是封给宗室或是外戚,偶尔也有文臣得以加衔。濮王的第十三子赵曙,也就是英宗皇帝,被仁宗过继来为皇子前,便是个团练使,人称十三团练,比窦舜卿的观察使还低两级。

以窦舜卿如今的官位品级,已经达到在国史中留下一份传记的资格。一般来说,官阶升到窦舜卿、郭逵这般地步,名位便已做到了顶,天下武臣中也不过三五人的地步。就该喝着热茶,晒着太阳,等待致仕了。

前任的那位让人印象模糊的秦凤兵马副总管,已算得上老迈无用,而窦舜卿的年纪比他还大上一点。郭逵是在陕西、河北都留下累累功勋的宿将,所以当开拓横山的战略需要一个稳妥的后方时,他便被赵顼钦点去镇守延州。

可窦舜卿的才具世间并无传说,只是他的籍贯是相州,与两朝顾命的元老大臣韩琦是乡里乡亲,他能得升高位,多得韩琦助力。而韩琦如今是反变法一派的主心骨,纵然离开了京城回到相州,他的阴影依然盘踞在变法一派的头顶上。

王韶就很担心窦舜卿来秦州后,会与韩琦一呼一应,使得拓边之计沦为空谈。韩冈现在远远的盯着窦舜卿,他已经忘记了追查眼神的主人,而推算着新来的副都总管会给秦州官场带来什么样的变局。

“玉昆!”

“嗯?”耳边一声唤,把韩冈从思绪中惊醒,王厚带着王舜臣不知何时挤到了他的身边。被抢去位置的几人嘴里嘟嘟囔囔还在抱怨着,但帮王厚推开人群的王舜臣只一瞪眼,他们便如落水狗一样抖了几下,乖乖的让了开去。

“昨天回来,大人为了上报硕托、隆博两部的事,便连夜去翻经略司架阁库【注1】里的故纸堆,想找出过去处理蕃部相争的堂扎,好对着写奏章。最后想找的没找到,却找到了一个更有用的……玉昆你猜,大人找到了什么?”王厚很是兴奋,鞭牛已经快轮到了王韶,他也不去看,对着韩冈扯出一大段来。

“没头没脑的,我怎么可能猜得到……”韩冈声音突然一顿,将视线投到排在官员队列中的王韶脸上。虽然他装得若无其事,但已经很熟悉王韶的韩冈,还是能看出明显的透着喜色。

“是与古渭有关?还是抓到经略相公的把柄?”韩冈猜测着。王韶不是沉不住气的人,能让他兴奋如此,定然是有助于拓边计划的重要情报。而王韶翻的又是政事堂下发的公文——这称为堂扎——还与蕃部事务无关,那需要猜测的范围就很小了。

注1:架阁库就是中国古代的档案馆。一般来说,无论中枢还是地方的衙门,都会设有架阁库,用以存放过往公文和账簿、名籍等物。

ps:今天这段顺便说了下诗词的事,古人并不缺才智,如果想用抄袭的诗词来长名声,不能抄杰作,弄个二三流的作品凑数就行。若是抄袭的诗词等级太高,惹动了那些眼光毒辣的文学宗匠,想不露破绽是不可能的。

第三更,求红票,收藏

