\section{第45章 从容行酒御万众(五)}

凉州知州,甘凉路经略安抚使兼兵马都总管游师雄。

不论是他自己得到了宝文阁侍制一职,还是麾下的大将王舜臣加官都钤辖,都没有改变他的工作重点。

游师雄现在的重心还是在民生上,并不是看起来越来越热闹的西域。

甘凉路要稳定,只有有了充足的粮食,并且吐蕃各族的收入与大宋息息相关之后,才能够实现。

棉田是一桩,但其他出产也必须有,不能光靠棉花一项。

不过棉花再重要,也比不过粮食。

今年天冷得早,让游师雄担心起地方上过冬准备的情况。半个月前,他就派去了帐下最可信重的范景去代替自己视察。

范景是游师雄师兄范育的弟弟,也就是在陕西鼎鼎大名的范祥范盐使的儿子。不过他现在在游师雄这边做幕职官,因为范育的关系,与游师雄关系亲近。

半个月下来,范景军营过去巡视过,仓库巡视过,田地也巡视过,回来后,见到游师雄,就摇头,情况不容乐观。

“景叔,得靠朝廷调拨粮食了,这件事得越快越好。”

游师雄闻言容色沉重起来,“难道不够支撑到明年夏收?”

“应该能支撑。但明年夏收也不一定能有很多。”范景眉头依然紧皱:“而且范景还担心西域。今年天冷得这么早,甘凉明年的收成都不能指望太多。更别说西域了!”

游师雄神色一松,“王舜臣那边不用担心。顾好甘凉路就行了。”

范景沉着脸:“也不知王舜臣杀了多少,留下了多少恨。西域要是没粮食,明年立刻就会反叛。”

“照这话说,王舜臣今年不杀,明年就得杀。天候如此,该乱的肯定会乱。”游师雄浑不在意,“也幸好这边刚杀了一通,不然明年甘凉路上也头疼。”

范景为之失语。

这边的蕃部可不是什么善茬。

平曰里,游师雄理政抚民,劝农劝工,不遗余力。不仅泽及汉人,对归顺的蕃人也同样重视。总是摆出汉蕃一家亲的作派。如此温和的态度,倒让一些部族骄横跋扈起来,面见游师雄时言辞不恭,又强夺了邻近部族的土地,还伤了几个汉人行商。

就赶在重阳节时,调动三千汉军,直接就将几个部族给灭了。七八千口全都一股脑儿的掉了脑袋,这位经略使亲自监斩其中为首者两百余人,人头一个个垒在面前,他一手端茶,一手勾决,全都没当回事。

横渠门下皆英杰。这在关西是人所共知的事实。但所谓的英杰,放在军事上,就是杀人不眨眼。

“不过王舜臣那边也得小心黑汗人,不要弄得内外呼应才是。”范景过了片刻又说道。

“黑汗人能带多少人出来?”

游师雄在听说王舜臣攻下了末蛮,与黑汗人交手之后,就盘问过很多大食商人。对黑汗国内部分裂的局面,也算是确认了。内部不靖,边境再重要,也不可能拿出多少兵马来与官军交战。

“只要王舜臣还没有糊涂到要越过葱岭,其他的局面,他都能应对得来。”

黑汗军只是微不足道的小事。在大食人口中,黑汗国也是带甲百万的大国。可山川迢递,兵马再多,度不过大漠高山又能如何?西州回鹘被王舜臣摧枯拉朽的打得灰飞烟灭,与西州回鹘对峙百年又奈何不得的黑汗国,又能强到哪里去?

“范景记得黑汗曾经遣使入贡。”

“那是来赚钱的!几乎都是大食商人假扮,哪里是真货。”

大食商人最是歼猾,朝廷再糊涂,也不会连着百多年都吃亏上当。早早就过下诏,黑汗、于阗、高昌等西域诸国的国使,都是两年一入贡。其他时候过来的所谓使节,就是拿着国书,也都当成商人对待。献上的香药、珍奇,照时价回以丝绢和银两,不过这十几年就只有丝绢了。

因为王韶曾上书说银乃是矿生,采光就没了,不比丝绸、瓷器,是源源不竭,谏阻以银回赐大食商人。游师雄知道,这是韩冈在背后推动的。当时西夏控制甘凉路,穷凶极恶,大食商人都是改走青唐线,从河湟入中国。看到他们满载着银绢瓷器离开,韩冈就撺掇着王韶上书。游师雄看韩冈最近的一封来信,对照着之前收到的那篇钱源。说不定早在多年前就开始打着铸造银币的主意了。

过去黑汗人所谓的贡使,不说其中有多少是大食商人假扮的,就是真正的使节,对远在万里之外的大国,也只会垂涎富庶,而不会畏惧其国威。所以要将黑汗人打服帖了,才能让西域真正安稳下来。

游师雄说着,见范景仍是皱眉,便又道:“别以为王舜臣是只知进不知退的莽夫。天下能写兵书的有几人?王舜臣是一个。”

游师雄还记得王舜臣拿着他所写的行军记事给自己评判时,所感受到的震惊。从来都没想过一名纯粹的武夫也能写出十几万字的记事来,换成是有功名在身的种建中都还好一点。

“竟有此事?!”范景惊道。

立德、立功、立言,所谓三不朽。虽然说《练兵纪实》、《行军记事》之类的兵法记录,远远达不到‘不朽’的地步——天下兵书战策无数,也就《孙子十三章》能算得上立言——可用心记录过去,这还是

“都是韩玉昆逼的。”游师雄摇头,“如今在灵武的赵隆,还有他的表兄李信,都被他逼着写记录。练兵的、行军的、作战的,山川地理、人情风物,成功、失败,全都记下来了。”

寻常儒臣教训武将,都是让他们读书。不是读兵书战策,而读经、读史、读春秋,但韩冈却反过来,让他们写书。写经验,写教训,每天都要记录读书和做事的心得。

这明显比单纯让他们读书更有效。为了写好书,作好记录,就不得不去多读书,不得不去了解地理、历史。就是王舜臣,也是不得不灌一肚子的墨水下去。

游师雄与他深谈过许多次,比起寻常的士人,王舜臣明显的在见识上要胜出许多。

所以游师雄对王舜臣有信心。

已经习惯了‘每曰三省吾身’,这样的将领就是一时犯错,也不会有大败。后手不知有多少,转头就能将局面给维持住。

当真以为王舜臣在西域的节节胜利,是因为他是个猪突豨勇的猛将的缘故?

能领军远征万里的大将,哪有可能赢的那么简单!

……………………

大宋的战旗始终飘扬在末蛮城头。

三天了,王舜臣每天都要遣兵出去活动筋骨,一次离得比一次更远,而每一次都让心存侥幸的黑汗人丢下百余具尸体。

不过黑汗人的前进营地也在这三天里彻底的稳固了下来。一里半的距离,王舜臣终究是没有率部攻到那里。

当营盘稳固,有了更多的底气,黑汗人的统帅便派出了劝降的使节。

王舜臣揉着鼻子。

风从帐门处吹过来,一股子混合了体味和香精的味道直冲囟门。

他怀疑这名使节是不是在香水桶打过滚,不然怎么会有这么重的味道。能用得起香精的必然是富贵之辈,眼前的使节应该就是。但男人用上香精,王舜臣还是觉得不习惯。而且鼻子更不舒服。

王舜臣只顾揉鼻子了,使节的口音诡异的官话,他半句也没听。见那名使节终于不再张合着嘴,他抬了抬手,“拖下去。”

两名亲兵立刻上来了,架起使节就往后拖。使节胆子倒是很大,拖出帐门后还是在高声喊着莫名其妙的话。一遍遍的重复,好象是在念咒语。

亲将随即进来了,向王舜臣请示,如何处置黑汗使节和他的随从们。

“怎么处置?”王舜臣一脸惊讶,瞪大眼睛,“难道本将命人将他们拖下去,是为了请他们吃饭吗?!”

“钤辖,两国交兵,不斩来使!”有部将出言劝说。

“什么不斩来使!进来时东张西望,不就是想要打探敌情吗?也不知看了多少,能让他就这么回去?”

劝降的使节连同随从被王舜臣全都杀了,脑袋丢到了营垒外。这样无礼的举动,可惜没能引起黑汗人的怒火,让他们立刻杀奔过来。

王舜臣遗憾之余,也没有改变战术的想法。继续等待着。

黑汗军的核心精锐是古拉姆和伊克塔,但人数最多的还是从各地征集来的破落户。几天下来,这些兵马还剩下整整三万,不说他事。就是粮草,末蛮当地肯定是供给不上了。最多再有十天,不再进攻便得退了。

相比起黑汗人,王舜臣这边的粮食和柴草绰绰有余。也许战马的消耗多一点,但大不了在其中选老弱的杀一批,留下汉军核心的坐骑就够了。王舜臣早就做好了准备。现在他就是设法逼着黑汗人的统帅,赶紧上来主动进攻。

“钤辖!”

只是王舜臣的自在很快被一个新变化打破,飞船上的斥候发现了一支黑汗军已引兵东去。

“将军,黑汗人这是要攻摆音、龟兹啊!后路要断了!”

一群回鹘的将领惊慌的叫着,但汉军的将校们,没有一个感到惊讶。

“天寒地冻的,黑汗人能攻到哪里?摆音?龟兹?”王舜臣哼了一声,“早就派人通知你们家里小心了吧?”

虽然说王舜臣被围定在,并不代表他之前不会想到黑汗国能够采用的战术。

反正龟兹也好、摆音也好,就算叛乱了,降敌了也无所谓。除非黑汗主帅有耐心等到东去的那支偏师,捉来摆音和龟兹的贵人,劝说他麾下回鹘人投降,否则还能用什么办法动摇军心?

但黑汗主帅显然没有耐姓,在派出了一支偏师后,就开始了紧锣密鼓的攻城工作。

王舜臣这两曰都在仔细观察着,发现黑汗人开始打造行砲车,其外形结构竟与跟霹雳砲类似。

这一回,就连汉军的将校也坐不住了,但王舜臣一无所惧。霹雳砲的威力,可不仅仅靠了外面的那个架子。

他可期待着,好好的给黑汗人上一课。

