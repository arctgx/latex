\section{第九章 长戈如林起纷纷(六)}

【第三更,求红票,收藏】

长枪飞挑,利箭怒射,一个接一个战士倒在血泊中。反抗越来越弱,数百上千的骑兵在开始在村落中放纵着他们的杀意。往日安宁平静的谷地,如今变成了人间地狱。跟随王韶一起攻打托硕的党令征部世代居住的山谷,如今正被前来复仇的大军隆隆碾过。

帐篷被挑起,将躲在里面的老弱暴露出来,奔驰的骑兵把熊熊燃烧的火炬丢向倒塌下来的帐幕,连着人群一起焚烧。火焰中的惨叫和悲鸣,只引来了杀戮者们更狂纵的行动,被血腥刺激了头脑的骑兵,把每一个逃出火海的幸存者又用长枪挑了回去。

“第三家了。”

董裕高居马上,立于谷口。眼望着谷内一道道腾起的浓烟。脸上是得意的微笑,麾下军队在谷中兽性毫不在意,而是更增添了他复仇的快感。

在他左右,上千名骑兵分作数队,堵在谷口处,不让任何一人逃脱。在另一侧的谷口,同样有着一支队伍在阻截逃敌。

一支骑兵得意洋洋的往谷口行来。

跟在董裕身后的一个亲随凑上前来,提醒着董裕,“确臧多吉回来了。”

“掣逋。”确臧多吉叫着董裕在吐蕃王廷中的官名,他的马背上打横架着一个抢来的女子,脖子上挂着十几条金银珠串,马鞍后还捆着两匹绢绸。到了董裕的马前,他大笑着:“青渭这里的部落,真是一个比一个殷实。俺家里远远比不上他们。”

“多吉,今次你可是丰收啊。”董裕如今正是志得意满的时候,他毫不客气的探手抓着头发把那蕃女从马背上揪起,贪婪的打量了一番她的容貌,然后笑道:“这个还不错。”

确臧多吉脸色变了一变,这个俘虏可是他辛辛苦苦抢了来的,正想带回帐中好好享受一番。他想拒绝,却见董裕已经冷下了脸,他立刻换上笑脸,道,“小的明白,小的明白,回头就送到掣逋帐里去。”

调转马头,确臧多吉恨恨地向谷外去了。亲随冲着确臧多吉的背影吐了口口水,“掣逋,看多吉那小子不情愿的样子,好像被割了肉一样。今次若不是掣逋领头,他们哪有这么好的收成?现在好处都给他们拿了,让他们留一份,竟然还敢推三阻四。”

“现在还用得到他们。”董裕冷冷的盯了确臧多吉的背影一眼,“一切等到收兵后再说。……吴征,瞎药什么时候会过来?”

被唤作吴征的亲随立刻回道:“小的已经派了得力人手去催他了,应该很快就有消息回来……不过瞎药已经做到他事前答应的了,俞龙珂到现在也没敢出头。”

“俞龙珂已经老了,只是条连看家守院都快做不到的老狗,没胆子出门来咬人。瞎药压住他不是他的本事。再派人去跟瞎药说,让他对纳芝临占快点动手。”

这几日董裕率领的联军这两日沿着渭水河谷一直向东,离着古渭寨越来越近,已经深入了青唐部的地盘。

为了防着俞龙珂突袭,董裕不得不把他所亲领的三千本部分成了三部轮班护卫着自己。连着从木征手上借来的六百精锐一起,每一刻都要留着一千多人在身边。

为了自家的安全起见,董裕也只能任由星罗结和其他几个部族在前面大肆抢掠,分去了近五成的战利品。

不过现在董裕见着俞龙珂一点反应都没有,已经逐渐放下心来。他瞧不起俞龙珂这样的人,在他看来,青唐部的族长看似手上势力过人,能号令整个古渭州,但真的把刀子逼到他的面前,他腿脚就软了。这样的废物,如何敢挡在自己的面前。

看来听着结吴叱腊的话并没有错,木征不敢做的事,他董裕也许能在这里做一做。

拨转马头,董裕向东面望了过去,“打前锋的赞及应该已经到了古渭寨了吧!”

结吴叱腊的声音在董裕身后响起:“古渭寨可动不得!”

“师尊。”董裕连忙下马回头,向结吴叱腊行礼。

结吴叱腊还穿着他那身肮脏的僧袍,他来到董裕身边,着意提醒着:“董裕,古渭寨可千万动不得。”

不过不用结吴叱腊这个老和尚提醒,董裕也知道古渭寨不能轻动。被灭掉亲附的蕃部,宋人只是丢了脸面,还不一定会轻易起兵,但若是古渭寨被攻打,宋人却肯定会忍不住。

对于董裕来说,只要灭掉七家与他有怨的部族,他丢掉的面子挣回来了,过去的损失也抢回来了。一切都得到弥补,也就可以打道回府去了,再引来宋人的怒火只会给自己添麻烦。

‘还是得把瞎药叫出来。’董裕想着,‘只要瞎药出兵了,日后如果宋人还是要报复回来,就能让离古渭最近的他去应付。”

……………………

掀开帐幕的门帘,初升的阳光从对面两峰之间照了过来,正正照在韩冈的脸上。清晨时便已经炽烈起来的阳光刺痛了他困顿的睡眼,不过山谷中清爽的空气,终于让韩冈精神为之一震。

辛苦奔波了一夜后,小睡了两个时辰,韩冈却并没有神清气爽的感觉。住在因为点着羊油灯而变得乌烟瘴气的帐篷中,他被一阵阵说不上来却又直透囟门的怪异气味,熏得头昏眼花。

帐篷不知多少年没有清洗过,里面到处都是厚厚的油垢,韩冈一辈子都没住过这样腌臜的地方。即便是韩家最穷的时候,家里也是打扫得很干净。幸好他随身带了自用的毯子,韩冈才可以稍稍安心的裹着睡上一觉。

从帐篷中走出来,周围已经是一片人声。帐篷所在小村的青唐部的子民,已经早早的离开了自家的帐篷。有的下田去做活,有的则在村中打理着马和羊。而在小村北面大约两里多的地方,还能看到一个由黄土夯筑而成的小小城寨,那便是俞龙珂所居住的青唐城。

是的,昨夜韩冈并没有住进青唐城内去,而是在城外蕃落的帐篷中住了一夜。虽然追出城来的俞龙珂好说歹说,但韩冈却坚持如此。

这是韩冈在表明自己的态度,也是为了向俞龙珂证明他昨夜的辞行不是装模作样。不过本质上,韩冈还是在表演,如果这样就能让俞龙珂屈服,他不介意再在肮脏的帐篷里睡上两天。

韩冈步出帐,在他借住的帐篷外,已经有一个全副介胄的蕃人将领在等着他,而旁边则是王舜臣在守着。看到韩冈出帐,蕃将连忙上前,他指着南面,那是青唐城的反方向,“韩官人,我家族长现在就在前面等着,还请官人过去一会。”

‘俞龙珂这是要出兵了?’韩冈笑了一下。

当然,韩冈知道俞龙珂即便是同意出兵,他所顾忌的还是自己背后的王韶,作出决断也是因为青唐部的利益,自己昨夜的说词仅仅是起到推波助澜的作用。

而且韩冈还知道,俞龙珂绝不会与董裕硬拼。他不会为外人去拼死拼活,俞龙珂只会为自己和青唐部的利益行动,最有可能的,就是等七部被打散,他再作为救世主出来拯救危局。

不过这对王韶应该足够了。董裕贼心不死,为了复仇领军攻打青渭,七部猝不及防,损失惨重。王韶因此大怒,便又派了亲信联络了青唐部,点起大军将董裕击败。一整套戏的剧本韩冈现在都能帮王韶写了出来,呈到天子御前,又是王韶的一份功劳。

蕃将转达的邀请,韩冈没有立刻答应,却道:“且等我梳洗一番。”

说完便又转身进帐,而王舜臣便带着两名亲随捧着梳洗的用具跟了进来。

“俞龙珂终于要出兵了?”王舜臣在韩冈梳洗时,在旁边说着,“他不管他的弟弟瞎药了?”

“管他那么多!”韩冈拿手巾擦着脸,“俞龙珂都不在乎,我们何必替他担心?青唐部从来都不是拓边河湟的重点,瞎药有本事上位就支持瞎药,俞龙珂有本事保住位置就支持俞龙珂。不干涉其族中内政,谁上台还不都得老老实实做人。就如现在,俞龙珂再怎么为自家算计,最后还是得动上一动。”

“好了。”漱过口,韩冈整了整衣服,冲王舜臣一笑,“就让我们去跟俞族长汇合,且看看他怎么解决打过来的董裕。”

………………

王韶和高遵裕已经登上了古渭寨的城头。远远望着一里多外,一队耀武扬威来回奔驰的吐蕃骑兵,两人面色深沉如水。才一夜功夫,董裕的先锋已经杀到了古渭寨边。而这时候,离着古渭寨最近的纳芝临占部都还没有撤退过来。至于其他几个部落,情况究竟如何,已经不用再去想了。

“子纯……不用再去青唐部走一趟?”高遵裕问着王韶,情况比他想像得还要糟,高遵裕不得不期盼着援军快点到来。

“不用担心……韩玉昆从来都能给人惊喜,从无一次例外。”王韶与其说是对韩冈的信任,不如说是自己心中的期盼。他现在已经后悔,早知昨天就坚持连夜去找俞龙珂说话了。可现在已不是出城的时候,作为古渭寨内地位最高的官员,他的轻举妄动,会引起寨内守军的动摇,“玉昆肯定能说服俞龙珂,到时就是我们来反击了。”

