\section{第23章 奉天临民思惠养(上)}

黄裳理清了云南路的头绪,回到京师的时候,已经是新的一年了。

云南十州,在政事上,属于成都府路。但在军事上,则属于云南路。

依照当年章惇、韩冈出兵灭交趾的旧例,主帅熊本回京,而副手黄裳则成了云南路的第一任经略安抚使,同时也是理州——原大理府——的第一任知州。

滇池畔的善阐府复唐时故名,为昆州,而洱海畔的大理府则是理州,两州府为朝廷控制,更是移民的重点。其余诸州,都是以羁縻为主,一州之中,除了控制大路的附廓县,其他全都是一个个羁縻州。

那些立下汗马功劳的西南蕃部,朝廷用鲜亮的官袍代替了赏赐,大理国最好的土地给汉人占了,剩下的土地,就分配给了他们。

黄裳就是因为要主持分割土地,而耗费了大量的精力和时间在与那些蕃人首领扯皮上。其间还因为一些大理乌蛮、白蛮的部族不甘降顺,黄裳还组织了两次讨伐,要不是有着合格的幕僚团队,以及担任兵马副总管的赵隆相助,他连照看理州本州事务的时间都没有。

对于得到好处的各家蕃部,黄裳并没有就此让他们自生自灭,而是秉承了朝廷的命令,派了人教导他们种植药材,放牧马匹,砍伐木料,用来与汉人进行交换,同时还传授了来自中原的耕作方式,将他们从刀耕火种中解放出来。

有了从大理国俘获的生口,很多繁重的农活就能从本部男丁手中转嫁到新人身上。而汉人,除了加强对核心地域的控制和发展,便是通过无可替代的贸易,将西南夷从农奴身上剥削而来的利益,再盘剥过来。

这个做法,就像当年韩冈在交趾做的那样,让四夷在经济上离不开中国,最后逐步融合在一起。只要日后商贸往来紧密,那么就不用担心蛮部反叛的问题。

交州的发展很大一部分是来自于商业,尤其是海上贸易的大发展。

不过汉人几乎都集中在交州的州治海门县附近。海门县之外的广大土地,基本上被成百上千座种植园给分割,让原本穷困潦倒的左右江七十二洞洞蛮,过上了安逸的生活。

由于韩冈的建议,交州免征丁税,人口数量上没有太多隐瞒的,而田地数量隐藏较多。不过只要交州产出的粮食足够多,保证国中的粮价稳定,田赋的收入多寡,朝廷绝不会计较太多。而且交州的税赋,绝大多数来自于商税,收缴起来非是费力的田赋,衙门里的官吏上上下都嫌麻烦。

收买上层,拉拢中层,共同剥削下层,中国更能坐享其利,这样的交州,就是治理四夷之地的典范。

黄裳觉得,只要能好好仿效交州,将理州、昆州这两个膏腴之地经营好,再教导蕃人怎么经营他们的土地,后任的官员也不自己作孽,云南路可以就此安定下来,成为朝廷的下一个交州。

黄裳在云南路的半年多,都是在设法将施行在交州的政策,运用在他的理州和云南路上。

等到他回到京城,熊本都已经做了好几个月的参知政事了。

因为灭亡大理的功劳,熊本顺利通过了廷推。而且他还是近年来,难得的高票当选。不论是哪一党,都没有人跳出来反对他这个功臣进入两府。

但熊本最后并没有进入枢密院,有太后钦点,让他进了政事堂。

“是不是太后想要……”

黄裳轻声问着,一根手指在杯口晃了一晃。

“并非如此。”韩冈摇头,“熊本的任命是我建议的。”

“为何?”

“章子厚在密院太久了。”

黄裳眼睛在惊讶中一瞬间睁大了,但转瞬间又明白了一切,“这样啊。”

现任的参知政事是邓润甫与熊本,庶务由他们负责。

而宰相,大部分的公事还是在韩冈身上,苏颂虽说不称病了,但政事处理,基本上还是交给韩冈,只有重要的人事安排,或是军国大事,他才会开了金口。

“太皇太后近来不豫,这几日都要辍朝,勉仲你就安心的多歇两日,等几日后再上殿。”

“是,黄裳明白。”

交待了黄裳一句,韩冈起身赶往宫中。

上一次辍朝,就说没有几天了,但不知怎么硬是给撑了过来。过了一个冬天,本来以为还能坚持一阵,但现在病情又忽然加重,让太后下令辍朝五日,并命人祈福于大相国寺。

不过辍朝并不代表太后不理政事,照样在内东门小殿召集宰执。

第一个五年规划即将宣告结束。

近五年的时间里,大宋的财赋、户口、建设,还有军事上,都有了长足的进步。

第二届政治协商会议下半年就要召开,下一个五年的国是制定,就不会像上一次般那么顺顺当当。

有着亮眼的成绩,韩冈便希望将国是的方向掌握在自己手中。

拿着政事堂总结出来的报告,纸面上的数字,便是韩冈所拥有的控制下一次国是制定的底气。

“户两千零四十万又四千一百七十三户,口四千五百二十六万又一千六百零九,比起五年前,户数增长了百分之七点九,丁口增加了百分之八点二。”

大宋的人口统计只计算丁口,也就是缴税的男丁,而不计算老弱和妇人。不过以壮年男子的人数来推定总人口,已经确定无疑的超过了一亿。

“纯以户口来计,历史上所有朝代,无论汉唐,文景也好,贞观也好,都远远不如今日。”

韩冈的总结,让太后欣喜的点头。有什么能够证明统治,一曰武功,一曰治政,而户口和人丁的增长,就是治政最好的证明。丁口和户数的增长,只能代表成年男子数量的增加,而因为死亡率的下降,而带来的自然增长率的狂潮,现在还远远没有到来。

韩冈道:“而近五年中,经过厚生司统计,种痘幼子总数量是三千八百万,平均下来每年七百六十万。”

七百六十万这个数字,远比丁口的增加更重要。

“这么多!”太后很惊讶。

宰辅们也很惊讶,太后是不是没看厚生司的奏章,全国上下每年生育数量,这可是极重要的内容。

尽管这个数字实际上根本无法统计,但从小不能接受种痘的孩子,日后能被计入丁口簿的可能性也不大——这跟有钱没钱无关,就是五等户或是客户之子,甚至是乞丐,也有一堆富户和寺庙想要积阴德,帮他们付账,而是是否被归入了朝廷的统治范围的问题——所以只要计算种痘数,差不多就能等同于生育数量。

“也就是说,天下每年都要增加七百六十万人?”

“还要减去死亡人数,才是人口的增长。不过十余年后,每年的确至少有三百万男子,三百万女子成年。”

七百六十万人中六百万成年,百分之二十的夭折率,在卫生意识业已推广到全国,种痘法等防疫知识深入人心的时候,韩冈已经将这个夭折率计算得很宽松了。

但一年三百万丁口,也是个让人难以置信的数量。二十成丁,六十为老,四十年乘三百万,四十年后那就是一亿两千万,总人口则是一倍还要拐个弯,少说三亿人。这还是没有计入增长的人口导致的人口增长——这其实是驴打滚的复利问题,这笔账,在场的宰辅们也都会算。那是比三亿还要多几倍的数字。

“每年三百万丁……”向太后忽然发现自己不会做算术了,这个数字乘以四十年后,实在太庞大了一点。

“三百万男丁,三百万女子,也并非全然是好事。每年丁口的死亡数量,也只有百余万,算上女子也不会超过三百万。一旦朝廷应对失措,就是每年要多上三百万乃至四百万食不果腹的男女饥民。这是下一个五年和再下一个五年,乃至百年之后,都必须要考虑的一件事。”

其实韩冈说的就是失业人口。在没有福利的时代,失业率超过五个百分点,就可以让皇帝睡不着觉了。

“吾明白了。”向太后点点头,又对熊本道:“如果每一仗都能与熊卿打下大理那么容易,吾日后也不用操心了。”

“不敢,”熊本连忙道,“这是陛下的庇佑,也是诸公在朝中一力襄助的结果。”

向太后对韩冈道:“相公之言,不可不虑。日后国是,这一条必须加以注重。”

尽管不知道什么叫做失业人口,但向太后还是明白,几百万百姓吃不上饭会是什么结果。

韩冈很早就公开计算过,当天下的出产不能养活天下的人口的时候,那么就是大乱的开始。向外拓张,这必须是百年以内绝不动摇的国是。

不论是在名义上,还是实质上,中国南方的国境线,基本上已经达到甚至越过了千年后的位置,西北也基本如此。而自河湟开边之后,西南的吐蕃诸部,以逻些城【今拉萨】为首的诸多部族、寺院,都逐渐派人入京,先后得到了朝廷的封赐,在名义上,臣服于中原王朝。

现在在军事上,唯一还在阻挠韩冈实现自己目标的对手,就是北方的辽国。那是最后,同时也是最强的外敌。但在开疆拓土上,攻打辽国却不是笔好买卖。

比如南洋周边的小国,打下来再容易不过了,只要能够克服疫病,就是上佳的开拓之地。就是克服不了,也能解决了中国本土大量出现的剩余人口。
