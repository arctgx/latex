\section{第27章 夙怨难解杀机隐(上)}

秦州。

都钤辖向府。

都钤辖府的主人,如今正是四十上下年富力强的年纪。每日清晨,他便出来习武练箭,打熬筋骨。冰雪无碍,风雨无阻,乃是标准的武将之为。

校武场中,向宝赤裸着健壮的上身,一块块线条刚硬的肌肉宛如最出色的石匠雕刻出来。他将一条大枪舞得矫如龙蛇,枪风呼啸声声。去了枪尖的枪头如毒蛇信子般吞吐不定,记记不离要害,把陪练的两名小校逼得步步后退。压得陪练无还手之力,向宝毫无兴奋之意,双眼瞪起,长枪边舞边吼:“你们就这点武艺?秦州可真是无人了!”

年长点的军校不为所动,沉稳如一,只将一杆枪左右遮拦。而另一名年轻一点的小校,不忿被小觑,枪势随即转急,枪尖在向宝眼前虚晃一招,反手枪尾直敲向宝胫骨。

“这样才够味!”向宝痛快的一声大喝,双臂猛然一振,手中大枪顿时化作千万虚影,滚滚枪影如同石子落水,自身周一圈圈荡开。狂风平地飙起,呼啸化为咆哮,只听得哐的一声脆响,一条长枪眨眼间就飞出了战圈。年轻小校双手空空的被捣翻倒在地,而年长的军校只稍稍退了两步,握紧长枪将门户守得谨严。

千重枪影合而为一,又恢复成一条大枪的模样。向宝挺枪待刺,眼角余光却瞥到向安不知何时站到了校武场边。他随即收枪撤步,跳到了圈外。就这么练了一阵枪术,向宝已是汗流浃背,身子热腾腾的直冒白气。一见场中的较量停了,校武场边的两名娇俏可人的侍女,忙拿着手巾上来要帮向宝擦汗。

向宝不理向安和侍女,先走到年轻小校身边,抬脚猛踹了一下,怒骂道:“一点激都受不了,日后怎么带兵?!”

小校忍着痛,翻身起来,磕头谢罪。向宝也不理他,转过身来,脸色就好看了不少,对年长军校笑道:“刘仲武,你倒是稳重,当是能带好兵。”

刘仲武虽说年长一点,也不过二十五六的样子。但目光沉定如潭水,喜怒不显于面,的确是一脸的稳重。他抱枪躬身,“多谢钤辖夸赞。”

“你做得好我就夸,做得赖我就骂,没什么好谢的!等我赏你再谢不迟!”向宝说话也有着武将的豪爽。他左右看看,抬手指着侍女中的一人,“刘仲武,你觉得惜奴她怎么样?”

都钤辖身边的侍女哪有长得丑的,唤作惜奴的侍女也就二八年华,身材袅娜,娇俏如花。刘仲武看了一眼,便收回目光:“钤辖身边人自是好的。”

“既然觉得好,那就赏给你了!”向宝干脆的说着。

刘仲武身子轻震,抬头惊讶的看着向宝。见向宝正盯着他,忙低头道:“小人不敢!”

“哪有什么敢不敢的!”向宝哈哈大笑,“你若喜欢,就带回家去铺床叠被,你若不喜,那就拉倒了事!”

刘仲武沉吟了一下,见向宝不似作伪,放下心来。他也洒脱,不再推辞,跪倒谢恩:“多谢钤辖厚赏。”站起身来,看着俏丽的惜奴,他心中感激甚深,一旁的年轻小校更是满眼的羡慕。

随便将美女赠人,向宝也不在意,他带兵一向是以严罚厚赏著称。摆了下手,“行了,你们都下去罢!”等校武场中再无第三人,向宝回身过来,方问道:“八哥,有什么事?”

在族中排行十一的向宝面前,向安说话简洁直率:“十一,王韶带着那个灌园小儿回来了。”

“韩冈?!”向宝脸色顿时冷了下去。如今在秦州城中说到灌园小儿,不会有别人,只有刚刚落了向钤辖脸面的韩冈。

“就是他!王韶和他是昨夜进得城。”向安为向宝分析道,“既然王韶将韩冈带回秦凤,看起来不再是张守约来举荐韩冈,而是改为他举荐……这措大,由得两家相争,当真是炙手可热。”

“管他是谁举荐韩冈,又干我屁事!”没了外人在侧,向宝也不必将心底的火气掩藏,他现在最不想听到的就是韩冈两字。

“话不能那么说。如果是张守约举荐韩冈只能是武资,而王韶来举荐,则应是文资。韩冈做了文官,就省得有小人为了攀附十一你,而跟韩冈过不去。到最后,也不至于被人说些泄恨报复之类话来……”

向宝嘿嘿冷笑:“那又如何?真当这点小事能把我打压一辈子?我向宝可是京营出身,天子面前留名!今天降一官,明天又能升回去。大不了换个地方,我照样当我的都钤辖。”

如今由于与西夏战事不断,西军系统水涨船高,渐渐有压倒河北禁军的势头。自澶渊之盟后,河北数十年不闻战火。就连河北禁军中的佼佼者,如杨文广之辈,如今都是在西北立功,继而才升任显官要职。不过论起真正受到朝廷重用的,还是以京营出身的将领为主。

即便当年京营出身的葛怀敏,本人顸庸无能,临战时指挥失措,突围时又犹豫不定,以至在定川寨惨败给李元昊,葬送了数万大军,可京营系统的地位依然不可动摇——要知道,三川口之败的主帅刘平,好水川之败的主帅任福,同样来自于京营禁军!

向宝虽然是关西镇戎军人,却是在京营禁军中混出头来。他自幼从军,以勇力过人而闻名。虽然没有经历大的战事,世间流传的只有他在五原射虎、潼关驱贼的传闻,但照样顺顺当当一路升到了御前忠佐马步军副都军头。外放后不数年,便已是秦凤都钤辖、皇城使、带御器械。

向宝的差遣——秦凤都钤辖,是执掌一路军事的第三号人物。本官官阶皇城使,也差不多到了外任武臣的顶峰。如果再升一步,就是横班——大宋百万军中,总数只有三十人的高阶将领。再上,就是基本上不实际领军的节度使、承宣使、团练使等贵官。而横班往往不满员,如今地方上实际领军的将领里,官阶比向宝还要高的,其实不过十几二十人。

所以向宝有自信,这么一点小事不可能让他一蹶不振。何况向安在伏羌城已经当众教训了家奴,向灌园小儿赔礼。回秦州后,向荣贵又受了家法处置,自家已经做到这般地步,任谁也说不出二话。到了天子面前,也不过是个持家不谨的罪名。向宝他真正丢的,其实只不过是脸面而已。

对!只是脸面……

向宝的脸上闪过一抹阴霾。堂堂一路都钤辖的脸面,却让一个灌夫的儿子给刷下来了。向宝怎么可能不介意,唾面自干的本事他可没有。

“王韶离不了秦凤路,他还要开拓河湟……”向宝狠狠地说着。

提举蕃部事宜本是他的权限范围,如今却被王韶夺了去,所有的功劳都跟他说再见。前两年他可是不辞下节的去跟蕃人打交道,也颇收服了几个蕃部。王韶平戎策上说的那些话,自己更是曾一条条的上书天子。只恨自家文采不够,找的门客又不会写奏章,反而让王韶占了先去,连过去的功劳都没人认了。向宝恨得不止是韩冈,还有王韶,

“韩冈为王韶所荐,自是也离不开秦凤路。不信他们日后不犯一点错,总有落到我手里的时候……走着瞧好了!”

……………………

熙宁二年闰十一月初一。

秦凤路经略安抚司管勾机宜文字王韶上书举荐韩冈为官,充任秦凤路经略司勾当公事,兼理路中伤病事宜。另外还有两份附带的荐书,分别来自于雄武军节度判官吴衍,以及与王韶重新沟通过的秦凤都监张守约。虽然韩冈没能如张守约所愿,但结下的善缘也没必要断掉,韩冈的才能正摆在那里。荐韩冈为文官,张守约没权力,但荐韩冈管勾秦凤伤病事他还是有资格的。

对于递上来三份荐书,经略使李师中判了个‘可’字,都钤辖向宝连歪嘴的机会都没有,便交由马递驿传运送,发往京中的通进银台司,最后呈到了大宋帝国的政务中枢——中书门下,也即是俗称的政事堂中。如果一切顺利,政事堂很快就会批下来,转发给流内铨【注1】。等到韩冈亲去东京将自己的三代家状呈上,并通过流内铨的审核,他就能正式成为大宋的一名从九品文官了。

而在同一天,在曾经在裴峡谷中袭击辎重车队的末星部被举族剿灭之后,陈举、刘显里通西夏一案终于开审。人证物证俱全,陈家在秦州世代豪族,积累无数,经此一案,怕是都要烟消云散,不知会富了多少官员。翻手为云,覆手为雨,韩冈不动声色,便让延续百年的乡土豪门陷入族灭之灾,让一千帐蕃部血流成河,自己却踩着人头得荐为官。一时之间,人人侧目。

也就在这一日,韩冈大清早便出了城去,沿着河畔官道,径直向东。只有与他亲厚的王厚和王舜臣带了几个从人跟着随行。

秦州最近的半个月,连下了三场雪,地面积雪其厚近尺。身在城外,又没有个铲雪的民伕,广阔无垠的雪原上,已经看不到道路的痕迹,只有通过河堤以及几座零星修在路边的酒肆、凉亭,才能确认出倚河而筑的官道位置。

注1:有品级的官员属于流内官,无品级的属于流外。流内铨是审核低品幕职官资格的机构,隶属于中书门下,为铨曹四选之一。

ps:陈举即将族灭,挡在韩冈面前的新敌人正式登场,也越来越强。

第三更。靠着各位兄弟的捧场,宰执天下在红票榜上的位置越来越高,如今已经在二十上下。人都是得陇望蜀,俺也不例外,下周上强推,恳请各位兄弟红票再给力一点,让俺站上首页。俺也会以最用心的文字来回报大家。再次保证,一天三更,决不拖延。

