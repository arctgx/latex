\section{第20章 心念不改意难平(四)}

【第二更,求红票,收藏】

韩冈闻言便吃了一惊,堂屋中也陡然静了下来,几个人都是目瞪口呆看着韩阿李。韩阿李则很平静的对儿子说着:“都送去,要做就做得大方点。”

韩冈感觉自家老娘的语气,平淡得就像是过去家里做多了菜,让自己给邻居家送一点过去,混没有将这么一大笔财富放在眼里。

他笑了起来,自家已经算大方了,想不到韩阿李更加豪气。两百两银,三百匹绢,说送就全送了。就是万贯家财的豪富,也没这般大方的。

一两银如今时价一千八九百文,但内库的银钱由于成色更好,甚至可在金银铺换到两千文,大约两贯半——因为省陌制的存在,一贯在此时仅为七百八十枚小平钱,只有加上‘文足’或‘足’,也就是‘一贯足’,‘一贯文足’才相当于一千文——而一匹上等的江南贡绢少说也值三贯上下。换算一下,这五百匹两银绢,大约相当于一千三四百贯左右。

拥有百贯身家就是一等户了,而一千贯在东京也许还不算什么,但在秦州城里,足以买到一间河西大街上的铺子,或是两座像韩冈家这种位置上佳、精美坚固的宅子。而在乡村中,更是可以买到普通的中田千亩,换作上等肥田也能买到三百来亩。

韩冈明白,韩阿李并不是不知道赐物的价值,才会这么大方。自家老娘对银钱财货清楚得很,往年入城卖菜,一文钱都不会算错,是精打细算的行家里手。但她就是这般毫不犹豫把价值一千三四百贯的财物全都送出去。

这就叫仗义疏财吧?韩冈想着。若是换个人有这样的性格,身边多半就能聚起一帮兄弟了。有这样对财帛不动心的母亲,韩冈也不用担心家里人会给自己在官场上拖后腿了。

不过最终韩冈还是没有照着韩阿李说的去做。依然是送一半,留一半。并非他吝啬,而是因为他还要留些做本钱。等赚到钱后,再给张载送些过去。韩冈想资助横渠书院,而且有着长期的打算。那他需要的就是细水长流,而不是一锤子买卖。

“前些天跟爹娘你们说起的事。朝廷已经批复了。以孩儿的官位,古渭寨外能拿到七八顷地。”韩冈又跟父母说起更为重要的另一桩事,“等过几天,孩儿把秦州城里的事情处理好,就奉爹娘搬到古渭寨去。房子是现成的,孩儿也已经让人收拾了,一切都已打点好,搬过去就能住人。”

韩千六没有二话。虽然一开始他心里还有些抵触,想在秦州城附近买地,但前两天韩冈已经跟他说得很清楚了,道理也分析得明白,再没有别的想法。他点着头,连声道:“有田就行,有田就行。”

韩冈点点头,这边没问题了。韩千六只想有些事可以做,老是跟和尚说话也没意思,做儿子的也不能不为他着想着。

“不过到了古渭寨后就不要再下田了,孩儿自会安排人手听爹指派。”韩冈想了想,又提醒了一句。要是韩千六照着过去的习惯,挑着肥料去浇田,韩冈他可是会被人骂不孝的。

韩阿李在旁边打着包票:“三哥儿你放心,不会让你爹犯糊涂的。”

“爹种田是把好手,有爹指点,古渭寨明年肯定有望丰收。”

被儿子夸了,韩千六笑眯了眼,谦虚着:“种田是看天吃饭,要老天爷答应才行。”

“你爹种田上是没得挑的,在下龙湾的时候,哪家要下种开镰,不先来问问你爹?”韩阿李也夸着丈夫,说起农活,这没几人能比得上韩千六的。

韩千六好得不得了,笑过一阵。又问着韩冈:“三哥儿,我们搬去古渭寨后,这里怎么办。要卖掉吗?”

韩冈摇头:“怎么能卖?这么好的宅子,秦州城里也没几处。现在卖掉,再买回来就难了。还留着好了,孩儿回秦州也有地方可以住。而且日后肯定也要搬回来的,不会一直住在古渭……孩儿会找个得力的。”

又说了两句闲话,韩冈见父母有些精神不济,便让严素心和韩云娘服侍他们回房休息。堂屋中就剩下韩冈和冯从义这一对表兄弟。

见韩冈视线扫过来,冯从义忙上前一步,“三哥。”

“你坐。”韩冈示意表弟坐下,“自家兄弟不须这般多礼。”

冯从义依言坐下来,但动作还是很拘谨,一张交椅,只坐了前半边,腰板着。就像蒙学里的小学生,一点也不敢稍动。

虽然他跟韩冈从血缘上算是很亲近,但两家多年没有来往,论关系,还比不上邻居。刚见面时还好些,只知道他这个三表哥是个官身,在秦凤有点名声。但看到他不动声色,就把三个哥哥都弄进了大狱,冯从义心中就开始有些畏惧了。

而到了秦州之后的这些天来,耳边传的、眼里看的,更满是韩冈的光辉事迹。从病愈后被迫当了衙前,到现在秦州城中能排进前二十的高官,用的时间竟然连一年都不到。期间他做下多少大事,让天子两次降诏褒奖。这些丰功伟绩,让冯从义在韩冈面前越来越放不开手脚。

对于冯从义的拘谨,韩冈已经见怪不怪,等熟悉起来就好。他问着表弟:“前些天跟你说的事,计划得怎样了。心里到底有没有底?”

听韩冈问起自己的得意事,冯从义来了精神,很肯定的点着头:“有!只是赚多赚少的问题。如果古渭榷场能赶在八月之前开张,今年年终前,就能把本钱翻上一番。”

韩冈不去细问冯从义想怎么做,琐碎小事就交给他处理好了。他本人只要看着钱到手就行。“那明天我就安排你跟着王安抚一起去古渭。先把事情熟悉起来,那里的榷场也没几天就要开张了,肯定能赶在八月之前……为兄与青唐部的俞龙珂、瞎药都有些交情,在蕃人中多少也有些名声,如果你跟蕃人什么龃龉,直接报我的名字,至少在青渭一带,基本上都会给为兄一点面子。”

“小弟明白。”冯从义点头应下。

“不过,做买卖最重要的是要公道,‘信’字摆第一。宁可亏本,也不能坏了名声。面子是别人给的,却是自己丢的。现在为兄在古渭蕃部中的名声已经勉强能算是金字招牌,不想砸掉它,我还想把买卖做得长久一点。”

韩冈虽然用着开玩笑的口气在说话,但眼神却越发的锐利起来。在过去……甚至在现在,不法奸商以次充好,蒙骗蕃人的情况也多有发生。这让许多蕃部只跟交往了几十年的熟人做买卖,这也是为什么当初陈举能影响并控制几家蕃部的原因所在。韩冈如今因为疗养院的事,在蕃人之中有些名望,不想因为贪图小利而破坏了。

冯从义变得更加严肃:“三哥放心,这番话小弟一定铭记在心,不敢稍违。”

韩冈对冯从义的的态度比较满意,“你明天还要早起,先去睡吧。省得明早醒不来。”

冯从义犹豫了一下,回头看了看堆在堂屋中的一堆贺礼。

韩冈会意,道:“这些礼物就放着这里,等明儿我想办法处理。”他拿起冯从义写的礼单,对照着礼物看了一下,基本上都给整理得差不多了,“剩下也没几样了,不费多少事。”

“那小弟就告退了。”冯从义行了礼后,回房去了。

堂屋中只剩他一人,韩冈拿着礼单又看了看,直咂着舌头。看起来他家所在的街坊,果然都是些深藏不露的大户。不过礼尚往来,现在收了人家的贺礼,等日后也得还礼回去,韩冈倒是不想贪着些便宜。

过了一阵,韩云娘一个人从里屋出来了,韩冈往她身后看了看,不见严素心的身影。

“素心姐姐回去陪招儿了。”小丫头现在越发的心思灵透,不等韩冈问,便把话说了出来。

韩家父母的里屋还有个侧门,出门后走过只有一丈多长的雨廊,就是严素心和韩云娘她们的屋子,并不是每次都要从堂屋进出。

被小女孩儿看透了心思,韩冈却也不觉得有什么尴尬。说起来两个女孩私下里不知是怎么商议的,现在是一日一换,轮着服侍韩冈。不过在韩云娘来的时候,最多也只是搂着说些话,却不可能做到最后。

严素心自从给韩冈收房之后,才半个多月的时间,就变得丰润了起来,行动时,腰肢扭动也不同过去,兼有着少女和少妇的风情。如同一颗半边鲜红了的苹果,咬过一口之后,让人忍不住想把她变得彻底红透。

而韩云娘正处在从女孩向少女转变的过程中,青涩渐渐退去。原本过于纤弱的身材,渐渐长开,开始有了日后风华秀丽的影子。

这不同时期的女孩,各有各的风韵,当然让人没法儿评出高下来。

拥着韩云娘娇嫩软馥的身子,嗅着她身上的香气,说了些体己话。洗了澡之后,韩冈自去睡了。第二天清早,王韶陪同着李宪,还有两人的一众随扈,一齐出现在秦州城的东门外。而韩冈,领着他的表弟也一起到了。

