\section{第45章 千里传音飞捷奏(中)}

一个人,一句话,一份捷报,让局势彻底扭转。

王惟新带回的王韶胜利的消息,就像秋日草原上的野火,一下就传遍了整个河湟之地。

王惟新是绕道岷州过来,没经过河州,所以苗授和二姚,都是得到了狄道城的通知,才知道胜利的消息。

苗授在听到了狄道城加急发来的捷报后,狂笑了一刻钟之久,接着又连声呼酒,竟然大醉了两日。而姚兕、姚麟在吃惊之余,便暗自庆幸自己没有硬是要撤军回泾原。在河州的一万多宋军将士,是欢呼雀跃,王韶即已尽全功,他们的封赏自然不用担心被打折扣了。

等到狄道城所派遣的露布飞捷的金牌急脚经过陇西城的时候,原本浮动的人心,都被一下镇住。王厚兴奋之余,也疲于交接——王韶献奇策、立殊勋,已经有了进入宰执班的资格,作为他的儿子,王厚自然是成了众人瞩目的中心。

捷报向西传去,传在吐蕃人的耳中,却是再令人恐惧不过的噩耗。

青唐羌中,还在拮抗的对手,只剩最后的吐蕃赞普。大宋兵锋直逼青海之滨,势不可挡,让青唐王城内外惶惶不可终日。青谊结鬼章彻夜未眠,董毡则是当场砸坏了酒杯。卧榻之侧,岂容他人酣睡。这一句话,大宋周边的邻居们,只有辽人可以不当一回事。

对于弱小的土著部族,身边有个虎视眈眈的恶邻,又有谁人能睡的好觉?不仅仅是董毡和他领下的部族,兰州的禹臧花麻,更是愁眉不展。

不只是因为已经在河湟站稳脚跟的宋人,兴庆府已经有消息说,为了防止宋军攻打兰州,并以兰州为跳板,北上兴灵。已经准备扩大兰州驻军的规模,将现有的两千铁鹞子增加到万人,同时增添的粮秣消耗,却是要禹臧家来解决——下一步该怎么做?禹臧花麻陷入了犹豫之中。

值得吐蕃人庆幸的是,现在不论是哪一方,人人都知道这一场河州大战,终于到了结束的时候。大宋在河湟的地位因此战而确立,但大宋对于河湟之地的攻取,现在却也得暂时告一段落。并不会继续紧逼湟州,也不会立刻进攻兰州。他们都还有时间来考虑自己未来的道路。

狄道城这边,在等待王韶回师的这段时间中,韩冈顺服的听从了蔡曚的处置,闭门思过,待罪听参。

将繁重的公事丢到一边,读书、习文,为着八月份的解试做准备,韩冈的日子过得很是惬意。他现在就等着王韶领军凯旋,不过这一次,王韶和高遵裕应该不会翻越露骨山,当是沿着洮水河谷,经过岷州,向狄道城过来。

收复洮州、迫降木征的捷报,在出乎意料的时机送到手上,让韩冈预备的几个后手成了无用功。

请罪的奏疏早早的就已经送去了京城,韩冈可不会在伪传了诏书之后,不知及时补救,最后在天子的心中留下一个恶劣的印象。郭逵曾经有过现成的例子,韩冈知道要脱罪,要翻盘,具体该怎么做,他都有参照的对象。

想要翻盘,就是要在京中起波澜。就算没有王韶的捷报,韩冈也不会坐以待毙,釜底抽薪的手段,他更不会弃而不用。

给王安石的信,给章惇的信,给天子的奏疏,都在确认了临洮堡的胜利之后,以急脚递发送了出去。韩冈甚至说服了王中正,让他密奏天子,追回撤军的诏令。而李宪那边,韩冈这几日其实都在旁敲侧击,试图影响这位被派来体量军事的使节,让他也成为坚持保住河州的盟友。

如果王韶没有回来,这番布置将会是扭转局势的关键。韩冈相信,以天子对开疆拓土的热切,让他回心转意难度并不大。最低程度也可以让自己脱罪,并为日后卷土重来做好准备。

但眼下这些准备,随着来自洮州的捷报,已经没有必要再进行下去。不过这一番布置,并不是没有别的好处。

奏章和书信,至少体现了韩冈对战略局势判断的正确性,以及个人立场的坚定。这个表现,上至天子、王安石,下至王韶和更下面的将领,都看着眼里。对韩冈的声望有着推波助澜的影响——还有同样也上了密折的王中正,他也肯定也会因此而受益。

因而这些天来,王中正心情好得无以复加。当日一听到王惟新千辛万苦带回来的捷报,便立刻摆起谱来,回转衙门,不去出城迎接吕大防和蔡曚一行。现在更是乐得看蔡曚和吕大防的笑话,跟韩冈一样,在自己的住处杜门不出,将最后一点手尾做得尽量完美。

这是运气吗?

也算是。

但坚定的意志,才是关键。

韩冈一向意志坚定,吕大防派人来请他议事,他直截了当就拒绝了,“现下韩冈是待罪之身,静等朝廷的责罚,如何敢随意行动?至于公事,自有蔡运判全权处置,韩冈又有什么不放心的?”

吕大防的面子,韩冈不是不想给。但伪传诏令不管结果如何都是个罪名,这认罪的态度更为重要。而且看着蔡曚焦头烂额的模样也很有趣。

韩冈袖手不理公务,他身上的重负当然都落到了蔡曚的身上。要钱的、要粮的,要夏季应当发下的衣服和药物的,都一窝蜂的去找蔡曚,就连应该预备的吕大防和蔡曚的接风宴席,也是要蔡曚自己来准备。最后是沈括看不下去,才出手帮了一把。

而韩冈在事后也因此而训斥了手下的官吏:“别犯蠢事,蔡曚说什么,你们就做什么。莫多话,也不要推托。别落下把柄在他手里。他前次吃过亏,今次可是带了几十号人来,不会像前次一般,想杖责都没人拿棒子。要是给他找出了错来,拿你们撒气,我也只能干看着。”

“……可是……”

“没什么可是的,照常行事,不要拖延推诿,吕御史就不会让他乱来。”

韩冈倒不是心软,故意使绊子毫无必要,蔡曚也是他人门下走马狗,真有问题也是他身后的人。且就算能成功,也逃不过吕大防的眼睛。传出去,不仅会让蔡曚得到同情,甚至还会连累到自己的名声,还不如一切如常行事。

蔡曚本人是个可笑的废物,韩冈做起来都吃力的工作,更别提临时接手的他了。这样的状况,吕大防也无话可说,只能变成了现在的情形:吕大防遣人三请四邀,再三的请韩冈出山理事,而韩冈不加理会,正坐在桌前,翻着孔颖达的五经正义。

唐人的注疏还是追循汉儒的陈迹,孔颖达对经传的释义,与董仲舒、扬雄、郑玄等汉代大儒一脉相承。但宋儒对此早已看不惯,上承三代,直溯本源,一向自视甚高的宋代士大夫中,不只一个大儒在这么说。

韩冈觉得路中的解试时,参照汉唐注疏应该不至于有问题,没听说蔡延庆在经传释义有何发明。但如果礼部试时依然照着旧日的注疏去写,恐怕会泯然众人,而给刷落下去。

推测出题人——也就是明年的知贡举——的身份,韩冈有九成以上的把握可以肯定,标准答案最好是王安石的学术为依据。早年王安石所著的《淮南杂说》,韩冈手头上就有一部。是韩冈透了口风后,章惇使人寄来的。其中基本上都是王安石对经传的个人理解和观点。对于宋儒中,只知道横渠、二程两家的学说的韩冈,有着不小的帮助。而且也是今次进京考试时,最好的参考答案。

章惇如今跟韩冈关系紧密,在前一封信中,章惇已经说了,他很快就要受命去荆湖两路巡阅,目的便是荆湖南路的辰州、潭州、邵州,收复梅山、飞山等地的蛮夷。

就跟大宋的南方诸路一样,荆湖两路驻军的水平,恐怕连西军的脚跟都比不上。章惇知道从关西调兵不易,所以就只要几个能派得上用场的将领。刘仲武是章惇之父章俞的救命恩人,肯定不会被落下。而韩冈也把自己的表兄李信推荐给了他。

当时写信时,西贼尚未出兵围攻德顺,李信也没有领军去救援。韩冈是看着李信因为跟着张守约,几次三番的错过大战而官位停滞不前,所以才想让他去荆湖走一遭。但今次德顺一役之后,就不知会怎么样了,但想来章惇还是能要过去的。

此外章惇要的便是合格的医师、护工,最好是一个完整的团队,以便让征服荆蛮的大军,不至于受疾疫之苦。韩冈现在正犹豫着,不知该推荐朱中还是雷简。朱中勤学好问,又善于安抚士卒,在疗养院中人望很高。但雷简本是京中医官,他的医术在朱中之上,南方的病症他处理起来当比朱中更为得心应手。

韩冈也没想太多,大不了随便点一个就是了。自己现在的注意力,当还是放在复习功课上。

正这么想着的时候,外面的亲兵进来禀报,说吕大防吕御史前来拜访。

韩冈一笑,‘终于亲自来了’。

“快请!”韩冈起身相迎。

