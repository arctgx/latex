\section{第17章 往来城府志不移(四)}

皓月当空。

夜空中并无一丝云翳,月中的清辉毫无遮挡的撒落下来,给嵩阳书院一角的小院中的树木、房屋、地面,都镀上一层淡银色的光泽。

透过手中的黄铜圆筒,那轮散发着银色辉光的圆月,似乎就变得近在眼前。

游酢眯着眼睛,将镜筒一头贴着眼睛,不意声音从身后传来,“定夫,好雅兴啊。”

游酢闻声转回身,却是同在二程门下的杨时和谢良佐。

“原来是显道兄【谢良佐字】和中立兄【杨时字】。”游酢放下手上的千里镜,笑道:“中立兄这几日路上奔波,怎么没有早点歇息?”

杨时三年前中进士,得受官阙却不赴任,而是在程颢门下求学。不过之前都在洛阳,今天才到嵩阳书院中。

“一时没有睡意。”杨时走过来,笑道,“倒带累显道都没得睡了。”

“半夜观月,雅兴不浅。”谢良佐走到游酢身边,抬头望着天上的圆月,“可是起了诗兴。”

“诗兴?”游酢笑了起来,将手上的黄铜镜筒递给谢良佐,“用这个看了就没有了。”

谢良佐接过有些沉重的镜筒,睁大眼睛:“这就是千里镜?节夫前几天托人送来的?”

杨时闻言动容,“是洞烛千里,远观日月的千里镜!”

“没那么夸张,不过就是将远处的景物放大个几倍而已。远观日月也不确切,要是望着太阳,会照瞎眼睛的。就跟用放大镜点火一样。”游酢摇头道,“也只能看一看月亮。”

借助千里镜望着月亮,银盘中那朦朦胧胧、惹人遐想的暗影,却便成了稀奇古怪的斑点。瞅着原本肉眼看去如桂如兔的阴影,现在却如同一张麻脸的斑斑痕迹,游酢都不知自己日后怎么再去写有关月桂、玉兔的诗句了。

谢良佐拿着千里镜摆弄了一阵,倒是很快就知道怎么使用了,对准天空中的星月,将眼睛贴上去。

杨时看到谢良佐迫不及待的样子,不由得笑了。转头过来,又对游酢道,“对了,还没来得及恭喜令兄今科高中。”

游酢拱手还礼:“多谢中立兄。”

杨时指了指正拿着千里镜冲着天空啧啧称叹的谢良佐,“听显道的话,这千里镜也是令兄托人送来的?”

“家兄知小弟喜好这些奇巧之器,所以送了一具来。”

游酢的兄长游醇曾经做过韩冈的幕僚。因为在熙宁七年河北灾荒的时候,辅佐韩冈安置了数十万流民,因功授官。熙宁九年没有考上进士,便在二程门下苦读,如今终于考上了进士,上个月已经回到京城候阙了。

一架千里镜的的价格虽不低,但想买到则更难。就是京城的药玉作坊终于能出产跟大食的玻璃器皿一样的透明玻璃,但想要从中要挑选出无气泡和扭曲、能够磨制成镜片的玻璃片,依然是百里挑一。也就是游醇曾经在韩冈幕中做事的经历,让他能在军器监中攀上关系,可以买到产量稀少的千里镜。

杨时前段时间在洛阳,很清楚千里镜和显微镜如今在显贵子弟中有多么受追捧。那些衙内们吃喝玩乐腻味了,显微镜和千里镜成了他们之中流行的新目标。玩物丧志的议论也是有的,但只要种痘法还在世间流传,这样酸溜溜的话,只是自取其辱。而且千里镜的用处,只要抬头看看,就是一清二楚。

“千里镜乃是军国之器,与飞船配合起来,几十里外的敌军也瞒不过天上的眼睛。”杨时说道,“如今京城那里透明的玻璃也有了,将作监和军器监正在鼓足全力制造,准备给军中全都配发上。”

“千里镜是一个磨镜匠献上来的,去年一出世就流传开了。不过之前白水晶价格太高,大多数给磨制成了眼镜和放大镜,官宦人家正时兴的显微镜又要占去一大部分,千里镜的数量很少,就是想找到一块镜片都难。如今有了玻璃,日后当会越来越多。家底差一点的人家,以后也能买得起。”

“银河中果然都是星辰,不用千里镜,当真分辨不清。”谢良佐放下千里镜,回过头来道:“这样的军国之器,国人买得起倒也罢了,要是给辽人得去,可就不妙了。”

“就是想守秘也守不住,凸透镜和凹透镜的原理,早就给公诸天下。显微镜和千里镜的原理也是一样。只要是个手艺不错的匠师,有样品在面前,花上一点时间,终究还能仿造得出,就跟飞船一样。”

“早知道会从凸透镜和凹透镜上,引出显微镜和千里镜,天子肯定会下诏严禁韩冈的书作刻印贩售。记得前两年有个叫钟世美的国子监内舍生,他上书为天子所赞。国子监本要刻印他的文章,天子亲下诏说其文中‘有经制四夷等事,传播非便。'韩冈的书,可比些书生之见对四夷更有用。”

谢良佐又拿着千里镜去看月亮,一边还说道:“或许韩冈早就知道会如此,才泄露出一星半点。只要能传播出去,天下间总有才智之士能将之补全。”

杨时哈哈大笑,“或许。”却是不信。

游酢则沉吟起来,他倒觉得韩冈有这份心术。受到其兄长游醇的影响,在程门的弟子中对韩冈主张的格物之道是最有兴趣的一个,这也是游醇为什么要捎个千里镜给游酢的缘故。

“说起韩玉昆……”杨时像是想起了什么,“不知定夫你听说了没有,他这一次又要入京了。”

游酢略感惊异,抬眼问道:“太原知府不做了吗?”

“改判太常寺兼提举厚生司、太医局。”

“怎么是这几个差事……”游酢狐疑的问着。

“还有端明殿学士,之前的龙图阁学士还继续兼着。”谢良佐一边说话,一边继续望着天上月亮。

游酢惊讶起来,“还能继续兼着龙图阁?这不是已经在司马端明之上了。”

杨时摇头道:“品藻人物,岂在官位?司马君实一心只在独乐园中修通鉴,哪里会在意官阶。”

洛阳城中致仕的老臣数十,日日筵席不断。司马光这几年,时常与富、文等元老游宴,倒也不是那么死板的。游酢笑了一笑,却也不说什么。

杨时道:“端明殿一职,不过是酬韩冈前功。判太常寺,还有太医局、厚生司两个兼差,才是官家想要用到他的地方。”

谢良佐终于收了千里镜,走过来还给游酢。“礼家如聚讼,虽兄弟亦不容苟同。韩玉昆司掌太常,必是不甘寂寞,日后有的是笔墨官司与太常礼院打了。提举太医局和厚生司,则是天子是想用其才,任其能。从这几项任命看,天子当不准备让他有机会再立新功。”

天子的私心,当然瞒不了人。韩冈已经有好几次有机会入居两府,但都被天子给挡下来了。天子对韩冈的忌惮都成了公开的传言。民间对此颇有些微词。不过年轻点的士人,或是一般的官员,只要不是气学门下,或是与韩冈利益攸关,其实都不想看到他出头,二十多岁就出任执政。

游酢摇摇头:“韩玉昆若想立功,太医局和厚生司都有立功的机会。还不知他藏了多少本事,等着放出来呢。”

杨时反驳道:“除了痘疮,还有什么能致人死地的疫症,能病愈之后不再复得?须知韩冈本身可是对医术一窍不通,”

“这是韩玉昆自己说的?不能全信。”

游酢从他兄长那里听过了许多有关韩冈的经历,从不觉得韩冈是那种一板一眼的君子,多少次都让天子、宰辅和元老重臣都无可奈何,最起码的城府不会缺的。当年在白马县断的那个案子,不是心术过人,怎么能轻易解决这一桩困扰三十年来白马县历任知县的积案?

“但他瞒着自己的医术又是为了什么?”杨时却是不信。通晓医术又不是什么丢人的事,完全没必要隐瞒。

谢良佐也道,“韩冈也曾求学于两位先生门下,说起来也算是半个弟子。虽然有门户之争,但两位先生可从来没有说过韩冈人品堪忧。前几天,正叔先生还说他在敬字一字上做得甚好。”

当年韩冈立雪程门,使得如今洛阳连年画上都有他的形象。就像司马光砸缸的年画,在民间已经传了几十年。韩冈的形象同样的流行。顺带的,连程门也在这个韩冈为主角的故事中得到了极大的助力,嵩阳书院中的弟子越来越多,也是因为程门名气渐髙的缘故。

道学最重师道。敬师,方能传承道统。韩冈对师长的尊敬,影响了很多程门弟子。一个气学弟子,只因听过几句教诲,就对大程小程两位先生敬重如此,身为门下嫡传,又怎么能输给他?倒让程门之中的风气更为谨严。

不管怎么说,从尊师重道上,还有用兵抚民上,韩冈的名声在世间算是顶尖的,能力和德行都一流水准。说他人品不好、或是心术不正,可是要做好被人痛揍的准备。就是程门之中,贬低或驳斥韩冈的学问没问题,但指斥其人品却是少不了会被同门反驳。

