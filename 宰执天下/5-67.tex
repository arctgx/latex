\section{第九章 拄剑握槊意未销(15)}

【第二更。】

回到群牧司中,也不过是辰时。

处理了今日的公务之后,一摞抄件就送到了韩冈的案头上。

浅黄色的标准公文笺上,一列列端正的三馆楷书墨迹未干。每一份都是以某某官职加臣某开头,全都是奏章——而且是抄件。

这些奏章的抄件,全都是顶级的机密军情。除了两府和枢密、中书两处的寥寥数位高官以外,其他人没有资格查阅,只能依照职司不同,看到转摘出来的条目。

韩冈若还是仅仅是同群牧使,照规矩他就只能看到牲畜的损伤数字,其他数据只能通过传言得知。

不过韩冈已经直接参与到军机中来。为了能让他能尽早掌握最新的军事情报,免得上了殿后,还会因为情报不明做出错误的判断,或是耽搁宝贵的时间向他通报军情,韩冈在几天前便得到了同枢密院都承旨的差遣。

枢密使在职位上惯例是兼任群牧制置使,群牧使、副使则一贯兼任枢密院都承旨、副都承旨。韩冈在担任同群牧使之后,却并没有得到兼差。现在的职位,本来就是一个让他歇歇脚的冰窖,但眼下的时局,却不得不让天子给韩冈更大的权力。

不过韩冈眼下只看送来的情报,至于枢密院都承旨的实际工作,那是韩缜的职权范围,韩冈无意去跟他相争。也懒得争,只看这半个多月,韩缜忙得都没有来群牧司衙门一趟,将衙中所有的事务都丢给了韩冈,就知道枢密院都承旨的差事可不是一桩轻松的活计。

送到韩冈案头上的文档,基本上囊括了昨日晨间到中夜,所有送进京城的紧急军情。大体的内容,韩冈其实在早朝前便已经在发给他的简报上有所了解,但细节才是关键。许多时候,细节上的些微助力,都有扭转局势的机会。

排在第一页的是泾原军上报的伤亡统计。

已经放弃宥州的一万一千余人战殁和行踪不明,轻重伤一万六千,伤亡近半。已入流品的军官则伤亡二十七人——这是个很可怕的数字,绝大多数统率一个指挥四五百人的指挥使也不过是个殿侍,基本上都不到从九品。失踪、战殁和重伤的将校总数为二十七人,已经达到了出战军官的三分之一,其中还包括苗授和其子苗履。一般来说,在战场上,官位越髙,伤亡率就越小。从将校伤亡的数字来推算,泾原军的伤亡报告算是比较准确的。

看着表后列下的伤亡名单时,韩冈叹息摇头,上面有好几个姓名还是他有印象的,对他们的评价并不低,想不到就此毁于一旦。

就是三川口和好水川都没有这么夸张的伤亡比例,也就是当年定川寨之败,葛怀敏犯蠢,硬是往党项人的伏击圈中闯时,才有了这么大的将校损失。已经不是伤筋动骨这么简单了。还是深入敌境的缘故,一旦败阵,伤亡。如果换作是在境内失败,怎么逃都容易。

而环庆军的伤亡报告到现在还没有奏报上来,也不知高遵裕那边是怎么一回事,苗授重伤都没有耽搁,他倒好,比殿后的泾原军提前至少一天抵达韦州,动作却是慢条斯理。

高遵裕是完了。在兵败之后,他连上奏本,弹劾苗授不从军令,疏忽大意,没有觉察党项人的奸谋。今天排在第二的抄件就是高遵裕兵败以来的第四份奏本,在弹劾随军转运使李察措置无方的同时,也没忘了再将苗授拎出来骂上两句。

其实朝堂上都看得出来,高遵裕是在推卸责任,天子也对此很是反感,已经到了下诏痛斥的地步,想来这两天就该送到高遵裕的手中了。但苗授也少不了要治罪,高遵裕的弹劾并不能说完全没有道理。但苗授现在身受重伤,如果他就此伤重不治,不但不能加罪,还得加以褒奖,否则无意激励将校为国尽忠。只有等他康复才好治罪。

然后第三封则是高遵裕、苗授的联名奏请,请求朝廷允许他们放弃韦州,撤回境内,以免留驻境外,以至于军心不稳。

看到这一份奏章,便可知泾原、环庆两军算是废了。而且必须要及早遣官去前线安抚军心,否则下一封奏章上就不会是简单的军心不稳四个字。

而第四份还是来自于韦州,上奏本的是军中的走马承受。密奏称前日万余西贼追击而来,高遵裕在城中坐视不前,让贼军在城外耀武而去。而贼军的去向,竟是鸣沙城。

这分明是要抄截秦凤、熙河联军的后路!

幸好早上在收到简报时便有了点底,韩冈才没有跳起来。

而且昨日收到王中正发来的金牌加急,便已经说泾原、环庆兵败之后,西贼全师来攻,其所部与贼鏖战多日,杀贼数万,但折损亦多。已是独木难支,又恐有前后被夹击的风险,请求自葫芦河撤入泾原路境内。

王中正有临机决断、便宜行事的权力,必然是先行动然后才发奏报,眼下当是已经退入了泾原路境内。

不过这么一来,铁鹞子就能从应理城直插凉州身后,正在攻略河西的王舜臣可能会有危险,韩冈不能不为他担心。当然,王中正如今是无功而返,只为自己考虑,也会设法增兵凉州,应该会有人提醒他的。

然后是秦凤转运判官游师雄的奏折。他本来是负责熙河路方面的转运,不过当王厚在战前被确定主持了熙河转运之后,他就被调往泾原的渭州辅佐李察。这一个是因为他在环庆甚有威望,当年广锐军之叛,送给叛军的第一场败仗,就是游师雄在邠州指挥;另一方面,泾原路也是属于秦凤转运司的管辖范围,游师雄在两路都说得上话。

游师雄的奏折,是上报近日每天都有数十近百的逃兵逃入境内,请求朝廷对其人优加抚慰,不要依军律处置,以防兵变。从游师雄的奏折中看,他已经开始这么了。这一封奏报,从侧面解释了前一封高遵裕和苗授两人联名奏章的并非是杞人忧天。

后面还有鄜延、河东、河北等处事关军情的奏报,总共有二十余份。

将二十几份奏报匆匆浏览了一遍,韩冈从怀里的暗袋中掏出一本册子,对其中一些关键的信息进行简短的记录,以备使用。又与前些天收到的数据相对照。为了更加直观,韩冈前些天开始,就做了几份简单的图表,来加强对比,今天收到了新的数据,便动手在上面添了几笔。

将几张图表在公厅中张挂起来,韩冈搓着颌下短须,皱眉凝视着代表总兵力的那条下降坡度越来越大的折线,损失之大超乎想象,党项的铁鹞子在绝境中表现出来战斗力的可见一斑。不过在这几战中,他们的损失又该有多少?

结束了一天的工作,橙红色的夕阳下,韩冈回到了家中。

迎上来的管家向他通报:“龙图,黄秀才来访,正在小厅中等候。”

黄裳今科又落榜,不过他在国子监读书之余,也经常上门请教格物之学。韩家的家丁都知道韩冈很看重这个屡考不中的福建秀才,待客唯恐有哪里怠慢。

“哦……黄勉仲来了。”韩冈点点头,“我换了衣服就去见他。”

换了一身家常袍服,顺便冲了个澡,韩冈来到接待熟客的小厅中。

厅内摆着冰块,阴凉得很,黄裳悠然自在的坐在厅中,手上拿着卷书册,慢慢的翻阅。

“龙图。”听到韩冈进来的动静,黄裳放下书,不徐不急的站起身,向韩冈行礼。

回过礼,韩冈与黄裳分宾主坐了,信口问道:“勉仲方才在看什么书?”

黄裳拿起小几上的书卷呈给韩冈:“是苏子容学士新近出版的笔记《思闻录》,里面有一部分关于天文仪象的内容。可惜印数甚少,费了好一番力气才借了出来。。”

听了黄裳的话,韩冈微微一笑。大概是因为上次见面时,送了他一架显微镜的缘故,黄裳如今对格物学兴致盎然。就是没有后世扬名的道藏、武典,在格物学上有所成就倒也是不错。

“论起天文仪象法度,朝中当无人能及苏子容。”韩冈道:“他的这一部《思闻录》,我书房中也有,前些日子让印书坊制版成书后,就送了我一部。若勉仲有兴趣,借去也无妨。”

黄裳一听,连忙起身谢了。

应该是韩冈所著的《桂窗丛谈》的影响,现如今,有关格物的笔记渐渐的多了起来。沈括的新书正在筹备,而苏颂的笔记已经出版了。

和韩冈聊了一阵格物学术上的问题,黄裳忽然道:“龙图是否知道,余正道今天被捉去了御史台,他已经是第七个了,再过两日国子监就没直讲、教授了。”

“听说了。”韩冈点点头,“只是对其中的内情了解不甚深。”

“……不过这倒是小事。”黄裳见韩冈对此事不在意,也就识趣的不提了,转而问道,“黄裳在外听说辽人十万大军已至大同府,是不是辽人要南侵了?”

“哪有十万?契丹骑兵一人三马,十万骑,就有三十万匹战马,西京道可养不起那么多。要多打几个折扣。且说到他们入寇,也当不至于如此。撕毁维持了七十年的盟约,不论是在大宋,还是在辽国,都不是那么简单的一件事。不是权臣说上一句就能决定的。耶律乙辛也不会愿意冒太大的风险。”

“西夏虽小胜,但官军犹占据银夏。唇亡齿寒,耶律乙辛难道不会出兵援助西夏?”

“这跟耶律乙辛何干?”韩冈笑道:“辽国还不是他的。一个谋国权奸,勉仲你说他是会为自己的身家性命和权位考虑?还是会为辽国的未来考虑?为万世开太平的想法,会存在于耶律乙辛的心中吗?……不过话说回来,朝廷也不会将信心放在耶律乙辛身上,河东路是肯定要加强防备了。”

