\section{第12章 平生心曲谁为伸(九)}

【不好意思,这两天都忙着走亲戚,坐在电脑前的时间太少,请各位兄弟见谅,等过两天一定都补上。】

位于城西北的魏楼,市口不及惠丰楼,清幽不及晚晴楼,酒菜水准则比不上郝家园子,就连建筑,也不过是一座普普通通的两层楼阁,在秦州城中的几家大酒楼中,只能敬陪末座。

但魏楼有一桩好处,就是地基是建在一处四丈多高的台地,使得楼阁凭空高了三四层去。在楼上凭栏而坐,只要有着一对好眼力,便能将城北数里之内的动静一览无余。

韩冈和杨英此时正坐在魏楼二楼的雅座中。桌上摆着七八盘下酒菜,两副碗筷对放着。不过只有韩冈安坐在桌旁吃菜喝酒,而杨英却没怎么动过筷子,除非韩冈举杯相邀,否则他连酒杯也不碰。总是跟在王韶身边的这位亲信,自坐进来后就是一副心神不属的模样,时不时的站起身,透过敞开的窗户向外张望。

见着杨英又一次站起身,韩冈终于放下筷子,笑道:“杨兄弟,不用这般心急。一切谋划抵定,窦解也已毫无所觉的跳入陷阱,事情顺利得很,杨兄弟你何必忧心。”

“啊……是,抚勾说的是。”杨英凭栏望远,心不在焉的答着韩冈的话,心神依旧放在楼外的夜色中。

韩冈无奈的摇摇头,拿起酒壶,给自己的酒杯斟满。

杨英在瞪大眼睛观察着秦州北城动静之余,也偶尔回首房中。不是见着韩冈自斟自饮,就看看到他拿着筷子大快朵颐。

在针对窦舜卿的谋划逐渐推进,正进行到最紧张的时候,连机宜都忍不住派了自己过来打探消息,但韩冈这个主事者却依然能安坐如山,悠闲自在。长时间的紧盯着楼外夜幕下的城市,两只眼睛都已经开始发胀发痛的杨英,不知自己是该敬佩还是该生气。

但韩冈的心中并不似他外露出来的那般镇静自若,看似自得其乐的喝酒吃菜,实际上却是食不知味,担心着局势的发展偏离他所希望的方向。他与杨英一样都在焦急的等待着……等待着代表计划顺利进行的那一个标志的出现。

任何计划在施行从来都不会一点错也不出,事先规划得越复杂越完美,最后在施行的过程中就会扭曲得越厉害。韩冈已经将他制定的计划简化而又再简化,尽量能做到一切顺势而为,只在聊聊几处关键的地方让人推动一下,让时局发展的方向转到他所想看到的地方。

就如韩冈让王九在城中传播的流言,除了最后说王启年在家里留下了证据这一点外,其他几条都是实际发生过的,没一句虚言。秦州城的百姓都知道窦七衙内这半年来造过的孽实在罄竹难书,但因为他祖父的关系,却没人敢将之曝光出来。而现在关于窦解做过的好事的流言传出,吃过他苦头的受害者或是亲眼见证过他嚣张跋扈的旁观者却都会暗地里为之作证,并将之推波助澜。

所以王九等人所要做的,仅仅是在喝酒和闲聊时随口说上这么一句——‘喂!窦副总管家的七衙内的事,你听说没有……’完全不必要担心有人能查出源头。

而计划中剩下的几项也都是这样,用不着手下的人去冒什么风险,仅仅是举手之劳,但韩冈依然没有百分百的把握能肯定一切都会照着他预定的方向发展。

幸好窦解已如他所愿,终于到了王启年家。现在,最初制定的计划已经进行到最关键的一步。为了亲眼确认计划的成功,韩冈便来到了魏楼之上。

这个计划,韩冈没有并瞒着王韶,高遵裕那里他也是隐隐约约的透露了一点。为了表示对他的支持,王韶在儿子去了京城的情况下,便派了杨英过来压阵。高遵裕虽无心插手,但等到韩冈的计划成功,他自会出手给摇摇欲坠的窦舜卿全力一击。

“抚勾!”杨英突然猛地回转身来,方才焦急难耐的烦躁已经全然不见,变得眉飞色舞,喜上眉梢。他竭力压低了自己兴奋的声音,“净慧庵火起了!”

“哦,是吗?”韩冈淡然的一问,透出了一切尽在掌握中的自信,却将内心的真实感受完全掩藏。享受着杨英崇拜的目光,他站起身,走到窗边,远眺两里之外那一朵如夏花般绚烂的火焰,

“就不知傅勍什么时候到了……”

……………………

“前面转过去就是净慧庵!”

一声兴致勃勃的吼叫,伴随着暴雨骤雨一般的蹄声,响彻夏夜的街巷。一队二十多人的骑兵,掠过犹有行人的街道,在街角卷起一阵狂风。

而队伍中,刘希奭一手紧紧攥着马缰,一手按着被风吹得要飞掉的官帽,在心底破口大骂:‘尼姑庵烧了关我屁事?’

对于净慧庵的灾情,刘希奭该做的是回家睡觉,等明天起来后再打探消息。如果救火及时,那就当什么事都没有,如果城中值守官员救火不及时,牵连民宅过多,伤亡太大,他就要将之上报给天子。可不论怎么会说,救火之事都跟他毫无瓜葛。

可方才傅勍一听到潜火铺铺兵通报净慧庵起火,就急叫起来:“这可是不妙了,烧死和尚没什么,庵里的尼姑怎么能烧了?”就转过头大着舌头对刘希奭道,“刘官人,俺这就要去救火,不能奉陪!改天再请你喝……喝酒!”

傅勍虽是跟自己告辞,但刘希奭却不能立刻点头答应,必须先表示一下自己对灾情的关心,然后再表明要同去救火的态度。下面,傅勍就要打包票说自己肯定能成功救灾,不用劳烦刘走马;刘希奭接下来再退让一番,就算是将事做圆满了,可以转身回家睡觉——这就是官场上的惯常做法。

所以秦凤路的走马承受刚才便照规矩对傅勍道,“净慧庵竟遭祝融之灾,此非小事,本官还是与你同去。”

下面该轮到傅勍拍胸脯,可傅勍这位已经喝得醉醺醺的武官,却浑然忘了官场上的惯例,哈哈的笑着,“刘走马果然是豪杰!”

紧接着,不等刘希奭反应过来,傅勍便刷的一声抽出腰刀,踩着马镫站直了身子。将刀高高举起,高呼着:“儿郎们,跟本官一起杀过去!”

听着莫名其妙的话,刘希奭大惊失色。但身边悠闲的蹄声已然一下转急,一队巡城甲骑就在傅勍的带领下往净慧庵赶去。

刘希奭勒马不及,只能任凭坐骑夹在马群中,跟着一起很兴奋的在跑。他还听见一只不知身在何处的夜枭,大概被马蹄声惊到,发出了一声凄厉的尖号,在夜空中远远传开。

那声被惊扰后气急败坏的尖号,几乎就是刘希奭的心声。现在好了,被一起卷去净慧庵,自己再也脱身不得。在火场前面不等火灭就离开,一旦传扬出去,保不准就是一个临阵脱逃的罪名。给李师中、窦舜卿两人捅上去,天子岂能饶他?!

刘希奭盯住前面得意得挥舞着腰刀的傅勍,心中发狠,‘等到明天,就调你去守城门!’

……………………

位于城北的王启年家的宅院中,王家寡妇绑在一株歪脖子树上,嘴中塞了麻布,身上的衣服已经被马鞭抽了破破烂烂。她从被麻布塞住的嘴中发出呜呜的声音,眼眶里全是泪水,一直都在死命的摇着头。

窦解坐在一张交椅上,脸上满是不耐。他们已经问了快半个时辰了,但这寡妇却始终不肯承认王启年留下了证明窦解罪行的罪证。拖了时间久了,窦七衙内已经等不下去,他回头对站在身后的一名随从道,“钱五,你去把她的嘴撬开。明天还要出城射猎,不能再耽搁了。”

钱五长得斯斯文文,三十岁不到的年纪,但在秦中市井中,却是有名的阴毒。他现在一手托着王家幺儿的襁褓,伸到井口上:“想不到你家竟然还有口井?还真是方便。”他看着头摇得更急的王启年的遗孀,斯斯文文的笑着:“王家大嫂,不要再摇头了,只要你点一下头,说明白王老哥留下的东西在哪里,在下就把手收回来,放你们母子三人一马。不然在下的手悬久了,说不定会抖上一下。”

钱五等人正在逼问着,一片红光突然间洒满了庭院,外面紧跟着一片乱声大噪,一声声‘走水了’的叫喊伴着锣鼓响,不停的传入院中。

窦解听着心中惊疑不定,站起身回头看着红光照来的地方,那的确是一片火海所投射出来的光芒。他连忙点起一人:“快出去打探一下!”

“等等!现在不能出去!”窦解身后的李铁臂惊叫了一声,连忙拦住不让人把门打开。

“七衙内,现在出去被人撞上可就有些尴尬了。”钱五把王家幺儿丢给同伴,也跑过来提醒着窦解贸然出去的后果。

他们两人听到窦七衙内的命令,心脏都差点被吓得抽起来。门外脚步一阵接着一阵,一出门肯定就会被人看到。今夜他们来王家是为了湮灭证据,不是为了抛头露面。如果这时候遭人撞上,看破了身份,那可就是不打自招了。

窦解心中本是急躁,被两人阻止后更是大怒,厉声问道:“那谁告诉我到底是哪里走水了?会不会烧过来?!”

一名从人显是熟悉秦州城内道路,看了两眼红得发亮的火光,道:“那是净慧庵的方向。”

贴着门缝,听着外面动静的另一人也回头过来,点头道:“的确是净慧庵走了水,外面的人都在说。”

“那就没事了。”李铁臂放下心来,对窦解解释道,“净慧庵虽然跟这里在同一个坊中,离得也不算远,不过我们是在上风,又隔了一条路,火过不来。七衙内还是安心等一阵,等外面人少一点,再悄悄的出去不迟。”

“火烧不过来?”窦解问道。

“肯定烧不过来!”李铁臂肯定的点头。

“很好!”窦七衙内安下心来重新坐下,狞笑着,“那我们就继续问!”

