\section{第40章 中原神京覆九州(中)}

【第二章,求红票,收藏】

清晨,韩冈一行四人结了帐,启程离开了八角镇。韩冈并不知道他在西太一宫壁上写下的诗句,已经掀起了一阵波澜,即便他知道,也不会放在心上。

开封府就在眼前,冠绝天下的盛世繁华,彪炳千古的名臣贤相,留名青史的风流才子,此时,都在那一座煌煌巨城之中。

距东京城应该还有不短的一段距离,但除了路明外,其他三人已经分不清这究竟是城内还是城外,熙熙攘攘的街市,鳞次栉比的屋舍,怎么看都是大城通衢才会有的风景。刘仲武和李小六不时的回头,他们都在怀疑自己是不是在不经意间,已经穿过了东京城的城墙。

但开封的外城城墙还在前方远处,区区一道三丈厚的城墙,根本不能分割东京城的繁华胜景。

远远的,他们看到了琼林苑,被一圈围墙圈着,看不见里面的景色,只有墙内的树木探了出来。

对于天下欲得一榜进士而甘心的士子们来说,琼林苑算是一个圣地。唐时有曲江宴,专门款待高中进士第的士子们。如今有琼林宴,就设在琼林苑中。每逢大比之年的三月,进士放榜,新科进士们便簪花穿红,跨马游街,从宣德门一路走到城西的琼林苑中。那一天,数以万计的东京百姓都会聚在路边,围观赞叹。对十年寒窗,方才一举成名的士子们来说,这是至高的荣耀。

韩冈用眼角余光看了看路明。他身在琼林苑旁,却是言笑不拘,看起来真的全然放下了三十年来的心结。一朝顿悟,性子一转变得如此洒脱,倒让韩冈为之激赏。

在琼林苑北面,与其隔路而望的一片湖,便是同样有名的金明池。不同于戒备森严的琼林苑,九里三十步周长的正方形湖泊并未被墙围起。虽然现在还有军士巡守,但到了春天,位于开封城西,别称西湖的金明池,便会很坦然的向普通人敞开着怀抱。

“每年从三月初一到四月八龙华会,金明池都会开放给万姓游观。”路明习惯性的向韩冈介绍着路边的景点。“至是天子驾临,诸军金明池中争标,池东搭起彩棚,棚中士民数以万计,据说那样的胜景,不在正月十五上元灯会之下。”

“据说?”刘仲武奇怪的问了一句。

韩冈咳嗽一声,路明不以为意的解释道,“到了三月中,在下早就回乡去了。”

刘仲武略显尴尬,而路明貌似并不挂怀。韩冈则远远望着金明池,好像刚才那声咳嗽不是他发出来的。

韩冈前世曾经去过开封几次,复建中的金明池和琼林苑都逛过,但水泥本质的建筑完全没有此时屋舍的神韵,在无数仿古建筑组成的旅游景点中,根本算不上特别。

韩冈眼前的这座金明池,虽然无法走得太近,但仍能看见犹有冰层覆盖的湖面。湖心岛上的一座小殿,临水观风,独立于冰面之上。

只供天子使用的池中龙舟,就停在岸边上一处像是船务的空场上。听路明说名为大奥。透过池边林木的遮挡,可以看到有不少人在船上进进出出,估计是为了一个月后的天子驾临,而进行必要的整修。

从金明池的另一侧,一条玉带蜿蜒而出,汇入城濠,从西水关直入城中。由此看来,金明池其实也兼做调节护城河的水位之用。方方正正的金明池是后周显德年间修造,进行演练水战的地点。到了如今,虽然演练水战的初衷早已不再,但每年入春后的金明池争标,依然是一项盛大的节日祭典。

离着城门越来越近,周围行人也越来越多——只是还有十天省试便要开始,路上却是少见士子在外游逛,基本上都是留在居所,进行最后的复习冲刺。如昨日西太一宫中喝酒赏梅的那一群,其实是极少数的特例——在人群中穿梭,仿佛是在沼泽里跋涉,时时刻刻都要小心着不要撞倒行人。城门前的五里路,他们走了近一个时辰。当韩冈他们终于抵达城门下的时候,早已是汗流浃背。

韩冈站在护城河边,四面顾望。宽阔的城濠有三十步之宽,因为是冬天的关系,河上的冰面比河岸都要低上许多,河边是一排柳树,光秃秃的。但只看着树干上犹存的千条万枝,可以想见,春来万物生发,翠柳如锦的风情。

护城河对岸青黑色的墙体如波浪般的曲折,一眼望不到头。全长五十里长的东京城墙,保护起当世排名第一的巨城。高达五丈的墙体,也远远超过韩冈从秦州一路过来所看到的其他城池。

这就是京师。

李小六张着嘴,吃惊于京师的雄伟。而刘仲武扬起的眉眼,心中的惊叹也是掩饰不住。路明带着点小得意的去看韩冈,但韩三官人比刘仲武还要沉稳,半点讶色也无。

这下反倒是轮到路明吃惊了,他第一次看到东京城时,眼珠子差点掉出来。而他历次入京,不是没有跟第一次进京赶考的士子同行过,而他们,都是与他一般德性。

长安、洛阳名气虽大,但规模上远远比不上东京开封。韩冈还是从秦州出来的,秦州城虽比邠州要强,但总不能跟京城相提并论。韩三年纪轻轻,难道养气功夫都到了七情无碍的地步了?

路明为什么吃惊,其中的原因韩冈看得出来。乡下土包子进城,刘姥姥进大观园,都是一般惹人笑的。路明并非坏心,只是想看看自己的惊讶,但韩冈如何会让他如愿?

虽然眼前的东京城的确雄伟,但比之后世的南京城墙还是要逊色一点,更不能跟明代重新修筑的万里长城相比,所以在建筑上,靠开封城墙的规模就想震慑住韩冈,几乎不可能。如果是小桥流水的野趣,或是园林亭台的秀美,反而会让他赞不绝口。没办法,这不是东京城的问题,而是时代的差距。

不过眼前的东京城墙,并不是后世的那种拆了后又重建的水泥城墙,处处透着古意。虽然缺乏西北边寨的苍凉和硬朗,但有着中原的厚重,以及京师的雍容。韩冈虽不至于惊叹,欣赏的目光却也是少不了的。

就在城壕内侧,城墙根下,有一圈五尺高的矮墙——这等拦在城墙前的围墙被称为羊马墙。羊马墙与城墙之间的狭窄空间中,拥挤着一群群的羊、马还有猪等牲畜,这是羊马墙得名的由来。这些牲畜的主人都是远远的从京城附近一两百里的州县把牲畜赶来,就在城下贩卖交割。

平日里,羊马墙只是放置要贩卖的牲畜,充作市场。如果到了战时,羊马墙的作用则更为巨大。有了羊马墙辅助,城墙不再单薄,而是与城壕、羊马墙合为一个完整的防御体系。城中的士兵都可以下到羊马墙后,与城头上的守兵组成上下两重立体化的打击。

‘只是啊,’韩冈的笑容有些发冷,‘东京城墙修得再好也是无用,城中的人守不住谁都没辙。’守城者的意志力比城防更重要。张巡守睢阳便是明证,而几十年后,这座城池内外就要上演一幕幕活剧,则是更好的反面教材。

踏上城门前,横跨濠河的宽阔石桥,东京城的城西正门新郑门就在眼前。城门顶上则有着顺天之门四个大字——新郑仅是俗称,顺天才是本名。飞檐斗拱,金碧辉煌的三重城楼压在门头,没有军事建筑应有的肃杀,反而多了许多富贵气。就算城头上角旗密布,守卫罗列,也照样缺乏西北城寨给人的雄浑之感。

韩冈看了城楼几眼,便收回目光,自嘲的叹着。毕竟不是学建筑的出身,如果是梁思成那样的建筑家,看到北宋京城的城门不是画在清明上河图上,而是真切的出现在眼前,大概会兴奋的死于心肌梗塞。

随着人流抵达城门口,京师城门的检查却比想象中的要宽松许多,韩冈一行下了马牵着过了城门,并没有人过来查询。韩冈看了一下,只有身上带着大包小包,或是押着车辆的商旅,才会被拦下来缴税。其他人,城卫根本不会多看一眼。

这在秦州根本难以想象,除非是韩冈这样都认熟了脸的官人,不然哪个能逃过搜检?本以为洛阳、郑州等城池是因为在内地,所以不事防务,但大宋首都、一国重心,还是这般宽松,真的出乎韩冈的意料。

不过想想也是,据说每天被赶进东京城中的猪羊等牲畜加起来就有万只之多,鸡鸭之物更是数不胜数。而各地商旅官员或是本地住户,每天也总是有数万人出入,若是一个个查检过来,一天有三十六个时辰都不够。

穿过两重城门,以及城门间的瓮城,首先出现在韩冈面前的不是让他们心潮澎湃的东京城,也不是直通朱雀门的御道天街,而是一队滴滴哒滴滴哒的吹着喜乐,敲着小鼓的鼓吹班迎面走来。鼓吹班前还有举着棋牌的几对朱衣吏。而鼓吹班后,又有一队兵马压阵,再后面则跟着一溜扛着箱笼的人力。

看着这阵势,韩冈连同周围的人群全都避到大路两边,给这一队人马让出一条路来。

“是哪家皇亲要嫁女儿?”韩冈还没问个究竟,旁边就有人先问了。

“没看到朱袍子身上的金腰带吗?少说一个郡公。”

“那出嫁的当是县主了?!”

