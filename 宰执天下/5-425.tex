\section{第36章 沧浪歌罢濯尘缨(27)}

轻咳了一下,折可大换了话题:“战事才结束几天,辽贼那边定然有人心不死,万一派了刺客过来……”

韩冈笑了:“我一路走过来,在这里坐了有一阵子了,可没人认出来。”

代州城中士兵近距离看见过韩冈的极少数,韩冈换了一身衣服坐在城门口的店铺中,人来人往的,硬是没人把堂堂的枢密副使给认出来。

“自家人都认不出,还指望外人能认得我?!”

韩冈摇摇头,人要衣冠,这话一点不错。没穿官袍,走在路上也不过是个普通人,谁还会正眼看在小摊上吃凉面的人?

“可事有万一……”

“小乙你看看周围再说。”韩冈拍拍折可大的肩膀,让他回头看。

折可大依言往周围一扫,顿时就没话了。

这家小店中,只有三个亲兵占了两张桌子。但街对面的几家店铺里,却都有韩冈的亲兵坐着。两三个人一组,各自点了一桌子的菜,筷子慢吞吞的动,眼睛却都在瞄着这边。看模样,一有个风吹草动就会抢出来。

韩冈用兵向来求稳,这一回微服出巡竟也是防备森严。折可大本来只是想表示一下自己的心意,可看到韩冈身边的阵势,倒不知该说什么好了。

……………………

“勉仲,可知枢密又去了哪里?”

韩冈今天没有出巡的预定,章楶也没听说有什么突发事件。可是他在衙中,却左找右找找不到人,只听到下面的人回报,说是枢密换了身便装出去逛街了。

实际上已经担负起了知州之责的章楶,现在忙得恨不得一天能当两天用。听到韩冈竟然悠然自得的去逛街,自是气得七窍生烟。但是没奈何,他手上的事少不了要韩冈来处理,最后也只能找到了黄裳头上。

黄裳正在准备功课,在和议签订后已经有好些日子没参与衙中事务了。他在韩冈身边做幕僚,已经积功升到了从八品的卫尉寺丞,进入了京官的行列。就算日后不再立功,熬资历下去,也能晋身朝官的行列。但他还是想考一个进士出来,有出身和没出身,在官场上是截然不同的待遇。

当几名陌生的同僚坐在一起,首先会做的便是序年甲论科第。你一个二十一岁登科二甲十九名,我一个三十岁登科一甲榜眼,一个个有出身的同僚报了自己中进士的时间和名字,自家最后却来一句没出身。那样的情况,想想都觉得心中发寒。

就算才高名高如韩冈,做到了从七品的国子监博士都还要去考进士,甚至宁可放弃面圣的机会,也要先留在陕西考一个贡举的资格出来。黄裳不糊涂,知道想要走得更远,更稳,就必须要有一个进士的出身。

一甲、二甲不指望了,三甲同进士也行啊。只要打通上进的路就行。抱着这样的想法,自从和议之后便日夜苦读,韩冈等人也不打扰他,让他安静地准备考试。不想章楶却找了过来。

“枢密?不知道。”黄裳摇头,“是出去了吗?”

“勉仲你也不知道啊。说是去逛街了。”章楶说话的时候就在咬牙,心中发恨,只是跟着又叹了一声,“城中人心尚未安定,来往的又多是没关防的流民,枢密贸然出外,万一遇上几个辽贼派来的刺客该如何是好?”

黄裳闻言神色一凛:“……枢密身边跟了什么人?”

“枢密身边的亲卫都跟出去了。”

黄裳舒了口气:“那就不用担心了。枢密身边有那群亲卫,比我们在衙门里都安全。”他“怎么急着找枢密?”

“开封那边有回音了。”

“是召枢密回京……”黄裳说着自己就摇头,要召韩冈回京,肯定是中使背着圣谕来,要设香案接旨,哪里会这般无声无息,“是枢密奏章的回复?”

“嗯。”

“朝廷那边怎么说了?”

“朝廷那边看起来不想让枢密回去呢!”

黄裳的眼睛瞪了起来,惊异道:“全都准了?!”

“是啊。”章楶叹了一声,“没想到都准了。”

之前他们这些幕僚就推测过朝廷可能会有的反应。以韩冈的功劳和声望,如果朝廷那边当真不想让韩冈回京,只会用怀柔的手段,免得他气急败坏直接撕破脸来上表告御状。

现在韩冈的每一份荐书都得到了批准,那么朝廷的用意就很明显了,不想给韩冈回京的借口。希望韩冈能留在河东,再镇守一段时间。

“而且里面也没有召我和诚伯入京陛见。”

章楶和田腴任官代州,以常例说得让他们先回京城一趟。尤其是章楶,朝臣出知边地要郡,当先经过陛见、问对的环节,让天子确认他的能力是否适任。而田腴任知县,从选人直接转京官,也应该陛见才是。韩冈当年在河湟,升到了从七品的国子监博士都没有入京,那是特殊情况下的特殊例子。而现在的河东又不是兵凶战危、须臾间离不得人的时候,章楶、田腴完全可以离开。

“王平章这是怕质夫兄你和诚伯回京陛见之后,引动皇后调回枢密的心思……一点机会都不留。”

“枢密会怎么做?”章楶想要找韩冈,正是想问一问韩冈的打算。否则这件事吊在心里,便没办法安心做事了。

黄裳摇摇头,摊开手,韩冈也没有跟他说过对策。他的恩主虽很少隐瞒什么,但总是喜欢把要采用的手段藏在日常的对话里,一个不注意就会忽略过去。对此,黄裳也没办法:“不知道,不过既然枢密今天能安心逛街,肯定是有办法的。”

提起韩冈逛街,章楶心中就发堵:“那要等到什么时候?”

“至少等枢密把布局完成吧。”

“枢密的棋艺……”

章楶和黄裳对视一眼,又都笑了起来。

……………………

“别看了,既然坐下来了,该点菜了。”韩冈敲了敲桌子,提醒折可大,“秦二哥说,这家铺子的味道就是太原的知味楼都比不上,代州这里也留不久了,不尝一尝岂不是可惜。”

“秀才公,俺这里最好的不是面和酱。是醋,是好醋。”老汉不知何时走了过来,很是自豪的夸着,放下了一个小瓷钵,盖子一开,酸溜溜的味道就钻了出来,“是真正的并州老醋。”

“哦?那就更要尝尝了。”韩冈充满了期待。

天下醋以并州最佳——并州就是太原——并州的醋就是在京城也是有名的,河东人爱吃醋则更有名。韩冈当年在河东时也没少吃,很是有几分怀念。

不比千年之后,想买哪个地方的特产,总有办法买得到。但在这个时代,许多地方的特产,由于储存和运输的原因,就算他已经是天下间数得着的高官显宦,也没办法吃得到。或是尝不到正宗的原味。他在东京的那段日子,对并州老醋可是久违了。

在河东军中,醋跟酱都是必备品,比酒都重要。

在《武经总要》中,还记载着如何能随身携带酱醋之类的调味品。将干净的麻布放进醋中浸泡,然后拿出来晒干,再浸泡,再晒干,直到麻布吸足了醋,晾干后就可以随身携带了。吃饭时,只要剪下一片丢进汤中,就等于加了一大勺陈醋。

“这位……官人。”老汉又转回了身,折可大没穿官袍,但身上的军袍却没人会错认,而且还不是兵卒的穿戴,称呼一声官人并不会错,“俺这小摊子上就槐叶和甘菊两样,不知官人要点什么?”

折可大看了看韩冈,韩冈道:“我是槐叶冷淘。秦二哥是甘菊,他昨天吃过槐叶冷淘了,今天想要尝个新口味。”

“那俺也来份甘菊冷淘好了。”

槐叶冷淘是槐树芽榨汁和面粉做成面条,然后拌上作料。甘菊的做法类似,只是换成甘菊而已。这是上了宫宴的菜色,民间也常见。

老汉看起来的确是不负其名,手脚极其麻利,和面切面下面一气呵成,动作中有着韵律和节奏,是积年的老手。

哒哒哒的快刀切过砧板声中,韩冈侧头道:“今天听诚伯说了流民回乡的事。这件事,小乙你办得不错。”

折可大张了张嘴,要韩冈叫顺了口,以后该不会都是小乙了吧。

只是他不敢说出来,顺着韩冈的口气:“只是跑跑腿,不费什么事。”顿了一下,“轨道真是方便了。四百户一天就从忻口到了州城。换做走路,老老少少,还不知要走几天。”

折可大说话时声音压低,看起来有些鬼祟。

“为了这一条轨道,花了多少钱粮?用了多少人工?太平时日能排上点用场,不算浪费了。”

多谢两府和三司,尤其是三司的吕嘉问,自己为他回京,还帮忙说了几句。只是才做了两个月不到就从开封府转到三司使的任上,在支援河东的时候,他真是帮了大忙。

现在回头再看一看,吕嘉问的确更适合做三司使。他曾经执掌开封市易务,为此得罪了不少京畿贵胄,另外他在理财上也有一手,比起权知开封府,更适合做计相。而且当初在崇政殿上一番争执,韩冈还记得很清楚。恐怕皇后心里成见依然存在,不想让开封府交由吕嘉问来治理。
