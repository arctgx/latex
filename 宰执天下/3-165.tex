\section{第43章 竹纸知何物(上)}

纷纷扰扰的大典终于结束了。

当天天子起驾回宫,次日御大庆殿,颁大赦诏。

天下州县狱中,除大辟【死刑】及十恶重罪之外,其余过犯皆赦之。旧有被贬斥的官员,也在原赦之列。而群臣、军士都随之得到恩赏。在官、阶、勋、爵上,视品级、差遣来加以封赠。

韩冈当然也不例外。

但对于韩冈来说,散官、勋位的晋升,根本就是噱头。散官官阶升为从六品下的通直郎,勋位擢为正五品的上骑都尉,只不过是将身上官名拉长而已,全然没有一点实际意义。

真要说起来,还是分到的胙肉更为实在一点。无论是什么祭祀,供奉在神主前的猪牛羊三牲,都是将脑袋放上去,剩下的肉就是参与者各自均分,郊天大典自也不例外。文武百官、上万军卒人人都有。韩冈分到手的胙肉有十几斤,就是一头猪的前腿,可比通直郎、上骑都尉什么的油水更足。

郊祀恩赏也就是这样,除了金帛之物以外,基本上全都是虚的。即便看上去好像有点实际的东西,可只要想想国中所有的官员都能得享恩泽,就该清楚如此封赏还不如直接给钱实惠。唯有一干高官显宦,能在郊祀之后,得到几个荫补子孙的名额,这才是他们参加郊祀的价值所在。

不过韩冈还有安置流民的赫赫功劳,隔了一天,韩冈又得到了一份制书。

本官从从七品右正言,特旨转迁正七品的起居舍人——理所当然的,这一官职仅仅是标定品级、俸禄的寄禄官,并不是说韩冈要跟在赵顼身后,记录天子的一言一行,这个工作由起居院中的修起居注和同修起居注来负责。

只是到了朝官之后,本官、品级,都不再重要,仅仅关系到俸禄的多寡。重要的是资序,另外就是馆职、贴职这类文学职名,这代表着朝廷的看重与否,以及在官场中的潜力。韩冈的资序在做过了府界提点之后,就是第一任知州一级。而职名也从集贤校理,晋升为直龙图阁,离着腰金带、跨狨座的侍制,只剩最后的一两步。一旦跨过去,那就是朝中高官显宦的一份子了。

听着从宫中传出来的消息,赵顼其实本有意直升韩冈为天章阁侍制,但给王珪给顶回去了。他说韩冈得中进士不过一载,便得任侍制,未免有骇物议,虽有功勋,亦不当开此先例。对于王珪此议,冯京附和,吕惠卿只帮着韩冈不疼不痒的说了两句,而韩绛则根本没开口,赵顼最终也只能作罢。

韩冈听说了之后,却是一点也没生气。走得太快不是好事,在朝堂上一个靠山都没有,也并不是坏事。

不亲附当朝宰辅,端正居朝。这样的姿态,落到赵顼眼中,就是最受皇帝欢迎的孤臣。对于以宰执为目标,本身又已经离侍制只差一步的韩冈来说,现在所谓的靠山根本就是个麻烦,狗屎一般,沾到手上,洗都来不及,绝不可能自己往上贴的。冯京、王珪跟自己过不去反而是件好事。

而为了补偿韩冈,赵顼给了他一个开国县男的封爵。但韩冈直接就给辞掉了,这等虚衔一点意义都没有。辞了两回,到最后,又改成了给韩冈二子加官,并给韩冈的两位亡兄赠官。

制书拿在手中,韩冈回头看着尚在吃奶的次子,还有在院子中与姐姐一起来回跑的大儿子。这样的小孩子,都能给个官身,自己却要千辛万苦才能挣来。勤学苦读十载,都不如投个好胎。

荫补子孙是如今通例,韩冈也不会故作清高到加以拒绝,而且前面已经辞了开国县男的爵位,现在再拒绝荫补儿子,就未免给人故邀清名的感觉。而且看着王旖、素心她们都为此而开心的样子,再想想乡中的父母听到两位兄长得以封赠的消息后的心情,韩冈也难以提起拒绝的心思。

“官人。”王旖提醒着韩冈,“得要给大哥、二哥起个大名了。”

一般来说,小孩子都是上学之后才起大名,到了成人时,再起表字。不过现在两个小子有了官身,就必须将正式的姓名送上去。

韩冈也没多为此费神,依着这个时代的俗例:“为夫的名字出自玉出昆冈一句。玉乃石属,算是土行。五行土生金,大哥、二哥名字都从金字旁好了。”

不费什么事,长子韩钟,次子韩钲,两个大名就给定了下来。韩冈一边亲笔代写下三代家状,一边笑道:“日后老三、老四,可以叫韩锣,韩钵……”

四名妻妾一起急了,“官人!”

韩冈哈哈大笑:“说笑而已,不要当真。”

写下了家状,过几日就可以递上去,等着告身发下来。韩冈搁起笔,对着妻妾道:“明天为夫就该去军器监了。你们也趁着这两日,将房子给收拾好。”

周南道:“官人放心,今天明天也就收拾干净了。”

当年韩冈与王旖成亲时所租的房子,如今已经给租出去了。不过韩冈毕竟是做过府界提点一职,在开封府衙中人头熟,很容易就找到了一座还不错的院子。也是官产,归于开封府所辖。就是租金比起过去的那一座要高得多,当然,这也是因为位置好,面积大的缘故。

前后三进的院落,位于北城,周围都是官宦人家的宅子,向东望去,一能看到五丈高的皇城城墙。也符合韩冈的身份,不过租金也贵得可以,掌管家计的王旖正为着租金头疼。

韩冈的俸禄不算少,得到的赏赐也多,可他偏偏是个大手大脚。别的不说,三个幕僚得官,他就直接各送了五百贯财物过去做贺礼。而且韩冈看重自己的名声,从不收受重礼。前些日子,王旖生了儿子,韩冈收下的礼物加起来都不到千贯。家中连着仆婢,人口有三十五六,吃穿用度都靠着他一人。光靠俸禄,根本积攒不下什么余财。

“说起来,家里的年货差不多也该松到了,前些日子冯家叔叔不是来信说,要赶在腊月前上京一趟吗?还说要今次带着弟妹见岳父母。”王旖对着账本问韩冈。

“前两天听传言说关中雪灾,不过因为不想干扰到郊天大典,上报的奏章给政事堂压下来了。不知是不是这个原因给耽搁了。”

在院子里玩的两个小孩子,扑得满身是雪。素心正帮他们给擦着,听到了韩冈的话,惊讶回头问道:“怎么又有灾了?”

韩冈叹了口气:“大宋十八路,幅员万里,哪年会没有灾荒?”

王旖形状姣好的双眉为难的皱着:“这耽搁下来可就不好办了,年节的时候,礼数都要尽到……家里的积存已经不多了。”

“现在才腊月初,还有着二十多天才过年,别急,肯定就快到了。”韩冈对此满不在意。

正说着,忽有下人在外禀报:“舍人,冯官人从关西来了。”

韩冈低低一声笑:“说到曹操,曹操就到。”提声道,“还不快点将人请进来。”

一行车队进了韩家的院子。十几名护卫下了马,随行而来的八辆马车停在院中。载人的两辆,剩下的六辆都是装着货物。车斗中的货物高高的堆了起来,被油布和绳索给紧紧盖住。

冯从义大步走了进来,尽管才二十出头,但几年来的磨砺,让他的神情举止都有几分豪商的气度。

一见韩冈,冯从义就拜了下来:“从义恭喜三表哥加官进爵。”他身后,浑家高氏也向韩冈屈膝道着万福。

韩冈扶起了表弟,笑道:“你耳目倒是灵通。”

冯从义也笑道:“表哥如今名气大了。小弟进了城门,一使人打听表哥你现在的住处,就什么都听到了。”

“爹娘身体可还安好?”韩冈紧跟着问道。

冯从义连连点头:“都好、都好。姨母还让了俺带来了她亲手做的小衣服,说是给俺侄儿侄女的。”

冯从义说着,就从怀里掏出了一封来自于家中的书信,还有带过来的礼物单子,呈给了韩冈。然后又与浑家跟王旖见礼。

两边尽过礼数,王旖带着高氏进去安置,而外面随冯从义而来的护卫和车夫也都安置了下来。幸好是租了三进的大院落,否则也安置不下突然多出的二十多人。

与冯从义落座,等下人奉上茶汤,韩冈便拆开信,细细看了一遍。不过上面尽是问着孙子孙女的话,倒没见几句念着他这个儿子。

放下信笺,韩冈对着桌上鲜红的礼单敲了一敲,“有没有给荆南的信表哥送过去。”

冯从义笑道:“表哥放心。今年的年礼十月的时候,姨母就催着送去了荆南,就生怕路上给耽搁了。是钱管家亲自押送的,护卫的人手都是从庄子上招来的,什么都不用担心。”

“信表哥如今已经是永平县开国男,坐镇荆南的一方大将,花钱的地方比我这里还多。钱物还是要送足了,不能让他日常受窘。”

“表哥可是操心太过,姨母那边早就想到了。”

