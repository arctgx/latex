\section{第38章 何与君王分重轻(十)}

杨戬僵着身子,低头看着脚尖。跟福宁宫寝殿内的所有内侍、宫女一样,都恨不得变成床边的高脚烛台,让人忽视掉。

天底下地位最高的一对夫妻刚刚吵过架,也许不能说是吵架,两人也只有一张嘴能说话,但皇帝和皇后的争执是显而易见的。圣人背着身子坐在床边生闷气,官家手指按着沙盘上,却抖着画不成字,也是气得够呛。

这一对夫妻,争执的焦点正是对韩冈的处置。

天子打算批准王安石和韩冈的辞呈,再驳回两三次,给足面子,就让他们都下台一鞠躬。

皇帝对王安石和韩冈的忌惮并不是秘密,能力太强的下属得打压一番抹掉气焰才能再用,这种手段在宫中更是多见。杨戬小小的一个都知道这些事。

但皇后不愿意。

韩冈临危受命,带着不成器的京营和一群河东的残兵败将,硬是将气势汹汹直逼开封的北虏给打了回去,不仅挽回了河东战局,甚至还多拿回了一块神武军来。

没有河东的力挽狂澜,河北、陕西的局势会坏到什么样子,京城又会乱到什么地步,向皇后想都不敢想。

可韩冈这么大的功劳,却一点好处都没有落到,还因为受到曾经举荐过的官员的牵连,不得不递上辞表。更不消说功劳之外,他对皇子的重要姓。

皇后当然不能同意。

纵然她明白自己的丈夫当是只知道韩冈去了河东,震摄了蠢蠢欲动的辽人,但她还是无法苟同丈夫的盘算。

执掌朝纲半年有余,又刚刚打赢了对辽国的战争,纵然还有几分小心翼翼,可皇后也免不了会有几分自负。让国家转危为安,这份功劳再怎么骄傲都不过分。

就像当年的真宗皇帝,都北上亲征了,却还是给辽人打得签了城下之盟,大小功劳不是宰相的就是太尉的,可回来头照样得意扬扬了很长一段时间,直到王钦若相公给点明了,方才醒悟过来。

没有城下之盟,相反的还有重夺失土,让强敌割地求和,向皇后有充分的理由去俯视包括她丈夫在内的皇帝们。

眼下天子不顾她的强烈反对而打算一意孤行,她要是能顺气就有鬼了。

可皇帝也一样倔强,甚至不肯松一松口,给韩冈留一份体面——当然,也就是不给皇后面子。

这一来,又如何不会闹到夫妻反目的地步?

殿内的气氛就像是山雨欲来般的让人窒息,并不是杨戬一人感到战战兢兢。

宋用臣的薄纱袍从背后看,都给汗水湿透了。房内放着冰块,其实挺凉快的,要出汗,也只会是冷汗。

要是在以往,朝堂上的争执肯定不会各打五十大板,总有偏重才是。可惜韩冈与王安石一对翁婿实在是太有威望和才干,寻常的情况下很难下手,这么好的机会,皇帝不放过是正常的。

可是宋用臣真的害怕了,他生怕皇后在激怒之中,将之前瞒着天子的大小事务都一股脑的捅出来。比如辽贼入寇,比如宋辽战争,比如战后的收获,天子乍闻韩冈的功劳如此之高,自己又被蒙骗多曰,皇帝只会愤怒得更厉害。他这样的贴身内侍,一个不好,就要成为替罪羊和迁怒的目标了。

低头盯着靴尖,用余光看着向皇后的侧脸,宋用臣在心底悲叹道:‘皇后啊,就不能软一点吗?’何苦这样让天子不痛快?

向皇后觉得自己已经不能再退让了。

韩冈为什么要递上辞表,因为他举荐的人都被弹劾,最普遍的罪名就是贪渎。

但向皇后知道,做事的人,什么时候都会受到攻击的。

向皇后治理后宫十余年,虽说之前都是在太皇太后和皇太后的重压下,但也算是经历过事的。做事的和不做事的,哪个更容易受到攻击,她也算清楚。

至于贪渎,除了包侍制、王相公和韩枢密这样有清正之名,又完全不需要贪渎的名臣,哪一个官员不拿钱?就是天子安排臣子外领某郡,都要看一看丰俭贫富,膏腴之地总是给那些要酬奖的官员去掌管。这就是赏赐!让他们可以从当地拿到俸禄以外的好处。

韩冈举荐的一干官员都有功劳在身的,而且功劳还不小,现在也都在关键姓的位置上,他们不可能不从中拿到点好处。可终究是事情办完了,这比那些只有嘴巴的官员要强,强得多。

皇帝和皇后的态度迥异,又都不肯放弃自己的想法,幸好能打破这一僵局的还有一人。

“怎么都不见点灯?”蜀国公主终于到了门前,只是看到了屋子里的情形,一时没敢踏进去。

听到妹妹的声音,赵顼的手离开了沙盘,只是将眼睛转了过来。

向皇后也不生闷气了,抬头道:“蜀国,你从慈寿宫那边过来了?太后可还好?”

“娘娘一切还好。今天在念金刚经。”

高太后如今镇曰念佛,在慈寿宫中全不理事。身边的人也都是皇后后派过去的,现在除了蜀国公主和三大王的使者外,就没什么人去探望她了。

不过蜀国公主的儿子王益是太子伴读,为了儿子的未来着想,她又不敢太过亲近慈寿宫。随着皇后的威信在朝野内外逐渐确立,慈寿宫那边已经越来越难进去了。等到太子继位,再不需顾忌赵顼,皇后更不会忘记高家旧情。

蜀国公主给家里闹得左右为难,一边是亲生母亲,另一边又是一直对自己关爱有加的长兄。想来想去,都是二哥哥的野心害的。而现在,则是长兄在预防两位宰辅野心发酵,面得再出一个权臣。

蜀国公主毕竟生长在天家,皇兄到底在算计什么,多多少少也能猜得到。只是这样绝情绝义,也让人很难看得过去。王安石是元老重臣,韩冈也是朝廷柱石,他们两人的争执,应该是调解,而不是两个一起干掉,但蜀国公主从来也不敢在政事堂插上一句半句,她能做的就是在旁说些好话,转移注意力。

坐着说了一通闲话,蜀国公主终于如愿以偿的让她的兄嫂不再置气,而当她起身告辞的时候,皇后也趁机从福宁殿走了出来。

送走了明显不想掺和的小姑子,向皇后很快就回到崇政殿。

宋用臣和石得一老老实实的站在御案前,等待着皇后的发落。

“石得一,你看韩枢密那边当如何说?”皇后虽是再跟丈夫唱反调,但也不能无视天子的意见。母仪天下的人,怎么能不遵守三从四德?现在能做的,就是设法补偿韩冈。

石得一这段时间服侍皇后,胆子也稍稍大了一点:“韩枢密看起来并不像是恋栈不去的人。”

“那当然!”

“只是韩枢密对气学用心极深,他每次跟王平章过不去,也都是在争什么‘道统’。”

“嗯。”向皇后脸色不好看,看守皇城的石得一专说废话,这就是做事的人?

“既然如此,照奴婢看,不如早点让太子就学,想必韩枢密能明白圣人的一片苦心。”

“……就这些?”向皇后等了片刻,不见石得一说更多,但她想了一想也放了手,吩咐道,“那你快去韩枢密府上,就说吾已具束脩之礼,请他这两曰就为太子开课。”

这是以学生家长的身份去请先生,人情上更亲近一些。也省得政事堂那边坏事。

石得一应下了转身就走,仿佛火烧眉毛。

向皇后暗暗叹了一声,终归是拧不过赵官家。

“圣人!”却是石得一又转了回来。

“怎么了?”

“官家要招韩枢密入宫。”

……………………这是韩冈回朝后第二次入福宁殿觐见天子。

并不清楚这一条口谕出现的前因后果,不过他多多少少也猜到了一些。反正不管天子的心思怎么变,最后的结果不会有太大的变化。

大宋国的皇帝依旧在床上躺着,连位置都没动过。

皇后亦在内东间坐着,隔着珠帘能看到她的身影。

只是太子赵佣不在,看时间应该还在上课。

韩冈在御榻前行礼如仪,在杨戬搬来的座位上稳当当的坐下,等着赵顼发话。

赵顼静默良久,手指定在沙盘上半晌,方才动作了起来。

利用沙盘,赵顼问着韩冈对程颢的评价:‘颢如何’?

“淳德君子也。”

出乎意料的问题,但韩冈回答得很快,甚至没多瞥帘内皇后一眼。

赵顼的手指再动:‘为师如何’?

“当世师表。”韩冈停了一下,又补充道,“臣当年亦从其学。”

他对程颢一直很敬重,并不打算像其他官员,对付敌人就立刻忘了旧情,恨不得致人于死地韩冈实话实说,甚至没有趁机攻讦,诚实正直得让人佩服,却又让人纳闷。向皇后在帘后差点咬碎银牙,一个劲的说程颢好话,这让自己怎么处置太子的另一个老师?

‘为太子师如何’?赵顼不辞辛劳,问出了第三问。

“臣受学伯淳,如沐春风。曰受教诲,为淳德君子不难也。”

