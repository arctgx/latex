\section{第20章 冥冥鬼神有也无(25)}

街巷上看不见人影,如同末日降临。

当宋军在富良江上正面击败了水师,失去了最后的屏障,就算是街边的乞丐,也知道交趾国已经大势已去。

宋军人数并不多,也没有堵着城门,可城中偷偷溜走的军民官吏却为数甚少,毕竟还有数万穷凶极恶的蛮部散布在野外,出了城只会是死路一条。只是眼下的萧瑟清冷,让人感觉不出来,都城中生活着数万人口。

在偷袭船场失败之后,李常杰他原本准备离开升龙府,带着大越天子往南方巡狩,看看能不能避过宋军的攻势。可占城和真腊两国联手出兵的消息,让他只能放弃这个打算。

不过坐困愁城,就是等死而已。宋人都已经打到了升龙府,怎么看都不会就此收兵,或是手下留情,可谁还有办法?能抵挡得了宋军的攻势。要知道,眼下在城外设立营帐,准备攻城的宋军仅仅是安南行营的先锋而已,真正的征讨大军还没有抵达邕州。这一个月来的战事,其实是宋人主帅心急,抢先攻了过来而已。

可就是对上了实力不足的宋人,李常杰仍是连番败绩,且就是因为他一时莽撞的愚行才引来灭国的宋军,旧日的权臣如今已是树倒猢狲散。

现在还跟在他的身边的人,其实都是不知道该如何是好,只能挤作一团来互相壮胆。也有一些是转着更为阴狠的念头,但李常杰已经是无暇去跟他们计较。

不知不觉之间,李常杰发现自己已经到了黄龙庙。本想着往宫里去,却不意到了此处。

犹豫了一下,李常杰下马走进了庙中。

庙中空无一人,平日里香火鼎盛的黄龙庙里,现在都嗅不到多少常年缭绕的呛人的檀香味道。

前些日子还在庙里磕头求着神灵庇佑的人们,全都不见踪影。

今天还是阴天,若是在水师败阵的前几天,这时候的殿中肯定是挤满了人,都会来求天上的阴云变成暴雨降下。但宋人已经到了城下,也不再有人抱着这等幻想。

可以这么说,当宋人打过富良江之后,大越国已经确定灭亡。都没有了抵抗的心思,唯一会做的挣扎,就是不想被砍掉脚趾而已。说不定就有人计划着,拿自己与宋人交换一个承诺。

其实宋军的人数并不多,只有万人而已。即便他们再凶悍善战,凭着大越的军力,其实也能抵挡得了。但当一头猛虎领着一群恶狼攻过来的时候,那就是再也没有顽抗的余地了。

原本还对富良江和江上的水师抱着最后一丝幻想,可眼下幻想破灭,朝臣们已经暗地里计算着要开城投降了。他们多半还怀着幻想,宋人的仇恨都集中在李常杰和太后、天子身上,只要都交出来任由宋人处置,他们至少还能保着小命,说不定还能保住富贵。

宫中都乱作一团,倚兰太后只能抱着年幼的国王等着最后的结果。

听到了宋军攻到南岸的消息,占城也该出兵了。

眼下大越已是日暮途穷,再难有挽回的机会了。李常杰本人,也是心知自己是命在旦夕之间。平日里上门奉承的朝臣全不见踪影,而令行禁止的麾下军队,也都不再服从他的命令。

众叛亲离之下,若还不知道到自己即将面临什么样的结局,李常杰也枉费了他做了几十年的将帅。

走进黄龙庙,香案上的黄绸都拖到了地上,上供的果盘翻倒在桌上,时新的鲜果滚在地上、桌上。

李常杰皱了一下眉,亲自走上去,将案桌收拾好。果盘、香炉都摆放回原位,地上的蒲团也都是好好的放在了案桌前。

一切收拾完毕,站在香案前,李常杰犹豫了一下,就跪下来磕了三个头,这还是他第一次虔诚的祈求上天的帮助。

半晌之后,他从蒲团上站起身。身后的从人脸上都失魂落魄的表情,一直以来充满信心、有着绝对强势的主人竟然过来求神拜佛,这样的变化,哪能不让他们感到恐惧。甚至当李常杰因为久跪突然站起,差点失去平衡的时候,都忘了上来扶着他。

身子晃了晃,李常杰重新站稳脚,看了一看一众随从,连喝骂的精神都没了,起步跨出殿门。

也就在这时,一道闪电划破阴云密布的苍穹,一声霹雳响彻天地。随着闪电雷鸣,下一刻,风雨大作,黄豆大小的雨滴从天上砸下来。

李常杰惊讶的停住脚,看着眼前一下就变成瀑布一般的暴雨,他和他的随从们又齐刷刷的回头看着供着黄龙的大殿。

冥冥之中,难道真有鬼神不成?!

雨势越来越大,就像是天漏了底一般,九天银河从破口处倒悬而下。这样的雨势,李常杰极为熟悉,生活在这一片土地上的每一个人都熟悉无比,这绝不是旱季时会有的阵雨。

“天不亡我!”

李常杰喃喃的念叨着。就在最后一刻,就在他亲自祭拜黄龙之后,上天终于给了他一个回复。

“天不亡我!”李常杰一声大吼,怒目金刚一般的眼神瞪着身边的每一个人。

对着有上天和黄龙庇佑的主人,随从们都跪了下来,“恭喜太尉,贺喜太尉!”

李常杰没有多加理会,他冲进雨中,仰头迎着狂风暴雨。雨点砸来在脸上,有着轻微的痛感,这点痛楚,让他放声大笑。湿透的衣袍黏在身上,散落的发丝蛇一般的贴在脸上,状若疯狂。

一道道闪电照亮了交趾权臣在雨中高举双手的剪影,“是天不亡我!”

……………………

“只是下场雨而已,何必作无谓之忧。”

韩冈对此并不在意,听着外面的雨声,脸上带着轻松惬意的微笑,已经有一个月不见雨水,天上的烈日,许多河流湖泊都大大缩减了水面的范围,这时候一场雨下来,也只是补偿而已。

“一个月不下雨了,这时候下场雨也是正常。就算是旱季,也不是说一滴雨都没有,还是有些雨水的。等雨停了之后,就立刻攻城。”

但三天之后,依然不见休止的暴雨,让韩冈再也无法维持脸上轻松的笑容。询问交趾气候得到的所有回复中,都确认了一个事实——提前了近一个月,熙宁十年的雨季,已经降临到交趾的大地。

营地中的土地在雨水中变得泥泞湿润,在泥地中安营扎寨的官军士兵,即将功成的兴奋渐渐消退,已经开始抱怨起来。

只差一步就能攻下升龙府,却遇上了最让人感到畏惧的雨季。

数日前还平缓温顺的富良江,此时水流汹涌浑浊,一如黄河一般。此前暴露出来的滩涂,已经被淹没了大半,卷起的浪涛上泛着白沫,隐约带着怒吼与呼啸,仿佛有异兽翻腾于江中,似乎就是交趾的护国黄龙在江水中兴风作浪。

章惇、韩冈、李信和李宪四人脸色沉重的,听着外面的江水滔滔。隔了数里和一层帐幕,依然清晰无比的传进他们的双耳之中。

‘难道当真有鬼神不成?’

四人都紧紧咬着牙,就差了这么一步。

升龙府的城墙已经就在眼前,工匠们正在为攻打城池的霹雳砲日夜赶工,只要再有三五日,就能一举攻破升龙府,为这一次南征向天子交出一份完美的答卷。

甚至不用再打,城中不少朝臣都已经派了亲信出来,递交降表和效忠书信,愿意为官军做内应。人数之众,只要官军当真冲到城下,说不定这些叛逆就能点起自家的家丁,集合起来攻下交趾王城。

眼看着就要赢了,偏偏就下起了雨。而且是七八天甚至十天半个月,都有可能一直下下去的暴雨。随着雨季的到来,疾疫也会快速滋生,没有经历过这样难熬的季节,来自北方的士兵,又还能保证多少战力?而原本派人来联络的交趾朝臣,这时候都没了消息。

天时地利人和,用兵三要不可偏废。官军虽然此前三者皆得,可雨季一至,便是形势逆转。眼下身居险境,再不做出一个决断,说不准就是在升龙府外全军覆没的下场。

“不能再耽搁了!”章惇率先开口,他不觉得眼下的雨水有在短时间内停歇的迹象,“即使是要冒雨,也得立刻攻城。”

燕达也同意章惇的看法,每在雨中多停留上一天,军中的士气就回低落上一点,几日来他都在巡视营中,看到的、听到的,都是明证,“是得尽快攻城,官军战力远胜交趾,就算用不了飞船和神臂弓,只靠斩马刀和霹雳砲,也照样能攻进去。”

“官军的人数太少了。交趾士气已盛,此时用兵,兵力要更多一点才好。”李宪建议道,“最好能将诸部大军都召回来,一并攻城。”

三人都表了态,六只眼睛一起望向韩冈。

韩冈点着头,开了口,他的看法与三人一样:“正如李都知所说,人都要召回来。这一次,我们要囊土攻城!”

