\section{第33章 物外自闲人自忙(一)}

韩冈带着几个随从从转运司衙门穿过了小半个洛阳城,终于赶在约定好的时间抵达了程家门前。他的到来,并没有惊动到任何人。

没有旗牌官在前开道,也没有一众元随左右护持,连官袍饰物一样都没有穿戴在身上,除了久居高位养出来的气度风采,看起来就像是普通的世家子弟。

不过程家的司阍不会这么没眼力,韩冈他也早就见过了。虽然他没看到浩浩荡荡的随行队伍,但他还是一眼就认出了轻车简从的都转运使。当年韩冈在门前雪地里站了一个多时辰,给他留下的印象刻骨铭心。连忙上前来迎接,又让另一名司阍进去通禀。

韩冈下了马,随从们便将坐骑拴在程家门前系马桩上。随口与上来奉承的司阍说着话,一边看着程家的外墙。

老旧的宅院还是跟他上一次来时一般,没有什么变化,只是少了积雪,多了些生长在墙头和瓦片上的杂草。以二程的名望来看,这间老宅的确算不上宽敞。安贫乐道四个字,也能算是合得上。

韩冈只在门口稍作停留,程家的大门便从中而开。

韩冈登门造访,程颢、程颐并没有出来迎接,世上没有老师出迎学生的道理,但程颢家的儿子端懿、端本,还有程颐的长子端中、次子端辅,一个个都被遣了出来。

韩冈与程家算是通家之好,程家的子弟韩冈全都见过,见着也不拘谨。行礼问好,忙了一通之后,便领着韩冈进门。

韩冈被领进程家的正厅,一家之主程珦被请了出来。刚刚参加过文彦博的同甲会,也快往八十走的程珦须发皆白,如银雪一般,但精神旺健,面色红润,显示日常保养有方。比韩冈几年前上京考试时,看着还要年轻一点,久违的程颢、程颐则侍立在左右两侧。

韩冈稍稍加快了步伐,几步站在厅中央,向着程珦跪拜问候。就算他已经是国之重臣,但还是依照旧时的礼节,没有半点改变。

韩冈谦恭守礼的态度,让程珦、程颢点头微笑,恐怕连父母都没见他笑过的程颐,脸色也缓和了许多。

程珦一直以来都对韩冈很是看重,当做自家的孙辈来看待,上下打量了韩冈一番:“果然是一次比一次更出色。当年初次从子厚那里听说起玉昆你得官的经历,就知道你绝非池中之物。现在看来,老夫眼光还是岔了,竟比料想的还要出色。”

“老大人的夸赞,韩冈可当不起。”

“怎么当不起?!”程珦听着不高兴:“晏元献【晏殊】当年上殿就童子科,与他同时的还有一人名为姜盖,比晏元献还小了两岁,同时得了功名。晏元献为人实诚,深受真宗所重。而姜盖小器速成,行事骄狂,时论其非远器,日后果然以罪废。还有那杨亿,也是性格骄狂,每每以年少骄人,戏辱同列,最后是不及五十而卒。玉昆论秉性就是与姜盖、杨亿不同,倒是跟晏元献相仿佛。”他左右看看两个儿子:“你们说呢?”

程颢和程颐都是谨守孝道,哪里会反驳,一起低头:“大人说得是。”

程珦拿着晏殊比韩冈,等于明说他未来必然少不了一个宰相。若是寻常人说来,可谓是满口谀词,但程珦开口,倒像是长辈对晚辈的勉励和期许。

韩冈可没脸皮大剌剌的听着,起身连声说着不敢当。

“玉昆你也不要自谦。子厚一向最看重玉昆你,写来的信上也都在说日后光大门庭,非你莫属。可惜他看不到了,连天祺也是一样。”程珦说起两个寿数不永的表弟,就有几分激动,抬头对两个儿子叹息着,眼中泛着泪水:“子厚和天祺比为父要小上许多,都没想到会那么早走。”

而韩冈也听着黯然神伤,“韩冈受学于子厚先生和天祺先生。在两位先生重病之时,却没能随侍身侧……”

“子厚表叔英年早逝,儒林之中又少一贤人,天祺表叔也同样可惜。”程颐一声感慨。

程颢不敢让老父太伤心,忙对韩冈道:“听说子厚和天祺表叔的祭田还是玉昆你帮忙置办的,还有安置我那两位表嫂和表弟妹的宅院和田地,也是玉昆你出力为多。你尽的这份心意也足够了。”

“区区身外之物,如何能比得上列位先生对韩冈的教诲之万一。”

程珦毕竟年纪大了,方才说起张载又伤了心,与韩冈说了小半个时辰的话,终于撑不住,起身回去休息了。二程和韩冈送了程珦入内,回来后,又重新分宾主坐下。

换了一回茶,程颢对韩冈笑道:“玉昆治政之才闻于天下,熙河、东京、河北和广西,皆留有遗爱,德惠百姓甚多。如今到了京西,可是本地父老之福。”

韩冈叹了口气:“只是一旦被庶务所累,与学问上能下得工夫就少了。”

“难道玉昆在广西的两年,就没有在经义上加以钻研?”程颐神情严肃的问道。

“经义当然一直不敢放下须臾,几年读下来,体会也是深了一层。”韩冈想想说道:“不过学贵于有所用,这两年学生多是想着如何将格物致知的一些心得放在经世济用上。”

“经世济用……”程颢将这个词默默的念了两遍,笑问道:“是‘为天下开太平’吗?”

“正是!”韩冈承认。。

“再修襄汉漕渠也是为了这几个字?”程颢再问。

韩冈点点头。

“不过京西近年多灾,民生困苦,人人惮于兴作。眼下若大修漕渠,恐会有所阻碍。”程颐的话已经给韩冈很大面子了,若是寻常官员想靠着大兴工程来求一个加官晋爵,程颐批评起来可不会留半点情面。

“正叔先生放心,此番兴造,学生会尽量使用旧时的渠道,只要稍加疏浚便可,并不需要太多的人工。真正需要大兴土木的是方城山的那几十里。方城垭口处,地势要高出近十丈,学生也不打算一举完工,而是打算用个十年八年慢慢开凿。”

“十年八年?”程颐皱起眉头,程颢也有点疑惑起来,“恐怕天子等不及。”

“不仅天子等不及,学生也不可能在京西任职这么久。学生是打算先建一条六十里的轨道暂代。轨道铺设简易,只需三千人足矣。甚至不用动用民夫,只用本地厢军就够了。有了轨道,开凿渠道,就可以慢慢来。”

程颢想了一想之后,就点点头:“洛水上的几个码头,都能看到玉昆你发明的轨道,的确是易于输送,打造也是简易,省耗人力。”

“但这样一来运力不是要比水运要少上许多?马车总比不上船只载货多,玉昆你又该如何回复天子?”程颐为韩冈担上一份心。

“襄汉漕渠只是对汴水的补充,主要是运输荆湖、两广还有一部分蜀中的货物,至于纲运的大头,还是得着落在汴水之上。”韩冈解释着,笑了笑,“依靠轨道居中转运,虽然多了一层手续,但一年百万石也不难为之。”

得到韩冈解释,程颢很是满意的向程颐投以一个笑容:“就说玉昆必有手段。安南一役有玉昆从中调度,也是从军力到民夫都比过往的战事省俭了不知多少。”

程颐点点头,但很快又皱了皱眉,对韩冈道:“就是对交趾男丁的手段有些过了。”

“十万血仇不能不报。但尽杀之又有伤天和,只能想个折中的办法了。依其罪论罚,刖刑倒也不为过。”

程颢、程颐都听得出来韩冈不想在此事与人争执,他们也就不多说,毕竟隔得远了,交趾又非华夏,而且也是对交趾人在广西屠杀的报复,圣人面前都能辨说得过去。

方才的气氛像是质问,程颢也是想要缓和一下气氛,换了个话题:“玉昆任职京西主要是为了开凿襄汉漕渠,那你接下来可是就要往唐州或汝州去?”

“没有那么急。”韩冈说道:“学生之前已经荐了沈存中去唐州,他在土木工役上才具当世少有人能及,有他在,之前的一番测量规划学生也都能放下不管。”

“也就是会在洛阳多留一阵子喽?”程颢很是有几分喜色。

“的确是要多待一些日子,兴造工役的钱粮也需要有些准备。”韩冈道,“明天处理一下公务,还得去河南府拜访一下。”

听到韩冈提起文彦博盘踞的河南府衙,程颢和程颐对视了一眼,程颢就问道,“是要去拜见潞国公?”

“文潞公判河南府,学生依礼数还是该登门拜访的。”

“玉昆,望你记得这个礼字。”程颐脸色沉重,提醒道:“文潞公忘了,你可不能忘!”

韩冈开怀一笑:“有两位先生训诫,学生岂敢失礼。”

韩冈如此说,二程便放心了一点。程颢又叮嘱着:“洛阳城中还有几位老臣,玉昆你最好都得去拜侯一番,不要遗漏掉。”

“学生明白,都会抽空登门拜侯。”韩冈也不是不懂人情场面,该尽的礼数当然不会忘记。自己做得有礼,对方无礼,那就不是他的错了,“除文潞公外,富、郑、王、范、司马诸公,韩冈皆是闻名已久,早就想当面拜会。如今正好任职京西,自不敢有所疏漏。”

