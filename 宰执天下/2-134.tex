\section{第27章 京师望远只千里(四)}

“韩官人……”何四躬起腰,陪着笑脸,“有什么吩咐尽管提,小店虽然破旧了一点,热酒热菜还是有的。”

何四神态语调的微妙变化,韩冈看在眼里,心知多半是自己的身份暴露了。不过他也并不是故意隐瞒身份,只是天寒地冻,官服太过单薄,不得不换了一件厚实罩风的外袄。

韩冈身上的外袄。里面填的是大雁腹部的绒毛,用碱水洗过,填进衣服里,再用针线纵横缝成格子状,基本上就是后世羽绒服的式样,比羊皮、狗皮或是狐皮之类的皮草,都要保暖得多。

如今这个时代,棉花还没有推广,韩冈让人寻找的棉种前段时间才送到古渭。平民家用的冬衣、被褥,好一点的人家用的多是丝棉,也就是碎蚕丝——禁军厢军到了秋时都会下发几两作为冬衣填料——穷一点人家则是芦絮。不过鸭绒、鹅绒用的人也很多,而在西北,蓄养鸭鹅的不多,牦牛绒、雁绒就成了首选。

在何四的招呼下,韩冈、李信坐了下来,李小六抱着包裹站在桌边,等着何四安排下房间,好把行李放下。

不待何四吩咐,跑堂的小九便提了一壶热茶过来,殷勤的把茶斟上。雾蒙蒙的水汽只是看着,就能感受到一点暖意。

“还请两位官人喝两口暖暖身,等下吃些热酒菜,小人就想办法给官人腾出一间房来。”得罪客人是做生意的大忌,但何四现在没什么顾忌的。在后院占了房间的有好几个商人,俗话说民不和官斗,商人最是敬畏官府。借着官威,让他们把房间让出来也不难。

对何四将要做的,韩冈心知肚明,也没有阻止的打算。仗势欺人也罢,欺压百姓也罢,这个时代,官员总能得到最好的照顾。韩冈无意故作清高,放弃有床有铺的房间,睡到大厅里的桌子上去。传出去也没人会说他平易近人,为人正直,反而言官会弹劾他有失朝廷体面。

给韩冈、李信倒过茶,何四转手也给李小六倒了杯热茶,面面俱到得很是会做生意。只是这个小客栈实在残破了一点,就连茶也是寡淡得很,跟白水没两样,但用来暖身已经够了。穿得再保暖,顶着风雪中走了两个时辰,韩冈三人都冻得够呛。端起茶水,韩冈双手握着杯子,从瓷杯中透出的热力,温暖着冻得发木的手掌。李信、李小六都喝了几口,脸色顿时好了许多。

何四吩咐了小九把三人服侍好,就往厨房跑去。体恤着一路来的奔波劳累,韩冈让李小六也坐了下来。三人今天都累到了,一时没心力说话,安安静静的一口口呷着茶。方才被他们惊扰到的其他客人,收回了好奇的目光,回到了自己的桌上。

安静的厅中一角,隔邻的两桌军汉的声音响亮了起来,“都虞什么时候醒?现在该午时了吧。”

“都虞被那蕃狗害得够惨,这几天他忙得连个安稳觉都没睡好,”

“你还真是能安得下心?明天要是不能赶到京兆府,可是要受军法的。”

“马都抢了,还要动军法,欺负人也没这么欺负的。”

“前面的经略相公没把俺们当人看,现在的宣抚相公把俺们当狗看,现在蕃狗都踩到俺们头上了,日他鸟的,连后娘养的都不如啊!”

“俺们他娘的就是狗.娘养的!”

砰的一声响,不知是谁用力捶了一下桌子,杯盘丁玲桄榔的掉了一地。韩冈随声转头瞥了一眼,只见几个军汉脸上尽是愤愤不平的恨意。

李信本是默默地喝着热茶,听到这里便抬起头,低声问着韩冈,“广锐?”

韩冈点了点头。前段时间,为了增强麾下蕃骑的战斗力,环庆路广锐军的战马被韩绛硬是夺了去,转交给蕃人。这件事闹得沸沸扬扬,连古渭这边几支骑军的指挥使,都跑来安抚司打探消息,生怕王韶、高遵裕有样学样。

不过韩绛自夺了广锐军的战马之后,就没传出进一步的消息,也没听说他再夺其他骑军的战马。韩冈估计韩绛也是知道错了,只是做出来的事已经难以挽回,从蕃人那里夺回战马交还广锐军,结果也只会更差,只能将错就错下去,但这梁子可就结下了。

听着这几个广锐军士兵的言谈,的确是怨气深重。因为李复圭枉杀大将之事,环庆路的军心已经被伤得很厉害,即便已经换了一个经略使也没有用处,而韩绛的作为更是雪上加霜。前段时间听说此事时,就算是高遵裕也都在说,换作是蜀中,说不定就要起兵变了——因为宋初灭蜀时留下的血债太多,自此之后,天下各路民乱兵变的次数便以蜀地为最。王小波、李顺等人就不必提了,蜀中甚至还有军队因为配发的军服不如人,士卒愤恨不平而起事叛乱的。

不过这跟秦凤路一点关系都没有,而四川是四川,陕西是陕西,西军闹兵变的几率并不大。韩冈听着有些嘈耳,只想着早点吃完饭,安排了房间去休息。

何四和小九跑进跑出,手脚麻利的端来了酒菜。韩冈并没点菜,都是他们自己上的。牛肉有禁令;猪肉则被视为浊肉,宫中一点不沾,富贵和官宦人家吃得也少。这种路边小店,能拿得出手的除了羊肉就是驴肉,再加点过冬的咸菜和白菜,就没别的菜蔬。

而端到韩冈桌上的,便是一盘子驴肉,一盘子羊肉,都是选得上好精肉,还有三大碗羊杂汤。还有两壶刚刚烫过的热酒。

方才了这间小店的茶水,韩冈对这里的酒菜并没有什么期待。不过出乎他的意料,酒也好、肉也好,都比想像得要出色。尤其是酒,没有兑一点水,且是筛过了,倒在杯中清亮澄澈,酒香四溢。喝进肚里,感觉不比和旨、眉寿之流的名酒差。一杯下肚,连李信也都点着头,赞着酒菜的味道。

砰的一声响,从韩冈的身后传来。一个粗壮的军汉一拳捶在桌上,冲着何四吼道:“你这狗才倒长了一对势利眼,端给几个鸟货的都是好酒,给爷爷的酒却能淡出鸟来!嫌爷爷没钱付账是不是?!”

何四脸色变了,连忙摇着手,“客官,你这可是冤枉……”

但那军汉却无意听何四解释,手一伸,就把他扯了过去。脸对脸的瞪着何四,醋钵大的拳头举了起来:“冤枉什么?爷爷好说话,但这拳头可不好说话!还不给爷爷拿跟着几个鸟人桌上一样的酒来!”

何四给别人的酒中掺水,这是自做的孽。但被人骂到了头上,李信便脸色一板,握紧了拳头,正要站起来,可韩冈却一下压住了他的手。

韩冈看跳起来的军汉横眉竖眼的样子,摆明了就是喝醉了的兵痞,其他人应该也差不多。前面他们还都坐在一起抱怨,若是跟他们起了冲突,他们秉着同仇敌忾之心,一起上来动手也不是不可能。出门在外,凡事须先避让三分。眼下地方不对,韩冈决不想跟这些兵痞叫劲。千金之子,坐不垂堂,反正他有的是把面子找回来的机会和手段。

韩冈笑了笑,正要说话。一声怒喝猛然响起。“林贵!你做什么!?”循声望过去,却见一个中年军汉站在通往后院的小门处。

“都虞!”被唤作林贵的大汉惊叫着,连忙松开了手。何四幸运脱身,就手捂着喉咙,弯腰咳嗽起来。

中年军汉大步走了过来,两桌的赤佬便呼啦啦的全都站起身,看起来很有些威望的模样。他大概三十多岁,壮硕的身材看起来英武非常。他几句话问明了事由,转回来便向韩冈作揖道歉,说起话来是温文有礼,“在下邠宁广锐军都虞侯吴逵,我这几位兄弟性子莽撞,不合冲撞了兄台。还望兄台大人大量,不要与他们计较。”

“都虞!……”

林贵还想争辩,吴逵回头瞪了一眼,“你闭嘴,看你们闹得!”

邠州、宁州都是环庆路辖下,果然正是被夺了战马的广锐军。韩冈微微浅笑,面子是互相给的。吴逵低头,他这边也得给人台阶下,“酒后失言,也算不上什么大事。既然几位都觉得我这酒好,那我就请各位喝两杯好了。店家,再取几坛酒来,都算在我的账上。”

吴逵是个疏阔的性子,也没发现韩冈在他报了身份之后,仍旧安然坐着有何不妥。见韩冈做事爽快,他大笑着,拉了张椅子过来,就要跟韩冈说话。

不过这时候,大门又被敲响,匡匡的,像是有人在踹门。

何四忙不迭地跑过去开门,门一开,随着风雪一下涌进来七八个军汉。他们可不像韩冈进来的时候那么安分,领头的一人先一脚踢开挡路的何四,站在厅中高声道:“我家将军今天要住店,里面的人把房间统统都给让出来!”

狂妄的话语惹起了一阵骚动,只是从大门处又进来了十几人,围着一个近七尺高的大汉。看那大汉相貌是个标准的蕃人,可装束却是个有官身的武臣。

吴逵一下变了脸色,低低恨声叫着:“王文谅!?”

王文谅……韩冈心中一动,这好像就是夺了广锐军战马的蕃将的名字?

王文谅进来后,视线在厅中扫过,看到吴逵便一下定住,转眼就又笑了起来,“这不是吴都虞吗?事都办完了?”

吴逵脸色彻底沉了下去,咬着牙,两边的腮肉绷紧:“本官要回禀公事,要么是王经略,要么是韩宣抚,轮不到你这蕃人来说话。”

“你这张嘴还真硬啊……”王文谅呲着牙阴笑着:“宣抚相公可是对俺言听计从。俺要说这里面全是北面的细作,宣抚相公就能把他们的头全都砍了。”

厅中的客人们闻言都惊怒的叫起,也有心思灵活的就准备掏钱买平安了。

“是吗?”冷澈的声音从吴逵身后传来,“本官倒不觉得你有这能耐!”

