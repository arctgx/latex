\section{第37章 长安道左逢奇士(上)}

【昨天自动更新出了问题,把今天第三更的章节提前发了出去。这是俺的责任,俺不推脱。今天晚上十点加更一章,作为补偿。第一更,求红票,收藏。】

这样都会被撞上,刘仲武算是认了命,不再挣扎。第二天,便老老实实的随着韩冈在长安道上并辔而行。

从咸阳往潼关去,有两条路,一条是继续顺着渭河下行,一条则是先往南绕去京兆府。这后一条路,便比前一条要多上一天的时间。不过韩冈一开始就决定走长安去,想近距离的接触一下这座千古名城。而写在驿券上的路线,也是这么安排的。

出了咸阳城,他们的行程便离开了渭水,而是转往东南。路上的行人多了起来,都是往京兆府去的。作为数千年的古都,如今陕西路的重心,原名长安的京兆府人烟辐辏。从陕西西部的群山峻岭中出来,富庶的关中平原便出现在韩冈的眼前。

八百里秦川大地,举目无垠,不论向哪个方向望去,都是一条平坦的天际线。官道两侧的雪原之下,良田以千万计。周、秦、汉、唐皆籍此而得天下,实实在在的帝王之基。

走在通往京兆府的大道上,时不时的越过几家行商的驮马或是车队。商人重利轻离别,尽管还没有度过上元节,但性急点的商人们,早早的就留下妻儿看守家门,自己带着货物上路。

“嚯!”行进中,李小六突然指着前面,惊叹了一声,“那骡子还真能驼东西。”

韩冈远远望过去,就在前行的方向上,一座小山出现在他们眼前。被小山般的包裹压在下面是一头骡子,若不是能看到四条腿和尾巴,旁人还会以为是包裹自己在走路。

韩冈一行很快越过骡子,从旁边疾驰而过。他只瞥了一眼,却惊见包裹的前面竟还坐着一人。既要驮着包裹,还要背着骑手,韩冈不禁可怜起这头晃晃悠悠、随时都可能倒毙在路上的老骡子,‘唉,前世不修,阴德不够,没能投个好胎啊!’

越过骡子,并没有走多远,前路便堵了起来。韩冈对此习以为常,那是地方上的税卡,也是越过州界的标志。他一路过来,经过了不少处。不过再怎样的税卡,也查不到他这个官人头上。道路两边的积雪使得他们不便绕行,而前面的队伍又不长,韩冈和刘仲武便耐下心来等着。

几个税吏,再加上三十来个土兵,在税卡前挨个搜检。他们的任务与后世海关的工作差不多,都是向过关的货物征税,并没收其中的违禁品。尤其是从西夏的青白盐池那里来的私盐,绝对是最主要的稽查对象,除此之外,酒、茶、矾、兵器也都是一样严禁私运,列于稽查目录中。

税吏的稽查,无论是行人还是普通的商旅,皆是一视同仁,一个个包裹无论大小都要打开,搜检得十分细致。一个运气不好的胖商人,不合在包裹里放了十几饼团茶,便被拎了出来,东西被没收不说,还要罚上一笔钱。

胖商人在税吏面前分辩着,一口的蜀音让人听不出他在说什么,但看他不服气的样子,这十几块团茶应是他带着自用或是送人的。数量这么少,本也不可能是要卖的货。可税吏籍此向他开具的罚单,却让这个胖子在大冬天里,头上热腾腾的直冒着汗。

可税吏们不管。见胖子不服,领头的一个留着一撮山羊胡子的税吏,随手一指胖子蜀商,几个土兵便立刻冲了过去。三下五除二,便把胖商人和他的伴当捆成了两个麻团,就撂在路边的雪地里。而原本胖子蜀商带着的驮着绸缎的三头骡子,也被牵到一边。

只看税吏和土兵们难掩脸上的欣喜之色,这三头骡子连同背上的财货,究竟是没收入官,还是被私分,说不定还要计较一番。至于还给商人?韩冈从没听说过胥吏军汉们的道德水准有这般高度。

韩冈心中不解,他前面经过的几处税卡,全没有这般森严,也就是私盐和军器查得严厉一些,其他的违禁品都是一串大钱塞过去,便能挥手放行了。京兆府的税吏是吃错了药,还是没钱过年?这时间也不对啊!

韩冈想不通,也许其他商旅也想不通。可是有胖子蜀商做先例,后面的商旅们便没一个敢再炸刺,老老实实的接受检查。一个接着一个,最后轮到了韩冈和刘仲武这边。

两个税吏走了过来,瘦高的一个对上刘仲武,个头矮的一个找上了韩冈。

刘仲武高居马上,仰头看天,鼻孔瞧人。右手拍了拍他跨下这匹赤骝的脑袋,冷哼着:“看看洒家骑得什么马?”

“什么马?”瘦高税吏也从鼻子哼着回了一句,但他定睛看过赤骝后,立刻不敢再废话多舌。大宋缺马,尤其是战马。肩高四尺二就算合格,而刘仲武的爱马少说也有四尺五以上,十足十的河西良驹。这不是普通军汉够资格骑乘的,没点身份,谁能骑上去?

矮个税吏则来到韩冈马前,韩冈也骑在马上没动。他的眼睛没去瞧税吏,而是看着陷在雪地里胖子蜀商。原本因为紧紧勒着身体的绳子而涨得红紫的一张胖脸,现在已经泛白发青,大半条命都去了。有进气没出气的样子,动也不动弹,也没几口气了。

韩冈缓缓地抬起手,指着胖商人,慢吞吞的说道:“让他吃过苦头就够了,莫闹出人命!大过年的,你们想让你家钱大府过不痛快不成?”

韩冈的声音平平淡淡,口气却大,比骑着高头大马的刘仲武说话更有威严。两名税吏也是阅历颇深,都知道面前的两人不是他们能招惹得起,跑回去找了山羊胡子过来。

山羊胡子一来,看着韩冈、刘仲武两人的作派,便知是有些身份,或者有个好后台,但两个人就带了一个伴当,怎么看也不是有官身的样子。而他领的命,是陕西路排在前五的人物下达的,底气十足:“对不住二位,此是公事,小人不敢疏忽。左右只是查一下包裹,二位都是有身份的,想必不至于让小人为难。”

刘仲武不说话,转过来看着韩冈。有韩三官人在,轮不到他这个军汉出手。

什么时候这些税吏胆子变得这么大了?

怒意在韩冈的眉头聚起,锋锐如刀的眉眼在怒火中犀利如电,而他的声音则越发的轻和起来:“诸位尽忠职守,本官深感敬佩,明日去见了钱府君,倒要向他赞上两句。”韩冈说着,又从怀里将驿券和公文抽出来,向着税吏们亮了一下。

看到两颗鲜红的大印,山羊胡子倒抽一口凉气。走眼了!竟然真的是官!他干咽了口吐沫,正要说话,韩冈却笑道:“本官受命入京,只带着这两样。剩下的都是些不着紧的什物,你们要查尽管查好了。公事公办嘛……好说,好说。”

山羊胡子心中发寒,韩冈这话说的,摆明是记恨上了,他一个小小的税吏,哪经得起一个少年官人的惦记,忙赔礼道:“官人勿怪!官人勿怪!这也是奉了转运陈相公之命,不关小人的事啊……若在往日,小人纵有天大的胆子也不敢扰到各位官人啊!”

转运陈相公?转运使不姓陈,而转运副使则名叫陈绎,山羊胡子说得应该就是他,但这又关陈绎什么事?韩冈疑惑着。

转运司主管一路钱粮,其实是分司民政,甚至有时候还有审理案件的权利。如陕西,负责军事的经略司有缘边的秦凤、鄜延、泾原、环庆四路,加上以京兆长安为中心的永兴军路,总计五路,但转运司,却只有一路,就是陕西路。

按照朝中规定,路份监司官,如别称漕司的转运使,宪司提点刑狱使,仓司提举常平使,每年都必须花上一半时间来巡视辖下州县,而当监司主官不在衙门中,那各司的实际事务,便是由始终留在治所的副使来处理。论权位,转运使和转运副使差得并不太多。

只是转运副使地位虽高,但陈绎跟税卡之间还隔着州县呢,他怎么能绕过州官县官,直接插手税卡?韩冈一时之间想不通。

山羊胡子不停的对着韩冈鞠躬道歉,为自己辩解,也不敢再坚持搜检。反正韩冈是骑着驿马,臀后有着烙印,而挂在马鞍后的包裹又是不大,也不可能私下夹带。谁知道这位年轻官人身后有什么后台,过于尽忠职守反会害了自己,抬抬手,便示意要放行。

“不查了,那怎么行?”韩冈摇着头,正色说道:“大宋律条均在,尔等岂能轻违,纵使本官也不能大过国法去。小六,你把包裹都打开来,给几位‘官人’看一看!”

韩冈不依不饶,山羊胡子面色如土,几乎吓得要瘫倒。韩冈方才亮出来的公文、驿券,他只看清了大印,但韩冈是明明白白的官人作派,连这个记恨小人冒犯的脾气,也是跟他见过的官人们一般无二。

俗话说宁欺九十九,不欺刚会走,像韩冈这样才二十上下便做了官的年轻人,不是才学高,早早的考上进士,便是投了个好胎,承了荫补。不论是哪种,都是动上一下,后面就有一大堆亲戚朋友跳出来,最是招惹不起。山羊胡子在衙门中多年,哪能不知?即便是转运陈相公也不愿无故得罪这样的人。他忙带着一众手下,在韩冈面前跪着请罪。

一群税吏在韩冈马前磕头求饶,请罪声不绝于耳。刘仲武和李小六都看傻了眼,知县来了都没这么大的谱,好歹得来个知州通判还差不多。

韩冈冷眼看着,也不说话。并不是他不肯饶人,只是因为陈举和黄大瘤的事,他对胥吏没有什么好感。现在几个税吏犯到自己,心中便忍不住升起一股子戾气。过了好半天,他心中怒气稍可,方才问道:“到底是出了何事?”

看得出今次应是陕西转运司下了死命令,要不然哪个胥吏会为要缴给朝廷的商税,而跟官员过不去?能弄到这个油水丰厚的职位,没一个不是人精,轻易不会得罪人。

