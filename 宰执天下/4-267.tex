\section{第43章 庙堂垂衣天宇泰(14)}

在京中流布数日的传言终于得到了证实,韩冈以身份、地位,以及在医道上的声望作保证,上书天子,声明困扰了天下无数生民的天花——或者叫痘疮——已经被成功制伏了。

毫无疑问,这是值得亿万人为之欢欣鼓舞的喜事。再多的大捷,再辉煌的胜利,也比不了一份能让疾疫远避,惠泽天下黎庶的医方。

但与此同时,皇第七子建国公赵价因痘疮而夭折的消息也传遍了京中。

这一天,京城中弥漫着一股诡异的气氛。

有人笑,有人忧,有人则是摇头感叹。

但普通的官员百姓还是关心着自家儿孙的安危,尽全力去打听其中的究竟。不是什么秘密,也没人刻意隐瞒,韩冈写在奏章中的内容,当天午后便在京城官宦人家传开了,再过三五日,街边卖油炸馉饳儿的小贩,多半都能知道韩冈在广西发现了不得天花的养牛人,结合了早前在神秘的孙道士那里学到人痘之术,运用格物之道,得到了如今种痘免疫法。

一朝得授于仙,继而又辛苦寻觅十年,锲而不舍加上细致入微的观察,最后在广西出现了转机,这是很有传奇性的一个故事。

对发明了安全无害的种痘免疫之术的韩冈,京城军民自然都是感激不已。当然,对于之前隐瞒了仙家传授的人痘之术,多少有些腹诽。不过,要除去自家的子嗣最近几年因痘疮而病夭的那些家庭。

所以人人都在看着天子,看他打算怎么发落韩冈。

傍晚的时候,章惇若无其事的离开了宫城,神色如常的与同列告辞,回府后见到家人,也看不出有任何一样,直到踏进书房,才终于变了颜色,

“韩玉昆啊,韩玉昆,这次可真的做错了。”

章俞走进来的时候,就看见儿子手按着额头,低低的说着什么。

“是为了韩冈的种痘免疫法?”章俞站在门口,出声问道。

章惇听到声音,猛然抬头,看了一眼后就连忙站起身,将座位让给章俞:“大人回来了?”

章俞坐下来,抬头追问:“是韩冈出事了吧?”

“今天上午的事。”章惇点头后,警觉的反问道,“父亲大人在哪里听说的?”

“方才在樊楼听人说的,弄得都没心情喝酒了……”章俞身上还有着酒水和脂粉的味道。儿子都执政西府了,他还是照样喜欢呼朋唤友的招妓饮宴,往往夜半方归,“能在樊楼里面喝酒的,果然都不是简单人物,为父跟礼院张伯约和曹家的老四坐一起,听到消息就让妓女都出去了。谁想到还没说两句,樊楼上下都没了丝弦声。”

对于自己父亲的喜好,章惇无可奈何,“想不到这么快就传出去了。”

“寻常点的消息,从宫里传出来也需要一天两天,但军情从来不过夜,这一次的事,比军情又不知重要上多少倍。”章俞摇摇头,叹道:“事情太大了,前几天,种痘术的传言刚兴起的时候,就有人盯着通进银台司。咸宜坊第一区的那一位,比天子和东府恐怕都要早一步看到韩冈的奏章……虽然是抄本。”

章惇的脸顿时冷了起来:“贼心不死!”

“万里江山,亿兆子民,能死心吗?”章俞冷笑的说了一句,又正经起来问道:“天子是怎么看韩冈奏章的?”

章惇回忆起天子看到韩冈奏章后铁青的脸色,摇了摇头。天子一怒,伏尸百万,包括他章惇——胆大包天、让苏轼评价为‘能自判其命,故能杀人’——在内,所有大臣都不寒而栗。

“建国公的病夭,给了天子很大的打击。人都糊涂了,正常是该辍朝的,却一大清早莫名其貌的坐在了文德殿上,回到崇政殿也没有恢复,直到看到韩冈的奏章……”

“难怪。”在樊楼中听说今天天子依然临朝坐殿,章俞还觉得奇怪,这才知道整个人都伤心糊涂了,行事只知道照着日常习惯走。他本人是没有这个情况,但也曾经见识过。

“韩冈的奏章是走马递,从银台司直送进崇政殿?”章俞又问道。

“一直都是如此。要不然在政事堂中耽搁一天,情况还会好些。”章惇无奈的摇头,“韩冈奏章到的时候太不巧了,正好刚刚议定建国公如何追封——太师、尚书令、魏王,谥悼惠,从明天开始辍朝三日……”

天子没有抢过殿上力士手中的金骨朵,将御桌和摆在御桌上的奏章一起给砸了,章惇都为天子的冷静感到惊讶……或许是气到手脚发抖,站不起来了。天子当时可是亲自读着韩冈的奏章给他们这些臣子听啊!那个声音,本应在最让人恐惧的噩梦中才会出现。

章俞也快站不起来了。他现在是听得如同光着身子站在雪地里,然后一盆冰水倒浇下来,从囟门到脚底都直冒凉气。

天子也是人!新近丧子的父亲,谁的精神上能受得住这样的刺激?韩冈也真是倒运。

皇子前夜死,奏章今天到,这时机已经糟糕透顶了。偏偏抵达的时间,还糟糕透顶中的最要命的那一刻,真不知该如何去形容韩冈的运气了。

章惇算是知道当初文彦博在殿上兴致高昂骂着河湟损兵折将、祸国殃民,突然一封捷报送来,说是熙河路斩首几千几万,到底是什么感觉了。

自己还是旁观者,今天在殿上,都已经是心惊肉跳,韩冈在京西,襄汉漕运、种痘之术,两样大功攥在手上,恐怕正是志得意满的时候,但建国公病卒的消息传过去,他的心情也许会跟刚刚致仕的文彦博一样。

“仅有的两名皇嗣现在就只剩一个。不说之前几年夭折的皇子公主了,就是韩冈能早上一个月将种痘法传来京城,好歹能将建国公给保下来。”

“韩玉昆行事谨慎害了他。”章惇很无奈,“在殿上听天子读着,儿子就知道事情不好了。丧子之痛,怎么跟天子说理?韩玉昆的确有理由,但天子如今的心情,怎么会管他的理由?”

皇帝对臣子的要求是什么?

第一条就是忠,第二条是忠,第三条还是忠。所谓事君惟忠,才能啊,德行啊,都得放在后面。

整件事,韩冈不犯刑律,依朝规也无过错。但在天子看来,不管韩冈怎么打算,他留着能挽救皇嗣的种痘法没有献上去就是不忠的表现。

将心比心,如果自家遇上这样的事,自家好几个儿子死在痘疮下,而朋友还藏私,慢悠悠的找着更好的方子,章惇肯定是认为这个朋友该杀上千刀——幸好没有,否则章惇肯定要跟韩冈翻脸。

救急如救火,当年韩冈领军南下,救援邕州,一路走得飞快,打了个李常杰措手不及,怎么偏偏这件事上变成了慢郎中?

“真没想到韩冈怎么这般失策,过去看着多聪明的一个人啊。就是没有建国公的事,天子听说韩冈将人痘法藏了十年,心中也会好一阵不舒服。在奏章中,他根本就没必要将孙真人扯进来,直接说在广西无意中发现的不就好了?‘不经明验,不敢献上’,当做借口怎么也能糊弄过去了。换成是孙真人传授的方子,哪里需要试验?!”章俞为韩冈叹了口气,“可能是太顺了。年纪轻轻就是一阁学士,看人待物都没过去的灵气了。”

“天子这般作派,明天少不得就有御史上本弹劾韩冈。种痘之事上,韩冈并无罪。但欲加之罪,何患无辞?”章惇叹道,“那群乌鸦,看到有人要跌倒了,肯定就会围上去,不可能会放过的。一着不慎满盘皆输,光是为了这一件事,天子能一辈子不待见他。”

明明身怀能挽救多少皇嗣的奇术,偏偏拖了整十年。韩冈凭着才能、功绩得到的圣眷,这下子肯定是烟消云散。

韩冈的才能即便冠绝当时,天子若是耍脾气,就是不用他又该怎么办?

嘉佑末年,翰林学士兼三司使的蔡襄本有一造两府的资格,但他据传在是否让英宗皇帝继承大统的问题上有过反对意见,等英宗登基后,一被御史弹劾就被打发出去了。

照惯例,高官被御史弹劾,即便是宰相也要归家待罪,自辩或是上表请罪,乃至请郡出外。而天子则会将请郡的奏章驳上几次,这是为了顾全士大夫的颜面。偏偏就是落在蔡襄身上,英宗皇帝直接就批准了,根本就不驳。

韩琦为此还问英宗,“自来两制请郡,须三两章。今一请而允,礼数似太简。”英宗的回答很妙:“使襄不再乞,则如之何?”

天子看不顺眼,自然就没办法,韩琦尽管是顾命元老、助英宗登基的第一功臣,也不便帮蔡襄说话,让蔡襄去了南方,没两年便病死。

“如今朝堂上希合上意的佞幸之辈甚多,不知子厚你打算怎么做?”章俞难得叫着章惇的表字,神色很是严肃。

“韩冈无负于我,过去又多得其力,如今之事又非韩冈故意而为……”章惇摇摇头,正色回复,“若还有人若想以不实之罪加诸其身,儿子当会上书。”

章俞看了章惇半天,最后叹道:“那就先给襄州写封信吧,虽然肯定会有人给韩冈报信,但你这封信却少不得。”

