\section{第五章 九州聚铁误错铸(一)}

【此是第二更】

鸭子河畔的春捺钵的太师大帐中,几名秃发短辫、发结金环的女直人,正跪在耶律乙辛的面前。

这几位女直人有老有少,身上穿着粗糙,布料都是最低档的,甚至还有用大块的兽皮裹着腰。衣着装束与南方生活在辽阳以东的熟女直截然不同,是典型的生女直的打扮。只有领头的一人身着辽国官服,不过衣服已经是很旧了,甚至留下了洗褪色的痕迹。

在他们的膝前,则是一字横排的摆放着十几枚头颅,连包装都没有,直接将头发打结用绳子系在一起。

这些头颅全都是典型的契丹发式,剃去了头顶部分,剪短四周,在颅侧部位,则像帘子一样蓄两绺长发下来,垂于耳侧。

这些头颅看下来有一段时间了,至少五六天。虽说当是用盐抹过,可由于没有腌好,全都已经发黑发臭,正从断口处向外淌着浓汁,将耶律乙辛富丽堂皇的一顶大帐变成了城外的弃尸场,帐中臭气熏天,连香炉中烧的沉香都压不住阵脚。

一贯喜欢干净清洁的耶律乙辛却完全不介意,脚下的一枚枚发臭的首级,让心情变得十分的高涨

尽管出现在他脸上的表情,依然镇定如常,仅仅露出了一丝矜持的微笑,完全没有异样。可他的手却不听使唤的微微颤着着,稍稍泄露了一点他内心的激动。

自从来到鸭子河畔之后,耶律乙辛的情绪还从没有这么激动过。他当日启程离开冬捺钵,领军移驻东京道。对外宣称是去春捺钵的所在地。但实际上,却是调兵遣将直扑辽阳府,将从属于窝笃斡鲁朵的势力连根拔除,数日之间,辽阳府外被杀得人头滚滚,血色漫天。

平乱之事,是半点拖延不得。曾经亲手为先帝剿平皇太叔耶律重元之乱,耶律乙辛很清楚不能给叛军发展壮大的时间。

即便身处东京道的窝笃斡鲁朵只是保持沉默,甚至还没有举旗说要清君侧、为先帝复仇,但要用来警告一干有反心的猴子,耶律乙辛可没时间在意要杀的鸡会不会打鸣。

只要不肯顺服,杀了就是了。难道还要给他们时间合纵连横,会集兵力,将反旗举起吗?耶律乙辛做事,这一次也没有犹豫。

可能会反叛的势力,还没有做好开战的准备,就被耶律乙辛连根割断了。下面的士卒十不存一,唯一一件事让人遗憾,就是领头一干全都跑了个干干净净,一个比一个溜得更快。

幸好东京道能让人藏身的去处并不多,南面是耶律乙辛控制最为严密的地区,西面的上京道和中京道,也同样被耶律乙辛拿到了手中。东面是高丽和和大海,根本跑不了。只能继续向北,向比契丹更为野蛮,也更为桀骜的生女直部族逃去。

也许那些逆贼在逃亡的过程中,还转着说服几个女直部族,然后卷土重来的幻想。

可惜的是,耶律乙辛早已安排好了,前些年的五国部叛乱,大辽国的权臣就将自己的手伸到了混同江两岸。而之前直扑辽阳府的时候,耶律乙辛也拍了得力人手,去联络女直各部,让他们提前做好准备,等着猎物自己撞上来。

计划也许很粗陋,但结果却是让人满意的。

从轮廓上,耶律乙辛还是能认出来摆在前面的几个首级曾经的归属。

不顾迎面而来的恶臭,耶律乙辛将其中一枚头颅双手捧起。

漆水郡王,窝笃斡鲁朵的实际控制者,耶律乙辛的心腹之患,日后必然能成为叛党核心的敌人,眼下却成了一堆烂肉。

耶律乙辛为这枚首级理好了头发,捧到了近前,面对面的正视着,“一年之前,吾与兄尚谈笑甚欢,岂料一载易过,转眼间就已是天人相隔。”

伤心感怀的声音在帐中回响,眼角溢出了几滴泪水。不论任谁来看,都能从耶律乙辛的话语和神情中,体会到一股沁透人心的悲凉。

“如果兄台能与乙辛携手奉上,共扶幼主,堂堂大辽岂会被南朝所看轻。眼睁睁的看着宋人要攻打西夏,却无力相助。”

叹了一阵世事无常之后,耶律乙辛随即一扬手,将手中的头颅递给帐下的亲卫。

亲卫队长接过首级,自作聪明的问道,“太傅,可是要好生安葬?”

耶律乙辛随即一瞪眼,厉声喝道:“安葬什么?!挂出去,在辽阳城头上给我张上三天。三天后传首五京道。让所有人给我睁大眼睛看着,敢于违抗朝廷的下场!”

原本出现在耶律乙辛脸上的悲伤仿佛是梦一般,转眼间就无影无踪,再也看不到半点迹象。

马屁拍在了马腿上,亲卫队长忙指挥手下慌里慌张的做起了搬运工。一枚枚首级被搬了出去,随即外面就是一阵鸡飞狗跳,亲卫们一连声的喊着,让人将这些头颅都按照耶律乙辛的吩咐都送出去。

尽管拿出去的仅仅是十几颗头颅,可帐中给人的感觉却是一下就空了许多,只留下了阵阵恶臭,以及地毡上被脓水浸透的痕迹。

除此之外,当然还有少不了给耶律乙辛带来喜悦和感慨的这一班女直人。

耶律乙辛斜倚着身子,靠着虎皮软榻,“劾里钵,你完颜部此次做得甚好,如果让这些叛逆去了五国部,就少不了又是一场大战。”

领头的劾里钵,是完颜部的部族长,承袭了生女直部族节度使一职,在混同江两岸的女直部落中,一向被恭称为太师。

不过完颜劾里钵的太师只是叫得顺口而已,在大辽太师兼太傅面前,却连站起来的资格都没有。

“太傅的吩咐,小人岂敢不听。得到令旨之后,就派了族中人马,在各条路上守候。也是上天垂顾,太傅的齐天洪福相助,终于是给小人等到了。”

完颜部的部组长态度摆得很正,这让耶律乙辛很是满意。有心要好生赏赐一番,给世人立个榜样出来,让大辽上下,都知道他的慷慨。

劾里钵的身后跟着两人,一长一少。大的二十多,小的则只有十三四岁。

“他们是你儿子?”耶律乙辛心中一动,手指抬了一下,出言问道。

“回太师的话,大的是小人的兄弟,小的才是小人的儿子。”劾里钵见耶律乙辛心情好,知道这是难得的机缘,回身指着兄弟和儿子,向耶律乙辛介绍着,“这是小人的弟弟盈哥,最是武勇,此番奉旨杀贼,正是盈哥率先冲杀过去。这是小人的次子阿骨打,才十二岁,有些小小的运气。小人一听太师将至,就带了盈哥和阿骨打出来,只留了长子乌束雅看家。”

劾里钵身为完颜部的一族之长,虽说有个官职在身,但混同江两岸,身上带着节度使、团练使的女直族长多得手指脚趾加起来都数不完。

家里有不服他管束亲叔跋黑,族中有要翻脸的桓赧和散达。外面还有乌春、窝谋罕。举目皆敌,纵然劾里钵心中如他的父亲一般桀骜,不愿辽人插手进生女直的势力范围,可眼下也只能认命。

如果抱上了大辽太师的大腿,跟随着有实无名的皇帝,不论是跋黑,还是桓赧和散达,又或是乌春、窝谋罕,他劾里钵只要一根手指就能像碾臭虫一样将他们碾碎。

甚至是更北方的五国、东海两大女直部族联合,劾里钵也有信心与他们斗上一斗。

在劾里钵期盼的目光中,耶律乙辛不介意让自己慷慨的名声传得更广一点,

“完颜盈哥!”耶律乙辛叫着名字,手向后一伸,将身后侍从捧着的金刀拿过来,递了过去,“这一仗杀得好,这柄刀就赏你了。”

完颜盈哥双手高高举起过头,恭恭敬敬的接下。看着嵌着宝石、以鱼皮做鞘的宝刀,他喜不自胜:“小人得了太傅的赏赐,日后只要是太傅的吩咐,叫小人杀谁就杀谁。”

“阿骨打……”满意的接受了完颜盈哥的效忠,耶律乙辛的视线又转到了劾里钵的儿子身上,“可是亲手斩了耶律哈葛的阿骨打?”

“第一次上阵,运气好而已。”劾里钵谦虚着,但语气中不掩对儿子的自豪。

“胆气也不差。”耶律乙辛说着就赏了一张宝弓给阿骨打。

“劾里钵你好福气啊,兄弟、儿子两人都难得的英武。这样吧,盈哥就跟着我,至于阿骨打……”耶律乙辛看了看虽然年纪不大,但身材已经跟成人差不多的完颜部部族长的次子,“天子身边正好还缺个护卫。”

劾里钵心中欢喜,这一下子,大辽太师的大腿可是可是彻底的抱上了,连忙拉着弟弟和儿子叩头谢恩:“能得太师看重,是他们的福气。”

“小人一心一意,听太傅使唤。”完颜盈哥磕着头。而完颜阿骨打则是沉默的磕着头,看起来像是被吓到了。

耶律乙辛笑着点点头:“劾里钵,你部今日立此殊勋,本太傅不能不赏。日后完颜部的马税就此减半,我这里还有五百套铠甲,一千套弓刀,加上一千匹南朝的丝绢,也一并赏了你。从你上阵的。战死之人,给他家人五十匹绢五十两银,许他一个儿子做官。受伤的银绢减半。参战的,一人十匹绢。”

对跪倒谢恩的劾里钵,耶律乙辛俯身道:“只要你能一心效顺朝廷,我不会吝啬一分半点!”

