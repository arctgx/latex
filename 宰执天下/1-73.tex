\section{第32章 营中纷纷难止休(下)}

跟在赵隆身后,是先前进去的青唐部蕃人。他们都是结着粗大的发辫,盘在头顶,油腻腻的,多日没有洗过的样子。身上穿的也是一层层交叠起来的刺花袍服,内里是羊皮,外面则是上好的丝缎,形制与后世的藏族服饰的区别不是很大。领头的吐蕃人,肤色黝黑,风吹日晒的容貌判断不出年龄,三十到五十都有可能。

陪着蕃部首领出来的是刘昌祚,韩冈心知,能让刘昌祚亲自送出来,又能让王韶命赵隆引出厅门,这蕃人身份肯定不低。

“是俞龙珂的兄弟瞎药。”刘昌祚送着一行蕃人出门便回转厅中,两名亲卫带着他们继续往城衙的大门去。赵隆也要转进去复命,却被韩冈拉住,问了来人的身份,竟是青唐部族酋的亲兄弟。

“鸟名字……”王舜臣冲着瞎药一行离开的方向吐了口吐沫,他的父亲虽不是战死疆场,却是死于旧日与西贼对垒时所中的箭疮,每天夜中听着父亲躺在床榻上的呻吟,就是王舜臣幼年时代最深刻的回忆,论起对蕃人的看法,不论党项还是吐蕃,他比韩冈、王厚都要偏激,“蕃人就是蕃人,就不会起个正经名字!姓俞的弟弟,竟然姓瞎……该不是他家老娘给他们找了两个爹吧。”

韩冈失笑,蕃人的名字的确够怪的,但朝廷给归附蕃人的赐姓赐名同样不靠谱。赵思忠,赵保忠,赵尽忠,幸好没了赵全忠——因为不吉利。

“哪里不正经了……”王厚吃吃笑道,“‘鱼’‘虾’本就是一家吧?”

也许是王厚声音高了一点,瞎药突然停步,回头瞥了一眼过来,眼中带着冷意。

瞎药的眼神狼一般的桀骜不驯,还有着几分阴毒,王厚看得很不舒服,冷冷的哼了一声,韩冈则微笑着平视了回去。他上一次看到这样的眼神,还是另一个世界的事。韩冈前生曾经待过很短一段时间的某家公司,当时他所联络的某位客户的一个下属,也是有着如狼一样桀骜不驯的一对眸子。

韩冈的性子其实说起来也是一样桀骜,但他知道如何掩藏,而不似那个还没学会掩盖心思的蠢货。那人据说不久之后便莫名其妙的被一辆无牌大卡碾成了两段……野心大点没什么,可别写在脸上,哪家老大也容不下这样的小弟。

瞎药已经走远,韩冈却还在回想着他的眼神,俞龙珂恐怕也不喜欢看到瞎药这个兄弟,难怪大过年的把他踢出来送礼,“看起来瞎药不似会甘居人下的样子……”

“不甘居人下?”王厚怔了一下,突然阴笑起来,“他上面就只有俞龙珂了吧?不如我们就推他一把,让他跟俞龙珂争上一争。”

“对付一个小小的蕃部,还要用计?大军压境,容不得他有二心。如果不肯降伏,随手杀就杀了,用计……太抬举他了!”韩冈摇头。

如果目标仅是青唐部,挑动内乱那没问题。但现在的目标是整个河湟地区的蕃部,要收服人心,就决不能用些阴谋诡计对付青唐部。要对付俞龙珂,只有两个策略,一个是赐予高官厚禄来千金市骨,一个则是连根拔起、彻底铲除,用雷霆手段来震慑四周蕃人。

从感情上说,韩冈其实对蕃人持有强硬态度的向宝比较认同。不过他拥有的理性告诉他,在汉人远少于蕃人的河湟地区,只能以招抚为主,否则就是把吐蕃诸部推往西夏一方——秦州汉人才是十多万丁口,而单是古渭州的蕃人就能与秦州相当,而古渭以西,蕃人数量更是古渭的数倍乃至十倍——但单独对上一个部族,却有杀鸡儆猴和曲意安抚两个选择。

在王韶与韩冈商议过的计划中,镇服古渭应是河湟拓边的预演。诸多的蕃族,混乱的内部,再有便是外部势力的插手,古渭面临的局势,与河湟地区一模一样。使得古渭寨相当于一个具体而微的河湟地区。

通过在古渭的试行,一系列纸面上的措施、策略可以得到现实的验证,有问题的地方能及时修改,而得到确认的手段便可在拓边河湟时加以推广。更重要的是,能够籍此锻炼出在拓边河湟的行动中,派得上用场的人才。

自太宗之后,大宋再无开疆拓土之举,反而连连失地。拓边河湟,在本朝并无前例可循。可以信用的部下,几乎都如韩冈一样,并无实绩可言;秦州的军队,守土有方,而进取不足。而王韶自己,其实也是纸上谈兵,从来没有真正处理过实际军务。如果能通过在古渭的预演,锤炼出一支精干的队伍,王韶当然求之不得。

征服河湟的计划,大体是上就是通过消灭木征,夺取河州,来慑服以董毡为首的吐蕃蕃部。收服古渭诸部也是大同小异,古渭寨已经立定根基,相当于夺取了河州,再拿两个不顺从的蕃部下刀,便可趁势威服青唐,利用他们去压制古渭的其他蕃部……

“就是纳芝临占部人丁太少,不然就能通过支援他们来压制古渭诸多蕃部了。”韩冈不无遗憾的说着,他并不喜欢青唐部,如果纳芝临占部与青唐部实力接近,他肯定会提议拉拢前者,而消灭后者。

王厚点着头,他与韩冈有着同样的看法:“毕竟是汉家苗裔,好歹也比青唐部的蕃人要亲近一点。”

河湟蕃部其实并不全都是血脉纯正的吐蕃人,有很大一部分是唐时陷蕃汉人的子孙。唐朝对吐蕃的战事,自高宗朝起,便多有一战覆没十余万的惨败。薛仁贵惨败大非川,李敬玄、刘审礼败于西海【青海湖】,一次十一万,一次十八万,都是如同字面意义上的全军覆没,兵败被俘的将士数以万计。

而自从安史之乱后,大唐势力中衰,吐蕃乘势扩张。安西、北庭两大都护府与中原的联络被切断,河西走廊上的诸多州县皆尽沦陷于吐蕃之手,吐蕃大军甚至能在长安城三进三出,被因此而掳走的,还有世代居住在河西州县里的,数十万计的汉人也多半成为吐蕃的奴隶。

普通的汉家百姓,被吐蕃人‘穴肩骨,贯以皮索’,成了逐水草、牧羊马的奴隶;而稍通文墨的士人,则在手臂处被刺上‘天子家臣’的字样,被吐蕃赞普录为家奴。

三百余年的时间里,华夏贵胄渐次沦为胡虏。如今吐蕃部族中有许多原本是汉家苗裔。尤其是河湟青唐,也就是王韶的目标地区,很大一部分都是原本的汉人世家转化而成的吐蕃部落。

纳芝临占部,又称张家族,族酋皆为张姓。秦州有安家族,大马家,小马家;古渭有张家族,丁家族,再远点的,还有邢家、周家、章家等部落。其起源都是一个个吐蕃化的汉人世家。

这些有着汉人血统的部落,其首领酋长‘例会汉言,多识文字’,而且由于势力不强,屡屡遭受正牌吐蕃蕃部欺压的缘故,往往亲附于宋室。在王韶的拓边计划中,他们都是能成为有用助力的部族。

衙门外突然一片喧闹,像是在吵架的样子,打断了韩冈的思路。李信过去一阵打听,回来后道:“是硕托部和隆博部的在外面闹起来了……”

“硕托部和隆博部?”王厚对蕃部的了解,让韩冈叹为观止,这些日子所看过的资料里都没提到名字的小部族,王厚竟然一口就能报得出:“那两家是世仇,部领已经近着渭源了。因为争夺草场和水源,断断续续打了有几十年,这两年刚刚消停了一点……”

“杀人了!杀人了!”外面突然乱声大噪,打断了王厚的介绍,上百个嗓门一起在高喊。

“什么?杀人了?”王舜臣一下兴奋起来,“那一定要去看看……”

王舜臣刚刚跑过去,一队卫兵也慌慌张张地赶了出去。一个小吏急匆匆地冲进官厅内,很快刘昌祚便板着脸大步走了出来。他步履如飞,几步走到门外。转眼之间,围墙的另一边,便是一片寂静。

王韶也慢慢的踱出来了,阴沉了好几天的脸色却有了多云转晴的迹象。两个小蕃部在古渭寨中闹出了人命,刘昌祚肯定要落个管束不当的罪名。而与蕃部有关的事务都是王韶的分内事,这一次正是他插手古渭的良机。

看着韩冈迎上来,王韶不禁欣慰的笑起。若不是这位年轻人的谋划,让他到古渭来过年,也把握不到这个幸运的机会——区区一条蕃人性命,多半就会被刘昌祚所掩盖。

等到硕托部和隆博部因此而重起纷争,连最基本的蕃人情报都无法掌握的蕃部提举,便会成为关西官场上的笑柄,也会承受天子和王安石的不满。李师中、向宝之辈当然更会趁机攻击于他,以便夺回对蕃部事务的管辖之权——如果让他们成功,渭源便会筑城,熙河照样开拓,只是这一切的功劳就不再姓王,而是李师中和向宝的了。

真得多谢韩冈,王韶心里想着,不枉他向朝中递上荐章。声音带着笑意:“两部争斗,殴伤人命,不是件小事。且去看看刘子京是怎么处置的……”

ps:都说是盛唐弱宋。但如唐朝这样把子民几万几万的丢给蛮夷的情况,至少在北宋基本上没有出现过几次。

今天第一更,求红票,收藏

