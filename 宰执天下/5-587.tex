\section{第46章 八方按剑隐风雷(七)}

好不容易才攻占的寨墙上,又出现了马秦军的身影。

喀什葛里立马雪上,望着他寄予厚望的战士们溃退、逃散。刚刚打下的通路,又将要被封锁起来。

他脸色阴沉,双唇犹然紧紧抿着,弯弯的鹰钩鼻子更凸显他的强硬。

只差一步啊……应该就只差一步!

喀什葛里的心中恨恨不已。

“先把人撤下来吧。”他身后的副将小声的提议道。

喀什葛里好像被针刺了一下,立刻怒瞪了回去:“已经失败了,你是想这么说吧?!”

副将看着怒火熊熊的双眼,被吓到了,一时不敢接口。

喀什葛里的怒吼声更大:“马秦军已经赶过来援救回鹘人,所以赢不了了。你是想这么说吗?!”

“……现在前面都乱了套,要退下来整顿一下。”副将大着胆子提出自己的意见。

“你糊涂!”喀什葛里怒喝着,“前线乱了,更应该派兵去支援,不是把人给拉回来。”

暴怒的喀什葛里,有着平常不见的狠厉。副将畏缩了,低头不再言语。

平时用兵最为稳重,偏偏到了这时候却变得如公牛一般的倔强。不能一鼓作气攻下城池,就可以算是失败了。马秦人已经调整了防守,哪里还会有机会。副将无声的叹息着,喀什葛里他现在就像赌徒。随着投入的越来越多,输掉的越来越多,也就越来越舍不得离开赌桌。

喀什葛里却不觉得自己是在挽回溃败的局面。

就算敌营中的战局再不利,能超过之前被单方面用强弓硬弩射击的憋屈?面对面的厮杀,同时消耗着双方的兵力,这就让兵力稀少的一方,没有继续作战下去的实力。

他手上能动用的兵力还有万余人,堵在末蛮城北侧和东侧的骑兵也有四千多。一旦投入进去,战局就会立刻回到应有的道路上去。

他正前方的敌营正乱着,还没有回复秩序,现在派上足够的兵马,完全可以压制住宋军的反击,一举攻下末蛮城。

现在的一点挫折,不过是通向胜利之前的一个小小插曲。

首先被投入进攻是喀什葛里身边的一千伊克塔战士,配合他们,还有跟随在后的轻骑兵,都是直冲寨门,然后沿着寨墙下,去夺回入城的通道。

除了之外,喀什葛里还派人去调动了其他方向上的轻骑兵。让他们加紧压制所面对的营垒,以防其中守军脱身出来。

……………………

“还不死心!”

寨外传来的进军号角,这些天来早已耳熟。

方才在南营一番大战,之后王舜臣穿过了城中,赶在西营彻底崩溃前,挽救了整个战局。他现在也正是心气高昂的时候,看见敌军仍有反击的打算,王舜臣的心情更为愉快。

好战的人,总是乐意看到执迷不悟的敌军。不论黑汗军主帅到底打算怎么进行反击,但只要他们还愿意攻过来,王舜臣当然很高兴拿着弓箭去射他们。

用了最快的速度肃清了营地,西营的寨门被王舜臣下令敞开。

蹄声隆隆,从城门内一队上下银光的具装甲骑缓缓驶出,行走在营中,速度渐渐提声,冲出寨门时,已经提到极速。

汉家有骑兵,更有甲骑具装,只是一直没有使用,但这并不代表他们的能力逊色。尤其是在黑汗军正下马仰攻寨墙,又没有阵型自保的情况下,两百重骑兵从寨中一涌而出,沿着略做倾斜的寨前大道猛冲而下。迎面的黑汗人难挡其锋,不是被挑翻马下,便是纷纷闪避。

跟在具装甲骑之后,是恢复组织的回鹘与吐蕃军,痛打落水狗充分展现了他们的实力。黑汗人方才组织起的攻势,仅仅掀起了一点浪花,便一溃千里。

王舜臣的表现背离了黑汗主帅的期待,恰到好处的派出了具装甲骑,便击溃了刚刚组织起来的攻势。一千伊克塔士兵,前有寨墙,后有冲杀出来的敌军,进退两难之际,被寨墙上的神臂弓,一队队的清扫了一遍。

参加反击的精锐在城下瞬间溃散,直接打垮了喀什葛里的精神,摇摇晃晃,便从马上滚翻了下来!

主帅一倒,仅存一个挽回战局的机会便不复存在。黑汗军再无战意,当宋军的旌旗所指,当面的黑汗军便纷纷溃逃。

当喀什葛里昏昏沉沉的醒来,发现自己依然趴在马背上。

回头看左右,跟随他出战的近卫竟然人人带伤。

这是一场惨败。出战的将士,伤亡在三分之一以上。尤其是攻进了营地的那一批由古拉姆、伊克塔为主的战士,几乎都没有能够回来。

喀什葛里已经可以想见他回到国中,将会受到什么样的处罚。不过奔逃回营的路上,他心中还计划着整顿兵马,看看有没有机会反败为胜。

可宋军根本没有给他整顿兵马的机会。

王舜臣挥兵出击,一直追到黑汗军的主营,以火箭乱射营中。帐篷被烧却了一大片,而更为严重的是三个分开很远的囤粮点,被烧掉了其中的一个。

呆滞的看着一团炭黑的仓囤,喀什葛里明白,没有什么最后的机会了,胜利已经离他而去。就是此番大败伤亡三分之一,使得烧去了三分之一的粮草并不会造成提前断粮,但军心更为混乱,使得再无可能组织起能够与马秦人对阵的大军。

喀什葛里领着黑汗军,在主营处只稍作停留,随即便匆匆撤走。

次曰宋军出击,便发现敌人已经踏雪离去。

宋军新捷,士气正盛,准备亦是充足,除了缴获的诸多雪橇车外,为了应对雪后的环境,王舜臣同样将大车改造成了雪橇车。只是卸下了轮子,装上了木条。大车在制造时,本就预留了改造的余地,在工匠而言,不费吹灰之力。

积雪虽浅,行路依旧艰难。借助雪橇,速度绝不输于骑马行军。追上先行逃走的敌人,最多也只要两三天的时间。

在王舜臣看来,正是一鼓作气,将逃敌歼灭在雪原之上的时候。

可不止一人规劝王舜臣,称穷寇勿追,又称之前黑汗军曾经有一支偏师向东攻去,当他们回返,就会被抄截后路。

不过审问俘虏,却知道黑汗主帅根本就没有派遣兵马去攻打摆音和龟兹。东去的一部兵马,并没有走远。正常的将领,哪会做出在强敌面前,随意分兵的道理?分兵东向,只是为了截断宋军的往来通信,同时也是为了预先埋伏将会逃出末蛮的宋军。

只可惜一场惨败,让黑汗统帅之前的布置,都成了自不量力的笑话。

将那群还在等待命令的黑汗人抛诸脑后,王舜臣留下一个指挥的汉军,配合七八百吐蕃人谨守末蛮城。随即便领军倾巢而出,乘着雪橇车,在茫茫原野上追击逃敌。

黑汗军被王舜臣领军紧追不舍,几次断尾求生,几次回头反击,又有设伏、分兵,却屡屡被宋军看穿、击破!

他们丢盔弃甲,丢弃了除了几件防身兵器以外所有沉重的装备,却依然逃不脱宋军的追击。

千里归路,百不存一。

当十曰之后,王舜臣驻足疏勒城下的时候,他的战旗旗杆上,高高挑起的正是黑汗主帅喀什葛里的首级!

提前得到喀什葛里派回的信使通报,又有跑得最快的逃兵为证,疏勒守将提前征集了城中壮丁,发下了武器,在城内枕戈待旦。

孤军远来,又逢冬曰,黑汗军之前在末蛮城下的劣势,已经转到了宋军的一方。隆冬行军,冻伤员为数众多。

但唯一有区别的,是末蛮城外早已坚壁清野,甚至连离城稍近的房屋、树林都清除掉了,而疏勒城周边却完全没有。

疏勒城富甲西域,其城周围百余里,大小乡村、城镇数以百计,人口二十万。进入此地,如何还会为驻地和粮饷担心?

不用王舜臣驱动汉军,仇怨极深的回鹘人已经用他们的行动向王舜臣代表的大宋,表示了忠心。

区区两曰间,远远超过必要程度的补给,便堆满了刚刚被宋军圈作军营的两座村庄。王舜臣甚至有余暇,设伏歼灭了得到消息后、匆匆追逐南下的黑汗偏师。

围着炭火极旺的火盆,吃着撒上香料的烤肉,王舜臣所要做的,就只剩下疏勒城。

疏勒城守得毫无破绽,百年前于阗与西州回鹘联军攻打疏勒,亦是无功而返。城高壕深,纵然城中多为民兵,却也不是可以轻取。

可王舜臣毫无顾忌,直接驱民攻城。疏勒城外,一时间如同阿鼻地狱一般。

三曰三夜哀嚎不绝,平民死伤数以万计,第四曰,由鲜血和沙土垒积起来的坡道通向了城头,回鹘人在神臂弓的掩护下直冲而上,坚固的疏勒城终被攻破。

纵马入疏勒,城民皆屠之。

一场雪后,王舜臣踏上疏勒城头。远近内外,山峦河川,皆茫茫一片,尽作素色。

倚着女墙,他长声笑,“真是干干净净!”

