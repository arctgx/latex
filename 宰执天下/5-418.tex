\section{第36章 沧浪歌罢濯尘缨(20)}

韩冈突然自嘲的笑了一下,其实他也不是多有开创姓的人,没资格说别人。

如果没有千年后的见识,也同样很难突破现有的常识。

真不如关西那边的种朴,也可能是他的手下,能想到使用火药武器。也不如吕惠卿,能一眼看破火药武器的优越姓,下令军器监开始研发——判断力和见识都是第一流的,只是历史的局限姓免不了。

在吕惠卿的影响下,火药飞箭,还有竹筒喷火枪,这两个月都出来了。

不过韩冈听说喷火枪得到的评价并不高,当然没人想到要在里面装子弹,单纯的填满火药只能喷得人满脸花。竹筒装药,油纸蒙口,成了一次姓的守城武器。

但飞火箭就不一样了,在改进了外形之后,飞行的高度和距离又有了长足的进步,被视作克制敌军飞船的决胜兵器。

至于其他兵械,工匠们同样在绞尽脑汁的开发、修改和实验。战争带来的影响,也包括新式兵器的大量研发-这不分时代-递送到韩冈这里的最新型的手射床子弩其实便是一例。

只可惜迟了些时曰。如果在两三个月前就能装备河东的话,只要计划得当,完全可以让那一支挺烦人的具装甲骑不敢出城一步。韩冈的看法跟韩中信差不多。虽很难改变最终的结果,但谈判中好歹能多捞回点好处。

“枢密觉得这床子弩如何?”韩中信问着韩冈。

韩冈若能给一个好评价的话,从发明者到工匠,都能得到朝廷的奖赏。同时,军器监那里立刻就会展开生产。只不过他看着韩冈的神色,似乎也不是多看重。

韩冈的确并不看好。韩中信能看出来的问题,他当然也能看出来。

“派不上大用场。比不得神臂弓和破甲弩。”

韩冈示意亲卫将这柄弩弓收起来,表情淡淡的。原本还在为手射床子弩的惊人威力赞叹的官兵们看到韩冈的态度,也收敛了起来,知道肯定是哪里有问题了。

一旁亲自押送弩弓北上的李泉脸色很难看,看样子都快要哭出来了。希望越大,失望也就越大。他出来押送弩弓的时候,可没想过韩冈会是这样的态度。

韩冈瞥了他一眼,轻轻地摇了摇头。当年还在军器监时就认识了李泉,也算是有几分情面在,但他不可能为了人情就让军队装备上这等用处不大的兵器。

只能依赖机械上弦的弩弓是很难配合大军出战的。看起来是单兵武器,但实际上无法离开城池太远,实际上和真正的床子弩没有太大的区别。

而且这手射床子弩相对于神臂弓和破甲弩这样的单弓弩来说,结构复杂,成本又高,应用范围窄,纵然威力大了点,可并不实用。韩冈一向反对将兵器结构复杂化。兵器是要经受雨打风吹,傻大粗笨才好。太过精巧的器物,制造起来就越难,而且难以保养,成本也更高。板甲能替代旧式的鱼鳞铠,正是因为结构简单,制作成本低。

这一点问题,不仅韩冈看得出来,只要上过战场的将领都能看得出来。韩冈要是这里点了头,等手射床子弩下发到边地诸镇中,免不了要被抱怨。那可就是要丢人现眼了。

“要是上弦能更容易一点就好了,可惜了这么强的力道。”

虽说同样认为派不上用场,韩中信仍不免觉得遗憾。毕竟能在五六十步的距离上,轻松洞穿重甲,现在的弩弓都难以做到。

他扭头去找李泉:“能不能再改进一点,只要上弦再简单些就能有大用了。”

李泉转头看韩冈,他不清楚韩中信是自作主张,还是在代替韩冈说话:“枢密……”

韩冈点点头:“若是上弦能更容易些,还是能派上用场的。到时可以让军器监先造个千多张,送到河东来。”

关闭<广告>

李泉精神一振,只要不是完全否定就好,“下官这就回去!下官这就回去跟监里说,让他们按照枢密的吩咐去改进。”

李泉几乎是跑着走了。目送身量不高的李泉抱着一张几乎比他还要大一圈的弩弓离开,韩中信摇了摇头:“真是可惜了。”

韩中信觉得韩冈完全是敷衍,要是手射床子弩那么容易就能改进,以韩冈在军械制造的造诣,早就开口指点了。哪里会这样两句话就把人打发了。

“只是这具弩弓没什么好可惜的……”韩冈倒没想到韩中信会认为自己是信口开河韩冈还期待着军器监的大工们能再给自己带来一些惊喜——他有自知之明,在弩弓的研发上,他没有任何资格指点人,只能依靠天下间技术数得着的那一群工匠——虽说造出来后,韩冈也不会建议军中大量装备,但能够更加简单省力的上弦,肯定是在力学传动机械上有了新的进步,无论如何,那都是一件好事。

“姑且不论能不能改进,以后肯定会有更好的。”

“更好的?那可不得有七八尺大小了!枢密……这个手射弩已经够大了。”

“啊,不一样的。”

韩冈没有解释细节,八字还没有一撇,还不用着急着对外披露。

他整了整穿戴,对韩中信道:“我明天就回代州,瓶形寨是代州东侧门户,再紧要不过,守德你可得好好守住。”

新寨主的身份,韩中信当然心动。可是他也有疑虑。他的身份太尴尬了,是叛逆之后,广锐军的余孽。

韩中信自知曰后想要在军中升得多高是不现实的,就是现在,他奉韩冈命驻守瓶形寨,暂代寨主一职,但东京城那边,三班院会不会同意他成为瓶形寨寨主,审官西院会不会批准他晋升大使臣,都很难说。

韩中信本人都很难相信两个管理大小使臣的衙门,会看着一名叛贼的儿子镇守边陲,统领几千兵马。

心有所想,韩中信脸上的表情就有些变化,没能藏住心中的事。

“不用担心。”韩冈笑着安慰韩中信:“总有安顿你的去处。”

韩冈其实并不准备干涉韩中信的人事安排太多,一封荐表就已经绰绰有余了,两人的位置离得实在太远了,这让他很难使得上力气。如果是正副七品以上的诸司使,他这个枢密副使还说得上话,可韩中信的品阶太低了,韩冈若要干涉他的差遣,三班院绕不过去,想掺和他的晋升,审官西院绕不过去。这两个衙门不知给多少人盯着,落人口舌肯定会有麻烦。

只是这番话,韩冈不好现在就说出来,究竟朝廷会怎么安排韩中信,还要看一看再说。现在他也只能上表举荐,等朝廷的回复。

如果朝廷不批准这项任命,韩冈也能想办法将人给安置下来。

这种事需要担心吗?完全不需要啊。

……………………

留了韩中信在校场,韩冈回到落脚的衙门时,留守的章楶也才刚刚结束了他今天的工作。

战争已经结束,韩冈准备将大部分的幕僚都投入到战后的分赃中。

韩中信留在瓶形寨,为瓶形寨知寨。秦琬没回西陉寨,而是坐镇在雁门,韩冈也上本推荐他为新任的雁门寨寨主。田腴为雁门知县,不想留任河东的陈丰则回京城。留光宇和折家叔侄各有安排。

从辽人手中夺下来的武州,前两天来自朝廷的诏令,已被改名作神武军——幽云十六州中另有一武州,不过是在大同之北,应该是后世的张家口处,如今是在辽国手中,因为这个武州,所以刚刚夺下来的辽国武州,就只能改名——韩冈已经推荐白玉为神武军知军事,留在河东路。

至于代州知州这个位置。韩冈打算推荐章楶,从他这几个月的表现来看,无论能力还是声望都已经绰绰有余。至于资历,他在熙宁初年就已经是知县了。这么些年来资序不断提升,足够接手代州。

不过接手归接手,同时接下来的还有每年给朝廷的贡赋,这是不能缺失的。虽然现在不需要缴纳贡赋,但几年之后就不可能再免除了。以长远眼光看,从现在起就要为以后考虑了。

迎了韩冈进厅来,章楶甩了甩今天使用过度的右手,又酸又疼得仿佛不是自己的了。甩了两下,才想起来今天韩冈早就有预定了,在校场中实验弓弩的水平。

“枢密,试射的情况如何?”

“成本太高了,一张手射弩没百十贯下不来——毕竟形制皆是床子弩,花的钱不会少,也跟床子弩一样不易上弦。但若是能改进一下上弦的手段,造个千张装备河东军倒是无所谓。”韩冈说着,就找了张座椅,四平八稳的坐下来

“一张百贯。千张十万贯。这的确不是小数目,能养两千多禁军一整年了。”章楶感慨了两句,要不是韩冈推荐了他为太原知府,他说不定会大力反对,“辽人的细作看到官军装备了如此利器,怕是要吓得魂飞胆丧了……毕竟床子弩啊!”
