\section{第15章 自是功成藏剑履(六)}

河东最新的捷报已经在京城中传扬开来,前日还气势汹汹的御史台顿时失声,一下变得安静了许多。

当然,要弹劾人总能找到理由。但那样子就成了泼妇骂街式的胡搅蛮缠,纵使大部分监察御史能拉得下脸来,也要天子和朝堂愿意陪着他们丢这份脸。

国丧之期,太过惹眼的七十二家正店那样的大酒楼里不兴曲乐,人数寥寥。但小一点的茶肆、酒馆,依然高朋满座。议论的话题,当然离不开河东的胜局,以及御史台和河东经略使的交锋来。

韩冈在民间名声极好,杀的又是一直与大宋为敌的党项人,御史台将目标选定在他身上,不仅百姓,连士林中的清议也站在了他们的对立面。这一回看到监察御史们丢人现眼,到处都能听见幸灾乐祸的笑声。尤其是南薰门国子监附近的诸多酒馆,

“也不想想,堂堂龙图阁学士怎么会糊涂到这个地步?御史台太小瞧人了,这下子可不知回去要吞多少消风散才能缓得过气来。乌台边的唐家熟药铺生意又要好了。”

坐在一张漆料斑驳的方桌边,一名三十四五的中年士子豪迈的放声大笑。与他同桌而坐的两名士子则同样举杯而笑。正如韩冈为御史们所嫉,国子监的太学生们也同样对一干监察御史好感缺缺,有机会绝不会少笑两句。

中年士子放下酒杯,感叹道:“黑山党项南下,自然是萧十三的奸计。辽军混迹其中,若不是黑山党项为其掩饰,如何能做到?一旦数万黑山党项与契丹人里应外合,胜州还能保全吗?到时候,河东半壁亦是难保。幸而韩龙图早有所备,才能让辽人自取其败。”

“季明所言正是。诚可谓世有贤人,国之大幸。我钟世美虽也研习兵法,亦晓韬略,却自知难望其万一。”

钟世美坐在表字季明的中年士子对面,啜着杯中酒感慨不已。

“正甫兄过谦了,你前日一篇经制四夷的文章,几位学录可是赞不绝口。”三人中,最为年轻、相貌却最丑的一人操着两浙的口音说道。

钟世美摇着头:“哪里能比得过你周美成的文章。”

周美成尚要自谦,中年士子就跟着道:“美成你的诗赋,在国子监三舍两千四百人里,都是数一数二的。正甫兄还能凭着策论一较高下,我潘必正可只有俯首称臣的份了。”

“季明兄你是气学门人,在自然大道,我等可是远有不及。”周美成转着圈又恭维回去。

“只是去听讲而已,当年横渠先生讲学京中,虽说日日去聆听教诲,却未能有幸得入气学门墙。”潘必正很是惋惜的叹了一口气,他虽不能算是气学弟子,但对于韩冈提倡的格物之说,认同感颇高,平日里也多有研究,还拥有一架显微镜。

“季明兄,你既然有心在气学中一展长才,何不投入韩龙图的幕下?”钟世美问着,“令先尊在湖南、广西皆有遗爱,与章副枢交谊匪浅。得他一封手书,至韩龙图幕中任职岂是难事?你本有官身,也不会与韩龙图门客抢荐书。”

潘必正是开国名将郑王潘美的玄侄孙。不过关系隔得有些远了,君子之泽五世而斩,郑武惠王的遗泽轮不到他头上。没中进士就有个官身,还是靠了他的父亲。其父潘夙,曾经任职荆湖南路转运使、潭州知州,参与了章惇平定荆南之役。后来因其在桂州任上首倡交趾可取,在章惇、韩冈两人主持的平南之役结束后,又以此事而被追功封赏,潘必正由此荫补得官。在三班院中,他只是个挂名候阙的小官,在国子监中,也只是个普通的上舍生。不过因为潘夙与章惇的交情,潘必正想拜见章惇,的确不需要太费周折。

但潘必正摇摇头:“还是在监中得个出身方是正途。韩龙图若不是得了一个进士出身,如今怎么也惹不来御史台群起而攻。而且小弟有意研习格物之说,在京城里面还方便点。”

韩冈宣扬的格物之说,能将身边的事物剥丝抽茧的进行分析。理在万物之中,格之乃得。

眼下无论是韩冈的《桂窗丛谈》,还是苏颂的《思闻录》,又或是沈括最近新出版的《笔谈》,对自然万物的分析和描述,吸引了越来越多的士子。

好奇心人皆有之。无论如何,枯燥的经学理论论起吸引人的程度,当然远远比不上对天文地理自然万物的研究。拥有显微镜和千里镜的士大夫,他们用心在两件工具上的时间,也比研读经书要多得多。

将自然之道和儒家典籍捆绑起来的气学虽然没有新学独占官学的力量,也不如程学那般得到元老贵胄的支持,但出于自身的喜好而愿意去研习的士人数量,却远远超过其他任何一个学派。

一番推杯换盏之后,周美成忽然又道:“不过这件事全凭韩龙图的一张嘴,真伪如何能知?”

“周美成你说什么胡话。才一千多,还不知道有没有加上混入党项人中的细作。何须作假?”潘必正摇头道,“当年河东军不是已经阵斩五百辽人,那时候与其对垒的官军也不过是千多人。如今的又是用计,又是设伏,也才留下一千人,当真是少了。”

“说得也是。”周美成愧笑点头,“斩首要是能有个三五千就好了。”

宋夏之役,看辽人只敢在背后占便宜,却不敢与官军对阵,越来越多的宋人都认为官军拥有击败辽人的实力,只要能换上个靠谱点的主帅。在林林总总加起来超过十万的斩首面前,区区千余辽军,实在是微不足道。

钟世美沉声:“萧十三远不及韩龙图,被玩弄在股掌之上。但那个领军的辽将,当不是个简单人物。中了韩龙图的陷阱,还能断尾而退,非是等闲可比。日后与辽军交手,此人可是当小心提防才是。”

“那件事得多少年后了。”潘必正提起酒壶倒了一圈酒,“眼下也不知道天子派出去的使者,追回那份密诏没有。”

……………………

当天子的密诏抵达河东经略司治所的时候,韩冈也同样回到了太原城中。

就在州衙的内院里,韩冈焚香供案的接了赵顼密旨。送走了使臣,听到风声,从后院走上来的妻妾,四人都是面如寒霜,心头生怒,但更多的还是掩饰不住的担忧。皇帝想要为难臣子,做臣子连喊冤的地方都没有。

“三哥哥,要不要紧!?”韩云娘扯了扯韩冈的衣袖,就像过去一般。

“要紧什么,不就是要为夫低头认错吗?”韩冈微微一笑,“先别在外院站了,回屋再说。”

韩冈也有些纳闷,从时间上看,最后一份捷报,与之前的几份相隔得并不远,朝廷怎么会连等两天的时间都没有?照常理该是派人先查验,怎么跳过了这个关键性的步骤,变成了急匆匆的斥责。

心中的疑惑不自觉的说出口,王旖抬头看着韩冈,眼神中有着些许感慨:“官人一向不妄言,说是两万,也没人会怀疑官人谎报。”

就是这个原因?韩冈微微苦笑,这是太诚实的结果吗?也许吧。多半就是这个缘故,使得赵顼没有在查验战绩上耽搁时间,早早的就派人来打掉自己晋身两府的奢望。

“这又是何必?此番来河东,是为国宁边,本也没想过立功受赏。何至于如此?”韩冈叹息着。

赵顼的用意已是昭然若揭。但不过是执政而已,一顶青凉伞不出意外迟早能到手,他又岂会急在一时?

“官家把官人当成刘子仪【刘筠】了,以为官人虚火上攻,一定要清凉散才能病好。”周南冷笑着。自家的夫婿无罪被责,性格刚烈的周南哪里能忍得住不去嘲讽上两句。

刘筠是仁宗时的重臣,三入学士院而不得晋身两府,写诗抱怨道‘蟠桃三窃成何味?上尽鳌头势转孤。’最后干脆称病,不肯出来做事。自然,他这么做便少不了成为世人的笑柄。被石中立嘲笑是虚火上攻,一服清凉散便好。这‘清凉散’当然说得是非宰执不可得的青凉伞。以爱开玩笑而著称的石中立说话可谓是刻毒。

“执政虽好,我也不愿巴着求着。”韩冈摇头道:“若能将先生的神主迎入文庙陪祀,就是宰相之职,为夫却也可弃了不做的。”

韩冈的宏愿并非区区官场可以束缚,高官显宦不过是达成目标的阶梯,却绝不是他的目的。赵顼或是御史台那一干人等,未免太小瞧人了。不过从韩冈的心愿上来的看,赵顼现在做的也不能算错。

“官家的密诏,官人打算怎么办?”严素心问着。

黑山党项乃是辽人的内应,最新的捷报应当没有耽搁的就传到了朝廷那里。可这份责难的密旨一路上竟没有被追回。究竟是没有来得及,还是咬定牙关要给自己一个脸色?韩冈的心中还是怀着疑问。不过如何应对倒不需要犹豫。

“当然是上表谢罪。”韩冈笑得风清云淡,“雷霆雨露,皆是天恩嘛。”

