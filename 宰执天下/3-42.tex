\section{第15章 一笔定黜陟(下)}

韩冈醒来的时候已经是午后。

淅淅沥沥的雨声敲打在庭院中的青石板上,撑开窗户,潮湿而微寒的风立刻吹入了房内。清新的空气,让韩冈精神为之一振。

梳洗过后,韩冈顺着廊道往前厅走去,却正碰上今日休沐在家的王厚在观雨。

雨水从檐上哗哗的淌下来,一幕水帘挂在面前,王厚怔怔的看着。

韩冈走了过去:“终于下雨了。”

王厚扭过头来,“这个冬天,京东、河北雨雪都少,朝廷里面不少人都在担心呢。”

韩冈眉头一皱,回想起来,情况的确正如王厚所言。

他自上京之后,心神一直都放在考试上,根本都没注意多少天没有雨雪了。不,他是注意到了,还为两个月以来的好天气庆幸不已,完全忘了农事。

“幸好下了雨,开春下一场透雨,好歹能缓解一下几路的旱情。”

王厚抬头看着天上的雨云,似乎渐有散开的迹象:“若能再稍微下多一点就好。”

“是啊,最好再下多一点。”韩冈道,“今冬河北、京东无雪,春后田里的虫子恐怕要多起来了。”

“这也不是我们的能管得了的。”

王厚自嘲地笑了一笑,一个三班主簿,能管得什么事?就是他做枢密副使的老子,也不便在中书的管辖范围上指手画脚。倒是韩冈,可以在王安石面前提上一句。

与韩冈一起向前厅走,王厚笑着说道:“看玉昆高枕无忧的样子,当是高中无疑!”

韩冈摇摇头,“那要到发榜后才能知道。”

韩冈对自己的学问还是有着自信,但他更清楚,并不是所有有才学的士子,都能考中进士,运气也占着很大一部分因素。

韩冈向来不愿去赌运气,将自己的命运放在天数上,根本不合他的性格。在不触犯规则情况下,尽量让自己拥有一个更为稳妥的前途,就一直是他重点考虑的关键。

所以他才会在昨天的考试上,故意放弃了三条正确的答案,又一直拖到了最后才交卷,就是不想去依靠运气来决定自己的命运。尽管如此的确是形如作弊,但韩冈可不在乎这一点小事。论能力论功绩,韩冈比谁都有资格,即便论才学,他也不认为自己够不上进士的标准。

其实韩冈并不能确定自己其他二十七条一条不错,但从王雱那里了解了审题规则的他更为清楚,二十七条中格并不是死规定,可以允许例外。既然如此,只要能够表明自己的身份,这个例外他一样有机会拿到手。

只要身份表明,他就有很高的机率将自己的卷子呈到主考面前。而就算能全数答对三十道经义,史论上还有被点检、考试、覆考三方一齐黜落的可能性。

两边的成功率都不是百分之百,但从几率上来讲,当然还是前者更大一点。

为了能让自己卷子一路过关斩将,韩冈耽思竭率,用尽手段,而他的选择也无可厚非。同样的,他在史论上也下足了功夫,相信足以通过四名主考的评判。

当然,机关算尽太聪明的可能性也是有的,不过……

“这是个机率问题。”

陪着王厚走在雨声不断的长廊上,韩冈低声的自言自语。

……………………

其实试卷批改得也快。

叶祖洽、上官均、陆佃这十几位点检试卷,用了三天的时间,去批改总计五千余份的考卷。他们以批改墨义帖经为主,兼及策论。因为是检查有着正确答案的墨义,批改起来只耗眼力,却不用费神思量,基本上一个时辰,就能过去五六十份,平均一人三百多,不到四百多试卷,两天就批改完毕。多花的一天,是将批改过的试卷互相交换,检查其他人批改得是否有错误。

仅是通过墨义帖经这一项,就一下刷去四千多人。除了一些策论文章确实好到让人难以释手的卷子,没有达到二十七条中格这道红线的贡生,便全数被黜落了——虽然之后还有一次复核,但能起死回生的卷子,几乎不会有。

最后送到考试和覆考那里的卷子,就只剩一千余份。考试官六人,覆考官四人,这两道关口,主要是评判史论一部。加上点检试卷,三方的评分如果相同,便没有什么可以说的,若是不同,则呈交主考。这一项评判,就比较耗费精神,前后一共用了六天才宣告结束。

就在以明法科为主的诸科考试,全部结束,特奏名进士考试开始的时候。覆考官也终于完成了他们的工作,将最后剩下的近五百份卷子送到了曾布、吕惠卿等人的手上。

其中有两百余份没有争议,连过三关被确定可以中格的卷子;另外还有两百多份点检、考试、覆考三道评判之间不相合的试卷,需要四位知贡举来敲定。

四个主考要最后敲定四百名【注1】进士,耗费的时间更甚点检、考试和覆考。曾布、吕惠卿、邓绾、邓润甫四人各自默不作声的翻阅着考卷,厅中一时见只能听到沙沙的纸张翻动声。也只有看到纰漏过甚的卷子,拿出来当个笑料;或是有什么出色的词句,念起来交流一番。

时已近晚,确定了取中的试卷已经有了大半。就要到吃饭的时候,邓绾突然呵呵的笑了起来。

吕惠卿听见他笑得奇怪,搁下笔,扭头过去问道:“怎么,又看到什么有趣的卷子了?”

邓绾拍了拍卷子:“有趣倒说不上,但写的是不错。只是这份卷子多质而少文,不是河东举子,便是解自陕西。”

邓润甫也从阅卷的工作中抬起头来,反问道:“难道湖广利夔的文采就好了?”

“满篇说了这么多西事,也只有陕西的贡生才能写得出……”邓绾的笑容意味深长,转手递给了邓润甫。

邓润甫不以为然的接过试卷,看了一阵,笑容突然也变得跟邓绾一模一样:“变法拨冗,王业兴至百年;因循苟且,帝统止于二世。以兼并六国之法而治六国,何以不亡。此一句别出机杼,道前人所未道,难得,难得!”

吕惠卿惊讶的看着邓润甫。这两句说着变法的好处,的确让人满意,但邓润甫的评价未免高过了头。

“岂不见《过秦论》中‘仁义不施,而攻守之势易也’?此篇当是化用其义,岂可谓之道前人所未道?‘并兼者高诈力,安定者贵顺权,取与守不同术也。’天下一统,自当改弦更张。始皇禁文书而酷刑法,先诈力而后仁义,以暴虐为天下始,故而生死国灭,卒为天下所笑。这道理,贾长沙【贾谊】早就写明白了!”

“‘秦任商鞅,二世而亡’,谢公可没觉得贾谊说的有理。”曾布一边批改着试卷,一边却不忘跟吕惠卿唱着对台戏,“这一句中的见识不算差了,比谢安要强!”

吕惠卿摇摇头,正准备反驳,邓润甫却已经将卷子递了过来。吕惠卿拿过来展开细看,很快,他的唇角抽了一下,似是在冷笑。然后真诚的笑意浮了上来:“这一篇文章别的倒不论,唯独一个‘势’字说得甚好。汉高顺势而为,约法三章代暴秦之苛刑,遂得关中人心;王莽逆势而行,遽行古制乱天下之正道,故而身死国灭。皆是变法,顺势而为当是正理。”

“汉高、王莽,这还真敢写!”曾布随手在面前的卷子上点了一点,摇头道,“若是取中,恐怕贴出去后,西京就会有人问了:如今天下汹汹,皆为变法,按这卷子中的说法,是顺势还是逆势?”

“李昉不喜谈利害,秉政不改一事,只因其时立国未久,制度初定,不可妄为。可当今天子登基时的时势,丞相的百年无事扎子已经说得够多了,大势需变法,岂是群小所能移?只为西北之事,变法便是必然。兵事无粮饷不行,青苗、市易不皆是为国用而理财乎?河湟功成,亦是变法之力也。中国苦西北二虏苦久矣,富国强兵自是顺势!”

曾布不跟吕惠卿争了,低头看着自己眼前的卷子:“道理说得过去,只不过文字尚待琢磨,不甚佳。”

邓润甫立刻回道:“文字的确是不甚佳,但倒也够格取中了。”

邓绾也附和着:“只凭卷中一番道理已然可取,只是难置高等尔。不当以文字取士,否则何须弃诗赋而用经义?”

“一二等既不可入,权放在第三等。”吕惠卿手脚麻利,在卷首上用朱笔描了个圈子。

曾布盯着眼前的试卷,慢悠悠的点了点头。三名副手既然有着同样的意见,他也便没有反对的意思——那几句听着并不差——何况他也反对不来。只是当曾布又批了两张卷子,脑中忽然灵光闪过,啪的一声重重放下了笔,厉声问道:“这是谁人的手笔?!”

吕惠卿慢慢悠悠:“拆了糊名纸就知道了。”

注1:这两天去查资料,发现熙宁六年礼部试的录取人数是四百零八人,而不是前面写的三百人,从本章开始更正。

