\section{第七章 苍原军锋薄战垒(四)}

天是灰黄色的,狂风如同巨浪,一波波的扑向在风沙中缓缓而行的一队骑兵。

狂风从背后卷来,殷红的盔缨在风中飞扬,精铁头盔被沙石砸得沙沙作响,泛着金光的背甲也是噼噼啪啪的响着。十几步外的景物,在沙尘中都模糊起来。

几近千人的骑兵们低着头,分成三列在灵州城外的原野上沉默的走着。外围更远一点,还有几十名游骑,分散点缀在原野和沙尘中。

他们人人着甲,相比起步兵具装时裙甲、肩甲一应俱全,骑兵们的甲胄,仅仅是前后两幅铁板,只护着胸背。

但铁甲就是铁甲。只装备了胸甲的骑兵,依然可以归入具装甲骑的行列。

换在十年前,身着铁甲的骑兵全都是精锐中的精锐,任何时候都是护卫在主帅身旁,不到最后一刻不会拿出来。而如今则是探马、巡卒的标准装备而已。

姚麟双眼眯成一条缝,迎着风沙,扫视过他身后绵延逶迤的队列,

见队列依然严整,并无散乱,他便立刻转回头来,吐掉了唇中的沙砾,揉了揉鼻头,又皱着鼻子哼了两下,把钻进鼻孔的沙土全都挤了出来。

身为领军的大将,巡逻的差事本轮不到他。不过这是姚麟自愿,加上也有与党项人放出来的一支支铁鹞子一较高下的打算,才会在得到高遵裕的首肯下,带了两个指挥的马军出来。

跨下的瘦马保持着稳定的节奏,一步步的踏着沙土向前。但姚麟探手在坐骑的肩胛上抹了一把,上面已经是满是汗水,连黑色的皮毛也变成了灰色。

“先歇一歇脚!”

风此时似乎小了一点,姚麟便抬起手,将就地休息的消息传了出去。

亲兵们在队列前后一阵奔忙,近一里长的队伍缓缓停下了前进的脚步,只有外围的游骑依然活跃在风中。

下了马,就在路边上,姚麟找了个树桩坐了下来,依然是背着风。

主将歇下来了,但军官们可歇不了。抬着脚将躺了满地的士兵一个个踢起来,让他们带着坐骑、战马到路旁的河滩边饮水,把随身已经喝空的水袋就着干净的流水灌满。

这一次的巡逻,姚麟带出来的两个指挥,是沿着灵州川,巡视粮道安全。一天下来,来回已经有八十多里了。

从亲兵手上接过羊皮水袋,姚麟仰起脖子喝了半袋子。里面不是水,而是解渴的淡酒,比起河水,姚麟更习惯喝这个。亲兵从姚麟手上接回水袋,回头又跟两个党项人龇牙咧嘴的头颅挂在马鞍后。

姚麟看了一眼已经干瘪下去的两颗头颅,没什么兴趣的挪开眼睛。今天杀败了几支党项骑兵小队,斩首只有八个。

拍了拍身子,从衣缝中拍了一堆沙子出来。要不是因为抵达灵州城下的这两天,飞船因为狂风无法使用,也不至于让骑兵在营外来回奔波。

一艘位于三十丈高处的飞船,在白天的时候,能让大队的敌军无法潜入三十里之内。而到了夜里,也能借助星月的光芒,看到潜伏到近前的敌军,配合探马、暗哨,能让大军不受敌军偷袭之苦。

但飞船畏风,风稍大一点,就没法儿上天了。灵州城内也有飞船——契丹人能偷学去,西夏也一样能偷学——上午离营的时候,已经被狂风吹得斜了过来。现在风势更大,不是给吹跑了,就是已经收了起来。

现在各路探马散出去有五十里,中军的安全得到了基本保证,但党项骑兵的战马更多,可以轻易的跑出一两百里骚扰粮道。每日里官军和西贼的骑兵厮杀不断,斩首虽然不少,但伤亡也一样不是个小数目。

如果官军能开始全力攻城,想必西贼就没办法这么嚣张几千上万的向外派出铁鹞子。可惜抵达城下已经两天了,连攻城器具的材料还没有备足,还不知道的什么时候才能让官军踏上灵州城头。

姚麟当做椅子坐下来的树桩,应该是刚刚被砍下了树干,木纹上摸上去带着点湿润,渗出来的树汁还有些粘手。

灵州附近的树木不算少,但姚麟放眼望过去,触目可及的范围内,基本上都是手腕粗细,最多也不超过碗口粗的小树,略粗一点的就只剩树桩。

“不知彭七还能不能找到合用的木料?都几天了,一根大一点的木头都没进大营。”

几名军官安排好麾下的士卒,便聚了过来。

“找个屁!给了西贼近半年,没砍得只剩牙签,已经运气够好了。”

“再找不到,高总管那一关可过不去。”

“算他倒霉,谁让他轮到这个差事!”

灵州城周围几十里内,稍大一点能用在制作攻城器械的数目全都被砍了,而村庄中拥有木梁的房屋,也都烧个了干净。

想要攻城,就必须要有云梯、霹雳砲之类的器械。但眼下的情况,却是不可能在短时间内打造得出

两路大军带来的工匠有一百多个,只要调来一两千人配合他们,加上充足的原材料,霹雳砲应该很快就能造出来。

可巧妇难为无米炊,没木头谁都没辙。

“也是彭孙运气不好,要是灵州川的水多一点,也没这么多要烦心的事了。”

“还不是高总管不识天文地理的错。人在夏天过瀚海能晒得只剩骨头,灵州川还能多冒出水来?”

几个军官一齐扭头看着路边的河道,只有浅浅的一层河水,快到河中心了,也不过没了小腿,这就是经过瀚海后的灵州川。

从灵州川上放木排下来,本是高遵裕的计划。

横山北麓的树木虽说比不上南麓繁茂,但数量依然无穷无尽难以计数。而从横山下来的灵州川又直通灵州,就算灵州没有木料,到时候将树一砍,扎成木排,顺水漂流下来。打造攻城器械绰绰有余,多的也能用来搭建营寨,顺便还可以用木排运送些草料。

一开始所有人的确觉得高遵裕的计划很不错,但看了到灵州川的现实情况,就没人幻想了。灵州川的水流到瀚海中之后,上面晒下面渗,没有多久就只剩一尺多深。到了灵州之后,更是连给全军的饮用都只是勉强,何谈水运。

“灵州川是北流,比不上山南山溪水丰。靠的多是雪水,春天是水最多的时候,现在在瀚海里面都快晒干了,木排到了中途就搁浅,载货更是别指望了。”

“钤辖。”一个年轻点的军官问着姚麟,“是不是西贼一开始就打着主意要退到灵州城下了?把灵州周围的树都砍光,除非是年初就开始动手。”

姚麟还没说话,另一个高个子的军官就冷笑道:“不把我们诱到灵州城下决战,难道还敢在横山脚下跟官军厮杀?”

“不过瀚海,就凭西贼那本事,”姚麟指了指挂在马鞍后的西贼头颅,“就是送首级来的。”

“现在我们让西贼如愿了,就不知道西贼下面会怎么做了。高总管把苗总管当贼防着,只让环庆军围城,让泾原军在外面守备。两帅不合,这仗怎么打?”

“要高总管、苗总管能合得来,钤辖也不至于跟着我们一起出来。”

整整两个指挥的骑兵虽然人数不少,但对于一路都钤辖的麾下兵力来说,就显得太微薄了,姚麟要不是躲着大营里面两帅相争的风暴,何苦从大营里跑出来吹沙子。

苗授之前没有依从高遵裕的军令,打过黛黛岭与其会合,而是绕去攻打鸣沙城。粮草的确夺了不少,却也把高遵裕彻底给得罪了。

当两军抵达灵州城下会合时,高遵裕甚至打算夺了苗授的兵权,将指挥泾原军的权力交给姚麟来执掌。但姚麟哪里敢接手?一路副总管的兵权只有枢密院能剥夺。高遵裕得到的许可,也不过是指挥泾原军的权力,没有说将人事权也给了他。

两帅相争,姚麟可不敢掺合进去。

“当年苗授之父苗京战死麟州,他的功劳是救援麟州的主帅高继宣报上去的,苗授因此得到荫补,算是高家一系。这高继宣就是高遵裕之父,当年高遵裕能带着苗授去熙河沾光,就是看在这点情分上。不过现在两家是一点情面都不讲了。”

“俺觉得还是高总管心眼太小,不过是……”

姚麟用力向下一挥手,将抱怨给打断:“别掺合,也别多议论,管他们那么多。不是我们掺和的。”

姚麟自叹,要是在河湟开边时多立点功劳,在横山之役时的职位高上一点,如今也不会仅仅是个皇城使、都钤辖,还要躲着高遵裕和苗授。

一骑探马此时忽然由远处而来,破开风沙,在了姚麟亲兵的守卫圈之外下马。与亲兵说了两句,便被领到了姚麟的面前。

“皇城。”探马单膝跪在姚麟身前,“八里外发现铁鹞子一部,大约一千五百骑!”

“一千五?!”

“怕什么,我们是在上风口!”

“没错,逆着风可打不了仗。”

“皇城!出战吧!”

一众将校顿时兴奋的嗷嗷直叫,眼巴巴的将渴盼的眼神投向他们的主将。

姚麟抬起手,做了个安静的手势,“再等等!”

不过等到第二骑、第三骑赶过来,姚麟就不再多等了,一跃上马。不是方才他骑着的瘦马,而是另一匹始终随行的肩高四尺五寸、膘肥体壮的河西骏马。

环庆路都钤辖带在身边的都是精锐,不须多言,一看姚麟换马,哪个不知道这意味着什么。一个个摩拳擦掌,纷纷跨上了上阵时的战马。

姚麟将银枪提过来,向着西北斜斜一指,“杀过去,杀个封妻荫子出来!”

姚麟的鼓舞催动着人心,顿时引发一片低吼,吼声如夏日暴雨前的闷雷,压抑着即将到来的狂暴。

