\section{第三章 陋室岂减书剑意(下)}

只是初来乍到,贺方很清楚表面文章是肯定要做一做。至少不能让韩冈的家人,看破他与韩冈的不同。每天读书,习字,过去韩冈如何做的功课,如今贺方也照样去做一遍。每天早上起来刷牙洗脸后便是读书,也幸好这具身体十八年来的记忆基本上都保留了下来,贺方依样画葫芦并不算为难。

日复一日读着经书,贺方不免有些气闷。九经三传韩冈早已背得滚瓜烂熟,只要看了第一句,全篇都能背下来,甚至连比经书还多数倍的注疏都能背个八九不离十。这些记忆,贺方很顺利的继承了下来,一般只要提个头,自家就可以很顺利的背诵下去。不过贺方还是着意日日诵读,即便再深刻的记忆,如果不去时时温习,还是照样会消磨褪去。

放下书后,贺方时常在想,若他能带着韩冈的记忆回到千年之后,凭着自己人话鬼话说得都顺溜的口才,在百家讲坛混个露脸应该不成问题。

‘只可惜啊……’贺方轻轻叹着,韩冈的才学若是留在此时却也不过是寻常。韩冈留下来的不仅仅是记忆和书卷,还有他过去做过的文章和写过的诗词。文章倒也罢了,以贺方的水平无从评判,最多觉得有些地方缺乏逻辑,结论和论据对不上号。但做得诗词,贺方随手翻了翻,都觉得看不下去。

大宋本土已经承平百年,文风浓郁,才子辈出,流传千古的词句俯仰皆是。说塞上风光,有‘长烟落日孤城闭’,说送别,有‘对长亭晚,骤雨初歇’,说闺情,有‘泪眼问花花不语,乱红飞过秋千去’。

在贺方想来,韩冈的诗词水平纵然不能跟这些名家相提并论,也该有个一二成的水准,想不到却都些让贺方也觉得惨不忍睹的作品,韩冈竟然还用这些应该一把火烧掉的东西与他的同学们互相唱和!——韩冈在文集中记录下来同学作品,也是一般无二的水准。

‘这叫什么诗?!难怪关西出不了进士!’

若陕西士子的诗词歌赋都是这等水平,被江南的举子们杀个落花流水也没什么好惊讶的。将铺在桌上的韩冈和一群无聊文人唱和的七八卷诗集往书架上一丢,砸得书架一阵摇晃。

醒来不过十数日,韩冈的记忆贺方已经渐渐熟悉,但韩冈的身份贺方还是觉得陌生,总是以第三方的目光来看待前身,包括他的诗文。看到韩冈的大作,贺方也不去指望能作为借助。如果让贺方代替韩冈来考,莫说考进士,恐怕连通过州里的发解试都有难度。

贺方从韩冈的记忆中得知,通过解试后的士子,称为贡生,也可称为举人。但与后世的举人不同,这不是一种终身通用的资历,而是一次性的资格。这次通过解试,去京中考进士不中,那三年后如若想再考进士,还得先参加解试并通过,否则照样没有贡生资格。

而且今科解试在自己躺在病床上的时候已经过去,州中的贡生都已经选出,准备明年去东京城考进士。自家要想考,也得等三年后。

三年后才能买的奖券,中奖的机率又小得可怜。贺方完全没兴趣去测试自己的运气。除非朝廷能将进士科的考题,改为他更容易熟悉且对文艺天赋要求不高的经义策问,否则他便无望一个进士!

“难度太高了!”贺方摇着头,幸好做官发财的途径不止这一条。比如考明经——这是比进士科难度稍低的一门科举考试;比如投到一些高官门下,立些功劳等待推荐;又或是直接花钱买官——此时称为‘进纳’。

“买官?”贺方环视房中,哈的一声苦笑。至少在眼下,比中进士还有难度。

韩家已是穷困潦倒,安身的草庐还是租来的。而过去虽是在村中还能排在前面,但看看自己房中的这些从旧家中带出来的家具,寒酸之气也自透了出来。一张床榻、一面书案、一架书橱,两个木墩,仅此而已。

这几样家具的形制都很简陋,就是几根杨木横平竖直的拼接起来。没有打磨过,显得很粗糙。上面没有用一颗钉子,只用上了榫铆。尤其是书架,榫头凿得有些宽松,碰一下便摇摇晃晃、吱呀作响。书架上的几个格子叠放着百八十卷书,泰半是韩冈一笔笔亲手抄写下,再辛辛苦苦从求学的地方背回来的,有九经三传以及一些经传的注疏,甚至还有十余卷史记断章。

而另外的二十多卷,却是货真价实的宋版书,但皆是福建版,而不是国子监或是杭州的出品,更不是私家刻印的版本——论天下书籍印数之多,流传之广,福建版居第一,而私家版本最少。但论起质量来说,福建印坊卖的书籍却是最差的。而韩冈,也只能买得起福建出品的书籍。

桌上的文房四宝也是透着贫寒。两条都磨得只剩半截的残墨,一块没有经过仔细打磨的石砚台,半叠略显粗糙的黄纸,一具挂了四五只毛笔的笔架旁边又放着一个半尺高的竹节笔筒,里面装了七八支半新不旧的毛笔。这便是韩冈所拥有的所有的文具。

‘真是名副其实的穷措大。’

半个月下来,贺方渐渐将身体旧主的记忆融会贯通了小半,已经能活用此时的词汇,也能明白唯一有点来历的竹节笔筒上的几行行楷究竟是什么意思。

“青玉半枝,其理劲直。宜记其心,宜体其节。以赠玉昆。”

贺方将竹节笔筒拿在手中,轻轻的读出声来。很漂亮的书法,字如行云流水,又有一分端庄大气,不是俗手可比。就在笔筒上的铭字左下方,还用更小一号的字体写上了——‘大梁张载’——四个字。这是赠送者的名号,也是这具身体原主人的老师。

张载这个名字贺方依稀耳熟,好像在那里听说过,却又记不起来。他对宋代历史了解得很少,学校的历史课睡觉的时候居多,能让他依稀耳熟的宋人名号,在这个时代多少也应该是个名人。而在身体原主人的记忆中,他的这位老师也是被世人恭称为横渠先生而不名,在关中士林名望甚高。

一想起韩冈的老师,贺方的脑海中便闪过一个场景。一名四十多岁的中年人,中等上下的身材,平凡普通的相貌,可举止气度却是非同一般,处处透着刚正严毅。正在一间还算宽敞的土屋中为十几二十名学生讲经说文:‘有不知,则有知;无不知,则无知。故曰:圣人未尝有知,由问乃有知也。夫子问道于老聃,问乐于师旷……’

老师在上面解释儒家经典,一群书呆在下面奋笔疾书。如果不论教室的结构,和师生的装束,这样的场景贺方其实很熟悉。

“不,不能叫书呆……”

贺方摇摇头。韩冈跟随张载,除了学习儒家经典以外,还有着兵法、水利、天文、地理、射箭、音乐的课程,张载绝不是只会教学生死读书的老师,而学习儒家经典也不是全是解说空洞的大道理,其中需要用到的天文地理上的常识也很多,箭术更是先圣都要学生多练的课程。

正如韩冈房内的墙壁上挂着的一张三尺长的反曲弓,是黄桦弓身,有丝麻绞弦,制作得不算精致,但更有一分粗旷之美。贺方将弓取下,拉了拉弓弦,却纹丝不动。感觉很硬,大病初愈后没有多少气力的双臂根本拉不开。

按照记忆中的数据,这是一张一石三斗的强弓,也就是要一百三十斤气力才能拉动,是出门游学时自家二哥的赠礼,比起普通五六斗的猎弓强出了许多。韩冈靠着这一张弓,在上百名同学同时参加的射赛中,屡次杀进前五。其箭术绝然不弱,这一点也可以从他指腹处还没有消退的老茧可以看出。

翻来覆去看着自己一双骨节凸出的大手,贺方想着等身体稍好一点,就要加强练习箭术。原本身体所拥有的能力,经过半年多的空白期,又经历了换主的风波,已经渐渐模糊。贺方是个悭吝的性子,不会任其白白流失,不但是读书,还有射箭,都要重新习练起来。艺多不压身,多一项本事,日后就能多一种选择,来自前世父亲的教诲,贺方记得很牢。

射是君子六艺,古时儒生无不是文武皆备,一手拿书,一手执箭。韩冈的老师张载讲究的也是以六艺为本。在韩冈的记忆中,他曾随侍师长,见识过许多名家,甚至还有传说中的理学始祖程颢、程颐,而他们恰好是张载的表侄。

二程与张载都是儒学宗师,聚在一起便开始讨论着什么‘天地本无心,而人为其心’的问题……

“天地无心!?”

贺方突然怔住了,差点失声叫起,他怎么到现在才想起张载是谁!?横渠张载留下的名句可是挂在中学教室的墙上,自己看了整整三年,而在穿越前,又因被人引用,而在电视和报纸上看见了多次

——为天地立心!为生民立命!为往圣继绝学!为万世开太平!

这才是儒士该有的气度!

虽然在韩冈的记忆里,此时横渠书院尚未建立,四句铭传千古的豪言也未出现,但回想起留在韩冈的记忆中那一段深刻印记,也只有学兼文武、目纵古今、心系天下的张载才有如此气魄!

\begin{quote}
“为天地立心!\\

为生民立命!\\

为往圣继绝学!\\

为万世开太平!”

\end{quote}

贺方一字一字的吟哦出声来,一股豪情壮志在心底涌起。穿越后他还是第一次感受到自己与历史有了最直接的接触,恍惚间自己的意识已与韩冈难分彼此,

‘原来这就是我的老师……’

ps:张载的名字知道的人也许不多。但‘为天地立心’这四句,恐怕不知道的人就很少了。比起二程、周敦颐、朱熹等宋代其他儒学宗师来,张载的气魄心胸远远超过他们,文武兼才,是贯通六艺,心怀天下的真儒。只是没有收到一个好弟子,让他的学问化为流水,只有横渠四句千古流传。

