\section{第28章 官近青云与天通(八)}

“这是什么话?!”向皇后出离愤怒了,“是官家和我将他赵仲糺逼疯的吗?!” 

大发雷霆的皇后,让福宁殿上下都噤若寒蝉。一名名内侍和宫女都缩起了身子。 

“昨天夜里,官家都那般委曲求全了,就只想保住六哥。韩学士也不顾身家性命,只想请他赵仲糺出外,好保全六哥。但结果呢?!!装聋作哑!” 

“做兄弟的为兄长祈福难道不是理所当然的?做臣子的为君上求平安难道不是圣人说的纲常大义吗?!但赵仲糺他都不干啊!” 

“就这样的儿子,还硬是要留在京里。不理忠臣之言,连点母子情分更是都不念分毫,什么时候还记得官家是她身上掉下来的亲骨肉啦!?” 

向皇后气得几乎语无伦次,手脚直颤着,说着说着泪水就涌了出来,当着福宁殿中内侍和宫人们的面呜咽着:“要是太皇太后还在,要是太皇太后还在,有她老人家主持,哪里会变成今天这个局面……” 

“圣人,还请息怒啊……” 

几名贴身的内侍、宫女在旁劝解着,却是一点用都没有,直到外出的宋用臣回来,向皇后的情绪才渐渐稳定下来。 

她眼皮微微红肿,带着浓重的鼻音问着宋用臣,“赵仲糺的情况怎么样?医官说了什么?” 

“雍王已经安静下来了,雍王府的医官都受了韩学士的吩咐。” 

“韩学士已经知道赵仲糺的事了?”向皇后先有几分惊讶,但想到韩冈的身份,便觉得医官们通知他也是正常的,“韩学士怎么说?” 

“韩学士只是吩咐雍王府给雍王安排一间避光、避风的屋子,在墙壁上钉上厚毡和棉花,以防雍王自残!” 

“韩学士做得好!”向皇后毫不犹豫的夸着韩冈,“要是乱给药,还不知怎么会编排官家呢。” 

宋用臣有些尴尬,更加小心翼翼的更正道:“圣人,据雍王府的翰林医官回报,韩学士还吩咐了,要给雍王开方子。” 

“韩学士开了什么药?”向皇后先是怫然不悦,但又立刻问着宋用臣。 

“韩学士说:清凉散好开,但雍王要的至圣丹是没法儿开的。先开些镇心理气的方子,让雍王好好服用。” 

“韩学士说得好!”向皇后心中顿时痛快无比,用力一拍手边的桌案。清凉散的典故是京中流传甚广的笑话,她平日里闲聊时,没少听人说过。宋用臣这么一说,她便立刻就明白韩冈话中之意,“想要至圣丹,也不看看自己配不配!镇心理气的方子,开得是最好不过!” 

宋用臣唯唯诺诺,不敢接向皇后的话茬。 

“韩学士这样的才叫肱股之臣。”向皇后一声叹息,也不知是拿谁做对比。停了一停,她又问道:“当年有个姓章的小臣,就是曾经上书让雍王离宫,却被太后逼着官家将其发遣出外的那一个,现在他在哪里做事?” 

宋用臣想了半天,却完全回忆不起来。那大概是十年前的事了,纵然他那是就已经在天子的身边,但区区一个刚露头就被赶出去的小臣,哪里还能留下什么记忆。 

而且向皇后突然提起此人,原因不问可知。说实话,宋用臣甚至都觉得两宫之间,都快要到不死不休的地步了。冬至的早上,皇后还恭恭敬敬的向太后行礼,现在就跟仇人没两样了。难道说是这些年来,在心底里已经积攒了多少怨恨,到今天才爆发出来? 

也不敢再多想,摇了摇头,宋用臣老实的回答道:“奴婢不知。” 

向皇后有几分不快,看了宋用臣一眼:“去知会政事堂,将人给找出来。如此忠臣贬居在外,朝堂上却尽是些忘恩负义之辈,这是哪来的规矩?!” 

向皇后一想起昨夜王珪的沉默就恨得心口发痛。要不是官家说的‘使功不如使过’,要不是王珪摆出了痛改前非的姿态卖足了力气,她今天就要将当今唯一的宰相给踢到京城外去了。蔡确、韩缜哪个不比他强,韩冈从品行到能力,更是强出百倍。 

“圣人,雍王可以不论,但太后那边……”宋用臣都不知道该怎么劝母狮一般的向皇后,嘴张了半天,才挤出一句,“官家的名声要紧啊。” 

“官家的名声不好吗?”向皇后尖声怒道:“官家顾全兄弟手足之义,对两位大王和蜀国赏赐从来都没缺少过,甚至自己都舍不用的器物、珍玩,照样赐予弟妹。孝道一事上更是从无疏失,福宁殿十几年来就修补过一次,庆寿宫和保慈宫年年翻新。时新蔬果、珍宝珍玩,都是想到太后。晨昏定省,又有哪一天少过?高家更是人人富贵,难道官家做得还不够吗?!” 

宋用臣扑通跪下,虽然他只是阉人,但自总角受学以来,忠孝二字决不敢违,现在看到向皇后快要在明面上跟太后过不去,却不敢不规劝,“圣人。母慈子孝乃是常例,不足为奇。就是因为父母不慈,虞舜依然守孝如故,这才被千古称道啊。” 

向皇后胸口起伏,怒瞪着身前的宋用臣。 

这是当年高太后和英宗关系紧张,英宗抱怨说高太后待其无恩,韩琦劝英宗皇帝的话。宋用臣现在说出来,就是要向皇后以旧事为鉴,纵然太后偏心不慈,也不要让还在病榻上的天子,落得个跟英宗一样忘恩负义的名声。 

宋用臣苦口婆心的劝谏,向皇后心中对高太后的怒火稍稍平息了一点,但对雍王的恨意却又立刻涌了上来:“一个阉人都知道忠义,贵为亲王却还不知道!” 

宋用臣跪在地上,对向皇后给自己的评价,不知是该惊,还是该喜。 

“圣人,圣人,官家醒了。”在寝殿中服侍天子的小黄门跑了出来,急声禀报于向皇后。 

“官家醒了?” 

向皇后尤带着几分欣喜。赵顼仅仅是因为疲累才睡着了,但中风后的睡眠,谁也不敢保证病人会不会就此一睡不醒。 

坐到床榻边,先服侍过赵顼喝了药汤和稀粥。面对着睁着眼睛的赵顼,向皇后将今天发生的几桩要事,很是简省的做了禀报。 

“三叔刚刚上书了,愿意出京为官家祈福。奴家安排了蓝元震带上一个指挥的天武军和御龙弓箭直,一路护卫他去河北。” 

“六哥儿回去后就在抄写金刚经,说是要为官家求平安。” 

“王相公已经接了平章军国重事的制诰,明天就能上朝了。” 

“保慈宫那边,奴家方才让蜀国去作陪了,还请官家放心。” 

“雍王回府后就突发心疾,病狂了,脱了衣服在院中跑。” 

“韩学士不敢让医官用药,只敢让雍王静养。怕出了事,累了官家的名声。” 

向皇后絮絮叨叨,说话也不是很有条理,赵顼静静的听着。只在听到赵颢病狂的时候,眼神才波动了一下,其他时候,都是平静得近乎毫无知觉一般。 

不过到了最后,向皇后也没有提起给韩冈学士之封的话题,只是问道:“官家,还有什么吩咐。” 

赵顼停了半天,方才眨了眨眼,示意并没有吩咐。 

“奴家知道了。”向皇后起身,屈膝福了一福。 

赵顼垂下眼皮,甚至有些冷漠。 

赵顼和向皇后之间的微妙,站在后面的宋用臣尽收眼底。心道官家终究还是舍不得他的位置

英宗垂危时,用宰辅之议,立赵顼为皇太子,却因此而泫然下泪。文彦博退下后,就对韩琦道:“可见陛下神色?人生至此,虽父子亦不能无动于衷。”韩琦则道:“国事当如此,可奈何!” 

之后韩琦还说过纵使英宗病愈,也只能为太上皇的话。这两件事很快就被有心人传到赵顼耳中,赵顼由此而对定策元勋的韩琦甚是冷淡,自其出外之后,就再也没有动过招他入京任职的念头。 

在帝位面前,就算是以父子之亲,夫妻之情,也抵不过那控制亿万生民的权柄。 

…………………… 

“二大王疯了,三大王出外。这变得可真快?” 

“都是皇后垂帘了,留京就得发疯,不想发疯就不能留京。” 

“别乱说。几名宰执都在场,若不是太后真的犯了众怒,天子也不可能跳过太后,然后让皇后垂帘的。” 

“太后又不能出宫。宫中全由皇后控制。谁知道是真是假。” 

“舅姑尚在,新妇却出面管家的例子,世间还少吗?所谓子承父业。延安郡王为皇太子,不正是合乎人情?” 

时局变化得太快了,从天子发病,到现在皇后垂帘,局势就像天穹上被狂风卷动的层云,倏忽间变得面目全非。 

但在世人的心目中终究还是有几分疑问的。并不因为皇后和两府诸公的身份,或是韩冈的权威,而稍稍平息。但也幸好只有几分疑问,若不是韩冈的名声具结作保,市井中的谣言就不会这么平静了。 

韩冈对这些谣言根本没去打听,他自离开城南驿后就直接回家。 

赵佣要侍疾,当然还不能开课。再过几天就是腊月,就是正常的书院也会放假了,资善堂放假的时间更长一点,基本上要等到开春后。 

但这仅仅是京城,当天子重病垂危的消息离开京城,天下也会随之震动。 

冬至后的第二天黄昏,一骑快马奔进了洛阳城。 

