\section{第31章 战鼓将擂缘败至(八)}

这一等就是五天,中间绥德城派了两队信使来查看和督促,到了第二次,甚至还带了圣旨的抄本,不过都给罗兀众官有志一同的拖了过去。

如果接受了命令,直接在数万敌军眼皮底下撤退,自己的小命能否保全还得另算,另外少说也要在路上丢下一半人马。回到绥德后,要么是官位降个七八级,要么就是调任南方闲职加以编管,肯定是要受重责的。而若是能把人顺顺利利的带回去,屁事都不会有,天子看到罗兀城中的士兵能囫囵个的回来,难道会不高兴?韩绛要担心的问题,他们却不需要考虑。

当然韩绛始终没有下达措辞严厉的正式公文,这也是张玉和高永能敢于把催促退兵的信使直接糊弄过去的原因。都是在官场上混老的,其中的问题一眼都能看得出来。

不过,这样的好日子到了二月朔日的这一天,终于到头了。宣诏使臣王中正竟然在一队骑兵的护送下,快马进了罗兀城。

王中正这位曾经到过秦州体量军事、并送来擢韩冈官职的诏令的大貂珰,与韩冈算是有点交情,王韶当初为了能直接跟天子搭上话,也考虑过请一名中使到秦州缘边安抚司任职,而来过秦州的王中正和李宪,就是他心目中的两位人选。

虽然韩冈知道,这个时代的宦官,每每有敢于上阵厮杀的勇武之辈,王中正也曾暗示过想到秦州镀上一层金,但韩冈绝没想到,王中正竟然敢于带着一百多骑兵,就这么径自进了罗兀城。

王中正的大胆,张玉虽然不喜欢阉人,却也不由得赞了两句。

王中正显然很受落,笑道:“中正既受天子之命,自无退缩之理。”

“赵郎中怎么不来?”

赵瞻的本官是祠部郎中,张玉故意问着他为何不来,完全不掩自己心中的怨气。张玉这几天两次收到赵瞻的信件,言辞间很不客气,地位甚高的老将当然看得不痛快。

“赵郎中坐镇在绥德城中,中正跑腿惯了,所以受了这件差事来。”

王中正微微笑着,但眯起来的双眼中,却是寒光隐现。他倒不是主动来罗兀,而是为赵瞻所逼。当文官和阉官同任一职,当然是文官在上,阉宦靠边站。赵瞻使唤得理所当然,却并不代表王中正会乐意。

“怎么能劳动到都知?”张玉看了看王中正的脸色,突然试探的问道,“是咸阳那里出事了?”

老将张玉不是能随便糊弄过去的人。王中正点头叹气,毫不隐瞒的回答道:“赵大观【赵瞻字】心忧王事,欲救咸阳百姓于水火,不意吴逵狡诈,让攻打咸阳的泾原军损失不小。”

“所以急着要罗兀城撤军?!”高永能问道。

王中正又再点头称是。

好了,这下众人都明白了。

这是来要兵将的!以便把纠缠在罗兀战事上的数万大军解放出来,好去平定叛乱。

罗兀城中的一万七八千人,是选自鄜延、环庆两路的精锐,而为了保住罗兀城的退路,绥德的种谔、细浮图城的折继世,他们手上的近两万人也不得不留在两座城池之中。少了两路四万精兵,吴逵尽管是被重重围困在咸阳城中,但光靠从秦凤和泾原赶来的军队,却很难打得下来——秦凤、泾原两路都要留兵防守,能出动的兵力不会太多。而且还因为赵瞻的催逼,不得不仓促上阵,吃了一个大亏。

王中正一番话虽然说得曲言宛转,但其实已经是很严厉的指责赵瞻在做蠢事,要不然不会把因为赵瞻才导致咸阳兵败的消息,在罗兀众将面前透露出来。

‘这等文官,当真只会坏事!’

不知在场的有多少将领在肚子怨声连天。

赵瞻是陕西人,而且在关中的名气不小,韩冈听说过他。他为官的名声并不差,尤其在他曾经任职多年的河中府【今山西运城】有着很高的声望。但今次他做得就有些太过了一点,士大夫的脾气把王中正这位跟他一起来宣诏的中使,逼得没处站。看王中正的话中隐含的怨愤,就知道在赵瞻手上受得气不小。

张玉一切了然,便道:“现在是因为党项人在外围困,不得不谨慎行事。只要稍有机会,当会立刻退兵。还请都知少待两日。”

王中正没正面答话,只从怀里掏出一封信,递给张玉:“这是韩相公的手书!”

张玉微微变色,拆信而看。韩冈从张玉身侧瞟了一眼,只看到信笺的最后面是一颗鲜红的宣抚司大印。

看来因为咸阳兵败,韩绛也已经加入了催逼罗兀退军的行列。

韩绛是被逼的。罗兀城软磨硬泡的不肯立刻撤军,如果梁乙埋能在此时退军,而广锐叛军又被剿灭,事情说不定还会有点转机。但眼下盘踞在咸阳城中的叛军,让韩绛无法再等待下去了。

王中正等张玉、高永能等将领一一传阅过韩绛的手书,便又道:“中正今次还奉了天子下令弃守罗兀的诏令,这是赵大观于出行前转予中正的……在下并不希望用到。”

王中正虽然觉得张玉说的话很有道理,但他从天子那里接到的命令,就是让罗兀城撤军。与文臣不同,他这样的阉宦,根基来自于天子的信任,没有胆子去反抗天子的诏令。

张玉花白的双眉皱了起来,向王中正叫着苦:“可是数万西贼精兵就在几里之外虎视眈眈……并不是不肯从命啊!”

这时一名高永能的亲信冲进了帐来,很兴奋的叫道:“总管、都监,梁乙埋退军了!”

叫完之后,方才惊觉帐中的十几双眼睛瞪着自己,还包括张玉的。他的身子颤了一下,退了半步,仿佛落进了蛇群的老鼠,惊慌失措得瞪大眼睛。

韩冈在后面不由得苦笑起来,这未免也太巧了一点!

………………

党项人正在大张旗鼓地从罗兀城外撤军,一支支队伍跟随着旗帜,消失在北方的山峦之中。但有四支铁鹞子的千人队分列守护,监视着罗兀城的动静,提防着城中趁机出兵。

张玉完全并没有追击的意思。宋军惯用的战法,本就是守成有余,进取不足。列阵而守,契丹铁骑也要绕行,但说起进攻,却是千难万难。不仅战术上如此,连战略上也是一般,要不然种谔突击罗兀城,也不会这般让人惊讶。

种朴和韩冈跟着张玉、高永能上了城头,望着向北方行军而去的党项军。

“果然还是那么老套!”种朴轻声叹着。

梁乙埋应当不会真走,而是暂时撤过横山,改在银州等待。像一只伏在树丛中的老虎,等待着扑击的机会。以战马的速度,追上泰半步兵的罗兀守军,也不会有任何问题。只要在山间多多放出游骑,宋军的斥候也很难越过山脉,打探得到山背后的消息。

但这话对王中正是没法儿说的,反而会让这位大貂珰认为城中诸将是欺他不习兵事。

不过该如何应对眼下的局面,罗兀城中早有了预案。

一人计短,众人计长,其实这段时间以来。张玉、高永能提要求,韩冈出主意,下面的幕僚再作出方案来,这已经有了一点参谋部的雏形。为实现顺利退军的目标,他们准备了多套计划,也对党项人可能有的反应,做出了相应的预测和应对。

眼下党项人伪装的撤离,在预计中,其实是几率最大的一个。

以他们离开的速度,大约要一天的时间。故而到了明天,罗兀守军就不得不放弃他们坚守了多日的城池——当王中正听说党项大军需要一天才能全数撤离,他就是这么想的。

但高永能的做法却是与王中正的想法截然相反,在午后掩映在云翳之后的黯淡阳光下,他在城头上招来一名名军官,连番号令:

“你速去绥德城,禀报种子正,请他立刻北上接应。”

“你去通知细浮图城,让折继世盯着,别让梁乙埋绕道我们的前面去。”

“去把车马都准备好,随身带上五天的粮草。”

“带不走的军资,全都浇上油,待我的号令!”

“把城下的暗道好好封起来,不要让西贼发现了。”

“全军依照计划行事,两个时辰后,撤离罗兀城。”

高永能言出如山,城中的士兵如臂使指,依照他的命令,迅速的进行着撤离前的准备。撤退的方案,在多日的时间里,所有的环节都做了计划,传达到每一个指挥使手中。只要高永能或是张玉下令,所有人都会按照计划行事,知道主帅下令改变计划。

张玉和高永能眼下都不觉得有必要改变计划,他们只觉得这样指挥起来实在是太方便了,并不需要多说废话,每个人都知道自己该去做什么,只要得到命令就去做。

“难道现在就撤退?!”王中正终于反应过来,惊声问道。

“自然!”张玉微微一笑,回头看了一眼跟在身侧的韩冈。最好的方案是再拖上七八天,梁乙埋再有本事,也变不出来粮食,维持不了足够的兵力了。但眼下的情况,却是不允许他们这么做。既然如此,那就是以快打快,在无月的夜色下解决一切!

转回来,他对王中正道,“今夜就撤!”

最好的方案是再拖上七八天,梁乙埋再有本事,也变不出来粮食,维持不了足够的兵力了。但眼下的情况,却是不允许他们这么做。既然如此,那就是以快打快,在无月的夜色下解决一切!

