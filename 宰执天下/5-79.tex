\section{第十章 却惭横刀问戎昭(11)}

【不知十四点还能不能算上午。好歹是赶出来了。】

西陉寨外,战鼓已经敲响。

一名名契丹骑兵在城下来回飞驰,用他们的行动,嘲笑着寨墙上拙劣的射术。

城头上神臂弓弓弦不断鸣响,可离着五六十步,想要算好提前量,将自如游走的骑兵射下马来,还是国法难了一点。只是让人看得烦心,连骚扰都算不上,骑射想要射上城头,得贴着寨墙才能够成功。

西陉寨外百余步的地方,辽人竖起了几支长杆,杆上悬着一颗颗人头。隔得稍远看不清相貌,但从他们的头盔上还是能分辨出那是宋军的样式。

西陉寨北的山间,方才还直冲云霄的几道烟柱,已经有三道消散得近乎无影无踪,只有眯起双眼,运足目力,才能在一片浓绿的山头上,发现那仅存的淡淡烟迹。

已经确定有三座烽燧被攻破了,都是没能来得及撤回来。一座烽燧满编也只有十人,在辽人的攻击下并没有多少希望,能坚持将狼烟放起已经很了不起了。

燃烧着狼烟的烽火台还有最后一个,但秦怀信已经不抱指望,只盼着最后一座烽燧的烽帅能伶俐一点,看到情况不妙,就带着手下的烽子逃入山中。反正之前都已经下了撤离令,临阵脱逃的罪名不会加到他们头上。

城头上的弩弓,追着城下来回蹿动的骑兵。这群比老鼠还要滑溜的骑兵,用了三条性命测算出了神臂弓的有效射程之后,便踏着那条危险的界限肆无忌惮起来。他们在马上行动灵活得用双脚走路,就是用神臂弓也来不及瞄准射击。

但这并不是攻城的样子。如果当真要攻城,就不该是骑兵上阵,而是将攻城器械和步兵拉出来。

霹雳砲的结构并不复杂,若是能多看几眼,复制出来也不是难事。飞船都有了,何况更为简单的霹雳砲?若是今天辽人搬出十几架霹雳砲来,秦怀信也绝不惊讶。同时也不会担心,西陉寨的寨防并不仅仅是一堵墙而已,而是一套高低搭配的壁垒体系,不是十几架霹雳砲就能攻下来的。更何况对付攻城器械的武器他也是有的。

秦怀信都为此做好了准备,但辽人并没有拿出霹雳砲,或是其他攻城的装具。甚至连飞船都没有。纵然辽国先帝因飞船而亡,但上阵时,谁还管这些,需要用时肯定会用。可秦怀信偏偏就没有看到。

“看起来韩经略说的没错,并不是当真要攻城。”儿子秦琬叹道。

‘还是没有昏头。’秦怀信低声自语,让秦琬疑惑的扭头看过来。

秦琬没有在父亲脸上看出什么异样,过了一阵,他问道:“……那辽人还是会按照预计,去攻打左近的村子?”

“多半会如此。”据秦怀信所致,至少有六个村子不用经过西陉寨,就能从北侧进入村中。

秦琬轻笑道:“幸好都有了准备。”

雁门关防线,是以雁门寨为核心,主要兵力都放在雁门寨中。西陉寨由于处在最前沿,只有两个指挥的兵力,总数八百二十余人。只是因为之前代州传令,分了一部分兵力去几处村寨设伏,现在寨中还有四个都,剩下的空缺都是由征发起来的乡兵弓箭手来补足,好歹让寨中兵力达到一千两百。

另外西陉寨应该还有两个指挥的援军,这是之前雁门寨那里承诺过的,可不知到底什么时候能到。即便他们不到,秦怀信也有自信利用西陉寨外围的城防抵御下辽人的攻击,莫说眼前的区区三千人,就是十倍于此,他也有信心。

“飞船准备好了吗?”秦怀信问道。

秦琬转身张望了一下,回头来对秦怀信道:“二哥儿就在上面。”

但也没有说什么,脚不着地的感觉,秦怀信并不喜欢,但自从气球配发下来后,每一次启用,次子却总是争着上去。几次下来,秦怀信都懒得多说什么了。

巨大的圆球形气囊从寨内冒出了头,一众辽军的视线顿时汇聚,一艘绘制着哭笑喜怒四色鬼兽面容的飞船便在他们的注视下冉冉升上了天空。

宋人的西陉寨内部,已经比寨外的坡地更高出两到三丈,而飞船更是虚悬在二十多丈的高处,被长长的绳索牢牢的系在军寨的上空。

气囊上的鬼兽图案并没有让寨外的契丹骑兵产生半点惊惧。能随着天子射猎的五院铁骑,对于漂浮在天空中的飞船见识过的次数已经是太多太多,又不是什么稀罕物。最多也就诅咒一句飞船吊篮上的乘员,也从空中摔下来。

骑兵们仍在纵马而过,耶律盈隐领众守在西陉寨的寨门一里地外,屈指轻敲着身下的马鞍。他要在国中出人头地,就要做点事出来,所以才有了今天一战。军令状又如何,有两千本部兵马,就是败了,萧十三也不敢动手。何况等到他攻下几个村子,弄上几百个人头轻而易举。缘边的宋人家里都藏着弓刀,一堆兵器搜出来,摆在一起的首级谁能说这是民?

奇异的尖啸破空传至耳中,耶律盈隐疑惑的睁大眼睛,几道黑影划过眼底,下一刻,凄厉的惨嚎在耶律盈隐的身前不远处响起,但立刻便戛然而止。

惨叫声的落处,是三支五尺多长的铁枪。破风斩浪,无可阻挡的穿透了行进路线上的一切阻碍。最近的一支连人带马一起贯穿,箭簇已经扎入了泥地中,猩红血色的箭杆裸露在外,就连铁质的箭翎都从马腹下透体而出。

“床子弩!”

“八牛弩!”

惊叫声同时响起。这是宋人最为著名的神兵利器,比起神臂弓还要让辽人如雷贯耳。耶律盈隐站在一里外,正是为了防备床子弩,本以为已经是够远了,想不到竟然还能射到这里来!

护卫们身上鲜红的血液映在眼中,耶律盈隐脑中一片空白,一股几乎让全身麻痹的恐惧感直至发梢,稍停之后,回过神来,便立刻拨马而回。

城头上也是一阵失落的叹息声,澶州城外辽军主帅萧挞凛一击毙命,罗兀城下西夏枢密使都罗马尾亦是一箭而亡,但这样的奇迹果然是无法复制。

作为床子弩中威力最大的一个型号,如果八牛弩射出去的一枪三剑箭再稍稍准上一点,如果城头上的床子弩能再多两具,说不定今天就能立下扬名立万的大功了。

“要是再多几架床子弩就好了。”秦琬轻声叹道。

床子弩的射程和威力不负众望,但准头就跟远距射击的神臂弓一样让人叹息。神臂弓射出的木羽短矢,最远能达到三百余步,但那样的距离,基本上能偏出十几丈、甚至几十丈。想要比较精准的命中目标,基本上还是在七八十步的射程内,最多也不能超过百步。只有手持神臂弓的士兵们列成箭阵才能发挥出最大的威力,让西北二虏的骑兵一见之下,就远远的绕开。

而床子弩无论如何都不可能组成箭阵,一座城头上,数量寥寥无几,其实更多的是用来破坏敌军阵势和攻城器械的。

床子弩的射程几乎都在五百步以上,其中威力最大的八牛弩号称千步,甚至还有一千五百步的传说。但那基本上是得靠吹嘘,得借着风势、地势。但一里半的距离却是轻而易举,若是近至一里,甚至还能加上一点准头再射,十箭八箭下去,多半就能命中一箭。

这样的命中率,配上区区三架八牛弩,要想射中敌军的主帅当然不容易,可已经足够将城外的契丹铁骑赶得远远的。

有了八牛弩的威胁,辽军骑兵想要靠近城池,只能快马加鞭的一掠而过,不能在一里半的射程范围内多加停留,否则登时就会有几支一枪三剑箭射来。一支五尺长的铁枪,只要命中了,便可以宣告无救,谁敢冒这样的风险?

耶律盈隐便是不敢再靠近半步,被几张八牛弩远远的赶到两里开外驻足。这样的情况下,派到城下的去骚扰的骑兵,都要先跨过两里的距离,绕了一圈回来,还要跑两里,正常的战马哪一个吃得消?即便一人三马也支撑不了几次!

望着两里开外,已经变得模糊起来的城垣,耶律盈隐迫切希望他派去左近村寨劫掠的属下,能带着捷报和首级回来。

立足于城头上,秦琬远眺着,居高临下,视野的范围比起耶律盈隐要大得多,可以看得出对面已经没有多少作战的意志了。但他们偏偏还不退,理由秦琬用脚尖想都能才得到,契丹兵这是等着攻打周边村寨的士兵们退回来。

秦琬对此并不担心。

几处最有可能受到攻击的村寨,都已经将其中的妇孺撤进了西陉寨中。剩下的精壮在编制上都属于乡兵的行列,是缘边弓箭手,守卫家园时爆发出来的战斗力,就是寻常禁军都比不上。配合派出去的精锐,让辽人无功而返不难做到,如果安排得好,甚至能让其全军覆没——深处山峦之间的村子,

但不久之后,收到的信报却让秦怀信和秦琬变了脸色,六个村子有两个被攻破了,其余四处有两处的伤亡也不小,只有两处成功的伏击了来袭贼军,收获了近百匹战马——这是死的伤的和完好的加在一起的数字。

“攻打村子的辽兵总共有多少人?”

秦琬问着最后一名来报信的小卒,其部所守卫的村寨,正是两座被攻破的村寨之一。

小卒迟疑了一下,回复道:“……约莫八九百。”

秦琬眼中寒芒闪过,知道这个数字至少要打个五折:“也就是说,跟村中的人数差不多?”

回来报信的小卒低下头,不敢直视:“是比我们要多一点。”

同样的人数,这边还占着地利,事先又有了准备,竟然还是有两个村子被攻破,死伤不在少数。换做是自己,秦琬也并不认为有太高的把握能做到。

而且还不是宫分、皮室这样的精锐,多半是部族贵胄领下的头下军。

抬眼望着城外的敌人:‘果然还是不能小觑……’

