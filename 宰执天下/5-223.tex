\section{第24章 缭垣斜压紫云低(15)}

今夜注定是个不眠的夜晚。

赵顼的病情牵扯了千万人的心。

如果从高处望下去,可以发现内城中靠北的几座厢坊,灯火比往日要多得多,到了两更天,也没见几盏熄灭。

不知有多少人竖起耳朵,等着宫里面传出来的消息。

吕公著回府之后,只用了一刻钟叮嘱家里的儿孙这段时间要循规蹈矩,然后就回到书房开始写信。给家里的,给洛阳的,给相州的,给亲友的,一个时辰过去,桌上的信封已经多了五六封。

不仅是他,许多官员都在给亲朋好友写信。天子危在旦夕,帝位或将转移,政局剧烈的变动,在这过程中,便隐藏着一步登天的良机。

王安石也在第一时间得到了消息。可他只知道赵顼是在宫宴上发病,可具体的病情无由得知,这让王安石忧急上火,直后悔没有参加郊祀大典。只半日功夫,嘴角便生了燎泡,疼得厉害。

“爹爹。”王旁一手端着热水,一手托着两枚蜡丸出来:“这是上次玉昆遣人送来的牛黄清心丸。”

王安石捏开蜡丸,拿过来便就了白开水便吞了下去。

韩冈送来的成药或是药材,其品相全都是最高等级的。尽管韩冈不会去占官中的便宜,但只要他掏钱买,自然皆是真材实料。这也算是厚生司中官员、乃至吏员们的福利。

只是药吃了不会立刻见效,嘴角依然火辣辣地疼,王安石在厅内走来走去,坐不安稳。

“去玉昆、章子厚和薛师正家的人还没回来?”来回走了几圈,王安石又问着相同的问题。

王旁摇摇头:“还没有。爹爹还是先回去休息吧,等人回来了,孩儿会立刻通知爹爹的。”

“嗯。”王安石应了一声,却还是在厅中打着转,一点也没有去休息的意思。

眼下皇城落锁,其他途径打探来的消息全都不靠谱,只有被留在宫禁中的宰执和韩冈才会有准确的消息。王安石也派了人出去,但到现在为止还没有回音。这怎么让王安石不着急。

“相公,章副枢在外求见。”一名家丁匆匆进门向王安石禀报。

“他怎么来了?”

虽然疑惑,王安石仍是遣王旁出门相迎,将章敦请进了内厅中。

“子厚,你不该来的。”一见章敦,王安石劈头便道。

以眼下的局面,章敦亲自登门,极有可能给他带来很大的麻烦。届时御史发难,新党在两府中可就没人了。蔡确、薛向之辈,无论如何都撑不起新党的门面。

“现在的局面,派别人来转述是说不清的。章敦这是必须来见相公一趟。”章敦语气坚定的说着。

章敦事先并没有跟韩冈联系,但韩冈能想到的办法,章敦不可能想不到。而且与韩冈多年为友,有些事,不必韩冈明说就能感觉得到。只能在驿馆里待着的王安石的作用,比起两府中的宰执们加起来都要大。

王安石叹了一声,也不再多说什么,转而神色紧张的问道:“子厚方自宫中出来,不知天子病情如何?”

“幸而有玉昆在。”章敦随即便一五一十的将天子病发后,直至他出宫前所看到的一切,都原原本本的转告给王安石。

听完章敦的叙述,王安石沉默了许久。只是用力眨着发酸发涩的双眼,不让自己的泪水流出来。一想起当年才十八岁的皇帝,王安石的心便一阵剧痛。

那是他的皇帝,可也是他的弟子啊!

当年赵顼与自己一起讨论如何变法兴国,通宵达旦亦不知疲倦。一说起灭夏平辽,收复燕云失土,打下一个太平江山,那灼灼生辉的双眼,仿佛依然就在眼前。

虽然这些年来是疏远了很多,但来自皇帝的信任依然不减。也就在昨天,天子还漏夜来访,这份恩遇,前世罕有。昨夜听天子提起收复燕云,虽然言辞中对两府的保守多有不满,但还是打消了立刻攻辽的念头。不过赵顼也自信的放言,最多十年,十年之内就能举兵灭辽,完成夙愿了。

只是这突然一病,却让满腔的豪情化为泡影,还让坚持反对新法的旧党,等到了机会。

章敦说着自己的担心。

“难不成他们还能让二大王登基废除新法?”王旁质问道。

“还不至于让雍王登基。”王安石摇摇头。

赵顼既然有子,旧党若是拥立赵颢,只会在士林中留下恶名。旧党之中,在乎名声的人很多,如邓绾那般敢于‘放眼好官我自为之’的脸皮,还是没有太多人能比得上。

“有玉昆的话在,短期内的确还不至于如此。”章敦轻声叹道,“但太后垂帘听政的麻烦只会更大。”

王安石点了点头。天子人还在,加之自家女婿保证其能恢复说话能力。不论韩冈的保证有多少把握,又有多少是属于在拖延时间,王安石觉得短时间没人敢投注到赵颢身上。

且雍王不喜新法,太后也一样反对新法,既然结果相同,顺理成章的支持太后垂帘,自然要比支持雍王的人更多。换成幼主登基,或是延续现状,高太后必然垂帘,那时事情就会很棘手了。

不论是则天武后也好,还是本朝的章献明肃刘皇后,在垂帘之前,都已经有过亲历政事的经历。武后帮有目疾的唐高宗批阅奏章,刘后助病重的真宗皇帝处理政务,等到正式垂帘秉政,才能得心应手。可就算如此,在军政二事上,也算不得出色,只是权术上厉害而已。

仓促之间垂帘,又是从来没有经历政事,以高太后的性格为人,要说王安石不担心,那绝对是谎话。当年连亲手抚养其长大的曹太皇都不给面子,如今权柄入其手中,有岂会息事宁人,镇之以静?

“新法之功,世所共睹,难道还能废了?”王旁强调道,“那时国事必然败坏!”

“旧党的怨恨不在法度上,而是他们一直被压着不得上台。要是他们在乎国事败坏,当年国用入不敷出,亏空至千万贯的时候就不会只顾着拆台了。等司马君实之辈纷纷粉墨登场,就算恢复旧法中有何差错,只要全都推到相公和新法上就行了。”

“司马君实不是这样的人。”王安石摇头道,“以他的性格,若是他入居东府,新法纵然会废,也不至于诿过与人。”沉吟了一下,他补充道:“而且免役法应该不会动……当年司马君实可是写过变衙前役为雇役的札子。”

章敦难以苟同的摇了摇头,实在太天真了。纵然司马光曾是王安石的好友,而且司马光的人品也深得王安石的认同,但毕竟自当年割席断交之后已经是十三年过去了,司马光和王安石两人也从不惑之年走过了知天命的年纪,已经是在花甲之年上下,怎么还能以旧时眼光看人?

士别三日当刮目相待,何况十三年?只能看见旧时友人施展抱负,纵然两起两落,但天下由此改变。洛阳的司马光,又怎么可能还能保持住当年的心境?

一名朝臣最宝贵也最能有所成就的十余年时光,消耗在洛阳城的故纸堆中,对于一名有能力、有想法的士大夫来说,这个仇怨比杀了他还要重上许多。就章敦而言,生不能就五鼎食,死亦要以五鼎烹,在故纸堆中消磨时间,他宁死也是不会甘心的。

看得出章敦的不以为然,王安石又暗暗一叹,强打起精神:“现在说这些也有些多了。说不定天子吉人天相,很快就能康复。”

说罢,还轻笑了两声,只是笑声中只有沉重。

章敦板着脸,天子康复,那是韩冈都不敢保证的事啊。纵是药王弟子,也只敢说不日能开口而已。何况医者在病家面前讳言病、死二事,就算是病人活不过冬天,也只会曲言道若是到了春天,便不会有大碍了。

王安石和章敦两人——甚至包括王旁——对医理都有所了解。中风的后遗症既然到了瘫痪和失语的程度,那么病人本身,基本上也就能拖上一年半载而已。而且延安郡王才五岁,就算天子还能再多活两年,也撑不到赵佣长大成人,太后垂帘便是必然。

王安石和章敦都沉默了下去,王旁忽而开口道:“玉昆的事,要不要通知一下妹妹那边?”

“应该会有人通知的。”章敦说着,想了想,又道,“不过还是再派个人去说一下比较好。”

……………………

送走了前来报信的一名小黄门,得知韩冈将会留宿宫中,王旖便是一副忧心忡忡的模样。

韩云娘从通内院的小门中走进厅来,小心翼翼的轻声问道,“姐姐,官人不会有事吧?”

王旖有些勉强的笑了笑:“皇后和贤妃求官人还来不及,怎么会有事?”

“但……”韩云娘仍觉得王旖的脸色不对,还想追问,却被周南扯了一下衣袖。

出身自教坊司的周南,眼光只比王旖差一点。知道韩冈面临的到底是什么样的危局。插手进皇嗣传承,就像是走上了悬崖峭壁,只要一步错了,便是落入千丈深渊的结果。

周南松开扯着云娘衣袖的手指,玉容苍白。若是二大王登基,或许她自尽才是最好的结局。

“没事的,官人一定有办法。”严素心小声的安慰着周南。

王旖回头看了一眼,没有说话。她已经将韩冈留宿宫中的消息遣人传去了城南驿,现在需要做的,就是等丈夫的消息了。

