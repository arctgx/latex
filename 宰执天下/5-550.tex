\section{第44章 秀色须待十年培(六)}

臧樟火烧火燎的带着人赶出了军器监,向炮弹飞出去的方向追了下去。

十几人在内城的街道上狂奔,平平静静的街道上顿时鸡飞狗跳,一路上人人侧目。

内城贴近皇城,管束极严,人数略多的聚会就有被查问的可能,像这样的一群人在路上乱跑,才过了两条街,就被人堵上了。

臧樟火烧眉毛,急得大叫,“别挡路,我是军器监的臧樟!”

来堵路的是一个老办事的巡官,和和气气的说道,“既然是臧官人,那就好说了。只是天子脚下有规矩的,若有急事,一两个人赶路也不会拦着,这么多人,小的们也为难。要不臧官人你说一说,到底出了什么事,俺们也可以帮帮忙,实在不行俺回去也好交代。”

臧樟给慢条斯理的一番话磨得心头火发,跳着脚:“火器局的东西飞出去了,丢了你抵命?!”

跟着臧樟的有韩冈和章惇的亲随,都是穿了朱衣,上前来道:“这是奉了章枢密和韩宣徽的口令!事关军器监机密,不能外泄。”

几个上来查问的都只是公事所的逻卒,敢上来查问一名官员,也是知道这是人来人往的大街上,没那个青袍小官敢发作。这时一听到事关章惇和韩冈,立刻就不敢多说了。

但那个巡官心中还有些疑问,眼睛在臧樟和那两名亲随身上转来转去。

臧樟急得直跺脚,怒气冲冲:“不信就跟上来。也用得着你们!”

说罢,便绕过了拦路的逻卒,继续往前跑。那个巡官见状,知道事情不会小了,派了一人回去报信,自己则领着人追了上去。

臧樟心急如焚,边跑边上火。

火炮真正意义上的首次试射,便把炮弹打出了军器监的围墙,擦着皇城飞出去了。

虽然韩冈和章惇表面上都没有什么异样,但不需要有多少察言观色的本事,也能看得出来,那两位的心里都在打着鼓。不然何必将自家的亲随给派出来?

东京城内人烟稠密,哪条街上都是人,那么大的一颗铁球飞着出去,保不准就伤到人了。何况这边还是内城,官宦遍地。号称丢块砖头下去,就能砸出个员外郎来。万一撞上了哪家的皇亲国戚,这事可就不好收场了。

怎么就能出这种事?!

臧樟边跑边后悔,早知道就多查看一下,让人将炮口再放低一点就好了。

当初为了防止意外,也为了更好地确认火炮的威力,用来阻挡炮弹的木板是当初的三倍厚度,中间还夹了石棉作为缓冲,再往后面,靠着墙还有一个沙堆,火炮的的威力再大也不可能突破这么多重阻挡。可谁想到炮弹竟然蹭过了木板的上缘飞出去了。

现在他只盼着韩冈说得没错,炮弹只飞出一里,那边是天宁院,砸死几个秃驴真的没什么。只要正好不撞上官宦人家去进香,那就什么事都没有。

离开军器监快一里了,臧樟便把身边的人都派出去询问,方才有没有看到一颗铁球从天上落下来。尤其是天宁院,臧樟直接踹门进去,抓着做主持和监寺的老和尚翻来覆去的问,也一样没有消息。回头盯了院中几个细皮嫩肉的小沙弥几眼,臧樟啐了一口,阴着脸出门。

既然这边没有消息,那就只能继续往前去去查问。臧樟一路向前,沿着大街小巷挨个问过去。差不多已经了解到了所有内情,那名巡官也跟着一起去询问路人和住户。

随着摇头的人越来越多,臧樟的心也一点点的沉了下去,再往前,可都是高官显宦居住的崇仁、保和诸坊了。只是侥幸之心也升了起来,万一砸到了一个空地上,没人注意到,也不是不可能,

过了一条大街,保和坊就在面前。

进了保和坊大街,就看见前面拥着一群人,都是脸对着路边的一间大宅院,围墙占了半条街去。

再仔细看看,一群人围观的那一间大宅院的上方,还有很明显的灰尘没有散去。

“怎么了?出了什么事?”臧樟心头一跳,忙上前,扯住了一个围观者问道。

那名围观者显得极为兴奋,眉飞色舞:“郭太尉家的房子塌下来了,还是正堂,也不知怎么就塌了。”

臧樟的心当即就咯噔一下,感觉不妙了。

旁边一个人插话道,“照我说,肯定是空的时间长了,永宁郡公搬出去后,十几年都没人住了。”

“不才修过吗?前些天木料、砖瓦还运了好些车过来。”

“开封府修的。”

“哦……”倒是没话说了。

“谁家?!”臧樟方才只听到郭太尉,魂差点没飞出去,这时候方才回醒过来,颤声问:“哪个郭太尉?”

“还能有几个郭太尉?”那个围观者很是不屑的横了臧樟一眼,扯回了袖子,冷哼着,“就是才从河北回来的郭太尉!”

完了!

臧樟的心彻底冷了,一阵天旋地转。

哪里来的三百六十步,是五百步好不好!从军器监到保和坊,整整一里半啊。

军器监火炮实验,炮弹飞了一里半,把郭太尉家的房子给砸了,真是好笑话。

臧樟笑不出来,章惇、韩冈也就罚铜,方兴有后台,最多降官,他这个没后台又是工匠出身的军器监丞,可就要承担最大的责任了。

他怎么这么倒霉?!

……………………

郭逵正铁青着一张脸。

望着眼前塌了半边的正堂,他的脸色,跟当曰收到河东雁门失陷的消息时,也差不了太多了。

今天郭逵是切身体会到了,什么叫做人在家中坐,祸从天上来。

青天白曰,好端端的天上掉下块陨石。而且好死不死,偏偏落到了他郭逵家的正堂上。就是正堂也没什么了,却还把房子给砸塌了半边。

如果只是破个洞,那还好遮掩,直接让人收拾了,夜里找个瓦匠给补上。但现在事情闹大发了。房倒屋塌,无论如何都遮掩不过去。司阍方才来报,外面已经聚了不少看热闹的闲人,也许再过片刻,宫里就要派人了问询了。

若是天降陨石的消息传出去,那些该死的钦天监的天文官肯定有话说了。这陨石早不落迟不落,偏偏赶在郭太尉住进来的时候落下来,自然是上天的警告,要警惕郭逵。

当年狄青家的狗头长角还只是市井传言,现在陨石砸下来可是千真万确。家里有好几个打扫前院的仆役,亲眼看见一个东西从天而降撞到了正堂屋顶上,然后房子就塌了。

郭逵也暗喊侥幸,正堂若是在平时,只是打扫,并不进人。但自己才搬进来,迎来送往,倒是少不了人进人出。也只是今天下午,正好没有需要出面接待的贵客,安排在正堂服侍的家人就被叫去打扫庭院——家中人口少,现在只能是一个人做两个人的活。

若是方才有人在里面的时候陨石落下,不知要落下几条人命。

“大人,大人,儿子在正堂里面找到了这个东西。”郭忠义灰头土脸的从正堂里面钻了出来,大呼小叫的。

郭逵的长子郭忠孝并不在家中,次子郭忠义,就被郭逵派进去查看,有什么不能见人的东西立刻给收拾起来。

陨石砸到自家头上的消息肯定要封锁,罪名可以推到修缮寨子的开封府头上。等过些年,消息泄露出去,时过境迁,也就死无对证了。

郭忠义小跑着过来,手上托着个圆滚滚的黑球。看郭忠义的动作,分量不会很轻。

“这是什么?”郭逵问道。

“应该是个铁球。儿子在正堂里面找到的,寻思着家里没这东西,房梁上要放也不是用铁球的。”

正堂没有完全塌下来。郭忠义进去看看情况,就发现了这个铁球。又不是镇屋子放几枚压胜倒是有,可不会用到铁球。直觉上就觉得这很可能是将正堂砸垮的罪魁祸首,忙用手巾托了,赶着送到郭逵的面前来。

郭逵从儿子接过这个铁球,手顿时一沉。挺重的,怕不有十来斤。颜色黑黝黝的,还有石膏、石灰之类的黏在上面。

郭忠义有担心,也有好奇,问着郭逵,“大人,这会不会是陨铁?”

天上陨铁制成的刀剑,都是世间的重宝。传闻很多,但亲眼看见过的人很少。据说都是能吹毛断发、断金截玉。要是当真是陨铁,打造成刀剑,必然是天下闻名的神兵利器。郭家是将门世家,藏兵当然不少,但陨铁制作的兵器可是一件没有。若能有一件镇压百兵,那也不错。

太圆了。

郭逵心中有些疑惑,天上掉下来的陨铁难道都是圆得跟球一样吗?感觉应该像是矿石的样子。不过他也没见过陨铁,也不敢就这么否认。

郭逵用手擦了擦,拂去了上面的灰土,然后动作立刻就定住了。

的确不是陨石,不过也不是陨铁。

“韩冈!”郭逵猛地一声大吼,一时间怒发冲冠,翻手狠狠地将铁球砸到了地上。

咚的一声响,近在咫尺的郭忠义吓了一跳。偷眼大步往外走的父亲,他小心的将铁球捡起来,擦了一擦,定睛一看,顿时全明白了。

铁球上面正刻着几行小字:

元丰四年,八月甲戌,上工吕文,监造方兴,判军器监黄。

不是天上掉下来的,是军器监丢过来的!

郭忠义目光追着几乎要被怒火烧起来的父亲,心道:‘难怪!’

