\section{第20章 冥冥鬼神有也无(24)}

人头大小的石弹,突然从天而落,一头撞上了小型战船略显单薄的桅杆上。

吱吱呀呀的木料呻吟声中,桅杆奇迹一般的没有折断,只是被擦出了一个偌大的凹坑。可沉重的石弹又反弹到了站甲板上的一名水手头上。扑的一声轻响,方才还能说能笑的一个人,他的头颅就像是被砸碎了的南瓜,迸出来了在黑夜下看不出本来颜色的一地稀烂的瓤儿来。

石弹打着旋儿,在甲板上滚动着,甲板上一片混乱。但阮陶根本就无暇理会,瞪得铜铃一般的双眼里,写满了难以置信。

河道两岸没有完工的望楼,原本只是夜幕下凝固的黑影,可此时却随着河面上闪起的火光,也几乎在同时亮了起来。一团团跳跃的火光,从河口一直延伸过来,将一座座望楼下数以十计的活动的身影,全都投射到了阮陶和所有船上水手们的眼底。

那根本不是什么望楼!

高髙耸立的台座,以及架在上面的一头短一头长的长竿,阮陶认不出那是什么东西,但他可以肯定,那绝对不是望楼……

……而是陷阱!

阮陶回头,一个字一个字向着将船队带入陷阱的细作,倾倒着心中的愤怒,“你的眼睛究竟长在哪里?”

“小人冤枉啊,小人真的是冤枉啊!前日看的时候,的确就是望……”

细作的话声未落,面色阴寒的阮陶已经挥臂而下。映着火红光芒的佩刀,在细作的脸上砍出一片血光。细作前一刻还在扯着水师统帅的衣襟,哭诉着自己的无辜,下一刻就立刻没了声息,倒在甲板上抽搐着。

用力将脚边垂死的罪魁祸首踢开,阮陶心头怒火依然难消,就算杀光所有派在北岸的瞎了眼的细作,也挽回不了今次的败局了。

一枚石弹又正对着战船飞了过来,咚的一声巨响,船只轻颤之下,石弹深深的嵌入了甲板之中。船上混乱的叫喊声里,夹杂着阮陶百思难解的疑问,“那到底是什么东西?”

交趾水师的统帅,甚至有那么一瞬间,都忘记了迫在眉睫的危机,只知凝视着数十步外,火光缭绕的巨型战具:“那到底是什么东西?”

原来这就是霹雳砲!

黄元也不知是兴奋,还是感到骇异,反正他现在的身上起满了鸡皮疙瘩,听到一枚石弹带着隐隐的呼啸飞往火焰熊熊的河面,身子都忍不住在轻轻颤抖。

只是用木料、石头和一些铁件搭起来的架子,竟然能像丢一颗石子一般,将几十斤重的石块,抛到了六七十步外的河面上,精准的命中交趾人的战船。

见识过了神臂弓的力量,见识过了斩马刀的锋利,见识过了板甲的坚固,见识过了飞船的神奇,眼下又在亲眼见证着霹雳砲的威力。宋军装备的每一件利器,都让出身自广源州的黄元,感到庆幸、害怕还有兴奋。

看起来只是望楼,谁能想到那是能让战船和城池都灰飞烟灭的军国重器!

对于船场中的能工巧匠而言,只要有木头,能打造的可不仅仅是船只,霹雳砲只要不将配重的石块装起来,即便有奸细混进来,也认不出来那高高竖起来的木架究竟是个什么玩意儿。

而所谓的望楼,就是一圈竹架围着的霹雳砲,当然永远不会完工。霹雳砲只要抛竿竖起,再用竹子在外面一架,远远地也没人分辨清楚。

伪装的望楼所在的位置也是精心选定,正好卡着河口至船厂的河道上的几处要点。除非交趾人敢于冲进满是泥沼的芦苇荡中,否则他们能泊船的位置,也就那么寥寥几处。

前几天看起来像是在消极怠工,拖延望楼建造进度的士兵,这时候似乎是要清洗交趾细作对他们的污蔑,比任何人都要卖力气。梢竿刚刚嗖的扬起,将石块远远抛射出去,他们就立刻拉下梢竿顶端绳索,将新的石弹装填上去。

河面上流淌的火焰,并不能将坚实厚重的战船给点燃,但亮起的火光,已经将一艘艘战船的轮廓勾勒在看不见月亮的暗夜之中。

“盯着最后面的船,打最后面的那一艘!”

爆发式的吼叫,不知是出自岸边的谁人口中,但竟然随着夜风,模模糊糊的传到了听得懂汉人官话的阮陶耳里。先是一愣,然后看见一枚枚石弹当真集中到了最后一艘船只上,死亡近在眼前的危机感,立刻从阮陶的心中溃堤般的涌了出来。

这条支流的河道浅窄,要是最后面那艘战船被石弹摧毁,那就谁都别想跑出去了。

“快掉头!”

阮陶已经无心去记挂冲入船场中的李常宪,只看眼下的陷阱,就能猜想得到那一座船场根本就是龙潭虎穴,李常杰的弟弟不可能出来了。

其实不用阮陶吩咐,他的船队中,所有的战船都在受到石弹洗礼之后,立刻选择了撤退,在燃烧的河道中吃力的掉头。

“换石子!”

又是一声吼叫响了起来。

黄元捂着耳朵,就看见让自己耳朵嗡嗡直响的砲兵指挥使,将一个铁皮号角从嘴边拿开。然后身边的霹雳砲上就立刻换上了用网兜包起来的石子。

一声哨响,绞着绳索的士兵放开了手,配了重物的稍竿猛然一晃,一包包碎石腾飞在天空。并不结实、又没有收口的网兜在空中分解,从河滩上捡来的鹅卵石如暴雨一般落下,河面上猛然间暴起的惨叫声,让黄元心头都为之一颤。

甲板上的水手正经受着鹅卵石的洗礼。没有头盔、没有甲胄,正在摇橹、撑杆、挥桨的交趾水兵,在被河水打磨得光滑圆润的石子敲打下,像块豆腐一般脆弱。像雨点一般落下的石头,寻常根本就没法儿想像,在这猛烈的狂风暴雨中,水手们在甲板上打着滚,许多人都是头破血流,甚至有人额头上挨得重了,昏厥过去都快没了气息。

但仅存的水手们还是在咬牙坚持着,这个时候再不拼命,当真只有死路一条。戴起防雨的斗笠,披挂上同样用来防雨的蓑衣,在鹅卵石掀起的疾风暴雨中,船队中大半船只,虽是艰难地,却还是成功的调转了船头。

可是拖在船队尾部的战船,并没有转了过来,只转到了一半,就停了下来,横挡在所有战船的面前。

“该死!”阮陶一声咒骂,但最后那一艘战船上已经没人还能听到的话,他们受到最多的攻击,伤亡也是最大,根本无力再操纵船只。而且并不止一条,前前后后有四五条船都是如此。

“从旁边绕过去!”

阮陶一马当先,他的座舰最为灵活,在他的命令下,直接就绕过挡在前面的船只,偏向岸边划过去。

这一条支流,真正被确认可以航行的只有中间的水道,越往岸边则是越浅,但几条船挡在前面,这时候也只能借用。

少了几十条小船,又有数百人上岸,现在战船吃水已经浅了许多。尽管还有搁浅的可能,可不管怎么说,也只能赌上一把。船场中这时忽然响起的胜利的呼喊,更是坚定了阮陶的决心。

飞过来的石弹更加密集,时时刻刻都有着石块溅落下来,而船身也在震颤着,船底龙骨擦着河底的震动一直传到阮陶的脚底,但他的座舰依然还是在坚持着向前滑行。

咚咚的声响,是石子砸到甲板上的声音,而刷刷的拍击声,则是被碾开的芦苇在拍着船帮。提心吊胆的阮陶都恨不得能捂住耳朵,但下一刻,眼前忽然开阔,闪烁着星光的富良江终于出现在了他的眼前。

劫后余生的狂喜让他差点支持不住发软的双脚,可当他再回头张望,却发现只有三艘战船跟了出来。

“尽量多留几艘船下来!”韩冈在开战之前的吩咐,还深深的刻在黄元的脑海里。

他亲眼看着砲手们师如何拼命的拉索放索,将石弹石子投射出去,手脚一刻不停,到最后已是几近疯狂,十一艘来袭的战船,仅仅有区区四艘消失在富良江上的晨雾中。

正月初二的早晨,章惇和韩冈得到了写满了胜利的捷报。

匆匆一扫之后,章惇将捷报递给韩冈,笑道:“七艘。”

来袭的十一艘战船被留下了大半,有四艘停在河道中,另有三艘慌不择路下,搁浅沙滩,其他也是带伤。对面交趾水师的实力,整整下降了三分之一。但被俘的七艘战船对交趾的意义,绝不仅仅是水师损失的三分之一。

韩冈低头看过捷报,说不算大喜,但也足感欣慰,不枉这一次的一番辛苦,“可惜没能都留下来。”

“这可是没本钱的买卖!”章惇放声大笑着,“不用贪求太多了。”

韩冈莞尔一笑:“这一下过江可就方便了。”

正月初五,修理好俘获的战船,宋军和溪洞蛮部联军同时展开了渡江的行动。

两百里宽的江面上,万舟齐发。数以千百计的木筏,满载着数万蛮部精锐横渡江面。而官军拥有了七艘小型的战船,凭借着神臂弓等利器,在江面上轻松击败了交趾的水师,摧毁了大部分的舟楫,将战旗插到了富良江南岸。

交趾国的都城,也就在宋军的眼前。

