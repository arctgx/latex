\section{第八章 朔吹号寒欲争锋(七)}

日后看结果?

“这是买赌券吗?!买了之后再看结果?老夫当初推行新法,什么时候不是战战兢兢,遣人分至各路体量,唯恐出上半点差错。身居相位,做事难道是要一翻两瞪眼?!”

“想不到岳父也知赌博?”

“玉昆,老夫不是在跟你说笑!”

王安石盯着韩冈,脸上带了些许怒意。

韩冈的话实在太冲了一点,连尊卑都不讲究了——这是在说走着瞧吗?

“小婿也不是说笑。”韩冈依然在笑,“岳父说黄廉好,小婿说曹诵好,既然相持不下,小婿也只能说等日后看结果了。”

“火器局的事,难道黄廉做得不好?”

“做得很好,所以应该升任了,留下曹诵配合王居卿。”

“王居卿非是适任之人。”

王安石耐着性子跟韩冈说话,换作其他人,何曾会让他费心费力的解释、辩驳。

“但在韩冈看来,王居卿在军器监会做得更好。若王居卿就任军器监,韩冈可是有把握让军器监拿出让辽人望尘莫及的新式武器来。平章若是不信,韩冈也只能说等日后再来看了。”

……………………

韩冈走出王安石的书房有些急促,几乎就是被赶出来的。

“玉昆。”

王旁迎了上来。

韩冈和王安石在书房中说话,没有让他旁听。王旁去里面见过了妹妹和外甥,回头就看见韩冈从书房中里出来。

“到底怎么了?”

甚至不用进书房去,看韩冈的样子就知道是不欢而散。

“仲元,回头帮忙劝劝岳父,消消火。朝堂上的事,没必要带到家里来。”

王旁皱起了眉头,跟他的父亲方才在韩冈面前的模样真有几分神似:“玉昆你到底跟家父争个什么?”

“仲元,你可听说过大虫巡山。”

王旁点点头。

民间山中有山大王的说法,所以有俗话是山中无老虎,猴子称大王。王旁小时候听多了乳母说的故事,很多都是山大王吃了某个不听话的小孩子,可谓是黑色的童年记忆

大虫之所以会被称为山大王,就是因为会没事在山林里绕着,所以许多穿越山中的道路,冷不丁的就会冒出一只吃人的大虫。

“其实大虫所谓的巡山,不是想巡守地盘,而是用尿划定猎食的范围,通过尿液的气味,来警告同类和其他猛兽,莫要侵犯。”

说大虫,王旁当然不清楚,如果是说家里养的狗,王旁就明白了。狗在树下撒尿,是人都会见过。

但王旁笑不出来。

看似是闲聊时的趣闻,韩冈就是明说了,他是来跟王安石划分势力范围的,只是用作比喻的例子太过粗俗。

“这……”

他甚至觉得无话可说。

韩冈明确的说要与新党划分势力范围,要在朝堂上占下一片地盘,王安石要是能答应韩冈,就是白日见鬼。

韩冈陪着王旁在院中说话,“虽然说大虫这么做,看起来腌臜了一点。但这样的提醒,就避免了与同类或其他猛兽的冲突。两只猛兽打起来,非死即伤,对哪边都不是好事。”

王旁明白韩冈的意思。

现在各自退让一步,还能留些情分。若是变成了牛李党争,或是之前的新旧党争,可就是不死不休了。

只是要比年纪,王安石肯定不比不上韩冈。真要将情分消磨尽了,日后对自家的妹妹也没好处,那还有几十年的夫妻要做呢!

“愚兄明白了。”王旁点了点头,声音却有些发沉。

韩冈叹了一口气,化作一片白雾在初春的夜风中散了开去。

他不知道王旁能不能劝得住王安石,但总算是尽了一份人事,不过另外一个长辈的情况就更麻烦。

……………………

因为程颢是韩冈的半个老师,又曾为帝师,开封府对其还保持着一定程度的尊敬。

不过仅仅是抄走了所有学生与刑恕往来的信件,就让泰半程门弟子都慌了神。

还在坚持讲学的程颢座前,每天坚持过来听讲的学生越来越少,时至今日,就只剩下二三十人。

程颢苦中作乐,说圣门七十二贤,孔子三千学生中,贤人也只有七十二。而他这里就有二十多,比不上圣人,却也足够自豪了。

但这样的话,只是自我解嘲,改变不了现状。

在很多人眼中,程颢的门下教出了一个叛逆。

幸好昨日殿上传信来,将开封府中所有因为蔡逆一案被搜去的信件全都烧光,终于让程门上下都安了心。

“真是兴衰一瞬间啊。”周文璞远远望着程颢讲学的寓所大门,“两个月前,那里可是夜不闭户,士子出入不绝。”

宗泽摇头:“谁让出了一个刑恕?”

“不仅仅是刑恕的问题。开门受徒,贤与不肖,皆入门来。是道学本身的问题。”周文璞对宗泽道,“汝霖应该听说过‘物尽天择,适者生存’这八个字吧。”

宗泽怎么可能没听说过,当初就是韩冈以这八个字来辨析华夷之分,并将之解释为自然之道。

可如今很多儒者都在讨论这八个字的内涵,试图映证到人事中来。其中极端的,甚至拿着这两句话来解释世间万象。

“这也算是适者生存?”

“怎么不算?”周文璞道,“远的不说,就说几日后的大比。五千贡生中才得选出四百人,这是不是适者生存?而这些贡生,无一不是从地方的解试中杀出来的,哪一个脚底下没有踩着十几二十同列?再说为官,天下文武入流品者几近三万,可升朝者几何?能入两府又有几位?”

宗泽眉头就皱起来了,周文璞的话,正是那种极端的说法。但这种说法,偏偏可以与事实相映证。

尤其是在官员和考生中,这样的感触最深。文武百官,以及希望成为官僚的士人,想要一步步走上去,都要踩着更多人的脑袋。不能适应的,全都被淘汰了。

“既然如此,新法旧法也是一般喽?”

“当然。旧法也有是新法的时候,新法施行多少年后会变成旧法,终有不合人意被人替换。”

“应该不会太早吧?”宗泽笑道。周文璞的偏向,从他买狗做试验后,就越来越明显了。

“不在其位,不谋其政,新法何时被替换,不是我等能说的,但经史传注,人人可说。对经史的了解,文璞于汝霖是望尘莫及。想必汝霖不会不清楚,自孔子之后,五经的注疏到底变过了几次?”

宗泽叹了口气。

新旧党争或许已成过去,随着韩冈走入政事堂,这士人之中,新学和气学争论可就愈演愈烈了。

远远地听到了喝道的声音,让川流不息的行人车马有了一个短暂的停滞。

宗泽抬起头,又是哪家重臣在前面堵住了道路?

……………………

韩冈留了儿女在外公家住上几日,与王旖先回了家。

留下儿女,主要是想缓和一下与王安石的关系。毕竟是亲家,总不能变成冤家。

不过刚刚回到家,就在书房中看见从政事堂送来的急报。

像这样连夜送到宰辅家中的急报一般都是军情,这一次也不例外。

是有关辽军在日本的战报,还有求救的文书。

冬天去日本的海路不好走。海上风浪大,信使传递消息困难。尽管在辽人侵略高丽和日本之后,朝堂上下都在说海船需要加强研制,可是缓不济急,更好的海船哪是朝廷说一句,就能变出来的?

眼下虽是开春,也是信使冒死通过了风急浪高的大海,才将海岛上的消息,送到了大陆。

韩冈展开用火漆封缄好的公文,看了一下,就开始摇头。

辽国在日本国中的侵略速度太快了,而日本军队的表现也太无能了一点。

日本国中无时不在的地震给辽军带来的干扰,都比倭人军队更大一些。

三十年的和平,让宋军给党项人打得跟狗一样。而日本的和平,持续了三百年。这期间,日本国内也有内战,但低水平的内乱,完全无助于对外战斗力的提高。

无论技术、装备、战术,都完全落后于世界。

而且还有传说,说日本国内曾经有过禁肉令,许多男子终身未吃肉食。从个人战斗力上,倭人也远远无法与以肉和奶养大的辽人。

就在一个月前,平安京被辽军攻下,整座城池被彻底焚毁。

三百年的时光方才积累起来的繁华,在火焰中化为灰烬。

曾经作过侵略者,韩冈明白。屠城不是残暴,其目的是毁灭。毁灭日本的中心,缺乏领导者的国家,很快就会在降伏。

如果能够将所有识字的领导层一并清扫光,日本作为一个独立国家,将不复存在。

要是让韩冈给辽人支招,大概就是他在交趾所作所为的翻版。

若是辽国顺利的吞并了日本,或许再过些年,宋军在面对契丹铁骑之外,也将会面对以倭人和高丽人为主体的步兵。

不过韩冈现在并不是太在意。

辽国对日本的入侵,至少能让宋辽边境太平上几年。

大宋也正在资助高丽和日本的反抗军,拖延辽人彻底控制高丽、日本的速度。

而最重要的,在耽罗岛上,耽罗星主已经向大宋献上了土地,请求归附。

虽然对不起流亡于此的高丽君臣。但国与国之间的关系,必须要有相近的实力,才能够得到尊重。高丽在灭亡前,由于中国需要他们牵制辽国,所以愿意不惜代价结好他们

有了耽罗岛这个海外领地,大宋对黄海和东海的控制就上了一个台阶。

宋辽之间的决战,不仅仅是在陆地,未来也可能是在海上。
