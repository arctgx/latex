\section{第15章 三箭出奇绝后患(下)}

再一次沐浴在箭雨中,无法再承受更大的伤亡,劫道的蕃贼不得不撤离战场。这些蕃贼虽是勇武,但架不住关西男儿更为犀利的强弓劲弩。

“贼人前后出战,总计超过八十,而丢下来的尸首二十七具,有十一人是王军将的战果。至于俘虏,则有四名。”

战后,韩冈很快的计点出战果,点出几个看起来有些胆量的民伕,让他们去割下贼人的首级,以便过后请功。经此一战,韩冈在民伕眼中,已是让人又敬又畏的秀才公。

虽然韩冈曾说埋伏在身后的蕃贼人数不多,但最后冲出来的却不在前方来敌之下,根本是句安抚人心的谎言。但靠着他的强硬和支撑,民伕们仅用七人受伤,其中一人伤重的代价,便获得了如此大的战果。

可没人注意到,韩冈的背后衣襟早已湿透,第一次面临战阵,又要作为全军主心骨来指挥,他久病初愈、沉疴刚痊的身体差点就要虚脱。

‘幸好有个王舜臣。’韩冈为自己庆幸,若不是王舜臣独自在前方奋战,若不是王舜臣箭术出神入化。有内忧,有外患,这一仗他多半小命不保。

但韩冈的作用并不比王舜臣稍差,尽管在战斗过程中他完全没有进行任何具体战术的指派,但有他站在身后,民伕们表现出来的战力,却远胜过这群蓄势已久的蕃贼。

这全是靠着韩冈的冷静,带给所有人的士气。士气,韩冈现在才体会到,在古代战争中,士气究竟有多么关键和重要。

王舜臣坐在骡车上,处理着自己肩头的箭疮,脸上的神色则有些不甘心。虽然他一人对抗数十倍的敌人,表现最为亮眼。但最终扭转战局的,还是靠了民伕们的努力,以及韩冈的指挥。

当时王舜臣甚至已经被攻上来的蕃贼逼得站不住脚,但一阵适时而来的箭雨,将贼人尽数射散。不过三五轮齐射,分作前后两波来袭的蕃贼,丢下了近半的自家人,向树木深处退去。

看着同样坐在骡车上休息的韩冈,王舜臣的眼中也多了几分敬重。不仅仅是因为被韩冈可圈可点的战时指挥所救,同时也被韩冈的狠辣和果决所折服。

“这两个鸟货也真背运,碰上了韩秀才你。”虽然心中多了敬重,但王舜臣还是改不了满口跑鸟的习惯,口气也不甚好,“被一箭射死,连个喊冤的地方也没有。”

“不听号令,乱我军心。只能拿他们俩杀鸡儆猴!”

“不知吓得哪家的猴子?”王舜臣失笑。他看似粗豪,心思却也不笨。

韩冈呵呵笑了两声,也不作答,起身走到河边,将怀中的一个小包丢进渭水。薛廿八和董超死了,从军器库中带出的东西也便用不上,留在身上,保不准什么时候就反害了自己。

从河边转回,他却道:“今次来的贼人却也不好惹,死了三成才退,加上受伤后还能动的,伤亡都过半了!”

“都是在关西厮杀了几百年,能耐差点的,早就被灭族了。又是劫道,留不得活口,不得不拼命,有什么好奇怪的?”王舜臣一边说着,一边用匕首挑着嵌入肩膀皮肉中的箭头,突然倒抽一口冷气,“日他鸟的,这一箭够狠!”

韩冈连忙上去检查王舜臣的伤口。长箭被拔出来后,血水直往外冒,还好这一箭并没伤到筋骨,仅是貌似严重的皮外伤。用浓盐水清洗伤口并止血,缝合起来再包扎好应该就没事了。只是韩冈只有理论知识,却毫无操作经验,而且这里是荒郊野地,没有煮沸消毒,如何进行外科手术?

但韩冈再看看王舜臣的伤口,因为剔出箭头的动作过大,使得伤口外翻得厉害,还在向外渗着血。现在王舜臣看着还有精神,但等会儿就不见得了。如今这等情形,只能先急就章的草草处理一下,幸亏现在是冬天,应该不会容易感染。

“有谁会做针线活的?”韩冈大声问道。他连纽扣都不会缝,想在活人身上绣花,会绣出人命来的。但这么些民伕中,挑出个会做针线活的人来,肯定不难。

此时的布匹质量普遍不高,尤其是民间下层常用来做衣服的紬绢和麻布,从来都不是以结实耐用而著称。要不然,军中也不可能一年给士兵们发下四匹、六匹、八匹的紬绢裁衣服。棉布倒是结实,但北宋的棉花才刚刚推广种植,纺出来的棉布称为吉贝布,价格跟蜀锦差不多,没个几千几万贯的身家谁穿得起?

平常百姓只能穿着容易损坏的紬绢和麻布衣服。常坏的衣服当然要常补,有分教:白天走四方,夜中补裤裆。常年在外,身边没个女人的男人,不会针线活的还真不多。

正如韩冈所料,一个四十上下的矮个民伕出来自荐道:“小的十几岁时曾在裁缝铺做过学徒,虽然没能出师,但针线活还是能来上几手。”

韩冈看了看他身上的衣服,针脚缝得细细密密,“衣服是自家做的?还是浑家做的?”

“自家。俺还没娶浑家。”

在一个茶壶能合理合法的占据几十个茶杯的年代,下层百姓中的光棍为数实在不少。韩冈也不惊奇:“好,就让朱中你来缝。”

不仅仅是朱中,其他民伕的姓名韩冈都能一口报出来。多认识一个人,就是多了一份资源。就算是微不足道的民伕,可谁也说不准,他们什么时候就能派上用场。

韩冈对朱中附耳低语了几句,王舜臣便看见他领着朱中,捏了一根折弯了的缝衣针走过来。“你这是作甚?”

“把你的伤口缝起来!”韩冈解释道。

“缝个鸟!”王舜臣惊叫,胆魄过人的王军将难得有惊慌失措的时候,“没听说皮肉能用针线缝的。”

“三国时,名医华佗可是把人的肚子剖开,割下瘤子又缝起来的。只缝个小伤口不算什么!”韩冈看着王舜臣的惊惶甚至觉得有些有趣,“堂堂一个军将,刀砍都不怕,害怕一根细针?传扬出去,可不是多光彩。”

“……那你先拿别人练练手,再来给洒家治。”

韩冈考虑了一下,点了点头,的确这样才妥当。在一名被射中了大腿的伤员身边,第一次上阵的朱中,小心翼翼的用针线将伤口缝合。几个人死死按着伤员,让他不得动弹,嘴里也塞进了手巾,让他不会咬到舌头。伤口中箭头早被取出,又化了些盐水来清洗,只再用针线缝起来,包扎好,一切手续便告结束。

朱中应是第一次上阵,但看起来他飞针走线的手段甚为娴熟,几下子又帮着一名伤员缝合了伤口。韩冈看着生奇,再一细问,才知朱中的缝合技术是在被砍了脑袋的死囚的脖子上练出来的,半吊子的裁缝工作不好找,将死囚的脑袋缝回脖子上,也算是一笔养家糊口的外快。

“该洒家了,快点动手。”王舜臣催促道,看了一阵,也不觉得有多可怕了,而且在众人面前,他也不肯露怯。

示意朱中换上一根新针,韩冈嘱咐王舜臣道:“应该会有点痛,但再痛也不能乱动。若是有麻沸散就好了,一包药喝下去,只要药性未退,天塌了也醒不过来。”

“世上哪有这等药!?”王舜臣绝不相信。

水浒传里就有!韩冈笑了笑,道:“如今是没有,你且忍一忍罢。”

“尽管缝便是了,爷爷若叫一声痛,往后就不是爷爷,是婆婆!”

朱中已将从一块干净的布匹上拆下来的一根麻线穿入针鼻,正等着韩冈的命令。韩冈对着他点了点头,朱中也不犹豫,当即下手。只是钢针刚落,王舜臣便是猛的全身一颤。

“痛不痛?!”

“痛?!”王舜臣龇牙咧嘴得痛出一身冷汗,但依然不松口,“是痛快啊!日死他鸟的,好痛快!!”

不仅仅是朱中一人之力,在另外一边,韩冈也指挥着几个伶俐一点的民伕,一起动手处理伤情。

把最后一名伤员的伤口处理好,韩冈已是满头大汗。他并非医生,连一点医术都不通,但止血,清洗伤口和包扎这几项,他还是会做一点点。

王舜臣的左臂伤口已经给缝合好,并没有缝死,按照韩冈的意见,留个了口子好排脓。由于没伤到主血管,流出的血也不算多。

伤口刚处理好,王舜臣便生龙活虎起来。他右手拎着铁简,走到了四名俘虏面前:“说,你们是那个部族的,又是谁人通得消息。说明白了爷爷就不杀你。”秦州的蕃人都是跟汉人混居了几百年,也不愁他们听不懂汉话。

被问话的俘虏,脾气看起来甚硬,扭过头去,丝毫不加理会。

王舜臣可能是学了韩冈的行事,也不多话,挥起铁简便照头抡去,噗的一声闷响,打了个满地桃花开。他若无其事的甩了甩粘在铁简上红白相间的汁水,又指着第二人。

那人只见铮亮的铁简带着腥风一下指在眼前,脑浆和鲜血一滴滴在鼻子上,直吓得浑身直颤,嘴唇哆嗦着,想说却说不出话来。

王舜臣脾气腾起,眼一瞪,抬手又是一铁简敲瘪了那人天灵盖,两颗眼珠子噗噗迸了出来,连着血淋淋的筋肉,挂在脸上晃晃悠悠。王舜臣双眼再一瞥,在第三个人身上上下一扫,从黄脸被吓成白脸的汉子,不敢有任何耽搁,忙要开口。只是韩冈不知何时走过来,一脚踢在了他的下巴上。

“韩秀才?!”王舜臣又惊又怒。

韩冈摇了摇头:“没必要问了。”

“不把他们背后的陈举挖出来,还等什么时候?!”

“不,他们是听了西贼的蛊惑,入境劫掠,骚扰甘谷后方的的贼人!”

王舜臣眨了眨眼,忽然明白过来,大赞道:“好秀才!”明白了韩冈的用意,他便抬手又是两铁简,正正敲在最后两名俘虏的太阳穴上。

目送又是两人踏上黄泉路,韩冈冷笑道:“直接往陈举身上安罪名根本安不了,谁会信我的话?一旦今天的这些个蕃贼被确认是被西夏收买的奸细,那他们身后的部族也肯定会被揪出来。到那时,陈举与他们之间秘密交易,自然会暴露。”他冲王舜臣挤挤眼,“而且把这些人当成西夏奸细,好歹功劳也能大一点。”

王舜臣有些担心道:“那事情可就要闹大了。”

韩冈轻声而笑:“我只恐事情闹不大!”

ps:韩冈锋芒渐显,得官的手段也在其中了。

今天第二更.求红票,收藏。

