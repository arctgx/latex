\section{第15章 焰上云霄思逐寇(六)}

夜露深重,穿行在野外的草木间,很快全身都被露水给打湿。

紧张中的潜行,对体力消耗极大。浑身上下的水迹,湿透了的衣襟,也有一半是汗水的功劳。终于找到隐蔽的位置,他停了下来。由动至静,他急促的喘着气。只是喘息声压得很低,浅浅的呼吸只将面前的树叶给吹动。

眯起如鹰隼般锐利的双眼,望着不远处的一片营地。营地之中,一支支火炬闪耀着,仿佛星光浮动。一顶顶帐篷,犹如雨后山林空地上窜出的蘑菇,连绵成片。那里是敌军的营帐所在。

天空中的一片浮云移了开去,二月上旬的月光撒了下来。半轮明月此时已升上了天顶。他的手摸上了腰间的水牛号角,紧紧攥着。牛角沾上掌心的汗水,立刻就变得滑腻起来,让人感觉很是难受。

时间差不多了。他心里想着。深深的吸了口气,鼓起胸膛,将号角凑在唇边,然后,用力的吹响。

一道沉稳厚重的调子突然窜起,振动着空气。先是一支号角的独奏,吹响了几个节拍之后,就有另外几支加入了进来。同时鸣响的号角越来越多,很快就变成了一个乐班的合奏,在寂静的夜色中传得很远。

这并不能算是正式的夜袭,只能算是骚扰。两军对垒,针锋相对的安营扎寨。他们之间的交锋,就少不了是要从互相骚扰开始。韩冈不知道李常杰会怎么做,但他会照规矩来。

骚扰和防骚扰,偷袭和反偷袭,都是身为一名合格将领必须精通的科目。李常杰在交趾号为名将,尽管他之前的表现实在是有损名将的称号,可具体到行军守夜,他的表现也不算很差。

就算听到进军号声在四面八方响起,他营中的骚动刚刚升起,就立刻被平息下来。接着从营帐中冲出一队队全身武装的士兵,冲到了营地外侧的栅栏边,面向号角声传来的黑暗,警惕的瞪着双眼。

而就在吹响号角的同时,韩冈派出在外的暗哨也发现了交趾兵潜藏在黑夜中的身影。且早在暗哨发现他们的之前,派去骚扰敌营的士兵,就已经传回了一部贼军来袭的消息。

听到了营地外猝然响起的木笛声,受到信号的李信立刻让营中将一串灯火升到旗杆顶部。

几星火光在营地外闪了一闪,亮了又灭,但很快更旺盛的火焰就跳动了起来。

归仁铺附近,靠着官道两边,是以田地为主。两个多月没有人搭理,里面长满了半人髙的杂草。而更远一点则是一片片零星的小灌木林。设立营地的时候,只将大营周围清理干净,稍远处的草木都留置不动。为了隐蔽身形,交趾兵全都藏身在这一片丰茂的草木之中。

雨水稀少的冬天,让火势蔓延得极快。火焰在营外熊熊燃烧,转眼就随着夜风而扩散开来。今夜的风向不太好,有一部分火焰烧向了宋军的营地,不过被清理干净草木的外围成了防火带,也只有些烟漂了过来。

而更多的火头则是延伸向了草木更为茂盛的地方。当蹿进了一处小灌木林,数百人从火焰中抱头鼠窜出来,狼狈不堪的奋力奔逃,试图躲避将他们烧得焦头烂额的火魔。

相对于韩冈派出去进行骚扰的兵力,这个人数倒也不算少了。尽管用于正面作战人数肯定不足,可拿来偷袭只有三千人的营地,如果没有提防,可以很轻易的踏破大营。

“想不到李常杰这么看不起人。”苏子元望着从火场中跳出来的偷袭者,冷笑着,“以为我们会不做提防?”

“他浪费得起。要是易地而处,我也会忍不住派兵夜袭的。”手上有着一万多人的兵力,用五百人冒点风险,试图换来一次胜利,根本不是大问题。韩冈说着就打了个哈欠,转头对值夜的李信道,“快点回去睡吧,明天可就有得要忙了”

李信点了点头,吩咐了下面的哨兵们不要松懈,要提防交趾兵再来一次,转身也回帐去休息了。下半夜的防务,自有人来负责。

一夜过去,两边都没有受到太多的惊扰。除了归仁铺周围一片着火过后的黑灰色,仿佛什么事也没有发生。

但吃早饭的时候,两军还在打探着对面的虚实,总不能不知敌情的就开始作战。骑兵在两军之间的原野上奔驰,相对于昨日寥寥数十骑,今日则多了许多。人数不足的宋军骑手,只能暂避锋芒。不过他们在避让间,也会趁机

“都是交趾兵,不见广源军的踪迹。”

而且李常杰的将旗就在敌营中飘扬,很明显的就是他在主持着军务。

“看来李常杰是打定主意要挡住我们往邕州城下的路。”

“他将广源军隔在他的身后,我们想派人劝刘纪等人倒戈,恐怕也是无功而返居多。……他做的真是很聪明。”

只有表现出来足够的实力,才能得到广源州剩下的三位蛮帅的投靠。而在两边交流被交趾兵阻隔的情况下,想让刘纪三人看到宋军的实力,也只有击破眼前的李常杰所部。可是要是能正面击破李常杰,广源军到底是投效还是顽抗,就没人放在心上了。

“而且刘永的事也是一个麻烦,要瞒着刘纪也不容易。”

刘纪身为一族首领,不会因为兄弟手足被杀,而与大宋不共戴天,这是领导者最基本的素质。但如果加上一千部众,情况就大不一样了。除非被逼到绝境,否则很难指望他会如同黄金满一样主动投效过来。

“黄洞主,你弃暗投明,万一交趾人恼羞成怒,回军时挥师攻打你部,那将如何是好?”韩冈突然想到一件事,问着黄金满,他可不希望投效之人全家被屠的事情发生,“此事不得不防。”

“多谢运使挂念于心。小人已经派了得力之人,绕道赶回广源,通知族人暂避。交趾不会有余暇去追逐小人的族人,而刘纪他们当也不会帮着交趾人做这等自坏名声的事。”自家的事当然是自家最为关心,黄金满早就想到了,并不用韩冈多虑。

“那就好。”韩冈点点头。

望着正在原野中奔驰交错的骑兵,双方兵力上的差距,只从骑兵上就能看得出来。苏子元皱着眉:“兵力相差甚远,如果要获胜,当然得用奇兵才行。”

韩冈轻笑道:“我们的目的是什么?是凭着三千人将邕州城下的数万敌军击败?……不是啊,让他们撤出邕州就算成功。”

韩冈的首要目的是将贼军从邕州城中给逼出来,这就是胜利。至于歼灭当面的贼军,如果天上掉馅饼,让李常杰犯浑,他不会浪费大好时机。可如果为了一个胜利要冒太多风险,他是不会做的,成功的几率未免小了点。等交趾贼军撤军之后,追逼在后,等着扑上去的机会,这才是一个更轻松也更容易成功的办法。

“做到眼下的这一步差不多已经是极限了。”韩冈道,“我们的底细不可能欺瞒太久,手上只有八百兵的消息,宾州多少人都知道,要说里面没有交趾人的奸细,我是不会相信的。”

“黄洞主不是已经奉命派人去守着各条路口了吗?”

“只要有心通过,小心的避开封锁道口的守兵,怎么可能封得住?只是能拖延个一两日而已。”韩冈不会将自己的希望放在敌人的愚蠢上。料敌从宽,把对手想得厉害一点不会错。

“运使,派去邕州的人回来了。”

韩冈在抵达归仁铺后,便派了一小队人潜去邕州城,现在回来一个。详细的情报并没有带回来,不过交趾军正在撤出邕州,则是一个很明确的事实。

“小人离着邕州有三里地,看见城中的兵马都是在往外走。不论是交趾兵还是广源军,都在撤出邕州城。而且也能看见百姓在逃离邕州。”

斥候的回报让韩冈等人喜出望外,也绝对是一个好消息了。这是连续几场胜利带来的结果,不论是他们是准备撤退,还是准备向归仁铺攻过来,被打开的包围圈,已经给了邕州百姓逃生的机会

“他们能进抵邕州城,交趾细作要绕过昆仑关,恐非难事。”

“那该如何应对。”

“何须应对。现在心急的是李常杰,而不是我们。敌强就退,敌驻则扰,敌退我们就追。不硬拼,但也绝不能让他们好端端的回去。”

只要不是大队的人马,区区十人上下的小队,借着夜色潜行至邕州城外,并不需要冒太大的风险。同样的道理,昆仑关周边的一片山岭,也不是黄金满派出的那些人能够封锁得了。

也正如韩冈所料,这时候在李常杰的面前,站着一人。他穿着普通汉人的服饰,头上戴着帽子,只是在帽子没有遮盖到的地方,还能看青茬茬的头皮。

“宋军只有八百?!”李常杰不敢相信自己的耳朵。

“宾州城内城外都在宣扬,说尽歼刘永千人的只有八百荆南军,而且那一战宋军损伤只有四人。”

“速将刘永战死之事传给刘纪。”李常杰先吩咐下面的亲信,然后又沉吟起来,不管怎么想,他都很难相信、甚至不愿相信让自己惊师动众的对手,只有区区八百人,“其中必然有诈……不过要试上一试。”

