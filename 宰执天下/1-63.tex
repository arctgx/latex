\section{第29章 君意开疆雪旧耻(上)}

东京开封。

已近年终,开封府刚刚下过一场大雪。城中厚厚的积雪,昭示着明年的丰收,给了苦于今年南方旱灾的君臣们一点安慰。只是东京城内街巷上的积雪并不能久留,很快就开封府组织人力被清扫一空,不会阻碍行人。尤其是从皇城南面正门宣德门一直向南延伸到州桥的御街,宽达两百步,根本就是一座广场,却早已看不到半点残雪。

北宋开封的皇宫,论面积并不算大,至少远逊于隋唐时西京长安的大明宫。朱温在开封立都时,汴州早已为胜地,人烟辐辏,户口已愈十万,根本没有大兴土木的空间,只得把原来的节度使衙门改了改,住了进去。而五代各朝,都是纷纷而兴,纷纷而败,没有时间和财富在皇宫上下功夫。等到宋代周兴,太祖赵匡胤勉强将皇城整修了一番,而太宗赵光义登基后,想着扩建皇宫,却因附近的民家反对而作罢。

不过宫室再简省狭促,也不会在门面上省工料。宣德门为皇城正门,高近十丈,有五门横列,‘门皆金钉朱漆,壁皆砖石间甃,镌镂龙凤飞云之状。莫非雕甍画楝,峻桷层榱,覆以琉璃瓦,曲尺朶楼,朱栏彩槛’,与其说是座城门,不如说是栋修造精美的楼宇,故而也称为宣德楼。宣德门两侧又有两座副门,名为左掖门,右掖门,形制比宣德门稍小一些。

宣德门后,是一片面积可容万人的广场,广场之后的巨型殿宇便是开封皇城的主殿——大庆殿。大庆殿位于皇城中轴线上,是皇城中最为雄伟壮丽的建筑。但大庆殿只有正旦、冬至的大朝会,或与之同级的朝廷大典才会启用。如今日的朔望朝参,则只启用大庆殿西侧的文德殿。

四更刚至,天色仍是黑沉,冬夜的寒风依旧刺骨,可皇城前的御街上已经热闹起来。这一天是熙宁二年闰十一月十五,乃是朔望大朝参之日,仅比正旦、冬至的大朝会低上一等。在京的所有正八品以上、有朝参之权的文武官员,都纷纷踏足御街上,前往皇城参加朝会。御街上的官员,有身着金紫,随从多达百人的宰相、亲王,也有单身独骑的青袍、绿袍小臣。即便不算随从,只论官身,熙熙攘攘也足有四五百人之多。

因为朝会起得如此之早,走在御街上的官员随从们大半都是肚里空空。并非他们出来前厨中不开火,而是因为就在御街两侧,各有一条千步长廊,号为御廊。御廊之中,就有许多摊位做着早点生意,水饭、爊肉、干脯、肚肺、赤白腰子,南北餐饮琳琅满目,想吃什么就吃什么,根本不需要将家中的厨娘或是浑家唤起。以御街的宽度,并不会因为长廊中多了些摊贩而拥堵。

当官员们在御廊中吃饱喝足,陆续抵达皇城脚下后,都纷纷下马。宣德门五道城门,正门惯常紧闭,当天子出巡或是朝堂大典时才会开放。官员们皆是下马从宣德正门边的副门入宫。宰执官们同样走宣德旁门,不过却能独骑昂然自入。宰执身负军国之重,得享殊礼,可以直入皇城,在第二道门处方才下马。

又是一队浩浩荡荡的骑队抵达宣德门前,八十多人的队伍比起百多人的宰相随班要单薄一点,却已远远超过其他文武官员,这是执政才能享受到的待遇。八十多人以两名腰系金带的朱衣吏为引导,张起宰执才有的青凉伞,簇拥着一名身着紫色方心曲领公服,腰佩金鱼袋的中年文官,直抵皇城前。

一见其人骑马而至,犹在皇城外的官员们,纷纷避道行礼。比起见到方才入宫的宰相陈升之,还要恭敬上数倍。却是如今最得天子宠信,有扭转国家颓势、一洗百年积弊之心的参知政事王安石到了。

王安石骑在一匹普普通通的骟马之上,所穿公服上的紫色已经被洗淡了许多。他肩宽体阔,身材高壮如牛,只是面色黧黑,仿佛多少年没有好好洗过。曾有人说他和同样身材高大的文彦博,是牛形人能负重致远,乃堪为宰执之相,但如今担任枢密使的文彦博和王安石却是水火不容,如同死敌。

在宣德门处,王安石没有多做停留,驭马直入皇城之中。他和文武百官从宣德门进入皇城,正面的是大庆殿的广场。转向左经过一道分割宫城中部和西部的横门,抵达文德门前。王安石至此方才下马,徐步走进文德门中。

文德门后,是一条百步长的御道,直通文德殿。御道两侧,先是钟楼、鼓楼一东一西隔路对峙。钟鼓楼之后,隔着御道又是两条长廊式的宫舍,名为东西上阁门。文武百官穿过文德门后,并不是直入殿中,而是要按照文武分东西两班,在东西上阁门处列队,等待上朝。

王安石到得已经算是迟了,需要参加朝会的文武官员已经到了大半,两间阁门中站满了人,却是鸦雀无声,呼吸可闻。谁也不敢乱说乱动,宰相亦是如此。御史和阁门使们就在边上盯着,若有大声喧哗,或是站错班次,不是当即被喝斥,就是朝会结束后,被弹劾砸到头上。

王安石默不作声的从后向前走,东班的官员各自躬身退避,为他让出路来。王安石脚步不停,只在翰林学士班稍稍一顿,不知为何,六名翰林学士只到了五人,过去的老朋友、如今的死对头司马光却不见踪影,不知又是因反对何事而称病不朝。

想到司马光,王安石心中暗暗一叹。随着新法逐步颁行,均输法,青苗法、农田水利条约一项项出台,司马光、吕公著、滕甫,这些老朋友们也是一个个跟自己分道扬镳,甚至鼓动朝论清议横加反对。原本支持变法的,现在也因清议而沉默下去。

难道他们不知道国计如何艰难?!

太祖太宗的积累,在真宗皇帝迎天书,封泰山,大建上清感应宫的过程中,消失得无影无踪。仁宗即位后,好不容易有了点积蓄,却又由于党项叛乱立国,而砸进了陕西边陲的那个永远都填不满的无底洞里。国库至此已是勉强支应,但仁宗皇帝大行后四年,英宗又跟着驾崩,两次国丧的耗费终于将国库的最后一块遮羞布都扯了下来。

对此司马光给出的办法是什么?减少依例赐给参与国丧的臣子的封赏。

好高明的策略!

一千五百万贯的亏空,终于能省下几十万来了!

义正辞严的说着君子不言利,也不见他们辞了俸禄,捐了身家。如果所有的文臣都来个君子不言利,每年千万贯的亏空说不定真的能填起来。

但这可能吗?!

司马光敢这么提议吗?!

冗兵、冗官、冗费,这三冗是大宋财计步履维艰的主因。其中朝廷养起的百万大军,吞吃掉了财政支出的八成。其战斗力,也许还不如开国时,太祖皇帝麾下南征北讨的十万禁军。

为了减去庞大的军费开支,仁宗朝的宰相庞藉曾经主持过裁军八万的艰巨任务。他下了军令状,若有被裁士卒因此而叛乱,甘受死罪。但视庞藉如父的司马光,却从来没有胆量说一句裁军省费的话来,只是要天子节省再节省。

成事不足,败事有余,王安石早看透了这些清流。

越过一众翰林学士,他继续向前,一直走到队列的最前端。站进东班中自己的位置,王安石手持笏板,闭目不言,等待朝会的开始。如今在他的前面,只剩下最后的两名宰相,再上一步,便是位极人臣。

王安石没有等待多久,参加朝会的官员绝大多数都已到齐,上朝时间也到了。东上阁门使和西上阁门使计点过人数,作为监察朝臣礼仪的台官,御史中丞吕公著便领着两位殿中侍御史当先入殿。

他们与宰执班擦身而过,目不斜视,唯独吕公著瞥了王安石一眼,闪过一丝厌憎。他的御史中丞之位甚至可以说是因王安石而来,但吕公著却一点也不高兴。因为王安石并非善意,其目的不过是想将他时任枢密使的兄长吕公弼赶出京城。

吕公弼身为枢密使,执掌朝中军政,最喜欢说的话就是镇之以静,以和为贵,对王安石拓边西北的政策大加反对。与另一位枢密使文彦博一搭一唱,甚至差点将好不容易才夺到手的绥德城还给西夏人去。后为边帅反对,其事不果,便把夺取绥德的种谔贬到随州安置来安抚西夏。王安石难以容忍两块巨大的绊脚石继续挡在前路上,否则接下去他对军制、马制进行改革的将兵法、保马法必然会受到掣肘。

文彦博资历太老,一时难以动摇,而吕公弼虽为前朝权相吕夷简长子,但底蕴比已位列执政几十年的文彦博差得老远,何况他还有个做翰林学士的弟弟吕公著。所以就在不久前,吕公著他便被举荐为御史中丞,开始领导朝中的台谏系统。

本朝为防臣子弄权,把持朝政,宰执官和台谏中,通常不会有兄弟父子或是近亲存在。一旦出现这种情况,在位日久的一人必然要上书辞位,外放为官,从无例外。若是有人想赖着不走,御史们就有事做了,有时候甚至连姻亲同时出现在两府、台谏之中,都会受到御史们的弹章攻击。这是个不成文的惯例,很少有人敢违反,吕公著既为御史中丞,自身岂能不正,所以他大哥吕公弼在枢密院的日子也不会有多长了。

ps:重要的男配角出场了,贯穿了北宋后半段的新旧党争,也在这个时候上演着序幕。自古变革不易,无论是商鞅还是晁错,都没好下场。改动一下制度,便会得罪原有的利益集团,王安石的旧友也是一个个与他反目。

这一段是铺垫,也是对时代背景的必要描述。韩三现在休息一下,待会儿就会回来。

第三更,红票,收藏。

