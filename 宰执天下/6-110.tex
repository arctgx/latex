\section{第11章 飞雷喧野传声教(七)}

韩冈正从大庆殿前经过。

殿前的广场,还是一如既往的空旷无垠。

昨夜才下过的雪,今天一上午就被清扫干净了。

按田亩来算,差不多有五六十亩了,这么快便打扫完毕,看来经过了一场宫变后,在新任太尉种谔的手中,皇城中的禁卫们手脚越发的麻利了。

在经历了宫变之后,在任的八名三衙管军之中,除了张守约一个,太后不相信他们中的任何一人。而张守约因伤隐退,太后就算连张家才七岁的侄孙都封了官,又给了张守约一个多少年都没有授人的殿前司都指挥使作为奖励,也没办法让张守约再回来领军镇守皇城。

原本曾经就任过三衙管军的燕达,当年最为先帝所重,不过关西的两位太尉中,他的性格比种谔要沉稳,且比种谔年轻,所以被留在了西北。

而种谔虽然为先帝重用,但因为他一贯好战,又常常自行其是,其实并不受赵顼所喜。如今西夏已灭,将他拉回京城来任职,一个是教训士卒,另一个也免得他再动心思去找辽人的麻烦。

只是守卫宣德门要地的,还是李信主掌的神机军。

神机军眼下仅有六个指挥——在计划中最多也不会超过十个——可成员却是以西军中挑选出来的精锐为主,战斗力不逊于任何一支禁卫。

而且李信的资序和阶级,尽管离管军最低一阶的龙神卫四厢都指挥使还很远,但他所担任的神机军都指挥使,军中排序却仅在诸位管军之后。

因为殿前司与侍卫亲军司的三个都指挥使多少年都没有授人——这一回的张守约是致仕前特晋——在任的管军数量很久都没再超过八人。所以朝中已有传言,十一个管军之位后,恐怕很快就要再添上一个。就像当年太宗皇帝将龙神卫与天武捧日四厢都指挥使二职,归入管军行列一样。

传言终究是传言,李信作为韩冈的表兄,想成为管军,阻力来自于各方各面,但他和神机军受到太后的看重却是实实在在的。

神机营把守着皇城正门,经过改造的皇城城墙上,安放了十八门火炮,将整个东京内城纳入射程之内。而火器局新近铸好的两门万斤巨炮,现在正安放在宣德门后新修的炮位上。接下来还有两门,等到正旦大朝会时,就会拖到大庆殿前的广场上,与玉辂等礼器放在一起。

自入秋以来,每天早上上朝的时候,群臣都能在穿过宣德门的门洞后看见这两门格外巨大的火炮。

由于城门是在放炮鸣号之后方才开启,穿过去后,还能闻到空气中的硝烟味。

巨大的轰鸣,伴随着浓烈的硝烟,的确充满了震撼力,但这只是不能发射炮弹的礼炮。

韩冈一开始就认为那种好大喜功的火炮根本造不出来,现在造是造出来了,却变成了只能听个响的礼器。

整件事说起来让人啼笑皆非,不过韩冈还是很喜欢这两门礼炮,实在太适合作为礼器出现在仪式上。比起那些古董,这才代表着国家的未来。一个殷墟方鼎,一个重型礼炮,一古一新相映生辉。

在一干大典时,宣德门前都会有几只被训练好的大象相对而立,巨象高有丈许,重达万斤,獠牙长达数尺,门前一立,顿时就让人望而生畏。

而预定中的四门火炮也是如此,比起大象更有威慑力。这种纯人工的金属造物,远比自然界的任何动物,都能让人感受到其中蕴藏的恐怖力量。

在冬至日的大朝会上,先期被铸好的两门火炮便与殷时方鼎、唐时玉辂等所有礼器一并放在大庆殿前的广场上。

太后对这两具青铜礼炮十分满意,尤其是当火炮响起的时候,比起悠扬的钟声,更加让人振奋。

而对更为注重实际的韩冈来说,这几门礼炮的另一桩好处,就是为了承重,炮车的结构上也进行了一定程度的改进。之后用在普通的火炮上,效果会更好。

当然,要说吓到辽国的使者,那倒不至于了,只可能让辽人对火炮的研发投入更多的人力、物力和财力。

要是辽人决定与大宋对耗起国力,那还真是一桩好事。

从已经清扫干净的大庆殿广场外侧走过,眼前的白色就多了起来。

东京有着丰富的色泽。皇城内的建筑,有新近漆过的墙壁,也有斑驳的旧阁,除此之外,还有浅绿色的琉璃瓦。皇城外,则更是五色杂成。天上飘着的气球,楼前挂着的招牌,行人身上的衣服,可谓是多姿多彩。

除去人工留下的色彩外,夏天的主色调是浓浓的墨绿色,所有草木都在烈日之下泛着浓浓的绿意。到了冬月的现在,一场暴雪之后,就是白色了。

经过了一道长廊,韩冈抵达了内东门小殿,最近太后在此处处理政务的时候多了起来。

不过只有太后在殿中,天子赵煦并不在此处。

韩冈已经很久没有注意到那位小皇帝了。

不论是在文德殿,还是垂拱殿,又或是大庆殿,只要在太后面见外臣的地方,依照礼制,赵煦都应该在场。只有如崇政殿、内东门小殿这样太后处置政事,因为时间太长,不好让年纪幼小的皇帝枯坐终日。

不过这半年来,赵煦时常因病不上朝,大殿正中的位置总是空着,臣子们对此都快要形成习惯了。

韩冈行过礼,又被赐了座,就听太后说道:“今日是参政休沐,本是不应打扰,但这边有一封夔州路走马呈上来的奏章,却不好耽搁。”

太后想问什么,韩冈入宫前就心中有数了,低头道:“陛下有事传唤,岂能说是打扰,只是臣不知是何事,还请陛下明示。”

“据夔州路上奏报,权发遣黔州的黄裳,自抵任后于州中收留逃人众多,众家土官索回不得,正聚兵准备攻打黔州。此事,参政是否知晓?”

黄裳去夔州路的黔州任官已经有半年了,权发遣黔州、兼本路钤辖。他就任之后,只做了两件事,一个是招收流人垦荒种田,另一个就是修筑城墙,练兵备战。有韩冈在后面支持,又沟通了在西南的熊本,黄裳做起事毫无半点窒碍,甚至连夔州路的转运使、提刑使都赶上来想凑个趣,绝不似当年的王韶那般步履维艰。

夔州路上的土官一贯横征暴敛,对治下的子民,比对待牲畜还要苛刻,动辄杀人。每年逃到朝廷治下的州县的夷人成百上千,绝大多数州县官怕生事,所以都是将其拒之门外,但黄裳全然不惧,因为他背后有人。

黄裳是州官,日常奏报一向是发到政事堂中,若事关军机,则是发到通进银台司。若是走马承受奏报军情,同样是通过马递进呈到御前。韩冈能知道此事,多亏了通进银台司这个四面透风的衙门。

“夔州走马所说之事,臣虽无耳闻,却早已预料到了。黄裳既然收留各家蕃部的逃奴,当然会惹怒那一众土官。黄裳此前给臣写信,以及日常奏报中,也都说了这种可能,并请求朝廷允许他修筑城墙,并调来精兵加以防备。”

“黔州各蛮部向来不顺朝廷,此事吾亦知之。吾所担心,是黄裳初至夔州,尚不及半载,恩信未立,可能一战?”

这就是经历过宋辽大战后的太后,根本不怕战争,只是担心准备不足。

“黔州本有驻军,若能再调遣千余精兵,以一良将统领,此战当可高枕无忧。”

“精兵好说,有茂州在前,从关西调兵就可以了。装备了虎蹲炮,正好让他们上阵试试。但良将都不好办了,不知参政有何推荐?”

“此乃枢密院职分,臣不便干预。”韩冈低头说道。

“军国之事,政事堂什么时候不能干预了?参政尽管举荐,以供吾参考。”

向太后说话爽利,感觉越来越有执掌朝政的架势了。

“灭夏一战,关西有功将领中,陛下点选一人便可。”韩冈顿了一下,“不过黄裳资历甚浅,如苗履、赵隆辈,名位已高,功劳已著,就不方便先去茂州了。除非等到战事扩大,需要一举平蛮,再选任其人为帅。”

太后马上就领会了,“也就是要年轻一点的?”

韩冈点头:“正好历练一番。”

“吾明白了……但参政还没推荐啊。”

韩冈头微微疼了起来,想了一下,“种建中为臣同学,折可适曾在臣帐下听命,此二子臣所素知。”

“折可适是折家的吧?”

“正是。”韩冈暗暗叹了一口气,朝廷对那一家身处云中的诸侯提防始终,就算是太后都不能避免。又道,“也可以用罪臣,曲珍曾于盐州城下独自逃生,引罪夺官,于今仍待罪在家,以其老将,当能持重而行。”

“火炮太新,曲珍太老,他用不来。”太后很干脆的拒绝了,“种建中倒是可以,气学门下,有参政,有游师雄,当可大用。”
