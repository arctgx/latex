\section{第46章 龙泉新硎试锋芒(三)}

【第三更,求红票,收藏】

司马光的才智天下谁人不知?在仁宗立嗣之事上,司马光只写了几份奏章、说了两次话,就让仁宗最终点头。而在司马光之前,包拯、韩琦、欧阳修他们不知苦口婆心的催了多少次,都是无功而返。以这等眼光和才智,他又怎会看不出青苗贷的好处来?

青苗法是李参在陕西首创,施行有年,得到的评价也很高,所以王安石才会现在地方试行,现在又准备推行全国。而司马光却硬是说他在陕西看到的青苗法‘只见其害,不见其利。’

司马光之心,吕惠卿心知。

吕惠卿都定了调子,在王安石和他的助手们面前,韩冈也不介意拍拍司马光的脸:“若是借一还一,破产者几希。正是世间借贷多为倍称之利,下户方有破产之厄。如今青苗贷只要不强迫人借贷,百姓哪里还会有怨言?而富民要想贷钱生息,便不得不把利息降到与青苗贷同样的利率。百姓因此就有多了选择,不论公家、私家,让他们自选便是。此不是人情两便?

常言道货比三家,此事不必教,即便是妇人也是一清二楚。过去只有富民的高利贷,贫民无可奈何,只能受其所欺。若是官府和富民都有借贷,百姓便多了个选择,他们自会去选一个对自己有利的。青苗贷推行过程中有问题是必然的,天下有什么诏令会完美无缺的施行,但青苗贷的带来的好处却更大,司马内翰反对青苗贷,是只见其一,不见其二——太偏驳!”

章惇双手一拍,哈哈笑道:“货比三家这句说的好。韩、文诸公尽道青苗贷与民争利,他们的眼界,却连妇人都比不上。”

曾布道:“殊不知他们是不是装出来的!?”

章惇不屑的笑了一声:“文、吕、马之辈自然是装的,但有一些人,却是真糊涂。”

“司马君实从不糊涂,除了兵事,他比谁都聪明。”王安石是司马光的老友,他对司马光的了解当然比在座的说有人都要深。

“说起兵事,不是听说司马内翰要做枢密副使了吗?”韩冈突然问了一句。

曾布道:“司马君实辞掉了。加上前天的一次,枢密副使一职,他已经辞让了三次。”

章惇嘲笑着:“司马十二不敢做的。他过去在麟州闹得那些事,他自己最清楚。累得庞颖公左迁青州,没有颖公保他,他少不了要降上几级。”

韩冈前几天就听说天子有意让司马光担任枢密副使,归入执政之列。但他同时也听说了,司马光在兵事上完全没有一点可供夸耀的功劳,反而有丢盔弃甲的败绩。

章惇所说的庞颖公指的是仁宗朝名相庞籍——他在后世一样有名,韩冈了解到庞籍的事迹后,很奇怪为什么到了后世他就成了奸佞。庞籍既没有做贵妃的女儿,本人也不是太师,只有个太子太保的名头,死后追封司空和侍中,除了御下甚严,官声并不差——庞籍的儿子和司马光是连襟。嘉佑二年,庞籍为并州知州,主管河东北部边防军务。为了方便起见,庞藉便将司马光带去并州,做了通判。

庞籍兼管河东防务,因为自己年纪大了,无力去巡视地方,便让司马代他巡边。当司马光走到麟州的时候,接受当地知州、通判的提议,向庞籍建议在边境靠西夏一侧修建两座军堡。但最后的结果就是筑堡军全家覆没,将领郭恩战死。

战后论罪,庞籍把司马光建议筑堡的文书隐藏,自己担下了罪名。而后看到庞籍被削去节度使的职位,司马光心中不安,上书坦陈自己的错误,最终却并没收到处罚。因此事,司马光事庞籍如父,同时也接受教训,不愿再论兵事,反对任何扩张军队和战争的决策。赵顼让司马光为枢密副使,也算是讽刺了。

“不过不论司马十二做着什么官,他总是有资格去议论变法的。而新法……尤其是青苗法,在施行中,总是免不了会有些问题,而成了司马十二之辈攻击的目标。”吕惠卿问着韩冈,神色严肃得像是一位考官:“不知玉昆有什么想法?”

韩冈摇摇头,精神却是暗中一振,这个问题他同样早有准备。当即答道:“想法倒是没有,朝廷大事不是在下这等偏鄙小臣能议论的。不过……朝堂上的大事不论怎么定,究竟是用的什么策略,到最后,总得下发到地方,发到州里、县里甚至乡里,发到在下这样的从九品选人手中,让我们,还有更下面的胥吏去做事。”

曾布思忖了一下,问道:“……玉昆是想说司马君实,当然还有韩、文诸人,会鼓动州县里的小官和胥吏,抵制新法?”

“这也算是一个原因。”韩冈随口答过,通过抓住话题,来影响谈话的方向,是他的长项,可不会让曾布牵着鼻子走,“我等小臣和胥吏一向苦得很,俸禄微薄,要做的事却很多,做不好还要受上官训斥,甚至责罚。也就在前几天,在下还在驿馆中,见到了一个从鲁山县来到待铨选人。他在鲁山县【今河南鲁山】下面的三鸦镇做了两年管勾镇内烟火兼捉捕盗贼事,也就是监镇。

两年来他日子过得很是清苦,在下看他的衣服,都是打着补丁的。还听他念了一首在三鸦镇时做的诗,‘两年憔悴在三鸦,无钱无米怎养家,一日两餐准是藕,看看口里绽莲花。’。”

韩冈说完,而在座的几人都陷入了沉思。韩冈说这些自然有用意,王安石也好,吕惠卿、曾布、章惇也好,不会以为韩冈只是随口说个笑话。不过韩冈的用意也不难猜,以他们的才智也不过是转眼中事。

吕惠卿第一个反应过来,他也是哈哈一阵笑:“玉昆倒是说得好,不知濂溪吃得口中绽莲花的时候,作不作得出他的那首《爱莲说》。”

拿着周敦颐开了个玩笑,吕惠卿接着又道:“说起来,我过去在真州做推官时,曾经自苏州转迁来的监酒税的选人,他也是作诗感叹,‘苏州九百一千羊,俸薄如何敢买尝,每日鱼虾充两膳,肚皮今作小池塘’。”

章惇也明白了,他也道:“说起哭穷诗,我也听说过一首,是三班院的闲官所作,‘三班奉职实堪悲,卑贱孤寒即可知。七百俸钱何日富,半斤羊肉几时肥?’”

吕惠卿摇摇头,这首诗他也听过,很有些年代了。“那是哪年的老黄历了,还是真宗朝做的诗。如今的三班奉职的俸禄可不差。”

“岂止不差,不是有说‘三班吃香,群牧……’”曾布突然住口,因为下一句话,可是要嘲讽到王安石头上。

“‘群牧吃粪’是吧?”王安石笑着帮曾布将下一句补上,并不以为意。虽然他是做过群牧判官,但吃粪的事他却从来不掺和。

三班院是相对于流内铨的武官铨选衙门,管的是低品武臣,如刘仲武就是归三班院管。正如流内铨内外不论何时,总是有着几百名没摊到差遣的闲散选人一样。三班院中,也总是有着两三百没差遣的大小使臣。三班院另外,就是圣寿之日,参加饭僧进香的典礼。等典礼完毕后,剩下的香钱都会散给这些穷苦守阙的闲官们,聚在一起吃喝一顿。

而群牧监掌管着天下马场,虽然每年养不出几匹合格的战马——作为中书五房检正公事,曾布曾经看过群牧司的帐册,去年一年,全国各牧监出栏马匹总计一千六百四十匹,其中能做为战马的为二百六十四匹,剩余的则只能放在驿站里跑腿用。但靠着兜售马粪,群牧司却是从来不少赚钱。粪钱积攒下来的小金库,就是给群牧司的官员吃吃喝喝用的。

所以世间便有了笑话——三班吃香,群牧吃粪。虽然一个清高,一个腌臜,但饮风餐露的寒蝉,怎比得上滚着粪球的羌螂舒坦?

说了半天笑话,话题也是绕来绕去,完全扯不上正题,其实在座的每一个人却都是心底透亮,吕惠卿、曾布、章惇哪一个不是心有八窍,九曲回肠的人物;王安石性格虽拗,可更是才智高绝,哪能看不透韩冈弯弯绕绕的一番话下面,到底想说什么。只是他们不肯明着说出来罢了。

——韩冈是在要求给低层官员加俸禄!

给公务员加工资,这是一包包着糖的毒药。本来朝廷就是因为三冗而是财计年年亏空,最多的时候甚至达到一千五百万贯,这其中,有官员的一份功劳——冗官!而且是很大一份功劳,单是发给文武两班,总计两万余人的官员队伍的俸禄,差不多占去了朝堂财计的两成还多!但朝堂根本不需那么多官!

现在再提高低层官员的俸禄——如果按韩冈话中的意思,必要时,还要给吏员发俸禄——由此造成的巨额支出,青苗贷赚到的,均输法省下的,还有农田水利法新开辟的,这么些财政收入怕是都得填进那个新挖的窟窿里去【注1】。

注1:不要以为这个政策不合常理,到了熙宁五年,王安石便主动增发底层官吏的俸禄,好让他们能安心做事,而不祸害百姓——就是北宋版的高薪养廉——这里只是让韩三将之提前了两年而已。

