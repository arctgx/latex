\section{第30章 回首云途路不遥(四)}

韩冈与章惇公开在州桥夜市上会商。

前天晚上,从州桥经过的几百几千人,都看见了两名宰辅对坐在一顿都要不了十文钱的小铺子里。

那家卖烤肉的摊子,是否大赚特赚、是否事后弄出个宰相专座、枢密使专供来,京师百姓挺有兴趣,大报小报都大肆报道,但朝堂之上,可就全无兴致,他们只关心韩冈、章惇是否会因此而受到惩罚。

依祖宗之法,宰辅于都堂之外,严禁私会,以防臣子勾连,架空天子。即便臣子们真想要交通勾连,都有得是办法,但规矩就是规矩。

也许过去宰辅们私下里串通的情况不胜枚举,可是在明面上,公然聚饮的就是韩冈、章惇二人。

御史台为此整体出动。主要是弹劾韩冈、章惇无大臣体,以宰辅之尊,出入市肆——韩冈当年与薛向一起在小摊子吃饭,也就是这个性质。

只有少数几封,弹劾韩冈、章惇以宰辅之尊,不当私会。

这还是韩冈、章惇私下里让人安排的,免得惹起众怒——牢牢控制在宰辅手中的御史台,比一群疯狗更让朝臣害怕,有了主子,可就是主子指哪儿就咬哪儿了——否则现在真没有哪位御史敢于老虎头上扑苍蝇,那纯粹是在京城呆久了,想去南方品尝一下不要钱的酒和盐。

太后不得不将这件事重视起来。

苏颂将在月内便会正式上表告老,而在这之前,他已经在太后面前提过了。之后乞骸骨的奏表,不过是走个流程。

太后也曾极力挽留,而苏颂虽是感动不已,但并没有改变他的决定。

苏颂那边是走定了,而韩冈这边就跟章惇勾结起来了,这是要做什么?

宰相和枢密使两人同桌共饮,不论是哪位天子看到了心里都免不了要不安,太后又何能例外?

别的不提,首先异论相搅就玩不下去。更别说两人违背旧制,还明摆着就是要将宰相之位私相授受。

如果换成是先帝赵顼,看到做臣子的悖逆到如此程度,实在是史无前例,决定不会轻饶得了章惇、韩冈。

就是向太后在眼中,也觉得韩冈、章惇有些过分了。

往重里说,就算韩冈、章惇两人情有可原,但他们这么做了,有了先例,日后朝廷的规矩那还是规矩吗?

只是她还是不觉得韩冈会如此狂悖,肯定是有哪里给弄错了。

韩冈很快便被招到了内东门小殿。向太后质问着他:“相公,这到底是怎么回事?”

她其实并不怀疑韩冈会骗她。长久以来的信任关系,让她不会怀疑韩冈。之所以还要找韩冈来,只是不相信另一边的章惇

韩冈立刻扬声道:“陛下明鉴,臣与章惇只是出宫时同行,顺便在路边小坐,非为公事,只是闲聊而已。”

“就这些?”太后追问了一句,只觉得韩冈说得太过轻描淡写。

“陛下。臣与章惇结识多年,一向交好。后因识见不同,故而稍有疏淡。但同殿为臣,又并心合力辅佐陛下数载,闲来共语,也当是人之常情。”

太后皱着眉道:“但也不必在州桥夜市上。你看,御史台写来的奏章都有两三尺高,全都是在说相公和章枢密的。”

“陛下明鉴,臣与章惇正因为胸怀坦荡,并无阴私,所以才能坦然于州桥旁小聚。否则臣要与章惇私下勾连,难道还不能派人、写信吗?若是如此,怕也是外人难知,更不会有御史台的弹劾。臣今日所受弹劾,正是臣与章惇并无欺隐的明证。”

“不是因为苏相公要告老?!”向太后突然问道,难得的言辞犀利。

“陛下!”韩冈抗声道,“臣虽已知苏颂将请老,但臣可以父母妻儿为誓,前日与章惇相谈,绝无一字涉及相位!”

韩冈敢于拿着自己的家人发誓,不是他不迷信,而是他的确半个字都没跟章惇提起苏颂要空出来的相位。

“相公息怒,吾不是那个意思的。”向太后连忙安抚,等韩冈低头谢罪,她才又小心翼翼的问道:“那相公与枢密说了什么?”

“有说起天子的病情,章惇详细问了臣。又有说起当初章惇被贬出外,臣清晨送行的旧事,还提到了交州的种植园。此外还有曾经与臣一起那里共饮过的薛向,聊起他当年整顿六路发运司的作为。另有说起京城美食,此事臣与章惇各有主张。”

“官家的病情,相公是怎么说的?”向太后随即就问道。

“跟臣之前在殿上与陛下和群臣所言无异。具体内情,不得陛下同意,臣不敢外传一字。”

向太后点点头,这才像韩冈会做的事,只是又纳闷起来,“怎么又提起薛向那个叛逆的?”

“今年汴水纲运又是报上来多少毁损,故而臣与章惇一时皆有所感。薛向虽是逆贼,但才干卓异,财计、转运等事上,朝中无人可及。他败事之后,六路发运司中内事便一路败坏下去。”韩冈叹了一口气,“本是国士,奈何从贼。”

“都这么些年了,六路发运司还没整治好?”

“有薛向之材者朝中难寻一人。”

“相公也不行?”向太后不相信韩冈会不如薛向。

在她眼中,韩冈、还有章惇,都是开国以来少有的能臣,文武皆备,尤其是韩冈更加出色,而薛向籍籍无名,只是在钱财上小有才干,怎么当得起韩冈如此赞许。

韩冈道:“即使是为臣,遇上汴水,也只能另起炉灶,设法釜底抽薪。”

“那是相公想推行铁路罢了。”向太后笑着摇摇头,“如果真的如同相公所说,六路发运司败坏如此,那把京泗铁路和六路发运司合并,在沈括手底下挑选贤能,取代那些贪官污吏。”

“陛下,臣在西北时,曾经跟随王襄敏整编各军,整编时,总会将一干油滑又爱闹事的老卒另作一伍,绝不将新人编入,免得他们把新人带坏了。不是王襄敏和臣不想讲那些无用老卒一概罢去,实在是为免变乱,只能如此行事。”

向太后点头,“相公的意思吾明白了。”

韩冈向太后行了一礼,说了句太后圣明。

东拉西扯一番话后,太后也没有穷究到底的样子了,看起来这件事就到此为止了。

韩冈正这么想着,向太后突然冒出了这么一句来,“那旋炙猪皮肉真的那么好吃?”

“啊?”韩冈难得的一愣。

屏风后的向太后差点没笑出声来,的确很难见到精明厉害的宰相如此反应。

韩冈还是很快就反应过来了,老老实实的回答:“臣好此物,章惇不喜。”

“那相公和章枢密议论了,京师中何家小食最为上等?”

“正如陛下所说,市井中的皆是小食,哪里有上下之分,仅有口味之别。”韩冈道,“臣出身西北,故而口味略嫌浓重,旋炙猪皮肉和葱泼兔乃是臣之所好。而章惇东南人,更好清淡一点,据臣所知,当是洗手蟹和炒蛤蜊为其所喜。”

“哦。拿苏相公喜欢什么,相公可知否?”

“……苏颂年长,更喜素食。”

“曾孝宽如何?”

“臣与其往来不多,并不知晓。”

“沈括呢?”

“沈括好吃鱼脍,此与欧阳修同。”

“原来是这样。”

君臣两人就在饮食上的喜好稍稍展开来一阵简短的讨论,之后韩冈便告辞离开。

“就这么轻松过关了?”

消息传出之后,很多人觉得不敢相信。都有几分怀疑是不是韩冈逼得太后不敢细问此事。

而韩冈和章惇则毫不在意,各自上表自陈行动不谨,然后太后下诏,两人都罚俸三月。

韩冈清楚,这件事是太后相信他,所以才会轻轻放过。

但话说回来,太后即使是不想放过,最后的结果也不会有什么改变。

韩冈和章惇都已经是盘踞在朝堂上的参天巨树,根基深入到京师的每一个角落,就是有一个名正言顺听政治事的皇帝来,都不可能直接拿两人下刀。除非他想引得京师和朝堂一片大乱。

这件事算是过去了。只要太后、皇帝和世人慢慢习惯,等到时间久了,就可以有一处公开场合,让宰辅和重臣们共同商议国是。等到最后的投票,再在太后和皇帝面前开始就行了。

韩冈也没打算害太后,如果太后想继续垂帘听政下去,韩冈绝对会支持到底。也会尽力帮着向家日后能够安稳度日,不用担心亲政后的小皇帝报复。

当年章献刘后上仙后,立刻就有人告诉仁宗皇帝他不是刘皇后亲生,他的生母其实死于非命。当时仁宗皇帝都已经命人将刘家人都看管起来了,要不是章献并没有害了章懿皇后,刘家人会是什么下场想也知道。

当遮天的大树倒掉,后家想要保全,就得看皇帝的心情了。而韩冈,他愿意帮助向家人,不用去提心吊胆的过着天子亲政后的日子。

过了几日,当秋税的帐册全数汇聚京师之后,苏颂正式上表太后,自述年迈,愿归老乡里。

这份乞骸骨的章疏,在太后和苏颂手中几番来去之后,太后终于同意了苏颂的请求,但并不是立刻就允许他卸职,而是先以宫观使相赠,在京师赠其宅邸,留他住下来。

再这之后半月,锁院宣麻,章惇继承苏颂留下来的空缺,成为新一任的首相。

