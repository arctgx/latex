\section{第25章 闲来居乡里(三)}

在父母的房间说了些话,韩冈和他的妻妾各自回房去。

为了侍奉韩冈,四名妻妾都排了班,今天轮到王旖侍寝。

先安排了明日家中事务,又去洗了个澡,半个多时辰后,王旖才来到自己和韩冈房中。让两名使女留在在外间,举着一支烛台走进黝黑的里间时,正好看到就韩冈坐在桌前。

房中没有点灯,只有一抹淡淡亮光。隔着碧纱窗,银色的月直照进来,正照在韩冈的脸上,眉间有着深深的阴影,在想着什么。

知道丈夫在考虑事情,王旖轻手轻脚的走进来,将点好的蜡烛用纱罩罩上。晃动的烛光,在经过了白纱罩散射之后,顿时变得柔和了起来。

安放好烛台,王旖悄步走到韩冈身边,问着他:“还是在想着冯家四叔带来的秦州商人们的事?”

“嗯。”韩冈点了点头。在他的计划中,与秦州豪门的合作是长久的事,一开始就要考虑清楚具体的分配条件,让自己吃亏他当然不干。可也不能太贪婪,不然合作肯定以分崩离析而告终。

听到丈夫证实自己的猜测,王旖有些难过,低声劝着:“官人。你现在已是一任朝官,日后也必定能身居高位。家里的吃穿用度,本也不多。有你的俸禄也已经足够了。何必与那些商人打交道,落一个聚敛之名?”

韩冈摇摇头,他娶得这个妻子的确是贤淑,但却把自己看得低了。反问着:“你当我是郭逵吗?”

郭逵虽然是如今朝中首屈一指的帅才,但他对于财货十分看重,在关西边地参加回易的商队中,从来都不会缺少郭家商队的身影。郭逵的夫人史氏多次对此劝谏,说‘我与公俱老,所衣几何?……何以多藏败名?’

“不是!”王旖连连摇头,她从来也不觉得韩冈贪于财货,功名都视若等闲,更别说那等阿堵物。只是看到韩冈为着些货殖之事,而让冯从义找来那些商贾之徒,王旖怕传出去后,伤了丈夫的名声。

“那是什么?”

“……只是……只是……”王旖只是半天,却不知该怎么将自己心里的担忧,在不触怒韩冈的情况下给说出来,急到最后,几乎就要掉下了泪。

看着妻子泫然欲泣的模样,韩冈笑了,笑得温和,完全没有生气。揽着腰,抱着王旖坐在腿上,低头在她耳边柔声说着:

“韩家这一支,自胶西乡里来到关西已经有几十年,但至今也没打下稳定的根基,两位兄长死的太早,就只剩我一个。别看现在如烈火烹油一般,只要我倒了,韩家转眼就会败落。我现在只求韩家能扎根于陇西,以此为根基而开枝散叶。”

“聚敛并不是目的,得到的钱财也只是可供使用的工具而已。巩州新辟,若能深植于此,援引奥援,日后必为此地豪族。纵使不能代代进士,但做着名乡绅,也足以保守家门。我看重他们,其实也是为了他们背后的秦州大族。”

韩冈不辞口舌的解释着。他知道,云娘三女对自己的决断都是盲目的信任,所以从来没有怀疑。而王旖因为是大妇,主持中馈,就算是全心全意的信任,也必须要多问一句。若是一概不问,韩冈才是要担心的。

王旖低声:“原来是这样。”

韩冈知道这番话还不足以让人信服,又道:“何况有此心思的不只我一个。不然王处道何必从文官转了武资?他可是过两日就要到狄道任知县了。”

王厚早就有投笔从戎的打算,他的大哥王廓是进士,但王厚自知没指望能考上一个出身。早前他就从赵顼那里得了首肯,在三班院中做了一任之后,便从文职转了武职。就在前天有消息过来,内殿承制王厚,被派在了熙州州治狄道县担任知县——边疆州县,武官也可主持。

拿出王厚证明,王旖一下惊讶了:“王家二伯也是要移驻熙河?”

“王家家大业大,从江西德江分出一支来也是很正常的。何况王家在熙河的产业,也不能全让外人看着。”韩冈想起当年高遵裕、王韶和自己,三家垄断古渭榷场的情况,不由一叹,“就算是再是清高,也不能餐风饮露,追财逐利都是少不了。只要不弄错了赚钱的目的,也就够了。至于名声,外面用这事攻击不到我头上,放一百个心。”

韩冈又想起了种建中,那一位,可是为了从武职转为文职,而跑去考了一个出身来。

王厚、种建中两人对未来的想法不一样,所以作出的决定不一样。种建中本是将门弟子,所以要求一个文官也很正常。而王厚或者说王家则不同。

武将虽然远不如文官,而且还要从文官那里受着憋屈,但对于想稳保家门的人来说,走武将的道路反而是长享富贵的捷径。就算是诗书传家的书香门第,谁又能保证代代都有进士出来?若考不上进士,基本上一辈子都升不到可以荫补子孙的七品官。即便成功,第三代的荫补官连转为京官都难,只会一代比一代更差。

“别看王副枢如今煊赫异常,几个儿子都有荫封,可日后谁又能保证,王处道这第二代能升到高位去?或者保证王家的第三代第四代还有出色的弟子?家第两代而绝,王副枢岂能愿意看到?

既然如此,还不如学着种家。种隐君【种放】可是文臣,但到了种仲平【种世衡】这一代就转成了武职,现在用了两代人的时间,在鄜延路的清涧城扎下根来,已经成了关中首屈一指的将门世家。

若是处道能学到种仲平的一半成绩,日后也是王家几代富贵的一个保证。文官难有传承,但将门可是一代一代传下好几代。

而且正好王副枢儿子多,可以两边下注。分出一个王处道走武将的路子,又是待在自己恩信威望深厚无比的熙河路,哪有不稳步上升的道理?比起种世衡当年守清涧城,起家的情况可是要强出千百倍。”

其实韩冈最想仿效的是麟府折家。杨家将中的佘老太君,其实本姓为折,就是这一家的女儿。不过是以讹传讹,最后被换了姓名。

麟州、府州,再加上丰州,位于河东路西北角、位于黄河之西的这三州,与辽国西京道接壤,同时还是位于抵挡党项人攻击河东的第一线。镇守此地门户的军队,乃是宋军中难得的精锐。可这三州精锐,直接听命的不是东京城的赵官家,而是折家的家主。

麟府折家对于宋室来说,是镇守边的重臣,甚至可以算得上是诸侯。从五代时起,出身党项的折家便盘踞于河东路的西北角,当宋室成立,便投了过来。而宋廷并没有将其麾下的军队改编或是解散。而是将那一片地,留给了折家。

直到现在,河东麟府军依然是掌握在折家旗下。就算朝廷往麟州、府州派遣官员,可又有哪人会跟让所有下层吏员和军校都俯首听命的折家过不去?

其根基之深厚,地位之特殊,人望的高峻,兵马之强盛,种家虽然号为将门,却是根本比不上。这是因为历史、地理等多方面因素而形成的特例。在韩冈可以预计的未来中,折家的地位依然稳固,党项、契丹一日不灭,折家就不需要担心有兔死狗烹的一天。

韩冈当然羡慕折家,在这个时代,‘彼可取而代之’的可能性微乎其微。无名却有实的诸侯,已经是此时能取得的最好的地位了。韩冈也想在陇西模仿折家的地位,有着一半的水平就能常保家门,不需要名义的统治,一个实质上的控制权就够了。

靠在韩冈怀里,王旖轻轻点着头。自家的夫君都说得这么详细,她已经明白许多。“官人真是深谋远虑。”

“哪能算是深谋远虑?不过是自保之道而已。”韩冈自嘲的笑道,“岳父为国无暇谋身,那才让人敬佩的。只是学不来啊……”

王旖因韩冈的话沉默了下去,只要读过史书,谁能都知道主持变法者的结果。商鞅可是最好的前车之鉴,更不要说王莽那个法古到昏头的逆贼。谁也说不准王安石、以及临川王家,最后回落到什么样的境地。

韩冈不想妻子太过担心这些不知多少年后的事情。双手探进衣襟中,摩挲着她细腻的小腹,渐渐向下,转移着她的注意力。

“等处道来了之后,你也要跟他家的女眷多多走动。我可是跟他定了儿女亲家,今后可是要互相扶持几代人呢……”韩冈的手指已经探进了晕湿的洞穴,指尖每一记勾划,都能引起怀中娇躯的一下颤动。

竭力被压抑的喘息声,渐渐沉重了起来,王旖的身子也变得滚热。

感受着指掌间慢慢的变得湿润,韩冈低声在妻子的耳边喃喃着:“还是给为夫早点生一个嫡子出来,也别让人说我言而无信!”

