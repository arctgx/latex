\section{第27章 京师望远只千里(七)}

【新的一更,求红票,收藏】

游师雄也是张载的弟子,在同学中向以知兵著称。不过他并没有跟韩冈同窗就学的经历,因为就在五年前,也就是治平二年,他就已经中了进士。不过韩冈还是在张载门下见过游师雄一面,虽然当时的主角是游师雄,而韩冈则是在人群外的看客。

现如今,张载门下的出色弟子,或多或少的都有些联系。今年开春后,游师雄转任邠州军事判官,这件事种建中在给韩冈的书信中提过了。可韩冈并不知道吴逵跟他的关系如何。一般来说,文官武官之间的鸿沟比渭河还要宽上一倍,而吴逵正因李复圭之案而愤恨不已,这两天的闲谈时,韩冈便没提到游师雄。只是现在看来,两人还是有些交情的。

能见到闻名已久的师兄,韩冈也是喜出望外,寒暄了几句,问道:“景叔兄今次至长安,是为了拜见韩相公,商议军事的?”

“愚兄这邠州军判可站不到韩相公的军议上去,只是到京兆府来要钱粮的。不过韩相公既然,说不得也得过来拜见一下。前两天递了帖子,今天终于能进去说上两句。”游师雄自嘲的笑了笑,瞥眼看到吴逵还竟然还在一边站着,急道,“吴逵,你还不快进去,前面韩相公已经提到你的名字了!”

吴逵脸色骤变,给高高在上的宰相惦记上,可不一定是好事。他匆匆向韩冈告了罪,丢下手下的一队人马,飞快地走进了驿馆中。

吴逵的亲卫等在门前,但看门的守卫视他们为麻烦,将他们驱赶得远远的。只是此处正是巷中风口处,风呼呼的刮着,如同刀剑切割着行人的皮肤。韩冈想让他们换个地方去等候,不然迟早会生病。可这些广锐军的士兵一齐摇头表示拒绝。

一路同行两天,韩冈看得出来,吴逵在这些士兵心目中的地位很高,一句话、一个眼神,就能让他们悚然待命。但吴逵也不是全靠威严来镇压麾下将士,嘘寒问暖的事他没有少做。他是把手下当自家人来看,要不然这些悍勇之人也不会安分守己的等在驿馆外面。

吴逵进去了。韩冈和游师雄不便再堵在驿馆门前。由游师雄带领,往最近的一家酒楼走去,李信、李小六跟在后面。这次,换作了游师雄发问:“玉昆,你与吴逵怎么走在一起的?”

“不过是道上偶遇。前日暴雪,马嵬驿墙倒屋塌,入住同一家客栈,正巧碰上了。”韩冈简略的解释了一下。

“原来如此,”游师雄点点头,转而又问道:“玉昆,你从秦州过来,路上正好经过横渠镇,有没有去看望一下先生?”

“今次运气不好,先生正好得了蔡经略的书信去渭州了,没能遇上。不过看到了新修的书院,大体上已经修得差不多了,明年开春前当是能进人了。”韩冈无奈的笑了一笑,他几次经过横渠镇,都没有机会跟他的老师们打个照面。

“新的书院有四分之一的功劳是玉昆你的。愚兄这里都听说了,今次兴建书院全靠玉昆你送上的价值几百贯的财帛,不然先生毕生所想的这座书院,至少要到一两年后才能动工。”

“一点阿堵物而已,比起先生对小弟的教诲和栽培,不值万一。”韩冈随着游师雄穿过两条小巷,一边笑着说道:“先生要办书院,其既有此心意,做弟子的哪能不照办。有事,弟子服其劳嘛。今次小弟还看到了先生划的井田,的确有些意思。”

“有些意思?”游师雄略略提高了声调。

“有些意思!”韩冈很肯定的点着头。仅仅是有些意思而已,井田这种已经消亡了的土地制度,在现实的生活中实际上根本没有半点可操作性。

游师雄这时在一间食铺前停了下来,门面很小,也没有楼层,与其说这是酒楼,不如说是街边小店。

“这个食铺虽然简陋,但味道上佳,比起外面的大酒楼要强上不少。几次来长安,都要到这间店中吃饭。”游师雄带着韩冈三人走进去,店家便迎了上来,引了几人坐到了桌边,倒了茶来。“正好可以庆贺玉昆你不日便要高升。延州的将士可是翘首以待多时。”

“不知景叔兄从何处听来?!”韩冈闻言一惊:“小弟只是奉命进京而已,没听说要转调鄜延。”

“怎么还没听说啊,愚兄是从种彝叔那里听来的,当不会有假。”

事关前程,韩冈追问着:“种彝叔的信是怎么说的?”

“种彝叔给愚兄的信中,提过有关玉昆你的事情,说玉昆你开设的疗养院,还有沙盘军棋,都是发前人所未发,连种五都深为赞许。前几封虽然没明说,但看文字的意思,就已经是想要把玉昆你调到鄜延路去。而前日寄来的最后一封,已经点名玉昆你担任鄜延路的管勾伤病事。”

“管勾伤病事?!……竟有此事!”韩冈脸上有了惊讶,心里却是骂开了。韩绛未免太小瞧人,管勾伤病是临时差遣,根本不是正式的工作。想把他调到延州,好歹给个像样的职司,管勾伤病事做兼职可以,不可能当成本职工作去做。

“怎么,玉昆你不愿?”韩冈没有刻意掩饰他心里的想法,让游师雄看出了他心中的不快。

韩冈闻言反问:“景叔兄,你当真以为今次罗兀能成事?”

酒菜这时都端了上来,菜肴多是鸡鸭,味道是难得的好口味。但他家的生意做不大,的确让人觉得哪里有些不对劲。

“凡战者,以正合,以奇胜。用兵‘出其所必趋,趋其所不意。’如果党项人今次没有发觉延州那里的动静,出齐不意四个字,的确是做到了。”

“但接下来呢,孤悬在外的罗兀城,又能抵挡多久?”

在韩冈看来,不论韩绛和种谔都是太性急了。刚刚得到绥德城,便把眼睛放到了罗兀城头。尤其是种谔,他老子种世衡的耐心一点都没继承下来。种世衡当年筑起清涧城后,断断续续花了十年的时间,开辟荒田,收复蕃部,把清涧城的防御体系打造如铁桶一般。而正是有了清涧城这个基地,种谔才能在三年前彻底夺下绥德城。

“清涧城周围十七处寨堡总计用了十年才修造完成,大顺城到现在还在修筑中,秦州的甘谷城,如今建起才三年,虽然地势绝佳,但连成一体的附堡才不过三处……听说去年和今年便有两次差点就被攻破掉。即便攻下罗兀,要想能稳守,不是三年五载可以见功的。”游师雄不负知兵之名,在兵法上果然有长才,早就把攻打罗兀城的害处看透了。

韩冈很奇怪,“即是如此,景叔兄你为何不去找种彝叔,怎么跑来找我了?”

“玉昆你以为到了这时候还会有人听吗?愚兄已经给种彝叔去了四封信了,没少提这话,但就是没有回应。”游师雄与韩冈互相敬了几杯,此时多了点醉意,络腮胡子参差不齐,而当他眼神剔起,便更显得凶悍。让人不禁怀疑,他到底是不是进士。

“不知将此事说给王相公听,会不会让他警醒过来,改成了更好的做法。”韩冈像是在自言自语,很快就摇摇头,“就算能够说服王相公,但韩相公如今可是昭文相,会听王相公的话?唉,可惜国事……”

如司马光、韩绛这样的朝廷重臣,对游师雄刚刚入官五年的选人来说,都是要仰头看的。哪可能如韩冈这般轻轻松松的提起来。而像韩冈一年跳过几个台阶的情况,根本是个异数。普通一点的官员,少说也要费个七八年时间,才能能走完韩冈一年的道路。游师雄虽然是进士出身,又做了五年官,但论起本官官阶,比韩冈还要低上一级。

不过游师雄没有嫉妒的意思,他是按部就班,以进士之身,迟早会升上去的。放下心头事,两人继续喝酒聊天,韩冈久历世情,想要刻意与人结交,通常很容易就能打得火热。游师雄本就是他的师兄,互相闻名已久,今日一见,一番闲谈下来,都觉得不负传闻之名。

……………………

次日,处理完了一番紧急公务。韩绛在驿馆中端起了茶盏,喝了两口甘甜的茶水,问道:“不是说韩冈就在城中吗?怎么他的帖子还没递进来?”

听命外出的亲兵绕了一圈就回来了,他回来后对韩绛禀报:“回禀相公,韩冈今天已经启程东去了。”

韩绛的脸色闪过一抹阴云,不过转眼间就消散了,他微笑着,像是在赞许:“无事干谒上官,本是官场恶习。韩玉昆不从流俗,不媚显贵,的确是难得。”

“元智,”韩绛叫来常为他代笔的门客,“且去草拟一份奏折,就说大军北进在即,战事一起,损伤难免,望朝廷速遣韩冈至延州。”

元智愣了一下,小心翼翼地问道:“……还是请朝廷遣韩冈至延州?”

韩绛点了点头,没再说话。

