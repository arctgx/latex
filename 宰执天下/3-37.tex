\section{第13章 上元惊闻变(下)}

听到这个消息,向王厚再三确认,韩冈就没办法再安坐着读书了。

五十多岁的老人,一下从马上摔下来,伤筋动骨是免不了的。再怎么说都是未来的岳父,韩冈有着及时去探望的义务。

虽然其中还有些让人闹不明白的地方,但只要深思下去,韩冈更是觉得他有必要去王安石府上走一趟。

从王家借了马,韩冈一路赶到了相府。

根本不用再多话,韩冈只一亮相,相府的司阍就忙不迭的将姑爷迎进了府中。

章惇在元旦之后,就已经回返荆湖。曾孝宽出外巡视河北。新党核心层中,剩下的吕惠卿、曾布、吕嘉问也都到了相府之中。

当韩冈走进偏厅,王安石父子,加上吕、曾、吕三人,总共六个人就都在这里。

王安石本人并没有受伤,但黝黑的一张脸,现在黑沉得更加厉害。只是见到韩冈赶来了,他的脸色方才和缓一点:“玉昆你来了。”

“韩冈来迟了,不知相公可有大碍?”

韩冈一进门,便赶上去嘘寒问暖。关心的模样,让王安石心头怒气消褪了不少,连声说着:“没事,没事!”

韩冈问了几句,见王安石当真无事,才直起腰,问着:“今天究竟是怎么回事?”

他这么一问,王安石的脸一下又沉了下来,“还能是什么?有人想将老夫赶走!”

王旁过来拉着韩冈,低声的对他说了今日之事的来龙去脉。

今夜天子照例出宫观灯,在御街上饶了一圈后,又照常规回宫主持家宴。正月十四的夜宴,参加的都是宗室。但观灯时随行伴驾的重臣们,也要照规矩将天子送回宫中后,再参拜恭贺一番,才能各自回家。

赵顼的大驾从宣德门正门进宫,而宰执官照常例便是到了宣德门内再下马。但今天王安石从宣德门西偏门进门时,却被门卒给拦下,让他在宣德门外下马。

为王安石牵马的从人上前分说了两句,却被当头一棍打破了脑袋。混乱中,王安石的坐骑也不知被谁抽冷打了一棍,更把王安石也颠下了马来。只是他身边的元从多,没有让王安石出事。

从王旁嘴里听到了事情的经过,韩冈的眉头就紧锁了起来。

整件事听起来像是个闹剧,可他绝不会把今天的事看成是闹剧。在场的每一位都不可能这么看。

没有人指使,谁敢在宣德门拦住宰相?

日日上朝,所有的宰执官都是在宣德门内下马,怎么轮到就上元节时,就必须在宣德门外下马?

“这是分明要激怒相公。只要相公因此君前忿怒,便可攻击相公不逊,无人臣礼。”

吕惠卿最近刚刚顶了曾布的职位,成为中书五房都检正,本官又从太子中允一跃迁为右正言。而且看势头,过几日,恐怕还有更进一步的升迁。如今正是炙手可热的时候。

吕惠卿发话,曾布便默然不语。两人之间,关系明显的很是微妙。

“即是如此,又该怎么应对?”吕嘉问问道。吕惠卿说的话谁都明白,关键的是应对。

“当然是镇之以静,看看他们还有什么招数?”韩冈开口道。他既然站到了这里,肯定要出出主意。

就像方才吕惠卿说得,这分明有人故意要激怒王安石。以王安石的脾气,肯定要查个水落石出,这样可就要上当了。不如什么都不做,。

“玉昆!”王雱一下怒道:“大人可是宰相之尊。礼绝百僚、群臣避道。却受辱于小卒,莫说大人的体面,就是朝廷的脸面,可是一样也要丢尽。”

吕惠卿在旁接话:“但此事实在难以根究下去,不如按玉昆的想法,镇之以静,让天子知道相公的委屈。想来他们也是没有别的招数了,才会如此鲁莽灭裂。”

能驱使得动宣德门守卫的,数来数去也就那么几个人。而其中手段会如此粗劣的,更是呼之欲出。

这一个指使者,查不出来都能猜出来,猜出来后就知道绝对不能查出来。

怎么得给天子留点面子!

“就算不能追究出主使之人,但传话的、下令的都能追究出来,他们肯定会自己认下,倒是也可以将他们远窜四荒。”

“但主使之人,连天子都要相让。追究到底,天子也会难做。”

“可总有不能相让的时候!”王雱愤然之言,更进一步坚定了王安石的决意。

‘哪有这么容易的事啊。’韩冈暗叹了口气,这事的确有些麻烦,王安石父子两人都宁折不弯的脾气。不像吕惠卿和他自己,为了获得更多的利益,可以选择妥协或是退让。

就算是定了亲的女婿,但韩冈的发言权依然不如吕惠卿,可吕惠卿也没能说服王安石父子,韩冈也只能干瞪眼。

宁从直中取,不向曲中求。

韩冈过去倒是经常这么做,但他敢于下狠手,都是顺着形势而来,可从来没有背时而行。

这件事的关键,就在天子赵顼身上。王安石也许还把赵顼当成是当年对他如同学生一般言听计从的新立之帝,但韩冈对如今的赵官家,可完全没有半点信心——近来凡事种种,都能看得出天子的信赖已经不足以依仗了……

除了王安石这个身在局中之人,还有心高气傲的王雱,不论是吕惠卿、曾布,还是韩冈、吕嘉问,其实都已经看了出来,王安石的圣眷已大不如以往。

上元夜一会之后,韩冈继续回到王韶家读书。

王安石那边也没有第二天便急着上书,而是先保持了几天的静默。王安石毕竟是浮沉宦海多年,并不是愚蠢和盲目的认为天子一如既往的支持自己。他先去查证了过去的记录,看一看,上元夜宰执入宫是否要下马。只要当夜,守门士卒喊出来的这条规则不存在,就可以名正言顺的请天子下令,根究此事的来龙去脉,追查背后的黑手。

只是王安石失算了,天子没有以他的奏章为准,而是问起了其他执政和皇城巡检,他们过去在上元夜,有没有进入宣德门后才下马。

得到的回答很可笑,也让王安石心冷。

冯京说他忘了,依稀记得是有在门外下马的时候。吴充则是信誓旦旦,他过去上元节都是在宣德门外下马。陈执中装了病。王珪更是一问三不知。至于当事皇城巡检指挥使毕潜等人,则是异口同声,说从来都是当在宣德门外下马。

尽管多少年来的上元节,几千几万人都看着宰执们从宣德门西偏门进宫后才下马,但王安石的同僚们,就没有一个来为他来作证。

而吕惠卿等人却无法帮着王安石做证明。不仅仅因为他们不够资格,而且要是他们多言一句,结党的罪名立刻就能扣到他们的身上。这也是背后推波助澜的黑手所想要看到的。

世人都知道新党,天子其实也知道,可只要新党诸臣在他们的权限范围内做好自己的事,谁也不能说他们有党。但若是一齐上书,为王安石在此事上争个高下,那就没法儿推脱了。

只能眼睁睁地看着王安石一人上阵。

这种情况下,王安石势单力薄的现状便暴露无遗,而有心人就看到了自己的机会。

虽然不支持根究此事,天子为了安抚王安石,还是下令十名当值的门卒一起解送到了开封府受审,开封府判官梁彦明、推官陈忱知情识趣,将他们一顿杖责了事。

可就算这样,依然有人跳出来指责王安石无人臣礼,并弹劾梁彦明、陈忱,曲意迎奉大臣之家,妄自将天子宿卫决杖,宜当重贬之。

这一个胆大的御史,并不是旧党中人,与吴充、冯京同样也没有瓜葛。当知道究竟是谁上书的时候,几乎每一个朝臣都吓了一跳,不是别人,而是新党中的蔡确!

‘这是第一个吗?’

韩冈听闻之后,又长叹了一口气。看来了蔡确这只政治老鼠,知道所在的船只快不行了之后,已经开始准备换船了。

蔡确的确是个见风使舵的主,但他政治嗅觉的敏锐却是无庸置疑的。

他当初将对韩冈的承诺抛诸脑后,转头就攀上了王安石——章惇韩冈的大腿,自然比不上王安石——自此走上了飞黄腾达的道路。

现在他又看清了天子的心意,用一份奏章迎合了天子,更洗脱了自家新党的身份——论起大腿,自然是天子更粗上一点。

蔡确虽然只算是新党的外围成员,但他的临风转向,却已经将新党内部的不安定给暴露了出来。如果王安石不能让天子将之贬官,将新党内部重新凝聚起来,因为共同的利益而形成的这一派别,其崩裂将会难以挽回。

就在朝堂上还为上元夜的宣德门之变而争吵不休的时候,韩冈终于迎来了久等了的进士科礼部试。

元月廿三,天子以翰林学士曾布权知贡举,知制诰吕惠卿、天章阁待制邓绾、直舍人院邓润甫并权同知贡举。连同点检试卷、监贡院门、诸科出义、考试、覆考,等一干官员三十余人,一齐同赴临时充作贡院的国子监。

从这一天起,所有的考官都被锁于贡院之中,直到二月初十礼部试开始。

