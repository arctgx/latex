\section{第30章 随阳雁飞各西东(十)}

城下的战斗结束了。 

并不是已经将城下的敌军都斩尽杀绝,连首级都收回来了,而是两队辽军骑兵的快速穿插让种朴感觉到了危险。 

辽人高悬在空中的飞船,能看清城中的布置。在飞船上的指挥下,这两队骑兵轻巧的避开了霹雳炮的攻击范围,直插城下。 

不过一千辽骑,凭着城外的一千精兵,种朴并不觉得会输。只要在濠河边布阵,完全可以较量一下。但计算过得失后,他还是下令出击的战士撤回。人数上的差距实在太大了,一旦辽人和党项人以步骑相配合,就算加上城中可以调遣出来的两三个指挥的兵力,在最后取得胜利,付出的代价也是种朴不想看到的。 

‘真是可惜啊。’种朴轻声自语。脸上却堆着心满意足的笑容,在出战千名战士中,拍打着他们的肩膀,然后大声夸奖着他们的勇敢。 

无论官兵,人人都是满面笑容,能一举大挫敌军,不管谁来看,都是可喜可贺的一桩胜利。接下来只需要继续战斗下去,将党项人打垮,剩下的辽人总不能骑着马来攻城吧? 

人群外的李清看得出来,种朴他并不满足。 

若能将三千党项斩杀大半,党项军的士气就会彻底崩溃,这一战不会再派上用场。而以种朴所了解的契丹人的作风,他们也不会拿自己的性命去攻击坚城。 

这一战的结果很可能就这么决定! 

但现在,辽军骑兵逼退了官军,救回了大半党项士兵,士气未泄,战事多半还要拖延下去。

正如种朴所料,下一波攻势隔了一个时辰后又开始了。 

依然是党项人主打,他们这一次选择了远离城门和霹雳炮的位置,一个个带着土包,试图最耗时间的办法攻上城头。 

契丹人也开始配合攻击。一队队十余人的契丹骑兵开始绕城飞驰,间或向着城上射上几箭,寻找着城防上的漏洞。另有两支千人队,在一里外监视着城门,若是城中守军再想打开城门,他们启动后转眼便至。 

不过这样的攻击,城上的宋军应对自如。 

两支监视城门的千人队很好解决,种朴直接就将八牛弩挪了过去。城头上的这一动静被天上的飞船发现,原本还算紧密的队形,立刻变得松散起来,而大旗下的将领也退到了阵后。在澶渊之盟后,射程远远超过一里的八牛弩,是契丹人最为畏惧的守城利器。 

绕城骚扰的契丹骑兵更好解决。神臂弓计算提前量并不难,几十架神臂弓同时射击一点,三次中总有一次能将飞驰而过的契丹骑兵射落马下。几次下来,他们绕的圈子就越来越大,从马上射出的长箭,也尽往壕沟里落。 

至于仍在往城下冲的党项人,宋军给霹雳炮加了轮子,直接就推过去了。 

看着在石子和泥弹下抱头鼠窜的党项人,李清摇摇头:“今天若是破不了城,再想破城,除非城中弹尽粮绝。” 

一座兵力充足、城防顽固、粮秣充裕的军城,只要守军有坚定的信心,就算宋军来攻,也必然是旷日持久。别的不说,旧年贝州王则据城作乱,为了平定这一股叛军,文彦博和明镐可是绕城筑了一道围墙,用了一年的时间才攻破了贝州。既然不能一鼓即克,那么就只能拿时间和人命去填城壕了。 

“辽人还没有拿出霹雳炮呢。”种朴则多想了一点,“若是造得多了,就不得不出战了。” 

但直到日暮,他也没有看到霹雳炮的出现,仅仅是来回试探,然后在反击下退走。 

辽人没有一击破城的打算,看出了这一点的并不知种朴一人。如此稳稳当当的用兵手法,一天下来,种朴觉得他的对手根本不像是传说中的契丹人,倒像是大宋这一边的将帅。 

“援军什么时候能到?”城上城下皆点起篝火的时候,李清问着种朴。 

种朴放下汤碗——现在他只能喝稀的:“赵经略知兵,知道什么时候派兵最合适。” 

李清眉头皱了一下。去年徐禧守盐州,种谔也是‘知道什么时候派兵合适’,最后的确是大捷。但徐禧死得干脆,满城京营将士也死剩下不到一半,城破时逃出来的曲珍都被追夺出身以来文字,削职为民。若是赵禼也学种朴的老子一般行事,那么溥乐城,乃至韦州的结果都不会很好。 

“溥乐城肯定能守住。纵使庆州隔得远,盐州可离这里不远。”种朴看了李清一眼,“还是先想想辽人今天的攻势。” 

喝了一肚子肉粥后,种朴召集他的参谋们总结今天的守城经验,并合计一下为什么今日辽人攻城给人的感觉这么奇怪。 

最后的结论是估计可能是昨天的偷营打乱了契丹人的计划,追击时算是吃了大亏,所以对面的辽军主将没有忍住。如果不是这个原因,种朴觉得,他们应该是等到霹雳炮打造好之后才会开始攻城。否则前两日刚刚抵达城下时,就会立刻攻击,而不是扎营围城了。 

另外还有一个猜测,说不定也有在正式攻城前,消耗一下党项人的想法。一群鸠占鹊巢的强盗,肯定不想看到原主老在眼前晃悠。大宋怕辽人煽动青铜峡中的党项部族。恐怕辽人也担心宋人盯上了贺兰山下残存的党项人。 

种朴越想越觉得这个猜测有道理,否则就不能解释辽人这一回的奇怪举动。 

边境上的冲突很常见,死伤个百十人,到了朝堂上也不过是打嘴仗而已。宋辽两国都各有各的难处,不可能贸然开战。眼下这种规模的攻势在其背后,肯定有着更深一层的意义。 

“为了区区一小队辽军,而破弃维持了几十年的澶渊之盟,难道里面有耶律乙辛的亲爹不成?!” 

一个参谋的俏皮话让所有人都笑了起来,这个道理是没错的。 

种朴不信事情会这么简单,李清自也不信。不过以两人的身份,想太多并没有意义,只要守住城池,剩下的就要看朝廷了。 

接下来的三天,就是单纯的消耗战,用党项人的性命来消耗城中官军的箭矢等守城物资。收获的首级算下来至少能让种朴和李清官阶跳上两三阶——这还是西夏国灭后,种谔和韩冈让党项人的首级越来越不值钱的情况下能得到的功赏。 

而韦州的援军也来了。 

虽说援军只有一个指挥的骑兵,而且来援的将领还是种朴的兄弟种师中——他在去年调去了甘凉路,但一个月前又调回了环庆——不过对于困守在城中数日的三千官兵们来说,州中和路中都没有忘掉他们,当然是一桩极为让人振奋的喜讯。 

可在种朴、李清的眼里,这更是城外的契丹人想要抓大鱼的表现。放过这个区区三百多骑兵,说不定就可以吸引来更多的援军。 

围点打援,是极常见的战术。灭了援军,毁了城中守军的希望,破城也不再是难事——即所谓的‘外无必救之军,内无必守之城’。 

相对于援军,另有一件事更为重要。 

就是城外的党项人遣使联络种朴,带着书信,说是要投诚。如果城中能给予一定的支援的话,他们甚至可以里应外合,在契丹人的背后捅上一刀。 

这件事让种朴极为心动,一举击败来犯的数倍辽军,自然要比守城功劳大上十倍。有契丹人的首级垫在脚下,面对谁都能高出一头来——这两年,折家的上一代十六和这一代的老大,可都是鼻孔朝天长了。 

但也不免有些疑虑,谁也说不准这是不是计谋。参谋们一阵合计,觉得还是慎重为好。虽说党项人可能是被逼上阵,死得太多而想要报复辽人,但万一是辽人的计策,上当后事情可就无法挽回了。 

种朴难得犹豫不决起来,便问种师中:“廿三,你觉得呢?” 

种师中想了想,道:“记得去年俺上京时遇上折家的老七,曾经聊起过胜州一役,就是斩首两万多的那一战。当时折七说了一句,‘死了的党项人,才是好的党项人。’这话据他说是从韩学士身边的人穿出来的。看黑山党项的下场,倒不像是乱说。” 

种朴听出了种师中话中的意思,皱着眉:“换成契丹人不也是一样吗?” 

“需要掺和吗?”种师中反问:“管他是哪一族,全都是蛮夷。以夷制夷没问题,帮着蛮夷打蛮夷那未免就太多事了。” 

见种朴还是犹豫不定,种师中更进一步的说道,“既然党项人想要投顺朝廷,至少得交一个投名状才是。一人带一个契丹人的首级来作证明,这个要求不过分吧?” 

“廿三。”种朴发着怔,“你今年到王舜臣那里转了一圈后,到底学了些什么啊!?” 

“非我族类,其心必异。”种师中冷笑了一下,凑近了一点,“十七哥,说句实话,你可别见怪。招降纳叛这件事,莫说是十七哥你,就是五叔,都不够资格!这群党项人若是愿意归附,让他们去跟朝廷说吧!” 

“朝廷吗?”种朴想着,或许应该可以相信如今两府诸公,至少他们不是逼死狄青,又打算还回绥德的文彦博。而且那里还有一个虽不入两府,但声名更胜一筹的韩冈。 

也就在这个时候,韩冈正拿着一封信往崇政殿中去。 

这是折干交给都亭驿中的仆役的密信,让其转交韩冈。在谈判陷入僵局,商谈的大门向萧禧关闭之后,折干这位正旦副使果然有了动作。 

