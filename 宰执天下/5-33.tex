\section{第四章 惊云纷纷掠短篷(七)}

苏轼抵达京城的时间比预计的要早。

韩冈还以为至少要到三月中旬以后,苏轼才会被押解入京城。谁想到御史台派去捉人的那一位,真是字面意义上的捕快,手脚麻利,办事利索。算算时间,很可能都没在湖州歇脚,抓了人直接就启程返京了。

“苏子瞻甫进京城,就押去了御史台。看着阵仗,御史台中不是李定一人看他不顺眼。”

“天子是圣君,能洞烛下情,苏子瞻仅仅是发发牢骚而已,天子当不至于重责。”吴衍说的话,恐怕自己都不一定相信,但他还是自欺欺人的对韩冈说着。

“希望如此。”韩冈见吴衍对有关苏轼的话题一番敷衍,也就识趣的不多提了。转过来问正事,“各路保赤局的情况怎样了?”

听了韩冈相问,吴衍精神一震,“北方各路转运司都有一名判官专一提举种痘之事。治下军州的保赤局,上半年都应该能设立起来。至于南方,厚生司中也已经派了专人带着痘苗去了各路。蔡元长刚刚领了命去江南,察访两浙、江南东西还有福建的保赤局事。”

“蔡元长去南方了?”

吴衍笑道:“蔡元长也是有点私心。他福建人,最是关心乡里。为了一个木兰陂,几次上书朝廷。”

韩冈看吴衍的样子,想来是被蔡京的表象给糊弄了。不过蔡京如此卖力,在厚生司中一任,赚到的功劳不会少。

又聊了一阵,吴衍见时间不早,便起身告辞。韩冈亲自送了他出门回来,就摇了摇头。

韩冈本以为吴衍会帮苏轼说两句话,或是请自己在天子面前帮着缓颊。吴衍跟苏轼似乎是有些交情,好像还有诗文往来。

但当初他不敢搏上一把,帮王韶对抗李师中、窦舜卿和向宝,最后与拓边河湟的盖世奇功失之交臂。这一次,他还是一样没胆子趟一下浑水。

不知这一次,苏轼的亲朋好友,有几个会出头帮他说话……韩冈倒是不怎么看好。

坐下来没一会儿,去审官西院办事的王舜臣也回来了。

“今天回来还真够早的。”韩冈笑问道:“没跟人去吃酒?”

王舜臣咧嘴笑了两声:“三哥你也说过了,现在正是风尖浪口的时候。俺还想去阵前杀贼,怎么也要做出一副悔改的样子,这样三哥和王都知才能帮俺说话。”

“本来还想提醒你的,没想到是白担心了……早这样多好。”韩冈摇摇头,道:“王中正在战前可能还要回京一趟。他一人身兼两路,天子要耳提面命一番才能放心。到时候你正好可以随他一起回去。”

王舜臣立刻喜笑颜开,一叠声的说道:“多谢三哥,多谢三哥。”

韩冈叹了一口气:“你也别急着谢。王中正肚子里面没货,到了天子面前,就怕他说错话。”

“有没有货,得看功绩。”王舜臣哧笑了一声,“赵括倒是一肚子的好料,上了阵全都拉出来了。”

“王中正的功绩可都是虚的,惟一能依仗就是他的福气了。”韩冈也笑了笑,“反正你和王处道跟着他,我也放心得下。”

“三哥你放心,该怎么做,俺都知道。何况还有王衙内在旁边提点,不会出差错。”

韩冈嗯了一声,点点头,王厚是聪明人,而王舜臣吃过一次亏后,为了官复原职固然会更加勇猛作战,但行事也会更加谨慎。

“王中正来回一趟再快也得要一个多月,四十天的样子。等他回去后就该出阵。这一战不会拖过四月。在五月之前,怎么样都要动手了。”

王舜臣点点头。

他也知道天子、朝廷和军中都想赶在冬天前结束这场战争,否则就得拖到明年,钱粮的消耗吃不消。不再六月之前开战,想在冬天前结束战争,时间上就太紧了,“银夏好说,一个月就能打下来。就是往兴庆府去,秦凤和泾原两路中间有几道关卡,可能会耽搁一阵。”

“不会耽搁太久。”韩冈道,“党项人要是聪明,当会选择坚壁清野,而不是在边境硬顶。”

“其实照俺说,这一仗应该开春就打。党项的马还没养起来,天气又好,正是打仗的好时候。过了四月,天就热了,大漠里面能晒死人。要是拖到秋天,党项人的战马就能养起来了,到时候又是一重麻烦。”

“都这时候,还说什么,哪有后悔药卖?!”韩冈摇头笑道,“种五想买都买不到!”

王舜臣叹了一声,他哪里能不知道种谔的想法。

以种谔的性格,他怎么可能让他人来分自己的功劳?吃独食还来不及呢。

一开始的时候,种谔的打算就是以鄜延、环庆两路的兵马为主,以最快的速度攻下银夏,进而直取兴灵。

王舜臣还没来京城的时候,韩冈就这么在想。等到王舜臣到了之后,韩冈一问,便证明了自己的猜测。

种谔本就是爱吃独食,什么时候愿意分功给别人。覆亡西夏的战功,他疯了才会送给别人。

可惜天子和王珪插了手进来,其他将领也不甘心让种谔独吞这块大饼。最后互相扯皮和妥协的结果,就是眼下的这个臃肿榔槺的作战方案。

韩冈叹着气:“种子正也是老用兵的,他不会看不出来朝廷调动了这么多兵力,实际上根本排不上大用场,反而是拖累。”

三十余万正兵,加上数目更多的民夫,号称百万已经是很谦虚了。但这一数量级的军队,对于领军的将帅来说,与其说是胜利的依仗和底气,还不如说是自家的灾难。

任何一名手上的军队并不是规模越大越好,人越多,问题就越多,传个令都不方便——能指挥的了得那才叫军队,指挥不了的那是累赘。

动员起超出了这个时代管理能力的军队,这是韩冈不看好这一次战争的另一个重要原因。

“幸好还是分作六路,各有各的指挥,各有各的粮秣来源,要是合兵一处,不用打就输了。没有哪条路能支持得了接近百万级的胃口。”

王舜臣不会质疑韩冈的说法。说到随军转运,韩冈是当世数一数二的能臣,他都说不行,自然是不行。

“不用三哥说,俺也能想明白。给百万大军运粮运草到底有多难,看看东京城就知道了。同样是一百万张嘴,汴河上船多的跟黄河里面的鲫鱼一样。”

按照后世的说法,这应该叫做边际效应,超过一定规模后,增加兵力并不能给战力带来相应的上升,反而因为兵力的增加,拖累起军队战斗力的发挥。

王舜臣当然不知道什么叫边际效应的,但他带了这么些年兵,自是明白手下的兵不是越多越好。

“人马上万,无边无岸,以陕西的地势,几千人就满坑满谷了。两边加起来几万人的厮杀,一年最多也只有那么几次而已。三五千人在一个山谷中打上一仗,这才是常见。人马再多,就顾头不顾腚,指挥起来都难。”

……………………

苏轼进了御史台坐监,想必李定是打算将这些年积攒下来的怨气好好的发泄一下。

韩冈现在也似乎是在坐监一般,军队的口粮不用他操心,但从军的草料却不能不去照管

陕西缘边五路,都从民间征调了大量的骡马等牲畜用来运送粮秣,数目大概在两万余匹。这些牲畜对群牧司来说很麻烦。虽说几乎都是取自民间,不花朝廷一文钱。但草料的还是要管饱的,总不能让百姓献了自家的牲畜,还要自备草料,这就是笑话了,也没有这么做事的道理。

可驴、骡、马和骆驼的胃口一个比一个大,从份量上说,平均起来是人的十倍。即便没有油水,人一天吃两斤米麦也已经足够了,但马和骆驼这样的大牲畜,又是在山区里面负重载货,胃口一开怎么算也得要二十斤草料。只这一件事,群牧司就又是得和地方上的转运司打嘴仗了。

韩冈皱眉看着永兴军路转运司发来要草料的公文,耳边还有下属的抱怨:“延州有草料场,华州也有草料场,永兴军路的草料场比洛阳囤积的还多一倍,哭个什么穷!”

韩冈将公文放下来:“沙苑监的草料记得是以三年的份量囤积的吧?”

韩冈的下属这些天来已经摸透了顶头上司的脾气,基本上就是前线要什么,他就会如数提供什么,一点都不带讨价还价的。

“龙图!”他惨叫道,“这可是牧监以防万一时的马料。”

“留一半下来,有一年半足够了,存得时间长了,不怕烂掉啊?”韩冈毫不在乎,“前面不是将沙苑监里十二岁以上的种马都送去了鄜延路吗?就当是这些马的口粮好了。”

‘怎么能这么算?’虽然都没有说话,但一个个挂下来的脸,都是在说着反对。

“你们也不想想,鄜延和泾原的马军不是跟着环庆路一起喊着没有骑乘马吗?等这一批运粮秣的牲口上去,正好不就有了?从沙苑监为此支取一部分草料,有什么舍不得的?”韩冈出言点醒几个下属。

