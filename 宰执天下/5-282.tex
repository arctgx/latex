\section{第30章 随阳雁飞各西东(13)}

萧禧这几天觉得自己的副手动作忽然变得诡异起来。

连着两三天与韩冈单独密谈不说,每天甚至都有一封信让人送回去。带出来的使团成员,一天就走上一个。明显的会引人怀疑的做法,竟然做得毫无顾忌,萧禧都想想了,折干他到底有什么依仗。

很可能是背着自己与韩冈达成了密约,但萧禧拿出正使的身份去质问的时候,折干打个哈哈就搪塞过去了,只说了一句不干国事。

不过萧禧也不是没有心腹,折干纵然对他自己的亲信再三训示,可萧禧派去的人打探了一阵后,还是得到了一些内情。

通过支离破碎的回报,拼凑起的内容尚缺乏足够的细节,可最关键的核心,萧禧已经了解到了。

韩冈竟然想要通过做买卖来贿赂耶律乙辛!不用岁币,照样能让尚父拿到真金白银,据说能有岁币的一半!

难怪折干胆子会这么大。若是这件事真能给他谈成了,把自己这个正使撇在一边也算不得罪名了。

但萧禧怎么肯甘心?!这件事怎么能没他这位正使?

只是当萧禧将搜集来的消息拍在折干面前的时候,折干眼皮都没跳一下:“我乃尚父帐下的宫卫提辖,这件事不干国事,是尚父的私事,林牙想要操心,那也得先投了尚父的斡鲁朵再说!”

折干冷笑着说罢,便扬长而去,只留下了气歪了脸的萧禧。

可折干的确是理直气壮。他是宫卫,说明确点,就是耶律乙辛府中家奴。萧禧能代表大辽朝廷说话,但他不能代表耶律乙辛。而折干,他却可以。

萧禧阴着脸坐了半夜,便开始给耶律乙辛写信。

成事难,败事易。

既然宋人和折干刻意排开自己,那干脆就踹破他们的美梦好了。

韩冈出的主意,分明是宋人新法的路数,将过去所有参与边境商事的富户豪门的钱都聚到尚父的手上。

对宋国来说,送钱给大辽和送钱给尚父是一个送,那还不如用来讨好尚父,轻轻松松就避免了岁币的恶名。

宋人心眼太多,但说来说去,他们终究不可能直接将钱送出来。只要从榷场中走一遭,在原本就有商队与宋人做买卖的各大族、各豪门的眼中,这些钱就应该有他们的一份。如果尚父独占下来,他们又会怎么想?

这不是明摆着的离间之计吗?只要点出了这一点,萧禧相信耶律乙辛会做出选择的。

……………………

萧禧正在给耶律乙辛写信,想要坏了韩冈的计划,可这时候的韩冈,已经觉得他的计划都乱掉了。

韩冈发现自己实在是太小看了商人们以及他们的后台对金钱的看重。

他的计划纵然对外要保密,也不可能瞒着要去辽国的当事人,赛马总社那边很快就得到了通知。有韩冈的做保,也让许多人对与辽国做生意这件事放心下来。当天夜里,会首们就坐在一起挑选起出使辽国的舌辩之士。何矩是顺丰行的代表,全程参与其中,听到他每天晚上传来的通报,韩冈最后也只能苦笑了。

在真金白银面前,让宋人畏惧百年的辽国也变得闪闪发光起来,充满了诱惑力。原本准备由赛马总社选出一人做代表,以购买赛马为名去辽国拜见一下耶律乙辛。但齐云总社听到消息后,立刻就明说要参上一脚。

齐云总社和赛马总社中拥有投票权的上层,充斥了宗室、外戚、勋贵和豪商,各大行会的行首亦是争先恐后的往里钻,实际上根本就是京城上流社会的俱乐部,只差挂出招牌来了。跟来来往往的官员完全不是一个体系。

不过顺丰行虽说来自于雍秦,但在其中涉足很深。毕竟整个利益链都是通过两大总社挂钩起来的,连接的极为紧密。所以韩冈在京城本土上层中的影响力,比王安石、韩绛、蔡确都要大,而且大得多。当然,也是因为韩冈在医学上的名声的缘故。

现在韩冈偏袒,将好大一块花糕也似的肥牛肉丢给了赛马总社,齐云总社的会首、副会首们一个个都红了眼,打上门来要分账——毕竟能像顺丰行一般,在两大总社中都占有一席之地的,只有极少数。

为了争夺出使的席位,两家总社整整吵了三天。华阴侯赵世将捋了袖子亲自下场,跟人争得面红耳赤,还将来劝架的邺国公赵宗汉骂了一通——英宗的这位幼弟,倒是跟顺丰行一样,两边都挂了名。他学何矩缩头躲一边倒罢了,站出来就是找骂。闹到最凶的时候,甚至还有不少人连夜遣了女眷入宫,请皇后主持公道。

最后的结果,去辽国拜见大辽尚父的不再是一个人,而是组成了一个有八个人组成的使团,各自代表两家总社一批人的利益。

可也正是因为这几天他们闹得太不像话,秘密已经不再是秘密,朝野上下都在议论。御史们暂时还在观风色,但随时有可能某个愣头青就跳出来了。

支持此事的两府觉得颜面无光,对商人们的看法更形恶劣,皇后也觉得自家亲戚变得满身铜臭实在很丢人。加上选出来的人太多,人多嘴杂容易坏事,都觉得需要挑一个总掌大权的,省得去了辽国给自己人丢脸。

韩冈不希望看到官员插手其中,但官员们对商人的不信任是根深蒂固,如果两大总社最后只选出一人来倒也罢了,可现在人数多达八人,他也没办法了,不能直接拒绝,只能相机行事。

所以就在赛马总社将使辽的人选呈报上来的第二天,崇政殿中又聚起了两府宰执,除了韩冈,甚至连新任的御史中丞李清臣也在场。

韩冈的提议,说起来也是有些犯忌。扩大与辽人通商的规模,虽然要比纳款献土好得多,但过于信重商人,在士林中肯定会引起清议的反弹。宰辅们不得不提防会有犯迷糊的御史们坏事。必须要事前跟风宪官通个气,免得最后闹起来大家都没脸——总不能再把御史台洗一遍吧?

“必须要有一名得辽人重视的大臣出使辽国,否则只凭几名商人,如何取信辽人?这几日你争我夺,在民间几乎成了笑话。言谈间不利财货,以此辈为使,岂不让辽人小瞧了中国?”

首相韩绛很少发话,但今天却是第一个站出来。京中若是跟耶律乙辛做起生意,抢得是河北大族的买卖,不过韩绛并不是为此而说话,实在是士大夫脾气作祟,对商人将国事弄得乌烟瘴气看不顺眼。

韩绛的话说进了向皇后的心里,点头道:“的确的选一个良臣去辽国,免得贻笑外邦。韩学士,你看呢?”

“的确得有人总掌此事。”韩冈还能说不吗,对那帮人他实在没话好说,不过他的话中还是留了余地。

向皇后见提出此事的韩冈都不反对,便问其他宰辅:“不知诸卿可有推荐?”

大概是商量好了,向皇后刚刚点头,张璪便着就说道:“侍御史蔡京如何?”

“蔡京?”向皇后记得这个人,侍御史已经不是小官了,前段时间御史台只剩下三五人时,就有他一个。但对蔡京的经历和才能,向皇后却并不了解。

“蔡京在厚生司判官的任上曾出使过辽国,传授种痘法。”张璪说道,“在辽国亦有声望,更曾见过耶律乙辛。”

原来如此。向皇后不由点头,听起来的确是个不错的人选。她透过屏风望向韩冈:“韩学士,蔡京在厚生司中行事如何?”

“臣提举厚生司时,蔡京业已擢迁。”韩冈很想说蔡京不适任,但蔡京可是年年课最,考绩一年上下、一年中上,直接减了一年磨勘,在中书门下都有记录的,“但从衙中遗留文牍看,其人甚为称职。”

无论是从能力,还是经历上看,蔡京当然都是最好的人选之一。曾经出使过辽国,又是在辽国国中主持传授过种痘事务,有着很好的人缘。才学也是第一流的,在中书里也做过事。

但他不想让蔡京掺和此事,倒不是担心日后的六贼之首蔡京从中伸手捞钱,或是因此而积攒功劳,而是怕他在此事中偏帮福建商人——蔡京对乡里的照顾很是有名,为了家乡修建木兰陂,他可是多方奔走——这就会坏了韩冈让雍秦商会和京商联系更加紧密的计划。

“蔡京的确是个好人选。但他现在可是侍御史……”韩冈很快就找到了一个借口。现在没办法直接反对,先拖人下水好了。

也不知是韩冈催逼,还是自觉自愿,御史中丞李清臣果然站了出来。

“此非是待遇儒臣之法!”李清臣是个好帮手,只见他声色俱厉:“御史者,诤谏天子,监察百官。宰相欲令为商人奔走,朝廷欲以此来待遇儒臣?!”

韩冈本意就是要逼李清臣站出来为御史台说话,否则台中下属的口水能将他淹死。这也正好可以帮了自己。李清臣现在说出这番话,让韩冈如愿以偿。

不过李清臣的话让韩冈听着还是很不舒服,合着儒臣就不需要做正事了?

还以为现在是旧党在台上吗?
