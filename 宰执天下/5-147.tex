\section{第15章 自是功成藏剑履(九)}

战争结束了,又是将及年终,也到了对这一场战争做个总结的时候。

平夏之役,大宋的付出很多,损失也很多。收获不少,但让人捡了便宜的地方,同样不少。想到给辽人不费吹灰之力就占了一半的好处去,从天子到百姓,都是觉得憋闷不已。不过不论心情如何,该做的事还是必须要做。

朝廷对于战争的功赏,在赵顼的催促下匆匆忙忙的决定了下来,又赶在过年前发了下去。对于参战的士兵和底层军官们的功赏,朝廷没有太过计较,罪疑惟轻功疑惟重的道理人人明白。甚至对河东军的两万斩首,最后也没有进行太过严格的查验,而是全盘承认下来。但对高阶的将帅,他们得到的赏赐,完全是由需虚衔和财帛组成,并没有太多实质上的东西。将将领和士兵割裂开,可以有效的避免有心人引发兵乱。

其中最为惹人关注的韩冈,他得到的封赏,基本上就是虚名。一如御史台一轮弹劾之后世人所预料的,被拒绝在西府之外。而且顺便还连累了种谔。原本同入西府呼声甚高的种子正,也只能饮恨回归三衙,做个殿前司副都指挥使,就连本官官阶也止步于观察使,不但没能拿到节度使的名号,甚至连节度使留后都没有得到授予。王中正更是只有一个防御使。

实在的也有,韩冈六子,眼下全都得了荫补。其中他长子和次子的本官已经比照宰执家的长子,跳升到了京官序列的太常寺太祝。一般的情况下,臣子都会先拒绝一次两次,但韩冈直接就上了谢表。

“想不到韩冈都没有拒绝?”

“御史台还等着说他是心怀怨望呢,他怎么敢辞?”

“难道能不怀怨望?眼下的那些官职全都是虚的。”

“除了宗室、皇亲,没有任职过宰辅的臣僚中,眼下又有几人在勋位上能与韩冈相提并论?”

“也到顶了。等河东平靖之后,明年多半会调回京中任个闲差。”

韩冈不知道京中的议论,但他对自己得到的封赏并没有太多计较。

通奉大夫、检校工部尚书,上护军,东莱郡开国公,食邑四千户,食实封一千两百户。除了最后的食实封,可以让韩冈每个月多得到三十足贯的额外俸禄,其余得以升迁的散官、检校官、勋、爵、食邑全都是虚头,官、职、差遣都没有动,依然是右谏议大夫、龙图阁学士、河东路经略使兼太原知府。而且在河东经略使这边的便宜行事的权力,也因为战事的结束,而一并给去除了。

赵顼摆明了就是要用抬高虚衔来抵换将韩冈拒之于西府门外的不公。韩冈对此根本没有放在心上——尽管他本以为还能援引王韶的旧例,从龙图阁学士晋升为资政殿学士,不过即便没有如愿,也不是很在意。他眼下更关心自己在民间和士林中的声望。

王安石负天下重望三十年,但在他入朝主持变法前,民间知道有个王安石的可不多。但韩冈的名声,可是远布四方,随着厚生司推广种痘法的脚步,越来越响亮。而这一次受到的不公待遇,也让百姓们为之惋惜。朝堂上人事变化的因果,世人无从知晓,相对于没怎么听说过的宰辅,一个有着响亮名声的能臣,自然更为受到期待。

名望是个好东西。任职地方,可以顺利的掌握政务,号令也能得到遵循。更重要的是,韩冈主张的学术观点,得到的认同和学习,也随之变得更多。

韩版的三字经如今已经在陕西的蒙学中流传开来,就是因为有韩冈大力推介的缘故,比起正常的传播速度,要远远快出百倍。这就是名声带来的好处。而研习气学、格物的士子,越来越多。秉承牛痘的原理,以及由显微镜证明病毒之说,对免疫学的研究,也成了医学中最时兴的课题。

过了腊八,接连两场大雪如愿而至,让河东一路诸多军州的官民都放下心来。有了年末的瑞雪,明年的收成也就有了保障。府衙之外,已经可零星的听到了鞭炮的声音,过年的气氛,也渐渐的浓烈了起来。

后花园中的积雪,已经家里的几个孩子兴奋了好些天。而在银装素裹的庭院中,衙中听命的胥吏们,正忙着清理地面上的积雪,很快就在庭院的角落里,堆积起了一人多高的雪堆。

不过大雪带来的并不都是好消息。处理公务的西厅中,韩冈正听取帐下幕职官司户参军的汇报。

“城中因积雪垮塌的大小房屋共计二十六间,除了七人有些皮肉伤以外,尚幸没有更多的人员伤亡。”

韩冈一边听着报告,一边翻了一下手上的官文,眉头就皱了起来:“怎么昨天有人冻死?没有收容到通慧庵旧庙里面。”

“这两人是喝酒后在路旁醉倒,以至于被冻死。已经从酒家那里确认。家人也已经将尸首领走了。”

“官文上没有写啊。”韩冈说了一句,无奈的摇摇头,酗酒的情况在烈酒出现之后变得严重起来,尤其是北方的州县,每年冬天的夜晚,因为醉酒而被冻死在路边的酒徒人数,已经快赶上乞丐等无家可归之人了:“这就没有办法了。让夜里的巡城和更夫,经过酒馆的路上多照看点吧。”

“下官知道了。”

“城外的情况怎么样?”韩冈放下报告,又问道。太原城中是由府衙直接管理,城外的乡村,才是阳曲县的管辖范围,正如开封府之于东京城一般。

“明天阳曲县应该能将雪后的灾情报上来,再过个三五天,榆次、交城几个县也都能将情况传回来了。盂县离得最远,不过龙图之前已经下令让各县随时汇报,最多七天,雪后的伤亡情况也该到了。”

太原府的司户参军是积年老宦,对政务处理得心应手,让韩冈很是满意。

西厅内的对话传到外间,正在整理公文的黄裳和折可适都听得一清二楚。

“也只有这个时候才会有雪灾。”折可适低声细语,“这一年可就要过去了,比起往年,路倒的饿殍可是少了九成以上。”

“想想这一年,事情还真多,不过总算是有个了结。”黄裳望着窗外的雪景:“明年当会有个好年景。”

“外面都在赞着龙图的治理之功呢。”

“这当然是龙图的功劳!”

因为种痘法的普及,开创者韩冈在民间的声望极高。他临危受命,经略河东,心思多放在战阵上,太原府的政务其实在许多地方都有所疏漏。不过崇高的声望让他在战事之余,处理州中之事时,节省了许多口舌上的纠缠。

而且韩冈利用大批黑山党项作为劳动力,将路中数以万计的可能被动用的民力给节省了下来。今年夏秋的战事,也因韩冈的举措,没有太过干扰到河东百姓的生产和生活。

百姓是淳朴的,但又是精明的,战争的意义在他们之中没多少人会了解,但会给自己带来什么样的影响,不需要太多见识都会明白。韩冈的作为,在秋冬之季,让多少百姓将一颗提起的心放了下来,名声和人望都上了不止一个台阶。

“嗯?”正在检视文牍的黄裳突然出声。

“怎么了?”。

“划界使。”黄裳举起手上最新从京城收到的政事堂省札,“朝廷要派划界使来了。”

“谁是正使?”

“韩玉汝。”

“韩缜?!怎么又是他!”折可适失声叫了起来。

“可不就是他。”黄裳将手上的公函递给折可适,“从之前河东划界的先例来看,至少还要一年两年的时间。”

折可适看了两眼之后,眉头就紧紧皱了起来。

这是今年的第三批国信使。宋辽两国分割吞并西夏后,亟需将西北的新国界彻底划定下来。前一批使臣在耶律乙辛那里谈崩了,主要还是胜州之战的结果传到了南京道的缘故。不得已,朝廷换上了新的一批使臣。正如省札中所说,是由翰林学士韩缜领衔。

“当年韩内翰主持河东划界,可是一让再让,从山下退到分水岭上,十几间巡铺都给让掉了。还有上万百姓,也不得不内迁。当真对得起那份俸禄。”折可适撇嘴冷笑着。

“这件事不是韩玉汝的错。”从内间的大门处传来韩冈的声音。

“龙图。”黄裳和折可适忙站了起来,回头就看见韩冈从内厅走出来。

看着两名幕僚一眼,韩冈摇摇头,为韩缜辩解了一句后,却不再多说,将司户参军送到厅门前。

熙宁八年的时候,韩冈当时就在东京城。被辽国使节萧禧逼着割地到底是什么情况,他可比折可适要清楚得多。赵顼三番几次下诏,这完全不是韩缜的责任。

但也没有必要再纠缠过去的事,毕竟时至今日,即便是赵顼,也不会再对辽人的讹诈感到心惊胆战,而不敢应对。

