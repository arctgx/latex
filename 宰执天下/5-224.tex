\section{第24章 缭垣斜压紫云低(16)}

向皇后呆呆的坐着,望着灯罩中的烛火,完全没有丝毫睡意。巨大的压力压在肩头,而原本能为自己和儿女们挡风遮雨的大树,已经衰弱不堪,倒了大半,也不知道什么时候就会彻底枯萎。

该怎么办才好啊……

向皇后的眼中只有跳动的烛火被映照出来,玻璃灯盏的透明度远远高于纱罩,所散发出来的灯光,甚至能给人一种耀眼的感觉。向皇后形容憔悴,呆呆看着那团火焰,许久也没有移开眼睛。

“圣人。”一名女官脚步仓促走到向皇后的身边,“官家好像有些不对劲。”

向皇后愣了一下后,方才反应过来。她立刻霍然而起,转身的时候左脚甚至绊倒了右脚上,幸好有随侍身侧的女官扶了一把,要不然就会摔个结结实实。

分开围在床榻边的几名嫔妃,向皇后坐到了赵顼身边。

方才已经入睡的赵顼才过了半个时辰便又醒了过来,正在不停地眨着眼睛,乍看起来很像是要传递些什么。

见赵顼并非发病,向皇后提到嗓子眼的心脏又落了回去。但眨眼到底是为了什么,却让人一头雾水。她俯身看着丈夫,原本蹙起的双眉皱得更紧了几分。

“官家,是不是有什么事要吩咐?”向皇后轻声的问着。

只能看到赵顼在用力眨眼。

“可是要喝水?”

赵顼还是在眨眼,喉咙中还咕噜咕噜的直做声。

“是担心朝廷上的事?”

依然是眨眼做声。

向皇后一个问题接一个问题,却只见赵顼的眼皮越眨越急,发自喉间的声音也变得更急促,但光是眨眼和哼哼根本就不能传递信息。

向皇后急得头上生汗,嫔妃、内侍和宫女们都是心中发急。到底怎么才能领会天子的心意,寝殿中有三十多人,但眼下却没人能想出个招数来。

百般无奈,宋用臣吞吞吐吐的开口问道:“圣人,是不是招韩学士来?”

向皇后想了想,就摇了摇头。韩冈要有能力让官家开口,或是有办法让人明白官家的心意,方才就出手做了。若是他只有居中转达的手段,以韩冈的聪明,则是怎么也不会做的——非但无法取信于人,也没有任何意义。

一应人等愁眉不展,留守的御医已经被招进来了,但他们也同样是束手无策,只能看着皇帝发急的眨着眼睛。眼皮开闭间,看不到节奏,让人弄不清到底是在传递心意,还是突然发病的征兆。

正是想不出一个眉目的时候,近门处有个尖尖细细的嗓门突然开口,“圣人,不如用韵书!”

一句没头没脑的话,却犹如弥漫眼前的浓雾被狂风卷曲,让人眼前顿时一片光明。宋用臣手一拍,仿佛遭到当头棒喝一般失声叫道:“对了,正是该用韵书!”

寝殿内的数十道视线,也在同时转去方才那个声音冒出来的方向。却是一个面目清秀,只有十一二岁的小黄门,一对眼睛灵活得很,看起来便是聪明伶俐的样子。

向皇后已经想明白了韵书的作用,看了那小黄门一眼后,便立刻吩咐道:“快拿韵书来……就去官家的内书房找。”

宋用臣立刻点了一名管理内书房的黄门,就见他小跑着的出了门往书房去了。

去拿韵书的人走了,赵顼眼睛也不在乱眨,喉间的声音也静了下来,平平静静的躺在床榻上,胸口随着呼吸微微起伏。

在感觉到了这一变化之后,一时间,所有人都欣喜如狂。官家果然是清醒的,仅仅不能说话罢了。

向皇后和几名嫔妃立刻围到榻边,望着赵顼时眼中无不闪着泪光。过了好半天向皇后方回过头来,问那小黄门:“你叫什么名字,现在跟着谁做事?”

小黄门当即跪了下来,口齿伶俐的回话道:“回圣人的话,奴婢杨戬,现在御药院中听候差遣。”

“杨戬?”向皇后念了一下,像是要记住这个名字。又看看杨戬身上的衣袍式样,又问:“还到祗侯殿头了?”

杨戬身上的衣袍不是有品级的内侍公服,而祗侯殿头是内侍无品杂职——也即是小黄门——的最高一阶,再往上一级,便是从九品的黄门。

“回圣人,奴婢现在是内侍省内品。”

内侍省内品比祗候殿头低了有四五阶,虽然下面还有几个内侍官阶,但在地位甚高的御药院中,就没有更低的了。

“且升做黄门,以后就在福宁殿里服侍吧。”

杨戬立刻跪倒谢恩,俯下去的脸上欣喜欲狂。

御药院名义上是管理宫中藏药和药房,但在天子身边服侍的大貂珰往往都要兼一个御药院都知的差事,是宫中不多的几个能接近天子的地方。但对于普通的小黄门来说,在福宁殿听差,时时能见到贵人,才是往上爬的终南捷径。今天壮着胆子的一句话,就升到了从九品的内侍官,日后只要小心办差,更进一步也不为难事。

一部《大宋重修广韵》很快就被拿来了,向皇后拿着书坐在赵顼身边:“官家,妾身用韵书搜字。想要说什么,是就眨两下,不是就眨一下。”

赵顼眨了两下眼睛,看起来是在表示自己听明白了。

《广韵》两万六千字,分为平上去入四部,两百零六韵。利用韵书能做的,也就是按部分韵的数着字来看赵顼想说什么话。

只要能读书作诗,常用字属于哪个韵部肯定是一清二楚,赵顼发病前是不用担心,现在用韵书,传话之余,也能确定皇帝的神智是否清晰。

“妾身想问问官家现在可有不适?”向皇后翻开书页:“可是上平?”

天子的双眼眨了一下。

“下平?”

还是眨一下。

“上声?”

眼皮开阖两下。

向皇后随之精神一震。上声共有韵部二十九,下面就该在这里面数了。

“可是韵部在一董到十贿中?”

一下。

“是在十一轸到二十哿中?”

两下。

“是上声十九皓中的‘好’【注1】?”朱贤妃抢着问道。

赵顼的眼皮又眨了两下。

两下一个‘好’字出来,向皇后也罢,朱贤妃也罢,包括了嫔妃、女官和内侍,所有人脸上都露出了笑模样。赵顼既然说感觉还好,即便有大半是安慰人,但情况也不会太差。这样的交流虽然很麻烦,但总比

无法交流要强出百倍。

“官家有没有什么要吩咐的?”向皇后再次翻起韵书,“上平?”

一下。

“下平?”

一下。

“上声?”

两下。

“韵部一董到十贿?”

两下。

向皇后便一董、二肿、三讲、四纸的一个个韵部数过来,全都一下。最后终于数到上声十贿中的‘宰’字,赵顼的眼睛终于眨了两下。

宰有屠的意思,但赵顼现在肯定不是想要让人杀只鸡宰只羊来。

“是将宰相招来?”向皇后问道。

赵顼睁大了眼睛,一下也不眨。既不是‘是’,也不是‘不是’,向皇后只能再翻起韵书。

然后得到的是上平一东中的‘同’。

到这里,所有人便都明白了,向皇后自不例外。她问道:“官家,宰执大多已经出宫了,现下还在宫里的有王珪、薛向,可要先将他们一同招来?”

又是两下。

向皇后正要吩咐人去,但想了一想,却又回来问赵顼,“……韩冈也在,是不是也一并招来?”

两下。

向皇后略松了一口气,这个时候,韩冈是决然慢待不得的。若是丢下他,只找王珪和薛向来,就不知他心里会有什么反应了。

抬起眼,方才那个伶俐的小黄门就在眼前,向皇后便点了他:“杨戬,去请王相公、薛枢副和韩学士来。”

……………………

“小的杨戬,奉旨来请相公,枢副和学士。”

见到一名陌生的小黄门跑来传话,王珪和薛向全都失了仪态:“可是天子出事了?!”

韩冈也坐不住,一下站了起来。

杨戬连忙道:“相公勿忧,是皇后用了韵书,可以跟天子说话了。来请相公、枢副和学士,也是陛下旨意。”

韵书?

王珪、薛向脑筋转了一下,都有恍然大悟的感觉,只要能眨眼,用韵书的确就可以交流了。回头看看韩冈,都有几分佩服。

韩冈则很意外,赵顼竟然真的还清醒着,看来之前自己的一番拖延时间的表演,竟然给撞个正着。

既然天子已经可以间接的下旨,王珪、薛向哪里能坐得住,连看都不看这个杨戬一眼,急急的就跨步出门。韩冈则稍稍多看了一眼,只因为前面这小黄门自报的姓名。

杨戬。

韩冈觉得这个名字莫名的耳熟,想了一下后,便记起《西游记》和《封神榜》里的二郎神好像就是叫这个名字。

不过那是几百年以后的流传了,在这个时代,崇信灌江口二郎显圣真君的人甚多,两年前在京城万胜门外重建的灌口二郎庙里的香火,如今只比大相国寺稍逊。六月二十四的灌口二郎生日,其热闹程度,同样也仅比天子的寿诞和四月初八佛陀生日略输一筹。

但杨戬这个名字,却跟二郎神搭不上钩。现在的二郎神姓李,是筑了都江堰的李冰的二儿子,被仁宗封作灵应侯的便是。

凑巧跟后世的二郎神,刻意在王珪、薛向和他韩冈面前提到自己的名字,看来这个小黄门,也不是个甘于平淡的主儿,肯定是想要拼命往上爬的那种人。有机会就抓住机会,没有机会就创造机会。

注1:这其实是宋末才出的平水韵,韵部与广韵有别。不过一时间能找到的就是平水韵,只能用一用了。

