\section{第30章 众论何曾一(五)}

大宋的太皇太后自从十九岁入宫,基本上就再也没有出去过。深居这小小的一方天地,几十年来,她的脚走过的地方也不过宫城之内,还有京中的几处园林而已,但她每天都要活动一下腿脚。只是今天,曹氏只是绕着宫室走了一圈,越发的感觉到自己的腿脚变得不灵便了,“真的老了。”

刚刚坐下来,就听着外面有人通传:“太皇,濮阳郡王家命妇求见。”

曹氏听了,就有些不高兴。她对于濮王一系好感不多。她是仁宗的皇后,英宗只是过继来的养子而已。可英宗即位后,先是缺席仁宗皇帝的丧礼——好吧,这是病!所以她开始了垂帘听政——但之后赵曙病愈亲政,又开始闹着要追赠生父赵允让为帝。最后闹出一摊烂事,害得自己都在宰相面前哭诉过。要不是赵曙有着个孝顺守礼的好儿子,曹氏当真是想过将他给废了。

这段时间求到她这边的有不少,不过地位最重的濮王家的人都只敢捎带上一句,真正去的地方还是高太后所居的保慈宫。毕竟太皇太后对濮议的心结谁都知道,硬是上前来触楣头肯定没有好结果。

但此事已经过去了多少年,不好再放在心上。既然来了人,也不便不见:“让她们进来吧。”

赵顼这段时间真的头疼欲裂,这新的一年也就刚开始的两三天轻松一点。

刚刚在经筵中否决了王雱的建议,但文彦博的奏章还挂在心上,要怎么解决大名府六十万石的粮食缺口又是一个麻烦。而每天传到自己的求情声,也让赵顼无法得到清净。

赵顼是个孝顺的皇帝。对祖母和母亲的晨昏定省,从来不会忘记。从崇政殿出来,他就先往慈寿宫过来。尽管保慈宫近上一点,但如果现在去向母亲问安,去肯定能看到一群哭诉的妇人。相对而言还是太皇太后这边清净一点。

不过慈寿宫中还是有着两人在,赵顼认识她们,是他二伯家的人。只是她们见着皇帝过来,却在行了大礼之后,连忙告辞出去。求着太皇太后就够了,直接求到天子面前,反而没了转圜的余地。万一皇帝一口否决,金口玉言就会像钉子一样,将要救的人钉死在牢中。

赵顼向祖母行过礼,就听曹氏说道:“也只有官家来了,这边才算安静一点。”

赵顼愤然道:“都是为了那一干奸人,也不想想败坏了国政,对他们有什么好处。”

“官家打算从重处置?”

赵顼摇摇头,沉默的叹了口气。

“官家,老身出身武家,读书不多,但旧年却是一直在看着仁宗皇帝如何行事。”曹氏的话让赵顼侧耳静听,“仁宗皇帝惯守法度,事无大小,悉数交由外廷议定。”

“这个未免有些……”赵顼欲言又止,要是真的这么容易,他何必头疼。

曹氏看着孙儿,温声说道:“官家仔细想想仁宗皇帝的庙号因何而来。”

赵顼明白了,恶人让朝臣做,自己来加以宽恕。只要将其稍加宽纵,就能换来仁恕的名声。

不过这也只是和稀泥的做法,终究上不得大台面。自己此前也不是没有考虑过,只是不愿意就此放过那一干毁了天家名声的奸商。但现在看一看,也罢,还是糊弄过去好了。世上本就没有万全之策,能糊弄过去的办法许多时候已经是最好的选择。

赵顼低头向曹氏谢道:“多谢太皇教诲,孙儿知道该如何去做了。”

……………………

身为宰相,王安石却并没有传染上皇帝的苦恼。

对于那一群借着年节入宫谒见天子和两宫的时机,为大狱中的奸商们求情的宗室,王安石现在根本就不放在心上。民心所向,他不信奸商们还能翻盘。

王安石过去可是没少拿宗室开刀,先是说着‘君子之泽、五世而斩’,将天子的远亲全都从宗正寺中除名,只给太祖、魏王等几房留下一脉来承宗祧。后来的均输法、市易法,无不是砍在宗室们的经济基础上。

由于太宗得登大宝有许多值得商榷的地方,宋室天子对于宗室的提防一代代都没有松懈过。不论是将宗室们摒弃于朝政之外,还是刻意将宰相的排位置于亲王之上,无不是借用着士大夫的力量来压制宗室。

多少年下来,如今的宗室都是攀附在皇权之上,有影响力但没实力,才会在得到天子支持的宰相面前根本做不到正面抗衡。他们能做的也只是设法去动摇天子的决心,而不是能够像文臣一般强硬起来能逼得皇帝改弦更张。

要求情的尽管去求情好了,但如果天子想要将他们轻轻放过,王安石绝对不会允许!

抄没来的百万石粮食难道还能还回去?!向着天下亿万兆民承认朝廷这一次做错了,奸商们日后尽管可以囤积居奇好了,朝廷不会因此降罪的!

这完全是个笑话,年前因为粮价高涨而引发的市面萧条,其所带来的民怨尚未消散。若是将三十七名奸商轻轻放过,京城百姓们的怨气就会聚集到天家身上。更别说囤积居奇的行为如果不受的惩治,将会给日后带来多少恶劣影响!

作为宰相,有着三十年官场经验的王安石,地方上的情况他比天子了解得还要深入,从地方官员奏章看到的东西,也要比连东京城都没出过几次的天子多上许多。

京师乃天下之中,东京城的物价波动,理所当然的会影响到地方上的物价。当京城中物价一倍两倍向上翻到时候,京东京西、乃至两淮等地,物价也都是跟着向上急涨,而当奸商们锒铛入狱,中原各路的物价却又同样的在短时间内应声而落。

现如今,地方上的商人们都盯着这一桩案子。如果不能给予足够的处罚,他们必然又会兴风作浪。尤其是如今的灾情一步步的加重,商人们的得意必定会让百姓受尽盘剥。这一点,是王安石绝对无法容忍的。

心中有了定见,今日不当值的王安石就很平静的坐在书房中,一切就要看皇帝如何决断,然后才能决定自己要该怎么去做。

京城物价的危局刚刚结束,而流民尚未大批南下,上元节之前的这些天,对他可说是难得的休息时间。趁着闲暇,王安石将这两个月耽搁下来的《三经新义》拿起来开始审订。

《三经新义》是王学一脉对《诗经》、《尚书》和《周礼》【也称周官】的重新诠释。其中《周官新义》由王安石本人负责,差不多要成书了,厚厚一摞手稿就放在桌面上。王安石字如其人,急性子的脾气到了纸面上,便是如同斜风细雨,一笔行草透着峻急。

不过王安石今日正在考订的并不是自己的手稿,而是由王雱所编写的《尚书新义》,另外一部《诗经新义》则是由吕惠卿领头撰写。

“武王胜殷,杀受,立武庚,以箕子归。作《洪范》。”王安石批改的正是《尚书》中的《洪范》一篇。

洪范九畴,传说是传为箕子向周武王陈述的‘天地之大.法’,乃是以《洛书》为本源。在《汉书》中,就有‘禹治洪水,赐《洛书》,法而陈之,《洪范》是也’的这么一段话。

但经义局对于《洪范》一篇的重新注释,着眼点却主要放在利义之辩。

《洪范》九畴,就是九条治理国家的基本原则。其中第三条的八政,说的是治国的政务手段。而八政之中,食排第一,货排第二。食货之事,自然与利有关。既然三代之时,将食货放在八政的前两位。那么利之一字,当然就是朝政之根本。

其实这也是盱江先生李觏的见解。王安石的学说也有很大一部分来自于李觏。作为南方大儒的代表,李觏一改旧时儒门重义轻利的理论,而将利放在与义平齐的地位上。

不过李觏所说的利是公利,而非私利,要‘循公而灭私’,并非是杨朱的拔一毛而利天下不为也的自私自利。

王安石的观点亦是如此,秉承他教诲的王雱也是如此在书中如此写到:‘以利和义,而非以利抑义。利者义之和,义固所为利也。’

王安石看着正入神,王雱却回来了。抬头见着儿子脸色郁郁,王安石便问道:“出了何事?”

王雱坐下来将方才经筵上的经过说了一通,又道:“要是天子肯答应此时,流民将不足为患。”

“天子不可能主动让流民进入开封府地界的。”王安石摇头,他比经验不足的儿子看得要清楚,“京师外和京师内是两回事。就像京城内和京城外一样。让玉昆去白马县,不就是为了不让流民进京城吗?”

王雱无奈:“当初就不该将滑州并入开封府。”

“那样由谁来掌滑州呢?治事能如韩玉昆的可不多。”王安石笑了笑,“有文宽夫在大名府,流民还是要南下的……”

