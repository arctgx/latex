\section{第34章 山云迢递若有闻(15)}

达成了共抗宋人的秘密盟约,又同木征商议了一些细节问题,禹臧花麻便起身告辞离开。

他与木征今天达成的协议,实质上是取得了木征对禹臧家染指武胜军北方地区的认可,让他得以吞并掉武胜军北部与兰州接壤的部族和土地。有了木征的点头同意,对于北面的许多蕃部,禹臧花麻攻打和吞并他们,将是名正言顺,并不用担心其他地区吐蕃部族的反弹。

禹臧家掌控武胜军北方,而木征则直接控制武胜军的洮西地区。两家一起出力,将宋人的统治区域,遏制在临洮城周围二十里地范围内。

武胜军中,凡是有可能投靠宋人的蕃部,两家都会组织兵马全力铲除。并不需要他们出动多少本部兵马,禹臧花麻和木征都是准备利用其他部族的人马,消灭所有附宋部族——只要不抢到自己身上,这里的蕃部都会把兔死狐悲的心思给抛到脑后,而醉心于这等没本钱的生意。

禹臧花麻有把握,只要栽这些部族一个投靠宋人的罪名,就能不惹起其他部族反弹的情况下将他们剿除。到那时候,他会再看一看临洮城的宋军会不会为他们出头,如果坐视,有几家还会再投靠宋人?但若是宋人会为之出头,战事一起,钱粮的消耗可就要海了去了。

木征和禹臧花麻已经确认了对方的想法,他们都不会跟宋军硬拼,只求能消耗宋人的钱粮,让宋人在武胜军难以支撑而不得不撤离。至于他们自己,都是准备将本部主力撤回,选留精锐督促此地的蕃部作战。等宋人师老兵疲,再从中寻找取胜的战机。

营门处,随行的从人牵着马正焦急的等候禹臧花麻出来,浑身绷得紧紧地,手都安在刀柄上。周围的木征家士卒,则都是用着不善的眼神盯着他们。自从禹臧家投靠了党项,两边的仇怨在几十年间的已经成了死结,要不是因为宋人的威胁,禹臧花麻和木征根本坐不到一处来。

见着自家的族长被人礼送出来,一干从人终于放松了。只是又立刻紧张得提防着周围,防着木征军士兵会对禹臧花麻不利。

禹臧花麻只觉得好笑,回身向送他出来的木征行礼。他今次若不是有了万全的把握,如何会孤身入敌营?

木征的形势比他恶劣得多,如何还会再得罪他禹臧花麻。正如他对木征所说,无论兴庆府能不能支援他,禹臧家至少还是西夏的臣子,而木征家背后又有谁?

难易有别啊!

在彻底解决河湟之前,宋人应该不会去动他的兰州。

对于宋廷的既定战略,禹臧花麻和木征其实都很清楚。王韶平戎策中的内容,这两年早在秦州以西传开了,都是针对自家的计划,只要有些风头传出来,没哪家蕃部会不重视,会不去着意打听。

既然知道宋人的计划是先定河州,禹臧花麻在与木征的面会上当然就能很顺利的占到上风,但他也不会太过分,木征的底线,禹臧花麻无意且也不敢去触碰。

因为他需要木征把宋人在河湟多拖上两三年,至少得等背后的大夏国稍稍缓过气来。

只是……禹臧花麻更清楚,党项人对兰州的垂涎不止十年八年了,即便靠着与木征的密约和协议拖住宋人的攻势几年,但河州终究还是难守,等几年后,宋人北侵兰州,能帮自己抵抗宋人的党项军,会不会得寸进尺的在兰州盘踞下来,禹臧花麻心中也没底。

眼下在兰州城中,其实也有一支党项本族的铁鹞子,虽说被自己死死压住,也说不准哪日就会里应外合。

禹臧花麻翻身上马,离开木征的营地,犹自暗叹,‘这个族长做得还真是让人头疼。’

……………………

王韶并不知道木征和禹臧花麻的密约,但他从最近木征的行动中,看出了一点不对劲的地方,“木征在对岸扩建城寨了。”

高遵裕不以为然,“纯属浪费力气,在霹雳砲面前,有几座蕃人的城墙能支撑下来的?”

“所以说才让人想不通。”王韶难以理解木征的做法,“我们现在虽不会过洮西,但眼下冰层渐厚,到了隆冬,不是木征他杀过来,就是我们攻过去。他修城寨又能如何?即便没有听说过霹雳车,难道木征以为官军就没有其他攻城的手段吗?”

王韶想不通木征的想法,吐蕃人有修筑城池的传统——这点跟喜欢住在城外帐篷里的契丹人不同——但在离洮水只有十里不到的地方增筑城寨,等于是跟紧贴洮水东岸的临洮城针锋相对。

为了保护临洮城的安全,正常情况,他也需要在洮水对岸修筑一座小寨堡,以增强临洮的防御能力,并且保证临洮守军对洮水的绝对控制——就像有了襄阳,还需要修汉江对岸的樊城;控制了江宁,还需要据有长江对面的六合。

而木征紧邻洮水增筑城寨,等于是明摆着要于此驻屯大军,不会让宋军跨过洮水一步。

难道他真的有心与官军决战不成?!

王韶最终还是放弃了去猜测木征的想法:“先把临洮城修好,再修好南北门户的南关堡、北关堡。安稳住临洮南北,再向西去跟木征打个交道。”

“最好还能在抹邦山那条路上,也设上一两处寨子。好歹修一下都能行车,又通向渭源和岷州。”

王韶苦笑着摇头:“真要连路都修上,没半年时间都完不了工。”

高遵裕想了想,便放弃了。临洮本就耗用无数,再拖上半年时间,缘边安抚司哪有那么多钱粮。却道:“玉昆那里的情况好像不错。现在他那里的两千民伕,已经大部移到野人关了,庆平堡只留了两三百民伕在那里筑营房。”

“玉昆手脚是麻利,听说他在罗兀城也出了不少的力。”

“可惜罗兀城还是给烧了。”高遵裕笑得幸灾乐祸,突然他又想起了什么,从自己的桌案上抽出一份公文,“对了,玉昆昨日移文来说,野人关名号粗鄙,想要换一个名字。不如子纯你给起个吉利的名字好了。”

“哪有那么多吉利名号……既然通向大来谷,直接叫通谷堡好了。”王韶起名字不想用脑筋,都是随口一说,庆平堡如此,现在的通谷堡也如此。

“那就叫通谷堡。”高遵裕也没什么反对意见,他提笔在公文上把通谷堡三个字记下,又随口说道,“不知这座临洮城最后会给改成什么名字,希望能吉利一点。”

边塞大城的名字不是他们这些边臣能随便起的,得由朝廷赐予嘉名,许多时候还是天子来拍板。比如甘谷城,初名是筚篥城,修筑时的临时名称是大甘谷口寨,最后就是如今的天子赵顼给定下了甘谷这个名字。

“别管朝廷想叫什么,城筑好再说其余。”王韶在座位上翻起了账本,见着上面一条条用红色记录的支出,咂着嘴叹道:“这钱粮花得如流水一般啊……”

临洮城比渭源堡的路程远了百多里,单是筑堡的花费就当即翻了一番。当初修渭源堡时,钱粮问题已经是让缘边安抚司殚思极虑,最后是连蒙带骗的干掉了不顺的蕃部,同时把渭源堡给修起来。现在虽说朝廷的支持与旧时不可同日而语,但看着几十万贯转眼就没了踪影,王韶也不免心生感叹。

“可筑堡的进度还要加快,我都想着是不是要移文转运司,请蔡运使再征发一批民伕来。”

“不能了……”王韶摇起头,“宁可多花钱,不能再征发。再增添民伕,明年秦凤转运司能送来的粮食就很难保证了,不能弄得跟白渠一样。粮食比钱重要。”

“要不要让蕃人来帮忙?”高遵裕又提议着。

“就不知道他们能不能做事……”

王韶和高遵裕正为钱粮人手在苦恼着,忽闻帐外通报,韩冈在外求见。

“玉昆,你怎么来了?!”王韶和高遵裕都惊讶的看着不请自来的韩冈。高遵裕更是站起来急急的追问着:“可是出了什么大事?!”

韩冈点点头,“下官从俘虏的嘴里听到一个消息,在后面坐不住。文牍传递又浪费时间,干脆直接过来了。”他笑了一笑,“野人关离临洮又不远,不过两个时辰的脚程而已。”

“是什么消息?”见韩冈神色轻松,王韶的心放下了一点来,问着:“是禹臧花麻又在弄鬼不成?”

“禹臧花麻?!”高遵裕惊问道:“他难道又去抄截粮道了?”

“不是!”韩冈摇摇头,“两位安抚误会了。韩冈刚刚听到的这个消息,是说岷州那里有铁矿。”

“这事不是早知道了?”王韶奇怪的问道,“瞎吾叱和结吴延征两家的兵甲在蕃部中都算得上第一流的,不是有铁矿如何能有如此的装备?”

“但事先得到的消息中,可从没说过岷州铁矿的规模……那是远远高过我们事前的预期。”韩冈双眼灼灼发亮,“如果运作得好,一年百万斤生铁也是等闲。”

“百万斤?!”高遵裕先是难以置信的摇了摇头,但立刻他又重重捶了一下桌子,兴奋起来,“如果是真的,那就可以开军器院了!全军的刀剑甲胄,直接就可以在河湟这里措办。”

“不,不是开军器院。”王韶摇摇头,直盯着韩冈,“玉昆,你说呢?”

“军器院当然也要有,不过当务之急却是……”韩冈与王韶异口同声:“钱监!”

