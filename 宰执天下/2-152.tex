\section{第29章 顿尘回首望天阙(四)}

【没能如约把欠下的两章补上,这里要说声抱歉。不过不会忘的,肯定、一定、必定会补上。】

京城的冬天寒冷干燥,一不注意,双唇就会开裂。如今的京城中人,就算是平民百姓,到冬天都会弄点牛油或牛骨髓制成的口脂来抹唇。不仅仅是女儿家,就是男子也在使用。香气馥郁的油膏不但能保护双唇不受冬风侵袭,其香味也能给人以好感。

本来世间男人用的都是无色的口脂,但后来许多京中的浮滑浪子和不学无术的衙内,甚至用上了女儿家专用的红色口脂,来妆点自己。周南对这样画着女妆的惨绿少年丝毫没有好感,甚至觉得恶心,而文武双全、英气勃勃的韩冈,行事又体贴,才这般容易扣动了她的心弦。

许大娘心浮气躁的瞪着周南不紧不慢的动作,胸口一起一伏,仿佛台风降临前的汹涌波涛。最终她还是勉强收起怒气,柔声上前陪着好话:“乖女儿啊,今次就别闹了。要是秦二官人来了怎么办?他可是从来只点你来作陪!”

周南毫不理会口脂被轻轻抹在唇上,粉色的唇瓣一点点的被艳红所掩盖,轻轻抿了抿小嘴,鲜红欲滴的双唇如樱桃般诱人。只见周南回身说道:“今次下帖的章官人是中书五房检正公事,王相公的心腹戚里。既然他,哪能”

许大娘终于忍不住了,尖叫道:“你是真不知,还是假不知?!那可是雍王!也不看看你那个刚刚做官的穷措大,跟雍王哪里能比?!”

拿起梳子的手抖了一下,周南的心一阵阵抽紧,的确,跟天子的弟弟比起来,韩冈的地位的确差得太远。要不是昨日墨文带回来的消息,周南此时已经绝望了。

不过现在有着韩冈的承诺,她倔犟的脾气便毫不服软:“女儿只知是秦二官人。说是二大王,还要看到紫袍玉带才知道是不是。”

许大娘怒火中烧,脸上厚厚的敷粉绽出了一道道口子,仿佛遭受了地震的墙壁,一片片的开始崩落。她想拦着周南,但周南现在名声已经出去了,已经不是任打任骂的幼时。门外就有章家派出来的家人等着,王相公身边的红人,不是她一个教坊司教习开罪得起。总不能把雍王拉出来跟章惇打擂台。许大娘很清楚,雍王赵颢是绝不会跟那些见过他的官员们打照面的。

周南站起身,叫上自己侍女:“墨文,我们走。”

听见院墙外的车轱辘响起,又渐渐的远去。

一声尖叫穿得老远,砰砰的脆响,在房中不停的响起,“真真是气死老娘了!傍上一个芝麻官,看你小贱人能有什么好结果!”

……………………

夜色中的樊楼,灯火辉煌。欢声笑语伴随着婉转动人的曲乐,还有着一股醉人的融融暖香,一起在楼阁间浮荡。

前次韩冈被章俞在樊楼宴请,那时是在中午,虽然客人依然为数众多。但直到韩冈现在看见如被繁星点缀的五座楼阁,以及站在围绕天井的阁楼外廊上,上百名打扮得花枝招展、等候客人点选的妓女,才真正的体会到何为樊楼春色。

李小六从进来后就一直张着嘴,土包子的模样让人发噱,一直到有人上来迎客时,他都没回过神来。韩冈则是见识多了,随意赞叹了两句,报了章惇的名字,便被恭谨有礼的侍者引了进去。

韩冈抵达二楼的一处包厢前,章惇和路明便殷勤的迎出来,笑容可掬。只是见到韩冈身边只跟了李小六一人,他便奇怪的问道:“怎么不见王子纯?”

“王安抚刚刚被天子遣使传进宫中去了,留话要韩冈代表歉意。”

就在方才,韩冈和王韶正准备出门的时候,从宫中来了中使,把王韶叫进了宫去。王韶是朝官,本有资格上殿,天子要见他,也没人能阻拦,王韶也不会推托。至于章惇的宴席,就只能作罢。

这是不可抗力,章惇无奈点着头:“也是……这两日王子纯就要回关西,官家要见他也应该的。”

章惇虽是这么说,但他和路明的脸上,都有着一点失望之色。韩冈倒不以为意,王韶比起自己,可更是炙手可热,理所当然的更受欢迎。

照规矩留了李小六在外面听候使唤,三人一起进了厅中。

包厢内装潢之华贵,器物之精美,自是不必怠言,又有莺莺燕燕七八人,皆是娇艳如花,色艺为一时之选。娇声道着万福,向韩冈三人一齐行了一礼。

可韩冈的注意力,却是被一身素雅的周南所吸引。虽然周围的官妓都是上品容色,但脂粉淡抹的周南,明显更甚一筹。美目含情,犹如一汪秋水,射出情丝,就像丝萝一般紧紧地缠绕在韩冈的身上。

“久别胜新婚,玉昆你与周小娘子今日重逢,倒是热得我们没处站了。”

俗谚道赌场无尊卑,酒桌无大小。而到了欢场之上,其实也很少有人再摆谱,讲究着身份。章惇笑着调侃韩冈和周南,韩冈也是笑着拱手回应:“说到缘起,我俩还要多谢检正你这大媒才是。检正现下热得没处站,可不算是作茧自缚?”

韩冈毫不避讳的当众承认他和周南关系,周南的胸臆顿时就被一股幸福感所充满,芳心一阵狂跳,胸口发胀,仿佛要开裂一般。眼眶也红了,滚热的液体就从脸颊上划过,泪水竟是毫不自知的就流了下来。

周围的妓女也都是一下兴奋的轻呼起来,一片声的赶着恭喜周南。周南赠匕定情的故事在教坊司中无人不知,今日章惇宴客,请得周南的心上人来,她们都想看看究竟是何方神圣,夺了花魁的芳心,把雍王殿下都比了下去。而韩冈不负所望,年纪虽轻,但前途不可限量,相貌气度亦自不凡,更重要的是一颗真心,就已经远超诸多嫖客。

章惇看了这一幕,犹有深意的问着韩冈:“玉昆,我这大媒做的你真的不怨?”

韩冈笑了一下,章惇要问什么他很清楚,“德容双全,韩冈谢还来不及,怎么会怨?至于那些扰人的琐事,也不需放在心上。我韩冈虽是鄙薄,却也知信义二字,从不负人。”

“好个从不负人!”章惇拍手赞着,他也是豪爽不羁的性子,韩冈的作派,的确是太合他的脾气,而周南出淤泥而不染的贞烈也正得他敬佩。至于韩冈本人都不在意的琐事,他也不会放在心上,天子亲弟,怎么也不可能出来跟人争风吃醋。

拿过一柄酒壶,一盏银杯,章惇给韩冈满满的斟上一杯樊楼特产的和旨酒,“玉昆此言,当浮一大白。”

韩冈接过酒杯,正待要一饮而尽,却有人在旁边拦着,“这酒岂是韩官人一人喝得?”

韩冈一愣,却见拦着他的路明向章惇使了个眼色,又朝正被众女恭贺的周南呶呶嘴。

章惇像是一下被开了窍,哈哈大笑:“说的也是,交杯酒哪有一人喝的道理。还不请周小娘子过来。”却是要让韩冈和周南喝这交杯酒。

一阵哄笑声中,周南赤红着脸,低着头,小步挪着硬是被推了过来。原本很大方的性子,现在却满是羞怯,与韩冈面对面站着,头始终都不肯抬一下。而周围的起哄声,更是真的像是在闹洞房一般。

这个时代闹洞房的事,韩冈也见识过。两支交椅背靠背,上面架个马鞍,把新郎赶上去坐着,不喝满三杯不给下来——在前身留下记忆中,一直致力于恢复上古礼仪的张载,也曾经向学生们抱怨过,如今的婚礼越来越不成样子了——只是起哄喝交杯酒,还真是算不得什么。

此时的交杯酒并不是后世的交臂对饮,而是各自把杯中酒喝下一半,然后互换了酒杯,再把对方的残酒都喝光。

一名妓女倒了酒,硬塞进周南的手中。

彼此间呼吸可闻,在周南脑海中,周围的声音全都静了下去,消失无踪。周南她怯生生的抬起头,对上了韩冈坚毅的双眼,“官人,这杯酒……”

韩冈性子爽快,也不多话,一仰脖喝了半杯下去。他把酒杯放在周南面前,微笑着,默不作声的等待着周南的回应。

周南看着递到眼前的只有半杯残酒的酒杯,还有稳稳握着酒杯的韩冈的右手,泪水又忍不住的流下来。

她抬起手,将自己的酒也喝了一半。抬头灿然一笑,纯美的笑容如百花绽放,刹那间闪过的艳色摄人心魄,让韩冈也一阵目眩。

各自喝了半杯,便换了杯,两人对饮而尽。而周围不知是谁起了头,有人开始唱着诗经中的古曲,“桃之夭夭,灼灼其华,之子于归,宜其室家。”

琴声也叮咚叮咚随之响起,“桃之夭夭,有蕡其实,之子于归,宜其家室。桃之夭夭,其叶蓁蓁。之子于归,宜其家人。”

这是先秦时恭贺少女出嫁的歌谣,正是合着眼前一幕。而此曲,又是出自诗经中的《周南》一篇,让韩冈和周南回想起了两人初见面时的对话,不由得相视一笑。

