\section{第36章 沧浪歌罢濯尘缨(13)}

河东的战局一直牵动着京城百万军民的心。

代州陷落的消息传来,很多东京的富户都开始做起了南下避难的准备。

随着韩冈就任河东置制使,随着一支又一支驻扎在京畿的禁军北上河东,表里山河的战局终于渐渐稳定下来,胜利的天平也开始随着时间的推移,向大宋一方倾斜。

不过纵使在扭转颓势的太谷大捷之后,也没人认为大宋官军能够彻底击败入寇的辽军。绝大多数人都只希望河东能够与河北一样,遏制住辽军的长驱直入,守到辽军不得不撤退为止。

而之后河东官军与辽军僵持在忻代一线,正好达成了人们的希望,由此也逼得辽国权臣耶律乙辛派人来开封和谈。

局势的变化符合众人之意,只是谁也没想到还会有官军直取大同府的机会。辽军在大小王庄的惨败让京中官民一时为之失语,同时也让很多人看到了夺回幽云诸州在太行山以西部分的可能。

只是局面的变化再一次出人意料,在河东主帅韩冈的主持下,宋辽双方转眼间就达成了和议。

前几日京中曾有传言,是皇帝故意以封王之诏相逼,所以韩冈为了日后的前途着想,才放弃了攻取大同的打算。

在民间,这个谣言并没有传扬开,庆幸的人还是占大多数,可在朝堂上,相信了这种说法的人就有很多了。

一些官员觉得韩冈私心太重,让大宋失去了一个绝好的机会,另一些人则是觉得比战前不仅没有损失土地,还捞回了一个武州,也不算亏本了,而且很赚,只是话语中依然免不了要为不能赚得更多而感到遗憾。

但为韩冈辩护的声音还是有的。

“官军的气力已经到了极限了。负重行远,三日而竭,九日而亡。官军在河东数月,日日枕戈待旦,席不暇暖,纵使多在营中休息,又何曾能得半夜安寝?夺回代州乃是以轨道为助力,可轨道如何去得了大同?”

“为什么韩枢密能刚到河东便于太谷城下大败辽贼,并非其有鬼神之助,而是辽贼深入河东千里,已是人困马乏。却因为得到韩枢密的消息,鼓起余力连夜南下。却为枢密挡在太谷城外。”

“击败困顿城下的贼人就只需一羽之力。现在反过来,累的是官军,以逸待劳的是辽贼。”

“高粱河殷鉴不远,易州之败更是近在眼前。”

“这话怎么听起来这般耳熟?”蔡渭细眯着眼,隔着花墙传过来的话,好像是在哪里听过一般。

邢恕提起酒壶,一边给自己和蔡确的儿子斟酒,一边笑道:“宗汝霖在报上的原话,也就改了几个字,能不耳熟吗?”

“宗泽?”蔡渭不屑的从鼻中哼出一声冷笑。

那位出身两浙、却在战前游历过河东的年轻士子,自从成了河东战事专栏作者,便声名大噪。对于河东战局的分析远比他人更加几乎成了人所共仰的军事大家。

钟离子和楚仲连的名号甚至传扬到了边陲。据说不止有一名边臣具礼延请,希望能聘宗泽为幕僚。不过宗泽都辞以学业繁忙、无暇分身。

只是在更高的层次中,对宗泽的看法则是截然不同。

在很多朝臣看来,京中声名鹊起的年轻谋士不过是一个传声筒,只是某个人想要在京城说些以他的身份不方便说的话罢了。

“不过是韩玉昆养得一条好狗,名声倒是直追武侯、王猛和赵韩王[赵普]了。要不是看着韩枢密的面皮,早就把他给赶出国子监。”蔡渭冷笑道,也不在乎声音让隔邻的酒客们听到。

邢恕抿了一口酒,啧了啧嘴。

宗泽在齐云快报和逐日快报上的多番评述,对河东战局的分析可谓是精到。要不然也不会让那么多人信服。但他文字中的细节其实混淆了事前事后的差别,让河东的战果显得不是那么惊人。

在战后分析出辽军的败因很简单,但在战前就判断出辽贼已经成了强弩之末,同时还不惜以自身为饵——有此判断的难有此决断,有此决断的难有此判断——这正是名帅和庸人的区别所在。

可宗泽的话并不是在贬低韩冈的功绩,而应是韩冈的自晦之道。以他的身份,不能学人自污,也只能自晦了。

“其实何正通的说法与宗泽大同小异,不过他觉得河东方向还是犹有余力,如果再得河北、陕西配合,夺取大同并不是不可能。”

蔡渭放下酒杯,皱起眉:“何正通?章子厚要荐其入武学教书的何去非[注1]?”

“的确是他。”邢恕点点头。

虽然说何去非的名气远不如宗泽,可在国子监中亦以知兵著称。刚刚崭露头角便被章敦网络入幕中,已经有动议要将他和宗泽一并推荐入武学担任教授。

“武学教授可武职可文职,只不过白身无功受荐,入不了文班。就不知他们愿意与赤佬同列,在三班院中做个吃香的殿值了。”蔡渭嘴角扯动,幸灾乐祸的笑着。

武学的成员并不是以武将为主,而‘使臣未参班并门荫、草泽人并许召京官两员保任’,没有品级的为入流武官,无法得到荫补的官宦子弟,甚至是白身的平民,只要有京官推荐,就能进身武学,至少得到考核入学的机会。

但就是因为需要有京官以上的文臣保荐,使得许多有能力的底层武官无缘武学,反倒是一些无能之辈,依靠家中的背景,被荐入学中。

纵然武学的毕业生能够被选派为小型寨堡的寨主、堡主,但有根基的将门子弟,不需要经过武学,也能升任。而没有根基的武学毕业生,也没人敢把重要的职位交到他们手中。自从熙宁五年武学重启,到如今已近十载,可陆续毕业的武学学员,在这十年间的频繁战事中,却都没有什么出色的表现。

“那两人偌大的名声不过是事后空谈得来,要是应敌破贼都是动动嘴皮那么简单,家严也不需要日日殚精竭虑,镇日坐守在政事堂中,唯恐战局有所变动。……不过是马谡、赵括之流。”

“邢恕也听说过。就是韩玉昆本人,也是在说:退敌逐寇,不在奇谋,只在用心。”

蔡渭想了一想,用力的点了点头。

他看过枢密院与河东置制使司之间往来的文书。遇贼兵当如何,守关隘当如何,行军当如何,运粮又当如何,枢密院提出的每一条条款都把战事中要注意的细节都提点到,而置制使司中的幕职官以及韩冈本人的幕僚,都要负责其中的一个部分,并给予朝廷一个让人信服的回答。

从行军到食宿,从武器到甲胄,从寝具到医药。光是要安排十万人马的衣食住行,就能把人弄疯掉,除此之外,更还要作战。

邢恕不比蔡渭,有个好爹和好岳父。而且他为了自己的名声,近来也少进蔡确的宰相府。所以没机会看过枢密院和置制使司来回递送的文牍,但当年西夏猖狂时,针对陕西缘边各路的防秋事宜,朝廷都是说了又说,那些旧文牍,他倒是见识过——划一指挥八,检举指挥十一,仅仅是防秋事,其所虑之处,已是无微不至。

但这并不是枢密院授阵图遥控前线将领作战,虽然太宗皇帝和今上都喜欢玩这一手,不过在西府中,明智的重臣还是占绝大多数,都只会是指示需要注重的方向,具体的战术安排,朝廷不会干预,而是交托给前线的将领们。

“不说这些了。”邢恕见蔡渭没有谈论这方面的兴致,便改了话题,“现在战事已了,就不知吕、韩二位,哪一位能先进京了。听说吕枢密已经在运作了。”

“韩枢密也不输人。”蔡渭笑道,并不隐瞒蔡确私下里跟他说的话,“他可是要求朝廷移民忻代,以保河东北部早日安定。”

邢恕也听说了这件事。

韩冈用不再追索叛臣为代价,逼耶律乙辛放弃对武州。河东在北线多了一条可以通行的道路,河西的麟府诸州与河东本土连接得也更加紧密。

不过从另一个角度来说,从此以后河东多了一个必须设重兵防守的战略要地。尤其是武州的山势走向,有很大一部分是对北方敞开了大门,实际上想要守住,其实不容易。

要想做到这一点,只有一个办法——迁民。

代州、忻州急需更多的移民。武州也同样需要。

“以韩玉昆之言,忻代武到底要多少户?”

“至少万户。”

武州群山汇聚,真正可用的土地只有河畔的谷地,规模很小,最多也只能分设两县,甚至撤州改军的动议说不定都已经放到了政事堂的议程表上。可再小也是一军州之地,给大宋的四百军州又增添了一个成员。没有个三千户口,根本维持不了正常的生产生活。

另一方面。代州、忻州在辽贼入侵之后,人口的损失很大。不说要回到旧日的富庶,仅仅要想恢复到旧时本州粮草自给的局面,也必须增加六七千户口。

注1:中国古代兵书《何博士备论》的作者。
