\section{第11章 飞雷喧野传声教(16)}

跟过去以内侍和外戚为主的皇城司统领不同,如今执掌皇城司大权的是来自于陇西的王厚。

生长在京城中的内侍、外戚,只会盯着朝臣,以及市井中那些似有异心的言论。可是换作关西,类似于皇城司的监察系统,所有目的,全都是对外。

大宋、辽国、西夏之间互派细作的情况太正常了,尤其是关西这样的常年交战的地区,每时每刻都有探子越过边境,或者说,只要往来于边境上的,全都是探子。尤其是那些回易的商队,住在边境上的七岁小儿都知道,上上下下全都是细作。

从陇西调来的人,受命搜捕城中细作,打头第一桩便是去探来自于河北、河东两地的商人和商队的底细。

王厚说他们嗅觉好,那是一点不错。辽人细作身上的味道,完全瞒不过王厚那几位心腹人,转眼之间便揪出了几个。

接下来如何审问、深挖,就不是韩冈和王厚所要关心的事了,他们只要下面的人给出答案。

“不过这一回挖细作,都亭驿也派了人过去,枢密院那边怎么办?苏枢密会不会觉得皇城司手伸得太长了?”

王厚拿着酒杯问道,事涉职权,他不免要为下面的人担心。

对辽外交,由于南北并立,一向是枢密院的自留地,归于密院中的礼房管理。而大宋周边的其他国家,无论是西夏,还是高丽,则都是属于大宋的朝贡体系,向大宋朝廷称臣。与其官方往来,在三省六部的体制中当归于鸿胪寺,理所当然是在政事堂的掌握中。

这一回皇城司的动作,是奉了韩冈的命令,也就是政事堂,从枢密院的角度来看,可不就是侵夺职权?

“不用担心。”韩冈则摇头道,“苏子容岂会在意这等小事。”

“西府里面又不只苏枢密一人。”

尽管苏颂跟韩冈的关系不差,章敦也应该有点交情,但那是私谊,而皇城司侵占的却是公权。

当年新旧党争激烈的时候,东府是以王安石为首的新党说话,而西府则是吴充等旧党盘踞,御史台有名御史上书天子,要求枢密院都听从政事堂,而朝廷中也有流言说天子正这么考虑。王韶虽然不愿与王安石交恶,但也跟着吴充一起封印回家,整个枢密院都罢了工,这件事是王厚亲自经历过的。

西府可能容许皇城司侵夺公权?

韩冈哈哈笑道:“皇城司又不是东府辖下,处道你担心什么?”

又不是东府侵夺西府权柄,自不用担心。只要不盯着朝臣,谁还管皇城司看着哪边?

王厚将探事司丢给了向太后的堂兄和回朝后同提举皇城司的李宪,自己则只管亲从官和反间谍的事务。王厚的这番作为,让他在朝堂上少了不少敌人。

并非政事堂那边侵占职权,主事的王厚又如此识趣,皇城司就算有点冒犯,枢密院那边也不会太过计较。

韩冈不会相信章敦、苏颂会如何为难王厚,甚至曾孝宽,性格也是比较宽和的。

真正重要的还是抓到人,将京师里面的细作扫清,韩冈不指望能够将之一扫而空,不过不大动干戈,如何体现哪几门火炮的重要性?

自己这边越是重视,想必辽人也会更重视一点。

再多说了些许闲话,喝光了三壶酒,韩冈让人备了车,送了醉醺醺的王厚回去。

韩冈酒量不大,今天算是比较节制了,可起身后也有些头晕脑胀,平日里多喝葡萄酒,为了配合王厚的口味喝了烧酒,一时间身体也习惯不了。

素心见了韩冈的样子,忙着去厨房做了些醒酒汤来,当她端着一盅热汤过来的时候,就看见韩冈推开了窗户,站在窗前,望着一丝星光也看不见的夜空。

房中的暖意都给夜风吹散了,素心放下醒酒汤,走到韩冈身边,小声的问道:“官人,夜里外面冷,还是先把外袍披上。”

“用不着。”韩冈抬手将窗户关上了,回头道:“又下雨了。”

……………………

下雨了。

从张家园子出来的左禹望着天上皱着眉。

不是没带雨伞或是雨衣,而是来自上面的命令让他很头疼。

今天晚上的宴会上,有关那几门巨型火炮的消息,从开席一直被说到酒席结束。。

左禹仅仅是起个头,以河北边州人氏的身份多问了两句,就引来了一个晚上的吹嘘。

直到散了席,耳边才总算清净了一点。

这几日左禹赴宴,有关禁中火炮是被议论最多的话题,大辽的国使成了最大的丑角,而那几门火炮,已经被吹嘘成了一炮糜烂上百里的神器。

如果有可能,左禹真想去都亭驿问一问耶律迪,他要的是不是这些消息。
