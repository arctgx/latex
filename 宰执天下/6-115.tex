\section{第11章 飞雷喧野传声教(12)}

就像所有进入辽国境内的大宋使者一样,耶律迪在越过白沟,进入宋境之后,便开始用心记忆起宋国境内的道路来。

不过也正如其他来过宋国做使者的同僚们所说,宋人绝不会带自己走最快的道路南下,肯定会绕来绕去。

几天下来,至少有两次,耶律迪发现自己的行进方向已经完全偏离了南方,而是向东或是向西前进。而其他时候,也不是稳定的向南,同样是大幅度的偏离。

从南京道到宋国的都城开封,除了河流之外,并没有其他险阻。现在记下来的道路,却如此蜿蜒曲折,可见对日后入侵宋境根本没有用处。

“直接走大路不行吗,当真以为没有我们不认识路?”

“要不是尚父得顾及背后,哪里会给宋人捡个便宜去。”

“南朝自己也知道,否则何必这么胆战心惊。”

当第二次转向东行,背后传来随行官吏们的议论声,听得懂契丹话的接伴使脸色难看,而耶律迪则越发的笑得开怀。

看到宋人的作派,耶律迪信心满满。像大辽这样从来就不会在路程上做什么手脚。宋人的使者过了白沟驿后,一路向北,直至虎北口,绝不会绕上半点路。

“扯什么闲话,都闭嘴,也不看看场合。”

让伴当将话传下去,像是在训斥下人,却让迎接的宋官脸色越来越黑。

耶律迪却根本不在乎,悠然自得的看起了道路两侧的风景。

一畦畦田地,从官道旁延伸到天边。举目可及之处,皆有田垄交错。村寨随处可见,往来行人不绝。

宋人的富庶,第一次以最直观的形势,展现在耶律迪的眼中。

辽东的平原也是如此广阔,但论起人烟稠密,却是输得老远。

在耶律迪看来,只有析津府周围才能与之相比。可这一路上,皆是远离州城,只是普通县治下的乡村,距离河北的中心大名府还不知多远。即便如此,已经能够与五京道中最繁华的南京治下相媲美,要是到了南朝的北京,又会是如何的繁华?

但直到黄河边,耶律迪也没看到大名府。

无论在这百多年间,辽国派了多少细作,将河北的道路已经探查得多么深入,宋人都不会将北方的第一重镇,暴露在辽国的使臣面前。

冬日的黄河,没有传说中的汹涌,凝固在一片素白中。

站在黄河金堤下,耶律迪分外感觉到了自己的渺小。厚重高大的堤坝,向两头延伸过去,望不见头,望不见尾。

仰起头来向上看,得扶住帽子才行。金堤顶端,比一路行来的官道都宽,而大堤底端的宽度,又是顶端三四倍还多。

要是宋人的哪座城池的城墙能有黄河大堤的规模,任何情况下,耶律迪都不会动起攻打那座城池的念头。

幸好这是上百年不停地增筑而成,南朝每年都会将黄河金堤加厚一点,但只有对黄河才会如此,就算是东京城,南朝也不会年年增筑。

要是南朝是在短短几年内修起几千里如此规模的大堤,那样才值得害怕一点……

不。

耶律迪用自己的双腿丈量着黄河大堤时忽然摇头,南朝要是当真调集那么人力去修筑河堤,不用打南朝就完蛋了。

想到这里,他就暗恨起来,要是前一次南侵入宋境的几支人马,有一支能稍稍大着胆子南下黄河畔,驱使宋人掘堤放水,南朝这两年都别想有好日子过,自己的任务也能更加顺利的完成。

过了黄河,就是进入了宋国的中心。

耶律迪也不知是不是错觉,只觉得过河之后,天也似乎黯淡了下来,不再明净高远,总是灰蒙蒙的。

但道路两侧,的确更为繁华,更胜了河北一筹。有城墙的是县城,没城墙的是集镇,不过不论是县城还是集镇,都是同样的人烟稠密、行人如织。耶律迪还想拿国内作比较,可他想了半天,始终想不到哪个地方能比得上这里。

不过耶律迪自进入开封府界后就一直心烦,没空去看周围的风景。使节团中的成员在离开国境前都经过了警告,不要失了了大辽的体面,在河北的时候,都还表现得很好,不过在进入了开封府界之后,那些下人就像是没见过世面的土包子,伸着脖子左右张望,差点路都不知道该怎么走了。而自己一行过来之后,街上的宋人不过是避让。但丝毫没有畏惧,反而隔着老远指指点点。

一直没被耶律迪放在眼中的宋国接伴使终于找到了机会,“看来贵属当真喜爱本朝风物,若林牙和贵属有何需求,但可直言,只要能做到,在下必尽力相待……待林牙回国后可就难见到了。”

接伴使似是好意,但他的这种高高在上的态度,让耶律迪看得很不顺眼,“让员外见笑了,孩儿们是见得少了,日后有机会多来几次,习惯了就好。”

耶律迪与接伴使微笑着对视了一眼,皆在对方的眼中看到了掩饰不住的厌憎。

两国的关系仅仅是维持在一纸盟约上,彼此都恨不得将对方破城灭国,此时的友好往来,不过是给血淋淋的战场掩上一层白布。

又经过了两天的时间,东京开封府的城墙终于出现在耶律迪的眼前。

望着盘踞在西面地平线上的煌煌巨城,耶律迪一时不解,明明是从北而来,却自东接近开封。

到底是怎么绕过来,耶律迪还不清楚,不过他倒是明白为什么宋人要遮瞒。

在开封城北面的远处,有一片地方正冒着滚滚的黑烟。到了开封府后,耶律迪就觉得在这里连呼吸都不自在,而这污浊烟气的来源,便是那一处巨大的铁场。

那里是南朝出产钢铁的地方,诸多神兵利器,如斩马刀、板甲的原材料,都是从那里运出。每年产铁万万斤,是大辽全国产铁量的数倍,而这仅仅是南朝几个大铁场中的一个。

可惜宋人绝不会让辽国的使者去那里探查一番,所以就不可能从北门进城。

不过耶律迪也没那个兴致,大辽的铁场虽小,但装备契丹精骑已经足够了。宫分军、皮室军,甚至一些头下军,都装备上了铁甲、马铠。

这样的精锐,只要数千便能灭掉一个百万丁口、十万大军的国家。而大辽境内,可是有着十万以上的具装甲骑,以及数目更多的轻骑兵。

这可不是宋人在短短十数年间,就能弥补得上的差距。

同样是铁,重量又相当,可农夫手中的锄头,就是比不过勇士手中的钢刀。

离着东京城墙还有一段距离,周围就已经看不到田地了。只能看见连片的屋舍,连片的仓屯,还有一座座黑色的由石炭堆成的小山。

“那就是石炭场?”耶律迪指着不远处的黑色小山,问着身边的人。

“……是。”接伴使很是勉强的应了一声。

耶律迪明白,这一位肯定是猜到了自己想说些什么。

就是因为石炭场起了火,南朝前一任皇帝才会被自家儿子给闷死了。才六岁的小儿,当然不会为皇权而弑父,纯粹的意外。这件事在辽国国中传开,便被视为是前世冤孽造成的结果。不过南朝派来的告哀使所携带的国书中,只是简单地说了一下因病身故,并没有说出真相。

“阿弥陀佛。”耶律迪念了一声佛,“贵国先帝猝然晏驾,鄙国皇帝与尚父也是感同身受,收到消息当日,便开始辍朝,祷祝三日,以求冥福。”

接伴使低低感谢了一声,只是声音内外都透着心虚。

“熙宗皇帝尚在时,尚父与熙宗皇帝都念着两家百年盟好,故而对边境上的龃龉,以大智慧加以化解,这才保住澶渊之盟。对于熙宗皇帝的顾全大局,鄙国上至尚父,下至百姓都是感念甚深。也对熙宗皇帝的驾崩,感到惋惜不已。”

接伴使的脸色由红转白,又由红转青,他想当面反驳,但辽国使者几番挑衅,必有所图。若是他们有心破坏盟约,自己要是将话说死了,那就是犯了朝廷的大忌。心中纵是憋着一口气,也只能隐忍下来。

接着石炭场为由头聊了几句,耶律迪对黑乎乎的煤山没有再多的关注,视线一扫而过。之后的一座座粮囤,更是惹人瞩目。高耸的围墙,已经有了沿途县城城墙的水平。

这其实就是开封城外,一座坚固的据点,若有外敌入侵到开封城边,这里随时可以接纳各地的勤王军。

不过围墙里面的东西更让耶律迪关注,从正门口望进去,里面的一座座粮囤,看起来都是塞得满满的。

“丰年吗?”耶律迪低声道。

今年南京道上也是丰年,西京那边也是丰年。但粮秣堆满了仓屯的情况可没有那么多。

这是因为大辽朝廷以捺钵巡狩东南西北,同时也是就食四方。各地征收上来的税赋,只需要放在沿途的城市周围,供大军食用,用不着汇聚于京城。只有南京道上的钱粮,才需要运去北面作为补充。

像南朝这样,天下财赋聚集一城的情况,根本不可能在大辽出现。却也没有必要在京城周围留下那么多粮仓。

