\section{第23章 铁骑连声压金鼓(七)}

【工作关系,天天晚归,一更已经是竭尽全力了,睡眠时间少得太多也吃不消,还请各位能谅解。不过这种情况只是偶尔,大概明天就能好转。】

天上一轮黯淡的太阳还未有落山的迹象,但持续了数日的城池攻防战,始终未有停歇的厮杀声,到了现在,到了此时,终于从城下转移到了城头上。

伴随着从城外的一面白色大纛下传来的苍凉悲怆的悠长号角,数以千计吐蕃战士如同一群群蚂蚁,举着架架长梯,疯狂的冲向了城墙。

城墙上的西军将士,目瞪口呆的看着吐蕃人完全有别于之前多次进攻的疯狂。只有一丈高的墙体,仅仅是一条最原始、最简陋,甚至没有多少使用价值的防线。以蕃人的手艺都算得上是粗制滥造的长梯,只要设法送到城墙下竖起、架上,便是一条最简便易行的上城通道。

吐蕃人在号角声的催促下,凭借着上百条长梯,在短短半个时辰之内,就已经三次冲上了城头。没有壕沟,没有马面,没有羊马墙,星罗结城的城墙墙体在宋人看来,就是一个城防工程上的最典型的反面教材。

前日,苗授和王舜臣就是明欺着这座城池城防的脆弱单薄,轻而易举便攻入了城中,歼灭了星罗结部的主力。无论是苗授还是王舜臣,在这过程中,都没有少嘲笑过星罗结部的筑城水准。而如今,换作了王舜臣来镇守这座城池,原本城防上让他谑笑不已的许多缺点,现在却成了他现在的致命伤。

如果有壕沟阻隔,贼军根本冲不到城下,如果有向外凸起于城墙墙体的马面,就可以从左右交叉射击攻到城下的敌军。如果有羊马墙,便是有了上下两重立体防线,蕃贼根本上不了城头。可现在,无论守御在星罗结城中的西军将士,拼命射出了到底多少箭,都无法阻止吐蕃战士们的冲锋。

在禹臧花麻的亲自押阵下,吐蕃人的这一次进攻,就如同冲破堤坝的洪流汹涌而来,而城头上射下去的长箭,仅仅是绝望下投入洪水中柴草,非但不能堵上缺口,反而是浪费宝贵的资材。

城墙防线的脆弱,守城物资的匮乏,使得城头上缺乏任何一种行之有效的反制手段。城中的士兵不得不与与攻上城头的吐蕃人,展开了面对面的厮杀。

一名西军战士大喝着挺枪直刺,一声闷响之后,枪尖没入了心口,搠死了正要冲上城头的蕃人。但下一刻,刀光自下飞起,一招便斩断了尚未来得及收回的长枪。西军战士连忙后退,随即翻上了城墙的蕃人却蹂身而上,长刀挥舞,顿时划断了颈项。可紧接着,还没来得及炫耀一番、寻找下一个对手的蕃人,便被一支呼啸而来的铁简,轻易的抽碎了脑壳。

这样的场面,每一分每一秒都在城头上出现。一刀一剑的搏杀,是血淋淋的生命交换。吐蕃人在禹臧花麻的催逼下,拼了命往城墙上冲,而守城的汉人这一边何尝不是为了自己的生命,而在拼死抵抗。

城墙之上,红色的将旗仍在猎猎飞扬。大旗之下,王舜臣深深吸了口气。吸气声绵长不绝,如巨鲸吸水一般,把九月山中的凉意随着空气一起压进了着了火一般的肺中。

因常年使用而被磨得发亮的黑色牛角扳指,牢牢卡着长箭,稳定的搭在了弓弦上。紧握弓臂的左手向前推开,右手同时向后扯动弓弦,上百斤的力道灌注于弓身,一张三尺长弓张开如满月。

吐气开身,右手松开弓弦,嗡嗡的一声弦响,长箭闪电般的飞了出去。弓弦仍在剧烈的振颤,一声变调的惨叫,就从数丈外破空响起。

一名高达六尺近半的吐蕃战士,本来正挥舞着一柄如轮巨斧,独立对抗着五名守军。过人的武艺和超乎想象的神力,不但让他在对战中丝毫不落下风,甚至还能狂吼着箭步冲前,将一名闪避不及的对手劈头砍成两截。但在一道流光闪来之后,这名持斧高手便捂着右眼栽倒在地上。他一阵阵的抽搐着,白色的箭翎在指缝中颤动,露在外面的一尺箭杆证明了王舜臣射出的长箭,有三分之一以上透过眼窝,扎进了他的头颅中。

一箭射翻了一名应当有着豪勇之名的吐蕃战士,王舜臣面无得色。他连自满的时间都没有,也无暇去确认战果。吃力的喘了口气,右手从腰间一抹,又是一只长箭跳出腰间箭囊,出现在他的掌心中。用着右手大拇指上的牛角扳指扣箭搭弓,他视线移转,又瞄准上下一个目标。

在外人看来,已经是快得惊人的射击速度,却让王舜臣狠狠的吐了口吐沫。原本一呼一吸之间,就能射出三四箭的急速,现在已经降到了一半都不到。

王舜臣从左手持弓换到右手持弓,又从右手持弓换回左手持弓。两只手来回张弓,把他左右驰射的惊人箭术表演得淋漓尽致,但他就算这么做,也来不及回复双手手臂中逐渐消耗掉的力量。曾经急如一曲《破阵子》的铮铮弦声,如今已经变成了《八声甘州》,眼见着就要往《声声慢》掉下去。

不过王舜臣的神箭依然保持着足够的威慑力。他已经放弃了以普通的吐蕃士兵为目标,而是瞄准了攻上城头的蕃人中最为勇猛的战士,一箭射去,便给他带走一条的性命,就是禹臧花麻也要痛哭流涕。

一声声弦响,换来了一声声惨叫,双臂的酸痛只拖延了王舜臣射击的速度,却并没有影响到箭矢落处的精准。相反地,随着体力的逐渐下降,王舜臣射出的长箭越发的准确起来,每一箭都直奔双眼和喉间而去。如果说王舜臣气力完足的时候,他射出的长箭能把几丈外一只蟑螂钉在地上,那现在,他已经能把苍蝇蚊子送到墙上作壁画。

十几年来千锤百炼的箭术,让王舜臣几乎变成了一桩杀神,西夏人几次冲上城头,都靠着他的一支支如有神助的精准长箭,来力挽狂澜。

‘竟然已经到了主帅都要上阵博命的地步了……’

王舜臣的活跃维持住了战线和士气,但在这同时,也让许多有点军事头脑的官兵,唉叹起眼下形势的不妙。

呜呜的号角声还在鸣响,就在冲上城头的一群蕃人渐渐被逼的难以立足,正要被赶下城去的时候。一根粗大攻城檑木,被几十人合力抬了过来。城头上还在激战之中,无暇去理会他们。而他们到了城门处,便开始用着檑木去敲打着并不结实的大门。

从城门处传来的轰轰响声,让城头上的守军动作为之一滞,给了吐蕃人一丝喘息的机会。王舜臣正打算冲过去把抬着檑木的士兵全都射下来,可就在王舜臣所处的这段城墙处,七八名蕃兵一齐翻上了城头。

这几人的身材体格都远远胜过普通的士兵,身上的装备也不是普通人能所有。皮甲、头盔、钢刀,都是必须有着不低的身份,才能被分配得上。

刚刚站定,一名高大的蕃人便呼喝着当先扑了上来。王舜臣漫不在意,弯弓一射,便是一具尸体仰天躺倒。而他的亲卫们也都纷纷冲上前去,与这几个准备斩将夺旗的蕃人厮杀起来。

前方虽然受到的阻碍,但后续的蕃兵都跟着翻上了城头,转瞬间,在这一段城墙上,暂时形成了敌强我弱的态势。王舜臣站在他的将旗下坚守不退,用尽了最后的气力拉弓射箭。他很明白,一旦他和他的旗帜被逼下城墙,便是兵败如山倒的局面。而只要他的将旗还在城头上飘扬,城中士兵便都有了主心骨,能坚持到最后一刻。

但蕃人这一波冲上城头的攻势,顿时让王舜臣以及他身边的亲卫吃到了压力之苦。很明显,这些蕃人的目标都是以王舜臣和他身后的将旗,城墙上的其他几处防线的攻防战虽然重新激烈起来,但实际上那几处的热闹,都是为了不让王舜臣在短时间内得到支援而展开的。

刀光闪了几闪,刁钻的刀术出奇的犀利,王舜臣的几个亲兵被这些蕃人中的高手硬逼着退到了一边去,将他们要护卫的对象暴露了出来。

直接面对敌人,王舜臣心神丝毫不乱,勉力将几箭射出,又是几人翻倒,都是直冲要害而去。啪的一声响,在最不合适的时候,他再一次拉坏了他的长弓。

‘难道今天真的要归位了?’

将手中断弓砸向敌人的同时,王舜臣的脑中一瞬间闪过了这个念头。为了能够随身挂着箭囊,能顺利的射出更多的长箭,他连个匕首都没有佩戴。手无寸铁,就算以王舜臣对自己武艺的自信,也不能保证他能击败眼前的手持长刀的对手。

党项人这一边是苦心积虑,从族中挑选出来的高手,终于能砍到王舜臣的影子,几名被逼退的亲卫猛挥刀要杀回来。而王舜臣本人则脸色狰狞,正打算冲上前去,用空手夺一个兵器下来。

一声大喝声震城上,一柄手斧紧跟着呼啸着飞来。手斧在空中急速旋转着,化成了一只光轮,擦过王舜臣的身子,噗的一声闷响,深深的扎进了领头蕃人的头盖骨中。

王舜臣还没来得及去感激那名恩人,回过头去,就是听到一声的号角。但这次号角声不再古朴雄壮,而是带着点急促的味道。

‘是退兵号!’

围攻王舜臣的几人犹豫了一下,看着匆匆赶过来的守军,在看看已经弄到了一把长刀的王舜臣,心知自己已经错失了良机,便纷纷转身跳下城去。

‘这是怎么了?’王舜臣糊涂起来,退兵在成功前的一刻,禹臧花麻为何如此大方?

“难道是后路有警?!”王舜臣惊道。

