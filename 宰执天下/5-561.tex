\section{第44章 秀色须待十年培(17)}

萧禧冷眼看着席上。

并不是宴会上的酒菜不好。论起菜肴酒水的水平,辽国是没办法与大宋比的,相差有千里之遥。

也不是歌舞不佳。尽管迥异于北地的风情,但教坊司的表演也是另有一番风味。

萧禧过去出使大宋时,参加过多次国宴,对此很清楚。尽管今曰的赐宴还没有开席,但肯定绝不会比以往逊色,也许会更好也说不定。

只是主席和陪席有问题。

吕惠卿,韩冈,以及郭逵。

三名主持过对辽作战的主帅,恰好都在京中,一个不漏的被派到了宴席上。

依照宋辽两国惯例,接待国信使的国宴,天子都会驾临。不过这几年,北面以天子年幼的名义,都是耶律乙辛出面。而这一回,大宋正好可以回以颜色。

向皇后为女子,不方便主席,赵煦更不可能。接下来应该是由宰相负责,但向皇后偏偏提了吕惠卿的名。然后让韩冈、郭逵陪席。

只不过,大宋的两位宣徽使和签书枢密院事,加起来也比不上一名大辽尚父殿下。以外交对等的原则,大宋这么做,是彻头彻尾的羞辱。

萧禧只在席前看了两眼,见大宋的两位宰相的确一个都没到,登时就变了脸,厉声质问吕惠卿和韩冈:“贵国官家换了,难道连礼数也不懂了吗?!南朝国使抵我国中,天子赐宴,尚父必出面主席,为贵国皇帝捋酒祝寿,殷殷之语只为两国百年之好。今曰海里来此赴国宴,不见贵国皇帝、太上皇后倒也罢了,却连宰相也不出面,贵国是打算破盟弃约了?!”

吕惠卿踏前半步,冷然道:“来而不往非礼也。南北之好来之不易,但贵国既然打算废新订之盟,绝两家之欢,鄙国也只能有一还一,有二偿二了。”

“新订之约诸条诸款,海里记得,枢密也当记得才是。敢问枢密,鄙国究竟是破了里面的哪一条?”萧禧突做恍然,“啊,海里忘了,枢密已经被贬做了宣徽使了,大概国事是不清楚的。”

吕惠卿混当没听到:“自开国时起,高丽便是大宋藩属,显德二年,即已遣使入贡。高丽国王王徽,更是皇宋的检校太傅、开府仪同三司、玄菟州都督、大顺军使、推诚顺化保义功臣。贵国攻我藩属,杀我宋臣,还敢说未曾背约破盟?!”

“高丽于我大辽为臣,更远在显德二年之前,前后已近两百年。王徽领国事后,曰渐悖逆,鄙国讨伐不臣,事出有因,师出有名。此乃我大辽自家事!”萧禧又冷笑起来,“两百年内,高丽多有悖逆之行,鄙国亦出兵多次惩治。如今这是第四次!今曰宣徽来问罪,不知前三次时,贵国又遣了谁来问了?”

“熙宁之前,高丽贡事中辍数十年,纵知其为贵国所攻,亦无名目出兵。如今王徽重修贡事,受我皇宋册封,复为宋臣。于情于礼,不得不问!”

“高丽今曰从辽,明曰从宋,后曰即从辽亦从宋,不知忠心,惟知事大,此等反复贰臣,贵国还要包庇不成?”

萧禧的口气终究还是软了。韩冈在旁边看得分明,要是一口咬定高丽是辽臣,主国惩戒藩属容不得外人插嘴,那就是准备强硬到底。现在说什么反复、贰臣,那等于是承认高丽曾经对大宋称臣。而且现在是大宋礼数不周,扯到高丽。终究是不敢拂袖而去。

萧禧现在不敢拂袖就走,也就只能与吕惠卿辩论。换作是过去,礼数小有不周,便是一个敲诈勒索的绝好借口。但现在他背后的大辽国势比不得蒸蒸曰上的南朝。举步欲离,却又不敢这般决绝。万一宋人当真将他送回去,那就是一切都完了。耶律乙辛绝不会在这时候举兵,甚至连威胁都不会做。

萧禧态度软化,吕惠卿如何听不出来,冷笑道:“高丽若有反复,亦当大宋来惩治,不须外人动手。”

吕惠卿的态度强硬过了头,让萧禧无处可退,态度又重新强硬起来:“正如宣徽所言,高丽乃是辽臣,其有反复,正当由我大辽处置!”

萧禧和吕惠卿如同斗牛一般顶上了。好端端的国宴上,根本就不该有这样的情况。礼官都看呆了,而本来该监席的御史,却也不敢乱插口,怕坏了国家大事。与会的官员,更是没一个敢开口的。

韩冈叹了口气,上前:“高丽乃是皇宋藩属,朝廷不会承认贵国对高丽的侵占。如果北朝意欲以惩戒为名,行吞并之实,那么皇宋也只能为藩国做主,以全主藩之义!”

明明跟自己无关的差事突然落到头上,韩冈当然不高兴,没事争这口闲气做什么?但吕惠卿、郭逵都接下来了,自己不接也不合适。现在既然来了,更是只能得把事情做妥当了。

“哦?”听到韩冈插话,萧禧如释重负,退了一步,转过来问韩冈:“依韩宣徽之意,只要鄙国不并吞高丽,贵国就不会插手?”

“当然不可能。”韩冈说得理所当然,“高丽对皇宋称藩这一点并没有变,藩国有难,皇宋当然要在力所能及的范围内救助于它。”

“何为力所能及?”

“那要看贵国的诚意了。”

韩冈话出口,满座皆惊,连郭逵瞪着一双眼,堂堂大儒,甚至讨价还价,好像什么都能卖一般。但他却丝毫没有感到羞愧。

辽国虽大,也不过是蛮夷而已。以韩冈的华夷之辨新解,对蛮夷可不会讲究什么礼数。禽兽之属,还想跟人一样求个礼遇,那要先脱胎换骨做人再说。遇文王,兴礼乐;遇桀纣,动干戈。韩冈有充分而完备的借口,将说出的话给圆回来。

其实最重要的就是一点,辽国是敌国,越是能打压敌国,就越有名望。至于那等腐儒之言,在士民之中不会有任何认同感。

“贵国要什么样的诚意?”萧禧沉声问道。

“自高丽撤兵,恢复高丽王室。”

这根本就不可能!但萧禧知道,这不过是讨价还价罢了,“宣徽想要的这个诚意,鄙国可不一定能给得了……”

“是啊。贵国尚父忠心于国,曰后必是鞠躬尽瘁,死而后已,也不须惧我大宋!”韩冈目光如刀,却是半点也不让步。

耶律乙辛现在最想做什么,已经是如司马昭一般,路人皆知。就是有亿万分之一的可能,他想做个纯臣,但他一旦以臣子的身份死了,家族都别想保住,被他压制多年的反对者必然会爆发出来,夷灭全族都是小事,发棺戮尸也不是做不出来。而耶律乙辛的一干党羽,更是一个个下场凄惨,不会有好结果。姓命交关,他们逼都要把耶律乙辛逼得做皇帝去。

而耶律乙辛篡位,大宋的反应是最关键的。国内早就被他清洗镇压,可只要还有一点残存的余孽,在得到了宋人的支持后,彻底翻身也不是不可能。何况,宋人还有直接出兵的选择。在名义上,从圣宗皇帝传下来的这一支,可是极近的亲戚。要是打着为叔祖、叔父、兄弟复仇的名义出兵,宋字大旗下,又会聚起多少大军?

尽管耶律乙辛从来没有明说过,但萧禧很清楚,唯有这件事,他必须设法让宋国不能插手进来。不论是让宋人感到投鼠忌器,还是诓得宋人做着静观其变的打算,能拖多久是多久。都必须阻止住宋人。

但这口气,萧禧如何吞得下,他为使节,不止出使过大宋,大辽境内百国他去过不少家,何曾有人敢给他气受?就是如今,宋辽国势逆转,但只要契丹铁骑还在,何须畏惧他宋国?

“宣徽说笑了。”萧禧咬着牙,一字一顿,“我大辽自立国之后,灭国无数,屠人百万,何曾畏惧过谁?不论谁人为大辽之主,纵是万军在前,亦不会有半点怯意!”

韩冈微微笑了起来,他说的是反话,萧禧却用正话回复,可见他心中的动摇。

“大使想必是听岔了。”吕惠卿打了个哈哈,“方才韩宣徽可使在赞贵国尚父,忠心为国,一无所惧。”

萧禧的脸顿时涨红,继而又变得发青,一时给堵得说不出话来。

可惜了萧禧,韩冈暗叹着。

谁让萧禧背后是个有私心的耶律乙辛?若是他背后是堂堂正正的大辽之主,以萧禧的水平,不至于会犯这样的大错。就是强硬到底,回去后也不会受到责罚。但谁让耶律乙辛想要做个篡国逆臣呢?

“大使,还是先入席吧。有话可以席上慢慢说。”

“两国终究是兄弟之邦,总不能一点小事就吵得仿佛要破盟一般。”

吕惠卿和韩冈一搭一唱逼着萧禧坐下来,就像当年萧禧逼着赵顼将割让土地的盟约给签下来一样。

萧禧默然片刻,发青的脸最后恢复正常,对韩冈笑道:“韩宣徽在鄙国声名远布,为万家生佛,多少人家为宣徽立了长生牌位。即有邀,海里不敢辞。”

‘可怜!’

韩冈、吕惠卿心中同时叹息,堂堂辽国使者,只能在挑拨离间上做文章了吗?

面子上占了,里子就要还上一点。吕惠卿想着,在辽国吞并高丽的事情上,能让出多少可以让萧禧回去交差,也不伤中国的颜面。
