\section{第三章 收兵止戈留余恨(下)}

【第一更,求红票,收藏。】

进退不得。

这四个字,指的是跟随向宝自秦州出征的一众人等的现状……自然也包括韩冈。

进兵当然不可能,王韶都把事办完了,去古渭连残羹剩饭都没得吃。但退兵也不可能,没人下这个命令,谁也不敢自作主张。向宝早前给军中定的口令并不是‘鸡肋’,也不好就此提前打理行装。

就因为没了主心骨,如今永宁寨人心浮动——这件事韩冈则是要除外,他倒是乐得自在一点——全军上下,都在等着向宝能说句准话。

但向宝始终保持沉默,仿佛一场中风,让他的思考能力都随风而去。而他的现状已经在当天就急报秦州,但至少还要等到五天后,才能收到李师中的回复。

其实这两天,向宝的情况已经逐渐稳定下来,手脚都能轻轻的动弹,也不会再对韩冈喊打喊杀。但死仇是肯定结下了,尽管这笔帐主要会算到王韶的头上,可韩冈就在眼前,向宝的带着杀气的一对眼睛,总是盯着韩冈在转。

韩冈现在没事都少去见向宝,若是真的避免不了,都会选择人多的时候,身后也会跟着两人。以便向宝一时怒起,有人能拦着。不过当韩冈说过可以让中风患者从新站起来,虽然向宝本人对此坚决不信,但他的幕僚们都相信了。

除了每天都至少要拜见一次主帅,韩冈的剩余时间则是做自己的本职工作。永宁寨中本就有些生病受伤的士兵——这也是任何一座城寨都难以避免的情况——韩冈便趁机把疗养院的牌子在永宁寨中竖了起来。带着朱中为首的一队护工,还有伤病员的亲友家属,打理起秦凤路的第二座疗养院,就跟他当初在甘谷城时做的一样。

另外韩冈也遣人去了古渭寨报信。向宝被气得中风,整个秦州政局都要改变。而且王韶是当事人,他的立场十分微妙,必须要通知他及早做出准备。

所以两天后,王厚带着赵隆匆匆从古渭寨赶来,就不是那么令人惊讶。

一见王厚,韩冈便上前拱手道喜,“恭喜王机宜,恭喜处道兄。”

“恭喜家严,还是当面说得好。恭喜愚兄,愚兄可没什么好喜的。”王厚经历过一次大战,精气神都不同了,说话、性格都在渐渐改变。

“机宜一战得胜,再不会有人说,机宜在秦州是劳而无功,虚耗人力了。”

“这一战可没那么简单。”王厚摇摇头,似是感慨万千,“还是偷袭,又是两倍于贼的兵力,但一战下来,各部死伤都不少。木征支援托硕部的就只有四百兵,但全是精锐。董裕带着他们一个反击,差点就给他翻盘。”

论起兵事,韩冈的经验便显得不够用了。他疑惑的问着:“木征真的有那么强?才四百兵……竟然差点就让托硕部翻盘?”

王厚摇头,“即便在河州,像董裕带来的这四百人肯定也只是为数极少的精锐。如这四百有兵有甲且经过训练的精兵,木征最多也就两千上下,但已经足以让他在河湟雄踞一方了。”

“兵贵精不贵多,木征看来也是颇有见识……没能见识一下木征家的精锐,还真是一件遗憾的事。”

王厚笑道:“等过几个月,就会见到腻烦的地步。”

韩冈点点头:“机宜以蕃部破蕃部。平戎策上的团聚众羌这一条,已经初见成效。只要圣聪未被蒙蔽,机宜于渭源建城,于青渭屯田市易,都是指日可待。”

“团聚众羌主要还是刘昌祚的功劳。”王厚没有贪天功为己功的意思,而且韩冈又是自己人,在他面前也没必要自我吹嘘,“玉昆你早前说得没错,刘昌祚这段日子给向宝欺负狠了,心中怨意确是极深。他虽然不敢调动麾下兵马,但七支蕃部中,有四支是他叫来的。没有刘昌祚的助力,今次说不定要惨败。”

“刘昌祚论能力,在秦凤军中少有人能与之匹敌。但他偏生官运甚差,总是被上官压制。今次终于给他把握到机会了,他怎么可能放过?”

王厚点着头:“昨天刘昌祚听说向宝中风昏倒,当面虽然没话,但他回去后据说可是笑了许久。”

“刘昌祚被向宝压在头上,向宝坏了事,他不笑才有鬼。”韩冈不奇怪刘昌祚的恣意无忌,任谁被顶头上司坏了晋升的机会,都会如刘昌祚这般恨人恨到骨头里。他很理解刘昌祚的想法,像刘昌祚这样的组织中坚,如果被上司压着不给他的做事,哪个会甘心,换作是自己,早就刀枪一起上了。

“刘昌祚听着向宝中风之事后幸灾乐祸,但他的心中还是有些生疑。”王厚问着韩冈,“玉昆,向宝中风一事可是确实?”

韩冈清楚这句话才是王厚今天最为关心的一桩事,“小弟亲眼看到的,这一点向宝做不得假,而且他也没必要作假,装中风对他有没有好处。”

“竟然是真的。”王厚又是在感慨着:“家严听说了此事之后,就说一句物伤其类,兔死狐悲。家严的本意也不是想看到向宝最后成了这般模样。”

“向宝心怀不广,所以气急之下得了风疾。此非机宜之过,当不须耿耿于怀。”

“只是向宝是带御器械,虽然仅仅个虚名,但他在天子驾前也的确做过一阵,混个脸熟。如今的天子性格宽厚,现在听到向宝中风不起,天子那边多半少不了会有些芥蒂。”

王厚虽然没提他的父亲,但这段话只会出自王韶之口。王韶见过天子,那是在他的《平戎策》得到赵顼认同后,被越次招入宫中。那只是两年前的事,这么段的时间,赵顼的性格不可能发生翻天覆地的变化,这决定了王韶的评价不会有什么错误。

“难道天子会看不到机宜收复蕃部的功劳?”韩冈对赵顼没什么了解,但一个感情用事的天子,对臣子来说,可不是什么好事。

“这件事家严也说不准。不过愚兄想来,王相公应该能帮上一手。”王厚为之分析着,韩冈见他侃侃而谈的样子,不知从王韶那边听到了多少,“只是秦州军中,家严的名声可就不是王相公能照顾得了的了。”

说起来,王韶在秦凤军中的名声可能因为这次他横插一杠,再度滑落下去,毕竟是带着蕃人抢功劳,没有几个士兵会喜欢这样的官儿。而且向宝的失败虽然的确可笑,如果仅仅是吐血的话,他就是个丑角,但向宝现在中风昏倒,却能引来不少同情。

“不过士卒军汉们的想法并不重要,重要的还是朝堂。”韩冈如此说道。他在秦凤军中也有些名声,但这并不影响李师中和向宝跟他过不去。现在军汉们因为功劳被抢了,所以敌视王韶,但等到王韶领着他们挣了功劳,分发一些赏赐,他们的看法便会颠倒过来。

“玉昆说得是,他们的想法的确无甚关碍。”王厚点着头,“向宝既然卧床不起,这两天等秦州的消息送到,就肯定要退兵了。玉昆,要不要顺便到古渭去。你的疗养院就开在甘谷城和永宁寨中,其他寨堡可是会有怨言的。”

“人才难得,小弟一切亲历亲为,所以做得慢了。不过小弟这边,有个叫朱中做得不错,古渭寨的疗养院可以由他先把架子搭起来。”

韩冈写出《伤病管理暂行条例》,为军医之事定下规矩后,一切就可以照着规条来就行了,并不需要他本人事必亲躬。“小弟现在还得随着向宝回秦州缴令,都得等过上一阵子,再去古渭不迟。”

“说实话,家严虽然与向宝几乎势不两立,但毕竟离得极远。玉昆你天天在向宝面前晃来晃去,也不怕他心情不好?”

“向宝不是傻子,他现在也不疯。他还想着恢复发病前的健康。不可能得罪一个对医术有所了解,传说中是药王孙思邈私淑弟子的人物,”

王厚惊道:“难道玉昆你知道中风该怎么治?”

韩冈摇摇头,笑道:“我对此也不甚了了,但向宝以为小弟知道。”

王厚注意到韩冈用了‘不甚了了’这个词。韩冈说话一向谨慎,很少说谎,而且遣词用句都是依着标准而来。他既然用上了不甚了了这四个字,那他对中风还有有点了解,所以能让向宝误会。

“既然玉昆你早有准备,愚兄就能安心了。”

三天后,就是韩冈预计的时间,秦凤经略司的公文追到了永宁寨中,在命令中,李师中下令向宝带出来的队伍,及早回返秦州。一场剿灭蕃部的大战就这么虎头蛇尾的落下帷幕,只有王韶和刘昌祚两人得意,而其他参与进来的官员,多是偷鸡不成蚀把米,灰溜溜的回去了。

韩冈自在的骑马走在队伍中,本是跟随他的朱中等人已经与王厚一起去了古渭寨,而本来带在身边的药材等物资,也托王厚转交给了王韶。

就在秦州城中,向宝的几个亲族这时聚在一处,向着李师中哭诉:“王韶鼠辈,妒贤嫉能,窃据高位。今次向钤辖受其所欺,以至于遭受卒中之厄,还望李经略为钤辖主持一个公道啊。”

李师中点着头,心中却是在想,王韶的运气未免也太好了一点,竟然能把好端端的一个人气成了中风,这下向宝空出动的位置,不出意料,张守约肯定将会顶替上。如此一来,支持王韶的军方将领,便已经是钤辖一级了。

真的是运气!李师中这般想着。

