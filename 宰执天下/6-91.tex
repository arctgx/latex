\section{第十章 千秋邈矣变新腔(13)}

直到走到崇政殿前,韩绛还在考虑韩冈他所创设的新体裁。

不得不承认,换了一个名目之后,再回想起那几道题,给人的感觉就不一样。

没有了那种对不上的错位感,虽不是策问,却的确能有评判考生才学、眼界和应对的能力。

但这所谓申论,古之所无。只能说是有近似的文章。

既然是论,肯定是以议论和评价为主。在史论中,《六国论》《过秦论》这样就历史变迁进行论述的文章,也有就历史上某一件具体的事件,或某一个具体的历史人物,进行议论的文章。

在史论之外,也有针对经义中的某一条某一句的议论,还有对于现实事件的评论——许多奏章就是如此,尤其是御史们所上的弹章。

让申论引申的事件,都是本朝近事,但韩冈在题目中,除了评论之外,还要求考生对具体的事项给出自己处理的意见。

这就近乎是明法科的考题,要让应考之人,对题目中给出的几桩案子,做出判决并写出判词。也类同于释褐后身言书判中的判,要对官衙中的具体事务进行处理。

这样看来,就是论和判两种题材的结合。

换个角度来看,有些朝臣也的确会将奏章写成这个样子。而这么写出来的奏章,也的确比较让人留下深刻的印象。

作为宰相,韩绛一般也比较喜欢看到这样的奏章,论断和处理办法一起都给出来了,看着就爽快。不像有些奏章,总是云山雾罩,要在其中寻找实际内容,不知要费多少心力。

让新科进士们提前做一回朝臣,这也没什么不好。未来的一二十年后,他们少不了会遇到要写奏章进呈政事堂的情况,一篇申论正好可以让他们了解该怎么写奏章更合适。

的确是气学。但不是与新学争夺对经义诠释权的气学,而是以经世济用为目标的气学。

不过话说回来,如果没有策问,仅仅是申论的话,必定会惹起一番争议。但有了策问,再加上申论,非议就会少一点了。

就是那个一百分不好说,同样是新创,不知考官们能不能习惯。

更有一点,这两题,到底该怎么划分?

……………………

韩冈对于申论这个新定名的体裁没有太多的想法。

说句实话,他当年在殿试上所写的文章,已经近乎于申论了。

类似于申论的文章,韩冈每天都能看得见,现在只不过是给个名号罢了。

作为考题的内容,诗赋、经义都不可能评判出考生的治政能力。策问虽能看得出考生的眼界,实际上也是空对空。申论虽也是空对空,好歹还有些实质性的内容。

将自己的打算,透露给韩绛之后,韩冈暂时没有别的想法了。现在拿出来的题目只是作为范例,真正的考题,要到最后才会交出去。

跟随韩绛的脚步,走近崇政殿中,没有等待多久,太后的銮驾也抵达了殿内。

尽管刚刚结束了廷推,但新晋的枢密副使曾孝宽,并不是今日议事的重点。

说起来,今天并没有什么大事需要讨论。

一件是来自于朝中的奏章,请求太后指派宫人为西域将士制作春衣。并希望由此形成定制。冬日制春衣,到秋天时,再让宫人为西域将士制作冬衣。

在奏章中,说明了在西域,一年一件春装、一件冬装,完全不够使用,由宫女为他们制作第二套军服,自然能大获军心。同时奏章里面还引用了好几条汉唐旧例,为自己的提议做注脚。大概是想引起太后的兴趣,奏章里面还附上了有关唐明皇的那条轶闻。

由宫女为前线的将士制作征袍的先例不少,不用追溯汉唐,本朝其实就有。这件事也并不大,之所以要上书太后,只不过是因为宫内是太后自己掌管,政事堂不能插手,否则直接就在韩绛、张璪、韩冈这边给批下去了。

之所以需要讨论,是因为韩冈对此表示反对。

向太后对此大惑不解,若是别人反对倒也罢了,在西域征战的可是西军,领军的又是与他关系亲密的王舜臣,怎么也不该是他出面反对。

“参政,马、步禁军虽本有春装、冬装,但都只有一套。西域苦寒,又是征战不断,一套肯定不够用。”

说起军服,殿上没有人比措办过多年军需的韩冈更加熟悉,他当然知道西域那边对各种军需物资的渴求,其中绝不会缺少军服。

并不是因为被克扣——因为官吏与喝兵血的问题,很多部队的军服时常被克扣,但因为当时西北战事不断,要是冬衣下不及时,很容易就闹出兵变,不知要拿去多少人头来抵账,没几个蠢货会将手往这里面伸。

而是因为征战在外,衣袍的磨损会大大增加,一件能穿半年的好衣服,在外出后,能维持一个月就不错了。

但韩冈还是反对让宫人制作军服。

“臣不是反对加赐征袍,而是想知道,征袍由宫人制作,会不会给予报酬?”

“报酬?”向太后稍稍愣了一下,“……会有些加赐,平常都是有月例的,宫人闲下来也都会做女红,只是让她们改去做征袍。”

换做别人,多半会认为韩冈担心太后会趁机多给宫人酬劳——随时提醒宫中俭省,也是臣子的本分——但太后这一回倒是理解对了,韩冈不是担心给钱,而是不给钱。

“陛下明鉴,虽说有加赐,但应该比不了做女红的收入。要是一次两次,尚无大碍。但常年累月,宫人岂不怨?人若有怨,又岂能长久?太后心念西域将士,加赐衣袍,诚为美事。若因此而使宫人生怨心,又为不美。”

宫女不是闲着没事就勾心斗角,绝大多数的宫人没有太多的闲暇,各有各的事情要忙,闲下来也会做女红。

她们手上的一些零用钱,光靠俸禄不够,赏赐也不会多——就是后妃都是按月拿钱,偶尔收一些来自宫外的礼物,还要提心吊胆被人打小报告,哪有可能一赏几十贯、上百贯?——都是要靠人将自己的作品带出宫去卖,然后再将钱带回来,或是直接托人换成胭脂水粉、绸缎饰品之类。

突然间要她们做白工,最多给点象征性的好处,损失了大量潜在的收入不说,还要一年两次,年年如此,哪个不会抱怨?

西域的大军若是生怨,对处在深宫中的太后并无影响,可若是身边的宫女们生怨,威胁性就大了。

“吾明白了,参政顾虑的是。但西域的军袍怎么办?”

“一方面可以让宫人制作,只要给予与市价相等的酬劳。另一方面,出征西域的大军,下给士卒们的羊裘,可以不用收回。”

“什么羊裘?”

“禀陛下,朝廷因塞下苦寒,至冬日,便会在军中赐下羊裘,人各一领。至春暖便拘收,修补后收入库中,以待冬日。”

西北军中,朝廷下的羊皮袄,并不是就属于此人,而是跟甲胄、兵器一样,都是只有使用权,没有所有权。韩冈的意思,可以将所有权也给西域军中。

“这些不够吧。”

一件羊皮袄,向太后不觉得能顶多少年。

“的确不够。但先还是要先保证一春一冬两套军袍的质量。朝廷下的衣袍,往往不中格,或轻薄短小,或容易朽烂,总之不尽人意。”

“此事当严查!”向太后断然说道。

“陛下圣明。”韩冈行了一礼,然后又道,“此外还有一事。”

“参政请说。”

“皇宋广有万邦,南北、东西皆有万里之遥,各地气候不一,方今京中花开正艳。西域道上,积雪则尚未消融。而在交州,却是四时如夏。”韩冈停了一下,“《礼记》中的《月令》这一篇,说的是四时,却也只在黄河南北一带合乎时节。譬如邕州,四时如春,冬季亦无深寒。交州则是四时如夏。自陇右向南,在高原之上,吐蕃人世居之地,又多有积雪终年不化的峰峦。四方月令,多迥异于中原,却皆是‘莫非王土’。”

“参政是想说各地气候不一,所以下的装束也得视地域有所变化?”

大概是与衣服有关,向太后显得十分敏锐。

“正是。”韩冈点头,“至于究竟如何依照地域变化,臣请陛下交由有司负责,再交由专人剪裁。”

“就依参政。”

韩冈口中称颂,又行了一礼。

韩冈比较希望能在京城中就设立成衣舍,专门裁剪成衣。如今市井中也不是没有卖成衣的店,不仅有,而且还不少,不过基本上都是旧衣,当铺就是其中最大的来源。而新衣的成衣舍,实在很难生存。不过换成是为军中制衣,那就是另外一回事了。

“至于布料,衣袍还是以棉布最佳,丝绸往往不堪用。”

韩冈话音未落,向太后立刻说道,“参政应该知道的,棉布很贵。”

“普通的棉布如今已与丝绸等价了,而丝绸做不到棉布的结实耐用和保暖。”

如今的江南,也开始种植棉花,衣被天下的松江,即将出现在大宋,这就是韩冈和他手中的雍秦商会,必须要面临的问题。
