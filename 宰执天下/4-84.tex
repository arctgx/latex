\section{第15章 焰上云霄思逐寇(15)}

黄金满和苏子元领军北上,让关城中掀起一阵骚动。士兵们没有人敢于公开询问,但私下里为此交流的眼神中,都带着一丝惊惶。

韩冈暗叹一声,他说交趾军突袭宾州不会影响军心,可实际上他的麾下士卒看到黄金满领军向北,还是免不了会动摇。

“李信、黄全。”韩冈点了主将的名,“你们将交趾军偷袭宾州的事传下去,让将士们不必惊慌。”无法隐瞒的事就必须公布出来,只有光明正大,才能让谣言没有滋生的场所。

本来宾州城中有实兵六百,再加上黄全的一千人,就是一千六。除此之外,还有刚刚征发起来的保甲,不过他们还没有武器,只能作为守城时的补充兵源。有这么多人守城,即便突破宾州城防,也不可能立刻将城池给占据。只是既然处在被突袭的情况下,还是要做好最坏的准备。所以韩冈一口气派了两千兵,希望他们能尽快解决自己背后的敌人。

而韩冈现在就等着李常杰攻上来。他手上还有三千兵,兵力与前几天在归仁铺的时候差不多。

可相比起归仁铺简陋的营地,昆仑关要坚固得多。关城所在的位置并不算险要,延伸到两边山头上的关墙也不算高峻,与北方那些个名关相比差得老远。但整座关城也是精心修筑,只凭交趾人的攻城手段要想直接攻打那是不可能的。可李常杰的军粮就算足够,想运上来也难,而且还有士气问题,撑不了多少天了,时间是在他韩冈这一边。

从城楼中走出来。不知何时雨已经小了起来,天上云层看着也薄了许多,不再是沉重的铅灰色,而是发白发亮。云层中裂开了一条缝隙,一线阳光投了下来,照在昆仑关的关城上。还带着水迹的城楼瓦片闪闪发亮,被久违的阳光镀上了一层耀眼的金芒。

“这是天现吉兆!”

一声喊叫在背后响起,韩冈吃惊的回头一看,却见是何缮。

何缮紧跟着韩冈多时,现在终于有了说话的机会,他指着金光灿灿的城楼:“这是天现吉兆啊。日曜城楼,可见我官军有上天庇佑,交趾小贼却被雨淋多日。这一战我官军必胜!”

被他这一嗓子,城中的守军都朝着城楼上望过去,看见一片阴暗的天地,只有城楼顶上映着阳光,闪闪生辉,似乎当真有上天庇佑。对着城下跑马的敌军,也不再放在眼中。

李信对着何缮满意的点点头,他这一嗓子喊的正是时候。回头来,指着城下的骑兵:“交趾的偏师刚刚攻到宾州城,李常杰就开始进攻。隔了几十里山林,他们究竟是怎么联络?”李信将疑问抛向韩冈。

韩冈身子一震,这个问题此前被他忽略过去了,现在想起来的确满是疑问,这究竟是怎么回事?

皱起眉来苦思片刻,他不得不摇起了头,的确想不通,即时联络不可能,要说事前的约定,那就更不可能,谁能保证宾州一定能拿下?前面说李常杰疯了,说他孤注一掷,可没说他蠢。

“且等着宾州那边的消息,当会有个合乎情理的的答案。”

…………………………

几天以来,一直在耳畔持续不断的雨声渐渐的停了。

李常杰已经结束整齐,头盔、甲胄都穿戴到了身上。走出帐外,护卫主帅的两千兵马已经整装待发,正等着他发出前进的号令。而前军后军也都对他的命令等候已久。

李常杰他已经尽了最大的努力,如果不是确认了宋军真正的战斗力,他绝不会将自己逼到不得不决战的危险境地。

由于地理地势的关系,交趾对广南两路的宋军了如指掌,甚至比起东京城中的天子、宰相都要了解。广西宋军一贯拙劣的表现,让他看到了大获全胜的未来。但当李常杰与宋军中真正的精锐交手过后,才发现自己之前的判断完全建立在一个错误的基础上。

这还仅仅是来自荆南的军队,曾经踏平侬智高的北方大军还没有出现。如果他们出现了,不知又会有多么恐怖。

交趾一向看不起广源州,李常杰也看不起侬智高。侬智高的父亲还是死在交趾国中,可侬智高几曾打算过为父报仇?他只敢欺负宋人。尽管此后侬智高被灭与狄青之手,但击败侬智高也算不了什么本事,狄青凭着这件功绩就坐上了枢密使的位置,试问如何能让交趾看得起宋军。

可是李常杰现在知道自己是大错特错了——不,其实在邕州城下就已经知道错了——幸好还有挽回的机会,他的兵力依然雄厚,在他散去了在钦州廉州的所得之后,士气也提振了许多。只要这一次计策能够成功,阻挡在眼前石头一样顽敌一样会如同瓷器碎成千百片。

跨上马,抽住匣中剑,李常杰遥遥向北一指,同时响起的鼓号传达了他的号令:“前进!”

…………………………

到了午后,从后方快马传来的消息,让昆仑关上下都松了一口气。

偷袭宾州的交趾军,没能攻下城中,受到城头上的反击后,就向东绕过宾州,继续往东南去了。现在有黄金满带过去的一队骑兵盯着,这群人数大约在七八百左右的交趾兵,逃不过官军的追踪。

这个消息韩冈立刻让李信和黄全传了下去,不用在面对前方敌军的同时,还要担心后方受到攻击,欢呼声顿时响遍关城。

步出城楼,看着已经逼近到一里地外的交趾骑兵。他们所在的山道还算宽阔——昆仑关入山后的道路的大部分地段,其实都跟山外的官道一般宽度——但几十匹骑手都挤在短短的一段路上,隔着五六十步的距离,与他们对峙的宋军骑兵仅有十几骑而已,但交趾骑兵就是不敢越界一步。

“他们就不怕在烂泥地里摔了马脚?”韩冈对交趾骑兵摇摇头,转身对李信道:“现在终于可以确定了。”

“什么?”李信疑惑的问着,“那些骑兵怎么了?”

“不是骑兵,是突袭宾州的交趾兵的事。李常杰和这一部兵马不可能联络上,也没有打算联络,他们放弃攻打宾州、放弃得实在太轻易了。如果是约好打下宾州,而且李常杰也不至于那么蠢。”只要亲眼看了昆仑关这一片的山林,韩冈完全可以确定,没有后世的信息交流手段,靠着人力来传递消息,不可能将两边的进攻时间掐准,“他们的目的也不是打下宾州城,而仅仅是扰乱昆仑关的后方,让我们必须分兵去围剿。”

“但时间上……”

“能不能攻下宾州城,谁都说不准,可穿过山岭的时间完全可以大致确定。不需要联络,不需要约定,一路避实就虚,只要不给围攻上,只要出现在宾州城外,我们就要派兵去围剿,至少要留下一部分兵力去看守。”

“但这一部交趾兵绕过来,不为攻城,只为骚扰。总不是为了送死吧,还往东南逃去……”李信脑中似有灵光闪过,“难道是准备逃去横州!”

说着转身就往城楼里走,在厅中展开广西一路的地图。

韩冈跟了进来,就见李信指着简略粗陋的如同小儿涂鸦一般的地图,“宾州东南是横州,而横州南面是钦州,西南就是邕州。只要往横州去,他们不论是回邕州,还是干脆去钦州,都很容易。”

“这是硬吃我们兵力不足啊。”韩冈深深一叹,这是没办法的事,以己之长攻敌之短,李常杰行事正合兵法正道:“只要我们分兵,许多事就……就……”他盯着地图,眼神猛然一变,“左江出了邕州后,正好经过横州!”

“乘船回邕州……不对!”李信立刻摇头,惊声道:“是乘船下横州!”

韩冈脸色凝重的点头:“交趾军有船。官军已经退回到昆仑关,刘纪三人倒戈的可能性已经是微乎其微。如果广源蛮军此时已经渡过左江,回师南向,李常杰就不需要再分出大量兵力盯着他们。空闲下来的队伍至少有万人,他们只要在邕州上船,可以乘舟沿左江直下横州。在永定县北上,可以直插宾州东南,抵达昆仑关的背后。左江上的船只并不少,有三四天的时间,足以将五千兵马送到横州。从横州至宾州,一路上没有任何军寨和城池,不会有人防守。”

随着推测与地图印证着说出来,韩冈越发的确定自己的判断。自己此前是被李常杰的给蒙蔽了视野。先是看见他身边带了两万兵,以为他必须要盯住广源蛮军,不可能再有多少兵力可以调用。今日再看到他派了一支偏师翻山越岭攻打宾州,自己和下面的将领们的注意力全都被他这个行动给夺走了。

韩冈很是有些后悔,自己因为南下得太仓促,又是率领着不熟悉的队伍连番作战,并没有推行原本在西军时行之有效的参谋制度。的确是自己太疏忽了。其实如果有心建立,还是有时间建立起一个可以集思广议的参谋体系,尽管免不了粗糙和原始。但许多问题都能够交给下面的军官们去思考和推断,就算他们说得大半都是无稽之谈,可至少能拾遗补缺,给自己一个参考。这远比一人之力要有效得多。

李信不知韩冈正在后悔,“可是他们粮草的问题怎么办?这边还能用杂兵和骡马来运粮,但他们到了横州去,要赶着北上,没有时间去攻城抢粮吧?”

“前几天苏伯绪曾对我说过,广西内陆的军州,除了州城以外,下面的县治很少建有城墙。”韩冈摇头一叹,“永定县的存粮够他们连吃带拿了。”

“时间呢?”李信神色肃重,“他们会什么时候到?!”

“也许就在一两天之后,”韩冈和李信对视了一眼,同时点了点头,“要出战!”

李常杰放了那么多心思在他的前后夹击的计划上,他能不能想到是关城中守军会先打出来姑且不说,他对自己麾下军队没有信心是显而易见的。计划虽然精巧,可一旦看破,那就什么都不是了。

虽然昆仑关这边分了兵,但李常杰那边分兵的情况更严重,而且他还抱着偏师打到昆仑关背后的美梦,此时不打他个猝不及防,又更待何时?!

