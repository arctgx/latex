\section{第29章 坐感岁时歌慷慨(中)}

【见鬼的年终总结。欠下的一更还没还,反倒又欠了一更。脸皮都给自己丢没了,不敢再保证什么,希望元旦时能有空补回来。】

送过了灶神,过年的气氛便浓了起来。

噼啪作响的爆竹,时不时的就会响起个一声两声。王韶几次提笔,都是猝然炸响的声音打断了思绪。摊在面前的稿纸,涂涂改改的只能看到墨团,只应该是短短的一封信,却用了一个多时辰都不见进展。

又是一记爆声响起,多半是石子桥林家卖得特大号的爆竹,却如天上打下来的一声霹雳,震得窗户一阵哗啦啦的响。

王韶抬头怒视着窗外,他家所在的升元坊,多是重臣国戚所居,向来是安静的。一干重臣在大街上鸣锣开道,进了坊中之后,就立刻偃旗息鼓。也就是过年的时候,吵得让人心烦意乱。

低头看看的一团污糟的稿纸,王韶突然间就丢下笔长声一叹。这跟爆竹无关,是他心里面乱。

王韶摇摇头,朝局也乱。

两年前,王安石第一次辞相时,新旧两党对立严重,各拿着一桩案子要将对方给掀下去。闹得朝堂上成了一锅滚开的稀粥,弄到最后,是韩冈和韩绛两撺掇了天子,将王安石召回来了事。

但这一次是不可能了,紧跟着王安石之后,是冯京被弹劾出外,在此之间,天子完全没有挽留冯京的意思,又将吴充调任宰相,吕公著升任枢密,甚至还将郭逵这名武将也调回来做王韶的同事。

从眼下的这几件事上看,天子对王安石离开后的朝堂乱局的处理手段,不再是打算维持朝中的稳定,而是想着重新换上一批新面孔了。

将桌上已经全是墨迹的稿纸团成一团,丢在一边,王韶低头看着干干净净的桌面,他已经很久没有这种无所适从的感觉了。

并不仅仅是因为朝局的混乱,而是现在根本没人知道天子是怎么想的。

吕惠卿、章惇下手对付冯京的时候,恐怕不会想过最后得益的是吴充和吕公著。

没有任何证据,能说明张商英是,他身为御史当然不可能自己去拜见吕惠卿和章惇,但从他的表现来看,肯定是秉持着两人的心意。

他这位御史,所掀起来的波涛,直接搅乱了在失去王安石的镇压之后,本来就已经快要沸腾的朝堂。

吕公著自回到京城之后,天子的用心其实就有了点征兆,可当时又有谁能预料得到天子有意让他接掌西府。

王韶并没有想过这一次朝堂变局上他能坐到枢密使的位置上,尽管他也做了四五年的枢密副使,但资历和声望还是远远不足以担任与政事堂相提并论的西府之长。

王韶很清楚这一点,只是天子在任命时完全没有考虑过他,这还是很让人觉得泄气,可偏偏他就是连不甘心都做不到。只不过眼下天子的几桩任命弄得朝堂上风急浪高,不知又是何意?

郭逵是武将,他时隔多年之后,重又担任同知枢密院事一职,这一桩敇命,被知制诰封驳了两次,是在天子坚持下才通过的。难道宣徽使一职还不能表达天子对郭逵的看重,偏偏还要再让他进出西府一回?

而吕公著更是铁杆的旧党,当年与王安石闹得割席断交的人物。他做了枢密使,最害怕的不是曾经偷了他的奏章草稿泄露给王安石、被他骂为家贼的侄孙吕嘉问,而是吕惠卿和章惇,恐怕连他们也不敢保证,天子是不是有着对他们过河拆桥的打算。

一阵脚步声在外面的廊道上响起,奉旨回京诣阙的次子王厚在外叫门的声音,随即在书房外响起。

“进来。”王韶将毛笔在笔洗涮了一涮,用纸吸干之后,挂到了笔架上。

年头有些久了的书房门吱吱呀呀的响了一声,王厚跨步走了进来。在关西边地任职多年,王厚经过几番风吹雨打,早已成了精悍干练的一方守臣,举手投足都由一股慑人的魄力。

“赶了几千里路,怎么不早点休息?”王韶责怪的说着,王厚是今天午后才进的京城,回府后,问过安,吃过饭,就该去睡觉的。“明天就是五日常朝的日子,你也要上朝的,说不准天子都要赶着召见你……睡得少了,到了殿上小心说胡话!”

王厚淡然一笑:“出外巡边的时候,孩儿可是整宿整宿的睁着眼睛,只是中间与人轮班的睡一两个时辰。”

王韶皱起眉,训斥道:“你这个边臣,没事往外面跑那么勤作甚?想着被党项人埋伏吗?!”

“也要他们敢来啊。”王厚笑容冷冽,“现在不开眼的越来越少,多少部族想投过来。兴庆府那里更是笑话,都死到临头了,还闹着要不要撤帘归政。”

梁氏不肯放弃手上的权力,但秉常也到了亲政的年纪,就算外敌已经逼到了横山,可兴庆府中还是在争权夺利。这消息自是瞒不过横山内外诸多宋人的耳目,一早就传到了东京城中。身为枢密副使,王韶当然不会不知。

王韶抬头看着几个儿子中最为出色的一个,轻声一叹,指了指对面,“坐下来说。这时候过来,是有什么事吧?”

“也只是想找爹爹聊一聊。”王厚扯过来一张方凳,在王韶面前坐下,看看干干净净的一张桌子,转过来问着王韶:“听说这些天,朝堂上乱得很?”

“你问这么多作甚?”王韶听着脸色就冷了下来,“管好你手边的一摊事就好了!”

王厚不以为意,他知道父亲这是怕他万一在天子面前说漏了嘴,就是回到了关西后,朝堂上的事也不是他一个武将能说的,还有走马承受给天子做耳目呢。

“只是见爹爹吃饭的时候有些郁色,”王厚顿了一下,“所以有点担心。”

“朝堂上的事,你不该问……”王韶依然是板着脸,“为父也只能在旁边看着,你这个武臣就该有多远躲多远,谁来问你都该说不知道。”

王厚看见老子脸色沉郁,心中有了几分了然,遂转过话题:“方才听大哥说玉昆这一次终于也被召回京城了?”

“嗯。”王韶点了点头,脸色也缓和了些,“这两天就该到了。”

“这多久不见了……”王厚脸色多了分喜色:“上一次通书信,已经将孩儿家的五哥儿与他家的大姐将亲事说定了,这一次撞上了,正好可以把换名纳聘的事一次都做完了。”

自家的孙子能与韩冈结亲,王韶当然乐见其成。他只恨自己的内侄女没福气,要不然也不会给王安石捡了便宜去,不过现在孙子能娶韩家的女儿,也算还了愿。

“对了!”王韶神色严肃的吩咐着,“好生的教五哥儿读书,韩玉昆日后都要往两府中走的。你要是不能还他一个进士女婿,看看日后还有什么脸面去见他。”

“儿子也不是进士啊,要没脸早没脸了。就是玉昆他自己,也是靠了时运,换个时候连贡生都难做。”王厚笑得不以为意,能不能中进士那还真是得看运气了,“等五哥儿再大一点,就让他拜在玉昆的门下,若是日后还不了孩儿一个进士儿子,那就是他没脸面见我了。”

当着自己的看玩笑,王韶瞧见王厚咧嘴笑着,心道他这个二儿子当真是成人了,不像旧时,与自己说话时都带着一份胆战心惊。

心中暗叹一声,王韶开口说道:“当年王介甫辞相,朝堂上也是闹得不可开交,最后是考了韩玉昆出手,加上韩绛,打动天子将王介甫从江宁召了回来。”

王厚稍稍吃了一惊,他的父亲怎么又突然说起了方才严令自己不得询问和打听的消息,不过这也正合他的心意,“那这一次玉昆入京,能否挽回现在的朝局。”

“难。”王韶给出了一个极简洁的回答,“时势更易,已经不是两年前了。天子对新法的心思说不准。”

尽管从眼下国家财政的情况上看,这个时候天子不可能抛弃新法,熙宁六年以后,就没有遇上一年没有灾情——若不是有青苗、免役诸法,国库早就完蛋了——月初天子才下诏明年改元元丰,求个风调雨顺,但谁也不敢打包票,也不看看东西二府的都是由谁来主掌?

王厚点点头,表示他对王韶的话能够理解,想了想却又问道,“那韩玉昆会不会坚持帮吕惠卿?他跟章惇据说是在广西配合得极好,而且他与章父有救命之恩。不会看着不理吧?还有王相公的脸面在。”说着就有些发愁了。

“说不准。”王韶摇了摇头,“韩玉昆是个油盐不进的性子,当初王介甫在的时候,几次三番都没能压得下他举荐张载。如果他不想帮,可不会顾忌半点王介甫的面子。”

“况且吕惠卿和章惇也不一定需要人帮。他们唆使张商英弹劾冯京的时候,天子并没有坚持要留下冯当世,否则就该是张商英回去监酒税了。”王韶冷笑了一下,“恐怕冯京自己都没想到,吕惠卿下手会这么快。”

