\section{第六章 征近伐远方寸间(上)}

【第二更,求红票,收藏】

王舜臣自延安回来了。前些日子,他跟着王韶将托硕部一顿好打。打完后就请了假,回了延安府一趟,把老娘从老家接出来。他新近又被提拔了一级,眼看着就要做官人了,当然不能让老娘再在延安府为自己担惊受怕。

一别多日,王舜臣倒是有些想着韩冈、王厚、赵隆他们。将老娘安顿好,便兴冲冲的去找。推门走进王韶的家中,却听着赵隆的声音在喊:“日他鸟,怎么又给突袭了!?”

“谁让你没有及时展开队形!”这是王厚的声音。

“在玩什么?”王舜臣很纳闷,跨步走进王韶家的正厅。

房内的不仅是王厚,赵隆,还有王韶身边的另一个亲信杨英,另外,李信也不知什么时候从甘谷城回来了。四个人在王家的正厅里吵得热火朝天。一张一丈大小的方桌,被四人围在中间,桌面坑坑洼洼、花花绿绿的不知是哪家木匠造的。

“整队,反击啊!”李信面色狰狞的大吼一声,声音差点把屋顶震破。他双眼瞪着桌面,面红耳赤的模样,让王舜臣都被吓了一跳,什么时候这个稳得像山的锯嘴葫芦会吼出声来了?

“没用的,你们俩的兵被俺的五百铁鹞子从后方偷袭,全军混乱了。”杨英哈哈大笑着,他的一口江西口音让王舜臣听得累得很,也纳闷着,杨英总是跟在王韶身边的,怎么今天泡在了这里?

“不可能!哪里又冒出个五百铁鹞子来?”

看着赵隆捶胸顿足的模样,杨英笑得更是得意,“俺可是把五百铁鹞子藏在另一侧的山谷里,你的队伍过去时没发现。”

“胡说,俺们带的可是三千汉番骑兵,怎么可能没斥候!?”赵隆捶着桌沿,冲着杨英大叫。

“别弄坏沙盘!”王厚一声大吼,把赵隆捶桌子的手拦住。

‘沙盘?’王舜臣探头又看了那张奇形怪状的桌子,这玩意儿是叫沙盘?

而那边王厚拦住赵隆后,又责怪道:“谁让你事先没有下令!捶沙盘出什么气?”

李信抓了抓头,苦着脸问道:“那俺们现在下令成不成?”

“俺都杀出来了,你再下什么令?何况你们的三千骑兵被偷袭,又是被前后夹击,已经陷入混乱了!”杨英还是在笑着,赵隆气急败坏的样子,看起来让他看着很乐,“俺这回可是一对二赢了,愿赌服输啊。”

“俺带的兵怎么可能会被一个突袭就弄乱了阵脚,别太小瞧俺!”赵隆手一抬,好像又要捶桌子,但抬到一半,反应过来,连忙停手,一只拳头便傻傻的悬在半空中。

王厚也不理赵隆的抱怨,丢过去三枚骰子,“解除亲卫指挥混乱要十六点以上,十六点都不行。”

李信指了指桌上:“其他几个指挥呢?”

王舜臣就见着王厚低头翻着一本大约七八页的小册子,翻了两页,他的手停了下来,照着上面念道:“如果你的亲卫指挥能结束混乱,下一回合,只要掷出十四点以上,临近的几个指挥就能恢复。”

“不过在混乱中,被攻击损伤加倍,士气降低也加倍。你的士气现在只有四十点,只能承受两个回合的突击。”

王舜臣脑袋发懵,王厚、赵隆他们说的话,他每一个字都听得明白,怎么合起来偏生就听不懂了呢?

就看着王厚几人在房间里吵着,这么长时间了,他们甚至都没发现王舜臣回来了。

“王兄弟,你回来了。”韩冈的声音在身后响起,王舜臣惊了一下,忙回头,却见着王韶和韩冈不知什么时候就站到了他的身后。

只是他见韩冈的脸色有些难看,而王韶的脸色更为难看,简直都如锅底一般。王舜臣很少见王韶气成这副模样。

王韶狠狠的跨进厅中,虎着脸,一阵发作:“还闹什么?!都闹了一天一夜了,难道还不够?!”

厅中的争吵声顿时消失了,从菜市口上的喧嚣转为半夜古刹里的寂静。

王舜臣扯了扯韩冈的袖子,低声问着:“三哥,这是怎么回事啊?”

韩冈摇了摇头,连他事先都没想到事情会变成这样。

秦州不是东京,娱乐活动不多。除了长安以外,说整个关西就是一片娱乐文化的沙漠那是不为过的。不要说平头百姓,就是王厚这样的衙内,如果没有培养出逛青楼的爱好和体会到吟诗作词的乐趣,那他平常的娱乐活动,也只剩下棋读书了。如此乏味的日常生活,如果碰上了一个新奇而有趣的游戏,他们当然会沉迷进去。

这是理所当然的。

就拿王舜臣顶礼膜拜的种世衡来说,他曾经有一次要整修一座位于山头上的寺庙,一切都做好了,就是最后的一根大梁太过沉重,想拉上山既耗人工,又费银钱,实在有些得不偿失。

对此种世衡便想了个计策。

他先放出风声,说为了庆祝寺庙上梁,要办一个相扑大赛庆祝。而等到比赛当日,成千上万的百姓便涌到寺庙所在山头下。这时候,种世衡又说,大家一起动手,把大梁送上山去,也好早点看上比赛。结果他话音刚落,一群人便一拥而上,将大梁送到了山头。

其实种世衡玩得这一手也不算什么计策,即便是普通人,静下心来也能想得透。但偏偏上千人没一个去往深里考虑,都是想着赶紧把大梁拖上去,好去看相扑。这是日常娱乐太过稀缺的缘故。

前天当韩冈把类似于桌游的简易型的军棋推演教给王厚,又帮他整理了一份操作规则后,王厚便立刻沉迷了进去,还把赵隆、杨英,以及跟着张守约来秦州的李信一起拉下了水。

韩冈对此能够理解,只是王厚实在玩得过了头,昨天点着灯玩了一夜还不够,今天他和王韶都从衙里回来了,却还见着几人在玩。现在他看王韶的模样,砸了沙盘的心都有。

唉,韩冈暗暗叹了口气,不知道秦州城里有没有姓杨的大夫。

把王厚他们一起赶出了门去,连着王舜臣都遭了池鱼之殃。王韶拉着韩冈站在沙盘旁愁眉苦脸的叹着气:“官家年纪不大,跟二哥他们差不多。若是把沙盘呈上去,让天子变成二哥儿那幅模样,那我可就是罪人了。”

本朝自太祖之后的几个皇帝,都是爱对着阵图指手画脚。如太宗,他最喜欢的就是插手前线军务,经常把阵图夹在圣旨中发出去让前线将领照着来。真宗仁宗好些,但也玩过阵图游戏。英宗在位时间太短可以不论,而如今的天子,又是跟太宗一个脾气,喜欢插手前线军务,又是爱观兵耀武的性子,而且刚登基时就穿着盔甲跑去炫耀,若是给他得到军棋沙盘,少不得要沉迷进去。

王韶好歹也能算是个忠臣,当然不想看到皇帝变成跟自家儿子这般玩得废寝忘食,而且他也怕被御史指着鼻子骂,王安石那样的地位可以不在乎御史说什么,而他一个机宜文字,可没有把御史奏章当放屁的资格。

“天子受命于天,圣聪承于天际,岂会沉湎于军棋?何况朝中还有王相公一众宰辅,宫内又有曹太皇,高太后,怎么都不会让官家迷在沙盘里的。”

他虽然是在说着赵顼的好话,但言下之意却是管他去死。要是天子真的能克制自己的欲望,世上就没昏君了。可韩冈却不在乎。

王厚沉迷于军棋推演,当然不是件好事,王韶这个做父亲的都怒发冲冠了。但天子沉迷进军棋推演,对韩冈、对王厚、甚至对田计,也就是在沙盘上留名的几个人,却都是一桩可喜可贺的乐事。管教天子,自有太后、宰辅他们费神,韩冈他们只要享受军棋沙盘带来的好处就行了。

王韶想了半天,便自暴自弃的又叹了口气,道:“这事就不提了,等明天就把沙盘送去东京,省得再误事。”

韩冈点点头,这事本就该越快越好,若是泄露给窦舜卿去,那就麻烦了。

王韶在厅中绕了一圈,像是想起了什么,“对了玉昆,你昨天是不是写了一份文字,提议要在粮库中养几条狗来防盗?”

韩冈点点头:“最近不是说要在粮库中再添两个缺吗,下官觉得养狗比养人要省事,人的位子添了,再减下去却难。而狗就不会那么麻烦,不想用了,直接让人领走了事。”

“玉昆你这事就做岔了!”王韶却摇起头,“库中圈养猛犬的确有用,但没必要写成文字呈上来,说一声就够了。今次我帮你压下去,日后记着不要再写。”

“这是为何?”韩冈想不通。不立文字,怎么做事?

“玉昆你有所不知,旧年有一宋姓御史曾建言宫中应多养猛犬以卫宫掖,并说罗江犬为天下犬只之冠,其警醒若神……”

“然后呢?”韩冈问道,他心中突然有种不妙的直觉。

王韶长叹一声,却有着幸灾乐祸的味道:“他的名字自此就变成宋罗江了!也有人叫他宋神狗。御史也没法做,直接贬任外官。”

‘这……这也太惨了……’韩冈听着都觉得毛骨悚然,幸好王韶帮他把那份提案给压下去了。

“天下间口舌轻薄之人处处皆是,要谨言慎行,玉昆,你不想你的名字变成韩卢罢?”王韶难得说个笑话。

韩冈知道,王韶说的韩卢是战国策中所载的韩国名犬,若得了这个绰号,那真是一辈子都没脸见人。

他正正经经的点头道谢,“韩冈明白,多些机宜指点。”

