\section{第十章 千秋邈矣变新腔(三)}

“还真是慢。”张璪放开了要找的书,看了一下外面的天色,语带不快,“怎么现在才出来?耽搁了多少事。”

正是看着结果出来的时间时间差不多了,又拿到了阁试题目,三位宰辅结束了堂中议事之后,才没有立刻回厅,而是一起坐在这里等消息。没想到一拖多久。

“毕竟是制科。”韩绛和和气气,年纪大了,脾气也仿佛变好了一般,“考订试卷合格与否,的确要多议论一下才对。”

用得着吗?

韩冈暗暗摇头。

这个又不是进士科礼部试,需要排定考生名次,需要评判立论高下。崇文院的一众考官,只需要确认考生们解题的对错与否,书写上下文有无错讹,这样就够了。

“结果如何?”张璪问着堂后官。

堂后官来的匆匆,有些带喘。听到张璪询问后,也没先回答,而是向韩冈的方向瞥了一眼,但在对上韩冈的视线后,立刻又避开了。

看见堂后官的模样,韩冈心中有数了,脸色也微微沉了下来。

“十二人中只有三人通过,李之仪、宋涟和陈瓘。”

三人中,李之仪是韩绛推荐,宋涟为张璪推荐,陈瓘是元丰二年的榜眼,是在大名的吕惠卿所荐。其中李之仪、陈瓘皆是进士出身,有官职在身,而宋涟是布衣,为张璪门客。

而韩冈推荐的黄裳,却没有名列其中。

“黄裳呢?”张璪立刻追问道。

“黄博士没有通过。”

韩绛和张璪两人顿时回望韩冈,不无惊讶。

这一回制科重开,总共十二人应考。不仅仅韩冈推荐了黄裳,韩绛、章敦、张璪,甚至王安石都推荐了人去应考。不过其他人都是走了贤良方正能直言极谏和才识兼茂明于体用两科,只有黄裳是军谋宏远材任边寄科。

在这些人中,有榜眼,有进士前十,有同为宰辅的门客,但还是以黄裳通过的呼声最高——只因为谁都知道军谋宏远材任边寄这一科,可以说是为黄裳量身定做的。

尤其是通过阁试后的御试,黄裳必定能够通过。虽名为御试,却不可能让太后出题,只可能是由宰辅们将题目拟定进上。黄裳在他应考的那一科中,第一没有竞争,第二又有实际工作经验,其举主韩冈在朝中守边制敌经验最为丰富,在殿中为黄裳张目,纵使王安石、章敦齐上阵,也压不下他。

但黄裳偏偏在阁试上就落空了。而眼前的这两位,韩绛与张璪所推荐的考生,却同时通过了阁试。

韩冈自己不想作弊,对他人会投机取巧也有心理准备,只是事到临头,两边一对比,却还是发现心里一阵憋得慌。

面对韩绛、张璪投来的视线,韩冈回以苦笑,“看来黄勉仲当真是没有那个命。也要恭喜子华相公和邃明兄慧眼识珠。”

“哪里。黄勉仲的才干,朝中知者甚多。纵是一时不顺,也不会影响未来的仕途。”

这两位怕是都拿到了泄露出来的考题了。方才一个个正儿八经的琢磨考题的出处,原来跟自己一样,都是揣着明白装糊涂。

方才张璪说自己看不出那一道送分题的出处,韩冈当时就不信,一方面题目的难度的确稍低,另一方面,也是张璪演技差了一点,不如韩绛的水平高。

除了韩冈推荐的黄裳没有考中外,章敦所推荐的李和也没有考中。王安石推荐的孙冲也同样没有考中。吕惠卿推荐的陈瓘考中了,韩冈却不会去怀疑他。

韩冈素知章敦为人,不私其亲。他若是拿到考题,怕也不会给人,便是亲儿子也不一定会。

王安石的眼中更是揉不得沙子,孙冲虽是他的门客,能得到他的荐举,却不可能从他手中得到泄露的题目。估计三馆中的那几位,也没人敢拿着考题去奉承王安石。

至于陈瓘,吕惠卿离得太远,却没有可能帮他多少。

终究还是黄裳的那个状元头衔让自己大意了。韩冈想着。

纵然知道来自后世记忆中的状元头衔做不得数,但潜意识中,还是将黄裳的水平放在了状元一级上,认为他肯定能够通过考试。换作是对黄裳的才学没有什么信心,在别人都有可能作弊的情况下,韩冈也不一定会崖岸自高。做事总不能彻底黑下心去,也难怪自己推荐的人不能考中。

现在看一看,黄裳的这位记忆中的状元,还是比不上真正的榜眼。陈瓘可是货真价实的进士及第,不过韩冈也没听说过他拜在吕惠卿的门下,大概是同为福建人的缘故。

韩冈轻易的便认了命,这让韩绛、张璪突然间有些不适应。

他这样的态度实在太过奇诡,就两人所知,韩冈从不是简简单单就认输的人。

张璪想了想,问道,“那黄裳的考卷看到了吗?”紧接着又问,“对错如何,几题为‘通’,几题为‘粗’?”

这名堂后官显然已经有所准备,“试卷下官没有看到,但下官打听了一下,黄博士好象是‘通’‘粗’各居其半,仅仅差了一点点。”

这不是差了一点点,六十分及格,考了五十九分叫差了一点点;六题中通四题合格,只对了三题,那可差得多了。

“秘阁那边应当已经将名单进呈上去了,且去将考卷取来。”

就跟进士科礼部试和殿试一样,在批阅完之前,外界的力量想干涉都很难,取出试卷更不用说。但录取名单进呈天子之后,将试卷拿出来就很容易了。要不然,名列前茅的贡生们的试卷也不会满天飞,更不会有编订成册、由名儒点评过的程文存在了。

没有太久,十二名参加制举的士人,他们的考卷都顺利的被取来了。

随手翻开上面的几页,一看到黄裳的考卷,张璪的心里就咯噔一下,‘坏了。’

的确是坏了。

在崇文院考官们的评判下,六题中,黄裳的答案是三‘通’三‘粗’,没有达到‘通’四题的合格标准。

但是除了出自《周官新义》的一题暗数,剩下的五题,黄裳都指明了出处,包括方才韩绛、张璪、韩冈最后讨论的《春秋公羊传注疏》的那一题。

在这其中,黄裳将四题的原文准确引用,只有之前有关唐时宰相的一题,在引用上下文时,黄裳在人名顺序上出现了一次错误,所以被考官据此判‘粗’。

到这里为止,还没有什么问题。虽然说因为没有将《唐书·宰相世系》的姓名按原序列出,的确是苛刻了一点,但也勉强可以说得过去。

可除了《周官新义》和《唐书·宰相世系》这两题之外,另外一题,黄裳对《墨子·明鬼》中的一节的议论,被考官判粗,那就说不过去了。

这也就是张璪叫苦的原因所在。

正常来说,在阁试中,考生但凡能将题目的出处准确找出,并准确写出了上下文,之后的‘论’,只要不是写得太差,有犯讳或是白字,一般考官是不会穷究内容的。

毕竟能被推荐参加制科,都是当世有名的才子,至少才学卓异,超出侪辈。并不比名列三馆秘阁的考官稍逊。有的考生在儒林中的名气,甚至远在考官们之上。考生的论点与考官抵触,究竟是谁对谁错,根本都扯不清。难道说那些闻名于世的大儒参加制科,他有什么独到的见解,还要得到三馆中的官僚认同不成?

放在是直言极谏科,更是皇帝都要顶一顶,与考官不是一个路数,再正常不过。

能将阁试中刁难人的题目,全都找出出处,写明上下文,已经足以证明考生的能力了。

可这一回,三馆秘阁的考官偏偏将黄裳写对了出处,写明了上下文的一道论判了错。

在张璪看来,黄裳的这一篇论,除了论点异于新学、偏近气学之外,并没有别的问题,也没有犯讳,文采也算得上不错,不说有多出色,但以其他人的论述作比较,已经足以通过了。

试卷从张璪、韩绛的手中传给了韩冈,韩冈看了一下之后,神色立刻就变了。

‘韩冈铁定要闹事了。’

这是张璪看见韩冈阅卷表情后的第一个念头。

“玉昆?”他试探的小心问道。

韩绛也是一瞬不瞬的盯着韩冈。

韩冈要是想为黄裳讨公道,必然会拿着通过的三人的试卷作比较,这样一来,他们两人也不能置身事外了。

“这最后两题的出处是在哪里?”

在韩绛、张璪的盯视中,韩冈忽然抬头问道。略嫌阴冷的神情,又恢复如常。

“一题出自《春秋公羊传注疏》,另一题是出自令岳的《周官新义》……玉昆你应该知道吧。”张璪指了指传到韩冈手中的试卷。

韩冈的确是明知故问,看到前面的几份考卷,尤其是参考已经通过的三人的考卷,已经能让他明了题目的出处了。

一个出自唐代徐彦的著作,《春秋公羊传注疏》中的疏。

《春秋》是鲁国国史,为孔子编修,是为儒家最重要的经典之一;《公羊传》是流传下来的《春秋》最早也最重要的三家注释之一,为公羊高所著;汉代何休的《春秋公羊解诂》,是公羊传的注,是注释的注释;而徐彦所作的疏,便是注释的注释的注释。出自于此,又加上是前后颠倒、改换句读的暗数,能靠自己找出出处,难度不低。

另一题则是出自《周官新义》。虽然是今人的著作——也就是王安石所著——但这的确是得到官方认定的经籍注疏之一。不过在张璪看来,这一题虽说是暗数,未免扭曲的太过分了。八个字中,有四个无意义的助词,这样鬼才能猜得到。这一题,黄裳没有做出来,通过的三人中,陈瓘和李之仪也没有做出来,倒是张璪推荐的宋涟做出来了。

“这六题分别出自经、史与古人、今人的注疏。”韩冈指着试卷对韩绛、张璪说着,平静的脸上看不出愤怒,“说起来崇文院的几位的确是煞费苦心。”

韩绛默然不语,张璪点点头,皆在等待韩冈的下文。

“不过这是贤良方正能直言极谏,才识兼茂明于体用两科的考题吧?……军谋宏远材任边寄科的考题在哪里?”
