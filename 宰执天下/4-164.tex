\section{第24章 南国万里亦诛除(四)}

米彧喝了一口茶,润了润喉咙,“小人在京城的生意也多亏了冯行首的照应。故而设宴请了几次。在出京的前一天,在球场上,棉行的蹴鞠队十五比三大胜了车马行。回头庆功宴上说起平南之事,冯行首便放言说,有韩龙图和李将军在,必能攻破升龙府,大胜而归。冯行首向来不出虚言,他既然这么说,都没人敢跟他赌一把。”

米彧絮絮叨叨的说了长长一大段话,李信是怀疑他跟冯从义的交情,他忙不迭的为自己辩解了一通。

李信的脸上看不出信还是不信,以米彧十几年的江湖阅历,也看不出个究竟。在官场久了,城府也深了起来,“米兄是布商,如果是要贩货,当是往琼崖去,怎么往交州这个穷地方来?”

“交州怎么能说是穷地方。”米彧笑了起来,“既然交州的治所设在海门,想必章、韩两位学士是有心于此开港,日后交州财货,也能通过海路往来,不用翻山越岭。”

米彧小心的偷眼看着李信脸上的神色,“不过交州开港,要想做到如同杭州、广州一般,则是时日久长。非千万人之力,难以为之。”他站起身,向着李信躬身一礼,“小人不才,愿附骥尾,以效犬马之劳。”

米彧不介意将自己手上的一点家当全都砸给李信。结交上官,哪有不花本钱的道理?米彧也是读过一点书的,只是福建竞争太大,自知没有考进士的能耐,便下海从商。他一生最佩服的就是吕不韦,后事不论,那可是有着投资的眼光,做到了一国宰相的商人。

但李信不为所动,空口白话他见得多了:“交州刚刚经过战乱,三五年内都不见得会有什么出产。不知米兄有没有耐性等到州中安定下来?”

米彧当然不会有那个耐性,他还欠着人几万贯的钱钞呢。

“将军有所不知,其实并不需要等那么久。大宋地大物博,什么都有,只是没有好马。故而熙河路的根本是茶马互市,朝廷要在熙河路费尽心思,也是为了战马。等到路中的户口多了起来,又是有了韩龙图的提议、韩老封翁的主持,路中才开始种棉种粮,有了棉布的出现。但马才是根本。”

米彧对商场上的敌人做过了一番深入的了解,陇右棉行的兴起,他都是着意打听过。眼下在李信的面前说出来,却是正好证明了他与冯从义的来往并不是自吹自擂。

见李信沉思的点起了头,他精神一震,继续道:“交州能有什么。水果、木料,只要是稀罕货,在北方的确能卖上高价,眼下的确是要等上三年五载。而且算起净利,同样的一船货,都不会比粮食高上多少——一个是处理起来费时费工,另一个则是占地方。

眼下能立刻拿得到的,唯有香药!豆蔻、丁香、沉香、象牙、没药、白檀、鸡舌香,交州的这些特产,到了北方都能卖上高价……应该说是天价。”

李信脸色稍稍一变,“听说香药与盐、铁一般,都是禁榷的。”

“香药名目繁多,禁榷的只有犀角、乳香、龙脑。且国中转运,并不干市舶司的事。禁榷只能禁外番货,而从海门运到杭州,最多也只会被市舶司抽解一成做税,再和买【平价收购】三成而已。还有六成在手,只要卖出去,其利十倍可期。得利之大,只看交趾靠着与大宋的香药贸易,变成天南一霸,便知端的。”

但李信对此并不理会,油盐不进。何况米彧说的话不尽不实,“这样的买卖能做几次?”

“一次难道还不够?”米彧凑近了,神神秘秘的低声说道,“眼下想到这一节的还不多,只要一船便能有十万贯的收益,但过上半年,就只有两三万贯了。”言下之意,想丢开自己,去找表兄弟来转这份钱,可是缓不济急。

十万贯的确不少,但分到自己手上可就不多了。李信哪里会将这种带着风险的收益放在眼里。他会接见米彧,也只是想知道表弟和家中的消息而已。他在顺丰行中有干股的,每年都有一两万贯的稳定收入,而且还在不断增长,根本就不缺钱花。

心中有些不快的看着凑到近前的一张奸猾谄媚的笑脸,李信皱眉想着,‘难怪三哥儿不喜欢行商,都是这般货色。’

李信知道他的表弟并不是歧视商人,依照韩冈的说法,工商不分家,种出来的粮食即使不卖掉,也可以存在家里,总不会浪费掉。如果工坊里面出来的货物卖不出去,就只能空占着库房,让人饿肚子,只有贩售出去,才能算是有用之物。

但韩冈并不怎么喜欢单纯的行商,那等人不事生产,对国家益处不大。他更喜欢工农之徒,不论是农人还是工匠,从他们的手中都能够有所产出。而且商人若没有自己产业,就是无根之木,随便出点意外便是要倾家荡产。

所以虽然顺丰行如今生意越做越大,但根本还是在巩州乡里的土地和作坊上。没有牢牢抓在手中的根本,靠着棉布的主业,只是凭着江湖转运,如何能敌得过京城中的那一干豪门?

李信也不喜欢米彧这等打算赚一笔就走的商人,故意为难他道:“贩牛的买卖如何?交趾倒是牛多。江西、荆湖南方诸路,都从广西贩牛,听说洪州、江州等地,都不对牛只收税。只为了能多一点牛来耕种田亩。此事于国有益,若是米兄有心,我倒是可以去李知州那里关说一番。”

米彧脸色变了一下,但立刻又恢复了谦卑的笑容:“广西牛多,交趾也不少,可惜都是水牛,只能在江南养着。到北方还是得靠黄牛。”

百里不贩樵,千里不贩籴。这是如今做生意的俗语。

大宋的商税税率并不低,过税是两分,住税是三分,每过一座税卡,就要在成本上加上百分之二;当到了地头,开始贩卖,就又要加上百分之三。

路途越远,就越是得选择等带来高利的商货。否则一点利润,就会如同落入沙土里的清水一般,被沿途一座座税卡吸得一干二净。

从海路走,倒是可以免除了走陆路时,穿州过县多如牛毛的过税,但风险怎么算,海上泛舟并不是那么稳妥的,主要就是风急浪高的珠母海,比起从广州往扬州去的水路,风险要大得多,每年都要有几艘沉船。如果没有足够的利润,他凭什么要去冒那个风险?

“那还真是可惜,想不到贩牛的生意这般难做。”

李信也不打算多说什么了,他只要练好兵,打好仗就行了。有表弟韩冈,还有老上司章惇襄助,日后有的是机会晋身三衙管军,没必要跟这等小人结交。

要不是表弟几天前随口说了几句,准备怎么在交州发展生产,问清楚了表弟冯从义的近况,也就点汤送客了,哪里会跟区区一个行商说这么多废话,李信本来就是不喜欢多说话的性子。

看到李信有点汤送客的意思,米彧就有些慌了。他没想到还有不爱钱的将军,他可是听说郭逵郭太尉对贩运商货的爱好让太尉夫人都看不过眼了,出身关西的将领,哪一个不是养着几支回易商队,在军饷中还要拿着军籍簿上空额,克扣上一份钱粮下来。

连忙道:“不过往江东贩牛的海路,小人还是有几分熟悉,只是对耕牛的商情不熟罢了。若是将军能有片言相助,小人岂有不愿之理?

……………………

从海门港上船,到钦州下船,只用了两天的时间。再从钦州出发,抵达邕州,最多也只需要三四天而已。

比起全程陆路来,的确是省时省力。除了在海船上,不能脚踏实地,让人放心不下以外,倒真的没有别的缺点了。

经过了一年多的重建,钦州城和安远港已经大致恢复了旧观。

韩冈望着新旧参半的建筑,对章惇笑道:“日后海门开港,来往钦州的商队也不知会多还是会少。”

“只为了钦州的珍珠、玳瑁、珊瑚,商队就不会少。”章惇说道。

虽然不如廉州的合浦珍珠知名,但钦州也是产珍珠的。从船上看到数以百计的采珠人不断的出入海中,将一枚枚珍珠贝从海中捞起。

再望远一点,海岸沿线,如同小小的蛋壳一般浮在水面上的船只,数以千百计,每一艘船,就是一户疍民。而在两广福建的沿海诸州,这样毕生生活在船上的疍民,数以十万。

“如果能将沿海的这些疍民编户齐民,好生的安置下来,朝廷在广西的根基就会又稳上了一分。”

章惇闻言便是一笑,韩冈说是广西,其实是在说交州,他的一门心思都放在这上面。不过话说回来,只要交州再多上两三千户,那就是多了一倍的兵源。与蛮部的户口比例,也能更让人放心一点。

“疍民不知稼樯,除了捕鱼、采珠可就没有别的本事了。”钦州的知州在后面说着,“如果将疍民们都编户齐民,那钦州可就没珍珠了。”

“难道人还比不上珍珠?岂能贵物贱人。”韩冈反问道,“潜入深海,寿命长的不多,若教他们种地垦殖,又有几人不愿。”

