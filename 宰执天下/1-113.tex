\section{第43章 百里河谷田一顷(上)}

【第一更,求红票,收藏。从今天开始,更新时间改为零点,十二点和十九点。】

韩冈从流内铨徐步出来,李小六立刻迎上。虽然韩冈脸色看不出与进去时有何异样,宠辱不惊的气度让他很难外露出激烈的情绪波动,但李小六心知,没有区别便是好事。

“恭喜官人!”李小六嘻嘻笑着上前为韩冈贺喜。

“还要再等几天。”韩冈心平气和的说着,“只是刚刚通过铨选,要拿到告身才算。”

李小六并不清楚铨选和告身,但他会凑趣:“进士发榜到琼林宴之间,也隔了半个月呢。可谁能说没参加过琼林宴就不是进士了?”

“就你嘴会说!”韩冈摇头轻笑。

听见主仆二人的对话,周围投来的目光便带上了一点敌意,像刀枪一般戳了过来。韩冈不以为然,被一群守阙的闲官狠狠的瞪着,反倒有一点脚下踩人的痛快。

带着李小六离开嫉妒汇聚成的漩涡,韩冈一边走,一边计算着自己还要在京城待上几天。

自己通过了铨选,接下来流内铨定下韩冈的本官和差遣后,便要呈文政事堂,等政事堂审核完毕,又得移文官诰院。官诰院是制作和颁发告身的机构,并兼作审查,这一步手续没有五六天下不来。如此一算,韩冈想要拿到自己的告身,也就是证明自己官员身份的证件——虽然不会是个硬封皮的小本子,但实际的意义却是一样——至少还要等个十天半个月。

‘足够急脚递在京城和秦州中跑个来回再带个几百里了。’韩冈暗暗为官僚机构的效率叹气,想想自己已经出来了二十天,一日四百里的急脚递也能往秦州跑两个来回了。而自己最快也得到三月初才能启程返家,来往公文更不知跑了多少回了。韩冈眉头轻轻皱起,也不知他和王韶制定的计划到时能不能成。

回到驿馆,却见刘仲武已经早早的回来了。他尽管沉稳,但如韩冈一般的养气功夫却是没有,嘴角唇边的笑意怎么也掩饰不住。

“恭喜子文兄了。”韩冈笑着说道。

刘仲武看了半天,也没看出韩冈的表情中有没有藏着铨选的结果。他陪着小心的问着:“……那官人你呢?”

韩冈笑着点点头。而李小六帮他出头回答。提得高高的声音有着引以为荣的得意:“我家官人哪有不过的道理?!”

“说得也是!说得也是!!”刘仲武哈哈的笑了,“以官人大才当然轻而易举。”

韩冈坐下来,问着刘仲武,“不知今日天子有没有来看子文兄射箭?”

“俺也以为官家会来看看!谁想到枢密院都承旨来主考。”刘仲武虽是在抱怨,但话里话外都透着喜意,“不过俺也没想那么多,只顾着射。俺用两石弓步射了十七箭,托福却都中了。又换了马,马射十箭还是都中了。再换了弩,俺先拉五石的,又拉了六石的,轻轻松松。都承旨见俺有把子牛力气,就使人拿了七石半的硬弩来。那力道,跟架在城墙上的八牛弩也差不离了。俺是用出了吃奶的气力,方才拉开。”

能拉开七石半的硬弩,这把子气力,让韩冈为之乍舌。虽然军中一直有传闻说有人拉弩能过八石,但谁也没真的亲眼见过。而刘仲武的七石半,已是骇人听闻。韩冈往刘仲武的下三路看,这厮的腰腿气力当是不小,向宝送他的美女当是被折腾惨了。

“……最后都承旨看着俺卖力的份上,给俺判了异等,其他十几人都不好意思在俺后面练了。”

刘仲武一番话说的得意非凡,一贯的稳重不知去向。不过这也难怪,他得到的试射异等,比优等还要高上一级,非武艺卓异不可得,几年也不定能出一个。而授官,往往也会比正常的三班借职要提高一级,直接任三班奉职。如果不论文武之别,真要计较起来,三班奉职比韩冈的判司簿尉都要高。当然,文武之别实际上是存在的,即便是从八品的东头供奉官,西头供奉官这等小使臣中最高的两级,也不能说真比从九品的选人强出去。

刘仲武今次在殿上演练的都是弓弩。试射殿廷,顾名思义本就是考得射箭。大宋军中最重远程兵器,向来是三十六种兵器,弓弩居首,十八般武艺,射术第一。韩冈现在只为王舜臣感到可惜,他神技一般的连珠箭术如果在殿前施展开来,就算刘仲武也得退避三舍。看到三十步外的箭垛上一眨眼的功夫就长出一朵花来,任谁都要惊掉下巴。可惜啊……

“韩官人,今天要不要好好喝上一顿!”刘仲武过去是躲着韩冈,怕被他拉着喝酒,后来虽说认命不躲了,但也没有主动过,今天可是第一次拉着韩冈喝酒。

“能与子文兄共叙一醉,当然是最好。只是啊……”韩冈很遗憾的说着,“我等会儿还要去张、程两位先生家报个喜信。这样吧,明天在樊楼里摆一桌好了,来了东京一趟,也得见识一下樊楼春色。不然回去后一说,连樊楼都没去,谁会相信我们真的到东京了。”

韩冈会说话,刘仲武被拒绝了,也没不高兴,反而笑了起来。点着头,“说的也是,不去樊楼,那就是白来一趟东京了。”

韩冈午后再次去了王安石府。刚到门前,就看到一名宦官捧着一个长条盒子,领着几个从人走进王宅,不过很快他又带着盒子和从人被王安石的小儿子送了出来。瞧他的模样,这次宣诏终究还是失败了。

看着传诏的中使骑马离开,韩冈猜测着王安石到底什么时候才会重新出府理事。想来应该不用太久的时间,他看看王府前的街巷,停在这里的车马比起前几天又多了一些。随着圣旨和辞章的交替往来,朝堂政局越来越明朗,王安石的地位也越来越稳固,所以原本散去的官员,现在又重新聚在王家的府门前。宽有两丈的道路,已经被来访官员的车马堵成了一条羊肠小道。

韩冈进了门房,里面早坐满官员,他们的心意也是跟韩冈一样,都是在等着王安石的出面。这么些人也是天天来此,几天下来,各自都混了个面熟。韩冈会结交人,在众人中人缘甚好。他进来后,座中官员便纷纷跟他打招呼。等他坐下,便一起东拉西扯海阔天空的闲扯起来。基本上,在门房里的官员都跟韩冈一样,皆是坐上一个时辰半个时辰就起身,这是变法派的官员们在表明自己的态度。如果不来,等秋后算账,那就是得怨自己的腿脚不勤了。

王安石还在称病中。理所当然的,韩冈也照样还是没能等到接见。在门房处坐了一个多时辰,表示了一下恭谨的态度,便韩冈告了罪起身离开。出来时,日已西斜,但大门口的车马不见减少,反而多了一些。

离开王安石府,韩冈直奔小甜水巷的方向。从城西北的王安石府,横贯了大半个东京城,用了半个多时辰,方抵达张程两家的门外。

看到韩冈,张戬和程颢连问都没问铨选的事,等韩冈说起,也不过是点点头,直视为理所当然,根本都不替韩冈担心。也难怪,毕竟新官铨选难度实在太低,即便韩冈被两位主考的令丞使坏,还是一无所觉的顺利通过,由此可见,平日里的铨选有多么简单。

“通过铨选不代表能做好官,日后行事要记得上不愧天,下不愧地,不负天子,不负黎民。”程颢语重心长地说着。

韩冈恭恭敬敬的行礼:“多谢先生们的教诲。韩冈必日日铭记在心。”

一番训诫之后,张戬让了韩冈坐下。沉声问道:“玉昆。有件想请教你一下。”

韩冈连忙站起:“请教绝不敢当。有什么事,先生尽管问。”

“坐,坐。”程颢笑着示意韩冈重新坐下。

等韩冈落座。

“也不是什么大事……”张戬便用着漫不经意的语调说着,“只想问问玉昆你,有关在古渭和渭源屯田的事情。”

韩冈点了点头,道:“先生问对人了,此事学生正好知道。”

“说来听听……”

韩冈心中透亮,看来他和王韶的计划已经在朝中传开了,却不知御史台对此看法如何。只是不论程颢、张戬他们这些御史们现在持的是什么态度,自己在情在理都得让他们变成河湟拓边的支持者……至少不能是反对者。而现在便是得看自己的表现了。

韩冈心如电转,嘴里的回话却没有半点磕巴:“屯田渭水上游,是王机宜的收复河湟的第一步计划。欲收河湟,便必须收服当地众蕃。而蕃人多是畏威而不怀德,为了震慑他们,就必须在古渭和渭源派驻一支官军,必要时,还得消灭一两支被西贼收买的蕃部,以便杀一儆百。但不论是驻兵还是开战,物资粮饷消耗总不会少,如果全数依靠外运,不论是朝堂还是陕西转运司,都支持不下去。所以王机宜便想着在当地自行解决部分粮饷,故而便有了在渭河中上游两岸屯田的计划。”

张戬道:“最近王韶已经用专折将他的这份计划呈上来了。”

韩冈点点头:“学生出来时,已经听说王机宜正在写这份奏章,大体内容也有所了解。渭源至伏羌城,两百余里河谷,宜耕荒地近万顷,而能开辟成良田的地方至少千顷之多。如果将千顷良田开垦出一半来来,出息就已经足够支撑一支两千人的军队,而屯垦这么一点田地,只需要他们一年的时间。”

“是吗……”张戬漫声应了一句,沉默的看着韩冈一阵,突然间眼神化为刀剑,单刀直入的厉声问道:“那窦舜卿为何说秦州至渭源,宜垦荒田只体量得一顷四十七亩?!”

