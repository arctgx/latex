\section{第18章 霁月虚明自知寒(上)}

“去把《幼学琼林》拿来。”

章惇回家后的第一件事,就是吩咐下人将那一部新出不久的蒙书找来。

章家下仆没人敢问为什么,幸而章惇的书房中也收藏了,片刻之后,四部十五卷本的《幼学琼林》便放到了章惇的案头前。

飞快的从头到尾翻了一遍,章惇皱着眉头将书放了下来。

《幼学琼林》出自横渠书院,由韩冈加以修订,分为成语故事、诗词歌赋、自然地理、日常医用四部。

成语故事部,主要就是历史上的小故事,以及一些成语的本源,三皇五帝、夏商周都有一点,还有有关甲骨文的新发现。

诗词歌赋部,当然不会有艰涩深奥的,而是一些文字简单、脍炙人口的诗词,比如锄禾日当午;鹅鹅鹅;举头望明月、低头思故乡一类的,枯藤老树昏鸦那一首也被选进去了,只是作者为佚名。

自然地理部,就是气学的拿手好戏,现在的士林,只要看见自然二字,就立刻会联想起气学来。其中有天地自然间的各种尝试,还有十几项简单易行的实验,不仅证明了书中言辞的正确,更能吸引学生对气学的兴趣。

而日常医用,更是韩冈的特长。从日常清洁,到疾病防治,以及急救,溺水、烧伤、跌打损伤等意外伤害的急救和治疗,都说了一遍。牛痘发现的过程,也写在上面,甚至在文中,自承病毒命名之误,对自己的错误毫不讳言。

从内容上看,这是一部蒙书。脱离了五经的范畴,文字浅显易懂,内容则是以学以致用为目的,让学生不致成为死读书的措大,不至于在车船中让人‘伸伸脚’。

但这么一来,想要阻止韩冈在解试中加考一门可就难了。

从书名上可以一眼看出,这是‘幼学’琼林,并非气学的。韩冈要在解试中加考一门常识,题目只会在给小孩子看到《幼学琼林》中选,有多少士子有脸去反对?九岁十岁的幼子尚且能侃侃而谈,寒窗十载的读书人却畏之如虎,徒惹人笑。

虽说这一次一旦开了头,日后可就不是《幼学琼林》,而很可能是《自然》了。可韩冈现在仅仅是加考一些常识,而且内容又不似经义那般争议不断,有实验为证,根本挑不出错来。实在让人没办法付出巨大的代价去阻止。

章惇起身,推开门,走到院中。

夏日夜晚的星空,似乎也不如过去明亮了。而理应横贯天空的星河,则暗淡的几乎看不清了。

冬日烟雾满城的生活,章惇已经习惯了,但如今就连夏天,只要不刮风,天空中便也仿佛是蒙了一层薄纱。

京城北郊的钢铁工坊和铁器工坊,即是强国之本,也是京城军民对川贝、枇杷等清咽止咳润肺类药材需求大增的祸源。

国中每个月产生的钢铁,是十年前一年的产量,而质量更胜一筹。巨大的水锤更是让各色兵器和农具流水一般的生产,据韩冈声称,一旦蒸汽机被发明和投入使用,铁器厂中能够改以蒸汽为动力,蒸汽锤能够更加简单将钢铁打造成型——这不仅仅是韩冈在朝堂上亲口所说,更是《九域游记》里面所描述的未来。

这就是气学和韩冈带来的变化。管理工坊技术的官员,已经全数成了气学的门徒。韩冈甚至能给那些工匠子弟一个身份,只要他们去学习气学的知识,而不是去学习三经新义。而通过几年的学习和历练,那些工匠子弟的技术甚至超过了他们父兄。

气学的势力就这么一天天的膨胀,只是在朝堂上,一时还看不出来。

章惇不在乎新学的颓势和气学的扩张,新学并不是他的心血,自不是他的新学,可看着韩冈如此有耐心的将新党的根基一点点的刨开,作为对手,这实在是一桩很让人气闷的事。

《自然》中的数理问题,章惇看得头晕脑胀,那些用甲骨文中的生僻字符,充作所谓的代数符号,简直像天书一样让人费解。如果韩冈在进士科中加考天元术……不,就是给出半径,要让人去计算球的体积,不懂得计算公式,有几个人能做出来?

王安石当年直接从进士科礼部试入手,说动先帝,一举将诗赋改成了经义。

而韩冈不如王安石那般激进。先从殿试和考试方法着手,再增加诸科内容,一点点的进行改变。即使现在,也没有贸然做出将经义内容由新学改成气学。只是加考,只是百分制,却是坚定一步一步动摇新学的根基,最后,自然是顺利成章的彻底改变。

章惇望着黯淡的圆月,他已经切身体会到了,当年王安石为什么要与韩冈鱼死网破的理由了。

……………………

“章枢密肯定要跳脚了。”

韩冈的书房中,冯从义呵呵笑道。

刚刚抵达京师,便从韩冈嘴里,听到这个有趣的消息。

韩冈虽只打算先改动一下解试的科目,而且仅仅是加考,冯从义并不觉得韩冈保守。谁都能想到,韩冈这么一步步的对科举下手,现在虽不去与新学争夺官学的身份,但也是迟早的事。才智之士,哪个不懂得居安思危的道理?

“反对也无妨。”韩冈并不在意。

冯从义点头:“这倒是。到时候讨价还价一番,也不会吃亏。”

韩冈微微笑了一下,初来乍到的冯从义,还是没太理解他和苏颂掌握了整个政事堂的意义。

他现在已经是宰相了。行事激进,固然会引来对手的反击,但当他稳重小心的行动,那么反对者的数量也不会太多。只要还没有将床给抽走,大部分人还是愿意继续睡下去,而无视肯定会到来的结局。这是章惇都没办法改变的事。只看摆在桌上的东西就知道了。

随着韩冈的视线,冯从义的目光也落到了书桌上。

“这是……”冯从义看着一堆厚厚的卷册,不像是公函的样子。

“是行卷!”

“……不会是诗词歌赋吧?”冯从义笑着问道。

韩冈笑了起来,“没几个人会送错礼物的。”

冯从义明白的点头,“不过合眼的礼物不多吧。”

“的确。”韩冈笑容中有了继续无奈。

古有献文搏名之风,左思献《三都赋》与张华,刘勰以《文心雕龙》进沈约,便是有名的例子。

至唐时便有了行卷一事。来官宦门第的士人,往往都会带着自己的得意之作投递到高官显贵家的门房中,期盼能得到青睐,由此一举成名,或是名登黄榜,一句‘画眉深浅入时无’,便是行卷之文。

这个风气,现在也依然存在。不过如今天下士人都知道,想要进苏、韩两位宰相家门,诗词是没有什么用的,最好的行卷只有一种,能发在《自然》上的论文。

《自然》刊行于世多年,如今通过邮传遍行天下,通讯会员超过一万,而得到会员资格的只有两百不到。通讯会员只有一个铜扣作为标识,当新人订阅全年的期刊时,便会得到一枚。但只有发过论文的成员,才能得到会员的身份,拿到一枚银质的徽章。

由宫中大匠亲手制造的银质徽章极为精致。圆形的徽章上,代表地球的圆型图案被经纬线分割,正中央嵌入了一枚打磨过的蓝宝石。而通讯会员的铜扣同样是圆形的,经纬线只有纵横三条,也没有镶嵌宝石,完全是翻模铸造出来。

拥有一枚自然学社的银质徽章,便是叩开宰相家大门的敲门砖,出门扣在襟口上,识者无不称羡。

可惜能做到这一点的,凤毛麟角。

“哥哥不用急,以后肯定会渐渐多起来的。”

“没那么简单。”韩冈摇摇头,“毕竟读书的人还是少。”

“天下读书人,百万总是有的。”

“还是太少了。官是百里挑一,进士是千里挑一,可自然学社的成员,却是万里挑一。你说能有多少?”韩冈反问道。

“乡里读书虽少,可城里就不少了,只要蒙学中用气学的书,以后自然学社的会员肯定会越来越多。”

韩冈对表弟的敏锐很高兴:“说的也是。不能寄望于如今的士人,只能期盼蒙学中的那些学生,日后能有更好的成就。”

以大宋的富庶,城里市民阶层人数数量并不少。从比例上不超过两成,可人数上已经达到千万级。在这个世界上,比得上任何一个国家了。

市民家里的孩子,不用像生长在农村的孩子一样,五六岁就要随着父母下田,或是去打猪草、拾柴,又或给人放牛放羊,主要是去做学徒,或是做些零散杂工,不过那也要九岁、十岁之后。所以顺便上一下蒙学,做一个会写会算的学徒,找到差事要容易不少。

在韩冈而言,义务教育还有难度,但在城市中提倡幼童皆尽入学,就算只有三年时间,也足够培养出气学的根基来了。

一直以来,冯从义都是韩冈的代言人,韩冈的一言一行,被冯从义看在眼里,又如何不明白韩冈的目标和手段:

“如今关西蒙学,皆用横渠蒙书,六岁读《三字经》,七岁学《算术》,八岁九岁就能看《幼学琼林》,有关西百万幼子在,十年之后,气学将无可动摇。”

