\section{第14章 霜蹄追风尝随骠(20)}

暮色渐渐降临。

最后一丝阳光消失在西面山峦之下,天边的彤云也褪去了红光。自北而来的大军已经进抵柳发川营寨外,而寨中的宋军完全没有反应。

隔着最外层的一重栅栏,只能看到寨内一片火红,红彤彤的映得漫山的冰雪都泛着血色。

喊叫、呼号,随着夜风,从火光中传出来。连同滚滚浓烟,一并卷到耳边鼻尖。

“总管。”下面的将校冲着罗汉奴大叫,“肯定是萧枢密安排的内应!”

风声火声杀声并起,望着红光笼罩的宋军营寨,火焰在耶律罗汉奴的眼中也同样燃烧着。

扯着马缰的手,攥紧了,又松下来,但立刻又再次攥紧。喉咙仿佛被火焰在烧烤着,口干舌燥,让他不停的舔着嘴唇。

内乱中的宋军大营,外围的防线完全看不到有人守御。在耶律罗汉奴眼中,就跟纸一样薄弱。似乎只要一伸手,就能扯个粉碎。

攻下宋人的营寨,功劳什么的,耶律罗汉奴没兴趣,而且宋辽两家还没有撕破脸,也不可能公开给封赏。但宋人他们的武器、甲胄,不论是神臂弓还是斩马刀,或是板甲,都是一笔巨大的财富。能给自家的族兵装备上,就是宫卫也得逊色三分。

宋军大营乱成了这副模样,趁着现在的混乱攻进去,耶律罗汉奴有七八成的把握,在占尽便宜后能全身而退。剩下的两三成,则是少占些便宜,但照样能全身而退。

攥着马缰的手松了开来,接着却提起了架在马鞍前的长枪。

“儿郎们,都准备好了吧?!”

耶律罗汉奴一声暴喝,下方一片应喝声响起,由近及远,一圈圈扩散出去。在宋人的大营之外,他们早就忍耐不住了。里面那么热闹,哪有不去凑趣的道理。

长枪高高挑起,轻轻画了个一个圆,唰的劈下来指着前方的熊熊火光:“那就杀过去!!!”

……………………

“萧十三到底想要打哪里?”柳发川和暖泉峰同时传来辽军逼近的紧急军情,让李宪很是头疼。

辽军在东胜州的兵力,与胜州诸寨堡的官军兵力相差仿佛,不可能真正的分兵,同时攻打两个军寨。只可能挑选其中之一为真实目的,而另一个则是掩护。

“柳发川有辽人的内应,说起来应该是萧十三的目的。但暖泉峰那条路,能使用的兵力更多。而且暖泉峰的城寨还有大半没有完工,比不上柳发川的进度。”

“用不着去猜。”韩冈慢慢的翻着手上的手抄本,气定神闲,“来一个杀一个,两边都守住,任凭他有千般计,也别想有施展的余地。”

韩冈这些天来,倒是很清闲。订立了计划,做好了预备方案,让每一名将领和官员都对全局有了通盘的认识,当辽人来袭后根本就不需要紧张,也没必要手忙脚乱的,按照既定方案去做就够了。

“龙图说得是。”

李宪只觉得韩冈的杀性越来越重了,性情却是越来越稳。

不过这么说也不能为错,无论辽人有什么招数,只要不能攻下城寨,那就什么的盘算都没有作用。以力破之,原本就是一切计策的克星。眼下是官军处在守御的位置上,即使以辽人之善战,也打不破草草建立起来的防线。

没有更好的办法,也没有更新的消息,李宪暂时放下心头事,关注起韩冈的举动:“龙图看得是什么书?这两天手不释卷,好像看得都是这一本。”

“是家岳的书稿。”韩冈说着,扬了扬手上的书册,明显的手抄本,连封面上的书名都是随手题的字。

李宪没看清封面,扬眉问道:“是介甫相公的新诗集?!”

“不是。”韩冈摇摇头:“是有关训诂方面的新书。几年前就听说写得差不多了,不过因为国事繁芜,无暇修订。直到回金陵后,才有了空暇。到如今终于是定稿了,托人寄了过来。”

“训诂乃经学之本。介甫相公的三经新义一洗汉时传疏旧弊,如今新书一出,《尔雅》《方言》亦得让其一头。”

“是啊,要是刊之于世,新学的声势当是又上一层楼了。不过……”韩冈笑笑,却不说下去了。

“……”李宪张了张嘴,终于想起来韩冈不仅仅是通晓兵事的能臣,还是当世一大学派的核心,一心想要发扬气学的儒者。纵然韩冈与前宰相有翁婿之亲,但两人分属不同学派,在学术上相互之间争得你死我活。韩冈之前的官职,一直因为学派之争的缘故,而被王安石压制的传言,可是一直在京城中暗中流传。

想到韩冈的忌讳,李宪哪里敢接这个话题。

韩冈看到李宪的神色变化,了然一笑。

这部手抄本是王旁抄写,不过其中几篇还是王安石的亲笔。写信来说是请韩冈斧正。可以看得出在学术上,王安石没有将韩冈当成是自家的女婿。但从序中文字上,则显示王安石对这本书十分有信心——‘庸讵非天之将兴斯文也,而以余赞其始?’岂非是上天将兴斯文,以我来引发。不得了的自信。

王安石写这本书的目的,当是给新学添砖加瓦。使得新学地位更加稳固。

自张载病故,程颐入关中讲学后,儒门正统之争,如今已经进入了白热化。不过新学依靠权威,一直高高在上,只要想考进士,就必须去学习新学。国子监中,以新学为教材,培养出来的士子一批批的走进朝堂,那个效率,绝不是韩冈在河东这边拼死拼活举荐的几个幕僚能比得上的。

不过新学的位置是靠着权力维持,韩冈并不担心,要颠覆掉眼下的地位乃是迟早之事。真正的对手是二程。传承千古的理学,就是自二程身上发轫。

等胜州事了,边境上当会有一段时间的和平。那时候,该在正经事上多下些功夫了。

眼下还得给王安石回信,这本新书上有很多地方是韩冈难以认同的。但训诂学是一门大学问,韩冈的水平并不高,甚至可以说是低微,要怎么辩得过精擅引经据典的王安石,可是得颇费思量。

至于眼下柳发川和暖泉峰的战局……韩冈仰头看了看房梁和屋椽,塌不下来的,完全不用担心。

……………………

三面火起,熊熊烈焰窜起数丈之高,融金烁石之力。

浓烟包围了苦力们所居住的营地,熏得人睁不开眼睛,但一阵风后,融化开来的雪水,又为烈火催化,立时又是水雾弥漫。

在营地中地势最低的位置上,苦力营周围的木料、草料和石炭火焰正旺,焰气蒸腾,滚热潮湿的空气充斥在周围,让冬夜宛如盛夏。

火焰的灼烤中,萧海里紧紧地咬着牙关,滚热的浓烟和水雾,让他胸口火辣辣的阵阵作痛。

如果宋人的哨探能晚一点发现援军就好了,飞在天上的眼睛,在二十里开外便发现了大辽铁骑的来袭。这让他不敢立刻发动,直到所有人被赶回了苦力们所居的营地,隆隆如雷的蹄声才传遍了山谷间的营地。

周围是滔滔火海,剩下的一面则是一道栅栏。以宋人对党项苦力的苛待,在苦力营周围设上一圈栅栏不算过当。但这道栅栏过于牢固,而他们给关入苦力营中的时候,宋人搜走任何利器。

幸好萧海里还是设法藏起了七八把做工时所使用的斧头,而且还有跟他同样想法的党项人,同样私藏了好几把工具。只是当他们组织人手去劈砍栅栏,阻截他们的宋军便立刻就是一蓬飞矢袭来。

付出了几十条人命,萧海里终于确定如果不能集合所有人的力量,根本不可能冲出去。

他已经将自家人召集到了身边,招呼着下属时,放弃了半生不熟的党项语,而改回了契丹话。都这个时候了,也不可能再遮着掩着。

死了四人,病了有二十多人,三百人不到的数目。但在三千党项苦力中,却占了近十分之一,已经是让人不敢轻辱的力量。加上有着共同的敌人,当他表露身份,将营中各部黑山党项酋首都招集过来时,也没人敢对他的身份进行攻击。

“只要你们听我的吩咐,回去就奏请尚父和枢密,将你们安排到西阻卜的草场上去。到底是在宋人这里做工到死,还是愿意回去占西阻卜的地?!”

迫在眉睫的危机下,抓住了黑山党项迫切想脱离苦海的心思,萧海里很轻易的就用完全没有根据的承诺,让所有人听命于他。到了这个时候,就是谎言编织成的稻草,黑山党项们也甘愿去抓着不放。

有了共同的指挥,脱逃的行动立刻便井井有条起来。拿着木板充作盾牌,互相支援着去拆除外围的栅栏。神臂弓射出的箭矢,绝大多数命中了木板,坚固的栅栏被砍得木屑横飞,一切都十分顺利。

只是萧海里的心中却是警讯不断,就在阻挡他们的宋人背后,明明没有人作乱,偏偏却是一片乱声大起。营寨中吹号敲鼓不到百人,但鼓噪出来的声势,却仿佛整个营寨都陷入了混乱……

熟悉的军号声从营寨外响了起来,那是大辽铁骑的进军号角。一声接一声在寨外的山野中响起,只听号角声,便知千军万马正杀奔而来。

如同电光在脑中闪过,萧海里恍然大悟,惊恐万分的大叫着:“这是陷阱!”

