\section{第27章 更化同风期全盛(中)}

“学士……学士,学士!”

一声比一声更大的叫声,把黄裳从睡梦中给叫醒。

黄裳睁开沉重的眼皮,随行的伴当就在面前,坐了起来,“到哪里了?”

“到京师了。”

“这么快?”黄裳头脑昏昏沉沉,只觉得还没有睡饱。

“学士,就要到酉时了。李学政都已经出了舱。”

“酉时。”黄裳皱着眉,起身来,身子还是觉得乏得很。

夏日的午后,又在地方狭窄的船上,饱餐之后,除了睡觉,也没别的事好做了。

在云南辛苦了一年,返回京师的路上,黄裳发现自己越来越懒散,除了看书练字之外,剩下的时间,真的就只剩睡觉可做了。

换了一身衣服,黄裳走上前甲板,一条虹桥正从头顶上掠过。

四周的繁华也让他真切的感觉到——京师到了。

正站在船头上的一人,闻声回头,“学士,起来了?”

“年纪大了,吃不得苦,一睡就睡得多了。”黄裳自嘲的笑了一笑,走到云南路的副学政身旁,“履中,怎么出来了?”

“复在船中待得闷气,所以出来吹吹风。”李复说道。

李复是气学弟子,与黄裳一样,做过韩冈的幕僚。不过李复做幕僚的时候,还是韩冈奉旨攻打交趾的时候,等韩冈转任京西,黄裳才投到他的门下。不过现如今,两人都在云南路上,一个是理州知州兼云南路经略安抚使,一个则是提点学政副使,工作上往来甚多,因为韩冈的关系,两人天然的就感到亲近。

黄裳道:“是船太慢了,换成是马车,坐在车厢里都有风。”

“是太慢了。”李复叹道,“从方城山出来,一路坐船走了整六天。什么时候方城山的铁路能直通京师就好了。”

“恐怕有得等了。”黄裳道。

铁路运输替代不了水运。

一列货运马车,货运量只能抵得上一艘、至多两艘的纲船运力。而一列八匹甚至十六匹货运马车,行驶同样的距离,则要比纲船成本高得多。尽管速度快过纲船,但很多时候,速度并不是排第一位的。即是综合了运力、时间等因素,自方城山至开封的货运客运,还是以水运的成效比最佳。

听了黄裳的解说,李复似明非明,“既如此,那朝廷为何要修京泗铁路?”

“惠民河与汴水岂可相提并论。”黄裳摇头。

汴河水运是有其特殊性。一个是因为汴水水源来自于黄河水,因而逐年淤积,另一个则是汴水冬天时不得不停止使用,每年有超过三分之一的时间,是,再一个,为了维持汴水航运,每年支出的成本太高,筑堤、清淤、造船,朝廷投入的资金,让运输进京的纲粮的成本翻了一番。加上南来北往的民间货运都通过汴河,过多的船只,使得汴河河运时常堵塞。朝廷也是迫不得已,才会费大力气去修京泗铁路。

而自荆湖北上,有现成的汉水可以使用,完全没有汴水的弊端,过了方城山后,惠民河的水源来自于其穿过的汝水、颖水,以及两者的支流洧水、溱水等河流,水质优良,并没有黄河带来的泥沙淤积问题,也就用不着年复一年的去治理。

“原来如此。”李复点头受教,“多谢学士指点。”

……………………

“谁说不修的?肯定要修。”

两人抵京之后,韩冈设宴款待,宴席上两人说起白天在船上的议论,韩冈的回复却出人意料。

黄裳顿觉脸上发烧,他才在李复面前大放厥词,转脸就让韩冈给戳穿了。他忙问,“相公。怎么朝廷要修方城山到开封的铁路?!”

宗泽今日陪客,听到之后便在旁解释道:“朝廷有计划,打算自鄂州武汉修铁路直抵河阴。”

“鄂州直抵河阴?”京畿和湖北的地图出现在黄裳脑海中,却想象不出这条铁路会怎么修,他皱起眉,“这条路怎么修?”

“出鄂州一直向北,经过安州、信阳军,抵达孟州河阴。注1”宗泽说道。

“信阳军?”黄裳听得发愣,信阳可是山区,“那武阳关武胜关怎么过?义阳三关没哪条路好走吧?”

李复也对韩冈道:“相公,下官虽未走过武阳关,但也知道此处自三代便为险关,吴国破楚,也是先攻此处。在这里修铁路可比方城山难得多。”

韩冈拿着杯子,轻呷一口,笑道:“武阳关道,古名大隧隘道,要怎么修,应该不难猜到吧?”

“穴地为隧道?”黄裳和李复同时问道。

“这是最简单的办法了。”韩冈点头道,“若是武阳关隧道能够开凿成功。就有一条连接荆湖和京畿的直道,总长不过千二百里,京营禁军四天便能抵达鄂州。”

“可为什么是河阴?直抵京师不好吗?”

“因为经过勘探,若要修大桥过黄河,河阴宜村渡比白马渡更合适。汴口开于河阴,也正是因为其地滩窄岸坚,易于引水,设立闸口。”

黄裳之前就已经觉得自己吃惊的次数太多了,但他今天还是忍不住大吃一惊,“在黄河上修大桥?”

由不得他不吃惊,黄裳明白,韩冈要在黄河上修的大桥,绝不是什么浮桥——浮桥上铺不了轨道,只会是坚实的拱桥。但这种技术,黄裳听都没听说过,之前京洛铁路,为了修那几座横跨黄河支流的桥梁,可是费尽了周折,拖了近一年的工期,才给修好的。现在,韩冈却说要在黄河上架大桥了。

这怎么可能?!

“不要吃惊,这是十年之后的事了。”韩冈大笑,“十年之内,京师去湖北,还是只用方城山轨道。”

‘十年之后,可还能修得了吗?’黄裳觉得韩冈想得太好了。

那时候,天子当已亲政,韩冈这位权相,还能有多大几率安然留在东府之中?如果他离开相位,他所定下来的一系列计划,怕是只会被人束之高阁。

黄裳并不为韩冈的安危担心太多。

皇帝现在对韩冈有成见,这不代表日后还会如此,小孩子的爱恨太单纯,等他再大一点,就知道权衡利弊了。

难道熙宗皇帝不反感把持朝政近十年的韩琦?不认为韩琦在英宗刚刚晏驾、又有诊断说天子可能会复苏的时候,说若天子复苏亦只能为太上皇,已经逾越了臣子的本分?但最后给韩琦的恩荣,依然冠绝朝野,不论是两代三人相继镇守乡郡,还是两朝顾命定策元勋的碑文,都没人能比得上。

只是到了那时候,韩冈还能紧握相位的可能性,就实在是太小了。

现在春风得意,日后可能就是门庭冷落。

他看了看宗泽、李复,还是将到嘴边的话给压了下来。若要劝诫,还是等没人的时候私下里说最好,有人在场,手握大权的宰相不一定能听进去太多。纵使韩冈看起来有着十足的宰相肚量,可黄裳不愿意拿自己和韩冈的关系来冒险。

“好了,不说这些事了。过两天,你们与汝霖一起去参观一下蒙学,看看京师的蒙学是怎么治学的,回头想想对云南能不能有所帮助。”

要在云南路下面推广蒙学,黄裳和李复跟着宗泽视察京师蒙学,是顺理成章的一件事。两人皆同声应诺。

“相公,京师现在有多少所蒙学?”

坐下来后,李复问着自己感兴趣的话题。

“大大小小总计两百所。”

韩冈去年就把新的丝织技术拿出来作为交换条件,有心于此的富户豪门,除了各自向雍秦商会缴纳了一笔技术转让金之外,也都各自满足了韩冈的要求。

各地的丝织厂先后破土动工,江南、淮左,甚至河北,都有了人开办新式的丝织厂,为了赚钱,人人迫不及待。与此同时,在官府登记的蒙学,光是京师之中,就超过了两百所——其实这也非是难事,只要将族学改一下,就是一座现成的蒙学。所要做的,只是改动一下学习的内容。

“可比云南多得多了。”李复感叹道。

虽是学政,云南的学政可是不能与中原各路相提并论,十几年中,能有一两个人考中进士,就已经是文曲星高照了。

韩冈摇头道:“这是城中的学校,城外的还有更多。但平均一座蒙学只有二三十人,多不过百人,真要细算起来,读书的小学生,人数还是太少了。”

“辽国也开始办蒙学,”宗泽说道,“连课本都一模一样。京师倒好说,很多地方,现在还不如辽人。”

京师、陕西,这两个地方,是韩冈推广以格物为根本的新蒙学的重点。陕西不算富裕,但是没有了过去的战争负担,又有横渠书院中的学子尽力宣传,城镇中的学生入学率至少能达到四成。最重要的,是蒙学的学制和内部结构,也脱离了私塾的形式,而更加符合未来发展的需要。

但京师的蒙学,情况就要差很多,这完全与京师的经济水平不相称。幸而韩冈有耐心,他也不指望了两三年内,就能普及教育,这本就是百年大计。更何况,现在又有外界的驱动,情况正在逐渐好转。

不得不承认,内忧外患的辽国,的确有着承平之地所不能相提并论的动力。

注1:这其实就是日后的京汉铁路的路线。
