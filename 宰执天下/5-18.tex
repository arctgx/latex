\section{第三章 时移机转关百虑(四)}

韩家的年节红包给得丰厚。

四两一个的万福如意银锞子,外院的管家、内院的管事娘子一级的有四个,贴身服侍的大丫鬟和元随与他们平齐;下面的仆役、婢女也依着等级,三个、两个、一个的不等。

依照最近的银价,一两足色的银子能在金银交引铺中兑换一千七八百文小平钱,大略相当于两贯半,一个四两的小银锞子就是十贯,四个就是四十贯。

在京城中,韩家给的算是很多了,自是一片诚惶诚恐的谢声。

韩家治家,近于军法,一向是重罚重赏。

犯了错,能原谅的则训斥一番,扣个工钱了事;不能原谅的,虽说韩冈不喜杖责这样的肉刑,基本上不对下人使用,但逐出家门的惩罚,对于全家老小都在熙河路,处在韩冈阴影之下的韩家仆役和婢女来说,却比杖责个几百下都可怕。

在韩家做事的压力很大,缓解压力最好的办法就是金钱,让人认为自己受到的压力所换来的回报有足够的价值。若是即有压力,又没有回报,鬼才会忠心。

不过家里的小孩子,就没那么好的待遇了。就是一片半两重的银钱,正面是福寿康宁、背面是永保千秋,是宫里的赏赐,由名匠打造,精致倒是精致,就是不值多少。

韩冈五子一女,都是如此,装在红纸袋里。三个大一点的孩子,另外都有一套文房四宝,这个花的钱就多了。

拿了压岁钱之后,小孩子们就撑不住了,一个个困得直打哈欠。三个小的早就跟着周南一起进去睡了。三个大的却强忍着困意,支愣着眼皮不肯去睡觉。

韩冈看着孩子都困了,便说道:“金娘,带着弟弟去睡觉。”

金娘坚决的摇头,揉了揉眼睛:“孩儿要守岁。”

“孩儿也是!”韩钟、韩钲两个小子也一起叫了起来。

“乖。”韩冈拍拍女儿小脑袋,“睡一觉起来,爹爹带你们去放鞭炮。”

“爹爹骗人!”金娘仰起头,黑白分明的一对眼睛瞪得大大的:“正旦是要上朝的。”

“没错!”韩钟、韩钲一前一后的搭腔附和,“都是要上朝的。”

金娘很认真的看着韩冈,一字一顿:“娘说了,说谎就不许吃饭。”

韩云娘从身后将韩家的大女儿抱起来,笑道:“你们的爹爹没有说谎。皇帝看他辛苦,今年就不用上朝了,可以带你们去放鞭炮。”

“爹爹,可是真的?!”

金娘在韩云娘手中扭着身子,要转过去问韩冈。韩云娘力气小,被她一动差一点就脱手,旁边的乳母连忙接过去。

“爹爹不说谎,不然你娘会不给饭吃。”

韩冈忍着笑,却被王旖狠狠的瞪了一眼。

三个孩子都被抱着进去了,韩云娘追在后面:“别忘了要刷牙。”

孩子在的时候,韩冈要保持形象,现在轻松了点,伸着手脚靠在椅背上,惬意的眯着眼:“难得有一年能免了正旦大朝会。在京里就是这一点不好,早上的时间都不是自己的,连个囫囵觉都没法儿睡。”

正月初一不用上朝,是因为太皇太后曹氏近日病重,天子赵顼下诏免去了今年的正旦大朝会。说实话,参加大朝会之后的那点赏赐,绝大部分的臣子们都是看不上眼的。元旦时可以清闲一点,京中的朝臣们也都乐得轻松。

“太皇太后不会有事吧?”严素心一边帮韩冈剥着松子,一边问道。

“不知道。”王旖摇着头,“前日随班入宫探问,也只是在殿门口问安而已。病情究竟怎么样了,都说不清楚。”

“今天姐姐也要入宫吧?”

“嗯。”王旖点点头,她的今天午后还是要入宫向两宫和皇后问安。是为外命妇,北海郡君,这是她的义务,“不过今天入宫,估计也是一样。进宫那么多次,都没正经跟两宫说过几句话。”

“上朝那么多次,有几人能正经跟天子说过几句话?”韩冈笑说着,就又被王旖横了一眼。

官员有官员的圈子,夫人们也有夫人们的圈子。因为王安石的缘故,王旖是打不进绕着太皇太后和皇太后的那两个圈子。由于韩冈的因素,又跟新党那一片没交情,也就最近王旖的行情才好了一些,在后妃那里能说得上话。

韩云娘安顿了孩子之后,从内间出来,很是好奇的问道:“三哥哥,太皇太后是什么样的人?”

“你姐姐是命妇,都没怎么打过交道。我还是外臣,更没机会了。”

韩冈对太皇太后和皇太后都没怎么接触过。两宫太后基本上连娘家的兄弟都不见。当初天子曾想让曹佾——也就是曹国舅——拜见曹氏,让姐弟二人好好谈谈心,但曹国舅转眼就被送了出来。高太后的情况也差不多,都很懂得约束娘家人,不让他们多进宫中。

既然两宫行事都守着纲常礼法,天子就更不能亏待曹、高两家,从老至少,都是高官厚禄的养着。高遵裕想领军立功,天子也没有二话。

“原来是这样啊。”韩云娘点着头。

王旖也说道:“两宫贤德,能约束国戚,乃是国家之福。”

韩冈笑了一笑。

其实话说回来,这也是官僚们对宫中一直保持压力的结果。除了垂帘听政的时候,要是两宫敢妄见朝臣,言官立马就能追上去一通乱咬。他们不敢咬太皇太后和太后,难道拜见两宫的官员还不敢咬吗?

士大夫们最忌讳的就是被侵夺权力,不论是武夫还是深宫中妇人、阉宦,谁敢跟他们抢蛋糕,就会落到群起而攻的下场。仁宗皇帝当年的张贵妃,也就是后来的温成皇后,她的伯父张尧佐每一次想升官,都会被文臣们联手敲打,从包拯开始,每一位谏官都将张尧佐当成了练手的靶子,一封封弹章,能堆满一间屋子。

就是一干宦官能出外领兵,可让他们去议论一下朝堂政事,绝不会有好结果。

这一切,却正是合宋室天子的心意。无论如何,文臣之间绝不可能和睦共处,只要外面没了压力,自己就会乱起来,拉一派打一派,或是说异论相搅很是容易。而武人、妇人和阉宦,他们秉政的结果,千百年来的青史历历在目,对皇帝的威胁才是最大的。

宋室诸帝,在掌控权力方面,一向做得很好。

不过也就如此而已,正经事从不见有这等本事。

韩冈自嘲一笑,大过年的,想这些事等于是给自己找不痛快。拿着素心剥好的松子仁就着热酒,问王旖:“过年要送的拜帖都写好了没有?”

年节送拜帖,是京城中的习俗,就跟后世的贺年卡一样,人情到了就行了,没人能一家家的全都走遍。

“就等官人签名了。”王旖突然捂住嘴,闷笑道,“若是没有官人亲笔签名,人家可是会打上门来的。”

……………………

年节送拜帖,就是吕惠卿也不能免俗。

别说是执政,就是宰相也得送帖子给亲朋好友。

吕家兄弟如今在京城的就吕惠卿和吕升卿,吃过年夜饭,守岁守过半夜,年轻人去找乐子了,女眷带着小孩子毁了房中睡觉。

就吕惠卿和吕升卿都是劳碌命,又到书房中。

吕升卿看着兄长一封封检查拜帖,随口道:“今年是没有正旦大朝会了,不过西夏来贺正旦的使臣,在都亭西驿只歇了一日,就被赶了回去……天子杀气腾腾,梁氏兄妹再糊涂也不会感觉不到……真的是要打仗了。”

吕惠卿头也不抬:“薛向将会负责往关中转运粮秣,过了初五就要去洛阳坐镇。大约在四五月的时候,就该开战了。”

“领军的人选还没定下来吧?”吕升卿问道。

“六路主帅的人选只会是由天子自决。两府中,哪个有资格说?”吕惠卿将手上的拜帖放下,“王珪是天子说什么,他就做什么;元绛也差不了多少。吕公著因为陈世儒一案,被天子抓到把柄在手,什么话都不敢乱说了。薛向只能负责往关中运粮,郭逵又要去河北。为兄现在心力都在手实法上。这就是两府的现状。”

吕惠卿说着直摇头。两府中能为各路主帅的人选,跟天子争一下的臣子,眼下根本就不存在。

天子乾纲独断,说起来,吕惠卿对这一战的前途,有几分担忧起来了。韩冈的判断,现在看来真的有那么几分道理。而郭逵的去向,更是让吕惠卿加深了一层担忧。

“‘天下有道,则礼乐征伐自天子出。’眼下不正是天下有道的时候吗?大哥你担心个什么呢?”吕升卿开了个玩笑,随即又正经起来,“继续推行手实法才是当务之急。朝廷既然要用兵,钱粮肯定是缺的,”

现在朝廷的确需要钱,但另一方面,天子也需要一个稳定的朝堂和国内环境,以便将全部精神放在即将开始的西北战事上。两者之间如果发生冲突的时候,就需要权衡利弊,进而有所取舍。

吕惠卿所要做的,就是不要让天子将自己给舍掉,又要拿出让天子满意的回报。这等于是在一根绳子上走路,不能左、不能右,稍有差池,就是万劫不复。

不过这方面事就没必要对自己的兄弟说了,省得他多担心,在府界提点的任上安心做事才是真的。

吕惠卿点点头,轻描淡写的:“说得也是。”

吕升卿果然笑了笑就放心了,不再说这件事。像是想起了什么:“对了,李定是不是有什么盘算?前两天见他时,总觉得有些鬼鬼祟祟的,打了个招呼就走了。”说着,他皱起眉头,“刚刚任了御史中丞,准备拿谁开刀?”

新任御史中丞上台,刀子上总得见见血,肯定要办个大案出来,否则这个御史中丞做得也难看。所以李定最近的变化,落在吕升卿眼里,就不能不让他担心了。

“这件事啊……大概是什么情况我知道,你不用乱担。”吕惠卿带着讽刺的笑了一声,低声道:“不过有仇报仇,有怨报怨罢了。”

见吕升卿仍是疑惑不解,吕惠卿笑意更是意味深长,“李定是聪明人,他不会蠢到破坏朝堂上的大好局面的……”

