\section{第七章 烟霞随步正登览(五)}

王安石正准备安排将选票发下去。

他已经就任平章军国重事,故而主持殿上推举 一事,他便责无旁贷。

由于推举在朝堂上别无先例,能够借鉴的就只 有两大联赛的会首选举。

为了避免在选举时,先表态的选举人影响他人 的判断,所以都是填写选票,而且不能是匿名,必须写上自己的姓名。选票就是类似于章疏的折子。如果换一个文字方式,再加上理由,几乎就跟举荐的奏章差不多了。

不过在王安石的心中,再如何与举荐相像,也 改变不了这是一场从无先例,模仿民间赌赛的会社来决定宰辅人选的闹剧。不论新党中最为核心的几位都因这一事而兴奋不已,四处奔走,可王安石看来,依然还是闹剧。

只是他无法反对。

变法是王安石一生的主张,他总不可能说一句 祖宗之法不可变,来反对韩冈的提议。即便有办法将韩冈的提议驳回了,最后太后肯定还要征求韩冈的意见,总不能自己跳出来推荐宰辅?两害相权取其轻,只能依从韩冈。

而且就是王安石,也不想开罪满朝重臣。那些 都是朝廷中的中坚,集合起来的力量都要。但要指望他现在的表情能够缓和一点,也未免太过强人所难。

当王安石站在文德殿上,只打算早一点结束推 举,各归各位。可惜他的愿望,刚一开始,就被人打破了。尤其是站出来的还是他最为厌恶的几人中的一个,这让王安石心情更加恶劣,甚至叫出了声来。

“沈括?!”

王安石的声音饱含着怒气和憎恶,沈括闻声, 身子就是一颤,脸色也霎时变白,不过他还是坚持着站到了大殿中央。

“臣……”

“沈卿若有事需奏禀,且待推举后再说不迟!”

听太后的口气,明显的也不喜欢沈括。沈括刚 一开口说话,就立刻打断了,将他的嘴给堵上。

沈括脸色发白,差点都没能站稳脚。

他背叛韩冈,在朝堂上早已不是秘密。

韩冈为了推荐他,与吕嘉问和李定等新党核心 交恶。甚至有传言说,王安石和章敦这段时间以来之所以渐与韩冈疏远,以至于在这一次的推举中,都没有保持中立,也是因为韩冈看重沈括,打压吕嘉问和李定的缘故。

但一听说能够有机会进入两府,立刻就将韩冈 丢到脑后,转头去奉承新党了。

很多人乐得看韩冈的笑话,可最为信重韩冈的 太后似乎就不那么高兴了。

“若是他事,臣当然可以推举之后再说,但唯有 这一件,却不能!”沈括苍白的脸上,多了一分坚定,“臣沈括敢问陛下,叛臣安可为宰辅?!”

帘后没了声音,王安石脸上的怒容也不见了, 代而起之的是深深的疑惑。而吕嘉问、李定等人,也都惊得瞪大了眼睛。

不过他们接下来都不是看沈括,而是去看韩 冈。

可惜在韩冈的脸上,人们看不到任何表情变 化。

这是沈括的独断?还是韩冈的谋划?

事前没人会认为韩冈会用这等绝户计。

虽然泼对手一声脏水,是解决政敌最简单的办 法。这也是官场上最最常见的手段。但因为韩冈要推重气学,对自己的名声看得比官位更重。就算是吕嘉问、李定等人,也从来没有想过韩冈会指使沈括赶在推举之前,拿着叛逆的嫌疑,将最有威胁的对手给拉下来。

尽管互为政敌,吕嘉问、李定他们还是相信韩 冈的人品,不至于如此下作。

“臣沈括奉旨审问赵颢、蔡确谋逆一案。”

沈括再次开口,双眼明亮了起来,苍白的脸上 也多了一丝血色。

“近日搜检从犯刑恕、蔡京等人家中所藏信件, 其中多有辞理诡谲,惹人疑窦之处。涉及官员,京内京外数以百十计。”

沈括的声调没有什么变化,但殿中似乎一下冷 了许多。

百十计!

这个含糊的数字,让吕嘉问的心沉了下去,如 果不是韩冈主使,沈括哪里有这么大的胆子?这分明是要兴大狱的打算。

“臣得陛下诏书,凡事涉叛逆之人,皆可下开封 府诘问。”

沈括的声音大了起来。

他的话,让人无言以对。这是诏书上的必然要 写的,但诏书背后没有写明的用心,却是当时主持平叛的一众宰辅都打算息事宁人。

当初宰辅们能够同意将这桩案子下放到开封 府,一个是因为沈括本身性格有问题,软懦畏怯,另一个,就是沈括背后的韩冈,从一开始就表明态度,坚决反对深究大逆案,连同曾布、薛向这样的主犯都要饶了性命。

这两个原因,使得朝堂上人人安心,不会因为 递上宰辅家门中的一张名帖,或是与叛逆党羽的一份书信,而被抓进狱中去拷问。

可是没人能想到沈括会仿佛变成了另外一个 人。

究竟是谁站在他的背后?

“卿家查到了什么?”

虽然对沈括没有什么好感,但沈括正在审查的 案子,却一直挂在向太后的心上。

韩冈和宰辅们一直都在劝说向太后不要再穷 究,她也的确听进去了,可是沈括现在在殿上一提,被压下去的想法,便又给翻了上来。

沈括从袖中取出了一份奏章,双手举到了头 上,“今日乃是推举宰辅之日,可以参选者为两制以上官,拥有推举资格者乃是侍制以上官。但其中吕嘉问、曾孝宽、蒲宗孟、黄履等人,与刑恕、蔡京等蔡确党羽相往来,其嫌疑不可不查。若推举之后查出这一干人事涉大逆,却有被选入了两府,必会贻笑朝野、外邦。”

沈括没有说该怎么办,但每一个人都知道沈括 的想法。

有嫌疑的人,既没有选举权,也没有被选举 权,去掉了这一批人,不论是沈括、还是韩冈,都不用在担心名次的问题。

…………………………

张璪心中正在啧啧称叹。

这一场推举还真是惊喜多多。

该狠的时候就狠下来,这沈括糊涂多年,这一 回倒算是变聪明了。最终的决定权在太后手中,让太后心里舒服了,说不定转头就能进两府了。

另外,如果沈括是听从了韩冈的吩咐来打击对 手,可见韩冈已经没有了与吕嘉问、李定等人正面交手的信心,而不得不使用一些上不得台面的把戏。

换而言之,韩冈在太后面前提议时,他本身也 不过是想找一个解决问题的办法,推掉太后打算给他的两府人事的推荐权。

若沈括不是听从韩冈的指派,而是自把自为, 那情况就更有趣了。

韩冈可就是要面对朝臣的攻劾和鄙视,他的名 声也将在朝堂上会一落千丈。

如此一来,气学还能够走多久?

…………………………

没有李定!

章敦偏过头去看御史中丞,不是怀疑李定暗通 沈括,甚至韩冈,而是感慨御史中丞这个位置。

看起来沈括也是在担心御史中丞手中握着他的 把柄。若给李定在殿上翻出来,沈括的脸就难看了。以李定手中掌握的资源,只要他想准备上一两手,来保证他能够顺利通过选举,也就是一句话的问题。

而沈括暂时不提李定,就等于是拿着对方的把 柄,与李定互相威胁,最后一同保持沉默。

如果从沈括的才智上来看,他这么做不足为 奇。可是沈括在政治上从来都没有做出过正确的选择。每次想要改换门庭,都会遇上最坏的结果,从来没有说他能够做对一处选择。

是韩冈吗?

章敦原本坚定的信心,变得犹疑不定起来。

…………………………

李定虽然不敢直接抛出沈括背后的把柄,但他 那边也的确给抓住了一个关键。

“开封府说京内京外的有叛逆嫌疑的官员百十人 之多,不论此事真假,沈括赶在推举之前上奏此事,其目的不问可知。”

只要不是曝光沈括的把柄,李定就没有太多的 顾忌,照样可以指斥沈括有私心。

“臣沈括颟顸愚鲁,又于国无功,不敢与诸贤相 提并论。宰辅之任,臣沈括力不能及。”

沈括辞去了参与选举的机会,这让他之前对吕 嘉问等人质问,变得正大光明起来。

既然他都推掉了参与选举的机会,那他所说的 话,也就多了几分可信。

只是向太后心中隐隐的觉得有些不多,沈括的 意见与宰辅们之前的请求可是完全相背离的。尤其是韩冈,那可是一直都在请求放过被擒获的叛臣,将他们流去四荒。

“韩卿,你看如何?”

“此番叛乱乃是仓促而行,真正会参与到其中的 又能有几人?多还是寻常往来,留下的书信,也多是寻常探问。可一旦入了狱中审问,什么样的口供都能拿到。”

韩冈的意见果然不与沈括是一路,向皇后很快 问道,“韩卿是什么想法?!”

“臣请陛下降诏,蔡京、刑恕等从逆的叛党,其 所有私人信件当尽快毁弃,以安朝野人心,也能让世人共仰太后和天子的仁德。”

韩冈环顾朝堂,许多道视线内所蕴含的感情, 在这句话后都变得充满了感激起来。

“烧掉吧!”他语气坚定。

