\section{第28章 大梁软红骤雨狂(四)}

【第三更,昨天的缺章补上了。求红票,收藏。】

当初渭源之战时,从古渭到渭源,长达一百多里的粮秣军资的转运,就已经耗尽了秦州泰半民力。而且那只是要维持连民伕加士兵,总计五千人的一个月的需用。而在绥德、罗兀能做到部分自给自足前,至少要两到三年的时间,鄜延路都要征发民伕,去运送粮秣。

如此一来,对鄜延百姓来说,是个灾难,而对当地的官员来说,同样是个灾难。

地方的官员是什么样的德性,韩冈再清楚不过。事情不做,便宜尽占,除了一些有望上进的,其他大多数的官员就是一副混吃等死的模样。一旦要组织民伕转运粮秣,或是干脆把民伕赶上前线去筑城,少不得就要劳动他们的大驾,想让他们不抱怨是不可能的。

再说了,还有个司马光在长安守着,几乎使用放大镜在盯着陕西的各个角落。只要地方上有一点风吹草动,他肯定要第一个跳出来的说话。

人都是这样,总是看到自己想看到的东西,司马光是这样,韩冈、韩绛他们的也一样——出兵罗兀,韩冈看到危机,韩绛、种谔则看到胜利——如果有什么与他们的期待相反,就会想办法将之抹去。不过区别在于,蠢货是在自己的思考中抹去,聪明的人则是在言辞中抹去。

司马光、韩绛他们究竟是聪明还是蠢货,韩冈不知道,但他能清楚的认识到这一点,实际上他的观点不会太过偏驳。所以他的批评,并不是放在战斗的胜负上,而是主要专注于粮秣转运的问题上。不管在何时何地打仗,只要不能像蕃人那样因粮于敌,后勤运输总是问题最多、事情最难的一个环节。批评后勤问题,那是一批一个准,绝不会说错。

“兵无粮不行,在出兵之前,还是要先看一下究竟能不能把足够的粮秣运送到罗兀,而且是要在不引起鄜延民乱的前提上!”韩冈语气坚定的总结着,每一句话背后,都是写满了自信两个字。

说话要让人信任,首先要表现出自信来。自己都不能相信自己,何谈让人信任。

韩冈自陕西来,又是参与执掌军务。在一般人的心目中,天然的就对陕西地理兵事了若指掌。而韩冈与王安石一问一答间,表现出来的自信,完全印证了他作为一个专家的形象——通常的情况下,说话的语气、语调,也就是技巧方面的有效表现,比起正确真实的内容,对于博得他人信任来,反而更为重要。

韩冈话说得虽然浅显,但他朗朗言辞间毫不动摇的自信,以及一直略显失礼却坚定不移的目光,还有毫无犹豫磕绊的流畅阐述,却会让人不由自主的相信了他的这番话。

王安石现在有点头疼了,这样的情况下,如何能让韩冈面圣?

当今天子现在虽然对横山那里的胜利消息日夜期盼,每天都对着武英殿中的沙盘一遍又一遍的推演着战局,将阵图、计划一份份的发往延州。但他毕竟耳朵根子有些软,自宫中长大的皇帝,绝不可能想自己眼前的年轻人这样,有着一对决不动摇的眼神。

一旦韩冈站到了天子的面前,指着沙盘上,将他方才所说的那些话复述出来,最后会有什么结果,真的难以估计。

天子对韩冈的重视,王安石心中很清楚。赵顼日日都要走一趟的武英殿中,每一块沙盘背后,都是打着韩冈的标签。而韩冈对于军中医疗的推进,更是得到了所有陕西将帅的看重。

正是由于郭逵、王韶、韩绛、种谔等人对韩冈的重视,使得赵顼更加确认韩冈的才能。既然韩冈在天子心目中留下了熟悉兵事的形象,那他的观点不可能不影响到天子的看法。

王安石事先也绝然没想到,韩冈会如此旗帜鲜明的反对出兵罗兀,就算执掌河湟开边,与横山拓土有瑜亮之争的王韶,也不会这般坦率直言。

这么想着,王安石感觉到韩冈的表现好像有些反常,

“韩冈,你可是不想去延州?!”他突然问道。

被戳破了藏在心底的想法,韩冈在一瞬间有了那么一点动摇。但是他很快收拾起,把心防重新武装,“为君分忧,不分天南地北,何处不可去?但明知不可为而为,让卒伍平白枉死,下官却不敢相从!”

拿着冠冕堂皇的话为自己的私心做外衣,这样的人和事,王安石看得多了。没想韩冈本质上竟是这样的性格,他有些不快说着:“那就是不想去了。”

要我去也可以,只要能满足条件。韩冈道:“朝廷有命,下官自当领命而行,不会拒绝。不过下官有一点要事先报予相公。无论此战是胜是败,无论下官是否有功绩,朝廷事后的封赏,都不要把下官的名字加上去。”

王安石惊讶了起来。韩冈不要可能会有的功劳,看似谦退,实际上却等于是再说,若此事有何意外,不论什么罪名都不要栽在我头上。

‘他当真认为罗兀守不住?!’

韩冈当然能肯定罗兀守不住,所以才敢这么说。

自己的这个条件如果被王安石如实报上去,天子会怎么看?韩冈无法确认。但这点其实并不重要。实际上,正如王安石所说,他只是不想去延州罢了。

因为不想去延州,所以韩冈才会大力反对出兵罗兀。他反对的理由,就是因为罗兀城下必败。韩冈可以确定,至少有九成以上的可能,韩绛在横山方向上这一轮的攻势,将会铩羽而归。

这并不是因为粮秣问题——

夺下罗兀城后,只要守上半年就够了。因为西夏人在横山统治的脆弱性,甚至等不了半年的时间。罗兀城一旦能稳定的在横山深处留上半年,西夏人在横山地区的统治权其实就可以废掉了。没有了西贼的威胁,安全的粮道,运输起来就很方便了。

但韩冈无法说出这一点。他总不能说,在他所记得的历史中,西夏安安稳稳延续到了蒙古入侵。而眼下的情况,如果横山失却,西夏覆亡就在眼前。

既然西夏没有灭亡于北宋,那今次的冒险计划就不可能成功。虽不能说百分之百肯定失败,但只要有七八成是败定了。只是说话的时候,必须为自己留条退路,“今次一战或许能侥幸取胜,但若是朝堂上下习惯如此冒险,日后的失败可能会更加惨重”

‘不意韩玉昆如此倔强。’隔着小门的单薄门扉,王旁听着里面的交谈,他很难相信,韩冈竟然会这么当面顶撞自己的父亲。

“二哥,怎么了,偷听到多少?”清脆的声音在背后悄声响起,但落到王旁耳中,却差点叫了起来。

看到自家妹妹王旖正在身后,侧着脑袋看着自己,“别闹了!”王旁脸皮有些泛红,被自家妹妹看到自己失礼的地方,他也觉得有些不好意思。

王旖向着门扉处探头探脑,就跟十个月前的一幕重现。时隔近一年,她的好奇心不见减退,“又是韩冈?他又来京城了?”

……………………

由于韩冈的不合作,王安石没有达成目的,他最后也并没有留下王韶和韩冈吃饭,可以说是忘记了。而王安石没有说话,王旁便不敢主动留人,不过韩冈倒没忘了他,当王旁来送行的时候,两人约好下次有空,到外面转一转东京城,顺便喝点水酒。

王安石坐在书房中,考虑着方才的一番对话。韩冈已经表明了自己的立场,王安石可以强迫他过去,但这样他就不可能不担心,韩冈会在公事上采取不合作的态度,或是消极怠工。而且韩冈有勇有谋,不是普通的官员。如果仅仅是让他去处理伤病,这样的做法实际上是太浪费了一点。

王安石一时拿不定主意,直到自家的二女儿过来催促吃饭,才让他暂时放下去思考问题。

坐回到饭桌上,王安石还是一如既往的盯着摆在桌上的一盘菜在吃。吴夫人问着丈夫:“大哥快要抵京了吧?要不要派人去迎他?一大家子拖儿携女的,许多地方的都不方便。”

王安石两子两女,长女早已出嫁,长子也已娶妻。而次子王旁已经与庞家结亲,等长子王雱到了京城,就要办婚礼了。

“大哥都做了多少年官了,许多事不必太替他乱操心,他自己心中都有数,哪里会有什么不便?”王安石丝毫不为自己的儿子担心,自幼聪慧的长子王雱是他的骄傲,完全不需要担心。

吴夫人听了,像是放下了心来,“等大哥回来,二哥成婚。剩下的就是二姐儿的婚事了。”

王旖脸红了,娇憨的摇着吴夫人的手:“女儿不嫁,一辈子都要陪着爹娘。”

“胡说!再拖下去就没人要了。”吴夫人说了女儿一句,回头就对丈夫发作道,“还不快点帮二姐找个好人家。不要老想着变法、变法,齐家治国,先把家齐了再说。”

