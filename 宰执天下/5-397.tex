\section{第35章 势颓何来回天力(中)}

折可大跟着韩冈从战地医院出来,抬头看看天色,已经是月上枝头了。

韩冈一行用了近两个时辰,才结束了对医院的视察,比之前预计的多了许多时间。

“想不到都这时候了。”折可大摸了摸咕咕叫的肚子,小声的说着,都有些抱怨。

黄裳同样小声的说着:“那是枢密仁心啊。如果只是看几眼,走一圈。一刻钟都不要的。”

折可大点了点头。医院中的伤病员太过热情,人人都想得到韩冈的看顾和问候。而韩冈也尽量满足他们的请求,所以才会在医院中盘亘良久。

就算是吴起再生,凭他为士兵吮痈的恩德,也绝做不到像韩冈这般得到士卒们的敬爱——‘卒有病疽者,起为吮之’,连老妇都知道那是要他们去死,而韩冈却是尽量为士卒们寻求生机。甚至为防止伤亡过重,而始终避免与敌交战。

主帅如此受爱戴,如何打不了胜仗?

但韩冈此刻的脸色却是沉郁的,就像夜色一般。这让折可大和黄裳都尽量避免大声的说话。

黄裳循着韩冈的视线望过去,在医院一角,几具担架被抬上了两辆大车,随即便向营外南侧的滹沱河码头驶去。

他心中暗暗一叹,也难怪他恩主的心情会这般差。

如今代州城下的大军数量接近三万,但由于还没有开始攻城等军事行动,在外的斥候游骑又十分克制,其中伤病的比例大约在百分之一以下。位于前线的这座战地医院,伤病人数也就两百出头。

这其中,基本上都是很快能康复的轻伤员,真正的重病号都会送往后方医院所在的忻口寨。

并不是说后方的医疗水平有多高——零和一或许有区别,但三和四其实没什么大的差距——而是免得这些重病号、重伤员在前线影响士气。

在这向后方的输送过程中,由于路途颠簸,要先坐船,然后转上有轨马车,很多伤病员都熬不到抵达忻口寨的时候。就算到了,也很难活多久。

每天都有几人这样被送走,其中却总有人最后蒙着头脸抵达目的地。这是十分无情的做法,但在军中却是不能不硬下心来,慈不掌兵,正是说了这一点。

就算是韩冈,有着偌大的名头,但他也不可能将这个时代的医学水准提高多少。能做的也就是卫生管理方面着手,预防疫病在军营中传播。

可除此之外,那些严重的外伤、内伤,还有一些疑难杂症,韩冈都只能眼睁睁的看着病人在床榻上苦熬,看着他们一步步走向死亡,却没有一点办法。

这些重伤员离开时,只要看到韩冈,都会变得很安静。韩冈总也在他们身边逗留得最久。纵然他们都知道韩冈不通医术,但得到药王弟子的看顾,就像是去极灵验的庙里烧了一柱香,总是能多安心几分。

跟随韩冈身侧日久,黄裳越发的清楚韩冈对普通人的关心,这不是一般士大夫居高临下的‘仁爱’,而是真正的重视。让信任自己的人期待落空,心情怎么会好得起来?

韩冈心情沉重,黄裳和折可大只敢小声的交换几句话,直到听到一声长叹,然后就看见韩冈往医院的一角过去。那里有个独立的区域,是俘虏中的伤病所在的病房。韩冈从主病区出来,便往那里过去。

这一回俘虏的人数不少,已经超过千人,不过其中重伤者人数很少,就是轻伤也没几个。一支刚刚经历过惨败的军队,又是俘虏,伤员的数量竟然才不到一百人。

如果打扫战场、清扫残敌的几支队伍下手能够轻一点,其实伤员应该更多的。但对于这些强盗,从太原一路走来的所见所闻,全军上下没有不将其恨之入骨的。如果是能走能跑的健康战俘,那的确不便下手,但那些无法走动的伤员,谁也不会有耐心将他们弄回营中,都是只带着轻便一点的人头回来。

死里逃生,让这些俘虏在见到韩冈的时候,于畏惧中有着浓浓的期待和欣喜。甚至在韩冈探视过后,那些辽军的伤员,都硬撑着下地来向他跪拜行礼。

在这些伤员中,最为虔诚的是西京道上一家小部族中的贵人,也是为族中领军的主将。这人伤得不轻,全身上下被缝了好些针,真的是运气很好才没有被打扫战场的士兵嫌麻烦生生割了首级回来交差,拜下来时身上的绷带都给染红了。


只是韩冈只是让人将他扶起来,却并不加以辞色。

“此辈在我国中烧杀掳掠,实是死不足惜。其中酋首,更是得明正典刑,以慰亡灵。”韩冈从战俘的病房中出来后就说着,“但我中国子民沦于贼手者不可胜记,少了这些强贼,就换不回他们。不得不饶了他们一命,能救的也得尽量救回来”

黄裳有个感觉,韩冈的这番话,与其说是在向幕僚解释,还不如说是讲给他自己听,像是为了说服自己。要不然不至于如此絮絮叨叨。

韩冈其实的确很想把这些血债累累的强盗砍了垒京观,至少也要学对待交趾土著的手段,不过不砍脚趾,而砍手指,以免他们日后再来犯境。

契丹人在忻代两地犯下的罪行,比当初肆虐广西的交趾军更胜一筹。只是这一回被掳走的百姓为数众多,他无法向对交趾那样,直接打过去讲其灭国夷族,就只能留下一点本钱,把人交换回来。

“用钱赎买是的一桩,朝廷不会舍不得这笔钱。不过能省一点是一点。用俘虏交换会更好。”

韩冈觉得耶律乙辛应该不会放过这个收买人心的机会。马上就能入账百万匹两的银绢,正好可以用在刀刃上,从手中有汉儿的部族手中将人买下来,然后换回被俘虏的战士。

“当然,如果这些还不够的话,武州也可以还回去的。”

无视幕僚们惊讶的目光,韩冈继续说道。

“辽贼入寇不过数月,忻、代二州已是满目疮痍。损失现在还无法计算,户口减半已经是最乐观的估计了。”

虽然不知道究竟损失了多少,但户口减半的确是最乐观的预计了,如果悲观一点,能有三分之一回来就很好了。

“至少得三年之内免除一切税赋。十几二十年后,才能勉强回复旧观。能多找点丁口回来,终归是一件好事。过两年重造户籍,代州衙门里面的气氛也会好看一点。”

每逢闰年,地方官府就会重新修订户口籍簿。

这一回代州遭劫,不仅仅要重造户籍,连田籍也要更造,以维持代州日后的秩序。只是这样一来,下一任的代州知州的任务就会很重,可以说是困难无比。

地方官府的之所以能将政令传达下去,是依靠一级一级、从州到县再到乡里的保甲。发号施令的官员好任命,但实际上的执行者,州县中的各色吏员、乡里的保正、甲长、书手,都不是能从京中或是外地可以调来的。

现在这些吏员跟百姓一样,已经不剩多少,缺少了他们的帮助,知州的政令连乡里都传达不到。

除了代州,忻州的情况也类似。而太原府被辽军侵袭过的地方,情况同样跟代州和忻州差不多。都是村庄被毁,百姓遭屠,财物子女被劫掠一空。

整个河东北部,完全给毁了,几十年内都回复不了元气。

这让韩冈的心情如何能好?

唯一能让人心情好的,可能就是代州城中的辽军了。

……………………

荒芜的代州城中,萧十三的心情只会更糟。

萧十三已经收到了从南京道上发来的通报,声称已于近日同南朝达成了和议,和谈的使臣很快就会从开封回来。

可放下耶律乙辛的亲笔书信,再抬头看一眼代州城中,一股荒谬绝伦的感觉就浮上心头。

但谁都知道,如果双方的实力无法维持最基本的平衡,无论什么和议,都不过一张用过的废纸。

大辽已经打不下去了,要不然和谈的使臣也不会答应这般屈辱的条件。西平六州丧师弃土,换回来的仅仅是一次性的百万银绢。

而且那还是在河东惨败前达成的协议,换做现在,宋人肯定不会甘愿继续执行。

国内的反对者正在聚集。尚父殿下不可能像皇帝一样,天然的便能让朝野内外从其所命。要是他有着一个皇帝的身份,事情就不会变成闲现在的模样。

耶律乙辛虽然让他守着代州城,这个命令是在大败之前发出去的,现在已经不可能再挽回了。

唯一值得庆幸的,就是自家的那点兵没有大的损伤。同时尚父派来助阵的那点心头肉也没怎么损失。除了盔甲、战马遗失了一些以外,就没有别的损失了。

与他们对比起来,西京道的各部头下军的伤亡就大得多了,营中怨声载道。在宋军的追击中,伤亡最重的就是他们。而现在叫着要回家的自然也就少不了他们。

人心全都散了,这些人还能派上用场吗?
