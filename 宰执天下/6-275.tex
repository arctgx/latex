\section{第38章 天孙渐隐近黄昏(中)}

韩冈放的狠话,让冯从义听得很舒服,就该这样对付那些心肝肺都黑掉的家伙。

别看他下江南时与之称兄道弟,但掉过脸后,冯从义可恨不得他们全都倾家荡产。

“仅仅是报纸还不够,”冯从义说道,“小说中要写,让那些说书人来帮忙宣传,还有杂剧剧本,让人把那些黑心商人的嘴脸都拿到光天化日之下。”

韩冈笑道:“那样的话,他们可就要成了过街的老鼠了。”

“正是要过街老鼠才好。走偏门的若是能够大发横财,那哪个人还会老老实实的去做正行?啊……”冯从义看了看韩冈,连忙补充,“当然,不能耽搁到推动工业发展的大事。”

韩冈点点头,他推动工业发展的心意不会动摇,“个人的武勇在军阵面前毫无用处,自给自足的小农生产,在机器生产面前,同样无法立足。此乃天下大势,洪水来势,谁能逆流而上?”

“莫说小农,就是过去的机械,在更新式的机器前,也一样无法站住脚。”冯从义垂下眼帘,对韩冈道:“小弟在城东的那间宅子,去年开了一家磨坊,从早吵到晚。小弟那外室闹了几次,小弟磨不过,想出钱让磨坊的东家搬个家,但他就是不肯搬,多一倍钱买他的房子也不干。仗势欺人,就是给哥哥你脸上抹黑。近处另开间磨坊,用低价将他挤走,又感觉太亏了,我是拿他没办法。”

“现在有办法了?”

“当然是抢先拿到蒸汽机,开蒸汽磨坊挤垮他!”

昔年汴河上还有水力磨坊的时候,利用汴河时有时无的水力,都能年赚十万贯。整个东京的酒楼正店,都是汴水磨坊来碾米磨面,而不是店里自己磨。

自从军器监的铁器制造取代了水力磨坊,无论是风力磨坊还是畜力磨坊,都比不上水力磨坊的使用方便。

但如今有了蒸汽机——按韩冈的说法是功率强大,成本低廉,随处可用——只要一台蒸汽机,加上碾米磨面的磨,光是碾米磨面一项,就能让京师所有磨坊关张大吉。

而水力磨坊,看着比蒸汽机能省下柴火钱,但那先得有钱买下汴河两边的贵价地,还得让朝廷同意出借汴河水力——这成本,可是要远远超过煤炭的价格。

蒸汽机只要能够投入实用,与之配套的碾米机和磨面机则很容易就能设计出来。有厚利在前,又可以借鉴水力、风力的机器,当然不会慢。

真要给冯从义抢先开了蒸汽磨坊,不仅他外室旁边的磨坊,京师其他磨坊都要关门了。

“其实,”韩冈听了之后,就说道,“你让几家店用他家的磨,做上一年生意,再请他上门做客,好生相商,再给他提供一个大一倍的好铺面,跟他合伙做磨坊买卖,他怎么会不搬家?”

“这还真是好主意。”冯从义鼓掌赞叹,可从他的表情上看,却没有太多惊讶,应该是早已想到过的,“要是江南那些黑心的家伙,都跟哥哥你一般仁义,喜欢双赢,就没有这一次的事了。”

韩冈摇摇头,“难哦。”

尽一切可能降低成本,扩大利润范围,这是资本家的特点。为了百分之三百的利润,资本家能吊死自己的绳子都能卖给敌人。

真要说起来,两浙丝厂厂主所做的每一件事,都是韩冈前世的世界曾经出现过的。他们不做,自然会有人做。

迟早雍秦商会中会有人觉得在开发新技术的同时,从工人身上盘剥一点好处,可以得到更大的利润。只要不给韩冈发觉,暗地里做一做也没什么。

现在之所以还没有,完全是因为现在的利润还足够多。而棉纺工厂的工厂主们现在还觉得为了一点钱,却冒着失去了韩冈信任的风险,未免有些不值得。

但韩冈都不敢冒险去考验人性,只能想着日后拿江南的丝厂厂主们,杀鸡给猴儿看。

“江南的丝厂就看他们怎么做吧,是生是死全,看他们自己。”韩冈说道。

“一切都是贪心的缘故,即是走上死路,也是他们自找。河北丝厂就没那么贪心,工人虽苦,可也没有闹到那步田地……这北人和南人,还真就是有差别。”

如此充满地域歧视的发言,让韩冈失声笑了起来,“那是因为河北不适合养多季蚕,只开一季工,想盘剥也盘剥不了多少,百姓受损也不重。要是气候跟江南一样,看他们怎么做?糊弄外面的说法,你不要自己也上当。”

冯从义不好意思的笑了两声,又有几分不服气的说道,“其实还是有些差别的。”

韩冈道:“要是西域办起棉纺厂,你看王景圣会怎么做。”

王舜臣驻屯西域,早就开始种植棉田。这些年,他占据了天山脚下的几处大绿洲,通过暗渠将天山上的雪水引下来。粮田不提,仅仅是开垦出来的棉田,就已经超过七百顷。这已经相当于关西、陇西棉田总数的十分之一。

“幸好他没想着要做。”冯从义庆幸道,“这要做了,西域都给他祸害了。”

王舜臣在陇西就有产业,棉纺工厂也有他一家,还没想着要利用这些属于官产和移民所有的棉田来纺纱织布。北庭、西域两大都护府也有官员曾提议过,由朝廷开办棉纺工厂,由此提供军需,并赚取军费。但韩冈就在中书,轻而易举的就以与民争利的名义给否决了。

“也是可惜,西域的棉花运不出来,否则棉布的产量还能增加。”

“关西那边是怎么传的?黑风驿一年只刮一场风,从正月初一刮到腊月三十,狂风一起,磨盘大的石头都满地滚,铁做的车厢都能给吹翻掉,修了铁路也没用。”

西域、陇西,相隔四千里地,而且中间还要经过几处整日狂风的荒漠。因而七百顷棉田的出产,基本上都是做成了冬衣冬被。一来棉花从西域运到关西不容易,运费远远高于成本。二来,西域也的确正需要这些填充料,比起羊毛,比起丝绵,单纯的棉花的价格当真不高。

“等到王景圣将黑汗国解决了,工厂需要煤和铁,也不能缺水,伊犁河谷是最合适设立工厂的地方。都不打算从中赚钱,而是”

“太远了,都管不到。”

“也不一定要管,日后自然有办法。”韩冈说道。

“就是移民也太远了,比起西域,愿意去两广、云南的还多一点。朝廷宣传两广、云南太多了,”

朝廷一直在鼓动移民,尤其是在报纸上是经年累月、连篇累牍,都在宣传移民,韩冈改革科举,新增的秀才、举人,都有朝廷核发的荒田证。只要移民,上百亩土地轻而易举到手。

江南在一千年前是什么样?两千年前又是什么样?不是无数先民持续上千年的辛勤垦殖,怎么会有如今的鱼米之乡?

从‘厥田唯下下,厥赋下上’的‘岛夷卉服、厥篚织贝’之地,到唐时的‘腰缠十万贯,骑鹤下扬州’,扬州的变化是显而易见的。

既然大禹时土地卑湿的江淮之地,能变成如今的胜地乐土,既然至隋唐时,亦只有寥寥数县的福建,能变成人文荟萃之地,那或雨水丰沛,或气候宜人的两广、云南,当然也能成为下一个江南,下一个福建。

朝廷持续不断的如此宣传,不断的为之鼓动呼吁,移民边疆的规模自是越来越大,虽不能说车水马龙,但数量上,主动移民的家庭,每年都超过五千户。理所当然的,愿主动前往西域的最少,都没超过三位数过,而且都是被判流配西域,遇赦不得归的犯人的家属。

“王景圣手下的军队,几乎都已在西域安家,娶了当地的妇人。过些年,朝廷再遣军去西域。只要娶了妻、生了子、分了地,相信绝大多数人都会安心住下来的。”

“能多派些就最好了。”冯从义一向支持开发西域,他又笑着说道,“上一次是西军,下一回该轮到京营了。”

“那得看情况了。”韩冈一句带过,“丝厂的事,你帮我多留意,过两年,朝廷就准备不再纳绢,而改纳钱了。”

冯从义精神一振,连忙问道,“朝廷打算发行多少银钱?”

“今年是两百万贯。”

冯从义心里算了一下,点头道,“那差不多就没问题了。”

丝绢在大宋之所以重要,那是因为丝绢在很大程度上,代替了货币的作用。朝廷的封桩库中,很大一部分存放的是绢帛,而不是钱币。

在过去,由于铜钱铁钱太过沉重的缘故,并不方便商人们带着走南闯北,所以质轻价高、易于携带的丝绢,就成了买卖时的货币,被称为轻货。

现在朝廷铸造大小银钱,价值、面值皆高,就是用来跟丝绢争夺高值货币的市场。

这两年,通过各种途径,流入大宋的白银数以百万两。朝廷现在能轻易的拿出一百多万两来制造银币将纳绢改为纳钱,夺取绸缎的货币价值,也就成为了可能。

一旦朝廷在收税时,将纳绢改为纳钱,对各地丝厂都将是一个沉重的打击。在过去,他们可以直接拿着丝绢去付账,去缴税,去购买其他商品,一时间花不出去,存在库房中也不用担心。但一旦税改,丝绢卖不出去,那就是要赔光棺材本了。

“但朝野必有异论,”

“不用担心,这样可以减少折变,是善民之政,没人能反对。”

百姓缴纳两税,有交钱,有交粮,还有纳绢的,只要官府需要,各地的特产都可以作为征收对象。

地方上的官吏,就借了这种混乱的税收模式,在征税的时候,随意的将缴纳上来的钱粮绢帛,折换成等值的其他税品。

交钱的折换成粮食,交粮食的折换成绢,交绢的折换成钱,在折换过程中,折换的比价则掌握在税吏们的手中,自然而然的,就成了牟利的工具。只折变一次,算是极有良心了,一般都要折变两三次,将税额上浮一半以上,多的甚至能有五六次,转了一圈重新回到原本要交物品上,变成了原来的两三倍。

对于这一残民之法,一直以来,朝堂上都有不少人提出要改正。但他们的呼吁,根本没有任何作用。

折变之法,本是自五代传承下来,大宋立国又有百年,利益早已盘根错节,在朝廷征税上得利丰厚的地方大族不在少数。在他们都反对的情况下怎么废除?

何况朝廷每年下拨官员和军中的俸禄,很大一部分都是实物。粮食、绢帛、布匹不说,填充冬衣用的丝绵,取暖用的薪炭,都是税收的一部分。

如果朝廷废除征收实物税,那这折变的问题就可轻易解决。剩下的,就是怎么压制住来自下层官吏的反对声。
