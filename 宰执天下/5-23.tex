\section{第三章 时移机转关百虑(九)}

【两连更,这是第二更。】

大约是黄昏的时候,王旖从宫中回来了。

比起正常的拜谒,回来得要迟了一些,韩冈从书本中抬起头,问道:“今天好像是又迟了,是皇后还是朱贤妃。”

王旖卸了妆,换了一身家常的便服出来。她在韩冈身边坐下,自己揉着脖子,“是朱贤妃,还以为会被留到宫门快落锁的时候呢。”

“怎么了?”韩冈问着,顺便将手伸过去,想要帮忙按摩。

王旖痛叫了一声,将韩冈的手一下拍开,“官人你手太重,骨头都给你捏碎了!”狠狠的瞪了韩冈一眼,又叹道,“真不想去宫里,每次戴花钗冠都觉得重,得让魏娘子好好捏一捏。”

王旖虽是这么说,却没有叫自己的梳头娘子进来——负责梳头的婢女,并不是仅仅负责梳些时兴的发式,更多的为主母的衣着打扮做参谋,有的还各有绝活,王旖身边的却是会按摩捏骨——夫妻两个正议论着宫里面的事,不便在下人面前说。

“方才谒见过两宫和皇后之后,就被朱贤妃拉去说了一通闲话,本来心想这下得拖到天黑了,可没多久,福宁宫那里就派人来,跟朱贤妃不知说了两句什么后,就打发了奴家出来。也不知道是不是天子那里招朱贤妃。”

韩冈啧了一下嘴,“还真是麻烦,每次进宫都要被拉着说话。”

“这还要多谢官人呐!要不然就能像娘一样,进去就出来了。”王旖冲韩冈抱怨的哼了一声,又揉起脚:“宫里面的娘子,一点都不知道体恤人。又是带着花钗冠,又是穿着朝衣,还从宫门开始,就绕着后宫走了好一圈。三个殿都跑过了一遍,累得脚都疼了。”

“弄点热水来泡泡脚,平常可不会走这么远的路。”韩冈说着提声叫了人进来,吩咐了去准备热水,“泡过脚就去吃饭,素心指挥着厨房,今天可是准备了不少好菜。”

“嗯。”王旖笑着点头。又看韩冈在卧室中拿着本书,突然觉得不对劲,“怎么不在书房里面看书?”

“问问你儿子吧。”韩冈无可奈何的摇头,“放鞭炮的时候,二哥儿不知怎么弄的,一发冲天响就飞到了书房里,一开始都没在意,过了一阵就见了烟。”

王旖惊了一跳,“走水了?!”

“是走水了。”韩冈状似无奈的叹气,“就烧了半幅帐子,但几盆水泼过去,书房里的书全都毁了,南娘和云娘正带着人在西厢里烤书呢。”

王旖愤怒起来了,“都是你,昨天说什么放鞭炮!今天只是书房被水泼了,明天烧了房子怎么办?哥儿姐儿心玩野了,以后不小心伤到自己怎么办!?”

“是为夫的错,所以老天将我的书房给毁了,这可是重罚啊。”

韩冈唉声叹气,王旖却气得牙痒痒的。狠狠的又剜了韩冈两眼,问道:“二哥儿他们呢?!”

“罚了他们三个去抄书,吃饭前得将三字经给抄一遍下来。字还得端正,否则就喝水过夜。”韩冈抬眼看着又有些担心起来的妻子,笑道,“饿上一顿没关系,为夫当年在子厚先生门下,一天一顿都熬了几个月,少吃一顿算不了什么。而且他们三个还不一定写不好,读书识字,可比为夫当年聪明。”

“官人倒是谦虚。”王旖说了一句,也暂时放下心来,“等正屋和退思堂都修好了,将书房搬回去,就不会再被鞭炮给烧了。”

“太皇太后的情况怎么样?”韩冈问道。

王旖摇摇头,“还是在殿外。”她的声音低了点,“看样子有些难了。几个翰林医官出来后脸色都不好。”

韩冈皱眉,想了想道:“过两天如果太皇太后的病情还不能好转的话,天子当会让宰辅去大相国寺烧香祈福,到时候就看人选和人数了……”

韩冈话说了半截,王旖却明白,宰辅们去相国寺的人数越多,地位越高,那么就代表太皇太后的病情就越重。如果是宰相王珪领着两府的全班人马去相国寺,那基本上就可以等着天子大赦天下了。

放下了太皇太后的病情,王旖看着桌上厚厚一摞书册,又看看韩冈手上的书卷,从字体上看,当是手抄本:“官人今天读的什么书?”

“正在看史论呢。”韩冈将手上的手抄本扬了一扬,“苏家父子的。如今空闲的时候多,正好多看点书。”

韩冈前生只知道唐宋八大家,只以为苏家父子三人诗词歌赋写得好,但后来才知道,文名可不仅仅是诗词歌赋。苏家父子当年在京中出名,靠的是史论和治策。

苏洵写了《几策》、《权书》、《衡论》,苏轼则是写了《进策》《进论》五十篇献与当时的仁宗皇帝,苏辙当时还差一点,但他也有十几篇论史的文章。

三苏父子文章一出,在京中又得欧阳修、梅圣谕等文坛宗主引荐,一时名声大噪。

韩冈今天将三苏的文章稍稍浏览了一遍之后,才知道为什么王安石要说他们是战国纵横家一流,的确全都是纵横捭阖的议论文。

“官人觉得三苏之作如何?”王旖很感兴趣的问道。

韩冈皱眉想了想:“三苏的作品主要是论,对史事的评论,以古讽今。不像司马君实那般,近似于单纯的史官了,而是秉承春秋之法,以史论明儒门大义,世间有称之为蜀学者,不为过当。如今的各家学派多论心性义理,以解经释义为上,蜀学偏近于史,算是个异类。”

王旖讶然:“笔削春秋……官人评价这么高?”

“该怎么说,似是而非,得其形而已。老苏倒也罢了,但苏子瞻的《进策》二十五篇、《进论》二十五篇,只是花团锦簇而已,更像是凑个整数,硬给凑上五十篇。”

韩冈翻了翻手上的书,指着其中一篇给王旖看:“苏子瞻的一篇《论诸葛亮》,说‘曹操既死,子丕代立。当此之时,可以计破也,何者?操之临终,召丕而属之植,示尝不以谭、尚为戒也。而丕与植,终于相残如此,此其兄弟且为寇仇,而况能以得天下英雄之心哉!此可间之势,不过捐数十万金,使其大臣骨肉,内自相残。然后举兵而伐之,此高祖所以灭项籍也。’”

王旖摇着头,她过去除了三两篇有名的之外,苏家父子的史论并没有多读,没想到里面这么不靠谱,“读过《三国志》就不该这么想。”

韩冈点头道:“所以说这是纵横家之流的想法,以为花点钱、动动嘴皮子,就能让敌人不战自溃。‘兄弟且为寇仇,而况能以得天下英雄之心哉’,从曹丕和曹植的关系上推到天下英雄上,这个引申,毫无道理可言,当真是一厢情愿!怎么不拿去比李世民?”

“爹爹过去也说是苏家父子是纵横术,一开始就不怎么喜欢老苏的史论。”王旖回忆道:“当大苏参加礼部试时所写的《刑赏忠厚之至论》,爹爹知道‘皋陶曰;杀之,三;尧曰,宥之,三’是杜撰后,就更是不喜欢了。”

皋陶是尧时的法官,他三次判人死罪,而尧则三次宽宥罪人。这一个典故是苏轼拿来证明尚书中‘罪疑惟轻,功疑惟重’这句话的前半句——这八个字,也是《刑赏忠厚之至论》这道题目的来源——但此一典故,主考官欧阳修不知道,副考官梅圣谕也没听说过,考官们没一个听说。

欧阳修和梅圣谕以为自己读书不广,不知道这一典故的来源,虽然其他考官认为无所依据要将之黜落,可欧阳修见文章写得又好,也就信了他。但当知道是苏轼所写之后,欧阳修一追问,竟然是杜撰!

“不谈文章好坏。从议论的原则上说,如果论据是伪造的,论证就毫无依据,论点也便不可能成立。整篇文章写得再好,都是不合格。”韩冈笑了一下,“时人将此事当做一段轶事,但要是这样的作风用在政事上又该如何?”

“是啊,就是这个道理。”王旖又道,“还有之后小苏在制举上,议论仁宗皇帝贪好女色,宫中贵姬数千,日夜游宴,不视朝政。这分明是拿道听途说之语博取直名,爹爹是主张黜落的。韩曾两位相公也跟爹爹同样想法,认为是污蔑天子,不过仁宗皇帝觉得本是求直言,不当以言辞罪人,还是将他取中了。”

“但岳父不是拒绝为小苏起草制书嘛?”韩冈笑道。

拗相公的脾气,在几十年前就倔强得让人头疼。他在担任制科考官的时候,认为苏辙应该黜落,没资格通过比进士科还要高一个等级的制举考试。纵然仁宗皇帝录取了苏辙,但当要给苏辙起草任命的时候,担任知制诰的王安石就死活不肯草诏。谁来劝都没用,最后硬是把苏辙拦了近一年。

听出了韩冈言语中的戏谑之意,王旖就又白了他一眼,“爹爹脾气就是这样,何况又没有做错!”

