\section{第38章 天孙渐隐近黄昏(上)}

“二哥,你可算是回来了。”

韩钟刚刚走进韩府的后院,就看见了他的姐姐韩锳。

少女容貌清丽绝俗,只是眉宇中多了些英气,冲淡了她的容貌给人的震撼。

韩钟见到了她,就像老鼠见到了猫,连忙低头:“姐姐。”

少女叉着腰,一点也没有大家闺秀的样子:“玩得忘了回家了?”

“哪里敢忘?姐姐要出嫁了,小弟断了腿也会爬回来的。”韩钟嘻嘻笑道。

“就你会说嘴。”韩锳脸红了一下,向前院张望,“四叔也回来了。”

“正在外面跟爹爹说话。我先进来拜见娘娘和姨姨。”

“娘娘在正屋里,跟阿娘说话。云姨带着六哥、七哥他们在后园读书,心姨在小厨房。”韩锳陪着韩钟往里走,打量着自己的弟弟,“黑了,瘦了。去江南玩了一趟,没把心给玩野了吧。”

“哪里玩了?这时候,冷得厉害,又湿又冷的,自骨头里发寒。在那边就想着早些回家。”

“你就胡说八道吧。玩了那么久才回来,不是乐不思蜀什么?”轻盈的抢先跨过一道门槛,韩锳回头看着弟弟,“可没去那些不该去的地方?”

韩钟愣了一下,醒悟过长风文学

www.

来后立刻就叫起了撞天屈,“我要真的去了,四叔还不揪着我回京,让爹爹打死我。”

韩锳抬高手臂,像安抚小孩一样,拍了拍韩钟的头,“好了,好了,就信你,就信你。”

“小弟这会回来,有给姐姐你带了礼物。还有几样是给姐姐你屋里玉竹她们几个的。”

韩锳道:“见人就送礼,都学得跟四叔一般了。”

韩钟笑道:“小弟这叫长袖善舞。来回一趟不易,多带一点礼物,也算尽尽心意。东西太多,就放在外面,待会儿卸了车,再拿进来。”

韩钟回来带了许多礼物。以相府之尊,韩家的公子们,手边当然不缺好东西。逢年过节、诞辰,都有人赶着送礼。但外人送礼,总不如自家兄弟知道喜好。

老三喜欢藏书,尤其喜欢同一本书不同版本的对比。韩钟带了一整箱。单只是,就有三家所出。

老四喜欢书法,韩钟就带了他亲手拓印的诸多碑文。在下面的四位兄弟都还小,韩钟就带了许多京师稀罕的玩意儿。

按照几个兄弟的喜好,韩钟将礼物一一备好。给父母尊长的礼物更是准备妥当。

听了韩钟数了好一通,韩锳笑容稍敛,担心的说着:“花了多少钱啊,零用钱别乱花,让娘娘知道了会挨骂的。”

“也没花多少,都是些便宜的,贵得小弟也买不起。有些是外公外婆要带回来的。另外还有一些,是要转交给二舅舅、二舅妈、京哥哥,还有越娘妹妹的。”

听到弟弟提起表妹,韩锳就没笑了,“前些日子,越娘来了家里一趟,看着气色不怎么好,宫里派来的人管得又严,只稍坐了坐就走了。”

听到王越娘的消息,韩钟的脸色黯淡了点,强笑道:“宫里面也管得太多了,还没嫁过去,就把嬷嬷派过来了。”

“谁让太妃是那种脾性。太后都没说什么,她倒是急得跟什么一样。”韩锳生着气,“娘娘上回进宫,路上遇见太妃,太妃连礼都没回。”

“娘娘气到了?”韩钟连忙问道。

韩锳摇摇头,“娘娘没说什么,是把太后气到了。第二天,还特意把娘娘请了去,代太妃道了歉。”

韩钟脸色微冷,“连礼数都不讲,太妃的名声也难怪不好。”

韩锳姣好的双眉蹙起,多了一分忧色:“真要是官家亲政了,听了太妃的谗言,还不知怎么看我们家呢。”

“不用担心,有爹爹在。”

韩锳立刻展颜笑了起来:“是啊,有爹爹呢。”

“哥哥去了哪里?”韩钟问道。

“苏姐姐一家到京师了,住在苏平章的府上,哥哥今天就到那边见岳父去了。”

“姐夫呢?”

“他还要读书呢。”韩锳回了一句,方觉失言,登时双颊绯红,抬腿狠狠踢了韩钟一脚,“还不是!”

“还不是什么?”

姐弟两人说这话,已经到了后院的正屋前。韩锳含羞挟愤的一声叫,倒让屋里的王旖和周南听见了。

听见里面问起,韩锳就瞪了韩钟一眼,跑进屋内:“娘娘,二哥欺负我。”

“二哥欺负你?你不欺负他就好了。”周南站起身,上前迎了韩钟进来,“二哥回来了。”

“孩儿拜见娘娘,南姨。”

韩钟先整了一下衣服,然后进屋,跪下来拜见王旖和周南。

王旖把儿子叫上前来,仔细的打量了一番,看见没有哪里磕着碰着,方才放下心来问道,“去江宁见到了你外公、外婆了?”

“外公、外婆身体都好,外公现在每天在家里读书写诗,隔两天还去一趟书院。”

王旖听了,却忍不住抱怨:“都病了一场,还不知道休息。”

韩钟笑道:“外婆也这么说外公。”

说话间,素心和云娘带着几个弟弟都过来了。严素心看着韩钟,对王旖笑道:“二哥出去了一趟,个头高了,人也干练了许多。”

云娘道:“就是瘦了些,是不是没吃好。”

“外面的口味是不如家里好。不过孩儿也没饿着,只是因为长高了一点,才看着瘦了。在外面的时候,四叔还逼着孩儿多吃饭菜,说是出门在外不比家中,口味就别讲究了,只有填饱肚子才有精神。”

韩钟难得出门一趟,又不是像去年,因为王安石重病才去的江南,一直守在江宁,而是在江东、两浙绕了一圈,经历颇多。

他笑笑说说,先让人出去取礼物,又拿出了从江宁带回来的信给王旖。

……………………

就在韩钟进后院的时候,韩冈正与冯从义说着话,“江南的情况怎么样?”

“其他都还不错,只有织户不好。”冯从义言简意赅,“民家自织的素绸现在卖不出去了,生丝也不行了。江南以耕织为生的五口之家,能耕种十亩地,一年能有十几匹绢,口粮、租子和税赋都从此中来。如今只剩下土里刨食,最多也只能养活三口人,这逼得农民要减租。闹佃的事情虽不如丝厂的事起眼,但数量确实比前些年都要多了。”

“这也没办法。”

韩冈没办法感概太多,小农生产被工业化大生产所淘汰,这是必然的事。除了转变成工艺品,只要是日常生活用品,手工制品无法与工业产品争市场。

冯从义也很冷静,“物竞天择,适者生存。争不过就是争不过,聪明的转行,有毅力的去学织绫罗,什么都没有的,那就只有被淘汰。”

机织的丝绸,以及机缫的生丝,质量比民家手工的要强。机械与手工最大的区别,一个是规模,一个是稳定,这两方面,都是机器占了绝对的优势。

如果是各种缂丝等特殊纹饰花样的绸缎,制造技术掌握在官府和极少一部分专业生产者手中。男耕女织的小农生活所生产出来的丝绸,只能是最普通的素绸。而机器生产的绸缎,正是处在这个等级。高档绸缎,机器生产不了,但机器生产出来的绸缎,却能以产量和质量上的优势,将民间手织丝绸的市场给冲垮。

小生产被淘汰,这是历史进程。虽然说是韩冈将车子给推动,但他现在也拉不停了。给那些受害者廉价的同情,就是鳄鱼的眼泪,反而是个讽刺。

“过些日子,朝廷会加大铁路的铺设。每个地方都会各自的特产,如果能运出来的话,也能弥补一下丝绢上的损失。在这方面,官府会加以引导。”

“这是好事。”冯从义点头,却又问道:“不是说南方修建铁路的条件不如北方,铁路修建的重心暂时还不会南移吗?”

“蒸汽机差不多可以用了。过些日子,你让商会去军器监那边购买设计图和生产许可证。”

冯从义喜笑颜开,“哥哥可要把价钱算低点。”

“朝廷为了蒸汽机花销了多少?能低得下来吗?一起凑个二十万贯出来,少了会有人说闲话。”韩冈说道,“图纸拿回去后,要跟商会里面的蒸汽机加以对比,不要全盘仿效,商会之前研究的基础绝不能放弃。”

冯从义郑重的点头,“小弟明白,哥哥放心。”

“那就好。”冯从义这么说了,韩冈便放心了,又问起另一件事,“秀州倭人坊你去看过了,情况怎么样?”

“那不是开丝厂,是开油坊磨坊。”冯从义冷笑,“进去的倭人,不给榨出骨头里的油,把骨头碾成粉,那些人都不会满足。也亏辽国能不要脸皮把人卖了来。”

“倭国的人口数量不比辽国少多少,耶律乙辛当然要未雨绸缪,免得日后麻烦。比起贩卖妇孺,他把倭国的男丁往矿坑里送,那才叫狠。现在这算什么?”

“还未雨绸缪什么?上层的倭人杀光了,现在连认识倭文的都没几个了,连个能出来领头的都没有了。”

“这是最聪明的做法。灭了倭国的文法,把倭国变成了女直、室韦一般的边荒部族,统治起来就容易多了。”

“倭国是完了,高丽也差不多了。过段时间,一样会往这边卖。”冯从义忧心忡忡的对韩冈道,“哥哥,难道就任由他们如此肆无忌惮下去?说不定商会里面也会给带过去,毕竟太节省成本了”

这是两条路线。一条是纯粹走技术线,通过技术发展降低成本,另一个就是靠压榨工人,做血汗工厂。路线不同,立场当然也不同。

单纯从效果来讲,说不上谁好谁坏,而且两条路线并不是对立,也可以参合而行。不过技术的进步能带来社会的进步,血汗工厂虽然也能,但进步速度就慢太多了。韩冈也不想雍秦商会的风气给败坏了。

“过几年,在报上爆出来,他们的日子过不好。就是倭人,那也是人,不把人当做人来看,有几个人会站在他们一边。”
