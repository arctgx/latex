\section{第20章 冥冥鬼神有也无(一)}

出京之后,韩冈一行就不断的在沿途的驿馆中换乘驿马急速南下。

才两天的功夫,就已经到了汝州,算是很快了。只是比起去岁和章惇一起南下,还是稍嫌慢了一点。

在官道上行路,一天走得路程皆有定数,一程、一程全都是预定好的。晨起出发,暮色降临时便能投宿在预定驿馆中。如果行程急,也可以兼程而行。一般他人兼程赶路,那是两程路一天走完。而前日韩冈和章惇南下时,则是两天走了七程的路。

当时不论韩冈和章惇都有些吃不消,只是军情如火,片刻也耽搁不得。为了尽快南下,而不得不那么做。章惇当日还说,‘不过一两年坐在衙门里,不意已是髀肉复生。’韩冈也有同样的感觉,完全没有了旧时在陇右,骑着马上一天走上小三百里的能耐。

不过这一次南下就不同了,韩冈在南方走了一圈之后,他又找回了旧日的感觉,重新适应起骑着马的长途旅程。

这一日同样是一路疾行,到了天色将晚的时候,抵达了汝州和唐州的交界,也就是方城山这一段。韩冈本来是准备一口气赶到唐州境内的方城县,可他回头看看跟着自己的四名幕僚,一个个都已经是大汗淋漓,连在马上坐直都没了气力。

韩冈在一座小小的驿站门前拉停了马,“今天就到这里吧,看天色也赶不到方城县去了。”

听了韩冈的话,李复四人几乎是从马上滚下来,两条腿直着打晃,扶着马鞍喘了好一阵气。

一行人所骑乘的驿马,中午时虽然在中途的递铺中都换过,可同样是累得厉害,呼哧带喘的鼻中喷着粗气,浑身上下都是汗水,将毛皮全都打湿了。

听到外面的动静,驿站中的驿丞忙迎了出来。一问过韩冈的身份,慌里慌张的行了礼,接着又手忙脚乱的指挥着手下的驿卒为韩冈他们的整理房间,

一名看起来有五六十岁,充作驿卒的老兵过来牵马,顺手在马背上抹了一把,手上顿时就满是水迹。他扯着缰绳,把马往后面的马槽拉过去,还低声咕哝着:“这一下,半个月不用开张了。”

韩冈耳朵尖,听到了驿卒抱怨。倒也没生气,摇头道:“终究不是西北的驿站,驿马少,还没一个匹好的。”

韩冈有着来自后世的记忆,天南地北的行程比任何人都多得多。但他在这个时代,也不过仅仅经历过关西、广西和中原,东部沿海地带都还没有去过一次。

相对而言,年纪最长的马竺,他游学天下的经历就丰富得多,也早一步缓过气,笑道:“京西的驿站还算好了。福建养在海岛上的州屿马,龙学你见了都认不出是马,比驴子还要小一圈,但还照样是放在驿站里使用。”

“陕西如今茶马互市,一年有了近三万匹青唐马。京中、陕西、河北的驿站之用已经是绰绰有余,就是东南差一点。”韩冈再看看还没有回过气来的李复、陈震和周毖,“这两天,你们都累了。等明天过了方城,到了罗渠镇后,就可以转官船。日夜行舟的话,过襄州、至江陵、穿洞庭湖,至潭州,再往下湘江、灵渠,到邕州这一路水程,也不用多少日子。”

李复四人的大腿内侧,这两天被马鞍磨得厉害,都破了皮,一直都在忍着。听说明天就能换船了,脸上都浮出了难以掩饰的喜色。只是陈震还故意感到遗憾的问道:“今天就不去方城了?”

“今天就在这里休息。”韩冈笑着,“也可以好好看一看着天下九塞之一的方城塞。”

李复抬起头,站直了身子,左右看看:“哪里来的山?”

“的确是看不到。不过这里的确就是方城山所在。”

从地理上来说,这是从南阳盆地东北侧的垭口。东有桐柏山,西有熊耳山,只有中间这几十里是个空当。在《吕氏春秋》之中被列为天下九塞之一,井陉、雁门等险塞并称。只是真要说起来,站在垭口中段的驿站处,向东西两侧看过去,都不见有高峻的山峦,最多也只是在接近地平线的地方能看到浅浅起伏的矮丘。

就在驿站的门前不远处,有一道宽达十数丈的沟壑,但里面的水很浅,看起来连膝盖都没不过去,也没有流动的迹象,老远就能闻到一股剧烈的恶臭味。

“这是河吗?”周毖低头往沟里面看了看,“怎么都不见水流?”

“这是当年准备沟通荆襄和京城的漕渠,只是方城山这一段地势高,两次开凿都失败了,没能通水。最后就留着这一段河沟在这里。”马竺对着韩冈和几个幕友笑道,“前两年因故去往鄂州探友,曾经在这条路上走过。”

“要是方城山的这一段有渠道可以通行,那就不用再经过汴河,便能走水路直下南方,而荆襄的纲粮,也不用再绕道汴河。”陈震道,“太平兴国三年,为了能让荆湖漕运直通京城而开凿漕渠。‘南阳下向口置堰,回水入石塘、沙河,合蔡河达于京师,以通湘潭之漕。’只可惜见事不见功,否则京城安危就不用全托付在一条汴河之上了。”

马竺、陈震,这两个关西人能将眼前这一条废弃河渠的来历用处说得一清二楚,并没有人感到惊讶。张载门下的入室弟子没有一个是只会读经书的书呆子,水利河工是这个时代最为重要的政务之一,只要有心出来做事,都必须知道一点。

周毖也仿佛是为了在韩冈面前表现自己,不甘示弱的说道:“白河入汉水,汉水入大江。沙河则是汇入淮水,走蔡河入京城。沟通两河水系,通漕运于京师。如果漕渠打通,就能从京城乘船南行,直入桂州。”

“可惜就是地势差了一截。第一次开凿,很快就被山洪冲毁了。第二次开凿虽然成功,但用了十余万军民所打通的渠道,最后的水深就只有一尺不到,”陈震指着下面的河渠,“许多地方只能没脚,勉强让空船走。”

“襄汉漕渠虽然两次都没能成功,但不是还成了一段?”李复说着,虽然他也不认识眼前的沟壑,但他对于国中的水利河渠,照样有着深入的了解,“沟通江陵汉江的漕渠可是凿通了。从襄州自汉江南下,不用一直走到鄂州【武汉】,直接可以通过江汉漕渠转入江陵,少走上千里水路。”

“水道不通,怎么说都没用。”马竺叹了口气,打算息事宁人,“可惜了。”

陈震突然笑道:“小弟对于此处地理不明,不敢妄言。不过如果仅仅是水势低浅,到有些变通的办法。”

“有什么办法?”几人一起追问。

“用斗门!汴河之上,可就是用斗门来调节水势,并放水淤田。”

“若是能用早就用了。”周毖摇着头,“斗门的确能调节水深,但汴河本来就有活水,眼前的这条襄汉漕渠,渠中之水只能没脚,斗门根本没用。”

四名同窗隐隐的在较量着自己的学识,韩冈倒是乐见其成,良性的竞争是好事。而且等他们到了任上,自然而然会各自分工。自己这边也会将他们的情绪控制住的。

他也跟着看了看眼前的这条河沟。当日南下的时候,他也跟章惇也聊起过有关这条废弃运河的事。

其实这里就是后世的南水北调的通路,应该是中线。丹江口水库的名气老大,韩冈还是有些印象。他记得南水北调的中线可以自流到北京,应该就是靠了丹江口水库的大坝来提升水位。

眼下当然不会有丹江口水库,所以两次开通,两次失败。第一次被暴雨冲毁了渠首的石堰。第二次的确打通了,但水位太浅,无法行舟。

只要方城山这一段漕渠能开通,就可以走水路从邕州直达开封,只需要中间换几次船——现在其实也可以,不过那是要绕道扬子江,入汴河,多了几千里出去——而更重要的就是方才陈震所说,开封沟通南方的命脉就不会只有汴河一条。

但方城垭口,看似平缓低矮,偏偏实际上就是高了那么一截出来,使得水道难以通畅。不过以韩冈看来,跨越方城山的这一段漕渠并不是不能使用。

渠道中水位低浅,这个问题也很容易解决。将河底掘深,再使用堰坝和船闸——此时叫做斗门——调节水位就可以了。调节水位的设施,汴河上就有,而灵渠中的三十六道斗门,从秦朝一直用到现在。

但正如周毖说的,单独一重斗门对水位的调节能力太小,而方城垭口的这一段渠道,则需要更髙的水位,所以不得不放弃了。可这个问题,只要能使用堰坝加上多级船闸,完全可以解决。通过堰坝将渠中水位抬高,并使用多级船闸来让船只通过堰坝,有很大的希望成功。而且实在不行,还有轨道呢,只要中间几十里周转一下,也不算麻烦。

不过这是后话了,眼前还是先休息,如何平灭交趾才是当前要应对的问题。

一起进了驿站,吃过简单的晚饭,韩冈等人就在并不算舒适、但还算干净的房中睡下。

只是到了半夜,一阵蹄声将韩冈惊起。

掀开被子下了床,听见外面一个劲的催促着驿卒快点换马。

韩冈有些觉得不对。夜中还在路上奔行,肯定是紧急军情。眼下南方有战事的当然就是广西。而且他们从这条路南下,而广西的战报理所当然的也是从这里北上。

韩冈披了衣服,唤了睡在外间的亲卫起来,“去问问是哪一路的人。”

亲卫忙忙的下去了,但只这片刻时间,铺兵已经换了马走得远了。亲卫回来后,禀报道:“驿丞说的,是从广西来的,但到底是何事却不知道,也不敢打听,但他没看到铺兵身上带着露布。”

