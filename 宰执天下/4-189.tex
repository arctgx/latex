\section{第31章 九重自是进退地(三)}

大宋的官员一直都是很悠闲的,就算是在州县中做着亲民官,也能找到与亲友出外饮宴的余暇。而相比起他们用在一些喜闻乐见的消遣上的时间,他们放在公务上的精力就未免太少了一点。

不过到了朝官一级,又是身在京城,那么很多官员三更天就要起床赶去上朝。尤其是冬天,一边怀念着被窝中的温暖,一边还要冒着刺骨的寒风敢去宫廷,这份痛苦让许多官员都怨声载道。

幸好礼仪性质的每日常朝,连天子都懒得出现,只让宰相押班。有时候甚至连宰相都不出面,过去曾有几次惹来了御史的弹劾。至于普通朝官,如果手上有实职,就可以不参加,没有实职的,也能隔三差五的请个假。

不过到了每隔五日的常参,以及朔望之日,或是正旦等大朝会的日子,那就怎么也躲不了了。

正旦大朝,在京朝官皆得与会,文官武官加起来也有上千人。还有带着一系列显赫官职的皇亲国戚,都是有资格且必须参加朝会。

半夜三更的京城道路上,全都是向着宣德门而去的队伍。

韩冈从家中出来,一路上不知见到了多少要参加正旦大朝会的官员,上了大路之后,汇聚起来的人流浩浩荡荡,让人不禁惊讶,京城之中哪里来的这么多官?

巡城的队伍也为数不少,避让韩冈一行的几支队伍,都没有什么精神,缩着脖子的为多。方才出了家门所在街巷,巷口的潜火铺望台上,还响着咚咚的跺脚声。

韩冈呼出一口白气,随即在空气中消散,今天的确是挺冷的。比起前几天韩冈入京的时候,温度下降了不少,这样的气温再持续几天,估计蔡河都要冻透底了。

转到了内西门大街,上朝的官员越发的多了起来,其中有不少相熟的,互相之间贺着新年。

韩冈一行继续往前,到宣德门已经不远了。这时从另一条道上转过来一支人数颇众的队伍。有六七十人之多,提在手上的马灯都是长长的一溜,韩冈看了一眼,就随队避让到路边。路上的其他官员,也全都退避路旁。

这是执政一级方才拥有的人数。

不同品级地位,能带在身边的元随数目是有定数的,韩冈作为龙图阁学士能有七名朝廷给发衣粮的元随,而执政是五十到七十,宰相则是七十到一百。看着眼前的人数规模,地位不高的官员当然得避让到路旁,让对方先走一步——何况还有宰执专有的清凉伞在后面张着。

这一队的身份韩冈差不多也知道了。眼下的几位宰执之中,东府三人,西府四人,除了还没来报到的郭逵,还有刚刚接任的吕公著,其他人都是已经做久了执政,各自被赐了宅邸。而方才的方向上,则并不是宰执被赐宅邸所在的位置。何况一众元随挑着的灯笼上,还有个端端正正的吕字。

大小几十名官员都目送着那一支人马前行,不意却看见他们突然停了脚步。一名元随骑着马向韩冈这边奔过来,“敢问可是韩龙图?小人奉我家枢密之命,特来相询。”

果然是吕公著。

不过他是怎么猜出自己的身份的?韩冈疑惑着。这条路上放眼一望,前前后后倒是有几十队之多。多的近百人,少的就是孤家寡人一个,最多带个伴当。韩冈一行人数不多不少,却也并不算起眼。真么不知自己的身份,怎么给看破的。他的元随打起的灯笼,可没有标上姓氏。

“正是我家龙图。”一名元随随即答道。

“正是韩冈。”韩冈亲口发出的回答也没有耽搁。

“韩玉昆,可否与老夫同行一程?”吕公著的声音并不大,但在安静下来的街道上,清晰地传进韩冈的耳朵里。

吕公著招呼他,韩冈并没有犹豫,随即打马上前,与吕公著打了个照面,行礼问好。

韩冈没有见过吕公著,但他对当今的枢密使闻名已久。

前代权相吕夷简的儿子,如今又做到了宰执的位置上。因为反对诸多新法,又曾经弹劾王安石,他当然算是铁杆的旧党。

当初曾被被吕嘉问偷了奏章,跺脚大骂这位吃里扒外的侄孙是家贼。不过吕嘉问如今在新党中,也是地位甚高,可比吕家现有的第三代、第四代要强,论能力也是不差的。

吕公著的年纪已经不小了,比王安石要年长,当年与韩维、司马光、王安石并称,精神看起来还是很不错的样子。五六十岁的年纪,其实正是宰执官们正当年的时候,能三四十便晋升宰执的也就那么一两个,更多的还是按部就班的晋升,从群臣中脱颖而出,在五十多岁的时候得到宰执的任命。

吕公著见了韩冈,并没有说什么久闻大名的废话,只是上下打量着韩冈时,神色中有着几分赞许。两人一同前行,韩冈稍稍拖后一个马头的距离,保持着恭谨的态度。

吕公著的语气沉沉,“张子厚实在是可惜了,这世上能贯通诸经,有所阐发的人也就三五人。本以为他能继续传习大道,想不到转眼之间就已归道山。”

韩冈的心情沉郁了下来,张载已经归葬横渠,自己作为传衣钵的弟子都没能送让一程,还是王旖请王旁代送了奠仪。

不过张载的学生大半还在京师,韩冈今次回京,这两日有不少人登门拜访。能光大关学门楣——不,如今当是叫气学了,张载的声望早已不再局限于关中——眼下只有韩冈一人。

“当年老夫在洛阳,曾经与子厚多有往来。”吕公著继续说着,“子厚的才学是不用说了。为人朴厚,忠勤于事,老夫举荐于他,也是想他能有补于朝廷。玉昆前岁举荐子厚,当也是如此作想吧?”

“先师欲昌明圣教,光大先圣之学,韩冈即为弟子,自当一效犬马之劳。”

吕公著当年担任御史中丞的时候,的确是推荐了张载入京为官。那还是熙宁二年的事。从这一点上,韩冈就必须对吕公著保持足够的尊重。

吕公著点了点头,“还有张天祺,也是可惜了。天祺为人甚正,是个难得的监察御史。”

张载的弟弟张戬,韩冈当年第一次上京,曾受业于他。前些日子也病死了,张载肺病转重,其实也有伤心的缘故。在吕公著当御史中丞的时候,张戬曾是他的下属,自是有些香火之情。不过张戬之所以被赶出京师,也就是因为他参加了吕公著所领导的御史台的大合唱,最后受到了大清洗。

吕公著一个劲的提旧事,韩冈觉得有些纳闷。不过应该不会有什么诡谲,以吕公著的身份,当不至于如此下作。

行走了一段,向左上了御街。内西门大街也算是城中数得着宽阔的大道,但与跨度两百步如同一个广场的御街比起来,还是差了甚远。

御街上的的人当然更多,韩冈跟着吕公著,后面一张清凉伞打着,倒是沾光了不小的光。

吕公著还与韩冈说着话:“张子厚的正蒙一书已经刊行于世,老夫也有了一套,翻看良久,兼有所得。其中道义阐述甚明,当真不愧是子厚。”

“正蒙乃是先师潜心天地,参圣学之源,道益明,德益尊,数载乃有所成。先师心血所聚,若能得知枢密赞许,必感欣慰。”

“正蒙诸篇,老夫最喜大心一篇。‘德性所知,不萌于见闻’‘圣人尽性,不以见闻梏其心’,以子厚所言,人心譬如明镜,不为外物所扰。”

“‘耳目虽为性累,然合内外之德,知其为启之之要也’。德性,见闻,并行不悖。‘两不立则一不可见,一不可见则两之用息。’”

吕公著能自去了解张载的著述,这当然是好事。但他歪解张载的理论,韩冈却是心中不快。

吕公著信佛是有名的,与富弼差不多,司马光都说他们对浮屠的崇信已近乎于佞——而为了能儿子好养活,把‘和尚’当做阿猫阿狗一般贱名,给儿子做小名的欧阳修,则是被司马光评价为躁,两者都偏于极端。

韩冈反驳的话,有着针锋相对的意思,但吕公著倒是一笑,并不以为忤,反问道:“不知格物致知作何解?”

“大其心则能体天下之物,体万物之理,即为格物致知。”韩冈很简略地回答,在刚刚集结成册的《正蒙》一书中,也有许多关于格物致知的解释,想来吕公著也不会漏看。

宣德门处,章惇算是来得早了。作为枢密副使,他身边不缺人奉承。与几名上来讨好的官员说着闲话,章惇原本悠然自得的神情猛然间就收了起来,眼神也变得锐利,只是瞬息之后就又变了回去。

宣德门前,多少官员都看到了,枢密使吕公著和王安石的女婿韩冈竟然是并辔而至。但许多人都怀疑其自己的眼睛来,肯定是看错了。吕公著怎么会与韩冈言谈甚欢?吕惠卿也紧锁着眉头,疑惑不解的模样。

抵达了目的地,吕公著和韩冈致礼后分了开来,韩冈已经看见了章惇,主动过去打了个招呼。

章惇双眼左右一扫,周围的官员全都识趣的散开了。

“怎么?是不是想问为何会吕晦叔同行?”韩冈半开着玩笑,先一步开口。

“若吕枢密当真想拉拢玉昆你,岂会刻意选在大庭广众之下?”章惇摇了摇头,韩冈既然如此说话,那就不用担心了。

宫门之前,也是交际的场所,只要不大声喧哗,也没有御史会如此不知趣。

韩冈跟章惇说了两句话,就分了开来。过来与韩冈寒暄的人不少,有些是认识的,而更多的则很陌生。

说了一阵话,原本在天顶的天狼星渐渐西斜,宫中钟鼓忽而齐鸣,宣德门的侧门吱呀呀的打开了,还在说着话的一众朝臣,也收起了寒暄,渐渐汇入皇城之中。

