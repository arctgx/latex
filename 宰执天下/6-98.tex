\section{第十章 千秋邈矣变新腔(20)}

王存正用力揉着自己的太阳穴,虎着脸在考场上转悠着。

不仅仅是他,其他考官也都是一脸头疼牙疼的模样。

之前韩冈让他们这些殿试考官列举禁字词,已经让他们感到头疼不已。

而今天上殿,殿试考题又是一变,不仅加了一道体例不明的考题,连评卷方式都做了前所未有的改变。这更让他们头疼了。

不用说,这肯定是韩冈的手笔,没什么可以怀疑的。

韩冈说服了太后,不仅给新科进士们,还包括给考官们,都出了一个难题。

尤其是评卷方式变得极为繁复,批阅之后竟然还要加减乘除一番,这对许多精研诗赋论和经义的考官们来说,比赶他们上马绕城飞驰一圈都难。

初考官、覆考官还好,只管评定等第。就跟过去一样,将试卷依水平高低以五等排列,一、二、三、四,加上犯讳或不敬这种列入第五等的卷子。

但详断官的任务就重了。不仅要评定初考官和覆考官们意见相异的试卷,给出最终意见,还要将试卷等级换算成百分制:第一等百分,第二等七十五,第三等五十,第四等二十五,第五等零分——这个零,过去的算经中不见,只在最近的《自然》中出现过,但之前质问时,韩冈却说关西给小儿开蒙的算术书中就有。

如果仅止于此,王存头还不至于疼得如此厉害,之所以感觉都要裂开了,因为在这之后还有一重计算。

出给考生们的是两道题,一为旧体的策问,一为新体的申论,分成两张卷子。将会分别进行封缄,然后评判。但这两张卷子的评分最后需要合并起来,不是简单的相加,而是两题分属各自乘以一个系数,最后计算出结果来,两边相加。

什么叫做系数?乘以零点七、零点三又是什么意思?

王存乍看到给考官们的说明时,脑中一团浆糊,这到底什么天书?其他考官也都是呆然发愣,完全看不懂。

幸好韩冈之后稍稍解释了一下,就是年利七分、年利三分,通过本金来计算利息。

好了,这一下子绝大多数考官都懂了,但还是觉得麻烦,毕竟家里放贷都是有账房在管,浑家来监督,他们这等一家之主是袖手不离,只管拿钱花钱的。

而且相较之前的评卷方式,现在还要计算分数,这真的是殿试吗?

初得题时,王存和一众考官都大起胆子质问韩冈。

韩冈则回道:“这是最简单的计算。诸位皆是进士出身,试问若是连出给十岁小儿的算术题都做不来,传将出去,世人会如何看?为何为进士者可以得世人看重,理政临民?只因其德才并举,超于常人。就是荫补出身人想要候阙注官,还要考钱谷计算,各位都是进士出身,难道还能比他们差了?”

言外之意,这也是韩冈出给考官们的试题。如果不能做出正确的评判,就意味着他们根本不够资格。连最简单的算术都能错,还能指望他们外放州县,不会给胥吏欺骗?

就任考官,若称职,则受到奖赏。若不称职,则受到处罚,这都是应有之理。比如排名时将天子最后所点状元放在下等,考官都会因为判卷不当被罚铜。

如果只是或许可能被罚铜,倒还有心理准备。现在已经关系到未来出典州郡,甚至晋升到更高层的机会,这样处罚结果,很难让人接受。

太后点头应允,觉得韩冈说得很对,尽管她很有可能也对‘十岁小儿都会做的算术’完全不懂,但作为臣子,对此又能抱怨什么?

事到临头,还能反口请辞不成?

还是早点解决这场闹剧……哦,不,是殿试。

王存与同僚通过眼神交流自己的想法,作为一名称职的官僚,他们的官场生涯的座右铭永远都是不求有功但求无过。

……………………

考官们时不时的从上面走下来,在殿中座位间慢慢的踱着步子。

尽管他们都阴沉着脸,一个个都像是被人欠钱不还的模样,连脚步声都重了一点,破坏了一干精神敏感的考生思路,不过宗泽并没有受到什么影响。

他拈着笔管,在考题发下半柱香之后,仍在仔细的审视着题目。

第一道题的体例宗泽很熟悉,而且猜到题的考生应该很多,只是在问垂帘听政以来政事有无阙失,以及改进的意见,策问而已。

不过在考生而言,正是因为猜到题的人太多,问题又太过空泛,这样的题目想要写好很难,想要在数百篇进士文章中做到出彩更难。

就算在礼部试结束,到殿试开始的这段时间里,宗泽专门针对不同的可能姓,写了六篇文章,加上过去精选出来的五篇,殿试考题可能会出现的几个大方向,都在这十一篇文章的范围之内。再加上百余条推敲已久的对仗佳句,宗泽自信可以应对各种情况。

今天所面对的这第一道题,正是在宗泽预备范围之内,而且是重点。

按照一般人的意见,这一道题中必须说好话。至少将太后临危受命后的艰难困苦写出来,同时将击败契丹的丰功伟绩也彰显出来。至于施政上有什么问题,当然是宰辅造成的,而不是太后的。

当初苏辙举制科时,参加的贤良方正能直言极谏科,拿着道听途说的谣言来攻击仁宗,宰辅和考官们都要将其黜落,但仁宗却说求直言却黜直言之人,仁宗如此做,主要还是顾及名声,让苏辙钻了空子。事后知制诰的王安石死活不肯给苏辙写诰敇,一方面是姓格执拗的缘故——要求黜落苏辙的便有他一个,另一方面,也为苏辙这种纵横家的手段颇为看不上眼。

但在参加制科时,苏辙已经有了进士的资格,所以可以有恃无恐。换作是殿试上,用同样的题目,看苏辙敢不敢这么写?

宗泽由于早有准备,对照着题目和令人惊讶的禁字词表后,发现原篇甚至连一个字都不用修改,直接抄上去就行。由于用心许久,宗泽自信至少可以得一个不过不失的分数。而将牢记在心中的文章默写出来,也不需要多少时间,正好可以留下更多的余地给第二题。

但即将落笔时,他无声的又重复念了一遍考题,接着又是一遍,最后宗泽放下了笔,翻到了第二题上。

之前开考时,宗泽就匆匆将两道题都浏览了一遍。第一题让他惊喜了一下,而第二题给宗泽带来的就是惊愕。

题目很长,考题的内容是对辽互市问题。但题目中,不像一般的策问只有一个宽泛的问题,而首先给出了六条资料。从澶渊之盟开始,每隔上十数年,河北互市的细节,以及澶渊之盟和去年的宋辽战争最后达成的和议。

而考题又分为三个小题:第一条是通过给出的材料说明边境互市对宋辽两国关系的影响和弊病;第二条是如何减少弊病、扩大优势,要考生给出方略;第三:对此方略进行论述。

这是一道同时包括策问和议论的考题。而且十分具体,让考生无法利用旧文章来拼凑。

方才扫了一遍,宗泽就立刻选择先做第一题。当宗泽再次放弃第一题,而开始准备第二题时,他又一次陷入了沉思。

这道题……不好做。

……………………

章惇拿着殿试的题目,目眩良久。

他很早就知道,政事堂那边,韩绛、张璪因为自知争不过韩冈,所以殿试考题上,便干脆放手让韩冈去做。但章惇没想到韩冈会给出如此别开生面的殿试考题。

评分方法就算了,章惇早就知道韩冈打算怎么做,并不感到惊讶。但考试题目就不同了。他不知道那几位考官在得知韩冈生造出来的评分方法后,有没有认真看过考题,可章惇敢肯定,当满殿的新科进士看过考题后,没什么人会不诅咒出题人。就算不能出声,心理面照样能骂到韩冈的祖上十八代去。

太后所出的第一题,是问阙政,这当是参考了先帝所出的历次殿试考题,才想出来的题目。宰辅们也平常的将之润色,。

而第二题,没有了参考对象的太后,就只能给出一个方向。也很简单——对辽。换作是其他宰辅,大概提笔就能写出十好几道与北方邻居有关的考题。而韩冈却用了半个时辰,还派人去翻过架阁库,拿了几道卷宗来,才出了这道让章惇都想骂人的申论题。

他看了看王安石,发现那位老人正盯着面前的题卷全神贯注。

果然如此!章惇苦笑了一下。

考题内没有半点涉及气学,评卷时的那点新玩意儿,跟考生无关,只与考官有关,而且也只有几个数字的关联。

可即便韩冈在考题中再牵扯一点气学的内容,王安石现在恐怕都没心情与他计较了。大宋的平章军国重事现在正盯着韩冈出的第二题。

就是宰相都做不好。

实在太难为人了。

“没人能入第一等、甚至第二等,第三等都难,能归入第四等已是万幸,大部分人别想把这一题写好。”章惇低声对韩冈道,“玉昆,你是不是把题目拿错了?这应该是制科的考题!”
