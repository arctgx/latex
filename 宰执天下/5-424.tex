\section{第36章 沧浪歌罢濯尘缨(26)}

蔡确依然留在宫中,临放衙时,他被皇后派来的中使给拦住了,说是皇后要见他。

蔡确欣然而往,只是他没想到,担任参知政事的曾布也被留了下来。

蔡确有些纳闷。皇后想要说什么保密话,或是私下里征询意见,只要留下一人就够了。同时留下两名并不和睦的宰辅,让蔡确想不透皇后到底要说什么?

瞥了曾布一眼,蔡确心道:留下的人也太多了。

不过说起来,也是两府的人太多了。等到吕惠卿和韩冈回来,人会显得更多。

两府已经都满员,如同被塞满的书箱,连一只笔都放不进去了。

正常情况下,皇帝肯定要唆使御史台为他分忧解难,可如今的皇后没有这个手腕。

到时候,皇后有没有清洗两府、给人腾位子的魄力?这可难说得很。

章献明肃刘后垂帘,曾下诏命重臣将家中子侄的名字呈上来,她将视情况重用。等名单一个个送上来后,刘皇后却翻了脸。只要是上了名单的人,就一个不用。

不用宰辅私亲,这当然是好事。可哪有这么玩的?这一下可是把宰辅重臣都得罪惨了。

还有这一回,司马光又是怎么灰头土脸的回洛阳的?

女人的心思不要猜。

蔡确有着切身体会,家中妻妾的心思都捉摸不透,皇后的心思怎么猜得透?

但也不需要他去猜了,皇后开口便是要议论的话题:“和议已经定下,连来自各路的援军都纷纷回返本镇。说起来韩、吕两枢密也都该让他们回来了。”

用眼角余光瞅瞅曾布,蔡确明白了,为什么皇后留自己和曾布下来。宰辅之中,坚持不让吕、韩二人回京的,除了王安石以外,就是自己和曾布了。

王安石不想女婿回来倡议气学、扰乱政局;蔡确只想挡住吕惠卿,免得他也升任宰相;而曾布则是韩冈、吕惠卿都想挡住。

曾布上前一步:“殿下,臣等只怕会辽人那边多生波折。”

“当初王韶活捉了吐蕃人,官家连夜亲笔草诏,要他回京。现在两位枢密的功劳比王韶当年大得多了,你们却说会伤了辽国的颜面,让和议再生波折。”

蔡确低着头。主持朝政半年多,原本生涩的皇后现在的确是不太容易糊弄了。

——只是不太容易。

曾布辩道:“镇守三路的主帅不在,臣等更怕辽国又起异心。”

“那就先让韩枢密回来。吕枢密去河东代替他,陕西交给郭逵。”

这个……皇后究竟是突发奇想,还是早有定计?

“但这对吕惠卿未免过苛,其恢复兴灵之功,不在韩冈退敌之下。”曾布只稍稍一顿,立刻又找到了借口:“如今迫于局势,吕惠卿与韩冈不得不在外稍留,以安地方。两人一为宣抚使、一为制置使,皆是非常之任,临危而授。如今和议已定,若是将吕惠卿调任河东,又何来临危之说?河东宣抚既不可授,难道任其为经略不成,这岂不是形同贬责?!且朝廷用人岂是儿戏,数日一变,让世人如何看待?”

屏风后的声音断了,似乎是向皇后给驳得说不出话来。

蔡确轻声一叹,也仅仅是不容易罢了。

现在他对吕、韩的态度,也只是拖一拖,拖个半年,人心定了,回来也容易打发出去。何况两人在陕西,河东,要挑出错来也简单。

饼就这么大,嘴多了两张,每人分到的可就少了。

而且朝野内外都公认的,吕、韩皆是开国以来数得着的能吏。一旦他们进入中枢,参与朝政,除了王安石、章敦、薛向几人,其他还不得都给挤到一边站去?

王安石当年以一参知政事,让两相两参都靠边站,弄出来个生老病死苦的笑话来,韩冈和吕惠卿说不定也能做一做。

只是皇后要调回韩冈的心思越来越迫切,像曾布这样硬拦着,究竟还能拦到何时?

蔡确看着曾布,摇了摇头。曾布在政事堂中就是爹不亲娘不爱,被王安石生生压着,倒是变成了倔驴的脾气。

他起身,向屏风后行了一礼:“殿下。臣有一言。”

“相公请讲。”

“吕惠卿功高,当授节以开府。而郭逵在河北,领军日久,不宜再留居大名,当迁。”

给吕惠卿一个开府仪同三司的名号,让他去做北京留守。郭逵加个节度使去河东好了。至于韩冈,当然就可以回来了。

“此事再议!”屏风后的声音饱含怒气。

蔡确愣了,难道自己说得还是太隐晦了?正要继续开口解释,曾布却快了一步。

“殿下。蔡相公之言,正合臣之心意。郭逵久在河北,军心归附,又不擅政事,当先行调回,授以节度之位,另择贤能以守大名。至于吕惠卿、韩冈,亦当厚加封赠,以安其心。”

屏风后的声音变得更加冷硬,“时候不早,吾也累了。相公,参政,你们且先退下吧!”

一阵环佩响,皇后竟是先行离开。

蔡确缓缓的转过身,死死盯着神色冷漠的曾布,视线似是要把他扎透一般,许久,化为一笑:

“这一回,可是多劳子宣了。”

……………………

时隔多日,折可大又回到了代州城。

前一次回代州没能见到韩冈,折可大正犹豫着是赶去瓶形寨,还是等着韩冈从瓶形寨回来——他要面禀韩冈的也不是什么要事急务,他的父亲早就写了公文用马递送往制置使司衙门了——可没两天就被田腴请去忻州城,去接收一批返回代州的流民。

“雁门县衙中六曹八班到处都缺人,实在抽不出人手。还有啊……那群石头里都要攥出油的奸胥滑吏,也靠不住。万一惹出事来,就是杀了他们头,枢密脸上也不好看。”

田腴当日就这么在折可大面前叹着气。折可大抹不过情面,点头答应帮了这个忙。

将两千多人,总共八百余户百姓陆陆续续安排坐上有轨马车,一路送到了代州城,一通忙活的折可大才得以拖着步子走进代州的西城门。

“小乙哥。”

不知有谁在街上喊着人。

折可大望着前面,沿着西大街一直往前,到了谯楼再向北走百步,便是州衙的所在地。前日在忻口寨就听说韩冈已经回到了代州城,可以直接往州衙去了。

“小乙哥!”

城门口人有些多,虽说才过去几日,但眼瞅着代州城的元气好像又恢复了一点。大街两边的店铺也有好几家开张了,只是些茶肆、食铺之流的小门面,也不知是不是原来的铺子——折可大估计肯定不是原主,代州西门大街这么繁华的地段,要不开些收益高的店面,根本赚不回租金来,而且有好几家门头上的匾额也对不上——可看起来就是有了些人气,不复之前的萧瑟零丁。

“小乙哥!”

折可大继续随着人流向前走,想早些赶去州衙,然后可以回去好生睡上一觉。突然间袖子就被人扯住了,耳边又是一声喊,折可大回头一看,才反应过来就是在叫自己。

那是他所认识的人,韩冈重用的秦琬的弟弟秦玑,平日里寻常见的,只是招呼自己时的称呼不对。

“小乙哥?”折可大一头雾水。

虽然在府州时,也有人这么称呼自己,但那也是少年时的记忆了。折可大可从来没有想到秦玑会用这般亲近的称呼。

只是当他茫然环顾左右,秦玑是不是在招呼另一个人的时候,却发现他要拜会的对象,正一身儒士青衫的坐在路边的店铺里。

一条布幡从铺中探出,看招牌是个卖冷淘的小店。不过再看看门额上的善庆堂三个字,以及铺中的摆设,倒是不难看出又是一个鸠占鹊巢的路边摊。

三张方桌,十二只小凳,就摆在店中靠门处。做面的摊子则在里面一点的内门外。一个五十多岁的老汉在摊子上忙碌着,旁边有一个八九岁的小丫头帮忙打下手。而韩冈坐得四平八稳,等着上菜的模样。

“枢…”

乍看见韩冈就坐在路边摊子上,折可大惊出了一身汗,刚开口就听得背后一声咳嗽。折可大不愧是折家家主的继承人,反应倒是很快,脸上挤出一丝僵硬的笑容,上前问好:

“数日不见,可还安好?”

他不知道现在该怎么称呼韩冈,只能含糊过去。

“小乙,过来坐。”

韩冈很大方指了指桌子侧面的凳子,示意折可大坐过来。

折可大斜着身子坐了下来,只挨了半边凳子。凑近了,他低声道:“枢密怎么就这么出来了?”

“代州西门口的陈冷淘可是有名的。面好,酱料也好。”韩冈又指了指陪着坐下来的秦玑,“秦二昨天吃过了,回来说好。今天左右闲得无事,就出来尝个鲜。”

‘闲得无事?’

折可大张了张嘴,想扯着满是灰土的衣襟说一说自己都忙得脚不沾地了,可再一想,还是乖乖的断了这个念头。

韩冈是不管民事的制置使,推荐贤才,安定郡国,那是他两府中人的权力。但州县中人事已定,再要插手地方事务,就说不过去了。所以他现在的确闲。至于自己,小虾米一样,倒不用担心会给人揪出来找茬。
