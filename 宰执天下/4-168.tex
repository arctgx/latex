\section{第25章 山水留连住多时(下)}

尽管米彧是知道章恂往交州来的,在泉州的时候还是亲眼看着他上船。但米彧只当自己什么都没看到,并没打算凭着同乡的身份,去接触章惇的弟弟章恂。

认了两家主人的狗,最后不是被赶出家门,就是上了餐桌。既然傍上了韩家,就要做一条诚实、忠心的好狗。

虽然王清脸上的表情,对米彧的表态看不出有何反应,但他的话似乎也变得热情了一点,米彧不知是不是自己的错觉。

“章家的十一公子,就是章家为了开设商行而来,接下来,章家也就要在交州插上一脚了。”

“不知章七相公家的商行打算在交州做什么营生,贩些什么货物?”米彧在泉州的石渚港看到章家的十一衙内上船南行后,在船上的时候便一直在揣测着,只是越想越是心惊胆跳,最后甚至是不敢再多想了。

“听说是什么都做。”

“什么都做?!”米彧脸色变了。不是他害怕听到的那句话,但结果都差不多。既然什么都做,那第一个不会放过的就是赚钱最为丰厚的香药贸易。

“放心!”王清笑着,“想必米二哥也应该知道,章七相公所在的莆田章家,在福建是赫赫有名的大族。前面出了个宰相,眼下章七相公也离宰相不远了。名门望族,做事那也是有讲究的,不会什么都一口给吞掉。”

从专为轨道而开的侧门进入交州城,王清和米彧一起下了车,两人的几个伴当就从车外搭脚的地方跳下来。

轨道线路设计时就埋下了伏笔,正好从顺丰行交州分号的大门前不远处经过,只要下了车,走上两步,就到了新近修起的顺丰行的分号。

顺丰行交州分号的占地并不大,因为商行的库房都集中在城东,占用地皮最多的建筑既然不在此处,围墙括起来的地面当然也就用不着太大。

新修起来的建筑自然到处都是簇新的。屋顶上的黑瓦,地面上的方砖,不像旧屋一般,有着青苔甚至青草。梁柱上都刚刚抹过了一层漆,不见一丝斑驳的痕迹,只是还没有完全变色。王清安排给米彧主仆的分了内外间的客房中,刷白了的粉壁看上去也尚未全然干透——毕竟仍是雨季,而且还是雨季中,雨水最多的时节——而挂在墙上的一幅山水,则也是墨迹新干的样子。

换下行装,借着王清使人送来的热水,米彧经过一番梳洗过后,整个人变得神清气爽,长途旅行的疲惫一扫而空。

王清此时派人来请,看看时间,离着饭点还远,接风洗尘的宴席自然还要过上一阵。他知道是王清为何事,便跟随派来请人的婢女,往去见客的偏厅过来。

王清就在厅外相候,也换了一身衣服。进了厅中,王清、米彧谦让了一番,分了宾主,在两张白木交椅上对坐着歇下来,两个相貌不俗的交趾女上来奉了茶,躬身出厅。

茶水就在手边,要商谈的对象则坐在对面。将茶盏举起来象征性的抿了一口,闲话到此为止,也就该说正事了。

王清迅速进入正题:“米二哥你的打算小弟是知道的,交趾的香药生意有多赚钱更是一清二楚。只是鄙号在此处人手不多,并不打算沾手此事。”

米彧心头一跳,顺丰行躲着香药贸易走,难道是不打算跟章家竞争,但王清接下来的话,让他松了口气,“所以鄙号打算入股米二哥的商号。以米二哥你为主,鄙号也就是做个跑腿的,占三成的股,平日派个人在行中查账就行了,一切还是米二哥你说了算。”

这是交换条件,而且优厚得让米彧难以想象,以他的计算,就算自己只占三成都是不会亏本的。而米彧一开始的打算,是六|四开,顺丰行六,而自己则只占四成就够了。

占了一条稳赚不亏的商路,就算欠着再多的钱,也没人担心会还不起,就算想要再借上一笔,只要能还清利息,都是很容易的。

既然顺丰行这般大方,米彧当然的有所回报,“三成干股,在下立刻就奉上。王兄弟可以放心,每年的红利在下肯定会按时送到。”

“不,不是干股。该出的钱,鄙号一份都不会少出。顺丰行只用了七年便发展成如此规模,天南地北都有分号,靠得就是信义二字,不会贪图不义之财。”王清变得很是严肃。

米彧心神一凛,连连点头称是。想来也该是这样,通过在京城了解到了冯从义的秉性之后,米彧才准备将接下来的注意力都放在交州的香药贸易上,要是换了别家,比如准备‘什么都做’的章家,他就只会选择设法捞上一票,而不是眼下的长期买卖。

王清代表开出的条件如此优厚,米彧完全没有拒绝的道理,很干脆的点头答应,找了个中人,将合约给定下。

在合约上签名画押,打上印模,王清微微笑了起来,漫不经意的随口说道,“若是米二哥你手上周转不开,鄙号也能出借一部分的,依着便民贷的利息就可以,”

“多谢王兄弟,不过在下还是有些闲钱的。”

米彧对此婉言谢绝,他另可借用高利贷。如果所有的本钱都从顺丰行借,到最后就会变成一个跑腿的掌柜而已,只有自己真金白银的出了钱,才是有资格与顺丰行一起做买卖,至少不用担心被吞掉。

将香药贸易的事敲定,王清也算是了结了一个任务。

在顺丰行的生意中,都是以各大工坊的出产为基础,而不是单纯的转运贩卖,靠差价赚钱。不过若有合适的机会——比如香药贸易——也不会全然放过,通过一笔笔投资,将别家的商行拉过来,依靠股份分润红利。与顺丰行合作的商人,米彧不是第一个,也不是最后一个。

这样做生意的手段,其实并不是赚钱最多最快的招数。放弃能赚大钱的生意,而与多家商会来往,王清作为大掌柜,多多少少也能猜得到,藏在背后应该有个更大目标。

但王清始终想不明白,更看不透。但他也没有多想,站起身,邀请米彧一同赴宴,不但是接风洗尘,也是为了两家的合作而庆贺。

……………………

韩冈刚刚抵达钦州。

这半年来,他主要还是在沿海的钦州、廉州、交州加上邕州四地之间频繁,间中只回了桂州一次,这是为了检查路中各州的财政情况,以便向京城汇报。

看了广西各军州的账本,韩冈忍不住要叹气,广西果然还是穷。他在开封担任提举诸县镇公事的时候,看到的账本,无论支出还是开支,都是要比广西的财计簿桑上的数字,要长出一截来。

算起库中的积存,这边除了桂州、邕州,多半是几百几千而已,而京城,随便一个县都是几万十几万。这还是大部钱粮,都往开封城中的几大库汇聚的结果——当真是不能比。

钦州的账簿就放在韩冈的桌案上。前后对照了,发现没有错,便将手上的旧簿和新账一合。顿时就是一团积灰喷了出来,让韩冈连着呛咳了几声。

“龙图。”服侍在身边的伴当关切的问着,上来要帮着捶背。

“没事!”韩冈摇摇头,让他去打开窗子。一股清新的风吹了进来,感觉就好了不少。

不再是龙学,而是龙图。

尽管比不上章惇跃入西府的光荣,但一个龙图阁学士的头衔,也足以让朝中九成九的文官羡慕和嫉妒了。要知道,就连做到枢密副使的包拯都不是龙图阁学士,后世所说的包龙图,根本是以讹传讹的结果。

大宋虽说是重文轻武,但军功的封赏,远远重过其他的功劳。治政即便再好,每一任的考绩都能拿到上上,四课二十七最一应俱全,也比不上一次斩首过千的大捷。

韩冈经手的是灭国之战,作为副帅,而且还拥有临阵指挥之功,他分到的功劳,并不比章惇那位主帅少多少。

但这一切都已过去,眼下更重要的是发展生产。

交州如今号称七十二部,实际上总计七十四个羁縻州。大大小小几十家的部族将富良江两岸的平原瓜分殆尽。

但他们并不擅长农业生产,所以韩冈的手段就是仿效熙河路的做法,从军中和治下,选取精通农事的士兵和百姓,将他们聘为农官,去指点蛮部的农业生产。

如果一干溪洞山蛮是要自己种地,他们多半都不会太尽力,但现在有交趾人代劳,当然不介意多费一点驱使时挥鞭子的力气。

为了保证地力,交州各部的土地都是采取轮作制,但出产不会少——光是之前半年,各部急就章的种植,生产的粮食已经足够他们近一年的食用——此外甘蔗的大规模种植,也都在酝酿之中。

只要再有一年时间,交州就能有个大变样,三五年后,就是堪比熙河路的富庶边州。

但韩冈心中忽然有了点惆怅。

放下父母、丢下妻儿的生活,难道还要再过上一年?身在南疆两年,间中只有匆匆一会,他也当真是想家了。

