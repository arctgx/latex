\section{第32章 吴钩终用笑冯唐(二)}

吴逵走在咸阳县的城头上。

城墙南面不到一里的地方便是滔滔渭水。桃花汛此时已经到了尾声,原本淹没了大半的河滩,现在也渐渐露出了河面。不过再浅的渭水,也必须要通过舟筏才能渡过。即便渡过之后,对面还有一营官军驻扎守卫。而除了南面之外,咸阳城其他的几个方向都驻扎有大军。一面河水,三面敌军,被困于城中的吴逵,已是插翅难飞。

是的,现在吴逵和他所率领的三千叛军,眼下便被重重围困在咸阳城中。

位于渭水北岸的咸阳县,是长安京兆府的北大门。在过去党项骑兵肆虐关西的几十年里,大宋君臣不得不考虑战局出现最坏的情况——也就是党项大军冲破缘边四路的阻截,直奔长安而来的情况。当北方防线被突破的时候,为了保护长安城的安全,计划中就是要以咸阳城为核心在渭水北岸展开防守。

尽管在这几十年里,纵使是李元昊纵横南北、大宋军几次三番全军覆没的时代,铁鹞子也从来没有在陕西穿越缘边四路的防线。可是为了抵御可能存在的危险,咸阳城依然是按照边境州城的形制进行修造。城高濠深,战具充足,武库、粮囤皆尽完备,远非普通县城可比。

但是,就是因为从未有过党项兵锋直指长安的危险局面,一直都是安定祥和的咸阳,也便在持久的和平时光里,失去了防备危险的意识和手段。当吴逵率领三千叛军沿着渭水东行的时候,不费吹灰之力的就攻进了咸阳城中。

以吴逵的本意,在咸阳城内补充过必要的粮草军需之后,便要渡河南去,在长安边上晃上一晃,逼迫官军进驻长安城后守卫,便设法跳出包围圈,照计划往秦岭进发。

可吴逵万万没有想到,燕达所率领的秦凤军的速度竟然会有那么快。燕达先是在西面的兴平,堵住了他第一次强渡渭水的尝试。而当他领军纵马奔驰,一举攻下咸阳后,再次准备渡河的时候,燕达的将旗也再次出现在渭水的南岸,在‘燕’字将旗的背后,是浩浩荡荡、几近万人的秦凤大军。而一直紧随在吴逵身后的泾原军,也追着他的脚步,一路跟到了咸阳城外。

两路合力,加之咸阳附近诸县的守军,一起将咸阳城包围了起来。

值得庆幸的是,这几路大军的行动不知为何,突然间变得愚蠢了起来。在他们还未做好任何准备的情况下,便展开了攻城行动。不但不讲究着围三阙一,把几座城门围得死紧,而在在攻城时,连城外紧贴城墙的民居也不做清理,完全没有拆毁、焚烧的行动。在吴逵的观察中,统领这几路围城大军的主帅,甚至不允许麾下将士们随意进入民居之中。

从仁德爱民的角度来看,官军的做法当然没有问题,而且值得褒奖。但从军事上看,却没有比着更蠢的做法了。咸阳紧靠长安,富庶无比,居住在城外的百姓有近万家。屋舍鳞次栉比,丝毫不逊于城中。如此密集的屋舍使得除了长梯以外的任何攻城器具抵达城下,这样如何能发挥城外官军人数上的优势?而且住在城下宅院中的百姓早跑光了,攻城的士兵在这些空屋中举着云梯来攻城,这不是引诱他吴逵防火吗?

吴逵当真放了火,在春雨贵如油的地区,又是春风正好的时间,一场火下来,攻到城下的官军不知死了多少。吴逵只知道那股子皮肉烧焦的恶臭味,在咸阳城里飘了三天才逐渐消失。

一场大火下来,围城的几路大军损失极重,包围圈也变得一戳即破的纸页一样脆弱。在这个时候,正是突围的好时机,但城中的叛军没有行动。

在邠州城外,因为官军的伏击而失去了得力亲信解吉的情况下,吴逵对叛军的控制降了一个台阶。一把火而轻而易举得来的胜利,使得下面的将校对城外的官军失去了畏惧。反而有空坐下来争吵,对接下来的逃亡方向也发生了分歧。只是这一番的争执,便让叛军失去了逃离咸阳的最后的机会。

才过了一天,秦凤军重新包围了上来,泾原军也包围了上来,名帅郭逵的将旗随着来自长安的永兴军路的两万大军抵达了。他们再次堵上了咸阳城外的城门。而刚刚被征发起来的民伕,则开始挖坑,拼命的挖坑,围着咸阳城墙,挖了一圈坑。所取出的土,也变成了一道一丈高的围墙,围着咸阳城绕了一圈。

此时的吴逵已然明了,他只有在等死和自尽之间选择道路。

不过再没看到仇人授首的情况下,他是怎么也不会甘心就死!

……………………

郭逵回长安了,不忘顺便带走他的将旗。

燕达知道,他的这位恩主是被赵瞻气走的。

郭逵在枢密院做过一任同知枢密院事,是天下有数的名将。虽然因为与韩绛不合,而不得不被调往秦州,但在叛军祸乱关中的情况下,天子和朝堂第一个想起的定海神针依然是他。

但就是这样的郭逵,还是被赵瞻气得发昏十三章,两人为了如何攻破咸阳城吵了七八天。最后,郭逵被比茅坑里的石头还要臭硬三分的赵瞻气得不行,转头调脸便离开了前线。

望着不断增高的墙体,望着不断加深的濠沟,燕达怎么也想不通,赵瞻的行事风格,为何会因为一场失败,而从极端激进偏移到极端保守。就像放在桌上的一枚铜板,突然自行跳起,从有字的背面,变成了带花的正面,让人感觉十分的突兀。

赵瞻当初匆匆抵达围城军中,先是夺走了燕达的指挥权,接着以攻其不备的名义,逼迫大军仓促攻城,然后八百多将士就被烧成的黑炭,还有同样数目的士兵受到了火伤。

有过这一次的挫折,当郭逵领军抵达咸阳城外后,再次聚众商议该如何解决咸阳城中的贼军,赵瞻便全然反对郭逵的破城计划,要求周围地方征发民伕,让他们挖坑夯土,围着咸阳修出一道外城来,还找了跟在他身边的叫赵什么雄的门客来指挥民伕来挖坑。

这是当年贝州王则之乱时官军所用的战法,以耗费时间、财税、人力无数而闻名天下,可赵瞻偏偏就是要这么做,美其名曰,不让一名叛贼逃脱。

不过除了被正主盘踞的咸阳,其他地方上的情况却是好了。

都已是三月上旬快结束的时候了,各州各县想乘着广锐军兵变的机会、跳出来搅混水的贼人,已经是被杀的被杀,被捉的被捉,剿得差不多了。现在只要在关中道上行走,不论顺着哪条路,经过的城池外墙上,都能看到一颗颗用鸟笼子装的脑袋,排得一溜整齐。

在招捉使燕达的命令下,这一放手狠杀,使得关中的风气顿时好了不少。许多积年老匪,都在这一波官府有志一同的行动中,变成了刀下之鬼,连个喊冤的地方都没有。当然,其中也有风声鹤唳的情况,错杀误杀也不在少数,

但话说到底,燕达好歹是把他的任务完成了,将战乱的危险压缩到只剩咸阳一地。这样的情况下,燕达的这个招捉使的职司也不能算是失职,至少到现在为止,东京城还没有送来走马换将的诏令。

只是现在燕达也无力进攻,当日冲在最前面的几路精锐,泰半折损在火海之中。有胆气、有实力的将校被付之一炬——连同他们麾下的士卒一起。一把火差点打折秦凤和泾原两军的脊梁骨,好脾气的燕达,都差点要让人把赵瞻绑起来丢进渭水。

要不是身上背了一个招捉使的名头,他也想走了,赵瞻的脾气当真不是让人能清净下来、好好做事。想到这里的,燕达不由得羡慕起郭逵来。作为取代了司马光,成为镇守京兆府的主帅,他真的只要坐镇长安就够了,并不需要上前线,赵瞻想找他麻烦都难。

游师雄这时走到燕达的身边,他在邠州城外的表现杰出,先是得到了赵瞻的看重,但因为军事上的议论不合,又被赵瞻所疏远。但燕达并没有挑挑拣拣,直接把游师雄给用了起来,任命他为自己身边的幕僚。

从游师雄手上接过的公文并不简单,上面是这段时间以来,所消耗的物资和钱粮的数量,实在是让人吃惊。比起罗兀城的消耗,这几日围城军的使用,也不输多少了。

‘真不知什么时候才能攻破咸阳?’燕达摇摇头,换了个想法:‘究竟什么时候延州那里的援军才能到?!’

罗兀守军顺利撤离的消息,连同大破西贼的捷报,早已送到手上,燕达知道原本被困在前线的数万精锐,现在正在兼程而来。但究竟会是什么时候才能到,燕达也不清楚,只知道大概就是在这几天的功夫了。。

正这么想着,快马来报,南来的援军已经到了十几里外了。

