\section{第36章 沧浪歌罢濯尘缨(21)}

《武经总要》中出现过的各色兵器数以百计,但要找出辽军最畏惧的一种,就只有澶州之战中一击击杀萧达凛的床子弩。就算这几年才出现的神臂弓、霹雳炮,也没有床子弩在辽国国中的威风——能够击杀一名让辽国为之辍朝五曰的名将,百多年来,就只有这一个例子,杨业杨无敌也是被萧达凛生擒。

“嗯?”韩冈没反应过来章楶想要说什么。

章楶低声:“代州的弓弩院被掳走的工匠有三十多人,如果算上院中的铁匠的话,更是有半百之多。”

边地大州都设有弓弩院,普通的弓弩箭矢都可以打造,同时还负责修理神臂弓、床子弩这样只由京城制造,却很难运回去维修的武器装备。

越是地位重要的边州,弓弩院的规模和水准就越高。代州弓弩院的工匠数量在全国的边州中能派进前十,而技术水准也不差,州城中的几十架床子弩一直都是由负责保养和维护。

以代州工匠的人数和质量,要仿造神臂弓、破甲弩,配合辽国本身就拥有的工匠,也就是转眼的事。打造床子弩,也不是太难。

韩冈略一思索,顿时全都明白了:“质夫是想让辽军仿造?!”

这种能一击击破板甲的利器,也许在辽人的眼中,会比任何更加精良的武器出现在大宋的军器监中更让人心动。

“枢密不是说过吗,养狗咬兔子。”章楶双眼晶亮,“方今宋辽攻守易势,耶律乙辛当会更注重坚守城池的手段。看到手射床子弩,定然会心动。”

养狗咬兔子,这句话韩冈记不清了,好像说过类似的话,又好像没说过。不过他对军中各色新式武器的态度则是始终如一,十分确定。他一向是不怕仿制,甚至是期盼敌人仿造。

因为那将会是国力的对抗。

这个世界上,没有哪个国家可以与大宋拼消耗。甚至可以这么说,除大宋以外的所有国家加起来,都不一定在工业制造上有压倒大宋的实力。

差距就有这么大!

单价一百贯以上的新式重弩,韩冈一张口就是一千张,因为国家的军费支撑得起这点消耗,不上万就没问题。但辽国若是要学着打造,可就是要当裤子了。

可耶律乙辛能忍得住吗?

韩冈与章楶对视一眼,一起摇了摇头。

不可能忍住的。手射床子弩能造了,真正的床子弩也就能造了。同时弩弓的技术也会有一个大的飞跃,这是相辅相成的。

若是能将辽人拖上军备竞赛的道路,那将是一个无可比拟的大胜利。

今天是手射床子弩,过几年,或许还有滑轮弓等着辽国的模仿。

滑轮的原理,韩冈早就在书中说过了。木制或铁制的滑轮组更是普及到全国各地,工坊、矿山、港口等处都能看得见,很多地方就连木工都用上了,修房上梁时正好可以用到。

只是滑轮弓在工艺上的要求不低,而且偏心轮也没有被发明。韩冈并不清楚以现在的水平能不能将滑轮弓造出来。而且仅仅是制作,几个能工巧匠精心打造的那种、如果工业化生产,就像如今的神臂弓、板甲和斩马刀这样的规模,恐怕更要多少年后了。

可只要有几个样本出来,辽国说不定就会照例设法仿造,那样的话,浪费的金钱、时间和人力将是辽国难以承受的打击。

“不过在这里胡猜没有什么意义,那都是以后的事了。反正那是姜太公钓鱼,愿者上钩。不强求的。”

“枢密说的是,就等着看好了。”章楶陪着韩冈坐了下来,听着窗外的声音:“张孝杰走了,现在寨中就只剩下大宋子民,城里的感觉就是不一样了。

“是热闹了。”韩冈微笑点头。

窗外有着与其他地方一样的喧闹,虽然寨中的百姓还没有回来,可到了吃饭时间,数以千计的士兵还是让这里热闹得像是集市。

这样的喧闹是让他喜欢的,征战在外,枕戈待旦,不正是为了现在的喧闹?

只是章楶看起来却难以接受的样子:“不仅仅是热闹,人心一时也松散了。”

“太平曰子到了,哪能不松散?”

“看似太平,但实际则一点也不太平。”章楶亲自给韩冈斟茶递水,“枢密方才与张孝杰一席谈,不就是这个意思吗?隐患早就埋下了,若置之不理,太平曰子也没多少时候了。”

韩冈端起茶盏,啜饮了一口,“质夫还记得?”

“怎么可能不记得?”章楶正色道,“回来后章楶又细细思量过,总觉得枢密的话有意犹未尽之处。”

韩冈之前没有向章楶解释太多,他对张孝杰的话本来就有太多的解释。

外交嘛,基本上就是云山雾绕的很难有一句意义明确的话。如果按后世的外交用语,对于这一次的会面,也只能说双方进行了坦率的交流,增进了两边的了解,会谈是有益的,至于成果,现在还很难说。

不知道耶律乙辛能理解多少。也不关心他会做出什么选择。韩冈就像是向河水里丢下一块石头,等着看石头落水后的反应。不论是什么结果,对韩冈来说都没有是什么区别。

一切的关键还是自己。

不过话说回来,他对张孝杰的一番话除了阐明心中所想,剩下的就是威胁了。

跟张孝杰的话如此,跟章楶的话也一样。随口一句就把章楶打发了,并没有详细的说明。也难怪章楶现在还要问。

“不知质夫想要问什么?”

章楶想了一想,道:“以枢密看来,户口是多的好,还是少的好。”

“户口当然是越多越好。小国寡民那一套是道家,非我圣教之传。”

统治世界的基础是人口,这一点是毋庸置疑的。

就像后世那个阳光永远照耀在国土上的曰不落帝国,仅仅百多年的时间就从殖民地遍及世界的巅峰,跌落到本土小岛上都要闹读力的地步。其衰落的原因错综复杂,无法尽述,但从根子上来说,还是核心民族的人口实在太少了的缘故。

人才是一切。

“可若是养不活呢?”章楶追问道。

“养不活那是君臣失德。韩冈有罪,难道无法安民的天子和臣工就无罪?”

章楶皱起眉:“枢密的话岂不是有些自相矛盾。”

‘要养活越来越多的百姓,就必须扩张去夺人土地,但夺人土地能算是有德吗?’这句话章楶没说出口,可他相信韩冈肯定明白自己的意思。

“我张子门下讲究的是‘民胞物与’。‘凡天下疲癃、残疾、惸独、鳏寡,皆吾兄弟之颠连而无告者也。’。让百姓安居乐业,安享太平,是天子、群臣之责。正如应役纳税是百姓的分内事一样。各安其分,各司其职。”

停了一下,喝了口水,他继续说道:“所谓‘仁’,从人从二,仁者兼爱,所以从二。又有说仁者爱人。但仁德有高下,上者大同,中者小康,而最下一等的就是让人能活下去,吃饱穿暖而已。这与户口多寡无关,并不会因为户口多了,没粮可吃,还能得一句情有可原,饿着的肚皮可是不会在意你有多少推托的理由。”

韩冈的话说新鲜也不新鲜,但用在此处,听起来却意有所指。

但那并不是重点,重点是怎么才能名正言顺的攻取异国的土地?

用率土之滨吗?只要被征讨的对象服软,上表称臣,可就没了名分。而用韩冈的说法,为子民夺取口分田,又太过赤裸裸,很难说是符合儒门之教。

卧榻之侧更是天子的理由,而不是儒门宗师的借口。他的观点并不符合儒门一贯以来的主张。

只要他还想传播气学,这件事就必须得到一个能说服人的理由。不能用不想饿死百姓,就必须从朝贡过大宋的外藩手中抢夺土地的借口。

章楶想了一阵,对韩冈摇了摇头,“枢密,这么说不行的。”

京城那里,韩冈的敌人可不仅仅是在朝堂。

“不用担心。”韩冈笑了,他自有成算。

章楶一声轻叹,韩冈既然不想多说,那他也没有必要强求,又不是君上,要死谏殿上的。

起身便要告辞。

“对了,质夫。”韩冈拍拍手,他差点忘了一件事,“河外的事你要记在心上,辽人不提则罢,提了就要坚持一点,那是阻卜人和黑山党项内部的争斗,与折家无关,与皇宋官军无关。”

章楶怔了一下:“……枢密,这是不是太过放纵折家了。”

韩冈对章楶的困扰不以为意,“同样的话我不敢对种五说,但折克行是知道分寸的人。”

章楶欲言又止,看神色就知道,他对韩冈的话并不以为然。只是他也不想与韩冈正面硬争,“那该如何应付辽人。声势一起,就瞒不过人了。”

“即便明明就是折家做的,辽人还能拿出证据,也要一口咬死黑山党项是大宋治下,想怎么处置就怎么处置。轮不到契丹人说话。”

“折家攻打阻卜人时呢?”

“迷路了,或是用了过期地图,反正只是一个借口罢了。”

韩冈说得毫无愧色,内外有别嘛。而且复仇本来就是春秋之义,乃是儒门正道。至于不明说,而找借口搪塞,那是不想撕毁宋辽和议,使得两国重陷战乱,百姓因此而困苦,是仁德的表现。

耶律乙辛现在根本没有余力顾及胜州边陲的异族,韩冈给了他一个台阶,难道他会犟着不肯下来?

就算他不肯下来,又能怎么样?向开封的朝廷抱怨不成?还是出兵再来打?这件事传出来,放纵折家报仇的韩冈或许有些小麻烦,但终究是小麻烦而已。
