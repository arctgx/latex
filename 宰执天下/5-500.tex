\section{第39章 欲雨还晴咨明辅(29)}

刚刚在韩家宣读的诏书,两府几乎比韩冈本人还要早一步知道。

蔡确听说了,章惇同样也收到了消息。

“竟然是国公!”刚刚上京的章恺惊讶莫名,这未免升得太快了。

“韩冈又不会糊涂到当真接下来,不过是太上皇后不想看到有人以为他失势了。”章惇语气淡淡的。

章恺眼底泛起狐疑之色,太上皇后当年的一句‘依卿所奏’在地方上都是经典的笑话了,她真的能想到这么深?

章惇摇摇头,垂帘听政都这么长时间了,如此简单的御下之术自当熟练掌握了。

“但莱国公又是怎么回事?国号几十上百,选哪个不好?太上皇后若是真有想法,不应该不知道真宗时的事。”

“哦,这倒没什么。”章惇倒是不在意,太上皇后见识上有所欠缺,这是短时间内无法弥补的,不比心术手段,“只要不是宋国公就行。”

章恺苦笑,大宋的国号,怎么都不可能给他人用,“可莒国公都比莱国公好。毕竟寇莱公的事还在前面。”

“莱国是小国,莒国可不是。”章惇摇头,“太常礼院不会在礼仪上犯蠢。”

封国大小有等级,首封国公必小国,继而才会是中国、大国。太常礼院可以在细节上做些小动作,来玩什么一字褒贬,但他们不会蠢到将把柄送到他人手里面。

至于曾封莱国公的寇准,在真宗晚年,被卷进了周怀政谋图废刘皇后、尊真宗为太上皇、立时为太子的仁宗为帝的公案中,寇准由此被贬去了岭南雷州,并病逝在那里。韩冈刚刚参与了拥立太子一事,现在就送了个莱国公的封爵,这不免给人以遐想。

“说得也是。”章恺点点头,觉得在这件事上议论实在是闲得慌,正如他兄长说的,韩冈肯定不会接下来,“反正都是虚名。还是田宅在手更实在。”

“京中的庄子市面上可少见。”章惇道,“十五六顷田,除了皇亲国戚,外臣哪有机会拿到手?”

“也不值多少……”章恺笑了一下,“哥哥若是想要,小弟这就去想办法。不过十五六顷,有钱还能买不到?”

“算了。”章惇摇头,当初就是自家的这位兄弟买地连累了自己被罢职,好不容易才回来的,可不想再被捅篓子,“措大眼孔小,老七你觉得不算什么,我可是觉得不少了,真要去搜罗下来,可是觉得拿着烫手。”

“哥哥怎么妄自菲薄呢?”章恺叫道,“我们章家诗礼传家,累世簪缨,岂是寒门素户出身的那一干措大可比?!”

“韩玉昆也是寒门,他是吗?”

章恺给噎住了,干笑了几声,道:“……韩玉昆当然也不能算。天授之才,贵气在骨子里。”

韩冈当然不是措大。靠了韩冈拉着章家一起在交州发财,章恺才有今曰的豪气。

韩冈在交州的布局,章恺都看在眼里,随着交州的发展,他早就佩服得五体投地。就是昧着良心,也说不出韩冈的坏话。

章惇总觉得,在对钱财的态度上,韩冈与他贫寒的出身并不相称。说是骨子里就与寒门出身的不一样,那倒是没有错。

要知道,学问与贫富不一定有关,但养移体、居移气,生长的环境不可能对人没有影响。

冯京商人出身,纵然是解试第一、省试第一、殿试第一,连中三元。本身又是风采过人,有名的风流倜傥,但他对财货的爱好,却与他的出身完全相符。金毛鼠的绰号,从长安叫到了京城。

而福建出来的官员,只因为八山一水一分田的家乡环境,便喜欢于各处置地,吕惠卿还有蔡确,都是如此。

韩冈出身寒门,并无世勋,现在却是能聚财,也知道散财,只是这一点,就有很多人不及他。

顺丰行做得都是批发的买卖,民间知道的并不多,朝中消息灵通的人虽说知道一点,但也不清楚顺丰行的生意有多大,又如棉布这样的特产,因为出产的商家太多,韩家也并不那么显眼。就是御史台,眼睛都只放在韩家在巩州的三百顷田地上。

但章惇清楚,白糖、棉布,还有香精、玻璃这几个暴利的行业,虽说韩家带动了一大批商家将其发展起来,可钱财并没有少赚。

韩冈的家底也就也同样变成豪富的章惇更清楚一点。

谁娶了韩家的女儿,那就立刻翻了身。可惜早就跟王厚的儿子订了亲。

钱财多寡,其实还是为子孙考虑。到了他们这一步,也只有手中的权力最为重要。

韩冈放弃了枢密副使,吕嘉问就欺上头来,让他不得不出手反击。现在太常礼院又开始做手脚,背后是谁姑且不论,不知韩冈会怎么做。不依不饶的要个说法吗?

章惇忽的呵呵笑了两声,韩冈若是那么糊涂,之前在他手下吃过亏的那么多宰辅重臣,就未免输得太冤枉了。

……………………“怎么是莱国公啊……”

当着周南、素心、云娘,以及冯从义和满院的下人的面,王旖没说什么。但回到后院,私下里,王旖就长吁短叹起来。

韩冈知道王旖为什么心情沉重,曾经的一位莱国公下场的确不怎么样。只是他本人并不放在心上:“岳父可为荆国公生过气?”

“最后又没除授!”王旖当即反驳。

从舒国公转封荆国公,的确是晋封,可连在一起,却应了诗经之中‘戎狄是膺,荆舒是惩’这一句。不过在韩冈得知后,直接就捅了上去。所以王安石现在成了楚国公,由小国直升大国,算是天子给他的补偿。

“是啊。不正是还没有除授吗?”

“现在是官人没有接。”

“难道为夫以后会接吗?”韩冈笑问道。

被韩冈一句句堵回来,王旖愤愤然的狠狠瞪了他几眼,最后也只能放弃,“终究是晦气。”依然带着些许嫌恶的口气。

韩冈哈哈笑了起来:“为夫可从来不在乎晦气不晦气。哪家毛神敢犯到为夫头上?”

王旖没好气的又瞪了丈夫一眼,也就把这件事给放下了。

王旖回房了,还在院中的韩冈,脸上的笑容就收敛了起来。

不是什么大事,就跟一脚踩到狗屎一样,只是恶心人。

这太常礼院,是太久没收拾了吗?还是说……韩冈皱眉想了一下,同样在心中放下了,转去找冯从义。

真的不是什么大事……“哥哥怎么来了。”冯从义正在灯下看着账本,见韩冈到了,连忙起身。

韩冈打量着冯从义的房间,虽然布置得很用心,但以顺丰行东主的身份,还是朴素了点。

“住得还习惯?”韩冈问道,他是知道冯从义在京城的宅邸有多奢华,韩家这间御赐的宅子,可远远比不上了。

“哥哥说哪儿的话,早年能有块木板躺着就很舒坦了。”冯从义满不在意。真的要在乎吃穿享受,就不会全国各地到处跑了。在外面的享受,哪里都比不上家里面。

看得出表弟的回话出自真心,韩冈点点头,道:“刚才忘了问。襄州的事解决了吗?”

京城的商家,想要在襄州港口要一个仓库来作为中转地。由于数量不小,加上背景很深,所以惊动到了冯从义,今天亲自去与对方的后台商议。韩冈也知道这件事。

“都商量好了。”冯从义点头,“襄州的地皮不卖,但可以长租,五年一重签。不过仓库和轨道要这边先修起来。”

“商会里面其他人怎么想?”

“都交托给小弟了。之前也都说好了,就这么几条。钱是小事,重要的是将人拉进来。”

“所以不卖地?”

“当然!”冯从义点头笑,“卖了地可就没现在的好处了。”

只要地皮还在冯从义等雍秦商人手中,京城商人就仰仗他们,若是卖了地皮,曰后翻脸都不用顾忌什么。冯从义并没打算在这个买卖中赚钱,所以契约才是五年一重签,他只想将京城商家拉上来一起去外面合作,而不是仅仅将关系局限在在京城里面。

韩冈对冯从义的明智很满意,这并不是他教导的,而是冯从义自觉去这么做。

影响力比钱财更重要。

越到高层,影响力和控制力,就越比家产的多寡更重要。

比如后世的富豪榜。其中有些富豪,往往能影响整个世界的商业秩序,一句话就能飘红飘绿。而有些富豪,纵然家产不会输给前者,他们的影响力则只是局限于一国中的某个产业。这里面的差距可就大了。

冯从义现在的情况,一方面背后是有韩冈的支持,另一方面,他本人的才干也让他在商界中如鱼得水。最重要的,在韩冈的影响下,他也学会不以钱财为念,而用更为宽广的眼光来看待世界。

冯从义为代表的雍秦商人在关西、在京畿、在荆湖、在广南的影响力,已经超出了世人的想象。而冯从义和他背后的韩冈,在雍秦商人中的影响力,也只有内部人士才清楚。

