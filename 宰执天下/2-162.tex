\section{第29章 顿尘回首望天阙(14)}

【第一更】

在众多羡慕嫉妒的目光中,韩冈叩首谢恩,再拜起身。这一拜一起,周南便已是他韩家的人了。有天子口谕为凭,再没人能从中作梗。

传过圣谕,王中正也不再摆出肃穆严重的架子,转而笑着向韩冈道喜:“恭喜韩机宜。”

以他御药院勾当、带御器械的身份来给一个选人传递口谕,可以说是屈尊了,正常来说,一个小黄门足以。不过能跟韩冈这个正得圣眷的年轻官员结交,他这位宫中屈指可数的大貂珰倒也乐意跑一趟腿。

何况王中正还想着出外博一个军功。就算今次韩绛功成而夺占横山,以他对赵顼的了解,河湟那边照样有立功的机会——韩冈能得赐美人,明面上的理由,也是因为他在河湟上的功绩。

“多谢王都知。”韩冈回礼虽是谦抑,但姿态也是不卑不亢,“前次在秦州,韩冈已经承了王都知的人情,不成想今次又是王都知送来天子恩泽。”

说着他回头示意李小六把刚刚准备好的谢礼拿过来,承到王中正的眼前。

王中正倒也不客气,让随行的小黄门把韩冈的谢礼接下——中使就算到宰相家传谕,也照样拿好处的,也算是约定成俗的规矩——接着不无遗憾的说着:“今次还要赶回宫去缴旨,不能久留了。日后若有机会,当与韩机宜你多多亲近。”

韩冈送了王中正出去,回来后,原本因为王中正来传旨,而不得不远远避开的众多官员,便纷纷上来恭贺。能得天子亲口褒奖和赏赐,这是他们想都不敢想的宠遇。

韩冈虽然也是不停口的感谢天恩,心底却是在冷笑,这哪里是天子的恩宠?这是回报!是交换!

是他韩冈借助了时势,用对天子有利的条件,交换了周南回来。相对于天子赵顼的得利,周南其实无足轻重。不过换在韩冈的心中,周南的份量也抵得过赵颢离宫对赵顼的价值了。

这是等价交换,韩冈不觉得自己欠任何人的人情。就算是章惇在中间帮着出了死力,但他这一次率先上奏,在天子面前可是立了大功,日后的好处绝不会少。

靠着莫名的恩宠而得来的地位,从来都不会稳固,但通过利益交换而建立起来的关系,便很难动摇。看看章惇,以他今次可以想见的丰厚回报,两人的交情自当水涨船高。而韩冈一路过来,能得到赵顼、王安石还有王韶的看重,不是因为他所立下的累累功勋,为三人带来了庞大的收获,还会有什么理由。

向着韩冈说了许多恭喜的话,驿馆中的众多官吏们也都知情识趣的不再叨扰他。都知道,后面还有一个绝色佳人等着韩冈。

前面韩冈接旨的时候,早就有人把喜讯通报给周南。等韩冈回到小院的时候,就看见周南站在门内候着。就是流着眼泪,捂住嘴,竭力不让自己哭出声来,连墨文也是泪满双颊,一边给周南递着手巾,一边则拿着另一块手巾擦着自己的泪眼。

韩冈怜意大起,走过去,轻轻搂住艳冠群芳的花魁。接过墨文递过来的手巾,擦着她脸上的泪水:“别哭了,今天可是你我大喜的日子,要是给人看到你现在的样子,说不定还会以为你不愿意。”

周南仰着脸,任由韩冈擦净泪水。充溢在她心中,满是喜悦。就像一阵狂风,吹散心头的阴霾,阳光洒落,让她不由自主的喜极而泣。

早已是处在绝望中的周南,怎么也想不到情郎进京也不过数日时间,竟然轻轻松松的就将自己救出苦海。虽是像物品一般,被天子赐于韩冈。可就是有了这道御赐的金身,就更是让她安心。这等翻手为云,覆手为雨的手段,使得周南对韩冈的感情中,又添了许多崇拜。

模糊的泪眼中,韩冈英气勃勃的相貌,越发得显得坚毅。周南痴痴的望着。哪个女子不希望自己毕生所托的良人,能够顶天立地,为自己挡风遮雨。她从小就看着教坊司中的那些姐姐,虽然在韶华正茂时被万众追捧,但最后能有好结果的却十中无一。始终萦绕在胸口的那种不知今生所托何方的茫然,直在韩冈身边时,才烟消云散。

韩冈搂着周南往里走,“如今得了天子亲许,南娘你就是我韩家的人了。不过现在京师,我又要赶着去延州,不能风风光光的纳你进门。若是不嫌仓促的话,今天就把好事办了。南娘你看如何?”

能早一点成为韩家的人,周南哪有什么不愿,自是千肯万肯。点着头,泪水又不知不觉的流了出来。

……………………

就在韩冈低声安慰着周南的时候,王中正已经回到宫中。

赵顼刚刚结束每日惯例的崇政殿议事,从西面传来的军情中抬起头来,问着王中正:“韩冈怎么说的?”

“官家重恩,韩冈当然是感激涕零,直说要鞠躬尽瘁以报天恩。”王中正说着赵顼爱听的话,又奉承着笑道:“官家既然已经将周南赐予韩冈,当是事了风息,也就不会再有人说二大王什么不是,太后那边也可以安心了。”

“……”赵顼一下沉默了下去,半刻过后,才点了点头,犹有深意的叹道:“但愿如此。”

天子直接插手的事,当然不可能这么容易就平息。赵顼心里也明白,他今次是风助火势,把暗中传播的轶事拉到了台面上来,亲自证实了传言的真实性。

这种情况下,御史台很快就会有反应。那些御史寻人弹劾,鸡蛋里还要挑出骨头,他的两个弟弟成年后还住在宫中,本就是惹人议论,只是前次压制的效果还在,没有人敢提。但赵颢今次是因为官妓而声名远播,跟章惇有同样担心的,绝不会少。加之士林清议对赵颢本有意见,到时群臣有志一同的攻击,赵颢还想留在宫中,群臣也不会答应。

又长吁了口气。赵顼其实本来已经把章惇奏疏丢到了一边去,以他的本意,也不想与弟弟勾心斗角。但看到二弟赵颢站在太后那里,一种发自心底的危机感让他改变了一开始的想法。

终究还是没儿子的错!

赵顼心里暗叹,要是今次宫中的两个有妊的嫔妃,能为他、还有大宋诞下继承人,他也就可以安心了。

………………

天子亲自出手,把亲王和选人的花魁之争做了个了断。这一消息,不过半日的功夫,就已经传遍了东京城中。

快要做新郎的王旁,很快也听说了此事。他对着房中绣花的妹妹王旖道:“天子钦赐佳人,韩玉昆倒真是艳福不浅。不过这风流韵事传得沸沸扬扬,。”

“我倒觉得很好啊……世间又有几人敢不畏亲王权势的?”王家的二女儿在一块绸子上飞针走线,还不忘跟王旁说话,“换作是那等龌龊之辈,连妻女都能献上去,更不用说定情的官妓了。韩玉昆也真是不负任侠之名!”

“爹娘现在正在帮你找人家呢……除了不是进士,还有家世稍逊,论相貌、论人品、论才智,韩玉昆都是一等一的,挑不出毛病来。你要是觉得他好,我就帮你跟爹娘说去,赶明儿就把你嫁了。”

王旁半开玩笑半认真的向妹妹提议着,而王旖则很干脆的摇头,“爱沾花惹草的男人,我可不要,爹爹那样的才好!”

王旁愣着半晌,摇了摇头,只觉得自己一辈子都别想弄清自家妹妹的心中所想。他伸头看着王旖绣上,“绣得是什么?狗还是猫?”

王旖手上的针线活停了下来,“……是荷花!”

这幅荷花图,她绣了好几日,本是准备送给王旁的结婚礼物,却被说成是猫狗,她一赌气也不继续绣下去了:“还是等二嫂嫁过来后,让她帮二哥你绣吧!”

王旁在旁暗自窃笑。他的这个妹妹继承了父亲的急脾气,要不是有母亲拿着戒尺强逼着,也不会有心去练习女红。不过逼出来的水平就不用提了,不比她的才学,写出来的几首小词,王旁觉得并不逊于曾布的那位诗才出众的夫人。

把绣得分不清是猫是狗的荷花图揉做一团,随手丢到一边,王旖拍拍手,对王旁道:“大哥这两天也就该到,还有四叔也来信说要回京。爹爹这两天就开心得很,还说要是六叔、七叔也能回来参加二哥你的婚事就好了。”

王安石家中排行第三,父亲王益总共有七个儿子。但王安石的两位长兄安仁、安道早亡,五弟安世也早死,只有四弟安国、六弟安礼还有七弟安上尚在人世。当初父兄早亡,没了顶梁柱的王家,就靠刚刚得官的王安石一人支撑,几个弟弟、还有两个妹妹都是王安石拉扯大,嫁娶都是由他一人主持,兄弟之间的感情也是极好的。

除了王安上以外,其余两人都是进士。如今他们都不在京中任职,王安礼在河东太原,王安上则在南阳做教授,也就王安国离得近,就在西京国子监教书,能在元日之后,趁年假赶来京城参加王旁的婚礼。

王旁的神色很是复杂,说不清是欣喜还是失落。无论他的四叔、还是他的大哥,都是天才横溢,十一二岁就名传士林。比较起来,自己就差得远了。

很勉强的笑了笑,“说的也是,要是都能到就好了。”

