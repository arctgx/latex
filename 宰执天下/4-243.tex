\section{第39章 遥观方城青霞举(八)}

【第三更。】

在城外绕了一圈,回到城中的衙署中,先将李诫安顿下来。接着就是正式签发的公文,让方兴去主持发运一事。

到了晚上,回到家中,王旖、周南和严素心正萝卜白菜的算着家计,家里听候使唤的仆佣站了一地,屏声静气的听着王旖的发落。

王旖治家已久,差事办得好坏,公款用得多寡,她那里都有本帐,说话时虽说是细声细气,却无人敢为自己辩驳。

看到韩冈,房中的人都站了起来。王旖迎上来:“官人怎么这么早就回来了。”

“在城外随便看看,顺便接到了人,也就没什么事了。”韩冈要进屋换衣服,随口就问道,“倒是你,不是说今天有事要参加什么庆生会,回来的比我还早?”

王旖跟着进了里屋:“那一位跟州中有头有脸的官人家的女眷都下了帖子,也是抹不开情面,才去了一趟。闹哄哄的,没个滋味。”

在王旖的服侍下,韩冈换了身家常穿的布衣襕衫,“谁敢在你面前闹啊……”

“就是在奴家面前,才是最闹腾的,换作是别家,跟前就没那么多人来吵。”

王旖是世家出身,宰相家的嫡出女儿,就是韩冈身份卑微,在官员内眷中也照样能压人一头,何况现在还是转运使夫人?出去都被人捧着的。

不过王旖并不怎么喜欢迎来送往,韩冈也不需要她在外为自己打点什么。夫人外交之类的差事,对她来说从来都是桩看不上眼的苦活。

有了王旖主持中馈,韩家的门户倒是清静。甚至连寻常最能穿堂入户的三姑六婆,也别想随随便便的挤进韩家的大门。尤其是佛家、道家的出家人,更是没有向家里请过一个。

韩冈换好衣服,和王旖一起出来。也不说话,就在旁听着王旖如何发落家事。

不得不说大户人家出身,从小就开始接受训练的王旖,在治家的过程中,表现出来的水准,让人惊叹不已。而且她不是一人亲历亲为,而是拉着周南、素心一起管家——韩云娘是个天真的性子,也不敢让她掺和进来。

一人主管,两人协理,家中的钱财、绢帛,乃至贵重器皿、用具、古董都有着一式三份的帐本在记着,取用和人情往来都要通过帐本留下记录。但凡大户人家,都会这么去做。家里的金贵器皿为数众多,少了一根银筷子都能自账目中体现出来。

账目清明,处理起来就方便了许多,王旖没用多少时间就将人都发落。回过头来,她就捧着茶汤,跟几个姐妹笑话韩冈:“平日里把家计的帐报给官人听,不是打哈欠,就是甩手说算了,只道官人不喜听。怎么今天太阳打西边出来了,突然想听家里面的这些杂事?”

几个妻妾都知道韩冈不耐烦听这些鸡零狗碎的家务事,全都丢给王旖这位主母,最后听个结果就了事,今天的确是特例了。

“外面都说你会持家,我说我这个做夫君都没见识过,也算是开开眼界,也能学着如何掌着漕司。”

韩冈开着玩笑的说道,方才旁听时,倒是很给脸面的没打哈欠。但他的确是没什么兴趣去听家中的财务报告,王旖掌管的财产,对于韩家来说,只是九牛一毛而已。亏了赚了,韩冈都无所谓,精打细算都没必要那般寒酸,他家可是如今国中顶尖的豪富,当然,是他家而不是他韩冈。

依此时的律法,父母在世,做儿子的就不得别籍置产,否则便是不孝。就是想买块田,也得放到父亲的名下。如果放在自己的名下,被人究举出来,官员少不了会被弹劾,庶民也会押进衙门里挨一顿板子。

当然,变通的办法也有。在一个没有分家的大家族中,保护自己利益的手段很简单,就是将置办的产业交给浑家,以嫁妆的出息为名,放在自家夫人的名下——依律,女儿家的嫁妆,丈夫都不得动用,如果哪家的新妇能将嫁妆拿出来支援族人,甚至是妇德的体现和象征——这样就不用担心给兄弟或堂兄弟给分了去。

韩冈是独子,倒也不用在乎什么。顺丰行的七成股权,以及熙河路的庄园田产,眼下虽全都是在韩千六的名下,但控制权现在就在他的手中,日后所有权迟早也是他韩冈的。

家里的情况,几名妻妾当然都知道。韩冈一直以来都不将眼下的家财放在眼里,她们也都觉得正常。

现在韩冈说他是要学着王旖治家的本事,哪个会信,王旖笑道:“阿弥陀佛,这奴家可当不起。官人财大气粗,不像我们眼孔小,倒是精打细算着,为家里的哥儿姐儿日后着想,一文两文都要攒着。”

周南拉着云娘笑道:“家里的哥儿姐儿有福了。官人不但是个文曲星,还是个财神爷,荒地里都能变出钱来的。姐姐又是能治家的,日后家里的哥儿姐儿还不知多享福。”

素心、云娘连着点头,韩冈的脸色则是变得稍冷了一点。

“授人以鱼不如授人以渔,留的钱多了是祸害。”韩冈说得干脆,“韩家的女儿嫁出去,都想着在婆家能过得好,有体面,这嫁妆就不能俭省。至于韩家的儿子,若男人不靠自己双手养活妻儿,也没面目见人。”

气氛突然冷了下来,周南三人都有些愣了,玩笑话当什么真,话说得也不中听。王旖不快的反驳道:“怎能这么说?不给儿孙个好安排,怎么开枝散叶,怎么承袭宗祧?”

韩冈一幅无所谓的态度,“君子之泽、五世而斩,谁家供奉能过五代的?百年后再过百年,牌位就早就可以拿去当劈柴了。”

王旖皱眉,这话可不好听。韩冈却不在乎:“话说得虽然是早了点,但大哥、二哥都已经开蒙了,这道理先得让他们明白。”

视线扫过几名妻妾,“我这个做爹的留个好名声,自能遗泽后人。但钱财留的多了,那就是祸害。说实话,我韩家门第浅薄,教养子弟的规矩,不早点立起来,日后麻烦只会更多。须知蓬生麻中、不扶自直,白沙在涅,与之俱黑。”

“怎么叫门第浅薄?”王旖一副不高兴的样子,“韩氏上起三代,唐末又有了一代文宗的韩吏部,这都浅薄,什么叫深厚?照奴家说,官人还是早点将族谱给定起来。”

王旖故意歪曲韩冈的本意,但维护韩家的心思,也是货真价实的。不过韩冈倒是不在意:“编家谱也得有人信,随便认祖宗也没脸再沾个‘孝’字。我韩家上溯个三五代,就得往三皇五帝夏商周去了,怎么编?!欧阳永叔【欧阳修】编修族谱,天下皆以其为范。可欧阳询唐初人,至黄巢时,近三百年,才得五世;欧阳琮在唐末,至仁宗才一百四五年,乃为十六世。”韩冈说着,就嘿嘿冷笑了起来,“世人都是给他个脸面,没人去认真计较,但有几个会给我韩冈脸面?不再踩一脚就不错了。要想福祚绵长,就得早些立下规矩。”

韩冈语气沉沉的,回来后就莫名其妙的说出这番话来,素心、周南都不敢接口,也不知是儿子哪里犯了错,可老三老四老五还在襁褓中,老大老二也不过才开蒙,哪里有犯错的能耐?云娘则是连连点头,她的心思单纯,只当韩冈说得十分有理。

王旖则是疑惑着,自家的儿子都还小,哪里会犯了错事,触了韩冈的心思:“平日里也不见说上一句两句,袖手掌柜做得比谁都自在,怎么突然间冒出这么多话……是出了什么事,还是听到了什么消息?”

“没什么事……”韩冈抬起眼,就是四对刨根寻底的眼神,笑了一声,“就是相州案定案了。不过是州里判错了一桩案子,却让好端端的一个新任相公栽了进去。我那个连襟,倒也没别的能耐,就学会了两个字……坑爹。”

挺新鲜的一个词说出口,性子天真烂漫的云娘就噗的一声,用手捂住了嘴,露在外面的两个眼睛弯弯的;放下心来的周南、素心扭过头去笑;王旖也咬着下唇,一幅想笑又不当笑的样子。

因为变法之故,她的姐姐在吴充家过得很是不愉快,舅姑那里都不讨好,吴安持性子软弱,让妻子受了很多委屈,归宁时每每向母亲妹妹哭诉,因为此事,王旖可是对吴家上下都没有什么好感。

“文六也是一样,文相公也是吃了大亏。”韩冈这一次倒是幸灾乐祸了,他跟文彦博从来都是互相看不顺眼。

因为这桩不算大的案子,文彦博致仕了,吴充去了江南。两个前宰相表示了谢罪之意后,天子自然也不能穷追猛打,御下宽仁。所以吴安持和文及甫并没有定罪,罚俸而已,一开始错判了案子的陈安民贿赂有司,因为天子想息事宁人,不过责降一官,编管远州。

“官人担心的是。奴家会好生教养大哥、二哥,不让他们日后变得……”王旖抿了抿嘴,“坑爹!”

