\section{第37章 青山声碎觑后影(二)}

昨日一场争夺谷地的战斗,最终以宋军的胜利而告终。吐蕃骑兵被逐出了支流河谷,宋军顺利的扎下营盘,拥有了一个稳定的据点。

但为了立寨,宋军伤亡了大约六百余人,其中直接战死了有两百多。连王舜臣都受了伤,连同一百多名伤势较重的伤员,被送到了珂诺堡的疗养院中来医治。至于还有四十多个更重的伤员,因为难以移动,就算送回来多半也是救不回来,所以最后是留在了香子城中。

而战果除了控制了支流河谷之外,还有一百一十多个斩首——斩首数与敌军实际战死的人数,并不是一一对应,后者往往是前者的两倍还多——相对起官军最后当在三百左右的阵亡数,以步兵对骑兵,这个交换比应该不算很差了。

王舜臣的伤只是看起来比较严重,实际上还算好,没有伤到筋骨。就是胸前一条长长的血口子,也不算深,只是看着吓人而已。是蕃人的长刀劈开了皮甲后,刀尖在胸腹上带出来的痕迹。

简简单单的皮肉伤,是王舜臣身边在疗养院接受过培训的亲兵紧张过度,用光了十人份的伤药不说,还几乎将一匹医用麻布就都卷到了王舜臣的身上。这番,虽然军医说没什么大碍,王韶觉得不放心,还是让王舜臣回来让韩冈好好确定一下。

自从韩冈开是改进军中医疗制度以来,如今的熙河、秦凤两路将校,他们身边的亲兵都是接受完整的战地急救培训——或者说,只有接受过完整的战地急救培训之后,才可以充任将校的亲兵——一方面,有着医疗能力的亲兵们可以为将校收服麾下军心,同时,他们也更能保护军官们的生命安全。

但大惊小怪的人还是有的,韩冈瞥了眼局促不安的站在一边、只有十几岁的小亲兵,摇了摇头。反过身来,用责难的口气问着王舜臣,“怎么不穿身好一点甲胄,有个护心镜,肩膀上多片硬甲,都不至于受伤。”

王舜臣胸腹处还是捆了一圈,但精神看起来旺健得紧。他哈哈笑道:“要射箭哪能穿着硬壳子,一身皮甲已经很碍事了。”接着又不服气的冷哼了两声,“要不是弓正好拉坏了,谁能近到我十步之内?直接就一箭在人脖子上开个洞了。”

“都是一路都巡检了,熙河路中能稳压你一头的将校,也就苗授之一人。还像小卒一样站在最前面,日后如何统领大军?”

“三哥放心,小弟以后肯定当心。”王舜臣道,“只出动万人不到,就硬从木征手上抢下一块地来。等姚兕过来,谅他也不敢指手画脚的说什么了。”

韩冈叹了口气,要是前日他身在前线,肯定要劝诫王韶的。不过是些去年临洮一役后,从泾原路传来的一些闲言碎语,何须如此提防?

王韶的想法韩冈能够理解,但他的本心却是有些不以为然。这等无谓的面子问题,其实太在意也不好。至于所谓的怕镇不住姚兕姚麟,光凭赵顼通过诏书交到王韶手上的‘便宜行事’四个字,足以将任何敢于违抗军令的将校,砍了脑袋下来示众了。

王韶就是挣一个面子而已!

只是既然王韶已经成功,那也就不用再提。

当王韶领军全数进驻了支流谷地,韩冈在珂诺堡这里加强了防御,又筹措了粮草输送上去。过了两天,姚兕领军到了珂诺堡。休整了一夜,便继续前行。

熙宁五年三月十八日,在河州城下,宋军已经聚集了两万兵马,而面对的敌人接近五万。这等规模的会战,近两年来,只在罗兀城下发生过一次。

随着前方的兵力越聚越多,韩冈也是越来越难放下心来了。吐蕃人占据了人数上的优势,抽调部分兵力过来偷袭后路那是必然的。

最有可能的就是珂诺堡,四处暗道给城中守军心头上蒙了一层阴影,谁也说不准会不会有第五处、第六处。韩冈明面上宣称已经将暗道都找了出来,让下面士卒不必杯弓蛇影,但暗地里还是让工匠营中擅于开洞掘土的匠师,在堡中多方巡视,以便确认寨堡的安全。

作为自陇西城过来的第三个中转站,珂诺堡对前方官军的意义不言而喻。王韶也是刻意留驻了大量兵力,来维护大军的后路。

现在在珂诺堡中有三千将士,但韩冈依然是放心不下——换作是别人,也难放心得下——前面是陕西半壁的精锐,一旦有所闪失,任何人都承受不起。

出身自广锐军的两千乡兵,韩冈早前就已经下令调到珂诺堡来,后方的转运有普通的乡兵弓箭手来押送就足够了。好钢还是要用到刀刃上,有他们负责这一段的粮道输送,其实就相当于多了两千精锐的战力。另外虽然对刘源有些说不过去,但韩冈需要这群广锐将校,即便他们现在只剩过去的一半实力。

招来亲兵,韩冈让他把自己的亲笔信连同令文送往狄道,希望刘源他们能在两天内赶来。另外又写信去给王韶,请他调一个指挥的骑兵回来,以防万一。

第二天,从前线被调回的骑兵到了韩冈面前报道:“末将田琼,奉命听候韩机宜指派。”

韩冈稍稍安心下来,尤想着刘源和广锐乡兵什么时候能到。

………………

狄道城。

沈括看着送到手上的公文,见着韩冈指明要将前广锐军组成的乡兵调去珂诺堡负责转运一职,他眉头微皱,‘韩冈未免太过相信这群叛逆了吧?’

韩冈要把这些不可信任的叛军调去最关键的位置,沈括是难以理解。虽然他们曾经表现过一定程度的恭顺,但叛军就是叛军,如果有可能,沈括是绝对不会相信他们。

可既然是韩冈的要求,沈括想了想,还是决定不加干涉。若是自家在中间横拦了一手,最后不论出了任何事,以韩冈在熙河路的发言权,能让人把罪名多多少少的栽在自己的身上。

因为即将共事,沈括还在京中时就着意打听过一阵,对韩冈有一定的了解,明白他不是什么心慈手软的角色,更不是什么可以欺之以方的君子。

还是不要无故招惹他为是,沈括心里想着,手上老老实实的批了同意。

这事很快抛到了脑后,沈括还有许多工作要忙着完成。不过两日,一众广锐乡兵被调集到了狄道城中,转头就要跟着刘源他们前往珂诺堡。

这几日,前方还没有展开决战的消息,仍是在对峙和小规模的交锋试探中。沈括暂时放下手上的公务,走到了广锐乡兵中间。他要亲眼看一看,这群叛贼究竟是否可用——毕竟事关重大。

两千乡兵,看起来普普通通。就是神态安稳,没有寻常百姓被调上前线的慌乱。

随便在跪下的人群中,点起了一个领头的,沈括问道:“你叫是什么名字,”

那名相貌朴实的汉子低头回道:“回官人的话,小人名唤尤三石。”

“三十?”见着尤三石老老实实回话,沈括感觉着汉子并不是那种凶戾的叛贼,倒也不是不肯悔改的,他笑了笑,“……排行倒偏后,族中兄弟不少。”

尤三石又躬了躬身,回道:“回官人的话,小人就一个兄弟,族人也不多。小人的名讳,是一二三的三,石头的石。”

沈括微微一怔。寻常的百姓,许多时候见到官员连话都说不好,哪有可能会出来指称官员叫错了名字,将错就错了。果然还是广锐军出来的,见过一点世面。根本都不会畏惧官威。

沈括一眼望过这两千乡兵,还有远远站在一旁,气势更为沉凝的前任将校,有些心神不宁。

这时几名骑兵从衙门处赶过来,匆匆的将沈括请到了一边。

其中一名骑手风尘仆仆,身上也是大汗淋漓,急急的在沈括耳边说道:“运使,临洮堡对面的禹臧家援军到了!已经增加到两万人!姚都巡请运使赶紧调发援军,迟恐不及!”

‘两万!’沈括心头一惊。

姚麟手上有四千兵,这一部分泾原军,顶替了前日在狄道城北方,护卫临洮堡修筑工程的秦凤军。守护起刚刚筑好的临洮、结河川两堡应该是安稳的。但禹臧家出动两万大军还是远远超出了预计。

“不是有包约的青唐部四千人马吗?!”他连忙低声问道。

“这样也才禹臧家的一半。而且姚都巡手中的四千人,有一千是拖在后面,驻守在河谷道兵站结河川堡中,不可能上去支援。”

听到军情,沈括心急如焚。只是单纯的随军转运使,不比韩冈有着经略司机宜文字的身份,能直接调动一部分兵力,然后让王韶事后认可。

他正犹豫间,突然瞥到了尤三石身上。灵光一闪。

总共两千人,分上一千总不为过。临洮堡要是出了什么错,这个责任他也担不起。调动乡兵、民伕,他随军转运使的签押和印信是足够用的……

重又走到广锐乡兵队列前,沈括心中暗叹,他这也算是病急乱投医了。

