\section{第32章 金城可在汉图中(一)}

暮色已深,崇政殿中变得十分晦暗。

两名内侍正拿着火引,一盏盏的去点着玻璃灯罩内的蜡烛。

向皇后并没有在批阅奏章,有点呆滞看着内侍将灯火点起。摊在她面前的章疏上,一个朱笔留下的字迹都没有。湘妃竹所制成的毛笔抓在手中,斑斑泪痕的笔杆动也不动,已经过了很长时间了,她都没有动笔的意思。

皇城司的石得一在外通名之后,匆匆踏进殿来。

向皇后抬起了眼,稍稍坐直了,问道:“韩学士已经出城了?”

虽然韩冈已经是枢密副使,可向皇后依然用着已经习惯了的称呼。

石得一连忙恭声禀报:“回圣人的话,韩学士一行是两刻钟前出的城。走得快的话,今晚就能抵达郭桥镇。明天到酸枣过河,抵达新乡后,从白陉北上,不日便能进抵太原。”

向皇后眼神愣愣的,也不知听没听到。石得一不敢惊扰到皇后,只得屏气凝神的站着,过了半晌,却又突然开口:“韩学士就没回家?”

“没有。”石得一十分肯定的摇头。

‘这才是纯臣的啊。’向皇后小声赞叹着。

堂堂执政出镇地方,至少应当在文德殿上陛辞,以尽君臣之礼。可河东事变,让一切仪式只能草草走个过场,当事人的韩冈更是浑不在意。

满朝文武,可有如韩冈一般能解民倒悬,为国抒难的?又可以一人如韩冈一般的视高官显宦如寻常?搜遍朝中,向皇后也找不到第二个可与其媲美的人才了。

“皇后,太子来了。”杨戬突然小声的提醒道。

向皇后立刻坐直了一点,吩咐道:“让六哥进来。”

立刻就看见身穿大礼服,头戴冠冕的赵佣在乳母带领下,前后宫女、内侍,然后跨进殿中。

“儿臣拜见母后。”赵佣在向皇后面前拜倒行礼。

向皇后眯起了眼睛,仔细观察着儿子在这一套繁琐的仪式中,到底有没有错,这关系到他在官员和百姓心目中的地位,乃至日后能不能胜任皇帝之位。

不过赵佣表现的很好、两天后就是赵佣正式出阁读书的日子,过年后刚满六岁的太子殿下为了这一天,已经整整练习了三个月的礼仪。

在赵顼基本上无法复原的情况下,皇太子赵佣已可以说是半个皇帝了。正常年纪,应该爱玩爱闹的时候。可此时的赵佣,却被教育得向一个老头子。

拉着赵佣,向皇后细细问着他这几日学习的成果。

赵佣老老实实的站着,神态端庄的汇报着自己的成绩。

并非是亲生骨肉,太子终究是少了一份亲昵。

向皇后暗自叹息,谁让她没能有个一儿半女,唯一的女儿都早早的夭折了,宫中好不容易保下来的太子和公主,都是朱妃所生。

赵佣对晨昏定省不敢有片刻耽搁,但也不会久留在皇后身边,汇报完毕后,就小大人一般的起身告辞,他还有亲娘那边要去请安。

“对了,韩学士临走的时候,推荐了几个人入国子监。”目送了儿子离开,向皇后想着,然后说着,“就照韩学士的心意去办吧。”

……………………

韩冈终于是走了。

这让蔡确稍稍松了一口气。他希望韩冈能在河东继续创造奇迹,但到底有多少把握能成功,韩冈没有说,别人也猜不到,似乎是不会太高。蔡确也只能暗中祈祷韩冈最后能凯旋归来。

而刑恕甚至还长舒了一口气,以表庆幸,“终于是北上了。”他低声喃喃自语。

“和叔你可别放心得太早。”蔡确摇头,“你可知道,韩冈今天在陛辞的时候向皇后求了什么吗?”

刑恕低了低头:“敢问其详。”

“他荐了你那十二名被带上京的同窗进国子监!”蔡确轻笑,却见刑恕脸色陡然一变。

蔡确笑容不改:“看看,多聪明啊。拿着受业于大程的名义,将那十二门徒转头就给荐到了国子监去了。”

世间都传韩冈尊师重道,可程颢抵京后,韩冈送了价值不菲的礼物过来,但七八日了,就上门拜访了一次,可有半点当年程门立雪的风范?怎么想得到他临走时就直接送了一个大礼,连考试都不用,直接被推荐进了国子监。

“和叔你说,令师这个情况下会怎么做?”

刑恕想了想,摇了摇头,“刑恕不知。”

不过或许也会默认下来。刑恕暗暗的猜测着。

虽说韩冈这一回的确是没安好心,但不论从什么角度来说,这都是对程颢弟子们的恩德,也是程颢扩大影响的机会。

国子监实行的是三舍法,从外舍、内舍到上舍,一级一级往上升,成为为数只有一百人的上舍生后,就有直接赐进士出身、出来做官的机会。也即是说,国子监生如果成绩的好的话,甚至都不用参加科举。

韩冈送程颢的弟子入国子监,纵然只是人数多达两千的外舍生这份人情他们也必须要领。否则不仅开罪了皇后,在世人眼中,也是不知感恩的无耻之辈。甚至还不能不去,否则皇后说不定会说一句不识抬举,半辈子就完蛋了。不论哪家的西席先生,让主母看不顺眼,都不可能安安生生的授徒授业。

只是这十二人若是太太平平的在监中学习,没有一点声息,那就代表程门的弟子叛离了师长。做弟子的都不能坚持师长传授的学问,那谁还会相信这位老师有足够的才华教授好弟子?但若是全都拒绝了,那结果只会更糟糕。而最坏的情况,则是他们进了国子监,却在国子监中与新党的成员起了争执。

程颢带来京城的学生,虽然特意选了一干老成稳重之辈,可他们大多数还是过于年轻,很容易被煽动起来。

“相公,能不能……”刑恕的话说到一半就停了,他相信蔡确能领会。

蔡确领会了,但他一口否决:“韩玉昆这一回挺身而出,两府是受了他的大人情,不能不还。”

韩冈若不接手,政事堂和枢密院都要为河东之事负责——他们要为天子背黑锅——反倒是韩冈这位前任河东经略,可以因为他推荐提拔的将校不涉败绩而脱身出去。

但韩冈现在以枢密副使的身份前往河东,等于是将整件事都拉到了自己身上,与两府中间便隔了一层。不论最后事情演变到什么情况,韩冈都是第一责任人,事后如果治罪,一个执政总能抵得过了,何况吕惠卿也少不了一并受责,这可是正副枢使,半个西府了。东府这边,完全不需要担心什么了。

“王介甫那边则更不会加以阻挠。十二人而已,国子监的直讲、讲书、说书加起来,差不多有两三倍。难道还能翻起天来?”

韩冈的用意无外乎牵制王安石和程颢,当他不在京城的时候,让王氏新学和程门洛学好好斗上一场。不过这件事,他是做得光明正大,并非以阴谋诡计伤人。

刑恕一叹,自然不便再说些什么。但不论真情假意,他都必须记住二程的教授之德,不得不站在二程这一边。

“不过韩玉昆也不好过。”蔡确很信任刑恕,甚至不介意透露一些机密的消息:“韩冈刚走,河北那边就送信到了。说是有细作来报,七天前,大约有万余名辽军骑兵转去了飞狐陉,并没有南下河北。”

刑恕的脸色顿时变了。这个消息是个不折不扣的噩耗。

飞狐陉的东头是辽国的蔚州,西段则是大宋的代州,以瓶形寨【平型关】为界。现在代州失陷,瓶形寨两头都是辽国的兵马,肯定是保不住。辽人一旦打通了河北和河东的联系,两边的兵马可就是要合兵一处,太原能不能坚持下去,就是韩冈也不会有底。

两府之中没人想看到河东兵败。韩冈这位深悉西北军事的重臣如果还解决不了问题,朝堂上真的就选不出人了。到时候,大嘴一张的辽人那边可不是好应付的,再来一个城下之盟,少说也得下去一两个宰相作陪了。

‘终究还是能赢最好。’蔡确想着。如果有可能的话,至少稳守住太原。河北则是守住三关和定州、保州一线。代州或是雁门收不回来的话,可以得拿兴灵去换。可真要是河北北界守不住,大名府以北就成了辽人的狩猎场,就算大宋想谈判交换土地,辽人也只会先抢个尽性再说。

蔡确轻声叹。无论如何勾心斗角,在面对外敌的时候,两府中的态度还是极为明确的。

一切可就要看韩冈的了。

……………………

韩冈一行人飞驰在荒原上,没有月亮的夜晚,只有阡陌纵横道路还能看得清楚,只要一个不小心,就立刻就会重重摔倒地上。

京畿的道路年年修补,但坑坑洼洼的情况还是少不了。如同西式的弯月照不亮地面,再跑下去,摔断骨头的可能性就会越来越大,但韩冈并没有减速的意思。

韩冈出京后一路走得极快,甚至连深沉的夜色也不在乎,从日头偏西,一直到掌灯时分,他领着家丁、部属,直接就奔出了四十里,酸枣县的灯火已经是遥遥在望。
