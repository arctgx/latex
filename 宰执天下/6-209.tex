\section{第24章 夜雨更觉春风酣(下)}

“那是韩相公家的车吧?”

离大图书馆还有半条街的时候,听到不远处有人说话。

秦观打着伞,顺便向街上张望了一眼。正从路中垩央经过的一队车马,马车前摇晃着的玻璃灯笼上,有着字迹分明的韩字。

马车前没有开道的旗牌官,自不是官员本人,而一行车马的规制,却远远超过了普通朝臣所能拥有的标准。朝堂中韩姓的大臣为数不少,但在韩绛离开之后,家眷还能有如此规垩模的护卫,那的确就只有一个了。

秦观转回头来,说话的那人眼熟,而他说的话也是耳熟,“……大丈夫当如是也。”

夜风清寒,雨声淋漓,话入耳时,不禁让人心下悚然。

说话的是同在国子监中的赵谂,来自西南渝州【今重庆】。

这个姓赵的,和其他姓赵的不一样。他父亲名为赵思恭,听这个名字就知道是归化的蛮夷得到朝廷的赐姓赐名。李继迁的赵保吉,李继捧的赵保忠,皆如此类。

赵谂十五六岁的年纪,就从蕃学被推荐进了国子监,在监中十分的显眼。且只用了一年就进了内舍,比起秦观的成绩还要强一些。‘大丈夫当如是’,归化蕃人这么说话自是犯忌,但出自一个十五六岁少年之口,倒不是不能理解,这本就是不知天高地厚的年纪。

看见年少轻狂的赵谂,秦观只有岁月易过的感概。

年已四旬,才学不差,文名更盛,小词在秦楼楚馆中流传得很广,‘山抹微云’更是让他在士林中名声大噪。

但他在科场蹉跎至今,元佑宫变之后又受了苏轼的牵累,连着两科被拒之门外。还是靠了几篇在《自然》上发表的论文得到了韩冈的赞许,才被安排进入了国子监中。

他年少时好读兵书,慷慨于文辞,稍长一点写下‘眷言月占好,努力竞晨昏’,到了连年科场不利,便只有化用小杜‘赢得青楼薄幸名’的一首《满庭芳》,时至今日,方以一部《蚕书》得到宰相认可。

“少游兄。”

身后的声音,打断了秦观的思绪。

秦观循声回头,却是国子监同学的毕渐。

“之进。你方才不是走了吗?”秦观惊讶道。

一同从国子监出来,毕渐回住处,他要去大图书馆,方才就分开了。

毕渐道:“邓府巷那边给巡检堵住了,得绕道回去。”

“出了何事?”

“邓府巷那边不是有座废园吗,说是要抓里面的乞丐。”

秦观失笑道:“上一次是拦着下水道,这一回就换堵路了?”

京师的暗沟近百里,里面都能行船,干燥一点的地方还能住人,藏了不少作奸犯科的贼人,而这些贼人出来时,很多也混迹在乞丐群中。包拯知开封府的时候都没能清理掉他们。还有一干无人居住的宅邸,都成了城狐社鼠的窝点。现在朝廷动手清理,城里城外已经抓了数百人了。

“说是天黑雨大,不小心跑了七八个,正堵住路挨家挨户搜检。”

“为乞丐夜搜民家,此事岂不扰民。”秦观摇头,抓乞丐没什么,但为了抓乞丐弄得夜入人家,他实在不能苟同。那些巡卒有哪个好的,夜里进了人家,就跟虎狼入屋,吃点拿点都算是轻的,重一点,家里的女眷都要遭殃。

“秦兄此言差矣!”

又是一个声音自背后响起,不过声音中的情绪比毕渐骄傲的多,却见是方才在一边说话的赵谂凑了过来。

“秦兄最近没看报吗?”赵谂自来熟的插着话,“说是乞丐,其实多是穷凶极恶的贼人,东城前两年不是有一家被杀绝了吗,犯人最近落网,就是在乞丐中抓到的。”

秦观眉头皱了一下,但并没有就此发作,赵谂的年纪还不到他的一半,与他置气毫无意义。

“这一点,小弟也觉得韩相公和开封府做得没错。”毕渐点头赞同赵谂,“本就是京垩城一害,又不知藏了多少贼子,如今边境上既然缺人,怎么能容得他们继续祸害京师良民?”

报纸上这段时间都在连篇累牍的说街头乞丐的问题,吹捧韩冈的政策,各种各样的证据一时都拉到了台面上,最惹人注意的就是许多无头案件,都从乞丐身上找到了线索,甚至犯人。而乞丐内部的倾轧,丐头对普通乞儿的欺压,还有拐卖良家子弟,弄残废了之后讨人可怜,此等事更是罄竹难书,读来只让人觉得字纸之中,满满的都是血泪。

唱莲花落的乞丐,在京师三百六十行中,也算得上是让人闻而生畏的行会之一了。乞丐讨要上门,那个店家不给点面子。当天夜里,就能有人提个净桶过来往门前一泼,害不了人也能恶心人。几次下来,哪家商家能不低头?做生意讲究的是和气生财,开门迎客,这门都开不了,还怎么做生意?

即使背.景再厚,跟乞丐置气也有失身份。本就是一点小钱就能解决的事,却把后.台给拉出来,主事者少不了要吃挂落,最后没有哪家不是出钱消灾了事。京垩城商家对乞丐忍受已久,现在韩冈要把他们全都送去西域、云南屯田,哪家不举手欢庆?

木笛声突兀而起,打断了三人的对话。只看见一名衣衫褴褛的男子,从毕渐方才过来的方向穿过街道,看穿着分明就是一个乞丐。

那乞丐跑得飞快,两条腿踢得街上水花四溅,后面追着七八个军巡铺的铺兵,一个个累得气喘吁吁,吹着木笛的军官气急败坏,但挺着一个富态的肚子,只能含恨落在了最后。

“你看,报纸上说得哪里有错,又非缺手缺脚,能跑得这么快,不是懒,怎么会做了乞丐?啊!”赵谂忽的兴奋地叫了起来,“真是找死。”

的确是找死。

秦观看着那乞丐在追捕下逃进另一条街,心下附议。

那条街道,韩家车队刚刚转了过去。

尽管都还有事,但三人仍在等着,没有离开。

正如他们所料,没等多久,只看见有两人拖死狗一般拖着那名乞丐回到大街上,几个铺兵点头哈腰,将那乞丐接了过去。而方才吹着木笛的胖军官也是一阵点头哈腰,送了韩家的两名下人离开,回过头来,就是狠踹了那乞丐两脚。

“那贼子或许有案子在身上,否则断不至于如此。”毕渐揣测道。

“有几个乞丐不犯事的?清光了了事,京师也能太平些。”赵谂冷笑起来,“太皇太后今日上仙,明日开始就要办事,这些乞丐也是犯在了风头上,肯定没好结果。”

秦观暗暗摇头,太皇太后自己都没好结果,一个儿子死于亲子之手,一个儿子因谋叛被诛,还有一个儿子喜爱医术,招了人研究疫苗,最近听说因为沾了病毒,染了疾疫,也没多少日子了——说实话,听到这个消息后,他真佩服韩冈,怎么有胆子去研究天花,一不小心命就没了。

“太皇太后自己都没好结果。”赵谂却把秦观心中的话说出了口,“谁还理会那些乞丐的结果?”

还真是敢说!

秦观与毕渐对视一眼,道理没错,说出来就有错了。再让赵谂说下去,给人听到了就是麻烦。忙打了个哈哈,然后匆匆告辞离开。

走了几步,两人都是摇头苦笑。

“还是太年轻。”毕渐轻声道。

“是太年轻了。”秦观也道。

赵谂读书虽不差,但时间的磨砺,人情世故乃至见识都差了许多。

不管怎么说,太皇太后都是先帝的生母。做亲娘的怎么处置儿子,打也好、骂也好,都没问题,就是勾结了奸夫,要害亲生儿子,被抓到了公堂上。抱歉,为全孝道,做母亲的还是不便处罚。如果儿子不懂事,下面也会有人提醒。若是儿子不依不饶,法官出面训诫都没问题。如果一切依法判决,反而会被诟病。

这类官司出得不多,但传得很广。秦观记得唐时就有过一出,嫌儿子碍事,便在奸夫的唆使下,到官府告儿子不孝。不孝之罪,是十恶之一,定案必死。但审案的官员发现了破绽,最后查了水落石出。而最后判决的的结果,却是法官意欲重惩,儿子愿代母受刑,最终母子和好如初。

这类事关人伦的大案,件件通天。如果处理得好,主审的官员完全可以藉此扬名立万,日后若是能达到国史有传的地位,本传也绝不会少了这桩案子。

但凡有些见识的官员,遇到这类案件,都会设法让案子变成母慈子孝的大团圆结局,一如《春秋》开篇,要杀长子郑庄公的姜氏,最后在隧道中,一个唱着‘其乐也融融’而入,一个唱着‘其乐也泄泄’而出,重修旧好,

为全孝道,不让亡夫为后人所议论,尽管太皇太后做了那么多事,太后也还是只能让太皇太后备极哀荣。

‘就不知太皇太后的赠谥会是哪个了?’

秦观心中想着,与毕渐告别,出示了自己的图书证,收起伞,在门前的木板上蹭了蹭脚,走进了大图书馆。

读书楼中的七八间阅览室内,有三四百位士子在这里通宵达旦。大概是下雨的缘故,今天的座位空了大概两成,寻常都是人满为患,无论白天还是黑夜。

这读书楼与后面藏书的几栋楼相隔了一堵高墙,藏书阁中藏书数以十万,书架重重,因而严禁烟火,只有白天才会打开。而读书楼中,只有靠墙的一列书架,上面只有常见书籍,历年《自然》,以及近日报刊,就没那么多顾忌——每间阅览室中,都有十几盏明晃晃的油灯,照得满屋透亮。

这世上,只有捐献给寺庙长明灯,却没有捐给学子的长明灯。大图书馆中所用的灯油,全都是钢铁厂那边出来的,炼焦后产生的废油中提炼,味道难闻,烟气也重,可量多价廉,城中百姓买得多,朝廷为此拨款也痛快。为了能让更多的士子有机会读书,朝廷给大图书馆拨款也痛快。

不仅仅是京师,天下的诸州,一座座图书馆拔地而起。而朝廷更是拿出了官职,为造纸、印刷给出了悬赏,降低印书成本,降低书价,让更多的人可以读书。

有多少士人,就是因为读书不多,而导致见识不足,最后永远只能仰望黄榜上的名字。又有多少儿童,因为书价太贵,而不得不放弃读书。

仅仅是经史两部,历代流传下来的传注、史集,便是数以千卷,普通人家有几个能买得起那么多书?官宦人家的子弟,更容易考上进士,并不全是因为父辈的权势。

而福建之所以文风鼎盛,进士数量始终保持在诸路第一,很大程度上便是福建的印书坊多如牛毛,书籍价格低廉。尽管福建版的图书以质量低劣、错讹众多闻名,但有错的书总比连书都没有要好,且为书校对错漏,也是学习的一种途径。

现在京垩城士子们手中的书,很多都是油墨印刷,用手一蹭就模糊了,但比起那些雕版精美、纸质优良的上品书,便宜的不是一分两分。秦观虽是官宦人家子弟,可若是在他面前,分别是十文一本和百文一本的书让他来挑,他肯定会选十文钱的。

朝廷欲让天下兴学,以多策来鼓励富户兴办蒙学。据说宰相所规划的第二个五年,就有蒙学毕业学子达到五十万的计划。但空有学堂还是不够的,学生们日常需要大量的书籍。便宜的书价,便是兴学中最重要的一环。

天知道,天下间到底有多少士人会因此感激韩冈所做的一切。甚至现在国子监出版的《科学》期刊,这部备受士林关注和美誉,刊载历年科举策论,以及国子监内部考试内容的刊物,也是受了《自然》的刺激,才告问世。

春雨滋润着大地,室内的油墨味道和淡淡烟气也仿佛春风,使人不觉沉醉其中。

从书架上熟悉的位置抽出一部书,翻到前一次停下来的位置,回到座位,秦观开始提笔抄书。

或许斯人此生不得归乡,但他说过的话,秦观依然记得分明,抄书方是读书。

以斯人谪仙之才,都要两抄《汉书》,只为科场登第,秦观又如何会吝惜自己的笔墨?


