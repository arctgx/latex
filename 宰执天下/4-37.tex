\section{第七章 都中久居何日去(五)}

韩琦终于死了。

在病榻上缠绵了数月,赵顼不停遣使送医送药,又以加封来冲喜,但最终还是没有能挽回这位相三帝、立二主的元老重臣的生命。

这个消息让朝中的许多人松了一口气,从欧阳修开始,从仁宗朝中叶开始引动天下变局的那一干名臣,终于一个个的退出了这个时代。

先是欧阳修,继而是吕公弼,现在又有韩琦,接下来,富弼、曾公亮、文彦博、张方平,这一干人都是垂垂已老,什么时候离开都不会让人觉得奇怪。

已是新旧交替的时候了。

就算是王安石、韩绛、冯京这样的宰相,在韩琦他们的面前都是小辈。

已经故世的欧阳修是一代文宗。当代文学之士,无不出自于他的门下。而韩琦则更是持天下名臣牛耳。国有定规,为官者不得在乡中任官,只有元老重臣可以衣锦回乡作为朝廷的恩宠。而有宋以来,能三次守乡郡的重臣,就只有韩琦一人。

赵顼却是很有点伤感,没有韩琦的扶持,就没有他父亲赵曙登基为帝,当然就更没有他现在的位置。

赵曙不过是濮阳郡王家的十三子,没有得到皇储职位的时候,他就不过就是个团练使,甚至不敢去想郡公这样的爵位,一个县侯就能把他打发了。而作为郡王家不知多少个孙子中的一员,赵顼更是不敢奢求什么,他幼年时是在宫外长大,从来没有享受过皇储该有的教育和重视。而十四岁之后,能成为一国的重心,全是韩琦的功劳。

收到韩琦的遗表,两府重臣们也议定了韩琦的谥号——忠献,以及他的追赠——尚书令。

“从明日起,辍朝三日,为尚书令、韩太师哀。”

就在崇政殿上,赵顼吩咐下去,命翰林学士起草诏书,这是元老重臣都能享受到的恩荣。

“蓝元震,朕欲于后苑为太师发哀,你且速去准备。”

蓝元震领命后去后院,准备祭奠用的器物。而赵顼提起笔,亲自为韩琦撰写着碑文。

饱蘸了浓墨的毛笔在展开的纸面上只是稍作停留,便八个字一气呵成——两朝顾命定策元勋。

八个字用着篆字书就,赵顼书法上佳,写出来的时候,也是气度自蕴。李舜举在旁边为赵顼按纸磨墨,看到天子为所写的碑额,暗暗点头。

这八个字也只有韩琦够资格收受。仁宗传为英宗时他是首相,而英宗传位今上时,他也是首相。顾命、定策,两桩功绩韩琦都是排在第一。

题下了碑额,赵顼又亲撰碑文,不劳翰林学士、中书舍人代为起草。而是自己亲自来写。

赵顼并无捷才,远远比不上在几百万名士子中冲杀出来的翰林学士。当章惇已经将诏命用四六骈俪的文字写好之后,赵顼又用了半日功夫,方才写好了几百字的碑文。

放下手中毛笔,赵顼又仔细的看了一遍,待到纸上墨迹稍干,他拿起来对李舜举道:“传朕谕旨,赐太师家中银两千五百两,绢两千五百匹,李舜举,你代朕将这幅碑文连着赐予的银绢一起送去相州。”

李舜举连忙走下去,跪倒接了圣旨。

“张茂则。”赵顼又点起另一位内侍中的高官——入内都知张茂则,“太师的葬事由你管勾,不得有任何差错。”

张茂则叩首领命:“臣遵旨。”

“童贯。”赵顼接着再点起今天在殿上当值的小黄门,“去查一查安阳知县是谁?”

童贯连忙去查找名单,转眼就回来报告:“是嘉佑八年的进士吕景阳。”

赵顼对这个名字没有什么印象,皱眉想了想:“再去查一查相州观察是谁?”

“是陈安民,于去岁上任。”

这个名字赵顼就记得了:“是文彦博的妻弟。”他又有点惊讶看看童贯,怎么这回不用查就能回答了?

童贯惯会察言观色,连忙道:“奴婢方才一起查看过了。”

“嗯,挺会办事。”赵顼满意的点点头,没想到长相完全没有一般内侍的阴柔的小黄门,心思竟然这般细腻,“过两日去御药院听候使唤。”

童贯立刻跪下来叩头谢恩,脸上露着谦卑和感激,心中则已是欣喜欲狂。

想要在宫廷中晋升,除了跟对人之外,就是要靠运气,只要抓到一次机会让天子满意了,就能一举飞升。过去童贯因为跟随李宪,加上又曾经多次担任传诏使臣,在天子面前留了名,就进了崇政殿中服侍。但他的运气就此而止,一年多也没见动过,但今天终于时来运转,给他抓到了机会。

赵顼岂会在意一名小黄门的心思,提声对着另外一名翰林学士道:“命相州观察判官陈安民、安阳知县吕景阳及入内都知张茂则同管勾太师葬事,许即坟造酒,以备支用,”顿了一下,“再命同知太常礼院李清臣,往相州即其丧祭奠。”

回头再看看韩琦的遗表,赵顼又提起朱笔来批复。重臣死前都有资格上遗表,推荐族中的子弟任官。按照官职高低,推荐的人数也就不同。不论韩琦在遗表中推荐了谁人,赵顼都是毫不犹豫的写了一个‘可’字。

慈寿宫中,曹氏也听到了这个噩耗——不过对她来说,韩琦的死也算不是噩耗了。

当初英宗即位后,曾因重病而让曹氏垂帘听政了一段时间,但赵曙病好之后,韩琦便以十分无礼的手段逼着她撤帘归政。而更重要的,还有濮议之争,到底要不要给赵曙的生父濮阳郡王追赠帝位,曹氏与赵曙对立严重,而朝堂上也吵成了一团,而在这番争执中,韩琦是站在赵曙的一边的。

因为这些事,曹氏对韩琦的感官一直都不怎么好,但他终究是大宋的忠臣。

“吕公弼死了,韩琦也死了。文彦博、富弼也都垂垂待老,没了元老重臣坐镇,日后这朝廷真不知会变得怎么样。”高太后就在慈寿宫中,对着她的姨母叹息不已。

“官家自有分寸。”

曹氏也自知时日不多了,已经没有多余精力去扭转她做着皇帝的孙子的想法。再说,如今的天子虽然一门心思的想着开疆拓土,但行事也随着年纪渐长而有了分寸,不会再偏听偏信,也懂得了该如何钧衡朝堂,作为皇帝,能做到这一件事也就够了。王安石虽然现今看似权倾朝野,但他对朝堂,再不会有熙宁初年那样的影响力。

韩琦死了。

一个时代结束了?韩冈觉得还不能这么说。

虽然韩琦在这个时代举足轻重,回溯数十年间的朝堂变局,都不能将韩琦排除在外。只不过韩冈毕竟没有在他浅薄的历史知识中,找到韩琦这个名字。论起对后世的影响,韩琦应该还远远及不上欧阳修。

他对韩琦的死没有什么看法,从别人的嘴里听到的传说总是隔了一层。虽然韩琦几十年前曾经担任过秦州知州,不过离着他的记忆实在太远,所以韩冈就不可能像王雱那样,连着几天都是喜气洋洋,虽然竭力装出悲痛遗憾的样子,却怎么也装不像,只是平平常常的度日而已。

辍朝三日,乃是朝会不用举行,并不代表天子和臣子不用做事。

王安石有他的事要做,王雱有他的事要做,韩冈当然也要操心着他军器监的工作。

这一段时间来,西方式风车的试作品断断续续的运行了一个月,终于确定了有效的结构。接下来就是打造更大的实用化风车,与中式的风车做对比,如果能成功的话,可以拿去抽水、磨面,当然,也可以用来驱动锻锤。

风车要想成功,还有一段路要走。但靠着军器监内外一起运作,轨道和有轨马车已经验证得差不多了。韩冈又上奏天子,在矿场推广使用轨道。节省下来的大量人力,可以投入到矿井开采中,也可以投入到生铁冶炼里,效率高上不止一倍。

剩下的且迫在眉睫的问题,就是焦炭。韩冈对炼焦的手段,只能让人用烧木炭的方法来烧焦炭。其间几个窑爆了好几次,最终确认了爆炸的原因,对窑口进行了改进,用竹筒释出煤气,并改动地面结构,用来收集煤焦油。不过要得到让人满意的成果,还要在进行一段时间的确认实验。

另外,韩冈也没有忘掉,在给天子的报告中,对自己倍加称赞的那名走马承受。

他的身份已经确定了,是个内侍,而不是武臣,名唤马缄。在宫中混迹的阉人,不可能连话都不会说,至少有五六成嫌疑——对韩冈来说嫌疑的比例已经够高了——恐怕王安石和王雱心里也有点疑惑,所以没有说明是内侍还是武臣。

马缄受了谁的指派,韩冈一时还没有查出来,但根子不会脱离两府。高阶内侍过了内常侍这一级之后,都会转为武职。到时候他们的晋升,就免不了要受到宰执们的影响。这也就是为什么宋代的宦官们闹不出事来的缘故,有文臣将他们当贼一样的防着,前途又被人攥在手里,在宰执们面前,再受宠的内侍也硬气不起来。

默念了两遍,韩冈记下了这个名字。从今以后,只要留意此人的动向,要找出幕后黑手,也不会有多少难度。

韩冈有时候会很健忘,但有的时候,记性可是会变得十分好……

