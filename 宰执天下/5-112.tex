\section{第13章 羽檄飞符遥相系(一)}

“大白髙国已经完了。”

“这时候赶回去,只会是送死。”

“我不会让族中的儿郎跟着大白髙国一起陪葬。”

从厅中传出来的声音,让外面的守卫吃惊的站直了身子。这是枢密使仁多零丁的声音,没有丝毫激动,只是在用平静的语调叙述。但他说出来的话,却让守卫的手不由自主的握紧了腰间的刀柄,

与磐石般立在大厅中央的仁多零丁对视了片刻,梁太后支持不住挪开了视线,压在膝头上的手也颤了起来。自从收到辽人突破北疆的消息之后,她就知道外姓诸部肯定会翻脸,只是没想到会来得这么快。厅中所有的外姓将领全都保持沉默,没有一个站出来指责仁多零丁。

梁乙埋高声大喝:“灵州之战都撑下来了,盐州城也攻下来了,难道还怕他辽人不成?他们离着兴庆府还远着呢!”

靠梁乙埋插了一句,梁太后缓过神来。看向另一位外姓的领军人物,带着仅存的期冀:“叶卿,你素来坚韧,当不会畏惧辽人!”

叶孛麻古拙的面容就像是石雕一般毫无表情:“太后、国相,你们还想维持大白髙国的体面,可宋人和辽人都是不会答应的。已经不是景宗皇帝的时候,靠着太祖【李继迁】、太宗【李德明】几十年的积累,南破大宋,北克大辽。这十几年,一点家底都消耗得一干二净,如今就是辽人不入寇,国势也是支持不下去的。”

叶孛麻和仁多零丁两人进来前并没有事先商议,当下同时发难,却是默契非常。

嵬名济从喉咙中挤出声来,低沉中带着满满的杀机:“仁多零丁,叶孛麻,知道你们站在那里吗?”

仁多零丁头也没回:“嵬名济,不要做蠢事。若我半个时辰后不能安安稳稳的从这里出去,保忠那孩子就会领兵攻过来,你是想要跟我仁多家的子弟兵拼一个高下吗?”

嵬名济眼神一寒,牙缝中咝咝有声:“你这叛逆以为我不敢?!”

“种谔就在外面虎视眈眈。”叶孛麻挪了一步,与仁多零丁并肩站着,“要是在这里火并起来,你嵬名家的兵纵然能杀光外姓诸部,也逃不过种谔的追击。”

嵬名济眉头拧起来了,梁乙埋见状,生怕他脾气上来当真要拼个鱼死网破,连忙拦着:“都这个时候,同舟共济才是应该做的。不论是回师抵挡辽师,还是干脆投效宋人或是辽人,大伙儿抱成一团,才能挣个体面。否则你打我,我打你,不是让宋人辽人去做渔翁吗?”

之前派出去的信使没有回来。梁乙逋和仁多瀚都被扣了下来,连随从一起都被扣了。梁乙埋本想着还能趁机激起同仇敌忾的心思,可没想到,人还没有回来,这边就已经图穷匕见。他一边说着软话,一边想着该如何解决问题。

要不要以缓兵之计拖一拖,然后出手杀光这群叛逆?还是退让一点。梁乙埋飞快的考虑着。

仁多零丁既然亮了刀子,便没有耐心等待:“我等本是羌人,与吐蕃源出一流。而嵬名氏乃是鲜卑种,景宗皇帝【李元昊】不是自称是北魏帝胄,拓跋后裔吗?借着我党项人的力量,你们鲜卑人已经做了几十年的皇帝,应该也够本了。”

仁多零丁说完就抿起了嘴。他曾经跟着景宗皇帝起兵立国,接连三次大败了宋军,之后与亲征的辽兴宗几番大战,又将辽人赶了出去。向西攻下了甘凉,向东也在河东咬了一口。当是尽管只是景宗皇帝麾下并不起眼的一名将佐,但当时的意气风发,便是现在也依然能记得清清楚楚。

在那个时候,怎么也没想到会亲眼见证灭国的一天。在作出决定时,仁多零丁考虑再三,可事到临头,反而就没那么多的感触了。仁多零丁在心中暗叹,终于走到了这一步,也没什么好犹豫的。

正式派遣的信使没有回报,但跟随着仁多瀚的随从,却悄悄潜回来了一个。想必叶孛麻他私底下也有派出人去联络种谔,当也带回了种谔的回复。并不是梁氏兄妹和嵬名济等宗室将领期待的结果。

这也是理所当然的。旧怨不说,一场大战才打得两边死伤惨重,怎么可能转眼间就联起手来?何况以仁多零丁对宋国的了解,种谔根本不够资格掺合对外的交涉。给出的回答,当然也就是以挑起西夏军中内乱为目的。

仁多零丁抬眼望着梁氏兄妹。得到了嵬名家的支持,他们才坐稳了位子,但他们给西夏带来的结果,却是灭亡。

‘投名状吗?’

种谔的要求可谓是用心险恶。

辽人的背后一刀刺中的不仅仅是大白髙国的背心,也直接打到了种谔身上。宋国的天子不会不去想,要是种谔不故意拖延,帮着徐禧守住盐州,等到辽人偷袭消息传来,就不会有盐州的陷落,也不会有这么惨重的伤亡。盐州城中的宋军还没被杀光,但剩下的也不多了。种谔是必须要有一个像样的功劳,这样才能挡住一切针对他的攻击,他如此强硬,也不是没有理由。

仁多零丁没兴趣按照种谔的心思去走。好合好散,跟嵬名氏争斗的结果就是种谔一家得意。反目成仇也许免不了,但无论如何直接动手的结果,仁多零丁都想尽量避免的。

当然,如果种谔之后为了功绩而攻击宗室所部,仁多零丁肯定是会做壁上观,不会干涉半点。

再没有什么好说的了,仁多零丁和叶孛麻告辞离开大厅,紧跟在他们身后,是所有的外姓朝臣,就连身为汉人的李清,也跟着一起出去了。

留着梁氏兄妹和宗室诸将在身后咯咯的咬着牙,仁多零丁跨出厅门。

该让梁氏兄妹和嵬名家诸将想想前路了,还有该怎么收拾军中。班直和环卫中的成员,泰半是各部贵胄的子弟和各部挑选出来的武艺出众的部众,接下来必然是分崩离析。

守在门外的班直侍卫对几名叛臣怒目而视,仁多零丁瞥了他一眼,年轻的面庞上是被背叛的愤怒。

‘年轻人啊。’仁多零丁感叹了一声,便把他抛去了脑后。

他还要考虑仁多家的前途,没时间多费心神在无关紧要的小人物上。

仁多零丁和叶家一样,无论如何都不可能投靠契丹人。尽管叶家的家底几乎都在兴灵周边。而仁多家也是前两年从宋夏边境的静塞军司撤到了兴灵。辽人攻下兴灵之后,他们两家的地盘将不复存在。可一旦投向契丹,那肯定是要遭受比过去重得多的盘剥——在西夏,大部族等于是与嵬名家共治国家,被盘剥的永远都是小部族。

对于人丁和地盘的矛盾,叶孛麻和仁多零丁都有极为清醒地认识。地盘从来不是问题,有人就有地盘,仁多家当初从边境撤回兴灵,为什么没人敢争?因为他手中有兵有将!

过去就是因为大部族所受到的盘剥要轻得多,所以经常有小部族投靠到两家的名下——就像宋国的平民寄田或投身到官宦人家的名下——尽管也要受到大部族高层的剥削,但总比之前要轻不少。可若是税赋盘剥重了,帐下的人口便会纷纷散去。

“仁多公。”叶孛麻走在仁多零丁的身边,低声道:“其实国相有一句话说得没错。若不能同舟共济,你我两部覆亡就在转眼之间。”

叶孛麻要定下盟约,正合仁多零丁的心意:“说得也是。过去你我两家的确是有些不愉快的地方,但从今之后,同进共退。”

事机紧急,并没有时间歃血为盟,叶孛麻和仁多零丁也不在乎这些繁文缛节,眼下共同的利益就是保障。

“如今的局面,契丹人是磨盘的上半扇,宋人是下半扇,我党项诸部就是磨盘里面的麦子,要被磨成粉的。想必仁多公肯定是要投靠宋人的,但完全依靠宋人,还是让人放心不下,何况家里的儿郎也得有一个落脚的地方。”叶孛麻看了眼身后,见其他部族都识趣得离得略远,方才问出重要的一句:“不知仁多公对此有什么想法?”

仁多零丁立刻回道:“召集你我两部在兴灵的部众,立刻撤过青铜峡。如此才有一线生机。与你家的打算一样。”

叶孛麻会心一笑,“眼下也的确只有这条路了。”

青铜峡是黄河进入兴灵平原的最后一道峡口,离灵州只有百里,兴庆府也只在两百里外。如果能及时占据青铜峡上游的谷地,依靠这个战略要地,与宋人讨价还价的资格也就有了。而且由于是战略要地,宋人日后肯定要在峡口处修建寨堡,派驻大军,到时候就不用再为抵抗辽人而费神。

“辽人南下,兵力必然不足。占了兴庆府和灵州之后,不可能分兵太多。要撤出去,没那么难。”

叶孛麻与仁多零丁有着相同的计划,没有多话:“我这就去安排人手,立刻赶回去。”

还来得及,两家的部众都好调集。而青铜峡的守军,鸣沙城的守军,主力都不是嵬名家的部众,这样一来,要占据那一片地难度也不算大。

