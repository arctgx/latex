\section{第一章 纵谈犹说旧升平(二)}

苏缄听得目瞪口呆,京城人的想法当真是让人捉摸不透,这打招牌的方法,亏他们想得出来。

苏颂啧啧叹了几声,又道,“飞船不好在船上生火,不然就会太重。但热气球容易,本来就是跟孔明灯一样,里面装了油、点了火,带条绸缎上天,能在空中悬上一两个时辰。若是到了夜间,气球中的灯火映出来,就宛如天上灯市。”

苏缄听得悠然神往,连声感叹。他的孙女儿则是趴在车窗上,一直在抬头看着天上随风轻舞的气球。

一路到了驿馆门口,苏颂和苏缄前后下了车。他们在驿馆中留个姓名,就能去苏颂府上住下了。

只是甫下车,就见到一名内侍在驿馆门前守着。

那名内侍显然是认识苏颂,见了人便双眼一亮,立刻小跑着过来。并没有照规矩行礼,而是在苏缄苏颂二叔侄挺直了腰,高声问道:“可是邕州知州兼广西钤辖、皇城使苏缄?”

一听问话中的称呼,苏颂苏缄便知这名内侍必然身负皇命。

苏缄上前一步:“正是苏缄。”

“奉天子口谕,诏苏缄抵京后即刻入宫觐见。”

苏缄也不惊讶,从今年年初开始,交趾国中的小动作便越来越多。单是他呈上去的奏折,就差不多有十几份,都是提醒天子,要加强戒备,并且请求天子下诏,让广西经略、同时也是桂州知州的刘彝不要再做蠢事。对于那个南方小国,朝中提防得很厉害,天子也十分关心。苏缄三年四诣阙,每年差不多有三分之一的消磨在路上。

他就在驿馆大门处行过礼:“臣遵旨。”

起身后,苏缄低头看了看身上的衣服,对内侍道:“黄门权且少待,等苏颂沐浴更衣后,便去宫中觐见。”

衣冠不具,身体不净,当然不能见天子,这是大不敬。虽然口谕中有着‘即刻’二字,却也不是急在这个地方。传过口谕,内侍的态度变得谦卑起来:“皇城请便,小人就在门口候着。”

苏颂正要送着苏缄入内,但内侍这时又转过来对着他道:“苏学士,陛下也有口谕,诏你入宫备咨询。”顿了一下,低声道:“是军器监里的事。”

苏颂点了点头,示意自己也听明白了。招来一名元随,吩咐他快点回府去取公服来。转身对着惊讶的苏缄一笑:“这样比回去换衣要快上一点。”

叔侄二人一起往驿馆中走。听到了外面的动静,被惊动的驿丞忙迎了出来。点头哈腰的为两人——主要还是苏颂这位集贤院学士——准备下了更换衣袍的房间。

苏缄方才听到了内侍对苏颂的传话,心中藏了几分诧异。方才在车上,他听说了苏颂即将调任应天府,也就是南京【今商丘】,与军器监根本没有干系。等着身边没了外人,他便问道:“前面子容你不是说要去南京应天府吗?怎么又跟军器监里有了瓜葛。”

“是为了水轮机。”苏颂苦笑了一下,“侄儿治学不精,一向心有旁骛,学得东西驳杂了一些,也不知什么时候传出了个博学的名头。弄得连朝廷要造器物都问到了侄儿的头上。”

“水轮机?”苏缄哈哈笑道:“难怪要问你。机械上的事,问别人都不如问子容你了。”

苏缄很快就换好了衣袍,而苏颂遣回家中的元随也很快带着他的一身穿戴回来了。各着朱紫,苏氏叔侄便在内侍的引领下,上马前往宫中。

一路进了宫中,天子正在殿中议事。苏缄、苏颂就被领到崇政殿外的阁门中等候传唤。两人刚到,正好就见到一人从前面的回廊转过去。是一个很年轻的官员,身材高大挺拔,穿着朱袍,腰悬鱼袋。

苏缄看得惊讶无比:“怎么宗室都能这时辰上崇政殿?”

“不是宗室,他就是军器监的韩冈!”苏颂笑了一笑,“才二十三,就已经赐了五品服色,正七品的起居舍人了。也难怪二十六叔你会误会。”

“哦……他就是韩玉昆啊!”苏缄略略拉长的语调中有着说不清道不明的味道,从心底里为着韩冈的年轻而惊叹不已。

自己在官场混迹四十年,同样也是进士,如今却落得转为武职,而且还仅是个正七品的皇城使,还不知哪年能熬上横班。不过苏缄倒也没有什么嫉妒之心,到了他这把年纪,少年时争强好胜的心情早就没了,一切早就看开了。等做完这一任,看看交趾人老实下来,就上表致仕,回老家养老好了。

苏颂仔细看着苏缄的脸色,见他对韩冈没有多少芥蒂:“二十六叔你若在交趾之事上有什么想法,如果正途不行,可以问一问,他如今在天子面前能说得上话的。”

苏缄听着苏颂的口气,似乎跟韩冈有几分熟悉:“子容,你与韩冈很熟吗?”

“水轮机的事还是韩冈先提起来的,就是为了能带动锻锤。而军器监新造的几具锻锤,天子也让侄儿来评鉴过。这月来跟他在崇政殿中见过几次,前两天,韩冈还来拜访过侄儿。”

“子容……韩冈为人如何?”苏缄问着苏颂,微沉的语气,似是有着些想法。

“为人也算是正直,至少是不忘本,举荐其师张载不遗余力。”

天地君亲师,尊师往往能与忠孝并提,韩冈一直以来不惜与王安石反目,都要推荐张载和气学的作为,其实为他博得不少赞誉。苏颂也是因为此事,而对韩冈有所赞誉。

“而且闻一知十,才智高绝,的确是难得一见的大才。前些天与他见面的时候,说起了算学上的一些事。想不到他在算学上,也有着别出一番心裁的见解。”

苏缄吃惊不小:“他才二十多岁吧,就连算学就精通了?”

苏颂摇摇头:“算不上很精通,但他简化了九章算经中的一些算法,本于‘天元术’【注1】,却更为完备。这套简化算法,可以推而广之,就像出去砍柴,手上多了一把好斧子。说真的,能想出这套算法,韩冈的确是高人一等,可惜使用不当,未有深究,完全是明珠暗投啊……若是使用得宜,九章算经可就要大改了”

苏缄对算学一窍不通,九章算经都没怎么看过。苏颂这个侄儿上知天文下知地理,博学闻名朝中,在算学和机械上是数得着的人物。看他说话时惊叹连连,尽管之中也有微词,但也可见韩冈的确得到了苏颂真心的认同。

苏颂见者苏缄若有所思,便问道:“不知二十六叔今日廷对有什么打算?”

苏缄也不瞒他:“桂州刘彝禁绝与交趾的交易往来,这点绝不可行,这等于是将边地所有的部族都推到交趾那边去。但整顿武备,还是该做的,已经不能再拖了。”

“桂州不是已经在练兵了吗?”苏颂奇怪的问着。

“练得应该是汉兵,而不该是溪洞土兵!”苏缄狠狠说了两句,转过话锋,“军器监的板甲还有神臂弓,最好都能下发一批到邕州的武库中来,在广西,只有汉兵才最为可信,只可惜现在的广西军是军令驰废,兵甲不精,不堪一战。前后两任经略,都只想着靠土兵来作战。”

两人正在说话,一名内侍过来通知,让他们去崇政殿外排队。苏缄苏颂都有些惊讶,他们觐见天子不是为了一件事,怎么一起得了通知。不过也不是什么值得大惊小怪的事,起身随了内侍往崇政殿去,远远的就见着韩冈立于殿门口等候传唤。

一见苏颂,韩冈就过来先行致礼。苏颂是庆历二年的进士,论辈分与王安石一代,韩冈也不敢失礼:“韩冈见过学士。”

苏颂拱手回礼,听着殿中似乎有争执声,他有些纳闷,“怎么回事?”

“原本该出来的,门都开了,但不知怎么的又争起来了。”韩冈叹着气,视线一转,转到了苏缄的身上。

苏颂为之介绍:“此乃家叔,现任邕州知州。”

“邕州?”韩冈一望苏缄,便又与他互相行礼。

等到重新立定,苏颂低声问道:“今日玉昆上殿,可是为了板甲局中事?”

韩冈点头而笑:“板甲局粗有雏形,一个半月的时间,已经打造板甲整两千套。”

而且这一个多月的时间,板甲局中各个作坊已经磨合习惯,正是全速开工的时候。兴国坊内,板甲局所在的那片区域日夜烟火不绝,叮叮当当的敲打声从来都没有断过。差不多快要到达一天三百件的第一期预定目标上。

“能不能给邕州下拨一批板甲?”苏缄在南方久了,说话做事一向很直率。

韩冈顿时面现难色,这不归他管啊,“此事得请于东西二府。不过据韩冈所知,板甲一旦下拨,当会以京营和陕西为先,河东河北次之。”

广西的禁军才多少?南方诸路的禁军人数,加起来还不到北方的十分之一。

天下禁军,三分在京中,三分在关西,河北加河东也占了三分,剩下的一分,就是零散的分布在南方各路。而且这些禁军,说起打仗只能摇头,论起吃空饷,则是一个胜过一个。怎么都轮不到。

至少在北方禁军换装之前,南方是没有半点机会的,就连韩冈都无法控制。不过对于苏颂,韩冈最近正想结好于他,面子不能驳,“这样吧,韩冈可以在监中设法挤出一批神臂弓来,只要经过中书批复,就直接给邕州发过去,不会耽搁。”

苏缄听得大喜,他求得就是此事。阎王好过小鬼难缠,许多时候就算打通了高层,下面也会给添乱。要说服天子容易,让中书宰辅点头也不难,但让下面的监司做事麻利点,可是千难万难。眼下有韩冈的承诺,就可以放下一半的心了——至少苏颂也说了,韩冈的人品不差,不至于会不守信诺。

几人在殿外又等了一阵,始终不见殿门打开,只听着殿中的争论声越来越大,就是离着殿门远了听不清楚,苏缄很有些纳闷:“里面究竟在说些什么?”

韩冈轻轻摇头,神色中有几分不以为然,轻声道:“是李逢谋反案!”

注1:天元术,是中国古代利用未知数列方程的一般方法,与现在代数学中列方程的方法基本一致,只是写法不同。其起源大约就是在熙宁年间。

