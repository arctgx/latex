\section{第39章 太一宫深斜阳落(一)}

【元旦第三更,求红票,收藏】

在京兆府度过了不眠的上元之夜,次日章俞的来访,虽然并没有时间加以深谈,但已足以让章俞将韩冈和刘仲武两人的名字铭记在心。别而后遇,韩冈的这一番做作,给人留下印象其实更为深刻,章俞的态度也便更为殷勤。

章俞邀请韩冈他们一起同去京师,只是由于行程的速度实在差得太远,两边还是无法同行。章俞又要赠钱赠物,但反应过来的刘仲武不待韩冈提点,也是自觉自愿的推拒所有的赠礼,这让章俞更加敬重。到最后,章俞几乎是强逼着韩冈和刘仲武答应,到了京城后一定要到他家中坐上一坐,方才殷殷而别。

“多谢韩官人。”回想起韩冈昨天说过的的话,刘仲武才深切的体会到韩冈的先见之明。他的道谢真心实意,没有半点虚假。

韩冈呵呵的笑了笑,很亲近的拍拍刘仲武的肩膀,“无妨,勿须在意。”

别过章俞,又被种家叔侄送出城门,韩冈一行继续启程。接下来一路,便是无惊无险,经过三百里潼关道,很顺利的抵达西京河南府,也就是洛阳。大宋西京,历史不逊长安,比起长安又更为繁华,甚至还有宫殿楼宇,不过韩冈他们也无暇游历。在驿馆中住了一夜,第二天便又由洛阳出发。四名骑手在中原大地的广阔平原上疾驰,数日之后,韩冈一行,终于来到了开封不远处的八角镇【今开封八店村】上。

离着京城只剩三十里地,但此时天色已晚,日头已经压在地平线上。即便现在以最快速度从八角镇往开封城去,也来不及赶在城门关闭前抵达城下。无如奈何,韩冈他们也只能在八角镇住上一夜,等明日再进城。

八角镇内并没有驿馆,韩冈一行便随便找了个看起来还算干净的脚店住下——世间的习俗,通过官府准许可以自行酿酒的酒楼,称为正店,而普通的小客栈,则称为脚店。京城中有七十二正店,而八角镇,就只有脚店了。

入店要了房舍,刘仲武便一头钻进马厩里照料他的爱马——一匹好马价值千金,刘仲武走了狗屎运才弄到的这匹河西良驹生了病,他简直比死了老子娘还要伤心。韩冈将行装安顿下来,过来找他,就见着刘仲武哭丧着脸,拿着不知从哪里弄来的药膏,要往赤骝的蹄子上抹。

刘仲武的赤骝在路上跑得太久,一千七八百里路,四只蹄子的蹄壳都磨掉了许多。前两天就已经有些跑不动了,在后面拖着,害得韩冈他们每天都是将将好才赶到驿馆中。

北宋还没有发明马蹄铁——至少韩冈还没有见过,赤骝的四条腿下面也没有安装——长距离的行动对战马四蹄蹄壳损耗很大,而在南方湿热的地方之所以难以养马,也是因为湿气容易伤了马蹄。

而韩冈知道什么是马蹄铁,也清楚大致的用法和形制,以大宋工匠的平均水准,按照要求打造几个急就章的蹄铁,钉上去也许不容易,但烙上马掌去却不难。如果韩冈前两天就告诉过刘仲武,在一路过来的铁匠铺中,连夜打上几对,说不定今天就不会来不及赶到京师,但他自始至终没有向刘仲武透露半个字。

就像马鞍和硬质马镫对骑兵的意义一样,马蹄铁也是能大大增强骑兵的战斗力。在还没有出现马镫、马鞍的汉代,手持重弩的汉军,可以以一当五的击败匈奴骑兵。而在群雄纷争的汉末,汉人照样能把北方的乌桓骑兵追着打。可到了出现了金属马镫的南北朝以后,北方游牧民族与南方汉人之间的战力对比渐渐颠倒过来。

当然,韩冈不会因为这个原因便放弃推广马蹄铁的使用。这样很愚蠢。已是公元十一世纪,西方应该已经出现了马蹄铁。如此有用的装具,迟早都会在东方流传起来。要想战胜敌人,不是将新武器深深掩埋,而是继续创造出更有威力的武器。

韩冈的想法只是不想让马蹄铁提前泄露出去,等他正式得受官职,开始辅佐王韶用兵于河湟。那时再放出来,由此挣到的军功,可比刘仲武的一点惊叹有力的得多。

韩冈在马厩外面看了看刘仲武悲痛欲绝的样子,心中也微觉歉然,觉得这时候还是不进去找他的为好。转回店中,路明走了过来:“韩官人,现在天色尚明,不如去逛一下镇中的西太一宫。虽然那里没有什么古物,但宫中的几株老梅还是值得一观。”

再过十天省试便要开始了,而路明却貌似全然不放在心上,俗话说临阵磨枪,不快也光,路明连佛脚都不肯抱一下,连复习都不作,真当自己是章惇那种想考进士就能考进士的奢遮人物了?韩冈暗自摇着头,对路明考中进士的机会又看低了几分。

既然路明本人都不在意即将开始的考试,韩冈也没有替他担心的道理。左右无事,他便留了李小六在房中看守行李,会同路明一起,往他所说的西太一宫而去。

镇外不远处的西苏村头便是西太一宫,于此相对的还有一座中太一宫,位于开封城中东南隅。为熙宁初年修建,最近刚刚落成,祭告时还死了一个三司副使,说是吃胙肉吃出了毛病,七窍流血而死——韩冈却想不明白,为什么三年未至京师的路明能知道这么多。

两座太一宫,其实就是祭祀东皇太一的神祠。太一又名太乙、泰一,史记有云:‘天神贵者太一’,是为天帝别名。屈原所著的楚辞《九歌》中也有《东皇太一》一篇,在中国的神仙谱系中排位很高。只是供奉太一的香火并不旺盛,还不如一般灶神,城隍,更不如如今世所流行的二郎神、紫姑神等莫名奇妙冒出来的神灵。所谓县官不如现管,大略便是如此。

尽管香火不盛,可太一宫毕竟是在祠部司中列名的道观,比韩冈老家的李广庙要大得多。但是在宫内洒扫庭院的火工道人就有十几个,由一个领着朝廷俸禄的庙祝管理。而韩冈从王厚和路明这里都听说过,朝廷中还有一类名为提举宫观的官职,专门用来安置贬斥或是求退的官员,类似于官场中的养老院,后世政协一类的地方。

这座宫观既然是隶属于官,当然也讲究着门面,殿宇重重,也有大小十几栋之多。主殿高达四五丈,单是露在外面的几根立柱就比两人合抱还粗。

“西太一宫这主殿虽然不大,装饰又乏华彩,可却是当年预都料亲自监造,坚实无比。当日主殿架梁,俞都料亲自把大梁放正,他从殿上下来,直说除非火焚地震,否则此殿千年不坏!几十年来,此殿数遇雷击,却当真一点事也没有。”

路明介绍起来,言辞引人入胜,像个标准的的地陪导游。不过他口中说的俞都料,韩冈则是一头雾水,便向他询问。

路明解释道:“就是都料匠俞皓,国朝以来木工第一人,号为当世鲁班。如今有三卷《木经》通行于世,天下木工皆以其为法度。”他指着东面的开封城,“开封城里的开宝寺塔便是俞都料所亲造,塔初成时,倾于西北而望之不正。朝中欲问罪,俞都料则道:‘京师平地无山,而多西北风,吹之不百年,当正。’”

“俞皓?”韩冈念着路明提到的姓名,莫名的有些耳熟,就是一时想不起来。若传言是真的,还真是不得了的名匠。他听得有趣,便问着:“那开宝寺塔现在呢?正了没有?”

“正好一百年的时候,给一把火烧掉了,那是庆历年间的事了。不过在这之前的确正了”路明手指上下比划着,“直直向上,一点也不偏。俞都料言之如神,所以啊如今京师里面却多了一层担心。”

“担心什么?”

“担心再过七八十年,京中的寺塔会不会都向东南面倒!”

韩冈听得哈哈一笑,路明这包袱抖得当真有趣。

路明陪着韩冈笑了一阵,继续道:“俞都料只有一女,据说已得其亲传,技艺不输乃父。有传言说《木经》三卷,其实是出自她手。后来招了赘,现在其后人应该还在京中。”

韩冈脚步顿了一下,他终于记起在哪里听说过俞皓这个名字。这不是他上学时出现在课本中里的那位俞皓吗?节选自沈括的《梦溪笔谈》中《梵天寺木塔》一篇古文,当时自己还是背了下来的。想不到俞皓不但在吴越国修过塔,在开封府也一样修过塔。能名传千古,能力当然不差。

谈笑间,两人走进主殿中。东皇太一的神像高居殿中,装饰得金碧辉煌。只是一张富态的圆脸下留着三缕胡须,这相貌却与韩冈见过的其他神像,如同一个模子映出来。

站在香案前,两人各自上前敬了一炷香,便跪下来行礼。瘦瘦高高的庙祝站在一旁,等着两人的随礼。

“东皇大帝在上,信男路明拜于驾前……”路明跪在蒲团上念念有词,而韩冈虽也跪了一跪,却是在四处张望。的确如路明方才所说,殿内没有什么装饰,至于建筑结构,韩冈毫无了解,也看不出俞都料的手段究竟是如何精妙。

