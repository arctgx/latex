\section{第48章 浮云蔽日光(上)}

郭忠孝放下了手中银杯,刚刚咽下的酒浆还在喉咙里烧着,几位同伴又拿着酒壶给他的杯中满上,“立之兄,多喝一点。高阳正店的醉缪,到了太原可就难找了。”

郭逵受了皇命,要去做太原知府。只是他在京中一坐一个多月,直到正月月底了,方才开始准备动身。

东京富丽繁华,又能亲近天子,许多官员都不愿出外任职,即便调任外职,也会拖着出外的时间。拖得时间长的,三五个月都有。

这样的现象,尤其以重臣们为多。郭逵打算等着正月过后再上路,他在外镇守四方多年,留京一两个月,天子都不好意思催着他这位重臣,最多也就一两个御史说些闲话而已,郭逵哪里会在乎。拿着黄河河冰正在解冻为借口,硬是坐在东京城中不动。

也就是时近二月,郭逵静极思动,无意在京中多留,也不管黄河还没有完全解冻,就要离京北上。

今天高阳正店中的宴席,就是为了给郭忠孝饯行而设。郭忠孝虽是将门之后,却是拜在二程的门下。结交的友人也都是文臣家的子弟,而非是将门的衙内。

不过宴上话题的主角却不是郭忠孝,除了倒酒、敬酒,尽是在说着在宣德门上拿了板甲出来,让宰辅们面目无光的韩冈。

一人放下了酒杯,带着几分醉意:“韩冈明知道铁船造不出来,只是玩个噱头而已,其实早就是在准备打造板甲了。什么日渐日新,骗鬼的……”

“那又怎么样,二府诸公不都上了当?朝中谁没给他幌了?何六你难道没上当?也就韩冈一人在肚子里暗笑着。”另一位双眼凸出,看人都是半眯着,近视得很厉害,但他的声音够大:“《浮力追源》说的似乎有那么几分道理,京里京外都以为韩冈造铁船来作为证明。谁想到铁船造不出来,但板甲却出来了。”

“陈定夫说得没错。韩冈为人狡狯无比。恐怕政事堂中两相两参哪个都没想到,他争判军器监这个位置,最后会是为了这个结果。”三十多岁,有些富态的中年人失声笑道。

陈定夫眼睛几乎眯成了一条缝:“要是知道韩冈是为了造板甲才去了军器监,吕惠卿会给他立功的机会?就是因为以为韩冈是要造铁船,所以才放了下心来,准备看笑话呢?”

“上当的不只吕惠卿一个,政事堂中其实还有一个上了当,最后偷鸡不成蚀把米。”富态中年身子往前凑了一凑,声音也低了点,“你们知道在上元节上,将灯船拿出来的究竟是谁?”

郭忠孝终于开了口,疑惑的问道:“难道不是韩冈主持的吗?”

“当然不是!”富态中年一口否定:“先是军器监的旧灯山在年节时坏了,那时韩冈还没正式去军器监上任。主持赶制新灯山的也不是他,而是军器监丞白彰。等灯船打造好的时候,都已经是正月十二十三了,韩冈和曾孝宽也就是这个时候方才看到。如果韩冈没后手,他即便毁了灯船重头再改做另外一具也不可能再来得及。到时候,造不出铁船,韩冈哪还有面目留在京城?天子也不会饶他。这算计得好是很好,可谁能想到,这却正落入韩冈下怀。”

郭忠孝狐疑着:“宾之兄,不是不信你。总觉得这事未免有些太牵强了!”

表字宾之的富态中年显然在官场上耳聪目明,冷笑着:“判军器监丞白彰已经要调任岭南监弓弩院了,你说是真是假?还有一个令史,也一同去了岭南。他们两个就是管着造军器监灯山的,他们的调职是韩冈的推荐。荐章上说二人打造灯山得力,举荐他们去了岭南任职。”

席上一片沉默,好半天才有人开口:“……好狠!”

“中书怎么会答应?”郭忠孝更为不解。

宾之笑道:“立之你难道还不明白?就是中书四人中的一位下得手,韩冈只是在报复而已。这件事,韩冈不怕闹出来。争到天子面前,倒霉的绝不会是他。所以中书才匆匆忙忙的准了这份荐章,要不是宰辅之威,岂能压得住白彰两人接受这份任命?”

“……此人到底是谁?”连方才带着醉意的何六,这时候也清醒了。

“谁批复的,谁就是灯船一事中的幕后人物!”宾之冷笑着,“你以为政事堂中的四位宰辅之间有多和睦,会为对方遮掩?韩冈是看准了时机递上去的。”

又是一阵沉默降临厢房之中。在座的都是官宦家的子弟,政坛上的勾心斗角也都看多了、听多了。但小小的判军器监与宰辅之间互相较量,非但不落下风,反而让人自食苦果,不得不学着蜥蜴断尾,这手段未免太过惊人。

“说那么多做什么?”列坐的五人中,唯一一位没有说话的拍起了桌子,“韩冈是奸猾没错,但他的眼界未免也太小了一点。拿着格物致知当幌子,但铁船说出来却做不到,要拖个十几二十年,甚至几十年。这一下,韩冈本人是春风得意,但你们再去看看还有谁去信张横渠的关学?”

“……这话尤公休说得对,韩冈的确是只看顾着自己。”何六点着头,“将‘格物致知’变成了踏脚石,说不定张载会气得不认他这个弟子。”

尤公休冷笑声中带着不屑:“人之所以为奸便是如此,无物不可利用,却不知正心诚意四个字,是跟格物致知写在一起的。”

韩冈少年成名,又是做了宰相家女婿,嫉妒者本就为数众多。现在找到了错处,哪还会有好话?

但对韩冈的攻击,郭忠孝却没有参与进去。当日他随父亲郭逵在大相国寺看见韩冈时,韩冈正逛着一家家货摊,还买了一套孔明灯。问他做什么,他却是说在买船。

怎么想都觉得有什么地方不对劲,韩冈的话似乎藏着深意,让郭忠孝隐隐的觉得答案就在这里。但偏偏就像隔了一层窗户纸,模模糊糊的没办法直接触摸到真相。

想着想着,郭忠孝的眉头就不禁拧了起来。

“立之,怎么了?”宾之问道。

“没有!”郭忠孝惊醒过来,摇摇头,“没有什么!”

但旁边的何六一拍桌子:“啊,是我们错了。今天是要给立之兄饯行,提韩冈那个厌物作甚?”

宾之这位富态中年立刻作了恍然大悟状,连忙道了一杯酒,敬向郭忠孝:“立之勿怪,愚兄在这里赔不是了。”

忽然下面大街上一片骚动声传了上来,隔壁的包厢中,接二连三的想起推开窗户的声音。

尤公休站起来,将紧闭的窗扇打开一条缝,寒风顿时从缝中刮了进来,而更为响亮的喧哗声也一起进来了。顶着寒风向外看去,只见下面黑压压的一片人,都仰着头向天上看着。

尤公休拉开了窗户,探出头,就看见隔壁包同样也在向天上望着。他顺着大众的视线望过去,顿时就是一声惊呼。

“出了什么事?”几人站起身,一起涌过来窗户边。

“怎么这么多人?”何六扶在窗台上,见者下面黑压压一群人,先是惊讶了一下,然后抬眼上望,便与郭忠孝、宾之还有陈定夫齐齐的瞪大了眼睛,“那是什么东西?!”

离着高阳正店差不多有五六十步的地方,有一个盘子那么大的异物,悬在二十多丈的空中,上圆下尖,不知道从哪里来,也不知道究竟是什么。

陈定夫眯着眼睛,只看着空中有个黑黢黢的东西漂浮着,就是看不清细节。但他有办法,从怀里掏出来个银圈雕花的水晶镜来,扣在右眼上——无论是放大镜还是眼镜,如今都已经传到了民间,不过能配得齐这两样东西的,也只有富贵人家——这一下子,看得也稍微清楚了一点。

那异物是个鼓鼓囊囊的球,下面垂下来十几条绳索,吊着个似乎是篮子一样的东西。从距离和下方的屋舍来判断,飞在空中的那一颗球至少跟房子一般大小。如此巨大的异物悬在空中,多半大半座京城都给惊动了。

“那个到底是什么?!”

盯着半天,还没人分辨出来那是个什么东西。非鸟非虫,更不是云翳,也不见有多大的风,能将如此巨.物卷上天空。

忽而一阵风刮起,天上的那颗球,向西漂了过去。街上的围观者大呼小叫的蜂拥而去,紧追着不放。门外亦是一阵脚步声,砰砰砰的从厢房外的走道上跑过去,转眼就看到一群人跑出门外。

“跟着去看看吧?”何六回头说着,也不等答话,推开门,就在门外守候的伴当惊疑的眼神中,砰砰砰的也冲下了楼去,其余几人也紧随其后。

看着空空如也的包厢,郭忠孝叹了一口气。举步出门,吩咐了伴当为自己的饯行宴会钞,也一起跟了下楼。不知是不是直觉,郭忠孝觉得方才看到的哪一个异物,肯定跟着韩冈有着脱不开的关系。

