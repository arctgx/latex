\section{第21章 欲寻佳木归圣众(十)}

‘闹剧可以结束了。’

看了看周围陡然间静下来的同僚,李格非想。

朝中的台谏官来了有三成,寂静无声,仿佛中午吃饭前的御史台,御史中丞不开口,所有人连咳嗽都不会有一声。

也就在片刻之前,开封府东城颇有些名气的沈家园子,还喧闹非凡,一群在外面总是黑着一张脸的御史们,正聚在一起,大小声的议论着他们近日的目标。

“沈括这一夜多半会睡不着了。”

片刻前的月下,有人捂嘴轻笑,有人纵声喧嚣。对外开放的私家花园,不像一般酒楼那般多有闲人出没,包下来后,众御史不虞犯忌的言谈举止被人首告。

王中正启程离京,距离他从太后那里接过差事,连一天都还不到,只过了一夜便上路了。

王中正如此勤勉,让很多人看到了他们想看到的东西。原本还想观望的一批御史,这一下子也忍不住了。一日之间,递到御案案头上的弹章,已经有十余份了。

谁都知道,太后并非对韩冈千依百顺,有什么不合意的地方,立刻就会否定。

从沈括几次挫败于廷推,以及诸如李南公没能入主三司等事例上来看,太后都是有自己想法的。

如今,韩冈又要推荐沈括,如果从太后的角度来看,这自是几次三番挑战她威信的举动。

国初,赵普为相。几次在御前荐一人为官,而太祖始终不允,最后甚至撕了赵普递上去的荐章。可次日,赵普将被撕破的章疏贴好后,再一次递了上去,太祖皇帝迫不得已,最终还是答应了赵普的请求。但这一番争执之后,太祖对赵普的情分还能剩下多少?太祖皇帝对赵普的看法,也不是从那几坛金子开始改变的。

也许今日太后在许多地方上要仰仗韩冈,而且还要念着平息宫变的旧情,但这些情分,能比得上太祖与赵普之间的情分?需要依仗的地方,能比得上为太祖谋划,夺取了御座的谋主?

“韩冈如此跋扈,当然要让太后知道,朝堂中有不畏权相的诤臣。”

“就算他有首倡平蛮之功。可官做到了宰相,功劳多少又有何区别,一切只在圣心。”

李格非本不想来这里听人说胡话,但总有人想要拉他这个殿中侍御史出头——至少可以壮壮声势,等出事了,还可以拿来顶缸。

龚原的小心思,李格非倒是看得明白,只是他今天一时不查,误上了贼船。来到此处,也只能暗叹还是安处厚聪明,自己糊涂。

份属同列的安惇根本就没来,章惇之前就动了心思,想将他弄出去,安惇现在正设法能以一个体面的方式离开,不打算再节外生枝,一切应酬都推掉了。

而龚原,则是台中的急先锋。听过他前段时间曾经去拜访过章惇,李格非原本猜测他是不是领了章惇的命,但之后听龚原对外所说的话,却又听不出有枢密使撑腰。以李格非对龚原的了解,如果当真有章惇撑腰,动作只会更张狂。

有这么些成员,御史台的威名,也难怪越来越差。

李格非上个月还见到了回京诣阙的张商英。

张商英就在那边叹,现在御史台是黄鼠狼下崽,一窝不如一窝。

张商英在台中时,也曾经斗宰相批枢密,尽管几次吃了大亏,如今只能在外州任职,但终究在士林中有着不小的名声,在御史台中,其名号更是如雷贯耳——多少人将其引以为戒,或是嘲笑他是属猪的,只会闷头向前冲,而不懂得相机而动。

李格非不知道要怎么评价那几位眼高手低的同僚,论起相机而动,张商英比之韩琦等谦卑已经差了不止一筹两筹,而如今的御史台,连张商英的一半水平都没有。

幸而这番得意张狂的喧闹,只持续到西南大捷的新闻穿街过巷,传到了花园中。

一众御史面面相觑:“胜得怎么这么快?”

大理好歹是南方大国,幅员犹在交趾之上,而且道路更为曲折。速胜石门蕃,那是因为石门关太近,出了富顺监就到了。可去大理路途遥远,孤军深入,不是该稳扎稳打吗?当年攻打交趾,章惇在桂州,韩冈在邕州,可是整整屯了一年的兵。熊本、黄裳再出色,能比得上章、韩二人?怎么转眼之间,就席卷大理境内。

李格非冷眼旁观了一阵,起身去方便。等他回来,尚未回到饮宴之处,却不意发现龚原正与人在树下低声交谈。

“苏相致仕不远,熊本入京又只是数月之间,太后还会将沈括拒之门外?”

树影中的那人看不清眉目,听他说话又将声音压低变沉,也分辨不出是哪位同僚。但话说得没错。苏颂不日致仕,熊本又必入枢密院,只从朝堂平衡上来看,沈括的任命就是不可避免。

“太后哪里会想这么多!”龚原厉声反驳。

在朝臣看来,维持朝中简单的势力平衡,太后能够做到,要不然就不会有拒绝李南公的三司使任命,但更深一层的权力运作,太后却还差得太多。否则就不会让苏颂、韩冈执掌政事堂,而让章惇、曾孝宽来管理枢密院,这算是什么样的平衡?

“又不是熙宗皇帝。”龚原低声说道。

世所公认,比起仁宗、英宗,熙宗皇帝绝对可算是手腕犀利的君主。变法初见成效,王安石便被踢到了,换上听话的王珪。一边压制碍手碍脚的旧党,一边又压制亲附王安石的新党,直到身边都是听话的帝党,能够老实听话的继续推行他想要的新法。以熙宗皇帝的心性和手段,要不是突发风疾,之后的十几二十年,直到他驾崩为止,朝中的大臣日子可不会好过。

若拿太后与熙宗皇帝相比,其差距不可以道里计。

“那该如何做?”

“弹章也上了,还怎么退?事到如今,只能进不能退!”

李格非无声冷笑,利令智昏到了这个地步,还有什么好说的?也许做别的事情,龚原都很合适,但当一个谋士,他还差得太远。

放轻脚步,李格非悄然离开。过几日,多半就要出城给他们送别了。

李格非突地苦恼起来,家里的宝贝女儿越发的难缠,也不知道有没有空来做两首赠别诗。

……………………

沈括一夜未眠。

早上起来一照镜子,几乎认不出镜中的那人是谁。

凹陷下去的眼圈青黑,眼中则是血丝密布。皱纹更深了几分,乍看上去,老了十岁都不止。

对着玻璃银镜照着,内室中便传来不耐烦的声音,“准备好了没有?”

“好了,好了。”沈括连忙放下镜子,让人过来帮他拿朝服过来,自己匆匆忙忙的梳洗。

挂在内室门口的珠帘哗啦啦一响,中年美妇便掀帘而出,柳叶眉高高吊起,怒气冲冲:“还没好?!”

沈括最是畏惧继室张氏,催促着下人整理衣饰,用热手巾擦了脸,再用冷手巾擦上一遍,用药水急急的漱了漱口,大声道,“这就好,这就好。”

“慌什么?!”张氏挥退了手忙脚乱的侍女,亲自上来帮忙**。

沈括的身子立刻僵硬了,仿佛被蛇盯上的青蛙。

张氏冷淡的向上瞥了丈夫一眼,哼了一声,却没就此再多说,整理着衣襟,道:“该是你的,就是你的。论功劳,你比谁差?没有你在外辛苦,韩相公能这般春风得意?都三次了,每一次都不见他插手帮忙,直到路快修好了,这才点头。这样的机会有多难得,今天若不能选上,还指望他下次再发善心不成?”

最后将腰带给沈括系上,张氏翘起纤细的手指,戳着沈括的脑门,恨铁不成钢:“你还想辛辛苦苦给别人做嫁衣?”

“为夫明白,为夫明白。”沈括连连点头。不管到底是真明白还是假明白,在他这位‘贤妻’面前,沈括从来只有点头。

“唉。”张氏叹了,上前轻轻的理好沈括的衣襟,拉直抚平:“过了今日,就能有一张清凉伞了,也能堂堂正正,到了明日,看谁还敢说你是壬人?”

沈括苦笑,纵有滔天权势也难堵天下悠悠众口,但张氏的话,还是把他给触动了,“夫人放心,为夫明白。”

…………………………

“存中?你这是怎么回事?!”

宣德门外,韩冈惊诧的对沈括叫道,就连晨曦将起未起的昏暗,也掩不住沈括脸上的狼狈。

“相公。”沈括拱了拱手,苦笑着,“今日事了,不论成败,沈括都不想再来一次了。”

“放心。这一次就彻底解决了。”韩冈哈哈笑道,丝毫不在意不远处的城门下,监察御史投来的视线。

御史台那边的一众乌鸦,韩冈留着他们不过是因为没有妨碍,人畜无害罢了,有些时候,还能派上些用场。真要开始咬人,自然是一棒子打死了事。

一名名手中握有一票的重臣陆续抵达宣德门下,有的上来问候韩冈,还有的则是自矜的站在一旁。他们手上的选票,决定了沈括的命运,也决定了未来朝堂上的稳定。

再过片刻,城门一开,朝会也就要拉开序幕。

为这件不算十分重要的事情,等待得太久了,韩冈……已经迫不及待。