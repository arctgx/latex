\section{第13章 已入苍梧危堞远(上)}

天灰灰的,大概是要下雨的样子。

空气中掺的水,比起凤翔府老家边上小酒店里卖的酒还多。当年掺水的酒,应该说是掺酒的水,李信记得他爹喝起来时,都是一边喝一边骂,越喝也上火。那时候,自己老父应该从没想到还有作封翁的一天。李信咂咂嘴,现在倒是没人敢给他的酒里掺水了。

‘这鬼天。’

一年到头,水面上都看不见冰,可湿寒的空气依然能钻过皮袄、棉袄,透进骨头来,李信觉得南方比起陕西的冬天还要冷一点。他在荆南已经有三年了,却还没有习惯过来。已经到了更南方的广西,情况还是一样。站在船头上只是过了片刻功夫,就已经手脚冰冷。

李信活动活动手脚,不知什么时候能调回北方去,等打完交趾得问一问表弟。看了眼岸上,每一艘官船,都是在十几名纤夫的拉动下,才能溯流而上。论起吃苦,他可远远比不上拉纤的。

“都监。”雷简从舱里走了出来,脸色还是青白的显着病态,不过已经能在船上站稳脚了。

李信回头,上下打量了一下,“雷兄,今天好一点了没有?”

雷简挺直了腰,给出了一个有点勉强的笑容:“好得差不多了。”

治病救人的医官反而病了,像是笑话,李信却是笑不出来。看到雷简现在终于能起来走动,心中的一块石头终于放下了,“那就好。”

“让都监挂心。”知道李信是个锯嘴葫芦,不会奉承人。点点头就当作安慰,雷简也算是见怪不怪。要不是有韩冈、张守约和章惇一路扶持,这样的性格怎么在军中爬上去?这一次又怎么可能压得过刘仲武,被天子点上领着南下救援广西的荆南军?这就是朝中有人的好处。

不过李信在荆南军中的威望倒也是十足真金,当年出阵都是身先士卒,下面的士卒都是服他。雷简也看到了,一起南下的几个指挥使,在他面前都不敢有二话。

掠过水面的寒风吹得雷简抖了一下,抱着膀子搓了搓,“现在到哪里了?”他问道。

“前面就是兴安县。”江面上的船只多了起来,沟通荆、广的灵渠渠口,就是冬天也一样热闹。

“都快到兴安了?!已经进广西了?”雷简吓了一跳,进入兴安之后,灵渠在望,就算是入了桂州地界。他在船上到底躺了几天?!只觉得刚刚离开潭州不久,怎么一下就到了桂州境内。

李信瞅瞅雷简,看起来病得不轻,头脑都糊涂了,这样的医生谁敢相信他开的方子,“雷兄,到了桂州城中,还是先将养个几日为好。”

这怎么行,他的副手可是等着要抢他的位置。“经略和运使招在下随军,岂是为了来桂州养病的。”这一次随军机会也是难得,雷简哪里肯放过。在太医局中,他的医术排着倒数,远远比不上给天子、太后看病的几个御医,但他升官一样不慢。靠得是什么,雷简很清楚。

雷简不肯听劝,李信再瞥了一眼便不作理会了,这事让他的表弟拿注意好了。

“已经到兴安,纤夫终于可以歇着了。”李信要管着他的兵,在最后一条船上坐镇,章惇和韩冈则是在中间的主船上。码头上传来号子声传到了船上,章惇和韩冈掀帘走了出来,“没有光,夜中灵渠不好走,纤夫得让兴安县换上一批,也需要时间。今天歇上一夜,等明天过了灵渠,就能到桂州了。玉昆,你看如何?”

“灵渠的水流是湘水往漓水去,入灵渠后就可以顺流直下,倒也不需要纤夫。一夜走到南面出口的灵川,天亮了之后,正好可以顺水去桂林。”

章惇惊讶的看了韩冈一眼,他这个陕西人怎么知道灵渠的水流方向。但他再往水面上看了一看,变恍然大悟。江中筑了堤,冲着上游还有尖嘴分流,而他们上行过来的还是人工开凿的渠道,只要想一想,当然就知道灵渠中的水究竟是从哪里来的。“原来如此,玉昆果然是心细如发。”

“不敢当。”韩冈曾经饱览过漓湘之间的风土人情,灵渠可不是第一次来,只是没有坐船在灵渠上走过而已。知道章惇是误会了,但他也只是谦虚一下,没办法解释。

“既然过灵渠不需要纤夫,那就好办了。”韩冈要连夜行路,章惇也不会反对。他们沿着湘水上溯一样,都是靠纤夫一路拉上来,纤夫走多快,船就走得多快,心急如焚也没用,现在终于从逆流变成了顺流,章惇也想走快一点,“就让兴安县换上一批熟悉水情的船工,让他们指点着过灵渠。”

湘江越往上游去,就越要依靠纤夫的手段。灵渠也有纤夫,不过只负责北上的船只,南下就是下水船,顺流直下。韩冈虽说是要急着过灵渠往桂州去,不过他的心里已经不是很急躁了。已经过了一个半月还多,邕州还没有传来噩耗,苏缄想保着邕州,应该就不会有问题了。

章惇、韩冈从京城南下,一路都是兼程而行,可也是足足用了十五天方才抵达潭州。等待奉召出动的潭州军做好一切准备,又用了他们两天的时间。而后沿着湘水一路上行,到今天抵达兴安,进入桂州地界,则是正好是第十天。

只用二十七天便从京城至桂州,其中有三分之一的路程,还带着兵——尽管是乘船——这个速度已经是足够快了,也只比来往广西和京城的急脚递稍慢了那么一点。

一开始在唐州的驿站中,碰上的广西来的信使,听说了邕州已经守了二十多天之后,韩冈和章惇都放下了心。一般来说,攻城战如果不能在十天半个月内便攻下来,攻城一方的士气就难保住了,如果不肯撤围的话,就会转为围城。而围城之战拖到一年半载都不鲜见。

只是到了襄州,又听到了桂州援军全军覆没的消息。情况一下子又变了。外无必救之军,内无必守之城。如果城中守军人心动荡,很可能会有内奸开城。

邕州军情,广西经略司一日一上报。韩冈、章惇一天天南下,尽管不可能总能在驿馆中撞上信使,可总能知道邕州城到底破没破。直到昨天,从距离上看,至少到七八天前为止,邕州城还是安稳的。

南下的一路上,韩冈和章惇的心就一天天的放下来。

章惇是荆南军的老上司,李信在荆南军中威望又是极高,韩冈的大名也在军中流传,随军就能大涨士气。他们执掌荆南军,如臂使指一般。以这四个指挥来为核心,可以组建一支超过六千人、有着足够战力的大军。在邕州城附近狭窄的战场上,要打穿围城日久、师老兵疲的交趾军,不需要太多的气力。

四十艘官船组成的船队,抵达了兴安县外的港口,在码头上停下,章惇便派了人下船去通知兴安知县。派出去的人才走到城门口,兴安知县就已经带着县中的官吏迎了出来。

已经与韩冈定下了行程,章惇无意跟他的下属多说废话,先问了如今的邕州战况,听说了没有

将连夜往灵渠去的打算说了一下,兴安知县就犯起难来,“经略有所不知,冬天灵渠水枯,得用斗门来蓄水,可这水一蓄起来,就流得慢了。再想要向南,就只能靠纤夫来拉,看不清脚下,想拉纤也难。”

章惇和韩冈面面相觑,再一问才知道,灵渠冬日水枯,连南下都需要纤夫。每到冬天,兴安县的有许多百姓就会主动过来拉纤,都是吃这碗饭的,冬天时打点零工贴补点家用,就算拉的是官船,也会给付工钱。不比陕西旧时的衙前役和夫役,全都是白工。

“军情紧急,给付三倍工钱,今夜就要过灵渠去!”章惇说得决绝,威胁着面现难色的兴安县官吏们,“本经略领军南下救援邕州,只以军法行事,尔等想一试法度不成?!”

顶头上司以性命相胁,又知道是绝对不能耽搁的要事,一干官吏忙忙碌碌了一个多时辰,等道入夜之后,穿上的士兵都吃过了饭,兴安知县才过来禀报一切已经准备好了。不仅仅是纤夫和熟悉水情的船工,连同灵渠上的各处斗门,都派人通知到了。

堤岸上纤夫喊着号子,两边还有人打着用火把为他们照明。进入灵渠后,渠中流水的确只有浅浅一层。不过当前面放下斗门,这一段河道中的流水就立刻涨了起来,就跟船闸或水坝一样。经过一处斗门,就用灯光来通知更前面的斗门。

“多亏了李师中。”章惇和韩冈站在船头上,看着船队在狭窄的运河中缓缓的前进,“灵渠上的三十六座斗门,还是李师中在广西提刑任上所修。那时候渠中还有礁石当道,也是他遣人凿开。”

虽然在秦州与李师中相交很不愉快,但他的功劳,韩冈也很大方的承认,“的确是多亏了他。”

在纤夫们的号子声中,用了一夜的时间,载着一千五百余名官兵的船队穿过灵渠,抵达了灵川县。接下来的路就好走了。在漓江两岸的苍翠群峰中顺流直下,只用了两个时辰,在中午的时候,桂州城已经出现在他们的面前。

就在码头上,已经提前得到消息的知州兼经略使刘彝、转运使李平一出城来迎接章惇和韩冈一行。

军情紧急,一切迎来送往的俗礼已经没人再放在心上,互相之间见礼通名、验了告身之后,章惇劈头就问道:“邕州情势如何?”

刘彝和李平一两人对视一眼,带着些惶然,“交贼近日封锁了大小道路,斥候难近邕州一步,从昨日开始就没有得到邕州的消息了。”

