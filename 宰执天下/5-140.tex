\section{第15章 自是功成藏剑履(二)}

京中寺院道观的钟声一齐鸣响,向天下万民昭告太皇太后崩于庆寿宫中。

一记记钟声穿云裂石,东京城的百万军民纷纷出屋,侧耳数着钟声响起的数目。

之前半年多的时间,因太皇太后不豫,辅臣时常奉旨入祷天地、宗庙、社稷及都内神祠,宫观寺院亦是隔三差五的设道场,五岳四渎、乃至天下有仙踪灵迹处的军州,当地通判都奉旨去焚香祷告。京中的那一等无钱买度牒,以至于做了几十年行童、童子、沙弥而不得剃度的男女,也都被特旨赐了度牒。

但这一切,都没能挽回太皇太后的生命。纷纷扰扰两百天,边疆战事不断,京城内也一直都是在忙乱着,直到太皇太后今日上仙。

太皇太后上仙,依制辍朝禁乐。

天子和朝臣依例都要朝临庆寿宫,祭奠太皇太后。御史中丞李定有监察百官之职,就在殿中盯着,看有哪位官员违了礼制。

宰相王珪身着丧服,领着群臣祭拜,宗室、皇亲亦在班列中行礼如仪。虽云辍朝,但在庆寿宫中的朝临仪式,一如常朝时的仪制。

李定坐在殿门后,紧盯着殿中朝臣们的一举一动,而他下属们的一对眸子,同样一如鹰隼一般,从衣袍查看到装束,一点细节也不放过。当年英宗驾崩,欧阳修在丧服下误穿了一件紫袍,由此引起了御史们的弹章交相而上。服饰是礼制的一部分,一点差错都代表着对太皇太后的不敬。

而这时候,天子并不在正殿中,宰辅们除了王珪不得脱身,其他人也都不在。皆是与赵顼一起在偏殿里议事——说是辍朝,那也只是不上朝而已,该做的正事不可能耽搁。

刚刚收复的河西,朝廷已经确定要新设一路,名为甘凉路。而银夏一地、以及兰州直至青铜峡的那一片数百里的黄河谷地,究竟是分割给原来的缘边五路,继续分区防守;还是干脆就设立一个银夏路来统管对北防御,将驻守在缘边五路的兵马给解放出来,朝堂上争论却得很厉害,到现在还没有一个定论。

新复之地需要治理,移民、垦荒,安抚土著,剿灭流寇,亟需大量的财力物力和人力,这就需要朝廷为此去筹措钱物和人手。同时扩张而来的土地,也代表着更多的官职,更多的功劳,以及更多晋升的机会,让许许多多有心边事的官员趋之若鹜。直面辽国的青铜峡和盐州,虽然没几人愿意去冒风险,可甘凉诸州,却是十分安全,且并不缺乏功劳的好去处。千方百计赶着趟上来走门路的很多,就是李定这边,也有亲友找上门来,求他为此关说。

不过今天的议题,应当不会局限在这几桩事上。李定瞥眼看了看殿中眼神犀利如电的几名下属,今天在庆寿宫偏殿议论的焦点,少不了跟河东有关。

韩冈犯下的错太大了。一下子竟敢上报两万三千斩首的功劳,未免太贪功了一点。若是三五千,朝廷随手就将赏赐给发了,没人会议论一句;万儿八千,天子也能捏着鼻子认下;但眼下可是两万三千,朝廷无论如何都不可能忍下来,御史台对此更是不会善罢甘休。

乌台之中,有名的如舒亶、张商英,没什么名气的如丁执礼、范镗等,总共不过二十多名御史,竟有三分之一为此上了弹章。

丁执礼、范镗等人,说韩冈御下失当,为部将所胁。而一向与吕惠卿走得近的舒亶,章惇旧年所举荐的张商英,则是上本弹劾韩冈贪功好杀,妄杀数万新附之人。

多名御史联袂弹劾一人,数年也不见得有一次。每一次出现,都会引发一场剧烈的朝局震动。基本上每一次的目标全都是宰执一级的高官。在正常情况下,即便如韩冈已经做到了镇守边地要郡的一路经略使,依然不够资格。只能说他当今的风云人物,身处风尖浪口,惹得监察御史们人人侧目,故而提前享受到了宰执级的待遇。

进了御史台,是为天子监察百官,不能怕得罪人。虽说监察御史都是选用有声望但资历浅薄的年轻官员,以利用他们年轻气盛的冲劲,为天子打压权柄在握的宰辅。但再年轻也有个限度,基本上都是三四十岁,十几年官场生涯才有资格。

一任御史,是晋身宰执重臣的终南捷径,若能让一名宰辅黯然而退,当即便能名扬天下,有了名声,便是日后入两府的根基。故而得选入乌台,在官场中是人人称羡的际遇,亦是监察御史们傲视同侪,敢于直面宰辅重臣的底气所在。可是韩冈的存在,却让他们黯然失色,眼看着他二十多岁就要走到宰执之位上,哪一个不想绊他一个跟头。而韩冈偏偏行事不谨,将把柄亲手送人,哪个愿意放过。只要此案一定,日后他纵能卷土重来,想要报复,恐怕也奈何不了已经身处高位的一众御史。

不论是否是偏近新党,御史们皆是将韩冈视为眼中钉。一夜之间,韩冈成了众矢之的。到了明天,弹劾韩冈的将会更多,就是李定他自己,如果不能顺水推舟,很有可能就会被盯着自己位置的某人,以不言韩冈之罪的罪名给弹劾了。

御史中丞能在一定程度上影响监察御史们弹劾的方向,却无权干涉或是阻止他们的弹劾,否则,御史中丞也将成为被弹劾的对象。李定不想开罪韩冈,但他也无法阻止下面的御史将韩冈视为眼中钉,何况他因为在清议中名声不佳,对下面的御史,也管束不住。

李定满是感触的叹了一声。

稳定了河东局势,又夺取了葭芦川大捷,韩冈在河东路经略使的任上已经是功德圆满。之后收复胜州的举动,根本是画蛇添足,落到人人喊打,也是他咎由自取,怨不得他人。就不知道偏殿中,正在议论此事的天子,打算如何处置他了。

但李定想错了,此时的偏殿,还没有说到对韩冈的处置。对病逝的太皇太后,需要讨论敲定的事,一桩接着一桩,还没有空出来针对韩冈。

赵顼听着臣子们报告太皇太后的后事准备,却是神思不属。

在真正的祖父母甚至母亲那里,都没有得到的亲情,刚刚去世的太皇太后给了他。每逢他处置政事过晚,太皇太后必然会亲自来探问,若饮食为此耽搁,更会亲自遣人安排,如此十余年,都没有例外过。

登基后不久,他身穿金甲,跑去太皇太后面前炫耀的那一幕,在记忆中犹如昨日刚刚发生过的一般清晰。但委婉劝诫他天子身穿甲胄非是国家吉兆、社稷之福的太皇太后,如今已经不在人世。日后想再向长辈炫耀自己的成绩,难道还能去一向对自己冷淡的母亲那里?

“太皇太后令旨一向称为圣旨,这园陵亦当可称山陵。”

赵顼突然间开口,正在读着刚刚撰写好的哀册的蔡确一下都愣住了。

几名宰辅面面相觑,也不知该说什么好。太皇太后的陵寝仪制,应当名为园陵,其制度依照昭宪、明德两位皇太后的旧例。可赵顼却偏偏要改为天子才能用的山陵。

不过天子一贯最亲近太皇太后,要怎么做还不是他一句话?太皇太后素日礼仪,比之天子,也仅是不鸣鞭。又有据传身穿天子冕服下葬的章献明肃刘后在前,也便没人愿意出来触天子的霉头。

“诚如陛下之言。”蔡确当先说道,“既如此园陵诸使当易名为山陵。园陵使,可由参知政事任职。而山陵使,当改由宰臣担任。”

“一切皆可比照山陵仪制。”赵顼道。

“那当以宰相为大行太皇太后山陵使,判太常寺为礼仪使,御史中丞为仪仗使,知开封府为桥道顿递使,翰林学士一人为卤簿使,诸事各归有司。”

吕惠卿冷眼看了一下很会抢风头的新任参知政事。

因为伐夏之役并非惨败的结局,辽人的偷袭为一力主战的王珪解了围,可以坐看他吕惠卿被人围攻。半个月前,蔡确升任参知政事。这个偏向新党的任命,很可能就是天子放弃自家的征兆。只是太皇太后新近大行,使得朝廷政局暂时不便有所更替。

也许等朝中这一番事了,就该轮到自己离开京城了。

“曹评还没有回来?”赵顼突然又问道。

这一次是元绛抢前一步:“已经遣河北沿边安抚副使刘琯去替换他,不日便可返京。”

太皇太后曹氏上仙,曹家的子弟都要入宫奉礼。其余子侄皆在京中,唯有侄儿曹评一人担任国信副使,随队前往辽国。他是宋夏开战后的第二批使辽使节,当第一批使节因辽人出兵吞并兴灵而奉旨回返后,他们是赵顼认命之后,派去与辽人商议西北国界的使节。

只不过说是商议,可谁也不指望能从契丹人那里占到什么便宜。曹评这个宗亲趁机出去占个光,混个资历,也没人在乎。

当年念兹在兹的观兵兴灵,到了今天,西夏终于是灭亡了。只是观兵兴灵的初衷却没有达到。长久的和平让人忘记了契丹依然是吃人的狼,这一回的教训刻骨铭心。

赵顼点了点头,国信使、国信副使是谁都无所谓,别丢朝廷脸就行了。过了一阵,他突然又问道:“今天御史台八御史共上本,弹劾河东安抚使韩冈贪功好杀,御下无方。不知诸卿如何看此事?”

