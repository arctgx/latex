\section{第36章 沧浪歌罢濯尘缨(25)}

“而且他们还缺乏自信。”章敦又说道,“赌赛终究是不登大雅之堂。别看如今的联赛声势浩大,但也只是赚钱的营生,以及打发时间的游戏。报社根植于此,当然也不敢有狂妄的念头。”

“……这也是好事。”吕嘉问笑道,“两家报社的心虚气短,正是背后无人指使的明证。之前可是有人猜韩玉昆也在其中呢。”

“我倒是没这么想过。齐云总社也罢,赛马总社也罢,都已经变得太大了,谁也控制不住的。”

小孩子耍不动大锤。各自控制蹴鞠、赛马两大联赛的两家总社,早已规模庞大。每年选举会首,都会有数百人来参加投票。两大联赛都是从关西发轫,跟韩冈有脱不清的干系。可有韩冈做靠山的棉行行会,如今在两大总社中也只是一个普通成员罢了。

“两家报社背后是两家总社,而不是哪一家宗亲或是勋贵。如此庞大的团体,的确没有哪家能控制得了。”吕嘉问点点头,又道:“可事前谁能猜得到会变成如今的局面?也许韩玉昆不是没打过背后操控的主意。只是他并不是当真能掐会算,没想到两大联赛传到了京城之后再也无法控制了。”

‘没那么简单!’

章敦家里的商号在交州跟韩家的顺丰行一起做着生意,顺丰行的底蕴他远比其他人看得要清楚得多。韩冈的性格,他了解得也更为清楚。

吕嘉问说韩冈事先没有想到蹴鞠和赛马会发展到现在的规模,可章敦却觉得韩冈肯定是想到了。只是韩冈从没想过去控制,他的目的似乎只是结善缘。留份人情。顺丰行和棉行行会如此简单的就在京城站稳了脚跟,便是韩冈做事方法的功劳。

韩冈这个人性格极为现实,做事也极有分寸,而且他最擅长的便是让利于人,然后团结起一批人来。在陕西,他把棉纺技术公诸于众,在交州,他将白糖技术与人分享。没有他的慷慨,河湟、交州都不可能有现在的繁荣。

但韩冈放弃了一家独享的利益,得到的回报却更多。若是他敝帚自珍,顺丰行永远也不会有现在的规模。论起心胸和远见,能让章敦佩服的人,在这世上其实是凤毛麟角,而韩冈倒是其中之一。

话在章敦的脑中一闪而过,却没说出来,顺着吕嘉问的口气:“说的也是,又不当真是药王弟子,掐指一算的本事韩玉昆是没有的。谁也控制不了的两家会社,不需要担心太多。有朝廷在,肯定能控制……只是反过来说,为何御史台都只敢在小事上做文章?不能断了报社的根?”

原本报社中负责撰稿和整理文稿的人,被编修、编校、修撰什么的一通乱叫。然后给愤怒的御史们参了一本,说是编修、修撰乃是官名,朝廷名爵岂容白身玷污,而且还是数得着的清贵之职。最后两家报社不得以,也不知是谁想出了个编辑的名号来,全都改了。

在民间,很多人都不直御史台的做法。去外面店里吃饭,哪家店里不是道一声官人、员外的。有本事把七十二家正店都参上一本。

“自然动不了。皇后都爱看。除非报社犯了大错,否则皇后不理会御史台的话,台谏官怎么跳都没用。”吕嘉问忽然眯起了眼睛,想到了什么:“子厚今日邀我过府,可是想说报社已难禁。现在只是畏于御史台积威,所以还不敢乱说乱动。等再过段时日,胆子大起来,可就是敢胡说八道了。”

以吕嘉问所知。皇后现如今天天都要看报,已是信之不疑了。就是换了皇帝上来,也不会拒绝另一个了解下情的渠道。相信民间流言的人们永远都比相信朝廷辟谣的要多。朝廷出面办报,绝不可能像两份快报一样将声音传给千万人。这的确是个大威胁。

章敦摇摇头:“不是。朝廷有刀,光有嘴皮子是没用的。何况报社后面都是富贵人家,又令出多头,就是有人起了异心,内部就会压下去。朝廷只要注意监察就可以了。”

“……那是要我提醒平章要小心西京。现在他们还没学,等学会了,立刻就能派上用场了。”

“也不是。文、吕、马之辈,只在洛阳办报,话只对洛阳城说,那是一点用都没有。东京城才是天下至中。”章敦冷笑着:“但他们的报纸却卖不到东京城来。来多少,就会被烧多少!没看《每日新闻》最后是什么结果,还有过去的那些小报,如今都没了踪影。”

断人财路如杀人父母。不论过去有多少联系,当洛阳的人想要把手伸入开封,在京城贵胄的眼中,就是来抢钱的。

吕嘉问越来越搞不懂了:“那子厚你请我来究竟是为了何事?”

“‘说的是辽国的风土风物,又不是大宋的内情,何须担忧。’‘吾未见好德如好色者也。可好色者又不如好利者。越多人对有意于辽国,日后征北,就有越多的豪杰谋士纷纷来投。’”章敦问吕嘉问,“望之可知这两段话是谁说的?”

“不是前些日子蔡确和薛向说的嘛。”吕嘉问当然记得,“记得子厚你也帮了腔。还有张缲也是。现如今可是人人皆谈北事。就跟当年河湟开边时,人人皆谈西事一样。”

刚刚结束的战争使京城中最为火热的话题,介绍辽国内情的文集、笔记也是印书坊中最受欢迎的书籍。

依照朝廷律令。任何臣子在接待外国使臣,或是出使他国的时候,一言一行都要原原本本的记录下来,作为奏章呈交朝廷。

比如现如今流传很广的《使辽语录》,就是两年前病逝、谥号忠文的陈襄将他担任国信使出使辽国时的记录结集,然后出版流传。苏颂最近也出版了一部使辽的记录,讲述了他出使辽国时的经历和见闻,这同样是他从自己旧日的记录中编纂出来的。

吕嘉问还听说最近有书商,向曾经出使过辽国乃至高丽的大臣们约稿,给出来的价格,甚至让吕嘉问都为之咋舌。可见如今讲述辽国的风土人情和山川地理的书籍有多么受欢迎。

这里面,有很大一部分因素是这一战的结果让士林一改过去对辽国的畏惧,开始对恢复失地有了信心。更有许多书生想一策成名,或是拿着对辽国的了解作为敲门砖,敲开一干重臣家的大门——秀才不出门,能知天下事,靠的就是书。

但民间对辽国的认识,却不是依靠这些书和读书人。能买书读书的终究是少数,绝大多数百姓还是依靠报纸来了解辽国。

“河东、河北两路鏖兵,没有两家报社开始邀请名家议论战局,京城的人心不会那么稳定。古北口的杨无敌庙,没有齐云快报刊载,没几个东京百姓会知道。辽国的国主年年巡游四方,春夏秋冬四捺钵之名还是靠了逐日快报的宣传,才在京城内普及开。两家快报的作用可不小。”吕嘉问说道。

“没错。”章敦点头,“这就是报纸的引导和教化之功……市井传言往往失真,道听途说而来的消息,并不可信。而朝廷的言论,还不如市井传闻让人信服。但报纸不同,从小处着手,几年下来,信用已经建立起来了。”

吕嘉问心中这时候已经有了点眉目,却还差一点没能捅破,紧锁着眉:“子厚的意思是?”

“借鸡生蛋!”

“借鸡生蛋?”

“没错,借鸡生蛋。有些话不方便在朝堂说,可以拿出去在报上说。虽然不是自家的,借来用用也无妨。”

有些东西就是不能抓在自己手里,也不能留在他人手中,就算不能控制,也得保持足够的影响力。

“气学讲究以实为证。列出户口人丁的数字,其实也是以实为证。我大宋国力远胜北虏。今日胜之,乃是必然。日后随着户口增长,还会越来越强。要说其中没有韩玉昆在后指使,望之,你信不信?”

吕嘉问皱起眉,摇了摇头,“但他控制不了。”

“也不需要控制啊。如臂使指难为,顺水推舟、借力打力、又有何难?借鸡生蛋难道那两家还敢拒绝不成?”章敦咧嘴冷笑了起来:“不,不可能拒绝的。平章若是借重他们来说话,他们可是会乐得不知自己姓谁!”

想一想,要是宰辅重臣都在报上写文章,那等于就是承认了两家报社的实力,更增添了快报的权威性。卖得肯定会更好,赚得也会更多。也不用再担心朝廷再跟他们过不去。哪能不巴上来奉承?

而且这两家快报既然新党用了,旧党就不会再用,甚至还会出言攻击。到时候,就算两大总社不愿意,也必须新党站在同一个阵营中了。

这件事还得尽快!

等洛阳那边有人先想到借用快报来说话那可就麻烦了……或许已经想到了,只是他们没能说服报社后面的贵胄富豪。毕竟现在谁当权谁得势还是很明显的。

对章敦的话,吕嘉问连连点头。

借重民间议论很多人都做过,可借重报纸来说话,却要转过好几道弯。毕竟快报的形式在过去就是传递流言蜚语的小报和揭帖,都是很多官员避之唯恐不及的。只是现在章敦一点破,就仿佛打开了一扇窗户。

“那我明日就去拜见平章,将这事与平章说了。”

“那就拜托了。”

章敦微微笑着,心道合则两利,也不知韩冈会不会承这个人情。
