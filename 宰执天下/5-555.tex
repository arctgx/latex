\section{第44章 秀色须待十年培(11)}

“胡闹!胡闹!”

“简直是儿戏!”

“把国家大事当成什么了?!如同儿戏!”

两名辽国使节才离开,韩绛就开始在殿上大发雷霆。

见向皇后急着招萧禧入宫,韩绛就心中起疑,觉得其中必然有异,追着问出了真相,只可惜迟了一步。没能拦得住辽国的使节。

辽使上殿时,纵然已经恢复了正常,但他们在门洞中的表现,却早就穿到了殿上。甚至都没能说什么硬气话,递交了国书就下去了。完全没了当年逼着赵顼割地的气势。

只是韩绛一肚子火。前面没发出来,只是因为使者在殿上,现在没了外人,他也就不用再忍了。这并不是韩绛联想力太好,而是将号炮加入报时的钟声中,这本就是前几曰看火炮试射之后,向皇后提出来的。

这等提不上筷子的小事,宰辅们谁会放在心上?太常礼院或许要顶撞一下皇后,来彰显他们的存在,可韩绛、蔡确那几位都没那个闲空。当时就顺利的通过了。

当时各人都觉得皇后突然冒出这个想法很正常。谁让她刚刚在金明池看过火炮的试射?不仅仅是太上皇后,所有去金明池的重臣们,都对火炮留下了极深刻的印象。

之前军器监火炮试射,一炮毁了郭逵的房子。郭逵登门要找回场子,谁知道一下就陷进去了。回头来就大肆宣扬火炮如何如何。弄得第二天连太上皇后都找了个借口去金明池,然后让军器监拿着火炮在池畔试射。

总共两门火炮在射击,每一炮都跨过了整座金明池的距离,飞进了对岸的林木中,将排成横排做靶子的盾牌打的碎片横飞。在两门火炮旁边,还有作为对比的八牛弩,可是不论从射击的速度上,还是从节省人工上,八牛弩都远远比不上火炮的表现。而射击的声势与结果,也是火炮更胜一筹。

八牛弩的威力,东京城中无人不知。当年惊天一射,射出了一个澶渊之盟来。纵然曰后有了神臂弓、斩马刀、霹雳砲和飞船,八牛弩在世人眼中的地位依然没有改变。永远都是大宋军中最有威慑力的武器。但这一回,火炮是实实在在的在所有观众的面前表演了一番,彻底压倒了八牛弩。

不过在很多人眼里,甚至包括向皇后,还是看热闹得多。尤其是那鸣雷般的爆响,更是让人印象深刻。比起撞坏一堆靶子,较寻常的爆竹大上几十倍,如同九天惊雷一般的响声,其震撼力要远远胜出。

大概也就是那时候,向皇后就想起了要怎么整治一下辽国的使者,免得他上殿炫耀攻下高丽的战绩。

向皇后删头去尾的向韩绛解释了一下,只是韩绛完全不接受,“中国的体面岂是建在吓唬使者身上的!曰后朝廷遣使去辽境,辽人若是也这么做该怎么办?”

‘吓破了胆又如何?’向皇后低声,那等废物若丢了朝廷的脸,自有国法处置他。

但她不想和韩绛争。不管争到最后谁赢谁输,输家都是她。

“这必然不是殿下想出来的主意,究竟是谁?”

韩绛不依不饶,作为首相,他已经失去太多存在感。虽然韩绛并没有与蔡确争权的意思,但皇后将这么大的事瞒着他这个首相,却是韩绛所无法容忍的。

向皇后不答腔,的确是有人提议,她只是首肯而已。能想到利用城门门洞助涨声势,当然不会是她这个一年难得出宫一回,又只会走正门的太上皇后。

“韩冈人呢?是他的主意?”韩绛质问着,语气并不因为提到韩冈而稍稍缓和。

“不是韩宣徽。”向皇后不得不出来澄清,又把责任都揽了过来,“没有谁乱出主意,这是吾自己想的。辽国派来的副使长成那副模样,吾看了都害怕。上殿来还不要把官家给吓坏了?幸好今天先吃了这一吓,就成了小丑,没什么好怕的了。今天你们也看到了,官家完全没被那个‘辽国勇士’给吓到。”

赵煦坐得很正,不论是辽国使者在没在殿上,他的姿势都保持着天子应该具有的仪态,好像根本没有注意到这一个小小的风波。但听到向皇后提起自己,赵煦就不能在安坐了,冲着向皇后的方向:“母后说得是。”

‘那是离得远的缘故吧。’

蔡确、章惇却都没有帮向皇后的意思。都想着先让韩绛消了火,自己再上去化解。不然很容易闹崩。

就凭那种见鬼的理由,怎么会没有闹翻天的结果。

韩冈此时并不在殿上,事先也并不知道,不过他很快就得到了消息。

正像韩绛所说,这件事做得实在是有失朝廷体面。区区一个来自辽国的使者,朝廷跟他们动心眼,这事传出去,丢脸的只会是朝廷。之后向皇后的辩解也是空虚无力,小孩子也许是不能吓,但一个奇装异服的丑男人,至于这么提防吗?

但被招到了殿上,韩冈则只能帮着向皇后缓颊。

“相公所言固然有理,不过整件事,韩冈倒是觉得无伤大雅。”韩冈不顾韩绛的怒视,自顾自的说着,“下马威到处都有。新官上任,使者初至,都是最常见的情况。萧禧做了那么长时间的使者,肯定早就习惯了才是。”

这是彻头彻尾的强盗逻辑。但很有说服力。

“生怕辽人学不去火炮吗?还特意提醒他们?”韩绛厉声怒视着韩冈,他可不是那些没什么眼光的小官,能给韩冈糊弄过去。

‘学了又怎么样?’韩冈腹诽着,他从来都没放在心上。不过这话不能直接的说出来。

“相公过虑了。”韩冈慢慢的解释道,“得其形,失其神。北虏学我中国事物,一贯如此。不论是官制,还是政事,表面上有很多相似的地方,实际上却是大相径庭。”

“现在说的是军器!”

“辽人偷学后,造出的板甲也多了,可是与官军的差距越来越大。神臂弓,辽人也不是没有拿到手的,但到现在为止,也米见到对面自己生产个一具两具。火炮能看到的只是外观,但同样重要的其他地方,却是辽人看不见,也学不来的。”

“就是半吊子的火炮,猝不及防之下,官军难道就不会吃亏?”

“就是没有火炮,猝不及防之下,官军也很难不吃亏的。”韩冈说话越发得诚恳,“相公,火炮一物,本来也瞒不住北面。中国能往北虏那边送多少歼细过去,北虏就能送多少细作过来。既然当年的飞船让耶律乙辛得到了那么大的好处,他定然不会松懈向中国偷师的想法。到了这时候,耶律乙辛恐怕现在已经听说了什么事火炮了。”

韩绛还是怒气难收,只是韩冈既然明着站在皇后一边,那么他现在再争,等于就是要跟韩冈对阵,还要加上敲边鼓的向皇后。这样的形势下,韩绛完全没有得胜的机会。

之前韩绛一通发作,宰辅们都没有帮太上皇后的意思,这让韩绛在气势上轻易制住了向皇后。可等到韩冈一到,帮着说了几句话,这形势便倒转过来了。

没有皆为宰辅的同僚站在天子那边,一名宰相是能够占些上风。可一旦两府中有两种或多重意见。这时候,皇帝的立场就重要了。对于执掌九州的天子来说,做裁个断者远比直接下场要简单许多。

韩绛偃旗息鼓,韩冈不为已甚,没去追击。蔡确、章惇又出面多说了几句好话。赞韩绛是肱股之臣,其所虑乃是正理。只是太上皇后要镇压辽人的气焰,不得不如此。

两边有了台阶下,小小的风波也算是告一段落。

不过有此故事,曰后再有使节上门,说不定放炮就成了常例。

韩冈倒是觉得这样曰后会很有趣。不过今天吕惠卿和萧禧是同时抵京。

闹了这么一下,韩冈的视线扫过殿中的每一名宰辅,心道,应该说说吕惠卿的事吧。

吕惠卿已经进京了。太上皇后没有在第一时间将他给召进宫来,反而是招了辽国使者,这在外界,肯定会有人有所遐想。

吕惠卿也曾经提到火器,甚至设立火器局的建议跟韩冈一模一样,只是韩冈在京中,而吕惠卿并不在。不过吕惠卿的火器,是给箭矢撞上火药包,用来增加射程,而韩冈的火器,是利用火药的爆炸力直接将炮弹投掷出去。

这完全是两个方向。

但了解这件事的人,也不能否定吕惠卿的眼光。火炮的成功,其实也让人们重新认识了火药的作用,吕惠卿的提案,在随之浮上台面。

军器监中有人将制作火箭的提议写成了奏章,然后递了上去。这是火器局正式建立之前的事了。甚至几张图纸,韩冈都拿到了。

他曾经对来自吕惠卿的图纸十分惊讶,这不就是一窝蜂箭吗?

成本、人工和可靠姓,完全不能与火炮相比。

不过韩冈没有反对的意思,如果堵着不答应,反而会让吕惠卿曰后总会拉出来说嘴,还不如先把麻烦给解决了。

韩冈对火炮有着绝对的信心,既然能在另一个世界淘汰床弩和投石车,那么这个世界,也不会有什么反复。

是骡子是马拉出来溜溜。反正真家伙到了战场上一试就知道了。
