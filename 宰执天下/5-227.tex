\section{第25章 晚来萧萧雨兼风(下)}

张璪进来了。

作为翰林学士在进殿前多半已经做好了一定的心理准备,但当他进殿时,看见太后、皇后、宰相、执政全都在列,身子还是猛地抖了一下。

尽管韩冈相信陈衍肯定已经对张璪解释了许多,但太后身边的内侍来传话,而不是天子身边的宋用臣、蓝元震等人,想必这位翰林学士肯定会有许多联想。

不过张璪毕竟还是为官多年的重臣,很快就恢复了平静。先通过韵书亲眼验证过赵顼的神智,然后便在准备好的小桌案上开始起草诏文。

招司马光入京。

七步成诗的能力对翰林学士来说是必备的技能。第一份诏书很快就打好了草稿,张璪提笔修改了几句之后,誊抄了一遍交了上来——看看时间,最多也不过两刻钟。

王珪草草看了一遍草稿,又给赵顼念了一通。

通过眼皮的交流,韵书翻到了上声二十哿,诏书的草稿便发还给了张璪,让他在正式的隐纹花绫纸诏书上誊抄——天子说了‘可’。

誊抄的时候,天子的印玺也已被找出来了。

当诏书写好,王珪又亲自检查过,向皇后便把着赵顼的手,攥着天子印玺在诏书上盖上了鲜红的大印。盖好印,宰相王珪落笔签押。

一封召还司马光的诏书便就此出台。

看着宋用臣接过诏书,用黄绫紧紧包扎好,韩冈咬紧了牙。这一封诏书,可就意味着旧党在沉寂了十数年后,再一次回到了执掌朝政的舞台上。

政局犹如跷跷板,一头翘起,一头便会落下。

韩冈并不觉得落下的仅仅是新党和新法。他的学派与新法勾连得太紧了。如今的成就,有多少是出自韩冈主导的气学?拓边河湟是王安石一力支持的,南征交趾领军的是新党中坚章敦,最后平灭西夏也是从一开始就在王安石和赵顼议定的变法方略中。当旧党重新登上舞台,曾经是新党拿来炫耀的这几件事,又怎么可能不被旧党当成靶子来攻击?韩冈和他手下的人何能置身于外?

难道要将希望放在旧党的宽宏大量上?!

就像赵顼不愿拿儿子的性命冒险一般,韩冈也不愿意去赌赵顼的算计能百分百的实现,更不会去赌旧党的人品。不要脸的士大夫,永远都会比要脸的多。欲加之罪,何患无辞?借口总是能找到的。

韩冈不喜欢陷入被动,也不可能眼睁睁的看着自己事业的命运落在敌人手中后,还能安心下来。

只是赵顼依然有条不紊的让张璪继续起草诏书。

司马光、吕公著,分别为太子太师和太子太保。而王安石……什么都没有。尽管只是虚名,但份量已经不下于宰执之位了。

尽管诏书没有参知政事们的签押,但并不是任免官员的诏令,仅仅是召臣子入京和两个虚职,在天子的印玺和宰相的签押后,就已经有了足够的法律效力,不愁无法通过。

通过三份诏书,赵顼十分直白的表明了他现在所作的一切,就是为了保住儿子能顺利登基。

三份诏书已经全部被黄绫包好,等天明之后,皇城、内城、外城开门,便会遣使出发。

看起来已经没有事了,赵顼也闭上了眼睛,但所有人还是在等着。

今夜还没有结束,应该还有一件最为重要,也是关键性的压轴要事需要解决。

韩冈在看王珪,不止一人将视线投向当朝宰相身上。额头和颈项上汗水涔涔的王禹玉王相公,一时间成了关注的焦点。

天子的态度都这么明白了,请立皇太子的动议,也该起头了吧?

前面赵顼说以司马光、吕公著为师保,那时候以王珪的聪明识趣,就该抢先一步请立延安郡王为太子——宰相在场的时候,副枢密使的薛向不好先开口。而端明殿学士的韩冈,则是不能开口提议。

但王珪没有任何动静,除了当着天子、太后的面,在三份诏书后签押副署之外,提也不提册立太子之事。

即便是诏书全都写好之后,他依然保持着沉默,只是在流汗。

战战惶惶,汗出如浆。

赵颢的神色一直很平静,但他现在想笑。对王珪的退缩看在眼里,冷笑在心头。

为了不受掣肘而用了这等没用的宰相。平日里是痛快了,但到了关键的时候,就是咬牙切齿也无法让一个废物变成谋国贤臣。

如今最重要的便是内禅,在赵顼还活着的时候,将皇位传给六皇子赵佣。

但内禅的事没人会催促赵顼,也没人敢催促赵顼,这需要赵顼自己提出来。臣子们只可能做好准备,亲如母子、夫妻,也不能径自开口让赵顼让出皇位。

可是连内禅的先决条件都达不成,那就是笑话了。赵颢当然更不会帮他的兄长。没有臣子开口,而由皇帝或是皇后主动提起,那么其中就有得空子可以钻了。

赵颢不屑的瞥了王珪一眼后,又将视线挪到了薛向身上。幸好不是章敦和蔡确——赵颢对他兄长的宰辅们下了大力气去了解——一个有名的胆大,另一个则最擅投机,没什么使他们不敢做的。至于薛向,胆子虽大,可惜已经老了。

视线最后落到了韩冈的身上。

赵颢很想笑出来,这样的窘境,不知道端明殿的韩学士是不是已经忍无可忍了?可惜他是最不可能开口请立太子的!纵然他是这座寝殿中最为期盼佣哥儿成为皇太子的几人之一,可他的身份让他不能开口。

看看皇兄怎么办吧。赵颢期待着。就算侄儿继承了大统,赵颢也不心急。时间有的是,身在深宫,区区一小儿,又能靠谁?

不需要太后狠下心对孙子如何,到时候,有的是想做王继恩的内侍。片刻风寒,一次惊吓,或是一点查验不出来的秘药,就能轻而易举的达到目的。就算太后知道真相又能如何,还能将他这个亲生儿子法办不成?

赵颢有足够的耐心。当他的皇兄真的像他日夜梦想的那般倒下,赵颢相信天命已经眷顾在自己的身上。不论怎么瘫在床榻上的皇兄怎么挣扎,命数就是命数,既然注定便不会再改变。

眼前的寂静,不就是最有力的证据吗?

第一次,赵颢觉得大庆殿中的那张御榻,已是触手可及。

凝重的空气压在寝殿间不知过了多久,仿佛要拖到天荒地老一般,赵顼终于还是再一次睁开了眼睛,眨起眼。

王珪一时间如释重负,连忙拿起韵书,继续做起了皇帝的通译。

上平十四寒——韩。

下平七阳——冈。

韩冈在众人的视线中上前半步,躬身道:“臣在。”

侍——讲——资——

没等赵顼将整句话用眼睛眨完,向皇后已经急着开口:“可是着韩冈侍讲资善堂?”

赵顼眨了两下眼,做了确认。

张璪提起笔,开始起草第四份诏令。翰林学士笔下的字如流水,一行行的流淌到稿纸上。这是早就确定了的任命,只要稍稍聪明一点的玉堂内翰,都知道该早一点打好腹稿。而张璪,甚至准备了两篇。

但赵顼的圣谕并没有结束。

上平一东——同。

下平十三覃——参。

赵颢不安的扭动了一下身子;张璪的笔也顿了一下,墨字的流水遇上了大坝,无法再轻快的流淌;王珪、薛向,乃至所有人的双眼也一下投向低眉垂眼的韩冈,眼神中只有震惊。

去声九泰——大。

‘想不到还真敢做。’赵颢心底里冷笑一声,又恢复了平静。因为他清楚的看见了他的母亲的双眉,向中间靠紧了一点。

想依靠韩冈?也得看看娘娘高不高兴。

可惜韩冈并不是那么讨他母亲的喜欢。或者说,只要跟王安石有瓜葛的,太后都不喜欢,包括从来跟王安石合不来的亲家吴充——或许其中有一部分是因为吴充脖子下的那个赘瘤。

当然,赵颢知道,更多的应是有他这个二大王的因素在。市井的瓦子中编排了那么多唐朝奸王夺女不遂,贫寒书生双喜临门的杂剧,太后若是能喜欢起韩冈,岂不是笑话?好歹也是最疼爱的儿子,而韩冈,不过是个灌园子。

但王珪的声音重又变得干哑起来,去声的诸韵部中一个个向下移过去。

最终,停在了第二十四韵部。

去声二十四敬——政。

同参大政。

也即是参知政事。

入居东府,副署诏令,为宰相之亚的参知政事。

张璪的喉咙也变得发干,正拿着笔打着草稿的右手仿佛重有千钧,甚至抖了起来,在雪白的宣纸上留下了一串墨团。嫉妒、愤恨、无奈、自怜,诸般心思涌上心头,啃咬着心口,一时间五味杂陈。

因为就在半年前,韩冈生日时,朝廷赐物的诏书正是由张璪所草拟。

学士以上的重臣都能在生日的时候收到朝廷的赏赐,宰辅们尤其多,这是朝廷给重臣们的体面。当时已经是龙图阁学士的韩冈也不例外。

但张璪也从那份诏书中了解到了,今天,离韩冈三十岁,还有半年!

一个尚不及而立的参知政事!
