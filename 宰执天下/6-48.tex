\section{第六章 见说崇山放四凶(15)}

韩冈的确是胆魄过人。

自河湟十年之后,都让人忘了他最早是怎么得到王韶的赏识。

不过,还是蔡确的失败最让文彦博扼腕叹息。

蔡确、曾布、薛向联手,推倒了一心延续先帝治国方略,换成了性格刚硬的太皇太后垂帘。

若他们成功,之后在朝堂上为了与王安石、章敦等人争斗,必然要援引外力相助。在眼下正邪截然两分的时候,蔡确能够请来的助力自然不会是他家。

而且太皇太后一向敌视新党,由其秉政,国政必然要恢复到祖宗之时。就算是蔡确不想拨乱反正,最后也是由不得他。过世的慈圣光献曹后,身为姨母、姑姑,还不是拧不过做侄女和新妇的太皇太后?

两三年后,重回朝堂的元老们,联合太皇太后之力,能将蔡确、曾布也一并给掀下来。彻底清除十五年来的重重乱政。

可惜韩冈这一骨朵之后,最后的机会都不复存在了。

蔡确从此成了叛逆,有宋一代都不可能再翻身。与其关系紧密的一干人等,这一回,日子也难过了。

蔡确的党羽就不提了。他的亲戚都一样要被这一桩的案子牵连进来。

据说韩琦家已经跟蔡确定下亲事。在婚事上,死掉的韩稚圭,他的儿子们倒是没有半点党同伐异的想法。一切都以维系韩家家门不堕为目标。可现在的情况,他们当初的目的已经完完全全的成为了水中月,镜中花。

还有冯京那位与蔡确联姻的前任宰相,也同样逃不过为人群起而攻的结果。

文彦博与蔡确没什么瓜葛,曾布、薛向就更不必提了。但文彦博现在并没有幸灾乐祸的想法。

刑恕竟然成了参与蔡确密谋叛乱的同谋之一,这一件事,让文彦博哑然失声。

刑恕的身冇份太尴尬了。他在洛阳城中,是很多人都看好的年轻一辈,也是西京元老们在京师的耳目之一。其交游广阔,常年在司马光、吕公著门下行走,又是二程的弟子,到处是朋友,出入元老之门,与其结交往来的衙内、士人多如牛毛。

比起吕公著、司马光来说,文彦博与刑恕算不得有来往。可他也不能置身事外。刑恕被牵连进谋反大案中,这是比司马光、吕公著败退回京,对旧党更大的打击。

在刑恕的家中,不知有多少与洛阳城中官宦人家的子弟往来的凭据,一旦给搜检出来,整个洛阳城都要鸡犬不宁。

纵然可以自辩清白,说自己与刑恕参与到叛乱没有任何关系,可这年头,谁没有点小尾巴?万一有人想来一个一劳永逸,文彦博本人都逃不过去。

文彦博白透了的双眉紧紧皱起,就连他也觉得这件事棘手了。对元老重臣的尊重,并不包括在叛逆之事上。尤其是新党诸贼等了这么多年冇,这么好的机会,就是文彦博也不觉得他们有任何轻轻放过的理由。

这样的情况下,至少得先做好准备。当事情真的来了,才能有所应对,不至于乱了阵脚。

“你有没有跟那刑恕私下里有什么勾当?!”

文彦博猝然问道,双眼紧紧盯着身前数步的文及甫。即使他一贯的对儿子不假颜色,也从来没有如此严肃的表情。

文及甫早就面无人色,惨白着一张脸。就算是文及甫也明白,朝廷对叛逆的态度,从来都是宁枉毋纵,何况文家眼下在朝堂上,举目皆敌,有所关联的朝臣,能挤进侍制班已经是难能可贵了。当真要面临朝廷天威,连个能帮着说话的人都没有。

与他常来常往的刑恕成了叛贼,作为与其关系亲近的自己又如何能轻易脱身。

但父亲的质问,他却不敢不答。若当真被认定与叛逆有所牵连,自己说不得就要自尽,以免为家族带来祸端。在这件事上,父子至亲也没有人情可说,总不能为了一个儿子,将其他子孙乃至整个家族都牵连进去。

在文及甫自己察觉之前,他就已经跪了下去,“儿子不敢欺瞒大人,刑恕过往一向常来奉承儿子。儿子却不过情面,也多与其敷衍。但决没有参与什么叛逆的勾当。”

见文彦博默然不语,他心中更是慌张,头脑急速转动,慌忙为自己辩解,“大人,想那蔡确和薛向都有拥立之功,寻常如何会谋叛?只是因为天子失德,方才起了异心。可太上皇才驾崩几日?儿子纵使有心为逆,也来不及与其共谋!”

文彦博沉默良久,最后叹了一口气,“……你去将你书房冇中的信和草稿都拿来。”

文及甫如蒙大赦,扶着膝盖挣扎起来,这才发现自己已经是浑身冷汗,浸透了内里的小衣。不过他也不敢抱怨什么,转身就脚步蹒跚的出了温房,往自己的小院去了。

一般来说,士人写信都会留草稿。就是才高八斗的大家,也会在写信给亲朋好友之后,留一份草稿在手中。那些私人文集中书信部分的底稿来源,都是留在家中的草稿。

文及甫过去可是有过写信为人关说,最后被牵扯进一桩大案中的前科。所以更是被文彦博严令任何信件都要留下草稿,以供日后查验和自辩。

文彦博不是不相信儿子的底限,而是不相信他的头脑。为人关说疏通是官场上的常事,但不懂怎么在文字上给自己留下余地,那就是少见的愚蠢了。而写给叛逆的信中,只有有一点含糊的地方,就能给人阐发出来,变成泼天的大罪。不亲眼看一看,文彦博是无论如何也不安心的。

文及甫很快就回来了,两名仆佣各抱着个箱子,里面全都是文及甫历年来收寄的信件。

一封封草稿被文彦博亲自翻阅过,不仅仅是写给刑恕的信件,还有写给吕公著、司马光以及其他一些与刑恕关系亲近之人的信件。

只是越看,文彦博的脸色越是难看。

虽然文及甫已经很小心了,但他的信件中很多都有言辞不谨的地方,如果真想要以文字入罪,那真的一点不难。

幸好与刑恕往还的信件中,没多少有问题的地方,不过与叛逆相往来就已经是罪名了。想要脱身,少不得要脱一层皮。

除非在朝中有人能帮着缓颊,否则朝廷就是顾念老臣的体面,文家的子弟也不会有什么前途了。

丢下了手中的信,文彦博长长的叹了一口气。温室中不通气,信上又满是灰尘,文彦博手中一个动作,透射下来的光柱中,就能看见无数灰尘虚影在晃动。

文彦博现在的心情就跟这些灰尘一样,乱哄哄的毫无头绪。

自己离开朝堂太久了,太后垂帘则不过区区一载,毫无旧恩可言。而朝堂之上,能够说得上话的几人,地位又远远不够。新党把持国政十余年,正人君子的亲族全都断了上进的通道。到现在为止,最高的也不过是一个侍制,想要说动太后,他们的份量还是太轻了。

而且自己与韩冈的关系更是恶劣,朝中几乎是无人不知。现在韩冈立下如此大功,想要巴结奉承的一干小人,恐怕都要争先恐后的踩上自己一脚,以求能够让高高在上的韩冈能够多看他们一眼。而朝堂上的其他人,冇更不可能为了文家,而与韩冈交恶。

是不是富弼就是看到了这一点,才计划着要跟韩冈联姻?

一个两个都是一个样啊,富弼的所作所为,让文彦博想起了韩琦,为了维持门楣,脸面丢一边也无所谓。

可有韩冈在朝堂上为其张目……甚至都不要韩冈说话,只要看到其与韩冈的姻亲关系,其他人自然会绕过富家去。

难道最后要求到韩冈头上?

文彦博虽老,却还是不甘心。但事已至此,他也没有别的办法。

挣扎良久,文及甫只听得老父一声长叹,挺直的腰背弯了下去,高大的身躯仿佛缩了起来,整个人更佝偻了几分。

“去拿纸笔来。”文彦博的声音中充满了疲惫,“为父要写信。”

稿纸铺在文彦博的面前,笔墨也准备好了。但文彦博面对稿纸,却久久不见落笔。

过了好半天,他方才一个字一个字的开始慢慢书写。笔端仿佛有千钧之重,让文彦博无法像往日一般笔走龙蛇。

在旁只看了两句话,文及甫的心就咚咚咚的跳了起来。这是给韩冈写的输诚信,是要向韩冈低头啊!

这么多年过来了,终究还是要向韩冈低头认输,文及甫心中一片悲凉,就是当年韩冈只是区区微官的时候,还做着枢密使的老父就已经奈何不得他,到了如今,更是气焰煊赫,让自家老父不得不低头了。

“相公!东京的急报!”

一名仆役匆匆赶来温房。

文彦博手一抖,大大的墨团出现在纸面上。

看着被污损的稿纸上除了墨团之外的区区百余字,文彦博丢下了笔,对仆役说:“拿来!”

这是来自东京冇城的最新消息。

文彦博展开来一看,动作立刻就凝固住了。短短数百字的纸页,他却看了足足有一刻之久。

双眉初时越皱越紧,但不久之后,就与脸上的皱纹一起舒展开来,到了最后,他竟放声大笑。

文及甫惊得目瞪口大,多少日子没见父亲笑得如此酣畅淋漓。

“大人?大人!”

文彦博精神振奋,抬手将桌上的稿纸揉成一团丢掉:“这下就好办了!”

文及甫茫然不解,只能呆滞的看着父亲。

文彦博这一回没有为儿子的一张呆脸而生气,反而笑着问:“知道沈括是哪里人?”

文及甫眨巴了两下眼睛:“……开封府的沈括?……好象是两浙……对没错,就是两浙!杭州的。所以当初先帝才会派他回两浙体量两浙新法推行情况。”

“嗯。”文彦博点点头,又问:“李定呢?”

“好象是扬州的。”

“吕嘉问呢?”

这又跟吕嘉问有什么关系?但文及甫不敢问,“吕晦叔乡贯莱州,他自然也是。”

“不,”文彦博摇头,“他是淮南寿州的……他什么时候帮北人说过话?”

吕嘉问如果从吕夷简那边算起来,他就是淮南寿州人,比江南离北方近一点,但依然是南方。

可若是说祖籍,吕嘉问则是京东莱州,说起来跟韩冈的祖上就是一个地方出来的。

但吕夷简、吕公著、吕公弼能说自己是北人没问题,他们的立场说明一切。但吕嘉问要说自己是北人,包管一群人吐他一脸口水。然后指着地图问,知道寿州在哪儿吗?!——他什么时候不都是站在南人那边?!

“韩冈是哪里人?”

“关西。不过祖籍是京东……大人这有什么关系吗?”

“有。”文彦博点头,随即又大笑起来,“既然韩玉昆有心,老夫又如何不捧个场?”
