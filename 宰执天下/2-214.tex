\section{第33章 旌旗西指聚虎贲(一)}

九月初的陇西已是深秋,草木皆已枯黄,一个月前尚漫山遍野的郁郁葱葱的绿色,现在则成了山岭间的点缀。河中渠中的流水依然潺潺,但叮叮咚咚的水声中,也已透着缕缕寒意。开犁播种的时候快要到了,道边田地中的杂草,已经被焚烧了一遍。王厚正骑着马,行在黑色田地中的官道上。他身后跟着一列车队,几乎都是空载,拉车的挽马头昂足扬,步履轻快的小跑着。

王厚是奉命押运粮草去渭源堡,现在才刚刚回返陇西【古渭】。一行车队接近了县城,于路遇到的商旅和行人多有认识王韶家衙内的,立刻闪到道边,让着他经过。

冬日已然不远,来往陇西的各地商旅又多了几分,都想赶在天气尚好的时候,为今年的生意画上一个完美的句号。城门口熙熙攘攘,王厚的车队虽然身份不同,不过还是在城门处耽搁了一阵。

进城后,亲自押了空车送去工匠营那里检修,王厚调转马头,纵马返回衙门。验了牙牌,进了大门,只见两名没见过面的从人牵了几匹河西骏马,往角落处的马厩走去。王厚与他们擦身而过,瞥眼见到其中最为高壮的一匹黄骠马的马鞍上,正正方方刻着‘仇雠未报’四个大字,文字用浓墨描上,底下朱红马鞍映衬着煞是醒目。

也是有了几分文人习惯,王厚的视线随马而走,盯这几个字多看了几眼。只觉得字体骨肉均亭,大有颜太师之风。马鞍一侧,挂了两支熟铜简,看马鞍给拉得歪倒一边,就知道这两支四棱铜简份量绝然不轻。王厚眼尖,只看到铜简的简身上有银光在闪,定睛瞧去赫然又是嵌了银的四个字。

‘该不会也是仇雠未报罢?’

王厚暗自思忖着,能用上肩高四尺半以上的上品战马,又配了朱鞍,纵还没得到遥郡的兼官,本官也该离横班不远了。这个等级的军头,一路也没几人。

他随口问着门前的司阍:“是哪家的将军过来了?”

“是环庆的姚都监。”

“哦,原来是姚武之!”

得到提醒,王厚一下恍然,想起了传说中在身边所有器物上都刻下仇雠未报四个大字的那个人物。

‘姚兕终于还是到了。’他边想边向内院里面走去,‘三种二姚,倒要看看,这二姚中的老大到底能不能跟三种比个高下。’

种家、姚家皆是西军将门世家。姚家这一代的姚兕、姚麟,少年时起便屡立功勋,很早开始便与种家第三代中的佼佼者——种诂、种谔和种谊三人并称,也即是所谓的三种二姚。不过在种谔飞黄腾达的现在,这个称号,姚兕姚麟都当不起了。

走到内厅门前,因是有客在此,王厚也不便随意入内。按着规矩让守门的侍卫入厅通禀。过了一阵,才被招了进去。

王韶正端坐在帅椅上,多年来风霜和劳碌染白了鬓角,让他比实际的年纪长了近十岁。但居移体养移气,王韶身荷重任,厚积如山的气势,也越发的凌人了起来。

在厅中东首,一名四十不到的将领也四平八稳的正坐着。方脸细目,肤色略黑,算是端正。只是嘴角紧抿,向下弯着,拉出深深的沟壑。一张脸死板着,像是被人欠了巨款……看他的脸色,少说也有十万贯。这位讨不回帐的债主,因为其父死于阵上,便在身边所有的器物上都刻下仇雠未报的标记,上阵杀敌,最是勇武无比。只看外相,姚兕的确英武不凡,不比种诂、种谊稍差,当是名副其实的名将。

姚兕见到王厚进来,便起身告辞。王韶亲自送了他出帐,转回来,王厚便把他运送粮草的任务向王韶交代清楚,缴回了令箭。

王厚顺利地完成任务,王韶这个严父也免不了要赞上两句。

得到父亲的夸奖,王厚心中也挺是高兴。笑说了两句,他才回头问着:“姚武之倒是来得快,朝廷下旨才没几天功夫吧,孩儿只是去渭源一趟,他怎么就到了?”

“大概是因为种谔吧?”王韶这已算不上是猜测,而是符合人情的事实。种谔已是三衙管军,而二姚还只是边疆的中层将领,他们怎么可能会服气?

“姚兕赶在第一个来,开战的时候,说不得也得让他占个先。”王韶又说着。

王厚点了点头。的确,姚兕行动如风,没有半丝拖延,必然要大加酬奖。而王韶能奖励他的,就是开战后一个可以吃肉而不是啃骨头的机会。

……准备开战了。

就在一个月前,在朝堂上反复了半年之久的争执最终有了定论。旧有的陕西转运使路被一分为二。东面为永兴军路,西面为秦凤路,设立转运司,分别以长安京兆府和秦州为治所。

在这次的区划调整中,等于是将原本同归一处管辖的陕西军务后勤,从此划分开来。缘边四个经略安抚司,东面的鄜延、环庆归于永兴军路转运司,西面的秦凤、泾原两个经略使路的后勤转运,则交由秦凤路转运司负责。

泾原经略使路的粮仓渭州,由于知州同时也是泾原经略使蔡挺的治理,几年来政通人和,风调雨顺,粮食连续丰收。加上因为蔡挺的坐镇,泾原从几年前开始,西贼就已经不敢随意涉足,这让泾原路的军粮损耗也减少了许多。因而州中的十几处粮囤中的粮食,几乎都是要满溢出来。

而将拥有从宝鸡到盩厔【今周至】这一片富庶平原、同为关中粮仓的凤翔府也划给秦凤路,其实也是表明了朝廷并不希望看到因为今年白渠流域的大面积减产,在粮食的问题上影响到河湟战略的顺利展开。

永兴军转运司因为年初的庆州兵变,原本最为富庶的白渠周边诸县,都成为亟待救济的地区,一两年内无力再向外做出任何后勤上的帮助。但有了渭州和凤翔府的支持,加上秦州亦是产粮区,而且军屯的成果也十分明显,使得王韶眼下没有后顾之忧。

有了朝廷的支持,彻底解决河湟的时间已经定在了明年夏收前后。而今年的任务,则是翻越鸟鼠山,攻下武胜军——也即是临洮——将大宋对河西的控制区,扩展到洮河流域。

要与木征直接对抗,还要防备之后可能的敌人,通远军眼下的兵力并不足以支持这样的行动。所以今次动员的是秦凤、泾原两路的军队。姚兕是第一个前来报到的将领,而接下来,泾原路和秦凤路的精兵强将也将汇聚于王韶麾下。

上万精兵汇聚一堂,如破堤之势,涌向犹未归附的临洮,让胡马远窜、不敢再行窥伺。再等到明年夏收,官军最后的一波攻势,将如洪水一般,将不肯顺服的蕃人全数淹没,不论是木征,还是董毡。

几年来的辛苦,就快到了最后的时刻,成功即在眼前,王厚幻想之中已是神飞天外,过了半晌才回过神来。王韶见怪不怪,已经低下头去看着自己面前的公文。

王厚不好意思的挠挠头,看看厅内厅外,忽然奇怪的问道,“怎么玉昆不在?”

“好像是酒场那里出了什么事,听了消息,就变了脸色出去了。”王韶没抬头,只用笔指了指门外,“玉昆这么久都没回来,二哥你过去看看,到底出了何事?”

王厚答应了一声,不敢再打扰父亲的工作,轻手轻脚的出了门去。

骑上马,带着亲卫,王厚便往城东行去。韩冈最近向王韶和高遵裕要主持并改造酒场的工作,而陇西县城,原来的酒场就设在城东。

王厚打马匆匆而行,但当他经过一处营区时,一片中气十足的吼声震耳欲聋的暴起,惊到了他胯下的马匹。

在战马嘶叫声中,王厚几乎是滚着跳下马,用力扯定缰绳,将惊慌中的战马安抚。回过头来,他恶狠狠地看着原本是空营的地方。

营中多了一群身穿锦袄、手持银枪的士兵,正排着整齐的队列,在校场上操演着阵法。这群士兵,大约四五百人,正好是一个指挥的数目。人人身高体壮,长枪挥动如风,队列严整似山岳,行动间阵型亦是丝毫不乱,看着就知道是精锐。

‘想不到泾原路的选锋都给姚兕带来了。’王厚长吁一声,怒气收止,‘蔡挺还真是大方!’

选锋并不是军中正式的编制,在枢密院的兵籍簿上也没有这个军额,但四个缘边经略司,都有选锋或是类似选锋军的存在。是各个经略司从配下的军队中,精挑细选的精锐所组成,基本上只有一个指挥,但战力可匹敌数倍的敌军。当初一举攻下了罗兀城的,就是种谔所率领鄜延路选锋。现在姚兕带来的,则是泾原路的选锋。

看了两眼泾原选锋的操演,王厚满意的收回视线。跳上已经安定下来的坐骑,往着酒场赶去。

