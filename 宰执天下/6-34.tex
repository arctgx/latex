\section{第六章 见说崇山放四凶(一)}

快天黑了。

韩冈退回班列中的时候,顺便向殿门外看了一眼。

隔着透明的玻璃窗,泛红的条状云已经变得黯淡了下来,正是黄昏将去,夜色即临的时候。

双脚微微有些酸胀,这提醒韩冈他在宫中已经一整天了。

上了两次朝,又议了一下午的国家大事,精神上还很亢奋,但身体上还是有了疲惫。

不过韩冈作为众臣之中,最为年轻的一位,真要比起耐力来,谁都赢不了他。

要将这一回的争论拖到夜里,甚至明天,包管他是笑到最后的一位。

尽管宰辅们都有了座位,可韩冈并不觉得自己会输给王安石、韩绛这些老字辈,就是章敦也不一定能赢自己,差了有十几岁呢。

其实向太后也赐了韩冈座,而且还因为韩冈不方便做下,连其余重臣都受到了厚待。

但自李定以下,谁也没有与宰辅们平起平坐的想法和胆量,全都坚辞了。韩冈此时还没回到宰执班中,不方便前后同僚都站着,自己却大喇喇的坐下,只能跟着一起站着。

在早已点起的灯火映照下,可以很清楚的看见侧前方的苏颂,他脸上已是疲色尽显,但在压制住宰辅之外的一众朝臣前,他还要在殿上苦熬着。

仅仅是如何给曾布、薛向定罪,政事堂上已经吵了快有一个时辰了。

之前韩冈直言要警惕未被拘押的贼党,要避免他们狗急跳墙。甚至让王厚和李信将火炮给拖出来震慑在京百万军民。

当时在炮声的威压下,一干重臣都沉默了下去,不敢拿自家性命打包票说不会有叛乱。

但随着王厚、李信逐渐控制城中的消息传来,殿中的气氛便随之一改。

李定等人,又重新兴奋起来。

宰辅们坚持维护自己的权威。

韩绛和苏颂都明确的支持了韩冈的意见。

城中人数众多的皇城司探子,他们中肯定有很多人或多或少的与叛乱有所关联,还有皇城司亲卫、御龙四直等禁卫成员,以及曾经与赵颢、蔡确、曾布、薛向等人过从甚密的官员,他们都在紧张的等待着朝廷的判决。如果对曾布、薛向两人的判决过于严格,最后引发大乱的可能性将会直线上升。

而当那群心怀忐忑的叛逆余党们看到朝廷饶了曾布、薛向的性命,就知道朝廷会实现承诺,不会再被人以危言煽动起来,

“但将朝廷的律法胡乱践踏,连叛乱都能保全一命,日后还会有谁畏惧王法?降一等为绞,留其全尸。”这是李定最后的让步,“叛逆不死,不足以儆世人。”

韩冈冷哼了一声。都是死,谁会在乎是成了包子馅,还是完完整整不见血?

韩冈自己都不在乎,想必那些面临死刑的曾布、薛向,也不会在乎两者的区别。

但是很多人在乎,所以绞、斩二刑并为列入律条的死刑——凌迟和腰斩皆不在刑统之中——但绞刑在等级上就要比斩降一等。不及斩则绞,不及绞则流。

说起来,绞刑由于并非立决重案,基本上都会拖到秋决开始后再施行。天下常有灾异,天家之中也常有人重病,朝廷大赦的次数远比想象的要多得多。在这前,如果能撞上大赦,那么就等于是逃过了一劫。很多判了斩的犯人都不在大赦之中,而绞刑多半都在原赦之列。

此外当地方将大辟的判决上书,请求审刑院和刑部批复时。斩刑的批准比例要远高于绞刑,绞刑的判决很多都会给改成流放,以体现朝廷的仁德和慎刑慎杀的态度。

从某种意义上,判了绞刑也就相当于后世的死缓。

尽管李定对曾布、薛向的态度不是要留一命,只是要给他们留具全尸。可不论是朝廷这边,还是在世人的眼中,绞刑就是破天荒的宽待了。

“曾布亦为士人,曾为执政。朝廷若要宽宥,可许其自裁,以全士大夫的体面。”

都不求明正典刑,而是留一个体面给他们……韩冈忽的心中一动,曾孝宽提到的就曾布一个,薛向给丢一边去了。

这真是个悲剧。韩冈暗叹。

谁让薛向他不是进士呢?天生就要受歧视。

“不可。”章敦坚持道,“万一有人不甘引颈就戮,贸然行逆,那样又该如何?”

饶曾布、薛向两人的性命,这是宰辅们给人看的,就是死了一个,也是伤了他们威信。

有了标杆在,下面的官员怎么都不会判死刑。一旦没有了两根标杆,那些从贼党羽,所受刑罚的判决上限,就是绞刑了。不论在朝廷还是在世人眼中是怎么看,在待罪的叛逆党羽们眼中,朝廷始终是要自己的性命。

“既然会从贼,就不要指望他们会畏惧王法。”韩绛看起来也不服老了,依然与人辩论着,“只有看到能保留性命,才会畏惧天威。”

权力果然是能让人充满精力的良药,少了一个蔡确之后,韩绛也开始焕发活力。

两边依然是相持不下。

人多嘴杂,这是一点不错。

如果仅仅是宰辅们共议,许多事几句话就能决定下来。

而加上几十名侍制以上官之后,利益各不相同,便很难做出一个让所有人都满意的决定。所以时间一分一秒的过去,天色由明转暗,他们还是没有讨论出一个结果。今天只是开头,日后若是持续下去,恐怕会更多。

韩冈并不期待殿中同僚们最后会因为效率低下的缘故,而决定定下一个能减少摩擦时间的议事程序来。

什么事直接由宰辅决定,什么事要招两制以上官共议,什么事得将所有在京侍制以上的官员一并招入宫来讨论。

若能定下这样的程序当然很不错,可没那么容易。

相互妥协,那是要建立在实力相当的基础上的。

对剩下的宰相、执政们来说,只凭这一次的功劳,以及蔡确、曾布、薛向三人事败后,更加集中到自己身上的权力,有足够的能力将这些还喋喋不休的侍制、直学士和学士们,一股脑的给干下去,换上一批听话的。

既然有机会有能力,他们为什么不这么干?而去委曲求全?

抱了这样的想法,绝不退让的宰辅,以及以为自己能投太后所喜的重臣针锋相对,崇政殿中的气氛也便越发得紧绷起来,

对立的双方让崇政殿再坐一直拖延下去。

韩冈等不下去了,再次出班,冲太后行了礼:“臣以为时间已晚,不宜延误过久,以免宫外犹疑。”

李定当即反驳:“此事不定,宫外又岂会不犹疑?”

韩绛也怫然不悦:“曾布、薛向不赦,宫外人心如何定?”

韩冈的提议同时引来了两边的攻击。

“韩冈之意,是可以先将此事搁置,把其他事先解决。最后再议论不迟。”

只是换一个议论的顺序,虽然都觉得韩冈有深意在,但他的提议还是无人反对。纵然毫不相让,终归都是累了。

向太后也是听得累了,松了一口气,问道:“依韩卿之意是要议论哪桩事?”

“除曾布、薛向二人之外,从逆被擒之人,数目不在少数。当交由何处审理,不如先将此事定下。”

“这事就交给开封府就好了!”章敦说道,“既然是东京城中的案子,自有开封府负责。”

正常的情况下,犯下重罪的大臣,或是一些争议性很大的案子,基本上是交由大理寺、御史台、审刑院和刑部一起上阵,有时候,还要派去内侍做监审。

但这一回,对叛臣的审判工作,却是交给了开封府。

李定皱了皱眉,却没有站出来表示反对。尽管沈括裁断的结果,肯定会秉承宰辅之意。

可一件事、两件事都要与宰辅们争执起来,在太后那边,就会留下一个恶劣的印象。

更重要的是,太后或许会为了所谓的执中而治,在同意了将审判权交给诸法司之后,便站在宰辅们的一边,将曾布、薛向给放过了。

两边都安抚一下,让事情可以早点解决。太后要是这么做,一点也不会让人奇怪。

李定之前大出风头,几乎成了重臣们的代言人,他不站出来,一时之间,也没其他人出来反对。

“也好,就交给开封府。”向皇后问韩冈:“韩卿,你意下如何?”

韩冈宁可是御史台、大理寺来审。

这一回叛乱,宰辅们要践行诺言,赦免从党之罪,就算没有及早反戈一击,也要免其死罪。

可沈括若是这样判决,不管将判词写得多好,终归难以得到多数重臣们的认同。而沈括本人名声又不是多好,真要被人找起麻烦来,根本防不住几手。

“臣无异议。”韩冈却如此说道,“相信开封府自会依律裁断。如若不然,还有大理寺、审刑院在后复核。”

要想通过判决后诸法司的复核,沈括不能对叛贼的党羽轻判。失入宽纵之罪,沈括担不起。而且判决被法司驳回,没了面子的沈括若不能请出太后为其主持公道,那他就只剩辞去权知开封府一职。

韩冈在想什么?章敦又是在做什么?

王安石在女婿和旧日门人的脸上打了个转,一时想不明白。

