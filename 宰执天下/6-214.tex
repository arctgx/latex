\section{第26章 惶惶寒鸦啄且嚎(中)}

“这么说,龚原已经走了。”

章惇拿着杯盖撇了撇浮起的茶叶,喝了一口。微涩的茶水,让喉咙舒服了许多。

身着红衣的家丁应声:“是。”

章惇放下茶盏,“是回家了?”

“龚管勾雇的马车,走的不是去新城城东厢的路。”

“哦,那他是去哪里?”

“只看到他往朱雀门的方向去了。”家丁脸色微变,躬身道:“这是小人的错,没有遣人追上去。”

“算了,这本也不是你们的差事。找个认识龚原的人,去城南驿问问,从润州来的吕知州去哪里了。”章惇挥了挥手,“快点去办。顺便叫余富进来。”

家丁退了下去,章惇又端起了茶盏,忽的一声冷笑,“就知道是这样。”

余富很快就过来了,面色如常,仿佛平时一般。

待他行了礼,章惇就笑道:“今天的事办得不错。”

余富欠身,然后静静的等待吩咐。对此,章惇更加满意。今天的这件事,确切点说,是办得很好。

余富并非是擅作主张。

哪家的司阍是主家的心腹人才能做。余富虽不是章惇的乡里,但从荆南开始,就是章惇的亲兵,从荆南到广西,章惇出征时他就守在帐门外。

不是秉承了章惇的吩咐,他如何敢自己做主?

龚原之前就已经与御史台一起上书,章惇当时就知道了。之后,开封府对龚原书信的处置,章惇也在第一时间收到了消息。

了解龚原的性格,了解他收到的冷遇,那龚原会找谁来泄愤,自然不难猜测。

本是章惇命余富晾他一阵,观其行止,余富便把事情做到十足十,且话里话外皆抓住了道理,不让上面的章惇难做,

“你跟了我也有好些年了,当年在荆南,没余富你守在外面,我也不能安心下来睡觉。有你守着我章家的大门,也是。不过以后就不用站了,坐!”章惇笑道。

不管怎么说,余富都是让一名进士难堪了,尊卑有别,要是章惇还坚持用他做司阍,不免惹人诟病。所以余富不方便再出现在京师,但他本来就准备给余富更重要的差事,这一回让余富离开,只是顺水推舟。

…………………………

看着眼前怒气勃发的一张脸,吕和卿明白,这是一个机会。

章惇在首鼠两端了许久之后,看起来已经有了决断。被拉出来证明他决心的,或者说,做投名状的,龚原不是第一个,想来也不会是最后一个。

章惇这番做作,一半给东府看,另一半,分明做给金陵那边,和还跟着金陵那边的新党成员看的。

‘要么跟我走,要么跟他走。’

在章惇在朝中支撑多年之后,新党势力大半归于他手,现在已经不需要老人在后面指手画脚,即使自立门户,也不担心没人跟从。

只是不知金陵那边,在听明白章惇想说的话之后,到底会是什么想法,又会怎么做?

吕和卿不知道,可他知道,至少王安石帮不了龚原。

王安石为了保住龚原留在京城,费了不少功夫,不仅跟章惇,还找了韩冈,请他不要再继续穷追猛打。

龚原被赶出御史台,韩冈正是幕后黑手。龚原带着御史台众人,刚咬过韩冈几口,韩冈狠命踹他一记,龚原都叫不了冤。

可王安石说了话,韩冈只能给他面子。

将龚原踢出御史台已经是不小的惩罚,放他回国子监不是大事——监中的新党成员多一个少一个都影响不了大局——若这点要求都不答应,韩冈与王安石的翁婿之情也就到了头。

只是韩冈给了面子,龚原再不知死活的话,王安石再想说话,韩冈也可以不加理会了。

但吕和卿又怎么会为龚原着想?对龚原的话不住点头,义愤填膺的心情更是溢于言表,“余富那厮我也见过,对人颇无礼,就跟他主人一样。章惇骄狂,如今正得志,谁不让他一头?”

不过他心中,却是藏了太多幸灾乐祸的情绪,‘丧家之犬,有本事去金陵嚎去。’

面子是相互给的,真说起来,龚原尽管是个文官,可终归不是现管,军巡院那边已经是给足了龚原面子,自古道拿人拿赃、捉奸捉双,军巡院做得也没错,捉了人,怎么不把证据拿走,不能确定的情况下,多拿一些也正常。现在受了龚原吩咐,人放了,东西也还了,少了一点,做罚金都不足。这样还不满意,那就别怪其他人不给面子了。

“于今得志猖狂的,可又岂止一个章惇?”龚原长声叹息。

“陋寒之家,窭人之子,故而只知锱铢之利,而不见大义。又狂妄而不进忠言,国事败坏便源于此。所谓名不正则言不顺。南征大理劳民伤财,以正大理君臣纲常为名出兵,最后却是夺人土地,从今而后,朝廷可还有脸面说辽人是非?又如何匡正藩国?”

龚原点头:“权臣秉国,虽一时见利,却不知大义已失。”

“可惜,如今东西两府分明已联手,诤言不仅难进于宫中,更难以宣之于众。”吕和卿一边说,一边关注着龚原的反应。

朝堂上发不出来,并不代表民间不行。士林之中的风向,曾经的御史,现在的同管勾国子监事,龚原有着足够的人脉去煽动。

但对吕和卿的话中之意,龚原却是懵然不悟,“是啊,纵使铮铮之言,却无人肯听。却只能见无数小人,秉权臣之意,荼毒百姓,骚扰良善。”“长此以往,民何以堪?民何以堪!”

说到最后,龚原愤然大叫,几乎拍案而起。

他进了御史台后,正欲一展长才,行平生志向,却不意中途为人所沮,以至于前途尽失,现在被人看做是落水狗,人人都想敲上一棒子。这其中的愤懑和屈辱,他在心底已经积蓄了许久。

吕和卿没有沾染上龚原的激动,冷静的摇头,“所谓荼毒百姓,骚扰良善,此皆小事。”

龚原的脸阴沉起来,“不知何为大事?”

“何为大事……”吕和卿森然冷笑,“京师兵马皆从宰相心意,此乃大事也。”

龚原脸上的怒意一点点的消退,盯着吕和卿却不答腔,等着他的下文。

吕和卿却没在意,继续道:“如今权臣反迹未显,人心犹在,忠直之士尚能挽回局面。再过几年,就只能‘试看今日域中,竟是谁家天下’。”

龚原的心脏猛地一跳,吕和卿终于是图穷匕见了。

吕和卿的这几句话,不只是说韩冈,甚至是直指太后——‘试看今日域中,竟是谁家天下’,可是骆宾王为徐敬业所作的《讨武曌檄》。

他恍然大悟。吕和卿附和自己的一番话,目的不是为权臣,而是意在太后,为的是几年后就要亲政的天子。

“太后有功于国。”

犹豫了许久,龚原艰难的说道。

‘无能之辈。’

吕和卿这样评价龚原,不是因为他没有支持自己,而是因为他毫无决断。

做臣子的听到这种话,要么拂袖而去,要么就击掌叫好,不同意,现在还留在这里做什么?

要是龚原真有本事,怎么会从御史台被发配到了国子监中?

“女主秉国,要么见识不明,为权臣所惑。要么便如武瞾,牝鸡司晨,威福自用。纵贤如章献明肃,不也有以天子服祭告太庙之举?”

“但……”

龚原欲言又止,他请吕和卿来,可不是为了与他辩论。既然有求于人,又怎么能一直反驳?只是他本以为能与吕和卿一拍即合,没想到却还是号不准吕和卿和他背后吕惠卿的脉。

“深甫可是想说,如今已非御史,对此无能为力?”

龚原叹道:“同管勾国子监,还能做什么?”

‘正是国子监中才好做事!’吕和卿心中暗叫。

御史台不论,国子监才是重点。

还没有做官,却已经开始指点江山,对已经成为官员的前辈,自是横看不顺眼、竖看不顺眼,觉得自己上位之后,肯定能做得更好。

从汉时的太学生开始,这些学生的愚蠢就没变过。但他们也是一如既往的好利用。更重要的,他们的名声,千年以降,总是一如既往的好——不做事,光说话,要讨好人当然简单——故而士林清议,便以太学生的声音最大。

要想让韩冈难看,朝堂上已无能为力,只有士林清议,方能有所成效。

尽管使动国子监必遭上忌,这么做,等于是放弃了近期翻身的机会,可等到天子亲政,眼前的朝堂便会天翻地覆。只要眼下在小皇帝的心目中留下一个印象,日后待其亲政之后,必有厚报。

吕和卿心急难耐,但还是强耐下性子,“深父莫要妄自菲薄,君子之行,自有遗爱。无论是在乌台,还是在国子监中,深父之望岂为官位所限。”

吕和卿几乎急不可耐的要挑事,龚原心中隐隐约约有了想法,试探道,“说得也是,御史台中终不会人人皆不知廉耻。”

“不,深甫,御史台虽能用,但如今人心离散,早非旧日乌台。若有一二诤臣,今日之事,又岂会容得权相猖狂。”

龚原稍稍坐直了一些,这吕和卿终于说出实话了,“难道是国子监?”

“正是国子监!”吕和卿斩钉截铁,“士林清议,民心所向,皆在国子监中。”
