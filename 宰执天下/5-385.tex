\section{第34章 为慕升平拟休兵(17)}

看着酒气熏天的耶律道宁,萧思温暗暗一叹,又立刻展颜笑道:“统领说的是哪里的话,枢密可没这么吩咐。没有统领坐镇于此,枢密哪里能放心得下?”

说着他就走到老将身边,先示意两名耶律道宁的亲信先退下去,而后方才低声劝道:“统领,萧枢密早前不是已经说了,宋人的撒手锏是翻越五台山的那一支西军!根本就没必要去攻打宋人的营垒啊。只要消灭了这一支翻山来偷袭的西军,就是那边的韩枢密也得认输了。在你这里,只要守住营垒就够了。”

“能翻过五台山的西军会有多少?”耶律道宁放下了手中的酒葫芦,眼中醉意不减,反问着萧思温,“有忻口寨兵力的两成吗?”

“那些都是关西的精兵,是南朝百万军中最精锐的一部。曾经出动了不到一万人,就在半年之内灭亡了一个有十万兵马的南方国家。按萧枢密的说法,得当成是跟皮室军、宫分军一样的精锐看待。别看忻口寨那边的兵力多,但都是没上过阵的废物。真要丢光了关西那边来的精锐,就是韩枢密也只能想办法保住他的忻口寨,没精力去干扰议和。”

耶律道宁哼了一声,却没有反驳萧思温的话。毕竟萧十三的计划,他之前也是知道的,只是有自己的想法而已。

萧思温看了看耶律道宁,暗暗笑了一下,又正色道:“不过宋人肯定要来攻打大小王庄的!虽然是为了迷惑我们的佯攻,以求我们忽略掉五台山的方向。但这样的攻势也不是那么好对付的。你这边的寨防还要再加固一番才好。”

“谁知道他们什么时候到!?”

“这些天不是有报说宋人在赶着修路吗?等路修好了,差不多就会到了。”萧思温指着远方,“现在一辆辆大车将路面给压得一道道车辙,那边的韩枢密就算再有能耐,也不能让装满粮食的大车不去压着翻浆的道路。想来统领你还有一段时间。”

耶律道宁默然不语。宋军的确在修路,但他派出去的探马只有很少人能在近距离看清楚那一条一直延伸到忻口寨的官道。据那些探马回来说,宋人可能用的是版筑一样的办法,运来了许多木板在路面上拼起了木框;也有可能那些木板是用来修补路面缺口的。

虽说耶律道宁对宋人所用的修路技术根本摸不着头脑,但这么特别的修路方法,从过去的经验来看应该很管用。

暮色降临。

自从宣宗皇帝耶律洪基从飞船上坠落之后,敢于登上飞船飞上天际的高官几乎一个都没有。可今天一下就多了两人。

猎猎夜风在耳畔呼啸,耶律道宁和萧思温站在飞船摇晃的吊篮中,各自举着千里镜,远眺着宋军的营地。

确定宋军在前线的大体兵力,确定耶律道宁上报的数目的正确姓,这是萧思温被遣来大小王庄所身负的另一个任务。

萧思温当然不愿意上冒风险,但萧十三的吩咐,他更不敢打半点折扣。在所有探查敌军兵力的手段中,现在的这一种其实是最安全的。

无数点营火在夜幕中勾勒出了宋军大营的轮廓。点点火光密集如星海倒映,在浓黑的夜幕中极为显眼。不过垂首下望,可以发现宋军大营中营火的密集程度,其实跟脚下的大小王庄相差不大,可见兵力也应该相差不多。在黑暗的更深远处,还有几团暗弱的火光,那里就是宋军大营周边几座小营,驻扎其中的兵力可见更加稀少。

“那里就是宋军的大营!”耶律道宁为萧思温解说着地面上的营寨。

“看起来攻过去也不是那么难啊。”萧思温眯着眼睛,在宋军营地中间,有着大批的空间被黑暗笼罩。这些地方必然是宋军战线的弱点所在。

“的确,攻进去很容易。但撤回来就很难了。”

宋军的阵地是以一座座村庄为中心而组成的防线,与陕西在山势中连绵不绝的寨堡营垒是一个模式的防御体系。不过由于是赶工而成,在耶律道宁眼中,整条防线还十分脆弱,到处都是破绽。

如果猝然出兵,出其不意,冲破这样的防线还是很容易的。但代州只是夹在两山之间的盆地,不是河北的千里沃野。狭窄的地势,决定了行动路线的单一。到了想要回返的时候,就要面对严阵以待的宋人,以及他们的陷阱。

关闭<广告>

不敢以大军冲击,那么就只剩下小规模的斥候刺探。可是任何一支军队,对粮道都是视若姓命一般保护。当宋军陆续进驻前线营地之后,加强了对粮道的防守,耶律道宁手下的探马根本靠近不了那一条经过了大小王庄的官道,一直都给封锁在数里之外。

但耶律道宁对宋军营中的兵力数目依然很有把握。在战线上,宋军的明哨暗哨安排得很多,铁蒺藜到处乱撒得让派出去的探马回来直喊心疼——汉人败家子,撒到地上的那些都是一等好铁。可是从炊烟的数量,从营中拉出去的粪车车次,还是很容易判断得出营中宋军的兵力多寡。

“统领,那是什么?”

突然自黑暗深处出现的几点飘摇星火,引起了萧思温的注意。那几点星火并非静止,而是不断移动,一点点的接近。

耶律道宁闻声,立刻将手上的千里镜转向了相同的方向,很快就发现了让萧思温发出惊呼的星火。星火暗弱,在宋军营中明亮的火光照耀下,那几点星火不仔细看根本就看不见。也亏萧思温能发现。

“那是什么?”萧思温再一次问道。

是灯火。

从晕黄的颜色和光线的抖动上,能判定是灯火。

耶律道宁搜索着脑海中的记忆,那几点灯火行进的路线,与官道的位置几乎重合。

“当是宋人。”他肯定的说着。

几盏灯火,前后相连,但数量并不算多,就像两三只萤火虫被串在了一根绳上。有些像一支一直纪律严明的小队借助着微弱的灯光在前进。

萧思温将千里镜死死压着眼眶,一直盯着那几盏灯火从西南方向驶向宋军主营。过了片刻,他的声音陡然拔高:“又来了一队!”

耶律道宁立刻向南方望去。的确,又是几点灯火从黑暗中闪出,沿着他熟悉的官道路线,向宋军主营驶来。

“嗯。”耶律道宁算算时间,差不多隔了半刻钟到一刻钟的样子

“是宋人的援军?忻口寨的宋军过来了?”

“不是,火光太少了。”耶律道宁摇了摇头。一支火炬只能照亮身周数尺的一点区域,要是宋军夜行的话,肯定是一条条火龙蜿蜒而至,绝不会是一星半点的火光。如果是隐秘的行动,则根本就不该点灯。

“只可能是马车,而且是车队。”

“不是运兵的?”

“绝对不可能!”

道路上隔了半刻钟才能看到两三盏灯火连成一队过来,可见车辆的稀少。要是运兵的马车,一夜恐怕一千人都运不到。

“这些天,这样的车队出现的次数很多吗?”萧思温问道。

耶律道宁摇头道:“这段时间宋人都在连夜整修道路,官道方向一直都是灯火通明。只有今天不是。”

萧思温抬起头,追问:“怎么今天就没有看到了?”

“大概是路修好了。”耶律道宁很没底气的说着。

冻土解冻后翻浆的官道究竟有多难走,这段时间他已经从辎重队的抱怨中已经感受到了。宋人运粮多用大车,对路面的破坏更大,按理说不可能很快修好才对。

“这怎么可能?!”萧思温方才还说宋军要到官道修好才能更进一步向前线集中兵力,但现在耶律道宁却告诉他,宋人已经将官道给修好了。

“也有可能放弃了夜间修路。”耶律道宁补充道,“点再多灯也比不上白天看得清楚。”

在夜里从远隔十里多的飞船上观察那两队车马,很难确定其行进的速度。不过自寥寥无几的数量上来推断,好像并没有修好,至少没有完全修好。

“这倒是没错!夜里修路,肯定有很多道路损坏的地方注意不到。所以才放弃了。”

耶律道宁和萧思温很快就认定了道路还要不短的时间才能修复,而宋军只能等到道路修复之后才能大量的进抵前线。

但现实仿佛是在嘲弄人,夜里刚刚信誓旦旦说宋军不可能借用官道快速转运,到了第二天白天,就发现从军营中一队队士兵正从营中鱼贯而出,聚集在在阵前。但不是一万,而是多达两万三万的样子。

“一夜之间,怎么可能到了那么多宋军?!”萧思温几乎是当面给一巴掌打懵了,他指着阵前列阵的宋军,“那肯定是假的,肯定是假的!”

已经顾不上搭理陷入了混乱中的萧思温,耶律道宁一把抓住身边的亲信,冲着被吓得苍白了脸的可怜人怒吼道,“还不快去点烽火!宋军的主力已经到了!”
