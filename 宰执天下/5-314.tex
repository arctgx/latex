\section{第31章 停云静听曲中意(21)}

李信的大纛就在全阵的前方,他本人也立于阵前。身后三百亲兵前后分作五排,阵型远比两侧更为单薄。

在宋军的两翼,骑兵解蹬下马,步兵倒是照常拿着神臂弓。

这分明就是陷阱的模样,但这个陷阱上的饵料却香喷喷的诱人垂涎欲滴。似乎只要一伸手,就能将果子从树上摘下来。

李信的自信到底从何而来?凭借这人数不过千五的精锐,就能胜过数倍的对手?

易州兵马都监在后开口:“李信精擅飞矛之术,听说他在广信军中特意挑选了一批身高体健的军汉,专门教习飞矛,号为选锋。对面跟在李信背后的,似乎就是那选锋军。”

萧敌古烈向后瞥了一眼,这名州都监别的本事不知道,这包打听的手段倒是一流的。

“不就是标牌手!?”耶律菩萨保听说过,宋军步卒中也有拿着标枪盾牌的士兵,远者投枪,近者挺枪直刺,御敌用盾。

“只有标枪,没有盾牌。只攻不守。”萧敌古烈放下了千里镜,他没看到盾牌。只看到被扎在地上的标枪如林。

以善守而著称的宋军,放弃了盾牌,当然是有一份自信在。但落在辽人的眼里,则就是挑衅。

鼓声沉沉,出战的宋军只有一千五六,莫说鹿角,连拒马枪都没有。这是极为明显的挑衅,或者叫做挑战。

麾下大军近万,面对远少于己的敌军,萧敌古烈和耶律菩萨保不愿堕了自己的名声。何况遂城主将出战,只要能杀了他,边界上的这一要隘将会唾手可得,功劳绝不会小。

“不能不打了!”萧敌古烈猛一咬牙,“若说精兵,我大辽也有!”

对宋人的军阵,不想打那就直接绕过去,反正步卒不可能列着阵去追骑兵。而若是想打的话,那就以重兵将之围困。然后如同狼群一般绕行左右,通过不断的骚扰和试探,消耗宋军的精力和士气,包围上一日半日,再强的军队也要崩溃。

萧敌古烈举起了手,招来了三名亲信将领。

“合里只。”他对领头的年轻将领指着阵前的李信:“看到他了没有!”

萧合里只点头道:“看到了。”

“把他的脑袋拿回来!”

“是!”没有豪言壮语,萧合里只简单的行礼接令。在他看来,不过只比吃饭喝水稍稍费事而已。

随即萧敌古烈又点起了后面的两名将领,“你们从两边攻,让两边的宋人忙一点!”

身经百战的将领,不需要吩咐太多。萧敌古烈号令一出,三部兵马随即奔行出阵。

号角声中,三支千人队分头而行,包抄向前方的宋军阵列。

“竟然都是铁甲!”

宋军的行列中不少人心中一凛,来自对面的银光刺痛了他们的眼睛。宋人对辽人的心理优势,有大一部分来自于更好的兵器和甲胄,看见了同样一身铁甲的辽军,不少人的心中都有了动摇。

虽说辽国的钢铁业终究不如大宋,但水力锻锤还是流传到了辽国。只是由于工匠和机械制造水平的关系,甲胄兵器的产量不过大宋的百分之一二,可装备一部分精锐,还是不成问题。绝大多数骑兵同样装备了半身甲,式样基本上就是最简陋的那一种。不过铁甲就是铁甲,再简陋的铁甲对箭矢的防御力,也能提高了几个等级。

“铁甲而已。”

李信探手从身边的亲卫手中接过了一支飞矛,提在手中。随着他的动作,身后的选锋们也跟着摆好了架势。

就是神臂弓射出的劲矢,面对铁甲也只能俯首,不可能一击便能致人重伤。但契丹铁骑身上那一套同样是前后两片组合而成的胸甲,却挡不住沉重锐利的飞矛。三斤多重的点钢飞矛,只要命中,不论是人是马,少说也能带去半条命。

虽说李家嫡传的飞矛之术,只要是称手的长短兵都能拿来掷人,韩冈的老娘就使得一手好擀面杖,但惯常用的标枪,其形制重量也有定制。

并非军中常见的飞铊,那仅仅是盈尺长的精铁短枪,也非南方标牌手惯用的四尺标枪。而是长六尺,重四斤,点了钢的单尖矛。枪尖锋锐无俦,重心更是恰到好处,左右盘旋时极为称手,投掷时更是能顺畅的将全力灌注进去。

这是军器监给韩冈面子,李信写信申请铁匠打造飞矛之后,直接从监中派了上工到广信军的弓弩院来,听候李信的指派,所造出的飞矛,成本甚至可比斩马刀,威力却也不遑多让。

骑兵飞驰,越过城外的田地,踏着犹如生铁一般的冻土,没有草木的地面上,一片黄土飞扬。大地在铁蹄下震颤着,如雷音从地面响起。

将兵马每五六百分作一部,前后多列,轮番冲阵,此乃是辽人克敌故伎。但李信出兵不多,站位紧密,攻击宋军的骑兵若是铺开来反倒浪费了人力。辽军的三支千人队便各自以百人队前后分列,相隔二三十步。在奔行中,自然而然的便分了开来,有如浪潮一般向宋军涌去。

冲击两翼的辽军领先一步,直冲目标而去。

百步之内,快马瞬息可至。神臂弓一射之后,便无再射的机会。一旦剪除了羽翼,宋人的中军再强也只能授首。

面对辽军的汹汹来势,李信安排在两翼的步卒各自转向,面对外侧包抄袭来的敌军。下马的骑兵都用短戟将战马的缰绳固定在了地上,张开了随身携带的神臂弓。不过除此之外,阵中的宋人没有任何反应,呆若木鸡,沉稳得让人心寒。

百步不发箭,八十步不发箭,到了六十步,奔行在最前的辽军骑兵,忽的向外一转,在阵前打横而过,后阵的骑兵一排排的紧随在后,脆弱侧面暴露在箭矢的射程中,但依然没有引得宋军发箭。

李信在广信军逐日练兵,一手鞭子,一手银钱,用了一年多才练出了区区五六个指挥可堪一战的精兵。他们并没有多少本事,只是听话,不闻号令,便不敢有所异动。

目标是李信的萧合里只,同样带着麾下骑兵在五六十步外兜了一圈,都没看到宋人中军有半点虚怯的模样。

转回到出发地,萧合里只举起了手中的长枪,先一步回转的侧翼兵马又早一步启动了。

六十步冲不动,那就四十步,一步步突前,不信宋人还能忍得住。只要神臂弓一用,没有拒马、没有鹿角,这样的军阵,接下来只能任人鱼肉。

辽军最后一队刚刚从阵前横过,这第一队又上来了。声势更猛,速度也更快。

城头上观战的文官武将,士兵百姓,心都提了起来,前一回辽人似乎只是试探,这一回来势汹汹,一旦应对失措,失去了弓弩的官军可就会直面辽骑的冲击。

面对冲得更近的辽人骑兵,两翼的阵列依然不动如山,虽说细看起来,已经有不少人的手在颤抖,但长年累月交替的鞭笞和赏赐,让他们养成了受命后方才敢动手的习惯。

四十步的时候,第一波辽军士兵在阵前回转,而马背上的骑兵也在转向时趁机张弓射箭,划着高高的弧线,落向宋军的阵列。小小的骚动无伤大局,后方城头上的人们松了一口气,战马的冲击力有限,两轮无功,第三轮必然会直冲而来,确定了敌军的来势,也就好应对了。

但下一刻,他们的脸色就惨白了下去。攻击中军的敌骑千余,却没有绕行,而是直奔中军大纛而来。

“辽狗!!”城上的宋贤厉声大叫。

两翼的骑兵最前面的两波都绕回去了,这就给人了一个中军也会回绕的错觉,偏偏这里是直冲过来。一点点错觉,就能乱了标枪投射的时机,九州之错,再无挽回的余地。

一声尖利的号角声就在同时自远方传来,两翼阵列前正一波波冲过来的辽军也突然不再转向,踏着最激烈的步点,配合着中军如浪卷而上。

中军大纛下,可只有三百人应对。

恐惧在一瞬间攥紧了城上所有人的心。

萧合里只紧紧盯着站在阵前的李信,与坐骑一起飞速的接近。人和马仿佛融合了在一起,躁动的血液将力量传到了枪尖。但他在敌人的脸上没有看到一丝动摇,只看到了宋军的主将高高举起了手中的长矛,然后带着风声一挥而下,直指前方,直指向自己。

大纛猛烈摇动,鼓音随之一变,暴喝声同时在阵中响起,两翼步卒和马军射出了上弦已久的箭矢,而李信身后,第一排的选锋重重的向前踏步,一丛标枪刹那间便腾起在空中。

何为选锋?

‘兵法曰:兵无选锋曰北。昔齐以伎击强,魏以武卒奋,秦以锐士胜,汉有三河侠士剑客奇才,吴谓之解烦,齐谓之决命,唐则谓之跳荡,是皆选锋之别名。’

‘须乔健出众、武艺轶格者,部为别队,大约十人选一,万人选千。所选务寡,要在必当!’

遂城军中六千余将士,李信只选出了三百。军中标牌手二十步中的为合格,而李信麾下选锋皆是三十步向上。

人数虽只有三百,却能作乾坤一掷。
