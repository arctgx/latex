\section{第29章 雏龙初成觅花信(上)}

长大成人,从来都是有两层意味。

出自韩冈之口,再考虑到天的年纪,想也知道,到底是哪一层意思。

王旖一时间不知说什么才好,脸微微变红,先瞪了口无遮拦的丈夫一眼,然后抬起手来,帮韩冈整理好了入宫面圣的装束。

看王旖的表情,韩冈知道自家的儿们有的苦头吃了,儿身边的小厮和使女,这几天都少不了被训诫一番。

不过也幸好有这么一位严母在家管着儿女,韩冈才有余暇去安安心心的处理朝政。

要是闹得向当今的官家一般,才十二岁便近了女色,又弄得身体虚弱,脚软得出福宁宫时差点就晕倒,韩冈不说无心用事,朝臣们口,也少不了成为被嘲笑的对象。

辞别了妻妾,外院早就准备好了随行的车马,韩冈登上马车,一行人便出门往宫城驶去。

听着车外的人声鼎沸,韩冈静静地合上双眼。对于小皇帝闹出的这一出,的确是出乎他的意料之外,但细思下来却是平常,甚至是觉得哭笑不得。

但天家的事,就是天下的事,宰相兼理阴阳,并掌内外,皇帝家的闺房事,韩冈照样得管,也管得着。

不过即便已经接受了诏命,韩冈也并没有快马赶进皇城。为了避免京军民惊扰等原因,他备齐了旗牌,慢慢,花足了近半个时辰,才从宰相府邸来到太后的面前。

韩冈没有在内东门小殿发现其他朝臣,只有他一个宰相被传召入宫。

向着太后躬身行礼,“臣韩冈拜见陛下。”

“相公终于来了。”

向太后本是等得心焦,即使心知以韩冈的性格,绝不会匆匆忙忙便乘马入宫,也依然忍不住心的焦躁。直到韩冈终于出现,就像是有了主心骨,整个人都轻松了下来。

“官家的事,想必相公已经知道了。”向太后叹着气,“这不成才的孩儿,又要劳烦相公了。”

尽管世间风俗还是将男女之事放到十四五之后,但十二三岁就谈婚论嫁在民间也并不鲜见,赵煦开荤,太后也没有觉得事情大到要惊动宰相的地步,也觉得不方便说。只是天因此而发病,就不能再隐瞒了。

“天事,便是臣的份内事。”韩冈略低了低头,“何谈劳烦二字?”

赵煦亲近女色,绝不是一日两日,福宁殿,也有太后派出的人,要说太后都没有收到消息,韩冈打死也不会信。若是将天的变化早早通知朝臣,做臣的也能及早做出应对,可惜向太后并没有这么做。

向太后道:“那依相公之见,此事当如何处置?”

“此事陛下不必忧心,自有故事可循。”韩冈道,“不知天现下如何?御体可还安康?”

向太后道:“尚算万幸。钱乙方才过来给官家看过了,官家并无大碍,只是需要调养一阵。”

韩冈一幅安心的模样:“那臣就可以安心了。”

太后、宰相一本正经的讨论天开始亲近女人了,听起来着实荒谬,但这的确是事关国家的大事。

天终于开了荤,论理说这是可喜可贺的一件事。皇帝玩女人这哪里是问题?不玩事情才大。间或找找内侍,虽是少有,可分桃断袖也是士林熟知的典故。

过世的高太皇性刚好妒,不让英宗皇帝接近嫔妃,曹后告诫,韩琦谏言,都是为了要让英宗能御女生,为天家开枝散。高滔滔听得烦了,她甚至回了曹太后一句‘奏知娘娘,新妇嫁十三团练尔,即不曾嫁他官家’,就是要把过去怎么管‘十三团练’赵宗实的手段,沿用到如今已经改名赵曙的新皇帝身上,把她嫡亲的姨母气得不轻。

向太后绝不操心日后天不能亲近嫔妃,她只担心天亲近得太多了。

韩冈话说到一半就岔了开去,也有些不高兴了,“相公安心了,吾可没安心。这桩事,相公也该给吾拿出个章程来。”

“不知陛下心意如何?”韩冈反问。

“官家才十二,就被人蛊惑,身边的人都不能留了。可吾就是担心这么做,朝堂又要闹上一阵了。”

如今天下安定,四民康安,边境上有强兵戍守,朝堂更是贤臣罗列,向太后平日里过得舒心得很,最烦的就是有人弄得她不能安生。

韩冈应声道:“其实此事如何处置,自有故事在。仁宗时尚、杨二美人受责出宫,便是前例,陛下的决定并无错处。至于朝堂之上,陛下久主朝纲,又何须担心?”

仁宗皇帝昔年在赶走了郭皇后之后,与尚、杨二美人,玩一龙二凤玩得日以继夜,所谓‘每夕侍上寝,上体为之弊,或累日不进食’,几乎就要精尽人亡,闹到‘外忧惧’的地步,还在世的杨太后几番告诫,入内都知阎应更是每天从早到晚的在仁宗耳边喋喋不休,最后吵得仁宗不厌其烦,也觉得自己的身体不行了,最后终于点头同意将尚、杨两人逐出宫去。

向太后听说过这件宫闱旧事,当年她随着赵顼进入皇宫之后,便被曹太后派来的老宫人耳提面命,要怎么服侍太才算是一名合格的太妃,这其没有少拿尚杨二美人的旧事作为例。

“相公的意思是就这么办?”

“若按臣的心意,此事当让天自己来决定。”韩冈瞥了一眼殿的宫人们,放声直言,“以仁宗之仁,郭皇后却不得善终,不免令人无憾。”

韩冈的话够直白的,说是挑拨离间都可以。

但向太后毫无介怀,而韩冈也并无一丝一毫诚惶诚恐的心态。

“相公这话说的有理。”向太后点头,“这件事得让他自己知错了才好。蓝从熙,你先去福宁殿,与太妃说,吾这就同韩相公过来探视。”她看看韩冈,“请相公随吾同去福宁殿问问官家。”

“臣遵旨。”

向太后坐上肩舆,韩冈跟随在后方,离开内东门小殿,一路往福宁殿去。

天寝殿,韩冈过去来得多了。

但自当今天登基之后,尤其是宫变之后,几年间便只有零星几次。

走进福宁殿,一切的陈设犹如五年前一般,几乎什么都没有变过,连正殿的那一张旧御桌还摆在原地。桌脚漆面斑驳,这么些年了,看起来也没有重新上过漆。

前些日,韩冈曾听说向太后准备将这桌换上一张新的,但赵煦却拒绝了,说是‘此乃先帝旧德,孩儿不敢弃’。赵煦的这番话传到外面,惹来了一阵唏嘘。赵煦好心办了坏事,只能说是夙世冤孽,尽管弑字脱不掉,可也没人怀疑他的孝心。但今日事发,可就有些问题了。

跟随太后走进天安寝的偏殿,围绕在赵煦身边的宫人,齐刷刷的矮了半截。

韩冈没看到郝随、刘友端、朱孝友,也没看到国婆婆,在钱乙确诊之后,赵煦身边的内侍、宫女,乃至乳母,全都给关了起来,福宁殿,尽是太后身边的人,杨戬领着人守在殿外。韩冈从抵达福宁殿门外开始,除了看到旧陈设,就是熟面孔。

赵煦惨白着一张脸,半躺半靠的倚在床上,看起来是想要下地来迎接向太后,却被其他人给阻止了。

寝殿的另一头,小门上的珠帘还在晃动。方才尚在寝殿照料他的朱太妃,在听到韩冈随行而来的消息之后,先拜见了太后,然后在韩冈进来前,就匆匆从另一头的小门处退了出去。只是在摇晃的珠帘对面,隐约可以看见有人影在窥伺。

“官家可还好些了?”向太后走到御榻边,关切的看着赵煦。

“孩儿多谢娘娘垂问,已经大好了。”赵煦匆匆说了一句,又看向韩冈,投过来的视线有些慌乱,“相公也来了。”

“陛下御体有恙,臣岂能不来?”

韩冈上前两步,沉着脸,语气冷然。身为底蕴深厚的宰辅,皇帝要是哪里做得不好,直接训斥也不打紧,更何况赵煦的帝位还是他一手保住。

向太后一见韩冈要教训皇帝,便连忙起身,离开御榻,让韩冈单独面对赵煦。

赵煦低下头去,细长的双手紧抓着浅黄色的被套。

也不知是不是在学他父亲,被褥外罩的颜色都退了,还是照样坚持用着。能够节俭是好,但现在可也帮不了他脱罪。

“陛下,亲近女色乃常事,却也要顾及御体。《春秋》便有云,‘是为近女室,疾如蛊,非鬼非食,惑以丧志。’女非不可近,惟需谨记‘节’之一字。”

韩冈在这边教训皇帝,向太后在一边听得有些脸红,在桌上随手拿起一个杯,让人来倒水,这些话本不方便当着女来说。

韩冈则是浑没在意,继续道,“圣教所谓庸,也有此意。不宜过,过则伤身,不宜戒,戒则无嗣。更何况,陛下又年幼,松柏日后纵能参天,但树苗时常常摇动,坏了根基,日后也难以长成。臣一番肺腑之言,愿陛下熟思之。”

韩冈的话一贯不多,赵煦待他训话结束,缓缓抬头,苍白的脸上双眸幽深,“相公的话,朕一定铭记在心。”
