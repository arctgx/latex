\section{第33章 为日觅月议乾坤(上)}

“娘娘是这么说的?!”

赵煦猛地站了起身。 但立刻就坐回来,紧张的望着水榭的门口。

“怕什么,你我母子说些体己话,有哪个敢打扰,打死了事。”

朱太妃凤目剔起,视线在门前掠过,她方才将赵煦身边清了场,可没人敢硬顶着。

十年前她以丽色闻名宫中,如今也依然颜色不改,脸上都看不见岁月留下的痕迹,但尖锐的表情,在容色上平添了一份狠厉。

赵煦紧张的神色也没有消退多少,勉强的笑了一下,“娘,娘娘当真是这么说的?”

“痴儿,要是不确定,娘怎么会跟你说?”

两年的时间,尽管身边亲近已被一网打尽,左右近侧皆是保慈宫中人,但向太后再怎么心狠手辣,也不可能将天子的生母一并处理掉。

只要还有这么一个缺口,赵煦的耳目便不会闭塞在区区宫城之中。

“但娘娘这么做,也不一定是让孩儿亲政。”

“官家,成了亲,便是成人了。成了人,还能不亲政吗?”

赵煦不敢如此天真:“可仁宗皇帝大婚之后,也没能亲政。”

“也有慈圣和你祖父。”

“可韩冈与章惇二人相互勾结,朝堂上又不见有一个韩琦。”

朱太妃探手摸着赵煦的头,几年前还是剃着光头,只留下几撮小角儿,如今已经把头发给留了起来,越看越像是大人了。

“娘是妇道人家,但也知道,天底下不止有权奸,也有诤臣。官家是人心所向,那些宰相堵不住。”

见赵煦仍是紧皱眉头,她心下一叹,“娘知道你担心保慈宫,要是她敢对官家做什么,娘也不会干看着,总能闹个灰头土脸,看她还能将娘给……”朱太妃话声猛地一顿,隐去了尖锐的表情,换上了一副语重心长,“官家一定要好生读书,不要辜负了太后的一片苦心!”

一名四十多岁的内侍出现在门前,眉浓目细,鼻钩仿佛鹰隼。

朱太妃在他的盯视下站起身,谆谆嘱咐了一番,然后莲步轻移,在一众宫人的护持下告辞离去。

冲着亲生母亲的背影,赵煦慢慢的弯腰:“小娘娘慢走。”

重新起身,赵煦的心里没有任何期待。

他没有朱太妃那样乐观,太后的这句话,也许只是为了不想亲口否决,而让宰辅们出来反对。

想起元佑以来,几乎只入不出、只内部调整的两府,想起两府中的那几位,赵煦完全不相信他们会轻易的将套在自己身上的绳索给松上几分。

一群窃国之贼,怎么可能给自己机会?

……………………

“太后是这么说的?!”

“冈亲耳所闻,岂会有假?”

“玉昆,是不是宫里面有什么言辞让太后难做?”

“没听说。子厚兄你听说了什么?”

“听说了也不会问了。”

两府宰执会于都堂。

苏颂照常例不至,郭逵告病,沈括居外。其余宰执,昭文相章惇、集贤相韩冈、枢密使张璪、知枢密院事熊本、参知政事邓润甫、参知政事曾孝宽,皆列席其中。

现任知枢密院事的熊本,在下首处听着两名宰相的对话,一边小心翼翼的从嘴里将一片茶叶给取了出来。

全都是草根树渣。熊本又小小的啐了一口,将碎末啐了出来。

这种炒青,他最早喝着还算新鲜,但时间长了,还是觉得过去的团茶更合口一点。偏偏政事堂中使用的茶汤都是附和韩冈的口味,多久日子没有使用团茶了。

即便政事堂总能从贡赋中得到不少团茶提供给官员们日常饮用,可如今也只是将之作为年节赐物的一部分,发给中书门下的所有官员。

这些都是之前政事堂中人为了讨好韩冈,才如此改了一通。

章惇如今虽是入主政事堂,可他根本就不在乎自己喝的到底是团茶还是炒青,就是白水,他也照样不在乎。故而政事堂中寻常提供的饮品依然是炒青。

不过真要想喝团茶,还是可以喝得到,但就跟大部分新人和客人一样,熊本还不愿如此张扬。

“太后若是真心。”就看见章惇皱了半天眉头,然后转向韩冈:“玉昆,你怎么看?”

“所以来请教诸位的意见。”韩冈又一次一推了之。

这是在唱哪一出?对唱吗?

熊本心下不屑,嘴角也拉了下来。两名宰相一搭一唱,如此默契,两府之中,还有别人说话的份吗?

自从章惇担任宰相之后,韩冈从未与他争权过。

尽管朝堂上大多数朝臣都心知肚明,韩冈他是以十年二十年为期,去培养气学的弟子。他的门生迟早会蜂拥于朝堂之上。

但十几二十年之后的事,有几个会去在意?

真到那时候,章惇不是回到泉州做太守,就是在平章的位置上没精力管事了。让韩冈去掌控朝堂又如何?

而现在,有韩冈的配合,章惇只要注意不去侵犯他的那点自耕田,便可以放心的去操弄朝堂大政,其余辅臣,也只能避退三舍。

熊本放下杯子,这茶喝得殊无滋味。

张璪之前本是知枢密院事,早前断断续续病了一年,照常例该自请离职养病,太后念着旧日之德,一直留他在西府之中。韩冈、章惇对此都表示了赞同。

现在张璪已经是枢密使,寻常做的事,就是附和章惇与韩冈。

干脆让章惇、韩冈兼领枢密使得了,熊本不论是在政事堂还是转到枢密院后,都一直觉得很憋气。

政事堂中,两位宰相都是战功煊赫,所以在军事上的发言权,决不在熊本之下。而且因为两人是宰相的缘故,声音甚至会更大一点。

熊本无意去比较谁的功绩更高,只在意是不是有人侵犯自己的职权。

天子大婚一事,本就没有西府说话的份。除非自己是做过宰相,又去做枢密使如文彦博那般才有发话的权力。自己一个晋身不过两年的知枢密院事,既没有根基,也没有底气去在这件事上插话。

“其实这件事,两位相公一言可决。”

熊本就坐在邓润甫对面,东府的这位参知政事脸色不太好,听他说出来的话,似乎也不怎么痛快。

“天子素来体弱,是否能够大婚,韩相公说一句,可比任何人都管用。”

不要宰辅们合力,只要韩冈说一句不合适,将天子大婚的时间拖到十七岁也没有关系。这是谁都知道的。

而章惇作为首相,只要在朝中无太大争议的情况下,将天子的婚期向后拖延一段时间,这同样不是什么难题。

“我等行事,事关家国天下。韩冈与医道上薄有威名,但天子大婚之事,岂能一身专决?更何况,天子的身体完全没有问题,随时可以大婚。”韩冈扭转身子,盯着浑身不自在的邓润甫,“我可以明确的对温伯你说,韩冈过去没有过用虚名谋取私利,今后也一样如此。”

邓润甫自觉失言,不敢与韩冈相争论。

其他人则各自做壁上观,章惇作为首相只能站出来,

“我看还是早一点好。”章惇沉声道,“朝廷好不容易才安生几年,没必要弄得鸡飞狗跳,多少人家难得安宁。”

天子**婚和天子十四大婚,哪个选择会让天子婚后亲政的呼声更高,当然是不用多想的。

而对于所有在做的宰执们来说,眼下的权力结构,没有改变的必要。

不论是邓润甫还是熊本,都不觉得自己能通过宫中的变动,抢下章惇或是韩冈的位置,一旦章、韩有失,得意的只会是京城外的那一干人。

“当如相公之言。”

张璪首先表示赞同。他的利益与太后紧密相连,又是章惇、韩冈的盟友,西府在他的领导下,大事小事都跟政事堂一个鼻孔出气。

“孝宽亦觉此事当尽快措办好。”曾孝宽随即附议。

“伯通?”章惇看向熊本。

熊本道:“儿女婚姻,自是父母定夺。既然太后有言,我等自当依从。”

韩冈点头:“韩冈之意亦如此。”

“太后的想法还没确定吧。”邓润甫道。

“不论太后心意如何,天子还是早些大婚为上。不过……”章惇对韩冈道,“玉昆。若太后心意不定,还望玉昆你能陈说利害,尽量说服太后。”

父母之命、媒妁之言,天下哪一户人家结亲不是如此?

天子大婚,同样也要按规矩行事。父母之命总不能少。强逼太后让天子大婚终究不是一件好事,几位宰执也都希望这是太后真心如此做想。

韩冈点头,“理当如此。”

韩冈做出了保证,邓润甫再无他话,点头同意。

宰辅们达成了协议,便各自散去。

章惇和韩冈留在了最后。

“如何?”

章惇端起茶杯,悠悠的喝了口冷茶。

“看起来没什么问题。”韩冈道。

以韩冈的为人,太后说要措办天子大婚,他怎么可能不问清楚?

要是为太后解忧,帮她说不好说的话,韩冈回头直接就安排人去办了,也就私下里跟章惇通个气,根本就不会在这里召集一众宰辅。

当然是试探。

“不过熊本心怀犹疑。至于邓温伯……”

“温伯那边不用担心。吕吉甫上来后容不了他。”

“那就当真没问题了。”

“那么,接下来……”

“就要看看哪家的女儿合适了。”
