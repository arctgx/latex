\section{第14章 飞度关山望云箔(四)}

宾州州衙的花厅中,何学究鼻青脸肿,一滩烂泥的瘫在地上。方才韩冈将他送给宾州百姓处置,差一点就被打死——要不是韩冈亲卫拦着不让下重手,他的确已经被打死了。

韩冈低头看这个标准的汉奸,“知道本官为什么要将你交给宾州百姓,”

何学究挣挫着爬起来,端端正正的跪好,头埋得很低:“小人不合从逆。”

韩冈身子前押,冲着何学究厉声道:“光是附逆从贼。只这一桩,断你凌迟都是该的。更别说屠戮百姓也有你一份!”

何学究咚的一声响头磕下来:“官人明鉴,屠戮百姓实不干小人的事,小人这辈子连只鸡都没杀过,当时可是尽力劝过的。”

“劝?你是分赃吧。”韩冈嗤笑一声,容色转冷,“刘永出来怎么会随身带个废物?你应该没有少出主意吧……”

“小人真的没有,小人真的没有出主意。”何学究连连磕头,这个罪名他是绝对不敢认的,“刘永杀人放火的时候,小人还在旁边规劝来着。”

“如果你只是在蛮帅洞主身边做个清客,那本官就用不到你了。”韩冈叹了一口气,这一位才智太低了点,换作是头脑灵活的,开口就该知道自己要用人,“来人啊,送他出去。”

两名板着一张脸的亲卫大步跨进厅来,左右将何学究夹了起来,就作势往外面拖。何学究心中慌了,奋力挣扎,“官人!官人!小人的确是谋主!小人的确是谋主啊!”

不见黄河心不死,不见棺材不掉泪,这时候才肯承认。“回来!”韩冈一招手,亲卫转回来,将何学究摔在地上,又大步走了出去。

“他当真派得上用场?”李信眉头都拧起来了,低声问着。

“只是传个口信而已。若是没用,那就真的没办法了。”韩冈低声回应。看看苏子元,脸色也一样是难看。

待何学究重新跪好,韩冈直接道:“你是刘永的谋主就好,将你的姓名籍贯报上来。”

何学究愣了一下,见韩冈双眼剔起,心惊胆颤的立刻回话道:“小人何缮,何为则.民服的何,缮宇葺墙的缮。本是柳州人氏。”

“何缮……”韩冈念叨了一声,让人捧来笔墨,又让亲兵拿出一个匣子,从里面抽出一张纸,提笔就在上面写了两个字,交给亲兵拿给何缮。

何缮看着韩冈拿出背面颜色纹理特异的那片纸就心中有了一点底。等到亲眼看到之后,更是浑身抖了起来。那页纸上只有寥寥数行,可有印文、有画押,填着姓名的地方墨迹淋漓,上何下缮,正是他的姓名。

何缮咽了一口唾沫,抬头望着韩冈:“官人……”

“本官奉旨南下,得赐空名宣札二十道,以备封赠功臣。现在这一道已经写上了你的姓名,只要本官将之送回京中三班院,那你就是大宋的一名臣子了。”韩冈示意亲兵将填好了姓名的宣札拿回来,就在何缮眼前晃着,“只要肯用命,朝廷又何吝爵赏!?就算曾经附逆从贼,只要改邪归正,照样能为朝廷所用。”低沉的声音犹如魔鬼在利诱,“何缮,你是想在广源州做一辈子的清客,还是想要弃暗投明,做大宋的忠臣?”

何缮喉咙很干,心跳很快,两只眼睛直勾勾盯着那张薄薄纸页。这样的宣札都是中书签押过后才发下来,每一道都能让一个平头百姓成为一名大宋国中吃着俸禄的官员,韩冈不可能拿着这么贵重的东西来欺骗自己。

当年的侬智高之乱,在广西就有许多人靠着狄青带来的空头宣札得了官身。最有名的石鉴,他当年可是广西不第秀才,但他帮着平定了侬智高之乱,现在则是在朝廷做了大官,听说都是入京了。广西士子考中进士不知有多难,哪个不想做石鉴第二。眼下多少读书人一辈子都在求不来的东西,已经写上了自己的名字,只要自己能让眼前的这位年轻的韩运使满意,那自己就追随着石鉴,成为一名货真价实的官人了。

大宋的富庶天下哪国能比,大宋的官员都是富贵荣华,能做大宋臣子,给十个洞主都不换的。何缮重重的磕下了头来,“愿为大宋忠臣。”

一旁的苏子元冷哼一声,要不是知道韩冈全都是为了救援邕州,他可绝不会同意给此人一个官身。

韩冈等何缮抬头起来,“想必何缮你也清楚,这一份告身不是这么好拿的。本官当年也是先靠军功入官,出生入死也没少过。不过朝廷给的回报也多,从入官到如今正好六年,已经做到了转运副使。

另外有一人的名字想必你应该听过,侬智高之乱时立过功的石鉴,他如今正在宣州做着知州。要不是章学士自请出外,桂州知州本来应该由他来接任的。想想吧,布衣入官二十年就是经略使,这一切是怎么来的,是拼命拼来的……何缮,你敢不敢拼一次?”

听着何缮心中正烧着一团火,脸上的疖子都泛着血红,抬头大叫道:“富贵险中求,小人敢不尽死力!官人有什么吩咐,小人拼了性命也去做得来。”

“很好。”韩冈点着头,“本官要昆仑关。”

……………………

“他要昆仑关?!”黄金满坐在大厅中,眯起眼睛盯着何缮。

“没错。”何缮点着头,在镇守昆仑关的蛮军将帅面前竭力不让自己的膝盖发抖,“正是昆仑关。”

黄金满嘴角扯动了一下,讽刺的笑容在脸上划过,“有本事就来攻打昆仑关,想凭张嘴就让俺将关口让出来,世上哪有这么便宜的事?!”

何缮摇了摇头,“就是因为打不下来,才会派我来劝说洞主。”

一阵哄堂大笑。连同黄金满在内,几个蛮将都放声大笑了起来,“打不下来才来劝?你说的那个韩运使恐怕不是疯子,就是蠢材!

“应该是即是疯子,又是蠢材!”

何缮脸涨的通红,只是背后传来的两声咳嗽,让他冷静下来。

何缮还记得韩冈的话:‘你之前附逆从贼,和刘永一起在宾州犯下的这些罪过,宾州百姓恨不得寝皮食肉,今日一战胜得如此轻松,也是百姓们的功劳。现在交趾兵犯大宋,在钦廉二州杀戮无算,眼见着邕州也要攻下来了,你说天子会怎样想?绝不会放过任何一个!’就是这番话让他有了底气。

在笑声中,何缮坚持说着:“韩运使和李都监作为先锋,带来的兵力,洞主应该也知道了,只有八百人。虽然这八百人将刘洞主的千名精锐杀了个干净,也不过折损了一点点而已,但要攻下昆仑关,还是略显微薄。而来援广西的荆南军主力,现在尚在桂州,要先筹备好粮秣军器,差不多要一个月后才能抵达。至于朝廷调发来平南的三十万大军,更是要半年时间。中国幅员万里,国力鼎盛,可是要从天南海北选调精锐过来,就要耽搁太多时候。若是等着大军前来,邕州难保。”

何缮环目一扫静下来的厅中,“韩运使从桂州领军南下,只为了救援邕州。如果邕州被攻下,也就不需要再来急着攻打昆仑关。只需在宾州等着朝廷大军抵达,到时候,十万大军杀到关外,试问洞主能挡得住吗?”

“那时候我们早回去了。”一个年轻的蛮将不服气的说着。

“你们能回去,难道官军就不能追过去?!还记不记得狄太尉?还记不记得侬智高?!”何缮的声音一下提得老高。

“广源州来过几次官军?”黄金满问道。

“两百年前,交趾何曾不属中国?”何缮反问着,“在下是为了救援邕州才派在下来劝说洞主。否则依着大宋天子的诏书,可是要将广源和交趾都斩草除根、鸡犬不留!李常杰说朝廷大军不能南下,那是骗你们为他赴汤蹈火。如今邕州将破,你们可分到一点好处?”

“在永平寨和太平寨,哪家没分到?”又有一名蛮将反驳着。

“那点点人口金帛,可是要拿命换的,可比得上朝廷的赏赐?”何缮看了一圈厅中的蛮帅蛮将,“韩运使让我来问诸位一句,同样是做看门狗,是给朝廷看家护院好呢,还是在交趾人的手下好呢?!”

厅中一阵静默,何缮说出了他们的恐惧。大宋太大了,而交趾太小,至于广源州则更加的小。大宋如同一只老虎,而他们仅仅是一只老鼠而已。老虎虽然再睡着,但只要一醒过来,一巴掌就能将他们拍死。而交趾人,根本不会帮着他们。

“要不是刘彝禁绝市易。我们也不会违抗朝廷。”有人嘟囔着。

“刘彝已经罢官,现在是章学士做桂州知州,平了交趾,市易就会恢复。”何缮催促着,“洞主,韩运使是一心想救邕州,如果邕州城被攻破,可就没有这等好事了。到时候,可就是玉石俱焚。”

“昨日已经上了城,邕州也就今天、明天了。”又有人说出来邕州的现状。

“那还不快?!”何缮厉声断喝,有着朝廷做靠山,他说话也越来越有底气。

“但关后就有李常杰派来的一队人马监视。你叫我们怎么让?”

“那是你们该去想的事。我只代韩运使来问,这关城你们让不让?这交趾人的狗,你们是不是要继续做下去?”

厅中又静了下去,所有人的眼睛都盯着黄金满。

黄金满沉吟了好一阵:“何缮,前面你跟着刘永,现在反过来帮着官军。我怎么能相信你?空口白话,总得拿点够资格的凭证吧!?”

见黄金满终于松了口,何缮也终于松了口气,虽然说的话全都是遵照韩冈的吩咐,但在黄金满面前,还是紧张得让背后都被汗水湿透了。

他向侧方跨出一步,将身后一直低着头的随从让了出来,“要凭证,我也有。”

众人一起望过去,那名随从抬起头来。挺起腰背,原本唯唯诺诺的跟班模样一下都没了,读书理民的官宦气度,简陋的外衣也压之不住,“本官苏子元,乃邕州知州之子,现任桂州军事判官。不知这个身份,够不够资格做凭证?”

