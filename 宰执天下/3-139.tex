\section{第37章 相叹投残笔(上)}

到了八月入秋,雨水反常的多了起来。中秋前后的月亮,藏在雨云中,一直就没露过面。

相州的雨,断断续续下了有半个月,原本已经渐渐稀少,可到了今天,突然又是一场暴雨突降。

昼锦堂有着良好的排水系统,只是雨水太大,如同瀑布一般,一时来不及排出去,院子中的积水差不多有半尺深。

之前持续了近一年的旱灾,在秋后淋漓的雨水中,让人逐渐模糊了记忆。

此时渐近深秋,天气已经冷了起来。连日的阴雨天,更是显得湿寒透骨。

窗门紧逼,厚实的门帘、窗帘将缝隙遮得严严实实,一缕香烟从三足香炉,让室内温暖如春。只有高处的一扇透进来一些清新的空气,还有不减停歇的哗哗雨声。

曾经的三朝宰辅,如今判相州事韩琦,就靠坐在床榻上。厚厚的锦被盖着腰腿,一脸的病容,不复当年的神采。一张小几案搭在床上,几上纸页墨迹淋漓,尚未干透的毛笔,很随意的横放在一方纯紫色的端砚上。

韩琦向后仰靠着,闭目养神。身后做靠枕的侍女,又轻轻的帮他揉着太阳穴。如此好一阵,这才重新睁开眼睛。不过写了几百字的奏章,脑中就一阵发木发胀,韩琦即便不想服老,现在也只能叹着岁月不饶人。

拿起刚刚写好的文字,韩琦默默地念了起来:“臣观近年以来,朝廷举事,似不以大敌为恤。彼见形生疑,必谓我有图复燕南意,故引先发制人之说,造为衅端。”

自从去岁第三次回到家乡任职,韩琦的奏章,都是家中的门客或是儿子来写,或是他只负责说,由人代笔,只是最后过目一下,签名画押了事。但是今次事关重大,韩琦并无意交给别人,甚至请人代笔都不行。

过去的几年,大宋朝廷行事,从来没有体恤过辽国的反应。既然见到新君登基后,大宋整军备战、开疆辟土,辽人当然会担心日后宋人北伐。与其等着宋人主动进攻,还不如先发制人。而辽人索取河东之地,就是最好的证据——这一事,就是韩琦打算用亲笔写下的奏章告诉天子的。

正要继续往下看,一个六七岁很是精神的男孩儿从外间跑进来,“爹爹,四哥来了。”

韩琦抬起头,一名二十出头的年轻人跟着进来了,是他的四儿子韩纯彦。

韩纯彦一进来,就对着男孩儿道:“六哥,出去玩去。”

韩琦最小的儿子韩嘉彦,熙宁元年出生,现在才六岁,比韩琦的好几个孙子都要小。听了韩纯彦的话,乖乖的走到外间,立刻就被乳母抱了出去。

见着弟弟出去,韩纯彦走到韩琦榻边,“大人昨日让孩儿查的事,儿子已经查清了。州里出去逃荒的流民,的确回来了不少,这些天陆陆续续有了几百户人家。”

“可问了南下后的情况?”韩琦动了动身子,有些吃力的问道。

韩纯彦道:“孩儿也使人问了。只要到过开封的,都没口子的赞着韩冈。说是逃难一趟还赚了本钱回来。”

“王介甫找的好女婿。”韩琦叹了口气。

韩冈年纪轻轻,做事理政却是朝中难得的人才。今年河北数十万饥民南下京城,才二十出头的韩冈竟然将之全数安置妥当,才干卓异,并不下于富弼当年。

虽然在安置流民的过程中,韩冈也不是全无破绽,韩琦也听说了有好几个知县和御史都有上书弹劾他,但顶不过赵顼对韩冈的信任,上的弹章全都留中不发,甚至将攻击韩冈最激烈的扶沟知县调到了荆湖北路管酒税去了。

想也知道,他们的弹劾成不了事。调去洛阳修堤的一万多流民,才一个月时间,竟然逃回三千多人,哭着喊着要韩提点去管堤防工役。有了这么多流民亲口作证,天子又怎么会相信他人对韩冈的弹劾?

又叹了一口气,韩琦便吩咐道:“四哥,你再去查一下,如果族中有人侵占了流民的土地,让他们都给退回去……若是有人不愿意,就从账上拿钱来买,说是为父买他们的。”

“孩儿知道。”韩纯彦毫不意外父亲的嘱咐,这等毁了家族名声的事,其父韩琦怎么不会让族人去做的,想想又笑道,“大人的吩咐,谅必无人敢不应。”

他又看了看韩琦,脸上已经有了些疲色,便关切的说道,“大人还是多歇着,孩儿先告退了。”

“等等。”韩琦叫住儿子,指了指桌上,“你看看这份奏章。”

韩纯彦听了吩咐,将字纸拿起来,边看边读了起来。

“……所以致疑,其事有七:高丽臣属北方,久绝朝贡,乃因商舶诱之使来,契丹知之,必谓将以图我;一也。强取吐蕃之地以建熙河,契丹闻之,必谓行将及我,二也;遍植榆柳于西山,冀其成长以制蕃骑,三也;创团保甲,四也;诸州筑城凿池,五也;置都作院,颁弓刀新式,大作战车,六也;置河北三十七将,七也。契丹素为敌国,因事起疑,不得不然……”读到这里,韩纯彦难以理解的停了声,皱眉问着韩琦:“大人,真的要如此上书?”

韩琦抬了抬眼皮,慢吞吞的道:“天子问政,做臣子岂有不答之理。”

辽使萧禧从年初受命至东京索要土地,到如今,已经是第三次来大宋。而且此次萧禧南来,还带来一个消息,就是辽主已经准备将女儿嫁给西夏国王秉常。

过去,契丹曾经嫁了一个公主给吐蕃人,如今臣服于大宋的吐蕃赞普董毡就是契丹女婿。现如今,大宋在关西咄咄逼人,北朝嫁一个公主给党项人也并不出奇。

只是这么一来,给大宋天子的压力就大了。西北二虏携起手来,是大宋君臣的噩梦。王安石在旱灾、蝗灾之后,虽然依然坐在宰相的位置上,但已经难以得到赵顼的信重,天子慌乱之下,想起了被他赶出朝中的元老重臣们,亲下手诏,向韩琦问政。而据韩琦所知,富弼、文彦博、曾公亮、张方平等人,也都得到了天子的手诏。

这可以说是元老重臣开始翻身的标志,韩纯彦本以为父亲会以三朝宰相的身份,安定天子之心。可没想到父亲会这般写。说以上七条是造成契丹人生疑的原因,最好的解决办法,就是‘以可疑之形,如将官之类,因而罢去。’,到时候,如果契丹人‘果自败盟,则可一振威武,恢复故疆’。

这是自相矛盾啊!

放弃交通高丽;放弃拓边熙河路;放弃在边境种植用来抵挡胡骑的榆柳;废除河北保甲;边境诸州不再筑城凿池;都作院和军器监打造兵器、战车,以及河北整备军力的行动也尽数停止。

这一番事做下来,到了契丹人南下时,如何能一振威武?

韩琦瞥了头脑混乱的儿子一眼,冷笑道:“想想王介甫是怎么与天子说的?”

对待契丹人的贪欲,王安石始终是主张强硬的对待。对于契丹人意欲重新划定河东地界的要求,王安石说着要寸土不让,并让刘庠、韩缜在谈判中有理有据的拒绝。

如果天子当真同意他的意见,当真放心下来,就根本就不需要向他们这一干被遣出在外的元老重臣问政。

既然天子现在下了诏书,问政元老。可见王安石的话,对天子来说,已经没有了说服力。这个时候,便是良机。

自太宗之后,赵家的皇帝都是这样。可有一个胆子大的吗?

韩琦做了那么长时间的宰相,历经三朝,又曾经亲自见证过仁宗当年与契丹谈判的经历,早看透了赵家子孙是何般模样!越是他这等见惯了皇帝的重臣,就越能看得透受命于天的那些人的本质,绝不会像乡里愚民一般,将皇帝当做神明般崇敬。

韩琦和声再问道:“四哥,依你之见,如果朝廷坚持不允萧禧所求,契丹人可会南侵?”

韩纯彦想了一想,摇头道:“应当不至于此。契丹内乱未已,百姓饥寒待救,而辽主又是荒于政事,成天游猎于荒野间,而朝中更是奸臣当道。虚言恫吓也就罢了,怎么会当真南下侵攻?!”

也就在熙宁五年,辽国北方大族乌古敌烈部起兵叛乱,虽然被剿平,却依然给辽国北疆带来极大的伤害。而去年,辽国又是全国性的饥荒,冬天,又是雪灾,牛羊冻死无数。

这样的情况下,辽人怎么敢南下用兵?其实辽国君臣要得也根本不是土地,而是要增加岁币,以便度过时艰,一如仁宗朝时的那一次增加岁币一般。

可是天子和世人仍将契丹当成了不是生产的蛮族,一旦有灾就到汉地来抢!其实辽国早就变了。韩琦看得明白,只是他可没打算说得那么透。

韩琦笑得深沉,如同当年坐镇朝堂之上,相三帝立二主的时候一般的笑容。既然契丹人不会南下,不利用这个机会,动摇王安石和新法,又更待何时?

