\section{第35章 历历新事皆旧史(下)}

“烧吧!烧吧!”

一个中年人在火场前喃喃自语。

他佝偻着背,熊熊的烈火照亮了他的面容,老实巴交的脸孔上有着与相貌完全不相称的狰狞。

他的右手齐腕而断,包扎手腕的纱布早被各种污渍染得看不见原本的颜色。尽管在医院中包扎得很好,但不去换药加上不注意卫生,已经让残余下来的半条手臂都开始发黑变色。

汹涌的热浪已经烤弯蓬乱的须发,从厂房入口舔出的火舌也几乎探到了他的脚边,但他仍没有挪动脚步,瞪大眼睛的死死盯着眼前吞噬掉他一切希望的工厂。

从烫伤到溃烂,从溃烂到截肢,从截取右手到被医师告知需要再截去整条手臂,只用了两个月。

好端端的活到三十五,只用了两个月就成了废人,这活下去还有什么意义?

一起烧吧,把一切全都烧个精光!

……………………

“烧啊!烧啊!”

年轻人左手拿着火煤,右手护着刚刚生起的小小火苗。

胸中的火焰早已熊熊,手上的火焰却细小如豆,他急得满头大汗,却连大气也不敢喘。

身后的大门半掩,在外面的同伴,已经快要抵挡不住那些护卫厂中的‘恶犬’,拖不了多久了。

焦急中,他回首门外,晃动的人影让他心中仿佛有恶兽在吼叫,而远处的火光则仿佛是对他的催促。

回头一见火苗终于稳定下来,他便立刻向前一丢。燃着的火媒划着抛物线落到了泼满油的丝绸上,黯淡的仓库之中陡然一亮,火势轰然而起,瞬息间扩散开来,攀上了仓库中一叠叠已经被扯得凌乱不堪的绸缎。

他被火势逼退了几步,火光变幻,映着表情也在不住变化。

仅仅两年,失去了桑园,失去了家业,原本殷实的家庭,现在只能依靠短工来维持生计。

想起自尽的老父,想起瘦骨嶙峋的母亲、妹妹,想起自己业已无缘的姻缘,他心中的火仿佛又开始燃烧,恶兽似乎又在吼叫,催促着他狠狠的抓起一匹又一匹丝绢,投向飞蹿上屋顶的烈火中。

烧啊,一切全都烧个精光!

……………………

“都烧光!全都烧光!”

一处又一处火头升起,白衣男子拿着千里镜,在楼阁上眺望着。

这是上苍在洗清一切不净。带来光明的火焰,会洗清那些工厂中的污秽和怨气,

几场大火,不仅可以回报明使,转天也能吸引更多的信众。

无灾劫,便无善信。

饥寒交迫,方会受到教义吸引。大灾大劫,才能让愚民敬畏主的威严。焚城之火,才会有满城的信众。

有此一火,这润州城中,光明的信众又将多上几分。

烧吧,把一切都烧个精光!

……………………

“烧得好!烧得好!”

火光映红了润州城半边天空,一个身着青袍的官员捋须大笑。

朝堂上的宰相苦心积虑来推行工厂,这一把火就像巴掌一样,打到了他的脸上。

一直以来,那些宰相所推行的重重变革,都没有大的挫折,现在终于出现了一个。

丝厂是他推动创办的,工厂大兴更是他所鼓励的。

士夫沸腾,百姓皆怨,还可推说子虚乌有,但此番火起,便再无法视而不见。

这场火,当可烧到庙堂之上!

烧吧,把一切都烧个精光!

……………………

一封急件在润州州城中匆匆写就。

由一名急脚递士兵骑着快马,送出了润州城。

京口上船,扬州下船,继而上马,越过还没修好的铁路工地,抵达泗州,乘上京泗铁路的快车。

四天后,来自润州的急报送抵通进银台司,一个时辰之后,便送抵韩冈等宰辅的案头。

死亡人数总计一百五十七人,失踪两百余,烧伤上千人。

两个数字触目惊心,尤其是死亡人数,几乎让人心底发冷的数字。

太平时节,又无天灾,突然间死了一百五十余人,又失踪两百多——这其中至少有一半已经葬身火海尸骨无存——而且不是意外,而是有人故意纵火。这桩案子,足以震动整个朝堂。

政事堂几位宰辅共聚一堂,

一开始被纵火的是润州的几处丝厂,原本目标只是厂房和仓库,但其中有一处丝厂的厂房靠近民居,火起之后,风助火势,将两个坊化为灰烬,顺便还将润州织罗务的仓库给烧了。

最后的结果,是两座丝厂尽毁,一座严重毁损,只有一座丝厂被守住了。这些丝厂的损失不计,只是织罗务库之中,就损失了三万余匹新成贡罗。

“织罗务的事暂且不论。”章惇右手向旁边摆了一下,做了个‘放在一边’的手势,他心情不好的时候,手上的动作往往就会比较多,“之后再细查。”

究竟是火势蔓延开来被连累到,还是有人想乘机来个死无对证,冲抵账上黑洞,现在谁都说不清楚。

“关键是为什么有人会烧丝厂。”他敲了敲扶手,继续说道,“此前十天,杭州盐官县丝厂被烧,之后两天,秀州处也有一家丝厂被烧,到了四天前,就是润州,同时四家丝厂被烧。这两天,说不定又有哪家丝厂被人放火烧毁。”

众宰辅先后点头、

章惇的猜测不是没有道理。已经有六家丝厂被人纵火了,谁人能肯定被烧毁的就只有这六家?从频率和速度来计算,润州急报在路上的这四日,多半还会有几家丝厂受到攻击,如果还没有警惕起来,赴前几位同行的后尘,也不是不可能。

章惇环目一扫,观察着在场的几位同僚,想要分析出有哪个人对他的话有着可疑的反应:“或许有人会说这是天怒人怨,丝厂夺民口食,故而横遭此报。但数日之间,三州丝厂先后遭劫,又岂是报应巧合能够解释的?其中必然有人为主谋,唆使民变。”

“子厚相公说得是,肯定不是那么简单的一件事。两浙山区和平原的民风截然不同,山中彪悍,山下软懦。若是婺、睦二州民乱,那是一点不出奇。山中村庄,为争水争地,年年都要打上几场。但苏杭润常湖这几州民乱,却是让人始料未及,必是有人在后主使。”曾孝宽道,“当寻究其主使之人,绝不容许其逍遥法外。”

“相公打算如何处置?”邓润甫问章惇道。

“命两浙路提点刑狱彻查此案,灾民令润州赈济安抚,若愿意屯垦边疆,酌情给付旅费。”

“丹徒知县当罢。”曾孝宽沉声道。

章惇道:“应该已经请辞了。”

通天大案,不论是否有牵连,当地的知县都要担上一份责任。若不知情识趣的上辞表请辞,就等着被弹劾吧。

再怎么样,也的把悔罪的态度表现出来,这样背后的靠山才能名正言顺的拉上一把,否则一个不知羞耻的评语加上来,就会变成臭狗屎一般,让人闻风而避了。

“希望他知趣。”邓润甫哼了一声,对章惇道,“当尽速另选贤能。”

“自然。”

参知政事先后表了态,章惇问韩冈:“玉昆,你看如何?”

“我亦觉得子厚兄的决定甚好。不过,可再选个人去一趟两浙,此事非小,当防微杜渐。光靠提点刑狱司和当地州县的奏疏,总是隔了一层。”

工厂是韩冈大力推动,现在出了事,他派人去两浙查个究竟也好,掩盖事实真相也好,都是情理中事。曾孝宽、邓润甫都没有异议。

章惇想了一下,道,“让宗状元去如何?”他问着韩冈,“他是浙人吧?”

“是,就让他去。”韩冈点头同意,这件事让宗泽去他才放心。

短暂的会议之后,章惇与韩冈留了下来。

“玉昆,你是不是有什么看法?”章惇直率的问韩冈。

韩冈点了点头,“之前子厚兄你和曾令绰都说,这件事别有蹊跷,并不简单。”

“玉昆你觉得不是这样?”

“其实我觉得这个问题很简单,”韩冈道,“归根到底,还是江南的工厂主太黑心了一点。”

章惇眉头微皱,道,“何以见得?”

韩冈道:“想必子厚兄你也知道,关西所创办的棉纺织厂数量比丝厂还多不少,棉花也与丝绢同样依然,雇佣的工人甚至是倍于江南丝厂,为什么关西就从来没有过工人烧厂的事?”

章惇道:“那自是因为无人唆使。”

韩冈反驳道:“若心中无怨,又有几人会因唆使而犯下如此重罪?”

关键就在这个唆使上。不是工人冲击丝厂,厂子也不会给烧掉。大部分工厂的防护都很紧密——丝绢本来就是另一种模样的货币——三两个人想要纵火,保准会被打出来,只有上百人的骚乱,才能得到纵火的空隙。

“在关西,棉纺工人想要作乱,回家提了弓刀出来就能干了。关西人哪家没几把兵器,两三张弓?可就是没人作乱。相反地,有不少贼子偷入厂中,被厂里的工人群起擒获,械送官府的例子。子厚兄,人心向背啊。”

韩冈语重心长的说着,章惇一时默然。

只追求利润,从来不在乎人命。黑心,贪婪,视人命如草芥,这是如今江南开办丝厂的诸多工厂主的标准写照。

但这些人虽说黑心,可如果是在同等技术条件下进行公平竞争,韩冈不觉得雍秦商会有获胜的可能。

江南的水力资源远胜于西北这一条,只是很小的因素,而且很快就会在蒸汽机上给拉平。真正的能让江南工厂主大获全胜的最重要的一条原因,是双方工人的待遇。

雍秦商会的棉纺工人,隔三差五就能吃酒吃肉,要不是棉布缺乏竞争对手,能卖上高价,谁会给他们那么好的待遇?这可都是成本。

但大宋的丝绢太多了,工业化的丝绸成本虽低于民户所产,而且质量稳定,但无一例外,都买不了高价。蜀锦等贵价锦缎,只有手中制作,现在的机械还做不出那个等级的丝绢。

开办丝厂的工厂主,即使想要把自家产品卖出高价,也不能超过民户的产品,否则就没人买了。而要压倒其他工厂的产品,除了压低成本之外,更是没有其他办法。

以资本天生的逐利性,压榨工人就成了必然。

‘这发展,真是让人眼熟啊。’韩冈苦笑着。
