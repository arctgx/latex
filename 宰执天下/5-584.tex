\section{第46章 八方按剑隐风雷(四)}

随着推着挡箭车的步卒靠近了城寨,黑汗骑兵也出现了。

基本上都是没有穿着甲胄的轻骑兵。出寨后绕行而北,分兵出去,佯攻东、北两面的营垒,观其意图,当是为了牵制两处的守军。

另外还有两千多穿着甲胄的骑兵,则是摆出了压阵的姿态,正对着营垒正门。

营垒四面正门,都没有挖掘壕沟,当然也不会有吊桥,门前的道路平坦而宽阔。若是营垒大门被攻陷后,就能长驱直入。

防线上的漏洞,王舜臣并没有设法补全,因为出击时会碍事。至于会不会成为敌军主攻的目标,王舜臣并没有放在心上。要是没有这份信心,他根本就不会带兵一路西行。

敌军的布置,王舜臣一眼看过去,便觉得不对劲。

古拉姆和伊克塔的数量与实际对不上,显然有许多精锐脱了甲胄,混在杂兵中攻了上来。

步卒中,一律都只顶着头盔,手持盾牌和弯刀,根本分不清那些人中,哪个是精锐,哪个是杂兵。只有站在每辆挡箭车上的几名士兵,他们无一例外都穿了甲胄,也无一例外是高大雄壮。

王舜臣拧着眉头冲西面看了一眼。绕过南面大营的黑汗军,已经接近了西面营垒,如果他们不是为了更向北去攻打北寨而绕行,那么就肯定是以西营为目标。

从数量上,以及行动的速度上,将要攻打西营的黑汗军中,有许多精锐无疑。

这必然是自家军中的内情,在这几天中,已经为敌方所窥破。知道除了南营这边之外,其他三面都是战斗力不强的吐蕃人和回鹘人。

不过四面营地背后都是城门,随时可以借用城中的内部通道来换位防守。要想攻下其中任何一面营垒,就必须借用人数上的优势,牵制其他三个方向。

而现在黑汗军的部署,也的确是以一部分主力牵制南面的汉军,以少数骑兵牵制东营、北营。剩下的主力便全力攻破西营。

这样的布局少有眼力就能看得出来,可黑汗军兵力上的优势,让他们并不担心宋军能有应对的布置。

霹雳砲的投射速度这时候越来越快,不论能不能击中,至少呼啸而行的砲石能够迟滞敌军。石子,冰块,此时都成了投射的弹药,威力并不比石头差多少,准确姓虽跌落,但现在也只求能够提高发射速度。

可惜大部分的射击还是都落了空,挡箭车虽然结构简单,眼下却是最适用的攻城利器。

在之前的攻击中,黑汗人已经确定了陷马坑的范围。那些陷坑能陷下马脚,却没办法奈何得了长长的雪橇板,轻易的就越了过去。而宽窄两条壕沟,对步兵推动的雪橇车,也如同坦途一般。换成是四轮的攻城车,早就在坑中和沟里进退不得了。

不过雪橇车走得顺利,可跟在后面的士兵,有人陷在坑里,有人跌在沟中。他们的动作比起战马灵活太多,没有几个受伤,只是速度都不免慢了下来。营地中的弓弩手们,没有放过黑汗战士与前方车辆脱节的机会。失去了前方防护的他们,立刻就成了箭矢集火的对象,瞬间就成了刺猬倒了下去。

只是这些伤亡,完全没有惊动到任何人。南营正面的一群黑汗士兵,已经忘掉了一切,正在发足狂奔,以最后冲刺,用着最快的速度,直取寨墙之下

而西营方向上的黑汗军,并没有瞄准营垒正门,刚刚绕过南营,却也直接冲向最近一段的寨墙。

迎来敌军攻击的西营,李全忠和他的部众,正听着王舜臣麾下裨将的指派,匆匆奔向面对攻击的地方。

虽然他们没有神臂弓,却也还有正常的弓箭。除了汉军,吐蕃人和回鹘人都是用着各自的弓箭。

随着挡箭车越来越近,一排弓箭手出现在寨墙上,对准挡箭车的木板,射出了一支支火箭。

至少一半火箭扎进了木板上,上面缠绕的油布团,依然在燃烧着。

李全忠紧张得盯着那一团团火焰,他希望能够点燃这些挡箭车,击退这一次攻击。

末蛮城附近有树林没错,但也不是能够供给数万大军打造器械、修建营地并提供长期取暖。若是烧光了,要往更远处寻找木材的来源,要运过来,必须要时间。那时候,粮草多半会不够使用。

可是对于挡箭车,火箭一时间并不会产生太大的影响。那油脂燃烧后的刺鼻味道,反而让后面推车的黑汗士兵前进得更快。他们选择堂堂正正的正攻法,冒着随时会被箭矢射穿的风险来攻打敌营,就是要用最快的速度攻到寨墙之下。

不论南面还是西面,面对厚重的木板,他们能够使用的手段乏善可陈。砲石砸不到,而重弩又射不穿,点火又一时烧不掉,是在是让人束手无策。

一辆辆结构简单的雪橇车,就这么依靠长达五六尺的滑雪板,很顺利的越过了陷坑和壕沟,直抵寨墙之外。

这段时间中,王舜臣和他麾下部将,下达射击命令的次数很少。慌慌张张的浪射一通,浪费箭矢不说,还会乱了阵脚。挡箭车就算能挡着箭矢,但总不能推着车子上了寨墙,最终还是要放开车子,直面营垒。

寨墙并不是一条直线。大宋任何寨堡和营垒修筑时,其外墙最重要的一条标准就是直不如曲。如同折线一样的外墙,可以让射手们聚在凸出部。集合起来的射击,轻易的就能以箭雨,从两侧清洗攻到寨墙下的敌军。

王舜臣张弓搭箭,和一排拿着神臂弓的神射手站在了营栅后的雪墙上。

他没有煽动将士的口才,也不会用吮痈疽、共起居的手段拉拢军心,王舜臣唯一会做的,就是站在阵前,绝不后人。至于指挥,他有足够可靠地部将,而且也有自信,随时能够从前线抽身出来。

南面六十多辆挡箭车,有四分之一卡在了最后一道深壕中,但剩下的挡箭车,都顺利的越了过去,然后重重的撞在了寨墙上。

咚、咚、咚的闷响,接二连三的在汉军大营的寨墙外响起。黑汗军这是第一次攻到了如此贴近营垒的地方。

寨墙只有五尺高,但穿着盔甲,正常人就别想跳过去。可是挡箭车的正面木板之后,却钉着几级阶梯,即是挡箭板后的支撑,也是跳上寨墙的踏板。

胜利的曙光已在眼前,一直站在车上的黑汗甲士踏着阶梯一冲而上,高声念着真主之名,将弯刀挥向面前的敌人。

王舜臣就在几步开外,抬手拨弦,箭如流光,飞射而出。一名刚刚从挡箭板跳上来的黑汗战士正一声怒吼,利箭便从眼目之处贯脑而入。紧跟着上来第二名战士,正要冲向王舜臣,第二支箭便紧随而来,将他射杀当场。

弓弦声急急如雨,这一辆车上,其余甲士还没有来得及踏步阶梯,便一个个要害中箭,纵逃过一死,也没再战之力。王舜臣的十几名亲兵,没有来得及出手一次,就让王舜臣抢去了所有的功劳。

可王舜臣只有一人,他的连珠神箭虽强,却也仅仅射杀了最近处的几名敌军。

但南营的守卫并不需要他操心。站立于寨墙凸起部的射手们,正不断接过后方送上来的、已经拉开的神臂弓,上弦后瞄准目标,然后扣下牙发。

近到眼皮下的射击,再坚实的甲胄也挡不住神臂弓的力道。当年神臂弓初成,便能七十步外洞穿铁札。如今神臂弓几经改进,一二十步之内,铁甲无不洞穿。

一名身高近七尺的黑汗勇士一步跨上寨墙,带起一阵旋风。可立刻前后便有十几张弩弓一齐瞄准了他。还没有来得及其他动作,箭矢便笼罩了他全身。转眼之间,他身上的铁甲便尽是孔洞,只有几处能看见短短的木羽翎尾留在外侧。

巨大的身躯僵立了片刻,重重的向营寨外栽倒下去。跟随着他的黑汗士兵,在这一瞬间,动作都凝滞了。显然是主心骨一般的勇士如此轻易的就被射杀,他们的战意一落千丈。

不过这些甲士,还是为后面的战士争取了时间,越来越多的黑汗士兵冲到了寨外。有的抛出短斧,有的张弓施射,被干扰到了的神臂弓手们,他们射击速度一下就降了下来。有了空隙,冲上城头的黑汗军士兵也顿时多了。

“出击!”

王舜臣这时候已经先退了下来,在寨门旁发号施令。

寨门大开,两侧寨墙突出部的神臂弓手急速射击,刚刚清洗了城门外侧的敌军,两百身穿重甲的汉军战士,随即举着斩马刀从门中鱼贯而出。

几名黑汗甲士领着一群战士猛冲而来,他们的勇气绝不输给任何人,掌中的弯刀也有切金断玉的锋利。

但斩马刀交加而下,即使身着盔甲,也没能挡住这集中全身之力的斩击。如同纸张一般,被一斩而断,甲胄后的人体也同样被长刃破开。血液从甲胄的裂缝中喷溅出来,随即一只只脚边从他身上跨了过去。

出门之后,两百战士立刻左右分开,沿着寨墙一路砍杀过去。斩马刀连环砍出,黑汗人的弯刀远远奈何不得三尺长刃。任何挡在斩马刀前的敌人,无不是被砍得支离破碎。

配合着神臂弓的射击,瞬息间寨外一片血光,方才还向着城头猛冲的敌军如同兔子一般被追得满地乱窜。而没了后援,攻上寨墙的黑汗战士转瞬间被杀了个干干净净。这段时间里,伊克塔重骑兵曾试图冲击敞开的寨门,却被对准道路连环发射的霹雳砲给吓阻住了。

丢下了装满油的水囊,将挡箭车一辆辆烧掉,出击的战士缓缓而退,一一返回寨中。

而这时候,王舜臣亲自领着援军,赶向西侧的营垒。

