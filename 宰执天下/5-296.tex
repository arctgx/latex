\section{第31章 停云静听曲中意(三)}

韩冈默不作声,几名宰辅同样默不作声,乃至对河北忧心忡忡的韩绛,甚至连感情上与赵顼最为亲近的王安石都不开口,皆是低眉垂眼,静待着天子的回答。

寝殿内的气氛忽然间莫名的变得诡异和紧张起来,向皇后纳闷的抬起头,看看左右,却有些弄不清楚情况。

不管是为了什么原因,赵顼在病榻上将两府中人随心进退是不可否认的事实。对此,朝堂上不是没有怨心,甚至反感。纵然是因此而得利,但上台之后的韩绛、曾布、张璪等人,其实都不想赵顼再这么折腾,否则步前任后尘就是他们了。

两个月来的朝堂人事频频更迭,对政局乃至地方政务产生了很大的干扰,另一方面,由于宋辽开战在即,宰辅的人选也不宜再轻易更动。

已经渐次磨合的两府都不希望天子在此时再掀起任何风波,那只会让刚刚被打压下去的旧党得利。同时让政局败坏下去。互相之间裂痕极深的宰辅们,这时候却一下子亲密无间的合作起来。

虽说不是硬抗,仅仅是消极且被动的抵.制……甚至连抵.制都算不上,只是不那么主动而已。但这已经将宰辅们的心意表现出来了。

静默中,宰辅们心中都有些不安。赵顼多年来的积威依然存在,万一他强来的话,没几个人敢站出来硬顶。

韩冈的呼吸也变得细了。若赵顼继续折腾,是人都会怀疑他的头脑是否清醒,宰辅们联手起来,轻而易举就能将他给架空掉。但由此而来的问题,却会给朝政带来难以预测的变化。而且在天子的威凌下联手,也不是那么容易。

似乎过了很长时间,又好像只是片刻,赵顼终于抬起手指,在沙盘上划了两个字:平章。

王安石应声道:“臣在。”

稍留。

既然是要王安石稍留,其他人当然就不需要留下来了。

这并不是宰执们想要看到的结果,只是比起赵顼的脾气上来,倒也不算最坏。

“臣等告退。”韩绛领着除王安石以外的其他臣子向赵顼行礼,而后鱼贯而出。

廊道中只有脚步声,从寝殿中出来的宰臣们依旧沉默着。或阴沉,或冷淡,从他们各自的表情上,完全看不出来有人为赵顼病情好转而感到欣喜。

赵顼只留下了王安石,这是要拉拢他吗?答案一目了然。没有人会看不出来。

以情分来说,王安石与天子是最深的。病榻上的天子动之以情,让王安石效死也不是难事。而平章军国重事如果得到了天子支持,可是能将两府都攥在手中。

赵顼这是想要让两府同仇敌忾?应该是有自信能压得住阵脚吧。韩冈想着。

蔡确慢了两步,落到韩冈近前,看着前面道:“天子的病势终于有了起色,可谓是国之大幸。当真是天佑……玉昆你说呢?”

韩冈毫不犹豫的点了点头:“相公说的是,的的乃是天佑。”

一瞬间的寂静,廊道中仿佛连脚步声都停了,不过立刻就恢复了正常,只是之前凝重的气氛似乎消散了许多。

章敦走到韩绛的身侧:“有郭逵在,河北必无大碍。”

蔡确的声音提高了一点点:“军器监要加快打造上弦机,越快越好,早一天发到军中,就多一分安稳。在军器上的缺额,也要尽快补足。”

韩冈道:“上弦机利于守城,不利野战。若是辽人南侵,野战也不可避免。上弦机省力的原理已经明了,再发明一件适于野战的上弦器并不算难。”

“玉昆可是成竹在胸?”薛向问道。

韩冈摇摇头:“韩冈只明其理,不知其用。得让专业的匠师去。若能以以爵禄悬赏之,不日当有所获。”

韩冈其实见曾经过一根绳子上带两个钩子的简易上弦器,给弩弓上弦时能省一半的力气。结构简单得只需捅破一张窗户纸。只要能想到,就能造得出来。

韩绛为家乡时本就是忧急于心,韩冈一提,他就一把抓住,“悬赏之事,政事堂接下了。只要能造出上弦机,小使臣不用说,就是一个大使臣都没问题。不过监中的制造,还得枢密院多费心。”

东府诸公各有分工,各自都有一滩事要管,普通一点的小事务直接在各自分管的范围内给解决了,只是有首相韩绛掌总。军器监归属东府,只是因为生产的是军器,军器监内部的官吏,基本上是枢密院这边影响力更大一点。

韩绛郑重其事的叮嘱着,章敦当然不能说不,点头道:“这是自然,还请相公放心。”

“河北当是不用担心了。可陕西之事,吕吉甫想要撑下来当不是那么容易。”曾布回头问韩冈道,“不知玉昆怎么看?”

韩冈跟曾布没什么交情,因为王安石的缘故更不好去攀交情,不过参知政事的曾布搭话,韩冈也不能拒之门外。

“该如何去做,前面韩冈已经报予天子。自是该做决断时就得做决断,首鼠两端并不可行。”韩冈停了一下,“吕吉甫若是能放下私心,这个决断,他是肯定能下的。眼下他面临的的确是两难境地,不论换成谁来做,想要面面俱到也一样不可能。不过只要吕吉甫愿意退上一步,立刻就是海阔天空。而且他之前可是重新启用了曲珍,他甚至可以做得更好。”

“这个决断可不好下。”曾布摇摇头,“种谔更不是省心的人,还是早点将他调离陕西为是。”

韩冈笑了起来:“若是溥乐城顺利解围,种谔当不能再留于缘边。但若是他直攻兴灵,而辽人起兵反扑,那就难办了。朝廷丢不起向辽人委曲求全的人啊!”

韩冈说得在理,但不好回答,给个武将拿捏住,哪位文臣不恼火?曾布顾左右而言他:“青铜峡的党项人呢?”

到现在为止,朝廷收到的消息,也只是在说青铜峡的党项余孽在仁多零丁和叶孛麻的率领下有异动。对宰辅们来说,这群西夏的孑遗很危险。

“泾原路有熊本主持,鸣沙城有赵隆抵挡,不需要担心。”章敦很轻快的说着,“还是要说说河北,要尽快设立四路行营了。”

“暂时还没必要吧?”张璪也投入了讨论中。行营的作用不是备战,而是作战,而且只会是为了应对大战才设立,旧年攻打交趾时就曾设立了行营,一旦设立行营等于就是在明说要开战了,“可以先做准备,至于四路行营,等得到辽人集结的消息再动手也不迟。”

“如今跟过去不一样。旧年辽人南下,会现在鸳鸯泺合兵。等兵马到齐后才会出野狐岭,经奉圣州【张家口、涿鹿一带】、南京道南下。但如今耶律乙辛冬天就驻扎在析津府【今北京】,若其意欲南侵,两三日内就能杀到边境了。”

章敦的话,让张璪有了些动摇。

“还是先征询一下郭仲通的意见比较好。”韩冈说道。

韩冈的提议不为文臣们所喜,但好歹有用。韩绛点了点头,“……说的也是。”

要不要成立河北四路行营,必然要征询判大名府、河北安抚使郭逵的意见。

陕西旧时有缘边四路,鄜延路、环庆路、泾原路、秦凤路,加上核心的永兴军路,可以说是关西地区的五大战区——此五路同归陕西安抚使管辖,只是这个职位已经形同废置,毕竟权力范围太大了,陕西又是常年用兵,让皇帝无法放心——后来多了一个熙河路,眼下则又有银夏路、甘凉路。与这八个经略使路相对应的,则是秦凤路和永兴军路两个转运使路。

而河北情况与之相似,同样有个总摄兵权的安抚使,以及分立的安抚使路和转运使路。漕司转运使路,有河北东路、河北西路。而主管军事的帅司,则是定州路、高阳关路、真定府路、大名府路,另外还有一个雄州知州兼任的河北沿边安抚使。只是有一点与陕西不同,那就是河北的帅司不带经略二字,单纯的安抚使。乃是澶渊之盟后,河北边境无战事,不需要经略军事,只需要安抚就足够了。

转任大名府的郭逵,现在是以签书枢密院事兼河北安抚使,统摄河北军权,四路帅司尽归其辖下。如今若是要成立四路行营,必以其为都总管。

“玉昆,令表兄现如今就在广信遂城,真要开战,可是首当其冲啊!”薛向小声的跟韩冈说话。

李信就在定州,确切的说是在定州路下的广信军。以广信军知军兼定州路钤辖的身份坐镇遂城。这其实就是旧年杨六郎杨延昭的职位。铜梁门、铁遂城,是河北边境上最重要的战略据点之一。一旦辽人南侵,遂城守军若不能阻敌于边境,那么剩下的任务就是反攻入辽境了。

韩冈冷然道:“既然受了重用,就得为国效死。没什么好多想的!”

章敦则道:“李信悍勇敢战,在河北亦有声威,他镇守遂城,日后说不定又是一个杨六郎。”

“玉昆。”韩绛听到了韩冈、薛向和章敦的对话,提声问道:“令表兄就是曾经在笼竿城下七矛杀七将的那位?”

“正是。”韩冈点点头。

章敦又补充道:“在荆南时其曾为先锋,只携一小校背矛出阵,日不移影连杀十余蛮将,之后更是五日破八寨。不过李信最难得的是治军严,肯听命。没有桀骜之气,非是那等骄悍不驯之辈。”

“果然难得。”韩绛闻言便点头赞许。蔡确、张璪、甚至曾布也跟着一并赞叹起来。

韩冈为表兄谦逊了几句,与章敦交换了一个眼神。暂时可以不用担心了,他们的称赞可不只是称赞。
