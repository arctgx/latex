\section{第39章 遥观方城青霞举(十)}

【第三更中午时出来,耽误了一点时间,实在不好意思。】

“也就是肯定不打算去河北?”王旖进一步追问着。

“这还用说。”韩冈听出了王旖的口气有些不对,搂得紧了些:“怎么,不喜欢河北?”

“不是。”王旖向后靠在韩冈的怀里,思忖着说道,“只是觉得这样做不太好。父亲上万言书,就是为了亲自主持变法,王枢密上《平戎策》,是为了亲自收复河湟;自己提出来的提议,不去亲手完成,总会免不了有人说闲话。这么重要的事,难道还能让他人去主持?现在官人说要在河北铺设轨道,却不自请去河北,天子也会怀疑官人的用心。”

“没关系。”韩冈满不在乎的说着,“奏表中该怎么说为夫难道还会弄错不成?但天子不让为夫去,那就没办法了。”

王旖幽幽一叹:“肯定是要回京城了?”

韩冈也跟着叹了口气:“其实为夫也是想在外面多轻松几年,只要有心,在哪里不是做事?还不用跟人勾心斗角,省了心思,当能多活几年。但我这边再不快一点,关西就没人了。先生的一番心血,做弟子的哪里能看着付之东流?还有为夫的毕生所学,也不甘心让人踩在脚下……”顿了一顿,“不管有什么人挡着,京城,我是肯定要回去的。”

韩冈的计划当然瞒不了枕边人,王旖早就知道丈夫的想法。为了推广气学,跟自己父亲都翻了脸。因为学术之争,丈夫所表现出来的倔脾气,并不比被称为拗相公的父亲稍差。

丈夫费尽了心力,才将横渠先生送入京中。依靠张横渠的讲学,好不容易气学的规模有所发展,但转眼就因为张载的去世,陷入了群龙无主的境地。如果有足够的时间,想必气学当能重新产生一名新的宗主。

在看到吕大临的行状之前,王旖知道,韩冈并没有打算去争夺这个位置。只是想着在自己的一亩三分地上继续深挖而已。但吕大临所撰写的横渠先生行状一出,韩冈就明白了有人迫不及待要毁掉气学的道统了。现在程颐就在关西,一次次的讲学正将气学斩草除根。

再好的情分,也比不上自身所学被人毁灭的愤恨。韩冈能为气学顶撞她的父亲,当然也可以为了气学而跟二程为敌。以韩冈的性格、为人,加上对气学的坚持,是不可能容忍出现张载过世后,出现气学衰微的情况。

写书,出书,用人,施政,都是为了维持气学的地位不衰,说到尊师重道,自己的丈夫的确是当世少有人能及。

“那李德新也该回来了吧,都耽搁了好些日子了。”王旖有点犹豫,“是不是出了什么事?”

“这倒没有,只是为夫想要多确认一下,所以让他在伏龙山多留些日子。为夫都派了十几个人去护卫他,天天都有消息传回来。”

“他回来后,官人是不是打算将防治蛊胀病的差事也一并交给他?”王旖问道,“今天黄夫人来了还在问,漕司让州中划定疫区,是不是为了给他们治病。还说黄知州向来勤谨,若官人还有什么吩咐,可以尽管说,肯定不敢推搪延误。”

“蛊胀病一时还找不到合适的方子,只能预防。为夫暂时也不打算让李德新分心,蛊胀病的事,自会安排其他人来做。”韩冈呵呵笑了一声,“黄庸倒是有心人,看到漕运有成,以为为夫准备下一步就要处理蛊胀病的事了,想先讨个好,日后也能分功。”

王旖还要说话。

“好了,好了。难道夫妻间只有这些话好说不成?别多担心了,正经事上为夫什么时候糊涂过?”韩冈笑了,拧过身,抓着手腕,将王旖压在床上,“前两天爹娘让人从家里带来的信上是怎么说的?”

王旖身子忽然一僵,抽开手,给韩冈一个脊背,“那你就去找南娘、素心和云娘去,闹奴家作甚?”

韩冈哭笑不得。

他这一房已经有五个儿子了,不用担心绝嗣,而长房、二房,则都没有子弟承宗祧,远在陇右的父母就希望周南、素心或是云娘能再生两个儿子,过继给两位早逝的兄长,让他们日后还能有个香火享用。而王旖生的儿子就不太方便过继,毕竟是嫡子。

“难道爹娘的信上就说了这件事?”韩冈扳着王旖的肩膀,俯身过去,低声说着,“再生个女儿吧,家里都是儿子也闹得烦心。”

……………………

的的的马蹄声中,冯从义望着远远近近,在碧绿中有着红黄杂色的山峦。

迎面而来的风,已经没有前些天的温暖,多了几分萧瑟,当真已经是入秋了。再过一阵子,到了十月,就要该下雪了,冯从义想着,等到这条路上被积雪覆盖,依靠雪橇车来实现的冬天的贸易线,就会立刻开启。

不过眼下还得骑在马上。

虽然还年轻,但冯从义在江湖上奔波了几近十年,基本上就不想跑得太远了。大部分的时候,都还是打算留在巩州,与妻妾和孩子在一起。

顺丰行在外面的生意,比如京城、襄州还有交州等几个重要的地点,都派了可以信用的人去查账,同时也把给两位表哥的礼物带过去了,有合适的手下在,凡事不必亲历亲为。

本来冯从义今年是准备在家里歇上一阵的,明年开春后再出去走动。不过秦州的几大商号都派了人来请,也不得不去秦州走上一遭。

从转运司这边来算,泾原、秦凤和熙河三个经略安抚使路,都算是秦凤转运司辖下。与顺丰行亲近甚至缔结了进退同盟的十三家大商行,也都是出身于秦凤诸州,没有一个例外。

韩冈在京西主持开辟襄汉漕运,只要能成事,襄州日后当是沟通南北方的枢纽,如果能在那里站住脚跟,他们秦凤路各大商行,也就是如今自号雍商或是雍州十三行的一群大商号,就能在南北转运的生意中,分润到很大的一部分利润。

冯从义去秦州,是为了此事。雍商在官场上有着不少关系,但这些关系都需要大量的利益去维持,不像韩冈,能反过来给各大商行输送利益。而且以地位来说,韩冈也是身份最高、且前途最为广大的一个——高遵裕和王韶,他们可以说是助力,但不能算是靠山——也就是有了韩冈的撑腰,成立不过数年的顺丰行在十三行中的地位,已经是排在最前面的了。

从发展潜力来说,能比得上顺丰行的也不多。以棉布织造为纽带,在京城站稳了脚跟,交州的白糖作坊,产量也日渐扩大,可以预期,必将成为顺丰行的另外一项支柱产业。而有韩冈作为靠山,日后的发展也同样是可以期待。

“东家,还有二十里就到秦州城,要不要先下来歇一歇,整整装束。”紧紧跟随在冯从义身边的一个护卫,上前提议道。

冯从义看看左右,再看看自己的身上,点点头,“先歇一歇好了。”

各家派出的迎宾,肯定就在前面守着。自家这边一个个都是风尘仆仆的样子,不整理一下,到了秦州城也有失体面。

马队停了下来,跑前跑后安顿人马的一众护卫,领头的是当年跟随韩冈在河湟上阵厮杀的刘源家的大儿子。由于身份的问题,他不方便跟着韩冈,眼下是给冯从义作着护卫。

在顺丰行中,有许多广锐军的子弟。大约占了三分之一,都是可以信赖的人选。

他们全家的性命可以说都是韩冈保下来的,如今在熙河路平静安稳,并且还算得上富足的生活,也是靠了韩冈和他的父亲韩千六才得来的。也因此韩家父子在旧日的广锐叛军中,有着很高的声望,几乎可以说是一呼百应。顺丰行要找人手,自然是得从他们中挑选。

他们过去叛乱的罪名,都在开拓河湟的过程中,清洗得一干二净,也不用担心人说闲话。而且他们全家都在熙河路,不能向内地迁移。有韩家的影响力在,不用担心他们会有何异心。

在路边的小店歇了脚,跟在身后的小厮端了一盆水出来给冯从义洗脸。

到了秦州城就要跟人扯皮了,襄州的利益肯定要让出一块来。雍秦的商人实力远远比不上京城或是江南,不抱成团就只有任人鱼肉的份。为了凝聚人心,就得摈弃一部分私心,钱是赚不完的,拥有更大的实力,能赚的钱就越多,不能因小失大。

但冯从义也不打算就这么简单的出让自己家的利益。等价交换,这是他的表哥跟他说过的原则,不占人便宜,也不做冤大头,这是冯从义做生意的底线。人心苦不足,养得贪了,日后一旦不能继续提供相当的利益,反而会滋生怨怼。

要交换什么,得到什么,都是冯从义现在需要考虑的事情。喝着伴当递上来的淡酒,冯从义半闭着眼睛,计算着,思考着。

