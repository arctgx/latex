\section{第32章 江上水平潜波涛(下)}

当天晚上,一盘油汪汪的红烧肉摆在韩冈的面前。

往常,韩冈都是立刻大快朵颐,但今天,他的筷子却每每绕过这道由爱妾精心烹制的佳肴。

“怎么不吃了?”严素心惊讶的问道。这可是韩冈最爱吃的几道菜之一。

“对养生不好。”韩冈叹道,他其实也是垂涎三尺,但既然他已经打定主意要节制,就不会再碰。

王旖带了孩子们南下江宁,至少一个月后才能回来,家里也变得冷清了许多。但韩冈与三名爱妾的关系,又重新紧密起来。

饭后,三女与韩冈聊了会儿,又拿上了今日的报纸。

严素心最不喜欢市井流言,却对报纸上的消息趋之若鹜,白底黑字,自比他事要稳定。

不过每次拿到报纸,严素心还都会去翻一下第三版,看一看新的情节出了没有,但一如既往的没有。

严素心心中怒火熊熊:“这些措大总是懒得很,这个月都断了几回了?,”

“有什么好头疼的。弄间小黑屋子关起来,什么时候写完什么时候放人。”

以前韩冈也最烦这种写书写一半就断掉的,或是匆匆糊弄一个结尾出来,又或是那些明明承诺了又做不到的,让人忍不住想把作者给拉了来关进小黑屋中,写多少字给多少饭。

现在有了机会,有了能力,当年留下的遗憾,岂能再让其继续成为遗憾?

“相公这个主意好,”严素心的眼睛亮了起来,

“别忘了,礼数要备足。”韩冈补充道。

……………………

蹴鞠快报诸多连载小说作者之一的甄五,趾高气昂的走进石婆婆巷中的一间小院,他的编辑邢立忠约在他这里见面。

走进门中,甄五看见邢立忠便大声笑道:“怎么今天约在这里,是不是刑兄的外室?”

邢立忠没跟着甄五一起笑,愁眉苦脸的:“真的不是,家里、社里天天挨骂,怎么还有心思置什么外室。”

“怎么了,出了何事?有什么难处,说出来,能帮我肯定帮。我们可是老交情了!”

甄五一幅义气冲霄汉的模样,拍着胸脯向邢立忠保证着自己绝对会两肋插刀。

“就是先生你的事啊。”邢立忠抬起眼,看着甄五,仿佛看鱼儿上了钩。“先生你这个月的份还没完成。当初约定好是天天交稿,现在才写了一半。”

甄五脸色一变,“是已经写了一半。”

“那也才一半吧。先生,所谓言而无信,不知其可。你这说话不能不算话啊。都月底了,还有一半的份没写呢。”

甄五就站在桌边,拿起桌上的茶壶给自己倒了一杯清茶,“圣人有言:‘取乎其上,得乎其中;取乎其中,得乎其下’。在下的确是说过这个月天天交稿,但这是‘取乎其上’,现在写了半个月,就是‘得乎其中’。不才甄五,只是遵从圣教而已。”

“先生!”

甄五啜了口茶水,放下杯子,“今天我还有事,先告辞了。这个月亏欠的,下个月肯定补上。”

邢立忠立刻拦到了门前,坚定地堵住了甄五离开的去路。

“还请让一让。”甄五拉下脸来,方才的义气冲天已经给风吹得一干二净。

邢立忠叹道,“若是小弟让开,先生你这个月多半就要‘取乎其下,则无所得矣’。”

甄五怒道:“莫要以小人之心度君子之腹。本……本人言出如山,既然答应了,就会去做!”

“只是做多做少就要两说了……”邢立忠帮甄五将下半句话说完,“小人斗胆,还请先生就在这里写,酒肉随时都有伺候。写完了,便放先生归家,写不完,里面这间屋子里面还有铺盖,就请先生在这里住下。小人听闻,先生每月拿到润笔之后,便会在外冶游一夜方归,家里面想必不会担心太多。”

房间的门开着,但房间内是黑暗的,连窗户都找照不进阳光。不过走到敞开的大门处,甄五便清楚的看见里面的陈设。

小黑屋只有一丈见方,一张床、一张桌、两把椅子,就把房间占得满满当当。

当然不会有甄五最喜欢的玉冰烧,也不会有劝酒时耳边的吴侬软语。

这是坐监吗?!

甄五心头火蹭蹭的往上冒,回头看见邢立忠依然不改的讨好笑容,恨不得甩他一嘴巴:“邢立忠,这事做得可就难看了。你可知无故拘禁他人是什么罪名?!”

“甄先生。甄学究。甄家兄弟!不!甄哥哥!甄老子!!甄爷爷!!!”邢立忠拿袖子在凳子上虚虚的擦了两下,硬是将甄五按得坐了下来,“俺都求你了,你老就在这里歇歇尊臀,把这一回先写出来如何?不说别的,多少人都在等着你这一回。一口气喘不上来,你说憋气不憋气?虱子叮在背心上,想挠挠不着,你说上火不上火?猴行者到底就没救出玄奘大师,你倒是先给个说法啊!”

甄五油盐不进,“写不了。这小说是想些就能写出来的吗?写的不好,坏的可是我这副金字招牌。”

“当真写不了?”邢立忠的笑容不见了

甄五咬定牙根:“决然写不了!”

“唉。”

叹了一口气,邢立忠放弃了,让开了门口。

甄五得意的扬扬下巴,正要说话,就见邢立忠对外面喊了一句,

“二位兄弟,看来真的只能靠你们了。”

两名一幅棺材脸的汉子,随即出现门口。

个头都不算高,却是往横里长。

两人一左一右,像门神一样守在门口。看相貌就是一贯横着走的,一眼瞧去就知不是好人。

软的不成就来硬的。

‘这些货,怎么就没给送去云南?’

甄五恼火的想着。

如今京城律法森严,便是窃盗,赃物满贯就要刺配云南。而街上游手好闲的泼皮都给寻了名目送去了云南。京师显贵无不大力支持,原来勾引家中子弟学坏,大多都是这一帮人与家中刁奴内外勾结,现在赶走了,自己家里再把刁奴送官,家里登时就清静了。

可这两位,一看就是欺行霸市惯的,怎么走在路上就没给人捉将官里去?

但甄五却毫不畏惧,难道《大唐三藏西域记》的作者甄五就当真只是甄五?以他的身份何须畏惧这些庶民。

“甄先生,请留步。”

其中一个泼皮开了口,倒是有几分礼数。

“没什么好说的。”

甄五冷着脸,便要从两人之间挤过去,但立刻就被人给揪住了。

这名泼皮发着狠,将手中的衣襟向上一提,甄五就只剩脚尖落地。

那人面目狰狞:“白天写不完,那就晚上写,晚上写不完,那就夜里写,我家主人嘴上长了个燎泡,就是等甄先生你的连载等出来的。”

甄五一阵心虚,住着胸口前的那只粗壮的胳膊,问出了口:“你们想做什么?!”

“奉主人的命,送些东西给甄先生。”

“什么东西?我不要!”甄五发起了读书人的臭脾气

但这两位却犹如强买强卖的奸商,不容甄五推举,“既然我家主人送出来了,就由不得先生不要。”

两人带来的礼物送到了,他们当着甄五的面,帮他拆了开来。

“座钟!”

只拆了一个外壳,邢立忠就大惊失色。寻常读者给甄五的礼物他见多了,但这么珍贵的器物,还是头一次得见。

“这摆钟就摆在这里,我家主人将此物送给甄先生,免得甄先生总是找借口推脱该完成的分量。”

邢立忠绕着座钟走了好几圈,越来越觉得这是一座兼具了美感和实用性的器物。

“怕不要一百贯吧。”邢立忠啧啧称叹。

甄五摇头,“礼太重了。”

这么重的礼,他可不敢收。守礼都是讲究交换的,从两人手里得到这么好的东西,自己要付出什么样的代价,可就难说了。

“这哪里重?我家主人的好心情,又岂是一百贯买得来的?”

甄五的脸色越发的难看,“这礼物我可不要,这间屋子我也不会留下来。”

“只怕由不得甄先生你了。”

甄五的脸色终于变了,但他现在想逃已经来不及了。

“本……本官。”甄五嘴唇抖着,终于泄露了自己的身份。

“在我家主人那边,鸿胪寺主簿也算不得个官。”那人冷笑着,“中太一宫,景灵宫,会灵观,都有的是位置以待贤人。”

甄五气得笑了起来,这三处都是宫观,专一养闲人的地方,比清闲得门口能让母鸡抱窝孵蛋的鸿胪寺都不如。只是身份泄露,却让他隐隐觉得不妙。

“别以为本官找不到人!”甄五发着狠。

苏颂可是做过判鸿胪寺,主簿虽是小官,当年每日相见,也算是旧部了。前任宰相,现任平章,苏颂的旧部,有哪家贵人敢欺上头来?

“苏平章虽为我家主人尊重,但我家主人可不会怕他。”那人走近了,在甄五耳边轻声说了一个字。

甄五脸色骤变,“为什么是我?!”

“谁让陈主簿写得一手好文字呢。”那人大笑着,“写得慢了,写得让我家主人看得不开心了,是什么后果……你知道的。”

甄五呆若木鸡。

两人扬长而去,邢立忠伸手拍了拍甄五的肩膀,一脸同情,“甄……哦,陈先生,还请多多努力。”

……………………

“还是官人说得对,这些懒鬼就该如此对付。”

严素心拿着今天的报纸,喜笑颜开。

但韩冈早就忘记了前两天的闲聊,只记得了今天听到的话。

太后打算安排皇帝大婚了。

实在不知该说什么了,说得再多都觉得是借口,只能这个月继续努力了。另外,祝祖国六十五岁生日快乐,祝各位节日愉快。
