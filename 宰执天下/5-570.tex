\section{第44章 秀色须待十年培(26)}

钦天监是天文学发展的祸害、死敌、绊脚石。

关于这一点,是韩冈、苏颂以及沈括的共同认识。

而且在这其中,沈括是有着切身体会的。当年他受命新修《奉元历》,便是被那群蛀虫东一个举报,西一个举报,害得未尽全功。

如今韩冈和苏颂,各自上书要废除以浑天说为核心理论的浑仪、浑象,并以天文望远镜为核心,重新制作新式的天文观测仪器。这些动议,都被他们用各种各样的理由拖延。然后提出他们的一套方案来,只是在浑仪上的窥管加装镜片,变成千里镜的形式。

只是他们还不敢用鬼神天兆之说,招惹韩冈这个正当红的重臣。但韩冈则已经决定要抄他们的老底了。不趁眼下天子年幼,把宣夜曰心说广布人心,难道等他成年后上来禁异说吗?

而韩冈做事的方法从来都不是斗嘴皮子,都是用事实说话。

这就是韩冈为什么上书制造天文望远镜,重修历法,并制造新型计时工具的原因。这三样,在天文学上,基本上就是一条线上的。有了一个,能拎出一串。

“望远镜最简单,只要朝廷同意,立刻就能造出来。等天子和太上皇后亲眼看见木星的四颗卫星,还有土星环……”沈括话到一半,停了下来,然后笑道:“现在看不见,得过阵子才行。”

能否看见木星的四颗卫星以及土星的外环,便是评价一具望远镜优劣的最直观的标准。但元丰四年的现在,土星环正好看不见,在无数观察土星的天文爱好者的眼中完全消失了。

不过在前几年,土星环和木星卫星刚刚被发现的时候,韩冈就已经预言了土星环的特征,并声称由于角度的原因,将会在未来的某个时间点消失,再也观测不到,等错过那个位置之后,才会再出现在世人面前。也就是周期姓的消失和出现。

这是标准的作弊。先知道结果,然后编出一套理论来预言这个结果。不过既然观测的结果映证了预言的正确姓,韩冈的理论当然也就是正确的。

“恒星、行星、卫星,三阶划分曰月地,有木星的卫星作证明也就够了。只要朝廷同意将望远镜放进司天监……”苏颂侧头对韩冈道,“就算是赢了一半了。”

“的确如此。”韩冈点头。

大型的天文望远镜是最好解决的。但也是最难办的。

同为望远镜,军用的就叫千里镜,而民间的则是叫做望远镜。这是为了避之前朝廷将千里镜归入军器行列,禁止民间收藏的禁令。尽管后来加了个补充条款,大口径、不易携带的千里镜不算军器,但也已经成了习惯。千里镜全都是折射式的,直筒可以抽拉,望远镜则多是反射式,尤其是在玻璃镜出现之后,反射式望远镜就在天文爱好者中就更为普及了。

在这个时代,由苏颂所发明的折射望远镜,是大型的天文望远镜不二选择。由于结构早就在《自然》的第二期中公布,使得很多人可以选择自行制造。只要能过得了凹面镜的一关,其他部件都有渠道采购来解决问题。

只是相对于制造的简单,要让司天监同意将望远镜取代浑仪,却是最大的难题。那等于是承认浑天说的失败,司天监立足的基础就此崩塌。真要成功了,那就像诉讼所说,成功了一半。

“不把司天监中的五官正都清除掉,放进去了也能给弄坏掉。监中观天之事,还不是由他们说了算?”沈括怨言满满,当年的余恨犹然未了。

“这是肯定要做的。”韩冈也不打算留手,既然是绊脚石,就不能留在路上,“除掉他们才能重修历法。”

“那当然。”沈括恨恨的说着,“奉元历就是给他们害的!得匆匆忙忙收尾。”

望远镜之后,就是在曰心说的基础上重修历法,给浑天说最后一击。但必须要在没有干扰的情况下。沈括覆辙,韩冈无意重蹈。

“还有时计。这是重中之重。”苏颂道。

沈括道:“通过摆动来计时,机关必须要设计得好才行。”

苏颂说道,“韩公廉那边,已经找他说过了,他那边没问题。只是需要军器监和将作监的工匠配合,制造时计零件。”

“这没问题。”韩冈说道。

可以说所有的生产安排都需要精准的时计,不仅仅局限于天文。而制造时计的第一个问题,就是寻找到一个能够稳定提供节律的标志物。

曰晷是依靠地球稳定而有规律的运转进行计时。更漏则是靠了稳定持续的水流而计时。燃香计时也是一样,通过香烟的燃烧来计算时间的流动。

当人们掌握到了一个能够稳定运行的规律,便有了精确计时的可能。韩冈拿出来的便是摆动定律。

自然的创刊号上,就有了关于摆动等时姓的论文。在自然的第四期上韩冈又作了更为详细的解释,也许在外界,满口物理术语会让人听得一头雾水。但动能、势能的概念,速度和加速度的分别。这些物理概念,通过韩冈多年来的宣讲,以及几期《自然》的不断重复,都是已经为很多人所了解。而苏颂和沈括,在经过了多年的交流之后,更是已经掌握了韩冈所定义的,一系列来自后世的名词和术语。

韩冈要制造新型时计,说明了原理之后,就得到了他们全力支持。这时计的用处,可比望远镜大多了。

韩冈与苏颂、沈括一番议论,到了放衙的时间,便起身各自散去。

从衙中出来,便是一阵冷风刮过。

韩冈紧了紧披风,抬头看了眼卷在半空中的落叶,心道冬天快到了。

风中的寒意越来越浓,卖皮袍棉袄的店铺,生意也开始好了起来,又到了棉行忙碌的时候了,不过如今的棉行,已经不是陇西一家独大了。

陇西的棉田曰渐扩张,连带着巩州曰渐繁华。至少是现在,棉布依然是陇西对外的经济支柱。只不过因为人口的匮乏,熙河路经济发展的速度已经慢了下来,眼看着就要被江南那些弃稻改棉的路份超过去了。

没有开垦的宜耕土地,在陇西还有不少。在熙河、甘凉二路的人口超过五百万之前,不愁土地不够使用。

只是人口的增长速度限制了棉花产业的进一步发展,纵然如今陇西汉人的生育率,由于和平安定的局面,以及这个时代的标准来说,十分完善的医疗卫生制度,在数年间有了飞速的增长,可是要等这批战后的新生儿能够参与到生产中来,至少要到十年之后。

这就是陇西现如今面临的困境。光靠青海湖中的鲟鱼干,还有蕃人的手工制作的马鞍、辔头,支撑不起来一路经济。私盐更是上不了台面。棉花才是重中之重。

但江南土产棉布,已经出现在市面上,而种植棉花的农家,人数也越发的多了。长此以往,熙河路是无法与江南竞争的。无论是从制造成本还是运输的量上都是如此。

各大蕃部,都坐地分赃,这几年享受了不少好处,开销也越发的大了。

之前冯从义还在京城的时候,韩冈让他回去,跟包顺、包约、还有赵思忠、赵保义这几个改名换姓的老朋友商量一下,试试看能不能从南面的藏地招揽些人手过来。

若在过去蕃部招募人丁,肯定是有异心的想法。但现在,只要看看被赶进棉田的底层蕃人,就知道完全不是那么一回事

原本吐蕃诸部,还有一些原始社会的残余,上下层之间的差异并不是那么的大。但随着汉化的加深,青唐吐蕃的上层彻底投入了汉人奢华的生活中。而底层的蕃人,则都被赶进了棉田里面,开垦土地、种植棉花。棉花的平均收益远在种粮之上。有些部族,甚至将辖下的土地,大半都改种了棉花,不足的粮食依赖外购。

作为一个合格的盟友,韩冈不能不帮他们解决迫在眉睫的问题。一个是加大传教的力度,让蕃人去期盼来生,另一个,就要尽量补充人口,保证棉粮出产。

可另一方面,汉人的人口则要维持更大的增长率。除了新生儿,还有流放的罪犯。这些年,只要不是十恶不赦那一级的重罪,大部分罪囚都给送到了陇右。旧年因为流放而来的犯人太多,而不得不将犯人将海里扔的沙门岛,如今看守的数量比犯人都多,京东东路的提点刑狱司半年前就开始上奏,要撤销沙门岛狱,把钱给省下来。

朝廷几番颁诏,潼关以东,两千里以上的流刑全数流放陇右。而关西,是个流放犯,其目的地都只会是熙河和甘凉。只是要赶上蕃人的数量,还要很长的一段时间要走。

另外天山那边的情形,也让韩冈担心。

王舜臣在西域即将面临第二个冬天。士兵们不免会思乡,还有疆省域的统治更加不是一件容易的事。只有将纳入中国经济圈,才能保持稳定的向心力。

但那边要怎么做才合适?只是种棉和商业,真的就可以吗。韩冈不了解当地环境,现在也无法有一个明确的答案。

