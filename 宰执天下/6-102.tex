\section{第十章 千秋邈矣变新腔(24)}

‘天监不远,民心可知’是仁宗时的故事。

其作者林希曾经是开封府试的解元,礼部试的省元,殿试时,一篇《民监赋》写得远胜同列,故而被考官们列为第一。

但其中两句‘天监不远,民心可知’犯忌,仁宗看了就不喜欢,林希也就因两句话丢掉了状元和连中三元的荣耀。而同科的章衡,也就是接替林希成为当科状元的幸运儿,他的破题则是很讨好的‘运起元圣,天临兆民’,远比林希更得仁宗的欢心。

同样的,让王安石丢掉状元的‘孺子其朋’就更有名了。这桩公案,时刻提醒着殿试的考生们,必须要注意文章中的遣词用句。

可是这一回,宗泽的策问犯忌之处其实甚多,太后和宰辅都没逃过,甚至于今党争含而将发的局面,也议论到了。太后根本就没看懂文章的内容,否则绝不会选宗泽。

“只不过状元郎的水平可以质疑,但状元郎就是状元郎。”韩冈道,“嫉恨也好,鄙视也好,都改变不了宗泽成为壬戌科的进士第一。”

宗泽在外游历的时日不短,但仅止于游历,见识虽不差,却也失之偏狭。对申论一题的回答,不能算是太好,而策问中论事,除了刚直一条让人赞赏,终究还是肤浅了一点。但太后既然点了他为状元,那状元就是他了。

殿试之所以设立,也正是为了让皇帝得以示恩进士,从而断绝过去那种座师与门生之间的关系链,使得新科进士感念天子而不是考官。这是代天子听政的太后的权力,做臣子的没有理由阻拦。

“黄裳明白。”黄裳语气沉重。

太后之所以会点了宗泽,不是因为宗泽的考卷内容,也不是太后的心胸有多宽广,太后只是记得宗泽当初所写的战局点评。尽管那只是一家之论,可既然被太后记下了,一个状元也就是命中注定【注,】。既然对宗泽都看好了,就算事后得知宗泽文章中的真意,也只会觉得自己得到一个诤臣。

这都是命数。

黄裳知道韩冈不喜欢这样的说法,可也忍不住这样去想。相对于宗泽的幸运,自己的运气就差了那么多。

恩主费尽心力做好的铺垫,自家却没能接上手,这就是运气。

如果只想做一个平平庸庸的官员,其实现在就已经足够了,有了进士的身份,又已经升做了朝官,还有军事和政事上的经验,这辈子最差也能在州郡任上养老。

‘可是啊……’他偷眼看了韩冈一眼,原本在一群老态的东府中显得格格不入的面容,在灯下则更为年轻,只是灯火在脸上留下的阴影,让人感到一种深沉的威严“这让人如何甘心。’

在韩冈身边久了,总有种奋进的力量,让人不甘平庸。看到多少原本被认为不可能完成的成就,在自己的辅助下一桩桩实现,又怎么让人甘心从此庸庸碌碌下去?

抛开了心思,黄裳对韩冈笑说道:“不过这一回殿试,宗汝霖虽是夺了状元,但气学得益更多,日后国子监中,又要多一门课了。”

“这也免得百姓遭殃。”韩冈说道,“难道发了大水拿论语去补堤坝不成?”

这一科的殿试,真正的赢家的确正是气学,是韩冈本人。

自从进士科成为众科之首,决定进士命运的科目,便成为士林中最重要的一个风向标。

今日韩冈硬是将申论放进殿试去,日后谁敢放弃对申论体裁的钻研?而与申论息息相关的气学,其中的著述,当然更是研究的重点。

如果申论仅止于殿试,那不在乎名次的考生还可以放一放,不去在意。可韩冈如今已经是参知政事,不论谁来看,只要站在韩冈的立场上,怎么可能不会想方设法的将申论放进礼部试的科目中?而以韩冈的年纪,王安石能挡住他多久?

而且以申论考核的内容来看,王安石又如何反对?

‘华辞无补于治’,此王安石变贡举法的理由,而背上一肚子经义,却不能用在实处,如何‘补于治’?

“参政说得正是。诗赋也好、经义也好,入朝为官最重要的还是得放在治事上。”黄裳道,“即便是王平章过来,也不能说不需要考一考贡生们的治事之材。否则身言书判,就没必要加那个‘判’了。”

自唐时传下来的规矩,新科进士释褐,要过身言书判四关。相貌、谈吐、书法和判事。

尤其是最后一条,标准是‘通晓事情,谙练法律,明辨是非,发摘隐伏’。尽管能够做到这四句的官员,实在是凤毛麟角,百中无一,但相貌、谈吐都不再成为拦路虎的今日,人们可以对结巴或丑陋的官员给予足够的同情和容忍,可没人能说官员不需要有办事的能力。

申论这一新体例,其目的也正是为了考察考生们是否对政务处理有着最基本的认识。通过对已知信息的审视和分析,抓住其中的问题,并给出一个具有可行性的解决方案,最后再针对这个方案加以论述。对官员眼界、常识都能考量到。

“不过参政接下来打算怎么做?可是下一科的礼部试……?”黄裳又问。

“我还没那么急。”韩冈笑得很轻松,时间在他这边,“以后再说不迟,先让铨叙的吃点苦头。不过下一科的礼部试,可以试一试百分制。这样经义上的错误,也可以用策论来弥补,不至于失去贤才。”

申论只是重点之一,推广百分才是更重要的一条。

放在还有诗赋论的过去,除了赋文是重点,诗、论两篇都可以放一放。而韩冈将殿试考试的分数换算成百分,申论虽在其中仅仅占了三成,却没人敢忽视,甚至只占十分都不敢忽视——哪个看不出来,这样的评分方法,在考试时一分都将是关键。

有了分数之后,策、论两事,就不一定要非此即彼,同时各为一题也是可以的。经义的部分,又能被计入总分之中。那些本因错题过多而被黜落的贡生,也有了逆转的机会。

“这样还能插进入更多的考题,申论不用说了,诗赋也可以,只占个十分,依然以经义为重,谁能说不是?”

与韩冈配合得久了,黄裳很容易看透韩冈的心思。

韩冈笑而不语,也许再过几科,礼部试的考卷,就会塞满了各式考题。

放水两个时辰,进水三个时辰,进出水同开,多少时间能将水池放空,这样的考题就算在全卷之中只占上三五分,又有谁敢放弃?

……………………

“三个时辰。”

韩钟做好了他的题目,忙拿着叫给父亲。不过还是比他的弟弟和妹妹要慢了一点。

给儿子女儿出的算术题,可比韩冈打算出给未来贡生们的试题更难。不是几个时辰放空,而是问放到一半或放到三分之一、放到五分之一,要多少时间。这样更多一重计算,也更难了一分。

答案对了,可韩冈还是仔细的看过他计算步骤之后,方才点了点头。

在韩冈做学生的时候,觉得一步步的写下计算步骤很麻烦,有些题目直接就能心算出答案,但当他开始教授弟子,答案虽重要,可确认计算方法才是最重要的。

“好了。快回去”

老大就要满十周岁了,不能再住在后院中,得当成成人来对待了。王旖和严素心正张罗着给他在外院准备单独的小院,还有住处的布置和准备,更重要的还有跟随他的伴当,免得学坏了。

这样的改变,也可以迟至十四五。但早早独立成人,

千年之后,如韩钟这个年纪,也有许多出外读书的学生,根本就没有太多可以操心的。

“等满了十四,就去横渠书院。”

金娘仰头问着:“哥哥不去国子监?”

韩冈笑着摸摸女儿的小脑袋:“国子监哪能跟横渠书院比,那里面能学到什么?”

“能进国子监就能中进士。哥哥不考进士吗?”金娘像个小大人一样认真的说着。

“等你哥哥要去考进士的时候,考题早就变了。”王旖笑道,“官人,是不是?”

“那当然。去横渠书院可以早点习惯一点。”

老大要去参加进士科,还是有十来年的时间。等到他和家里的老二去考进士,进士科的考题的确早就变了。而且是面目全非的改变,绝不是现在人们以为的申论。

百分制隐藏在申论之后,对考试科目的改变其实更大。

科举有数百年的历史,不论从诗赋转为经义的进士科,还是秀才、明经、明法、明字、明策、道举这样逐渐消失或不为人所重的科目,都是一样的评卷方式,而百分制可以改变所有科目,可以更为精细的安排考题,也适合安插进更多的试题。

也许到了这个时候,王安石和章敦应该想通了,但还能来得及阻止吗?

纵然是老瓶也得装进新酒去,老歌也要唱出新调子。

对科举考试的改变,正是从这里开始。

注:真实的历史上,元丰五年的状元黄裳也是一开始被排在第五甲,因为神宗赵顼记得他过去的文章,故而‘至唱名,令寻裳卷,须臾寻获进呈,神宗曰:此乃状元也。’
