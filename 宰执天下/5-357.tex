\section{第33章 枕惯蹄声梦不惊(14)}

喊杀声还在峰峦间回荡,而谷中的战场上则已经恢复了平静。

折克仁骑着马穿行在已经被撂荒的麦田田垄上,周围尸横遍野,旗帜兵械也落了满地。这是一片刚刚结束了的战场。

当折克仁自打扫战场的人群中经过,从一个指挥的指挥使,到最下面的小卒,都不敢稍稍抬头。战场上,唯一能趾高气昂的只有跟在折克仁身后的折家子弟兵。数量几乎相当的契丹铁骑,只用了一刻钟的时间,就正面被他们所击败。

原本埋伏在谷地两侧、准备前后夹击的代州兵,完全没有发挥任何错用,甚至在辽军开始逃跑后,都没能尽到阻截的责任。

尽管来得及逃走的辽军只剩下不到七成而已,且个个丢盔弃甲,但领头的契丹将佐手持骨朵,左劈右砸,似乎没费什么气力就冲破后路上代州兵的阻截,领着一半以上的兵力逃出了生天。

“是皮室军!竟然是皮室军。”

打扫战场的士兵们突然乱了起来,好像发现了方才交战的辽军番号。

“皮室军?!”

更多的人开始惊讶,毕竟辽国的宫分军和皮室军都是在河东军中赫赫有名的对手。

“是萧十三那鸟贼派来清剿的前部,竟敢直接踏进山中,胆子倒是大,就是不长脑子!”一名将校得意洋洋,穿在身上的盔甲暴露了他的身份。

秦琬脸色并不好看,代州兵丢人现眼,让他在折克仁面前好生没面子。

将为一军之胆,没有有人望有资格的将领统帅,又在官贼间反复,河东数一数二的代州兵就变得任人鱼肉,连群败兵都拦不住。换作是自家父亲在时,又几曾畏惧过什么皮室军?!

“也还不错了。”折克仁似乎看破了秦琬的想法,过来笑着宽慰。

秦琬恨恨的瞪着几个指挥使:“这时候都不拼命,当真王法是摆设吗?”现在他立场跟前几天截然相反。

“韩学士不是让人传了话,自全为重,所以是有恃无恐吧。”

“韩学士宽仁,当年广锐军也靠了他才没有被发配岭南。但这并不是这群贼囚能轻松脱罪!”

不过折克仁已经很满意了。经过了几天的整训,这一支临时的军队已经有了些模样,今天在战场上也没拖后腿。

在军需补给上,西面的徒合寨已经安排了人力将粮草运了上来,加上原有的积存,还能支应上一段时间,足够撑到将辽人赶跑的时候。

而在整备军力的同时,折克仁还选了一些人手,潜到忻州城外极近处的山林中,然后放火烧山。虽然此时气候潮湿,山火烧不起来,但滚滚浓烟足以昭告城中的守军,此时辽军已经无法控制忻州外围的局面。只不过这两天辽人加强了守备,能成功接近忻州城的斥候越来越少,被俘被杀的数量在直线上升,折克仁已经在考虑将人暂时给撤回来,省得浪费宝贵的人力。

但不管怎么说,随着一南一北麟府和京营援军两大主力一步步的接近,在折克仁的眼中,胜利已经离之不远了。

……………………

一支支由床子弩射出的踏橛箭,插在忻州的城墙上。

长而坚实的箭杆让攻城一方可以藉此攀援上城头,可是从城上丢下来的石块瓦片密如雹雨,狼牙拍和檑木更是直接砸断了扎进高处的踏橛箭。

一个上午的数次攻势皆是无功而返,而太过频繁的射击频率,反而使得近三十具床子弩,毁损了其中的四具。

忻州城的城防从半个月前,看起来就已然是摇摇欲坠,可直到今曰,却也只是摇摇欲坠,而不见被攻破。

“看起来还是不行啊。”一名国舅房的将领大声叹着气,“如果再能有些宋人,就能让他们垒土上城了,比光射箭要好。”

另外的几名将领也在大点其头,显然是说进他们心里去了。张孝杰闻言脸上闪过一层青气,强自压住心头怒意,转头对萧十三叹道:“哪里还有那个时间?”

‘而且也没人了。’萧十三心中暗道。

忻州城下,环绕着城墙,有着一片衣衫破烂的尸骸。都是抱着侥幸的心理没有逃离的百姓,被辽人驱来攻城。

之前萧十三领军南下,张孝杰本来是准备回镇代州,只是忻州城外的三千降兵突然作乱,使得他不得不赶来主持围困忻州的战斗。

但在兵力不足的情况下,除了分出兵马来防备逃入山中的宋军之外,剩下的兵力只够他驱赶百姓来攻打忻州。

用了两天的时间,驱使来的宋国百姓就在城下消耗一空。战争之中,普通人的姓命就跟草芥一般无足轻重。城中守军为了自己和满城上下的安全,如何会手下留情?尽管城上有许多人都能在给驱赶来的百姓中找到自己的亲友,可最后照样是箭矢无情。

没了任意牺牲的消耗品,能用的就只剩下小富即安的自家人了。可要想解决手下人出工不出力的现状,对萧十三和张孝杰来说,实是力有未逮。历代大辽天子都很难做到的事,耶律乙辛同样难做到,更不用说他们这两个尚父殿下的走卒。

在这样的情况下,为了能尽快攻破忻州,萧十三和张孝杰不得不加强了远程力量。神臂弓都集中起来使用,床子弩也一样,工匠们加班加点的打造霹雳砲,只为了突破上城时更简单一点。

正当两人绞尽脑汁的时候,城外的宋军却开始在山林中燃放烽火。紧邻着忻州城,最近处的几处山林都被潜入的宋军点燃。

尽管春湿浓重,草木难以点着,可宋人在林子里只是为了生狼烟,并不在意到底能烧掉多少林木。看到附近的山中烽烟频起,萧十三清楚,近几曰忻州城中的士气,就是被这些烽烟给撑起来的。

萧十三为此加强了拦子马的派遣,不惜耗费人力来扫荡近处的山岭,从前天起,山中烽烟的数量便陡然下降,只是他没办法消灭源头。

为了用最快的速度解决山里面的那些苍蝇,萧十三调动了整整两个千人队的皮室军,让他们直逼宋军。尽量将之歼灭,如果做不到,也要将其给赶跑,决不能让其留在忻州附近。

让手下最为精锐的一部兵马去对付一群乌合之众,萧十三要不是看在地势的关系上,也不会这么小心。

但萧十三的心腹精锐在进攻盘踞的一处军寨时,却遇上了另外的一支精兵,七八百人的样子,没有一见到人,就散入山林中胡乱放箭,而是当道列阵以待。一开始领军的辽将还以为是装模作样,可他领着这一支皮室军冲着军阵撞上去后转眼就败了,而且是惨败!

最为可恨的,是战败后又在山中为埋伏起来的宋军阻断了后路,冲破阻截回来的不到七成,而这些败兵原本带在身边的战马,除了胯下骑乘的一匹外,其余的基本上都给丢了。

这个结果差点没把萧十三给气疯掉。

就算是因为身家丰厚了,开始变得不想冒风险,但堂堂皮室军的威名呢,自负呢,难道都当马粪一样半路上给拉掉了?

这几天的攻防战,一来一往的几次下来,都是大辽这边吃亏。

派出去的探马好歹捉到了几个活口,知道了那一支精兵是来自府州的兵马。而且麟府军的主力很快就要到了。麟府军放弃一切,赶来救援。除非能趁其人困马乏之际迎头击败,否则就不可能来得及如果能攻下忻州,事情还有挽回的余地。从忻州出发的兵马,能镇压附近百里方圆的土地,让出山的宋军有来无回。可要是攻不下来,再犹豫的结果,就是石岭、赤塘两关的万余守军将会来不及撤离,以至于全军覆没。

张孝杰已经看到了最坏的局面,而战局不可能在短时间内打破,必须要退了。他瞥了瞥萧十三,琢磨着要尽快向这位主帅摊牌。

萧十三感受到了张孝杰的视线,突然抬眼问道:“要不要去看看?”

“什么?”张孝杰微楞。

“之前活捉的探马,不是让他去城下劝降的吗?差不多该安排好了。”萧十三说道。

张孝杰摇头苦笑:“多半不会有什么用。”

之前张孝杰就曾接二连三的派遣宋国的官员和俘虏前去劝降,甚至还让他们伪报消息——其实有好几座城寨就是这么不费吹灰之力就拿下的——可对于忻州城,却都没有造成太大的作用。

萧十三笑了笑,站起身:“现在情况不一样了,或许忻州守军身上也只差最后的那一分一毫。”

之前那是兵马不足,吹得再厉害也不容易让人置信,而现在他都带着主力赶回来了,虽然不可能让狗头们卖命攻城,但围着忻州城漫山遍野的营帐,也足够骇人。加上山中的南蛮子也被逼退,城中恐怕正是惶惶不安的时候。这时候,放些狠话也能让人相信,而且也是给了城中一些人出降的借口。

萧十三觉得这样挺好,有用没用试试再说,反正也不会费多少事。

被派去劝降的被俘斥候相貌很不起眼,普普通通的角色,萧十三远远地看了两眼,就没再报了希望,也没有过去封官许愿。

城下旗号鼓噪,城上冷眼相看,那名斥候被几名手持橹盾的辽兵护送到了城墙近前。

“我乃府州……”

话声刚起,城上便是一箭飞来,对准了的面门,却被身边的士兵用盾挡了下来。之前曾有降辽的宋官被派到忻州城下劝降,而后被乱箭射死,现在辽兵们就学会了帮忙拿着盾牌,而城上的守军也学会了不再浪费箭矢,大部分的情况下仅仅是散散的射上几箭,做一下警告。

挡住了几支箭矢,宋军的斥候再次探出头来,向着城上大喊着:“我乃府州帐下殿值张忠孝,辽狗已败!韩枢密、折府州转眼即至!坚持!坚持!!”

城上的箭矢顿时停了。

在后鼓噪助威的旗鼓也停了。

吵吵闹闹了多少曰子的战场上一下安静了下来。

萧十三和张孝杰神色陡然一变,而在斥候身边的辽军士兵全都煞白了脸。

当那名斥候还想再重复一边方才的喊话,便被横拖竖拽的扯了回去。

宋军斥候被拖到萧十三的面前时,已是被打得满口是血,甚至已经无力再站起,但他还是在笑,甚至洋洋得意的扬起了眉毛。

这副得偿所愿的笑脸让萧十三气急败坏:“把他给我拖出去碎剐了!”

“呸,辽狗!爷爷在下面等你!”那斥候狠狠的啐了一口。

被拉去帐外的已经开始动刀了,但萧十三没有听到一声惨叫,只听得一声声笑,“辽狗,爷爷在下面等你!”,不过很快就没了声息。

帐中有些静。能听见从忻州城传来的叫骂和呼喊。

萧十三和张孝杰对坐无言。

半晌之后,张孝杰干笑的出声:“要是宋人的将官都如此,我看恐怕连雁门关都打不进来呢。”

萧十三沉默着,没有任何反应。

“退吧!不能再打了。”看着萧十三的样子,张孝杰叹了一声,“不说军中的士气,看到方才的那一场,忻州城中无论如何都会坚守到底。”

萧十三抬起头来,像是被羞辱了一般涨红了脸:“但守住代州、忻州是尚父的意思……你可明白?!”

“但尚父更不会乐见西京道的兵马都在忻州和石岭关上被消耗一空!”
