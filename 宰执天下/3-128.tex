\section{第34章 雨泽何日及(三)}

与王雱和吕惠卿又说了两句,韩冈返身回到阁门中。

无视同在阁门中等待入对的同伴们探索的目光,韩冈坐下里沉思起来。从王雱那边,他稍稍了解到郑侠这个人,想不到竟然是王安石的弟子。由于不可支持新法,而被贬在安上门做监门官。

这就是王安石的错了,当断不断,必受其乱,既然不肯合作,远远地请出去就好了。即便不死心的想起用,也该安排一个清闲自在的差事,怎么让他坐了一个监门官?以为他是侯赢吗?最后好端端的师徒情分变成了仇家,韩冈也只能摇头。

郑侠不为权势所动,甘居陋巷而不移,从人品上,无可指摘。但这等人也是最麻烦的,固执、坚定、认为自己坚持的都是对的,自己反对的都是错的。同时因为他们的品德高尚,也让外人觉得他们主张的观念也同样有理。旧党的声势,现在有很大一部分是被他们所张扬起来的。

旧党之中,有因为利益而对新法恨之入骨的,也有郑侠……甚至程颢、程颐这等为理念而反对的。后者清正廉洁的名声,反过来就给前者镀上一层金,好像文彦博、冯京之辈,也跟郑侠他们一样干干净净、清廉洁白。其实呢……根本不是一回事。

想到要跟正人君子一较高下,韩冈也觉得很头疼,这等事太麻烦了,反而是打文彦博的老脸还轻松一点。

正暗自叹气的时候,一名班直走了进来。他在门内站定,高声道:“右正言兼集贤校理、权发遣提点开封府界诸县镇公事韩冈何在?”

韩冈立刻站起身:“韩冈在。”

“陛下有诏,着韩冈越次入对!”

“臣遵旨。”

从入觐的顺序上看,韩冈绝不会是阁中的第一位。但天子让他越次,当然无人敢有异议。

出了阁门,韩冈随着来通传的班直往延和殿去。他并不担心郑侠的流民图能起什么作用。流民图又怎么样,那都是他玩剩下的手段。

当年渭河荒田一顷和万顷之争是怎么解决的?沙盘又是谁献的?郑侠献流民图,与他献沙盘明古渭地理,都是为了更直白的向天子证明自己的正确。

要说应对,他有的是底气。

……………………

延和殿。

王安石此事还留在殿中,正为自己而辩护,“水旱常数,尧、汤所不免。陛下即位以来,累年丰稔,今旱暵虽逢,但当益修人事,以应天灾。”

‘禹水九年,汤旱七年,而民无饥色,道无乞人!’

贾谊的一番话,就在赵顼嘴边没说出口,他不想与自己的宰相发生争执。但王安石现在所说的一切,在赵顼耳中,都成了强辩。王安石说了一通还不够,还让自己招韩冈来相问,但想想郑侠的话,‘十日不雨,乞斩于宣德门外。’这命都赌上了,赵顼如何还能不信?!

赵顼想不到他辛辛苦苦这么些年,本以为百姓丰足,国家强盛,而在西北边境上的成功,也的确证明了这一点。但没想到市易法一出,就是遍地怨声。等到旱灾持续了半年,更是将大宋的老底都露了出来。

他看了一眼桌上的流民图,又想被烫到了一般,立刻将视线挪开。他的国家,他的臣民,生活得竟然如此凄惨,赵顼心中如何能好受?

听到外面的通传,韩冈终于到了。

赵顼眯起眼睛,就见他一直十分欣赏的年轻臣子,从殿外进了殿中,目不斜视的在大殿中心行礼如仪。

“韩冈。”赵顼第一次不是称呼他为韩卿,“这份奏章和图轴,你自己看一看吧。”

从李舜举手上接过郑侠的奏章和流民图,韩冈匆匆看了一遍,便行礼回道:“陛下无须忧虑。臣即为府界提点,自当尽力而为,不至使万千流民如图上所绘之状。”

“朕不是说日后的事,朕是问你白马县中如今的情况!”赵顼见到韩冈弯弯绕绕的避而不答,心中怒火噌噌而起,“郑侠指你阻流民于白马,使其不得至京城受赈,此事可否有之?!”

天子震怒,如同雷霆,但韩冈凝神定气:“郑侠说臣阻十万流民于白马,此事诚有之。”

赵顼闻言一惊,面上顿时泛起了青气。而王安石持着笏板的双手也一下抽紧,而韩冈平平静静的继续说着,“只是尚不及十万。至前日,有六万四千四百余口,延至今日,当已过七万。”

“七万流民……”赵顼其实知道白马县的流民人数,韩冈本来就是一日一上报,但现在这个场合听到耳中,这个数字就变得太过于沉重,让他无法承受。颤抖的手指着韩冈,“韩冈,你竟然当真将数万百姓阻于白马。”

“陛下不以臣资历浅薄,而用臣为府界提点,不正是为了阻流民乱京城吗?”韩冈反问着。他知道自己必须以快打快,根本不等赵顼说话,接着道,“臣斗胆敢问陛下,流民如今背井离乡,究竟是何原因?”

“那要问问你们了!”赵顼被韩冈弄得十分恼火,竟然跟王安石一样,都在强辩,还以为他好蒙蔽吗?

韩冈冷静如常,自问自答:“是因为乏食之故。若坐于家中即可饱食,任谁也不至于弃祖先、离乡土。所以河北流民南下,乃是为了就食而来。”

“这又如何?”赵顼冷然道,怒火似乎一下不见,只是眼神冰冷。

韩冈不在乎天子的语气,只要皇帝不再被流民图蒙蔽了双眼,而开始思考,他的目的就已经达到了。他现在所要做的,就让天子能冷静下来好好想一想。

“饿死是死,落草后被官军擒杀亦是死,后者好歹还能多活几日。若当真逼到绝境,就是陈胜吴广在大泽乡之事。所以臣斗胆再问陛下,六万、七万,数日后将至十万之数的流民,如果当真在白马县吃不饱饭,典妻卖儿,难道就不会往京城来求一个活路?他们若是要走,可是区区两千户的白马县所能阻?!”

韩冈质问得理直气壮,郑侠的攻击,只要揪住一点就够了。

赵顼一时没有词了。若是仔细一想,韩冈说得也是的确有理。他是被流民图给冲糊涂了,要流民当真忍饥挨饿,早就有人揭竿而起了。韩冈再有本事,也不可能阻止得了数万饥民。

韩冈见到天子终于沉吟起来,朗声道:“安居足食,这就是臣将数万河北流民,阻于白马县中的手段。郑侠以此来指臣有罪,臣甘当其罪!”

赵顼不知不觉的摇摇头,“是朕误会卿家了。”

赵顼这么一说,连带着立于一旁的王安石都放下了心来。

只听韩冈道:“郑侠远在京中,不知白马县中之事,只凭道听途说而言。陛下英睿之性,希世少伦,受其蒙蔽,乃是图绘之故。而臣至京师,请对入觐,亦有一图要呈于陛下御览。乃是白马县中各流民营,布置、陈设之规划,逐日将施之于京畿各县。现被留于殿外,陛下可命人取来一览。”

赵顼一听连忙道,“快去取来。”

一名小黄门立刻小跑着出去,而韩冈低头敛去笑意。

如果他一上来就指责郑侠一个守门官,根本不可能知道白马县中事,那顺序就错了。要先让天子开始自己思考,然后才能攻击对手,否则很容易惹起逆反心理,反而更生怀疑。

赵顼现在则是有些尴尬,因为一幅图,而发了这么大的一场无明火,还让韩冈受了委屈。

蓝元震在白马县看到的一切,都原原本本的上报。赵顼这段时间,一直在关注着白马县的流民安置情况,要不是被流民图一下弄昏了头,也不至于会怀疑韩冈的作为。

干咳了两声,赵顼道:“如今河北南下流民已近十万,到了五六月间,人数还会更多。不知韩卿可有把握,使其不至为乱?”

‘成了!’韩冈终于心中大定,赵顼对他的话已经信了八九分,否则不至于有此问。他微一欠身:“以黄河之汹汹,不破堤,不为患。流民虽众,若安抚得宜,亦不至为乱。必不致使陛下烦忧。”

“旱情不过七八个月,怎么就至于如此。”赵顼很是疲累扶着额头,不管怎么说这场旱灾的确造成了大批的流民,而赵顼也不免怀疑其来是不是德政不施的缘故,所以郑侠的流民图才能惹起他这么大的一场火气,那仅仅是一根引线而已,火药早就在赵顼心中积存了起来:“禹水九年,汤旱七年,而民无饥色,道无乞人。朕怎么连十分之一都做不到?”

韩冈瞥了一眼王安石,开口道:“乃是天灾过甚,新法行之日短之故。”

对于韩冈,赵顼不需顾及太多:“三年耕而余一年之食,九年耕有三年之储。自便民、免役诸法施行于世,至今已有五载……”

“三代之时,以井田授土,人皆有土地,出产自有预留。”韩冈回道,“如今之世,富贵之门,拥田不啻千顷;而贫者无立锥之地,日夜辛劳,方得一饱。故而富者坐安于室,不事稼樯,收租取息,一年即有三年之积。而贫者日常所得仅能果腹,何谈积蓄防灾?如今流民,率为贫户,岂有拥百顷之田而亡命于道者哉?!”

