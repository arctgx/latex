\section{第26章 惶惶寒鸦啄且嚎(下)}

与吕和卿的密谈结束后,用了两天的时间,龚原终于可以坐下来喘口气了。

国子监中的学生,从十几岁到三四十的都有,但主要还是二十出头的为多。

学生们寓居地多在民家、僧院,主人家被骚扰到的不在少数。而且国子监生们对主张气学的韩冈,一直以来都有一股怨气。

解试要加考《幼学琼林》的自然部,那绝不是要多读一本书的问题,要是当真以为这么简单,那简直跟猪一样蠢,十几年的书就白读了。肯定是要把气学相关的内容,都要融会贯通,否则随便出上一题什么池塘四角四棵树的问题,或是气压与高度的关系,又或是速度和加速度的题目,那就都要抓瞎了。

而且到了策论的时候,到底该采用哪一派的观点,更是让人愁。考官的身份,即有气学出身,也有新学出身,没有标准答案的考题,

百分制的考试,事先划定了得分点,气学的部分,至少要占二十分,国子监的学生差不多两千五,而能拿到贡生资格的考生只有百人。只要差上一分,就要落下几十名。被逼得要去学习气学,学生们的怨气自然免不了。

这样的一群年轻气盛的读书人,又是有着治国平天下的宏愿,更对气学有成见,要在里面煽动起三五百人来,就实在太容易了。

龚原从几名过去曾经对气学多有抨击的学生身上着手,并没有花费太多时间,就让他们激动起来。他这两日大部分的精力,其实是放在对自身的保护上——扇摇学子的罪名,聪明如他,当然尽可能的不沾在身上。

龚原当然知道吕和卿对他的唆使没安好心,就算当时还有地方没有想明白,事后回想起来也完全想通了。但韩冈视其为仇雠,章惇又将他拒之门外,不去找吕惠卿,难道要坐等被秋后算帐不成?

既然投了吕惠卿,冲锋上阵是情理中事。

但让龚原下定决心,按照吕和卿的说法去做,最关键的一点,是距离亲政已经为时不远的官家。

“又在闹事什么?”

毕渐起身望着不远处喧闹的庭院,只能模模糊糊听到有人在哪里高声说着什么。

秦观摇摇头,“都是在找死的。”

秦观知道那边在闹什么,但他不关心。他现在只希望自己能够考中进士。

在苏轼之后,他就成了惊弓之鸟。但出身南唐将门的他,至今还抱着一展长才的梦想,故而投身气学门墙。两改《蚕书》,三次投稿,花费的心血,比作出一篇千古绝唱都难。

但龚原不觉得自己是找死。在他看来,最多也只是蛰伏一阵,等到天子登基,他就能咸鱼翻身。

煽动学子,龚原当然不会将自己的亲戚给拉出来,购买贼赃的事,有口供、有人证、还有物证。相对有些身家的商人,普通百姓在追捕丐贼的过程中,受到的骚扰更多。

韩冈要抓乞丐实边,出首又有奖赏,五十里城墙内的乞丐都给抓绝了,莫说乞丐,就是流民都给抓走了。在这过程中,京城的百姓被骚扰得不轻。闯门的人手上拿着诏令,没一个进士出身,有几个敢强硬的?

女眷被骚扰的报告有十几起,还有两家户主被打成重伤,这都是事后传出来的消息。等到传到学生之中,所有的数字就翻了十倍,重伤也变成了被打死。

‘自太祖定鼎以来,未见京师有此之乱。’

‘京师震恐,百姓惊怖。’

‘宰相篡权,民间只知有宰相,不知有天子。’

‘十年之后,北虏之乱,恐现于中国。’

这一干的危险舆论,开始从国子监中,慢慢的向京师传播。

吕和卿安坐在城南驿中,听着官员们的议论。

心中自有几分得意,韩冈或许能够度过这一关,但灰头土脸是少不了。

虽然结果只能期待以后,但现在能出上一口气,也是件美事。

……………………

韩冈早早的便得到了消息,处理政事之余,抽了个空对宗泽道,“汝霖,你跟两家报社有交情吧?”

宗泽点头,“在监中读书时,下官还是靠了给两家报社撰文才得温饱。”

当年辽人入寇河东,宗泽用了两个笔名,为两家报社分别撰写河东军情分析,两头赚钱。尽管他这么做的不算地道,但跟两边的编辑部都有着不错的交情,在他中了状元之后,这份交情也顺理成章的更加深厚起来。

“汝霖你家中不是行商,怎么会连温饱都做不到?”

宗泽道:“居京师,大不易。”

韩冈呵呵笑了起来,宗泽和白居易虽不是同类才子,但同样能够在京师活得很好。

笑罢,对宗泽道:“有空的话,去两边帮帮忙吧。”

宗泽眼中闪着精明:“怎么一个章程?请相公吩咐。”

“依法行事就够了。”

宗泽心领神会:“宗泽明白了。”

宗泽领命出门,韩冈又提起笔,开始批复公文。

这点小事,不值多费心神。

……………………

“坏了,坏了。”

刚刚从开封府狱中被放出来的商人早起后,刚刚拿起报纸,便大叫起来。

“老爷,怎么了?”

“这,这是疯了吗?”他把报纸一丢,“简直是疯了,跟韩相公打擂台,这捡了便宜还卖乖,当初就不该求到他身上。”

匆匆忙忙的换了衣服,叫了车马,用最快的速度前往龚家。

但当他来到巷口,却发现一群如狼似虎的衙役正一脚踹开龚府的大门。

这商人一屁股做到了地上,“这下真的完了。”

……………………

两家报社,在头版头条,刊登了有关丐贼之案的最新新闻。

三千三百九十四。

一万又八百一十一。

四百六十九。

一百零三。

一千两百零七。

蹴鞠快报上,很有特色的列出了五个数字。

京城及开封、祥符两赤县,抓捕丐贼共计三千三百九十四人。

含勒索在内的案件,总计一万又八百一十一件,其中劫杀要案四百六十九。

为了断案,大理寺、审刑院、开封府的三厅两院,以及京畿各县的县尉、典史、刑曹孔目官,全部汇聚京师,总计一千两百多人的庞大审判团。

几个衙门每日灯火不歇,日以继夜的审案。

近十天来,席卷京师的风暴,最后定下来的大辟名单多达一百零三人,凌迟、腰斩等重法要犯,共计的十五人。而官府所捕乞丐中,涉案者居其半数。

报纸上只是将几个数字这么一罗列,不用说什么扰民了,任谁都知道,放任那些贼子不管才是害民。

‘一万八百件案子,十天不到就全审完了?’吕和卿把驿馆中茶杯也砸了,他实在没想到韩冈竟敢如此不要脸。

十年审结一万多件案子,这的确是笑话,其中当然大有情弊。

国子监中的谣言,不过是将受害人数扩大了十倍,而韩冈这边,却立刻将不知多少无头公案,全部栽到了这三千多丐贼的手中,尤其是那一百多个被勾决的名单上。心黑皮厚,让人望尘莫及。

一口气解决了那么多案子,开封府上下不仅仅可以轻松许多,而是上上下下都能授奖受赏。实在是可喜可贺。

而在无知的民众眼中,一千两百名法官去审一万八百件案子,平均每人才九件而已。

而且世人有几个会认真去分析数字的真伪?大多数人还是为下面案件的细节报道所吸引。

有个十岁出头的宗女上元节随家人出门去看灯,从此一去不归。现在查到她下落的时候,已经在城外的乱葬岗里了。

类似的事屡见不鲜,可登在报纸上,却足够耸人听闻,也更能引动人心。多少百姓家里的孩子遗失,普通人即是没有亲身体验,在他的附近,也肯定出现过。

历任开封知府都想要处置这些败坏京师治安的贼子,只是不能根除。

现在朝廷做了,百姓如何不拥护,韩冈的声望也顿时又涨了一截。

‘对丐贼所涉诸案,须从重,从快,不论牵连何人,一并查处到底,让京师百姓从此能得以安寝。’

太后如此批示,也让她的名望更加高涨。

几日后,赛马快报上又出现了一则后续报道,国子监学官龚某,以情害法,关说有司,收赃奸商因而得以逍遥法外,今有御史上表弹劾,龚某被拘入台狱。

‘没有收赃的奸商,就不会有窃盗。这话说的没错啊,那些贼人,要不是有人帮他们销赃,怎么可能去行劫盗之事。’

‘要说该死,奸商也算一份,那帮奸商说话的赃官,也一样要重重处置。’

已经结束了。

韩冈将报纸折好放下。

只看报道,韩冈就能想象得到,民间会对龚原是什么样的看法。

这一次的纷争,不是简单的朝廷对清议的斗争,更包含了舆论权归属的问题。

一个是自汉代以来,便掌控士林舆论的太学生团体,一个则是新兴的商业传媒集团。

双方对阵,究竟哪个能取得胜利,如果不是牵连到自己,韩冈倒是很乐意在旁边看好戏,顺便推波助澜。这样对开启民智好处更多。但现在的这个情况,他就算不愿意,也得掺合进来。

若是是几十年后,国子监生们多半能赢,毕竟办报的鱼龙混杂,与朝廷太贴近的话,也会启人疑窦,很容易丧失公信力。不过现在的报纸是新生事物,且两家快报都贴近民生,深得百姓喜爱,相对而言,国子监生们就太曲高和寡。

不过,决定胜负的还是舆论背后的那只黑手。权力在握,又怎么可能会输跟一群只有嘴皮子的书生?

一封弹章上抵御案,龚原随即铛锒入狱,拘入台狱中待勘。国子监的骚动,立刻烟消云散。

台谏本非一体,纵使龚原,或者说他背后的吕惠卿唆使了几个人,可无论是章惇,还是韩冈,夹带里还是有几个听话的御史。

龚原所做的事,在官场上太普遍了。但要因此去定他的罪,却也不是什么难事。

就像写诗一样,欲加之罪,何患无辞?只要有心去找,总能在诗里找到那么犯忌的词句。

所以才了赛马快报上的那一篇学官龚某以情害法,关说有司的报道。

很快,御史台又查明其煽惑学子的行迹,向太后因此大怒。煽动人心,这本就是朝廷最为忌讳的重罪。不过因为王安石为其举主,故而留了他一条性命,追夺出身以来文字,被送去了云南种地。

而收了龚原的信,徇私枉法的军巡院都巡检,则是因其在丐贼一案上颇有功勋,又是为龚原所蒙蔽,故而不加重惩,并准其将功赎过,最后只是罚铜了事。其中种种,明眼人自然看得清楚。

陛辞之后,吕和卿惶惶出了京师,他确认了章惇的倾向,也确认了韩冈的势力,现在他确认了一点,在天子亲政之前,眼下朝堂的局面,将无人能够动摇。

“就放他一马好了。”韩冈对章惇道,两人并肩走在皇城中,“跳梁小丑,不足挂齿,子厚兄还有更重要的事要做,不是吗?”
