\section{第40章 岁物皆新期时英(六)}

一番梳洗,换了身宽松的衣袍在房中坐了下来。

慢慢的品着略带酸味的醒酒汤,韩冈突然叹了一句:“章子厚家的酒可不好喝。”

云娘正在韩冈身边,端了醒酒汤过来后就帮着磨墨。听到他说话,就问道:“三哥哥在章枢密家喝的什么酒?”

“交州的甘蔗糖蜜酿的酒。”韩冈道,“家里的白糖作坊剩下的糖蜜,也都给他们家了。每年有不少船从交州运出来。”

“都没听说过。”云娘气鼓鼓的说道,“既然拿了我们家的糖蜜,怎么也不见送几坛过来?”

韩冈不禁一笑,“酒药不行,弄出来的烧酒味道太杂,章子厚自家都不吃。全都在交州。”

韩冈到现在也很遗憾,朗姆酒的原材料有了,酿造技术也不差,就是弄不出好酒来。只是偶尔成功一两次,下一次就又是失败。这其中只能怪酿酒的酵母不行了。酿酒的酵母是有讲究的,品种不对就酿不好酒。要不然官府也不能通过控制酒药、窖池,来实行酒禁。

“不是有酒禁吗?”云娘好奇的问道。韩冈所创的烧刀子天下闻名,但韩家酿酒,却只够送亲朋好友的,剩下的就是香水的原料。一船船的往外运,那是怎么都不敢想的。

“酒禁归酒禁,也不是没有变通的办法。”

朝廷力行酒禁,不过对于官员和贵胄们自家酿酒自家吃,就不会去管了,小批量的赠送甚至外售也是睁一只眼闭一只眼。但章家在交州的酒场规模就太大了,最后是借了黄金满的地皮,在他儿子的辖地中酿造,直接就在当地给卖了。那些海商买过来当做压仓货,运到一些专司回易的私港中,转眼就能卖光。福建、浙南,那边的小酒馆里面,都是交州的糖蜜酒。

听了韩冈的解释,云娘点点头,“不过三哥哥方才说的不是这糖蜜酒吧?”

“嗯。”韩冈应声。

在自己面前轻言浅笑的云娘,早是几个孩子的母亲,不是当年一句话就能糊弄的小丫头了。

“什么糖蜜酒?”

严素心跨进门来,亲自用托盘端了一碗馉饳儿。半开着盖子散着热气,隔着老远就能问道香味了。

韩冈笑了一声,没有作答。

“是说三哥哥方才去章枢密家吃的酒。官人说章枢密家的酒不好喝。”

“酒不好喝?……筵无好筵,会无好会,当然不好喝。”素心皱了皱鼻子,拉起云娘出门,“云娘,我们先出去吧,不打扰官人。”

韩冈拿起调羹,盯着漂浮起来的腾腾热气,却没有动手。

蔡确要对御史台动手了。

在司马光入京的那一次,其实御史台已经给清理过了一遍。但蔡确看来,打扫得肯定并不干净,还有很多地方需要再一次的清理,以便能更好的控制台谏。

只是不知蔡确会怎么处置御史中丞?是一并解决,还是援引李清臣入东府?

韩冈想来想去,当还是连根铲除的居多。

三相两参,宰相位上还有一个空缺,但参知政事的空额却没了。张璪、曾布,他能将谁给踢下来?

放到西府……也要章惇肯答应!

苏颂虽说是顶替韩冈的位置,年纪已长,不会争权。章惇乐得有这一位做同僚。而让一个刚交五旬的李清臣进来,嫌西府不够乱吗?那一位可是与安阳韩家有亲。

只能请出去。

对付御史台,蔡确和章惇看起来已经达成了默契,蔡确想要独相,而章惇暂时没有转去东府的想法,一东一西,联手掌控朝局。

宰辅之争,看的就是谁更能影响御史台。宰辅想要掌控朝局,第一个就要控制住御史台。

一切的关键都在乌台上,那是皇帝用来压制相权的工具,一旦落入宰相之手,就是皇燕京有被架空的可能。

而不论是哪一位宰辅控制住乌台,两府中的其他人,立刻就低下一头去。他们本人纵不惧,可如果下面的门人都一个个被御史干掉,在朝堂上也不会有任何影响力了。

曾布和张璪恐怕都想不到蔡确这么快就会动手,棋差一招,便是缚手缚脚。

韩冈也挺意外。现在想来,肯定是蔡确已经在韩绛那边得到准信,才会选择现在动手。

韩绛已经七十岁了。依例官员七十而致仕——最迟七十岁就要退休了。除非有那个必要,需要这位老臣镇压朝局,否则朝廷一般不会破例,而朝中物议也会逼迫他主动辞职,御史台等闲更是不会放过。

曾孝宽之父曾公亮,也就是主编《武经总要》的那一位。熙宁初年已年过七十,仍在朝堂为首相,这是因为他支持新法,所以天子希望他留下来,御史台故此也没有弹劾他。但立刻就有人写诗道‘老凤池畔蹲不去,饿乌台前噤无声’,逼着曾公亮自请致仕。

看韩绛现在闲云野鹤的作派,当也不会久留。而韩维、韩缜也都六十多岁,在朝堂上没几年了。韩家的子侄,未来十几年,说不定都要靠蔡确、章惇他们关照,没必要逗留太久,以至于惹人议论。

清除任何一个有可能上位的潜在对手,这应该是蔡确想要做的。而短时间没有进入东府想法的章惇,就是最好的盟友。

只是蔡确真的这么信任章惇?而章惇当真就对宰相之位没有兴趣?

应该不可能。可如果章惇真的想要宣麻拜相,必然要利用任何可以利用的力量,包括借重自己的影响力。正是因为章惇暂时没有那个想法,才会摆出了与蔡确合作的态度。但章惇不可能白白与蔡确合作,不知蔡确拿了什么与他作交换。

韩冈猜测了一下,确定不了究竟是什么条件,打算改天问问章惇去。

究竟是怎么回事,韩冈很快就抛到了脑后,由得他们闹去,他现在要做的事很多,没精力多分心。要有闲空在朝堂上争锋,就不会辞去枢密副使的差事。

李诫来了信。

韩冈一边吃着类似于后世馄饨的馉饳儿,一边看着信。

李诫眼下正在河东去修轨道。从三交口往北一直到忻州,那一段在两山之间的谷地中,还要越过一重关隘,算是最难修的一段。不过以现有的技术,还是能够解决问题。

只是究竟是穿过石岭关还是绕道赤塘关这件事上,李诫与其他人有了异议。

从太原往代州,赤塘关道路远比石岭关好走。可如果要说那条道比较近,肯定是石岭关。

从太原北上忻州,一条路向北穿过石岭关就是了。但石岭关山势峻险,进出关口坡陡弯急,轨道想要走石岭关,需要凿山开道,不是做不来,可至少要花费更多的时间和人工。而向西绕道赤塘关,一来一去多了七八十里路,但除此之外就没有太大的阻碍。穿过十余里的山谷,再转而向东,就石岭关背后了。

开国之初官军伐北,久攻晋阳【太原府】不下,便移兵石岭关,意图阻断北方的契丹援兵,孤立晋阳城。但石岭关关险难克,便又分兵先攻赤塘关,绕道石岭关后,前后夹攻,方才攻克。这一回宋辽大战,如果不是萧十三主动撤出两关,韩冈的主攻目标也只会是赤塘关。

绕路是没什么,避开了困难的地形也方便修路。可八十里的弯路就是多八十里的成本,维修、运作都大大多于直接穿过石岭关。李诫主张凿山开道,走石岭关,而其他人则是想按照先期制定的计划,绕道赤塘关。

韩冈从心理上支持李诫,只是并代铁路若迟迟不能通车,河东的经济复苏将会遥遥无期。

而且西府中薛向年纪也不小了,正在争取近年内将泗州通向京城的轨道给铺设出来。京泗铁路,他仿照河东的命名法,连这一条轨道名字都给预定好了。只等着并代铁路完工,估计他是没耐心多等。

迟一天通车,朝廷就多损失一份收入,李诫想要凿通石岭关的计划,韩冈也无法支持。

‘什么时候能有蒸汽机就好了。’韩冈不是第一次这么想了。每次看到骡马拉着车厢行走在轨道上,都会涌起同样的想法。如果有蒸汽机车的话,走石岭关或许不一定要凿通关隘。

几口吃完了夜宵,韩冈从书桌旁的多宝格上,拿下了一支长条形的木匣。

木匣中放着一卷卷轴,张开来却是一张绘制得十分精细的图标,乍看去像是升官图,但又有很大的区别。图上的线条是树状的,分叉很多,类似于但现在已经有了很大名气的生物树,却又有很多地方聚合为一。

这不是什么升官图和生物树,而是韩冈私下里绘制的科技发展路线图。

韩冈在图上做的标记,除了他本人以外没人看得懂,上面的符号让整张图看起来就像是某个与算学有关的图表。就像新一期即将刊印的《自然》之中绘出图来的贾宪三角。

这一期《自然》中的数算一章,将会有对贾宪的介绍。同时还有将贾宪的增乘开方法和增乘方求廉法,用现在逐渐规范起来的数学符号和数学语言,重新加以解释和阐述。

司天监中不是没有人才,在算学上,有很多能力出众的伎术官。贾宪就是其中之一。他的《黄帝九章算经细草》在世间流传很广,程序化的高次幂开方法让多少数学家受益匪浅,只是本人早就不在人世,这让韩冈觉得十分遗憾。若是贾宪仍在人世,他的科技发展路线图上能填上的空缺会多上许多。

蒸汽机在路线图上排在了很后面,之前有锅炉、抽水机,活塞,齿轮,轴承,还有压力计,阀门。韩冈对蒸汽机的发展了解得很肤浅,也只能列出这些发明创造,而且并不确定那些可以省略,哪些是绝对必要。相对于诸多前置科技,蒸汽机的理论却很容易阐明。

他已经在写一篇有关动力的文章。将人力、畜力、风力、水力都拿出来一一分析,然后对蒸汽动力进行详细阐述,并加以鼓吹。其中水气化为蒸汽由此产生动力,跟火器的原理,也有一定共通的地方,在文章中韩冈就拿了出来进行说明,并阐明区别。

看起来这只是理论上的分析,但实际上,就是在说明火器和蒸汽机,要什么方向上去改进和发明。明眼人都能看得出来。

这一桩桩、一件件,都是要事先打好的基础。没有根基,就是无土之木,离水之鱼,只有在民间形成风潮,有了足够的受益者,这样等自己重新回到两府,就能有足够的底气去推动真正的发展。

韩冈重新卷起了图表,收好,放在一边。

久久一声轻叹,还是需要耐心和时间!

