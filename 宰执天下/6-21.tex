\section{第四章 力可回天安禁钟(下)}

笃的一声轻响。

一支长不及尺的木羽短矢疾飞而来,只在韩冈眼底留下一抹残像,正正的扎进了章敦脚下,深深的钉进了石缝中。

韩冈的背上噌的出了一身白毛汗,再看章敦,也是脸色青白。

藏在人群中的弩手被撞到的时候,正好扣下了牙发,但他原本可能命中张守约或郭逵的弩箭,却吃这一撞飞到了章敦的脚下。如果当时那张神臂弓上的箭簇再抬高一寸,飞到台阶上,就是在章敦的心窝里了。

韩冈深吸一口气,平复了动摇的情绪,“宫中什么时候给藏重弩!?”

皇城才多大,如果拿着神臂弓站在皇城城墙上,说不定一箭就能飞去福宁宫。若目标是靠近西南两侧城墙的两府,瞄准哪位宰相都没问题。

禁卫中又不是没出过叛贼,谁敢将远程的兵器交给他们?寻常守护皇城,也用不到弓弩。即便是御龙弩直、御龙弓箭直,招箭班、弩手班,班直们手中有弓有弩,但他们的箭袋里的箭矢也只有三五支,还都是装饰品。

“是怕宫里面的人跑出去吧,拿着弓弩是比拿着刀剑方便。”章敦表情冷硬,“石得一准备得还真周到。”

皇城司手中的弓弩已经开始射击了,都在瞄准张守约和郭逵。就算是再不通兵法,也知道擒贼先擒王的道理。

但身处一窝蜂的人群中,弓弩手们的射击都被干扰到了。石得一在军阵指挥上的无能,让皇城司叛军的优势完全发挥不出来。只有一半能对准目标,但即使是准确的射出去,也只是命中了挡在郭、张二人之前的班直。

可班直们多有被近在咫尺的神臂弓吓到的,不由自主的便开始后退,让前方的阵线一下子便千疮百孔起来。

“幸好不会多!”

皇城司私藏神臂弓,一旦给查实,石得一逃不了罪名,而且是谋图不轨的重罪。在过去,他完全没必要做那等蠢事,也不会给送宫内的敌人这么大的把柄。石得一搜罗神臂弓,必然是在决定叛乱之后。而其目的,就像是章敦所说,更多的当是防备封锁皇城后,有人偷偷溜出去通报其他宰辅——一张弓、一柄弩,所能控制的范围,自是要比刀枪要大得多。

如此仓促的搜集,数量必然不多,现在又是问询后仓促赶来,更不可能将分散到皇城各处的弩弓给一起带来。只是再少,也有二三十张之多。

这不是韩冈的推测,而是亲眼看见拖在后面的皇城司叛军主力已经追了上来,韩冈只扫了一眼,就在其中发现了五六张弓,两柄重弩。推测总共有二三十张,已经是往少里算了。

但张守约在弩箭的威胁下毫无顾忌,踹着几名畏缩起来的班直屁股,一边毫不客气的大骂着。他没有穿着甲胄,却比全身覆甲的班直更为蔑视那一支支飞来的长箭。

“没事的!只是皇城司。”

章敦紧紧盯着台阶下的战况,跟韩冈一样,明知流箭的危险,但未避免动摇军心,便没有退避半步。

“当然!”韩冈很肯定的说着。

只要郭逵和张守约不出事,区区二三十柄只能射击一两次的弓弩,又能济得了什么事?

石得一和一群城门兵,当真能胜过有郭逵和张守约指挥的班直?

神臂弓虽强,但这边也有重甲。而其他方面皇城司的一方就差太多了。

“贼人的弓弩不会多!有老夫和郭枢密在这里,他们瞄准的也不会是你们……你们还不配!就是射中还有盔甲挡着!冲过去,让他们射不出来!是想守一辈子的殿门吗?!”

张守约连踢带骂,“老夫在西面就没见过你们这样的废物!有功劳都要放着,有官做都要躲着,天上掉饼都不知道捡一下,节~度~使~啊!杨明,郑万,你们想日后后悔一辈子吗!”

在老将的呵斥下,班直们手中的刀剑和骨朵,终于克服了恐惧,又开始发挥作用。

在张守约的指挥下,几名班直禁卫中的军官领头,连续两个反冲,硬是砍杀击伤了最前面一排的叛军,将战线给压了回去。

一名没有穿甲的武将,正是方才随郭逵和张守约出来的将领之一,他现在不知从哪里钻出来,脱掉了碍手碍脚的官袍,冲在了最前面,一手提一只骨朵,见人便挥砸过去。其勇力过人,叛军中竟无一合之将,手起锤落,血光四溅,惨叫连声不绝。

在他的带动下,几名班直便紧随而上,如利刃般将敌众一划而开,直奔石得一而去。

叛军中的弓弩手乱了手脚,石得一同样是在封官悬赏,呵斥叫骂,但他完全不懂如何临阵指挥。他手下的弓弩手还在射击,可石得一却没告诉他们是该继续瞄准张守约和郭逵,还是射击已经杀到面前的班直。

但意外总是发生在安心之后。瞄向张守约和郭逵的弓弩比之前少了大半,但一支长箭越过了前方的几重铁甲,正中了张守约的胸口。

张守约的倒下,让叛军一阵欢呼,石得一也更为兴奋,振臂高呼,反击陡然猛烈起来,攻入敌阵的几人顿时陷入了重围。

韩冈的脸色陡然一变,“子厚兄,你先回去找李信来。”

也不等章敦回应,就要向台阶下走去。

章敦一把拉住韩冈:“玉昆,还有郭仲通!张守约也还在!”

韩冈迟疑的停住脚,只见老将捂着胸口,在敌人面前努力站稳了脚,紧抿着嘴,一点也不让痛楚暴露在人前。

郭逵从身边的班直手中抢过一把刀,甩手一挥便切过一名退得最快的禁卫颈项。

提着沾血的刀,在那名禁卫的惨叫声中,他冷声喝道:“郭逵就在这里,谁敢越过这级台阶,悉斩!队正退了斩队正,都头退了斩都头,郭逵退了就斩郭逵!看看你们是试郭逵的军法,还是上前去博一个封妻荫子?!”

郭逵的脸阴沉着,在军中几十年的积威,硬是压得无人敢于再退一步。

禁卫们一阵嚎叫,硬是反冲了回去。

出手接替了张守约的指挥权,完全不同于张守约的指挥方式,郭逵却同样稳住了军心和战线。

军心一定,在郭逵的指挥下,班直随即便反压了回去,迫得贼军步步后退。而被围在敌军中的那名将领和几名班直,更是不退反进,直向敌阵后杀去。叛军刚刚重新组织起来的十几名弓弩手,便给他们一冲而散。

“杀得好!”章敦大叫。

从指挥千军万马的将帅,重新做回一名都头,张守约和郭逵都明显表现得很不适应,甚至有负盛名。但表现再差,比起对时机的把握,对战局的掌握,还是远在石得一之上。

皇城司的攻势虽猛,也只是仗着一股子蛮勇,张守约的受伤就像是火上浇了一瓢油,但郭逵的出手却是一蓬沙土,直接就将火势给压了下来。

叛军阵脚渐渐散乱,石得一狂躁的大喊大叫,也无力控制他手下的乱象。

“大局已定了。”韩冈对章敦说着。

“放开!你们放手!”

杀鸡一般的尖叫从背后传来。韩冈和章敦侧脸向后看去。

几名武将押着赵颢从殿内出来,后面还跟着王厚,手中挑着一支长戟。

赵颢半边脸肿了起来,唇角带血,显然在擒住他的时候,没有给这位亲王殿下留半点面子。

而另一侧,王厚手中的长戟上,高高挑起了一套衣冠。紫袍金带金鱼袋,还有长脚幞头,那是宰相身上的装束。

赵颢被强押在台阶上,亮相于众人眼前。

章敦指着赵颢的脸,一声大喝,“蔡确已死。齐王已经束手就擒。尔等还不速降?!”

这是压断了骆驼脊梁的最后一根稻草。

望着已成阶下囚的二大王,又见宰相的穿戴,己方败势已成,谁还有心再战?

在士气大振的班直猛攻下,皇城司叛军支撑不住,阵型崩溃,数百人四散而逃。而赏格诱人的石得一更是被几十名班直穷追在身后。

就在大庆殿前的广场中,一追一逃。双方的距离渐渐接近,很快便要赶上。

跟随石得一逃窜的亲信,一刀捅在石得一的腰上,举起腰刀,大声喊:“我杀了石得一。”

但他话声未落,随即便被追兵剁翻在地。

几十名禁卫一拥而上,还没有咽气的石得一被乱刀一阵疯砍,转眼便成了一滩肉泥。

韩冈在后面清清楚楚的看到了这一切,感觉最后就像是闹剧。不过再是闹剧,现在也终归是结束了。

“忘了问一下究竟是他究竟是为什么要叛了。”他叹了一声。

“多问何益?”章敦反问韩冈。

的确不能多问。若是让他攀咬出太多人就不好办了。

但韩冈心中总是挂了一件事,蔡确一向爱投机,但这一回,未免太过果决了。

“蔡确曾与曾布一起劝说太后行废立事,以安人心,太后严辞拒绝了。石得一在其中必是不甘寂寞的。”

“什么?这个消息我怎么没听说?!”

“这就要问玉昆你自己了,为什么会不知道?”

