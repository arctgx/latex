\section{第一章 纵谈犹说旧升平(六)}

【还有一章放在下午两点左右更新,夜里赶不出来了。】

入夜后就开始下雨,不大,绵绵细细的雨丝,正是清明时节沾衣欲湿的杏花春雨。

韩冈起身告辞,吕惠卿略加挽留,便让下人送他出去。

韩冈走后,吕家两兄弟都没有移动,依然坐在偏厅中,只是一时间沉默不语。下人进来收拾灯盏,厅中凝固的气氛,让他动作僵硬的将厅中的蜡烛都换了新的之后,就急急的走了出去,仿佛身后有鬼在追赶。

偏厅的窗棂斜斜的支着,屋外的细雨投不进来,但屋中晕黄摇曳的烛光却映了出去,将院中几株芭蕉的影子打在了院墙上。被微风细雨轻轻摇晃的芭蕉,落在院墙上的黑影却是张牙舞爪,像极了影戏上的妖魔鬼怪。

吕惠卿透过微敞的轩窗,瞅着新近刷过的院墙粉壁上一只只变幻莫定的瞳瞳鬼影,心中暗暗自嘲,方才与韩冈的一席谈就像是这墙上的妖魔鬼怪,只能在影中攒动,丝毫见不得光。不过只要有用于自己,见不得光也无所谓,与魑魅魍魉打交道也是可以的。

吕升卿不知坐了多久,腿脚也有些麻了,始终不见吕惠卿对方才之事的解释,终于忍不住:“韩冈虽非等闲之辈,可兄长备位参政,何须至此?”

吕升卿反应慢,并不代表他的才智差,方才兄长和韩冈赤裸裸的进行利益交换,让吕升卿听了从心底里觉得难堪。他的兄长可是参知政事!

“觉得丢脸?看开了就半点不会了。”吕惠卿浑不在意,他很早就明白了一件事,妥协这个手段在官场上必不可少。

虽然很早就知道韩冈绝不简单,之后也一次次调高对他的评价,但韩冈能如此之快的就走到这一步,吕惠卿也不得不为之惊讶。

尤其是韩冈在军器监中的行事,更是让吕惠卿只能自叹不如。腹有锦绣已可算是最苛刻的评价,他胸中当是有着一番与众不同的天地。通过浮力追源,还有板甲、铁船、飞船这一些已经造出来的,或是还在努力的,一切种种,让吕惠卿明白,在秉持着格物之说的韩冈的双眼中,世间万物都是与常人不一样的。

这样的人物,平起平坐的对待,真的丢脸吗?吕惠卿已经不这么认为了。

当然吕升卿的态度也不奇怪。他与韩冈方才的谈判内容的确过于赤裸裸,仿佛锱铢必争的贩夫走卒,有失士大夫的风度。

但韩冈不是朱余庆,而吕惠卿也不是张籍,该婉转曲言的时候就婉转曲言,该直截了当的时候就直截了当。‘画眉深浅入时无’式的来往交流,在两个重视实际、厌恶纠缠繁琐的官员面前,其实一钱不值。

省去了无聊的宛转赘语,直指本心,这样的交锋其实更为坦率。虽非焚琴煮鹤之辈,可放在两人如今的关系上,所谓的舌华清言、儒门风流也只能雨打风吹去了。

“当年王介甫就没能压得住他,为兄前日也的确是做错了。现在改正过来,绝不是什么丢人的事。”吕惠卿看了看仍是满心不痛快的弟弟,“若是自始至终都将韩冈拒之门外,视之为敌。韩绛、冯京、王珪、吴充他们会怎么想?肯定是以欣喜居多。”

“但韩冈到最后也没有答应!”吕升卿怒冲道,他生气其实也有这个原因,“说了半天手实法,他连头都没有点一下!”

“韩冈难道打算做一辈子孤臣?要想有所发展,就必须要让张载上京讲学,所以是不用担心的。”吕惠卿没再多说,调转话锋:“这一桩谋反案,天子绝对不会让王介甫牵涉进去。但韩冈他作为王介甫的女婿,总不能对此案听之任之。冯当世、吴冲卿之流,也说不定会有些不该有的想法,所以今次也是难得的机会。”

吕升卿听着心头一动,回头向外看了一下,凑近了压低声音问道:“难道这一次能将两人请出去?”

“很难吧……”吕惠卿轻叹一声。坐到参知政事这个位置上仅仅才有半年时间,但已经足以让他迷恋上掌控天下政局的感觉,无时无刻不在考虑着更进一步的控制朝堂,“不过若是没有斧锯,要想拔掉一棵树,不是一下子凭蛮力直接硬来,而是要先一点点的去摇、去晃。”

“那手实法该怎么办……”吕升卿知道,这个法案是让吕惠卿脱离王安石阴影,成为新党核心的关键,而不是像如今,依然还是受着远在江宁的那一位的庇荫。

“这就要放在最后了。”吕惠卿陡然变得轻微起来的声音,似乎在说着心底的无奈。

如果换个情况,比如冯京被赶出京城;王珪老老实实的做壁挂;韩绛虽为首相,却依然无法控制朝政;那么吕惠卿说不定就会设法让王安石一辈子回不来,由他吕惠卿一直将变法大局给掌控下去。

但现实的情况让他不会也不能滋生与王安石为敌的想法。冯京、王珪甚至吴充都不甘寂寞,韩绛尽管暂敛锋芒,但也绝不会甘于平淡。眼下的局面中,吕惠卿必然要维护王安石这面新党赤帜不倒,以维护自己坐在政事堂中这个位置的稳固。

“手实法还要放一放,政事堂中不靖,就不能推行。”

吕惠卿说着。前段时间,他的确有些自负了,毕竟是跟王介甫斗了数年的人物,要想抓住他们的把柄,不是那么容易。但提前制定手实法的预案不能算错,只要冯京一去,就可立刻推行天下。

……………………

快到家的时候,雨水忽而转急,原本如丝如雾、轻微得几乎感觉不到的细雨,哗哗的打在青石板铺起的路面上,让前面的道路变得模糊起来。

不过韩冈家门前的这一条略嫌僻静的巷道,每家的门户之前,都会在入夜后挂上两盏灯笼,用来照明。一盏盏青纱灯笼中的烛光,穿透了雨雾,映照着夜色,散射处一圈圈同心的光晕。

雨水顺了油布雨衣不断的向下趟着,雨点用力的打在帽上,啪啪的连绵不绝,都能感觉到从高空雨云中直落而下的重量。

春来天象多变,尤其是多雨的清明,官员随行的扈从们都会在马鞍后带着一包油布衣,在骑马时穿上好用来遮风挡雨,而不像普通百姓只穿着蓑衣。

不过旧时的油布衣遮风挡雨的效率并不高,所以韩冈早在秦州的时候,就提了一句,并模仿后世雨衣和雨披的式样,各做了几件。也不知是怎么传播的出来,如今连京城中贩卖的油布衣,也全都改成后世的式样。只是现在恐怕也不会有人知道,这是他判军器监的韩舍人随口一句的结果。

一队或披着雨披,或身着雨衣的骑手,转进韩家家门前的巷道。

望着前端,隔着一段就有一团晕光的小巷,一行人就将缰绳轻提,减缓了速度。

就算在白天,都是慵懒而宁静的街巷,入夜之后,更是变得寂静无比。钉了蹄铁的马掌,踏在青石板上,传出清脆的声响。只是蹄声也不再那么急促,仿佛散步一般的慢了下来,哒哒…哒哒的响着,不会惊扰到邻居。

巷中东头第四家,就是韩家。整条街巷,也就只有六户人家。虽然比不上一户就能占了半个坊的豪门大宅,但占地其实已经不算很小了。远比一条两三百步长的小街上,挤进上百户人家要宽敞得多。

韩冈在家门前跳下马,两个司阍的家丁正跑过来牵马,就看见一个纤巧的身影从小门处钻了出来。

“云娘,怎么出来了?”

“三哥哥你都这辰光都不回来,三个姐姐都急得很,奴奴就出来看看。”

都快十八岁了,但几年来,一直都备受韩冈宠爱的云娘,还是一幅娇痴的模样。春夜依然清寒,下了雨后就更感觉着冷。韩云娘小小的身影披着连帽斗篷,将身子裹得紧紧的,只有几缕秀发调皮的从抛出来,

“去了吕吉甫的府上,没人回来通知吗?”

韩冈一边说着,一边就在门下脱下了身上的雨披,后面的伴当忙将一柄精巧的油纸伞递到他手里,张开来打着向家里走。韩冈喜欢自己打伞,这个习惯,在此时的官员中算是另类。背地里有人嘲笑过,不过韩冈安之若素,还当众说过,等日后升了执政,有了清凉伞,再让人张着不迟。

韩云娘与她的三哥哥挤在一把伞下,踮着脚穿过空旷的前院。仰起头,就只能看到宽厚坚实的肩膊。不高兴的嘟起嘴:“哪里有?姐姐都派了人去军器监问!”

韩冈回头看看跟着自己一起牵着马进来的八名伴当,这几位都是一脸无辜的望了过来。

叹了口气,摇摇头。带到京城来在家里奔走的仆役,其中几个心思灵活的都被韩冈安插进了军器监里做吏员。而现在跟在韩冈身边的伴当,个个老实听话,且忠心耿耿。只是就没一个聪明伶俐到提醒韩冈一下,派人通知家里。

“是我一时忘了。”

“那南娘姐姐的生日有没有忘?”

“……当然没有!”韩冈难得有点慌张的说道。

