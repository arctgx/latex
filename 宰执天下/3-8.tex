\section{第三章 参商稻粱计(下)}

对于第一次参加地方举试的士子们来说,解试,就是他们踏上官员之路的第一道关口。拿起笔时,总有些心惊胆战,生怕有哪里错漏。

平日里只是读书,哪里有挑战这等事关命运的关口的经验?往往就会不知所措,脑袋里的文字,全都不翼而飞。许多士子,都是经过了几次考试之后,有了足够的经验,能在考场上充分发挥自己的实力,这样才考上了一个贡生。

但对于韩冈而言,他经历得已经太多了。生死线上都走了几个来回,这点小场面根本算不得什么。

何况他还有援军——尽管没有事先沟通过。

这一场考试,主考官蔡延庆是个关键,他掌握着韩冈今次考试的结果。如果蔡延庆前面见了他,情况反而危险。没有见面,就足见今次的主考官有着避嫌的心思——如果在取了韩冈,被人揭发两人考前见过面,不管其他考生有没有被蔡延庆接见过,那就是黄泥落到裤裆里,怎么都说不清了。

现在韩冈就能确信,蔡延庆不会在今次的考试中跟自家过不去。

而且他韩冈的身份其实就已经确定了,只要蔡延庆不糊涂,就不会故意使绊子。更要压制蔡曚,省得被连累到。只要蔡延庆这个主考不使坏,在秦凤路这个偏僻之地的一次宽松已极的小考中,取得前四名的成绩,韩冈还是有着足够的自信。

眼前的这份考卷的难度,对韩冈来说并不算高。为他量身定做的策问就不用提了,那是十道经义,虽然是有难有易,但难的题目都在论语等韩冈较为熟悉的经典上。而他感到棘手的易经,题目却是‘飞龙在天,利见大人’之类的段落。

韩冈对经义浸淫颇深,甚至完全放弃了诗赋之道。而不像其他士子,就算明知今科之后,进士试已经改为经义策问,却还是要兼习诗赋,以免在与其他士人的交流中变成笑柄——慕容武就是如此。但韩冈却是踏上一条路后,就一意精进,全部旁顾,真要算起来,他这三年放在经义上的时间,不见得就比慕容武或是厅中的其他考生,少上多少。

胸有成竹,韩冈动起笔来当然如有神助,一行行端正的蝇头小楷出现在答卷上,没有半点迟钝或磕绊。

就在韩冈开始考试的时候,两个考官都没有留在厅中。要是不经意中看到了考生的试卷,那就有串通作弊的嫌疑。有七八个老成的小吏在里面看着,进来前也检查过是否有夹带。

大约两个时辰后,考生先后交卷,各自离开。而到了第四个时辰,最后一名考生收起了笔。

蔡曚和蔡延庆仍都在候着,到了夜中,一叠重新誊抄好的试卷副本,放到了他们的面前。

“转运、运判,经义的卷子已经誊抄完了。策问的卷子过一阵就送上。”

小吏恭声在两人身前说着。

蔡曚也不跟蔡延庆多话,直接把卷子当先拿过来翻看。他是第一道关口,而蔡延庆是最后拍板的。

经义不同于策问,答案都在书上,考得就是对儒家经典的熟悉情况。十五份卷子,一个时辰不到就已经批完。有的是圈,有的是钩。好的多加几圈,最差的,则是钩掉后,又划上一个叉。排好了自己拟定的名次顺序,蔡曚就将卷子传给了蔡延庆。

蔡延庆接了过去,只翻了几翻,就把其中的一张挑了出来,对蔡曚道:“这一份未免放得太后面了吧?”

蔡曚面现冷笑,蔡延庆果然还是看出来了。但他也无所谓,一切早有准备。随手在卷子上点了两条,都是易经的题目,“转运请看这两条,可是符合先圣之言?”

‘当然不符,因为这是张横渠的一家之言。’

张载在洛阳坐虎皮讲易时,曾经被他的两个表侄夺了位子,没有继续开讲下去。但在易经上,他还是有所发明,钻研颇深。这份卷子上的答案,跟儒家先贤全然不同,但却分明是张载的学说。

蔡延庆当然知道,他还知道这是谁的卷子,“先圣无释义,注解皆是后人所撰。这份卷子虽然别出新意,但未必没有道理。”

“其余被黜落的卷子,他们的答案难道也是未必没有道理?”蔡曚反问着。

蔡曚拿着张载与《五经正义》释义不同的地方来出题,就是为了要确认韩冈的所在,并且将之黜落。与只考策问的殿试不同,在地方解试中,经义的顺位在策问之前。如果经义不过关,策问写得再好也没用。

不过蔡曚并没有将被挑出来的这一份卷子,肆无忌惮的列为最后一名。这份卷子上,除了有关易经的两条外,其他八条其实都没有什么问题。而排在五、六、七位的三人,其实都是对了八条,所以就被列为第八。

蔡延庆不说话,却去翻了翻前面四名的卷子。一看之下,就指着第四名的卷子,“这一句不通吧,怎么能算对?”

要挑刺很容易,就算是十题十中格、被列为第一的卷子,也不是每个字都跟书本上一样。而要在对了九条的第三名和第四名中找出一个毛病,将他们与后面的四名降为一个等级,并不是什么难事。

蔡曚的语气变得深沉起来:“下官觉得这个答案只是略有不同而已,本意还是符合圣人之言。”

蔡延庆摇着头:“还是偏了一点,不能算中格。”他将方才惹起争议的第八位的卷子抽出,放到第四名的位置上,“反倒是这一份,应该放在前面。”

蔡曚没有再争论下去,此时下面誊抄文字的胥吏已经将一叠策问卷子送了上来。

策问的题目是蔡延庆出的,是以河湟为题。在这方面,韩冈自然是当仁不让的专家。写出来鞭辟入里,深刻入骨,而其他十几份卷子,就明显的显得肤浅了许多。

虽然一眼就能看出哪一份是出自韩冈的手笔,但在这份卷子上,蔡曚就不敢将之丢到后面,只能放在第一。差距实在太大了,想做手脚都难。而且前面的经义卷的争执,就已经足以让韩冈和蔡延庆都惹上一身麻烦。

只要考完之后,私下里把蔡延庆将韩冈经义卷的名次上提的情况,模模糊糊的透露出去,没有被取中的考生肯定都会认为自己是被刷落的那一个。

情重关己,被刷落的人必然跳出来闹事。到时候,蔡延庆和韩冈将功名私相授受的罪名,就可以彰之天下——若有人质疑,只要看看蔡延庆出的题目就知道了。

韩冈如今身份地位已经不同旧时,要拦着他很难。但要坏了他的名声,顺便让蔡延庆跌个跟头,蔡曚做起来却是轻而易举。

同时要知道,在御史台中,不是没有胆子大的!

——韩冈究竟有多遭人嫉妒,蔡曚更是再清楚不过。

蔡延庆慢慢的读着眼前的策问。蔡曚的想法他一清二楚,但他才不在乎。他拉了韩冈一把那又怎么样,天子难道会为这点小事而把韩冈的贡生资格给刷掉?

开什么玩笑,韩冈可是功臣!

蔡延庆早想好了前后应对。为了熙河经略使的位置,付出些代价也是应当的。蔡曚把事情给闹大了,对他来说反而是好事。这样一来,韩冈就必须要承他的人情。

作为熙河路实质上的第三号人物,从一开始就跟着王韶,胼手胝足的将大宋最年轻的一个经略安抚使路拉扯起来的韩冈,他在天子面前的发言权绝对不低。

而他蔡延庆,就只要韩冈在御前为自己说上一句话就够了。

策问看完,最后的名次就按照蔡延庆的意思定下了。蔡曚并没有争辩,他就等着发榜后,将流言放出去。

考生们的正卷被拿了过来。接下来,要检查卷子上有无错字、别字,还要确定有无犯杂讳——犯了讳的卷子就会直接黜落,没有容情的余地。

找出第四名的正卷,拆开蒙在上面的厚纸,最右侧被蒙起的考生个人资料一栏,映入两人的眼中。

“慕容武?!”

连蔡延庆都惊得差点要叫起来,‘怎么不是韩冈!?’

蔡曚脸色大变,刷刷刷的连拆十数份,但后面的卷子中,韩冈的名字都没有出现。

蔡曚的手抖了起来,蔡延庆的脸也泛起苦笑。

向前拆看,第三名不是,第二名也不是。

而排在头名的那一份,在姓名一栏中,赫然写着‘韩冈’二字。

蔡曚颤着手,拿起那份卷子,工整的三馆楷书中锋芒内蕴,已是有着大家的风范,想从卷面扣分,却做不到。他又一个一个字的扣着,也找不到一个错字、别字、或是犯杂讳的地方。

转运判官的脸色变得又红又青。

蔡延庆低声轻笑,笑声渐渐的放大,到最后一直笑道快要喘不过气来,“好个韩冈!好个韩冈!……经义、策问竟然皆是第一!这一下,名次该定下了吧?”

……………………

“为什么玉昆你没按着先生的主张答题?!”

就在两位考官批改考卷的同一时刻,正在韩家,与韩冈对答案的慕容武惊问着,声音中有些困惑,更有些不满。

“权变而已。”韩冈答得轻描淡写。

当师长的教导和现实相冲撞时,韩冈可不会如这个时代的士子们那般纠结。在这方面,他依然保持着千年后的作风。

标准答案必须要遵循,即便是自己不认同,即便是错的,但终究还是标准答案。

前生所经历过的几百次考试,让韩冈知道该如何选择。

“凡事有经有权嘛……”他轻松的笑着。

易经过多的经义卷有问题,以河湟为题的策问卷同样有问题。以韩冈的才智,还有事前的心理准备,他当然看得来。但不管出题的人有什么盘算,他只要做好自己的考题就够了。

韩冈只要一个贡生的资格。

如今,他已经到手了。

