\section{第十章 却惭横刀问戎昭(13)}

【白天赶报告,晚上回来码字,说起来算上这一章,今天已经写了一万多字了。】

篝火映红了千百汉番两族战士的脸。

秋日的星空下,欢歌笑语回响在肃州城外的酒泉池旁。一堆堆篝火布满城外的原野,天上的星辰,地面上的篝火,交相辉映,一齐在波光粼粼的池中留下闪烁的的倒影。

王舜臣离开凉州之后,便领军西行,走得并不快。过胭脂山,破删丹城,然后在攻克有两个党项部族盘踞,总计三千兵马驻守的甘州时,花了一些时间。之后搜集军资、休整将卒、发动甘州的吐蕃部族和汉人巨室,同样费了几天的功夫。不过当王舜臣从甘州重新出发的时候,跟在他身边的,已经是三千官军,以及高达七千汉番两族联军。

就这么不慌不忙的稳步前进,王舜臣又顺利的抵达了肃州。万余兵马围城,肃州城中不及千人的留守夏军完全镇压不住局势,当天夜里城中内乱爆发,城内的几家大户打开了城门。次日天明,宋军兵不血刃的进驻了肃州城。

肃州乃汉时酒泉郡,霍去病远逐匈奴,曾驻兵于此。肃州城下有泉,其水若酒,相传乃是霍去病倾酒入泉中。

王舜臣拿下了肃州城后,便遍邀城中汉番两家族长、耆老,于酒泉池边将随身珍藏的数坛烈酒一起倒入泉中,更是搬光了城中的数百坛各色酒水,一并倒了进去。

他就这么在千万人的注视下,在酒泉中用头上金盔舀起一杯:“旧年冠军侯在这酒泉边与将士同饮,今日本将与众儿郎重来旧地,如何能独享美酒,当与尔等同饮此泉!”

这一刻,宛如冠军侯重临人世,数千汉家儿郎当先齐声呼应,争先恐后的舀起泉水,吐蕃士兵也为这狂热所沾染,跟着一起舀水同饮。

王舜臣再一次举起头盔,向着来自于肃州城中的族长、耆老,“且共饮一杯,今日同为宋臣,太平富贵当与尔等同享!”

夜宴就在酒泉边开始。

化入无数佳酿烈酒的泉水,其实依然没有多少酒味。但围着热腾腾的篝火,周围是欢腾笑闹的歌舞,纵然酒泉不醉人,但人已然自醉。

多少吐蕃人围着篝火,跳了一圈舞蹈,满头大汗的回来,拿起一只牛角杯,就在小湖旁舀起清洌的泉水,合着杯中弯月,一饮而尽。而来自秦地的汉家儿郎,更是拿起头盔,在湖中舀起酒泉,高唱着秦腔,与不通言语的同伴共饮。

酒泉畔,王舜臣举杯相邀,与一名名将校士卒,族长、耆老,痛饮酒泉泉水。汉人和蕃人的隔阂,在这一夜也消失无踪,把臂同饮酒泉,宛如兄弟一般。

用银刀削下一片片的烤羊,伴着泉水一齐下肚,王舜臣一声长啸,声震三军,继而放声大笑:“今夜好生痛快!”

笑罢,他跳将起来,高高举起的金盔在他掌中闪闪发亮:“本将已经遣人回去向天子讨酒去了。等到数月后本将打下了瓜州、沙州和玉门关,领军回师,天子的赐酒也该到了。到时候,酒泉池畔,再与诸君痛饮!”

……………………

凉州、甘州、肃州,在东面的战局一时间陷入沉寂的时候,西面传来的消息,则是不断传递着官军节节胜利的喜讯。

“王中正这一回靠着王舜臣又露了脸。偏偏每次他都有这个运气。”

“西贼安置在甘凉的兵马几乎都给调去兴灵,王中正和王舜臣都是捡了便宜。”

“可惜甘凉仅仅是附带而已,比不上银夏,更比不上兴灵。今天在崇政殿上,天子又是没有提到那一路的战况。”

秦凤、熙河联军不敌携胜势而来的党项大军,再快要打过青铜峡的时候,却不得不撤回国中。他那一路最后的封赏,只能寄希望于王舜臣在河西甘凉节节胜利。可也因此,他那一路几乎都要被遗忘了。远远游离于主战场之外,除了偶尔几封捷报,报称官军攻下了甘州、肃州,就是天子都能连着几天不提王中正的名字。

池畔小轩中,蔡确三支手指捏着精致小巧的银杯,投过稀疏的窗棱,望向窗外的风景。

盛夏的气息只剩一点残余。窗外荷塘中,荷花落尽,莲蓬也被摘采一空,仅有一片片或完整、或残缺的荷叶,和几根高高挑出水面的残枝。

已经是秋天了。

战争开始时是初夏,如今则是初秋。持续了一个夏天的战争,如今还在继续着。前半个夏天,战火如荼,官军先胜后败,而后半个夏天,战事则略嫌沉闷,除了不断向西的一支偏师,官军和西贼,都没有太大的动静。

但这样的平静,无法持续太久,当时间进入了秋高马肥的八月,人心的躁动已经如同战鼓声一般响亮。

蔡确把玩着酒盏:“河西的进展,天子没有放在心上。不过韩玉昆巡视代州,雁门便小胜一仗。对上缘边弓箭手,辽人竟也没有占到便宜。天子倒是为此欣喜不已。”

“那是地利的缘故,在山道上,辽人的骑兵施展不开。当年折家在丰州立功,斩了皮室军数百级,也是这个缘故。”

难得有此见识,蔡确很是欣赏的看了坐在对面英俊的青年官员一眼,又叹道:“韩玉昆胆子大,不在乎跟辽人起纷争。可萧禧就在京中,闹到了朝堂上可就让人头疼了。”

“不知韩冈会怎么看徐禧之事。退保银州、夏州是他的提议。如今官军驻守盐州,跟他之前的提议差了许多。”

“韩冈不需要冒险,之前灵州之败已经让他大涨了声望,接下来只要种谔守住银夏,他就彻彻底底赢了。试问韩冈如何会支持吕吉甫?他的心思,天子又怎么会看不出来?”蔡确笑容中带着几分讥讽,“所以吕吉甫会去支持徐禧。若是他说一句稳守银夏,功劳就全是韩冈的。”

蔡京低了低头,拿起酒壶,为蔡确斟酒,并不接话。

“元长如何看待盐州的局势?”

蔡确放下酒杯让蔡京倒酒。在他看来,这个年轻人很是有几分眼色,能力又出众,在厚生司中的工作很出色。虽说是同宗,过去并没有太多的来往。如今投入自家门下,只要在经过几次考验,倒是能当成心腹来倚重。

“守银州、夏州,肯定是要比守盐州容易。西贼想要攻打银州、夏州,从兴灵攻来,有千里之遥,其间还要越过瀚海,艰难可想而知。从粮秣上来计算,最多也只有七八天的时间来攻城。凭党项人的手段,这么短的时间,怎么可能攻得下来,到时候就得退军。一个不好,就是灵州的翻版。”蔡京显而易见的在西事上下了苦功,回答时并没有半点犹豫,“不过攻打盐州,同样有瀚海阻隔,相对于夏州,也仅仅少了两三百里而已,西贼的粮秣的确省一点,但也省得不多,最多也就半个月的时间。差别没有想象中的那么大。”

“而官军这一边,从青岗峡、櫜驼口这条路北上盐州,比起通过无定河的粮道要近得多。櫜驼口本来就是李继迁为了贩售青白池盐而设的榷场,走过这条道路的盐枭不知凡几,道路也修得甚是完备。当初高遵裕的环庆军便是赶在种谔之前,将盐州攻克。粮草由此北上,怎么看也不会有耽搁延误的问题。”

“官军粮草无缺,以逸待劳,西贼又只能设法速战速决,拖延不得。这样一看,吕吉甫冒着风险其实没有想象中的那般大。”

鄜延军退守银州、夏州,缩短了官军粮道的同时,相应的也拉长党项人的补给线,在已成荒墟的盐州、石州、宥州,即便是党项人也无法得到粮草补给,打到夏州城下,最多也只有七八天时间来攻城,而后就必须撤军了。绝对是立于不败之地,这一点,朝中都是公认的。

“但徐德占能不能守住盐州,却还有些难说。”蔡京又补充道,“虽说他正在调集民夫增筑城防,仓促之间,也不可能将盐州打造的固若金汤。”

了解西夏的困境,这一点不足为奇,但蔡京对盐州本身还有了解,就很难得了,许多事不是他这一级、又没有去过陕西的官员能打听到的。蔡确对此算是比较满意:“想不到元长竟对边事如此了解。”

“在下此去北方,说是领队去传授种痘法,不过见大辽的那位尚父,肯定少不了提到边事。”

这一回使辽,为了能安抚下辽国,为了正副使节的人选,朝堂上很是伤透了脑筋。直到最后才决定调回沈括,让他担任正使。副使照规矩应该是选择一名武将,但这一次面临的局势不同,又负有传授种痘法的任务,所以设了两名副使,一文一武,其中文副使就是蔡京。

“说起边事,沈存中当然远远强于在下,又是去过辽国的,一切都熟悉,不会受辽人所欺,说不定还能逼得辽人出乖露丑。到时候,辽人要捡软柿子捏,多半在下这个年轻识浅的副使下手。”蔡京微微一笑,“为朝廷脸面计,西北的兵事只能囫囵吞枣的多记上一点了。”

