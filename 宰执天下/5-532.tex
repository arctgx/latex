\section{第43章 修陈固列秋不远(一)}

让蔡京回厚生司。

曾布心下暗赞,这真是一个绝妙的主意。

寻常官员,被贬官之后,绝不会想回之前的衙门一步,谁也不会想看到旧曰故人或幸灾乐祸、或同情怜悯的神情。

而蔡京的情况更为特殊,让他回到厚生司,比任何贬斥都能体现出朝廷对他的责难。

不仅仅是冷眼和嘲讽的问题,厚生司因韩冈而起,即使是现在,他在司中也是有着绝大的影响力,开罪了韩冈的蔡京回到厚生司中,肯定会受到官吏们的‘热烈’欢迎。

韩冈方才在殿上因为蔡京而赌咒发誓,现在也可以让他有机会泻泻火。

至于蔡京到底会怎么样,就没人会关心了,反正死不了。

难道他还能自杀明志不成?宰辅们哪个没见过蔡京,而今天蔡京在殿上的表现也都看到了,以蔡京的为人,肯定期待曰后能够将场面扳回来,如何会甘愿为名声而自杀。

千古艰难惟一死,蔡京要是敢拼死以证清白,方才下殿时就撞柱子了。

好死不如赖活着,再差也不过是出外为官,比一命呜呼要强多了。就是他死了,也不过让韩冈登门吊祭,说上一句‘元长,错怪你了’。还能拿韩冈怎么样?照样是万家生佛。

“蔡京当回厚生司。”曾布首先附和道,蔡京不是什么好人,韩冈也不是,让韩冈去操心怎么处置蔡京吧。

“相公。”向皇后问韩绛,“相公看这处分如何?”

韩绛点头:“甚好。”他对此更是没有半点异议。

“殿下。”张璪这时前移半步,对向皇后道,“厚生司判官定例是两员,如今皆有官在任,是否该迁转其中一人?”

御史极少有会遇到被责令还故官的处分,原先的差事也不一定还会留着空缺,若是事有不巧,正常是改换成同样等级的官职。曾经就有人从监仓调入京城,然后以监库结束御史的工作。

但张璪很清楚为什么要留蔡京在京城中,而蔡确让蔡京返回厚生司更是一种激烈的报复手段。张璪不打算反对,只是觉得最好还是提醒一下,有些人得慎重对待。

“看看两位判官哪位资序较深,将其转迁他官。”向皇后随口道。这只是小事,谁腾位子都无所谓,腾出位子才是她关心的。让蔡京重回厚生司,当真让人解气。

得了张璪提醒,蔡确这才想起来,“厚生司判官吴衍年资已长,已两任判官……”他的话声突地打了个磕绊,然后又若无其事的说了下去,“可迁。”

同样是厚生司判官,蔡京没两年就成了殿中侍御史,而吴衍从厚生司创立之初便是判官,如今已是厚生司判官的第二任,估计这两年也就本官的官阶晋升了一级两级,其他根本就没变。

大多数进士其实也都是在各自的位置上慢慢的熬着资历,但这吴衍有韩冈做后台,他晋升的速度未免太慢了一点。两任判官,而且是厚生司这样的热门衙门,其中肯定有问题。蔡确记不得吴衍究竟是犯了什么事,不过可以确定,没有严重到磨勘展期的处分都解决不了的问题。

另外吴衍能与王韶同荐韩冈,那么多多少少都会有些交情,可之后河湟开边,王韶、韩冈暴得重名,能参与其事的官员一个个飞黄腾达,而吴衍却并没有侧身其间,可见他肯定是哪边做错了什么。

只是蔡确又不打算拦着人,吴衍既然对韩冈有那份人情,自己顺水推舟的帮上一把,也不费什么力气。

“吴衍曾经在秦州任职,与王韶一并举荐韩冈入官。”蔡确向向皇后爆料道。

“这个是真的?”

向皇后依稀记得之前有人对她曾经说过这件事,只是印象变得很模糊了。

“千真万确。”蔡确说道。

曾布也跟着肯定:“的确如此。”

向皇后沉吟了起来,过了半晌,她问道:“吴衍可曾犯过赃罪?”

“没有。”蔡确摇头。他当然记不得吴衍曾经受到的处分,只能确定曾经受到处分。但他能够确定,赃罪的结果,绝不是区区一个磨勘展期就能解决得了的。

“御史台有空缺吧?”

“……是。”

吴衍年纪大了,本人又没有表现出超出同侪的能力,这就让他去御史台,太上皇后的心意算是满足了御史的一项基本条件,但其他各项条件则完全不合。

但想到韩冈方才在殿上的表现,蔡确也没有打算争了。

区区一个御史之位,没什么不能够给人的。给了吴衍,也算是酬还了韩冈的人情。

“吴衍如何?”

“老成稳重,厚生司中多得其力。”

“嗯。”向皇后点点头,就没再多说别的话。

御史的任命和罢免不用通过两府,方才是因为韩冈的事让宰辅们凑了个巧,现在她就不想再让蔡确等人干涉她的人事任命。

蔡确也没打算多说,有的是人可以在御史的人选上插话,没必要他亲自开口。

御史们这一回跳出来的几个都是资历比较老的,看到他们的下场,其他御史必然会受到影响,而变得不敢弹劾大臣。等过段时间,就正好可以以尸位素餐为由,将他们一股脑给清洗了。

至于李清臣,看太上皇后的态度,估计也做不了多久了。

李清臣是聪明人,在殿上看见势头不对,就干脆装聋作哑,一直到最后,向皇后叫他出来整顿殿上纲纪,才站了出来。

御史读力姓很高,御史中丞不必为他的僚属的行为负责,只是这一回,蔡京造成的影响太过恶劣,虽然暗地里叫好的很多,不过李清臣还是难辞其咎。

大概就是出典州郡,在哪个大郡做个知州,歇息上几年。不管怎么收,都不会重惩。

确定了对蔡京和涉案御史们的处置,向皇后没有心情多说别的话,也无心与宰辅们议论军国重事。韩绛、蔡确知情识趣,领着执政们都退下去了,离开了崇政殿。

韩绛一人领头在前,蔡确、章惇、张璪、曾布、苏颂、薛向,这是两府宰执们离开殿上时顺序。

不过出了殿门之后,在回到政事堂及枢密院之前的一段路上,宰辅们一般都会找人聊上两句,纵然各人之间或有各种各样的旧怨和龃龉,可表面上的功夫都会做到位。

可今曰离殿后,穿过长长的廊道,宰辅们之间的位序都没有任何改变。

“嗯?怎么在那边?”

出了宫城,没走多远,薛向突然扭头朝文德门的方向望过去。

几名宰辅随即一同扭头,只见方才大闹朝会的几名御史此时刚出了文德门,正往右掖门缓缓而去。

旁边还有班直随行,甚至有押送的味道。

人不少,可蔡京却是孤伶伶的单独走着。赵挺之几人,明显的跟蔡京拉开了距离。

“狗一般的东西。”蔡确哼了一声,眼中满是快意。不想做狗的蔡京,却变成了落水狗,这哪能不让他感到痛快!?

“可惜了他的一番心机啊。”曾布冷嘲着,这又是一个正常做事的官员,被不按理出牌的韩冈给干掉了。

韩绛、章惇、张璪、薛向无不露出了深有同感的神情。

就凭蔡京敢当着文武百官的面,直言太祖曾是周世宗忠臣,传出去必然名声大噪。敢说直言,为国无暇惜身,这都是御史的优良品质,直言犯上,完全是立功受勋的表现。

可惜他指责的是韩冈,而且是不惜用自己晋升两府的机会去反击的韩冈。

寻常御史弹劾重臣,在最坏也不过是贬斥出外,然后升官回京的流程保护下,求的是一举功成,从此名震天下。最好能像韩琦一般,一纸撂翻四宰执。

但蔡京今天的举动,分明就是为了求名而来。任谁都知道现在没人能够扳倒韩冈,王安石都只能用自己平章之位,将韩冈从枢密副使的位置上给拉下来。蔡京不会蠢到以为自己能凭自己的攻击干掉韩冈。

不求获胜,只求一个名声,蔡京的成功姓本来是很大的。

换作是其他宰辅,或是韩冈采用正常的处理流程,请求皇后给个公道,逼皇后做出抉择,那样蔡京就能达成目的了。

不过是贬官出外,可从此就名震天下。不论世人是不是将他衔之入骨,但士林和官场上,都会有赞许他的声音。有了名声,官位还会远吗?

但韩冈的疯狂举动,让他的小心思从此化为泡影。

他这辈子就只能成为韩冈的影子,唯一的作用就是被人利用来牵制韩冈。

就是现在的小皇帝亲政,又或是出了什么意外,换了其他宗亲做天子,只要想要压制韩冈,便会想起今曰的赌约。

将蔡京丢到京外,然后当面将此事给韩冈一提,韩冈就是已经做到了宰相,也不得不老老实实的辞官——大臣可以在同僚面前厚脸皮,但皇帝表现出赶人的态度,也只能聪明的请辞了。

即便蔡京的才干足够抢眼,以后掌控朝堂的皇帝也不会想要重用他,让他回京就任要职,因为韩冈远比他更为出色,更抢眼。为了压制韩冈,区区一个蔡京,只是一个随却随用的牺牲品。

“蔡京求仁得仁,又有什么好计较的。”苏颂冷淡的说道,然后转身往西府走去。

不是要让朝廷不去重用韩冈吗?这回他已经完成心愿了。

求仁得仁!
