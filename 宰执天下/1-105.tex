\section{第41章 辞章一封乱都堂(二)}

【第二更,收藏,红票】

“王介甫这回是要走了?”

程颢不论何时何地,无论身前有人无人,向来都是坐得端端正正。后世的被儒生们顶礼膜拜的明道先生,此时也不过三十多岁,可饱学鸿儒的气质,寻常人五六十岁也是拥有不了的。虽然是与自家人闲谈,但程颢肩张背挺的俨然姿态,即便站在朝会上,再挑剔的御史也找不出毛病来。

相较下来,张戬便放松了许多,靠着交椅后背,他冷笑着,“不过以退为进罢了。因为韩稚圭,王介甫是上了告病请郡的札子,但天子现在是怎么想就不知道了。不知是要留还是要放。”张戬说到这里,不满的哼了一声,“不管怎么说,韩琦的话总比我们这些御史管用。”

张载、张戬与程颢是关系很近的表叔侄,而程颢与张戬又同在御史台中,更显得亲近。最后连在京中的宅子,都是租在一起。两家后院还有一道小门通着。三人经常坐在一起议论朝政,探讨经义,他们的妻儿也一样互相来往走动。今日台中无事,张戬和程颢就坐在一起,闲聊起来。话题不知不觉中,便转到了王安石的身上。

程颢轻轻叹着:“若王介甫能稍听人言,也不至于闹到这般田地。”

“听也没用,均输、青苗、农田水利,哪一项不扰民?改是没处改,可王安石能听着劝把三法尽废?!尤其是青苗法,官府出面放贷!朝廷体面要不要了?!又是拿常平仓做本钱,若有天灾人祸,缓急间拿什么去救人?”一提起青苗贷,张戬便是一肚子火,越说越怒。他一贯瞧不起放贷的,连世间常见的僧寺放贷都被他批过,何况官府亲自上阵。

“天琪表叔,你这话就错了。”程颢不同意张戬的偏激,“若从救民济困论,青苗贷不为不美。如当年李参之于陕西,王介甫之于鄞县,都曾救民甚多。只是如今王介甫一改初衷,以求利为上,原本利民的青苗贷早已面目全非。为了多得利息,地方均配抑勒青苗贷,不需要贷钱的富户也要他借钱,朝廷的体面为其丢尽,故而当废。只不过若是能少取利钱,继续行之亦为不可。”

张戬惊讶道:“伯淳,你前日谏章不是说青苗贷不当取利息吗?”

程颢笑道:“这不过是进二退一之法。虽然是说不当取利息,但此事官家绝不可能答应,只求能少收一点就可以了。世间事本是如此,求之为十,通常也只能得之三四。”

张戬觉得程颢妥协得太多了,不过他知道他表侄的性格便是如此,也不与他争论青苗贷的话题。另挑话头:“吕献可【吕诲】前岁曾言,王介甫‘大奸似忠,大佞似信’,‘误天下苍生者,必斯人也,如久居庙堂,无安静之理。’当日,司马君实还说‘未有显迹,盍待他日’,如今观之,吕献可一条条说得还有错吗?只恨吕献可没能早将安石逐出朝堂,让朝野不安如许。”

程颢闭口不论,并不附和。去岁吕诲任御史中丞,以十条大罪攻击王安石,不止说王安石‘大奸似忠,大佞似信’,而且还说他‘外示朴野,中藏巧诈,骄蹇慢上,阴贼害物’。可王安石刚刚任参政连半年还不到,变法才开始,如何能犯了这么多的罪行?

而且其中还有一条,说得是一小臣章辟光上书,劝赵顼把已经成年的弟弟岐王赵颢遣出宫去,因而惹怒了高太后,要将其治罪。王安石支持章辟光,反对治罪,但吕诲却借机攻击王安石是离间两宫,朋奸附下。这样的说法有些太过了,程颢看不过眼。章辟光劝天子将成年的弟弟遣出宫去,哪有什么错?成年皇子都不宜居于禁中,何况亲王?

这都是御史惯常做的,攻击宰执以博清名,即便输了,也不过是到京外任几年官就回来了,一点后患都没有,反而每每因此而升官,哪个不愿?程颢却是不喜欢:“吕献可只是碰上了而已,他弹劾宰执多少次,也不过碰上了三两次。御史正言,当是论事不论人。朝廷设谏官,拾遗补阙那是没问题,但以言攻人,却非应有之理。”

张戬反驳道:“既如此,何必让御史有风闻奏事之权?”

“风闻奏事不是妄言妄语。”

他们两人已经为了如何做御史争论了许多次,每次都没争出个结果。程颢看似温和,其实甚为固执。他任御史里行一年多来,从来都是就事论事,从没有对同僚进行人身攻击。

赵顼曾经问他何以为御史,程颢则回答道:‘使臣拾遗补阙,裨赞朝廷则可,使臣掇拾群下短长,以沽直名则不能。’

赵顼很喜欢这样性格的臣子,多次留下他来深谈,甚至有几次拖到了中午之后,让服侍赵顼的内臣抱怨说他‘不知官家未曾用膳?’

因为程颢是这样的性格,尽管他对王安石提出的新法令有些不以为然,但新法中对的承认,错的指出,并不会一口否定。也因如此,一力反对新法的张戬,就对程颢的态度有所不满,

可张戬拿程颢没法,辩论不是对手,就算偶尔占上风,可看到程颢那副永远都是平和浅淡的笑容时,就没有了胜利的感觉。程颢的笑容,就像一个性格平和的老先生,看到顽皮的小孩子时,那种自然流露出来的夹杂着些许无奈些许戏谑的温和笑意,一点也不像跟自己年岁相当的样子。

张家的一个老仆,这时进来递上一张名帖,“禀御史,外面有位小官人,说是校书的弟子,今次因事入京,便来拜上校书。”

“大哥的弟子?”张戬伸手接过名帖。

程颢看了一眼封面:“弟子韩冈?是子厚表叔门下的哪一位?”

“韩冈?”张戬念着名字,“好像是有这个人。年岁不大,个头蛮高。表字唤作玉昆,玉出昆冈。家世挺普通,但比谁都用功。”

韩冈这个名字他真的耳熟,模模糊糊的有些记忆。张载的弟子他几乎都见过。前次回乡,虽然吕家兄弟走了两个,游师雄也考上了进士,但其他弟子皆打过照面。韩冈当时虽然不显眼,但见了多次,总是能留下些印象。

“请他进来吧。”张戬对老仆说道。

“不知是赶考,还是入京求学的?”程颢随口问着。

“赶考的去年就该来了,若说是入京求学……”张戬想了一下,又摇摇头,“国子监收人也不会赶在礼部试前。”

很快,老仆引着两个人转过庭前照壁。张戬和程颢站起身,就在厅内相迎。

“天琪先生,伯淳先生。”韩冈在张戬、程颢面前拜倒,“末学晚生韩冈,拜见两位先生。”

程颢、张戬两人,韩冈都不是第一次见,甚至都有听过两人讲学的记忆。只是当时他的前身身处张载的众弟子之中,并不起眼,也不指望他们能认出自己。

程颢气质纯粹,谈吐温雅,谦谦君子,温润如玉,就是对他最好的写照。永远都是平和淡泊,无论如何争论,也不见其动怒急躁。与他交谈,顿觉如沐春风。一代理学宗师,诗书醇化气质,也是理应如此,却比他总是一张棺材脸的弟弟要强。而张戬的眼神便利了许多。他二十多岁便中进士,少年得意。又因张载的缘故,而在关西很受敬重。如今做了御史,故而性格上有些锋锐。

这边程颢和张戬两人看着韩冈,也觉得这位年轻人举止自如,形容出色,礼仪上也无所缺,没有一点小家子气,的确是张载弟子的风范。

略叙寒温,三人延礼落座,见韩冈欲言又止,心里透亮的张戬便笑道:“玉昆你到得不巧,大兄日前被派去明州查案了,不知什么时候才能回来。”

“那还真是不巧!”韩冈脸上的失望并不是装出来的,他又欠了欠身:“不过能见到两位先生,已是不虚此行。”

张戬问道:“记得玉昆应是秦州人氏吧?今次入京不知为得何事?”

“学生刚刚得荐秦凤经略司勾当公事,今次入京是来流内铨递家状的。”

“入官了?!”张戬惊讶之色在眼中闪过,看着韩冈过分年轻的面容,“玉昆你才二十吧?”

“学生刚过十九。”

“十九就为官……勾当公事,这是连差遣都有了!”张戬的惊讶再也掩饰不住,监察御史的常识告诉他,韩冈得到的这项任命并不合法度。‘真的还是假的?’他不由得怀疑起来。

程颢一直沉吟着,这时突然问道:“前日听说秦凤机宜王韶、雄武节判吴衍还有都监张守约一起荐了一人,因为年齿不足,而由官家亲下特旨……”

韩冈点头:“正是学生。”

听到程颢提醒,张戬也想了起来。若比耳目消息,御史台在朝堂诸司中可是排前面的。即便是军情信报,监察御史都有资格查询和过问。官家下特旨给一个从九品选人差遣,在御史台中,也算是个小小的新闻了,“原来就是玉昆你啊……”

