\section{第13章 晨奎错落天日近(19)}

一殿皆惊。

韩冈话音甫出,王安石就变了脸色。

他想做什么?李定瞪着韩冈。

朝会或再坐后,臣子留对并不鲜见。朝会上要处理的议题很多,许多事无法深入的去了解,所以天子经常会在事后留下一二宰辅、重臣,来询问详细。但自请留对的情况,在王曾设计干掉丁谓之后,可是十分稀罕的一件事了,偶有发生,全都是要针对某位重臣。

众目睽睽,韩冈面无异色,重复道,“有关两日后共商国是一事,臣尚有还有一些细务需要禀明太后。”

细务?!

曾孝宽摇摇头,还能是什么,肯定先一步去游说太后,试图在廷议开始之前占据更大的上风。

但这件事,既然太后不反对,其他人也不可能说什么。

王安石深吸了一口气,默不作声的照常退出了内东门小殿,在他的引领下,即使是章敦、曾孝宽和李定也没有强行留在殿中。

“没事的。”李定略快了两步,走近曾孝宽身边,“昨夜不是已经计议过吗?不论他怎么折腾都没用。”

曾孝宽抬眼看了看前面的王安石,点了点头。

两府之中,会支持韩冈的,当有韩绛、苏颂,张璪多半也会站在韩冈一边。郭逵肯定不会说话。最后会支持王安石和吕惠卿的,就只有章敦和曾孝宽。平章军国重事能不能被归入宰辅班这很难说,不过韩冈若想要将王安石拒之门外,还是有的嘴仗打。

既然韩冈在宰辅中占据优势,王安石也不可能坐视韩冈利用他的优势。

韩冈所拟定的共商国是,肯定是跟推举宰辅一般,以票数多寡定胜负。

不过推举宰辅,宰辅们只有推荐候选者的权力,选举权在与会朝臣们的手中。而共商国是肯定就不能这么办了。宰辅们举荐同列,天子所不能忍。但要宰辅们不能参与到国是的决定中去,无论如何都说不过去的。韩冈也不会糊涂到不利用自己在两府中的优势。

既然韩冈能够提议召集重臣共商国是,那么王安石也能建言如何商议,尽管都是投票选拔,但投票的方式可就很多变化。

韩冈打算讨好下面的朝臣,王安石也不可能崖岸自高,既然都要拉着下面的朝臣们一起来投票来决定是否改易国是,那么干脆就再给他们一些好处,无论名位高下,都有着与宰辅相同的权力——一人一票,数多者胜。这样才能彻底发挥出新党在人数上的优势,让韩冈惨败而归。

韩冈只要想借用他在宰辅班中的优势,必然会为其他朝臣所憎恨。如果他不去借用,而是跟新党有着同样的想法,那么韩绛、张璪和苏颂会怎么想?

王安石的奏章已经递上去了,现在已经送到了御前,只等着太后拿起、打开。

怎么想,曾孝宽都不觉得韩冈还有获胜的可能。

或许国是的确会稍稍改变一点,以满足朝臣们对秉国之权的需要,但最关键的一项,也是已经与王安石的颜面、以及朝廷风向紧密相关的一项,绝不会有任何改动。

韩冈经此一败,必然要蛰伏一段时间。接下来,就是御史台发威的时候了。不用去针对韩冈本人,他在朝中的势力可没几个是干净的,他曾经举荐过的门客同样如此,一番摘叶斩枝之后,谅韩冈也不能厚颜继续在朝中安居。

“……不过还是要及时打听到详情,也好做出应对。”

曾孝宽猛然从思绪中惊醒,却只听到李定的最后一句。不过到底在说什么,曾孝宽还是能听得出来。

“自是当然。”曾孝宽点头。他回头望了一眼背后的宫阙,其中正在此刻发生的对话,当然是必须要尽快探明。

……………………

王安石已领着一众宰辅退了出去,除了韩冈独立殿中之外,内东门小殿中再无第二名外臣。服侍太后的宫人,也只有十来人还在殿中。

“参政,”人前人后,向太后对韩冈都是以职衔代称,除了他之外,也只有王安石才如此,“请先坐下再说话。”

韩冈谢恩之后,落落大方的坐了下来。

“参政要说的究竟是何细务?廷议上如何拟定国是?”

向太后心中很诧异,也有几分好奇,韩冈特意请求留下来,到底是有什么话想说。

“有关廷议的大小事项,臣已在今日的章疏中写明。之所以召集群臣共商国是,一是因为国是乃君臣共议,非是二三人可以拟定;另一条,也是使各项国是能够得到更好地考订,以免施行后难孚众望。”

“请参政说细一点。”向太后说着,又遣人将韩冈的札子取来。

韩冈开始向太后解释起他对这一次共商国是的廷议的想法,太后专注的听着,同时翻看着今天才收到的札子。

“国是之议,从多不从少,宰辅与朝臣共议。”对照着韩冈折子中的文字“参政打算是一视同仁?”

“两府、百司,职司虽有高下,可皆是皇宋的臣子,共商国是时不需论及尊卑。”

韩冈当然是弄要一人一票,本来他这就是顺便卖好朝臣的提议,当然要将事情做得大方一点。做得小家子气,还不如不做。

“参政说的是。”太后点头赞赏。

这话不是出自卑官之口,却是由政事堂的参知政事说出来,的确让人有一种错位的感觉。但向太后并不觉得吃惊,在她看来,韩冈本来就不是贪恋权势的大臣。就是眼下与王安石斗得你死我活,都是为了各自的学术,而不是争权夺利。有这样权欲淡薄、才干卓异又忠直可信的臣子的确让人安心,不过因为学派之争,以至于干扰到国事,也让向太后颇感无奈,人无完人,不外如是。

“且虽说身处两府,能看得更多,考虑更为全面,但细节上也是需要有人拾遗补缺。”

太后应着韩冈的话继续向下看,札子上,接下来的内容正是韩冈现在所说,不过奏章中也有提及,只有宰辅才有资格提出更易国是的动议。

所以说这番话的前几句,才是韩冈要强调的。宰辅们的地位必须维系,所以提案权只能是在宰辅手中。改易国是的动议,只能由两府中人提出,他官不得言。

“这是老成谋国之言。”

太后多多少少能看出韩冈的想法,但提议和决定,哪一端权柄更重,就不必多说了。不过是对两府的妥协,而且两府的确比普通朝臣看到得更多更远,否则为什么他们是宰辅,而别人只能听命行事?

而且最后的结果,交由天子决定。也就是说,现在垂帘听政的太后就有否决一切的权力。

向太后将韩冈的奏章,从头到尾看了一遍。对里面的内容没有什么不满。基本上就像是宰辅廷推,让她在被推举出来的候选者中进行选择一样,这一回也并没有去削减她手中的权柄。

“如果改易国是,”

“正如当日平章主持变法,异论一时甚嚣尘上,但王平章出入京师,却是人人渴盼,甚至有‘安石不出,奈苍生何’之语。如今太后若欲改易国是,

虽经重臣共议,但事后也必有反复,等到施行时,定会异论迭出,以沮坏国是。”

“的确如此。”

经历了那么多,向太后至少了解了一些人性,成功修订国是后,失败者绝不会跟胜利者合作,而是会更加强烈的去反对。

“若依参政之言,此事当如何避免?”她问道。希望韩冈能有一个让人安心的答复。

“避免人言难,让人无法沮坏则易。只要稳。”韩冈道:“新政施行时不须着急。昔年变法失之峻急,实乃先帝登基未久,亟需有所成就以安朝堂异论。”

韩冈稍稍停了一下,他相信太后对当年朝堂上的韩琦、富弼、文彦博等元老重臣记忆犹新,这番议论,肯定能让太后表示信服。

听见太后低声称是,韩冈立刻接上去道:“而陛下听政两载,功业有目共睹,已无需如此匆忙,可缓缓而行,以稳妥为上。”

这个对比,让太后听得很开心:“参政说得好。的确得稳,不要那么着急。不过还有呢?”

“此外维护国是。国是既定,便不容非议,施政可以议论,但国是三五年内就必须坚持到底,不能今日新订,明日便改。”

韩冈的一番话其实是以太后同意修订国是为前提,向太后对此也没有觉得有哪里不对。

只是她有些纳闷,韩冈特意留下来,难道就是为了说这些,而不是更重要的话?王曾逐丁谓的故事,她现在也知道一点。韩冈犯同僚之忌,却当真只是‘细务’?

但直到一番对话后,韩冈退出内东门小殿,向太后也没从韩冈嘴里听到她想要听到的‘大事’。

半天后,韩冈与太后在殿上的对话传到了王安石那边,

并不是因为韩冈在太后面前进了多少谗言,而是因为什么也没有。他将这份情报递给了身边的吕嘉问。

吕嘉问翻来覆去的看了几遍,眉头越皱越紧,最后他抬起头。

“平章,宫里面看来都已经站在令婿那一边了。”
