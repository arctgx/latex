\section{第32章 荣辱凭心无拘执(中)}

韩冈与李南公说着话,态度诚恳谦逊,也不提河南府上下的失礼,看似全然不放在心上。

李南公和一众僚属,还有韩冈的幕僚都对他的态度十分疑惑。都说是少年意气强不羁,韩冈正是意气风发的时候,还不到三十便是执掌一路漕司,之前也是一路高歌猛进

在关西、在广西都是说一不二;在京城,天子都得给他一份面子,为了他,两名御史被逐出京城,这个消息,甚至比韩冈还要早一步传到洛阳;说起来,文彦博、吴充乃至他的岳父王安石,前任宰相、现任宰相,都被他顶撞过不知多少次,这根本都不是秘密。

几乎都没受过什么挫折,做起事来也是凭着自身的才干强硬无比,遇上这等奇耻大辱,如何能忍耐的下来?李南公从侧后方瞅着韩冈含在嘴角的微笑,总觉得里面充满了杀意,就连他的说话,也似乎带着针对着文彦博的深意。

但韩冈当真是觉得这点小事根本没有什么关系,虽然他作为当今知名于世的儒臣,又是大儒张载的传人,礼法之事虽不能说了如指掌,可也算得上是精通,但要说他多放在心上,那也不至于。

不管怎么说,他都不可能像这个时代的士人一般,将繁文缛节看得太重,不来迎接也不是什么大事,仅仅是个态度问题,他并不会因为这等事伤到自尊心。

而且洛阳是龙潭虎穴,多少致仕的老臣,多半是以文彦博马首是瞻,强来是不成的。韩冈可不会跳起来要跟文彦博争个高下,既然以宰相为目标,就要表现出宰相的气度来。

所以他一路上言笑不拘,对沿途几年不见的洛水畔的景致也是赞赏不已,完全不见半点愤然之色。

转运司的衙署位置离东门不远,韩冈一行人进城后不久便到了目的地。偌大的衙门,前面是施政的公署,后面就是韩家的新住处,为韩冈接风洗尘的宴席也已经在大堂中设下了。

迎接转运使的接风宴本来应该是兼西京留守的河南知府为主,在河南府衙中为韩冈设宴。李南公也是到了昨日见到河南府还是一点动静都没有,才忙不迭的去派人到左近的酒楼去订餐。仓促之下,也订不到需要花费大量时间来烹调的酒菜,十分的普通,用来款待转运使,着实是有些寒酸。

完全不是该拿来接待转运使的低水平的接风宴,让宴会上的气氛有些阴郁,甚至连乐班的表现都是二流水平。不过胜在韩冈笑得开朗,很快就把气氛调整过来了,多少人放下心来,新来的这位年轻的都转运使,论起器量,看来并不算太差,就是不知道是不是装出来的了。

酒宴之后,与会众官各自告辞离开,李南公也与韩冈说了几句话后,一并离去。衙署中的一个小官指挥着吏员打扫大堂,韩冈与方兴一起返回后宅——韩冈幕僚清客中,有资格参加宴会的也只有被韩冈征辟为转运司管勾公事的方兴。

就在前面的大堂上酒宴正酣的时候,在王旖的指派下,家里人都安置了下来。连同一众幕僚、清客,王旖也都安排好了房间和服侍的人选。

王旖治家的手腕,倒不愧是大家闺秀出身,如何掌管一个大家庭,都是从小就开始学着来做的,这两年也有了实践经验,加上几个妾室也帮忙,家里的事不必韩冈多吩咐就安排得一一当当。

另外也是韩冈一向注意维护王旖的地位,只有大妇的身份地位稳固,这样才能保证家中和睦。也不是说韩冈要厚此薄彼,平常值夜都是轮班来的,就是普通的三人共事,也要分出个高下,谁为主导。韩冈主外,王旖主内也是合情合理。

韩冈回来时,解酒汤由严素心亲自给他端上来了,周南带着几个孩子去休息了,小孩子吃不住累,路上兴奋过度,到了地头,吃了些东西,就困得撑不下去,全都去睡觉了。

王旖和韩云娘正说着话,看到韩冈进门,就都站了起身。

韩云娘对今天河南府上下官员的冷淡,为韩冈义愤填膺起来,当着韩冈的面就开始抱怨,“这文相公好不晓事,他不来倒也罢了,怎么都不让别人来?”再看看韩冈仿佛在说着他人的满不在意,更有几分不满的问着,“三哥哥你怎么一点也不生气?”

“为夫当然也生气,”韩冈脸上的微笑一点不变,“不过文相公恐怕正想看到我生气,没必要让他如愿吧?还记得前些天为父给大哥二哥说的北风和太阳的故事?一味强来也不一定能达到目的。”

韩云娘是为韩冈生气,但韩冈自己都没当回事,那就代表这件事根本就没什么关系。她嘟起的小嘴,很快就放松了下来。

韩冈端起严素心的醒酒汤来喝,初来乍到,内院的小厨房中,能让严素心满意的食材为数甚少,临时派人去买也不比在东京一般能买到合意的,韩冈将不够味道的醒酒汤一饮而尽,咂咂嘴,有些不怎么满意。

“过两天就好。”严素心看韩冈似乎对自己方才做的药汤有些不满,为了自己手艺被拖累得没有得到应有的评价,她也有了几分不高兴,“等厨房里的材料补全,官人再来尝一尝。”

“还好。”韩冈笑笑说道,“洛阳有洛阳的风味,不必跟开封一样。”

王旖瞅瞅韩冈,觉得他的这句话似乎又有些夹枪带棒的味道:“官人,今天的事,当真没放在心上?”

韩冈笑容收敛起来,正色问道:“你说潞国公今天做的事是对还是错?”

“当然是错!”王旖其实也是很生气,“就没听说有这么做的!”

“要是为夫为此与潞国公打起笔墨官司,甚至伺机报复,那是对还是错。”

王旖的回答就没有前面那么干脆了,犹豫了半天,“似乎也不太好。”

“娘子说正是。”韩冈呵呵笑道,“他错了,为夫却不能错。潞国公既然倚老卖老,我这个末学晚生就让他卖好了。反正这样做下来,最后丢脸的绝不是为夫。”

面子是人给的,脸是自己丢的,自己只要做得越是宽容敦厚,就会越发的反衬出文彦博的心胸狭隘——毕竟是年纪大了,脾气也会变得倔强古怪起来,如果换在是文彦博年轻的时候,韩冈觉得他应当不会做这等蠢事。

韩冈安抚的拍拍韩云娘的背,又对王旖道:“这件事就这么算了。去派人拿着为夫的名帖,写上学生韩冈顿首再拜,还有礼物,送去伯淳先生的府上。明日为夫倒是有空,也该去看看了……还有吕与叔【吕大临】,自先生故世后,便去了嵩阳书院,也不知是不是要转投程门,正好可以顺便见一下面。”

王旖问道:“官人不也是半个程门弟子吗?”

“为夫是想看看他将先生的行状写得怎么样了。”韩冈解释道,行状是叙述逝者世系、生平、生卒年月、籍贯、事迹的文字,多由门生故吏或亲友撰述,是日后墓志甚至是留名国史的个人传记的依据,“这么长时间,至少草稿该打好了。”

……………………

韩冈这边尚没有动静,但文彦博的所作所为已经传遍了洛阳城。

不以为然的有之,摇头暗叹的有之,幸灾乐祸想看热闹的则为数更多。韩冈这位年轻气盛的都转运使到底会怎么反应,人人都想看个究竟。

富绍庭当天晚上就把这一件事传到了他父亲那里,还疑惑不解的问道,“潞公是不是有什么打算,行事怎么如此颠三倒四?”

“文宽夫他不就是这样的人吗?只是年纪大了,越发的刚隘狠愎。”富弼敲着手中的玉如意,不屑于文彦博的作为,

其实富弼过去与文彦博关系还不错,仁宗时,富弼主持开六漯渠,政敌贾昌朝曾暗中唆使司天监的两位官员说开六漯渠是仁宗皇帝龙体欠安的主因,要以此来构陷富弼,而这件事就是文彦博一手压下来的。不过文彦博的品性,富弼了解得更清楚,恋栈权位,行事刚愎,这都是有的,

“都七十五了,还不自请致仕,你以为他是什么性子?……倒是韩冈,为父倒是想看看他会准备如何应对。”

“任谁都不能忍吧?”富绍庭想了一想,“听说韩冈没有当场发作,在后面的酒宴上,出来的人也都说他言笑自若。但儿子想来,他少不了要记恨上。最近韩冈春风得意,天子都为了他将御史赶出了朝廷。潞公如此‘礼遇’,想必不是上书朝廷,就是借职权跟潞公过不去,河南府中的事务也不是挑不出错来。”

富弼冷冷一笑:“韩冈若当真这么做,日后就不足为虑了。”

富绍庭惊讶的咦了一声:“王介甫不也是这样?当初也没少辞相、称病要挟天子,多少人被他逐出京城!”

富弼摇摇头,“那是为公,此是为私。韩冈若是做出此事,哪里能与他岳父相比。”虽然政见截然相反,但富弼也不会否认王安石的人品。

他的声音顿了一顿,“不过如果韩冈做得大方,以后你倒可以多亲近亲近。”

