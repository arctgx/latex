\section{第33章 物外自闲人自忙(十)}

童贯刚刚抵达洛阳,就听说了文彦博和韩冈上演了一出将相和……或者说,是相逢一笑泯恩仇——似乎是都不贴切——反正是诸如此类的佳话,之前不利于文彦博的谣言,似乎一下就烟消云散了。

童贯隐隐的有些恼火。他身负明暗两道皇命,一路上都不敢耽搁,从东京城出来就直奔洛阳。进了洛阳城时,离着郑国公富弼的寿辰,还有五天之多。

他这么辛苦的兼程赶来,就是想将这个差事办得漂亮了,在天子面前讨个好、得句夸赞,但两边眼下既然已经说和,那么就是天子都不愿再去穷究谁对谁错——

——府漕两家势不两立对天子来说绝不是好消息,童贯估计如今的官家在福宁宫中是做梦都在盼着文、韩二人能和睦相处。只是之前的形势看起来和不了稀泥,才不得不派他童贯出来将此事探查明白,回报京中,以便加以处置。

‘怎么就这么快呢?’童贯都纳闷,韩冈这件事未免太过软弱了,应该再拖几天才是。何必急着去河南府衙,让文彦博再煎熬上几天难道不好?等他童贯将此事查问明白,回报天子之后,再去也不迟啊……

童贯脑中转着的全是私心,但他也不会蠢到表现出来.点着头赞道:“文相公和韩龙图果然还是有肚量,能一释前嫌也是一番佳话……”他接着又问被密召来驿馆中禀报的此地走马承受左丰,“市井中对此是怎么说?”

左丰低着头回话,虽然他的官品不比童贯低,但童贯是在崇政殿中听差,眼下也是代表天子而来,而他左丰则是在皇城之外充当天子耳目,差距实在有些远,“没人再说文相公的不是了,就是之前河南府衙的官吏没有出迎,也说是府衙中的属吏误会了文相公心意。但也有人说,韩龙图是为不让河南府在兴修工役时扯后腿,才不得不上门负荆请罪。”

“负荆请罪?”童贯眼神顿时一凛,厉声问道:“……这是谁说的。”

左丰不知打听到了多少种偏向不同的流言:“外面有不少人在说。文相公是有心给韩龙图一个难堪。没有出城迎接,并不是衙中属吏误会了他的心意,而是为了给韩龙图一个下马威。韩龙图知道强龙不压地头蛇,只能去赔小心,第一次没做好,才不得不去第二次。不管怎么说,现在都是都转运使去河南府衙,而不是判河南府来漕司衙门,到底是哪一边势弱,一看就知道了。”

韩冈负荆请罪?童贯摇摇头,不能这么说,也是难以想像。应该只是帮文彦博解围,卖好而已,并不是向西京留守卑躬屈膝,“这个传言是什么时候传出来的?”

“也就是今天才一下传开的,昨天还没有听说,小人也是今天中午的时候才收到。”左丰回话道:“基本上都是这么说。说是韩龙图怕漕运被人扯后腿,所以忍气吞声,不得不第二次上门,做坐足了两个时辰,才敢告辞离开。”

眼下的两种说法,一种是韩冈宽仁大量,让文彦博都要承他的人情。另一种则是韩冈委曲求全,希望文彦博不干扰他去开凿襄汉漕渠的工役。

童贯心中疑云大起,两种说法都有些问题,尤其是第二种:

‘韩冈应该不是这个性子!’……‘是决不是委曲求全的性子。’

童贯对韩冈的第一印象,就是当年他跟着李宪抵达熙河,当时王韶和高遵裕领军翻越露骨山追击木征残部,一时音信全无。

韩冈区区一个刚做官才两年的小京官,硬顶着带着退兵诏令而来的使臣,抵挡住了西夏和吐蕃的反扑,保住了熙河一路。这样宁折不弯的强硬性格,如何会为保证漕运供给而向文彦博弯腰?恐怕是会为了设法将文彦博给请走而努力。

童贯忽然觉得放在自己眼前的是一团乱麻,他的任务就是解开这团乱麻,将整件事的内中隐情原原本本的查验出来,以便度过此次的难关。

……不对!童贯忽然醒悟过来,他的任务并不是把事实探查明白后告知天子,而是要让天子相信自己的话是事实。如果天子不信,真的也是假的,若是天子相信,假的也是真的。

也就是说,只要自己表现得好,天子对河南如今的内情了解,都会来自于自己。他一个低品内侍,就像是一枚能左右天平平衡的砝码,决定了名为天子的天平的倒向。只是在此之前,童贯必须先确定自己的倾向……不过这一选择很好做出来就是了,童贯都没怎么去想,就已经有了决断。

若是一个七十五,一个五十七,该偏向哪边,也许还得费一番思量,但眼下文彦博七十五,而韩冈则是二十七,偏向谁难道还需要多想吗?

更不用说他童贯跟韩冈打过不少交道,当面能说得上话。而跟文家则是一点交情都没有,那文彦博,更是只在朝会上远远的见过,一个身量高壮的老头儿而已。

尽管自己是宦官,但日后也少不了也有要依靠两府的地方,宰执官们不但能掺合入内侍的晋升,更能坏事。管勾皇城司的石得一,可是吃了士大夫们的不少苦头。童贯乃是聪明人,自然知道该选择哪一边。

不过此事不能做得太明显,作为天子家奴,需要是不带私人立场的公正,如果偏向太大,天子那一关也不好过。

童贯皱眉组织着语言,该怎么说才能让天子满意,不至于误会自己,但同时还要表现出一定的倾向,让天子的心意也跟着倾斜,‘这份差事,果然是不容易。’

……………………

富弼已经听说天子的使臣今天赶在城门合钥之前,带着礼物进抵驿馆。作为洛阳的地头蛇,他更是连府中的走马承受被招进去问话的消息都打探到了。

“果然老夫的生辰只是附带,主要还是文宽夫和韩玉昆的事。文宽夫这一番闹腾,倒是让天子都记挂在心上。”富弼自言自语的口气似乎有些小抱怨,但脸上的淡定,让人一点也看不出来他的真实心情。

挥了挥手,让报信之人退了下去,还政堂中又只剩富弼喝着当归饮,一名老仆在旁服侍。

富弼是如今洛阳城中最清闲的一位元老,一个是因为富弼自致仕归乡后,便以老病为由,少见访客——他在洛阳亲朋故旧数千,若是开门见客,从早至晚都不得清闲,同时他的脚的确一直有病;另外一个原因,乃是府漕之争吸引了太多人的注意力,让富家门庭也变得清净了一点。

不过富弼今日的清净并没有太久,只过了片刻,就有人来报:“刘秘监来访。”

“刘伯寿可是好一阵子没来了,快请!”富弼说着就起身,在老仆的搀扶下降阶相迎。

富弼也不是所有客人都不见,一干耆老,包括刚刚过世的邵雍,都是经常走动。富弼崇佛,洛阳的几位高僧大德也是常来往,刘几刘伯寿也是其中之一。

刘几的官位虽不算高,但刘氏乃是传承数百年的世家,从北齐一直延续到今,代代有人出仕,且世牒具存,不是吹嘘出来的,在洛阳城中声望不低。

在庭中富弼与之对行了礼,一起回到厅中坐下。等下人奉上了茶汤和菓子,富弼就有:“伯寿有半年多没上门了”

“冬天畏寒,不喜出门,开春又生了懒病,这两日方才病愈。”年纪都大了,说话也没顾忌的,刘几内外张望了一下:“彦国生辰将至,怎么你这还政堂中也不见多少喜庆?”

“并非是逢九逢十的正经日子,也不准备大事操办。有事也是小儿辈忙着,我这里倒是清闲。”

刘几瞅瞅富弼身上的一袭没有花样的素色直裰,“清闲是清闲,也是越发的清俭了。”

富弼微微一笑:“只为惜福之故。如今连荤腥也少沾了。”

“当真要受戒做居士了?”刘几不以为然,喝了口茶后问道,“听说彦国明日意欲往坟寺剃度一僧?”

“确有此事……伯寿你身子懒怠在家,耳朵倒是到处跑。”富弼笑说了一句,又道:“此人言谈可喜,礼佛甚诚,只是贫不能具度牒,故而顺水送他一程。”

“好个顺水送人一程。”刘几笑了起来,“不过彦国你坏了几个,才度得一个,世尊前不能论功啊。”

富弼有些疑惑:“此话何从说起?”

“是刘贡父【刘攽】前日在偃师说的,是指你去年度得那个和尚。刘贡父说彦国你‘每与僧语,往往奖誉过当,其人恃此傲慢,反以致祸,攽目击数人矣,岂非坏了乎?’”

“刘贡父总是口舌上不饶人。”富弼不快的皱了一下眉,转又笑道:“方外之士,无碍世人,坏了也就坏了。若是换作一亲民官,那又当如何?”

“这话说的好,只是佛祖不爱听。”刘几拍拍手,凑近了一点,“不知彦国你觉得如今府漕两家之事,是好了还是坏了。”

“……文宽夫如何说?”富弼反问。

“还没去问过。”刘几顿了一顿,摇头笑了笑,“恐也不当问啊。”

