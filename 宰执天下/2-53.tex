\section{第12章 平生心曲谁为伸(七)}

【第二更。今天在外面聚餐回来,赶到现在,才把今天的份赶出来。明天后天两更肯定维持不了,一更也不是很有把握。不过欠下的章节,俺后面都会补上,还请各位书友放心,】

“寻常的丝绢可就只有一千三五百一匹!”李小六在后面听着乍舌。

而韩冈知道,这并不是胖掌柜乱报价。由于原材料产量的稀少,棉布可不便宜,跟蜀锦差不多。但这个价格还是不对。

他抬眼看了看胖掌柜,露出了一个看透了一切的笑容,“你这怕是西川的货吧?”

胖掌柜脸色一变,急道:“官人这话怎么说的,这可是实实在在的黎货。”

“本官前月去京城,真正出自黎人之手的吉贝布都是十贯起跳,最好的折枝凤团广幅布能卖到三十贯一匹。而西川和广南的货色,就要便宜一些。但凡吉贝布,若是只卖七八贯,那都是转运路上不慎浸了水,坏了品相,只能打折卖。”

韩冈对棉花很感兴趣,特意打听过行情,对此是一概门清。他见胖掌柜还要辩,给出了最有力的一击,“以琼州往秦州的路途,一匹吉贝布的运费都不止这个数目。在秦州能把价钱压得这么低,只会是西川的货,要么就是从河西过来。还是说,你这是浸了水要打折的货色?”

胖掌柜被韩冈砸得一时说不出话了,谁能想到一个官员会对布匹的事都了如指掌?

韩冈不为已甚,摇头笑了笑:“算了,我等小官,官俸微薄,不论是真吉贝,还是假吉贝,都是穿不起。还是挑丝麻的好。”

韩冈不再追究,放了一马。胖掌柜又楞一下,便很乖觉的承认了下来:“官人心明眼亮,说得正是。小人这也是生意上的声口,不这么说就难卖不出去。但这布是实实在在的好,小人也没有高开价骗人。既然官人能看出这匹木棉布来自西川,想必对此也是深有了解,小人却是对这木棉布一窍不通,实是明珠投暗,待会儿小人把这匹布给官人送到府上去,也算是有德者居之。若是顺便,小人还想请官人在其他官人面前品评两句,日后小人也好多得几个官人照顾生意。”

韩冈摇头失笑,瞟了一眼谄笑着的绸缎铺掌柜,心道这贿赂的手法还真是千年如一。而且这胖掌柜说话尽带着些文酸气,但遣词用句却是有些可笑。他不置可否,却问到:“你既然认识本官,那你可知本官在安抚司中执掌得是何事?”

胖掌柜精神一振,“官人执掌的是军中医药,办的是疗养院,救人无数。这小人怎么会不知?秦州城也不会有人不知道的!”

“那你可知安抚司里的王机宜是做什么的?”韩冈继续问道。

“小人当然知晓!”王韶跟李师中、窦舜卿还有向宝之间的争斗,可是秦州城里有名的八卦,也一样是口耳相传,尽人皆知。

“王机宜可是难得的英雄好汉,把秦州西面的蕃人管得跟自己儿孙一般听话!”胖掌柜比出个大拇指,赞道:“这几个月两次大捷,杀得蕃贼几万人屁滚尿流。听说前日大战,渭水都给蕃贼的尸首堵上了。凭着王机宜的功劳,日后定能跟韩相公一样当上宰相。”

“蕃部只是其中一件,还有呢?”韩冈像是在考试,一句接一句的追问着。他又回头看韩云娘和严素心,见着她们还在那里比着两匹绸缎的好坏,看样子也不是短时间内能作出结论。韩冈并不介意趁机多说几句。

“还有的就是屯田吧?”胖掌柜这回想了半天才想起答案。王韶与窦舜卿的荒田之争,同样是在秦州城中传言,但传得不是那么广,由于时间久了,对此还有兴趣的人也不多了。

“屯田是一项,还有就是市易。”韩冈为之补充。

胖掌柜终于觉得有些不对劲了,眼前的这一位怎么对他一个做小买卖的说这些话?

“官人,是不是有事要差遣小人?”他小心翼翼地看着韩冈脸色。

韩冈笑了。他抬起手,在空中一划,掠过堆满店中的丝绸,“秦州种桑麻的少,这是水土不宜的缘故,故而丝麻皆要外运。但甘州、凉州却早在唐代能种木棉,秦州的水土与河西相仿,想必也能让木棉生长。而且秦州闲地也不少,分出两三千顷来种木棉却也不难。”韩冈回过头来,对胖掌柜说着,“本官说的话,还请原样转告贵店东家。”

胖掌柜浑浑噩噩的点头答应了下来,没弄清韩冈究竟是什么用意,只知道韩冈想着在秦州种棉花。突然间脑中灵光一闪,他顿时醒悟过来。难道韩官人他是要邀请东家一起参与此事?

他再看一眼韩冈,难道今天这位年纪轻轻就以才智闻名秦州的韩三官人,是为了邀请东家,而特意走进这家铺子的?此事可真的要与东家好好说道说道了。

韩冈却没有那么多想法。今天的事是他看到绸缎铺中的棉布临时起意,不过联络秦州商户却是他筹划已久。而推广种植棉花他也早有考量。明清时棉布取代了如今惯常所见的丝麻,成为民间最常用的织物。既然历史潮流如此,韩冈理所当然的要顺流而行。

在秦州种棉比种桑要简单,桑树要能大量取叶,少说也要三五年。但种棉只要栽培得好,却是当年就能收获。同时比起丝绸麻布,厚实的棉布当然在冬日深寒的秦州更有用处。

用减免赋税的口分田来吸引民户,而用高利润的棉田来拉拢秦州大户。如果能得到贫富两个阶层的支持,王韶开拓河湟的根基也会变得坚实起来——这是韩冈准备要在王韶面前说的话。

——冠冕堂皇,却非真意。

棉田推广,不是短期内就能建功。这不像粮食,该怎么种才能有收获,种过田的农民们心中都有个数。但棉花在秦州可是个稀罕货。

第一年,只能先开个几十亩的试验田。如果成绩不错,那第二年就会扩大到三四顷。两年时间,勉强可以让人初步摸索出在秦州这片土地上种植棉花的技术来,而收获也让旁观者看到好处。接下来的几年是大举推广的时间,但想要到大量收获利润的时候,却是要等到五六年后了。

五六年的时间,天子等不及,王韶等不及,韩冈更不可能等得及。开辟棉田,其实是拿未来的收益跟豪门富商做利益交换。王韶希望能得到他们的支持,而韩冈本人也是想着能与他们联手在市易之事上插上一脚。

当然不是为赵官家,而是为自家考虑。

北宋的商业发达,所以铜臭之物便分外受人喜欢。别看士大夫们各个摆出富贵不能淫的态度,自命清高,不屑俗物,但他们中的绝大部分听到叮当作响的声音,耳朵就会立刻竖起来。

这世上没钱可不行。韩冈的品级是官员中最低的一级,俸禄一月也不过五贯不到,加上一点惯例的灰色收入,也就勉强十贯。韩冈前面说自己买不起吉贝布,并不是哭穷的虚言。

艰苦朴素,让家里天天吃素,只有寥寥可数的几个清官能做到,韩冈做不来。他要让自己的家人过上富足的生活,充裕的金钱是少不了的。韩冈不想贪污受贿,家里也没个田产,剩下的道路就只能做点小买卖了。

只是韩冈要插手市易之事,不能明着来。王韶把这一块都划给了元瓘那个还俗和尚,韩冈不好明着掺和进去。据他所知,元瓘在对此很上心,也做了不少工作,他已经先一步联络起足够的人脉来。韩冈如果在明面上跟他竞争,要费大力气不说,还会开罪王韶。

所以要采取迂回战术。韩冈想着过几日给邠州去一封信,看看路明能不能来秦州。自家支援他开一间商铺,联络秦州的几家商行,往即将开在古渭的榷场做些买卖,只要他不去与元瓘争夺事权,韩冈确信王韶会睁一只眼闭一只眼。

韩云娘和严素心终于选定了两匹绸缎,一匹素色隐纹,一匹则是带着龟背花纹的赭色缎子。韩冈看着,这两匹好像就是两女一进门时当先拿起来看过的。

胖掌柜不肯收钱,直说要送给韩冈,但负责拿钱袋付账的严素心知道韩冈不会贪这个便宜。最后一番退让,胖掌柜给韩冈打了七折。最后胖掌柜对韩冈他们笑道:“官人可以陪两位小娘子去逛逛街市,小人现在就遣人把缎子送到府上去,不劳官人烦心。”

韩冈道了声谢,在点头哈腰的胖掌柜相送下,出了店门。他回头跟胖掌柜说了两句告辞的话,而韩云娘和严素心已经先走在街上。

一阵蹄声不知从何处传来,声音由远及近,来得飞快。

竟然入夜后在城中奔马,难道出了什么大事?

韩冈惊讶得循声望过去,数息之后,一群骑手便带着隆隆蹄声,猛然从十几步外的十字路口处冲了出来。他们一行有四五骑之多,转过街角,他们用力扯过缰绳,几声马嘶之后,便毫不犹豫地冲上了人流熙熙攘攘的河西大街。

街面上顿时慌乱起来,街中的行人车马忙不迭的躲避这几个疯狂的骑手。严素心先急着去抱招儿,而韩云娘却怔住了还没反应过来。

韩冈看着心中大急,连忙抢前一步,左手将小丫头扯到怀中,右手又用力拉过抱着招儿的严素心,四人一起向后疾退。

