\section{第39章 遥观方城青霞举(五)}

【元宵节喝了酒,头昏脑胀的,睡了一觉才起来写。对不住各位书友了。】

虽已入秋,但御园中的草木依然茂盛。

桂花离着盛放还有几天,不过已经有一丝若有若无的甜香,在空气中浮荡。

赵顼扶栏观水。朱婕妤、邢婉仪等几名正受宠的嫔妃则带着皇子皇女,在不远处的凉亭中,等候着赵顼的召唤的同时,闲聊着各自感兴趣的话题。

池中的荷叶残落了许多,荷花自然早就败了,一颗颗莲蓬被挑在水面上,无甚可看之处。但大宋天子的双眼却盯着水面,不过两眼焦点茫茫然,显然没有落在荷叶上。

五日前,赵顼收到了京西转运司的奏疏。韩冈在奏疏中汇报了襄汉漕渠的最新进展。虽然渠道依然在方城山处中分,但方城垭口轨道的修筑完成,代表着襄汉漕运的替代通道已经可以投入使用。在襄州连通京城的水道全线贯通之前,这条替代通道将为襄汉漕运。

韩冈更在奏疏中说明,方城轨道两端的转运港口预计将会在九月底完工,故而申请将荆湖两路和京西南路的总计六十万石的秋粮,通过新开辟的渠道运送上京。

这份申请赵顼已经批复了下去,中书也签押过了。他肯定是要看一看韩冈的成果。到底能不能见功,能有多少运力,这关系到大宋是否能再多上一条联系南北的生命线。

荆湖两路,在章惇收复荆南之后,一年的纲粮数目有一百二十万石。赵顼当然希望这条通道能有一百二十万石的运输能力,如果不行的话,一百万石也能接受,再少可就没意义了。

若是能比一百二十万石多,那自然更好。汴河一年的纲运是六百万石,但除此之外,还有多达数倍的商货运输。一条漕运通道,不仅仅归官府所用,民间也当能享用得上。

只是赵顼并不是很指望襄汉漕渠能与汴水一较高下——漕渠的运力与渠中的水量有关,沟通黄河、淮水和长江三大水系的汴水,拥有的水量不是京西几条细窄的河流可以相提并论,襄汉漕渠即便全线畅通,最多也只能是汴水的补充,而眼下还只能用轨道暂代,恐怕也当真只有最多百万石的运力。

在襄汉漕运投入使用前,赵顼都不会太过记挂在心上,真正让他陷入沉思的,还是种谔的上书。乘着西夏国内梁氏和乾德的母子不合,起兵征讨西夏,将盘踞大宋西北的这个国家彻底覆灭。他的提议,对赵顼有着莫大的吸引力。

种谔从来都是好战的,赵顼当然不是不明白,但种谔过去所表现出来的战略眼光,却是远超侪辈,每每见功。他既然提请开战,自然是看到了西夏的弱点,有立功的可能,否则此等良将,也不会拿着自己的名望地位来赌博。

但赵顼作为天子,不可能只听信一人的意见,种谔也不是不会犯错的将帅。其他臣子的观点都要听取,而赵顼本人,对于时局也有自己的认识。

契丹人对西夏的支持能到哪一步,这一点就是困扰赵顼乃至整个朝堂的最大问题。

要是契丹国中有变,西夏可就完了。很多时候,赵顼都在想,如果那位掀起了叛乱的皇太叔还在就好了,或者现在的权臣耶律乙辛有造反的胆子也好。一旦辽国内乱,赵顼能毫不犹豫的下诏发动讨伐西夏的战争。

但耶律乙辛现在只是个权臣而已,还没有做到挟天子以令诸侯的水平,而辽主耶律洪基在做了几十年皇帝后,在国中也有足够的控制力。不过也不是没有机会,耶律乙辛害死了辽主唯一的儿子,眼下看起来似乎没有动静,但这件事迟早会闹起来。耶律洪基如果想要铲除耶律乙辛,辽国国中肯定会有为时不短的动荡,那时候就是机会了。

离开白玉栏杆,赵顼走近凉亭,一名名宫中佳丽全都站了起身,以万福相迎。

“在聊着什么?”赵顼进了凉亭,坐了下来。

生下了皇第六子、也就是如今宫中排行最长的赵傭的婕妤朱氏笑道,“正猜着今年联赛的头名究竟是谁。”

京城中只有一项联赛,就是如今正红火的蹴鞠联赛。入秋后,歇了一个夏天的蹴鞠联赛就要重燃战火。

自五年前,棉行将流行于熙河路的蹴鞠联赛带到京城之后,经过了区区数年的发展,蹴鞠联赛就成为了京城中最受欢迎的运动,比赛制度也已经完备了起来。

旧时京中,就有以踢球为主业的齐云社,多家球队聚起来比赛,但远远不如现在蹴鞠联赛的刺激。受到所有人疯狂的追捧。

那种软绵绵的表演脚法的球赛,早已被硬朗、凶狠的拼杀所取代。比赛中经常有球员争球时撞得头破血流的场面。京城百姓过了上百年的太平安定的生活,难得受到血腥气的刺激,喜欢上的这个味道的球迷们一个比一个更加疯狂。

联赛的制度也是吸引球迷的法宝,主客场制,循环赛积分制,多支球队组成的联赛,让一年之中的大部分时间,东京军民们都能看到比赛。在漫长的赛季中,支持着自己所喜爱的球队一步步走向胜利,更是忠实的球迷们的共同心愿。

猜测冠军谁属也是球迷们共同的爱好,赵顼早就见怪不怪:“猜到了是哪一家?”

“现在甲级联赛积分排名第一的是车马行,队中的几名大将都没有伤病,下半赛季保持上半赛季的水平,头名跌不出他们的手心。连齐云快报也这么说。”

一名才人则反驳道:“齐云快报上的说辞做不得准。上次棉行的鲁七明明是伤了腿,报上却还说没有伤”

“登载的是棉行球头游勇的话,他当然不会实话实说,兵不厌诈嘛。”

“快报上也说了,车马行只是暂居第一,后面两家追得紧的很,只要错失了一两场,就会从头名落下来。”

赵顼都纳闷,怎么都看了齐云快报?

刊载新闻消息的小报,东京城中很早以前就有了。但像齐云快报这样专业性的小报还是第一家。

由主管联赛赛务的东京齐云总社创办的这份报纸,每一次的比赛日之后,总是会及时刊登比赛结果,积分排名,以及对各场比赛的点评,各支球队的球员被访问后说的话,也都会刊登在报上,许多有关球赛的新鲜名词也是从这份报纸上推广到每一名球迷的嘴里。同时还少不了球队赞助者出钱打的广告。

宫廷中,只有重复得太多无聊的娱乐活动,要不然仁宗皇帝也不会眼巴巴的将宫外的女相扑叫进宫来表演,为此还挨了司马光一顿批。

永远都不缺乏新鲜感的蹴鞠比赛,当然要比抛绳、飞竿之类的百艺表演要有趣得多。虽然嫔妃们一年之中看不到几场比赛,每一场比赛只能从齐云快报上看到结果。但她们中的许多人对于各支球队如数家珍。

“今年甲级联赛的头名就三家争,第四名往后,积分都差了不少,赶上来的机会太小。倒是降级区就堆了五支球队,不知哪两支会降级了。”

“棉行下半赛季再不努力,说不定他们会真的降级,只比倒数第二的甜水巷多一分。”

“要不是棉行队的鲁七上次受了三个月的伤,在病愈之后也没能恢复旧日的水平。加上乌克博也回乡去了,要不然也不会败落到如此地步。联赛中最早的元老之一!”

赵顼侧耳倾听着嫔妃们对宫外的比赛的评价,在争论时,她们之间甚至都模糊了尊卑高下,甚至将皇帝丢到了一边。赵顼插不上话,他几乎抽不出时间来看比赛,连看快报时间都不长。

虽然蹴鞠联赛发轫于熙河,据说还是韩冈首倡,连如今通行于世的规则也是韩冈所制定。但熙河路诸州毕竟是都不大,平均每州也只有十几支球队,合在一起踢比赛就够了。但京城不同,人口百万之众,由于联赛的发展,加上丰厚的奖金刺激,这些年组建的球队多达百余支。

这么多球队当然不可能聚在一起比赛,所以就有了联赛分级和升降级的制度。甲乙丙三级联赛,每一级都是十二支球队,前两名升级,后两名降级。

至于剩下的小球队,则是实行的赛会制,聚起来踢淘汰赛。将京城通过纵横两条中轴线分成四个区,各区中小球队先通过淘汰赛决出冠军,然后四个区再通过循环赛决出前两名,取代丙级联赛的降级球队。

每个赛季的上半赛季,是三月初到五月底,下半赛季则是从八月中开始,到腊月中旬结束。到了正月时候,还有一个金球赛。甲级联赛的前四名,乙级、丙级联赛的前两名,争夺一个铜质镀金的足球模型,当然,还有高达千贯的奖金。而从去年开始,三月初八,前一年甲级联赛前两名在金明池,又多了一场在天子面前表演的争标赛。

金球赛的决赛,以及争标赛,最后都是在金明池边的球场举行,这两年,赵顼都带着嫔妃们来看球赛,宫中的不少人变成狂热爱好者有一半是在看比赛后。连宫中举行的蹴鞠比赛,也被改成了新式规则。

三级联赛的各支球队都有固定的球场,附近的居民一般都是他们的支持者,就像棉行队,已经是城西的第一号球队,里面出来的任何一名球员,出去吃饭都能碰到人请酒。虽然今年几个主打接连受伤,在上半赛季落到了最后,但他们的支持率依然极高,球迷们对他们支持的球队都是不离不弃。

只不过也有疯狂的球迷,闹出来的乱子不是一桩两桩,最后支持不同队伍的球迷间的斗殴时常可见。一家家酒店茶肆成了不同球队球迷聚集的大本营,在十天一场的比赛前后两日,都是最热闹也最容易出事的时候。

御史们没有少弹劾蹴鞠联赛扰民、致乱、败坏风气。但蹴鞠联赛早已形成了一项横贯黑白两道的庞大产业,丰厚的利益将上至宗室、下至小吏数以万计的,都拉到了一条船上来。叮叮当当的铜钱撞击声,让反对的声音变得微不可闻。更别说京城球迷以十万计,谁都不会眼睁睁看着联赛被人毁了。

东京城是天下流行的发源地和制高点,诗词、学术、娱乐,能占领京城的,就能占领天下。当蹴鞠联赛在京城受到欢迎的时候,当然也随之传播到地方上。很快天下州县就都会组织起联赛来了——韩冈所首创的蹴鞠联赛。

‘又是韩冈!’赵顼想着,这一位年轻的臣子,总能带来奇迹。就是随便在踢球上颠簸了两句,都能引发一阵风潮。

这样的人物,似乎在哪里都能立下功劳。等他结束了京西的差事之后,该将他调到哪里去呢?赵顼拿不定主意,只要不是京师……

