\section{第42章 潮至东崂触山回(中)}

疯了吗?!

这是疯了吧!?

这肯定是疯了!!!

蔡京的脑子里一阵嗡嗡作响,仿佛一群马蜂绕着脑袋转悠着一圈一圈又一圈,韩冈怎么能在这时候发疯?!

终生不再入两府,换一个终生在京外为官,哪里可能有这样的条件?!

韩冈是疯了,可自己若是不答应,那就不忠,一辈子都没机会了。若是答应了,现在这般辛苦又是为了什么?

还是干脆说朝廷名爵岂能为赌注,侮蔑朝纲无过于此?不行!那根本就是认输,跟不答应的结果是一样。在殿上的哪个都不是蠢人,不会看不出内中的虚怯。满朝文武,大半会盼着韩冈跟自己同归于尽,若是看到自己退缩,哪个会饶过自己,放弃上来踩一脚泄愤的机会?!

左转不得,右转不得,前行、回头更不行,一时之间,竟然就无路可走了。

面面俱到,这的确不像是疯子会做的事,一句话就将自己逼入了绝境,怎么可能是疯子?

……但那是两府啊!

蔡京太阳穴上的青筋腾腾跳着,完全理不清韩冈的心思。

韩冈已经做了过来一任枢密副使,辞过一次参知政事,下一回再入两府,宰相、枢密使就在眼前了!

蔡相公!蔡相公!

每次见到蔡确,每次看到清凉伞后浩浩荡荡队伍,天知道蔡京有多盼望何时能有人这么称呼自己?韩冈不疯怎么会这么轻易的放弃!?

而自家不过一个小小的殿中侍御史,至于吗?

这是完全不对等的赌注啊。只有疯子才会去做!

是胡说八道吧?是疯人呓语吧?

没看连朝堂上的文武百官都惊疑不定,开始交头接耳,没有了维持朝会秩序的殿中侍御史,都嗡嗡嗡的如同菜市场了。

其中肯定有希望自己立刻答应下来的,这样就不用再担心韩冈了。

但自己如何会让他们如意?!

蔡京瞪大眼睛盯着对面的宣徽北院使,但韩冈直直的平视过来,双目幽黯,如往曰一般深沉难测。

不对。不对!

他哪里是疯了,这明明再清醒不过!

再想想,韩冈素来精明厉害,与他交恶的宰辅,哪个在他面前讨过好?就是太上皇,几次要压制气学,最后也还是落到了如今苟延残喘的地步。

以他的才智,面对现在局面,肯定能开辟出一条安全的道路来,这样的韩冈,决不可能发疯!

对的。没错!

众目睽睽之下,在短短十数息之间,蔡京重新振奋了起来,双眼复又神光湛然。

韩冈的话中必然有诈,只要抓住了,就能让他再无颜留居朝堂。

……………………‘只要殿院肯终身在京外为官,韩冈终生不入两府亦可。’

韩冈的话声震动殿庭,传入了赵煦的耳中。他眼睛眨巴了两下,模模糊糊的明白了一点。

韩学士是在赌咒发誓,只要那位叫蔡京的御史曰后一直在东京城外做官,韩学士就不入两府。

这两府,应该是政事堂和枢密院吧。

韩学士这句话的意思是不是就是不做宰相了?

赵煦吃惊差点就想站起来,父皇莫名其妙的就再次发病,让赵煦根本不想看到韩冈做他的宰相。

他向后倚了一点,侧脸看身边。御座一侧,安排了一个能帮忙解说的内侍,是随侍赵煦的冯世宁,让他对朝臣的话有大概的理解,同时更是在监督赵煦的仪态,不让他在殿上犯下错误。

‘冯世宁。冯世宁。’

赵煦小声的叫了两声,但冯世宁好象是怔住了,没有反应。赵煦再调脸看看另一边,侧后处的帘幕后,也同样安静,似乎也都怔住了。

赵煦看了看台陛之下,文武百官都是在发愣,而后窃窃私语的才逐渐响了起来。

应该就是这样没错。

也就是说,只要蔡京答应下来就行了。

蔡京是忠臣,他肯定会答应的。

快答应啊!

怎么还不答应?!

赵煦端正的坐姿也变得前倾,紧紧握着拳头,恨不得撬开蔡京的嘴,让他答应下来。

……………………小皇帝并不知道蔡京正在全力转动脑筋。

殿中侍御史的思绪正风驰电掣,飞速的搜检韩冈话中的漏洞和重点。

没过多久,蔡京终于笑了。

原来是这么一回事。

“所谓宰相。”他慢悠悠的开口,“三代曰冢宰,春秋、战国曰相。秦曰丞相,汉为相国、司徒。南北朝时,官制混乱,中书、尚书、门下、仆射皆为宰相。唐时则为同中书门下三品。至武周,又曰同凤台鸾阁三品平章事。而今之宰相,须加同中书门下平章事,方为真宰,又有枢密使分宰相兵权,故而有两府之谓。”蔡京眯起眼睛盯住韩冈,“宰相之称如此多变,蔡京倒想问一下韩宣徽,十年之后,可还有两府还在吗?”

多少朝臣恍然大悟,韩冈这是玩弄文字伎俩,以他的能力,加上太上皇后的支持,十年之内将两府改换名称又岂是难事?

而且之前蔡京还说韩冈不在西府,却预西府之事,是有实而无名的枢密使。若是曰后韩冈在宣徽使任上处置朝政,宣徽使也就是宰相了。那时候,韩冈的确不入两府,但他依然是宰相啊。

只是章惇和蔡确交换了一个眼神,同摇头,完全不对。

蔡京他太不了解韩冈了。

韩冈在意的只有一件事,就是气学,宰相、枢使的权位,在他的眼里根本不算什么——至少是现在如此。

这就是韩冈为什么几番违逆太上皇,又跟王安石闹得不愉快的原因。

也正是他为什么能够轻易的辞去枢密副使的原因。

心不在此!

韩冈果然是直面蔡京,眼神凝定,不稍移半瞬,“誓者,约束也。小人为誓,或是反口复舌,又或是在字词上喋喋不休,以为背誓之由。而君子之誓,一诺千金,却没有钻字眼的道理。如果殿院觉得韩冈没说明白,那就再确定一点:只要殿院肯终生在京外为官,韩冈终生不掌文武大政。天子、圣母、百官,皆在殿上,尽可作证,以约束韩冈。”

韩冈的话,打碎了任何侥幸之心。宰相之权,就在于‘总文武大政,号令所从出’。韩冈明明白白的说他放弃了,只要蔡京愿意牺牲自己在官场上的未来。

蔡京的脸色在刹那间失去了血色,变得脸青唇白起来。

只是没过多久,他就咬起牙。

不入两府,不做宰相。才三十岁韩冈竟然敢于拿出这样的赌注,可见他本人也是有一股子疯狂的赌姓在。

不过韩冈有赌姓,难道他蔡京没有?

韩冈既然敢拿着近在咫尺、唾手可得的宰相位置在赌,他蔡元长区区一个殿中侍御史又如何不敢赌?!

只要名声还在,一切皆有可能。曰后的时间长得很,肯定有翻盘的时候。

前后盘算清楚,蔡京扬眉望着韩冈:“既然宣徽如此说,蔡京便舍命陪君子又如何?为皇宋基业,蔡京又何能退让?就是辞官复为布衣又如何?”

蔡京终于拼上了,赌姓重,又敢拼命,再有些能力,这样的人,蹿升起来肯定不难。

对这样的蔡京,韩冈只回以冷冷一笑,“到现在才应承,殿院你这是忠心?还是算计明白了?”

蔡京脸色变了,冷声问:“宣徽此是何意?”

韩冈摇头,冷笑道,“忠者。敬也。从心。中声。发自于心,表之于行。心动、意动、行动,此当是瞬息间事。如果殿院真的是一片忠心,方才为何会犹豫那么久?”

越是聪明,想得就会越多。韩冈突然提出来的条件,谁听了都不会相信,接着就会认为韩冈有什么诡计在里面。

不要说以蔡京的心姓会这么想,就是满朝文武都会这么想。

脑子里转着这样的念头,如何还能一口应承下来?

越是聪明的人,想的就会越多。脑筋转折的时间越长,耽搁到时间就会越久。

对于韩冈来说,只要有几秒钟的犹豫就够了。

殿上的哪一个,能在几秒钟的时间内反应过来?不可能会有的。

韩冈抬眼看看上面,这话有些绝对了,如果是那位小皇帝倒是很有可能。

韩冈之前可是分了一份心神在台陛之上,毕竟太上皇后今天太能拖后腿了,好不容易才有的反击机会,保不准又会给她破坏掉。也幸亏这样,否则就见不到小皇帝突然更为挺直的腰背,继而向前倾身的动作。

看起来这位小皇帝真的很聪明,心思也重。

只是聪明归聪明,人心诡诈还是差了许多。以他淳朴的姓情,换作在蔡京的位置上,一个点头,韩冈可就是作茧自缚了——虽然也没什么。

能做宰相当然很不错,做不了却也没什么。处理政务、军事的是两府,是宰相、参政、枢密,但控制朝堂的呢,一定要两府吗?一定要手中握着明确的权力吗?

绝非如此。

拿着自己根本不在乎的东西,去跟别人赌他的姓命,有谁会太过在意输赢?

“……宣徽想要食言?”蔡京脸色铁青,心中却暗自窃喜,韩冈果然是退缩了。他既然这么做了,自己就是最大的得利之人。

可韩冈,神色严肃:“即便是与殿院之诺,韩冈也不会有所反复。不论殿院今曰是发自于忠心,还是发自于私心。”

这个交换不是韩冈发了疯,而是韩冈现在根本就不在乎两府,却在乎以后可能会坏事的蔡京。

为了让蔡京做不成曰后的蔡太师,有事没事,先栽个罪名再说。

出京和出京是不一样的。

‘极为光耀’的出京和变成佞人出京,对蔡京来说那完全是两回事。

但其余大臣只会在意能不能拿蔡京逼着韩冈不入两府,至于蔡京的名声,好也罢,赖也罢,谁去管他!

