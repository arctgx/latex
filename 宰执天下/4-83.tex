\section{第15章 焰上云霄思逐寇(14)}

曾经肆虐城外的贼军,都已变成了战利品,首级用盐腌了之后放进了仓库,等待经略司派人来点验,而战线也稳定在昆仑关,但宾州城的紧张气氛没有得到缓解,并没有恢复正常的景象。

城门现在一天依然只开启两个时辰,内外进出的搜检也依旧严密。在四座城门上,都挂着装着人头的小木笼子。自从韩冈传回严查手持度牒进出关卡的交趾细作,宾州城门的检查工作就没有放松过。

这些天下来,奸细被杀了二十多,其中少不了有冤枉的,但其中几个得到确认的,就让宾州城内的百姓双手支持将眼下严密的搜检工作继续保持下去,直至交趾人撤回国中。

住在昆仑关边,宾州城内的居民都很清楚南面的那片山岭,无法阻挡真正有心穿越过来的敌人。当领兵出援邕州的韩运使,在交趾兵的追逼下,被迫退回昆仑关的时候,人人都在担心他能不能守得住那座并不坚实的关口。更重要的是,宾州城离着最近的山林,仅仅只有五六里,说不定交趾贼军什么时候就从山中冲了出来。

宾州城单薄低矮的城墙,给人以虚假的安全感,这些日子很少再有城中居民愿意离城出外。现在进出城中的多是挑了柴禾菜蔬进城贩卖的农民。由于下雨的缘故,更因为宾州城外的村庄前日遭了大劫,这些天,柴草菜蔬的价格水涨船高。唯一值得庆幸的,就是城中不缺粮食,粮价依然保持在正常的水平。

不过连日阴雨的天气,也全然是坏事。城内的上万军民都在盼着交趾人早一点退军,好恢复旧日稳定的生活,连着下了几天的大雨,不少人都觉得再继续下个几日,交趾人不想退也得退了。

而黄元也是这么在想着。

因为是韩冈的命令,他率领一千族中儿郎来到宾州城,作为守军的补充。

一个是因为粮草。以昆仑关的规模,在关城中驻留下两家的兵马没有任何问题。但前面向归仁铺运送粮草的牲畜虽说都是从宾州城中搜罗来,可粮食则很大一部分则是由昆仑关运往归仁铺,而那些粮草在撤退的时候全都丢光了——且还因为雨水的缘故没能烧起来——当全军回到昆仑关,关城里的存粮就显得有些少了。

当然,韩冈更多的还是要提防宾州被突袭。交趾军前后夹击关城的确是最后可能的情况,突袭宾州城同样能起到打乱昆仑关城守御的作用。用兵贵奇,有雪夜下蔡州的李愬为先例,要说韩冈、李信这样历经战事的将帅会考虑不到这个问题,那也愧对了他们读过的那么多兵书战策。

韩冈派来的援军,倒是很受宾州城中的欢迎。虽然他们跟前日在城外杀人放火的贼人,都是来自广源州。但黄元他们既然已经弃暗投明,加上交趾军正有着攻打昆仑关的打算,有这么一千人守备城中,还是能让宾州百姓安心不少。

黄元穿着一领韩冈赐下来的盔甲,很是骄傲的站在宾州城的城头上。城楼挑起的飞檐挡住了直扑而下的风雨,看起来还是要下个几天的样子。

黄元很感谢韩冈只带了八百兵来,要不是韩运使手上的兵力不足,也不至于这些好差事都能落在他们这些刚刚归顺的广源蛮身上。让他和他的兄长都在这一场战争中立下了战功。

黄元很珍视自己得到重用的机会,不论是白天黑夜,刮风下雨,他也照样一丝不苟的执行着韩冈的命令,就算是让他们冒着风雨来到宾州助守,军中的些许怨言也都被黄元强力压制下去。

天色仍是昏暗的,就算站在一丈多高的城墙上,在雨幕中也望不了多远。城头只有一队队绕城巡逻的守军,城门处则是有些杂乱。

两道鹿角拦在城门前,城门又只开了半边一条缝,仅留下容许一人通过的窄路。想要从窄路进城出城,都要经过严密的检查,免不了要为此耽搁许多时间。不过如果有人敢于为此闹事,格杀勿论,城头上吊着的一排首级中,就有两个这样的蠢货。

黄元就守在城西,他放在这里的士兵是最多的。城南直接通向昆仑关,并不需要太担心。如果交趾人从山里出来,最有可能就是来攻打离山林同样近的西门,而城东也同样离山不远,那里也是重点之一。至于城北和城南,受到攻击的可能性都要小一些,两处的兵力也稍少。不过放在城中还有两百人的预备队,必要时也可以去急救。

黄元自认为这样的布置应该是不错了,遣人回昆仑关的报告,韩、李、苏三位也没有说不好。就算一下有万人来攻,他也能抵挡个一时三刻。

手握刀柄,他一时踌躇满志。困于小小的广源州哪里算是英雄,他并不是长子,没有继承大首领的权力,与其等着父兄分他百十个部众,做个小小的洞主,还不如投入大宋官军之中,见识一下大宋的富丽繁华。

正在为未来浮想联翩,背后突然吹响了告急的号角,那是从城北传来的。缠绕在黄元脑海中的美梦,被惊慌失措的号角击碎。他顿时清醒了过来,是敌袭!

黄元脸色突变,但他的第一反应并不是领兵救援,而是向城下用着自己能发出的最大音量在吼叫着:“关门!快关门!”

在惊急之中,他一时忘了说官话,但站在城门前的士兵心领神会。这时候,上面只会有一个吩咐。告急的号角不论在哪里吹向,城门都得立刻关闭,以防受到贼军偷袭。

丢下手边的差事,回身就窜回城门中,沉重的大门从内被关闭。拥挤在城门前等待入城的百姓,同样大惊失色,纷纷冲向城门,只是他们慢了一步,压了一条缝的城门一下就阖了起来,咚的一声闷响从门内传出来,连门闩都给合上了。

就在被堵在的城外百姓哭号声中,城北的号角声再一次响起,声音更显急促。黄元脸色变得更厉害,提刀下城,从西门处的守军中点起两百人,往城北赶去。西门这里还有三百人驻守城墙,他也不担心会有什么意外。

可黄元离开不久,就在西门外,约莫四五百人呐喊着从雨幕中冲了出来。手上还有十几架简易的长梯,向着城墙直扑过去。

李常杰用金银财帛和封官许愿,从一万人中挑出了七百兵,顶着狂风暴雨穿过了山间小道。

对于大军行动,气候是个大问题,不是第一流的将帅,绝做不到率领部下在风雨大作的时候行军打仗。可换作是少数精锐的奇袭,天候的影响却能减低许多。突袭宾州的几百人虽然少,但能在风雨中通过草木森森的山林的他们,其战力也是第一流的水平。

突袭宾州城的行动的确是个冒险,但冒险并不是百分之百的会失败。

……………………

下雨天,大部分人的心情都会低落起来,也很少会有人会喜欢冒雨出外。守在关中,有建筑遮风避雨,可留在城外的交趾兵,他们的住宿条件,可不会太好。

“今天交趾人的骑兵已经冲到了关城前。要么李常杰是打算孤注一掷,要么就是他要撤军了,防着我们追杀出去。”

已经下了四天的雨,眼见着交趾人快要待不下去,韩冈想瞅准机会,给李常杰好生送一送行。李常杰在广西杀人放火,没有礼送出境的道理,肯定是要打上一场。

“李常杰驻扎在山中驿站,他前日又是追着我们身后一路赶过来。他帐下的士卒能随身携带粮草,最多也就三两天的份量——就是我们走的急了,没看着那把火生起来——光凭手上的干粮,现在就该断粮了。想要在雨中支撑起供应过万兵马日常食用的粮道,对人力的消耗可是个大数目,李常杰坚持不了几天。”

“如果他当真撤军,就可以追杀回去,交趾兵士气低落,他们挡不住官军!”

“就是神臂弓是个大问题。”李信叹着。

这些天来雨水不断,湿气过重,使得弓弩的威力大减。神臂弓发射时的声音都是软绵绵的,完全不见正常发射时铿锵有力的弦声。以檿桑为身、檀木为弰、麻绳扎丝为弦的神臂弓都被湿气侵透,失去了该有的威力,用牛角、牛筋和牛皮胶的战弓更是一点力道都没有了。

“没有弓箭,难道就不能打仗了?”韩冈反问,又道,“追杀贼军,用得上弓弩的时候也没多少。”

“万一李常杰打算孤注一掷呢?”苏子元问着。

“除非交趾兵已经绕到我们背后来了?要不然,李常杰疯了才会在这时候就出来攻打关城。”韩冈摇摇头,笑了起来。

但只过了片刻,他脸上的笑容就无影无踪,“交趾兵绕过了昆仑关?!”

派在山里监视敌踪的哨探跪在下面头也不敢抬:“回运使,他们没走小人几个巡视的道路。只是今天早上,看到了出山的地方有人马经过的痕迹才发现。大约千人的样子,从方向上看是往宾州去的。”

“有没有通知宾州?!”苏子元急问道。

“已经有人追过去了。”哨探的声音低了点,“就不知道能不能赶得上。”

李信紧紧咬着牙,这算是一个大失误,想不到李常杰竟然当真孤注一掷,真是疯了!不过李常杰派过去的应该的确不到千人,要是兵力过千过万,就算穿行的是荒僻山野,也没有发现不了的道理。

苏子元立刻对韩冈道:“运使,要立刻派人去救援宾州,迟恐不及!”

韩冈皱着眉,从昆仑关到宾州距离并不算短,如果直接冲过去营救,跑到半路就没有力气了。

苏子元三人都在看着韩冈,等他做出决断。

“伯绪,我给你两个都的荆南军。黄洞主,你领两千本部,一同去援救宾州。这一路宁可走稳一点,也不要中途吃了埋伏。”韩冈嘱咐着,这时候再也不能乱,“偷袭宾州城的交趾兵绝不会太多,就算城池被攻破,他们也压不住城中的反抗,更守不住四门。只要稳扎稳打,宾州城即便丢了,转眼就能夺回来。”

“唯命。”苏子元站起身,抱拳行礼。

黄金满则是单膝跪下:“请运使和都监在昆仑关中静候我等捷报。”

“不,我们要准备出战。”韩冈同样站起身,“不论攻向宾州的那一队交趾兵怎么打,如果李常杰这边配合不上,一切都是无用,肯定会攻过来的。”他充满自信的笑了,“我军养精蓄锐多日,士气正旺,宾州小乱,也影响不了军心。任凭李常杰计谋百出,也照样得丢盔弃甲!”

