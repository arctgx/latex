\section{第11章 飞雷喧野传声教(二)}

邮政的推行工作正在有条不紊的进行中。

而且由于韩冈就任参知政事,进度甚至更快了几分。

在地方而言,眼下的准备工作不过就是确定,然后给城市里各厢坊中的住户钉上门牌号码。至于邮局和邮递所的安排,那还要放在后面。

这样的准备工作,对收税也有帮助,在地方上没有什么反对的意见。只有少数几个州县官觉得是扰民,写了奏章上来要为民请冇愿,太后没理会他们,而韩冈也不管他们真的是糊涂,还是懒病犯了,直接发文申斥。

再过段时间,韩冈还准备向各路派去邮政察访使,不需要地位多高,只要他们对各地邮政筹备和推广工作进行督促和考察。就像当年王安石为了推行新法,派人去地方督促和检查新法的推行情况一样。作为参知政事,韩冈找得到足够的人手来帮着他。

与王居卿说了几句扯偏了的闲话,韩冈便招了王居卿和方兴一起上车,继续向前方一里地外的火器实验基地行去。

片刻以后,一行车马人等穿过寨门,停在了寨门后的校场上。

王居卿被人扶着从车上下来,环顾左右。营区的东南角,是方才在外面就看到的内堡,另一侧的几处建筑,由于旁边堆着木头和砖石,应该是仓库。除此之外,整个营地之内,就没有其他建筑了。

不过在向北的位置上是厚实的土堆——甚至不能叫做土堆,而是土坡,一直堆到了快要跟三丈高的寨墙平齐的位置上。而土坡之前,竖着一块块木板。那里是实验火炮的地方。

上来将马牵走的士兵,一个个都操着浓重的关西口音,看外形,也是高大粗犷的关西大汉模样。

王居卿在就任判军器监后,便知道了这里的详情。

这一处火器局外院,本是一处旧军营。原本驻扎在这里的军队,早就移防京西,枢密院只安排了一小队人来看守——在驻泊制代替更戍法之后,开封府界中类似情况的旧营垒有不少处——由于火炮实验的危险性和特殊性,之前为了给火器局一个保密的试验场,朝廷特地将这座军营划拨过来,为防有奸人窥伺,这些天还突击维修了一下破损的寨墙和营垒。

守卫这里的都是从关西调来的禁军,一方面希望以此来减少守军与外界的交流,一方面也是尽量多保留一些有着卓越战绩的队伍——这当然是韩冈的提议。而且为了保密,周围两里内的人家都给迁了出去。尽管开封府事前是特意挑选了一处地广人稀的去处,但这一片旧牧场也有两个小庄子,一百来户人家。

不过王居卿不知道在寸土寸金的开封府,到底是用什么手段将这里的百姓给迁走的。

“以前驻屯在这里的禁军,是侍冇卫马军司的一个指挥。”

韩冈的一句话,让王居卿顿时明白了,恍然道:“原来是牧监的地!”

“确切的说,是配属马军的放马地,不归群牧司管。”

其实都一样。

因为这些官地都给私吞掉了,而且吃大头的无一不是有权有势,想要虎口夺食,难度可想而知。

王居卿就越发的纳闷了:“那到底是怎么做的?”

“这就多亏了沈存中。”

在太祖太宗的时候,京畿一带地多人少。为了养马,直接就划了二十四个牧监,这还是大的,归属群牧司。而配属给马军的牧场,就是各个指挥的私有地,不过随着时间的推移,这些牧场基本上都给侵占光了,此处虽是适合牧草的沙土地,也一样不例外。

原本就是附在军营外的军用地,全都是属于官产,后来军队移防才为人侵占。虽说这些土地已经被耕种多年,但从地契上却是官府的地,朝廷将之索回名正言顺。而且还是给了一定的补偿——一亩一贯的现钱,或是八亩换一亩的代州无主田地。

如此补偿,在收回的过程中,官府理所当然用了点强制性的手段,也理所当然的闹出了些乱子,同样理所当然的,这里的地冇主颇有几个皇亲国戚,带领着当地的村民,将事情闹到了开封府。可想而知,开封府若不能满足他们的要求,下一步就是太后那边了。

正常情况下,开封府绝不会为军器监的事兜底,官僚嘛,遇到麻烦总是往外推的。可偏偏当时新上任的开封知府,姓沈名括。

沈括当时才上任,正想展露一下才干,也是因为关系着火器局,不过他没有麻烦到韩冈——以才干来说,朝中能比得上沈括的也没几人——而是很聪明的选择分而治之。

拿出预定的补偿款,在不远处八角镇上西太一宫附近的沿街官地上盖出了两排二层门面房,下面可开店,上面能住人,以此来利诱一干拆迁户。

任谁都知道,京冇城附近的镇子上的铺子有多金贵。远比百亩田地都值钱,尤其是八角镇这种位于主道上又有驿馆的大集镇,而沈括盖出来的铺子数量只有十一二间,加之沈括还暗地里收买了最穷的两家人领头,很快就带动一批亲近的邻居,人多粥少下,立刻就被争抢一空,让没抢到的人扼腕叹息。等到这一批抢完,他又在稍稍偏远点的地方盖了十套房子,又安置了十户人家。

两次下来,最后什么都没拿到的民户依然占了大半,但经过了之前的房屋分配,大部分也不再闹了,等着开封府的好处。剩下还在闹腾的,就是那几位自恃地位、又瞧不起开封府给出的好处的皇亲国戚们,以及以他们马首是瞻的十几户。

其中有一家地冇主还是宗室,而且是郡公,太宗的后人,论辈分是当今天子的叔祖。仗着自己皇亲国戚的身冇份,遣了奴仆将开封府派去的吏员打了出去。沈括得知之后,没有多说废话,联络了方兴直接去告到太后那里。次日这位郡公便给削了爵,直接降到了县伯——火器局事关军国重事,谁敢阻挠就是往刀口上撞,很快土地本也不是他们的。

有了这个例子,立刻就没哪家的皇亲国戚敢再多纠缠,看到皇亲国戚如此,其余百姓也不敢在闹了。但这时补偿款都用在修建屋舍中,已经没钱给补偿。无奈之下,他们只能选择了代州的那些无主田地。而有了如此明显的对比,整件事在民间,就变成了贪心吃亏的典型,倒是没人为其抱不平了。

王居卿知道因为廷推一事,沈括与韩冈有些心结,但韩冈能如此平和的赞扬沈括,想必那些传言只有三分能信。

“原来如此,沈府尹在治才上的确难得。”他点头道,“但占了官地,还能有补偿,这未免也太好运了一点。”

几年前修开封城墙,拆掉的民宅也不是一间两间,清理的坟茔数量更多,而这一回给熙宗修山陵,迁出的陵区民户绝不会比眼前的这一处要少。那些还都是他们自家的地皮,而这里更是官产,朝廷的手也未免太松了。

“谁让他们占得时间长了?”这其实是韩冈的主张,不能让贫民吃亏。至于富民,尤其是皇亲国戚,吃着喝着都是朝廷的,又占了这么多年冇便宜,也该知足了,“有的人手中的地,都是经过了好几次转手,虽没红契,白契却是都有。买了这些地的百姓,花的是真金白银,总不能让他们吃亏。”

不论盖着官印的红契,还是没改官印的白契,都有法律效力,不过两者相冲时,以前者为准。但世间有很多人做了买卖后不愿付那笔契税钱,所以还是以白契为多。

“还是别说这些了,不能再耽搁时间。”韩冈看看天色,他过来可不是在议论怎么拆迁的。他问方兴,“都准备好了吧。”

“已经准备好了。”方兴点头,然后在前引路。

韩冈和王居卿是过来视察火器局新试验场的运作情况,在干掉了郭逵家的正堂后,朝廷终于决定将火炮及所有火药武器的实验,彻底搬出了京冇城。

开封府加派了千余名厢军,用了两个月,终于将这里修整完毕,移交给了军器监。

火器局的实验部队入住此处有半个月了,工作应该上了正轨,正好是过来看一看的时候。

试验场的靶子,就设在那座土坡前。

这座两丈多高的土坡完全是用麻袋装土堆成,千多人背起麻袋来垒砌来只用了十天,同时还在营寨外围添了一道壕沟。

配发步军指挥的虎蹲炮,如今已经定型,马上就要开始量产。现在正在这里进行大规模、高强度的测试。

虎蹲炮的结构很简单,直接就可以铸造成型。加上火药和炮弹定装,使得速度提升,是完美的步兵野战武器,使得这种超轻型的火炮,比起名气更大的野战炮、城防炮,更早一步开始量产。

十八门虎蹲炮在地面上一字排开,相距只有两丈左右,前面四十步外就是一块块木板制成的标靶。

韩冈过来之前,这些虎蹲炮正在试射,前方的靶子早如蜂窝一般,标靶之后,更是满地铅子。韩冈、王居卿他们一路缓缓行来,到了此时,还能感觉到之前炮火留下的余温。
