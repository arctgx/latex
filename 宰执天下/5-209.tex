\section{第24章 缭垣斜压紫云低(一)}

出宫的时候开始下雪了。

不是鹅毛般的雪片,或是柳絮一般的细雪,而是一粒粒的冰渣子,被横过御街的劲风一把抄起,然后狠狠的砸在脸上。

风雪扑面而来,韩冈皮糙肉厚,早惯见了风霜。摘下手套,用力搓了搓脸,便浑若无事的顶着风雪驭马前行。

御街两侧千步廊内的灯火在风雪中忽明忽灭,让空荡荡的廊中更显幽暗。宽达百步以上犹如广场一般的御街,也笼罩在黑暗之中。宣德门城楼上如星如月的灯火,也穿不透风雪拉起的幕布。只有离宫回家的官员和他们的随从一队队的提着灯笼,照亮了周围的一小片地。韩冈环目四周,这样的情形似曾相识。

“明天看起来要更冷了。”

薛向说话时带着浓重的鼻音,将斗篷又裹紧了。寒风直往他衣襟里钻,恨不得连头到脚都给裹住,只是要跟韩冈说话,没好意思将口罩也戴上。

韩冈仰头看了看天:“雪要是再大一点就好了,今年冬天,京城这还是第一场雪。”

因为入宫问对耽搁了一点时间,回编修局后不得不多费了一番功夫将今天的工作给完成,等到再从太常寺中出来已经过了黄昏。赶在在皇城城门落锁之前出门,却撞上了巡视地方才刚刚重新回朝的薛向。

“那还真得多下点雪,去年京城这里十月就下雪了。”

“去岁河东也是连番暴雪,太原府还被雪压塌了一些房屋,不过今年过来便是一个丰年。”韩冈说道,“这场雪下得大一点,明年当也一样能是丰年。”

“若是明年又是丰年,可就是连着四年丰收了。元丰这年号可也算是名副其实。”薛向笑着,脸上的皱纹更深了几分,“想起熙宁七年、八年的时候,真是恍若隔世。”

“……嗯,的确如此。”韩冈又想起了那一年帮着王安石与旧党过招的日子。同样是在郊祀之年,但早已是物是人非。想一想,也不过过去了区区六年而已。

这六年的时间,辽国两个皇帝驾崩,西夏灭亡了,新法的地位稳定了,旧党在外苟延残喘,不过在高层中,真正的新党也变得寥寥无几,最后的胜利者是当今的皇帝。而韩冈,则是从一介京城知县和提举诸县镇公事这样的中层官僚,成为了真正的重臣。

这几年的天候仿佛是要对之前几年的灾害进行补偿,各路连年丰收,官仓收之不及,米价几乎被打压倒了最低点。

“有天子圣德庇佑,当真是天下之福啊。”薛向正说着话,突地又是一阵寒风掠起,吹得他手足冰凉,不禁打起了寒战,“真是够冷的。”

薛向在马上冷得发颤,一张斗篷遮不住全身,身后张起的清凉伞也不能遮风挡雨,反倒差点将举着巨伞的元随给刮翻掉。

韩冈偏过脸看着薛向在寒风中瑟缩的样子,道:“枢副是不是穿得少了点,这个天气受了寒可不好办。”

“不能跟玉昆你比身体,不过多喝两杯热酒就没事了。”薛向扯起冻得发僵的嘴角,勉强笑道,“听说官家冬天最喜欢喝的便是杨梅酒,醇而不烈,只是得从两浙运来。”

韩冈也知道赵顼喜欢杨梅酒,宫里面的嫔妃对于各色浸了鲜果的烧酒都很喜欢,正如薛向说的,醇而不烈,有的还因为放入白糖而使得口感更好。但烈酒就是烈酒,喝多了一样会醉人,而且因为口感好,感觉不到烧酒的刺激,更是会让人不知不觉中喝过头。说实话,如今北方酗酒的问题已经远比烧酒出现前要严重得多,尤其是在冬天,各大城市都时常见到喝多了而倒在路边冻僵的尸体。

不过小酌几杯倒是无妨,韩冈邀请薛向道:“枢副若不嫌弃,不如就由韩冈做东,在前面的夜市中喝上两杯如何?朱雀门下王家现烤的旋炙猪皮肉,还有梅家刚出炉的鸡皮、鸡碎,配着热过的水酒,倒是正合适这个天气。”

薛向侧脸望向韩冈,不过在暗弱的灯火下却只能看到一幅剪影。挺直的鼻梁直透山根,线条刚硬,从面相上说,当是心智坚毅不为任何事情所动摇的人物。如果在光线明亮的地方,韩冈脸上随时随地都带着的温文笑意,好歹能冲淡了一点面相给人带来的坚硬执拗的印象,但在此时此地,当外在的伪装被黑暗掩去,韩冈的本性反倒更加清晰明了的呈现出来。

在薛向眼中,这是最让人觉得头疼的类型。就像当年的王安石,也像曾经一直盯着他不放的几名御史。幸好配合这种性格的,并不是如茅坑里的石头一般又臭又硬的头脑。

“玉昆即有兴致,薛向哪有不奉陪的道理?”薛向立刻点头,也不在乎夜市上三教九流混杂,哈哈笑着:“听了玉昆你的话,馋虫都出来了。”

出了宣德门,沿着御街一路向南,经过大约两里的路程,便有一片灯火密集如星海。御街经过州桥跨越汴河后,穿过内城南门朱雀门直抵龙津桥前,长约一里的路段上,便是赫赫有名的州桥夜市。

御街热闹的只有早市,到了夜里就轮到南面一点的州桥了。每当黄昏过后,州桥夜市便热闹起来,各色摊铺百十家,各色杂嚼【小吃】琳琅满目。不过乍起的风雪,让今夜的客人比往日少了近半。许多摊主甚至都还没开张,望着白茫茫的夜雪发着愣。

韩冈和薛向沿着御街跨过州桥一路过来,没人多看他们一眼,从州桥出内城的官员多了去,谁会费神注意他们。

只不过当他们在朱雀门下停下步子,明显是领头的两名身着紫袍的贵人随即下马,所有的摊主和客人都愣住了,人数不及往日多,却依然热闹着的市面陡然间安静了下来,无数道目光切割过风雪交加的空间,落在两人的脸上。

几乎没有朱紫高官愿意在人流溷杂的夜市上吃喝,倒是衣着青绿的小官和吏员,在这里吃饭时候比较多。虽然这两队人马不知何时收起了灯笼和旗牌,不想让人看出身份。但浩浩荡荡的元随队伍,有点见识的人都知道,这两位少说也是两制官以上,甚至更高。

韩冈和薛向都不在乎周围人的眼光,下马后走了两步,就直接在王家从食的摊子上坐了下来,四周的摊位和桌面,便立刻都给两人的随从给占去了。原本的客人,一见到他们的这番声势,随即结账远避,不想惹起无谓的麻烦。

“店家。”韩冈不待元随出头,自己先一步招呼着店主,“旋炙猪皮肉挑顶好的给我上四份,煎夹子、猪脏各两份,烧酒也来先两壶的。这天冷得够呛,要快一点……啊,可别掺水!”

店主带着颤音的高声应答,让店铺里的小二去舀酒烫酒,自己则忙不迭去挑已经渍好的大块带皮猪肉去炭火架子上去烤。

店里的人看上去虽然有些慌,但动作还算麻利。韩冈点点头,随即招来一名元随,让他去梅家铺子,去买鸡碎鸡皮腰肾之类的杂食来下酒。

“……批切羊头,姜辣萝卜,梅子姜、莴苣笋也别忘了都来点。”韩冈自自在在的吩咐着,完全

再一看周围,他和薛向的元随们的或站或坐,在外面围了一圈,却没有一个要点菜的,把周围几家铺子的生意都耽搁了。便又道:“其他人自己点,别空占着座位。”说罢,向韩信比划了一个手势,让他去负责。

衣服和脸都在灯火下闪着一层油光的王家从食的店主面对着蓝汪汪的炭火,记挂着身后的两名显贵,心里面直发慌。

方才那名年轻的官人点菜的时候,乍看上去便是常来常往的熟客,甚至王十三当真是依稀觉得面熟,曾几何时来店里坐过。但那身服饰,无论如何都是从来没有出现在这件铺子中的异类。

两人一个已入暮年,一个则正当年华,年岁相差得很远,但都是金紫罩身。身上的紫色公服,腰间的金丝犀带,无不在提醒人们,他们身份上的高贵。可是两人坐在这看上去甚至有几分腌臜的铺子中,却没有任何别扭的神色,自在得就像是坐在樊楼的三楼上,饮着眉寿酒,听着花魁唱曲跳舞一般。这样的气度,他还从没有见过。

滚开的熟水里煮着洗净后的碗筷。多人共用的碗筷如不用滚水消毒易传染疾疫,经过厚生司的一番宣传,已经在京城中人所共知。就算是因此而大幅增加了炭火上的成本,也没哪家食铺敢于懈怠一点。或许过些年,食客们的神经会放松一点,但在牛痘法.正普及于世的现在,厚生司在卫生防疫上的发言,世人当成圣谕一般遵从。

王家的小二从滚水中瓷碗和酒盏里专门挑了没有被磕碰出豁口的两件,又抄起了两对筷子,用盘子装了,连同已经烫好的热酒,一并送到了韩冈、薛向的桌前。

在顶棚被熏黑的架子上扫了一眼,又瞅了瞅远处向铺子内偷偷张望的人群,薛向笑道:“今天玉昆你我在这铺子里一坐,明日乌台恐怕就又有事可做了。”

韩冈微微一笑,抬手给薛向倒酒:“债多不愁,他们说他们的,我们喝我们的。”

薛向仰头一阵笑:“玉昆说得好,债多不愁,任凭他们去说好了。”
