\section{第44章 一言镇关月燎辉(下)}

其实不需要王中正提醒,天子随意更改诏令的情况很常见,莫说韩冈,大部分的官员基本上都明白。什么金口玉言,什么君无戏言,都是说着好听而已。

周公能逼着成王将错就错,桐叶封弟。但到了唐朝时,就没人信了,柳宗元还为此扯了一通。换作是现在,朝中的臣子们是更进一步,不把天子做的错事拧回来,那是绝对不会善罢甘休的——不管是真的错了,还是在他们眼里觉得皇帝错了。

要不然为何不论大事小事,朝野之中的大臣们都喜欢一封接一封的上书。那就是要用洪水一般的文字,用着更响亮的声音,把皇帝给洗脑。

深宫妇人之手养大的皇帝,要是能如王安石、王韶还有韩冈这般在红尘中久经历练的官员一般,性格坚毅如钢,不为外事所动,反而是一件不可思议的事——当然,对于臣子们来说,固执己见的皇帝也会很让人头疼的。

如今的天子赵顼就是一个典型的例子,比起他的父亲要差了很远——直到三十多岁才确认了皇储地位的英宗皇帝,他行事就稳重许多,毕竟在宫外的风雨之中待了几十年——尤其在军情上,往往听到风就是雨。

弃守罗兀的事就不说了,足够赵顼后悔七八年。从去年底熙河经略司这里的临洮之战开始,体问军情的敕文、手诏从来都没断过,事无巨细,都要过问。而且还爱对战事指手画脚,每次的作战计划都要事先呈上去。河州之战前,还送了幅阵图来,说是要让王韶在河州城下这般布阵——那份阵图倒是没有直接就给丢到架阁库中去,王韶还是带在身边,不过也仅此而已——太宗皇帝的爱好隔了几代,倒没人当回事了,赵光义所拥有的权威,并不是赵顼可比。

话说回来,韩冈将李宪带来的诏书给隐了,甚至伪传诏令,蒙蔽了下面的官兵,这个罪名也不会小。而且若真的有第二道撤军诏令传来,韩冈自问肯定再难顶住,也做好了最坏的打算。

他已经准备好要让天子像弃守罗兀一样后悔的手段。

——如果在没有外敌的情况下,将已经到手河州,甚至熙州给放弃,韩冈倒想看看赵顼会有多长时间睡不好觉。反正巩州不会让出去,只要保住陇西、渭源一线的根基,也足以让大宋在几年后卷土重来。

“来人!”

用着伪传的诏令安抚下麾下将士,韩冈回到官厅,匆匆写下一封手书,交给了领命而至的亲兵,“速速送到王都巡那里去,让他依照事先商量好的方案去做。”

亲兵接过信没多话就匆匆出门去了。

王中正却正好过来拜访,回头看着行了一礼后就离开的亲兵,神色就变得古怪起来。

“不知有何急务?”他问着。

“临洮堡熬了这么长时间,也到了动手的时候——不好再拖了,也不需要再拖。”韩冈并不打算瞒着王中正,过一两天,也就会传开了。

王中正一听,就立刻上前一步:“可有把握?”

“战事难以逆料,如果能继续与西贼对峙下去,其实缺乏粮草的他们必然会不战自退。”看到王中正欲言又止,韩冈笑道,“但六七分把握还是有的。只要临洮堡那边能退敌,至少还能保着熙州的。”

韩冈已经可以确定西夏人那边的粮草已经撑不住了,熙州北部的坚壁清野的绝户计早在一年前就开始施行,再出色的名将也变不出粮食。王舜臣如果真的出击,甚至不需要跟仁多保忠和禹臧温祓决战,只要他能保着一队人马进入临洮堡,围城的西贼就不会再有半点士气。

对于韩冈的决断,王中正倒是有些信心。点着头,“那在下就等临洮堡的捷报了。”王中正说着坐下,沉默了片刻,便唉声叹气起来,“要是王经略那里早点有好消息传来,那就更好了。”

不像王中正被忧虑所困扰,韩冈的想法是一回事,但他说出话却十分的乐观:“没有消息并不一定是坏事。好消息没有,但坏消息其实也没有啊!”

王韶至今渺无音讯是很奇怪的一件事。如果他败了,应该会有败兵返回。如果更进一步,是全军覆没,那回来的就该是木征。但到现在,都是什么都。韩冈只能猜测是木征和王韶两边都陷在了露骨山中,或者是突然之间,露骨山路变得不好走了,让军情一时无法传回。

不论是何种情况,前面韩冈都已经移文河州,请苗授再加强露骨山口的防卫。至于姚兕、姚麟两兄弟,据苗授所言,是以结河川堡的安危,作为撤军与否的关键。只要今次诏书中的真实内容不传到他们的耳中,想必他们两人也不愿放弃已经落到手中的功劳。

做好了应对的准备,‘现在就等第二道信使来了。’韩冈想着,不来最好,来了他也能设法让自己脱罪。

而到了五天后,王厚连夜送抵狄道的密信终于又立功了。宣诏的使臣的确有来了一波,从离开东京城的时间上看,他们其实就追在李宪之后,只差了一天而已。不过不像李宪一路加急而行,仿佛是急脚递一般。今次宣诏的使臣就稍稍慢了一点,照着比正常略快的行进速度前进,还在渭源堡歇息了一夜。同时是早早的就派了人来,让韩冈出城迎接。

从王厚的信中,宣诏使臣的人选明确了——吕大防,曾经的韩绛帐下的宣抚判官,横山攻略中,与韩冈同在韩绛的幕府之中。这是是个正人君子,他的三个兄弟还是韩冈的师兄。本人熟悉兵事,而且更是文官,地位犹在韩冈之上。

宣诏使臣的人选是有特定含义。李宪是中使,夺文官之权是件犯忌的事,天子不会在诏书中让李宪来顶替韩冈的职位,最多也只会给他一个体量军事的权力。而选了曾经在宣抚司中担任判官的吕大防来宣诏,情况就不一样了。他有绝对的资历和能力,来取代韩冈,更不会惹起文臣们的反弹。

不过王厚却又在信中说明,吕大防的任务并不是夺权。诏书的内容王厚已经提前得到了——在诏书中,熙河路的指挥权将转交给蔡延庆,而蔡延庆眼下正因为德顺军的战事而焦头烂额,所以不知怎么回事,却是变成了由秦凤路转运判官蔡曚来接收韩冈的职权——王厚能得知诏书内容,也全是靠了蔡曚在陇西城的一番宣扬。

转运司衙门中的大菜小菜并不和睦,这一点就算是包顺包约这样的蕃人都知道。韩冈不知蔡延庆是为了什么而将蔡曚给丢出来担任接收大员,如果是嫌着他在秦州太碍事,而特意找个理由踢出来,那蔡延庆就做得真是太过分了。

……………………

已经在城外守了不短的时间,李宪好不容易才看到姗姗来迟的韩冈。

韩冈比预定的时间迟了有半个时辰才到,李宪觉得很是纳闷。同为宣诏使臣,他本不需要迎接吕大防,但因为默认韩冈隐了诏书,行事劳而无功,不得不想后来之人低头,甚至是提前出城来迎接。

此时东方已经能看到一抹尘头出现,韩冈方奔马赶来,差一点就要比天使来得还要迟上一步。

‘大概是因为临洮堡赢了的缘故。’李宪猜测着。

这两天来,李宪已经确认河州、熙州的局势。韩冈隐匿诏书也不是没有道理。

临洮堡得胜,王舜臣不但将久困中的城堡解围,更是派遣了包约领兵,将敌军远远逐离。熙河路已经大体平定,王韶就算再也回不来,洮州被木征控制,河州也照样能安定下来。

王韶带走的是三千人,而景思立全军覆没的也不过两千。加上此前几次战斗,今次河州会战。出战者近三万,连同王韶的三千人在内,伤亡总计也不过一万上下。这点损失,其实分摊到熙河、秦凤、泾原三路后,也不算多少了,三路经略司都支撑得起。虽说伤亡的这一万人都是精锐,但更重要的是多了两万在大战中历练过的将士!

同时韩冈所领导的转运系统,已经充分证明了他们的能力。支撑五万人一下的战事,完全不成问题。对于陕西缘边的崎岖地形来说,一个战略方向上,能动用的兵力充其量也最多五万人。真正论起兵事,李宪比王中正强得太多,他知道要让三百里外的前线保证粮秣充裕,到底有多么难得。

可就算这样,还是一样要撤军。李宪也不免要为韩冈叫屈,他已经做得很好了,却还免不了功败垂成。

如果能继续强硬下去,将吕大防也一般儿顶了,最后说不定就能将眼下的胜果给保护下来。

但李宪更明白,韩冈根本不可能再拒绝第二份诏令。

选了比韩冈高上几级的文臣来宣诏,究竟是怎么回事,李宪很清楚。

韩冈已经失势定了!

望着两边都逐渐向自己靠近的尘头,李宪暗叹着,天子的运气还真是不好。

