\section{第11章 安得良策援南土(二)}

一个多月前,从丰州被党项夺占开始,朝堂上的气氛一天冷过一天。到了今日,交趾兵围邕州的消息传来,崇政殿中的温度已经降到了冰点以下。

皇城中的任何一座宫室,只要天子驾临,夏天就会放置冰块,冬天则要升起炭火,让天子在御榻上坐得舒心。从鹤型香炉中飘散出来缕缕香烟,缭绕在梁柱间,让天子所在的每一座宫阙,都宛如天上仙宫一般。

但韩冈觉得这殿中的温度还是够冷的,而且冷清,尽管人数比平常要多了好几倍。

今日的崇政殿,不再仅仅是五六宰辅加上两制班的十余重臣,而是扩大到了侍制一级,加上几个重要的且有关军事的监司主官,会聚一堂,共同讨论如今要面对的问题。

不过相对于迫在眉睫的紧急军情,难以区分的责任,借题发挥的臣僚,以及愤怒的天子,这个才是更棘手的问题。

赵顼看着满朝文武,雷霆怒意在眼中汇聚,火焰在胸中熊熊燃烧。

谁能告诉他这到底是怎么一回事?!

北面有敌,南面有敌,国中还有内患,而天上的警兆才过去不久,为何一时间出了这么多乱子?哪件事都让人焦头烂额,现在却一起堆到了面前。

北面的战火是自己主动挑起的,赵顼不会为此事而秋后算帐。但事情拖到契丹人都牵扯进来,赵顼又怎么可能不上火?

也就是在丰州陷落后的半个月,西夏就派人上京,说是要拿丰州换罗兀。

赵顼一听,好悬都没忍住将那名浑身带着羊骚.味的使节下旨赶出宫去。不就是仗着辽国已经站在了他们的身后吗?只要辽国还没有正式的传递国书,赵顼可以完全不加理会。以为拿下区区一个丰州,就能逼他就范,未免太小瞧他这位大宋天子。

至少在当时,赵顼还认为丰州很快就能夺回来,运气好一点,说不定银、夏之地也一并到手。所以就把西夏使节晾在了城西的都亭西驿,不去理会。

但紧接着传来的消息,就是种谔攻打银夏不成,只保住了控扼山口的赏逋岭寨;然后是河东的军情,麟府路加上太原府总计接近三万的收复丰州的大军,因车辆不足,难以越过积雪深重的山道,被阻于古长城一线;接着又有淮南、江东告急,说是因旱蝗而流民生,且已现盗贼,恳请将当于今冬发送京城的六十万石粮秣留于本路赈济;最后一击来自于南方,交趾入寇,钦州、廉州接连失陷。

随着这些不利的消息从朝堂上传出去,西夏使节报出来的条件便改成了用丰州交换绥德城。

对,不再是换罗兀城,而是换绥德!

换绥德?这是天大的笑话。一旦绥德还回去,罗兀城当然也保不住,连同横山南麓全都丢了回去。从他登基后的这些年来,在鄜延路的进取开拓,全都化为了泡影。

如果这时候开价依然是罗兀城,赵顼说不定真的换了。但如此狮子大开口,身为大宋天子他也难以忍受,直接就命人将这位会见风涨价的西夏‘奸商’强送出境,甚至连会否将丰州送与契丹,都不去多考虑了。

可是昨日西夏使臣刚走,契丹贺正旦的使节也到了。而且来的是赵顼最不想看到的萧禧。当初几次作为使节来索要土地,萧禧的一张看似敦厚的笑脸,赵顼看得就是咬牙切齿。

萧禧带来了辽主对太皇太后的问候,同时敦促大宋与西夏两家罢兵。依照澶渊之盟,辽主耶律洪基是赵顼名义上的叔叔,而他又将女儿嫁给了夏主秉常,是西夏的国丈。以长辈的身份劝说子侄们不要闹了,这倒是名正言顺的。

只不过这层亲戚关系,仅仅存在于国书中,并没有人放在心上。耶律洪基用来劝说赵顼放弃对西夏开战的,并不是国书或是萧禧的嘴皮子,而是在西京道的兵力调动,让太原府连夜送金牌告急抵京。

而且让赵顼痛心疾首的事还不止如此。两日前,皇五子赵僩夭折在襁褓中。好不容易他赵顼的子嗣才增加到两人,这时候又只剩三子赵俊一个了。

内忧外患,沉重的担子压在赵顼的肩头,让他一时间甚至觉得呼吸都变得艰难起来。而下面的臣子仍旧在争吵,吵得他头疼欲裂。

“交趾小国,自李日尊时起,便疏于朝贡。朝廷念其国小人寡,加以优容。岂料其枭獍之心,不感朝廷恩德,反而干犯天威,凌犯中国。当选良将,起大军,破其城、灭其国,俘其太后、国主,执于陛前问罪!”

这是刚刚入京诣阙的一名侍制在兴奋的叫嚣着战争,但说的话跟没说一样。哪个不知道要对交趾兴兵报复,关键是怎么做!是缓是急,又是该调哪里的兵将,还有交趾入寇的责任又该由谁来负,这些才是争论的要点。吼两句倒是容易,想在天子面前挣个好印象,也不是这么做的。

所以吕惠卿很是嫌恶的瞥了一眼,“调兵遣将,膺惩南蛮,这是应有之理,可当务之急,乃是速调兵马,救援邕州。”

“广西与京城相距数千里,远隔重山。京中接到战报,立发信时,就已经过去了近一个月。如今在京中点集兵马,选派良将,再快也还要一个月,缓不济急。兵法有云,趋百里而争利则厥上将军。有五岭阻隔,不论从哪路调兵,又何止千里之遥?如今的当务之急,不在救援,而在于如何收拾后事,让贼人不敢复窥中国。”

吴充反驳着吕惠卿。又向赵顼道:“陛下。沈起、刘彝贪于边功,接连生事,方致今日之变。臣请陛下将之重责,以儆效尤,并选派精密毅重者替刘彝而任桂州,属之方面,付以便宜,并命其选举部下文武将吏。其本路职司官,朝廷为之遴选,令其协力从事,招集户口,各安本业。待情观便,临事制宜。再发禁军南下,并令募本土丁壮,分屯缘边城寨,使之足以保守要害,更可相于救赴。则贼不敢复窥于内!”

“吴枢密。军情如火,岂能耽搁时日!”吕惠卿厉声说道,“交贼欲壑难填,不论邕州是否可保,王师不至,贼人绝不会收手。王师南下越迟,贼人肆虐越久。广西万千生民,枢密都打算放弃了吗?”

“彗星出于轸宿,此天传警讯。若是早做防备,岂有广西黎庶今日之惨状?!”

听听,一下子就转回到争论这是谁的责任上去了!

赵顼听得心中发恨,直咬着牙。这祖传的异论相搅,跟他要改变的祖宗之法一样,都是临到大事便出乱子。

“如果要救援邕州,当可从荆湖调兵。”韩冈站了出来,在赵顼愤怒爆发之前,说出了他的意见。这其实是他与章惇昨日商定后的意见,“旧岁为定荆南,荆湖南路兵甲皆足。如今荆南平复,潭州的驻军能南调者当为数不少。”

韩冈说着就瞥了一眼章惇。

章惇上前一步:“李信、刘仲武皆为良将,潭州守军亦颇多经历战事的锐卒。”顿了一顿,又补充道,“且荆南为瘴疠之地,从此路调兵南下,不虞多病伤军。”

潭州是荆湖南路的治所,当初章惇领军平定荆南山蛮,就是调发了潭州守军为主力,不过核心则还是从陕西调去的一批将校士卒。如今这些人,一部分回了陕西,一部分被调往他处,但剩下的也为数不少。其中李信、刘仲武都已经飞黄腾达,依靠在荆南的几年战事,皆升到了都监一级。也都是名震南国的新一代名将。

韩冈、章惇两人一搭一唱,一看就知道他们私下里已经有了默契。

赵顼觉得这个意见还不错,荆湖两路本来就是南方的战略中心,依靠长江和汉江、湘江这些支流的水路交通,向东趋江南;向西溯巴蜀;北上汉江可至襄阳,进而入中原腹地;南下更可凭籍湘江、灵渠和漓水,而至桂州。依靠历经战事的精兵强将,当能给交趾人一个好看。

但立刻有人出来反对:“荆南新定,正需强兵良将镇守,岂能随意调离?”

殿上众人看过去,竟然是蔡挺在说话。这是怎么回事?疑云丛生,而蔡挺则继续道,“荆南新复之土,若无重兵镇守,荆蛮之中当有反复。若被其探知广西交兵,起兵呼应,南方必生大乱!”

被人怀疑其自己的功业,章惇立刻反驳:“岂可因未兴之变,不救已生之灾?!且荆南之地经由王师扫平,又得陛下垂恩,山蛮早已臣服,岂有再叛之理。”

“江东今岁荐饥,前日诸州皆报饥民做过。而江西如今亦告饥馁,若事出万一……潭州驻军尚可急趋江西!”

冯京的发言很平静,但赵顼听到之后,心脏便一阵阵的抽紧。

四面边声连角起,而国中又似乎是烽烟遍地。什么时候,他的天下,竟然变成了如此动荡?

