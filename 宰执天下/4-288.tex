\section{第46章 了无旧客伴清谈(十)}

冯从义当真说到做到,只在府上歇了一天就整顿行装,准备离京返乡。

他也是本事,走的时候连驿券都弄到了手,一路都能用驿马,免费的住驿站。在熙河路,以冯从义的身份拿到一张驿券轻而易举,想不到在京城依然不费吹灰之力。

送驿券来的是开封府的左都押衙。乃是府中六百公吏最顶尖的几人之一。相对于官员调动频繁,这等公吏中的老行尊,几十年的差事做下来,拥有的权力甚至不比那四位推官、判官要小,还在六曹参军之上。

可韩冈听外面陪着冯从义一起出面接待的家人说,那位都押衙可是冲着冯从义大官人前大官人后的喊着,比对亲娘老子都亲切。

自古到阎王好惹小鬼难缠,开封府的吏员也不是以勤快著称。如果没有关系,就是学士、侍制这样的高官,想要拿到驿券,想必也不会有这个速度,更不会有都押衙亲自送上门,多半得遣人去开封府三催四请。

冯从义让伴当将驿券收好,神色如常,只是当成了一件微不足道的小事。不过当他过来辞行时,看到韩冈不能苟同的表情,就哈哈笑了起来:

“三哥你也是太过自清了。天下各州如今都开始盛行蹴鞠联赛,官府多了那么多税入,难道还当不起一张驿券不成?你想求得干净,也不看看人家怎么做的。哪家子弟出行,不从朝廷这里拿驿券?他们是常年白蹭朝廷便宜,小弟可是一向自律,寻常都是用着家里面的车马,在驿馆里住下,房钱食料钱都是给足的,也就这一次没办法才破例的。”

韩冈摇摇头,也不说什么了。这个时代的风气,拗不过来,后世也一样抓不过来,还是就当没看到好了。只要给官中有所回报,填补损失,心里也算是能说得过去。

冯从义上了马车就走了。最迟到明年仲春,想必熙河路那边就会有好消息传来了。

看到冯从义在京城中的份量,韩冈也就放下心来。

金钱的魔力本来就能所向披靡。

以利诱之是拉拢人最简单有效的手段。冯从义能在京中拥有这么大面子,还不是他出面筹办蹴鞠联赛,拉着勋贵豪商一起出来赚钱?

王安石主政多年,将宗室往死里得罪。但在宗室们的眼中,韩冈这位女婿的名声却是还能过得去,这依然是钱的缘故。

一支普通的球队,就算没有打进季后赛,一年几十场比赛下来,光是门票钱就是个绝大的数目,加上赌资抽头的分红、球场上的广告,至少万贯。而且球队成绩越好,门票、分红和广告的收入就越多。

商业繁盛的东京城,自从热气球拖着广告条幅上天之后,商家仿佛一夜之间都开了窍。如今每逢比赛日,热气球拖着广告上天不说,球场边也是一圈广告,而且连球队的队服上都绣了广告了。

刚刚结束的那一场季后赛,淮康军节度使,英宗皇帝的嫡亲六哥赵宗晖家养的踊胜队,队服胸口上都绣着张戴花洗面药一洗便白的广告。韩冈昨天从冯从义那里听说,这一代的张戴花为此花了整整一千贯——这笔钱,都能捉个进士女婿回来了!

天下熙熙皆为利来,天下攘攘皆为利往。

球队能赚钱,直接分配赌金抽头的齐云总社手上自然不会没有钱

冯从义如今虽然只是一众股东中的普通一员,但他作为开创者拥有的发言权依然份量十足。而且作为商行推举出来的几个代表,就是那些勋贵也压不下他,要不然大长公主家想要改变赛制,还要派说客到他门前?

在冯从义的主张下,开封府六百吏员,私下里从主办蹴鞠联赛的齐云总社这里拿到的钱,比从天子手里拿到的钱都多。开封府的官员,也都有一笔灰色收入。

活生生的财神爷,哪里能不给面子?敢不走齐云总社的路子,私下里赌球坏规矩的,全都找罪名给关进牢里。联赛的眼热的不少,可哪个地痞泼皮敢往里面伸一伸手?

昨天夜里,冯从义就跟韩冈聊了好一通如何通过赌马将京城富户豪门都拉进来一起赚钱的手段。言语间,对韩冈以赛马竞标为主的想法,赞不绝口。

要组建马球队成本太高,属于高端类型的比赛,而赛马竞标,属于低端,成本低,训练也简单。虽然想要玩得好,砸钱不会在少数,但门槛毕竟不高。尤其是低级联赛,主要还是以普通的马匹为主。

以蹴鞠联赛为范本,冯从义甚至都规划好了赛制。依然是由地方的队伍组成联赛,以多场比赛的总积分来派定最后的胜负。而比赛的项目分为短距离、中距离、长距离的各级争标赛,以及田忌赛马式的分队争标。最后挑选从地方联赛杀出来前两名,参加总决赛,决定一个赛季的冠军谁属。

只要能错开蹴鞠联赛的时间,不但能将吸引一批对蹴鞠不感兴趣的人们,还能将埋头于蹴鞠联赛的球迷和赌客也一起拉过来。

培育好马需要时间,但买马还是很快的。只要联赛组建起来,三年之内当有成效。十年二十年后,赛马运动与蹴鞠一样遍及天下,到时候朝廷只要能拿得出钱来,好马当是要多少有多少,而且还能一年年的享用长久,群牧司也会有事可做。

不过不论是十年,还是三年,都是以后的事了,眼下韩冈还是很清闲的,群牧司中无事,主要的事务是一封封求回书的名帖。

而厚生司的判官又来登门造访,但这一次不是吴衍——他被派出去负责开封府界二十多个县的保赤局组建和监察工作,一年之内,一个月能回一趟京城就了不得了——而是蔡京。

韩冈拿着名帖,怔了有片刻光景,过了一阵才反应过来,“快请。请他去偏厅。”

换了身见客的装束,韩冈来到偏厅,一名身着绿袍的官员随即起身。

“蔡京拜见龙图。”

一拜一起,动作舒缓自如。这位千古名人,相貌未免太英俊了一点,眉目俊朗,身材颀长,让人一见之下就自愧不如之感。

而且韩冈也记得他当年在西太一宫。当年的那一首《天净沙》,早就传唱出来,韩冈虽然没脸去剽窃,署上自己的名字,但他和路明两人的身份都给好事者挖了出来。只不过两人一直都不肯承认罢了。而当时在西太一宫的几名士子究竟是谁,韩冈当然也听说了。

但韩冈无意与蔡京多有瓜葛,还了一礼:“久闻元长大。当初元长为木兰陂一事多方奔走,韩冈也多有听闻。心慕已久,今日一晤,乃知传言非虚。”

韩冈说得基本上就是顺口的恭维,不过他能听说自己引以为傲的木兰陂,蔡京还是有几份自得,“微末之劳,相较于龙图的累累功勋,乃是萤光与皓月之别,龙图之誉愧不敢当。”

韩冈微微一笑,客套话说完,请了蔡京坐下。

让下人换了茶,方才问道:“不知元长今日来访,可于韩冈有所指教?”

韩冈的话中透着生疏,蔡京却哈哈一笑,“龙图说反了。种痘之术,乃是源自龙图格物之功,自是得向龙图请教。蔡京自观横渠正蒙,其中有言‘大其心则能体天下之物’,龙图仰观天,俯观地,体天下之物,得天地自然之道用之于人事,可谓心之大矣。”

“体物体身,道之本也。大道玄远,韩冈只得微末,远当不得元长之赞。”韩冈谦逊的说着。

从蔡京的一番话中,可以听得出来,他对气学还算熟悉,对韩冈的格物之说也几分了解。至少是下了功夫。

如果不论蔡京的身份,听到有人对格物和气学有心深入了解,韩冈多半会在视察其人品能力之后,提拔或是荐用。

可惜的是,对于一个留名千古的奸相,韩冈对他的信任度,完全是负数。也许千古传言有误,但韩冈不认为自己有必要去冒风险,也不会自大到认为自己能控制得了他。

能为大奸大恶,必有大智大勇,这句话,韩冈很是认同。如果有可能,不着痕迹的打压一下蔡京,韩冈不介意伸一次手。但蔡京能力卓异,在官场上也是如鱼得水,一个熙宁三年的进士,仅用九年时间,便晋身朝官,而且还在中书五房担任过检正公事。这份际遇,比起当年的吕惠卿、曾布甚至章惇都要强出不少。而且从他流传后世的名气来看,日后仕途如何也是可想而知的。

这一等放在麻袋里,立刻就能脱颖而出的人才,在自己面前所表现出来的仿佛发自肺腑的谦逊,完全没办法掩盖他藏在心中的自傲。

蔡京还记得韩冈。韩纲似乎是不记得了,但当年西太一宫中的擦肩而过,由于那首传唱天下的小令而让他记忆深刻。

熙宁三年的时候,自己意气风发的进士及第,释褐得官,与刚刚被举荐的韩冈相差仿佛,而且还多一个进士,任谁来看,都是他蔡元长更有前途一点,但如今九年过去了,两人的地位已经是天壤之别,差之甚远。

现如今,在厚生司中做事,人人羡慕功绩将会从天而降,但往深里说,却是捡了韩冈的便宜。嫉恨毫无意义,蔡京也从来不会浪费自己的心力。韩冈虽然名位已近宰执,但眼下他停步不前,而自己则是稳步上升,迟早会有追上去的一天。

对于这件事,蔡京从不怀疑。

