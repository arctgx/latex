\section{第28章 临乱心难齐(一)}

十月中。

四五天前的阴云蔽日让满朝上下欣喜不已,但到了前两天的清早,一轮红日再一次升上天空,毫无遮挡的将阳光撒向大地,彻底击碎了天子和群臣们的幻想。接下来的几天,又都是万里无云的好日子。供给东京水源的金水河都落了两尺,京畿一代的旱情就不问可知了。

所以这些天来,赵顼心情不好,王安石也很是烦闷,在崇政殿上的奏对,基本上都是说完公事便就此告退。不过今天有些特别,等王安石这位宰相说完公事后,赵顼竟然有心说起闲话:“王卿,你的女婿在白马县可是一鸣惊人啊!三十年积案,他到任七天竟然就破了。”

王安石已经听说了这个案子。韩冈在白马县安定下来后,就派人回乡将妻儿搬来同住。派回去的亲信,在经过东京城时,顺便稍了封信回来。里面就说了白马县的情况,顺便也将前日刚刚断过的三十年的这桩争坟案说了一遍。

看着信中所说种种,王安石越发的对于韩冈不能帮上自己而感到遗憾。能力那是没话说的,军事、治事都早有明证,而刑名断案竟然也是一样的出色。刚刚到任还不到七日,就解决了一桩三十年的积案。只可惜自己的这个二女婿,千方百计的要将他的老师塞进经义局。不忘本的做法王安石很欣赏,但干扰到自己的策略,那就不能容忍了。

王安石一拱手:“昨日韩冈写信过来,的确提到了此案。说他三问白马县民,人人皆依忠孝而答。一句世间可有哭坟不哀之孝子贤孙,引得万众齐呼,此案便由此而定!可见忠孝之道乃是人心所向,亦是陛下教化之功。”

赵顼就喜欢听这样的话,脸上顿时绽起了笑容。在他得到的消息中,并没有多提百姓的反应,而是详细了描述了韩冈是如何设局让何阗自己跳进来,从文字中赵顼能看得出来,皇城司在白马县的耳目,对韩冈这番断案的手段可以说是心悦臣服。

“以韩冈之才,置其于百里之地。其实算是大材小用了。三十年积案随手便破,虽然让人惊叹,但也是情理之中。就是那个何阗,因一己之私,连讼有司竟达三十年之久。这等刁民,韩冈怎么没有严加处置?!”赵顼不解的问道。在他看来,以大不孝的十恶之罪,直接将何阗处死都是应该的。就算大不孝的罪名勉强些,韩冈又是心好,好歹也是要刺配啊!

“何阗所犯刑条乃是‘诈欺官私取财’之下的‘冒认’一条,依律赃不满贯者免刺,而未得者更是又要减二等。两顷田地虽然价值千贯,但既然是未遂,也就是笞三十而已。这个罪罚,以知县之权,可以恕之。”

王安石是有名的好记性,书房架子上的几千卷藏书,随便抽一本下来,提个头,他就能全篇给背下来。宋刑统中的律例,他也背得滚瓜烂熟,随便就将何阗的罪名、刑罚给举了出来。

看着赵顼还想说些什么,王安石又补充了一句:“何阗也是读书人。”

赵顼听了之后,咕哝一下就不言语了。

对,这就是读书人的好处,就算是干犯律条,也能得到一定程度的通融。

士林中有骗了同僚几百两金器的状元,有诓骗资助自己考上进士的妓女饮下毒酒的学士,这一干人都被士论所不值,律条也照样是犯了,追究起来,罪名还不轻,但他们一样升官发财,一点事也没有——因为他们是读书人。

即便何阗为两顷祭田,背宗弃祖,连讼三十年,使有司不甚其扰。打上一顿板子给个教训,乃是合乎律法。但法理无外乎人情,何阗是读书人,饶他三十板,不是要照顾他,而是要照顾读书人的脸面,否则怎么能体现朝廷对文士的重视?

而且更重要的,当初支持何阗的基本上都是白马县的士子。要是真的扒光了何阗裤子,露出屁股来打板子,一记记的都是打在之前支持何阗的士子们的脸上。

这又何必呢?

韩冈还要继续治理白马县,那些士子在名义上都是他的学生。韩冈已经通过这一案将他们给慑服,但若是得寸进尺,反而会引起他们的反弹。

这番道理韩冈在信中说得也明白。何阗经此一案,已经声名尽丧,虽生犹死。这对他来说,其实已经是最大的惩罚。说不定过些日子也就死了,根本不用板子来帮人上路。律条不是死的,可以灵活选用,何阗的下场已经足以使人警醒,除了官员受累以外,又没有受害者,就没有必要再多此一举。

又说了几句,王安石从崇政殿中告辞出来。

回到政事堂,儿子王雱正在厅中等着他。

王雱到了中书过来,是要说着经义局中的公事。王安石虽然提举经义局,但他基本上不往经义局去,只能劳烦王雱来禀报。

作为宰相,王安石身上的兼着的差事不少,编纂朝廷政令、律法的编敇局,编写国史的史馆,还有就是编写科举教科书的经义局,这些文事、政事方面的职司,都是要他这个宰相来提举。

不论是法律条令,还是国史,又或是国家教材,都是宰相身上的任务——就如《武经总要》,署名的曾公亮,他当时就是宰相;《太平御览》的主编李昉,当时也是宰相——这是宰相的权力范围,提举之位不会交到别人手上。就跟后世国务院的最高领导,许多时候都会兼着某某领导小组一般——官僚社会,古今如一。

不过王雱说是来禀报经义局中的最新情况,其实也只是借口而已,王安石稍稍问了几句,就放到了一边。父子两人谈论的乃是事关天下的要事,回到家中都讨论不完,要在政事堂中继续。

现在王安石面临的情况很是危急。这并不是政府中事——新党之中,吕惠卿和曾布之间关系依然紧张,可王安石自问还镇得住他们。而诗书礼三经的释义,也差不多快完成了。《诗经》、《尚书》两部,是自己列出大纲,而由王雱、吕惠卿领衔编写,只有《周官》一部,是由王安石自己亲自写的。新法的推行还算安定,政事、军事、财务等方面的变革都是卓有成效。

眼下,会直接影响到王安石官位的问题,还是今年的旱情,以及明年在预料之中的饥荒和蝗灾。

“京畿一带的出苗的情况,下面都报了上来。玉昆写的信中,也说的很清楚了,黄河滩上尽是蝗虫卵,亿万之数,来年就是漫天飞蝗。而白马县的麦田,眼下也只有六成出苗。情况的确很糟。儿子在经义局中,还能听到外面的消息,说是市井中已经开始有人在暗中囤粮了。”王雱脸色沉重,瘦削的双肩似乎都有些支持不住现在的压力,“不知能不能让东南多运一些粮食进京,就算只有十几二十万石,关键时候拿出来,能一举让那一干奸商折光老本。”

王安石的神色与儿子一般沉重。如果灾害继续严重下去,他作为宰相,肯定要负全责。天人感应就是攻击他下台的最有效的武器。尽管在重臣中,相信这一理论的人并会不多,韩琦、富弼、文彦博、吕公著,乃至司马光,都不会信。但并不妨碍他们拿着这个作为武器,来攻击自己。

“两浙从入秋后也少雨,秋粮比往年减了有两成,润州都报了灾情。能保证一百五十万石的额定,两浙转运司已经是竭尽所能。其他几路,情况也不算好,淮南也一样有灾。润州干旱,方才已经奏请官家拨常平司粮三万石,此前报了饥荒的淮南东路的真州、扬州,也各拨三万石,募饥民兴修农田水利。”王安石叹了口气,“而且最近气温骤降,汴河转眼就要封口。就是有再多的粮食也运不过来。”

“不知能不能今冬不闭汴口?”王雱提议着。

“可河冰怎么办?”王安石想摇头,突然又停住。到了冬天。汴河因为河中上冻,就要封住汴口,停止航运,等到来年春时解冻后,才会开启汴河河口,让纲船南北通行。不过若是能利用上这个冬天,京中的情况也许会好转不少。

见到父亲心动,王雱立刻提议:“不如急招景仁【侯叔献】回京来问一问。”

侯叔献在新党中出了名的精擅水利,是新党中的中坚力量,也是王安石处理新法实务的重要助手。

侯叔献从熙宁二年农田水利法和均输法一开始推行,就开始接手灌溉淤田等事,经由他手所淤灌出来的良田,多达万顷之多。原本汴河两岸,因为洪水决堤而造成的两万顷荒废的盐碱地,在他的治理下,也已经恢复了很大一部分。现在他是都水监,管理天下水利。不过因为他又兼着河北水陆转运判官,现在不在京中。

王安石没有多作犹豫,点头首肯,拿起笔墨,便书就了一份堂札,画了押,盖了印信,让书办送了出去。

望着透窗而入的灿烂夕阳,王安石叹道,“希望景仁能早点回来。”

