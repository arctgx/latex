\section{第38章 逆旅徐行雪未休(一)}

【第四更,2010还剩最后一天,三十万字的承诺即将达成,努力的求红票,收藏。】

天色阴沉了下来,正月十五的天空,泛着沉甸甸的铅灰色,灰色的天空,白色的大地,却在天地的交界处模糊起来。风也起了,不算凌冽,却足够寒冷,看起来要下雪的样子。路就在脚下延伸,韩冈一行离着千年古都也越来越近。

路明不愧是常来常往于东京和关西之间,对道路熟悉得很。他骑在骡子上,指着南面偏东一点的方向,“再过十七八里,就能看到京兆府城了。”

韩冈点了点头,十七八里的路程,只要一个时辰便能走完,应该能赶在城门关闭之前抵达城下。只是他低头看着骑在骡子上的路明,心中有些抱怨,若不是他的骡子脚力太差,耽搁了行程,他现在就应该住进长安城中的驿馆里去了。

听着路明的话,韩冈一行速度便稍稍加快了一点,让路明的骡子追得有些吃力,一边走,一边不爽的叫唤着。

只是行不过一里,他们的速度又降了下来,骡子不叫唤了,但路明叫唤了起来,“怎么啦!怎么啦!出了甚事,怎么堵起来了?”

就在他们前面,不知为何聚着一群人。七八辆车马都停了下来,连同百来人,将通往长安的官道堵了个严严实实。官道两侧的田野中,积雪深厚超过三四尺,并不像官道上的积雪已经被熙熙攘攘的车马行人所碾平。原本因为路基的缘故,应该比周围要高上一尺的官道,现在却仿佛陷在雪地中间。只要积雪未化,前路这么一堵,想下了官道绕路前行都不可能,就跟方才的税卡一样。

“怎么回事?!”韩冈也纳闷着,他和刘仲武驱马上前,赶开了挡在前路的人群,把他们逼到官道边。不管身后有多少抱怨,挤到了最前面。

“狼!”路明像女人一样尖叫了起来。

“不是大虫就好!”韩冈冷冷的说了一句。此时还没有诞生环境保护这个词汇,虎狼熊罴满山乱跑,陕西靠近秦岭的各处州县,城里没钻进过老虎的屈指可数。韩冈家的下龙湾村,基本上隔个两三年就会来只大虫做客,路上看见老虎都不奇怪,何况是狼……

就是数目多了一点。

官道的前方,堵住行旅的地方,令人难以置信的聚集着二三十头饿狼。在狼群的中心,是一匹被啃掉了许多皮肉的死马。马尸的大小有限,只有最壮的几头狼能挤到马前,埋头于马尸之中,一条条的血肉被撕下来,嘎吱嘎吱的嚼碎骨头的声音听着让人牙酸。剩下的饿狼都在外围不停的打着转,眼睛莹莹透着绿光,不时的,有几头想挤进内圈分一杯羹,却立刻被一爪子拍回来。

而那匹死马脖子上,还系着缰绳,脱缰的车厢则在死马边上,被狼群围在中央。狼群之外,还有五六辆与狼群中的那辆同样形制的两轮马车,车上的人都下来了,十五六人的样子,有男有女,都在惶急的看着狼群中的马车,想上前,却又不敢,一直都在犹豫着。

“车里有人!”刘仲武一声惊道。

“嗯!”韩冈点了点头,他也看见了,也听见了。吃不到肉的一群饿狼就围着死马和车厢打转,总有几头不耐烦的想跳上车子。车厢门口的布帘抖个不停,而尖叫声穿过布帘的阻隔,也隐隐约约的传到了围观者们的耳中。

冬天觅食不宜,少有大股狼群出没。平日里见到的多半是孤狼,最多也不过三五头一起出动,见到人往往远远的就跑掉了,根本不敢在人来人往的通衢大道上久留。韩冈不论是在秦州,还是在今次出行在外,都在野地里碰上过几次狼。比家养的狗要瘦弱许多,只是一眼看去,便知道它们的凶悍。

但从来没有一次,韩冈同时看到过这么多狼。吃饭的嘴聚得越多,找到的食物便越不够分,不论是狼,还是人,其实都是一样。如眼下一次聚集起这么一大群饿狼,必然会有原因。

“这群畜生,都是给血引来的。”刘仲武突然冒出一句,解释了韩冈的疑问。

韩冈再仔细一看,才发现雪地上有一长串血迹,血迹两侧还有一对已经模糊不清的车辙痕迹。这几十头狼肯定不是一伙,而是被血腥气从四面八方吸引过来。那支车队在狼群出现时没有及早抛下受伤的马匹,现在才会被围住。

韩冈望着被狼群围困的车厢摇了摇头,眼下形势并不妙。车厢里的人没有及早弃车,是个最大的错误。狼的本心是怕人的,一开始的几匹孤狼绝不敢跟人斗。车中人下了车,完全可以直接向前走。有着马尸吸引狼的注意力,人根本就不会有事。但时间一点点的拖下去,饿狼到得越来越多,这时候,已经变成想走也走不了的情况了。

而且随着血腥气飘散得越来越远,一头头饿得只剩一把骨头的瘦狼也不断的从官道边的野地里窜上来。仅仅是韩冈在这里等的片刻时间,狼群的数量又增加了三四头。再拖下去,区区一匹死马肯定不够越来越多的饿狼食用。到时已经受到刺激的狼群,肯定会开始攻击其他的马匹和人类,那一支车队说不定全都得葬身狼腹。

“韩官人,怎么办?”刘仲武问着韩冈的主意。虽然他是在向韩冈征求意见。但见他突然变得深沉起来的神色,韩冈心知就算自己反对,刘仲武也定会自行行动。

路明插话提议道:“还是赶紧回头去方才经过的镇子上找救兵,只要来了一队人,包管把这些畜生都驱走。”

为了掩饰自身的怯懦而提出的建议,并没有实际的意义。刘仲武不给路明半点面子:“真的等我们把救兵找来,人都死干净了。韩官人,你说怎么办?”他再次征询着韩冈的意见。

“不就几十头狼吗?它们又有吃的在旁边,有什么好怕的。”如果是群没有食物的饿狼,韩冈不会去凑热闹,就算运气好没有自己陷下去,被咬伤一口都不得了。但既然有一匹死马供狼群食用,便不必去怕这群狼还有攻击自己的闲心。韩冈把绑在鞍后的包裹丢给李小六,开始检查自己的武器装备。

刘仲武弹了一下弓弦,嗡嗡的弦鸣表明他的两石长弓的状态良好,“希望车里的是个美人,也不枉洒家一番辛苦。”他轻松的笑着说道。

刘仲武并不是个死板的闷葫芦,其实也会说个笑话,人缘也很不错。要不然他当日启程往京城去的时候,就不会那么多兄弟来给他饯行。

韩冈则一边整顿装束,弓箭和佩刀都是一次再次的确认是否整齐,一边还不忘给刘仲武泼了盆冷水:“决计不会是美人,多半是把老骨头!”

“官人你能看到?!”刘仲武觉得自己的视力应该在韩冈之上。他可是以眼力敏锐著称的,能将百步外的人脸相貌看得一清二楚,冬天里,能一眼看见雪地里的白毛狐狸。而日日对着油灯读书的措大,怎么可能还有双能看透车窗布帘的好眼神。

“想都能想到!……那辆车里坐的是整个车队的主人,而且还是说一不二的性子。”韩冈抽出腰刀,查验了一番是否完好,便又收回鞘中。

“官人你怎么知道的?”刘仲武小心翼翼的问着,难道韩冈能掐会算不成。若他真有这本事,日后还是要躲着他远点走。

从头到脚检查了一遍,韩冈最后拍了拍身子,发现没有任何疏失,一切都已经准备完毕,他这才指着官道两头远远围观着的人众,向刘仲武解释道:“没看到路两头围了多少人吗?若非只有车里的人才有权拿主意,车队里的人早就该出来悬赏驱狼了。但他们主人不发话,下面的仆人谁敢越俎代庖?”

韩冈又回头向西面看了看天色,天空中的铅灰越发的黯淡了起来。他对刘仲武道:“快入夜了,再不动手可就难说了。”

刘仲武哈哈大笑,“就等着官人里这句话!”

一声喝斥,两人同时提弓驱马上前。隔着二十多步,把坐骑拉横过来,在马上张弓搭箭。韩冈和刘仲武的动作吸引了所有围观者的目光,而车队中的成员,也发出了低低的欢呼声。路明惊得说不出话来,韩冈亲口说过他是文官,怎么胆子这般大的?

噌噌两声弦响,两支长箭同时激射飞出。众人正要欢呼,却见刘仲武的一箭扎进了雪地里,箭尾全没了进去,旁边正埋首于马尸肚子里的一头饿狼,连头都没有抬上一下。而韩冈的一箭则更出色,夺的一声,射到了马车的车辕上。

“日他嘬鸟!”刘仲武摇头骂了一句,他箭术并不差,但手指都冻得发僵,使不上劲,也把握不好力道,而且在马上还难张弓,同样的问题也出现在韩冈身上。两人又射了两箭,便只看见箭矢乱飞,却一头狼也没射到。

