\section{第46章 了无旧客伴清谈(八)}

‘这也是当然的。’韩冈视之为理所当然。自家的老子是老封翁,娘亲是老封君,在陇西县中是数一数二的大户人家,不论走到哪里都会有人给份面子。何况还有他这个儿子在。

冯从义喝了口茶,振起精神:“三哥你是好些年没回去了,都不知道陇西现在变化有多大,城里早挤满人了,城外原本的榷场早就被住家、商户围起来了。城内城外的坊廓人口加起来,快赶上秦州坊廓的三分之一。现在都说要扩建城池,将城外的住户都包进来,州衙那边说是过了年就向朝廷申请。过些天,说不定会有信来,请三哥你帮上一把。”

“听你这么一说,若有机会,还真的想回去看看。”韩冈说道:“至于给陇西扩建城墙,这一点愚兄怎么可能会不帮忙?不用说肯定都会出手的。不过扩建的城墙到底打算怎么修,这可是要先给我说一说。人、财、物从哪里筹备,规模到底多大,城墙形制如何,都得给愚兄说一说。”

“那还用说!若是三哥不明不白的胡乱答应下来帮着说话,一旦修得不好,最后岂不是要怪到三哥头上。”冯从义立刻说道,“到时候肯定会让州衙里给三哥你说明白的。”

官员在外,也会关心家乡的事,许多时候,州县有什么工役,去请动那些在朝中为官的乡里重臣,十分常见。

该说的事都说了一通,冯从义无意中瞥了眼书桌,正看到韩冈放在桌上的一张纸。

“青唐羌、沙苑监、保马法、州屿……”冯从义皱眉看了一看,回头问道,“这列的是军马的来源吧?”

“嗯。”韩冈应了一声,“当了同群牧使,虽说不想多管事,总得关心一下这方面的情况,做得太难看,愚兄也逃不了罪责。”

冯从义拿着纸坐下来,多看了几眼,又抬头问道:“三哥,这军马的来源,怎么能把那个地方漏掉?”

“什么地方?”

“女直啊。”

所谓女直,也就女真。盘踞东北的蛮族,日后祸乱汉土,给中华文明带来深重灾难的那个女真。

“不是没想到,女直人手中的马,愚兄当然想要。可高丽怎么绕过去?”韩冈摇摇头,“马政若有外国参与其间,那是太阿倒持。”

说到女真,就必须提到高丽。大宋与女真山水相隔,联络不便,绝大多数的情况下,必须通过高丽来中转。

“年初的时候,不是派过了使节去高丽,还怕他们做什么?”冯从义问道。

“派的是安焘,现在的判厚生司。可一样没用啊,做生意的商人,可不是官府说什么,就做什么。”

朝廷从熙宁八年开始,就与高丽这个辽国的属国有了正式的往来。就在去年,为了震慑高丽,夸耀大宋的实力,天子赵顼还特意让明州船场打造了一艘万料海船,亲自题名为‘凌虚致远安济神舟’,在今年年初,供如今的判厚生司安焘出使高丽。

而高丽商人作为中间商,在中国和日本,以及中国和女真之间的贸易上赚取差价的行为,更是从立国时就开始了。

现如今,与女真人做买卖的,有中国的商人,更多的则是高丽的商人。朝廷想从女真人那里弄到战马不是一天两天了,但最终这些商人弄到手的,却多是东珠、貂皮、鹿茸之类的珍货特产,战马却没有几匹。

“所以说商人做事不靠谱,眼珠子都钻进了钱眼里。”韩冈叹气。

冯从义笑了起来,顺手在纸上添了两个字,“凡事只看钱,这是商人的本分,再靠谱不过。小弟在各地捐钱捐物、修桥铺路,还不是为了名声好赚钱。战马的确价值高,但那终究是活物,在船上要吃要喝,装得多一点就会病死,少一点浪费空间,而且还犯契丹人的禁令,反而不如北方的特产来得赚钱和保险。”

韩冈看着纸面上的女直二字,皱了半天眉头。如今的女真,还不需要放在心上,以现在大宋的发展,日后更不需要放在心上,只是他们手上的战马,却没有人

听说每年辽国从各部女真那里收上来的贡马数量大得惊人,有说是一两万,有说是五六万的,有说十几万、二十万的——这当然不可能,但从最少的数量上来说,能有一两万已经是很让人羡慕了——贡马,是不花钱的。

而且辽国可不是宋国朝廷,荤素不忌,大小通吃,游牧民族出身,来自于草原上的契丹人,他们对马匹的要求可是高出十几倍、几十倍,品相差一点的都不可能收下来。而且除了女真,他们还有草原这个大马场。契丹人没有只从女真人手中压榨战马,而放过草原上的阻卜人的道理。更不会放过其他属国,吾独婉、惕德、东丹、直不姑,这些大属国,越里笃、剖阿里、奥里米、蒲奴里、铁骊这些小部族,乃至西夏,哪一家敢不给契丹人上贡战马?

说起来还真是让人羡慕。

“照小弟看。”冯从义继续说道,“看看是不是拿官职悬赏上来,同时设立专门的市易司,来负责处理对女真的茶马互市的业务。若是能占据一两个海岛,贴近到辽国国境,说不定能联络得更方便一点。”

“事关辽国,朝堂上不怕盘剥百姓,却会担心节外生枝。只能少量的买。”

“那就没办法了。”冯从义摇着头,“如果只是少量的话,天竺马、大食马也不是买不到,广州蕃坊里面居住了多少蕃商,可惜就是买来了,靠牧监中的那群人也养不出好马。”

马政的败坏不是单纯一个原因造成的,而是内因外因的集合,在韩冈看来,几乎是无解的。要说官营牧监不好,可唐代前期的几十万匹战马,全都是出自牧监,而不是私人。可要说官营有多好,眼下的例子能让人说不出话来——这是管束上的问题,让豪门富户将官营牧监当成肥肉,而朝廷没有从一开头就加以制止,日积月累,现在想改正都难了。王安石主持撤并牧监,也只是承认现实。

牧监都已经撤了,只剩一个沙苑监,根本没有用处。韩冈也没有回天之力:“富有富过法,穷有穷过法。既然真正的战马还是得买来,那就干脆还是以少数的骑兵部队配合大批量步兵,这本就是大宋官军对敌的正道,继续下去好了。”

冯从义也听得出自家表兄的无奈,附和道:“手上有什么菜,那就得做什么饭。的确是没办法的事。”

“是啊,只能这么做。”韩冈偏着头,对冯从义道:“说来也好笑,群牧司里现在就有人打着主意,准备谋划什么户马法,逼着富户去养马。”

“强逼富户?是从保马法改过来的吧。”

韩冈更正道:“保马法养马可都是自愿的。”

冯从义笑了,“三哥都做过转运使了,怎么还不知道下面的事?多少地方推行保马法时就是强逼着来的,现在换了户马法,不过是正名了罢了。”

“就是正名不得!”韩冈怎么会不知道地方官员提高政绩的恶劣手段,“只要朝廷还不承认,日后也有改正的余地。一旦正名了,错事都变成对的,想改正都难了。”

他一声长叹,“其实也不能怪他们,要不是各个牧监都废了,朝廷又要用兵,哪里会逼得人去想这等找骂名的主意。强逼着富户去养马,祖宗八代都别想安生了。”

冯从义突然眯起了眼:“三哥,其实要想人主动养马也不是没办法啊……”

韩冈狐疑的瞅着表弟脸上诡谲的笑容,“你有什么办法?”

冯从义微抿着嘴,很是有两分得意,神神秘秘的,“三哥你可知道,巩州的富户,钱多的,直接养着一支球队,钱少的,几家联手养上一支。没有几家手上不攥着一支球队的股,光是门票和赌金的分红,都是一笔大数字。”

都说到这份上了,韩冈哪还能不明白,眼睛一亮,脱口而出,“赌马?!”

“是马球……”冯从义先愣了一下,旋即醒悟,“就是赌马!现在外面的蹴鞠联赛哪有不赌的,场场都有几千贯的赌资进来,到了季后赛和总决赛,都没见过少于万贯的!”

韩冈知道表弟是误会了,也不说破:“组成马球队,马匹、骑手少说也要十几对,没几家能养得起。如果仅仅是竞速,长程、短程的骑马争标,一家就只要养一两匹马,参与者就能多上一点。”他站了起来,轻快地在书房中来回走着,“当然,有马球队也是好的,养得起就去玩马球联赛。只养得起一匹两匹的,就让他们去玩争标。各有各的去处。”

“那小弟这就去安排!”冯从义也跳了起来,“等三哥你上本之后,就在京城中将骑马争标赛给操办起来。”

“不,这件事由你来提。”韩冈摇摇头,“这是义哥你想出来的,愚兄岂能夺你之功?等你提上之后,愚兄再上书赞同就行了。”

