\section{第11章 五月鸣蜩闻羌曲(六)}

【不好意思,迟了一点。第一更,再求一下红票,和收藏。这两项的数据现在都是有些不够给力。】

烈日高照,除了躲在树荫里得意的欢叫着的夏蝉,就只有藉水的水流声哗哗不绝的响着。道边草木的叶子都在烈日下蔫了下去,但沿着四丈宽的官道,迎面走来。

他们虽然人数不多,装束更是五花八门,但气势昂然,俨然一支胜利之师。高高举起的旗帜比起路边蔫掉的叶片要精神许多。而他们所骑乘的战马,大概是受到主人心情的影响,各自踏着轻快的步伐。路边悦耳的流水声是欢快的进行曲,为他们的前行做着的伴奏。

在夏日艳阳下,越过陇城县城与秦州州城之间的三十里地,两名秦州西路蕃部提举所率领的队伍却没有半点疲累的模样。高遵裕脸上的笑容也随着他们离秦州越来越近而更加灿烂,完全不在意从额头上滚滚留下的汗水,这样的笑容一直持续到他看到空空落落的秦州东门。

青唐部在渭水边胜利的消息,应该早在三四天前就抵达秦州,而王韶他们的行程也应在两天前送到秦州州衙之中。但理应迎接凯旋大军的官员们,却一个也没有出场。空空荡荡的城门前的道路上,只有知了在叫着。

高遵裕的脸一直黑了下去,挂得老长,而王韶却是开怀大笑,韩冈也是轻笑了两声,对高遵裕道:“他们气急败坏了。”

想了想,他又添了一句,“心胸如此,此辈不足虑。”

李师中现在还坐镇在陇城县。在王韶他们驻扎在陇城县的昨日,李师中是随便找了个冠冕堂皇的借口往北去视察水洛城了,正好避过得意洋洋的王韶和高遵裕。

而留守秦州城中的窦舜卿则是又病了,秦凤路兵马副都总管总是病得很及时,又痊愈得很及时。他的健康状况只跟局势有关,情况不对就缩头做乌龟的本事,也只有他这个世家弟子,才能玩得这般娴熟流畅。

至于向宝,他杜门不出已有多日,倒不是因为不想看到王韶和韩冈他们得意的那张脸。秦凤都钤辖即将调回京中的传闻已经在秦州城中传扬开了,秦凤路的官员们都是现实得很,就等朝中发来的公文证实,对向宝发出命令都是采取拖延无视的态度。这种情况下,向宝也只有选择关起门,在家扎王韶、韩冈的草人。

王韶和韩冈对此早有所料,他们过往的经历已经告诉他们今次会受到什么样的接待。但高遵裕不同,他对这般无礼的待遇毫无心理准备,正在兴头上却被当头浇了盆冰水。心头却并不是发寒,而是一阵难以遏制的邪火。

“等到朝廷封赏下来,就可以让李、窦二位好好看看了。如果那时他们还在秦州城的话。”

韩冈越来越看不起李师中、窦舜卿之辈,心中狭窄的模样让人发噱,如果换作是自己,笑着上前亲切拥抱都没问题,何况出城说些恭维话?

他又回头看看青唐部的两支队伍,无论俞龙珂还是瞎药,神色都起了点变化,也不知他们有没有看出问题。他提醒着王韶和高遵裕,“机宜、提举,不能让得胜归来的将士在城外久等。”

王韶立刻会意点头,“不用理会他们这群鸡肠鼠肚之辈,大张旗鼓,让全城都知道,王师得胜而归!”

……………………

在城中安顿随行蕃部近三百人的队伍,是个不小的麻烦。秦州军中排在前三的人物都摆出非暴力不合作的态度,弄得下面也是有样学样,但最后把高遵裕这张虎皮拉了出来,韩冈还算轻松的解决了这个问题。

仔细挑选了得力的吏员,让他们好生招待这群立了功的蕃人。又跟俞龙珂和瞎药打了招呼,请两人约束一下他们这些不懂礼数的手下。韩冈倒不怕俞龙珂和瞎药现在还能闹出什么事,已经到了自己的地盘上,一切都由不得他们。但他们手下的一群蕃人,却都不是省事的主,如果在秦州做下浑事来,李师中的弹劾就又有好题材了。

“本官已经下令让人在营地外好生护卫,防止有人骚扰贵属。族长你完全可以放心自己的安全。”

同样的话,换了个人称,韩冈又对瞎药说了一遍。

不过两人都是聪明人,都知道宋人对他们这些蕃人的顾忌,也清楚韩冈到底是为了什么而说这番话:“韩官人放心,不会让官人为难的。”

韩冈自营中出来,冲在外面领了一队骑兵的王舜臣点了点头,“这里的一切都拜托王兄弟了。”

王舜臣对韩冈拱了拱手:“三哥放心,不会让他们闹起来。”

韩冈笑了一下,走进了,反手用手指对身后的营盘一指,“有机会多表演一下你的箭术,给他们每一个人的都好好见识一下。让这些蕃人知道,秦凤路除了刘昌祚,还是有个堪比李广的神箭手的。蕃人都是畏威而不怀德,不要怕冲突,只须小心不要弄出人命。出了事,我会帮你的担着。”

王舜臣连连点头,韩冈赞了他两句,让他听得浑身都舒坦。他呲着牙笑着:“三哥你放一百二十个心。俺肯定会好好跟这群蕃子谈谈心的。“

把蕃部的事处置妥当,向王韶、高遵裕禀报过,韩冈又想起他自己手边的事来。

仇一闻已经被高遵裕惦记上了。现在高遵裕正恨着窦舜卿,任何能让副都总管不痛快的手段,他都不介意用上一用。

高遵裕不是心胸宽广的人,以韩冈这些天来对他的了解,新任的蕃部提举跟李师中、窦舜卿都是一路货色。对功劳很贪,对责任则无心负担,而对他人的不敬,却是狠狠的记在心底,想着等到时机就去报复。

韩冈不能眼睁睁的看着高遵裕去玩他的小手段。就算不能把仇老郎中的徒弟救出来,也不能让仇一闻也跟着陷进去。对于高遵裕玩着阴谋诡计的手段,韩冈并不觉得有什么大不了,但弄到跟自己有交情的人身上,韩冈却不能忍受。

把仇老头子安排到自己的家中,让严素心和韩云娘好生款待。韩冈便想问一下这里的地头蛇,仇老郎中的弟子现在的情况究竟如何,还有这件事,仇一闻到底说得是真是假——不是说仇一闻说谎,而是同一件事,不同人持有的看法都不同。谁也不能保证仇一闻说的事情,不是被他的立场所扭曲。

他找来李小六,吩咐道:“你速去把王九和周宁都叫来,说我有事问他们。”

王九、周凤已经在成纪县衙做了半年多了,县中内外的一应事务都已经熟悉。而他们与州衙吏员之间,多少也应该有些交情了。要询问州狱中事,少不得要通过他们。

州衙所在的县治,知县都管不了城中之事。州城内的大小事务,都是由州衙处理。就如成纪知县,他就只能管辖秦州城外的成纪县辖区,对城墙以内,却没有插足的余地。

仇一闻的弟子是在城中为窦舜卿的重孙诊治,那他现在的位置,只会位于州衙大狱之中。而韩冈虽是在州衙内做事,但经略安抚司与秦州是两套班子,只是统领两套班子的是李师中一个人罢了,而两边下属的官员,都是互不干涉。也只能希望那几个被他安插在成纪县衙中的钉子,能起到一定的作用。

很快,王九和周宁来了。他们见了韩冈,就立刻恭喜他又立新功。而韩冈不说废话,直说道,“今次请你俩来,倒是有桩事要问你们。”

……………………

窦舜卿这几天心情正不好,在院中的树荫下坐着,死板着脸,两个婢女不断的摇着扇子,也没能扇去他心头的火气。

日后快近天顶的时候,窦舜卿的长孙从院外进来,向他行礼请安。

“怎么才回来?昨夜到哪儿去了?!”窦舜卿看着孙子青黑色的下眼圈就气不打一处来,“你亲儿子死了,也不见你难过一下!镇日的往青楼里跑,也不好好读点书出来!”

“死了再生就是了,也不是生不了。”窦解对死了个儿子毫不在意。只是他看着窦舜卿的脸色沉了下去,连忙转口道:“给幺儿治病的那个党项郎中肯定是西贼内奸,奉了西贼的命要害我们一家。”

窦舜卿有些疲累的摆了一下手:“这事就随你去做,别把事情闹大。”

“怎么能不闹大?”窦解这时神秘兮兮的凑到自己的祖父耳边,“大狱里的党项郎中是个叫仇一闻的游方郎中的弟子。而仇一闻,如今却是一直都在帮着灌园小儿弄什么疗养院,在军中收买人心。任用西贼奸细的师傅,韩措大这究竟是安得什么心?”

窦舜卿眼定定的盯着自己的孙子,一个字一个字的问着:“这事是谁告诉你的?!”

“是孙儿打听来的。”

“胡说!”窦舜卿对自己的孙子哪还不了解,他能打听青楼里的头牌花魁喜欢什么颜色的肚兜,却不会把心思放在正事上半点。

“管他是谁说的。能把那个灌园小儿整治一番,岂不是一桩美事。把他弄进大狱里好生料理一顿,说病死也就病死了。种家的人都能瘐死,还怕弄不死个灌园措大?”窦解扭着手狞笑起来,“这也能让人知道爷爷的手段。”

