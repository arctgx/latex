\section{第七章 苍原军锋薄战垒(二)}

深夜的东京,依然有着炎炎暑气。

傍晚时的一场骤雨,并没有将气温压下来,反而因为多了温热的湿气,让夏夜更显闷热。

韩冈穿着一袭单薄的短衫绸裤,手上摇着把蒲葵扇,靠在在一张藤屉子躺椅上。编织屉面的老藤深褐发亮,连绵不断的水波纹花样当是费了工匠不少手工,躺在上面凉快透气,而且还不像竹床那般硌着慌。

李宪是个有能耐的人,在征南的时候,韩冈就了解到了这一点。比起运气好得让人无话可说的王中正来,李宪这位阉宦,才当得起通晓兵事这个评价。

李宪在河东路第四将副将訾虎被袭身死之后,立刻领军北上,先利用帐下为数不多的骑兵,吊住了回窜的两千铁鹞子,步兵则在分兵后用最快的速度连续毁了百里之内大大小小十一个水源地,又作势要毁去更多的水源,勾引这群还有心继续袭击官军的铁鹞子撞上来。

李宪成功了。打着各个击破主意的铁鹞子咬上了兵力最少的一支,只有两千人不到,但那是折家家主折克行亲自率领的一支精锐——还有一支同样数目的精锐由李宪亲领——直接崩坏他们的牙齿。

由于宋军禁军已经普及了铁甲、斩马刀和神臂弓,对精锐和非精锐的判断失去了最关键的依据,只能从人数粗略判断宋人实力的铁鹞子,被折克行的反击造成的伤亡超乎预计,一次交锋就失去了所有的信心。而接下来,溃败后的铁鹞子又遭到了宋军骑兵的追击。尽管在被反击和追击的过程中,他们加起来的损失依然不到总兵力的三分之一——这是骑兵的优势所在——但这一支作为奇兵而被派出来的铁鹞子,已经失去了实现他们出战目的的可能。

但李宪的运气终究还比不上王中正。李宪在解决了铁鹞子之后,就不得不全军南下,向种谔靠拢,以求得到补给。

而王中正在收复兰州、攻克卓啰和南军司之后,在天都山下焚毁了西夏的行宫,还在龛谷川边发现了一座御庄——这是西夏国主名下的庄园——里面囤粮近八万石,加上还没有收割的田地,十余万石总是有的。这座御庄不知为何成了被党项人遗忘的角落,偏偏给王中正撞上了。就是打下了鸣沙城的苗授,也不过得到了窖藏粟及杂草三万三千余石束而已。

苗授打下鸣沙城,高遵裕攻克韦州,都是十天前传来的消息,现在他们两人应该到灵州城下了吧?韩冈摇着扇子,想着。

这个时代的信息传递速度太慢了一点,对于已经深入环庆和泾原两路,东京城中只能得到他们十二三天之前的消息。不过从时间和路程上计算,顺利的话,应是已经看到灵州城了。

在计划中,六路人马是要在灵州城下会合,可眼下就只有环庆、泾原两路做到了。

王中正还有好些天的路要走,而种谔和李宪完成计划的可能性更小。河东路的民夫损失过大,粮草全都得依靠鄜延路。而鄜延路的情况,也不会好多少。而且韩冈也不相信,党项人派出来骚扰后方的奇兵会只有两千骑。

这样的情况下,诸路兵马齐聚灵州城下的话,后勤上压力就太大了,也不可能实现。

可只凭环庆、泾原两路的人马,到底能不能打下灵州城?

赵顼和王珪似乎很乐观,但韩冈却不这么看。而且打不下来的结果,只会是惨败,连全身而退的可能都不会有。

但从韩冈的角度来说,坏事中终究还是有点好处的。

王中正来不及赶到灵州城下,一旦前方溃败,他肯定不会再主动冲上去,王舜臣更是向西去。鄜延路和河东路粮草不济,很难渡过瀚海。

从这个意义上说,即便败阵,除了环庆、泾原两路之外,其他几路的损失不会太大。只要西军不丧失太大的元气,日后也有卷土重来的机会。

尽管西军的败阵是韩冈所不想看到的,但事已至此,又不是自己造成的,韩冈也不会将责任揽到自己身上。

十年前的韩冈,在这个季节正缠绵于病榻之上。八年前的韩冈,也不过是个刚刚立了点功劳的小官。

那时候,他绝不会自大到认为自己能立刻改变这个国家,最多也就在王安石面前煽风点火一番。

但随着官位的升高,曾几何时,就变成了凡事都要心想事成的心思?

过去做事,都是顺势而为,借助天子或是权臣的力量,达成自己的目的。眼下则是顶着皇帝想法,还想心想事成,就不是那么简单了。顺势、逆势是两回事。

越向高处去,身上的束缚就越多。

还在熙河路的时候,来自于朝堂上的压力被王安石和王韶顶着,自己只要把手上的工作做好就行了。

到了如今,手上的差事对韩冈来说仅仅是举手之劳,而国家大事,韩冈却又还差上一点资格,才能名正言顺的参与进去。

现在的处境,其实就是太过于想干涉朝政的结果。纵使他想保着西军,但别人不领情也没办法。

表明了自己的态度,做好了自己的工作,西军中关系最紧密的几方都不会有太大的危险,还有什么好挂心的?既然改变不了,应该直接放下。

在过去,韩冈从来不会将结果幻想得太完美。如果付出的努力能有三成的回报对他来说就算是及格了。达到六成便可以称之为满意,至于更高的回报,不要去奢望,只要能保持这样的豁达,结果就是时常而至的惊喜。而眼下的局势,利用得好的话,也是能有惊喜的。

韩冈突然笑了起来,期待有惊喜的想法,也是不该有的。

“官人在笑什么?”

周南在门外问着,掀开帘子走了进来。为了通风,书房的大门敞开,只用纱帐做了一道防蚊的帘子。

“我在笑我这段时间想得实在太多了。”韩冈笑着扬了扬手上的扇子。

韩冈的回答没头没脑,周南却也没多问。回身从身后的一个小丫鬟捧着的托盘上拿起一个盖碗,递到韩冈手中。

青玉色的瓷碗触之冰凉,外壁上凝着细细密密的水珠。揭开碗盖,里面是一碗细白如凝脂的冰镇酥酪。

周南在韩冈身边坐了下来,手中拿着一柄绣着牡丹的轻罗团扇,宽松的袖口褪到了肘弯,莹润光洁的半截小臂露在外面,在灯下愈发的肌肤如玉。韩冈舀了一勺酥酪,放在周南的手臂旁,在灯下看着,倒还真的差不多。

“难怪有如酥如酪的说法。”韩冈由衷的感叹着,视线却往上移。产后两个月的周南腰身已经恢复如初,而原本就丰满的地方,则更加丰盈,而且比起手臂,色泽尤胜一筹。

美目似嗔似怒的瞪了韩冈一眼,周南坐直了身子,将衣襟裹得更紧了一点。

韩冈笑了一笑,想看的时候,总是能看到的,低头专心到今晚的甜点上。

冰镇的酥酪,用鲜羊奶、白糖和醪糟为原材料,做好后冰镇了,冰凉爽口,还带点酸甜的口味。如果再加些时新的鲜果,如蜜.桃、西瓜,味道就更好了。跟后世夏日解暑的冰激凌一类的冷饮也差不多,正是夏日最受韩家儿女欢迎的甜点。

不过酥酪成本不低,在外面也只有正店一级的大酒楼有卖,剩下的就是豪门显宦才吃得起。韩家当然不会吃不起。酥酪既可以热着吃,也可以当做冷饮,营养也不差,便被当成食补的方子,给儿女日常食用,王旖周南她们也是经常吃。

周南摇着扇子,她其实有些怕热,抱怨着:“官人你都不用冰块解暑,害得家里都跟着你一起吃苦。”

韩冈将最后一勺酥酪,送进嘟起的小嘴,笑道:“心静自然凉。放再多冰块也比不上自然的凉风。”

韩冈做到了龙图阁直学士,冬天有赐炭,夏天有赐冰。一天有三十斤的赐冰,不过也没大用。三十斤说着不少,也就两水桶,唯一的好处就是干净,是冬天从金水河中取上来的。

金水河是宫中专用的饮用水来源,宫中不多的几口甜水井,专供天子一家,下面的宫女内侍全都是要靠金水河的水。河水流经城中坊廓时,渠道上都盖着厚重的石板,还有巡卒防止有人偷水,水质一流。

而北方的豪门宅院,基本上也都不会缺少专门藏冰的冰窖。在韩家厨房下的冰窖里,也存了大量的冰块。不算多,也就两三万斤,十几个立方而已。

家里有这么多冰,韩冈却不喜欢。他并不喜欢用了冰块后的阴湿感觉,热一点也无所谓。

韩冈拿起扇子换了个手,顺带着也帮周南扇着风。周南很享受的眯起眼晴,像只猫一般蜷在韩冈身边。

“你也不要太贪凉,刚生过孩子没多久。”

周南嗯了一声,却也不睁眼。

韩冈的六儿子已经快两个月了。本来周南怀孕时安安静静的,都以为是个女儿,谁想到又是个儿子。

韩家不缺儿子,韩冈倒是想再来个女儿才好,生下来知道是儿子时,甚至还有些失望。不过这等抱怨不能传出去,否则天子听了,能气疯掉。

不过生儿子也好,稍大一点就能过继给两个兄长了,还了父母的心愿。来自于后世的韩冈本是不在乎这些事。何况过继给兄弟房后,还能推出去让两个过世的兄长养?还不是养在自家家里!只是个名义而已。

