\section{第32章 金城可在汉图中(17)}

只剩七天了。

来自北方的探马越来越多,而大宋的游骑,每日从城中清晨出发,回来时总会多多少少的少些人。

借住寺院的制置使司衙门中的空气,一日.比一日沉重。

唯有韩昂与众不同,一派轻松的让人看了心中发恨!

“胜者在敌,败者在己,我能做的只是做好准备,先立于不败之地。至于能不能赢。能赢多少,那就要看辽人的表现了。”吃完饭后,韩冈端着一杯热茶在偏厅中慢慢品着。

“不只是要看辽人。”章楶补充,“还要看河北。郭逵若能弄出个大捷来,河东就能平定一半了。”

“要是郭逵能打到燕京城下,辽军直接就会拼命的往回赶过去。”

“不可能打燕京的,多半是攻一攻易州就差不多了。”

韩冈很喜欢这个气氛,幕僚们的商议往往能给彼此带来启发。当然,也包括韩冈。

易州勾连飞狐陉。从飞狐陉向西,便是代州。本来飞狐陉东半部属于辽国易州,西半部属于大宋代州,现在却是都给辽人占了。不过一旦易州被攻下来,析津府的南大门被打开不说,仅仅是飞狐陉东半部落入宋人之手,对于河东的辽军而言,便是要面临灭顶之灾的危险。

“那只是飞狐陉,井陉怎么办?”陈丰问道。

“寿阳不丢就没关系,丢了寿阳还有平定,丢了平定还有承天军寨【娘子关】,过了承天军寨那才是河北的井陉县。”章楶说道,“这与飞狐陉不同。东面是辽国的易州,西面的代州再失陷,瓶形寨【平型关】纵然地势险要,可两面夹击而来,一样是守不住。”

章楶对地理的熟悉让人惊叹,就是黄裳也不能如此举重若轻的举例。

“但那样,辽人在河北不就没有兵了?”

“为辽人担心作甚?就算现在,南京道中的辽兵也差不多有十万呢。”

“易州岂不是危在旦夕?”

章楶道:“这十万人是整个南京道的总兵力,真正能参与到易州之战中的不会超过三万。从兵力上来说,还是以官军占据了绝对优势。”

“但飞狐陉和井陉怎么办?”黄裳反问道。

易州本来就是南京道上的重镇,又连通飞狐陉,驻军本不在少数。但在座的所有将帅都不担心这些兵马。他们怕的是打到一半,辽人的大股援兵赶来怎么办?

万一辽军从井陉和飞狐陉杀出来,到时候前线的官军甚至有全军覆没的可能。当年太宗皇帝之所以功败垂成,败在了燕京城下,就是因为耶律休哥早一步率军赶到了幽州,出乎太宗皇帝和开国众将的意料之外。

“且不说辽军会不会从两陉谷道出兵,以郭逵的老道,会吃这个亏?”章楶摇头。

郭逵不会放过这么好的机会。若是没有这点抓住时机的能力,他也妄称名将了。

就是不知道辽人给他准备下的是什么样的招待。耶律乙烯不可能不加以防备。这就要看郭逵和耶律乙辛谁更棋高一着了。

“麟府军差不多该过汾河了。”话题从河北绕了回来。

黄裳立刻道:“再过几日或许就能到忻州。”

“希望那时候,忻州城还没有丢吧。”

“保住忻州可没那么容易,位置卡在大路上。”

河外的麟府军主力虽然在胜州前沿,即便韩冈让其放弃胜州,也至少要一个月以上才能调过来,但一部分镇守府州的核心兵力则可以将这个时间缩短四成,而将最后的目的地自太原改为忻州,则更是只剩一半。

尤其早在韩冈之前,王.克臣也下了调令,再有了韩冈之后的补充,麟府军的出现将会出人意料的早。

忻州城还未陷落。确切的说,是至今还没有陷落的消息。

不过没人看好忻州城。正卡在代州、太原之间,而且是控扼着唯一一条官道的位置,战略位置至关重要,辽人必然会全力以赴的攻打。今天没有消息,明天说不定就有了。

至于忻州会不会出人意料的坚守住,那的确不是不可能。只是韩冈早前也曾对他们几个幕僚说过,这件事可以期待,但不要指望。而来援的河外军会怎么选择前进的方向,还是让人担心。

“若来的是西军,根本就不用这般担心。”田腴叹道。他可是在横渠书院正儿八经学习过的,比谁都清楚韩冈在西军中的威望,更对西军的战斗力有一份迷信。这份了解,不是黄裳等人可比。只有章楶多少了解一点。

“可西军才来了七千啊。”黄裳同样长叹息。

现在从汾河谷地上来的援军只有七千人,而且只走到了阳凉关。抵达河谷北端出口的介休,还有不断的距离。而到达能直接支援太谷战场的平遥县,更是遥远。在十天之内,不会超过一万。真正要能够达到足够改变战局的数量,则不仅仅是时间的问题。

“枢副有没有给吕枢密写过信?”陈丰突然问道。

黄裳摇头:“没有。只给朝廷上过奏章。”

陈丰表情发苦,这么说来七千援军根本是关中主动派来的。可能是吕惠卿得到了朝廷的诏令,直接调动了河中府的兵马——河中府虽然属于关中,但其位置却在黄河之东,是长安面对河东的屏障。

“也就是说,短期内,能派上西军也只有七千人了?”

这完全是杯水车薪。以西军的兵力,才挤出七千人根本是打发叫花子。再怎么说,长安也该有兵。兴灵之役打得再激烈,也不可能将整个关中的兵力全都抽调走。

黄裳嘿然冷笑:“对朝廷来说,他至少是派了。”

派与不派完全是两个性质。不派兵,不论河东结果如何,都要面对朝廷随之而来的怒火。可只要派了,这就代表吕惠卿将河东放在心上。就算没有更多的援军,那也是形势所然,事后也不能说他的不是。

这一点连陈丰都明白,如果想要让在陕西的吕惠卿全力相助,除非韩冈向他低头。可以韩冈的心气,可能会向吕惠卿低头吗?

“都是为了国事,哪有低头抬头的说法?如果真的有必要,枢副肯定会低头的。”田腴对韩冈的性格为人还算了解,“现在枢副既然连封公文都没往关中送去,自然是有很大把握的。”

韩冈当然有足够的自信。作为他的幕僚,章楶、陈丰在他脸上看到的永远都是自信满满的神情,看不出半点虚怯。那份从心底透出来的自信,是绝对伪装不来的。

可莫说几位新人,甚至就是跟随韩冈时间不短的黄裳和田腴,心里也有些犯嘀咕。支撑他们信心的,并不完全是韩冈对计划的解释,而有很大一部分是因为计划来自于韩冈他本人。其他人也基本上如此,甚至更甚,其信心几乎全都来自于韩冈。

如果是在太原城中,当然是另作别论,可惜现在是在太谷县。城防水平在诸多县城中,绝对可以排在前列,但与河北、河东、陕西的府城、州城相比,还是有距离的。

可想要达成目的,进驻太原城是没有任何意义的,躲在南面的山谷中,同样没有任何意义。

韩冈坐镇在太谷,还让来援的京营禁军在威胜军铜鞮县集结,然后依照他的吩咐,将大营安扎在威胜军最北端的南关镇,到太谷县南端的盘陀一线,与太谷县相距不超过四十里。而这座谷中连营形势的大营,其在谷地北端峪口处的前进营地,与太谷县则更近了一半。

虽然现在抵达的兵力并不算多,可接下来的十天,将陆陆续续还有三万兵马齐集太谷县南方的大营之中。

其实要是把大营安在山口外,再接近太谷县十里,甚至五里,那辽军是绝不会过来的。

太过稳固的犄角之势,将会让任何攻打太谷县的军事行动成为笑话。只有保持现在的距离,才会让太谷县成为一块让辽人忍不住咬上一口的肥肉。

从峪口到太谷,超过二十里的平原地带,步兵要走上半日的路程,足以让骑兵发挥出自己的威力。将成阵列的步卒拖住拖垮——至少契丹人应该有这份自信。

……………………

韩冈硬是给了辽军施展的空间,其用心不问可知。都不用多说,章楶、黄裳、田腴都看得出来,自然辽人也能看得出来。

来来往往的远探拦子马早就将韩冈在太谷县周围的布置打探得七七八八,虽然在这过程中受了一些损失,也跟宋国的百姓、以及宋军的游骑有过多番交手,但比起得到情报,那点损失实在算不了什么。

理所当然的,韩冈的计划便在辽军将帅中引起了激烈的争执。

“那是明摆着是陷阱!”

“可只是太谷县啊……太原打不下来,区区一个县城如何打不下来?”

“怎么可能只是一个县城?!那只是鱼饵,没看到后面的钩子吗?!”

“但尚父的吩咐怎么办?”

“那是因为尚父还不知道韩冈的打算。”

张孝杰烦躁的敲了敲桌子,让大帐中的声浪稍稍平息了一点。他与坐在身侧的萧十三交换了一个眼神,皆在对方的眼中看到一丝无奈。

其实争论的最后一句话并没有说错。

‘韩冈在哪里。’

当耶律乙辛说出这句话的时候,韩冈还没有进入太原府。

以当时掌握在耶律乙辛手上的情报,尚父的意思也不过是围定进入太原的韩冈,一批批的击败赶来援救韩菩萨的宋军。若是行动快一点,在半道上截住北上的韩冈,那更是一桩不可思议、可以留名千古的战绩。

但韩冈现在可是坐镇在太谷县,根本就没有继续前进半步的打算,反而想着将他们引诱南下。

这一点,萧十三可以肯定,耶律乙辛是绝对不可能事先猜测得到韩冈竟然会以自身为饵,引诱大军南下决战。

对萧十三来说,就算现在直接退走,劫掠来的战果也足以填饱任何人的胃口,回到朝中,绝不会受到尚父的斥责。即便丢掉的那块肉,会让人在日后的日子里,一想起就会后悔不已。

是的……绝对会后悔。张孝杰可以肯定。

一旦击败宋人在河东最后的抵抗,得到战利品,会丰厚得让人难以割舍。现在所劫掠到了一切,也不到那时的十分之一,甚至百分之一。

那可是开封府啊!

韩冈在河东人望极高,又是被朝廷派来主持河东军事的宰执,人人视其为久旱甘霖,皆认为其必能挽救河东于危亡。但相应的,一旦韩冈死了或被擒,整个河东的抵抗将会立刻土崩瓦解。辽军由此甚至可以一直打穿河东,直取开封府。

在几十里外,有数万宋军随时可以出动的情况下,攻打太谷县的确需要冒上一点风险,但得来的回报则太过丰厚。丰厚得能让萧十三和张孝杰以及他们手下的一众将领,忍不住去赌一把肥肉后面的钩子锋不锋利。

这枚钩究竟能将猎物吊起来,还是让猎物一口咬坏,这就是韩冈和他们对赌的赌盘。

赌还是不赌?

再次交换了意见,萧十三和张孝杰很快就做出了决定。太谷县又不是龙潭虎穴,试一下又能如何?

派了人回去看住了石岭关的后路,又放了重兵在榆次城。如果数日之间破不了城,那就直接撤好了。只要不给宋军围上来的机会,韩冈又能如何施为?难道人还要跟马来比脚程不成?

……………………

夜色渐浓。

普慈寺的依然灯火通明。

一封封急报从北面接连传来,异色的烽火也从北方一直烧到了太古城下。

“怎么了?”见黄裳突然间没了动静,章楶疑惑的问道。

黄裳回头,带着一抹意味不明的笑容:“按枢副的说法,是客人们到了……鱼儿上钩了。”
