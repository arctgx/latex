\section{第13章 晨奎错落天日近(23)}

会议才刚刚开始,两边就针锋相对。

韩冈和王安石将正正经经决定国家大政的殿堂,变成了吵架的市口。

“何谈一鼓作气?”韩冈的声音大得就是在吵架,“河北有一名帅,便能保河北一路平安,但进兵燕蓟,却是胜率渺茫,且败则不可收拾。此时欲用兵于北,是拿国运孤注一掷。”

“陛下。”吕嘉问转身对太后道,“韩枢密献火炮,自谓神兵利器,远胜床子弩。如今神兵已铸千万,却不敢逾越雷池半步,即是如此,又何须空耗钱粮铸炮?”

“陛下,臣昔日说火炮,能做大军御寇的依仗。而吕主计今日的依仗,非是火炮,倒是嘴炮了。辽国幅员万里,带甲百万,岂是易与?若贼人侵疆,国中生乱,则不得不急。若欲兴兵讨境外敌国,则不得不稳。”韩冈转而望着王安石,“昔年先帝问策王平章,只因国库空虚,而臣反对仓促开战,也正有国中钱粮不足这一条。”

王安石沉声道:“西北罢兵,军费移至河北,足以供给战事之用。”

“战事一开,金水银水亦难济。若是不能一战而定,如陕西一般几十年纠缠不休,平章还能说‘足以供给战事之用’?”韩冈反问王安石,又道,“收复汉家故土,天下士民所望,自不必说。但天下士民盼望的是收复,而不是因收复而带来的惨败。前日平章与吕宣徽畅言北伐,敢问能否一战而定,从此北虏不再寇边?”

“伤有轻重之别,贼有大小之分。举兵攻辽,即便不能一战而得百年安宁,也能让河北得到堪比河东雁门的屏障,北虏大军望山兴叹,使天下士民能安享太平。”

吕嘉问代王安石避重就轻,韩冈冷笑,“兵无常势,水无常形,胜负之机,往往一线。以北虏百年底蕴,纵孙武子复生,亦不敢言必胜。吕主计不敢称必胜,却又自知之明。但既不能必胜,贸然北进,只为一口闲气不成?”

他说着,又对太后道:“陛下,昔年勾践攻吴,十年生聚、十年教训,二十年灭吴。如今先帝生聚教训十余年,事功仅得其半,若仓促起兵,十年辛苦,或将付之流水。以臣之见,仍当厚植国力,再期以十年,十年之后,灭辽不为难事。”

韩冈、王安石、吕嘉问等人,你一句,我一句,分毫不肯相让。

壁脚处的李格非听得啧啧兴叹。

‘这才开场吧!’

一切还是因为太后出场后的那句话,李格非向御座的方向望过去,连遮住太后的帘幕都看不清楚,不过帘幕之后的太后会是什么样心情,多少还能猜到一点。

开场第一句,就被大臣给驳了回来,太后的脾气即使再好,也免不了要动怒。唾面自干,娄师德有那份好脾气,但太后一介妇人,怎么可能会有?

不论是王安石,还是韩冈,只看方才的表现,都是半步不让,翁婿二人之间就像是死敌一般。接下来无论是站在哪一边,可都是把另一方往死里得罪。

一边是势力遍布大半个朝堂的元老,另一方则是得太后全力支持、名望重于天下的新贵,不论站在哪一边,所要面对的敌人都是强大得让人绝望。

即便其中任何一方在现在的情况下,都奈何不了对方的首脑,可拿下面的人开刀,却都是轻而易举。

能够在今日殿上拥有一张选票,离开两府的距离就不远了。都走到了十步之内,谁人能够无视清凉伞的诱惑?而现在想要进入两府,就必须在朝堂中得到足够的支持,没有一个还不错的人缘,另一方面,他们也需要太后的准许。

如果之前还能幻想一下不会受到报复,现在看一看双方剑拔弩张的样子,就知道这完全是幻想。

幸好自己还差得远,李格非暗暗庆幸。身居高位,固然是桩美事,可也有高处不胜寒的风险。

身负于殿中监察朝臣举动的任务,但李格非可不想现在跳出来打断双方的争吵,还是安安静静的看下去更安全一些。

李格非不觉得自己有哪里做得错了,殿中侍御史也不独他一人,不都没有站出来维持朝堂秩序?!近处还有韩冈的心腹方兴,一样站得安安稳稳的。

这样想着,李格非多打量了方兴几眼,随即就惊讶起来。

今日的会议开场就紧张激烈,韩冈得到太后的支持后,仍然受到新党的围攻,方兴虽然与其他朝臣一样关注着上首处的争吵,但紧张的程度并不算深,反而有几分有限的感觉。

是因为这是翁婿内争,外人干脆看热闹?

新党、旧党相对,韩冈虽与新党决裂,可气党和新党就没有相对的意义,总之不那么贴切。稍稍刻薄一些的,就是称呼王党、韩党,以姓冠之,比拟于唐时的牛李二党;更刻薄一点的就是翁党、婿党。但不论怎么称呼,都是在说韩冈自成一派,与王安石打擂台的事实。

但自家可以这个态度,方兴怎么也是这般,还是说他已经胸有成竹?

李格非想不明白。

这时候愿意蹚浑水的并不多,很多朝臣都不想在这个时候被迫选边站。

双方的唇枪舌剑不见止歇,原本为了直接解决争议而举行的会议,因为一人的不服气,再一次陷入了混乱,向太后心中不耐,“够了!”

她刚刚张开口,就听到下面一声更加响亮的呵斥,“够了!还在吵什么?还是说有人觉得,今日之会不合时宜?”

朝臣们惊讶的发现,存在感一直都比较单薄的首相韩绛站了出来,

韩绛没理会班列之外的王安石、韩冈等人,怒瞪着殿中的御史们:“殿中侍御史何在?!有人渎乱朝仪,尔等为何坐视不理?!”

李定陡然变色。

韩绛出面维持朝纲,这是在讨好谁?自然是太后。

而太后又是站在谁的一边?那就不用说了。

韩绛对韩冈的支持,这还是第一次如此旗帜鲜明。任谁都知道,在如今正在争论的伐辽一事上,出身河北灵寿的韩绛,是绝对支持韩冈的。但国是不同,之前支持韩冈,只是反对一场战争,现在与韩冈站在一边,却是在反对整个新党。

所以这段时间朝中都猜测韩绛即使有偏向,也不会公开表明支持谁。三次为相,韩绛已经没必要再蹚浑水,灵寿韩家的地位,谁在台上都动摇不了。就像洛阳那几位元老,即使败出朝堂,每天生日,朝廷照样要遣使问好,逢年过节,赏赐照样不会短少。

但韩绛还是表明了立场,这当然让许多朝臣惊讶莫名。甚至王安石都不免心中动摇,回头深深的看了老友一眼。

排在班列后方的陆佃也不安的挪动了一下身子,一下子变得不自在起来。他没想到韩绛会支持韩冈,这或许是韩冈如此自信的原因所在,有太后,有首相,的确能够分庭抗礼。

方兴只排在他后几位,一切都看在眼里,冷笑了起来。

别忘了韩相公是哪里人?!河北灵寿啊!

如果能够一战击败辽人,那当然最好,河北就此太平了,但吕惠卿能做到吗?

韩绛了解一切,吕惠卿为了回朝,所玩弄的那些伎俩,又岂能瞒得过韩绛的眼睛?

韩绛与老朋友对视了一眼,眼中没有交情,只有决绝。

既然不准备打虎,却偏偏要去捋虎须,事成之后,自己悠哉悠哉的回京为相,却在河北留下了一个烂摊子。吕惠卿是福建子,自不担心辽人的铁蹄,可韩绛不能不为乡里担心。

在韩冈站出来之后,他就彻底的站在韩冈的一边了。不管王安石和吕惠卿是当真想与辽人打上一仗,还是只想借机混些功劳回京,韩绛都不能容忍有人拿着河北的安危做自己的垫脚石。

有了太后的支持,有朝中唯一的宰相支持,韩冈已经不再处于劣势,这一次的胜负,一下就变得扑朔迷离起来。

韩绛一怒,在短暂的震惊和静默之后,王安石、韩冈、曾孝宽、吕嘉问齐齐请罪,向太后道,“不用多说了,相公、枢密对于今国是,可有何提议?韩参政,此事是你先提出,你先来说。”

韩绛成了敌人,吕嘉问心中正怒,但太后这么一发话,他差点就要笑出来。

太后或许是要帮韩冈,让他先声夺人,可惜的是,她这是帮倒忙。

先开口不是好事,等于先暴露了虚实,后面的人可以根据他的提议而做出调整,原本因为韩绛而五五开的胜率,至少又有一成倒了回来。

韩冈这一次当是有苦说不出。

韩冈也是停了一下,才迈步出班,朗声道:“先帝念兹在兹,不过富国强兵。新法施行十有余年,国仅小康,尤未富也。于今皇宋生民亿万,一人一年仅食两石,亦要两万万石。今日国中积储多少?臣闻‘国无九年之蓄,曰不足;无六年之蓄,曰急;无三年之蓄,曰国非其国也’,今日皇宋,可有九年之积?”

他瞥了一眼王安石,而后继续道,“且不说九年之蓄,三年之蓄可有?大宋幅员万里,无一年无灾异。十分国土,有灾异者,多则十之五六,熙宁是也;少则十之一二,今日是也。若无千万粮谷,如何保得住中国无乱事、无流民?若要北上攻辽,收复故土,如今所作的准备更是远远不够。”

“所以今日朝廷所要做的,是继续变法,而不是抱残守缺,不求进取!”

