\section{第十章 千秋邈矣变新腔(六)}

“平章在说什么?”

向太后的问题传入王安石的耳中,这位位极人臣的平章军国重事,发现自己竟然如此失策,竟然将还没有确认过的消息当了真。

但他一瞥眼,看见自己的女婿后,立刻又醒悟过来。

竟是给这小儿算计了!

韩冈在自己的面前,从来都是直截了当,即便立场迥异,但也可见其直。

不管怎么样,如果是在官场上,立场当然重于人品王安石当初为了变法,明知来投的许多官员,各有各的问题,但只要他们。反正他的对手们,那些自命清白的旧党重臣,也没几个是干净的。

但换成是自家女婿,人品可就要比立场更重要了。至少在今天之前,王安石还是从来没有怀疑过韩冈的品性,即便每每被气得七窍生烟,但这个女婿,王安石自始至终都认为找的没错。

可今日韩冈为了黄裳被黜落一事,将做考官的蹇周辅他们逼得来找自己,转头又上殿求见太后,等到自家匆匆赶过来,要将此事分说个明白,太后却回了一句,‘平章在说什么?’

换作是别人,王安石还不至于如此疏忽大意,但面对自家女婿,王安石都没多想,韩冈的儿女这几天还在家里小住呢。

一时不查,落入了如此窘境,王安石没有气急败坏,也没有丝毫畏缩,深呼吸了几下,压下了心头怒火,仰头直言:“臣说的是这一次制科的阁试,黄裳被黜落一事。”

“黄裳被黜落的事,吾方才就知道了,今次制科就三人入选御试……难道是弄错了?”向太后的声音中充满了疑惑,“参政,王平章方才所陈之事,参政可知晓?”

“臣已知,黄裳的确是被黜落了。方才臣因为不解黄裳落榜,曾遣人去崇文院求取黄裳考卷,对此知之甚详。”

“究竟是怎么回事?”

向太后立刻追问,让王安石这般气急败坏而来,肯定不会是小事。而且王安石入殿拜礼后的第一句,她也是记得清清楚楚。

“黄裳于阁试六题中,只有一题不知出处,此外有四题写明了出处,并正确引用了前后文,剩下的一题,也仅仅是在列举七十余唐时宰相姓名时,与原书有一条错讹,其位置顺序错了,故而被判错。”

“姓名前后顺序?这可不能错,之前为了杂压合班,可是吵了好久,啊,当时参政还没回来,当是不知道……嗯,也许知道,参政应该看了朝报吧?”

“……臣知道此事。”韩冈停了一下才回道。

“参政也知道,当时出了这件事,可真是不合时宜。”向太后叹了几声。

朝堂上文武百官的站位顺序,关系到其地位高下,也关系到官员们相见时的礼节,不同官职排在什么地方之前早有规定。但前段时间,也就是韩冈还在河东的时候,太常礼院上书说之前的合班之制有错,要改一改。只为了这件事,朝堂上下吵了好些天,奏章一时间都比军报都多,让向太后想起来就头疼。

叹了几口气,她随即又不解起来,“不过黄裳都作对了四道题,怎么还会被黜落?不是六题里面四题判‘通’就通过吗?”

“因为黄裳有一条论上被判了‘粗’。四题之中有一题,因为新学与气学论述有别,黄裳依气学的道理做答,所以被知阁试的蹇周辅等人,判了‘粗’。”

“……哪一边是对的?”向太后突然变得小声了一点。

韩冈微微一笑,朗声道:“臣当然主张气学和黄裳。”

“嗯……平章呢?”

王安石冷着脸:“昔年先帝一道德,将传于天下。方今天下士子皆以为是,礼部试中亦皆以为是,制科阁试,又何能例外?”

“臣不知平章何出此言?我等治学,岂能以朝廷权势压人,而不穷究其理?”韩冈摇头,“中有一章,可见平章对诗经浸淫之深。不过对中的小宛》这一篇里面的‘螟蛉有子,蜾蠃负之’这两句,臣之所见,与的解释有些区别,敢问平章,对错如何?”

韩冈这是当面给王安石难看,在这一条上,王安石根本无法辩驳。

现在世所共知,螟蛉义子的说法是彻头彻尾的错误。不要说王安石的有错,就是流传了多少年,由毛玠作注、郑玄作笺、孔颖达作疏的,都错了。

揪住千古以来诗经释义的错误,证明了格物致知对经义的价值,是气学发展上的一个里程碑,由此在士林中被视为新学的头号挑战者,而不是众家异说中的一家。

向太后也听说过这件故事,因为螟蛉义子的说法实在是太有名了。

王安石脸色更冷,硬邦邦的回道:“已然改易!”随即又辩道,“区区一条,能证明其他都有错?”

“既然改了,也就是之前平章的见解是错的,也就证明平章的著作并非十全十美,能万世不磨,为世人圭臬。那么今天的这一条,就又当真没错吗?”与王安石的黑脸相对应,韩冈脸上一直维持着若有若无的微笑,“一书,即便是其,在最近从殷墟中发掘出来的,也已经有了一些值得商榷的地方了。”

“荒唐之言,荒谬之论,完全不值一驳。”王安石哼了一声,“朝廷不遣重臣监守殷墟,不说盗掘猖狂,就是世间也多了一干无知乡儒,拿着片有几条印痕的龟板和骨头,就敢对经典指手画脚。”

几年过去了,韩冈当年揭开的盖子,如今正在持续不断的冒着热气,出现的成果已经烫伤了好些大儒和一直以来作为主流的观点。王安石的新学更是成了攻击的重点。不过现今在儒林中已经有了些不好的风气,一些儒者都开始将颠覆性的观点托名殷墟出土,而宣讲于人,弄得儒林的风气越来越差。

韩冈随即道:“沙砾之中,亦有真金,只需格物致知便可。”

“平章!参政!”见王安石和韩冈的争论已经向不知所谓的地方滑过去,向太后连忙提声提醒。

王安石和韩冈立刻停止了争论,恭听太后训示。

向太后问道:“参政今日求见,是不是也有为了黄裳被黜落这件事。”

韩冈瞥了王安石一眼,却承认道:“就此事,臣的确有想法要禀报于太后。黄裳明明是军谋宏远材任边寄科,却跟贤良方正能直言极谏、才识兼茂明于体用两科做一样的考题,这是要招揽精擅兵法的贤才,还是书呆子?臣不讳言,以臣的才识,去做今科的考题,也肯定过不了。”

韩冈自陈过不了阁试,可当今看谁能说他不是朝中戍边帅臣中的一把好手?

“参政太自谦了。”向太后连忙说道,“那以参政的意思,是要让黄裳通过,还是重考?”

“不论是对是错,既然知阁试的蹇周辅等人已经定下了结果,就不能再改。改易已定登科名单,此先例不当开。并非臣认为黄裳不够资格上殿御试。只是朝廷威信远在黄裳一人之上,即便是错,也必须将错就错。”

“……参政这是公忠体国之言。”向太后感慨着。

王安石听得心中冷笑。到了这时候,韩冈肯定要撇清。不过韩冈还是承认他有打算对黄裳落榜一事报与太后,只是放在了代州的几件事之后。这让王安石感到意外。难道韩冈还不想最后决裂?

“平章?”向太后问着王安石的意见。

王安石立刻道:“臣无异议。”

“既然不是为了黄裳,那参政想说的是什么?”向太后问道。

“臣想说的是三馆秘阁。崇文院想来是朝廷的储才之地,选入其中者皆当是儒林英才。可蹇周辅等人连科目不同,考题自当不同道理都不懂,说其滥竽充数或许过当,迂腐颟顸这四个字,蹇周辅等人却是逃不掉。”

韩冈很难为黄裳再争取,既然考官已经判定了他落榜,事已至此,想要挽回是不可能的,走制科这条路的前途,黄裳已经没有可能了。但韩冈可以让那几位考官付出代价。

暗地里送了考题的人,韩冈知道是谁,但他无意去追查这两人背后是谁。而提议将黄裳黜落的人隐藏得太深,韩冈无法分辨到底是谁,但他可以确定,这些都不是他的人。

“迂腐颟顸?”

“蹇周辅几近六旬,赵彦若也有五旬,此辈皆是老迈不堪,却仍得以留在崇文院中……”

“参政是要将他们都外放地方?”

“不。”韩冈又摇头,“当初范文正公曾经说过,一路哭何如一家哭。放蹇周辅诸人出外,祸害的可是一州一军的百姓,数万军民官户,几十万人口。两害相权,还不如留他们在朝中。”

“到底该如何罚?”太后问着。

“不当罚!”王安石立刻叫道:“无罪岂能处罚?!无罪受惩,蹇周辅等人岂能再觍颜留在朝中?三馆秘阁之中,何人补缺?”

韩冈立刻道,“自古至今,只闻国家缺贤,未闻朝廷缺官。”

爱干干,不干滚。
