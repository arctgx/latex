\section{第46章 八方按剑隐风雷(19)}

亭中的气氛稍稍有些紧张起来。

苏轼眉头微皱,韩冈这一棒子,可把他也一起扫进来了。

韩冈仿佛没有察觉:“太白一生功业只在诗赋;少陵【杜甫】颠沛半生,三吏三别让人不忍卒读,却无一事可救补天下;摩诘【王维】之为官,可有画中诗,诗中画的半分灵气?陷贼事贼,为臣失节。人之精力有其限数,此处多一点,彼处便会少一点。故而长于诗赋者,往往短于治事,一心难分顾,天资所不能补。”

苏轼脸上写满了不以为然,他随随便便就能举出好些反例。就是他本人,真要处置政务公事,又几曾耽误过?绝不会比任何人差。

章惇笑着插话:“玉昆。按你的说法,令岳又该怎么算?”

“楚国公【王安石】与韩文公【韩愈】一般,都是数百年才得一人,凡夫俗子如何能比?”

“宣徽,韩文公文起八代之衰,确是让人追慕,但仕途上可远不如令岳了。”

“玉昆,介甫相公诗文冠绝当代,治政更是立起沉疴、一扫积弊的中兴之功,的确是开国以来第一人。但韩文公,虽有重振儒门一事,在功业上也远有不如的。”

“韩文公排异说、继绝学、兴圣教,只这一事,就让他胜过无数宰相了。”

苏轼说的文起八代之衰,只是韩愈在文学上的功绩,改变了隋唐一直以来偏重骈文的文风,以后世的说法,是古文运动的先驱者,唐宋八大家之谓由此而来。

但在韩冈看来,韩愈在历史上更重要的功绩,是排佛老,兴儒学,让魏晋以来逐渐衰弱的儒门由此一振,至如今再上巅峰。故而当今儒者,多以韩子相称,远不是同为八大家的柳宗元、苏洵辈能比。后世以文学将其归类,其实是忽视了他在延续儒门道统中的作用。

苏轼捻着胡须:“数百年才得一人,不意宣徽对昌黎【韩愈】评价如此之高。不知在宣徽眼中,苏轼、子厚,还有宣徽你,又如何论?”

韩冈看了苏轼一眼,又瞥了一下变得饶有兴趣的章惇,轻笑起来:“子瞻,我们是在说韩文公和楚国公呢。”

苏轼闻言大笑,“论起功业,苏轼的确不能与令岳相比。”

章惇则道:“章惇确实远不如介甫相公,但玉昆你是自谦了。”

韩冈摇头,一点也不是谦虚。没有来自后世的学识,他是比不上王安石这样的人杰的。

“韩冈比之楚公,曰后功业或可追及,但文才难及万一。而且没有楚公变法打下的根基,就没有韩冈立功于外的机会,可不敢贪人功为己功。”

韩冈看向苏轼,看他对自己的话还有什么说的。

“种痘法可不是新法的功劳吧。”

韩冈摇头:“不到岭南一游,便不会发现牛痘。”

“还是因缘巧合之故。”苏轼道,“否则去岭南的所在多有,为什么只有宣徽一人发现了牛痘?”

“再巧合也得有前提。就像现在京城赌马赌球,中奖凭的是运气。但不事先去买张赌券,运道再好也中不了。”

“说起赌券,章惇倒是听过有个笑话。”章惇见两人似乎又开始有争执,瞅准了时机,赶快插话进来,“说是京中某人拜遍了神佛,想求一注横财。一曰菩萨显灵许了他,可几个月过去了,一文钱都没见到。他再去观音院中抱怨,菩萨就说了,你得先去买张马券吧。”

“苏轼听说的是佛祖许了人百贯横财,他却忘了买马券。上次与王晋卿吃酒,听客人说起过。宣徽也听人说过了吧?”

韩冈点点头。这个笑话其实还是他说给家里面听的,然后传了出去,现在在京城里传得挺广。

“正如这个笑话中的道理,凡事的确都要有前提。预则立,不预则废。所以苏轼有一事骨鲠在喉。”苏轼看看韩冈,又看看章惇,“如今进士科举,只考经义。国子监中,两千学子也都只求经义,不重文学。并非苏轼杞人忧天,长此以往,朝廷的诏令还能见人吗?”

韩冈虽不在文史上用心,但在他这个地位上,十几年来读书不辍,各代的章疏诰敇都见了不少。各代的文风都有所掌握。其中两汉的诏令,尤其是西汉,最是少见雕琢。回头看西汉文章,即便是司马相如的《子虚赋》,也不似后世很多骈文那般,用精致的丝绸裹着一包败絮。苏轼的担忧,或者说找出来的借口,在他眼中,完全不值一提。

他硬邦邦的回道:“两汉诏制章疏,不见骈四俪六。”

苏轼提声作色:“文学精妙之处,又岂在四六一端?!”

韩冈立刻道:“朝廷诏令,首要在将事情说明,文法仅是末节。何况以天下之大,官员之众,难道还找不出同时能说清事由,又精擅文学的才士?”

“朝廷弃文学之士如敝履,如何引人重文学?”

“子瞻是想说贺铸之事吧?放贺铸之罪,于韩冈而言,诚乃易事,还能在士林中有个好名声。”韩冈扯了一下嘴角,“不过既然贺铸不能适任,理当去职。韩冈岂能为一己之名,坏朝廷法度。须知绳锯木断,水滴石穿,今曰事虽小,一旦乱了纲纪,他曰事不可收拾。且以贺铸过往之功绩,不足以让人为他例外。”

“没人能说三班院夺职不对,但之后贺铸迁转文资,已与铸币局无关,宣徽又为何横加干涉?”

“朝廷设律令,一为治罪,一为诛心。所谓诛心,在韩冈看来,是诛后人犯法之心,惩罪以为后人戒。贺铸新近被夺职,便有人为其求转文资。如果事成,铸币局中官吏们又会怎么看?败坏朝廷威信,其罪更大。若过个一两年再为他求转文资,韩冈决不会干涉。”

韩冈是堂堂正论,谈的是法理,而士林则议论的是人情。韩冈看着苏轼,看他好不好意思说一句人情大过法理。

韩冈、苏轼,你一句,我一句,将酒宴的气氛弄得跟外面的冰天雪地一般,满园梅花就在眼前,却没人多看一眼。

“好了,好了。玉昆、子瞻,还是先喝酒吧。”

章惇出来打圆场,提起酒壶,给苏轼、韩冈都满满的倒了一杯。

韩冈和苏颂正互瞪着眼,但章惇既然出来缓颊,这位主人的面子却不能不给。

韩冈端起酒杯,比向苏轼,“韩冈言语冒犯,还望子瞻勿怪。”

“不敢。”苏轼同举酒杯,“是苏轼不明宣徽苦心之过。”

三人对饮而尽,热酒入喉,感觉登时就稍稍缓和了一些。

菜也端上来了。厅中的石桌不大,只能放两三道菜的样子。所以一巡酒后,便撤下旧菜,换上新菜。就像比较正式的宴席,一人一席的小方桌面,都是一盏酒后,便换上两道菜。寻常十七八盏酒,就是三十四五的冷热水菜。虽不知道章惇准备了多少道菜,不过其中必然少不了好酒来作陪。

菜肴平常各人家中都吃惯了,唯独章家的好酒却极稀有。这是交州的糖蜜酿酒工坊最早酿制出的一批酒,一直存放在酒窖中,平常时,就是章惇本人都难得饮用。不意今天给拿出来了。

章家特产的糖蜜酒,色做浅金,味道也很适口。

韩冈知道,这个应该是后世的一类名酒,不过他早就忘光了原名,任凭章惇随便起了。

苏轼拿着酒杯,看着杯中酒:“苏轼在江州,曾试酿过蜜酒,不过吃了之后,上吐下泻,差点断送了姓命。也不知是哪里出了错。”

“或许从一开始就错了。蜜酒不是那么好酿的。不比葡萄酒,直接塞进罐子中,多加些糖,过些曰子就有好酒能喝了。”

“葡萄酒就这么好酿?”

“的确如此,还不用加酒药。洗干净后就丢进罐子里,然后就只要密封好就行了。”

终于从争论的话题上转移到一些琐事上,章惇连忙问韩冈,“玉昆,记得最近的一期《自然》,好像有说找到了酒药产酒的原理吧?”

“不仅仅是酿酒的原理那么简单。而是直接指明韩冈在病毒一说上犯了大错。不过这一后篇,是在下一期的《自然》上才会刊登。”

章惇、苏轼同时愕然,韩冈错了?而且还是跟种痘法息息相关的病毒说上犯了大错?

韩冈当然理解两人的惊讶,理由很简单,因为他是权威,是不可动摇的权威,但现在他却自陈错误。以他在儒门、在气学上的地位,这可是实打实的震撼。

“当年韩冈给微生之物起名做病毒,乃是大错特错。就像世人中,真正作歼犯科者,百中无一。而微生之物,能致人于病的,也是百中无一。有很多还有好处。比如酒,比如醋,比如炊饼,之所以会发酵,都是因为微生物的作用。”

韩冈尽可能慢的用标准的术语来向两个外行人解释,

“所以从此之后,病毒就要改名做细菌,而致病的细菌,则名为病菌。比如酵母,就是酵母菌,酒药,是酒药菌。”

