\section{第43章 庙堂垂衣天宇泰(八)}

【第二更】

“昨天唐州来报,六十万石纲粮已经有八成经由方城轨道抵达了汝州的山阴港,想来最多再有六七天,剩下的事就能有个了结。接下来,转运司这里就有了些空余的时间。”

先将自己的成果展示了一下,韩冈接着说道,“我打算在利用这个冬天,在转运司下面设立一个暂编的卫生防疫局,暂由……”他抬眼看看坐在下首的党项神医,“李德新来主管————眼下工作是集中在推广种痘法上,培训各州派来的医生。但这个卫生防疫局,日后也不用仅限于种痘一方面,可以让他们参与到平日里工作中来,若是遭了灾,缓急间也能让他们派上用场。”

韩冈的话,就是金口玉言一般,面前的人没一个能站出来表示反对。整件事按着韩冈的吩咐一路做下来,对李德新、对黄氏兄弟,都是一桩好事。

“事重情急,这件事得越快越好。”黄裳补充道。

“明天成立如何?”韩冈不信黄历上的宜忌,当然也不在乎什么良辰吉日,“开张庆祝还得押后,不能不顾正经事。”

“明天?……”黄庸察言观色,发现韩冈并不是在开玩笑,就立刻点头:“那就定在明天。明天我就来门前候着。”

“哪里能惊动一州之长?”韩冈哈哈笑了笑,像是听到了一个有趣的笑话,“派些医生来就够了。”

“卫生防疫局中能接收多少医生去求学?”黄庸追问。

“多多益善吧,以韩冈想法,但凡悬壶济世之辈都能懂得种痘才好。”韩冈轻叹摇头,“不过眼下也不需要太多,有一二十人,够使唤就行。”

韩冈和黄庸,你一句我一句,没有什么争论,便将在襄州推广种痘的整件事给敲定。

达成了拜访韩冈的目的,整件事也算是告一段落。韩冈、黄庸端着茶盏,用茶水润着喉咙。黄裳这时乘机开口:“龙图,学生心中有一疑问,不知能否为学生解惑。”

“勉仲请讲。”韩冈放下了茶盏。

“世间医术都说痘疮本为胎毒,因外感风邪而发。但学生看龙图的牛痘免疫法,就觉得痘疮似乎完全是外感,而与胎毒无关。不知对错与否?”

“想不到勉仲对医术也是了解甚深。”韩冈笑说了一句,“正如勉仲之言,以韩冈之见,痘疮纯为外感,非是胎毒。”

“不知二位有没有听说过显微镜?”见两人一齐点头,韩冈继续道,“旧年韩冈发明凸、凹透镜,只是用来给人做眼镜。不过近年来,有人将两种镜片叠放,就有了显微镜。能将镜下的细微之物放大三十、四十倍,一寸大小的虫豸,显微镜下看起来能有四尺。一根发丝可与手指相比。佛家有云,一碗水中也有四万八千小虫。如果将一滴河中或井中的清水放在镜下,就能发现水中尽是活物,不仅是水里,土中,树上,家中,到处都有这些所谓‘小虫’的东西。韩冈将之称为病毒。”

“病毒?”黄裳脑中转着疑问。

“能致病的毒物,自然就是病毒。”混淆了细菌和病毒的定义,韩冈说道,“寻常人身体康健,如同拥有高墙深垒的城寨,病毒难以为害。但换作是老弱或是小儿,就是要了人命。不同种类的病毒,引发的疾病不同。天花或者叫痘疮,也有引发此等恶疾的病毒。”

黄庸眉头紧锁,一时难以接受韩冈的说法。吃喝之中,难道自己当真将那么多病毒吃下去肚去。

韩冈则不管他,继续道:“病毒细小,更胜微尘。飘散在空中、水中,不经意间就能窜入人体内滋生,又能随着呼吸、咳嗽等途径,散播开来。这也是为什么一个人发了痘疮等传染病,周围都有可能染上的缘故。”

“其他病症……”黄裳试探的问道,“比如痨病,也是由于人与人之间接触多了才会感染,是不是也有痨病病毒?”

韩冈点点头,知道黄裳想问什么,“的确是有的,不过想要趁势造出疫苗,还是有些难度。找到发病的原理,才能有针对性的去寻找治疗手段。痘疮算是一个典型的例子。算是因人成事,没有孙师的灭毒种痘法,也就没有现在的牛痘免疫法。至于其他病症,就要看个人的研究了。”

达成了自己的目的,又了解了来龙去脉,还被韩冈上了一堂有关免疫学的课程,心满意足的黄庸和黄裳也不打算在漕司衙门中多逗留。要安襄州百姓之心,明天就要配合着将卫生防疫局成立起来,今天晚上甚至得熬通宵。遂起身向韩冈告辞。

在两人告辞的时候,韩冈送了一台显微镜给黄裳,微笑道:“闲来无事,也可当个消遣,也许不经意间就能有所发现。”

显微镜市面上根本买不到,全都得靠人自己打造。如果真要算一算价值,韩冈送出来的这一架显微镜至少得在百贯上下,算得上是很贵重的礼物了。黄裳为此还多谢了两句,却是没有推辞,看样子对显微镜和韩冈所说的那一段话,有着很浓的兴趣。

黄庸则是拿了韩冈的《肘后备要》,说是要带回去仔细研读。韩冈也不小气,不过是抄本而已,本来就是希望能颁布于天下,成为官员们施政理事的参考书。要是能成为《水经注》、《齐民要术》一般的策问必读课本,那就更好了。

他也不怕泄露出去有人剽窃冒名,都已经可以算得上是卫生医护上的权威了,抢别人的成果不费力气,被人想混走他的成果,却是千难万难。

送了黄氏兄弟回来,韩冈笑着对李德新道:“下面可就要靠德新你了。”

李德新立刻应道:“龙图放心,小人定当用心做事。”

“有你这份保证我就放心了。”韩冈点着头,进了外书房中坐下。又对李德新道,“对了,德新,你还没有表字吧?”

李德新摇摇头,他又不是读书人,还是党项出身,哪里来的表字。

“还是得有个表字,”韩冈说着,“日后你为官朝中,没有一个表字,称呼起来也不方便。”

韩冈的话都说到这地步了,李德新哪里还会不明白。面现喜色,一揖到地,恭声道:“小人的表字,还请龙图赐下。”

韩冈略作沉吟,道:“你既名德新,那表字也就该从此而来。记得《书》中有‘惟新厥德’一句。德惟一,故有‘咸有一德’之语。而‘新’字,则有更易重生的意思。不如就叫做易一吧。”

“易一……”

李德新咀嚼着这个十分别致的表字,有些想笑。都说韩冈不会起名,长子、次子,一个韩钟、一个韩钲,就没在姓名上费过神。‘易一’怎么看也不觉得有多深得寓意,当也是韩冈随口所起。

不过该谢还是得谢,韩冈是一片好心,李德新又不是没有眼色的人。随即跪下来磕了两个头:“多谢龙图赐字。”站起身后还笑着,“从今往后,也算有个合适的称呼了。”

“可不是,没有一个表字,如何能在官场中行走。”韩冈冲着李德新笑道,“以‘易一’为表字,也是希望你能更易旧时之行,一心向国,永为汉臣。”

李德新浑身一下绷紧,脸色瞬间变得煞白。抬起眼,就正对上韩冈锋芒不露,却沉重得如同山峦一般的眼神。

脸上虚假的笑容已经收起来了。没有愤怒,没有失望,从眼神中传递而来的只是单纯的压力,几乎让人窒息。每当午夜梦回,冷汗淋漓的从床上坐起,李德新就知道迟早有一天会如此结果。

李德新一点点的弯下腰,屈膝跪倒,额头紧贴着地面上的青砖:“龙图……小人罪该万死……”

韩冈抬起手,示意李德新站起来,不要摆出一副五体投地的动作,“都这么多年了,过去的事我也不想计较。仇老将你当亲儿子看,我于情于理也不能让他因你而伤心失望。”

一想到已经在天水县隐居的仇一闻,李德新涕泪纵横起来,喉头哽咽着:“小人对不起先生……小人对不起先生……”依然跪着不敢起身。

韩冈居高临下的盯着李德新的后背。仇一闻是他的老交情,在秦凤路遗泽甚多,韩冈也得给他几分面子,如今他的弟子有事,韩冈就是要处置,也得先知会一下仇一闻。

对于李德新来说,仇一闻弟子的身份就是他的护身符。有仇一闻在,韩冈怎么也得给自己一个体面。就听见韩冈道:“如今你试行痘法有功,不论过去有过何等错失,倒也都能抵得过了。”

李德新呼吸一滞,连忙跪得更加毕恭毕敬:“多谢龙图恩典。小人必一心一意,为龙图将事情做好。”

“好了,易一。”韩冈挥了挥手,“你下去歇着吧,明天开始可就有的忙了。”

李德新倒退着离开了空寂的偏厅中,只剩韩冈一人。

静静的坐了许久,最后他站起身,返回后院。

