\section{第34章 为慕升平拟休兵(15)}

战争已经结束了。

至少对京城的军民来说是这样的。

尽管河北的鏖战依然未有停歇,河东的对峙也还在进行当中,每天从城门运出去的军资装满了一辆辆马车,但辽国的求和使者已在京城盘亘多日,每天在都亭驿中的谈判一结束,当天的谈判成果在当夜就能传得满城都是。

整个谈判过程,一直都暴露在公众中。到了现在为止,辽人的使者虽然不断退让,也只同意将以增加十万岁币的要求缩减到五万,同时两国边疆恢复到战前的状态。东京士民都相信,再有个几天,就能达成岁币维持不变,边境也恢复如初的皆大欢喜的协议。那时候,太平的日子就当真回来了。

冷清下来的酒家重新坐满了食客,两大联赛的赛场上也再一次欢腾起来,任谁都知道,这一场战争,只差达成协议、签订和约这两步了。太平的日子就要回来了。

“都亭驿那边还不肯松口?”

“昨天才压到岁币加增银绢各两万五千匹两,今天怎么也不可能再降的。”

“讨价还价,买菜乎?”

“也跟买菜差不多了。”

“应该很快会答应吧。拿半个代州换回兴灵和武州,耶律乙辛的面子也不算丢。”

“皇后的那位堂兄差不多也该回来了吧?”

“要看耶律乙辛肯不肯放人了。”

“留着又能如何?还能拿来当人质不成?一刺史耳!耶律乙辛也要顾及面皮。”

“当初韩玉昆提议让商人去跟耶律乙辛打交道,这是个好主意。如果能将收益跟耶律乙辛计算清楚,他当是不会再闹着要岁币了。”

“就是怕他觉得骑虎难下,必须要赌上这口气。”

“那也无妨。让耶律乙辛三分又如何?现如今银贵绢贱,五十五万银绢可以给他,换成十万两银,四十五万匹绢就可以了。”

“这是朝三暮四中的猴子吧。”

春风已经吹进了皇城中。

在京的两府诸公济济一堂,从排位最高的王安石,到最后面的薛向,基本上心情都很放松。战争已经到了尾声,正常进入收尾阶段,不会再有大的波澜。

虽然他们每天还要在海量的军需物资的批准书上签下自己的大名,但从各自本人的角度来说,也已经不再需要坚持战争。

之前宰辅们不能主张议和,因为那与向辽人投降无遗。一旦同意和议,地位还不算稳固的他们,会连同皇后一起,都将失去天下士民的信任,乃至对朝堂的控制。

但现在不同了。辽人主动遣使来求和,已经证明了他们的能力,以及坚持与辽人作战的正确性,剩下的只是讨价还价如同买菜的工作而已。

这一场猝然而起的战争,使得新生的新党政府彻底站稳了脚跟,并打掉了所有反对者的气焰。不论是在前线的吕慧卿、韩冈,还是在中枢的王安石、韩绛等人,在天下士民心目中的地位,都有很大的提升。因为天子中风而带来的动荡,也由此安定下来。

国有贤臣。

街头巷尾多有操持这个观点的百姓。要不然怎么能着快就逼退辽军。原本连河东都快占了,这才两个月,就只剩半个代州,河北那边更是稳守着防线。

当年寇准要将真宗皇帝架到澶州,逼着他渡过黄河,但如今只用几个枢密使守边,就把辽人给打回去了。现在双方大军都停在了边境线上,也就没有了战争存在的土壤。场战争打到了陷入僵局,最好的做法就是两家收兵,通过谈判解决问题,不要再空耗国力。

吕慧卿和韩冈的名望自不用多说。就是郭逵,也有很多文臣觉得可以破例给他一个节度使,以褒奖他守住河北的功劳,再厚给赏赐,让他安心的去养老了。

唯一值得忧虑的地方,就是在外的枢密使对和谈的态度。

“韩玉昆那边怎么办?”

一众人望向王安石。

韩冈说是愿意接受和谈,但韩冈提出来的条件,却让人啼笑皆非。

已经夺占的兴灵、武州都要保下来,代州也要拿回来,作为交换条件,则是岁币依旧。当然,兴灵和武州都是旧日盟约中承认其为辽国属地,所以为了名正言顺的拿到手,皇宋可以拿钱买断。自此两地与辽国再无瓜葛。至于出价呢,也就十来万贯——相当于投入进一场顶级马赛中的赌资,还不到蹴鞠联赛季后赛时,砸在关键场次中赌金的三分之一。

好吧,这基本上就是乡下的土豪恶霸侵占穷人田土的路数,先强占下来,然后给两个小钱把地契的问题解决了。

虽然不知道韩冈为什么对这一套流程如此熟悉,可他要面对的的对手不是只知道忍气吞声的贫苦百姓,他们的牙齿很尖,他们的爪子够利,那是头会吃人的大虫啊!

这根本不是为了和谈而提出的条件,而是要引发战争才会提出的要求。

耶律乙辛若是听到这个条件,会答应那肯定是他发了疯,再开战事那才是清醒的。

只是没人愿意挑头出来开罪韩冈,让岳父来教训女婿是名正言顺了。

“吕吉甫的意见今天也到了。”王安石是君子,但三度为相的元老却难欺之以方,“他同意和谈,但辽人开出来的条件,他绝不同意。兴灵之地绝对不能交出去。”

众皆默然。

吕慧卿的态度,在座的任何一人事先都能想到。官军夺取兴灵是这一场战争中最大的亮点,是未竟全功的平夏之役的终结,也是吕慧卿恃之以晋升宰相的功绩,要是当作交换条件还给辽人,他肯定是不干的。

王安石环顾厅内,每一名同僚的神色变化他都收入了眼底。章敦是坦然回望,蔡确眯起眼不与任何人交流,韩绛、张璪同时低头喝茶,曾布、薛向都皱着眉。神色动作各异,心中的想法也肯定不会一样。

与王安石对视一眼,又扫了眼同僚们的动作神态,章敦轻轻啧了一下嘴,“吕吉甫之前已派了五千兵马去抄耶律乙辛的老巢,计算时间,这时候说不定已经到黑山下了。”

这五千兵马,是吕慧卿分别从青铜峡党项余孽和银夏路马军中挑选出来的精锐骑兵,在河东溃败后,立刻被打发上路。

不论此行成败,都能逼迫辽人分兵防守黑山河间地,也间接的帮助了河东的战局。就是韩冈,之后也要感谢吕慧卿的围魏救赵之计。

只是一旦尚父殿下的斡鲁朵受到重创,宋辽两国之间现在的和议,很可能也一并化为飞灰。

故而在得知辽人遣使求和之后,派去陕西征询吕慧卿意见的中使,也附带了命吕慧卿将已经出发的兵马召回的诏书,只是能追上这支北上骑兵的可能性微乎其微。

吕慧卿为了一个集贤相已经不惜一切代价,而韩冈更是不打算在土地上与辽人妥协,真要按照辽人的意见强行议和,就等于是要面对两名军功显赫的枢密使的愤怒。

而且现在政府之所以能得到士林的赞许,是因为让辽人主动遣使来议和,这是前线和中枢配合紧密,且又得到了皇后全力支持才得到的结果。若是变成前后相攻的局面,皇后会站在哪一边在座的都没有把握。而战局会不会因此发生变化,进而导致政局的变化,那就更难说了。

蔡确轻咳了一声:“韩玉昆既然有信心,就让他试一试。一旦成功夺回代州,耶律乙辛就算暴跳如雷,我们又有何惧?”

“可如此一来,京营的损伤必重。”张璪犹犹豫豫的说着。

这段时间,支援河东的京营禁军,最大的人员损耗一个是水土不服,另一个则是各种意外,真正战死的数量微乎其微,这让许多人都放心下来。外地的禁军损失再大也干扰不到京中,但京营禁军一旦有了惨重伤亡,在有心人的控制下,京城内外肯定会出些乱子。

“邃明参政此言差矣。”薛向很难得的主动站了出来,“此辈皆食朝廷口俸,从没有比河北、陕西的禁军少上一文钱、半匹绢,怎么西军和河北军能与辽人拼命,他们就不能?朝廷养着他们又有何用?!”

“但也不能将他们往火坑上堆。”

见气氛不好,韩绛连忙转移话题,“吕吉甫的意见也得好好想一想。”

“兴灵的重要性不用多说,西贼凭借兴灵、银夏两处,拮抗王师多年。现在尽归我有,也的确不方便再还回去。”

“此外,武州的形势也类似。拥有了武州,代州与河外麟府便连成了一线。相互支援无须绕道太原。河东北界自此便能高枕无忧。”

“但现在想要守住武州,难度着实不小。”

“是因为粮草?”

“河北、陕西只要支撑到五月夏收。但河东要打下去,就必须要得到及时的补充。”沉默的曾布突然开口,“河东民夫已经征发到了近二十万,牲畜七万余,光是民夫、牲畜,一个月食用的粮食、草料加起来就是三十万石束之多。再加上忻口寨参战各部,可就有六十万石了。”

宋军一天的口粮是两升,一个月就是六斗。按照韩冈的要求,民夫的食用也同样比照这个数字。所以消耗量的粮食都由官府兜底。这就使得粮草的消耗没有预计的那么疯狂。

两府之中对韩冈的意见莫衷一是,最后也只能压后再议。不止是王安石,连韩绛、蔡确都在叹息,可惜缺了一个主心骨。当然,要是韩冈能早一点收回了代州,就没有现在这么多争执了。

两府中的争执,韩冈并不知道。但他确信自己还有一些能拖延的时间。而且他还有一个助手——吕慧卿只为了他的宰相梦,就肯定要保住兴灵不失。加之章敦、薛向两人的态度还是偏向自己一点。整个枢密院,可以说是站在同一个阵营中。

除非皇后和自家岳父想看到东府西府之争,不然就必须设法安抚和弥合分歧。不管他们的态度如何,对韩冈而言都是争出来的时间。

太原府境内的简易轨道半个月后就要贯通了。忻州境内的轨道也差不多了。这是用在矿山上的简易型运输轨道,且皆是在盆地之中使用,不过民夫和牲畜的征发数量,还是可以减少一半以上。

韩冈所想要做的,就是一步步的向前,再向前!