\section{第12章 锋芒早现意已彰(一)}

十个了。

昨夜王厚说,辽人派了细作窥探军器监,试图从中谋取那几门大将军炮的底细,他带来的人正加紧搜捕

今天便收到了皇城司发来的公函,说是抓到了十名细作。

这数量倒是出乎韩冈意料。

不过这间谍的活计,就像是瓜和瓜蔓的关系,只要抓住其中一个瓜,就能扯着蔓子,将其他瓜一个个的给摘出来。

韩冈并不怕辽国的细作能打探到什么,两国在工业上的差距,以及间谍本身的素质,注定了他们看不到真正宝贵的地方。就是朝堂之内,也没多少人了解到工业体系的重要性。

就像被用作刑具的硫酸,化工产业的标志三酸两碱之一,可在皇城司众人眼中,不过是种可以用来吓唬人的东西。

王厚方才过来当笑话说给韩冈听时,也完全没有意识到硫酸的意义。

把化学药品当做刑具,这个想法是挺新潮的,不过要是他们能亲眼看一看掉进硫酸池中活人是什么样子,恐怕就不敢这么干了。

当然,能将人淹进去的硫酸池还太遥远了,硫酸到现在为止还没能做到规模化的连续生产。但制备的原理,跟韩冈十几岁时学到知识已经很吻合了。

通过实验,人们已经知道,将硫磺燃烧,或是煅烧黄铁矿,会产生含硫烟雾,用提炼过的硫酸来吸收含硫烟雾效果比水更好,可以由此来制作更多的硫酸。

有了硫酸,制备盐酸、硝酸便不再是梦想,乃至之后的纯碱、烧碱,也有了可能。

化学工业,总算是有了点雏形。

但有几个能看到这一点的?又有几人会看重这一点?

就算是放了辽国奸细去军器监和铁场绕上一圈,他们对硫酸不是视而不见,就是见到之后,敬而远之。

去过军器监和铁场的朝中官员也不少了,可所有人只关心钢铁与火炮的产量,对其中的组织,以及各种副产品完全不在意。

韩冈不觉得自己能比这个时代的英杰强出多少,他唯一可以自恃的,就是眼界。有近千年,由亿万人所垒砌的高台,站在上面,看得比此时的任何人都远。

皇城司还在继续努力,试图从已经抓住的细作嘴里撬出更多的东西。

韩绛在了解了内情之后,明确指示,要把辽国安排在京城中的内奸一网打尽,而对硫酸,就当个趣闻咂咂嘴便过去了。

而韩冈在王厚离开后,继续处理政事。

在京百司和监司州县,每日呈递上来的公函是个巨大的数字。

东府中的三位宰执分工合作,大事协商,小事则各管一摊。人事主要是韩绛抓总,张璪、韩冈也占上一块。至于政务,韩绛放手得比较多,张璪和韩冈将之瓜分。偶尔对于没有划分清楚的部分,会有些争执。

其中有一项是没人跟韩冈争夺的,就是厚生司、太医局系统。谁也不会跟韩冈争夺在这个重要性在朝堂中已经排在前十的衙门。

但韩冈在翻看厚生司递上来的奏章和呈文的时候,还是希望有人能为自己分担一点,或者厚生司中的主从官们,能多注意一下,不要事无巨细都发上来。

就像今日,又有一人自称发明了伤寒疫苗,进京来献给朝廷。厚生司不敢怠慢,立刻具本奏闻。

韩冈对此只是付之一笑,提笔批复。

这些年来,有过不少人声称发明了新型疫苗,针对不同的病症,不过所有的疫苗都被证明了是错报,甚至是骗局。

现在但凡有人献上疫苗,都要他自己先试一试,如果被感染后不得病,那就再进行动物实验。

所以到目前为止,只有天花被确认是可以通过种痘来免疫。

一些种痘后死亡的案例,经过排除,九成九以上,都是因为其他病症甚至意外。剩下被确认是因为天花而死的病例,则基本上是在种痘前就已经得病了。

随着时间的过去,已经有两年,国中城镇中没有发生死亡人数达到两位数的天花疫情。而乡村,上报的疫情数量也越来越少。

所有衙门中,保赤局的名声,在民间可说是最好的一个。而保赤局中的官吏和医生,在地方上,也远比州县官更有人望。厚生司的地位,就是依靠保赤局给天下士庶种痘而得来的,同时在运作上,一年也有一两万贯的收益,不需要朝廷补贴。

短时间内,很难有第二种如牛痘一般功效显著的疫苗出现,厚生司要做的,就是将保赤局的工作深化下去,持之以恒。

韩冈有把握,再过十年,天下各路的主要城镇中,不敢说可以消灭天花,至少能让天花这个病症只会出现个案,而不再成为肆虐一方的疫情。

不过另一件事,韩冈就不能大笔一挥就批复下去,甚至都不方便一人来做决定。

这是有关以太医局为主,同时厚生司参辅,出人出力设立医学的决定。

仿效国子监与武学,设立医学,培养医疗人才,最后再安排考试,将医师培养正规化。

这是韩冈的想法。

师徒传承的医术,一直是世间流行的主流。还有一种,就是转业或兼职的士人,范仲淹就说过‘不为良相便为良医’,很多士人读了几本医书,又揣摩一阵医理,再多搜集几张方子,就敢给人治病了。

这样的培养方式,的确能出名医,但更多的还是庸医,决不可能与正规化的教育相提并论,上正轨后,每年就能有一批的达到合格标准的医疗人员,。

尤其是外科技术,依靠旧式的师徒传承,技术能保住不退步都难。想想吧,三四个人在一个小屋子中围着一具尸体,不说能研究出什么,周围的邻居有几个不会出首举报?

而在国家默许的情况下,半公开的进行大规模的研究和对照,这才是促进医学技术发展的最佳途径。

本来太医局中就有培养医生,这时候独立出来,设立医学,依靠之前的基础不会很难,如果年前能初步定下来的话,明年年中,就能开始着手医生了。

医生的含义自与后世不同,而是跟贡生、监生一样。真正

如今对医职人员的称呼,正式一点是医工、医官,民间也有郎中之类的俗称。韩冈也打算正规化,分成上下等级——这也是他现在所能做的,在制度上先弄出个有用的框架了。

废除医工的称呼,日后只有医生和医师。

想要成为医生,就得通过了考试或是得到推荐,然后才能在医学中读书。经过学习,再通过考试,就可以成为正式的医师。

但医师只是开始,之后还有更长的路要走。

“驻院医师,主治医师,主任医师……这么分倒是简单明了。”

韩绛拿着韩冈的计划书,颇有兴致的细细看着。

韩冈向韩绛细细解释着:“所谓驻院医师,就是在那些通过了考试,却还不能独挡一面的医师,他们还需要在医院中训练上几年,等通过了下一级的考核,才能晋升为主治医师。”

“又要考?”

韩绛为之咋舌,看韩冈所分的等级,想要做到主任医师,所要参加的考试次数,快要赶上进士科了。

“总比贡举简单,贡生的资格一科可就只有一次。这里成了医师,不犯大错不会夺其功名。”

“这样人会越来越多吧?”

韩绛很敏锐的抓到了其中的关键所在,名义上是在模仿进士科,但这头衔一直能拿下去,岂不是能参加考试人数会越来越多。

“若驻院医师,迟迟不能晋升主治医师,还可以选择离开医院,自个儿去悬壶济世。诊所只要有医师资格就可以开了。晋级考试并不是一定要考,肯定有很多人有自知之明。不会一次”

“在医院里可是能拿朝廷的俸禄!”韩绛提醒道。

“从医最高能成为翰林医官,而在医院中任职,也算是拿朝廷的俸料。但并不是拿了朝廷的给俸,就能算是官的。外面的卒伍,一年还能拿多少俸料呢。”

“等同于卒伍,怕是很多人不愿。”

“至少能有个希望。特奏名录用的一干人,不都是文学、助教,有几个能入流,得到品级?”

韩绛想了一想,便摇头笑了起来,韩冈的话有些牵强,但最后还是得看结果。

“韩冈打算上请成立医学,主要还是为医院提供人才。京师的城东、城西两家医院已经培养出来大批人才。但十年之内,医院最多也只能普及到各州,在州城设立一座医院。京府大城则可以两所、三所,甚至更多,只会缺人,不会嫌人多。”

“朝廷供给得起?”韩绛问道。

“自是自负盈亏。”

一个医疗体系若不能自己赚钱,就没有发展的可能。

在这个时代,除了维护统治的军队与官僚体系可以在财政收入中分到一块大饼外,剩下的开支就是在典礼上的花销了。至于医疗,从来都是赚钱的。

大规模的医疗福利,只有后世才能做到。而且在效率、成果、开支三者之间,后世也没办法得到一个完美的平衡。

韩冈没本事超越这个时代,他只能铺出一条路,让时代前进得更快一点。

“城东,城西两家医院,可从来就没亏过本。诊金虽便宜,但收入足以支持两件医院的运作。”韩冈又补充道。

“嗯。”韩绛点了点头,不要朝廷多花钱,肯定是一件好事。

“那现在已经悬壶的医工们怎么办?”张璪不知何时走了进来,看起来也旁听了很久,“他们连医生都不一定能考上。”

见是张璪,韩冈就先起来让位给他,等自己也坐下来了,他才说:“就算能考上医学,也容纳不了那么多人。”

“眼下正在行医的医工,都可以得到一个考试的机会,通过考试,可以拿到同医师的资格。韩冈不会在这方面为难人,熟读医书、明了医理,再懂得一点急救的方子,就算是过关了。”

“同?……”

“只能是同了。”韩冈说道。

总不能让一名野狐禅,和医学培训出来的人才,一起诊断病人。

韩绛不置可否,继续翻着韩冈的计划书,看了几行,就又问道:“医官得从军?”

“医官不仅得有才干,还得有功绩,否则何以为官?想要成为医官,必须得有主任医师的资格,但主任医师,不一定都是医官。必须要在军中累积服务时间,进行一段磨勘,之后才能被征选为。想要考主任医师,至少得有十年医疗经验,其实这段时间中,就可以开始就任军医了。”

“现任的医官如何?”

宰辅都有推荐医生的权力,就任宰相不仅可以荫补儿孙、亲戚;门客、仆人、私人医师也都可以沾光。韩绛也不能免俗,这两年,也推荐了几个。

“考试吧,没别的办法。既然能成为御医,想必这样的考试肯定能通过。”

韩绛没有再多问,韩冈也没有继续多废话,最后还是要看结果才对。

早间在崇政殿议事后处理了一个上午的公务,等到午后,韩冈又被招进了内东门小殿。

尚未近前,韩冈就看见自家的岳父大步迎了上来。

“玉昆,今天也要入宫?”

“太后有招。”

“是医学之事?”

韩冈低眉垂眼:“不知,但此事不日自会禀明太后。”

王安石闻言,却不置可否。

韩冈倒是觉得很惊讶,他的岳父什么时候变得这般好脾气了?

韩冈这是步步紧逼,眼下只是医学界,伎术官的最高品级不会超过六品但韩冈准备在明法科之外,设立明算科与明工科的消息,已经在朝堂上疯传很久了。而且还有传言,武学那边他也准备有些动作。

韩冈打算给士人提供更多的入仕通道,由此一步步的加强气学的影响力,等到王安石致仕,还有谁能够拦着他将手伸向进士科?

王安石在朝堂上已经久无动作,随着时间的过去,他在政事堂的中心地位也会下降。韩冈对此虽然放心,但也不免担心是不是伪装。

不过这个时候,韩冈也无心计较。与王安石辞别,韩冈通名后就走进了内东门小殿中。

太后就在殿中,看到韩冈,才放下了笔,说道:“参政。”

“臣在。”

太后问的直接:“那个皇城司是怎么回事?怎么开始捕捉辽国细作了。”

“皇城司奉旨护卫军器监,辽人为重炮所摄,近这几日密谍不绝,皇城司捕贼也是遵从圣意。赖陛下庇佑,如今已报有十人就擒,经审讯,皆已招认。”

“可会屈打成招?范阳郡公就说有一个与他多有来往,当不会投贼。”

那些个‘河北’商人果然还是有关系的。但韩冈不紧不慢,“按道理说,审案应该是罪疑从轻,甚至不当问罪。这十人中或有被误抓、并屈打成招的,但辽国国使正在京中……”

韩冈欲言又止,话只说了半截,但太后会不明白?想也知道不可能。

“若如此,还是得多关上几日。”

“臣遵旨。”韩冈低头躬身。
