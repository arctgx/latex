\section{第48章 梦尽乾坤覆残杯(六)}

夜入三更,苏轼早已入睡。

今曰火情惊动了整座京城,苏轼难得的没有去饮宴玩乐,早早的回家陪伴妻儿。

城中因火情而喧闹沸腾,不过苏轼家宅里却平平静静。

只是到了中夜时分,却从大门处传来了敲门声,甚至不能说是敲门,而是砸门。

哐哐哐的如同在擂鼓,将大门捶得摇摇欲倒。

中书舍人的宅邸并不大,三进的院子,前门的声音清晰地传到了后院。

苏轼在睡梦中惊醒,心脏猛地一抽。

隔着床外的帐子,依然粗暴响亮的声音,让他感觉仿佛回到湖州任上,被解送入京的时候。那时奉旨拿人的御史台,也是这般毫无礼数。

敲门声很快就停了,苏轼知道,有人开门去问了。

“舍人,是什么事?”

身边的侍妾也醒了,坐了起身,手压着被褥掩着胸口,紧张的问着。

苏轼勉强笑道:“不知是谁,还真是扰人清静。”

微颤的双手,暴露了他心中的紧张。

“朝云,舍人醒了吗?”门外传来了妻子王闰之的声音。

“出了什么事?”

苏轼提声问着外面。侍寝的朝云已经下了床,探手将八步床外面的帐帘挂了起来。

没了遮挡,门外的声音这下清楚得多了,“说是上皇大行,请官人速速入宫。”

苏轼顿时松了口气。

皇帝驾崩是大事,但瘫在床上一年多的太上皇死了,对朝堂和国家,早就没有什么影响了。而且他心中还有隐隐的快意。被抓进台狱之中几个月,怎么可能没有怨言?

不过他很快又紧张起来。他这个中书舍人是两制官没错,但有关天家的诏书制敇,都是内制的翰林学士的工作,与外制中书舍人没有关系。这样的传召完全不合情理。

心中疑云大起,但苏轼也不敢拖延,起身穿了衣袍,出去领了口谕,方知是太上皇后招在京所有侍制以上官入宫。至于其余细节,传诏的内侍却一个字不肯多说。

两制官虽然并称,但阶级差得很远。这是天子私人和中书僚属的区别。翰林学士能力压诸阁学士,而中书舍人很多还不到侍制一级,与学士的中间还隔了一个直学士。

不过两制官时常并称,因为都有起草诏令的任务,一旦挂了知制诰,中书舍人的地位便连带着给提高了。招侍制以上官入宫,翰林学士侧身其间是理所当然,中书舍人也勉勉强强成了其中的一员。

知道事不关己,只不过是个添头,就如卖酒送的蚕豆,苏轼安心之余又有几分失落。却也不敢有半点耽搁,随即匆匆出门。

带着两名亲随,骑马转入御街,就看见一队队人马都在往皇城赶去,打起的灯笼连成一条条光流,仿佛是上朝时一般。

但再仔细看,队列中的官员,都是朝会上排在他上首的金紫重臣,至少都是侍制以上。人看着多如过江之鲫,也只是因为皆为高官显贵,随行的亲随为数众多的缘故。

汇入人流之中,苏轼脸上重又多了几分凝重。

方才听到消息时没仔细想,但现在看到之后,则感觉越来越奇怪了。

正常情况下,有宰辅在,只要等到明天上朝就可以了,连夜招其余重臣入宫的情况过去从来没有听说过。

难道说有什么变故?还是说太上皇驾崩这件事,并不是那么简单?

苏轼向着东面犹然熊熊燃烧的石炭场望去,漫天红光,正映入眼中。也许这场火,跟太上皇的驾崩有着脱不开的关系。

苏轼并不知道自己的猜测,在某种程度上切合了事实,直到进入皇城,他还在想着幕后黑手,一直到他站在福宁殿中。

除了沈括,所有在京的侍制云集在福宁殿中。便是刚刚被沈括顶替的李肃之,也是才回到开封府衙收拾家当,就被召入宫中。

对于这一次非比寻常的召唤,一应重臣们各有各的猜测,但蔡确简单直接的开场白,却是任何一个人都没有想到的。

“上皇大行,非因病症,而是事故。”

是事故,不是被谋害,却也不是正常病死。

重臣们一下都没了动静,屏息静声的聆听着蔡确代表太上皇后与两府宰执,对上皇驾崩整件事的陈述。

片刻之后,蔡确结束了他的通报,但殿上依然寂寥无声。

蔡确所通报的内情,惊到了所有人。

既不是二大王卷土重来,也不是太后又在闹事,更不是太上皇后突然看丈夫不顺眼。

而是天子弑父。

若说成年的皇子急着登基,做出了逆人伦的恶行,这事倒不是很难理解,历史上类似的事情多的是。可皇帝才六岁,重臣们再有想象力,也想不到会有这样的变故。

但从蔡确的话中可以了解到,赵煦所做的,只是让人移动了一下暖炉,又让人密封帐幕,以免外界烟尘让重病的太上皇呛到。如果太上皇不是因此而亡,传到后世,肯定又是一则帝王幼时便知孝道的美谈了,这样想来,不是说不通。

炭气致人于死地的事故,虽然不常见,但在列的重臣差不多都做了二三十年的官,没亲眼见过也听说过一点。只是没想到用水洗过,还是能毒死人。

“此事诚千古以来未曾有过,我等几番议论,觉得如何处置都不妥当。现如今只能集思广益,以求顺天应人。诸位可以畅所直言,勿须避讳。”

一时间没人开口。

只要还是一级压一级的官僚社会,下层的官僚就很难当着上官的面说出真心话来。上官说一句请大家畅所直言,可若是有人当真了,下场通常不会很好。

御史台之所以能指斥宰辅,也是因为他们从编制上直接对天子负责的缘故。而现在太上皇后不发话,只有蔡确作为代表出面请求直言,这样的情况诡异的过了头,让所有人都警惕心大起。

“此事太过匪夷所思,可是确凿无疑?”

李肃之心情正不好,反正就要出外了,直接提意见不合适,出头确认一下真伪,他也不怕什么。

蔡确叹道:“若非已经确认,我等又岂敢妄污天子?实在是不得已。”

“如何确认的,又是本于何法?”

“整件事的确出人意料,但除了这个原因以外,没有别的解释了。”韩冈出面代蔡确说道:“上皇久病,但另外三人却皆康健。若说暴疾疫症,不可能只发生在床帐中,福宁殿中其他人却一点事没有。若说饮食有异,上皇与帐中其余三人并非同饮同食,甚至三人都不一样。但偏偏三人死后与上皇的状况一般无二,唇颊皆红如生,正符合炭毒而亡的症状。上皇的遗蜕不能亵渎,不过其余三人就在偏殿,待会儿诸位可自去查验。”

韩冈的辩白之词,听在苏轼的耳中,却是他背后的太上皇后急着想自证清白。这话说得像是公堂上被告自陈无罪一般。也难怪太上皇后不说话,她现在主管宫中,太上皇有事,世人第一个会想到她。

只听得李肃之又追问道:“烟炭之气有毒,此事尽人皆知。但经水洗去了烟灰后,难道还是有毒?”

韩冈摇头,他现在是绝不会承认他早就清楚所谓烟气之毒的底细,只能含糊回应:“可能只是没有洗脱干净,也可能炭气之中的毒姓与烟灰无关,现在只是推测而已。不过再仔细想想,烟毒要真是与烟灰有关,天下文士也没多少人能活着了。”

文房四宝中的墨块都是用搜集了松枝、柏木之类的烟灰与胶调和之后制成,尝过墨水的人很多,也没听说过有人被毒死。

“而且世间公认的常理往往有错,螟蛉有子,腐草化萤便是如此。各位回去后,可以依照同样的条件进行实验,以证明真伪与否。”

韩冈的推测倒是很有几分道理,李肃之点点头,不再多问。

当然,他不可能就这么释去疑心,整件事实在难以想象。只是问了几句话后,觉得还是见好就收为妙。

现在情况不正常,宰辅们放弃了掌控朝局的权力,而征求重臣们的意见,过去什么时候见到过?韩琦当年逼曹太后撤帘归政,就是同为宰辅的富弼都没通知一声!事关曰后权柄,能多抓一分就是一分,谁会嫌手中权多?所谓事有反常必为妖!

李肃之退下来后目不斜视,但他知道,周围同僚的想法基本上都会跟自己差不多,都是疑虑重重。这么大的事,没有人敢于随便表露立场。而且现在宰辅们虽然是在征求意见,但谁也不敢说他们是不是放线钓鱼,让有异心的鱼儿自己蹦出来。问清楚事情缘由后,他就等着宰辅们自己先给个标准,李肃之可不信王安石、曾布、章惇、韩冈这等极有主见的辅臣,现在会没有一个想法?

学生犯下了大错,作为老师的王安石和韩冈都不能脱罪,可现在他们两人还是好端端的站在宰执班中,态度与其他人相同,分明就是已经有了共同的立场。

