\section{第18章 青云为履难知足(20)}

安南道经略招讨司。

只看经略和招讨两个词就知道,对交趾的战争,目的是讨伐、是吞并,要将之前多少年所受到的侮辱和伤害,用最狠辣的手段彻底报复回去,并不是打到交趾献上降表就了事的。

安南道行营马步军都总管兼经略招讨使,理所当然的是广南西路经略安抚使章惇。而广西转运使韩冈则顺理成章的成为副招讨使。

“……并兼任随军转运使一职。”

“终于还是要攻打交趾。”张载喉间带着嘶声,“眷惟安南,世受王爵。抚纳之厚,实自先朝,函容厥愆,以至今日。”

“这篇檄文就是在昨天家岳亲笔写就。虽然是刚刚出门视事,不过全篇早就打好了草稿。本来就是为了等到正式的招讨司成立,才留到今日发出。”

苏昞看着抄录下来的邸报上一篇短短数百字的檄文:“玉昆,你的岳父是在后悔。”

“任谁都在后悔!”张载吃力的说着,“如果太宗之时就能一举将丁氏平灭,便无今日的李氏之患。”

“嗯。”韩冈点着头,“所以这一次,要一举解决后患,为广西一开太平。”

“说的好!”苏昞、范育同时拍案叫道。关学弟子们没有一个是怯战、畏战之辈,既然出身在关西,早就习惯金鼓争鸣。只恨不能凭双手结束战争,而不会畏惧战争。

“听说兵马副总管是燕达。”吕大临道。

“就是燕逢辰。将从关西诸路调集四十七个马步指挥,总计一万七千人。”出战一切都是军事机密,除了已经暴露出来的信息以外,韩冈不能说得再细了。

吕大临疑惑着:“是不是少了点?”

“会号称二十万的。”韩冈轻笑着,然后正色道:“两万大军用来决战已经绰绰有余!而且还有诸多曾经受过交趾欺压的蛮部都会投奔而来,齐攻升龙府。”

“‘比闻编户,极困诛求,暴征横赋,到即蠲除’?”

“正是。”韩冈点头道,“第一批的五千兵马将会从秦凤、泾原调集,由燕达亲领,将会用最快的速度南下。”实际上是十四个指挥,五千三百人,“同时西向关中的河北军也已经发文调动,等他们到了之后,剩下的兵力就会立刻出发。”

张载点着头,正要说话,突然猛地咳嗽起来。捂着嘴,五脏六腑仿佛都要被咳出来一般。

“先生!”韩冈等几个弟子都立刻从座位上站起,围了上去。

张载咳嗽了一阵,换过气来,然后就立刻斥开他的弟子们:“没事,我没事,你们都坐回去,快坐回去!”只是掌心、唇上还带着鲜红色的血。

韩冈、范育、吕大临几人都犹豫着,但在张载凌厉的眼神下,却不得不后退。彼此望望,都能看到对方眼中难以掩饰的忧色。

张载的病如果是在山清水秀、空气清新的地方静养,不要劳累过度。虽然最终还是治不好,但可以一直养着,不至于快速恶化下去,至少可以多撑上几年。

但张载拒绝了韩冈的提议,他选择了继续在京城中传道授业。尽管在最终确诊之后,甚至不能与学生们坐得太近,但张载还是想要尽可能将自己该做的事给完成。

在没有抗生素的时代,这就是绝症。韩冈依稀记得大概有什么药能治张载的病,也曾想过去发明。可究竟如何去发明,他根本就不知道。唯一能确定的就是在没有运气的情况下,根本不是个人能完成的工作。只有有了足够的权力,给出大略的方案,让人去试验,用上几年十几年,实验上百次千次,才有可能成功。但这样,时间上根本来不及。

喝了几口水,张载将唇上和掌心的血迹擦干净,仿佛什么事都没有发生过,“玉昆,你今天来是不是也有想找几个同门入你的幕中参赞军务?”

韩冈收起心中之忧,点点头,“还得先生准许。”

……………………

这一天,韩千六也抽空去了审官东院,中间一点波折也没有。

本来他历年的考绩都在上等,加上还有韩冈这个儿子,又是王安石的亲家,根本就没有什么人敢跟他过不去。就是新的差遣还没有给定下。

韩冈从张载那里回来时,韩千六早一步回来了。

在父母的房中坐下,韩阿李对儿子道:“娘和你爹商量过了,如果还是在陇西做农官,那就继续做下去。如果是调到其他地方,或是其他职位,就直接回陇西养老。”

韩千六也道:“为父也就会种地,除此以外都没有别的本事。做个恳田的农官倒也罢了,其他的差事可都做不来。”

“三哥你也别失望。”韩阿李指着韩千六道:“做人要有自知之明。叫你爹审案,不知会判出多少糊涂案。让你爹做账,也只会是一笔糊涂账。有多少本事做多少事,其他就勉强不得。”

父母的这个想法,韩冈当然不会反对,想做就做,不想做就休息,在陇西做个老封翁也自在。笑道:“爹这两年也辛苦了,歇一歇也是好事,正好可以多看看家里的庄子。”

见儿子不反对,夫妻两人也笑了。韩千六道:“其实前些日子听说要上京的时候,俺本来就是准备辞官告老,顺便回乡走一遭。”

“回乡走一遭?”韩千六的话有些奇怪,韩冈略略一想,也明白了说得是哪里,“是胶西?”

“是啊。本来准备是让你爹衣锦还乡的。”韩阿李叹了一声,“不回胶西乡里走一遭,谁知道三哥儿是哪个地方的人?”

韩冈点头笑笑,还记得他真实籍贯的的确不多。,

标注在韩冈身上的标签,是关西、关学、秦州、天水、熙河,与潼关以东拉不上关系。这样的情况,谁人能想得到韩冈的祖籍与潼关以西没有半点牵扯,都道他韩冈出身关西。

到现在为止,除了推荐过他的,和一些个会特意翻看他档案的人——韩冈依稀记得就一个蔡延庆——还有王安石这样的亲戚,另外就是审官院的官员,除此之外就没什么人知道他的老家是在靠着海边的京东。甚至当初进士揭榜时,他所登记的籍贯也是入贡时填写的地点秦州。

如果韩冈老老实实的去考举人、考进士、然后再去做官,基本上第一道关口就会暴露了籍贯。为了不在福建、江西这样两百选一、三百选一的路份中挤独木桥,移籍到北方易于中举的路份参加贡举的现象,在这个时代十分常见。

只是冒籍就贡的做法,跟后世的高考移民如出一辙,当然也会受到同样严厉的打击,朝廷不会容忍这等作弊的行为。但韩冈是通过秦凤路中的锁厅试,得到了参考进士的资格。官员们的籍贯问题,并不会干扰他们在哪里参加贡举考试,根本就没有人会去仔细检查官员们的祖籍。

以至于近年来,韩冈碰到过的攀亲之人,都是说自己的祖籍是秦州。

“本来你祖父几十年前从胶西乡中,一直到关西来讨生活,是为人所迫。折了本钱是真的,没法回乡也是真的。不过就是没折本钱,也是回不去了。”韩千六语调沉重。

“是么。想不到还有这回事!”这话还是韩冈第一次听说。

他张口想问问是怎么回事,但看看父母的神色,决定还是不问了。子不言父过,不好让父亲说祖父的不是。

不过从父母此前对此讳莫如深的情况上来看,也许是做了什么不好的事,坏了名声,才不得不离乡远走。而自己做了官之后,韩千六也没有想着与族人联系,炫耀一下儿子的本事,肯定也是这个原因。只是自己现在官越做越大,才又动了心思。以他韩冈现在的地位,祖父就算犯了论死的重罪,也照样能洗得干干净净。何况人都死了多少年,怎么也不可能再翻出来。

儿子没细问,韩千六倒也没在意,继续道:“只是都过去这么多年了,你祖父在秦州过得也还算快活。过世的时候,提都没有提胶西乡中的事。本来的确是想着衣锦还乡,可前两天去你岳父的府上,看了你岳父的那些八竿子打不着的远亲,觉得还是算了。”

“……全凭爹爹的意思好了。”韩冈对于密州胶西的老家倒没有什么在意的,本来就从没有去联系过,有没有那些亲戚,对他来说并不重要,上门来就照顾着,不上门也没必要去管他们。

韩阿李听着儿子的语气,以为他不认同,解释道:“亲朋好友当然都要照顾,但有些人不值得照顾。不见你二姨家的两个,也就是这两年才老实下来。”

这个世道,不照顾亲友,是会被人指脊梁骨。韩冈对母方的亲戚都很照顾,李信、冯从义不说,就算二姨家的两个不成器的儿子,现如今也被教训过后,都在老家中老老实实的做着土财主。

李信曾经拉了一个去军中,但还是吃不了苦,另一个韩冈见都不想见。只是世间的规矩要遵守,所以花钱让他们老实做人。他们还是母族,属于外族,如果又多了一批本家亲戚,更不定会出什么事。韩阿李和韩千六全都是在为儿子着想。

韩冈点头:“孩儿知道的,爹娘的顾虑没有错。”

