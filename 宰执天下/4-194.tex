\section{第31章 九重自是进退地(八)}

在店中用了一个多时辰吃酒说话,到了午后时分,韩冈和章惇方从棉行楼中出来。

回头看看两层高的楼阁,黑瓦白墙、不饰纹彩的酒楼,乍看上去确有着几分来自西北的粗犷,还有悬在门前的两只热气球,看着就比压在棉行楼头上的那一干酒楼要大上一圈,制作则更为精细。

以行会为名的酒楼在京城中并不稀罕,就是原本行会的会所对外营业。棉行楼新开,老牌的马行楼在正店中的名气也不算大,不过有如今七十二家正店排第一的樊楼——京城引领天下风气,说成是天下第一也不为过。那原本是矾行的行会会所,称为矾楼,只是以讹传讹,变成了樊楼。

章惇在等伴当取马回来的当儿,问着韩冈:“这座酒楼以棉行为名,当是玉昆你家的产业吧?”

韩冈没有否认,只是稍稍解释了一下:“这是棉行的会所,中间隔着好几层,而且也只是一部分。”他笑了一下,“过些年,糖行说不定也要在京中开店了。”

章惇没有笑,如今在交州,的确有他家的产业。韩冈有点铁成金之才,了解熙河路变化的章惇,当然也想沾一沾光。但没看到实际出产的白糖之前,他也不可能去幻想自家未来能分到多少多少。而且钱财一物虽是重要,可若是与权力比起来,那就根本不值一提。

“吴冲卿为宰相,希合其意者甚多。他对你成见已深,玉昆你难道就没想过后果?”章惇抬头看着挂在入口处的匾额,意有所指。

“难道吴相公有办法将我置之于死地不成?”韩冈冷笑着,到了他这个地位,除非是谋反之罪,贪赃枉法都已动不了他分毫,“只要不入京师,自然平安。”

“玉昆,你当真是这般想?”章惇回身过来,到现在他都没有放弃对韩冈的劝说,“贼咬一口入木三分,只要咬你一口就能得到宰相的赏识,又会有多少人能忍耐得下来?”

“在熙河路的可不止我一家。”韩冈依然神色平静。

一直以来他做得都很聪明,从不吃独食。在熙河路,想要查他的老底,可是要掂量一下能不能抵挡得了韩冈、王韶还有太后家的反扑。

章惇摇头轻叹,他知道韩冈这一个优点,在交州的时候,他更是亲身体会到了这一点。

不过这并不代表韩冈能就此高枕无忧。自古无罪而遭构陷,最后身死族灭的臣子实在太多太多。在政坛上,将某人治罪的结论,总是要比他的罪证要更早一步出现。如果想将那一个人置于死地,罪名总是很好找的,“欲加之罪,何患无辞?”

韩冈知道自己在吴充眼里总是碍眼的,等到将新党中人一个个清除出去,迟早会轮到他韩冈。但若是吕惠卿上台秉政,他韩玉昆也不一定有好结果。

没有一个强力的势力支撑,韩冈的地位并不稳固,退居江宁的岳父王安石帮不了他,娶了高家女儿的表弟冯从义也帮不了他。在熙河的产业,也保不准有人垂涎。其实韩冈现在能站在章惇身边说着闲话,主要还是靠着的是自己的才干,只是天下从不缺乏人才。

辞别了章惇,韩冈上马回到了家中。接下来的时间,自然是好生的休息,陪着儿子女儿一起玩,

内心的担忧并没有浮出水面。如今新旧两派的交锋是在朝堂之上,无论哪一方都没有余力去扩大打击面,目标只会是朝堂上的对手,而不是正在等待入宫觐见,已经等得不耐烦的韩冈。

章惇今天的这番话应该还是危言耸听的居多,如今大宋在军事上的成就是有明君在上的结果,是赵顼全心全意推行新法结果,若是重启旧法,岂不是否定了他十年来的操劳辛苦?

在韩冈看来,赵顼应当能容许对新法进行小范围的修改,藉此来缓和一下对旧党的关系,让这些年来分裂为二的朝堂能有所恢复,不过赵顼绝不会就此否定此前十年的成果,那是他心血的结晶。

韩冈不认为自己会猜错赵顼的想法,也许身份地位的差距会让人的想法天差地远,但人性是共通的,在新法推行卓有成效的时候,在军事上节节胜利的时候,指望现在的皇帝为实行新法认错,这可能吗?!赵顼之所以要推行新法,还不是因为前些年被辽夏两国的欺辱过甚。若是旧党能给出一个不受二虏欺辱的方略,王安石又怎么可能会被启用?

在韩冈想来,沈括应该是揣摩到了一点赵顼的的想法——有着那样的夫人,察言观色肯定是把好手——只可惜他的小算盘打错了时间、用错了方法、找错了人,也没有认清自己所站立的位置。

韩冈则是一向有自知之明,对自己的原则也是坚持到底。并不会因为任何人改变。

而且韩冈正要去主持的是对建都在开封的大宋有百利而无一害的工程,天子也不可能允许有人干扰到这个关系到都城未来的任务。至少在结束前,赵顼不会容许有人对韩冈横加指责,影响到此项工程的进行。

两天的时间很快就过去了了,沈括最近传来的消息很不妙,据说已经被确定要出外了,只是他还有些事没有交待清楚,天子赵顼暂时还没打算将他放走。

不过韩冈希望他与沈括的交情,不会因为此事而有所损伤,不管怎么说,他并没有盯上三司使这个位置,这应该能让沈括得到些许安慰。

如果沈括出外的话,自己还是该去送他一程,韩冈如此想着。

终于到了入觐的时候了,韩冈并没有打算在赵顼面前表演什么或改变什么,一切还不到时候。光是向赵顼说明自己打算在京西路上怎么做,就要花费不少时间。

清早在文德殿参加过没有天子出现的常朝,韩冈又在宫中等了半天的时间,终于有人来通知韩冈去武英殿。

不是崇政殿,而是武英殿,赵顼到底想做什么,韩冈也已心知肚明。

久违谋面的赵顼比起韩冈上一次见到他时,似乎又憔悴了一点,肤色惨白中泛着没有生气的青色,似乎是一年多来操劳过度的缘故。大概白天和晚上都是太劳累了一点。

等韩冈行过礼,赵顼便笑道,“安南一役,若无韩卿一力主张,难有如此顺遂。收服交趾,此功韩卿当居其半。”待韩冈连声谦逊了一番后,赵顼接着说道,“韩卿为国事操劳经年,难得入京一回,这一个年节可是好生休养一番了。”

倒还真是好理由!

韩冈恭声行礼,为此谢过赵顼的恩典。

赵顼见韩冈恭顺,笑了一笑。先问了韩冈一阵关于交州的大小事务,韩冈对交州内外之事了若指掌,回答也是显得游刃有余,让皇帝甚为满意。

不过这些事都是韩冈或是广西的官员曾经在奏章上提起的,赵顼此时相问,不过是开场的寒暄罢了,又说了一阵,终于转到了正题上,“韩卿曾经在奏章中提起重修襄汉漕渠一事。依韩卿之间,这襄汉漕渠能否修起,又有多少功效?”

“京城百万军民,人口浩繁,食用皆由汴水。仅仅是粮纲,一年便有六百万石之多。臣闻狡兔亦有三窟,而百万之城,唯有一水相系。若有一日,汴水断绝,开封焉能存续?”

“襄汉漕渠不是没开凿过。”赵顼在前,韩冈跟在后面,走到一幅沙盘前,虽然京西度还不算高,但方城山、伏牛山,还有沙河等一众河流,都在上面准确地表示,“太宗皇帝可是两次前任督办,却又两次无功而返。不知韩卿有何良策?”

韩冈听说最近国中的地图和沙盘的制作上了一个台阶,现在看到这些新作的沙盘,发现传言并无错误。

听说正是沈括主持并改进了一些测量方法,绘制飞鸟图——也就是排除地貌所引起的距离误差,从空中取直线确定两地的距离——并由此而制作了沙盘。

有了还算准确的沙盘,韩冈解说起来就方便了许多,指着沙盘上的方城垭口,将自己的计划从头到尾向赵顼说了一遍。

赵顼皱着眉,“用轨道越方城垭口转运……这会不会太麻烦了一点。”

“由轨道居中转运的确是多了一重手脚,不过这仅仅是开始,轨道易于修建,先靠着轨道来转运纲粮。于此同时,襄汉漕渠依然要继续挖掘。不过经臣计算,方城垭口处的河槽要下探五六丈之多方能得见成效,此事非集数载之功不可。”

“京西的户口不多,不知韩卿需要多少民夫?”

京西诸州府,尤其是南路的唐邓数州,一直以来都是户口稀少、甚至有许多荒地没有开辟。如果是在熙河路,这还并不奇怪,但是在京城数百里的范围内,竟然还有多少荒地,这就很让人纳闷了。

不过眼下不是追根究底的时候:“如果是先修建翻山的轨道,三千人足矣。至于开始开凿河渠,只要不催促赶工,民夫也能多能经受得住。”

