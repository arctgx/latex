\section{第11章 城下马鸣谁与守(19)}

昏黄的天地中,盐州城孤伶伶的矗立着。

党项骑兵从城墙底下奔驰,成千上万,竟在绕着城池旋转。霹雳砲投出的石弹、床子弩发出的铁枪,还有神臂弓射出的劲矢、城上投下的灰瓶、油罐,都对他们没有产生一点影响。

在他们的手中,一张张战弓带起一声声弦鸣,不住的向城头上射出长箭,城头上的守军如石块一样像城下坠落。

城头上的守军越来越少,而围在城外的西贼却越来越多,只听得惊天彻地的一声巨响,厚重高耸的城墙就在一瞬间垮塌下来。

铁鹞子们欢呼着,嚎叫着,涌向城中,黑压压的一片将城池覆盖,如同蚁群掩盖了地面。竖在城池中央的‘宋’字大旗,百丈高,数人合围,如同一座高塔,却在刀枪的挥砍,重重的倒了下来。

落到地上的大旗,被战马踏过。旗杆砸在地面上的震动,却变成了铁蹄的鸣响。

一名契丹骑兵践踏过宋军的战旗,跃上了一条长堤。堤坝绵延千里,不见头,不见尾。堤坝内侧的河水浑浊无比,如同泥浆,又仿佛一条黄龙。浪涛奔涌的大河同样看不见头尾,隐于白云之上。

堤坝之外,是一片燃烧着的土地。只能看得见熊熊的火焰,燃烧在大河的北岸。滚滚的河水掩不去生民的哀嚎,在契丹骑兵过来的方向,有着无数人凄惨的哭号。

不知何时,画面又起了变化。

这一次是东京城,高耸的城墙,巍峨的皇宫,铁塔行云,汴水唱晚,当夜幕将临,一盏盏灯就亮了起来,各色的灯山排列在御街两侧,照得天地如同白昼。可就在城外,是无边无际的大军,黑色的铁甲沉沉如阴云,将偌大的东京城团团包围。

转过身,身后是全都是熟悉的面庞。

祖母苍老而睿智的眼神里,满是失望。母亲严厉的表情仿佛在诉说着不满。弟弟翘起的嘴角,蕴含着的全是讥笑。

你不配当一个皇帝。

瘦弱的仁宗皇帝,躺在病榻上的父皇,还有更远处相貌都模糊的几个身穿十二章服的身影,全都抬起手指过来——全都是你的错!

声浪铺天盖地,千万人一起在怒吼,都是你的错!全都是你的错!

一声压抑至极点的惊呼,赵顼从噩梦中惊醒时,浑身已经被汗水湿透。

“官家?”身边的人被惊醒了,支起手肘撑起了身子,令赵顼沉醉的娇躯被透过帐帘的微弱灯火映在另一侧,留下一个动人心魄的剪影。贤妃朱氏的声音清柔,“可是有那里不适?”

“没事。”赵顼摇摇头,一场噩梦让他惊魂未定。不想看到爱妃脸上的忧色,他提声问道:“李舜举,什么时候了?”

就在榻旁不远,一个尖细的声音响了起来:“回官家,才四更初。”顿了一下,那个声音又道,“官家,李都知现下还在盐州,今夜宿直的是奴婢宋用臣。”

……盐州……

赵顼沉默了下去,方才出现在梦魇中的场景又浮现在眼前。过了片刻,他才提声道,“去准备热水,待朕更衣。”

“官家……”朱妃的轻呼中饱含着担忧。

今日轮值宿卫寝宫的宋永臣惊讶的声音也再一次响起:“官家不再多睡一会儿?”

多睡?怎么还能睡得着?身子的确是困倦得没有什么气力,头也疲累得发痛,真的很想好好睡上一觉,但意识却是无比的清醒,宁人痛恨的清醒。

盐州被围,西北战局糜烂,辽人的使节又在京城叫嚣,连着多日都夜不能寐,除非西北大局抵定,否则怎么能安然入寝?

赵顼抬眼看看头上的黄绫帐子,用得时间久了,染在上面的明黄色,已经变得十分黯淡,几近于土黄。他不嗜声色,戒绝一切奢华,吃穿用度尽可能的俭省,甚至还不一定比得上一个奢侈的朝臣——那个蒲宗孟,平日洗漱都有小洗面、大洗面、小濯足、大濯足、小澡浴、大澡浴的区别——如此的排场,赵顼何曾有过?可换回来的是什么?一场接一场的惨败啊!

“官家,”帐外的宋用臣,他音调中带上了点哭腔,“再多睡一会儿吧。这样下去,官家你的身子骨可吃不消……”

“朕知道。”赵顼有些不耐烦的应了一声,但这是宋用臣的忠心,却也不能骂上两句。“盐州那里可有消息?”他坐起身,掀开帘子问着,想避开前面的话题。

宋用臣摇摇头,小声的回道:“没有。”

“种谔和高遵裕呢?!”

宋用臣更为小声:“也没有。”他偷眼看了下赵顼的脸色,见没有什么异状,才又小心翼翼的说道,“官家,若是有军情来,肯定会立刻报与官家知晓的,或许捷报就在这两天。”

“真能有捷报那就好了。”赵顼轻叹了一声,又抬起眼,“河东也没有消息?”

宋用臣还是只能摇头。

为了保证夏州和盐州之间的通路,河东军的骑兵全都给了种谔,现在阻卜骑兵乘势杀入河东境内,光靠步兵根本追之不及。

韩冈早前告急的奏章,虽然没有明说什么,但哪里看不出其中的抱怨。若是稳守夏州、银州,兵力何至于会捉襟见肘到防线上处处漏洞的地步。

宋用臣欠着身站在床榻前,见赵顼没有再睡个回笼觉的打算,也在心里叹了一声,终究还是放弃了劝说。回头示意了一下,一名宫女便端了参汤上来,让赵顼就着漱了漱口。

朱贤妃也起来了,帮着赵顼披好了衣服。宋用臣等内侍、宫女便簇拥着大宋天子往殿后的净房过去。

赵顼身上裹着深黑色的羊皮皮裘,将殿中的寒意拒之于外,“太皇太后那里还有消息吗?”

宋用臣立刻回道:“半个时辰前,庆寿宫那边还说一切安好,请官家勿须忧心。”

“嗯。”赵顼点点头,“那八哥呢?”

宋用臣的回复迟疑了一点:“……这几天都有钱乙在照看着。”

听出了宋用臣话语中的顾忌,赵顼黯然惨笑:“难道这座皇宫,当真是不利皇子?八个儿子啊……就只剩一个六哥了!”

“官家!”宋用臣急声叫道。天子口.含天宪,这种话是不能乱说的。

赵顼一声长叹:“钱乙是当世小儿科的圣手,他都治不了,也就是命数了。”

赵顼的话中已经是认命了,宋用臣听得提心吊胆的,一颗心如同十五个水桶打水,七上八下。太皇太后拖不过今年冬天了,八皇子眼下多半也没多少日子,要是西北再来个噩耗,皇帝还能不能承受得起,想想都让人心忧如焚。他现在宁可西北那边永远都没有消息,也比坏消息传来的要好。

前些日子因为赵顼的发病,在宫中朝中都引发了一场混乱。尽管只是轻微的晕眩,但人心的浮动却是怎么也压不下去的。而且天子的身子骨究竟如何,他这样的近侍再清楚不过,若是有个什么万一,到时候只有一个六哥,那个局面可就怎么收拾?

赵顼泡在热水中,温热的感觉,让整个人稍稍放松了下来。洗澡水热得有些发烫,里面洒了香精,有着一股淡淡的兰花香。

赵顼仰靠在木桶中,感受着水中的热力渐渐渗入体内。身体和精神在清淡的兰花暖香中完全的松弛下来,似乎就要睡去。

没有人上前打扰,内侍和宫女都屏气凝神,一声不发,绝不敢惊扰到赵顼的休息。

不知过了多久,宋用臣的声音响了起来,但不是跟赵顼,而是不知跟谁在说话。隔了一重镂花的木门之外,宋用臣与人交谈的声音很是模糊,赵顼没有去仔细分辨。依然紧闭双眼,休养着精神。

“官家!官家!”宋用臣突然响起的呼声中全是惊喜,木门被推开,他跌跌撞撞奔了进来:“河东胜了,河东胜了,是大捷!”

“……大捷……”泡在水中的赵顼,脑筋还有些迟钝,一时间没有反应过来。

“是大捷,河东路大捷!”宋用臣高声强调着。

“大捷!?”哗啦一声,赵顼在水中坐直了身子,就看到宋用臣举着一份奏表在面前展开。

宋用臣的手也抖着:“韩冈和李宪具表上闻!官军尽歼攻入河东地界的阻卜贼寇。斩首近两千,其余或俘或降,漏网者寥寥无几。”

“好!好!”赵顼除了叫好,甚至都没有其他的话可以说了。这么些天来,总算是有个好消息了。

将奏表交给赵顼,宋用臣悄悄的退出来,留着天子在里面欣喜欲狂。

赵顼抓着河东路的奏表看了一遍又一遍,奏章都已经被水濡.湿,他还舍不得放手。外面宋用臣又不知再跟谁说话,赵顼也没有去在意。

片刻之后,赵顼神清气爽的换了一身衣服出来,眉眼间尽是欢喜。还在想着今天去崇政殿,要好好的商议一下怎么赏赐这份功劳。

可宋用臣脸上的喜色已经消没不见:“官家,环庆路高遵裕上表请罪。其领军至櫜驼口,遇西贼五万坚守其地,一时攻之不克,所部伤亡惨重……”

“攻之不克,伤亡惨重?”赵顼头晕目眩。环庆军这一路的援军玩了,盐州城怎么办?!

