\section{第12章 锋芒早现意已彰(16)}

元佑元年腊月廿九,耶律乙辛登上禅让台的消息终于传到了京垩城。

耶律乙辛的篡逆,大宋朝臣们早就有了心理准备,可当真被确认,还是有几分难以置信。

五代之后的大宋,能亲眼看到一个谋朝篡位的逆臣,真不是一件容易的事。

于此同时,辽国的新年号也一并传到了宋人的耳中。

“天统。”章敦咂着嘴,“大哉乾元,万物资始乃统天……好大的口气。”

曾孝宽笑道:“其实北虏大多不读书,起个正统为号更能让人明白。”

“正统……”章敦笑了起来,“耶律乙辛当真这么不要脸,还真的不好对付了。子容兄?”

“啊,说得是。”苏颂很是勉强的附和了一下。

苏颂心情显得更加沉重一点。

如今大宋朝野为攻辽还是不攻辽各执一端,争执不下,其肇因便是耶律乙辛篡位一事。之前,尚无确切消息,双方的争执一直拖了下来,可到了今日,已经到了不能不决断的时候了。

现在还只是孤证,再过两日,从不同渠道得到证实,如何对待耶律乙辛的辽国便再无拖延的余地。

开战,到底能不能打赢辽人,谁都没有把握,除了吕惠卿在外叫嚣之外,知兵的宰辅,都在担心粮草资源的问题。而支持开战的王安石,话里话外,也是寄希望于通过战争来引发辽国内乱,而后趁机胜之。

要说不开战,士林清议、民间舆论皆曰可战,几位宰辅如何压得住悠悠众口?即便不理会,可吕惠卿的独断独行,如何制止?多少战事,都是先从边境开始,最典型的就是交趾之战。至少大宋这边两任邕州知州先行整军备战,是确凿无疑的。

从宰相的角度,当然不希望在军国重事上被臣僚所挟持,但是要想变动吕惠卿这一宰辅级重臣的职位,就必须要有太后的许可。

之前向太后并没有对此做出决定。曾经通过对入寇辽人的反击,确立了自己在朝堂中的地位,向太后有很强的开战倾向。

韩冈尽管反对,却也没有主动去发挥他对太后的影响力,这让吕惠卿等支持开战的朝臣,得以安坐朝堂内外,继续鼓吹战争。

话说回来,王安石当朝元老,声望无人能比,吕惠卿在朝中也是根基深厚,整个新党在他们身后,急切之间,即便是先帝在位,想把此等重臣给打垩压下去,也没那么容易。

天下万民,如今都视此时为攻辽良机。而韩冈逆民心而动,名声不免受损。

对那些愚民来说,韩冈依然需要顶礼膜拜的药王弟子,但在士林中,那些儒生又会如何想韩冈?他们可不是愚民,考中进士就能当成文曲星了,在士大夫眼中,韩冈在学术上,可算得上是自创一派的宗师,可绝不是什么神仙人物,没必要敬畏如神。

一旦韩冈的名声被破坏了,气学的地位不问可知。

‘玉昆啊玉昆。’苏颂心中念叨着,‘现在不是写书的时候吧。’

……………………

转脸就要是新年了,两府之一的枢密院要处理的事务比平日多了几倍。日常能聊天的时间也没有多少,何况枢密院中的几位枢密的交情,也没好到可以没事就聊天。

耶律乙辛正式登基的消息,虽然让决定开战与否成了近在眼前的急务,但毕竟还没到眼前。

聊了两句,几位枢密使就各自回去处理公事,曾孝宽进厅前,往苏颂的方向看了一眼,那一位枢密院中年岁最长的老臣,正皱着眉头,显得极为忧心。

而另一侧的章敦,却神色恬淡,完全没有困扰的样子。

章敦的反应,印证了曾孝宽一直以来的猜测,暗暗地点了点头。

……………………

章敦平平静静的走回自己的公厅,即便耶律乙辛已经做了皇帝,现在也还不是作出决定的时候。

这段时间以来,章敦对北上攻辽的态度始终暧昧难明,尽管他也反对出兵,可说道具体怎么应对战事时,章敦却退到了后面。万一事情有变,肯定会改弦更张。名垂青史的机会,没有哪位儒臣能够不动心。

当朝廷决定伐辽,到底谁为主帅?

韩冈对于不出兵的坚持,等于是退出了帅位的争夺。少了一个呼声最高的竞争者,其他人的机会立刻就大了许多。

平灭辽国的主帅,吕惠卿想做,章敦当也想做。

但这个主帅之位不是那么容易就能坐上去的,章敦和吕惠卿,谁能让北地军民心服口服?他们可比不上韩冈,亮亮名号就能让军中健儿奋勇争先,也比不上韩冈,一句话可让地方豪强都俯首帖耳,更比不上韩冈,将旗号一张,就可让北虏、西贼都望而生畏。

更何况这一战的准备还没开始,章敦岂是会愿意不明不白的上前线去。

一旦他上来前线,后方交给谁?韩冈吗?还是那些不靠谱的同僚?

说实话,十万级的举国之战,章敦不相信任何人。

河东、河北的资源远不足以支撑大战,就是韩冈这等巧媳妇也变不出钱来。

任何一场对外战争,都要一路、甚至几路的资源来支持。

对西夏时,陕西、河东以五倍、六倍于党项的人口,方才支撑起了连年战事。而为了封锁党项的铁鹞子,更是集中了举国之力,才在陕北崇山峻岭之间修建起了一条千里防线。

辽国的实力十倍于西夏,想要进攻这样的大国,没有中枢居中运作,聚合天下财赋,向前线输送,战事如何能支撑得下去?

吕惠卿好战,是他别有一番心思。都是带过兵的人,怎么可能不知道粮秣的重要性?对军需输送的底细也是门清。真要到两国鏖兵时,怕他也是避之不及。

…………………………

曾孝宽回到公厅中坐下,摆在案头上的茶汤正冒着热气。

案头上的公文,依然厚厚一堆,似乎从早上开始就完全没有变动过。天知道,方才他出去之前,已经批复了不知多少本了。

曾孝宽很是难解吕惠卿的想法。

在过去,吕惠卿的心思还是很容易让人看透。比如他明明对经学有自己的一番见解,却偏偏要去为王安石修《三经新义》;当年王安石第一次罢相,推荐他进入东府后,便急切的推出了手实法——吕惠卿的想法,实在是太明显了。

相形之下,韩冈过去的行事风格才当得起渊深莫测四个字,或者说,总是与常人的思路拧着来,让人猜不到。

可这一回,吕惠卿的想法却同样的让人难以明白。

打仗的事,曾孝宽不觉得自己能胜过有经验又在边地的吕惠卿,可兵要吃粮却是不用多说的。

前方打仗,就要后方尽力供给,若后方不能供给得上,这一仗不用打也知道是输定了。

朝中中层如今充满了新党成员,若有王安石在朝中领垩导,说不定的确能保证前方的粮饷供给。

臣子能够架空皇帝,而几乎控制了在京百司的新党,加上王安石,或许可以保证一干反对出兵的宰辅,无法干扰前线的战局。

只是那其中有韩冈啊!

在军事上,韩冈一个人的破坏力比起其他宰辅加起来都要大。

虽说当年韩冈在罗兀城表现得很正直,可吕惠卿愿意将胜负放在韩冈的节操上?到了他这个位置,又有谁会相信其他同僚的品德?

正猜测着,曾孝宽随意翻动着公文的手,突然定住了。

这是从河东递上来的奏章,并代轨道通车后,军需转运节省下来的时间。

轨道?

河东的轨道,只修到了太原,从太原往京师来,是太行山,所以依照计划是往南修到关中去。韩冈的意见就有一条是要等联通关中和河东的轨道,以及京师通河北,京师通泗州的轨道全线通车之后,再去考虑对辽作战。

曾孝宽站了起来,走到架阁前自行翻找起来。

找了一阵,便翻到一叠抄写下来的副本。这是前段时间,他特意让人送过来的。

翻了一下,里面的纪录便印证了他的记忆。通过轨道,的确有可能解决粮秣军需的运输问题。

如果紧急修筑河北段的轨道,并不是不可能,至少有河东的先例。

河北有着足够的人力,甚至铺设的路线位置,也已经勘探了多年,调来能工巧匠,征发起民夫,直接就可以动工。

沿途虽然要征用民间田宅,可大义在前,即便是阀阅世家,也不敢阻拦片刻须臾。而且有韩冈提议的干线支线之分,恐怕相州韩家,巴不得轨道能早一点修到他家门口。

这是不是推动了吕惠卿开战想法的一个原因?

……………………

王旖从睡梦中醒来的时候,发现前面的灯亮着,韩冈坐在桌前,正写着些什么。

“官人,怎么还不睡?”

“想到下一段怎么写了,就起来记下来,免得明天起来忘掉。”

王旖披了衣服起来,好奇的问道:“不是为了辽国?”

“吕惠卿能有几分是想争打仗?不过是想沾点便宜罢了。”

王旖在韩冈身边坐下,“以北虏之势,岂有便宜可占?”

“在他的位置上,什么都不做就没有机会,只有先做了,才能找到回两府的机会。”韩冈停下笔,冷笑道,“左右败了也不过外任几年,还是跟现在一样,你说他赌还是不赌?”

“可爹爹呢?”王旖不相信自己的父亲是跟吕惠卿一样的想法。

“以吕吉甫的关系,要说服岳父可就简单了。”韩冈笑着,冲妻子招了招手,“来,看看为夫的这几段写得怎么样?”

王旖依言接了刚刚写好的一段字纸,韩冈在旁轻轻敲着桌子,吕惠卿的想法,现在看了一阵,还真的不难猜。
