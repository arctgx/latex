\section{第十章 千秋邈矣变新腔(四)}

黄裳已经到家了。

刚刚进门,妻子便带着家中的婢女迎了上来,一如平日娴静的帮黄裳更衣。

待妻子安静的取走外袍,黄裳问道:“怎么不问考得如何?”

“官人考得如何?”

黄裳妻子的问话漫不经心,精神像是全放在黄裳汗湿的内裳上。

黄裳微微一笑,叹道:“总算是考完了。”

他接下来要做的,便是换件衣服,在家中等待消息,消息来得不会太迟。

就像方才与他一同出皇城来的其他参加阁试的考生一样,在家里等待着自己的命运被决定。

与黄裳一起参加考试的十余位考生,似乎都没什么想与人交流的意思,离开了皇城后,相互间便匆匆打了个招呼,然后分道扬镳。就算有把握通过阁试,之后还有一场御试等着他们,一众考生,既没有心情,也没有时间,与竞争者交流。

与通过了礼部试便已经确定了进士资格不同,仅仅通过阁试,并不代表拿到了制科出身的资格。大多数通过阁试的考生,最后在御试中,依然还会落榜,以第五等的评价成为失败者。想要一个第四等难如登天,第三等开国以来更是只有两人得到。有时间结交对手,还不如回去复习应考。

“博士肯定能够过阁试。”帮着黄裳更衣的一名小女婢叽叽喳喳。

黄裳笑了,问着家养的小女婢:“何以见得?”

“有小韩相公推荐,博士的学问还用说吗?”

“那可说不定,能进阁试的都是有两府的相公推荐。”

小女婢摇头表示不信:“他们哪比得小韩相公?”

“单个比不上。但下面有人啊……”黄裳轻声叹着。

都说上面有人,但下面有人才是最为可畏的。太祖皇帝黄袍加身,何曾靠了上面?

一边想着,黄裳一边听着妻子的吩咐,将湿透了的内裳从身上剥了下来。

“怎么出这么多汗?”黄裳妻子抖了抖刚刚剥下来的内裳,全都是汗水。

小女婢也慌忙端了热饮子来,让黄裳端着,自己则拿着一块干布帮黄裳擦着背后上的汗。

“一时急得满身汗。”黄裳喝了一口热饮子,一股子暖意从喉间传到了全身,整个人都放松了下来,笑道:“也幸好就这么一场,这样的阁试,为夫可不想再有第二次。”

在经过了这一场阁试之后,黄裳像是刚从水里被捞上来一样,内裳的前后襟都已经为汗水湿透,方才在冷风地里一吹,便浑身发冷。只是呼吸到了新鲜的空气,在室内时的憋闷便被一扫而空。

虽然仅仅是完成了阁试,他却像是卸去了心中块垒,浑身上下都轻松了不少。成与不成,就等考官们如何评判,一时之间,黄裳也不想太放在心上。

“是题目太难了吗?”

“的确有难的。”黄裳在妻子面前一向坦然,等着常年在外游学的自己,在家中一直毫无怨言侍奉舅姑,礼敬兄嫂,这样的妻子,让他极为敬重,“不过让为夫为难的,可不是那难题。”

“那是什么为难?”

“王平章和韩参政,考中和黜落,为夫在这两边有些为难。”

“为什么?”黄裳的妻子疑惑的睁大眼睛望着丈夫。

虽然妻子容貌普通,年岁已长,但不经意间的神情,还是让黄裳心头一颤。

轻轻握了一下妻子更在为他套上一件新亵衣的手,“王平章、韩参政这对翁婿是道不同不相为谋。王平章如今势大,崇文院中都是他的人,若是为夫不从其学,就只能饮恨今科。而韩参政的气学,有堂皇大家的气象,正与为夫相合。平常怎么写都无所谓,但今天偏偏遇上考题要两边选一边,为夫可是为难了许久……”

在黄裳去做留到最后的一道题,选择如何回答时,他苦思半日,最后还是决定按照自己的想法去做,而不是屈从于时论。不论当今科举是否以新学为圭臬,使得无论何家学派的贡生都必须对其低头,在制举的阁试上,黄裳并不打算依从新学的见解来论述自己的观点。

除了这一道题之外,还有一道题,黄裳无法确定出处,其他四题中,黄裳有三题还是很有把握的。

黄裳唯一没有确定出处的一题,在短短两个四字句中,竟有一半是助词。想要通过被助词分割、且顺序与原文完全不同的四个字来找出出处,未免太过为难人。黄裳在这一道题中,充分体会到了出题人的恶意,看了两遍之后便聪明的选择了放弃。

剩下的一道题做出来,却没有把握的一题,是要他列出七十余名唐时宰相的名单,明显的又是出题人想要为难考生。黄裳虽然全都写出来的,但还是有些没把握。从出题人的角度来看,多半名单的顺序也会是评判的依据,否则这道题也没太高难度了。

在有一道题没有做出来,一道题又缺乏把握的情况下,黄裳面对论点要在新学和气学之间选择落足点的时候,还是选择了坚持自己的见解。

考中制科,rì后便能够高官显宦,由此回报对自己栽培多年的韩冈。但在新学和气学之间,不畏权势,坚持己见也是一个回报,如果委曲求全,如何面对一力宣讲气学的韩冈?

今天能为了御试的名额,屈从新学,日后也有可能为了前途,而背叛气学。与其这样一步步的发展下去,不如现在就坚定想法。

“原来如此。”黄裳的妻子点着头,手脚麻利的给黄裳套上在家穿的外套,看起来完全没有在意。

“君子行事,言不苟合,行不苟容。与其曲己意,媚上官,还不如长舒胸臆,如此方能还韩参政恩德之万一。”黄裳不怕多话,费尽口舌,也要跟妻子说明。

“官人说得是。”黄裳的妻子帮丈夫整理着襟口,听到后,便屈膝到了声万福,“正该如此。奴家虽然读书少,但也知道知恩图报四个字。既然要在韩参政和王平章中间选一个,那根本没什么要多想的。”

看来妻子是不在意,这让黄裳放下心来。比起外面的风波,宁静的家中,是黄裳最是安心的地方。

而且现在也不一定说肯定过不去,就不知那位眼神阴冷的主考,是否会畏惧新晋参知政事的权势。

……………………

史馆修撰蹇周辅的眼神是有名的阴冷,加上过于瘦削的脸颊,站在房屋的一角,都不用说话,直接就能将小孩子给吓得哭不出声来。

当年他在御史台,几次奉旨审案,都是痛痛快快的就将事情给办下来了。犯人如竹筒倒豆子一般说着口供,这与蹇周辅表现出来的态度和表情不无关系。

他现在脸上依然阴气森森,只是面对他的是日常相伴的同僚,都吓不到人。

蹇周辅指着手中的一份考卷,“这份卷子写得不错,六题写出了五题,就是这一条论不对。”

“五题?谁这么能耐?”

几名考官一起拥过来,仔细的读了起来。

跳过了唯一一道被跳过的地方,翻看每一题的回答。这张卷子的主人其实在试卷中,讲各题的出处全都指明了。

“论似乎是差了点。”一人皱着眉头。

“不仅仅是差,议论的方向错了。”蹇周辅摇着头。

“的确是错了。”另一人附和他道,“王平章肯定不会答应。”

“可要将之判‘粗’,其举主能答应吗?”又有一人在旁问道。

“王平章更近一点。”蹇周辅笑道,笑容一现便收,又恢复其木然、阴森的外在表情。

“这就两条了。其他都没问题吗?”

“没问题了。”第一人点头。

“不,还有一条。”蹇周辅低声道,“这一条人名的顺序错了。”

蹇周辅少年时与范镇、何郯为布衣交,但范镇、何郯显达之后,蹇周辅却累考不中,最后是通过特奏名入官,之后才考中的进士,比起昔年老友,迟了不知多少年。如今他已近六旬,距离重臣的班列依然遥远。但蹇周辅在昭文馆、史馆、集贤院和秘阁这三馆一阁组成的崇文院中,算是老前辈,说话有些分量。而朝廷任事,也往往先选择他这种老成稳重的三馆中人来主持,一来二去,倒是威望日高。

这一回的主考,就是看在蹇周辅的年纪和才识,这是朝廷任用他的主因。

“这就三题了。都判粗的话,此人可就要被刷落了。”

“朝廷开制科,其用意,各位应该明白。不让滥竽充数者充斥朝堂,我等才会奉旨参与知阁试。制科只待当世大贤,宁缺毋滥,但凡可判可不判的错处,全都算成错误,没有必要保全。”

蹇周辅坚定的说着。不过他已经看出这份试卷的主人。

虽然有弥封官,但看了几眼之后,试卷的归属很容易能够确定。这一次的制科总共就十几人,不是数千人参加的进士科,要想一卷卷的对应上,也不会很难。

在这一次的考试总,黄裳的答案中规中矩,在十二人中排在前列,但不论是什么样的作品,即使再完美,只要有心去找,总能找到错处。而这一次,不是简单的错误。

“但这是黄裳的卷子啊。”一名考官叹道。

蹇周辅顿时瞪起了眼:“我等奉旨监考,难道首先考虑的不是完成太后交托的任务?崇文院是天子的储才之地,也是朝中最为清要之处,难道要畏惧一参政?王平章如何会让他的女婿当面大逞凶威?!”

蹇周辅一点儿都不怕。
