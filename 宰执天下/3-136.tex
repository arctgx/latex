\section{第36章 望河异论希(二)}

“……编管恩州【今河北清河】……”

在一次次上堂听审的过程中,郑侠已经变得麻木了,当听到最后的判决,却也只注意到了其中的四个字。

御史台定罪,再交由开封府发落,郑侠的案子在没有惊动任何人的情况下有了结果。

对堂上主审知府孙永的话充耳不闻,郑侠低低的道了一句:“去沙门岛又如何?”

一开始,士林中对他的支持度还是很高的。还没有被收押进御史台的时候,有不少人私下里赞他有胆识,甚至旧识王安国都过来见了他一面。

可等到同天节前暴雨如注之后,郑侠就知道,士林中的风向肯定就要转向。

联系起韩冈在殿上的一番奏对,郑侠坐定了欺君罔上的罪名,让他有口难辩。

现在谁能相信他当初是当真赌了性命?!

这些日子里,在御史台狱中并没有受到折磨,在审讯时也被没有根究什么同党,吃喝居住上更没有被克扣,但郑侠心中仍是十分痛苦。

对于他来说,名声比性命更为重要。

在士林中声名尽丧还好说,自己的一片赤胆忠心换来的却是天子的误解,更是让郑侠心丧若死。与其到河北恩州熬着大赦,还不如到犹如鬼门关的沙门岛【今庙岛群岛】里住着。

依着刑律,配隶重者沙门岛寨,其次岭表,其次三千里至邻州。也就是说,在刑罚中,流放岭南则比流配三千里要重,流配沙门岛比岭南还要重上一层。

至于所谓的编管,则是连官身还保持着,只是被拘束在城中不得出城,往来书信要受检查而已。

孙永在宣判的时候,嘴里就说着,这是皇恩浩荡。只是郑侠却不想要着浩荡皇恩,另可多受点苦。

孤伶伶的无人相送的出了城后,郑侠还是不时的念叨着。

“郑官人,沙门岛还真去不得!”

领头押送郑侠的老公人和气地与郑侠搭着话。他是开封府中的积年老吏,知道轻重,别看郑侠现在声名尽丧,被赶出京城去,但坏名声也是名,只要朝堂上风向一转,或是说得悖逆一点——皇宋易主,说不定他立刻就能翻身。

“怎么?”郑侠没好气的反诘着,“难道沙门岛上还敢行李庆故事?”

沙门岛上只有重刑犯,有些死囚被赦了死罪后,也发配到沙门岛上。由于发配者日多,渐至千人以上,而沙门岛上给囚犯的口粮配额却是只有三百,而且还不能加派,当时管着沙门岛牢城的寨主李庆就将多余的犯人往海里扔。两年间,丢进海里丧命的犯人有七百之多。直到熙宁二年,当此案被登州知州马默揭出来后,顿时震惊朝堂内外,天下闻者无不为之惊骇。

老公人骑着马跟在郑侠身后陪着话:“就算李庆悬了房梁,沙门岛还照样是鬼门关,去得多,回来却没几个。”

“德政不修……”郑侠从牙缝里迸出四个字来,让老公人听着心惊肉跳,不敢再说了。

郑侠的官身还在,出行照样有马骑,有车坐。他从京城北上后,就乘上了驿马,而一同随行的浑家则坐着车子,就这么一路往北去。

一行人出京北上,在封丘县住了一晚,第二天清晨起来出行。正是五月的时候,天上的太阳火辣辣的,到了快中午的时候,路上已经看不到多少行人。

“郑官人,已经是白马县了,到了前面的铺子就歇一歇吧。”

郑侠没理会,在马背上望着路边和天上,时不时能看见一小群、一小群的蝗虫飞来飞去,冷哼着,“蝗虫遍野,现在还吵着要不要修河堤……”

‘修河堤……’

老公人一下看向郑侠,看着他脸上的神情变化。从这口气中,想必这位郑官人即便在台狱之中,也照样听说了这场惊动朝堂的议论,而且还清楚是那位让他入了台狱的韩玉昆所掀起的。

老公人在开封府衙门里面几十年,官场上的勾心斗角早就看多了。郑侠怎么说都是败下阵来的,肚子的怨气不用想也知道寄存了不少。

但眼前看到的,的确如看门的郑官人所说,一眼望过去,地里蹦跶的尽是蝗虫,密密麻麻的连道路上都有。还有不少蝗虫飞了起来,在空中横冲直撞,甚至撞到人马身上。不过在道旁的田地间,一群群的鸡鸭欢快的跑着,但最多的还是人。男女老幼各自举着大扫帚,在田地中用力扑打。

看着白马县民在地里灭蝗,郑侠一行人又向前走了一阵。终于前方出现了一面绘了‘茶’字字样的小角旗,高高的挑起在路边上,比起一边军情递铺挂起的旗子还要起眼。而角旗的落处,就是一座茶棚。几根柱子撑起了棚子,用麦草盖着顶,下面的一幅阴凉之地,让在太阳底下走了半日的人们看着就忍耐不住。

“先歇一歇吧……”郑侠对着押送他几名公人说着。

道边茶棚下,卖茶,也卖解暑的凉汤。一个老汉拿着扇子坐着,面前一摞碗,紫铜大壶放在缸里镇着。郑侠过来时,里面就只有一个行脚商。

郑侠坐下来,卖了几碗茶汤,一碗自己喝,一碗给了马车里的浑家,剩下的给了押送自己的公人们。

喝了一口解暑汤,口味比起东京要差多了,但郑侠也不在乎。就听见行商操着河北口音,跟着卖茶老汉搭着话:“这蝗虫来的不是时候,辛辛苦苦种下的麦子,这一下子都完了。”

“还好,还好。小韩知县拿钱买蝗虫。苗被吃了是可惜,但人拿蝗虫换了米面吃就没事了。别说,现在看看还真扑了不少,县城四门外都在烧着。”卖茶老汉指了指北面白马县城的方向,几道烟柱模模糊糊的往天上散去,“烟都冲天了。”

而就在茶棚不远处,就有几个胥吏摆开了换米的摊子。三斤蝗虫换一斤米或是五文钱。蝗虫极轻,一斤能有近百只,又会飞又会跳,捕捉起来着实不易。但架不住田中的蝗虫多,一扫帚下去就能扑下五六只。

蝗虫易捕捉,使得换米的人为数不少,使得官府派出来的这个换米点都排出一条人龙来,多是老人或是小孩子,背着口袋来换米。一名身穿绸缎的乡绅旁边站着,压着队伍不乱。下面一名书办坐在张小凳上,在一本册子上做着登记。

但也有觉得不该浪费时间来换的,行商喝着茶汤,望着烈日下的队伍:“这排队看着一排就要小一个时辰,排着不累吗?一斤蝗虫晒干了也能剩三两,磨成粉合着面吃,好歹也是荤腥,还能看着点油水。”

“蝗虫鸡鸭吃得欢,喂猪也行。人怎么吃?”坐在茶棚下,卖茶的老汉摇着头,拿着蒲葵扇赶着苍蝇虫子。

“怎么不能吃?”行商浮在脸上的笑容却似乎是在叹气,“河北的树皮都给蝗虫啃光了,现在人都改吃蝗虫了。”

卖茶老汉为这个世道叹了口气,道:“蝗、旱从来都是连着的,要多下雨才能好。就是官家生日前才下了一场透雨,隔了两日,又下了一星半点,月底的时候下了一场稍大的。怎么说这雨水还是少,根本不解渴!”

“京畿好歹有三场雨下来,可怜河北就见了一场雨。而且是到了地面上就没了影,一点也看不出来雨迹。一旱七八个月,都是朝堂里面闹的。”河北行商有了点愤世嫉俗的口气,“听说你们这里的知县是王相公的女婿吧?”

“说得是小韩知县吧?已经升做府界提点了,现在县中事是侯县丞代管。”

“这么快?”行商惊讶道,“真不愧是宰相女婿!”

“小韩知县跟他岳父不一样!别看在县中才做了几个月。老汉几十年看见过的知县里面,他算是第一了。”卖茶老汉为韩冈分辨着,比出了个大拇指,“诸押司在县衙里横行了三十年,去年冬天将米价涨到一百三十五文一斗的也有他一份。后来怎么样,被逼着捐出了两万石来买命!现在县衙中哪个公人还敢伸手要钱?”

“还有那个三十年的案子!”卖茶老汉左手蒲葵扇一挥,“两家人争一片祭田,争了整整三十年。多少任知县都没办法,官司都打到州里过,知州也只知道将案子发回来。可小韩知县一到任,当着全县百姓的面,一转眼就将案子破了!”

“那还真是一名能吏!”河北行商赞叹着。

“谁说不是呢?”卖茶老汉突然又叹起气来,“就是做得太好了,才半年就升了官。要是能在县里做个三年五载那该有多好!”

“好官总是升得快!”河北行商笑道,“相州的韩相公不是三十多岁就做相公了嘛!”

“小韩知县多半也能三十出头就当上相公,到时候,天下百姓就有福了。”卖茶老汉又叹道:“只是这么好的官,还有奸人骂!”

将后面押解郑侠的公人当成了郑侠的随从。看着郑侠坐在一边、默不吭声,卖茶老汉搭上话来:“这位官人从京里来,一看就是有见识,肯定听说了这一件事。”

郑侠不置可否,低头喝着茶。

