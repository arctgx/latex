\section{第14章 霜蹄追风尝随骠(11)}

巍巍黑山已被白雪覆盖。黑山下的黄河两岸,也笼罩在一片雪白之中。冬天的风刮过雪原,但在白色的原野上,却交织了血与火的痕迹。

无数生活在此地的党项部族被驱离了家园。肥沃的河间地上,党项部族的身影越来越稀少。一支支部族,人数有多有少,穿过荒漠一步步向南逃去。

屈野川通向大漠的通道,又是一队党项骑手踏上了这条道路。从他们肮脏的衣袍和疲惫的神色上可以看得出来,之前已经经过了一段不短的行程。

眼前的土地,从平缓的沙砾过渡到起伏的坡地,被积雪覆盖的地面下,已经可以看到河道的轮廓。

队伍中,一名满脸周围的老者看着面前被不知多少马蹄踩得碎乱的雪地,这是前人留下的痕迹。可能是前面有过不少部族从这条路上走过,一路上留下了不少可供识别的残迹。从荒漠中行走了数百里,他们并没有迷路,还算顺利的抵达了屈野川前。

并不算深邃的谷地中,再往前就是一条很单薄的栅栏,栅栏后,是零零散散的七八个守卫。但就是这么单薄的一条防线,这队人马也不敢硬闯,离着栅栏远远地便停步下马。

一名老者排开众人,走到栅栏前,与守卫的头领们,一个年轻的军官打了照面。

年轻军官一扬手,一群宋军士兵出现在四周的山坡上,弩弓上面的箭头闪着夜星般的寒光,遥遥指着包围圈中的逃难者。

被一圈弓弩四面围住,惶惶不安的逃难者立刻反射性的拿出弓刀,与四周的宋人遥遥对峙着。

气氛紧绷的一触即发。

“那是神臂弓!”老者额头上冷汗涔涔,他可是曾经带着族人南下攻宋,吃过神臂弓的不少苦头。距离这么近,神臂弓射出的箭矢能把人射个对穿。连忙回头急声叫道:“不要轻举妄动,把手都放开来,全都放开来!”

老者在这些人之中的身份显然很高,在他的呵斥下,所有人虽是犹犹豫豫,但最终都放下的握着弓刀的手。

不过周围的宋军士兵依然没有松懈,手中的神臂弓依然稳稳的端着。

“将……将军……”老者又回过头来焦急的望着那名领头的年轻军官。

年轻的军官很和气的样子,来回打量着老者身后七八十人的队伍,笑得眯起的双瞳中眼神犀利,“看来吃得苦头不小啊。”

老者操着半生不熟的官话,结结巴巴的道:“牲畜都丢光了,原来可是有一百多帐啊。”

蕃部的帐跟汉地的户是一个意义,一帐便是一个家庭,至少一个壮丁,还有妇孺、老人。一般的情况,一百多帐差不多在五百人上下。眼下一百人都不到,除了两三个老头以外,基本上都是青壮年和小孩子。

年轻军官咂了一下嘴,似笑非笑的:“契丹人下手还真是狠辣。”

老者可不指望宋人能有多少同情心,党项人与汉人的仇怨,不比汉人与契丹人之间的仇怨稍逊。看到这个年轻军官的反应也就知道了,他恭恭敬敬的行礼:“所以小人是诚心归附大宋,做朝廷治下的良民。”

老者张着浑黄一对眼,期待的看着年轻军官。

年轻军官很好说话的样子:“收留你们也没什么。前些天,已经有不少黑山党项部族南下了,都给安排了安居的地方。只要你们愿意归附,自然会让你们住下来。不过有句话要说在前面,肯定是比不上你们之前的黑山河间地——能比得得上黑山河间地的马场,大宋也没有,所以别说朝廷亏待你们这群逃人。”

“能有块地落脚已经是万幸,哪里还敢贪心。”

“那就好。”年轻军官点头,又道,“留个姓名,还有族帐名号,等会儿就派人带你去安置的地方。”

这么宽松的条件,老者一口答应下来,道:“愿从将军号令……可是将军你看,我们被契丹人赶过来,一路上匆匆忙忙的,牲畜都丢光了,人也只剩这些了。能……能不能给我们一点粮草。”

老者说着,做出一副可怜相。

“要粮草,当然可以。”年轻军官笑了起来,“但不可能白白给你们。”

老者一听就放心了,只要愿意提要求,就代表他们还有用,“是不是要出兵?要是打契丹人的话,我这里还有二十个能跑马能射箭的汉子。”

“用不着。”年轻军官一摆手,“区区辽人而已,我们大宋还不放在眼里。只要你们愿出人出牲畜帮着运送粮草,官府会先给你们配发粮食。”

“运送粮草?”老者没想到会有这个答案。

“现在正好缺人手从后方运粮草上前。既然你们愿意听从号令,为官军出人出力,也排得上用场,只要用心做事,朝廷自会为你们提供口粮。至于马匹草料,给你们安排的驻地,有草场,应该够过冬了。”

老者楞了起来,这样的安排,他事先想都没有想过,实在是太宽大了一点。

“怎么,不愿意?”年轻军官又眯起了眼睛,“河东经略小韩相公吩咐了,这件事不会强迫你们,不愿就算了。”

老者苦笑起来:“要是不听号令,可会有口粮?”

“当然不会有!”年轻军官的回答斩钉截铁,“不为朝廷出力,还想有好处,世上哪有这么便宜的好事了?对了,有一件事忘了说。押送粮草时,携带的武器只能是弓刀。长兵或是重器,一律不得携带……你也知道的,你们初来乍到,我们也不得不小心。”

老者一阵点头哈腰:“小人明白。不过官人多虑了,你看我这个小族,哪有什么长枪、骨朵,能有张好弓就很了不得了。”

“正如你方才说的,既然能出二十人从军,出二十个运送粮草,当也不会成问题。你们这些人的口粮先期给付半个月的粮,至于后续的粮草,要你们为官军运送粮草来赚。”年轻军官显然不知对多少人说过这番话,熟极而流下,语速变得飞快,“从府州或麟州到旧丰州,路程都不超过五天,只要能老老实实的做好差事,不要担心家里面会挨饿。你们跟着我的人走,他会带着你们去安置的地方。”

他说完见老者没有反对的意思,就一挥手,便有一名骑兵跃马而来,到了这一队党项人面前,说了几句话,让老者留了姓名和族帐名号,便领着他们继续向下去了。

目送着这一队人马远去,又是一名相貌相似的年轻军官走过来。

听到动静,前一名军官头也不回,道:“龙图的好手段啊,七哥儿你说是不是?”

“按龙图的说法,这叫废物利用。”望着党项人越走越远,折可适眯起双眼,“指望他们跟辽人斗,都是白指望。但他们的马都是好东西。从府州、麟州运一趟粮草,要耗用大量的人力、畜力,也还要给付人畜粮草,蕃人、汉人都一样。既然没什么区别,与其征发民夫,惹得路中骚动,还不如用这些蕃人。”

“废物利用……龙图一点口德都不留啊。”折可大哈哈两声笑:“党项人已经到了有两千多,再有个三五千,差不多也就足够了。若再来个两三万,连修城的人手都够了。”

“修城就不指望了……党项人的手艺啊!”

“……就能运粮也不错了。这些天来南下的黑山党项越来越多,看来辽人是铁了心将黑山下河间地中的党项人全都赶走……还当真是帮了大忙了。”折可大转头问道,“七哥儿,你跟在韩龙图身边,有没有听到什么?”

“听到什么?”折可适不解的问道。

“之前经过一番杀鸡儆猴,辽人已经将黑山下剩下的党项部族全都集合起来,马上就要南下了。龙图就没有事先说到这种可能?”

“还以为说丰州的事呢。”折可适笑了一声,又道,“别把那些蕃部当场傻瓜啊!”

“七哥儿,你前面说的丰州是什么意思?”折可大却没有理会折可适后面的话,追问着前一句的缘由。

“旧丰州已经夺回来了,之前小弟想知道现在的丰州会不会移镇回去,将占下的地归还给府州……还以为哥哥也在想这件事呢。”

折可大震惊道:“你就这么直接问了?!”

折可适摇摇头:“哪敢正面问,旁敲侧击的提了一句。”

“韩龙图怎么说?”折可大连忙追问。

折可适苦笑了一下:“不能指望丰州会移回去。割下去的肉,朝廷多半是不会还回来的。龙图虽没有明说,但听他话中的意思,旧丰州很可能新立一州。”

“果然。”对旧日的失土,折家上下都没有幻想过,折可大也知道朝廷的行事,不会这么儿戏的,“这事指望不了。黑山党项被辽人驱用的事,龙图怎么说?”

“其被契丹人夺取了老家,就算被逼迫,也不会老实听话。只要派人联络一下,给他们一个出路,谁还为一点好处都没有的契丹人卖命?”折可适道:“黄裳一开始就此事问过了龙图。你知道的,龙图一向都有让人将所有可能的情况都列下来,然后一一确定应对方略的习惯。黄裳问龙图遇到这个情况该怎么办,龙图只说了四个字。”

“哪四个字?”

“文攻武卫。”

