\section{第20章 冥冥鬼神有也无(11)}

交趾没有冬天,没有四季,只有雨旱之分。

如今正是旱季,天蓝的通透,只有几朵薄云点缀其上。

身下的肩舆随着轿夫的步子,一起一伏的上下轻摆。李洪真抬头望着天空,轻声一叹,这样的天气,还要持续数月之久。最是适宜出行的气候,自然,也就适宜用兵。去年李常杰和宗亶就是在此时领军北上,而现如今,北方的敌人南下,也是选在了这个时候。

北方边境的防线,在宋国的奸计下,半年来已经被戳得千疮百孔,甚至可以说是不复存在。被左右江三十六峒的蛮军扫荡过,群山攒聚的地区,现在找不到稍大一点的村落,没有周边乡民的支撑,北方的任何一座城寨都不可能再有抵挡宋军的能力。从边境一直到富良江,都无险可守。只有一条并不算十分宽阔的富良江,如何防得住从北面涌来的复仇大军?

李洪真这一年来多少次叹息,李常杰将宋国当成烂泥一般易于揉捏,这件事真的是做得大错特错,太祖太宗留下来的大越,就在奸臣、淫后的败坏下,眼看着便要毁于一旦了。

“四太子!四太子。”

宫门已经在望,李洪真乘着肩舆正往宫门去,后面突然传来了唤声。他回头一看,叫他的是兵部侍郎黎文盛。

黎文盛最近与李洪真走得甚近,甚至近于阿谀。李洪真也需要更多的在朝堂上派得上用场的棋子,并不介意将原本属于李常杰一系的黎文盛,收归自己的门下。

兵部侍郎隔着老远就下了自己的肩舆,匆匆来到李洪真身侧,扬起头压低声音问道:“不知四太子听说了没有,章惇将献降表的使臣都赶回来了!”

这么大的消息,李洪真自然听说了,心知黎文盛也不过是打算以此起头而已。他嗤笑一声:“光是一张降表,奉还掳来的汉人,宋国皇帝如何能答应?”

“所以章惇还说要罪魁自缚去东京城受审。”黎文盛仰着脖子,随着肩舆往前走的样子有几分可笑,像是被捏着脖子拖着走的鸭子,不过黎文盛看不到自己的模样,只能看到李洪真嘴角边淡淡的笑意,“这当也是宋国皇帝和宰相的想法。”

“罪魁?”李洪真笑了,李常杰怎么肯去东京城?不过事情再往下发展,说不定就由不得他了。到时候,所有的罪魁也都能送进东京城去。

“章惇这一句话出来,太后和李太尉可就不能降了。”黎文盛饱含深意的冲着李洪真微微一笑,“四太子为大越的中流砥柱,可是要为君分忧啊!”

在交趾国中,只要是皇子,除了朝会之上,平常时候皆称为太子。而高品的妃子,则多称皇后。

李洪真排行第四,是李日尊的亲弟弟,故而被称作四太子或是洪真太子。他对李乾德的即位,一百个不服气。李日尊是三子,而李洪真则是四子,如果李日尊无子,论理就是该由他即位。但偏偏李日尊到了中年之后,一下就得了两个儿子。

这件事让人很是奇怪。李日尊之前一直无子,是到了四十多岁,纳了如今的太后倚兰之后,才连得两子。而别的嫔妃,还是连个屁都没放出来。这其中的缘由,要么就是外面纷纷传说的倚兰有神佛襄助,要么就是其中另有鬼祟。

李洪真虽是李乾德的王叔,是宗室的身份,但他手上照样有着一部分兵马,这是他自保的底气,也是他窥视大宝的本钱。

黎文盛的态度很是明白,甚至太过直率,而李洪真则是满意的冲他点了点头,仰天一声长叹,“本想做个悠闲王公,只是天不从人愿。”

说话间,李洪真的肩舆已经与黎文盛一起入了宫城之中。

紫宸殿前,交趾国正等着朝会开始的文武百官,并没有大宋朝会时的森然戒律,几人一群的正在议论着刚刚传来的噩耗。

“这一下就只能打了。”

“大越人丁数以十万,人人皆可上阵,何须畏惧区区数万宋军!”

“不要小觑了宋人。得赐旌节的帅臣是章惇,辅佐他的是韩冈,而实际领军两名大将则是燕达、李信,这些文官武官,哪一个不是打惯了仗的?这一战可不能硬拼!”

“别忘了北人不服南方水土,到了我大越国中,就该知道什么是瘴疠瘟疫了。只要能守住升龙府,不用半年,宋人就得退军了。”

“得先拖到明年二月才行。”

“正月一到,雨水就开始多了,只要抵挡两个月便足矣。”

“听说宋人南下军队才到了五六千,等全数到齐,肯定要到明年了。”

“雨水一起,瘴气便会跟着起来,到时候,宋人至少病死一半。”

李洪真抿起嘴。一众大臣竟然天真的将希望寄托在疾病上,也不知道他们想过没有,万一宋人不生病怎么办?

黎文盛在李洪真耳边冷笑着,“指望宋人会有因为疾疫,不知道大败了李常杰的韩冈是什么人吗?药师王佛座前弟子转世!荆南军到了广西一年了,派了多少密探过去,也没听说他们有多少人病死。听说在在邕州,有几十名中国给皇帝太后治病的医官,日夜给士卒们传授医术,闲暇时还给当地百姓问诊施药。”

这一桩桩事都不是秘密,但国中百官却一个个视而不见,听而不闻。

‘该清醒了!’李洪真抿着嘴。

一对眼睛望着立于一侧的李常杰,想必他不至于会跟其他人一样,听说过韩冈的传闻之后,还能将胜利寄托在交趾的气候之上。转头又看了看其他几个以明智著称的大臣,都是阴着脸,并不与他人交谈。视线转到另外一边,李洪真的一名党羽暗暗指着李常杰,向他使了个眼色过来,李洪真点点头,心领神会。

几声净鞭响起,交趾国的文武百官忙排起队,走进紫宸殿中。

李常杰并未站在班列之首,在原顾命大臣、太师李道成暴卒之后,他为邕州之败上表请罪,由辅国太尉降为金吾太尉,官阶也贬斥三级,并罚俸一年。不过,李常杰的请罪也就是做做样子,谁也不敢当真以为他在军中已经是过气了的人物,驻屯升龙府内外的三万天子兵,大半都对李常杰唯命是从,他的一句话,比起现在坐在御榻上的大越皇帝管用一百倍。

当然,李常杰说话的份量,比起坐在年幼的李乾德身后、用一道薄纱遮起面容的倚兰太后,倒也不至于重上一百倍。

倚兰太后隔着幕帘,问着群臣:“宋臣章惇行事不可理喻,将我意欲通好的使臣赶回。如今其聚兵邕州,谋图南下攻我,不知诸位卿家,有何良策却之?”

这应是开战前例行的询问,太后的一番话,也不是她随口能说得出来的。交趾众臣静静的站着,都在等待李常杰出来说话。

李常杰身形欲动,但李洪真当先步出班列,“臣有言欲禀于太后和陛下。”

帘幕后的倚兰太后明显愣了一下:“……太傅请说。”

李洪真躬身一礼:“宋人在邕州秣兵历马,大越已是危急存亡之时,臣忝为宗室,太祖太宗之子孙,不敢侧身事外。愿领一军北上,抵挡宋军入寇。”

当朝太傅,天子的亲叔竟欲自请出征,这话一说出来,殿上顿时一片骚动,李常杰木然的脸色也猛然间有了些变化。

李洪真暗暗冷笑:‘别以为我不知道,你不是准备推荐我去北方吗?现在我主动去。’

李常杰要防着李洪真为首的宗室,卖了他和倚兰太后给宋人;而李洪真难道会不防着李常杰对自己下手。打听到消息,心知难以避过,就立刻主动出手,以进为退,要让李常杰犹豫难断,怀疑他另怀鬼胎。

李常杰似乎是犹豫不定,倚兰太后也不能决定到底是点头还是摇头,万一李洪真投降宋人,事情就麻烦了。

在掌控朝堂的一派首脑陷入沉默的时候,突然间站出来是李常杰的党羽礼部尚书陈仲和,他的出现打断了李洪真咄咄逼人的气势,“太后,臣有一策,不动刀兵,也能让宋人自取其败。”

只听声音,就知道倚兰是精神一振:“不知陈卿有何良策?”

“韩冈威重广西,而章惇则声名不显。如今宋国安南经略司以章惇为主、韩冈为副,这是轻重倒置。当以计间之,使两人不和,自相纷争。”陈仲和为李常杰争取着时间,“章惇既为主帅,权柄当在韩冈之上,而李信听闻又是韩冈的表兄。既然与韩冈交恶,章惇也不会再任用李信。只凭毫无经验的燕达,如何能攻入我大越?”

“此计大妙,当即刻命细作往邕州去施行。”倚兰说着,那眼睛瞥着李常杰。

李常杰终于开口:“此计当行,但战事依然难免。宋人攻我,自当全力相抗。如今兵马粮秣都以备齐,只待发兵。既然太傅慨然请战,臣请太后即刻下旨,点集兵马,以太傅为帅往门州驻守。”

说话的口气平平静静,看不出到底是不是真心同意。

李常杰要让他去门州,李洪真对此并不在意。不论去留,他有路走,他只是想抢着先机而已。略略抬头,望着殿上的李乾德和他背后的倚兰,“请太后、陛下下旨。”

