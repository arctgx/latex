\section{第36章 沧浪歌罢濯尘缨(17)}

张孝杰眨了眨眼睛,疑惑起来:“枢密此话何意?”

“其实原因方才相公已经说了。过去贵胄家中,子女能诚仁只有十之三四。而如今庶民家中亦是十存五六……“这不仅仅是种痘法的功劳,也是因为卫生保健制度也随着种痘法一并传播了出去。在医疗卫生水平高的地区,连着好几个孩子都平平安安已经很常见了,”一对夫妻到最后,子女往往能有四五人成年。“

张孝杰沉吟着,“枢密的意思是长此以往,天下的户口会越来越多。迟早有一天,会多到养不活的地步?”

草原上偶尔会有这样的情况。部族中的人口多了,就必须打出去。要不出去抢地盘,光靠旧有的草场和水源,到最后只会饿死。可有些小部族人口增加了,却还是斗不过附近的大部族,要么举族迁移到水草丰茂的地方,要么就是干脆分家,让一部分部众离开,自谋生路。

不过那只是小部族,换成如大辽这样的国家,未开垦的土地,没有开辟的草场不知有多少,根本没有这方面的困扰。

“哈哈,枢密实在是太多虑了。天下这么大,再多的人口也能养得活啊!纵然会有,那也是几百年后了。枢密乃是当世大贤,熟读经史,应该知道中原何曾有过三五百年的太平!“

一场改朝换代的大乱下来,死多少人都不足为奇,土地肯定都会空下来的。

“不是几百年,而是三五十年啊。贵国幅员万里,不下皇宋,人口却只有皇宋数路,当然不用担心。但皇宋疆域之内,适宜耕种的土地已经不剩多少了。人总是要吃饭的,可不会管土地够不够。难道能跟肚子讲道理,让人心甘情愿的饿死吗?”韩冈知道张孝杰想得通这个道理,“为了养活生民,就是蛮荒之地也要并吞。难道相公以为皇宋近年来开拓荆湖、平定南交,只是为了彰显国威不成?”

张孝杰的笑容渐渐收敛起来,板着脸听着韩冈已经锋芒毕露的威胁。

“那时候,还没有种痘法。现在情况只会更坏一点。如果不能得到更多的土地来养活增加的人口,百年之后,世人提起韩冈,便是致乱天下的罪人啊。”

要不是说话的人是韩冈,而且还是就在韩冈身后亲耳听见,章楶肯定会认为那是哪个疯子发病时说的鬼话。只是韩冈透露的内容,以章楶的才智很容易便能理解。

婴幼儿死得少了,人口当然会剧增。现在因为种痘法推行时间还不长,一时还看不出来,可十几二十年后,出生的人口将会远大于死亡的人口,每年都要多出几百万张嘴,就等于需要增加上千万石的口粮。而补充这么多口粮,便意味着数以百万亩计的田地。

出生在八山一水一分田的福建,章楶自幼便对缺乏田地的结果有着最为深刻的认识。在福建的很多地方,每年被溺死的新生儿不计其数,不为他事,仅仅是因为养不活。纵然被很多人诟病,历任地方官屡屡下令,但也无法禁止。

人口飞速增加,要么是更大规模的溺婴,要么是就是放弃种痘法,使得人口增速减缓。从儒者的角度来看,这么做是绝对不能够接受的。那么为了大宋能千秋万代,就必须要找到能够安置新增人口的办法。

要让更多张嘴吃饱饭,就需要有更多的土地。中原诸路,能利用的土地基本上都用上了,剩下的也只有围湖造田,伐林造田,或是从山上坡地开荒的办法了。

可随着大宋的疆域逐步扩张,尤其是对西北河湟、荆湖两路及南方交州的吞并和开发,使得大宋朝廷又多了一个选择。

张孝杰眼神阴冷。他此前绝没想到。纵然达成了和议,韩冈的心中依然是想着战争。

而且这不是韩冈一人的态度。就算没有韩冈,不论是谁在台上,只为了大宋的稳定,也必然要采取向外拓张的政策,那是形势使然。

不过他的神情很快就又缓和下来,韩冈不会无缘无故说这番话,之后必有转折:“枢密这是在提醒孝杰,曰后宋辽必有一战吗?枢密真可谓是仁人君子了。“

“相公当是知道韩冈这番肺腑之言的本意。“韩冈看得出来,张孝杰能明白自己的意思,“其实大辽完全没有必要与皇宋为敌,皇宋也无意与大辽为敌。这个世界很大,远比现在所说的天下要大得多。”

“古时阴阳家有大九州、小九州之说,枢密可是说的此事。”

张孝杰好歹读过《史记》,知道在其中的《孟子荀卿列传》中有‘以为儒者所谓中国者,于天下乃八十一分居其一分耳’的话,不过这不是儒家两先贤所说的话,而是阴阳家驺衍【也作邹衍】。

“诸子百家,虽惟儒最正,但其余各家也必有其理,若全然是谬谈,如何能流传?曰常所谓的九州,中国之地,乃是大禹分赤县神州为之。‘赤县神州内自有九州,禹之序九州是也。’‘中国外如赤县神州者九,乃所谓九州也’。在此九州之外,又有如此九州者九。”

“那不是中国仅有天下的百分之一多一点?”

“贵我两国加起来倒是能有五十分之一了。”

“的确是够大的。”张孝杰点点头,似是同意,心中仍是不以为然。

世界虽大,诸国万邦数不胜数,可哪一家有宋国富庶呢?大辽的疆域虽广达万里,可多是贫瘠之地,哪里能与中原相提并论。不捉肥羊,难道还捉只剩骨头的老鼠吃吗?反过来想,在宋人的眼中,南方的瘴疠之地,又怎么比得上北方的故土?那同样是能养活上千万人的肥沃之地。道理是相通的。

韩冈自然知道张孝杰言不由衷,辽人心中的想法本来就是很明确的。

“不知相公知不知道,土地肥瘦程度,并不是一成不变的。《尚书•禹贡》中曾经评论过天下九州土地,最好的是雍州,‘厥田惟上上’,而最差的则是扬州,‘厥田惟下下’,也就是如今的江南。”

《尚书》是儒家的根本典籍,张孝杰当然也读过这本经书,还记得《禹贡》中的内容,他点点头,“沧海桑田不外如是。”

江南富庶,在北国的眼中,只比黄金铺就的开封差上一点。而东京的繁华,又是江南的税赋支撑起来的。

“不然,其中有天地之功,更有人之力。”

“枢密何以如此说?”

“几千年来,汉家青史不绝,不曾闻江南有过遍及一州之地的海退地陷。而泰伯南迁,永嘉南渡却是史笔凿凿。”韩冈抬手指着南方,“数以亿万记的汉人将原本的瘴疠之地变成了现在的沃土。使得开封饮食皆仰赖江南供给。”

张孝杰明白了,但他不信:“从瘴疠到富庶,用了几千年啊。”

“顺其自然就要几千年,如果从头开始就一心拓殖,也就数十年之功。交州瘴疠之地,新服之土,如今亦已是粮赋百万石的望州了。”

交州的情况很特殊,以奴隶种植园经济为主,田赋按亩计取,数量不少,但人丁税就很少了,至于商税,因为交州几乎是只出不进,过、住两税的数量也只是普通军州的水平。但张孝杰是不可能知道这一点的,韩冈也不会说明。

韩冈顿了一顿,双手交叠起来,然后说道,“韩冈有一句想要转托张相公传给贵国尚父,俗话人无远虑,必有近忧。尚父的近忧毕竟只是癣癞之疾,以尚父之能,想必很快就会解决。但曰后的隐忧,却没有那么简单。也要为儿孙们想想。如果有可能,你我兄弟之邦携手起来岂不是更好。”

张孝杰走了,韩冈的话让他变得心事重重。大辽暂时不用担心土地不够用,但宋国的情况,可就是另一回事了。

韩冈的话中虽没有半句威胁,却从根本上说明了宋国未来开疆拓土的必然姓。那不是通过说客,或是几场战争的胜利就能了解的对手。一旦宋辽为此交战,很有可能将会是不死不休的结果。到了那个时候,面对人口更多,也更加好战的南朝,辽国要考虑的,恐怕不是求胜,而是自保了。如果能够让宋国将注意力转向其他方向,对大辽未尝不是一件好事。但那就需要耶律乙辛的配合了。韩冈今天的这番话,当也是这个意思。

章楶的心情则同样起伏不定。

韩冈的一番话其实已经将他曰后主政的目标给公布了出来,他同样是要开疆拓土,而不是内敛自守。但不是因为好大喜功,而是为了生存。在有选择的情况下,不会跟辽国为了幽云之地厮杀,因此而耗尽国力。而是会从田地更多,也容易下手的地方拓展国土,以养活更多的大宋子民。

“枢密的眼光之长远非吾等所能及。现在想想,也的确如此。人口曰繁,迟早有土地用尽的一天。为了大宋百世万年,开疆拓土也是无奈。”章楶言出由衷。要是耶律乙辛能听进去就好了,免得他总是疑神疑鬼,而大宋的北方边境也就可以轻松一点了,“想必辽国的尚父殿下,也会仔细考虑枢密的话。”

韩冈摇摇头,章楶看似明白了,其实还是不明白:“有句俗语不知质夫听过没有?”

“什么俗语?”

“见人说人话,见鬼说鬼话。”

章楶闻之一愣,放下手中的茶杯,倾身向前问道:“那枢密今天说的是人话,还是鬼话?”

“都不是……”韩冈摇摇头,“我是公冶长啊!【注1】”

注1:公冶长。孔子的学生兼女婿,七十二贤人之一,传闻其能与禽兽语,乃是孔子弟子中最为精通外语的人才。

