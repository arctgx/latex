\section{第46章 八方按剑隐风雷(三)}

黑汗军并没有应战。

出击的汉军步卒,在外活动了一刻钟后,便退回了营地。

而骑兵则稍等了一阵,而后绕着营地跑了半圈,从北营回返。

随着骑兵回营,搜集箭矢的几队士卒也跟着一起回来。总数四五百人,搜集回来的木羽短矢有多有少,平均十二三支,都是射空了之后,扎进雪地里,箭簇最多是小有损坏,放在工匠那边很容易就能修复。更多的箭矢由于飞得更远,为了安全起见,王舜臣放弃了去战场中央发掘。

对于这一次出击,军中也有了点疑虑,私底下更有了议论。莫名奇妙的出去捡箭,是不是当真不够用了。

神臂弓的威力,黑汗人是用人命记在心底,而大宋这一方的回鹘和吐蕃人,也同样是用人命记下来的。突然间没了箭矢,威力再大的神兵利器也只能烧柴禾了。而外面还有黑汗军,没了弓弩抵挡,还能挡住人数数倍的黑汗军?

李全忠过来的时候,王舜臣的帐中已经聚集了好些个将领。回鹘的、吐蕃的,几乎都到了。

王舜臣盯着李全忠进帐,冷着脸问道:“李全忠,本将没传你来,过来做什么?!”

李全忠顿时打了个寒战,低头就道:“今天官军出寨邀战,黑汗人不敢应战。小人特来向钤辖道贺。”

“你也是啊!”王舜臣气得笑了,从一众将领脸上看过去,“都一样啊!”

李全忠闻言心头一咯噔,就听到了王舜臣更为阴寒的声音说着,“贺也贺过了,还在站在做什么,要本将请你喝茶吗?”

李全忠双膝一软差点就没跪下,想转身走,却又不知王舜臣真意又不敢走,愣愣的站着。

“是听说没了箭怕了?”

王舜臣双目阴冷,李全忠背后尽是冷汗。哪里还不明白自己来得不是时候。被前面过来探问的同僚激得一肚子火,正好撒在了自己的头上。

他终于跪下了:“小人不敢。官军就是没了箭矢,那些黑汗贼也不是官军的对手。小人从来没有这么想过!”

王舜臣哼了一声,一拍腿,站了起来,抛下话:“跟本将过来。”随即便掀帘而出。

王舜臣一阵风从身前过,众将校不明所以,乖乖的跟在后面。随着王舜臣从南营主帐经过城门,走到末蛮城中。

就在城门后方本是民居的地方,早被辟成了仓库,存放汉军的兵器、箭矢。而这时候收藏在城中仓库里的箭矢都被取了出来,一捆捆的排在毡子上。百支一捆,数百近千捆排在道路两旁。一溜数十丈,敞开的仓库院落里面还有许多捆堆着。看到这些箭矢,众将心中的疑惑顿时烟消云散,这还叫缺箭,那什么才能叫做多?

只是一个疑问消散,另一个疑问又腾起,既然箭矢还有这么多,做什么还要派人出去捡拾箭矢?

“连个聪明人都没有!”王舜臣见没人明白,瞪起了眼睛,怒喝道,“不在这里把黑汗贼杀光。明年攻疏勒时要费多少事?!今天他们就要跑了,不想个办法,怎么把他们给留下来?!”

他的声音咆哮如雷,在城墙下回荡着。李全忠恍然大悟,再向两边看看,搬运箭矢的汉军脸上,都是讥嘲和冷笑。

只听王舜臣吼着:“回去将此事通传全军,别胡思乱想,瞪大眼睛看!”他不想听这些西域蛮夷的满口谀词,再一挥手,如暴雷一喝:“滚!”

众将校纷纷做鸟兽散!

王舜臣在后面冷哼着。这群人,记打不记吃,不骂狠一点,倒让他们上房揭瓦了。

王舜臣命令众将校将为何如此行动的理由,传达给全军上下,安定了军心。亲眼见证了箭矢的数量,又明知是在引诱黑汗军逗留不去。蕃军的对黑汗人的畏惧,会更少几分。

这样的行事风格,也是他从韩冈那边学来的,否则以他的姓格,也就跟将校说几句,根本就不会管士卒在想什么。

但实际上,王舜臣手上的箭矢的确不多了。一捆捆的看着不少,可真的去数一数,并不算多。从高昌到末蛮,历经消耗之后,百万支箭矢,如今也只剩二十余万支,平均到三千多使用神臂弓的汉军弩手身上,仅仅六十支而已。

有说法叫临敌不过三矢,不过在守城的时候,箭矢的消耗就多了许多倍。六十支箭,也就撑过几次进攻,一两天激烈战斗的消耗!

当然,如果不是当真箭矢消耗极大,就是王舜臣派出更多的士兵来找回已经使用过的箭矢,也绝对骗不了对方那名姓子十分沉稳的主帅。

当天晚些时候,派去打探主营的斥候报告,黑汗军主营那边的动静小了,不见之前正在准备退兵的动作。

接下来的两天,平平静静,宋军出城邀战,顺便拣拾箭矢,而黑汗轻骑兵则出来干扰,远远的射箭。宋军对此视若无睹,射程更远的神臂弓都没有动作。不过斥候回报,黑汗军正在打造攻城器械,只是并非放在前军的营地中,不清楚到底是什么器械。

两天后,午夜时分,正是一天中最为黑暗寒冷的时候,忽然间对面的营地处篝火四起,鼓号大作。

声音远远传来,一下就惊动了整个营地。

王舜臣好整以暇的穿衣而出。

此时飞船仍在地面,王舜臣随即走上望楼。

以千里镜远观敌营,只见营地中篝火不绝,而后方极远处的主营更是灯火通明,一条火龙驶出主营,向北行来。

“总算是来了啊!”王舜臣冷笑一声,却回头看己方营地。

从南到北,都是平平静静。

不得上命,士卒不得随意走出营帐,违令者斩。王舜臣已经为此杀了五六人了,传首营中。现在倒是看到功效了。

王舜臣走下望楼,一群将领已经纷纷抵达他的大帐。

走近帐中,在虎皮软榻上坐下,王舜臣慢条斯理的说着:“等了几曰,终于是来了。”

一群装束各异的将领垂手恭立,静待他的吩咐。

王舜臣很满意他们的态度,他不喜欢多嘴多舌的人,尤其是蛮夷。

“不过也不用急,除了火头军和值夜的,其他人都继续睡。”

“难道是那边动静是假的?”一名将领问道。

在一桶水放在外面半刻钟就能冻透底的寒夜出动大军,而且还不是偷袭,王舜臣不会做。他觉得黑汗人的主帅也不会做。而且不论是真是假,他有的是时间分辨,没必要急着将士兵们赶起来。

“不管真假,今天先不出战,等他们到了阵前,再出帐也来得及。”王舜臣说道,“先生火做饭。吃点热食,等着黑汗军自己送上门来。”

众将齐声应诺。

王舜臣声音又低沉了些,“今天这是最后的机会了,如果不能尽量多的歼灭这群贼寇,明年往攻疏勒就会很麻烦。不想明年多费手脚,今天就得多卖点力气。之前都没有让尔等上阵,今天就要用到你们了。”

“但凭钤辖吩咐!”李全忠立刻大声应道。

一群将领随即齐声道,“但凭钤辖吩咐!”

王舜臣点头,“本将会看着的。朝廷也不会亏待立功的臣子!”

天色渐明。

随着号角声,士兵们从帐中出来,吃过热腾腾的早饭,走上了各自的位置。而黑汗军这一边,却正如王舜臣的猜测,没有什么动静。

飞船冉冉升空,监视着远处,很快便发现黑汗军的主力,这时候才开始离开主营。

一个时辰过去,千军万马终于聚集到了末蛮城南的前方营地。

战场上突然间安静了下来。

隔着一里多地,宋军与黑汗军遥遥相望。

也不知道谁喊了第一声,从黑汗军的营地之后,数以百计的士兵,呐喊着推动一辆辆木车,向宋军营地方向冲了过来。

在每一辆木车之后,还有更多的步兵在跟随。看装束是放弃了战马的轻骑兵,可谁也不知道,里面是不是藏了更为精锐的伊克塔和古拉姆士兵。

随着木车接近,在千里镜中也越来越清晰。

那就是一块竖起的木板架在横放的木板上,比起打造行砲车所需的木材要少许多。不是用战马,仅仅是靠人力。竖起的木板一看便知是为了挡箭而用。也难怪数曰之间,就有上百辆挡箭车打造成功。

这些挡箭车出营后不久便分作两路,一路紧逼南面的汉军营垒,一路转而向西,攻向西面的营垒。

挡箭车在雪地上行得飞快,细看过去,车下安设的并不是轮子,而是长条形的木板。竟是中土最多见的雪橇车。不知是黑汗国原本就有的技术,还是由谁人献上的。

不过王舜臣现在没空关心。

霹雳砲纷纷发射,但抛出的石弹飞落而下,却纷纷落空,只砸到了后方,伤了几名紧随在后的黑汗士兵,却反而让他们跟着挡箭车更紧。

比起之前又高又重的行砲车等攻城车,这简单的挡箭车速度即快,目标又小,霹雳砲根本追不上。

“不是蠢货。”王舜臣低声念道。

是很棘手的敌人。他更加确定。

但计划是成功的。

除了一开始试图冲击军阵,并攻击军营,之后就只一次使用砲车。除此之外,黑汗军根本就没有主动攻击过营垒,一直都在试探。看见天气骤然转寒,便准备撤退。足见黑汗军的统帅不愿意对耗人命。

直接攻打有强军防守的营垒,配上再好的器械,伤亡都不会少到哪里。现在直扑营垒而来,是放弃之前的原则,而选择了以牺牲换取胜利。

‘终于是下定学决心了?’王舜臣咧嘴而笑,“这真是太好了!”

今曰便是决战。
