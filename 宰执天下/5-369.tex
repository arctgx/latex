\section{第34章 为慕升平拟休兵(一)}

向皇后自是没听过宗泽这个名字,只是对两家报社同时启用同一个人来评述河东局势感觉有些奇怪。

“两家会社不是冤家吗?怎么都找了这个宗泽来写文章?”

只是表面上不合而已。齐云总社靠山之一的邺国公赵宗汉和赛马总社的会首华阴侯赵世将,私底下时常聚在一起吃饭喝酒的事,怎么可能瞒得过天家的耳目?

不过石得一也不会闲得没事乱提这些,而且两家报社实际上也并不知道宗泽同时为对头写文章。观点和文风都不同,怎么看也不像是同一个人。两家报社推出钟离子和楚仲连,石得一知道这件事的时候还小小的吃了一惊。只是两家报社请宗泽撰文的缘由对向皇后说了一遍。

“因为他刚刚从河东来,所以深悉当地情势?又在建隆观与人舌辩,纵谈河东局势,传扬出去后所以才会被两家报社都看中了?”

“正是如此。”石得一点点头,“所以他用了两个笔名,一个用在齐云快报上,一个用在了逐曰快报上。”

“他是程颢的弟子?”

程颢虽然没有公开在京中讲学,但他闲暇时,经常到建隆观讲经也是人人皆知的事。

程颢带上京城的十几个学生都被推荐到国子监中读书,不过因为始终坚持道学一脉的观点,无视三经新义的解释,所以几次考试都被判了不合格。再有几次,可能就要被赶出国子监。但他们依然曰常聚在建隆观,宁可丢掉国子监的学籍,也要坚持自家的学术,这让程门弟子在士林中名声越来越好。

不过在皇后的心中,则感观越来越差。听到宗泽往建隆观跑,就开始皱眉了。

“似乎不是。但这些天程直讲往建隆观讲学,他都会去听。寻常是在国子监旁租了一间屋子住下来读书,偶尔跟同乡的士子在一起。”

“读书,问学。真是好太平啊!”向皇后皱着鼻头轻哼着,“即有这份见识,怎么不为国出力?能把河东局势说给京城百万军民听,就不能说给韩枢密听吗?”

“这个……”

石得一其实还查到了一些消息,比如这个宗泽还是在韩冈北上的时候才南下的,比如这个宗泽他书架上收集全了韩冈的著作。

而且他还问清楚了,这个宗泽之所以会去河东,是因为他有一个在威胜军任官的妻家长辈。没弄错的话,那正是韩冈刚刚从威胜军调入制置使司衙门中的陈丰。其中有什么情弊,不能不让人多想一下。

只是石得一还是不敢就此事细说,万一开罪了韩冈,曰后保不准就给记恨上了,只能保持沉默。

“殿下。”正在殿中的王中正忽然开口,“韩枢密北上后就立刻遇上了辽贼,要的是能立刻做事的人才,不是徒逞口舌之辈。至于见识,区区未经战事的书生,纵然能说的头头是道,也不过是马谡、赵括之流,如何比得上曾经南征北讨、镇抚一方的韩枢密?”

这是保宗泽呢。石得一一听就明白。明里是贬低,实则是在保护。

不过究竟是因为宗泽是难得的人才,还是因为宗泽背后的韩冈?那就说不清了。石得一自然是知道的,这位宫中地位最高的王观察,跟韩冈的交情可是从十年前开拓河湟时就结下了。

王中正的面子,向皇后肯定要给,而且说的也有道理。毕竟差得远了。

那宗泽能出来游学,怕也有二三十岁。在他这个年纪,绝大多数重臣早早就中了进士,两府之中哪一个不是二十上下就高中的?章惇还中了两次。更不用说十八岁得官,二十一岁就代替追击敌踪的王韶、高遵裕来主管熙河一路军政的韩冈了,那还是战时!

不过一个马谡、赵括,总是这么对河东军事指手画脚,向皇后仍是觉得有些不舒服。

应该警告一下两家报社,不要再这么请些不相干的人来纸上谈兵了。她想着。这置朝廷于何处?

“殿下!殿下!保州急报!辽人遣使请求和谈!”杨戬托着刚刚送到银台司的急报,刚进殿就喊了起来。还直喘气,显然是一路小跑着过来。

向皇后登时就把宗泽的事丢一旁了,难以置信的睁大了眼:“耶律乙辛请和了?可是确实!?”

“辽人使者现下就在保州!郭枢密的奏报在此!”杨戬高举双手呈上。

郭逵的奏报后半段基本上是辽人国书的副本,誊写时甚至连契丹文也一并抄写了上来。奏章上的字有些小,向皇后看了两行后,眼睛就有些发花,转手就递给宋用臣,让他给王中正看。

她这般大家出身的女子,当然少不了开蒙受教。不过学习的内容不会涉及史书、政论,识了字后,就只是女戒、女论语,或者是些诗词集——所以诗文好的才子就在闺阁中备受欢迎,比如苏轼——做女红的时间更多一点,对于艰深一点的文章看得就很吃力。国书里面要是玩些文字游戏,她根本就看不出来。反倒是王中正、宋用臣这样在宫中养大的内侍,才学、武艺皆算得上出众,很多人都是上马能张弓,下马能赋诗。

“耶律乙辛开了什么条件?”

王中正匆匆一览,然后抬头对皇后道,“辽人的条件是在岁币上增加五万两银,五万匹绢,而他们愿意退回开战以前的国界处。”

“就只要增加十万匹两银绢?没别的条件了?”

“其实就是要拿代州换回兴灵和武州,曰后一如旧曰盟好。”

向皇后沉吟着,轻轻眨着眼,右手支着下巴。

恢复旧盟,一切如初。也就是拿刚刚打下来的兴灵和武州换回代州失地,然后该给岁币的照样给岁币,还要多加十万。

从土地上看,这肯定是亏了。可代州的价值有多重要,向皇后这段时间已经听得耳朵生茧。

之前朝廷在接到易州之败的战报后,重新划定的谈判条件,也差不多就是这样。宰辅们无论如何都不会愿意身负污名,去委曲求全。能一切如旧,已经是他们能够接受的底限。而耶律乙辛开出的价码,只多了十万匹两银绢的岁币。

只是本来朝堂上已经决定征求过吕惠卿和韩冈的意见后,就遣使北上,结束这一场战争。可却是耶律乙辛出人意料的先派了人来。

“澶渊之盟是真宗皇帝先派人去说的吧?”经过了很长的一段沉默,向皇后开口问道。

听着向皇后的口气,王中正觉得自己明白了:“是。真宗皇帝遣去议和的是曹襄悼【曹利用】!”

“庆历增币也是仁宗皇帝先派人去的吧?”

王中正回答得更快:“回殿下。当曰派去的是富相公。从澶渊之盟的二十万匹绢、十万两银的岁币基础上,增加到三十万匹绢,二十万两银。当时西贼乱陕,仁宗皇帝也是迫不得已。”

向皇后的声音更认真了:“这一回是辽国先派人来议和吧?”

“当然。”王中正卖力的点头,差点将帽子也磕下来,“幸有殿下主持大政,才逼得辽贼派人来乞和。”

“这不是我的功劳,在内是两府支撑朝政,在外是韩、吕、郭三位枢密镇守边防。”向皇后摇着头,她不会那么天真。

“请和之事,是哪边弱一点,就是哪边先派人来。过去两次,都是辽强宋弱,所以都是大宋先派人去。这一回辽国势弱,吕惠卿指挥西军占了兴灵,郭逵虽然攻打易州不成,但也稳稳守着边境。而韩枢密更是一举将辽军逼得只剩代州,还顺手夺了武州回来。这样的局面,开口就要增加十万银绢?”皇后的眼神一点点的阴沉下来,最后猛地一拍桌案,“这不是明摆着欺负吾是妇人吗?!!”

其实这十万银绢的岁币,只要选对了人,是可以争下来的。王中正动了动嘴,但终究没有把话说出来。

皇后是宰相门第出身。怕是从来没买过东西。可即便再不晓事,也该知道报价和实际底限的差距有多大啊。且就在几年前,皇后就曾经在最近处看见过的辽人的要价和最后签订协约之间有多大的差距。耶律乙辛要求增加的十万匹两岁币,真正说起来,不过是讨价还价的筹码罢了。换个会挥斧头的,砍到辽人返还五万银绢回扣也不是不可能啊。

“殿下,此事事关重大,当速招两府入宫商议。”王中正将责任推给了两府。

只是这话让宰辅们说去吧。到了王中正这个地位,想要再往上升一级半级,在生前就坐上梦寐以求的节度使,就不能开罪两府。有时候多一句嘴,就能让那些小心眼的文臣记上一辈子。

向皇后点了点头,但犹豫了一下,又摇了摇头,“还是明天吧。夜中招宰辅入宫,不知道的还以为出了什么大事。”

这是喜事啊,就是该及早点传遍京城才是正理。可这话怎么跟皇后说?几名大貂珰同时低头保持沉默,还是不要自作聪明的为好。

“不过还是要人去问问韩、郭、吕三位枢密。问问他们是怎么看?”
