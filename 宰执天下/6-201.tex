\section{第22章 鼓角连声彩云南(上)}

轰然一声巨响,仿佛只存在于苍穹之上的雷鸣,出现在大地之上。

来自于西南群山中近百部族的族长、贵胄和战士们,在与大宋官军并肩作战的这几个月里,已经熟悉了这样的声响。

从一开始的惊慌失措,敬之如神,到如今的视若平常,不动声色。

甚至有几个聪明的部族之长,在见识到了火炮的威力之后,设法打探到了其运作的原理,甚至火药的成分,动起了如何让自家的匠人,打造这种武器的念头——大一点的部族都不缺匠人,会铸钟的工匠也不是没有,而且火药这东西的原料,西南也不是没有。打造出来后,放在自家的寨子中,也多了一件保家的利器。

不过今天,第一次见识到火炮轰鸣时的惊慌和敬畏,重新降临到他们的身上。

来自群山之中的人们可以逐渐无事宋军声势烜赫的火药武器,但他们却不能对炮口前绽开的血花视若平常、不动声色。

五门火炮的炮口余烟袅袅,炮手们忙着清理炮口和炮膛,之前绑在炮口之前的死囚,已经化为了满地的碎肉。地上的残骸还没有收拾,新的一批又被拉了上来。

火炮不住轰鸣,血色弥漫于刑场之上,熊本却在放声大笑,“都说这是好办法吧?”

站在一旁的赵隆点了点头,同样在笑:“的确是。”

在宋军主帅纵声大笑的场面中成为背.景的,是上百名脸色苍白、捂嘴欲吐的蕃人族酋。

望向熊本、赵隆的眼神中,充满了畏惧,那些已经化作了血泥碎肉的物体,原本还是他们的同伴,一同听命南下,相互间争夺功劳和战利品。为胜利而醉酒,为俘获而喧嚣,现在已经烟消云散了。

“多亏了看了报纸。这么好用的手段,多亏了那伪帝能想得出来,所谓推陈出新,不外如是,不外如是啊!”

辽国伪帝耶律乙辛是怎么对付他的敌人的?他的手段通过各种途径,传到了宋国,之后又被刊登在了报纸上,传遍了天下。

原本是被当做伪帝残暴不仁的证据,现在却被熊本活学活用。

其实要说刑苛,大宋一点也不输给北方的邻居,凌迟、腰斩之类的法外之刑,从来都没有断过——刑统之中,死刑不过斩、绞,其余更为酷毒的刑罚,全都是法外之刑。而在军中,法外之刑尤为多,这个时代的文臣武将都有相同的想法,要想震慑一干杀人放火都毫不在意的赤佬,只有用更为残暴的手段来让他们感到恐惧。对付手下的蛮夷,自然手段更多了。

不仅仅是观刑的蕃部成员,就是赵隆,虽是在一同说笑,但心中也不免丝丝冒着寒气。

只有熊本,完完全全的投入进去,兴奋地仿佛看了一场大戏一般。

没有穿甲胄,也没有穿官服,普通的日常服饰,看起来就是一名斯斯文文的老学究。但他终究是让西南夷俯首帖耳,蛮夷家的小儿不敢夜啼的熊经略。

‘文臣……文臣……’

那些蕃人仿佛被蛇吓到的青蛙一般,赵隆望着他们,有那么三两分感同身受。

这些闻名天下的帅臣,又岂有一个好相与的?又有哪个会心慈手软?小觑了他们,自然会落到这样的下场。

赵隆是亲眼见识了王韶当年是怎么碎剐杀良冒功的士兵,也知道了韩冈是如何让交趾人都只长八根脚趾,熊本有样学样的把犯了事的蕃人绑在火炮炮口前,他除了一点点的兔死狐悲,真的是一点也不惊讶。

不过些许感同身受的同情心很快就烟消云散,自幼生长在战区,赵隆他对不顺的蕃人一直都抱着杀之而后快的念头,现在的一点感慨也不过是手段太过酷烈罢了。

真要让四方蛮族,从此不再为中国之患。一个是教化,让所有夷人都羡慕中原的文明,甘愿接受朝廷的统治,另一个就是震慑,以煌煌武功,在教化完成之前,让首领们不敢率部作乱,这就需要朝廷投入大量的人力物力。所以一直以来,大宋多是以羁縻的形式,来解决边境上的问题,放弃了开疆拓土的追求。相对而言,赵隆还是更喜欢现在的做法。

“此辈畏威而不怀德,不多敲打,就不知道自己姓什么了,总好像朝廷有求于他们。大帅处置他们正是时候,这下子该知道好歹了。”

前段时间,朝廷为了让这些蕃部来配合官军的作战,给与他们的条件实在是太宽大了。

大理国中各部族的子女任其自取,财富也任由其攻夺,只要他们肯派兵跟随官军一起南下,助长声势。

相对于在韩冈手上受了一年多磨练的广西诸蕃部,一直被打压的西南夷,突然间为朝廷用兵大理所倚重,如此剧烈的转变,不免让那些蕃部的首领认为是朝廷因为形势窘迫,不得不给他们好处。

既然知道朝廷少不了他们,自然有人会得陇望蜀,想要捞上更多的好处。尽管官军表现出来的战斗力让他们不敢倒戈一击,但行动间不免恣意妄为起来。

为了争夺俘获,各部族私下里火并了不是一次两次;而对敌,也是稍遇敌踪便立刻知会官军来援,对此,熊本一直视若无睹。

但熊本的沉默,不过是等待爆发的时机。

不论在何地,汉人总是能够让自己生活得更好。不论在大理,还是在其国周边的蛮部之中,绝不缺乏汉人的身影,而他们也几乎都是有着丰裕的家财。南下之战的一开始,熊本便以行营总管的名义,严令诸部不得掳掠汉儿,但凡俘虏中的汉人,必须将其全家老小一并送还,而且严禁掳掠其家财。

一开始,每一家蕃部都老老实实的奉若圣旨,但几个月下来,终于有人忍不住将手伸到了身家丰厚的汉人身上。

而这一回,熊本就没有再当做没看见。

火炮的交替轰鸣,已经超过了十轮,地上也多了一层厚厚的血肉。

三个犯事的部族,其族中的大小头领,只要身在南下的队伍中,都被绑上了刑场。昨日熊本还写了一封信,让人加急送往后方黄裳那里。斩尽杀绝,鸡犬不留,这就是熊本的打算,他打算用几千人的血,来强调汉人的地位。

至于三家部族的部众,则被驱赶攻城,昨日在龙首城下死得干干净净。

利用他们的牺牲,一天之前,赵隆率军攻下了重兵把守的龙首城。

龙首城是大理国都北面的最后一道屏障,从几百年前修筑时开始,就与龙尾城一起,保护着南诏、大理两朝王家在洱海之滨最为核心的一片土地。

龙首城护卫着大理国都的北面,而龙尾城则护卫着其南方

即便这一片土地中心,从古太和城,迁到到了十余里外的羊苴咩城——也就是如今的大理城,龙首城和龙尾城的地位也是没有改变的。

一战便攻下了大理国都城的北面屏障,一支前锋更是直接进驻了被废弃的南昭国都太和城,这一场的战争,分明已经到了最后的阶段。

所以熊本才不怕杀人,也必须要杀人。不管大理的军势有多么颓废,但大理毕竟是千乘之国,享国日久,对付官军固然力有不逮,但对付随同南下的蕃部,大理军还是有信心的。

熊本正是眼见着大理军偷袭越来越猖狂,而自己一方的蕃军又往往不堪硬战,方才找理由将那些不擅外战却精于内斗的家伙一一解决——没人想看着稳拿稳的胜利从自己手中溜走。

大理军并不算弱,开战以来的几次交锋,大理军已经了解到了宋军惯用的破城战术,只要大理军退守到城寨中,立刻就会被宋军用不知什么样的手段破坏了城池,再坚固的城墙,会在霹雳一般的巨响之后,变成一地的碎石,从来没有一座城寨,能在宋军的攻击下,守住三日。

所以在宋军终于穿越崇山峻岭,来到大理国的核心地区之后,大理军开始针对性的布置,而不是龟缩在城中。

而跟随宋军南下的西南诸蕃的贪婪和残暴,也让段、高二氏彻底联合起来,同时也带动了大理国中的诸多部族。旧日的矛盾在灭族的危机之前,只是不值一提的小事。这种最基本的见识,很多时候,自诩文明的汉人内部都欠缺,但这一回,反倒是大理国的君臣给了熊本等人一个惊喜。

求和的使者就是在这个时候走进了大宋的军营。

并没有拉着人去还没有收拾好的刑场走上一遭,熊本就在行辕中接见了这位使者。

“我曾听说大理城户户飞花,街街流水,乃是南国第一的去处。如此胜景,毁于我手,实在不忍,若贵主能够自缚出降,本帅便放过这座大理城。”

使者还有些胆量,夷然不惧,“小人曾闻,上国发兵犯境,是欲为我大理国拨乱反正,不知今日,经略相公让吾主出降,可是奉了圣旨?”

“我知道你们还不死心。”熊本冷笑,根本就没有为这点言辞上的小把戏所拿捏住。

“本帅在京中便听人说,大理四季长春,百花不尽。如此佳处,既然你们都不在意,那么本帅也只能先拿下再去在意了。”
