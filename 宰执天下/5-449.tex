\section{第38章 何与君王分重轻(七)}

午间的席上,王安石出奇的精神。

尽管一如往曰的盯着面前的一盘菜,又时时陷入沉思。可看着就比之前有精神得多,说话和眼神都有着慑人之威。

这让吴氏及王家的子女都心中纳罕,不时的去看韩冈,不知道他是怎么将王安石给刺激得精神起来。

但王安石并没问韩冈那一句的来源——只有王旁时不时的瞥眼过来——甚至什么都没说。要是他问了,少不得就要费一番口舌推到无名氏的身上。

不过在送韩冈的时候,王安石才对韩冈丢了一句出来:“玉昆,你那一句沧海桑田,老夫记下了。”

“什么沧海桑田?”韩冈回家的时候,王旖就忍不住发问。

韩冈也没隐瞒,跟王旖将书房里的事说了,算是解了她的疑惑。只是她又怔怔的看了韩冈半天,眼中尽是惊异。

“怎么了?”韩冈心中不解。

“没事,没事。”王旖忙摇摇头,问韩冈:“全篇呢?”

“什么全篇?”

“那分明就不是对句,只可能是结句。”

“算是吧。”韩冈漫不经心的应道。

王旖兴致高了起来:“对句好可得,结句好难得。官人的那一句既出,当再无人敢借用李长吉的‘天若有情’……气象不同!”

一句诗的好坏只有放在全篇中才能得到正确的评价。一点墨迹,只有正正点在眼眶中,才有画龙点睛的效果。换做是石灰粉过的墙壁上的黑点,那是拿笔时打喷嚏,不小心将笔尖摁在墙上——家里给孩子就读的书房墙壁上,都是这样的黑团团。

不过韩冈凑上的一句,不是对联的下联,也不是需要对仗工整的颈联、颔联,看着像是一首律诗的尾联。好坏且不论,倒是硬把‘天若有情天亦老’的意境拔高了一层。正如王旖所说,气象不同。原诗和一干借用的都是自怜感慨为重,而现在的一句却是厚而大。诗言志,至于此,无余事矣。

“娘子太高看为夫了。”韩冈摇摇头,差得太远,而且是全方位的,“没全篇。就这一句,应时应景。听仲元提到,突然想起来的。”

王旖又盯着韩冈半天,发觉他真心不想说,便长长地叹了一声,回到了正题上:“爹爹就是倔脾气。官人你若是不去说那一句,说不定真的一切都放下了。”

“哦,看来为夫也算是做了件好事了。”韩冈笑说道:“岳父跟为夫一样,是劳碌命,闲不下来的。”

王旖变得不高兴起来:“官人既然自称劳碌,不知是为何劳碌?”

“教化万民啊。为夫最大的愿望就是人人读书识字。一百人中出不了一个人才,那就一千人,一千人中不出了一个,那就一万人。就学的人越多,人才就越多。而能让弟子青出于蓝者,方可称良师。如夫子者,更可谓之至圣先师。”

“‘吾与女[汝]弗如也’?可颜子只有一个。”王旖一下就抓住问题。论语中,孔子自承不如颜回,但复圣也就这么一个。何况焉知不是圣人自谦?只有一个弟子超越自己,按韩冈的说法,怎么能为至圣先师?

“娘子家学渊源。”

“《论语》都没读过,怎么能算上过学?”

“那‘三人行,必有我师’呢。夫子三千门徒,其中倒有一千个能做夫子师,这算不算青出于蓝?”韩冈是半开玩笑了,“先人不过通往大道的一级台阶。让后人借力走上去,能够更近大道。”

他比韩愈更进一步了。师不必贤于弟子,韩冈则是干脆说师长是弟子的踏脚石,能让后人更贴近大道。

王旖摇摇头,她实在是很难理解韩冈的想法,也不该说什么好。

韩冈也不想再说了。

他甚至连吕惠卿都不放在心上。朝堂之上,自有蔡确和曾子宣跟他打擂台,不要想有清静的时候。

而他本人的态度,这个枢密副使不做也罢,将挂在身上的靶子丢到一边吸引箭矢,自己也就能够轻松上阵。

真正的争夺是在太子那边。谁都想要一个传习大道的皇帝学生。但资善堂处,还有这些天几乎被人忘掉的程颢。

韩冈的半个老师,现在似乎比王安石更得。王安石还要分心政事,而程颢的心力就全在教学上。如沐春风般的授课,不仅仅在京中士林渐渐受到尊敬,也让太子赵佣和伴读的王益——王诜与蜀国公主的独子——都乐于上程颢的课。

这才是大问题。

……………………

韩冈、王旖带着孩子到家不久,冯从义就跑过来了。

韩冈昨天才抵京,没有来得及跟冯从义深谈。这段时间,京城钱币波动极大,冯从义主持银号,免不了被牵连进去。

直到韩冈从河东送了一篇文章,在报上发表了。

冯从义将韩冈的《钱源》看了一遍又一遍,简直五体投地。

按韩冈的说法货币只要维持住信用,就能通行于市。可是现在京城百姓对大钱的信心一落千丈,除非韩冈能继续得到重用,否则折五大钱就很难继续发行。

“现在的折五大钱都快赶上陕西当年发的交子了。”在韩冈面前,冯从义叹着气,钱币贬值对他的生意影响很大,想想也不免抱怨。本来以为韩冈回来后会有所改变,孰料。朝堂上的变化直接让他的希望落空。

“交子现在不能发。陕西、蜀中倒也罢了,都是有缘故的。可京师腹心地,哪里随便发纸钱?”

“可惜不是陕西、蜀中。否则就没那么多麻烦了。”

“因为朝廷的信用撑不住。”

韩冈很清楚,执政最忌讳的就是挑战民众的底限。百姓对朝廷的信心向来不足,至少现在还不是使用纸币的时候,必须是硬通货才行。韩冈之所以不动交子的心思,就怕朝廷兴起时,在中原、江南开始发行纸币。

毕竟蜀中、关西是特殊的货币区,早就习惯了,但中原腹地不能这么玩,至少现在不行。拿张纸出来,信用无论如何都很难维持,一旦贬值就会没有底限。

其实这个时代已经有了很明显的信用制度。最典型的例子就是交子。

庆历年间,为了应付西北边境上曰益膨胀的后勤需要,困于粮草输送的朝廷,便扩大了入中的制度。并为此发了六十万贯交子,送粮到前线,便能收到交子作为凭证。而交子抵代的金钱,则是让转运使辖下诸军州拨还。虽说按朝廷的要求是‘止当据官所有现钱之数印造’,有准备金就印多少交子,仅仅起到是兑换的用途,免去运输钱币到前线的耗费。可实际上,交子的印制数量都远在准备金之上。

之后西北战乱不断,入中纳粟的政策不得不继续推行,陕西交子也跟着发行,至今为止已经发行了二十多界[通届]。陕西的交子以两年为一界,每界到期,便让民众将手中旧交子兑换成新交子,由于兑换有名为贯头钱的手续费,还有因故无法来兑换被强行作废的交子,朝廷每年都能有二三十万贯的额外收入,这还不算超发的部分。而蜀中一直在发行的交子不论从时间还是规模,都远大于陕西,所以朝廷从中牟取的好处就更大了。

朝廷在交子发行的过程中,所得到的好处让几任天子都难以割舍,对朝廷财计有着巨大的补益。故而相对的也会维持交子的信誉。当交子贬值,当地衙门都会出钱购回一部分交子销毁——纵然不知道货币信用的本质,可行动却是合乎道理的。

除了交子之外,还有盐钞。同样也可以算是钱。交子以现钱为储备金,而盐钞那是以盐为储备金来发行的货币,商人拿盐钞可以去兑换相应数量的解盐。有有现钱公据,类似盐钞,不过是改成在在京师兑换钱币,类似后市的支票,朝廷直接拿来在陕西购买粮草。

但这些都拥有着货币属姓的纸张,都不能得到人们的信任,陕西的商人一拿到手,往往就在当地给卖出去了,换成现钱落袋为安。而收购这些票据的,却是宗室和戚里,回到京师的交引铺来足额兑换。

“如果朝廷的信用能维持住就好了。带上一摞钞引,比随身带着两辆装满钱的大车都安心。”

远程商贸中,聪明的商人都采取往返贩运的形式。并不直接运回货币,而是在当地采购特产,称为‘回货’。在过去茶叶不由官卖的时候,川盐陕茶常互为回货。

再比如成都府路将蜀锦运往中原,兑换成白银回川。陕西四川虽然铁钱可通用,但运输相当困难,所以常常借助银绢。

蜀中征收的赋税泰半是银绢,有半输银绢的规定,上供也以丝帛居多。同样价值的白银和绢帛远比铜铁钱要轻,易于输送,这就弥补了蜀地铁钱价值太小,运输不便的缺点。

如今的陕西,情况也类似。尤其是秦凤转运使路,缴纳的税赋是由棉布代替了绢帛,一半是现钱,剩下的一半是白银和布匹。
