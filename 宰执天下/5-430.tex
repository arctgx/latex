\section{第37章 朱台相望京关道(01)}

起来的时候,王安石先拉开窗帘望了望天色。

透过新近装起的玻璃窗,看不到天上的星月,黑沉沉的一片。当是阴天无误。不过打开窗后,迎面而来的风很是凉爽,没了前两天让人烦躁的暑热。

吴氏被王安石的动作给惊醒了:“今天要上朝?”

“是要上朝。”王安石叹了一声,转过身来,“你再睡一会儿吧。”

“哪里还能睡得着。”吴氏也起来了,叫了外屋的使女进来服侍更衣。

换上了朝服,匆匆用过粥饭,来到院中,王旁和出行的元随队伍已经在等着了。

“大人。”王旁上前问安。

他很早就起来了。这段时间在粮料院只挂个名,实际上主要的工作还是管家。家中迎来送往的大小事宜,都是王旁来处置。至今也没有上朝的资格,不过他要为父亲王安石准备,上朝日时还是免不了要早起。

小心的服侍着父亲上马出门,王安石突然想起了一件事:“二姐今天要带钟哥、钲哥回来,别忘了准备了。”

“二妹妹昨天就派人来说过了,这两日暑热,就不让钲哥他们过来了。”王旁纳闷,自家的父亲怎么不知道,“今天下午就二妹妹自己来。”

王安石楞了一下,脸色又黑了两分,这一回闹得,连女儿都生分了,“你娘知道吗?”

王旁小心翼翼的看了王安石一眼:“就是娘昨儿说给儿子的。娘还说这两天去常乐坊那里看看。”

王安石心情更坏了,他没想到吴氏竟然绝口不提,“辽人多诡诈,且河东北方诸郡人心不稳,只为这一事,玉昆就得多留在那里几日。”

他也不知是在向谁解释,说了两句,摇摇头,挥鞭驭马往皇城去了。

夏日的朝会比起冬天来,要让人感觉好很多。

不仅仅因为不用冒着凛冽的寒风,也因为只有清晨时才有的凉爽。

迎面而来的凉风,尽管带着城外铁场的烟火气,王安石胸中的郁闷也为之消散了些许。

王安石为平章军国重事,其位权柄极重,军国重事无不可以与闻,故而之前被约束到五日一入朝视事。

不过在战争期间,王安石则日日上朝主政,实际上已经把军政之权牢牢控制在手中。不过战争结束之后,他又回复到之前的状态,并没有恋栈权位的意思。

不是因为觉得麻烦,而是心怀愧疚,所以尽量避免多去见赵顼——每天宰辅们都要入宫探问,王安石正是为此才不愿意多去朝堂。

他能得到今日的地位,能尽情施展自己的才华,实现一直以来的抱负,都是当今天子重用他的结果。现如今却要将国事欺瞒,纵然有着充分的理由,但心中还是免不了有着沉重的负罪感。

……………………

“王平章今天的脸色看起来不太好。”

文德殿前的队列中,张璪低声跟蔡确说着。两人在政府中势单力薄,自然而然的就走得近了。倒是曾布,却谁都不理,似乎要做一个孤臣的模样。

“平章家的娇客无所不用其极。做岳父的脸色如何好得起来?”蔡确笑道。

前日当政事堂收到了韩冈与张孝杰对话的记录,大发雷霆的王安石到了最后也只能决定看看再说,最后什么有意义的决议都没有做出来,

“那番话也亏韩玉昆敢说。传扬出去,东南西北都难安稳了。尤其是陕西那里,吕吉甫一直都在想办法回京,得了韩冈的提醒,还不知会怎么做。”

妄启边衅的罪名一向不轻,这是为了约束边臣不要贪功生事,而且在朝堂上的宰辅们一般来说也不喜欢会破坏朝中政局平衡的战争。但韩冈的话却是给边地守臣的野心找足了借口。

无论是在张璪眼中,还是在蔡确看来,韩冈的一番言辞都是彻头彻尾的威胁。无论朝野都会因他的一番话而动荡起来。

按说朝堂的变动不关小民的事。可韩冈是种痘法的发明人,他说出来的话,又是与种痘法紧密相连,怎么可能不会引起民间的议论?那毕竟是韩冈说出来的,同样的话从不同的人嘴里说出来,引起的反应当然不会一样。

一旦天下士民听闻韩冈的言辞,恐怕都会毫无保留的相信他的话,而希望朝廷能解决这个其实并不是很急迫的问题。

尽管如此,两府却对此很难驳斥或压制。韩冈评价自己的功业,而且是贬低,外人如何能插话?而且从道理上说,他的一番话没有半点不对的地方。

道理极为朴素,百姓吃不饱饭要么饿死,要么造反,后者的可能性还高一点。而要让人吃饱饭,就要开辟出与人数相适应的田地来。但要做到这一点,就要看是什么地方了。

蔡确是福建人,很清楚在他的家乡,那些平民百姓为了保证能养活家中已有的子女,最后会怎么处理之后生下来的幼子。

除了种痘法之外,韩冈还有一系列有关医疗厚生方面的成果,也都推行到天下。

别的不说,蔡确的族中,近些年来所生育的幼子,夭折的比例比十年前要少了近半。原本是五五开,现在至少能有七成了。

这个看似喜人的势头,却正好印证韩冈一番话的正确性。

因为能长大成人的幼儿多了,田地增加的速度赶不上人口的增长。如果不能增加可以耕种的田地,增加的人口也就会变成水里的亡魂。

可在韩冈本人而言,这一番话肯定是借口。为了回到京城的大棋局上而下的一手。

两府之中,人人都是眼睛雪亮。谁也不会相信韩冈只是乱说话。只是到现在为止,他们根本不知道韩冈的计划到底是什么?可以选择的手段太多了。就像是石头砸进了缸里,同时破了几个洞,不止一个地方会漏水了。

处在相同的位阶上,张璪怎么可能明白不了韩冈的想法,反正情况再坏也坏不到哪里去,他还有什么好顾忌的?这也是免得有人把他当软柿子来拿捏。总结起来,终归就是一句——

“他是唯恐事情闹不大!”

“谁说不是……只是韩冈这么一来,陕西那边也少不了会有动作。”蔡确:“谁让吕、韩都有便宜行事之权。”

“不过宣抚、置制不可久任。拖也拖不过一年半载。”

现如今,两府以御寇备辽、以防反复为由,让吕惠卿和韩冈继续以宣抚使、制置使的名义,留在陕西和河东。这样的情况下,他们两人手上的便宜行事的授权是不可能收回的。

此外宣抚使和制置使都是临时性的差遣,并非经制官,这就是棘手之处。经制官,一任两任三任三年六年九年的丢在任上,都没有任何关系的,很正常的人事安排。但宣抚使、置制使权柄过重,因事而置,事毕则罢,若是久任多年,即便现在不会出一个藩镇,有了故事循例,日后也是重蹈唐时覆辙的肇因。

蔡确并不在乎日后会不会变成中唐晚唐的局面,他一点都不放在心上。可是如果两府决定让吕惠卿或和韩冈以宣抚、置制二职久任地方,肯定会引来极大的反对声,这便是给了皇后以借口。上下相逼,两府何能一意孤行?届时朝堂上的风向一变,吕惠卿就必然会借力返回京城。

可难道还能任命他们为安抚使不成?那可是形同贬责。赏罚不公,同样会掀起轩然大波。

“其实能有个一年半载也差不多了。”

把他们拖在京外,总能寻到错处。且如今因为对辽战争的胜利,两人名望大增,可晾上半年之后,声势就不会有现在这么大,到时候怎么安排都容易。

“平庸之辈自是如此,但吕吉甫、韩玉昆可都是敢做敢为啊。这一回不正是明证?”

他们可绝不会缺乏挑战底线的胆略。

韩冈当着张孝杰的面所说的一番话,传出去就是给了吕惠卿再次整顿武备的借口。甚至韩冈本人都有充足的理由整军备战,保护边地的百姓在辽人的鼻子底下开田种地。让朝堂为之提心吊胆。

蔡确设身处地的为吕、韩二人着想,如果他处在两人的位置上,一切的关键就在那‘便宜行事’四个字。

“那怎么办?”

“现如今也只能头痛医头,脚痛医脚了。”蔡确很无奈,“陕西、河东就不消说,光是开封市面上的谣言都不知道该怎么办了。”

“昨日不就已经通知两家会社,不得刊载任何有关的话题。皇城司都遣人去书局盯着了?”

不消蔡确、张璪多提。他们两人前些天就坐在政事堂中,共同讨论该如何处置妄报国家机密的两家报社。但最后也还只是不了了之。把两家报社查封其实更好,可一旦那么做,就更会惹起谣言,原本不信的都会相信了。

蔡确记得前日报上曾经刊载了这一回重造籍簿所统计出来的天下户口的总数。本来蔡确只是觉得是商人逐利之举。但现在看来,却似是另有一番缘由了。

不过报纸刊不刊载已经不重要了,当年没有快报的时候,谣言照样禁不住。

说起来一味的堵并不是上策,以两家报社的在都下士民心中的地位,应该要好生利用才对。
