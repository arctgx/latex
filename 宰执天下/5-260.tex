\section{第28章 官近青云与天通(27)}

开始书写诏书时,孙洙手有些颤,这个参知政事的人选是谁也没有想到的。

一旦在宣德门外张榜公布,不知会惊到多少人。

时隔七八年,沉浮于南方诸州。

想不到天子竟然还能记得他。

曾布曾子宣。

……………………

“竟然是曾布!”

“曾布为参知政事?!”

韩冈猛然坐直了身子!与苏颂对望的眼神中满是讶色。

已经不是‘新党大兴’的问题了,天子这明摆着是等不及形势自然而然的发展,而是光明正大的要逼新党分裂!

曾布可是被王安石恨之入骨,与吕惠卿都是死对头。但这个曾布,毕竟也是新党的干将,一旦他上齤台,一样会坚持新法,只是跟王安石、吕惠卿肯定合不来。

控制得好的话,异论相搅同样可以成立。

赵顼虽然病重,但帝王心术还是用得这么溜。得到的结果远远出乎韩冈的预料,“家岳这一回可是要跳脚了。”

“当真这般恨曾子宣?”苏颂有些惊讶。

“恨之入骨。”韩冈很肯定。自曾布叛离新党,他从王安石口中听到曾布曾子宣这个名字,加起来也不到十次。

苏颂沉默了片刻,嘿然一叹:“这就是大拜除!”

韩冈点头附和:“的确是大拜除!”

不约而同,韩冈和苏颂都是将重音落在‘大’字上。

只要在朝堂上有三分经验,一看招入三名翰林学士草诏,就知道是什么样的情况了。但更迭两府人事,是一桩极为精细的手术。绝对不会如今天这般剧烈。即便因为国政需要,通常也要一年半载的时间。

王安石主掌变法,政事堂中的生老病死苦,分了王安石太多的精力。赵顼欲加以改变。可除王安石和曾公亮以外的三位,也是用了一年的时间才逐步更换完毕,换成了对新法掣肘不多的一批新人。

韩琦旧日曾一纸落下四宰执,那时倒是特例,今天则同样是特例。

曾布上来了,但两府中似乎人选还是不足。韩冈和苏颂又猜测起剩下的可能,最多也只有一两个空位了。

章敦转入东府升任参政的可能给他们共同否定了,这不是受到了韩冈的牵累,而是在曾布任参政后,东府中不需要再多一名新党。

“籓邸呢?”苏颂问着,“曾在开封府做过的那位。”

“是说孙曼叔?”韩冈立刻道。孙曼叔就是孙永,韩冈在开封任职时的老上司,去河东时的前任,韩冈与他颇有些交情,可他并不是个好人选,“孙曼叔更近于旧党,上去就会被弄下来。蔡确、吕惠卿容不下他。”

“愚兄说的是孙和父。”苏颂更正道。

“孙固?”同在籓邸,孙固的确也做过开封知府,不过韩冈仍摇头,“他的脾性可是跟他的名讳一样硬啊!”

元丰初,京城中已经被传言将要晋身枢密院的孙固,因为反对伐夏,被踢到河北去了。如果当时他松松口,绝不会是又回去知真定府的结果。而且他的立场也偏向旧党,上来就是被围攻的份。

此外曾经在两府中任职过的老臣们,元老们不用去考虑了,那是笑话。吴充前些时候已经病故,冯京倒是还活得滋润,但因立场关系,也是没戏。

韩绛年纪太大,快七十了。加之底蕴不足,回来也撑不住局面。当年以两任相国的资格,都压制不住政事堂,手腕实在是弱了点。而且他回来还会把韩缜逼出去,有不如无。

元绛元厚之年纪更大,已经养老了,更不可能卷土重来。

真正有资格就任两府的人选也只有这么几个,韩冈数来数去,也没有更合适的。

苏颂看了韩冈半天,突然问道:“玉昆,你怎么不说自己?”

韩冈咧嘴一笑:“小弟是不愿意……”他看看苏颂,“而子容兄你是不需要提,天子考虑两府之选,必然少不了你。”

苏颂没理会韩冈的后半段,追问道:“为何不愿意?”

“还是再过两年吧。小弟的年齿摆在这里,现在上去心里也不踏实。何况若是闹将起来就没时间做正事了。”韩冈冲苏颂笑了笑,“小弟倒是觉得子容兄你是最合适的人选。”

“愚兄不可能的。”苏颂很干脆地摇头。

“为什么?”韩冈疑惑起来。

虽然苏颂跟自己走得很近,又有姻亲。但他的年纪已长,在两府中做不了几年,完全没有章敦那般让人担心。且即便他不是赵顼心目中的第一人选,可因为近于气学,只要韩冈这边辞位,苏颂绝对是个最佳的替代选择,必然能得到韩冈全心全意的支持。

“籍贯啊……”苏颂对韩冈在这里犯糊涂有些惊讶,“玉昆,你不觉得两府中南人太多了一点吗?”

韩冈眨了眨眼睛,随即恍然。不比后世,如今地域之别,其实被看得极重。


南人不可为宰相,世传是太祖皇帝所说。而寇准知贡举,据传也曾经将南方士子大加删落,还说又夺南人一状元。到了王安石主持变法,司马光好像也拿他的籍贯说过事。而起新旧两党中,籍贯之分也十分明显。北人多旧党,南人则多隶新党。

眼下两府之中,天子大用新党,所以南人成了主流。章敦福建人,吕惠卿福建人,蔡确福建人,王安石江西人,曾布江西人,薛向、郭逵是另类可以不计,韩缜倒是河北的,有名的灵寿韩,可他眼下就是孤家寡人一个。

若是按照这一张名单定下了两府人事,再加一个福建籍的苏颂,两府之中北方人的比例的确是低过头了,而福建籍的宰执数目也未免太高了一点。

那么再接下来,就不再是新旧党争,而是南北之争了。情况反而会比之前更麻烦,天子稳定朝纲的心意也不可能达成。

不能身登两府,苏颂却毫无芥蒂的对韩冈笑道,“所以愚兄不可能入两府,之前也没有提乡贯淮南滁州的张璪,但玉昆,你可是北人啊。”

韩冈现籍关西,祖籍京东,当然是标准的北方人,但他不愿意凑热闹,摇摇头,继续喝酒吃菜。

苏颂却道:“不管玉昆你愿与不愿,只看你的身份、籍贯,天子不会落下你。”

“为什么不可能是韩子华【韩绛】替代?”

“说不定真的会有他。你一个,再加韩子华,就算韩玉汝不得不离开,也说得过去了。”苏颂看着手上的酒杯,“新旧两党处置完毕,现在天子应该想到籍贯了。”

要向平衡南北,必然要有个北人宰相。韩缜的政治倾向并不是新党,他是不可能被提到宰相位置上的,那只会让他成为众矢之的,根本坐不稳位置。维持现在的参知政事已经很勉强了。而韩绛现在却成了最合适的人选。而韩冈身份特殊,还是太齤子师,宰相之位不可能给他,可做参知政事或是枢密副使也能有足够的影响力。

小半刻后,拜韩绛为宰相的诏书出来了,而韩冈为枢密副使的诏书也只隔了两刻钟。

一切尽如所料。

“糟了,家里没人啊,别糊里糊涂的接下来。”韩冈虽是这么说,身子却动也没动,倒是开玩笑的意思居多。

苏颂也没催韩冈,这本来就是笑话,“拜除的诏书当会留到了明天的官衙中宣读。”

但片刻之后,韩冈和苏颂都跳了起来。蓝元震竟然背着个黄绫包裹带着十几名班直,找到了西十字大街横巷里这间不起眼的小酒店中来。

‘好个皇城司!’韩冈和苏颂的眼神中隐隐闪过怒意。连重臣都敢派人跟踪,改日揪住几个不长眼的,好好敲打一番!

但现在两人都不可能发作,只能出店到了院子里,小小的院落挤满了韩冈和苏颂的随从,根本就不是受诏的地方。

幸而拜除执政,不可能在小酒店里完成。蓝元震先满脸堆笑的向韩冈道了喜,然后就催促他快快回府接诏。韩冈摇头,辞而不受,三句两句就将蓝元震打发走了。

蓝元震走时倒也不以为意,宰执的任命,受命者肯定是要做作一番的。

一名受清凉伞的相公差点就在他家的院子里接了诏,躲在厨房里的店主一家已经有人吓得昏过去了。韩家的一名元随不耐烦,过去泼了两瓢凉水将他弄醒,让人继续上菜。

韩冈和苏颂重新坐定下来,苏颂笑问道:“玉昆,你现在还不想做吗?”

“我可不凑热闹!”韩冈摇摇头,他坚持着。

但接下来的消息让他眉头皱了起来。

或许是感受到了韩冈拒绝时的决绝,新一份诏书出来了,却不是有关两府的——程颢为资善堂说书,王安石为资善堂翊善。

说书和翊善都是资善堂的讲读官,与韩冈同为太齤子师。这两人,一个与韩冈有半师之谊,一个更是韩冈的岳父,平章军国重事。无论哪一个,都能在资善堂里压韩冈一头。

韩冈叹了一声,天子终究还是要压着自己。

“玉昆,枢密副使,你还不接吗?”苏颂语气沉沉的说道。

一抹嘲讽的笑意浮现在韩冈脸上:“天子以为小弟不担任枢密副使,就压不下新学洛学吗?”他的眼神转利,“若说新学、洛学,乃至其他学派,都是师长建个房子,然后学生们在里面叠床架铺。但气学不同,是一代更胜一代,后人学习前人经验,改正前人的错误,一步步向前。哪个能走得更远,站得更高,还用说吗?!”

对韩冈而言,《自然》期刊的意义,可比枢密副使重要得多,在刚起步时,他并不打算分心。何况一张清凉伞乃是自家物,迟早到手,有必要向皇帝低这个头?

不干就是不干!

而且皇帝的算盘,可不一定打得响。有些事,不是他把握得了的。

韩冈笑容中的自信,真实无虚。
