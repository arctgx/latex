\section{第28章 夜钟初闻已生潮(六)}

“陛下,官家御体并无大碍。”钱乙再一次重复,“只需注意日常饮食。孔有云,饮食,人之大欲存焉。”

钱乙对病家一向毫无欺隐,给小孩看病,最重要的就是得到其父母家人的信任,只是这一次,一贯直言的钱乙,尽管还是对诊断的结果没有隐瞒,却是难得的采取了委婉地说词,尽量避开直接叙述问题。

向太后胸口剧烈起伏,正竭力克制住自己的怒意

杨戬从侧面偷瞄了一下,当即一个哆嗦,太后的脸黑得吓人。

天一怒,伏尸百万。太后手权柄,丝毫不下于天。太后一怒,不要伏尸百万,只要宫伏尸,正巧撞在气头的小命可就悬了。

之前钱乙禀报自己给天的诊断结果,跟现在完全一样,都是一句‘饮食,人之大欲存焉’来说明。

对有着士人应有的常识,或有着相当于士人知识水平的人来说,钱乙的话说得并不委婉。不过太后只能背下《女论语》,《论语》却不成,所以并没有听懂。

但他杨戬听明白了。《论语》还算重要的一句,被钱乙漏说了两个字,从小就在宫接受教育,水平至少能做个乡学究的杨戬,话声一入耳,立刻就像是鞋底里进了一颗小石般硌着难受。

如此重要的问题,杨戬不敢对太后欺瞒,低声报告了自己的发现,只是心不免忐忑,如果今天不当直,就不必蹚这汪浑水了。

尽管学识不高,可向太后已经独力与那些老奸巨猾、一个个都是当世人精的宰辅重臣们打了五年的交道。周旋日久,在得知了孔原句的她,怎么会不明白钱乙想说却不敢说的话来?

‘饮食男女,人之大欲存焉。’

这就是孔在论语的原话,之后告与孟辩论时,也有过食色性也的说法。正常来说,当然没必要避讳,即使是在太后面前,但如果碰上了难以启齿、更不能轻易外泄的病症,钱乙一时间不能请太后屏退左右暗地里禀报,就只能用这种方法来告知太后。

他不想明着介入太后和天之间,但又不能不说。若是政治水平比医术超出几条街的那几位翰林医官,想必不会让自己落入如此窘境,可钱乙只有半吊的政治头脑,说出这番话来,实在是让他绞尽了脑筋。

太后终于克制住了自己将要爆发的情绪,疲惫不堪摆了摆手,“钱太医,你先下去吧。”

“微臣遵旨。”

听到太后的话,钱乙立刻忙不迭的拜礼而出,出殿的时候不小心绊到了高高的门槛,脚步一个踉跄,要不是反应快,差点直接就摔了出去。但站稳之后,他连尴尬的时间都没有,匆匆忙忙就疾步离开。

钱乙不知太后是忘了吩咐自己之后去给天复诊,还是故意不吩咐。即使失去了御医的资格,钱乙也不觉得可惜——以天的情况,现在已经不需要儿科医师了。

向太后心五味杂陈,在别无外臣的大殿坐了良久,突然一声:“杨戬!”

正自怨自艾的杨戬一个机灵,“奴婢在。”

“去召王正来。”

杨戬怔了一下,随即警醒过来,赶忙领命而出。刚转身准备踏出殿门,就隐约听见太后自言自语,“官家的御体重要,官家的御体重要。”

赵煦的身体的确重要。

赵煦自娘胎里便体弱。自出生后,补药从无一日不喝。自小到大,几乎就是一个药罐,时至今日,甚至连母乳都没有断过。

母乳,世所谓仙人酒,一直以来都被视为滋补圣品。只看新生儿在断奶前的一两年时间,就长高变重一倍两倍,就知道母乳有多养人。

《自然》也对母乳喂养有着极高的评价——尽管这个时代,不用母乳喂养幼儿的几乎没有——而且《自然》还有载,初乳最为贵重,内饱含母体自有的免疫之物,用以保护嗣。

富贵人家的女,往往多病的缘故,就是出生时是由乳母喂养,母乳元气不足,根基没有扎牢。天家不说了,高门显贵家,夭折的儿孙跟普通平民百姓家的比例相差不大,这明显不正常。

经过了《自然》详细剖析,即使是高门显宦家的新生儿,也不再完全交给乳母,都能吃到生母的初乳,再交给乳母,期间还会让生母喂养一段时间。

而赵煦,尽管他出生后,生母朱太妃错失了喂养第一口奶的机会,但太后一得知初乳有补于幼,就张罗着募集乳母,而且要正怀孕、不日即将生产的。

不过初乳,一名产妇也就能有三五天的量,所以天若是要以此来补身,就得长年累月,一年就要使用一百多乳母。

没有哪位宰辅敢于放任太后如此去做,所以韩冈便以有伤圣德、误民赤为由,竭力劝阻了头脑发热的太后,转头倒是将牛初乳推荐给了太后。

因为牛痘正以极为显眼的速度,不断减少天花造成的夭折,这使得向太后对韩冈将人初乳改成牛初乳这件事,完全没有心理障碍。

不仅仅是太后在给皇帝的每日补品套餐,将人初乳换成了牛初乳,其他许多看了那一期《自然》的富户豪门,本来是有心试一试初乳的好处,一听说太后做了什么,也都纷纷改弦更张,使得京畿和江南的小母牛的价格陡然间贵了三成。

在母乳这件事上,太后为了天,殚思竭虑,唯恐做得不够好。在其余补品上,太后也是一样的劳心劳力,正如同样在《自然》上出现过的蜂王浆。

蜂王浆能让蜜蜂幼虫长成蜂王,物性自是滋补,说起来对幼最好。天自幼体弱,只要蜂王浆有出产,都每天在吃着。

反倒是蜂蜜,由于只能让蜜蜂长成工蜂,宫就不给小皇帝吃了。进用的甜点,都不再掺蜂蜜,而是改成白糖。

“……只看官家今日的情况,这些补药的效果的确是好,就是好过头了。”

在王正到来之后,太后便又絮絮的跟这位老臣,将今天的事说了一通,

王正低下头去,一幅苦瓜脸。他都已经是宁国节度使,诸多节度使职也是位居前列,转头就要拥着娇妻美妾和千万家赀养老去了,何曾想就被拖进这场漩涡?

论功劳,他不仅仅是征战四方皆有胜绩,居也有定策之勋,而且还是忠心耿耿的老臣,宫变始终尽其忠节。朝廷的那些士大夫,绝不敢以阉人相视。即使回家养老,朝廷都得跟王安石、韩绛那等元老一样,倍加优遇,甚至再加一两个节度使衔,成为两镇、三镇的节度使,都绝不是不可想象的一件事。

不论是哪家在朝堂上,太后、天谁人掌大政,他王正都会有一个好结果。国史之,宦者传上,也少不了他的一篇本传。可是即将脱身的现在,太后竟然想将他给拖进漩涡里。到时候,天亲政,将这一件旧事给翻出来,他王正少说也要给扒下一层皮来。

只是事已至此,王正也只能老老实实认命,问道:“不知陛下打算如何处置福宁宫人?”

“跟着官家的宫女,尽数禁于玉阳院,等过几个月再做考量。”

但福宁殿可不仅仅是宫女,王正随即问道:“内侍呢?”

“……”太后想了一阵,“去做杂役吧,死罪可饶,活罪难逃。”

王正的头都痛了起来,太后的处置实在是想当然了。

以他多年的经验来看,对付没有尽到自身任务的伙计,杖毙跟赐死,那是最稳妥的。

尽管这么一来,勾决的名单又要加长了,但这是为了未来而做的准备。否则等赵煦登基之后,肯定会将这一批人都召回去,今日与事的所有人,全都得到大霉。

要么什么都不做,要么就把事情做绝。

这是王正的看法,可是太后到现在为止,还没做出过如此的决断。

王正对此也无可奈何,难道要他手把手的去交太后怎么做事?即是一时成功了,莫说天会记恨于心——深知不知有多少人在嫉妒他自己,王正不觉得这件事能瞒住天——就是太后本人,事后回过神来,也不会喜欢一名家奴,对主人家的事指手画脚。

“陛下,福宁宫,还是尽快换上一批老成持重的宫人,侍候天。”王正挑着不疼不痒的建议说着。

“吾也是这么想的,官家还年轻,必须要”

“然后请陛下告知两府。”

向太后皱起眉,“此乃家事。”

“陛下,自古便有人说,天家无私亲,天之事,皆是公事,韩相公也说过这样的话。”

“那就请韩相公,”向太后一口答应,瞟了转过脸去的王正一眼,她又低声吩咐,“先只请韩相公。”

韩家的休息时光,被匆匆而来的令使给打破了。

韩冈听了令使报告之后,便命家妻妾去准备入宫的公服。

王旖取了件衣服,匆匆来到韩冈边,“到底出了什么事?这么急匆匆的招官家入宫。”

韩冈看了看身边一脸稚气的二儿,无奈的叹了一声,“天长大成人了。”
