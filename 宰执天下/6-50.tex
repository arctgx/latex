\section{第七章 烟霞随步正登览(二)}

时隔数载,范纯仁重新踏进南薰门。

超越天下任何一座城市的富丽繁华,让范纯仁身边随行的子侄和家人都忍不住让目光流连在街道两边,只有马背上的范纯仁,目不斜视的望着前方。

不过他的心思却完全没有放在数里外御道尽头处,那座高高耸立的城门上。

早在五天前,离京冇城尚有三百余里,范纯仁就听到了一个消息。

如果耽搁一日,就没他的什么事了。范仲淹的次子却是赶在廷推的前一天进入了东京冇城。

虽然说前些年在庆州时为种诂所讼,被贬黜信阳军,但很快就被重新启用。尽管齐州知州的地位不高,但身为宝文阁侍制的他绝对是有资格参与到推举宰辅之中去。

一旦他在宣德门登记了自己的姓名,等待入宫面君,那么明日的朔日朝会,就有资格参与进去。理所当然,大宋首次推举宰辅的会议,没有人能够将范纯仁排除出去。

只是如此推举之法,史籍不载,到底是参与,还是表示反对,范纯仁现在还无法作出决定。

……………………

“范尧夫?!”

正往国子监去的叶祖洽突然勒住了缰绳,望着迎面而来的一队人马,仔细辨认了几眼:“果然是范尧夫。”

与其同行的丁执礼吓了一跳,抬头望着:“范纯仁?!他怎么回来了?!”

朝堂之中,范纯仁也算是有名人物。在朝野内外资历声望都不低,而且是铁了心、死不悔改的旧党。

“当然是诣阙。”

“他是侍制吧?”丁执礼问道。

“宝文阁侍制。”

“这半月回京的侍制里面没有他啊。”

能参加廷推的人选名单早就在京冇城传遍了,计算行程能在选举之日前抵达京冇城的几名诣阙侍制,也都在名单之中,这里面可没有范纯仁的名字。丁执礼记得至少还要两三天的时间,肯定是在廷推之后。

“也许是走得快,大概是听到消息了。”叶祖洽摇头,“不过可说不准他会参加廷推,还是干脆一顿大骂……这也算是变法了。”

“……听说范尧夫性子刚硬?”

“忠直嘛……听说范文正自己都说纯仁得其忠。忠心事主,无暇谋身,所以看不顺眼就要说出来。”

熙宁三年的状元郎口气中有着掩不住的讽刺。

“纯仁得其忠……那范五呢?”

“纯粹得其略,所以才能就任并州。只是现如今太原可不需要谋略之士,是要休养生息。”

范仲淹有四子成人,范纯佑,范纯仁,范纯礼,范纯粹冇。

三十年前范仲淹守关西,范纯佑便是其助手,不过后来得病,早早病亡。剩下的三子之中,范仲淹曾经评价道,纯仁得其忠,纯礼得其静,纯粹得其略——也就是谋略。范纯粹现在河东,新进的知太原府,是韩冈离开河东后才走马上任。

“范文正公的谋略也算不上多出众,得其传承,最多也就是勉强谨守门户。”丁执礼又在望着越来越近的范纯仁一行:“不过范尧夫他可真是心急啊。”

叶祖洽冷哼了一声:“多他一人不多,少他一人不少。就算他有心,也改变不了什么。”

从现在流出的消息上看,韩冈能得到的支持可是少得可怜。

一方面,比起这几日频频交接群臣的李定、吕嘉问等人,韩冈完全没有动作。但另一方面,也是韩冈太过出色,以至于其余大臣不约而同的对他进行压制。

在可以选择的条件下,如韩冈这样太过于突出的同僚,没人愿意他进入两府。如果是太后来决定,那谁都没办法,可现在决定权落在了侍制以上的重臣们手中,哪里可能会推举韩冈再入两府?

重臣们尽管不清楚韩冈入两府之后会做些什么,但他至少知道什么叫做生老病死苦?

熙宁初年,王安石第一次进入政冇府,区区一介参知政事,挤得其他宰辅没有立足之地,老的老、病的病,无能的在叫苦,心眼小点的干脆就气死了,只有王安石生气勃勃。

韩冈当初第一次就任枢密副使,是因为北疆不稳,而且任期内他几乎都不在京冇城中,而是在北面主持军务。等到回京,没多久就因为误诊先帝之病而请辞。没有多少时间让人感受到他的威风。

但这一回,可不会有辽人入寇的意外了。如果进入两府中,少说也能坐上三五年。而太后又对他言听计从,如此一来,就是当年王安石的翻版,其他宰辅还有立足的余地?而韩冈为了巩固自己的地位也会大力提拔自己的部属,从而控制朝堂。眼下各位占据了重要职位的重臣们,一两年后,能剩下一半就不错了。

从宰辅到朝臣,只要不是韩冈一系,眼下都是有志一同。有消息说,参加选举的侍制们会尽量将韩冈压在第四名。

要么就是太后否决掉这次选举的结果,让提议的韩冈丢尽脸,无颜入两府,要么就是太后承认现实,放弃韩冈,从中选的三人中选取一名提拔入两府之中。

不论是丁执礼,还是叶祖洽,两人都参与过熙宁六年礼部试的阅卷,当年韩冈就是在他们手上中了进士,当年还没有进士便已经是朝官的韩冈,现在更是远远的超过了他们。所以他们私心里也想看见韩冈再吃一个亏。

“嗯?那是哪一家的?”

丁执礼突然眯起了眼睛,只见不远处,一人突然从街边的酒店中出来,拦住了范纯仁一行。

“似乎是就是在这里守着范尧夫的。”叶祖洽亦凝神细看。

“看装束不像是东京冇城这边流行的打扮。”

“嗯。倒像是西面土包子,不过又不像是关西。那边可真是不会裁剪,白白浪费了好布料。”

“莫不会是西京?!”

“文、富会支持韩冈?”

叶祖洽和丁执礼对视一眼,同时大笑起来,那怎么可能?!就是韩冈是北人,但他也是王安石的女婿啊。

……………………

“景贤拜见侍制!”

郑国公富弼的侄孙在范仲淹的儿子面前恭谨行礼。

范纯仁对待富景贤仿佛是自家的子侄,“好些年不见,贤侄都这么大了。”

“已经六年了。景贤还记得当初随三叔出东水门送侍制南下的事。”富景贤说着,从怀里取出一封信,双手递了上去,“这是家叔祖命景贤给侍制送来的信。”

范纯仁笑着点头,接过信,又命人空出一匹马来,让富景贤上马。富家人,就是他的子侄一般,一点也不会觉得生疏。

庆历之时,富弼与范仲淹是最紧密的政治盟友,一在东府、一在西府,共同推行新政。

与那个专门坑队友的欧阳修不同,富弼在很长一段时间中,一直被范仲淹连累。其出使辽国时,所携国书都被人篡改,日后其首次自两府出知地方,也是因为跟随范仲淹。而之后,范仲淹病逝,他的墓志铭也是富弼主笔冇,不擅诗赋的富弼还写了一篇吊祭范仲淹的祭文。而且范纯仁早亡的长兄范纯佑的墓志铭,也是富弼亲笔撰写。

相对于一直往来不绝的富弼,因为欧阳修在范仲淹神道碑上所撰写的范仲淹与吕夷简同时复起之后“二公欢然相约,共力国事’的那一段,倒是很早就疏远了——范纯仁认为自己的父亲自始至终与吕夷简未曾和解,便将那一段给删去,欧阳修却说‘此事所目击,公等少年,何从知之?’由此而疏离。

另一方面,富弼当年科举不第,转头却得以去考制科,最后制科得中便是范仲淹举荐之功,且富弼能做晏殊的女婿,也是因为范仲淹在晏殊面前的大力推荐。

富弼在《范文正公仲淹墓志铭》中所写的两句‘师友僚类,殆三十年’,便是两人情谊的最好总结。

信上别无他语,只是普通的问候。范纯仁与富弼,以及富家的子弟常年鸿信往来,逢年过节都要致书问候,今日信中的内容与平日别无二致。但隔了数百里,特地派了侄孙来送信,说是普普通通的问候,也要人信才是。

范纯仁将信纸折好放回信封,然后命左右离开一点,直接问:“郑公有何吩咐?”

虽然在范纯仁面前侃侃而谈,但富景贤还是有些紧张,范富两家的关系虽不必多说,但范纯仁从来都不是因私情而废公事的人。

旧年王安石入政事堂推行新法,宰相富弼阻拦不得遂告病回乡,范纯仁便上本指责富弼是‘恤己深于恤物,忧疾过于忧邦’——怜恤自己比怜恤外事更深,忧虑自己的病情超过忧虑国家——所以是‘致主处身,二者皆失’为君主效力和为自己安身立命,二方面都有过错。

“……不知侍制可曾听说推举宰辅一事?”

“自然。”范纯仁点头,但随即皱起眉,“不过依行程,纯仁可是要在朝会之后入京,在给郑公的信上也是这么写的。郑公如何会遣贤侄来此处侯纯仁。”

“景贤离家前叔祖有言,侍制一向忠于王事,上京必然兼程,只要在南薰门内守着就好。”

“……知我者郑公。”范纯仁眼神闪动了一下,叹了一声,“郑公如何说?”

