\section{第26章 任官古渡西(六)}

一般来说,能作为祭田,用来奉养祖先坟茔和宗祠的田地,都不可能太差,而且京畿一带的地价绝不便宜。韩冈年初时欲在京城买房,顺道问过开封周边的田价。普普通通的旱地都是十贯往上——这还是出产不丰、位置偏僻的下田。如若是靠近村庄、道路的上等良田,那价格更是要翻番了。当时韩冈打听过了开封府的田价和房价后,便收起了在京城置房置产的心思,老老实实的租了一间靠河的院子。

白马县这边,虽说离着京城稍远,但还是属于津梁重镇,现在又成为了畿县,地价不会比开封府周边低到哪里去。两百一十五亩田,韩冈估计着至少也有两千贯。

“这祭田分作几片?”韩冈追问道。

胡二越发的惊讶,韩冈的每一句都问到关键上,很少有官员会对田宅买卖的如此了解。低头答话:“都在一处。就在清水沟边,是一整片水浇地……”

那就更贵了!

宋朝不抑兼并,田宅买卖频繁,有‘千年田换八百主’的说法。土地易手频率如此之高,许多时候,经常能看到将一片田七零八落的分卖出去。一顷的整片土地,几次转手之后就会变得支离破碎,属于几十户不同的人家。

大户人家的田产也都是东一块、西一块,甚至分散到不同的州县中。这样的情况下,越是完整的田地,卖的价格就会越高。而有些田主,为了能让自己家中的田地合并在一处,都是大费周折,陷人死地的情况也不是没有。当初李癞子要强买韩家的三亩菜园,便是因为那三亩地,正好可以让李家的河滩田连成一片。

如今次这样两顷多的一整片地,而且还是灌溉设施良好的上等田,那三千贯那是没得跑了。

韩冈摇了摇头,一片价值三千贯的田地,难怪能打上三十年的官司。

“旧时的田籍,还有当年能作证的老人,难道都没有了?”韩冈继续问着。

“回正言的话。当年黄河决口,从东京一直淹到滑州。白马县的人不是死在洪水里,就是阖家一起逃难。等到水退归乡,回来的也不剩多少。加之第二年县中的田籍簿册因为县衙走了水,全都烧了个干净……”

听到这里,站在一边旁听的方兴就一声嗤笑,“这买卖做得漂亮!”

韩冈也是眼神变得冷了起来。这一干胥吏做得也太绝了,一下就让他想起了当年的陈举。一把火烧掉了所有的存档,几乎就是死无对证了。

此时的契约分为白契和红契两种。过户时在官府中登记缴税并盖了印后的田契称为红契。不经过官府,只是买卖双方私下里过户的田契,则称为白契——因为没有朱色官印的缘故。按照律条规定,田宅成交后,不及时去官府申报缴税,被查实后是要受到处罚的。但罚不责众,真正照着律条处罚的情况,其实极少见。

另外打官司时,两种田契都是合法的,都可以用来作为证据。而且当红契与白契相冲的时候,照律条来说,是该以红契为准,但官员们断案,往往都是以时间靠后的为准,并不注意是否经过官府。

所以烧掉了田契和丁产簿后,因为水患的缘故而没有了户主的田地,只要随便拿出一张白契,就能将合理合法的吞下。除非有人叫真,去开封三司里的户部司,将县中上缴的田籍和丁产簿给翻出来,否则这份田就占定了。如果再交上一份税金,将大印盖上,基本上这个案子就翻不回来了。

“何家本来就不是大族,只有三房而已,一次洪水之后,几乎都不在了,只有何允文过了两年才回来。虽然手上没有地契,因为墓碑还有界碑上都留有田主姓名,加之何允文手上有系谱,又找了两个证人,便把这片田判给了他。后来又盖了印,将这份田契在田籍簿中给登记上了。”

“此中必然有情弊!”方兴低低评了一句。

“那是自然。”韩冈冷笑一声。证人好找,衙门难缠,这等不靠谱的证据,不知何允文花了多少钱才让田产给认定下来。

示意胡二继续说下去:“又过了三年,原告的何阗迁回本县。他回来后,就递了状子声称墓中的何双垣是他的祖父,要夺回这份田产。”

“他有什么证据?”韩冈问道。

“没有!没有田契,只有族中谱系。”胡二摇头,“两人身上虽说都没有地契,只有族中谱系,但何允文有证人啊!所以第一次判案就已经断了何阗输。”

“那这个案子怎么几经反复,整整拖了三十年?!”

“麻烦就麻烦在这里。证人虽然帮着何允文,但何允文家富裕,而何阗贫寒。谁都知道,这证人是怎么回事。”胡二叹道,“不过何阗是读书人,平时也作一些诗文,跟着一帮士子交好,帮他说话的有很多。所以重新递了状子到了州中,便发下来重判,这下子,结果就反了过来。只是但何阗毕竟没有证据,所以等到原任知州离任后,何允文重新递了状子,这坟和田又断回给他。”

方兴听着连连摇头,久讼不决乃虽是常见,但这个案子,能来回多少次,也的确是个奇葩了。

“刚种了一年地,输的一方再来打官司,结果又是反过来。为了这片田地,十几年中来回反复了三四次,县里闹过,州里也闹过,最后甚至闹到转运司和提点刑狱司。但两个衙门判出来的结果还不一样,之间又变成一番笔墨官司。现如今,当年作证的几个证人在十年前就已经死光了,从那时开始,这个案子就再也没判过,就是一任任的给拖下来,田也是给荒着。”

“原来如此。”

前面看过了状纸,现在又听着胡二的一番叙述,韩冈对于这个案子大体就有数了。

的确不好判!

官司打了三十年,水患还要在往前上溯五年。当初能出来作证的老人,早就死得一干二净。现在能拖出来作证的,当年也不过十几岁二十岁的样子,说出来的话,根本无法让人信服。原告何阗和被告何允文还活着,也都六十七十了,不可能给他们用刑来求个实证。

也难怪历任的白马知县都拖着,没有人证物证,要想让人心服口服,让原告和被告都不再上诉,难度可想而知。

这个时代可没有终审定案的说法,只要不肯认下判案的结果,就可以继续上诉。县里不行去州里,州里不行去路中,路中不行,还有东京城里的登闻鼓。而且官员流动得又快,前一任判下的案子,下一任也许会顾及前任脸面,不去改判,但也有可能会重新审理一番。韩冈可不想丢脸,让后来人耻笑。

方兴紧锁着眉头,他在旁边听了也头疼,根本断不清的案子。他上前半步,正想提醒韩冈不要贸然接下,就听着韩冈吩咐胡二道:“明天开审此案。你去通知何阗和何允文二人,本官要先看看人,将事情问个明白再说!”

胡二闻言便是一愣,明明都跟这位年轻的知县说了,这个案子没法儿断,怎么还不知道好歹。但他立刻低头应诺,一点也不拖延。心里则是在想着,吃点苦头也好,这样才会信重自己。

胡二离开,韩冈回到后厅。连同听到消息的魏平真和游醇也赶了过来,韩冈挑了陈年旧案作为他到白马县的第一个案子,作为幕僚都不可能坐得住。

就见方兴急着满头汗:“正言,怎么能这么快就开审?!”

韩冈慢悠悠的不在意,吩咐着下面的侍从端茶上来,“这个案子很难吗?”

“所有的田籍都是这些年新造的,追溯到最早也就三十三年。证人也几乎都死光了。什么凭证都没有,谁能断得了?而且当年又不是没断过,还不是给翻案了?日后再给翻案,可是要受罚的!”方兴提醒着韩冈。

韩冈满不在意的笑道:“不过是依律罚铜而已。又不是失人入死。家产析断的诉讼,错了也只是赎铜七斤。”

“还有展磨勘啊!”

就跟记过一样,赎铜罚俸不仅仅是罚钱的问题,随之而来的还有展磨勘的处罚。原本定例的三年磨勘,要拖到四年、五年才能迁官。对于减一年磨勘,‘杀人亦可为之’的官僚们,这等于是要了他们的命。

“不用担心。”魏平真拦着还要说话的方兴,他虽然还不清楚此案的内情,但看着韩冈的模样就知道可以安心了,“正言可是胸有成竹了。”

韩冈冲着惊讶的望过来的方兴和游醇微微一笑,“不用担心,这案子我还是能断的……”顿了一顿,韩冈神色变得严肃起来,“九月开犁。麦子种下去了有近一个月了,但缺水灌溉,出苗的情况并不好。而且还要防着明年的灾情,不能多费时间纠结在这等争产的案子上,要速战速决!”

韩冈上任的时间不巧,正好是秋播后最忙的时候。作为知县,他不能安坐在县衙中,必须去乡中查探灾情。什么事都不干的官员,官场上也是有的,但他们很快就会被上司、御史或是走马承受给弹劾,除非有文彦博那等资望,才能让天子反过来将弹劾者调离。

知县、知州之所以被称为亲民官,就是他们要直接面对百姓,一州、一县的生产生活都在他们的控制之下,与千万百姓息息相关。比起那些幕职官、监司官来,身上肩负的责任要重得多。

韩冈自知身上重任,所以现在要做的,就是立威。通过一桩桩公明方正的断案,在白马县,立下说一不二的声威!

