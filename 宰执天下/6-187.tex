\section{第21章 欲寻佳木归圣众(七)}

“相公回来了?”

周南停下了手中的针线活,问着过来报信的使女。

正苦着脸坐在一旁一起绣花的金娘,立刻支棱起了耳朵。

“金娘,手别停。”周南脸都没回,就知道了自家女儿什么状况。

“相公刚回了,刚换了衣服,正在外间面客。”使女低声禀报着。

周南点了点头。

韩冈不好声色,也很少参加私人宴请,放衙之后,除了家里,其他地方找不到他的人。

但他每天回到家中时,都少不了继续处置公务。

宰辅家的门槛一向吸引人。每天想要拜见宰辅的官员等各色人等,都在递了名剌后,在门房里坐着。就是明知主人不会接见,也会坐满一个时辰再走。不仅仅是京城的宰辅家如此,从京师的高官显宦,到地方上的官员,权势强一点,皆是一般。

韩冈晋身两府之后,觉得这么多人的车马堵在家门口,让他回家都不方便,便改了这个规矩。

每日固定十人,派发号牌,先到者先得。剩下的再从前一天递上来的名帖中挑选十人,会派人按照留下的地址上门去通知。

这样就免得上京的官员们耽误时间,也减少了家门前交通堵塞的情况。若是亲信和其他重要的官员,自有其他渠道进入府中,这就不必多说。

“大概还有一个时辰的样子,你去跟严姐姐说一下吧。”

在政事堂中,下面的官员谒见,大多数三五句话就打发了,回到家中,说话的时间就能延长一点,但终究也不会太长。而且韩冈不喜欢晚上花费太多时间在会客上,总是在饭点前会见客人——对外则是声称不想耽误客人吃饭。一般来说,高官家的门房不会提供饮食,晚上谒见主家的客人,如果准备不足,很多都是饥肠辘辘。

使女出去了,周南回头就看见女儿嘟着嘴,低头绣着绷子上的绣品,只是手劲稍大了点,准头也差了些,一根绣花针上下翻飞了几次,竹篾编的绷子竟一下子断了。

周南叹了口气:“都这么大了,还当自己是小孩子吗?哪家的姐儿到了你这年纪,不在家里苦练女红的?”

“爹爹说了,过得去就行。”

女儿小声的嘟囔,也没逃过周南的耳朵。

一对纤长的柳眉先是高高挑起,然后便又无奈落下。搂着女儿的肩膀,周南轻声道:“金娘,你爹是男人,女儿家的事他不懂。你爹与王家大郎他爹,恩若骨肉。大郎他娘也是个和气的人,你嫁过去,不必担心受多少刁难。可如今天家的女儿出嫁后都要受气,宰相家的女儿又何能例外?王家又是大族,日后出嫁少不了被人挑剔的。德言容功这四项,金娘你若是做不好,娘家丢脸没什么,你爹也不在乎,但你在夫家,还怎么过得好去?”

小时候就活泼爱闹,长大了更是变得倔强,拧起来周南都压不住。韩冈不在乎女儿闹些小脾气,还笑说是这倔脾气从周南身上传下来的。周南每每气得没办法。不过现在她也知道怎么对付女儿了,耐下性子来讲道理反而管用。

……………………

韩冈回到后院的时候,只有周南迎了上来,“官人回来了!”

“你姐姐她们呢?”

“正在后院置办乞巧的什物,已经让人去通传了。”周南手脚麻利帮着韩冈脱下了见客的外袍,递上一块冰镇过的湿帕子让他擦脸,“官人今儿怎么这么迟?”

韩冈回家后会客的时间一般都是固定的,今天却比往日多花了半个多时辰。

用冰手巾擦过脸后,顿时一身的清爽。听见周南问起,韩冈从身后的使女手中拿过一卷纸,递给周南,神秘的笑道:“你看看这幅画。”

周南疑惑的打开来,却是一人的绘像,但这幅画,与常见的画有着截然不同的观感。

周南惊讶的张大了眼睛。

白色的纸面上,用黑色的炭笔画上一名女子的半身像。

这个时代的绘画风格,正处在一个剧烈的动荡期。原本仅仅是为了随时绘制地图才出现于世的炭笔,如今则成为天下画家都少不了的工具,打草稿少不了,出外速写风景、人物也都比毛笔更合适。

由于炭笔的使用越来越多,纯粹的炭笔画也多了起来,韩冈将之命名为素描。素描的画面,由于有浓淡明暗之分,加上视觉上的透视效果,往往比旧时的工笔白描更显逼真,但如此栩栩如生的绘像她还是第一次看见。

“真的好像!”周南惊讶的说道。

其实还差点,韩冈心道,但以这个时代的的眼光来看,绝对是超乎想象了。

“是李公麟所作。”韩冈道。

“李伯时?”

“嗯,国子博士李伯时。”韩冈笑着说道。李公麟的这个表字起得好,还没做博士的时候就有人喊他博士了,现在做了博士,就更加名副其实。

周南惊讶的再看了一下画面,摇头不信:“要说是他人之作,奴家倒是信了。但这分明不是李伯时的手笔,差的太远了。”

“是吗?”韩冈皱眉看了一阵,亦摇头道,“这是李公麟亲手拿过来的,他当不会夺人之名。”

工于作画的李公麟,其名气在京中远比他国子博士、中书编排官的官位要强,本身又是进士,所以在京城士林中很是受到尊重。不过李公麟不喜与高官显宦结交,周围的朋友都是一般的骚人墨客。

“可是……”周南仍是一幅难以置信的表情。

韩冈家里有一副李公麟的画作,是一名马童牵着一匹意气风发的赛马的绘像。韩冈意外得到,给喜好绘画的周南收藏了起来。

带着金牌和大红缎带的冠军马,那神采飞扬的模样,还有身上一块块浮凸的肌肉,仿佛跃然纸上,而前面的牵马人,探前的左手仿佛要摒开热情的众人,右手则紧紧攥着缰绳,将马童在夺冠后,对冠军马的重视展现得淋漓尽致。

但那副骐骥夺冠图,远不如眼前的这一幅绘像精致。仅仅两尺见方的绘图上,人脸占了大半,人物的表情栩栩如生,甚至脸上的一沟一壑,都能分辨得出来。

这用笔的作风,完全不是一个人了。

“李公麟在京师这些年,也没听说他来拜访过官人,怎么今天上门来了?”

作为韩冈的下属,几年来,李公麟可从来没有登门造访过一次,突然造访,周南觉得总不会是心血来潮。

“为了驸马都尉王诜啊。”韩冈道:“他与王驸马是好友,如今齐鲁大长公主重病,若有个万一,太后岂能饶得了他事主无状之罪?”

齐鲁大长公主是英宗与高太皇太后之女,也是先帝仅存的妹妹。因为太皇太后的事,向太后对这位小姑子只会更好,甚至热情过了度。日常封赠远超应有的水准——只看封号便可知一二——唯一的儿子前一日更是刚封了团练使,说是为了给大长公主冲喜。

而驸马王诜与大长公主的关系,是有名的恶劣,若是大长公主不治,王诜自然不会有好果子吃。

韩冈将画摊平在桌上,“这幅画就是他拿来讨好为夫的。”

“有什么特别之处?”

周南素知自家的丈夫对琴棋书画无一所好,诗词歌赋同样是毫无兴趣,李公麟如果只是拿着一幅好画来,不至于耽搁韩冈这么长的时间。

“你可知道这幅画是怎么画出来的?”

周南仔细的看着这幅画,发现连光线从哪里照上人脸,都能从画中看出来,其精细可知一二。她一向工于画技,但对此却是如同。

摇摇头,她期待的看着韩冈。

“是通过暗室画出来的。将人像通过几组安装好的镜子和透镜投影到暗室之中,直接描画投影,不仅仅人物逼真,连光影效果也更为切合现实。”

因为韩冈很早之前,便将投影、透视等仅了解皮毛的绘画名词,公然的登上了《自然》。尽管说出来的东西十分粗浅,但这就是戳破了一层窗户纸,让一干天赋杰出的绘画大家找到了进步的方向。

周南腾地一下就跳了起来,急急的问着:“官人可知那暗室是怎么造的?”

“当然,不过为夫不会说。”韩冈吊着胃口,“看下下一期的《自然》吧。我还希望李公麟,能画一些带色彩的画,试制各色颜料,什么都尝试一下。”

西方的油画家,很多为了寻找更好的颜料或是溶剂,都精研过化学和矿物学。如果仅仅对纸墨笔砚研究透彻,那对科学发展的贡献就太少了。李公麟若是能多研究一下颜料,绝对是一件好事。

“官人……”周南抱着韩冈的手臂,娇声叫着,一下子好像回到了过去,满身成熟韵味都换成了少女时代的娇憨。

“自己对照着文章试验才有趣,现在说破了可就没意思了。”韩冈眯起眼睛,享受着手腕中那动人的触感,却丝毫不为所动。

“可是《自然》里面,多少文章奴家都看不懂。”

“太后都能看懂,南娘你怎么会看不懂?”

向太后与许多闺秀一样,文化素养并不高,识字而已,远比不上周南这种能与士大夫唱和的花魁——相对而言,王旖就是一个异数了。

周南一下甩开了韩冈的手臂,冷了下来,“是啊,太后能看懂,我们看不懂。”

韩冈轻拥着爱妾,“闹什么脾气,太后看懂的也就是那几篇简单的养生文章。”

《自然》一刊,已经成了天下最受欢迎的读物,朝野内外,不知多少人都在订阅这一期刊,里面的内容也被许多人奉为圭臬。

据韩冈所知,宫中也是大客户,太后更是一期不落,不过她主要也就看一看医药和养生方面的文章。而她看过之后,却都会遵循文章来行事,将宫中的多年俗例丢到一旁。

譬如蜂王浆,出自几年前的一片说蜜蜂内部社会的观察论文。蜂的分工说了,蜂巢中的产物也说了。

工蜂、蜂王之类的虫豸之事,知道了也就是个乐子,也就文人在文章中又多了一个能比喻、借喻的东西。在民间,则是养蜂的手法有了进步——有了后世通行的蜂箱,取蜜的手段也不再是直接割走蜂巢。而在宫中,则是日常的补药都因此而变。

蜂王浆和蜂胶成了贡品,也有臣子得此为赏赐。

韩冈就受赐过几次蜂王浆,还有过蜂胶。韩冈的父母,也常年服用蜂王浆、蜂胶和蜂蜜——陇右那广阔的油菜田,让韩家每年都有大量的蜂蜜出产——据信中说,身体好像越发的康健了。

揽着爱妾的娇躯,韩冈再次低头看着桌上的人像素描,不由自许而笑,这个时代,已经不可能再回到过去了。
