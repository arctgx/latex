\section{第36章 骎骎载骤探寒温(五)}

润州夜战,官军一战而胜。逆乱润州的明教妖贼旋起旋灭,近两千贼众授首,而官军损伤仅仅八人。

自号圣公的妖贼渠首——卫康的首级,也在润州夜战的三日后,连同他的两个儿子一个侄子的脑袋,被一位保正一并送到了州衙中。他们是在化妆逃窜的过程中被村人发现,然后被当地的保正率众击毙。

在卫康之后又陆陆续续的又明教贼众自行归案,或是被械送官府,待五六日后,已经没有几份相应的报告了。

至此,方可说此役已是大获全胜。

明教妖贼起事不过两日,肆虐范围也仅仅是丹徒一县,但县中伤亡不可胜计。数以千计的乡民被劫掠、被裹挟。战乱至后,丹徒县中门前挂上白布幡的家庭,十之七八。

除此之外,财产损失也极为惊人。之前丝厂被烧的尤、陆两家,这一回更是满门被烧杀一空。其余大户,除了一个以乐善好施闻名乡里的李家被贼人放过,只要处在乱贼经过的路径上,没有一家能逃过一劫。

丹徒县内的十余家生产丝织、陶瓷、玻璃的工厂,皆毁于一旦。甚至那些只雇佣三五人,仅仅为同村村民服务的油坊、磨坊,也全都被乱贼捣毁。

如此惨烈的伤亡,如此巨大的损失,责任自然是落在知州杨绘的头上。而立下平乱之功的景诚,不管此番变乱他之前要付多少责任,如今有军功在手,又有铁打的靠山,已经被视为即将飞黄腾达的热门马。

因此即使就在平乱后的第二天,杨绘从州衙后院中走出来,试图亡羊补牢,挽回一些局面,也被景诚连同州中官员一起顶了回去。可想而知,州中的官员会将多少责任推到杨绘的身上。

接下来的五天里,景诚忙碌于抚恤百姓,计点伤亡和损失,宗泽则等到了泗州来的援军。他们将会暂驻在润州,宗泽也会留居几日,等待朝廷那边新的命令。

从事后对俘虏的审问中,宗泽和景诚,自卫康的角度,了解到了这一次妖贼作乱的来龙去脉。

看过审问的报告后,宗泽忍不住苦笑出声。他实在是想得太多了。料敌从宽,这话是没错,但是宽,也是得有界限的。

卫康最早的计划,并不是谋反,而是准备集合润州的教众,收拾家当逃离润州,前往浙西山区暂避风头。那边才是明教传播最广、信众最多的区域。山谷之间的穷乡僻壤,也是朝廷管辖不到的地方。

若不是州中派了丹徒县尉去抓他,卫康在次日夜里就要动身上路了。而所谓的伏击,不过是听到州城信徒的走报,仓促间率人躲到庄子附近的桑园中。只是看到县中人马毫无防备的走过去,发现有了机会,才临时起意从后袭击。

在轻松拿下了丹徒县尉,感受到官军的无能之后,卫康的目标终于变了。变成了扩大声势,吸引更多的明教教众一同起事,而不是丧家犬一般的逃到浙西——尽管卫康还是打算去浙西,但他打算尽量带更多的部众走,这有助于他在浙西的同伴那里维持自己的地位。

因此,他蛊惑了一干信众,席卷丹徒县的各个乡村,裹挟了大批百姓。当他手下的人众超过两千之后,他又有了攻打润州州城,博取更大声名,搜罗更多财货的想法。

之所以留下城池北面不攻,是有人给卫康出的主意,想的是大张声势,围三缺一,放出一条生路,使城中人心难以固守——这是说书中经常出现的计策——而后此人便被卫康封为军师,如今也成了官军的斩首功之一,在一堆头颅中也分不清谁是谁了。

但卫康和一干反贼的眼界,还没有扩大到润州城之外。因为担心京口方向上的援军,在来路上放了哨探,却没想过去伏击。

贼人终究还是不敢跟禁军为敌。毕竟官军的战斗力,这些年在四方小国身上得到了无数证明,越发的被世人所熟知。

卫康熟悉州县中的弓手、土兵,也知道润州城中的兵力,但他对禁军却完全不熟悉,更不会清楚官府内部调兵的流程,并不清楚驻泊润州的禁军绝不会轻易出援,周围军州的禁军也不会那么快出动。

同时卫康没有认为自己能够顺利攻下润州城,他想的是如果不能在短时间内拿下润州城,或是官军援军赶到的话,就依照原计划撤往浙西山区。充裕的兵力,可以不用投靠同教中人,而是直接鸠占鹊巢。

故而当夜润州城外,卫康就是驻扎在最易撤离、距离京口也远的西南方,而不是在宗泽所猜测的北面做伏兵。他所派出的劝降使节,便是从南面而来,要不是景诚被宗泽的判断带偏,当时就能判断出卫康主力所在的位置。

从头到尾,卫康都只是兵学上的外行人。但凡揭竿而起的贼寇,要么吸纳掌握知识的士人,要么经过多年阵上搏杀,否则永远成不了气候。

这一回八名西北出身的老兵,带着一百多壮勇,夜袭贼人营地,轻而易举的就造成了极大的混乱,冲散了卫康的营地。当城中主力出阵,就彻底奠定了胜局。如果只看战果,这是一场八比千八的大捷。

这一过程中,之前击败丹徒县尉的十几甲士,完全没有起到任何作用。战后的搜检,那十几具铁甲也都先后被缴获。

所谓的铁甲,只是民间铁匠打造的铁板,带了点弧度,前后各安一块,用皮索一系,勉强能说是胸甲。当这种‘铁甲’让勇武有力之人穿戴上之后,区区土兵、弓手的确是抵挡不了。

可比起正牌的铁甲来,却完全不在一个档次上。尽管只是用州中武库中的库存货装备起来,出战的八名老卒足以轻易挑翻那十几名甲士。

不管怎么说,这批铁甲就是卫康蓄谋已久的最好证明。什么官逼民反,什么丝厂害民,都是污蔑之词。十几副铁甲一摆,什么话都不必说,这就是最好的解释。

从丝厂被烧开始,一切都变乱都是明教所为。之后一段时间,所有对工厂的攻击,都可以说成是明教党羽所谓。

以卫康亲信为主的口供,在细节上,还是有些问题。

比如遣人焚烧丝厂的真凶,被说成是一个信教成疯的疯子,想要多收信徒所以煽动了工人去烧了工厂。这很难让人相信。

再比如卫康围困润州时所做出的选择,不论是让宗泽来看,还是让景诚来看,都是蠢到家了。外行人的想法,在内行眼中,很多都是天马行空,让人无法琢磨的。这种自作聪明的犯蠢,即使是专家,也根本捉摸不透。只是完全归咎于卫康在兵法上的外行,还是有些说不通的地方。

不过这些口供来自于卫康的亲信,以及一干附贼的党羽,但毕竟不是兄弟子侄这样的血亲,更不是卫康本人,有些问题是肯定的。

因而又经过一番谆谆劝导,景诚和宗泽才得到了他们想要的口供——有些事可以直接报上去,有些事就得打个埋伏。

就像卫康的铁甲,不过是为了与邻村争水而做得准备,两块铁板拼起来就是铁甲,分开来则可用来摊饼,只是外形别扭点。真要下去细搜,家里存着类似器物的绝不止卫康一家。但这样的事要是传出去,又会引起一番轩然大波。还不如就这么压下去,然后在州县中多宣传宣传私藏铁甲究竟会有什么样的法度。不然这份功劳,不知要给打几成的折。

还有卫康伏击丹徒县尉的事,照实说,也远不如将卫康说得更加狡猾狠厉的好,将贼人说得太胆怯,于丹徒县尉的名声有损,说得强一些,这样对战殁的丹徒县尉也是一个安慰。

又用了两日,待景诚将他的那份名为请罪实则表功的奏章写好,宗泽也将他的奏疏整理完毕。两份奏章中的内容经过很好的协调,重要的关节都可以相互映证,细节上有些参差,乃是必不可少的伪装。

不过在宗泽给韩冈写的密信中,倒是一点没有隐瞒,原原本本的将整件事说了一遍。

给朝廷的奏章送出去,景诚和宗泽终于可以松了一口气。

总算是结束了。

尽管还有许多善后事务要处理,但提供给朝廷那边的材料,足以给此番谋逆大案下定论了。

是功是罪,是赏是罚,就看朝廷那边怎么认定了。

景诚、宗泽两人,也终于有闲暇坐下来先喝杯茶。

火炉上吊着一柄小巧的长嘴银壶,里面正烧着水。景诚手持蒲葵扇,轻轻的给红泥小火炉扇了两下风,又从一支银盖玻璃小瓶中,取出了两块金花小龙团来。小心的拆开外面的金帛,又将价比黄金的团茶块更加小心放进茶碾中。

景诚有条不紊的准备着茶汤,宗泽静静的看着,忽然开口:“宗泽战前臆测太多,倒是让诚甫兄见笑了。”

景诚抬头一笑,“倒也没什么,如果事情发生在关西,汝霖你可就是算无遗策了。”

“不。”宗泽肃容说道,“若是在关西,贼人根本就攻不下任何一间村寨。就是关西乡中十二三的少年,若有个一两百,手持兵械,也能赢得了他们。”

“是吗。”景诚一声轻噫,心中自是不信。

“关西的蒙学、小学,每天都有半个时辰的时间,用来列队操练。虽然只是排列队形,练些强身健体的拳脚功夫。但到了冬季保甲操练时,蒙学生上场演武,阵型队列比他们家里的父兄强上许多。”宗泽像是要倾吐些什么,“三年蒙学,不只是读书识字,更重要的是增长见识,同时也在学习的过程中,学会恪守纪律。这才是精兵之本。”

“或许吧,但江南民风与关西毕竟不同。汝霖你乡贯两浙,想必比我更清楚。”

宗泽默然不语,摇了摇头。

景诚双手推动着精致的小茶碾,将茶团一点点的碾碎,头也不抬的问道:“此次两浙变故,有明教担下来了。但相公日后打算怎么处置,是否就这样。”

“诚甫兄怎么看?”

“此番事变,虽有明教作祟,实肇因丝厂,此事不寝,工人依然受东主盘剥,长此以往,其何以堪?以我看来,日后火焚厂房之事必将再现。”

宗泽默然片刻,道:“张因考绩下中,展磨勘三年,段炜任满转迁宫观,段将老迈,将斥其自乞骸骨,而陆子石素无官声,宗泽出京前,御史已经上表弹劾。过几日,将会有一份朝报发往各路军州,想必会给人提个醒。”

景诚停了手,对宗泽摇头,“恐其不易。”

宗泽道,“佃农闹佃之事自古未绝,士卒闹饷也年年都有,工人为了工钱闹事又何足为怪?官府只要维持住不将事情闹大,最终他们会取得一个平衡。而且此番事后,想必江南也不会有几家丝厂,再敢于苛待工人了。”

民不可轻。民畏官,但官也一般畏民。

两浙百姓的两税和身丁钱,多是以丝绢的形式缴纳。所以江南就产生了一种专门用来缴税用的丝绢。正常只能织一匹的生丝,缴税的丝绢至少能织出两匹来,黑心一点甚至能能织出五匹。

这类丝绢上的经纬线,最恶劣的情况,稀疏得能钻过蚊子。宗泽曾见韩冈拿了一匹到中书,半开玩笑的说,连纱窗都做不得了。在过去,朝廷会把这类丝绢当做军饷发下去,不过韩冈治事之后,不合标准的丝绢都被禁止下发,而是按照产地发回原州县,让当地官员自己处理。

上有政策,下有对策。

这是宗泽听韩冈说的,不仅仅是上级对下级,百姓对官府依然有办法。最坏的情况,就是揭竿而起。

面对雇主,百姓又岂是好欺负的?只要官府不干涉太多,迟早会有一个平衡出来。

“但愿如此。”景诚说道。
