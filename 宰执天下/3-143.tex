\section{第38章 心贼何可敌(中)}

已是深秋。

万物萧瑟,一阵秋风扫过,道上落叶纷纷而起。除了一些常绿的松柏,也只有田间的麦苗还是绿的。

田间的老农总是有些心惊胆战,中秋前后的雨水不小,在黄河行还形成了小小的秋汛。但到了九月之后,雨雪又不怎么见了。开封府中,也就在前几日下了一场转瞬即止的小雪,落到地上就不见了踪影。

如果今年冬天仍不下雪,明年的收成就没指望了。而那时候,开封府的常平仓,也再难以支持如今年这般数以十万计的流民。

不过晴朗的日子,却是出行的好时节。

秋高气爽,晴空万里。蓝色的天幕,澄澈得仿佛透明的一般。

沈括骑在马上,身后的随行人员多达上百。这一支人数众多的队伍,出现在官道之上,一路向北疾行。行人看见举在队列前的旌旗,皆是避之唯恐不及。

沈括正是志得意满的时候。

前日他成功的从枢密院的故纸堆里翻出了证据,证明辽人索要的土地,过去是属于大宋所有。呈与御览之后,天子大喜过望,现在就遣了他奉旨前往辽国,谒见辽主耶律洪基,将此事分说个明白。

近冬时节,去辽国谈判是个苦差事。

辽国虽分五京,东京辽阳府、西京大同府、中京大定府、上京临潢府以及南京析津府,但这五座京城,并不是如大宋的四座京城一般,是作为首都、陪都的形式存在,而只能算是地区的中心城市——也就是五京道的核心,说是首府更恰当一些。

历代辽主都是保持着游牧民族的习惯,带着他被称为斡鲁朵的宫卫,以及文武百官,在国中分四季逐水草而居。除了登基、册封等大典之外,很少进入这几座京城。

辽主这等游牧行为,并不能算是荒于政事。这是他们的习俗,也是震慑和拉拢四方异族的必要手段。辽主四季巡游的行营大抵都有固定的地点,称为捺钵——这才是辽国的京城。春天在鸭子河,夏天在吐儿山,秋天于伏虎林,而冬捺钵则是在广平甸。

在草原上踏着冰雪行进,宋人很难习惯那等高寒之地,不过沈括心头一团火热,却是等不及的要见辽国天子。

“还有多久到白马县城?”沈括招来随行的伴当,问着。

“回校理的话,前面就是!”

沈括眯起眼睛,有些近视的他,稍稍远一点的地方就是一片模糊。不过他也有办法解决,从怀里掏出一个中间略凹、周边镶银的水晶圆镜来——这是天子赏赐之物,以奖励其在清查旧档并献上熙河路全图的功劳——扣在左眼前。顿时,地平线上的一座城池便出现在镜框中。

从京城往辽国去,或是从辽国往京城来,只要不是冬天黄河冻结的时候,两国使节过去通常走孟州的浮桥。不过现在白马县也有了浮桥,就不需要再绕路了。

一行人都是骑着马,七八里的距离很快就走完。进了白马县城,就在驿馆中歇下。

沈括是身负皇命的使节,不便随意离开驿馆。他本以为已经算是身居高位的韩冈会自重身份,最多派一个家人来送践行之礼。没想到刚刚歇下没多久,韩冈却以故旧的身份亲自来访,到了驿馆与沈括见面。

沈括惊喜的出门相迎,只见韩冈在门前先行致礼:“存中兄,许久不见,向来可好?”

沈括连忙回礼,“一向久疏问候,还望玉昆无怪。”

坐下来先行寒暄了两句,韩冈就赞道:“存中兄之材,远过小弟。早前存中兄所献的熙河路山河地理图,小弟看了之后,便是自叹不如。昨日又闻天子诏存中兄搜检枢密院故牍,小弟就知道,存中兄必能有所收获。”

见韩冈毫无芥蒂的说着自己的得意之举,沈括,连声谦虚道:“当不起玉昆之赞。舆图沙盘是玉昆首倡于前,愚兄只不过是东施效颦而已。至于搜检到旧岁两国所议疆地书函,那是天子圣德庇佑之故,非是愚兄之能。”

“存中兄太自谦了。以兄之材,使辽一回,那契丹的山川地理,当尽在胸臆之中了。”

韩冈看得出来,沈括如今正在兴头上。

王安石去过辽国,富弼去过辽国,能作为使臣——尽管不是贺正旦、贺生辰的正式使节——出访辽国,日后的前途可谓是一片光明。

沈括现在自然满心都是热火,要在辽国天子面前争出个谁是谁非来,驳回辽人的无理要求,不辱使命,凯旋归朝。

可韩冈已经从王雱那里了解到了天子的真实心意——竟然已经准备屈服了——如此一来,沈括在辽主面前表现得再好,也是无用功。

契丹人可以用道理说服,但那是在大宋君臣坚持立场的情况下。

狼和小羊的故事,韩冈三岁就听过了。韩冈从不认为,一方的主君已经屈服的情况下,作为代表的使臣,还能通过谈判来解决争端。自身已经将软弱二字写给对手看了,那就别指望能在谈判中占到多少便宜。

其实这一次,契丹那边不过抱着讹诈的态度,只是想顺手沾点便宜罢了。可谁知道赵顼竟然当了真,以为契丹当真要南下侵攻,却是糊里糊涂的要将土地划给辽人。

这其中几位元老重臣当真是立了‘大功’了。

宋辽交锋大小八十一战,只有一战得胜?有这么信口开河的吗?

韩冈都想见一见,张方平在天子面前提及此事时,究竟是什么一副嘴脸,而沈括则自顾自的拉着韩冈说起了他的得意之举,“愚兄在枢密院用了七天的时间,找到了契丹西京道朔、应、蔚三州发来的公函,函中所及,皆是以古长城为界,距今所争之地有三十里远。”

辽国西京道的朔、应、蔚三州对应着大宋的河东,一直以来都是以古长城为界。但这个国界,其实并没有立下界碑,没有正式的国书确定,仅是在两国的公文往来时,有所提及而已。两国守边的军队,一般都是保持着一定的距离,空出来的中间地带并不去占领。

偶尔,戍边的军队也会在空白区域搭建军巡铺,但无一例外的都会受到对方强硬处理。要么直接发兵拆除,要么就通过所属州郡发文让其自己拆去。这样的情况,两国其实都有,但一点边界摩擦,都会在澶渊之盟的光辉下给化解过去。

这样的边界相处模式,一直以来都成为了惯例。韩冈在后世听说过的所谓打草谷的情况,澶渊之盟后,其实是很少见的。而萧禧如今强要以分水岭——也就是分割滹沱河和桑干河两大黄河下游支流水系的山脉为界——就是打破了已经约定俗成的惯例。

可是,萧禧不过是信口开河而已,他对当地地理都没有稍加了解就来索要土地,明摆着就是个随便找来的借口。

“萧禧一开始时说,以分水岭上的土垄为界,偏偏长连城那一段分水岭上都没有土垄!”沈括说着便哈哈大笑起来,笑过则又接着道道,“若愚兄所料不差,萧禧必然是在辽主面前夸了海口,如今骑虎难下,所以才半点也不肯通融。只要能在辽主面前分说明白,使其知道理曲直,必然不会再有他议。”

“当是如此。”韩冈点着头,附和着沈括。

心中却是冷笑,什么叫疏不间亲?耶律洪基是信他臣子的话,还是信宋人的。

‘唉。’韩冈暗暗叹着。其实还是自身软弱。否则管他契丹君臣怎么想,自身硬了什么问题都不会有。

土地岂能轻易许人,最后的谈判结果若是真的要割地,士林肯定要翻天。

连匈奴人都知道土地宝贵。

冒顿是将汉高祖刘邦围在白登的雄主,汉时的和亲之策,就是他打下来的。东胡人要宝马,要女人,冒顿单于都给了,但等到东胡人又来索要土地的时候,他却是立刻举兵,率领部众灭掉了东胡,使匈奴称霸草原。

如果沈括够聪明,就干脆直接给岁币上加上一笔,就算十万、五万,想必契丹人都会答应下来。反正有匈奴可汗冒顿作为榜样,有富弼作为前例,他就算许诺一点岁币,事后在士林中还能保持一点名声。

不过沈括也仅仅是传达大宋天子的意见,并非主持谈判的全权使臣。真正在河东边界负责谈判的是韩缜、吕大忠、刘忱。他们能不能顶住契丹人和赵顼的两面来压力,那都是未知数。

只是韩冈觉得,沈括他自己好像并没有意识到这一点,误以为天子会支持他,所以才有着一副气壮山河的态度,要是知道了赵顼的真实心意,怕是现在就笑不出来了。

对于沈括来说,能帮着解决天子解决了这一场危机——尽管仅存在于天子的心目中——必然能因此而得到天子的青睐,继而受到重用。

韩冈想了想,还是没再多说。他跟沈括的交情没到那一步,若是交浅言深,事后沈括也不会为他保密。让沈括继续保持着幻想好了,说不定真能如他所愿。

