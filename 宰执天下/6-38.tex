\section{第六章 见说崇山放四凶(五)}

“太皇歇下了吗?”

杨戬静悄悄的走近保慈宫。虽然是在问,不过他已经有了肯定的结论。

否则保慈宫中,不会这般安静。

“已经歇息下了。”值守保慈宫的御医如此回复,“太皇太后喝下了药汤,就睡过去了。”

御医的脸上有着长长的几道血痕,可得出来是指甲留下的痕迹,其中最长的一条,从左边额头拖到了右边下颌,将相貌还算不错的一张脸,给变成了凶狠狰狞的盗匪。

“辛苦了。”杨戬由衷的说道。

与那些被认定为叛党的罪臣不同,太皇太后不论做了什么事,都不会,更不可能受到任何处罚。

父母杀子女,都不用判刑坐监,连理由都不必给出来,溺婴之事天下各州各县都没烧过,何曾见哪家官府为此抓人了?而子女若是弑父弑母,甚至只要故意殴伤父母,报官后,基本上就只有死刑一种判决。

太皇太后尽管已经被软禁在保慈宫中,又是犯下了谋逆的重罪,可若是她受到任何身体上的伤害,刑律和舆论依然会站在她这边。负责看管她的内侍、宫女还包括御医等人,都因此而束手束脚,免不了要受些伤。

“这也没办法。”御医叹着,“太皇太后今天脾气大了点。”

‘只是大了点?’

杨戬环视周围,一多半的宫女和内侍的脸上身上,都有着不同程度的破损和伤处。

不过她们不会像王中正一般对高滔滔请求其体面一点,只能任打任骂,但她们心中的想法却一般无二。

太皇太后没有太皇太后的样子,撒泼耍横如同一个年纪老大的泼妇。

究竟是老了之后学会撒泼,还是泼妇变老了,其实宫中早有定论。

‘圣性严毅’,已经是对高滔滔最善意委婉的评价了。亲手将她抚养长大的慈圣光献皇后,说她一句都会被顶回来。

但之前再如何脾气大,也还可说一声严毅,但方才一阵闹腾,什么形象都没了。好歹也是在宫里面长大的,竟如市井妇人一般骂街殴人。还弄伤了好几个。

‘幸好来得迟了。’

杨戬暗念着自己的幸冇运,悄步走进内厢。可能是踩中了地上的碎瓷,他的脚底发出嘎吱嘎吱的响声。

太皇太后的床榻没有垂下帐子,直接就能看清她的一举一动。

服侍太皇太后的宫女和内侍都是方才才被派来的,一个个神情紧张,肃立在壁角,大气也不敢出。

房内的空气因此紧绷着。另外还弥漫着一股奇异的香味。

杨戬嗅了嗅,转身问道:“这就是给太皇太后的药?这是什么味道?”

“正是方才给太皇太后喝的药。是阿片的气味。”

“主药是阿片?”杨戬不通医术,但他曾在御药院做过事,尽管宫中御药院并不是单纯的一个主管宫中用冇药的机构,但这也是其中的一项职责之一,杨戬多多少少拥有一点有关药材的常识,“对身体是否会有所妨碍?”

“这点份量只会让人快速入睡。”

杨戬点点头,那就是量大便有毒了。

不过这也无妨,有毒性的药材多得是,控制好份量就是。至少近期内,得让太皇太后继续活着。

走到床榻边,高滔滔正披头散发的躺在床榻上,睡得正沉。

‘久违了。’杨戬恨恨的想着。

之前杨戬就因为这位老婆子而被赶出了福宁宫。不过塞翁失马焉知非福,这一次黜落,在他而言却是一件幸冇运的事。

离开了福宁宫,就没有撞上先帝意外暴毙的那桩公案。据杨戬所知,当时就在福宁宫中的一干熟人,都被赶出了皇城,据说是被圈禁在敇建的寺观中,但也有说法是那些人全都被秘密处死了。

而更大的运气是他避开了这一回的叛乱。宫中绝大部分有官职品级的黄门,在这一回的叛乱中,都受到了叛逆们的挟制,要么从逆,要么就是死路一条。大部分人都服从了,而不肯屈从的那一小撮人,全都被处决了——只有王中正一人因为身冇份太高而例外。但随着宰辅们拨乱反正,从逆的宦官则尽数被囚禁起来,等待他们的要么是死,要么便是生不如死。

经此一事,宫内上百名黄门以上的宦官,今天一天几乎被一扫而空。而杨戬,则是因为早早的被踢到了宫中的清冷衙门中,便被叛军给丢到了脑后。根本就没想起来要将他拉进去。而等到叛乱结束,宫里面如杨戬这般还有官身的内侍官不到十人了。

虽然此前杨戬犯过一些错,但这给名字还留在向太后的心中。在宋用臣、石得一,以及两人的党羽尽数飞灰湮灭之后,杨戬便被重新启用,多多少少也能做些事的。

仔细观察了一阵,见太皇太后的呼吸平稳如一,杨戬确认了御医的判断。

回头问着亦步亦趋跟过来的御医,“阿片的药效会持续多久。”

“看各人的情况了,有轻有重。跟药冇品的份量和本人的身体有关。”御医皱着眉思考了一阵,“太皇太后这是第一次喝药汤,到底会持续多久还不清楚。不过半夜时间总是有的。”

“醒来后就又是麻烦了。”杨戬笑了一笑,又问“这药能一直吃吗?”

“三餐不能断,药倒是看情况了,只要太皇太后情绪不激动,一般是不用给她喝药的。”

杨戬沉吟一下,道:“你们要为自己着想,外面的事不要随便告诉太皇太后。尤其是二大王的消息。”

“太皇太后肯定已经知道了。”御医低声回复,“要是连二大王会有什么样的结果都不清楚,太皇太后方才也不可能大闹一场。”

杨戬叹了一口气,他出来时崇政殿那边都已经在写诏书了,现在二大王要么挂在了房梁上,要么挂在灯架上,母子连心,怪不得太皇太后方才会闹。

“二大王是罪魁祸首,太后和相公们饶不了他。可三大王还在,何苦这样闹?弄得不好,三大王那边可就难看了。等太皇太后清醒过来,就这么跟她说吧。到时候,你们小心一点,不要让太皇太后出事。”

”若是事情不妙,下官会命人再熬一份阿片汤。“

“那你们就小心应付,实在不行,就继续让太皇太后喝汤。总之一件事,就是不能让太皇太后有半点损伤。”

“下官明白!”

……………………

“太后。王中正回来了。”

来自殿外的通报,打断了崇政殿内还没停止的议论。

“让王中正进来!”清冷的声音里有着几许急切。

向太后方才直接就在崇政殿上,让翰林学士写了一封奏章,叫了王中正过来,命他领人去赵颢那边宣诏。言辞举止中的迫不及待,就差对王中正说一句快去快回。

不过,王中正倒是能体贴上意,当真是快去快回,都没耽搁多少时间。

赵颢的判决,必要的程序都没有走,只能事后将之补齐。虽然从大理寺在走一遭才算是最好,但局势不稳,与其等到天黑之后,担心有人谋图不轨,想救出赵颢再决死一搏。从根子上断掉最是安全。

回到崇政殿的王中正一身的杀气腾腾。虽未着甲胄,却依然威仪自蕴。看着冇王中正现在的样子,外人一见,都不会怀疑他当世名将的身冇份。不过他真正领军的时候还不一定是这副摸样。

王中正在殿中拜倒:“臣奉旨去责令齐逆自裁,如今事毕,臣特来缴旨。”

“终于死了……?”向太后像是松了一口气的样子。

自从先帝赵顼病发之后,宫中朝中的规矩一切都不正常了。

皇后与亲王无相见之礼,从礼制上说,向太后和丈夫的两个弟弟根本就不能见面,见面了之后连拜见的礼仪都无处可循。亲王与宰执之间也是连说话都犯忌讳,但蔡确还是与赵颢勾连起来。

如果是在先帝赵顼安好的时候,赵颢岂敢轻犯?

现在赵颢死了,终于正常了。

“是太后仁德,许其自裁。”

“仁德?便宜他了。”她毫不客气的说道。

是太便宜了。

殿中几位宰辅心有戚戚焉。

正常就该是凌迟的,在他身上的肉被割尽之前,就不能让他痛痛快快的死。

韩冈则无所谓,人死了就行了。

绞死、斩首、腰斩、凌迟,一个比一个更为酷烈。但再残酷的极刑,其实都只是为了震慑世人,或是出上一口气。

“首恶尽数伏诛。臣请太后速速传谕城内各处,朝廷对赵颢、曾布、薛向三人的处置。”

这份要公诸于世的诏书已经写好,就在王中正领旨去请赵颢上路的时候,翰林学士已经将大诏书写完成。

“王中正,这份诏书给你,出宫晓谕百官众军。曾布、薛向流放交州,其子侄兄弟皆流放雷州、新州诸州。只要能够敛手服罪,朝廷不会加以重惩,但若还有人胆敢死不悔改,朝廷和吾也绝不会再宽宥。”

王中正登时跪倒,接下了旨意,随即转身出了殿门。雷厉风行处,完全就是一名令行禁止的名将。

目送了王中正离开,向太后问着下面的臣子:“好了。这下就不会有问题了吧?”

王安石躬身一礼:“太后已经宽厚如此,除了一二穷凶极恶之辈,谁还能不感念太后的恩德。”

韩绛也道:“便是有人还能惑众,待诏书一宣,其众也会纷纷散去。”

“那接下来还有什么事?”向太后问道。

还有什么事?!

吕嘉问想帮太后扳扳手指。一、二、三,没见少了那么多人吗?

“没有了。”韩冈出班道,“为安定人心,朝廷需要一切如常。且如今首恶尽诛,余波渐平,不能为些许小事而耽搁了正务。”

他瞥了眼几位心浮气躁的同僚,尤其是吕嘉问,不用那么急!

