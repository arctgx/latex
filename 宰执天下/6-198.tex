\section{第21章 欲寻佳木归圣众(18)}

赵令璀找到他父亲赵世将的时候,被戏称为老马弁的华阴侯赵老爷子正在马厩中。

华阴侯府在城外别业的后园,有一半的地皮给马厩占去了,旧时让无数宗室艳羡的池畔垂柳,现在被砍得一株不剩,而赵令璀自幼玩耍的水榭,也是不见了踪影。而院外属于华阴侯名下的百亩坡地,也给改造成了跑马地,给赛马日常训练之用。

这一切的变化,只成就了老马弁之名。

赵世将一身短打,灰褐色葛布衫背后汗湿的印子,让他看起来就像外面码头上的力工,手上拿着毛刷,正小心翼翼的给他最心爱的一匹黑色骏马刷洗着。轻手轻脚的模样,比奶妈抱刚生下来的娃儿都小心,赵令璀都没见过赵世将抱孙子时这么谨慎过。

正如员外郎不敢跟园中狮比,赵令璀也懒得去嫉妒这些畜生。

华阴侯别业的马厩中,闻不到多少异味,干干净净的,马粪和湿掉的草料,不会超过半个时辰就会被清理走。

马栏前的石质水槽中,流水潺潺。在夏日,侯府只用自深井中取出的净水来饮马,通过水槽从井口那里将水一路引过来。而马厩中的住户,一匹匹赛马除了洗刷的时候,都穿着精心缝制的防蚊衣,以防蚊虫叮咬。更不用说,夏天冰块,冬日火炉,寻常富贵人家才能有的享受,一干赛马只会得到的更多。说起来,的确惹人嫉妒。

可这又怎么样?

他的五弟赵令格,曾经抱怨过,赵世将对他的那些四只蹄子的畜生,比给儿子、孙子都舍得花钱。赵世将当时就戳着五儿子的脑门,大骂道:‘养马是净赚的买卖,你们全是折本的生意,能比吗?’”

若是没有赛马,华阴侯府就会与过去一样,外面光鲜,内部则破落潦倒。

多少贵为国公、郡王的宗亲,逢年过节送礼,只能从库房中挖出之前收到的礼物来转赠,赵令璀每逢节庆,查验礼物时,都能看见几件眼熟的,全是从他手中送出去的东西。出去饮宴,能去正店的更是少数,朝廷给的那些俸禄,填饱家中大大小小十几张嘴都不够,谁敢去七十二家正店花销?去街边小店吃点小菜,就差不多了。

只要不是太宗濮王系出身,大略如此。要不然,也不会有豪商之家,县主十个手指数不完的情况。

而华阴侯府,这些年来却红火得很,出门在外,也被人高看一眼。担心家中几个心性不定的纨绔子弟,会因为月例增多而变本加厉,这种奢侈的烦恼,也只有富贵门第才能有。

走到赵世将身后,赵令璀轻声道,“爹,让儿子来吧。”

赵世将没理会,把刷子在水桶中涮了一下,又悉心的刷洗身前的爱马。

马厩中总共十四匹马,也只有眼前的这一匹乌骊,才会得到赵世将无微不至的照料。

黑色的骏马,毛皮光滑得跟丝缎一样。从上到下,没有一根杂色。

这匹从西域不远万里运回的骏马,以乌骊为名。莫说京师,就是天下各路,也有成千上万人知道,华阴侯赵世将的马厩中,有一匹神骏无比的天马,堪比浮光、掠影两匹御前神驹,是京城中屈指可数的顶级冠军赛马,也是天下间身价最高的种马之一。

骊就是黑马,前面再加个一个乌字就重复了。

赵令璀曾经指出过这一点,赵世将只反问了一句,‘想叫盗骊吗?’

盗骊是周穆王的御马,这匹马连乌骓之名都不敢起,怎么还敢用周天子御用的马名?即使是现在的乌骊,还有人说,是不是想要鲤鱼跳龙门。

自证据和结论同样可笑的赵世居谋反案之后,太祖子孙人人噤若寒蝉,赵世将的行事也低调了许多,否则,又何必早早的辞去了赛马总社会首的位置?

过了一百年了,太宗一系,还是将他们当做贼来防着,现在吃喝玩乐,一样少不了四面八方猜忌的眼神。

仔细查看过马蹄上的蹄铁,拿着手巾擦了擦汗,赵世将这才起身:“向四怎么说?”

“越国公说,韩相公当是有意为之。”

“哦。”

赵世将淡淡的应了一声。医学已经建了,下面自是要建立工学、算学。韩冈到底想要做什么,看王安石就知道了。

他拿了根近几年与天马同时传入中国的胡萝卜——这种颜色和气味都很特别的蔬菜,不知为何特别受到乌骊的欢迎,大概是家乡菜的缘故,一看见赵世将将胡萝卜夹在掌心中递过来,立刻兴奋的唏律律叫了起来。

让爱马啃着手中的胡萝卜,赵世将回头问,“向四他当真觉得韩冈是想让宗室贵戚插手进去?他觉得这件事有我们说话的份?”

“越国公说了,去上韩相公的工学、算学,出来最好也只是诸科出身。真正的世家子弟,考不上进士的,都会选择荫补,这比诸科出身的前途都要好。”

赵世将点点头,这世上,有荫补出身的两府中人,却没有诸科出身的宰相、参政。

赵令璀又道:“越国公也说了,我等家中子弟,并不是人人能受荫补,纵是太祖太宗的子孙,一出五服,除了玉版留名之外,也与凡人无异。进士考不了,想做官,也只有诸科一途。无论工学还是算学,其实有一半是给宗室、外戚家的子弟准备的。”

赵世将道,“多少穷措大摩拳擦掌,能从他们手里面抢来多少?”

赵令璀摇头道:“数算也好,营造也好,哪一桩不要钱财支持?又岂是连书都买不起的儒生能置办得起?”

“可惜冯四回去了,找不到人问了。”赵世将拿了手巾擦汗,叹了一声,没说信还是不信。

韩冈最近在经筵上的一番话,冯从义又正好将水力缫丝机等机器丢出来,这两件事很容易就让人联系起来。想要得到丝织上的好处,那么肯定就要支持韩冈的想法。

只是这个决定让人很难做,这毕竟是要让一直作为旁观者的宗室、贵戚加入朝堂的纷争之中,至少要摇旗呐喊一番。得到缫丝机的好处难以计算,而工学、算学对偏远宗室也同样好处不少,不过不付出代价就想吃下好处,这世上也的确没那么好的事。

尽管赵世将已经不是赛马总社的会首,可依然是冠军马会的会首,对马会的影响力无与伦比,身家在宗室中也是顶尖的,平日里周济亲戚不遗余力。在太祖后裔里,人望极高。只要他一句话,多少人愿意为他奔走。。但负担了举族上下的性命,这个决定可就越发的难下了。

‘……还真会为难人。’

老华阴侯声音不大,没让儿子听见,但乌骊一下就支起了耳朵,左右转着。

……………………

从公文上抬起头,韩冈捕捉到了宗泽脸上欲言又止的表情。

“怎么了?”

“相公。”宗泽犹豫了一下,问:“你是不是打算整顿武学?”

“这个?不是!”韩冈扬了扬手上的公文,然后否定得很干脆,“那个烂摊子,避之唯恐不及啊。”

纸上谈兵和实际指挥,完全是两个概念,而军事上急需的是什么样的人才,能明了的朝臣依然很少。

所以设立在武成王庙中的武学,尽管有好些年头了,武举次数也不少,但那些学生,到现在为止还没有一个成才的。

大宋的武官系统,在册两万余,出身各不相同。将门世家、军班行伍、潜邸亲随、外戚成员、士人及文官从军、武举选拔、宦官、蕃将、吏人、宗室,林林总总,百门千道。

但其中宗室、外戚和潜邸,是基本上不会上战场的一拨人,虽说除去宗室外,外戚、潜邸两家出身,是三衙管军的一大源流,不过被朝廷倚为干城的,还是真正能够上阵的将领。

将门世家有传承,军班行伍靠搏命,大多数都能打仗,上阵的也是他们。而宦官、文官领军,几乎都是以监军和帅臣的身份,真正要上阵的,也还是武官们。

而武学出来的学生,尽管有个出身,但他们的职位安排,不像进士和诸科出身那般有章可循,勉强安插到了军中,无不被排挤。再加上这些学生,几乎都是学文不成,才退而习武,属于军中出身的数目极少,更是难以成才了。

想要把武学办好,就先得将混乱的武官出身给整理一遍,但这未免太得罪了人。韩冈暂时还不打算去插手武学,章惇若有心,就让他去做好了,反正那是枢密院的地盘,而且现阶段的敌人,暂时还不需要普及军事学校。

“那相公是打算做什么?”宗泽问道。

“看一看办学校到底会出什么问题?”韩冈讽刺的笑道:“武学是个好样本,能犯的错都犯了。”

“相公的确是打算最近就开设工学和算学?”

“谁说的?哪有这回事。”韩冈一口否定,“要办也是以后。”

没人会认为韩冈之前在太后和天子面前,说‘才士多种多得’只是信口而言,从王韩翁婿之争上看,两家争夺的焦点必然是学校。现在人人皆知,韩冈在他将《幼学琼林》列入解试内容之后,要更进一步了,或许一时不会拿国子监下手,但传言已久与明工科、明算科配套的工学、算学,肯定要设立了。

可韩冈现在却一口否认,这让宗泽迷惑起来,“相公为什么在经筵上那么说?”

韩冈笑了起来。

宗泽若不是困于时代的局限,不会想不到。

算学、工学、乃至农学,韩冈肯定是要设立。弘扬格物之说,需要大量的气学弟子进入官场,走进士一途,竞争性太大,而诸科,就简单了许多。尽管诸科出身很难晋升高位,但是当做事的人遍布朝野,气学的地位又有何人能动摇。

只不过,已经传扬已久,又没有多少阻力的事,又何必让他堂堂宰相在经筵上多费唇舌?

“是蒙学。”韩冈道:“想要种田收粮,难道不是先播种吗?”
