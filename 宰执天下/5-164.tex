\section{第18章 向来问道渺多岐(一)}

唐时修建的华佗祠,在苏颂看来,远比不上州衙宽敞。但眼下州衙正在整修,也只能先在这里暂时安顿下来。

供奉华佗的正堂是不能侵占的,苏家的一大家子百十口人,全都挤在后面给庙祝等人居住的厢房中。狭窄局促的空间,使得苏家上下都怀念起州衙的宽敞。

上个月一场百年不遇的冰雹,毁伤亳州城中屋舍千余间,因此而丧身的城中百姓多达,受伤者更是不计其数。

亳州的州衙也在这一场冰雹中毁损严重。从前门到后苑,几乎每一间屋舍楼宇,都被砸出了一个甚至几个窟窿,房上的屋瓦几乎都损毁殆尽。就连最为坚固的大堂也不能例外。

现任亳州知州的苏颂不得已之下,只能从州衙中搬出来,选调工匠过来将亳州州衙翻修。而在雹灾中,州衙附近的建筑同样毁损严重,一时间也只得先借住在城东尚算得上完好的华佗祠中。

眼下雹灾过去了一个月,受灾的百姓已经安置的差不多了,可州衙修缮完工的时间依然遥遥无期。不仅是大半屋瓦都得替换,大多数房顶的结构损伤都不轻,要替换的地方是在太多,没有三个月以上的时间,根本完工不了。

而更大的问题是亳州城中到处都要重修屋舍,木料、石灰这样的建筑材料价格飞涨,而亳州府库在支出了大量钱粮来救济难民后,剩下的库存,已经不足以购买到足够的材料,以完成修补工作,眼下就只能慢慢的挨时间,等材料的价格跌下来。

就在这样的情况下,苏颂收到了韩冈的来信。

“想不到竟然是让为父去帮他修药典。”苏颂看过韩冈的来信,并不置可否,只是将信递给身后的儿子苏嘉。

苏嘉是苏颂的次子,一直随侍在苏颂身边,看清楚韩冈信函上的要求后,眉毛都挑了起来,“韩冈太无礼!先贬低《本草》为己张目,又邀大人同修药典,此事可是正人作为!?”

苏颂的脸色上看不出喜怒“韩玉昆的品性当不至于如此。而且他说的也没错,《神农本草经》的确纲目不明,眼下是三百余条分作上中下三品,这样还好翻检查阅。但编纂药典,可是药材方剂以千计,仍以三品区分,到时候想找个药材或是方子,也无从措手。”

苏颂轻吁了一口气,“为父曾谒王原叔【王洙】,因论及政事,其子仲至【王钦臣】侍侧,王原叔令其检书史,指之曰‘此儿有目录之学。’王原叔、王仲至父子二人的学问你也是知道的,方技术数、阴阳五行、音韵训诂,无不通晓。能博通如此,便是深明目录之学的缘故。”

“那也不能将大人呼来唤去,视大人为何许人?”苏嘉兀自不忿。

苏颂摇摇头,道“韩玉昆为药典修纲目,打算纲举目张,将目录之学用在药材之中,拿着一条索子将钱都串起来。观其书信,有将天下万物皆囊括进来的心思。这样的气魄,少有人能及……他到底想用什么样的标准来区分,为父倒是很想知道。”

苏颂的目光中充满着对未知事物的好奇。对于天地自然中所蕴含的至理的追求,才让他没有如其他士大夫一样,沉湎于饮宴作乐,或是诗词歌赋之中。

在天下数以千万计的士人中,能遇到韩冈这般同样探索着自然之道的同好,对苏颂来说,是多少年也难有的惊喜。

朝闻道,夕死可矣。

苏颂的心性虽不至此,但能比旁人早一步问道,却是比什么都开心的一件事。

“回京城也不错,为父其实也有地方要韩玉昆帮忙的。”就在苏颂的身边,放着一具架在支架上的千里镜,比寻常的千里镜大了几倍,最前面的物镜,竟有碗盏大小。苏颂抬起手,摩挲着光滑如丝的黄铜镜身,“大宋自开国以来,太祖《应天历》、太宗《乾元历》,真宗《仪天历》,仁宗《崇天历》,英宗《明天历》,直至如今的《奉元历》,这历法一朝一修,但就没有一个准数。熙宁时,沈括掌司天监,举卫朴参校司天监历法事,但其所订《奉元历》其实也是错漏百出。气朔之验、五星之验、交食之验,合于实者仅为十之六七。为父出使辽国,两边的历法竟然硬生生的差了一天。”苏颂眼神一下凌厉起来,“辽人的历法竟然比中国的还准,这可是要颁赐天下万邦的律历!”

接受中原王朝颁与的年号和历法,是藩属臣服的标志。将错误的律历赐给藩属,昭示天下万民,可知朝廷会多丢脸。

“如今五星和日食偏差一年比一年更严重。为父早就有心重修历法,韩冈既然要为父帮他,那为父请他在天文上帮个忙也是理所当然。”

“……儿子从没听说韩冈精于天文历法,三垣二十八宿,千万星辰他能辨认出多少个?”

“你错了,韩玉昆看到的远比任何人要深远。”苏颂长声喟叹,轻轻敲着千里镜的镜筒“我等看到的外相,他看到的是本质。日月星辰的变化之本,韩玉昆早就看破了。没人能想到,五星循环那么简单就能解释通透了。”

说着他又回头冲着惊讶莫名的儿子笑了一笑,“亳州受了一番大灾,百姓暂时是安定了,但衙门也毁了,接下来都是要在这华佗祠中苦熬,还是交给后来人的好。”

…………………………

章敦的提议,韩冈考虑再三之后,才写了信给苏颂。而苏颂的回复,很快就到了他的手中。得到肯定的回答之后,韩冈便立刻上书天子,请求将苏颂调回京中,同编修药典。

对于韩冈的这个请求,据后来从崇政殿中传出来的消息,政事堂中为此事是有过一番激烈的争论的,但最终还是由赵顼拍板,同意了韩冈的请求,派人去亳州给苏颂传诏。

不过赵顼也顺便给了苏颂一个翰林侍读学士和判光禄寺的差遣,毕竟将苏颂这个等级的高官调回京中,不可能只让他做一个药典编辑,这样可不符合优待儒臣的道理。

苏颂的这个判光禄寺,和韩冈的判太常寺基本上是一样的情况,都是有数的闲差,如今是管着祭祀时供奉的酒菜、胙肉等事。

从其名下属吏,大大小小加起来只有二十一人上就可以知道,光禄寺其实比太常寺还要清闲几分。相对而言,翰林侍读学士这个给天子讲学的经筵官,倒是要比九卿之一的光禄寺要重要和忙碌一点。

调回苏颂的诏书发出去了,赵顼给药典起的名字也确定了下来——《本草纲目》。

当韩冈从赵顼口中听到这个名字时,刹那间便是心头一紧,这个巧合未免有些匪夷所思了,让人不自禁的要往坏处想过去。

不过往深里去想,韩冈从开始提建议编纂药典,就一直在说纲目分类,由此影响到赵顼的思路也不是不可能。就像司马光当年在经筵上进读《通志》八卷,本意是以史为鉴,资于治道,赵顼便援引‘商鉴不远,在夏后之世。’这一句,起名做《资治通鉴》。

天子赐名之后,《本草纲目》编修局也就成立了。主要助手有苏颂、林亿和高保衡,除此之外,还有为数众多的名医,他们将为本书编纂订提供技术支持。

编修局的地点,韩冈安排在了太常寺的衙门中,虽然厚生司的位置更好一点,但那里闲杂人等太多,不是能安心编书的地方,而太医局的占地又太小了,腾不出空间来。

朝廷提供给编修局的经费一个月有三百贯,数目是不少了,但比司马光的《资治通鉴》编修局还是要低一等,毕竟药典和史书在此时是不能相提并论的。在局中打下手的吏员,韩冈也从太常寺、厚生司和太医局,调了一批过来。

地方、人员和财务都筹备完毕,剩下的就是该怎么做了。理所当然的,这就必须要主编韩冈来定下基调,这是他的权力,也是他的职责。

立秋已经过了,处暑时节,正是一年中最热的一段时间的末尾,天上日头依然如炼铁炉中的炭火,火辣辣的仿佛能将大地给烤焦。

林亿与高保衡并肩走进了太常寺,冷清的大院,也让他们感到浑身上下一阵清凉。对于两名不属于本司的官员的到来,太常寺中的官吏视若无睹,也就是行个礼而已,也没有人上来帮他们引路。

不过两人也不需要有人引路,这十几天来,他们已经来此造访了好几次。在太常寺一角的院落中,便是他们接下来几年要忙碌的场所。

走进编修局的院落,正厅中门大开,韩冈就站在厅中,当面挂着两幅画,远远看去,画上的图案赫然是两棵树。

林亿和高保衡对视一眼,加快了脚步。

但当两人走近厅中,便发现这并不是两棵树,只是图案如同树一般的分岔。在每个分岔上,都有莫名其妙的名词。而两幅画的左上角,有着简单的题名动物、植物。

“端明,这是……”高保衡指着两幅图画,疑惑的问着韩冈。

“这是生物树。”韩冈转过身“也是这一次分类的纲目。”

