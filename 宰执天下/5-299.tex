\section{第31章 停云静听曲中意(六)}

权知开封府。

作为天下首善之地的地方官,在朝堂的政治版图上自然有着极为特殊的地位,其重要性自不待言。

从最直观的角度来说,就是开封知府的面君次数仅次于宰执,跟御史中丞和天子私人的翰林学士差不多。而从职权上,京畿一路的政务、刑名、转运、军事,权知开封府都有管辖权,最少也有参议之权——地方上有帅司安抚、漕司转运、仓司常平、宪司提刑等四大监司分管路中权柄,但京畿,就只有一个权知开封府。

真要计较起来,天子、宰相也可以说是在开封知府的治下。

这是能沟通内外的首都亲民官。

要不然开封府尹在开国之初也不会是皇储的代名词,使得真宗之后,就没哪个臣子能拿到开封府尹这个官职,只能担任低一级的知开封府,甚至是低两级的权知开封府。

所以在钱藻几乎可以确定要离任,究竟谁来接任,就是朝堂上很重要的议题。

“钱藻的手脚好快,御史台的弹章还没递上,他都已经上表待参了。”并肩踏进崇政殿外的东阁时,章敦小声的对韩冈说着。

“还能怎么办?总得有人出来挨板子吧!”韩冈其实并不太喜欢这个结果,不过这也是官场上的规则:“他这个开封知府本来就是要背黑锅的,不论是不是钱醇老的责任,出了这么大的事,总得向京城上下给个交代。何况皇后本就因为郊祀之事耿耿于心,他哪敢恋栈不去?”

西北边境上的军事冲突并没有因为年节而平息,韩冈和章敦依然不得闲。正月初一好不容易可以不理政事,但初二就得进宫议事了。不过两人抵达崇政殿比较早,宰辅们还没到齐,只看到了一个蔡确,只能先在群臣专用的东阁等一等。等人到齐了,先一起去福宁殿问安,然后再回来议事。

“因为没什么伤亡的一场火离任,钱醇老恐怕也是松了一口气才是。”章敦与韩冈一起,远远地冲走过来的蔡确行了一礼,依然没停口:“开封知府也该换一换了。”

侍卫们在外面站得远,章敦说话没避人,蔡确耳朵也尖,倒是听到了,回礼后笑道:“在说钱醇老的事?”

“是啊,正说他这一回运气不好呢。”章敦回了一句,问韩冈:“玉昆觉得谁接任比较好?”

“此事可是韩冈可以妄言的?”韩冈当即反问了一句。

章敦会意的笑了一笑。

韩冈现在自保的心理很重。不是西府中人,却插手军务,纵然有充分的理由,可也已经是人人侧目,哪里还敢插手到如此重要的人事任命上?可不知有多少人拿着小本子在一笔笔的记账呢,等着几年甚至十几年后翻出来。

“其实换个人也不是坏事。”蔡确微笑道,“天子沉疴将起,上有王介甫平章军国事,下有得力之人安抚京师,朝堂倒是能安定一点了。”

章敦深以为然的微微颔首,可韩冈却不以为然。除了曾布的胡子给烧光以外,这一场火真的不是什么好事。

韩冈觉得钱藻此时落职,等于是在刚刚稳定的地基上,又砍下了一根柱子。钱藻不是新党,但他行事沉稳,朝堂真的乱起来,对宰辅们是绝然不利的。但即便是同一件事情,站在不同的角度来看,却是截然不同。

韩冈看蔡确、章敦两人的神色,再听到蔡确的话,似乎他们是从中看到了压制王安石的机会,“难道已经有……”韩冈的问题刚出口,就立刻警觉断了音。

不过已经迟了,蔡确尽量压低自己的笑声:“玉昆终究是没忍住啊!”

韩冈苦笑的摇摇头,任谁来都会想多问一句的。

聊几句私密的话,是拉近关系的好手段,虽然韩冈对蔡确有着很重的戒心,但他也不得不承认,蔡确以此交心的手法的确很有效。

“究竟是谁?”韩冈也不再避忌。

章敦不瞒他:“是王和甫。”

王安礼?!

韩冈顿时瞪圆了眼,不大不小的吃了一惊,亏他们想得出来!

这是准备不让王安石说话吗?

“出乎意料?”章敦笑道。

韩冈无奈的一声叹:“那还用说!”

他虽然不知章敦和蔡确什么时候联络上的,但他们的想法的确很出人意表。

不论天子说了什么,又有什么打算,他打算借重王安石的声望、地位,却已经明明白白的摆在台面上。接下来若是让赵顼如愿以偿,王安石必然会大举插手军政二事,让宰辅们两边站。

王安礼可是王安石的亲弟弟。当王安礼做了开封知府,王安石就必须在许多事情上避嫌。让王安石继续做个不管事的平章军国重事,才是对所有宰执——包括韩冈——最为有利的局面。

韩冈暗自庆幸,幸好没人提名自己。否则有他这个现成的女婿在,没必要刚刚赶出去的王安礼再赶回来。让韩冈去跟他岳父打擂台就是了。

不过也没人敢提名自己。

包拯担任开封知府时才一个龙图阁侍制,之后才升了直学士。而韩冈现在可是端明殿学士。他推了参知政事、推了枢密副使,谁还敢拿着一个开封知府给他难看?

蔡确和章敦等着韩冈的回答,虽然韩冈很明显不愿插手朝堂人事,但有些事得到他的支持就会很方便。

韩冈沉吟了一阵,却轻声叹道:“天子的病情可是刚刚有所好转,皇后可是正高兴呢。”

蔡确脸色微微一变,韩冈的言下之意,就是得让赵顼心情痛快一点,皇后那边不会想看到有人拂逆天子的心意。若是天子想任命什么人,宰辅们最好不要顶着来。

蔡确和章敦你看我,我看你,心中有了深深的疑惑。难道韩冈已经知道前夜赵顼跟王安石说了什么了?可论脾性,王安石应该不会跟女婿泄露才是。

韩冈看得出两人眼中透出来的疑问,又摇了摇头,当然没有。王旖今天才回门,他又没有登门造访,哪里可能知道说了什么。

但他们怎么不想想王安石的脾气究竟有多倔?若赵顼诚心相托,拗相公怎么会去在乎王安礼正做着开封知府?若王安石真的不管不顾,两府诸公又能拿他怎么样?难道要上本弹劾不成?还是说他们另有安排?韩冈看看蔡确,又看看章敦,以他们两人的才智,有后手是必然的。

蔡确,肯定是蔡确提议。韩冈同样可以断言。章敦得到通知时也应该很迟了,估计是最后一个,否则必然会通知自己此事——孰轻孰重,章敦不会掂量不清。韩冈瞥了蔡确一眼,眉微皱,这一位用心很深啊!

“玉昆认为当如何?”章敦飞快地看了外面一眼,然后问道。

“以医家论,当让病者一切如意为是。”韩冈不管他们有什么后手,还是觉得自己的主意比较好。

“一切?!”蔡确提高了音调。

韩冈的回音很肯定:“一切!”

蔡确和章敦对视了一眼,沉思着坐了下来。

宰辅们陆陆续续都到了,当王安石他也到了之后,韩冈和章敦就过去跟他说话。

韩冈本以为章敦和蔡确举荐王安礼的默契仅仅是存在于他们两人之间,不过蔡确却在同张璪行过礼后,就一齐聊了起来,看来是跟他也勾搭上了。还真是本事啊,韩冈心中暗叹,连张璪都给拉下了水,也不知他们什么时候就有了联系,共同准备推王安礼上位。

回复 2楼2012-11-29 00:54举报 |

799597976
朝请大夫8

一个宰相、一个枢密、再有一个参政,推动王安礼回京任职已经足够了。只要他们提议,其他宰执肯定能明了他们的想法,基本上就是不赞同,也不会反对。

最让韩冈感到意外的是曾布也到了,只是变得面白无须——其实不能用面白来形容,曾布身材短瘦,肤色也只比王安石稍好一点。当他和身材瘦高的韩绛并肩进来时,韩冈脑中立刻就浮现起一幅龟鹤延年的贺岁图来——脸上有几处不算明显的伤痕,公服簇新,估计是昨天得赐。

当所有人全数到齐,阁中众臣便一齐前往福宁殿。

寝殿之中,赵顼今天的脸色似乎又有所好转。见到群臣跪下参拜,叩问安康,便用手指很顺畅的在沙盘上画了一个好。

照常规,这时候就该回返崇政殿了,但韩绛却开了口:“陛下,殿下。”他冲着赵顼和向皇后行了一礼,“钱藻以除夕火灾措置不当上表请辞,如何处置,臣等不敢擅专,还请陛下示下。”

向皇后不意韩绛会在寝殿中提起此事。不过开封知府的人选,就是她和宰辅们商议了之后,也会回来跟赵顼商量。故而不以为意,坐定了等待丈夫的意见。

王安石也有些诧异,不过同样不是很在意,正常现象而已。皇帝病情有所好转,做臣子的当然会有人偏回来。

赵顼听了询问,眼睛似乎亮了一下,以指画字,“平章之意如何?”

王安石便回道:“钱藻京师用事二载,无所大过,只是近日雪、火二灾,其人措置不当,不宜再为开封知府。”

几名宰执明会于心,王安石除夕夜必然得天子面授机宜,若是前几日,王安石绝不会开这个口。

得了王安石的回话,赵顼便不再问宰相、执政,直接画字道:“准其辞。”

“臣领旨。”韩绛领旨后,又问,“敢问陛下属意何人继任?”

赵顼停了一阵,手指落在沙盘上:“吕嘉问如何?”

这是个出乎意料的人选。说起来,已经有好些年没注意到这个名字了。那是当年与背叛的曾布争执不下,最后两人一齐出外的新党干将。论资历,还远远不足。就是王安礼、孙固,都比他更不会宰辅们感到意外。

正常是应该有所争执的,但韩绛却领头拜倒:“臣领旨。”

向皇后惊异的睁大了眼,狐疑的望着韩绛,又一个三旨相公?
