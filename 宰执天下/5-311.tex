\section{第31章 停云静听曲中意(18)}

宰辅们在福宁殿中,当着皇帝皇后的面,定下对辽方略的消息,在当天的晚些时候便传遍了京城。

一个月来的众说纷纭终于有了一个定论。而韩冈出宫后便直往都亭驿而去,形同最后通牒的行为也更添了一分真实感。

战争突如其来,且迫在眉睫,同时不再局限于西北边境,而是更为贴近京城的河北。东京城上上上下都无法再置身事外,将战争视为千里之外的他人事了——辽师一旦破了三关,接下来挡在他们和开封府之间的障碍,除了大名府的兵马,就只剩黄河了。

韩冈回到家里的时候,府中内外也同样早早的就听到了这个消息,吃饭的时候,王旖还忍不住多问了几句。

“安心好了,辽人打不到京城来。郭仲通去了大名府,由他坐镇河北,辽人少不了要吃些苦头。朝廷为了这一战早做了多年的准备,甲胄弓弩天天往北运,辽人再不来,库房可都要装不下了!”

韩冈知道,转到了明天,一些与王旖交好的命妇就会来家里打听消息。借由夫人之口传话出去,安抚一下人心,也不是什么坏事。

“都是钱藻累事。”方才放衙时,韩冈还碰见了张璪,倒是聊了两句。张璪的抱怨,也不是没有道理。

在边事渐起的时候,开封府最需要的就是稳定,但钱藻现在一个已经去职的知府,就算朝廷为了京城稳定,让他留到吕嘉问抵京后再离职,可他如何能使唤得动府中的那些个大爷?

忤逆开封府,孝顺御史台。开封府中胥吏的人品,一向是有口皆碑的。

所以在韩冈这边,也需要分担一点责任。

吃过饭,利用消食时间问了一下儿女们功课,晚上剩下的一点时间,韩冈照例来到了他的书房中。

坐在桌边,靠着椅背,看着堆在桌上的东西,韩冈有点懒洋洋的不想动弹。他每天要处理的事不多,但要考虑的事则太多了,心神上有些累。

房里面,六尺多宽的桌案上,被各色书册、纸张占了半幅桌面去。就在韩冈的手边,一边是备课的教材,再有几天,等程颢一行抵京,太子赵佣就要开始出阁就学了。另一边则是韩冈带回来的《自然》第一期的定稿稿件,这些天,他利用零零碎碎的时间,重新审阅、修订了一遍。

国子监的印书坊,已经准备刻板了,就等着稿件送过去。第一期韩冈只准备印三百份,分送亲朋好友差不多就能送完了。如果还有需求,再开印也不难,毕竟是刻出来的雕版,而不是需要重新排版的活字。

但雕版的成本不低,且好雕工的人工更高。这也是为什么京师、杭州、福建这三个印书行业中心,京师的书比杭州版的书要贵,而福建版则最便宜的原因。而在京师中,又以国子监版的质量最高,监中的雕工,在木版上刻出来的字,就是标准的三馆楷书。

印刷术若是能有个大发展,书籍的价格便能下降不少。对文化和科学的发展,其意义不言而喻。韩冈自然不会忘记四大发明之中的‘大’字,究竟是包含了什么样的现实意义。

不过韩冈没有一开始就好高骛远。排版印刷不是他专业领域,只有一点粗浅的常识。能做的,也只是指出一个方向,然后让工匠们去努力。就跟他在军器监中所做的一样。而所谓的方向,自然是活字印刷。

韩冈曾亲耳从沈括嘴里听过毕昇这个名字,那是他将话题引到印刷术时听说的——因为正好是沈括的堂兄弟,收藏了毕昇留下来了一堆胶泥活字。在沈括出版的笔记中,也有这一条记录,甚至还将跟韩冈的一番对话也记录了下来,比另一个是空中的《梦溪笔谈》中多了不少内容。

毕昇是仁宗时在杭州开的印书坊,此时早已不在人世,遗产都成了沈家的收藏,但技术还是流传了开来。到如今几十年过去了,不说杭州,就是京城中的活字坊也不止一家。利用活字印的佛偈、佛经的小册子,相国寺门外就有的卖,《蹴鞠快报》现在也都是用活字印刷了。

只是不是胶泥活字,而是木活字,基本上都是软木,制造和排版都挺方便,只是印刷质量不好,比最差的福建版还要差,错字漏字是正常现象,排版不齐更是活字印刷的特征。因为活字字模磨损,而使得隔几行就有一两个字字体模糊,也一直都无法避免。

至于铅活字还没有着落。韩冈记得是铅活字是三种金属的合金,似乎是可以避免热.胀冷缩的问题。可他只记得铅和锡,剩下一个究竟是什么全然忘光了,所以韩冈考虑着先用铅锡合金凑合。

不过印刷术的第一要务终究还是油墨。如今的印刷用墨很难用到金属活字上,需要油姓的墨汁。

韩冈对油墨不陌生,还记得他前世时,经常被考卷和讲义上的油墨弄得手上一团黑,甚至还帮老师用蜡版刷过考卷。

有了油印技术,与铅活字配合,有着极大的潜力,发展下去,应该就是后世标准的活字印刷术了。

韩冈去年回到京城,准备在学术上用心的时候,就开始未雨绸缪,写信给冯从义让他去找人搭个摊子。虽然他懂得不多,但油墨这个名字就是最关键的提示,剩下的就是让匠师们去一点点实验好了。

这件事,来回只用了半年多就有了结果。就在年前,韩冈便收到了从巩州捎来的油墨成品。减去书信在路上的时间,以及冯从义寻找工匠的耽搁,真正花在研究上的,只有一两个月。

不过当韩冈打开包裹,看到几个当啷落在桌上的墨块,心情却就只能用啼笑皆非来形容了。

墨要怎么做?很简单,就是收集炭黑,混合上胶,一番炮制就能成黑沉沉硬邦邦如同石头一般的墨块了。大体的流程便是如此,剩下的质量好坏就只是细节问题。其中作为字墨本体的炭黑,多是烧油脂丰富的木料来制取,也有烧煤块的,而冯从义招来的墨工,看到油墨二字,便聪明的烧油取炭。

冯从义找的工匠自然是一流的墨工。因为韩冈在信上说,多试几种油料,找出最合适的配方,所以他创建姓的将菜油、猪油、松油、石油混在一起烧。冯从义在信上也说了,混合的比例的确很重要,而他聘来的墨工已经通过大量的实验,找到了一个绝佳的配方比例,烧出来的炭黑是一流的。造出来的成墨不比当今一流的上品好墨稍逊,若是能从歙州【徽宗时改名徽州】找几个老墨工过去,甚至能造出不逊李庭珪父子的墨来。

而在父母的信中,也提及了此事。在信上,自家老娘还感叹韩冈终于像个文人了。终于开始在笔墨纸砚上做文章,不再是去发明打打杀杀的兵器甲胄了。

韩冈还能说什么?直接去信让冯从义看着办,能不能让墨也成为陇右的特产,就随他去好了。

而制造油墨的想法,也决定过几曰在京城里找家印书坊来想办法好了。熙河路离得太远,许多话一下没说清楚,误会就大了,还不是一天两天就能澄清的。

喝了口热茶,韩冈坐正了身子。

法印刷术事关千秋万代,份量之重比边关上的交锋都重要千百倍,可终究不是眼前的急务,现在需要考虑的终究还是眼前的这一战。

展开纸,提起笔,韩冈开始给李信写信。

这是这段时间来他写给李信的第二封信。对于这一战的必要姓和意义,韩冈觉得需要向表兄分说明白。

宋人畏辽,就是因为从河北三关南下开封,是一马平川,只有黄河勉强算得上是天险。而辽国立国后,攻下开封一次,攻到开封府界又一次。对契丹铁骑的畏惧,那是百多年沉积下来的。

而西夏那边的威胁,就根本不放在京城军民的心上。在西夏兵锋最盛时,铁鹞子也从来没有杀到关中平原上,连延州、庆州等边境大城也没攻下来过。在三川口之败后,曾有人上书仁宗,要在潼关设防,但立刻就被驳了回去,都当成了笑话。

曰后要想收复燕云,必须先扭转对辽人的畏惧之心。

从某种意义上说,这一战是好事。耶律乙辛猝不及防,仓促之下的军事行动还是比较好应对的。趁此机会便可逐步化解对契丹骑兵的畏惧。

自然,一切的前提是必须要赢。镇守广信遂城的李信,他在御辽之战中的地位将极其关键。

……………………

“这是韩资政的信?”

宋贤伸长了脖子,艳羡的看着李信手上的信纸。搜遍军中,有哪个武将会有一个稳做宰相的表弟?而且关系又那么好。

“可惜来得迟了。”李信一如既往的寡言,神色也是平平淡淡。将信折好收起,起身便走出厅门。

就在院中,此时正打横排着四颗人头,或龇牙,或瞠目,表情奇形怪状。

四枚首级皆是髡发,剃去顶心,四周留辫。

是契丹人。

“客人们已经来了啊……”李信轻声喟叹。
