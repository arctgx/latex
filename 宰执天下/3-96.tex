\section{第29章 百虑救灾伤(一)}

白马县的县衙这一天突然忙碌了起来。不是二门以外的县衙大堂、二堂,而是二门之后的内庭。

刚刚雇来没有多久的使女婆子,拿着抹布水桶,仔仔细细的打扫着每一个角落。而也在整理离着大件的杂物。如今天干物燥,几个月来的雨雪,加起来还不能没过桶底,空气都是雾蒙蒙的,全是灰土尘埃。莫说园中的花木全都变成了灰黄色,就是室内的家居摆设,也同样只要半天功夫,就能落上一层灰。

韩冈虽然好洁,每天也让下人打扫着房中。但男人眼中的干净,与女人眼中的干净,定义是完全不一样的。过去的一段时间可以糊弄过去的地方,从今天往后,却再也不能视而不见,随随意意的一带而过。

——昨天晚间,有消息传来,县尊的夫人今天就要抵达白马县中。王相公的女儿,另外还要加上几个妾室,同时还有韩冈的一对儿女。知县的一家子终于到齐了,当然要好生的清洁一番。

韩冈虽只是让人将自己居住的院落打扫干净,安置一些必要的什物,但有心在韩冈面前表现一番的仆婢或是胥吏,又有哪个会放过这么好的机会?当然是手忙脚乱地将韩冈吩咐下来的事情尽量做到最好。要不是韩冈本人的性格这段时间已经让所有人都看在了眼里,自掏腰包买了贵重摆设来卖好韩冈的,人数绝不会少。

今天韩冈本人也没有像往常一样,远赴乡中视察旱情。而是就去了城外不远处的流民营。

这座在预定的设计中,能容纳几万人的营地,如今只有一点雏形。进入其中的流民,也不过两百多户而已。不过该做的准备,韩冈一点都不会漏掉。从食物到饮水,从居住到行动,吃喝拉撒的一应事务,韩冈都是全盘放在心上,有一点问题传到耳中,便及时将之处理。

这些天,流民营中最重要的事情就是打井。

人多的地方,病疫自然也会多。流民营一旦聚集了上万人之后,一不小心就是一场大瘟疫,尤其是到了春天之后,死上一半都不是不可能。

这卫生情况乃是重中之重,韩冈就是靠了医疗制度而出头,当然不可能不放在心上。而在这其中,洁净的水源是保证病疫不至于大爆发的关键所在——这个时代,最为洁净的水源,则只会是井水!

今年秋冬,大旱成灾。在十月份的时候,因为田中的出苗率只有六成不到,在韩冈还没有完全接手县中事务的时候,白马县民就已经自发的开始疯狂的四处打井,要用地下水来灌溉土地——不要多,只要能出苗就行。

从十月到十一月,只一个月的时间,白马县中新开出来的水井就多达两百口。其中大量出水的就只有十分之一。靠了这么二十多口井,加上原有的一些,也的确浇了一部分地出来。只是对于县中整体的苗情,乃是杯水车薪。

如今在流民营这边打井的人力,韩冈用的自然就是流民,从官库中掏出钱粮来雇佣他们掏井。精壮的汉子下井中掏泥,而妇孺老人则是打打下手。而负责在流民营附近寻找水脉,确定凿井地点的,则是请了一个在前面县中百姓四处开井时出水最多的井师。

如今虽然天寒地冻,可也就地上三尺被冻得发硬。一镐下去,就只是一个小坑的情况,到了深处就不见。过了冻土再往下,要容易许多,随着越挖越深,土地渐渐湿润变软,从泥地渗出来的水也是越来越多。

那些流民中的精壮,都是脱得只剩一条犊鼻裤下井去挖,通过轱辘将混了地下水的泥土一桶桶的挖上来。堪用的劳力有两百多,开井的进度也比正常要快,不过六七天的时间,同时开的二十眼井中,就有八眼出了水。

也就在昨天,韩冈收到妻妾儿女即将抵达白马县的消息之前,流民营的井出水的消息也送到了韩冈这里。

今天早间处理完公事,韩冈便带人来到流民营中。

被指定为流民营甲区保正的,是带着一家老小三十余口南下的老汉,连同着一个村子逃难的都在一起,人口多,势力大。加之这一片的都是乡里乡亲,互相之间,绝大部分都能攀上亲的。这个姓张的老头子年纪最长,也能镇得住他的晚辈。

见到韩冈一行抵达大营门口,张保正便带人迎了上来,紧跟在他身后的,是点了掘井位置、立了大功的井师。

在韩冈面前,张老汉让下面小子捧上了几个瓷碗。韩冈看着盛在碗中微显浑浊的井水,点了点头,至少是能用了。

“不过最好还是要白矾啊!”他低声叹了口气。

京中七十二家正店之首的樊楼,最早其实是叫做矾楼。就跟同为七十二家正店的马行楼一样,本是行会的会所,后来才改为对外开放。

矾楼之所以会变成樊楼,是因为朝廷将矾业归于官府专卖,矾业行会最终解散的缘故。

就跟食盐一样,此时百姓在日常生活中,对于明矾的使用,乃是普遍的情况。世间的大户人家,都会用明矾来澄清日常用水,无论井水河水。而普通的寒门素户,如果有条件,也会购买一些明矾来使用。

明矾在此时人们看来,就是最好的净水之物,也是韩冈眼下能想到的净化。泥浆水就算煮开了,也没人愿意饮用,如果能加上一点明矾。

不过这就未免太过奢侈了一点。贫寒人家都没有用白矾净水的,哪有从官库中拿白矾出来给流民用的道理?韩冈要是这么做了,必然会引起一番议论,不过用在疗养院中,则不会有任何问题。

参观过出水的水井,水量都很充足。有几口废井,其实也能渗出点水,不过水量不多而已。

对于这些流民的工作,韩冈很是满意。从这个速度来看,两个月内,还能开出几十口堪用的水井。不过要用来提供给足够万户流民使用的水源,却有些不太够。更确切点说用水桶取水,对水井的利用率太低,不足以供给更多的使用者。

最好能造出从井中提水的器械,类似于水车的那种,用畜力或人力来拖动。不仅是供人饮用效率太低,同时用来灌溉田地,用井水一桶桶的提上来,也是太浪费人工。

韩冈早已看到其中弊病,前些日子就给出五十贯悬赏。征集能够大量提升井水的器械。

要知道,在流民中从来都是不缺乏人才的。旱灾、水灾,也不会因为人的才能而将之放过。管你有才无才,是贫是富,一体都受灾。还是老子那句让人说滥的话,‘天地不仁,以万物为刍狗。’——天地无私亲,对于万物一视同仁。

在艰难困苦的生活逼迫下,人们往往都能迸发出日常所见不到的才华。这等危急关头的爆发力所创造出来的结晶,正是韩冈所期盼看到的成果。更别说高价悬赏,白马县的百姓中也有人为此而心动。

如今这个时代,有用来放火、救火的唧筒——只看抽得是水还是油,也有利用畜力、人力的水车,更有通过滑轮从水井中用水桶提水的牛拉井,集思广益的改造一下,不是弄不出来可以用来大面积灌溉的器械。

但出乎韩冈的意料,当初的悬赏的确让人们蜂拥而来,其中几件的确有着不小的可行性,但让韩冈心动的,不是那些还没有造出来的器械,而是那位在寻找水脉上有长才的井师。

这位井师就以井为姓,排行十六,以行辈为名。听着口音,乃是蜀人。在韩冈面前,井十六说道:“……不需要人力畜力来提水,水井只要打得够深,穿透了石层,就能自己涌出来。”

“此事真的可行吗?”韩冈有所疑问。深达几百米的自流深井当然好,他也不是没见识过,但这个时代的技术要求能做到想打深井就能打吗?

黑黑瘦瘦的井师点头回话:“回县尊的话,小人过去曾经打过!”

“打过?”韩冈立刻追问着:“不知那眼井在哪里?”

井十六却跪了下来,“还请县尊赦小人之罪,小人方敢说。”

韩冈不喜这等要挟之举,但眼下的情况让他不介意赦免一个人才,更不介意问上一问,“可是杀人行劫?”

“不是。”井十六连忙道:“若是做下此等恶事,小人怎敢说出口?”

“那就没关系了。”韩冈笑道:“只要不是论死重罪,其他的过错本官就帮你担待着。如果当真能打出自流深井,救了本县百姓的灾伤,本官甚至可以奏请圣上封你为官。”

井十六.大喜过望,磕了几个头,抬头道:“禀县尊,小人乃是蜀人。”

韩冈点头:“能听得出来。”

“小人出身于富顺监。”

韩冈神色一变:“盐井?!”

井十六低头道:“县尊明察,小人本是盐户,祖传的点井之术,后来遭人陷害逃出来的。”

“原来是家学渊源。”韩冈这下对这个黑瘦的四川汉子有了几分信心,“不管你受了什么委屈,只要能立功,本官保你能有衣锦还乡的一天!”

