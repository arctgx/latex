\section{第35章 历历新事皆旧史(中)}

“在下前两天刚去过城南的养济院,那些小娃儿真是可怜。”一个满脸油光,相貌可笑的胖子,在多景楼这座润州最为胜丽的名楼雅间中叹息着,“我张德生是读书不成,只能行商。可那些官人,读书进学,一个个把书都读到哪里去了?连孤儿孤女的口粮都能克扣。”

只看这张德生一身没有花色的朴素绸衫,浑身上下没有半点金玉之物,没人能想到他就是润州最大的丝绸工厂主,背后还有着一个世家大族撑腰。

“怎么会少了?”张德生对面的儒生连忙道,“每人每月有十八斤口粮,太后和相公们的德政,谁敢克扣?”

张德生哈哈的笑了一阵,忿然作色,“对,对!要不是有太后陛下和章、韩两相公的德政,这些弃婴可都是要葬身沟渠,朝廷给付的口粮,也不会有人克扣。只是小孩子不知道好歹,吃得太多……”

“张兄!”

那儒生屁股上好像生了疮,坐立不安,连咳了几声,脸都变了色,不敢让张德生再说下去。

张德生长声叹息,垂下的眼角悲天悯人,“朝廷给的或许不少,但一干雀鼠居中盘剥,能落到小娃儿头上,就太少了。小娃儿正是长身体的时候,饥一顿饱一顿,一个个面黄肌瘦的,阿弥陀佛,让人看不过眼啊。”

那张本有几分可笑的胖脸,仿佛镀上了一层光,变得庄严肃穆,让人望之生敬。

“那张兄后来又给养济院捐了一笔?”儒生一边问着,一边拿着筷子夹了大大的一个虾圆。

“捐了一些。”张德生没有丝毫得意,反而更显低落,“当时带了钱少了,回去后便想着让家里送了一车粮过去。只是又一想,若是给个百八十石,多是多了,但肯定没两天,都给那些‘雀鼠’给分了去。便只能先给了五石米,不够人分的,好歹能多留一些,剩下的,等下次再给。”他叹了口气,拿着筷子指着外面,“这世道,连行善都要思前想后,唉……”

书生拿丝巾擦了擦嘴,离席起身,向着张德生恭恭敬敬的作了一揖,“张兄德行高致,急公好义,小弟敬服。今日回去,必在报上一彰张兄之德。”

“当不起,当不起啊。”张德生连忙跳起来,“在下捐钱捐物,也只是理当如此,岂是为了搏名?”

“张兄你这话就错了。如张兄这般德行,小弟不在报上为之彰显,那还有什么事值得宣扬的?小弟主持这份报,就得告诉润州百姓,这世上不止有只顾一己之私的小人,也有如张兄这样的纯德君子。教化生民乃是圣人之教,若能告知世人,善人能得善果,这便是教化了。非为张兄之德,也是为了教化之功。”

一个时辰之后,张德生的马车回到了家。

待马车在前院停稳,从车下来了一个酒酣饭足的胖子。

一张胖脸越发的油光,刚刚跟润州快报的副主编吃过饭,张德生心情很好。他拿着牙签剔着牙,一步一晃的进了正屋。

屋中一个老苍头等候已久,见了张德生,连忙上前行礼,“四老爷。”

看了看那老苍头的脸色,张德生自顾自又继续剔牙。等到从牙缝中,挑出一块粉红色的肉,他方斜睨着眼睛,吊着嗓门:“怎么了?又出了什么事?”

老苍头愁眉苦脸,“禀四老爷,丝厂那边的工人又在闹了。”

“又闹?!”牙签啪嚓两段,张德生瞪起眼睛,“闹什么?是嫌钱少?一个月一贯半的工钱叫少?我还管他们吃管他们喝!你叫他们去问问,这润州百里方圆,有没有比我更大方的东家!”

“小人也这么说。可那些工人说……说……管饭只有中午一顿,饭又稀,还多黑米,吃着有怪味。还说……”老苍头吞吞吐吐,边说边观察着张德生的脸色。

“还说什么?”张德生挂着脸问。

老苍头低下头,“还说老爷一直拖着工钱不发,只能从账上借支,年底拿工钱抵账时还要记利息。”

张德生重重的哼了一声,“绢卖不掉,我拿什么钱给他们?契书上也写明了,一季帐一季还,最迟年底结清。我去年年底没结清吗?我可是半点没亏欠他们!”

“可他们……”

“什么他们!”张德生暴怒道,“那群穷骨头,都是看你软,觉得你会帮着他们说话,才敢闹。别忘了,给你工钱的是谁,是我,还是那些穷骨头?要不是看着你女儿的份上,早就把你开革了。你回去对张武说,谁敢闹事,都抓起来送到官里去。”

老苍头被骂得抬不起头,嘴也不敢回,只知道点头。

“嫌没钱,不会做乌龟叫自家的婆娘去卖啊。那样来钱最快!”张德生骂骂咧咧,发作道,“过两个月,倭国的奴工运来,就把他们都开革了。这班贱骨头,等了他们还不上账,看老子怎么收拾他们。”

骂了一阵,张德生把自己小妾的父亲赶了出去,另叫了一个管事进来,“倭国那边还有多久才有新货来?”

“回老爷的话,秀州来人说,这段时间倭国管得严,新货到得太少。又说请老爷放心,等到辽国皇帝同意,就能光明正大的发卖了。”

“什么皇帝,是伪帝!”张德生没好气的更正道,“利这般厚的买卖,早就该做开了。还拖,拖到什么时候?这一来一回少说耽搁了我半年,这可就是少赚半年的钱。还要多受半年那些穷棒子的气!”

张德生发着牢骚,管事的不知该说什么,低着头等牢骚发完。

等到一通抱怨发泄完毕,张德生才又对管事的说,“到时候留几个人下来,怎么操纵这些机器,还得先教一教,等教会了再开革。还有,工厂里面管事的,不需要什么本事,只要听话,只要听老爷我的话!”

……………………

“那些丝厂的工人当真是惨。”田轸回到编辑部,刚换了衣服,就连声道,“在工厂里只做了半年的工,就双手溃烂,双脚浮肿,瘦得脱了型,都不成人样了。你们是没去看过,张家的丝厂,整座厂房到处是湿漉漉的。又热又闷,在里面待上半日,就连气都快喘不过来了。那些工人就要在这种厂房里面做工,还得把手探进滚水中取线头,简直不把人当做人。”

一名编辑语带调笑,“张德生可是有名的张大善人!”

“善?”田轸朝底下啐了一口,“去了养济院就只给了三贯多钱,五石米。这几天就只见他上了酒席,就是好一阵宣扬,还以为他捐了三百贯、五百石呢,原来就舍了这么一点。”

“我听说,昨日那张胖子在金山寺捐了八十贯的香钱,僧众一人一匹缎子,用来裁衣。而且他家的老封君每个月定例的要给金山寺和常乐庵各五十斤香油,点长明灯用。”另一个编辑说道。

田轸气哼哼的说道,“不做人事,还想在佛祖面前讨好,等他死后,不下地狱才有鬼。”

第一位编辑道,“死后的事,死后再说。现在的事,谁也拿他没办法。开丝厂的陆、张、尤、段皆为郡望,哪家没三五个进士撑腰?张德生的亲叔可是在河北做知州。”

“说什么呢?”从门口走进一人,正是陪着张德生吃饭的儒生,“张德生那些商人是奸猾,可他们没犯王法啊。杀人放火,官府能管,不给工钱,官府也能管,这做工太苦,官府怎么管?又没人逼着那些工人去丝厂上工,觉得苦,那就不去好了。青天白日,纵是郡望,也不可能逼着人去他们家里做活。而且……”

“而且什么?”

“我听说段家现在已经在用倭人做工,开革了不少丝工。等张家也学了这一招,就不用听那些抱怨了。”

“怎么会没抱怨?世所谓男耕女织,少了纺织的进项,只靠土里刨食,又有几家能吃饱饭的?倾家荡产的也是所在多有。”

“吃不饱饭可以去拓边啊。”那儒生道,“没看朝廷将养济院改归了保赤局吗?流民也好,乞丐也好,只要未满十二,朝廷都不会白白养着了。若是没饭吃,趁早去官府报名,到边疆拓荒。听说西域虽多荒漠,但雪山脚下水土亦佳。而云南新疆,则是四季如春,土地肥美,更胜江南许多。”

“物离乡贵,人离乡贱。岂不闻故土难离?”田轸反驳道。

“既不肯做工,又不肯移居,我看不是故土难离,也不是做工太苦,而是懒吧?照他们的想法,恐怕是盼着朝廷白白养着他们最好。”

田轸一时气结。

“授人以鱼不如授人以渔。朝廷其实做得够好了。实在没饭吃,朝廷还是会给你地,种了粮食自己吃。难道这些还不够吗?张德生之流纵然残苛,可为什么倭人能吃得了苦,他们吃不了?还不是懒!”

田轸忿然道,“等那些工人闹起是非来,朝廷会这么说吗?”

“润州有朝廷的兵,对岸还有铁路。真出了乱子,就算能弄出些声势来,十日之内,就能平定下来。”

“出事了,出大事了!”儒生的话音未落,一个编辑就跑了进来。

“段家的丝厂起火了,张家丝厂也乱了。出大事了!”

田轸惊讶的与其他几位同事对视了一眼,这乌鸦嘴今天竟然对了一回!一

