\section{第七章 烟霞随步正登览(七)}

“臣举翰林学士曾孝宽。”

这是第四位被推举出来的合格候选者。

章敦虽然没有问过曾孝宽,但他觉得曾孝宽应当无意在枢密副使时便出来与人竞争。资历比吕嘉问等人更老、曾经做过同签书枢密院事的曾孝宽,因故出外又重回朝堂的他,现在的目标应该是参知政事,而不是枢密副使。

而且曾孝宽与韩冈曾经共事,关系不差,只要不与韩冈竞争,说不定还能得到韩冈的一臂之力。所以曾孝宽之前一直没有动作,便不为章敦所在意。

不过当沈括这株墙头草又倒了回去,还指称曾孝宽与逆党相往来,嫌疑甚重。为了能够压制住韩冈,曾孝宽就必须填上沈括留下的这个空缺——本来沈括就让人无法信任,一旦沈括态度有变,在一干人的计划中就是曾孝宽来顶替他。

还是蔡确累人。章敦想着。作为始终坚持新法的宰相,在新党内部,其实是自立一派。他的倒台,也连累了一大批新党成员。同时曾布身边的一批人也都垮了台。现在够资格进两府的新党核心,也就是曾、李、吕这么几人。

“不经两制,不可入选。”

王安石又否决了一名没有两制经验的朝臣直接获得两府的入选资格。

章敦的视线,这时从王安石,转到韩冈,又从韩冈转到了其他宰辅身上。

没有人向旁边多看一眼,所有人的注意力现在都集中在了推举上。可能除了章敦,没人察觉到经过了沈括的一番话,蔡京在没有定案的情况下就成了叛逆。又或许有人注意到了,却与章敦一样,有各种各样的顾忌而无法站出来。

之前蔡京只是因为有嫌疑才被收进去进行审问的,而且是生拖硬拽,理由极为牵强,到现在为止,都没有一份判词能定了他的罪名。

虽然说沈括是韩冈的人,开封府的那位章判官辟光对韩冈也奉承备至,若哪天报称蔡京在狱中‘病死’,一点都不会让人觉得奇怪。可想要将蔡京明正典刑,就没那么容易了。就章敦所知,还是颇有些人等着蔡京瘐死狱中,好揪出韩冈在背后的黑手来。

可是现在蔡京无声无息的就成了叛逆,连他与友人的信件都被一股脑的焚毁,之后蔡京还想为自己辩论,怎么辩?从开封府的记录中?

这算是暗度陈仓吧!

相对于蔡京的罪名,韩冈能不能在这一次成为枢密副使,根本无关紧要——他本人都将两府中的职位让了多少次了。而定下了蔡京之罪,日后便没有什么能够阻止韩冈成为宰相。

不论这一回能不能选上,韩冈可都是赢家!

章敦望着殿上,王安石已经在询问还有没有想要参加选举或是推荐他人参选。

他随即闭上了眼睛,现在就算想要阻止也来不及了,而且他还没有打算跟韩冈彻底决裂。韩冈作为朋友十分可靠,要是作为敌人,想想就让人心头发凉。过去与韩冈的配合,现在想想也觉得十分愉快。

不再去多想韩冈和蔡京,章敦也终于可以确定,沈括是在配合韩冈。

韩冈和沈括的一搭一唱,成功的让太后同意烧去所有罪证。

虽然沈括的确是请求太后要搜检信件,但结果最终还是太后在韩冈的建言下,亲自下令焚毁。

有了韩冈的配合,沈括的提议才不是得罪所有人的自杀行为。这样的配合就是被人看出来也无所谓。沈括做恶人,韩冈做好人。从此之后,许多朝臣睡觉也能多安心一点。

如果不是这样的结果。即便日后此案审结,也没有牵连到其他朝臣,可那些被搜检出来信件,还是必须封存入档。说不定哪一天,这些信件就会被政敌翻出来。

在几位叛逆收藏的书信中,相对于被沈括点名的数人之外,普通的朝臣还是占了绝大多数。沈括并没有点出他们的名讳,只要信件被烧掉,他们就能高枕无忧。谁也不想被人日后在给刑恕、蔡京、苏轼,又或是曾布、薛向的私人信件中,抠着字眼,从里面找出叛逆的证据。

就是韩冈行事的风格变化,让人一时间难以适应。在过去,韩冈的行事风格会更大气一点,而不是用这样的小伎俩。

“可还有人要举荐?”

王安石提声询问,让章敦惊醒过来。木已成舟,此时再多想也没有意义。

三问之后,无人作答,候选者的推举算是结束了。

写在屏风上的名字,到了最后,也只有李定、吕嘉问、韩冈、曾孝宽四人。

只要不糊涂,这两天都该知道中选者将会在那几人中产生,也会知道,重臣们手中的选票早就被瓜分殆尽。

登名上去之后却一票也没有,未免太难看了。做陪客没什么,但丢人现眼可是绝大多数人都敬谢不敏的一件事。除非与对方有仇,否则谁也不会出面提名。且就算被提名了,也肯定会拒绝上场丢脸。

选票早已发了下去,宰辅们没有资格干预选举,与学士和侍制们一并被赐座。有选举之权的重臣,又被赐桌,一套笔墨纸砚放在小桌案上。

韩冈低头看着自己面前的选票。

侍制及以上官有选举权,而两制官及以上才有被选举权,现任的宰辅则是什么都没有。韩冈前几日复授资政殿学士,加翰林侍读,作为选举者和候选者,手中也有一票,而且完全可以投给自己。与他相同,同样来参选的三人,手上都有着选票,只是不知道他们都会选择什么人。

所谓选票,基本上就是上奏给天子的章疏形式,里面已经印好了文字和花样,只要在空缺处填上姓名。此外,就必须加上自己的姓名。作为当朝重臣,必须要为自己的选择负责,韩冈也不想让下面的人来个匿名投票。南北之争,必须要明明白白的表现出来,才能让太后有所感触,也能让韩冈聚拢更多的人心。

“意欲推举何人,只需在空格处写上其人姓名。并在卷末,签名画押便可。若以上候选四人皆不合意,则空缺不书,只需签名画押。”

王安石最后提醒的规则,早就为所有人熟知。几乎每一人都做好了选择,拿起笔后,无人犹豫,一个个落笔如飞,转眼之间,便放下了笔,将选票折好,递给来收取内侍手上。

韩冈写下自己的姓名,熟练地画上了押记,然后折起来交给等候已久的内侍。

这就是在选举枢密副使?

王安石站在臣子班列的最上首向下望着,暗暗叹息。若是换一个场景,也许就是那两家专司赌博的总社的会首选举了。

朝廷的脸面也不用讲了,过去的制度也废掉了。从今以后,每当两府出现一个空缺,接下来就是四处封官许愿,向人求票。而有中有选票的侍制,又能待价而沽,货比三家,将自己手中的选票卖上一个高价。

说不定,这样的推举日后还会扩大,扩大到宰相、扩大到平章,那时候,买上一票又得有多少钱?

这就像买龘卖官职,一开始拿出来的只是不起眼的小官,可时间日久,为了能满足更多的需要,朝廷拿出来的官职就会越来越大。汉灵帝不正是这样?为了聚敛钱财,就连三公九卿都能卖,整个朝廷从上到下,连一点体面都不留存了。后汉之亡,实肇于桓灵,这一句话,无论谁来看,都是没有一点错的。

可王安石也知道,下面的每一名侍制都不会愿意放弃刚刚到手的权力。这就跟宰辅一般,没事谁还会觉得自己手中的权力太大了?只会觉得小。官做得越大,心思也就越大,当韩冈给了他们打开了门,让他们得以跨进之前无法涉足的地域,若再有人将他们赶回门后,不论之前对个人有多少恩德,在朝堂上又多高的威望,那就是死敌。

片刻过去,选票已经集中在了台陛阶前,由一名小黄门用托盘托起。选票数量并不多,但分作两叠堆放,看起来也是很有些份量。

王中正受命唱票,从台陛上下来。

殿中众人的目光都汇聚到了他的手上,原本已经很是安静的殿堂中,此时更是静得能听见呼吸声。

拿起放在最上方的一份选票,王中正双手展了开来。

“宝文阁待制、礼部郎中、判鸿胪寺陈睦——”

数百人的期待中,王中正抑扬顿挫的念出了最后的落款,让人知道究竟是谁投出的这一张票。视线左右横扫过朝堂,稍停了一下,他念出了受选者的姓名:“吕嘉问。”

屏风前的一名内侍随即拿笔饱蘸了浓墨,在吕嘉问的名下写上墨迹淋漓的一横。

吕嘉问的嘴角顿时多了些笑意,这是开门红,再吉利不过。这一票就在他预期之中,只要能够一票票的积累下去,就是

远远地超越韩冈。就是太后想要偏袒,也得顾忌人言。就算之后会恶了太后,很快就被赶出京龘城,但清凉伞的资历有了就不可能再抹去。等个几年之后,天子成年,机会也就来了。

至于什么弑父……总有不在意的臣子,也有不想在意的皇帝。

