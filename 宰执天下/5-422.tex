\section{第36章 沧浪歌罢濯尘缨(24)}

吕嘉问叹了一口气:“说真的,这三十万钱绢发下去,还不一定能消停。河北、陕西和河东都看着呢,京营能闹,他们就不会?京营毕竟没怎么上阵,看看伤亡,去一百个,回来还能有九十九,这是打仗吗?子厚当年领军去岭南,路上死的人都不止这个数吧。”

“好吧,下次写信给韩玉昆,我代你问他,为什么只让京营死了这么点。”

“真要是死伤多了,还要添一笔抚恤。”吕嘉问皱眉想了想,顿时不寒而栗,“真要是那样,我还是递辞表出外算了。换萧何、武侯来做差不多。”

章惇摇摇头:“这个烂摊子,萧何之才,武侯之智,都没办法……只盼着接下来几年能休养生息了。”

“……其实也不是没有办法。”

“嗯?什么办法?”章惇立刻追问道。

吕嘉问摸了摸袖袋,掏出一枚铜钱来。比寻常的小平钱大了一圈,厚了些许。章惇眼尖,一眼看出那是一枚元丰重宝。面值五文,俗称折五钱,一枚抵五枚小平钱的大钱。这折五钱才出来没多久,通行于世也才一年多的模样。

章惇两根手指拈起来正面反面看了一阵,想到了什么,神色突然一变:“望之,难道折五钱还不够,还要铸更大的不成?!”

“当二十、当五十的都有人提了。是度支司的人。”吕嘉问摇了摇沉重的脑袋,章惇猜个正着,“五枚小平钱用的钱料,就能铸一枚折十大钱。这是一倍的利。当二十、当五十的利钱就更高了。”

三司衙门下辖户部、度支、盐铁三司,由此而得名。其中度支司便是掌管朝廷收支用度的衙门。

“折五钱都不该铸,何论折十钱?!当二十、当五十就更不该想了!”

王莽做过的蠢事,怎么能学?何况仁宗时为了应对西夏战争造成的亏空,也曾经铸过大钱,铸过铁钱,可结果也仅仅救急,折十的大钱很快就贬下去了,铁钱也只能两枚、三枚的抵一枚小平钱。

折五钱论其中含铜量只相当于三枚半小平钱,纵然朝廷要求是当五文钱来使,可百姓的眼睛是雪亮的。发行才半年的时间,在市面上,折五钱就从一开始五文降到四文,再从四文降到三文钱在用。这个币值一直保持到现在。

当二十、当五十的大钱铸出来,朝廷拿这个钱发俸买物,等于是明抢。远的不说,京城内肯定要先乱了。

章惇沉着脸,“现在还没到饮鸩止渴的地步!”

“这还不都是打仗打的?库中没钱,度支司最苦。幸好这一战结束的早,要是再多打半年,当十、当二十的大钱说不定就要出来了。”

陕西、河北、河东,三个战场,大宋的北方全都卷进了战争之中。参战的禁军、厢军和乡兵,总数超过五十万。如此规模的一场大战下来,无论哪朝哪代,国家财计都会变成一个烂摊子。三司下面的储备,要留下很大一部分来救济河东、河北遭受兵灾的百姓,还要应付每月的俸禄支给。人穷志短,有人动歪心思很正常。

“政事堂那边肯定不会同意的。”

“现在当然不会。等到需要钱又寻不出财源的时候,子厚你看会不会!?”

如今新党当政,绝不会留口实给旧党。除非被逼得没有办法,否则绝不会铸造更大面值的钱币。可一旦遇到无法解决的危机的时候,旧党的那些人说些什么也就无关紧要了。

吕嘉问举杯喝了一口凉汤,放下茶盏后又问章惇:“对了,今天子厚你邀我过府到底是为了何事?……可是吕吉甫和韩玉昆二人的事?昨天见平章时问过了,平章还是觉得让他们在外面多留一段时间比较好。”

关闭<广告>

吕惠卿远在陕西,章惇已入两府需要避嫌,曾布更是旧怨未了,重臣之中,现如今只有吕嘉问与王安石走得近,能时常往平章府上走。有时候也帮章惇、王安石之间互相带个话。此时交流最多的话题,还是有关韩冈和吕惠卿二人。

正常来说,一场大捷之后,肯定要观兵御前,献捷陛下。主帅都会被召回京中,连同功臣一起受到天子亲自赐予的褒奖。但宋辽之间刚刚达成和议,为了顾全辽国的面子,以免其恼羞成怒,不宜大肆庆贺。同时也免得给辽人侦知三路主帅同时回京,让河东、陕西和河北的军情又起反复——这是两府阻止皇后将吕惠卿和韩冈立刻召回京中的理由。为此,甚至两人的封赏都没定下,一旦传诏封赏,两人要求入京谢恩就麻烦了。

虽然十分牵强,可既然皇后都认可了,远在千里之外的两位辅臣都只能望而生叹。可那也只能拖上一两个月。

“不是这件事。既然平章决定了,章惇也没什么好说的。”章惇给吕嘉问提醒了,坐下来说了一大通话,还都没说到正题上,“介甫公既然不在意自家女儿还在韩家做新妇,我等也没必要替他担心不是?”

从章惇的本心上,其实并不希望有人过来分自己的权柄。

正常情况下,枢密使和知枢密院事都是枢密院的主官——知枢密院事稍低半级——但两者不会同时存在。也就是说,不可能即有枢密使,又有知枢密院事。只有熙宁初年,文彦博为枢密使时,朝廷又升了陈升之为知枢密院事,这是唯一的例子。如今章惇便是援引此先例,加之枢密使吕惠卿又受命领军在外,他才得以成为知枢密院事,执掌西府。

只是当吕惠卿和韩冈回来后,又会是什么样的情况?一个是开疆拓土,一个是拯危救急,放在往曰,两人的功劳升做宰相都绰绰有余。可宰相只能补上一人,韩冈争不过吕惠卿。资历上的差距让韩冈没办法越过吕惠卿。而且吕惠卿的官位本来就在韩冈之上,理所应当先一步升去东府。

吕惠卿做宰相,章惇自觉有运筹之功,可以接替吕惠卿的枢密使一职,而韩冈则接任知枢密院事。看起来两人并立,曰后少不了在枢密院中争权。可章惇清楚,韩冈的目标是广大气学,回来之后可是有得忙了。哪有时间与自己相争?吕惠卿一起回来的话,韩冈更是没空了。

这段时间以来,章惇和章楶一直都有联系,与韩冈同样没断过书信往来。只是韩冈既然并没有写信来明着求助,章惇也乐得不去招惹是非。暂且看着,等时机到了再出手不迟。

“那是什么事?”

章惇抽一份报纸,就是之前他让儿子找出来的那一份,指着上面的图表:“望之,你看到这个了吗?”

吕嘉问瞟了一眼,“这不是平章说要彻查的吗?!可要是查出来就见鬼了。”他嘿的冷笑一声,“真想要不漏消息于外,先把石得一那阉竖杀了再说。”

皇城司和两大报社交换消息的事,在上层并不是秘密。章惇知道,吕嘉问也知道。

“杀了他也不管用。换个人来做,一样少不了要借重两家报社的耳目。两家报社也是,有皇城司通消息,现在世间都说他们是为民喉舌了。”

“为民喉舌?”吕嘉问失声笑了起来,“台谏也是自诩为民喉舌呢。这让两家怎么不打架?……一路货色!”

“谁说不是?”

两人都是被台谏盯着咬过的,对乌台的成见根深蒂固了。

御史们喜欢拿着百姓为自己张声势,弹劾时动不动就说百姓皆怨,民生困苦,朝廷用人之误一至于斯。这样的台谏官,要说十成十,倒是有点绝对了,可要说是九成如此,那绝不是冤枉人。

绝大多数的御史,只有在实现自己的目的上,才会有为百姓说话。比如名声、人望或者是成就感,又或是为了后台而上书。这还是好的。毕竟是有了好结果。

而更多的情况是打着为民说话的幌子,来实现自己的目的。那样的情况下,为民喉舌的姿态也只是一个伪装,实际上连个好结果也没有。

那等行事中允平正、不为私心而上本的御史,章惇没见过,吕嘉问也没见过。

“望之你可知以两大报社势力之广,背后的京城贵胄富户几乎又都合在一处,还能与皇城司暗通款曲。一篇文章百姓看得见,士林看得见,朝廷看得见,宫中也能看得见。如此声势,却在御史台前低头认输?”

“能不低头吗?那可是御史台!”

“没错,因为是御史台!因为他们怕!从骨子里怕!”

两家报社的后台都是明摆着的,宗室、外戚、勋旧还有富商。他们在朝廷中的政治地位其实很低。就算其中有王公侯伯,有陶朱猗顿,也比不上一名御史说话的分量。

谁敢试一试诏狱之威?名满天下如苏轼,一封弹章便让他在狱中蹲了小半年。两大会社的会员们个个身娇肉贵,谁会愿意招惹乌台的那群咬住就不松口的疯狗。

当年变法的时候闹起来,是因为夺了他们的财路。尤其是市易法施行,吕嘉问主管在京市易务,也不知有多少家国戚勋贵扎了他的草人。但现在有了新的财路,哪个还不去好好享受,去闹个什么?
