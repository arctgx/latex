\section{第13章 晨奎错落天日近(六)}

“足够久了?”韩冈重复了一句,然后点点头,“吕吉甫若是要在河北边境上做点事,时间的确足够了。.”

王安石静静的看着韩冈。韩冈的话中讽刺意味太重,与以往总是胸有成竹时的态度差了太多。

“挑起边衅,只为了让吕惠卿重回两府。岳父,这手笔未免太大了一点。”

“不会打起来的。”事到如今,王安石也不用瞒着谁,“官军打不了,辽人也打不了,吕吉甫也没考虑过要大打一场。”

“对,吕吉甫只是想要做个样子而已。”韩冈依然是尖酸刻薄的口吻。

“玉昆,你失态了。”王安石叹道。太少见韩冈这般冷嘲热讽,他的作风一贯是单刀直入的。

“当然会失态。”韩冈笑了起来,“这件事上,是岳父你错了。岳父你这一回私心之重,小婿始料未及。在过去,即使是新法中不合人意之处,韩冈也都是认为平章的初衷是好的,但这一次,完完全全看不到有任何公心。”

王安石不为所动:“此事无害于国。”

无害于国?

韩冈冷笑。

如果一切如王安石、吕惠卿所愿,烽烟不起,当然对国家无害。可边境上的冲突依然少不了,军民伤亡如何能避免?

但在高高在上的大人物的眼中,那些牺牲,只是单纯的数字而已,不过是要付出的成本——在王安石看来,正是于国无害。

韩冈也不会为此而指责王安石什么。

“可在小婿看来,恣意妄为的边臣,却是对皇宋的未来不利。”韩冈冷冰冰的说着,“幸好,还有挽回的余地。”

“想靠那刘舜卿吗?”王安石反问。

……………………

刘绍能站在寨门上方,望着黑暗的北方,身后一名小校低头恭声,

“都监,知州请都监去州中,有要事商议。”

刘绍能缓缓回过头来,打量着这名从州城匆匆赶来的小校。

“诺。”他应声。

用微笑迎上小校惊讶的目光。

知州刘舜卿要将自己召去州城,究竟所为何事,刘绍能用脚趾头想都能想得到。

一旦进了城,没几天功夫别想脱身。

但有些事,刘绍能已经安排下去了,刘舜卿的动作实在是太迟了。

……………………

“刘舜卿会怎么做,小婿并不清楚。”韩冈摇头,“以他的姓格,只会去做他该做的,包括禁止属僚挑起边衅。或许迟了,或许早了。”

王安石漠然以对,扭头看着夜色笼罩下的皇城。

“当然了,岳父和吕吉甫也不一定需要挑起边衅,只要辽人那边有些异动就够了。”

“何为异动?”王安石头也不回的问着。

“十万大军叩关可以算,千余骑兵行于界上,同样也可以算。八千皮室在彼处,为边事出来一两千撑腰,此事年年都有。”

……………………

“刘绍能还没到吗?!”

刘舜卿在院中来回踱着步子,步伐快而重,偌大的院子,七八步走到墙边,转过身,再七八步走到另一堵墙下。来来回回,走了不知多少圈。

他接到消息的时候太晚了,他实在没想到刘绍能和他背后的那位大人物会这么心急。

“应该快了。”一名部将低声回道。

“快了是多久?!”刘舜卿停住脚,扬眉瞪眼的暴喝道。

部将连忙说着,“末将已经派人去探,马上就会有回报!”

重重的哼了一声,刘舜卿再度踱起步子。

早在两个月前,便有一部辽军进驻涿州各县,与雄州隔着一条白沟对峙。

经过细作确认,来到白沟对面的涿州的皮室军有八千之多。而细作的回报中还说,他们打听到其中有一支是从高丽撤回来的精锐,皆是人马贯甲的具装甲骑。

本来大部分雄州的军民还不知道出了什么事,等到耶律乙辛称帝的消息传来,边境上加强守卫,一切便真相大白。

流言随即甚嚣尘上,如今有有谣传说整个南京道上,总共增加了十万大军,耶律乙辛打算篡位称帝之后,从大宋这边抢上一把作为庆贺。

只是也有消息说,其实南京道没有更多的兵力。耶律乙辛为了顺利称帝,已经将南京道上的驻军调去了上京临潢府去了。这八千皮室军,只是虚张声势,

南京道是辽国的财赋汇聚之地,也不像其他四京道上,被各大部族占去了大片的土地,几乎是完完全全属于大辽朝廷、大辽天子。

一直以来,辽国皇燕京通过朝廷派出的文武官员,牢牢的控制了此地。然后通过此地得来的财赋,来控制广及万里的疆域。失去了南京道,辽国三五年间就会分崩离析。

当年耶律乙辛主控朝政,也是设法先掌握了南京道,由此奠定了弑君篡位的基础。等到弑君之后,清剿东京道的反对者,耶律乙辛也没少从南京道调兵。

相反的谣言,却都有各自的道理。

但听出巡的马军回来说,遇上的对面巡卒,跟之前的都不一样了。

过去的几十年间,雄州边郡巡逻国界,与辽人的逻卒遇上时,还能打个招呼,说两句笑话,甚至相互交换点特产。据说还有交情好的,能坐在一起喝点小酒,一起骂骂各自的上司的。

雄州此处塘泊众多,原本是黄河及其支流破堤之后留下的河塘,在真宗时便加以修筑,使之成为阻拦辽军南下的一道天险。从真宗开始,直到如今,这样堤防整修工程始终没有停止过。可是在一切都冻结的冬天,越过这一条塘泊防线,就太容易了。

一旦刘绍能挑起边衅,辽人的大军随时可能会杀过来。

正常的交锋,刘舜卿绝无二话,拿了朝廷的俸禄,就该好好做事。但为了某个大人物的野心,去与辽军对垒,未免太冤枉了一点。

刘舜卿狠狠的跺着地砖,仿佛那长条形的砖石是刘绍能和他靠山的脸。

他可从来没从吕惠卿手上拿过一文俸禄!

……………………

“这边厢吕惠卿大喊着要攻打辽国,讨伐逆贼,那边厢就边境告急,辽军准备入寇。”韩冈指着远处灯火辉映的地方,那里是不夜的东京城,“人人都会清楚,这必然是吕吉甫私下里做的手脚。”

“无害于国。”王安石道。

“更是查无实据!”韩冈补充道,“即便有实据,也查不出来,”

“行事岂能畏避人言!”

“人言士论,吕吉甫岂会在乎?而且结果只会是吕吉甫想要的结果。”韩冈摇头笑,“岳父当心知肚明,士论清议之后必定会站在吕吉甫的一边。否则岳父和吕吉甫这般辛苦又是为了什么?”

……………………

临近年节,吕惠卿的妻妾们正给给家里年幼的子侄和孙辈们准备过年的压岁钱。

红绸袋装起小小,里面是几枚钱币。数量虽少,却都是铸币局新造的金银钱。

吕惠卿轻轻拈起一枚金钱。

钱有半两之重。中无方孔,形似小饼。

这是韩冈说了很久的模锻压制成型的钱币。

钱上纹路精细,背面的如意图样,正面的元佑元宝字样,还有外廓上的小齿,都是一丝不乱。

小小的金钱,完美的犹如一件工艺品。而银钱同样如此精美。

要不是听说铸币的模子损坏严重,铸币局早就将金钱、银钱推广出去了。

现在这样的金银钱,尚不能公行于世,只能作为压胜之用。

朝廷赐予重臣,而吕惠卿又给了家里的孩子。

岁岁年年,都少不了这一回。

是的。岁岁年年!

吕惠卿将金钱丢进装钱的小篓子中,叮当一声脆响。

岁币是皇宋立国以来最大的屈辱。

兄弟之邦只是一个名分,而岁币才代表着宋辽两国之间真正的关系。

如果有哪位宰臣能够废除岁币,立刻就是天下人心目中的英雄。

之前就算是击退了入寇的辽军,夺占了灵武之地,还在西京道上啃了一口下来,朝廷也没有废除岁币。

但这一回耶律乙辛篡位,给了朝廷最好的借口。

无论哪位宰辅,都有心借此良机,废除旧曰盟约,不再向辽人提供岁币。

而处在河北的吕惠卿,正好有着绝佳的地利。

只要为此打上一仗,甚至不要打仗,仅仅是调动了辽军兵马叩关,这份功劳就得算在他的头上。

那时候,即使是京城中宰辅们都说要废去岁币,功劳最后也不会落到他们头上。

难道在安全的地方动动嘴皮子,比得上实际临战的功劳?只要朝廷不再奉上岁币,任谁都会说这是吕惠卿击退辽人的功劳。

就算打不下一座城池,甚至大军未向北越过界河一步,这功劳都是他吕惠卿的。

而断绝了岁币,辽人会不来吗?

本来就不惧辽人入寇,又有了火炮装备军中,还担心挡不住契丹铁骑吗?

他所要做的,只是改变一下先后次序。

将朝廷断绝岁币激得辽人大动干戈的事实,说成是因为自己的进攻才结束了耻辱的岁币。

只消倒因为果。

只要先行动手。

……………………

“成为宰相不过等闲,回到京城更不是难事。”韩冈望着天上,没有了月光的干扰,银河比平曰分外清晰,“吕吉甫需要的,岳父想看到的,是能够和小婿抗衡的声名。”

“非是小婿自大,但如果只从名声上,吕吉甫的确差得小婿太远。正常情况下,他永远也比不上小婿。除非曰后有什么变化,让小婿身败名裂。”

王安石静静地听着,任凭韩冈仿佛自言自语的述说。

“可这一回,耶律乙辛给了他一个机会。耶律乙辛篡位,断绝岁币供给是既定事实。可如今吕吉甫一力主战,一旦辽人大军压过来,即便仅仅是威吓也好,吕吉甫只要将之拒之门外,废除岁币的功劳却能全在他一人身上。啊……还有岳父。但想必岳父是不会与吕吉甫争功的,新学需要吕惠卿。”

王安石眼皮动了一下,可沉默依然,并不去评价韩冈说的是对是错。

“但有一点,岳父和吕吉甫大概弄错了。”

韩冈语气沉了下来,转身看着王安石,双瞳映着不远处的灯火,闪烁如星,

“辽国皇帝的确需要岁币。富彦国当年出使辽国,曾经对辽主道,若辽宋通好,皇宋以岁币赠之,则‘人主专其利,而臣下无获’;如若两国交兵,再无岁币,则‘利归臣下,而人主任其祸’,辽主当然会选择岁币。”

“试问没钱怎么使唤人?军中的神臂弓手,齐射一次都要记一份功,有一份功,就得有一份赏。辽人也好,武夫也好,忠义之心比不过对财货的贪欲。有了大宋每年送上的五十万银绢,辽主就能收买诸军、诸部人心,牢牢的控制住国政。”

“耶律乙辛当年也需要岁币,作为权臣,最不能丢的就是财权、军权。没了岁币来收买人心,他连三五千兵马都控制不住。他更不能丢了岁币,否则连名声一并都会丢掉。”

“可是有一点,小婿想问一下岳父。”韩冈盯着王安石的双眼,缓缓道:“耶律乙辛,他现在还需要岁币吗?”
