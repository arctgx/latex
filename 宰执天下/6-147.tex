\section{第13章 晨奎错落天日近(12)}

“不过话说回来。”韩冈说道,“章子厚去家岳府上,会是什么结果还说不定,或许比我们想的要好也说不定。”

“章子厚若当真无意,就不该去令岳的府上,至少去之前该派人过来说一声。”

“上门归上门,买卖不一定能成啊……”韩冈道。

“两头喊价,价高者得,这是买扑啊?”苏颂哼了一声,“当真如此,倒可谓是扑枢了。”

韩冈失声而笑,性格敦厚的苏颂嘴巴刻毒起来的时候,也分毫不输人。

买扑是从国初延续至今的包税之制。对酒、醋、陂塘、墟市、渡口等处的税收,由官府核计应征数额招人承接,是为买扑。而买扑的过程中通常会有两家或多家竞争,一般是以价高者得。

章敦先从韩冈这里得到了晋身宰相的许诺,掉过头来就拿着韩冈的报价去了王安石那边。看到韩冈的报价,王安石怎么敢不拿出点好处来安抚章敦?但这样的做法,的确是跟买扑别无二致。

军班出身的狄青做了枢密使后被称为赤枢——赤佬的赤;而先附和王安石而得以拜相,后又奏请废除制置三司条例司,所谓得鱼而忘荃的陈升之则是被称为荃相,加了一个章敦的扑枢,可谓是鼎足而立了。

“不过,玉昆。”苏颂放缓了声调,“不必着急啊。令岳与颂一般年纪,垂垂已老。吕、章二子亦早非新近,日暮不远,你却青春正好,不必急于争一时短长。”

“子容兄肺腑之言,韩冈必铭记在心。只是……”

“因为令岳?”苏颂问道。

在外人看来,才三十出头的韩冈的确有的是时间,现在还是打根基的时候,再有一二十年的时间,气学根基已固,整个朝堂局势都在他的影响之中,那时候,还有新党什么事?

韩冈本也是这么想的,所以他无论是在枢密院还是在政事堂中,都没有太多的争权夺利的行为。可惜王安石也明白这个道理,总想着让遏制气学的发展,还竭力培养吕惠卿这个接班人,尽其可能的不让韩冈和气学出头。

“不如直接呈与宫中,请太后做决断。”

韩冈摇了摇头,“边地不稳,总得有人在河北坐镇。章子厚不愿去,只有吕吉甫了。实在找不到人,请动太后也没办法。”

苏颂叹了一口气,韩冈不想过多的借助太后的力量,从他提出推举宰辅一事上,就可见一斑。否则轻轻松松就能回到两府之中,何须那般麻烦。

两人正说着,一名伴当匆匆而来,在韩冈耳边轻声说了几句。

韩冈双眉一皱,对苏颂道:“章子厚出来了。”

“可还真够快的。”苏颂惊讶,问道,“是不欢而散?”

韩冈笑道:“这可打听不到了。”

“玉昆你觉得会是哪般情况?”

韩冈想了想,“往好处期待,往坏处准备。”

苏颂闻言,哈哈笑了起来。

的确是要往坏处准备,否则他跟苏颂在一起商议做什么?

韩冈给了章敦做宰相的机会,不过宰相手中权力的大小,却不是来自于一个同中书门下平章事的头衔。

参知政事压倒宰相的例子不胜枚举,根基不牢的宰相很容易被架空。背离了自己扎根的新党,就像鱼离开了水一样。同中书门下平章事的名头虽好,也要实权在手才能算是有意义。

若是上了韩冈的贼船,好吧,是得到韩冈的支持之后,作为立身之基的新党,是跟随自己分裂出去,还是彻底抛弃自己,章敦本心里不可能没有疑虑。

何况韩冈曾经提到过的一点想法,也肯定让章敦心中生畏,毕竟韩冈的想法不是简单的权臣,又或是文彦博所说的与天子共治天下,已经可以说得上是悖逆了。章敦好端端的,也没理由去冒风险。

一开始的消息中也说了,李定同时被拉了去,似乎是作陪。王安石此举有要挟之意,但若李定是以老朋友的面目出现,而不是以御史中丞的身份出现,章敦的逆反心理不一定会被挑起,却肯定会好好想一想如何选择了。

“看来玉昆是胸有成竹了。”苏颂笑罢说道,“若变成坏结果,打算如何坏令岳的好事?”

韩冈嘴角微翘。苏颂猜得的确不错,如果章敦当真有所反复,再想要与王安石、吕惠卿相争,的确是难了,但扯人后腿的事,做起来还是简单一点。

一开始韩冈就是这么打算的,再早一点的盘算里面,可从来没有将章敦这个变数算作自己的一方。韩冈邀请苏颂,抱怨也只是附带,他们的时间没多到可以浪费在抱怨之中,

“胸有成竹是不可能的,但好歹有些准备。”韩冈长身而起:“章子厚那边只能等着消息,吕吉甫那边既然想立功,就让他继续看顾着河北,至于家岳,不想过太平日子,那就如他的意好了。”

前面太过于看重朝堂中和平安定的大好局面,总想着做到斗而不破,将章敦拉过来,也是想让王安石和吕惠卿知难而退,现在将目标放低一点,眼前其实还是海阔天空。

……………………

送走了两位客人,王安石一声轻叹,“多亏了李资深。”

幸好章敦没有完全被蛊惑,也幸亏请了另外一名客人作陪。

“大人。”代父送客至巷口的王旁回来了。

“送走了?”

“已经走了。”王旁脸上忧色难掩,进言道:“是否要好好与玉昆谈一下”

“怎么谈?!”王安石脸色顿时一沉,“有太后为他撑腰,他何曾愿意好好说说话!”

出手将章敦拉拢过去,这等于是触到了王安石的逆鳞。在曾布事后,王安石分外容不得有人背叛,而故意引诱章敦背离,韩冈的行为,怎么可能不让王安石怒火中烧?

“原本只是争于国事,他不愿吕吉甫回朝就算了,做什么鬼祟手脚,这岂是正人所为?”

王安石语气激动,王旁紧紧皱着眉。他父亲这般模样,其实很少见,看得出十分痛心。

自家父亲对妹婿的欣赏,王旁比谁都清楚。正因为这份欣赏,让王安石对韩冈绝不会有半点留情。

王旁虽然才智不高,可站得近,也看得比别人更清楚,王安石和韩冈之间的争斗是如何变得激烈起来的,党争也是这样一步步的恶化下去的。

这样下去,又会是亲家成为仇家了。

王安石却没理会儿子,径直走进了自己的书房。在书桌前做下,盯着烛火沉沉的思考了起来。

章敦虽然给拉了回来,但看得出他本心还是犹豫不定。而日后能成为新党之首的只会是吕惠卿,章敦一辈子多半都会在吕惠卿的阴影之下。以章敦的脾气,他肯定是不甘心的。

但投到韩冈那边的劣势更为明显,章敦虽是一时心动,可显然也有着疑虑,否则以他的决断,不至于首鼠两端。

若是在当年,直接就把他如曾布、沈括的旧例给处置了,纵使章敦现在是枢密使,可之后先附和新法,得相之后又反戈一击的陈升之,一样给赶出了京城。若还有当年的权柄,去一章七又算得了什么?可现在却万万不能了。若章敦当真背离,对新党的打击太大,已是承受不起。

不过终究是挽回了,加上河北那边天随人愿,一切总算是恢复了正常。

王安石很是庆幸。

原本局面或许当真会如韩冈所说的那般,没想到还有峰回路转的一天,吕惠卿事先安排的一个伏笔,却砸出了皮室军。

现在情况有变,太后纵使再偏向韩冈,能压制朝堂所有反对者,但辽军可不会听太后的,辽军叩关,又有谁能去镇守边关?韩冈吗?

若能击败这一支辽人驻扎在南京道上的主力,不仅能够动摇到耶律乙辛那个伪帝的统治根基,幽燕也决不是梦想。

若是在自己的手中完成当初的计划,日后也能坦然的去见熙宗皇帝了。

……………………

“将章七说回来了?”

李定回到家的时候,同住的堂弟迎了上来。

李定微微皱了下眉,情知兄弟来问,定是有人委托他打探消息,不过李定对族人一向亲厚,不习惯板起脸来拒绝。

遂随口敷衍了过去:“章子厚心思本是坚定,投效之说只是谣传而已。”

章敦的阴私之事,李定并不打算对外透露半句。别人怎么猜,是别人的事,他可不打算做搬弄口舌的小人。

王安石为了将章敦给拉回来,给他的好处可不小,甚至要比吕惠卿还要先一步进入政事堂为宰相。

王安石以平章的身份去推荐,以自己致仕为交换条件,不愁太后不答应。

韩冈若是阻止章敦为相,登时就是他的死敌了。

而且推举宰辅一事,是韩冈所发明,若是廷推宰相,有王安石率新党众人同举,章敦必然中选。

吕惠卿再回来,还是先从枢密使开始。不过吕惠卿如今在外,第一目标还是回朝,之后怎么转到宰相的位置上,那是得另说。而且到了相当的地位上,手上的权力高低,主要还是得看夹袋中的门人,这一点,章敦远布如吕惠卿。

李定收拾了一下,准备梳洗睡觉,可半刻钟不到,便被人唤起。

来报信的承旨官忧惧带着惊恐,“中丞,辽军叩关了!”
