\section{第13章 晨奎错落天日近(一)}

因为昨日的消息,韩冈起得早了一点。走进待漏院中时,才是五更天。

不过明日是正旦,今天需要处理的事不少,韩冈没睡多久就自动醒来了。

厅中只有韩绛,见到韩冈来,起身走到厅门外,笑道:“待漏院中少见玉昆。”

韩冈先行了礼,道:“总是起得晚,每日都只能赶在皇城门开时才到。也就没什么机会来这里做一做。”

“怕也是喜欢热闹一点。”韩绛笑着,与韩冈携手进厅。

“也不独韩冈一人。”

“的确,也就郭仲通还能陪老夫在这里说说话。”

待漏院是百官晨集准备朝拜之所。宰臣待漏院,位于宣德门西侧,宰辅们参加早朝,抵达皇城外时,可以至待漏院中休息,免得在城门外吹风。

冬天的待漏院,炉火生得很旺,室内暖洋洋的,甚至显得有些燥热了,让穿戴整齐的韩冈很不舒服。

韩冈往日极少往待漏院来,一般都是卡在点上进皇城,懒得进进出出,更不愿早起。

宰辅们在待漏院休息的次数也不多,大部分时候也都是跟韩冈一样,卡在晨钟敲响、皇城城门开启前后,抵达宣德楼下。

但今天韩冈到得稍早,不想在外面吹风,干脆就进来歇一歇。只不过,方才韩绛说的两府宰臣中,除了他外的另外一个待漏院的常客却还没到。

“今天怎么就子华相公一位?”

韩绛知道韩冈想问谁:“郭仲通今天告病了。”

韩冈神情一凛,“什么病?”

郭逵年纪大了,一点小病都是大事。尽管他是武官,可他若告病辞官,依然会对朝堂局势有着很大的影响。

“北病。”韩绛吐出两个字,端起茶盏慢慢的抿了一口。

韩冈愣了一下神,方才明白过来。失笑道,“病来如山倒,还真是一点不假。”

昨天耶律乙辛篡位的消息传来,今天多半就要在朝堂上争个胜负了——新年在即,在京百司连开封府在内早封了印,公务一概等来年再办,但军国大事是不能耽搁的。

郭逵嗅到了风色,不敢卷进来。

不过只要不是当真发病了就好。如今天寒地冻的,室内室外温差很大,一个不注意就会感冒。在这个没有抗生素的时代,感冒若是变成肺炎,就要面对鬼门关了。

韩绛叹了一声:“郭仲通他也是难做。”

没到最后的关头,他表示一下意见,但真要两边对垒,郭逵这等武臣真还不敢乱掺和。

“但他当真这般说?”韩冈笑问。

“对外当然是外感风寒。”韩绛无奈的笑着,又道:“今天怕会有不少人告病。遇到这种时候,告病的总不会少。”

韩冈笑了一声,朝堂中人告病,真病的时候的确不算多,总是避风头的时候更多一点。

“‘知一国之政,万人之命,悬于宰相,可不慎欤?’”王禹偁的《待漏院记》,以青石碑嵌在待漏院的墙上,被岁月模糊了字迹,不过还有一副是吕夷简亲笔所录,挂在正厅中的墙壁上,韩冈念了一句,对韩绛叹道,“能独善其身、谨慎从事,总比胆大妄为要好。吾等宰辅,一言一行,攸关天下万民,岂能不慎?”

可能是想起了罗兀城的旧事,韩绛的神色变得沉郁起来,“的确是要谨慎才是。”他抬眼看韩冈,“玉昆,你当真觉得此时平辽不可行?”

韩冈不指望韩绛能够如何帮自己,但只要他有所倾向,那就足够了。

“这一次,家岳和吕吉甫何曾想过举兵平辽,恢复幽燕?否则就该上一道平北策,将方略说个清楚明白。”

“战端一开,北虏主力南下,其身后必有起事之人。”

“将攻辽胜利的希望寄托在辽人的忠义上,家岳和吕吉甫不会办这样的蠢事。”

韩绛脸色稍变,“哦,那在玉昆看来,介甫和吕惠卿主张对辽开战,会是什么原因?”

韩冈笑了一下,“相公当是已经猜到了,何须韩冈多嘴?”

“……至少不会败,是也不是?”韩绛肃容问道。

“有八九成把握能成,这样的买卖当然可做。”韩冈像开玩笑一般的回答着。

“但……为何玉昆你要反对?”

“兵形如水,把握再大,也保不准一点意外就给输掉。如果再等几年,宋辽两国国势差距更大,那时候,就不是八九成,而是近十成了……何况火炮军势未成,北地防备还没有开始调整,现在开战,还是显得太仓促了一点。”

从一开始,韩冈都不认为开战之后,吕惠卿敢冲着析津府进兵,他依然是意在朝堂。

虽然韩冈尚无法确认王安石是不是觉得把握十足,所以才对吕惠卿的提议顺水推舟,可战争是政治的延续,却不当是政争的延续,他这么做错得是有些过头了。只是自己将黄裳放到西南,明面上的理由,也同样够牵强。

唉。

韩绛再次长声叹息。两府之中,数他对韩冈的战略眼光最为信服,韩冈说不宜作战,那的确是不宜作战。

可是连韩冈都说有八九成把握不会输,那么怎么去说服王安石放弃这个想法?

避开了让人头疼的话题,韩冈和韩绛继续喝茶聊天,到了宣德门开,也没见第三位宰辅来到这间专属宰臣的待漏院中暂歇,已经报病的郭逵当然也没来。

除了郭逵之外,两制以上重臣之中,有一个感冒发烧的,还有一个腹泻不止,又有报称头疼难忍,总之要等几天才能来上工。

怕卷入党争到了这一步,多半是因为对新旧党争犹有余悸。

不过文臣宰辅们倒是都到齐了,没有哪位宰执愿意留下一个怕事的印象,纵使选择中立,也不会投弃权票。

入宫前,韩冈与王安石见了礼,又与匆匆而来的章惇打了个招呼,还见到了与他同来的侄子——嘉佑二年丁酉科的状元郎章衡。

章衡比章惇年长十岁,仕途却不比章惇顺遂,今日却是为了陛辞。章惇向来不喜私亲,坐到枢密使的位置上,也不曾见对家里的亲戚有何照顾。

章衡资历和身份完全可以更进一步,但是若朝中无人援引,也还是回不来。相对于章衡,章惇的另一位族亲,精擅兵事和治政的章楶才更有晋身高位的希望。

章惇大概态度不会变,可万一王安石全力支持开战之议,那么他恐怕就会设法去寻找顶替吕惠卿出任主帅的办法。

章楶现任代州知州,又有军功在身,有他在河东支持,章惇想争夺一下伐辽的主帅之位,几率不比吕惠卿要小。

而王安石,韩冈就不想多考虑了,自家岳父已经做出了决定,想要说服他,大概只比登天简单点,尤其自己的理由还不那么充分。

入了皇城,一班朝臣就在垂拱殿中等待着太后鸾驾的到来。韩冈站在自己的位置上,等着净鞭响起,思虑依然没有脱离即将到来的争论。

时间一分一秒的过去,当韩冈从思绪中惊醒,发觉时间过去了太久了一点。

为什么太后到现在还没到?

一种既视感让韩冈陡然心悸。

正当韩冈准备出来探究真相的时候,一名内侍慌慌张张的从后殿进入大殿中。

是杨戬。

殿中的每一位朝臣都认识这位跟在太后身边的阉宦,但依照上朝的规矩从来也不该由他先出来,而且后面更不见太后。

平章军国的王安石神色大变,“太后!杨戬,太后怎么了?!”

杨戬左右看看,想凑近了低声告知王安石。

却听到韩绛的一声断喝,“还不快说!”

杨戬吓得脚一软,差点没滚倒,肚子里的话也给惊出来了,“太后有恙,方才在来垂拱殿的路上晕倒了!”

垂拱殿中,顿时一阵嗡嗡的窃窃低语响起。

韩冈、章惇同时出列。

“今日谁人殿上当值?”韩冈点起殿中的班直头领,“去通知王厚、李信,严守宫门,若无两府签押关文,不得放一人出入!”

“种谔!”章惇大喊着统辖天武军的太尉之名,“还不速出殿去,守卫宫禁?!”

拧脾气的种太尉都没空对章惇的呵斥皱眉,三步并作两步,咚咚咚冲出了殿。

殿中气氛仿佛有铅汞压着。

这怎么回事?

太后当真是生病了,还是又出了什么意外?

要是当真生病,太后的病情又怎么样了?如果没病,又是谁做了手脚。

曾经的记忆在许多朝臣的脑海中泛起,很多人在午夜梦回时,还记得血溅朝堂的那一幕。难道今日要旧事重演?

两府宰执相互间交换了一个眼神,韩绛提声道:“太后有恙,我等宰辅当去觐见太后!”

底下的朝臣,搔动声更大的几分。

若是天子有恙,宰辅直接去寝宫问安,一点问题没有。可现在是要去太后的寝殿,男女有别,可是一点都不合适。

要不然,为何太后面见朝臣,要隔上一重屏风?

但韩绛却不在乎,与王安石交流了一下,便对张璪道,“邃明,你来押班,退朝后便过来。其余宰臣,随老夫入内觐见。”

杨戬愣愣的站着,不知道该拦还是不该拦。等到王安石和韩绛走到身边,他下意识的探出手,“平章,相公,此事……”

话未说完,却被王安石随手一推,给推到了地上。

王安石昂然而过,韩绛也视而不见。

韩冈紧随前面的王安石、韩绛,走过杨戬的身边,低声抛下话,“还不快起来,前面领路。”

脚步却不停,越过他,跟着向后宫过去。
