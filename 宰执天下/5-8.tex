\section{第一章 庙堂纷纷策平戎(八)}

【很对不住各位书友,有事耽搁了。今天补上。第一更】

韩冈并不知道王舜臣犯下的蠢事给有心人爆了出来,也不知道吕惠卿和王珪已经通过徐禧达成了默契。他只知道已经是腊月廿五了,灶神都上天两日了,他却不能休息下来。

在京百司都在送灶神的那一天过后锁了印,按常例只需要安排人轮值就够了。但为了筹备即将面对的战争,为大宋骑兵提供战马的群牧司却照样要上工。

尽管群牧司这两年提供的军马,基本上全都是买来的,不过茶马互市的贸易也是有着管辖权,分派军马的差事同样在职权范围之内。总之,涉及有关军马的文案中都绕不过去群牧司衙门,必须要有群牧司的大印盖上,以及群牧使和同群牧使的签押。所以韩缜和韩冈都偷懒不得,下面的官吏当然也放不了假。

这就是官僚社会的特点,无论多么的无稽,多么的没效率,该盖的章,该走的流程,一个都不能少。

韩冈曾经在章惇那里看到一封唐时的诏书——收集前代的诏书也是这时代文人的爱好,——上面从天子,到尚书、中书、门下三省诸位宰相,再到实际经办的官员,以及抄复归档的书吏全都在上面留了名字,每一个章,每一个签押,加起来,比正文还要长。而眼下的官衙中,并不比唐代好到哪里,甚至更麻烦。

韩冈骑着马带着伴当,按时抵达了群牧司衙门。

群牧司中正是忙乱的时候,大小官吏奔走在庭院和走廊中,繁忙的情形跟政事堂中差不多。不过政事堂是乱中有序,而这边则是无头苍蝇。

平常的群牧司清闲得要命,却油水丰厚,朝中的官员和宗室,皆是尽可能的将自家子弟往群牧司里塞,反正不需要他们做事,知道领钱就行了。可眼下要做正经事,而且是急着要办的,这些滥竽充数的官员被逼得鸡飞狗跳,全都没了招数。

看到院中一个个慌慌张张、却不知该做什么好的官吏,韩冈终于真切的感受到战争终于又来了。过去不论是在西河,还是在广西,也不管事前到底做了多少准备,在开战之前,军队的驻扎地总是提前一步变得兵荒马乱起来。

群牧司中的乱象,让韩冈知道河北的轨道,还有赛马赌马,肯定都要放一放了,一切都要给即将到来的战争让路,没有人有多余精力去顾及这些事。

这就叫做计划赶不上变化。

韩冈摇头失笑,许多事本就是谋事在人,成事在天。计划和现实背道而驰的情况,也不是一次两次了。反正这两件事,做成了对国家有好处,做不成对他本人没影响,拖个几年时间,倒是无所谓了。

到了正厅中,依然是乱哄哄的一团。倒是韩冈的出现,让厅内顿时安静了下来。无论官吏,都上来拜见韩冈这一位同群牧使。

韩冈坐在正位之侧,低头看着向他行礼的几十名官吏。如果是在熙河、广西或是京西,自主持的衙门中,官吏跟这一窝烟熏的马蜂一样,他可是不会轻饶。可惜有吕公著和韩缜在,他就不便越俎代庖了。

左右看看,不见韩缜出现。韩冈心道果然如此,要不然衙门这么乱,韩缜早就该出来呵斥了,“内翰呢?”他随口问道。

厅中领头的一名勾当公事出来答话:“回龙图的话,内翰今天上午当在枢密院轮值。”

韩缜和韩冈同姓,担任的职位只有一个‘同’字上的差别。加上两人身上都有学士衔,衙门里面的僚属,便称呼韩冈为龙图,翰林学士的韩缜则是内翰——翰林学士为天子私人,又称内制,故而简称内翰。

“午后回来?”

回话的小官有些迟疑:“当是处理完西府的公务就会回来。”

恐怕今天就不一定能回来了。差不多已成定局,要筹办的事很多,但三位西府执政肯定会有大半时间的在崇政殿中,韩缜兼任的枢密院都承旨,是西府的大管家,只会比群牧司更忙。

兵马从来都是合在一起说的,枢密院与群牧司也不能分开。但凡群牧使一例都是兼任枢密院都承旨,更确切点说应该是反过来,是枢密院都承旨都会兼任群牧使,此外枢密使也会兼任群牧司的最高长官群牧制置使,孰为主孰为次,分得很清楚。

韩缜虽是不在,但韩冈也不可能趁机做些捞过界的事。这几天来,他一直都努力在做个合格的橡皮图章。

让下面的官吏各自去做事,乱就让他继续乱去,韩冈往自己办公的东厅走,随口问着紧随在身边的书办:“今天还有什么公文要签押的?”

书办弓着腰答话:“有二十余份,都已经送到东厅去了。”

“内翰是否都已经签阅过了?”韩冈问着,走进了东厅所在的院子。

书办跟着进来,他本就是群牧司安排在韩冈身边听候指派的胥吏:“有十一份是从枢密院转过来了,昨日内翰都已经顺便批阅签押过了,不过剩下的十份则没有。内翰今天午前在枢密院,这些事都是急务,所以就先拿来。”

韩冈就在入厅的台阶前停下脚步,深深的盯了书办一眼,“我之前说过吧,内翰没有签阅过的文字不要拿来给我,送回正厅去。”

书办的脸色都青了,马屁拍在了马脚上,而且还是违命,说明他没将韩冈早前的吩咐放在心上,这可是大忌。忙不迭的点头,“小人明白,小人明白。”

只是他不明白,为什么韩冈连一点竞争之心都没有,三十不到的年轻人,争权夺利的心思不该缺的。而且还是神仙弟子,怎么会甘心屈居人下。

想不通归想不通,书办进了厅中,在韩冈的桌上,挑了十几份卷宗出来,安排小吏送去正厅。

韩冈在自己的桌前坐下,今天要批阅的文件就摆在案头上。

群牧司中大事是群牧制置使与群牧使商量,小事由副使处理,余事群牧使自决,同群牧使的工作只剩签字画押。

韩冈将手上的几份公文都看了一遍,其中有一半的签名是枢密院都承旨韩缜转群牧使韩缜,难怪说他昨天顺便签阅过了。

提起笔随手就签了字。其中有几份其实韩冈都看出了些问题,但不是原则性的错误,也便毫不在意的副署上自己的名字,并画上押记。

翻阅动笔,加起来只用了韩冈一刻钟的时间。将手上的笔一放,把十一份卷宗推给书办,“今天就这些了?”

“就这些!”书办快手快脚的收拾好,“那小人就派人送去正厅了。”

韩冈摆了摆手手,示意他自去。

书办安排人去送文件,厅中的小吏就换了热茶上来。

走进群牧司衙门只用了小半个时辰,韩冈便已经悠悠然的靠在交椅上,小口的啜着滚热的茶汤。如果韩缜不回来,今天的事也就这些了。这么轻松的工作,韩冈十年官场,这还是第一任。

不过韩冈并没有像群牧司的底层官员过去做的那般,抱着杯热茶,翻着最新一期的蹴鞠快报,然后与同僚讨论着该在哪一队头上下注比较好。

喝了杯热茶之后,他就从桌上堆成一座小山的旧年档案中抽出一本来,一页一页的翻看着。

旧档在架阁库中多的有几十年,少的也有数载,积攒下来的灰尘虽然给清理过了,可翻开来的时候,还是尘埃飞散。不过韩冈依然看得聚精会神,时不时的提起笔,在一个空白的小本子上记录下一两句话或是几个数字。

在同群牧使的位置上,韩冈不用管事,也不便管事,可对于司中事务,他还是要做到心中有数。架阁库中的旧档,韩冈自从来到衙门中报道之后的第一天,就开始挑选关键的部分翻阅,许多数据还做了记录。前面批阅文件,他看一遍就知道有没有问题,就是这些天狠下功夫的缘故。而书办将还没有给韩缜签署过的公文拿到韩冈这边,也是因为他看到了韩冈翻阅旧档,以为要找韩缜和群牧司官吏的碴,要不然也不敢自作主张。

如果现在天子问韩冈有关马政方面的问题,军马存栏数,牧监田亩数,群牧司各部门官员人数和日常开支,韩冈的嘴皮子半点也不会磕绊。十年内的具体数据,他能张口就报出来。再往前,几个有代表性的年份——比如太祖开国,太宗即位,高粱河之败,澶渊之盟,元昊叛乱——,也全都在韩冈肚子里。

要想说服人,最好的办法就是让人以为自己是专家,而要证明自己是专家,详细具体的数据是最管用的。两世为人,商界官场都是骗徒横行的地方,韩冈知道如何伪装。

又翻过了一年的记录,韩冈将装订起来足有一寸厚的档案丢在了桌上,里面没有清干净的灰尘一下腾起老高。

“终于没有拿抽了原件的档案给我看了。”韩冈抬手拂开灰尘,叹道,“连做旧都懒得做,真当我好糊弄吗?”

厅中七八个胥吏闻言皆是悚然一惊。想糊弄韩冈的两个老吏,被他揪出来交给韩缜处置,最后被打断了腿逐了出去,这一桩公案也才过去了三天而已。那两位在群牧司中做了几十年,也可算是老行尊了,但一顿板子下来,人都废了。

书办陪笑道:“都是他们不开眼的缘故,现在绝不会有人再敢欺瞒龙图。”

韩冈瞥了书办一眼,似笑非笑,然后就看见书办的脸色又开始发青。端起刚刚递上来的热茶,他吩咐道:“去将天圣二年的河西买马的记录和天圣六年的记录找来。”

书办急急的领命出去后,转眼却又回来了:“龙图,传诏的天使来了。”

