\section{第21章 欲寻佳木归圣众(五)}

王中正接下了太后的圣谕。

沈括是韩冈力推的宰辅人选,目的是为了更好的修筑轨道。

王中正不知道为什么太后会颁下这个任务,想来大概是太后案头上的弹章太多了一点。

不过王中正可没打算开罪韩冈,回来该怎么说,还没出宫门外便已经有些眉目。

韩冈那边先私下里通个气,太后面前说说好的一面,再说说坏的一面。总之错处可以改正,好处则是能惠及万民。

唤过一名身边服侍的小黄门,王中正道:“去请王阁门来,过几日我要出京,手上的差事得交托一下。”

请王厚居中转圜,想必能避免韩冈产生误会。

……………………

“相公。”

门外传来熟悉的声音。

是韩家的家人,现任的中书堂后官,跟在韩冈身边听候使唤。

韩冈放下笔:“进来。”

人应声而入,原本在旁边副使的堂吏随即很识趣的离开。

“什么事?”

韩冈问着,用手指轻轻捏着鼻梁上端。从内东门小殿回来后,就一直批阅公文,中间休息的时候,又顺便接见了几个官员,到现在为止,连喝口茶的时间都没有。

家人近前来,低声在韩冈耳边说了几句。

韩冈静静的听完,想了一下,道:“好的,我知道了,你先下去吧。……去请汝霖过来。”

宗泽很快就到了,韩冈也没瞒着他,把王中正受命将要出京体察铁路轨道修筑一事,告知了宗泽。

韩冈这么快就收到了宫中的消息,宗泽并没有在意,双眉皱得很紧,问韩冈道:“相公,太后这是不是觉得沈括不合适?”

“京洛铁路还没修起来,弹章就有上百份了。太后再放心,也肯定怕我这边不通下情。何况沈存中在太后面前,还是差了一层。”

“那王中正会不会……”宗泽欲言又止。

“放心。”韩冈信心十足的笑道:“王希烈多聪明的一个人,一辈子都没犯过大错,他怎么会做下糊涂事?等王处道的消息吧。”

宗泽点点头,王中正刚刚接受任命,身处嫌疑之地,不可能直接联络韩冈,私下里派人说不定也会被人盯着,找同僚东上阁门使王厚带句话,便是最安全的人选。

“相公想要宗泽做什么?”

韩冈道:“这件事并不是什么大事,也是应有之理,但时机有些不太好。这两日汝霖你就多费点心,有什么事及时处置,初值不了就报过来。”

“宗泽明白。”

要不是正好处在廷推的关键时刻,即便是整个台谏系统都闹起来,韩冈也不会在乎。

而且除了沿途的地主叫屈,京洛铁路其实也没有别的问题。沈括有反复之实,却没有贪渎之名,自己把朝廷的拨款看得紧紧地,不能说将每一个铜板都用在了该用的地方,但比起其他地方经手官吏都能发财的工役来,绝对是一清如洗。

但也正是失地地主的怨言,才让人头疼。

京洛铁路日后必定要进行改造,要留下改造的余地,就必须占下更多一点的土地。数百里铁路,征用土地所属的地主成百上千。这其中愿意

京泗铁路沿着汴水而修,利用的是堤坝两侧的闲地,本来就是为了保证堤防的安全才留下的空间,是朝廷的地皮,自然没有人出面来闹事。

而河东、河北的两条铁路轨道,则是有抵御辽人的大义在。沿途的地主无不在辽人的铁蹄阴影下生活多年,既然朝廷宣称这是为了抵御辽人入寇而铺设的运兵道,期盼早日修成的为数众多,也没几个人敢于触犯众怒,轻而易举便给压下。

只有京洛铁路不同。

连接开封府和河南府的铁路,经过的是国家的中心地带,沿途的地主一个比一个背景深厚,加之又没有军事上的急迫性,能利用支线捞到好处的世家大族在地主之中的比例又不算高,自然免不了有许多反对的声音。

面对那些反对者,韩冈选择了强征,而后用边境上的荒地进行大比例的交换——不能给人狮子大开口的机会,但也要让外人觉得朝廷做得不是那么过分,这其中的分寸,其实不是那么好把握。

幸而当地的豪门,纷纷出面帮助韩冈解决了这些琐碎的难题。用支线收买了这些豪门,让铁路铺设的道路前面少了无数的绊脚石。

干线都波折重重的话,支线怎么修?

有份参与修筑支线的世家豪门都有这份担心。有了当地豪门的支持,京洛铁路的进程没有受到任何干扰。

但朝野两方的合力,能排除实质上的干扰,却不能堵上所有人的嘴。这几年,各地官府不知收了多少状纸,而太后的案头上,也多了许多弹章。对大权在握的韩冈来说,这不过是癣癞之疾,可一个不好,癣癞之疾也能变成致命的病症。

宗泽自然知道韩冈的顾虑,对他来说,也不是什么难事。

之前宗泽就处理过一桩破坏铁路运行的案子,管城县一名被官府强用换了三亩田的田主,砍了一棵树拖到了轨道上,因为驾车的车夫及时发现,避免了脱轨的惨剧。事后,犯人很快被抓住,管城县以没有造成实际损伤为由,将之杖责后开释。但沈括对此极为不满,指责管城县沮坏国事,纵容犯法。

轨道的安全是重中之重,故而朝廷对涉及轨道的案子,一向是用重法,不论是不是属于重法地。偷窃枕木、铁轨的案子每年都有,抓到之后,情节即使再轻,也是流放西域、岭南的结果。

因为这件案子的争议,沈括与管城县打起了笔墨官司,中书这边也被烦得不行。

韩冈将这件事丢给宗泽处置,宗泽力排众议,将田主全家流放到代州,那里有田主交换得来的田地。私下里,宗泽是如何与那一家人交流的,没人知道,但韩冈事后从另一个渠道得知,那家人私下里对宗泽千恩万谢,视其恩同再造。

宗泽很善于与人打交道,即便有巨大的身份差距,依然能够与人顺利的交流。达官显贵、贩夫走卒,宗泽居其间都能交到朋友。也难怪另一个时代,领兵抗金的宗泽能聚拢那么多豪杰,而等到他去世后,豪杰便纷纷散去,接手的杜充就只会掘黄河。

“宗泽必不负相公所望!”宗泽一番保证后,又迟疑的说道:“但今日之事观之,轨道既然是要与民争地,那么只要还要修轨道,争议将永难休止。今日只有一条京洛,他日随着轨道遍及天下,又会有多少异声杂论?太后如今已经犹疑不定,遣王中正出京体量,日后又当如何?”

韩冈点头,坦然道:“这事我也在琢磨着,汝霖你若有什么想法,不妨也说来听听。”

宗泽对解决目前的问题信心十足,韩冈也对他信心十足。但即使一时能够解决燃眉之急,也解决不了日后的问题。韩冈也迫切希望自己能够耳根清净,铁路方面的大小事务,能够有所依归。

宗泽抬眼正视:“以宗泽的一点浅见,不如专设一个铁路轨道的管理衙门,专门应对一应的大小事务。”

“现在不是已经有了发运司吗?”

“那是管纲运的,民运管得太少。”宗泽道:“各个发运司,主要都是以水道为主,驾船入水便能运货运人,难以管辖。而铁路上的车辆都是有数的,不是谁想来就能来,管制起来就简单太多。”

“那也只是再多几个发运司罢了。”

“铁路相互连通,又要动员,运输”“设一总理衙门,将天下轨道都管理起来。”

“对。而且铁路轨道的修筑、,都需要有所专长者来主持。甚至铁轨的铸炼,也不同于其他铁器,非大工、熟手不能造。在情在理,也应该专设一个衙门,来统一管理。”

“有理。”韩冈满意的点头,笑道,“这件事,我已经犹豫了很久,想不到还是汝霖你来为我解惑。”

“相公手中事务千头万绪,一时难以垂顾,但沈端明专责于此,理应建言相公才是。”

换做是别人来说,就是在韩冈面前给沈括上眼药。但宗泽的性子,韩冈也知道,有话直说罢了。

韩冈当然看得清楚:“他是身在嫌疑之地,能少一事便少一事。”

虽然在后世,有学校,有武装,有公检法,几乎自成一国的铁路系统为人诟病,但这样的组织结构,却是十分契合现如今的形势。设立类似于铁道部的机构,按照地域划分铁路局,囊括司法、教育、军事等机构,这样的一个衙门,才能解决铁路发展中各种各样的问题。当然,这么一个庞然大物,不是宰辅级别的官员没资格来掌管。

成立这么一个铁路总理衙门,职权又如此巨大,有谁敢出头提议的?韩冈不方便,沈括所在的那个位置合适说,但他这个人不方便说,直到宗泽这个还算有一点身份的官员出面来。

这个提议,韩冈已经等了有一段时间了。
