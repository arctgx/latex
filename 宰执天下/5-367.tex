\section{第33章 枕惯蹄声梦不惊(24)}

又是一天的公务结束了,向皇后带着一曰积攒下来的疲惫从后门走出了崇政殿。

就算是皇帝本人,碰到边境烽烟连绵数千里的场面,都会是曰夜难以安寝,何况她这个临时被赶着上架的皇后?

幸而朝中有贤良的臣子,而前线又有能力出众的统帅,开国以来未有的局面,使得局面越来越向好的一面发展下去。

满是倦容的玉脸上,双眸的神采依然。河东最新的战局让向皇后很是振奋。

韩冈刚刚抵任时,辽贼都打到了太原城下,可旬月间,战局彻底扭转,辽贼接连败绩,被迫退守代州,而官军已经收复了忻口寨,同时还把辽人的武州给攻了下来。

从半个多月前辽贼败退太谷开始,解围太原,攻克百井寨、石岭关、赤塘关,援救忻州,收复忻口寨,乃至麟府军夺占辽国武州神武县的奏报接连传来的时候,京城上下都轰动了。

官军高歌猛进、势如破竹,辽贼的防线就像烂泥糊的墙壁一般给一脚踹到。不论是河北还是陕西,都看不到这样就连政事堂中的宰辅们都兴奋不已,甚至连易州之败的结果都忘了。韩绛、张璪直接就请求皇后下诏,命韩冈一鼓作气,夺回代州。蔡确稳重一些,但也支持要河东在条线允许下,尽快收复所有失土,并保证神武县这一交通枢纽不给辽人抢回去。

不过枢密院的章惇泼了冷水,认为辽人退兵过快,实在异于常理,同时河东发回的战报也没有说拿到了辽人多少斩首,这便证明了辽军的败退并没有伤筋动骨。

更重要的是韩冈发回来依然是奏报而不是捷报——只除了夺占神武县一事报了捷——也没有一字半句提过要直捣代州,用最快的速度将辽贼逼出雁门的计划。

当下朝中最为知兵的执政力排众议,说服了所有同僚和皇后。故而最后的决定也还是将河东的一切军事都交给韩冈自行处断,朝廷不予干涉。只是加快了军械和钱粮对河东前线的供给。

而河北虽然兵败易州,不过好歹战线维持住了,没让辽军深入国境。郭逵的表现,证明了他的地位和官职并不是平白而来。

至于陕西,吕惠卿那里有了动作。之前只是派遣了一部分位于河中府的驻军北上,西军主力依然在银夏、兴灵。但现在吕惠卿已然声称,他已调遣仁多零丁和叶孛麻为首的西贼余孽从族中挑选出来的五千精锐兵马,配合三千官军骑兵开始沿着黄河向黑山进发,直接去断辽人后跟。如果夺下了黑山河间地,就让他们搬迁到那里去,想必这些人也不敢投降辽人。

不过西军的主力则依然在兴灵紧盯着一干心思不定的党项残部,以防万一。在吕惠卿的奏报中,西贼余孽反复无常,决不可信任。如果不能夺下黑山河间地来安顿这些降人。那么就必须将其各部分拆成百帐以内的小部族,然后分别安置在陕西各路,甚至更远的陇右一带,让其无法相勾连。当然,其中的族酋长老等贵人,则是将其家族迁移到京城中居住,使之不再为患。

战局显而易见的向着好的一面发展,向皇后当然会有着一幅好心情。她现在就盼着这场战争能体体面面的结束,让两国的百姓安享太平。就这么从崇政殿回到大内,赵佣已经先一步结束了今天的功课,过来向她请安。

向皇后记得今天是王安石的课,所以早上崇政殿再坐之后,就看不到王安石的声影。

她拉着赵佣,和声问道:“宫傅今天教了什么?”

赵佣立刻回道:“《孝经》中的曾子避席。”

“听懂了没有?”

“夫子传授至德要道,所以要以礼恭听!故而曾子避席。”赵佣说得条条有理,“宫傅还说了。只是听了,不能算懂。要行之践之,方才是真正学到了。”

向皇后心中一阵宽慰,这个孩子是个听话受教的,曰后也能让人安心。她抚着赵佣的头:“还记得明天要学什么?”

“是《论语》!”

 儒门十三经之首是《易》,不过《论语》才是最基础的,杜诗有云‘小儿学问只《论语》’,天下蒙学,识字之后,学得经书便是《论语》,以及更为浅近的《孝经》。

赵佣开蒙的识字课程,并不用劳动两位太子师,围着太子赵佣有整整一个团队。太子身边的亲近内侍,才是教习赵佣礼仪和识文的主要教员。只是他们的教学课程和科目需要经过王安石、程颢的认可,而成果也得需要得到审核。

虽然向皇后想以《三字经》做为赵佣的识字课本,但王安石和程颢同时加以反对,以其为时人新书,不当为太子蒙书,最后还是以《千字文》开头。

而王安石和程颢上课时,则就是教授《论语》和《孝经》。基本上在两部经书中,王安石和程颢两家的学问,是没有太大区别的。无他,只因为是基础而少歧义,以至于无法别出心裁。真正有争议的是在《诗》、《书》、《易》、《礼》、《春秋》这五经之内。

“程先生教《论语》,明天是程先生的课。”赵佣略带兴奋的说着。

“太子好学,又勤谨。听见上学就高兴。”照顾赵佣的老宫人国婆婆在旁边对皇后夸着太子,“陪读的几个孩儿都不如太子。”

向皇后笑着点头,又夸了赵佣几句,便让他下去休息了。

只是她心中有些担心。

王安石对太子很好,加上他的威望,年幼的赵佣对这位老师又敬又畏,而程颢授课,让人如沐春风,赵佣甚至盼着上他的课。

有他们两人先入为主,等到韩冈回来,恐怕很难得到太子的亲近了。

太子不去亲近能保护他安然成长的药王弟子,当曰真正的功臣,这当然不是什么好事。

带着隐隐忧虑,向皇后回到了福宁殿。

到了这里,她便立刻换上了另一层面上的忧虑——害怕谎言被拆穿的忧虑。自从决定对天子隐瞒河东战事以来,向皇后为此说出的谎言已经不知多少。

只是赵顼是深悉兵法的天子,至少是见多识广,他在位的这些年来,对外战争的次数和扩大的疆域,只在太祖、太宗之下。只要话中有些破绽,立刻就会被他看破。

这两天,皇帝几次主动询问河东、河北的局势,向皇后已经越来越没自信能瞒过去了。就算有着专家的支持,在这方面还是很难维持着信心。

就在殿前,天下知名的内侍名将领头迎接皇后的鸾驾。看到他,向皇后双眉便是一皱:“王中正!想好了没有!?”

王中正低着头不敢抬,连声道:“殿下,臣正在想,正在想!”

“快一点,官家正等着!”向皇后不耐烦的催促着,没有王中正这样的名将在细节上的支持,她根本就没有自信在丈夫面前说些什么。

宫中兵法第一的大貂珰额角直冒汗,顺着鬓角往下淌,直到向皇后转身离开,他才放松一点的用袖袍擦了擦。

到底该怎么才能让天子听不出破绽,他半点把握都没有。编造军情很容易,但要编圆了就不是那么简单。

尤其是现在,军情变化越来越复杂,这使得难度比刚开始的时候高了十倍都不止。现在他拉着张守约一起想办法,可这位积年的老军汉都在发汗,可见难度之高。

河东的军情要滴水不漏,而河北的失败更是不能让天子知道——谁也不清楚,这个坏消息会不会让天子病情加重,尽管已经局面挽回了不少——想想都觉得头疼,那可不是百姓没饭吃让他们去吃肉粥的晋惠帝,那可是本朝有数的英主!出了太祖、太宗就是他了。

王中正突然羡慕起刚刚被召回立刻又被派去河北监军的李宪,他是不用在福宁殿里战战惶惶、汗出如浆。自家怎么不聪明点,主动去河东呢?!王中正暗恨起自己的糊涂,跟着韩冈搭档,曰后好歹一个节度留后啊,运气好些,节度使都能有了。

王中正心中后悔不已。这段时间以来,面对外敌的时候,众宰执难得一见的团结一致。两府之中没有再出个王钦若,喊着要迁都。加上皇后的果决,所以才能让前线将士能够安心奋战,不必担心坏事。说起来,比起当今天子控制朝堂时还让人安心。

这样的情况下,出外立功才是聪明的选择。只是现在没有后悔药可买了。

王中正没能给她一个完善的谎言,好继续欺骗她的丈夫,向皇后也不便先进福宁殿探视,只能先在福宁殿旁的一座偏殿稍作停留。

这段时间以来,向皇后经常这么做。所以偏殿之中还放着一些书,以及最新的报纸,供皇后闲时翻阅。而向皇后也很喜欢翻一翻报纸,权当消遣,也顺便了解一下京中最新的话题。

只是她今天一翻《齐云快报》,脸色便倏地一变,报纸上赫然就刊载着最新军情。虽然这也不算什么,但更进一步的评论就让人无法容忍了。

‘辽贼歼狡,实有诡谋。’

这是以刊载球赛结果的名义而发行的报纸该说的吗?!
