\section{第25章 鸟鼠移穴营新巢(上)}

在大狱里待了三天,乔二狗终于见到了太阳。

狱中的小窗户朝北,房间一直都是阴湿的。不过铺子上的草还算干净,没有臭掉,也没有多少虫子。房间中有股焦油味,涂在墙上地上防跳蚤和臭虫。方便也不是用净桶,而是专门的水沟,斜砌着,通到更深的沟里,用水一冲就干净了。

狱中的牢头提着刀每天来回巡视两趟,中午给饭的时候,就会过来说一通,监中变得如此之好,是韩相公的德政,你们这些贼骨头命好云云。

乔二狗早年进过一次开封府狱,两边的对比之下,觉得牢头说得的确没错,可是他好端端的给抓进来,据说也是那位韩相公的命令,这哪里不让他感到满腹的冤枉气。

狱中再干净,他们这些乞丐却也是脏的,没了跳蚤臭虫,也还有虱子。

抓着身上的虱子,乔二狗跟着同伴走出了狱中。

这两日,一起被抓进来的同伴,有两个被拖出去了,再也没回来,其他倒是好得很,与乔二狗一起有吃有喝。

在狱中,乔二狗还看见不少老朋友,有一些很打过几架,为了争夺一条街的乞讨权,乔二狗这个年轻力壮的乞丐,为丐头没有少冲锋陷阵。不过乔二狗没在狱中发现他的丐头,其他熟识的丐头他也是一个都没发现,只看见了他们的属下。

“会不会要杀了俺们?”

与乔二狗一起讨饭,也一同在雨夜中被抓的叶小三浑身发抖。

乔二狗长了叶小三几岁,也比他更有见识。“杀人也要先吃一顿断头饭才是。你没听隔壁的陈瘸子说吗,这是韩相公找不到人了,只能抓俺们去守边,报纸上早提过了。”

“说什么话!”

旁边的牢头听见声音,横眉竖眼的呵斥过来,乔二狗立刻藏头弓背,又是一副乞丐模样。

一群人被赶着离开了待了三天的牢狱,从后门出来,就看见一排大车停在巷中。十几人一辆,几十名乞丐,就这么全被赶上了五辆车子。

旁边骑兵持弩同行,车队左弯右绕,最后穿过了一道大门,终于停了下来。所有的乞丐都是第一次坐马车,幸好车子是运货的敞口车,倒没人晕车呕吐。

乔二狗在人群中中缩头缩脑,尽量不惹人注意。眼睛却没闲着,一路上左看右看,发现这是他认识的地方。

在京师多年,大小军营他都认识。倒不是要来这里讨钱,而是防着走错地方,这些赤佬可不比商家,下手又黑又重,就像三天前下雨的那个晚上,过来追捕他们的军巡铺巡卒,平素里都有钱孝敬,但官面上的命令一下,立刻翻脸无情,就跟狗脸一样,说翻就翻。

啊,就是那种大黄狗。

盯着那条狗,乔二狗想起了过年时吃的那锅狗肉,不经意间已经被赶到了狗面前,抬起头,狗上面有张桌子,桌子旁边立了个军汉,桌子后面还坐了个人,读书人的模样,拿着笔,身前铺着一张纸。

‘应该是个书办。’乔二狗想着。

“姓名。”

书办头也不抬,一边拿笔蘸墨,一边问着。

“啊?”乔二狗一愣。

“苏学究问你姓名!”

桌旁的军汉一声呵斥,乔二狗连忙道:“小的姓乔,贱名二狗。”

“这个‘狗’?”

书办指了指脚下,一跺脚,趴在地上的大黄狗立刻站起来,冲着乔二狗汪汪汪的龇牙咧嘴了一番。

‘等爷爷出去,就拿你下酒洗秽气。’

乔二狗心中发狠,脸上则堆起笑,“小的不识字,应当就是这个狗!”

“狗字不雅,去掉犬旁,加个草头。乔二苟。”

刚换了名字的乔二苟一脸迷糊,“这不是一样。”

“写起来不一样。”书办终于抬头,“下一个。”

“还不让开!”嫌乔二苟动作太慢,桌边的军汉一脚踹来,“原来是狗,现在是草狗,真楞得跟草扎的狗一样了?”

用力冲前面吐了口吐沫,回头盯了一眼书办,乔二狗心中恨恨,‘爷爷是能咬人的狗,却给弄成草扎的。等有一天,爷爷发迹了,也让你做一回草狗。’

“老实坐下!”

就在乔二苟心中痛骂的时候,他已经被领到了校场的另一头。

眼前一张凳,旁边一盆水,然后还有一个拿着剃刀的军汉正虎着脸看他。

“坐下,闭嘴,闭眼,不要说话。”

一声一呵斥,乔二苟只敢心里骂,却不敢违抗命令。

老老实实坐了下来,闭上眼睛,就感觉到头顶上一阵窸窸窣窣的声音,肩膀上也能感觉到不停的有东西掉下来,最后一捧水当头泼下。

等到被人从凳子上提起来,乔二苟便发现自己被剃了个光头,原本满头油腻还跳着虱子的乱发,现在只能摸到一点点湿漉漉的头发茬子。

这下要做和尚了,乔二苟心道,听说少林寺和尚能吃荤,不知能不能混进去。

大相国寺的和尚明面上戒律森严,其实不仅不忌荤素,连女色也不怎么忌讳,时常上门驱邪,或给人送子,这就更强出十分了。可惜人家是敇建,官家都常来往,乔二苟不指望自己能进去。但少林寺肯定需要能打的,不肯交租的佃户,想要侵占田地的富民,没些棍棒拳脚,怎么保得住这份家业?

“进去洗干净。”

乔二苟一边幻想,一边跟着人来到了一间大屋前。

从敞开的门口,能感觉到一团湿气扑面而来。

‘莫不是浴堂?’乔二苟想道,‘是不是要洗澡?’

韩相公说疾疫只因脏,讲究干净,所以京师内外,遍地浴堂。但乔二苟自己却觉得那是放屁,不干不净,吃了没病,他做了这么多年乞丐,身上就没干净过,也没见自己病死啊。

乔二苟在浴堂前,胡思乱想,等到将韩冈骂到了十八代,突然推了他一把,大骂着还不脱了衣服滚进去,这才发现,周围已经都是一个个光头了。

“二狗哥。”

听到有人叫,乔二苟瞪大了眼睛,费了半天才认出是叶小三。

一块儿吃了两年饭的兄弟,剃了光头,再脱了从来没洗过的衣服,人整整小了一圈,显得更黑更瘦,乔二苟差点没认出来。

“快进去,快进去!”

站在门口的军汉大声的赶着已经脱光了衣服的人进去。

乔二苟三两下脱掉了身上的破布,与叶小三一起被赶进屋中。举头张望,他发现这里果然是个浴堂。只不过只有湿气,没有热气。

‘大概是嫌烧热水太费煤炭,所以干脆省下来?’

乔二苟想着,却也不怕。冷也好,热也好,都不过是洗个澡。从来都是打不怕骂不怕,他乔二苟哪里还会怕冷水。

但浴堂里面不仅是冷水,而且还有军汉。五名壮汉站在浴堂中,提着棍子瞪着每个人。

“下面一路上都要坐车。干干净净的车子,你们这群贼骨头坐上去后,少不得要弄得一车的腌臜。你们自己染病没什么,把病留在车厢里,你们这些贼骨头死一百遍都不够!……所以给我洗,要洗得干干净净,重新做人。”

在提着棍棒的军汉们的命令下,一群光头光身的乞丐,两人一组,互相之间拿着丝瓜瓤子,用力的刷着自己和对方身上积攒多年的污垢。

“要洗干净了!”

“别图省事!”

“眼瞎了,这么一大块脏东西都没看到?还不搓下来!”

身后几个士兵提着短棍来回走,看见有人草草了事,立刻就是一棍。

乔二苟挨了两下,疼得差点嗷嗷叫。跟他一组相互帮忙的叶三也挨了一棍,不敢再浑水摸鱼,拼了命的洗刷对方。因为没有热水,一开始还觉得冷,但很快就热了起来,火辣辣的烫。最后两人与其他人一样,身上红得就像是刚出锅的螃蟹,只感觉连皮都给搓破了。

从澡堂中出来,乔二苟身上是火辣辣的烫,身下却是凉飕飕的——浴堂里面还有剃刀,不过是剃下面。

现在他浑身上下光溜溜的,就像是刚出生时的模样。

看着周围一个个赤条条的身子,自己也精赤着身子的乔二苟莫名其妙的就有些想笑。可指使了他们一天的人,一点空闲也不留,很快就传下口令,让乔二苟与其他人一起排着队去领衣服和鞋子了。

乞丐从不知纪律为何物,但他们知道军棍,在队伍中不老实的也同样是一军棍,队伍便排得跟接受了半月队操一般整齐。

春风中精赤着身子,乔二苟冷得瑟瑟发抖,下面的物件都快要缩进去了,方才还想笑两声的心情现在是一点也不剩了。

幸而只排了小半刻队,乔二苟也领了一件衣服。他急匆匆的将叠好的衣服抖开,却发现这是哪里什么衣服,就是一口钟。一块布裁开,再缝起来,两边没袖子。穷和尚常穿,富和尚就看不上了。只是做乞丐的没什么挑拣,乔二苟拿起衣服,赶急赶忙的套在了身上。除了衣服,还有一条草绳做腰带,一双草鞋穿起来。

几十个光头都穿得一样,乍一看,倒是几十名沙弥聚在一起。

不过沙弥是不用刺字的。

乔二苟咬着牙,看着自己的右手手背上,被龙飞凤舞的刺上了四个字,又揉进了特地调好的墨汁,使得字迹鲜明。

乔二苟不识字,但旁边有识字的人念——云南戍边。
