\section{第28章 官近青云与天通(20)}

蔡确的发言总是这么恰到好处,让人惊喜连连。

多少喜欢去看蹴鞠球赛的文武官们同时闪过了一个念头:

这脚球补射得好!

韩冈与司马光交锋三五回合,刚一占上风,蔡确就趁势将皮球踹进了球门。

韩冈瞅了面容整肃的蔡确几眼,无奈的将视线转回了脸色紫胀的司马光。盯上王珪留下来的位置的蔡大参,自然是要在皇后面前露个脸。

而帘后的向皇后,她并不知道蔡确之言的真伪与否,她对此根本没有了解。不过司马光被堵得闭了气,倒是能做个证据。

但她对细节也很有兴趣,回头看看宋用臣,宋用臣会意,弯下腰,低声道:“圣人,这件事王观察应该知道,他当时就在陕西军中。”

所谓的王观察,就是王中正。他本官是观察使,正五品的贵官,内侍兵法第一。这段时间正领军镇守宫掖。

向皇后点点头,到底是什么样的情况,事后问问王中正就知道了。

蔡确的这一脚的确是稳准狠兼备,司马光没了反击之力,韩冈也觉得差不多该收场了。

可章敦却跟着发话。他质疑:“记得当年广锐叛军并没有打下邠州,反而吃了一个亏,最后不得不绕城而走。并不是如参政所言,大掠邠州。”

过分了!韩冈心道,捧哏不要做得这么明显好不好?!

他更加紧张的望着不远处的太子太师。司马光都这把年纪了,身体不会太好。要是在殿上发病,甚至中风,就不太好了!

可他已经来不及阻止,只见蔡确当即精神一震,高声道:“吴逵这个广锐军邠宁都虞候,直至官军开始进筑罗兀,被调往庆州镇守边防前,正是驻扎在邠州城中。他领军南下,人情地理皆惯熟的邠州是最容易被他攻破的。但幸而邠州有个年轻有为的判官。见邠州驻军北上庆州之后,城中兵力并没有加以补足,自知无法坚守,便率兵出城偷袭贼军前锋。虽然这一战侥幸赢了,其实也是险到了极点。一旦他败了,邠州将立刻陷落。只因城防不固,兵力不足,不得不如此。那位判官,名为游师雄,却也是横渠门下!”

蔡确当年曾任邠州司理参军,因献诗于宣抚陕西的韩绛,才被荐到时任开封知府的韩维门下,韩冈第二次上京便正好与其有一段因缘往来。

尽管蔡确离开陕西时,横山攻略刚刚展开——罗兀筑城和广锐之乱是发生在韩冈离京后——不过在横山之役宣告失败后,通过仍在陕西的旧友,蔡确对广锐之乱前后的陕西局势仍了解得十分深入。

听蔡确将当年事娓娓道来,向皇后再去看司马光时,就更多了几分厌弃。韩冈的同门,只为了给司马光收拾手尾,就不得不冒险领兵出外偷袭贼军,而不是固守城防以求安稳。司马光在关中,差点就坏了国家大事。

司马光脸色通红,嘴唇抖着,却发不出声来。他甚至无法辩驳!毕竟这是实实在在发生的事。回去查查旧档,就能将《谏西征疏》、《乞罢修腹内城壁楼橹及器械状》和《乞不添屯兵马》这三份他在长安任上所进呈的奏章给翻出来。这也是为什么他被撤了知京兆府的差事,派到了洛阳,主掌西京御史台的缘故。

司马光的窘迫,让韩冈看得暗暗摇头。

缺乏地方从政经验,这是司马光最大的弱点。在二十岁得中进士之后,直至五十三岁知长安京兆府这个大府资序的要郡之前,他没有任何亲民官的主官经验,知县、知州、一路监司主官他都没有担任过。

寻常的进士要就任兼领一路兵马的要郡,最快也要有两任知县资序、两任通判资序,两任知州资序,然后再看运气,至少要升到侍制以上,再有几任路中监司的主官。在这段一般长达二三十年的时间中,至少有一半时间得在地方任亲民官,剩下的则是在京城或是路中监司担任资序相当的职位。可司马光,则基本上都在朝中度过。

签书苏州判官事,签书武成军判官,并州通判,开封府推官,这是司马光在担任知京兆府兼永兴军路经略使之前的全部地方经验。

在苏州任上,因为其父母相继亡故而解职丁忧,司马光只做了一年多。

除服后,司马光出任武成军判官,也就是滑州,签书判官事两年。

之后他就回到了朝堂,直至十年后,司马光因其连襟之父庞籍知并州兼河东经略,被荐为并州通判。司马光上任后,代庞籍巡视边地,主张在麟州筑堡失败,损兵折将,连累得庞籍被贬知青州。庞籍帮司马光担下了罪责,司马光此后便视之若父,事庞籍之妻如母。这一任,两年而已——在并州通判前,司马光其实还跟着庞籍去了郓州,主管州学半年多,不能算正式工作,也没有什么功绩可言。

并州事毕,司马光回到开封,任职开封府推官。两年后便改修起居注,判礼部。在这期间,司马光最有名的是论交趾麒麟祥瑞,还写了一篇赋文来讽谏。

从此他一直留于朝堂,任官知谏院、翰林学士等清要之职,直至王安石开始变法。

三十余年的时间,司马光在地方上只有佐贰官和幕职官的资历。除去滑州、开封这两个畿内差遣,司马光在外地的任职时间更是只有区区三数年。且不论是在并州通判任上,还是在开封推官任上,司马光都没有表现出足够的能力。

司马光比起其他从地方上一路稳稳爬上来的官僚,最为欠缺的地方就在这里。更是远远不能同在地方上施展才华而不愿入京的王安石相提并论。

当蔡确拿任职地方时的挫败和纰漏来攻击司马光,司马光是毫无还手之力。

话说回来,蔡确本人也极度缺乏地方经验。升朝官后,就没有离开过朝堂。基本上走言官路线,从监察御史,一路升到御史中丞,现在又成为了参知政事——亲民官的经验远比司马光更欠缺。可是到了他这个地步,也没有司马光在地方上出乖露丑的失误,反而没有破绽了。

而且司马光现在也没办法驳斥他。已是血涌上头,晕眩一阵跟着一阵。外表看着没什么变化,但能站定脚跟已经是他在竭力平复心情的缘故。

韩冈始终都在关注着司马光,看到他现在的样子就知道有些不好了,再争下去,太子太师当真能晕厥在文德殿上。

“各位以辅臣之尊,陛前相争,喧哗如街市口角,到底成何体统?!”

一声斥责,突然响起在殿上。

众人循声望去,一名风姿挺秀的御史步出班列,在大多数御史前面跟着司马光一起弹劾王珪的时候,没有出班的御史也就剩下寥寥数人。

这个人,韩冈还认识。

“臣监察御史蔡京,劾司马光、蔡确、章敦、韩冈,殿上失仪,有失大臣体,当一体罚铜,以作惩戒!”

蔡京倒是聪明。可谁也不能说蔡京错了,甚至司马光还得感谢蔡京收场,至此他方能定一定神。

只有殿中侍御史才能名正言顺的维护朝仪,而蔡京现在只是御史而已。前面他没有跟着跳出来攻击王珪,现在站出来,却是正好合了绝大多数人的心思。总不能让好端端的朝会,变成蹴鞠球赛后卷堂大散的球场。坚守维持朝廷纲纪的本职,当能给皇后留下一个深刻的印象。

“臣等喧哗殿上,有罪。”

从蔡确开始,连同御史,包括韩冈、司马光在内,几十名朝臣同向皇后和太子行礼请罪。

一言震朝堂,让宰辅们同请罪,蔡京有些得意。

帘后的向皇后却气冲冲的哼了一声,“两边打板子,到底是谁错了,当吾看不出来?!”

“圣人!圣人!”宋用臣又开始冒汗了,“司马宫师年纪大了,只看太子也该给个体面!”

向皇后闻言立刻向赵佣那边望了一眼,五岁的小孩子仍端端正正的坐着,动也不动一下。可朝会拖太久了的确不好,向皇后也不想再耽搁时间,挥挥手:“都免了,归班吧!”

韩冈回到班列中,他已经不再看司马光了,而是吕公著。

方才司马光被群起而攻,吕公著竟然就在旁边看着,没帮司马光说话。

他到底在想什么?

韩冈很有几分纳闷。就这么让司马光成为众矢之的,最后灰溜溜的返回洛阳?让赤帜蒙尘对旧党可不是一桩好事。

诚然,司马光顶撞皇后,已经犯了大错。而且在朝臣们面前,连皮都给扒光了。但就这么将之抛弃,旧党的人心怎么办?壁虎断尾求生,但断了后半截身子,还能活吗?

韩冈看不透吕公著的心思。

但司马光完了是肯定的。即便福宁殿中的天子还要给他两分体面,司马光自己都不会有脸留在京城。皇后也不可能留着他。

而自己这边,司马光的攻击虽然给了敌人们灵感,但终究还是无甚大用。

想以药王弟子来攻击自己,韩冈早有心理准备。本就是避免不了的事,他这几天得到的弹劾中,就有这么一条。

可相对于后患,用处则更大。无论如何,韩冈不可能放弃医学权威的身份。而且这种胜负手,也不会随便乱放的,多是用来阻吓对手。就算被人忌惮,又能怎么样?韩冈从来都不在意这等小事。

