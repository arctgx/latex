\section{第30章 臣戍边关觅封侯(四)}

“……又是机宜说的?”韩冈问道。

“没……错!”王厚真的是喝多了,有些话根本不该说都说了出来。他饧着醉眼,醉晕晕的道:“大人说了,王相公的青苗贷就是……就是为了填补国库亏空,筹措军费,跟什么救民疾苦根本没关系。否则何必这么着急。均输法才闹得沸沸扬扬,主持均输的六路发运使薛向受得弹章叠起来等身高,却没隔两个月又把青苗贷给推出来?玉昆,你知道什么是青苗贷罢?”

韩冈当然知道什么是青苗贷,因为这一条政策本是出自陕西路,是前陕西转运使李参在任时首创。一年中,农民最困难的日子,便是春天青苗刚起、青黄不接的时候。许多农民都是在此时向富户借下高利贷,最后被驴打滚的利息弄得破产。

李参有鉴于这一点,便在春天向农民借出常平仓里的粮食或是钱财,等到秋收再连本带利的收回来,当然这个利息远小于平常民间的借贷。而王安石在地方上的时候,也实行过类似的借贷,据说百姓多承其惠,公私两便。但如今王安石推行青苗贷,目的却是聚敛,救民的本质已是附带。

韩冈笑了起来,政治这东西目的根本不重要,结果才是关键,道:“听说青苗贷利钱才两分,‘夏料’是正月三十日前借,夏收时还,‘秋料’是五月三十日前借,秋收时还,两项借贷都是两分利。换算成年利,也才四分。即便目的不是为了民生,但实行起来却也当得起公私两利……”

如果当初能用两分利借到钱,自家也不用卖田了。可惜啊,当时摆在韩冈父母眼前的只有李癞子的高利贷。李癞子用着高利贷盘剥了村中三分之一的田产,多少家老子没还清就死了,儿子跟着还。韩千六宁可卖田也不敢借,就怕连累到儿孙身上。而如李癞子之辈,哪乡哪村没有几家?他们都是乡里的大户人家,如果青苗法推行,等于是断他们的财路,抢他们的生意。

“不过……”韩冈话锋一转,声音变冷:“恐不会受豪绅世家所喜。”

一方得利,必有一方失利。既然官府把借贷的年利率压到了百分之四十,贫苦百姓虽然高兴了,朝中也可得到一笔收入,但原来通过高利贷聚敛钱财的大户豪族必然心有怨艾。这个时代,投资的途径不多,除了田地外,官户、宗室、豪商、富民,许多都是靠高利贷来赚钱,年利五分是良心价,六分七分才起步,一年息钱跟本金一样多——也即是‘倍称之利’——才是最普遍的情况。

韩冈中学时就学过了阶级论,虽然课程无聊的让人想睡觉,但到了社会上加以印证,却是至理。扯落温情脉脉、忧国忧民的虚伪面纱,让人一眼就能看清许多言论和行为背后的吃人本质。个人能背叛阶级利益,但阶级本身却不会背叛自己的利益。

王安石要充实国库,从虎口里夺食,等于是将官宦世家、豪门富民这个统治阶层彻底得罪,他们不一个个跳出来反对那就是天下奇闻了。当然,基于‘君子不言利’的世风,没人会赤裸裸为自己的利益叫嚣,但他们总能找到看似正大光明的理由。

“大人也是这么说。”王厚猛力甩了甩脑袋,想让自己清醒一点,“但只要让官家看到国库充盈,至少几年内不会有事。如今王相公要在全国推行青苗贷,首先试行的便是河北、河东和陕西三路。秦州沿边,蕃人众多,又是与西贼作战,所以没动静,但关东诸州府可是都已经将本钱准备好,就等明年开春了。”

“但至少要等到明年夏收秋收以后,府库中才能充实一点。”韩冈沉声说道。如果只能依靠青苗贷的收入,王韶的行动至少又要耽搁大半年。拖得时间越长,对王韶就越不利,一直看不到成果,王安石也不可能无条件的一直等下去。

“玉昆,你不知道。自从李师中上任后,就拿着钱粮不足为借口。大人想修渭源堡【今渭源县】,在渭源堡开榷场,他都推说财用不足。如果大人硬要修城,他也不是不同意,就从供给北面诸寨堡的钱粮里扣一部分下来支转。玉昆你说,这些钱大人能动吗?!”

“不能动。”韩冈叹了口气,摇了摇头。动了那些赤佬的钱,王韶还能在秦凤路待吗?李师中掌握着秦州财计,就算王韶得天子和宰相看重,但李师中毕竟是顶头上司,他要压制王韶,能用的手段太多了,

“所以得等青苗贷的息钱到账,那时候李经略也无法找借口了……不,那时候直接根本不用经过李经略的手,直接让政事堂下令,通过陕西转运使将钱转给机宜。反正王相公已是债多不压身,被李师中怨恨也不会在乎。”

“没错,大人就是这么想……王相公推均输法,推青苗贷,都是聚敛之术。大人也看不过去,但为了平生之愿,也只能……”

王厚的声音突的一顿,没有酒喝,他的醉意消退了许多,终于反应过来前面话说多了。有些紧张的对韩冈道:“玉昆,这些话你可不能对外说。”

韩冈轻笑,笑意中透着讽刺。没办法,此时人都是讲究着个视钱财如粪土的名声,忌讳赤裸裸的追求利益,但私底下评说两句也无甚大碍:

“王相公为财计推新法,朝中已是沸反盈天,反对声只会越来越大,王相公身负天下重名三十年方才入朝,就不知他的名声还能撑上几年。不过只要能在三五年之内将河湟吐蕃收服,王相公纵使倒台,也与机宜无关了。”

王厚点了点头,“封侯之赏,是家严平生之愿。朝中局面如何,家严不愿去理会,只望能安安心心收复河湟。”

“这可是最难的。大将在外,天子不疑者有几?三人成虎,以曾子之贤,其母也不免惑之。天子对机宜的信重,可比得上曾子母子至亲?”

曾参是孔子的弟子,平素最有贤名。但一次一个与他同名同姓的人杀了人。亲朋好友听说后,忙去找曾参之母,让她早点逃跑以防株连。别人说了一次两次,曾参的母亲不相信,但到了第三次,曾参的母亲就跳窗跑掉了。

王厚给韩冈绕糊涂了,酒醉以后,头脑也是变得迟钝,“玉昆,前面你说王相公纵使倒台,也与家严无关。怎么现在又说家严会被三人成虎?”

“还没明白吗?”韩冈悠悠然的说道,“我说的其实是时间啊!机宜必须在王相公失去耐心之前,作出一番成绩,还必须抢在王相公失去天子信任之前,收复河湟!若是耽搁了时间,日后再不会有如今的机会了。”

王厚恍然,连点着头,“玉昆你说的是。”只是马上又唉声叹气起来,“只是说得容易,做起来就难呐!除非能赶走李师中。”

对于李师中的问题,其实王厚曾经有意无意的提起过。韩冈也考虑过不少办法,但想来想去,却想不出一个好主意,“去一李师中,又来一张师中,除非机宜能接任秦州知州,有苦劳而无功劳,在任的经略相公哪个会大力支持机宜。”

“接任秦州知州?哪里有那个资格。”王厚苦笑,“家严中进士才十二年。只任过一任主簿和一任司理参军,之后便因参加制举落选而弃职客游陕西。资历实在太浅了,莫说秦州这等要郡,就算普通的下州知州,也做不了。这点资历,当个知县过一点,做个通判则是勉强,高到顶,也仅是一军知军。不然天子为何不让家严直接担任秦州知州,偏偏只给一个经略司机宜?”

“知军?”韩冈脑中仿佛有道灵光闪过。

在宋代,州一级的行政区划,还有府、军、监等名号,比如长安就是京兆府,秦州北面还有个德顺军,蜀中则因富产盐井而设立了一个富顺监。一般来说,曾为古都,或是曾为天子潜藩的州,会升格为府,通常比州要高上半级——可算是后世的副省级城市。

而军则是属于战略重点区域,户口数量不足,辖下县治只有一两个,不够资格为州,只能称作军——在韩冈理解中,相当于省管县。至于监,那是相当于地市级的大型国有矿业集团。

“如果在秦州西面设立一军,不知机宜有否机会担任知军?”

“渭源?丁点大的寨子,户口才几百!”

“不是渭源,是古渭!”从伏羌城往渭河上游去,一百八十里抵达古渭【今陇西县】——因其为唐时渭州而得名——再过去六十里,才是渭源。

“古渭建寨已经有二十多年,聚于城寨周围的蕃汉户口不下千家,足以支撑起一个军的基本户口!”韩冈越说越兴奋,经略司只掌握兵权,控制不了财权,一旦王韶成为新的古渭军知军,渭源必然会划归古渭管辖,那李师中根本没有办法再在资金上卡王韶的脖子。

同时在西北边境,县改军,寨改军,都是极常见的事。渭州北面的镇戎军【今固原】,便是在至道三年【西元997年】由高平寨改为军,户数至今也不过才一千多。秦州东北的德顺军,更是在庆历三年【西元1043年】由笼竿城升军。古渭建军,只要政事堂通过,天子首肯,便再无阻碍。

“古渭……建军……”王厚喃喃念着,眼睛越来越亮。啪地一声他重重地一拍桌案,跳将起来,拉起韩冈的胳膊,“走,去见大人去!”

ps:北宋的高利贷是吃人的,一年利息把欠账翻倍,是很普遍的情况,多少豪族世家官宦靠着高利贷来充实家财?数也数不清楚。虽然青苗贷的目的是为了充实国库,但其作用却是把世间通行的利息压到百分之四十,其间,断了多少人的财路,惹怒了多少敌人。

断人财路如杀人父母,所以王安石积攒的三十年人望,就转眼间化为泡影。他的政敌司马光也许是个正人君子,但并不意味着司马光所代表的阶级是正人君子的集团。身为旧党赤帜、领导世间舆论的司马光,以及以士大夫、豪商、皇族所组成了既得利益集团,两者的结合,便是变法的最大阻力。

如果以为这样的裂缝可以用些拍拍脑袋便想出来的小手段弥补,那就是天大的笑话!利益的争斗是你死我活,这才是本质。想双赢,也看人家肯不肯。

借用一句俺从论坛上看到的一句话做总结:

世界从来不简单,历史何尝会温柔——by马前卒。

