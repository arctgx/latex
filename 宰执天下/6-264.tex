\section{第35章 历历新事皆旧史(四)}

“汝霖。这件事可就要拜托你了。”

会后,韩冈回到厅中。端着新出的搪瓷茶盏,捂着手,问面前的宗泽。

宗泽拱了拱手,“相公既然将此事交托下官,下官必竭尽全力,彻查此案,不教一贼脱逃。”

宗泽没有推脱这桩回乡查案的苦差事,韩冈赞许的点了点头,却听宗泽问道:“这件案子,不知相公怎么看?”

“虽说两浙路几处丝厂接连被焚的确蹊跷,但工厂苛待工人也是事实。没有他们的贪心,贼子也煽动不了那么多人,陇西棉厂办了近二十年,也没见被人烧了。”韩冈看了宗泽一眼,道,“橘生淮南则为桔,生于淮北则为枳,无他,水土异也。工厂设于北方,奄然无事,设于南方,则乱事迭起。无他,民风有别也。北人重于义,南人重于利。北人顾义,办厂得利,与工人均分,故而四方闻招工,则熙熙然而就。南人逐利,办厂得利,则悉藏于家中,锱铢不与他人。今观北方棉厂之安,南方丝厂之乱,南北之分昭然可见。”

宗泽在官场中浸淫日久,但这番话只听到一半,还是涨红了脸。韩冈的根基在西北,但他从来没有歧视过南方的士子。沈括、黄裳、宗泽,哪个不是南方人?宗泽从来没想到韩冈突然间会攻击起南方人来。

一等到韩冈说完,宗泽便立刻大声驳道:“相公此言大谬!”

“这后半段话的确是错了……”韩冈很直率的点头,“好了,这地域歧视先收一收吧,这一次丝厂遭火焚,的确是有几成缘由是因为南北之别,却绝不是全部。但是汝霖……”他抚着茶杯,低沉的说着,“你得承认,南北的差异是的确存在的。南方的那些工厂主,有钱有势,有亲族,有靠山,却不知道聚众二字有多可怕。还当在他们工厂里做工的,跟他们的佃农一般吗?”

“那北方……”宗泽又欲争辩,但话刚出口,便猛然醒悟。

韩冈抬了抬眉毛,道:“北方多结社,又多保甲,寻常便见多了几百人聚集一堂同做一事,怎么处置,上下皆有心得。也不会糊涂到把自己工厂里的工人往死里逼。”

“河北丝织业的情况其实也不好,过去辽人多河北丝绢,但如今海运已通,铁路也同样贯通,北方的丝绢价格一降再降,一座同样规模的丝厂,在河北只能赚到江南的一半。若是河北的工厂主学江南,河北丝厂的工人肯定早就揭竿而起了。但北方民风彪悍,家族庞大,很少有人敢于明着鱼肉乡里,而且北方拖欠工钱的情况很少,尽管在明面上,在北方丝厂做工的工钱要少于南方,大约只有八九成,可怨声载道的情况并不多见。”

“不同地方都有各自的特点。北方的工厂因为民风和风俗而不忧动乱,而朝廷的工厂,多在京畿,人数数以万计。谁敢克扣工人钱粮,那就是祸乱京师的罪人,没人敢担这份责任。”

朝廷的产业多如天上繁星。钢铁厂、玻璃厂、眼镜厂,还有铁路、矿山,论收益,论规模,雍秦商会的成员加起来也比不上朝廷辖下产业的十分之一。

在这些国有企业中,小工皆有军籍,大工更是有望为官,人人都是拿着朝廷的俸料钱。加之军器监、将作监管束甚严,两府又极为重视,工人们温饱无忧——当然,除了矿山。不过大多数矿山开采了多年,矿工们早就习惯了那样的生活,不像江南的丝工,基本上都是破产农民就职,完全适应不了工厂里面的管理制度。

雇佣人数超过两百的私人工厂,在南方的绝大多数地区,是个新生的事物。劳资双方都是新手。一方有着资本家固有的贪婪,却没注意到工人与农民在行动力上的差别;另一方则还没有适应参与工业化生产时所必须遵守的纪律和工作强度。所以在矛盾产生的过程中,激化和爆发成了常态,等到大部分人都在磨合下适应,如今的乱象当会弱化,然后……持续下去。

像丝厂这样劳动密集型的工厂,工作环境又极端恶劣,其实雇佣男子远不如雇佣妇孺。易于管理,也不用担心她们会串联作乱。可惜在大宋,想雇佣妇孺做苦工,难度可不小,而且有儿童蒙学入学率作为官员考核标准,官府也不会坐视。

宗泽沉默的点了点头,在这方面南北的差异的确存在,不用韩冈说,他自己也清楚。

见宗泽服了气,韩冈更是语重心长:“之所以对汝霖你说这些,只是希望你去了两浙,不仅仅是抓捕贼人。那只是治标,却不能治本。”

“下官明白。”宗泽说道。

“其实有了这场乱子,江南的工厂主们自然会收敛一点。”韩冈笑着说道,揭开盖子,喝了口茶。

利益争夺本就是你来我往,在争斗中取得一个平衡。不过这平衡并不牢固,随时都在酝酿着下一次的动荡。但韩冈还是希望,宗泽这一回下江南,能让这个平衡维持得更久一点。

“不过汝霖你方才也说了,乱民集中在两浙,其中必然有其因由。至于这因由……”

“必是妖人邪教,否则绝无可能煽动多地丝工。”早在几日前,第一家丝厂受袭的消息就引起了韩冈的重视,几天里,宗泽多方查证,早已想得通透,“只看数日间,相隔数百里的丝厂相继乱起,便可知这些妖人势力定然不小。”

“嗯,说得有理。”韩冈道,“等到了两浙后,汝霖你可向提点刑狱司多借些人手,若有变,可发金牌急脚上京,至少两个指挥的神机军能调出来给你。”

宗泽心中一凛,“当不至于此。有官府……”

“汝霖!”韩冈打断了宗泽的话,“当往最坏处做打算。我曾听说过西域的一句谚语,面饼总是涂了肉酱的那面先落地。”

“下官明白了。”宗泽一瞬间的惶惑之后,又恢复了冷静,斩钉截铁的说道,“但下官会竭力阻止事态恶化到那般田地。”

“相信汝霖你一定能做到。”韩冈展颜笑道。喝了口水,他又道,“如今铁路纵横如阡陌,千里之行只需数日。日后穿州过县的贼人将会越来越多。像这一回的煽惑、纵火的案子,只靠一州一县,要破案着实不容易,甚至交给一路都吃力。若是有人沿着铁路犯案,从扬州行到定州,这样的贼子凭现有的人力怎么抓?”

……………………

躺在摇晃的床上,宗泽久久没有入眠。已经在南下的路上,他还在想着韩冈早间说得那些话。

尤其是最后,韩冈透露了要设立新衙门的打算,很有可能要在维持铁路治安的军队之外,增设一个专一用来捕盗的衙门。那时候,追捕江洋大盗,可就是由这个衙门,在各州县和提点刑狱司的辅助下来进行。

不知道到时候,会被人怎么说了。

宗泽暗叹道。

就像这一次的事。韩冈刚刚推动朝廷颁下鼓励工业的诏令,突然间就出了漏子,必然会有人开始攻击韩冈的政策。宗泽匆匆南下,便是要解决这个问题。

宗泽很明白,这些攻击,只要他把差事办好就能解决了。把他派去江南做什么?就是把幕后黑手挖出来,然后将责任全推到他们身上去。至于那些残苛贪婪的工厂主,韩冈也给了他处置的权力,要不然也不会多费唇舌说了那么多。

这也是改制带来的问题。尽管韩冈没有明喊着变法、改制。但在不知不觉间,韩冈已经将制度改变了很多。

宗泽就在韩冈身边,对此看得十分清楚。甚至还听韩冈说起过,他对城市与农村的看法——韩冈当时使用的词汇很陌生,但宗泽的确是听懂了。

一直以来,农村与城市在经济上最大的区别,便是一个是生产者,一个是消费者。

城市虽富,可财富皆来自于四方田亩。尤其是开封,富丽甲于天下,但百万军民,皆仰食于江南,文武百官,俸禄皆来自于四方。

但随着开封府的工业开始发展,钢铁、玻璃、等产业占据了各自大半市场,来自于工业上的财政收入,其实已经超过了开封府界之内的夏秋二税,与包括铁路印花税在内的商税一起,占据了总税赋的近八成。

这一方面有开封府界内的田土多属于世家大族,税赋本少的缘故。另一方面,也的确是开封工商大兴,远过旧年。十年间,开封税收增长两倍有余,单单只靠田亩两税,怎么也不可能有如此迅猛的增长。

但在韩冈眼中,旧日的财政体系,已经不能适应日渐繁盛的商贸体系,甚至连政治体系,都远远跟不上时代的变化。

“开封府内,以六曹治民,以两厅理民,以三院安民。但铁场户口,不在开封府内。”

这是前段时间,韩冈私下里对宗泽说的。

来自于铁厂的税赋……朝廷压根就没收过铁厂的商税。铁厂的盈利,直接就送进国库了。要研发,要改建,要增产,决定权都在朝廷手上。收税?那朝廷要亏多少钱?!

宗泽知道,韩冈对此一直都有想法。可韩冈到底要怎么改变,宗泽并不清楚。

他只知道,这个天下,就像他现在乘坐的列车,已经在韩冈设定的轨道上跑得越来越快,快得无法再停下来了。
