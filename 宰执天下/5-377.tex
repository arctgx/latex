\section{第34章 为慕升平拟休兵(九)}

飞马追上了最后一名辽军骑兵,用神臂弓射中战马后腿,韩中信最后用着他的铁锏,将那名不幸的骑手连着头盔和脑壳一并给敲瘪了下来。

短暂而又激烈的战斗之后,韩中信所部顺利的解决了辽军的这一支探马小队。

在两军探马的交锋中,很少有这样一边被完全歼灭的结果。若说原因,除了运气之外,更多的还是韩中信作为一军副将,他率领着这一队探马,乃是秦琬从军中特意挑选出来。人员精锐,器械精良,是其他探马小队所不能比的。

此外在马背上用弩,尤其是神臂弓这般大型的蹶张弩,很是需要一点技巧。韩中信的这个小队都是几次出巡的精兵,韩中信本人也着意练习过一番,而辽军是初学乍练,这就又差了一筹。

结束了战斗,除了两个望风的骑兵,其余士卒纷纷下马,一部分赶过去救助落马的兄弟,另一半则去收拾还没有断气的辽人。

不知是运气还是甲胄坚固的缘故,落马的几人中,只有两名士兵不幸丧生,其他几个都是或轻或重的外伤,学过一点医术的韩中信一一检查过后,确定只要回去看过军医,都能保住性命不会落下残疾。

待韩中信检查过伤员,并和学过急救术的两名队正一起给他们处理过伤处,十八个辽军骑兵的首级已被卸了下来,各自拴在马上。由于韩冈规定,一场战斗的斩首,由全体参战官兵分享,并没有产生争抢人头的冲突。这让韩中信省了不少事。

漫步在战场上,半绿半黄的麦苗在杂草中拼命挤出头来。荒芜的田地没有引发韩中信的感慨,相反,他的心情很是轻松,甚至是雀跃。这场战斗对他来说很有些纪念意义,能领军上阵杀敌,而且还轻松取胜,是他在军中站稳脚跟的第一步。

不过当他拿起辽人所用的重弩时,脸色还是变了。果然是神臂弓,而且刻在从弩架上的铭文来看,是元丰二年军器监所出品。监造和大工的名字,韩中信都听说过。

河东军中,代州是最先换装的一批,熙宁三年、四年就用上了神臂弓。所以平夏战后补充战损也是最早且最多的一批。从铭文上就能看得出来了。

“巡检,真的是神臂弓。”一名士兵抱着一堆神臂弓回来,举着给韩中信看,“都分不清哪张是我们丢的了。”

“元丰二年的是辽贼带来的,熙宁年间当是我们的。代州的武库可是全落在了辽贼手中。”韩中信用脚尖踢了一下脚下无头的尸骸:“去查查他们身上的盔甲,肯定也是从代州武库中得来的。”

盔甲也是战利品,几名士兵过去吭哧吭哧的把甲胄都卸下来。兵器、刀剑也都集中在了一起。

“巡检,这些辽贼带了不少好东西啊!”一名士兵叫了起来,手上举起了个羊皮小口袋。

打扫战场的士兵在辽人的尸身上都掏摸了一阵,无一例外的都摸出了一小袋金银珠宝,堆在一幅羊皮上,闪闪的光泽晃花了一群没见过世面的穷赤佬的眼。

韩中信瞥了一眼,心中估算,最多也不过两三百贯的样子。

“你们都分了吧。”韩中信家底不少,看不上这点小钱,“给蔡九和张狗儿多留一份。”

蔡九和张狗儿就是战死的两名士卒,多给他们的遗属一份,自然不会有人说不。可是说不的没有,动手拿的也没有。人人都站着,没一个动手的。

对于战利品的分配,军中管得不算很严,尤其是这样零碎的收入,上面的将领也都是睁一只眼闭一只眼。这里都是老巡兵了,畏惧军法不敢分账的情况,让韩中信都为之楞然。

不过当他多想了一下,也就明白了过来。自己要是不拿,下面的士兵也不敢大着胆子分赃。只有自家先拿了,这群士兵才会将自己视为同伴,这也算是投名状。

想明白后,韩中信也就不充大方了。只是在一堆黄白的财物中,他没看到心仪的东西。正想随便捡几件银首饰,眼光一扫,倒是在兵器堆里发现了一柄腰刀。

探手拾起腰刀,两尺长,刀型略弯。刀鞘色泽沉黯,却不知为何带着隐隐的绿芒,从纹路上像是鱼皮,很是精致,上面还附有粗犷却不失.精美的银饰。这样的风格让韩中信一下就喜欢上了。

可是当他拔出刀来,眼中的欣赏就不见了。腰刀看着应该是折锻铁,经过了多层折叠锻打,只是不论从质地,还是锋锐上,都远远比不上军器监中出产的精品。

韩中信腰间就有一柄刀,是得官后由韩冈所赐,能一刀砍下首级而锋锐不损,是柄名副其实的宝刀,出自于军器监的名工之手。而这一柄,也就刀鞘很不错,刀身比军器监量产的腰刀强不到哪里去。

韩中信微微的摇了摇头,可惜这刀鞘配不上自携的宝刀,不然直接就把刀给丢了。

‘看来要欠周独手一个人情了。’别人都是刀鞘配刀,韩中信倒是打算弄柄好刀来配这个刀鞘。

见韩中信将刀收回鞍袋,士兵们齐齐松了口气,兴高采烈的开始分赃。片刻之后,地上的小堆金银珠宝不见了,人人都是一副心满意足的表情。望着韩中信的眼神,也更见亲热。

能打能杀的军头才能得到士兵们的认同,如果再大方一点,那就是不仅仅是认同,是进一步得到士兵们的忠心和喜爱。

收拾过了战场,探马们就开始准备重新出发。再迟一点,可能就有新的一队辽兵赶来了。

一群士兵踩着铁镫,给神臂弓重新上弦。

“要是单手的就更好了。”一名队正举了举手中的神臂弓,“两只手在马上太麻烦。”

“你当军器监没造过啊?”韩中信摇头笑,“单手的小弩连马背上防箭的毛毡都射不破。造出来后,给人拿去打鸟去了。”

给神臂弓重新上弦时,有两张直接就坏了,其中一张碎裂的弓臂反弹起来时还将躲避不及的士兵砸破了鼻子,弄得满脸是血。

只是损坏的神臂弓才两具,而从辽人那里收回的则有二十六七张虽然不能做到一人三具神臂弓,可有好几个辽兵都是一人携带两具,看起来应是有些地位。

“倒是赚回来了。”韩中信笑道。

不仅仅是神臂弓,甲胄、弓刀、鞍鞯、辔头都是能换来丰厚赏赐的战利品。四十匹契丹马驮着这些收获,夹杂在小队中返回土墱寨。

韩中信骑着高大的河西马,被士兵们簇拥在中间。看着今天的收获,心中欢喜不已。

只是当他回望着身后荒芜的原野,心中的喜悦一下就消退了,在半年前这里还是男耕女织的乐土,现在已经不见人烟的荒原,不知要多少年才能回复元气。

辽贼该死!他一下握紧了腰刀。

……………………

辽军开始给前线的远探拦子马装备神臂弓。这个消息很快传到了忻口寨。

并不是每一队都能有韩中信那样配置和运气,也因此,就很难出现同样的结果。

在马上拉开步弓,的确很不容易。但将提前张开的神臂弓举起来并不是一件那么难的事。辽军很快就适应了马上使用弩弓的战术,这使得前线斥候的伤亡在几天之内陡然增加。

韩冈面前有一份伤亡报告,在醒目的位置上有两幅坐标图。

这是韩冈让人画出的图表,最开始很多人不习惯,但习惯之后,就觉得这图表比数字和文字更让人一目了然。

章楶也在看着同样的报告。前一幅图的纵横坐标分别是日期和伤亡,后一幅图则是牺牲的士兵和战果对比。两幅图表上的曲线变化,让章楶看得触目惊心。

“怎么伤亡这么大?”

韩冈苦笑道:“我们跟辽军骑兵的差距就这么大。之前靠着神臂弓胜过一筹,现在是回归了正常水平。”

其实从后一张图表上看,宋军骑兵阵亡的数量虽然比之前大幅攀升,但与战果基本上还是一比一代州骑兵的骑术并不比辽军骑兵差到哪里,而战斗**也高于辽军许多。只是骑兵数量上的差距,会很快把代州骑兵的血耗光。

一旦没有了外围的耳目和护卫,辽军的动向就会变得扑朔迷离。大军最多前进到崞县。再往代州去,行军之路就会变得很危险了。被萧十三放在必经之路上的近万兵马是个严重的威胁。

当辽军用轻骑兵牵制和迟滞军队的行进,到了合适的时机,一队具装甲骑冲撞过来,西军还好说,京营禁军是肯定要崩溃的。

“我们需要更多的骑兵。”韩冈说道,“京营禁军中的骑兵要打散了补充上去。但这还不够!”

“要把阻卜人调进关来?”章楶问道。

“当然。阻卜人的家眷要转去忻州或太原暂住,而能派上战场的青壮……就让他们好好卖一卖命吧,可不能白养着他们。
