\section{第41章 辞章一封乱都堂(三)}

【第三更。求红票,收藏】

张戬记得韩冈家世并不好,甚至不是书香门第,更不能与种建中那等将门弟子相比,但就是因为如此,才显得不到二十便引动天子颁下特旨的韩冈是如何不简单。

“玉昆你能同时得王韶、吴衍和张守约三人青眼,才学当是不差,怎么不安心下来多读两年,也好考个进士出来?”

“秦州虽大,却也摆不下一张安静的书桌。”韩冈感慨着,“外有西贼肆虐,内有蕃部不顺,年年烽烟不断,怎能安心读得下书去?”

韩冈的话惹得张戬颔首称是。当年李元昊举起叛旗,张载同样有着投笔从戎的心思,若不是有范仲淹、韩琦一众名臣来镇守关西,动荡的局势也容不得张载、张戬安安心心的读书下去。“既然玉昆你是王韶所荐,那应是为了开拓河湟喽?”

“正是当年子厚先生首倡之议!”

“开拓河湟,钱粮、人马都要千里迢迢的转运过去,秦州百姓便要受罪了。”有个知兵的兄长,张戬当然对开拓河湟的战略有所了解,其利弊亦是心知。

“……总得试上一试!一旦真能收服河湟蕃部,秦州便为腹地,生民也便不用再受战乱之苦,这是一劳永逸。”韩冈年轻的脸上透着坚毅,“其事虽难,若是还没有做过便放弃,心中总是不甘心!”

这话若是由他人说出,张戬必然拍案怒斥,而程颢也要摇头,语重心长地开始劝诫。但韩冈是张载的弟子,并非外人,年轻人的冲劲却是让张戬和程颢看着喜欢。即便他说出的话有些幼稚,但想来也是因为太过年轻,思虑不足的缘故,不是本心上有错。

只不过河湟之事,得王安石之力甚多,张戬和程颢这时又想起称病请郡的王安石。心道‘王介甫若去职,韩玉昆的职司,也许要生变数了。’

……………………

中书门下。

也即是政事堂内,一名又高又胖的堂吏脚步匆匆,沉重的脚步声传遍廊中。

曾布听到脚步声,放下手中正在读着的老杜诗卷。他身为检正中书五房公事,总理并督察中书门下吏、户、礼、刑、工五房吏人公事。职位要津,庶务繁芜,但凡发往政事堂的公文都要管着。平日里都是忙得团团转,也只有今天,他自任职以来才第一次这般轻松过。

胖堂吏走到门外,对里面喊道:“都检正,三司方才又来人了,急着要昨日发来待批的公文。”

“让他再等一等!”曾布摇摇头,拿起茶杯啜了一口,“此事需待王大参回来再批。”

“小人明白!”胖堂吏今天已经好几次往返于前院和检正厅,得到的回答都是一样——等王大参回来再批。但这一请示的环节他不敢省,自以为是,砍头的可是自己。

胖堂吏转身要走,曾布自后面叫住他,把他唤进公厅来:“曾相公、陈相公,昨天可曾说什么?”

胖堂吏是曾布的亲信,既然曾布有问,便不敢怠慢:“昨天王大参从宫中出来就没回政事堂,后来宫里传出消息后,曾相公和陈相公便想立刻下堂札停止推行青苗法,但赵大参却说,是王大参弄出来的事,得让他自己自己回来废除。”

“赵阅道帮了大忙啊!”曾布笑着,心里却对赵抃没半点感激,却在想赵抃一点担当都没有,又不敢做事,难怪总是在叫苦。

曾布昨天一听到宫里传出来的消息,就赶去王安石府。他跟吕惠卿、章惇等一众变法派的中坚官员都在门房候着,待了整一天,也没见到告病的王安石,不过把心意传到就已经够了。只是曾布没想到,他这么一走,昨天在政事堂中竟然发生了这么多事情。

山中无老虎,猴子称大王——尽管有两只猴子的官职比老虎要高——“还真是有趣!”

胖堂吏则在不无忧虑的看着堆满了曾布桌案的厚厚几撂公文,忧心忡忡。“都检正,积压了这么多公文,不会有问题吗?”

“你担心个什么?”曾布站起身,徐步走出门,回头望着北面的宫城,崇政殿就在他视线落下的方向,“不用急!参政很快就会回来!”

崇政殿。

赵顼现在很烦躁。他低头盯着铺在御案上的王安石的请郡折子。‘臣请辞’几个字一入眼,就像被烫了一下,视线随即便离开了那份辞章。年轻的皇帝并没有料到,只因韩琦的奏章,他犹疑了一下多说了几句,王安石的反应便会这般激烈。

好歹是出身在皇家,宗族中有形无形的勾心斗角也见得多了。赵顼登基时日虽短,但王安石为何会如此做,他还是明白的。而王安石的目的,赵顼也一样清楚。

可韩琦是三朝老臣啊!相三帝扶二主,没有韩稚圭,英宗坐不稳皇位。他赵顼能坐在这个位子上,有韩琦的功劳在,他的恩德不可不念。韩琦说的话即便不相信,也得做出个相信的样子,这才是顾全老臣体面的做法。

但王安石那边又该怎么办?听他自去,不再变法?那钱哪里来?军队如何整备?失土如何收复?二虏如何降伏?!

罢去新法可以!罢免王安石也可以!但你得给我个富国强兵的方略来!

韩琦给了,让他‘躬行节俭以先天下,自然国用不乏’。但将每年朝廷收入的五六千万贯全部吞吃掉,还要带饶个几百万贯封桩钱的三冗——冗兵、冗官、冗费——有哪一条说的是皇帝?这些钱几乎都是被数万官员,百万军队,还有几千宗室花去的!

仁宗、英宗,还有他赵顼,哪一个是奢用无度的昏君?没有啊!仁宗皇帝大行前,身上盖的被子是旧的,用的茶盏是素瓷的。先皇登基四年,病得时候居多,宫舍、出游,会花大钱的支出一项也没有。连大殓,也是因为距离仁宗驾崩才四年,国用不支,费用一省再省,害得自己连孝心都尽不了。而他赵顼呢,自登基以来何时奢侈过一星半点?!这样的情况下,自家再节俭,能节俭多少出来?即便自己一点不用,也不过省下几十万贯。这对三司账簿中越来越大的窟窿来说,是杯水车薪。

王安石不能走!从昨日想到今日,赵顼越发的肯定,王安石不能走!要想富国强兵,实现自己的梦想,就不能放王安石走!

如果不能两全,必须要做一个选择的话,赵顼很清楚该选谁!

崇政殿中,宰执、两制,决定大宋国策的十几位重臣都在等着赵顼从沉默中醒来。站在宰执们的下面,司马光平心静气的等着。不同于曾公亮、陈执中的心浮气躁,不同于文彦博、吕公弼的急不可耐。几位翰林学士中排在第一位的司马君实,始终都是保持着冷静的态度,仿佛变法的存续、王安石的去留,如流水过石,在心底没有引起一点动摇。

不知过了多久,赵顼抬起头来,神色间没了犹豫:“变法刚刚开始,王卿实走不得!司马卿,你为朕草拟一份慰留诏书。”

赵顼的话,让宰执们一阵骚然,而司马光应声答是,接下了旨意,退后去写诏书。他是翰林学士加知制诰衔,正是有资格草拟诏书。

“陛下!”文彦博却是当先上前:“天下纷纷,皆为新法。新法悖时难行,天下士大夫无人不言。王安石既已然自知,何不从其愿,放其离京?!”

“文卿何出此言?!”赵顼又惊又怒,他知道文彦博与王安石互为政敌,但天下纷纷之说,未免也太过了一点。别以为他年轻不晓事,青苗贷的实行过程中的确有问题,但使人监督并修改一下,当是能解决。只要修正了,青苗贷对百姓只会有好处。他当即批驳,

“更张法制,于士大夫诚多不悦,然于百姓何处不便?”

文彦博生于真宗景德三年【西元1006年】,到了如今的熙宁三年,已年过花甲,几近古稀。六十五岁的他老迈龙钟,身子佝偻着,皮肉都松弛了。但宽大的骨架子一旦挺直,数十载为相而产生的压迫感,便宛如一团阴云沉甸甸的压向年轻的皇帝。他冷笑,从唇缝中挤出的苍老声音,就像从崇政殿外呼啸而过的寒风:

“陛下!天子为与士大夫治天下,非与百姓治天下!”

竟然敢这么说?!

赵顼闻言一惊,双眼瞪住文彦博。而文彦博则垂下眼帘,但身子站得更直。殿中的重臣们没有任何反应,仿佛没有听到文彦博的话,又好像默认了说进他们心里的这一句。

对,文彦博说了大实话。无论是变法,还是反变法,两派之间的笔墨往来,尽管都是冠冕堂皇的说着是为天下百姓着想,但实际上考虑到百姓只是附带。青苗贷能稍稍惠民,却伤了士大夫的利益。文彦博这是在提醒赵顼,不要忘了天子之位的根基在哪里。

朝堂上每每争论治国之策,都是把百姓拉出来为自己的话做背书,哪一个不是摆出为民请命的态度。三年来,赵顼还是第一次从臣子的嘴里清楚的听到治理家国的本质。即便过去王安石与他谈起青苗法的本意,也要遮遮掩掩,不肯把话说透。

是不是该谢谢文彦博?这些年来,这位文相公还是第一个肯跟他说这些大实话的臣子啊!

