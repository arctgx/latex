\section{第一章 庙堂纷纷策平戎(五)}

【学习一天,没法儿上网。先发昨天的两更出来,待会儿还有一更。】

夜幕降临。

二十余支儿臂粗细的巨烛高燃,照得吕府招待近亲戚里的小厅中亮如白昼。

这样数目的宫中用烛就是天子早年都舍不得多用,朝政处理得晚了,才会点起几根来。也只是如今口袋里面有钱了,才会在崇政殿、福宁宫见得稍多一些。

吕惠卿倒是不在乎被人说他奢侈。论穷奢极侈,他还远远比不上,连茅房厨房都放上蜡烛照明,一设宴就喝上一夜的寇准。

而且御用的贡品价格能比平常货色高出十倍去,其实也不过是掺了点上等的香料,基本上就是从天子手里捞钱罢了,实际价值远远比不上价格。吕家现在用的巨烛,就没那么离谱了,价格很正常,也照样掺了些香料,仅是不及御用的高档而已。

巨烛的照耀下,吕惠卿和吕升卿两兄弟,招待着突然造访的徐禧。

虽说是吕惠卿的儿女亲家,可徐禧选择在这个时间上门拜访,自然不会是为了聊一聊天气,联络一下感情。

“吉甫。”酒过三巡,徐禧图穷匕见,叹气道:“平夏之事,如何能让王禹玉占了先去?”

吕惠卿正举杯喝酒,没空说话,吕升卿帮衬着笑道:“王禹玉是宰相,本就排在大哥之前,怎么能不让他占先?”

徐禧横了吕升卿一眼,你是在说什么胡话的想法没明说出来,却在叹气:“难得的机会啊。”

“就给王禹玉好了。从种谔上书时,王禹玉就看上了这一件好事,孜孜以求,只是被韩冈给耽搁了。现在好不容易又重新浮上水面,这时候想横插一杠,抢他的风头,那是会将王大丞相向死里得罪。”

“开罪一庸人又何妨?”徐禧还当真敢说,在吕惠卿兄弟二人面前毫无半点讳言,“辽国内乱,一时自顾不暇。此次西夏又传来外戚干政,母后囚子的消息。而官军正是兵强马壮,良将如林。天予弗取、反受其咎,此等大好时机,错过一回,就不会再来。缘边六路,粮饷俱足,六路齐发兵,试问西贼如何与官军拮抗?”

吕惠卿素知徐禧好谈兵事,平日里说得最多就是平夏伐辽,就连文章诗词皆是偏向此类话题。由于他的诗文才华甚高,还在士林间出了好一阵风头,只是这两年被苏轼给压下去了。

徐禧本人对王韶、章惇的际遇,没有少嫉妒羡慕。在吕惠卿面前,多次流露出统领大军,一展胸中大才的想法。在吕惠卿眼中,他的这位姻亲基本上可以跟赵括、马谡相较高下,夸夸其谈的文士,叶公好龙的书生。

没有王韶、章惇和韩冈那样的实绩,放言兵事全都是空话。韩琦当年也是空谈兵事,葬送了数万精锐,要不是当时两府之中尽是庸碌无能之辈,他至少还有点胆气,早就完蛋了,哪里会有相三主、立二帝,成为两朝顾命定策元勋的风光?

这也算是如今士林中的风气。

在过去,大宋坐拥百万大军却连御敌于国门之外都做不到,必须要用卑辞厚赂来讨好夷狄,故而人心厌武——失败多了,自然会厌恶起来,此乃人之常情。吕惠卿也曾见家里面的子弟,因为支持的蹴鞠球队连败,而气得干脆不再看比赛。

而现在中国军力大振,平河湟、定荆南、收横山、灭交趾,一桩桩大捷撩动着人心,想学着班定远的士人就一下变得车载斗量。徐禧也不过是其中一人而已。

吕惠卿当初与徐禧结了亲家,一是因为他对新法的支持,另外也是因为天子对徐禧很是看重,加上徐家又是江西名门,姻亲甚多,也有引为助力的想法。而徐禧在担任监察御史的那段时间,也的确给了吕惠卿不少的帮助。

只是对于徐禧的性格,吕惠卿心中则就是有所保留了。“王禹玉只是在附和天子的心意而已,如果天子被郭逵、韩冈说服,恐怕就会改弦更张。到时候,王禹玉多半也就不会坚持要出兵了。”

徐禧放声长笑,拍着吕惠卿的手背:“吉甫,此话差矣!王禹玉在东府日久,几近十载,却无丝毫建树。观国朝百年诸多宰辅,才干政绩位列其上的不知凡几,可秉政比他时间长的却没几人,无他,听话而已。取圣旨、领圣旨、已得圣旨,三旨相公之名,卒为天下笑。如今二虏内乱,天子意欲先观兵西北,继而北收幽燕,这就需要朝堂上有贤相主持,王珪可能担得起这份担子?天子英睿,自然知道王珪不是能架上房的栋梁之才。故而王珪眼下才会尽力的想表现,若不能于西事上有所成就,他在政事堂中的时间可就不会太长了。”

王珪的盘算,吕惠卿只会看得更清楚,那几乎已经是司马昭之心了,但他也没必要跟徐禧明言,举杯道:“德占所言甚是,只是此事与惠卿何干?”

“怎么会没关系。王禹玉如今只想保着权位,全力迎合天子的心意,试问此等良机如何能轻易放过?”徐禧双目灼灼有神,盯住吕惠卿,神色中尽是急切。

吕惠卿却笑得从容淡定,仿佛事不关己:“王禹玉有了元厚之、薛师正相助,又是迎合天子的心思,多我一个不多,少我一个不少,王禹玉无求于我,我沾一身腥又是何必。”

“难道吉甫你就别无所求了吗?”徐禧沉声说道,“王禹玉一心要攻打西夏直取兴灵,吉甫你现在则是将身家赌在手实法上。你们是各有所求。如果吉甫你助其一臂之力,想必王禹玉决不会阻挠或干扰手实法的施行。”

想要知道的都是知道了,吕惠卿沉吟片刻,端起酒杯,“德占言之有理。惠卿受教了。”

听见吕惠卿终于松口,徐禧心中大喜,“不敢当。徐禧也只是想看朝廷在外能观兵兴灵,在内则手实法顺利实行罢了。”说着亦是举杯回应。

“徐德占还真是敢想,只不过是口才好,会写文章而已,当真以为自己有武侯之材。”酒宴之后,在席上没捞到几句话说的吕升卿送了徐禧回来后,坐下来就冷笑,虽然方才酒席上,徐禧没一句说自己想去陕西,但吕升卿如何听不出来,“看他的样子,恐怕还是像赵括、马谡更多一点……难道当真要举荐他去陕西?”

吕惠卿正在书房中喝茶消食,听到兄弟相问,放下茶盏,“他那边都打通了王珪的路,我这边拦着,岂不是平白无故的得罪人?”

“他已经走了王珪的门路?!”吕升卿顿时吃了一惊,瞪大眼睛,“不可能吧!刚才根本就没说啊。”

“方才他说得那番话还听不出来?”吕惠卿从鼻子里笑了一声,低头又端起茶杯。

吕升卿干笑着:“委实听不出来。”

“想想他为什么说只要我支持举兵伐夏,直攻兴灵,王珪就不会阻挠和干扰手实法的推行?”吕惠卿提示。

吕升卿还是茫然不解,摇摇头,很是疑惑的道:“这有什么问题?”

吕惠卿心中暗叹,自家的兄弟才学不差,给诗序做的注解,王安石和王雱都没法改,就是反应实在有些迟钝,其实并不适合在官场上做事。不卖关子了,详加解释:“‘朝廷以经术变士人,十已八九变矣,然盗袭人之语而不求心通者,亦十之八九。’这是徐德占当年在天子面前说的话……以他素来喜爱夸大其词的性子,应该说只要我支持王珪,王珪也应该反过来支持手实法才对是,为何这一次说话如此保守。”

“……王珪也真能信他。”被点破之后,吕升卿也想明白了,啧着嘴,“王禹玉乃是当朝宰相,手上不知多少人要安排,徐德占空口白牙的竟然从他夺下一块肉来,还当真是本事。”

“只是讼棍吃两头的手段罢了。”吕惠卿注视着桌上的烛台,纱罩中的火光映在眼中,“徐德占在王珪哪里,肯定是张着愚兄的幌子!不然凭他也进得了相府的大门?”

吕升卿心中顿时腾起一阵怒意,几乎要拍案而起,愤然道:“也亏他敢做!”

“他怎么不敢做?”吕惠卿语调平淡,“现在不就给他办成了吗。也难怪他卖力,韩冈跟他一样是熙宁二年由布衣得官,又是同在熙宁六年榜上锁厅登第。现在两人差距如此之远,不就是因为军功上远远不及吗?徐禧哪里会甘心。想要向上爬,能利用的当然都要利用。反正正合我心意,顺水推舟一把也无所谓。”

“也只是让他一时得意。”吕升卿呆了一下后咬牙发狠,“贸贸然去了陕西,看谁会听他的吩咐!”

“那要看他的本事了。”吕惠卿漫不在意,“徐德占的事,愚兄倒是不担心。西军的那一干骄兵悍将,也正缺人去磨一磨。不是说连年大捷都靠他们出力,朝廷就不敢动他们了。”

吕升卿想了一想后,苦笑道:“……以徐禧的脾气,说不定还真的做得出来。”

