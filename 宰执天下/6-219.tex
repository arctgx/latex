\section{第28章 夜钟初闻已生潮(上)}

“这就是水运时计台?”

受到苏颂邀请的韩冈,来到了司天监在内城一角的院落中。

墙外是潺潺的金水河,其中分出的一条支流,穿入了院落之后,一座楼阁的地下。

“正是水运时计台。”苏颂带着自豪的向韩冈介绍道。

韩冈抬头,他知道,这一座水运时计台,在原本的设计中,应该是叫做水运仪象台才对。

早在仁宗的时候,苏颂便曾经有意制造一座由水力驱动的仪象台。

在他的设计中,仪象台的最上部是浑仪,用来观测夜空,浑仪上面设计了可以自由开阖的屋顶,用以防止日晒雨淋。

浑仪的下一层是象征天球,演示天体运动的浑象。在苏颂的设计中,天球的一半隐没在“地平”之下,另一半露在“地平”的上面,靠机轮带动旋转,一昼夜转动一圈,真实地再现了星辰的起落等天象的变化。

再下一层就是时计,按时报告时间。最下面是引用水力的机器,通过水力来驱动整座仪象台。

但在气学格物一派,浑仪浑象被打入了冷宫之中,望远镜淘汰了旧有的天文仪器,成为了观测天空的新宠儿。

所以苏颂想要制造的浑仪浑象,变成了水运时计台。上面也有观测天文的仪器,不过是固定角度的望远镜。据他所说,要编列星表,确定哪一天的哪一个时刻会看到哪一颗星星,由此来确定时间。

苏颂的想法很有意思,不过韩冈觉得,没有三五十年的持续观测,连行星卫星、日食月食的运行表都不一定能够编列成功,更别说通过星星来计时了。不过天文观测是必要的,没有第谷,就不会有开普勒的行星三定律。观察宇宙,了解宇宙,需要几十年、上百年持续的观测,所以韩冈不会去泼苏颂的冷水,而是带着敬佩去期待着。

现在这座水运时计台,规模如同一间小楼,几乎有三层高。走进台中,一座楼梯联通上下,耳边是淅沥沥的水流声,还有机器轱辘轱辘的转动声。

脚底下的水流带动了时计台中的机械结构,通过一个个齿轮将水力传送上来,引导着时计指向相应的位置。

这或许可算是世界上最早的机械时钟了。韩冈想着。他不清楚西方现在到底发明了钟表没有。

但正处在中世纪的欧洲,在来到中国的大食商人嘴里,是不值一提的穷乡僻壤,是野蛮穷困而又不守信诺的异教徒的地盘,当他们怀念的说起希腊和罗马的时候,毫无例外的都在大加嘲笑此时欧洲人的愚昧。恐怕即是欧洲人发明了机械时计,也会被苏颂现在的发明所掩盖——以大宋现在的情况,苏颂发明的时计很快就会传遍天下。

通过稳定的水流来确定时间的流逝。所以水运时计台最难的一点就是如何保证机械运转的稳定。

苏颂亲自设计了时计中的机械结构,通过一个特别的轮机和齿轮杠杆结构,让上方水斗中流下来的水流,带动齿轮稳定的转动,而流下来的水,再通过楼底的水轮机送回上面的水斗中。虽然看起来很麻烦,但这可算是跨时代的技术了。

韩冈仰着头望着这座巨大的时计台,一开始的时候,他可不会想到苏颂能弄出如同一间房子这么大的机器来,随时随地都必须有人在这里进行维护,这个使用的成本未免太过惊人,只有朝廷的公帑,才能支撑得起此类器物。

不过,刚刚修成不久的水运时计台并不是苏颂邀请韩冈参观司天监的原因,他想要展示的并不是这一座时计,而是放在另一个房间的时计。

在另一个房间里,是一座一人多高的座钟,身材稍矮一点的人,踮起脚尖也不能与顶端平齐。只不过若是将这一座钟,放到外面的如同一座小楼的水运时计台旁边,那就是小猫与老虎的对比了。

座钟正面的上方是表盘,上面有着上端不一的三根指针,下方有一扇玻璃门,座钟的玻璃门后,钟摆正平稳的摆动着。

钟摆的摆锤看起来就像是武夫手上拿着的骨朵,外形几乎一模一样,只是变得扁而宽。而摆锤上连着的摆杆细长。

摆钟的本源,在另一个世界上为伽利略所发现,不过在这里成了韩冈所提出的原理。所以苏颂在成功之后兴高采烈地来找韩冈,向他感谢发现了摆动的意义。

钟摆的摆动总是保持着相同的时间,只要将这摆动通过齿轮连杆传送到指针上,就是一个标准的时钟。剩下要解决的问题,就是如何让钟摆维持摆动。

一刻刻,一时时,一天天,一月月,长久的摆动下去,这是最困难的一步。

不论是人力畜力都做不到这一点,现在的钢铁技术离弹性发条还有很长的距离,当然也不可能实现。在韩冈束手无策的情况下,正在设计水运时计台的苏颂,很自然的就想到了利用水力。这就是为什么座钟会跟水运时计台一起进行设计和研发的原因。

但到了最后,苏颂终于发现,座钟完全不需要什么水力。只要制造一个重锤,系上一根长绳,再将这条用丝线缠成的绳索绕在一个齿轮中心延伸出来的杆子上,在重力的影响下,重锤不断下降,带动齿轮转动,这就是现成动力。齿轮带动齿轮,一级级的传导着动力,通过一个小巧的擒纵装置,补充钟摆在运动中损失的能量,保证钟摆不断运行下去。人们所要做的,仅仅是隔上几日,重新将重锤绕上去。

重锤作为动力驱动,这其中只是一层窗户纸,剩下的,用现有的技术立刻就能完成。但在韩冈对此懵然无知,无法出面指点的情况下,苏颂和他手下的能工巧匠,在已经了解钟摆原理的情况下,只是绕路就用了整整五年的时间。

“这个时计误差是多少?”韩冈上下打量了座钟一番后,开口问道。

虽说眼前分明就是一个摆钟,但到了准点也不会报时,所以韩冈还不能随口一个座钟、摆钟,只能称之为时计。等到日后将之安置到钟鼓楼上,或是在城市的中央,修建起一座巨大的有表盘有指针的钟楼来,准时准点,联动起钟声,再称之为摆钟才名正言顺。

“每一天的?”苏颂问道。

韩冈点头,“就是一天的误差。”

“大约一刻钟。”

韩冈差点呛住,这是个极其夸张的比例,但这毕竟是新生事物,无法求全责备。

“水运时计台的误差就好一点,每天只有一分钟。”

有了机械时计,就有了划分时分秒的需要,韩冈向苏颂提出了自己的意见,而苏颂也愉快的接纳了韩冈的想法。将一个时辰划分初、正两小时——如子初、子正——再将一小时划分为六十分钟,分钟划分为六十秒,一切都符合韩冈的习惯。

“中间停下来上弦怎么办?”韩冈又问。

苏颂道:“可以用水运时计台进行校准。”

“那……”

“如果水运时计台坏了,就用摆时计校准。”苏颂冲韩冈笑了一下,他知道韩冈想问什么,“如果两个时计同时坏了,最后会用日晷来进行校准。而且每天正午,都会按照日晷来调整。”

每天正午都会处在正南方的太阳,人们给一天定义的本源,理所当然的是校准所有计时工具的标志。所以日晷,便是天底下最为精准的时计。

韩冈可以放心了,不会出现那个敲钟人按照计时的炮声来敲钟,而放炮人按照敲钟人的钟声来放炮的笑话。

相对于日晷,摆钟准确性还差得远,甚至还比不上刻漏。不过话说回来,这毕竟已经是一个让人惊喜的开端,没必要苛求太多。好与坏暂且不提,有没有才是最重要的。

犹如一间三层小楼的水运时计台不可能推广到民间,而摆钟却完全可以,现在终于有了一座有推广和利用价值的时钟,这就是韩冈一直以来所期盼实现的目标。

韩冈连声夸赞,苏颂对韩冈的反应十分满意,笑着道:“玉昆你放心,下面将会尽力做的更加精准。”

“那就再好不过。”韩冈迫不及待的搓了搓手,对苏颂道,“现在用刻漏来排定列车发车,浪费了太多时间。如果能使用更加精确的时计,现在发车的频率至少能增加一半。”

苏颂神采飞扬,灼灼眼神,在朝堂中时完全看不到,“也就是说,平白的多了半条铁路出来?!”

“只要拉车的马匹能跟得上,一条铁路说不定都有可能。”韩冈摇头叹了起来,“只不过现今已经在运营的各条铁路上,所使用的各色马匹已经超过一万五千匹了,再多一半需要增加的草料至少一百万束。”

“一百万束草,如果是苜蓿的话,差不多五十万亩田了!”苏颂叹息道。

“这么多的田地可不好找。”

一亩地能出产的干草料,也就两三束的样子。用五十万亩地来种草,即使不要良田,也是个巨大的数目。

“还是去找大辽的新郡王吧,用马是不可能了。”苏颂笑道。

韩冈笑了起来,摊着手,“那也要等人先发明蒸汽机,再去献给辽国的那位皇帝才行。”
