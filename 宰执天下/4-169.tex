\section{第26章 鸿信飞报犹绝迟(一)}

一封只有两页的信函拿在手中,韩冈却是翻来覆去的看了半天。

这是王旁寄来的私信,与自家妻妾的信件一并送来。虽然信并不长,但里面说的事不少。比如蔡挺在殿上突发风疾,比如吕公著回京,比如天子因为司马光修资治通鉴而暑病,特遣使赐药洛阳。但最重要的还是王安国的去世。

蔡挺在殿上发病,基本上他的政治生命算是完了。如果他不主动请辞,御史们的弹章能把他家门口给淹起来。枢密院刚刚多了名枢密副使,眼下就要又少了一名。人数依然不变,但西府中这几年来的固有格局已经发生大变。且吴充、王韶在枢密院的时间也已经很长了,很可能短时间内会有个变化——至少王韶出外的可能性很大。

而吕公著,他是铁杆的旧党,旧年还是他推荐了王安石,而后却因为反对新法而出外。包括他在内,一干旧党重臣在数年间陆陆续续的都被赶出了京城,由此确立了新法的权威。但眼下吕公著回京,让人不得不猜想,天子是否有意重新启用旧党。

这一点,在天子对司马光的看顾上得到了确认——绝不可能仅仅是因为听闻司马光在独乐园中中暑而特意赐药,以司马光旧党赤帜的身份,这么做的政治意味太重了。至少在过去,天子不会做得如此直接。

这三件事与韩冈的关系都不大,但接下来却跟着王安国身故的消息。

韩冈与王安国来往并不多,王安石的三个兄弟,最反对变法的就是他。但王家兄弟之间的情分很深,当年王益早亡,一家老小的衣食住行都是靠着王安石一人的俸禄支撑起来的,作为长兄,王安石为兄弟做了很多,而几兄弟对他敬重,也是不必说的。去年王雱病逝,今年王安国又病故,自家岳父会是什么样的心情,韩冈多多少少能体会得到。

按道理说,既然是王安国病逝,王旁就不该在告哀的信上牵扯其他杂七杂八的事。不过两遍一看,他这位内兄的用心差不多也能领会了。

“看来进益不小啊。”韩冈在小厅中自言自语,王旁在出来任官之后,这两年在各方面都有所成长,从这封信中也便能看出一二。

尽管王旁他在信上连只言片语都没有涉及,但韩冈能看得出来,王安石的心境有了变化,天子也有心对两府人事加以更迭,内忧外困,自己的岳父多半在宰相位置上做不久了。

‘是准备过河拆桥吗?’

韩冈虽是这么在想,心中却没有半点愤怒,只是为他的岳父感到几分悲哀,兔死狐悲、物伤其类的感怀。

赵顼这么做,是在尽天子的本分。

从政治的角度上说,新法几年内狂飙猛进,这时候肯定是需要稍微缓和一下。而且王安石控制朝政的时间也太长了。弱势、听话的宰执官,做个十几年都没问题,天子不需要为此而担心,而一个强势的宰相,三五年就已经让人嫌太长了。

而且这些年来官军胜绩累累,即便年年灾异,但朝廷的开支依然能维持平衡,赵顼富国强兵的夙愿已经成为现实,剩下的目标就是厉兵秣马,剑指西、北。以眼下的情况来看,只要将已经成型的法度和条令继续保持下去,达成最终的目标也只是需要时间而已。

从这方面看,王安石不再是必不可少的了。如果王安石能够主动请辞,多半就能留下一道君臣相得,善始善终的佳话吧。

韩冈摇了摇头,王安石不可能在相位上待一辈子,迟早要走的,趁着眼下国势大兴的时候离开,也算是个好结果了。日后朝堂上若有动荡,他再回来镇住朝局,这就是元老重臣的作用。

这一切应该就是在半年内有个结果,自己只要等着看就行了。

将信叠起收好,韩冈拿起桌上的一张名帖看了看,叫了门外的亲兵进来,“去门房,领武福、俞亭二人去偏厅。”

武福、俞亭是钦州疍民的首领,昨日韩冈派人传话今天过来,丝毫不敢推搪的就按时赶着上门来听候吩咐了。

韩冈到了偏厅的时候,两名疍民首领正局促不安的站着,见到韩冈终于出现,便连忙跪下来通名行礼。

韩冈坐下来看着两人,他们身上穿得甚是光鲜,一身绸布做的袍子,头上的帽子遮住了与汉人有别的椎髻,看不出来有什么特别的地方,除了肤色黑了一点,就是两个普通的富家翁,连肚子都是一般儿的装满油水。

待到两人战战兢兢的站起来,韩冈温和的笑着,“前日本官从交州泛海而回,正好看见有人在海上采珠,故而找你们来问一问。”

两人对视一眼,像是松了一口气,武福从袖中抽出一张礼单,恭恭敬敬的弯下腰,双手递上来:“相公,这是小人的一点孝心,微薄得很,不成敬意。”

“本官不是要你们的珍珠,一颗颗都是人命,本官也没心思拿。”韩冈摇摇头,看都不看的让他将单子收回去,“采蚝几百几千才能有一两颗上好的珠子,还要防着鱼虎【鲨鱼】,这份生计可算是辛苦。”

两人以为韩冈是故作姿态,便又劝了两句,等到韩冈一声怒喝,偷眼看到他的表情,才确认了这位年轻的转运相公当真是不想收礼,讷讷的将礼单收回去,“……相公说得是,的确是辛苦。”

韩冈悲天悯人的叹着气,“每年夏秋时节,又多有台风。靠海的州县年年遭灾,昨天我翻看籍簿,最近的十年,年年少说都有几十人殁于风灾。你们在海上,恐怕灾伤更重。”

“相公当真是心慈。我等在海上,哪年不死人?家家户户都有死在台风天里的。”

“即是如此,那为何不上岸买地,换个稳当点的生计?”

“都是这么想啊,可怎么也做不到!相公知我等辛苦,可钦州人哪里会管?我们疍人一说要买地,价钱都能翻上天去。”俞亭叫着苦,“小人两个几代辛苦,才攒了点身家,好不容易才置办了两块地,一间房。其他的人还不如小人,有点钱买点穿戴就散尽了,哪里还能置办得下?”

“方今交州新复,正乏人口,若是尔等能迁往交州,置地倒是方便的。”韩冈喝了口茶,漫不经意的提了一句。

“相公,小人都是习惯了钦州的水土,突然去了交州,水土不服。”

“交州也不愿,若说路程,也不过是顺风时往南一天的水路罢了!”

两人面面相觑,终于发现韩冈是认真的这么在打算。武福扑通一声跪倒,“相公要小人做牛做马都行,可这交州是万万不敢去。交州的风浪可比钦州更重!”

“不是说让你们置地建屋了吗?当然不会住在水上。”

“这……可是没钱啊。”

“那就更不用担心。到了交州之后,买地是另外算得,而官府都会给你们分配一份永业田,不要你们一文钱,足够温饱支用。日后有了田地,也不用再怕风浪,也不用再吃采珠的苦了。钦州沿海总共上千户疍民,估计也没有几个家有产业的。只要搬个家,就此有了产业,日后也能给子孙一个安稳的生活。”

韩冈一句句话,让他们无从推脱,武福和俞亭两人愣了半天,最后一咬牙,连连磕头道,“相公明鉴,小人世世代代的在水上讨生活,再苦再累都是一代代传下来的活计,总归是手熟。突然要小人去种地,可连锄头都不知道该怎么拿,只会将自家给饿死。”

“邕州左右江的溪洞蛮部也不会种地,但他们现在不还是在交州开垦荒地吗?总是能学着来的,官府也会派人指点怎么耕种。且刚开始的两年,不会收你们的税赋,若有灾,官府还会有赈济,一切都不用担心。本官也知道,一开始肯定是辛苦,但过些年也就能好起来,日后子孙不用再吃采珠捕鱼的苦,也不用再怕台风,这岂不是一桩美事。”

韩冈不厌其烦的为两名疍民首领解释着,但两人尽管砰砰的磕着头,额头都红了,但就是不肯答应下来。

低头看着脚前的两个磕头虫,韩冈的视线森森如寒水。

关于收编疍民的事,韩冈其实可以直接发布一道公文,传达自己的命令,剩下的具体工作自有地方州县来完成。

他都已经做到了转运使,为了这点事,亲自征求当事人的意见,其实说来即有失身份,同时也不并合乎官场的规矩。

这等于是不相信钦州知州的能力,同时若是出了乱子,也没办法将罪过推到下面的人身上,只能自己全数承担,算是自讨苦吃。聪明人都不该也不会这么做的。

不过韩冈只想看一看领导一地疍民的到底是什么样的人,并尽力将这第一步给走稳了。只要这个开头打得好,日后福建、两广,甚至还包括浙南,上万里的海岸线和江口、河口,总计十万的疍民,都可以按部就班的编户齐民,然后寻找合适的地方将他们安置下来。

