\section{第15章 自是功成藏剑履(11)}

冯从义商人的习气,听说有钱赚,眼睛都能发金光。

玻璃水银镜冯从义只是听韩冈提起过。但他素知韩冈的为人,不轻易出言,却言必有中。既然曾经郑重的说起过,又要让他去投资开发有关的技术,那么当然是门赚钱的好买卖。

西北的棉布,交趾的白糖,还有南北货转运贩售的生意,已经让韩、冯两家富甲一方。但支撑家族根本的产业,永远都不会嫌多。玻璃和镜子,乃是与生活息息相关。只要能将技术掌握在手中,就算日后扩散出去,只靠细水长流,那也是能传承几代人的富贵。

冯从义沉浮商海多年,喜怒不形于色那是基本功。但在上千万贯的远期收益面前,却也压不下心头的兴奋。别说他区区一介豪商,就是天子,听说了一年能几十万的纯收益,也照样不能免俗。

看着表弟的模样,韩冈微微一笑,低头又看着手上价值千万的小纸片。纸片上一排排小字写得很密,不仅仅是原材料的配方。除了那几条之外,还有烧制时要注意的关键,以及专用炉灶的修筑和运用。

“嗯?”看着看着,韩冈忽然神色一变,忍不住发出一声惊噫。

“哥哥,怎么了?是不是有哪里不对?”冯从义立刻紧张得问道。不要花大钱买来的却是个骗呆子的赝品。

“透明玻璃的烧制肯定都要附带工艺的改变,不仅仅是配方的问题。你能注意到这一点,的确是很好。”韩冈扬了扬纸片,“尤其是关于这烧料的炉灶,跟之前用的炉灶大不一样。”

“这是肯定的。将作监可是新修了炉灶,跟之前的样式都不一样,当然是跟白玻璃有关,小弟怎么能不去详加打听?”冯从义指着纸片上的最后一条,“从炉子里排出来的热气,透过穿过外带的烤炉,可以鼓进去的风加热,能节省不少炭火。”

“不仅仅是省炭火,更是在提高炉温。”韩冈语气郑重。

若不是在这张纸上看到,韩冈还想不起来这项关键性的技术。以蓄热室交换炉中带出来的热量,不仅仅可以用在玻璃的烧制上,炼钢炼铁上有更大的用处。韩冈在军器监时,这项技术还没有开发出来。

“炉温?”冯从义疑惑的问着,这个词可没听说过。

“就是炉子的温度。就像长短轻重一样,将寒热用数字来衡量,是为温度。这是愚兄最近想要做的事。旧有的度量衡不仅不精确,而且太偏狭。冷了热了都能感觉得到,但到底多冷多热,可就没个准了。”

冯从义半懂不懂,想了半天,试探的问道:“是不是炉温高了,就变得更热?”

“正是,温度越高,就代表越热。所以说白玻璃是个好东西,测量温度的器具,我已经设计出来的。可是没玻璃,就只能是纸上谈兵。”韩冈淡淡的提了一句,眼神深沉起来,“好了,不用多想了,眼下我也只是个想法而已,具体怎么测算还得慢慢考量。说一说吧,来太原见我,到底是为了什么?仅仅是玻璃上的事,我想用信应该就够了。”

就是之前商议利用棉花将甘凉路的汉番诸部拉拢过来,纳入棉行的势力范围,韩冈和冯从义直接也只是写信而已——不过是用了密文,以防被人偷看——只凭白玻璃,用不着冯从义亲自来太原。

“三哥说得是,要是仅仅是玻璃这件事,的确写封信也就够了。其实还有另外一桩要事,必须要让三哥知道。”

“什么事?”

“不知三哥还记不记得吴逵?”冯从义凑近了,将声音压低下来。

“当然记得。”当年让罗兀城功亏一篑的罪魁祸首,但被他带累的广锐军又是韩家在巩州的根基,韩冈怎么可能会不记得,“怎么,听到他消息了?”

“有人在沙州看到他了,身边带着十几人。”冯从义脸上添了几许阴翳。

广锐军出身的子弟,是顺丰行中的主力。而广锐军在巩州、熙州开辟的一座座农庄,里面出产的棉花,也是顺丰行收购的主要对象。每年还没有开始播种,便以契约定下当年的收成,并事先给付定金。这种旱涝保收的策略,是由韩冈当年亲自定下,让广锐军上下对韩冈死心塌地。

由于广锐军这些年来安分守己,加之在拓边河湟时的奋勇,如今朝廷和地方上的州县对他们已经不是当成叛贼看待。可吴逵一旦出现在河西的消息传开来,朝廷肯定就要紧张起来,对广锐军残部加紧提防。而任用许多广锐军子弟的顺丰行,避免不了的要受到影响。

“这件事确定吗?”韩冈问道。

“看到吴逵的是顺丰行派在甘凉路的掌事之一,也是广锐军出身。据他说,他看到的人虽然跟当年形象大不一样,但一眼看过去,就是吴逵没错。”

“就这样?”韩冈眉头皱起来,就凭这点证据,完全证明不了什么,“连长相都变了,怎么能那么确定”

“其实据他说,只有五六分相像。小弟也不可能就这么一惊一乍。”冯从义沉声道,“但他在甘州留下的姓名,可是叫做武贵。”

“武贵?”韩冈的眉头微皱,倒还真是很相近的两个名字。

“而且这一回西夏归附的汉将,领头的叫李清。他手下有个第一得力的部将,也是姓武名贵。据说此人乃是熙宁四年五年的时候投奔西夏的,只用了几年就在李清帐下出人头地,能力、手段都十分了得。”

“这个武贵现在怎么样了?”

“他在盐州城下的那场大乱中,不见了踪影。说是死了,但也有人说,他是带着一众兄弟去投奔了辽人。反正在那场大乱中,跟他交情好的兄弟,全都失踪了。而甘州城的武贵,他身边也有十几个伴当。”冯从义长出了一口气,低声问道:“三哥以为吴逵到底死了没有?”

“当年就没有确认他的死信。跟侬智高一样,都是被烧得面目全非。狄武襄当年没把侬智高当做战果报上去,韩子华【韩绛】也没敢报。军中也有传言说他去了西夏。只是后来一直没有消息,才没了那些谣言。”

当年吴逵的死信由于无法确认尸体的身份,并没有报上去,但基本上都认为他死了。可若是这一次,才安顿下来没几年的广锐军多半又要乱了。

从姓名、时间、行动这些地方来看,武贵的嫌疑实在太深了——甚至不能叫做嫌疑,完全可以确认,武贵就是吴逵。

“他现在还在甘州吗?”韩冈追问道。

“已经不在了。”冯从义道:“说是有人看到他一伙十几人向西出了玉门关,去了西域。不过是不是故布疑阵,那还真是说不准。”

韩冈沉吟了一下,抬眼道:“……不要想太多。吴逵此人,我与他有过一段往来。他的性子,多少了解一点。既然在西夏国灭之后去了西域,多半是没有再回来的打算。等到日后朝廷收复西域,说不定才会再听到他的名字。”

“三哥既然这么说,那我就放心了。”冯从义点点头,“那名掌事我就让他留在甘州主持分号事务,也叮嘱过了。一年半载,也不用担心会传出来。等过了一年半载,就是传出来,也不用担心了。”

兄弟俩说了一番话,也到了掌灯的时候。王旖遣了家丁来传话,酒饭已经准备好了,催两人吃饭。

韩冈和冯从义都饿了,出了书房起身往前面去。

冯从义边走边问,“不知三哥在河东还能留多久?”

“最多半年吧。”韩冈道,“河东这里,天子不可能让我留得太久。之前我也写信给你了,河东可以放一放,摊子不要铺得太大。”

“小弟明白。”冯从义低了低头,又问,“之后不知天子会安排三哥回京,还是会外任其他州府?”

“多半是回京。天子怕我功髙难赏啊。”在心思通透的表弟面前,韩冈一点也不遮掩,“要是去了其他路州,再立下些功劳又该怎么办?我眼下都是开国郡公了,还能向上封国公不成?”

“国公放在三哥身上不是迟早的事?”冯从义笑道:“做了相公,国公自然就能有了。”

“任官宰相,数载之后,便能封国公。但食邑过万户,也就能封国公了。天子不就是怕我再立功勋。若是爵同宰相,到了大殿上,站在哪里才合适?”

冯从义陪着韩冈一同叹了一声,转而又道:“说起国公,小弟前些日子在京城,听过介甫相公要转封了,说是因为灭夏之功,本因于变法之利,而且谋图西夏,也是介甫相公在任时先行主持的。”

“哦,是吗?”韩冈想了想,“之前是舒国公,这一次,不知道是什么了。”

封国也是分等级的。首封封小国,继而中国,而后大国,之后还有两国国公。外姓生前不封王,到了两国国公,便到顶了。如秦、楚、魏这样的国公名衔,不是老资历到几任宰辅,基本上不可能有机会得到。韩琦是魏国公,富弼是韩国公,都是大国。而王安石的舒国公则是小国。

“到底是什么,得要看太常礼院了。不过介甫相公也在第二任宰相任上才封国公。之前从相位上退下来,也只是开国郡公而已,好像就是太原郡。三十不到便因功封郡公,大宋开国以来,还真没人能比得上三哥。”

“比得上也好,比不上也好,都没什么好计较的。记得我过去曾经说过龟兔赛跑的故事吧,跑得快的不一定一直都在前面。”韩冈摇摇头,“接下来的几年,可能要清闲一下了。”

“正好可以用来治学。”冯从义道,“三哥的学问越高,小弟也能一并沾些文气。说起来也是关西魁星不利,好不容易才出了横渠先生和三哥。若三哥不能将关西的士子都收归门下,日后做了宰相也坐不安稳。就跟我们这些雍秦的商人一般,若不能抱成团,便只有被人踩的份。可一旦并力相向,就是京城,也能站下一块落脚地!”

韩冈笑而不言,但冯从义的话正是说到了他的心里。官位的高低,他可以不计较。但有些事,可是要好好争一争了。

