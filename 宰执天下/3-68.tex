\section{第25章 闲来居乡里(一)}

北京大名府。

六月盛夏,热浪滚滚。

炽烈的阳光没有半分遮挡,直直的落到了大地上。

汗水滴到晒得滚烫的路面上,转眼就会消失不见。空气在阳光下晃动着,带着远处的景物都模糊了起来。

大名府城外的东湖上,尚有着一点微风。碧绿的荷叶铺满了半幅湖面,朵朵白莲亭亭玉立。只是看着,便觉得清凉起来。

可偌大的东湖中心,就只有一艘画舫在莲叶间缓缓行驶。丝竹之声若有若无,在湖面上流淌。而在湖岸边,还有一众军士守卫。纵使汗流浃背,也不敢离开湖堤半步。看到这份阵势,路上本就不多的行人,都是远远的避让开去。

撑着画舫的艄公,戴着斗笠,有一下没一下的慢慢推着竹竿,让沉重的画舫一点点的移动着。

两名十一二岁的小使女,蹲在船舷边,探着细细的手腕,将画舫经过处的一个个莲蓬摘下来。用几个小篮子盛了,捧着进了船舱中。

船舱之内,有着丝竹歌舞。

一队乐班坐在角落处,前面是一幅帘幕,挡着他们望向舱中的视线。而在船舱中心,六名色艺俱佳的妓女,随着乐曲且歌且舞。艳丽动人的舞姿,让坐在四周的宾客们看得目眩神迷。

这是司空、河东节度使、判大名府——穷贵极富的文彦博在宴客。

自从离开了枢密院出外之后,不论是在河阳府,还是在大名府,文彦博所做的就是饮宴,游历,累了,就在府中读书、休息,政事那是丝毫不理。

河北东路的转运判官汪辅之前些日子刚刚巡视过大名府,对此颇有微词——转运司有监察地方州县官治政的任务在——但文彦博却是一点也不在意。

小儿辈的牢骚琐语,他做了几十年宰执的元老重臣岂会放在心上?!更别提他身上还有一个司空兼节度使的头衔,是为使相,论品阶,王安石都要在他之下。

这一日,他看着东湖上荷花开得正好,便邀了一帮宾客来,都是大名府的名士。船舱中,十几桶冰块放在角落和隐蔽处,暑气全被挡在了画舫之外。这样的享受,也只有几十年宰执的文彦博才能用得起。

保养的极好的右手捋着雪白的长须,半眯起的眼睛藏着深如渊海的心机。看着是歌舞,心中却没人知道在想些什么。

进来后的小使女将一个个装着莲蓬的篮子放到文彦博和众宾客的几上。文彦博身后的两名侍女,一个打着扇,一个则拿起莲蓬,帮着剥了起来。

轻微的一声碰撞声,让画舫轻颤。就听着一串脚步声,从舱外的船舷过道上响起,文彦博六子文及甫,出现在舱门外。

宾客们纷纷起身,向着文家的六衙内行礼问好。

文彦博慢慢的抬起眼,问道:“六哥,你怎么来了?”

文及甫刚刚乘着小舟,从艳阳下来到清凉的船舱中,还是一副汗流浃背的模样。他走进来,与众人打过招呼,在文彦博身边低声道:“大人,汪辅之那厮竟然上书朝廷弹劾大人!”

文及甫怒形于色。富弼当初被李中师所逼,竟然要交免役钱。现在又有人弹劾到自家父亲头上。元老重臣的脸面朝廷都不在乎,竟然让这一干小人欺上门来。

但文彦博不为所动,依然是慢悠悠的问着:“他说了什么?”

文及甫更凑近了一点,贴着文彦博的耳朵要说话。

文彦博瞪了儿子一眼,眼神中的厉色瞪得文及甫向后一仰。探手端起用井水镇过的酒杯,“即是监司弹劾老夫,此等公事,有何不可对人言?”

看着愣住的儿子,文彦博也不免与富弼一般,有着虎父犬子之叹。宾客们十几对在看着,再私下里说话,到外面可就要传出流言了。不过是个转运判官弹劾而已,有什么好好在意的。这消息很快就会传出来,现在弄得神神秘秘、紧紧张张,反而会让人以为他文彦博怕了。

干咳两声,当着宾客们的面,文及甫不便将自己了解到的汪辅之弹章上的内容都说出来,便简简单单的归纳成三个字,“汪辅之说大人‘不事事’。”

“就这个?”文彦博反问一句,毫不挂怀的样子,让竖起耳朵的宾客们都没了探究根底的兴致。

“此必是得当朝之人的授意!”文及甫背对着外人,恶狠狠地说着。

“要是王安石有这么蠢就好了。”文彦博自言自语道声音低得只有儿子能听到,“河北东路的转运判官是该换一个人了。”

“大人!……”

“此事让天子来决断,做臣子的何须操心?”文彦博提声长笑,“雷霆雨露皆是君恩,老夫一生栉风沐雨,到也不在乎多沾上一点。”

文彦博说的狂傲,但有谁能反驳,三朝宰辅,元老重臣,本来就有倚老卖老的资格。

说了一句后,文彦博眼一低,见着文及甫的腰上别着一个透亮的圆形琉璃坠饰,是他没有见过的。

“这是什么?”

“水晶阳燧,又叫放大镜。”文及甫忙摘下来,放到文彦博眼前,“不仅可以用来聚光引火,而且透过此镜,能放大对面的东西。听说是韩冈画了草图,而后天子让将作监的名匠打磨而出,奉与二圣。就跟此前传说能分光为七彩的三棱镜一样,才一个月功夫就从宫中传出来了。儿子也是看着大人读书不方便,所以从京中托人带了一个过来。”

“又那个灌园小儿弄出来的东西?”儿子当面表示孝心,文彦博并不理会,但听到韩冈的名字,便皱起眉头。

因为过去种种,文彦博对韩冈成见极深。前日韩冈在琼林宴上,凌逼杨绘,以下犯上,文彦博听了这件事后,便没有半句好话,什么天理自然,哪有朝廷纲纪重要?!后来听说韩冈荐了张载和二程入京进经义局,他才没有再说什么,心中也想看着王安石和韩冈翁婿二人打擂台的笑话。

只是看到韩冈弄出来的东西,天生就是一股子厌恶,扬手示意儿子将其拿回去,“阳燧不是铜镜吗?怎么是透明水晶……以奇技淫巧媚于天子,王安石越来越下作了。”

韩冈发明的放大镜,文及甫虽然不知怎么歪到了王安石头上,但不敢回嘴。讪讪的收了起来,附和的问道:“大人是否要上书天子弹劾?”

“且观其自败即可。”文彦博冷然说着,但一转眼就看到文及甫闻言愣住,问话中带上了一点怒意:“怎么?!觉得为父说得不对?”

“呃……不!没有。”文及甫忙着低头,哪敢说自己是因为惊讶而发楞。

过去在朝中的时候,他的父亲可是看到不顺眼的事情就立刻上书的。文彦博眼下的转变,让文及甫惊讶不已。但他也不敢多问,文彦博在家中亦如严君,丝毫不加以颜色,文家诸子一向是畏其如虎。向着舱中的客人拱手告辞,然后匆匆告退而出,坐着小船,又往岸上去了。

方才父子间的一番交谈,舱中众客仿佛充耳不闻,都是盯着美人歌舞,一点也不分心的模样。

文彦博看着他们,哼了一声。转头透过竹帘,望着亮得发白、闪着阳光的湖面,冷声自语:“且待其自败!”

……………………

七月流火,而陇西的六月,就跟放在火上烤一般。

路上的行人也少了,城外的榷场也冷清了不少。连巡视城中的甲骑,也都是将巡班改变时间,以避开了白天的高热。

韩冈自京师回到家中已经有一个月出头了,陆续来拜见他的宾客,也终于少了起来。

穿着一身宽松的袍服,躺在树荫下的摇椅上,悠然自得的看着近日的堂报。云娘旁边为他轻轻打着扇子,。十六岁的她越发的娇艳动人,举止乖巧。

王旖从外面进来,看到她,云娘连忙站起。

“云娘妹妹你做你的。”王旖让云娘坐下,到了韩冈身边,“官人,姑姑说明天冯家叔叔就要到了,要准备着为他接风洗尘。要问问官人,有什么要安排的。”

舅姑,就是公婆,从古到今都是这般称谓。但王旖喊着舅、姑,韩冈一开始听着也有些觉得怪异,现在渐渐才习惯。倒不似云娘,直接就喊爹娘。

“家里的事,你和娘商量就好了,这些事,你们看着办。”

男主外,女主内。主母的作用,本就是主持中馈,让丈夫可以安心处理外事。王旖乖巧有礼,对舅姑孝顺,每日晨昏定省,从不缺礼数。对于韩冈的三名妾室,她也是尽量亲近,并不争夜,一点也没有宰相家女儿的傲气。韩阿李对这个儿媳妇欢喜的不得了,人前人后没有少夸过她。现在家里有什么事,都要跟王旖商量着。

“那奎官和金娘快十个月了,周岁转眼就到,也要准备一下了。”

从礼法上,韩冈妾室所生的孩子,也都是她的儿女。王旖也是善抚如子女,每日悉心探视,让提心吊胆的周南和素心都安心下来。

按照如今的风俗,小孩子不能起太贵气的名字,以防夭折。韩冈的小名自己都不想提。一对儿女的小名,还是韩阿李起的——奎官、金娘,韩冈听着觉得不算坏。

“你们商量着来吧,问问南娘和素心的意见。”韩冈很是放心。

