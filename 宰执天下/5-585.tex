\section{第46章 八方按剑隐风雷(五)}

相对于南营的轻松取胜,西面的情况却极为危急。

既然理应了解南营是最精锐的汉军驻守,黑汗军的统帅也不可能以南面为突破口。

攻击南面营垒的人数虽众,其目标依然是牵制。真正的攻击重点,还是在西面。

王舜臣在黑汗人开始进攻时就看出了这一点,但他却没有分兵去助守西营,而是选择了用最快的速度击败对手。

否则两面平摊的结果,就是黑汗人以人数取胜,营垒被攻占。

就在王舜臣展示他让人叹为观止的连珠箭术的时候,挡箭车已经越过了西营外侧的壕沟。

黑汗人改造出来的雪橇车,在雪地上奔走的速度,出乎于守军的意料之外。来不及做出恰当的反应。而利用雪橇车越过寨墙,更是李全忠等将领所始料未及。

排在寨墙上的射手还没来得及换上刀枪,就被当先跳上来的几名黑汗战士,挥刀砍杀了一片。

惨叫声接连响起,箭手们的战线乱了。原本还畏惧弓箭压制的许多黑汗士兵,这时候终于找到了空隙,从挡箭车后闪出来,趁机冲上了寨墙。

防线上一片乱象,正当那些黑汗人要扩大优势的时候,城墙上一片羽箭飞射而来,顿时便射倒了冲在最前面的几个黑汗战士。黑汗军的攻势为之一顿,还在寨墙上的吐蕃与回鹘士兵,终于可以脱身下来,重新整顿阵脚。

末蛮城中,王舜臣放了一个指挥。即是守城的兵马,也是随时可以调动支援的预备队。

此时西寨告急,预备队立刻就调出了三个都转向西侧。就在城头上,张开神臂弓,居高临下,对准刚刚冒出头来的几名黑汗士兵一阵攒射。

除了主力据守的南营,其他几面的营垒,都是宽而浅,正是为了保证城头上的守军能够及时的给予支援射击。

不过黑汗军的攻势仅仅是一顿,随即便又汹涌澎湃起来。

当第一名黑汗战士跳下寨墙、冲入人群,疯狂的追砍着胆战心惊的回鹘人,紧随其后的黑汗士兵士气大振,纷纷跳上墙头,纵使箭矢急如雨落,可还是守不住逐渐崩溃的战线。

寨中的守军,开始纷纷退往城中。背后有退路的时候,没有几个人会选择死战到底。

只有营地中作为中坚,为数两百的汉军,他们依然坚守在寨门处。

他们聚集在一起,用斩马刀和神臂弓击杀每一个想要靠近寨门的敌人。只要寨门不失,能进来的就只有步卒。这样还有挽回的余地。一旦黑汗人的具装甲骑都冲进来了,西面营垒的败势将再难扭转,而城池也可能随即被攻破。

强兵的标准是什么?是敢拼敢杀,是胆气,不是光靠手中的武器。两百汉军战士牢牢守住寨门,如同中流砥柱,让混乱的营地有着一小片秩序。

汉军还在坚持,等待着援军的到来,但回鹘人的胆量已经涓滴不剩。

几天来,看着汉军摧枯拉朽,用强弓硬弩将黑汗军射得人仰马翻,姆百热克心中一片豪情壮志,每天出帐时都用力攥着刀柄,想要为自己死在黑汗人手中的亲友报仇雪恨,提着头颅回去炫耀。

可现在敌人近在眼前,只要一挥刀就能砍中,可他的手却抖起来,双脚也不听使唤。提着刀,站在雪墙上,看着这名跳上来的黑汗人一刀一个砍杀了两名同伴,然后转过身来,又面对着自己。

同伴的血喷溅到了脸上,敌人面目狰狞的脸转了过来,姆百热克的神经终于崩断。一声惨叫,丢下弯刀,抱着头,从雪墙上滚了下去。

冲上来的黑汗甲士没有上去补上一刀,甚至没有多瞥他一眼,跳下雪墙,就直奔敞开的城门而去。

……………………

‘赢了。’

千里镜几乎卡进了喀什葛里的眼眶里。但他毫无所觉,胜利的果实已经抓到了手中,只要再用些力气,便能牢牢握在掌心里。

喀什葛里并没有带着他的将旗。而是选择潜身在西营方向上的攻城大军之中。

在他的指挥下,黑汗军的表现远比攻击南营的同袍更为出色。

他麾下最为勇猛的战士们,推着挡箭车,攻到寨墙后,半点阻碍也没有冲上了寨墙,转瞬之间就杀进了敌营之中。

现在越来越多的战士已经攻入敌营,只要他们再能打开寨门、攻下城门,这一仗就赢定了。即便攻不下城门,只要能杀散马秦人之外异族仆从军,这一战的胜利也就在他的手中了。

少了大半回鹘人和吐蕃人组成的仆从军,区区三四千马秦人,那就完全不需要畏惧了。那点人数,就是守城也做不到,更不用说继续作战下去的士气。

这几天下来,从马秦人统帅的布置上,就可以看得出他的仆从军的战斗力是多麽的低下。马秦将军甚至不敢将他们带上战场,只让他们守在营地中。但如此无用的几千军队,却是支撑马秦军士气的关键。

现在双方兵力之比,是三比一。在防守一方而言,这样的比例还有着获胜的可能。尤其是寒冬已至,在野地里扎营的危险越来越高的现在。马秦军上下都会由坚守到底的信心。

可是若他们失去了异族的仆从军,那双方兵力的对比就是十比一了。这么悬殊的比例,马秦人的战士如何还能维持获胜的信心?而自己以十倍的兵力围城,区区末蛮城,还可能攻不下来吗?

他们的箭矢已经不多了!马秦人甚至到了不得不暴露这个秘密,而到战场中央寻找还能收回的箭矢的地步。但那又有什么用?除非是先知尔撒,否则一块面包养不活几千张嘴。找回来的箭矢,也许一天,也许两天,或许把半天就会用光了,而自己绝不会给他们再出城寻找的机会。

面对攻到城下的敌人,守军没有了箭矢,守住城池的可能姓可以说是微乎其微。

胜利也就在眼前!

…………………………

城门就在眼前。

敞开的大门仿佛在邀请他走进去。

阿迪望着无数胆怯的回鹘兔子涌进去的城门口,拔脚狂奔过去。

被选为先锋的他,现在表现得勇猛无比。

双手都拿着弯刀,将挡在面前的敌人一刀刀的劈开。

虽然出兵时,尤素普曾说过,只要拿下营地,那些投石车就能被利用起来反过来射击城上。

但眼下马秦人愚蠢的敞开了城门,这样的机会,难道要放弃?一旦在混乱中夺取城门,就能够获取最后的胜利。

在阿迪的眼中,这份荣誉真主已经送到他的手中,怎么能不要?

随着阿迪向城门的冲刺,跟随在他身后的黑汗士兵越来越多,很快就聚集了一百多人。

放弃了追杀寨中到处逃窜回鹘人,直接冲着城门杀过去。

仿佛热刀破开牛油,砍杀着挡在前路上的敌人,他们顺利的前进着,攻到了门下,穿过了并不深邃的门洞,终于攻进了末蛮城中。

控制了门洞,这道城墙就失去了作用。

可还没等阿迪为胜利兴奋起来,眼前的一切让他的脚步停下了。

末蛮城门后,并不是宽敞笔直的大路,而是一圈一人高的矮墙。逃进城中的回鹘人都被挡在了矮墙圈起的城门口。在矮墙的另一侧,是一名名拿着弓弩的马秦士兵。

黑汗人所不知道的,这样的矮墙只有南门没有,而在东西北三处城门内,都有用拆下房屋的砖石瓦片和木料修起来的围墙。只留下在道路侧面的一个出口。

“所有人都趴下!”

“全都趴下!”

城头上,矮墙外,拿着十字弓的马秦人一个劲的在喊着什么,几百张口合作一声,阿迪却完全听不懂。

只看见他们毫不客气的用箭矢射击着慌乱的回鹘人,而回鹘人则一个两个趴在了地上。

转眼之间,涌进城中的所有人中,就是有阿迪他们这些黑汗人还站着了。

天地间陡然静了,看着前方趴在地上一动不动的人群,还有围墙后的十字弓,冷汗从阿迪浑身上下流了出来。

没等他下一步动作,蓄势已久的齐射随着铮铮弦鸣,阿迪的眼前就被箭矢所充满。

……………………

从一开始,王舜臣就没有相信过汉军以外其他番军的战斗力。甘凉路上的吐蕃人的德姓,他攻凉州、攻甘州的时候见识多了。至于回鹘人,他王舜臣满手血债,别说用他们上阵,万一战局不妙,回鹘人少不了会逃跑,甚至返身一刀。他只相信自己带出来的人。汉军是最为可靠的力量。

所以王舜臣毫不讳言的向异族部将申明了这一点,并且在东、西、北三门后,加筑如同瓮城的矮墙——既然不信任,就要做好营垒被攻破的准备。

如果战事不利,可以逃城来,但进城后必须听从命令,否则迎接他们会是强弓硬弩和能在七十步外射穿铁甲的利箭。

每一名异族士兵都知道这一点。只是在这些天来的胜利,让他们都忘掉了这件事。不过当今天他们败阵之后撤回城中,不论是吐蕃人还是回鹘人,在汉军用弓弩提醒之后,纷纷听命趴下,纵然号令都是汉语,但总有人听得明白,而其他人也都能回想起之前的吩咐。只有攻入城中的黑汗人还没有反应过来,站立不动。

这是最好的目标,几次呼吸之内,便都成了满地的刺猬。

再无人敢于异动,然后弩弓手就开始点杀还藏在人群中的黑汗人。谁敢站起来就毫不客气的一箭射过去,数百张神臂弓盯着,任何一点异动都会引来二三十张硬弩的齐射,连同周围的倒霉蛋,一起做了箭下鬼。

而跟在第一批黑汗人之后,冲进城门的黑汗军源源不绝,但汉军的齐射更是连绵不断。敞开的城门向怪兽一样吞吃掉了所有冲进来的同袍,还在城外营中的黑汗战士终于觉得不对劲了。

他们聚集在城门前,用盾牌挡着头顶的箭矢,却不敢向内走上一步。

下一刻,一支长箭从门内飞了出来,正中一名身材最为雄壮的勇士的面门。长箭一支接着一支,瞬息间七八人被射倒在地,无不命中要害。

随着箭雨,从黝黑的门洞中,传来一声冷喝:

“给我杀!”
