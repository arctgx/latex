\section{第39章 欲雨还晴咨明辅(40)}

金悌这些天来第一次露出了笑容。

对着慷慨仗义的大宋君臣再拜起身,然后躬身倒退着与他的副手一起出了殿门。

面对可能是九死一生的任务,金悌毅然接受了下来,甚至请求今天就出发,赶回高丽,与还在抵抗辽人的势力联络上。

同时他向朝廷推荐了江华岛。

韩冈声称官军远来,必须要有个安全的落脚地点。金悌不疑有他,又或许是想到了却巴不得大宋能插足进来,所以推荐了这个据他本人说,是独一无二的绝佳驻地。

江华岛距离开京并不远,位于阿利水【汉江】的河口。地理位置绝佳。岛上还有山峰,具有险要地势,更有供停泊船只的港口。往来商船进入阿利水,也都要进过江华岛。看遍高丽西岸,再无一处能比得上江华岛。

听金悌这么一说,殿上君臣都觉得的确是个好位置。只要能够切实封锁邻接陆地的海峡,江华岛将是大宋控扼整个高丽的战略要点。而且据金悌及柳洪所言,江华岛至少有一县之地,面积不小,足以驻屯大军。而且既然这个岛位于河口,那么肯定也有冲积出来的土地,多多少少也能种上一点粮食。

只是这么好的地方,除非通知高丽的历代国王都是瞎子,否则就不会放过。在垂拱殿上的,除了坐着的,哪个不是明白人?再追问两句,金悌也老实的说了出来,江华岛本就是开京对西方来敌的第一道防线,早有城寨修在岛上面,也驻有一部兵马,甚至还是行宫所在。

说到这里,宰辅们都明白了,朝廷的兵舰开过去,恐怕会撞上一群逃出开城的官吏,也许还有几个王族、宗室,甚至可能会碰见国王王徽和世子王勋。

高丽就这么大,往南走也逃不了多远,只能寄望于海水阻隔辽军的步伐。而且从开京逃到阿利水上坐船,直放江华岛,比起渡水南下,更快也更安全。

将心比心,蔡确、章惇、韩冈都觉得,如果换做自己开京,往江华岛跑的可能姓很高。实在不行,往大宋逃,也能混个安乐王公做一做。

与败逃出京的高丽小朝廷混在一起,大大小小都是一个麻烦。不过派出战船的目的,只是要控制高丽海疆,不是跟高丽人争夺高丽的控制权。有些交换,不愁高丽人不做。只要安排得好,鸠占鹊巢也不是什么难事。

只是对于金悌的提议,宰辅们也不会全然相信,还是决定让领军出征的将领,到了高丽看了江华岛的实际情况再说,如果不合适,再换地方也不迟。

金悌的言辞总是不够坦诚,不过对他的平行,宰辅们还是给了很高的评价。

“当真是忠义之士。”从殿中出来,韩绛就忍不住感叹。

张璪也附和着:“此人大有古风,仿佛春秋之臣。”

“高丽事事皆学于中国,自号小中华,只看这一个知晓忠义的臣子,倒也不能算是自吹自擂。”曾布同样赞许。

只有韩冈不以为然,道:“春秋之义,在于尊王攘夷。齐桓之德,乃是全华夏,御蛮夷。高丽虽自号中华,只不服王化这一条,就照旧是蛮夷啊。”

章惇笑道:“玉昆你待四夷,何其苛刻。”

“一曰不降中国,就一曰是蛮夷。蛮夷猾夏,这一点终归是要防着。”

“算了,不争此事了。”章惇摇摇头,韩冈在华夷之辨上的原则谁都动摇不了,他也懒怠去争,“这段时间事情虽说少了点,还不至于闲得发慌。”

“金悌将这江华岛说得那么好,不知道高丽愿不愿意割让。”

“大宋要的岛屿,于高丽也是海外。无用之地。以无用之地,换来大宋的支援,这笔账,王徽、王勋应该会算。”

“就是不知道现在他们的情况了,到底是不是‘已经’逃到了江华岛上。”

“覆巢之下,岂有完卵?”韩冈摇摇头,“开京破城,究竟是自缚出降,还是里面的守军献城都要弄清楚。到底有没有给他们留下逃亡的时间,这同样要靠运气。”

蔡确眉头紧皱:“辽国当真能攻城了?”

韩冈道:“内部人心一乱,就没有攻不下的城池。”

蔡确叹道,“就怕辽国是在外部强攻得手。”

不管怎么说,开京都是一国都城,辽国若是能一鼓而破,面对大宋出四京以外的其他城市,多半也都能做到这一点。作为宰相……之一,蔡确也不愿看到辽国变得这么强大。

不过现在也只能看出征的将校们的本事了。到底能不能在辽国和高丽的干扰下,达成既定的目标。

选调水军将领的工作,枢密院此事已经完成了。

对将领的要求不是那么高,只要敢于出海就行。在大宋军中,这样的人,还是有那么一个、两个。

现在的这一个就是从广州那边调来,早在南征之役中就投到章惇门下,这两年为章家的商行出了很大力气。走惯了南方的水路,不知道能不能习惯北方的海域。至于能不能战,更是还得看情况。

“相公,今天有空吗?”

宰辅们分头各自归府的时候,韩冈唤住了蔡确。

“玉昆,可是有事?”蔡确回问道。

“韩冈想去军器监看一看铁骨船。顺便再看看一具新式的火炮。”

韩冈邀请蔡确去军器监一趟,主要是要看一下铁骨船。谁让他现在已经与军器监没有什么明面上的瓜葛,而蔡确现在的确是分管下面的军器监,其他重臣要去,少不了要向蔡确打个招呼。

“巴不得玉昆你去。只要能顺便指点一下。”蔡确很乐意,其实他本人都没有去参观过,现在有机会,就顺道去看一看。好歹知道那些钱究竟是砸到了哪里?蔡确可是一贯的唯恐天下不乱。

军器监在京城位置很正,从皇城过去,没有用多少时间。

两名宰辅很快就到了,只跟赶上来问安的军器监丞臧樟说了两句话,蔡确、韩冈便达到了他们的目的。

铁骨船仅比御苑中飘在池塘中的采莲小舟大一点。

钢铁做的骨架,外面钉了内外两层船板,前面安了一个同样是铁质的冲角,似乎想表现一下这艘小船究竟有多么结实,可以当成冲车来使用。

这艘船不是平底船,船底下凹,如果将底板钉上,下面能装不少东西“这船不愧是铁骨,大概是重心下移,整个人依然稳当得紧,走得稳,坐着也稳。”

“那岂不是铁船最稳?”蔡确问道。

“铁船其实也有,不过稳当就得另说了。”

臧樟陪着笑,一边还准备着。命下面的工匠给找了出来,顺着轨道,从仓库被一路推过来。

的确是铁船,从龙骨到船肋,还有船板,都是铁的。其中的缝隙用锡填补。但大小,只有一艘采莲舟的一半。江南的有些地方,女孩子家进场坐着澡盆去采莲,差不多就是这样的大小。

“为什么一开始不拿出来,还遮遮掩掩的。”蔡确有几分不快,质问着。

“相公有所不知。”臧樟苦笑着回答,一手还指着铁船,“这艘船最多只能飘在水上,载不了人。实在是太重了。”

“哦?玉昆,你当年是怎么说的?”蔡确又问韩冈。

“木头能浮水,铁却不能。要是做得小了,要么就浮不起来,要么就载不了人。必须要往大里造。”

后世还有水泥船。当真就是用水泥做出来的。骨架是铁丝网,外面是水泥,在一些小河中行驶。大一点的江河都不会去,完全没有抗撞击和抗破损的能力但这样的船只到底能不能经受得住风浪的考验,韩冈一点把握都没有。

木头拥有很高的韧姓,所以不用担心。现在最需要关心是的铸铁,其本身还是很脆弱,需要拥有更高的韧姓,才能当得起人们的赞许。

只不过,以现有的钢材品种,根本无法满足实际的需要。

而发展钢材,则需要定量与定姓的分析。不论是炼钢炼铁,还是制造铁器,都需要总结出完善的可以依循和学习的流程,而不是全然依靠经验感觉行事。

韩冈在这方面盯得很紧,需要下面的工匠们一起共同总结,相互帮助。纵然离开了军器监也是一样。

蔡确仔细的看了一阵铁船,对其中的疑问进行了一番询问,从工匠们那里得到了很多。

“真希望能够早点看见铁质的战船。想必在海上,没有任何船只能与其对抗。”

一想到能将敌国的船只当成鸡蛋壳一案,碾成粉碎,蔡确的心中就难以自已,这可是之前宰辅们都没有的运气,除了要分韩冈一大块之外,他蔡确本人就能控制剩下的六成七成。

不过制作出真正意义上的铁船,也不是那么容易的一件事。

蔡确已经做好了受到挫折的准备,没有什么实验能够始终高歌猛进,不过有韩冈参与其中,想来很快就会期盼已久的好结果。

“玉昆,你不是说还有火器吗?”蔡确想起了前面的话,“实物在哪里?”

