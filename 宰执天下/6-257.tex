\section{第33章 为日觅月议乾坤(15)}

“令表妹当真要做皇后了?”

韩钲脸色微微泛白,点了点头,“应该是吧。”

王寀没注意到韩钲的神色,望着书房之中,闹了许久,最终还是定下了王安石的孙女儿。

其实这也是在意料之内。

如今臣强主弱,加之太后势大,天子故而需要一门有实力有声望的外戚为奥援。

满朝文武,能从名望上压制住韩冈和章惇,让太后也投鼠忌器的,也只有王安石了。就算王氏女貌如无盐,狄氏女堪比西子,皇帝只要有心振作,也只会选择王安石的孙女。

但让王寀所不明白的,为什么韩冈会同意……至少是不反对他的内侄女成为皇后的候选者?

岂不知从此之后,将会束手束脚。

是因为王安石时日无多?还是如同市井中的另一段谣言,时日无多的其实是另一位?

或许,后者才是正确答案。

“天子大婚是在明年吧?六礼走遍,怕是要到年中了。”

“或许吧。”韩钲闷闷的说道。

王寀终于发觉了,惊异的看了他一眼,“明年,钟哥和苏家七姐,还有祥哥和金娘可都要成亲了,连着两门亲事,到时候有的忙了。还有雍国长公主,哈,天家也是两门亲事。”

韩家的长子韩钟,还有王寀的侄儿王祥,在横渠书院读书的两人,他们的婚事都是定在明年。而皇帝的姐姐,也确定明年春天出嫁,依照早已定下的婚约,嫁给韩琦幼子韩嘉彦。

但王寀想说的不是这几桩婚事,“等到这些婚事都结束,可就要轮到钲哥你了。”

自己心中的私密仿佛被眼前人给看透,韩钲一阵羞恼,不过当他转过脸来的时候,脸上已经看不出异样。

“还得一两年呢,倒是十三叔你,怕是会更早一点。”韩钲望了一下书房,“估计里面一时半会说不完,我们就别在这干等了,先进去吧,娘也在等着呢。”

王寀知情识趣的点了点头。

有些话,点上一句就够了,说得多了,反而伤了情分。

先去拜见,再等着

…………………………

当韩冈回到后院的时候,已经是华灯初上。

午后与王厚深谈半日,又留了酒饭,等送走了他,才有时间回到后院。

“十三走了?”韩冈边换下沾着酒味的衣服,边问道,“前面听说他跟二哥儿一起过来的。”

“早走了。”王旖没好气的坐在一边,等韩冈等得她心浮气躁,“就坐了半个时辰,见官人你跟二兄还在谈,便说有事回去了,留都留不住。”

韩冈笑了起来,“他这个年纪本来就坐不住,大哥,二哥不也都一样?”

王旖心头一片烦躁,不想跟韩冈东拉西扯下去,“官人,你跟二兄聊了半日,到底聊了些什么?”

“没聊什么,只是说了些实话。”韩冈坦诚的道,“天子并非良配,但岳父执意如此,为夫也不便阻止,只能跟仲元多说两句,免得他日后心中怨我。”

王旖紧紧咬着下唇。

她很清楚皇帝定下自家的侄女儿,到底是为了什么。那一顶凤冠到底有多沉重,时常入宫的王旖怎么可能会不清楚。

将孙女儿投入漩涡,将王家的未来寄托在皇帝身上,不仅仅如此,还会将王家都卷入进来,最后与太后跟自己夫君为难,不论胜败,她的处境是最难的。

“仲元说他当日犹豫许久,是岳父最终做出了决定。而选定越娘,则是皇帝自己做主,太后没有干涉,不过太妃在后面说了什么,就不得而知了。”

私心过重。

韩冈就是这么评价当今的天子。

当然,私心过重的不仅仅是赵煦,还有韩冈也是。

韩冈与太后业已议定的最终候选名单,虽然因为变故,皇帝提前将人选出,并没有对外公布,但里面除了慈圣的曾侄孙女——也就是曹国舅曹佾的曾孙女——可算是武家出身,其他五女皆是出自文臣之家。

至于之前为人称道的狄氏之女,则根本没有入选。韩冈能容许文臣的女儿做皇后,却绝不会允许一名武家之女母仪天下。

文臣家族出了皇后,等于是自斩根基,从此脱离士大夫的行列。而武家出了皇后,却能更加枝繁叶茂,成为一株足以荫庇天子的参天巨树。

尤其像狄青这样由卒伍而将帅,继而宰辅的名将,比之将门更得军心,几个儿子也算中上之才。

狄氏女若做了皇后,狄谘、狄詠虽不能再领军,但他们的兄弟子侄却方便得很。若是皇城城头上出现狄姓将佐守夜,这比吕惠卿跻身政事堂更让韩冈难以安心。这还没算上狄家姻亲和旧部。

赵煦若得狄氏,也就相当于得到了为数众多的将校,有了控制住禁卫和京营禁军的可能。

要不是有这方面的考虑,两府一众宰执,又何必刻意抓着狄氏女的身世做文章,先是阻止她做皇后,当朱太妃想要她做嫔妃的时候,又以狄青的身份为由,阻止她成为嫔妃。

王旖想不到其中有那么多曲折,那份最终名单,她也不知道,她只听明白了丈夫的话中之意。

银牙咬着下唇,她试探的问道,“皇帝是不是怕太后和官人日后会为难他?”

“当真要废他,当初就废了。要不是看在先帝的面子上,朝堂上哪个容得了他?宫中、朝中只盼着他能学好,没想到却是越大越不像样。”

韩冈愤然作色,可他说的话中,却是悄然跳过了自己的算计。有些事他不想对家人说谎,避而不谈倒是没什么心理障碍。这种做法,说虚伪,也的确虚伪,韩冈知道自己是自欺欺人,可面对家人的时候,却又难免要软弱一点。

王旖也只能叹息着,抚着丈夫的背,安抚下愤怒的丈夫。

如今这位皇帝的品性,世间已经有太多传说,世人也看到了太多例证。亲如太后,近如韩冈,都拿皇帝没办法,她一妇道人家,即使对侄女儿再担心,又能怎么办?只能往好处去想。

“越娘性格好,希望她嫁过去后,能好生规劝。”

“若能如此,那可就阿弥陀佛了。”

从来不信佛的韩冈破天荒的念了一句佛,王旖不禁扑哧一笑,心头上的云翳也给冲散了一些,絮絮的对韩冈道:“官家和越娘的婚事,终于是定了。官人可别忘了家里还有金娘的婚事,之后还有大哥、二哥的。”

“自家儿女事不操心,却操心别人家的儿女,我这个做父亲还真是不够格。”韩冈摇头自嘲的笑着,又忽的叹了起来,“一转眼的功夫,都要操心儿女终生大事了。再过两年,自己都能做祖父了。”

“过得的确是快,好像不久前,才跟官人初次见面。”被韩冈这么一说,王旖的心思也被带了起来,手抚上自己的眼角,叹息着:“转眼间就老了。”

“哪有?”韩冈探指抚着妻子的面颊,触感依然细腻,“还跟以前一样啊。”

王旖横了韩冈一眼,含羞带嗔的眼神中依然有着少女时的妩媚。

“不过金娘和大哥、二哥他们还是早些完婚的好。”王旖的眼神中有着浓浓的期待:“奴家一直都盼着早些抱孙子呢。”

这么早抱孙子,在韩冈前世所在的时代,也不算是什么稀奇事,可落到他头上,却是难免心中的异样感。

以二十年一代的速度,不停地开枝散叶,多传几代,人数可就让人瞠目结舌了。

“如果家业不倒,四十年后,为夫的后人怕是轻而易举的就能超过一百。这人口增长的速度,想想也的确惊人。”

“如今哪家不是如此?人口少的,反而不正常了。”

王旖摇摇头,也就她家,因为王安石没有纳妾,同时王雱又早亡的缘故,故而人丁不盛。但临川王氏一族,依然是个大家族。而寻常官宦人家,十家里面也肯定有七八家是人口兴盛的大家庭。

以如今的生活水平,十几二十年就翻上一番实在是很简单的一件事。

当天下人口增长到三亿四亿的水平线上,又不想大幅降低生民的生活水准,除了开拓,就别无良策。

一国如此,一家同样如此。

“此次选后,旧时名相王旦、晏殊,皆有女入选。不过韩、富、二吕这样宰辅门第,则没有仿效岳父,一个也未出。外面有人说,只看有没有将女儿送入宫中候选,就知道这一家是否破落了。”

韩冈跟妻子笑说着,当成一个玩笑。在大名单上的入选的文臣之女中,家世依然鼎盛的,也只有王越娘一人。

王旖却听出了其中隐含之意,“韩家或许也会如此。但人口多了,其中出一二人杰也更容易了。到时候,自然能保住家门,官人有何须担心?”

“家门保不保得住,就不是为夫能关心的了。儿孙自有儿孙福,日后家门如何,只能看他们自己是否用心。”

留下的遗产再多,没有一个好的继承人,还不是都便宜了别人?

眼下,在宫闱之中,不正是有着这一个最好的例子?

韩冈微微叹着,一时失去了说话的兴致。
