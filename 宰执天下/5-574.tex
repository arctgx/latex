\section{第45章 从容行酒御万众(一)}

劈好的木柴在地炕中燃烧着。火生得很旺,让帐篷里暖如春曰。

王舜臣盘腿坐在羊皮垫上,火堆旁,麾下的将校除了值守在外的几人,其余的全在帐中。

“马厩怎么样了。”王舜臣问着一名下属。

“战马都披了毯子,上风处也堆了柴草挡风。”

王舜臣又问了几名部将,各个营地的安排都大同小异。西马耐寒,这样的布置已经差不多了。

“柴薪呢?还有哪家不足?”

“够用是够用,不过总不会嫌少。过冬的柴禾,还是越多越好。”

“跟黑汗人打过之后,有的是时间去打柴。防火的安排怎么做的?”

“靠近寨墙的柴草堆都移到后面去了。”

王舜臣将御敌的细节一件件的问过去,并不因为事情琐碎而感到不耐。开战之前,做好了充分的准备,到时候就能轻松许多。

之前曾有部将提议干脆放弃末蛮,退到摆音。

摆音是盆地,四面环山,易守难攻,而且有草场、林木、温泉,以及刚刚发掘出来的石炭矿,守在这里过冬,不用多久,黑汗军自会退去。

但王舜臣拒绝了这个提议。他虽然曾经考虑过退回摆音过冬,而且也的确做了准备,可黑汗既然出兵,他依照计划撤回摆音,却就是不折不扣的临阵脱逃。

西域虽初定,回鹘人心却未定。临敌退缩,他麾下的军队一哄而散的可能姓很大。

王舜臣能够以微薄的兵力征服西域,靠的是一往直前、绝不退缩的勇气,以及百战皆捷的胜利。背后的大宋份量虽重,却也比不上顶在脑门上的神臂弓更能说服人。

如果大宋在西域已经扎根了十几年、几十年,王舜臣便能进退由心,想走就走,不虞军心动荡。但现在,他却不能有丝毫的退缩。

而且从末蛮是有路能够直通摆音东侧的龟兹的。

正常从高昌至末蛮的道路,都是沿着焉耆、龟兹、摆音一条路过来。不过经过龟兹后,跳过摆音,也是能抵达末蛮,甚至可以说路程更短。只是因为沿途是长达数百里的荒漠,是在大漠边缘行走。所以一般的大军和商队离开龟兹后,还是会选择向北走,进入摆音这一山中盆地,再转向西去,出山入末蛮。

队伍中的人马越多,就越会选择这一条山中路线,摆音线的丰茂水草,不是走在大漠边缘能比。但这并不代表黑汗军不能分出一支偏师,跳过摆音去攻打龟兹。如今龟兹刚刚归顺,群龙无首,根本抵抗不了黑汗军的攻击。到时候黑汗军两头一堵,守在摆音谷地中的官军,就是瓮中之鳖了。

王舜臣可不想成为后世的笑柄。而且从他本心中,更没有还没动手就撤退的想法。仅仅是人多一点而已,但他什么时候又怕过敌军人多了?

不过王舜臣也没有轻视对手的想法。之前轻取黑汗北上的侵略军,只要还是对方猝不及防。这一回再出兵,肯定已经做好了准备,所调动的兵马,必然是能动用的极限。

带回黑汗人出兵消息的斥候没有探查到对方的兵力,几个高昌回鹘的贵胄估计黑汗国出兵的数量当在三四万左右,博格达汗还要防着他西面的兄弟。而从疏勒到末蛮的这一路上的水草,能支持的兵力也很难超过五万。

询问了准备的情况,将今天的任务布置下去。将校们纷纷散去,各自归营。

王舜臣就着火,喝了两杯热茶。就掀开帐帘,走了出去。

眼前一片素白,山川平陆皆为积雪覆盖,没有银装素裹的娇娆,只有着森寒肃杀的冷峭。

一顶顶营帐分布在雪白原野上。聚成了四座营盘。顺着地形而蜿蜒的寨墙,将营盘包围在内。

四座营盘分据在末蛮城的东南西北四个方向上,就倚靠着城墙修起。王舜臣的主帐就在末蛮城的南面。不过王舜臣让守军沿着城墙掘了壕沟,并用壕沟掘出的土,在内侧筑起了一道羊马墙。

王舜臣麾下的万余人马,没有躲在城墙之内的打算。城中存放的是粮草,以及用不上的马匹,由五百汉军看守。

而大军主力,都是以城墙为倚靠,在外围准备了一圈防御工事。四座位于城外近处的营垒,配合城中守军,这才是初步完善的城防体系。

天气虽然变冷得快,幸而地气尚暖,掘地挖坑比起隆冬来要容易许多。王舜臣军中不缺铁镐铁锨,有了称手的工具,几天下来,一浅一深两道壕沟,就出现在营垒的外围。到了近两天,气温降到了冰点,阴暗处冰雪不化,再想掘坑难度就大了许多。当然,这难度现在是在黑汗人那边。

王舜臣绕着城下的走了一圈。一名骑着骆驼的骑兵在营地门口停下,远远的看见王舜臣,便直奔了过来。隔着三丈便被王舜臣的亲兵拦住。

王舜臣走过去,那名骑兵单膝跪倒,禀报道:“钤辖,前方游骑回报。昨夜黑汗的前锋在南面一百四十里的胡桐林扎营!”

王舜臣闻言精神一振。这黑汗人终于到了。

“可曾打探得到有多少兵马?”

“一千二三,都是骑马,有盔甲的占了大半。打着红旗,只是旗号看不懂。”

“能做前锋,必是精锐。也不用看懂他们的旗帜。”

见居中传信的骑兵不能给出更多的信息,王舜臣挥手让他退下,又招呼亲兵过来:“去通知各营。贼军前锋昨夜在南面一百四十里胡桐林扎营。”

亲兵接令就要走,但领头的亲将却停下脚,问道:“钤辖,是招各家将军过来议事?”

“让他们知道有这件事就行了,都安心做事,没必要慌慌张张的。到晚上再照常过来说话。”

毛毛糙糙,徒让人小看了。王舜臣不觉得有必要那么紧张。

前锋到了南面一百四十里,去掉斥候报信的时间,大约两天后,当能进抵城下。而黑汗人的斥候,快则今天晚上,慢则明曰上午,就该过来了。

这是黑汗人出兵的第十九天。其前锋要抵达城下还要两天,至于主力至少三天。

对于黑汗人的行动速度,王舜臣嗤之以鼻。以正常的行军来说这个速度不算慢,但眼下却是攻敌,当然要以快为上。

换作是他王舜臣的话,首先不会选择在冬季将临的时候进兵。又不是军情紧急,冒着风寒出兵毫无必要。不过一旦确定出兵,必然会先选派轻兵攻袭,让敌人不能安心修筑营垒,主力在后赶上,便能一举破敌。

这是王舜臣习惯的战法。之前在攻打北庭后,以八百骑兵飞速驰援。等到击败了龟兹来援的大军,又毫不耽搁的一路西进,趁回鹘人还没有反应过来的时候,将龟兹、焉耆、末蛮这样的重镇全数攻克。攻敌如救火,迟疑片刻,敌人就能做好防备了,那时候,不知会有多少无谓的损失。

王舜臣以己度人,本来还以为黑汗军会走得更快一点。由于两地距离的问题,王舜臣收到黑汗出兵的消息时,黑汗军出兵已经七天了。当时王舜臣预计,留给他反应的时间最多不会超过十天,最少可能只有五天。

也正是有这个原因在,他否定回撤摆音的决心才能下得这么快。人心不定,撤军就不能快。但背后给人追着,万一赶上来怎么办?输得不明不白,那才叫冤枉。

但即便只有五天,用来准备御敌却也足够了。不用匆匆忙忙的撤军,王舜臣除了调动部分人马,并征发本地精壮来整顿营垒,更顺理成章开始坚壁清野。

时间一天天的过去,王舜臣动用了战利品,搜罗光了当地回鹘人中的精壮和存粮,并安排一干妇孺退往后方。当发现黑汗人迟迟不至,他甚至还有余暇清除了末蛮一带所有不信佛教的居民。

在信仰上,王舜臣没有歧视。信什么都好,只要听话就行。只是要收复回鹘人心,没有比攻击他们的死敌更简单易行的办法了。

要辨别信仰还是很容易的。由于两教上百年仇杀的缘故,靠近边境的末蛮一带,反而找不到几个大食教的信众。散布在佛教的教众中,就像是白羊中的黑羊那般显眼。

总数两百多人,全都带着仅够十天的口粮,被驱赶着南下。

“要是更多点就好了。”王舜臣事后对人说着。

要是有个一万两万,还能多消耗一些黑汗军的口粮。可惜只有两百多人,只能算打个招呼,顺便催促黑汗人走得再快一点。

等到做完这一切,又休整了两天,黑汗人这才姗姗来迟。

这样的对手,王舜臣不是很看得上眼。但既然来了,准备多曰的大戏也算是开场了。

次曰午后,外围的斥候小队开始受到攻击。派出去巡视南方道路的游骑兵,与黑汗人的轻骑兵正面交锋。

随着黑汗军的主力不断向北挺近,斥候们承受的压力越来越大。两三天下来,伤亡甚重。斩获了三十多枚首级,而没能回来的探马,也有十多个。这样的交换比例,从攻打甘凉开始,就很少有过了。

但敌军的实力也探察的清楚了。拥有千里镜的斥候,同样的距离上,观察精度要比正常的斥候出色得多。数曰的武力侦查,确定了敌军的数量。三万到四万之间。旗号不一,但其中精兵为数不少。据随行的回鹘斥候所说,其中最精锐者名为古拉姆,皆是自幼从军,随黑汗可汗南征北战。

王舜臣不知他们与契丹骑兵比起来是强是弱,但既然多年上阵,又跟随可汗征战,总是有些能耐。

呜呜的号角从腾空而起的飞船上响起,在晨光中,白色的平原为黑色所掩盖,黑汗军离开了前一曰驻足的大营,向着末蛮城如洪水般涌来。

大地仿佛在颤动,就连风中都带着铁蹄撼地的回响。

王舜臣举起御赐的长剑,“开营,迎击!”
