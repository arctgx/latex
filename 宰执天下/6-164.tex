\section{第15章 经济四方属真宰(上)}

元佑五年三月戊申。

阴转小雨。

在日记本上写下两句,韩冈抬眼看了一下摆在窗边的温度计,玻璃管中水银柱的最上端指着标记着十的刻度下面一格。

九度。

开封府农历三月的夜晚,九度这个温度还算是在正常的范围之内,至少比去年三月的一天夜里气温骤然降到冰点之下要好。

温度计的概念,已经在《九域游记》中出现,不过更早一点,韩冈已经在《自然》中提出了温度的概念,并将水的冰点设定为零度,沸点为一百度。并且在之后的论文中,通过的托里拆利实验,确认了大气压的存在,顺便对水的沸点进行了补充,也就是用猜测的口吻,明确了气压对沸点的影响。

而将韩冈的猜测进行证明,在韩冈就任参知政事后,便成了人人争先恐后的一件事。最近的一次,便是是通过《自然》期刊的组织,让各地的会员和通讯会员,分别在庐山、雁荡山、五台山等山脉顶峰,以及三十九处出于平原上的州县进行了为期三个月的实验,参与的会员及通讯会员总共超过三百人,最终确认了高度及天气变化对气压的影响,同时也确认了气压和沸点的关系。

不过尽管相应的概念已经提出和验证,韩冈也很早就选调工匠试制温度计,但适合制成温度计的玻璃管制造起来有些难度,玻璃管内部空间很难保持平直均匀,至今为止,成功的希望依然只能放在运气之上。

但韩冈家里,还是有好几支温度计和气压计,每天的每个时辰都会有人将之记录下来,最后抄录两份,一份集结成册,另一份送到韩冈这里。

韩冈从桌上拿起记录纸条,从昨日申时开始,到今天申时,每个时辰的温度变化,都在纸条上。并不需要太多,只是几句话吩咐而已。

韩冈提起笔,一丝不苟的将最高温度和最低温度写进了日记。这等记录天气和气温的习惯,若能持续上几十年,而且能够一直留在京城,肯定会是研究开封天气变化的重要资料。韩冈不仅仅自己这么再做,他还在《自然》中选择了

而且不仅仅是温度计和气压计,还有雨量计,记录下每场雨水的数量。设立气象局,暂时还有不到时候,但通过《自然》进行气象学知识的普及,将之从天文学中分割出来,已经在进行之中了。

日期和天气之后,韩冈继续他的日常功课。

不独是他,不少士人都有记日记的习惯,今天遇到的几桩大事,大都简要的记录下来。曾布就记日记,而且是自幼便记,他被抄家的时候,抄出来的日记本装满了一辆车。

韩冈看过其中的几卷,上面的都是些文过饰非的东西,责任都是别人的,功劳都是自己的。尤其是当年曾布叛离新党的那一桩公案,全都王安石、吕惠卿、曾孝宽的错。韩冈也很荣幸的在其中成了倒坏水、给王安石出谋划策的角色。

韩冈在自己的日记本,也不会太客观,不过他日记本中的内容,记录科技和工业进步的比较多,政争几乎没有。没空的时候,只会是将天气、气温记录下来,然后干脆的跳过这一天,闲起来,则会当做练字,顺手写上一些记忆中的科学理论,当成自己猜测。

笔锋在纸面上挪动,忽忽数百字。放下笔时,正好听到外面的云板响了几声,要等的人差不多该到了。

可惜精确计时的钟表到现在为止,还没有被发明出来。不然,就不用云板声来传达家中刻漏所指示的时间。

尽管韩冈早已给出了摆钟的原理,但想要看到使用钟摆来确定时间流逝的座钟,还得需要某位工匠的灵光一闪。若是有了钟表,能够更精确地测量时间,不论是生产生活,还是军事行动,都能从现有的水准上再上一层楼,更是进入工业社会不可或缺的关键道具。

正想着该如何再激励一下工匠们,外面家人扬声通禀,“相公,王学士来了。”

判军器监、枢密院直学士王居卿终于到了。

就如王居卿数年间,虽说职位未迁,可贴职已经积功升到了直学士一级,韩冈现在也已是宰相——同中书门下平章事兼集贤院大学士。

半年前,韩绛上表乞骸骨,遂以中太一宫使告退。苏颂独相半年,然后二月初,几番辞让后,韩冈正式就任宰相。

诏书一出,朝中全无异论,本就是水到渠成一件事,而在他就任宰相之后,家中仆役便一夜之间全部改口——相公二字之贵重,让他们这些做仆婢的也觉得与有荣焉。

“请他进来吧。”

韩冈说道,将桌上的日记本收起来,来到外间,顺手从书架上抽了一本书出来,等着王居卿。

“居卿拜见相公。”

没过片刻,王居卿便被引了进来,韩冈在阶上相迎。

互行了礼,韩冈将王居卿引入厅中。

相让着坐下,王居卿一眼便看见了韩冈出迎时,随手放在小几上的那本书。

《九域游记》。

“相公也在看此书?”王居卿问道,试探着韩冈是否有什么深意。关于这本书和韩冈的关系,外界议论得沸沸扬扬,韩冈始终避而不谈,王居卿这还是第一次看见韩冈拿着这本书。

韩冈方才是随手一抽,也没注意是自己的书。不过听见王居卿提及,便拿起来扬了一扬,“寿明以为此书如何?”

“奇书也,所以能遍传天下。虽说的是子虚乌有之事,却有七八分成真的可能。”王居卿道,“只可惜不知作者何人。”

“既然佚名,大概也是不想让人多探听吧。”

韩冈写下这部书的时候,就是为了针对王安石以上追三代为名进行变法的借口。他想要的世界,不需要以三代为名——那样百姓不明白,而士人也不会信,朝臣们更是都知道是借口——直接用这部书来告知世人,里面种种,有韩冈已经说过了,也有未出现但可以印证的,在很大程度上,能够通过努力去实现。

是科幻,更是现实。

不过由此引发的热潮,是韩冈本人也始料未及。现如今,京城的瓦子里,除了说三分等评话之外,又多了一个说九域,而且也带动了士人写作的热情,现如今,甚至在报纸上都有了连载小说,大多是将评话进行改变,像韩冈这样现实主义作品很少,而是上天入地,无所不能。

“相公说得是,想来的确是这样!”

韩冈的话,听着就是承认了,只是警告不要说出来。王居卿当然不会违逆韩冈的心意,这本来就没有什么好探究的。尽管这部书,说起来其实可以算是对新党开战的号角,向天下士民

有的人看的是书中人物的悲欢离合,有的人看的是书中的衣食住行,有的人看的则是书中的地理人情,而王居卿,在书里面,找到了军器监的目标和方向。在他看来,这几年能稳守住军器监,得到了诸多赞誉,完全是依靠自己从《九域游记中》得到的灵感。

“相公觉得这部书写得如何?”

韩冈摇头,“诗词不甚佳。”

打了几年的交道,王居卿多多少少也能算是比较了解韩冈。韩冈的脾气,正常情况下,可以说很不错。只要不去挑衅他,正正常常的说话,韩冈也很乐意跟人聊天,甚至说个笑话。偶有冒犯,只要不是存了恶意,也能大度的容忍下来。

“其他的确不甚佳,不过一篇‘滚滚长江东逝水’,足以光耀全书了。不知相公如何看?”

韩冈微微苦笑了起来。他在书中插进这首词的时候,还有些开玩笑的心思,可现在他只希望后人不要将放入自己的文集中。

“我也是这么看的。”韩冈说道,“全篇诗文百余首,惟有这首临江仙最好。只可惜作者不得扬名。”

王居卿惊讶的看着韩冈。从韩冈的脸上,他能感觉到,韩冈是真切的感到遗憾,并非是在开玩笑。这让本已认定韩冈是此书作者的王居卿,变得不那么确定起来。

《九域游记》这部小说家言在世间流传很广,可得到的评价中,诗词和文笔都是居于末位的,除了一首《临江仙》之外,都被人批得一塌糊涂。而这一点,也让世人认定此书出自于文采不佳的韩冈手笔。

可是那首临江仙,也不是没人批评,但书中的其他诗词,都让人无法为其辩护,只有这首词,才能让人有足够的底气去驳斥他人的攻击。所以就有人以此为由,怀疑起这根本不是韩冈的著作。

韩冈如今春风得意,正是准备一展宏图的时候,怎么会有‘浪花淘尽英雄’和‘是非成败转头空’的感慨?更不可能写下‘古今多少事,都付笑谈中’这样的词句?

王安石当年初得志,手握变法大政,他当时的诗句便是驳斥反对者的‘丈夫出处非无意,猿鹤从来不自知’,甘愿鞠躬尽瘁的‘明时思解愠,愿斫五弦琴’,以及感恩天子的‘应知渭水车中老,自是君王着意深’。对比起王安石,书中的那首临江仙完全不附和韩冈的心境和际遇。

王居卿不知道是不是该猜测下去,不过不管是谁写的,那都不重要。重要的是这部话本中的内容,完完全全都体现了韩冈的心意。

“的确是可惜了。”王居卿道,“此书洋洋百万字,天文地理、人情世故无所不包,却是独树一帜,古之所无,又是诗文所不及。”

韩冈点了点头,笑道:“确是首开先河。”

文笔再差,诗文再烂,借用和剽窃的内容再多,也改变不了这本书的历史意义。这也许不是这个世界的第一部长篇小说,但肯定是第一部长篇科幻小说,这一点是不必妄自菲薄的。

“旧日有水井处皆有柳词,如今周美成也不遑多让,但一篇文,使罪不为罪,耆卿、美成远不及也。”

“解剖?”韩冈想了想,问道。

王居卿点头,“正是。”

刑律中毁损尸体本是重罪,即使是死于谋杀,进行检验,也会尽量不去毁损尸体,多是会从外表去看。可现在除了官方的尸体检验之外,发现的无名尸,多要交送医学院进行解剖研究,找出死因报备官府,同时也能顺便进行一下研究,不过之后必须火化埋葬。私下里,医学院隔一段时间还会延请僧道,做个水陆道场。

“外科医术,不去认清人体构成,如何去医人治病?这几年医学院中,不知扫除了多少古书中的谬论,不仅仅是外科,内科、妇科、小儿科的医术,都比过去进步了。”

“相公说得极是。只拿着书本,琢不出美玉。死读医书,成不了良医。”

“做事难,难就难在要本于实。得看实际,而不是看文字。”

王居卿起身,拱手恭声道:“相公放心,居卿明白。”

王居卿今日来见韩冈,不是别的原因,而是因为他要出外,去淮南东路就任转运使。

韩冈并不支持蔡确、吕惠卿那种在京中,从初入朝官一路升到宰辅的经历。一直以来他都比较赞赏那种几年京中,几年地方的任职方式,尤其是御史台,必须经过一任亲民官,才有资格进入。如果让不做事的清流掌握了话语权,是对所有认真做事的人的讽刺。

王居卿在军器监的位置上时间已经不短了,应该出外去涨一涨经验,再回来时,当能再上一层楼。

而王居卿就任淮南东路转运使最大的问题,就是淮南去年的旱灾,导致了这个春天分外的难熬。如何赈灾,是地方的工作,是常平使的工作,同样也是转运使的任务。

“寿明你打算如何做?”

“淮左粮秣不缺,唯一可虑者,唯有赈济一事。当官民协力,共度时艰。”

“赈济?”韩冈想了想,道,“赈济当然不容易。行善哪有难么简单的?行善积德,能泽被子孙数代。人人都想子孙福寿绵长,可惜有几人能做得好?我曾听家严说起过,昔年密州乡里曾有位善人,他家先祖起初只是位塾师,一年不过十来吊钱。后来乡中因他年高望重,就推他做了乡老。他老人家从此到处募捐,广行善事。那些念阿弥陀佛的,穷人家两个铜板都能给他化去一个,而他家连着尼姑庵里的钱都会募了来做善事。”

王居卿明白韩冈想说什么了,配合的回了一句:“神通广大。”

“那是!”韩冈道:“到了他家曾祖不在的时候,十几年积善行德,家里就已经积聚下几百贯钱。到他祖、父两代,正好是黄河接连泛滥,青、徐之地赤地千里。州县中知道他家肯做善事,就把他家推戴起来。”

王居卿呵的一声,低声道:“老鼠入米仓。”

“这就是善功,功德之多……”韩冈摇头啧啧两声,“等到他家老父去世,庄上的已经存了好几十万贯、数百顷地了!

“相公放心。”王居卿肃容说道,“居卿此去,必不使此等人得逞。”

“没有此辈善人,寿明你做得成事吗?”

“赈济离不开州县豪右。不过也不是让他们予取予求的。”

官吏要过手,富户要过手,朝廷发下的赈济,到了灾民手中,十不存一。怎么给灾民多留上几分,便能看出主事者的才干了。

“以寿明之才,淮左的灾伤我是不担心的。但我还是希望寿明你能做到公私称便、官民称道。”

王居卿要大用,就必须让他有更多机会展示自己的能力。王居卿一向以事功见长,所以韩冈就让他去淮南这个转运的中枢之地,好好的表现一下。做得好,直学士的直就可以去掉了。若是做得不好,那就只能继续主持实务部门。

韩冈希望自己手边能多一点有干才的助手,这样一来,自己也能更轻松一些。

