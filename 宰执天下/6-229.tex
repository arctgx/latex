\section{第30章 回首云途路不遥(中)}

赵煦的运气也算好,早早就出了事,否则再持续一段时间,身体真要垮了。

现在最多休息一个月,差不多就能恢复正常了。

“希望天子能接受这一次的教训,日后不要再糊涂了。”韩冈将文件折了一下,递回给宗泽,“至和、嘉和那十年,仁宗皇帝时不时的便缠绵病榻,全都是年轻时留下的病根。”

宗泽跟在韩冈身边时间长了,也经常听到韩冈评价历代天子,只是他地位还不到,不可能拥有宰相才有的洒脱,只能讷讷的道:“天子的确是要好生调养。”

不论宗泽有多出色,他心中积累下来的沉淀太深厚了。许多事,他是无法靠自己的力量去踏破。

宗泽拿着文件出去了。

目送宗泽,韩冈觉得这件事暂时可以放一放了。

犯事的人业已开始了万里之行,名义上的受害人则躺在床上休养生息。

皇城司的人在看管着剩下的三位曾经承受恩泽的宫女。等到她们被确定是否怀孕了之后,再作处置。

朝堂上也达成了共识,一切恢复如初。

韩冈伸了个懒腰,这件少年初识风月闹出的一场风波,也该风平浪静下来了。

“相公。”刚刚出去了的宗泽,这时候突然又进来,手上一份文件,“这是今天的简报。”

韩冈收起伸展开的双臂,接过来,“有什么消息?”

除了赛马和蹴鞠两家快报之外,京师还有许多小报。只要还没有威胁到两家的地位,都会被放过一马。甚至在其中,有许多家小报社,都有两大报社的股份在。

包括韩冈在内,都把新闻报纸当成是自己了解民间舆论的窗口,但那么多份报纸良莠不齐,而且内容也不可能全然是的,所以就有了简报。

除去两份快报需要亲自浏览一遍之外,两大报社的内参,皇城司的日报,还有多份报纸的简报,都是节选,通过不同角度的报告,让韩冈得以了解京师内外的一切重要新闻。每天午后,通过五房检正,放到每一位宰辅的案头上。

“是天子的。”宗泽手指着简报。

宗泽递过来的简报,翻开的那一页,韩冈只一瞥,‘官家’,‘太皇太后’,两个词就映入眼帘。

下面还附了一份原版的报纸,打开看正面第一版,刊名新京新闻,下面的头条又是如此。

“什么时候,京城的报纸变得这么大胆了?”

过去的报社,就算想要报道与天家有关的新闻,都要想方设法的回避直接描写。就像白居易的汉皇重色思倾国一般,明明白白写的是唐明皇,却要用个汉皇来遮掩,如今也一样,专有名词要用其他词汇来替代。直接说官家、太后、太皇太后的报纸几乎没有。一旦犯了戒,开封府的大狱会让东家、主编和编辑们,知道什么叫做‘你有言论上的自由,我有处置你的自由’。

“必是有人在背后指使。”宗泽沉声道。

韩冈一目十行的浏览了一遍,主要说的就是皇帝在太皇太后丧期欢娱过度,以至卧床一事。

现在市井中的传言,主要是说皇帝是为人所诱,所以才会犯了大错。十二三岁的小孩子,在男女之事上本就是懵懵懂懂,很难经受得住这方面的引诱。这一次的事,将责任推到皇帝身边的人身上,对皇帝的名声最为有利。而且太皇太后的丧期,说实话,世人也并不是那么在意。民间的婚丧嫁娶,过了百日,朝廷就放开了。

在这份报纸上,并没有说皇帝什么不是,而是在尽力的帮皇帝解释他做下的那等事并不违背礼法。向世人说明,作为皇帝,赵煦不需要守对祖母的齐衰之礼。

乍看起来是在帮忙,如果只看字面上的意思。但这份报道给人的感觉,就是赵煦不再是类似于被害者的身份,而是一位抓住律法上的漏洞,恣意妄为的昏庸之君。

新京新闻特意要点明赵煦无罪这一点,等于是此地无银三百两,让世人认定其罪。

新京新闻之所以起了这个名字,韩冈倒是能猜得到。当是因为其位于京师新区之中,故曰新京。

在几年前,因为重新修整城防,京城被加以扩大。原本外城城墙之外的大片民宅,被一条长近百里的矮墙包围了起来,名为京师外廓城。从此之后,大宋东京,从里到外,便是宫城、皇城、内城、外城和外廓城。

外廓城的城墙并非由土夯筑,而是柳条篱笆墙,实际上的防卫,则是交由七大十一小总计十八座的火炮棱堡负责。只要棱堡还未陷落,敌军甚至不能跨进外廓城的防线半步。只是这棱堡现在还没有完全修起来,至少还得有三年的时间。

不过这并不阻止外廓城中的百姓从此可以自豪的自称是天子脚下的京城爷,而不是过去的乡下人。自然,办报起个名字,也是新京新闻。

韩冈抖了一下报纸,纸质很差,油墨质地也不佳,这是一份针对的是占了京师男子人口大多数的只上几年私塾的普通人的报纸,是以最廉价的印刷成本刊印出来的小报,当然也是最廉价的,街头休息的时候,一两文钱买来,等到看完之后,还可以拿着包点东西回家。

京师中的大多数小报,大多如此。报纸上面的内容,也是怪诞不经,多是乡间的各种神怪传说,也有蹩脚的诗词,还有偏近鬼神、甚至色情的小说连载。新京新闻在其中并不算特别。但正是这样的一份小报,竟然刊登了别人不敢刊登的文章。

“这报纸是谁出的?”

“新京新闻社的地址位于城外京南厢的明义坊,去年九月十五创办,原为五日刊,今年五月开始,改为日刊。在开封县的记录上,报社东家姓王名春,白身,开封人,就住在明义坊内。主编纪茂直,出身楚州,八年前来京长住,没有通过解试的记录。另有编辑两人,皆是今年四月份聘入。”

宗泽对答如流,说得十分详尽,新京新闻的老底都给揭开了。这不是宗泽的能耐,而是五房检正公事下面的吏员,在韩冈一直以来的命令下做好的功课。

报纸影响舆论,所以报社中所有从事文字工作的从业者,都必须上报到当地衙门,在中书门下,也有记录下来的副本。如果没有得到批准,便自行办报,等着皇城司或军巡院的人打上门吧。

“就两个编辑?”韩冈惊讶的问道,加上总编才三个人。一份日报,再怎么粗制滥造,也不可能只靠三名编辑就完成这么大分量的内容。

两家快报都是隔日一期,而新京新闻竟然是每天都有出版。没有两家大报社的底蕴和人才,每期就只有一页四版而已,但每日都要出一期,无论如何都不是三个人能做到的。

“全都是抄的。京城的小报,几乎都是如此,你抄我,我抄你,只要两三个编辑,先摘抄来填满大半版面,再随手写点东西,最后用广告填满。不要太多成本,每个月都能赚个十几二十贯。”宗泽冷笑了一声,“要不是两家快报严禁其他报纸刊载联赛的相关消息,被抄的会更多。”

按照宗泽的说法,办一份日报,一年下来,净赚能有一两百贯了,难怪京城的小报层出不穷。

韩冈又拿起这份报纸来看。字印的密密麻麻,新闻、广告、小说连载,总计两三万字之多。就算其中大半是摘抄,也有几千字是自己写出来的。也难怪这么多报纸,能存活下来的不多。

道德经才五千言,是老子毕生学识的总结。而现在一天几千字的印刷品,放在过去,能让人写一辈子。文字的价值,随着教育的普及,真是越来越不值钱了。

当然,这是好事。

新京新闻是发行点位于新城的小报,其背后理当没有太大的靠山。但新京新闻既然敢于公开报道此事,所谓没有靠山的判断可以丢一边了。

多半是哪一家的暗子,先留下一闲笔,到了关键时候,可就是草蛇灰线的伏笔了。

将赵煦放在风尖浪口上,硬是要把他丢进脏水泡上一泡。背后是什么人,从谁最能得利,就能想明白。

“是三大王,还是濮王家的那一堆?”韩冈皱着眉头自言自语。

“相公,要不要去查一查?”宗泽提议道。

“查得出来吗?”韩冈摇摇头,“这就跟屋子里的蟑螂一样,看到一个,后面能藏一百个。能都抓出来吗?”

“蟑螂?”宗泽不解。

“啊,赃郎。”韩冈立刻更正。

蟑螂的古今之音太相近,蟑螂二字其实就是赃郎变化而来,他一不注意就发错了音。

“哦。”宗泽没大注意,只是当成了韩冈咬字不清,“那相公的意思,就是不查了?”

“抓人吧。”韩冈道,“所有相关人等,交付开封府法办。”

掘后.台的事就算了,但既然敢把宫闱秘事都登在报纸上,自当依法行事。

该罚的罚,该流的流,没有二话可讲的。

“是。”宗泽行了一礼,便欲离开,韩冈却又叫住了他,“汝霖,顺便把白泽叫进来。”

白泽进来了,他是韩冈的家丁,被安排在中书做堂后官,平时代韩冈传个话找个人。

“白泽,你去西府,跟章枢密说一声,今天晚上,我请客。”
