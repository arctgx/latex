\section{第八章 朔吹号寒欲争锋(15)}

光是将这些资料拿进来,就用了五六人。

其中也就包括了有关蔡京这位无足轻重,却又至关重要的犯官的判决。

在刑恕‘自缢’之后,很多人在放心之余,还在意的便是蔡京是否伏罪这件事了。

而在今天开封府呈上的卷宗之中,终于确认了蔡京的罪行。多日的审讯,终于让蔡京服罪。

在有关蔡京的卷宗中记录着,蔡确在叛乱之前,曾经找蔡京私下里计议过。而蔡京为大逆案,也是出谋划策,尽心尽力。可谓是蔡确的谋主,是这一次大逆案的推动者。

据蔡渭口供,蔡确曾在叛乱的几天前在他面前称赞过蔡京的才智,并声称会让蔡京官复原职,甚至就任御史中丞。

所以蔡渭才会在事败之后,逃去蔡京家中,打算与蔡京一起逃出京城。不成想蔡京临事反复,甚至想欺瞒朝廷,博取奖赏。

另外开封府还查证,蔡京在绑了蔡渭出府之前,曾经在家中烧掉了一部分信件。

开封府问询过蔡京家中的亲信仆婢,众口一词都证明叛乱之日蔡京离家之前,的确烧过了一部分信件,而在叛乱的之前几日,蔡京也多次出外。由于蔡京在京城中的名声极差,他和他的家人都极少外出。接连几夜多次外出,突然间行动规律产生变化,是很明显的犯罪征兆。

不过蔡京否认烧掉信件,也否认曾去拜访蔡确,更否认与蔡确共谋。

一切的指控他都否认了,还反过来说这是来自权贵的陷害。

权贵姓甚名谁,蔡京在开封府的大堂上将那两个字说得字正腔圆,听不出一丝福建口音。但记录中却看不到这个名字。

张璪抬头望着对面的‘权贵’,年轻的面庞上看不到岁月的痕迹,只有眉心上有着几条纵向的纹路,显然经常皱眉苦思。但张璪很清楚,这一位让别人皱眉头疼的次数,应该是他本人的十倍、百倍。许多人皱眉之后,绝不仅仅是眉心上多了几条纹路这么简单。

有的送了命,有的贬了官,还有的就是等待着朝廷最后的决定。

低下头去,张璪的注意力重新回到开封府的奏报中。

所有被关进开封府的人犯里面,在蔡京身上花去时间最多。

在一次次拉锯中,主审和陪审的官员多次声明,如果蔡京肯认罪,则能饶过他一条性命。若是怙恶不悛、死不悔改,便是要严办到底。朝廷纵是宽大,也不会将恩赦赐予不愿悔改的贼人。

不过即使这么说,蔡京也不肯松一下口。

断案最重口供,若犯人不肯认罪,这桩案子就无法结案。不论是有证人的证言,还是充分的物证,都必须要犯人服罪才行。

只要蔡京咬定牙根不去认罪,这件案子就结不了。在急着结案的情况下,蔡京就只能做另案处理,那时候,便还有一线生机。

直到四五日前,很多关心这桩大案的朝臣还是觉得,开封府最后恐怕只能让蔡京直接瘐死狱中,而不是能拿到蔡京伏法的供状。

但不知出了什么事,蔡京突然间却一口承认了所有的指控,包括他是蔡确叛逆谋主的指控,也包括他暗藏侥幸,希望能够蒙混过关的想法,一起都承认了。

据陪审的大理寺、审刑院的刑法官所说,最后一次过堂,蔡京的身上依然没有一点伤,就是整个人萎靡不振,变得痴痴傻傻的,完全不见了当初意气风发的模样,也不见之前否定指控的坚决。

张璪在开封府也有一个耳目,根据他的说法,开封府在审讯蔡京的时候,完全没有用刑。

从头到尾,即没有打,也没有夹,什么刑具都没有给他上。

一开始也只是用御史台对付官员的故技,以肮脏的饮食,来消磨蔡京的意志。

只不过在蔡京始终不肯服罪之后,审讯的方式突然改变了。

不再过堂,而是改在了阴暗的牢房中。吃照给他吃,喝照给他喝,只是用灯光照着脸,不让蔡京睡觉,又不知从哪里拿来两支铁条在蔡京耳边锉着。

据说那种铁条摩擦的声音,听了之后,就让人浑身发毛。

那位耳报神在张璪面前回报时,两只肩膀一抽一抽,显是心有余悸的样子。他毫不隐瞒的告诉张璪:当时没多久他就夺门而出,可事后一回想起来,心里还是燥得慌。

在这样的折磨下,蔡京只熬了两天,就变成了要他说什么,就是什么,而且整个人都废了。

之后过堂,蔡京除了点头说是,完全没有别的反应。沈括拿着供词一句句问,蔡京便一下下的点头,然后签字画押按指模,一气呵成,顺利通过。

想起耳目回报的内容,张璪心里就一阵发寒,这到底是什么刑?蔡京这个最不该软的,偏偏就软了,难道真的有那么酷毒?

由于没有实际体验,张璪不知道蔡京受到的折磨有多恐怖,但从回报之人的表现来看,已经足够让人惊骇。只是旁观者,就变城那副模样,那亲身体验折磨的蔡京,能支撑两天已经是很了不起了。

而且张璪还确定了一件事,能想到此种拷问之法,这样的人还是不要招惹。

……………………

韩冈对张璪总是张望自己感到很奇怪,难道自己脸上有什么地方脏了?

可韩绛那边完全没有异样。而且方才进出厅中的堂吏,也会提醒自己才对。

想了一下,韩冈就放了下来,继续翻看开封府进呈的卷宗。

由于蔡京最终还是认罪,开封府在判决中给他留下了一条性命,不过对他判罚是流配西域。

而蔡卞被蔡确、蔡京拖累,没能像苏辙一样仅仅是贬官,而是夺去了官身,就此成为平头百姓,且又空出了一个好位置。

蔡确、曾布、薛向,在两府中,提拔任用了不少官员。这些官员,身上都贴着蔡、曾、薛的标签,尽管没有参加叛乱,但他们想要一点不受牵连,自是不可能。不说别的,他们屁股底下的位子就是一块块绝好的肥肉,吸引着多少垂涎欲滴的目光。

只不过蔡确、曾布、薛向三人留下来的这些蛋糕,要瓜分起来还是很费些时间。

由于他们的党羽人数实在太多了一点——在京百司,到处都有他们的身影。位置也关键——蔡确不说,薛向掌握六路发运司和三司多年,汴河转运和朝廷财计上的官员多少都是他亲手提拔起来的。事关京师的稳定,一个不好,京城大乱,汴河水运又乱了套,东府的三位,哪一个都逃不过罪责。

在不损害朝堂稳定的前提下,清理三位叛臣在朝堂上留下的色彩,是一桩旷日持久的大工程。不仅需要精心筹划,更需要耐心。

但是清理他们的亲族,就是一件迫在眉睫,而难度稍低的问题了。

朝廷意欲息事宁人,不过其亲友不能不加惩处,仍留其在高位,当然不可行。

这不仅仅是三五人调任偏远小郡的问题,而是一大批。除去已经被定罪流放的如曾巩、曾肇,剩下的依然至少有几十人要去职、贬官。

最典型的就是苏轼的弟弟苏辙。

苏辙正在楚州通判任上,比起自变法一开始就唱反调的苏轼,苏辙因为先接受了王安石的征辟,做了制置三司条例司属官,之后却在天子面前大唱反调,故而比苏轼的官路更为坎坷。

不过这一回苏轼都仅仅是追夺出身以来文字,并流放交州,遇赦不得归。他的弟弟不过是受到牵累,当然也不会太重。

韩冈翻了一下张璪亲笔写下的提议:“泰州西溪盐务?”

这算是很轻的处罚了,还是在淮南。除了辛苦一点,至少还是一名官人。

“倒是不算重。”他对张璪笑道。

“够重了,西溪多蚊蚋,自春至秋,人不能露天而坐,牲畜也得以泥浆沫身,否则必至病。”

“是吗?”

张璪道:“范文正公曾为此职,曾有诗句记西溪蚊蚋,‘饱去樱桃重,饥来柳絮轻。但知离此去,不要问前程。’”

韩冈笑道:“不意文正公也有拈轻怕重的时候。”

“范文正提议修海堤,当是怕了西溪的蚊子。”

张璪说罢,便轻笑了起来。

不过朝廷如今若是安排苏辙去做盐务,想必他连迫不及待的赶着去上任,生怕朝廷会变卦。

如苏轼的兄弟苏辙,曾布的亲族要怎么处置,都是需要大费思量的一件事。

苏家在蜀中不大不小也是个名族,亲友众多。而南丰曾家更是江西数一数二的名族,连曾家的女婿在内,曾经一科七进士,西北好几个州加起来都没这么多。曾家的姻亲更是遍及南北,王安国便是娶了曾家的女儿。

曾巩、曾肇之外,曾家在官场上尚有其他子弟多人,遍布朝野内外。不过既然是南丰曾家的成员,当然一体受到牵连。

经此番打击,曾家几代人的努力化为泡影,日后能不能重新崛起,希望十分渺茫。

韩冈对曾家没有太多的关注,若是士林为其叫屈的声音太多,让他们去修《太平广记》之类的类书——至于史书就不可能了,那可不是犯官亲属能做的位置。他现在关心的是考试

——这一科的礼部试,终于要开始了。

