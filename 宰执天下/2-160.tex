\section{第29章 顿尘回首望天阙(12)}

【第二更,求红票,收藏。】

当天夜里,在一家僻静的酒楼里,韩冈和章惇又坐在了一起。

章惇刚刚落座,却又站了起来。向着韩冈一揖到底:“玉昆,今次之事,愚兄实在对不住你!”

韩冈没敢受章惇的大礼,很仓促的闪到一边。但心中还是有几分不快,言辞便锋锐了一点:“蔡确毁诺,非是检正之过。检正何须越俎代庖!”

“不,愚兄是待苏子瞻来道歉的。”章惇摇了摇头,正色对韩冈道,“昨天去审教坊司的那个小吏的时候,苏子瞻就知道他自己做岔了。但判状已经发出去了,追也追不回来,只能徒唤奈何。”

韩冈神色不动。章惇继续说道:“本来你和周小娘子的事,也是一桩佳话。若是子瞻事先听说了,真的会成全了玉昆你。只是阴差阳错啊……”

章惇其实也挺替他的老友感到无奈的。苏轼的性格,结交数十年的章惇很清楚,若是周南贸然申请脱籍,身后又没有什么奢遮人物,苏轼肯定不会同意,所以他前日提醒了韩冈,千万不要把申状递到推官厅去。

只是当韩冈和周南的事闹得满城风雨,士林清议倒向韩玉昆的时候,以苏轼的聪明,就不可能站到反派位置上去。而且以他爱凑热闹的性子,从中推波助澜,帮着韩冈把事情闹大,才是苏子瞻会做的事情!

但他的这位老友聪明归聪明,偏偏是又个行事疏阔的人,判状前也不是先打听一下,周南申请脱籍是为了什么原因。但凡多问一句,也没今次的事了。可叹现在判状一出,在士林中,苏子瞻可算是丢了大脸。

章惇为苏轼低头,光是看在他的面子上,韩冈都不能再继续计较。章惇胸中一股任侠之气,为友两肋插刀的做事,也让韩冈甚有好感。

“不知者无罪。既然是检正为苏子瞻说合,韩冈哪能再纠缠不休。”

韩冈的话,虽不代表已经冰释前嫌,但也是无意继续下去的表示,章惇挺高兴的替苏轼谢了。

“……还有蔡持正,方才与他碰面时,他说是过两日要向玉昆你摆酒致歉。”

章惇说起蔡确,就不如提到苏轼时那么诚挚。说起来,蔡确其实也是阴了他一下,让他在韩冈面前丢了脸。章惇心中理所当然的不痛快,也有几分看不起言而无信的蔡确。可是现在的形式,让他必须帮蔡确说话。

韩冈默不作声端起茶盏,慢慢地啜着杯中的茶水。

蔡确这等人,总是会为选择对自己能带来最大利益的一条路,毁信背诺之事虽不会刻意去做,但与利益相冲时,该如何选择他们都绝不会犹豫。人不为己,天诛地灭,蔡确当是抱着这样的想法。而且要向韩冈这个并非进士的小小选人示好,恐怕蔡确心中也觉得憋屈。

而章惇对蔡确的态度也已经很明显了,疏远,但不会针锋相对。

章惇等着韩冈的回答,房中一时静了下来。一杯茶,一口口的慢慢喝光,掌中温热的瓷盏渐渐冷了下去,韩冈突然单刀直入的沉声问道:“相公要荐蔡确为何官?!”

“三班主簿。”章惇脱口而出,说出来后才‘啊’了一声,摇头苦笑,自觉失言。

“三班主簿啊……”

这是主管低阶武臣的三班院中的文职,不算低了。蔡确的确是阴了韩冈、章惇,但因此而得到了王安石的看重,从结果上看,他的选择是没有错的。巴结王安石的亲信,当然不如直接示好王安石本人。

而王安石现在正愁手上人才匮乏,连个半疯颠、爱乱说话的唐坰都启用——那可是上书说要斩韩琦、文彦博脑袋、以便推行新法的狠人;王安石想着千金市骨,所以便提拔了他——可见他手上究竟是多么缺乏人才。如蔡确这样旗帜鲜明的进士,王安石有不重用的道理。至于蔡确毁诺一事,就算韩冈和章惇说出来,王安石也不会太计较。就像章惇,名声也不算好,还不是照样被重用?

韩冈沉吟了一下,蔡确有王安石的看重,加之自己再来两天就要离京,周南那里还要处置,没时间找蔡确麻烦。想了想,帐要慢慢算,先把利息拿笔回来再说,便道:“最近家表兄在三班院那里颇不得意,也许今次试射殿廷可能会有人从中作梗……”

章惇先是一楞,然后就放松地笑了起来,韩冈肯提条件,便是与蔡确和解的表示。他虽然不值蔡确为人,但王安石现在要用蔡确,韩冈与其过不去,不会得到王安石的支持,反会让自己从中为难。

“这是小事而已,三班主簿品位虽不算高,但在三班院中,也能说得上话。不过蔡持正要去三班院上任,还需要一阵子。今次试射殿廷最好让李信称病,等到年后的下一科。”章惇为韩冈想着主意。

韩冈皱眉问道:“称病误考,可会有什么挂碍?”

章惇摇头笑着:“玉昆你多虑了。入京的文官武官,水土不服的情况多得是,三灾八难谁也避免不了,何独令表兄能例外?”

“那就要多劳检正了。”

韩冈不提蔡确,只拜托自己,看起来还是心中有着芥蒂。当然,章惇心中也有芥蒂,蔡确的确是落了他脸面,“玉昆你放心,这次决不会让人打扰了。”

“至于周小娘子之事……”提及周南,章惇则是犹豫了一下。本来能顺利玉成的好事,却被苏轼和蔡确联手给坏了。一个是他的挚友,一个则是他荐给韩冈的助力,说实话,这让章惇这个中间人觉得很有些对不起韩冈,“王相公已经答应帮你,不过眼下风高浪急,想脱籍却是要等到两个月后,”

‘两个月?!’

韩冈听了后,就皱起眉头。他哪里能放心?才一天就捅了篓子,还要几个月?!韩冈可不会把信心放在王安石的承诺上,变数实在太大了。

他不答章惇的话,却岔开话说道:“天子仁德,雍王孝悌,宫中如今倒是平和得很!”

“……”章惇眼睛越瞪越大,以他的才智,韩冈话中隐义当然是一听便明。

常常逛教坊的成年亲王,竟然还能住在宫里,难道是嫌天子戴得长脚幞头颜色不正,要抹些绿漆上去吗?正常情况下,当然是要将其赶出宫去!

而且已经不是第一个有人上书要请雍王赵颢、高密郡王赵頵离开宫中。一年多前,曾有一名小官章辟光就上书天子。但在高太后的反对下,赵颢、赵頵都留了下来,反倒是章辟光被赶去了南方。

天家无私情,赵顼对两个弟弟被太后强留住在宫中,心中若能高兴那就有鬼了。如果能趁此机会把赵颢请出宫去,赵顼难道还会怪罪不成?

不过事情有这么容易吗?高太后那里边绕不过去。而且韩冈这么做是为了什么?赵颢已经不可能再出头与他争夺周南了,突然继承大宝的可能也是微乎其微。

“如果官家能下旨放周南的话……”

“官家下旨……玉昆你是这要让雍王自己请辞?!”章惇已经越来越明白韩冈的行事作风,他的想法对章惇来说很是新鲜,但可以正确推测。

韩冈点了点头。如今事情越闹越大,已经大到必须处理的时候了。如果是为了天家名声而选择帮赵颢遮掩,那么周南就会被遣出京城,而让天子来决定,事情可就不一样了,顺利的话,就能留下一段天子为人结缘的佳话。

章惇都佩服起韩冈,也亏他能想到,驱逐雍王,卖好天子这一手段。对付眼下的情况,一个是不加理会,将风潮拖下去,拖到有人在来处置,这对赵颢的名声是最好的。还有一个方法,就是特旨将周南赐于韩冈,这等于是明着承认雍王犯了错。实际上,这么做了后,赵颢只能申请避居宫外。

为什么赵颢出宫来必须要隐姓埋名,从这一条想过来就很容易明白,并不需要多少才智,只是需要把两件事联系在一起的胆量罢了。

对……是胆量,而不是头脑。

章惇已经听明白了韩冈的用心,问题是他到底敢不敢上书提醒天子呢?

章惇当然敢。

富贵险中求,蔡确如今的加官进爵就是个好例子。而示好天子的机会更是难得,章惇当然不会放过。当然,有章辟光的例子在,章惇也会把文章写的隐晦一点。但这份功劳,他却要生受。

至于韩冈,一句话就撬动了内宫局势,因势利导的手法当真是无双无对。

章惇暗骂自己前面是糊涂了,周南之事竟然要让韩冈等上两个月,他怎么可能会等,两个月中的变数实在让人无法安心。一般来说有人只会无奈的等下去,而韩冈却直截了当的把天子都拉出来帮忙。

章惇看着韩冈,目光中不无敬佩之意,但也有几分感叹,‘难怪吕吉甫要说他是贾文和!’

