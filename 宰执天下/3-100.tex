\section{第29章 百虑救灾伤(五)}

保丁校阅的场面并没有什么可说的,完全乏善可陈。

县尉冉觉虽然对此十分上心,但在经历过开边之战、见识过最为勇猛的关西禁军,还有吐蕃、党项两家精锐的韩冈眼中,保丁们的表现也就比笑话好上那么一丁点。

如果是笑话倒也好了,还不至于像现在这般让韩冈看得昏昏欲睡。也就偶尔能发现一两人的箭术还算过得去,差不多能在上四军中混个中上游的水平。

不过冉觉很是自豪。在他眼里,方才上场的那些保丁们的表现,不比护堤的厢军稍差,与白马镇附近的那两个指挥的宣翼禁军也差不了太远了。如此精锐,若是当真来了盗贼,绝对能将其一网成擒。到时候自己也能脱离选海,得入京官——依照真宗年间颁布的条令,县尉如果能尽擒十人以上的一伙盗匪,就有改官的资格。

在韩冈的面前,冉觉领着大保的保正们,昂首挺胸等着的犒赏。韩冈则是随口赞了两句,照规矩将预备好的钱粮散发下去。只是在离开时,却亲挽一张一石五斗的硬弓,一箭射中了五十步外的靶心。这个成绩,在方才的箭术比试中,只有寥寥数人达到了。

韩冈丢下弓时,什么话都没说,只是摇了摇头。但所有人都明白,知县到底要说什么:

‘再练练吧!’

从校场回来后,游醇来见韩冈:“正言若有闲暇,还是要多往县学中走走。到了十五之后,县学就要停课。在这之前,照例是要开考,这题目还是得由正言来出。”

照规矩,县学是每月一小考,年终一大考,连续三次小考最下,或是大考不过,便要当即开革。朝廷不会用宝贵的资源来养废物,韩冈对此举是双手赞同,但要让他这位关学嫡脉出题去考较此间的士子,免不了会在题目和答案跟程颢的弟子起冲突。

韩冈本想着还是算了,如今真的没有多余精力去照管这些他名义上的学生,只是条令规定要做的事,却是不便推搪:“过两天我就去县学中。只要是用心向学的,当让他们过个好年!”

敷衍过游醇,魏平真又问道:“听说今天文司空的儿子又来了?”

“文及甫?他是去京中拜见他的岳父,路过而已,不过明天我还要送他一程,尽一尽人事。”

文及甫要去东京城,今天正好落脚在白马县中。不论从官场的礼节上,还是从关系上,韩冈都要按照他的说法‘尽一尽人事’。

文彦博的六儿子文及甫是吴充的女婿,吴充的大儿子吴安持则是王安石的女婿,而韩冈与吴安持是连襟。说起来,他跟文彦博都有点瓜葛亲。但这点亲缘,在如今的官场上根本不算什么。随便将任何两位重臣拎出来,差不多都能三五转之内,攀上亲戚关系。

韩冈对这等蜘蛛网一样的官场生态叹为观止,不过看看也就算了。亲戚关系什么都决定不了,王安石、吴充这一对亲家可是死对头,而韩冈与太后都能攀上关系,但他最为亲近的还是一点亲缘都没有的王韶父子。

文及甫是不是拜见吴充,韩冈其实无从得知,但他赶在过年前跑去东京城,回大名府后,少不了会给文彦博带回去第一手的京中新闻,韩冈算算时间,差不多该到了正戏该上场的时候了,不知道文彦博听说王安石将宿州的存粮当真运抵东京后,又会是一个什么样的表情?!

……………………

韩冈正盼着好戏开锣,而京城中,垫场的开幕戏其实已经开始了。

京城中的官场上,现在正在嘲笑王安石的慌不择术。他此前力排众议的提案,如今成了最大的笑柄。冬日开河口的措施还没有施行,为此而打造的器具已经宣告破产。

于汴河河口处的汜水船场所打造好的碓冰船,在黄河中进行试验的时候。虽然安置在船头上的大碓的确敲开了接近一尺厚的冰层,但驶进河中的木船却立刻就被河道中的流冰所挤毁碾碎,差一点,就连船上的船工都一起给送了性命。而且还不只是一艘,而是新近打造出来的总计四艘的碓冰船,全都毁在了黄河之中。

这个消息传回来,官场上、市井中,立刻就有了酒席上的谈资。

“我早就说过,冬天开汴口根本不可能,现在看看怎么样,还能开吗?”

“王相公这下黑脸要变白脸了,硬是强着天子御笔题朱,现在不知他要怎么去见官家?”

“今年是好戏连台,先是上元节宣德门的一棒子,然后是琼林宴上丢石头,再来就是天下大灾,如今再以此事收尾,这才叫做完满!”

自吹先见之明的,说风凉话的,幸灾乐祸的,不一而足。除了新党以外,几乎所有人都在这次失败的实验上找到了优越感。

冯京、蔡确正坐在的冯参政府的暖阁中,喝酒聊天的同时,也不免带上这一桩东京城眼下最流行的笑话。

两家刚刚定下了儿女亲——就在半个月前,蔡确为他的长子蔡渭,向冯京家的十三娘下了聘礼。

从只能用诗词来奉承宰相的小臣,到如今御史台的第二号人物,蔡确只用了两年的时间。不论是在开封府任上顶着新任的知府刘庠,还是进了御史台后对恩主王安石反戈一击,每一步,每一个转折,蔡确都没有错过半点。

蔡确的行事作风,引来了不少警惕的目光,但让冯京很是看好这位新任的侍御史知杂事的官运。能够准确地揣摩上意,能在恰当的时间出手,说不准过上个几年,就能给蔡确他挤进政事堂中。定下这门亲事,日后当少不了好处。

也正因为已经成儿女亲家,蔡渭作为御史台的副职,快过年的时候到参知政事家拜访,就不会引来多少议论。

商家出身的冯京素来善于聚敛,一个金毛鼠的匪号尽人皆知。但在冯京家的暖阁中却看不到半点金玉之物,装饰素雅简洁。不过若是将注意力放在陈设上,暖阁中每一件器物其实都是有来历的古董。看似简单的客厅中,却隐隐透着富贵气。

红泥小火炉上放了个烫酒的水煲,水煲中咕嘟咕嘟的响着。而酒气从浸在热水中的酒壶散出。几个银碟中的酒菜不算多,却做得极精致,甚是还有冬天极为难得的绿叶菜,乃是靠着温泉种出来的。

蔡确喝了一口冯京亲自斟上来的酒水,酒气立刻直冲囟门,一股火辣辣的感觉顺喉而下。蔡确被冲得呛咳了几声,皱眉看着这杯盛在雕花银杯中的热酒,烫过后竟然还这般烈,“这酒水是蒸过的吧?”他问道。

冯京陪了一杯酒,却是一点事都没有,只是英俊的脸上有些泛红而已。他笑着回答:“喝惯了就好。烈酒可以去阴湿,阳气虽重,但在冬时饮上几杯却无大碍。”

“只是喝多了就不行了。肝乃木性,遇烈阳则枯,酒喝多了会伤肝。”蔡确如此说着,却将杯中酒一口干下。

“这话还是韩冈说的。”冯京呵呵笑了两声:“王相公家的女婿虽说一直不肯承认,这医理却比谁说得都透。”

韩冈对烈酒的评价,如今早就在士大夫和医生们的口中流传。连同烈酒的蒸酿之法,也同时传遍了京畿一带。虽然蒸酿过的酒水过于劲烈,但好这一口的人还不少,尤其是到了冬天,更是祛寒的良法,多有趋之若鹜的。而按照韩冈的说法,酒乃至阳之物,所以在一些医生手中,用烈酒伴服丸药,也成了标准的医方。

“前两日,李士宁开了一方丹药,就说是要用热酒伴服。一枚大丹伴着烫过的烈酒服下去,浑身的阴寒全都不见踪影。”在蔡确面前,冯京并不避讳自己服外丹的习惯,“这韩冈,在医理、医药的见识,的确是难得一见的精深,要说他不是见过了孙思邈,这传承又是哪里来的?”

蔡确回忆起当初在章惇的宴上见到的韩冈,现在想起仍是觉得他的确不简单:“韩玉昆不但医理过人,在机械上,他也是过人一等啊!”

“说的是雪橇车?”冯京抬了抬眼皮,笑问着。

蔡确点了点头,“当然!”

一个是宰相的副手,一个是御史中丞的副手,六路发运司打造雪橇车的行动当然瞒不过他们。一份天子经由中书下达的诏令,需要参知政事副署,御史台也有权过目。王安石让薛向做的事,冯京和蔡确都有资格掺上一脚,但他们却都放了过去。

一方面是王安石已经被逼到绝境,现在与其当面顶撞,并没有任何好处,反而会因困兽之斗,而将自家给栽进去。另一个也是因为他们不相信王安石能成功,等到他失败后,再踹上一脚将会更为省力。

其实王安石要开汴口,造碓冰船传到外面后,又有几个人相信他能成功的。后来又多了一个雪橇车,虽然王安石对此尽量低调,但在东京城哪有秘密可言,反倒转头就给传遍了。

碓冰船乃是都水丞侯叔献所献。而这都水丞更是如今朝中首屈一指的水利大家,他提议的碓冰船尽数毁于流冰之中,成了东京城内的笑柄,难道韩冈在水利上的才华还能比他强?

“王介甫是病急乱投医。熙河路的奏章我也查了。雪橇车的确有用,但都是三五辆一队,送些消息酒水和银绢犒赏的。从来没有说熙河路的粮秣运输能靠雪橇车来完成。要将几十万石。”冯京冷笑着,重复的强调:“这是病急乱投医!”

