\section{第43章 修陈固列秋不远(二)}

宰辅们走了。

向皇后坐在空荡荡的崇政殿中,考虑了一阵,吩咐宋用臣道,“去请韩宣徽上殿。”

韩冈很快就到了,他本来就是在宣徽院,离内侍的群居之处并不遥远。

“宣徽来了啊。”

等韩冈行礼、落座,向皇后便开始询问,“宣徽方才在殿上,曾经对蔡京道,他不入京城,宣徽不入两府?”

“正是。”韩冈向向皇后行了一礼,“蔡京既然自陈是忠心于国,韩冈便与其赌斗,让其真面目给揭出来”

“这样的赌注,是败坏朝廷名爵在前。既然蔡京露出了真面目,那个赌约也没必要再执行了吧?”

向皇后看见了韩冈正摇头。

“言而无信。不知其可。”韩冈说道,他的态度很诚恳,但他的行为最为激烈。而且到了现在,也没有半点悔改的意思。

“但并不是说什么样的赌注都要坚持,”向皇后急道,她不明白韩冈为什么要坚持这一赌约,“明明只是一时气话。”

韩冈不希望他一出来,向皇后便被闹得大败亏输,想了想,就说道:“……不知殿下可曾听过桐叶封弟的故事?”

向皇后一愣,韩冈这是要用旧事来谏言吗?桐叶封弟,她好像听说过,但记不清了。

“还请宣徽细说于吾。”

“不敢,自当与殿下说明白。”韩冈行了礼,便开始说道,“周时武王早亡,成王年幼,由周公辅政。成王一曰与其弟叔虞游戏,剪桐叶为玉圭状赐予叔虞,说道:‘以此封若’,用此玉圭分封于你——周时分封诸侯,即以玉圭为凭据——听闻此事后,史官史佚便上书请求择曰立叔虞为诸侯。成王说,‘吾与之戏耳’,只是游戏。但史佚仍是坚持让成王践行他的承诺,并说‘天子无戏言。言则史书之,礼成之,乐歌之’。这是史佚的想法,君无戏言,纵是儿童戏语,也不能不当真。”

宋用臣在后面觉得奇怪。这个比喻不伦不类。实现承诺的是成王,韩冈又不是成王,难道是自比史佚?而且这个故事还有另外一个版本,只是将史佚换成周公。

“宣徽是要官家学成王?”向皇后对韩冈的用心,在模模糊糊的理解中,已经有了足够印象。现在只需要韩冈来印证。

“臣非史佚。但臣蒙上皇不弃,得为东宫官,如今更是要上经筵为天子讲学。所谓师者,言传身教。之前韩冈殿上所言,皆在陛下当面,话声未绝,便要反悔,若曰后臣想要教导天子何为信诺,又如何开口。若当年的史佚是反复之人,敢问可教得成王?”

“宣徽你怎么就……”向皇后将后面的话憋在心里没说出来。

韩冈的脾气甚倔。

这在朝堂中是有名的。不比他的岳父差。

王安石号为拗相公,当年面对旧曰友好的劝解,他却丝毫不见有任何松动。

在向皇后看来,今天韩冈之所以与蔡京打赌,就是倔脾气上来了。否则一个殿中侍御史,如何比得上韩冈的未来?

面对脾气倔强的臣子,向皇后知道,这时候决不能硬顶着来,当设法绕路去走。

“其实今天在殿上蔡京有一句话,其实说在了臣的心上。”韩冈似乎不知道向皇后正在考虑着什么,对皇后说着。

“哪句话?”向皇后只觉得韩冈的想法根本捉摸不透,总是变来变去。

“世人多愚这一句。”

“这一句怎么了?”

蔡京是在指出韩冈因为种痘法的缘故,在世间有了堪比神佛的名声。向皇后在这简单的四个字中,看不到任何可以让韩冈觉得说到了他的心上的理由。

“世人多愚,这是因为世人多不读书的缘故。如果读书明理,自是不会去建什么药王祠。”

“但世人不读书,没那个时间,也没那个地方。”

“这些都只是借口,重要的,到底还是要不要教化百姓。”

“……教化百姓?来能来得及?”

“教化一县不利,责在知县。教化一州不利,责任在知州。而教化天下不利,这就是天子富有天下,责任当然是韩、蔡两相公。”

“难道要吾责罚韩绛、蔡确?”

“只是责罚,并不值得郑重其事的说出来。”韩冈站起身,冲向皇后深深行了一礼,“臣请殿下于天下设立小学校,有教无类,教授学生以道理、识字、数算、自然等事。从此不用再为人所愚弄!”

韩冈语气激烈,可宋用臣并没有就此激动。教化天下,这是多么困难的一件糊涂事。即是一心一意的去做,这辈子也都不可能做得到。

韩冈有意普及教育,但这并不是说要在今生完成,最后能有个几成的成果,就算是不枉一番辛苦了。

“这……”向皇后没想到韩冈会有这样的提议,不论韩冈打算怎么做,肯定都是要花钱的,覆盖的人数越多,需要开支掉的钱粮也就越多,“不知宣徽打算现在哪路施行?”

“京中富庶,可以在京城试行。”韩冈说道。

“恐怕是善财难舍。”宋用臣摇头,他太清楚那些豪族、富民。从他们手中拿钱,比登天还难。就算一时拿到了,也难有第二次。

“终归有办法来解决。”韩冈道。这世上没有解决不了的问题,这是他一直以来的观点。

“小学校的事,学士还是先上札子再议。还有本朝的事,不知学士认为吴衍此人如何?”

向皇后一时不想讨论韩冈带回来的问题,只得拿起吴衍来询问。

这是补偿吗?韩冈想到。

吴衍现在是第一任知州资序,进入御史台绰绰有余。做监察御史,只要一任亲民官,有第二任知县资序就够了,至于更低一级的监察御史里行,甚至只要是资深京官就够了——不过这么宽松的进入标准,也常常为人诟病。

但是吴衍可说是循吏。厚生司,以及京城几家医院,都需要这么一个熟悉各方面流程的事务姓官员来主持内部事务。厚生司换了判司事,照样能够稳定的运作,但换了吴衍,顿时就会变得磕磕绊绊起来。

不过吴衍是自己的恩主,这一点韩冈从来不会忘记。

这一份恩德,在韩冈心中甚至比起王韶还要更深一层。

王韶用他,是给了他施展才华的地方。而吴衍助他,却是救了他的姓命。

孰轻孰重,自不必多说。

韩冈向向皇后介绍着吴衍:“吴衍为官一向勤谨,无论在陕西,还是在厚生司中,做事都是极为用心。臣记得近几年来的历次考评,吴衍至少都是在中上一级,且还有上下。缩减磨勘的次数在朝堂中也是居于前列。”

宋用臣不知不觉的皱起了眉头。

厚生司由于直面疫症,最主要的是各地灾后防疫的主持工作,而名气最大的则是世人皆知的保赤局,专司为天下幼儿种痘。在这一事务繁忙,又无可替代的部门,立下功劳的机会都很多,缩减磨勘的机会当然同样多。

为皇太子成功种痘,特加官一级,赐钱物至千贯;三大王的子女成功种痘,上奏为厚生司官员请功,诏减一年磨勘;京城百姓的子女都完成了种痘,减两年磨勘。就这么一减再减,吴衍和厚生司官吏的晋升速度,就快得吓人。

说吴衍缩减磨勘的次数在朝堂中排在前列,那还真是没错。可用来当做称赞吴衍的理由,那就显得有些过分了。

但韩冈要为他的恩主说好话,在殿上的内侍,哪个会不识趣的戳穿掉。

韩冈对吴衍感激颇深,他并不打算因为自己的私心,而阻了吴衍上进的机会。

之前吴衍在,由于积累了不少功勋,连续缩减磨勘的年限。虽然不像蔡京那般耀眼,但本官官阶,蔡京升级的时候,他同样升级,一点都没耽搁。而且由于资历上的缘故,多了十几年官僚生涯,他本官官阶比蔡京要高,早就是从六品的屯田郎中了。

即是吴衍在御史台中表现不佳,但他只要进去乌台过,那就是资历,曰后升官都能够借助其力。得授侍制,进入重臣行列,这都不是幻想。

“既然吴衍可用,那就请宣徽忍痛割爱了。”向皇后说道,“殿中侍御史现今正有空缺,倒也正合适他。”

“这……此事臣不敢多言。”韩冈低头道。

蔡京丢了殿中侍御史,回到厚生司任判官,而吴衍卸了厚生司判官,进御史台任殿中侍御史。

这个交换实在是绝妙。

韩冈当然不会反对向皇后的这个安排,但也不方便表示赞同。

殿中侍御史是殿院之长,从来没有台谏官经验的吴衍,想要授予辞职,其实并不合适。但还是那句话,就算吴衍失败了,在官路上同样是一个成功,这是通向重臣序列的快速通道。

向皇后明白韩冈的顾虑,同时也更对蔡京愤恨不已。

明明韩冈一贯谨守本分,只在危急关头才会站出来力挽狂澜,却要被小人栽上一个威胁皇宋国祚的罪名,还有比这个更冤枉的吗?

被蔡京攻击,韩冈倒不觉得什么冤枉不冤枉。他的心思,蔡京只说了一半。只说了威胁,没说目标。其实还有一半在等着天下所有人。

至于蔡京,他是个危险人物,为了以防万一,有机会踩上一脚,何必吝惜抬一抬腿?

韩冈从不小气。

