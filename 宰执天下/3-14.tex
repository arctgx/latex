\section{第六章 日暮别乡关(下)}

【连续两天写到三点,白天还要上班,感觉快吃不消了。从明天起,恢复正常的更新时间。但两更不变,各位不用担心。】

吕惠卿回来了。

这个消息,在刚开始的几日,没有在京中引起太大的关注。

虽说吕惠卿是三年前新党的第二号人物,但因为回乡丁忧耽搁了三年时光,现在已经是时过境迁。

旧党的几次反扑,他不在场;横山、河湟的两场大战,他也不在场;诸多法令的制定、修改和推行,他同样不在场。不但官位停滞不前,连积攒下来的人脉都断了。

且在他回乡守制的这二十七个月里,曾布已经取代了他的地位,成为了王安石的助手。章惇去了荆南,博取一个开疆辟土的功劳。王韶已经建功立业,成了宰执班中的一员。更别提当日那位曾经在王安石府上侃侃而谈的还未入官的士子,现在已经是从七品的国子监博士。吕惠卿反观自己,竟然还是正八品的太子中允。

不过天子和王安石给吕惠卿安排的差事,还是让人明白了他所受到的看重。可这不是吕惠卿想要的,只能说,可以勉强接受。

王安石执掌着中书,但并不是代表他在政事堂中能一手遮天,冯京、王珪都不是省油的灯。真正让王安石和新党控制着朝局的是两个职位,一个是判司农寺,另一个则是中书五房检正公事。

判司农寺,统领着司农寺这个新法修订编纂的机构,各项条令法度自此而出;而中书五房检正公事,则就是王安石在中书的第一助手,辅助其处理天下政务,权柄甚至直逼冯京、王珪两个参知政事。

如果韩冈在这里,他会说,这个两个衙门,一个管得是立法,一个管得是执行,剩下就差一个监察机关了。

而监察机关——御史台,新党其实也已经控制住了。御史中丞邓绾一直以来都是新党安插在御史台中的关键人物,三年来,一步步的升到了台长的位置上。

对于邓绾,旧党恨之入骨。而邓绾本人,也不是德行高致、无可挑剔之辈,王安石并不是很喜欢他,只是不得不用,所以一直进入不了新党的核心层。

吕惠卿不会去抢邓绾手上的权力,他的志向不在于此。但如果判司农寺和中书五房检正公事这两个职位,不能拿到一个在手中,那他在新党中的地位就不可能稳固得下来。

可吕惠卿现在得到三个差遣——判国子监、天章阁侍讲、同修起居注——离他的目标还有很远的距离。

判国子监这个差遣,也许日后会很重要——对新党的未来很重要!因为昨日吕惠卿在相府中听到王安石亲口所说,他日后有意废除科举考试,而以学生在各级学校中的成绩来给予功名。如国子监,只要能在其中升入级别最高的上舍,就能得到一个进士出身,抡才大典将会为之大变——不过吕惠卿当下只想考虑现在,无意去顾及未来。只有重新进入新党核心,他才会有多余的精力。这个职位有等于无,唯一的用处,就是明年的礼部试他应当能插上一脚了。

旧日的集贤校理这个贴职,升为天章阁侍讲也是理所当然的升迁。吕惠卿本来就是崇政殿说书,现在自然得升任侍讲,以便在经筵上为天子讲学。在一般人的眼中,这个能经常见到天子的职位已经是难得的美差了。可在吕惠卿看来,还不足以弥补他这三年远离朝堂后,造成的与天子的生疏和隔阂。

只有同修起居注这一差遣,才是让吕惠卿松下一口气,知道天子和王安石依然有心大用于他。毕竟能终日紧随官家脚步,再不济都能混个脸熟。而若是如自己这般才学,那就是能让自己飞黄腾达的踏足云鹤了。

剩下的关键当就是曾布了。

当年王安石手下三大将,他吕吉甫回乡守制,章惇现今又出外,曾布一肩挑了七八个差遣。当今天子曾问王安石,曾布身上的差遣是不是多了点。王安石回道,能者多劳,曾布不会耽误公事。

现在吕惠卿回来了,便是一门心思,要从曾布手上抢下几个差遣来,回复他旧时的地位。只是他现在缺乏人脉,要跟曾布斗,实乃力所不及,且王安石也不会偏向任何一边。

自从回京后,吕惠卿已经想了好几日,新党中的成员这几天也见了不少,还当真给他找出了一个人来——新近出头的吕嘉问,因为对新法忠心耿耿,而备受王安石看重。且吕嘉问跟曾布不算和睦,应该是个能派得上用场的人选。

刚刚结束了随侍天子的工作,吕惠卿坐在崇文院的史馆厅中,依照定规,书写着天子今日的起居录。崇文院近着中书,甚至有一条近道联通两个公廨——毕竟宰相都要在崇文院中兼职,王安石本人就是昭文馆大学士。故而崇文院的小吏,往往是消息灵通程度,仅次于两府属吏的一帮人。

吕惠卿正在端端正正的写着起居录,本就是书法大家,一笔三馆楷书同样写得出类拔萃。只是快要收尾的时候,却听到外面突然变得有些乱,一帮小吏不知是在絮絮叨叨的传着什么小道消息。

放下笔,吕惠卿回头对随侍的胥吏道:“去问问出了何事?!”

小吏出去片刻,便回来了:“禀侍讲,是华州的急报!六天前的丙寅日,陕西地震,少华山崩,生民死伤无数,急求朝廷下令赈济救援!”

“……是吗?”吕惠卿不动声色,抓起笔重新面对桌上的卷册,头也不抬的说着:“我知道了,你且先下去吧。”

小吏依言出去了。

吕惠卿就手将笔一丢,一靠椅背,仰头看着比三年前又破败了一点的厅堂屋顶。他脸上的神色似喜非喜,似忧非忧,让人难以揣测他的心情。

只是听他喃喃念着:“这下可是有得麻烦了。”

……………………

华州位于潼关道上,境内的少华山、太华山,峰峦险秀,很有些名气。可今次的地震,让少华山上的一座山峰崩塌了下来。

在天灾都会算成人祸的这个时代,天子和宰相对于地震和山崩负有不可推卸的责任。

这个认识,在百姓心目中根深蒂固。而在士大夫中,有见识的儒者多有不信这套董仲舒编出来的天人合一之说的。但其中一些人因为所处的立场,却让他们拿起了这套趁手的工具,敲打着他们在朝堂上的敌人。

自从前日,少华山山崩的消息传到陇西,韩千六回来就念叨了几次,还问韩冈是不是王相公有什么不行德政的地方,然后让韩冈去了京城后要小心行事。

换作是马相公上来,也是一般……天地何预人事?!

但这话韩冈不能说出来。大部分的儒者其实心里也是透亮,但外面还是要装着去相信天人感应,否则就没有了能约束天子的有效工具。

现在用祖宗之法已经压不住皇帝了,若是有人跳出来说天地异变跟天子没关系,肯定会被群起而攻。如果事不关己,新党一侧其实也是会想着能有个钳制天子的工具。

‘天变不足畏,祖宗不足法,人言不足恤’,这其实是韩琦栽给王安石的罪名。后两条王安石很干脆的认了,也为此而辩解了一番。只有第一条,王安石不敢直接否定,而是曲言带过。

要想压制住天子,不靠天地,还能靠什么才能名正言顺?

但现在就有些麻烦了。韩冈最近也有听说市易法在京中推行困难,自河州大捷,王安石得赐玉带之后。新党的势力已经走过一个高峰,避免不了的要进入下行道走上一阵。今次的地震山崩,很有可能会起到推波助澜的作用。

不过这也不干他的事。短时间内,王安石的地位依然不可动摇。大宋地域广大,地震山崩乃是常事,隔个十年八年就有一次。更别提刚刚收复的洮州,前几天也是一场地震。如果不是有心人要搅混水,一般的灾异都不会影响到王安石这等根基深厚的宰相的地位。

天气一日日的转寒,也到了该上京的时候。冬月出发,在腊月初赶到京城,可以安稳的准备参加明年二月的礼部试。

当十月上旬,第一场雪在陇西落下的时候,周南、素心和云娘开始为韩冈收拾行装。衣服、药品、银钱,一样样的都是精挑细选的出来。在韩冈面前,三女都是笑着,尽力服侍着韩冈,但转过身,她们都会背着人抹着眼睛。

一夜缱绻之后。严素心只穿着小衣下床,修长笔直的双腿裸露在外。韩冈从身后看着,挪不开眼神。

素心从箱子拿了一套冬衣过来,厚实的棉布布料是新成立的织造工坊的杰作,里面则是填着丝棉。这件冬衣针脚细密得让人难以相信是出于人工。韩冈捏了捏她的小手,拉到了眼前。指尖上上面还有个针刺出来的红点,而中指处,还能看到长时间戴过顶针的痕迹。

“这些日子都是在缝着这套衣服?”韩冈这两天白天时都看到严素心打着哈欠,本以为是家里的大哥儿太吵,现在终于知道是为什么了。

素心将手抽回,催着韩冈:“官人先试试吧!”

佳人的一番心意如何能辜负,韩冈起身将这身衣袍给穿了起来,却是不宽不窄,不长不短,正正合身。也不仅仅是严素心,周南、云娘都给韩冈缝了一堆衣服。如果都要带上,那就要多带两匹马才够装。

看着韩冈的一身俐落的装束,严素心先是满意点头,但眼眶渐渐的就红了起来。

这毕竟是行装啊!

韩冈叹了一口气,将她拥在怀中,双臂之中的娇躯轻轻颤着,抽泣声低低的,却清晰可闻。

“不要哭了。考完之后,也许还能回来一趟。就算不成,也会尽早将你们都接过去。”

出发的日子很快就到了,十一月初八,是宜出行的好日子,只是天色微阴,看起来像是暮色提前降临。

韩冈带着两名伴当,在家人、朋友的送别下,离开了他战斗、生活和学习过的地方。

韩冈骑着马,已经远远的离开了饯行的十里亭,但他回过头去,却还能远远的看到仍留在原地的那群亲友。

重新正视前方,韩冈放下了心中五味杂成的感情,用力挥了一鞭。胯下的坐骑陡然加速,带着他向着更加起伏的前路,奔驰而去。

