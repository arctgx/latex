\section{第33章 枕惯蹄声梦不惊(五)}

当侦骑再三回报说周围辽军的探马已经全数消失之后,章楶终于放下了心来,辽军真的是撤了。

在战前,制置使司中军议时,对辽人的预测,其围城绝不会超过三天,孰料只过了一夜就走了。

一击不中,远彪千里,让人追之不及。这是辽军作战一向的特点。但放在眼下,还是让章楶喜出望外。这代表了辽军失去了在野冇战中大获全胜的信心,而选择了见好就收。没了锐气的契丹骑兵,也就不是那么值得畏惧了。

要知道,韩冈甚至事前还给了他三封提前写好并封装起来的军情公文,如果太谷县当真被辽军围得水泄不通,让他代为每天按时发回东京。现在可是全都作废了。

放出了大批游骑巡视周边,章楶率领着四千援军前部,用了一天的时间顺利的抵达了太谷县。后面的主力走得稍慢,还要一天的时间。

望着城墙外远没有收拾干净的战场,章楶暗自心惊。地上的箭矢密如野草,那一夜怕不射冇出了有几十万支之多。而城南那一片繁华之地所化成的废墟,也让他感慨不已。纵然事前已经知晓肯定保不住,但亲眼看见却又是另外一回事了。

章楶带着众将校在衙中拜见韩冈。

仅仅是数日不见,章楶却憔悴了很多,想来他在营中劳心劳力,决不轻松。而章楶看韩冈,虽然眼神锐利如昔,但眼窝也是陷了下去。

不过韩冈和章楶并没有寒暄感慨的余暇。下一步该怎么做?这是必须尽快议定的。

去太原。

这是理所当然的结论。

从主持军议的韩冈,到只有站在壁脚列席的一众指挥使、副,全都对此毫无异义。

辽军眼下绝不会仅仅撤到太原就停下来。只要稍有战略眼光就能看得出来,太原城下绝不是一个有利于辽军的决战地点。

从太谷之战已经可以看清楚韩冈用兵的方针,是以势压人,绝不会仓促急进。是那种吃口饭,都要用手巾擦上三下嘴的那种谨慎。只要短时间内没有攻下太原城的能力,逼得韩冈挥军急进,所谓围城打援就不可能成功。

既如此,萧十三又如何会在太原城下浪费时间?返回代州,并巩固代州才是改变现在不利局面的最好手段。

“要是辽贼能直接回大同去,倒是省了我们多少事!”军议上,有人调侃着。所有人都了解到了辽军的窘境,气氛也就变得轻松了不少。

“北虏会这么简单就放弃石岭关吗?会这么简单就让出代州吗?会这么简单就放弃雁门、西陉吗?”韩冈却无心言笑,他冲着一应将校属僚摇着头,“不可能的!若尔等是耶律乙辛,难道不会想拿着代州换回兴灵吗?只为了他自己,至少也回复到开战前的状态,顺便还能多添几分岁币!”

章楶点头,“想必耶律乙辛这时候连使节已经都派出来了,要逼朝廷就范!”

大雄宝殿中轻松的气氛顿时凝重了起来。

人人皆知,之前韩冈就在《御寇备要》中宣言过,绝不会让强盗顺顺当当的带了赃物回去。否则食髓知味,日后将会永无宁日。

韩冈是绝对不会答应耶律乙辛开出来的条件,既然如此,收复代州便势在必行。自然而然,援军当是也得继续北上。

但这也就意味着,自从入寇河东,辽军还没有真正意义上的失败过。没有攻下太谷县,仅仅是战略目标没有达成,兵力损失在总体中其实并不算大。只要那么多兵马还在,想要夺回代州,就少不了要与他们交手。要攻取险关名城,还要野冇战克敌。

对此有必胜信心的,并不算多。之前在太谷逼退辽军也只是守城而已,只看韩冈让援军慢如乌龟的行军,就是到制置使本人也不看好野冇战的结果。

“辽人虽然主力犹存,但现在各部战马能上阵的不会太多了。”黄裳开口说道,“可以算一下,这一个多月来,他们的战马究竟跑了多远。”

陈丰紧跟着:“两千里,只少不多。”

“现在才是初春。”田腴补充。

三名幕僚配合娴熟,这等于就是韩冈亲自说话。但他们并不是以势压人,就是一个从未涉及战略决策的指挥使也很明白他们想要说什么。

从入犯之后,辽军骑兵每日来回劫掠,战马不得停歇,之后还要驮着抢来的赃物。辽军的每一匹马,这一个月来,跑动的距离绝不会少于两千里,而且都是负重,且又是在刚刚经过了严冬的初春。正常的战马吃不住这样的劳苦。

这等于就是让一名饿了三个月的人,又背着三五十斤的重物每日来回跑,纵然能用干草粮食将肚皮填满,身体状况也好不了。

“辽贼南下,倒毙在路上马匹数目不算少。这两日枢密派出去的侦骑回报,在太谷县周边,至少已经发现了两百余匹战马的尸骨,这还不算被百姓发现,然后隐匿起来私分掉的,也没包括被辽贼自己吃掉的。"

“在辽军的营地里,发现了不少战马的碎骨残肢还有内脏,并没有完全被填埋起来。他们在吃死掉的战马!”

实际的证据比起空洞的推测有效得多。黄裳和田腴前后说完,气氛又缓和了不少。

“相对于总数,其实还是少的。”韩冈端着茶,做着总结,“但最重要的并不是在于战马怎么样?而是在于辽贼怎么看待他们的坐骑。”

“如果将战马当做消耗品,死了就丢,那辽贼当会不惜一切与我决战。若是当成自家物,损了伤了,可就会肉疼心疼啊。”

不会有人不清楚,就算是宫分和皮室这样如同禁军上四军的劲旅,也都是自备甲胄、战马和弓刀。国有征召,正兵便自备弓马甲兵应冇召而起。

战马都是自家的财产,而且是最为贵重的财产之一。死了一匹马,不仅仅家产的损失,还意味着驮送赃物的畜力又少了一匹。这个损失就大了。

萧十三之前能用太谷县中的财物,甚至中原、东京那样的画饼,来率领麾下诸军南下。可现在就没那么容易了。无功而返,加上不断看到周围战马大批伤亡,必然会使得一大批辽将选择更为保守的方案,而失去决战的意志。

“这番话要传达下去,让每一名士兵都明白,不要畏惧辽贼。因为他们没什么可怕的。一群强盗而已。”

韩冈看着指挥使们,中层军官是支撑一支军队的骨架,没有一群有能力有胆略的军校,就不会有一支强军。

将校们齐齐行礼,韩冈的吩咐就是军令。

“当然,能多削弱辽贼一分,对我们来说,胜利就会更轻易一点。之前我已经传令路中,命各地军民尽可能的拖延辽军行动的速度。一天不行,那就半天。半天不行,那就一个时辰。一个时辰不行,那就一刻钟。能拖延片刻,就拖延片刻。”

韩冈希望胜利的天平能多向自己这边偏移一点,即便只是一点点,或许到了最后,就是决定了胜负的关键。

休整了一夜,韩冈便领着一部兵马和他的制置使司先期赶往太原。章楶则留了下来,他要迎接后方的大军,然后安排他们继续前进。

军议时韩冈和他的幕府就判断过,萧十三到了这个节骨眼上,尽快退出太原,守住石岭、赤塘二关,依靠山河地势来保住代州,就是辽人最好的选择。

但韩冈并不是一厢情愿认为辽人该如何如何,依然很谨慎的在沿途放出了大批的侦骑,让自己手中为数不多的骑兵,去仿效辽军的远探拦子马,一部分继续去追踪辽军主力的动向,另一部分则搜检周边,以防萧十三留下些什么。

在这一过程中,得到了当地残存百姓的鼎力相助。有了谙熟地理的向导,能够藏兵的去处被一一查看。作为外来者,辽军几乎不可能有办法将大股的军力在某个隐蔽之处埋伏起来,等待着时机在背后捅上韩冈一刀。

而搜检的结果也证明了这一点。萧十三完全没有拖泥带水,辽军已经撤得干干净净。一路上被搜寻出来的,仅仅是加起来不过两百多骑的脱队者,然后被韩冈下令不论死活,一路吊在在道旁。

确定了道路的安全,北上太原的大军行动也就快了许多。

之前是稳,以防为辽人所乘。不过现在,辽军既然已经撤离,那就完全没必要再慢慢磨蹭。恢复正常的行军速度,甚至更快一点,自是理所当然。只是因为沿途的村庄被毁坏殆尽,困于食宿的问题,却也没办法以最快的速度强行军前进。

不过一路疾行,韩冈率军抵达太原的时间,却正正好卡在了他许诺的二十天之内。

言而有信,无过于此。

王.克臣投笔而叹:“子以四教:文、行、忠、信。为将五德:智、信、仁、勇、严。文武之道,皆在一个信字上啊。”

言罢,率满城官吏军民出城相迎。

韩冈却并不进城,而是就在城外安歇。

他前一日刚刚得到了韩信传回来的消息。忻州城依然在坚守中。而原本投敌的代州军,在秦怀信的儿子秦琬和韩信的策动下,已经全数反正,潜入了忻州的山中。

“辽贼有腹心疾,前后夹击,岂有不败之理?石岭、赤塘二关,已是官军掌中之物!”
