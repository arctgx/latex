\section{第11章 五月鸣蜩闻羌曲(四)}

【第二更,求红票,收藏。最近两天的红票很不给力啊,各位兄弟看完后能不能多投两票。】

“求援?”韩冈本是把仇一闻往门里请,听到这一句,动作便停了,奇道:“这秦凤路上谁还能给仇老你脸色看?”他在秦凤路上行医多少年了,救下的人命成千上万,任谁也得给他一点面子。

“韩官人你也太看得起老头子了。”仇一闻唉声叹气,“官人们要跟老头子过不去,老头子就要躲着走。老天爷要收人时,老头子的脸面也一样没处挂。

就像前些日子,老头子在夕阳镇上碰上个卖炭人家的女儿肚子大了起来,说是偷了人吧,可有了喜也不至于躺在床上不能动弹,而且才三四个月就大得跟十月怀胎的样子,实在不对劲,便把老头子请了去。

老头子过去把了脉,的确不是个喜脉,当是怀里生了痈,但看她肚子胀起来的样子,怎么施针下药,这肚里的痈都是消不下去了,也只能等死。韩官人你说说,这老天爷硬是要收人的,该是怎么个救法?”

‘开膛破肚,把瘤子给割出来。’韩冈一点后世的医学常识还是有的,不过肚中的瘤子长得这么快,多半还是恶性,即便在千年之后也不是那么容易能救回来。

不过韩冈也不能在仇一闻面前表现自己多有见识,立刻就说道:“仇老,小子的医术你也是知道的,当真是一窍不通。也就是在……”

“好了,好了,”仇一闻虽是求人,还是不改倚老卖老的脾气,打断了韩冈的推脱,“这事老头子也知道。韩官人你要藏着掩着,谁也没办法,你真的把人给救回来也就认了。”

韩冈摇头无奈的苦笑两声,看来仇老头是认定他身怀医术了。不过这也难怪,普通人对医道并不了解,所以韩冈的话还能蒙混过去。但仇一闻老于医药,当然知道韩冈主持的疗养院究竟有多难得,而他对于五行生克用于医道上的见识,又是如何发人深省,怎么可能是跟萍水相逢的一个普通道士聊了两天,就能学到的?

天气燥热,门边树上的知了大合唱也是让人听着头疼。站在门前说话的确不是礼节。韩冈请着仇老郎中进了待客的厅中,谦让两句各自坐下,又让人送了茶汤上来,他才重又问起,“既然仇老你不是来找小子教训医术上的事情,那究竟是为了什么事?”

“也就是月前的事,老夫的一个徒儿在秦州城里做着郎中,不合医死了一个两岁的小娃子——其实也不能算他医死,本就是病重。老夫的徒儿只是扎了两针,又开了个药方,到了第二天就没救了。现在那家人把老夫那徒儿送进了大狱里,说是要治他个庸医杀人的罪名。”

“这样就告了?”韩冈难以置信。

医生治死病人,尤其是幼儿,在此时根本算不得什么大事。连当今天子的子嗣都是生一个死一个,若是这样就要治御医的罪,太医局里就没活人了。韩冈眼前的这位老军医,他的医师生涯中,怕也是亲手给几十个小儿送过终。

所以韩冈听着有些糊涂,心里也是奇怪,“此事应该不大啊……难道是六七十岁才生的独苗?”

仇一闻摇头:“死得是个小幺儿,前面还有两个三四岁的哥哥。”

啪,韩冈一拍桌子,心头有些火气,“那还告个什么?!这等夹缠不清的人家,仇老你在秦州城里找个熟人说上两句公道话,也就过去了。世上有几家没夭折过小儿,天家都免不了的事。这都要递状子,日后谁敢做医生?”

“谁说不是呢……可老头子的脸面不够用哇。”仇一闻继续叹气,“老夫平日里从来不进官宦家的门,医的多是平头百姓和军汉,真要有事求人的时候,认识的几个军头,根本派不上用场。官人你是管勾路中伤病事,又跟着管蕃部的王机宜,说起来这事还真是非你不可。”

“……这又是从何说起?”韩冈更糊涂了,路中伤病事指的是军中伤病,勉强也可以附带上军中家属,但与平民无碍,而王韶的提举蕃部,与医药之事更是不搭界。

“病家身在军中,我那徒儿跟蕃人又有些瓜葛,这不是正好两边都对得上?”

这根本是强词夺理!韩冈都想掀桌子了,‘哪里对得上!?’

而且这仇一闻人老嘴碎,说了半天都是夹缠不清,说不到个点子上。韩冈深吸一口气,平了心头火气,“仇老,你还是把此事来龙去脉给小子从头到尾的分说一下,那样,小子才好知道该如何去做。”

“老夫方才也说了。就是秦州城里一家小儿病了将死,找了几个医师都不敢开药方,摇头就走了。最后找我家徒儿去治病。我那徒儿心肠软,虽然那小儿是没救了,可他想了半天,还是决定把死马当活马医,开了个偏方。只是他自不量力,到最后还是没能救回来。那苦主就恨起来了,揪着说我那徒弟是庸医杀人。”

庸医杀人的确是要治罪的。照书上方子开药,治死人还有个说道,但如果别出心裁,不依正方,添减药方中的君臣佐使,致人于死的,依着疏律,韩冈记得那是要徒两年半——也就是劳教两年半。

“哪是徒两年半!真要这么轻,老头子也不会来找韩官人你了。”仇一闻急了起来,雪白的胡须直颤着,“现在丧家是告我那徒儿是违方诈疗,诈取钱财!本是要以盗论,现在又死了人,论罪是要被绞的!”

“绞?!”

韩冈真的觉得这件事情有些奇怪了。违方诈疗骗取钱财和不依正方致人于死,都是疏律中的条款。但在唐律疏议中,这两条关于医生的条款,其实很少被使用。药医不死病,真的药石无用,家属一般也就认了,谁还会跟医生过不去。要是这件事传扬开去,以后也没哪个郎中敢去上他们家的门了。

该不会碰上了北宋版的医闹了吧?可如今的时代,普通人比后世仍可算得上是淳朴,由于极高的幼儿夭折率,也不可能有人会对夭折一个不是独苗的小儿就闹得天翻地覆。而就算病家闹上一通,也换不来多少赔偿,只会让其他医生对他们家望而却步。

“仇老,你应该还有话没说出来吧?”韩冈眼神一变,如刀一般刺着仇一闻。他可不信事情会有仇老头说的这么简单。

“唉……”仇一闻又长吁短叹了一阵,磨得韩冈快没有耐性了,他才把整件事的关键说了出来,“我那徒儿,不合是个党项人。”

“党项人!?”

仇一闻点点头,“就是党项人。”

一个党项人,在汉人的国家里治病救人,这是什么样的精神?韩冈没去想这个问题。但一个党项人把人治死了,病家又在军中,很可能跟西贼厮杀过不知多少次,他们看着死去的儿孙,会有些不好的联想,也是可能的。这只能算是仇一闻的徒弟运气不好,还有就是太多事。

不过话说回来,真要说起民族成分,大宋这边的党项族人其实为数不比西夏少到哪里去,忠心耿耿的也不少。河东有名的麟府折家,就是党项人出身,但他们家从宋初便归附,跟契丹、西夏打了不知多少年,是有名的将门世家。而近一点的镇戎曲家,也是有着党项血统。

据韩冈所知,在秦州城中的几个衙门里,也有不少党项人在做事,而缘边的寨堡,也颇有几个党项籍吐蕃籍的军头。关西一带蕃人部落数不胜数,人丁也不比汉人少到哪里,单是秦州就有大小部族数百,在边境军州中,看不到蕃人才是怪事。异族在秦州坐馆,其实也不能算出奇。

“光是为了个党项身份,就把人送进大狱,这实在有些过分。若是一切都如仇老你所说,我肯定会要为令徒分辩上几句。”韩冈摇摇头,以民族成份取人,却是把那些忠心于大宋的异族往外推,并不是件有长远眼光的作为。

仇一闻听着大喜而起,向着韩冈拱手深揖,“那老夫就为我那徒儿多谢韩官人了。”

韩冈连忙站起身,扶住他的双臂,拦住仇一闻的行礼,“仇老的礼小子可当不起。”

一番谦让之后,韩冈和仇一闻重新坐下来。

喝了两口茶,韩冈突然想起一事,仇一闻还没跟他说清楚过病家的身份呢。前面仇一闻说是病家是军中人,但以仇一闻在秦凤军中的人望,怎么还会有人跟他过不去?逼着仇老头子在大热天里,赶到古渭来找他韩冈?

韩冈越想越不对,这老头子是不是故意把我绕了进来?

他连忙问道:“仇老,不知今次究竟是哪一家这么跋扈?无论县里还是州里,都不会让他这么胡闹吧?”

仇一闻慢慢的抿了口茶水,然后轻描淡写的说着:“是窦副总管……”

仇一闻声音不大,韩冈一时没有听清,问道:“谁?”

仇老狐狸放下茶杯,抬头望着韩冈,说道:“是秦凤路上的窦副总管。”

“窦舜卿的孙子?!”

“重孙。”仇一闻为韩冈更正。

‘就当我没听到这回事吧!’韩冈心里想着,‘这开什么玩笑!’

