\section{第48章 斯人远去道且长(三)}

【第二更,求红票,收藏。】

提及太祖时官场上的种种轶事,虽然有很多文人私下里抱怨,实在是有辱斯文,韩冈却觉得很有趣。赵匡胤这位戎马生涯数十载,靠着黄袍加身得到皇位的太祖皇帝,是从背叛和战乱中的五代挣扎过来的,本就不可能对文酸看得很重,即便要靠他们治理国家,压制武将,也不会如太宗朝之后的这百年,士大夫拥有至高无上的地位。

当然,士大夫的这个‘至高无上’,只是个比喻,正如近日传扬开来的枢密使文彦博对天子说的那两句——为与士大夫治天下,非与百姓治天下——是跟着皇权来的。真正深入此时人心的至高无上,还是指的是能在御街上有一条专用道的那一位。

路面两百步宽的御街宽阔得像广场一般,但街道本身并非完整的一片。就在御街中央,是六十步宽、专供天子出行所用的御道。御道两侧各有一条水道与外围普通行人使用的道路分隔开来,将宽阔的御街分成了三部分,而为了防止行人不慎落水,在水道外侧,还有两条黑色木杈组成的栅栏,从皇城南门一直延伸到外城南门。

御道所处的位置,就像后世的高速路中心的安全岛。不过不是绿化带,而是给天子出城南郊祀用的。御道严禁闲杂人等踏足,但御街是东京城的中轴线,不可能让其将城东城西分割开来。故而每隔百步,以及与横街相交的路口处,遮拦中央御道的黑木杈都会空出一段,水道上也架起石板,以便让行人通过。

韩冈等人穿过御街后,重新翻身上马,继续向城西去。而他们身后,蔡京突然停住脚,惊讶的盯着韩冈他们远去的背影。

“三哥,怎么了?”在蔡京身边,与他并肩同行的一个年轻士子见蔡京停步,回过头来奇怪的问着,“出了什么事?”

蔡京的视线追逐着韩冈等人的背影,喃喃自语:“大概是看错了吧……”

“看错什么?”年轻士子更加疑惑的追问道。他长得与蔡京有几分相似,俊秀不输蔡京多少,看得出来他们有着很近的亲缘。其实他就是蔡京的兄弟蔡卞,表字元度【注1】。两名俊秀出众的年轻士子站在大街上,周围的女眷顿时就把眼神移了过来,或正大光明的盯着,或是暗中瞥眼过来,或明或暗的打量着两人。

蔡京回过神来,对着蔡卞笑道:“就是我在西太一宫曾经遇上的那两人。刚才骑马过去的几个人中间,有两人跟我当日看到的很像。”

蔡京这么一说,蔡卞登时恍然。一说起西太一宫中的两人,不会是别的,就是已经在京中传唱开的那首小令的作者和他的同伴。即使蔡卞当日没有参加那场聚会,也不会不知道。“究竟是不是他们?”他眼望着西面,问道。

去国子监认了下门,回来时,就与百寻不着的目标擦身而过,这世上哪会有这般巧的事?蔡京回头望望已经消失在人海中的身影,摇摇头:“说不清楚,可能真是认错了。七哥,还是回去了。今天养足精神,明天可就要上考场。这些事,等考完后再说不迟。”

……………………

周南呆呆的望着镜子,新磨的铜镜亮得眩眼,一张如花似玉的俏脸正映在铜镜中央。眉不描而翠,唇不点而红,两汪秋水能人把心都醉进去,白皙细嫩的脸颊上没有半点脂粉的痕迹,却是清丽无双。只是今天,月妒花惭的一张脸没了神采,怔怔地发着呆。

“周南,你真是太傻了,他到底有什么好……”周南对着镜子嘤嘤念着。自起床后,只梳洗了一下,就穿亵衣坐在镜前发怔,不停的自说自话,如同魔魇了一般。

周南一手托着下巴,看着镜中的自己入神。右手则紧紧的攥着,掌心中似乎还残留着昨夜感受到的温暖,让她舍不得放开。

没了外衣的掩饰,一层薄薄的白纱亵衣完全掩盖不住发育得过于出色的双峰,在胸口处被高高的撑了起来。纱衣通透,映出了下面的宝蓝色抹胸,而亵衣衣襟交接处,则露着一抹雪腻微光。

周南穿得如此单薄,尚幸房内火生得极旺,温暖如春,让她不虞被冻着。但服侍周南的小丫鬟在旁边不免要担心着,犹豫了半天,然后才轻声问着,“姐姐?要不要再加件衣服?”

周南什么都没听到。她自幼时起便入了教坊司中,被逼着学习琴棋书画,歌舞诗赋,到了十四岁开始行走于各家酒席宴会上,先是跟着艳名高炽的几个姐姐,后来便独自出来,这期间,她渐渐打响了声名,被称为花魁行首,多少人为她的一颦一笑而心旌动摇,也有假正经的,但他们总是在偷偷的看自己。就只有一个人,虽然他看着自己的歌舞,又跟自己谈笑,但实际上却是视若无睹,嘲讽起来又一点口德都没有。

周南突然又恨恨地咬起牙,因为韩冈在樊楼中的几句话,让她受了多少嘲笑。本想着要好好报复他一番,却没想到再见面时,他只是不经意的倒了一杯茶,就让自己的心都失落了。

“不过就是一杯茶啊……想请你喝杯茶的,京师里不知有多少,受宠若惊的该是他才对。”嫩如春葱的纤指轻轻点着镜子,周南对着镜中的自己细声的说着话。

这两年她见过不少达官贵人,也有过宿儒名士要她陪酒,但他们在自己面前,就像传说中的孔雀,尽力表现自己的才学,但有几人是真正关心的看过自己一眼?有几人会想着喝酒伤身,而为自己倒上一杯热茶?他们总恨不得将自己灌醉灌倒,好一逞他们令人作呕的欲望。

只是……他究竟是因为自己而温柔,还是举手之劳的习惯?

周南突然间想哭,没想到喜欢一个人的感觉是这么难过。而且他今天就要走了,再到京城时,又不知是何年何月,也许那时,自己已经不在东京也说不定。

对了,一定是要去送他,不然一别之后,他又怎会记得一个只见过区区两面的歌妓?!

周南一下站了起来,丰盈的胸口一阵让人口干舌燥的轻颤。猛然间的动作,晃掉了她束发的金钗,满头青丝如瀑布般披散了下来,顺滑得一如最上等的绢绸。

只是一转身,周南突然又犹豫起来。才见过了两次就巴巴的赶去送行,会不会让他认为自己轻浮?她的心一下抽紧,突然间痛得厉害,血色自脸上褪去,双唇都白了。

‘才两面而已,怎么就会喜欢上那个冤家?!’

“墨文,你去……把这手帕……不,去让人备车。快!”周南心意一变再变,但最后,还是忍受不住噬心的相思,要见上那冤家一面。小丫鬟答应了就匆匆忙忙跑了出去。

但墨文刚下楼,周南忽尔又惊叫了起来,光着一双脚在闺房中团团转着,她现在才发现,自己头发完全散了,衣服也还没换,而服侍她的墨文却已经跑出去了。

光洁如玉的一对小巧天足慌乱的踏着从关西羌人那里贩来的羊毛地毡,只听着歌舞双绝的花魁在慌慌张张的念着:“怎么办?怎么办?”

……………………

新郑门的三重城楼在眼中越来越大,周围的车马行人也越发得多了起来。进城的,出城的,在城门前都免不了要堵上片刻,这里总是最为拥挤的地方。

刘仲武没有骑着他的赤骝,虽然他的这匹爱马的脚伤已经好得差不多,但他还是舍不得再骑上去。最重要的,刘仲武现在已经是名官人,本官品级比韩冈还要高一级的三班奉职、秦州边境者达堡的堡主,已经有资格用一下驿马了。

骑着一匹毛色有些发灰的骟马,带回秦州的土产由身后的赤骝驮着,刘仲武在马鞍上坐得笔直。也不左顾右盼,下巴扬起,眼睛直视前方。表情上看不出什么异样,但春风得意四个字从他的姿态中透了出来,看起来就像一个跨马游街的进士。

突然间,他‘咦’了一声,抬手指着前面。韩冈顺势望过去,只见一个老者正带着几个仆从守在城门前,却是章俞在那里候着。

刘仲武立刻拍马上前,韩冈向两位师长告过罪后也跟了上去,两人在章俞面前下马,韩冈便问道:“怎么敢劳动章四丈为晚生来送行?”

章俞故作不快:“玉昆你这是说的见外话了。我们交情是极好的,怎么能来不送上一送。”

张戬和程颢这时也骑马赶了上来,先看了看章俞,便向韩冈道:“玉昆,不向我们介绍一下?”

“啊!”韩冈连忙为两位师长介绍起章俞,“这位就是学生曾经向两位先生提起过章四丈。”

“章……章!?”

注1:关于蔡卞中进士的年龄有两种说法,一说他是二十三岁中进士,一说十三岁。不过第二种说法有着明显的错误。

第一,在蔡卞的宋史本传中,根本没提到他十三岁中进士的事。司马光七岁砸缸的事在他的本传中都有记载,蔡卞才十三就中进士难道还比不上砸口缸不成?在北宋,中进士是士人最大的荣耀,而十三岁中进士,不入本传是不可能的。

第二,蔡卞的侄子、蔡京之子蔡條,在他写的《铁围山丛谈》中,提到蔡卞不少次,却并没有说起蔡卞十三岁中进士。

第三,前文中也说过,在北宋官员得差遣是有年龄限制的,荫补等无出身的官员要到二十五岁,而进士等有出身的官员也要到二十岁,但蔡卞是中了进士后便担任了江阴主簿,很明显不可能才十三。

