\section{第七章 烟霞随步正登览(十)}

都大提举市易司。

从王居卿身上的官职就能看得出来,这是铁杆的新党,而且是新党之中擅长财计的成员。

王安石提议朝廷设立开封市易司,是熙宁六年旧党第三次反扑的起因,也是当年曾布叛离新党的导火索。当年差点让新法就此折戟沉沙。

自吕嘉问之后,这些年来但凡能坐上这个位置的,无一不是新党的骨干。王居卿当然也不例外。不过他是以侍制之位就任市易司,属于高职低配,故而不得不加上都大提举的前缀。

不论王居卿的选票投给吕嘉问、李定和曾孝宽中的哪一位,最后的结果都是韩冈居末,与人并列。

那时候,韩冈就要在两难之中作出决定,是放弃推举之法,交给太后决定一切,还是点头答应由重臣们重新推举出第三名来。

蒲宗孟相信,这个主意不止他一人能想得到。毕竟宰辅推举之法的根本,就是齐云、赛马两家的选举法。在听说了朝廷将要开始以推举之法选择宰辅人选,很多人都打听过了两家总社选举会首究竟是什么样的一个流程。从两家总社的选举法推导出来,让韩冈自食其果,这并不需要费多少心力去谋划。

只要最后结果是新党当选,具体的人选不论为谁,除了当事人和不多的几个推举者之外,其他的新党成员都不是很在意。只要能让韩冈丢人现眼、无颜就任就可以了。

蒲宗孟从韩冈、吕嘉问、曾孝宽这么一个个看来,连他所推举的李定也没落下——都选不上才好。

但每一位新党成员都不会希望韩冈能够上位,尤其是在范纯仁、孙觉、李常这样的旧党支持下上位。

朝堂上的空缺就这么多,要是韩冈上位,肯定要酬谢他的支持者,就像原本被举荐者对举荐者感恩戴德一般。那时候,不知会有多少职位被旧党瓜分走。

韩冈到底有多得圣眷,每一个人都看得很清楚。他的提议,恐怕太后都不会拒绝。

有人会将自己的命运放在韩冈的正直无私上吗?认为韩冈会大公无私的保留新党的职位,让他的支持者们希望落空?

蒲宗孟不知道别人怎么想,反正他是不愿意韩冈能够被选上。因为他明白这一次推举,就是韩冈在举旗招兵,而且事实已经证明他的做法十分有效。

有着这样的目的,又已经有了初步的成果,韩冈又如何会亏待投效自己的朝臣?千金市马骨的故事,六七岁的小皇帝都会耳熟能详。

不过有一点让蒲宗孟很担心,这让他的两只眼睛一眨不眨的盯着正拿起最后一份推举折子的王中正。

原因不在韩冈身上,而是这一票的归属者王居卿。

王居卿不是个多有名气的官员,可以说是默默无闻的就一点点爬了上来。但这位肯定是有才干,否则不至于没有多少名声,却能被安排在如此重要的职位上,还能被授予一个侍制。

也许别人不记得王居卿的出身,但在学士院多年的蒲宗孟却记得很清楚。

并非是因为他曾搜集过所有选举人的资料,他的记性一直都在下降,就算昨天看过了,今天也不一定能回忆起来。不过亲笔起草授予王居卿天章阁侍制一职的诏书,蒲宗孟不可能会忘记,在诏书上耗费的心力让他记得十分牢固。

王居卿的出身是登州蓬莱!

但凡知道王居卿的祖籍,又想到韩宗道和李承之在前的先例,不少人都开始担心起来。

新党的身份,北人的出身,这让王居卿的立场变得模糊不清起来。

应该不至于。

章敦想着。

虽说南北分立,新旧党争,但并不可能截然两分。北人可以投向新党,南人也可以投向旧党。

韩绛是标准的北人,但他就是因为支持新法,而在相位上一坐多年。王居卿虽是北人,他可也是铁杆的新党——因为旧党不要他。

王居卿虽然才干卓异,而且也是进士出身,但没有一个显赫的家世,又没有结交过权臣,也不会以诗文会友,不论他在地方上做出多少成就,在变法之前,一直都被把持朝堂的旧党视为俗吏,可用而不可重用。

司马光可以光说话不做事,因为他是儒臣。而王居卿为盐铁判官能做到‘公私便之’,做京东转运,能‘人颂其智’,在整治河防时,能让朝廷以他的规划为后世法度,但在旧党儒臣们眼中,他终究不过是一个言利至从官的俗吏。

韩冈招聚旧党,可谓是神来之笔。但他也因此会失去许多新党中人的人心。像王居卿这样的官吏,只有在重视才干的新党之中才有他们发挥的余地。

而且就他所知,王居卿的这一票早早的就为吕嘉问所预定,这将是吕嘉问的第八票。

章敦等待着王中正宣布这最后的一票。

熟练的打开选票折子,王中正先于任何人看到了王居卿的选择,微笑一现即收,

“天章阁侍制,户部郎中,都大提举市易司王居卿——”他略略拖长了声调,迎上数百道迫不及待的目光,然后清晰的吐出了两个字:“韩冈!”

相对于之前的骚然,现在的文德殿上一片寂静。

事前已经与吕嘉问约定好的王居卿,背弃了诺言,将他的一票转投向了韩冈。

第一次受到如此多的关注,王居卿抬着头,直直望着书写着姓名和选票的屏风。那名小黄门正提着笔,在韩冈的名下,补上最后一笔。

第七票。

至此尘埃落定。

王中正放下了手上的折子,转身回来向太后缴旨。

“韩卿七票,吕卿七票,曾卿和李卿各为六票,加上韩卿的一票弃票,总计二十七票。”向太后现在的语气听起来轻松了很多,“韩卿和吕卿并列头名无异议,同应入选。可曾卿和李卿也同样并列,平章,你说这第三人该如何选出?”

“请太后决断。”王安石简短的回了一句,就闭上了嘴。

向太后皱了皱眉,觉得王安石的回答有问题,又问韩绛:“韩相公?”

“以臣之见,可重新再行推举,让方才诸位从曾孝宽和李定中推举一人来。”

向太后想了想,点头道:“……如此甚好。”

“陛下!”曾孝宽却叫了起来,“无须再选,臣曾孝宽情愿退出!”

“曾卿可是确定?”

曾孝宽行了一礼,“臣自知声望才干不如李中丞,情愿退出让贤。”

“是吗……”向太后正待应允,李定那边又叫了起来。

李定岂能让曾孝宽博一个不爱官的好名声,同样道:“臣于豁达上不如曾孝宽,人望也不如吕嘉问,功绩才干上更不如韩冈,方才群臣推举也已有结果,臣既然居于末位,也不能再厚颜求取西府之职,臣愿退出。”

“都要退出?吕卿不会也要退吧?”向太后没有挽留的意思,转头却去问吕嘉问。

“臣请太后决断。”

“韩卿?”

“此事乃臣提议,臣岂会退?请太后决定。”

韩冈轻松的说着。

自始至终,韩冈都没有太担心,就算这一回败选,也还有下一回的参知政事。

只要太后看清楚了新党党同伐异,南人把持朝纲的现状,必然会对朝堂设法加以改变。再加上在北人中煽动起同仇敌忾的氛围,下一回参知政事的选举,他就绝不会输。

韩冈可以通过请求,请太后做一些安排,但终归不如让太后主动去感受,然后依靠自己做出决定。而今天的一幕幕,韩冈相信太后已经有所警惕了。从她的询问中,就能看得出一点来。

“好吧。”向太后点头,“曾卿,你是玉堂之首,由你来草诏。”

曾孝宽楞了一下,然后应声站了出来。

他是翰林学士承旨、知制诰,刚刚退出选举,就被拉来草诏,说起来也太过讽刺了。

不过太后这一回没有去内东门小殿宣翰林学士上殿草诏,而是在文德殿上直接开始宣布名单。也许这将从此成为定制——两府的成员,将会在文德殿上选出候选者,接下来由太后或天子来决定最后的入选者,并在殿上当庭草诏、宣诏。

曾孝宽提笔等候,群臣屏息聆听。

太后开口了,却不止一条,“资政殿学士韩冈,可枢密副使!”

“枢密副使韩冈,可参知政事!”

两府是一个门槛,跨过门槛之后,如何选择就是在太后身上。总不能宰相都由群臣推举出来。所以只有参知政事和枢密副使可以由选举产生,而宰相与枢密使,则是从两府之中现有的成员,以及过去曾经担任过两府之职的前任宰辅们来就任。且两府内部的流动,也由太后所掌握。

但这样的变化,实在是让臣子们始料未及。

韩冈入西府为副使,这是事先确定的,因为今天只是在选举枢密副使。等就任枢密副使后,再调去东府为参政,没人能够指责。只是未免儿戏了一点。

可太后不等群臣回过神来,紧接着又道,

“西府现在又有缺了,半个月后的大朝会,还得再来选一次。”

