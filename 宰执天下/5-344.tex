\section{第33章 枕惯蹄声梦不惊(一)}

夜幕终于降临。

天上的星月被厚重的云层所遮挡,缺乏足够光亮的夜晚是伸手不见五指的一团浓黑。

黑暗中,乌鲁正舔着干燥的嘴唇,半蹲半跪的蜷着身子,望着不远处的城墙。

在他的身后,是一片乌压压的人影,那是三百名与乌鲁血脉相连的兄弟子侄。而在更远处的黑暗中,还有更多的契丹儿郎潜伏于地,正暗暗蓄力,等待着最后的命令。

将伴随他们同时抵达城墙下方的,还有五架长梯。那是随军的工匠赶了一夜共之后得到了的军器。乌鲁得到了其中的一小部分。

从设为营地的村庄潜来攻城前最后的候命地,乌鲁并没有受到任何阻碍。

遭受了十倍以上的大军围困,城中的守将、城中的士兵都不敢出动,就算是韩冈本人在城中,也决然不敢驱动手下的将兵打开城门。

离开城墙还有百步,在烈风劲吹的夜晚,这个距离上,并不用担心城头上的神臂弓。六寸长的木羽短矢在近处杀伤力惊人,但距离一远,风都能吹跑。只有冲到了城墙脚下,才会需要担心来自头顶上的劲矢。而从此处跑到城下,只需要几个呼吸的时间。

之前出发时,乌鲁底下的兄弟儿郎多有抱怨,要是有屋子挡箭,也不用在田地里战战兢兢的慢慢磨蹭。

但现在,乌鲁只想感谢。

感谢天时,感谢地利,也感谢即便心怀怨艾也能耐着心思等待号令的儿郎,这让乌鲁对今夜的进攻拥有更多的信心。

不过他们还在等待,除非来自后方的号令开始响起,否则他们就可以一直潜伏下去,直到夜幕消退。

在等待中,一记战鼓陡然拔起,敲动无数人的心。鼓声在营地中传递,就像点着了烟花爆竹的引线,立刻就在南北两侧城门前,惊起了一片狂潮。

数以百计的士兵在官道两侧疾行狂奔,并通过房屋与铺面,躲避城中的观察。

‘不要跑得太快,也不要跑得慢了,攻入城中之后先去找富户,衙门中的一切都可以交给枢密和他的中军。’

这是乌鲁出发前得到的叮嘱。现在看来,的确是最好的方案。他只求实利,至于户口籍簿那些玩意儿,交给更有责任心的人管好了。

不过宋人的反击立刻就到了。

一簇簇从空中降下的火焰,落到了城外鳞次栉比的房屋中,星星火光立刻便划破了城下的黑暗。

不知是风带起了火势,还是火助长了狂风,星星点点的火焰在须臾间便扩散了开来,房屋、商铺也一间间的被火海吞入。跳动着的光芒染红了半幅天空,吸引了无数人的视线。

一串短促的号角声开始呼叫,听在耳中之后,许许多多的辽国士兵直起身来开始冲锋。他们只知道沉默的冲锋上前。没有吼叫,没有狂呼,人人口中都含了一枚钱币,让他们在冲锋的时候不会发出半声呐喊。

乌鲁领头前冲,在他的背后是三百同族,他们拖着五架云梯,准备在城下给竖起来。只要能够成功,眼下还是神出鬼没的电影院,必然会将他们的丰功伟绩一点点的给挖掘出来。

七十步,五十步,四十步,三十五,乌鲁疾步狂奔。

城墙的黑影在视野中占据了越来越多的地盘,只要再有两三次呼吸,就能如事前计划一般的攻到城下,但这时砰的一声巨响从空中传来。

一团烟花在太谷城上的高空中炸开,艳色的礼花绽放于天际,成为天空中最为耀眼的存在。

城头上丢下了一团团用稻草扎起来的草球,轻飘飘的没有伤到任何人。但来自于草球上的浓烈气味还是让乌鲁鼻子猛地一抽。

“是油!”乌鲁一声惨叫。

话声没落,一团团草球就像是灯火一样齐齐亮起。城下闪耀的火光,将所有来袭的敌人从夜色中割离出来。城头上的箭矢便立刻有了准头。

箭矢如雨,但远比细密的雨丝更加危险,在火光和黑暗交错的地方,越来越多的被压抑的惨叫声出现在城下。

城墙上每隔一段距离就有一个突出于外墙的台基,也即是所谓的马面,而敌台就建在马面之上。相邻的敌台可以相互支援,直接从侧面射击城墙脚下的敌军。

韩冈在后世自然去过几百年后才修建的长城,同样是敌台,同样是每隔一段距离就有一座,不过长城上的敌台是与城墙一体的砖石建筑,而此时修在马面之上的敌台,却与城门上的谯楼一样,都是木结构的建筑。

而现在,这些敌台正盯着辽军的一举一动,并从箭孔中射出一支支锐利如电的箭矢。

韩冈在进入太谷城后的这段时间,他所着手的工作除了在战略上的布置以外,还包括了太谷县城的防御安排。

最为明显的就是城头上的变化,将之前只剩基座的敌台重新搭建了起来,虽不高,但一座座箭屋也让原本光秃秃的城墙变得爪牙锋利起来。

不过这一些,并不是韩冈的主意,而是先有人提出议案,然后经过商议讨论、补全细节之后,韩冈再加以批准。

如何以现有的人力物力财力,万无一失的守住太谷城,这是韩冈提出了要求。具体的方案是交给幕府成员来完成,为了在他面前表现自己,人人唯恐有所疏失,同时在韩冈有意无意的操纵下,他们也开始不顾人缘关系,去挑对手提案的错处。经过了这么样一番折腾,一系列的守城方案就变得严谨周密且具有极高的可行性。

来自于城上的箭矢越来越密,仿佛一阵阵伴随着狂风的暴雨。任何人都无法在箭矢风暴中逃离或穿过,但还是有越来越多的市民聚集在城外。

“射得好!”

城上兴奋的叫声引动了城下的欢呼。城上是一批批穿着铠甲、手持重弩,并向着敌军射击的士兵。在城下,则是一队队正当盛年的平民百姓,被聚拢在一起,每隔一段就有这样的一片百姓聚集的场所。而在北门附近,甚至还有一群和尚,闪亮亮的光头反射着场地中的火光。

在瞬息间的欢呼之后,包括和尚在内的平民,他终于了解到了工作的辛苦。

给一具具使用过的神臂弓上好弦,然后集中起来,用吊篮吊上城去,再由专人分发给城上的士兵,并收回已经射击过的重弩,然后用吊篮送下城去。

每一个人需要做的都很简单,专人负责上弦,专人负责射击,专人负责递送运输,再由专人负责统领和监察,事先练习过几日后,眼前的一切都是有条不紊,如同流水一般顺畅。

“照俺说,还不如把那些贼秃也拉到城上去,射杀了辽贼后,顺口一句阿弥陀佛就超度了,多省事?!还省得日后做水陆道场了。”

“契丹人有几个能得人以佛经一段来送行的?能得高僧大德念声佛,九泉之下也能瞑目啊。”

城头上,一群武官笑得肆无忌惮,从他们的笑声中甚至能感觉得出来,在这生死攸关的守城战中,他们都有着充分的信心。

军官们大多数都经历过战争,但他们从没打过这么轻松的会战。连上弦都不必自己动手,只需瞄准敌人,扣动牙发,并不需要消耗什么的气力,反倒是身上的甲胄更会累着人一点。

这样的守城战,又有什么艰难的?

任凭辽人狡计千万,在高墙深垒、连绵箭雨面前,还是要靠实力来扛过去,可他们过得来吗?

城头下的阴影里,悄然巡视至此的韩冈一行听到了头上传下来的笑语,虽然负责北壁守备的将校脸色难看,但随行韩冈的幕僚却相视而笑。军心士气如此,守不住城就是笑话了。

“那些和尚虽然都是该戒的不戒,但杀生戒都还是不敢妄破。真上了战阵,也就是平民百姓一般。”韩冈轻声叹,像陕西缘边弓箭手那样能与禁军相提并论的乡兵,在内地是不用指望能见到多少的。河北那边都悬。燕赵之民私下里好勇斗狠是不假,但勇于私斗、怯于公战的人实在太多太多。

黄裳也道:“上阵临敌真的不是那么简单。可以用他们的力气,别指望他们的胆量。”

韩冈见多了初次上阵的新兵是什么模样。在城头上,只要一支无意中飞上来的流箭,就能让一群新兵趴在地上。这样的新人,就算十个八个,也远远比不上一名有经验的老兵管用。但只要在城下给弓弩上弦,就算是从来没杀过人的一群平民,却也是很简单了。有着城墙的保护,不用担惊受怕,只需专心于神臂弓和上弦器。

北门下的一群和尚,平日一个个有钱有闲,拿香客信徒的香油钱养得白白胖胖,虽比不得东京的和尚敢挟妓招摇过市,但带着假发逛窑子,顺便勾搭良家女子,无论是军官还是士卒都实在见得太多。不过现在一个个光着膀子,满头油汗的给神臂弓上弦,倒也不是那么碍眼了。

之前的数日,制置使司发出军令,调集城中壮丁练习如何使用上弦器。虽然曾经递上政事堂上的畜力上弦机只有三架,但有把子力气的精壮汉子,四里八乡的乡民都逃入了城中的太谷县城内,却绝对不会缺少。

“早就说了,这些贼秃就是闲得慌。就该让他们累一点,省得总是动歪心思。”

韩冈的幕僚们大多都受到了他的影响,对于佛门的看法,对僧人的观点,与他别无二致。不交税,不纳粮,还要从百姓那里收取供奉,除了少部分人以外,整个僧人阶层对国家并无大用

在战阵上,不是随便哪个人就能适应那样的氛围。拿得稳刀枪已经是凤毛麟角,能学会合理的分配体力,不在一开始就把体力耗尽。

不过当原本由一人来完成的工作被分解开来之后,一切便再也不需要担心。
