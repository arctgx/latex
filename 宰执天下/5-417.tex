\section{第36章 沧浪歌罢濯尘缨(19)}

弓臂近六尺,弩身几乎跟人等高。

如此巨大的重型弩弓,硬是把身高也有六尺、比寻常人要高出一头的韩冈,映得矮了三五分去。

说是床子弩小了。说是单兵弩则又大了。

但是给人拿在手中,谁能说这是床子弩?只是这样的重弩拉开来,怕不要八九石近十石的力道。射击出去的箭矢,估计能把穿了铁甲的士兵给串成肉串。

韩中信乍看到这般充满了暴力气息的重型单兵弩,眼睛都不由直了起来,一时忘了要跟韩冈回报他的差事。

“这是克敌弓?!”

军器监中正在制造比神臂弓和破甲弩威力更强的新兵器的消息,已经在军中传了很有些曰子了。连名字都传了出来。

自从确认了辽国的正军几乎换装了铁甲之后,许多人都在期待新弩弓的出现,能重新确立在远程攻击上宋军对辽军的优势。

克敌弓正是其中备受期待的一种。

“不是,克敌弓只是神臂弓放大了尺寸。这个是手射弩。”韩冈将这手射弩递给韩中信,拿着看了半天,手也酸了,“是军器监的新玩意儿,方才才送过来。要是张孝杰不走,还准备在他面前试上一试。”

“不对吧。《武经总要》里面的手射弩明明是床子弩。手射合蝉弩,手射斗子弩……”韩中信愣愣的接过这具看起来威力就极为恐怖的重弩,手一沉,差点就没脱手掉了。

“全名就是手射床子弩。”韩冈摊了摊手,笑了起来,“军器监就是这样,起名直白得很,一点都不动脑筋的。”

其实他也一样。板甲这个名字,真的一点脑筋都没有动,都没认真想过一个威武雄壮点的名号。

军器监的起名天分,不是韩中信要考虑的问题,他从韩冈手中接过了这具手射床子弩后,就在估计这东西的重量,“怎么这般重,怕不有二三十斤吧。”

“二十二斤,还不算托架。”

韩中信顺着韩冈的视线移过去,就在一边,有根扎在地上的双头铁叉。叉上的双尖中央,正好可以架着一具手射床子弩。

用托架的重弩,韩中信还真没见过,“加上托架呢?”

“加上这架子,该有三十斤了。”

“都快跟半身的骑甲一样重了!”韩中信又看着这手射床子弩,掂了掂手,感觉真个跟盔甲拿在手上没多少差别。

“不带头盔和手脚上的配件,那真的就差不多了。”韩冈点头道。

他的力气就是在军队中也是很不错的水准,但韩冈拿着这张重弩,就知道凭着自己的臂力,不可能稳稳地站立着持弩射击。二十二斤的总重量不用托架支撑,很难稳定射击,力道再强,不能瞄准也是无用。

可如若算上加装的托架的份量,基本上就是又一副不加配件的半身板甲。而且就算有托架,拿上拿下也吃力,军中没几个士兵能用得好这种缩小版的床子弩。

“守德,你觉得怎么样?”韩冈问着韩中信。

韩中信皱着眉头。从初见时的惊讶,到仔细审视之后,他的神色也变得失望了起来,重量远不如神臂弓轻便,自是难以在行军时携带,这纯粹是防御姓的武器:

“明明是床子弩,两根弓臂!军器监改个名字就能拿在手上用了?怎么上弦?难道只能用上弦机?”

在旁听着两人对话的。

“的确只能用上弦机,人力是没指望的。”韩冈笑了笑。

韩中信闻言叹了一声。

这具弩弓一看就知道,必须要由机械上弦,人力的话少说也要三五人合力,但这张弩比床子弩小的多了,人多了,连搭手的地方都没有。不比放在地上的床子弩,有绞盘绞索,还有坚固沉重的地盘。

“力道肯定极强。但他背在身上的有三十斤啊。野战时没法儿带。倒是在城墙上守城时,可以把弩弓架在雉堞上,还能丢掉累赘的托架。”“不过还是得先试试看,试射之后,末将才敢评说。”

关闭<广告>

“这是当然的。”

韩冈将手射床子弩又从韩中信手中接过来,递给了一边的亲兵,让他去给手射床子弩上弦。校场边的畜力上弦机嘎嘎响了一阵,出来后在韩冈的示意下,交到了另外一名粗壮勇悍的军官手中。

韩中信认识那名军官,七尺一寸的身高,无论在哪里都会十分显眼。而且他还立下了不小的功劳,斩杀的辽军将领有好几个,在战场上更是因为硕大无匹的体格,帮袍泽们吸引了多少箭矢。

床子弩形制的重弩,用的自然是一尺多长的铁羽箭,而不是之前的木羽短矢。

上了弦,上了箭。

在身高体健的壮汉手中,重弩没有用托架支撑,便稳稳的瞄准了五十步外人形架子上的铁甲。

咚的一声响,弯曲的弓弦一下抻直,巨大的后座力使得人猛地向后一仰,弓也扬了起来,但箭矢早已飞窜而出。

根本就不是双眼能够追上的速度,下一瞬间,铁羽箭出现在了七十步外,夺的一声斜斜的扎进了地里。

射空了?

校场边围观的官兵一阵叹息。

才七十步的射程。

白白用了那么大的架势。

“不对,中了!”

韩中信眼睛尖,大步上前,走到盔甲前,肩窝偏下一点的地方一个极为显眼的黑窟窿,竟是一击洞穿了甲胄。

箭矢命中的位置位于胸甲的边缘,接近肩甲。厚度比当胸处要略少一点,而且在架子上没挂好,使得背甲和肩甲之间有着缝隙,铁羽箭从前面穿出后,直接从缝隙中钻了过去。

这算不上是成功的试射,不过在五十步外还要追求命中胸前要害的准确度,那未免是强人所难了。而且已经足以看出这手射床子弩的威力了。

虽然不是将背甲一并贯穿,如果是人穿上铁甲的话,只要洞穿一层铁甲也就够了。五十步外,彻底穿透铁甲,又飞出二十步才落地,如此力道,一击把五十步外穿了铁甲的士兵射成肉串当是不在话。威力完全超越了神臂弓,破甲弩也难以与之匹敌。

一群军官随着韩冈来到靶前,摸着胸甲上的洞口赞叹不已。

“怎么样?能派上用场吧。”韩冈问着韩中信。

韩中信犹豫了一下。在他看来,这种东西根本没有大批量装备部队的价值。重量太大,不方便携带,类似于床子弩,至于威力,其实把敌军放近了,神臂弓也能射穿板甲。

这就属于那种两边不靠的兵器。绝不可能像神臂弓、板甲那样人人装备。但用对了地方,还是很有威力的。

“自然是没问题。没有派不上用场的兵器。只要不是粗制滥造,用对地方都能克敌制胜。何况威力如此的神兵利器。”

“质量不用担心,军器监这两年还算是会办事,也算用心在做事。”

经过吕惠卿和韩冈两任判军器监重点整顿的物勒工名制度,使得军器监产品的质量检测,虽然没有他们主持监中事务时的严谨,但这些年也稳定在相当程度的水平线上。并没有出现甲胄一戳就破,战弓一拉就断的情况。

“枢密说的是,自从朝廷同意铁器入民间后,农具都是军器监的最好。”

韩中信越来越会说话了,韩冈有些不太满意。停了一下,接过茶盏喝了口茶,方才又道:“守德,如果是你的话,你怎么用这个手射床子弩?”

韩冈喜欢出题,时不时的就是一个问题丢出来。跟在他身边的幕僚和亲信都习惯了。很多人的能力也是这样锻炼出来的。

韩中信想了一下:“守城时用最好。神臂弓力道不足,八牛弩及远不及近,这个手射床子弩最合适。再厚的盾也没用。”

韩冈摇摇头:“等平安队夺了头名,都不见得能看到辽贼硬攻城池。”

平安那是一支老牌的队伍,后台是车马行,所以起名平安。开封蹴鞠联赛的创始球队,但总是在降级区沉浮,好些年下来了,都没人指望平安队能拿下一个第一来。

“野战也可以。像选锋一样,一个将挑选出一个指挥弩手来,专一射击辽贼的具装甲骑。具装甲骑全都是合在一起冲阵,抓住机会,可以轻易地将一支具装甲骑给歼灭。”韩中信眼中闪烁着自信的光芒,“枢密,这样的床子弩只要数量足够多了,上弦机能跟得上。辽贼的宫分军来了都没用。”

只要数量足够多……韩冈想着韩中信的用词,真的是会说话了。

手射床子弩的问题正是产量不足。

要把一件武器缩小,并不是依照图纸把零件尺寸按比例缩小就行了,床子弩做得大了反而容易打造,这般精巧的床子弩,对材料质量和工艺精度的要求高了不止一个等级。制造时间上,要比神臂弓长了太多。

工时和材料消耗太大,制造难度也高,制造速度提高不起来,在成本上并不合算,但军器监还是献宝一般拿了出来。

很早以前,韩冈就有了这个感觉,现在就更严重了——军器监中缺乏一个具有开创姓头脑的大师级人物,现在全都是匠人,只敢改进,而不敢扬弃。
