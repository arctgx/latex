\section{第38章 何与君王分重轻(13)}

任谁都知道,韩冈的第一次课,不是教太子读书,而是给皇帝和皇后看的。可是谁也不曾预料到韩冈竟然会上了这么一堂课。

蔡卞皱着眉头,盯着桌上的教学记录。

国子监与资善堂紧密相连,好几个讲读官都在资善堂兼了一份差事,蔡卞也是其中之一。

虽然没有像当值的同僚一样,亲耳聆听了韩冈的第一堂课。但才下课没多久,留堂的三道题目,就从皇城内传到了南薰门的蔡卞手中。

隔邻教室中也正像放在火炉上的水壶,热闹喧腾。一群国子监生正为韩冈的题目吵吵嚷嚷。

“这叫什么啊!出的到底叫什么题?国子监里有几个能做出来的。”

“别的不说,太子才六岁。白乐天半岁能识‘之无’,可他六岁时也写不出‘此恨绵绵无绝期’吧。”

“没听到韩枢密说的最后一句吗?可以问人!官家、圣人想要的不是君子,是太子。韩枢密也就是要教太子兼听则明的道理。”

“这是卖菜卖惯了。上门的客人想要什么,他就卖什么。”

“有几户人家聘西席先生,不是打算教个进士及第出来?有哪个皇帝不想要个有为的太子继承皇位?”

“多了去了。要我给你数数吗?汉武帝、唐太宗……”

“别抬杠。汉武有瘫……”

那几个学生说话简直是肆无忌惮,尽管最后半句给吞了下去,可还是够悖逆的。真要计较起来,可是指斥乘舆的大不敬罪。说的人杀头有份,听的人也少不了一个流放。

蔡卞动了动身子,想站出去训斥,但又忍住了,只是记住了外面几个人的姓名。

太学三舍,外舍、内舍、上舍。不升内舍、上舍,就别想做官。就让他们在一辈子烂在外舍好了。

“吵什么呢,宗汝霖那边还真摆出来了。”

就在蔡卞听着隔壁吵吵嚷嚷的时候,宗泽从隔邻正在重修司马庙的木匠那里,找来了尺子和锤子,还有一团墨线,摆弄了半天。倒是重现了课堂上的那个实验。

不过尺子不是搭在桌子边缘,而是搭在宗泽的手指上面。

看着宗泽手指上摇摇欲坠却偏偏掉不下来的尺锤,教室中静了下来。

前面国子监生们都是在吵韩冈的用心,但亲眼看到了不可思议的实物,是人都会想要知道这到底为什么。

“既然韩枢密摆下了阵势,肯定是想要太子去找人答案的。也不知王平章和伯淳先生对此能给出什么样的说法。”

宗泽说着,声音不大,却清晰的传了出来。

蔡卞的手一沉,正是他现在所忧虑的。

韩冈的教学,明摆着是针对王安石和程颢两人的课程。如果两家避而不论,到时候皇帝怎么想?皇后又会怎么想?

……………………向皇后正茫茫然,与陪她说话的蜀国公主一样表情。

韩冈第一天上课所出的题目让她们都是一头雾水。

韩冈所出的题,肯定是有其深意在,只是让人想不通。而明面上的答案,也同样让人难以计算。

“从一加到一百的那题倒好说,应该是为了磨六哥的姓子。”向皇后像是自言自语,又像是对蜀国公主在说话,“六哥从小就聪明,上了学后,什么都是一学就会,聪明外露不见得是好事,懂得收敛才好。若能磨一下姓子倒也不坏。”

“六哥可比我家的大哥聪明多了,说不定一下就算出来了。”

“那可不容易。一步步加上去,整整一百步,中间错一点可就全错了。六哥有时也会犯迷糊,昨天背论语,背着背着就跳了句。”

“说的也是。这一题,不要聪明,只要小心。”

向皇后点了点头,又道:“可那锤子尺子,就像戏法一样的,让人完全看不懂了。”

蜀国公主也不懂,不加锤子,尺子都肯定会掉下来,把锤子系上去,反而不掉了。要说是戏法,可不论谁来做,都是一样的结果。而且韩冈还不在场。哪家变戏法的能这么变?

宋用臣回来一说,再亲手一摆,在宫里问谁都摇头。

“不过韩枢密特意说可以问人。王平章、程修撰与韩枢密同在资善堂,据说又在争什么道统,说不定就是韩枢密在给王平章和程修撰下战书。”

“那这一题可就做不出来了?”

地位丢一边,品姓也不论,只说学问,王安石和程颢可都是闻名天下的大儒。韩冈拿来下战书的题目,宫里面可真找不到人来做。就是朝中,也不定有人有这个能耐。

向皇后不多想了,只等着结果来,也就再两天而已。若能将王安石和程颢问倒,那也不坏。这也就能让人知道谁才最合适当太子师。

至于最后棋盘上放麦粒的那题,向皇后倒是多想了一阵。

最后麦粒的数目应该很多,所以章惇、沈括才不赌。两人都是高材博学,不会上当。说不好,可能会有几百石呢。

但她总觉得似乎又没那么简单。

向皇后很喜欢下棋,只是大概因为很少输的缘故,其实水平并不高。她也有自觉,毕竟没什么人敢赢她的彩头。不过这并不影响她对象戏或者说象棋的兴趣。

韩冈棋艺也不高。她曾王旖那里听过几句。棋艺不高的韩冈能让章惇、沈括不敢赌,那输掉的结果肯定是赔不起,甚至可能是赔得太多,不敢冒风险。

“也有可能是韩枢密虚张声势,故意诳人。”蜀国公主猜着,“麦子做彩头比起几百上千贯来实在是不值什么,反而让人心中生疑。”

一粒、两粒麦子,就算每一格翻一倍,到了六十四格,也肯定多不到哪里去。比起韩冈给的彩头实在差得太远,让章惇、沈括心中生疑,不敢贸然去赌。

“就像开盅前那样?”向皇后问道。

“有点像。”蜀国公主道。

逢年过节,闺阁中赌彩头,向皇后和蜀国公主各自年幼的时候也没少玩过。也知道上了赌桌,就算心中再没底,也要装出一副胸有成竹的样子。有时候吵吵嚷嚷最闹腾的,反而是最心虚的。

不过向皇后觉得韩冈不会这么简单。虚张声势的手段,毕竟不登大雅之堂,不应该拿来当作太子的课程。

“吾已经交待让宋用臣去找人数麦粒了。看看到底有多少。应该快了。”她说道。

宋用臣回来得比想象中要晚不少。

向皇后已经等得有些不耐烦,蹙着眉问:“一合到底有多少?”

宋用臣欠了欠身,袖口抖着,抖出了些麦粒,“禀圣人,一合就有五万粒之多。”

宋用臣当真让人去拿了一合小麦数数。还不只一个,七八人各自数各自的。数了整一天了。报上来的数字却乱得很,从一万多到十万都有,看着就知道有些人根本就没用心。但要复查一下,时间又不够。不过他也不敢说自己找的人不靠谱,折中一下,报了个五万。

幸好向皇后和蜀国公主都没怀疑。

“十合一斗,十斗一石。一石麦子不就有五百万了。”蜀国公主轻轻啧着舌,对向皇后笑道:“看来章枢密和沈括真的是被韩枢密给唬住了。”

要一石粮食当真能有五百万麦粒,六十四个格子每格都能分上近十万。就算按韩冈所说放麦粒,越到后面放得越多,可一开始才一粒、两粒、四粒、八粒啊。

可是不知为什么,向皇后还是觉得没那么简单。

“再等等刘惟简的消息。”

数麦子有宋用臣,但计算棋盘上要放多少麦粒的差事,向皇后就让刘惟简去算了。刘惟简现在在管左藏库,精通钱谷之术。

不过刘惟简回来得比宋用臣还要晚。

“怎么这么迟?”

“禀圣人,奴婢早前算过一边之后,觉得结果匪夷所思,心道多半是算错了。就去了司天监,让司天监帮忙。谁知道,司天监当值的冬官正算了一遍,却跟奴婢算得一模一样。”

司天监虽人浮于事,水平又差得可以,但基本功还是有那么一点的。要不然刘惟简也不会去找他们。

“匪夷所思?”向皇后瞅瞅一边的棋盘,问刘惟简,“填满棋盘到底要放多少麦粒?!”

刘惟简从袖中掏出一卷纸,展开来照着念:“启禀圣人,到了二十八格的时候就超过一亿了【注1】,一亿三千四百二十一万七千七百二十八。再往后二十七格,到五十五格,就是一亿三千四百二十一万七千七百二十八的一亿三千四百二十一万七千七百二十八倍。而到了最后第六十四格,更是第五十五格的五百一十二倍:九百二十二兆又三千三百七十二万零三百六十八亿又五千四百七十七万五千八百零八。这还只是一个格子,若是将棋盘上六十四格全都加起来,是第六十四格的两倍去一。一千八百四十四兆又六千七百四十四万零七百三十七亿又九百五十五万一千六百一十五……”

蜀国公主完全怔住了,刘惟简绕口令般的数字她听着就糊涂了,可再糊涂,也知道这是个不得了的数字。

一亿三千多万的一亿三千多万的五百一十二倍!还要再乘二,减一!

只是六十四个格子而已!怎么会变得那么多?

向皇后也整整愣了半天,最后惊讶失声:“这么多!?就一个六十四格的棋盘,还没算中间的楚河汉界呐!”

朝廷每年几千万贯石匹两的收入,都兑成钱的话,合几百万万钱。在向皇后看来,已经是了不得的大数目了。可这个数字跟棋盘上的麦粒比起来,却是差了一亿倍。

“一石小麦五百万粒!这到底有多少石?”

刘惟简粗粗的算了一下:“三万亿石还多。”

向皇后更觉得恍惚了:“够吃多少年的?”

“天下人口一亿多,一人一年吃四石。也不过四五亿石吧。少说六七千年吧。”刘惟简也不知道这样算对不对,反正再怎么样都不会少于一千年。

向皇后又是怔了好半天,方回过神对蜀国公主苦笑道:“难怪章枢密和那沈括不肯跟韩枢密作赌,就是官家倾家荡产都赔不起。”

蜀国公主也是苦笑道:“几百一千贯对万亿石粮食,韩枢密真真是太会戏弄人了。”

“章惇和沈括能一眼就看破,论起才智,其实也不差了。肯定是让人望尘莫及。”向皇后点头说着。

不管怎么说,这是比直接上表推荐要委婉得多。

注1:东汉《数述记遗》中记载,古代有上中下三种进数法:下数以十递进,十万为亿,十亿为兆,十兆为京;中数以万万为亿,万万亿为兆;上数以亿亿为兆,兆兆为京。通用的一般是下数,不过这里为了方便起见,选用了中数。

