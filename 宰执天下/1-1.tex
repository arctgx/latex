\section{第一章 劫后梦醒世事更}

从出租车跳上下来就直奔检票口,贺方终于一身大汗的在最后一刻赶上了回上海的飞机。直到在东航的美女空姐不满的目光下跨入机舱,他才整个人放松下来。

贺方不是能让航班停下来等人的主,若是误了机,虽说费些口水公司应该就会给报销多出来的帐,但是要他跟会计室的老处女扯上一个下午,即便是老于世故的贺方也不会有这么好的兴致。

“好了,终于是赶上了!”贺方在座位放松着手脚,懒洋洋的不肯再动弹。

为了赶上预定的航班,贺方昨夜没能睡好觉,现在一点精神也没有,连系安全带时也是慢吞吞的,被过来检查的美女空姐狠狠的瞪了两眼。

飞机已进入预定高度,开始在空中向目标城市飞去,机舱广播提醒着乘客们现在可以放开安全带。机舱内人声嘈杂起来,空姐也推着小车走进机舱。不过贺方却拉下眼罩,靠在椅背上闭目养神,不知不觉已经进入梦乡。

突如其来的猛然一震,机身剧烈的摇晃起来。贺方从睡梦中惊醒,正想找人问明白怎么回事,机舱广播应时响起。不过也不需要广播,只看舷窗外透进来的火光,就知道到底发生了什么。

贺方脸色惨白,紧紧抓住了扶手。据说飞机失事的几率小于百万分之一,他买彩票从来都没中超过二十块的奖,难道今次竟要碰个头彩?!

火势蔓延得很快。转眼间,舷窗外流淌在银色机翼上的火焰已经吞噬了最后一个发动机,覆盖了整支机翼。巨量的燃油从发动机的破口处喷出,在机体过处的轨迹上爆燃起来,延伸在机身之后,如同传说中神鸟朱雀的火焰尾羽般灿烂。巨型喷气客机的双翼就这样拖着数条长长的焰尾,从空中坠向地面,仿佛一颗火流星划破深黯的苍穹,在夜空中分外醒目。

提供给舱中电力的紧急线路在最后一个发动机被吞噬的前一刻已经失去的作用,机舱顶部的数列应激照明灯在几下闪动后突然全数熄灭,连同座椅一侧的小灯一起都黑了下去。机舱终于陷入了黑暗中,除了机舱外的火光再无一点光明。原本就已经被恐慌所笼罩的乘客们,现在顿时引发了他们一阵凄惨哭嚎。

贺方紧贴着舷窗而坐,被安全带牢牢束缚在窄小座位中。机翼上被烈风鼓动着的橙色火焰猛烈的燃烧着。闪烁的火光穿过舷窗透入机舱中,映得贺方的面上忽明忽暗,耳畔充斥着尖叫和哭泣。

不知为何,贺方此时出离了恐惧,反而是心如止水般的平静。他看着周围的一切,却感觉像是坐在影院中欣赏一部新近出炉的灾难大片,对即将面临的结局并没有多少真实感。

舷窗外的熊熊火焰照亮午夜时分的万米高空。‘如果站在地面上仰望,应该让人惊叹的景色吧。’贺方心中胡思乱想。

一团灿烂的焰火在空中爆开,贺方在这个世界的时间就此凝固。

………………………………………………………………………

意识犹沉浮于黑暗中,但从身体的各个部位传来的不适感逐渐将贺方从昏迷中唤醒。那种感觉不是受伤后的疼痛,而是从骨髓里透出的虚脱,如同失血过多的反应,浑如当年胃出血后躺在病床上那般浑身发冷无力。

浑身虚软的感觉很让人难受,贺方还是觉得很高兴。只要有感觉,且不论是什么感觉,至少代表他还活着。能从空难中活下来,再怎么说都是可喜可贺的一桩事。只是很快贺方却又恐慌起来,因为他发现他的脑袋里多了许多不属于自己的记忆。

‘韩冈?那是谁?!’

贺方心中猛然一惊,意识彻底清醒了过来。头脑中莫名多出一段的不属于自己的记忆,完全是另一个人的人生。从幼年到成人,以韩冈为名的十几年的人生岁月留下的痕迹琐碎而完整。但这份记忆并不属于二十一世纪,而是千年之前、因时光久远而众说纷纭的宋代。

‘不会吧……被千年老鬼上身了?’

贺方感觉像是被梦魇住一样,怀疑自己是不是在事故中伤到了头部。他吃力的想睁眼看看周围的情况,但薄薄的眼皮却如有千钧之重,怎么也睁不开去。用尽了浑身气力,也不过让眼皮动了么一两下。

“醒了,醒了!爹爹!娘娘!三哥哥醒了!”

一个少女惊喜的呼声随着贺方眼皮的微微颤动而响起。少女的声音娇柔脆嫩,还有着甜甜的糯音,但传入贺方耳内却变成了黄钟大吕,震得头脑一阵发晕。而后一片杂声响起,身边又多了一男一女略显苍老的声音。他们为贺方一点微小的动作而兴奋不已,话音中满怀着惊喜,可贺方的心却一点点的沉了下去。

贺方自大学毕业后,走南闯北十来年,全国各地的方言就算不会说,也能混个耳熟。但身旁三人说的竟然完全不是他所熟悉的任何一种方言,音调怪异,有几分陕西话的影子,但也有一点广东话的腔调。

‘是古音吗?’贺方联想起脑中多出来的千年前的记忆,‘难道不是我被鬼上身,而是我做了鬼上了别人身,而且还是宋代古人的身!’

一念及此,贺方心中更为混乱,一阵阵的抽紧。虽然喜欢拿着手机翻一翻网络上穿越系的小说,但贺方却不会去相信真有一越千年的事情。只是如今的现状,却容不得他不信。

存在即是合理。

贺方一直秉持着这样的观点。他现在能清晰的听见身边三人喜极而泣的声音。这不可能是幻觉或是做梦!脑中的记忆这样告诉他,传入耳中的话音也是这般告诉他。

梦境也好,幻觉也好,都不应该超出自己所拥有的知识范围。但传入耳中的莫名稔熟、同时却与任何方言都不相同的语言,以及头脑中还残留的不属于自己的记忆,完全否定掉了这是幻觉噩梦的可能。

‘不会真是穿越吧?!’

回想起过去看过的一些打发时间的小说,贺方的内心越发的混乱。难道真的是越过千年的时间,来到过去的世界?若真的发生了这种事,要怎么生活下去?

混迹在在社会最底层,贺方是绝不愿意,但像一些书中的主角那样硬生生背下几百首诗词的本事他可一点不会!虽然对历史了解很少,但贺方至少也知道,不会吟诗作对很难在古代顺顺利利的混个出身。

还有现在的家人,他要怎么面对?而分隔在另一个世界的父母,现在又怎么样了?

纷乱的思绪不断消耗着贺方不多的一点精力,很快的,他又陷入了沉睡之中。

……………………

再一次醒来,贺方是被隔壁房间传来的声音所惊醒。

“韩兄弟,听说秦州城里又来了一位名医,姓李,在京兆府名头响亮的了不得,多少高官贵人争着延请他上门诊病。去年韩相公的小妾宿疾恶发,李大夫几针下去便断了根。韩相公千恩万谢,到府中都不用通报的。今次李大夫来秦州访友,正巧县里陈押司的小儿子得了风邪,又转成肺痨,也是与你家三哥一般,但他是药到病除,转眼就下地能跑能跳。虽然这李大夫\footnote{宋代医官多以大夫为号,如和安大夫、成和大夫,称为伎术官。所以民间对医生便多以大夫相称。}诊金贵点,但用来救命也没人说不值……”

一个刺耳的公鸭嗓音传入耳中,不知为何,贺方的心中便是一阵怒意上涌。这种江湖声口,听着就知道是在胡吹。借着高官显宦或是明星偶像的名头来垫脚进行的骗局,在社会上闯荡多年的贺方如何会不熟悉?就是没想到一越千年竟然被人用在了自己的身上。

“李癞子!你上次说的那位诸葛大夫,俺家千求万请用六亩田换来的药方,却屁用都没有!你现在还来骗俺?!小心老娘老大耳刮子打你!”

极彪悍的吼声,却让贺方心中感到一阵暖意,这是‘他’母亲的声音。但他马上又担心起来,因为从‘母亲’的话中,能听出很明显的动摇。

“俺真是太冤了!”只听得被唤作李癞子的公鸭嗓门叫起了撞天屈,“阿李嫂你想想,这天下间哪有包治病的神医?就像如今的李大夫,也不能拍着胸脯说一副药下去,就能让你家三哥活蹦乱跳的站起来。但终归是一条出路,总不能看着你家三哥就这么病下去吧?田卖掉还能再买,人没了可就买不回来了!”

“……李癞子你不就是贪着那块河湾边的三亩菜田吗?尽着教俺家卖田。老娘在这里说了,就凭你出的那几文钱,卖谁都不卖你!”

“阿李嫂看你说的,俺岂是要贪你家的地?你卖谁俺都不会插话……不过话说回来,你家的那块菜园,村里有哪家买得起?也只有俺才出得起价!要不你也别断卖了,先典给俺,拿到钱给三哥儿治病。若是以后有了钱也可以再赎回来。”
