\section{第27章 京师望远只千里(三)}

时值月末,一场突如其来的暴雪,掩盖了秦凤路通往关中腹地的官道。

鹅毛大雪铺天盖地,天地之间皆是白茫茫一片,山峦河川尽被掩去了踪影。即便今天的黄历上正正印着宜出行三个字,却不会有人会认为在这种天气下离家外出,会是件吉利的事。还在路上艰难跋涉的行人,无不是叫苦不迭,而躲在家中避雪的人们,也要担心着雪势过大,压塌了家里的屋顶。

不过还有人对这场雪欢欣鼓舞,并不是想着瑞雪兆丰年的农夫,而是一些开客栈的店家。

比如在北莽山下官道旁开店的何四,他这路旁小店由于离着东面的马嵬驿只有五里多地,往常一天能有两三个客人住店已经很难得了。大多数的时候,后院的客房都是老鼠比人多,只能靠着卖些茶水吃食来贴补家用。但从前两天开始下雪时起,住店的客人立刻多了一倍,到了今日,雪势突然转急,一连三四家商队都不得不停了下来,挤到了何四家的这间有些破败的小客栈中。

先披着蓑衣从小门出去,把门头上挑起的酒旗抖净积雪,挂到门口更显眼的地方,再回来在厨房里吩咐自家的浑家,把每盘菜的份量弄少一点,酒坛里再多掺一瓢水,何四便又喜滋滋的转回厅中来。

厅中火盆倒是升得很旺,何东主也算是有良心的,并没有把火炭像酒菜那样做了克扣,不然照着现在寒风从遮掩不住的门缝中一个劲透进来的样子,这厅堂就不能待人了。

小小的客栈大厅中,此时挤满了客人。除了当年开张时,亲朋好友来捧场的那一天,何四还是第一次看到自己的小店中,每一张方桌边,都有人围坐着。互不相识的陌生人挤在一桌,吃着没甚滋味的饭菜,喝着明显掺了水的村酿,扯着天南海北的话题。何四坐回到收账的柜台后,让自家做跑堂的内侄来回服侍着客人,自己则听着客人们聊天。

说话的都是些商人,厅中的几十人里商人占了大多数。不过在最里面的角落处,有八九个军汉占了两张桌子,正大碗的喝着酒,不与商人搭话。

“……真的要打了?”一个少说也有三百斤重的胖子压低了声音问着。他身后站着一个五大三粗的伴当,身上衣袍一看就是贵价货,再加上他身材的缘故,一身衣服就得抵人家两身、三身,当是个身家丰厚的豪商。。

同坐在一桌的一个瘦子则嘲笑道:“也不看看这兴平县,往年少说也有二三十万石新粮要从下面的这条官道去秦州,但今年自入秋后,可就没看到半车粮食往西边去的……三军未动,粮草先行,韩宣抚把送去秦凤的粮草全都截了下来,不是为了打仗还会为了什么?”

瘦子身上的穿戴远不如胖子商人,显然不是一路人。胖商人奇怪的问道:“不是听说秦州那里又是一个大捷吗?秦州每年的出产能喂饱自己就不错了,他打仗的钱粮是哪里来的?”

“当然是秦州本来的积蓄喽……”这次是坐在胖商人身后的一人回过头来,他留着半寸多长的头发,穿着一袭打着补丁的僧袍,显然是个很久没有理发的和尚。这和尚桌前有酒有肉,嘴上油光光,看起来就是个好说嘴的:“你们不知道吧,这其实都是韩宣抚闹得。韩宣抚跟郭太尉水火不容,前些日子把郭太尉赶到了秦州,后来又怕郭太尉趁机立功,就一点钱粮都不拨。”

“师傅却是说错了。”瘦子直摇着头,“韩宣抚虽然跟郭太尉不合,但他不调钱粮跟怕郭太尉立功没关系,秦州可是设了缘边安抚司,几次大捷的功劳全是安抚司的,跟郭太尉和小燕太尉都没关系。”

另一张桌边,一个老者放下筷子,插话道:“今次在渭源堡也不能叫大捷,听说不过是个平手而已,两边的死伤都不小。你们想想,前两次大捷有钱有粮,蕃人都肯听命,不费吹灰之力的就斩首几百上千,把敌将一个个都砍了脑袋。今次没了钱粮,秦州的官军只能自己上阵,王安抚被围在渭源堡不说,最后还让那个蕃人头领大摇大摆的走了。而且要不是那个有名的韩玉昆领着一支蕃军绕道贼人背后去,渭源堡说不定真的就给破了。”

“原来如此。”几人的闲聊吸引了多数人的注意,听到难得一闻的内幕消息,无不点头。

“说得那么多,朝廷打仗跟俺们有什么关系?只要今次带的东西能卖上价就行!”厅中一角,一个一身短打的中年商人开了口,只是他操着蜀地口音,当是穿过陈仓蜀道过来的蜀商。

‘呸,蜀蛮子!’一众陕西商人都啐了一口。无论是横山还是河湟的战事,都是关系到家乡的安危,每个人都一直放在心头,对这个蜀商不屑一顾的反应,却都记恨了上。

胖商人又问起老者:“老哥,你说的韩玉昆是不是那个孙真人的弟子?”

“那还用说!除了他还有哪个韩玉昆?!”

“孙真人的弟子?是唐时的那位孙真人?……几百年前的人了,哪收来的弟子?”中年蜀商性子和说话有些惹人烦,也没人理会他,倒是正在角落里喝酒的几个军汉抬头看了他一眼。

“韩玉昆不仅是孙真人的弟子,在秦州设了好几座疗养院,救了千百条性命,而且他还是横渠先生的弟子,文武双全。天子几次下旨褒奖,当官才一年,就已经升了两次还是三次官,日后肯定能中进士、做相公的……”老者也不知从哪里听了这些事,见众人都竖起耳朵静听,得意得喝了一口酒,抖擞精神,便要再说上一通。

“店家!店家!”大门突然被匡匡的用力敲响,一个刚刚变过声的嗓门在外面高声叫着。

何四的内侄连忙过去挪开门闩,还没等他拉开大门,厚重的门板便被人从外一下推了开来,风雪立刻伴着新的客人卷了进来。

进来的旅客总共三人,都披着厚厚的斗篷,上面全是白花花的积雪,看不清相貌。三人走进来一点,大门立刻被关上,刮进来的风雪又被堵在了外面。

三人脱下斗篷,露出的是三张年轻的脸。最前面的十五六岁的半大小子,当是方才敲门的,看穿戴是个伴当。而后面的两人一高一矮,矮瘦的青年相貌普通,大约二十多岁;而他旁边身材高大的年轻人看起来只有二十出头,比矮个青年要小上两三岁,不过气质很特别,斯文中透着英气。

何四连忙迎上来,除了前面的小伴当,后面的两人穿戴皆不差,尤其是高个青年,当是有些身份的。“客官是打尖还是住店。”他问道。

高大的青年笑了笑,视线绕着客栈大厅看了一圈:“这辰光,只能住下了。”

“可有上房?”小伴当上来劈头便问。

何四躬了躬腰,表情谦卑中透着无奈:“三位客官你们看,还真是不巧得很,小店的几间上房都给人定下了……”

小伴当不等何四说完,就回头苦着脸对着高大青年道:“官人,你看这事……”

“出门在外,没什么好计较的。也没必要一定要上房。把马照顾好,随便来一间房,只要干净就行了!另外再来点吃得,要干净的。”高大青年说得平和,听口气仿佛是已经放低了要求,可眼下厅中几十人,夜里却都是要睡桌子的。

何四做久了生意,见过的人成千上万,也算是有眼色的。只看了三人腰上的兵器就知道他们的身份绝不简单。寻常百姓除外,最多拖根杆棒、带条朴刀,能光明正大携带兵器的,军汉居多,出家人其次,剩下的就是官员。

‘要是穿了公服就好办了。’可惜三人都穿着出行的衣袍,何四一下确认不了三人的真正身份。虽然他有权力查看路引,但实际上官府要求的住客登记只是表面功夫而已。从来都不会几人照着去做,客人说什么就是什么。现在说要查路引,肯定会惹起怀疑。他便冲跑堂的内侄使了个眼色,“小九,你去把三位客官的马带到后面马厩里安顿好,不要失了照看。”

唤作小九的小二会意点头,连声应了,转身便出了门去。李小六把斗篷一披,也连忙跟了出去。

伴当可以站着,但眼前的两位年轻人却不可能站着吃饭。何四正想办法要腾出一张桌来,先把两人安顿下,小九就已经回来了。他贴在何四耳边,声音细如蚊蚋:“姐夫,都是驿马。肚子上都有烙印,不会有假。”

何四悚然一惊,能动用驿马,三人的身份不问可知。他看着满满当当的厅中,苦笑着上前跟人赔了半天不是,好不容易在那几个军汉旁边腾出个空地来。而小九已经从后面般了一张落满灰、瘸着腿的桌子。何四把桌子擦了又擦,又找来砖头把桌子脚给垫上。

一通忙活之后,他拿来登记簿,小心翼翼的问着:“不知客官贵姓。”

高个青年吐出了一个字:“韩!”

