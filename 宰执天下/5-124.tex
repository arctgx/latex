\section{第14章 霜蹄追风尝随骠(八)}

阴云密布,看着就要下雨下雪的样子。应该是正午的时候,却是一股股寒气逼人活动不开手脚。

折可大站在山坡上,低头俯视着下方的一片工地,神色中的阴郁跟天上的铅云一样浓得化不开。

刚刚焚烧过的地面是一片黑色,草木被清理一空。就在一片灰黑的土地上,两千多人正拿着斧锤铲锯各色工具忙碌着在修营寨。虽然人力远远不敷使用,但在几个关键的据点上安营扎寨还是足够了。

吆喝的号子伴着重锤此起彼伏,将一根根碗口粗细的木桩,敲打到地里去。一条长达两里,拦路而修,一直延伸到道路两侧高坡上的栅栏正在成形。

不过折可大还是觉得太慢了,这片营地要到明天才能完工,想把一应防御体系完成,更是要七天后。不是临时性的行军大营,要具有最基本的防御能力,时间和人工都是省不了的。而且要抵挡大军围攻,日后还要进行大量的增筑、改建,甚至重修。

北面三十里外,便是辽国东胜州武清军所在。辽军从东胜州河清军南下,只要一个时辰,就能杀到柳发川边的前军大营。辽人的铁骑随时可能出现,现在的进度实在是太慢了。要抵挡辽人的大军,可不是用木头营垒就能解决问题。

远处一片黄烟腾起,三十余名骑兵从北面飞驰而来,直扑营栅,被外围的守卫拦了一下,不过立刻就被放行,然后朝着这片坡地奔来。

远远的就认出了领头的骑着栗毛马的骑兵,那是他父亲的兄弟,排行十六的折克仁,折可大大步上前相迎:“十六叔。”

纵马上了山坡,就在折可大身边,折克仁跳下马来。大家族中的成员,年纪和辈份没有半点关系,他的年纪也只比折可大长上一岁而已,由于脸庞略圆,又没有留须,看起来比折可大还要年轻一些。

回头望了一下下方的工地,折克仁皱眉摇了摇头:“看来这营垒真要到明天才能完工了。大哥儿,单是营栅还要多久?”

“天黑前应该能完工。”折可大应了一声,回问道:“道路那边呢?有什么麻烦吗?”

折克仁也不管脏不脏的就一屁股在路边的一方青石上坐了下来:“只是挖坑而已。一上午挖了有三五万吧,两里地,能跑马的山上、坡上都照顾到了。还有三条沟。足以挡住辽人了。”

“三五万,陷马坑挖得这么快?!”

“能崴了马脚就够了,也不是要把整匹马都陷下去,一铲子下去坑就出来了……坐!”折克仁拍了拍身边的石头,示意让折可大坐下,“说起来,军器监的铁锨还真是好用,只发了五百把还真是少,都是能打造兵器的好铁,刃口都能看到钢花,也亏他们舍得……”

折可大先把石头上的灰土掸了一下,方依言坐下,“现在朝廷又不缺铁,一年据说都有万万斤了,光是徐州的生铁就比以前全国都多。没看如今朝廷多大方,一说打仗,铁甲、斩马刀全都发下来了。”

折克仁道:“要不是韩龙图,最新的这一批可到不了手。”

折可大笑起来,“若是朝廷严令不许,韩龙图会这么做吗?”

折克仁呵呵笑了两声,“不说这个了。你看看这边的土地,还真是好地,方才在前面看人挖坑就这么想了,挖出来的土是真正的膏腴,可比府州的地强多了。”

折可大笑道:“所以这边是养马地。来家的马有许多都是在这里放养。之前是王家。现在则是我们折家了。”

“朝廷,是朝廷的地。”

“嗯。”折可大站起身,“等到南面的援军到了,正好可以把马都放养在这里。”

折克仁嗤的一笑:“别指望太原的马军。夏天、秋天全都被拉出去巡逻、打仗,今年冬天还不只要死多少。而且粮草还是问题。”

河东骑兵的情况不妙,折可大不用听后方的消息就能知道。自幼就骑在马背上,熟悉马性,怎么可能不了解战马的极限在哪里。

折可大也跟着起身,“到了冬天就好了。屈野川冬天的时候,能把河底都冻起来,一场雪后,通过雪橇车运送粮食倒也方便。”

“还是等步卒吧。马军动了,步卒也会跟着动。马军到了之后,最多半月之内,所有援军就都能赶过来。到时候,我们这边也就轻松多了。”折克行伸了个懒腰,“知道你这边的进度,我也放心了。先去前面,到了入夜前就会回来。”

“十六叔还请小心,辽人随时可能会过来。”

“放心放心,我那边好歹有一个指挥呢。辽人若是人多,我会跑回来的。要是人少……”折克行嘿嘿一笑,眉头一挑,“那就却之不恭了。”

“也许辽人过来只是动动嘴皮子,威胁上一两句,可不一定会立刻动手。”

折克仁咂着嘴:“谁耐烦跟他们说嘴,杀过来,就杀回去,韩龙图不是说了吗,有什么事,他撑腰。”

折可大可以确定,韩冈肯定是没说过这个话,但大意是不会错的。他的七弟送来信中,也明确说了韩冈的态度。问题是如果当真闹出大乱,韩冈会不会信守诺言。对于一名文臣,折可大可没那么有信心。

作为下一任家主,折可大必须要以一名家主的身份思考问题,不是在家中等着位置掉到自己的头上来,必须有带领族人在辽国的威胁下生存下去的能力。领军能力是第一的,但并不是唯一的。思维、行事,必须切实承担住数千族人,十万子民的未来。要有抵御辽人的侵袭,甚至包括来自东京城的压力。若是没那份能耐,到了那个时候,除非是天子降诏指名,否则他折可大就绝没有希望继承府州知州的位置。

从他自幼受到的教诲中,绝不可能将折家的安危放在一个没有什么关联的文官身上。他知道他的父亲为什么会想主动收复旧丰州,也清楚为什么要与韩冈联手,但他并不明白为什么要与韩冈走得这么近。韩冈的年纪可是比自己还小一点,年纪轻轻便身居高位,性格和为人必然激进。若是他有什么计划需要折家冲锋陷阵,到时候会不会给折家带来什么灾难性的后果,可就说不准了。

但这个想法,折可大仅仅存于心中,并不方便拿出来明说,尤其现在必须借助韩冈的权力,更是一个字都不能随便说出来。

折克仁上马走了,去边境继续他的任务。只要毁了道路,辽人的骑兵想要杀到营地这边,可就没那么容易了。

阴云的遮挡下,看不到太阳位置的变化,但天色的确是一点点的暗了下来。下方的营地开始点起火炬,折可大想着折克仁差不多该回来了。

但直到天完全黑透了,才有折克仁亲兵一骑奔回,他上气不接下气,“大……大郎,十六将军出……出事了!”

折可大劈手抓住那名亲兵,厉声喝问:“出什么事了!?”

“十六将军被辽……辽人的流矢射中了!”

“什么?!”折可大表情更为狰狞,“十六叔可有大碍?”

“伤……伤……伤了一只耳朵!”

………………

两天之后,韩冈得到了发生在柳发川的边境冲突的确切消息。

事先在营垒前挖好的陷马坑和壕沟挡住了辽人的斥候,被破坏的道路让骑兵不可能纵马奔驰,但辽人撤离时,随手射出来一支流矢,射中了折克仁的耳朵,削掉了半只。然后折克仁立刻出兵边界,虽然没有造成伤亡,但放火烧了两个辽人的巡铺泄愤。

“终于开始了?”韩冈将这份紧急军情丢回桌上。

边境冲突是避免不了的。澶渊之盟以来,河东、河北的边界从来就没有消停过,烧一两个巡铺或是烽燧,都是极常见的事,韩冈一直在等待这场戏什么时候开锣。

虽然这个开场让人觉得可笑。但那一箭,应该是瞄准折克仁射过去的,绝不是简简单单的流矢。只伤了耳朵是运气而已,要是那一箭偏个几寸,就是出了人命的大事了。以折克仁的身份,绝不可能同于边境的平民,可以轻易的息事宁人。

“龙图。”折可适脚步沉重的拿着一封公函进来,“辽人东胜州移牒府州,质问官军犯界烧其巡铺之事。”

韩冈拿着辽人的通牒看了没一会儿,另一封公函也被人送来了,“龙图。李都知昨日已经率部出发。预计四天后,抵达麟州。”

“看来能赶得上呢。”韩冈对黄裳道,“勉仲,帮我起草给萧十三的文书,让他们交出射伤折克仁的犯人,要不然,就依辽法给予合理的赔偿。否则日后辽人侵界,将不问情由,立诛之。”

“折克仁烧辽国巡铺之事怎么办?”黄裳惊问。

折可适立刻道:“这是辽人先犯界伤人,才会有的报复!”

韩冈也道:“只要他们给折克仁的耳朵一个交代,大宋也会负责赔偿巡铺的损失。”

“龙图,这一封通牒发过去,辽人当不会善罢甘休。”黄裳沉声提醒,“难道龙图想与辽人开战。”

“不想。”韩冈摇头,“就是因为不想,才必须做好大战一场的准备!要让辽人明白,我们绝不是可以讹诈的对象!”

