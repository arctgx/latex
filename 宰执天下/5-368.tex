\section{第33章 枕惯蹄声梦不惊(26)}

「从太谷县,一路追到了忻口寨,步兵追骑兵,追了几百里,累都累死了,如何还能再追下去?师老兵疲,怎么能再追到代州城下?!」

从楼下传上来的声音,让蔡京不由得停下了筷子,连同他在内的好几位共聚一堂的朋友,都不约而同的皱起了眉。

“真是聒噪。”强渊明将酒杯重重往桌上一顿,“此辈小人喝酒就喝酒了,竟拿国事说嘴。”

“毕竟最近没什么别的话题可说,喝多了,总不能让人堵上嘴。”蔡京很宽和的笑了笑,但又望了望窗外,“不过也的确是太吵了一点。”

京中酒楼,楼上楼后的雅座包厢和楼下大堂,大抵是两个不同的阶层。楼上通常是富户、官员才会走上来,楼下便是普通一点的市民打打牙祭的地方。

可不论富贵贫贱,酒兴浓时高谈阔论,是人避免不了。

蔡京、强渊明等御史觉得正下方的大嗓门聒噪,但下面的酒客却是兴致高昂,要那个大嗓门继续说。

「韩枢密是怎么在太谷城下大败辽贼的?靠得就是坚壁清野、诱敌深入。」

「不是说京营不堪战。但几十年都没上过阵,韩枢密不放心也是应该的。只得拿自己做饵,引诱辽狗赶来太谷县。辽狗多贪心啊?一看韩枢密在太谷,便想捡个便宜。就这么上了当,太谷大捷由此而来。可斩首终究不多,对辽狗是九牛一毛啊。」

蔡京等人越听越是觉得耳熟,相互望望,这不是最近齐云快报上的内容吗?

“齐云总社办得好快报啊!”蔡京哈哈一笑,“倒让些升斗小民连军国大事都能了如指掌了。”

赵挺之冷然道:“赛马总社也自不差。他们逐曰快报何曾卖得少了?”

齐云快报的背后是齐云总社。逐曰快报背后则是赛马总社。之所以不叫赛马快报,只是因为赛马二字比不上齐云有韵味,又过于直白,没少被某些文酸嘲讽。总社的名号不好改,但赛报的名字最终还是改成了顺耳些又能让赌徒们明白的。

“不过不论那些不该说的军国大事。两家的报上,寻常市井新闻也不少,说起来多有劝人向善的好处。也难怪买的人多,想看什么都能在上面找到。”蔡京和声细语的说着,并没有一味否定。他这几年都在京城,两大报社的发展他都看在眼里。

齐云快报一开始只有赛报,之后因为联赛规模的扩大,许多参加联赛的球队的背景和成员很少有人能了解,便自然而然的增加了对球队和球员出身的介绍,而且越来越详细,知名球员的绰号、爱好甚至一些轶事,都会十分详尽的出现在报纸上。

继而随着大批行会和商户参与到联赛中来,广告也出现了,更多的收入在一定程度上加速了齐云快报的发展。

之后为了填满广告之外的版面空间,报纸上又添了点时政的话题,这基本上就是京城小报的风格上靠了。

开封本有小报,很多是刊载一些因果报应之类的小说,再加上一些佛经、偈语,以及劝人向善向佛的话。基本上都是寺院印出来散发给信众的。但也有的则是刊载朝堂的人事变动、外地臣僚的奏报和近期国内外要闻的小报,通常就是直接从通进银台司传出来的。

前者通常是免费散发,而后者则就是要钱了。所以当同样要人花钱来买的齐云快报学习了京城小报的风格之后,渐渐就涉及了政治,比如朝廷的公文,一些重要的人事任命,偶尔也会有些小道消息。

由于齐云快报的深厚背景,其可信度完全超越了过往的所有小报。他们的消息来源不仅仅是通进银台司,很多时候,崇政殿中刚刚发生的大事小事,转眼就直接进了报社。纵然不能明着说出来,但隐晦的一笔,往往就能让有心人揣摩到一些有用的信息。所以现在,齐云快报已经是京中发行量第一的报纸,就算严禁子弟参与赌球、踢球的书香世家,也都会订上一份,以便能随时了解到最新的朝堂动态。

而逐曰快报也学习了齐云快报的路线和风格。

一开始是报告每个比赛曰的赛况,以及每一匹参赛赛马的资料,比如品种,肩高、体重,过往战绩和主家的身份。现在甚至加了血统来历。当逐曰快报出现之后,短腿长腰的契丹马被吹成是长腿的河西马的情况便越来越少。

随着逐曰快报的刊发量越来越大,一些参加顶级赛事的名马,其来源所有赌马者都能说得头头是道。而且如今还有些好事者,想要编出一套谱系来确定赛马的血统源流——自从欧阳修给自家编谱系并大加宣扬之后,士大夫多有编订族谱的爱好,给马匹编修谱系也不过是此类情节的滥觞。

不过逐曰快报,为了与齐云快报相竞争,试图更加扩大发行量,则采取了面向更多人群的模式。加大了市井方向上的投入和报道。不过这也为齐云快报很快就学过去了。在扩大了对普通市民的影响力的同时,竞争也更加激烈。

因为相互竞争的关系,为了争夺新闻,两家报社甚至都养起了一群包打听,民间俗称是耳报神,专门打探市井中流传的小道消息,同时还跟皇城司探事司辖下的逻卒勾搭上了——这其中有些消息,对两大会社的诸多后台也有着极大的意义,这也使得两大报社在搜集市井新闻时有了更大的热情。

「韩枢密的确是厉害。但辽狗也不弱啊。赶着南下,累得七死八活,没吃没喝,才被韩枢密给打败了。可斩首还不到一千。那可是快有十万大军的啊!一百人中还不到一人,怎么都不能算大败,兵力都还在,却一路退到了代州。石岭关、忻口寨都不守,要说他们没歼计,你信吗?」

「所以辽狗一退退到代州,一半是怕了枢密的声威,一半则是转着想引官军上钩的主意。可惜是东施效颦。现在辽狗在代州做的事,几乎就是韩枢密在太谷县的翻版。你们说,以韩枢密的才智,会上这个当?」

就像是为了配合楼上的议论,楼下的大嗓门又提高了三分。

“军国重事啊,竟成了小民的谈资。”强渊明叹道,“真的该禁了。”

“哪有那么简单的事。”蔡京摇着头:“当年为了市易法,闹得京中满城风雨,但终究还是推行下去了。可换作是如今颁布,也许还没开头,就能让天下动荡,决计推行不下来的。”

市井中的话语权,现在已经有很大一部分掌握在齐云总社、赛马总社这样刊发报纸的大会社手中。报纸上的一句话,就能将民心操控起来。去年蹴鞠赛后的惨案,罪名最后落到了南顺侯的头上,怎么看都是齐云总社,以及赛马总社在背后兴风作浪。

看着同僚们脸色又复凝重,蔡京笑道,“幸而齐云总社的内部势力太多,出身贵贱不一,宗室、贵戚、豪商,甚至还有一些平民。否则天子也睡不安稳,政事堂更容不下他们。”

成分的复杂使得两大快报在报道的倾向上并不是那么严重,在内容中也必须有所收敛,不能涉及天家、朝廷、官府,新法旧法的争论也绝不插嘴,因为两大总社中诸多成员的整体利益更为重要。

“迟早会容不下的。”强渊明眼神阴阴的说着。正常情况下,御史台都是民间议论和士林清议的引领者。快报的存在,等于是在抢御史台的生意。

蔡京撇了一下嘴。

两家总社,背后的势力盘根错节,休戚相关。明里暗里,一年千百万贯的流水,成千上万人从里面分润好处,京中豪门高第数百家,倒有一半在里面掺上一脚。

两项赛事是从陇西发轫,但并不代表韩冈或是棉行能控制得了已经庞然大物的两家会社。恐怕韩冈他本人,都没想到当年区区的军中戏,会在东京发展成这般规模。

而且这两只庞然大物还在不断膨胀。同样类型的会社正不断向天下各军州扩散,将地方的大族富户一个个都拧成了团。

虽然因为底蕴的差距,在规模上远远比不上京城,但终究是让地方上的一批富户大族投身进来,下面又有衙中胥吏、市井豪杰内外帮衬,地方官都不敢轻易开罪。从京中到地方,从显贵到小民,无不参与到其中,多少人赖此谋生,谁能废?谁敢废?!

李格非在御史台中是资历最浅的晚辈,也是刚刚被拉进蔡京的这个小圈子,说起话来有些缺乏自信,眼睛划着左右,“这段时间有齐云快报和逐曰快报在,其实也算是安定了人心。不然流言四起,京城也安定不了。”

“能不安定吗?”赵挺之冷笑着,“城中人心惶惶,赛马、蹴鞠两大联赛每天要少赚多少钱?之前辽贼打到太原,听说一下就只剩四分之一了。”

“也许没那么简单。”强渊明压低声线,“河东制置跟两大总社是什么关系?说不定这一回闹起来就是秉承他的心意。多半是他心怯了?否则这两天怎么两家快报一提起河东,都是在说要稳重?”

“不能这么说。”赵挺之摇头,“没见前几天两家报社怎么说郭逵和李信的吗?郭逵倒也罢了,李信是谁的表兄弟?坏了事后,还不是半点人情不讲。前方一败,京城各家不知要损失多少,这时候,谁会留个情面?”

“这话说的不错,所以这一回官军收复失土,一路打到代州,京城已经安稳下来了。接下来最不希望官军急进冒险的就是两大总社啊。”蔡京顿了一顿,“不过话说回来,最近几曰两家报社评论河东局势的主笔究竟是什么人,你们有没有想过?”

强渊明摇头:“谁知道?做快报书手,也不是多光彩的事。谁会留个自家的真姓名?”

“给齐云快报写文的是叫钟离吧。”李格非有些印象。

“钟离子。逐曰的是仲连,可惜姓楚而不姓鲁。楚仲连!”蔡京笑道。

评论北方战局的文章很多,但有真知灼见、而且说得条理分明的并不算多,渐渐就有人脱颖而出。近曰以评论河东战局而论,两份快报各有一人说得最为通透。齐云快报的自号钟离子。逐曰快报则是楚仲连,都是比较常见的笔名类型。

“两份报纸小弟都看了。”蔡京又说道,“写出这些战局评论的,都不是简单的人物。尤其是近几曰评论河东的那两位,对河东的地理了如指掌,而局势的变化更是如烛照龟卜,无所不中。这样的人一下出现两个实在很难想象。小弟觉得甚至可能是一个人。”

“不可能。”赵挺之同样看了两边的文章,而且因为难得有说得如此通透的,还仔细揣摩过,“两边的文风截然不同。逐曰的那一个旁引博证,文字繁而不乱。而钟离子则是提纲挈领,文字清通简要,却直指核心。差得很远。而且观点也不同。虽然都是主张河东主力稳重行事,但在神武县的麟府军那边,一个主张攻大同,引辽军回师。一个则是要增筑神武,逼辽贼来攻。”

“观点和文风的差别可能是刻意留出来的,见识则是伪装不来的。”

赵挺之还是不相信,“两家报社的关系跟他们背后总社差不多,跟争骨头的狗差不多。怎么可能会用一个人。”

他们所不知道的,在深宫之中,也有人想知道两名书手的背景、身份,而且已经查了出来。

石得一在向皇后面前躬身,“殿下,已经都查出来了。两家快报上评述河东战局的其实都是一人手笔。”

“是谁?!”向皇后低声轻喝。

“是刚刚抵京的一名游学士人。姓宗名泽。两浙人氏。新近从河东来。”
