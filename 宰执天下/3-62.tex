\section{第23章 内外终身事(上)}

吕惠卿、吕嘉问,并坐在王安石府的偏厅中。

吕嘉问前日外出视察京东的市易务,今日刚刚从南京应天府【今商丘】回来。他一近东京城,就立刻听说了琼林宴上的那一出。对于杨绘与韩冈一番交锋后结果,吕嘉问也是乍舌不已:“想不到杨绘他竟然自请出外了。”

吕惠卿冷笑着:“杨元素是倒了大霉,琼林宴上声名尽丧,输了场面,更丢了人,在东京城中成为笑柄。再不出外,留在京师继续受人笑吗?!”

“杨元素找韩玉昆不快,那真是自找苦吃。”吕嘉问虽然没有见过几次韩冈,但他对王安石二女婿其人其事,也是着力打听过一番,“向宝的中风还没好,窦舜卿已然致仕,苏子瞻现在还在杭州,雍王老老实实的住在宫外,但凡跟韩冈过不去的,真的没有一个有好结果。”

“文彦博当初也曾几次三番的要拿着韩冈敲打王韶和相公,最后出了什么事,你知道的。”吕惠卿笑说着。

“也差点中风那次?”吕嘉问呵呵笑道,“凶名卓著,真乃是天上岁星!”

“当真惹不得啊……”吕惠卿也是长叹着,“那个韩玉昆!”

“在玉昆说什么?”王安石换了身家常的宽袍出来,正好听到了后半句。

“正说韩玉昆在琼林宴上的事呢。”吕惠卿口改得很快,总不能当着岳父的面,说女婿是个扫把星一样的人物,“当着天子的面,拿着石头往水里丢,这事有些过了头。但他后来的那段推演,却是很又几分道理,说起来还有些唯识宗的味道在,不知是不是因为横渠靠着长安的大慈恩寺的缘故。”

唯识宗,又称为法相唯识宗、法相宗,是玄奘法师传下来的嫡脉,其祖庭就在有着大雁塔的大慈恩寺。只是唯识宗自晚唐后就已然式微,幸好王安石和吕惠卿对此都有研究。

经过隋唐的佛道大兴,其实宋儒各派经义中,无不融合两教的理论。当世大儒几乎没有不去研究佛理道法的,就算是一向排斥释老,独尊儒术的洛学、关学二家,也是一样。如张载,他就是在研究了佛法和道法之后,才重新回到儒学的殿堂。更别说王安石这等贯通三教,能为《老子》注疏,能以偈语名世的全才。

“是因明学吗?”王安石随口问着,坐了下来。

唯识宗是浮屠诸宗中,研究因明学【近于后世的逻辑学】最为精深的一派。吕惠卿这么说,就是觉得韩冈借用伽利略的那一段逻辑推理,有点像是佛教中因明学的论辩术。

“正是!”吕惠卿点头。而吕嘉问却是一头雾水,只能呆坐着。毕竟能与王安石一起讨论各家法门的,新党中,也只有吕惠卿、王雱等寥寥数人。

王安石想了一阵,摇头道:“是有几分相像,不过与《成唯识论》中所言因明之法,还是有些不太一样的地方。玉昆于此事说得太少,不过几句话,一桩事,不便就此下定论的。”

“这些都是枝节了,日后可以再问。”吕惠卿带着一点刺探,道:“倒是韩玉昆与相公家二小娘子的婚事也快到了,到时候,都得备上一份礼啊!”

听着吕惠卿提起二女儿的婚事,王安石苦笑起来。又是一个跟自己不是一条心的女婿。要不是韩冈人品还算过得去,是为了师门而赴汤蹈火,王安石悔婚的心都有了:“尽给经义局添乱。”

从话语和神色中,吕惠卿看出了王安石的苦恼。宽慰道:“韩玉昆的确接连被天子召见,但不代表他当真能说动天子,要不然,征召张载入朝的诏书就应该下了。”

“就是说动天子下诏又如何?”王安石半沉下脸,冷然说道。

吕惠卿听着一喜:“……相公的意思是?”

“不行就是不行!外面不是说我拗相公吗?”王安石神色坚定,语气也毫不动摇:“不管韩玉昆在算计着什么,经义局绝不能让人!”

……………………

就在王安石发狠的时候,韩冈正在崇政殿中。

从李舜举手上接过一块白水晶。侧面为三角形的柱体晶莹剔透,在掌心反射着照进殿中的夕阳,闪闪发亮。

韩冈不过是在前天向赵顼提了一次,才两日功夫,竟然就给磨制了出来。而且用的是通透无比、一点气孔都不见的白水晶,磨制得也是光亮透彻,几乎看不到上面的磨痕。这等手艺,不是普通的大匠能做到。

毕竟是皇帝啊,言出法随。随便一句,就能让人没日没夜的赶工。水晶贵重,但对皇帝来说,可算不得什么。

韩冈仔细看着手上的三棱镜,一点也不比后世看到用精密机械制造的水晶制品稍差。如果能借用这名大匠,说不定过些日子,就能将透镜给磨出来了。

“这个就是三棱镜吧?”赵顼说着,“方才朕用来对着一线阳光,的确是散出了七彩,正如虹光。与韩卿你说的一模一样。”

韩冈前日受诏廷对,趁热打铁的说起了三棱镜。赵顼对此很感兴趣,立刻命名匠打造。

“彩虹多出雨后,而且必须是天上放晴,太阳出来的时候。乃是残留于空中的水汽,折射了阳光的缘故。其本质,与三棱镜分出的七色光乃是同理。”

这些前日赵顼也听韩冈提过了,了解阳光的组成,便能明了了万物的颜色从何而来。当时只是听着而已。不过当三棱镜磨制出来后,才有了直观的认识。

赵顼笑着接回了三棱镜,在手中把玩着,“格物致知之说,越格越是有理,的确是让人欲罢不能。张载在儒门至道上,的确是别出一番新意。”

“不仅仅是家师,洛阳二程亦曾言格物。”韩冈瞅准了机会:“如今儒门各家,都有其合理的一面。若能集天下名儒,共议诸经新义,由陛下亲自裁定。石渠阁和白虎观的盛事,亦可重现于今世。否则闭门造车,难免贻笑大方。”

石渠阁会议和白虎观会议,是西汉宣帝和东汉章帝时,聚合天下名儒,共同讨论儒家经典的盛会,由天子亲自主持和裁定,自此留名于后世。

“…………”赵顼一阵沉默,连拿在手上的三棱镜都放下了。

赵顼当然不想让外界或后人来嘲笑他。但他更知道这事不好办。各家学派之间的纷争虽不能说是你死我活,但也可以算是相悖如参商,若是强凑到一起来,几乎就是的爆竹,点着就爆的。

而且就算是石渠阁和白虎观,在会议上得出的结论,没多久就被全盘抛弃了。要不然流传至今的汉儒注疏,就不会是东汉末年的郑玄私人所著。

赵顼的犹豫,韩冈看在眼里,心头闪过一阵遗憾。

他知道天子为什么会犹豫。现在王安石已经在撰写《周官新义》,而王雱和吕惠卿则是在注释《诗经》《尚书》。张载多说《易》,但对于《周礼》也是同样熟悉。韩冈也清楚,如果张载进京,要说的绝不会仅仅局限于格物上。要是诸家大儒都进京,更别提讨论的场面会有多火爆。

看起来自己还是太过心急了一点。赵顼虽然对格物感兴趣,但也只是兴趣而已。经义局的目的是‘一道德’,韩冈能看得出天子也是这么盼望的。而韩冈的提议,却不能带给赵顼统一的经典释义。几家学派之间裂痕比起新旧党之间的鸿沟还要深,根本不可能像白虎观和石渠阁两个会议那般,能得出一个让各方面都能稍微认同的结果。

‘果然啊……’韩冈暗自叹了口气,即便让赵顼对格物之说有了兴趣,通过落物实验造出这么大的声势,可在权力面前,还是不堪一击。眼下天子只是对格物致知有了一点兴趣,但这点兴趣却难以抵得过他对稳定朝局、控制思想的欲望。

看来走上层门路终究还是不行,还得靠自己。等自己的地位再高一点,如王安石那般地位才差不多。

既然如此,那还是退而求其次的好。用事为十,得之二三。亏是亏了,不过能让格物学的名声在京城中传播开来,也算是不错了。

韩冈的目标很明确。

他对后世科学的记忆只剩初中的水平,对经史子集的了解,也只限进士科举的考试范围。至于哲学,的确是重要,但众家纷纭,韩冈没力气在上面花费太多的精力。所以他一直将关学的衣冠披在身上,日后自有人去总结归纳。韩冈所要做的,就是推广这个时代对自然科学的认识,改进生产力。如果工业能够兴起,新兴的利益集团就会去要求更多的权力。

所以眼下一开始计算的道路走不通,那就换一条好了。只要能最终走到目的地,走的是哪条路,就不是那么重要了。

想通了之后,韩冈便没有再出言推荐,而是放开来,就着三棱镜,跟赵顼讨论起光学上折射和反射来。

等日后将基础筑牢,再卷土重来不迟!

