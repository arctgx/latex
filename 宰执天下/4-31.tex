\section{第六章 仲尼不生世无明(中)}

横渠镇的五月燥热无比,又半个多月没下雨了。尘土被风卷起,头顶上的天空都仿佛用灰黄的纱帐蒙了一层。不过眼下正好是麦收时节,地里正是一片金黄,这个时候没有雨水反而是件好事,不用担心收上来的麦子遇水发芽了。

就在一处满是新栽杏李的山坡脚下,一架巨大的风车正在夏风中轱辘轱辘的转着。将清澈甘甜的地下水不断的从深达近二十丈的深井中提上来。

因为正是收割时节,不需要浇灌田地,流往田中的渠口都落了闸,清澈的井水便义无反顾的顺着用水泥和卵石铺底的水渠,一路流向镇口,用以给人畜饮用。镇中有水井,但水多带着一点苦味,不及横渠书院下的深水井甘甜,虽然仅是一口深井,现在却在浇灌田地的同时,为横渠镇上的几百户人家提供水源。

张载正站在书院的山门前,俯望着山坡下的一片在数月间,由青葱翠绿转为丰裕金黄的大地。清风吹动了麦田,也吹动了山下的两具风车。轱辘轱辘的车轮声,就不停的送到留他耳朵里。

五十多岁的张载,这些年身体一直都有病。今年转过年来,他的气色又差了几分,脸上还是带着不健康的晕红,唯有一对眼睛深邃无比,仿佛能洞烛世间一切虚妄。

他得意门生苏昞此时正随侍在侧,指着书院山门下的一块块麦浪起伏的田地:“除了两顷多开在山坡上的田,书院周围的三十多顷田地,现今都已经是水浇地。虽然才开始收割,还不能确定收成几何,但今年肯定是一个丰收年景。”

张载点头笑着:“水浇地比旱地要强上数倍,要不然白渠周围数县,也不会成为关中粮仓。”

苏昞的心情很好,手上有粮,心中不慌。在横渠书院中,他还负责管账的工作,为师弟们安排食宿,都由他来操心,不能让来求学的士子们饿着肚子,为了满足这一最低目标,苏昞也是操碎了心,

“等到晒谷之后,书院后面的几个粮囤肯定能堆满。别说一年,三年之积都能存下了。”苏昞喜滋滋的盘算着,去年还有今年的横渠镇上的丰收,让他一向为书院担忧的心,终于可以放回去一大半。

一提起韩冈这位弟子,张载的心情就变得很好:“要好生的谢一谢玉昆了。”

“这是肯定的。”苏昞对韩冈的感激是最深的,要没有韩冈出谋划策,又舍得捐财捐物。如今的书院中,那里还能每隔几日便有点荤腥下肚?那些都是用钱换来的。而且没有韩冈的全力宣扬,横渠书院如今也不会有这么多来自于关中以外的学生,已经占到了三成还多。

有着韩冈的支持,横渠书院这两年来的发展很不错。当然,韩冈并不是一直当着横渠书院的金主,将自己赚到的钱,一五一十的送给他的老师张载。就是对当今的天子赵顼,他的臣子中,也不会有人忠心到这等地步。授人以鱼不如授人以渔,横渠书院周围的一片山坡地并不值钱,但种些易打理的果木,两三年后就能有出息。

而且此处多风,造风车开磨坊就很方便了,另外山脚下又开了深井。通过属于书院的六顷田,加上风力磨坊和为周围田地提供浇灌田地的井水赚到的一些钱,横渠书院能将求学于张载的近两百名士子全都安置妥当。

张载回身慢慢的往书院中走,从他身旁经过的学生,都是在向他行礼之后,这才恭恭敬敬的离开,一个个醇厚有礼,有别于世间的乡儒。

正门后面的庭院中,树木都是不高大,皆与书院同年,也就是三五岁的样子。张载指着院中一角的两株并排的柏树:“这两株柏树还是书院落成时我亲手所植,也不过才几年时间,就长得这般高了。”

苏昞抬头看着这两棵柏树。新修起的房屋,房屋的主人都会亲手栽种几棵树木,算是做个纪念,有时候,小树苗几十年后就变成了参天之木,甚至能留存数百上千年。但张载亲手种下两棵柏树,相距不到两尺,却并不同命:“只可惜一枯一荣,命数有别。”

“枯荣生发,天道也。生灭自然,又何须兴叹。草木如是,人亦如是。存,吾顺事,没,吾宁也。”张载回头教训着苏昞,“季明,得道亦须守道才是。”

苏昞愣了一下。然后便退后一步,向着张载一揖到底,“学生谨受教。”

“不须如此。”张载摆摆手,示意苏昞站起来。他回头再看了一看这两株柏树,眼底还藏着一丝不舍:“再过一阵,可就看不到了。日后再见,又不知会到何年何月。”

“先生已经决定要去京师了?!”苏昞惊喜的问道。

“是要去的。”张载点着头,“不入京师讲学,如何宣扬气学之道?韩玉昆为此竭心尽力,也不能辜负了他。”

昨日从镇上的驿馆送来一封有天子和中书签押的调令,给了张载一个集贤校理的馆职,并命他及早入京。所以书院中人心有点浮动,不知道张载这一去,何时才能回来。但几个主要的弟子,都建议张载领下此项任命,气学若想发展,就必须将声望扩大,好将关中气学推广到天下去。

张载正说着话,忽然猛地捂着嘴,撕心裂肺的咳嗽了好一阵,苏昞连忙过来拍着背,过来半天,张载才停止了咳嗽。无奈摇摇头,生老病死都是躲不过的,张载也自知他的归期已近:“这个身子也拖不了多久了。”

苏昞神容一黯,勉强笑道:“京中名医甚众,必能有医治好先生病症的医师。”

张载没去理会这明显的安慰之词,自己身体自己最是清楚,慨然一笑,为韩冈的努力而感叹,“只为了这一个集贤校理,玉昆在京城可能又跟他的岳父闹开了。”

苏昞却笑起来,王安石、韩冈这对翁婿,的确是很有趣:“韩玉昆也帮了王相公不少的忙,想来他们翁婿两人也不会闹到分道扬镳的时候。”

“王介甫也是难做。论起性子执拗,韩玉昆不比他差。”张载轻笑着,他可不是没见过王安石。

张载说笑着,但苏昞心头还有一点不痛快,“韩玉昆和吕微仲好不容易请动了王禹玉,荐先生判国子监,虽说只是进二退一的打算,没想到王介甫连一个直讲都不肯留给先生。”

“不能入国子监其实无妨。岂不闻‘蒙以养正’四字,养其蒙使正者,圣人之功也。国子监中孜孜以求的乃是一个官字,反倒是蒙昧未明的童子,更易导其向道之心。”

张载回头望望掩隐东侧的偏院中,从中正传出琅琅的读书声,声音皆为童稚,读得又只是论语,一听就知道这是蒙学中的小学生在读书。

只是带着小孩子尖细嗓音的读书声,听在张载的耳朵里,却如大礼韶乐一般让人舒心,“二月蒙学重开,拿着系着葱的竹竿往窗外抛,这开聪明的风俗,可比举试前参拜二圣庙更合正道。”

苏昞默然点头。儒门弟子参拜圣贤、拜祭祖先,只是一个‘敬’字,而不是有所求。为了能考中进士,去拜子路子夏的庙,实在是莫名其妙,的确是偏离正道了。

张载叹了一口气,重又振奋起精神来:“《正蒙》一书,已经成书大半,明年当能见全功,希望这部书能让人多看一看。”

苏昞半弓腰的行了一礼,正色道:“正蒙数万言,学生已一一用心记下。但字多难断,学生斗胆,敢请以分章区别,以便成诵。不知先生意下何如?”

正蒙一书,是张载毕生心血的结晶,但眼下看来则只能说是残金碎玉,断简残章。是一句句、一段段言论的集合,条理性并不完备。在苏炳坤看来,需要重新整理一遍,并加以最基本的注释。

张载扶着那一株已经枯朽的柏树,微微笑着,须发在风中轻拂:“小儿抓周,百物俱全,无意条理明之,取者亦难。的确需如季明你所言,区分章节。不过吾作此书,譬如此一枯株,根本枝叶,无不悉备……可也只是枯枝而已,充之荣之,则须尔等之力。”

“……学生明白。”苏昞略略欠身,张载的意思就是将分章分节的任务交给他们这些弟子,而他本人就不管了。

张载慢慢的向着书院中堂走去,一边走一边说话:“上京后,还要再多见一下韩玉昆。他一向偏于自然,俯仰见天地,亲手开辟一条蹊径,又以实物相验,的确是难得。但须知天地之间不有两则无一,仅是自然之道,就只得一偏,最后难见其成。”

“学生知道。”苏昞低声说道,“不过玉昆不过二十出头,要做到天人两道并行不悖,本来就有些难。他能追着其中一门深入考究,已经是难能可贵了。”

“是否是难能可贵,见了他之后就能明白了。”张载呵呵笑了起来,带着喉间的残喘。

笑声中,清风又起,山下的风车转得更急,轱辘轱辘的,如同车轮,直往东去。

