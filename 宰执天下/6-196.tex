\section{第21章 欲寻佳木归圣众(16)}

冯从义从马车上下来,立刻就有一股热浪扑面而来。

天色都已经黑了,但东京城中依然炎热。刚刚从暗格中放了冰块的车厢里走出来,分外感受到冷热对比的强烈。

眼前的门庭并不显眼,黑漆的门扉,也有别于官宦豪门带着门钉的朱色。

但正门两边,一直延伸到两侧巷口,长达五十丈的围墙,让人知道,能在五十里长的城墙内拥有如此规模的宅邸,其背[这也是禁词?]景又岂会是寻常的人家?

一位身穿赤红军袍的男子立于门前,看见冯从义下车,便迎了上来。

“小人彭孝,奉命来给冯大官人引路。”

冯从义跟在彭孝身后进门,只带了一名伴当在后。

夜深了,只是灯笼的光,让冯从义看不清前方彭孝耳后的刺字。但从装束上,当是禁军的成员。

枢密院三令五申,各家官宦门第,不得使唤禁军。但冠军马会中的成员,除了大量使用厢军在门下奔走,也无一例外借用禁军士卒使唤。朝廷给这些贵胄出行的护卫,都是出自禁军。既然出行时少不了护卫左右,那么让这些护卫顺便做些其他事,就是枢密院都不好说些什么。

即便是上四军,在护卫天家的同时,也是天子仪仗。除了轮值皇城,平日里进行训练的科目,亦多半与作战无关。韩冈和章惇努力争取,方才改变了少许——但也只是少许而已,相较于河北河东陕西的边地禁军,京营禁军的日常训练,只有神机营等少数部队,才能达到应有的水平。

幸好西军那边情况还好一点,至少熙河一带,由于韩家做出表率的缘故,没有拿现役士兵当奴仆的现象。而种家这样的将门,也知道收敛一点,只是难以免俗。

“冯四哥。”

“冯兄。”

“冯大东家。”

等在厅中的主人、客人,只看散官、勋位,没有哪个是在三品之下。但冯从义一到,立刻成为最受欢迎的客人。

只看石阶上,每一级的青石条板上,都刻着的蹄踏飞燕的骏马,熟悉赛马的人氏,都知道这里的主家到底是何方神圣

——冠军马会。

每次回到京城,冯从义只要离开韩冈的府邸,立刻就会被无数请帖给淹没。

其中有王公贵胄,也有无名小卒,而冯从义本身,也要分出一部分精力去处理顺丰行和平安号京城分号的事务,对邀请必须有所取舍,但唯独有一家的邀请,他不会拒绝。

冠军马会的邀约,冯从义即使再忙,也会推开一切来赴会。因为在这里,他并不代表自己,还代表他身后的那一位。

而冠军马会的成员,也不会因为冯从义的身份,而小觑于他。背后有个做宰相的表兄,自己再有一个富可敌国的身家,手中从不缺冠军马,任谁都能在这里得到应有的尊重。

不过今天的热切,还有一番别的因素。

冯从义在京中最为熟稔的老朋友,也是最熟不拘礼的宗室,更是赛马总社第一任会首华阴侯赵世将,三巡过后,低声问着冯从义,“冯四哥儿,听说这一次,你家商会又弄出好东西了。”

冯从义放下酒杯,轻松的笑道,“会首说得是缫丝机?”

冯从义连推脱都没有,赵世将神色立刻热切起来,“当真弄出缫丝机?”

“小子哪里敢骗会首,是家里的一些工匠闲时弄出来的。”

“成效怎么样?!”

“比起过去的抽丝机,只要十分之一的人工,将现在的棉布纺机改一改,也能用在丝织上,还能再减八成人工。”冯从义眯着眼睛笑道,他与赵世将说话,厅中客人虽各自聊着天,但都是时不时的瞥眼过来。

“当真!?”

赵世将声音陡然高了起来,而周围的说话声都停了。

冯从义也稍稍放开了声量:“当然是千真万确,只不过呢,雍秦商会内部不说了,小弟家中可是做着棉布的营生,这丝绸上事上就不怎么用心了。除了将图纸给了一部分与商会中人,现在,连机器都在库房中落灰。”

“这……这也太……”赵世将张着嘴,胡须都在抖着,这也太浪费了,但他立刻就反应过来,“是担心什么?”

冯从义低声笑:“钱一家赚不完的,有现在这么多已经够了,再多,那可就患了。”

“原来如此。”

一群人都跟着赵世将点头。冯从义这般说法,肯定是韩冈在背后的指示。而韩冈的为人,说出这种话,一点也不让人感到惊讶。

但韩冈知道的收敛,需要担心日后,他们这些皇亲贵胄,又担心什么?

而起大宋从南到北,都能出产丝绸。产量最高的是南方,而西北最少。西北如今被棉花占据,对丝绸生产的利益,并不是那么垂涎,但他们这些家族,哪个又能把将人工缩减到几十分之一的缫丝机、丝绸织造机不放在心上?

眼前十几双发亮的双眼,冯从义暗叹,要不是韩冈严令,他如何会放弃这么大的一笔利润?

冯从义对每一期《自然》都不会漏过,上面有许多文章,都是蕴含着难以想象的财富。

两年前,就有一篇关于养蚕的论文,让冯从义看了一遍又一遍,然后牢牢的记在了心间。

那篇论文上面,对于现有的养蚕技术,进行了改进。尤其是通过各种新型的器物,比如温度计、湿度计、显微镜,对进行监测,进行了详细的叙述。

通过温度计,来稳定孵化的温度。由此孵化出的蚁蚕,不但孵化时间整齐,出蚕的比例高,而且体质也比过去孵化的蚁蚕要好。

甚至文中还提出了湿度的概念,摒弃了《蚕赋》和《齐民要术》中的‘燥湿是候’这般模糊的说法,而将空气中的含水量量化,用去了油的头发来牵引指针,这样制成的湿度计,可以将蚕室的水汽,控制在一个稳定的区间,避免蚕病。

蚕虫极易生病,各家各户,每年蚕月到来时,要上香、要祭祀,蚕室打扫干净后,甚至不能进外人。而在地方上,就是催租催税,也都会避开养蚕的时节。可就是这样,蚕月过后,蚕茧颗粒无收的依然为数众多,而只收了少半的也不在少数。

若这论文上的技术有用,只要蚕茧的产量,能加上一成,以天下蚕户之众,增加丝绢产量的就是一个难以想象的数字。

而且,这样做的话,成本并不算很高。温度计已批量出现在市面上,其价格随着玻璃的普及——不仅仅是玻璃便宜了,而且玻璃工匠也变得更多,手艺也更好——大幅下降。

不仅仅是温度计,银镜、千里镜、望远镜,还有显微镜,价格都在下降,而质量则在不断上升。就像现在的显微镜,其物镜的镜头,过去还要工匠设法打磨,现在就是一颗小小的玻璃珠,玻璃工匠将其凝结成一个完美的球形,这样做出来的镜头,比过去由最好的工匠制造的水晶镜头,还要出色许多。

小门小户,准备这些器物,当然还是承受不起。但如果一开始就是仿效棉布纺织那般规模来做的话,这就是一个小小的甚至不值一提的支出。

但是,当冯从义兴奋的写信跟韩冈了解作者的底细的时候,他才知道,这篇论文又是韩冈列出的大纲,然后让人去写的。

韩冈一直在支持这样的研究。尽管他没有成立什么机构,最早的时候也没有让自家的人做跨行业的研究,但出自《自然》上的一些论文,对养蚕业和丝织业都有着极大的促进。

对于自家的表兄什么都知道一点的天赋,冯从义已经习惯得无意去感叹,不过看到韩冈要他安排人去研究缫丝机和丝绸织机,却又让他不要涉足丝织业,他就只能叹气了。

不过韩冈的想法,他还是能够理解的,昨夜韩冈的一番嘱咐,更让他加深了这个认识。

有关丝绸业的工业化生产,与韩家、冯家并无关系。韩冈无意在棉布之外,再开辟一个战场,冯从义考虑之后,也觉得自己无法再去挑战天下的蚕户。

水力机械能对纺织业起到翻天覆地的作用,如果是之前毫无基础的棉布,不会有太大的问题。棉布从一开始,就是半工业化生产,从纺纱,到织布,并没有像另一个世界那样,是从小门小户的单人纺车、织机那样遍布天下,成为大规模的生产阻力。从成本,到人力,棉纺织业都不会造成已有的产业毁灭,也不会让数以十万计的小民倾家荡产。

但丝绸业早已是国家支柱,每年上缴的税赋,有很大一部分是通过丝绸而来,若是一个人能完成十几人的工作,民间会有多少人失去他们仅有的收入?

对于这样让无数人记恨的事,韩冈无意去做,冯从义也不想大损阴德。最重要的是,如果韩冈从中取利,势必要影响到他的名声和地位。

既然如此,当然是给最合适的人来完成。
