\section{第20章 心念不改意难平(七)}

【这是昨天的第二更,因故耽搁了。今天试试看能不能把昨天的两章都补回来。求红票,收藏。】

一步步的从城头上下来,韩冈回眼顾望。郭逵仍站在城墙上,眺望着城外的山川。五十岁的宿将,只留下了一个在烈日下坚定如钢的背影。

通过方才的一番对话,韩冈明白郭逵对自己的看重,并不是因为要与王韶别苗头,而是单纯的认同了自己的能力。这让韩冈不免对郭逵升起了一点知己之感。

不过知己归知己,但在韩冈看来,缘边安抚司方面的工作还是得放在第一位,第二位才是疗养院的事。

郭逵让他权衡两者轻重,韩冈的确也权衡了,可结果却没法让郭逵如愿——如果天子跟郭逵一样,把韩冈倡导的军中医疗制度看得很重,在这方面得到的功劳能在河湟开边之上,韩冈会毫不犹豫地选择后者。可惜的是,除了郭逵以外,韩冈接触到的每一个人,都更为重视河湟开边。

王舜臣正在城门门洞中等着韩冈。不过他不向头顶上的郭逵和韩冈,在炎炎夏日还要晒着太阳。门洞中凉风习习,坐在竹制的交椅,喝着凉茶,再惬意不过。而且旁边还有一群守门兵卒,手上扇着风,口中则皆是奉承。

王舜臣刚做官没几天,就连升了四级,官运亨通四个字都不足以形容他的进速。现在他身边还没有亲信服侍,有不少人想在他面前混个脸熟,好求个出身。

当韩冈从城头上下来的时候,王舜臣正翘着脚,很悠闲的享受着。不过一见到韩冈下城,他便一下跳起来,丢下众人迎了上去。一起向城中走了几步,他低声问着韩冈:“三哥,郭太尉找你到底有什么事?”

“你说呢?”韩冈反问道,脚步不停。

王舜臣迈开大步追着上去:“该不会要三哥你转投过去吧?!”

“转投?”韩冈修长英挺的双眉拧了起来,声音也透着若有若无的寒意:“我什么时候做过王家的门客了?!”

以如今风俗,如果成为官宦人家的门客,就算定下了主仆关系。即便日后为官,见到旧主或是旧主的子女,也得保持尊敬,身份关系并不会改变——这是故时门阀旧制残留下来的痕迹。

但王韶只是韩冈的举主,而且并不是唯一的举主。虽然以地位论,王韶远在韩冈之上。但在韩冈眼中,他跟王韶是拥有共同目标的盟友,而决不是主从。王韶举荐韩冈,是为朝廷举荐,是为他的目标而举荐,并非是对韩冈的恩赐。没有王韶,韩冈照样能做官,当时张守约已经要举荐韩冈了。

所以王韶、王厚也从没有——或者说从不敢——以恩主自居,把韩冈当成下仆呼来喝去。

听出了韩冈声音中的怒意,王舜臣悚然一惊,知道自己说错话了,忙干笑了两声,“俺这不是担心三哥你跟王安抚闹得不痛快吗。”

“开拓河湟不仅是王安抚的事,也是我韩冈的事。自当与王安抚同心协力,又岂是他人能干扰得了?……郭太尉很看重疗养院和军中医疗救护,希望我能把心神多放在上面一点,方才也是说得此事。”

韩冈微笑着,眉头也舒展开来。他不会把王舜臣的一时失言放在心上,只是不想让他以为自己跟着王韶是因为盲目的忠义之心,才故作发怒——日后的事谁也说不准。而最后他也没有瞒着王舜臣,一个巴掌一颗甜枣,总不能一直严词厉色,让王舜臣跟自己离心。

郭逵重视军中医疗救治,也给韩冈打了鼎力支持的保票。就在当天,韩冈便把准备好的申请和计划一起递了上去。

关于秦州疗养院的地址,韩冈早已选定了,照例是军营。而驻院医师,还有有着护理经验的护工,也都安排妥当。

韩冈圈定的军营,原本驻扎了一个指挥的禁军,秦州的禁军一向高傲。但在郭逵的命令下,却也老老实实的迁到了秦州城中的另外一处军营,跟人挤着睡觉。

若是在往日,营中这么急着搬迁,更换戍守、驻扎之地,总得会闹上一闹——通常不是营里的士卒,而是周围做着小买卖的生意人,他们的衣食父母都是营中的士兵——但今次不同,韩冈只是在门前站了站,安抚了几句,不但摊贩没一个敢作声,周围开店的住家也都是老老实实。

韩冈本以为他们是预计到疗养院办起来后生意会更好,所以才不闹腾。但后来听仇一闻说,这是韩三官人名气太大的缘故。

韩冈听着心里不舒服,他在秦州只是把仇家斩草除根,欺压良善的事却从来没做过。不过仇一闻向韩冈解释,这是韩冈是药王弟子的传闻在作怪。

世人都是见庙就拜,不管信与不信,小心点总是没错的。若真是得罪了药王弟子,日后生起病来可不得了——毕竟谁也不敢拿自己的小命去赌韩冈的身份。

韩冈对此不知是该气还是该笑,他并不希望自己被药王弟子的身份束缚住,也从来不承认,不然日后有得苦头吃。不过越离奇越怪诞越有神秘色彩的谣言,往往更容易传播,韩冈清楚这是堵不住的,所以他现在考虑着是不是用革命的谣言对抗反革命的谣言。

平整土地,修整房屋,清理院庭,再加上病房中的布置,这些事早就有了规划,无论物资和人力,韩冈也都早早的定下了。等营中军队一迁走,立刻就开始动工。

由于这座疗养院是位于秦州城中,韩冈希望能成为一个让人传诵的典范,故而比甘谷、古渭两处的疗养院下得功夫更多。虽然无法奢侈起来,却是尽力做到了整洁干爽,美观大方。

营中的道路都是用砖石铺就,就算下雨也不会弄得泥泞不堪。下水沟渠也尽数改成了暗沟。夏日不易移栽树木,但韩冈已经为行道木和园林留下了空间,等到明年开春便可以把树木移植过来。疗养院中特有的长条交椅安置在道路边,在营区一角还能看到一座凉亭。

改做病房的营房整修一新,原本该在屋顶上的茅草也都换成了黑色屋瓦。石灰抹墙、水泥铺底是不用说了,病房的门窗都是重新打造过,关闭起来便是严丝合缝,外有挡雨棚,不虞暴雨侵袭。而病房内的床榻,都是改作了单人床,而不是甘谷、古渭两地的通铺隔间。虽然这单人床只是床板搭在土台子上而已,但照样让郭逵派来查看工程进展的官吏摇头说这实在太奢侈了。

半月后,疗养院的整备终于完工,韩冈请郭逵给疗养院题了名,做了匾,挂在入口的大门上。这期间李师中离开了,韩冈跟着去送了一下。而古渭寨王韶那边,他直接安排了王舜臣把父母家人一起护送过去,这个态度比去信解释管用得多。

在疗养院开张的那一天,郭逵带着一众官员来捧场。众人在营中一处处的参观过去,仇一闻和他的弟子李德新在前面做着解说员。

韩冈跟郭逵走在一起,只拖后了半步。郭逵一路走来,对韩冈的布置赞赏不已。进了病房,先是赞过了平整的水泥地面和雪白的石灰墙,又看了看排得整整齐齐的几十张床位,回头笑道:“前两天看过的人回来后都说玉昆你忒大方了,把个伤病营弄得跟住客的正店一样。现在看看,还真是没说错。玉昆,你把营房做成这样,到底能收治多少人?”

“这是要按病榻多少还有合格的医生护工数量来算的。现在秦州疗养院中总计有两百四十张床位,而院中的医生和护工,大概能照顾三百到四百人。”

“也就是说,添加床位后,最多就能同时住进四百个伤病?”郭逵问着韩冈,“是不是少了点?”

韩冈向郭逵解说:“秦州城,包括城外附近五十里内寨堡的马步禁军、厢军,总计在两万上下。除非是爆发疫症,否则两万人中会病到卧床不起的,在同一时段怎么也不会超过两百人。”

“若是与西贼开战,打起来后,可就不止这么些了。”

“如果是胜仗的话,伤亡最多两成。除去阵亡的,真正需要住院治疗的也并不会太多。若是败仗,能逃回来的,也没几个需要住院。”韩冈说道,“以下官浅见,军中的每一个百人都,最好都有一两个了解急救之术的士兵。能在大战后能处理一下轻伤,帮重伤员止血,以便能送到后方拥有疗养院的城寨中医治。如此,当能少上不少枉死之人。”

郭逵沉吟了一下,“……说得倒是有理。但这些懂急救术的士卒哪里找。”

“从军中挑选聪明稳重的,送到疗养院中轮训就是了。急救术学个十天半个月就能掌握,也不需要费多少心思,再让他们背几张能治头疼脑热的便宜方子,也同样不难。每月支俸加个一两成,当是会争着来做。”

“主意的确是不错。这样疗养院中的护工人手也不会缺了。”郭逵笑了笑,“但这些懂医术的士卒总得有个名目,不能跟普通的士兵混为一谈,但称呼他们为医生、郎中也不太合适。”

“不如叫卫生员吧。”韩冈脱口而出。

