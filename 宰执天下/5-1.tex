\section{第一章 庙堂纷纷策平戎(一)}

【今天晚上八点上电台,有兴趣的朋友可以去听一听。】

韩冈自随着童贯两名内侍走进宫中之后,就感觉到了周围有着一股说不出怪异的气氛。一路上成了围观的对象,就是他献上牛痘后,第一次进宫,也没有说像今天这般受到众人瞩目。

童贯在前面侧身引路,仗着跟韩冈有交情,笑道:“龙图运筹帷幄,远隔万里毙虏酋,宫里面可是没人不吃惊。”

“那日后有人因飞船而亡,岂不都是我的罪过了?”

“两件事不一样啊……而且几十条人命换一个大辽皇帝,怎么都值得的,这可是禁军百万兵马都做不到的事。”童贯奉承着韩冈。但看到韩冈没有任何得色和笑意的眼睛,他就笑不出来了。干咳了两声,老老实实的回头在前面引路。

在通名声中,韩冈踏进崇政殿。

殿中的几位宰执投过来的眼神,倒是没有什么变化,这让韩冈松了一口气。但赵顼则是掩饰不住心中的兴奋。

待韩冈拜后起身,已经耐不住性子的赵顼长声而笑:“韩卿,可听说了辽主因何而亡?通于天,绝于地,可都是韩卿的功劳!”

尚书中‘绝地天通’一词,竟然是做了这等解释,耶律洪基可算是贻笑后世。王珪立刻就凑趣的笑了起来,但韩冈没笑。

“此事臣岂敢居功。”韩冈躬身,“汉质帝夭亡,事在梁冀,不在做肉饼的御厨。”

韩冈的比喻有趣,赵顼呵呵笑了两声,“韩卿说得也是,虽说少不了韩卿的一份,终究还是耶律乙辛的功劳。但也是辽宣宗失察之故。一家父子都亡命于此贼手中,现在连孙子都成了耶律乙辛的掌中傀儡……用人之误,一至于斯!”

“辽宣宗?”

韩冈疑惑的声音并不大,可同样处在兴奋中的王珪耳朵似乎比日常灵敏了十倍,立刻在旁解释:“就是追赠辽主的庙号。”

随着耶律洪基死因一起传来的是耶律阿果登基的消息,改名延禧。大行皇帝庙号宣宗,谥仁圣大孝文皇帝。

“以故辽主治国的功罪,自是当不起一个‘宣’字,但既然耶律乙辛既然受了所谓的遗诏,当然得给故辽主一个上佳的庙号。”

“坠自百仞高空,还来得及下遗诏?”

听到韩冈这么问,赵顼哼了一声:“据称辽宣宗弥留之际,留下遗诏,命魏王、太师、北院枢密使耶律乙辛辅政,处分军国重事。故而耶律延禧,晋封耶律乙辛为郑王,太师兼太傅,尚书令,赐铁券几杖,入朝不拜,上殿不趋。”

这已经不仅仅是权臣这么简单了,耶律乙辛现在挟天子以令诸侯。过些日子,恐怕就是加九锡也说不定。

当然,辽宣宗不是病死,而是比坠马而亡更为无稽的坠天而亡,接下来辽国就肯定少不了内乱——是百分之百,而不是之前的八九成。

“说不定辽国内乱,两边打到最后,还会有一方求到朕的头上!”赵顼嘴角翘起,想起了儿皇帝石敬瑭。

“陛下!”枢密使吕公著站了出来,“澶渊之盟誓书犹在,宋辽乃兄弟之国,至今未改。今日陛下殿上之言,可能传到宫外?!”

吕公著很会扫人兴致,赵顼顿时就收敛了笑容:“吕卿说得是。等辽国告哀使抵京,便选使去吊祭。”

不过他的情绪很快就又高涨起来,“辽国内乱可期,必无暇西故。这一下,攻打西夏也就彻底安心,能够直捣兴灵。”

这下轮到韩冈来扫人兴致了,“陛下,兵法有云,昔之善战者,先为不可胜,以待敌之可胜。七百里瀚海难渡,粮秣难以供给,并不是辽国或是西夏内乱可以改变。”

“韩卿难道不知梁乙埋已经囚禁了秉常,梁氏复又垂帘听政?”

王珪附和着:“西夏权相囚其君上,国中亦当内乱。其即为大宋藩属,自不能坐视。当举师直入兴灵,以讨权奸!”

韩冈事前没想过梁氏下手会如此果决,毕竟给秉常找了辽国公主的还是梁氏兄妹。不仅仅是韩冈,就是同样深悉西事的郭逵也是一样没有想到——倒是有几篇请战的奏章中提到了,可与其说几篇奏章的作者是对西事的准确判断,还不如说是他们中了奖。

将做皇帝的儿子囚禁,自己出来掌权的过去只有一个武则天。东京城中的君臣,谁能想到梁氏敢这么做?再怎么说秉常都是梁氏唯一的儿子。

这半年来,除了景询之外,并没有听说其他属于梁氏一方的重臣被杀,韩冈一直认为西夏国的局势不至于有大变动。至于禹臧花麻说兴庆府中内乱的信,基本上半年一封,早就没人信了。

要说耶律洪基驾崩,辽国即将陷入内乱,这件事宋人能看出来,党项人当然也能看出来。让嫁过来的辽国公主也从飞船上掉下来,也不足为奇。但对梁氏直接囚禁秉常,韩冈还是很难理解。不怕国中也发生内乱,乃至各大部族人心离散。难道还有什么是他不知道的?

只是不论西夏的情况变得怎么样,韩冈都坚持他的观点,“陛下,粮草是变不出来的,万一西夏坚壁清野,毁弃沿途存粮,引诱官军深入至灵州城下。届时只要一支偏师骚扰粮道,官军的攻势便难以为继。总不能把胜利的希望全然放在西夏内乱上?”

赵顼很意外韩冈的坚持,皱着眉头,心中很是不快。韩冈是朝中屈指可数的擅长军事的文臣,领军经验也不缺,他的支持对讨伐西夏的灭国之战能起到很大的推动作用。而他的反对,则就会被反对开战一派拿出来当做证据。

吕公著就是个好例子,他的观点与韩冈相同:“秉常一年送马、驼三万与辽国,国中民怨已深。梁氏政变,许是有恃无恐,不能认定其国中必有内乱。”

元绛反驳道:“没有了辽国为依仗,西夏国中人心定难安稳,如何会无内乱?”

吕公著回道:“敌国人心岂可恃?兄弟阋于墙而外御其辱,官军攻入西夏境内,双方未必不会同仇敌忾!”

元绛冷笑道:“西贼贪于财货,朝廷以爵禄诱之,如何同仇敌忾?”

两边的争论,让赵顼的脸色越来越难看,这时,薛向站了出来。

“陛下。古语有云,天予不取,反受其咎。”态度一直暧昧不明的新任枢密副使也终于开口,“如此良机不把握,日后还会有这么好的机会吗?七百里瀚海的粮道的确不易输送,但维持到打下灵州,还是能够做到的……”

王珪随即接过话头:“灵州一下,试问兴庆府又如何保全?”

韩冈眼神瞥了薛向一下,不知他是不是跟王珪做了利益交换,今天终于表明了态度。薛向在粮秣转运之事上也是权威,他的支持,不啻是对韩冈说辞的反击。

一番争论,到最后也没有得出结果,只能各自散去。但从赵顼的态度上看,韩冈知道自己失败是必然的。

韩冈并不是很在意最后的结果。他被挡在两府之外,就是因为年轻,行事激进,不适合居于两府。现在他需要表现出自己的稳重,而不是算无遗策。

但话说回来,不宜冒险进攻兴灵,也是他对战略局势的判断,并不因为自己的需要而改变。

……………………

吕惠卿回到家中的时候,吕升卿已经先回来了。

吕升卿今日回京,要留在家中过了年后再去继续他的工作。他今天下午在开封府述职时,也听说了辽主的死因。

听了兄长说了一遍今天崇政殿中的议论,吕升卿惊讶道,“想不到韩冈现在竟然还在反对直取兴灵。该不会是因为无法领兵,所以反对吧?”

“韩冈现在哪里还会想要功劳,往外推都来不及。”对弟弟的猜测,吕惠卿冷哼了一声,“若不是他的年纪,早就在两府中坐着了,何至于会去群牧司?”

“那他为什么还反对?眼下的局势千载难逢,一旦辽夏两国局势稳定下来,就再也没有这么好的机会了。”

“韩冈前面都已经说了不能直攻兴灵,他怎么方便改弦更张?这不就显得他思虑不周吗?”吕惠卿眉头皱了一下:“而且,应该也有一部分原因是韩冈当真觉得直攻兴灵太过冒险了。”

“韩冈发明的飞船,使得辽主送了性命,还让辽国陷入内乱。总觉得巧合过了头。”吕升卿很是疑惑,今天午后听说了此事后,开封府衙中里面就没人办公了,全是在议论此事,“会不会真的是他的谋算?”

“韩冈要是真有这个本事,之前就会赞成攻打兴灵了。粮秣之事,不比谋算辽主更简单?”吕惠卿和韩冈抬头不见低头见,熟悉得很,韩冈的才智谋略皆是上上之选,这一点的确不假,可要说他能做到谋算惊鬼神的地步,吕惠卿哪里会相信。但他又沉吟起来,“……不过免不了天子会这么想,所以韩冈坚持不能直攻兴灵,当有这份心思在。前后如一,才能显得心中坦荡,以便化解天子暗地里的猜疑。”

自家兄长的判断,吕升卿从来不会怀疑,点头道:“想来应该就是这样。”又是一笑,“只是世人多愚,现如今,韩冈的名声恐怕又要更上一层了。”

