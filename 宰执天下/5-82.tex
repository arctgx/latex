\section{第十章 却惭横刀问戎昭(14)}

【十二点前后还有第二更。】

蓝田县外的吕家别庄中,吕大防一身素色的麻衣,坐在空寂无人的庭院中。

就在灵州之败后,朝中下诏,命吕大防就任庆州知州,代替高遵裕的职位。可就在诏书来的前一天,吕大防的亲兄弟吕大钧,在永兴军路转运司任上因病故世。本就无意参与这一场战争的吕大防乘机辞了就任庆州的诏令,告假回家,为亲兄弟服丧。

还在丧期之中,吕大防虽与人对坐,但摆在石桌之上的,却并不是酒水或是其他的饮子,仅仅是两杯清茶。

“为了给盐州输送粮秣,民夫已经征发到庄子上了,县里说了,要十一人。”

吕大临没有出仕,几个兄长都在外面做官,家里的产业基本上都是他在管。县里发单要人,平常都是自己处置了。不够眼下既然吕大防在家,便得向他请示。

吕大防不插手弟弟的工作,道:“该怎么安排,一切照旧例。”

“小弟知道了。”吕大临没什么表情的应了一声,停了一下,他又开口道:“从年初开始,调集民夫的单子就没有断过。今年的夏收就因为人手不够,没旱没涝,什么天灾都没有,白渠上的几千顷田地,收成却硬是比往年减少了一成。”

吕大防沉默着,慢慢抿着渐渐变冷的清茶。

“三哥就是生生累死的!”吕大临阴沉着脸,“辗转于途而枉死的民夫则更是不知凡几。都已经败得那么惨,这一仗,怎么还能打下去!?”

吕大防叹了一口气,放下手中的素瓷茶盏,“为什么朝中将徐德占的鄜延路体量军事兼计议边事改成了陕西路计议边事,还将李长卿【李稷】也分派过去佐理军中转运?现在只要是有关西北兵事,徐德占都能插话,谁还能压得住他?朝廷一心要守着银夏,谁来说都没用。韩玉昆在朝中说了那么多,可天子依从了一句吗?”

吕大临愤然握拳一捶石桌:“吕惠卿私心太重!”

“不仅仅是私心太重这么简单。”吕大防与吕惠卿打过不少交道,对其也算是有些了解,“是吕吉甫为人高傲,耻为人后。新法诸条,泰半出自他手,为什么他做了参知政事之后要另起炉灶,大兴手实法?因为他根本就不甘心做萧规曹随的曹参,即便他前头的是王安石也是一样不甘心。何论王安石女婿的韩冈?守住了银夏,那是韩冈的建言之功。而守住了盐州,就是他吕吉甫的慧眼独具、远见卓识!你说吕吉甫会怎么选?”

“这不还是私心?!”吕大临反诘道。

“私心也分几种,此乃功名之心,非是利禄之心。”吕大防垂着眼皮,看着杯中的茶水,“若只是为了做一宰相,吕吉甫学着王禹玉循规蹈矩、谨守上命就够了。眼下只要他依韩玉昆之言,保住银州、夏州,就可以等着天子御内东门,锁院宣麻了。但这也要他甘心!”

“利禄之心,仅损私德。功名之心,可是会祸国殃民。灵州之败,不正是王禹玉起了功名之心的缘故?若他能安于利禄,岂会有如今之失?吕吉甫对功名看得太重,自然也就将国事、百姓看得轻了!”

吕大临对吕惠卿颇看不上眼,言辞也不甚客气。

吕大防在官场上打滚的时间足够长,虽说对吕惠卿与兄弟有着同样的看法,但他心中倒是感慨更多一点——哪个士大夫不想一个留名青史?可惜吕惠卿心不正。

“吕吉甫的确是用心不正,迟早自取其败。”吕大防顿了一下,又补充道:“但说不定这一战当真能赢。现在谁敢保证说盐州必败?从兴灵往盐州,是几乎连水源都没有的七百里瀚海,从青岗峡往盐州,是三百里盐路。有这条盐路在,粮道其实已经打通了。”

……………………

“盐州能撑多久?”

折可适刚刚回到府州,就被拉倒了家中计议大事的小厅中被人问话。

看看左右,自家父亲和几个叔伯都到了,兄弟辈中,还有一位叔祖父。折家算是有实无名的藩属,在府麟、丰三州势力虽大,但也因此受到朝廷忌惮,能在外州任职的子弟几乎一个都找不到。要聚会时,人倒是到得很齐。

折可适现在是灰头土脸,无暇打理的须发乱蓬蓬的。从十四五岁起,每次上街总少不了有闺秀、妇人驻足回头的折家七衙内,一个月之内在盐州、夏州和府州之间绕了一个圈之后,跟个人见人厌的乞丐也差不多了。

折可适现在最想做的就是洗个澡睡一觉,但长辈坐了一圈,幽幽的双瞳都盯着自己,也不敢喊累,老老实实的站着回答父亲的问题:“盐州城中的粮囤现在大半都是空的,驼队和民夫都赶不及运粮。这个时候西贼来攻的话,能守上十天就很了不得了。”

厅中啪的一声响,折克行重重的拍着几案,叹道:“徐德占不该修城的!”

“吕惠卿就不该将兵事交托给他,给种谔、给李宪,甚至给王中正都比给他好。贪大喜功,”

“多了一万增筑城防的民夫,根本存不下多少粮草。”

“……如果西贼一个月后来攻城,说不定还会有转机。”

“西贼会放过这个机会吗?”

厅中只是折家核心的成员,身为将门世家的子弟,最基本的战略眼光没有一人会欠缺。

“故善战者,致人而不致于人。”折克行道,“无论官军占着盐州,还是夏州,都能逼得西贼挥师来攻。大参和徐禧只看到了占据盐州,使得银夏之地尽归我有。可不论官军是仅仅屯兵银州、夏州,还是连盐州、宥州一起占下,党项人都必须将官军赶回横山以南。否则无定河沿岸的上万顷良田以及盐州的万亩盐池,不论哪一种情况都是保不住的。”

占据了会战主动权的一方,胜利的天平将会大大的倾向过来。

徐禧占据盐州,也是逼迫西夏来攻的手段。

但相对于银州夏州,盐州的位置就太靠前了。这样是对党项人有利,并缩减了官军的优势。唯一的好处,就是胜利之后,吕惠卿和徐禧由此能功成名就。而从旁观者的角度来说,折家的上下三代将领,一致认为没必要为个面子的问题,硬是要占着会减小对敌优势的位置。

“小韩经略也是知道不对了。要不然李宪也不会到了晋宁军就停下来不过河。”

折可适忽然又开口,厅中众人听着神情都是一变。

“什么时候的事?!”折克行急躁的追问道。

“就是孩儿回程的时候。李经制的将旗还在晋宁军,不见有大军过河。孩儿私下里问了,是太原那里传令让李经制留在黄河西岸,不要过河。”

“看起来这一仗是输面居多。”折克行叹了一句,韩冈的战略眼光在文臣中算是第一流的,他都抱着同样的看法,基本上,可以说事确定了。

无力的挥了挥手,让折可适站到墙边上去。

折家的核心密会,折可适等有幸与会小字辈都只能站着,听着叔伯们的对话。折家的规矩如此,长辈们说话,小辈没有资格随意插嘴。即便是折可适,被郭逵看重,称为将种,日后基本上就是下一代的家主,可照样是没有特殊的待遇。

折克忠眉宇间怒气缠绕,“一帅无能,累死三军。高永能和曲珍,还真是冤枉,到时候,少不了要问罪!”

“还得看运气,西贼来得迟了,修好城民夫一退,粮草囤积上来。盐州城就不是那么容易被攻破了。”

“西贼濒临亡国,哪里还可能耽搁时间,要筹措粮草和运输的畜力,一个月的时间已经绰绰有余。这几日,要不去攻盐州,除非是嵬名家和梁家想去东京城逛樊楼了。”

“这一回,能保住西军的元气,就是万幸了。”

“打仗哪有一直赢的道理,输输赢赢,习惯了就好。”苍老暗哑的声音响了起来,上一辈中硕果仅存的折继长,坐在现任家主折克行身边一直都没有说话,这时忽然开了口。老家伙咳嗽了两声,抬手抹了一把脸,像是刚刚睡醒了一般,“胜败兵家常事,还有什么看不开的。”

三川口等三次惨败的时候,老家伙就在军中,更是亲身经历过旧丰州的陷落,亲眼看到从唐末便与折家一般世镇丰州的王家与之偕亡,然后折家的府州就给割了一块过去成为新丰州的地盘。这些年,官军翻了身,将党项人压着打,说解气也解气,但也不过如此,想要一举灭亡西夏,折继长从来没有这么奢望过。

他站起身,反手捶了捶腰,叹了一声,“年纪大了,经不住困,老头子先去睡了……”在子侄们的目送下,他向厅门走了几步,突然又回头过来,“当真灭了西夏,胜了契丹,还不一定是我们折家的幸事,凡事多留心几步,为日后着想……顺着大势走!”

