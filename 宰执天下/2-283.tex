\section{第46章 世情如水与天违(上)}

端午过去已经五天了。前些日子弥漫在东京城大街小巷中的艾草味道,也终于在初夏的风中,飘散得无影无踪。

这一天起来,院子里的石榴花开正艳。

朝阳的照耀下,火焰一般在枝头上跳跃的重瓣红花,透过支起的窗棱,透进王雱的房间。随之而来的,还有一句‘佳人携手弄芳菲,绿阴红影,共展双纹簟。榴花照影窥鸾鉴,只恐芳容减。’

王雱的浑家萧氏坐在梳妆台前,对镜梳妆,一手拿着梳子,一边问着夫婿:“这是欧阳永叔的咏石榴吧?”

“正是!”王雱也在整束着容装,一名小婢正吃力的举着厚重的官服,要帮着王雱穿戴起来。

看了窗外一眼,王雱摇头笑了一声。窗外哼歌的是照管庭院的仆娘。一个四十多岁的老佣妇唱着此曲,情景上未免有些不搭。

“欧九重病,已经没几日了,听说遗表都写好了。恐怕再过一两个月,《醉翁》一篇也就成了绝响。”王雱惋惜的说着,欧阳修虽是旧党,但诗词文章却是极好的,王雱也是很喜欢。

“……去年元夜时,花市灯如昼。月到柳梢头,人约黄昏后。”萧氏轻吟着欧阳修的名篇,不像丈夫还要想着党争,她的心中就是单纯的惋惜。

“明年上元可就真是要‘不见去年人,泪湿春衫袖’了。”

低头捏了一下床边还在酣睡中的儿子的小脸,王雱对仍是一脸遗憾的妻子道:“你还是睡一会儿吧。夜里奎官哭得那么厉害,你也是一夜没睡好了。”

他的这个宝贝儿子,也不知犯了哪路阴神。自从随他入京后,隔三差五就在夜间啼哭,哭起来就没停。光靠婢女奶娘也让人放心不下,萧氏都是一夜起来三四次的照看着。

“还没去问过安呢。”

“不必在乎这些俗礼,爹娘都不会在意的。累了就多歇息,夜里奎官怕是还要哭。”

“官人,听说大相国寺中有个叫愿成的和尚,擅长符箓咒,惯会医治疑难杂症,不如请他来看一看奎官。”

王雱微微皱起眉头。他对鬼神之事一向不信,更别说符箓之类的巫术。僧人修符箓那更是让人觉得怪异。不过自家的儿子夜啼不止,日久必然伤身。名医既然治不了,能抓住一根稻草也是好的,

“那就请他来府中好了,但也别太过期待。”

“奴家知道了。”

与浑家又说了些闲话,王雱出了小院,往父母所住的院子走去。他一向好交接,朋友众多。为了方便呼朋唤友,王雱住在相府东边靠外墙的地方,有个小门可以直通出相府去。方便是方便,但每天往父母那里的晨昏定省,就要多走不少路。

走到王安石夫妇居住的院落,正看到二弟王旁也正走过来,后面还跟着弟妇庞氏。

兄弟两人一个照面,王旁夫妇同时行礼,“大哥。”直起腰后,看看王雱身后,王旁问道:“大嫂和奎官呢?”

“昨夜你大嫂没睡好,今日有些不适。”王雱说了一句,又看了看天色,“时候不早了,今天是大起居,还得早点入朝。”

说着就领头进院向父母请安,而王旁跟在后面,脸色则是有些难看。

王安石夫妇此时早已起床,还有跟着父母住的王旖也在。请安之后,一家人就在一起吃了早饭,王安石和王雱起身进宫,还不是朝官、连正式差遣都没有的王旁则是回自己的院子。

被上百名元随围在中间,父子两人往宣德门的方向过去。十几对棋牌在前驱赶着闲人,一路上碰到的行人和官员,一看到宰相驾临,皆是立刻避让到了路边。

群臣避道,礼绝百僚,这是宰相的威严。

马蹄敲击着厚重的青石板,清脆的如同雨打芭蕉。王雱就在马上,正与王安石说着话:“章子厚要出外,曾子宣已经兼了四五个差遣,吕吉甫的丁忧更是要到九月才能起复……”

王雱没说下去,他相信父亲能听明白他要说什么。王安石手下现在真正能派上用场的人手还是少。除了章惇、曾布,还有守孝在家的吕惠卿三人外,也就曾孝宽、吕嘉问等寥寥数人可堪大用。

“韩玉昆还是太年轻。若是让他入京任官,有骇物议的事可以不计较,但资历太浅,一时还是难以派上用场。”王安石摇着头,“何况他也不会愿意。今次河州之事,以他的脾气,闹到最后说不定会辞官。”

为了保住河州,韩冈连给王安石和章惇的私信都走了急脚递,要不是王安石在通进银台司那里安插了人手,韩冈的私信说不定就直通到天子的案头上。正常情况下,谁敢如此犯忌?!不过韩冈连诏书都顶了,看他信中的说法,甚至连矫诏的事也一样做了。与此相比,他擅用急脚的罪过,真的不算什么了。

“河州真的难以挽回吗?……临洮堡那里的可是赢了。”

因为韩冈的奏疏,还有王中正的佐证。在朝堂上已经吵了两天了。河州到底该不该撤军,前日在被天子确定了之后,现在又被重新摆进了议事日程中。

“临洮堡解围,熙州可保无恙,但与河州无关。现在先保住出战前的形势才是最紧要的,河州只能等日后了……没有了王韶,熙河路只能先求自保。”

王安石也想保着河州,但一时之间,他却找不到接手熙河经略司的合适人选。西夏进逼德顺军,关西诸路的主帅都不能轻动,连召蔡挺回京的诏令都被追回了,哪里还有其他能压得住阵脚的选择?

而且在目前的局面下,谁都不会为王韶收拾他留下的后患——运气不好,可是就会把自己给搭进去。就算有心开边的大臣,也都是会选择暂时退军,日后再来攻打河州。这样不但稳妥,还能给自己留一个立功的机会。

这就是为什么放弃河州的决定能通过的道理——满朝文武,找不到一个想保住河州的。

“但有苗授,有韩冈,并不需要让人来接手熙河。王韶说不定还会有消息,再等他个一两个月。等到河州平定,就算他不回来,也一样不会有事了。”

“怎么可能……那几个位置保不住的。”

让韩冈或者苗授暂代熙河路的做法根本不现实。一路经略,那是人人要抢的位置。落在韩冈、苗授的手上,就像小儿闹市持金,哪能不惹起他人的觊觎。

王雱又要争辩,就听到身后一身唤,“相公,元泽!”

是曾布和章惇两人赶了上来。

“怎么……出了何事?”在后面看到王安石父子似是在争执,曾布追上来就问着。

王安石叹了口气,“还是河州的事!”

曾布看了看王雱,笑道:“今天到了崇政殿再商议便是……再怎么说,熙河路总是能保住的。”

“军国重事,岂可谋于众人!?”曾布说得轻描淡写,王雱急得上火。气头上来,脸色都有些发白。按了按一阵发慌的心口,他对王安石说道,“前日没能阻止吕大防就是一个错字,现在再不及时改正,恐怕就再难挽回了。西府岂是会弃了河州就甘心的?”

王雱是一意支持韩冈,他早年就说过河湟若不能抚而有之,日后必是中原之患。如今若是从河州撤军,河湟开边大受挫折,这是他所不想看到的一幕。

“熙州不会放弃的,不论是谁提议都会压下去。至于河州……”王安石摇了摇头,关键还是在王韶的身上,没有王韶,他怎么保住河州?

“要保住河州,还不就是一个拖字?……”章惇叹着,他地位不够,前日没能阻止第二道诏令的发出,这让他遗憾了好几天,“如果没有吕大防,玉昆还是能拖住的。”

“但现在吕大防早到了熙州,第二道诏令可不是像第一道那么简单,韩玉昆如何再抗旨?河州的苗授更不敢反对。加上前面矫诏的事,韩玉昆、王中正少不了要受责罚。冯当世选了一个殿中侍御史去宣诏,不就是为了要一网打尽吗?”

韩冈会抗旨,一开始所有人都预计到了。本来在诏书上就松了口,还选了李宪去,明摆着就让韩冈来挡着。当时冯京和吴充都没有反对,谁能想到是他们欲擒故纵的伎俩,等到第二道诏令一下,都知道上当了。

“总是要保着他的。”王安石轻声说着。

曾布笑道:“韩玉昆少年得志,稍受挫折也非是坏事。”

“以韩玉昆所立诸功,时至今日,只为一太子中允,实是刻薄过甚。前日讲筵后,天子亦曾言及此事。以韩玉昆的未赏之功,有什么罪过抵不了?”章惇心下冷笑,他知道曾布一向不喜韩冈。一直认为韩冈性子太过激烈,行事不顾后果。殊不知变法之事,如逆水行舟,是不进则退,不勒以严刑峻法,如何能压服得住一干反对者。

路上的短短时间,一时争不出个眉目。说话间,就已经到了宣德门处。

