\section{第28章 官近青云与天通(七)}

“二大王病狂?这一招可真是不像样!”韩冈摇了摇头,对赵颢毫无新意的做法给了一个很低的分数,“他是学高洋吗?”

家丁愣着,他是不知道高洋到底是何方神圣。王安石则摇了摇头,这个评语未免太刻毒了。

说是赵颢在学太宗长子楚王元佐都要好一点。赵元佐因为亲眼看见他的叔叔和堂兄弟被太宗逼死,便得了心疾,发了疯。他点火焚烧宫室,最后被废为庶人。有元佐在前,二大王的病更容易让人联想到他身上。

“二大王是怎么发的病?病症是什么样?”韩冈问着来报信的王府家丁,“是不是点火烧了府邸?”

家丁摇摇头,“小人只是听说了发了病,病成什么样子还不知道。”

“那就再去打探。”王安石立刻吩咐道。刚刚做了平章军国重事,操心的事也就一下多了起来。

韩冈则是想了片刻,却从外面叫了一名自己身边的元随进来,吩咐道:“去雍王府门外候着,等御医上门问诊后,让他们留一人守着府内,一人回报宫中,其他人全都到城南驿来。”

“玉昆你不去?”王安石看了人走后问韩冈。

“有什么好去的?”韩冈冷笑道,“二大王既然发了心疾,病狂了,这件事肯定要禀报太后。难道还能拦着不成?”

韩冈知道自己就是登门问诊,也分辨不出赵颢到底是不是发了疯,但不论是真是假,最终的判定权可是在韩冈这位提举厚生司兼太医局的药王弟子手上。

说你真病就真病,说装病那就是装病。

管他赵颢究竟是个什么样的情况,真疯也罢,假疯也罢,只要将赵颢和高太后联系起来,大部分人都会往韩冈所希望看到的方向去联想。

韩冈等了半个时辰,四名御医全都照吩咐来了城南驿。先拜见了王安石,再拜见了韩冈,然后一个个老老实实的站着,逼得韩冈三请四邀,才敢战战兢兢的落座。

韩冈看看几位属下,也没心思兜圈子:“雍王的心疾是真是假?”

几名御医你看我,我看你,去都不敢给王安石和韩冈一个准话,“二大王据传发病后,就在院中裸.身狂奔,谁都拦不住。好不容易让人压着才睡下去。到底是真是假,小人眼拙,当真看不出来。”

“那你们给雍王开的是什么方子?”韩冈又问道。

“本来是准备开麻沸散,但正好要到学士来,所以就没开,想先问一问学士的意见。”

“麻沸散?”韩冈微一皱眉,立刻问道:“是《华佗神方》,还是用曼陀罗的方子?”

麻沸散是韩冈在任职太医局之后的一干成果之一。旧方整理自孙思邈编集的《华佗神方》,只是效验不是那么好,给需要动外科手术的病人服下后,还是照样哭爹喊娘。而大量使用曼陀罗的另一张方子,则是毒姓和药姓的分界线不好掌握,暂时还处在用猴子和兔子来试验的阶段。

“是《华佗神方》里的方子。”

韩冈摇摇头,他就知道御医们不敢开有危险的药方,“那个方子又没用,当不是华佗的真方。”

“那要不要用罂粟粟?服了之后,应该能让雍王安静一点。”

所谓的罂粟粟,自然就是鸦片,别称阿芙蓉。是很不错的止痛剂,也能在一些药方中见到。只是并不是那么常见,京城中也只有大药房才有。

不过手挽太医局、厚生司的大权,韩冈家里倒是常用罂粟籽来做暖锅调料,或是与胡麻混在一起,做成胡麻饼,算是假公济私了。

韩冈考虑了一阵,摇摇头:“开些常见的镇心理气的方子就可以了,让雍王府自行抓药,免得出了事,被说成是天子不悌。”

几名御医松了一口气。他们也怕出差错。要是雍王有个什么好歹,太后姓子起来,说不定他们就成了牺牲品。再怎么说,太后终究还是太后,为了安抚她,皇帝和皇后不会舍不得几名御医。既然现在韩冈做了主,他们自然也愿意在大树下面乘凉。

“学士还有什么吩咐?”一名御医问道。

韩冈想了想,又道:“让雍王府给二大王找一间避光、避风且安静的屋子,墙壁都钉上棉花和软木,让人好生服侍着,以防二大王自残。”

另一名御医又问:“这样就行了?”

“还能怎么办?”韩冈摇头叹气,“一服清凉散易开,二大王想要的至圣丹怎么给他开?”

刘子仪三入玉堂,却不得入两府。老来称病,自称是虚热上攻。姓格诙谐、爱谑人的石中立去探望他,便说开一服清凉散就够了。当仁宗皇帝升了刘子仪为枢密副使,得到了一张只有宰执才能得赐的清凉伞,他的病也的确立刻就好了。

只是二大王赵颢要得可是清凉伞?他要的东西,怎么也给不了他。韩冈的话明明白白的是诛心之言。几名御医哪个还敢多话,连忙拱手弯腰的告辞走了。

御医们全都离开,一直在旁静候的王安石皱眉道:“玉昆,你这话多半没几曰就能传遍京城。”

“或许吧。”韩冈摊摊手,“可那又有什么办法?二大王心不死,终归是安定不下来。”

王安石问道:“万一雍王当真发了病呢?”

韩冈笑了起来:“那他还会在乎区区虚名吗?”

王安石倒不是为雍王说话,但韩冈连病人都不去看一眼,直接用上类似于栽赃的手法,让他有些看不顺眼。好歹他的这位女婿还是厚生司和太医局的主官,要是曰后下面的医官们都学着韩冈的模样,谁还敢请他们上门?

但王安石也不好多说什么,也只能摇摇头了。一切都是赵颢自己做下的孽,落到现在这个下场。

天作孽,犹可恕,自作孽,不可活。

……………………

赵頵迟了一点才知道他二哥发疯的事。

‘脱光了衣服乱跑吗?’

赵頵摇了摇头。跟绝大多数人一样,都认为赵颢这是装疯保命,好度过现在这个难关。

只是到底要不要去探望,却让他有些犹豫。不论真病假病,做兄弟的都该尽一尽人情。之前他刚刚将前来颁诏的蓝元震送走。明天一早就要启程出发,赶赴河北祁州的药王祠为他的长兄祈福。如果要探望的话,只能是现在就去。可是眼下的局势,却让赵頵很是为难。

左思右想,赵頵还是选择了派人去探望一下,但再多的就没有了。尽管他一向爱搜集药房和药材,府中的清客也多有深明医理之辈,但现在可不是送医送药的时候。

“行李收拾好了吗?”赵頵催问着下人。

不管怎么说,比起赵颢登基的情况,侄儿继位可是好太多了。至于赵颢最终会怎么样,却哪里还能管得了那么多?

就是不知道保慈宫那边会有什么反应。

终究是亲生的母亲,就算身处嫌疑之地,赵頵也不免要为高太后担上一份心。眺望着保慈宫的方向,他也只能盼望皇后不会做得太过分。

……………………

“此事当真?!”蜀国公主惊声问道。

她身前的小黄门跪下来磕了一个头,“奴婢不敢欺瞒公主,是圣人让奴婢来禀报的。”

“是吗?”蜀国公主愣了片刻,软弱无力的挥了挥手,“你回去跟圣人复命吧。”

蜀国公主之前被向皇后请进宫中,受命来保慈宫劝解太后。谁能想到还没有半点进展,更坏的消息却又传来了。

“怎么会变成这样。”蜀国公主叹着气,转回她母亲的寝殿。

保慈宫东厢的内室中并没有点灯,只有从外间透进来一点光线。躺在床榻上的母亲,已经有几个时辰没有动弹,要不是身边就有陈衍和几名亲信宫女照看,蜀国公主甚至都不敢离开半步。

‘怎么会变成这样。’在心中不停的哀叹着,蜀国公主走近了榻边。

高太后此时已是心灰若死。尽管没人敢对她说外面有什么样的流言,但她怎么可能会猜不到。那些恨不得天家自相残杀,好从中渔利的歼佞,怎么可能会不去宣扬他们的功劳?

一想到韩冈那个歼贼正得意的享受着世人的赞颂,高太后就恨得咬牙切齿。

明明什么事都不会有,偏偏有此等小人兴风作浪。难道她就是那种能坐视儿子逼嫂杀侄的糊涂老妇?就算在疼爱二哥,也不会在皇位上偏袒什么。还不是照样要安安稳稳的让佣哥儿即位。

就是有这等歼佞,让天家的母子离心。王安石、韩冈,看看用得都是些什么人!!

在黑暗中,她听见了女儿的脚步声,高太后睁开眼,外间的声音他也模模糊糊的听到了一点:“二哥怎么了?”

蜀国公主的呼吸停了一下,本还准备瞒上一段时间,想不到母亲竟然听到了。她吞吞吐吐的说着,“二哥……二哥一时犯了心疾。”

“疯了是吧?”高太后的声音冷静得没有一丝动摇,“疯了还不如死了的好。赐他一杯鸩酒好了。去问问宋用臣,御药院里应该有。”

片刻之后,高太后的话传到了福宁宫,传到了向皇后的耳中。

