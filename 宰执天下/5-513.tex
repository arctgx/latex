\section{第39章 欲雨还晴咨明辅(42)}

被当朝宰相怒视着,韩冈的神色依然平静,蔡确心头的怒意一点点的消退下去。

韩冈不是刘攽、石延年那样爱开玩笑的人,既然说得肯定,那么面前的这段松木也许的确就是他所说的火炮。

蔡确第三次将视线投向那段木头,这一回观察得更加仔细。

还是松木。一丈长短,一尺粗细。前面的一端开了孔,碗口大小,而且是小碗。再仔细瞧瞧,应该用锯子将这一段松木竖着锯成两半,然后将一段木芯给掏空,另一端则保留原样留着。最后用铁钉钉回去,接着用铁箍箍好。

蔡确看不出个眉目,但明显的是粗制滥造的产品。

“玉昆,这到底怎么用?”

韩冈招了方兴过来,“试过几次了?”

“四次。换了两门炮。这是第三门。”方兴回话道,“第一门炮试射了三次,第三次,木头被炸裂开来。第二门炮,可能没做好,火都从缝子里面冒出来了。所以又给这第三门加了两道箍。”

“准备发射吧。”韩冈让方兴过去,回头又对蔡确道,“火药威力不小。炮管不结实一点,就会跟鞭炮外面的纸壳一样炸碎掉。木头还是不够牢靠。不过也就试一下,让人看看这是怎么用的。”

蔡确这才发觉,空气的确有着淡淡的硫磺味道。

“已经试了几次,看来是不会有问题了。”

“没亲眼看过实验,韩冈也不知道到底行还是不行。”

“玉昆过于自谦。若没有把握,当不会在殿上夸口。”

‘我是说的没把握是这松木炮。’韩冈还是没把话说出口,松木炮只是在故事里面听说,实际上并没有接触过,不比他想要打造的金属火炮,四处旅游时见过不少实物,青铜的,熟铁的,他都见过。

而且这个松木炮也是临时赶工的结果。现在还没有真正意义上的车床、铣床、镗床,只有简单的加工平台。在造这具木炮的时候,并没有派上什么用场。第一门炮是用凿子硬凿出来的,接下来做第二门炮的时候,有个工匠提议,直接用烧红的铁棒将树干中央灼烧成碳,然后铲掉后再细细打磨,以配合尺寸。说快也是很快,一天下来,就造好了三门。

韩冈和蔡确说话间,方兴那边已经准备好了。

几名杂役将火炮在架子上用绳索固定好,又用绳索将架子牢牢的绑定两旁廊道上的几块石础。炮口正前方,是那块厚木板,但跟着又牵了只羊过来,拴在木板前面。

“之前几次都没有用活物做靶子,这一回试试。”

另外一名杂役则捧了一只碗出来,里面黑乎乎的,装满了火药。

“火药一多起来就很危险,不敢放得太近。”方兴又走了回来,与韩冈、蔡确说道。

那名杂役拿着火药,却没有直接往火炮口中倒,而是先倒在了一块丝绸上“这是?”韩冈疑惑着问。包起来火药,让他有了一种熟悉的感觉。

方兴解释道:“火药太细碎,总是漏了许多在内壁上,一开始怎么也弄不进去,浪费了许多。所以就想了个办法,干脆就用细纱给包起来。这样也方便。”

杂役用丝绸将火药包好,揉了一揉,调整了一下大小,便顺着炮口塞了进去,另一名杂役早就拿着一根一头长木杆在等着,见火药包放进去了,就拿起木杆往里面用力的捣实了。

紧接着就是一枚直径只有三寸、铸好的铁球放进了炮口里面,同样将之捣实。这是韩冈吩咐下来的。对大金作来说,是个很简单的工作。

“这是炮弹,相当于霹雳砲的石弹。都是砸出去的。”韩冈说着,让人拿了另一枚给蔡确看。

蔡确聚精会神的看着,到了现在,他已经渐渐明了其中的原理。只是还不知道威力如何。

火药、炮弹都放好,另有一名杂役拿着根铁钎从火炮上方扎了进去,蔡确这才发现火炮上方靠后的位置,有一个孔洞,不算大,但看起来是深入到火炮内的空洞中。

那名工匠杵着铁钎用力捣了两下后,抽出来,看了眼铁钎的前端,然后便将一根细绳放进了。

“那是引火绳,点火用的。”方兴继续解释,看着前面都准备好,又道:“相公、宣徽,要点火了,还请稍稍移步,”

他说着,指了指火炮的侧后方。那里用草袋装土,堆出了一道墙来,倒像是防洪时的样子。

面对蔡确疑惑的目光,方兴陪着笑脸:“相公,宣徽,这火炮是急就章做出来的,说不准会不会就这么爆开来。两位身系国家安危,还是稍稍站远一点比较安全。”

方兴也难做,为了表现火炮的威力,不能减少装药量,但又要保证在场的两位大人物的安全,着实让他头疼。

韩冈不让他为难,“持正相公,你看,我们稍退几步?”

蔡确点点头,往那道墙后走。

“那他们呢。”韩冈问跟过来的方兴。

“宣徽不用担心,他们会去那里。”

方兴说着指了指院墙。韩冈顺着望过去,那里站了一排禁军士兵。

火器局配属了一个指挥禁军作为护卫。这比斩马刀局的一百人要多得多,跟如今的板甲局相当。十几名禁军士兵都在院墙边上候着,墙上还斜靠着一张张大型的橹盾。

走到草袋墙后,透过缝隙,望着前面的松木火炮。蔡确这时候也有点紧张了,这火炮的威力看起来不会小,否则这么郑重其事又是为何?

火炮周围的人跟着散开了,都躲到了橹盾之后,只有一名小兵拿着火折子小心翼翼的靠上前。在引线上凑了一下,就转身飞快的跑开。

但那引火绳一点动静也没有。

方兴脸色尴尬,看起来是首领的大匠上去骂了两声,夺过火折子上前点着了,倒退着回来。

院中陡然间静了下来,人们的注意力都集中了起来。

引火绳滋滋燃烧着,那点火星转眼就没入了火炮之中。

蔡确双手握紧了,双眼一眨不眨的看着前面。

下一刻,橘红色的闪光在眼前划过,震耳欲聋的爆响轰然而起,一股子青白色的浓烟弥漫在前方。

蔡确被惊得后退了一步,然后再上前。转头看着韩冈,正神色凝重的看着前面。

“好了。”

待硝烟散去,方兴第一个绕出草袋墙,韩冈和蔡确跟在后面。

火炮还在原处,看起来跟之前没有什么区别。只是从炮口、引线入口以及接缝出,还冒着一丝一缕的青烟。

蔡确再向目标看过去,隔了三十步,厚厚的木板上是一片怵目惊心的血红,木板前面的羊早就倒下去了,血水在地面上洇了开来。

身边人影一闪,韩冈快步上前。

蔡确想了想,随即跟了上去。

一群人一起上去,只一看,立刻就有很多人移开了脸。

那枚炮弹击中的是羊的头部,眼和脑的上半部直接就消失了,烂乎乎的一滩黏在木板上。但羊的身子还在抽搐,血就这么一阵阵的流了出来。

铁球落在血泊中,根本看不出来能放在掌心中的铁弹丸,能隔着三十步将骨头一起给打烂掉。就是换了穿了甲胄的士兵过来,肯定也是连里面的士兵一起给砸死。

蔡确强忍着剧烈的恶心感,多看了两眼,终究还是移开了眼,张望了一下三十步外的火炮,对仍是沉着脸的韩冈叹道:“玉昆,这火炮果然是堪比霹雳砲啊。”

“还是比预计中的成品差了很远。”韩冈摇摇头,“火药要改进,不能使用来作鞭炮的玩意儿。木头也要换成青铜或熟铁,只有坚实的铜铁外壁,才能承受住精制火药爆炸后的力量。而那股推力,完全可以将十几二十斤的铁球发射到数里之外。现在用的是玩具。差距之大,就跟小孩子用叶子编的的盔甲,与真正的铁甲一样。”

“数里之外?!”蔡确瞠目结舌,“八牛弩都没有那么远吧。”

“小儿玩的竹弓不过十步,军器监的黄桦弓,百步亦能及。究其原理,却还是一样的,都是利用弓背、弓弦的弹力。这种松木炮,既然三十步外能击碎硬度可比铁甲的羊头骨,如果调整好角度,射程超过两百步亦不在话下。要是换成爆炸威力更强的火药,更加坚固的铁炮,千步又算得了什么?”

蔡确点点头,往回走,让人收拾残局。走到松木炮旁,他伸手拍了一拍,“玉昆,曰后铜、铁火炮的成品也是这般大小?”

“看形制了。发射同样大小的炮弹,铜炮、铁炮可以做更小一点,毕竟比木头要结实许多。若是与这松木炮差不多大,就可以发射更大的炮弹了。”

“霹雳砲是要竖着放的。”蔡确沉吟一阵,忽然又道。

“嗯……没错。”韩冈心中疑惑,不知蔡确想要说什么。

“所以战船上放不下,最多在斗舰顶上安一具拍杆。”

韩冈明白了,却是很吃惊,蔡确的脑筋怎么转得这么快,“换成火炮,可以甲板下一层开舷窗,一扇舷窗后面就是一门火炮,一层两侧可以放上几十门。只要船载得动。”

蔡确闭起眼睛想了想那样的场面,突地摇了摇头,“这样的战船,只要一两艘就能跟上千张弩弓相当了。用在水战中,轻而易举就能毁掉几十艘艨艟斗舰。”

“差不多。”韩冈可是知道风帆战列舰的威力。

“小一点的船呢,能放下几门火炮。”

“就是千料海船,也能载下数万斤货物。一门火炮至多不过三五千斤,怎么也能撞上五六门。甲板上霹雳砲能装几架?而且越高的越不稳。”

“因为重心吗?”蔡确笑着道。

“正是。”韩冈点头。“曰后真要与辽国开战,火炮战舰可以护送官军渡海在辽东登陆。更可以保护官军占据榆关。堵住东京道通往南京道的唯一要道。”

“玉昆……”蔡确叹道,“你这是一番苦心啊。”

