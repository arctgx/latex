\section{第十章 千秋邈矣变新腔(二)}

考生们正心怀忐忑的等待着礼部试的结果。

而此时,一张字纸,正在韩绛、张璪、韩冈手中传阅。

韩绛近乎全白的双眉紧紧皱着,手指捻着胡须,眼看着就是一根根的揪下来。

最后他指着其中的一条,有些没把握的问道:“是《多方》中的一句吧?”

“相公好眼力。”韩冈道,“‘民不克永,多方之义’,虽然是掐头去尾,前后颠倒,连句读都改了,正是出自《尚书·多方》。”

韩绛顿时松了一口气,除了一开始就看出来的出自《唐书》中的一题,这是他辨认出来的第二道题,道:“可是‘乃惟以尔多方之义民,不克永于多享’?”

“正是。”

六道题,韩绛只认出了其中两道的出处,不过他已经很满足了,至少没丢人。以阁试的难度,以他的年纪,能辨认出两题的出处,当真算是多了。

韩绛呵呵笑道:“年纪大了,记xìng也不济了,就认出了这么一题。实在是惭愧。”

韩冈道:“相公说哪里的话,阁试中的哪道题不是为了为难人才出的?”

张璪也道,“这题张璪可是想了半天,实在是弄不清出处,原来是暗数。”

‘民不克永,多方之义’原文应该是‘乃惟以尔多方之义民,不克永于多享’,义民是一个词。

这是暗数中的一题,将断句的位置变了,又故意前后颠倒。以三代文章的艰涩,这样颠倒改换,其实照样能附会解释一番,想说通还是可以的。

但也正是因为出自于《尚书》经文之中,如果认不出来,就不能算是合格的儒门弟子。经典的原文都做不到倒背如流,十年寒窗又到底耗费在哪里?

之前辨认出来的第一题,找出出处很容易,要做出来却难。而这道题,连一卷的标题都在题目中,可以说,这是六道题中最简单的一题,算是送分。

张璪说他想不出来,韩冈半点不信。

其他五题难度都要比这一题要高。这毕竟是为了刷落滥竽充数之辈才设立的考试,六题之中能有一道出自于诸经的本文中,说实话,是给考生留一份情面,免得颗粒无收太过丢人。

张璪盯着字条看了一阵,指着第一条:“‘陨节苟合其宜,义夫岂吝其没;捐躯若得其所,烈士不爱其存’。这是晋书中的一段吧,《列传·忠义》一卷。”

韩冈点头:“开篇明义,乃《忠义》之序。”

韩绛向后招了招手,一名堂吏立刻回头在书架上翻找起来。不过每一部史书都是卷帙浩繁,找起来一时并不容易。

“第五十九卷。”

见那堂吏翻找的麻烦,韩绛提示道。有了张璪、韩冈的提示,到底是哪一卷,他还是记得的。

拿起这《晋书·忠义》一卷,韩绛翻开了封皮,抬眼就对张璪、韩冈道:“还是邃明、玉昆眼力好,一言中的。”

“运气而已。”张璪摇头。

韩冈也谦虚的笑了笑。

这是明数中的一题,出自诸史中的题目。由于是一卷的序文,只要有心准备了,一般都会记住的。再看其文字内容——‘陨节苟合其宜,义夫岂吝其没;捐躯若得其所,烈士不爱其存’——其实也等于是提醒了出处。

这一题也算是简单的。

不过自史记后,至本朝总共十九部史书,排除掉欧阳修私修的《五代史记》【也就是后世所说的《新五代史》,此时被收入禁中,不算是官定史书之列】,以及被排除在官定史书之外的后晋刘昫所编著的《唐书》【此时尚无新旧唐书的说法,官定《唐书》就是宋祁、欧阳修所主编的《新唐书》】,也有十七部,数百万字,在里面随机抽取一句,终究是比出自经典原文的题目要难一些。

“这就已经三题了。邃明,玉昆,还能看出几题的出处?”韩绛问道。

这六道题目,韩冈都很眼熟,不过他可不方便说自己知道所有题目的出处,他屈起手指:“《唐书·宰相世系》《书·多方》、《晋书·忠义》……这一题。”他先指了指纸条上的最下方,接着屈起第四根手指,“是出自《墨子·明鬼》的上篇。”

张璪漫不经意的扫了韩冈一眼,“想不到玉昆对诸子也有研究。”

韩冈笑道:“先师明诚先生说‘民胞物与’,墨家说兼爱。有不少人都说本门要义与墨家相近,为了辩驳此番谬论,韩冈可是费了不少功夫在《墨子》上。”

“原来如此。”韩绛点头。

有关气学与墨家之间的纠葛,还有其他学派对气学的抨击,这些事,他多多少少也有些耳闻。寻常士人能钻研一下道、法、兵和纵横四家的著作,已经很难得了,不过韩冈能够了解墨家传世的文章,却也不值得惊讶。

“剩下的两道题呢?”张璪问着,看起来兴趣盎然。

韩冈再看了看纸条,其余两题全是诸经注疏的内容,而且还都是暗数,改变过句读和顺序的。以难度来说,这两题算是很高了。如果一个不好,黄裳就有可能两道题都做不出来。

“剩下的两道,恕韩冈眼拙,实在看不出出处。”韩冈摇头。他的确知道,但他理应不知道。

“这就已经有四道了……”韩绛喟叹着,“世人有轻浮的说玉昆你当年能中是运气,天子钦点的进士第九也是特恩。但能分辨出四道出处,玉昆你去考制科,照样能上殿。”

“相公谬赞了。不说韩冈当年能不能比得上现在,就是这几道题,韩冈也只是认出了出处,当真要做起来,可不一定能拿到一个‘通’。”

张璪哈哈大笑,“玉昆,你太自谦了。”

“不是自谦,是当真过不了。‘三入十二人,四入三人’。”韩冈指着一开始就被翻出出处的一题,“一看就知道这说的是唐宰相。”

韩绛、张璪都点头。三入、四入,除了说入三省为宰相,还能说什么?这一题的出处,是最好辨认的,也是一开始就认出来的。

只是这一题却一点也不简单,反而是已经辨认出的四题中难度最高的一题。

韩冈指着这道题对两人道:“可这一题出处好说,以此为论也好写,左不过是世族、寒门的那些事罢了。但……这前后文怎么引用?!”

阁试六论,每一题都是要先判断出题目的出处,接下来是将前后文都引用下来,再依据前后文来写出一篇不少于五百字的论来。

而问题就在这个引用上,这是要将前后文全都默写出来,决不能有助词之外的缺漏。

阁试的考题,就是要让人对经史子集烂熟于心,而且因为要将题目的前后文都引用,是必须要全背下来。制科之难,难就难在阁试。

如果是出自经书就很简单,韩冈都能做到,而出自于史书,比如序、赞、论——也就是一卷的开头,或是最后的论述,也同样不算很难——都是重点要背的。但有些题目实在是为了刁难人才特意出了出来。

“‘唐宰相三百六十九,凡九十八族。再入者五十七人’,‘三入十二人’,‘四入三人’,‘五入三人’,”韩冈拿着刚刚找出来的《新唐书·表第十五》——这是有关宰相世系的最后一卷——指给韩绛、张璪看,“加起来总共七十五人,少一个人名就是不‘通’,谁背得下来?”

这是明数题,出处通过逻辑推理就能找出来,但前后文的引用就太难为人了。

韩绛摇了摇头,韩冈说的的确没有错,要将七十五人的姓名全都给写出来,的确难度很高,这种大数量的列举,很容易出现错漏。不知道题目的内容,有几个会刻意去背下来?就算是背了,混在其他几百万字的文章中,恐怕很容易混淆一二。

“当真一个个去数姓名,多半会漏上一个两个。”

张璪则道,“须得对此书滚瓜烂熟。记下了列传,当然也就好列举了。”

如果黄裳之前已经将列传传主的姓名都记下来了,功业也记下来,一个个去列出来还是有可能成功的。

“若是顺序出问题呢?顺序错了可是能拿到一个‘通’?”

“顺序错了一点,也不一定是不通。”

“就怕不是一点。”韩冈摇头叹道。

“那就看黄裳到底准备得怎么样了。不过既然是玉昆极力推荐,想必定能过关。”

韩冈微笑着点头:“韩冈就代黄裳多谢邃明兄吉言。”

六道题,有两道黄裳肯定能做出来,送分的《尚书·多方》,《晋书·忠义》。《墨子·明鬼》这一题,由于士林中曾经有声音说气学近于墨家,相信黄裳也对此研究过,应该比较熟悉。而《唐书·宰相世系》这一道,就要看黄裳的底蕴到底有多少斤两了,七十余人的姓名,而且顺序还不能有错,难度可想而知。

剩下的两条,都是经典注疏中的内容,变了句读和顺序。韩冈指着这两题问张璪,“不知这两题,邃明兄可有头绪?”

张璪摇摇头,“张璪只知道这一题当是出自《chūn秋公羊疏》,不过也没把握。得把书找来才行。”

侍立在侧的堂吏立刻翻身去找,一名堂后官匆匆走了进来:“相公,参政,秘阁那边结果出来了。”

