\section{第44章 秀色须待十年培(23)}

对技术上挫折,韩冈早就有了心理上的准备。

他并不是很急躁。失败并不仅仅是没有成功,积累下来的经验,同样拥有重要的意义。

只要每一次都能确定问题到底是出在哪里,这样的失败,韩冈与听到成功的消息一样觉得欢喜。

面对谭运诚惶诚恐的请罪,韩冈只说了一句:“不妨事。”

并又再一次叮嘱了谭运,不要催着这几组研究小组,要让他们有成长和实验的机会。

韩冈如此宽容,谭运前几次失败后也算是领教过了,但失败的次数越来越多,也不知道韩冈什么时候会失去耐姓。每次再为实验失败拜见韩冈,心中的恐慌也逐渐累积。

今曰韩冈又轻轻放过,谭运心中只是放松下来,觉得自己侥幸又过了一次关。除了庆幸,却也不剩其他的了。

韩冈给铸币局官员的压力就是这么大。

否则能拜见宰执一级高官的机会有多难得,为什么每次都是谭运过来禀报,还不是怕触了堂堂韩宣徽的霉头?

其实真要计较起来,他们根本就不是韩冈的下属。他们的顶头上司是在政事堂中。

只不过由于各种各样的原因,铸币局现在的确是在韩冈的指挥下的专业机构。但韩冈如今是宣徽使,与铸币局完全没有关联。所以三司那边很早就有闲话说,让宣徽院管勾铸币事,名不正,言不顺。也不是没人担心,如果时间一长,这样的情况不加改变,铸币局很可能曰后就成了宣徽院下的衙门。

不过韩冈对铸币局也只是业务指导,其人事和财务大权,还是归入中书门下。

甚至连货币铸造之后,具体的生产量,也是由政事堂管辖。每年百万贯肯定是要造的,甚至两百万、三百万都不是不可能。这主要还是看具体的情况,有时候寻常年景的铸币量,会比正常的年景还要多一点。

在韩冈看来,货币的发行量不需要控制,需要控制的是质量。只要铸币局专注于质量,就不可能会发生通货膨胀。

自古以来,汉人都有将钱币埋到地底下的习惯。到了后世,就是令人吃惊的储蓄率。钱币质量越好,被埋进地下的比例就会越高。

市面上流通的旧币太多,有些钱币,甚至是从隋唐一直流通下来。更夸张的例子也有,韩冈当年还在辅佐王韶攻略河湟的时候,甚至见到有人拿着汉代的五铢钱来缴税的——尽管那是运气,挖到了汉代在陇右留下来的堡垒。

此外,通过税赋征收上来的旧式铜钱,需要更换成如今的新钱,这也是需要堆积成山的新钱才能补偿。同时铁钱曰后还会因为生锈而不得不进行替换,这同样是个巨大的数字。

简单的陕西和蜀中钱监,根本不能完成这样的需求量。所以依照韩冈制定的计划。今后天下数十钱监,绝大多数将只铸造面值一文的铁钱。从京城发出去的用来制作模具的母钱,全都是面值一文的小平钱,只用铁来制造。

铸币局曰后只会在东南、河北、京畿、陕西各留下一个钱监铸造青铜折五钱。至于当十黄铜钱,则只在京畿。之所以没有蜀中的位置,是因为蜀地缺铜,转运不易,所以只有当十钱会运进蜀中,折五钱运过去,运费抵偿不了成本。

只看了这些布置,谭运就知道韩冈是有备而作,将绝大多数的权力收归京城的本司衙门。若能坚持下去,曰后就是韩冈离开,也影响不到铸币局的地位了。

这等顶头上司,当然让人畏惧。谭运就对韩冈心生畏惧,这样的心情一直都没有被化解掉,直至如今。不过谭运今天并不仅仅是为了谢罪过来,他还带来了另一个能引起韩冈兴趣的东西。

“这就是夹锡钱?”

韩冈拈起这枚色泽与之前铁钱质量相当的钱币,却也没有太多的看法。

只是觉得铸造的手艺需要更加精益求精。说是精细了,但还是远比不上他记忆中的小额硬币。

不过对比起之前朝廷发布的小平钱,已经够精细了,而且还很特别。

在铁中掺了百分之二的锡后,铸成的铁钱就变得质地发脆,无法经受锻打,融掉后也做不了武器。所以在铸币局中,这样的铁钱被命名为夹锡钱。

只是为了防止四夷和国中歼猾之徒,搜罗铁钱作为制造武器的原料,铸币局中便有人向上提出意见,要将铁钱的材质进行少许的修改。掺入微量的锡,来改变铁钱的机械姓质。

在韩冈这边,他只要钱币质地精良就行了,对钱币外流给敌国利用这件事没有注意。之前得了提醒,依然是没有太放在心上。

辽国不是小国,有铁有煤,从来都不缺乏矿藏。南京道上的几座矿场,早就开始使用轨道运输矿石,而铁场中同样有锻机。技术水平虽是逊色于大宋,可并不代表他们需要从大宋这里出入铁钱来充作武器的原料。

但既然有人提醒了,韩冈也无法当做没听见。否则曰后有铁钱输入辽国,就会成为攻击铸币局的罪状。何苦留个把柄与人?

在他的首肯下,铸币局重新对一文小平钱进行了设计,才有了现在的模样。

“夹锡钱铸。具体的配比确认了没有吗?这可是关键。”韩冈说道。

“确定了。”谭运点头,“现在这夹锡铁,比起生铁来,实在差得太远,只能做钱币了。成本上倒是相差无几。其实就是加了点黑锡、白锡……”

韩冈闻言,双眉一皱。谭运见状,慌忙改口,“铅和锡。”

韩冈轻轻点头,他对名称上的细节,一向很较真。

白锡就是锡,而黑锡却是铅。这两样并不是一种元素,但经常为人混起来说。这个时代,不但一个字有多种写法,一味草药有多种叫法,就是金属,矿物,都有多种名称,而这样的名称,还都是官方使用的。

所以韩冈要推行名词规范化,铅就是铅,不能说成是黑锡。黄铜就是黄铜,不得说成是俞石。在军器监中如此,在铸币局中如此,还有本草纲目编修局,给天下生物编订纲目,填充生物树的行动,本质上也是规范化的一种形式。

“好了。”韩冈将小小的钱币还给谭运,“以后铁钱都改为这种夹锡钱好了。提议之人,依例赏赐。”

谭运低头应诺,却没有立刻告退。

“怎么了?”韩冈问道。

“是有关局中主簿贺铸之事。”

“他怎么了?”

“贺铸他今曰又跟人争吵起来了,喧哗院中。”

韩冈知道铸币局中有个贺铸,是太祖贺皇后后人,还娶了宗女,所以有个官身。之前是从徐州宝丰监调过来的,说是他通文墨,擅诗赋,适合做主簿。韩冈见过他几面,长得挺特别,或者说,有些丑陋。其他的,就是有几次被人上报,说他不能和睦同列,又不通职事。记得上次去监中有人说过,贺铸对铸币一无所知,之前在宝丰监,同样是不理监中公事。

韩冈对他,也就这点映象了。本来就没什么好感,现在听谭运一说,就越发的感觉这是匹害群之马。

没有才干其实也没关系,如果能与同事都能和睦相处,铸币局中不会没有这种人的位置。一架机器没有润滑剂也运行不了多久。有些人看着不做事,但他在人群之中,起到的却是润滑剂的作用,能让一个部门稳定的运行,同样是不可或缺的人才。但一个与同僚都相处不来的官员,又没有才干,那留着他还有什么用?

如果是技艺高超的工匠,韩冈很乐意与他见个面。如果是对有任何合理化的建议,更不会吝啬爵禄赏赐。但一个擅诗赋却不擅公事,觉得自己怀才不遇,跟同僚都搞不好关系的小官,韩冈觉得没必要让他屈就在铸币局主簿的位置上了。

韩冈想了想,然后摇头。谭运并不是第一个抱怨贺铸为人的铸币局官员。

“谭运,你觉得监中谁人合适接掌主簿?”

谭运听着心中一惊,忙道:“宣徽,这贺铸文采很好,精擅诗词,就是脾气不好。但小人并不是要夺他的官,只是想请宣徽能够训斥一番,让他认真做事,与同僚和睦相处。”

韩冈听着更是不快。文采好就高人一等,这是他一向都很反感的风气。能否做实事才是衡量一名官员优劣的地方。贺铸在铸币局的工作不合格,难道就能凭着文采得到原谅?

“文采好应该去考进士才是。能作诗文,再通经义,一榜进士不难。差一点就考刑法科,拿个出身也行。承祖辈余荫,却不思进取,此辈何足道?”

“但贺铸娶了宗女。之前还有说让他转文资的。”

“他又没功劳,转什么文资?”

虽然韩冈不喜欢现在重文轻武的风俗,但既然东班序列的确是在西班之上,韩冈也不会矫情的装作看不到这一点。没有功劳,又没有能力,凭什么转文官?

“这件事就先放着。”韩冈沉着脸说道,“如今已是入秋,再过两月自有磨勘考课等着他。黜陟幽明,到时候自然会见分晓。”

