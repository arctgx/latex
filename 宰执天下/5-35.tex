\section{第四章 惊云纷纷掠短篷(九)}

【很对不住各位。断更了一天。】

目送韩冈离开崇政殿,赵顼的眉头始终没有松懈下来。

韩冈会为种谔火烧火燎的跑来请求入对,并说事关军国重事,让赵顼心中也不免生出一丝隐忧。

但韩冈请求收回早前发出的诏书,赵顼却万万不能答应。

他可没颁下许种谔便宜行事的诏书。种谔这一次违抗军令,为争功抢先出征,几坏朝廷大事。此风如何可长?

强令种谔回兵,的确会伤了鄜延路近十万大军的士气,但只要粮饷充足,士气这玩意儿而还是很好鼓动的。赵顼相信到了灵州城脚下之后,鄜延路的士气不用耗费唇舌去鼓动,就能自己冒出来。

而默认种谔的行径,则是会给其他几路一个极坏的榜样,到时候人人赶着出兵,却不管有没有做好准备,那么结果只会更差。

两边都有坏处,但两害相权取其轻,赵顼权衡一番后,没有任何犹豫的便下诏严令种谔回师。就是韩冈来劝谏,也无法改变赵顼的想法。

但注视着韩冈步出殿门,赵顼心中隐藏的担忧却变得沉重起来。

韩冈毕竟是西北出身,论起对西北军事的了解之深,朝中现在就唯有他一人而已。韩冈如此心急的要为鄜延路的辩解,赵顼都不能咬定是他错了。以韩冈之前反对急进的态度,也不能将他今天袒护种谔之事归结于私人交情。

赵顼头正疼着,现任御史中丞李定已经在殿外通名了。

依照今天入对的次序,方才赵顼就该召见李定了,韩冈说是事关军国重事,才抢前了一步。

李定进来叩拜行礼之后,就呈上了一封折子:“陛下,这是近两日台中审问苏轼的口供。凡前日所劾种种,其皆已服罪。”

赵顼随手翻了翻,不用李定详细解说,只看了供状,就已经怒气勃发了。

之前御史台对他的所有指控,苏轼竟然全都承认了。讽刺盐法、讽刺水利工程,讽刺免役法、讽刺便民贷,藏在诗句中的险恶用心,苏轼在御史台的审讯中全盘招认。

赵顼不是蠢人,自是明白,犯人对罪名承认得竟然这般爽快,要么是受刑不过,要么就是在掩饰更重的罪行。

“可曾用大刑?”他直截了当的问道,双眼不放过李定脸上的任何变化。

李定低眉顺眼,回答则是肯定有力:“苏轼名高当世,辞能惑众。为避人言,台中不敢用刑。”

好个不敢用刑!赵顼怒意更盛。苏轼当真名气大,连弹劾他的御史台都只敢审问而不敢拷问。

“此案必须深究到底!”因为方才跟韩冈的一段对话,赵顼情绪已经很是烦躁,现在则更深一层,“李定你给我好好的审问。审明白苏轼他到底说了什么?做了什么?又有多少人与他书信往来的,一同讪谤朝政?这些人,都给朕一个不少的审出来!”

天子的语气中饱含的怒意,能吓昏胆量小点的朝臣。李定则喜出望外。事先准备好的一肚子劝说赵顼穷究到底的言辞,根本就没派上用场,

他叩首领旨:“臣遵旨。”

赵顼虎着脸,握起拳头在御案上捶了一下,他现在完全没有宽宥苏轼的想法。

原谅臣子的冒犯,这份德量,赵顼自问也是有的。

当初仁宗皇帝被臣子喷了一脸唾沫星子,又差点被汗臭薰昏过去,回宫后还要抱怨两句。可他赵顼,过去每次召见吴充,吴充项下赘瘤臭气熏天,他回宫却是连抱怨都没有过。

因为他知道,吴充等人再怎么争,心思终究有一部分是为了国事,不全然是私心。

但苏轼不同。在赵顼看来,苏轼完全是怀着私心在发泄心中的怨气。看这个不顺眼,看那个不顺眼,谁在台上,他就看谁不顺眼,只有自己最聪明。

其实这样的人,赵顼也见得多了,一般来说,也只是一笑了之而已,赵顼跟不会放在心上。

可苏轼偏偏又是名声极广。若说韩冈在外界被传说是药王弟子,那苏轼就是货真价实的文曲星。他的诗词,人人喜爱,他说出来的话,也自然多有人信服。

这样的人议论朝政,纵使仅仅是诗词上做文章,可他带来的恶劣影响,是普通人说上一万句都比不了的。

赵顼无法容忍有人诋毁他的心血,尤其是能煽动人心的臣子。看到了供状,若说他对苏轼没有动杀心,那可是彻头彻尾的谎言。

赵顼当真想一刀下去,让所有人都闭上嘴。

他自登基之后,整整用了十二年的时间,才让大宋一步步的强盛起来。眼下的局面是他一手打造,心血浇灌,就如同亲生儿子一般。哪个父亲能容忍自己辛辛苦苦拉扯大的儿子被人污蔑?

说起变法,世人想起的都是王安石。可王安石的去留,只是一句话的事而已,他做不了权臣。

如今已经做到了强兵富国的大宋,的确王安石主持变法得来的结果。但王安石是在他赵顼的许可和控制下主持变法。赵顼在变法上投注的心血和精力不比任何一名臣子要少,而且他的冒得风险可远比任何人要高……而且是高得多。

他赵顼可是将大宋天下都押上去了。

如果变法失败了,王安石不过丢官去职而已,连商鞅那般的性命之忧都没有。可是对于赵顼来说,国事一蹶不振,自己的声望落入谷底,甚至有帝位不保的风险——为了推行新法,宗室都被他得罪干净了。

变法带来的好处,是赵顼所挂在心上的成就。所以王安石尽管已经去职,但新法依然还在稳定的运行,无他,只是因为变法是赵顼的心意。

真宗、仁宗的时候,一听到边关急报,没人会认为是好消息。不是辽人要趁火打劫,就是党项人又破关杀进国中劫掠。在那些年中,边疆一旦有军情,东京城中总会一夕三惊,各种各样的谣言总会传得遍地都是。

可如今呢,一夕三惊的是党项人!是契丹人!

大宋官军已经有实力彻底百年之患了。

在这个时候,竟然还有人敢说新法的不是,而且还传播得极广,煽动士民之心,这是赵顼完全不能容忍的。

得到了天子的全力支持,欣喜的李定起身退了出去。

赵顼端坐在御案之后,脸上的神色如同极北的冰山,与外面温暖宜人的春光截然不同。

他不会轻饶了苏轼。他不会再让反对和争论干扰朝堂。

结束了对西夏的战争,接下来就该准备对辽人作战,以图收复幽燕云中。但契丹不是西夏可比,即便会有内乱,但也照样不可轻辱。

对西夏,只要动用陕西和河东的军队,再添上几万京营禁军,就已经是绰绰有余。可攻打辽国,则是举国之战,要动员全国上下的所有力量。而在举国之战的时候,必须国中内部要安定,不能前面正打着仗,后面却突然翻了天。

为了达成自己毕生的心愿,赵顼不介意先拿人开刀。

……………………

韩冈出了崇政殿,便与与李定擦肩而过。

在拱手揖让中,韩冈敏锐的发现御史中丞眉间如有春风拂面。而韩冈不用照镜子,也知道自己现在是阴沉着一张脸。

好了,这一下身上的责任全都卸掉了,顺带还跟种家缓和了关系。

韩冈竭力不让自己心头上的轻松情绪在举止和言辞间泄露出来,但脚步还是比正常时要轻快上少许。

不过对于这一次的战事,韩冈则是越发的悲观起来。

十根手指伸出来都是各分长短,此番出阵的六路,也是各路有各路的情况。出兵多有出兵多的麻烦,数以十万计的大军,有品级的武官都数百近千,什么样的人都有。

有人智,有人愚;有人激进、有人稳重;有人爱用奇兵,有人则喜欢临堂堂之阵。不同的性格带出来不同的军队。要整合他们,并不是粗暴的截长补短,将出头的椽子打压下去就能成功的。

天子和朝臣对战争充满幻想,以为西夏就是个破房子,一脚就能踢倒。换个时代,多半就回叫嚣着三个月内灭亡西夏,投鞭断流的什么的了。

韩冈只希望最后的结果不会落到最坏的场面,三十年养精蓄锐的结果不要一朝断送就好了。

幸好王中正应该清楚这一点,也不会蠢到将所有希望都押在灵州上,河西走廊的凉州肯定不会放过。与其跟高遵裕、种谔他们的抢大饼,还不如先将自己碗里面的肉送进肚子里去。

一旦官军控制了河西,收复了银夏,即便这一次没能成功的夺占兴灵,西夏国的结局也不过再拖上三五年而已。

韩冈现在另外还是有些担心辽人。

都说辽国这一次必然内乱,却让大宋君臣的期盼许久的喜讯却始终没有消息,现在只是从回归的正旦使身上知道,辽国新任天子在太师兼太傅的陪同下,一如往年的前往鸭子河的春捺钵,主持延续了百年的头鱼宴。

耶律乙辛的决断让人心生敬意。不是每一个权臣都敢带着皇帝在国中四处巡游,但耶律乙辛却

如果辽国没有发生内乱的话,那么耶律乙辛必然不会坐视宋军灭亡西夏,肯定会出兵。或许会遣兵援助西夏,或者是干脆出兵河北,以图围魏救赵。

幸好郭逵去了河北。

有郭逵坐镇,韩冈也能放下心来,倒是可以将河北之事放在一边。

不过话说回来,这都是最坏的情况下才会发生的事。在韩冈看来,耶律乙辛能将自己的位置稳定住已经很了不起了,想要援助西夏,恐怕只能是口才和道义上的帮助而已。

不至于会到那种地步。

韩冈摇摇头,将无谓的担忧抛诸脑后,但不知为何,他的脚步确有几分沉重起来。

