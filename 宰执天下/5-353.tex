\section{第33章 枕惯蹄声梦不惊(十)}

“正是这个道理。”秦琬心如电转,表面上看起来却毫无芥蒂,笑着拍了拍手,欠身前问:“十六将军既然来此,想必府州的援军应该已经到了吧?”

这是秦琬最关心的问题。如果府州的援军到了,就是兵权都给了折克仁又如何?这一回的功劳也不算少了,一个官身少不了,曰后在边境上多用点心,迟早能升上去。眼下自然是越早解决入寇的辽贼越好,可不是争闲气的时候。

“早得很。”折克仁眯了眯眼睛,“之前麟府的主力全在胜州御寇,调回来可不容易。我之前也对这几位指使说了,折十六来此就带一个指挥先行一步。”

‘一个指挥?!’秦琬眼眉轻跳。

果然还是为了夺取兵权。不过想想也是,正常的将领在面对战事时,都不会放过扩大手中兵力的机会。只是折克仁明说身边仅有一个指挥三四百的兵马,又用这样的高压态度,难道是想打算就此逼反众人,然后趁机下手?

但更让秦琬感到后悔的是自己前面太大意了。之前直接进来,都没有注意到有这么多人在寨中。也是这段时间,来投的士兵和乡勇太多,反而就疏忽了。果然是大意不得,万一来的,而是有异心,占了营寨,自己可就是自投虎口,会冤枉得死不瞑目。

折克仁脸和心都冷着。这几个叛贼明明都已经犯了死罪,还不想着以功赎罪,竟然躲在山中,坐视忻州被围攻,只派了三五百去拖延辽军。

还有这个秦琬,连局面都控制不住,也是个废物。但凡有些能力,就当排开这些贼子,稳稳的抓住兵权。若是还有些公心,更不该离开大军,去护送一个家丁,而该去领军救援忻州——他可是忻州出来的。

至于那个韩信,折克仁更不放在心上,能说服叛军,那是仰仗韩冈的声威,与本人无关。

现在的当务之急究竟是什么?是忻州啊。是正被辽贼围攻的忻州!

若能救下忻州,整盘棋才都活了。

为了达成目的,折克仁甚至不介意解决这几个叛贼,纵然他手上只有一个指挥,而且是以急行军的速度经过了长途跋涉,才赶到了忻州的附近,也足以压服这里的五千乌合之众。

折家眼下能确实控制的编制——也就是私兵——在胜州之战后,就被一系列的人事调动和移防,降到了四个指挥。但由此也都变成超编的状态。折克仁的这个指挥,装备只比上四军稍逊,战斗经验则远胜,且是人数多达六百八十的有马步人,其实是等于两个指挥。

有了近七百精锐做核心,这样的一支军队,就不再是乌合之众了。折克仁可不信,凭自己的家室声望和地位,以及这支叛军叛而复降的惶恐内心,干掉几个叛将后,会控制不住这支叛军。

此外,虽然方才也听说了一点辽军受挫北撤的消息。但具体的情况不明,一群叛贼都没有说个明白,只是说刚刚得到消息不久,究竟是当真被韩冈击败,还是主动后撤,都没有定论。在折克仁的心中,这群人说的话,他是一点都无法去相信。

所以当他听到秦琬说起他这一次出去送人,也打探到了萧十三败退太谷的消息,而且是从辽人俘虏嘴里听说时,依然是疑心重重,眯着并不算大的眼睛:“你确定?”

秦琬遽然起身,就冲外面招呼了一下,片刻之后,两名随从便四只手拎着十几个人头进来了。都是顶心剃去头发的髡发发式,标准的辽人。

首级在厅中摆做一溜横列,秦琬站了起身,朗声道:“十二个。一队辽贼探马都在这里。”

这是之前秦琬送韩信时,在半路上发现的敌人,是突袭而收获的战果。但这一次突袭,也让秦琬所召集的十五人的小队付出了三死两伤的代价。

“这个也是?”折克仁拎起了十二枚之外的第十三枚首级,发髻虽然被打散了,但依然可以看得出是汉人的装束。

“是辽贼找的向导,也是通译。”秦琬解释道,“韩枢密的消息就是先从他嘴里知道,然后又分开来审过几个辽贼来确认。只是嫌带在身边麻烦,就只带着首级回来了。”

“你会契丹话?”

秦琬点点头,“少年时曾随着商队去大同走过七八回。身在雁门,也不可能不去学几句契丹话。”

折克仁直到这时候才对秦琬正视起来,熟悉山川地理的将校在军中永远稀缺,而且之前若秦琬没有说谎,能让一队辽军探马一个都逃不掉,也不是说着那么简单的一件事。

而且他心中还涌起一阵狂喜,辽军当真是被韩冈击败后才退回来的。能退第一次,可就能退第二次。

他看了看秦琬,又瞥了几个指挥使一眼,挥了挥手,对叛将们道:“你们先退下吧。我跟秦殿侍说几句。”

折克仁理所当然的将自己视为了主将,指挥使们则是先瞅了瞅秦琬,见他没有异议,便各自低下头,很是顺从的退了下去。

“秦殿侍。”折克仁开门见山的问道,“对围困忻州的辽贼,你是怎么想的?”

“当然是想解围、逐寇!”秦琬脱口而出。

“那你现在是按照自己的想法去做的吗?”折克仁质问着。

“要是能那么做就好了。都是不堪使用啊。”秦琬无奈苦笑,“能让他们不去帮助辽军,已经很不容易了。除了三百挑选出来的精兵,剩下的就是让辽军分兵之用。”

这群人就是让辽贼分心用的,只要还在活蹦乱跳,辽军就不得不分出很大一部分精神来提防。若能救援忻州,秦琬早就去了,但他比折克仁更清楚这一支已经失去了胆气和骨头的军队,其实是不堪使用的。

而且他并没有权力去杀人立威,也没有足够的威信去统领全军,至今为止,他也只与韩信一起,控制住了那三百多挑选出来的精兵。剩下的,则是通过影响力来维系他们聚而不散。主要作用就是充门面而已,然后坐食几处军寨及附近村庄的粮草。

“十六将军,敢问剩下的麟府援军将何时抵达?”见折克仁凌冽的眼神略显平和,秦琬抓住时机立刻问道。

折克仁道:“主力还是往代州去。只有我这个指挥是往忻州来。”

秦琬心中一阵失望:“看辽贼的样子是打算坚守石岭关,然后猛攻忻州。”

“辽贼那是白费力气。只要忻州还在,就不可能守得住石岭关、赤塘关。关键还是保住忻州。”

“忻州那边,秦琬之前已经派人进去通知了,其中虽然是颇费了一番气力。而且辽贼之中,不过之前类似的招数玩了好几次,忻州城中已经不会有人相信他们。”

折克仁又眯起眼睛:“真的有把握将消息传进去?”

秦琬沉默了片刻,然后实话实说:“……不,没有把握。”

之前萧十三只是派了几千人马围住了忻州,辽军对忻州的围困就跟渔网一样,到处都是洞,小一点的鱼虾很容易钻进钻出。但现在辽贼已经很明显的在全力攻打忻州,而且是在拼命了。主力返回石岭关之后,甚至没多加休整,便立刻就要准备攻城。之前秦琬之所以能遇上那一队探马,也是因为被派出来监视外围,以防秦琬的这一波兵马偷袭攻城大军。

“那就大张声势!”折克仁大声说道,“将辽贼为韩枢密所败的消息传出去。一旦亲眼看见了,忻州人怎么可能会不知道?代州人也必然会知道的!”

当越多的人听说了辽贼败阵的消息,反抗的力量就会越大,辽贼在代州也就越不安稳。腹背受敌的危险越来越重,纵使萧十三胆子再大,也不敢再留于忻州境内,继续围攻忻州。

“这两天韩兄弟差不多就能到太原了,韩枢密当在太原城中。”秦琬点着头,“韩枢密一来,便逼退了辽人,正是军心士气正旺,定然会赶来救援忻州。之前秦琬只因兵将不堪驱用,才不得不分散用兵。现在有了十六将军的兵马,也就有了主心骨。秦琬如何不敢奉陪?!”

……………………半个多月不见,韩信瘦了不少,但双眼神光湛然,看起来更加精明干练了。

韩冈上下一打量,不由得点点头,笑道:“有模有样了啊。奔波了这么些天,有什么感受?”

韩冈的话像是长辈对家中晚辈的话,韩信却是正正经经的跪下行礼,“韩信有负枢密所托。”

“你要真有负所托,辽贼就不会退到了百井寨。”

百井寨北控石岭关路,与石岭关二寨相应援,加上近处的赤塘关,三座关隘相互支撑,如果都有着足额的驻军,即是其中任意一处被攻破,仍可以维持防线,并有很大机会夺回失陷的关隘。

辽兵分兵驻守百井寨是意料中事。但整条石岭关路,却只守百井寨,就代表辽军的指挥已经放弃了峡谷中大半道路。而正当道路的百井寨既然被萧十三派驻大军,想必就是辽军预设的关键点。

百井寨当道而立,占据居高临下的位置,控扼石岭关路而修筑。地势极险要,但官道并不经过寨中,而是从主寨下经过。当然,百井寨还有几座附属的寨子。各自守住周边要道,守护百井寨主寨。

就像药有君臣佐使,大部分的边寨,或多或少都也有附属的小寨,或是堡垒什么的。由此方能让一座城寨,成为附近百里的中枢。控制田地、工坊和关卡。

“终究还是要打过去的。迟早要打,越靠南面就越安全。与其在平原上与辽军相抗,还不如就在这一条山路上与辽贼决一雌雄。”韩冈放声笑着,他只确定整件事该不该做,而他的幕僚们则就去讨论该怎么做。

“城墙不易夺。即便夺下来,也少不了要砸进去多少兵马的姓命、”

“我们可以把太原城的八牛弩运上来。”

“用破甲弩压制住城头,拿八牛弩的铁枪钉出上城的台阶。”

“终究是要厮杀一场。”

军议上的混乱,韩信视若无睹,留光宇看得眉头深皱,“要是给走马发回去,朝廷怕是要改弦更张了。”

开玩笑的口气说出来的话,却体现了留光宇心中真实的想法。

“不用担心。”韩冈却笑道,“辽军败退的奏报此时当已至京师。”

没加入争吵和议论的折可大和韩信没听出来,但留光宇却又觉得韩冈的遣词用字似乎有哪里不对,“奏报?”

“奏报。”韩冈点头,“我可没脸说是捷报。”

