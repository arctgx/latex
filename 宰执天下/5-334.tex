\section{第32章 金城可在汉图中(13)}

“井陉道出现辽兵了?”

郭逵的声音很轻,却又足够沉重。这意味着河东局势仍在持续恶化,而丝毫没有好转的迹象。

郭忠义的声音更为沉重:“是的,寿阳城外已经有辽兵出没。平定军的人现在就在外面,等候大人接见。”

寿阳是太原辖下,过了寿阳还有一个平定军【今阳泉市】,直到过了平定军的承天军寨【娘子关】,才是河北地界的真定府。

寿阳和平定军都是属于河东,但面临敌侵,却遣人来河北寻求支援,更进一步印证了太原府的危局。

“现在连井陉中都有辽军出没,可见榆次县多半已落入了贼手。”郭忠义双眉间聚起的沟壑,已被忧心所填满,“代州陷落,太原门户洞开,河东可用之兵已经不剩多少了。河东若有个万一,河北可也难保全这是唇亡齿寒啊。”

郭逵摇了摇头,“韩冈今早遣人送来的信,你难道没看?他既然那么有把握,何须为他担心?”

郭忠义眉头皱得更厉害,正想再说,外面的亲兵正好进来禀报,“枢密,丰、谷两将军求见。”

丰祥和谷维德都是自河东领兵来援的将领,为太原军中的正将。这时候求见,到底为了什么郭逵也能做大心中有数。

之前他将河东的战况瞒了七八天,现在终于瞒不下去了。

两名将领被引到郭逵面前,行过礼后,根本就不站起身,直挺挺跪着。

“你们想要做什么?”郭逵几十年的军队不是白混的,这两人到底想的什么,他一眼就能看得出来。

丰祥和谷维德对视一眼,齐声道:“末将所领部众,都在担心家中老小。太原局势危殆,不免会有人担心,不知他们留在太原的家人是否安康。”

“想必新任河东制置使,也就是新任的枢密副使,两位都认识吧?”郭逵没有回答问题,而是接了一句反问。

丰祥和谷维德当然知道他们当然知道。在听说韩冈将去河东坐镇,下面的军校士卒很多都兴奋,都说有了韩冈,河东就安稳了。可是现在看情况,就是药王弟子也不顶用。

两人各自眼里都透着狐疑,不知道郭逵为什么提起韩冈。

“这是他的信。”郭逵将韩冈今天所写的信函特意拿了出来,让人递给两名将河东籍的将领:“若有人胆敢抗命,乃至懈怠,可斩之勿论!”

郭逵语气森然,但两名将领却早习惯了郭逵的眼神,正要一条条的述说自己的迫不得已。只是郭逵却根本不给他们时间:“也别跟老夫说什么军心,韩玉昆在信中也是说了,你们能在河北将辽贼打得越狠,就越能逼着耶律乙辛将河东的贼寇调回去!”

一句话就将大门给关上了,让两位将校无功而返。

郭逵绝不会放人,在他的计划中,来自河东的这两万人不能有任何闪失。

从一开始,郭逵就没想过能在边境上的第一线将辽军阻截住——那根本是不可能完成的任务。他是准备以自边境到大名这数百里的土地作为战场。

边境上霸州、雄州、保州一线的兵马虽众,但任务只是迟滞辽军的攻势,打下辽军的气焰,第一要务是保守住这几个战略要点,等到辽军更进一步南侵,各军州的兵马就可以反攻入辽境。

决战的战场在大名周边,这是郭逵事先预定好的,但局势的发展,却让出乎他的意料。谁也想不到,河北没事,反倒是河东出事了。而原本会赶来支援的辽军,现在都往河东去了。

只是在边境线上的辽军,终究还有着近五万兵马,而南京道中,却还有刚刚从东京道调来的至少三万渤海、女真各部族的头下军。加上本来属于南京道的一部分戍守军队,郭逵将要面对的将是十万人马。

稍晚一点的时候,帅府行辕中的大小官员齐齐来到郭逵的面前,正式的军议是任何时候都少不了的,尤其是现在的辽军动向,以及对河东局势的应对,人人都想知道。。

不过有人更关心其中的一支辽军的,“不知枢密怎么看哪三支攻入河北腹地的北虏骑兵?”

现在两国的战场仍是拉锯在边境线略偏向大宋的一侧,大多数辽军的留守精兵拼尽了力气也没能再越雷池一步。不过千里之堤终有溃于蚁穴的道理。漫长的边境线,不可能用军队将其全数堵在国门之外,

这一段时间以来,甚至已经有多达三支,总数近万骑的宫分军攻入了河北腹地。不解决他们,官军的主力根本不可能大举反攻。此乃腹心之疾,一不小心能断送了所有人的性命。

“不要紧,有人正跟着他们。”郭逵坦然说道。

当那三支宫分军攻入了河北腹地后,郭逵立刻派出了手上大半的骑兵力量,分成六部,让他们追摄在辽军身后。不与其交战,而是紧紧跟随,进则跟进,退则同退。一幅虎视眈眈的姿态,配合着地方守军,逼得辽军不敢贸然分兵劫掠地方。

这六部官军分别跟着各自的目标亦步亦趋,轮流盯防,就像一道绳索入寇辽军的套在脖子上,随时都有其吊死在法场上的可能。

所以郭逵才这么放心,只要小心提防着那三支官军不被辽军给吞吃掉,那么他就有足够多的手段将三支宫分军最后给撕碎了吞下去。

“敢问枢密,接下来给怎么做?”

“攻打易州,这是河北救援河东最有效的手段。”

“枢密是要官军去攻易州?可那三支辽军怎么办?总不能放着吧。”

“我自会去将他们给解决了。腹心之疾必须尽快解决。”郭逵摇头道,“让广信军的李信为先锋,统领各部兵马!在本帅抵达易州之前,先行攻打。”

虽然郭逵说得隐晦,在座的文武官员哪个听不出来,郭逵这等于是将整个战役的指挥权交给了李信。

郭逵这是投桃报李。要不是韩冈在朝中屡屡相助,现在又置己身的安危于不顾,支持河北给与辽人决定性的一击,那么郭逵又怎么会不识趣,将李信给投闲置散了?。

何况李信的能力、经验、功绩,以及他现在所在的位置,和他手下的那一部精兵,足以让他就任易州之役实质上的指挥官。

而且韩冈面临河东的危局,不用担心李信不拼命。至于调到李信手下的将校,最差也有王安石和章敦在,也不愁有人但敢不听号令。

现在郭逵唯一担心的就是韩冈能在河东坚持住,若他能成功,那么自然是河清海晏、天下太平,否则,天下局面糜烂将再难挽回。

……………………

刚刚离开铜鞮县城,章楶就被人给拦住了。

一名只有二十出头、身着武服的年轻人,被带到了新近就任河东制置使司参议的章楶面前。参议官此时骑在马上,虎着脸死死盯着身量比他高上不少的这名年轻士兵。

不仅是章楶,就是他身边的将校士卒,也是一个个眼露杀机。

“这什么意思?”

“参议没明白?那就是小人没说清楚了。”这名年轻人丝毫没有畏惧之色,平静沉稳得仿佛惯历了风浪的老水手,“小人得枢密相公的吩咐,命所有北上来援的官军,可在铜鞮县稍事休整,待大军齐集时再行北上。”

章楶的眼中闪起了凶光。虽然是文臣,却有着百战武将的威势。

他是章敦新近推荐到韩冈幕中,担任制置使司参议一职,算是高级幕僚。随着第一批从京城出发的骑兵一路北上,准备赶往太原与韩冈会合。

太原局势之危殆,从一匹接着一匹南下的信使身上就能看得分明。章楶恨不能插翅赶到韩冈所在的太谷县城。却没想到,离着太谷县还有一段距离,就有人敢明着来欺他。

他抬起手,几名亲兵便将腰刀抽出了半截。只等章楶一声令,便立刻拿下胡说八道的奸细,好生的拷问一番。

只是这个年轻人声音却依然平静:“小人不是辽人的奸细,所传的话,也是枢密相公亲口所言。”他从怀里拿出一封信:“一切都在这封信上。”

章楶的亲兵接过信,上上下下看了一通,然后方才交给他的主人。

章楶比他的亲兵看得更加仔细,上上下下的查验各项暗记,直到确认了外皮的真实性,才一把撕开了被火漆封好的信封,抽出了里面的信纸。

纸上没有文字,只有一排排奇怪的符号。但这就是信,这是韩冈和威胜军事先约定好的密书。

两边事先约定好用同一本书的同一版,以页数、行数和字序来代替具体的文字。满篇尽是数字,不拿到原本,根本就解读不出来,辽人的奸细自然无法伪造。

而且这些数字,完全不是文字,而是一些数字的代码,是码头上写在麻袋或是箱笼上的记号,也就是所谓的码子。辽人的奸细也不可能有这么偏门的常识,即便是章楶本人,也只能认出这是什么,却不知道哪个码子对应哪一个数字。

看了这份密信,章楶顿时便信了五分。不过真正让他释疑的,还是这名信使的身份给随行的人叫破了。

“参议,那是韩枢副家的家丁,小人曾跟他打过照面。”

章楶眼神一变:“你是韩枢副的家人?”

“小人正是。”

