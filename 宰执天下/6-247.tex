\section{第33章 为日觅月议乾坤(五)}

章惇闲定的坐在椅上,听着侄儿的回报。

“不愿?可说了为何?”

“没有。”章缜摇头,停了一下,“可能是意气。”

章惇摇了摇头,拍了拍扶手,“都做到了这个位置,哪里可能是意气?”

“那是为何?”

章惇呵呵笑道,“韩三与他岳父为了道统明争暗斗十几年了,哪里可能低头认输。”很快,笑声又低了下来,“左右他也看不上这个位置。”

“什么?”章缜没有听清。

“没有。”

派了章缜出去后,对韩冈可能会有的反应,章惇也有所预料。

韩冈要是遣人去跟王旁议亲,摆明了就是贪恋权位,在王安石面前平白的低了一头。相反地,王安石若是让自家的孙女做了皇后,新学就要输气学一头了。

如果韩冈当真对学术比权位更加看重,现在的反应倒是十分正常的。

不过当初韩冈所说的那些话,章惇现在依然记得很清楚。韩冈当真是比现实更加重视自己的理念,日后怕是会一步步实现他的初愿。

只是若韩冈当真被王安石借着孙女扯了后腿,他还要靠什么手段去推行自己的想法?

这可就让人捉摸不透了。

见章惇沉默不语,章缜担心的问道,“七叔,可还要紧?”

“没事。”章惇展颜笑道,“有王平甫在江宁,他还想进两府呢。”

语气轻松的将侄儿打发了出去,但章惇心里却不觉得王安礼能拦着王安石剑走偏锋。

王安石和王安礼关系并不好。

章惇曾听说王安礼做了江宁知府后,就初上任时与王安石见了一次面,之后便再无往来。

前几个月,他还听说王安石微服出游时遇到带着整套仪仗出巡的王安礼,直接躲到路边的民家中,不与王安礼打照面。

王安礼的放荡形骸,一直为王安石所不喜,尤其是在王安国的丧期,王安礼还召妓饮宴,更是在王安石心里留下了极深的芥蒂。

最重要的是,王安礼对新法的态度一直暧昧,更是让两兄弟之间的嫌隙越发的深了起来。

若是王安礼得知消息之后到半山园去闹,说不定反而会推了王安石一把。

当年熙宗皇帝留下的情分,王安石更不会忘记。

那时候,还真的麻烦了。

……………………

送走了客人,回到后院的韩冈跟妻妾说了方才的会面。

“官人,到底是为何不愿?”

严素心狐疑的问道,这的确是个解决问题的好办法。

“是官人不想向老平章低头吗?”

韩冈慢条斯理的喝了口茶,“你们可知道,三代内近亲生下的子女,先天疾病和痴愚的几率要比寻常人高出十倍。”

“此事当真?”周南惊道。

虽说韩冈还有不是王旖所生的儿子,不会有血缘上的牵连。但王越娘是嫡女,韩冈若让庶子去求取,可就失去了姑表结亲的意义了。寻常人家将女儿嫁给表兄弟,不就是图了有一个嫡亲的姑母、姨母做姑姑,能得到照顾吗?

现在按韩冈的说法,这样的亲戚连结亲都不能。

“我骗你们作甚?”

“啊!”严素心突地轻叫了一声,“可冯四叔和李二叔已经结了亲,薇薇和肃哥。”

“这是最近保赤局才通过病历统计出来的,事前哪个知道?”韩冈摇摇头,又道,“四弟家里就是生了有残疾的孩子,也养得起一辈子,不用担心,事后下不为例就行了。”

只是单纯的几率问题,韩冈也不担心。

倒是江宁那一边,才让人要多费心思。

如果从阴谋论的角度出发,这件事极有可能是想打破章、韩体系的官员做了幕后黑手,不过也可能只是一个小小的意外。

但不管怎么说,当信交到了王安石的手中,王家的越娘就有很大可能成为母仪天下的皇后。

韩冈虽不记得他前世的历史书上,记载了王安石成为外戚的史料。可在这个面目全非的世界,一切已变得皆有可能。

就不知道王安石会不会舍了面皮了。

“可要是这样,越娘做了皇后该怎么办?”

“成了也没关系。”韩冈看得很开,“多个做皇帝的内侄女婿,这不是好事吗?”

……………………

夏去秋来,随着日照的时间渐渐缩短,天子选后一事,正顺利的进行着。

朝廷向上千家庭发出了问询信函,总共得到了五百余名候选。被派去查看的宫使,最终从中选取了八十余适龄少女入宫受训。

这些少女,皆是出自高门元勋之家。

其中最让世人惊讶的,便是王安石嫡亲孙女的入选。

这意味着,临川王家从此由书香门第转为勋贵世家,王安石的名声由此大受牵累。士林中为之沸沸扬扬数月之久。

此外士林中也有传言,这是王安石不想让韩冈继续留在相位上而施展的绝户计。

只是在最终结果出来之前,韩冈的地位依然不会受到动摇。

不过,在这一次的选拔中,脱颖而出的并不仅仅只有一位平章孙女,还有一位枢密使的孙女。

进入宫中之后仅仅一个月,两位少女便渐渐走入世人的视野,成为最有可能被选为皇后的候选者。

一个是一开始便备受看重的王安石之孙,王旁之女。另一位,则是已故的枢密使狄青之孙,东染院使狄谘之女。

狄谘是仁宗朝名将狄青亲子,如今正在河北担任钤辖,驻兵定州。王安石的孙女入选,使得京师一时间都认为皇后人选已经定下,但随着狄家女从定州抵京,却渐渐的有后来居上的声势。

连韩冈都不免感到惊讶,王越娘的优势太大了,而狄家女是怎么做到与她平起平坐的?

待妻子从宫中回来后,他便有几分好奇的问道:“狄家的女儿如何?”

王旖道:“前次不是与官人说了吗,狄家女儿相貌出众,越娘要略输一筹。”

“这为夫知道。为夫问得是其他方面,品性举止谈吐,德、言、功。”

相貌绝艳这个消息,狄家女进宫前就有传说了。

狄家诸子的相貌遗传其父狄青。狄青生前上阵,都要带一个铜面具,免得太过俊秀的相貌为敌人所轻。狄青的孙女,论其相貌来,自然也是极为出众。

王旖之前从江宁回来,送侄女入宫的时候,曾经看过狄家的女儿,相貌的确超凡脱俗。

回来后就感叹说,不仅侄女儿比不上,就是遗传了周南七八分相貌的金娘,也比不上她。

韩家的宝贝女儿虽然继承了母亲的明艳,但也遗传了韩冈略嫌刚硬的眉眼,所以以如今的审美观念,要输了她母亲一筹。但能在相貌能胜过金娘,可也是凤毛麟角,京师上层的小圈子里,金娘算是能独占鳌头。这一回让王旖都承认不如,当真是令人惊讶了。

“相貌另说,狄家女儿的性格、举止,也都出类拔萃,看着就惹人怜。才学虽比越娘差了些,但宫中选后,也不注重这一点。而且武将家的女儿,体质也好,听太后说,是个好生养的。”

还有一点王旖没说。家世上,王越娘也占不到太大便宜。

近二十年来国势大张,南征北战,辟土灭国,名将层出不穷。狄青的功业与之相比起来,已经变得很不起眼。

但他毕竟是真宗之后,唯一一位做到枢密使的武将,而且还是军班出身,自卒伍而至节帅。加之壮年猝死,也让人惋惜不已。家世上,狄家女也不会输给王越娘太多。

“如果官人反对越娘,那多半就是狄家女儿做皇后了。”

韩冈摇头,不置可否。

他宁可给天子安排一个文臣宰辅家的女儿,也不能是武家的。

枪杆子里出政权,这句话,韩冈须臾不敢忘。

狄家的女儿做了皇后,危害可比内侄女做皇后要大得多。

“太妃怎么看?”韩冈问道。

“太妃好像更喜欢越娘。”

王旖明白,朱太妃是看上了王安石的身份。只有外有奥援,才能让赵煦的地位稳如泰山。否则太后和宰相联手,轻而易举就能换个新皇帝。

“官人呢?”

“若是狄家女入选,还不知道种十七那边会怎么编排为夫呢。”

狄青、种世衡旧时有瑜亮之争。在西夏尚存的那段时间,种家一直比较敌视狄青,如今才放下了。

韩冈与种家关系紧密,王舜臣、李信乃至赵隆,这些西军新生代中的领军人物,都与种家交情匪浅。而种家本身,三种之名威震天下,其中种谔更是在宫变之后一年就升任节度使,做了三年殿帅之后,又出镇河东,厉兵秣马等待伐辽的时机。至于种朴、种建中、种师中,也都为军中中坚,

故而熙宁之后,种家在西军的地位急剧蹿升,如今不仅是西军第一的将门,更是禁军第一等的将门世家。将狄家彻底踩倒了脚底下。

若是狄谘家的女儿突然做了皇后,种家心里肯定要不舒服一阵。

但王旖知道韩冈只是在胡扯,他当真决定站在哪一边,绝不会是因为要顾及种家的心情。

“官人,你当真希望越娘做皇后吗?”王旖正色问韩冈。

“做了皇后的姑父,可就是天子的长辈了,为夫怎么不愿?”

韩冈开了个玩笑,只是见妻子一派正经严肃,也收敛了笑容。

“岳父会走这一条路,是意料之外,却也在情理之中。越娘,为夫不会赞成,也不会反对。”韩冈这段时间以来都是这个态度,现在依然如此,“至于狄家女,她先天不足,最后一关,她过不去。”
