\section{第23章 铁骑连声压金鼓(二)}

【第二更,求红票,收藏】

王舜臣紧紧咬定牙关,两腮上的肌肉硬得发僵,耳中几乎都能幻听到臼齿碎裂的声响。亲兵从旁看着,惊见从他嘴角处都沁出了血来。

王舜臣知道,这是对他面释放俘虏时打折右臂的报复,但骨折可以长好,而砍断四肢,人还哪有命在?只恨他方才一念之仁,没下狠手。早知道放出去的蕃贼会出这等主意,他直接就下令将他们剁了祭旗!

城头上的守军看着蕃人得意洋洋的残杀被俘的袍泽,无不感同身受,没有一人忍耐得住,纷纷向王舜臣看来。

“侍禁!乔都头已经准备好了。”一个亲兵过来提醒道。

王舜臣抬起手,便要下令让在城门口准备好的骑兵出城救援。但不知为什么,他总觉得在敌军的阵列中,有些让人难以觉察的微妙动作,让他的手举在半空中,出城二字也卡在喉间,怎么也挥不下去、说不出来。

王舜臣盯着敌军军势狠狠得看了又看,这其间又有一名战俘被砍下了四肢,惨叫声传遍了战场上空。

多少人焦急的等着王舜臣的命令,但他最终还是把举起的右手收了回来。他不能冒险,很明显的陷阱他不能踩进去。不过王舜臣现在很明白,若是让继续让蕃人在战场中表演下去,对他麾下将士们的军心士气打击太大,而自家的威信也会一落千丈。

他紧紧攥起拳头,喝道:“神臂弓在哪里?!”

六十步的距离,是神臂弓大展神威的场地。从一年前神臂弓开始配发关西,开始为各军换装。经过一年的时间,这件神兵利器已经逐渐普及开来。王舜臣手上就掌握着整整一个都的,装备了神臂弓的弩手。

很快,一队弩手上来了,他们手上都提着一张四尺多长的重型弩弓——在军工技术独步天下的大宋,也被视为军国之器,由天子亲自命名的神臂弓——其弩身前端带着的铁质脚蹬是神臂弓有别于过往弩弓的最大特征。

而就在这段时间里,战俘们一个接着一个被斩下四肢,不论他们是恳求,还是破口大骂,都没有改变他们的命运。手段变得越来越熟练的蕃人,甚至有能力在斩下四肢时,用绳子扎紧他们的伤口。看着被斩去四肢的战俘,用尽最后的气力像条肉.虫一样在地上蠕动,围观的蕃人们无不拍手狂笑。

“给他们个痛快!”指着六十步外的战场中央,王舜臣的声音有着说不出的疲惫。他不能踩进敌军的陷阱,但他也不能眼睁睁的看着被俘的袍泽被凌辱。要死也得痛痛快快的得个全尸,被零碎的切割成一块块的,做鬼都没法投好胎。而且还有那些个正得意的蕃人,就算他救不了自家的弟兄,王舜臣也要他们陪着一起上路。

率领神臂弓队的都头发了楞,动了动嘴唇,欲言又止。王舜臣的一个亲兵在旁小声的提醒他,“侍禁,不是要救……”

“救!救得回来吗?!”王舜臣旋风般的回转过来,指着还在被折磨着的袍泽,眼中尽赤,如鬼神一般的气魄压得众人不敢再劝,又是一声暴喝:“快!”

若非王舜臣此前已经在众兵面前展示过了自己世所难匹的神射,用明明白白的实力确定了身为主将的威严,他现在的这个命令,肯定会被一群人涌上来劝得收回去。

但眼下,王舜臣的号令没有人敢稍打折扣。一群神臂弓手在城头上排定,用腰腿的力量上好弩弦,听着一声号令,一起扣下牙发,将安放在槽中的短矢怒射出去。

不同于长弓射击后总是传着袅袅余音的弓弦声,神臂弓力道极大,扳开牙发后,就是嘣的一声短促暗哑的鸣叫。数十上百声连在一起,便如同巨兽的怒吼。将六十步外的一切,无论是蕃人还是汉人,全数钉在了地上。

不论好歹,尽数杀光,王舜臣的决断让金鼓号角一刻未停的战场上,刹那间化作一片死寂。但下一刻,西夏一方洋洋得意的号角声重又响起,仿佛得胜了一般,欢快的奏响着。不过可能是慑于神臂弓的威力,蕃人并没有伴着响起的号角再来攻城。

尸骸处处的战场上,突然有了一阵难得的空白,除了前几次来攻城时倒在地上幸而未死的吐蕃人,在没有其他活物。不过在蕃人们做出了残杀战俘的行动后,每一个还能动弹的身体,都会被几支长箭从不同角度给贯穿,原本还有机会逃回去的吐蕃伤员,转眼就给杀了个干干净净。

尽管箭矢的数量已经严重告急,但王舜臣还是任由他的士兵在这件事上浪费一点,他们心中的怒气必须得到发泄,否则就会影响到士气,让军心不稳。

而王舜臣本人,心中也有一团火气要出来:“让乔四先去休息,晚上我用得到他!”

………………………………

野利征坐在青唐部族长之弟瞎药的主帐中。

不同于由禹臧花麻派出的、在木征那里吃了个软钉子的同僚,身为党项豪门野利家的重要人物,又在朝中有着一个团练使职位的野利征,在瞎药这里得到了最大的尊敬。不但坐在帐中最尊贵的位置上,有瞎药和他帐下的耆老一齐奉酒,甚至还能看到汉女的歌舞——这是一个有求于瞎药的商人花了大价钱买来的。

野利征对此安之若素,他可是受了君命来此。比起他的地位和身份,甚至在小小的青唐部中连族长都还不是的瞎药,在他面前本是连站的资格都没有。能让瞎药坐下来说话,是他野利征的为人宽厚,也是因为他想早点完成他的任务,回到山北去。

野利征本不想出来跑腿,他最想的是领军作战,而不是出来给人当说客。但他偏偏不幸分在禹臧花麻手下,在卓罗和南军司任官。禹臧花麻这个吐蕃汉子娶了宗女,被封做驸马,又坐拥禹臧家的十万户口,就算到兴庆府,太后国相都要以礼相迎,不是他野利征能开罪得起。

无奈的野利征只能打起精神,跟瞎药周旋,封官许愿的话说了一通。按照出来时禹臧花麻给他定得底限,无论如何都是要保证,不能让瞎药的军队出现在渭源城——即便是瞎药投效了过来,也不能让他出兵——对于禹臧花麻的担心,野利征能够理解,瞎药从背后捅死董裕,禹臧花麻不再提防他一手,那就是太蠢了。

‘幸好他识趣。’野利征在欣赏歌舞之余,用眼角瞥了一眼瞎药。他今次来见瞎药,没有说上几句,青唐部族长的弟弟便毫不犹豫的接了官状,做了大夏国的一名钤辖。虽然有说过要帮忙出兵,但被自己拒绝后,便绝口不提。

至此,野利征便是算是完成了他的任务,便能安安心心的坐下来看着歌舞。他曾听说过,东朝派在秦州的专门负责招揽吐蕃人、名叫王韶的官员。把招揽了青唐部族长兄弟作为他最大的功绩报了上去。现在野利征真想让王韶看看他所招揽的这个吐蕃人究竟是什么德行,这卑躬屈膝的奉承样子,相比王韶也很少看见。

瞎药倒是没觉得他的行为有什么不对,作为夹在宋夏两强之间吐蕃人,两边通吃才是正常的做法。对上衣食父母,他也不介意弯弯腰。而在他的心中,其实也隐隐的对王韶试图维持他和他兄长之间的平衡很是不满。

而且瞎药想要更大的地盘,更多的子民,这些汉人都不会给他。七部余族,王韶宁可补充给张香儿那个废物,也不让青唐部从中分一杯羹。

一名亲信匆匆走进,用着吐蕃话向瞎药说了句什么。瞎药脸色顿时一变,紧张得向野利征看过来,看见野利征利一直在喝着酒,便也用吐蕃话回了两句。

野利征竖着耳朵,暗自冷笑。瞎药应该想不到,他可是会说吐蕃话的。

‘原来是有名的韩冈来了!’

…………

被人恭恭敬敬的迎进了寨中。韩冈第一眼就看到了,身上的服饰完全不同于吐蕃、也不同于大宋的一群人。

“是西夏人!”韩冈的一名亲卫低低的喝了一声。

智缘脚步一停,吃惊的望了过去,看看那群人,又回头看看韩冈。

“的确是西夏人。”韩冈板着脸,点头确认道。

从那群人中投过来的视线上看,他们也是大吃一惊的模样。这也是因为韩冈在青唐部中人望极高,谁都不会拦着他,一见到就把他往主帐请,才会就这么给撞上了。

智缘脸色霎时变了,不过他好歹也是经过多年修行,念了两句般若心经,心情就逐渐平复下来。转头看向韩冈,却发现他则是面带微笑,心神凝定的样子。

看见韩冈的神色,智缘暗暗放下心来,只是他并不知道,韩冈越是怒气勃发的时候,便笑得越是灿烂。却只当韩冈现在的神情是胸有成竹的表现。

