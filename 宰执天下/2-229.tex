\section{第34章 山云迢递若有闻(十)}

‘想不到竟然有结吴延征……’韩冈暗中遗憾,‘可惜没斩了瞎吴叱!’

今次蕃军来袭,本来在渭源堡外,就留下了一百五十多枚首级。而刘源夜袭蕃军,至少也会有两三百斩首,说不定四五百都有可能。加上一名敌军主将的脑袋,说起来,的确可以算是大捷了。

可是韩冈想的是全歼!

两千多名蕃骑,若是能留下大半,这份功绩就足以让他晋升朝官——而不仅仅是京官。如果是在白天交战,在前后堵截的情况下,韩冈的计划很大机会能够实现。

让人遗憾的是,刘源是在夜中突袭的敌阵。

受到骑兵偷袭后,就算是蕃人,只要他们稍有头脑,都会往山上或是密林中跑。贼人跑得漫山遍野,刘源有天大的胆子也不敢冲到山坡上追击。以他手上不到三百的军力,能把道路上的敌军给荡清,就已经算是很努力了。

而逃入山野的蕃人,肯定是抄小道回去洮西。如果是大队人马,堵着两处渡口就行了,他们也没处躲。但十几人、七八人,甚至单人匹马的情况下,找个山坳躲几天,找个部族投靠两天,风声收了再回洮西,都是很容易的事——散了的鸭子,想要捉回来,哪有那么容易?

只是从刘源的角度来看,他的选择并没有错。拖延敌军只有苦劳,而斩将败敌才是大功。如今的军中,不论换作谁人,都不会为了成全别人,而牺牲自己的功业。更别提刘源他们这群被流放的叛贼,正盼着用功劳洗刷自己过去的罪衍,好给儿孙留一条上进的出路——他们当然更不会放弃这么好的机会。

而韩冈也没想过刘源有这般能耐,敢领着不到三百名、配不齐战马的士兵去夜袭数倍于己的大军。他现在想想,自己在军事上的确有些保守。当然,在常胜和不败之间,韩冈会做出的选择是唯一的。

不过胜利终归是件好事,即便是韩冈,也不能说他之前的计划就一定能够成功。十鸟在林,不如一鸟在手,有了眼下已经确定的斩将斩首,也就别去想计划中的全师全歼了。

王中正比韩冈看得开,不像韩冈有着患得患失的想法,他为着平白到手的功勋,而兴奋得脸上生光:“尔等以微薄之军,败数倍之敌。前有卫营守城,后有斩将败寇,明明之功,难有一见,吾当上书,为尔等向天子请功!”

韩冈连咳嗽都来不及,王中正就把不该说的话说了出来。见着信使满脸惊喜的跪倒叩谢,王中正又站着生受,他只能出言转圜,“广锐军虽有旧时之过,犹有今日之功,只要尔等忠勤于国,终有一天,有洗脱旧过的时候。天子洞烛内外,公明严正,断不会绝了你们改过自新的机会。”

韩冈的话算是补救,不让刘源等人,听到王中正的话后,有着过多的期待。

韩冈望了望厅外,夜色依然深沉,但星月都已向西落去,计算时间,再过一个多时辰,天就要大亮了。

“王君万!”韩冈点起了脸色犹然阴沉的渭源堡主,“你到外面整顿蕃人兵马,天亮后去将刘源给替回来。来袭的吐蕃人只是逃散而已,人数尤众,随时可能会重新集结。你要穷追猛打,不得让他们再有整军的机会!”

韩冈算是有功大家分,不能让刘源等一干广锐将校把功劳都赚足了,同时也担心刘源他们经过一天一夜的鏖战,会乐极生悲,出什么意外。

王君万虽有些不情愿捡这个便宜,感觉实在有些丢人。可军令如山,他不敢稍违,便跪下接令:“末将遵命!”

倒也不说其他废话,转头就出门去了。

韩冈转对王中正笑道,“还要快点向临洮派去信使。王、高二安抚,听到这个消息当能轻松一点了。”

王中正此时脑中还是被大捷的消息冲击得晕晕乎乎,韩冈说着什么,他都点着头,“是,是,韩机宜说的是。”

自到了通远军后,他什么都不用做,只要等着功劳就好。就像坐在梨树下,等到梨子熟透后自己掉下来。而在罗兀也是一样,都是坐享其成。可几次功劳下来,他王中正在天子面前,便是宫中首屈一指、精于兵事的中官。

真宗朝的秦翰秦仲义,是宦官中名将,其大名至今流传在宫中。秦翰北抗契丹,南平蜀乱,西定党项,一生征战,身披四十九创,功业不再曹玮等名将之下。可他到了晚年,一逢阴雨便浑身酸痛,最后暴卒于宫中,哪比得上自己这般,找个好地方坐着就是了。

这才是聪明人的做法!

王中正得意之情充满了胸臆。

……………………

天色已经发白,响彻了一夜的喊杀声与夜色一同消散在晨雾中。

刘源提着长枪,溜着战马,在伏尸满地的道路上漫行。

枪尖染着一层血凝后的紫黑,夜中的一场混战,在昨日白天的便已经开始激烈起来的血液,一直沸腾到现在。刘源也不知自己究竟杀了多少人,在浅银色的月光之下,看着眼前的晃动着的黑影便一枪搠过去,要不是事先在右臂上都绑了白布,说不定连自己人都给杀了。

前任广锐军指挥使在马背上坐直了身子,环视左右前后,数着身边的同袍。让他欣喜的是,经过了一夜混战,身后的兄弟并没有折损多少。

他安心的笑了一笑,毕竟都是老上阵的,知道如何在战场上保护自己。

刘源只有一人一马,还有一支长枪。不像他下面的兄弟,不是在马鞍后挂着一两个首级,就是横绑着几面旗帜。大战之后,战利品遍地都是,他们都是看着好的才捡了起来,兵器甲胄稍有破损,便弃之不理——只有战马、首级和旗帜,是必须一个不漏的收集,其余的战利品,有没有装起来都无所谓。

经常上阵的将校士兵,都知道该如何收集可以记功的战利品。现在每一个出战的前广锐将校都是骑在马上。原本他们还是带着挽马、驿马上阵。可是到了此时,挽马、驿马虽然还在,可是都是用在背负战利品上了。每一人皆是骑着四尺多高的战马,有的人还多牵着一两匹。

“刘指挥,下面该怎么办?”有人问着刘源。

刘源一时间,不知该如何回答,下面该怎么做,他也没有什么头绪。今次他把拖延变成了偷袭,可算是有违节度。但有功劳在手,也能支应得过去——只要说是敌军脆弱不堪,本是牵制而已,却在试探的攻击中一哄而散——那就谁也不能说他是故意为之。

正犹豫的时候,西北面的来路上,一片蹄声撼地,上千骑兵在行军时才有的威势,竟然从渭源堡的方向过来。

很快,从后方而来的军队已经转入众人眼帘。千军万马淹没了谷地,到了刘源等人的面前,才收缰止步。这是一群蕃骑,不过中间则有一群宋人。

大军停步后,一个接近三十岁,长得十分英俊的将校排众而出。

看见这名将校,刘源略一犹豫,便下马行礼,“小人拜见王堡主。”

低头看着刘源,曾经的叛将就跪伏在马前。王君万不是妒贤嫉能之辈,虽说暗恨刘源抢了自己的头功,也没想着太过为难他们,只是面如严霜的转达着韩冈命令:“韩机宜让你们回去,下面的事就交给本官了。”

王君万挺胸直背,在马背上低头盯着刘源的后背。摘果子摘得如此理直气壮,刘源等人心头隐隐怒起,一时间忘了自己的罪囚身份,并没有低头接令。

王君万形状姣好的双眉一轩,一提银枪,便要怒喝。

幸好韩冈早知会有如此僵局,派来的信使会说话,更会察言观色,见气氛僵硬起来,眼见着有争功火并的迹象,连忙站出来,对刘源他们道:“机宜知道尔等一夜辛苦,立有殊勋,所以让你们回去休息。你们的功劳都记下了,王都知听说尔等阵斩了结吴延征之后,也说要为你们向天子请功。”

这正是刘源等人想要听到的话。斩首再多,功绩再繁,如果没人上报,还是一样的白忙。以他们的罪囚身份,又都是打着叛贼的记号,若是被人吞没了功劳,连哭的地方都没有。

现在听说被派来监军的王中正已经答应要上报天子,而向他们郑重承诺的韩冈,名声又是出奇的好,过去答应的事,都一一做到,眼下说把他们的功劳都记下来,那就一定不会有错。

来摘果子的小事,大功在手,再多朝廷也不会给他们加官晋爵,还不如留给别人,省得把周围的人都给得罪。

刘源抬起头来,两边的山岭之上细细簌簌的还有人影在晃动。这是逃窜上去的吐蕃人。要缀上他们可不容易,这份功劳就转给王君万他们好了。

刘源冲王君万拱了拱手:“且祝王堡主马到功成,小人不才,不能随侍左右,得领命先行回寨了!”

