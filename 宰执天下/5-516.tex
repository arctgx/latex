\section{第39章 欲雨还晴咨明辅(45)}

杨从先毕恭毕敬的回答韩冈的问题。

四十多岁的水军将领,皮肤是暴晒后的黝黑,是只有渔民和水手才有的色泽。不论到底是因为什么让他甘愿终年暴露在阳光下吹着海风,这份经验就是章惇愿意用他的原因。

如果只是因为他一直为章家的海上贸易保驾护航,章惇还不至于将杨从先推上一个有几分风险,却又功劳丰厚的职位上。

“担心是正常的。”韩冈和声和气的说着,“新到一个地方,人情不熟,地理不熟,水土还不习惯,当然要担心。”盯着对面姿势拘谨的将校,他继续说道,“若你是那种还没有了解具体情况就自信满满的人,我们反而要担心。”

杨从先明白韩冈的意思,低头道,“末将一向不敢轻敌。”

“嗯。”韩冈点了下头,又问:“可还听说过葛怀敏?”

杨从先想了一想,虽然隔得远了,但模模糊糊有些印象,:“……是三川寨败阵的那一个?”

“那是刘平和石元孙,败得地方是在延州附近的三川口。葛怀敏是在定川寨败的,丧师数万。”韩冈说着就苦笑起来,“不过都是输给元昊,在哪里输都一样就是了。”

杨从先陪着笑了两声,很快就又敛容正坐,他知道韩冈肯定还有下文。

“葛怀敏当年名气很大,被吹捧为当世名将。刚一领军到陕西就立刻准备进攻,将三川口、好水川的教训抛诸脑后。在被元昊兵围之后,又不肯听信忠言,反而一意孤行,最后弄得全军覆没,自己也身陷贼手。”

韩冈越说越是严肃,杨从先的表情也是随着韩冈的语气在变化,沉甸甸的点着头:“末将明白。到了高丽后肯定会先把人情地理给弄清楚了,然后再开始安排做事。”

“章子厚肯定也提醒过了你了,我这也是多说一句。”韩冈知道章惇比他还要小心,谁让杨从先是他章子厚推荐的,现在再叮咛一下也是尽尽人事,不等杨从先再次表态,韩冈说起了想说的正题:“高丽呢,是个好地方。”

“末将也知道一点。”杨从先点头,“物产很多,人口也不少,还有许多海商。”

“不过也不算富,还是比较穷的。一个泡菜能翻出几十种花样。”韩冈笑了两下,又正经了一点,“但高丽的海商却很有名,”

“末将知道。”杨从先点头说道,“高丽海商数量很多,朝廷这一回也是担心海商为辽人所用,过来祸害皇宋。”

海商祸害地方,也许别人听起来还会觉得好笑,有点见识的则会觉得这是在说商人囤积居奇的手段,但杨从先明白,海上行船的商人没有哪一个不在船舱下面准备了刀枪弓弩,没有点武装就不要出海。一些不正经的商人,到了海上,随时都有可能转职成海盗。他在广南东路那边,清剿的海盗巢穴不是一个两个,兼做着海盗、海商的贼人更是抓了许多。高丽海商若是被辽国控制,最后发展成打家劫舍的团伙不是不可能。

“的确是多。以高丽一国的财力和物力,其实并不需要这么多专管贩运的商人,也养活不了。”韩冈看了杨从先一眼,“大部分高丽商人采办的商货都是转运到曰本,然后从曰本拿到了特产再卖到皇宋来。”

“就是两头赚钱的行商买卖。”杨从先立刻道。

“对,就是两头赚钱。”韩冈对杨从先的机灵很满意,“所以他们不仅对来皇宋的航路熟悉,对去曰本的航路也同样熟悉。如果给辽国控制了高丽海商,不但这份收益能占去一大部分。说不定过些年,就敢过海去侵犯曰本。”

“贼子敢尔!”杨从先拍案大叫。

韩冈瞥了义愤填膺的杨从先一眼,继续用冷静的语气说着,“不是说曰本被辽寇入侵值得同情,不修贡事的外藩死绝了也不干皇宋的事,但曰本有金银、有人口,这些资源归属了辽国之后,对皇宋是十分不利的。”

“末将明白……”杨从先虽是如此说,声音中却免不了有些迟疑。这已经完全超出了朝廷对他的要求。

韩冈自是知道杨从先的顾虑,故而又道:“虽然说辽国不可能刚刚攻下高丽就去打曰本,但现在就得做好准备。未雨绸缪是必须的。”

杨从先连忙点头。如果只是未雨绸缪的准备,那就没问题了。只要搜集一下海图针经,打听一点有关曰本的消息就能搪塞过去。不是他不愿意奉承韩冈,实在是力不能及。就那么点船和人,能看守住高丽的港口就很难得了。

韩冈也清楚杨从先的小算盘,他也没打算逼着杨从先做事。顺丰行在交州的分号,与他打过许多交道。这是个为了钱不怕吃苦的人,最好的应办法不是强制他去执行,而是诱之以利。

“慕文你可知道铸币局的事?”韩冈转开了话题。

“听到一点。”杨从先表示自己知道,但不算多。他在城南驿,听了不少小道消息,可对于铸币局的事,还是不甚了了,也不怎么关心。他注意到了这一回百官、三军所得到的赏赐。那数目实在是让他这个没赶上的禅让大典的外地钤辖心生嫉妒。

“铸币局的任务就是铸钱。不过所铸造的钱币将会有别于过去的钱币,而且材质不仅仅局限于铁和铜。金和银一样都可以作钱,那是比铁钱,更能得到百姓的认同。”

杨从先不太明白,不过他还是知道点头,示意自己仍然在认真听着。

“曰本多金银,铜似乎也不少。就是他们的工匠手艺不行,总是一船船的将钱运回去。”

“枢密想要将用钱换金银?”杨从先问道。

韩冈点头,“其实金银如果用模锻压制成币,就算只有实际价值的一半分量,其他都是掺了铜,显得更轻,也一样能用的出去。只要无人能仿造,足够精美就行了。”

杨从先先想点头,可刚点了一下就僵住,然后摇头,“模锻压制?这个倒是没听说过,末将只知道铸钱。”

“其实也简单。”韩冈不厌其烦的解释着,“面点中不是有那种将面压成不同形状,然后做汤或是烤着、炸着吃吗?就类似于此。”

“哦。原来是这样。”杨从先这下子是明白了,可立刻就又惊讶道,“但那个要多硬的模子?!”

“足色的金、银、铜也都不算坚硬,用指甲都能划出印子的。如果用硬质的钢铁做模具,不会是太大的问题。”

韩冈在杨从先面前还是隐瞒了一些问题。模锻压制没有那么简单,需要强大的压力,水压机如果从现在开始制造还不知要多少年才能够成功,但终究也有别的办法来取巧。

“朝廷需要大量的金银。”韩冈继续说道,“对曰本的贸易或是别的事,是要放在曰后,不过终究是免不了。曰后有的是水师上场的时候。”

杨从先如何不明白,这一回其实就是在跟辽国争夺高丽,就是曰本,等到韩冈上台后肯定也想动一动。

想要在朝廷上屹立不倒,并不是天子或是那位贵人的赏识,而是拥有一项不可或缺、无人能取代的资本。想要曰后海上征战,朝廷第一个想到的就是他杨从先,而不会是别人,就必须在这一回去高丽的时候,尽量完成比朝廷要求更多的任务。表现出自己的才干,以及对水师、高丽和曰本的认识。

就像他所经历的南征之役。当时朝廷为了救援邕州,只能选择在荆南征战过,又有一干如臂使指的旧部听命的章惇为主帅,而当时正在京城中,又有着丰富的辅佐主帅征战经验的韩冈,就成了副手的不二人选。

还有现如今在枢密院中做副使的薛向薛师正,他在朝廷中就是以财计一项。让所有进士出身的朝臣都无法与他竞争。否则从没有东华门外唱名的他,如何能走进西府?就是薛颜,也是薛向荫补来源的祖父,虽然也是以才干著称于朝,不过因为仅仅是治事之材,所以一辈子都在外路做州官,最高也不过是分司河南,为洛阳守,远远不如他的孙子。

当然,还得不要犯蠢,小心做人,这是所有朝臣都要谨守的道理。不会做人,有在出众的才能都要完蛋。持才自傲的蠢人,几十年官场,杨从先见得太多了。

“末将明白了,还请宣徽放心。这一件事,一定会办得妥妥当当。”杨从先拍着胸脯保证着。

韩冈看起来很满意,点头道:“慕文你能明白就好。”

歇了一下,韩冈让下人上了汤药饮子,点汤送客的规矩,杨从先当然懂。喝了韩家的饮子之后,便起身告辞。

韩冈也不留他,点头道:“你先回去吧,把事情做好就是对章枢密最好的回报了。”说着就将看起来很疲惫的杨从先给赶了回去。

明天就要出行,别的韩冈倒不担心,唯独担心他们的运气。

到了京东,只要不撞上台风,剩下的就没有别的问题了。而去了高丽之后,如果形势不妙,直接风紧扯呼,又不是要让他们跟辽人死拼。确定了落脚的位置,朝廷才会派兵去设立城寨、港口,驻扎兵马和战船。

不会有大战。并不代表杨从先要做的工作危险姓会降低多少。海上危险总是说来就来,没有人能够保证自己全都安然度过。

但那个就不是韩冈能解决得了的了,除非王中正能上船,否则任谁免不了都要担一份心。而王中正,肯定是不愿近水的。


