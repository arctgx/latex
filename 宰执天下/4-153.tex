\section{第21章 涉川终吉黄龙锁(中)}

从南门开城,到上城的土坡堆到城头,之间也就相隔了一个时辰的样子,接下里,就是一队先锋冲上东门城头,兵不血刃的打开了城门。紧接着西面、北面的城门也都陆续被打开。

到了傍晚的时候,安南经略招讨司和安南行营的衙署,已经在东门的城楼上驻扎了下来。升龙府的四座城门、加上两个水门全都已经控制在官军手中。

军中的主力,也都进驻了城中,除了四座城门各驻扎了两个指挥,便是以东门为主,将周围的一片房屋清理了出来,让官军驻扎了进去。而攻入城中的溪洞蛮军则是放开手脚来烧杀抢掠,但按照事先的约定,分给一众蛮部只有人口,城中的财物都是官军的。

“他们在外面抢得也够了,这时候还要跟官军来争?”下面的部将有人眼红不已,要不是章惇、韩冈、燕达连下禁令,保管就有人忍不住冲出去一起抢劫了。

李信刚刚从城中巡视回来:“这些蛮人倒是有桩好处,心眼实。城中挂了桃符、贴了春贴纸的人家,都没有人敢动!”

“如果官军不是能以一当百,看看他们还会守约定?”燕达摇了摇头,“大帅、副帅,军中的孩儿们强压着可不行。要么都没有,要么都有。哪里能让蛮部,官军坐看的道理?!”

“皇城周围,还有交趾兵坚守着,蛮部也没能杀过去,那些才是大户。寻常的交趾人,可有几个富庶的?”章惇完完全全变成了山寨大当家的口吻,“更别说这些蛮人还要有求于朝廷,哪里敢吞没得太过分?”

“李常杰目下已经退守皇城,这一件事,可是要先行解决!”

由于溪洞蛮军的,城中的交趾军民并没有停止反抗,从捕获俘虏口中,也得知李常杰和一干对交趾忠心的将领,带着两千多兵马,回到了升龙府北面中心处的皇城中。

韩冈则觉得没有必要,越是这个时候越是得以稳妥为主,他转过去对章惇道:“以韩冈看来,还是暂时不要对皇城进攻,先以东门为核心,清除出一片没有交趾人的安全区域来,修好营垒再说。乾德小儿和李常杰都飞不上天去,只要守住城门,巡视好城墙,他们绝对逃不了的。”

“韩副帅所言甚是,不过内城也不能放着不理。”李宪对交趾的皇城使用了一个稳妥的称呼,“只要内城不破,城中的反抗就不会停止。而且不捉到李常杰和交趾郡王,这一战也还只能说是未竟全功。”

章惇沉吟着,李宪是反对韩冈的意见,但他现在已经占据了升龙府,灭国的大功可以是说确定了,再也没人能拿走。交趾的太后也好、国王也好,都只是添头而已。

不过为了日后麻烦——比如像狄青那样,因为无法确定自焚而死的尸体究竟是不是侬智高,因而功劳打了个折扣——身兼经略招讨使和行营兵马都总管的章惇,还是想将罪魁祸首和交趾的太后国王一并械送东京城,让朝廷来处置。

“今日天色已晚,明日再攻内城也来得及,不过也得派人看好内城城门,防着有人打算乘隙逃窜。”章惇两不得罪,大局已定,要多多考虑政治上面的事情了。他顿了顿,又道:“将朝廷的檄文先射进去,让里面的人知道天子和朝廷的宽容。”

王安石亲笔撰写的讨伐交趾的檄文中,可是明着说不会追究十岁不到的李乾德的罪名,捕获到东京后,也是会好生养起来。在这时候提醒皇城中人,章惇是打着分裂李常杰和倚兰太后的心思。

雨已经停了,天上的云翳依然覆盖了整幅天幕,但云层已经薄了许多,甚至能看到丝丝缕缕夕阳的红光。空气依然是湿漉漉的,且又闷热,让人很不舒服,而这样的天气,要持续半年以上,直到快入冬时才会结束。

章惇和韩冈站立在东门的城楼上,并没有去在意身上的汗水,仅仅是沉默的俯望着城墙周长近十五六里的交趾王都。

这是一座中国式的城市,纵横东西南北的十字大街分割了城池。一条条南北、东西走向的大街小巷,又将城市分割成更小的部分。

就在夜幕将临的时刻,城内有着淡淡水雾,远近的街巷看着都有些模糊。一栋栋简单的竹屋,那是贫民所居住的区域,好一点的类似于中国的屋舍,这是城中的富户。如果往城中去,还能看见面积广大的府邸,那是官宦高门的居所。

除此以外,城中到处都能看到一座座或高或低的佛塔,虔信浮屠的交趾人,违背了多手佛门戒律,有如今的下场也是半点也不出奇。

喊杀声传了上来,可以发现正向着升龙府城的中心地带渐渐深入过去,直奔皇城而去。

不论苟延残喘的交趾人打算在退守皇城后如何顽抗,他们的结局已经注定下来。

……………………

李常杰依然是不掩宿将的威严,在城头上的失态,现在已经恢复过来。

他就是不到黄河心不死的性格,就是穷途末路的时候,也会不甘心的挣扎一番。

而且李常杰觉得还有机会。宋军兵力不足,无力控制整个升龙府,而蛮军又不堪使用。只要等到城中的宋军守卫放松下来,就能带着乾德设法逃出城去。眼下宋人放任蛮人在大越境内倒行逆施,只要等待时机,依靠当今的大越天子在手,几年或是十几年后,复国也不是梦想。

他脚步匆匆的来到天子寝宫的天宁殿,太后倚兰正搂着乾德在殿中等候。

一进殿中,李常杰就跪了下来,“太后、陛下,臣无能,没能守住城池!”

“太尉勿须自责,这都是运数。”倚兰挤出了一个惨淡的笑容,“太尉数日不眠不休,事已至此,还是先歇口气吧。”

立刻有人奉上茶水,李常杰口正干,一口气喝了下去,放下茶盏,他就急着说道,“太后、陛下,现在宋人还没有攻到皇城前,要换了衣服,臣已经安排了忠心的部将坚守皇城,为太后、陛下争取时间。”

倚兰抱着儿子没有动弹,神色淡漠得似乎什么也没听到。

李常杰心中发急:“不能再耽搁了。臣知太后、陛下难以割舍,但眼下离开,日后还能回来。如果此时不走,只能与城同殉。”

倚兰又摇摇头,李常杰正待再劝,忽然感到腹中一阵剧烈的绞痛,踉跄了两步,正疑惑着,抬头看着倚兰淡漠的双眼,一下全都明白了。

看着自己一手支撑起来的女人,突然倒戈相向,李常杰双手骨节捏的咯咯,武将的凶戾之气便欲爆发出来。只是四名膀大腰圆的宫女忽然左右前后的架住他,牢牢不让他动弹分毫。

李常杰咬牙切齿,挣扎着问道,“为什么!?”

“这大越国不就是太尉给毁的吗?为何还要再问?”倚兰幽幽的说着,“若不是太尉你把持军国之事,大越如何会有今日的结果。祸国殃民,李常杰,你难道还不该死!?”

李常杰狰狞的表情平复了下来,眼神中充满了寒意,“太后,只要臣一句话,就能让你母子二人万劫不复。”

“谁说的?”随着几声惨叫,宗亶负手缓缓走进了殿中,先向着倚兰和乾德行过礼,转过来看着李常杰。

“宗亶……”李常杰瞪圆了双眼,充满怒火的眼神瞄着宗亶,“去过宋国疆土的将帅,宋人都不会轻饶。你别以……以为……”话说到一半却没有力气再继续下去。

“别以为投降之后就能免死?”宗亶代替李常杰将他说不下去的话补充完整,他点着头,“这件事我当然知道!”

一阵几乎深入骨髓的抽痛又从腹中传来,但李常杰仍怒瞪着宗亶,看着他能说出什么话来!

“算起来我也是侬人,与七源、广源都有亲,降顺之后,我便是死了,好歹也能保住家里的香火。”宗亶冷静异常,说着自己的性命,却仿佛平常议论军事朝政一般,“大越亡国,有我一份,与国同殉也是无妨,但家中的子嗣挂心啊!太尉,这一次就请你先走一步,过一阵子,在下多半也会跟上去的。”

李常杰手指着宗亶,荷荷做声,却是一句话也说不出来。肚中的绞痛越来越厉害,不知是下了什么药,浑身上下都一阵阵剧痛。豆大的汗珠从额头上冒了出来,脸色蜡黄中泛着青气,两只眼睛像鱼一般的凸起,布满了血丝。

“太尉,你安心去吧!”宗亶立在李常杰的身边,低头看着已经有出气没进气的李常杰,平平和和的送了一句。向倚兰和乾德告辞,他便转身向殿外走去,不再多看李常杰一眼。

走出天宁殿,宗亶来到前面的紫宸殿,向着殿中的李常杰一众亲信说道,“太尉不甘城破之后受宋人之辱,已经拜别了太后、天子,自尽殉国。太后有旨,开城,出降……”

宗亶的话听在耳中,几人面面相觑,你看看我,我瞅瞅你,不知道该信还是不该信。

“怎么?想违抗太后懿旨不成?”宗亶声音沉了下来,冰冷的双眼一扫众人。李常杰已经一命呜呼,至少在此时,代表着太后和大越天子的他,说话的份量是最重的。

“……不敢!”他们的李太尉肯定已经死了,自己再硬下去,跟随李太尉共赴黄泉的,可就是自己了。原本他们就对李常杰计划并不情愿,但被他的积威压着不敢反对,但现在可就不同了。一个个都争先恐后的跪了下来,趴在地上恸哭不已,哽咽着:“臣领旨!”

