\section{第36章 骎骎载骤探寒温(上)}

景诚在码头上来回踱步,时不时的抬头望着江面,心中焦躁,“怎么还没到?”

润州丝厂被烧,连绵大火烧死烧伤士民不计其数。润州知州随即请辞,辞表虽还没批下,但已经待罪于家,州中内外大小事务,全都落到了他这个通判身上。

出了如此大案,朝廷派遣专员察访自是在情理之中,如果是普通人倒还罢了,还是宰相的心腹人,景诚尽管手上有一堆事情要做,可他还是得到渡口来候着。

“通判。”身后的从官代他抱屈,“你与韩相公、熊参政有旧,便是来的是状元郎、中书检正,也不敢慢待你,又何必在冷风地里站着。”

有旧?景诚淡淡的瞥了那人一眼,情分是用在慢待对方心腹上的吗?那有旧可就变得有仇了。当真以为自己年轻气盛,扇点风就能逗起火来?

景诚他的父亲和叔叔相继殁于王事,父亲在熊本手下战死,叔叔是与韩冈并肩作战时战死,祖父又亡于秦凤兵马总管任上,可谓一门忠烈。最重要的是他与韩冈、熊本都能攀上关系,中进士仅仅十载,便做到了权发遣通判的任上。区区一个三甲进士,却追上了一甲的升官速度,没有宰辅照顾,又怎么可能有这样的进步。周围人也都看在眼里,就连知州平素里都敬他三分。

但景诚明白,上面的照顾是念在父辈的情分上,要是自己不识趣,那什么样的情分都会烟消云散。

韩冈是什么样的人?万家生佛?别说笑了,那是一心要进文庙的主儿。为了气学能跟他岳父拧了一辈子。

韩冈要推动天下广建工厂,以安无业之民。之前江南各路,已经有人说丝厂夺民口食,朝廷都没理会,仍在一意推动韩冈的政策。现在工厂出了乱子,印证了之前的话,堂堂宰相怎么可能容忍?

现在,几百条人命大案,败坏了他的法度,坏了他的学术,管束不力的州县,还有几位贪鄙害民的工厂东主,谁都脱不了身。但板子最终会落在谁的身上,全得看这次下来的钦差的心情了。

而且这一次来的还不是别人,是两浙出去的状元郎,是韩三相公的心腹人。为了什么?难道就是为了区区几间厂子?不是!韩冈最担心的是‘国是’!‘国是’的重要性,从二十年前的新旧党争开始,便为所有官员所熟知。宗泽这一番南下,可以说是身负重任。

位卑而权重,此等新贵面前,别说现在吹些冷风,就是天上下刀子,景诚都要守在这里,不求有好处,只求一个安稳。把人奉承好了,免得恶了他,最后给牵连进去。这个节骨眼上,竟然还有人使坏,景诚没空发火,但这一个个他可都记下了。

“出了这么大的事,中书门下又遣使南下,我等不摆出个认罪讨好的作派,这不是自己往坑里跳吗。”景诚语气温和的对幕职官们说着,不管心中怎么想,对外,他总是一副好脾气,由此也得了一个好口碑,“受风也就这么一日,总比日后吹个十几年的冷风强。”

景诚一番话,几位从官听了,齐齐拱手:“多谢通判提点。”

景诚是个老好人,翻来覆去说的都是他们知道的,但这面子还是要给。换作是知州,可不会这般好心。

“知州这一回可是要摘印了。”

“杨知州他怕什么,本就要致仕了,纵使引咎请辞,朝廷也照样要给他一点体面。”

“知州不是开罪过韩相公吗?哪里能容他自自在在的致仕。”

“他怕什么?朝堂上少不了人会拉他一把。”

润州知州杨绘,十几年前便就任过翰林学士,可惜犯了大错,在琼林宴上更是坏了名声。在南方各州做了十几年的知州,自学士之位上一降再降,连议政之权都没了。这一回就任润州之后,转眼便要致仕了,这辈子都没机会再入朝堂——谁让宰相还是当年那位在华觜崖上让他丢尽颜面的韩相公?只不过,若是这一回韩冈要借机往死里逼他,还是会有人出来拉他一把,总而言之,翰林学士的体面该有还是得有的。

“都少说两句吧。”景诚回头,打断了属官们的窃窃私语,“杨公已闭门自劾,何苦再说他是非?”

“通判有所不知,”州中的录事参军对景诚道,“可知知州的自劾上是怎么写的?”

怎么写的,景诚当然知道。杨绘自己往坑里面跳的没人能拉他。

“怎么写的?”其他几个还不了解情况的官人齐声问道。

“知州与韩相公有着积年旧怨,这一回为了脱身,便在自劾的奏章中狠狠的咬了韩相公一口。”录事参军冷笑着,环顾周围,“你们觉得他能成事吗?”

除了景诚之外,人人摇头。

韩冈有擎天保驾之功,故旧遍布军中,即使是明君在位,想要动这样的权臣,也得小心翼翼,谨防反噬。如今是太后垂帘,对韩冈信任有加,一个小小的知州怎么可能动得了这位当权的宰相。

景诚则懒得与这些人多费唇舌。江南官场的风气败坏不是一日两日,说人是非、掇拾短长的事情从来不少。

这一回杨绘少不了栽个跟头,但体面同样少不了。做过了两制官,身份便于他官有别,即使是宰相,也难行快意之事。

景诚现在只担心一件事,宗泽怎么还不到?

到了黄昏的时候,所有人的耐性都给消磨光了。当派去江对面打探消息的吏员回来时,包括景诚在内,一个个都急不可耐,“宗状元可是出什么变故?”

吏员摇头,“瓜洲那边没人见到朝廷来的人。”

景诚脸色大变,“糟了!”

“怎么了?”几位官员见状,都紧张起来,。

景诚脸色泛白,“宗状元已经过江了。”

“哪里?”

“是微服。”景诚说道。

几位从官的脸色也难看起来,宗泽选择了微服查访,摆明了不信任润州的官员,带着恶意而来。一想到他这番举动是宰相在后授意,所有人都如坠冰窟。

景诚已顾不及形象了,冲着手下的官吏们大声呵斥:“还不快去查!城中各处客栈、僧舍都查清楚,有没有生人入住!”

……………………

就在润州官吏守在渡口的时候,过了江。

他奉旨南下,只把一名老仆带在身边。除此之外,仅有四名堂吏跟随。一路轻车简从,并没有仗着身份,一路骚扰州县。

他从扬州出来后并没有走瓜洲渡过江,在他的计划中,没打算先与润州的官员见面。

宗泽出任过地方,下面能做的手脚,他哪里不清楚?要是给人在半路上截住,一路作陪,接下来就只能看到下面想让看到的东西。说不定一个不好,还会被人设计陷害了。

昔年文彦博守成都,朝中有人弹劾他贪墨,并御下苛刻,几至兵变。朝廷遣御史何剡前往成都体察详情。文彦博听说之后,暗地里遣了亲信在入川道路上迎上了那位钦差,然后一番好生款待,招了营妓,谎称为家姬出来陪酒,一番歌舞将何剡迷得晕头转向,扯下营妓的汗巾写了一首艳词。可等到何剡抵达成都,正准备作威作福,文彦博在宴席上把那营妓唱着艳词出来一亮相,何剡哪里还能查案?只能灰溜溜的回京复命,报称查无实据,就此让文彦博顺利过关。

这些前人的典故,宗泽在中书门下听了许多,各色是非装了一肚皮。他并不是要微服私访,即是那样做了其实也查不到多少东西,但与其一路与人勾心斗角,还不如先跳出去,到各地走一走。

……………………

当润州官场上重新得到宗泽的行踪,已经是一天之后。然后他们就眼睁睁的看着宗泽在润州下面各县绕了一圈,最后才施施然的进了润州城。

待宗泽一行入住馆舍,夜中的润州州衙倅厅灯火通明。

疲惫不堪的景诚坐在上首,灯火下,两个黑眼圈分外浓重。在他的一侧,是同样憔悴的属官们,另一侧是润州治下的知县们。

景诚环顾左右,声音沙哑,“这几日下来,想必各位已经明白了。察访使过润州而不入,进城后又闭门谢客,主使者是谁,相信你们心里都有数。宪司已经调派人马大索四方,说是要斩草除根,免得日后再添烦恼。又说只要一纸诏令,朝廷的兵马五日之内就能抵达润州城下,不怕有人敢造反。诸位,这一回都将把息事宁人的心收一收,也把轻易过关的心思收一收,朝廷这一回是要下狠手了,别再幻想朝廷的板子会高举轻放。”

景诚说得人人一身冷汗。

这几日,两浙提点刑狱司几乎是疯狗一样的到处抓捕明教教众,各州各县对此怨声载道,润州辖下诸县镇也是给闹得鸡飞狗跳,但这个时间,谁也不敢抱怨出声,宪司动作如此之大,没有得到授意是不可能的。

可是在州县中任职,保境安民是分内之事,若是起了民变,宪司能推脱,亲民官却推脱不得。

“我等该如何做?”丹阳知县急声的问道。

丹阳民风彪悍,偏偏又多有明教信众,提点刑狱司在县中大动干戈,眼看一堆柴草上就差一把火了,早急得心如火烧。

“你等先安抚百姓,明天,等我去拜访了宗察访使再说。”
