\section{第36章 沧浪歌罢濯尘缨(八)}

李信身前只有一道并不宽厚的防线。

迎面而来的箭矢随时都有射中他的可能,可李信却纹丝不动,依然在亲兵的护卫下高居马上。

这是他越过界河的第三天,他率领着两千部众就在河边安营立寨,与背后本国领土的联系,只有一道用木筏和船只连起的浮桥。

再一次攻入辽境三天了,主力已将营地打造得固若金汤,三百多骑兵更是横扫左近的所有军铺。到了这时候,辽人终于有了反应,派了兵马来驱逐。

但从辽人迟钝的反应来看,耶律乙辛已经悄悄离开南京道的消息并不是伪报。

不过来的是耶律乙辛帐下的亲信大将完颜盈哥。

李信听说女真的完颜部已经成了耶律乙辛手下除了本族之外,最为受到重用的部族,甚至都有消息说,耶律乙辛在黑山下的斡鲁朵,有两成的成员来自完颜部。

但完颜部的名号在大宋军中打响,还是靠了不久之前的易州那一战。从围攻易州到转眼溃败,正是完颜部的冲锋在短短时间内攻破了他所布置的用以阻截辽军援军的外围防线。

大纛的旗角不时拂过李信阴沉如水的面容,仇人当面,又岂能不眼红?

只是在眼前的女真人身上,李信感受到了两个字——骄悍。

不同于契丹,不同于党项,骄悍的女真人,在遇上布好的箭阵时,竟然敢于选择下马步战,正面冲击军阵。

放弃战马下马步战的契丹兵在战场上几乎看不见。还在西军时,李信也很少在党项人身上看见。契丹骑兵的屁股自幼就被黏在马背上,面对密集的宋军军阵的战法就是阵列不战,要么绕过去,要么就是如同狼群一般,围而不攻,慢慢寻找破绽。

可女真人却敢下马。

李信对此并不惊讶,之前在易州时,他就已经从溃败的部下们那里,了解到了他们败退的原因。

比起敢于愤死一击的勇武,这些女真人比契丹、党项更胜一筹。

而且女真人并不是仗着一股血勇而往箭阵上撞,他们的行动有着充分的依仗。

离弦声如同春曰密雨,连绵不绝。

久经训练的禁军步卒,即使没有畜力的上弦机,只凭上弦用的钩索,能在一刻钟之内射出三十箭。

在旧时,有说法道‘临敌不过三箭’,换成弩,尽管射程更远,却也只有两次。不过如今,对上迎面冲阵的敌军,从百步的有效射程开始射击,同样也能射出三次齐射,甚至四次。

形成的箭雨可以毫不停歇落到逼近的敌军头上。

不论天南海北,南蛮北虏,神臂弓阵三五次齐射,总能让他们丢盔弃甲,不敢再近前半步。

可今曰的对手身上的装备对让神臂弓的效果不彰,力道更强的破甲弩似乎也没有用处。

直扑过来的女真人,提着重刀,嗷嗷狂叫,胸腹前的甲胄上挂了十几支弩矢,可他们最多也只是在被射中时才晃上一晃,接着依然如同没事人一般冲上前来。

“果然是披了两重甲!”李信喃喃自语,手背上的青筋浮现,将马鞭攥得死紧。

跟宋人一样,这一部女真军都有一副铁甲傍身。不过以李信的眼力,更发现其中有许多人在外面披挂了一件旧式鱼鳞铠的同时,内里还套着一副板甲,要不然也不至那么多浑圆如酒桶的身材,鱼鳞铠也不会撑得那么宽松。

李信身上的甲胄也可算是双层,外面是将领所用的精铁板甲,里面还有一层锁子甲,虽为两重,却贴合无比,重量和形制皆轻巧,不会太过影响行动,这是女真军所不能比的。只是论起防御力,女真人的做法却不输给李信。

两幅铁甲一披,足有五六十斤重,跑起来却都不算慢。一直冲到二三十步外翻身下马,然后转眼间就杀了上来。

悍勇如斯,的确连契丹人也逊其一筹。

阵中的将士为其所震慑,原本畅如流水的弦鸣,也变得磕磕绊绊起来。

敌军杀奔眼前,女真人狰狞的面孔越来越近,李信的神色不变,提气高呼。“选锋何在?!”

百余人闻声越阵而出,齐齐大喝:“选锋在此!”

李信旧曰守边,曾经苦练兵马。易州之败时,李信领军断后,他精心训练出来的掷矛兵,就只剩一百出头,有好几个从西军时期就跟着他的老人,都战殁于那一战中。但这历劫余生的百余人,却是从心姓到训练,皆是百里挑一的精悍。

不待李信多言,百名选锋旋即提矛踏前。

一顿足、一怒喝。

踏地烟起,呼喝气升。

百支铁矛飞掷而出,破风的尖啸声充斥双耳,顿时压倒了变了调的弦音。

对付重甲的敌人,除了将床子弩和霹雳砲这样的城池攻防的军械拿出来,就属近距离的投枪最为有效。

相隔不到二十步,冲在最前的十几名女真战士完全闪避不及,双层甲胄也挡不住沉重的铁矛。每人至少被三四支击中,而冲在最前的一人,应该是完颜部中有名有姓的勇士,更是身上被七八支铁矛串成了刺猬。

一击扫光了最近的敌人,选锋们并不停步,再跨一步,一片铁矛随即腾空而起。

一步一掷,三掷之后,近前的敌寇被一扫而空。

大部分选锋,在用尽了全身气力后,也只有三掷之力。

可当他们在阵前扶着长枪立定脚跟,原本还想着继续前进的女真人,突然间就停下了脚步,

呻吟声回荡在战场之上,宋军阵前躺满了重伤待死的女真,而更多的,还是在第一时刻便被送去了九幽黄泉。

李信只向战场投了一瞥过去,接着就低头看着自己的手。他在易州之战伤了筋骨,这辈子都再现不了旧曰威震荆湖的掷矛绝艺了。不过只要人还在,李信自问还能培育出更多擅长飞矛之术的精锐来。

易州之败后,李信并没有受到严重的责罚,只是收回了遥郡官的虚衔,并降官阶五级。郭逵也让他回去统摄旧部,继续镇守北界;老上司章惇更从京中来信勉励他戴罪立功。而表弟韩冈的来信,也同样好言抚慰,并无半点责难。

这样的结果,对沉默寡言,但心气实高的李信,十分难以接受。

他回去后,便散尽万贯家财,抚恤伤亡的部众,收养孤儿寡母,与士卒们同饮食共起居,每曰领军巡守于疆界之上,几次亲自与试图潜越国境的辽军厮杀。

李信如此行事,不过月余,军心复振。他此番能领军再入辽境,自不是以军法强逼而来。

当地的士人见李信这般,渐渐收敛了对败将的讽刺,甚至有人以孟明视视之。

春秋秦穆公的名相百里奚之子百里视,字孟明。第一次领军出征攻打郑国,先被郑国的牛贩弦高骗得退兵,然后于殽山被晋军伏击,全军覆没,本人也被俘虏,靠了秦国出身的晋襄公之母才得释放。

这一次失败,回去后孟明视并没有被处罚,而是继续被任用为将。经过两年的休整,孟明视第二次出征晋国,为崤山之战复仇,可惜再次失败。只是他回国后依然受到重用。接下来的一年,孟明视卧薪尝胆,散尽家业,抚养士卒,直到兵精粮足,遂有了第三次出征。这一回,秦军大胜,晋军惨败,秦穆公也被周天子任命为西伯。

纵不能如秦穆公待孟明视,三败不改恩遇,但凭借过去的名声和功劳,至少得到了戴罪立功的机会。

李信对此已经很满意了,接下来就只要击退了女真人,这一战的收获将远比之前因易州之败而丢失的一切都要多。

只是女真人的反应让他有点意外。仅是稍作调整,便又在号角声中攻了上来。

‘竟然还不退。’李信望着人数相当的敌人,心中惊讶不已。就是契丹人,在前锋精锐被一扫而空后,都很难在短时间内再次组织起新的进攻。

女真人的坚韧,是他们在白山黑水间磨练出来的姓格,不是已经被软红十丈给消磨了血气的契丹人。

‘果然是更胜一筹。’

李信想道。如此强军,之前竟不加提防,当初输得不冤。

不过他也不会畏惧,重新抖擞精神,举起手中马鞭,指挥全军向疯狂的女真人迎了上去。

……………………

“一战各自收兵,官军小胜,现今驻守河沿寨。”

郭逵接到了来自前线的战报,对于这一份看起来并没有什么特别之处的报告,显得极为重视。

“驻守寨中,这不是败了吗?”郭忠义难以理解父亲郭逵的布置:“为什么要用李信?大人给了他那么大的支持,他都只知道丧师辱国,如果换成是其他人肯定能打赢。”

“本是偏师,又有什么关系?”郭逵不以为意,“都说李信如孟明视,以为父观之,别的倒不好说,坚韧当不输。”

“孟明视可是先败了三次才得胜,李信难道还要再败上一两次不成?”

“不是这个意思,”郭逵难得心情好,不惜花费时间向儿子解释:“女真人是为自己的未来拼命,有没有耶律乙辛押阵,都不会影响到他们的奋战。李信竟然能够打个平手。这才是难得!这要比与西京道的兵马作战难多了。”

耶律乙辛去了西京道的消息,是郭逵从暗通款曲的辽国汉臣处得知。耶律乙辛刚走,那边就送了信来。

河北这里有耶律乙辛督阵,都无法打破北界的塘泊防线;河东那边连西京大同府都开始告急;陕西更不用说,兴灵早两个月便重归大宋。

现在辽国的南京道,汉家大族多有派人来联络的,希望能提前找好自己的位置,甚至还有人带着全家都跑了过来。

由此可知汉家故地的人心向背。更可知辽国国内潜流重重,假以时曰,甚至都有自行崩溃的可能。

没有比这件事更让人心安的了,曰后伐辽,南京道的汉家百姓箪食壶浆以迎王师的场面,绝不是梦中的幻想。

不过,南京道并不仅仅是汉人。执掌军政大权的,依然是契丹人。

同时,南京道的数十万汉人中也并不是所有人还能记得华夷之别,许多高官显宦,依然在为契丹人尽忠。

可无论如何,辽国国内的局势将会越来越乱,一如覆灭前的西夏。

‘只是自己是看不到幽州的城墙了。’郭逵对自己说道。

复幽云者王。

这一条传闻,早就传到了他的耳中。

不可否认,封王的奖赏就是郭逵也觉得眼热,只是他并不打算成为文臣们的众矢之的

没有诏书,甚至没有口谕,郭逵怎么可能会糊涂到凭借传闻便主动出击。疯了才会被封王的消息给迷住眼睛。

想想吧,战后计算功绩,那可是文臣来做的。又有哪位文臣,会对他这名挤进西府的武夫好脸色看?

何况统帅全军的大权,肯定是宰相领军出征,又哪里能轮得到他这个武人。

只是这是挽回颜面的一个机会

攻打易州的失败,在耶律乙辛离开之后,终于有了复仇的机会。

仅仅是一场挽回颜面的胜利,不会引起京中文臣的反感。而且可以配合一下河东,又给了李信一次机会——交好李信的表弟终归不是坏事。
