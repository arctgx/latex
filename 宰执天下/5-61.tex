\section{第九章 拄剑握槊意未销(九)}

【晚上还是有事,不过今天还是赶出来了。明天上午照常更新。】

如同五雷轰顶。

苏轼看着眼前排得整整齐齐的鱼鲊,手脚冰冷,脑中一片空白,呆愣愣的半天都没用动弹。

“苏直史,还是快吃吧,冷了就不好了。”提着食盒进来的小吏,温声劝着苏轼。

监管台狱的吏员对任何一位进士出身的官员都是彬彬有礼。尽管现在成了阶下囚,但一两年后就翻身的例子不胜枚举,谁会给自己日后找麻烦。

但苏轼完全没听到小吏到底在说什么,脑中嗡嗡的直叫。

人犯在台狱中的饮食,一向由家人负责。在入狱前苏轼跟儿子苏迈约定好,平常送饭送菜,只送菜蔬和猪羊家禽,但当朝廷定罪,且是死罪的话,就改送鱼。虽然改变不了什么,但至少能有个心理准备。

虽然一直都有不好的预感,下狱后就被问五代以来可有誓书铁劵——死囚才会问五代,他罪只问三代——但事到临头,苏轼却发现自己并没有自以为的从容。

哪里还有心去吃饭,苏轼摇摇晃晃的来到狱中一角的小桌旁,桌上笔墨纸砚俱全,这是为了让他写自供状的。

苏轼终究不甘心就此而死,磨开了墨,提笔便是一首七律,给弟弟苏辙的绝命诗。但苏轼知道,他的诗作肯定会被献上去给天子,只要有个几日的耽搁,说不定还能来得及打动天子收回成命。

‘圣主如天万物春,小臣愚暗自亡身。百年未满先偿债,十口无归更累人。是处青山可藏骨,他年夜雨独伤神。与君今世为兄弟,更结来生未了因。’

心情激荡下,一首转眼写成。小吏也识字,看了苏轼的新诗之后,脸色就是一变,回头看看食盒,却又看不出来其中到底蕴含了什么样的信息。

而苏轼紧接着就又是一首,‘柏台霜气夜凄凄,风动琅珰月向低……’

“苏直史,宫中的天使来了!”

从门外一声叫喊,让苏轼的手为之一颤,一滴墨汁从笔尖滴落,浓浓的墨团转眼就在纸面上殷开。

圣旨来得好快!

苏轼惨然一笑,本以为还有几天的时间,想不到竟然这般心急。他放下笔,颤巍巍的站起身,瞥了角落处一眼,那里藏着他惯服的青金丹,如果一次吃得多的话,就是登仙之药。

到了狱中后就藏了起来,本想着实在受不了了,就一了百了,可终究没下定决心。想不到还是要用到了。

回过头来时,前来传诏的内侍已经到了牢房门前。

蓝元震曾经见过苏轼,那时候的苏轼文采风流,气韵冠绝当代,但如今成了狱中一住数月的阶下囚,已经是骨瘦伶仃,须发皆乱。

暗叹一声,蓝元震就在门口展开圣旨,“苏轼接旨。”

苏轼跪了下来,颤声道:“臣……臣恭聆圣谕。”

李定没有来,舒亶也没有来。这些日子日审夜审,两人总会到场一个,想不到赐死的时候,他们两个都没来看自己的笑话。

妻儿老小现在不知是还在湖州,还是已经先到了自己当初在常州买的田宅中安居。兄弟、儿子都是受了自己的牵累,现在也不知怎么样了。是不是还在为自己而奔走。

文才害人,悔不该作诗。

苏轼心中自悲自苦,也不知蓝元震到底在念个什么。

等到蓝元震将一封诏书念完,身后小吏推着他让他领旨谢恩,苏轼才有了点反应,泪如雨下的跪伏着:“罪臣苏轼自知讪谤朝政、罪孽深重,死且不恨。可天使是否能宽容半日,让罪臣见一见家人。”

蓝元震愣住了:“不知苏水部此话何意?”

“苏水部,是监江州酒税,不是……别的。”小吏在身后提醒。

苏轼呆滞的没有反应,蓝元震摇了摇头,明白了苏轼到底是为什么没有听清楚,根本看到自己出现后给吓糊涂了。

“苏轼,如今乃是天子圣恩,可本官监江州酒税,还不快叩谢天恩。”蓝元震将圣旨中的核心内容重又向苏轼说了一遍。

本官水部员外郎的品阶不变,直史馆的贴职被剥夺,然后去江州监酒税,仅此而已。根本算不上什么处罚。一些监察御史,如果弹劾重臣失败,往往也就是这样的惩处,本官不变,变得仅仅是差遣,过两年就能爬回来的。

心情大起大落,苏轼茫茫然的向着前来宣诏蓝元震叩谢天恩浩荡。

“苏水部,回去后好生洗个澡,去一去晦气,过两日可就要南行了。”蓝元震很和气的叮嘱了苏轼一句,然后快步离开了牢狱,回宫缴旨。

几乎是被民间的舆论所迫,不得不放了苏轼一马,天子如今的心情,可不是很好。

可不要被迁怒了。蓝元震心中有几分忐忑不安。

拿了圣旨,御史中丞、殿中侍御史都没有出现,就派了一名小吏将他送出了台狱。

乌鸦在台前的槐树上飞舞,但狱中只惯见老鼠、蟑螂的苏轼却是贪看不已,儿子苏迈并没有在门前等候,只有一个远亲和一辆马车。

看见苏轼出来,他是一脸惊喜:“天可怜见,官家终于是开恩了。维康【苏迈】近日盘缠用尽,去陈留筹措了。这两日的饮食本是托付给小弟,没想到就才一顿而已。子瞻你怕是还没吃吧?不管那么多了,先回去洗个澡,去了晦气后,好生吃上一顿酒。”

难道这就是送了鱼来做晚餐的缘故?苏轼一时啼笑皆非,竟是差点被吓死。

“听说了吗,苏直史已经定案了。”

“听说了。是监江州酒税吧?”

“从知州贬到了监酒税,还真是够重的。”

“已经很轻了,前面不都是说要论死的吗?现在连本官都没动!”

“……说得也是。”

樊楼之上,不少房间传出的曲乐在这一晚变得雀跃起来。

灵州之败的确出人意料,酒宴上谈兵痛饮的人也少了,但终于有了个好消息。尽管有当年周南之事,但苏轼因诗文入罪,在秦楼楚馆之中,并不乏同情之人。

但也有人为此而感到遗憾。

“真是算他运气。要不是有传言出来,多半还要关上半年。死罪不一定有份,但好歹一个编管,追毁出身以来文字也不是不可能。”

“谁说不是呢,天子也是要脸面。不过这谣言传出来的时候也巧,正好卡在节骨眼上,否则当真会依律处置了。”

“其实这等于是借势凌迫天子。天子为了名声不得不放了苏子瞻一马,但心里怎么也少不了芥蒂。只要天子在一日,苏轼就一日别想再出头,好生的在江州写诗吧。”

“谅他经此一事,也不敢再乱写诗词了。”

由于天子插手,苏轼讪谤朝政一事就此定案。惩处之轻,让人出乎意料,不过联系起此时京中流出的谣言,却也就没有人为此大惊小怪了。

但苏轼的责罚虽轻,可曾经向他通报消息的苏辙、王诜全都被牵连贬官。而其他与其鸿雁往来的友人,也都各自被罚铜。只是终究不是重罪,只为了给一番辛苦的御史台一个交代罢了。

不过话说回来,这个交代显然无法让李定坐稳御史中丞这个位置了,第二天,辞章便送进了崇政殿。

“真的不管官人的事?”周南端着夜宵进了韩冈的书房,却没有立刻离开,而是问起了今天的新闻。

“此事跟为夫何干?”韩冈反问,低头看着书信。

“官人前些日子还说不让苏子瞻做田丰吗?”

“为夫说过吗?”韩冈皱眉想想,摇了摇头,“忙都忙不过来,哪记得这点小事。”

周南手肘撑着桌子,凑近了凝视着韩冈,双眸弯弯,带着笑意,“官人就尽管骗奴家好了,反正奴家什么都会信的。前些日子听官人说了之后,奴家去查了三国志,才明白为什么官人会这么说。这两天听外面的传言,怎么听都像是袁绍和田丰那一段。”

“真要说起来,苏轼只被贬官,还是靠了岳父给天子的奏折。圣世安可杀才士,没有这一句推了天子一把,哪有这么快结案的道理?苏轼被拘入御史台,就连最亲近的张方平都没有为他上书,反倒是岳父、章子厚他们站了出来救援……”韩冈呵呵笑着,也不知在笑谁。

可惜了赤壁赋和大江东去,‘小舟从此逝,江海寄余生’多半也不会再出现,不过也许会有庐山赋或是鄱阳湖赋,或许能抵得过了。

但苏轼之事,放在眼前的天下大局上,只是个微不足道的小插曲罢了。种谔、李宪暂时不用担心了,眼下还是要看王中正那两路的情况,秦凤、熙河两路联军便首当其冲,希望赵隆、刘昌祚他们两人能有所表现

……对了,不知王舜臣那边怎样了!

韩冈终于想起了在六路汇聚灵州的战事中,还有一支小小的偏师正在向西进发。

王舜臣收复凉州的消息通过加急文书发送到京城后,朝堂上还欢呼鼓舞了一阵,毕竟是河西故地是个百多年终于回归,官复原职的诏书随即就发过去了。不过转眼灵州之败也传到京城,几天下来,朝堂上下都把他给忘了,韩冈都没能例外。

在周南惊奇的目光中拍了拍脑门,早点把总参谋部建起来就好了,多少人拾遗补缺,哪里会有这么多幺蛾子的事。

不过创设一个新的部门,必然少不了从既得利益者手中夺取权利,韩冈现在也只是想想而已,倒也不会指望提出来就能有个好结果的。

他也曾在军中推行过参谋制度,有用归有用,但之后也没有流传开来,没有哪位将领愿意分割自己的权力。

还得慢慢来。

韩冈叹了口气,喝着掺了金银花的解暑凉汤,思路转回到凉州,王舜臣那边的进展应该很顺利吧。

