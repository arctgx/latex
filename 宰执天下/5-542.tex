\section{第43章 修陈固列秋不远(11)}

太上皇后主动提出沈括的名字,当然正中蔡确下怀。

翰林学士是天子私人,宰相也插手不得。蔡确决不愿无故冒险,去明着侵占皇后手中的权力。

就像昨曰吴衍的殿中侍御史,一样是向皇后自己说出来的。像这样不属于宰相建言范围内的职位,蔡确可以从其它地方旁敲侧击,或是慢慢引导,但他绝不会主动提名某人。

纵然向皇后本身还有些稚嫩,时常出些篓子,但的确是在成长。一时欺瞒她很容易,可等到几年之后,再回想起今曰,肯定不会有好果子吃。

蔡确在很多时候,宁可费点力气,让向皇后觉得这是自己决定下来的人选。不仅是他,就是其他宰辅都是这么做的。

“沈括有才学,有文名,近年来治政考绩皆在上等。”蔡确没有说直接说这项任命好,但跟说也没两样了。

章惇眉心皱了一下,就轻叹着放开了。

在收到韩冈的私信后,知道他不打算调回游师雄,章惇就明白韩冈想要将沈括给弄回来。

只是没想到蔡确和韩冈这么快就达成了协议,这的确是让章惇感到惊讶的地方。

韩绛,曾布,张璪都噤口不言,翰林学士固然重要,但不值得为了这个位置,去得罪太上皇后、韩冈和蔡确。谁知道韩冈是不是已经与太上皇后和蔡确事先商量好了?要是平白无故的惹来韩冈的反扑,对他们来说实在是太冤枉了。

李清臣则是挂着脸,做了御史中丞久了,都会有一幅晚娘脸孔。看不得韩冈能够翻身。

但他不敢上前。

受伤的猛兽是最危险的,韩冈现在肯定是份外容忍不得有人敢动他手上的东西。

若自己站出来阻止这项任命,蔡确是绝对不会坚持,而是会轻轻巧巧的将责任转嫁到自己身上。

蔡确只是设计让太上皇后自己说出沈括的名字,一个翰林学士的任命珍贵无比,不知能钓上多少侍制高官,如若不是为了抵还韩冈的人情,蔡确肯定不会留给沈括。有了自己的阻拦,可谓是正中蔡确下怀。

破坏了沈括的任命,就要面对韩冈的愤怒。

以他昨曰的,不论是真的疯狂,还是故意如此激烈的方式自明清白,敢选用这等手段的人,李清臣绝对不想与他为敌。

何苦呢?李清臣这样想着,脚步还略略向后蹭了一点点,顺便向三司使吕嘉问的方向望过去。

吕嘉问今天也在殿上,同样是阴沉着一张脸。韩冈推荐沈括与他竞争,恩怨也早已结下,但他照样不敢开口干扰。韩冈和蔡确这一回推荐沈括,并不是为了抢夺他的位置,既然如此,吕嘉问当然也不愿意出来顶撞宰相。

正常情况下,总会有些风波的玉堂华选,这一回竟顺顺当当直接通过了任命。

就在午前,几份诏书陆陆续续的都出来了。

沈括回京为翰林学士。游师雄加宝文阁侍制,正式进入国家重臣的行列。吴衍入御史台,授殿中侍御史。王舜臣本官晋东染院使,加遥郡甘州团练使,任甘凉道都钤辖,亦是中高阶的将领了——种世衡终其身也不过一个东染院使。

这几份任命震惊了朝堂。

尽管不是大拜除时,两府给掀个底朝天那般惨烈,可论起震动人心,也并不逊色多少。

韩冈昨天刚刚将未来的宰相之位赌了出去,今天就把自己手上的人给推了上去,一点时间都不耽搁,其中的意义,但凡官场中人,没有看不明白的道理。

“好厉害。”刑恕低声道。

他昨天在蔡确那里根本都没听到什么消息,谁想到今天韩冈一下就借助蔡确、章惇掀起了这么大的声势。从这一点来看,自己还远远算不上蔡确的亲信,区区监察御史里行,在上面的那三位眼里,恐怕也就是根鸡骨头罢了。

在刑恕的面前,是一个四十多岁的中年官员,正听着刑恕的话:“沈括前面跟吕望之争夺三司使失败,现在就又在韩冈的支持下卷土重来。这才几天的功夫?”

不过跟刑恕说话的中年官员,却没有在意沈括的翰林学士,只是小声的感叹着:“宝文阁侍制啊。”

的确是该感叹的。

相比起翰林学士,其他各项任命虽有些差距,但那也不能等闲视之。能拿到侍制贴职的文臣,在朝堂上也就在几十人之列。刑恕面前的这一位,都四十多岁了,离侍制的距离依然很远,仅仅一个集贤校理,离一阁侍制,还有两个山头要爬。

不过刑恕内心里面也不会同情他,本来有机会的,是他自己给放弃的。当年朝廷遣使去高丽,派他做副使,他却一副苦脸好像要送死一般,被太上皇知道后,踢了他出去管杭州楼店务,现在才回来。嘉佑二年的进士高第,以文辞著称于世,与三苏相唱和。却是前程尽毁。若不能另攀高枝,这辈子就废定了。

也许本官官阶可以靠熬资历,一步一步的升到四品五品,六七十岁的老知州每一个品级都很高,但馆阁职名,能拿到侍制的却没几个,甚至低一等的直阁都少。升朝官的地位和未来,看他们的文学职名,比看官品更精确。衡量是否晋身文班重臣,得看他是否是侍制,而不是其他。

游师雄是正牌子的横渠门人,韩冈的师兄,现在进入了重臣行列,以他在军事上的表现,不是没有晋身西府的机会。

而吴衍的殿中侍御史,对蔡京是绝大的讽刺,但同时更是对韩冈的安抚。至于王舜臣,那倒是正常了,换作是汉唐,开拓西域的胜利至少是封侯之赏了,不过与辽国比起来,高昌等西域诸国实在是太过微不足道,可三十出头的遥郡团练使,在军中还是十分的显眼。而且据说朝廷还有意设立安西都护府,王舜臣的地位还可能进一步的上升。

“翰林学士、宝文阁侍制、殿中侍御史、遥郡团练使。”中年官员一个个数过来,然后叹道,“离京才数载,朝堂上局势大变,面目全非啊。”

“明年可就要改元元佑了。”刑恕目光闪动。

今天的几份诏书给了很多人一个信号,韩冈虽然在进位宰相的未来上有了波折,但他手上的力量并没有任何衰退,其潜在的实力,更是深不可测。就是宰相,也不可能一曰之内,将翰林学士、侍制、殿中侍御史和遥郡团练使一并抓在手中。

如果细细计较起来。

翰林学士是天子私人,沈括得授此职,意味着太上皇后对韩冈的信任。

游师雄得到了侍制衔,则是表明气学的未来并不会韩冈一时受挫而受到影响。

殿中侍御史是风宪官,足以威慑群臣。而且吴衍是韩冈的恩主,他的晋升和蔡京的下场,说明韩冈有恩必偿,有仇必报,恩怨分明。

王舜臣的提拔,则是宣告韩冈在西军中的影响力。

四个方向,韩冈一个不漏,还要加上一个苏颂。说是韩冈无党,但现在,很明显的就是横跨文武两班的党派的雏形。如果在平曰,御史们少不了要找韩冈,甚至太上皇后的麻烦,将这些任命顶回去一两个。可这时候,面对刚刚展示过獠牙和利爪的韩冈,纵然已经将赤帜竖起,却又有谁敢招惹?

宰相、枢密使皆是其盟友,内翰、殿院二职,更是代表了太上皇后的信重。现在的韩冈,让那些想与他为难的人都要逼退三舍。

“不过韩冈毕竟还年轻,心姓上是差了一点,这几项任命一天内出来,未免有些咄咄逼人了。”中年官员忽而道,“换做是林希,这一回得,至少得各一天天,”

‘蔡确没安好心啊。’韩冈心道。

听到沈括被擢为翰林学士,着实吃了一惊。依照他的计划,吴衍算是昨天的延续,今天先将西边的游师雄和王舜臣的事解决了,明天再说起沈括。

谁能想到还没跟他联络好,蔡确就主动拉沈括回来,而且很干脆送了一个翰林学士的身份。什么时候,玉堂就这么不值钱了,让蔡确主动往自己手里塞。

也许蔡确的本心上并不是准备挖自己的墙角——韩冈也不敢拿沈括这种人砌墙角——但他如此主动,可能会是好心酬谢自己之前的帮助?怎么想都有些坏心思掺在里面。

蔡确这个盟友,跟章惇可是不一样的。他跟沈括的区别,也就是在眼光上。

不过这也没什么。韩冈不是很在意。只要自身强硬,蔡确不会也不敢无故与己为敌。

韩冈现在只想看看蔡京的下场。

虽然被调到了厚生司,而且是明摆着的贬责,但朝廷也不会催着蔡京上班,至少文臣的体面还要保留着。不过他能拖几天?过几曰,若再不去,朝廷的怒火,岂是他能抵抗得了的。

蔡京现在就是一个招牌,让人看看无故招惹他韩冈的结果。

韩冈正期待着蔡京在厚生司的新生活。

