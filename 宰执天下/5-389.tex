\section{第34章 为慕升平拟休兵(21)}

达楞念着这个让他感觉有些拗口的单词,他眼前的两道远远延伸出去的铁条,也很难看得出这是一条运载量大得惊人的道路。

“这就是轨道。”折可大站在铁轨旁,向阻卜部大部族长的亲信解说道:“这只是两百里长的忻代铁路的一部分,等全线联通之后,从忻州到代州的两百里地,就是满载着人和货也只要一天的时间。”

这个自称是阻卜大部族长磨古斯的亲信,已经得到了韩冈的重视。既然身兼枢密副使和河东置制使的韩冈都重视这个北虏,折可大也不会拿着架子待人。

折家在草原上也有名声,折家家主的继承人的地位在达楞的心中自然不会低。这就是为什么韩冈不让文官而让他这个武将来招待达楞的原因——这一点,折可大不用人说,自己也明白。

而且折可大也知道这个阻卜人不简单。韩冈让他拿来二十个辽人首级证明他自己的身份,达楞倒也二话不说,找了几个部族的族长,几天之内就给凑齐了。

不费吹灰之力就弄来了二十个辽军首级,让折可大很惊讶。一个战果就是一份封赏,河东制置使司为辽军首级给出的馈赏,对穷困潦倒的阻卜人而言,那就是能换来全部家当的一个大数目。达楞能虎口夺食,不仅证明了他的身份和能力,也体现了他背后的磨古斯在阻卜诸部中的地位。

但现在,看着眼前能改变一切的运输工具,达楞却再没半点为自己的身份和靠山而骄傲自豪的心思。

一日运兵数万到百里之外。就是在草原之上的辽人,也很难有这样的速度。而运送大量的物资则更是不可能有这样的速度。而且一旦这么做了,马匹肯定要损失一大批。马是牧民们的命。根子,任何一个部族都不会驱动自己的命。根子去做送命的买卖。可宋人要做到这一点,只要付出几百匹劣马罢了。

由十来匹劣马拉动的有轨马车,速度赶不上奔马,却也跟骆驼、驮马带着货时差不多,但载货量却是骆驼、驮马的几百倍。一列车就能把一个小部族的家当全都装上去。

难怪都说宋人有钱,只是为了临时运送几万兵马,就能把几千几万斤的好铁钉在地上来修路。

如果说传言中人人绫罗绸缎、家家金银珠宝的南朝,是让草原上的阻卜人魂牵梦萦、垂涎欲滴的肥羊的话。那么能毫不在意的把精铁放在野地里的行为,在被契丹人封锁了铁料的来源,甚至很多时候只能用骨头来做箭簇的阻卜部中,是奢侈到足以让人望而生畏,乃至顶礼膜拜了。

达楞素知磨古斯的雄心,要控制草原与契丹平起平坐,甚至还有取而代之的想法。而达楞作为磨古斯的亲信部众,不仅忠心耿耿,更愿为磨古斯描绘出来的未来而出生入死。那是达楞所知的唯一能改变命运的道路。

但在亲眼看见宋辽两国之间的战争之后,他为磨古斯煽动起来的幻想一下就烟消云散。

无论宋辽,都不是阻卜部可以对抗得了的巨人。他现在所在的河东战场,仅仅是宋辽三个大战场中的一个,且是规模最小的一个。

不论是陕西还是河北,两边都是十万以上的甲兵在长达千里的战线上厮杀,只有河东,是在一个个被山峦约束的盆地中打仗。可眼前双方依然是数万战士正面相峙,这是不亲眼看到根本都不能想象的场面。

而且在辽人已经摧毁了当地的所有村庄城镇,掳掠屠戮了无数当地百姓,受到如此重创,但宋人还是能动员出更多的军力,更多的辎重,来跟辽军作战。在草原上,任何一家部族都不可能会有这样的韧性和底蕴。宋人的实力,就像是九河汇入的北海,不论分走多少水量,都不会见到干涸。

赢不了的。

达楞摇着头。

不论是那一边取得了优势,阻卜部都是赢不了的。

腾出手来的辽军,能将阻卜部的叛乱一掌碾碎。

而宋人若是掌控了草原,同样不会给磨古斯翻身的机会,相对于军力,他们更有足够的财富来买通所有的草原部族,让磨古斯成为孤家寡人。

达楞悄然的瞥了仍在夸夸其谈的折家子弟一眼。

与其在宋辽两家的夹缝中挣扎求存,去幻想那百中无一的可能,还不如赶紧卖个好价钱,就跟已经冲宋人的年轻宰相摇着尾巴的西阻卜各部一样。

……………………

轨道。

萧十三当然听说过轨道。

尽管他没见过,可在朝堂高层,任何人的耳目都不会太闭塞,只是他没想到宋人能用在这里。

韩冈的事迹在辽国国内传播得很广,其中有真有假,但无论真假,都是被夸大的不像话。只是这些传言多半集中在医疗领域,以及军事上的那些发明。

至于用兵、治政,也只有最高层的一些人才会去关心。轨道就是其中鲜为人知的一种。毕竟在辽国,只听传言是很难产生直观的认识。

“你确定是轨道?”萧十三问道。

“我想不到有别的可能了!”张孝杰摇着头,“那是韩冈的发明,第一条轨道也是他亲自指挥修筑,能一个月轻轻松松运送六十万石,换成人又该是多少?人可不要搬来搬去的费时间!”

南京道的几处出产石炭和铁矿石的矿山,就是南府宰相张孝杰奉了耶律乙辛之命亲自主管。

依辽国官制,南北两府的左右宰相是总理契丹政务,为北面官,并不分管汉人;南京道的事务基本上都属于南面官管辖。但耶律乙辛还是让已赐姓耶律同时出任北面官的张孝杰去管理,可见有多看重那几座铁矿和石炭矿。

而在张孝杰的管理下,宋国在矿坑里使用的轨道也一并被学了过来。萧十三没亲眼看过,一时想不到宋人的伎俩,但张孝杰可是一切亲历亲为,让辽国的军事生产不至于被宋人拉得太远。

这一回要不是突破了河东,需要他来此调节出征各部,同时管理劫掠来的收获,接受俘获了工匠,他现在还会在南京道,去督管军器的生产制造。

当他听说宋军在一夜之间,就多了两万兵马。一开始虽是怀疑过耶律道宁和他麾下的人马是不是瞎了眼,没探查出潜行而来的宋人。但当更新的消息传来,说宋军的兵马还在增多,他立刻就想到了轨道。

飞船、板甲、霹雳炮,既然这个战场上已经出现了这么多刻着韩冈印记的军器,那么再多一个,又有什么值得惊讶的?那根本是理所当然的一件事。

“乘坐有轨马车不会耗费多少体力,而且有了这一条轨道,韩冈可是想要运来多少军队都可以。”

宋军的出动速度吓到了耶律道宁,吓到了萧十三,也吓到了萧十三之下的所有辽军官兵。不过在得到了一个合理的解释后,所有人都还能放下心来,因为那样的手段不可能更多了。只要在这里守住了代州的门扉,宋军最后也只能看着东北的方向望而兴叹。

但是现在,宋人既然能在一夜运送万人行进百里,那么依靠轨道从开封到忻州又要多久?从长安到忻州又要几日?也就是说,眼前的宋军会越来越多,一直增加到他们承受不住为之。

张孝杰如此肯定,让萧十三心中最后只剩下一星半点的侥幸,“再看一看。只一夜功夫,应该不会有问题。”

张孝杰暗叹了一声,但他也没有多说,一夜时间还是等得起,从代州赶来的援军也需要一定的时间休息。

宋军并没有隐瞒他们的行动。接下来的这一夜里,他们几乎炫耀一般的用着有轨马车将后方的兵员和粮草运抵前线。

亲自在飞船上用千里镜观察的萧十三和耶律道宁,从夜幕中分辨着那一缕缕微光,数着一列列抵达前线的有轨马车,直至心脏如沉大海。

当太阳再一次升起,除了三千留守在忻口寨的士兵,韩冈手中的兵力达到了三万八千余人,其中拥有近六千骑兵,驻扎在前线的七个军寨中。被兵马填满的营盘中,一道道炊烟如同山中的密林,震撼着对面的每一位辽人的心。

这一支宋军的主力。距离代州只剩最后的四五十里。谁都知道,宋军的主帅肯定正在准备缩短这个距离。

与此相对应,经过了昨日的大举增兵,辽军的总兵力其实也达到了两万。在兵力上,仍是只及宋军的一半,明显占了下风。

但在过去,凭着这两万人马,哪一个辽军将领都是有信心与数倍于己的宋军周旋一番。可是眼下这两个数字的对比,却说明了双方在运输能力上的差距,骑兵的支援竟然比不上以步兵为主的宋军。

宋军用轨道运兵的消息,让辽军的士气加速跌落——纵然萧十三对此下达了禁令,但由于辽军内部的组织结构的问题,这些消息还是很快的传遍了军中。

底层的士兵不知道什么轨道,但他们知道如同神佛化身的韩冈。当大军处于优势时,这一点无关紧要,可一旦处于劣势,韩冈的声望就成了压断脊背的最后一根稻草。

除此之外,萧十三还发现了一个同样急迫的问题。

由于辽军正军,多为一人双马、三马。两万大军所拥有的战马竟多达五万。

这就带来了一个极为现实的问题——大小王庄的粮草不够吃了。

望着对面紧闭的军营,张孝杰如同被浸在冰海中,韩冈可能不战而胜啊!
