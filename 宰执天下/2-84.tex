\section{第20章 心念不改意难平(三)}

【第一更,求红票,收藏】

辞别了王韶父子,韩冈踏着月色往家中去。

天朗气清,一轮半月正在天顶,银色的月光毫无阻挡的照着韩冈脚下的路面。更夫手上的梆子声从临街传来,长长短短的几声,告诉韩冈现在已是二更时分。

韩冈没想到会在王家待得这么晚,在说过了郭逵和李宪的事后,又讨论了屯田和市易的事——王韶明天就要领着李宪去古渭,自己大概还要在秦州待上十天半个月的样子,许多事必须现在就商议出来——不知家里等急了没有。

入夜之后,秦州城惯例的宵禁让街上已看不到一个行人。以皮革为底的官靴踏在石板路上,没有什么声音,只有身后传来的马蹄声哒哒的响着。

李小六牵着两匹马,静静跟在韩冈的后面。他不清楚韩冈为什么要走着回去,但他知道什么时候该保持安静。而且韩家离得王家又不远,就算慢慢走,一刻钟也就到了。

韩冈正需要这份安静,能让他想些事情。他想的当然不是郭逵的事。就如他早前对王韶说的,察其言观其行。要先看了郭太尉接下来会怎么做,才好作出应对。而不是事前东想西想,自己吓唬自己。

韩冈想得是自家的事。他撺掇王韶向朝廷要求土地和贷款的提案,已经得到了肯定的答复。以他的身份,在古渭寨边上,靠着河滩处,弄上七八顷好田不成问题。而向衙中借贷,至少能有七八百贯,加上家里的积蓄……还有今次他升官应该能收到的贺礼,林林总总一千五六百贯不成问题,这些钱作为本金也够了。

并非韩冈贪于财货——他现在更看重的是自己的权势和地位——而是这世上当真是无钱不行。

商业繁荣的结果,自然带来人人爱财的风气。北宋承平百年,世风越发的奢靡。韩冈去东京城,去的几家酒楼,无论碗碟皆是银器。关西这边的风气好上一点,可秦州城中,但凡有点余财的人家,都少不得穿着绸衣,套着丝履,绝不在吃穿上节省。

而官员么,像王安石、包拯那样清正廉洁、只靠俸禄吃饭的官员毕竟是少数——而且无论王、包,文字、书法皆不差,靠着润笔也是一笔不小的收入,韩冈可没这本事——为了争娶十万贯嫁妆的寡妇,把官司打到天子面前的两位宰相就不提了,连刚来的郭逵都是个好财货的主。

郭逵一年来镇守鄜延,前面跟党项人打得你死我活,后面照样派着亲信带着商队去西夏回易。据说郭逵的夫人为此劝过他,好不容易才收敛了一点,不过不是不再回易,而是把赚到的钱多分了一份给参与回易的士卒——这是高遵裕前段时间打听来的消息。

韩冈猜高遵裕大概是想抓郭逵的小辫子,好用在日后,才仔细打听郭逵的事。不过对于做到节度留后、检校太尉这一级的高官来说,赃罪也好,回易也好,根本就不是罪名。所以高遵裕才会把这事当作笑话说出来。

世风如此,韩冈为了自家打算,当然得想办法置当家产,以养家人。田地、货殖,农商二事如果做好了,家财万贯也是轻而易举。以韩冈在古渭的地位,联手王韶、高遵裕,这两件事当真不难。

同时只要能加强他在蕃人中的人望,回易之事也会更加安全,也可以买到更加优良的蕃货。

韩冈在古渭寨设立的疗养院,为他在青唐等部的蕃人中争得了不小的名声。前次去古渭,遇上的蕃人只要听说他的名字,都少不了向他行个礼。而俞龙珂和瞎药都托人带过信给他,为送族中的病人到疗养院中治疗,而向韩冈求人情。

韩冈现在都想着,是不是在渭源堡开一个小型的疗养院,用以救治蕃人,好让自己的名声再响亮一点——人脉是资源,才能有时不足为凭,而人脉却是长久的保证,这个现实无论在哪个时代都是一样。

主仆两人一前一后的走着。拐过街角,迎面就是一溜气死风灯。灯笼提在一队巡城甲骑手中,幽幽的灯火昏黄,只在灯外,有一圈光晕。

两边猛然打了个照面,韩冈从纷乱的思绪中惊醒。

“什么人?!”从骑兵队列中紧跟着就传出了一声低喝。刷刷几声响,那是拔刀的声音。

韩冈停住脚,心头微怒,有几个奸细会光明正大的走在大街上的,不是巡城路线的小巷子多得很。李小六从后面上前报着他的名字:“是缘边安抚的韩机宜!”

一个灯笼挑了过来,对着韩冈主仆上下一晃,照出了韩冈阴沉着的一张脸。

韩冈在秦州大小也是个名人了,认识他的人不少,现在又穿着官服,身份当做不得假。看到冲撞了新近得意的韩机宜,巡城的队正吓了唇都青了。连忙带着手下下马行礼,为方才的无礼连声道歉。

一群士卒单膝跪在韩冈面前,一叠声的说着,“还请韩机宜恕罪,还请韩机宜恕罪。”

“罢了,尔等也是尽忠职守,本官也不会加罪。尔等自去,”韩冈不耐烦的挥了挥手,“这半个月都没下雨了,天干物燥的,巡察时都注意点。”

“小人明白,小人明白。”巡城队正点头如捣蒜,起来后,也不敢在韩冈面前直接骑上马。这一队巡城不得不牵着坐骑,一直走到十几丈外,方才上马离开。

见着他们诚惶诚恐的模样,韩冈发觉自己不知不觉间,也是有了不小的官威。

经了此事,韩冈便不再在路上耽搁,也上了马,直接回到家中。

开门的是韩冈找来守门户的一个老兵,是从经略司里找来的。五十多岁的老夫妇,又没个子女,亲眷也没几个,韩冈看在他老实忠勤的份上,把他调了来。现在韩冈家的排场日大,没有些得力的仆佣的确不方便。

这老兵开门后一看到韩冈,便连声道着恭喜。韩冈点点头,笑道:“等明日,自有一份赏赐下来。”这话他是对着老兵和李小六一起说的。

韩冈升官,连两位过世的兄长都得了赠官,这喜报早早就有人通知了过来。韩冈得到的赏赐,连着韩冈大哥、二哥的告身也一起遣人送回家来。

街坊邻居相处了有了近半年的时间,听到消息,都过来道贺,与韩冈,送得贺礼堆满了半间堂屋。而韩冈进门时,已经是二更将晚,来贺的邻里早已各自都散了。

几根蜡烛照着堂屋,严素心、韩云娘在忙里忙外的整理着礼物。而冯从义则是坐在一边,对照着礼单和礼物,并在账簿上一一记录下来。这些人情往来,一桩桩都要记着,今次邻里送来贺礼,等有机会,还要还赠回去。韩冈瞧着他们忙忙碌碌的样子,觉得得给自家招些个可靠的仆佣的需求更迫切了。

韩冈跨入堂屋,惊动了三人。立刻,道贺的声音一齐响起:

“恭喜三哥。”

“恭喜三哥哥。”

“恭喜官人。”

听到外间的动静,韩阿李的声音从里面传了出来:“可是三哥回来了。”

“正是孩儿!”韩冈应了声,正想走进里屋向父母问安,韩千六和韩阿李已经先一步出来了。

看到韩冈,韩千六激动不已,“三哥儿果然是没白读书,这官升得一次比一次快。还给大哥、二哥争了一份告身来。”

韩冈笑道:“孩儿官位还不够,只让大哥二哥受了追赠。等再过两年,孩儿一定会为爹娘博个封翁封君的诰敇出来。”

韩千六听着点头直说好,韩阿李却有点不高兴:“升官是好事,但有几个向三哥你这样冒风险的,这个几个官都是卖命换来的!三哥你前日从古渭回来什么也不说,尽瞒着家里,要不是今天来送告身的衙役说了两句,娘还给你蒙在鼓里。”

韩冈孤身夜闯青唐城的事没在父母面前提过提过,都是含糊了过去,韩家就剩他一个独苗,出了意外,哪里找人承香火?韩阿李气得有礼。

韩冈也不得不笑着赔罪,“孩儿不是怕娘你担心吗?”

“怕娘担心,你就不会尽做着这些冒风险的事了!”

不过韩阿李气了一阵也就过去了,毕竟儿子还好端端的在眼前。看着供在两个儿子灵位前的两份追赠告身,韩阿李抹着眼泪:“想不到大哥、二哥也有官身了,若是他们还在,不知该有多好。”

“大喜的日子哭什么!”韩千六说着。

“三哥这是光宗耀祖的事,该挑个好日子祭拜一下。”冯从义则在旁岔话。

“过几日,当是要把灵位都找人重新做过。”韩冈随口说了一句,又问韩阿李,“今次孩儿因功得赐绢银总共五百匹两。不知家里还有什么地方急需要用钱的地方?”

韩阿李知道他儿子现在但凡说话必然藏着心思,擦擦眼睛,直问道:“三哥你有什么地方要用钱?”

“孩儿本想着给家里置办些田产。不过最近听说子厚先生从京中辞官回横渠镇乡中,说是要办一间书院。教化关中子弟。只是办这书院耗费不小,子厚先生做官多年也没挣下多少身家,现在正愁着钱不够。而孩儿在子厚先生门下时日不短,深受子厚先生教诲,一直无以为报。就想分出一半给子厚先生送去。”

“这是应该的!”韩阿李说话毫不犹豫,“没有横渠先生,也没三哥你今日的光彩。知恩不报,读书就读在狗身上了。照娘说,家里现在也不缺钱用,也不必一半一半了,都给你先生一起送去!”

