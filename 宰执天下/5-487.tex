\section{第39章 欲雨还晴咨明辅(16)}

新帝的年号,太常礼院已经给出来了。年号本不需要多动脑筋,弄得多高雅。或者说,因为是要给天下臣民看的,用意越简洁明了越好。如今算命多有拆字的,这年号好坏也多用拆字来测一测。

熙宁末,上皇赵顼欲改元。太常礼院进美成、丰亨两名供赵顼挑选。‘美成’二字中,‘美’可拆成‘羊、大’,而‘成’字中,又有个‘戈’,羊大遇戈,这是被屠宰的兆头,不可选。‘丰亨’的‘亨’,下面是个‘了’,比‘子’少一横,叫做为子不成,赵顼当时正为皇嗣头疼,用了这个年号不是触霉头吗,遂改了元丰。

这一回太常礼院进呈了三个年号。效率难得高上一次,只是一点意义都没有,迟一两个月其实都没问题。现在才年中,明年正月初一才会正式改元。又不是太宗即位,都十二月了,还硬是将年号给改了。

不过既然太常礼院已经拟定了,也没人会说不是。

一个是元佑。元继承元丰,佑就是佑护。希望能延续元丰的好年景,愿上天继续保佑国家。

另一个,则是明泰。因曰月而国泰民安。这是有奉承太上皇后的用意在。

最后一个,是天佑安国。

天佑重复了唐昭宗的年号,但加了安国就不是重复了。四个字的年号此前不是没有,太宗皇帝的太平兴国便是,真宗的大中祥符也是。更早,还有太初元将之类的。天佑安国,就是太上皇、太上皇后二人佑护,国家安定。

意思很浅显,不过一时不容易做决定。

向皇后看了一阵,问下面的臣子:“诸卿觉得哪个年号为上?”

“请殿下自定。此非臣子可以置喙。”

几名宰辅无不推辞。

如果是事关各方利益的国家大事,在列的臣子肯定都会有些想法。但只是年号,说起来仅仅是汉武帝弄出来的东西,根本就不合古礼,只是确立中华正溯而已,用了中国的历法,那就是中国的臣子,除此之外,宰辅们哪个会太放在心上?现在帮忙做了决定,有了好兆头不会有功劳,弄得不好,就是一身麻烦。

“天佑安国……明泰……元佑,”向皇后念叨着,突然问赵煦,“官家,可有觉得合意的?”

“元佑。”小皇帝回道。

向皇后微微一愣,她本没指望赵煦会应声,“为什么?”

“儿臣听着好。”赵煦答道。

“嗯,也好。官家觉得好就好。”向皇后也不是很看重年号,只是觉得几个都不错,无法做出决定,点了点头,对朝臣们道,“诸卿若没有意见,那就选定元佑吧。”

宰辅们自不会反对,韩绛径直上前领下旨意。

蔡确与对面的章惇交换了一个眼神,皇帝才六岁,应该不会这么早就能理解年号中的意思吧?

只是看着端严肃穆的正坐在御榻上的赵煦,两人心中都不由得升起一个念头——这个小皇帝看起来决不会是仁宗。

仁宗皇帝能在御座上一坐十余年,直到太皇太后刘氏自己病死。这份耐姓不是普通的官员就能拥有。还有着更为长远的眼光和头脑。能维持住朝堂的稳定,一朝君子一朝臣的情况虽然有,可交替时并没有那么激烈,很是平稳结束了交接。虽然看着懦弱,但那份宠辱不惊的气度,都不是之前之后的几位天子能比得上的。

而眼下的这位小皇帝,虽然离亲政还远得很,但现在所表露出拉来的冷静、聪慧,不是年龄可以约束,甚至可以让人一时间忘掉了他的年龄。

当然,现在考虑那么多还早得很,新的皇帝,新的职位,新的差事。要衡量的地方越来越多,无论东西两府,都需要大量的时间去准备,至于赵煦今天做出的选择,也没什么好在意的,还早,毕竟还早!

解决了几件根本不需要议论的议题,接下来才算进入需要正式讨论的环节。

蔡确出班对向皇后道:“百官三军犒赏事不可拖延。需要尽快散发下去。还请殿下召三司使上殿,与两府共计议。”

向皇后没有意见,她也急着解决现在的问题。点了一名内侍,她吩咐道:“速去三司招吕嘉问。”

韩冈有意沈括。但沈括给王安石否定掉了。吕嘉问的能力,到底能不能适任,向皇后抱着悲观的态度。

之前要填补战争造成的亏空,所以要鼓铸大钱,但就这么简简单单的一件事,却闹得京城钱价大跌。要不是韩冈适时发表的那一篇《钱源》,京中的币值怎么稳定下来?而现在又开始波动,便是韩冈辞官的结果,吕嘉问这个三司使在中间,到底做了什么?

向皇后并不知道吕嘉问能给目前已经是苟延残喘的朝廷财计,开出什么样的一副药方,不过吕嘉问直接摊手向她要钱,也的确是在向皇后的意料之中。

吕嘉问也就这点能耐,怎么可能突然间冒出什么神机妙算来?

“就是韩冈来,也只能开内藏库。”吕嘉问放声直言。

向皇后哼了一声,她根本就不信吕嘉问的话,半句都不信。以韩冈水平怎么可能只会将手往天家的口袋里伸?都是没出息的儿子,才会总是惦记家里面的钱,而不是去外面为家里赚钱。

她斜睨着吕嘉问,也就这点出息!

吕嘉问对皇后的鄙视,似是毫无所觉,只说道,“巧妇难为无米之炊。没有钱粮,三军如何能安稳。三军若不安稳,京城都要要动荡。京师若不稳,将会是天下震荡。”吕嘉问又道,“运筹为何事,就是钱粮。人、财、物,哪一个都不能缺。”

“难道相公不知道内藏库中还有多少钱粮,随口一句就要上千万贯的来贴补。”向皇后抱怨着,“拿光了内藏库,天家靠什么稳定下来?吃穿用度怎么解决?难道天家的体面就不要了?”

吕嘉问苦口婆心:“殿下,国家安康才算是体面,若百姓皆怨,纵有万贯亿贯,这有什么用?”

“当年上皇就已经答应了,每年从内藏库中,拿出六十万银绢贴补国用。平常时的犒赏,朝廷拿一部分,内藏库再拿一部分。现在呢,六十万银绢都填不满你们的胃口。犒赏更是一文都不想出。是不是要欺负我这妇道人家?”

“殿下息怒。此事可以询之于众。”吕嘉问有些阴狠的给向皇后出了一个主意。

向皇后没上当,“想要人人都知道国库空了吗?还是想要天下人知道,你只能从吾这个妇道人家手上骗钱的。”

若向皇后当真听了吕嘉问的意见,到处问怎么才能不开内藏库就把赏钱发下去,这太上皇后以后也别出去见群臣了。抱着金库不可能犒赏百官三军,这是什么皇后?吝啬至如此,还能垂帘听政吗?

向皇后很愤怒,吕嘉问这根本就不是上奏的态度,完全是要拖人下水,让人不得不帮他一把。

“殿下息怒,吕嘉问也是一片向国之心,并非有意冒犯。”章惇出来支持他的同僚。

“那章卿你说怎办?”向皇后问道,“要犒赏的三军都属于枢密院来掌握,不知章卿有何高见?”

章惇自己站出来后,皇后的火力就转移到了他和韩冈身上,吕嘉问也不需要自己再为他保驾护航。

“臣亦无法。纵有,也赶不及秋税,更赶不及转眼就要散发的赏赐,正所谓远水解不了近渴。”

说来说去,还不是你们这些宰执无能。向皇后抱怨着,但她还是没有将话说出口。这不是她一个人的事,也不可能当着宰执们的面,去骂他们无能。这点面子还是要给韩绛、蔡确留下的。

只是等她听到,吕嘉问打算怎么铸钱,终于是忍不住了,“铸造各色钱币,防人盗铸,同时维系币值不跌,这些是吾早前听韩枢密说过的。根本是一模一样。不知吕卿除此策之外,还有没有自己想的办法?”

吕嘉问知道皇后会生气,早就有了心理准备,不管怎么做,他都会在朝堂上与人对立起来。当着皇后的面提起前任的策略,不论是最后的结果如何,都是会惹怒皇后维护旧臣的心思……

“当然一样,本来就是韩枢密想要用的计划。”吕嘉问并不遮掩,“嘉问愚鲁,三军犒赏只能想到内藏库。不过铸钱之事,嘉问考虑了很久,拾遗补缺,总算是有些眉目了。”

拾遗补缺在哪里?向皇后都不想生气了,早一点将眼前的事给解决掉,就能早一点结束今天的议事。

“就这么办好了。”她没什么力气的摆了一下手。

她已经不想再听吕嘉问说下去了,都是拾人牙慧,剽掇他人见识,可现在的问题,不是替人抱不平,而是将这些肯定行之有效的方略,早点在京城中施行,那样的话,困扰自己的问题也就解决了。

那么,也就不用看太多吕嘉问的一张脸。

另外还有一件事,既然吕嘉问用了韩冈的策略,她就可以找韩冈入觐,备咨询,正好用在这个时候。

如此定下来的策略,向皇后她才能觉得安心。

