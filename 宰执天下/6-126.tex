\section{第12章 锋芒早现意已彰(七)}

隆隆的雷音仿佛自天边传来。

但来自宋国的使者们仰头上望,看到的却是一碧如洗的万里晴空。

“是火炮。”

在炮声的间隔中,种建中轻声说着。

使团的成员都从帐中出来了,可无人接下种建中的话语。

使团中不会有人不清楚这是什么声音,在京冇城的时候,他们见识过太多,只是他们都没想到辽人能够拥有火炮,并展示出来。

这样的一件被视为替代八牛弩的神兵利器,大宋朝野的许多人都以为能够继续吓阻辽人几十年,可现在却已经出现在辽人的手中,并成为辽人用以炫耀武力的手段。

即便辽国窃取飞船的事例在前,且有关辽国制造火炮的流言也一直在流传,可是直到此时此刻之前,还是没人愿意相信谣言乃是现实。

王存强作欢颜,指着周围哈哈笑道:“这般小气,可见尚父还是心虚。”

“内翰说得是。”向英颤声应道。

一支千余人的宫分军,将使团的营地围了里三重外三重。看旗号,是来自景宗皇帝的那一支。本来使团中还有为辽国达官贵人问诊的医官,现在都被堵在了使馆内。王存说,这是辽人害怕使团成员窃取机密、戳穿底细,也不是说不通。

可是这样的解释,一厢情愿的内容太多了一点。

向英心中直念着阿弥陀佛,道理说得再好,也架不住鞑冇子发疯,谁知道耶律乙辛得到火炮之后,会如何骄狂?一群蛮夷,用道理去推测他们的想法,难道不是一厢情愿?

当真杀了他们这几个使节,朝廷还不是要顾全大局?就是太后想为自己叫屈,下面的重臣都会出来劝说太后不要为了区区小民闹得两国之间再起烽烟。相比起战争造成的损失,也算不了什么。

不仅仅是向英,使团中的很多人都是惨白着脸。这个道理,他们也能想明白。

耶律乙辛使用火炮,正是为了进一步收摄人心。而大宋使团的拒绝,又会对人心产生什么样的影响,稍微想象一下,就可以知道有多严重。

耶律乙辛篡位已是迫在眉睫,拒绝一名大权在握的皇帝,不给他面子,结果会如何——不论耶律乙辛是否正统,手中的权力却是实打实的——谁都能想明白。

但朝廷不可能与一名篡位者打交道,除非彻底不要脸了,又或是不值一提的小国。如辽国这等平起平坐的大国,双方天家又通过盟誓缔结了兄弟之约,在情在理,朝廷都不会承认耶律乙辛的篡逆之举。

朝廷视耶律乙辛为篡逆之贼,他们这个使团就是身处敌境,几十人的安危,便完全取决于耶律乙辛的理智和心情。

“可惜了。”种建中低声说道。

“如何可惜?”王存转头问道,并非质问,而是请教的语气。

种建中是种谔的侄子,使团的副冇使,可终究比不上横渠门下更受人敬重。

当年胡瑗主持太学,门下弟子皆以守礼著称朝野,一见举止便知是否为安定门人;而横渠门下,则是以文武兼备、长于实才而闻名。

已经高居庙堂的韩冈就不说了,游师雄也已入重臣之列,再传弟子黄裳如今正在西南,而今科状元、以知兵闻名朝中的宗泽,据闻也可算是张载的再传弟子。

而种建中这名出身将门世家的亲传弟子,也远比其余将门子弟更为出名。

种建中以明法科入仕,虽比不上进士科,但这也是朝中正经的文班出身之一,比荫补更为人尊重。在王存眼中,种建中仍可算是能够共语的士大夫,而不是粗鄙的武夫。同时种建中历经战火,如今的官位也是靠战功而来。如今众人身在辽国,为辽军围困,他对局势的判断,能决定使团上下行止。

“要是耶律乙辛能再迟两年篡位,朝廷就能出兵讨冇伐罪臣,为辽人拨乱反正了,如今多半只能坐视。”

不要迟两年,一年就够了!

向英心中大叫,要是耶律乙辛明年才篡位,肯定不会轮到自己来北方吃苦受累、担惊受怕了。至于怎么捡辽国的便宜,那就是两府诸公的事,与自己这等外戚没有任何关系。

王存苦笑起来,“彝叔,这是现在你我要考虑的事吗?”

“担心辽人是多余。”种建中笑道:“有些事即便在意,也在意不来。现在闲来无事,不去想如何用兵,还能想什么?”

种建中笑容中有着无奈。

身处矮墙下,不得不低头。耶律乙辛要真是想将使团发配北疆,他们根本没有反抗的余地。

一行人都在辽人的掌控下,要生要死全凭尚父殿下的一句话,除了坚持不与辽人苟合的态度,使团根本没有别的的办法。

如果朝廷想要趁机攻打辽国,理由都是现成的;如果朝廷不想作战,就算辽人杀光了使节团,开封那边也只会装聋作哑。

现在还不如多想想日后领军,怎么击败外面那群宫分军。

但朝廷是不会出兵的!

来自关西,在河东又有好友,对京师也了解甚深,种建中很清楚朝廷会做出的反应。

耶律乙辛赶着要篡位,现在的大宋却很难抓到这个机会。

要是连通京冇城和保州的轨道都已经修好,现在可就能直取燕蓟之地,将辽人驱逐至燕山以北。要是关中通往太原的轨道已经与并代铁路联通,关中的财赋也能支持河东用兵,收复云中。日后有机会,还能继续向北,彻底将契丹、女真等异族征服在汉家的车马之下。

可上一次的战争仅仅过去一年多的时间,残破不堪的河东路不说,就是程度较轻的河北路,也不足以储存出足够的粮秣,以供军用。而能够大量输送军资的轨道还在图纸上。

此时绝非适合出兵的时机,种建中半是遗憾、半是庆幸的想着。

现在与辽决战,能够领军出征的只有功成名就的一干将帅,李信、王舜臣、赵隆之辈,早已在战场上证明自己,他们还能有机会领有一军,可自己还没能来得及积攒战功和经验,朝廷岂会重用?再有满腹策谋,也只能望而兴叹。但再过些年,好歹能来得及轮到自己。

连最为知兵、又了解辽人的种建中都不看好前途,许多使团成员真的就绝望了,

炮声已经结束很久,使团营地还是死寂一片。

打破了寂静的是整整一天都没有过来的馆伴使。他向王存行过礼后,便说道:“奉尚父之命,今日晚间,请王大使、向、种二副使赴宴。”

向英脸色顿时煞白一片,他现在最怕的就是与耶律乙辛正面相遇。

不要说禅让大典,即便是普通的宴会,如果耶律乙辛穿着天子服来到席上,他们这些宋臣便不决能入席,必须掉头离开。

王存端端正正的向馆伴使回礼,接下了邀请。送了馆伴使离开,他回头对种建中、向英道,“各自去准备一下吧。”

向英惨笑着,“朝廷这时候恐怕还是什么都不知道吧!”

“不。”种建中摇头,“多半已经知道了。”

……………………

大宋朝廷此时已经收到消息了。

几乎是在得到河北边境急报的同时,一封密信便通过金牌急脚,从代州一路南下,用了三天冇的时间,抵达开封。

那是来自辽国内部的信件。不是通过河北,而是经由河东传信。

宋军曾经打进西京道,也顺道留下了些种子。

垂涎大宋的辽人很多,但憎恨耶律乙辛的为数更众。

一朝天子一朝臣。而耶律乙辛虽是耶律姓,却非太祖阿保机之后,作为一个新势力上台,他想要培养属于自己的势力,就必须去清洗早已盘根错节的朝堂。

尽管耶律乙辛掌握辽国朝政多年,但辽国很大程度上,依然是一个松散的国家体系,地方豪强势力众多,他想要重新征服,也没那么容易。

辽国国中,高官显爵中对耶律乙辛心服口服没有多少,如今大宋势强,耶律乙辛秉政又名不正言不顺,有了反对奸佞篡位的借口,愿意跟大宋这里合作的还是很有那么些人。不论怎么样,在大宋这边先留一个善缘总归不是坏事。

辽国内部对于局势的变化总要更敏感一点,耶律乙辛在上京道刚刚有了些动作,便立刻有人开始联络大宋。

从代州南下的金牌急脚,还附带了一封辽国乙室部一名重臣的密信。信中的内容无外乎邀请大宋出兵,为辽国拨乱反正。到了第二天,就是到了第二封,第三天,便是第三封、第四封。

有了这么多证据,辽国国内有变,此事已是确凿无疑。摆在大宋君臣面前的就这么两件事——

要不要承认耶律乙辛;以及拒绝承认之后,是否要出兵辽国。

不承认耶律乙辛是宰辅们共同的认识,韩冈在其中态度尤其强烈——他若不是儒门宗师的身冇份,还能讲一下变通,可他一旦主张,三纲五常还要不要讲了?即便韩冈对此嗤之以鼻,可现在还不是公然抛弃的时候。

但说道是否要出兵的时候,所有宰辅则一起表示反对,惟有一人例外。
