\section{第38章 一夜惊涛撼孤城(上)}

幽幽的灯火下,韩冈正在翻阅着一卷这两天运抵珂诺堡的兵械粮草的入库帐。

珂诺堡作为支撑前线两万大军的核心重镇,越来越,只是韩冈他眼神飘忽,焦点完全没有落在纸面上。

今日决战的消息就在开战后不久,就传到了他的耳中。前方正在决战,身在后方,韩冈怎么也不可能放下对战事的担忧,而心平气和的处理公事。他今天并没有耽搁政事,不过就连吃饭喝水,也一般的心不在焉。就像谢安的那般,就算淝水之战后,再怎么心平气和的说着‘小儿辈胜了’,出门时还是照样会心情激荡的不看脚下。

从河州城下,传回来的消息一个接着一个:

先是新型霹雳砲进抵河州城下;

接着是沉默已久的吐蕃人拼死反扑;

然后是两支吐蕃军,猛攻宋军的侧翼,但在大宋箭阵前,皆是无功而返;

继而又是模模糊糊的一句官军小挫,让韩冈担心不已;

但很快,姚兕统领泾原军大发神威,一举击溃当面的敌军的战报就传了过来;

而夜幕降临前,韩冈得到最新的消息,就在未时过后,霹雳砲开路为大军开路,硬是砸塌了河州城的东门,官军前锋冲上了,虽然没能立足,但破城已经近在眼前。

传回来的消息,一个比一个更让人振奋。只是韩冈看过自己这两日让人私下里制作的河州城下的沙盘,如果王韶继续往河州城中攻去,出战的大军将会与位于支流河谷中的本阵大营脱离连接,如果这时候吐蕃人攻打官军的后路,很有可能会吃上一个大亏。

想到这里,韩冈突的由摇头自嘲的一笑,他这算是白担心。跟着王韶的将领们,都是西军中才能卓异的名将,哪里会犯这样的错误。即便出了点岔子,也有足够的手段来弥补。

反倒是他这边,也许是香子城,也许是珂诺堡,两个地方至少有一处应该有动静了。不然等到王韶的帅旗飘扬在河州城中,再来偷袭后路,就会面对士气正旺的大军,根本没有什么机会了。

连禹臧花麻都拼了家底来攻打临洮堡,在这样的压力下,木征不会将希望放在硬拼这条路上。可以确定,只要能派上一点用场,木征必然都会选择用上。

就在韩冈这么想的时候。香子城外,散诸四野的游骑,以及立于各处高地上的暗哨,或前或后的都发现了有大队敌军自小路来袭。但他们发现时,来袭的吐蕃骑兵距离香子城已经只有十余里,这时候,想出去在道上伏击已经来不及。按照预先的计划,王舜臣立刻接管了香子城中的防务,并立刻派人向王韶和韩冈通报。

而香子城下游的珂诺堡,也在香子城急报抵达的片刻之后,内内外外响起了警号之声。三短、三长、再三短,这样的号声不是韩冈聚将的手段,而是预定中敌军来袭的暗号。

听到了报警的号角,韩冈微微皱了皱眉,不慌不忙的收起了刚刚传到手上的急报,还有铺在桌面上的帐册卷宗。站起身来,轻轻吹灭了油灯。

小帐融入黑暗中,韩冈走出了门外。此时夜色正浓,星汉灿烂,半轮残月挂于天际,将远近群山染上一层银辉。

城内一片喧嚣,被警哨惊动的士兵纷纷离开温暖的被窝。十几名就在城中的军官匆匆赶来,脚步急促,由远而近,立定在韩冈的身前。

韩冈正抬头望了望山巅半月,叹了口气,回头对着众将校道:“恶客临门呐!”

略显轻松的话语,让有些心慌的将校们平静下来。这时一阵轻微的脚步来到了身后,韩冈侧了侧头,是刘源赶了过来。

韩冈笑了笑,刘源所站的这个位置,可算是一个承诺了,就是想得有些太多。

“田琼!”

韩冈并不入帐点将下令,而是就站在星月之下,点起了城中唯一的一名马军指挥使。

四十多岁才混到一任指挥使的田琼,虽然不是什么惊才绝艳的名将,更没有王舜臣、赵隆那等惊世骇俗的武艺。但他本人在军中多年,凡事领命而行,做事中规中矩,胜为小胜,败则小败,虽不起眼,但足够可靠,是组成了几年来,战无不胜的熙河军的中坚力量。

他听到韩冈命令,当即踏步出列,躬身道:“请机宜吩咐。”

“你速领本部赶赴香子城,为之助守。让王舜臣必须坚守一夜,明日等我解决了骚扰珂诺堡的敌军后,必然会去领兵救援!”

田琼领了韩冈的命令,一名亲兵奉命又将韩冈的令箭递到了他的手中。将令箭揣入怀中,田琼便在行礼之后,匆匆离开。

韩冈不担心珂诺堡的守卫,虽然珂诺堡的重要性远过于香子城,但既然吐蕃人分兵攻打两座城堡,其中必然有一座是主攻方向。

而韩冈的判断……那个主攻方向是香子城,所以要命田琼在骚扰珂诺堡的敌军在抵达堡下之前快点出发。即便是他猜错了,但以珂诺堡眼下的守卫能力,也足以抵挡万人以上的攻击。

田琼走了。做事稳妥的他在赶过来之前,早就命跟在身边的亲兵去通知他麾下的战士。在他刚刚离开不久,韩冈还没有将守城的任务分派下去,一片蹄声便已经猝然响起,穿过吱吱呀呀打开的城门,向南方的香子城赶去了。

“希望田琼不要太急躁。”刘源在渐渐变小的蹄声中,压低声音对韩冈说着,“只要能让香子城中的守军知道他们到了,城中就不会有事了,不一定要进城的。”

“不用担心田琼,王经略会看重他,就是因为他做事一向稳重。”韩冈顿了顿,又用着只有自己才能听到的声音道,“即便是败了,他也能拖延上一阵时间。”

韩冈将任务一一分派下去,谁人该做什么,他早有腹案。而在这之前,也让这些军官们准备过——有备无患一向是韩冈众多座右铭中排得很靠前的一条——现在下令,也不过是走到程序,顺便激励和安抚一下人心。

十几名军官,得到了韩冈的命令后,一个个都赶赴自己的岗位上去。

待身前众官走了个干干净净,韩冈回头看看刘源,又点起了站在外侧的几个亲兵,一一向他们嘱咐:

“让住南二营里的那群民伕准备起来,去军器库中领取战具。”

“拿我的令箭,去军器库中,让狄四打开库门,给民伕们分发战具。一如禁军例,神臂弓包括在其中。”

“去疗养院,将占了五、六两病房的那群人给叫起来,让他们去军器库找他们的老下属……如果有人要问,就说这是我韩冈的严令,违者必斩!”

韩冈回转头来,冲刘源笑了一笑,“两边各自知根知底,想必不需要让我来分派谁该管谁!”

刘源却是越听越是惊骇,月光照在脸上,却是一片发白,最后竟化作一声惊叫,“机宜!此事还请三思!”

驱用叛军、驱用叛军将校,现在都不算什么罪过了。但让广锐将校重新统领其那群来自于广锐军的民伕,重新与他们的部下汇合在一处;让那支已经消失在枢密院和三衙的军籍名簿上的队伍,再一次出现在世间,这个罪名,不是韩冈一介朝官能担当得起的。

“没关系,事急从权。只要能打退贼人,怎么都能说的过去。”

韩冈对此事早有准备,他之前移文调集广锐叛卒组成的乡兵民伕来珂诺堡时,就已经有了这样的想法。虽然沈括在中间克扣了一半,但正好刘源这帮将校也因故折损近半,统领起来也不会变得官多兵少。

即便再是精锐的士兵,也只有在优秀的将校率领下,才能发挥出百分之一百二十的实力。广锐军何能例外?而在之此前,别看刘源这群由将校组成的队伍所向披靡,也别看广锐军卒也同样是能轻易压倒敌军。但他们所表现出来的战斗力,尚不及其应有战力的七成。

想要度过眼前的困局,就要将所有的手段都用上。重组区区广锐叛军又能怎么样,最后能赢就行了,到时再分开就不会有事……

在前线的战局尚未确定的情况下,韩冈绝不会让王韶有丝毫分心旁顾。这场决战同样关系到他的命运,韩冈不介意做些旁人不敢做的事情。就犯了点忌讳,只要有战功填补上去,天大的篓子都能给弥补起来。

韩冈对此事是胸有成竹。

再怎么说,他都是一名文官,而不是被人鄙视、提防的武将。

武将做出此事,重的夺职,轻的也要降责远恶军州,但没人会相信一位屡立战功的文官会有叛心,更不会对其忌惮。只会说他有能力,能驭使叛军。

安抚似的拍了拍刘源的肩膀,韩冈笑着道:“去城头上看看!……先看看是吐蕃人中的哪路豪杰,来攻打我这珂诺堡,再等着你的兵过来!”

