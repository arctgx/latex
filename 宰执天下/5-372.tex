\section{第34章 为慕升平拟休兵(四)}

位于忻口寨的宋军越发得咄咄逼人,其对神武县的控制也一天比一天更为紧密。

摆在代州辽军面前的只剩三条路。

一条是坐守代州,第二条是反击忻口寨,第三条则是争取收复武州神武县,之前已经试过两次,可都失败了,若还想收复,就意味着必须从代州抽调更多的兵力回来。

现在最大的问题是不是哪个选择,最后都不会有太多战利品。

宋人能下大本钱去激励那群猪狗日出的阻卜人,可萧十三做不到。就算他能放话说收复武州赏多少多少,攻下神武县城赏多少多少,砍下折克行的人头又赏多少多少,可这也要有人信才行!大辽的家底有多少,他麾下有几人不清楚?!

伤亡高、收益低,萧十三能动用的只有他的本部兵马。可这万余精锐,就是尚父攥住西京道的手。如果没有这两支亲附尚父的皮室军和宫分军,西京道上的大小部族绝不会对耶律乙辛俯首帖耳。甚至仅仅是受到大一点的损失,除了一个大同,其他州府,萧十三都没有把握控制住。

而主动进攻宋军的防线,同样是人人避之不及。这些天来,光是斥候探马就损失了上百人了。宋军骑兵随身带着三张事先上好弦的神臂弓,一接战就先射箭,根本就不在乎这样会造成多少弩弓损耗。财大气粗到让萧十三连愤恨的力气都没有了。

所以到了最后,做出的决定就是等待宋军来攻打代州。等他们出了忻口寨之后,就可以利用契丹精骑在野战中击败他们。至于斥候,宋军财大气粗让人无力,还是减少一点。反正大军出动,是瞒不住任何人的。

这一方略,不是最好的,也不是次好的,甚至不是第三好的,明确的说,是让人可以继续观望下去的方略。只是等而已。

与其说是对战事的规划部署,不如说就是简单的两个字——再议!

张孝杰为此急得上了火,嘴角边好大一个血燎泡。而萧十三则是头发掉的厉害。

不过当一封信函从南京道经过飞狐陉送抵代州,萧十三和张孝杰终于是松了一口气,他们再也不用担心了。

“尚父终于是下定决心了。”张孝杰轻声一叹,将来自南京道的军令递还给萧十三,上面有着耶律乙辛向他通报的最新战况,以及朝廷议定的方略:“早就该与宋人好好谈一谈了。”

“之前不可能啊。”萧十三摇了摇头。之前从上到下还有着改变战局的实力和欲望,想与宋人和谈不可能得到支持。

虽然战争中间主动联系南朝,的确有些丢人。不过再这样坚持下去,不知还要损失多少儿郎和战马。这段时间萧十三整日价的心惊肉跳,生怕下面的人再也忍不下去,直接带着兵马就撤退。

现在终于是好了,一旦辽宋两国展开和谈,就又能恢复到澶渊之盟时的和睦。安享宋人送来的好处。

“我现在就怕忻口寨的那一位不甘心!”

“由得了他做主吗?”萧十三哈哈大笑,“宋人的皇帝病得快死了,现在由一个妇人主政,而朝堂上,更是无能之辈居于显要之地。有着这个时间担心他,还不如多照看一下军中的士卒。”

张孝杰摇头,“损伤又不大,有什么好看的。”

萧十三轻松的点着头,不算战马的话,的确是损失不大。

西夏故地那边虽然是耶律乙辛的支持者,但都是一些首鼠两端的支持者。所谓的支持,只存在于口头上,以及耶律乙辛收拾反对者时,他们站干岸看风色的行动上。耶律乙辛将西平六州分配给他们,不过是给外人看的,顺便还收回了一部分更靠近国家中心的土地,还有朝堂上多出来的位置。而且还是异族为多,死活都没人在乎。

而忠诚于尚父殿下的主力,在这一场战争中都没有受到太大的损失,加起来也不超过万人。无论是在河东,还是在河北,都是宋人的伤亡更大一点。

正是有着这样的战果,萧十三和张孝杰才有信心,南朝的皇后和她所任用的南朝宰辅绝不愿意将战事再继续下去。

韩冈不甘心又能如何?他现在正在做着收复代州的准备又如何?他可是远在河东,手伸不到开封城去!

如此一来,他要么就去遵从南朝朝廷的诏令。要么就必须尽快有所行动有所收获,使南朝朝廷觉得继续打下去比和谈的好处更多。可那时候,大辽扭转局势的机会也就藏在其中了。

关闭<广告>

……………………

韩中信懵懵懂懂的辞了韩冈出来。

他虽然机灵干练,头脑灵活,可韩冈说的话他却是想不通。

分明就是有澶渊之盟在前,辽人都求和了,朝廷怎么可能还会追究下去?百姓的仇怨终究比不上一份双方罢兵的盟誓。就是韩冈的本心,也是不想这一场战争继续下去。

但韩冈为何能这么自信,朝廷会坚持对战争中百姓的损失穷究到底?难道是要将在外君命有所不受?

韩中信不能不为韩冈担心。若是惹怒了朝廷,以韩冈的身份纵然不惧,但终究不是什么好事。而且朝廷想要干扰前线的作战,也是容易得很。忻州、太原的知府,以及来自京营的将领,包括折家,只要朝廷有诏令到,他们只会听朝廷的话。

不过就算是在苦思冥想中,韩中信也没忘记向刚刚回返、正准备拜见韩冈的黄裳行上一礼,“机宜回来了?”

黄裳则诧异的看了韩冈的这位旧仆一眼:“韩信,怎么神不守舍的,出了什么事?”

韩中信不敢有所隐瞒,也忘了提醒黄裳自己已经改了名字,将整件事原原本本的向黄裳说了一通。并且还将自己的疑问也说了出来。

“这是因为枢密是枢密的缘故啊。”黄裳当即笑道,他看了看一头雾水的韩中信,“没听明白?”

韩中信摇摇头,很绕口的一句话,他怎么可能听得明白。恭恭敬敬的向黄裳行了一礼,“还请机宜指点。”

“你可知道什么是朝廷?……所谓朝廷,往大里说是在京的文武百官。往小里说可就仅仅是天子和两府。天下军国事,无不是天子和两府来处断。为与士大夫治天下……”黄裳为文彦博当年的话冷笑了一下,这一句自宫中传出来后已在世间流传多年,让许多士人为之击节叫好,可有多少人想过,他们够资格被文彦博看作共治天下的士大夫吗?“其实是与两府共治天下。所以我才会说,‘因为枢密是枢密的缘故’”

韩中信沉沉的点头,他算是明白了。

他的恩主,现在就是朝廷的一员。朝廷的决定,必须得到韩冈的同意。至少在河东战事上,是毋庸置疑的。

……………………

方才韩中信离开时,脸上犹存疑惑。

韩冈看到了。他也知道,自己的话听起来甚至有些矛盾的成分在。不过韩冈的确是半点不担心,东京城那边敢绕过他来与辽人达成协议。

原因很简单,他本人就是朝廷的代表之一。而且在军事上拥有等同甚至超过宰相的发言权。没有他的点头,与辽人和谈就不会有任何结果。

这已不仅仅是担任枢密副使之前单纯依靠经历和威望得到的发言权,现在更是已经加上了制度的保证。

除非他在河东失败了,否则权威加上制度得到了权力,是不可能动摇的。

没有对辽人求和的消息多费神,韩冈继续批阅他面前的公文,这时黄裳在外通了名进来。

韩冈立刻放下笔,他等黄裳回来等了好几天了,“情况怎么样?”

黄裳行过礼后就点头:“全都准备好了,可保万无一失,秦玑做事还算妥当。”

“还是觉得我把秦玑派给你不合适?”

“其实此事交给韩信来做是最妥当的。”

“不是韩信了。”韩冈更正道,“中信,中庸的中、智信仁勇严的信,韩中信!表字守德。”

“是枢密所赐?”黄裳眨了眨眼睛,没想明白哪个典故能把中信和守德联系在一起的。

“嗯。毕竟要做官了,不能再用旧名了。他之前跟秦含之配合得甚好,这一回就让他们继续配合。至于那件事,秦怀信壮年而亡,实在是可惜,看在他的份上,便给他儿子一个机会。左右有勉仲你在旁看着,具体的操作谁来主持都一样。”

“多谢枢密的看重,黄裳必不负枢密所托。”黄裳谢过韩冈的信任,又问起了辽人遣使求和的事,“郭仲通怎么会想起来传话给枢密。是不是他对和谈有什么看法?还是他想要知道枢密的态度?”

“郭仲通不会有任何看法:因为他不是文官。他也不是来确定我的态度:因为就算早一步知道,对他也没有任何意义。”

韩冈一口否定了黄裳的猜测。郭逵想来在文武之别上谨守本分,不会做出任何与他身份不合的事来。这仅仅是单纯的通知,或许带一点示好的成分在。毕竟这么重要的一桩事,早一步收到,就能早一步做出布置。

“枢密打算怎么办?”

“前面我跟守德说了:他们谈他们的,我们打我们的。按部就班,稳扎稳打,给我打到代州城下!”

一日后,秦琬和韩中信奉命出阵,领军开赴土墱寨。
