\section{第30章 肘腋萧墙暮色凉(九)}

与种谔把将要施行一应事务敲定,韩冈便告辞离开。种建中留了下来,韩冈的建议,还要他来具体承办。而种朴则说是要送韩冈,趁机跑了出来。

离开主帐,韩冈并没有回自己的住处,而是往城下的疗养院去。

就算在上元夜的深夜,罗兀城的工程也完全没有停歇的意思。一群群民伕,有气无力的喊着号子,站在已经初具规模的城墙墙体上,牵着木桩上的绳索,一下一下的夯着新堆上去的泥土,将城墙一点点的加高上去。用木板做成框子,里面留出的空间堆满黄土,再用桩子夯实,就是如今通用的板筑,其坚固程度并不输给砖石。

每一处墙头,都能看到夯筑民伕的身影。而不仅仅是城墙上,城中规划好各处建筑的地方,都有民伕伴随着咚咚的夯土声喊着号子。而城墙外围还有数千民伕,拼命挖掘着壕河,其中取出的泥土,正好用来可以修筑城墙和建筑。

按照此时的计算方法,一个民伕完整做完一天的工作,计为十个工。普通的寨堡,大约在二十万到四十万工,比如新渭源堡,是双堡夹河的结构,新筑北堡是四十万工,扩建的南堡是三十三万工。而罗兀城的工程量,则是一万民伕一个月的工数,也就是说,总计三百万工!

这个数字在北宋已经是了不得的大工程了,工数几乎跟当年秦州州城的扩建差不多——秦州城可是周长近十里的州城——而且还是集中在一个月内完工。

一般情况下,修筑城池的工程都不会聚集这么多人力。一方面,管理上的压力实在太大,另一方面,粮草供应上的麻烦,也足以让管理后勤的官员疯掉。正常的千步军城的修造,标准工期都在百日以上,而当年秦州为了赶修甘谷城,秦凤路全境动员,也花了五十多天。但今次韩绛、种谔为了赶在西夏人反击之前完成,预留得时间就是一个月。所以才拼命的堆上人力——光是在隆冬季节从冻得如铁一般的地上取土,好用来夯筑城墙,就用去了近四分之一的人力。

不过忙碌归忙碌,一见到韩冈,周围的士兵、民伕,便纷纷跪拜下来,有的还连连磕头,脸都贴在地面上。

这样的场景,韩冈倒是见多了,不以为意。在古渭,那些虔信浮屠的蕃人,做得更夸张的也有。但种朴倒是羡慕不已,以他的衙内身份,下面的士卒也的确要向他跪拜,但如此虔心的,可是一个都没有。

在罗兀城周边,总计三万余士卒民伕心目中,韩冈的名声极好。救死扶伤的医生,拯危助困的官人,任何时候都是能得到他人的尊敬。而在韩冈到来之前,其实也已经颇受期待——种谔为了安抚人心,把韩冈的事迹向民伕和士兵进行宣传,也是主因之一。

韩冈一边点头回礼,一边问着种朴:“抚宁堡那里情况如何?”

罗兀城是罗兀防线的核心,但与之同属一个防御体系的在建寨堡还有两处,抚宁堡就是其中之一。位于罗兀城的侧后方,守护着罗兀与绥德之间的交通线。现在种谔的副将折继世,就在那里主持营造工程。

韩冈前日往罗兀城来,就从抚宁堡工地的旁边过去,不过因为赶着到种谔这里报到,没有分心去看——从程序上,也必须是到了种谔这里报到之后,才有资格去巡视工地。

韩冈这两天和种建中都在罗兀城忙着,倒是负责逃卒和民伕的种朴去了抚宁堡一趟。

听到韩冈想问,种朴踌躇了一下,“……折继世去年得了风疾,天子都派了御医来看护。虽然命是救回来了,也没哪里瘫了不能动弹,可现在就是时常头晕,经不起累,性子也躁了点。”

韩冈瞥了种朴一眼,从他顾左右而言他的样子,抚宁堡的情况可能不太好。不过韩冈也不在乎,他现在唯一能肯定的是今次一战必败,作为一名管勾伤病的官员,对于这一等级的国战,并没有改变局势能力,而且也没有那个心思。他只把自己的事做好就行了。

“方才忘了跟大帅说了,明天我想去抚宁堡看一看。那里的工数只有罗兀的十分之一,如果民伕管理得好的话,应该比罗兀城更快完成。”

从预定的工期来算,不论罗兀还是抚宁,都不会超过三十天。

种朴听到韩冈要去抚宁,道,“玉昆你明天去抚宁,顺便把粮草给送去。上次运去的粮食,那里的该吃完了。”

“我知道了。”韩冈点点头,顺路而已,他回头望了望满是存粮的罗兀旧城,“也幸好罗兀城这里西贼囤积了足够的粮草。要不然改从绥德运粮来,任谁来也只能束手无策。”

种朴笑道:“西贼这是自作自受,本是为了开春南侵的储备,现在全都便宜了我们了。”

西夏人囤积在罗兀城的粮草,就是为了南侵。如果是秋后出征,可以轻易的就食与敌,但在开春时南侵,就必须自备口粮,以防劫掠不足。

而把粮草堆放在罗兀,山南的粮草理所当然的该存在山南,没必要运到山北的银州。从银州到罗兀,这十里的山道,骑马过来很方便,但运送辎重就麻烦了。把从横山蕃部勒索来的存粮,先翻山运到银州存放,等到出兵时,再翻山运回来,西贼也没那么多人力畜力。

当然,这也是西夏人本来就没想过离着绥德六十多里的罗兀城会被攻打,更没料到会被攻破。而当时守卫罗兀的西贼将领,只记得放烽火求援,却舍不得焚烧粮草。而当城池被攻破,再下令放火,刚刚点起火头,就立刻被早有准备的宋军给扑灭了。

“故智将务食于敌。食敌一钟,当吾二十钟。”韩冈背着《孙子兵法》里的应景章节,种朴听着自己老子被称赞,也是感到与有荣焉。

……………………

兴庆府的王宫中,梁太后、梁乙埋兄妹,还有一众重臣,正会聚一堂,讨论着眼前的局势。

罗兀失守,横山即将沦陷,前日消息传到兴庆府,整个西夏小朝廷都被这场千里之外的地震给惊呆了。垂帘听政的梁太后当即下旨,把国中能立刻动员地来的精锐全数征发,但各个部族却有些阳奉阴违。

所有的党项部族都知道横山是国之命脉,但半年前以举国之兵南侵,却近乎于无功而返,出战的部族人力物力还有士气都损耗极大。如今宋人一反常态,主动攻击。其气势汹汹,让许多部族暗地里都起了心思。

但梁太后和梁乙埋这对兄妹倒是安之若素,幸好他们事先早有了准备,若没有现在这个后手,还真是要出乱子。

梁乙埋的亲信罔萌讹,前些日子奉命秘密去了辽国,也是刚回来了不久。他带回来的消息,让梁氏兄妹有底气去通知各个部族和重臣。因为罔萌讹见到了辽国的太师赵王,并从他那里得到亲笔手书和许诺。

大辽太师、赵王耶律乙辛是如今把持辽国朝政的权臣,与梁乙埋在西夏的地位相当。西夏国的大臣们,当然不会不知。他的承诺,比起沉浸在游猎之中的辽主耶律洪基,要靠谱一百倍。

“我大夏也受了辽国册书。辽国当不容宋人欺凌于我。赵王亲口许诺,如果宋人犯我疆界,意欲灭我而后快,当以二十万大军助我!”

梁太后当日在朝堂上,把耶律乙辛的亲笔手书向大臣们炫耀时,声音提得极高。

辽国不会坐视宋人吞并西夏,这就是梁氏兄妹想要向国中传递的信息。

宋人也许会天真的奢望,维系了七十年的澶渊之盟会继续维持下去。但同为蕃人,党项人却很清楚,盟约就是拿来撕毁的,他们跟宋人签订的和约不止一次,可都是刚拿到了岁币,转过脸来,就去宋境去劫掠。维系盟约的关键,不再盟约本身,而是在于实质上的利益是否值得去维护。

梁乙埋很有信心,他能确定西夏的存续,对辽人来说,比起五十万岁币更为重要——而且也不需要辽人真的出兵,只要做个姿态,宋人还敢冒险吗——而党项各部,和手绾兵权的重臣们,也都通过耶律乙辛亲笔书信确认了这一点。

这几日,逐步汇聚在兴庆府外的部族军已经超过三万,加上原本就驻扎在兴庆府的五万常备兵,已经占到了国中正常调兵极限的半数。兵力不断增强,让众臣们的信心倍增,开始高呼着要夺回罗兀城。

一名亲信的内侍这时小碎步的跑进殿中,高声禀报:“秉太后、国相,黑山军司团练使赫里颜率本部两千已抵达北门外!”

听到这个消息,殿上骚动起来。

“赫里颜也来了,他可是平日里走得最慢的一个,不看到好处,绝不出手的。”

“看到他都来了,其他还在观望的,当是也会出动了。”

“再等两日,兴庆府的兵力肯定能超过十万。”

“不等了。”梁乙埋有了决断,“宋人那里正在加紧增修罗兀城,拖上一日,我们要夺回罗兀就难上一分。我们先走,让后面的自己赶过来!”

