\section{第24章 携眷西返家(上)}

【这是补昨天的份,十二点左右,还有一更】

拜过天地,喝过合卺酒,王旖第三次被送入洞房。

依照此时的婚仪,新妇迎进门后,先入洞房坐床,名为正坐富贵礼。然后新郎用同心结牵着新妇出去拜天地,先祖和父母。新妇的盖头,由子女双全的妇人拿着机杼——也就是织布机的梭子——来挑起。最后于厅中,用破成两半的匏【葫芦】为酒器,交换着喝过交杯酒,然后第二次送入洞房,而这一次则是新妇反过来牵着新郎。

重入洞房后,协理婚宴的妇人将合卺酒所用的两瓣葫芦一正一反的放在床下。两新人又要掩帐换妆,换下上黑下黄的大礼服,王旖换了身大红吉服,而韩冈则是套了身绿袍、戴上了花幞头。在礼官的催促下,出去敬过亲朋好友三盏酒。到了这时候,才没有了王旖这位新娘的事。

作为新郎官,韩冈还要继续应付一下客人,而王旖坐在洞房中床边,低垂着头。

两根儿臂粗细的龙涎香烛,映得洞房中通亮。天子赐予的绸缎和器皿,与诏书一起,供在桌前。大红色的喜帐,被两支金钩挂在了床沿。

洞房之中,除了王旖之外,只有陪嫁的两名使女,平日里就在服侍着王旖。不过在这时候,新妇不便说话和动作。两名亲近的使女,也都遵照事先的教训,如木雕一般站在不敢乱说乱动。

王旖静静的坐在床边,呼吸都是柔柔细细,身子一点也不动弹。只是红色的丝巾绞在手中,抓得紧紧的,显出了她心中一点也不平静。

方才坐床时,当韩冈一坐在身边,她浑身都立刻绷紧起来。并不是出于畏惧,而是不习惯和紧张。

第一次听说韩冈这个名字,尚是在三年前。那时候,她还只是把韩冈的经历当作是唐人传奇一般的故事听着,就像小时候听着张乖崖侠客行径的传说。不论是在军器库中射杀三贼,还是在送粮途中与寇博命,都是一波三折,让人听着都不禁为其提心吊胆。听过韩冈的故事,王旖对他当时是有了几分好奇,但却从来没有想过日后有什么交集。

后来,因为韩冈跟着二哥交好,两边渐渐有了书信往来。王韶从秦州遣人送信上京时,韩冈也会随之带来一声问候。在二哥王旁口中,便时常听到听到这个名字。

而韩冈参与的河湟开边,是父亲最为关心几桩事之一。就算以王旖对时政的浅薄了解,也很清楚熙河方向上的开疆拓土,对于一力主张加强军备的父亲有多么重要。因而他的名字,在王旖最为敬仰的父亲的嘴里,出现得也是越来越多。加之在关西的一桩桩功业,当父兄与人一谈起当今朝中的年轻俊杰来,韩冈这个名字往往都能排在前面。

而很快,一直为自己担心着的母亲,也不时的提起韩冈。到了这时候,父母的心意也渐渐的明了起来。论起才能、功业、品貌甚至名声,韩冈都是很好的。王旖也知道,就算是少年时就已经声名大噪的大哥,在功绩上也很难跟他相比。

只不过要看夫婿,也不能只看这些地方。

姐夫吴安持是枢密使的儿子,学问、相貌、人品也皆不差,而且幼年时还是见过面的,与大姐更是青梅竹马的身份。两家是门当户对,无论哪一方面都没有半点可挑剔。但是这样的婚姻,最终还是成了一个悲剧。

大姐未出嫁时是多活泼的性子,蹴鞠、秋千都是她带着自己玩着。但嫁到吴家几年过后,便一下变得少言寡欲,浑身暮气,新近做的诗词,也满篇尽是悲意。这两年,大姐只要回来省亲一次,母亲就会哭上一次,连着父亲也是好几天都阴沉着脸。

王、吴两家原本都是走得极为亲近,要不然也不会结下亲家,只是现在反目成仇,让大姐在婆家饱受责难。王旖真的很害怕自己最后会变成大姐那般。让父母伤心,是做子女最不孝顺的表现,还不如不出嫁,丫角终老——当日去见韩冈的时候,王旖当真是这么再想。

只是……

咿呀一声,洞房的房门这时被人从外推开。

一群人笑着在外面将身穿绿衣的韩冈推了进来,乱哄哄的说了一通好话,然后大队人马又去了前厅。

正式婚礼的酒宴应酬,不像韩冈早前纳妾那般是由本人负责,而是由知客来应对。韩冈出来后,只是向客人敬五六盏酒,受了他委托的王厚和冯从义便代他招待起客人来。

新郎进了房,如同雕塑一般的两名使女识趣的退了出去,在外面轻手轻脚的关上了房门。

房中变得只有两个人,王旖觉得自己的心脏跳得厉害,不知道走过来的那人是不是听到了。

韩冈见着坐在床边,绷得僵硬的王旖觉到有些好笑。方才就感觉到,心惊胆战的把自己当虎狼一般。

“怎么?”韩冈走过去,“还是害怕我?”

王旖摇着头,但随着韩冈走近,就变得更加慌张起来,一时不知该说什么。混乱中,一直转在心中的疑惑翻腾了上来:“官人……官人……官人你为什么要娶奴家?”

“娘子你该不会自那天后就一直在想?”

看到王旖的点头,韩冈笑了。想不到自己竟然给她带了这么深的疑惑。他虽然是喜欢算计人心,但总有疲累厌倦的时候。回到家中,对家人便不想动什么心眼,有话尽量直说,“虽然说一开始不免有些其他原因,但我娶你,只是因为你当日是为父母来找我。”

韩冈看重王旖的就是这份孝心。以他的身份,政治婚姻是避免不了,想自由恋爱根本是痴心幻想。能碰上一个孝顺父母、心地好的女孩子,那是再难得不过,遇上就不能放手。

坐到王旖的身边,韩冈将她的手强拉过来攥在掌心里。另一只手强硬的托着王旖小巧下巴,转到正对着自己,向慌张羞涩的双瞳中深深望进去:“娘子你有一辈子的时间来看我说得到底是真是假。现在只需要看着我,不要想其他事!”

韩冈动作很是强硬,被王旖压在心底的记忆被打开,当日在清风楼上,被韩冈强吻的一幕,一下又浮了上来。

午夜梦回时,都害羞得惊醒的那份记忆,此时又再现在洞房中。

双唇离开被吻得喘不过气来的妻子,韩冈的手又探上了她的腰间。

王旖不敢动弹,紧紧的闭着眼睛。在出嫁前,王旖被教授过男女方面的知识。就在压在箱笼底下,还有几本春图,连同几个几种姿势的瓷塑像。只是到了临阵之时,被母亲和叔母一番教诲后强记下来的东西,一下子就不知了去向。

王旖僵硬着身子,家中谨守礼法,虽然不至于男女七岁不同席那般严苛,但过了十岁之后,父兄连她的闺房都不再踏入一步,更别说被陌生男子触碰。她强忍着羞涩,但还是听着韩冈的话,任由他解开罗裙,将衣衫一件件退开。

韩冈主动引导着动作笨拙的妻子,动作也是尽量温柔。直接触碰到肌肤,韩冈立刻感觉到正在触碰的娇躯一下又绷紧起来。当他的手拿开,王旖才放松了下来。但他重又触摸到细腻柔软的酥胸,身子又再一次绷紧。

韩冈不由笑了起来,觉得这样的女孩子,当真是单纯得可爱。将被剥得如白羊一般的娇躯放倒在绣着鸳鸯的锦缎上:“春宵一刻值千金,娘子……我们也该安歇了。”

……………………

一声拖长了声调的鸡鸣,让帘幕低垂的床榻有了动静。

王旖撑着床铺,勉力想坐起来。可是平常的时候,很轻松的动作。不仅仅是下身私密之处火辣辣的疼着,身子骨也几乎被揉散了,浑身上下一点力气都没有。一想起昨夜,从一开始的僵硬拘束,再到后来不由自主的迎合,她就忍不住红了脸。不敢看躺在身边的夫婿,只用尽双臂的气力想要坐起来。

当她快要起来的时候,一只手突然按在腰间。王旖浑身一惊,双臂中好不容易才积攒一下就没了,登时就倒在了一副坚实的胸膛中。

韩冈搂着纤细柔软的腰肢,在妻子耳边轻笑着:“待晓堂前拜舅姑。起这么早不知要拜谁?”

若是在家中成婚,婚礼的第二天,新妇还有一道上拜舅姑的程序要走。要鸡鸣即起,洗手做羹汤,然后奉于舅姑,也就是公公婆婆——当然,这是后世的称呼。但王旖不需要,韩冈的父母又不在京城,她起来后,根本就没有长辈需要拜见。

王旖被韩冈搂在怀中,几下挣扎不开,红晕着脸,就不敢再动弹,声音低低的:“奴家要服侍官人,不能起得迟了。”

“你昨夜服侍得够好了。”韩冈咬着耳朵一声笑,“也没能好好睡,今天没必要这么早,再睡一会儿也没关系。”

因为韩冈的话,王旖的脸热得发烫,乖乖的嗯了两声。

韩冈几个月都没近女色,需索过甚,王旖初承恩泽当然吃不消,很快又沉沉的睡了过去。

当看着妻子又睡去后,韩冈则精力充沛去外面活动了一下筋骨,洗澡更衣,就觉得浑身神清气爽。

回头望望洞房,人生大事也算定了。

