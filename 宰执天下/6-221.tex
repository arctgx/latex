\section{第28章 夜钟初闻已生潮(三)}

“辽国伪帝当真是要封工匠王爵?”

厅中的幕僚,或不解,或羡慕,或嫉妒,或冷嘲,反应不一而足。

众人的态度尽收眼底,吕惠卿点头,“千真万确。”

吕惠卿收到这个消息的时候,并不比别人迟到哪里,一开始他还以为是说书人说《九域》时改编出来的故事,随后在口耳相传中,以讹传讹的变成了惊动天下的谣言。直到数日后,他方才知道这竟然是事实,耶律乙辛不惜王爵之赏,用以招揽能工巧匠。

这其中有传播距离太远导致了信息扭曲了缘故,吕惠卿怎么也不可能相信耶律乙辛会封出一字并肩王这种玩意儿,另一方面,也是吕惠卿不相信堂堂一国之君,会轻忽君子,任用小人,把工匠置于儒生之上。

不过面对一名窃国大盗,世人可以说他品性,却不能说他的眼光。尤其是在他的统治下,大辽的势力日渐扩张,灭高丽,灭日本,国势昌盛,若不是大宋的国力,也在同步增长,辽主早就观兵开封府。

辽国能大举扩张,依靠的是甲坚兵利。而这一切,都是从大宋这里学来,从韩冈手中学来,即使吕惠卿一贯敌视韩冈,也无颜否认这一点。

但耶律乙辛如此重视老对手,这让吕惠卿心中未免泛起一阵酸味。

二十年前韩冈不过区区一措大,现在却已经高居朝堂之上。

他所主张的气学,也是自成一体。张载留下气学已经不存在了,剩下的只是韩冈的气学。如果剥下儒家的外皮,下面的完全是匠人、农人之学,讲究着技近乎道,却把根本都抛弃了。

留下了心浮气躁的幕僚们,吕惠卿离开了公厅,返身往后院走去。

冷静下来之后,吕惠卿却不觉得耶律乙辛的选择错了。

辽国一贯弃儒如敝履,也不闻其国事因此而衰颓。五季之时,早有人喊出了‘天子者,兵强马壮为之’。

得天下也好,昨天下也好,并不是一定非儒不可。

文景治世,治国的是黄老之说,汉武独尊儒术,天下户口减半。汉宣帝说汉家制度是‘王霸道杂辅之,奈何纯任德教,用周政乎?’

到了后汉,图谶成了儒门显学,放在如今,儒林之中若有谁主张图谶之学,绝无其容身之地。

名义上吕惠卿也是当世大儒之一,新学学子皆从其学,但实际上,他对儒学并没有那么大的坚持。真正的儒生,早就不存在了。当今大儒,无不是拿圣人之言证一己之见,当真孔子复生,怕也是被打成异类。

儒门千年来一变再变,前日为显学,今日为异端,哪个才是真正的儒家?

二程那边,会说众论皆有失,皓首穷经不若穷究道德性命,以明其理。韩冈会说,圣道邈不可及,需要不断追索,日渐日新,才能近于圣人之道,而如何追索,就要靠格物致知了。

“耶律乙辛这是在帮韩冈吗?”

吕耆卿跟在吕惠卿的身后,不解的问道。

吕惠卿摇头,“耶律乙辛只会恨韩三不死,帮他作甚?”

“那是不是离间之计,让朝廷提防工匠……朝廷中必有人会上当。”

吕惠卿闻言失笑,他这兄弟异想天开惯了,想得太曲折,哪里有人会这么糊涂?当年张元吴昊投党项,得了偌大的富贵,引得陕西人心浮动,可没人说将落第的士人都抓起来砍了。

现在就把国中的能工巧匠都管束起来,这是帮辽国大忙。

被吕惠卿的连续否定,吕耆卿也不猜了,随着吕惠卿,慢慢走,问道,“不知朝廷会怎么样处置?是提高悬赏吗?”

“耶律乙辛敢做,是他不怕有人反对他,自家的产业,想怎么处置就怎么处置?韩三就是想要多拿出点好处,朝廷上都会有人非议,他总不能拿出朝官或大使臣给人。”

吕耆卿摇头。

莫说韩冈给不了,就是他当真拿出了升朝官和大使臣的官位赏人,也肯定比不上一个郡王。

“那他怎么做?”

“什么都不用做,等到辽人打上门来,自然不会有人再拦着他了。”

“或许此事正如韩冈所愿。”吕耆卿低声道,“耶律乙辛远在万里之外,如何得知蒸汽机事?若非韩冈,又有几人知道蒸汽机。耶律乙辛如此作为,或许正入其彀中。”

“你想太多了。”

“或许是小弟想多了。不过如今韩冈威信日高,声望日隆,日后若有变故,他想做个纯臣,下面的人也不会答应了。”

吕惠卿皱起眉:“十七,慎言!”

吕耆卿笑了笑,“不过申生居内而亡,重耳在外而生。韩冈虽得太后信重,却不免得罪了官家。如今兄长,跳出了那汪浑水,只要再等几年,自然能回到朝堂中。”

吕惠卿摇了摇头,他并不怎么担心自己的前途。

王安石在江宁府创立了金陵书院,每日教书育人,忙忙碌碌,过得好不开心。大多数的时候,新学内部的事务都交给了吕惠卿。

章惇不愿意引用王安石旧年的党羽,又与韩冈和睦相处,许多人因此而投靠了吕惠卿。皇帝的经筵上的侍讲,气学和新学各半,新学的几位侍讲中,又有一半亲近吕惠卿。吕惠卿很容易便能够通过那几位侍讲,对天子施加影响。

等到天子亲政,对朝政自然会有所更易,到时候能让他挑选来替代韩冈的臣子,又能有几人?

当然,若是太后想做章献,韩冈又能不要脸皮,吕惠卿倒也不在乎多等几年。

“不说此事了。”吕惠卿不想继续这个话题,“你去晋江看过了,那边的情况怎么样?”

“小弟前日已经去晋江那边看过了,不过只看了缫丝厂。章家新修的缫丝厂占地近三十亩,招了数百工人居其中,剥茧,选茧,煮茧、缫丝、整理。成品还要抽取检验,一条条依序而行,生产的生丝虽不如过去收上来的最好的,但也是在上等,而且质地均匀,。”吕耆卿凑到吕惠卿耳边,低声道,“那缫丝机说是十倍与旧机,照我看,至少二十倍。一担茧子才抬进去,转眼就光了。虽然章家遮着掩着不肯明说,但照小弟看来,这么一家厂子,一天下来,没有五百担,也有三百担。”

但一名工人必须要在热气蒸腾的厂房中站上五个时辰,不停的走动,手指还要不断的探到开水中.将线头挑起,这样劳作,使得缫丝工的手指很快就会烂掉,身体也会垮掉。这样的事,吕耆卿就不会跟吕惠卿明说了。

吕家前段时间通过前台的人,从雍秦商会那边拿到新式的缫丝和织造技术。

通过水力驱动机器来缫丝,但缫丝还要热水,这就需要锅炉,纺织需要动力,这就是水力。将这些集合起来,新发明的缫丝机,效率十倍于旧式的手工缫丝。纺织机的效率更是提升了十数倍。

韩冈之前用机织丝绢的技术,弄得人人心痒。之后以支持蒙学为条件,将这个技术对外进行转让。当然,转让技术的钱还是要给的。雍秦商会为了研究这项技术,付出的代价并不小,给钱也是应该的。

在签约的时候,雍秦商会给这笔钱起了一个很奇怪的名字——授权金——谁给了钱,就授权他可以用新技术生产丝绢,一县只有一家能够得到授权。

而且在签约的时候,双方都约定好,只有付出了授权金的商家,才能够使用缫丝机、织机来生产丝绢,若是有人敢于在没有得到授权的情况下擅自仿效,则合众共惩之。此为专利之权。

吕惠卿并不觉得,韩冈是为了将图纸卖得更多一点,才约定了专利权,否则绝不会同意一县只有一家能够得到授权。虽然不知道韩冈到底打了什么主意,但没人会喜欢竞争者,也没人会喜欢,自己费了心思、花了钱钞,方才得来的东西,被人轻轻松松抄了去。现在有了一个类似于行会的组织,解决这样的问题,就简单了许多。

吕惠卿族兄弟二十九人,中进士的只有其中七个,剩下的有的借助家中势力出外为官,也有的闲居乡里,更有的走南闯北。龙生九子,各个不同,这也是应有之理。吕耆卿一直以来都是在家中经营,吕惠卿见其无事,又觉得他有些才干,便将设厂的事托付给他。

不过吕惠卿知道自己的这个兄弟说话时总会喜欢夸大,说是三五百担,实际情况大概要打个对折,甚至更多。不过西北的那些商人说,十倍于民家手工,这的确是没有说谎。

“现在章家只愁蚕茧不足,急急的将附近的桑园也盘下来十几处,就等下一期收茧子了。”

在过去,经营丝绸,一般都是从民间采购个人织造的绸缎——也就是所谓男耕女织的理想生活的产品——最多是采购生丝,自己家里见织造工坊。现在就只要买蚕茧就够了。甚至可以不用买蚕茧,自己家里置办桑园,用桑叶跟蚕户定下用蚕茧还账的协议,最后只要再贴上一些小钱,就能把所有的利益都拿到手中,而风险,则全都留给蚕户。

吕惠卿问道,“家里织坊的情况如何?”

“厂房已经建好了,那边的人也过来看过了,说是没有问题。等机器运过来、组装好,还要把人找来训练,再试行一段时间,确定一切完好之后,就能敞开收茧了。”

吕惠卿状似满意的点点头,又道,“不过你们也要留心,不要什么都听人说,全都靠着雍秦商会那里,迟早会被坑了。自己也要学,学通了,就懂得如何改进。”

“小弟当然明白,”吕耆卿忽的又笑起来,“就这么把下金蛋的母鸡给卖了,就算卖出了黄金价,也还是亏本。真不知道韩相公是怎么想的。”

吕惠卿微微一笑,“既然韩冈他心有所求,又怎么能不让出部分好处来?!”
