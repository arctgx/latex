\section{第22章 瞒天过海暗遣兵(三)}

【第二更,求红票,收藏】

韩冈上京时曾听说过智缘的名字,也听说过他的手段。

三命僧愿成,医僧智缘,是大相国寺中名声远播的两位僧侣,总在豪门达官中行走,当时刚刚入京的韩冈无缘得见。

愿成善于算命,观人体貌便能断其三生休咎,说起人生过往能分毫不失,并言及前生后世历历如真,所以人称三命僧。而智缘的医术更为神奇,世传他只要只手诊脉便能知人贵贱休咎,甚至可以按父脉而知子祸福,所言精准如神。京中官宦贵人趋之若鹜,不是延请两人上门,便是亲自登门造访。

不过对于愿成和智缘两人的传奇,韩冈当时听了便哈哈笑过。三生轮回本是飘渺,诊父知子更是荒谬,他是半点不信。

但据说王安石却是真的相信智缘的本事,有传言他跟天子谈及智缘时,说道‘昔秦医和诊晋侯之脉,而知其良臣将死。夫良臣之命乃见于晋侯之脉,则诊父知子,又何足怪哉!’

但只是这个关于王安石的传言,韩冈却有些怀疑,一是天子与参政在宫中私下里的闲聊,怎么这么容易就传出来,在市井中被人口耳相传?第二,若是王安石真的相信智缘的本事,当智缘自告奋勇来秦州,当不会吝啬一件紫衣【注1】。

很可惜,当韩冈看到智缘的时候,他穿得袈裟还是赤色的。

韩冈与王厚并肩进厅时,王韶和高遵裕正陪着一为身穿赤色袈裟、五十上下的僧侣在说话。除了智缘,自不会由他人。

韩冈、王厚向王韶和高遵裕行礼如仪,直起身又转过去面向智缘。智缘大咧咧的坐着,王厚便欲作揖。可眼角看到韩冈直着腰纹丝不动,便也跟着停住了动作。

智缘没有官身,韩冈不会自降身份先向他行礼。尽管智缘是方外之人,不用俗家礼法。但既然要为朝廷拓边河湟,来西北边境追求名望功劳,就不要装出个高僧大德的模样来——当然,其中最关键的,是韩冈不喜佛教。若是面对饱学宿儒,即便没有个官身,韩冈也不介意谦恭一点。但对上吸民膏血、不事劳作的僧人,他可做不到恭敬谦卑。

对视了很短的时间,智缘见韩冈并不打算先见礼,脸色便是微变。他磨蹭了一下,终于还是起身向韩冈合十躬身,“小僧智缘,见过韩机宜、王机宜。”

智缘的声线浑厚圆润,如同禅唱。其声自丹田出,一张口,醇和的声音就在耳边回响。用这副声线向人解说经文,论人祸福,也难怪能挣下如许名头。

韩冈方才拱手回应,“大师善医之名,韩冈闻之久矣,如雷贯耳。素慕尊颜,却缘吝一面。今日得见,终遂平生所愿。”韩冈老于世故,这恭维式的套话说得极为顺畅,王厚跟着韩冈说了一通,各自哈哈笑了两声,重又坐下来说话。

虽然甫一见面,就有点不愉快,但韩冈并不否认智缘的魅力。这和尚相貌端正,阔面大耳,甚有佛象。身材虽不高大,但端端正正的坐着,如同一口青铜钟,身子毫无一丝偏倚,一看更是不脱高僧大德的形容。而且其说话间恂恂有儒者之风,儒释道三家的经典也是信手拈来,讨论起九经经义,虽无韩冈精深,但他旁引博证,把佛道两家的经文为儒学经籍做注解,却也丝毫不落下风。

等说过几句闲话,堪堪到了饭点,王韶使人布下宴席斋饭,将古渭寨中的大小官员如苗授、赵隆、杨英他们一齐唤了过来陪客,给足了智缘脸面。

坐入席中,智缘指着饭菜又说起了养生之道。凭着他医僧的名头,一番话说得王韶、高遵裕都心悦诚服。最后他甚至即席赋诗,与王韶这个进士相唱和,风头完全把韩冈盖了下去。

被智缘抢去风头,韩冈并无丝毫愠色。他本就希望智缘的本事越出色越好,这样才能为河湟开边之策去说服更多的蕃部。在大航海时代,基督教的传教士们往往精通天文地理医学建筑,每一个都是多面手——只有过人的才能,才能让传教的对象信服。先让自己成为信任的对象,然后才能把教义灌输出去。而智缘的出色,也就让韩冈看到了成功的希望。

智缘的才学的确过人,尤其是身兼三家之学,能让不少士人甘拜下风。不过这也难怪,如今的儒林风气,是儒释道三家互相印证,三教一家的说法,不论哪一派都有人提出过。儒释道三家,经过千年的并存发展,早就不复旧时的泾渭分明。许多时候,在民众中佛道与其说是教派,还不如说是民俗。

而从世风上,已经融入世俗的佛门道门都日益兴盛,信众无数。就算是崇儒排佛的士大夫,他们的家人也会到寺庙里烧上两柱香,比如韩冈的老师张载、还有程颐程颢,都是对浮屠二字深恶痛绝,但韩冈可是亲眼见过,张载的家眷、程颢的夫人去庙中烧香。

可能是酒喝多了的缘故,同时也是因为对智缘十分欣赏,高遵裕突然为智缘叫起屈来,“以大师之德才兼备,还得不到一件紫袍,实在是委屈……政事堂中诸公却是太吝啬了。”

智缘不以为意的笑道:“天子和王相公本是要与贫僧僧官之位,但贫僧心想未见寸功,非有长才,便以口舌得官,来秦州后却难以见人。故而对王相公推辞道,‘未见事功,遽蒙恩泽,恐致人言。等有功于朝廷,再与官亦不迟。’”

高遵裕愣了一下,立刻更加热情的赞扬起来,“视名利官位如粪土,大师果然德行高致!”

智缘口宣佛号,“钤辖过奖了。贫僧今次自请来河湟,也是不忍此地汉番之民再遭兵焚之苦。故而愿深入不毛,弘扬佛法,劝蕃人臣服于朝廷,从此共享太平之乐。”

“好个共享太平!大师以慈悲为怀,足以让朝中庸吏愧煞。”王韶轻轻击掌赞许,举杯敬向智缘。

智缘以茶代酒,与王韶对饮之后,放下茶杯,问道:“贫僧前日过秦州,承蒙郭太尉与燕太尉不弃,设宴款待。在宴上听说近日有一星罗结部屡有不顺,其族长别羌星罗结聚兵露骨山麓,意欲反叛。不知可有此事?”

王韶犹豫了一下,点了点头,“确实有此事。”

关于别羌星罗结的种种不顺,秦州那里早就通报过了。只是郭逵和燕达会将此事告知智缘,让王韶有些不快。

见王韶没有否认和隐瞒,智缘就席上向王韶:“贫僧来此,便是为了规劝蕃人归降朝廷。如今有星罗结部不顺于大宋,却是再巧不过。等明日贫僧便去露骨山下,劝说,”

王韶脸色丝毫没有半点变化,仿佛前几天批准突袭星罗结部计划的并不是他。“大师初来乍到,对蕃部内情尚未了解。还请大师在古渭少待几日,先熟悉了这里的地理人情,再去蕃部不迟。”

智缘又念了一声阿弥陀佛,道:“安坐古渭寨中,如何能熟悉蕃部内情。何况拖上一日,其不顺之心便盛上一日,若是拖延下去,说不定就有大战连连,死伤枕籍。”

“大师心慈,不忍见生民涂炭,韩冈深为敬佩。”韩冈向智缘拱了拱手,表示了自己的敬意。转过来对王韶道,“安抚,以下官之见,既然智缘大师一心想去,不如就准他去了。蕃人虔心礼佛,以智缘大师的身份,行走在蕃部之间,也不会有任何危险。”

王韶和高遵裕、还有所有知道即将实行的计划的官员,都惊讶的看着韩冈,这等于是把智缘往鬼门关里推。

王韶正要拒绝韩冈的提议,而韩冈却抢先一步道:“不过能否先请大师去纳芝临占部的吹莽城和青唐部的青唐城走一趟。托硕大捷和古渭大捷,得两家之力甚多,而战殁者亦多。大师若能去两城做一场法事,将之亡魂超度,其善莫大焉,亦能让两部更加恭顺于朝廷。”

智缘想了一想,点头道:“机宜有命,贫僧不敢推辞。”

“在下就为两部先谢过大师恩德。”韩冈起身向智缘行礼,“蕃人盼大师久矣。原本河湟一带最有名的僧人唤作结吴叱腊,在此地多有其弟子信众。其后因其不守佛门戒律,鼓动董裕在青渭残杀劫掠,在古渭一役跟着董裕一齐被斩杀,”韩冈指了指王舜臣,“这功劳还是他的。”

“阿弥陀佛。”智缘低头合十,对王舜臣道,“念佛而逆佛,口诚而心不诚,结吴叱腊死后必入地狱。斩杀此獠,王檀越阴德不少。”

王舜臣听得眉飞色舞起来,他杀人放火的事没少做,虽然为人豁达,平日里有时也担心死后会下地狱。但智缘说他杀人就救人,算是积攒阴德,让他放下一块心头大石,哪能让他不高兴,“多谢师傅,多些师傅。”

注1:宋代僧侣,如果译经之功,或是升任高位僧官,便能得赐一件紫色袈裟和法衣。名义上非高僧大德不与,但实际上,只要有亲王、宰执官或是地方监司官推荐,就能由中书门下颁下紫衣牒,可穿紫衣。

