\section{第17章 往来城府志不移(六)}

韩冈抵京,安顿下一家老小并没用上太多的时间。

他到宣德门去留名轮对,就得到通报说朝廷已经赐了府邸。本来韩冈还以为会在城南驿住上几日,才能等到开封府的消息,却没想到安排得这么周全。

因此抵达京城的当天,韩冈一家便入住了位于旧城左军第一厢信陵坊的宅院。这个速度,王旖也是感到惊讶不已。

“当初爹爹从江宁上京时,也是先让哥哥先进京找落脚的地方。”

“是治平四年的那一次?”韩冈问道。

“还能是哪一次?”王旖反问了一句,又道,“当是爹爹还让哥哥在司马十二丈家附近找宅子,说是做邻居好。”

“看来当年岳父跟司马君实倒是交情不错。”

“嗯,爹爹过去一直都在家里赞着司马十二丈的道德学问。”王旖的脸色有了几分黯淡,“只是因为变法,便反目成仇了……不仅司马十二丈,当年与爹爹交好的朋友大半都分道扬镳了。”

“那是因为岳父坚持到底的缘故。”韩冈深有感触,“大凡能成大龘事者,无不是性格坚毅的智勇之士,什么样的阻碍都会毫不在意的跨过去。”

“官人这是在自吹自擂吧。”王旖神色一换,带了些笑。

“算不上自吹自擂吧。难道为夫在这一事上会比岳父逊色?”

王旖轻哼了一声,却也没否认:“你跟爹爹就是一个脾气。”

话题给扯偏了,但很快还是回到了眼前的宅子上。

韩冈之前曾想过在京城买块地皮来修间宅院,弄一间属于自己的家宅。不过一番考虑之后,还是觉得算了。谁也说不准什么时候会被发遣出京,要是到外地任职,建起来的宅子还要留人来看守,浪费钱财,浪费人力。其实也就是这个原因,除了决定将籍贯移到京城的人,很少有官员会在开封置产。

韩冈倒不是缺人缺钱,不过没必要做得太显眼,炫耀自己的财富可不是聪明的做法。而且京城之中,寸土寸金,好地皮早就给人占去了,论起位置和规模,如今在东京城里面能买到的宅院,都远不如韩冈能从朝廷手中得到的官邸。

就像冯从义,他在京城置办了三处产业,但在城中的两处最大也只有三进而已。能安顿下韩冈全家的宅院,都是在东京城的城墙外京西第一厢的天泉坊。那都不属于开封府直辖,而是由祥符县管了——只有东京城中的区域是由开封府直接管理,城外就归于开封府下各县来管理。

冯从义的宅子,韩冈住进入没什么,但离得皇宫太远,上朝时能少说也要一个时辰,而且那间宅子周围人来人往很是麻烦,还是住在城中分配下来的官邸里面方便。

不过大部分官邸都经过了不少任住户,破旧得可以。之前担任同群牧使时的那间宅院,好不容易才整修了一遍,韩冈就去了河东,倒便宜了后面的人。这一次开封府给安排的宅院乍看起来还不错,但也有许多地方需要修补。

韩冈和王旖商量了,明天使人去开封府,去找专管官邸修缮的官员,让他们调些工匠来将破损的地方给修补起来。

安身的宅院还只是末节,重要的是觐见天子。

韩冈入觐奏对的日子定的是两天后,并没有像当初从广西回来时被一晾多日,但也没有迫不及待的当天就让他入宫面圣。

天子的态度很大程度上决定了韩冈在即将上任的岗位上能有多大空间施展手脚。从这一次的待遇上看,赵顼这个皇帝还是希望韩冈能在任上做出一定的成绩。

有了两天的空闲,韩冈便接受了王安礼和章敦等亲友的邀请,而上门求见的官员和士人,则是基本上都是推掉了。不过开封府中负责官宅修造的官员,却是轻车熟路的上门来拜会。

来府上的开封府官员,当年在韩冈担任提点府界诸县镇公事的时候有过几面之缘。韩冈看在故旧的情分上,见了他一面。不过这名官员也算知情识趣,没敢多耽搁韩冈的时间,说了几句奉承话,回忆了一下当初韩冈在开封府中任职的旧事,便起身告辞。

韩冈送了他到厅门,那名官员回身请韩冈留步,又问道:“不知端明在照壁上还有什么吩咐?”

“照壁?”韩冈闻言有些疑惑。

“还是端明当初在修群牧司宅子时传授的手艺,用碎瓷拼出了王都尉的《烟岚晴晓图》。如今京中的府宅里都开始用碎瓷片来拼接图案,不过还是碎瓷片的多。京畿各窑的碎瓷本是堆积如山,但才两年的功夫,就已经耗用得差不多了,开始有瓷窑专门烧制各色瓷片,用来在照壁、还有墙上拼图。”

这算不算又开辟了一个产业,韩冈有些想笑了,他现在还真有些怀念那块照壁上的星星。

“弃物亦能派上大用场,京中一说起此事,就有人以端明比之晋时的陶桓公。”

韩冈想了一想,问道:“陶士行?”

“正是。”

韩冈笑了,“那还真是不敢当。”

表字士行,谥号一个桓字的陶侃,是陶渊明的曾祖,东晋时的名臣。善于用兵,更会过日子。造船剩下的木屑也不丢,到了大雪天拿出来洒在路面上防滑。韩冈利用碎瓷的行径,的确跟他相类似。

只是如今民风奢侈,外面传韩冈似陶侃,好意没多少,想来更多的是笑他寒门出身的小家子气。但这样的讽刺对韩冈来说只是清风拂面,毕竟在种痘术面前,什么样的嘲讽都不可能成为主流。

韩冈看看面前一张讨好的笑脸,他当是不知道其中的意思。

随口对照壁提了几个要求,这个开封府的小官告辞走了。为了登韩家的门,他送上礼物不算便宜,是一对透明的玻璃花瓶。不过韩家的规矩照例是拒收,等人走后,只把单子呈给了韩冈。

透明的玻璃大约是元丰元年年底出现,当时韩冈还在同群牧使的任上,但那时的透明玻璃还很难制造,甚至得靠运气,也不可能成为透镜的原料。不过当时原理和配料已经总结得差不多了,加之将作监和军器监看到了曙光将临,同时加大了投入,所以到了元丰二年年底,拥有蓄热室、能够可以开始小规模成批次制造的炉窑终于出现了。

才半年的功夫,韩冈倒没想到这么短的时间,透明的玻璃制品就投入了市场之中。不过想想也是这个道理,毕竟玻璃镜片需要磨制后才能使用。磨镜匠的能力决定了对原始镜片的需求,剩下的产能要释放,当然就得用在各色器物上了。但比起瓷器,玻璃可以在其他方向上起到更大的作用,而不当放在装饰用的花瓶上。

在有心人的引导下,这个时代技术扩散的速度是极快的,或许再过两年,就能用平板玻璃代替窗纸。一尺见方的大玻璃一时造不出来,巴掌大小总不会有太多的技术难题,到时候在窗户上做个镶嵌功夫就可以了,只是价格上一时间肯定是个让普通人承受不起的数字。

作为有心人,韩冈很期待玻璃烧制的技术能有更大的进步,试管烧杯等仪器对化学有着极为重要的意义,镜子、灯盏,同样都是玻璃可以大显身手的位置。他之前已经跟冯从义商量过,准备在巩州设立的玻璃作坊,将不会在器皿上多费心神,而是努力开发新的应用,也就是水银镜和油灯。

家中之事稍定,就到了入对的日子。

韩冈入觐,被安排在早朝之后。天子不坐常朝,有实职差遣的官员往往也不需要在礼仪性质的朝会上浪费时间,但还没有正式就任的韩冈,却依然得一大早去文德殿。

排班轮次,韩冈自是排在前面。下面站着一堆胡子花白,没有职司,空领俸禄的老家伙。但论起位置的重要性,判太常寺等三个差遣加起来也比不了镇守边地的一任经略使,更不用说身为天子私人的翰林学士。

在王珪的引领下,向着空无一人的御榻礼拜之后,韩冈便往崇政殿去等待天子的召见。

赵顼并不打算过于冷遇韩冈。听话的臣子很多,但听话的臣子到王珪那个水平的却屈指可数。同样道理,有能力的臣子数量并不少,但水准能到韩冈这个等级的,也一样是凤毛麟角。

当结束了今天的议事,宰辅重臣们一个个鱼贯而出,他也不做休息,直接召韩冈入殿。

待韩冈再拜起身,赵顼便赐了座,道:“韩卿镇守河东,接连大捷。朕能在京中高枕无忧,韩卿之力也。”

“此非臣之功,乃是陛下圣德庇佑,将士用命。”

经过这几年的折腾,那种君臣相得的气氛是不存在了。君臣之间的寒暄就跟应付故事一般,这样的对话,让人有似曾相识的感觉。不过君臣相得的气氛本就不是常态,韩冈一直以来就没有想过要靠皇帝的恩宠得到什么。

