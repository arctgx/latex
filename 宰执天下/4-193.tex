\section{第31章 九重自是进退地(七)}

韩冈不说话了,低头喝酒吃菜。

既然已经安排了御史对付沈括,那么章惇来找自己当然就不可能是来抱怨的。

至于其中理由,韩冈基本上也能猜得出来。

在新党的全力反扑下,沈括不会有好结果,而吴充也不可能去硬保着犯了大错的三司使兼翰林学士——难道他敢对天子说,三司使和翰林学士应该向宰相负责,而不是对天子负责?——恐怕当御史弹章一上,为了向天子证明自己的清白,先行出手将沈括踢出去的就是吴充。

既然沈括接下来少不了会引罪出外,那么他留下来的两个位置,当然就分外惹人注目。

翰林学士倒也罢了,学士院里面的六个位置常年有缺,很少有满员的情况,又是天子私人,执掌内制,天子不点头,谁也打不了主意。但三司使是一国计相,也是往两府中去的几条主要道路之一,盯上这个位置着实不少。

章惇见到韩冈不说话,低着头喝酒吃菜,也就知道韩冈领会了自己的来意。他在韩冈面前习惯了直话直说,将筷子一放:“愚兄来此之意,想必玉昆现在也该明白了吧?”

韩冈沉默着,拿着酒杯在手中旋转,像是要在杯子上看出花来。这是牛角杯,有些寒酸的杯子上面镶了银边和几颗玛瑙之后,就成了摆在东京正店中也不逊色的高档货。

棉行楼是新开的酒楼,本是棉行的会所,顺丰行在里面占了不小的股份。修起来才一年多,但现在已经设法取得了酿酒权,成为东京城七十二家正店之一——这七十二其实也是虚数,实际上只有六十多家。

酒楼中的菜肴以关西的风味为主,连主打的酒水都是烧刀子。独特的风格,让一部分东京人对此嗤之以鼻,但也有些人却喜欢上了这一家的烈酒和蜜炙羊腿,成了常客。

不过韩冈对于关西风味的菜肴早就习惯了,再盯着看也看不出花样来。章惇耐心的等着,过了半天却看到韩冈摇起头来,“力所不能及,这一桩事,韩冈接不下来。”

“玉昆!”章惇不快,“你就打算看着新法一步步的被废除吗?!”

“不有李奉世吗?曾令绰也可以吧?”韩冈推脱着,推荐了另外两个人选,“这两位可要比小弟更合适。”

李承之和曾孝宽,是如今新党的中坚,都是有资格担任三司使,乃至翰林学士的人选。韩冈相信章惇不会没有备选方案,而这两位有机会更进一步的人选,会坐看着机会落到别人头上。

尤其是李承之,当初最早在王安石面前提到章惇的就是他。王安石曾经质疑过章惇的品行不佳,不打算起用,也是李承之说他才能过人,让章惇得到了一个见面的机会,与王安石一席深谈之后,从此便受到重用。

章、李二人交情匪浅,如果李承之出来接任三司使,对于章惇来说,当然是个好消息。不过章惇还是希望韩冈能够出来接下这个烫手的位置,“眼下这个时机,李奉世和曾令绰都不是合适的人选。三司使这个位置,只有玉昆你最为合适。”

章惇开诚布公的劝说着,他相信只要韩冈愿意,吕惠卿也会全力支持。以如今朝局的现状,如果不能再多些助力,将吴充和吕公著两人给顶住的话,新法大业很有可能毁于一旦。

只是韩冈不干,他与新党有一份香火情在,却不是铁杆的新党。吕惠卿、章惇都是靠着新法发家,而他韩冈却并非如此,功劳远远超过得到的官职,没有一次是靠着依附新党,以幸进而得官。

眼下若是有新党中人的全力推荐,他强取一个三司使,不是不可能,但之后又该怎么走?

天子的态度已经很明白了,是想要他在地方上多呆几年,不论韩冈本心如何,至少眼下他没有打算与赵顼顶着来的意思。

强要争取一个三司使,即便成功也没有什么意义。而翰林学士如果没有天子全力支持,单是文字水平这一关,韩冈就过不去,反而会丢人现眼。如果这两个职位不能让韩冈在晋身宰执之位的道路上向前一大步,那他辛辛苦苦的去争,又有什么意义。

“子厚兄太看得起小弟了。不论李奉世还是曾令绰,他们都久在朝堂,从中书到监司,担任过多少职位,内外之事远比小弟要熟悉,真要放在三司使的位置上,小弟要想与那两位媲美,可是有些吃力。”所以他依然坚持推辞,“吴相公虽然反对新法,但现在毕竟还没动手,子厚兄何须心忧。更何况,吕吉甫手上当有对策,吴相公到底能不能压得住他,还得两说。”

章惇摇着头。李承之就算去做了三司使,也只会是个中规中矩的三司使,寻常的时候,他虽不会有开创之功,但也不会将国家财计弄得一团乱。

只是眼下旧党做着宰相和枢密使,三司使的位置上如果不能放着一个强势的人物,最后只会在中书和枢密院的联手压制下,成了仰人鼻息的部门。新党在政事堂和枢密院的版图已经渐次沦丧,再失去了对财权的控制,有着正常思维能力的官员都知道最后的结果会如何。

但章惇拿韩冈没办法。别人都是想着高官厚禄,一看到能有晋身的机会,根本就不会放过,偏偏韩冈推三阻四,根本就不把三司使这个职位放在心上——好歹也是号称计相,不是地方上的都转运使能比,有过三司使的经历,就代表着有了参与掌控国家全局经验,韩冈眼下缺的可就是资历。

韩冈一切都很清楚,但他还是没有兴趣。

一直以来,机会是靠自己挣来的,而不是别人施舍的。落到眼前的大饼,里面到底有没有钩子这件事当真不好说。章惇当不至于害自己,但吕惠卿那边就难说了。

韩冈又低头看着桌子上的菜碟。自己与吕吉甫的关系,从来就没好过。所以韩冈都不问吕惠卿到底是怎么想的,章惇也聪明的不提吕惠卿的事——尽管如果韩冈打算接手三司的职位,没有吕惠卿根本不可能成功。

“以小弟看来,三司使一职还是以李奉世为佳。他曾做了检正中书五房公事,又去河北、陕西、两浙担任过察访使,而且免役法的首倡者便是他,”韩冈说到这里就抿了抿嘴,嘴角流露出一丝讽刺的笑意。

如果李承之的任命当真能从天子面前通过,朝堂上的风向其实能转过来一点——政治意味很深。不仅是对沈括指责役法的言论的反制,同时也能通过李承之,从政事堂那里,将属于三司的财权抢回来。

原本在王安石当政的时候,中书的权威横跨军政财三方面,这也是为了易于推行新法。不过现在是吴充担任宰相,集中到他手上的权力,也就变成了用来狙击新法。

针对三司的财权被中书所侵占的现状,由做过检正中书五房公事、熟悉政事堂内部事务的李承之做三司使,当然是个上佳的选择,因为他足够了解对手,这一点是韩冈也比不上的——他在中枢的经验太少了,一个军器监说明不了什么。

章惇没有办法了,韩冈说的这些,他难道没有考虑过?正是因为他权衡了利弊,所以才来找韩冈。

但既然韩冈不愿意,也只能退上一步。无奈的点着头,“李奉世的确适合做三司使,今天回去,就知会一下吕吉甫,迟了恐会生变。”

韩冈与李承之往来次数不多,但也算有份交情在。当初韩冈为开封府界提点的时候,李承之是纠察在京刑狱,明里暗里帮着韩冈省了不少麻烦。韩冈推荐他而不是曾孝宽,也有还人情的心意在。

看着章惇无奈的样子,韩冈笑道:“其实就算吴相公想要在免役法上做文章,由此产生的后患,他可解决不了。要不然,当初也不会由家岳来做宰相了,富彦国、文宽夫哪个不是先被天子询问的?”

“玉昆,你可知道一旦开始废止新法中的任何一条,接下来只会废除得更多。一旦废除之后,再想恢复可就不是那么容易了。摔坏的瓷瓶,还能盛水吗?”

“子厚兄,你是福建人,自幼惯见了大海,可曾见到只涨不落的潮水?”韩冈眼神深敛,“小弟在交州的那段时间,常常看见潮涨潮落,落潮时海水退得越远,涨起来的水位可就会越高!”

“朝令夕改,百姓要吃多少苦。”

韩冈叹了一口气:“可惜这一点,哪里能说服得了别人?”

“别人?”章惇声音变得有些生硬,“哪个别人?”

“吴相公,吕枢密,还有洛阳的那几位,”韩冈笑了一笑,“实在太多了。”

章惇盯着看了韩冈半天,最后放弃了,也不指望韩冈会说出悖逆不道的话来。

韩冈也不再提三司之事,而是端起酒杯,跟章惇说起了最近在京中声名鹊起的王韶家的十三郎来。

