\section{第20章 土中骨石千载迷(14)}

压力就摆在赵顼的案头上。

赵顼将一份份奏章摊开在御案上,面色凝重的看着。

一篇篇千余字、几千字的奏章,内容如出一辙,看了其中一篇,其余就可以当成废纸扔掉。但上书臣子的签名,却一个比一个份量更重。

当年为推行新法,赵顼将一干老臣请出了东京开封府,将他们安置到其余三座京城中。这是在免除朝中反对变法的杂音之余,对老臣们尽可能的优待。

只是这些老家伙们可不是心甘情愿的退出朝堂,每一次朝局动荡,他们都不会放过这一攻击新法的机会。

一个月不到的时间,西京河南府、南京应天府和北京大名府的一干老臣,都上本请求发掘殷墟,并专设有司,负责全盘事务。用发掘出来的殷商的金石甲骨,来印证儒门诸经。

在奏章的最后,都还不忘添上一句内容相似的话:上古遗物再现,此为陛下福德所佑,是儒门盛事,更是祥瑞之兆。

这二十多份奏章还是离得近的大臣们所上,离得远的一干旧党臣子们,要么还没有收到消息,要么就是奏章还在路上,赵顼不觉得他们会息事宁人。好不容易等到一个能对新学群起而攻之的机会,怎么可能舍得放过?肯定会跟嗅到了伤口上脓血腥味的苍蝇一样,嗡嗡嗡的就聚拢了过来。

赵顼眯着眼睛,眼皮的缝隙中闪着冰冷的眼神。

熙宁七年、八年的时候,辽人趁火打劫,硬是从河东划走了七百里土地,那时候插手到其中的一干元老重臣,他们的撺掇之言,赵顼也记得清清楚楚。这一回,他们究竟又是有着什么样的打算,赵顼不可能不明白。

摆在面前的奏章,加上过往旧事带来的回忆,赵顼很难对那批老臣有太多的好感。

那些老臣在台上的时候,国家是什么样,自己将他们赶下台后,国家又是什么样?

灭了西夏,收复了西域,南海的小国在交趾灭亡后,只要再谋划几年,就可以向北

赵顼并不觉得自己除了照顾老臣们的体面以外,有必要在军国重事上接受他们的指手画脚,他已经听够了,也受够了。

瞥着桌面上的一份份奏章,赵顼很想直接丢到崇政殿后的架阁库中去。

可赵顼更清楚,若是就此将殷墟拒之门外,安阳地下的上古遗物便无法避免的要失落民间,万一韩冈或是别的学派,给出了一个让人无法辩驳的证据,‘一道德、同风俗’的初衷,就没办法依靠新学来实现——并不是所有的问题,都能够依靠权势来解决。

来自相州韩家的奏章,排开上面的虚浮辞藻,则满是抱怨的文字。对韩冈揭开殷墟所惹起的动荡,不仅是韩忠彦的奏章,还有相州知州的奏本,也是在抱怨连天。赵顼在相州的耳目也有着同样的回报,而且将情况说得更加危急,为了让赵顼都为之惊讶的收购价,竟然是户户掘土,家家挖坑,一时间成了风潮。

如此一来,就算朝廷将此事搁置,殷商旧物照样会被不断的发掘出来,只是由明转暗而已,并散布到各家学派手中。解释权落入,就可以乘机用以攻击新学,乃至新法。这样的局面又该如何应对?难道要焚书坑儒不成?!

不过这对赵顼来说,依然仅仅是桩小事。只要他一意不加理会,谁能奈他何?所谓拒谏,又有什么大不了的?

可是将眼睛蒙起来,并不代表眼前的敌人或是阻碍就能消失无踪,反而是把整件事的控制权交托出去,在赵顼眼中,却是让他无法容忍的。身为天子,难道只为了赌口气,就扭过头去,而放弃对天下士林的掌控?这份权力,赵顼可是绝对不愿意放开手的。

自然,造成眼下这一让人进退两难的局面的韩冈,这个有能力却从不让人省心的臣子,赵顼一想起来,要皱眉头。

要是韩冈有王珪的性子,或是王珪有韩冈的能力,那该有多好?

在殷墟之事上,王珪的态度一直都是暧昧不明,甚至是偏向于打压新学的一方。看起来除非自己明确态度,否则他的宰相绝不会立场分明的站出来。

许多时候,有王珪这样的宰相很让人顺心,但有时候,赵顼也觉得,这样的臣子,终究是挑不起大梁的。在大齤事上,比不上王安石,甚至吕惠卿。

让宋用臣将这些奏章扫到一边,赵顼低头看着桌案上勾勒着金色花纹的深色漆面,让他不省心的还不只这一桩。

私下里在国号上做手脚的太常礼院,让赵顼也是一肚子火。‘戎狄是膺,荆舒是惩’,不是韩缜提醒,每天忙于国事的赵顼,都不会注意太常礼院在改国转封的事上玩的小动作。

尽管这个小动作,王安石也不会在意。

赵顼让人翻出了当年封赠其为舒国公时王安石所进的谢上表。表章中对这个国号就说了:‘久陶圣化,非复鲁僖之所惩’——‘戎狄是膺,荆舒是惩’正是出自《诗经·鲁颂》,赞的是鲁僖公的武功——可见王安石是浑不在意的。

但这并不代表可以再封王安石一个荆国公——未免欺人太甚,也完全失了赵顼褒奖这位谋国老臣的初衷。

幸好有韩冈为王安石鸣不平。至少在学派之争以外,韩冈还顾念着翁婿间的情分。并不是经常可以看到的,为了打到某人,就先从人品开始攻击。

韩冈对新学的攻击好歹还是明着来的,而太常礼院却是鬼鬼祟祟用阴招,仿佛能给王安石一个荆国公的封爵,就能占多大便宜似的,可以躲在阴暗处暗暗窃喜。

对比起来,至少韩冈在品行上还能让人看得顺眼,是君子所为,而太常礼院的一干人等,可就是彻头彻尾的小人了。

一时还是无法打定主意,中午的时候,赵顼带着左右为难的心情回到福宁殿。

他每日清早便上崇政殿来,一对儿女的晨昏定省,都要放到中午或是午后。可在他的寝宫中,赵顼只看到了女儿,却没有看到儿子。

“六哥儿怎么了?”赵顼变了颜色,急着问道。

“均国公早上有些发热,请了钱乙过来,说是并无大碍,喝了药,睡下去发汗就能好了。”

赵顼松了口气,但一颗心依然高高提着。

赵顼现在有一对儿女,也只有这么一对儿女。论起身子骨,都是不算太好的样子。

尤其是作为皇嗣的赵佣,夏天生了场病,入秋后也没敢让他累着,一直在养着病。病恹恹的样子,让赵顼看得心忧不已。且不提能不能保得住,就是日后这样老是生病,万一生变,怎么争得过他的叔叔。

面前的一张桌上,御厨整治出来的菜肴色香味俱佳,又有活泼可爱的女儿在旁,但赵顼吃得食不甘味。被人拿捏在手中的把柄,的确让赵顼不痛快,但有些事也必须稍稍退让一点。

这一日午后,王珪又被招入崇政殿。

很难得有这样的情况,王珪知道这是天子终于有了决断,低眉顺眼的等着皇帝的发话。

“殷墟之事,就让王安石去主持好了,他为正,韩忠彦为副——毕竟是在相州安阳,得有个韩家人看着,不能惊扰了韩琦。”赵顼漫不经意的说着,“王卿你就看看给个什么名目比较好,三馆和国子监中,有哪些人调动起来比较方便,明天报上来。”

王珪愣在了那里,殷墟的事,让王安石去主持?!天子怎么会有这样的想法?

但赵顼没理会发愣的王珪,负责草诏的中书舍人就在旁边的,他只是通知宰相而已。

赵顼要吩咐的,并不止这一桩,尽可能平静的语气道:“均国公年纪也不小了,差不多也该封郡王了。王卿你去跟太常礼院商议一下,明天一并给朕一个回复。”

这一下,不仅是王珪,就是中书舍人也一起发了楞。

赵顼声音微沉:“王卿,可是有什么不妥?”

王珪一个激灵,登时回过神来。

大宋的皇子,并不是一生下来就能封王,而是一步步的晋升上来。从国公晋郡王,由郡王晋封王爵,而王由于国别大小不同,又分个三六九等,一级级的慢慢升。在这过程中,还要封个节度使,侍中之类的官职。

当今唯一的皇嗣眼下便是均国公,向上升一级,自然就是郡王。

但这位六皇子未免太小了一点,才五岁!就算是如今实质上的嗣君,但要封王还是嫌太早。在王珪看来,至少要等到七岁才合适。

仁宗当年五岁方封庆国公,七岁才封寿春郡王——现如今的皇嗣形势,跟真宗晚年时差不多,都是只有一个儿子,所以当年的故事旧例,用在现在也是合适的。王珪特地让人找了六七十年前的旧档出来查看过,便是为了能够更加有效应对。

可是天子既然这么说了,王珪也不敢争辩。提前个一两年就提前好了,没必要在皇嗣的事情上与天子顶着来。嗣统之事,即便再不起眼也不得了的大齤事,逆了天子的心意,那么想坐稳政事堂第一把交椅,只能是做梦了。

“臣遵旨。”王珪低头躬身,不带一点犹豫。

这就是为什么王安石在东府之中两进两出,如今黯然退隐金陵,而他王珪从熙宁四年进了政事堂后,就一直没有离开的缘故。

赵顼看着王珪并不反对,点了点头,“差不多是时候了,资善堂也该重开了。”

