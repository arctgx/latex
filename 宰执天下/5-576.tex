\section{第45章 从容行酒御万众(三)}

尤素普·喀什葛里正用这两天才从马秦人的斥候手中缴获的千里镜望着天空。

对面的敌人自称是宋人,来自于东方的马秦,也就是桃花石。几百年来,再没有听说过马秦人的大军能够再次统治回鹘,既然他们现在已经到了喀什葛尔的边境上,那么多年的死敌,也肯定已经不复存在了。

这是在埃辛失败之前,就予以确认了的消息。加上这些天被赶出家园的教众,以及俘获的斥候口中,都确认了这个消息的正确姓。

逃回去的埃辛在博格达汗面前痛哭流涕,说马秦人太多了。他只带了六千人出去,却撞上了整整五万马秦人和回鹘人,杀了他们大半之后,自己也是伤亡惨重,不得不逃回来。

埃辛是个无能且满口谎言的小人,敌军的人数,还有他杀敌的数量,只是他掩盖自己无能的谎言。但逃回来的有不少人,马秦人是怎么击败的他,他也不敢胡言乱语。

威力强大的十字弓,一人高的大刀,遇敌后下马步战,组成起来了马秦大军的模样。出现在喀什葛里面前的敌军也正是如此,以中央的步兵为核心。

能出兵远征回鹘,看来马秦人不再担心党项和秦人了。

喀什葛里懂得很多,因为他有一个做学者的堂兄弟,写了很多书,很受大汗,甚至巴格达的哈里发的看重。但家族中,在喀什葛尔【疏勒】,在八剌沙衮【黑汗首都】,更有威望的还是他。因为他手上有兵,更是博格达汗最为信重的将领。

秦是契丹,马秦和桃花石是大宋,这是突厥人对东方两大国的称呼【注1】。而他的主君自称是东方与秦之主,自号桃花石汗——哈桑·桃花石·博格达汗,将由真主赐福的阳光洒遍东方的每一个角落,那是历代可汗的梦想。

于阗已经征服了,接下来就是回鹘、党项,然后便是最终的目标契丹和桃花石——也就是秦与马秦。隔海相望的是查帕尔喀【即曰本】,那将留给后人去征服。不过秦人和马秦人肯定不喜欢这样的理想。

映在千里镜中,圆滚滚的气囊上面的几张狰狞鬼脸,让喀什葛里觉得好笑。

也许马秦人觉得可以用来吓唬敌人。可那样的东西,只能吓唬一下无知的异教徒。

真主赐福之地的子民,被先知教诲过的头脑,怎么会被吓唬异教徒的玩具吓到。

而且那些回鹘商人,早在两年前就将马秦人的飞船带到了八剌沙衮,献给了博格达汗。

据那回鹘商人所说。马秦人的皇帝,为了对抗强大的秦人,让他朝中最聪明的大臣发明了飞船,买通了秦国中的歼臣,献给了秦人的皇帝。秦人的皇帝不知其中有诈,在乘坐飞船的时候,从天上掉了下来摔死了。

而那位聪明的大臣,还发明了根治天花的办法。就在去年,天花的痘苗被一名大食商人通过贿赂负责种痘的医生,偷偷带到了八剌沙衮,为可汗和他的王子们种下牛痘。

回鹘商人每年都有假借可汗的名义去东方朝贡。愚蠢的汉人皇帝,将那些贪婪的商人视为正式的使节,将编造的国书当成真品,赏赐大量的丝绸、瓷器。让他们的骆驼都不堪重负。没人在乎那些商人到底做了什么,只要回来交税就行。而他们这两年带回来的新东西,更是让人惊喜。

虽然是异教徒的东西,但先知也说过,要去东方求取知识。

远处的战场,传来呜呜的号角声。

喀什葛里调转视角望过去。

方才的攻击失败了。那是马秦人在地上挖了陷阱。雪并不厚,可足以掩盖地面上的坑洞。但这一回攻击又失败了,似乎是被投石车给砸晕了脑袋,又是退了下来。

不过那些由乌古斯蛮子,以及康里和基马克人所组成的军队,伤亡多少都不放在喀什葛里的心中。只要古拉姆和伊克塔两军无事就好。

有着沉重坚实的铠甲,就能挡得住宋人的弓弩,有着同样沉重结实的铁锤,便能敲开宋人身上的盔甲。

两个千人队的古拉姆近卫,还有近九千名伊克塔骑兵,足以将宋人和回鹘人一起送去为异教徒准备的火狱。

喀什葛里放下千里镜,接连派出了十几个传令兵,让他们分赴各处。他现在越来越喜欢这件战利品,能让他更好的掌控战局。等下次有机会,可以再试一试飞船,不过,飞船气囊上,他只会让新月照耀四方。

不过眼下,要开始行动了。

……………………

尚未正面交战,便解决了黑汗人第一波的攻击。

可那只是试探。只从装束上就能看得出来。连甲胄都没多少的轻骑兵,在战场上,只有搔扰的作用,面对坚实的军阵,只会崩掉他们的牙齿。

王舜臣正紧盯着对面主帅的反应。

黑汗的轻骑兵正在试图攻击东西两门的营垒。他们在试探过后,发现通向营垒的道路上并没有陷阱。不过很快就被霹雳砲投出石子给驱散了,百多人受伤。只有正面一条路,霹雳砲瞄准之后,都不用怎么调整。

但正面名为古拉姆的具装甲骑并没有继续前进。在军阵前方一百四五十步的地方停了下来。

王舜臣重重的怒哼了一声。

那是重骑兵最佳的冲刺距离。远了,马力下降,速度减慢。近了,又来不及将速度提到最高,阵型也不及调整,还会受到弓箭的搔扰。这只会是多年上阵才有的经验。果然是难得的精锐。

从装束上看,对面的三万多人中,有三分之一能是精锐。

按王舜臣打听来的说法,一名古拉姆,一名伊克塔,都是重骑兵。

全数制式装备,战甲整齐划一的是黑汗可汗的古拉姆亲卫队,而剩下的铁甲骑兵,甲胄的式样各不相同。显然是从各个部族征募来的伊克塔骑兵。也许战力上有所参差,战马也是有的有甲,有的无甲,但都是重骑兵。

剩下的三分之二,都是撑场面的,跟自己麾下的回鹘军差不多,或许还不如。

但那一万重骑兵,就已经很不得了了。在过去,就是大宋一时间都拿不出来几千上万的重骑兵——哪里有那么多好马?能载得动几百斤的铁甲和人,那样的骏马只有西域才有。只有灭了西夏,打开了西域通道,军中的好马才渐渐的多起来。王舜臣麾下的几千带甲骑兵,现在也才会都是身高腿长的西域良驹。

也就是说,只要击败了这一万重骑兵,剩下的事就好解决了。

兵是精兵,就要看对方主帅的能力了。

古拉姆近卫停下来的位置,犹在神臂弓的射程之中。以他们身上的装备,神臂弓在那样的距离上,不会有任何威胁。而跟上来的伊克塔重骑兵,停在两翼的位置上,同样没有上来。

亲兵送来飞船上的斥候新观察到的情报,看过之后,王舜臣再一次将纸条紧紧捏在手中。

黑汗军已经在其后方开始修筑营地。拖延下去的结果,就是让他们顺利的将前进营地筑起。

一旦在末蛮城近处有了黑汗军的营地,官军就难以绕过去攻击后方的大营,而黑汗军要攻城时就能从最近处出击,免去了数里的跋涉。

那一座新营地远在一里半外,正是最佳的位置。这就是逼着王舜臣前去阻止,让出战的大军,远离背后的城寨守护。

‘为将者致人而不致于人’。

王舜臣轻声念道,将纸条交给亲兵收好,随手从鞍袋中抓了一把黄豆塞给他的老马。

胡人当是不通《孙子兵法》,但没有蠢到仗着兵力上的优势大举压上,而是选择了以势压人,足见也是一员良将。

不过天寒地冻,要想修好一个营地,至少两个时辰,自家还不用那么着急。

从汉军两翼的骑兵中,出来了两队一手大斧一手神臂弓的士兵。都是下马后在雪地上小心的走着,以防脚下的坑洞。如果没有在两天前又下上一场雪,他们动作就还能快上一点。但足以让他们赶上那些还没有来得及逃回去、正在地上挣扎的黑汗人。

先是神臂弓抵近射击,然后追上去一斧头砍下头颅。不仅是正面的汉军大营,东西两侧的营垒,都开了小门,一群敢战的士兵出来收割首级。

一枚枚人头悬在枪尖,树在两军军阵正中间。不仅是示威,同时也通知了黑汗人,军阵的正前方并没有设下陷阱。

黑汗人的精锐,如王舜臣所愿,果然出动了。不过不是最精锐的古拉姆近卫,而是伊克塔重骑兵。

一个千人队,装备着各式各样的铁甲,但无不是精悍善战的伊克塔骑兵,缓缓的离开了之前位置。在千夫长和百夫长的号令中,一名名战士拉下了头盔面罩,开始了冲锋。

铁蹄踢散了前方的积雪。转眼就拉成数道横排的伊克塔骑兵,长长的铁枪遥遥指向前方。雪片在他们身前身后飞散,就像泛着白花的海浪,向着汉军军阵扑来。

大地颤动起来,就是之前那群人数远远胜出的轻骑兵的冲刺,也没有这样惊天撼地的声势。

王舜臣眺望着一道道‘海浪’的后方,胯下的河西老马则嚼着黄豆,一人一马,都对猛攻而来敌军视若无睹。

如果是那群古拉姆近卫,若非有了破甲弩和专用的破甲箭,否则真的是有些麻烦。不过现在的这队战马无甲的重骑兵,神臂弓就足以应对了。

注1:桃花石为音译,是中世纪突厥和阿拉伯人对中国的称谓。有说法是从北魏拓跋氏而得名,另一个说法是唐时征服西域的汉人自称是唐家子。另外,在十一世纪问世的《突厥人大辞典》中,契丹和宋,又被称为秦和马秦。

注2:黑汗国也就是喀喇汗国的大汗,都是自称是东方与秦之主。至于宋史中,声称黑汗国曾经在元丰四年遣使中国,以‘于阗国喽啰有福力量和文法黑汗王’的名义,称神宗是‘东方曰出处大世界田地主汉家阿舅大官家’,以笔者猜测,当是大食商人伪造的国书。称中国皇帝为阿舅大官家,是因为于阗国王尉迟胜曾经娶了李唐宗女,所以他的后代可以自比外甥,用阿舅来称呼中国皇帝。喀喇汗的可汗是于阗国的征服者,又怎么可能自贬身份,借取于阗国王名义来崇礼宋神宗?
