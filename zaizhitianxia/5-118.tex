\section{第14章 霜蹄追风尝随骠(二)}

得到了李宪的副署,韩冈请求出兵旧丰州的奏章便发了出去。

但韩冈并没有坐下来等待东京城那边的回复,他在就任河东路经略使的时候,便从天子赵顼那里得到了便宜行事的权力,军事上的行动,并不需要事先禀报天子和朝廷,事后得到追认也没问题。

与韩冈敲定了出兵的细节,李宪当天就回了晋宁军。韩冈本人,就任河东之后,太原、雁门、晋宁到处跑,马上又要去麟州。而在韩冈的麾下,李宪基本上也是一个劳碌命,远不能跟高歌猛进、意气风发、心想事成,反正只要是吉利话都能用上的王中正相比。也就比运气差到极致的高遵裕和苗授强些。

……其实也足够了。没有经历什么惨败,手上的军队也没多少损失,还有两三个说得过去的大捷,可以说是不负天子所托。如此,李宪也不敢奢求太多了。下面若能抢在辽人之前,拿下旧丰州,挣点功劳回去,至少在王中正面前不至于低他一头。

李宪回晋宁军整顿兵马,他手上的兵力基本还是以步兵为主。前段时间出兵银夏,退回来后又镇守在晋宁军,之后还在葭芦川跟阻卜人打了一场。虽然没有经历大战,可得经过一段时间的休整之后才能出征。

李宪昨日离开太原的时候对韩冈坦言,最好是得到了葭芦川大捷的功赏后再让他手上的河东军主力上阵,这样才保证有足够的士气。

韩冈并没有强令李宪即刻出兵,只是苦笑而已。然后答应朝廷的诏令一到,便将功赏给发下去——之前韩冈已经从太原的府库中准备好了足数的钱粮,并不要等着朝廷的划拨——军中的问题和弊端,韩冈了解得很深,知道逼李宪也没用。

当年太宗皇帝领军攻下太原、灭亡北汉,对士卒不加赏赐便从太原出太行直扑燕京,想打辽人一个措手不及,可惜却是攻城不果,惨败而归。高粱河之惨败,一方面是辽军的援军来得太快,出乎大宋君臣的意料之外,另一方面便是有军心士气因赏赐不及时而低落的因素在。

宋军的传统传承自五代,开拔、上阵,全都得发钱,立了功之后,更是得及时发给赏赐。韩冈在军中颇有声望,但名气再大,也没有孔方兄的面子大。

韩冈一开始时对此很不习惯,也很想有所改变。只是随着对这支军队了解,发现必须推倒重来才有解决的可能。再好的苗子到了污水里,都会给染得漆黑——根子都是黑的。

这不是建个武学或是军校什么就能处理的问题,技术战术可以传授,可一支军队的根源和秉性是几乎不可能更改的。也许圣人或伟人能做到,但韩冈可没那么自大。

幸好周围的军队也都是这幅德性,甚至更差,就是一群强盗。所以不必渴求跟后世比肩,只要比周围的军队强就行了。

大航海时代的各国军队,其实不也是一群强盗?但并不影响他们去抢.劫全世界。

除了银绢之类的赏赐,需要运送的军粮就不需要太多的担心。之前河东的粮食为了保证供给前线的将士,有许多就囤积在黄河西岸的晋宁军以备转运,向北运去麟州也算是比较方便。

不过也不可能吃多久,太原这边也得及时给他们补充。因为不仅仅是李宪现在手上的步卒,另外还有骑兵。

借调给鄜延路巡视道路的五千骑兵,韩冈已经派人去调回来。可战马的情况很让人担忧,想要他们上阵的话,多半会有些麻烦。

在太原还有两万马步禁军,这是河东路仅存的预备队,无论是萧十三在雁门关外挑衅,还是阻卜人骚扰葭芦川沿线,这部分兵力都没有动过。不过韩冈之前已点起了其中的五千人,包括三个指挥的骑兵,准备带他们一同出发。

预定中的军队各有各的问题,先期攻城掠地的主力暂时还是折家手上的兵力。等到半个月之后,全军才能开始正式进驻麟州,向屈野川的上游攻过去。那时候,对手就该是回过神来的契丹人了。

据折家最新的情报,辽国西京道的兵马正在镇压黑山下河间地残存的反抗——之前通过河间地南下的辽军,仅仅是将已经逃散了大部的守军击溃,并没有完全压制——一时间还无法分心旧丰州。不过以辽国的实力,将河套上下清洗一遍,并不算是什么难事,最多也就半个月的时间。等西京道的兵撤回来之后,肯定是要将注意力转到南方来了。

到时候,他们会有什么样的反应,倒也是不难猜。

韩冈在灯下忙碌着,等今明两天,将太原府这里安顿好,便要动身启程,去往麟州坐镇。

他只希望东京城那边早点传来消息,给个确定的恢复,不要拖得太久。

……………………

盐州收回来了,银夏之地肯定是稳当当的拿在手中。而西夏也亡国了,该死的全都死了。

但收到这些消息的崇政殿中的大宋君臣,却没有一个能为此笑出声来。

给契丹人捡了便宜。自己用了多年的心血才削弱了西夏的国力,正与西贼进行最后的决战,契丹人却趁此机会,不费吹灰之力,就把自家一直想要却没能打下来的土地一口吞了去。

赵顼听到这个消息后,就一直铁青着脸,连着多日都没有一个笑容。

不过盐州城的陷落,和一天后局势的转变,反过来到可以说是徐禧的坚持是正确的。至少吕惠卿这样在御前为自己辩护着:“若是盐州城能多守一天……只要一天,西贼将不战自溃。但城中守将守城不力,而城外援军又不能及时而至,故而差了这么一天。徐禧以下,数万将士因此殁于王事,闻者无不扼腕叹息。”

就是赵顼也能听得出来,这是把责任往种谔、高遵裕和逃出来的曲珍身上推。

在盐州城破之后,西贼随即因为契丹入侵而发生内乱。秉常和梁氏兄妹全都死于内乱之中,而种谔趁势攻挥军攻打。一场大乱之后,斩首数以万计。战乱二事接踵而至,嵬名一族在这之后已经什么都没有剩下了。仁多零丁、叶孛麻还有李清,非嵬名氏的党项部族和汉军,则全数投降,之后还帮着清除嵬名家逃散的兵力。

这一战中,种谔就算是捡了便宜,也是有功的。三万多斩首,即便可能源自于杀降,但终究是立国以来,数目最多的一次战果。而面对西贼的阻截,高遵裕和环庆军一触即溃,种谔好歹还冲了过去,差了一天,的确让人悔恨,心中堵得慌,但就此降罪有功的种谔,却是有些说不过去。

而且如果依照韩冈、种谔等人退守银州、夏州的方略,由于距离一下远了三百多里,需要调集更多的粮草,西夏出兵的速度便快不起来。一拖时间,契丹人渔翁得利的机会也就小了一些。即便他们还能捡个便宜,可大宋官军这边,损失决不会有盐州城那么大。

因为坚持驻守盐州,派出去的京营禁军伤亡惨重,这些天来,城中日日有人出殡。皇城司每日的汇报中,提到了许多。

病容未消的清瘦面孔上是难以舒展的愁眉,赵顼叹了一声:“现如今,西夏亡国。韩冈请求出兵收复旧丰州,及屈野川一路,以防给辽人占据。不知诸卿有何意见?”

“诚如韩冈所言。”吕惠卿抢先说道,“屈野川乃云中锁钥,一旦占据屈野川,河东西界便可延伸至大漠。只要防住北面来敌,河东路便可以安居无忧。”

王珪并不开口,如今在政事堂中,存在感越发的薄弱起来,完全的融入到三旨相公这个角色中,将自我完全放弃。揣摩着赵顼的心思,全力支持。并不像吕惠卿那般高调。

他一力主张的西北一战,虽不能说劳而无功,但收获远远不及预期,而付出和损失的也远远超出预计,尤其是与辽国的收获和付出做过对比之后,更是让人觉得憋屈。之后,吕惠卿主张的盐州之战也损失惨重,若没有辽人相助,也是惨败的结果。

两场惨败之后,政事堂中肯定要换人了。但什么时候换,却还没有确定,外面的猜测都是西北局势抵定,那时候政事堂中面孔就要大变样了。

“万一惹怒了辽人……”元绛有些迟疑。

“那辽人就不担心惹怒朕?!”赵顼猛然间的爆发,让元绛不敢再多话。

在元绛看来,黄河西面、大漠东面的那一片地,有没有其实也无所谓。又不是多富庶、多肥沃,仅仅是黄河西边的荒地,让河东的西侧得到一个缓冲,挽回一点颜面而已。相对于兴灵、银夏的争夺,战略意义差得很远。

其实河东本来就有黄河做屏障,黄河之西的土地,荒芜、贫瘠,不过是添头。但在现在的情况下,能挽回一点颜面就是一点。总比给辽人占尽了便宜,自家却毫无还手之力要好。

“韩玉昆是有耐性,眼光好,能等到这么好的机会。”从崇政殿出来后,元绛轻声赞着韩冈惯会捡便宜。

“也许,西府之中很快就要多上一位新人了。”吕惠卿笑着回头,“是吧?子厚。”

“嗯,多半如此。”章惇言简意赅。

