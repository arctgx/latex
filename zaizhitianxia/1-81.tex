\section{第35章 把盏相辞东行去(二)}

一旦正式对河湟吐蕃开战,王韶军权独立,必然会有一个缘边安抚使的头衔下来。到时在王韶幕中,王厚理所当然的会得到一个名为‘书写机宜文字’的职位——不是‘管勾’,是‘书写’——这是安抚使的权利,可以任命家人、仆役为书写机宜文字,也就是私人秘书。

只要王韶本人做得好,便可以正式授官,这是王厚仅有的机会。要不然,必须等到王韶功德圆满,收复河湟后,立下的功劳足以让几个儿子一起沾光,才能获得官职荫封。

窦解一个油头粉面的衙内,来秦州后又沉湎于酒色,不费气力却得到了正九品的官身,对荫补之事耿耿于怀的王厚当然看他不顺眼。

刘希奭与窦解互相见过礼,又引来与韩冈、王厚相见。

窦解则随意的向韩冈和王厚拱了拱手,便自顾自的坐了下来,一拍桌子,对两名歌妓道:“怎么不唱了?我窦七可是特地来捧场的。”

‘是砸场,还是捧场?’

韩冈看了看刘希奭,秦凤走马的脸色并不好看,他作为主人都还没有说话,窦解却喧宾夺主。当真以为凭着他祖父的权势,就能在秦凤路上横着走了?

韩冈自从转生以来,在这个时代接触了很多人和事。地位高到李师中、向宝、王韶,地位低到黄大瘤、李癞子,心机都不少。年纪轻的,如王厚、王舜臣,也都有些城府,或者说都是一些聪明人。如窦解这般浅薄的纨绔子弟,韩冈还是第一次见到,‘该不会是装出来的样子吧?’韩冈总是习惯性的将人往聪明里去想。

王厚向韩冈使了个眼色,眼神中有着几分喜色。这是好事啊,窦七可是把刘希奭强往王韶这里推。

刘希奭脸上的不快只是一闪而过,笑意又堆了出来,招呼着韩冈和王厚重新坐下。琵琶弦动,牙板轻敲,两位歌妓又唱了起来,还是柳屯田的曲子词。

曲乐声中,几人随意地说着话,可窦解只理会刘希奭,却对韩冈、王厚全不答理。而韩冈、王厚也不自找没趣,也只跟刘希奭说话。

窦解上桌,方才吃的旧菜便撤了下去,惠丰楼又换了一桌菜上来。刘希奭和王厚对前面吃得一盘鲜嫩的酿豆腐赞不绝口,细嫩弹滑,洁白如玉,又没有咸苦味,口感远远超过他们过去吃过的任何一次豆腐。现在又端了上来。掌柜亲自来介绍,说是城内天宁寺的特产,过去只用在寺内素斋上,只是最近香火少了,才开始提供给惠丰楼等秦州城内地几家大酒楼。

“这是用石膏点的,而不是卤水。”韩冈随口把底细揭穿。虽然此时还是天宁寺意欲掩藏的秘密,但后世豆腐种类花样繁多,本质上却还是盐卤豆腐和石膏豆腐两种,这点小常识他也还是有的。

“石膏?”王厚、刘希奭一起问出声来。

韩冈解释道:“寻常都是用卤水点豆腐,故而有股子咸苦味,如果用的是石膏,便是如现在的这一道般鲜嫩。”

王厚摇头赞叹着:“早知玉昆博学,不意连庖肆之事亦能通晓,到底还有什么是玉昆你不知道的?”

“不愧是韩玉昆。”刘希奭随手又敬了韩冈一杯酒。

“若是说起种菜施粪,抚勾应该也是一样熟悉。”可能是韩冈得了两人的赞,让窦解心里不痛快。他的话里带着刺,却透着浅薄。连刘希奭都听着不舒服,冷冷的瞥了他一眼,更别提王厚,差点要拍案而起。邻桌也是一阵响声,却是李信和杨英两人一个拉着一个,硬是把双眼怒火熊熊的王舜臣和赵隆压在交椅上。

韩冈没有理会窦解,笑着说:“也不是韩某博通,而是恰巧知道天宁寺每月都要买上一批石膏……”

“看来韩官人的确不是博通,而是包打听啊……”窦解歪着嘴笑着,说话越发的刻薄。

王厚和刘希奭都不禁皱起眉头,窦舜卿的这个孙子怎么这般说话?连做人都不会,真不知窦家的家教是怎么教的?窦舜卿一贯的喜文厌武,曾经有传言说他想将自己的武官身份改成文官,只看他连孙子都训不好,转了文官也是丢脸。

凡事总想图个嘴上便宜,喜欢打压别人来抬高自己,这样的浅薄小人韩冈倒见得多了。如今韩冈地位不同了,在走马承受面前与窦七衙内争起闲气,反而会毁了自己辛苦打造的形象。

但给人欺上门来也不合他的脾气,韩冈偏头看了看王厚,又对刘希奭笑道:“处道兄应该是清楚的,如今医治骨伤,总少不了一味石膏。在下很快就要提举路中伤病事宜,在情在理都得要打听一下秦州各种药材的行情……”

韩冈没说下去,但王厚和刘希奭却已经听明白了。韩冈因为要打听药材的行情,从而得知了天宁寺在争购石膏,又从中推断出天宁寺做豆腐的诀窍。这一层层的推理,便体现出了韩冈的头脑明锐,闻一知十。

“这些年来,天宁寺每隔三月就要进个四五十斤石膏,若说是有人热毒缠身,非用石膏这等大寒之物不可,也不至于一用十几年,当成饭在吃。”

韩冈的解释倒是合情合理,刘希奭暗暗点头,又暗自给了他一个心细如发的评价。

自从被推荐入官以来,韩冈以尚未授官为由,对路中各处伤病营不闻不问,连他亲自起名的甘谷疗养院也没再涉足半步。刘希奭本以为韩冈是那种得了官后便无心政事的一类人,但从他暗中打听药材行情的一事来看,韩冈对他自己要负责的事务还是很上心的,也难怪王韶那般看重他。

“见微知著,王、张、吴三位果然有眼光。玉昆当真是大才。”刘希奭举杯又向韩冈敬了一杯酒。

“哪里,走马过奖了。”韩冈回敬刘希奭,王厚也端起杯子凑个热闹,不经意间,窦解已经被晾在了一边。

对窦解这样的人来说,无视便是最大的侮辱。偏激的性子,根本容不得人小觑半点。一个灌园小儿,一个阉人,还有一个幸进之徒的儿子,竟然都当他不存在,在那里自说自话。窦解的心中顿时浸透了屈辱,熊熊怒火燃起。

而韩冈还在跟刘希奭谈笑着,毫无拘束,根本看不出是第一次见面的样子。王厚对此并不惊讶,只要与韩冈打过交道,只要与他没有仇怨,都是很容易便跟他亲近起来,他本人不也是这样的?

刘希奭与韩冈有说有笑,觥筹交错,不是官场上的应酬,也不是一开始别有用心的刻意结交,刘希奭是真的觉得与韩冈喝酒聊天是件很愉快的事。甚至不知不觉中,话题转移到河湟拓边上之后,刘希奭也浑忘了要避忌一点。

与君子交,不觉自醉。

韩冈前世毕竟有过长达十六年的正规的学习经历,虽然所学到的知识,与如今世间流传的学问有所冲突,无法有效运用。但学习方法却能贯彻古今,将之运用到儒家学术的攻读上来,同样无往而不利。科学知识故且不论,十六年正规化的教育培养出来的逻辑思考能力,就已经让刻苦钻研的他立于不败之地。

其实就算没有留在身体里的记忆,只要有充分的时间用来学习和交流,他照样能在面对这个时代的饱学之士时,丝毫不露半点怯意——这是韩冈的自信。

而且从精神年龄上说,韩冈比他的外在要年长得多,早早有了稳固的世界观和人生观,性格、为人都已经成形,又是冷静现实的性子,几乎不会为身外之事所干扰。同时他还有有足够的社会经验,与人交往起来得心应手。

北宋与千年后的时代,社会、风俗、人情都有了翻天覆地的变化,但人性依旧,使得韩冈混迹在北宋的社会中依然如鱼得水。

这就是韩冈的优势所在。也是依仗着自己的经验,韩冈正小心的准备着从窦解这里探一下窦舜卿的老底。

“……再过一年半载,等王机宜在古渭和渭源将根基打好,到那时,立功的时候便到了。”韩冈抬眼像是在对刘希奭说话,但眼角却是在关注着窦解的神色。

不出意料,窦解冷笑一声:“富相公、文相公这些元老重臣,没一个喜欢妄起干戈。”

“别忘了韩相公。”韩冈第一次接过窦解的话头,出言反驳,“相三帝、扶二主,富、文可比得上?!他可是支持拓边河湟的!”

“谁说的?!”窦解仿佛听到了一个很好笑的事,“韩相公怎么可能支持王韶!?他可是骂了也不知多少次了。”

‘蠢材!’韩冈眼中藏着嘲笑。

窦解的脾气性格,韩冈一眼便看个透底。自高自大,心胸比针尖还小,又乏城府,浅薄无知。这样的人总以为是众人的中心,最受不得轻视。把握到窦解的性格,设个陷阱让他自己跳进去,也不需费多少力气。窦解这么轻易便上了当,让韩冈一点成就感都没有。

窦解脸色也变了,说了不该说的话,话一出口就已经后悔。

刘希奭面沉如水,双眼透出的寒意能把人冻结。他当然明白,赵顼把窦舜卿派来秦凤,不是为了给王韶拆台。可从窦解的话中,窦舜卿的偏向已经展露无遗,而且谁是幕后,也已经清楚明了。秦凤走马头痛欲裂,这件事他是上报好,还是不上报的好。

窦解脸色阵青阵白,让王厚看了很解气。而韩冈却站起身,对刘希奭行礼道:“今日一会,多承走马盛情。只是天色不早,明日韩冈便要启程,还是先告辞了。”

刘希奭愣了一下,又苦笑着点头:“也罢……就到这里吧。”

ps:韩三快走了,不要着急。

今天第三更,求红票,收藏。

