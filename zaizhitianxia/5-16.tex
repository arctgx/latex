\section{第三章 时移机转关百虑(二)}

【昨天住的地方断网,现在两章连发】

司马光与王安石反目成仇,王旖对其完全没有好感,听着韩冈对他的批评,想了想就道:“官人说得是。”

得了妻子的赞同,韩冈脸上又多了一份笑意:“棉花、白糖世间所无,熙河的棉田、交州的甘蔗,都是为夫一手开创。不从他人手上侵占,而自行创造新利。为什么先圣称赞开女闾赚皮肉钱的管子?”他瞟了一眼欲言又止的王旖,“可不只是因为他助齐桓公‘九合诸侯、不以兵车’的仁;也不仅仅是‘一匡天下,微管仲,吾其被发左衽矣’的功。更是因为他开创了一个产业,不与民争利的缘故,‘民到于今受其赐’!”

韩冈说得离经背道,甚至有污蔑先圣的怀疑,王旖这一回就变得张口结舌、面红耳赤。

低头看了看王旖,韩冈发觉自己说的似乎多了点。这本是他日后要拿出来推广自己政策的理论依据,只是还没有一个系统化的总结,破绽还是很多。

他不打算就这份理论再多说什么,再说就要漏了底,“所谓治政,当以公私两便、经久可行为上。量利害之多寡,审人情之顺逆。不过道理是这么说,做起来就难了。新法诸条其实还是急了些,岳父是受了天子所累。而岳父的脾气也是极刚硬的,所以才会硬顶着士大夫中的压力推行新法。换作是为夫,多半是会想方设法的绕过去。此为夫不及岳父之处。”

王旖素知丈夫虽然很尊敬自己的父亲,但对于一些法规、政策也是颇多微词。丈夫自承不及父亲,其实也是在批评新法推行时的问题解决手段太过粗暴了,许多时候,也有些变通的办法。一时间就沉默了下来。

“好了!”韩冈觉得气氛不对,“说得远了。今天的事是为夫不好,娘子大人大量就原谅为夫一回。以后家里关账放在年后计算,将年节最忙的时候跳过去,省的累着。”

王旖也是聪明的女性,就哼了一声,娇嗔道:“官人你这甩手掌柜也知道奴家辛苦啊……”她靠在韩冈怀里,“其实家里的日常用度靠着官人的俸禄已经绰绰有余,可人情往来的花费就太多了。家里这还是官人你不蓄伎乐,要是再养一班歌伎女乐,然后学着那般人整日游宴,就不知还要花多少了。”

韩冈舒舒服服的搂着妻子,笑道:“侍制以上的,哪里还要靠俸禄吃饭?只要差事不差,伎乐游宴都是小事而已。”

做到高官的任上,从来都不是靠薪水吃饭。就是王安石,在金陵也是陆陆续续置办了几百亩田宅,亲朋好友也是时不时的会送上一份厚礼。更别说一干和光同尘的重臣们,手上的权力一年随随便便都能换个几千几万贯回来。韩冈现在能玩得起的,他们一样能玩得起。

“朝廷给那么多俸禄,就是让人不要走歪门邪道,一心事上。”王旖哼了一声,当然,这不是针对能开辟产业而兴利的韩冈,“如果没有那些无谓的开销,单纯的日常用度和人情往来,朝廷的给俸已经足够了。”

“说得也是。”韩冈点头,“平常日用,一两千贯怎么都够花用了。”

“还有禄米,家里只用买肉菜,米麦都不用花钱。柴薪也有给,茶酒厨料、盐酱都有赐,最大的一项开支就没了。一个月一百五十贯,怎么都够花了。”

韩冈是龙图学士、同时也是右谏议大夫,两个职衔在官俸表上都有相对应的薪水级别。不过韩冈领俸禄时,领到手的俸料钱,不是两者相加,而是两者中取高。

龙图学士一个月的俸料是一百二十贯,每年春冬还有衣料,绫罗绸缎加起来二十余匹,另外做冬衣的丝绵五十两,近两年又增加了棉花一项,二十斤。而从四品右谏议大夫的本官就低了十几等,俸料只有四十贯,衣料的数量也只有龙图学士几分之一。

馆职、贴职的贵重,就是从这里体现了出来。所以说对重臣而言,本官——也即是寄禄官——的高低已经没有太大意义了,连最重要的确定俸禄的作用,也被馆职、贴职所替代。

除此之外,还有同群牧使这个差遣带来的添支有十五贯——虽说实职差遣跟俸禄多寡无关,但不同职位也有不同的情况,清闲繁剧各不相同,所以有了添支这个名目,重要繁忙的职位多给点,清闲卑微的职位少给点。三司使和开封知府算是最忙的,有一百贯之多,韩冈的水平是不上不下。

逐步增加到一千八百户的食邑,食实封是四百户。不过食邑的收入不是亲自去收,而是朝廷给付,一户一年三百文,这就是一百二十贯,一平均,一个月又多了十贯。

另外韩冈还能从朝廷那里得到每月给付的餐费,以他的职位有十三贯。作为福利,夏日赐冰,冬天赐炭,逢年过节也有赏赐,七七八八加起来,一个月平均差不多能有一百八十贯以上的正当收入。

如今一斗白米的均价一般是七十五到八十文,依照产地稍有差别。一贯差不多正好能买一石——因为省陌的缘故,一贯是七百八十文——韩冈一个月的俸禄,能买一百八十石。

但韩冈不用买粮,因为他还有三十石的禄米,实发六成,米麦各半。九百斤米,九百斤麦,近两千斤主粮,以韩家的人口是绰绰有余了。每天另有三升酒,三斗厨料,盐一年四石。还有韩冈是龙图学士,家里的有七名元随还能得到朝廷发下的衣粮。

相对于一日忙碌只能挣到百文,买点米,买点菜,然后就剩不了多少的普通百姓,韩冈的收入已经是高到让人难以想象——要知道,一个从九品,文官月俸六贯,武官四贯,衣料也少,禄米也少,如果能在外任职,每月还能多两口羊,两顷职田,若是在京就什么都没有,差了几十倍去;至于吏员,重禄法之后才有工资拿——可宰执们的收入,少说也有韩冈的三四倍。

所以说越是重臣,待遇就越好。

“不过升到侍制以上的重臣之后,要养亲戚,要养门客,要蓄养伎乐,迎来送往还要送上一份厚礼,薄了就有失身份,光靠俸禄是远远不够的。”韩冈叹了一口气,“为夫也不记得是谁了。说是某人刚刚升上侍制之后,向他伸手的亲友宾客就多起来,旧日还能经常吃肉吃酒,一下就沦落到以素食为生的状况了。”

“所以要想不残民,不争利,要么学包孝肃,亲友宾客不相通问,要么就是学为夫,设法兴利,以补贴家用。”

“都是不好学呢。”王旖幽幽一叹。

夫妻俩低声说着话。时候也不早了,除夕事情又多,没过多久听到外面有人唤,王旖连忙匆匆忙忙答应了一声,让人在外面候着,又慌慌张张从韩冈身上撑着身子起来。

“看着就知道不是做贼的料。”韩冈靠在躺椅上,头着双臂,笑道:“上一次也是慌慌张张的。又没人敢进来,你慌个什么?”

“越是没脸没皮了。”王旖冲韩冈啐了一口,脸说着就红了起来。过去韩冈也有白天强来的时候,也曾差点被人撞破过。

王旖暗自庆幸着幸好外面的使女守着门不敢靠近,不然真是别做人了。

她对着铜镜整理着散乱的头发,一边对韩冈道:“听说京城市面上已经有了透明的玻璃,官人当初在军器监的时候,就为此定过赏格了,是不是军器监出来的?”

“为夫也听说了。过些日子就能有玻璃银镜了,显微镜也能更进一步。”

玻璃的这个名字,还是韩冈确定下来的。

这个时代的玻璃,透明的并不多,多是不透明的彩色材料,而且有好几种名字。有叫琉璃的,又叫药玉的,还有发音与玻璃相同,而写作另外的字的。而从西方贩来的透明玻璃制品,则被称为蕃琉璃、假玉。

韩冈还在军器监中的时候,就为透明玻璃下了赏格,以官职和三百贯重金悬赏,就是他离职后,这份悬赏也没有被废除。对于韩冈在工艺制造方面的表现出来的能力,后继之人无人敢挑战。

王旖听韩冈提起过,也清楚玻璃跟镜子有什么关系,“日后真的造出了玻璃银镜,这就又是一个产业了?”

“算是吧……不过铜镜匠人渐渐的就要改行了。”

“这不算争利?”王旖挽好头发,转回头来,俏皮的笑问着。

韩冈笑道:“玻璃银镜价格极贵,能买得起的人是极少数,可不会与平民争利。为夫悬赏透明玻璃,只是想着用在显微镜、眼镜和放大镜上。这个眼镜和放大镜过去用的都是白水晶,用得起的寥寥无几。若是透明玻璃出来之后,就能降下价来,那可就是真真正正的一门新产业了,于国于民,都是大利。”

王旖整理好衣服,回过身来,向韩冈福了一福,笑道:“那奴家就恭祝官人明年大吉大利了!”

