\section{第五章 冥冥冬云幸开霁(六)}

齐王府被围得水泄不通。

两百多班直禁卫,以及一个指挥的天武军卒,守定了齐王府外的围墙。

按照王厚出来时,从郭逵那边领到的命令,那是一只老鼠都不许逃掉。

这个要求未免太过苛刻。

不过如果目标只是府中的人,那依靠就从军器监那边拿的一批强弓硬弩,王厚还是很有信心守住齐王府的围墙,

王厚现在就骑在马上,正面便是宽达两丈的齐王府大门。中间的正门紧闭——平常都是如此,除了赵颢出入,或是贵人上门,正门都不会开——而两边的侧门也关着的。方才在班直赶来的时候,便一下关上了。

不论是兵围府邸,还是宣读诏书,门都没开一下,甚至连个出头问话的人都没有。

王厚不知道齐王府内是不是还抱着一丝侥幸,但他可没打算在这里耽搁太多时间。

撞门不易,寻梯子也不方便,但王厚从军器监中,借到的可不仅仅是几百张弓弩。

王厚的背后就是赵家老三曹王赵頵的府邸。

赵顼的两个弟弟的王府,是相对而建,只隔着一条宽约五丈的街道。

见王厚领兵而来,徐王府的大门也同样紧闭,看见对面的齐王府被围,一样不敢多问。

不过窥探就少不了,围墙上也免不了有些杂音。

听到身后有动静,王厚回头看了一眼,墙头上冒出了一溜脑袋,而正门旁的侧门,也被拉开了一条缝,几双眼睛从里面窥探着。

不过见王厚回头,墙头上转瞬就没了人,刚刚拉开一条缝的小门,也立刻关紧了。

“上阁。”

王厚身旁的内侍回头看看,不无担心的问着王厚。

“没什么。跟曹王无关。”

王厚望着正面,一动不动。

说话的内侍也在马上,几乎与王厚平齐。

这名内侍怀中插着一封卷轴。看他身上的服饰,就知道还未入流品,但怀中的卷轴,只要熟悉朝事,一眼就能从纸背花纹中看得出来,那诏书才会用到的绫纸。

王厚没理会这名内侍,宫里面还没给安定下来。

石得一、宋用臣久在宫禁,地位又高,门生弟子无数,与他们有瓜葛的宦官,在宫中有职守的内侍中占了大半去。剩余的一些有资格宣诏的内侍,现在都在大庆典上赞礼朝会,一时间竟只能拉了一个连从九品黄门都不算的祗候高品来宣诏。

不过管宣诏的内侍是几品官,仓促写成的诏书是不是有哪里不对,只要将二大王家里给封锁好了,重要的人犯一个不漏的给抓起来,再搜查到罪证,对王厚来说那就是功德圆满,可以回宫缴旨了。

齐王府正面朱红色的大门上,铜钉给擦得锃亮,相形之下,大门上方的几条白绸就显得黯淡了许多。

前几日王厚和韩冈在寻找大图书馆地址的时候,还顺道在巷口看了几眼。

当时王厚还感叹,二王府邸比韩冈在京城的家宅要大得多,建筑也出色得多,先帝待两兄弟也算是厚道了。

谁知几日后,二大王就再没那个福分了。

不知当日二大王知道韩冈往来这边,会是什么想法?有石得一在,肯定是瞒不过他的。或许今日的宫变,在其中推了一把也说不定。

具体的情况,王厚猜不到,不过也没多少兴趣去猜。

只是等的有些无聊。

“上阁,要不要小人再去叫一叫门?”

见王厚始终没有动静,内侍更加小心翼翼的问着,完全没有传诏天使、奉旨监军的威风。

王厚今天立下了大功,他背后的靠山功劳更大,新上任的知西上阁门使的位置一下就坐得稳当了。

当初授王厚以西上阁门使,以他的资格还是差了点。不过韩冈在里面使了点力,让太后与东西两府都同意了这项任命。

而且朝野内外对英年早逝的王韶评价很高。十年来的西北战略,都是遵循着他的方案。在西夏灭亡之后,甚至到了有人将他的《平戎策》与诸葛亮的隆中对相提并论的地步。认为是释皇宋百年之困厄,救关西生民于倒悬。

看到王韶盛年病殁,在倍感遗憾之余,世人无不觉得先帝对他亏欠许多。所以在人事安排上,韩冈为王厚争取一点补偿,朝廷里面很难有合适的借口来反对。

这项任命本属于超迁。可凭王厚今日在殿上的表现,他肯定能得到太后的信任。也许接下来的多年时间,他都会在京师中掌管禁卫兵马。

眼看王厚身上衣袍已经红得就要变紫了,换作是宫内的大貂珰,说话都要放几分尊重,何况正指望着能凭今天这一回的出场,挣一份官俸回来的区区祗候高品?

“没必要。”王厚拒绝得十分干脆。

太皇太后的情况不知清楚,但二大王现在的状况,王厚是知道的。

赵颢与他的儿子——那位被抱上御榻的伪帝——都被关在了宣德门的城楼上,由郭逵亲自镇守。

已经是彻头彻尾的叛逆,对叛逆的家眷,完全没必要给予什么优待。

“上阁!”那内侍突然又叫了起来。

王厚也变得面色凝重,望着齐王宅内,那里正冒起了几股黑烟。

“起火了!”内侍失声叫道,“上阁,里面起火了!”

“我看到了。”王厚语气平静。

“上阁。”内侍惊讶的望着王厚,“要快救火啊!”

“不,你们注意不要让火势蔓延。府中人出来,都必须要看管起来,若有人敢于反抗或逃窜,杀之无论。”

至于救火,没那个必要。

这句话王厚没说出口,但听到他命令的人都明白了他的心意。

王厚完全无视,内侍也不敢打扰,闭上嘴等着王厚的命令,抬头看着那愈发浓烈的烟火来。

……………………

‘那是……’

宗泽陡然间停下了脚步,惊讶的从巷口往巷中望去。

‘……班直?!’

很难想象当今皇帝的亲叔叔的府邸,会被兵马围上,而且还是禁卫。

‘到底出了什么事?’

宗泽在拥挤的人群外猜测着。

通向二王府邸的街口,早围上了一圈看热闹的路人,宗泽只能仗着自己骑马,借着高度的优势,向里面张望。

他刚刚从城北回来,就碰上了兵围齐王府的一幕活剧。

‘没听说拜文昌庙,会应在看热闹上啊。’

宗泽头脑中转着莫名其妙的念头。

供奉了子路、子夏的二圣庙,前日宗泽已经出南薰门去拜过了。

今日又往城北来,拜过文昌庙。虽然不知道来自梓潼的文昌星君,会不会只保佑蜀人,这好歹是京城中两座主管文运的祠庙之一,拜上一拜总无坏处。

宗泽出来烧香,与其说是求神拜佛,不如说是调整心境。所以也没有呼朋唤友,而是独自出门。

静静的上一炷香,布施点香火钱,嗅着庙中的香烟味,因省试在即而变得浮躁起来的心情,也一点点的安定了下来。

不过回程时,撞上一出好戏,是他所没有预料到的。

今天是大祥后的第一天,依例是开大庆殿的大朝会,在京文武百官和宗室都要入宫。班直在这一天围了二大王府,用脚趾头想,就知道肯定是赵颢在宫中犯了事,让太后不再顾及脸面。

能够造成这样的结果,二大王的罪行必然不轻,多半会跟帝位归属牵扯不清。疯了一年多,不好好的享受余生,还故态复萌,又开始得陇望蜀,这就是自己寻死呢。

由于班直封锁巷口的缘故,宗泽只能远远向内望去。二大王家门紧闭,而对面的三大王家同样家门紧闭,两边都不见有人出来

“肯定是坏事了。”身边有人低声议论,又有些骚动。

齐王府中竟然又起了火。但距离最近的班直,却没有一个上去救火,动也没动一下。

坏了事是肯定的。不甘寂寞的二大王一夜之间就疯病不再,任谁都知道他想趁先帝大行的这段丧期,出来搅风搅雨一番。

天家的那对叔嫂之间的关系有多恶劣,从传言中就可知端的。

可向太后从二大王‘病愈’开始,就出人意料的一直忍到现在。但忍耐的时间越长,这爆发出来的怨恨就越深。

而这场祸事的程度到底有多深,只看班直们的态度就知道了。

不过赵颢只要还有一分卷土重来的可能,只要太皇太后还有东山再起的希望,只要太后的旨意没有太过决绝,过来的禁卫行动就不会太过狠厉。

“那是什么?”

忽然围观的人群中,一阵窃窃私语。

只见一辆由四匹马拉动的双轮马车从大街北面驶来,车身外蒙了一层布套。也不知里面装了什么,布套被顶出了奇怪的外形。

这辆车本已很显眼,但更为显眼的是车身周围的士兵,多达上百人。

“火炮。”

宗泽低声自语。

双目放光的看着炮车咕噜咕噜的从面前驶过,宗泽突然想:

‘这一回轮到二大王了啊。’

……………………

“来了!”王厚突然向巷口看过去,又抱怨着,“真够慢的。”

内侍顺着王厚的视线望过去。

只见一队士兵进了巷口,之后又是一队,再后,就是一辆马车缓缓驶了进来。

只看马车碾过青石路面的声响,就知道马车上的货物有千斤之重。

“那是什么?”内侍惊问。

“火炮。”王厚回答。

军器监离皇城不远,要不然前几天也不会一炮打中郭逵府。而两位亲王府邸,当然也同样在附近。

方才从军器监借了一批弓弩,顺便的,王厚也奉韩冈之命,让人从火器局中拖了一门火炮出来。

铜炮身,铁炮架,钉铁的木炮轮,揭开布罩的火炮暴露在世人面前。

炮手一阵忙碌,火药、炮弹很快装填完毕,黑洞洞的炮口对准了齐王府的大门。

“李彦。”王厚叫着内侍的姓名。

“上阁有何吩咐?”李彦连忙问。

“捂住耳朵。”王厚道。

“啊?”

“捂住耳朵!”

