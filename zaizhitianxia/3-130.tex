\section{第34章 雨泽何日及(五)}

一幅画卷铺开在御桌上,不过不是泼墨山水,也不是工笔美人,而是简简单单的一幅由不同颜色的线条和图标组成的舆图。

在图纸上,实线代表的道路纵横交错,营中各坊的界线用虚线表示,红色的线条是沟渠,蓝色的则是引水道。一座座简易房舍是小小的方框,水井的标志却是○中加个井字。风车、茅厕、各色地标都有独特的图案来表示。却不似过往的舆图,是山就真的画座山,是水就真画条河,亭台楼阁、房子、屋子都照着原样绘在图上。

而赵顼已经习惯了韩冈的这种图纸风格,当初从关西送来的地图,就是渐渐的都转换成了用图标符号来标志山水城寨。看起来虽然不如旧时直观,但更为清晰明白。

对着图纸,图轴一侧密密麻麻的注解,再加上韩冈在一边则不厌其烦的回答着心中的疑问,赵顼很轻松的就将韩冈在流民营中的一番布置在脑海中形象的绘制出来。

从提供给流民们的简易屋舍,到饮水道的设置,再到临时保甲的设置,防火防疫的应对,只剩老弱妇孺的家庭的安排,甚至还有粪便的处理,细致到生活的方方面面,每一个细节都尽量考虑到,从中也可知道韩冈究竟费了多少苦心。

看着比上次觐见时,似乎瘦了一些的韩冈,赵顼有着深深的感慨。在他看来,治政的才能上,朝中能与韩冈相比的官员还是有不少人的,但能如韩冈一般用心的,却是极少数。

‘毕竟还是寒素出身,所以才会对流民感同身受。’赵顼暗自想着。

从舆图上抬起头,赵顼点头而笑:“韩卿果然用心,这一下朕也可以放心了。”

韩冈退后半步,躬身道:“臣愧不敢当。”

一直以来,韩冈与王安石若即若离的态度,才是赵顼相信韩冈说辞的关键。

吕惠卿、王雱、吕嘉问这一干人,在天子面前为新法辩上千句,也比不上韩冈轻轻巧巧的三五句话。

娶了女儿是一回事,但在政治上,韩冈没少拆王安石的台,尤其是经义局一事,闹得翁婿离心,赵顼也是清楚的。

在赵顼的印象中,韩冈对于新法,有的认同,有的反对,对于不了解的法度绝不会盲目说好,这次才是为人正直的表现。

所以赵顼想听一听韩冈对郑侠的看法:“韩卿,郑侠妄言白马之事,以不实之罪弹劾卿家,不知卿家觉得该如何处置?”

韩冈没有犹豫:“郑侠妄言臣过,臣心中亦是不忿。然朝廷治政,不当以言辞罪人。臣愿陛下斥其谬言,容其改过。”

赵顼瞥了王安石一眼,这又是韩冈跟他岳父不一样的地方。王安石很多时候,都是要将反对者踢出去京城,反而赵顼要设法保着朝堂上有不同的声音存在。

只听韩冈继续道:“郑侠于疏中言之凿凿,道所绘流民乃其亲眼所见,治罪于他,料其难服。臣恳请陛下将郑侠转调府界提点衙门,或是白马县中为官,让其亲眼一见微臣如何安置流民。”

赵顼差点失声要笑起来,韩冈看似稳重,但还是年轻气盛,硬是要将郑侠折服。从这番话中,可见他的自信,但赵顼不会拿救治灾民之事冒险。

他现在对郑侠的看法很差,哪里会让这等奸人就任关键的职位,摇摇头,“这一事,朕就不能允你。朕虽不欲以言辞罪人,然朝廷自有法度在,郑侠区区一监门官,擅发马递已是一桩罪过,而妄言无据之事,更是难赦!卿家不必多言了……”

……………………

吴充今天不知第几次搁下了手中的笔。桌上堆着的公文足有尺许,等待他批复的军情文案一封接着一封的从承旨司送来,但他面前的公文只见增高,不见削减。

但承旨司那边并没有来催促,吴充枢密使的身份不提,另外,承旨司的前任长官,前枢密院都承旨李评也就在吴充这里。

李评是娶了太宗女儿万寿公主的李遵勖的孙子,算是外戚出身。极受天子宠信,常常留他下来聊天。但李评极端敌视新法,没少在赵顼面前攻击免役法等事,王安石几次三番要将其治罪,都给赵顼保了下来。不过在两年前,李评私改枢使文牍被王安石抓到,将之逐出了京城,外放保州为官。

李评在外任官两年不到,便被吴充找了个由头召回了京城。新党这一段时间,都忙于应付市易法和旱灾带来一系列攻击,根本无心理会这等小事,使得李评顺顺利利的就重新回到了开封府。

李评被外放的保州【今保定】位于河北,吴充设法将其召回,自然有一番用心在。只是吴充却没想到,竟然有人先行下手,看情形他的亲家应该熬不过去了——而这人,竟然还是一名城门小吏。

“真没想到城门还有一个侯赢般的人物。同在大梁城中,相隔千年,足可相辉呼应。”李评虽是外戚,任着武职,但口才和才学都不差,要不然也不会在与赵顼聊天时,‘上色未尝不欢也’。

吴充身为枢密使的矜持让他不便放声大笑,但仍是忍不住抿着嘴:“王介甫如今众叛亲离,曾布是一桩,郑侠也是一桩。”

“树倒猢狲散,正是这个道理。”李评笑道:“下官方才听宫内传来的消息说,昨夜官家拿着流民图一夜都没合眼,长吁短叹,几至泪下。官家为百姓忧心如此,我辈如何能妄食俸禄,而不想方设法为天子解忧?!”

“自当如此。”吴充点了点头。

方才院中的吏人来报,对面的东府之中,王介甫身边的一众走卒,群居一堂,惶惶不可终日,多半也是知道末日将临。只是等到现在还没有消息,让吴充心中焦躁不已。

门外的廊道上响起了急促的脚步声,吴充和李评一起望过去,只见一名小吏来到门外。通了姓名,却是方才吴充派出去的亲信。

亲信进了厅中,看了一眼李评,走到吴充身边低声说了好一阵,方才直起身来。

吴充神色不动,只是沉默的挥了挥手,示意来人出去。在李评询问的目光中,过了半天他才一拍桌案:“好个韩冈!”

……………………

韩冈随着王安石从延和殿中告退出来。

虽然王安石神色依然没有太大的变化,连步伐也依然保持着宰辅重臣的沉稳,但跟随在身后的韩冈,还是听到王安石极轻声的舒了口气,这一道险关总算是跨过去了。

翁婿二人一前一后,沉默的穿过宫廷,在许多人的注目中,一路回到了政事堂前。

韩冈没有随着王安石往东府里去的意思,他是开封府的僚属,不是宰相的,众目睽睽之下不便跟着王安石回政事堂:“岳父,小婿这就要去见孙府尹,不知岳父可有什么吩咐?”

王安石定住脚,回头看了韩冈一眼,动了动嘴唇,想说些什么,但又化作了一声长叹,将感谢的话收起,正色道:“玉昆,你可知从今日以后,再难有安稳的一天?”

王安石想说什么,韩冈当然清楚,“小婿已经有所准备,义之所在,虽千万人吾往矣。”

从普通朝官往宰执位置上走,身上怎么可能不背上几十上百本弹章?王安石做了这么几年宰相,弹劾他的奏章叠起来等身髙,而吕惠卿、曾布等人,同样都没有少受弹劾。

争权夺利,哪有不下狠手的道理。官场越往上,位置越少。你上去了,别人就要下来。韩冈现在已经是府界提点,再往上走,每走一步,就不知要踩下去多少人。而别人要上位,同样也要踩着韩冈的头上。

过去韩冈虽说升迁之速,建国以来屈指可数,但也不过是一个年资浅薄的普通朝官。又跟王安石因经义局闹翻了脸,所以旧党没有将他当成攻击的目标,而想着看翁婿俩的笑话。就算去年年底的纲粮抢运,外界所知的韩冈的功劳也只是发明雪橇车而已。

但这段时间他在白马县的一番作为,已经引起了所有有心人的注意。加上他升任府界提点,只要顺利的将流民安置好,就是帮着新党稳定了大局。相比有许多人不会愿意看到韩冈成功,接下来,必然就是暴风骤雨一般的攻击。

即将成为众矢之的,韩冈早有了心理准备,迟早都要经历的,早一点也不是坏事。只要天子信任,自己这边不出大错,任何弹劾都会无功而返。但关键的问题是,他必须得到政事堂的全力支持,而不仅仅是开封府。

韩冈道:“小婿即为府界提点,进入京畿的流民若有不妥,便是小婿的罪过。外人的弹劾小婿不担心,只担心有人坏事。”

王安石对此知之甚深,“今日得了玉昆你襄助,总能再撑上一两个月。安置流民之事尽管安心去做,老夫不会让人动你分毫!”

