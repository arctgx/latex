\section{第16章 千里拒人亦扬名(下)}

王舜臣急得冒汗,赵隆看着韩冈的眼神中则明明白白写着疯子二字。但韩冈一点也没疯,他也不怕得罪向宝。因为这里不是秦凤路兵马都钤辖官厅,不是秦州州衙,不是向府,而是伏羌城!是处在军机要道、来往官员军马无数的伏羌城!

他那一箭,是故意没有射中——不然区区三五步距离,箭术退步再多也不至于失手——但既然射了出去,肯定会就在短时间内传遍整个秦州!在他们周围,究竟有多少双眼睛看见了刚才的那一幕,根本算不清楚,只能看见周围的观众聚得越来越多。当韩冈一说出要在这里等,周围便轰然叫好!

看客们的喝彩声韩冈充耳不闻,王舜臣和赵隆的劝诫也是不加理会,只背负着手,仰头看天。心中却是在默默的盘算着利害得失。

韩冈也是被逼无奈,若是让向荣贵把车拉走,自己也被拉去抗肩舆,陈举会怎么做,根本就不用想。想让向荣贵为他说话,那更是个笑话!拦截军需,罪名可大可小,若是没爆出来,什么事都没有——看向荣贵肆无忌惮的样子,以前并没有少做——可一旦闹出来,连向宝都不肯往身上揽,向荣贵一个钤辖家的家奴能担待得起?如此局面,他韩冈若是不拼命,那就是死无葬身之地!

但把事情换个方向去想,既然拦截军需是个罪名,那向宝就不敢将之公开——就算他拉得是地方上的人和车,而不是运送到前方的军需辎重,被揪出来后,也照样少不了要吃点苦头——闹得越大,他韩冈就越安全。只要站得正,行得稳,向宝对韩冈也无可奈何。

因为韩冈是士子,而向宝是武臣!

在大宋,文武殊途。韩冈方才说的做的,王舜臣便说不得做不得。一个是士人,一个是武夫,官僚对他们容忍度是截然不同的。

韩琦韩相公对犯事的从官能一笑而过,却可以随便拿着一点小错,去杀一个久历边事、战功累累的将领。只为了给将领的上司狄青一个下马威。狄青为他的手下焦用去叫屈,并称焦用是立过功的好男儿的时候,韩琦却说:“东华门外戴花游街【注1】才是好男儿!”如焦用这等武夫,不过是杀鸡给猴看的鸡罢了。鸡被杀了,狄青这只猴子,也的确被吓得不敢再说话。

向宝纵然身份显贵,还有一个带御器械【注2】的加衔,却也别想对一名有跟脚的士子想打想打,想杀就杀。暗地里也许没问题,但摊开在阳光下,给他十个胆子也不敢。

事情既然已经闹大了,若是向宝还敢为今天之事跟他韩冈过不去,不知会招来多少弹劾!想表现出气节的文官,天底下太多太多,连李师中听说后,都要为此事上书,否则监察御史那里少不得会反过来给李师中参上一本。

文官会官官相护,但遇到武臣……是乘机卖好还是踩上两脚,端得看心情!看时机!

何况这件事上,向宝他完全不占理。向宝派过来主事如果够聪明,那就只有一件事可以做——

“在下向安,见过韩秀才!”正如韩冈所料,没等多久,一名看起来有些身份的小老头子来到韩冈面前,向荣贵就跟在他的身后。只是向荣贵一去一回,一张瘦脸已变胖了不少,双颊肿得如同发起的炊饼,红得发亮。

向安回手指着脸被打肿的向荣贵,“方才家奴无知,竟然开罪了秀才。在下已经教训过了他,若秀才仍觉得不够解气,在下便当着秀才的面,再给他一顿家法便是!”

韩冈还了一礼,容色依然冷淡,“官人有心了,韩某方才之气,为得是国法,并非为己。韩某奉命押送军资,如何能改为私家奔走。都钤辖私事又岂能凌于国事之上。若以为韩某只会纠结于私怨,就未免太小瞧我了!”

“秀才果然宽宏大量。在下以小人之心度君子之腹,有罪,有罪!”向安躬身一礼,看上去真心诚意。

韩冈眉梢一跳,暗骂道:‘老狐狸!’杀了黄大瘤,阴了陈押司,诳了吴节判,吓了向荣贵,今次,还是他第一次遇到滑不留手的对手。

“话虽如此,但秀才毕竟是读书人,如何能服这贱役。不如跟小老儿回秦州,成纪知县当不会驳小老儿的面子。”向安诚诚恳恳的劝道。

只要韩冈低了头,跟着回了秦州,这件事上,便没了向宝的错。再有人拿此说事,有错的只会是前后反复的韩冈。可他不愁韩冈不点头,衙前是什么样差事,天下谁人不知,甘谷城里的那位专会在衙前身上剥皮抽筋的管库,更是名声显赫。能脱离差役之苦,就算丢脸又会有谁不干?,

韩冈退后一步,一揖到地。如果刚才韩冈留给众人的印象是刚直严正,现在的表现却与方才截然相反,一转眼就变得卑躬屈膝。

‘终究还是露了原型!’向安眯起眼,虽是如己所愿,却仍忍不住心生不屑。周围的不少人也与他一般想法,韩冈的前后表现实在差得太远:‘这也是读书人啊!’

直起腰后,韩冈却对向安道:“君之美意,韩某心领。只是人无信而不立,韩某既已受命,自当全始全终,哪有中道而废的道理?”

韩冈的回答,完全出乎向安的意料。刚才那一弓腰,难道只是为了谢绝他的好意?!

周围的观众也是一片哗然:‘能脱离苦海却还死赖着不走,这秀才疯了不成?’

“不识好歹!”向荣贵捂着肿得越发得高起的腮帮子,嘟嘟囔囔的骂了一句。

韩冈理也不理,最有效的鄙视就是漠视,何况向荣贵回去后,怕是只有死路一条。

他打断想开口再劝的向安,道:“国法不可妄违。释某衙前之役,县尹可,府君可,而君不可。韩某承蒙不弃,欲救某于苦海,实是铭感五内。可既承君之盛情,便不能陷君于不义。这悖国法、逆军规之事,韩某怎能让向君来做?!此违圣人之教,韩某又岂可为之?”

咬文嚼字的一番话后,韩冈又一揖到地,把礼节做足,不待向安回应,转身便走。顺势对着王舜臣、赵隆等人摆了摆手:“没事了。我们去营里!”

王舜臣正在震惊中,赵隆的嘴巴到现在也没能合上,听到韩冈说话,便糊里糊涂的跟着他往前走。走了几步两人才反应过来,‘俺怎么成跟班了?’

一众民伕也都懵懵懂懂的赶起骡车跟在后面,把脸色阴晴不定的向安抛在脑后。不经意间,韩冈的领导地位已经得到了所有人的认同。

王舜臣本是自负其能的人物,会接下吴衍的任务,也是只是欣赏韩冈在军器库中的手段和胆量,顺便让陈举难过一下。只是他现在看着走在前面的韩冈,却多了几分敬服之色。裴峡谷中的战斗姑且不谈,单是方才对上向荣贵和向安时的表现,已足以让王舜臣折服。

赵隆也是又惊又叹盯着韩冈的背影。他绝非怯弱之人,若是孤身面对百十个西贼,他照样敢斗上一斗。但如果他遇上的是自家的军官,就算只是一名巡检,他便不敢稍有违逆,更别提一路都钤辖——无他,怕累及家人。

可一个毫无凭藉的穷措大,却义正辞严的拒绝诱惑和威胁,将一路都钤辖的亲信家人驳得哑口无言。读过几年书,还有个名为‘子渐’的表字的赵隆,心中突然冒出了孟子说的几句话:‘贫贱不能移,富贵不能淫,威武不能屈,是为大丈夫也。’

韩冈昂首阔步独自走在前面,他走到哪里,哪里的人群就自动为他分开一条道路。神色庄严肃穆,但心中已笑开了花。他还记得前世曾听过的一句话——推销员推销商品在本质上其实是在推销自己。韩冈如今身份已变,但他依然知道,该如何推销自己!老天爷送上门来的机会,他如何不去把握住!?

得罪了押司,得罪了知县,得罪了都钤辖,韩冈如今是债多不愁身,因为他的情况不可能再坏,也因为他有底气。对于一名没有官身、缺乏背景的贫寒士子来说,声望就是一切。有了名望,他的地位便稳如泰山,权势不能侵,富贵不能欺。

韩冈追求的就是名望!他前日挑战陈举,名声已经遍及州城内外,他现在挑战向宝,名声难道还传不到秦凤路中吗?等他不惧权势、尽忠国事的名声打响之后,又有谁能动他?陈举?还是向宝?

军器库一案,裴峡谷一战,还有方才的一箭,等这三桩事传扬开去,在秦州道上,他韩冈不大不小也该是个人物了!

注1:指中进士。在北宋,每科科举结束后,进士们便会骑着马带花游街。从东华门一直走到城西的金明池,参加琼林宴。

注2:顾名思义,就是在天子身侧可以携带武器的护卫。在宋初,属于实职,在天子身边轮班宿卫,定额为六人。但到了后来,渐渐演变成了赐给近臣、功臣的荣誉加衔。再打个比方,如果此时真有御猫展昭,那他官职的真正名号就不是什么四品带刀护卫,而是带御器械。

ps:许多事越是放开来做,越是有成功的机会。若是畏首畏脚,失败便是必然。

今天第二更,求红票,收藏

