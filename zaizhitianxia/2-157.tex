\section{第29章 顿尘回首望天阙(九)}

【第一更。】

周围嘈杂的声音一瞬间都静了下来,人人为之侧目。韩冈也惊讶的看着甘穆,什么时候小小的吏人已经有这个胆子跟官员说话了?而且还是教坊司中拉皮.条的王八。这本事见涨啊!

韩冈一瞬间燃烧起来的怒火几乎能燃尽整个大厅,不过这外放的怒意转瞬即逝,全都给他压在了心头上。

而甘穆却只觉得扬眉吐气,能让他看到自己平常见了都不敢抬头的官员们气急难言的样子,走这一趟都值了回票价,而且还没有白白浪费脚力。他背后站着雍王赵颢,何惧一个选人。虽然今次来没得雍王吩咐,但他和许大娘这么贴心,让雍王知道后,总少不了他们的好处。

他再一次重复着:“周小娘子时常有贵客临门,无暇分身,还请官人不要来找了。”

厅中窃窃私语的声音一下大了起来。雍王赵颢私下里出宫找些乐子,最后看上了教坊司花魁周南的消息早就在市井中传开了。对于厅中的大小官员来说,这些欢场上的风流韵事也从来都不是秘密。周南算是个奇女子,许多人都知道她在为人守节,就是不知那人到底是谁?

可现在教坊司派人气急败坏的过来,基本上是不打自招了。几十对羡慕嫉妒的眼神向韩冈望过去。这个年轻人,不但王、韩两位丞相都看重于他,连名妓也垂青他,人和人都差别怎么这么大?!

不过嫉妒归嫉妒,甘穆的无礼还是惹起了许多人的不快。不管韩冈是不是与周南有私情,雍王殿下争风吃醋到这个地步,未免做得太过了一些!士人倚红偎翠,嘲风弄月那是风流盛事,你亲王跑出来棒打鸳鸯算什么?不少官员的脸色都阴沉了下去。

“小六,你待会儿去安仁坊走一趟,今天晚上我要宴客,让周小娘子把时间空下来,等我的消息。”

韩冈仿佛什么都没听到,甘穆说的话,就从他耳旁划过。也许教坊司中人以为凭赵颢的身份,他韩冈只有乖乖退让的份。但士大夫的尊严就算天子也不敢轻辱。私下里找韩冈说话,没有问题,但在大庭广众之下,明着排除情敌,这分明是上门找打来着。

什么叫同仇敌忾?看看现在厅中官员的眼神就知道了。韩冈自知招人嫉妒,如果是跟其他文人争风吃醋,这里面的大多数人只会占到自己的对立面去。可换作对手是亲王,韩冈的身边就都是支持者!东京城内的士人们虽然不会为韩冈明着出头,但私下里激起的士林清议,足以让赵颢灰头土脸。

韩冈将得意的冷笑藏于心底。他特意走出来见客,而不是把人招进去说话,本是存了以防万一的心思,但没想到教坊司的乌龟竟然这么配合,进京后,还没有这般好运过。

‘难道是终于转运了不成?!’

韩冈的吩咐,李小六毫不犹豫的点头应下。韩冈又回头看了李信一眼,李信会意,安仁坊是教坊司的老巢,这龙潭虎穴不好闯,李小六细胳膊细腿,这副身板,经过不了几次折腾。李信站了出来:“我和小六一起过去。”

韩冈针锋相对,是明着跟甘穆身后的雍王过不去。厅里的官员现在都是看好戏的模样,而原本在客栈内部的住客,听到消息后,也出来了好一些。虽然有着官人们的矜持,不会像普通的看客那样围成一圈,但他们坐在韩冈座位的旁边,随意的点了两个菜,竖着耳朵、斜着眼睛,这等吃饭喝酒的样儿,其实更惹人发噱。

李信和李小六赶着就要出去,韩冈也不理会甘穆,权当没看到这个人,直接就转身打算回自己的小院。

甘穆冷嘲似的声音从背后传来:“韩官人,你看不起小人倒没什么,不理会小人也没关系。但周小娘子今天不便见外客,官人还是不要让人白费力气了。”

韩冈脚步停了,但他没说话,而是李小六帮他出头道:“周娘子倾心于我家官人,京城多有人知。我家官人要见周娘子,难道还有人要拦着不成?”

甘穆在后面嘿嘿冷笑,神色张狂,“拦着又如何?难道你一个小小的选人还想跟二大王争!?”

此话一出,顿时把厅中人都得罪了,在座的基本上可都是选人。人人面色不善,就看韩冈如何处置了。若是不能让他们满意,他们可就要自己出头收拾人了。

韩冈叹了口气,这个白痴,以为雍王的名头是这么好借用的吗?赵颢听到了,肯定会恨不得拿杖抽死他。遇上这样愚蠢的对手,韩冈都觉得胜之不武。

也不理会小人得志模样的甘穆,韩冈直接唤来驿丞。锋芒毕露的双眼笼罩住在城南驿中奔走多年的老吏,惊得他如同被猫盯上的老鼠。愠声道:“你也看到了,也听到了……知道该怎么做吧?”

驿丞连连点头,转头叫来人手,指着不知末日将临的甘穆:“还不把这个满口胡言的疯子绑起来送到衙门里去!雍王殿下,也是你敢污蔑的?!”

被驿卒左右架住,甘穆惊慌失措,得意神色全都没了,他想不通他怎么要被抓,挣扎着,连声叫道,“俺是教坊司的人!俺是教坊司的人!俺真是替二大王来的!”

驿丞听得额头直冒虚汗,在他地盘上闹了这一出,前面没能拦住已经是个罪过了,现在再任由甘穆扫尽天家体面,那还会有好结果?他还想多活两年呐!

连忙飞起一脚招呼到甘穆脸上,把他踢没了声。指着满脸溅血的教坊司小吏,破口大骂,“你这鸟贼,竟敢冒二大王的名头说事?!还不堵上他的嘴!?拖出去!”

两名驿卒不知从哪里拿了块油晃晃的抹布过来,硬是塞进了甘穆嘴里,横拖竖拽的把人拉了出去。

甘穆呜呜闷叫着被强制退了场,韩冈冲着周围官员拱了拱手,神色坦然:“让诸位见笑了。”

“哪里,哪里。韩兄人物风流,有此一事不足为奇!”

“目无尊上,语出悖逆,如此小人,就当严加惩处。韩兄做得正是!”

周围一片声,或调侃,或愤慨,无不支持韩冈的做法。甘穆的下场,让围观的官人们觉得很解气。而韩冈处置的手段,也没有任何可以挑剔的地方。

若是韩冈方才与甘穆争吵起来,必然会让人小瞧了去。但他连一句话都没跟教坊司的小吏多说,直接命人将之处置,这才是士大夫应有的作派。

参合了一出闹剧,韩冈与驿馆中的官员们的关系拉近了不少。早有人招呼韩冈坐下来说话。前两天,韩冈忙里忙外,把许多拜会和邀请搁置一旁,让人以为他是崖岸自高、目无余子的狂傲之辈。但现在,韩冈坐下来言笑不拘,品茗聊天,畅谈天下之事,不着痕迹的与人拉近关系,却让人不禁觉得他当真个好相处的朋友。

韩冈坐着大厅中与人闲谈,等着李小六和李信带话回来。

他还是决定还是乘热打铁,早早把周南脱籍的事情办妥。章惇昨天设宴邀请,拿着蔡确作为筹码跟韩冈做了交换。但这不代表韩冈能就此安坐在家,等着蔡确把事情办完。在官场上,首先就要学会做人。不论章惇那里已经许了蔡确什么,他这边都要把礼数做周全了。韩冈今天让李小六去请周南,就是为了由自己设私宴邀请蔡确,好将周南脱籍一事正式托付给他。

正如韩冈所期待,李信和李小六很顺利的将周南邀请到。一开始当着不让他们去找周南的许大娘,却因为甘穆的事,紧急被召去了教坊司內衙——尽管驿丞不愿把事情闹大,但有几十个文武官员盯着,他也不敢把甘穆直接送回到教坊司,而是送去开封府。以他的攀诬宗亲的罪名,少不得一顿好打。

蔡确就在开封府中,从头到尾听说了这一桩事。暗赞着韩冈手段,这一件事闹将出来,最多半个月就会在京中传播开,周南和韩冈的关系就挑明在世间。君子成人之美,周南的节烈深得人赞,不爱亲王而钟情于选人,更是能博得士大夫们的赞许。

眼下她要委身韩冈,谁会阻拦她脱籍?雍王赵颢都没那个脸皮。

韩冈品位不高,但正如章惇所说,思虑清明,眼光长远,而且在兵事上多有建树。冯京阻止天子召见他。换个角度来看,何尝不是天子对韩冈的看重,让冯京这个参政觉得有所忌惮。天下选人数以万计,有哪一个能像韩冈这样能惹出了天子和参政之间一番交锋。

这样前途无量的年轻人,蔡确当然还是选择多多亲近,有卖好的机会,更是不会放过。而等他接到了韩冈的邀请,更是发觉韩冈在他这个年纪的官员中,的确是难得的会做人。

当夜,就在驿馆中,韩冈宴请了蔡确。接过周南亲手奉上来的美酒,蔡确拍着胸脯,把她脱籍一事应承了下来。韩冈都已经把风头火势掀起,蔡确也只需顺水推舟。惠而不费,举手之劳,他蔡持正自不会推脱。

一番酒喝得兴头上,从外间忽而传来一声惊呼,“韩大府去职了?改由前任河东路都转运使刘庠接任!?”

