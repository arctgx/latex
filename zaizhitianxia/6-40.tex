\section{第六章 见说崇山放四凶(七)}

小殿中静了,宰辅们集中过来的视线一下就阴冷起来。

传话的小黄门浑身一颤,声音都哆嗦了起来,“东……东莱郡公呢?”

“韩冈已经回去了,今夜宿直宫中的是两府。”韩绛的声音很冷。

“多……多谢韩相公,多谢韩相公。”

小黄门连声谢,然后走得飞快。

小黄门走后,殿中依然保持着安静。

苏颂捻着胡须,还真是意外。

无意间,听见坐在旁边的章敦咕哝着,“司马十二不冤……”

从西窗外望出去,苏颂暗道:‘是不冤。’

……………………

王中正过来的时候,韩冈已经准备睡觉了。

一天的折腾下来,他也累了。有两府宰执一同镇守宫禁,有王厚、李信控制京城兵马,不可能再闹出什么大事。

“入宫?”

听到这个词,韩冈也愣了。都打算提早点去睡觉,谁想到这时候还派使者过来。

这到底是怎么回事?要是想要自己留镇宫中,前面还在崇政殿上的时候说句话就行了。

难道说又出了什么事不成?韩绛他们都解决了不了?

总不会是张守约那边病情有变。因为重伤不便移动,老将军做完手术后就在宫中安歇,韩冈出宫前还顺便看了一下,情况还算稳定。可就算病情有变化,也不该为此招外臣入宫。

“才出宫,怎么又要入宫?”王旖担心的问着。

“为夫怎可能知道。不过是王中正来,宫里面不会有什么变化。”韩冈摇摇头,心中亦是不解。

若非这回是王中正亲自过来传诏,韩冈绝不会放心入宫,转头就会去通知王厚。让已经兼任皇城司管勾的王厚先入宫,确认了郭逵的动向,控制住宣德门,韩冈他才会进宫去。

“当不会是什么大事。”韩冈对妻子说道。

“不是什么大事,也要半夜招官人?相公、枢密还都在宫中。”

“谁知道呢。”韩冈长身而起,“得快点了,不能耽搁。”

“是啊,太后有招。”王旖哼了一声,让人去取韩冈才换下的官袍。韩冈也命人出去,请王中正稍待片刻。

取了衣冠来,王旖过来服侍韩冈穿戴。放下了心,就不免抱怨起来:“都要睡下了,大半夜还折腾人。”

“莫说就要睡了,便是衣服裤子都脱了,人也睡下了,也得起来入宫。就是家中正着火,该放下也得放下。”韩冈叹了一口气,对王旖苦笑着:“谁让拿了这份俸禄?”

王旖嘟了嘟嘴,还是过来帮着韩冈整理穿戴,将袍服一件件的套上身去。

将内外袍服全都穿好,王旖拿起压制衣襟的方心曲领,踮起脚尖,要围在韩冈的脖子上。

韩冈轻轻压住了布带,对王旖道:“想起来了,又不是上朝,穿朝服就闹笑话了,换公服就行!”

今天是大朝会,韩冈穿的是朝服——貂蝉冠、罗袍裙、白花罗中单、大带,以及方心曲领。而日常上衙和陛见是穿的都是公服,紫袍、金带和金鱼带就够了。

“早不说。”王旖白了韩冈一眼,放下了手中的饰物,唤了外厢听候使唤的婢女:“快去取寻常穿的公服来,可别让太后等着着急。”

韩冈清了清嗓子,“是要快点,不能让太后和两府诸公在宫中久候。”

…………………………

程颢讲学的寺庙中安安静静。

正是做晚课的时候,寻常时,就算到了半夜,寄寓此处的学生们也不会放低辩论的声音。

可今夜,一群士人如行尸走肉般坐在讲学的课堂中,没有大一点的声息,只有偶尔响起的窃窃私语,如灵堂守夜,鬼气森森。

或许当真是在守夜了——

——为道学。

游酢想着。

程门的弟子在操行上一向被二程耳提面命,故而时常为士论所赞。除了当初在国子监中与教授新学的教授们闹了一场之外,一直都是德行的典范。从来没有说哪个弟子犯了事,牵连到学派上——在律法上也没有如此牵连的道理。

如果是学术之争,使道学受到朝廷的打压,那在士林中,反而是增光添彩。

可如今道学门下的刑恕,却是掺和进了谋逆大案中,这事情就两样了。

刑恕日常结交广泛,好友无数。横跨新旧二党,从宰辅家的子弟,到还没进入国子监的士人,他都有说得上话的友人。在同窗之中,几乎没有跟他的关系恶劣的,多年来诗文往来成百上千,就是游酢本人也曾经与刑恕通过一两次信。

一旦刑恕家里给查抄,只凭这些信件,就能让许多程门弟子从此毁废终身。而程颢、程颐,更是逃不了一个授徒无方的罪责。

二程一倒,道学又如何能够存世?

“刑七怎么就能做出这等事?!”

“当初就看刑恕此人险恶,只是其恶不彰,故而才与其敷衍。”

“刑恕一向多诡诈,欺世盗名,多少人为其所瞒过,谁知道他竟然如此悖逆不道。”

学堂中有人窃窃私语,渐渐的,说话的人多了,声音也稍稍大了起来。

游酢看过去,都是平常奉承在刑恕左右的门徒,现在就在撇清关系了。

过去他们可不是这样对待刑恕的。

早年韩冈在张载门人中所受到的期待,就是刑恕在程门弟子中收到的期待。

当年在韩冈以格物致知之说,重举气学大旗之前,他在张门弟子中,一直都被当做是十几二十年后,气学在朝堂上的依靠。是未来的支柱。虽然学问不佳,没多少人认为,他能在学术上有多大的成就,起到什么样的作用,但足可以做一个称职的护法。

而刑恕此前游走于西京显贵之间,在京城又是宰相家的座上宾,从上到下,人面广,人缘好,到处都有朋友。谁都认为他的前途远大,虽然做不到钻研经义,成不了饱学鸿儒,但足可以成一名护法。

程门想要发扬光大,刑恕这样前途远大的弟子,就显得尤为重要,绝大多数的二程门人,都与其相友善,那些目的不单纯的学生更是对刑恕巴结奉承,可现在刑恕一犯事,全都变了嘴脸。

“韩玉昆曾求学于先生门下。想必不会坐视先生受到牵连。”

“对。今天就是韩相公亲自拨乱反正,有其在朝堂上主持,必不会让先生受辱。”

游酢皱了皱眉。

寻常时,他们在私下里好像没少攻击过气学和韩冈,但今天立刻就把过去的言论丢到了葱岭西面去了。

“不必多说了!”程颢不知何时出现在学堂门前。一贯和善,接人待物如同春风一般的前任帝师,此时却是声色俱厉,:“和叔犯法,自有刑律在!朝廷自会依律审判。尔等即无人参与其逆行,又何须担惊受怕,求于他人?若当真犯了大律,求到别人头上又有何用?”

程颢不是惯于训斥人的,但严词厉色的几句话,让好几人头都低了下去。

“先生说得是。”吕大临跟着程颢一起过来,他从后面站出来,“相信朝廷不会让无辜者受冤屈。”

“正是如此。”游酢点了点头。

不管是不是已经引咎辞职,程颢终究还是做过赵煦的老师。赵煦若逊位,程颢同样损失惨重。从这一点来说,刑恕的确是背叛了程颢、背叛了程门道学。

程颢曾为帝师,刑恕却在谋逆,这岂不是欺师灭祖?要说程颢参与刑恕谋逆,从情理上说,就说不通。肯定牵连不到程颢头上。而以程颢的为人,只要没有真凭实据,他也肯定不会允许有人将他的学生都牵连进去。若是程颢求到韩冈那边,更不会有事了。

可几名僧人连滚带爬的跑来,其中还有住持和尚,见到程颢,就叫了起来,“伯淳先生,伯淳先生,外面被官兵包围上了。”

堂中一下就乱了,“怎么会有官兵?”

“肯定是来抓人的。”

“谁之前跟刑恕有勾结?!”

“肯定有人。”

“慌什么!”吕大临怒喝一声,转身对程颢道:“先生,学生去看一看。”

堂中惶惶不安,游酢等几位弟子过来扶着程颢坐下,见他们不为所动,一群程门弟子这才稍稍安定了一点。

片刻之后,一人跟着吕大临回来,在程颢面前行了礼,“小人戴光,奉王上阁之命,前来护卫大程先生。”

所谓王上阁,应该是王厚。而王厚背后的靠山究竟是谁,这是不用想的。至于王厚怎么指挥起皇城司的人,也是用不着深究。

“皇城司的人是来给先生看门的!?”

游酢一下就把握住了重点,看门可是有好几种。

皇城司的人给帝师看大门也能说得过去。不过换做另外一个角度来看,未尝没有让他们将功赎罪的用意在。既然刑恕参与了谋逆,程门弟子中未必没有第二个刑恕。

但没过多久,游酢就看到另外一位得意弟子杨时过来了,蹲在炉子边,双手烤火。

杨时之前已经先走一步,现在却又回到了这里。

“怎么回来了?”游酢问道。

“方才从御街那边过来,看着韩三出宫,却没看到两府诸公出来。所以就回来知会一下。”

“这么可能?今夜他该留在宫中才对。”游酢惊讶莫名,韩冈今天立下了泼天的功劳,理应与宰辅们地位相当。

吕大临冷道:“若无韩冈,便无今日之变。他怎么能留在宫中?”

“都回去睡吧。”程颢不想听这些,赶着学生离开。在庙中寄寓的就回房间,在外租房的就回各自的住处。

游酢跟着吕大临等几名同学做一路走了。

天寒地冻的夜风中,吕大临问着游酢:“朱雀门可能出不去,定夫今日到愚兄家中小歇如何?”

游酢的住处在外城,若是要出去,少不了要经过朱雀门,但今天的情况不行,他点了点头:“多谢与叔兄,如此小弟就叨扰了。”

吕大临和游酢相互客气了几句,就抵达了御街。

这是一队人马正从前面的巷口上了御街,然后转向北面过去。从吕大临和游酢这边,能看清提在亲随手中的一盏盏玻璃灯笼,玻璃灯盏上的‘韩’字字样,直直的映入眼中。

吕大临一脸的困惑,“怎么又入宫了?”