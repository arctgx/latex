\section{第12章 共道佳节早(一)}

时近腊月,京城中越来越有节日的气氛了。

不但,市井街巷中的行人为了即将到来到年节忙忙碌碌。连朝堂上的气氛,也是变得跟不断响着鞭炮声的元日一般火爆异常。

这段时间,枢密院和御史台,因为博州军库赃罪一案起了争执,最后却将政事堂拖下了水。

一开始是御史台控诉博州军库一案,枢密院定罪不当,应当将此案交由博州本州衙门重审,而处置此案的枢密院详检官刘奉世,却是偏袒着他在此案中有瓜连的亲戚,却让纠察刑狱司去定案,硬是要坐实博州官吏此前错用刑律之罪,此罪一定,当然就没有改审的权力。

为了这一件事,枢府和乌台两边公文往来一阵后。御史台首先按耐不住,将战线拉长,新近上任的权监察御史里行张商英,为了展现自己的能力,开始攻击枢密院中老吏任远,恣横私徇等十二事,并弹劾枢密院上下勾连,结党庇之。

王韶本不想掺和这些烂事,刘奉世、任远这些官吏徇私枉法的事,他也看在眼里,都滚蛋对他更有好处。且王韶是因边功而得入枢府,在京中根基不稳,最安稳的策略就是凡事不出头,做好手上的这一摊子事,维持住自家在西事上的发言权,慢慢营植自己的势力。做过几年枢密副使,再外放几年边帅,五十上下的时候,便可回朝登上枢密使的位置了。

只是御史台不仅仅是揪着任远之事不放,不知怎么就有传言称,御史台中有人向天子上书,请求将枢府的事权交给中书。

虽不知其中真伪,但事关密院权柄,就算是传言也必须做出反应。所谓是可忍,孰不可忍。枢密院上下这次是同仇敌忾,王韶即使不愿,也不得不站到了吴充、蔡挺这一边。

原本王韶在河湟时,被执掌枢密院的文彦博三番四次的刁难,恨不得让王安石兼任了枢密使。但现在换作他担任枢密副使,却难容东府侵犯西府之权。

因为这个传言,西府中的三个正副枢使,从两天前开始,就一起不赴院中值守,并把大印送到了中书去。

不是要事权吗?那就交给你好了。

枢府大印,政事堂当然不敢接受。

王安石被将了一军,说实话,他这也是糊里糊涂的便挨了一刀。枢密院和御史台的意气之争,莫名其妙就变成了东西二府权柄谁属的交锋。为了在天子面前自证清白,无意总揽大权,王安石不得不抛弃了张商英这个刚刚由章惇举荐上来的御史。

经此一事,王韶和王安石的关系虽不能说是破裂:王韶昨天还连夜还写了信,今天一大早就遣长子送去了相府,向王安石道歉,并述说自己的苦衷。但实质上,王韶和王安石之间已经有了疏远的迹象——其实就算没有此事,王韶和王安石一为执政,一为宰相,本来就不便来往的太过密切;加之王韶只求开边建功,从来都没有认同新法的想法,分道扬镳,可以说是不可避免的。

虽说对跟王安石渐渐疏离,早是有着心理准备,可王韶这两天还是有些不痛快。毕竟今次是被人拿去当了枪使。会跟东府闹起来,也并不是为了自己的利益,他的心情当然不可能好。

而且今次之事,很明显这是有人刻意在转移视线。将政事堂拉下了水,把一开始的刑案归属权的争夺,变成了两府之间的政治.斗争。为了维护枢密院的威权,御史台也只能吃上一个哑巴亏了。

朝堂上的政局变幻莫测,也让刚刚侧身朝堂的王韶叹为观止。一句流言不但让吴充脱身出来,而且还反手给了政事堂和御史台一棍子。要是没有这一档子事,因为包庇胥吏任远的行为,吴充应该下台,而他的亲信枢密院详检官刘奉世也别想有好果子吃。

不过在这一件事中,也能看出了天子的倾向,以及他跟王安石的关系了。若是放在熙宁二年、三年的时候,王安石尽管连宰相都不是,枢密院若敢这般欺到政事堂的头上,王安石能当即撂挑子给天子看。但现在,王安石已经不便也不敢这么做了。

身在京中,王韶也知道王安石的确不易。今次两府一台的三方之争,王安石吃了个暗亏,让吴充更加稳坐枢密使的位置。而在市易法上,皇城司越来越多的活动迹象,已经表明天子并不再彻底的信任王安石送上来的报告。就在昨日,听说天子还质问王安石,为什么最近京中的水果涨价了,外面的行商都在抱怨,市易务转卖水果,这般行事是不是太繁细了?

虽然王安石当时已经长篇大论的顶了回去,但王韶听说此事后,也是想上本与天子说上两句。

繁细?市易务就是做这个事的,怎么叫繁细?

天子连有司内部的事务都干涉,才叫做繁细!

什么叫‘元首丛脞’?《尚书》中的这句话,就是不要让天子不必去管这些琐碎的细务,只需主持着大方向上的战略就够了。而天子注重细务,忽视大略,就会‘股肱惰哉!万事堕哉!’——做臣子的会懈惰,如此万事都会堕废。

如今的天子啊,勤勉是不必说的,聪慧也是实实在在,就是什么事都想抓到手中的这种性子,跟太宗皇帝一脉相承,让臣子无所适从。

王厚新近转迁三班院,他今日从衙门回来时,便先去了书房中。请安问好后,又对王韶道:“外面的吃食好像又贵了几分,一斤林檎果都十八文了,不知道是不是又有人在捣鬼。”

“年前物价贵上一点是很正常的,但不可能再涨了。”王韶虽然不涉家计,可作为一国执政,对外面情况还是很了解,“有汴渠运来的诸色南货在,明春之前,京城的物价怎么都不会再涨。”

十月末黄河上东,汴渠随之封口。但在这之前,依靠均输法而得到了对汴河南北货运的控制权,通过汴河运来的货物大半掌握在市易司手中。靠着这些商货,足以打压下京城的物价。

“但到了明春就不行了,库中存货清空,而南方的新货一时间又运不上来,控制着其余诸路货源的京城豪商们,必然会一齐动手。”王韶微微冷笑。

只要对京城历年来的物价波动情况稍做了解,得到这一点结论很容易。王韶相信王安石、吕嘉问他们不会没有准备,就是不知道他们有什么后手了。

“其实市易法也不坏。”王厚坐下来跟父亲说话,“过去各地进京商货,全为各家行会行首们所把持,但凡不肯将货物贱卖给他们的,在京中连间仓库都租不到。现在可以卖给市易务,再由市易务转发下面的商号,真正吃亏的也只是各家行首而已。”

“凡事要看长远啊……”王韶意味深长的说着,“市易务新创的时候,必然有一番振作,人人勤谨,不敢有丝毫懈怠,凡事必得尽力做得最好。但过了一两年再看看,什么千奇百怪的事都能出来。除非能不断修订整改,最后形成能维系数十年的条贯,这样才能算是大功告成。”

王韶这是经验之谈,‘鲜克有终’的事他见得也多了,他看了看儿子,忽而笑道:“二哥你旧年读书,多少次发狠说要从此用功,但哪次不是一开始用心几日,后面就放羊去了?”

王厚脸色一变,事情说着说着,怎么都扯倒了他的头上,很是尴尬的讪讪笑着,“孩儿不是读书的料,坐下来也看不进去。要是有大人读书时的一半耐心,也就去考进士了。”

“那你在武职上好好做吧,只要记得凡事要以一贯之。”王韶叮嘱了儿子两句,又将话题转到了市易法上,“今次的市易法掀起的风浪太大,还不一定能等到一两年后。别忘了,站在那些货殖之徒背后的,都是些什么人呐……”

王厚默默颔首,他当然知道站在京城豪商们背后的究竟有哪些人?只看隔三差五就从宫中传出小道消息,说两宫哭诉,欲费市易,而天子坚持不允。后台究竟是谁,已经很明显了。只是又有谁能将之解决?

疏不间亲,骨肉至亲时时刻刻都在耳边说着,总有挡不住的时候。天子不断加派皇城司的探子,新任管勾皇城司的蓝元震不断报上去的细碎小事,让王安石都觉得头疼。

市易法最后的结果,王韶总之是很难看好的。

父子两个正相对而谈,一阵脚步声急匆匆到了书房门前。王韶皱起眉,他领军日久,最是看不惯不稳重的行为。

敲门声响了两下,王厚上去拉开了门。出现门外的一张脸上,喜色难掩。王家这名仆人急急的对书房中的两名主人道:“相公,二郎,韩官人已经到了,现在就在门外面。”

“什么?!玉昆到了!?”王厚惊喜的叫了起来。

“本来以为能更早一点,没想到还是拖到了快到腊月了。”王韶一连声的催着王厚,“二哥,你还不快去将玉昆给请进来!”

