\section{第五章 九州聚铁误错铸(三)}

“太尉,天使很快就要到了!”胖粮官却很沉稳的提醒着。

啪嚓一声,种谔一下拧断了交椅的扶手,从牙缝里迸出声来:“将在外,君命有所不受!”

“太尉,下官只是来传话的。”胖粮官顿了一下,又着重强调道,“是奉了李运使的吩咐!”

“哈哈哈……好个李稷!”猛然间惨笑起来的种谔,宛如走上末路的孤狼,“真够狠的!真够狠啊!!”

“五叔!”种建中叫了起来,“小侄回去说服李运使!”

“有屁用!”种谔回头冲着侄儿吼着。

“还是有用!”种建中毫不退让的在种谔的面前站直了身子,与他对视着:“李运使是功名中人,只要能让他相信五叔可以给他带来泼天的功劳,完全可以说服他把圣旨顶回去!”种建中毫不在乎在李稷亲信面前这么说话,他咬着牙,“这时候,只要能顶住圣旨,打下夏州、宥州后什么都有了!”

“让李稷顶着圣旨没什么关系,可他不敢得罪其他人啊。这么多人都伸着双手想要功劳,京营的三万废物上阵没屁用,但在京中使坏可是拿手啊!李稷已经打定主意了,要不然他让粮官上来送信做什么!?”

种谔指着来传话的粮官,让这个胖子缩着脖子不敢多说一句废话。李稷派他来,等于是明摆着说粮草别指望了。

如果只有圣旨,李稷还可以壮着胆子顶回去;如果只是其他几路文武官员的嫉妒,李稷也可以不在乎;但同时得罪天子和其他几路的将领、官员,李稷无论如何还没有疯到那个地步。这样的情况下,只有选择逼种谔退军。

“都想着分功,尽知道扯人后腿。”种谔回头嘶吼,声嘶力竭:“这一仗他娘的还没赢呢!”

种谔的吼声让外院起了一阵骚动,一群将校拥到了门口。

“现在回师,到一个月后再出兵,是打算要我们在六月过沙漠吗?!”种谔愤怒的吼叫着。

“回师?”众人几乎不能相信自己的耳朵,“怎么能回师?!”

种谔捶胸顿足:“此时还是雪水,天还不热,尚能在大漠里走。回去后再出来,可就是他娘的要在六月酷暑的时候,穿过没有任何遮拦的瀚海了!”

“河东路、鄜延路的人马过不去,就凭高遵裕,他能打下灵州城?!”

“鄜延路完了……”

“鄜延路完了啊!”

瞪起的眼睛里燃烧着熊熊的怒火,一道殷红的血痕从种谔的嘴角溢了出来,染红了已经斑白的胡须。

看着种谔这般模样,门外的众将人人面色惨然。但要让他们出来劝种谔继续进兵,至少在外人尚在场的情况下,没人敢出头。

种建中这时候反而冷静了下来,这件事已经瞒不住下面的兵将,而在全军只剩几天粮秣的情况下,也不可能继续指挥他们攻打夏州。

他在种谔身边低声提醒道:“五叔,军心已乱,还是退兵吧!”

……………………

“种谔退兵了?”

只用了三天的时间,银夏的最新战况就传到了兴庆府中。

可不论是梁氏兄妹,还是嵬名荣、嵬名阿吴、仁多零丁、叶孛麻等重臣,都不敢相信自己有这么好的运气。

西夏如今的高层人物再三确认,负责情报消息的梁乙逋则十分肯定:“路上跑死了二十匹好马,三个方向的哨探同时传递消息回来,不会有错。”

确认了消息之后,疑问随之而生。梁乙埋疑惑道:“是不是有什么诡计?”

“都这时候还要什么诡计?”仁多零丁耷拉下来眼皮掩不住眼中的精光,“种谔全力攻城,夏州守不住三天。他会退兵肯定是后方出问题了。”

代替被囚禁的儿子坐在御榻上,梁氏问着:“能有什么问题?”

仁多零丁皱着眉头,即便以他之智,也想不通后方出了什么问题,能逼着种谔要在大捷将至的时候退军。这完全不合常理。

“会不会因为断了粮?”叶孛麻猜测道。

“罔遇厄他哪里有这本事,将银州、石州的粮草都扫空了,不给种谔留下一点?”嵬名阿吴反问。

在打探到宋人的战略之后,由于上上下下都缺乏御敌于国门之外的决心,银夏之地几乎已经给放弃了,精兵强将全被抽回兴灵,只剩最前沿的寨堡还留有一部分守军。这样的情况下,名臣亦要束手,何况一向没有什么表现的一干庸碌之辈。

嵬名荣也跟着补充:“且要是当真缺粮的话,种谔就不该浪费时间去攻打弥陀洞,可见他手上的粮食很充分。”

叶孛麻的脸色难看起来,嵬名家是宗室,但叶家也是外姓的大族,嵬名阿吴和嵬名荣还真是一点面子都不给。

“呃……”梁乙逋突然出声,将殿中所有人的视线都吸引到自己的身上后,他说道:“其实有消息说,种谔此次出兵是鄜延路自作主张的行动。当初得到的宋军出兵时间,其实是正确的,只是种谔贪功没有理会。”

“这又如何?”嵬名阿吴问道。

“所以会不会是因为种谔私自出兵,惹怒了东朝皇帝,所以被召回去了?”梁乙逋说出了自己的猜测。

殿上一片寂静,而下一刻,寂静便被一阵狂笑声砸碎。

嵬名阿吴毫无顾忌的放声大笑。他丝毫不在乎梁氏的脸面,没有他所代表的宗室支持,梁氏在囚禁了儿子后还想稳定兴庆府中形势,只会是做梦。何况梁乙逋的猜测的确是可笑。

他收住笑声:“东朝皇帝的圣旨,顶回去有什么大不了的?种谔都要打下夏州了,只要占了夏州,罪名都是功劳。”

嵬名荣冷笑道“当年的罗兀城,要是种谔不遵旨退兵,横山早几年就给宋人占了。吃过一次亏,他还能犯第二次蠢?”

仁多零丁叹道:“就算是种谔之前是自作主张的出兵,可他一路进展顺利,当是让泾原、环庆、秦凤、熙河即刻出兵配合,何至于将种谔调回去。”

梁乙逋的脸一阵青一阵白,他哪里想到随口一句,就让嵬名家的人给抓到了。梁氏在上面板起脸,就连梁乙埋看儿子眼神都多了几分恨铁不成钢的羞怒。

“其实也说不准,梁副枢的猜测也许是对的。”一直站在最下首处的李清突然打破了长久以来的沉默。

太后梁氏神色一动:“李卿,此话怎讲?”

李清站出来,先行了一礼,依然是汉人的礼节,让梁氏看了,就有两分不喜,梁氏反对汉礼汉俗,曾经下诏严禁朝臣用汉人服饰、朝堂上也禁用汉家礼仪。但李清是汉人,而且还是在帮梁乙逋说话,只能先忍了下来。

李清自从他因为在囚禁秉常一事立下功勋之后,重臣议事时就有了他一个位置,虽说比仁多零丁的侄儿保忠还要靠后,但毕竟进入了重臣的行列。不过李清对自己的位置摆得很正,却从没有主动发言过,今天还是第一次破例。

他环顾殿中,朗声道:“东朝如今国势强盛,又趁辽国内乱之际来攻我大夏,朝堂、军中,无视我等为釜中鱼,俎上肉。故而从河东到熙河,人人都想着抢一个头功。要不然,就不会有眼下六路齐发的战法。”他顿了一下,让众人有时间思考,继而接下去道:“种谔提前出兵,等于是抢了其他几路的功劳。其他几路当不是不想出兵,但他们连粮秣还没有筹备完成,这就只能设法将种谔拖回来了。不光是东朝皇帝一人的功劳,还有陕西各路将帅相助。”

“没错,正是这个道理。”梁乙逋连忙点头。李清在殿上帮他说话,让梁乙逋心中多了几分感激。他还想继承父亲梁乙埋的位置,人望和脸面是万万丢不得的。

梁氏兄妹,嵬名家的两位大将,还有仁多零丁、叶孛麻都在深思一阵之后,承认李清说得的确有几分道理……不过也仅仅是有几分道理而已。

“全都是猜测。又不是种谔肚子里的蛔虫,谁知道他是怎么想的。”嵬名荣依然嘴硬。

但梁乙埋却不可能不站在儿子一边,“想想倒是在情理之中……”他叹了一口气,“景询若还在,倒还能问问他了。”

“且不管究竟是什么原因,种谔回军让我们缓了一口气。”仁多零丁道,“至少可以确定,一个月之内,鄜延路的兵马很难再出征。”

“希望能如此。”梁氏叹了一声。她才三十多岁,但眼角、额头都有了明显的皱纹。秉国十余载,在得到权力的同时,付出也一样很多。

退回班列的李清头略略低了下去一点。

换作是十年前,一国之主哪里会说出如此弱势的话。但李清更清楚,西夏已经离灭亡不远了,就算没有宋人的攻击,也维持不了几年了。

这几年西夏国内的形势岌岌可危,为了维持国用,旧年积攒下来的一点老本几乎都给掏空了。

而且西夏国中的财政状况本来就是入不敷出,从立国时开始便是如此。如果没有通过与宋人的回易、劫掠和岁赐来填补亏空,西夏国内的统治根本维持不下去。

