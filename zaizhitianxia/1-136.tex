\section{第48章 斯人远去道且长(四)}

【第三更,本卷最终章。继续征集红票和收藏。】

虽然张、程二人与章惇互为政敌,但并不认识没有官身的章俞,直到听了韩冈介绍,他们才惊讶的发现面前的这位甚有风度的富态老者,竟然是传说中私通岳母的败类。

张戬勃然作色,当即就要发作出来。程颢却拉了张戬一下,提醒他不要乱发火,张戬心中怒意难消,但被程颢阻着,却也不得不狠狠的回头盯了韩冈一眼。

章俞私通岳母,章惇私通族叔小妾,父子二人的品行皆是卑下不堪。程颢张戬都是虔信儒学,最重纲常伦纪。对于章俞这等悖人伦的行为,他们深恶痛绝。但两人都抱着君子隐人之恶,扬人之美的想法,并不在韩冈面前提及此事,只是没想到韩冈会跟章俞走得那么近。

韩冈在关西道上救了章惇之父的性命,张戬和程颢也是知道的,也清楚因为这个原因,韩冈多次受到章惇的宴请。虽然明白章俞是感念韩冈和刘仲武的救命之恩才过来送行,但张戬还是很不高兴,而一向性格温文尔雅的程颢,也不免皱眉。

亲眼见着章俞和张戬程颢之间紧绷的气氛,韩冈不由得庆幸,幸好王安石那边没人来送行,章惇还好解释,王安石本人身份贵重也不会来,但若是吕惠卿、曾布,或者是王旁来了,那麻烦真的就大了。

送行的事还算小,若是他给变法派支招的事给捅出来,那就是把张戬、程颢往死里得罪了,不用说,肯定会臭了名声。

不过他出的那几条绝户计,王、吕等人都不会帮他宣扬的,韩冈可以确定,他们甚至不会承认有这几条计策存在,只会说是每一条每一款都是为了利国利民。这关乎他们的形象和声望,对政治人物来说,没有比这点更重要了。

公布韩琦等人的放贷取息之事姑且不论,若是改动青苗贷之名,为低层官吏加俸目的是为了打击反变法派的这件事,传到了天子的耳朵里,赵顼心里会怎么想?即便是过去韩琦吕公著司马光他们那一派攻击新法,攻击新党成员,依然要在脑门上写下忧国忧民一片公心几个字的。

党争之事可以做,但不可以说,这就是潜规则。不能像欧阳修那么糊涂,受了吕夷简的激,写出个朋党论,说小人可结党,君子也可以结党。拥有同样的目标,拥护同样的纲领、组织完备的政党只在后世才有,放在此时,但凡党派,无一例外都不过是个争权夺利的利益集团而已,即便现在不是,日后也肯定是。所以范仲淹才悲剧了,没有觉悟的欧阳修也悲剧了,到现在一身脏水都没洗干净。

所以韩冈很安心,能带着笑在两位师长和章俞之间做着缓冲。正如早前程颢训诫韩冈那样,行事说话不可悖于人情,即便章俞过去行为不端,但他来为两名救命恩人饯行却是没有错的,是知恩图报的行为。张戬和程颢都不能为此发作,更不能赶章俞走,毕竟他们只是韩冈的老师,而旁边还有一个刘仲武。

张戬苦苦忍耐,不想在弟子面前失了身份,程颢的性子则洒脱一点,苦笑两声也就放开了,幸好两人算是韩冈的尊长,不必送韩冈到离城十里的郊外,出了城门,就算到点了。

就在城门外,找了家干净清爽的酒店。几人在二楼坐下。让店家上了酒菜,各自劝了几杯酒。皆是浅尝即止,没有多喝。

酒过三巡,章俞执杯问道:“玉昆在京师住了也有一个月了,如今即将离京,不知可又不舍?”

韩冈想了一下,回道:“东京富丽繁华,甲于天下,却不是宜住人的地方。”

“是不是因为人太多,住的不习惯?”章俞笑着问。

“……也许是吧。”韩冈怔了一下,然后点了点头。虽然他过去千万级别的城市也待过许久,那些百万级都排不上号,但在他如今的这个身份里,他所经历的百万人口的大城,只有东京开封。

“怕不全是!”章俞像是看透了韩冈的含糊其辞,追根究底的问着。

“若是能多听得两位先生的教诲,那住哪边都是无所谓了。不过还是心有挂念!”

“挂念着秦州的事?可是哪家的好女儿?”章俞哈哈笑道,“难怪玉昆你会拒绝王大参的推举。要是你点一点头,就能在中书里做事了。”

韩冈又是一怔,转念一想,忽然明白了章俞的用意。再一瞥被惊到了的张戬、程颢,心中暗喜,章俞这忙帮得真是好。他谦虚的笑道:“跟儿女私情无关,只不过是想着做事全始全终罢了。”

程颢欣慰的点头笑了起来。张戬也脸色稍霁,道:“平常人都盼着能在东京任官,玉昆你却往外走。不受官禄之诱,不枉你平生所学。”

“同为天子治事,本不该分京内京外。韩冈也是按着先生们过往教诲行事。”

韩冈和章俞一搭一唱,让饯行宴上的气氛为之稍缓。

对韩冈的本心而言,东京虽好,却也不是久留之地。他先前已煽风点火,现在便得隔岸观火。在京城这座舞台上搅风搅雨,过了把瘾之后,韩冈乐得离开接下来的狂风暴雨远上一点,躲在秦州挣自己的军功。

在王安石稳固自己地位的这段时间里,王韶必然能得到最大限度地支持。只要没有人扯后腿,河湟开边的难度其实并不高,毕竟依照王韶《平戎策》中的计划,他的主要任务,不是征战,而是收服。即便动起刀兵,也是以杀一儆百为目标。

韩冈还记得有一次与王韶谈起过历朝历代的开边拓土,炎汉四百年里,韩冈对卫霍敬佩有加,对班马赞不绝口,但当时王韶却说这些都不差,但他最羡慕的却是司马相如。韩冈很奇怪,写些诗赋勾引寡妇的文人有哪里值得羡慕?问为什么,王韶则叹了一口气,答道‘无人作乱’。得到提示,韩冈从记忆中找到司马相如的传记,也不得不苦笑点头。

司马相如奉使持节定西南夷,‘至蜀地,蜀太守以下郊迎,县令负弩矢先驱,蜀人以为宠。’对比上司马相如的所受到的拥护,王韶的境遇就可悲得很了。至少韩冈就无法想象,王韶到秦州,李师中领着一众官吏出城相迎,窦舜卿、向宝等人跨弓持弩为王韶打前站,秦州父老皆认为他们这么做是件荣耀之事,会是个什么模样!这实在太疯狂了。

真是人比人得死,货比货得扔,但之后的哪一朝又能跟充满大无畏的开拓精神的汉代做比较?即便是唐朝,在安史之乱后,也成了一个任人蹂躏的小姑娘了。哪像汉朝,即便到了军阀混战的末年,照样控制着边境的领土,追着乌桓、羌人这些异族打,‘国恒以弱灭,而汉独以强亡’本就是说了这个道理。

自古送别皆以诗赋表离情,张戬和程颢却无意如此。韩冈本不擅诗词,他们也不会让韩冈难做。饯行宴后,他们对韩冈殷殷的一番叮嘱,便与他举手挥别。作为官员,今日己送人,明日人送己,都是常事,再无半点小儿女态。

韩冈冲着两位师长一揖到地,便翻身上马。刘仲武等了一阵子,见韩冈终于过来,便等不及立刻再次动身。章俞和路明还要再送一程,按他们说法,要到城外十里再回头。

只是没行多久,突然一个小女孩挡在了路前,冲着韩冈他们喊着:“可是秦州的韩官人?”

韩冈很诧异的看着小女孩:“我就是韩冈!你是……”

“这不是周小娘子身边的小女使吗?”章俞一下叫破了小女孩的身份,又转过来对韩冈低声笑道:“恭喜玉昆了。”

“小婢墨文,我家姐姐想跟韩官人说两句话。”墨文认认真真的说着,韩冈顺着她手指的方向望去,只见就在不远处,大树旁,马车边,一个俏丽脱俗的身影正静静地站着,一双如含情秋水的双瞳也定定的望着自己。

韩冈向章俞他们说了抱歉,便下马朝周南那边走去。走得近了,韩冈便看清了在周南的脸上,有着欣喜、羞涩,还有显而易见的紧张。

“周小娘子是来送韩冈的吗?”

韩冈的单刀直入让周南猝不及防,擅长歌唱如百灵鸟般的她一下变得笨拙了起来:“……是……是来见,不,是来送官人。”

“那就多谢小娘子的一番心意。”

“不……”周南很大胆的抬起头,一双本是柔波隐隐的双瞳变得坚定,与韩冈对视着,“小女子不想送官人,只望能常伴君侧。”

这下轮到韩冈发怔了。最难消受美人恩。说起来他对周南也很有好感。一个在物欲横流的污秽场所,还能自保清白的女孩子,的确很让人佩服。虽说有律条规定官妓禁止陪夜,只能局限于陪酒和歌舞,但实际上官妓陪夜的事从来不少,而周南的这份坚持更显得难能可贵。而且她又喜欢上自己,韩冈怎么能不心动?

但韩冈却不知道,周南的这份心意能维持多久,她又能在教坊司这个污水缸保护自己多久?韩冈都不能确定,也无法确信。

周南站在车边,静静的等着韩冈的回答,身子却在微微的颤抖。女儿家的心事都给摊在了阳光底下,就像是在公堂上等着最后的判决。

韩冈的沉默,让周南的心一点点的沉了下去。一阵酸楚涌上心头,哀恸欲绝,一颗颗泪珠从脸上滑下,落在了地上。周南急转过身,掏出汗巾擦干了泪水。返身从车上拿出一个小包裹,这是她本要送给韩冈的饯行礼,勉强笑道:“小女子蒲柳之姿,的确不足以侍奉君子。这是给官人的饯行之物,只代表小女子的一点心意,还望官人勿要拒绝。”

看着周南强忍着苦楚而露出的笑容,韩冈怜惜万分。他轻轻摇了摇头,也没辩解,只从怀里掏出一把匕首来。拔刀出鞘,刀身上银光闪烁。这是当日王韶赠给韩冈当饯行礼的银匕首,本是在古渭寨时,蕃人送给王韶的礼物。韩冈将之带在身上,却是因为水浒传看多了,怕蒙汗药、砒霜什么的,用来试毒。

周南疑惑不解看着韩冈。却见韩冈将匕首在左手掌心一划而过,顿时拉住一道浅浅地血口。周南猛捂住嘴,将惊叫压在喉中。

韩冈将刃尖上带着一点血丝的匕首递过去,道:“请小娘子再等三年,三年时间,我也该能回东京了,也该有足够的实力让小娘子得脱苦海。到了那时,若小娘子心意仍如今日,韩冈必不负你。”

看着递到眼前的匕首,周南脸上又滑下了泪水,却不是因为伤心,只是当她看见韩冈手上那个浅浅的伤口还在渗着血,立刻忘记哭泣,手忙脚乱的拿着自己的汗巾帮韩冈包扎起来。

周南包扎伤口的手艺比甘谷疗养院里那些粗使打杂的民伕还要差了许多,长长的汗巾歪七扭八的卷着伤口,倒真的把血止住了,不过这也是伤口本来就不大的缘故。

韩冈回头看了看在官道上静候着的同伴,对周南道:“行程不能再耽搁了,今天还有几十里路要赶。南娘你也不必多想,只要好好照顾自己。说不定也不需三年,我们就可再相会。”

韩冈欲走,“官人!”周南怯生生喊了一声,又把那个小包裹递了过来。

韩冈笑了,摊开左手,染了血渍的丝巾展在周南眼前:“有这个就够了。”

只在乎一片心意,不为财帛所动,周南终于安心下来。她把匕首紧紧地贴在胸口,自己芳心所托,确是良人无疑。

韩冈往回走。周南紧追出几步,朝着韩冈喊着:“官人,别忘了你说的话!小女子会等你三年的。”

韩冈哈哈笑着:“我韩冈骗人的时候不少,可从不欺心。”

在周南的目送中,韩冈一跃上马,挥手而别,渐渐向西行去。

