\section{第31章 离乡难知处(上)}

已经是春风送暖的二月下旬。白马县北面的黄河水中,可以看到流冰越来越少,最多三五天内,两岸的交通就能恢复通畅。

因为黄河解冻的缘故,判大名府文彦博向朝廷要求补给的六十万石粮食,并没能运过去。在黄河冰上通道依然畅通的那一段时间里,到位的粮食仅仅只有十五万石。继而便因为黄河冰面开始破裂,这一补给的过程便停顿了下来,一直到现在都没有恢复。

由于大名府的常平仓已经不能支撑近十万流民的日常食用,流民也不得不开始向粮食更多的南方转移。隔着黄河,这段时间都能看见对面的黎阳津那里,越来越多的流民在堤岸上徘徊。

现在韩冈都有些怀疑文彦博向朝廷索要六十万石粮食,就是为了推卸责任。以文彦博的老于政事,不可能不知道黄河交通封闭的时间。他赶在黄河快要解冻的时候要钱要粮,很可能就是算好了时机,即便京城这里将粮食都准备好了也运不过去。现如今,大名府常平仓中的粮食已经吃完了,不要说京里的天子不能责怪他,就算是饿着肚子的流民也不能怪罪于他文宽夫,而只会将怨气投到京城的宰相身上。

河北流民南下,控扼要津的白马县就是必经之路。

旧滑州是东京城在黄河南岸的门户,而白马县则是滑州的门户。作为滑州州治所在,白马县紧邻着黄河,白马渡是河北通往京城的两个主要渡口之一。而从滑州的东北方,另一处重要的渡河地点,河北东路的开德府——也即是濮阳——往京城来的官道,也要从白马县东南角穿过。

位于交通要道上,白马县每年的商税收入甚至要高于田赋,要不然渡口镇的户口数也不会超过县城。只是到了流民南下的时候,交通便利就变成了一桩坏事。看着黄河对岸的流民,再想想数日之后,成千上万的河北流民涌进县中,任何人都会不寒而栗。

奔腾的黄河水冲击着位于大河中央的一座礁石,发出轰隆隆的如同雷鸣一般的声音。说是礁石,其实已经可以算是一个山包,说是小岛也可以,被两岸的百姓称为居山。居山形状如龟,差不多有二十丈上下,堵在河中心,只是稍稍偏向白马县这边。与现在韩冈以及他的幕僚们所立足的汶子山,只隔了百步之遥。

汶子山其实也只有二十丈左右,大小还不如居山,却也算是白马县中的一处难得的景致。韩冈站在汶子山的山顶小亭中,望着对岸沉吟着,而他的三名幕僚则在亭外说着话。

从山上望下去,就能看到一架风车,小小的就如同玩具。但实际上,这座风车足足有三丈高,从井中提出的水如同涌泉一般。

为了能大批量的制造风车,韩冈采取的是分包制度。打造出两台样品后,一台架在水井上作展示,剩下的一台则拆散开来将扇叶等部件分派给本县的木匠铁匠来打造,各自照着样品做着一个部件。

人多力量大这句话很有道理,只要组织得力,就能创造出奇迹。只盯着一个简单的零部件,工匠们上手得都很快,出产则更快。而原材料的准备,韩冈全都分派给各乡各村,谁上缴得多,谁就有优先权。

汶子山下方不远处的这一架风车,就是县中的工匠们将零部件送来后组装起来的。由于没有后世的标准化工业,零件都有各式各样的毛病。但大体上不会差太远,如果尺寸不合适的零件,能改造的便就地加以改造,改造不了的重新做。组装时通常都仅是打磨了一番,换上了几个零件后,就能顺顺当当的组装了起来。

不过这些风车,不像韩冈记忆中的荷兰风车,一座小屋上伸出四面长长的扇叶。却像是一面面船帆拼出来的,中轴为立式,直直的竖着,远远地看过去,就像是一个走马灯,随着扇叶可随风向自动调节,清风吹来,便咕噜咕噜的转动起来。

韩冈对于机器了解不多,看到这般容易就打造出来汲水用的风车,使得他对这个时代工匠们的手艺赞赏不已。而有了风车,一口口深井便有了真正的用武之地。

一开始打出第一口深水井用了十多天的时间,但当韩冈借助流民之手开始推广之后,负责凿井的本地村民,却一个个如同吃了药一般卖力,到现在不过一个多月的时间,全县打出的深井有一百四十余口,而其中出水的,则有三十一眼,每一个乡都至少有一眼深井。这么高的比例,算是运气很好了。

风车架在水井处,有风时用风车,无风时用畜力,日夜不停的汲水。有着三十一眼深井,至少能应付过去眼前的旱灾。魏平真和方兴甚至都为此做了诗,而各乡的深井出水时,也都大摆宴席加以庆贺,只是蝗灾还是免不了要让人头痛。

此时早过了惊蛰,从地里爬出来的若虫细小如蚁,可蹦蹦跳跳的爬得满地都是,啃噬起花草树木、田间的麦苗也是毫不费力。

站在黄河岸边的山包上,看到脚底下密密麻麻的蝗虫幼虫,游醇只觉得头皮发麻。刚刚孵化出来就已经是铺满了地面,若是让它们长成了飞起来,那就是遮天蔽日,这还了得?!

也幸亏韩冈在县中的威信高,已经组织起了人手来扑打,从汶子山上望下去,能看见有上千人沿着河堤排开阵势,举着笤帚向着地面扑打着。看起来要灭掉这一段的蝗虫并不费什么气力。

但区区白马一县的灭蝗顺利,对于黄河两岸的河北河南几百里蝗区来说,根本无济于事。河北蝗灾已经近在眼前,而京畿这边,也极有可能爆发蝗灾。

方兴不停地跺着脚,蹦跶到他靴子上的蝗虫让他恶心的要命。

游醇忧思难解:“春麦正是发芽的时候,这时候蝗虫出来,也不知能留下多少。”

春麦早在元月底就播下去了,韩冈作为宰相的女婿,通过王安石弄到些种子,还是比较容易的。整个京畿各县都要春麦种子,而白马县靠了韩冈,不但第一个拿到手,而且从比例上说也是最多的一县。几乎将所有已经确定绝收的田地,都补种上了。

方兴一边跺脚,一边道:“我们这边好歹有正言在,河北那边还不知道怎么办呢。”

魏平真望了一眼亭中的高大身影。回过来摇了摇头:“流民就在河对面,河北还能怎么办?倒是先想想我们这边怎么办吧!”

三人现在都知道了韩冈的心意,也差不多确定了王安石将韩冈安排到白马县,就是为了要将河北流民堵在这里。

“可惜只有一县之力啊。”方兴摇摇头,对王安石的吝啬有些看不过去,“要想都救助下来,不是白马县能做到的”

“若是正言权柄再大一点,那就好了。”跟在韩冈身边几个月,游醇对韩冈的一番作为看在眼中,虽然因为自矜,没有明着说出来。但他对为治下百姓,殚思竭虑的韩冈已是敬佩不已。游醇相信韩冈有了更多的权力之后,能做得更好。

“节夫是要复滑州?!”魏平真转头过来,惊讶的问道。

“复滑州?”游醇不知道为什么魏平真这么说,他只是随口感叹,并没有这个意思。

但方兴在旁听了,仔细一想,却觉得恢复滑州的想法的确好处不少,“白马作为京县,那就是通判的资序。现在正言第二任通判算是做了,再往上就是知州资序了。如果滑州恢复,以正言的品阶,甚至权发遣的前缀都不要,直接权知滑州就可以了。白马可就是原来的滑州州治,如今的县衙就是旧时的州衙。正言升任滑州知州,只要换块牌匾,连门都不要出的。”

游醇想了一阵,也随之兴奋起来:“如此一来,有这一州之力,救助起流民来当然也就容易了许多。更别说以正言之材,治理州郡也是易如反掌,滑州三县之民,也能免了蝗旱二灾之苦!”

“可是有人肯定不愿意啊……”

反对的声音并不是出自游醇、魏平真和方兴,而是来自他们的身后。

三人急忙回头,竟是韩冈不知何时到了身后,正微微笑着。他们急忙躬身行礼,连声请罪。

“无妨。”韩冈倒不在意他们在背后说什么,何况还是自己的好话。但他们所说的恢复滑州的提议,朝廷允许的可能性并不大。

尽管如今行政区划的变动十分频繁,远比千年后要容易。但才一年多的功夫,就喊着要恢复,等于是在此前撤并二州的倡议者——好吧,其实就是曾布——的脸上打耳光。

而且前年滑州和郑州并入开封府,也是两州的乡绅父老求来的。就如后世的京城,公共交通的费用远小于地方上的城市,这个时代开封府的赋役也远远小于外路州县——这是京城人的特权,也是朝廷为了维护稳定所付出的代价——同时少了州郡衙门的几十个官员以及数百衙役,两州百姓也要少交许多额外的杂捐。

“当初是两州百姓联名情愿,如今还能让他们联名吗?”韩冈摇着头,这根本不现实。

但他的眼中自信不减,要安抚下入京的流民,舍我其谁?!

