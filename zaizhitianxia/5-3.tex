\section{第一章 庙堂纷纷策平戎(三)}

韩家的司阍接了名帖后,就安排了郭忠孝在门房中等候,自己则进了府中通报。

韩家待客还是很有些规矩,就是坐在门房中,也有一份茶汤和菓子来招待,一点也不像刚刚起家不久的寒门素户吝啬,却也不似暴发户一般的喜欢炫耀。

但在门房等候主人接待的这个体验,对郭忠孝来说已经不知是多久以前的事了。他的父亲郭逵早早的就担任了一任执政。作为同签书枢密院事的儿子,郭忠孝在年纪长到可以出门访友的时候,已然没有几人可以让他待在门房中,看椽子上留下来的水渍。只要表露一下身份,基本上立刻就会被迎进去,即便郭逵只是武将,但武将的地位高了,文臣在场面上也必须给予足够的尊重。

‘有年头没有修了。’郭忠孝百无聊赖的想着,老房子都是如此。

几年前,韩冈还是郭逵面前的后生晚辈,在寻常人眼中,甚至还比不上郭忠孝。但如今,韩冈与郭逵已经平起平坐,相差仿佛了。说句难听话,还没有考上进士的郭忠孝,连嫉妒都不够资格。

端起白瓷茶盏喝了口茶,口感微涩,但比只经过一道蒸青的散茶要好很多,却又跟小榨去水,大榨去茶汁,去汁后置于瓦盆内兑水研细,最后压模成型的团茶又差得很远。

郭忠孝端着茶盏,就着灯火看了一下,在杯中舒展开来的茶叶是标准的散茶模样,只是口味独特,不知是出产自哪里的新品。

不过对于韩冈身边出现一些新奇的事物,郭忠孝已经见怪不怪了,世人也是如此。不论是官场、战场还是儒术,医术,韩冈都有震惊世人的事迹,无数例证早就证明了这一点。

又啜了一口,感觉还是不错,郭忠孝两口喝光杯中茶,放下茶盏示意再续水。

在门房中侍候客人的韩家家丁,立刻就怀疑起郭忠孝的身份来了,枢密家的儿子怎么这么没见过世面。

脚步声响,韩府司阍从门房的内侧小门走进来,抱拳行礼:“郭衙内,我家龙图已在内厅相候,请移玉趾,随小人来。”

郭忠孝心中暗叹,就知道韩冈不会自降身份来出迎。

司阍和另一名仆人,提着灯笼在前引路,郭忠孝和他的随身伴当跟在后面。领路的司阍不是在官场中有些名气的断了一条腿的那位韩家看门人。他的腿脚还算灵便,但左肘一直向内弯着,走起路来也不伸直,可能是左臂在战场上伤了筋。

在郭家的庄子上,其实也有一批身有残疾的老兵。都是跟着郭逵出生入死后的亲兵,最后不能再上战场,被郭逵养了下来。但郭逵不会让他们去守大门,影响郭家的体面。但韩冈不在乎,照样使唤。开始时,还被人嘲笑韩家的门第浅薄,到了如今,完全变成韩冈仁人仁心了。

地位变了,郭忠孝心道。庶民犯蠢,那就是蠢事,而名人犯蠢,可就是轶事了。

绕过照壁,韩家正院的院墙下,放置着一堆堆砖瓦、木料等建筑材料,虽说在夜中,那只是几堆模糊的黑影,但石灰的味道是瞒不了人的。郭忠孝心知,韩家刚刚搬进来,多半是要重新整修一下宅邸。

韩冈的同群牧使宅子比起普通朝官一进两进的院子要大得多,可相对于执政级的郭府则要小不少。穿过一重穿堂,前面院落的左侧灯火通明的房间前,站着两名身高体壮的汉子。自然这就是目的地。

韩冈就在偏厅中,等着郭忠孝,外面有两名家丁守候。

郭忠孝选在夜中来访,当然不是来叙旧,更不会是以二程弟子的身份来讨论学术上的问题,只可能是奉了郭逵之命,私下里来联络,商议如何应对眼下的局势,甚至是订立攻守同盟什么的。

即然郭忠孝是以同签书枢密院公事之子的身份来拜会,身为龙图学士和同群牧使的自己就没必要出迎了。

“龙图,客人到了。”门外传来声音。

韩冈步出厅门,却没有走下仅有两级的台阶,看着院中走过来的郭忠孝。

“郭忠孝拜见龙图。”见到正主,郭忠孝徐步上前,躬身行礼。

韩冈也不更正郭忠孝对自己的称呼,还了一礼,寒暄两句侧身邀郭忠孝入厅,“还请厅中说话。”

两人入厅后分了宾主坐下,下人又奉上了茶汤。郭忠孝喝了一口,是门房中的茶水同样的香气和味道。

在灯火通明的客厅中,郭忠孝更加确定杯中茶汤并不是蒸青散茶冲泡出来的深绿,而是更为浅淡的一种黄绿色调,依然有别于团茶:“龙图家的茶倒是特别,不知是何名色,何处所产?”

不意客人拿着茶叶当做开场白,但韩冈也不心急,道:“就是秦岭山中的野茶树产的野山茶,也没想过要取名。山坳里的一小片茶林,一年的出产仅有百来斤,是当地山民的自用。我只是偶尔尝过一次,觉得合口,就干脆将每年多余的出产给买下来了。”

郭忠孝摇摇头,笑道:“此茶口味特别,不仅仅是野山茶的缘故。”

“是制法有别的缘故。寻常茶叶皆是上屉蒸青,但蒸法耗柴薪,山民俭省,直接就在锅上炒了。比不上龙团工序繁复,不过喝起来倒是别有一番风味。若是立之觉得不合口,韩冈就让人换了龙团来。”

“不必了,这茶虽与世人口味不合,却正合在下心意,家父应该也喜欢。”

“即是如此,待会韩冈就让人包上两斤赠与立之。还望不要嫌少,已经是年终,韩冈手上也只剩七八斤了。”

“多谢龙图厚赠。”郭忠孝又喝了一口茶,越发的觉得这茶合口味,不过他今天来不是为了喝茶的,而是有正经事。叹了口气:“如今攻打西夏,也是如同这野山茶一般,合乎天下人之心,可惜不合龙图和家父的想法。”

郭忠孝并不是上佳的说客,话题转得有些勉强。韩冈的:“辽夏两国同时内乱,如此良机千载难逢。瀚海虽是难渡,但如今军中名将如林,精兵无数,攻下兴灵也不是不可能。韩冈也只是觉得直取灵夏稍嫌冒险,希望能够稳妥一点,并不是觉得不该攻取西夏。想来令尊郭太尉,也不会认为此战必败吧?”

“的确不是。”郭忠孝摇头,“家严也只是想着能够稳妥一点。”

甫一见面,韩冈对自己称呼他‘龙图’受之不移,郭忠孝就知道今天的差事不好办了。这样的一幅公事公办的态度,并不见亲近,有些话就难以说出口。

“那不知立之今日夜中来访,不知又是有何事指教?”韩冈问道。

他不信郭逵敢在这时候去幻想讨伐西夏的主帅之位。

不是说郭逵会担心走了狄青的旧路。只要郭逵在得胜后立刻辞官归隐,文官们也不会去跟他过不去,而天子更是要荫封他三代以作酬劳,郭家至少能安稳三代而不虞门第衰落。而是说郭逵绝不会蠢到认为自己会同意以稳步推进为条件,帮他夺取西军主帅之位。

韩冈的基本盘在西军,他绝不可能反对攻打西夏,也不会同意让外来的将帅得到主帅的位置。先取兰州、银夏的方略,只是体现了韩冈稳妥的性格,并不会与灭亡西夏的总方针相违。而郭逵虽说多次在关西任职,可并非西军出身,他想要虎口夺食,韩冈怎么也不会支持他。

郭忠孝却在反问:“如果朝廷当真以兴灵为目标而兴兵,不知以龙图之见,当如何用兵?”

韩冈看了郭忠孝两眼,随即扳起了手指:“西夏乃万乘之国,自当全力而攻。出兵兴灵,大的方向为四路,从出兵的地点细分下来则是六路:

河东军过黄河,直取西夏腹地,破祥佑军司,入银夏,趋灵州,这是第一路;

鄜延路所部沿无定河北上,越横山,攻取银、夏,进而越瀚海攻灵州,这是第二路;

环庆路兵马穿过青岗峡攻韦州,越瀚海取灵州,这是第三路;

泾原路军从兜岭走沿葫芦河北进,攻灵州,这是第四路;

秦凤路兵马翻越柔狼山,沿黄河取灵州,是为第五路;

熙河路官军会合河湟蕃军攻下兰州北上,截断西凉府和甘肃军司的勤王援军,并向东攻灵州,这是第六路。”

“全军会合在灵州城下?!”郭忠孝抬眼问道。

韩冈冷笑:“这样的规划当然可笑之极,可一旦以兴灵为目标,又有谁甘心落后他人一步,为他人作嫁衣裳?都会往灵州赶,根本拦不住——将在外君命有所不受,天子的话都没用——还不如事先做好准备,省得因为粮草不济而饿死,反正只要攻下灵州城就够了,以官军的实力,任何两路兵马合力,应当都能做到这一点。”

郭忠孝沉默了一下,叹道:“……龙图的说法跟家严一模一样。”

“所以韩冈想问,郭太尉究竟是什么打算?”

“龙图当真认为辽国内乱,就一点也不用担心了吗?”

韩冈神色终于变了:“郭太尉想要去河东?!”

郭忠孝没料到韩冈反应如此之快,惊异之下点头道:“用兵以奇胜,亦须以正合。辽国虽说内乱在即,但也不是百万大军会捉对厮杀。家严对辽国内情稍有心得,真正会参与内争的也只是各部贵胄名下的头下军,以及从属于各斡鲁朵的宫分军而已。西南、山后诸军会参与其中可能性并不大。仅仅是一西京道,就有十万兵马。焉能以其国中内乱,而轻忽视之?”

