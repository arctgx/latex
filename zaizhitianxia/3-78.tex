\section{第26章 任官古渡西(三)}

九月末,天候渐寒。可天上的太阳依然明亮,照得行人身上暖洋洋的。

天朗气清,乃是赶路的好时候。从白马津往京城去的官道上,行人车马便是络绎不绝。

韩冈一行离开京师不过一日,第二天出发后不久,就看到了滑州胙城县的界碑。

在界碑前,韩冈停了马,跟在后面的三人也都停了下来。低头仔细看了一看界碑,韩冈回头笑道:“滑州还真是近,这么快就到了。”

“滑州都被撤了,这界碑却到现在都不改,开封府中干管此事之人真该打板子。”

紧跟在韩冈身后的这名三十出头的南方士子,唤作方兴,总是带着笑,微圆的脸看起来有些滑稽。他乃是江西金溪人氏,是王安石推荐过来的幕僚。不过要说是王安石推荐,其实还不如说是靠了王雱的缘故。

方兴与王雱自幼相识,当初王雱在江南任官,他便在其幕中。前岁王雱进京,方兴也跟着来到了京城中。先是被推荐去了国子监中读书,但今科的科举,却连贡生的资格都没拿到,遂断了进士之念。这些天在京中待了无聊,却是跟王雱求了个人情,来到韩冈这里。

韩冈第一次与方兴见面,先听了他自我介绍了一通后,又听他说道:“方兴族兄向有令名,与相公有旧,相公亦曾有文一篇赠予族兄。”

韩冈当时并没有反应过来,遂摆出了礼贤下士的模样:“敢问是哪位大贤?”

“大贤不敢当,大名唤作方仲永。”

方兴爆出答案,在旁的王雱哈哈大笑,韩冈也似是自嘲的摇头失笑,但心中却是微感不快。方兴拿着自己的族亲当玩笑开,觉得有点让他难以接受。不过一表三千里,论起族亲也是远到不知哪里去了,拿出来当笑话介绍自己,也算不上什么罪过。

“走得快一点,今晚就能进白马县。”在界碑旁,韩冈顺着道路向北面望去,不过入了滑州地界,离着白马县还有几十里地,“就不知白马县中有什么让人棘手的大户豪门?”

“这倒没有没听说,想来也不会有。”方兴为了能在韩冈幕下做事,还是请了王雱帮忙,看了不少白马县的资料,“白马县虽是畿县,但户口却是最少,两千四百多户人家,丁口八千,不过是中县而已。”

韩冈算是在考试,之前见面的时候,并没有多问,那样不太礼貌。听了方兴的回答,他也是有所感触,“白马县原来并不差,乃是河津要地,三十年前还算得上是紧县望县一级。但仁宗年间,连着几次河决都撞上了,人丁流失大半,到现在都没有恢复元气。”

“所以白马县最让人头疼的就是律讼多。”说话的魏平真,在四人中年纪最大,已经有五十岁了,乃是王韶所荐。为人老成持重,阅历见多识广,“尤其以田宅上的瓜葛官司为甚,而且根本断不出个是非来——有的是全家户绝后,外来的骗子冒籍来夺田,有的则是原来的田主来要回自己被占的田地,完全分不清真假。听说有打了二三十年都没见分晓的……都是河决的缘故啊!”

“现在要是有着河决时的那么多水就好了。”方兴却是在抬头看看蓝得一丝纤云都没有的天空,“有多长时间没下雨了。”

最后一名身矮而瘦的儒士,相貌普通,双目晶亮,操着一口福建腔:“此乃是德政不修的缘故。”

“节夫此言差矣。”

游醇游节夫是程颢推荐来的弟子,他还有个弟弟叫做游酢,现在就在程颢门下就学。韩冈还不知道他能不能在政务中派上用场。不过就算派不上用处,韩冈也照样会恭敬有礼的待着他,怎么说程颢的面子都要顾着的。

但王韶荐来的乡里魏平真却没有那么多顾忌:“其实水旱交替,如同阴阳相转,乃是天道。阴盛阳衰、阳盛阴衰。连接几年水患,接下来便会停上几年,跟着就是连着几年旱灾。此是天道循环,与人无关。试问尧舜施政又有何错处,为何洪灾遍于天下,需要大禹来治水?”

游醇瞪眼要辩,韩冈却抢先一步问着魏平真:“前些年京畿有水灾?”

魏平真虽然是德江人氏,但他在京城已经住了有二十多年,近五十岁的他,对于京城内外一切消息,都比韩冈这等小辈要明白,“从嘉佑元年开始,再到治平初年,这七八年时间,京师不知淹了多少回了。”

他扳起手指一一为韩冈数着:“嘉佑元年【1056】四月,京师大风雨,六塔河决,水注安上门,坏官私庐舍数万间。嘉佑二年五月至六月,京师雨未停,水冒安上门,门关折,城中系伐渡人。嘉佑三年,京畿河溢,坏民田。嘉佑六年,京师久雨,至冬方止。治平元年【1064】,京师自夏至秋淫雨不止,坏真宗及穆、献、懿三后陵台。治平二年,京师大水,坏官私庐舍无数,军民死者一千五百余人……”

“原来如此!”韩冈点着头,却是在阻止魏平真继续下去。

尽管是魏平真是平铺直叙,没有添加多少感情。但听着就是怵目惊心、不忍卒听。韩冈本来是想用来阻止游醇的辩论,可不是要听京畿有多少苦难的历史,更不是为了要将游醇气着。

韩冈的想法,老于世故的魏平真能看明白,笑了一笑,道了一句:“看着旧年的雨,如今的大旱说不准还有几年。”

……………………

韩冈并没有急着往白马县赶,照规矩要白马县中官吏、乡绅出来迎接他,所以午后到了胙城县后,就歇了下来,并派得力之人去白马县通知抵达的时间。

其实也不需要韩冈派人通知,白马县也在开封府地界中。韩冈刚出城的时候,就给诸立派人给缀上了。倒不是怕他少年心性,弄出微服私访的把戏,而是想要提前做好迎客的准备,争取留下个好印象,

出开封后的第三天,韩冈终于抵达了白马县。

刚刚进了白马地界,就见着一群人远远的迎了上来,隔着老远就在喊着:“可是平灭虏寇,威震关西的韩正言。”

韩冈在马上抱拳:“正是韩冈!”

姓名一报,就见着这些人连忙跪下,一片声的恭维:“我等白马小民,在此恭候正言多时。正言弱冠之龄已是名震海内,听说正言来此任官,我等真是三生有幸。”

韩冈微一皱眉,未免做得太过了一点。连忙下马将其中年纪最大,胡子全都白了的几个老家伙,全都搀扶了起来:“几位老丈大礼,韩冈年幼,可是折受不起。”这几个看起来都有八九十岁了,上了紫宸殿,天子都不好意思让他们跪拜的。

一番礼节之后,韩冈重新上马,一路行到离县城十里地,又是一拨人在路边候着,还是扎着彩棚在迎接,满口的好话奉承,一碗碗迷汤灌过来。

到了五里地,就是第三波接着。等到进了县城,前任知县凌庄带着白马县的一众官吏在县衙前候着。

见到韩冈,凌庄就堆起笑脸来迎接:“久闻韩正言的大名,如今方得一见。在下于白马三年,无所建树,如今有正言相代,必能一济白马县父老倒悬之苦。”说着,就要请韩冈入内,交接大印和县中内外账籍。

大印其实是小事,关键的是库中的账目和库存不能有差错。韩冈带来的是顺丰行在京城的掌柜,同时魏平真也是精于财计。再加上韩冈自己对账目上能玩花活的几个关节,也是了若指掌,所以根本不担心有什么。先由顺丰行的掌柜把第一道关,让魏平真把第二道关,最后自己再出面审核。三道关卡,不信有谁能过去。

但韩冈并不急着查验:“此事并不着急,韩冈早前曾经为王副枢筹划粮秣转运一事,知道点验库存不是一天就能完事的。且等明日再说。”

几句话,就摆明了车马。魏平真捻着胡须轻轻点头,而诸立等一干吏员则是脸色微变。

韩冈分明是在说他来交接,对于库中帐籍,绝不会睁一只眼、闭一只眼的走过场。而且明着说自己在熙河曾经给几十万大军管着粮秣补给,更是在警告白马县官吏,不要想着可以蒙混过关。至于将点验库存的事拖到第二天,就是给了白马县官吏们一夜的时间,如果此前还抱着幻想,没有去弥补亏空的话,今天晚上就不要睡觉了,赶快把漏洞给补上。

不过是几句场面话,但该说的却都说了,就跟混了几十年官场的老狐狸一样。诸立心思微沉,的确是精明干练,不好糊弄。

前任知县凌庄好象是没有听出其中的隐义,笑呵呵的道:“是不用急,是不用急。既然如此,还请正言入内,下官已经让人办下了接风酒,正等着正言入席。请!请!”

说着就拉着韩冈的手,一起往县衙中走去。

