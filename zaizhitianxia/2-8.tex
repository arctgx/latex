\section{第二章 边声连角不知眠(四)}

【第二更,求红票,收藏】

韩冈不介意跟向宝顶牛,反正也不知道什么时候会挨上秦凤都钤辖的反手一刀,现在斗斗气,也没什么大不了,正好可以看看向宝的反应。

对,就是向宝现在这种参杂着不屑、嘲笑和居高临下的神情,前面的怒意倒像是他伪装出来的。

看起来到了地头就要把脖子洗干净了,韩冈想着。向宝的这副模样基本上是铁了心的要置自己于死地,连计划都做好了。

想不到这家伙真的要发疯……韩冈现在觉得请王韶做个双保险当真是作对了。就是不知道王韶那边能不能成事——检验他在秦州这两年的成果的时候真的到了。

三军未动,粮草先行。不过古渭、永宁诸堡,都备有大批粮秣和军资,不必向宝操心。但出战的饷钱却少不了要筹办,还有胜利后赏赐,都得准备好。

向宝手上有人,他的幕僚水平也不差,办起事来熟能生巧,不过一夜工夫就已经筹办得差不多了。而韩冈既然负责军中医疗救治,也不客气,跟在里面要钱要物要人。李师中和向宝既然让他负责随军医疗,但总不能让自己两手空空的变出药来。

药品物资不齐备,真的要治罪的时候,韩冈可是有得话说。

韩冈存了这个心思,便狮子大开口,不出意料的,他申请的药材、布匹之类的物资,就只发下了不到三分之一。

韩冈当即去找向宝:“敢问钤辖,领兵在外,粮草不及三一,不知可否出战?”

“你想说什么?”向宝冷冰冰的直接问着,懒得跟韩冈拐弯抹角。

韩冈这下也不弯弯绕,直言道:“下官已经是一省再省,但发下的药材、布匹等物仍只有三分之一不到。若是药材不足,让受伤的将士们白白枉死,那究竟算是谁的责任?”

向宝死盯着韩冈,他想不到这个灌园小儿竟然还真的敢在自己面前‘理直气壮’,一点都不担心后事,“本帅过去领兵,可没这么多手尾。”

“所以过去才死得多。”韩冈说得更不客气。

“来人!”向宝又狠狠的盯了韩冈一眼,叫过来身边的一个亲卫,“去跟管库的庆思道说,把韩冈要的都给他配齐。”

‘配齐?’韩冈心底冷笑着,‘包扎用的布匹也许能补齐,但伤药若能配齐,我就跟你姓向好了。’韩冈他在勾当公事厅的十几天辛苦并没有白费,隶属于秦州和秦凤经略司两个系统的库房里的存货数目,他都是能做到心里有数——加起来连他现在要求的六成都没有!韩冈可是专业人士,药材需要多少全任凭他一张口。

“对了,下官还有一事。如今配发下来的许多药材都是在库中存放已久,往往朽烂不堪……”韩冈说着又从怀里掏出一团土块样的东西,展示给向宝看,信手一捏便成了粉末,“看看,这样的伤药如何能用?”

向宝看着从韩冈指缝中簌簌而落地粉末,算是明白他的想法了,这是韩冈给他自己在找退路!——‘药材备不齐,救不了人,可别怪我。’

向宝忽而冷笑:‘不过既然你已到了我麾下,怎么挣扎都是没用的。’

他也懒得敷衍韩冈,一摆手:“库中的东西你自去搬,能用的则带上,用不了就留下。”

……………………

到了午后,一切准备就绪,向宝便领着一众幕僚佐吏启程出发。他手下的兵还在永宁和古渭,秦州城里的军队,李师中本是一点也不打算给的。

不过也不知向宝又跟李师中怎么打得饥荒,竟然把秦凤经略视为心头肉的一个指挥的骑兵弄到了手。有着五百骑兵作陪,再加上运送军饷的车队,向宝的此次出行也算是颇有些气势了。

韩冈骑在马上,随着队伍前进。向宝的将旗在他前面百步,而他的身边,是三车子的药材和布匹。

“韩抚勾,怎么你家的王机宜没能来送你?”向宝的一个段姓幕僚过来搭话,看他脸上的笑模样,也是想着看韩冈笑话。

“王机宜事务繁忙,也是有要事缠身才没法儿过来送行。”

段姓幕僚知道韩冈是在随口搪塞,王韶昨天午后紧急出城的事,并不是什么秘密。不过王韶的身份一向特殊,秦州城从来都是来去自如,李师中都管不到他。段姓幕僚也不指望能在韩冈嘴里问到些什么。

只是他一转眼,又看见韩冈低着头在扳着手指:“怎么了,抚勾是在算吉凶?”他带着笑意的问着。

“韩冈只是算着时间。”韩冈回了他一个笑容,“算算还有多久才能到古渭。”

段姓幕僚一指队列前后,“尽是车马,没有一个步行的,也就是四天上下。”

“四天吗?”韩冈点了点头,跟着看了看排在队伍前后的骑兵,如果没有这个指挥的骑兵的话,的确四天能到,但多了这五百名骑兵,情况就不一定了,向宝的这位幕僚,肯定没有经历过战阵。

四条腿比两条腿快,那六条腿呢?

实际上,为了节省马力,也为了保护战马。跟随向宝出征的这个满编指挥的骑兵,每天只有一个时辰的骑乘时间,其余时候都是下马步行。

韩冈暗地里笑称他们是六条腿的骑兵。朝廷没马,不可能像契丹那样,每一个正兵几乎都能配上一人三马。连秦州的骑兵,也都只能是一个人配一匹马。韩冈从刘仲武那里对骑兵有了最直观的认识,很清楚以这支骑兵的行进速度,没有七八天时间,到不了古渭。而四天后,他们方才抵达永宁寨。

在永宁寨,向宝终于得到了他今次要指挥的军队,而韩冈的麾下,也多了一群人。被一纸调令紧急调到韩冈麾下的是甘谷城的朱中和甘谷疗养院的半数护工,他们收到秦州的调令后,就在伏羌寨等着向宝和韩冈一行。

韩冈前世听过一种说法,说死在战场上的军官,有两成是伤在背后。韩冈也很清楚,上了战场后,出些意外很正常。如果向宝肯让他在古渭寨设立医院,那他的安全可以得到保障,但如果跟在向宝左右,说不定就会出点意外,比如一支流箭什么的——当然,这是指向宝发疯的情况下会做的事。

如果向宝足够理智,绝不会命人直接拿刀子捅自己,也不会玩什么流箭意外,最有可能的是给自己栽一个罪名,然后把王韶拉下水,这才符合向宝自己的利益。在军中杀一命官,向宝是给自己添麻烦,还得不到什么好处,除了能出一口鸟气。

所以韩冈现在还不到这么紧张,王韶那边的保险姑且不论,反正到最后,他还有摔断胳膊腿这一个断尾求生的招数在,要保住性命倒真的没什么难度。

为了整顿兵马,向宝在永宁留了两天。韩冈也初步把他的随军医院建立了个框架,朱中等人有了几个月的经验,比起韩冈来,手法更为熟练。

这一日,向宝终于把永宁寨的兵马整顿完毕。清晨时分,太阳刚刚升起,在校场中集合了今次出征的四千兵马。

就听着向宝站在点将台上放声豪言,而下面士卒们的欢腾声一浪接着一浪。

“只要尔等奋力杀贼,朝廷就绝不会吝惜赏赐!”

“杀光托硕部的吐蕃胡狗,回来自有金银美酒!”

“眼下我有四千大军,再加上古渭寨还有六千兵马!另外又有数十蕃部十万人马听候使唤,”向宝一口气把古渭寨的兵力翻了个好几番,“小小托硕一部,如何能当得我信手一击!”

向宝多年带兵,知道如何鼓动起士兵们的狂热,连大营门口的守兵都不知不觉的走进校场,跟着向宝一起热血沸腾。

向宝高高举起酒碗,誓师出征的血酒就在碗中摇晃。

一个风尘仆仆的士兵这时突然从大门处闯进来,他所骑的马匹上,用竹竿高悬着一条白绢,绢上密密的尽是文字。台上众官的注意力都一下落到了他的身上,他的这副模样是所谓的露布飞捷,身份是铺兵,传递的是捷报。

就听着他在营门处一声大吼:“大捷!大捷!王机宜在古渭寨运筹帷幄,调集七部兵马近万,昨日大破托硕部,生俘其族长以下酋领近百人!”

急脚递的铺兵吼了一声就跑了,赶着去秦州报喜,书着捷报的白绢如旌旗般猎猎飞扬。这名铺兵的目标并不是永宁寨,只是经过时看了这里人多,他就冲进来顺便喊上一嗓子,这也是露布飞捷的用意所在。

全场一片安静,静得仿佛在守灵。每个人都听清了那位铺兵的喊话,但没一个人能即时反应过来他到底说了些什么。

很静,很静,所有人都沉默着,虽然他们这时已经明白过来,但他们的沉默仿佛是在对方才的狂热做遗体告别。

哐啷一声,向宝举得高高的酒碗落在了地上,碎成了千百片。随着青瓷碎片的飞散,血酒为之四溅,沾湿了他的马靴。

向宝整个人摇摇欲坠,耳中嗡嗡直响,只有方才的那句话在耳边响着:

王机宜在古渭寨运筹帷幄,调集七部兵马近万,昨日大破托硕部,生俘其族长以下酋领近百人!

王机宜!

七部兵马?

大破托硕?

托硕族已经败了?这么说来,他方才的表演,不完全成了笑话?!

向宝突然觉得眼前一片鲜红,莫名的人影在视线中晃来晃去,就像他就年在集市上看到的灯影戏。他们好像在说些什么,但向宝什么也听不清楚。

看不清、听不清,头又昏得厉害,他突的心中一阵烦躁,用力的推开周围的人。可下一刻,秦凤都钤辖的视野便完全黑了下去。

所谓釜底抽薪,不外如是。韩冈看着一头栽倒的向宝,微微一笑,缓缓地踱上去,

‘想不到新店开张,第一个上门光顾的,竟然是向钤辖呢……’

