\section{第29章 百虑救灾伤(12)}

跪在通往前庭的屏门前的白马县押司,在冬日的寒风中冻得脸色铁青,胡须上缀满了白霜。又没有戴帽,花白的头发也曝露在风中,一丝一缕的乱发随风飘着,看着一副可怜兮兮的模样。

这已是诸立在县衙中前下跪的第三天。当天子下诏根究粮商不法之举的次日,诸立就跑来向韩冈请罪。但韩冈一直没有理他,任凭他清晨来、夜中去,连着跪了三日。

三天来,在县衙中进进出出的人不少,都看到诸立跪地。县中百姓纷纷在议论,县尊是不是要拿诸家开刀。开封那边的事,白马县中百姓也都听说了,诸立本就是跟那些被捉将起来的奸商们混在一起的。王相公的女婿要动手,当然不会放过诸立。

此前高价卖粮,诸立的确招了不少怨恨。但后来赶在天子诏令之前降价售粮,人们也都看在眼里。现在看着他五十岁的人在寒风中连跪了三天,老百姓心肠软的居多,外面的舆论都对他都有了一点同情。

今天,韩冈并没有从他身边径直走过去,终于停下了脚步。低头看了他后脑勺半天,开口问道:“你家还有多少存粮?”

终于等到韩冈开口,诸立心头一松,身子便摇摇欲坠。用着最后一份精力,强自保持着心中的镇定,不敢有丝毫隐瞒的老实回答道:“有两万一千余石。”

这个数字让周围的衙役和韩冈身后的三名幕僚都忍不住一声惊呼,县中的仓储也不过是这个数字的两倍而已。深藏两万石,诸家的确是在囤积居奇。

“都拿出来捐个官!”韩冈丢下一句后,就转身离开。

穿着一对厚底官靴的脚从眼前移走,诸立浑身的力气消失得一干二斤,一下瘫软的坐在了地上。一直躲在一边的两个弟弟立刻跑上前来,紧张的问道:“大哥,怎么样了?”

诸立只是点头,兴奋和放松让他的脸上恢复了一点血色:“保住了,保住了。”

捐出两万一千石虽然肉痛,但换算成如今的米价其实也不过是两万多贯而已,诸家还负担得起。用这份钱买下全家的安稳,怎么都是合算的。

要是韩冈一本奏将上去,说白马县吏诸立‘赋性奸猾,囤积渔利’,那被捉进大狱的三十七家粮商之后,就要再多添一个白马诸立,一家老小全都要完蛋。

而见到诸立点头,诸霖两人也都软了脚。几天来他们夜夜都做着噩梦,每次都是从身死族灭的结局中惊醒。现在韩冈终于松了口,好歹也能睡安稳了一些。

三名幕僚紧追在韩冈身后,只有游醇皱眉问着:“为什么要放过这个奸商。”

韩冈回头看看三人,方兴和魏平真全无讶色。看来这两人已经知道自己的心意。自家让诸立跪在这边三天都不加理会,其实已经可以看出他无意治罪,否则第一天就可以将其下狱。只有游醇年轻,没有看出来其中的门道。

韩冈轻笑道:“大鱼小鱼都已经入网,有没有虾其实也无所谓了。”见着游醇要争辩,他又接下去说道:“再说前面还没事发的时候,我让他降价他也听命降价了。不管诸立当时转着什么心思,至少没在行动上给我弄鬼作祟。且既然早在诏令出台前,诸立就已经降价售粮,再处置他就有点说不过去,罪名加到他身上也有些勉强。”

从心底来讲,韩冈其实也是想顺手将诸立一起给扫进去,当初吩咐他降价售粮的时候,也不是没有一份算计在内其中。但天子下旨清办粮商的时间比预计的迟了两天,这使得遵照韩冈吩咐、平价贩售米面的诸立‘囤积居奇、至使民变’的罪名就很难成立了。

如果强要将其弄进狱中,用的借口就会显得太勉强。到时候,这反而就会成为对手反击的一个突破口。被人以一点攻其余,审理其他粮商的时候,就少不了麻烦了——其实这也是后世许多案子中,将人另案处理的重要原因之一——现在也只能放其一条生路。想想,自己前些日也的确性急了一点。

韩冈走进大堂中,接着又道:“也是诸立足够聪明,三天来只是一个人跪着。要是诸家的三兄弟一起来跪,我也只有将他械送大狱了。”

若是连着两位赵家的女婿来跪着求饶,其行径就等同于威胁,韩冈若不拿他们往死里办,那才叫有鬼。诸立并没有这么做,而是将姿态放到最低。在县衙中总是以强硬姿态现身的诸押司,腰骨如今软起来,也是跟面条一般。

“不过就此放过他也太便宜了。”游醇依然耿耿于怀。

“所以正言让他跪了三天。”魏平真道:“如果不是这一跪,正言放过他也会有些议论。”

方兴跟着道:“何况正言已经将他赶出了县衙,又挖了他的根,放过他也就跟放过一条死狗一样,无甚大碍了。”

游醇先是一愣,然后一下恍然,接着却又忧心冲冲起来:“就怕他有官身后,就盘剥百姓,将入粟的花销全都赚回来。”

魏平真眼睛一翻,笑着反问:“有官身就会有差遣吗?”

游醇张口结舌,而方兴也呼呼的笑了起来。大宋的官员数目是实阙的数倍之多,有多少官儿一辈子能轮上一个好差遣?

韩冈让诸立拿了家中所有粮食出来捐官,绝对是一个惩罚——纳粟捐官,得到官位都很小,也没有晋升的空间,而且还容易被歧视,得差遣极难,一个肥差则更是难上加难,所以很少有人这么做。正常情况下,都是花钱娶个宗亲回来,从此有官位有靠山——而且当诸立有了官身之后,就不可能再做吏员了。

诸立虽然帮着两个弟弟娶了宗女,挣了两个裙带官回来,但自己却一直保持着无官一身轻的状态,不是他做不了官,而是在衙门里的利益太大了,舍不得去做官。但现在被韩冈硬逼着买下一个不想要的官身,攒了三十年才在白马县积攒下来的影响力,转头就会化为泡影。

影响力,是威望、权位和人脉的综合。诸立的声威、地位和人脉关系,都是靠着他在县衙中做了三十年押司而渐渐聚来。现在职位不存,而且还是因为高价卖粮的缘故,而被知县处罚,他的威望从此不再,地位无存,人脉当然也不可能再保住。这还不如直接捐出来修桥铺路来得好,至少那还能攒点阴德、聚些人望,为子孙后代留点余荫。

而诸立一去,县衙胥吏中就再无人敢阴私作祟。本来被诸立压着的胡二等人就算上台来,也都要对韩冈低眉顺眼,不敢有所依违。县中上下如臂使指,应付起明年的大灾,韩冈便又多了一份把握。

……………………

“这是在玩火啊!”

文彦博将邸报一下丢到了几案上,王安石处理粮商们的手段,让他嗅到了一丝不妙的味道。

士大夫们没一个能看得上那群攀附着天子,吮吸百姓膏血的裙带官。他们的死活根本不会放在文彦博的心上。只是王安石将他们置于死地的手段,让文彦博深感不安——他竟然是挑拨民意!

在文彦博看来,王安石做得实在有些太过头了。

虽然大臣们为国事而上书时,都少不了带上民心、民意,皆作出一副为民请命的架势。可真要说起将百姓们鼓动起来做事,没有一个会答应。

水能载舟,亦能覆舟。

这个道理有谁不知?民众的聚集,对于统治者来说就代表着危险。

禁淫祀,禁邪.教,推行礼法,宣扬纲常,让治下百姓循规蹈矩,这才是官员们该做的事。

文彦博当年能做上宰相,乃是靠了剿灭贝州王则煽动起来的弥勒教之乱。被煽动起来的百姓有多么恐怖,文彦博比谁都清楚。那些被邪.教蛊惑了的教众,一个个如同疯子一般不顾生死。要不然王则坐困愁城,只占据着小小的一座贝州城,竟然让朝廷的十万大军围攻了数月之久,最后靠着挖掘地道方才破城。

王安石处置粮商们的手法看似痛快淋漓,可这等煽动的手段如果用错了地方,带来的后果必然不堪设想。

但文彦博知道,王安石已经渡过了这一关。裹挟民意之后,如今的宰相已经重新树立起自己的形象。同时在三十七名粮商手中抄没的粮食有一百三十万石之多,而田地、银钱还未统计。这一大案,算的是开国以来净赚最多的一桩案子。对于天子、朝堂来说,多了这些粮食,应对起明年的灾情更多了一份把握。

现在的情况下,甚至连攻击王安石都难。也只有盼着大旱继续下去,才能用天人感应的道理,以及源源不断的流民,将其逐出政事堂——虽然这也算是靠着民心民意,但煽动和利用是两码事,文彦博在心中为自己辩解着。

不过粮商们落得如此下场,京城的豪商们恐怕都要起着兔死狐悲之心。王安石此前已经通过均输法和市易法彻底与豪商们对立起来,这一次下手又如此狠辣,试问哪一家豪商不担心日后王安石会食髓知味,找借口将他们灭门了。

恐惧心能让人发疯,文彦博……深悉这一点。

