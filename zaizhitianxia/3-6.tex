\section{第三章 参商稻粱计(上)}

八月的秦州,平静得紧。

没有外扰,没有内忧。风调雨顺的太平日子,除了树上的知了不停声的在叫着,就没有别的让人烦心的事了。

可知州沈起却是烦躁得脑门上、脸颊上都生了一片疙瘩。听着单调的蝉鸣,他恨不得像京中的殿帅宋守约一样,命人将衙门里的大树上的秋蝉全都给打掉,好让自家的耳边能得一点清净。

这两年来,他无大功,无大过。没在熙河战事中捞到便宜,但也没有被西贼所败,而吃到苦头。前日的德顺军被困,笼竿城既然未破,那他也就没有什么罪名。

安安稳稳,和平安定,这是秦州的三十万大宋子民梦寐以求的生活。

不过这样的安稳,正是沈起所不想看到的。

泾原路的蔡挺走了,在京中做了枢密副使;熙河路的王韶也走了,在京中转眼就要做上枢密副使。

就他沈起还在这里!

看着临近两路的主帅一个个飞黄腾达,沈起心急如焚。喝到嘴里的凉茶,压不下心头的焦躁。遮在头顶上的树荫,只能挡住秋老虎一般的炽烈阳光。

身为边臣,求着盼着的就是军功,要不然他眼巴巴的跑到西北来吃什么苦?!

这鬼地方,春天沙尘,夏天暑热,秋时就要防备着西贼,冬天又冷得厉害。哪比得上东京城的安逸?就算不能留在京中,以他的身份地位,求个江南美地的差遣也非难事。

可他就是贪着泼天的功劳来到了秦州,只盼望着能在此地沾一点韩稚圭的福运,能让他大展拳脚一番。

可惜的是,李师中和郭逵都没能从王韶手上分到的功劳,他同样没有能得到。

河湟那么大的一块饼,熙河路上下吃得差点撑死,却一点也不留给外人。

王韶当了执政,高遵裕成了贵官,韩冈像甩狗屎一般将罗兀、咸阳的功劳全都扔了,还照样升到朝官上——国子监博士!从七品!还有那苗授、王舜臣、王厚、傅勍、赵隆,全都加官进爵,一个个仿佛是腰肋下绑了开封李家的烟火,点了火后就直往天上冲,

而秦州上下,则几乎都要饿死。

钱粮都支援了熙河去,但熙河还是吵着说不够,沈起连续两年的考绩也就是中平。而张守约那边又有多久没升官了?景思立好不容易抢到一个参加河州大战的位置,偏偏还战死了!整整两千秦凤精锐,全都成就了禹臧、仁多两家名声。就一个王存得了个坚守城池的功劳,但退敌的首功还给王舜臣拿走了。

不患寡而患不均。

沈起几十年来,读了那么多遍圣贤书,没有哪一次像现在这般,觉得圣人说得话当真是太有道理了!

站起身,围着院中的老槐绕起了圈子。沈起一身薄纱外袍,背后却都被汗水湿透了。两个侍妾给他打着扇子,都没让他少流点烦出来的热汗。

眼下秦州是打不起来了。会州、会州,秦州北面的会州,柔狼山以南的这一片地,若是打下来,离着兴庆府就没多远了。可眼下常平仓中也没多少存粮,天子更不会支持任何冒险的行为。

沈起的脚步停住。

但熙河却还有机会,湟水之滨的董毡不过是将一个拖油瓶送到了巩州蕃学,并没有表现出让人满意的恭顺之心。而北面的兰州,也同样被并不顺服的禹臧家控制着。

王韶现在离开了熙河,而高遵裕又做不了熙河主帅。如果能抢到这个位置,即便只能派人试探,他都有办法让一场斥候间的战斗,变成连绵一路的血战。到那时,就是他建功立业的机会了。

……嗯……不是他为了自己的加官晋爵而妄开边衅,实在是那些吐蕃蕃人不可相信,应当剿之而后快。安定了吐蕃人之后,才好北上兴庆,平灭西夏。

主意已定。

接下来,沈起要考虑的就是,该如何得到熙河经略这个位子。

‘该走谁的门路呢?’

这是个问题。

……………………

再有十天,秦凤转运使路中报名参加今科锁厅试的官员们,此时已经到得七七八八,或前或后的到了转运司衙门这里报了到。算到最后,就只剩韩冈一人未至。

“韩冈是不是不敢来了?”蔡曚冷言冷语。他在秦凤转运司的时间不多了,已经有消息说,要将其调任到蜀中或是荆湖去。

“大概是有事绊着了。”年初的时候,也就是河州大战期间,蔡延庆在陇西待了不短的时间,多多少少知道一点有关韩冈的情报。“听说他的两个小妾都有孕在身,说不定现下正在等着。”

“原来是个贪恋女色的巫蛊之徒罢了。”蔡曚冷笑了两声。

“韩冈若是只有这么简单,如何能屡立功勋?运判还是不要随意臆测。”

“韩冈擅长捧拍之术,若非如此,如何能三天两头的升官。”

“蔡曚!”

蔡延庆直接叫着僚属的姓名,眼神冷冽。在士大夫的交往中,如果当面直接叫着对方的名讳,那就是很严厉的叱责了。

蔡曚神色也变了,嘿嘿冷笑起来:“转运这般维护韩冈,难道是想着接王韶的手?!”

‘这个时候怎么就聪明起来了?’蔡延庆皱起眉。他的确有意接手熙河经略司,转运之功,绝对比不上一路统帅的功劳。但要想得到这个位置,就必须让天子点头。这其中,王韶等一众熙河官员的发言权将会有着很大的影响力。

只是他口中不能承认:“熙河经略由谁接手,那是天子和政事堂考虑的事。运判未免想得太多了!”

“究竟如何,各自心知。”蔡曚起身,向着蔡延庆一拱手,“下官尚有他事,先行告辞。”

临走出门时,他又回头,“下官既然同判锁厅试,就不会任凭一个滥竽充数之辈混迹于朝堂之上。朝廷抡才大典,也容不得有人将私相授受。”

“运判说的是,自当如此。”蔡延庆,

蔡曚狠狠的一甩袖袍,转身离开。

蔡曚也只有在这个场合,才有机会为难韩冈。出了锁厅试之后,官品已在蔡曚之上的韩冈,根本都不必用眼角瞥他一下。

蔡延庆抿起了嘴。如果给蔡曚坏了事,为了一个贡生资格而跟韩冈结下了仇怨,那还真是冤枉到了极点。

韩冈此人,终究不是池中之物。就算能在这里给他一个绊子,终究也不可能拦住他一辈子。这样的人才,迟早要升上去的。疯了才会与他结下这样的死仇。

何况韩冈的才学并不差,只是与所有的陕西士子一样,拙于诗赋罢了。驻扎在陇西,参加河州大战的时候,蔡延庆与韩冈就有过几次深谈。

从谈话的过程中,能看得出韩冈在经义之上浸淫甚深,并未辱没张横渠的名声。而策问更不必说,见识、眼光就已经决定了他写出的策问的水平,只要稍稍注意一下文字,到了礼部试和殿试时,都不会输给任何人。

与其他一同参加锁厅试的官员的平均水准,韩冈要在锁厅试上得一贡生,根本不是什么难事!

如果蔡曚想在其间下黑手,多半会是落到作茧自缚的笑话。

‘不如就这么做好了’

蔡延庆没有干涉蔡曚的意思,让他自去闹笑话。闹得大了,他蔡延庆再出手相助,这个人情当要卖足!

……………………

辞别了父母,辞别了两个最为亲近的妾室,与照看两个孩儿的云娘打了个招呼,韩冈便启程上路。

从陇西到秦州的两百里地,韩冈只待了两个伴当。熙河经略司中上下,有上百个职位,但其中就是没有一个参加锁厅试,好跟他一起同行。

韩冈的博闻多才,在熙河十分的有名。一听说他要参加锁厅试,原本有心的都各自散了,就没人敢去跟他争位置。锁厅试失败的后果,他们承受不。

一路来到秦州,韩冈在西门前亮出了身份,守门的城门官连忙将他送进了城中。

很长一段时间没有来到这座边陲要郡,韩冈走在路上,都在对比着记忆中的城市和现实的差别。一直走到城中央的衙门前,与几个的没有功名的读书人擦肩而过。

韩冈并没有打算在外面找地方住,他家就在秦州城中,那间小院虽然不大,但布置也足见匠师心中丘壑,不是等闲的人物。

唤了一名伴当将行李送到自家的旧院,韩冈自己仰头而入。同时参加锁厅试的只有区区在内的十来个人,其中还有一张很熟悉的面孔。

慕容武已经有了明经的出身,但他有着更高一层的心思。一看到慕容武,韩冈就会记起他曾经见过一面,就当即魂归道山的凤翔知府李译,那个家伙还真不关他韩冈的事,完全是被疾病打到的。

“思文兄,好久不见!”韩冈上前打着招呼。

“原来是玉昆!”慕容武惊喜无比,他一直都在等着韩冈,现在终于可以说上些话来。他立刻跨前两步,亲热的拉着韩冈的手,“你可终于来了。”

