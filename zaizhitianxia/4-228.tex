\section{第38章 岂与群蚁争毫芒(一)}

“自从三川口、好水川和定川寨三次大败之后,我西军上下厉兵秣马三十载,才有了如今的强势。西夏国势日蹙,而西军的强悍,天南地北都有明证。”

种建中记得自己幼年时,时常能听到祖父壮志难酬的叹息。但如今的西军南征北战的累累功勋,已绝不下于开国之初,跟随太祖南征北战的那一支号称大梁精兵甲天下的十万禁军。

把玩着手上的茶盏,种建中眼神中有着积郁百年的深沉,“已经差不多是时候了,该跟党项人做个了解了。为了种家这一三代宏愿,我们可是等得太久了。”

种家连着三代都投入对西夏的作战之中,王舜臣只要想到这样的种家,耳畔便随之响起了回荡了上百年的金戈铁马。眼下是最好的机会,有着最大的把握,如此千载难逢的良机竟然还有人从中阻挠:“连小范老子【范仲淹】的儿子都得罪了,当然不能再等。”

就是方才王舜臣和种建中所说的,他们跟范仲淹的儿子范纯仁闹翻了,现在已经将其调离了环庆路。

范纯仁知庆州、兼环庆经略,而种诂是环州知州,正是其属下。去年年初,种诂将一批犯了法的熟蕃发配南方,却是在庆州被拦了一下来。范纯仁以此等熟犯罪行查无实据为由,将他们送到宁州羁押起来待,而宁州知州史籍,却是范纯仁大力举荐的人选。

表面上看,这仅仅是一场普通的官场上的权力争斗而已。但本质上,种家和范纯仁之间其实完全是理念之争。当年种谔曾经被人举荐为秦凤都监,范纯仁言其不便。对于开疆拓土的看法,种家和范纯仁截然相反。

当初元昊猖獗,范仲淹曾为庆州知州,在当地甚有威名,赵顼以范纯仁知庆州,便是这个缘故。但范纯仁在廷对时曾推却道,‘陛下若使修缮城垒,爱养百姓,臣策疲驽不敢有辞。若使臣开拓封疆,侵攘夷狄,非臣所長,愿別择才帅。’不过赵顼还硬是将范纯仁调来了庆州。

范纯仁有着这样的态度,自然与种诂和种谔不合,种家诸子皆在边境为官为将,与范纯仁的一干政敌联手起来,将他给赶走了。虽然这一过程中也赔了个种诂进去,但少了束手束脚的挡路石,其实对整体的计划还是有利的。

“攘外必先……安内。”王舜臣也不知从哪里听说过这句成语,“环庆内部算是安靖了,眼下可以往外看。兴庆府里的细作应该为数不少了。”

“份量最重的还是汉人。西贼军中是有汉人的……朝堂中也有。”种建中冷然一笑,“党项猖獗时,他们为党项人做着走马狗,领头南侵。但到了如今,他们不会跟着党项人一起去死的。就是张元吴昊复生,也只有向朝廷低头,求个恩典的份。”

王舜臣兴奋起来,出着主意,“让他们去支持秉常,好好闹上一闹。”

“支持梁氏才对。……秉常希望借重契丹人的力量,他这个契丹女婿现在最想做的事是借了契丹兵来,将梁氏一扫而空。殊不知这个引狼入室的想法,吓跑多少原本支持他及早亲政的朝臣。”种建中冷笑着,在他看来,当今的西夏国主,比石敬瑭还要蠢。契丹人的支持岂可为凭,他们的兵可是那么好借的?不说日后还本了,就是利息以西夏的国库也承受不起。

只听他对王舜臣道:“你这里是环庆第六将,正是处在最前线。如果朝廷要举兵北向,第一个出动的必然有你定边城的四千兵马。”

“到底还要等多久?”王舜臣性急的问道。能成为灭夏大军的先锋之一,自然是难得的荣耀,但举国之战,想挣一个先锋官的身份,并不是那么容易。可不是坐在这里的种建中能私相授受,这份荣耀,想要争夺的为数众多。只有越快定下,自己出任先锋的可能性才会越高。

“快了……很快。”

种建中没有给出一个明确的回复,但对王舜臣来说,已经足够了。种建中的性子向来是言不轻发,他既然这么说,肯定是有所准备的。更重要的是,王舜臣对种家的各项计划都有所了解,不是有了阶段性的成果,种建中不会这么说。

“已经上书朝廷了?”王舜臣追问道。

种建中点头:“五叔前两天刚刚将奏章交马递发去了京城。”

“前两天?!”王舜臣心道,‘还真是一点都不耽搁。’

不过这也不足为奇。种家,尤其是种五,好战的程度当世少有人能匹敌其一二。上书攻取绥德是他,提请进筑罗兀的是他,两年前要攻略横山的是他,现在叫嚣着要灭亡西夏的也是他。

继承了父亲种世衡的遗志,种谔份外看不得眼下这种虚伪的和平时光。现在外界都说法是种谔不死、兵事未已。但西军上下,连同关西的百姓,却都想着能早点将西夏给灭掉,还陕西一个真正的太平盛世。

“五叔已经上书了,”种建中眉宇中满载着兴奋:“只要天子下定决心,朝廷批复下来,最多只要一年的筹划和准备,到了明年就能举旗北向,杀奔兴灵了。”

“不知到时候,兴庆府里面还能不能争出个结果来。”王舜臣明显的想看着梁太后和秉常母子之间的好戏。

种建中笑了:“前几天,去兴庆府的商队回来说,帝后两派斗得是越来越凶了,说不准什么时候就撕破了脸皮下来。”

王舜臣点点头,端起茶杯喝了一口。像是想到了什么,脸色忽然一变,放下杯子就问道:“……要不要紧?”

“什么?”种建中一时没有明白过来。

“现在还派商队去跟党项人做买卖,这件事到底要不要紧。”王舜臣为种家现在还派人去西夏国中而担心,虽然是为了打探敌情顺便做个生意,但不代表没人会抓着此事而做文章。

“怕什么?”种建中满不在意,“没有商队去兴庆府打探,我们哪有可能坐在这里谈天说地。”

尽管一直都想着灭亡西夏,但这并不影响种家跟党项人做买卖。参与回易的种家商队之所以最近人数少了一些,并不是种家正上书要准备与西夏决战,所以有所收敛,而是对面能拿来交换的财物越来越少的缘故,即便是出产自青白盐池的池盐,最近也是越来越卖不上价了。

生意是生意,公事是公事。光靠俸禄和陕西贫瘠上的田地是维系不了一个将门世家的日常开支。更别说收买对方部族,探查西夏国内军政,都是靠着派出去的诸多商队带回来的情报和钱财。

作为大宋不多的几个将门世家的成员之一,种家上下都很清楚,要想维系家门不堕,依靠的不仅仅是官职、土地、财产、子弟、门客、戚里,其所掌握的敌情和密探的资源也是关键,是能带来胜利的法宝。也只有能带领麾下将士为天子不断夺取一个个胜利,才是永保家门的唯一方法。

兼职做着间谍任务的商行,并不止种家一家。就王舜臣所了解的,把持熙河路大半商事的高、王、韩三家的商行,已经将手伸到了河西。凉州、瓜州和甘州,旧日属于已经覆亡的六谷部的吐蕃人,现在通过一支支商队,纷纷暗中缔结了投效的约定,只要官军打下兰州,攻克河西门户的洪池岭【乌鞘岭】,就会立刻起事,将党项人从甘、凉诸州给赶出去。至于兰州,党项人堆在城中的军队已经超过一万,但只要官军有意北进,禹臧花麻立刻就会里应外合,将城门献上来。

熙河那里已经准备好了,眼下环庆路这边也准备好了,以王舜臣的经历和身份,两边都可以任选。不论在庆州立功,还是在熙河立功,王舜臣也都不在意,他只求能上阵杀敌,并不会对地点挑三拣四。

一番话说着,王舜臣的下属已经在他的吩咐下安排好了宴席。心情大好的王舜臣请了种建中入席。

“知道新任的庆州知州是谁吗?”喝了两杯酒,种建中忽然问道。

“谁?”

“高遵裕。”

“怎么是他?!”王舜臣惊道。他在熙河路,深悉高遵裕的脾气,也是个贪求功名的主。有他在,联手撺掇天子攻打西夏,应该是又多了两分成算,但同样是因为有高遵裕在,种谔可是在北攻西夏的时候,难有大展拳脚的机会。

种建中对此并不在意:“世事难如意,但至少高遵裕不会拦着不让打西夏,总比范尧夫要强。高遵裕好对付,谁压谁还说不定。至于范尧夫,就留给韩玉昆去头疼吧。”

王舜臣迷糊起来:“……范经略跟韩三哥怎么拉上关系了?”

“韩玉昆不是在京西吗?”种建中笑了笑,“朝廷降罪范尧夫,是落直龙图阁,责知信阳军。”

“京西的信阳?”王舜臣立刻问道。

“还有哪里的信阳?”种建中脸上多了点幸灾乐祸的笑容,“范尧夫在环庆路这里让我们整整吃了两年苦,也该韩玉昆尝尝滋味了。”

