\section{第29章 百虑救灾伤(十)}

“诸立,你可知现在白马县的粮价。”白马县衙的花厅中,韩冈问着垂手站在厅中央的衙中押司。

诸立腰更弯了一点,谦卑的答道:“小人知道。”

“眼下都已经是腊月十九,粮价却还是一百三十五文一斗。再这样下去,县中百姓的年节可就没法儿过了。”

诸立保持着沉默,并不接口,等着韩冈继续。

“想必你也听说了,如今南面的纲粮已经运抵东京城,不但在京中发卖,也会散给京畿诸县。白马县这边有一天三百石的定额。纲粮从东京运过来,也就接下来一两天的事情,可以说粮价很快就要跌下去了。”

“听说是多亏了正言的发明。”

“粮价既然要降,就不能让其再涨上来。本县有意发文,将白马县中的米价定为八十文一斗。为防有人为奸,一人一次只能购买一斗。诸立你是县中最大的一家米行东主,不知你能不能当先做出个表率?”韩冈顿了一顿,又道,“……本官也不占你便宜,只要你愿意打这个头,本官可以在你家明年的税赋加以减免。而且卖出多少,等纲粮抵达后,我就补还给你多少。”

诸立低下头去,掩起脸上的冷笑,不让韩冈和他的三位幕僚看到。

白马县离着东京城有一百多里地,但诸立他与行会联络得勤力的很,消息日日传递往来。东京城眼下是什么样的情况,他心里都有数。

韩冈担心县中百姓过不好年,几乎是强逼着自己给粮食降价。但诸立觉得这位年轻的白马知县,现在更要操心的应是他的岳父才是。

发运司辛苦从南边运来的粮食,大部分都给官户买走了。几处市易务卖粮的地方,都是排起了一里长的长队。排上一天,就只能买上一斗粮,百姓原本的期待都化成了怨气,可是眼见着就要爆发了。

不过就是因为王安石现在已经陷入绝境,诸立才不会蠢到跟韩冈硬顶。别看此时韩冈和颜悦色,好言好语。如果自己不点头,保不准王相公的好女婿就会用上强硬的手段,以维护自家的威信。要是在快成功的时候,被当成杀给猴子看的鸡,那未免就太冤了一点。

低头弯腰,拱手行礼,诸立毕恭毕敬、老老实实的说道:“正言说什么,小人就做什么。正言让小人将粮价降下来,小人回去后就就将水牌全改了,一陌一斗。”

一陌是七十八文,比起韩冈的要求还低了两文。诸立此举可谓是老实听话。

但将店里的存粮低价卖光又如何?诸立根本就不在意!

他早就将手头上的大多数粮食都存放在乡下的庄子上,以待明年开春——基本上粮商们都是将粮仓放在城外,要是全囤于城中,别的不说,这租地存粮的地皮钱就要吞吃很大的一部分利润——老实听命的卖光了店中的几百石米面,不信韩冈还能有借口去他庄子上抄家去!至于补还什么的,有最好,若是没有,看看韩冈还有脸再对自己要求什么。

而韩冈似乎没有看出来诸立的小心思,对他的回答很是满意:“如此最好,还望你尽快施行。”

诸立恭声答诺,告辞退了下去。

看着诸立离开的背影,方兴立刻转过身来:“正言,诸立答应得如此爽快,其中必然有诈!”

韩冈嘴角扯动了一下,像是在笑,但眼神冷得如同厅外池塘中的寒冰:“这一点我当然知道。”

阳奉阴违的事谁不会做,就算不违背自己的命令,韩冈也能为诸立想出许多变通的办法。

“看正言的样子已经是胸有成竹,想必对此局面早有所料,也做好了应对了吧?”魏平真微微一笑,问着韩冈,方兴和游醇都望了过来。

韩冈点头:“是有些措施,日前王元泽过来的时候,就已经就此商议过。”

现在京城粮价的问题很麻烦。在粮商们卖力的做着绊脚石的时候,想要赶在年节前将粮价降下去,就必须一口气放出大量存粮。

大灾还在延续,加上一直以来的徘徊在高位的粮价,哪家哪户不担心日后断粮,都想多买一些存在家里。虽然一天一万五千石的数额,用来供给百万军民其实勉强也够了,但架不住人人都想多买一点。

韩冈为此估算过——也让魏平真算过——想要用卖粮来平抑粮价,少说也要一下散出百万石储备粮,甚至两百万石,这样才能将高高在上的粮价一下打垮。如现在这般细水长流式的零卖,根本无济于事。东京军民百万,官户买一点、富户买一点,贫户再买一点,一天一两万石转眼就瓜分干净了。

所以有着宗室撑腰的粮商们,能稳如泰山的将粮价保持在高位上,就是在逼着王安石开常平仓。常平仓一旦敞开,他们立刻就会降价。

不过对于眼前的窘境,王安石、王雱、韩冈,还有新党一众,都不是没有预计过。相应的应对招数,皆有所准备。

官与商之间的争斗延续了几千年。官员遇上的并不一定都是没有后台背景的商人,官商才是最为普遍的情况。怎么化解有着宗室背景的商人们的攻击,新党自然有着未雨绸缪的计划。韩冈对诸立的一番话,也不过是计划中的一环罢了。

对上三对好奇的目光,韩冈笑了一笑,“这时候也不用瞒着你们了。办法很简单,就是将所有运抵京城的纲粮都平价卖给粮商,由他们转售。”

……好让绊脚石不再成为绊脚石。

……………………

“卖给粮商?!”

吕惠卿此言一出,顿时满堂大哗。虽然有朝规在上,许多官员都忍不住发出低低的惊讶。

御史中丞邓绾霍然起立,从他位于殿门后的小交椅上站起来,恶狠狠地一扫殿中,“君前何敢喧哗!?当知失仪之罪!”

也只有绳纠百官的御史可以在朝会上大声插话,弹压众官。

御史台长一怒之威,殿上顿时安静了下来。但人们心中的疑惑却难以消弭。

只听吕惠卿继续说道:“如今百姓欲购官粮,只有几处可去,往往要自朝至晚,方能买到一斗。如此粮价如何能降。所以以微臣之见,不如命市易务将新近上运的纲粮以七十文一斗卖与粮商。而将东京内外的米价一律定为八十文一斗。此十文的差别,便是给付粮商的代售之费。”

这是妥协!这是退让!

听到吕惠卿的一番建议之后,每一位大臣都是如此在想。看到没办法将粮价打压下去,王安石为保权位,便去卖好那些奸商!

一斗让利十文,一石就让利百文,每天的一万五千石那就是一千五百贯,如果持续两个月差不多接近十万贯。王安石授意吕惠卿将十万贯全送给粮商,拿着朝廷的钱财来买下这一干与宗室勾结的奸商不再发难!

立刻就有人站出来,“坐视奸商盘剥百姓而不制,反与其同流合污。此乃奸邪之举!”

就连冯京一时间也疑惑起来,‘王安石这是要跟粮商们媾和?!’

‘此乃与虎谋皮!’吴充暗自摇头,不意王安石如此不智。十万贯争如百万贯?恐怕粮食落到那些奸商手中,就由不得王安石来做主了。

但他们将视线投往站在最前面的王安石身上,严肃沉重的一如既往。原本的判断却渐渐动摇,这根本不符合王安石的为人!

忽然他们心中闪过一丝明悟:‘难道……’

……………………

听韩冈说完,一阵静默之后,魏平真突然叹道:“王相公和正言的这一番谋划,甚有深意啊!”

游醇和方兴都点着头,完全同意魏平真的说法。几个月的相处,使得三人已经了解韩冈的脾性,知道他绝不会向粮商们低头服输。具体会怎么做,他们其实已经可以猜测得出来了。

韩冈笑道:“如此作为,也只是为了四个字而已。”

游醇立刻问道:“可是仁至义尽?”

“是欲取先与吧?”方兴说道。

魏平真沉声道:“乃是骄兵之计。”

韩冈呵呵笑了两声,却不正面回答谁对谁错,“很快答案就会揭晓,三位还是拭目以待吧!不管怎么说,既然那一干粮商挑起了战争,就只有你死我活一个结果。”

韩冈虽然语带笑意,但说得内容却让魏平真三人仿佛有一阵寒流来袭。

——韩冈竟然将粮价之争定义为战争!

韩冈在这次反击的计划中,所起的作用绝对不小。他说的话,基本上就可以说是王安石的意思。既然是战争,那就如韩冈方才所言,结果只有你死我活!这代表着王安石,绝不会对粮商们宽纵半分。

天色将晚,韩冈送了魏平真三人离开,又回到花厅中坐下。他们的回答其实都沾边,但只是对所用手段的评价,并没有说到本质。

宁静的花厅中,火盆内的木炭燃着幽蓝的火光。偶尔有木炭在火中噼啪一声,除此之外再无杂音,只有韩冈的声音低低:“其实裹挟民意更恰当一点啊!”

