\section{第27章 京师望远只千里(九)}

【这是补昨夜的一章,今天还有两章。】

天寒地冻,无定河已经被冻透了底,绥德城外亦是白雪茫茫。

种朴给冻僵的双手呵着气,从城门后的阶梯走上城头。翻修不过两年的绥德城城墙上的积雪已经扫清,露出了仍是黄姜色的夯土地面。堂弟种建中正拿着一封信站在城上,扶着雉堞,望着城外的眼神有些呆滞,许久也不动弹一下。

“十九!怎么在发呆?不冷啊!”种朴大喊着,砰砰的跺着脚,对冒着风站在城头上的种建中,感到很不理解。

种建中回过神来,收起了手上的信,回头笑道:“怎么会不冷!”

“真冷就不会傻站着了。”走到种建中身边,陪着堂弟一起望着漫山的雪景,种朴侧过脸问道:“又是你同学捎来的信?”

种建中摇了摇头:“是子厚先生的信。”

听到横渠先生的名号,种朴神色肃穆了几分,“横渠先生有说什么?……是不是罗兀城的事。”

种建中笑了笑,摇摇头:“子厚先生不会在私信里论公事的,什么也没说,只是叮嘱要多读书,不要误了功课罢了……”停了一下,他又补充道,“子厚先生现在已经辞职归乡,应该还不知道兵出罗兀的消息是真是假。”

“你的那个景叔兄可是知道的一清二楚。”

“游景叔可是邠州军判啊……西贼耳目所寄如今都放在鄜延一地,要将其引开,少不得靠环庆、泾原和秦凤三路帮忙。现今环庆路那里动作不断,游景叔怎么可能不清楚其中的内情?”

种朴伸手掸了掸面前雉堞上的残雪,双手撑着就坐了上去,返身冲堂弟冷笑着:“也就他会好心来劝,其他人都在想看我们的笑话呢!”

种建中叹了口气,如今尸位餐素者遍布朝堂内外,因循苟且者众,想要进取一番,都会被各种各样的阻碍所束缚。如今有当朝首相坐镇后方,干扰是没有了,但想看笑话也就更多了。

不过那些或明或暗的反对者不是没有道理。

只要略通兵法,稍悉地理,就知道在罗兀筑城的风险究竟有多大,等于是把全部身家放在赌桌上,而且不是赌单双、比大小,而是几个铜板一起扔,要丢出个同面的浑纯出来。

但换个角度去想,也就因为这个战略实在太过冒险,所以才没人会相信。真正得到消息,明确的知道韩绛领下的西军将会兵行险招的,其实寥寥无几。

西贼也绝不可能想象得到,一直行事保守的大宋官军,会胆大到沿着无定河突进六十里!

出兵几十里去敌国打草谷很容易,都是倏去倏回,见到情势不妙,转身就能跑掉。可是在敌境修造寨堡,却要动用大量的民伕、厢军,要守卫工地最少几十天的时间,这对领军将领的压力,对出战大军的压力,不言而喻。

自从元昊起兵反叛以来,大宋用兵从没有这般大胆过。从来都是在自家控制区内侧几十里的战略地点,修筑核心城寨。而附属于这些核心城寨的寨堡、烽堠,才会放在控制区的边缘地带。至于向西夏一侧深入修筑寨堡,基本上都没有过几次。而一举前进六十里,这种疯狂,没人能相信。

虽然修筑罗兀的流言已经传遍了关西,可有人相信吗?在横山南北流传的谣言数不胜数,要想在这些无穷无尽的谣言中寻找到真相,就跟在海岸边的沙砾上寻找珍珠一样困难。

西夏人不会相信的,前段时间在罗兀修筑的与烽堠没两样的百步小堡更是证明了这一点。如果梁乙埋真的确认了官军的计划,至少也要打造出一个能驻军千人的大寨。

出其不意,原本也许只有十分之一的成功几率,现在却至少有一半的可能能得胜归来。

“他们怎么就没想到,一旦夺取并守住了罗兀城,横山蕃部有多少还会继续跟着党项人?”

“他们不知道,党项人年年在横山蕃部中点集大军南下,横山蕃又有几家没有怨心?”

“罗兀一落,西贼就再无翻身之力。”

“打仗哪有不冒风险的。要不输很容易,一辈子窝在家里。如果要取胜,当然冒风险。李愬雪夜下蔡州,难道不是冒险?继迁逆贼袭银州,难道不是冒险?不还都给他们赢了。”

“天子、中书都支持此战,钱粮充裕,兵马精熟,西贼防备不高,没有比着眼下再好的的局面了。如果今次错过了,十年内不会有更好的机会了。”

种建中还能记得种谔当初是怎样的慷慨陈词。一向话语不多、威严冷峻的五叔,前日见过韩绛后,难得喝醉了:“燕达本是吾之副将,现今却成了秦凤路副总管。燕达跟着郭逵的青云直上,你以为韩相公会看得惯?只要今次成事,我也能……只要今次成事……”

种朴的声音打断了种建中的回忆:“……今次配属在大人麾下的,总计两万精锐。如果能一举攻下罗兀,河东军至少能派来过万人马支援。再加上各路配合进军的兵力,是实打实的十万大军!”

种朴眼睛发亮,话声中透着少有兴奋:“十万啊……真正的十万可战之兵!可不是随随便便就能碰上的。”

种建中点着头。他经常在史书上看到一场小小的会战,双峰动辄出动十数万、数十万大军的记录。但作为出身将门世家,现在实际参与军务的新生代将领,很清楚那些记录根本不靠谱。

在一个小小的州县中聚集十万以上的军团,要消耗多少粮食,多少草料,配属的民伕要有多少,征发的牲畜又该有多少,驻军的营盘该有多大,互相之间将如何联络,这等实际上的难题,不是不通兵事的史官拍拍脑袋就能解决的。

事实上,能有三五万可战之兵,天下都去得了。

如今次在没有水道运送粮草的西北山区,出动十万大军,无论人力物力,都几乎达到了陕西能承受的极限了。今次若败,就如种谔所说,十年内都难有这么好的机会了。

从城中突然响起蹄声,一名骑兵直奔种朴和种建中两兄弟所在的城墙而来,“两位小将军!高、折二将军已经到了,太尉请两位速速回衙。”

高永能,折继世,种谔的两个副将都到了。

“终于到了!”种朴哈的一声跳下来,拍了拍身上的雪片,搂着种建中的肩膀,“走!十九,我们去见两位将军去。”

……………………

从宫中回到驿馆,已经是午后时分。

抵京一个多月,王韶这已是第四次被召入宫中。与他儿子当初入京时的情况一样,受到了天子超乎一般的重视,引得京中人人侧目,还有羡慕。

王韶并不着急回去。如今的缘边安抚司刚刚经历过大战不久,无论内事外事,都不会有什么问题。

另一方面,如果真的出问题了,对他来说也不是一件坏事。可以让天子知道,河湟少不了他王韶。

只是王韶的心情还是不好,因为韩冈的事。今天他在宫中刚刚听说,韩绛第二次的上书天子,要把韩冈调去延州任职。

如果只有一份请调的奏文,韩冈完全可以辞去。照常理说除非是受到贬责,否则文臣对于官职不满意,有权不接受,也没人会去强迫他接受。可是韩绛接连上了两份奏文,表现得恳切如此,韩冈再想拒绝,事情就不会那么简单了,天子和王安石那边都少不了施加压力。而韩冈本人,想来也不会冒着激怒皇帝和两位宰相的风险。

宰相韩绛宣抚陕西,以他的身份,当然是什么都能要到最好的。只要他觉得能派得上用场,提上一句,无论人和物,都会源源不断的送到他的面前。王韶看着天子和朝堂的重心都放在横山,就像一个妾养的庶子,看到受到父母宠爱的嫡兄时的感觉。

计算时日,韩冈抵京也就在这两天了。王韶曾想派人先去通个气,顺便问问韩冈的心意。但他个人派出去的信使,怎么可能跟朝廷的马递较量速度。恐怕人还没到半路,韩冈就已经离开了秦州。所以他只能静等韩冈抵京后,再与他联络。

王韶心中不痛快,回到房中,命人不要打扰。便拿出笔墨纸砚,练起字来,这是他平日消减心头怒气的做法。只是刚刚把墨磨好,房门又被敲响。王韶不快的抬起头,“什么事?”

“安抚,有人在外求见。”

“是谁?”

“是韩机宜的表兄李信。”

王韶一下丢了笔,“快让他进来!”

李信累得够呛,灰头土脸的,甚至都没有来得及擦洗。但在见到王韶的时候,动作仍旧稳稳当当,渊亭嶽峙。

李信从怀里掏出了一封信笺,双手呈给了王韶,“小人表弟在京兆府听说今次被召入京,是为了调任延州,心中不安,所以就让小人连夜赶来跟安抚联络。”

王韶先是一愣,“原来玉昆已经知道了。”转而又惊讶起来,不知这李信是怎么赶来的,若是走的驿站,韩冈哪里弄来的多余驿券?

不过李信怎么来的是小事,韩冈派他来的做法,才是王韶在意的关键。

这是韩冈在表明态度。奉命入京的官员,基本上不可能抵京的当天就去中书候命,至少也要在驿馆里歇息一夜。有这个时间,什么不能商量?但韩冈还是不嫌麻烦的把李信提前派了过来——表现了他以王韶马首是瞻的态度。

王韶的心情好了许多,展开信,细细审读起来。

