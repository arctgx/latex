\section{第13章 羽檄飞符遥相系(四)}

【承诺中的第三更】

“是的。”嵬名济板起来的一张脸,正经八百的,看不到半点戏谑的笑意。他重复着:“梁乙埋狼子野心,试图夺权篡位,刺杀了太后。”

眨了几下眼睛,嵬名秉常终于明白过来了,他立刻追问,“梁乙埋呢?!”甚至没有去关心他的母亲。

“当然是被杀了。”嵬名济厉声道:“如此大逆不道的贼子,如何能留他!”

秉常低下了头,肩膀耸动起来,捂着嘴,呵呵的低笑藏在掌心中。肩头随着低低的笑而抽动,过了半天,压抑在喉间的笑声终于忍耐不住,爆发了出来,他哈哈的狂笑起来,“杀得好!杀得好啊!……”

心中的狂喜再也忍耐不住,大白髙国的皇帝猛然跳起来,挥舞着双手,高呼乱叫,“杀得好!杀得好!”

嵬名济静静的等待着,等待他的皇帝将几年来被囚禁的怨气发泄出来。等了许久,却也不见秉常的疯狂有个休止。

“陛下。辽军南下了!”嵬名济提声叫着在御帐中欢跳的皇帝:“契丹人从黑山威福军司南下,数万大军往兴庆府杀过来了。”

“什么?”秉常的笑声戛然而止,心中的狂喜也不见了踪影。

“辽军南下了。”嵬名济重复着这个噩耗,“契丹人趁我们跟宋人决战的时候从背后捅了我们一刀。耶律乙辛之前对我们全力支持,但实际上却是一直是想灭了大白髙国!”

嵬名秉常呆愣着,像是不能理解嵬名济的话,又像是被这个消息吓得怔住了。

“陛下勿须忧虑。当年李继捧献地附宋,只有太祖皇帝坚持不降,身边就只剩十一人,要躲在棺材里才逃出了夏州城,躲进了地斤泽。之后也历尽坎坷,连太后都被捉去了东京城。但最后呢,得了银州、得了灵州,最后打下了这么大的一片基业。都是太祖坚持到底的结果。”嵬名济激励着他的皇帝,“只要我嵬名家的大军还在,只要陛下能坚持到底,日后必然能恢复国土,恢复旧日的荣光,像景宗皇帝一样,吓破宋人和契丹人的胆!”

秉常一直愣愣的,嵬名济的一番长篇大论之后,好半天才反应过来,摇起了头,仿佛听到了一个很好笑的笑话,笑了起来,“辽人来攻兴灵?辽人是来帮朕的!朕要回兴庆府,有辽人来帮朕,朕才不需要去地斤泽!”

“陛下,契丹人真要是来帮陛下,肯定要先派人来联络,怎么可能一句话都没有就杀进来了?他们是想要我们的土地啊!黑山下的河间牧场地已经被抢走了,现在他们想要的是兴灵,是兴庆府,是大白高国的都城!他们不是来帮陛下你的!”

嵬名济的当头一棒,让嵬名秉常刚刚直起的腰身又弯了下去。他愣愣的发了好一阵的呆,才期期艾艾的说道:“真的不行,就投降契丹吧。”

嵬名济整个人都僵住了,浑身如同被梦魇住时那般无法动弹。他不能相信自己的耳朵:“投降契丹?兀卒,你是要投降契丹。”

秉常不喜欢别人用党项语叫他兀卒,译成汉语是清天子的意思,他只喜欢臣子们用汉语称呼他陛下和官家,但这时候,嵬名济已经理会不了那么多了。他目瞪口呆,愣然的看着秉常。

“朕是大辽宣宗皇帝的女婿,是驸马,就是到了临潢府,也该给朕一间宅子才是。”几年的囚笼生活,已经消磨光了嵬名秉常的锐气,他颓然的叹气道:“兴庆府既然是辽人想要,那就给他们好了。朕要做个安乐公,耶律太师总不会赶尽杀绝。”

嵬名济还想辅佐秉常,中兴大白髙国。遵循太祖皇帝李继迁的榜样,以图东山再起。但出现在他眼前的却是一个说胡话、没志气的天子。像一盆夹着冰的冷水,将嵬名济给泼醒。又像是一柄重锤,将他刚刚腾起的美梦,击得粉碎。

“大白髙国的确是完了,是完了啊!!!”

嵬名济仰天狂叫一声,倏然站起了身子,在近处俯视着身材瘦弱、脸青唇白的大夏国君,一股子戾气涌上心头,‘既然如此,就没必要留着他了。’

嵬名济低头看了看,没有白绫,只有系着外袍的一条丝绦,‘够用了。’

看到嵬名济探手解下系着甲胄外袍的丝绦,秉常心中腾起了一股不祥的预感:“嵬名济,你要做什么?!”

嵬名济默不作声,一步步的逼向前,只是将长长的丝绦两端缠在双手上,留下中间的两尺。两只手骨节凸出的手紧紧握着拳头,青筋根根迸起。

他杀了梁氏兄妹,要助秉常复辟,恢复大白髙国的旧日荣光,可现实又是怎样?看看这个让人恶心的东西,要有太祖、太宗和景宗的一成能耐,就不会叹着气要投靠辽人了。

“嵬名济,你到底要做什么?!”秉常质问的声音尖利的如同女人。

“干什么?”嵬名济攥着丝绦,面目狰狞,“陛下你安心去吧。大白髙国既然要亡了,你自尽殉国,也是尽了天子的本分。”

“逆贼!!!!”

嵬名秉常肝胆俱裂,他看得出嵬名济绝不是在开玩笑。一声尖叫,他猛然冲前,求生时生出的一股子蛮力,竟然将身高体壮的嵬名济一下撞开,趁势就冲了出去。

守在御帐外的都是嵬名济的亲卫,原本被梁氏兄妹派来的看守,都被他们全部清理干净,一个个手握斩马刀默默肃立。虽然听着里面似乎有争吵声,但隔了一层牛皮帐,里面的声音已经模糊不清,当秉常冲出来的时候,一个个都措手不及。而更外围的士兵们更是惊讶,天子怎么逃了出来?

嵬名济铁青着脸紧追着出了大帐,劈手从门口的亲卫手中夺下一柄斩马刀来。

瞪着在前面跌跌撞撞奔逃的皇帝,嵬名济心头的怒火越燃越烈,烧得眼前一片血红,热得脑中只存下一片杀意。手中长刀一紧,三步并作两步,重重的几步追到秉常身后。他腰部反拧,全身的气力都鼓了起来,就这么向前用力将长刀向前一挥——

弧月般的刀光闪过,奔跑中的人影一刀两断!

御帐前,刹那间安静了下来。

暴怒下的愤然一击,汇集腿力、腰力和臂力,爆发出了难以置信的力量,将年轻的西夏国主劈成两段,两截身子从中折分,啪的落在了地面上。

秉常趴在地上,似乎还不知道自己身体的变化,努力的向前爬着,一边还回头哭叫道:“不要过来!不要过来!”

千百支火炬的照耀下,数百人呆然地看着他们的皇帝哭喊着,用手撑着半截身体一下一下的向前挪动,长长的肠子也随之一点点的从身后漏了出来。

在黄赤色的火光中,鲜红的血也仿佛是黑色的。浓浓的黑,就像如椽大笔,饱蘸浓墨后在地面上划了一道。

思维和空气仿佛同时凝固,没有一个人能反应得过来。就这么眼睁睁的看着秉常爬行,倒下,挣扎,最后一动不动。

不知过了多久,微风掠过,一个尖利的叫声击碎了沉寂:“他杀了兀卒!”

而后千百人一起惊叫:

“他杀了兀卒!”

“他杀了兀卒!!”

“他杀了兀卒!!!”

秉常再如何不好,也是西夏的皇帝,如果在帐中被勒死,没外人看到,出去后报了暴毙倒也罢了。但现在是嵬名济于众目睽睽之下亲手将他一刀两段,事情已经变得不可收拾。

在一片喊声中,狂怒中的嵬名济终于回过神来。他所侍奉的主君半截身子趴在了血泊中,另外的半截则远远的掉在了后面,而做下这一切的长刀,却在自己的双手中。

千百支火炬照得周围亮如白昼,千百只眼睛看着拿着刀的自己,也不知有多少人亲眼看见自己一刀将皇帝劈成两端。

宗室中的重镇,以嵬名为姓的宿将,终于承受不了这样的压力。一声狂叫,远远的抛下了手中的斩马刀,抱头狂奔。没有人敢拦着他,就这样看着嵬名济冲出了人群。

阴暗的角落里,吴逵双眸映照着火光,亮如星辰。他没料到带着一封信来,却看到了这一幕好戏。亲眼看见臣子弑君,这可不是随随便便就能看到的戏码。

西夏国主就死在御帐之前,没有人将注意力投注到吴逵这一边,但弑君的凶器,却被嵬名济丢到了他的面前。吴逵双眼扫了一下周围,见过没人注意,脚往前一伸,脚尖一动,便将那柄斩马刀挑到了自己手中。

夹钢打造的锋刃经过了精心的打磨上油,锋利无比,在火光下还莹莹泛着青光,甚至没有沾染多少血液,也难怪能将一个大活人拦腰斩成两段。

这是标准的大宋军器监造的斩马刀,弧度,长度、宽度、重量,皆有定制。几万把刀放在一起,都不会有什么差别。

在刀面上的最下方,有着几列小字,凿上去的,歪歪斜斜。这是制刀工匠和监造者的姓名,以及时间。辽国和西夏都仿造过宋军制式的斩马刀,但没有一个会在刀上留下以待追查的记认。

就着火光,能隐约看得清这些字,四字一列:熙宁八年;六月壬子;上工魏申;锻造何安;监造臧樟。

只有最后一行是五个字:‘判军器监韩’。

吴逵啧了下嘴,竟还是熟人。

