\section{第一章 庙堂纷纷策平戎(11)}

【写得比较慢,接下来的两更,放在明天。】

韩冈脚步顿了一下,随即又举步向前。看似平静,随性问道:“哦,是什么时候的事?”

韩冈的口气有几分无礼,王珪不以为忤。韩冈和王舜臣的关系并不是什么秘密,而且他现在的反应也证明了一些猜测。

‘事前当不知道此事。’王珪侧脸注视着韩冈面上细微的变化,‘多半与郭逵没有联系。’

王珪虽以三旨相公著称于世,可没几分察言观色的本事,光说取圣旨、领圣旨、已得圣旨,就能走到宰相之位上?笑话!

那不过是世人嫉妒罢了。王珪从不认为自己有哪里错了,天子的看重才是一切。

自家的诗作因为用金玉富贵之词多了点,就被亲兄长称为至宝丹,士林中也多有嘲笑。但中秋入宫写应制诗,能从后宫嫔妃那里得到满满两袖子的润笔,也就他王珪一人。‘寒蝉凄切、对长亭晚’、‘忍把浮名、换作浅斟低唱’,如此落魄寒酸,可配得上宫中的富丽堂皇。

被人称作三旨相公又如何?到了路上,谁敢不给他让路?满朝文武,又有谁能当得起自己的一礼。只要自家还在宰相之位上,一切的批驳,都是源自嫉妒的诽毁之言。

“那是三年前攻取横山时的事了。”确认了韩冈的不知情,王珪就淡然一笑,“说是王舜臣当时领军北上,灭了沿途的两个部族,以老弱首级充做军功。”

“三年前的事,至今才报上来?”

王珪点头道:“今晨的急报入京!”

“……这还真是巧了。”韩冈声音低沉了下去。

这是看上了王舜臣的位置吧。有资格争夺鄜延路先锋官的将校,跟王舜臣肯定是抬头不见低头见,日常也是称兄道弟。如今为了抢功,脸皮都撕破了。

“的确是巧。”王珪点点头:“早不报、晚不报,偏偏是这个时候。”

韩冈说得巧,并不仅仅是王珪说的那一层,还有郭逵。不过因为是今天的急报,巧合的几率应该占到八成。另外赵顼知道自己跟王舜臣的关系,竟然一点都不提,倒是让韩冈有些恼火。

“这件事可是要彻查?”韩冈又问道。

王珪道:“因为涉及到杀良冒功,天子震怒,的确要下诏狱彻查。不过郭仲通出来说,谎报军功涉及人数众多,大战将临,不宜动摇军心。所以最后是不予深究,只将王舜臣夺官。”

也就是不用查就已经认定了王舜臣的罪名。不过韩冈倒没有为王舜臣喊冤的打算。他不敢帮王舜臣打包票。横山大战前后,王舜臣写来的信中,有不少抱怨,说是没有立功的机会。从王舜臣的信中来看,再加上韩冈对他的了解,谎报军功的事,王舜臣做得出来。

“可是要他戴罪立功?”韩冈继续询问。

“天子震怒啊。”王珪回头对韩冈意味深长的笑了笑。

他倒想看看韩冈到底不会不会为王舜臣争取出战赎罪的机会。

韩冈默然不语,随着王珪的脚步继续往前走着。

迎面来的侍卫和内侍,见到王珪,远远的就叩拜行礼。都没有看到王珪正惊异的抬了下眉毛,然后就是一幅若有所思的表情。

郭逵帮王舜臣争到了一个不深究其罪,仅仅夺官的待遇,但上阵立功的机会却没帮他争取。想来韩冈应该帮的——所以前面天子根本就没提王舜臣,省得听韩冈为其辩解——但韩冈却没有。

韩冈在后面没有看到王珪的神色变化。也许在他人看来王舜臣,但对韩冈来说,只要保住了性命,没什么大不了的。过个两年,凭自己的面子再敲敲边鼓,转眼就能升回来。

“让他吃点苦头也好。”沉默的走了一阵,韩冈开口道,“省得不知天高地厚、军律森严,日后犯了大错,想悔改都没机会了。”

“说得也是,玉昆能想得开也是好事。”王珪语重心长起来,“王舜臣少年成名,是西军中的名将。如今虽然犯法受责,切不可自弃,一心奉公,日后必有再起之时。”

“韩冈必会将相公的话转述给王舜臣。”韩冈诚挚的说道,“能得相公的教诲,王舜臣定会感激涕零。”

谎报军功,甚至可能是杀良冒功,这件事在两名重臣眼中,的确算不得是什么大事。都不会认为这件事能让王舜臣一蹶不振。

虚报军功其实是随大流,基本上没有不这么做的,而且朝廷私底下也有鼓励。这是炫耀功绩振奋人心的手段,尤其是在仁宗的后半段,经常有防守住几万十几万的党项大军攻势的战报,然后斩首个三五级、十几级。当真算不得什么大事。

至于杀良冒功的问题,性质比较严重,韩冈并不能说没有。以王舜臣的性格,在攻略横山的时候,顺手屠两个部族充功劳,也不是不可能。但要说横山蕃部是良民,陕西的猪都要笑了。

陕西缘边,不知有多少人与横山蕃部有仇,西夏的步兵主力步跋子,就是横山蕃人所组成。上百年来的,随党项骑兵攻入宋境烧杀抢掠从来都少不了他们一份。

除了一些熟蕃,剩下刚刚降伏的横山蕃,死光了才好——抱着这样想法的西军将校,其实是主流。在朝堂上,虽然不好明说,可私下里认同的人也不少。

一路走到政事堂前,王珪驻足,韩冈也随之停了下来。“王舜臣在这个时候被人揭出来旧年谎报军功之事,想必并不是对天子一片忠心,而是看上了他的位子。”

“相公说得是,不过也是王舜臣行事不谨的缘故,否则就是有人看上他的身份,也没办法用这个理由,去夺王舜臣的职位。”

韩冈说是这么说,但对他而言,整件事的来龙去脉,乃至郭逵相助到底是不是巧合并不重要。就是给郭逵算计一下也无所谓。还没有晋身两府的官员,能让一名执政小心翼翼的进行利益交换,不敢唐突冒犯,也足可以感到自豪了。

韩冈虽然不至于如此,但不论郭逵到底是事前还是事后得知此事,没将王舜臣一棒子打死,还主动在崇政殿中帮了忙,这份人情,韩冈领了。

已经到了政事堂外,韩冈向王珪行了礼,便转身往群牧司的衙署中去。

王珪最后看向韩冈的背影,掩饰不住的疑惑终于露了出来。

赶在开战前揭开三年前的旧事,整件事可以算是陷害了。天子不会不知道,这么做的目的是为了夺取王舜臣的先锋之位。但谎报战功是欺君,问斩都是可以的,天子也不会么有那么好的心情去体谅王舜臣。

其他人的反应,合乎情理。但韩冈的态度很是让王珪纳闷,很明显对王舜臣被被夺职而乐见其成。要说是因为自己无法参与其中,而故意杯葛对西夏的战争,这一点都不像韩冈的为人,而且一旦攻取兴灵,被挡住立功机会的王舜臣肯定咽不下这口气,最后两人很可能会反目成仇。韩冈也不会这么蠢。

那他为何会有这样态度?

王珪脸色突然变了,他想到了一个可能。是韩冈一直以来都在说的,他不看好一举攻取兴灵的计划。

从韩冈的本人对兵事的经验和能力上看,他说得多半是实话。

王珪的心有了一丝动摇,当他立刻就稳定了下来。宝都押进去了,现在哪有改弦易辙的余地,还不如坚持到底。

……………………

傍晚的时候,韩冈从衙门中回来了。心里装着好几件大事,但他的身份让他不便去拜访郭逵。

涉及利益交换如此私密的要事,除非是儿子、兄弟这样的血亲,或是跟随身边几十年的亲信幕僚,否则谁敢将自家的把柄交到他人手中。韩冈身边没有这样的人,底蕴可以说还是差了一点,

外面又开始下雪了,瑞雪兆丰年,明年是个好年。但韩冈的心情还是郁郁。

王旖看出了韩冈心中藏着事,没有怎么犹豫,很直接的就问道:“官人,怎么从衙门里回来后,就变得这般模样。是不是上殿时,出了什么事。”

韩冈说道:“今天均国公和淑寿公主种痘,官家把王禹玉和我都召了去。”

“……天子怎么把士大夫当成了医工一流,”

“有王禹玉作陪,算不得什么大事,还是用着为夫的手段救人,为夫也没觉得丢脸。”

“是不是均国公有什么地方不合意?”王旖只剩下这个可能了。

韩冈摇摇头。今天赵佣是第一次出现在外臣的面前,虽不知道他有没有机会成为下一位天子,但表现还是不错的。

“不是这件事。”韩冈也不卖关子了,将王舜臣的事向家里面的人都说了一通。

原来如此,王旖算是明白了。对此也不惊讶,这是常有的事。而败了丈夫心情的原因,肯定不是这一桩。

就听韩冈继续道:“原来此战取胜还有个六七成,运气好点,甚至八九成的把握。但现在看来,肯定要打对折了。”

还没开战就开始争权夺利,韩冈越发的不看好这一次的战争。不过以现在西军的实力,翻盘的机会依然存在,而且就是败,也不会败得太惨。这一点让韩冈心中感到几分安慰。

