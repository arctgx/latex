\section{第25章 阡陌纵横期膏粱(一)}

【这是补昨天的,等会儿还有一章】

初冬十月,今冬的第一场雪,随风而至。

雪不大,只下了半个时辰便停了下来,很快就云破日出,冬日稀薄的阳光也洒了下来。薄薄的雪层在阳光下越发的显得单薄,盖不住田地中刚刚探出头的嫩绿麦苗。可看着阡陌连绵的田野间,郁郁葱葱的绿被白色模糊了开去,韩千六还是忍不住开怀的笑了起来,连带着王韶、高遵裕、韩冈这些一起出城视察田地的官员也都喜笑颜开。

瑞雪兆丰年,今年冬天如果多下几场雪,来年的丰收就可以期待。

夹在秦岭和六盘山的余脉之间,古渭寨所处的盆地,是渭水自处源头鸟鼠山后的第一块盆地,方圆数十里,为旧时渭州的中心地带,宜垦荒地面积广大,除去划拨给纳芝临占部的一部分南山脚下的土地不算,也轻易超过五千顷。

一千九百一十七顷又八十二亩,这就是古渭寨周边已经登记造册的田地数目,而其中的半数,是今年新开垦的荒地。因为是新辟之地,对于在此处屯田,随时会应召上阵的乡兵弓箭手们来说,已经为他们打了许多折扣的田赋并算不了什么,不像中原的乡村中那样为逃避田赋,有大量的隐田存在。

可以说这新开辟出来的九百多顷地,就是王韶用来证明自己屯田之功的最好的证据。不过这些新辟之地,收成不会太高就是了——为了能用最快的速度开垦出大量田地,缘边安抚司采用了集体耕作的方法,大量使用马匹来拉犁,派出了古渭的驻军,动用了整整五百匹驮马和两倍于此的耕牛,调拨了预定中要分发给移民的耕犁,将划为官田的近千顷荒地在数日内耕作完毕,而分配给官员们的私田,也顺便让他们一起开垦了出来,并播下种子。

这种粗耕漫种的做法,能种一收五就已经是很高的比例了;一百斤种子,收上来两三百斤也是常有的事。但数量是第一位的,先开辟了足够多的田地,在天子面前就有了说话的底气,也可彻底结束有关古渭荒地多寡的争论。关于收成问题,可以等日后人口繁衍,再推行精耕细作的技术——先解决有没有,再考虑好不好,缘边安抚司上下,都秉持这样的观点。

所以王韶现在漫步在田间地头,望着广袤的原野,问着韩冈:“玉昆,春麦之事你打听到了多少?”

韩冈追在王韶身后半步:“关于春麦,下官已打听过了……”

“春麦?”高遵裕不习农事,还是第一次听说春麦,回头打断了韩冈的话,“有春天种的麦子?!”

韩冈答道:“西域冬日酷寒,比陕西尤甚,寻常麦苗熬不过冬天,只能种植春时下种、入秋收割的麦种。就如甘凉兴灵,其实也都是以春麦为主。”

因为韩冈的缘故,在屯田上担了一份差事的韩千六,跟古渭寨的各路官员接触得多了,在王韶和高遵裕面前也不会再战战兢兢。他种了一辈子的冬小麦,春麦也是第一次听说,故而问道:“三哥你拽着文,俺是没听太明白。是不是说西域冬天冷,种下的麦子都会冻死。所以得种那等在春天播种,到快入秋时收获的麦子。没错吧?”

“对!”韩冈点了点头,“冬麦和春麦的习性不同,种子也不可能一样。春小麦的种子,孩儿已经让各家商人去打听了,顺利的话,年前可以让他们带些种粮回来。几十家商队,就算一家一驼,也能有个几千斤种子了。”

王韶抬头向远处望去,神采内蕴的双眼,看见的是美好的未来,“等到了明年开春,还可以多开垦三五百顷地,到时正好把苜蓿和春麦都种上。”

高遵裕笑道:“单是古渭一处,就有两千五百顷田地,到时候,看朝中诸公还有什么可说的。”

韩千六皱着眉头,指着田垄下的麦苗:“冬麦种了几十年,不会有差错。但春麦是第一次种,恐怕脾性不熟……”

王韶哈哈一笑,摆着手不以为意,“广种薄收,先求个广字再说。关于怎么种才好,慢慢试着来就是了。”

望着王韶向前迈步的身影,韩千六欲言又止。他是种田的老把式,对田地向来是精耕细作。今次新开田地,全是大把的种子撒下去,虽然眼下长得还说的过去,但等到开春后,肯定照看不过来,只能看天吃饭。现在又让他随随便便就把不熟悉的作物种上去,这比撂荒还让他为难。

“其实春天主要还是以苜蓿为主。人吃粮,马吃草,几千匹马牛牲畜需要的牧草,也不能光靠后方。春麦仅是试种而已,几千斤种子下去,也用不了多少田。”

韩冈过来向自己的父亲解释,韩千六虽然并未释然,但以他的性格,儿子既然这么说了,他也不会在众人面前反驳。摇头叹了口气,就嘟嘟囔囔的跟着往前走。

王厚侧过头,低声对韩冈道:“几千斤种粮的确不算多。但对商队来说,便不是小数目了。即便分给几十家,但一驼西域特产的香药等物,至少能换来等重的蜀锦,至少近千贯。换来一驼种子才多少?就怕那些商人不肯带!”

“所以我有个想法,把种子当成进场税,商队带来的种子越多,能在榷场买走的商货就越多。不一定要限于五谷,瓜果菜蔬的种子也可以。而且这些种子必须要能长得起来,最好附带种植之法。如果有人敢带来一些劣等种苗,那他第二年就没有进榷场的机会了。”韩冈对此已经有了腹案,只等瞅准时机向王韶、高遵裕提议。现在说给王厚听,也算是征求一下意见。

“西域有那么多作物可用?”

韩冈摇头,笑着王厚的眼界,“西域各色作物多不胜数。像胡麻【芝麻】、胡瓜【黄瓜】、芫荽、西瓜这些瓜果菜蔬不都是西域而来吗,比起香药珠宝来,这些才是最珍贵的宝物。”

王韶和高遵裕顺着田垄绕了一圈后,视察了小麦出苗的情况,中午时分,便抵达了纳芝临占部的主城吹莽城。张香儿早得了消息,摆下了几桌宴席,等着缘边安抚司的高官们入席。

张香儿让人端上来的都是山里海里的特产,虽然这个‘海’指得是青海,但整条的从青海加急运来的湟鱼并没有经过名厨调味,仅仅是炖汤,却已经是鲜美无比。而笋、菇之类的山珍,各色禽兽野味,更是丰盛异常。

张香儿劝过一巡酒,又指着端上来的一盘鲜红透亮、被切得薄薄的卤肉片,向众官僚介绍道:“这是金钱肉,本就是大补之物,治气血虚亏。现在又加了益气补中的黄芪在里面,最是滋补元阳不过。”

“张香儿,你这话是从哪里学来的?”高遵裕惊讶地问道。金钱肉是古渭特产,在座的都吃过。但加了黄芪的做法,却是一次听说。而且张香儿还说得一套一套的,在他想来,一个蕃人怎么也不可能对药膳有多少了解。

“是疗养院的朱郎中,找小人要黄芪给院中伤病补身子,小人心想,黄芪既然能给人补身子,跟金钱肉并在一起岂不是大补,便一起下锅烩了来。”

张香儿这么一说,众人都把眼睛望着韩冈,朱中可是韩冈的得力手下。

韩冈不知朱中究竟是从仇一闻还是雷简那里,学来的这些东西。不过如今世人都重养生,许多士大夫都是药汤不离口的,点汤送客都成了习俗。所以对于药汤、药膳,不少郎中肚子里都有一堆心得和方子。

韩冈夹了一片金钱肉放进嘴里,感觉比记忆中的味道还要好一些。他点了点头,笑着说道:“朱中可是仇老郎中的得意门生,又是从雷简这个京里来的医官学来不少方子,此物不会有差。”

见到了药王弟子首肯,众官便纷纷举箸,风卷残云一般将驴鞭制成的金钱肉吃了个干净。

在纳芝临占部吃了一顿,王韶等人也不急着离开。为官本是清闲,忙得脚不沾地的只会是吏员。尽管缘边安抚司一向忙碌,但近来无论是战事还是政事,都已经告一段落,正好是悠闲度日的时候。

高遵裕午后都要小睡一番,张香儿安排了他歇息。王韶则在院中慢悠悠转着,要消一消食。等高遵裕起来,再到山中看看风景。王韶已经很长时间没有诗文问世,也想趁着这个机会,展露一下当年身为德江才子的才华。

只是没过多久,一骑急速的奔入吹莽城,把一封公文交到了王韶的手上。

王韶把用蜡封缄好的公文拆开一读,脸色就变了,“朝廷来要让瞎药、张香儿入京。”

“当是要赐姓了!”韩冈闻言心头一喜,只是他又看着王韶的脸色,却不像见到好消息的模样。

“大人,怎么了?”王厚也看不对,随之问道。

王韶脸色阴沉:“俞龙珂也要一起进京!”

