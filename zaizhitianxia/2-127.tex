\section{第25章 阡陌纵横期膏粱(六)}

千里之外,一连串的咒骂,正在王韶的肚子中酝酿。

在京城中,除了赵官家和寥寥几个宰执以外,其他人无法也无权干涉河湟之事。而且只要有了天子和王安石全力支持,枢密使文彦博也拿他没有办法。但王韶怎么也没想到,他这么快就被人拆了台,而且还是天子亲手拆的。

自入京后,觐见天子的程序按部就班的完成。从王韶开始,一直到随行的蕃人,一个不少的都到了赏赐。也不知俞龙珂和瞎药两人从哪里听来的故事,当天子说要赐姓时,他们便一起说平生多闻包拯包中丞是朝廷忠臣,乞求官家赐姓包氏。现在俞龙珂改名包顺,瞎药改名包约,至于张香儿,他本就是汉名,也不用改了。

以青唐部族长为首的三人肯到京城表示顺服,代表着王韶平戎策第一步的完美实现。天子颁制书,署诏令,并盛赞王韶‘不烦大举之兵,靡事称饷之役,以戎拓地,震慑遐荒,开信示恩,辑绥怀附。’恩荣无比。

一时之间,王韶便成了在京城中风头最劲的人物,邀请、示好络绎不绝,如同行星围绕太阳旋转,让王韶差点昏了头去。幸好他自出关西之后,吃了亏多了,更清楚这些奉承今天能来,明天就能去,完全做不得数。

可几年来,王韶还是第一次从京城中听到人们的欢呼声。由于地理位置上的关系,秦州一向不被京城的官员们重视,听说过河湟二字的寥寥无几。但眼下一切渐渐都在变化,越来越多的人听说了王韶努力的结果,随着拜访他的高官显贵越多,赞美声便显得更加响亮。

志得意满四个字充斥在心间,只是王韶的好心情只持续到今天,片刻之前:

“调韩冈去鄜延?!”

王韶陡然提高的声调仿佛在质问天子,在寂静的崇政殿中显得格外刺耳。他顿时惊觉自己已经可以算是君前失仪,陪伴在侧的枢密副使吴充也投来不快的目光。虽然声音又勉强回复正常的水平,但王韶的反对声却坚定异常,“陛下,此事万万不可!”

“为何不可?”王韶的反对也是在情理之中,赵顼不以为意,但他的反应还是要比天子预计中的激烈不少,“延州半年之内便要见功。而河湟明年开春前不会有大的动作。把韩冈调去也是为了能够更好的用兵横山,等到韩绛并吞千里横山之地,再将其调回秦州也不迟。”

“而且关西的钱粮也不足,现今都给了鄜延,秦凤没有多少余量,只够补上渭源之役的亏空。”吴充补充着赵顼没有说出来的关键。

今夏陕西大旱,不过秦州夏收之后才旱情爆发,对于冬小麦的收获,并未造成太大的影响。而且秦州河流众多,加之处于源头,小麦以外的其他作物虽然都是秋收,但用水可以用河水弥补。而秦凤以东诸州,却是旱了整个夏天,连渭河水面都降了三尺,一点都排不上用场了。

不需要吴充强调旱情的影响,王韶从秦州往京城来的一路上,听说的、看到的,就已经让他忧心不已。低低的叹了口气,王韶收拾起心情,却还是想保住自己的墙角不被人撬走,屈己利人是美德,但在官场上,却是笑话:“因疗养院之事,韩冈在河湟之地声名远播,武胜军中亦有多家蕃部因其之名,意欲来投。如今此事刚刚有了眉目,贸然将其调离,恐怕会功败垂成。”

赵顼未曾想过王韶对这个调令反应如此激烈,好像真是离了韩冈古渭那边就要出大问题了一般。虽然事实情况正是如此,不过赵顼并不想改变自己的做法。横山、河湟两地的重要性孰高孰低,他看得很清楚。主持进筑横山战略的是宰相,而主持河湟拓边的王韶,离宰相之位还有千万里之遥。

只是如王韶这等屡立功勋的臣子,赵顼一般来说都是宠礼有加。尤其是他还盼着王韶接下来能继续高歌猛进,把木征和董毡一起提来,让他能像对包顺、包约两兄弟那样,给董毡叔侄赐姓赐名。这样的想法,让赵顼不便用着强硬的态度对待王韶:

“朕还记得王卿早前曾多次上书欲升古渭为军,此事朕亦早有考量。但前时古渭诸蕃并未顺服,就算强行升格,也不可能让此地顿时变成人烟辐辏的军州,最多也就跟那些个羁縻州相仿佛,不如不设。不过眼下包、张两家都已降伏,古渭已定,再提此事便是顺理成章。”

当年真宗皇帝伪造天书,闹得国中乌烟瘴气,王旦一代贤相,一贯的贤明正直,却跟着胡闹。何故?还不是因为真宗赐了他一酒壶的珍珠。对一国宰相来说,一酒壶的珍珠算不得什么,但这可是天子送的贿赂!雷霆雨露皆是天恩,天子给脸,做臣子若不老老实实收下来,等日后可就没脸了。

现在赵顼摆明要用古渭升军一事来向王韶交换韩冈。古渭升军本就是水到渠成之事,用韩冈来交换,其实还是亏本——有药王弟子坐镇后方,前面的士兵胆气便能装上三分——可王韶有拒绝的权力吗?何况韩冈又不是他的儿子,能任他摆布

就是王韶犹豫的短短片刻,吴充粗短的双眉已经拧起来。他脖子上长了颗比李子略大、比毛桃略小的肉瘤子,如果离了近了,还能闻到一股子异味。若在唐时,入官四审——‘身言书判’中的第一项,吴充就通过不了,痤病之身,岂能侍奉君上?而且论长相,别说与另一位枢密副使,以英俊倜傥著称于朝的冯京相比,就是跟他的亲家王安石比起来,吴充都差得太多。

不过在注重才学的大宋,吴充身体形骸上的缺点,便显得无关紧要。从考上进士开始,他便一路晋升,其进速不在亲家王安石之下,已经坐在了宰执之位上。

既然已是枢密副使,理所当然便要维护枢密院的权威。他倒是没去介意王韶对皇帝的口气,朝臣不给天子台阶下的情况常见得很。但对于王韶的不干不脆,天子还没有发火,吴充就已经听得很不舒服了——什么时候官员调动要征求官员上司的意见了?!

就算韩琦、富弼这样的前任宰相,在遇到得力部下被一封诏令调走后,也只能私下里抱怨几句。只有见到看好的下属被左迁,才能为其上书说几句好话,就这样,他们也不敢说把那人再调回来——否则,一个结党的帽子就要扣到他们头上去。

“韩冈被天子亲擢于布衣之中,”吴充说道,“天子有命,他当不至有推脱搪塞。”一句话堵上了王韶的嘴。

赵顼也跟着道:“韩冈自入朝后屡立功勋,疗养院,沙盘,军棋,无不是别出机杼,发前人所未发。而在军中,亦是战绩彪炳。朕一直都想见见他,就是隔了两千里,古渭局势又一直吃紧,所以才拖到今日……今次韩冈调职延州,依例也须入京一趟,正好可以招韩冈入觐。”

赵顼早就想见韩冈一次,只是不得其便,如今正好是趁势而为。今年年初时,韩冈的名字仅仅是在他耳边一带而过,眼下才不过过去一年的时间,就已经成了秦州举足轻重的一名官员。

能举荐韩冈,王韶当然是功臣,但若是王韶回去后,撺掇一下韩冈,说不定就会让韩冈拒绝这项调令。如果此事发生了,赵顼都不知该怎么发落王韶,不论是治罪,还是放过,都让人心中难以决断。

在这种情况下,最聪明的做法,就是不要给人犯错的机会——趁王韶还没回去,先把韩冈叫来京城再说。

王韶无可奈何,韩冈虽然是他最得力的手下,又是自己亲笔所荐,但给韩绛挖了墙角,他也只能干瞪眼。天子支持韩绛的冒险,而且就在昨天,韩绛还跟王安石一起宣麻拜相。加上韩绛兼领的是昭文馆大学士,而王安石只是号为史馆相的监修国史,从名义上说,韩绛才是首相,王安石却是次相。

天子、宰相的组合,王韶根本斗不过,换作是哪一家来也都只能俯首听命。如今,关西钱粮尽入韩绛之手,兵将皆领延州之命,陕西多年来的积累都给压到了罗兀城上。如果胜利倒也罢了,但一旦失败,恐怕就是让陕西、河东两路数年内都无法重新振作的惨重损失——不仅是物质上的,也是心理上的。

‘这完全是孤注一掷!’

澶渊之盟后,王钦若曾说寇准劝真宗皇帝亲征是赌场上的孤注一掷,把天子当作筹码丢了出去。本是救国于危亡的名相,便因此恶了天子,被贬斥出京。从后人的角度看,王钦若摆明了是谗言,当时的情况已是逼不得已。

而如今,韩绛在横山的冒险,并非因为危亡在即,仅仅是天子贪心、臣子贪功的缘故。这就是眼光和胆略的差别。尽管如今的君臣,依然保持着对外战略的掌控力,但跟寇准比起来,他们还差得太远。

‘看你怎么收场!’

这不是心怀怨毒的女人所施用的诅咒,而是看透了本质,看透了迫在眉睫的战局的变化,才得出来的结论。唏嘘的口音,有着难以言喻的魔力。呢喃的话语透了凛凛声威:

“看你怎么收场!”

