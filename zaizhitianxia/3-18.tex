\section{第八章 四句千古传(下)}

星月西落,东方已经能看到金星的身影。

吕大忠在冬日凌晨的夜风中,眨了眨酸涩的眼睛。不意稍作探讨,就已经是一夜过去。他的年纪还在张载之上,精力不济。虽然年纪大了,睡眠自然会减少,但今夜消耗脑力过甚,却是头都疼了起来,分外感到疲累,远远比不上年轻人的耐力。

“觉得如何?”吕大钧走在身边,在旁问着兄长。

“很有些意思。”吕大忠放下揉着太阳穴的手,“至少今天终于知道了为什么树上的李子会掉下来。”

吕大临却道:“万有引力之说只是臆测而已,非有实证,且一时无法确认。”

吕氏三兄弟同往居所走去,还不忘说着方才在张载书房中的讨论。

“对,的确韩冈说不一定是正确的……”吕大忠对弟弟道,“但此前又有何人将李子落地拿出来钻研?几千年来,都是视为平常之事,从未根究其理。如果韩玉昆的这一假说,能带动得起世间治学开始讲究起格物致知,即便最后证明是错的,也已经是善莫大焉。”

“何况韩冈推导得还是很有道理。凡物无力则不动,这一点谁都知道,推车的车夫比我等还要清楚。至于‘如果无力改变,将会永远保持现有的状态’……”吕大钧说得很慢,显然这种说法让他觉得很拗口,要不是韩冈在给张载的书信中已经提到了不少次,他也不可能一夜之间,就能,“韩冈的这一条定律,也可以说是没有错。冬日渭水之上,常常能看到实证,如果没有……阻力,冰面上的行人、车辆当是能永远的滑行下去。”

“既然这一条定律得到确认,那么树上的李子落地,丢上天的石头总会回来,其中必然也是有力存在,也就是万有引力。”吕大忠接口,“李子、石块只是眼前的小物。日月星宿,包括脚下的大地……或者按韩冈的说法叫做地球,都是靠着万有引力而维持着互相绕动。”

三人中,虽然对韩冈万有引力的说法存有疑问,没有全盘接受,甚至吕大临更是全然反对。但对于韩冈今夜涉及的天文之说,他们却是没有一口加以否决。

因为张载的宇宙观便是上承着旧时的‘宣夜说’。万物皆气所凝,‘日月星宿亦积气中之有光耀者’,大地也是气积而成。至于地圆之说,早有明证,天圆地方也只是错讹,士林中的有识之士,无不是接受了大地为圆的说法。

韩冈的几个见解,本也是吕大忠他们日常所秉持的观点而已。只有月绕地而行,地绕日而动,金木水火土五行星也是与大地等同,这一条让他们暂时无法接受。

“不过韩玉昆的这个说法,就能解释了为何五星会逆行。绕行之速各个不同。就像两匹奔马,后马追过前马,返身看去,被追过去的前马便等于是在后退了。”

“也可以说明为何水、金二星,始终近日而不偏离。”

吕大忠和吕大钧一搭一唱的说着。金星、水星永远都在太阳附近,所以金星有启明、长庚两个名字,而水星更是长久的被遮挡在太阳的光辉之中,很少能被人见到。在过去,没有什么人去解说其中缘由,只有韩冈,大概是因为要格物的关系,所以盯了上去。

说起来,韩冈的观点虽然特别,让人一时无法理解,但却能很好的解释他们所知的一些天文学的现象。

吕大忠、吕大钧都是为此而深思,而不是一口否定。

“真不知道他是怎么想出来的,难道真有所谓的天授之才?”吕大忠半开玩笑的说道。

“‘生而知之’那可得是圣人!韩冈却还差得远。”吕大钧摇了摇头。“但先生所言的‘大其心’,旁人难以为之,韩冈却是做了个十足十。”

吕大忠为之失笑,如果只看韩冈的一番言辞,竟然事涉日月星辰,可见‘大其心’已经到了狂妄的地步。只是一转念,吕大临便是沉着脸,不开腔。

吕大钧走了一阵,见到吕大临的脸色,便奇怪的问着:“怎么了?”

吕大临摇了摇头,“只是觉得大道非在此处。”

吕大忠则回道:“大道本就存在天地万物之中。如果想追寻大道,就必须去了解万物。”

“且不说这些,格物之说总是尚显粗浅,力学三律还没有得到更多的实证,现在韩冈所阐述的不过是些残章断简,要想最终确立吾道之地位,不是三五十年就能解决。”

而吕大钧却道:“不知大哥没有没有看出来,总觉得韩玉昆在这格物之说上,藏着掖着呢。就如今日,好像也只说了一半。”

“他自己也没有把握,只能将已经可以确定拿出来。”吕大忠猜测着韩冈的心思。

吕大临冷笑道:“不拿出来推敲,还想靠着一人之力,就将其全数推演出来不成。”

吕大钧摇了摇头。

敝帚自珍的韩冈的确是有些不对。不过方才在讨论时,就是他们的这个弟弟辩难得最为激烈。韩冈不敢随便将尚未明确的粗浅理论拿出来,否则肯定是逃不过质问和指责。

"这些其实都是小节。”吕大钧说着,“我等年纪即长,时日无多。要想光大关学门楣,也只有靠年轻人了。”

吕大临却冷哼一声,“就怕他年轻识浅,根基不深。妄言大道,最后反而会走入歧途。”

“慢慢看着来吧。”吕大忠道,“我等做师兄的,日后时常提点就是。注意一点,不至于会让他走偏了路。”

吕大临又不说话了。他这个大哥就是太好人,韩冈在这个过年的时候去京,只要他能的中一个进士,日后必然飞黄腾达,怎么提点他?

………………

韩冈躺在客房中,隔着一层薄薄布垫,后背的正下方就是木板。

房中一桌一榻,桌上只有一盏油灯。再没有其他的装饰和贵重事物。简单朴素,这就是词典中能挑出来的最好最温和的形容词。

如此简陋的小屋,韩冈不知多久没有住过了。一时之间,他睡得很是不惯。枕头太硬,房中也不算干净。但他还是忍耐着,没有表露出不喜之色。这是必要的做法,也是理所当然的做法。

张载没有给韩冈安排好一点的住处——说起来在书院中也不会有如同酒店一般服侍的客房——躺在铺了几层厚布缝起来的床铺上,旧年作为张载学生时的生活,又回到脑海中。

两点一线,偶尔会是三点一线。这就是当时韩冈学生生涯的全部记忆。

摇头挥散了单调而充实的学生生涯,韩冈也在回忆着他和助手们今天所说的一番话。

韩冈如今正设法将后世的物理之学融入儒门之中,行的是李代桃僵之策,功利之心不谓不重。但张载在交谈和商讨中,明知韩冈的名利之心,却没有大加斥责,只是多提了两句让韩冈正本清源。

能成为一代学宗。张载的心胸气度,还有眼光见识,都不是凡俗可比,绝对是出类拔萃的第一流人物。韩冈拿出来三定律,还有日月运行之道,张载都能很快理解,并能有举一反三。

而韩冈的理论就此得到张载的认同,但在三吕的询问下——也许可以说是诘问——让他差点溃不成军。要不是心中对这些道理的坚持,几乎都要改弦更张。

这就是儒士讨论经义时常常出现的辩难,目的虽不一定是要否定对方的观点,但尖锐的言辞加上锋利的切入,让准备不足的士子折戟沉沙。

而韩冈坚持了下来。他要坚持宣讲关学,后续的困难苦厄都是他自找的,怨不到他人。

如今的士林之中,各家学派互不相让,如同百花争艳。但到最后,能挣出头来的只有一个。

有点像是春秋战国时的百家争鸣,笑到最后的只有儒家。

韩冈来自于后世的记忆中,此时的各家学派,能传承到后世的只有程朱理学。

韩冈知道王安石是文学大家,是诗人,是改革者,但王安石在经义上的学术观点,并不存在于他记忆中。

至于关学,可怜得就只有横渠四句流传下来。而张载,竟然是在历史书上,成了理学的开创者。

真是个让人笑不出来的笑话。

身为张载弟子,又拥有后世的记忆,使得韩冈有心要改变这一切。

今天的讨论仅仅是开始,虽然他在学术界的名望并不高,但横渠四句既然已经出世,在其中插上一脚的韩冈,已经在他的同学们的心目之中,建立了他的地位。

为万世开太平。

这样的宏愿,听候就让人变得进取起来,而韩冈确实也是朝这个方向努力。

若是韩冈能日渐高升,那么他背后的书院,乃至有名有姓的官员,都会日后学派大战的主力。

统领着他们,韩冈自问若能将之收服,就是大半个关中士林清议落到了自家的手中。

到了那时,韩冈才可以说是,他的目的,就是要为万世开太平!

