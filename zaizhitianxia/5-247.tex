\section{第28章 官近青云与天通(14)}

吕公著是枢密使兼太子太保,韩维是资政殿学士,也是又名颖昌府的许州知州。他们两人紧跟在韩冈之后,联袂造访城南驿。本已是浪涛滚滚,人心浮荡的驿馆中,更是如同被丢下了两座山,掀起了阵阵巨浪。

王安石本就是住在城南驿中,这件事倒也罢了。韩冈多半是来见他的岳父的,也不会有太多人误会。但吕公著和韩维,竟然刚刚得知司马光抵京,便立刻登门造访,城南驿中的大小官员,都不免暗暗升起了一种山雨欲来风满楼的预感。

好戏连台啊!官僚们一个个或兴奋,或紧张的交换着眼色。王、马、吕、韩今日的会面,肯定会在朝堂上掀起一场巨浪。虽然风浪起时其中必然危险重重,可不管怎么说,朝局一旦动荡起来,就不会那么容易平歇下来,而机会便隐藏在其中——住在城南驿中的官员,大半都是选人和低品的京朝官,而且是以候阙的为多。

“驿丞何在?司马宫保可在馆中?”也不知是吕、韩哪一家的亲随,先一步跨进了大门,高声的吆喝着。身着官赐红衣的元随趾高气昂,挺胸叠肚的模样,直视院中诸多官员如无物。

周至慌慌忙忙的跑上前,“司马宫保已经入住馆中,正在厅内与王相公、小韩资政叙话。”

那元随一听,登时就是一愣。身在相门,自然不可能不知道司马光和王安石的恩恩怨怨,如何能想得到摆放的对象会跟老对头和老对头的女婿坐在一起。等他醒过神来,便慌慌张张的转身往外走。

周至也跟着往门外小跑着过去,别的不说,枢密使来了,身为驿丞无论如何都得出门迎接。

就在驿馆的正门外,枢密使和资政殿学士加起来的上百名元随,几乎将大门都给堵上。而吕公著和韩维,才刚刚下马。

听到下人的转述,吕公著眉头先往中间聚拢了一点,但立刻有舒展开来,转头对韩维笑道:“君实来得太快,本来还准备出城相迎的……想不到君实竟然与王介甫和韩玉昆撞上了,还真是巧了。”

“是巧了。”韩维点点头,又道:“许久不见君实,不知近况如何,还是早点进去。”

吕公著虽然看着平静无波,但心中早起波澜。

司马光来得太快,却正好跟王安石撞上。而且韩冈竟然也来了,驿丞说司马光和两人在内叙谈,也不知现在是个什么情况,但至少已经将自己的准备都给打乱了。

周至转身引路,领着两位他远远开罪不起的重臣入驿馆中。只是刚刚抬头向前,顿时就怔住了,脚步也不动分毫。

而吕公著也在同时双目瞪圆,从门内出来的人让他完全出乎意料,他低低一声喝:“韩冈?”

韩维不认识韩冈,但辨认正从正门内跨出来的年轻人身上的服饰是没问题的。除了少数宗室之外,这个年纪依照金紫的臣子自然只有韩冈一人。确切的说,身穿紫章服,腰围御仙花带,却不佩戴金鱼袋,只会是新就任翰林学士的韩冈——翰林学士是只围金带而不配金鱼袋,直到入两府之后,才会两样同时佩带,称为重金。

但让两人吃惊的不是韩冈,而是韩冈出门相迎的行动。只见这位新晋资政殿学士大步迎了上来,一直走到面前。

这里可是大门外啊!韩冈的身份就是跟吕公著相比,也不输多少,更是与韩维只有资历上的差别。并无旧谊,又非世交,没有那份交情,怎么会出正门?

他们来拜访的是司马光,就算出迎也该是司马光的儿子出来,韩冈越俎代庖算是什么道理?即便是王安石同在馆内,跟司马光叙话,出来的也该是亲生儿子的王旁才对。让身居高位的女婿出了大门来迎接地位相当的同僚,换作是心眼小一点的人,可能会记恨上一辈子。而韩冈年纪轻轻便为显宦,心高气傲是免不了的……

其实司马康和王旁就跟在韩冈身后。只是他们一看到韩冈,司马康和王旁便立刻成了路边的甲乙丙丁,完全被吕公著和韩维给忽视了。

在吕、韩二人面前,韩冈停步,随即一揖到地,“韩冈见过吕三丈、韩五丈,家岳和司马十二丈正在内中相候,还请随韩冈入内。”

周围一阵抽气声,围观的官员们脸上只有震惊之色。衣着金紫的重臣,殿上拜见天子也就这个礼数了——非是正式拜谒,寻常时见了天子并不一定要大礼参拜——这是晚辈对尊长的礼仪。

吕公著和韩维惊讶更甚,韩冈这分明是以晚辈的身份在行礼,连称呼都没有使用正常的官称。如果是作为身份相当的大臣,韩冈断不至于行如此重礼。

醒过神来,吕公著、韩维立刻回礼,只是眼神依然是犹疑不定的瞅着韩冈。至于韩冈身后的王旁和司马康,虽然跟着上来行礼,但直接就被两人给忽略过去了。

王安石和司马光此时已经并肩站在正门内的院中,正等着两位多年不见的老友。笑意盈盈,看不出之间有任何芥蒂。

这一回是韩冈引路开道,领着吕公著和韩缜进了大门。

犹在院中的官员都不敢再上前,连声音都渐渐收止了。东宫三师加上大韩、小韩两位资政殿学士,寻常时,哪里能在城南驿见到这样规模的重臣集会?几人在院中一站,仅仅是互相行礼,一股难以言喻的威压就随之扩散开来。而韩冈不以官位自傲,自居晚辈的举动,让人在惊讶之余,也有不少人点头暗赞。

“晦叔、持国,久违了。”王安石笑得最是真挚,大步上前与旧日友人行礼。

吕公著脸上的表情变得稍稍僵硬了,就算他的城府再深,也因为这番突然而来的变化而措手不及,心情一时难以安定下来。

王安石到了京城多日,登门造访的旧日亲朋甚多,尤其是在他成了平章军国重事、确认留京之后,更是宾客盈门,这两日才稍稍好了一点。但吕公著并没有来拜会王安石——韩维也没有,不过他上京才两日,却还能说得过去。

“晦叔、秉国。”司马光也跟着上来行礼问候。

韩维字持国,但司马光的父亲名池,因而避讳,一直称持国为秉国,字不同,意思却是相同的。只是韩维的兄弟韩绛做过宰相,韩缜现在是参知政事,更是近期拜相的最为热门的人选。但韩维虽字持国,却跟宰执之位距离甚远。

一起进了方才对坐闲谈的小厅,各人谦让了落座。并不是按官位,而是自然而然的按照年甲,年纪最长的韩维坐了上首主位,然后吕公著、司马光、王安石这样排下来,一切一如旧日,只是座位上的人与过去完全不同了。

嘉佑年间,四人都在三四十之间,正值壮年,亦是闻名朝堂的少壮派官员,时常抽空相聚,论史、论诗、论文、论政,纵谈天地万物,当时何曾想过会有分道扬镳的一天。

韩冈坐在下首,纵然还是晚辈,还不至于让他和王旁、司马康一视同仁的侍立在侧。

相隔多年重新坐在了一起,就算心中依然有着深深的鸿沟,王、马、吕、韩这嘉佑四友,二十年前四人相聚的日子也不免浮上心头。

“记得当时是包孝肃置宴设酒吧?”吕公著笑问道,“包孝肃是群牧使,君实和介甫是群牧判官。”

“当时包孝肃举杯劝酒,光不能喝酒都勉强喝了,就是介甫你硬是不肯喝。”司马光问韩维,“秉国你说是不是?”

韩维点头道:“介甫的脾气一向执拗,听说弄得包孝肃都下不了台。”

王安石笑着道:“但安石没有少跟持国、晦叔和君实一起共饮吧?”

王安石只论旧谊,司马光也是半句不提今日朝堂和新旧法之争,吕公著看起来也没有破坏气氛的想法,随着一起谈笑,只有韩维言语稀少,与传言中喜好结交的性格完全不同,让韩冈有些惊讶。

“介甫最让人羡慕的倒是有一佳婿。”吕公著冷不丁的将话题跳到了韩冈身上,“玉昆如今声名广布,北至辽土,南至日南,人人视玉昆为万家生佛。”

韩冈向着吕公著欠了欠身:“虚名而已。韩冈徒有虚名,学问远未精湛,当不起三丈之赞。”

“玉昆可不是虚名,富彦国一直都赞你是宰相器。”司马光说道。

司马光略显削瘦,须髯不长,看似是轻松的在谈笑,但眼神中一点笑意也没有。应该不是错觉,韩冈想着。

“富公为人宽厚,提携晚辈不遗余力,就是往往失之过誉。韩冈愧甚,绝不敢当。”

厅中诸人各自异心,正在说着无聊的话,驿丞周至敲门进来,说是酒席已经准备好了。已经受够了这种怪异气氛的司马康和王旁立刻起身,但看看四位长辈和韩冈都没动,当即就僵住了。正犹豫着是不是该坐回去,司马光却也跟着起身。

“不意都到了这个时候。”王安石看看外面的天色,回头笑道:“君实一路奔波劳累,的确是不该再耽搁。”

“若是寻常时候,应该让人好生筹办一番,不过眼下天子重病,不便太多奢华,只能以简素为主了。”这是韩冈之前吩咐下去的,所以在入席前代为解释了一下。

没人会对酒菜简薄而感到不满,若是按照正式酒宴初坐、再坐的从菓子、甜点一直吃到冷盘热菜,一盏盏酒的排下去,几十盘菜吃过来,拖到半夜都是等闲。

谁有那个耐心?!韩冈没有,王安石也没有,司马光、吕公著和韩维更没有。
