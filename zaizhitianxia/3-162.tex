\section{第41章 礼天祈民康(九)}

【对不住各位书友,今天中秋得陪着家人,所以只有一更。】

熙宁七年冬月廿九,冬至前日。

六天前,天子赵顼留宿于大庆殿中开始斋沐,拉开了三年一次的郊天大典的序幕。昨日,赵顼祭拜过太庙,并在太庙中斋戒。而今天,终于到了最后的仪式开始的时候。

刚过鸡鸣,天还是黑的。夜风劲烈,看不到月亮的夜晚,只有被风刮得忽明忽暗的数百只火炬,照亮了大庆殿前广场上。映出了广场中,数以万计的人马、车辆,正是天子的大驾卤簿。

所谓卤簿,就是仪仗。

大驾卤簿,仗下官一百四十六员,执仗、押引、职掌诸军诸司总计二万二千二百二十一人,另外还有伴驾的数千文武官员,以及车辆、马匹,甚至还有六头大象,此时都聚集于大庆殿前的广场之中,等候天子从皇城的主殿中出来。

数万人在广场上各就其位,站得分毫不乱。除了宰执之外,数千官员都是按照本官来派定位次——差遣仅是职司,只有本官才有品级。

右正言属于谏官之列——诗圣杜甫做的拾遗,其实就是正言的前身,只不过被改为正言——故而韩冈的位置也就在谏院之中。

尽管天子前日在韩冈转调判军器监一职后,又特赐了韩冈五品服色,也就是所谓的赐绯银,纵然只为七品,亦可身穿红色五品公服,腰间配上银鱼袋。但绯衣鱼袋是日常所穿公服,在今日的大典上,所有的官员都得身着朝服——朝服都是用绛色衣袍,鱼袋例不佩戴,另有作为饰物的配绶区分等级。

只看外袍,韩冈却与站在大庆殿前的其他官员没有多少区别。不过他头上戴的不是三梁、五梁的进贤冠,而是以铁为内框,上方缀有两枚珍珠,凸起仿佛尖角的方形冠冕——獬豸冠,也称法冠。

獬豸是传说中跟随在上古刑官皋陶身边,能辨是非曲直,能识善恶忠奸的神兽。皋陶在刑狱中被供奉,而獬豸的图案也是贴在监狱大门上的。自先秦以来,獬豸都是刑法的代表,獬豸冠也就成了言官、谏官、刑法官们的装束。不过现如今,也只有在朝堂大典时才穿戴。平日里,就算是正儿八经的御史,也还是戴着长脚幞头。

上方下圆的獬豸冠是以铁条为梁给撑起来的,虽然看着不错的,但戴在头上就未免显得沉了一点。戴惯了轻便的长脚幞头,韩冈一时还没有习惯过来獬豸冠的沉重,时间稍长,脖子就有些发酸。

想着如何不为人注意的活动一下脖子,韩冈却没注意到有多少双眼睛都在背后看着他,暗地里也在议论着他。

“看不透啊。”一名须发皆白、差不多有六十多岁的老京官从韩冈的背后收回视线,声音很低,却充满了疑惑。

韩冈前日廷对上的细节,只是在核心层中传播,并没有悉数传到下面来。所以底层的京朝官们从粗略的传言中,完全看不透究竟是怎么一回事。

冯京不想韩冈入中书,韩冈本人也不想入中书,但两人到底是为了什么便翻了脸?据说冯京当日回到政事堂中,连个好脸色都没有。

在崇政殿上开罪了冯京。而拒绝了韩绛的举荐,也同样开罪了另外一名宰相——已经不是仁宗、英宗的时候,过去拒绝宰执们的举荐,可以说是品行高致,眼下可是关系到站队的问题,韩冈的行为摆明了是拒绝了韩绛的招揽——韩冈的所作所为,怎么都让人想不透。

“区区一个七品官,竟然四面树敌?当真以为远在江宁的王介甫能护着他,还是圣眷一直能保着他?”

与老者并肩站着,身上的配绶毫无二致,可相对而言要年轻许多的官员则猜测道:“该不会吕参政不想让他去中书,所以他才不去的吧?”

老者反问道:“要是韩冈当真站在吕吉甫那一边,他怎么会不去中书?”

不管韩冈投了谁,他都该去担任中书五房检正公事。眼下无论哪一位宰辅,在得到了掌管中书各房庶务、文牍的都检正的支持后,完全有可能将对手在政事堂内给架空掉,就像当年的曾布,帮着王安石架空了其他宰执一般——毕竟这个新创设不过数年的职位,一开始就是为了让当年还仅是参知政事的王安石,能顺利的掌控朝政而设立的。

“那就只有一个可能了。韩冈是准备在军器监大展拳脚,不想受到其他的干扰。他不是自称传习格物之说,于此事上有所擅长吗?说不定能”

老者驳道:“这样一来,他不就又得罪了吕吉甫?吕吉甫如今可是兼着经义局,又是前任的判军器监。韩冈在军器监只要想有所成就,就必定会得罪吕吉甫。”

“但他拒绝了韩相公的举荐,不是与吕参政结了个善缘吗?”

“哪有这种道理。”老者低声笑着。东府参政和七品正言之间,可没有交换的说法,韩冈岂够资格?如今的朝堂非此即彼,不去投效,又哪里来的善缘可结?

数声净鞭响过,殿前鼓乐合鸣,打断了两人的对话。官员的特技在瞬间发动,神色刹那间变得肃穆庄严,方才的议论仿佛从来没有出现过。

天子步出大庆殿,群臣、万军一起跪拜下来,山呼万岁。这呼声,如同山崩海啸,千呼万应,在广场上空回响。

随着天子等上玉辂,蹄声、脚步声和鼓乐声便响了起来。

先是六头大象起步,继而开封令等六引导驾,清游队百余骑夹道而行,前队仪仗两百余人持朱雀、黄龙、风伯雨师雷公电母等旗,与太常前部鼓吹——笙、箫、笛、笳、鼓、钲——又数百人紧随其后。

然后司天监、持钑前队、前部马队、步甲前队、前部黄麾仗、六军仪仗、引驾旗、御马、班剑仪刀、五仗、左右骁卫、左右翊卫、金吾细仗、左右卫夹谷队、捧日、奉宸,十几二十队总计上万人一批批的穿过宣德门,沿着御道向南过去,导驾官才开始起步。

通事舍人、侍御史、御史中丞左右分行。正言、司谏、起居郎、起居舍人同样分行左右。在后面谏议大夫、给事中、中书舍人、散骑常侍为大驾玉辂的先导,而两名宰相,是导驾官最后一队。

等到紧跟着导驾官的殿中省仪仗的大伞、雉尾扇、华盖等器物过后,载着天子的玉辂才在御马的拉动下启动。

玉辂之上,当今大宋天子端坐着,仿佛庙里的塑像一般。

天子的玉辂还是从唐高宗显庆年间传下来的旧货色,已经有四百年的历史,多少代皇帝经手。虽然之前整修过一次,但毕竟是几百年的老古董,一动起来就是吱呀作响。赵顼坐在上面,不但摇晃得有些难受,而且冷得厉害。

这玉辂四面透风,只有一层轻薄的纱帐遮住御容。外面的视线穿透不进来,可子夜的寒风却能毫无遮挡的吹进玉辂之中,悬在纱帐上的小铃叮叮当当的响着。不比寻常的马车,座位下面还能放着小暖炉,天子玉辂从来都不考虑这些舒适上的问题。只想着如何装饰精美华贵,符合天子的身份。

左青龙、右白虎,龟背为纹,四角栏杆有圆镜、鸟羽。就是连根支撑黄盖的柱子油画刻缕、金涂银装,各色陈设世间所无。可赵顼坐在上面就是觉得冷。

赵顼不是没有考虑过造新的,前年——也就是熙宁五年——就新造了一辆玉辂。在除夕的时候放在大庆殿前,准备在第二天正旦大朝会上展示。不过天降横灾,搭在玉辂外面做遮挡的棚子竟然倒了下来,将新玉辂给砸坏了。天意如此,赵顼也只能老老实实坐着四百年的古董。

赵顼现在身上的穿戴,从内到外都是按照礼制,可就是不按照时节。若是在圜丘上祭祀时所穿戴的衮冕,外面还能多罩两层,可现在他穿的依然还是通天冠、绛纱袍,并没到换衣服的时候。只有到了青城行宫,进了大次之中,才会换上正式的祭服。那些在典礼上有司职的,如担任大礼使的韩绛,桥道顿递使的孙永也是一样,现在都穿着朝服,到了地头上才会换上祭服。

从宣德门出来一路南下,还没过了州桥,赵顼就已经冻得脸青唇白。

韩冈行在队列中,作为导驾官中的一员,他离着天子的玉辂倒也不远。身边的同僚在寒风中各个都有些瑟缩,只是在天子驾前不得不强挺着腰。但韩冈却迎着风,一点也不觉得冷——比起关西的酷寒,东京城的冬天根本算不了什么。

韩冈自前日接了诏命,并没有立刻去上任,他还要参加各项仪式。右正言的本官本是定俸禄的空衔,也只有到了奉祀的时候,才变得有实际意义。

不过对于上任后,该怎做他都已经有了规划。对章惇,他说他准备萧规曹随,这并不是谎言。韩冈的确并不准备更动吕惠卿定下的制度。在吕惠卿的监督下,这两年打造得军器精良远胜过往,军器监中的官吏必定早就被他驯服了。

韩冈贸贸然去改变制度,不论他设计的新制看着有多好,施行起来肯定要吃个暗亏——虽说县官不如现管,但韩冈不认为他能在吕惠卿干扰的情况下,将差事办好。即便做好了,也挡不住有人说不好。

韩冈知道,现在外界对他的选择都是疑惑不解。这个局面换作他人来,也的确是破不了,只能向吕惠卿俯首或是选择干脆离开。放眼今日,只有他韩冈,才有这个能力。

天渐渐的亮了起来,大驾卤簿一队队的出了南薰门,渐次进抵青城行宫。随着东方的太阳跃离地平线,号角齐鸣,天子的御驾终于抵达了青城。

