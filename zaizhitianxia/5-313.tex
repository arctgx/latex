\section{第31章 停云静听曲中意(20)}

越界两日,攻入广信军境内的辽军,已经清扫了遂城以外所有的村落,但前锋军中作为主力的六千兵马一越过边界,便直接压到遂城之外,驻扎在城北偏东最近的一处村庄中,遥遥压制住城中守军。

两名主将萧敌古烈和耶律菩萨保已经在小村子中歇了两日,纵然不去打草谷,下面的人不会短了他们身边的好处,可两人体内的血液已是躁动不已,恨不得能代替手下在宋人境内烧杀抢掠一番。但来自耶律乙辛的严令,让他们不敢少有违逆。

这一日城头上鼓声大噪,城外的辽军便如被惊起的飞虫,立刻活跃起来。

萧敌古烈和耶律菩萨保也立刻从安营扎寨的村庄中出来——就是北面离遂城最近的村子,只有四里——率队抵近遂城,拿着千里镜望着城头上。

千里镜在契丹高层将领们手中也普及了,纵然从宋国流传过来的千里镜最次的也有五六百贯之多,但也没几人会在这件事上省钱。本也没有打算攻城,笨重榔槺的飞船没带上,但千里镜两人手上都有一具。

“终于是准备出来了?”

“大概是想提一提士气吧。”

“会不会有援军?”

“远探拦子马哪个都没回报。这一马平川,大股的宋军出来后还能瞒得了人?”

萧敌古烈和耶律菩萨保你一句我一句的交换着想法。在两人望远镜中,遂城的城门中开,两列贯甲持矛的宋军骑兵从城门中鱼贯而出,越过吊桥后,在城壕外分列官道左右。

遂城算是军城,城外并没有民居聚居,最近的村落也在三里外。一里之内连大一点的树木都没有,只有官道上,还有些酒家和亭舍,只是已经给烧得干净。有着足够的布阵空间。

“两个指挥啊,站得倒是整齐。”耶律菩萨保笑了一声。那些骑兵,即是军阵的一部分,也当是保护列阵时脆弱的军队的作用。只是在他眼里,实在不值一提。

萧敌古烈也回以不屑的一哼:“也只是整齐罢了。倒是甲不错,到时一家一半。”

“行啊,这事好说。”耶律菩萨保一口应承。还没攻入宋境时,还有提心吊胆,现在倒是一分一毫的担心都没有了。他所在的这一支进兵顺利无比,最前面的都已经杀进保州了,都不见有宋军出战的,在边界上也没人出来阻截,哪里还会将宋军放在眼中?他们两人可都是多次参与过征讨国中叛乱的战争,从女直人打到阻卜人,没一个好对付的。

面对出战的宋军,萧敌古烈和耶律菩萨保视线的落点却是放在城头上。如果城上的防守稍有松懈,他们就要试一试能不能趁机冲入城中,可惜云集在城头上的神臂弓和极为显眼的八牛弩让两人不敢冒险。

紧接在骑兵之后,则是步卒。上下一身雪亮的铁甲,连脸都给遮住,拿着长矛带着弓弩,亦是分作两部,在两部骑兵外侧列阵站定。

“倒是光鲜!”耶律菩萨保咕哝了一声,对宋军不屑一顾,但对他们的装备眼红无比。

“押阵的是谁?”萧敌古烈想知道谁领军出阵。两翼看起来已经站定了,只留下了中央的空隙。

一面赤红色的大纛就在这时从城中缓缓移出,旗面血红,而边缘上则镶了一圈黑边。一个斗大的黑色篆字文的‘李’字,绣在旗面正中央,在旗帜之下则是十几骑骑兵以及一批步卒。

辽军中起了一点骚动,能用上这面大纛的自然不会是底层的将领。

“呵,竟是李钤辖亲自出来了。”耶律菩萨保瞪大眼睛叹了一句,便回头笑道:“药师奴,你家主人弟子的表兄出来喽!”

小名药师奴的萧敌古烈立刻返过来嘲笑回去:“菩萨保,你不是靠那李钤辖的表弟来保佑吗?不知跟李钤辖对上阵,人家还保不保你了?”

“没药上菩萨,也还有文殊师利、观音、大势至诸菩萨呢。可药师奴,你可是人家家中子啊!”

“不闻菩萨都有亲吗?开罪了药上菩萨,文殊、观音怎么还会保你?”

两人对视一眼,却是放声大笑。

萧敌古烈大笑着回顾左右,“你们说是不是?”

周围一片寂静,片刻后嘎嘎如鸭鸣的干哑笑声才一个跟一个响了起来。两人身侧的部将和亲兵聪明的都知道要奉承,但偏偏笑不出来,却又不敢不笑,只能从脸上挤出笑来,看着就怪,听着也怪,一下就冷了场。

韩冈在辽国民间口耳相传,是药师琉璃光如来,也就是药师王佛的座下弟子药上菩萨转生,随着种痘法在辽国的推广,虔信佛教的辽人多有人尊称他一声韩菩萨。

萧敌古烈,小名药师奴,而耶律菩萨保,更是直接把小名当成大名来用。辽人信佛,药师奴、菩萨奴,佛奴、观音奴,菩萨保、观音保,这都是辽国国中极普遍的名号,作小名的极多,沿用到大名上也不少,就是在官场上,也多有撞名的情况。

萧敌古烈,耶律菩萨保拿着民间的传言说笑,敬意不见半分,却是戏谑之意更多,自然是对李信并不放在心上。契丹的骄横乃是一贯如此,对南朝的将领根本都不放在心上,尤其前几年还逼着宋人割地。西平六州的失陷,也不过确认了耶律余里是个蠢货,而当地的头下军各个废物罢了。只是下面的人不配合,比起上层的贵族,地位越低的百姓越是对佛教虔信,对韩冈也更为敬畏。

闹了个没趣,耶律菩萨保咂了咂嘴。看着依然拥在城头上的一排宋军,又不无遗憾的叹了一声,“可惜李信胆小得跟鹧鸪一般,都出战了,还让那么多人守在城上作甚?”

“自然是因为没胆子啊!”萧敌古烈心中不屑,笑了几声后,脸色一冷,阴狠狠的问道:“要不要杀过去?!”

耶律菩萨奴摇摇头,“还是再等等吧。既然李信出战了,应该不会装装样子就缩回去。”

“详稳,总管,不可小觑那李信。”跟在两人身后,是易州的兵马都监,“李信做了多年的将领,南征北战,陕西、荆南、广西、河北,天南地北都镇守过。打过的仗几十数百,杀过的人成千上万,不是普通的角色。”

萧敌古烈立刻一声冷喝:“想那李信,打的杀的不过是些猪狗一般的蛮夷,可有比得上我契丹男儿的?!”

“都能让南人欺负,哪里比得上我大契丹!”

“脚趾头都比不上啊!”

众皆大笑,纷纷附和,终于不再冷场了。易州兵马都监嘴唇动了动,也没再说什么了。

跟随大纛和李信同出的只有三百余人,却是个个高大雄壮,站在最中央的位置,气势硬是压倒了两翼人数更多的骑兵、步卒。

当这一群人出阵,就没有士兵在后,最后出来的,是一辆鼓车,被十几名推着从城门中缓缓驶出。一出城门,节奏徐缓的鼓声立刻就传遍四野。

鼓声不停,鼓车也没有停,随着鼓车驶上吊桥,刚刚站定的宋军军阵就开始前进了。

马军提着缰绳,小碎步的前进,而步兵则踏着鼓点,一步步的徐步向前。

脚步声与鼓声汇合一处,不见一丝乱,每一步都撼动着旁观者的心神。从濠河边前行了百步之多,方才停了下来。

辽军的两名主将一开始还挂着不屑的笑意,可随着宋军军阵的移动,笑容也渐渐收敛。到最后,出城的宋军前行百步而阵列不乱,萧敌古烈和耶律菩萨保的脸色全都变了。

兵法云,军阵有气,望气可知强弱。耶律菩萨奴两人自然不懂什么望气法,但能不能打那是一目了然。

列阵简单,但列阵后在进退中保持队形,难度可就要翻了番的往上涨。萧敌古烈和耶律菩萨保纵然没有吃过猪肉,也见过猪跑。十步一驻足,二十步一整列,正常的步军阵列绝不可能一口气向前走上百余步。

出阵的宋军总数不过一千五六,但这一千五六,却绝对是精锐。军行如山岳,缓缓压了过来,甚至给人以喘不过气来的感觉。

“打不打?”耶律菩萨保低声道。

宋军军阵所处的位置,已经越过了城上守军的保护范围。除了八牛弩,就连神臂弓都没办法在这个距离伤人了。宋军主动脱出城墙上的保护,看起来就是要堂堂正正的打上一仗,这边照理说也不能

萧敌古烈犹豫着。阵列不战是契丹对宋军时的惯例。自尊心什么的,根本不放在他们的身上。往石头、树桩上撞的那是走投无路的兔子,不是青牛白马八部落的子孙。

但对面的宋军这时又有了动作,居于阵后一名骑手从马背上下来,穿过前方的军阵,径直来到阵前。那面血红的大纛就跟在他身后,同样从阵后来到阵前。而在那骑手的另一侧,跟着一名身材高大的小卒,背着一捆标枪,也是亦步亦趋。

当那名骑手在阵前站定,随行在后的大纛被掌旗官用足了气力,一下夯进了土中,旗面猎猎随风,斗大的‘李’字飞扬,城头上猛然一阵欢呼,连鼓声也陡然间高亢了许多。

辽军一方寂静无声,萧敌古烈和耶律菩萨保脸色阴冷,死死那名金盔金甲的骑手。易州兵马都监在后小声道:“那就是遂城主将李信!”

