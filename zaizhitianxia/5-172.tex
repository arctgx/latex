\section{第20章 土中骨石千载迷(一)}

“总算是清净了。”

张商英拿着一柄高丽折扇,在掌心轻轻打着拍子。一名教坊司中第一流的歌伎在厢房中婉转而,动人的歌声从半开的窗户中传出楼外,而前些日子的嘈杂喧闹,终于不再出现了。

前段时间,一个十三间楼,一个清风楼,是那些喜好格物的衙内们聚在一起谈论天地自然地方。甚至有人模仿了诗社的形式而结社共论天地自然。这些人互相都不服,每每争论起来,都让酒楼中的其他客人不胜其扰。

“在正道上走不通,只能走旁门。但旁门左道毕竟不是正途,一旦天降雷霆,根本就避不过。”

禁令一下,两间酒楼中不见前些日子高谈阔论的衙内们。不仅没人谈论千里镜和天文星象,就连显微镜也一并没了人来谈论。

蔡京与张商英对坐,蔡卞打横相陪。蔡卞难得有一次与人共饮,张商英兴致高昂,拉着蔡京和蔡卞两兄弟,你一杯我一杯的喝着酒。

蔡京看似带着酒意,举杯邀张商英对饮而尽,“非是天觉,也无人能直言气学之非。”

张商英哈哈的大笑,“说什么格物致知,致知的结果倒成了玩物丧志。”

“韩冈所学不正,故而有如此结果。”蔡卞甚至有几分期待,期待天子的用意不仅仅是在千里镜上,“今日在经筵上,天子也在说风俗不同,道德不一,国必难治,民必难安。”

“说的没错,正是这个道理!”张商英点着头表示赞同。

蔡卞是王安石的弟子,因而被安排在国子监中授课。在去年的太学案中,他仅仅是被风尾扫过,没出什么大篓子,也就连带的受了点处罚。这一次,天子要在经筵上开讲《字说》,便是很巧的就让他出了头。若是换作是蔡京上本,蔡卞想要得到崇政殿说书这个职位,可就没那么容易了。说起来蔡卞的运气的确是让人羡慕。

赵顼已经年过而立,在课业上并不打算花费太多的心神。经筵已经由初登基时的逐日讲学,变成了隔一日开讲一次。加之经筵官也有七八人之多的,蔡卞要不是正好是开讲《字说》而被提拔起来的经筵官,半个多月才能够轮到一次——虽说依然是让无数朝臣艳羡,但毕竟比不上现在隔三差五就面见天子的际遇。

“《字说》乃万世不移的经典,故而得了天子看重。圣意如此,世人皆有共论,又有谁敢跟天子拧起来?”

蔡京并不认为王安石的《字说》能当得起蔡卞的评价。说起来《字说》解字,皆是以今字楷书为解。中心为忠,如心为恕,算是解得妙的。坡者土之皮,滑者水之骨,也勉强能说得通。但豺为才兽,熊者能兽,这样的解释,怎么想都不对。

也不看看从古时到如今,这字形变了多少。蔡京是书法大家,大篆小篆汉隶,楷书行书草书,以及各种的变体,他都是得心应手,当然看得出来《字说》中最大的症结。

上古以六书造字,象形、指事、会意、形声、转注、假借,可不是只有会意一条。形声虽多兼会意,但不能偏到所有的形声字都当成会意字来解释。

但蔡京却无意指出来,若有必要,为《字说》做注疏他都不会有问题。

“气学这一次大败亏输,朝廷真的想有所振作,也没有气学的落脚点。”张商英轻摇着扇子,这一回上书合了天子的心思,对他来说,也这绝对是是一桩喜事。

“韩冈求胜心切。若是稳一点,也不会有今天的事。”

要想在朝中混出头,就必须有自己的班底,而想有自己的班底,就要有足够的名望。同样的道理,学术上也一样如此。韩冈他的功劳已经足以留名史册,日后的宋史之上少不得有他的一篇列传,但以蔡京对韩冈的认识,这一点还满足不了韩冈的。只是他太年轻,家世又缺乏底蕴,想要在各家的纷争中脱身而上,将关中的子弟都招揽至门下,光凭防疫之术还是不够的,学术上必须有成就。

“就是稳了又能如何?待介甫相公的《字说》遍传天下,气学也只能去找匠人和纨绔们做传人了。”张商英望着窗外的街道,“这一份诏书,倒是便宜了军器监。”

一辆马车正从楼下经过,车边的护卫明显多于正常情况。昨日政事堂下了堂札,着令开封府将收缴上来的千里镜,全都转送去军器监武库。

清风楼离开封府不远,对街的巷中就是府库的正门,一车价值万金的千里镜从府库中出来,便是从清风楼前的正街上转去军器监中。

视线随着马车远去,张商英唇边的得意更加明显。

蔡确当年弹劾王安石合了圣意,才几年功夫,就已经做到了参知政事。张商英绝不认为自己会比蔡确差,两府中不到十人的位置,他也有意在十几二十年后占上其中的一把交椅。

当上了御史,只要做得好,两府中的位置就绝不是奢望。

多年前第一次入御史台,张商英知道自己是太心切了,惹起了枢密院同仇敌忾,以至于被天子和王安石抛弃。这一次卷土重来,两府暂时不能请动,那么离两府只有一步之遥的韩冈便是第一人选。

就让韩冈继续当这块垫脚石好了,张商英不介意多踩上两脚。

……………………

笼在纱笼中的蜡烛噼啪响着,夜阑人静,除了烛花轻爆,就只有远远地随风而来的更鼓声。
苏颂夜不能寐,都快三更天了,心绪乱得让他没有一丝睡意。伴读的书童也给赶了出去。

近一个月时间,他和韩冈才将凡例写好,一个崭新的分类方法,使得书中章节也必须做出相应的调整,为此很费了一番功夫,到现在也没有竟全功。

但苏颂今天晚上并不是因为此事而夜中难眠,今天白天,来自韩冈的一个请托,让苏颂心烦意乱难以安寝。脑中一团乱麻,坐在桌前却是纹丝不动的有一个时辰了。

“大人。”苏嘉在书房外敲门。

苏颂的书房不是随便什么人都能进来的,对于他们这些士大夫来说,比起人来人往但卧房,自己的书房更为私密。不过儿子过来,当然不能拒之门外。

“大人还是早些安歇吧。都快三更了。”苏嘉端了茶进门后就劝着苏颂,“明天大人不是要开经筵为天子讲学吗?。”

天子这些年来,一般都是隔五日才去文德殿上朝,平常的时候,都是让宰相押班,带着一群不厘实务的朝官拜舞了事。苏颂有实务在身,可以不赴常朝。但明天是他上经筵为天子讲学的日子,精神不好,犯错的可能性就不会小,纵然只是罚铜,那也是一桩丢人现眼的事。

苏颂应了一声,却仍是动也不动,低头只顾盯着身前的书案。

苏嘉看着苏颂面前摊了一桌子的药材,地上也都是没有清理的碎渣,弄得好端端的一个书房,变得乱七八糟。忍不住气道:“韩冈不过是想借着编修药典来宣扬气学,却让大人为这一部《本草纲目》殚思竭虑,连睡觉也不安稳。大人你这是何苦呢?”

“你懂什么?!”苏颂突然发起了火,冲着儿子呵斥,“圣学乃万法之宗,医药之学何能例外?!医典中论及圣学,本就是韩冈他该说的,不说才有错!明日到了经筵上,为父照样会说。”

苏颂一贯好脾气,一年到头也不一定会发一次火。今天的怒气突如其来,苏嘉张了张口,却也不敢多说上一个字。

训得儿子不敢说话,苏颂冷哼一声,这个话题算是揭过去了。不过到了明日的经筵上,却是没办法跳过的。

苏颂是翰林侍读学士,在经筵官中排在最前面的,品阶远比侍讲、说书都要高。不过经筵官的地位高低,是要看他们与天子相处的时间长短来评判。苏颂当然是远远不及新党的那一拨人马。

大宋天子特设经筵,让臣子来讲学,这是在向天下臣民表示皇帝重视文教,同时也让天子多了一个了解外情的渠道,增长学问只是末节而已。相对的,诸多臣子也利用了这个机会,来争取天子的认同。

王安石安排吕惠卿和王雱做崇政殿说书,吕惠卿升任执政后,也安排了自家的兄弟做崇政殿说书,就是为了利用给天子讲学的机会,将自己的政治观点灌输给天子。比起崇政殿中,一群宰辅重臣争着说话,互相之间还监视着对方,经筵上一对一的讲学,能将事情说得更细,也便更容易说服天子。

这些天来,新学一脉的经筵官,将气学视作眼中钉,在经筵上连番攻击气学中的观点。韩冈主张的自然之道难以争论,但儒门的根本还是在经义上,张载和韩冈的论述中,可供攻击的地方很是不少。

所以韩冈才会请到了他苏颂的头上。

下这个决心并不容易,在经筵上为气学张目,这等于是要让苏颂彻头彻尾站在韩冈这一边。不过对于新学,苏颂的确没有好感,而他本人的学术观点,这些年来,也的确是与气学越走越近。

而且因为天子诏禁私藏千里镜,苏颂不得不将自己心爱的千里镜交了上去。那可不是三五十贯就能买到的低档货,光是为了磨那镜片,苏颂可是卖了宿州的六十多亩上等水田。虽然在外面恍若无事,面对韩冈也一点不露心思,但苏颂的心中可也是恼火至极——不仅是外物,还有他那些借助千里镜的发现,全都得束之高阁了。

别看新学如今借助《字说》的问世,天子的垂青,一时间声势大振。可一旦有人能拿出真凭实据来戳破画皮,冲上将会是争先恐后。

真凭实据,苏颂手上就有。气学讲究着以实为证,这一份证据,新学无论如何都没办法辩说过去。纵然天子咬着牙坚持,但士林中对新学虎视眈眈的可是为数不少,到时候,纵然有皇帝主张,但新学想聚拢士林人心,也就只能靠科举了。

硬顶着风头来,或许会让让天子难堪,只是落到头上的责罚,也不会太重。大不了出外,对苏颂来说,也不算什么大事。到时候,又能有足够的时间来研究天文星象了。

终于有了决断,压在心头上的巨石也就不复存在。苏颂如释重负的站了起来,小心的将桌上的药材收好。转身看见儿子毕恭毕敬的站在身后:“还站着做什么?还不去睡?为父可是要去睡了。”


