\section{第22章 瞒天过海暗遣兵(六)}

【辛苦了三个月,终于有了过万收藏了。多谢各位兄弟捧场。接下来,俺会更加用心去写,回报各位的支持。第一更,求红票,收藏】

事情顺利的按着计划在发展。听到苗授已经传回了预定中的军情,韩冈当即怒色上脸,厉声说道:“别羌胆大妄为,本不过一跳梁小丑,竟然屡次阻挠王事。前次星罗结部追随董裕来犯古渭,当时已经放过了他和他的兄长。没想到此人怙恶不悛,竟敢一犯再犯。天作孽犹可恕,自作孽不可活,今次不能再放过!亦得让蕃人明白,朝廷不只有和气春风,亦有风暴雷霆!”

韩冈严词厉色,演技则稍稍过了点,但身在厅中的胥吏们则纷纷暗自叫好。并不知道内情的他们,生在关西、长在关西,拥有着对异族刻骨铭心的仇恨,韩冈的一番话正是说到了他们的心底。

王厚紧跟着拍案而起:“玉昆说得好!此等蕃人,若肯顺天应人,及早归顺,朝廷必不吝赏赐。但若是如今日的别羌星罗结这般愚顽不化,就该严加处断,以儆效尤!”

高遵裕颔首赞同:“玉昆、处道说得正合我意。此贼不除,何谈安抚河湟。”

“速传赵隆来!”王韶随即下令。

他拿起笔,先飞快地写了一份告急奏文,令人加急传回秦州。接着又几笔写好了一份军令,签过押盖过章,交给高遵裕签名。等赵隆奉命赶来,王韶便把封缄好的军令递给他,并交代道:“你领一队速去渭源,让苗授仔细体量敌情,凡事可临机处断,必要时便当直捣敌巢,擒别羌而归,以振朝廷的声威!”

赵隆慨然领命,单膝跪倒,双手接下军令,接着雄壮如山的身躯霍然而起,转过身,踏着沉重的脚步,一阵风的跨出厅外。

王韶低下头像是思忖了一下,自言自语:“苗授如果出兵星罗结,渭源便无人执掌。若蕃贼趁渭源空虚突袭城中,在前方的苗授别说得胜而归,恐怕自身难保。看情形不能不去渭源坐镇啊……”他抬起头,对高遵裕道,“接下来几天,古渭可就要劳烦公绰了。”

高遵裕点头对王韶笑道:“子纯大可以放心去,古渭自不会有失,我就静待子纯将捷报传回。”

“玉昆,”谢过高遵裕,王韶接着又吩咐起韩冈:“渭源虽然一开始就为了提防蕃贼突袭,准备下来的粮秣兵械都不少,但军情多变,谁也说不准会有什么变化,准备下来的物资也许并不够,必须要从古渭送上去。这前后军中转运之事,我就交给你了。你在古渭多多准备下粮草军器,将之及时运抵渭源军中。”

韩冈抱拳回道:“安抚放心,下官必然竭心尽力。”

得到两人的承诺,王韶松了一口气的模样,“城中军力有一半去了渭源护翼筑城,幸亏有他们在,不然只要百名蕃骑,就能把筑城的民伕都给杀散掉。不过古渭已经少了一半兵,剩下的就不能再调动了。二哥,你去通知张香儿来见我,明日要让纳芝临占部随军一起出发。”

王厚领了命,便急匆匆地出去找张香儿了——在渭源传来紧急信报的时候,要将这条老狐狸从他老巢里挖出来,不让他装病躲避,也只有王厚等寥寥数人能做到。

从刚进门听到蕃人突袭渭源的消息,到现在王韶命苗授迎战,并决定亲自坐镇渭源,一连串事情的发生,让智缘都反应不过来。看到了方才的一幕精彩的演出,智缘根本就想不到,这一切的安排其实都是早就确定下来的,不过是为了隐瞒这边主动挑起战火的真相,而刻意在他面前表演了出来,希望他能将之传回秦州。

出战在即,王韶和高遵裕都无暇与智缘闲谈,道了声不是,便让韩冈送他回住处安歇。韩冈转过头来,将智缘送出厅外,叹道:“可惜了大师一片苦心。本想着送大师去劝服别羌,谁想到他会一条路走到黑。其人自寻死路,也救不得他了,还请大师在古渭稍留几日,等渭源捷报传回,再前去抚慰亡灵。”

智缘没有应声。王韶的处断有个地方让他想不通,他问着韩冈:“为何不命青唐部出战?论起军力,青唐部当是在纳芝临占部……”话刚说到一半,便警觉道,“贫僧多言了,还请机宜恕罪。”

“无妨!这些事就算大师不问,我也是要说的。大师日后要行走在河湟边地,对蕃部的了解是少不了的功课。”韩冈向智缘解释道,“前次两战大捷,都是安抚驱动蕃人打下来的。俞龙珂和瞎药至今仍未完全顺服于朝廷,也是因为他们自负手上的军力,而不肯屈就。如今有了机会,也得让俞龙珂和瞎药看看官军的实力,省得他们以为自己不可替代。”

听了韩冈的话,智缘欲言又止,因为韩冈回答的并不是他的问题。韩冈会意笑道,“纳芝临占虽然名义上是蕃部,但都是当年的陷蕃汉人的后裔,族酋皆为张姓。素来亲附朝廷,在这里,要比青唐部这等真蕃亲近得多。可以当汉人看待的。”

“原来如此。贫僧受教了。”智缘竖掌行礼,“若今次能彻底击败别羌,日后当可趁势夺下狄道,平定武胜军。”

韩冈现在所处古渭,与木征的河州之间,隔着一片方圆两百里的土地。原本是董裕的领地,今名武胜军,属于后日的临洮县,是黄河支流的洮水【今洮河】流经的地方。在唐时此地属于兰州,而在五代,则是被命名为武胜军。

如果说收服如今的吐蕃赞普董毡,是王韶拓边河湟的最终标志,那么击败木征,攻克河州便是实现目标的必要条件。而挡在河州之前的武胜军,就是要最先占领的地盘。

韩冈指着西面的山峦,“真要计较起来,渭源也算是武胜军地界了。翻过渭水源头的鸟鼠山,对面就是洮水。武胜军的中心狄道就在洮水边,渭源离仅仅隔了一百多里。”

“若贫僧记得没错的话,狄道就是临洮。‘北斗七星高,哥舒夜带刀,至今窥牧马,不敢过临洮。’如今临洮沦于胡虏数百年,不知何时能重现旧日大唐的盛况。当年玄奘大师取经而回,其经文经过多年,已经零落不堪。若是能交通西域,从天竺重新把经书迎取,可是能流传千古的大功德。”

“大唐之威,的确是让人追慕。”韩冈悠然长叹,“今日辛苦争夺的河湟,只是当年的数州之地。而统归大唐的安东、安北、单于、安西、北庭、安南六大都护府,其地域之广,任何一个都能比得上今朝的半壁江山。”

在安史之乱前,虽然唐朝对吐蕃有着几次大败,但其国势还是逐渐延伸到西域葱岭。而当时的河湟之地,也是大唐所领。安西、北庭两大都护府,在西域为汉家拓张了万里江山。这一番辉煌,至今犹在西北百姓口中传唱。连当今天子赵顼,也是对唐太宗的功绩深为敬服,进而追慕不已。

当然,河湟之地沦于吐蕃人之手,同样是被唐朝丢掉的——是安史之乱后中原势力大衰的缘故。自吐蕃开始,沙陀、党项、契丹纷纷侵入中原,所造成的后果,说句难听点,就是如今宋室始终难以振奋的主因。而偃武修文的国策也是因为晚唐五代子弑父、臣弑君的武人之乱,给宋初君臣们留下了太过深刻的恐惧,才顺理成章地形成。

但这些,韩冈就无意再提,要比就往好处比,比烂则是毫无必要——毕竟,总是有更烂的。老是想着后面还有更差的,反而就没有上进的动力了。

智缘也不会说出这些煞风景的话来,他更是为韩冈的话勾起了心思,隐藏在他眼神中的,全然没有半点属于出家人的平静:“当年李卫公等诸多名将,败突厥,破回鹘,让胡人不敢东顾。如今,汉家天子欲重定西土,不世功名,也正在今日!”

……………………

“观察,东朝的王韶自领有缘边安抚司之后,越发的咄咄逼人。今次他能在渭源筑堡,明日就能穿过大来谷到狄道筑城。等到他控制了的武胜军,不知观察到河州还能保得住?”

木征半闭着眼睛,靠在一堆毡毯中。他在宋为河州刺史,在夏则是河州观察使。宋人称呼他一声刺史他应下,眼前的这位禹臧家的使者称呼他观察,他也照样应下。

“这不还没打到狄道嘛,我只要保住河州就够了。”木征懒洋洋的说道。

木征并不是很有野心的人。他对用兵扩张没兴趣,也无意跟他的叔叔去争吐蕃赞普的位子。只要不打到他家门口,他最多也只是派点兵凑个热闹,绝不会跟人硬拼。

当年其父瞎毡早亡,他被逼得放弃河州,躲到西北的安江城。后来他聚集部众,也只是打回河州就停手了。武胜军原本是他二弟董裕的地盘。董裕死后,木征顺理成章的接收了这片土地,但他没有留给自己,而是交给了他的另一个弟弟瞎吴叱。比起他的那个心比天高的二弟来,木征的性子可算得上是小富即安。

