\section{第30章 肘腋萧墙暮色凉(四)}

见到韩冈,王文谅显然有些尴尬。犹豫了一下,不知道是不是该上来跟韩冈见礼。

而韩冈却站起身,不仅是为了王文谅,更是为了跟着王文谅后面进来的那个。

“原来是王阁职,韩冈有礼了。”韩冈先向王文谅打了个招呼,然后对其身后的种建中笑道:“彝叔,久违了!”

种建中跟着王文谅一起抵达延州,前来求见韩绛。他见到韩冈,当即大喜过望,等到王文谅与韩冈见礼后,就连忙上前:“就猜到该是玉昆你。”他拉着韩冈的手笑道:“方才进城后先去了驿馆,正听说有个韩官人来了,不过赶着过来帅府,没能细问,但想着就该是你……家叔和愚兄在绥德日盼夜盼,盼玉昆你多日了,怎么到今天才到延州?”

“小弟可是离了京城后,就紧赶慢赶,没敢耽误一刻行程。”

韩冈与种建中谈笑了两句,也请了王文谅一起坐下来,等着里面的传唤。

韩冈没提被关进大狱里的吴逵的事,此事与他无关,他也不会为一个萍水相逢之人而出头。应酬式的跟王文谅说了两句,他便问种建中道:“今次种帅半日克复罗兀,威震雍秦。小弟来延州的这一路上,正看到露布飞捷过处,各州各县的官民无不赞着种帅的功绩。不过罗兀虽得,但西贼必然想要重夺回来,彝叔在种帅帐下参赞军务,怎么有闲来延州的?”

种建中听着韩冈相问,顿时眉飞色舞起来:“愚兄是随王阁职一起押送缴获的首级军械而来!家叔领军夺占罗兀之后,西贼当然不肯罢休。当日银州守将西贼的都枢密都罗马尾,便领军两万,意图救援罗兀,不过在马户川为高都知所破,而后数日,都罗马尾又聚兵三次来攻,其兵力一次多过一次,但皆为我所败,旗帜鼓号丢了无数,最后再也不敢来了。这数战,总计斩首一千两百余级。而罗兀附近的部族也纷纷归附,已经计点出来的,有三部共一千四百余口!”

“一千两百余级?!”韩冈脸上的惊容却是难再掩住。败敌人数能胡吹海吹,但斩首数作假却是麻烦,而且就算作假也容易被人看破。如果这个数字是真的,横山这边的斩首功,又将反超河湟,成为天子登基以来第一功。

“是啊!”种建中得意的笑着,“辛苦了许久,终于可以望河湟之项背了。”

“谈什么项背?”韩冈摇头苦笑,“就是不算斩首数,吐蕃也不能与党项相比,何况斩首已在河湟之上。当是望尘莫及啊……”

韩冈自认不如,种建中兴致又高了三分。凑近了,低声对韩冈道:“游景叔前日又来了一次信,说当日在京兆府遇上玉昆你,对突进罗兀之策,好似也是不以为然。”

韩冈不意游师雄竟然把私下里说的话转述给种建中,暗骂游景叔多嘴之余,有着几分尴尬。忙解释道:“那是因为小弟担心罗兀距绥德过远,粮秣军资难以支持的缘故。”

种建中哈哈笑道:“玉昆是多虑了。家叔事前早已侦知横山粮秣尽集于罗兀,故而出兵时,就只待了三日的口粮。而等打下罗兀,便尽以夏人屯粮为食。计点食用,所将步骑两万,并民伕万人,共耗官米二斗二升,草六束!”种建中张着双手,用手指比划了几个数字,洋洋自得的继续说着,“家叔的那匹韩相公亲赠的河西龙驹青电,嘴刁得很,就是不肯吃党项人的粮草。要不然,也不会有这二斗二升米和六束草的消耗。”

粮草耗用的数目是否正确姑且不论,种谔在罗兀城中的大丰收当是确凿无疑——没哪个将军敢在粮草问题上给自己吹嘘的,只会叫着不够吃。

也就是说,事实证明了韩冈的担心是杞人忧天。

韩冈一直以来都是对韩绛主持的横山攻略报着否定的态度,而现在种建中当面拿话驳他,他心中却也没什么不痛快的。

攻下罗兀,当是情理之中,以韩绛和种谔这半年多来的精心准备,若是做不到,那就是笑话了,西军的脸面都能丢尽。但守住罗兀,可就没那么容易了。孤悬在外的城池,究竟能在西夏人的攻势中守住多久?——那可不光是粮草方面的事务。

从年初二攻下罗兀,到现在过去八天了,捷报当是已经传到京中,种建中也押着战利品到了延州,而西夏那边,兴庆府也当收到了消息。如果梁氏兄妹还有一点战略眼光的话,肯定会立刻点集大军前来。就算环庆、泾原和秦凤那几处会出兵牵制,都不可能阻挡党项人对失去横山的恐惧——以党项人征召部族战力的速度,还有兴庆府与银夏的距离,韩冈估计种谔大概有一个月的时间做准备。

能否赶在他们到来之前,把罗兀城的防御体系建好——至少修造出个大概——难度可不是张张嘴那么简单。前线的核心城寨,其基本规模,是战时至少能容纳万人驻守、平日也要能放下三千兵驻屯的千步城。甘谷、古渭、清涧、绥德、大顺,无不如此。

即是说,罗兀那里至少在一个月之内,要修好一座周长千步的城池。另外,罗兀防线不光是罗兀一城,周围协防的附堡,守卫后勤线的军寨,都要敢在一个月之内打造完毕。而且还有城中的防守物资,也要在同时运抵罗兀。

可如今是冬天,天寒地冻的冬天,土地冻结的冬天。一边在河谷中不停的受着寒风的侵袭,一边还要从冻得跟石头一样的地面上取土筑城,民伕们能支撑多久?这可不是个容易回答的问题。

不过身在韩绛的门厅中,韩冈觉得还是少说为妙。附和着提了一句:“只要能守住罗兀,得到横山,那西事也就定了。”

“我皇宋待蕃人最是宽厚不过,而西贼则是刻薄已极。一旦横山蕃部看到西贼难挡我皇宋兵锋,那时就会纷纷来投!……横山一附,西贼指日可平!”

从种建中的这句话上,就能知道韩绛厚待王文谅这个蕃人的用意所在。

王文谅听话好用只是个末节,最重要的是韩绛有着千金市骨的盘算。横山蕃部都在看着,看着大宋如何对待蕃人。当他们看到王文谅这名西夏前任国相门下的家奴,竟然在大宋混得风生水起。当然会有投靠大宋,自己应该能得到更好待遇的想法。

不过可能就是因为韩绛太想把王文谅这蕃人的变成马骨的缘故,他在陕西的人缘看来很不好。不然种建中在跟自己说话的时候,也不会一句话也不带着王文谅说。

吴逵是一桩,种建中又是一桩,从王文谅的人缘中看,韩绛并不是会用人的那一个类型。瞧着脸上写满不耐烦的王文谅,韩冈倒有三分期待,千金市骨的戏码如果玩不好的话,可是会变成千金买堆臭狗屎,最后烂在手中,香飘千里。

作为以宰相身份统领陕西、河东军事的宣抚使,要来求见韩绛的官员有很多。不过王文谅和种建中显然很得韩绛看重,韩冈也只跟他们谈了一小会儿,从内间出来的侍从就把两人叫了进去。

种建中向韩冈陪了声不是,就跟着王文谅一起走了进去。

两人后至,却能先得到韩绛的召见,韩冈并没什么异议,这是理所当然的,人家可是带着战利品回来的功臣!

过了一阵,种建中和王文谅出来了。王文谅先走,而种建中跟韩冈又聊了两句,也告辞说要去拜访几个朋友。让韩冈见过韩绛后,回驿馆先等着,晚上两人再好好喝一顿酒。

韩冈答应了,继续在帅府行辕的门厅中等候。时间慢慢的过去,他渐渐就觉得有些不对劲了。后至而先入的不仅仅是王文谅和种建中,来求见韩绛的官员一个接一个被叫进去问话,却就是不见有人来传他韩冈。门厅中的官员不断变换,就韩冈一人始终坐着。

到了傍晚时分,从里面出来一个小吏,说天色已晚,相公视事劳累,已经倦了,命门厅中的众官吏有事明日再来。

在门吏奇怪的眼神中走出帅府大门,韩冈心中隐怒,这是分明是韩绛故意怠慢于他。

当初他去京城,虽然在王安石府的门厅中等了近十天,但当时王安石正拿着辞官的幌子逼天子继续变法,根本不见外客。而当今次上京,王安石就忙不迭地派人来请他。从他韩冈入官以来,何曾受过如此慢待?就是在天子面前,他韩冈也是极受看重,只是因为各种阻碍,才没能登殿面圣。

就是不知道韩绛的怠慢究竟是何缘由,韩冈百思不得其解。难道是他在王安石府上的言辞,传到了韩绛的耳中?但也不至于故意晾着,他可是两封奏章调来延州做事的,要想跟他韩冈过不去,先得让他把事情做起来,空晾着反而不好找茬……

