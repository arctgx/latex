\section{第34章 雨泽何日及(六)}

王安石已经回到了中书门下自己的公厅中。

坐在熟悉的座位上,王安石轻声一叹,如果不是韩冈在殿上的一番陈词,扭转了天子的想法,现在他要做的,就是回家写奏疏,自请出外。

对着站在身前的儿子和助手们等待结果的眼神,王安石微微笑道:“勿须多虑,多亏了韩玉昆。”

前面已经有了一点模糊地消息,现在终于从王安石口中得到确认,吕嘉问顿时如释重负,方才他在心中不知念了多少声阿弥陀佛,眼下这一道鬼门关总算渡过去,不由自主的,一声佛号就脱口而出。

对上一起投过来的视线,吕嘉问有点尴尬的笑道,“关心过甚,见笑了。”

“谁能笑望之你,”曾孝宽摇头苦笑:“我等方才都失了分寸,也就是吉甫沉稳。”

王雱瞥了一眼曾孝宽,道:“也多亏了吉甫,要不是他打听到了郑侠献了流民图,猝不及防下,韩玉昆怕也难应对如常。”

吕惠卿回以温和的笑意。他一开始的焦急倒也不是装出来的。王雱为王安石和新法忧心不已,吕惠卿当然也是同样的关心,只是顺序要反过来,新法在前,王安石在后。但后来稍稍冷静下来,就已经全然安心。

他对王安石道:“惠卿素知韩玉昆之才,当年初上京时就侃侃而谈,如今新法推行得力,也少不了他的一份功劳。试问他怎么可能的不用心辩驳?”

‘可惜啊……’

心思与言辞截然不同,但吕惠卿的笑容没有什么异样。

本来他是想等韩冈在天子面前将白马县之事辨明,自己入宫再请对。吕惠卿有足够的把握将天子的心意彻底扭转回来。只是没想到韩冈一个人就将事情办成了,甚至比自己准备做的要更上一层,倒让自家的一番心思化作了泡影。

这一下子,只能收起心思再等上一段时间。

吕惠卿现在是满心的不甘。

从本官来说,他和韩冈都是从七品的右正言。只是到了朝官一级之后,本官高低已经不重要,重要的是差遣、资序和馆职贴职。翰林学士可比要集贤校理要髙得多;中书检正、判司农寺、集英殿侍讲,更不是区区一个白马知县可比,上朝时排定班次,自己能排在前面的二三十位,而韩冈差不多得在殿门边上。

但韩冈转眼就是府界提点,或许过上几日,就能又追过自己。就算追不上来,可见着年纪只有自己一半的韩冈能与自己拥有着同样的官阶,吕惠卿怎么可能心中没有疙瘩?

不过如果给了自己力挽狂澜的机会,就能立刻跨上一大步,将韩冈远远的抛在身后,让曾布嫉恨不已。

王安石要为大旱负责,避免不了的要辞去相位,但要保住新党,吕惠卿本有着足够的自信。

废新法?那是旧党之流只能在梦里实现的幻想。

说句难听话,如果天子现在尽废新法,转眼就要坐吃山空。到时候朝廷养着的文官武将胥吏士卒,连带着他们的家人亲友,数百万张嘴张大了要吃饭,看看天子又能怎么办?

大手大脚的花惯了钱,怎么可能再节省得起来。已经给胥吏发了几年俸禄,突然说不发了,看看下面闹不闹?更别说这两年给官员的加俸,给军中的加俸,难道还能再削减?

别看如今旧党见到大灾连年,叫得春天的猫狗一般欢快,真换了他们上台来废掉新法,比熙宁初年更为严重的亏空,谁能解决?是坐拥千顷土地的韩、富、文,还是只知道要天子节衣缩食的司马光?

只要仗着这一点,天子就根本不敢动新法一下。就算一时废掉,也要重新恢复。

可惜了这个机会。

吕惠卿暗自惋惜,又与王雱、吕嘉问一同,开怀的笑起。

王雱笑过,又想起了今天的功臣:“不知道玉昆在开封府那里能不能说服孙永,今次河北流民可就全得靠他来安置了。”

“不用担心。”吕惠卿道,“孙曼叔现在巴不得有人能帮忙处理好流民。”

换作任何一位开封知府,若是听说有人能解决涌来开封的数以万计的流民,肯定是大喜过望,恨不得立刻将手上的这一摊子事交出去,而且会全心全意的支持,不会拖任何后腿。不管怎么说,流民都是在开封府的治下,出了点事,不但韩冈要遭灾,连开封知府也少不了要受牵累。

正如吕惠卿所言,接下来的数日,有天子、有宰相,再加上开封知府做后盾,韩冈顺顺利利的将府界提点衙门接手,在他的指挥下,天下汇聚于开封一府的庞大资源,开始源源不断的流向旧滑州三县。

韩冈对河北流民的决战之地,也就打算放在旧日的滑州。

……………………

身在安上门,听到了御史台来人带来的‘送御史台根堪奏闻’的通告,郑侠没有丝毫动摇,上书数日来毫无音讯传回,他已经猜到了今天的结果。

平平静静的将公事向下属交代清楚,郑侠回头对着领头来捉人的吏员道:“好了,可以走了。”

在官员中闻之色变的御史台内,郑侠昂首阔步,没有丝毫畏缩,挺直的腰背,严肃的神情让他带着一分悲壮。

被押解进御史台的三堂,郑侠在堂中站定。一名御史高高坐在上首,一拍惊堂木,高声喝问:“郑侠,你可知罪?!”

郑侠昂起头,坚定地双眼盯着堂上的御史:“若说擅发马递,郑侠甘当其罪!若说将下情禀明天子,使权臣不能蒙蔽圣聪,郑侠则不知何罪之有?!”

听到郑侠的回话,蔡确叹了口气,他实在不想神这一桩麻烦的案子,但御史中丞邓绾报请天子后,将差事交到自己手上,他也不愿因为拒绝而开罪天子。

蔡确明白自己能在两三年间,就做到御史台的第二号人物,靠着的就是揣摩圣意。

罪轻罪重,端看天子的想法。如果天子接受了,那就什么罪名都不算数。

开封民妇妄敲登闻鼓寻猪算不算有罪?但太宗皇帝收了这桩案子,那就不是罪过,官府还要赔一头猪钱出去。

蜀中老秀才题下反诗‘把断剑门烧栈阁,成都别是一乾坤’算不算有罪?可仁宗认为这只是穷措大急着要官,就不算罪过,还给了他一个司户参军做安抚。

郑侠的上书,虽然是擅发马递,只要天子接受了他的奏疏。蔡确就会批一句情非得已,将罪愆给掩过去,发遣到开封府,让孙永给郑侠一个申诫了事,最多将其踢出东京城,让他到外地做官。

但现在赵顼既然不接受,而是正经八百的发到御史台来定罪,蔡确也不会违逆天子的心意。

当然,说郑侠妄言白马县中事,构陷朝臣的罪名,蔡确不会认同,那是要直接驳回去的。要不然,一贯风闻奏事的御史们全都得要下狱。同时,蔡确也要表现一下自己的气节——反正郑侠擅发马递,那就是铁打的罪名,没有必要在其他事上纠缠。

只是郑侠的态度让蔡确很不舒服。乌台何等地,连御史们吃饭的时候都是禁绝言笑,犯了就是要罚俸。哪一个来到御史台中的官员不是战战兢兢?就算有人胆壮得如虎如龙,三五天之内也要乖乖的变成一只猫、一条虫。

能在台谏之地抬头挺胸的只有御史!蔡确就是要将监门官现在表现出来的这股傲气打掉:“郑侠。你可知前日天子问起韩冈如何处置于你,他是怎么回答的?”

郑侠一声冷笑:“奸佞之辈自不会有好话!”

“韩冈说,‘朝廷治政,不当以言辞罪人,愿陛下斥其谬言,容其改过’。”

“惺惺作态,沽取直名!”郑侠的回答毫不客气。

“韩冈还奏请陛下,调你入府界提点衙门或是白马县,他说要让你心服口服。”

郑侠头仰得更高:“郑侠若要为高官显宦,早就可以做了,何须韩冈来?君子正人,岂会五斗米折腰?”

‘还真是嘴硬。’

蔡确笑了笑:“韩冈前日在延和殿中又说,他清晨曾见石上有水,树上有露,乃是降雨的征兆。想来郑侠你在安上门处也看到了吧?”

郑侠终于变了颜色,一张严肃傲然的脸,转瞬就涨得通红,愤怒的说着:“此乃污蔑!”

“污蔑?”蔡确哈哈一笑:“这两日,天上阴云渐多,今日更是不见艳阳,寒风阵阵,说不定当真就要下雨了。”

当韩冈在延和殿上奏对的一番对话传出来后,蔡确知道自己的亲家是不能如愿了。招了个好女婿,王安石一时还下了不了台。

而且韩冈手段高明,郑侠拿来赌命的一手,竟然轻而易举的被他化解了过去,顺便还将罪名栽了回去。听说了韩冈的手段,蔡确都有些后悔,过去他做的事太得罪人了,是不是找个机会,再与韩冈拉一拉关系。

低头望着终于不能再高傲的仰起头的郑侠,蔡确志得意满的冷笑一声。如此也就够了,这个案子其实没得审,郑侠又不是不认罪,而眼下形势尚未见分明,蔡确也没有将之重惩的打算,最多一个远州编管而已。

呼啦啦的一阵带着水意的风卷进堂中,将蔡确正要说出口的话挡了回去。然后就听见外面一片骚然,不知多少人在乱喊乱叫,轰轰的如同雷声,就连一向被威严沉重的气氛所包围的御史台,都一下沸腾起来。

蔡确疑惑的望着堂外,不知出了何事。但很快他就明白了,一道雨幕落了下来,落在了干涸已久的大地上。

听着外面的万众欢呼,和淅淅沥沥的雨声,蔡确轻轻拍了拍手,对着似喜似忧的郑侠:“十日不雨,乞斩于宣德门外。郑侠,你说得还真准……与韩冈一样准!”

