\section{第32章 忧勤自惕砺(下)}

朝会之后,王安石率领辅臣至东郊祈雨,而曾布等一干臣僚则得以提前面君入对。

听了曾布对市易务行事不依法度而败坏民生的一番奏报,赵顼面有喜色,“朕久矣闻之,非卿不得言。”

赵顼当然欢喜。此前他曾多次因为市易法惹起天下议论,而有心废止,但全被王安石给挡回来了。赵顼没有实据,只能听之任之。但灾情越发的严重,许多奏章都说这是推行新法所致。而新法已经推行五年,此前并无灾异,只是从去年开始才有了大灾,赵顼想来想去,当是施行了最后一部市易法的缘故。

现在曾布秉公直言,正是他忠心表现。市易法是新法之中最得争议的一条法令,如今被查出事端,换作是结党营私之辈,必然将其中情弊给瞒下来,以讨好宰相,并防止政敌藉此攻击。这等蒙蔽圣聪的行为,是每一个皇帝都难以允许的,却有无法避免。故而曾布所为,让赵顼看到了一个忠臣的出现。

等到王安石入宫回禀祈雨之事后,赵顼便立刻问道:“曾布言市易不便,卿家知否?”

赵顼的发问突如其来,王安石却神色平淡。最为信任的助手反戈一击,这一刀子等于是捅在他的心口上,但经过了一夜,他已经调整过来。遂点头道:“知之。”

赵顼双眉一扬:“曾布所言如何?”

王安石立刻回道:“曾布与吕嘉问有隙,其相争亦有牒文可见。”

王安石将曾布的一番奏报,说成是对吕嘉问的构陷,赵顼不快的说道:“可朕亦曾听人言。京中多有卖尽家产,遭市易务关押枷固之辈。人数之众,以至于市易务乏人监守。”

王安石随即说道:“既如陛下所言,此人必知卖产者及受刑者之所在,陛下何不明示其人姓名,交付有司推问?若确实有之,市易司隐而不言,其罪固不可轻恕,当严惩之。若无实据而妄言,不知陛下包容此人于政事何补?”

赵顼叹了口气,王安石永远都是这样的理直气壮:“王卿可知,这数月来太皇太后和太后在宫中日夜长叹,心忧天下因此而乱。”

王安石的眼神更为严厉。妇寺干政,乃是国中大忌。赵顼在廷对上拿出太皇太后和太后的话来说,换作是平常,王安石都能强硬的给堵回去。但眼下的形势,让他不便抓着此事来发作。

深吸了一口气,他沉声说道:“陛下宣示两宫忧致乱,臣亦忧致乱。诗曰:‘乱之初生,僭始既涵’。臣之所忧,正本于此。陛下试思诗书之言不知可信否?如不可信,历代不当尊而敬之,开设学校以教人,孔子亦不当庙食。如其可信,祸乱之生即源于此。”

‘乱之初生,僭始既涵’的下一句就是‘乱之又生,君子信谗’。王安石直指赵顼轻信谣言,才会致使祸乱,而非关市易务之事。

不等赵顼说话,王安石抬起头,声音转厉,“齐威王三年不治国,一旦烹阿大夫,举国莫敢不以实情禀上,国遂治,兵遂强。僭生乱弱,信生治疆。如此,臣愿陛下熟计之!”

春秋齐威王三年不治国,身边小人环伺。即墨大夫善抚民,却被威王小人日夜以谗言攻之,而阿大夫不安民治政,却买通近臣,日日得到称赞。不过齐威王派人暗访得实情,将阿大夫和身边小人一齐下了大鼎烹死。自此,无人再敢欺瞒于他,而齐国遂兴。

但王安石拿齐威王比拟当今之事,乃是强辩,赵顼也明白,以王安石之材,一件事正说反说他都能找到典故来做证据。只是要看有没有道理罢了。

王安石说了这么多,赵顼也变得有些疑惑,也的确觉得当派人调查清楚再说:“既如此,且令曾布与吕惠卿同根究市易务不便事,待二人诣实回禀,再论。”

……………………

司农寺的公厅中,吕惠卿很快就得到了消息。在一瞬间的惊讶之后,是对背叛者的愤怒,但很快,一丝淡淡的笑意在嘴角浮现出来。

“曾子宣太心急了。”

这真是个好消息。

曾布叛离新党,得益的当然是他吕吉甫。

司农寺是新法的立法机构,而中书检正则是负责推行,原本都属于曾布的差使,现在皆由他吕惠卿来主持。但任谁都该明白,以王安石的性格,决不至于如此厚此薄彼,曾布其实必有任用。可惜曾布是一叶障目不见泰山,完全给怨意蒙蔽心神了。

曾布的倒戈一击,对于整个新党的确是个大挫折,但对吕惠卿来说,却是个良机。

吕惠卿环视左右,他刚刚入主的公厅中,还有着旧人留下的痕迹。陈列、摆设都是由着曾布的个人习惯,但吕惠卿相信,只要一个月,他就能让这处新法的核心之地,成为他手上得力的工具。

当然,曾布现在并没有输。如果他能在市易务之事上,能说服天子,将吕嘉问论之于法,那他就会是第二个蔡确,以忠心受到天子的看重,升任执政就是转眼间事。不过若是他败了,京城之中可就再没有他落脚的余地。

吕惠卿从袖中抽出一份早已写就的文书,本来他正犹豫着发出的时机,不过现在就没有什么好忌讳的了:

“本寺主行常平、农田水利、差役、保甲之法,而官吏推行多违本意,及原法措置未尽,弊症难免。今榜谕官吏、诸色人陈述。如有官司违法之事,亦可一并投于本寺按察。”

吕惠卿默念一遍,两指捏着薄薄的纸页轻轻一抖,唇边绽出一抹得意的笑容。

此文一下,曾布之叛就再无转圜的余地!

……………………

夕阳终于没入了地平线下,夜中河上无法行船,渡船都在岸边下了碇。

白马津的渡头上,点着火炬,灯火通明,照得内外如同白昼。

今天最后一批抵达南岸的流民,就在渡口外排着队。他们都在粥棚盛了热腾腾的菜粥,一边填着肚皮,一边听候着安置。

抵达白马县的流民,都是依着乡族籍贯来安排,是小聚居,大杂居。来自同一乡的流民住在一起,可以互相照应。但上到县一级,流民就必须打散,以防其中有人串联起来作乱。不过也是视人数而定,并不是那么死板。

“今天渡河的流民有三千三百一十八人。”今天的人数终于点算完毕,韩冈在渡口内厅听着汇报,王旁和方兴一起走了出来,“连黎阳那边也免了渡资,渡河来的流民果然一下就多起来了。”

方兴笑道:“黎阳的杨知县也是聪明,若是他不将渡资免了,流民必然都要等着免费的船坐,几万流民不知何时能渡完。流民多留一天就是一天的麻烦,若是逗留在境内出了事,要比推卸责任,他肯定比不过正言。还不如一起免了渡资,就算有人拿来说事,也可以请正言出来顶着。”

王旁道:“今天天子已经允了玉昆的奏疏,想必杨知县得到消息后,也可以安心了”

一串急如密雨的蹄声这时从南面过来的官道响起,由远及近,声音渐渐变大,很快一名骑手埋头大汗的来到渡口旁。他跳下马,几步走近前,将一份递给韩冈的随从。

王旁回头看了一眼,又转回头来,“不知是哪里来的消息?”

“大概是京城又来问流民安置的事。”方兴猜测着。

流民渡河南下,黄河上的几个渡口,隔三五日就要将过河的流民人数上报中书。而白马县这里,更是天天要禀报开封府。白马县现在每天都能收到京城传来的公文,而韩冈这几天因为渡口初启,就都在白马渡坐镇。也吩咐了下来,抵达县中的文书都要立刻转到白马渡这边来。

方兴瞅了瞅黑黝黝一片、只能听到哗哗流水声的黄河,再望望黄河对岸的大堤上,一字排开十数里的火光,不由的感叹起来:“若是滑州浮桥能重修就好了。”

旧时滑州黄河上设有浮桥,但屡屡因水涨而冲毁,如今不得不仍以船只来摆渡。现在黄河出潼关后,也就是孟州河阳津,还有东面的开德府【澶州,今濮阳】处有浮桥。

王旁听了,心中顿时一动:“浮桥?”

“嗯!”方兴点了点头,“有了浮桥,黄河上可就日夜都能行人了。正好如今要驱用流民,工钱也不要太多,加之黄河水枯,建造浮桥也方便,更不虞洪水冲毁。”

王旁听得连声称是,急忙问道:“此事玉昆怎么说?”

方兴摇摇头,他也是刚刚才想到:“尚未与正言提及。”

“那还不快去说?!”王旁催促着,兴建浮桥。

“正言。”方兴在王旁的催促下,来到韩冈身侧,就想跟他提及浮桥之事。却不意发现正低头看着手中信笺的韩冈,他神色有些不对,“正言,出了何事?”

韩冈折起了信笺,摇头叹气:“一滩烂事!”

