\section{第20章 心念不改意难平(十)}

从郭逵那里出来,韩冈就有些后悔,自己方才是不是说得过于隐晦了一点。要是郭逵没听明白,把他的话当成是敷衍,就有些让人头痛了。只是再一想,郭逵好歹在官场中沉浮多年,不会如此迟钝。

韩冈并不是想要弃王韶投郭逵,但他还是希望能由久负盛名的宿将来主持河湟开边的战事。河湟开边虽然是以招抚为主,但最终还是少不了一战。为了能让这一战的胜率增加一点,选择能力更强的将帅,也是理所当然。

王韶不是名将,而郭逵是。王韶有着战功,在军事上也有才华,但他的经验和威望远远比不上郭逵。在面临大战的时候,郭逵只要亮个相就能振奋起来的士气,王韶就要长篇大论,跟将领们一个个面谈才能做到——而且还不一定。在遭逢危局的时候,郭逵能让军心坚韧如山岳,而王韶不拿起屠刀,就无法将浮动的军心镇压住。

如果郭逵跟王韶水火不容,如李、窦之辈把韩冈当作攻击的对象,韩冈当然会设法反击。但郭逵却是向他表示善意,有着重用于他的想法,那韩冈还有什么理由要跟郭逵为敌?可是他再怎么想,以郭逵和王韶的性格,最终冲突起来的几率至少都会在八成以上。

难道还要帮着王韶把郭逵赶走,就像李、窦、向三人那样?同样的情况一次次的重复,朝廷上对王韶的肯定会产生看法,而韩冈自己想想都觉得烦。

韩冈穿过庭院,心中还在想着怎么才能调和王韶和郭逵之间的关系。一抬头,却惊觉州衙大院中,捧着大叠大叠的卷册的小吏比平日多了数倍。韩冈挥了挥手,示意迎面过来的那些抱着大摞卷册的小吏直接过去,用不着行礼。

“又到了要忙的时候了。”就在韩冈还是做着勾当公事的时候,他手下的胥吏就已经在叹着了。

每隔三年,一到八月,秦州……确切的说,是全国各地的州衙县衙还有路份监司就会一下忙碌起来。并不是因为到了征税的时节,夏税在六月,而秋税在十月,而是为了三年一更造五等丁产簿。

五等丁产簿记载了户中人丁和家产数额。而家产数额确定了户等,而从一等到五等的户等,则决定了赋税数额。

今年正好是时隔三年的重新划定民户户等的日子。为了确定接下来三年税收数目的,县中的胥吏要下到乡里,与乡中里正、书手一起,丈量土地,点验家财,然后确定户等。

把这些数据搜集起来后,就一式四份的重新造册,一份县中自留,一份送到州中,剩下的两份则分别送入路中监司和京城的三司衙门。这一套流程,从八月开始,一直要持续到年终,中间还穿插了秋税,每一个吏员都是少有能喘气的时候。

韩冈突然发现,自己方才好像耽误了郭逵的工作。郭太尉不仅是秦凤经略,同时也是秦州知州。他的任务并不局限于军事,同时包括了政事、民事。

重造簿册,对亲民官来说,是最重要的一件工作。千年前,萧何随军入咸阳,第一件事就是控制了咸阳城中的户籍簿册。而如今边境蕃人纳土归降,要做的第一件事便是编定户籍,并呈交朝廷。

虽然韩冈并不知道三年前是个什么样的情况,但他确信,今年的州衙县衙,将会格外的繁忙。

朝廷新近颁布了免役法,改变了延续千年的徭役制度,变差役为雇役。各家各户只要交上了免役钱,就可以免除原本会弄得倾家荡产的差役。而旧有的衙前、工役、苦力等徭役,便由各级衙门使用征收到的免役钱,通过雇佣人力来完成。

为了准确的统计出各家各户需要缴纳的免役钱,重造五等丁产簿便是不可缺少的关键一环。

同时随着免役法的实行,重禄法也跟着公开。各路胥吏将在今后三年内,逐渐开始由官府来发给俸禄。原本的胥吏从编制上说,属于长名衙前,是服役之身。就跟其他服徭役的百姓一样,都是自备钱粮,他们的吃穿用度,官府根本不予理会。

如果胥吏不盘剥百姓,那唯一的结果就是坐吃山空,把家产折耗干净。而等吏员们有了俸禄,朝廷也就可以名正言顺的严肃吏治,制止他们再向百姓出手。虽然这是能算是良好的理想,但终究还是会有一点改善。即便是一丁点,只要能比过去好就行了。

前几天听说了重禄法的公布,以区区一个选人的身份,却能影响到朝廷策令,韩冈当时心中就平添了一股指点江山的痛快。当初他给王安石的几条建议,看起来真的是一步步的在施行。

走出州衙,李小六牵着马迎上来,而同在门外的还有一队骑兵。作为缘边安抚使司机宜,韩冈跟当初的王韶一样,有了一队亲兵护卫。

“机宜,可是要去古渭?”李小六把缰绳交给韩冈,出言问道。

“当然!”韩冈双手一搭马背,转眼就骑在了马背上。他方才就是向郭逵辞行,想说的话即已送到,接下来就是离开秦州,赶往古渭。“你们准备好了没有?”他回头问着李小六和一众亲卫。

亲卫们跟着一起上马,在马背上一抱拳:“还请机宜下令。”

韩冈正要动身,李信从州衙中疾步赶了出来,叫道:“三哥,等等!”

韩冈一见,不得不重又翻身下马,“不知表哥有何事?是不是要小弟带话去古渭?”

李信摇了摇头,喘了口气,把气匀了,便对韩冈道:“是钤辖让我带话给三哥你。”

“钤辖说了什么?有何要事?”韩冈虽是在问,心中已经隐隐有了答案。

果不其然,李信对韩冈道,“钤辖倒是没什么要事。只是要三哥你去古渭时,顺便带话给渭源堡的王君万,让王君万尽心做事,他家中钤辖自会遣人照看,无需担心。”

韩冈点头:“小弟会给王堡主把话带去的。”

“没了!”李信顿了一下,忽而又道,“对了,今天早间,钤辖还提起三哥儿你当初拒绝了他的举荐,而接了王安抚荐书的事。赞三哥你有眼光,会选人。”

‘那老家伙还在为当初的事耿耿于怀?’韩冈有些不快,随即他便醒悟,这是张守约在提醒……甚至不能叫提醒,而是明着在开骂了。

韩冈当时在张守约和王韶的两份荐书中挑挑拣拣,并没有多少关系。但如今他已经受过了王韶的恩惠,再投往郭逵,名声肯定要完蛋。

韩冈看得出李信心中也是这么想的,否则也不会一口气说这么多话,就是为了提醒韩冈别走错路。

“请表哥转告钤辖,韩冈多谢他提点。”

韩冈现在只恨自己对历史了解得太少了,若是知道河湟开边成功与否,如果成功又是由谁人主持,他现在就不会这么纠结了。

不像现在,韩冈只觉得他想在郭逵和王韶之间找平衡,等于是挑着千斤的担子走在只有半尺宽的独木桥上,一个不稳,便会落到桥下跟流到龙门处的黄河一样湍急汹涌的河水中。

但这副担子,至少在眼下,他还是准备挑下去的——这是他所能确认的,实现他最终目标的成功率最高的一个方案。

河湟开边,早在开国之初就吸引了无数文武英才为此画策定计。曹玮,范祥,张载,甚至向宝,皆有光复汉唐旧地之志,只是由于内敛自守的国策,始终无法施行。如今因为励精图治的新帝登基,王韶的平戎一策终于有了用武之地。有志于此的文臣武臣,便渐渐云集而来。

王韶、高遵裕、郭逵,他们哪个没有开疆拓土的念头?不过王韶有王韶的目标,高遵裕有高遵裕的目标,郭逵也有他的想法,而韩冈同样有着自己的目标和期许。大方向或许相近,但选择的道路和手段,以及最终的目的地却无一雷同。

不同于王韶写在平戎策上,为朝廷并吞河湟,收复吐蕃,剑指西夏的初衷。在河湟之事上博取到足够的军功,为日后能在官场上不断前进打下坚实的基础,这一很现实的目标,才是韩冈的追求。

他目前最大的期望,便是河湟开边能在熙宁五年之前能有个阶段性的成果——因为熙宁五年的下半年,就是癸丑科进士试的地方解试时间。如果不能在解试中,取得一个贡生的身份,便无缘参加三年后的科举。

为了能在官场中走得更远,韩冈迫切需要一个进士身份。虽然进士头衔可以由天子赐下,但由此荣幸的,几乎都是出自宰执之家,且早有文名的子弟,就连孙复、胡瑗这样名儒都没能得赐。韩冈想要混进去,其难度比起科举还要高上十倍百倍。

而熙宁六年进士科考试科目的更改已经确定,从诗赋改为经义策问,这番变动,对于在诗赋上浸淫已久的才子们是个灾难,但对于韩冈这样放弃了诗赋,而把经义背的滚瓜烂熟的读书人,却是个天大的喜讯。

在科举考试的转型期,文采飞扬的才子会因此而在科场中折戟沉沙,而对于有所准备的士人,金榜题名的机会却大大增加。

韩冈早已有所准备,他很清楚熙宁六年癸丑科的举试,是他得到进士出身的唯一机会。一旦拖到熙宁九年,当那些刻苦攻读的才子们适应了新的考题,总有事情分心的韩冈不可能与他们相争。

“还有两年。”别过了李信,骑在马上,韩冈轻声自语。

要想赶上熙宁六年的科举,和熙宁五年下半年的解试,就必须在两年中击败木征,夺取河州。一旦拿下河州,控制了洮河流域,盘踞在青唐王城中的董毡,就不得不顺服朝廷。而亲身参与其事的韩冈,只要再有一个进士头衔,他的前途将会是一片坦途。

