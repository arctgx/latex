\section{第23章 弭患销祸知何补(四)}

快到中午的时候,华阴侯赵世将带着两个伴当,一身平民的打扮,一路到了城北的惠临院。在门前翻身下马,便被一名知客僧迎进了院中。

惠临院不是京城中有名寺观,占地并不算大。正殿中供的是观音菩萨,也就没有什么大雄宝殿的牌匾挂在上面。

知客僧领着赵世将从殿前过,却在门口停了步。本就是心情不好的赵世将皱着眉,嘴角往下一拉,眼角也挑了起来。

知客僧笑道:“鄙院的观音菩萨像是从普陀迎来的,最是灵验不过。小僧看华阴侯今日似有忧色,想必是有心事。何不敬一柱香,求菩萨保佑,也能得一个心安。”

赵世将垂着嘴角盯了笑容可掬、相貌讨喜的知客僧两眼,却是不发一言的进殿去进了一炷香,丢了一串钱才出来。

跟着知客僧往后院去,赵世将冷声道:“这些天的确运气不好,若是能转运,当来还愿。”

“华阴侯是有大福气的人,本就有神佛庇佑,今天礼敬菩萨,不日当有喜信。”

知客僧一路说着好话,领着赵世将到了后院的一间禅房门前。通名后禅房房门吱呀打开,一名三十五六的中年人和穿着袈裟的白须老僧走了出来。

老僧是院中住持,知客僧见了他,便退到一边。老僧向着赵世将合十稽首:“华阴侯,小僧有礼了。”

中年人则站在台阶上朗声笑道:“三一,你可来迟了。”

“昨天接了九十七叔的帖子,今天起来却没敢耽搁,眼下还不到午时,是九十七叔来得早了。”赵世将先向着中年人行了一礼,口中却不让人。转过头又对老和尚还礼道:“守端师傅,赵世将有礼了。”

住持守端和尚请了两人进了禅房中,亲自给两人重新倒了茶,“邺国公,华阴侯,还请两位稍坐,酒饭很快就送上来。小僧不便打扰,先行告退。”说着便退出了门去。

禅房中的陈设很是朴素,桌椅上也都是横平竖直的线条,没有任何多余的花纹,只有香炉中散淡淡的檀香。

赵世将没让自己的伴当进来服侍,房中就只有他和对面坐着的邺国公赵宗汉。一口就将杯中的茶给喝光后,赵世将就自己提起茶壶,给自己又倒了一杯。

“九十七叔,今天没有别的客人了?”赵世将问着。

“就请了三一你一个。”赵宗汉笑道。

太祖一脉的字辈是德惟从世,赵世将是太祖的嫡脉玄孙。太宗这一脉则是元允宗仲,赵宗汉是太宗曾孙。两人辈份差了一倍,赵宗汉本人在他那一房同辈中的排行排在九十七,纵然赵世将年岁要长上五六岁,但他也不得不道一声九十七叔。

说亲缘,两人其实已经很疏远了,但要说熟悉程度,却是时常见面的,不过也就这半年因为赛马联赛的关系,才真正熟悉起来。

赵宗汉在蹴鞠和赛马场都有投入,而赵世将却正好是赛马总会的会首——诸多宗室之中,只有他最不在乎脸面,直接出来为赛马总会撑腰,堂堂正正的做会首。不像齐云总社,虽然每一家球队的东主都有资格在总社中做到会首、副会首,但家里养着球队的宗室贵胄,从来都是派代理人出面,没有说自己出头的——一起喝过几次酒后,交情倒也是有了三五分。

喝了两杯茶,解了口渴。住持和尚就领着几个清清秀秀的小沙弥,将一席素斋送了上来。

晚秋时节,加上京城附近有借着温泉种蔬菜的人家,还有不少蔬菜,加上一些笋干、豆腐、素鸡、素肉,倒也有七八道,对两个人来说,不算少了。

这一间惠临院,素斋做的不错,但名气不是很大,香火并不旺。只是清静也有清静的好处,换作是大酒楼,人来人往,就是特意挑了包厢,说话一样都不方便。

素色的瓷盏倒满了米酒。没经过蒸馏,也没经过窖藏,酿好了就端出来,就是口味很淡的素酒,尽管是过了筛,但还是有些浑浊。从饮食上能看得出来,这间惠临院中的僧人还算守清规,比大相国寺娶妻吃肉的花和尚们要强不少。

菜肴和酒水的口感都不错,但赵世将并不觉得今天赵宗汉请客,是为了喝酒吃菜。可是当他准备开口的时候,赵宗汉总是给他劝酒:“先喝酒吃菜。这惠临院里司厨的证慧和尚,厨艺虽不比上大相国寺和报慈寺,但也不差了。”

等到酒过三巡,赵宗汉才放下酒杯和筷子,神色也正经了一些,“三一,场面话我就不多说了。今天我在这惠临院里面摆酒,想说什么,想必你也知道了。不知三一你是怎么想的?能否直说来?”

“九十七叔既然要小侄直说,小侄自是无有不从,不过还是想先问一句,这一次的事,齐云总社是不是准备认命了?”赵世将说话直接了当,跟着说道:“若当真如此,我这赛马总社的会首也不便插手到齐云总社之中。”

“认命什么的,我从来就没想过。但这一次的事,也不是一家的事,门户之见暂时得放下一阵子。”

“这不是一家的事?”赵世将咧开嘴笑了,“九十七叔,该不会只想凭这一句,就要赛马总社为齐云总社冲锋陷阵吧?”

赵世将的话直率到了无礼的地步,赵宗汉却没有升起应有的愤怒。只要赵世将肯坐下来说话,就已经算是成功了一半,剩下的一半,就要看赵宗汉的说服力了。

“如果仅仅是要捉罪嫌,那是一点关系都没有的。但眼下御史台可是想拿着聚众为由,冲着蹴鞠联赛下刀,就不是那么简单了。”

虽说御史台为此事出动是顺理成章,可华阴侯赵世将的脸色还是一下就难看了许多,只觉得桌上素瓷器皿的反光刺眼得很。作为太祖皇帝的后人,他一向知道做什么事才能让赵光义的子孙放心,可是眼下他想做些让人放心的事,看来都难了。

“九十七叔,“赵世将沉声道:”“想必你也知道,我从这赛马联赛中得到的那点好处,要拿出多少来周济族人,若是没了这笔钱,多少人家今年的年关是没法儿过了。”

赵宗汉满意的点了点头,“多亏了是三一你,换作是别人,也不会有这么大方的。可惜御史台这一回却什么都不关心,只在乎能不能如愿以偿。”

靠山吃山,靠水吃水,没有什么好说。若是有人连这一点都不肯松口,赵世将可不会硬生生的咽下这口气。

“宗室没了体面,官家脸上也不好看。不说别的,就是下面的庄户跟邻村争个水,族长也须得出头。一族同宗,一穷一富不算出奇,但相差再大,也得维持一个最基本的体面。好歹我们这几千人也是宣祖之后啊,前两年,朝廷连问都不问一句的断了一多半的钱粮,好些宗室给夺了玉牒。眼下不止一户人家,靠了蹴鞠和赛马两项联赛撑场面。若是再把联赛给绝了,难道要我们赵家人去讨饭不成?”

赵世将说着说着,火气就噌噌的上来了。拍着桌子,砰砰砰的震得桌上的酒盏筷子乱跳,连温酒都晃了几晃,差点给倒下来。他是宗室中有名的火爆脾气,发起火来就是前任和现任的濮国公也不愿意直接面对,

“要是哪个御史敢议论赛马一句,我就去太庙哭太祖太宗去!当年忍了王安石,那是国库无钱无粮,要为君分忧。如今钱堆在仓库里,绳子都要断了;米麦存在粮囤中,连壳子都要烂了,光用钱都能把辽人都砸死了,还要夺我等穷鬼的口粮。列祖列宗在上,可是能看得过眼?”

赵世将跟炙手可热的濮王一脉来往并不多,只有眼前的赵宗汉有着共同的爱好,倒是比他人都要熟悉。这一回两人约在了不惹眼的寺院中,都是一家人,利益又相通,私下里说的话,也没有什么需要避讳的成分在,说起话来便也没有什么好顾忌的。

赵宗汉点头附和着,“每年我这个国公,就算是官家赏赐,也不过是百多两黄金、千多两白银,钱绢几百一千的,这又能济得什么事?!家里的女儿的嫁妆都置办不起啊!这一回若当真禁了联赛,难道还能指望官家将内库给我们分了不成?”

“唇亡齿寒,赛马总社这一次会配合齐云总社,九十七叔当可以放心了。”

“赛马总社愿意配合,的确是桩喜事,放心就难说了。”赵宗汉苦笑:“张商英前两次跟韩冈过不去,官家没站在他一边,没有派人治张志中的罪。他多半是打定主意有机会便去咬韩冈一口。借着韩冈的力,说不定日后真的给他做到两府中去。”

什么叫异论相搅?就是甭管地位有多高,权势有多煊赫,或是多么受天子看重,朝堂上必须有个跟他唱反调的。

只要有哪个地位还说得过去的朝臣,能长年累月的跟韩冈过不去,等到韩冈任职两府的时候,他多半也能被提拔起来,只为用来钳制韩冈。不说别人,参知政事蔡确就是从骂王安石开始受到天子的重用。既然韩冈日后晋身两府不过是时间问题,那么张商英想做个异论相搅的另一方,其实也是合情合理。

“齐云总社打算怎么做?”赵世将才不信齐云总社对这件事没有预先的谋划。

“闹事的罪魁祸首不抓出来,这一回事情就不算完。但反过来说,如果能快点结案,剩下的就是嘴皮子上打仗。谁胜谁败,得看官家站在哪一边了。”赵宗汉眼神灼灼的看着赵世将,最后一步少不了要靠在宗室中,名声甚好的赵世将来做。

赵世将眼神一凝:“也就是说,只要结案……”

赵宗汉肯定的点头:“只要结案。”

