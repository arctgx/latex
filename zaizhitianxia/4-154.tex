\section{第21章 涉川无咎黄龙锁(下)}

清晨的时候。

动荡了一夜的交趾国都终于稍稍恢复了平静,而紧闭了一夜的皇城城门也在千万人的注视下轰然打开。

交趾国王李乾德身穿白衣,双手自缚于后,交趾国的王印玉玺用白布裹着,挂在脖子上。亡国之君并不论年岁大小,十岁不到的孩童,就这么从城门中低着头走过来。在他的身后跟着交趾太后倚兰,另外还有避入宫中的一众朝臣随行。除此之外,所有的士兵都被远远地拦住,不让他们近前。

两千余名官军将士已经控制了皇城城门门前的广场,周围的房屋、建筑也都在宋军的控制之下。经过一夜的奋战,城中的抵抗已经渐渐被击溃消灭,而同时失去的生命也多达万人。

自太宗皇帝平灭北汉之后,这还是大宋官军第一次攻下一个敌国的都城。经过行营参军们一番刻意宣传,军中上下人人都感到与有荣焉。被特意挑选过的士兵们,人人精神焕发,手持斩马大刀,一个个挺胸叠肚,用扬起的下巴和鼻孔冲着弯腰驼背走过来的交趾君臣。

章惇已经换了一身戎装,腰佩长剑,英姿焕发;而韩冈也是穿着盔甲。相比起批了二十多斤的盔甲仍是不脱文人色彩的章惇,韩冈一将精心打制的山文甲套在自己的身上的。一名年轻武将的形象顿时压倒了他身上的文士色彩。倒不是他要装佯,眼下升龙府未定,出意外的危险性还是很高的。

燕达、李宪,还有一干经略招讨司和安南行营的文武官员,都在等着见证历史性的时刻。只有李信没有这个运气,他是负责领军监视城中异动,以防交趾人趁机将南征大军的将帅们连锅给端掉,没有机会参加这一次的观礼。

李乾德终于走到了受降将帅们的面前,扑通一声跪倒在地,紧随着他的一众人等,也同样是向着两位帅臣以大礼相待。

章惇、韩冈不避不让,他们是代天巡狩、讨伐不臣,小小交趾郡王的跪拜,他们都受得起。

“罪臣乾德,不幸为逆臣所欺,以至有今日之果。今天兵征讨,耀兵天南,罪臣愿举国降顺,以求天子宽宥。”

李乾德这一番话还带着奶臭味,说得虽不算结巴,但也明显全是事先备好的。而且说话只是说话,他抬起头来的眼神全是恨意,如同毒蛇一般,全无半点悔改的想法。

韩冈、章惇对此倒是毫不在意,以他们两人的权势,回到了京城之后,捏死他这个降顺的反王,跟捏死一只虫子一般容易。

李乾德一番话说完,便又五体投地的跪拜下去,整个过程恭顺无比,只要不看他的眼神,不知会有多少人被这个小孩子给骗过去。

章惇、韩冈等几名将帅并没有太过注意李乾德的表情,他们的眼睛一直都在扫视着交趾国王、太后身后的群臣,左右来回,并没有看见任何一个与传说中李常杰相貌相似的官员。

“李常杰何在?”章惇厉声问道。

一名就跟在太后、国王身后的交趾官员,听到提问,便扬声回复,“李逆昨夜自知冒犯上国,其罪难恕,已经畏罪服毒自尽了。”

“自尽?”

韩冈眼神闪动了一下,李常杰要自尽不是在家里,不是去太庙,却是巴巴跑到皇宫里面自尽?这种鬼话他是一百个不信。但不下手,解决李常杰,估计也不能这么顺利的投降。

交趾如今的境地,都是李常杰的功劳,但他在大宋这边可落不着个好。没能在李常杰活着的时候,将他拖到邕州忠勇祠前血祭英灵,实在是太便宜他了。

“尸首现在何处?”章惇立刻问道。

面覆重纱的倚兰太后向后一招手,一名小内侍连忙从后面递上了一个一尺见方的木匣,“李逆的首级就在匣中。”

揭开木匣的盖子,将里面的存货暴露出来。沾满血迹的头颅,如同恶鬼一般狰狞,只留下了一个死不瞑目的眼神,瞪大的眼角处都有道血痕流淌下来。死人相貌都会由所变化,此时乍看起来,这名只能看到脑袋的死者,让人无法确认到底是谁人的尸骸。

韩冈招了一人过来,是随军南下的何缮。他也算是行营参军的一员,曾经身为刘纪幕僚的他与李常杰有过数面之缘。冲着盒子里面仔细看了一看,点点头又摇摇头,“似乎有些相似,只是小人拿不准。”

韩冈并不满意这个答案。黄金满他们其实都见过李常杰,但他们都在外围守卫,要找过来还需要点时间。正想着是不是找他们来确认,倚兰已是俯身拜倒:“此物千真万确,下邦岂敢欺瞒上国使臣。”

章惇又看了盒子里面的首级两眼,转过来,毫不遮掩的亮在的李乾德眼前,“李常杰常年上朝,大王必然熟识,此物可当真是李常杰的首级?”

发黑的血迹残留在脸上,临死前因为剧痛都咬烂了下唇,如此恐怖的画面展示在眼前,李乾德浑身一颤,看着就想往后退。但立刻又强行忍住,点头道:“正是,此物正是李逆的首级!”

小儿魂识未全,若看了此等恐怖的画面,惊悸而死都是可能的,好歹也会重病一场。章惇为了查个究竟,下起手来可是没有半点容情。可李乾德看到人头还能这般冷静,也不算简单了,日后说不定还当真是个大患,幸好给提前拔除。

虽然还不是能完全确定,但章惇也不想再深究了。跟韩冈不同,对他来说只要抓到太后和国王,该有的封赏都会有,而活生生的倚兰和李乾德是没法冒充的。

“即是如此,就将连同尸首一起装起来,送到邕州去。”章惇让人将盒子收起,接下来就该发落这些人了。

“招讨相公,”倚兰太后忽然开口:“祸乱上国的元凶就在此处,若仍欲加罪,其罪只及吾母子之身,恳请招讨相公且息雷霆之怒,饶过满城良贱。”

章惇双眼微微眯了起来。没人会认为倚兰这话只是单纯的挂念着交趾百姓……‘难道是还想着能遗爱交趾不成?!’他略低头,瞥了一眼李乾德,隐含的威胁之意不用开口出声已经表达得很清楚了。

燕达冷冷哼了一声,也是眼神凌冽,一扫过来,便让所有降人噤若寒蝉。

韩冈则是带着笑意,上前和声道:“交州之事,太夫人勿须挂心,朝廷自有安排。此番上京,路途遥遥,还是尽早动身为是!”

他言外之意哪里还有人听不明白,也就是明说尔等已是阶下之囚,就别再多动什么心思,从今以后,交趾一地已经与李家毫无瓜葛了。

章惇冷笑着,都这时候了,还会给交趾留下一点翻身的余地。更不多话,他直接让人将倚兰、乾德以及宫人内侍安排了上了车,先送到官军已经完全控制的城东安置,择日送上京城。

转过头来,章惇问着打头的那名官员:“尔乃何人,在交趾朝中担任何值?”

“他是宗亶!”何缮叫道。

章惇、韩冈闻言,神色登时为之一变,就见那人低头行礼:“在下宗亶,拜见两位相公。”

‘都这时候了,还心存侥幸?’

不过这时候也不急着处置他,章惇指着一众交趾降臣,“全都押送东城去,回邕州后再行审问!”

交趾朝臣一个个都押走了,驻守皇城内的两千多官兵也放下刀枪走了出来,李常杰一死,没有了一个主心骨,想反抗都没那个胆子。

空荡荡的交趾皇城已经被宋军所占据,立于又一座紫宸殿中,韩冈问着章惇:“子厚兄,这座内城中的财物该如何处置?”

“该怎么做就怎么做好了。”章惇很干脆的说着,“不过内城库中的禁物全都要收拾起来,有些悖逆之物,可不能流出去到私人手上。”

有了主帅的首肯,官军对交趾王都核心区域的洗劫也开始了。有组织的清洗,比起盲动水平的抢劫,效率要高上百倍。杀人放火的事,都是军队惯做的。

交趾一国肯定是穷的,大半百姓都是连一件好一点的衣服都没有,但轮到王公贵族倒是不穷,相反还挺富裕。另外虔信浮屠的交趾人,他们建立的寺庙也是一个比一个富裕,加之许多人都逃到了寺庙中,使得寺庙比起豪门大户还要有钱许多。

只是宋人也多信佛,官军杀到庙门前,便不知该怎么做了

“逃进文庙倒是可以饶那么几个,逃进佛寺算什么?”韩冈可是半点不在乎,“不过不得在寺庙中杀生,将人赶出来再说。至于财物,佛门弟子有戒律在,不会在意身外之物。”

也有幕僚劝过章惇,但章惇反问道,“交趾虽小,亦是万乘之国。以万人破万乘,这番辛苦如果不是贪求之后的回报,又有谁会如此卖命?”

就这样,交趾王都被洗劫一空。而与此同时,章惇和韩冈则商议着该如何处理交趾国的后事。

“升龙府是不能留的。富良江江口有镇名海门。与升龙府一样,亦曾为旧唐安南都护府及行交州治所。”韩冈对此早有腹案,“若皇宋重设交州,当于此地立城。”

