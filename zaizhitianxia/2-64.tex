\section{第14章 卧薪三载终逢春(下)}

【赶到现在,实在困得受不了了,不过还是完成了承诺。第一更,求红票,收藏。】

在渭源待了两天,仔细确认了筑堡的地点,王韶又领着大队回返古渭。

虽然从渭源到古渭的一路上,看到的都是羌人。但那些羌人,一看到王韶亮出来的棋牌,便是立刻闪到路边,有的甚至跪下来叩拜,比汉人看到高官棋牌还要恭敬许多。

王中正看着王韶的威势,眼热不已:“左正言在蕃地果然积威深重。两次大捷,倒把这些蕃人的桀骜不驯给打掉了。”

王韶却是无甚喜色:“蕃人叩拜,不如汉儿一揖。这百多里地,汉人是实在太少了。要想稳定西番,必须加快屯田的速度。没有数万户口,镇不住这里的蕃人。纵使一次过砍个千百个首级,让蕃人心惊胆寒,但过个几十年,他们又会故态复萌。”

王韶说的一点不错。自古渭到渭源这条沿着渭水河谷的道路上,除了熟羊寨这个算不上战略要地的歇脚用的中继点,设有宋人的军寨外,其他地方皆是蕃人的土地。韩冈倒是想见着几个汉人,但除了身边的这些人,见到的都是把袖子脱了半边的吐蕃人。

其实真正说起来,窦舜卿说三百里渭河没有一顷宜垦荒地,其实也不算错。河谷中的这些荒地,被吐蕃人占了几百年,都可以说是他们的土地。王韶要在这些土地上屯田开垦,其实是违反了赵顼早前下过的不许夺占蕃人土地的旨意。但自古以来,古渭州就是汉人土地,真要论起土地归属,所有吐蕃人都没地方站了。

而土地的所有权问题本质上就是跟实力有关。现今吐蕃人已不复在长安城三进三出的荣光,在古渭的势力并不算强。区区一个青唐部又不敢跟官军相争,不及早占据渭水河谷移民屯田,等到吐蕃人中出个李元昊或是李继迁一般的人物,那就是第二个西夏,又或是换作了党项人过来吞了此地,那情况就更是糟糕了。

王中正也听得心有戚戚焉:“左正言所言甚是。此亦是天子所担心的。等回京之后,吾亦会向官家奏请及早在古渭招民屯田,以充实边地。”

“如此,王韶先多谢都知御前赞言之德。”王韶在马上对王中正拱手称谢。

“不敢当。”王中正摆着手,“吾此是为国而言,左正言何谈‘谢’字。”

王中正再次向王韶保证了他对河湟开边的支持,也让王韶更加坚定了将王中正请来监军,作为联系天子的助力。

一路再无他话,自清晨天色刚刚泛白之时就离开渭源,到了华灯初上时分,韩冈终于跟随着王韶回到古渭寨。

高遵裕此时就在寨中,见到王韶等人回返,便登时出门相迎,而另一人也迎了出来——却是纳芝临占部的族长张香儿。

张香儿最近精神状况好了不少,不再颓丧,迎出来的时候脸上带着真切的笑意。

一来是因为纳芝临占部的损失比当时董裕攻来时听到的要小不少。丢掉的多是财物,烧掉的也不过是座吹莽城,但人员损失并不多——纳芝临占离得古渭很近,是最后一家受到进攻的部族,早做好了逃跑的准备,看到董裕大军,几乎都翻山越岭跑了,只死了些躲避不及的。比起其余六家被董裕打得残破不堪的部族,纳芝临占部的运气,实在好得让人羡慕。

另一个原因,就是王韶准备将被董裕摧毁的其余六部的残部交给张香儿,由他一并统领。虽然六部残破,部众皆是流离失所,但对纳芝临占部来说,却是最补的一块肥肉。更重要的是,纳芝临占部一旦收拢了六部余众,朝廷划拨给七部的补偿和救济,也将全数交给张香儿。

高遵裕、张香儿,还有回到古渭寨的刘昌祚迎着王韶、王中正一阵寒暄,一起回到城衙。张香儿当即向王韶禀报:“小人前日奉机宜之命,清点六部残余。如今户口已经点算出来:总计三千一百六十六帐,八千余口,马一万余匹,牛三千余,羊两万,其余财物则剩得不多,而各家的土地都已经给青唐部占去了。”

王韶向高遵裕看去,高遵裕点了点头,他派了两名清客,跟着一起去清点人数,知道张香儿没有在其中作假。

“既然已经点算完毕,那从今天起,这三千残余就归入纳芝临占部。”王韶在城衙中,对张香儿再一次嘱咐着:“不过这三千余帐,都是你纳芝临占部的子民。本官不想看到你厚此薄彼,以至于六部余族与朝廷背心的情况出现。这一句,望你能谨记在心。”

张香儿连忙跪下,“小人不敢。小人对天发誓。但凡纳芝临占的部众,不论出身何处,就是小人的兄弟姊妹,尊长子侄,绝不敢对他们刻薄半点。”

“希望你日后行事,不忘今日所言。”王韶又说了几句,弹了弹手指,示意张香儿退了下去。

王厚冲着张香儿的背影呶呶嘴:“这人选得是不是太差了一点,”

韩冈笑道;“是差了点,但缓急间,也找不到更好的人选。”

“三千帐蕃部部众,当在一万五千到两万口上下。而六部残余的三千帐就只有八千口,几乎都是精壮。”刘昌祚接口说道。秦州西路都巡检精明强干一如往昔。

且有消息称,因为他在甘谷城的功绩,以及留下的威望。大约只能在秦州军中挤进前十的刘昌祚,即将跳过排在他前面的几位武官,接任张守约留下的位置——秦凤路兵马都监兼甘谷城主。但他现在还只是一个都巡检,兼着古渭知寨一职。

“有这八千精壮充实进部众,纳芝临占部的实力又上了一个档次。至少可以在俞龙珂和瞎药中间,做个左右摇晃的不倒翁了。俞龙珂势强,就与瞎药结盟,俞龙珂示弱,就反过来跟瞎药为敌。相信此事张香儿能做到。”

韩冈如此说着,王韶、高遵裕和刘昌祚都一个个都点着头。

无论是大宋,还是王韶本人,都不会容许青渭一带由青唐部一家独大。可官军要保持超然的姿态,对蕃部内部的纷争尽量要做到不偏不倚,这一点,是天子和王安石都耳提面命过的。所以就必须另外找一家过来。一直对朝廷恭顺有加,军令不敢稍违的纳芝临占便被挑选上了。

尽管如今青唐部接近于分裂的态势,俞龙珂和瞎药的实力相近,在他们中间便形成了一个平衡,但这种均势并不稳定,随时可能打破。为了避免俞龙珂两兄弟,在蕃部中就必须有一支可以平衡他们两人的力量。

王厚突然提议道:“必要时还可以推动青唐部分家,分成两个部族。瞎药不是想当族长吗,这下也可以如愿以偿了。两部对峙,当会为了博取朝廷支持而努力卖命,可以省掉朝廷多少事。”

“多此一举!”王韶毫不客气的批评者自己的儿子,“维持现状就可以了。俞龙珂和瞎药名义上是一家,实则已经分成了两部。俞龙珂占着名分,但有智有勇的瞎药更得青唐部人心,本已是分裂之局,由张香儿维持两部稳定,并不需要你多事。”

“可张香儿和他的纳芝临占实在让人放心不下。”王厚争辩着。无论户口、地盘、财富还是军力,纳芝临占都不占上风,而差得最远,就是张香儿。他的才智决断跟俞龙珂和瞎药比起来,实在差太远了。

“也不是全指望他。”韩冈跟王厚一样,都有些看不起张香儿,不会把希望放在他身上,“要维持青渭稳定,光靠蕃人是不够的,至少还要有汉人插一手。古渭寨中的士兵难以为持。招民屯田是唯一的解决之道。”

“韩抚勾,这样做倒是不错,但无论屯田还是市易,本金都是少不了的。不知李经略会不会批下来?”因为跟窦舜卿不合,刘昌祚几乎算是投进了王韶这一派,不过他耳目局限于边地军寨中,对秦州城内的变局却是不甚了了,却为王韶的行动担心着。

“不用理他,他什么都做不了了!”靠着托硕、古渭两次大捷而来的军功,又不再需要顾忌李师中、窦舜卿他们的掣肘,王韶说话的底气也足了许多。神采飞扬,神清气爽,宛如春天到了身边。

刘昌祚听着王韶的狂言,便有点发怔。韩冈向一头雾水的都巡检解释道:“向钤辖已经要回京修养,窦副总管则是被他的孙子连累,这两件事,相信都巡是知道的。而李经略,天子本就有将他替换的意思,他在秦州的时间应该也留不长了。”

王中正笑了一下。他前日就已经王韶和李师中之间紧张的关系和宫宴上发生的事,用急脚递传回京中。如果天子真的宠信王韶,必然会将李师中调走。

“总管、副总管、钤辖若是一下子都换了,军中怕是会有些不稳。”刘昌祚也是在官场上浸淫多年,一眼就看出了问题所在,“为了镇服军中,也许官家会派个厉害人物来秦州。”

王韶哈哈笑道:“再怎么样,总不会比李师中他们三个同气连枝时的情况更差。而且天子肯定会选个支持开边之策的知州来。”

半个月后,消息从京中传来。继向宝卸职回京,窦舜卿奉旨诣阙之后,李师中因此前阻扰开边的旧事被翻了出来,因他秦州荒田数目前后述说不一,被按了个奏报反复的罪名,责降一官,又调离秦州,至淮南东路的舒州担任知州去了。

至于新任的秦州知州、秦凤路经略安抚使兼兵马都总管的身份也传来了,其人姓郭名逵。

看着王韶突然苍白起来的脸,韩冈突然有了一点因荒谬而极度想笑的感觉,‘真的不比李师中他们三人都在秦州的时候更差吗?’

这春天可真短暂。

