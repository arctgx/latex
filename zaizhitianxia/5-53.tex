\section{第九章 拄剑握槊意未销(一)}

鄜延、河东两路联军十余万人马,在夏州、宥州、盐州一线驻扎有半个多月,每天除了派兵四散巡视周围,以防西贼偷袭,就没有别的动静。

对于眼下进退两难的境地,下面的士卒和中间的将校们都觉得这样根本没有意义,除了赌博,都没有别的事可以做了,尤其是听到环庆、泾原两路高歌猛进的消息一天接着一天的传来,更是让一些有心争一个封妻荫子的将领抱怨连天。

而两位主帅和监军也同样觉得眼下情况糟透了。

奉旨体量军事的徐禧在鄜延军之前粮草不济的时候,坐镇绥德,逼着李稷不得不尽全力保证全军的粮草供给,等到官军打到宥州后就赶了上来,打着主意就是想跟随大军去兴灵,谁能想到竟然被瀚海所阻。

李宪连着多少天都没有好脸色,只是在巡视军营的时候,才会装出一副轻松的神情来。

至于种谔,则一直都是死板着脸,仿佛别人都欠了他几千几万贯一样。他本就不是一个宽和的主帅,靠的是声望和功绩,没必要给自己伪装。

眼下的鄜延、河东两路大军的情况,粮草暂时是足够的,行军打仗对粮草的消耗最大,但一旦停下来,就会减小许多,加上后方又加强了运送的力度。尽管其中李稷下手杀了不少试图逃逸的民夫以儆效尤,但前方的军队短时间内不需要担心自己的肚子会挨饿。

可现在最大的问题不是粮草,而是前路。总不能在宥州坐上一辈子吧?种谔、李宪还有徐禧,想过瀚海都想疯了。眼睁睁的看着高遵裕和苗授杀到灵州城下,将城围起来攻打,哪一个不是如同猫儿在挠心挠肝的抓着。但瀚海的水源被破坏了,里面都是粪尿,人喝不得,马喝不得,根本过不去。夏日过瀚海本就是难,若是没有了水,那更是自杀。

随着灵州围城日久,许多人都认了命,种谔、李宪都是一日沉默过一日,就是徐禧三天两头的鼓动出兵,甚至在五天前,还跑去撺掇掌管京营的几名将领,想要先一步过瀚海,好歹占点便宜,只是没人愿意,想方设法找借口推了了事,将徐禧气得头脑发晕。

不过从两天前开始,营地中的气氛就变了一个样。

“五叔,已经确认过了,高公绰和苗授之的确败了,四天前已经有人逃回了韦州。听说是西贼掘了河渠,让高公绰功亏一篑。”

种谔的营帐中,种建中笔直的站着。虽然身子一如既往的如同劲松一般挺拔,但脸上的疲惫十分明显的表露出来。他风尘仆仆,脸上、身上都是灰蒙蒙的,就连殷红的盔缨上都是一层黄土,显然是刚刚走过了一段原路。

“大伯、七叔那边情况怎么样?”种师中急着追问。

“你大伯、七叔需要你这黄口孺子担心吗?你还在吃奶的时候,他们就上阵了!什么风浪没经历过?”

种谔的呵斥,让种师中吓得一缩脖子。可种建中、种朴,还有同在帐中的几个亲信将校都看得出来,种谔的嘴虽然很硬,可脸上的忧色却是怎么都掩饰不住,毕竟那是他的亲兄弟。

种谔心情很是浮躁。种谊就在环庆路,种诂在泾原路,这是种家多方下注的结果,也代表了西军将门种家的势力。

可由于自己的原因,无论种谊还是种诂,他们都受到了主帅的打压,一直都不能尽情的展现自己的才华。现在的情况下,不知道他们会不会被丢出来殿后。如果是在全军崩溃的时候殿后,最后会有什么样的结果,就很难说了。

“可知两路的损失多少?”种谔沉声问道。

种建中苦笑着摇摇头,“能回来这么早,肯定是跑得最快的。”

“既然都是逃回,可见高遵裕和苗授已无力控制麾下各军,很有可能已经被打散了。”种朴皱着眉,深思着说道,“情况殊为不妙!”

一名将校问道:“太尉,要不要去救援?”

种谔摇了摇头,他虽然想去救自家兄弟,但他更清楚这并不现实。

种建中冷静的道:“隔了几百里,根本来不及,一时间也不过去。”

“那该怎么办?”种师中心急的问道,“中路已经败了,西路又是王中正统领,西贼的下一个目标肯定是他。等到西路被击败,就剩我们东路的鄜延和河东两军,这仗还怎么打?”

“当然不能再打了。”种建中叹道,“……现在全军上下,还有士气吗?还不知高苗二帅送了多少好处给西贼,要是他们身穿板甲、拿着神臂弓来与我们对垒,下面的将校士卒还能有多少战意?”

种师中闻言愕然,看了自家兄长一眼后,就抬头望着种谔,“五叔!”

种谔中指敲着交椅的扶手,默然不语。种家子弟和亲信的将校都屏声静气的等着他最后的决断。

“太尉,徐宝文派人来了。”帐外亲兵打断了种谔的思路。

一名小校在外通名之后,走进了种谔的大帐。在种谔面前一抱拳:“太尉,学士有请太尉共商军事。”

种谔脸色不愉,徐禧是越来越过分了。呼来喝去的,他区区一个体量军事,当自家是宣抚使吗?

前两天刚刚得到消息时,徐禧就将种谔和李宪请了过去,说是要商量一个方略出来。当时种谔和李宪同时推脱,事情不知真伪,加上环庆、泾原两军的现状如何也没有查探清楚,怎么能遽然下决定。

种谔之后派了种建中去打听,想必徐禧和李宪都派人了去韦州。现在终于确认了败阵的消息,徐禧坐不住也是必然的。但他表现出来的态度实在是让人恼火,只不过种谔还不打算跟徐禧撕破脸皮,尚有用得到他这个热心兵事的文臣的地方。

当种谔抵达徐禧营帐的时候,李宪已经在里面了。三人匆匆见过礼,徐禧就迫不及待的开口道:“环庆、泾原两路兵败的消息,想必子正和子范【李宪字】都已经确认了吧。”

种谔和李宪交换了一个眼神,同时点了点头。

李宪叹道:“没想到会败得那么突然,听说已经将灵州城的城墙砸塌了一半。”

“高公绰和苗授之太过于疏忽大意了,明明身边就是黄河,怎么就没去想西贼走投无路之下会决堤放水。十万大军啊……唉,高苗二帅怎么就这么糊涂!”徐禧感慨不已,连声叹息。

种谔在徐禧的脸上只看到了幸灾乐祸,心中暗骂了一句,就跟着叹道:“实在是没想到,竟然是真的败了。西贼也算是有决断了,能想到决堤放水。这也不能怪高公绰和苗授之,从瀚海走了一遭后,西贼破坏水源肯定是小心提防,但水淹三军,却实在是出人意料!”

徐禧看看李宪、又看看种谔,两名主帅只在这里叹气,却硬是不顺着话题向下说,心中顿时就有些怒意上涌,但他随即收起怒气,露出一个温文尔雅的微笑:“不知子正、子范对于眼下局面,有什么想法?”

种谔和李宪又交换了一个眼色,这下就换做种谔先开口:“中路已败。秦凤、熙河两路联军组成的西路便会首当其冲,如果王都知也不幸战败,接下来我们就独木难支了,将会是被各个击破的结果。”

种谔话声一停,李宪就跟了上去,“种太尉说得正是。少了中路的联系,我们跟西路就被分隔开了,眼下西贼士气正旺,人人用命,比起之前人心涣散时要难对付得多。”

眼见种谔和李宪都在推脱,徐禧脸上青气闪过,提高了嗓门,厉声反驳道:“两位别忘了,六路出兵,任何两路都有于西贼一较高下的能力。如果按照东、中、西三路来划分,其中任何一路都不会输给西贼。就是高苗二帅之败,也是失察之故,非战之罪。”

“这是没有败阵之前的说法。”李宪摇摇头。

种谔也道:“灵州之役后,西贼声势复振,现在鄜延、河东二路没有办法在后路随时可能被断绝的情况下,守住整个银夏之地。只有先退回夏州、银州,将粮道守稳。”

“谁说没办法?”徐禧挑起了眼眉,朗声道,“官军守住银夏,西贼就只剩兴灵一地能出产粮食。官军夺了盐州,西贼就连财源也一并断了。天气暑热,只要等秋凉便可。水源被毁,一两个月后,也自然会干净下来。”

说的是很听,不是没有道理,但要能做到才行啊。占了灵州,西夏就亡了,但灵州打下来了吗?

李宪和种谔都是暗自摇头,要是能守得住,他们怎么可能会甘心撤退,放弃已经夺占的城池和土地?!

徐禧却更加兴奋,脸色涨得通红:“吾曾听闻,兴灵之地,田土肥沃,沟渠以千百计,乃是塞上江南。其地田土半麦半稻,足以支撑百万人食用。不过开战时在四月末,那时麦田还没有完全收割,而稻田更是几个月没有照料,试问这样的情况下,他们还能有什么收成?只要能等到秋天,官军的机会可就来了。”

