\section{第43章 庙堂垂衣天宇泰(17)}

何正臣和黄廉匆匆而起,正准备与其他几名御史一起上书,将天子动用宦官去彻查京西转运司的打算给顶回去。可当他们回到御史台中,却听说彻查京西转运司的诏书在知制诰那里就已经给驳回去了,只留了诏韩冈入京的那份。

何黄二人相视而笑:“徐德占果然刚正。”

“朝中也没有几个士大夫喜欢看到阉宦四处乱跑。”

成功的让天子收回了派遣中使体问京西转运司不法之事的决定。但之后何正臣、黄廉二人联合其他几名御史,上书请求天子派遣朝臣去京西彻查韩冈的时候,却没了消息。

一天过去了,宫内什么动静都没有。对要弹劾韩冈的御史们来说,没有消息就是坏消息。不能打动天子,想要找韩冈的错失,只会是自取其辱。

“肯定有人从中阻挠。”何正臣厉声道。

黄廉皱着眉头:“这几日,天子不在福宁殿,就在崇政殿。宫内不会有人帮韩冈,如果有奏章经过政事堂,总会有风声透出来,不会一点消息都没有。”

答案已经很明显了,何正臣和黄廉同时冷笑,“章惇。”

虽然别的途径不是没有,政事上的宰执也可以避开他人耳目直接上书,两制官中也有机会,但最大的可能还是与韩冈相交莫逆的枢密副使章惇。王珪、吕惠卿、元绛,有哪个会这个节骨眼上帮韩冈说话?

“也许是苏颂。”另一位御史在旁插话。

何正臣诧异的问道:“韩冈跟权知开封府的苏颂有什么关系?”

“韩冈跟苏颂一直有往来,死在邕州的苏缄更是苏颂的族叔。苏缄之子苏子元是韩冈在广西的下属,一直跟着韩冈。现在他坐着邕州知州的位置,也是韩冈推荐的,据说两家已经是姻亲了。”

“苏子元跟韩冈是姻亲?!”

黄廉和何正臣一听,当即就遣了台中两名干练的吏员去通进银台司查问。

御史不仅能风闻奏事,对朝臣的奏章都有阅览权,去通进银台司查问,完全在他们的权力范围之内。不过需要御史亲自去查实的情况不多,许多时候,派个人去问就够了。御史台里的胥吏,是京城中消息最灵通的那一部分人中的一份子。

半日之后,派去通进银台司的吏员回来了,对黄廉、何正臣等几名弹劾韩冈的御史禀报道:“苏大府今日上书,请求天子尽快下诏推行种痘法,并且还转递了万民书。京城之内,上千名士绅和百姓联名向开封府申请推广种痘免疫法。”

何正臣当即就怒道:“苏颂现在不忙着陈世儒弑母的案子,倒还有闲心管闲事!”

“这是好事啊。”黄廉笑着说道,“韩冈挂个药王弟子的身份将种痘法拿了出来,本来就受忌惮,这万民书可是往棺材上钉钉子,天子可不会待见。”

“药王弟子……”何正臣咬着牙,“韩冈奸狡的地方就在这里,明明说话行事都引人往那个地方去想,偏偏就是不肯承认他跟孙思邈的关系。”

两人都有些遗憾,要是韩冈肯说一句他是孙思邈的弟子,事情就好办了。可惜韩冈人前人后,从来都是矢口否认。

不过在这个世上,有些罪名不是看你做了什么还是说了什么,而是看你是不是能够做道。只要有那个能力,罪名就定了。

“苏颂那里好办,陈世儒那个案子能让他审得焦头烂额。只是章惇怎么办?”何正臣作难,不打掉韩冈在两府中的后台,想要动他,还是有些麻烦。

黄廉笑道:“章惇不是多干净的人,台中可就有他的把柄。”

招韩冈入京的诏书,很快就抵达了襄州。

这时候,受韩冈所托,编纂三字经的邵清、田腴,他们的努力终于有了成果。

家中的幕僚走了大半,韩冈送钱送物,倒是做到了善始善终。不过邵清、田腴都留了下来,还适时将三字经终于给修改定稿。

仅仅算是一篇字数中等的文章,韩冈匆匆一览,脸上就多了几分笑容,“多劳彦明【邵清】、诚伯【田腴】,这一篇,正是我先要的。”

三字经编写完成,韩冈欣喜不已,亲自题字作序,并制作封皮装订起来。看到上面只有自己的名字,没有韩冈的姓名,两人都惊讶不已。韩冈则笑道,“此书乃二位之功,韩冈岂能占为己有?”

对于韩冈的正直,邵清和田腴都很感动。世间编纂书籍,基本上都只留下主编的名字。主家让幕僚代为编书,也不是为了让幕僚留名。虽然韩冈在事前已经说了只想看到成品,不会夺他们的心血,但当真说到做到,还是有几分出乎意料。

李诫和方兴都回来了,神色都有些不安,他们的命运已经跟韩冈挂上钩了,不可能一走了之,也不愿一走了之。

韩冈则好言宽慰:“不用担心,一码事归一码事。你们的功劳谁也不能掩。至于那些不实之罪,都是因我而起,我怎么也不会让人栽到你们身上。”

安抚了幕僚,辞别了妻儿,韩冈随即北上京城。不数日,便回到了开封城中。在这几天中,黄廉、何正臣几次上书催促天子下诏搜查韩冈贪渎之罪,却都没有回音。

“天子还是看重韩冈,否则不会将彻查京西、熙河的事,拖延至今。”

“再看重也不可能比得上过去了。早半个月就能将建国公保下来,七名皇子只剩一个,韩冈隐匿不言的罪名有多大?天子的心结是解不开的。”黄廉冷笑道,“韩冈现在是进了京城,可还有谁能帮他。”

章惇是焦头烂额,御史台弹劾他父亲章俞和弟弟章恺侵占民田,开封府官各怀观望,畏避佥书。只能归府闭门,上书自辩。而苏缄也因为受到牵连,同时加上陈世儒弑母案而无暇他顾。

重臣之中,能帮他说话的两人都有了麻烦。

韩冈就在这个时候,进了城南驿。

“韩龙图?!”驿丞一声变了调的惊叫,让驿馆大厅中的所有人都望了过来。

“是小韩学士?”

“是韩龙图!”

“韩龙图,种痘法当真有用?!”

“小韩学士,有没有带痘苗上京?!”

不过刚刚登记了姓名,在城南驿中的官员全都涌了过来,甚至连照规矩递拜帖都等不及,簇拥在他身边,追问着种痘免疫法的详情。

韩冈甚至连梳洗更衣的时间都没有,在大厅中被人围着动都动不了。而消息很快散布出去,驿馆更是里三层外三层的被赶过来的官绅所包围。直到一个时辰之后,一名中使终于让韩冈身边清净了下来。

童贯带了赵顼的口谕来到韩冈的面前,“官家有旨,宣右司郎中、龙图阁学士、京西路转运使韩冈即刻入宫陛见。”

韩冈没有动弹。童贯一愣,忙低声催促道,“韩龙图,官家可是一听说你到了,便忙着招你进宫。”

韩冈根本就不理会童贯的催促:“御史所论,宰相亦得避位归府待罪。御史数论韩冈于京西、熙河行事,不彻查分明,哪有入宫面圣的道理?”韩冈从袖中抽出一本奏折,双手递给童贯,“这是臣之自辩,请天使代为呈送陛下。”

童贯为难了半天,看着韩冈神色中的坚持,叹了一口气,将奏章接了过来,转身离开。

韩冈也回身往驿馆内走去。在众目睽睽之下,他硬顶着天子的使臣不肯松口,恐怕很快就会传扬开了。到了他这个地位,一点也不能软,一旦松了口气,事情只会越来越糟。

要为人师表,名声是关键。坏了名声,谁来相投?不把贪赃、结党、所用非人的罪名给驳了,韩冈是绝不会入宫的。

他在心底冷笑着,既然有求于己,这帝王心术,还是收一收比较好。

“韩冈硬顶着没有入宫?”何正臣眼睛都亮了起来,这是他几天来所听到的最好的一个消息。

原本还在担心韩冈的口才能扭转乾坤的人们,这个消息让他们几乎要弹冠相庆,这是自寻死路!摆出诚恳认罪的态度,天子看在他的功劳和苦劳上,说不定在敲打一番之后给个恩典,能将这件事轻轻放过,但眼下韩冈硬得像块茅厕里的石头,事情只会越变越糟。

可第二天,何正臣呆呆的站在御史台中的公厅内,难以置信的发问:“全都驳回了?”

黄廉也是呆愣的,只知点头:“天子将所有的弹章都驳回了。”

……………………

“当然要驳回,几个皇子公主因痘疮而夭折,的确是事实。如果牛痘能早献上一个月半个月,皇第七子建国公说不定也还能保得住。眼下天子可就只剩一个儿子了。”

数日后,洛阳富府,窗外白雪皑皑,室内融融如春,香炉中青烟袅袅,与茶香、药香相合。太师致仕、韩国公富弼正与儿子富绍庭议论着京城近日种种。

弹劾韩冈的奏章堆起来差不多能有他半个人高,但天子留中的留中,驳回的驳回,完全没有责罚韩冈。甚至以襄汉漕运开通之功,加食邑四百户以作褒奖,并唐州沈括、汝州方静敏、转运司管勾公事方兴等有功官员皆有封赏,布衣李诫也得授从九品,进入官员的行列。

而坚持弹劾韩冈的何正臣、黄廉二名御史则是被贬斥出外。这个结果,让绝大多数观众跌碎了眼镜,当然,其中并不包括富弼。

“都只剩一个皇子了,在这时候,跟发明了产钳和种痘法的韩冈过不去,”富弼冷笑,“最高兴的会是谁?”

