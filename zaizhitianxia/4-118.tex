\section{第19章 萧萧马鸣乱真伪(四)}

“这不可能!”

折克柔几乎不能控制自己的双脚,颤抖的手按在交椅旁的小几上支撑着。只是厚实的松木打制的茶几立刻给带着抖了起来,几上的茶盏丁玲桄榔的碎了一地,茶水全都泼在了地上。

河东、关西都赫赫有名的折家家主,此时抖得厉害,“这不可能,契丹人怎么会为西贼火中取栗?!是准备毁了澶渊之盟?岁币不想要了!?”

“所以他们打着西夏人的旗号!”

折克行与西贼骑兵交手几十年,大战小战数百次,从没有见他畏惧过,可今天却是铁青了脸。因为很有可能在进攻丰州的同时,受到契丹人的袭击,更因为堂兄眼下的失态。

折克柔心中被契丹铁骑的马蹄声给充满,并没有感受到兄弟心头的怒意:“打着旗号又能瞒得过?”

现任的府州知州是折克柔,不过州中的政事兵事都是由他的堂弟折克行拿主意。前任的府州知州折继祖是折克行的父亲,不过折继祖的职位是继承自他的兄长折继闵,也就是折克柔的父亲。所以在折继祖过世前所上遗表中,就请求朝廷将府州知府的职位交给折克柔。只是折克柔的身体一直不好,为人庸碌,实际上府州的执掌者是折克行。如果折克柔病死了,下一任折家家主就会落回到折继祖长子折克行头上。

‘该正视现实了!’折克行瞥了堂兄一眼,已经确信必然是辽人:“唇亡齿寒,契丹人本就不可能看着大宋一步步将西夏平灭,都把女儿嫁给了秉常,难道还不能派些兵马?只要是打着西夏人的旗号,就算是被拆穿,辽人也能推个干净。”

“此事不要妄下定论!”端坐在尊位上的郭逵双眼半睁半闭,将自己浮动的心情藏在了双眼的眼帘之下,“必须要查探清楚再说。”

折克行低头一叹,还要怎么查探?!

并不是一名哨探的回报就让他深信不疑。此前派出去的几十名哨探,有四分之一陆陆续续的都将发现辽国骑兵的消息报了上来。上报的辽国骑兵人数有多有少,最少的三十余骑,多的则达到了五百骑。尽管他们都打着西夏的旗号,但穿着打扮,甚至所乘战马的品种,都与党项人有着很大的分别。

如果是关西禁军,肯定不可能认出他们的身份,毕竟从来没有接触过,多半会认为是西夏国中一个装束特别的部族罢了。但这里是河东,不但与党项人交手,同时也日夜提防着盘踞云中的契丹人的侵袭。契丹人和党项人的装束区别太大了,可以说是一见便知。

“会不会是党项人假扮的。”折克柔忽然又问道。看出了破绽一般的大声说了起来,“他们的行军的路线不对!派出去的斥候所发现的疑似辽人的骑兵部队的地方,是位于丰州外围的官道上,辽国骑兵怎么可能会为党项人看守门户?”

折克行暗自摇了摇头。契丹人当然不会为党项军看守门户,可辽国西京道绕道丰州转往府州来的小道,却也正连在那条官道,过去丰州去往辽国的回易商队,在那条官道上时常都能发现。

可这话他不好开口,私下回易可是重罪,折家的家产这些年就是靠着转口贸易来维持,但这样的挣钱手段虽然私下里人人都知道,可一旦拿到明面上,就算是以折家的地位身份来说,也是一桩大麻烦,怎么能随意将把柄送人,折克行只能选择从沉默。

“那为何他们不打着契丹的旗号?”梁从吉尖着嗓子反问。光洁的下颌,尖细的嗓音让人不会误认他的身份。不过梁从吉是领军的将领,而不是作为监军的走马承受。

这位在仁宗朝受到重用的内侍,曾经镇守在大顺城,领八百兵大败来袭的党项军。如今积功为皇城使,河东都钤辖,只是宠幸程度远不如如今炙手可热的王中正、石得一等大貂珰。

“如果打着契丹的旗号,岂不会惹怒辽人?”折克柔却是似乎一下子变得思路清晰起来,“现在只是换身装束而已,旗号还是西夏的。辽国总不能说禁止党项用契丹服,谁也不能说他们有错。”

“从装束到战马一起假扮?”梁从吉反再一次反问,“衣服、头发好说,但几百上千匹契丹马怎么来的?”

天南地北的马种放在一起,普通人分不清个一二三,但他们这些老行伍怎么可能分不出来,他们派出去的斥候又怎么会分不出来?党项人用得多是出自贺兰山下的河西马,与辽国惯用的契丹马外形差别大得很,只要对马匹稍有了解,就能区分得出。

“以西夏国力,想弄到一两千匹契丹马并不算难,直接跟上京道的阻卜部族交易就行了。前两年辽国不是才平了阻卜之乱吗?党项人惯爱玩这些鬼名堂。”折克柔提醒着,“别忘了好水川和三川口!”

堂上众将都沉默了起来。好水川之战,嵬名元昊在泥银盒子里面装了带哨子的鸽子,宋人打破盒子之后,飞上天际的哨声就是伏兵齐出的信号。三川口之战,嵬名元昊更是派遣奸细伪作延州范雍的信使,催着刘平连夜行军踏入他提前安排好的伏击圈中。

论起喜欢用计,西夏可比辽人要凶狠得,折克柔说的并没有错。说这是党项人伪装出来的可能性也并不为零。

梁从吉叹了一口气:“也就是说此事不需要管,继续按照已经定下来的方略,继续攻打丰州?”

折克柔一滞,张开口却回不了话。他可以逃避现实,不承认契丹已经选择帮助西夏,但他不能逃避到在用兵方略上冒风险。

谁敢冒险照着原定的方略继续进攻丰州?

就算是折克柔也不敢这样提议。

万一当真是契丹骑兵,而且党项和契丹配合起来,这样问题可就大了。若是就在攻打丰州城的时候,契丹骑兵突然从哪条山间小道中冲出来,正在进攻中的将士连反击的能力都没有。

“如果将辽人对西夏的支援算进来,眼下的兵力是肯定不够的。”折克行打破沉默,“必须立刻通知缘边诸关寨,雁门、瓶形、麻谷、土蹬都要通知到。火山、岢岚、宁化三军也都的派人去。另外还得派急脚递回京城,奏请天子,给河东、给府州添支兵马,等雪降之后,再行攻打丰州!”

“此事还没有确定,岂能妄报?”郭逵慢慢地开口说道,直到现在,他还是稳如泰山一般,并没有为契丹来袭的消息所动摇。

折克行质问:“难道太尉不打算将此事报回京中?!”

郭逵抓着交椅扶手的右手一下握紧,已经很久没人敢这么跟他说话了。不过郭逵的怒火很快就收了,折克行这是关心则乱。

丰州是府州的门户,如果不能尽快夺回,日后府州腹地就是党项骑兵纵横的马场。前一次已经失败了,如果这一次再失败,几年内不可能再有进攻丰州的能力。折克行根本不敢拿家族去冒险,万一错了一步,折家就是旧年控制丰州的王家阖门死难的结果。

郭逵摇了摇头,看了同样立于堂中的走马承受一眼。走马承受可以现在就将这个消息报回去,这本就是他的职守范围。但河东经略司限于地位,却不能妄报,必须要有真凭实据才行——谁来通报,性质完全不一样。

“总不能看到穿着契丹服饰、骑着契丹马的骑兵,就说成是辽国来援助西贼。”郭逵心平气和的说道,“究竟是与不是,先打了再说。”

“万一辽人当真来助战,该如何是好?”梁从吉问道。

“只要稳一点,也不怕他辽人能有什么能耐。即便发现的当真是契丹骑兵,只看眼下他们还打着西夏的旗号,可见尚不愿暴露自己的身份,对西夏的支持也是还很有限。”郭逵冷笑一声,“只要辽人不能汇聚大军,我们又何须惧他区区数百骑兵。再想想丰州现在的粮草还有多少,不可能支撑太多的兵力。只要战事的时间拖得稍长,丰州城中的各部兵马都会要自相纷争。”

折克行紧抿起了双唇,上阵打仗不想着克敌制胜,却盼着敌军自起纷争,哪有这样打仗的。只是郭逵是主帅,

郭逵知道折克行不服气,笑了一声,长身而起:“何况契丹骑兵来了又如何?铁甲、陌刀加上神臂弓,列阵而战的官军,试问契丹如何能抵挡?有飞船在天上监视,契丹骑兵又如何偷袭?丰州城小而坚,但我有霹雳砲、床子弩,试问哪一段的城墙能够防得住?!”

“官军已远胜过往!”郭逵放声直言,“区区荆南军便能以千五破十万,麟州、府州现有六万大军,岂能畏敌如虎!?”

“要一举夺回丰州,铁鹞子挡在面前,用刀劈开;步跋子挡在面前,用箭射开;就算辽军拦在我们面前,刀箭齐上,谁也别想挡着!!”

