\section{第七章 飞将庙中风波起(上)}

就在韩冈埋首于案牍,勤练于刀弓的时候,金秋九月忽忽而过。一眨眼的功夫,就已到了将军庙酬神的日子。

十月戊子,已是深秋。天上一片云也无,瓦蓝色的天空高远澄净,正是秋高气爽,草满羊肥的时候。可从北方刮来的寒流已经渐渐犀利起来,冬天的脚步也越发的近了。

韩千六同着十几个被邀来喝酒的乡邻们,一起往村西不远处的李将军庙走去。李将军庙祭祀的是西汉飞将军李广。庙后就李广的坟墓,坟前墓碑上‘汉将军李广之墓’几个大字还是当年时任秦州知州的韩琦韩相公亲笔撰写。

由于李广在史记中备受称赞,在关西一带名声也很高,尤其是他家乡的这座飞将庙,向来香火不断。不但有附近的善男信女,还有各地慕李广之名而来的骚人墨客,更有官府遣人照料,四时八节都有祭祀。李将军庙就在下龙湾村村外一里处,逢年过节,村民们也都会来此祭拜,若有个病灾,更是会到庙中,上炷香,许个愿,借李将军的神力禳解一番。

当日韩冈重病不起,已是无计可施的韩千六和韩阿李来到庙中捐了二十斤香油,又许了几个空头愿。此举虽是无稽,但却很有效验,韩冈的病自此之后很快便好了。这也是韩千六为什么要来还愿的缘故——人能欺,鬼神却欺不得。

韩冈比他的父亲先来了一步,比他更早的是韩阿李和小丫头,她们一大清早,天色才蒙蒙亮的时候,便带着大包小包的食材赶去了庙中,准备酬神后的宴席。

走在通向飞将庙的道路,韩冈步履矫健。多日的修养和锻炼让他精神焕发,身子虽仍消瘦,可当日因病而深深凹陷下去的脸颊,已一点点的红润丰满起来,走起路来也渐渐有了足下生风的感觉。

为了实现自己的梦想,韩冈每天读书笔耕不辍,这样的辛苦换来了他对儒家学术以及张载的气学理论更进一步的了解。如果持续下去,韩冈相信,最多半年,他理论研究的工作就能有个小成。

除了读书研究,韩冈每日晨起后,还有固定的射箭练习。他现在已经可以拿起挂在自己厢房墙壁上的一石三斗的硬弓,而不是继续使用软绵绵的旧猎弓。那张硬弓他天天都要拉上百十下,权当锻炼身体,渐渐的已能拉开到一多半的程度,以这个速度,到明年正月,应该就能完全恢复健康。

到了将军庙,韩冈先是去厨中看了看韩阿李和韩云娘准备得怎么样了,却马上被赶了出来——君子远庖厨,这句话就连女人都知道。闲来无事,他便在庙中游逛起来。他前生曾经来过天水,也曾进过李广庙中。从自己经历的时间上算,不过是两年前,但从外在的时间上看,却是千年的时光。

千年前后,李将军庙变了许多。楼台殿宇,树木草石,都不一样了。李广的墓身、墓碑,也自完全不同。不过最大的区别,还是殿堂四壁上游人的题字。此时不是后世,有闲暇有雅兴四处游览的泰半是士人,所以留在墙壁上的签名不是‘到此一游’的俗笔,而是一章章或是赞颂飞将之功、或是悲叹李广难封的诗篇。

可韩冈随意看了看,只觉得这些大诗人能把自家的作品公诸于众,还是很有些胆量的——无论诗还是字,就算以韩冈本人现在的水准,在里面也都是能排个中上。

“唉……”韩冈瞧着满墙的墨迹,摇了摇头。其实还不如直接写个‘某某到此一游’呢。倒是题在西壁上的那两首赞李广的‘将军夜引弓’‘不叫胡马渡阴山’,与庙额和墓碑一样,同样出自韩琦,这些字却能算是一流的书法。

自古以来,能流传千古的,多半是名篇杰作,而那些没有流传下来的劣作,实际上肯定是百倍于此。大李、老杜的诗篇留传到北宋的也不过各自千余首,但诗仙、诗圣一生所作,又岂止千数,万首也不止啊——想想后世那位脸皮老厚的十全老人,仗着皇帝的身份可是留下了十万首诗词!——以李杜的绝顶诗才,也不过十分之一的杰作,何况远逊于两位的闲杂人等。任何时代,佳作的比例就像是河里淘金,总是砂石多,真金少。

庙中正殿上点了几盏长明灯,满满地好几缸香油。为了保佑韩冈能病愈,韩家夫妇也捐了二十斤。不过谁也说不清其中有多少点了灯。韩冈只看殿内昏暗的灯光连殿上的李广神像都照不分明,再看守庙的老兵【注1】却是满面油光,肥头大耳,心知其中少说也有一半是给这只油耗子给干没了。

老兵在将军庙中值守多年,也是韩家的熟人,看到韩冈,忙上来打招呼。其实他早早就看到了韩冈在殿中闲逛,可原本韩冈长得牛高马大,提起弓来,倒像是军汉。现在瘦下来,再穿了让人举止舒缓的宽袍大袖,反而更多了点文人的逸气。韩冈形象大变让他一时没能认出,直到走得近了,方才瞧清这是韩家的老三。

“是韩家的三秀才罢?两年没见都快认不出来了。”

“啧啧,个头都赶上你爹了,长得也越发的俊俏。走到街上,不知能引来多少家的小娘子看顾。日后肯定能结下门好亲。”

“就是还有些瘦,病还没大好啊,要多养养。前日听说你生了病,俺是担心得不得了。韩大哥和阿李嫂来供香油,俺还多添了两斤油。”

“听说这些日子,三秀才你日日读书,比以往还要用功得多。再过两年,肯定能考个进士回来,也让我们这个村子沾沾文曲星的光。”

老兵噼里啪啦说了一通,韩冈连插嘴的机会都没有,还被硬扯着袖子,脱不开身。幸好庙外一片人声传来,他方得空告了个罪,逃了出庙。

韩千六带着请来的客人到了,韩冈站在门口,将他们一一迎了进来。众人寒暄了一阵,也便到了开席的时候。

将军庙的正殿不是韩家能用,便只向庙中借了偏殿。几张桌子在殿中摆开,一群人围坐着。几个大盆菜,荤菜猪羊鱼,素菜藕菘韭,再一桌配上一坛酒,这样的宴席其实跟后世也没什么差别。当然,世上还有一人或是两人一个独桌的宴会,但那等宴席可不是寒门素户能置办得起。

酒菜很快便摆满了桌子,韩千六举起酒碗,正想谢谢诸位邻里这些日子的人情。但就在此时,一人走进偏殿殿门,却是里正李癞子。

李癞子不请自到,偏殿内的气氛顿时便冷了下来。在座的都知道,李癞子与韩家并不亲近,最近因为田地的事好像还结了怨,他贸贸然跑来,总不会有好事。

韩冈心中也感觉着有些不对劲。自己重病卧床的时候,李癞子天天撺掇着家中卖田卖地,连最后仅剩一块菜田也不放过。但自从自己病好后,前日挨了韩阿李的一顿骂,这李癞子便偃旗息鼓了好一阵。现在突然蹦出来,却不像是想要重新与自家修好的样子。听说里正老爷这些日子尽往城里跑,不知与他的亲家暗地里在谋划着什么。

韩冈倒不是担心他能弄出什么妖蛾子来,关西田价低廉,普通的上等田一亩不过两三贯,差一点的就仅值几百文甚至百来文,韩家在河湾上的三亩两角的菜园由于肥力充足地势优良的缘故,在上等田也能算是顶儿尖的,韩家典卖给李癞子收了十贯半,实际价值大约是在二十贯的样子。

不过要劳动到陈举,这点钱甚至还不够让他张一张嘴,以他的势力,少说也要五六十贯才能买动他说上一句话。为了二十贯,花上五十贯,没人会这么蠢。如果李癞子只能请动他的亲家,身为士子的韩冈可不会把区区一个县衙班头放在眼里。他安安稳稳地坐着,看着李癞子能玩出什么花样来。

虽是恶客临门,但主人也要以礼相待。韩千六站起身,迎上前去:“原来是里正来了,俺忘性大,倒是忘了请你。多有得罪!多有得罪!亏得还没开席,先坐下说话。”说着便让人再搬一张凳子过来。

“不用麻烦了,俺说句话就走!”李癞子摆摆手笑道,“俺今天不请自到,一来呢,是来贺韩兄弟你家的三哥身体康健。二来呢,则是有见要是须跟韩兄弟你说一声。俺刚刚接到县里的行文,最近县中衙前不足,要各乡各村安排着人手。俺看了名单呐……”李癞子摇着头啧啧两声,“正好有韩兄弟你的名字啊!”

注1:北宋的士兵,他们的工作并不局限于打仗。尤其是厢军,更是从事各行各业的都有,唯独上阵少见,比如跑堂的,有酒店务,比如砍柴的,有樵采指挥,比如拉纤的,有广济军,比如疏浚河道,有清塘军……等等等等。而看守官方祭祀的庙宇,为官员家中打杂,也都是用的士兵。

PS:第一个高潮即将到来。各位兄弟不要吝啬手上的红票啊!

