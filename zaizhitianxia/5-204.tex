\section{第23章 弭患销祸知何补(12)}

韩冈醒了过来,打了个哈欠。睁开眼帘时,眼前依然是一片黑暗。

严素心就睡在身边,修长的身躯紧紧贴了过来,呼吸拂动着耳畔的发丝,轻轻细细,几至微不可闻。

几声发闷的咳嗽从外间传来,有两名婢女和一个年纪略大的婆子在外面值夜。估摸着时间,应该还不到三更天,但也快了,睡个回笼觉是没可能了。

昨夜韩冈特意早睡,就是为了今天的朔日朝会。朔望之时的朝会,也只比元日的大朝会低上一个等级而已。比常朝要严谨得多,规模更大,天子也不能像常朝时那般直接留个空座位给不厘实务的朝臣礼拜。有职司在身的朝官可以无视常朝,以免耽搁工作,但逢到朔日望日的朝会,只要不是抱病,没有哪人能够逃过,必须要早起。

韩冈手上有正式的差遣,而且多达三个。平常的朝会不用参加,但朔望朝参那是没办法逃的。

抬眼望着上方黑沉沉的帐帘,韩冈静静的躺着,等着到点后,外间的人会进来知会他起身。

时间过得很快,自那一日崇政殿中面圣廷对后,不过转眼之间,就已经是十一月初一,建子之月的第一天了。再过几日,便是南郊祭天的日子。

资善堂重开,皇子出阁入学,便是放在冬至大典之后,也就是十天后。

韩冈对此倒是有些担心。从出生后就养在深宫里的赵佣,第一次在世人面前露面,而且是以大宋王朝未来的继承人的身份露面,这个过程若是出了一点差错,便会造成极其恶劣的影响。

虽然说,赵佣绝不可能出现在南郊圜丘的祭天大典上——不论是他的年纪,还是他的身体,都不可能支撑得住那等漫长劳累的仪式,那可是要在斋戒数日之后,于高台上吹上半日的冬日寒风——但郊祀之后的宫宴,那是肯定要出场的。还不到五岁的皇六子,能不能在宫宴上有着过得去的表现而不出差错,真的说不准。

一切都得等到冬至才能见分晓了。阴极阳生的日子,希望能有破开朝中阴霾的力量。韩冈知道,在很久以前,冬至所在的仲冬之月,亦曾做过一年之初、万象更新的正月。

在周时,冬至所在的仲冬之月、建子之月,才是一年之首的正月。而朔日的这一天,便是一年的开始【农历的十二个月对应十二地支。仲冬之月为子月,北斗斗柄指向正北,冬至在此月中,为如今通行的夏历的十一月】。要是天下还是用的周历,那么韩冈昨天晚上就该给儿女们发压岁钱,而现在应该全家人都在守夜呢。

只是几千年来,天下通行的历法尽管万变不离其宗,总是在黄帝、颛臾、夏、商、周、鲁等古六历中轮转,但自汉武帝太初元年改颛臾历为夏历,以至于这一年有月亮有十五次阴晴圆缺之后,夏历系统始终是历法上的主流。

以孟冬亥月【夏历的十月】为岁首的颛臾历,仲冬子月【十一月】为正月的周历、鲁历、黄帝历,季冬丑月【十二月】为正月的殷历,只能偶尔得见——王莽行殷历十五年,魏明帝用殷历三年,则天皇帝改周历十一年,唐肃宗变夏为周更是只持续了半年——基本上早就被丢进了故纸堆,如今通行的历法,源自夏历,是以孟春的建寅之月为正月。

历法,是最近几日苏颂和韩冈谈论的比较多的话题。

由于从太宗开始就对私人研究天文采取比前朝更为严格的禁令,大宋的天文学水平下降得厉害,如今的历法在节气和日食月食上始终没有算准过。《钦天历》、《应天历》、《乾元历》、《仪天历》、《崇天历》、《明天历》、《奉元历》,不过一百多年的时间,为了弥补不断出现的错讹,历法就改变了六七次之多。

前几年沈括曾经接手过司天监,但他在这个几乎已经成为几个家族世代盘踞的衙门中,根本无法施展自己的才华。加之当时又是兼齤职,最后费尽了气力才有了一个《奉元历》,但这个《奉元历》依然不算准。月食、日食和五星占候上,总是有些差错。

可能是天子对这个情况有些厌烦了,前几天,让有这方面特长的苏颂兼了主管天文历法的司天监的差事。

变得更加忙碌的苏颂,到了本草纲目的编修局中,也拉着韩冈讨论历学。弄得韩冈现在满脑子的都是建子、建寅,月犯五纬,太白昼现什么的,变得一团浆糊。更别提元法、岁盈、月率、会日、弦策、望策、损益率等专有名词,不回去翻书,根本就弄不懂。

不要在自己不擅长的领域与专家交流,这是韩冈长久以来的坚持,也是最聪明的做法。不过韩冈的为人不喜逃避,反而喜欢以攻代守,所以他反过来拉着苏颂谈了一通恒星、行星和卫星的区别,以及日食、月食的成因,甚至还有万有引力,好歹没有露了底。

在过去,韩冈也不是没有跟苏颂讨论过天文星象,也曾稍稍透露了一点自己的观点,至少大地是球状的理论早就跟苏颂讨论过了。只是系统化的描述,这还是第一次。

日、月和五大行星运行的规律,是天文历法的基石。建筑在日月运行的观察上才得以编订的历法,正确的寻找出其中的规律,当然是重中之重。相对而言,那些名词反倒是枝节了。

韩冈不知道苏颂信了几分,只是苏颂在听了他的话后,神情很是严肃,看模样并没有将他的观点当成是胡言乱语。话毕竟是要看人说的,韩冈说出来的话,分量自然是不一样。

不过天文星象上的事,并不是韩冈目前关注的重点。

王安石就要抵京了,以他的才智,不可能看不出发掘殷墟这个行动对气学的意义。不论谁来主持,都是格物致知的体现。

虽说在发掘的过程中,占据了甲骨文的诠释权,能稍稍弥补一下因为《字说》而造成的失分,但被韩冈推入被动的局面却是没有改变。

王安石的性格有多倔,韩冈可是有着切身体会。拗相公的外号也不是白叫的,他愿意接受朝廷的任命,自然是为了给新学张目。说起来,等于是受了韩冈的逼迫,这口怨气相信不会缺少。等见面时,估计还有得头疼。

但王安石能上京,王旖是最高兴的,而且王旁也应该跟在他身边。管了几年的江宁粮料院,估计王旁也是够憋屈的,能卸下这个差事,兴奋的心情不会比他的妹妹稍差。

希望自家的内兄和浑家能帮着说合,韩冈可不想跟王安石吵起来。以经史为基础的辩论,对手还是王安石,韩冈可是一分一厘的自信都没有。

反正是肯定要头疼了,韩冈想着,不知不觉间却又昏昏然然的睡了过去。

“官人!官人!快要到点了。”

似乎从极远处的地方,传来一声声发急的呼唤,韩冈能感觉得到身子也被人用力推着,将思绪从一片混沌中拉扯了出来,韩冈一惊而醒,这才发现自己想事情的时候竟然睡着了。

眨了眨眼睛,打了个哈欠,韩冈终于算是清醒了一点。他坐起身,对比外面值夜的婢女、婆子还要小心的严素心笑道:“做了官就是这点不好,小时候好歹都是天亮了才起来。”

严素心有些疑惑的歪着头,“记得爹娘常说,官人读书的时候,都是听到鸡叫就起来了。”

韩冈又打了个哈欠:“……那是爹、娘偏袒,为夫小时候做的事,在爹娘眼里全都是好的。”

“所以官人才让钲哥、钟哥他们不要起得太早?”

“小孩子嘛,长身体的时候,多睡一点是正常的。等年纪大了,想睡都睡不着。”韩冈笑道。

韩冈拥有两份记忆,对于幼年时的前尘往事,已然模糊混淆在一起。说话前不想一下,就分不清究竟是出自哪一人?不过只要没有大的关碍,韩冈也不会刻意去分辨,本来就已经没办法区分开了。

已经是三更天,韩冈没有时间多耽搁,起床洗漱,匆匆吃了早餐,便在一众元随的护卫下,赶往皇城南端的宣德门。

宽阔的御街有两百步宽,大小如同一个广场。御道两侧的千步廊中,几乎都是早饭的摊子。即便是官员之中,能像韩家这样厨房不熄火的人家,终究还是少数,大部分的朝官早起上朝,有许多都是在御街两侧的千步回廊中解决早饭问题。

当然,这些摊点的服务对象,更多的还是官员们的随从。若说上早朝的官员们只有一部分能在家吃早饭,那么几乎所有的随从,就是韩家,也不过是出门前拿两个热馒头而已。待到官员们汇入宣德门,才是客满为患的时候。

在宣德门前的广场上下了马,随行的元随们便将马匹牵走,退到了广场外。韩冈自门中穿过,向内侧的文德门走过去。一路上过来,身前身后都是衣着朱紫青绿的朝官,其中主动跟韩冈打招呼为数不少。

文官们有文官们的圈子,武将们有武将们的圈子,那些身上背着节度使、观察使、金吾卫上将军之类官职的宗室、国戚们则是另一个小圈子。互相之间有些泾渭分明的感觉。但偏偏向韩冈示好的官员来历却没有这样的区分。

幸而人群没有因为向韩冈行礼而让队形稍乱。都是有经验的官员,早就知道如何应对。认识的行礼打个招呼,聚在一起也是压低声线,没有一个大声谈笑的。

司掌朝堂风纪的御史们正在一侧虎视眈眈,若有人在宫中失仪,等朝会后就会报上去,到时候少不了一个罚俸的处分,背上这个处分,钱财还是小事,重要的可就是磨勘的时间要增加,又多了几年的闲空。

韩冈能感受得到御史们的视线,正像刀子一般在自己身上划来划去,只是他浑不在意。在前面看到了两位皇弟,并肩站在一起说话。不过片刻王珪也到了,赵颢和赵頵给王珪让出道来,天子以下以宰相位份最尊。群臣避道,亲王也不能例外。

除此之外,韩冈还看到了李清臣。他是新上任的判太常礼院。

据说李清臣此次从河北调回,本来不是这个任命,而是准备让他去做三司使的。而太常礼院事前也没有听说,可是突然之间,政事堂便改了任命,并得到了天子的许可。确切点说,应该是反过来,太常礼院的任命,是天子的独断,政事堂只是在诏书上副署而已。李清臣原本就做过太常礼院的同知,这一回正巧太常礼院人事大变动,便让他接手下来。

但问题不仅是前任知太常礼院被发遣出外,院中的有三名礼官,也一并被请出了京城。事前一点消息都没有传出来,便让太常礼院。这在朝堂上,也是很少见的。

不过一干消息灵通的朝臣,还是知道太常礼院到底是哪里犯了天子的忌讳。‘狄戎是膺,荆舒是惩’,太常礼院玩得小花样,也许王安石不在乎,也许有许多人根本就没有联想,但既然已经遍传京城,就必须给王安石一个交代。

王安石已经确认进京,最多还有几天就该到了。天子命王安石去发掘殷墟,在许多人眼里,说不定他还有机会重返相位——毕竟王安石才刚过花甲,还有足够多的时间执掌政事堂。
