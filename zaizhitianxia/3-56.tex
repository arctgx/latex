\section{第21章 论学琼林上(中)}

所谓琼林宴,就是在琼林苑这座皇家园林里,为新科进士举行的贺宴,是唐时曲江宴在宋时的翻版。

琼林宴由天子亲赐,作为宾客参加的只有新科进士才够资格,而陪席的,是以翰林学士、龙图阁直学士为首的学士和馆阁官。能参加琼林宴,对于天下数以百万、意图一跃龙门的士子来说,是无上的荣耀,也是他们寒窗苦读的动力。

更别提在赴宴之前,进士们还要戴着御赐的金花,骑着马从天街上招摇而过,沐浴在东京城近百万军民羡慕赞叹的目光中,一直抵达最后的光荣之地。

排在三百八十四位的慕容武,韩冈从今天甫一见面,就见着他一直在笑。多少次他想摆出庄重的样子,竭力掩饰自己的兴奋,但无论怎么努力,慕容武也抿不住翘起的嘴角。

不仅仅是慕容武,只要韩冈双目所及,在东华门外的天街之上,几乎每一个新科进士都是兴奋难耐的表情。韩冈却是无法融入进去,只是当作一项必须完成的任务而已。抱着这份心情,韩冈就像混入白羊群中的一头黑羊,自己都觉得扎眼。

尽管投生于此已有三载,想要融入这个时代的民风人情,依然不是那么简单的事。对于韩冈来说,考中进士,不过是多了个有用的头衔,一份证书,让他日后加官进爵不会再有阻碍。兴奋——当然有,但只在殿试之后,并没有持续到现在。

而在这个时代的士人眼中,考中进士却是一生的荣耀,几十年后都可以拿出来当话题,还可以放进族谱中,让千百年后的子孙,知道有个考中进士的老祖宗。

如慕容武和其他四百零六位进士这样的兴奋,韩冈能了解,能理解,却不能感同身受。

‘毕竟还是外来户。’

韩冈想着,与所有进士们一起,在东华门下,为了天子恩赐的衣靴笏而拜礼。

天子向进士们赐了绿袍、官靴和笏板,这也就是所谓的释褐。褐是平民的衣服,脱去平民的衣服后,代表进士已经拥有了官身。自此之后,从民而官。家中的户等也被单独制册,列入官籍之中,不再属于民籍。

众进士一起换上官袍之后,一开始就穿着公服的韩冈就不再显得那么的显眼。

九品以上是青袍,也就是蓝色的官服;七品以上是绿袍。四品五品为朱色,三品以上,那就是紫色。不过赐五品服,赐三品服的很多见,毕竟官品难升,宰执或是地方的守臣中常常有品级不高的,所以为了朝廷体面,都是特赐的朱紫袍服。

而新科进士,尽管封官依然是从九品,但都能穿上七品服,这是朝廷对他们的奖誉,也是要让进士的尊贵由此而体现出来。

韩冈辛辛苦苦三年多,立了多少功勋,才得了一件绿袍。而普通的士子,只需用三场考试,就在服色上追平了他。进士之贵重,便由此可见。

一齐发下来的,不仅仅是衣靴笏,还有用金丝、红绿二色彩绢扎成的金花一支,用来插在帽子上。

这朵金花是宫廷名匠所制,做工的确精致无比,金丝缠成的花蕊清晰可辨,轻轻垂上一口气,就在风中颤动。韩冈看了一阵,便满不在意的往鬓角处插了上去。不像当年的司马光,还要为此纠结一番。

一部宫廷鼓吹从右掖门中出来,而一群马夫也牵着马一起来到东华门前的广场上。

鼓乐齐鸣,四百进士一齐上马,向南从宣德门离开皇城。拉出来的马匹都是从禁军中特别挑出来,本就是温顺无比,加上前面有马夫牵着。就算是从来没有骑过马的南方进士,也不用担心在游街的过程中有何意外发生。

皇城之外,此时已经是人山人海。东京城今日万人空巷,男女老少离家而出,都是为了来看一看新科进士,沾一沾文曲星的文翰之气。

上四军派出的禁军作为先导和护卫,一队仪仗跟进,一班鼓吹紧随其后。再后面,便是以状元、榜眼打头的四百零八位进士。进士之后,还有数百名骑兵跟随。

为了让新科进士享受一下这一荣耀的时刻,上千人的队列一路走得很慢。道路两边都拥挤人群,当进士们的队列经过的地方,那一段道路上就响起一阵喝彩声。而无数仕女,更是挥着手绢,希望进士能望去一眼。

骑在马上,被陌生的人群欢呼叫好。韩冈只觉得他们狂热的程度,堪比后世的追星族。身旁是第八名的留光宇兴奋的脸色涨红,仰着头,在马背上将腰挺得笔直。而紧跟在后面的叶涛,韩冈回头望过去时,也是一般无二的神情。看来真的只有他一人,无法融入这样的气氛。

从宣德门往琼林苑去,先是沿着御街南下,然后在州桥之前,转向西行。一路经过内城的郑门,外城的新郑门,然后抵达城外的金明池,而琼林苑就在金明池畔。

总计七八里路,走了近一个时辰,速度之慢,可想而知。

在琼林苑门前下马,护卫的,穿过敞开的琼林苑大门,走进了这座皇家园林,有别于宫中殿阁给人感觉的端正厚重,而是多了许多秀气。

亭台楼阁,错落而置,环绕在树木、花卉之中,湖水在其近侧。假山、花木等位置的安排,显得匠心独运。每一座建筑顶上,所铺设的墨绿色琉璃瓦,给园林更增添了一股古拙文雅的味道。远远望过去,比起韩冈前世见识过的江南园林规模上要远远超出,而尊贵之气,更是私家园林无法相比。

宫廷宴席的仪式有其定规,不能有丝毫错误。在琼林苑的主殿中,摆下了五六百席,皆是单人的席位。由知贡举的曾布压宴,一众学士、馆阁,在上首陪席。

状元余中领着一众新科进士,按照事先通知过的礼节,行礼、入席、奉酒、谢恩,一步步,都是按着压宴的曾布指派。直到奉酒三巡之后,方才算是结束了这一整套仪式,各自放松了下来,也允许了在席间走动。

小桌上的菜肴,韩冈只动了动筷子,宫廷置办的宴席也并不算出色,而且这样的大宴会,古往今来都是一样,并不是用来吃饭,而是以互相交流为主。

很快,余中先来找韩冈。

这些日子,状元郎忙得脚不沾地。在新科进士要举行的一系列仪式中,状元位份最尊,凡事都要有余中和两名榜眼领头。而具体到一系列的活动,前三名更是要作为主事者,一力承担。

来到韩冈身边,先与韩冈互敬了一杯酒。然后道:“过两日期集,须去国子监拜黄甲、叙同年。还要请韩兄一并做一下《同年小录》,以为日后亲近之用。”

韩冈笑着一拱手,弯了弯腰:“韩冈谨受命”

韩冈很好说话,半开玩笑的回复,让余中也哈哈笑了两声。

礼部试之前在清风楼上的初次碰面,韩冈很是给余中面子。使得余中反过来,也变得愿意亲近韩冈。韩冈的前途光明,且正得圣眷,聪明人都不会与他为敌。何况余中这个好名好利,本身就热衷于仕途的。

前段时间,余中还因为自己中状元而兄长被黜落,而向天子上书,要用自己的状元换兄长一个进士出身。这种做法就是典型的沽名钓誉。

在官场上,的确是有用自己的官阶、功绩,来为自己的亲友赎罪或请官的,但并不多见,而且大部分还不能成功。至于用自己的状元换亲友登科的,却是古往今来第一遭。以余中的心性和才智,应当知道他的这个要求绝不可能实现。真的要换,在礼部试后,就可以上书的。不过余中的名声因此大噪,连天子都觉得他在孝悌做得甚好,赐了余中兄长一个没品级的助教职位。

余中很会做人,又跟韩冈说了几句,便转去找其他几个同在一甲的几个进士。《同年小录》,也就是今科进士的个人资料档案,就跟同学录一般,不过比同学录更繁复。

考生个人的姓名字号、排行生日、籍贯家世,三代名讳,母亲、妻子的姓氏乡贯,乃至考了几次科举,等详详细细的资料都要录入进来。另外还要加上天子颁发的举行科举的诏书,省试、殿试的考官姓名,两次考试的题目,林林总总,不一而足。最后还要请官中的印书局来雕版造册,新科进士人手一份,并呈递天子、中书、还有诸多考官。这当然不可能靠三五个人就完成,需要每一个进士都动手。

余中离开了,韩冈便又变成了独自一人。慕容武在后面被人绊住,叶涛人在对面。而其他进士都与韩冈不熟,不是没有想结交他的,却一时不敢过来。

韩冈喝了一杯酒,正准备站起来,主动过去打招呼。同年可都是日后的人脉,没必要崖岸自高。

这时却走过来一名身着红袍的贵官,远远的叫了两个字,“韩冈。”

