\section{第47章 不知惶惶何所诱(上)}

【第三更,求红票,收藏】

时间过得很快,转眼间已经到了二月中旬。

天气还是有些轻寒,但汴河两岸的垂柳枝条已经有了融融绿意,而站在汴河边,也能看到河面上的冰层一天天的消失无踪。街巷上的行人因为天气转好的缘故,多了不少。

不过街巷上的气氛稍显紧绷,本来前些日子还有些对自己充满自信的士子,在街上游逛。但再过三天就是科举的礼部试,从七八天前起,街上和酒店里的读书人,倒真是一个也见不到了。

而韩冈这边,自前日在王安石府上慷慨陈辞之后,他就没有再去见过王安石。当日所言的几条计策,王安石究竟用还是不用,也不是他所能左右的。韩冈明白,王安石他们不是自家手上的傀儡,自己怎么说他们就会怎么做,他们有自己的判断和选择。

但韩冈更清楚,他的一番话已经在王安石等人的心底埋下了种子,等到合适的时候就会生根发芽。不管怎么说,就是看着老鼠一个劲的在面前蹦达,即使没有任何危害,也已经够恶心人了。何况领衔旧党的诸多元老重臣,还有身为赤帜的司马光,他们不是老鼠,是老虎!

韩冈的一番言论就是恶魔的劝诱,开花结果不一定是现在,但总有茁壮成长的一天。

以韩冈对章俞的救命之恩为名,章惇则来过两次。但两次会面,章惇绝口不提有关变法之事,韩冈也当什么都不知道,也是一点也不提。而刘仲武,于章俞同样有救命之恩,韩冈看章惇的样子,对他很是看重,看起来即便在向宝面前失了宠,刘仲武还能在章惇幕中混出头来。

在等待告身发下的这段时间里,韩冈一众逛过了类似于后世娱乐中心的桑家瓦子,在里面听了说三分,诸多杂剧,还看了一场光着上身只穿兜裆布的女相扑。

桑家瓦子是娱乐场所,而大相国寺则是小商品市场。趁着每月五次大相国寺开放,所谓万姓烧香的日子,韩冈进寺内入乡随俗的烧了几柱香,但主要还是参观游玩的用意居多。

万姓烧香只是个名义,实际上大相国寺开放的目的却是集市。尤其是从大门到主殿,有卖花鸟虫兽的,也有卖家用摆设的,东京城里诸多尼庵道观,也在相国寺中有着固定的铺位。那些尼姑道姑日常无事时做的女红,都在摊子上摆着发卖。

与一到相国寺,就双眼发光的路明和刘仲武不同,韩冈对逛街店的兴趣不大,两次都是走马观花的转了一圈——第一次来时就买了点带回秦州的礼物——便往后殿走。

不得不说韩冈过去对大相国寺有很大误会。这座皇家丛林名义上是一座寺,但其实是几十个僧院组成。而且里面的和尚不是一个宗派,有律宗,也有禅宗。

律宗的弟子端正严肃的双手合十,低头念着经文,而两个禅宗的和尚在旁边晒着太阳打打机锋,这样的情况很常见。但不论是哪个宗派,香火钱都是要收的。

两次到大相国寺,韩冈都在寺内转来转去,香火钱给得不少。这不是他虔信浮屠,而是想找几个有点水平的和尚去秦州。无论是党项还是吐蕃,每一个蕃部几乎都是虔诚的佛教徒——惯做的杀人放火,并不会影响他们对浮屠的崇拜。

所以韩冈当日给王韶出的主意中,便有一条就是向河湟蕃部派出。可韩冈现在发现他想得太简单,走马观花一样的闲逛,要是能撞到一个有心一建功业的和尚那就有鬼了。而且东京城如此繁华,那些贼秃又怎么会放弃花天酒地的夜生活?!

此时和尚娶妻的情况不少,‘没头发【和谐】浪子,有房室如来。’这是如今对娶妻生子的僧人的戏称。当韩冈看到一个光溜溜的秃脑袋旁边,傍着一位千娇百媚的美人,他便放弃了搜寻,这个问题让王韶头疼去好了。

这一天,韩冈久等不来的告身终于发到了手上。

官诰院的官厅中,一名黑黑瘦瘦的苍老文官,展开画轴一样的告身,正用着一股子怪异的广南口音,念着上面的文字。

韩冈对此很是遗憾,本以为今天能见到正担任监官诰院一职的苏轼,却没想到只是一个吐字不清,腔调怪异的广南佬出来。

韩冈在下面垂手肃立,努力想听明白他到底是在说些什么,但这个黎或是李判院见鬼的广南腔调,让韩冈听得一头雾水。只听清了自己的名字,并确认了他的告身不是由四六体骈文所写——当然他也不够资格。只有侍从官以上的告身,才会四六骈骊,写得文采飞扬。如韩冈这等青袍小臣,他的官诰能由骈文写就,只会是遇上官诰院的官员和书办想练练笔的时候。

正常的京朝官和选人之间有着天壤之别,礼节的问题忽视掉也无所谓。今天显然心情不好的官诰院判院,并不想跟韩冈说什么恭喜之类的套话,他将韩冈的告身装回到锦囊中,递给一边的令史,反身就走了进内厅去。

令史和令丞差一个字,但一个只是小吏,而另一个则是官人。判院能拿大,而尚书省中的积年老吏,敢于欺蒙上官,却不会无缘无故得罪人。

他笑眯眯的走到韩冈面前,弯腰低头,双手将告身锦囊奉上。

韩冈一笑,接过锦囊。回头使了个眼色,站在院中等候多时的李小六,心领神会的走上前,捧上了一贯铜钱。这是新官得铨后,惯例给人的赏赐。

这钱令史收得心安理得,韩冈交得理所当然。而除此之外,韩冈在拿到告身前,还向官诰院缴纳了三足贯的大钱。这叫绫纸钱,也可以说是工本费,不交的话,官诰就拿不到手。前两天,韩冈让李小六吃力的将三千枚小平钱挎在身上的时候,不禁想着,官僚机构果然都是一个德性

令史恭喜了韩冈两句,拎着钱串子送了韩冈出门,便走了回去。韩冈拿着价值三贯的锦袋,盯着缎面上的云纹看了半天,突然右手用力,五指一收,里面撑起官诰绫纸的两根纤细木轴,就在他的掌中弯曲变形。

“官人?!”李小六在韩冈身后惊道。

韩冈慢慢的松开手,告身所用木轴的质地应该很不错,一下就恢复了平直。

韩冈掂了两下,轻飘飘的。为了这个像画轴一样的东西,他费了多少辛苦,因他而死的冤魂也不知多少了,因为他,很快朝堂上又要卷起轩然大波,辛苦到最后,也不过换来了这个东西……而且拿到手上前,一个猥琐不堪的小吏露着一口破烂的黄牙,跟他比了三根手指:“三贯。”

虽然只是工本费,但韩冈还是觉得心里怪怪的。

把锦囊收进怀里,韩冈领着李小六离开官诰院衙门。就在官诰院大门外,路明满面笑容的迎了上来,“昨日刘官人得官,今日韩官人得官。果然是烛花连爆,可喜可贺。”

韩冈笑着,方才复杂的心情好似已消失无踪:“折腾了这么久,终于能拿到手,也算不枉我的一番辛苦。”

“官人得官之艰,这世上少有人能比。”路明深有体会的点头附和,完全没有一点羡慕嫉妒之意。

韩冈得官之辛苦,路明已是一清二楚。他这些天来,一点一滴从李小六、刘仲武还有韩冈本人这边,打听到了许多支离破碎的信息,如同拼凑一幅散碎的拼图,路明拼出了韩冈从布衣一直到今天走出官诰院的艰难道路。

路明有时在想,如果是自己处在韩冈的位置上,怕是骨头都能拿来敲鼓了。

时已近午,韩冈三人在路边找了家脚店,找了个僻静的角落,点了几个酒菜,韩冈便把告身从怀里取了出来。

打开锦囊,抽出告身,是个木轴长度只有不到一尺的小卷轴。

据韩冈所知,宰执官的告身都是金花五色绫纸所制,而且是十六七层绫纸裱糊起来,犀角为轴,彩丝系带,由紫丝网罩着,连装告身的袋子也是用最上等的云锦缝起。

而他手上的这个从九品的告身则是最普通的五六张白绫小纸叠合,用的是木轴青带,袋子也是普通的锦缎。

路明和李小六伸着脖子盯着韩冈手上的这个卷轴,不管形制再简陋,但这毕竟是官员的凭证,多少人一辈子都弄不到到手。

“官人,快打开看看。”李小六催促着。

韩冈嗯了一声,满不在意,他的差遣早定,经略司勾当公事兼理路中伤病事宜,判司簿尉的本官究竟定得

如何,其实并不重要,只是关系到俸禄多寡而已。

解开卷轴上的系带,韩冈将之展开。几行端正的楷书占去了告身卷轴中心的位置。

“密县县尉?”托前世走南闯北的福,韩冈地理的水平很高,很快便将自己的本官与记忆联系起来,‘是新密市吧?’

密县县尉就是他的本官了,不过韩冈不用去密县应差。这个时候,在密县必然另外有个县尉,管着县中兵事和捕盗,那是他的差遣。而韩冈的密县县尉只是发工资的凭证,他的工作在秦州。

说起来差遣和官职分离的这个见鬼的官制是在很好笑,不过这也是从晚唐五代流传下来的后遗症,不是轻易可以改动。

只是韩冈又纳闷起来,能在后世留下名号的地区,怎么是下县?

判、司,是州中官职,簿、尉,是县中职司。因为天下四百军州,两千余县,人口、税收、地理、历史各不相同,所以这些州县就被分个‘赤畿望紧上中下’等七个级别出来。由此而来,同样是从九品的判司簿尉,其实里面也分了个三六九等。

新入官的进士,他的本官会是望州的判、司,或是次畿县的簿、尉,而九经则下一等,为紧州判、司,望县簿、尉。再往下,是五经、三礼诸科。而韩冈这样布衣入官,则是倒数第二档,下县主簿县尉而已,只比花钱买官的进纳官高上一点。

