\section{第六章 见说崇山放四凶(六)}

“既然如此,那各位卿家就先回去吧,以安人心。”

向太后没多考虑就同意了韩冈的提议。

“陛下。”韩绛上前奏禀,“变乱余波未息,今夜两府当宿直宫掖,以防万一。”

屏风后面的声音冷了下来:“昨夜宿卫的是那两个逆贼,今天该谁了?!”

东西两班的宰辅们面面相觑。

太后这个心结留得够重的,听口气就不对。

但宰辅们要宿卫宫掖,在宫中生变的夜晚,谁也不可能当作无事一般打道回府。万一出了乱子,他们也能在第一时间来处理。

“……以臣之见。”章敦说道,“两府还是全都留在宫中为是。”

“那就诸位卿家就都留下来好了,其余卿家,都回去吧。”

向太后说着就起身,只听得屏风后一阵环佩急响,群臣连忙恭送太后退朝,待他们抬起头时,太后一行已经消失在通向后殿的小门中。

大臣们相互交换着眼色,太后走得如此之急,甚至没有多留下一句。

要知道,虽然残存的两府宰执都留下来了,但‘其余卿家’中,还有两位不是宰辅,地位却能平起平坐,功劳也更高的人。

太后没提到留下韩冈,王安石也没有被留下来。

韩冈今日立下如此殊勋,最后却被太后给忘了。

不论是真的忘了,还是故意没理会,从这件事中来看,她对韩冈的信任还剩下多少?

数十道视线窥探着王安石和韩冈,猜测他们两位会不会主动留下。

“……都回去吧。”王安石停了一下说道,他面向韩绛:“子华,今天宫里面就拜托你了。”

韩绛点点头,“介甫放心。”

王安石转又对其他人道:“翰林学士照常宿直玉堂,其余都不要在宫里面留了。”

下面的一应重臣,除了翰林学士要在玉堂中轮值,其余人等,都不需要、同时也没资格留宿于宫禁之中。不过韩冈虽站在他们之中,但明显的不属于他们的行列,王安石自己主动出宫之余,还一并要将自家的女婿也带出宫去。

王安石如此做,韩冈脸色如常。一言不发。随着同列,一起退出了崇政殿。

吕嘉问与身边的同僚交换了几个眼色,又望着韩冈在前的背影。

十几只眯起的眼睛中,都在疑惑不解之余,也有几分幸灾乐祸的神色。

王安石和韩冈现在都不能随意入朝,只有重新回到朝堂上就任实职,才有那个资格。所以不论是明天、后天,只要太后不提起他们,他们都不能主动入宫。只要他们两位的实职差遣一日不定下来,一日就不得随意入宫,必须等待着太后传唤。

但看方才太后的言行,她的想法就可以明了了

今日的宫变,全都是天子致祸。

没有韩冈坚持要保住小皇帝,蔡确、石得一、宋用臣叛不了,也不敢叛——根本就没有理由。

太后不可能想不明白,先致祸,再解除,在这之间,韩冈他有什么功劳可以称道?

而且韩冈在明面上立功太高,其余宰辅看起来已经联手王安石要压制他了……说反了,是王安石主动联手其余宰辅,要将自家女婿给压下去。

从这边看来,韩冈最后就算能回到两府,也难以施展手脚。

不过幸灾乐祸的时间并不长,吕嘉问心中的念头又转到了上面的几个空下来的位置上。

不知太后什么时候会招内翰,御内东头小殿,拜除宰执?

吕嘉问心中火烧火燎。

功劳虽不及宰辅,但忠心可不会输给任何人。

只要太后能提拔自己入两府,他吕嘉问愿鞠躬尽瘁死而后已。

……………………

“官人怎么还不出来?”

周南在灯下焦急的说着。

严素心坐立不安,就站在门口向外望:“是啊。都什么时候了,再忙也该派人送个信回来。”

云娘紧紧咬着下唇,手上的针线活早就没有按着样子来绣了,手上扎了一个个血点,都没觉得痛。

“别急,再等等。”

王旖说着冷静,但紧紧皱起的双眉,让人一眼就能看出她心中的忧虑。

韩家内外灯火通明。

后堂中,韩冈妻妾都聚在一起,等着家中的主人回来。

就算是李信和王厚先后报了平安,黄裳等一众门人也都来问安

但韩冈本人却始终没有任何口信传回,这让她们一个个都放心不下。

王旖宽慰着几位姐妹:“今天官人多半会留在宫中。多半稍晚一点就会派人出来传信的。”

“可是……”周南欲言又止。

王旖明白,摇头道:“没事的。”

可她也是一样难以安心。只要还没看见韩冈回来,终究是放心不下。

“京城虽好,还不如在外面过得安心。”

王旖闻言苦笑了一下,严素心的抱怨说到了她的心里。

早上送了韩冈出门,对王旖来说,今日不过是寻常的一天。但到了快中午的时候,她就听闻宫中有变,之后又得到了确切的消息。

二大王贼心不死,竟然联络了蔡确和两名权阉发动了宫变,囚禁了太后和天子,大喇喇坐在了大庆殿上,等待群臣参拜。逼得丈夫在殿上挥锤杀人,而且是宰相,方才扭转了局面。

王旖乍听闻便惊出了一声冷汗,好半天都没有回过神来。

黄裳等门人上门来,名为安慰、实则沾光。谁都知道,经过这件事后,再没有什么事能阻止韩冈回到两府。

可在王旖看来,这做官都做得提心吊胆,每天都要在刀尖上走路,还要与政敌相争,又为了道统,四面树敌,这样的官做得还有什么意思?

一日之间,或入云端,或坠泥沼。其得失进退,皆是归于天命——天子之命。

还不如退到地方军州上去。

一旦退出朝堂,按照多年来的惯例,留在朝堂上宰辅绝不会赶尽杀绝,天子也会刻意保护。

天道好还,报应不爽。谁知道过个几年,这一位被赶出京城的失败者,会不会卷土重来、东山再起?皇帝也需要留一把刀子,用以威慑朝堂。

就是当年新党对旧党,从上到下皆视如寇仇,欲除之而后快,最后还不是一个个在地方上安享富贵?

而且做到韩冈这个地位,离开京城到地方任职,谁还敢劳动他做事?就是每天开宴饮酒,来自京城的诏书,也不会是斥责,而是问一下酒够不够喝,钱够不够花。

好生的休养几年,让家里安安心,也能教导着儿女们成才。

可事情哪有那般容易。

王旖叹着:“也要官人愿意才行。”

韩冈有其目标,他要施展抱负,就必须留在京城中。可这样一来,日后如今天这般要担惊受怕的日子可能会更多。

正苦恼的时候,却听见外面一片人马喧哗,那声势是她日常听惯了。

王旖惊讶的站了起来:“是官人回来了?”

的确是韩冈回来了。

韩冈在外院没有耽搁太久,门人如黄裳,都以为韩冈会在宫中值守,早就告辞走了。

没有什么事需要吩咐,他很快便踏进了内院,

王旖已带着周南、素心和云娘,在门内等候,看见韩冈,便一起

反复说着:“官人回来就好。”

“都哭什么,乱臣贼子,为夫杀得还少了吗?过去也不是没有亲手杀过,何必担心。”

一想到差点,方才韩冈,心神松懈,便再也难忍住了。

韩冈“还以为你们看到为夫,会问怎么这时候回来了。”

“官人今天怎么回来了?”

韩冈微微一笑:“没了事情,当然要回来。难道没事留在宫中不成?”

韩冈笑了笑,便收敛起来。以王旖的聪明,应该明白自己今夜没有留下宿直宫中的问题。不过再大的问题,也要比今天早上,太皇太后坐在屏风后时要要强出百倍。

回想一下,今天的确是险。性命攸关之处,不比他当年刚刚病愈的那段时间稍逊。

幸好是过去了。

可到了这个地位这个年纪还要与人搏命,真要说起来,肯定是做错了。

方才太后没有留下自己在宫中,若是往好处想,是太后神思混乱,以至于疏忽了。可事情哪可能那么简单?

经此一变,总会有些想法。

坚持保住小皇帝的是自己,不论此事对错,政治上反复多变是致命错误。就算错了,韩冈现在也打算坚持到底,拖个几年,等风波平息之后,再说也不迟,现在则是绝对不行。

……………………

“真是想不到。”

沉寂了不知多久,张璪突然冒出了一句。

“谁都没想到。”章敦道。

“本人也是。”张璪说。

韩绛皱了皱眉:“想不到什么的,用不着提了,今天想不到的事太多了,不多这一件。”

章敦道:“的确不用提,事后总不可能晾着……功劳就是功劳。”

“嗯。不错。”苏颂略点了点头。

虽然全都没有主语,不过到底在谈论的是谁,宰辅们各自都是清楚的。

“算是好事,不论从哪边来说……”章敦侧脸对苏颂道。

苏颂也没有否定。

回想起来,就是太后偏信韩冈一人,才会酿成今日的大祸,差点将太后和韩冈他们自己都烧进去。要不然,乱臣贼子是不会有任何机会的。

但从今日之后,就不可能再恢复到过去了。

韩冈本人也意识到了这一点。

所以他今天在朝会重开后的作为,可以说是苦心积虑。

宰辅以外的一众朝臣,是韩冈出言带进崇政殿的。但最后又是韩冈将这些人给带出去。

他们除了唱反调,完全没有起到任何作用。不过,给太后看见他们在唱反调就已经达成目的了。

今天晚上,太后肯定会来召见众人。

一名内侍匆匆而来,几位宰辅看过去,是方才跟随在太后身边的一人。

‘终于来了。’韩绛、章敦、张璪、苏颂都这么想着。

“东莱郡公何在?”那名内侍问道。