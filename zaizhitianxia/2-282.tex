\section{第45章 千里传音飞捷奏(下)}

吕大防终于还是觉得要拜访一下韩冈。

在狄道城的这几日,他走访许多地方,同时也视察兵备、转运和医疗等后勤方面工作。越是深入的了解这座城市,他便越是发现,在这座城中,韩冈留下的痕迹实在太深。

疗养院就不说了,根本是韩冈一手创立并推广,如今在关西军中,无数人对他感恩戴德。而乡民和蕃人,更是对这位传说中的药王子弟视之若神。

转运方面的兵站制度,也是韩冈所制定。靠着这看似在路中浪费了太多人手的程序,三百里的道路上,粮秣损耗为之大减,而民伕们的怨言也变得只有很少一点,让人不由得深思起其中的道理。

作为转运中枢,狄道城中所有行之有效的制度,都跟韩冈脱不了干系。蔡曚贸然接手,却没有将之维持正常运转的能力。最后的结果,自然是怨声载道,诸事无不延误。

下面的官吏都是听命行事,其中没有推诿和拖延,这一点吕大防看得很清楚。中间形成的混乱,全是蔡曚一番错误的命令所造成的。

所以蔡曚满心怒火的叫嚣着要重责五十杖、一百杖的时候,就立刻被吕大防给拦住了。板子真的打下去,事情就不是简简单单的能解决了。

吕大防想过要出面帮忙,但他的身份却不对,更绕不过已经心生嫌隙的蔡曚那一关。他又想让韩冈来帮手,反正韩冈的闭门思过在王韶的捷报之后也就是笑话了,有功无过,又思什么过?有他吕大防协助,当能轻易压倒蔡曚,不至于再添乱。但韩冈就是不肯出来,就是要让蔡曚的蠢事昭示天下。

眼下狄道城中的局面越来越乱,要是不能及时将之处理,河州前线保不准就要断粮。若是因此而坏了眼下的大好局势,蔡曚和自己日后被责罚事小,让天子因此而更为倾向新党,问题可就大了。

并没有犹豫多久,吕大防便来亲自请韩冈出面,国事为重,个人的脸面只是等闲。

被韩冈请着坐下来,吕大防没有寒暄,也没有拉近关系,而是立刻发问:“玉昆,你可知如今的狄道城中已经一团乱了?”

韩冈的笑容游刃有余:“韩冈是待罪之身,此事是心有余而力不及。”

“事急无暇谋身,玉昆你何罪之有?”

“蔡运判可是不会这么看。”韩冈哈哈笑着。

吕大防的脸色,在笑声中冷了下来。虽然韩冈的话是指着蔡曚,但言下之意,吕大防听得明白,韩冈指明他前面的一番话只是个人的看法,就连蔡曚都说服不了,更做不得数。

有罪无罪只有天子够资格评判!——这话韩冈没明说,但两人都清楚。

“玉昆,今日运粮队又没能出发,你就不担心河州因此而乱?”吕大防换了个角度来劝说。

韩冈却是稳坐钓鱼台,“如今大局已定,癣癞之疾也不坏不了国事,御史不必太过心忧。”

他就是要看着蔡曚捅篓子,他就是要坐视吕大防无计可施。论起关系,吕大防虽然于己亲近,但韩冈可不会因私废公。吕大防和蔡曚背后都站着同一拨人,不将这两位一起坑进去,斩掉伸向河湟的贼手,他如何能安心的离开?

吕大防与韩冈渊深难测的双瞳对视着,从中没有找到一丝泄愤的情绪。他终于明白了,韩冈拒绝出手并不是因为一时之气,而是有着很明确的政治意图。

即是如此,吕大防确认今天是不可能说服韩冈了。心火上升,不过转眼就给他自己压了下去。韩冈的态度是正常的,总不能只允许自己压着人打,却不准他人反击的。

吕大防看着眼前的这位在关西官场上声名鹊起的年轻人,在温和的笑容下面,是一颗难以动摇的心。吕大防一生阅人甚多,心知这样的人物,只能用道理来说服,动之以情是没用的,“玉昆,河湟开边已尽全功。但你可知道这几年来耗用多少钱粮,日后为了维持这一路安危,每年又要输送多少?”

韩冈笑了,这一事,他可比任何人都清楚。吕大防想用此来说服他,那是班门弄斧。

“如果能保证每年两千户的移民,再有五年的时间,熙河路就能在没有大战的年份中做到自给自足。就算移民的数量降到过去几年的三五百户,十年内也一样能做到自给自足,不须外路支援。”韩冈对吕大防说着,“家严分管经略司屯田之事。家严这两年一番辛劳,单是巩州今年一年的田赋,就已经可以支撑三万大军三个月的食用。而巩州屯田的开始,至今也不过才过去两年!”

“蕃人岂会这般容易收服?屯田处虽云荒地,但实际上就是汉人侵犯蕃人土地。蕃人不乐于此,日后战事必然不断。官军四处扑火,二三十年内,岂会有没有战事的年份?”

“要使蕃人顺服,当设蕃学于诸州,化夷为汉。教化一事,是重中之重。让蕃部首领之子去蕃学就学,他们是质子,但教习忠孝之义后,日后他们统领族中大权,自然会亲附我皇宋。”“至于眼前的动荡,那是免不了的。不过就算蕃人反叛不断,只要在村寨中设立保甲,并以精兵屯驻要地,河湟当可无恙。”

“保甲法……”吕大防微一沉吟,决定还是单刀直入,“玉昆,你对新法怎么看的!?”

韩冈讶异地看了眼吕大防严肃的面容,决定还是保持自己一向的观点,他在程颢、张戬面前如此说过,在吕大忠面前也如此说过,就没有必要在吕大防面前隐藏:“新法多是善法,只是施行中有所偏差。比如最近的方田均税法,虽然乡绅多有不喜,但贫民之中,却多有乐之者。三代以井田定天下田土,方田之法中,却是又几分井田的用意在。”

吕大防微微的皱了皱眉,真不愧是张载的弟子,说起田制便是井田。洛阳的二程那边也在说井田。甚至是王安石都没少说过井田,却是一点都不现实,只是这个年轻人让他有些琢磨不透,对井田的看法,不一定是真的。

“不知玉昆你可听说过市易法?”吕大防又问道。

“市易法?”韩冈模模糊糊的在章惇的信中听说了一点,最近就要施行的法令,但具体内容却是一概欠奉。

摇了摇头,就听吕大防解释了一通。

“谁提出来的?这……这……”这是疯了不成?!后半句话韩冈吞到了肚子里,但他真的觉得提出这项法案的人真的是想钱想疯了。

剥去优抚小商贩的面纱,这项法令根本是抢夺京城豪商手上最后一份大饼的宣战书。青苗贷,均输法,都已经将京城豪商们手上的利源一点一滴的剥夺,韩冈不反对从他们手上拿钱,但做事不能做得太绝,凡事留一线,日后好相见。

豪商们的背后可是一户户宗室,龙子龙孙们现在有许多都是靠着联姻的豪商们的经济支援才能勉强度日。豪商们没钱了,宗室们都要饿肚子。如果市易法当真推行,熙宁二年反新法的高潮,多半又要在今年再现——别指望他们不会反击。

韩冈的震惊,吕大防看在眼里,情知不是作伪。

但韩冈却没有吃惊多久,静了静神,道:“同声相和,那是党。事事反对,那也是党。新法之中,在下是有所取舍。新法之中,青苗、均输是善法,保甲、将兵,在关西行之有效。农田水利,只要行事者能收起好大喜功的心思,在不扰民的前提下稳步实行,亦是良法。但保马、市易,在韩冈看来就有待商榷了。”

“如此玉昆为何不上书言及此事?”

“不在其位,不谋其政,韩冈现在只是一介边臣,哪有说这些话的资格。”韩冈目光变得深沉起来,“韩冈两兄皆殁于国事,国仇家恨俱在,誓与西贼不共戴天!至于其余,不是韩冈有资格说的。”

韩冈明确的向吕大防表明了自己的态度。听说了之后,他原本全力支持新党立场已经有了动摇,至少觉得做得太过火了一点。但他更为明确的告诉了吕大防,如果不能支持开边河湟、攻取西夏的国是,那他韩冈也绝不会站到旧党一边。

这其实也是王韶秉持的观点,谁支持他立功,他就站在谁的一边。

吕大防有些失望,他看得出来,韩冈说得是真心话。而且他更能看得出来,眼前的这位年轻人的想法,不会轻易的更改。

河湟开边的成功,让始终支持他的新党更加受到天子的赞许,也必然能让王安石的地位更加稳固,当几天后,捷报送进崇政殿的时候,市易法必然会被推行下去。

‘一战误国啊……’吕大防暗叹着,放弃了说服韩冈的想法。

也就在三日后,就在五月初十,露布飞捷的信使奔进了东京城中,在原本就已经风急浪高的朝局中,掀起了更大的狂涛。

