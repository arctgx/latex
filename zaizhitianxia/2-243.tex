\section{第35章 重峦千障望余雪(八)}

一年过得很快,转眼就是除夕。

禹臧家的军队已经退回了兰州。但前面不过两个月的时间,他和包约【瞎药】两家,将熙州北面的蕃部几乎全都洗了一通,让他们过年都过不好。道上的盗匪多了许多,只是没人敢来抢狄道,都冲到其他没有受灾的蕃部去了,这一个除夕,熙州北部将会热闹非凡。

可王韶现在所在的狄道城【临洮】,却是静悄悄的没有任何声音。今年的雪出人意料的大,厚厚的雪层能没进大腿根,远处近处的山峦皆是银装。露着一圈灰黄底色的一座狄道城【临洮】,仿佛就成了雪海之中一座孤岛。

韩冈前面派来了信使。二十多岁精干的年轻人骑着马,在路上走了六天。出来的时候,信使身上的穿戴跟一头熊一样,毛皮都裹到脚尖上。可一路行到狄道,照样还是冻坏了手脚。听着疗养院中的医官说,至少有两根脚趾保不住了。

这样艰难的局面下,王韶也不敢多派人手回去联络。看起来在明年二月雪化之前,跟后方的联系,怕是就只能靠着几天一次、损耗极大的驿马来传递。

“报……”拖着长音的一声叫唤,一名小卒通报之后跑进公厅中,跪下来就向王韶禀报道,“陇西城那里来了一队人马。”

“一队?”王韶强调的问着,韩冈没事派这么人过来做什么,人多了要多消耗多少驿马?就算是他是一路经略使,都是感觉着舍不得。

报信的小卒点着头,“一队人从南边来的。”

“怎么可能!”

王韶这下坐不住了,猛的站起身。南线虽说要平坦一些,可毕竟比现在所走的鸟鼠山北线多了近一倍的路程,如果走这条路,少说也要的多上两天的时间,人和马怎么能吃得消的。

小卒想了想,又补充了一句,“他们还带着六车的辎重。”

王韶差点就要骂起来了,‘雪地里走车?!胡说八道。’

王韶一百个不信,可是眼见为实,当他走出官衙,就看见一队车马驶了过来,总共的确是有六辆。

每三匹马就拉着一辆车,深一脚浅一脚的踏着狄道城中仅剩的一点冰雪,走到了衙门前。在车上高高堆起的货物,让人看了乍舌不已,也是心生疑惑,不知怎么这么沉的车子如何在雪地中行车。

王韶看得清楚,那几辆车上没有装一个轮子,只是在下面钉了两条窄窄长长的木板,木板在前端翘起。马车过后,后面就是长长的两条平行的印痕,从远处直拖过来。能弄出这种怪异的车子,不会有别人,只会是精于机关巧器,甚至在高喊以旁艺近大道的韩冈。

“这是韩玉昆让人打造得?”王韶先让人开始卸货,转头就把领队的小校拉过来询问。

小校却是一问三不知,只是从怀中把今次的货单和要接收者签书的公文,连同着一封韩冈给王韶的书信,一起递了上来。

等到高遵裕收到消息,赶来的时候,六辆车上的物资都已经卸得差不多了。六辆车中都是装着今年年节犒军的货物,基本上都是惯例的银绢茶酒。看到其中三辆车上满载着的酒坛,卸载辎重的士兵都欢呼起来。过年没酒喝可不成,从巩州千辛万苦送来的其他军资,他们都看不上,就是这几十坛最好。

“这是什么车?”高遵裕的第一句话就是这么问着,没轮子的车任谁都是觉得很新奇。熙河副总管疑惑着,绕着车子转了一圈。

王韶把手上的信折起,回答着高遵裕的疑惑:“玉昆称之为雪橇车。”

“雪橇车?”这个词让高遵裕很陌生。

“陆行乘车,水行乘船,泥行乘橇,山行乘檋。这说的是大禹治水时乘着何物出行。”王韶看了看茫然的高遵裕,补充道,“出自于《夏本纪》。”

“你们起名,总少不了个出处。韩玉昆该不是把大禹出行的橇车给重新打造了出来吧?”

“差不多,现在看看,这雪橇车在泥沼中也同样能前行,不至于会陷下去。”

高遵裕又绕着车子看了一圈,道:“其实用驮队也一样吧?”

“马驮的货物,哪有用车拉得多?驼了货物,马匹走起了也会更难。”

王韶的解释让高遵裕连连点头称是,啧啧赞叹着:“真不知韩玉昆是怎么给想出来的。”

“说是因为减少了摩擦力的关系。轮子在积雪上行走受阻,把轮子换成滑板,就减小了摩擦……还有参照了雪鞋的原理,什么压强、压力的。”

以自然之道为纲目,来考虑如何解决问题。而不是如工匠一般不求甚解,知其然不知其所以然;只知如何,不知为何。这是韩冈在信中写给王韶的话。

韩冈说得道理,王韶粗粗一览也没有看得太明白,高遵裕同样被一堆新名词给弄得糊涂起来。

王韶把信递给高遵裕:“玉昆的信上还画了图,设计了另外一种冰车,下面不是滑板,而是两条刀刃。说是冬天在河道冰面上使用。”

“玉昆这是要做公输般【鲁班】吗?”高遵裕都不知该说什么好,摇着头,接过信,“药王弟子不做了?”

“越来越搞不懂他在怎么想了。”王韶也是摇着头。韩冈在信中解说他所格致出来的自然之道,王韶很是难以理解,只是仔细想来,还是有着几分道理。

韩冈的心思并不是区区开边之事就能局限得了的,再一次认知到这一点后,王韶都感觉着有些泄气,“只要真有用就是了。”

“要不要试试看玉昆设计的冰车。”高遵裕看着韩冈在信中画得设计图,腾起了一些兴趣。

“再说吧,现在河上都是厚厚一层雪,走不了冰车。这些雪橇车,就是从洮河河面上过来的。”

“是绕得竹牛岭和抹邦山?”高遵裕现在才听到这队辎重走得哪条路,跟王韶方才一般的惊讶,“没人冻伤!?”

“不是骑着马容易兜风,坐在车上冻得就不会太厉害。而且玉昆让人把雪橇车设计得精妙,座位下面还有放火盆的地方。”

在高遵裕来之前,王韶就已经上上下下里里外外的把车子全都打量了一遍,里面的构造,也都了解了。

他让人把车夫的座位掀开来让高遵裕看,在车夫的座位底下,有着一个很大的空间,被木板分割成一个个格子。而正中的一格在内壁镶着隔火的铜皮,里面放着一个暖炉,暖炉的三条腿嵌在事先钻好的槽中,而暖炉的盖子也是带着卡子,不会在行驶中动摇。由于暖炉所在的这个中间的格子是前后镂空的,能够通风,木炭就在暖炉中缓缓燃烧,将暖意带给座位上的车夫。暖炉所用的木炭,就堆在座位下的其他格子中,走了几天,只用了一半还不到。

高遵裕盯着车座下的格子看了又看,再一次叹道:“当真要做公输般了。”

“不管韩玉昆是不是要做公输般,他终究是把过年的犒赏都运来了。”王韶看着摆在衙门前的一坛坛酒水,心中也放下了不少忧虑。

但这时,一名骑兵从西门处狂奔了过来,翻身下马,一下跪倒在王、高两人身前,“启禀经略、总管,洮西三里外,有数百蕃人的甲骑在活动。”

“又来了?”

“怎么胆子肥起来了?”高遵裕听着消息,脸上狰狞而笑,“就拿他们当过年的大礼好了。”

“多半是董毡插手了。”王韶猜度着,“木征也不是傻瓜,不会为董毡挡风挡雨,终究还是要把他的叔叔给拖下水的。”

………………

木征对他的三叔没有多少好感。他本人可是唃厮罗正牌子的嫡长孙,吐蕃赞普之位本来应该是他和他父亲的,只是阴差阳错落到了董毡的手里。

年轻的时候,木征还窥伺过那个已经算不上尊贵的位置,只是年纪渐长,变得有些懒散起来,只想保着他的河州。但心里一直都有想法,因而跟董毡始终不和。

可眼下的局势,容不得木征再跟董毡不合下去。

董毡不会太过尽力,这是木征清楚的。毕竟在平戎策中,明摆着写的是联合吐蕃诸部,而不是对抗。但谁都知道,如果董毡不能表现得出一位赞普该有的实力,那么新成立的熙河经略司不介意在吃掉河州这个正餐之后,把青唐王城当作饭后的消食汤水,一起给吞进肚中。

所以权衡利弊,最后在木征低头之下,董毡还是派兵来了,整整一千精锐甲骑,并承诺如果宋人攻打河州,他会再暗中派人来支援。木征这个不听话的侄子做邻居,让人很是头痛,偶尔还会让董毡感到胃痛。但换作宋人做邻居,却不是头疼胃疼就能了事的,那是要他给大宋做牛做马兼做狗啊!

身边有一条随时可能反噬的狼,总比换头张着大嘴的老虎过来要强出百倍。董毡不愿与宋人明里对抗,撕破脸对谁都不好,但暗地里襄助木征,他怎么都能派得出人手。

