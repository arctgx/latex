\section{第20章 冥冥鬼神有也无(19)}

李常杰的话,并不能带给人以信心,但聪明的倚兰太后,也不再多问。李常杰说是什么,那就是什么。

连同李乾德在内,三人其实已经被紧紧地捆在了一起,休戚与共。一旦李常杰无法控制升龙府中的局面,接下来,无论是倚兰还是李乾德,除了一死之外,都只有被献给宋人乞降一个结局。

倚兰太后重又开口,话锋由北转南:“占城那里又该如何应付?”

李常杰冷哼一声:“制矩为人色厉胆薄,升龙府破了他敢来,升龙府没破,给他十个胆子都不敢越境一步!”

倚兰如此知作,李常杰也算是心情好了那么一点,但一想起北方的局面,他的心情立刻又坏了下去。

所谓驱使鬼神,只出自门州逃人的口中,并没有确实的证据。从门州溃逃的士兵,人数也不多,以临阵脱逃的罪名斩了之后,也没有怎么流传出来。

另外一件事则更应该担心。自从宋军开始进攻之后,来自北方的情报便断绝了。北方山林深处的大小道路,时不时的就会走过一队溪洞蛮兵,谍报就算想将打探到的情报发回来,也无路可走。眼下更是停止了富良江上的摆渡,除了几个隐秘的去处,南北的交通已经完全中断。

现在唯一能确认的,就是宋军在发兵前,抵达邕州的西军只有五千余,加上荆南军,可堪一用的兵力只有七千——那些新军,没必要不用计算在内。

五千是全部装备铁甲的精锐,若是与之前交过手的荆南军比较起来,必定要胜出一筹,唯有水土不服一条,可以让他们不战自溃。

只是韩冈——曾经让他在胜利刚刚降临的那一刻,又将他打得大败而逃的韩冈——在宋国却是以医术闻名。不过李常杰不相信他能制服得了南方的瘴疠。而且交趾在失去了人和的时候,也只能依靠天时和地利,无论如何,都要拖到下雨才行。

……………………

理应是数九寒天、腊月隆冬的时候,头上却是一轮炽烈得似乎能将水烧开的太阳。

本应是在冰雪中艰难的迈着脚步,冷得瑟瑟发抖,还要担心着鼻子耳朵会不会冻伤后掉下来。但眼下却是汗流浃背,背后的衣袍就结了厚厚的盐霜。

这分明是关西盛夏时都少见的暑天,却是在快要过年的时候遇上了。

不过韩冈为了交州难捱的天候和水土,用了近一年的时间,做下了充足的准备。解暑的药材、驱蚊的药材、补充流失盐分的盐水,韩冈领导下的转运机构,不仅仅是在后方运送着资材,也包括随军行动,合理安排全军的饮食,保护着上万大军的身心健康。

只是外界环境上的剧烈变化,人也许还能适应,但牲畜就不可能。从关西一路南下,军中的战马损耗得有些大。士兵们能喝干净的开水,总不能给马也烧开水。人能脱衣服,马却不能将身上的长毛都褪掉。士兵们哪里不舒服,还能说给医生听,但马的一张嘴是用来吃草料的,没办法说话。

同样是饮食和气候上的问题,西军的将士们还能在医护人员的悉心照料恢复健康,逐步适应交州的水土;可是他们所骑乘的战马,已经有许多一命呜呼。

自进入平原地带之后,战马损耗的情况也越发的严重起来。韩冈每天起来,听到的头一个消息,就是昨天死了多少匹战马,又有多少匹战马病倒。

“幸好有滇马来补充。”韩冈叹着气,看着手上怵目惊心的数字,“否则关西最精锐的几个骑兵指挥,屈指可数的几个人人有马的马军,差不多近一半人要转成步卒了。”

“旧年国中,骑兵有马的十中不过一二,也就这两年熙河开了马市之后才好了那么一点。现在又有了滇马,南方日后也就不用再怕骑兵只有两条腿了。”燕达倒是笑着,一路顺利进兵,他的心情也是越来越好,“还要多谢副帅赠马,要不然末将就只能骑着四尺高的吗上阵了。”

韩冈摇头道:“逢辰你说哪里的话。我这文官坐牛车都行,你身为要上阵的大将,却是万万骑不得劣马。”

燕达从关西带来的坐骑在穿过交趾北疆山区后也病倒了,军中的兽医医治不了水土不服的疾病。燕达没奈何,就只能去刚刚送来的数百滇马中,挑了个头最高的一匹来骑乘,只是矮子里拔将军,到最后还是一个矮子,与正儿八经的河西良驹没法儿比。

韩冈见着燕达身为主将,却骑着一匹肩高还不到四尺的矮马,实在不像样子,看着就让人笑话,哪里还能摆出大将的威风,震慑军中?就把自己一直以来骑着的一匹有着北马血统的良驹送给了燕达,尽管还是比不过燕达之前的黄骠,但也算是看得过去了。

“诸峒蛮军在外面杀人放火,做得都是斩草除根的活,但李常杰始终没有出来一步。”燕达为坐骑谢过了韩冈,说起了正事,“接下来该怎做?是否要”

章惇沉吟着,“还是要慢一点,解决了富良江北面,再考虑南面,要以防万一。”

“现在才是腊月上旬,我们还有近两个月的时间来打进升龙府,要稳着一点。”韩冈说道,“而且交趾人将船都拖到了南岸,还要用些时间,将木筏或船只打造出来。”

未虑胜、先虑败。章惇和韩冈的谨慎,让燕达感到安心。

尽管已经打得交趾人不敢渡江来反击,可就算将新兵都算进来,掌握在安南行营手中的兵力毕竟也只有万人,一旦疏忽,就是万劫不复,完全没有失败的余地。

不过话说回来,官军兵微将寡,能打到富良江边,就是功劳;没有攻下交趾王庭,这并不是罪名。即便现在章惇、韩冈领军回师邕州,都是大功一件。

当然,燕达不会认为章惇、韩冈两人会就此罢手,见好就收。

两名主帅的心中都是转着将交趾彻底灭国的打算,一劳永逸的解决南方的敌人。要不然一系列有损两人声名的举措,就不会从安南经略司中给传出来,这都是为了铲除交趾立国的根基,其当务之急就是要清理土地上的人口。

从经略招讨司的临时驻地出来,燕达就遇上一队押送生口去城北营地的士兵。一行人中,男女老幼都有,踉踉跄跄的在长枪之下走着。

舍不得耕种许久的土地,舍不得居住数代的家宅,更舍不得烧毁家中后院的存粮,恋土的农耕民族不到最后一刻就不会逃离家乡,能下决断的毕竟是少数,等到州县官们纷纷南逃,连组织撤离的主心骨都没有了。

没有山峦可以藏身,由河流冲击而成的三角洲平原,就连面积稍大一些的树林都少见,除了渡过富良江,也没有别的道路。但十数万人要渡过大江,哪有那么容易。道路都给拥堵了起来,到最后,就只能成为俘虏。

同情?

燕达的心中当然是有一点,周围士兵们眼中也隐隐藏着一些。当看到饱经磨难被换回来的汉家百姓之后,一点同情全都烟消云散。

而且燕达看到更多的还是复仇的快感,尤其是从邕州征召而来的新军,一个个都与交趾有着血海深仇,亲人罹难,屋宅不存,如今终于能在交趾国中报复回来,心中只有痛快畅意。大部分针对平民的差事,都是交给他们去完成。

常言道匪过如梳,兵过如篦。

杀入富良江北岸的数万洞蛮,即是兵又是匪,一梳一篦的来回扫荡,城镇乡村都毁于一旦。而在他们劫掠地方的同时,宋军的脚步继续向着南方缓缓前进。但富良江下游的这片平原毕竟不大,数日之后,慢慢行进的大军,已经能听到富良江江水的流淌。

“唐时的安南都护府驻地、现今的交趾王都。这座矗立于富良江畔的南天都会,自唐武德四年立城以来,历经多方之手,名号改换不定。从始建时的紫城,到唐末的罗城【螺城】,又在五代南汉守将吴昌岌自立后改名为大罗。

不过这之后,交趾的都城都在南方的华闾。直至李公蕴代黎氏立国,以大罗‘宅天地区域之中,得龙盘虎踞之势’,遂将交趾王都由华闾迁至此处。迁都之始,李公蕴率众宿于城下江边,有黄龙现于御舟,以其祥瑞,故改大罗为升龙府,迄今已有六十余年。”

何缮对交趾王都的介绍,让人觉得他这个降人还是有些用处。不论他是早有所知,还是得官后才临时抱的佛脚,有这份见识,倒是没有浪费韩冈给他的那一张空头宣札。

“现在还有近十万流民聚集在富良江北岸。”燕达问道,“诸峒蛮军都是摩拳擦掌,要将他们分食。这件事,是交给三十六峒和广源军去做,还是由我们动手?”

“只要官军压过去,他们自然就会四散而逃,只要将他们驱散就行了。”章惇无意去让手上单薄的军队去卖苦力,“剩下的事,都交出去,只需要盯着南面!”

