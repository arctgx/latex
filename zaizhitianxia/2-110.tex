\section{第23章 铁骑连声压金鼓(六)}

【回到家都是十一点了,好不容易才赶出一章来。实在不行了,下一更等白天再赶。】

目送着围堵在城下的敌军一点点的远去,王韶心神略略松弛下来。号角声仿佛还在耳边回响,但城下已经空空如也。王韶暗暗赞着禹臧花麻说放下就放下的决断,换作是其他人来领军,大概都是得撞得头破血流后才会收手。

尽管他还担心着王舜臣和他的一千余名被困于星罗结城中的士兵。也准备从城中挑出千人左右的精锐,紧追在撤走的敌军背后,让他们无力顺势攻打星罗结城——禹臧花麻撤退的原因,让人颇费思量。王韶想到的几个答案都有道理,让他难以确定——不过现在,王韶只想好好放松片刻。

但王厚却很快打破了他的幻想,他匆匆带着一人走上城头。王韶认得他,却是韩冈的亲信。禹臧花麻撤围,与古渭的联系已经恢复了畅通,信使进城也没什么好惊讶。

“玉昆到底怎说?”等他行过礼,王韶立刻问道,“可有援军?”

亲信点点头:“有。机宜已经说动了瞎药巡检。”

苗授稳守的营盘,还有王韶主持的渭源堡,都在禹臧花麻所率领的吐蕃大军的攻击下,稳稳的守了下来。即便是在攻势最为猛烈的时候,王韶都没指望过援军。

他本以为要来也是俞龙珂先来,青唐城离古渭寨只有三四十里,以高遵裕和韩冈的手段,当能把俞龙珂那只狐狸从洞里给逼出来。

王韶却完全没想到,韩冈在来渭源的半道上就听说了消息,直接转去找更为接近的瞎药了。不过瞎药比他的兄长更为不驯,要让他火中取栗,难度比牵出一只老狐狸要难得多。出兵跟禹臧花麻敌对,瞎药从俞龙珂那里学来的随风而倒的态度,已经变得更为倾向于大宋。

“韩玉昆是怎么说动的瞎药?”王韶帮他的父亲问出了想问的话。

信使便把韩冈做的事,从头到尾、原原本本的说了一通。王韶和王厚虽然已经对韩冈的行事风格习惯了,但他直接斩掉了野利征,还是让他们吃了一惊,而对韩冈放弃了一桩能让他名扬朝中的大功,也颇为感佩。

“玉昆帮了大忙啊。”听完之后,王韶便喃喃的说了一句,随即他猛然抬头,对王厚道,“快去把苗都巡找来,今次得让禹臧花麻来得去不得!”

……………………

日出之后,城头上的空气中,仍弥漫着火炬燃烧后的焦灼味道。等日上中天,过了半日都还没有消褪掉。空气中的灰尘,将前几天天顶上澄澈如水的蓝色,染上了一层暧昧的浑浊。

王舜臣闭着眼,靠在雉堞上假寐着。夜战一场,城上城下都是累坏了。吐蕃人的兵力也只有王舜臣的两倍,昨晚一起熬夜,没有谁能休息下来。不仅王舜臣这边累得够呛,今天城下的敌军也没有继续进攻。

只是就算是攻来,王舜臣也是半点不惧。按照正常的战力交换比,两千多蕃兵也就勉强能跟一千精锐禁军相抗衡。若不是顾忌他们都是骑兵,而且攻打渭源堡的主力随时可能回返,王舜臣早就派人出城去野战了。

王舜臣的一个识字的亲兵,在他身前秉报着昨夜的损失,“昨夜出战者有两百零三,有四十二人没有回来。剩下的重伤病三十余人,都不能在短时间内重新上阵。”

王舜臣脸色如同头顶的天空一样阴沉,跟随他出城突袭的只有两百人,没能回返的就有四十二人,而且现在躺在病床上的,还有三十多人。他带出去夜袭的,都是精锐中的精锐,不成想损失竟然如此之惨。

王舜臣闭着眼睛,亲兵不知道他在想什么,犹疑中,声音便停了下来。

“怎么不说了?”王舜臣一下睁开眼问道。

亲兵连忙对王舜臣继续说道:“箭矢还有一万两千余支,已经集中起来,分配给擅长箭术的人。不过守城的器具就没有办法了。”

城中箭矢极度紧缺,加上没有油料,没有木石,连烧水的柴草都不多,守城的器具更是欠奉。宋军虽然善守,但巧妇难为无米之炊,缺乏足够的守城物资,王舜臣也只能让他的手下,做好与吐蕃人在城头上硬碰硬的准备。

王舜臣心中很是纳闷,他这里又不是大来谷那样的交通要道,也不是藏着有多少金银财帛,本就是做空空荡荡城池,蕃贼怎么会紧咬着不放?吐蕃人也好,党项人也好,他们打仗都是为了抢钱抢粮抢女人,什么时候也不会去做亏本生意。

但王舜臣却发现他如今所面对的,都是有组织的精锐,坚韧性上比起寻常蕃人要强出许多,所以他很吃惊:‘他们究竟是什么人?’

地面上传来的隐隐震动打断了王舜臣的猜测。他一跃起身,向东望去。只见尘烟扬起于天际,如雾气一般遮掩了东方山峦中的谷地,隔了一阵后,数以千计的骑兵出现在他的眼前。

号角声起,千军万马踏地而来,听在城内守军耳中,便宛如勾司人的锁链在悉悉作响。

围在城外的敌军一下多了近一倍半的人马。城头上,人人惨白了一张脸,原本就是被围攻的状态,已经渐渐不支。现在又多了一彪生力军,让他们完全失去了信心。

王舜臣看着神色变得麻木起来的下属,心底的一番狠厉之气勃然而起,“不想死的都给俺听好了!蕃人不过才六七千人马,什么时候蕃贼不到守军十倍,就能破城的?!都给俺打起精神来!”

他高声吼着,毫不犹豫地说着瞎话:“没有人想被人说裤裆里的两个蛋,被蕃人吓缩了去吧?别丢了关西汉子的脸。守住今天,王安抚明天肯定会带援军来!”

……………………

围攻星罗结城的西夏营寨中,禹臧花麻自马背上跳下。几天下来积攒的疲累,丝毫没有影响到他动作的矫捷,倒是腾起的烟尘,让他咳嗽了几声。

尽管已经从渭源堡下撤军,但禹臧花麻并不是要立刻顺着大来谷,撤回到鸟鼠山西侧去。他虽已经达到最初的目的,可今次劳师动众,甚至还向木征的弟弟许愿赠礼,却连一座城也没打下来,这等白跑一趟的事,禹臧花麻没打算去做。

出兵劳而无功,做了一次亏本生意,定会大伤军心士气。禹臧花麻心知手下的一众小蕃部的族长,跟随他出战,目的是为了财帛女子,可不是什么忠义。如果不能枪些东西回去。贼不空手这四个字禹臧花麻没听说过,即便听说过也不会用到自己的头上,但他的想法却是与这四个字正巧相合。打不下渭源堡没什么,但连星罗结城都打不下,那就太丢脸了。

禹臧花麻下马,几名禹臧家的将领随即跪倒在他面前。脸贴着地面,头不敢稍抬,带着深深的愧色,向禹臧花麻请罪。

禹臧花麻居高临下的盯着他们的后脑勺,眼神中满是恨铁不成钢的愤怒。

一开始禹臧花麻就没有幻想过能顺利的攻下渭源堡,能把星罗结城攻下来,就已是不虚此行。但他没想到,他留下的人居然无能如此。他都把本部的精锐都交给了他们几个,自己则是带着附庸部族的联军堵在渭源堡。以近三倍的兵力攻打一座残破不堪的小城,城中又是孤军,竟然到现在也没有一个结果。

“你们爱跪就跪着好了,试试看能不能把星罗结城给跪下来。”禹臧花麻狠狠地丢下一句,大步走进主帐中。

几个将领抬起头来,面色如土,他们想不到禹臧花麻会如此愤怒。这让他们一时失了方寸,不知该做什么为好。不过很快禹臧花麻的亲卫走出来,把他们唤了进去。

“说,你们还需要多久才能把那座城给打下来。”禹臧花麻虎着脸问道。

帐中安静了一会儿,一个犹犹豫豫的声音响起:“……一天。”

禹臧花麻随手拿起手边的一个茶杯砸了下去,“哪来的一天?!”

他撤退时以一千精锐守着后路,让王韶无法近距离的追击。从中争取到的时间,禹臧花麻想着用来一举破城。但他争取来到时间也是有着时限,王韶绝不会丢弃星罗结城中的士卒。禹臧花麻对被封锁在城中的士兵数量有所了解,足足三个指挥,上千人的兵力,王韶绝对损失不起。就算道路被死死堵住了,他也肯定会从其他地方设法绕路赶来救援。

按照禹臧花麻的计算,在王韶的追逼下,他只有一天不到的时间。如果半天之后,他还不能攻下星罗结城,剩下的选择就只剩饮恨而退这一条路。

刚刚继承了禹臧家族长之位的禹臧花麻绝不会让自己名字,跟失败连系在一起。他阴冷的视线如毒蛇信子般舔着一众将领的脸,盯着他们的心脏一阵阵的抽紧。

只听得这位吐蕃大酋的声音,冰冷得能把九月变成腊月,“三遍号角之后,若是再攻不上城头,皆斩!”

