\section{第七章 苍原军锋薄战垒(三)}

【两更补上】

过了一阵,周南突然问道:“官人。这一仗当真是输定了吗?”

“在横山一役后,西夏国势如江河倾颓,而大宋则是蒸蒸日上。如果步步为营,西夏必灭。就像这一次,如果只动用鄜延、环庆和河东三路,以银夏之地为目标,西夏必败无疑——夏天的瀚海可比横山难走多了。但现在官军直奔灵州城下,一千里地走下来,早就是师老兵疲,而西贼则是以逸待劳,反而变成官军拖不起了。”

“不是有官人的霹雳砲吗?”

韩冈摇摇头:“霹雳砲名气那么大。党项人能烧光所有的粮食,会蠢到在灵州附近留下制造霹雳砲的材料?”他苦笑着,“灵州是坚城。还在太宗皇帝的时候,就整修过一次,那时候灵州还在官军手中。等到灵州落到党项人手里后,也没有停止对灵州城防的修护。没有足够的攻城器具,想要攻下灵州,是痴心妄想。”

“今天太常礼院知院家的李夫人来拜访姐姐,就在说官军赢定了,也不知是谁说给她听的,姐姐也不好回她。”周南对军事也不是太懂,但至少是知道韩冈反对这一次西征的方略。

“战场上没有说必胜必败的,为夫是觉得官军输面居多,但并不代表官军必败,赢下来也不是没有可能的。”韩冈看了周南一眼,讶异道,“想不到你们妇道人家,也议论这些事。”

周南立刻道:“我们可不会议论。是姐姐的手帕交!”

正房和妾室之间的地位还是有差别的。在家里,韩冈的四位妻妾性格都不错,挺和睦的。但外面的夫人们来访,对周南、素心和云娘理都不会理。

“那你们平常议论什么?”

“要忙着家事,还有哥儿姐儿的功课要操心,也就说说闲话……”周南道:“今天还听素心说王家的六夫人昨天来找姐姐,又是为了苏子瞻——她一向是爱苏子瞻的好词——关在御史台狱这么久都不放,是不是真的要论死了?”

“要真的定了罪,会不会觉得很解气?”韩冈问道。

周南不高兴了,用力捶了韩冈一下:“奴奴哪有那么小心眼。吃点苦头就好了,哪还有恨到要人死的道理。”

韩冈揉了揉被捶的肩膀:“这么大的案子,不会很快审结,总得有个一年半载。就算断了死罪,也要等秋决才是。何况本来也不是什么大事,只是讪谤朝政。天子就算想杀鸡儆猴,夺官编管也能达到目的……”他想了一下,“照为夫想来,如果西夏顺利地打下来,天子心情好,多半就会放了苏子瞻。”

“如果赢不了呢?官人你不是说这一仗输面居多吗?”

韩冈咂了下嘴,“……那就得尽量不让他做田丰了。”

……………………

罗兀城在战前乃是守御边境的寨堡,因为西夏人几年来一直都很老实,算是很清静的地方。当年的守将王舜臣,每次会绥德,都说守在罗兀城能淡出鸟来。

可如今的罗兀城,城门处车水马龙。一辆辆车、一队队人马从几个门中进进出出。时不时的在城门口就有一起或大或小的骚动。要么是车辆损坏、驮马失蹄,要么就是车马迎面相撞,总少不了将城门堵上一时半刻。

“这要到哪天才能将城里的粮草都运上去。”转运副使吕大钧从门外走进来,满头大汗,“这兵站一程程的,卸货、装货耽搁的时间也太多了。”

“谁让在熙河路行之有效?”章楶从账本中抬起头,看着吕大钧从小吏手上接过湿手巾擦着脸,苦笑道“河湟之役经过了这么些年,兵站制度已经在陕西各路推广开了,但并不是有了兵站就能顺顺当当的运送粮秣。空学了皮毛,没学到本质,就是现在的情况。”

“要是韩玉昆当年也是为十万大军运送粮秣,成就不了那么大的名声。”吕大钧摇摇头,接过一碗冷茶,几口喝了下去。

十万人马和三数万人有着本质的区别,加上地理和路程,韩冈来了也一样没辙。这并非人力能挽回的局面。

将茶碗丢给小吏,终于感觉舒坦了一些的吕大钧坐了下来,“而且韩玉昆会在河湟开边时推行兵站制度,那是因为熙河路本来就没几户汉人,缺乏足够的民夫,是不得已而为之。鄜延路根本就不需要这么做。”

“倒也不一定。”章楶瞥了吕大钧一眼。听说学派上的纷争,吕家跟韩冈关系不睦,现在看来似乎不是谣言,“若这一次当真是韩冈代替李资深来主持粮秣转运,以他的手段,至少要比现在强。没看到昨天枢密院发来的札子吗?他可是好手段,同州沙苑监的种马全都调来了,堵得李资深什么话都说不了。”

吕大钧沉默了片刻,叹了一声后又摇了摇头。昨天的院札中还明说了,不论牲畜、人力的缺口有多少,都会超额补齐,只要求将尽快将粮草运到种谔手中,不得延误。这么一来,李稷怎么将罪名往枢密院和群牧司上推。

章楶冷笑道:“天子还给李运使下诏了,可‘斩知州以下乏军兴者’。想想吧,只要是有碍军粮转运,知州以下,一律可先斩后奏。这样的建议,多半也是韩冈向天子提议的,否则时间不会赶在一起。杀人不见血啊,看看李资深还有什么借口?”

李稷上书说用来运粮的牲畜病死太多,这等为自己找退路的手段,吕大钧、章楶这一干下属都看在眼里——说句难听话,他们暗地里都是支持的,李稷能藉此脱身,他们一样能。

可京城那边的应对却极为狠厉。牲畜要多少给多少,人手缺多少补多少,加上天子赐了先斩后奏的诏令,将李稷找的借口全都给堵上了。如果李稷不能给前方的种谔和李宪补足粮秣,罪名将全都落在他身上。而吕大钧和章楶,作为转运司中成员,连带责任一样少不了。

吕大钧有些灰心丧气,叹道:“依愚见,不如调回一些兵力,来守住粮道。这样往前运的粮草也能少点,粮道也更安全。反正不堪使用的军队,留在种子正手上的实在太多了。而且民夫逃散得太多,至少要补上一点。”

“在李运使眼里,这是让种谔日后可以推卸责任,怎么让他答应?”章楶摇摇头,“就是他答应了,军中的将校又有几个甘愿回师,为他人作嫁衣裳?”

“只恐民夫不胜其苦……李资深已经命张亚之督管道上转运。张亚之行事一向酷毒,不知他这一回要杀上多少人。”

“说得也是。”章楶叹了一声,“延州连妇人都征发起来运粮了,至今仍有一成多的田地还没来得及收割,就是收割了,也有许多没有脱粒晾晒,明年还要不要吃饭?”

这些年,章楶他都在陕西的仓司、漕司中打转,对其中的情弊,他了解得很深。这一次的确不妙了。

章楶担任转运判官的这段时间以来,眼里看的,耳中听的,都觉得李稷都快要疯了。眼下派亲信督管粮道,更是疯得彻底。光是杀人解决不了任何问题。不是杀得多了,杀得人心寒了,就能将粮草运送上去,这要靠手段和能力,决不是一杀了之。

而且朝廷似乎也是疯了,赶在五月开镰前出兵。眼下不仅仅是鄜延路都没有来得及将所有的粮食全都收割下来,其他几路的情况都差不都。今年的粮食还能靠常平仓补充。可眼下就算将西夏打下来了,明年年初的粮食缺口又该怎么办?

“兵足食不足,这一仗打下来,无论胜败,关中都是元气大伤。”吕大钧叹道。蓝田吕氏偌大的家业尽在关中。眼下的这一仗,吕家的损失很大,今年别指望有什么收成了。到了下半年,一旦不能及时翻耕土地,种下明年的口粮,就得动用家里的库房了。

“郭逵和韩冈都是反对急进兴灵,主张缓进。如果这一次仅仅是攻取银夏,河东、鄜延、环庆三路加起来十万兵马就足够了。根本不用我等坐在这里长吁短叹。”章楶叹道:“可惜天子不听人言,只听着王相公的撺掇,否则何至于此?……听说没有,辽国根本就没内乱,数十万大军已经压倒了鸳鸯泺。一个不好,就是万军齐发,到时候,别说攻下兴庆府,就是开封府都麻烦了。如今的这位王相公,可不能指望他做寇莱公。”

虽然是章惇的族弟,且又是福建人,但吕大钧觉得跟他倒是挺合得来,“还是指望泾原路和环庆路吧。高遵裕和苗授应该都到了灵州。一旦他们将灵州打下来,这一仗也算是赢了。”

“报!!!!!!……”一个拖长了声调的小校跑了进来,在吕大钧和章楶面前扑通跪倒,“副使、运判容禀。北方急报,泾原、环庆两路兵马已于壬辰进抵灵州城下,即将挥兵攻打灵州!”

