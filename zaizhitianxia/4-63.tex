\section{第14章 飞度关山望云箔(二)}

“贼军逃了?!还带着抢来的人口?”

韩冈和李信都面面相觑,怎么有这么蠢的贼人?但领头的韩廉却说得十分笃定,他是亲眼看着贼军驱赶着生口从村子里出来。

将信将疑的心情一直追到离正在撤离中的贼军还有三里地的时候。就在一片浓烟升起的庄子东南面,看到在广袤的田野上拼命向南却慢得如同龟行的人群,韩冈、李信才相信世上真有这么贪婪到愚蠢的蛮贼,“怎么有这么蠢的人?!”

敌军就在眼前,李信眼中燃起了火焰,“韩廉!你去盯着蛮贼,让他们再走慢一点!”

十几名骑兵应声就一抖缰绳冲了出去。韩冈和李信只带了二十名作为斥候的游骑,不可能让他们上阵厮杀,但用来阻碍骑兵更少的敌军行动,却十分方便。

“举旗!击鼓!吹号!”

李信的战旗举了起来,宣告大宋王师到来的鼓号声,在原野上向四面八方传的了出去。以行军队列行进中的队伍顿时停步,用着最快的速度整队,转换成作战阵型,开始追击敌军。

听到了鼓号的呼唤,推头看到了来援的官军,被掳走的百姓纷纷反抗起来。而为了吸引宋军来攻,押解他们的蛮贼一点也不手软,开始砍杀不肯听命的百姓。隔着一里的距离,前方的惨叫声清晰可辨,更可以看到前方蛮军的杀戮。见到这一幕,战旗大幅前倾,号角和鼓点更加急促,自韩冈以下,八百余名官兵的愤怒从鼓号声中传出。

可就被在追击的时候,蛮军依然没有任何动摇,用刀枪催逼着百姓前行。“是不是有问题?”随着韩冈一起前行追击贼军,苏子元越看越是不对,“官军都快追到他们了,贼人怎么还不肯放弃百姓?”

“伯绪前面没看出来?”韩冈很惊讶的看着苏子元,“没看到那座村子吗?如果是烧的是房子的话,烟气哪里会有这么浓?还看不见多少火!?”

“里面有伏兵?!”苏子元倒抽一口凉气,转头望着不远处正在燃烧的村庄。

“唉。”韩冈叹了口气,“伯绪你知道西军每次大败都是因为什么吗?……是伏击!关西千山万壑,官军与西贼交战,哪一次不是提心吊胆,防着西贼的伏兵?追击的时候,更是要左右看着两边的山沟。”他惨然一笑,“这可是几十万条人命换来的经验。说起演技,这群蛮子可比党项人差多了。”

“运使你是打算将计就计!?怎么不……”苏子元一声惊叫,瞪大了眼睛,指着冲锋在前的官兵,“难道他们都看出来了?”

“荆南平蛮,都是在山中走,哪有不防备埋伏的?下面可是连什伍都知道了。没看方才过村子的时候安排了最精锐的一队靠着村子在走?没看到始终离着村子有三十步的弓箭射程?没看到下面士卒的眼睛方才都盯着哪边?没看到殿后的又是哪一个都?我们可不是张守节。”韩冈笑得很开心,只有先骗过自己人,才能骗过敌人,“贼人会使计,多半也是兵力不足的缘故。村子就那么大,外面还生了烟,最多藏下一两百人。这点伏兵,随手就能解决。”

“为何不直接攻击村子,应该能将前面的贼人引回来吧?!”

“万一他们砍杀百姓怎么办?”韩冈反问。

苏子元沉沉的点了点头,虽然是冒了风险,但将计就计的确是最好的办法。

韩冈笑了一笑,他还想要见识一下荆南军的实力。赶了一天的路,麾下的士卒没有久战的气力,如果换上其他情况,他肯定会先让士兵休整后再出战,但蛮贼的自作聪明让韩冈看到了一战而胜的机会。

用湿草做出了浓烟滚滚的模样,遮掩了村中伏兵留下的痕迹。而宋军竟也没有细查,盯着前面的队伍追了上去,一切顺利得难以想象。刘永兴奋的捏着拳头,透过护村的矮墙向外张望。宋军就在他的眼前追击而过。转眼间,前锋已快要追到离开的队伍,而后军也都越过村子四五十步。一切都按着计划发展,只要解决了眼前的宋军,回到昆仑关,就能让胆小如鼠的黄金满看得眼睛红掉。

“二洞主,该冲了!”何学究狠狠叫道,“别让宋人有时间张起神臂弓!”

刘永一直耐着性子,就等着这句话。随即一声大吼,一马当先直冲了出去,两百名精锐也紧紧跟随着他,一齐冲出了村子

当身后一片吼声响起,正在追敌中的宋军回头一看,一群面上满是刺青的蛮贼,正哦哦怪叫着,如同恶鬼一般从背后冲了上来。而前方又是一片吼声,原本正驱赶着百姓拼命向前的贼军,这时候也纷纷返身杀了过来。

“好了。李信!指挥追敌之事由我代理。至于后方,由你来处置!”韩冈驭马前冲,冲着前军高声吼着,“贼军已经中计。后方一百多小贼而已,有你们的李都监在,足矣!速速击破眼前贼人,救出我大宋子民!南下之战的头功,看看谁人当先拿到!”

随军的小鼓更加急促的敲了起来,这是加快进攻的催促,数百渴求一战的荆南精锐欢呼起来,纷纷冲向敌军。

在前方贼军中混杂着百姓的时候,官军不便动用神臂弓,但作为荆南军中的精锐,刀斧用得也一样不差。作为先锋的一个都,手持大斧旋风一般冲入敌阵,血光顿时冲天而起。

重达十几斤的精铁大斧挥砍时,都会带起一阵猛恶的呼啸,如同狼入羊群,当者披靡。广源蛮军拿着刀盾想要抵挡,却哪里能抵挡得住。脆弱的刀枪盾牌一劈就断,连同后面的蛮兵,搂头给一斧头劈开。

与此同时,被蛮贼强掳的百姓趁机挣脱了束缚。但他们没有逃跑,而是怒吼着冲向返身对战的贼人,向着焚烧他们家园、杀戮他们亲友、蹂躏他们妻女、抢劫他们财产的强盗,用手、用牙、用一切能用的武器,奋力撕咬过去。本已是难以抵挡,猝然之间又受到前后夹击,蛮贼顿时溃不成军。

而后阵此时,李信已经跳下马,两名亲兵捧着十几支掷矛,身后是为数八十人的选锋,都是李信模仿关西的习惯,从他麾下数千军中挑选出来的精锐。一百六十多只眼睛,冷冷的看着冲杀上来的贼军。

刘永冲出来的时候,距离宋军后阵就只有四五十步,这个距离神臂弓根本来不及拉动。不过他没想到宋军反应极快,转眼就是殿后的队伍堵在自己的面前。

但眼前的宋军只有手上兵力的一半不到,他哪里会放在眼中。四五十步转瞬就只剩一半,刘永冲在最前面,手上的大刀瞄准身穿一身山文甲的李信,金光闪闪的甲胄已经炫花了他的双眼。他用足了气力一声大吼,要把自己的得意给吼了出来。然后……他就见到他的目标,踏前一步,以双眼追之不及的速度挥了一下右臂。

‘为什么他右手连肩甲都撤了?’最后一个疑问刚刚在刘永的脑中亮起,传入耳中的尖啸声尚没有引起他任何反应,一阵麻木的冲击就从面门传来,转瞬之间,所有的意识就都沉入了黑暗中。

一支掷矛从刘永的面门扎了进去,轻易击碎了脆弱的鼻梁,穿过了软腭,扎透了舌根,最后带着血红的液体从颈后穿了出来,将广源州大首领的亲弟弟,钉死在地上。

就在李信展示着他名震军中的掷矛之术的同时,他一手训练出来的选锋,也同时掷出了手中的铁矛。只要在三十步外击中放在地上的银盘,就能揣着坏掉的盘子回家,李信模仿着种世衡的练兵法,在这时候见到了功效。

连串的破风声后,接上去的是一声声惨叫。两百蛮军伏兵,选锋们只是一击就解决了三分之一。而掷矛接连投出,转眼之间,只剩下最后一名蛮贼茫茫然站着。他已经被巨大的冲击夺去了所有的神智,不逃也不降。下一刻,七八支掷矛同时贯穿了他的身体,仁慈的将他送回到他的首领身边。

轻轻松松的一战就解决了自作聪明的敌军,只用了比吃饭多上一点的气力就是近千斩首,下面的士兵喜笑颜开,打扫着战场,等着宾州城中的官民出来相迎。

但为首的韩冈、李信都是阴沉着脸,苏子元更是连眼睛都红了。躲在村中的何学究被揪了出来,他磕头如捣蒜,为了保住小命,将自己知道的军情和盘托出。

一名徐姓秀才献策,李常杰用了堆土成山的策略。一点点的将土山向邕州城头上堆。不过城中多次募集敢死之士出城劫杀,筑山的进度缓慢。可是在李常杰指挥下,邕州城已经接连战死了一个都监和一个供奉官,“一个叫薛举,一个叫刘师古,这是今天早上刚刚收到的消息。”

“没有其他的了?”

何学究磕着头,“小人不敢有半点隐瞒!”

韩冈嫌恶的看了何学究一眼,一挥手:“将他拖出去。”

两名亲兵走过来,一把将人夹起。何学究惊得呆了,拼命挣扎,大声叫道:“官人,你说过不杀小人的。”

韩冈冷眼了看着白读了圣贤书的汉奸一眼,“我是不杀你。但宾州的百姓会不会杀你,就看你到底做没做孽了!”

何学究被拖下去,三人皆默不作声。虽然还没破城,但李常杰用得手段却是正打在邕州的死穴上。从他们的进度上看,邕州最多最多也就再坚持三五日的时间。

而且这还是两天前消息,如果要救邕州城,剩下的时间也就两三天了。是等后方大军过来,还是设法继续前进。

苏子元看向韩冈的眼神中带着乞求,但他不敢说出来,这关系到韩冈、李信和近千将士的身家性命,他不能指望韩冈为此冒风险。李信紧锁眉头,昆仑险关天下闻名,仅仅八百疲兵根本攻不过去,而抄小道则有全军覆没的危险,这个风险他不能带着他的兄弟袍泽去冒,但他不能当着苏缄儿子的面,说放弃救援,只能选择沉默。

过了好一阵,韩冈终于开口,用着就像是出去吃饭的语气:“我们要拿下昆仑关。”

