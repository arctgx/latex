\section{第27章 京师望远只千里(二)}

依偎在韩冈怀中,嗅着熟悉的味道,沉稳的心跳声从紧贴着的结实胸膛中,一声声的传入耳内。若是在平日里,当被韩冈抱在怀中,严素心自幼坎坷、始终缺乏安全感的心,很快就能平复下来。只是今天,她却有些难以平静。

前面王厚过来,别的话她没听清,只听到了最后几句,也是她最在意的。“官人……又要去京城了吗?”她幽幽问着。

“……嗯!”韩冈沉沉应了一声。

自入宦海,韩冈与家人便是聚少离多。平常总是在外面奔波,归家孝顺父母的时候也难得有几天。现在好不容易能歇下来几个月,过些清闲日子,却又被一封诏令召去京师。

韩冈感觉到抓着自己衣襟的一双小手突然握紧,而瘦削的肩头也有着轻微的颤抖。

“不会太久的,很快就会回来。”韩冈搂着少女坐下,在她耳边好言抚慰着,一遍遍地诉说。素心把头埋在韩冈怀里,怎么也不肯抬起来。

大腿处传来充满弹力的触感,黑翼的秀发透着诱人的香气,带着鼻音的抽泣反而引起了心头的,韩冈搂着少女的双手渐渐不规矩起来。

他手上的动作不急不忙,手指摩挲着白皙的颈项,感受着落指处的细腻。然后拨开襦袄的领口,指尖在纤细秀气的锁骨上划过,轻轻按在锁骨交汇处的凹陷上。秀丽的小脸扬了起来,紧闭着双眼,晶莹的珠泪犹挂在长长的浓睫上,微微张开的鼻翼呼吸略显急促,初雪般的双颊染上一团红晕。韩冈的手便更加深入的探了进去。

“三哥哥!”韩云娘在外面叫了一声,推门进来,正看到素心被韩冈搂坐在床边。已是衣襟半解,圆润的肩头露在了外面,一团白嫩纤巧的雪腻正握在韩冈的大手中,如同面团一般变幻着形状,粉嫩的一点红莓在指缝中半隐半现,而一线细若萧管的呻吟,也在同时渗入她的耳中。

过于刺激的画面,让小女孩“呀!”的一声惊叫,连忙红着脸退了出去。跑到走廊上,她又羞又嗔的回头啐了一口,瓜子小脸血一般的绯红,手捂着脸,热得发烫。但握在晒得黝黑的大手中的那一抹雪白,却一直在云娘眼前晃着。她羞恼的瞪着眼前薄薄的两扇房门,“还是白天呢……”

严素心很快就红着脸从房中走了出来,身上的衣裳已经穿戴整齐,只是脸上还是如同晚霞映照。

韩云娘明明已经害羞的不敢睁眼,但脸上的羞涩没有影响她的发挥,在素心面前故意歪着头,问道:“这么快就结束了?”

反而是年纪大的少女受不起云娘这等促狭的眼神,脸都要烧了起来,结结巴巴的:“我……我……去厨房做事了!”

吃晚饭的时候,素心都是低着头,脸色红仆仆的,不敢跟人正眼相对。小丫头则是有些不高兴的样子,嘟着嘴没言语。只是听到韩冈把聘妻病故还有被召上京的两件事一起都说出来,两女却都又惊呆了。

韩云娘是两件事都不知道,而严素心也仅仅知道韩冈即将要去京城,并不清楚韩家未来的主母已经不在人世。突然听说此事,她们心中在惊讶之余,都是五味杂陈。

而韩千六那边,则花了一阵时间方才消化了这些消息。他有些拿不准的问道:“已经下了定,该算是亲家了。要不要去上个香?”

“还没成亲,没这个规矩。再说,又是在江南,哪里去上香?”韩阿李叹了口气,为着自己没过门的儿媳,叹道:“也是个没福气的孩子,听说还是少有的贤惠,真真是可惜了……三哥儿,你和厚哥儿他舅家刚刚定亲,也不算丧妻,是用不着服丧。只是娘心里虽说也急着想看到你娶亲,但人情面上一定要做好。刚走一个就立刻找新的,这点就不好,娘劝你最好等过半年再重新寻亲也不迟。”

“娘教训得是,孩儿明白的。”韩冈点点头,他娘这样处理的确是妥当的很,也跟自己想法暗合。

“娘知道三哥儿你一贯稳重,多余的事就不用我多说了。你后天就要走了,明天要养足精神。今天晚上,有什么事就自便好了,素心、云娘都行。”韩阿李说话百无禁忌,原本还在惊讶中的素心、云娘两人,都把头低得看不见人。

吃过饭,韩冈先陪着父母闲聊了两句,方回转自己的书房。书房中,灯火隔着窗户纸透了出来,两个动人的剪影正映在窗户上,说话声也从房中传出。

“……就怕三哥哥到了京城后,被狐狸精给迷住……赵家大哥上次还说那人是京里有名的花魁娘子。”

“听说官人一直都给人家写信,每次边上有人去京城,都要亲笔写信去联络。”

“肯定是狐狸精!不然三哥哥绝不会一直写信过去。”

韩冈听不下去了,推开门:“在编排我什么坏话?”

“官人!”“三哥哥!”

两女大吃一惊。玉色的脸颊殷红如血。在背后说人坏话,却被人听个正着,没有比这更让人尴尬了。两名少女都站了起来,低垂着头,红晕爬上了脸颊,修长的颈项有着天鹅一般动人的曲线,闪着更胜人一筹的的光泽。

“没……没有……”韩冈目光灼灼,让想为自己辩解的云娘声音渐渐低了下去。

韩冈笑着坐了下来,拍拍大腿,示意二女都坐过来。搂着两名少女香软的娇躯,想起了人在京城的周南,再怎么说都已经隔一年的时间了,她的心是否还能保持原来的纯净?会不会受到他人的欺负?信笺不同于语言,白纸的黑色字词并不直观,难以让人放心下来。

……………………

政事堂的公文里催得甚急,韩冈没有慢悠悠的准备时间。第二天衙门里还在评说昨日.比赛的胜负,但韩冈已经手脚麻利的,把眼下他手上所有的公事都做了总结和整理,移交给他人代管。而家中,素心和云娘则是帮着韩冈整理着远游的行装。

第三天清晨,并没有看黄历的余暇,韩冈带着李小六上马启程。父母,还有云娘、素心,皆倚门而望,遥遥相送。

到了城门口,汇合了一众亲卫,他们将会把韩冈护送到秦州。而寨中主帅高遵裕,领众出城相送,举杯辞别。韩冈相熟的几个亲友,赵隆正领军巡边,来不及赶回来。王厚、王舜臣,一直送了他到十余里之外。

一路朝起暮宿,不数日便到了秦州。

韩冈身兼两份职司,即是缘边安抚司的机宜文字,也是秦州经略司的管勾伤病,既然被传唤入京,到了秦州后理所当然的也要向郭逵打个招呼。而郭逵的反应,也正是符合了韩冈早前的猜测。

“玉昆高才,此去京师,当有一番大作为,”郭逵举着酒杯,不吝在酒宴上、在众官面前,展现自己对韩冈的青睐。

“承蒙经略夸赞,韩冈愧不敢当。”

一个晚上都在混乱中度过,前来搭讪的对手被郭逵全数带走。韩冈从郭逵的神色中也看不出什么异样。过了一阵,韩冈正准备结束这场无聊的宴会,一名白发苍苍的老将进入了他的眼帘。

是张守约!张守约这位关西军中的老军头,因为燕达这个毛头小子撞大运似的抢到了他头上,便一气之下跑到了连接秦凤、泾原两路要道的中心要镇——水洛城,还上书自请镇守水洛,没事就不肯回秦州来。

只是为了今次陕西河东诸路共同攻取横山之事,秦州已经很久没有接收到关中腹地发来的钱粮,所有城寨、军队都消减了不必要的开支,勒紧裤腰带过日子,水洛城自也不会例外。张守约今次行事时便是徒唤奈何——再不来要钱,年就别过了——只能跑回来向郭逵抱怨,跟他叫穷。

另外今次李信也要去东京,就跟去年的刘仲武一样——试射殿廷。籍此博取一个官身。虽然按理说,李信年后再往东京去也来得及,但韩冈既然现在就要赶往京城,张守约便把他发派了出来,也顺便护送一下韩冈。

张守约摇晃着酒盏,酒香四溢,“什么时候后成立了古渭州路,我就要申请调职去那里任总管或是副总管,不受毛头小子的气!”

“设立新路?没有那么容易吧?”韩冈摇头表示自己的反对,在酒宴上他多喝了两杯酒,脑袋都有些发僵。

老将自得的笑了一笑,韩冈没看透的,他却是都看透了,“如果夺下了武胜军的狄道。肯定要设一路经略司。秦凤路在缘边四路中已经是地域最广的一路,再扩张下去很快就会被距离所束缚才是……缘边四路都是为了针对北面的敌人而设立,现在秦凤路一边要在甘谷城一线对抗党项人,一边还要支持开拓河湟,分心二用,事所难成。”

“一旦夺下武胜军,必然要专设一路,用来针对党项人的侵袭。古渭的缘边安抚司只会再扩张,而秦凤路就可以重新把精力放在北面。

