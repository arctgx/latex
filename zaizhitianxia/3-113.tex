\section{第30章 众论何曾一(六)}

过了年之后,时间一转眼就快到了上元节。

这些日子,白马县中并无大事。也就是京城的一些消息,让白马县的百姓们竖起了耳朵来打听。

到了正月十二的时候,一座座灯山已经在县衙门前扎起,论规模和华丽的程度,肯定是比不上京城那一座座过了冬至就开始准备的彩灯鳌山,但节日的气氛也算是出来了。

这些彩灯,都是县城中各家行会出手。其中最为卖力的,却是白马县中的粮行。这在往年,是不可想象的——粮商们一向低调。但当诸家一口气捐出了家中所有的存粮来换一个官身之后,只要长了耳朵都知道这是韩冈的手段。

听说了京城粮商们的下场,看到了诸立开罪知县的后果,如今哪个敢来触宰相女婿的霉头?不让白马县热热闹闹的过个上元节,弄得冷冷清清凄凄惨惨的样子,试问韩县尊如何会放过?

既然商人们舍得出血,市面上自是看着就热闹多了。商铺、民家张灯结彩自不用说,连着县衙中隔绝前后的屏门前,也挂上了两串彩灯。

城中热热闹闹,城外的节日气氛也不算差,这一个新年,过得其实都不错。

市面上的物价降低,乡民们花钱的欲望也随之大增。手中的余财除了留一部分用来购买粮食之外,也拿出了许多来置办年货。

流民中的精壮在韩冈的安排下,于县中几处被井十六点出水脉的地方打井,他们的卖力,也换来了还算丰厚的报酬,除去了日常开销,给家人换身新衣也许还不足,但花上三五文钱,弄两盏小灯意思一下,绝大多数流民还是舍得花这份钱。

至于韩冈本人,年后的这十几天来,也是收到了不少好消息,主要就是水井的开凿。

自从第一口深井出水之后,日前韩冈一口气就铺开了三处。现在其中有一处已经见水,尽管依然不是自流井,但在大旱之年,能见到水就是一桩喜事。故此听到深井出水的消息后,有不少乡绅跑去喝了井水,继而转头就联名向韩冈情愿,要在村中也开凿几眼深井。

一口好井对于农民的意义无需赘言,跟田地一般都是能留给子孙的财富。旱年两村争水闹出人命来的案子,韩冈能在县衙架阁库中找出一摞子出来——这还是在许多人命案没有报官的情况下留下来的。

就在黄河边上的白马县,对于苦于旱涝二事的百姓们来说,一口据说能常年出水、且不受灾异影响的水井,怎么可能不受重视?更别说深井的井水甘甜清澈,在冬天舀起来时还带着地气的余温,不是那些只有一丈两丈,最多也就三五丈的浅水井可比。

榜样的力量是无穷的,一口好井让人看到了希望,现在许多村子都要开凿深井。韩冈也就趁机将一干流民分派过去指点他们。由井十六点下位置,然后由几名流民带着一干村中的健壮动手开凿。

流民们指点如何凿井,当然不会免费,负责食宿的同时,理所当然的也要给些工钱。这一下子,就给县中省了不少开销。韩冈现在都在盼着深井开凿的名气,能早一点传播到外县去。如此一来,肯定有饱受大旱之苦的外县的乡绅或者是官员来引进这份技术。到时候将学到技术的流民们都派出去,自己这边也可以轻松一点——有了正经的工作之后,流民当然也就不再是流民了。

而此前韩冈为了能不用人力而提取深井井水,以用来灌溉田地,在县衙外的八字墙上挂出了五十贯的悬赏。利用畜力的提水机械,张榜之后就立刻得到解决,根本没有耗费时间,竟然有七八个人来争抢这份酬劳。韩冈让他们各自去做出个样品之后,就将他们打发了。等到样品验证有效后,再让成功之人均分。

而利用风力,前两天,也有人过来揭榜,声称知道如何打造风车来汲水。

只是当韩冈细细询问过之后,来揭榜的那一位却被戳破了谎言。他仅仅是曾经见过用来磨面风的风磨,只能画出外面的样子,并不知道风车的具体结构。来揭榜却不过是打着蒙混过关的想法,想着趁机捞上一笔。

但只要知道在这个时代,已然出现以风为动力的机器,对于韩冈来说就已经足够了。他可是深受天子看重的朝官,同时还有着一个做着宰相的岳父。所以对于带来消息的这一位骗子,韩冈判了他十五臀杖作为欺骗的惩罚,另外给了五贯作为消息的奖励。

在明确了这个时代有着风车实物之后,韩冈就打算传信东京,看看京中的大匠们有没有打造风车的手段。以他见识过的工匠们的能力,只要给出原理和要求,多半就能得到一个满意的答复。

王旁已经在白马这边住了快十天,每天给韩冈拉着在白马县中四处跑,虽然累着,但心情却还不错,都有些乐不思蜀的样子。只是父母就在京城等着,他总不能在外面过上元节。

昨日王旁向韩冈辞行,今天韩冈就带着几名幕僚出城来送他回京。没有临别的诗句,只有几杯水酒,还有韩冈请他带回去的礼物。不过更重要的事情,是韩冈将在京城中寻找会打造风车的匠师这一事,拜托给了王旁。

“若能用风车汲水,田地灌溉就不需再等待天时,如今的旱灾也就不再。白马县上下企盼,可都要靠仲元兄及早传回佳音。”韩冈与王旁肩并肩,一边走着一边说道。

王旁差不多是拍着胸脯来回答:“玉昆放心,愚兄必然不所托。”

“一切都拜托了!”韩冈深深一揖,与王旁道别。

一同来送王旁,等着宰相家的二衙内走远,游醇低声问着两名同僚,很是不解:“风车取水之事,正言为何不直接向朝廷上书,何必转托私人?”

魏平真笑道:“请王二衙内帮忙,可以靠着王相公。上书朝廷,最后也是要落在王相公手上。与其冒着不知被谁丢到角落里的风险,还不如直接一点更为方便。”

方兴也道:“现在可不会有多少人敢将正言的奏章丢到一边,但耽搁时间可是免不了的。中书之内,一封并非军情的奏章不走三五日,怎么可能能递到宰辅们的案头上?哪比得上王二衙内的一句话。”

魏平真和方兴其实心明眼亮,韩冈这么做,等于无端的分功给王旁。等到王旁将人找到,韩冈很有可能就会将这份事交给他来做。要不然这些天来,韩冈一直将王旁带在身边又是为了何事?不过话说回来,自家现在也在忙得团团转,恨不得有人能帮把手,一点功劳分给他人,他们也不愤恨自己手上的饼少了一块。

何况王旁还是宰相的儿子,能多多结交绝不会是坏事——两人虽然一个是王雱所荐,一个是靠了王韶,但要说他们跟荐主有多亲近,那就是开玩笑了。若真的是心腹,根本就不会转荐出来。

跟在韩冈身边几个月下来,这位以七品朝官的身份来做知县的右正言到底要做什么,两人都已经看得很明白。在白马县城外的几处流民营,只观其规模,就知道这根本不是一县之地该分管的事。足足能容纳数万人之多的营地,怎么看都只要要州府来治理。白马县只有两千多户口,若无背后的支援,绝不可能负担起比县中户口还要多上几倍的流民。

至少现在,魏平真和方兴都可以确定,韩冈来担任白马知县,绝非在外界大肆流传的缘故。只从韩冈身上,就可以发现王安石对于今年的灾情,早已有所准备。

韩冈听着身后幕僚们的窃窃私语,他不知道魏平真他们在说些什么,但想来多半是在说刚刚离开的王旁。

事情其实还是很简单的,主要还是因为有夫人在吹枕头风。韩冈其实是可以直接上书,但通过王旁去问王安石,其实也是一般。既然没有区别,能顺便解决一下家中的问题,自是公私两便的好事。

王旁跟浑家庞氏吵闹不休,在韩冈看来还是太清闲的缘故。就算没有多少才干,但王旁终究是读过书的士子,不可能没有做一番事业的志气。而现在他却是留在家中陪着父母,看着父兄、亲戚,以及来往的宾客,商讨着国家大事,当然心中有份发泄不出来的怨气。愤恨、自卑,诸如此类的负面情绪,都不会缺少。如此一来,疑心病也随之而生。如果让他有些事可以做,就不至于会将精力都放在疑神疑鬼上。

就不知道王旁究竟要多长时间才能够回来,这边的灾情可不等人。

一路回到县中,经过看不出正在受到旱灾侵袭的市面,还有行走在街巷中人们脸上的笑容,韩冈的心中充满了成就感,这是他精心治理的结果。他现在只盼望到了一两个月之后,白马县百姓们的脸上还能有着如今的这份笑容。

