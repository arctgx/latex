\section{第12章 大厦将颓急遣行(上)}

咚的一声闷响,伴随着竭力压低的惨叫,下一刻,清脆的碎裂声从陈举的书房中传了出来。

黄德用拿手捂着头,从指缝处露出的额头皮肤上乌青一片。只要一放手,就可以看见他额头上刚刚长出的瘤子跟脖子上一般大小。在他的脚底下,是一地的石头碎片。石头碎片只看那色作青紫的温润,还有其中一块碎片上那枚圆滑的凤眼,就可知这石头碎片的前身,定是难得一见的上品端砚。如今在地上粉身碎骨,看着着实让人可惜。

被人用端砚砸了脑袋,一向气焰甚高的黄班头却连叫痛也不敢。只按着痛处,老老实实的站着。不过他脑门上挨着的那一记实在够重,虽然没见血,但眼前闪烁着金星,脑袋嗡嗡直响,却像是千百只闪着光的苍蝇围着自己打转。

拿价值千金的端砚丢向着黄德用脑门的那一位,看着黄德用痛得站不稳的样子,走近了很关切的嘘寒问暖了一句:“黄班头,很疼吗?”

被那人在耳边一说,黄德用浑身一颤,忙放下手,低着头肃然而立,两个瘤子一上一下交相辉映。只是看他龇牙咧嘴的样子,肯定是痛得厉害。能让黄大瘤老老实实的人物,秦州城中并不少,但能让他发自内心恐惧的,却也只有陈举一人。

年近五十的陈举外表并不起眼,中等的个头,长得黑黑瘦瘦。可胜在相貌忠朴敦厚,长得慈眉顺眼,脸上总是带着一点谦卑的笑意。对于年轻人来说,他是个可亲的长者,对于长官来说,他是个可信的手下。这样的一个实诚人,第一眼就能博得上司的好感,如果再能办事得力,哪个长官会不信重?

也就是这个貌似慈祥的中年人,让几任知县含恨而走,多少官员无可奈何。陈举的势力,不仅仅局限在成纪县,在军中,陈举有人,在蕃部,陈举有人,在京城,陈举照样有人。曾经有一个进士身份的主簿,想挑战陈举的地位。但最后的结果,是主簿被贬去琼崖孤岛,而主簿的妻女则一起给陈举收入房中。陈举三十年把持着成纪县的内外事务,而越发的根深叶茂。

陈举又瞥了黄德用一眼,眼底的憎厌一闪而逝。黄德用此人胜在听话好用,所以就算有点贪色,他也从没放在心上。哪里会想到为了一个才十二岁的小丫头,竟然闹出了那么大的乱子。

想到这里,陈举心中更恨:‘十六岁就敢孤身出外游学,远行千里,这样的人岂是好相与的?!而且还是横渠先生的弟子,也不想想他的同学里有多少家衙内!他的老师又有多少好友!’

还有自作聪明的刘显,陈举也是恨铁不成钢。韩冈一个毫无凭籍的措大,敢在大街上与黄大瘤直接翻脸,分明是个胆大包天的光棍脾气。这样的人竟然还把他放在德贤坊军器库的位置上,只想着能一举两得,就没考虑过什么叫鸡飞蛋打?他陈举只收了八十贯,就把监军器库的位置给了那个胆小怕事的周凤,到底是为了什么?!

踩着砚台的碎片,陈举在厅中重重的踱着步。这砚台是他最喜欢的一方端砚,而且还是老坑出来的石头。是他从一家破落的官宦人家费了不少心力才弄来的,若拿到外面去卖,少说也要上千贯。但现在却在他脚底下发出嘎吱嘎吱的悲鸣。

陈举用鞋底碾着砚台碎片,恨不得这些石子是韩冈的脸,能狠狠地踩在脚底下!

这是陈举的书房,除了黄德用外,其实还有七八个人高高低低站着一旁。他们都是陈举的亲信,当军器库事发后,便被陈举紧急召唤过来。他们看着一砚台砸在黄大瘤的脑门上,皆是噤若寒蝉,生怕陈举将怒气转移到他们头上。

他们都在等着,等着有人将进一步的消息送回来。

更鼓咚咚咚的敲响,听着鼓点,刚刚交了三更。警号传遍秦州城时是二更天,到此时才过去了一个时辰,天上的半轮上弦月甚至还没有升到天顶。

秦州城毕竟有宵禁,巡城、更夫、潜火铺铺兵,还有在高耸的城墙上来回巡视的守城军卒。一整套严密的监察体系,让夜中秦州城的大街小巷举步难行。陈举能在德贤坊军器库事发后,不到一刻钟便收到消息,再过了半个多时辰的时间,就把手下从全城的各个角落给找出来,他的势力之大也可见一斑。

终于,当更鼓敲在三更一点的时候,一名亲信下人进来禀报:“押司,刘二爷回来了!”

书房中的众人精神一振。陈举忙道:“还不快请二爷进来!”

刘显听到传报,拖着沉重的双脚走进陈举的书房。他今夜是将功赎罪,卖足了气力去打探消息。自家瞎了眼,把一条五步倒当成了菜花蛇抓了起来,如今被狠狠地咬了一口,就算死了也只能怪自己不长眼睛。

“现在人在何处?”看着刘显进来,陈举急急问着。

“现下都在州衙里。韩三,王五和王九都是。”刘显说着摇了摇头,“都没有下狱!”

此时的规矩就是这样,管你有罪无罪,在定罪之前,定是要在狱中走一遭。而韩冈和王五、王九三人手上都沾了血,按律条,当时就要下狱的。而节判吴衍没有依律行事,分明已经将罪名认定给刘三和他背后的人物了。在场的众人都是老于吏事,怎么会想不明白?神色也是更为不安。

“不用担心,小事而已。”陈举温言安抚手下,他不信区区一个穷措大真能翻了天去。但韩冈的狠辣果决,让陈举看到了自己年轻时的影子。他不禁有些感慨,江湖越老,胆子越小,也只有年轻人才能这么毫不顾忌后患。

刘显给陈举出着主意:“韩冈其实可以暂时放到一边,最重要的还是军器库。只要军器库里的窟窿不给查出来,刘三的事怎么都能推掉。”

“也不过万来贯的亏空,填上就是了,钱从俺这里拿。”陈举说的轻描淡写,但随随便便就能拿出万贯家财,就算在东京城里也不多见,“除了钱以外,兵器上亏空今早之前查清数目,差多少就跟赵彬借多少,李相公再怎么查也不会查到都作院【注1】去的,就算查到了,让工匠们随便造些抵数的也不费多少功夫。”

陈举其实他心中也后悔,如果早知有这一档子事,他提前几个月改改帐册,就能将亏空填上了;又或者不吝啬一两万贯钱钞,直接把窟窿补上也没现在的事了。

“但现在德贤坊被州里的人盯着,钱物就算拿来了,怕是也送不进去!”一名亲信提醒道。

刘显嗤笑一声:“放在县衙里不就行了。只要数目合上,再在帐目上加个转库,谁还能说不是?”

陈举点了点头,这么做就算想挑刺也挑不出来。轻轻松松的解决了最大的问题,剩下要面对的便是韩冈带给他们的困难局面。而陈举此时也有了腹案,“关键还是在王五和王九身上。他们是给韩冈吓住了,也怨不得他们。”

只要王五和王九肯改口,光凭韩冈一张嘴,连口吐沫也吐不到他陈举身上。陈举转身对着站在书房角落里的一名高壮青年,道,“小七,你找个机会跟他们俩见一面,就说是俺陈举亲口说的,前面的事可以既往不咎,但……”

“押司!”刘显突然出言打断了陈举的话,叹道:“押司有所不知。刘三他们身上皆有刀伤,而且都是砍在要害上!……是王五和王九的佩刀。”

陈举的话说不下去了,韩冈做事竟然滴水不漏,哪里像十八岁,根本是条八十岁的老狐狸。半天后,他方才恨恨吐出几个字,“好个韩冈!”

书房中的众人面面相觑,而黄大瘤的脸色越发的难看。他们都知道,既然作为当事人的王五和王九已经拉不回来,那解决刘三一事的办法就只剩一个。刘显欲言又止,陈举则是犹豫了片刻,最终摇了摇头,长叹了一口气,对黄德用道:“黄兄弟……你先回去吧。”

黄大瘤呆住了,他如何不明白陈举让他先回去究竟是什么用意。他惊叫道:“……押司!”

刘显走到黄德用身边,扶着他的肩头,柔声道:“黄家老哥,你先回去歇息一下,今天够你累的。”

黄大瘤的脸色白得如石灰粉过一般,瘤子泛着铁青色。一天前的此时,他还躺在净慧庵妙心尼的床上,搂着美貌的光头尼姑,惦记着韩家的小养娘,可十二个时辰之后,他已是面临绝境。白天在普修寺门前时,黄大瘤怎么也没想到,一日之间,风水轮转,竟然是他看不起的穷酸措大把绞索套在了自己的脖子上。

绝望的看看陈举,又看看刘显,黄大瘤扑通一声跪倒在地,抓着陈举的靴子,哭喊道:“押司,你看在俺往日的情分上,留俺一条活路罢!”

“德用你这是作甚,你是俺的兄弟,俺怎么会不留你活路?!”陈举面无表情的说着,退后了一步,用眼神示意站在门口处的另外两名亲信:“还不将黄兄弟好生扶将出去!”

两人会意点头,这是让他们监视住黄德用,以防他在绝望中做出什么事来。他们一手捂住黄大瘤的嘴,一边从两边将他架起,硬夹着不断挣扎的黄班头,出了书房。

“二弟,待会儿你去追上黄德用,跟他说,俺保他的妻儿安安稳稳一辈子,让他放心去罢!”陈举难得的收敛了脸上伪饰的笑容,脸色阴沉的可怕。

刘显点了点头,示意自己听到了。陈举转过身,透过半开的窗户,直直望去州衙的方向。没人看见他的表情,只是半天后才听见他从牙缝中迸出的两个字:“韩冈!”

注1:地方州县中,负责制造兵器弓弩的机构,一般只有边疆的州郡才有设置。

ps:第一个敌人解决了,更强的敌人又紧跟着过来。想看着韩冈在继续踩人的同时,走向更高的地位,请不要吝惜手上的红票和收藏。今天第三更

