\section{第十章 千秋邈矣变新腔(18)}

榜前人山人海。

宗泽按时在卯正起床,但已经迟了一点。

租了一匹马穿街过巷,走到贡院的前街处,就再也走不进去了。

京城出租马匹的贩子很多,街口桥头都能看见,当客人租马之后,就会一起跟着过去,抵达目的地后就将马给牵回来,若是客人片刻就回,更可以再赚点回程钱。宗泽就是跟这租马贩子说好看了榜后就回去。

看见前面堵得水泄不通,莫说一时半刻,就是一个上午过去也不见得能少些人。租马给宗泽的小贩,登时就急了起来,耽搁片刻,他会少赚多少?家里的浑家孩子都要吃饭。只是宗泽是贡生,过来看榜的,保不准就中了。哪里敢催促?只能绕着马打转。

宗泽见状,一笑了之。付了钱,打发了这贩子牵着马回去。

站在巷口,望着前面乌压压的一片人群,宗泽也作了难。

里面的没有出来,外面的拼命向里挤,前前后后都堵在这边,还不知什么时候能挤到榜下看看上面有没有自己的姓名。

巷中并不都是考生,那些穿戴奢华,前呼后拥的贵人,反而占了大多数。

跟随在这些贵人身侧的仆役,一个个膀大腰圆,身高体健,显而易见,就是为了去抢人做女婿才这样。

宗泽正这么想着,就听到轰的一片声,前面的人群忽然让开了一条路。

几个急着在外打转的,正想往里面挤,忽然就踉踉跄跄地被推开,紧跟着一队健勇就昂然而出。

前面的几条大汉左推右搡,之后七八人在中间夹着了一个书生,后面一个又高又宽又厚的贵人压阵,最后还有几人守着后路。

排开众人后,一辆马车正好就开了过来,几人将书生往车中一丢,后面的贵人随即上车,其他人上马的上马,步行的步行,护着马车扬长而去。

宗泽瞠目结舌。

他曾经与那名被架走的书生打过照面,那是从蜀中那边过来的贡生。虽然没说过话,但那一位的脾气倒是不小,宗泽与他见面的时候,其实是看见他正跟人吵架,蜀人特有的口音一听就明白。

当时这一位连劝架的都一起骂了,一人舌战群儒,丝毫不落下风,让宗泽对他的印象极为深刻。

这样也被抢走了?看起来丝毫没有反抗的余地,再犀利的唇枪舌剑也难抵四条大腿粗细的胳膊,当真是秀才遇到兵有理说不清——也难怪东府里面最年轻的那一位,遇上太皇太后和宰相联手的叛乱,根本就不动嘴皮子辩论,直接挥锤敲碎脑壳了事。

但这榜下捉婿的也够厉害,果然是捉,生擒活捉,比起县里的快手捉贼,就差上索子了。

宗泽没有为这般有辱斯文的举动愤怒,反倒觉得有趣。如果咬定牙根不愿结亲,难道还能当真强逼着未来的朝廷大臣拜堂?左右这样的事只要不摊到自己头上,那就是三年才得一次的打诨杂剧,站在台下,不看白不看。

感觉上有了些乐子看,宗泽就不心急着去看榜文了。此时结果已定,若榜上有名,迟看一步也不会被人抹去,若榜上无名,早看一步也一样找不到自己的姓名。

宗泽随着人流一点点的往里面蹭,小一个时辰过去,终于能看见聚在榜单下的一群人了。

这段时间里面,宗泽又看到几次好戏。有几个贡生如那位蜀地贡生一般生拉硬拽的被架走,也有几个是自己随着人走出来的。不过他们出来的时候,皆是前后左右都有重兵严防死守。生怕给他人拦路劫走。

不过在,也看见了哭到晕倒在地,被随行的朋友架着离开的落榜贡生。五千贡生,才四百余人中选,其实失落而归的贡生,在离开的人群中还是占了大多数。

宗泽看到一张张失魂落魄的面庞从擦肩而过,不免心下恻然。

宗泽熟悉的同学张驯就站在榜下,身周里三圈外三圈围着一大帮人。

遥遥望着张驯意气风发,不用去看榜单,宗泽就已经知道,张驯这一回定然是高居榜首,得中省元,否则又如何会有这般气派。

张驯的才名,早就遍传京中。在国子监中,本也是不需要应考就能直接从上舍直接受赐进士及第,只是有一次考试没有考好,才不得不来参加解试、省试。然后,就轻松过关。

早在考前,国子监中学官们就在议论。以张驯的才名,状元不好说——这与太后的心情有很大关系——但进士高第必然少不了他一个。这一回高中省元,宗泽也不感到惊奇。

五千人中第一人,纵使还不是状元,却也是值得夸耀一辈子的事了。张驯欣喜若狂,也是常理。张驯身边的人,也都如众星捧月一般,将他团团围住。

宗泽不打算凑热闹,离开榜单十几步,他就立定了脚。和其他考生一样,眯起眼睛,引颈而望,从密密麻麻的一张名单中,寻找着自己的姓名。

从右侧最上的张驯开始,一个个姓名从眼前掠过。有熟悉的,也有陌生的,但始终没有看见最为熟悉的那两个字。

视线在榜上飞快横扫,一个姓名跃入眼底,又立刻掠过去后,但随即就停住了。再返回过去,那个熟悉的姓名就出现在眼前。

宗泽脑中微微一晕,身子也轻轻晃了一下。心脏剧烈地跳动了起来,呼吸声也变得大了一点。

第九十四。宗泽,两浙,国子监。

名次、姓名、籍贯,以及得到贡生资格的发解试。

宗泽排在第九十四位,不算很高,但也不低了。在四百五十五人中,名列前百,在宗泽自己来说,也不会再奢求什么。

而且省试中的名次高下做不得数,即便是名列榜末,也跟位列省元的张驯没有太大的差别。省试定去留,殿试才定高下。真正的名次,要在殿试上才会排定下来。

十年寒窗,宗泽在读书时付出的心血不比任何人要少。要说他对进士资格不放在心上,那纯粹是骗人。

在乡里,回乡的新科进士总是得到最热烈的追捧,而出身小商贩的祖父,也总是拿着本乡历年高中的前辈,勉励宗泽认真求学。

耳濡目染下,尽管宗泽有着为万世开太平的宏愿,但进士资格,依然是他心中最重要的目标之一。

只有有了进士的资格,才能够实现自己的抱负。君不见,如今继张载之后,执掌气学大纛的韩冈,也是在有了朝官资格之后,还要去考一个进士出来。

数年前,横渠四句教刚刚开始传出关西,宗泽的书房中便开始挂起亲笔书写的‘为天地立心,为生民立命,为往圣继绝学,为万世开太平’这四句话,对气学的好奇与探究之心也是从那时开始。

毕生宏愿终于实现了第一步,激荡的心情反映在脸上,依然只是淡然的一笑。

纵然心中欣喜欲狂,想要将着喜讯与家中的老父、老母分享,但宗泽也做不到像身边不远处,一位同样高中的贡生般大笑大叫。

不过这样也好,那位正大叫大笑的贡生,已经被两拨人一左一右的扯住了胳膊。两拨人的为首者,一边瞪着竞争者,一边三千五千的开始报数。

而宗泽身边却没有任何人。如秃鹫一般,守在榜下的贵人和仆人们,在仔细审视过宗泽看榜之后的反应,便都不感兴趣的挪了开去——他的反应实在太平淡了。

这对宗泽是个再好不过的反应。正要不惊动任何人的转身离开,就听见榜下传来一声大叫,“汝霖,恭喜了!”

抬眼看过去,竟是张驯在大声喊。

顺着省元看过来的方向,宗泽就成了众人瞩目的焦点。

宗泽年少,二十出头的模样在大多数人眼中,显得年轻了一点。

俗话说三十老明经,五十少进士。那是熙宁六年前还没有改动考试科目时的事了。熙宁六年后,明经科被取消,进士科改考经义,进士的平均年龄也有了些许降低。但再是降低,也没有降到随随便便就能看见二十出头的少进士。

不过真正有才学的士人,都是二十出头、三十不到就高中进士的。稍有见识便知道,这般年纪的进士,往往就意味着三十年后的一位金紫重臣,甚至有望身登两府。

本来宗泽一派温润醇和,气定神闲,没有其他列名榜上的其他贡生一般心浮气躁。看起来也不想是高中的样子,倒像是来看热闹的——在这榜下,颇有些无关的士人想要过来见识一下,以此来勉励自己。

但高中省元的张驯这么一声喊,宗泽立刻成了众矢之的。

再见他又是年轻,投来的目光中又添了几分炽热。

周围人的眼睛已经开始冒起了绿光,宗泽心中大叫不妙,拍着那位已经开始被人抓着手挣来抢去的仁兄的肩膀,大声喊了一句:“汝霖兄,恭喜了!”

大部分人的视线,转向了那位走了运的贡生,而宗泽趁机就往外走去。

张驯脸色冷淡了下来,盯着宗泽的背影,看着他就这么消失在人群中。
