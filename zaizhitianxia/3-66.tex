\section{第24章 携眷西返家(中)}

【下一更夜里赶不出来了,明天中午时补上。】

新婚燕尔,韩冈作为丈夫又是温柔体贴,王旖也放下了心头事,脸上的笑容渐渐多了起来。

过了三日,王安石那边送来了冠花、彩缎等物,是为‘送三朝礼’,而韩冈、王旖这对新婚夫妇,也照规矩回门去拜见王安石和吴氏。

此时有着名为‘会郎’的礼仪,新女婿第一次拜门,岳家都要广设宴席,用以款待女婿。一起向王安石夫妇行过拜礼,王旖就被吴氏拉去了里间,王家的三姑六婆正在里面等着三堂会审。而韩冈则被留下来应对着王家的亲戚。

今次王安石嫁女,王安国、王安礼正好都在京中守阙,只有最少的王安上仍在太原。三兄弟的子侄辈,也有十来人,以王雱最长。不过这些王家子弟都是被教训得极为守礼,韩冈这个新女婿入席,各自老老实实上来敬了一盏酒,却没有如寻常人家哄闹起来灌酒的。

过几日还有王安国女儿要出嫁叶涛,王家还有着忙,宴席没有拖得太长时间。喝过酒后,王安石将韩冈招到书房里,一口口的啜着醒酒汤,说些闲话。

方才见到二女儿的时候,王旖脸上的笑容瞒不过人。知道韩冈待她甚好,王安石对他这个二女婿藏在心中的一些芥蒂,也为之烟消云散。

喝了两口凉汤,王安石问着韩冈:“玉昆,准备什么时候回陇西去?”

韩冈欠了欠身:“劳岳父垂问。此事定在十天后。再迟了,路上暑热,就有些不方便了。”

一般的进士,在琼林宴结束后,基本上就要回乡省亲。有家室、或是要回乡完婚的走得早些,而在京城做了女婿的,则是走得迟些,不过很少有等到婚礼满月之后才回去。再在家乡住上两月,大部分的新科进士,都是到了十月之后,才陆续回京候阙。

韩冈考虑得周全,王安石点了点头,又道:“替我向亲家问好……你岳母还有礼物要送给亲家母的,到时一并带上。”

韩冈起身,向王安石拜谢:“小婿代家严家慈谢过岳父、岳母。”

“这是做什么呢。亲戚间来往乃是应当,没能将二姐送到陇西去成亲,本来就是我这边失礼。”

命韩冈安坐下,王安石沉吟道:“前日在中书看到巩州蔡延庆的奏报,亲家在熙河路所管的屯田一事,两岁的考绩都是在上下,如此勤谨极是难得,政事堂前日已有堂宣,为之迁上一官。”

儒家讲究着中庸,基本上不会有极好或是极坏的评价。在唐宋,上中下九等考绩中,上上考绩从不与人,上中也是极少——是要立殊勋方可——基本上上下就已经是顶头了。如张九龄那等贤相,唐玄宗给他的钦定的考绩就是中上。韩千六两年上下,在算是了不得的评价,磨勘减个一两年,直接迁上一官都是应该的。

韩冈欠身谢过,王安石不避外人可能有的讽刺,为韩千六加官,肯定是要感谢的,“家严做事一向勤勉,小婿在家严那里学到的很多。”

“德在才上。才士易得,德士难觅。亲家虽非才学之士,但德行过人,如今官场上也是难得。玉昆你若能效之而不移,日后当是能为国之柱石。”

韩冈的父亲虽然目不识丁,但能做事,而且做得好。王安石对韩千六也是十分赞赏,他最后决定招韩冈为婿,其实也是有着韩千六的一份功劳在,“去岁秋冬,开垦的官田据说又多了一千两百顷,就算只是一亩一石的薄田,今年也是多了十余万石的收成了,五月份,当是有好消息来了。”

一说起熙河的发展,王安石就是满带着欣喜。巩州、熙州两地的田地连着棉田一起,现在可以肯定已经在五千顷以上,虽然都不会是多肥沃的良田,但有可比没有要强得多。

韩冈道:“本来在预计中——只要不改年号——到熙宁八年之后,熙河路军民的粮草供给就能自给自足。之后再过两年,靠着榷场和岷州钱监,加上开始收取的税赋,熙河路日常驻军的饷银,以及官中的耗用,可以保证一半以上的本路供给。不过从现在的情况看来,只要没有大灾,两事都可以提前至少一年。”

“那样就好。”王安石听着很满意,缘边四路,也只有秦凤一路能保证自给自足,其余三路,比起韩冈所描绘的熙河路未来,可是要远远不如,“金城三郡之地,汉室乃是中国故土,如果能固本培元,不再拖累朝廷财计,日后也不用担心会有所反复。”

“但熙河路的关键还是在户口上。如今开发出来的,也就是通远军改成的巩州,熙州狄道城,加上岷州铁城堡一带。至于狄道城向北,直至临洮堡的一片河谷地,还有河州的一片谷地,都是因为户口不足,却还一时无法去开发。若是汉人不能继续流入,熙河路中的发展恐怕会后继乏力。”

王安石叹了口气:“陕西厢兵已经汰撤了不少,但愿意去巩州的还是不多。”

熙河路招募移民,都是保持自愿原则。用免费分配的土地和免税制度,来吸引在内地过不下去的百姓实边。所以强迫被淘汰下来的厢军移民,那是不可能的。

而且缘边四路由于常年战事不断,几乎没有排不上用场的军队,汰撤的厢军数目极少,而熙河更是没有一名。真正厢军汰撤的大头是在永兴军路,也就是长安为核心的关中腹地。驻扎在关中平原上的不堪战的厢兵,总数多达三四万,去年一口气被汰撤了一半。只是相对于熙河路这等边远荒僻的新疆土,丢了饭碗的厢军士兵,更愿意留在关中找口饭吃。

“玉昆你所说的事,朝廷都有考虑过。”王安石道,“熙宁以来,每年大辟常过三千。其中真正犯了刑杀重罪的并不多,多是贩运私盐等事。政事堂现在有考虑赦去此等人死罪,可杀可不杀的一律发配熙河。”

“死囚……”韩冈迟疑起来。

大宋主客户总计两千万余户,人口总数可以肯定是在一亿以上。这么一个帝国,每年处决的死囚,超过三千人。这个数目不能算小了,而且一般的囚犯,更是接近百倍。死囚中一部分是杀人、劫盗,另一部分则是经济犯罪,多以贩运私盐等严令禁止的走私行为为主。而贩运私盐,直系亲属都要连坐。

而三千人这个数字并不包括军中,单是熙河路,去年就杀了两百多犯了军法的士卒——尽管熙河路去年是处于战时,有着特殊情况,不过推及全国百万大军,恐怕也是接近千人了。

而且这也不算是别出心裁。至少在半年前,赵顼就已经下诏让各地州县尽量将罪囚流放熙河,而不是旧有的沧州、荆南、两广等地。同时罪犯,死囚也只比流囚近一步而已。但这不是没有别的问题:

韩冈叹道:“就怕坏了熙河路的风气。”

贩运私盐那可不是普通的走私贩,黄巢就是贩私盐的出身。私盐贩子好勇斗狠,能打头的几乎都是有几条人命在手。好勇斗狠其实也是好事,但更大的问题是,此等人桀骜不驯,很难约束他们遵守军法。无组织无纪律,上阵岂能堪用,若是收录入军中,到时候把熙河路搅得乌烟瘴气,韩冈更是不想看到。莫说死囚,就是流放沙门岛、通州海岛等岛上牢城的重刑犯,韩冈都不想要,听话受教的厢兵比起他们这些罪犯来好得太多。

王安石却道:“这些罪囚各个勇武,如果能教训得宜,未必不能当大用!犹如广锐军一般。”

那等罪囚哪能比得上广锐军!韩冈叹道:“就怕他们勇于私斗,怯于公战。”

这个时代的士大夫,总把战士立敌千军的勇武和市井流氓的好勇斗狠混淆在一起。公战和私斗是两回事。怯于公战而勇于私斗的,世所常见。要是王安石做决定前,能问一问通晓兵事的武将,或是经历过战争的文官,就不会犯这等错误。

“这事就再说吧。”

王安石转述的其实是天子的想法。减少死刑数量,不论是在后世,还是在如今,都是一项德政。要不然唐太宗时,一年只有几十个死囚的故事,也不会被宣扬成旷世难遇的德政标杆。现在赵顼想减少,王安石不觉得要在这件事上,违逆天子,能少杀人当然是好事。至于,死里逃生,不觉得他们还敢有什么胆子乱来。

韩冈见王安石的样子,明白此事应该是定下了,就等两三个月后,赶在各路提点刑狱司上缴冬至大辟的名单之前,将之公布天下,以此来作为天子的德政。

即是如此,韩冈也就笑了一笑,不再谏阻:“也罢,那等死囚即便想作乱,熙河路上还有三尺钢刀给他们预备着。大不了杀一儆百,相信都下得了手!”

又说了一阵话,王旖被吴氏送了出来,洞房不留空,就算是回门的日子,也不能留在岳家过夜。不知王旖在内间说了韩冈什么好话,吴氏看着韩冈这个女婿,满意的不得了,晚上一家人吃饭的时候,不停的让下人将好菜往韩冈这边送。

吃过晚饭,韩冈和王旖向王安石夫妇告辞,回返家中,等着这对新婚夫妇的自然又是一个满是柔情蜜意的夜晚。

