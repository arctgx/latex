\section{第33章 物外自闲人自忙(六)}

童贯大着胆子插了一句话,终于得到了一个在天子面前露脸的机会。

给宰相贺诞辰,这算不上是什么难得的差事,也就油水丰厚一点,胸有大志的童贯心思还没有放在这些阿堵物上。但明面上的差使之下,还有一个秘密的任务,这一点才是体现了天子对他的信任。

童贯心花怒放,不过他的面上却是严肃端正,一丝不苟的跪下来叩头领命,从态度上,半点也不见得天子重用之后的兴奋。

这一点让赵顼看着他的眼神,更添了几分欣赏,他过去可见过不少次,宫内宫外的臣子、内侍遽受重用,拜谢时连手脚都忘了怎么摆了。宠辱不惊总是难得的优点,而且童贯沉稳如此,当是能将差事办好。

童贯将自己学到的宫中礼仪施展得十足十,同时尽量保持着冷静的心态和谦卑的神情,他很清楚,越是在这个时候就越是要保证自己能做到全始全终,不让天子有一丝不快。对于像他们这样在宫廷服侍皇室的内宦来说,一辈子能撞上的机会也就那么一两次,如果把握不到,那就在宫里面伏低做小个几十年好了,永远都别想得到一个官身,遑论转为武职。

宦官有属于自己的内侍官阶,从无品级的贴祗侯内品,到从八品的内东头供奉官,总共十一阶。到了内东头供奉官之后,内侍再想升官,就会转为武职,从此归入武班,也就是说高品的宦官能出掌军职,领军作战是有所凭据的。

不过这么一来,宦官升到高位之后,就会受到政事堂和枢密院的制约,不可能再出现晚唐时凭心所欲废立天子的‘定策国老’,也就不会出现所谓的‘门生天子’。

童贯一心的就是想在边疆建功立业,继而得到天子的信任。童贯时常幻想,如果自己能有秦翰的武勇,他老师李宪的军事素养,再加上如今即便不在宫中、却也最受天子信重的王中正王大珰的运气,日后必然少不了一个节度使。

宦官追封节度使,不是没有先例,童贯眼下最大的梦想就是节度使,即便是追封都是好的,至于目前,则是希望能在天子面前继续得到任用,等到地位高了之后,到时候也能收几个弟子,收一个养子,在宫外再收养一个能传宗接代的儿子,这样也算是弥补了自己的缺失。

童贯领了圣旨,从崇政殿中退了出来。天子已经将为富弼贺寿的圣旨写出了文字,等两制官将赵顼的草稿加以润色,再经过政事堂的检查之后,就可以传回到赵顼的手边,让童贯带着礼物去洛阳。

虽已经还没到傍晚,但天色已经黯淡下来,童贯抬头看着渐渐爬上殿顶的一轮满月,当这轮月亮变得只剩一半的时候,就是富弼的生日了,自家也当已经身在洛阳城中。

……………………

一场惊动京城天子的风波,洛阳城中自然也不会那般容易就停息。

韩冈在拜访了河南府,按传言说是黑着脸出来之后,又在衙署中安安稳稳的处理了几天公务,并没有再去拜见其他致仕的老臣,更没有去找文彦博的麻烦。

韩冈之前当先拜访文彦博,只是因为文彦博是判河南府兼西京留守,是属于公事上的往来。而其余重臣,皆已致仕,拜访他们则是人情上的交往,必须在公务出来完之后——至于二程,则是因为与韩冈有师生之谊,天地君亲师,师排第五,例外一下,没人能说不是,反而要夸韩冈尊师重道。

只是韩冈变得沉寂下来,对于洛阳士民来说,就像好书看到一半被人打断一般,下面会怎么发展,看客们都没有能得到满足。弄得洛阳士民心里如同塞进十几只耗子,在里面抓挠着,心里面一个个焦躁无比——尽管这番争锋都跟他们毫无牵扯,但能看到两名重臣的争斗,对于难得有娱乐活动的时代,却是比起刚刚在洛阳兴起的蹴鞠联赛,还要让人感到迫不及待。

眼下都二月中了,一年一度的洛阳牡丹花会也即将开展,可文彦博和韩冈的府漕之争,却让元丰元年的牡丹花会,一时间,失了许多颜色。

魏紫、姚黄,这是人人都知道的名品,如今也算不得什么了不起的了,往年到了现在这个时候,都会有几本独家拥有的新品牡丹的消息传出来,唯独今年与往常不同,一点消息都没有,只有文彦博和韩冈之争,在西京的酒楼茶肆、脚店客栈中,议论的最多,议论得最广,眼下还是受封潞国公的文彦博,和龙图学士韩冈。

“这韩龙图到底是怎么想的?都几天了,也该给个明白的说法,这样吊人胃口不是事啊!”有人这么抱怨着。

“谁能知道他在怎么想?以韩龙图的身份,根本就不需要太给文相公的脸面。接下来肯定会有好戏看。”有人如此期待着。

“韩龙图应该会上奏天子,让官家给个公道。直接上门应该不会了,想来他不会再让人给赶出来。”有人如此确信着。

但韩冈根本就不理会外面的传言有多么让人瞠目结舌,也不去理会外面的那些闲人对他的期待有多么无稽,他现在都在想着儿女上学的事的烦心。

韩冈家的老大和在家里最受宠爱的女儿,现在都已经六岁了,已经到了该读书的年龄。但韩冈不想将他们送进蒙学之中,而是希望自己的妻妾能在家里为他们打好基础。

论及家学渊源,韩冈不如王旖,论起琴棋书画之类陶冶情操的艺术,韩冈又不如周南。有她们两人做蒙师,加上自己家里还养着一群同门,完全可以从中优中选优,让他们帮着加强教育。

但韩冈也不会将责任全都推给王旖她们。韩冈有心为子女编写一部蒙学的教材,不管怎么说,后世的教学有着十几亿人作为证明,这个时代的蒙学教材可是远远比不上后世之万一。

韩冈已经将大纲和章节全数罗列出来,关于算学的前几章也已经写好了,不过想要推广和代替就有教材还是很麻烦,毕竟区区一篇千字文,文字上都是经过千锤百炼,不是韩冈凭着记忆闭门造车所能比。

王旖在看过韩冈的草稿纸后,也明显的不感兴趣,她的丈夫写得太粗率,文字上缺乏精雕细琢,连半成品都算不上。只见她放下草稿,柔声劝道:“官人,文相公那边再这么继续闹下去也不好,也该给个说法了。听说昨天在漕司之中有人议论此事。官人你亲口对人说,当时是自己主动告辞,如今文潞公深受污名,非己所愿……”

韩冈看看王旖,想了一想,点头道:“的确是该给个说法了。”从书桌上拿出惯用的纸笔,让书童帮着将墨给他磨好,韩冈随即在纸上刷刷刷的飞快的写了几行字。打好草稿,就拿着笔在上面点点划划起来。

王旖看了草稿一眼,立刻就吃了一惊:“求见潞国公?!官人你还要再去见潞国公?”

韩冈的态度还是依然故往,平静带笑的点着头:“为夫的确是打算再去见潞国公一面。”

“还是有怨气?”王旖小心的问道。但她看着丈夫的眼神中似乎又有几分释然。

如果一点怨恨都没有,要么韩冈已经修炼到了宠辱不惊的程度,对于受到的羞辱毫不在意;要么眼前的一切就是他一手造成的,所以早已有所准备。无论是哪种可能,都会让王旖觉得她的丈夫未免太可怕了一点——幸好不是这样,自己的丈夫虽然无意去陷害,还是留了些许怨恨,这才像个真正的人。

“怨气是肯定有一些的,但文潞公如今深受市井流言所扰,我再去一趟,就是帮他澄清一下传言。”

“文相公会不会生气?”在书房中一直保持沉默的韩云娘问道。

“我只要问心无愧便足矣。潞国公会怎么想,我也无法约束得了他。”韩冈摊摊手,笑着表示自己的无奈。

其实韩冈抵达洛阳以来,他做的每一步都完完全全符合正道,全都让人无法指摘。他不打算改变这一点,今日要做的事,当然也是要做到问心无愧。

而韩冈的妻妾只会偏向她们的丈夫,这一次的事,要错也是文彦博有错在先,要不是他没有依照礼数派人去为韩冈接风洗尘,世人又怎么会误会他将登门拜访的韩冈从府衙中赶出来?对于文彦博,韩冈一家都没有什么好感。

写好了信,韩冈又从头到位的查看了一遍,确定了文字上没有半点疏忽,韩冈便收拾了一下,将信纸装进信封,唤了一名老实听话的仆役让他送去河南府衙。

“也不知道文潞公能不能接受,说不定看到信就撕了。”韩冈对妻妾笑道,“不过不论是他接受还是不接受,为夫都能安心了。为朝廷做事,能做到问心无愧这四个字,也算是没有缺憾,不会有任何问题。”

