\section{第18章 诸士孰为佳(中)}

策问。

考验是进士的眼界、见识、才学,以及文笔。另外在文字上,对于建言轻重程度的把握,也很关键。换个简单明了的说法,就是要会揣摩圣意,文章需要深刻透彻,但不能说出过重的话,否则就会变成一个悲剧——现在站在陛前的王安石,就是现成的反面教材:

王安石庆历二年参加科举,就是在殿试的考卷上,写下了‘“孺子其朋’这四个字。一个二十二岁的年轻人,竟然把周公教训周成王的语句写进试卷中,来教训已经做了二十年天子的仁宗皇帝。自然不会有好结果。仁宗皇帝亲笔一划,到手的状元飞了去,只得到了第四名。

有了前车之鉴,后来的士子都已学着不再去犯这等蠢事。天子让你畅所直言,却也不能当真直言无忌。想喷皇帝一脸口水,博个直名,等拿到进士头衔再说不迟。

韩冈自然也不会去犯。看到今次的考题,他已经完全放下心来。

此前对殿试考题的揣测,在大方向并没有错,把握得很准。而天子的心意,通过王韶、王雱这两位重臣和近臣,韩冈已经有了深入的了解。对此作出的应对,是显而易见的充分而完备。

在礼部试结束之后,为了准备殿试,韩冈已经作了八篇的模拟作文,针对预计到的可能情况,作了不同方面的论述。其中有一篇正好跟今次的题目相吻合,而其他几篇,也都有可以用来参考和借用的词句。韩冈要做的,就是将之默写下来,稍加修改而已。

韩冈胸有成竹的振笔疾书,草稿纸上,转眼就出现了一行行墨迹淋漓的小字。

下笔畅快如此,实是平日作文时难得一见的情况。乃是准备已久的文章,自家修改多次,又经过王韶的修订,书写起来自然不会有半分滞碍。可即便如此顺利,韩冈也没有去争夺前几名的想法。

天下聪明人数不胜数,能从百万士子中杀出来的集英殿上这四百余人,眼光长远的也所在多有。韩冈很清楚,不仅仅是自己能猜测得出今次的考题。四百零八人中,至少有十分之一能事先推断得出同样的答案。至少叶涛,王安国或者是王雱,都不会忘了跟他提上一句。

但韩冈很安心,只要注意不要将大宋历代天子的名讳带出来,就不会有失败的危险。进士已经到了手中,区区名次而已,何须他孜孜以求。

赵顼在殿中慢慢走着,只有王安石和李舜举跟在身后。

见到天子过来,考试中的贡生要起来行礼的时候,便会被赵顼所阻止。他是来看考生应考的,不是来打扰考试的。

赵顼的视线在一份份卷子上掠过,只要上面有让他眼前一亮的词句,赵顼就会稍稍停步,记下这一个考生的姓名。

从前到后,又从后走到前,在考桌前后的空隙中,天子、宰相悄无声息的踱着步子。

赵顼又一次停下了脚步,前面正在奋笔疾书的贡生很是年轻,但他写好的那部分文章,却是能让天子为他驻足。赵顼在他身后看了良久,别的故且不论,单是满篇华彩的文字就很让他很是喜欢。

留意了一下籍贯和姓名。

龙泉叶涛。

赵顼默默念了两遍,这个名字,他将之记了下了。

重新起步。赵顼又向后走了两排,只是这几十名考生中,再没有像叶涛一样让他眼前一亮的,但他还是停步了。并不是为了方才经过的那些个考生。

就在赵顼的左手边,有一名贡生,在蒲团上跪坐得笔直。眼神专注于纸笔之上,肩张背挺,身形气质看着就不同于其他的士子。

赵顼回头看了看自己的宰相,王安石明白他的意思,点头回应,轻声道,“就是韩冈。”

这个就是韩冈!

虽然是跪坐着,但他的身形气度,在周围的一圈士子中依然是如鹤立鸡群一般。从侧后方看去,只能看到宽阔的肩膀,还有挺直的鼻梁,另外就是展在桌案上的试卷。

雄壮的身材,端正的书法,坐在集英殿上的韩冈,在赵顼眼中,的确是个文武双全的模样。

对于韩冈的形象,赵顼满意的点了点头。看着他一直渴求一见的臣子,正心无旁骛的笔走龙蛇。

韩冈在贡院中一直拖到最后才交卷的事,赵顼也听说了。明白韩冈并非是七步成诗、落笔如江河的捷才,而是喜欢深思熟虑、揣摩再三的士子。现在写起来如此顺畅,当是灵感来了。赵顼也经历过这等文思如泉涌的时候,此时若是被打扰到,断掉的思路多半就接不下去了。

赵顼不想打扰到韩冈的行文,只准备看上两眼,就打算离开。但视线落到试卷上,两脚便迈不开了。一直站了好一阵子,从头到尾的将已经完成的部分看了两遍,才慢慢的又点了点头,回头对王安石低声说着:“果然不错。”

王安石在后面也看到了韩冈的文章,却是在摇头:“文字尚有待琢磨。”

殿试的交卷速度,要比礼部试快上一点,不管怎么说,也没人敢让天子等到三更之后。

今次殿试,开始的早,结束的也早。

到了午后时分,天子已经转回到后殿休息,而最后一名考生,也终于交上来自己的试卷。

接下来,就是批改的工作了。

以曾布、吕惠卿为首的知贡举的那批考官,并没有出现在殿试上,而是由赵顼另外任命一批官员,担任考官——详定官、编排官、弥封官。

殿試审核之制,与礼部试差不多一样,仅仅稍有区别。

应考举人交卷之后,先交付编排官,去掉卷首姓名籍贯,改以字号数字来排列。然后给弥封官,指挥三馆书吏誊抄、比较。接下来,交付考官定等,再次弥封后,交送覆考官再定等。前后定等完毕,最后交送详定官启封对照考官和覆考评判的异同。详定官最后确定下名次,将试卷誊本重新缴还给编排官,揭开籍贯姓名,与本卷中的字号对应,将确定下来的名次,呈递给天子。

——到了这一步,基本上就是礼部试的翻版,大同小异。但接下来就不同了,因为评判出来的结果要交给天子审核。这一事,就会改变进士们最后的排名。

到了快入夜的时候,赵顼给殿中等候结果的考生们赐了酒食。听着前殿的谢恩之声,今科进士名次的榜单,连同考生们的试卷正本,一起呈到了赵顼的面前。

余中、朱服、邵刚。

这三人是考官们定下的前三名。

赵顼将三人的试卷找了出来,文字和内容都算是出类拔萃,这个排序并没有问题。

接下来,赵顼看着后面的名单,一直看到了二三十位,也没有看到他方才关注的两个举人的名字。着意找了下叶涛,竟然被放在了第五十六名,归属第三等。再看看韩冈,则更惨一点,却是第三百八十四名,排在第五等,几乎是最末了。

看着自己看中的贡生,竟然被放到了如此之后,赵顼只觉得自己的眼光被侮辱了,心头便多了几分不快。

让李舜举在一摞四百张的卷子中,找出两人的考卷,赵顼便聚精会神的看了起来。

叶涛的文章很好。赵顼方才就是因此而停步良久。只是现在看起来,内涵的确是显得空洞了一点,没有说到多少实在的东西。所以被置于第三等,这不算考官的错。可赵顼又读了两遍试卷上的的文章,感觉仍是很喜欢,直接用朱笔抹去了试卷一角上的‘五十六’,改写了个‘九’上去,将之提到了前十名中。

相对于叶涛,韩冈的情况就正好相反。文采只能算是中平——不过比起前日赵顼特地要来的韩冈在礼部试上所写的史论,还是要强上一些——但每一个段落,每一句话,甚至每一个字,都紧紧扣着题目。

新法推行几年来的施政利弊,尤其是在陕西一地推行情况的论述和评价,可以说是一针见血。与对党项和吐蕃的战事紧密相连。没有哪一位陕西来的贡生,能有韩冈这等深刻的手笔。也没有一个来自于其他地区的考生,能对他们所了解的当地情弊,说得如韩冈一般通透。

赵顼明白,韩冈毕竟不同于其他考生。参加过横山攻略,参加过咸阳平叛,并且是从头到尾的经历了河湟开边的一切艰难困苦,更是枢密副使王韶,在熙河路上最为重要的助手。经历之丰,在他这个年纪,当世已是无人能及。

站立的角度不同,看事的眼光不同,行事的经历也不同。赵顼看着韩冈的这份卷子,感觉着完全不同于其他士子的文字,哪里是考卷,分明就是一份来自于陕西边地重臣的奏疏。

平实、直观,让天子看到施政上的弊病,同时给出的意见又能让朝堂很容易的作出相应的解决方案。

这样的奏疏,也只有精于政事的名臣才能写得出来。

赵顼平日里也喜欢翻看旧时名臣的奏章,韩冈今次的策问,比起那些名臣,也只是在文字上有所欠缺而已。

所以韩冈的名次才被评得这么低。类似于奏疏的风格太过于特别,当然不会受考官们所喜。但赵顼不同,他所处的位置,让他与考官们看人看事的角度就不会相同。

他们不喜欢韩冈的这篇策问,但赵顼就十分喜欢。

所以,韩冈的这个三百八十四名,就只有让排在前面一位的慕容武降下来填上。

