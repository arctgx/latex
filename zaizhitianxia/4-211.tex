\section{第33章 物外自闲人自忙(七)}

昏暗的灯光摇摇晃晃,投在地窖墙壁上的人影也是晃来晃去。

长宽皆不及一丈,高仅七尺,狭小的地窖中,只有一人一桌和排满墙壁的书架。

在污浊的空气里,盯着面前的书稿久了,纸页上的文字,就像是有了自由活动的生命,如同水里的蝌蚪一般游来游去。

司马光努力了半天,也没能看清稿纸上的下一句到底是什么。尽管他凭着记忆还能记得一点,但看不清文字,也就别想再写字了。

今天只能到这里了,司马光想着。在地窖之中,看不到时间,不过从地窖中空气的情况上看,也就两个时辰的样子。

年纪一大,眼神是越来越不济。

编纂《资治通鉴》,司马光惯例是先排列从目,然后将找到的史料,按照纪年法将编纂出长编,而后再从中挑选合用的条目,并加以删改和叙述。数万卷的史料、几千万字的原本,都要靠着一双昏花的老眼来检定和筛选。

的确是用得过头了。

资治通鉴的主编拿下夹在鼻梁上的眼镜,用力眨了眨酸涩发干的双眼。就在编写《资治通鉴》的过程中,他从四十多岁意气风发的翰林学士,变成了如今坐在地窖中的垂垂老者,眼见着转眼就要六十。

年过花甲啊。昏黄的油灯下,司马光无声的笑着。这十年他究竟是怎么过的?!

写书本也需要一个安静的环境,司马光前几年在园中挖了个地窖写书,被人当做奇闻异事来宣扬。但司马光之所以躲在地窖里,一个是因为里面冬暖夏凉,另一个就是足够清静,清静得足以让他抛却所有让人心如火焚般的煎熬。

重新戴上眼镜,亲手收拾着桌面,将今天书写和校对过的稿纸全都分门别类的放好,又慎而又慎的将眼镜拿下来,放进一个填了丝绵麻絮的小盒子中。

水晶眼镜的确是个好东西,司马光自从拥有之后,就当成宝贝一般珍视。虽然用得时间长了,眼睛就会变得很难受,但比起旧时他用得放大镜,仍要方便不少。

就像治病要对症下药,这眼镜也同样要看人来配带,有近视镜,有老花镜——这两个名字似乎是韩冈所起——不但人人不同,就是两只眼睛的情况也不一样,要找到一个合适的镜片,就要以一片片的去试。

如今的东京城,公卿们要选用眼镜,都是从几十片磨制好的镜片中,挑选出合用的,再让匠人为镜片打造合适的框架。有夹在鼻子上的,也有架在耳朵上的。

司马光这副眼镜是两年前由天子所赐。当时他向天子禀报说,受两代帝命而编纂的《资治通鉴》已经修成了一百七十多卷,天子赵顼闻之欣喜,赐下了一批财物,其中就有这副水晶眼镜。这自然与司马光视力配合不上,只是能稍微改善一下而已。儿子司马康倒是建议换上一副更合用的,但去东京城配镜并不现实,而且价格未免太高了一点。

用着如今风靡天下士绅的眼镜,司马光也不禁要赞一句王安石的女婿本事当真不小。

从地窖中拾级而上,推开一扇低矮的小木门,扑面而来的清新空气让人为之一振。尽管下面的地窖不是没有开辟通风的出口,但在里面待得久了照样还是憋闷。

“君实秀才,今天这么早就上来了?”

自幼侍奉司马光的老仆吕直就守在地窖门口,听见里面的动静,就立刻从小杌子上站了起来。

“早?”司马光抬头看着天色,在阴暗的地窖里坐得久了,夕阳的阳光依然显得分外刺眼。现在鲜红的落日还没有完全沉到西面的群山下,“还不到酉正?”

“快到了。”吕直立刻回道,“君实你下去有一个半时辰了。”

比起预计得还要早,司马光心情差了一点:“有没有客人来?”

尽管士大夫之间正常拜访,都会先写一封帖子,确定时间,但总有例外的,司马光并不是多问。

老仆低头回道:“刑秀才来了,正和大郎在棣华斋里说话。”

“刑和叔来了啊。”

独乐园在司马家宅院的东侧,一汪池水中有一坞榭名为柳坞,一座小桥连接于岸上。东南是巫咸榭,正对着巫咸山。巫咸榭后是赐书阁,一时间用不上的书籍都放在里面。司马光的住处是在园中主阁东侧的小阁中。

司马光原本是要去午睡的,不过他听说了刑恕来访,便转头向外走。他弟子门人读书的地方便是外面的棣华斋。刑恕是他的门人,要不然司马康也不会在棣华斋接待他。

离开看不到名木名花的独乐园,司马光往着前院走去。棣华斋中并没有什么人,只有两个熟悉的声音从小楼下的厅中传出来。

“韩冈这一手当真是出人意料!”

“该说是绝妙,潞国公没给气中风就算好了。”

听到了儿子和刑恕正在高谈阔论着什么话题,司马光又暗道一声,王安石的女婿本事当真不小。

闹得洛阳沸沸扬扬的一桩新闻,司马光再是躲在地窖里,也不可能茫然无知。对于这一次的事,起因自然是文彦博做得差了——司马光并不怎么欣赏文彦博的穷奢极侈,从性格上两人并不相和,只是有共同的政治对手而已。

司马光不会偏向文彦博,但之后韩冈的行事,虽然从道理上挑不出毛病,也没人能指称韩冈哪里做得错了——韩冈甚至已经对外宣称是他主动从府衙中告辞,试问他哪里还有错?!

但看实际的结果,司马光就觉得韩冈是有所欲谋的。这么多年、这么多事的看下来,司马光早已明白,王介甫的那个女婿,可是聪明绝顶的人物。

刻意放重了脚步,里间的谈论立刻停了。当看到司马光出现在门口,司马康和刑恕都站了起来行礼。

“和叔来了。”司马光平平和和的说了一句,在座位上坐下来,一杯茶水立刻就递到了他的手边。

司马光喝了口茶,漫不经意问道:“在说什么呢?”

“还是潞国公和韩冈的事。”刑恕回道。

“又出了什么事?”司马光问着,前面刑恕说韩冈做得绝妙,又说文彦博会气得中风,倒让司马光好奇韩冈又做了什么。

“韩冈早间递了帖子去河南府,说是要明日拜会潞国公。”

司马光皱眉:“明天?”

“就是明天!”刑恕用力的点头道。

“好一个韩冈!!”司马光板起脸,摇着头,为文彦博的处境长叹一口气。

身为前任宰相的元老重臣不是想拜见就能拜见的。人家也忙,呼朋唤友、吟诗作对,邀风赏月,什么五老会、同甲会,都占了文彦博日常大半的时间,偶尔他还要处理一下公务,尽一尽判河南府兼西京留守的义务,哪可能是一个‘小小的’都转运使想见就见的?

韩冈第一次拜会文彦博,那是公事,文彦博前面做得错了,只能给他一个面子。正常想要再登门,先去排队去吧!文彦博预定的行程中,来往的朋友身份都不低,全都是熬老了资历,用了几十年的时间将本官的品级升到了三品、四品、五品,不会为韩冈一个年轻后生让路。

只是眼下遇上这件事,韩冈说是明天上门,文彦博就必须留在家里候着。因为他上门是帮忙澄清之前文彦博受到的误会,人家给了这么大的面子,文彦博别说不见,就是见得迟了,他的名声就会更差上一分。

“所以学生才说,文潞公这一次肯定会被韩冈气得不轻。”刑恕摇头苦笑,似乎是对文彦博处境深有感触。

“但他这么做,外人看来是帮潞国公解围了。宽容大度,乃是难得的君子。如果敢说韩冈不是,那就是小人之心度君子之腹了。”司马康道,“方才和叔还说,他刚刚从程府过来,程伯淳也听说韩冈要去,还赞着他器量难得。”

刑恕也是二程的门人,为他们辩护道:“伯淳和正叔两先生一向忠厚,不识诡道诈术,加上韩冈又善于伪饰,故而为其所欺。”

“……谁让文宽夫有过在先。”司马光为文彦博感到遗憾,当真是老糊涂了,要是在几十年前……不,就是十年前,文彦博都不会犯这种错,“韩冈此子奸狡诡谲,外示朴厚,内含诡诈,文宽夫一时错失,就给他抓到了机会。”

“但潞国公依然得承他的人情,日后也不便再与他过不去了。”刑恕说得有几分义愤填膺,但他私心里却是佩服韩冈的手段。

轻描淡写的就将文彦博的气焰给打压下去,完完全全合乎正道,不见一点烟火气。堵得堂堂潞国公有口难言,真的就是方才他跟司马康说得,没给气得中风就是好了。

接下来韩冈去南面主持襄汉漕渠的修造,洛阳这里要是敢在钱粮上拖一下后腿,文彦博的老脸也不要见人了。

“明天潞国公见韩冈,至少要坐上一个时辰,才能洗掉外面的传言。”刑恕摇起头来似是在叹息,却透了一分幸灾乐祸出来,“这一个时辰,可不好待……”

