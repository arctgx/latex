\section{第30章 随阳雁飞各西东(八)}

身后一片沸腾,种朴低着头奋力挥鞭。

种朴带着三百余骑宋军骑兵,连连抽打着胯下的坐骑,拼命的向溥乐城逃窜回去。

数以百计的辽军骑兵奔驰出营,衔尾追来。

同时追上来的,还有一支支长箭。

宋军骑手们的背甲或头盔上,时不时的就有叮的一声响,一击冲击便随之传来。

不过一应坐骑都披挂着防箭的厚毛毡,而骑兵们也都好端端的穿着战甲,黑暗中的马弓更是失去了绝大多数的准头。箭矢虽急,却也没有产生多大的效果。很快,后面的辽人就只是零星射上一两箭,一心一意的追击。

种朴左颊一阵阵的抽痛。方才他在箭雨中回头招呼麾下将士时,嘴里突然间就多了一股浓浓的铁腥味,还有个异物贴着舌头,似乎是中了箭、受了伤。但偷袭城外的敌营失败,被一路追杀。在他的心头上,更重的还是计划失败的挫折感。

辽人的军营不好踹,但党项人可就容易对付多了。聪明的党项人全都换了装束。但他们身上的臭气改不了。何况还有更多的连装束都没改的党项人在?

种朴的目标就是他们。

党项人不可能为辽人卖命,国破家亡的现在,已经失去了过去与官军大战数百回合的锐气。这样被强逼上战场的军队,一旦在夜中被突袭,不是炸营,就是崩溃。而少了数千党项人,再想要攻打溥乐城,辽人就要用自家人的姓命去填城墙外的壕沟。

但种朴看到的问题,辽人也同样看到了。

党项人在溥乐城外的军营,就是摆在陷阱上的诱饵。

刚刚走到党项人的营寨边,连绵数里的七八个营寨,须臾片刻之间,便一下变得灯火通明。

奔逃中,种朴心中大呼着侥幸。幸好今天是阴天,幸好身上的装备精良,幸好带出来的都是精锐。

而最大的幸运,就是他出来前已经做好了最坏的预计。在彻底落入陷阱之前,终于反应过来,并及时脱身。

城墙就在眼前,城头上火光缭乱,背后的追兵却仍紧咬不放。越来越重的蹄声,显是没有放弃的意思。一旦溥乐城为了救回自家人而打开城门,他们就可以趁势杀进城去。一击破城的机会,契丹人当然不会放过。

跳动的心中只剩下兴奋,契丹骑手们的呼吸跟他们的坐骑一样激烈。再有半里就能追杀入城,宋人积攒在城内的物资便能尽数收入囊中。这可是宋人在年节前的积存,收获远比平曰里更加丰厚。

但就在此时,军鼓声在前方城头上猝然响起,鼓声中,紧追不舍的辽军突然一骑骑的莫名翻倒在地。

当领队的将领从热血沸腾的追击中冷静下来,这才发现就在前方不远处,有一支人数颇众的步军,正在向他们急速射击。城头上火光太过眩眼,让人根本看不清前方的黑暗中到底有多少宋人,能看到的只是一片密密麻麻的模糊黑影。

嗡嗡的弦声细密如雨,整齐有序。

这是陷阱!

半刻钟前,才出现在种朴等人脑海中的判断,现在又出现在辽人的心中。不过比起种朴,他们的运气就差了很多。

这不是力道只有六七斗的马弓长箭,而是神臂弓以数百斤力道射出的利矢。

锋锐,犀利,充满力道,而且有着更强的准头。

城头上一片亮光,城上的宋军士兵不可能分辨得清在城下的黑暗中奔驰的骑兵,纵使六具八牛弩都架上城头,也抓不住射击的目标。

但同样身处黑暗中的这两个指挥近千人的神臂弓手。他们就在城外,背对着城头,双眼早已习惯了黑暗。在他们眼前,追逐而来的双方骑兵,人和马都映照在来自城上的火光下。

箭矢落处,人惨嚎,马惨嘶。

一名名辽人军官在箭雨中,用着契丹话大喊着,试图收拢麾下的兵力,但他们随即就成了最为显眼的靶子,被乱箭射下马去。

追击种朴的契丹骑兵几近千骑,与守在正面的宋军弓弩手人数相当。可在长达两里多地的追击中,已经给拉出了一条长长的队列。追在最前面的两百多精锐,成了近千柄神臂弓集合打击的目标。最开始的一波箭矢,就将这两百多精锐骑兵射落了近半,让他们失去了秩序,更让被堵在后面的骑兵没有了一开始的冲击力。

从追击敌军,到被敌军伏击,这个转换过于剧烈,慌乱也随即传染开去。原本还有可能奋力一搏,挽回败局,但人心一乱,领头的军官又被射杀,就再也没了机会。

这是种朴事前埋伏在城外的两个指挥,并不是他的先见之明,而是他的参谋们对偷营计划补完后给出的意见。

当年在罗兀城,种朴就觉得韩冈提出的参谋共议的制度很不错,虽然事后几乎所有的西军将领都觉得这是多此一举,没有再延续下去。但种朴自**领军后,却在自己的麾下挑选了一些精明能干的军校,让他们共参军议,并处理军中庶务。

今天种朴决定了出城给辽人迎头一击,剩下的具体方案,便交由这些军校来完成。用城中唯一的一个骑兵指挥去偷营,同时准备两个步军指挥出城做接应。并不算很复杂的计划,却带来了挽回颜面的回报。

身后的追杀硬是被神臂弓截停了,种朴随即放缓了战马的速度。回头看看已经乱作了一团的追兵,精神一震的溥乐城主,随即便带着骑兵们如狼似虎的又返身杀了回去。

这是一个漂亮的反败为胜,千余追兵被打得四散而逃,也不知有多少人在黑暗中坠马受伤。

不过当辽人后续的援兵赶来,种朴也知道见好就收,及时的率领在城外的马步军,陆陆续续退入城中。

种朴选择了在后半夜黎明前,人们最为困顿的时候出兵偷袭,当他退回城内,东方的天空,已经泛起了微微的亮光。

“城主!”

“巡检!”

“十七郎!”

一看到种朴的模样,围上来的一群人脸色陡然变了,慌乱的大声叫着。

“仇老在哪里?!”

“还不快去医院请仇老来给城主医伤!”

面对围上来的部下,种朴想捂住脸,今夜的一战简直丢人现眼,幸好最后捞回了一点老本。但他的左颊上正插着一支长箭,却是怎么也捂不住。方才在城外厮杀时,种朴完全没有感觉到什么异样,但现在一歇下来,一阵阵的抽痛便让种朴坐不住,也站不住。

名闻关西的老军医仇一闻很快就赶来了。他头发胡子全都白了,可精神却好得很。明明前些曰子,在韩冈和他弟子李德明的举荐下,朝廷赐了一个官身。但他还是在关西各路的军营中到处游走,不愿接受轻松一点的差事。这些天,正好逛到了韦州这里。

对于这位行医几十年的老军医来说,如何处理箭创,就跟吃饭喝水一般简单。拿着钳子将箭杆贴着肉夹断,手指探进种朴张大的嘴里,攥着箭簇一拔。随着锈迹斑斑的箭簇带着血水一并涌出,剩下的就只需要清洗伤口和缝合了。

脸颊上的贯通伤火辣辣的疼,种朴双手紧紧攥着拳头,指节发白,却是一声不吭。

仇一闻很满意种朴的配合,拿出一个葫芦,递给了种朴。

种朴接过酒葫芦,拔开塞子,浓烈的酒香立刻散了出来。

周围的士兵嗅到酒味后,齐齐咽了一口唾沫。这是慢慢一葫芦的烈酒,而且还是极醇正的烧刀子。放在军营里,十个人里面少说也有两三个愿意拿半个月的俸料钱来换这一葫芦的美酒。但放在此时,则是用来洗伤口的。

“用来漱口,用力点,好消毒!”

仇一闻的吩咐,种朴不敢不从。大大灌下一口酒,只漱了漱口,一半酒水从创口中喷出,剩下的一半则噗的一口吐了出来。都是鲜红的,还有一阵钻心的剧痛。

“好痛快!”种朴咬着牙大叫道。

“再来。”仇一闻逼着种朴再继续。

一葫芦烈酒漱口,吐出来的酒水中血色渐渐的就淡了。

“仇老,城主的伤可还要紧?”一名种朴的亲信紧张问道。

种朴听着就不痛快了,“不就是中了支箭吗?多大的事,蚊子叮了一下而已。”

“别动!别说话!”仇一闻用力拍了一下种朴的脑袋,毫不客气的教训道。

仇一闻的江湖辈分极高,甚至还跟种世衡那一辈的西军将领们打过交道,种朴一个后生晚辈,就是靠官位都摆不起谱,只能老实听话,不敢再乱动。

须发皆白的老军医带上老花眼镜,拿着一只放大镜,仔仔细细的查看着种朴脸上的伤口,最后松了一口气:“还好没伤到大血管,缝起来上了药就不会有大碍了。就是伤口长好之前得忌口。”

招了副手拿了消过毒的针线过来给种朴缝伤口,老军医年纪大了,手不如年轻人稳定。

种朴身上套着一身将军甲,防护力远胜普通士兵使用的九件套的全身板甲,更不用说骑兵的半身胸甲,从头到脚都能防护到。如果每一件配件都装备上,除了眼睛以外,不露一丝破绽。

但他为了方便指挥,也不想拖累坐骑,只是装备了头胸腹背等几个要害位置上的部件,还将护面给卸了下来。在阵上运气极差的被一箭射穿了面颊,还带去了一颗槽牙。

伤口缝好后,种朴叹着气,“这下破相了。原本就比不过十九相貌讨人喜欢,这一回更差了五分。下回再同他去逛窑子,那些婊子都不带正眼看了。”

“窑子里面,有钱的就是祖宗。怀里揣个百十贯,我这老头子去了照样不缺人奉承。下次去,见人就打赏,看看你兄弟能不能比得过。”

“有这闲钱,还不如用来教训士卒呢。今天能一下射退辽人,可都是平曰练出来的功劳。”

“那就别抱怨了!”仇一闻说着,用棉絮沾了一种散发着莫名气味的黑色药膏,往种朴嘴里面塞,“膏药要贴着伤口,不要松开了。”

种朴乖乖的将药膏贴着内侧的伤口,一股清凉感从伤口处扩散开来,疼痛突然间就减退了许多。

帮种朴收拾好伤处,仇一闻收起药箱,让身后的小童背了,拄着手杖在副手的搀扶下往城下去。绝大多数的伤兵都在那里歇着。不过种朴要观敌情,没办法到随军医院中治疗,仇一闻只能上门看病。

种朴起身送行,顺带一脚踢起两名亲兵,“看什么看,还不去扶着!”

刚刚送走仇一闻,号角声便从各座城门处响了起来。

夜里的厮杀让辽人还是吃了一个不小的亏,终于忍不住要开始进攻了。

种朴几步跨到城墙边,望着辽人攻来的方向,城外旌旗招展,气势汹汹。真的是要进攻了。

“好!”种朴用力拍着雉堞,“就怕你们不来!”
