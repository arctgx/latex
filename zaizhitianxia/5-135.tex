\section{第14章 霜蹄追风尝随骠(19)}

夜色笼罩下的营垒工地中,依然灯火通明。

肩上近百斤的负担,空空如也的肚子,让萧海里步履维艰,踉踉跄跄的走在通向外围寨墙的狭窄小道上。

过去根本没有做工务农的经验,但十天来的磨练,让他即便还是摇摇晃晃,还是能将装满黄泥的担子稳定在肩头上。不过萧海里痛恨这样的磨练,痛恨这种被皮鞭和刀枪逼出来的本事。

明明是领着三百骑兵来投的酋首,在预计中应该是被宋人好生款待,甚至应该是奉承的对象,却被一体同仁的被发遣来挑土夯筑。

在进入宋军的营地后,武器战马都被收走,反抗者立刻被诛杀。在宋人的围困下根本就没有反抗的余地。

一阵阵钻心的疼痛让萧海里扭头看着挑着担子的肩膀。皮又破了,血水都渗了出来,已经被磨破的羊皮袄又粘上了新血。血泡破掉的位置,等长好后多半就是一层厚厚的老茧,就跟手上拉弓挥刀练出来的老茧一样。但当年磨出来的老茧让他欣喜和自豪,而肩膀上的老茧只会让他成为笑柄,就算回去后,也不能在老茧退去之前再光着上身。

不过只要能回去,丢脸就丢脸好了。

“磨蹭什么?!”

一声呵斥,皮鞭破风随即响起,萧海里背后便是火辣辣的一阵剧痛,痛得他他趔趔趄趄的连着冲了几步才稳下来。萧海里听不太懂汉人的话,但皮鞭和剧痛比什么言语更加容易让萧海里明白宋人监工心中的不耐烦。

萧海里狠狠的咬着牙,低头下去将担子稳在肩膀上。到底什么时候来救援的人才能来,早知道会有今天的情况,找个借口推掉这个差事。

三千多人被分成了整整一百队。只有每天完成最好、最快的前十队,才有饮食和劳役上的奖励,其余绝大多数的小队,全都是以不饿死和第二天能继续做工为基准,得到每日的口粮。

犯了错之后,直接被鞭笞至死,甚至拉到众人面前吊死和斩首的苦力,每天都有十几人。而想从这里逃跑,根本不可能。日以继夜的工作耗光了每一个人的体力,而周围监视工地的宋人更是一点破绽也不露,十几天来,出逃的上百人,一个不留的全都悬头于辕门前。

萧海里许多时候都在想,还不如饿死在大漠里。在这里为宋人修筑城寨,甚至比饿死还痛苦的。

在宋人压迫下,周围的党项人都是敢怒而不敢言,但越是这样严酷的压迫,爆发起来就会越严重,只要一个机会,洒下一点火星,便能将宋人专心修造的这间大营给烧起来。萧海里之所以能忍耐下来,正是在等着这个机会。

萧海里阴冷的视线绕过周围的监工,只要机会一到,这里的宋人,他一个都不会留。

只要再忍上一阵……

“再这样下去谁都活不了!!”

一声大吼,在前面离萧海里只有几步的同伴,突然间丢下手中的担子,只把中间的木棍拿在手中。左右一荡,便将靠得最近的两个监工挑飞到一边去。

“阿鲁带!”萧海里心中大急,那可是他的亲弟弟。都忍了这么多天,这时候忍不住,可就前功尽弃了。

“还等什么!?”阿鲁带反吼回来,“想挑土挑到死?!”

萧海里正想再说什么,十几名监工立刻熟练的围了上去。内圈的几人手持杆棒,外圈则是一张张扯开的神臂弓。

几支箭矢从周围监工的宋人手中射出,神臂弓巨大的力道在近距离轻易的将箭矢射进了泥地里,顺利的将几个蠢蠢欲动的苦力给吓得不敢动弹。

神臂弓就在身边,萧海里动也不能动,就算那是自家最为亲厚的兄弟,却无法伸出手去。就这样眼睁睁的看着他被乱箭射穿了身躯。

萧阿鲁带是难得的勇士,连生命也一起放弃的爆发却没有撑过片刻。挥舞着棍棒连着打翻了几个监工,但立刻便是箭矢齐发。在极近的距离上离弦的木羽短矢,完全穿入了他的身体中。

从弟弟身上流出来的血,一条小溪般蜿蜒到萧海里的脚边。低头看着脚尖上晕开的暗红,萧海里静静的站着。要不是还有一线脱难的希望,萧海里就要在这里与人拼个你死活我。

“他刚才是跟你说话吧?”两名监工拖走了萧阿鲁带的尸体,又一人站到萧海里的身前。

萧海里方才与自己弟弟的对话,不可能不引起宋人的注意,肯定要审问明白。

但萧海里动都没有动弹,只在盯着脚尖。地面上隐隐有着震动,细微得让人难以察觉。但在一直期待着援军到来的萧海里的感觉中,就如同晨钟暮鼓一般响亮。

终于来了。

萧海里惨然而笑,不知是悲是喜。他看着脚尖的暗红,明明就差这么一步了。

但机会终究是来了!

宋人这边有了反应,刺耳的号角声从高高飘扬在半空中的飞船上传了下来。有了天上的眼睛,任何突袭难度都高了十倍。

正在盘问萧海里的监工闻声立刻放下了一切疑问,其他监工也一下子紧张起来,仿佛听到阵上的鼓号。

正在工地上忙碌的苦力们,便在号角声中被驱赶进他们起居的专门营地。已经有不少人意识到必然是辽人来袭,只是却还没有一人领头站出来。

跟随在人流里,萧海里深吸一口气,身子紧绷起来,时间终于到了。

……………………

人马上万,无边无岸。

不过那还是以步兵为标准的说法,换作是骑兵,则只要一半的数目差不多就够了。从飞船上望下去,满坑满谷都是黑压压的辽人骑兵。

从北方向柳发川大营逼近的辽军骑兵,差不多五六千。

作为当世最强的骑兵,漫山遍野的压到了尚未完全完工的营寨之前,压迫感远不是区区党项可以比拟。

从武清军到柳发川寨,只有不超过四十里的山路。自武清军南下,只要半天就足够了。仅仅经过几十里的奔驰,对于惯于苦战的骑兵们来说,正好是大战前的暖身,是战力提升到最高的时候。

但身处柳发川营寨中的折可大和折克仁,在两人的脸上却看不到半点惧色。

“萧十三是不是糊涂了?我们占着地利来防守,什么城寨守不住?区区的五千兵马,也想来攻城?”

“辽军主力的目标看来是暖泉峰。对柳发川这边的攻击,应该是个幌子。”

“那就不要向胜州请援了?”

“按既定的方略去做就行了。之前在胜州已经定下了,就照着去做好了。”

折克仁、折可大之前都奉韩冈的命令,参与了作战计划的讨论。针对契丹人可能会有的进攻方向,心中都有数。而在战前胜州的谋划中,绝大多数辽人可能使用的策略,都进行了预测,然后做出了应对的方案,让统领各处大营的将领都知道该怎么去做。

折克仁正要传令给下面的士兵,但就在已经有了大致轮廓的城寨一角,突然间一片火焰升腾,浓烟滚滚而起。

一名小校纵马狂奔而来,远远地便大声高喊,“十六将军!小将军!那些苦力叛乱了!”

“还要等你说?”折克仁居高临下的俯视下方的营地:“已经看到了。”

积蓄起来的怨恨突然间爆发出来,被圈禁在营地的党项人最终还是选择了反乱。

“竟然能忍到现在,看来黑山诸部被人从老家赶出来后,就没有了该有的锐气。”折可大更是看不起这些被辽人打得丢盔弃甲的废物,“恐怕只要半个月就能彻底的给解决。”

“由着他们乱好了。”折克仁冷笑道,“肯定是那一队契丹人先动的手,然后黑山诸部才敢跟进。”

折可大抓抓头:“萧十三不会以为三百人在这里做工潜伏的过程中,不露一点破绽吧?”

若是一人两人倒也罢了,三百人,伪装成一个部族,怎么可能一点破绽都没有。光是党项话就是一堆破绽了,还有做事的习惯,都不可能让人一点疑惑都没有。

而且之前已发现了两支辽人,他们虽然全都被换成了斩首功,可谁也不能说萧十三就派了总计七百人的这两支兵马。为了搜检可能会有的另外一支兵马,所有人都紧绷了神经,早早的就将萧海里的那一队给揪了出来。

清查出这一部的身份,折克仁并没有为自己的先见之明而得意半分。更重要的疑惑就在他的心中。既然辽人的内应在这里起事,那么柳发川便应该是他们的主要目标才是。

“不是暖泉峰!出现在暖泉峰那边的确多半是辽军的主力,但萧十三的目标,肯定还是在柳发川。他想里应外合!”

先兵压柳发川,然后主力出现在暖泉峰。让所有人都认为抵近柳发川的五六千骑只是个幌子。但实际上,真正的目标还是柳发川。

柳发川与武清军之间是一片山岭,中间的道路并不算很宽阔。对于更善于奔驰的骑兵来说,战场回旋余地并不大,五六千骑兵已经是绰绰有余了,再多了,也不过是轮番上阵而已。眼下只需要一锤定音,根本不需要缠战。

