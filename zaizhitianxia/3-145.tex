\section{第39章 苦心难成事(上)}

韩冈一声长叹。

除了天子,除了与契丹的争执,这句话不会有别的解释。

‘敌理屈则忿,卿姑如所欲与之’——

——怕契丹人说理不得便恼羞成怒,所以只能为了两国的和平安定着想,干脆从了契丹人的要求。

真是个绝妙的逻辑。

“韩琦要废将兵保甲,以释契丹之疑;富弼要天子含辱忍垢;文彦博倒聪明,没在奏疏中多说,别人都是长篇累牍,就他四五百字便交上来了,但也说了河北饥荒,难以抵御辽骑。”章惇的愤怒难以遏制,用力一锤桌子,正放在桌沿的银质雕花酒盏当啷啷的掉到了地板上,“自毁长城,示敌以弱,现在又‘姑如所欲与之’。妥协退让,能消得了辽人的贪心吗?”

“还说这些做什么?!”韩冈脸上挂着霜,声音也仿佛在冰雪里浸过一样:“契丹不会南侵,那一干元老哪个看不出来,明着欺君罢了!富弼竟然还说‘近闻陛下决为亲征之谋’,朝中有哪人说要天子亲征了?!张方平说宋辽大小八十一战,只胜了一次。他是板着指头数的吗?!”

“道听途说都不至于!”章惇狠狠的说道。

房间的门吱呀一响,酒楼的小二探头进来,他在外听到了房中怒气冲冲的声音,又听到了酒杯落地。但他一露头,顿时就是四道充满怒火的视线钉了过来,吓得他忙把头缩了回去。

韩冈满心的怒火过了半天也没有消散的迹象,只是怒极反笑,表面上已经看不出一点异状:“韩琦、富弼,他们回想当年为国奔走于辽宋之间,领军抵挡元昊叛军的过往事迹,不知还愧不愧!”

韩冈来自千年之后,不论再怎么争权夺利,营营汲汲,对国家民族的荣辱,总是在心中有一个位置。

来到这个传说中积贫积弱的时代后,除了早年签订的岁币、岁赐之外,他却从没有亲眼见过大宋对外卑躬屈膝的场面。而且看着皇帝,推行新法,又整军备战,的确有着振作之心。不论是在熙河路开疆拓土,还是在横山针对西夏人展开的攻略,虽然一胜一败,但都能从其中看到皇帝一扫积弊,改变对外军力不振的雄心壮志。

这一切,让韩冈认为后世的传说有所偏差。只是没想到他看到的只不过是个伪装,当今的皇帝,外面装饰得再漂亮,内里还是如同真宗、仁宗那般气短虚怯,契丹人只用了一句恫吓之言就将画皮撕了下来。

韩冈其实本也有了心理准备,毕竟前几月开始,就在闹着了。还与王雱一起商定了借机行事的战略。可是当真事到临头,还是忍不住心里的火气。

“本以为会拖过郊天大典之后,否则天子有何面目去祭祀天地及太祖太宗?没想到这么快就撑不住了。郊祀之中用掉的那些钱钞银绢,还不如拿出来犒赏军民,整修武备,如此才对得起太祖、太宗。”

今年是郊天之年。冬至日,天子率百官至东京南郊,合祭天地于圜丘。这是三年一次的盛典,是国家祭祀典礼之中,排在第一位的大典。在国事中,是重中之重。赏赐百官及众军,并大赦天下,通常的花费都要在三五百万贯。

韩冈言辞之间一点也不客气,甚至直接攻击朝廷大典,章惇却深有感触。他长叹着:“君忧臣劳,君辱臣死。天子受此奇耻大辱,大臣却坐食朝廷俸禄,岂有此理,当真是岂有此理!”

韩冈的心中完全没有章惇的这一等感慨。此时的士大夫,由于自幼接受的教育,或多或少都有那么一点忠君之心,但韩冈完全没有。原本他认为赵顼值得辅佐,几次相见,也算是留下了一些好感。可现在就要打上问号了。只是这个时代没有挑三拣四的权力,让他十分遗憾。

“天子乱命,丧权辱国。此非臣之罪,而是天子有过。”韩冈冷冰冰的说着。

“不管怎么说,愚兄都是要为此上书,而士林中必然也会有所应对。”章惇也不介意韩冈说的话,如今当面骂皇帝的多了去了:“到时候,清议一起,看看韩缜、吕大防他们有哪个敢于听了天子之命的。”

韩冈跟着道:“小弟也会上本谏阻。这一事,太伤国家体面,也会留下后患,对日后不利。”他再叹一口气,“蛮夷畏威而不怀德,且欲壑难填。天子自以为的忍让,只会被视为退让,到时候其步步紧逼,又该如何对付?”

过去的士林清议,基本上都是跟着新党作对的时候多,谁想到此事一出,两边却是要合流了。

这算不算‘兄弟阋于墙,外御其侮’?韩冈甚至感觉到事情的发展,当真出人意表,甚至变得有些荒谬。不过这也是好事,当年他与雍王争夺周南,就是用着士林议论来压人。如今若能借这个机会,弥合一下两边的矛盾,对新党也是好事。

只是两人对视一眼,在对方的眼中,都是看到一丝无奈。方才说的事,他们当真会去做,但实际上的作用,也只能算是赌气而已。上奏谏阻若是有用,就不会有今天的事了。

不论是章惇,还是韩冈,他们在此事上的发言权实在太小了,远远比不上众位元老的功劳。除非是对付荆湖山蛮或是吐蕃人、党项人,否则都是只能坐看事情一步步的变坏下去。

“屡谏不从,家岳怕是不能安于相位了。”韩冈幽幽说道,“出了这一档子事,许多人不便弹劾天子,只能来弹劾家岳了。”

怒火收起,他现在又回归到现实中来。自当日与王雱商议之后,王安石苦苦支撑了近一个月,始终抱着一丝幻想,以为能说服最终天子。可如今天子主意已定,再不辞相,日后等着背骂名吧!

章惇闻言脸色一变,立刻点头,“相公最好早点辞相,否则弃土辱国的罪名,必然会加在相公身上,到时候,洗都洗不掉。”

王安石作为新党的领袖人物,一直以来饱受争议。说他‘刚愎’,说他‘不晓事’,说他‘不恤人言’,说他是不折不扣的拗相公,这些评价,几乎都为世人公认,但说他是伪君子、真小人的一干诋毁,却没有人去相信。

尽管王安石他强行推行新法,得罪了多少官员士子,惹来了多少攻击。但无论谁的攻击和弹劾,都无法在他的人品道德上找到半点可以指摘的地方。

道德水准,是如今评价一个人贤愚不肖的主要指标。新党中人,只要有一定的理智和头脑,都知道要在什么地方维护王安石这面旗帜。可以攻击他的施政,但不能让他的人品受到质疑和诋毁。

章惇也知道不能让王安石背上割地失土的罪名,这个污点沾到身上后,不是那么容易洗脱的。

“只怕外面的言论现在都会归咎于家岳了。”韩冈苦笑了一下,“不能谏阻天子,本来就是宰相的过错。”

章惇站起身,酒也不喝了,菜也不吃了,急着道:“愚兄这就回中书去。玉昆你今日应该留在京城吧?回去后好好劝一劝相公,要赶紧写辞章了。”

“小弟当然明白!”韩冈也站起身。

人嘴两张皮,以韩、富、文门生故旧之多,要将失土的罪名栽到王安石身上,也不是什么难事。在失去了天子的支持,王安石在高层是孤立无援,新党根基不厚的窘境,在对契丹一事上表露无遗。

这时候,只有先退一步了。退一步海阔天空,将反对割地的态度,通过一封辞章表现在世人眼中,让奸计难以得逞。

韩冈回京城奏事,都是照规矩住在驿馆中,从没有例外过。他行动做事,在小事上也都注意着,不给人留下口实。不过他今天却没有去驿馆,在去了开封府向知府孙永汇报了这一个月来的工作情况之后,就直接往相府去了。

韩冈抵达相府的时候,王安石和王雱都回来了。被领进书房,韩冈发现两人的脸色也都不好。

一等韩冈进来,王安石就道:“玉昆可是来劝老夫辞相的?”

“岳父难道准备附和天子不成?”韩冈反问道。

王安石道,“此事老夫岂会附和,但不能不加以劝谏。”

韩冈紧跟着就问道:“天子不听奈何?”

王安石脸色一变,但又立刻道:“终究还是会听的。”

拗相公就是拗相公。韩冈看得出来王安石是在赌气。而且是在跟韩琦、富弼他们赌气。过去天子都是信着自己,可偏偏遇到大事的时候,却又相信那一干被逐出朝堂的老臣们说的奇谈怪论起来——王安石不服气。

但旁观者清,韩冈从这两年来天子对王安石的态度上,已经看得很明白,赵顼已经不再是熙宁二年的那个王安石说什么就信什么,如同学生对师长一般尊重王安石的天子了。

他看了一眼王雱。王雱先是叹了口气,然后道:“大人,如今还是听了玉昆的提议吧。”

