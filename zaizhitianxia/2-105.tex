\section{第23章 铁骑连声压金鼓(一)}

【第一更,求红票,收藏】

张弓搭箭,弦飞箭出,平常人要两三次呼吸才能完成的动作,在王舜臣手中,却陡然加快了数倍,仿佛时光的流逝变得迅疾起来。长箭搭在弓上的时间就只有一眨眼的功夫,只看着弦在颤,听得声在响,一道道白光破空闪过,却无人能辨清箭矢是如何飞出。

箭矢如雨,就算用盾牌也遮挡不住如毒蛇吐信一般精准的箭矢,其落处惨叫声连成一片,几十张嘴一起合奏出哀痛的乐章。单靠王舜臣一人之力,就抵得上一队出色的箭手。从他手中射出的箭雨,彻底压制了冲向城门的敌军,使得从城下回射上来的箭矢寥寥无几。

被王舜臣领头的宋军弓手连番攒射,被阻截在城下的西夏士卒终于等到了撤退的信号,如同潮水一般退了下去。就像落潮后沙滩上的虾蟹贝壳,在城下,他们也留下了数十具尸体,还有同样数目的伤员。

西贼的号角声中,城头上猛然响起了一片彩声,守城的士卒们为他们主帅的神射连连叫好,投向王舜臣的视线中全是崇拜。自从前日接仗后,王舜臣就站在最前线,无论是防守时的城墙顶,还是反击时的排头兵,王舜臣一直处在这样的位置上。他拉坏的长弓已经有五六张,身上的甲胄最多时,插上了十几支长箭。

真要说起来,王舜臣做为一名将领并不合格,为将者,一人身系千军之重,奋死拼杀是底层军官和士兵的工作,统领着上千兵员的将军应该是在后方指点全军。只是王舜臣还没有适应身份的变化,虽然已经心知冲杀在前不再是他的工作,合理准确的命令才是他要完成的任务,但一听到战鼓声响,便忘记了他是统领千军的将领,只记得把敌人一一射落下马。

贼军退下去稍作休整,王舜臣便命人上来收拾城墙上的伤兵。四名臂缠蓝色布带的士兵随即带着十几人跑上城头,用简易的担架把几个运气不好中了箭的伤兵抬了下去。前段时间,郭逵和韩冈确定的军中医疗制度,在秦州最精锐的禁军中已经开始推行。如今已经有三分之一的指挥有了经过短期培训后的医工,虽然还做不到一个百人都就有一名的水平,但一个指挥都保证了至少有两人可以轮换。

靠着这些医工,王舜臣不必担心战地救护上的问题,一间小小的战地医院就设在城中央、原属于星罗结部族长的大屋中。而有了战地医院,许多轻伤员在处理过伤口之后,便主动归队,不像过去那样需要专门派人把轻伤员一个个逼起来作战。

靠着在敌军重重围困下,仍能维持着士气的千余士卒,王舜臣稳稳守着这座破烂的星罗结城。这座在大宋只能归入堡一级的小城,连城墙都是破败不堪。但城墙的地基却打得极为牢固,刀子划上去就只留下一道白痕。

城墙从地面到齐胸的地方,墙体的颜色也不同于上半段。只要对西北寨防稍有了解,就能一目了然的看出来星罗结部的这座小城堡,是建立在隋唐旧城的基础上的。而周长仅仅三百步的城垣,也说明了这座城不过是隋唐年间,边地最为常见,兼做烽燧之用、护卫通往西域的交通要道的大型驿站罢了。

王舜臣望着远处敌军,而在他手边的墙头上,排了一圈面目狰狞的首级。这并不是前日突袭时的斩首,而是不肯顺服的俘虏。苗授领军突袭星罗结城,斩首数百,而俘虏更多。正常情况下,这些俘虏都会被释放,让他们自谋生路。而在王韶的计划中,则要把他们迁到古渭寨附近,移交给纳芝临占部。一方面酬奖张香儿的功劳,另一方面,也正好可以把隐隐控制大来谷这个要道的星罗结部地盘给腾出来,交给更为可靠的部族。

不过王韶只来得及带走了第一批,在西夏人来袭前,王舜臣用了两天时间又捕捉了数百人。当西夏骑兵突如其来,杀到城外。只来得及关上城门的王舜臣,不敢把这些俘虏留在城中,不然厮杀正酣的时候,被人从背后捅伤一刀,可是会让人死不瞑目的。

当时想趁城中慌乱揭竿呼应的一群俘虏,被王舜臣随手杀了个干净,首级全都吊在城墙上。而剩下的俘虏,便被他扒光衣服,敲折了右臂放了出去。虽然王舜臣这么做,等战后肯定要受到责罚。但他现在可不在乎,光是因为被偷袭而失落在外的两百军卒,就已经够他喝一壶了,释放俘虏这些小事根本算不上什么,保住眼前的小命再说其他。

轻轻敲着城墙雉堞,窜入鼻中的是首级开始腐烂的恶臭。只是闻得久了,王舜臣很容易就忽视掉这个让人作呕的味道。

现在让他头疼的事很多。虽然不知援军什么时候会来,但粮食还是足够支撑一段时间,而星罗结城因为本就是修在溪流边,又有好几口旧朝留下的古井,不用担心水源问题。最让王舜臣头疼的是他手上已经没有多少箭矢了。

宋军以弓弩为上,最常用的对敌手段就是万箭齐发,将来敌射成一群刺猬。一支箭从箭簇、箭杆再到箭翎,基本上要七八文钱,战场上的一个指挥列阵攒射,就能把价值几十贯的箭矢全都射出去。如今天下诸国,也只有富得流油的大宋能让士兵在交战时,仿佛不要钱的往外拼命射击。就算是辽人夏人上阵,都要设法节约着用,而吐蕃人更不用提。

就是因为养成了习惯,而且一开始就没有准备在星罗结城久留的打算,原本随身带着的箭矢就不多。一天下来又都射去了大半,此时平均一算,每人的箭壶中就只剩十支箭可用了。现在城中守军因为受伤不多,才能保持着士气,但若是没了弓箭,面对面的厮杀起来,事情可就难说了。

又是一通号角,打断了王舜臣的思路。抬眼看着远处又骚动起来的敌军,他随手便拉了一下掌中的长弓,接下来,又是它出场的时候了。

扳指刚刚扯动弓弦,只听得啪的一声轻响,长弓的弓臂唰的挺直,带起的断弦抽在王舜臣的脸上,一条细细的血线便从他的脸颊上流了下来。

王舜臣脸抽了一下,吐了口唾沫,把断弓丢在脚边。这张弓方才连续使用,现在终于支持不住了。这也是他今天用坏的第四张弓,原本精心保养的两张上品硬弓全都毁了,现在用的军中制式硬弓,质量不算出色,很容易就会损毁。

“拿弓来!”接过手下亲卫递上来的长弓,王舜臣转了转手腕。他能左右开弓,一条胳膊累了,就换另一条胳膊,再加上他射击只求准求速,不求力道,今天射得虽多,却也没伤到胳膊。

‘再射个几百箭也没问题。’王舜臣心里这么想着。一边活动着手腕,一边盯着对面的敌阵,今次好歹再给自己添个上百战绩。不过他的手突然停住了,今次出阵的敌军不是几百上千,而是仅仅数十人。

来人越走越近,王舜臣的脸色则一点点的阴沉下去。几十人中,有十来人是被反绑着双手,他们不是吐蕃和党项,而是汉人,是王舜臣被俘的部下。他们被绑到了阵前,离城墙隔着六十步,被硬按着跪成了一排。那是箭矢难及的位置,一石多的普通战弓就算能射到六十步外,也不会剩下多少力道。

城墙上,上千只眼睛盯着这几十人的动作,不知他们是要劝降还是要斩首立威。而王舜臣看了两眼后,脸色突然白了,在他被俘的部下身后,有好几个吊着右臂的蕃人,这是被他下令敲折了手臂赶出去的俘虏。

“乔四!”王舜臣一声大喝,“带你的人到城下准备!听到我的号令,出城救人!”

一名粗壮的大汉躬身应诺,转身下了城去。

向手下最精锐的一个骑兵都下过命令,王舜臣右手往后一伸,“拿弓来!”

他的亲兵们一齐愣住,王舜臣的左手上不是正提着一张弓?

王舜臣回头一瞪,把左手中的战弓甩手丢了,“还不快拿硬弓来!”声音更添了几分急躁,掩饰不住的怒意已经处在爆发的边缘。

及时反应过来的亲兵一通手忙脚乱,急急忙忙的找到了一张两石出头的硬弓。王舜臣试了一下手,便甩手丢在地上,如爆雷的怒喝道:“没有力道更强的吗?!”

周围的亲兵,你看看我,我看看你,都是一脸无奈。如果是在秦州,力道达到三石的强弓也能从武库中给翻出来。但在眼下,能有两石的硬弓,已经是很难得了。

正怒瞪着手下的亲兵,身后突然传来一声惨叫,王舜臣猛回头,只见一名蕃人拿着一条血淋淋的胳膊在手中晃着。而他的一名被俘的军卒,已经滚倒在地上,右臂没了,鲜血淌了一地。等他滚得没了气力,另一名蕃人上前去,踩住背,把剩下胳膊和腿一起都砍了下来。

