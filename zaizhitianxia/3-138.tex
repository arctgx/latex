\section{第36章 望河异论希(四)}

韩冈低头看着手上整理出来的文字,与方兴的汇报对照来参考,最后点头道:“进度不错,辛苦了!”

方兴陪笑着:“是提点的竞争奖励管用。”

韩冈每天用奖励来鼓励各组竞争,每天总计一百五十贯的悬赏,只取前十名赏赐,就让一万多人拼了命的干活,一天的进度几乎能抵得上寻常的两天。正常情况下,民夫们怎么也不可能的这般勤力。

王旁则叹道:“也是玉昆待人宽厚,才能得民伕信任。得了信任,才会如此卖力。”他看着大堤上,隔着一段就有一座的工棚,里面不仅仅是民夫们休息的地方,棚子下面还排着一只只盛满了水的水桶,不时的就能看到有人过来舀上一瓢灌下肚去,“换作是其他地方,哪家会给民夫们提供盐水喝?”

方兴也道:“民夫在烈日下辛苦做活,流汗极多,我们这边掺了盐的凉水都是为他们准备着,一天差不多都要用上一石半的盐。就是不知其他地方能不能做到。”

“难说啊……”韩冈喟叹道。他能管着开封府的流民,监察沿河各县的工役,却管不到京畿以外去。

昨日中书下令,征调了一批流民往洛阳那边去修筑黄河大堤,这虽然如了韩冈之愿,但要指派流民一路走过去,还是要费不少周折。最关键的是不能让他们往东京城去,想想也只能安排他们沿着大堤走。而流民们到了洛阳后,那里的官员想来也不会如自己一般用心,民夫的伤亡率不用想也会大于白马这边。这等于是自己将他们送进虎口,韩冈的心中总是有点难以释怀。

看着韩冈心情有些沉郁,方兴识趣的转圜道,“如今东京一段河堤已经动工,洛阳也要跟着动工,过几日,从洛阳到大名的河堤都要开始修筑。”他感叹着,“黄河之患,在沙而不在水。日前准备用浚川杷来疏浚河道,目的也就是为了驱沙。提点的方略,由不得天子不心动啊!”

“谁让玉昆说出来的道理,都没人能驳得了?”王旁附和的笑着,“”

韩冈摇头:“有些人只是暂时观望、等待时机而已,不是当真认同。”

束水攻沙的方略,前些日子从王安石口中说出来后,就在朝堂上掀起了轩然大波。毕竟是将过去行之千年的治河手段全盘推翻,反对的奏疏如雪片般飞来。可细细数来,真正反对的最为激烈的仅仅是一些想博取名声的小臣,最大的也不过是几名御史而已。旧党重臣一个个都闭着嘴,富弼、文彦博等人都没有说话。

韩冈的提议很有些道理,加之杨绘的例子、还有郑侠的例子都摆在前面,谁愿意出头成为东京人的笑柄?而且韩冈的性格也渐渐地为人所知,言不轻发,行必有据,这两年一桩桩的事迹验证着,又有谁敢立刻跳出来丢人现眼?至少要等到他失败之后再出手。

再说要弹劾人,没必要迎着对手的长处去,那不是自找不痛快?安置河北流民的过程中,有的是机会。只要是为官理事,就不会没有出错的时候。不说构陷二字写来之易,就是要找茬,也是一找一个准。

有些人的想法,韩冈不用费心去猜都能看得明白。

所以朝堂上的纷争只用了十来天就没有了声息,只不过私下里讨论的就有很多了。

有人支持韩冈,他们翻找古籍,在《汉书》中找到证据。在《汉书·河渠志》中,张戎说‘水性就下,行疾,则自刮除,成空而稍深。’也就是跟韩冈说得是一个道理。

但也有人反对,毕竟这一方略过往从无有人施用于黄河。据说在宰相府上,反对声最为激烈的是都水丞侯叔献,他一口咬定束水攻沙绝不可行,不是韩冈说得道理不对,而是工程难度太大,能夹水攻沙的内堤根本修不起来。

不过因为碓冰船一事,王安石明修栈道暗渡陈仓,将侯叔献顶出来让人当笑柄,而暗地里采用了韩冈所创的雪橇车,最后一举翻盘。韩冈因此事而备受赞许,而侯叔献则成了韩冈的踏脚石,所以有许多人都认为侯叔献这实在挟忿报复。

韩冈与京中联络频繁,争论传言皆有耳闻。

许多言辞,只能报之一笑,连反驳都嫌浪费口水。不过也有一些,却是很有些道理。比如侯叔献所言,韩冈也为之深思。

不过韩冈好歹也知道,束水攻沙是明清时代出现的治河手段,那时候的技术条件能用,此事也不会有太大的问题。

说起治河,韩冈其实也只记得束水攻沙这几个字。但推敲其中道理,却总比现在一味的加固堤防,可每隔几年十几年就有一次破堤改道要强。

束水攻沙最坏的结果也不过是下游破堤如故,可只要能将开封这一段堤坝稳固住,这就是功劳。且现如今京畿周边全线动员,就算放弃了束水攻沙的方略,光靠重新加固起来的大堤,其实也能撑个好些年。到时候,说起来还是他韩冈的功绩。

而之前所用髙筑堤坝并开支河分水势的策略,也即是西汉末年贾让提出的‘分杀水怒’的方略,并不是不好,还有若能分水分到后世那等让黄河断流的水平,那还要头疼什么黄河决堤?可现在做不到,每分一次水,水流就越缓,沉寂下来的泥沙就越多——这何时是个了局?反倒是束水攻沙看着能拖得长远一点。

经过一段时间的讨论,韩冈的幕僚们也都完全认同了这个观点。

方兴道:“等到今年冬天内堤开始修筑,洪水未至时就能束水攻沙。而到了行洪期后,又可以缓解洪水冲击外堤。大河金堤必稳若金汤。”

王旁望着河心滔滔浊流:“‘多用巨石,高置斗门,水虽甚大,而余波亦可减去。’这是真宗皇帝当年说如何在汴河上修斗门的口谕。如果洪水水势高涨,多余的水就会从斗门上漫过去。而内堤的作用,有一半也近于此理。”

韩冈摇摇头,心中也不知道该叹气还是该感慨,就连王旁都能随意举用故事,而来源还是皇帝。

河防之重,实重于泰山。黄河三天两头决口,决口后,就是一泻千里,梁山泊——官场文字上称为梁山泺——是怎么来的?就是五代至宋初,黄河多次决口,每一次决口,洪水多半都涌向东面,最后在古巨野泽处潴留,汇聚成浩浩荡荡的八百里梁山泊。

作为通往京城的运河——五丈河的源头,梁山泊水产丰富,同时又是将京东东路的出产运往京城的起点,但当初形成梁山泊时,京东东路死了多少百姓,淹了几座城池,如今的人们都还能记得——就在真宗皇帝的天禧三年【西元1019年】,黄河决口,其位置就在白马县,‘岸摧七百步,漫溢州城,历澶、濮、曹、郓、注梁山泊’——白马县的县城都是重建的,前一座就在地底下埋着。

黄河的不驯,逼得当今世人不得不精研水利,所以连皇帝都能随口说出个一二三来。生死攸关,此事也不足为奇。

所以具体施工,韩冈并无意插手。他提出的仅仅是思路。以自己的水利知识,对比起如今的水利工程学的水平,韩冈并不认为在技术上,他有什么能指点人的地方。韩冈也见识过汴河靠近京城的一段,堤坝、水闸、桥梁,任何一处都闪烁着能工巧匠们的智慧。韩冈并不认为自己能胜过他们,而想必他们也能给自己带来惊喜。

在工地上,大批的木滑轮组已经用在了夯土的木桩上,省了不少人工。而运土上堤费时费工,韩冈张榜悬赏,前两天就有人来献了一架修堤飞土梯【注1】,可以将泥土通过滑轮和绳索很容易的运上堤去。工程的进度能如此之快,除了韩冈在管理上的功劳,也有简易机械大量使用的原因在。

而且方兴、魏平真,这等幕僚在政务处理上的手段以及见识,都要强于一般的官员。而稍逊一筹的王旁和游醇也逐渐历练出来,加上手下的官吏听命得力,做起事来也是得心应手。

上下一心,反对之声几希,虽然忙着,韩冈的心情还是很不错:“明早我要去胙城县看一下那里的流民安置情况。郑侠就要仲元你费心了,明日早点送其出境了事。”

王旁苦笑着点点头,以韩冈如今在白马县受到的尊敬,郑侠就算在驿馆中都待不安生,自家等会儿回城后,也还要去驿馆一趟。如果郑侠受到折辱,对韩冈的名声也不太好。

次日清晨,天刚蒙蒙亮,韩冈就带着一队人马准备前往胙城县视察。

一片蹄声向着西门而去,忽然前方几匹马伴着一辆车,从城西门处的驿馆转出来。几匹马上,唯一的一名官员韩冈并不认识,可就算是用鼻子猜,也能猜得出来究竟是谁。

竟然是郑侠!

隔着十几步的距离,两人都发现了对方。

差不多是相看两厌憎,韩冈无意上前,而郑侠更不会上来相见。韩冈遥遥的拱了拱手,就见郑侠转开视线,不顾而去。

韩冈摇头一笑,也全然不放在心上,一鞭望空轻挥,向着初启的城门行去。

注1:就在熙宁九年,神宗重修东京城。内臣黄怀信等献修城飞土车、运土车,‘并创机轮发土……所省者十之三。’

