\section{第43章 百里河谷田一顷(中)}

【今天第二更,求红票,收藏。再提醒各位兄弟一遍,更新时间改了。】

韩冈眨了几下眼睛,脑子一时没转过来,又想抬手去掏耳朵,只是给他忍住了。

‘听错了吧?……肯定听错了!这怎么可能……’他自嘲的笑了一笑,这才问道:“窦观察说得多少?”

张戬神色冷然,吐词清晰,不带一点含糊,每一个音都缓缓的咬得很准:

“一顷四十七亩。”

韩冈终于确认自己的耳朵没有问题,但接下来,他又确信窦舜卿的脑子出了问题。

他从来没听过如此荒唐的一件事,两百里的河谷……不,窦舜卿说的是从秦州到古渭,那就不是两百里,而是三百五十里。长达三百五十里的渭水和藉水河谷,秦凤路副都总管竟然说荒地只有一顷四十七亩!

荒天下之大谬,滑天下之大稽!

即便是千年之后,以十余倍于此时的人口,天水一带的荒地都不可能只有一顷四十七亩,翻上一百倍,一千倍还差不多。而在秦州人丁总计只有十二万,而蕃人人丁也不会超过三十万的熙宁三年,方圆几千平方公里的渭水中上游,竟然敢说只有一顷四十七亩宜耕荒地。这要是什么样的胆子和头脑才会说出的昏话?!

韩冈先是大怒,继而又是摇头失声而笑,笑过一阵,才起身向张戬程颢谢罪:“是韩冈失态了,还请两位先生恕罪。”

“无妨。”程颢一摆手,在他看来韩冈情绪的波动才能体现他话语的真伪:“玉昆你还是说说这究竟是怎么一回事吧。”

“两位先生,若要韩冈说,那没有别的,就是窦舜卿欺君罔上,为倾轧而不顾国事,其心可诛。一顷四十七亩地面有多大,不必韩冈再说。区区一个大相国寺,就占了十五六顷的地皮,金明池周长九里三十步,水面百余顷。难道秦州到古渭,连十个金明池的平地都找不到?!

秦州到古渭之间的渭水和藉水总长超过三百五十里,这一点,去枢密院一查军铺里程便可知晓。三百五十里有多长?从东京往西京洛阳是三百五十里,往南京应天【今商丘】是三百里,往北京大名又是三百五十里。东南西北四京所括田地不啻千万顷。即便秦州西北都是山地,但山谷之中,河水两岸,难道不是宜耕平地?!会只有一顷四十七亩?!”

韩冈一番话理直气壮,说得合情合理,语气更是斩钉截铁。张戬程颢都露出了深思的神色。韩冈也不停下来喘口气,此时他气势正盛,正是乘胜追击的时候,

“所谓由微见著,见一叶落而知天下已秋。萁子见纣王用玉著而知殷之将亡。窦舜卿欺君罔上以至如此猖狂,他今日能妄言三百里河谷只有荒地一顷四十七亩,他日未尝不能伪造军籍,贪污军饷,甚至讳败为胜,欺瞒朝堂。两位先生皆是御史,难道不该奏明天子,穷治窦舜卿欺君之罪,斩其首以正纲纪?!”

最后一句,韩冈狠狠暴出。以一介从九品的身份,对高高在上的窦舜卿喊打喊杀,程颢无奈的摇摇头,而张戬却没有呵斥他的无礼,沉吟了半晌,他又道:“……按窦舜卿所言,一顷四十七亩只是荒地数目。若是有主的,即便是蕃人,也不能计算在内。而王韶的万顷也是说的无主荒地。”

韩冈笑了:“天祺先生有所不知。远的不说,单是开封府,寸土寸金,但没有开垦的田地,难道就找不出一两顷来。韩冈西来,在黄河滩边,河堤之后,可是看到了不少长满衰草的荒地。天下四百州两千县,哪一州哪一县的宜垦荒地没有个千百顷?

再说秦州荒田,窦舜卿的解释更是可笑。体量荒地,并不是蕃人说哪里是他的,便把地算到他头上。总得是世代居住、开垦、放牧的地面才能算。打秦州主意的蕃人从来不少,总不能随便一个部族出来说秦州城是他家的,就把秦州城给他们吧?

甘谷城所在的甘谷不过六十里长,就有田四五千顷,里面虽有上万蕃人定居,他们也闹了多次,但最后也不过给了他们一半田而已。秦州地面广大,十倍于内地军州,但人烟稀少,不及江南一县。地大人少,可能没有荒地?”

韩冈一阵话就像疾风暴雨,把窦舜卿的奏章戳得到处是洞。稍稍喘了一口气,他有些疲惫的说着:“虽然说了这么多,韩冈却是不敢相信,天下竟然会有如此明目张胆欺君罔上之人。非是韩冈有胆怀疑两位先生,实是此事太过匪夷所思,不知天祺先生、伯淳先生,能否将此事的来龙去脉为韩冈说上一说。”

张戬和程颢交换个眼神,各自点了点头,程颢开口,便详细的向韩冈说明这一桩荒谬绝伦的公案来。

事情其实很简单。王韶的奏章是半个月前,也就是韩冈刚刚离开长安,走上潼关古道的时候,就被送到了天子的案头。赵顼见奏折上说得有情有理,心道有了万顷屯田之地,困扰他多时的河湟拓边的粮饷问题,便可以得到部分解决。

欣喜之下,赵官家便立刻下诏让秦凤路确认,以便能及早施行。但十天后,也就是今天,秦凤路发来的回复却说,王韶所言万顷宜耕荒地并不存在,经过经略司窦舜卿窦副总管的一番考察测量,发现所谓的荒地,只有一顷四十七亩!

如此一来,王韶便犯了欺君之罪,得到了攻击王安石的新武器的一众臣僚欣喜如狂。中书门下和枢密院同时下令彻查王韶之罪,御史中丞吕公著也明确说要去写弹章,而御史台的其他御史也不可能放过王韶。张戬和程颢则想到韩冈正好是王韶所荐,又从秦州来,便想从他嘴里再问个清楚。

韩冈皱着眉,双手十指交叠拢在身前:“这事就更是奇怪了。天子下旨确认王机宜奏折所言是否属实,十天后就收到了回复。以急脚递的速度,从秦州到京城要四天或五天,从京城到秦州也是一样。来回一次要八天到十天。即便按八天算,留给窦观察体量荒田的时间就只有两天。

两天时间,窦观察便量完了秦州到古渭的三百五十里河道,而且还精确到一顷四十七亩。这是荒地啊,不是田地,没有田籍可查,只能一寸寸的亲自去量,而且秦州又没有为蕃人建过五等丁产簿,他怎么确定地皮是谁家的?

更可怪的,是此时天气尚未回暖,连汴京道上的积雪都没有半点融化的迹象,何况西北高寒之地。今年冬天,秦州一带没少下雪。尤其是渭水自伏羌城以上,几场暴雪之后,积雪最厚处达三尺许。人难行,马也难行,原本两天的路,少说也要五六天才能走完。学生出来前便亲眼见到李经略为此散了常平仓的钱谷,相信秦州雪灾之事已经上报给政事堂。依然是一查便知。

这样的天气,各家蕃部哪家不是杜门不出?究竟是谁家向窦观察报备,确定自家的领地位置?若窦观察真的是用了两天就走完三百五十里雪路,丈量完所有的荒地,同时联络上与路的百十家蕃部,这手段,区区秦凤路副总管可安不下他,枢密使都有资格做吧?”

韩冈又是一番夹枪带棒、语带讥讽的长篇大论,程颢和张戬听着苦笑摇头,他们不怀疑韩冈之言的真实性,因为韩冈说得完全在理,并且给出了可以查明的证据。

如果不是像韩冈这样直接当事人来说明,他们这些御史坐在几千里外的京城,怎么可能知道地方上真实的情况?都是当地官员怎么奏报,他们就只能信着,最多心里存疑而已。即便地方两家纷争,也无从作出评判。要么去翻旧档,要么就是选择自己认为可信的一方,而不可能追查事实。无他,距离太远,事实难明。

其实天子也是一般受欺。别看赵顼兢兢业业,一日二日万几。但实际上他看到的,听到的,都是群臣想给他看的、想给他听的。就算他从宫中派出去一队队的宦官充当走马承受,但实际上,已经融入官僚队伍的内侍们,根本动摇不了早已成型的现实。

不论下面的臣子分为一派,还是两派,甚至多派,他们上奏的文字少不得都是偏向自己一方的。而要从扭曲的文字中寻找真相,即便是宦海沉浮多年的名臣也是勉强,何况自幼就住在东京城中的年轻皇帝?这并不是他所能做到。

程颢、张戬做了多少年大臣了,当然知道这一点。古来昏君,有几个是真心毁掉自己国家的?即便是商纣、隋炀,也不可能眼睁睁看着自己的国家衰败下去,还能开心的玩乐。还不都是言路闭塞,奸臣充斥周围的缘故!

“不知此事李经略是如何说?”韩冈这时方问起自己最关心的问题,若是没有发现李师中早前所写的奏章,王韶也不会一张口就是一万顷。而一旦李师中因前事不敢发言,窦舜卿的攻击却也并不足为虑,“窦观察查出来的一顷四十七亩,跟去年李经略说过的一万顷完全相悖,李经略难道支持窦观察的说法?”

“李师中自称他当时是初至秦州,为王韶所诓骗。”

韩冈忽而冷笑:“……李经略才智高绝,欺人时常有之,被人欺却从来没有听说。”

