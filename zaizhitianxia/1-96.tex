\section{第38章 逆旅徐行雪未休(六)}

【新年第二更,俺不会嫌红包多的。】

只是看着种建中的表情,韩冈心中有了点不好的感觉:“该不会是雄武军吧?”

种建中哈哈赞道:“玉昆果然才智过人。”

这个‘果然’可不好。韩冈脸色虽没什么变化,脑仁子却疼了起来。想不到郭逵竟然要擢迁雄武军节度留后。

秦州的军额便是雄武军,像韩冈的举主吴衍,就是雄武军节度判官。虽然本官与实职差遣无关——王韶的本官是太子中允,但赵顼连个儿子还没有呢。吴衍的本官是大理寺丞,而他也不在大理寺上班——郭逵应该不会来秦州。

照理说是如此,可有个万一呢?万一郭逵转任雄武军节度留后是朝中给出的一个信号,那就让人头疼了。

郭逵有雄心,有才能,有威望,有地位,更有经验。但他最大的问题,就是喜欢大权独揽。在鄜延,种谔被他挤兑。若是他到了秦州,王韶还有站的地方吗?要知道王韶与李师中、向宝两人合不来,便是因为权力之争。郭逵在关西在军中的威望远在李师中和向宝之上。他来秦州任职,开拓河湟的战略应该还会继续下去,但在那之前,王韶肯定会先被踢到一边。

韩冈和种建中对视一眼,一齐苦笑,谁都别说谁了,一个郭逵就让两家头疼得都要裂开来,运都倒在一个人身上。

“对了,”说到绥德城,韩冈便想起今天在路上遇见的山羊胡子,以及从这位老税吏口中所听到的消息,“不知几位听没听说过,转运司陈副使下令陕西全境税卡加强税检,即便拥有官身,也不得私带商货过关。”

种詠和种建中听后顿时陷入深思,陈绎的做法反常得让他们难以置信,而种朴却没有考虑太多,直接摇头道:“不可能吧,那要得罪多少人?陈副使什么时候有这个胆子了?”

“说是因为提供给绥德城的钱粮不足,必须要加强征收。”韩冈将陈绎的理由平平实实的说出口,等着种家三人的反应。

砰的一声响,种朴当先拍案而起,双目圆瞪,怒发冲冠。他厉声叫道:“他竟敢这么说?!”

“竟有此事?!”种詠也一样吃惊,再次重复追问着,“可是确有其事?!”

“小侄区区一个从九品,编排转运副使作甚!?”韩冈反问道。他是秦州官员,鄜延路的问题根本与他无关,陈绎的小动作也扰不到秦凤去,他相信这一点种詠能想得明白。

“项庄舞剑,意在沛公。”路明阴阴的在旁插了一句,尽力表现自己的存在。

种建中狠狠地一锤桌子,“这是驱虎吞狼之计!”

陈绎的用意,不但种建中想得通透,连种詠和种朴都看得明白。不外乎煽动人心来干扰绥德。即便他的命令最终被阻止,也可以名正言顺的不为绥德城提供足够的钱粮。

种建中又愤愤不平的继续说道:“难怪陈绎下令不得在环州、庆州这些缘边军州发放青苗贷,还说要留常平仓物,准备缓急支用,原来是为了演得更像一点。”

“王相公岂能容得了他?!”路明立刻问道。

韩冈为他解惑:“陈绎正是为了堵王相公的嘴才这么做的。”

陈绎越是用常平仓为借口不肯散财散物,越是用钱粮不足为理由停止发放青苗贷,便越是显得他加强征税的正确性,也更理直气壮地去卡绥德城的脖子。

而且他用绥德虚耗钱粮为借口,停止发放青苗贷,又要留用本该用于青苗贷放贷业务的常平仓储备,等于是用王安石的左手打他的右手——颁布青苗法的是王安石,倡导绥德战略的也是王安石——也许可以让王安石找不到任何处办他的借口。

陈绎算是把世情人心算到了极点,不愧是长于刑名的官员。若是在提点刑狱衙门,他的表现肯定要比转运司要强。韩冈很佩服陈绎,而王安石就不一定了,任何计策都有个适用的范围,若是以力破之,直接办了陈绎,那是什么谋算都没有用。

空气凝重,几人默默地坐着,气氛沉凝的仿佛是在为人守灵。种家叔侄三人都是紧皱眉头,韩冈和路明都挤出同样的表情陪着他们,也就刘仲武,看起来显得很轻松。

“算了……算了……不提这些烦心事了。”种建中照空甩了甩手,似是要将束缚着自己,使得自己难以施展的绊索全数扫开。要想对付陈绎,除非朝堂上有人出手,凭着他们几个,什么办法也没有。“对了!玉昆,你猜小弟今天还碰到了谁?”

“没头没脑,我怎么可能知道。”韩冈看着就他和种建中在说话,其他几人都在便听便喝,便拿起酒壶站起来,给每人都倒了一杯。

“是游景叔!”

“你遇到游景叔了?”韩冈放下酒壶,坐了下来。种建中的话,让他有些遗憾自己走得慢了些。

游师雄游景叔算是韩冈和种建中的师兄了,在张载的诸多弟子中,游师雄的才能也是出类拔萃的一个。以经义大道论,横渠门下,以蓝田吕氏兄弟——吕大临、吕大钧、吕大忠——三人为最,而以兵事论,则是以游师雄为首。

种建中年纪尚幼,但将门子弟在兵学上的才能也不容小觑。至于韩冈,留给众同学的印象,却是箭术还不错,但刻苦过了头的书呆子一个。谁想到他如今已经被荐为官身,现在正要入京递上家状?

不过游师雄并不只是长于兵事,文学一样出色,早早的便考上了进士,是治平二年的龙飞榜出身【注1】,让张载的一众弟子甚为羡慕。而在张载的弟子中,蓝田吕氏兄弟里的吕大忠、吕大钧皆是进士及第。吕大忠中进士比张载还早,吕大钧则与张载同科,即便这样,他们依然敬张载如师长。

游师雄如今在,名望在外,张载的弟子们当然都是佩服不已。尤其是种建中和韩冈这样偏向兵事的弟子,更是如此。

“上次听说游景叔时,他应是在仪州任司户参军,现在到了京兆,是调还是升?”

“什么升、调?”种建中摇了摇头,“他是武功人【今陕西武功县】。今次是到转运司述职,顺便返乡省亲的。”

“人走了没有?”韩冈急着追问。

对于如游师雄这般才能地位皆高的师兄,韩冈自然很有兴趣结交一番。后世讲究四大铁,此时也讲究着同乡、同年、同门,与同为横渠弟子的同门兄弟拉好关系,自己的根基也便会更加稳固。

“今天清早便回仪州了,就在道边匆匆说了几句。”种建中有些遗憾,游师雄进士中得早,跟他和韩冈这样的小师弟只有几面之缘,没能深交,今次巧遇,却又是一叙而别,“说起来,游景叔已历三考,磨勘也过了,大概明年便要转任。若是调出关西,再见可就难了。”

种詠一起叹了口气,他年纪即长,亦久历世情,对此感触更深。此时便是如此,见面难,再见更难。道左一别,再听闻时,也许已是阴阳重隔。

韩冈却是笑着,洒然道:“何必做小儿女态!酒在杯中,人在眼前。与其长叹,不如醉饮!”

“说得好!”种朴拍手笑道。

韩冈几句,豪爽无比,正合种朴脾气。他站起来举杯邀约,众人便轰然和应,一番痛饮,宾主尽欢。

种建中与韩冈同学两年,关系只是平平。但今夜偶遇,一番相谈,只觉得与韩冈意气相投,人物风采为生平仅见。酒后席散,种建中和种朴便硬拉着韩冈去秉烛夜谈。

直至次日清晨,谈天说地了一夜的韩冈,方被种建中兄弟俩给送了出来。韩冈的才学见识皆是一流,纵然无法像当日对王厚那般借势纵论,使人五体投地,但已经足以让种家二子深感敬服。

回到自己院中,三间厢房的房门都是大开着,无论刘仲武还是路明皆不在房中。李小六这时已经起来,韩冈走进房门,吩咐一声,他便端来了梳洗用具。

拿着滚热的手巾擦着脸,韩冈顺手指了指隔邻,问道:“刘官人和路学究呢?”

李小六回道:“刘官人一大早去马厩照看他的马去了,好像蹄子磨得厉害。路学究则牵着他的骡子出去了,不知是要做什么。”

韩冈随口应了一声,示意自己听到了。

路明的骡子本是昨日那位倒运的胖蜀商的,还附带着一驼价值不菲的货物,路明从邠州带来的土产别看多,却卖不上价,邠州的名产只有一个——就是田家泥人,一对能值十贯有余。除此之外,并没值钱的东西。要不然,路明的那头老骡子的背上,货物也不会堆成一座山。

而从蜀商那里弄来的货物,只看包裹外形,就能确定是蜀地特产的绸缎。蜀锦贵重,即便是最便宜的绢罗,也至少值得三四十贯。只是如今关西税卡森严,韩冈又答应带他一起上京,骡子不可能跟得上驿马的速度,干脆全卖出去换成盘缠。对于路明的想法,韩冈很清楚。

刘仲武的马蹄子,韩冈则没兴趣。他心中只在奇怪一件事,他预计中应该到的人,怎么还没消息?

韩冈正想着,这时房门被敲响,李小六过去打开门,一名驿卒走了进来,恭恭敬敬的双手递上一张名帖,道:“外面有个老员外要求见两位韩、刘两位官人。”

韩冈接过名帖,便微微一笑,喃喃念了一句:“终于来了。”抬头对李小六道,“快去把刘官人请来。”

李小六应了声便要出去,转身前顺势瞥了一下名帖封面,上面端端正正的写着一排小字,其中字体较大的四字,便是——

浦城章俞。

注1:龙飞榜:新皇帝登基后第一次开科取士,便称为龙飞榜。宋英宗赵曙登基后第一次开科,就是在治平二年。

