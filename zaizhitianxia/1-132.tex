\section{第47章 不知惶惶何所诱(下)}

【第二更,红票,收藏】

韩冈向张戬解释着:“这对百姓是好事。因为官府把低息贷款的名头打出来后,天下富民再想维持倍称之利便不可能了,如果想继续放贷,就只能把利息降到跟官府一样,这对百姓不是好事吗?天下百姓哪能承其恩惠?”

“玉昆你还年轻,不知其中情弊。”张戬摇摇头,果然还是历事不多、受了蛊惑的缘故,“州县胥吏多伪诈,皆尽小人,而州县官也往往受其所欺。一旦实行青苗贷,他们能上下其手的机会太多了。别的不说,提高利息,减放本金,这都是他们做得出来的。”

经历过陈举、黄德用之事,韩冈很清楚地方上的胥吏们有多么无法无天,但讳疾忌医却也是要不得的,“如果依着青苗贷原来的名字,百姓都听不懂究竟是何事,只能任凭地方官吏所欺。前些天不是有个陈留知县,他在衙门外贴了三天的布告,又在乡里贴了三天的布告,回过头来便撕了布告,说无人请贷,在陈留县不用推行青苗法。可这么短的时间,又不向百姓宣传,贴了几张纸,又怎么会不让人犹豫?而如今利民低息贷的名字说得清楚直白,又有谁会闹不清?”

张戬紧锁着眉,连连摇头。在他眼里,韩冈现在就如同一头犟牛,死咬着牙坚持自己的意见。“放贷收息,朝廷体面还要不要了?”

“朝廷的体面由百姓中来,百姓富足,朝廷自然有体面。”

“玉昆你可知道,一旦青苗贷推行下去,尽管如今的富民不能再放贷,贫民不会再受他们的盘剥,但主管青苗贷的官吏,却只会一步步的比早前更加酷毒。”

‘我当然知道,不论是什么样的政策,都会在施行的过程中变得对权力者越来越有利,旧的利益集团被打倒,新的利益集团便吸着他们的血茁壮成长,这不是理所当然的吗?’

韩冈腹诽着,神色间却装得一本正经:“但总不能看着天下百姓一直受着富民所欺。学生家自先祖父起,便是以务农为生。两代人四十年的辛劳,一亩一分的积攒下了百亩田地,但学生一场重病就把几十年的积累全毁了,若不是学生病愈得及时,如今也不知要背上多少债务!如果当时有息钱低一点的借贷,学生家的田地产业何至于被人剽夺的半点不剩?”

韩冈与张戬第一次争论起来,不过韩冈小心的控制事态的发展,不使争论变成争吵。他也不想日后跟自己的师长变成势不两立,所以得提前打个预防针,省得张戬和程颢听说他投了新党,以为自己受到欺骗。

程颢倒是觉得韩冈说得有理,出身寒家且受过高利贷欺的韩冈,若是不支持青苗贷,反而奇怪了。而且韩冈对官府借贷的看法,也符合程颢的本心。程颢本就是不反对帮助百姓,救人急难,只要不是以牟利为目的,利息降上一点,青苗贷行之亦可。

不得不说信任是有惯性的,韩冈对青苗贷——不,现在改叫利民低息贷款——的赞赏,张戬虽然难以认同,只要韩冈做得不出格,不跑去为新法鼓吹,张戬还是愿意相信他这个学生。

照旧在张戬家吃过饭,方才的一点芥蒂也是一笑了之,饭桌上,张戬听说韩冈已经拿到告身,便问起了他接下来的行止,韩冈道:“能在两位先生门下就学,是学生几世修来的福气,惟愿能常随先生门下。不过如今学生已经拿到了告身,不能再耽搁了,现定得后日启程。”

“既然已经拿到告身,那就是官人了,为天子牧守百姓。且谨记勿残民,勿贪纵,行事以清正为上。”

程颢也跟着道:“吾观玉昆你不是在学问上能有所发展的性子,但为人处事都分寸,日后必为栋梁之才。别的话也没有可送你的,只要你能记着你读书的一点心得,凡事体仁心,尊立法,行中道,也就够了。”

韩冈站起身,恭恭敬敬的答道,“两位先生的教诲,学生必谨记在心。”

……………………

第二天,是章惇休沐之日,韩冈和刘仲武拿到告身的事他也听说了,便再次邀请了韩冈一众,在他们离开前做一小聚。

一见韩冈,章惇便拉着他到一边低声笑道:“最近署中事多,也是玉昆你的功劳。你出个了计策,我等便要忙个脚不沾地。”

韩冈摇头笑道:“编修此言,韩冈可当不起。而且现在脚不沾地的,不是编修,而是文吕司马之辈。”

韩冈和章惇哈哈又是一阵笑,让不知来龙去脉的刘仲武和路明摸不着头脑。

互相谦让着坐下,章惇拍了拍手,道:“今天请来的校书【注1】,虽然年岁不大,却以歌舞双绝名震教坊,最难得的是洁身自好,让人激赏不已。”他神秘一笑,“玉昆见到她,定然有份惊喜。”

只是看到来人,韩冈惊喜倒没有,却当真吃了一惊,“周小娘子?”

“周南拜见章编修,拜见韩抚勾,拜见刘官人。”周南笑语盈盈,完全不见几天前的怒意。只是当她避开章惇,视线掠过韩冈时,却是凤目含嗔,狠狠地盯上了一眼。

韩冈以笑容回敬过去,就见到周南气得银牙咬着下唇,用力扭过头去。韩冈轻笑了两声,觉得这样的歌妓真是难得。正如章惇方才所说,洁身自好的周南,应该是尚没被污染的女孩子,若是久历风尘,什么样的心情都能掩盖在营业性的笑容之下。

章惇大概是从其父章俞那里听到了什么,便让周南陪着韩冈,而他和刘仲武身边的则是普通的妓女。周南沉默的陪着韩冈喝了两杯酒,便下场翩翩起舞,而悠扬婉转的歌声,竟一点也没有被动作所打乱。

韩冈轻轻击掌,的确是歌舞妙丽,极尽妍态,当得上歌舞双绝的称呼。

章惇极会做人,知道韩冈不擅诗赋,便在酒宴上半句不提酒令,对句,射覆之类的惯见娱乐。说了几句笑话,又跟刘仲武和路明对饮了几杯,章惇凑近了,压低声音说话。

“玉昆,听闻你是横渠张子厚的弟子,”章惇提起张载时,撇了一下嘴,提起张载这位姓字同音的同年,他心中就有些怪异,“你在经义上,应该有所心得吧?”

“在下才疏学浅,诸经只是泛泛读过,算不上精研。”韩冈谦虚着。

他的经义水平,如果是面对的是普通的半是运气半是才气考中的进士,也许还能一较高下,但章惇是想考进士就能考上进士的正牌子的才子,他的才能可不仅仅是诗赋。韩冈在章惇面前,现在还没有自大的本钱。

章惇低头把玩着拿在手上的朱砂色的酒盏,翻来覆去看了几遍,对韩冈笑道:“这是钧州民窑的货色,红得不透,晕得不匀,比起内用的正品,差了不止一筹。”

“民间也不会有内用之物。”韩冈说道。对章惇有些不屑,通过转换话题,来掌握对话的主动权,自家玩得更溜。

章惇又压低声线,低得只让韩冈一人听到:“经义之事,说难不难,说易不易。若是真的钻研进去,一生也不能穷尽,但如果只是想学以致用,三年便有所得。”

‘三年?!’韩冈心中一动,带着疑问的神色看向章惇。章惇这时又抬起头欣赏着身前的歌舞,似无所觉,前面的话仿佛不是出自他口,却又微不可察的点了点头。

韩冈会心一笑:“韩冈谨受教。”

“你能明白就好。”章惇便拿起酒壶,给自己酒杯斟满酒喝了起来。

‘如何会不明白!?’毕竟章惇都说得这么直白了。

韩冈当然明白,没事章惇何必问着这些事?章惇可不是爱说废话的人。看起来自己以前猜得没错,王安石还是打算变革科举制度,虽然这一科已经不可能,但下一科的考题,必然改成经义……学以致用,说不定还有策问。

‘这三年里,是不是要按着章惇的提议,去攻读儒家经典?’韩冈陷入沉思,对周南的绝妙歌舞视而不见。真有‘泰山崩于前而色不变,麋鹿兴于左而目不瞬’的气派。

见着韩冈这副作派,周南气结,动作也乱了一点。尚幸被她及时补救回来,没给外人察觉。一曲舞罢,周南又坐回韩冈身边。剧烈的舞蹈之后,少女喘息着,额头上细密的汗珠晶莹剔透,俏脸晕红,丰盈的酥胸轻颤,淡淡的香气从她一侧飘进韩冈的鼻尖。

周南气喘得厉害,右手用力压着心口。方才她为了弥补一时的失态,强换动作,便走岔了气,胸膈隐隐作痛,心中就恨得想咬韩冈的一块肉下来。她伸手拿起酒杯,准备喝点酒水压一压。

韩冈突然伸出手,把酒杯从周南手中拿开。被一只滚热的大手攥着,周南脸一红,忙把馥软纤细的小手从韩冈掌中抽开。她又羞又恼的瞪过去,她往常遇到客人都讲究着身份,哪会这般无礼?

而韩冈却是毫无所觉的抬手给她倒了杯茶,柔声道:“气急不可饮酒,还是喝茶好一点。”

周南愣愣地看着韩冈递过来的茶水,怔了许久。

章惇在旁看个通透,笑言:“玉昆当真怜香惜玉。”

韩冈微微一笑,心中却在疑惑,难道他这么做现在很少见吗?

注1:唐胡曾《赠薛涛》诗:“万里桥边女校书,枇杷花下闭门居。”薛涛,蜀中能诗文的名人,时称女校书。后因以“女校书”为歌女的雅称。亦省称“校书”。

