\section{第40章 雁度长空迹不彰(上)}

【高估了自己码字的速度,低估了今天单位里的事情。到现在才有第三更,等会儿加油写,夜里还有一更,明天上午再一更,不会少。】

初秋时分,已经进入了一年中最适合狩猎的季节。

当今的大辽皇帝耶律洪基,最喜欢的就是乘着飞船直上云霄,俯视着属于他的国土,而后从飞船上下来,去追逐猎物。

耶律乙辛远远的观望着。天子能开心畅快的游猎,正是由于有他帮着处理国中政务。

大辽的权臣不禁会想,如果皇帝能把精力全放在这两样事情上就好了。朝堂上的那些鸡毛蒜皮的小事,完全可以有他这样忠心耿耿的臣子来全权处理。

太子就是不认同耶律乙辛的这个想法,最后落到生母赐死,自己幽闭而亡的境地。但耶律洪基绝不是耶律浚。耶律乙辛能以谗言离间夫妻、父子,唆使耶律洪基杀了皇后、废了太子,但他面对耶律洪基时,可用不上这一条手段。

耶律浚之死是因为父子亲情压不过权力之争,自己只是煽动了耶律洪基心中潜藏的恐惧,让他以为耶律浚有篡位之心。经过了皇太叔耶律重元的叛乱,耶律乙辛知道耶律洪基最怕的是什么。

耶律乙辛能掌控朝政,究其原因就是在平定耶律重元之乱时立了大功。他犹记得在大帐中确认了耶律重元的谋反,当时的大辽天子脸上的表情,是在愤怒中,透着深深的恐惧。变了调的声音,泄露了他对失去权力的恐惧。

如今耶律浚已经死了,天子不会再担心一个死人跟他抢皇位,当初的恐惧和由此而来的愤怒也就烟消云散,剩下的就是父子之情,疑心自然就会悄然滋生。到时候说不定只要谁说上一句话,就能让当今天子改弦更张,认为儿子谋反之事是纯属污蔑。

耶律乙辛很清楚,自己的权力,是嫁接在皇权之上。失去了皇权撑腰,他耶律乙辛又能使唤得了谁?自家的性命与权力密不可分,一刻也不能松手。可耶律浚的冤死,就是悬在头顶的一把宝剑,耶律乙辛也说不准什么时候大祸就会降临到自己的头上,那时候,就是灭顶之灾。

手抖了起来,耶律乙辛突然发现,对失去权力的恐惧,不论是天子,还是自己,都是一模一样的,没有任何区别。

“大王。”萧十三快步走近了耶律乙辛的身边,手上拿着一片纸,“西京道急报,乃西夏朝中事。”他手递出去的同时,偷眼打量着耶律乙辛的脸色。

耶律乙辛警醒过来,瞬息间恢复了正常。接过纸片,匆匆一览:“西夏翰林学士景询以赃诛……”抬起头来问道,“景询是谁?”

萧十三回覆道:“景询是宋人叛臣,饶有谋略,屡助党项攻宋。十几年前投奔西夏的时候,南朝还指名要西夏将景询交还。”

“难道景询是宋人奸细?”耶律乙辛惊讶的问道。

西夏同时向辽、宋称臣,如今更是势弱。如果抓到两国的细作,也不敢以间谍之罪将他杀了,必然是要以别的名义处决,当做根本不知道奸细这回事,为空留人口实——何况现在宋人正磨刀霍霍,就只缺个借口。由此推断,宋人要讨还景询的举动,当是个幌子,推景询上位的助力。

“不是。”萧十三摇头否定,又发现自己的口气太硬了,忙缓和了一下,“应该不是。景询是梁氏的人……谋主。”

耶律乙辛闻言眉头一挑:“记得秉常是六月亲政的吧?”

“正是。”萧十三点头,“秉常六月亲政,才三个月就杀了梁氏的亲信大臣。这小孩子还真沉不住气。”

耶律乙辛摇摇头,叹了一声,“烂泥扶不上墙,西夏眼下都到了这步田地,偏偏碰上了如此糊涂的一个国主,当真是运气差透了。”

“大王说得是。”萧十三附和道:“放秉常亲政,梁氏本不甘心,还是大王使人发国书质问,才不得不放权。只是谁都没想到,不过三个月的功夫,就杀到了翰林学士的头上,若是再过个一年半载,还不杀到梁乙埋头上?梁氏肯定要拼死反扑了。”

耶律乙辛眯起眼,皱眉不语。

秉常这个小孩子,不要说跟继迁和元昊比,就是有他曾祖父德明的五分耐性,也不会做出这等让宋人欣喜欲狂的蠢事。

经过了三年,宋人肯定已经将横山南麓给安定下来,粮秣军器也备足了,多半已经准备好要用兵于西北。眼下对于西夏来说,正是需要同舟共济的时候,偏偏秉常却杀了自己舅舅的心腹。

耶律乙辛沉吟良久,忽而问道:“……你说这件事,宋人会不会知道。”

萧十三肯定的回答:“这么大的事,肯定瞒不过宋人的耳目,两边打了这么多年,细作肯定几十上百的派过去,有什么风吹草动,都不可能瞒得过,我们在河北还不是一样有人。”

“秉常轻佻急躁,梁氏根基深厚,肯定会斗起来。”耶律乙辛微微一笑,“宋人应该不会放过这么好的机会。”

萧十三双眼一亮,事情既然发生了,后悔也没用,还不如趁机利用。如果大辽能表明的秉常的支持,想必西夏年轻的国主只会更加迫不及待。到时候宋夏开战,大辽就能乘机插手进去。

“如果与宋人打起来……”耶律乙辛省略了主语,又只说一半,但意味深长的眼神,让萧十三彻底明白了魏王的用意。

宋人攻夏,大辽肯定要插足进去。若是进一步引发了宋辽开战,只是为了朝中的稳定,耶律乙辛的地位就能稳如泰山。胜可加功,博取诸部的支持,至于败……情况也不会比现在坏多少。

“大王妙计。”萧十三低头拜服。

耶律乙辛以五院部出身,其父更外号“穷迭刺”,能坐到现在的位置上,这政治上的谋算,当然是无人可及。

“大王!末将拜见大王。”

又是一人通报后过来拜见耶律乙辛,却是前几天派出去办事的萧得里特回来了。

“事情办成了?”耶律乙辛漫不经心的问着。

“办成了。”萧得里特脸上的笑容狰狞,“太子妃……”

这个称呼刚出口,就在耶律乙辛一下转得狠厉的眼神中停了口下来。咳了一声,道:“这一下就能安泰一些了。”

“当是如此。”耶律乙辛点头微笑。

‘哪里能安泰得了?!’大辽魏王的心中依旧冷得如冰一般。不过是震慑一些首鼠两端的贼人,让他们继续观望下去。敌人依然还是敌人,只是由明转暗,变得更加危险了。

毕竟当初聚集在故太子耶律浚身边的人很多——这也是为什么他耶律乙辛能打动天子的原因——太子死后,明面上的同党治罪的治罪,离散的离散,多少人畏于自己权势而不敢开口说话。但同情耶律浚的在朝堂上依然存在,只要给他们找到动摇自己的机会,事情就危险了。

话说回来,就是耶律浚不是自己下得手,是自行病死的,恐怕许多人也照样会设法将这个罪名栽到自己身上。只要有机会将自己掀翻,不会有人不愿趁机踩上他耶律乙辛两脚。

“这一趟辛苦了,先下去梳洗一下,歇上一天。等明儿再来见我。”耶律乙辛温言道:“接下来要借重你的地方还很多,不要累坏了身子。”

“多谢大王垂顾。”萧得里特跪下来磕了一个头,却没走,“虽然大王吩咐小人的事已经办好了,但小人听说陛下最近将皇孙招进宫中自养。这件事非同小可,如果不早作预防,恐有不虞之祸。”

萧十三诧异的看着萧得里特,这是他早就知道的事,但萧得里特应该不会无缘无故的说出来。

耶律乙辛瞥了萧得里特一眼,淡然一笑:“萧茹里算不上聪明,相貌也不算出色,就是运气甚好,竟然生了两个千娇百媚的女儿,还真是让人意外。糺邻……”他亲切的称呼着萧得里特的表字,“你说是不是?”

萧得里特浑身一颤,遍体生寒,脸色一下变得很难看,他跟萧茹里的密谈只是方才回来时的匆匆数句,怎么耶律乙辛就得到消息了,而且说的什么话都知道了。

想笑两声化解尴尬,却发现自己现在发出的笑声嘶哑难听,仿佛公鸭在叫。连忙跪下来:“小人也没想到萧茹里竟然转着做国丈的心思。方才小人回来时,本是准备绕路避开人多的地方来见大王,没想到就碰上了他。萧茹里他拉着小人说,他家的两个女儿,天生贵相,乃宜男之体,如果将她们两个献入宫中,当能生出个皇子来。这番话,他当是为自己说的,但萧霞抹的妹妹进宫都两年了,到现在也没有生出皇子来。如果当真能带来一个皇子,如此一来,我等也可高枕无忧了。”

耶律乙辛沉吟不语,萧得里特仿佛看到了希望,凑近了道:“虽说能不能生出皇子,这要看天数,不是靠什么宜男之体,但多一人都多一分把握,总比现在干等着要强。”

耶律乙辛似乎是被触动了,抬眼看着萧得里特,“这件事我会让张孝杰安排的,你且去跟萧茹里道喜吧。”

萧得里特大喜过望,又向着耶律乙辛说了一通好话,匆匆忙忙的就走了。

“办成了事,却不急着回来报信,反而跟人商议给天子献女。”萧十三转过来对耶律乙辛道:“萧茹里跟萧得里特是远支的堂从兄弟,还是姻亲,若是萧茹里的两个女儿当真生出个皇子来,萧得里特恐怕就会更张狂了。”

耶律乙辛的视线追着萧得里特的背影,眼底生寒,凛凛如三九寒水。过了半天,他方才张开口,

“你知道吗?”大辽魏王轻声说道。

萧十三躬身侧耳。

“我最不喜欢的……就是天数。”

