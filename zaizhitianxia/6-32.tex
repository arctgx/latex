\section{第五章 冥冥冬云幸开霁(11)}

震耳欲聋的一声巨响。

王厚的视野再次为青烟所笼罩。

嗡嗡耳鸣尚未停歇,在还未散去的硝烟外,又是一声巨响传来。

齐王府的正门,连同一旁的门房,在五次射击之后,彻底化作了碎石瓦砾,腾起的烟雾像硝烟一般又扩散开去。

“好了。”

王厚在灰土扑来之前,轻松的念了一句。

正门不复存在,再也没有能阻挡攻入齐王府中的脚步,再想拖延也已经不可能。

不过耽搁的时间已经足够了,齐王府中的火势,也到了难以扑救的等级,不需要再多拖延多少时间。

远处旧封丘门上的红旗已经飘起,李信在朱雀门那边应该已经看到了。

北门尽数控制,其余几面的城门,想必李信也不会耽搁。有李信控制住内城,自己这边手脚慢一点就不会有问题。

伸手扇了扇扑面而来的灰土,王厚呵呵笑了两声:“不愧是火炮。”

李彦捂着口鼻,待烟尘变得稀薄,面前散落一地的残砖断瓦便映入眼中。

‘这就是火炮的威力?’

李彦低头盯着还散发着余温和烟气的青铜火炮,五炮就击毁了王府正门,换成是城门,千军万马护持下,几十门炮合力射击,也应该不会需要太长时间。

只是皇城的礼炮不论,备受世人期待、理应是保家卫国的火炮,其两次公开射击,都是以京城内的贵胄显宦的府邸为目标,就像是受到了诅咒一般。

李彦记得上一次是郭逵、郭太尉、郭枢密,这一回就轮到了二大王。想想还真是不吉利。

灰烟散尽,齐王府前院中一片血红。

门后的正堂,与正门隔了十数丈的院落。偌大的院落之中,到处是残肢断臂。

因为要防备,府中很多护卫,以及本应是看守的班直,都聚集在前院中。

几次火炮轰击之后,十几人送命,十几人在血水中翻滚哀嚎,剩下一些人则都是愣愣的站着,看起来早被吓得魂飞魄散。

郭逵家的正堂被一炮击毁,只是运气不好。齐王府的宅邸刚修起来没几年,正堂上的琉璃瓦还亮得能反光。

只是五炮之中穿过大门命中正堂的三炮,有一炮击中廊下的柱子,合抱粗细的大柱从石础上塌了下来,连带着小半边的屋檐也一并垮塌。

从倒下的大门内,能之间看到黑洞洞的炮口。

火炮的威力让人咋舌,比起霹雳炮更胜一筹,惊醒过来的齐王府人众,都失去了反抗的意志。

李彦心头一抽,就是他也没见过这么多尸体,也没见识过什么叫做血流漂杵。

但王厚见识过,见识过太多太多,远远不是区区一座小院中的区区十几具尸体可比。

拔出腰刀,直指前方,王厚厉声大喝:“进门!”

跟随王厚而来的一冇帮班直禁卫,终于等到了命令,随即踏过满地的瓦砾,直扑府中。

但冲进正门废墟,他们的行动立刻就变得呆滞起来。脚下的惨状直接冲击心灵,吓到了一众禁卫,穿过院中时,一个个都是踮着脚在血水中寻找没有染红的地方。

王厚来到门前,啧了两下嘴,反过来对身边的李彦笑,“鹌鹑寻食时倒是这么走路,够小心的。”

京中的班直,没有几个见过血。

百年来父子相承,皆以高大女子为妻,几代下来,班直禁卫看起来一个个人高马大、精壮健勇,外国使者来朝时,一见便低了一头去。

可真要上了战场,其实还不如从边州州军中随便拉出来的一名老兵。

王厚能在尸堆旁面不改色的吃饭喝酒,但不论是跟随他的班直,还是齐王府的那一批,看到满地血水和残尸,就一个个如同被吓住了的鹌鹑。

王厚没有走进齐王府中,就站在门槛上看着。脚边的瓦砾中埋了一人的尸体,看不出全貌,只有一只手伸了出来。多半是齐王府的司阍,刚刚毙命于火炮和连带的危机中,

“李彦!”

“小人在。”李彦连忙低头听候使唤。

“你进去去确认一下。”王厚毫不客气的使唤着李彦,“该捉到的一个都别让他们跑掉。活要见人,死要见尸。”

“小人明白。”李彦肃然接过军令。

齐王府中需要确认捕捉的只有寥寥数人,那些仆婢只是被连累,跑掉几人都算不上什么问题。

但片刻之后,李彦赶来回报。

这一回来攻击最重要的目标之一——齐王妃,已经自尽了。其余重要人物,包括赵颢两个年幼的女儿,以及还在襁褓中的嫡子,倒是都控制住了。

“果然。”王厚咕哝了一句。这才是正常的发展。赵颢要做的事,自瞒不过其妻,如今事败,就是不自尽,宫里面也会送酒和白绫来的。

李彦递上了一封信,“齐王妃还留下了这封遗书,说是要呈给太后。”

王厚摇摇头,没接过信封,他可没兴趣。

“你回去呈上去就好。”

反正内容只会是求太后绕过年幼的子女一条性命,要不是存了这条心,就应该带着子女一起走了。

这是赵颢的第二任妻子。

前任齐王妃并非病故,而是与赵颢多年感情不和,算是京城中有名的怨侣,最后在已经过世的曹太皇的安排下出家为尼。

不过也可以说是她的运气,这一回齐王府中,从护卫到仆佣,还有最上面的赵颢一家,没有人能逃得掉,只有那位下堂妇或能逃过了一劫。

安排了人手将重要的俘虏先行押回,剩下的仆婢护卫,还有一些不相干的门客,则从对面的赵頵家里借了几间屋子来关押。

赵頵还没有回来。南班官这时候能放出来就有鬼了。全都是赵家人,在局势未平之前,怎么可能让他们出宫?若是有一二宗室为人裹挟,甚至拥立,京城不知要添多少乱。

没有主人在家,王厚敲门进来,硬是逼着他们借出房屋,还要担着罪囚逃跑的危险。只是没人敢反对,不论是这一家的女主人,还是朝廷和宫里安排的官员和内侍官。就是赵頵的乳母想要倚老卖老,被王厚扫了一眼,立刻就老实了下来。

飞快的安置好了齐王府的俘虏,王厚便叫来了今天的搭档:“李彦。”

“小人在。”

李彦在王厚面前愈加谦卑,王厚本身就要在皇城中任职,还别说他与韩冈和王中正等人的关系,轮不到区区一小内侍不敬。

“去蔡相公府。”

“……不救火?”

李彦望着开始在府中蔓延的火势,惊讶的问道。

烧光了事,王厚想着,免得藏在里面的书信害人,还牵连到自己被人怨恨。

“没那个时间。跑了叛逆亲眷怎么办?”他反问道。

‘要跑早跑了。’

王厚在齐王府这边耽搁了太多时间,但李彦不敢说出口,只能低头,“小人明白了。”

王厚言出而行,随即上马前往蔡确府邸,班直护翼左右,炮兵跟随在后。在背后留下了熊熊火焰蔓延的废墟和一群看客。

王厚冷笑着,这么长的时间,蔡确家里的子女亲族,这时候也该有些人逃出去了。

韩冈冇虽然没说,但王厚之前与李信定计,都是集中兵力先攻一点,但依照计划还是要分出几十人,去蔡府守着大门。

可王厚偏偏只派出了十来人。这么点人,当然看守不住蔡确家的围墙,里面的人想要跑出去,只要翻过一道墙。

不过这些人就是祸害,逃去谁家,谁就是叛党。

而蔡确家人,能逃去的地方,京城中又能有几处?

王厚尽管刚刚进京,但有关韩冈不能出任宰相的消息,早就传进他的耳中。

蔡确嫉妒韩冈,担心韩冈动摇他的地位,指使族亲蔡京弹劾韩冈。使得韩冈不得不立下毒誓,自证清白。

这件事,在关西早都传遍了——攸关关西士子未来前途的大事,由不得人不在意——事发后还没半月,就已经到了王厚的耳中。不知多少人对蔡确恨之入骨,也包括王厚一个。

这一回宫变,韩冈立了大功,王安石等宰辅还是跟在他身后。而且两府一下空出了三个位置,肯定要填人进去。

在内外皆安的太平时节,少几个宰辅多几个宰辅都不是什么大事。而如今正逢乱局,尽早填补上空缺,朝廷也能尽早安稳下来。

王安石要是做回平章军国重事,朝廷内外就比较容易稳定下来了。而韩冈,若是不入两府,那就太说不过去了。

而且之前安排朝堂人事,都有先帝赵顼在背后控制。而这一回两府出缺,则将是向太后乾纲独断。比起一直压制韩冈的先帝,太后对韩冈要倚重得多。

唯一的问题,就是韩冈之前立下的誓言。

韩冈受权臣陷害,为誓言所困,从此不得不遭人钳制。王厚身为好友和姻亲,又岂能眼睁睁的看着韩冈难在宰相位置上坐得安稳?自然要为其分忧解难!

一路赶赴蔡确府上。

蔡府内外早就得到了消息,开门等候发落。不过看守不严,仆佣也有不少人逃走,而蔡确之弟蔡硕,其子蔡渭,都已不知去向。

王厚没再耽搁,丢下李彦处置余事,抓了一名识路的班直,领了二十余人直奔蔡京家而去。

不论蔡硕、蔡渭是否投奔蔡京,先去将蔡京抓起来再说。拷打一番,死了也就是死了。这时候,还能有谁为蔡京喊冤?

快马而行,将及蔡京家宅的路口,领路的班直突然一下拉住缰绳,惊叫起来,

“蔡衙内?!……是蔡渭那叛逆!”

‘蔡渭?’

王厚闻言一惊,也跟着一把扯起缰绳,勒停了坐骑。

只见几名家丁装束的汉子,将疑似蔡渭的男子五花大绑,押出了巷口。后面跟着一名相貌俊逸的官员。

他望着那名叫出声的班直,又看看穿戴明显不同的王厚,拱手一礼:

“在下蔡京。刚刚擒获了这名叛逆,正要押送去开封府投官。”
