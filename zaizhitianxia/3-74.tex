\section{第25章 闲来居乡里(七)}

看到纺纱工坊中的一架架机械,众掌柜们都忍不住连连点头,甚至有几个动手去转动纺机,看一看到底是怎么去运作。都想着回去后,就学着给打造出来。

“不过诸位家中的纺纱作坊最好留在陇西,熙河路毕竟人少,又是偏僻,而且戒备森严,外人不容易混入。放在秦州可就不一定了。若是其中的奥妙给关东的人学去了,我们可是谁也争不过。”

“不必冯掌柜来提醒,我等哪个不明白!”成轩等人异口同声。为了守护自身利益,谁也不会犯傻。

冯从义说得没有错。工坊如果留在熙河路中,纺机改进的这个消息传出去就已经很不容易,等传到南方织造商们的耳朵里,派人过来打探消息,更是不知要过去多久。如果是放在秦州这个西北重镇里,消息散布的速度,却是陇西的十倍。说不准哪一天,就全给人一股脑的学去了。

用了新式纺机之后,棉纱的成本大大降低,赚到手上的钱财谁会嫌多?前面在棉田中多付出的钱钞,在这里却补了回来。

见着成轩他们眼中的喜色,冯从义知道今次的事算是成功了。韩家纺纱工坊的效率瞒不了人,只要有心打探,迟早都能打探得出来的,又不是多难造的机械,多看几眼就能学走,还不如早一步拿出来做人情。

只是多保密一年就是成千上万贯的收入,有几人能放得下?韩冈却是没当回事的就丢了下来,这份心胸和眼光,让冯从义敬佩万分。

成轩也走过来,想着冯从义一揖,正色道:“韩太常的心胸,世人难以企及,我等实是感佩万分。”

韩冈的名声如此之大,累累功绩更是惊人,谁也不会把他当成一个不识货殖的傻瓜。这么大的利益,说让就让,实在让人敬服。

而见着纷纷过来要自己代为想表哥转达敬佩之意的商人们,对于韩家,对于顺丰行,对于韩冈本人,对于他们的未来,冯从义更加的信心十足。

……………………

冯从义回来的时候,韩冈正看着张载的来信。

韩冈从京中回来时,也没忘了探望二程和张载。虽然他成了王安石的女婿的这桩事,的确有些让他们不太喜欢。但在他推荐关、洛两家入经义局共参诸经新义的消息传开之后,这桩婚事给张载和二程留下的心结,也就烟消云散。

对于韩冈通过实验推导出来的理论,当日回来时经过横渠镇,已经跟张载讨论过整整三天。现在又是书信往来,不再是韩冈,连张载也有心要将气学和韩冈的理论完全融合起来。

见着表弟回来,韩冈收起了信。让了冯从义坐下,道:“今天可是辛苦了。”

“倒也算不上辛苦。”冯从义摇摇头,又道:“与承恩村的合同都已经定下了,都没有意见。之后协议每年一签,具体的条款在签约前,会在行会内事先加以沟通,以防有人抬价收购,乱了行规。”

“那行会怎办?”

“也就在这两个月,过些天我再去秦州一趟。”

“再跟他们多说一句,这门生意是要做上几十年的,赚一时,不如赚一世。不要因为一时的贪心,坏了日后合作的可能。”

冯从义笑道:“表哥是在白担心,都是生意人,这个道理相信他们都懂。”

生意场的本质虽说就是利益,但也是要讲人情和信用的,不可能赤裸裸的利益争夺或是交换。即便是后世,人脉多寡还是衡量一个业务员水平高低的重要标准。交情和关系,往往抵得上几千几万贯的投入,而信誉更是重中之重。

“白担心那是最好。”冯从义的话,韩冈不以为忤。想了想,他又道:“下次讨论成立行会时,不要忘了把王家给拉上。今次没有带上王家,还有些说道。但到了组建行会是还不带上王家,脸面上可磨不开。”

“那高家呢?”冯从义问道。

“……至于高家,等行会准备成立之后,再拉进来不迟。”只要在行会成立前,将两家拉进来,即便有芥蒂,也不会有太大问题。

今次组织商人去订立棉花购买合约,韩冈并没有知会一起掌控熙河路商贸往来的王、高两家的商行。不是韩冈不带两家玩,而是那些商人没有几个愿意跟枢密副使和太后家一起做买卖。只有现在跟韩冈敲定了合作之后,才有胆量接触王、高两家。

虽说王韶、高遵裕现在都离开了熙河路,但各自都是升迁。王韶的枢密副使就不用提了,高遵裕则是去了河东路,比起新成立的熙河路的副总管,河东路兵马副都总管,明显要高上一两级。两人虽然离开,可留下的阴影则更为庞大。

与他们这样的庞然巨.物合作,谁都害怕自己的一份被吃掉。以怡和号为首的多家商行在秦凤路是地头蛇不假,在秦州,他们也不怕王、高两家商行在商业上的竞争。可棉布这门生意是要做到京城里去的,在东京城中,秦州的地头蛇只能算是黄鳝,而强龙依然是强龙。

要不是韩冈一向不独食,今次的表现又足够大方,开出的条件更是让人无法拒绝,也没人愿意跟宰相家的女婿一起做生意。齐大非偶这个成语,不论是在谈婚论嫁上,还是生意场上的合作,都是有几分道理的。

韩冈不怕王家,也愿意与王家分润利益,但他却着实担心太后家的胃口,不忘再三嘱咐着表弟:“不要想着靠你的岳家,那是条鳄鱼,能将所有的份全都给吞下去,一根骨头都不会留给人。”

冯从义是高家的女婿,但只是远支而已,真要让高家独占东京城棉布市场,他的岳父岳母也占不到多少好处。而冯从义也知道他的根基在哪里,点头应道:“小弟明白。”

“如此就好。”

韩冈不再有什么担心,但冯从义却还有着一份隐忧:“只是今次拉了这么些人进来,摊子铺得如此之大,若是不能三五年内有足够的棉田开辟出来,到时候,行会也很难维持得住。”

设立行会的目的是赚钱,今次秦凤、熙河几家商会要成立行会,共襄盛举,便是为了棉布的潜在利润实在太大。若是就一年几十万贯让各家来分,那个愿意付出如许大的精力。就是秦州城内的粪行,一年还有十来万贯的周转。

“不用担心。”韩冈对此都有考量,只是没有跟冯从义说,“先不说日后的移民,眼下有了广锐军为首的汉人弓箭手作为榜样,就可以引诱蕃兵弓箭手们群起仿效。只要他们种好了地,日后又是一个棉花的重要来源。”

“蕃人?”冯从义惊问着。

“当然是蕃人。不给他们一个种田赚钱的路,日后他们开始多种粮食可就麻烦了。”

可以参考一下,几百年之后,西方列强控制下的殖民地,是如何发展和稳定下来的。主要靠的就是单一化的经济生产,将殖民地纳入自己的产业链中。所以当殖民地独立之后,从原有的产业链中脱离,国家经济会有一个暴跌的时期,能否再恢复,就看各国自己的治理水平了。

韩冈的想法就是要引诱开始农耕的蕃人种植棉花等经济作物,就像如今朝廷引诱游牧为主的蕃部大量的养马用来交换茶叶。

一旦熙河蕃部都被归入到大宋的经济圈中,生产生活全都离不开大宋的商业活动,就算有人唆使他们反叛朝廷,也会被他们给反过来打翻掉。

这个时代可没有民族独立的潮流,而是四方蛮夷都对汉家文明顶礼膜拜,以至于有传言说,契丹的皇帝要在银佛背后刻下‘愿来世生中国’的字样——虽然韩冈不知此事是真是假,而‘中国’是不是指得大宋。但从吐蕃贵人对汉物的喜好,以及如今对大宋官员的敬畏中,还是能看出一二来。

“按照朝廷的规定,给予归顺的蕃兵弓箭手的田地是一百亩,小头目两百亩,大头领是三百亩。但蕃人不会种田,漫种薄收,勉强糊口而已。如此下去,当然会难以安定下来。若是能让他们变成靠着种植棉花来赚钱,必定能吸引其他蕃人陆续投效,下山种田。”

“那还要靠着姨父来指点他们了。”

韩冈点头,这也是给自己父亲韩千六一个发光发热的机会。

现如今韩千六在移民中名望很高,尤其是本不擅稼穑的广锐军,他们若没有韩千六所在的屯田务的帮助,来陇西的第一年就要绝收。不过在蕃人中的地位,韩千六就远远不如自己开创了疗养院、为吐蕃贵人们治病疗伤的儿子。若是韩千六出手帮助他们学着种田,蕃人弓箭手们的生活富足起来后,也必然念着韩家的好。

如果父子两代都能与蕃人结下深情厚谊,后面韩家的几代人,都能从中得到极大的好处。

与韩冈就棉纺业的前途说了一通之后,冯从义起身告辞。

韩冈送了他出去,回来后,躺在躺椅上回想着整桩事情是否有所疏失。

棉纺上的事情解决,今次回乡,该做的事都做完了。再歇息些日子,便要离乡返京。

不知这段时间,京中的朝局变成了何等的情况。

