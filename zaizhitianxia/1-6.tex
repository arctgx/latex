\section{第四章 异世缘从天地成(全)}

“三哥哥。”熟悉的甜糯声音从厢房外响起,打断了贺方的回忆。平和的笑意随即出现在他脸上,“是云娘罢!你自进来好了!”

韩云娘应声倚着门倒退着进房,手上捧着个食盘,上面摆了一口小砂锅,还没开盖,羊肉小米粥的香气便已经冒了出来。

“不知刚吃过吗,怎么又端来了?”贺方问道。

“都已经过午了。”小丫头轻笑着,粉色的双唇中微微露出的一排皓齿如同编贝一般整齐雪白,很难想象光靠柳树枝就能把牙刷得这么白。她轻手轻脚的将食盘放在书桌上,顺手便收拾起被散放在桌案和书架上的书册。

“过得这么快?”贺方觉得自己只不过读了读书,又陷在回忆中一阵子,怎么一眨眼的功夫便到了中午。

“三哥哥你读书入了迷,当然不觉得。”韩云娘手脚麻利的得很,三两下的功夫,凌乱的桌面便被收拾得整整齐齐。就在书桌上打开锅盖,又把木勺放进锅中,小丫头转过头来扶着韩冈坐下来吃饭。

贺方坐在桌前,低头看着眼前热腾腾冒着香气的小米肉粥,前世在社会上摸爬滚打而被煅炼出来的一颗坚如铁石的心脏,竟然有些抽紧。

此时农家的习惯都是一日两餐,早一顿,晚一顿,闲时吃稀,忙时吃干,每日都是勉强填饱肚子。但贺方刚刚占据的这具身体久病虚弱,现在便是一天三顿的将养着。每天三四个鸡蛋,一斤煨得烂熟的羊肉,还有浓浓的小米菜粥,父母不惜家财,养得贺方一日【和谐万岁】比一日康健。不过他现在是知道了,每天吃得这一日三餐,究竟是怎么换回来的。难怪家中一点田地都不剩,每天父母仍要一起出去,而后很晚才一身疲惫的回来。

“怎么了,三哥哥?快点吃啊,冷了就不好了。”韩云娘看着贺方坐着不动,小声催促着。

贺方摇摇头,放下心事,现在他的这副身板操什么心都没有用。他对站在一旁准备服侍自己吃饭的小丫头笑道:“过来一起吃罢。我一顿也吃不了这许多。”

韩云娘白皙的小脸噌的一下红了起来,受到惊吓一般的向后退了小半步。她不知何为司马昭之心,但她的三哥哥的心思却是清楚明白。自从病愈之后,三哥哥就一改过去的严肃,常常轻薄于她。跟三哥哥更加亲近,小丫头的心里自然是千肯万肯。但耳鬓厮磨的亲昵,已经渐知人事的韩云娘总是羞涩不已。

她秀丽双眸盯着脚上的绣花鞋,不敢看着韩冈,声音细如蚊子哼:“还是三哥哥你多吃点,才能早日好起来。”

贺方看着那一抹艳丽绯红,少女瞬间绽放出来的娇羞让他目眩神迷,原本沉重的心情不由轻松了许多。抽空就调戏一下温柔体贴的小萝莉,对他的精神健康很有好处。

贺方欠起腰,把韩云娘一把扯了过来,“我在吃你在看,这样也没滋味,两人一起吃才香甜。”他手上用力,却想把小丫头拉着坐在怀里。

父母在外吃苦劳累,自己却在家中搂着小女孩儿吃饭。这倒不是贺方没心没肺,而是他很清楚,回报父母的最好办法,就是尽快恢复健康,不论身体还是心情。如果硬是要跟父母一起吃苦,拖延了康复的时间,只会让他们的辛苦操劳失去了意义,那反而是不孝。贺方并不是矫情的人,既然觉得做得对,就不会再考虑其他。

被贺方强拉着手,韩云娘小脸越发的殷红如血,用力挣扎着,怎么也不肯坐下。看着不能得逞,贺方半带调笑的凑在小丫头晶莹如玉的小耳朵边低声说着,“爹娘都出去了,家里就我们两个。”

滚热的呼吸透入耳中,小丫头连耳根都热得通红,挣扎也不由软了下来。但还是不好意思坐在贺方怀里,只侧着身子坐在了贺方的身边,被他一手搂住了纤腰。

灯下观美,自有一番风情,而到了白天,小丫头的娇俏可爱更是遮掩不住。尤其是一双眸子,黝黑深亮,羞涩时,眼皮低垂,长长的睫毛掩住双眼,如同深潭般幽深,开心时又会闪亮起来,配上无邪的笑容,编贝般的皓齿,几乎能把人的魂魄都陷进去。她身上穿着的粗布襦裙半新不旧,虽无损她的容色,只是让贺方看得有些心疼。

按照此时的习惯,婢女称为养娘。而在韩家,小丫头不仅仅是做养娘,其实还有一重童养媳的身份在。也不一定是贺方身体的旧主,一开始韩家父母的打算,就是韩家三兄弟如果日后有哪个娶不上媳妇,就让小丫头配给他——其实,这也是关西乡村里惯常的做法,单是下龙湾村中就有十几家里养着童养媳——等韩家老大娶亲,韩家老二从军之后,就指给了韩冈,只是现在则全便宜了贺方。

韩云娘本人自是知道韩家父母的打算,现在却也是把三哥哥当作自家的良人看待。贺方病愈后对她的亲昵,她半是羞涩,却也有几分欢喜。

贺方搂着小丫头温软纤细的身子,你一勺我一勺,两人花了半个时辰方分着把一锅羊肉小米粥吃完。

吃过饭温存了一阵,小丫头跳起来收拾碗筷,贺方则整了整衣冠,徐步踏出门去。他的身子渐渐恢复,已经不需人扶,也可自行出门散步。每天出外走走,虽是感觉着有些累,不过贺方还是坚持着一天比一天多走上一段路。唯有加强锻炼,才能早日恢复健康。在没有抗生素的时代,要对抗疾病主要还是得靠自己。这几天他都是到河边走上一阵再回家,以培养体力。

行走在村中的土路上,贺方借助散步重新熟悉着周围的环境,顺便寻找可以发家致富的道路,让父母不至于那么辛苦。

从小就表现出读书天分的韩家三哥,在小村中很受敬重。在路上遇到,村民们都是先上来嘘寒问暖一阵,让贺方感受到了一丝暖意,而贺方亲切有礼的回应,也让村民们感到惊喜,都道韩家三哥越来越有读书人的气度了。

一路上,他不停与相熟的邻里打声招呼,虽然从邻人惊讶的神情中,贺方进一步体会到过去的韩冈的确不是亲切待人的性子。不过韩家老三到底是在外游学了两年,回来就就病倒,还没来得及与村人打上交道。贺方与前身的不同完全可以推到两年的时间上去,并不至于会让人疑惑。

走了一阵,已经能听到哗哗的流水声,饱含着水意的空气也扑面而来。下龙湾是个不大的村庄,位于两山夹谷之中,村北远山其色苍莽,村南山色苍翠,哗哗的河水水流声则从村子北面传来。那条河名叫藉水,河对岸便是秦州州城。藉水向东流淌,过了百里之后便汇入渭水——也即是渭河。如果没有党项人的威胁,这里其实是一个很宜居的村落,但既然其位于边塞,便也免不了要日夜担惊受怕。

“毕竟是北宋啊……”贺方暗叹着。若是后世,陕西那是中国腹地,根本不需要担心外患的地方。在那个时代,自家只要安安分分的做事便也不会有什么危险,战乱是个陌生得只能在新闻和书本看到的名词。但在此时,却是他实实在在要面对的问题。

陌生的天空,陌生的土地,以及陌生的时代……贺方的心情忽然有些低落,不意被脚下的石头绊了一下,脚步一个踉跄,差点便要栽倒。但一双小手将将好从后伸来,将他给扶住。

“三哥哥,小心一点。看着脚底下……”

“嗯……”贺方应了一声,回头看看,韩云娘不知什么时候跟了上来。一双会说话的眸子正担心的看着他。

对了!至少还有家人。贺方侧头看着小心翼翼搀扶着自己的小丫头。在这个时代,还有应该陌生,心中却怀着一份情谊的家人。

‘也罢……既来之,则安之……’贺方心中说不清到底是无奈,还是认命。一越千年。天意如此,纵使不甘,又有何能为?

“既来之,则安之!”站在潺潺的藉水边,扶着少女的肩膀,远眺着对岸的城池,贺方再次重复着。深秋的熏风沿着河面拂来,不知从何处带了一丝甜甜的桂花香气。宽大的青布襕衫随风飘动,削瘦的身子却稳稳地站着,没有一丝动摇。

尽管贺方很想重生在一个富贵家庭,但能再活一次已是天大的机缘,凭空多出来的一条性命更值得珍惜。何况还有关心自己的家人,贪求太多恐怕要天打雷劈了。贺方很看得开,可以说是豁达,既然莫名来到这个时代,也无从得知该如何回到二十一世纪,他现在所能做的也只是让自己和这里的家人过上更好一点的生活。而第一步,便是抛弃旧日的自己,接受新的身份:

“……我是韩冈……我是韩玉昆……”

(从本章之后,主角的名字变为韩冈)

PS:好了,主角终于认清了现实,贺方消失,而即将改变历史的韩冈诞生了——虽然他现在还在调戏小萝莉。

ps又ps:和谐万岁。

