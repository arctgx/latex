\section{第六章 见说崇山放四凶(八)}

“枢密,回来了。”

“倒是不慢。”

在城头上向外眺望的士兵回过头来,郭逵应了一声,起身下城。

拾级而下,当郭逵从城头上下来,王中正和韩冈一行已经抵达宣德门外。

宣德门的侧门吱呀呀的打开。

火光摇曳,在赤红的光线照耀下,可以看见王中正和韩冈都对郭逵出现在门洞里吃了一惊。

“郭太尉!?”

“枢密?”

“玉昆,你来得到不慢。”

郭逵笑道。他知道王中正是去做什么。

王中正出宫的时候,也经过了一番询问。郭逵亲自下城来问过王中正,方才放了他出门去。

韩冈早下了马,与王中正一同走进城门门洞中,到了郭逵近前,行礼后便问:“枢密怎么下城来?”

放在这个节骨眼上,郭逵再严格,也没人能说他什么不是。只是坐镇中军帐的大帅来营寨门口检查出入,不可能不吃惊。

“今晚不同,不亲自看着,就放心不下。”

郭逵说着,冲韩冈颇有深意的笑了笑。

韩冈怔了一下,随即了然,点了点头:“多劳枢密了。”

郭逵的确老辣,有他亲自迎进来,自己就没什么好担心了。

郭逵转身陪着韩冈和王中正往城内走:“张希参那边可就要拜托玉昆你了。”

“以张老太尉于韩冈的恩德,韩冈岂敢不用心?太医局中最擅外科的医官早前就都入宫了。没有伤到脏腑,只要能熬过感染一关,张老太尉很快就会痊愈了。”

郭逵和韩冈说话,王中正一言不发,也不催促。

不过韩冈自己心中有数,知道不能多耽搁,与郭逵说了两句,待出了深长的门洞,便与郭逵告辞。

“对了。”王中正离开前却问郭逵,“太尉,方才还有谁出门?”

郭逵摇摇头:“没有。”

“这样啊。”王中正冲郭逵拱了拱手,“多谢太尉告知。”

韩冈心中又是一片疑云。

王中正出宫之后,太后竟没有再派人请王安石入宫?!

不过他没有将惊讶再表现出来,与郭逵告辞,便往深宫中行去。

深夜的宫室,不同于白天雄伟华丽,更让人觉得阴风惨惨,寒气逼人。殿阁之下的灯盏,只能照亮周围的一小片区域,大片大片的黑暗笼罩在宫室上。

每一次看见夜里的皇宫,韩冈都会觉得这里的确不是一个好住处。难怪日后会有圆明园和颐和园,不论哪个时代的皇宫,恐怕都是阴气过重,不宜人居的地方。

走了几步,韩冈却发现方向并不是往宰辅们宿直的地方过去的,也不是天子的寝宫。太后的寝宫当然更不可能。而是一处在国家政治中,地位十分特殊的一间殿宇。

“内东门小殿?”

天子要拜除宰辅,都会前往内东门小殿,宣翰林学士来书写诏书。向太后于此召见,正常来想,也该有着极深的政治意味。

王中正脚步不停:“太后就在内东门小殿中等候。”

“太后若要拜除,应该招翰林学士才对。”韩冈轻声道。

“究竟是何事,还请东莱郡公询问太后,岂是中正可以多嘴多舌?”

韩冈与王中正交情颇深,对话也不像普通的宰辅与内侍般,充满了隔阂和歧视。既然王中正这么说,表明他也不清楚是什么原因。

韩冈不再多问,与王中正快步而行。

心中则揣测着,韩绛、章敦几位,此时会不会也在内东门小殿中。

当韩冈抵达内东门小殿外,王中正进殿通禀时,结论出来了,没有。

透过敞开的殿门,韩冈并没有看到韩绛、章敦他们,只有向太后人在殿中。

是已经召见过了,还是在自己进来之前,太后根本就没有召见韩绛他们?

得到里面的通传,韩冈一边猜测着,一边跨进殿中。

……………………

‘差不多该到了。’

如果太后当真派人去传韩冈。

章敦不清楚太后到底有没有这么做。但有不冤枉的司马光在,章敦确信,向太后当会去招韩冈入宫。

只是现在不可能让人去打听证实。

内宫中再私密的消息,从宫里面传出来,也跟水透过渔网差不了太多。宰辅们想要去了解,渠道多得是。消息灵通与否,差别只在迟早。但在明面上,打探宫内阴私,却是不能触犯的禁忌。

而且章敦也无法确定太后招韩冈,会是什么事?

只能猜猜会是在哪里召见韩冈。

首先不可能在内宫中。先帝尚在时还好说,可如今没了男主人的家宅,哪里能让男子夜中进出?

难道是内东门小殿?那还真的不妙了。

韩冈的打算,章敦怎么想都觉得不妙,只是没办法对外面公开。

章敦可从来都没觉得韩冈是半途而废的一个人,打定了主意之后,都会千方百计达成目的。

韩冈对赵煦的坚持,很难说不是因为他的目标,而这一回的宫变,便是由此而起。

这样的韩冈一旦重归两府,在外又没了蔡京的牵制,以他的能力,日后不知会将朝堂给闹成什么样。

章敦彻夜难眠,张璪也同样无法安睡。

只有韩绛找了个理由先去内间睡了,只是不知他到底能不能睡着。

苏颂自己也没睡。为了观测天空,他习惯了晚睡,甚至彻夜不眠,只在白天抽出一点时间补觉。

对坐立不安的章敦和张璪,他都觉得好笑,

以韩冈的功劳,受到重视不是理所当然的吗?

不过章敦究竟是什么时候开始与韩冈生分的?一点都没有感觉到征兆。

苏颂推开窗户,涌进室内的寒气,顿时让人睡意尽消。

不过天上的星星又看不见了。

苏颂失望看着无光的夜空。

冬天的东京城,日月星辰总是比其他地方要黯淡许多。就连晴日天空中的蓝色,也是蒙了一层灰,远不比上记忆中的澄清通透。

什么时候才能有一个好天气。苏颂想着。

……………………

只隔了一个时辰,韩冈重新来到太后驾前。

换了一身日常的公服,行动也轻便了许多。只是心中疑惑难解,却远比脚步要沉重。

再拜而起,得到了太后赐座,韩冈坐下后就问道,“不知陛下漏夜招臣入宫,可有何事?”

“辛苦韩卿了。今日是吾的不是,以为卿家今晚应该在宫里宿直。”

听到太后这么说,韩冈一时间都不知道该怎么回复才合适。

说自己不是宰辅,所以不能留在宫中?这感觉就是在求官了。今日之事,太后或者是无心,但他却不能不多心。

韩冈正在斟酌着怎么回覆。就听向太后又说道:“今日多亏了韩卿。若非卿家,吾母子性命不保。卿家于吾,是救命之恩。”

韩冈站起身:“这是臣的本分。”

“卿家安坐。”向太后让韩冈坐下,叹道,“可满朝文武,能尽到这个本分的不多。”

韩冈头疼了起来。这话本没什么,就是当着众宰辅的面说也一样。可现在,宰辅们都在宫中,却单独召见了自己,就架不住有心人要联想了。

“未能尽到本分的,也就区区数人。罔顾圣恩者,毕竟是少数。”

韩冈如此说,屏风后的声音,也不再追究,问道:“两府里面的那三名逆贼,一个死了,两个流放。不知韩卿觉得该怎么办?”

怎么办?

韩冈微微一怔,这让他怎么说。

白天的那么多话是白说了吗。不都是在说之后怎么办?

想了想,道:“一如既往便好。稍待时日,陛下可以静观有何不尽如人意之处。”

“卿家话的确有理。不过吾觉得国家大事,不宜耽搁延误,得尽早弥补。两府阙额,卿家自是其中一人,剩下的两个谁比较合适?”

终于明白太后想说什么,韩冈心中顿时叫苦不迭。

这话若是正常的出自天子之口,他说不得就得跪下来请罪,或是自证清白。这明摆着就是皇帝的猜忌。但出自向太后口中,却又不是那么一回事了。

不过韩冈也不可能一口答应下来,然后推举谁谁谁上来填补空缺,更不可能大喇喇的说一句舍我其谁。

“请殿下圣心自断,此非是臣等可以妄言。”

“卿家尽可直言,吾素知卿家为人,不须顾忌。”

韩冈口中发苦,这不是难为人吗?

进退宰执,这个权力太烫了,韩冈现在还拿不到手上。真想要应承下来,立刻成为众矢之的。

当然,他不是没有想法。

只是现在的情况太过顺利了,让他怀疑起是不是章敦私下里跟太后说了什么?不过只要自己看不出私心,就无所谓。

沉吟了一下,韩冈说道:“陛下可知御史?御史之用,在于绳纠百官,威慑宰辅,使人主耳目不为权臣所蒙蔽。所以御史进用,其人选便不能由宰执议论,而是御史台与内翰共荐。”

当御史台有空缺之后,就会由御史台的正副手——御史中丞、侍御史知杂事,以及翰林学士来推荐人选,由人主在其中挑选合意的人选。

韩冈相信太后肯定知道这个规矩,所以他说道:“所以陛下既然属意微臣,那两府阙额,便不宜再由臣推荐。”

