\section{第48章 斯人远去道且长(一)}

【今天第三更,求红票,收藏。关于茶壶、壶盖和茶杯的票选终于结束。鉴于大多数书友的选择,壶盖路线宣告终止,韩冈的未来还是在复数的茶杯环绕之中。】

同样的夜色下,有人拥美邀醉,但也有人伴着孤灯,守在空寂的公厅中。

吕惠卿今天正好轮值,孤身守在他的官厅里,外厅中倒是有两个老兵,本是为了服侍署中值守官员,而派在官厅处听命的。不过他们现在早蜷在火盆边,快活打起呼噜来了。吕惠卿无意将他们唤醒,要睡就让他们睡,等到需要时再叫他们也不迟,反正他现在还学不来文彦博的手段。

那位枢密使当年在成都任官时,逢着冬日大雪,便兴致大起,没日没夜的摆酒赏雪。守卫士卒又冻又累,吃不住了,就拆了亭子烧来取暖。文彦博当时没有发作——真要发作了肯定会惹起兵变,蜀地兵变是有传统的——而是让人继续拆亭子。但到了第二天,秋后算帐的时间到了,为首的几个全被他拎出来杖责发配。

吕惠卿也坐在火盆旁,手上拿了份公文在读着。火盆里的贡炭闪着蓝幽幽的火光。由柏木烧制成的贡炭,燃烧时没有多少烟气,外面是买不到的,倒是两府中年年都有供给。虽然已经渐渐入春,但天气还是昼暖夜寒。抬头看看承尘上几处透风的缝隙,吕惠卿不由暗叹,白天时,有太阳晒着,还感觉不出来有多冷,但到了夜间,一阵寒风从缝隙中透进来,穿堂过户,便能把人的手脚都一起冻得冰凉。

政事堂的几十座楼阁,无一例外都已经有上百年的历史,皆是年久失修,而且当年修造的时候,就只注重着外表光鲜,这保暖的问题从来就没有考虑过。每年到冬天都会有人抱怨不迭,说一定要整修一番,可只要天气稍暖,这一茬马上就没人提了。

并不是没有钱去修,虽然请朝廷划拨,会有好事的御史出头骂上几句,但各司账面上的公使钱,还有一些私底下的结余,把官厅修缮个十遍八遍都是够的,不过各院厅的主事不是想着各自分肥,就是转着一起去樊楼等上等酒楼好好快活一下的念头,除非被火烧了房,不然谁会把钱用到官厅上?

反正依照故事,在京诸司里,没哪人能守着一个位置几年都不动弹,小吏或许还有可能,但官员绝对不会有这种情况,多是一两年就换了位置。就算开始修缮公厅,倡议者自己肯定是享受不到,或是享受不久,等他调了职,新上任的地方多半会有几个漏风的洞在嘲笑他为他人做嫁衣裳。既是如此,又有谁会去做这等自家种树他人乘凉的蠢事?!

朝中都是这等混吃等死的庸碌之辈,也难怪新法推行如此艰难。吕惠卿把手中的公文丢到身后的桌案上,又是一份诉说青苗贷伤农的奏章,但通篇没有一处提到实据,亏上书的还是个知县。这等人,在韩、吕一派中,怕也是是走卒一类。

门外廊道上,突然响起了一阵脚步声,夺夺的木底鞋敲着廊道地板,在公厅的门口停下。吕惠卿心中一动,暗道;‘这下可不好了。’

“吉甫……”果然,曾布先叫了声门,径自推门进厅,当他看到外厅中的呼呼大睡的两个老兵,便立刻大发雷霆:“尔等还不起来?!官长熬夜值守,尔等怎敢偷懒!”

外厅中登时鸡飞狗跳,两名老兵被惊起后,见势不妙,当即就跪了下来,没口子的认罪求饶。

吕惠卿听得吵得慌。自家仆从,他一向管束甚严,但听候使唤的老兵,觉得不好就换一个,何必吵得失了身份。他对外厅提声问道:“今天不是子宣你轮值吧?怎么有闲来此?”

曾布丢下两名老兵不理,走了进来,很不高兴的说着:“吉甫,你也不管管?”

“误了事自然会治他们的罪!”吕惠卿平直的回了一句,又一次问道:“子宣,你怎么现在还留在衙里?”

“相公交代下来的事,要赶着办完,待会儿就回去。”曾布几句话解释了原委,可能是感冒了的缘故,他说起话来有些瓮声瓮气。

两名老兵这时战战兢兢的走了进来,对着吕惠卿,又扑通一声跪下请罪。吕惠卿不耐烦的往外挥了挥手,示意他们退下去,“今次就不罚你们了,下次再犯,就是两罪并罚。”

老兵们千恩万谢的退了出去,曾布找了绣墩坐到火盆旁,烤起手来。嘴里抱怨着:“子厚倒是会享受,到了休沐之日,还真的就不来了。”

“他是为韩玉昆饯行去的。”吕惠卿用火钳往火盆里添了几块木炭,看着火苗重新旺起,他问着曾布,“明天去不去送他?”

曾布摇摇头:“哪有那个闲工夫,已经让人送了份礼去驿馆里……相公大概也不会让仲正去送行,多半也是送份盘缠,尽尽礼数。”

吕惠卿深深叹了一口气,道:“谁让相公觉得韩玉昆锋芒太盛,不宜赏誉过重?须先磨他两年性子,而后方好大用……其实相公本不会有这个想法,如果韩冈不是说了最后那段话的话……”

其实吕惠卿也是觉得暂时压一压韩冈比较好,少年早早得志,对他日后并无好处。而且韩冈做事定计并不顾后果,王相公担心他日后会走偏了路也不是没道理。不过韩冈的策略虽然后果堪忧,但好处也是显而易见。

那天韩冈在王安石府上说了那么多,事后吕惠卿归纳起来了三条内容:改青苗贷之名;以重禄养吏;曝韩、吕之辈私心;这三条,吕惠卿都有打算陆续施行。

第一条其实已经做了,因为这是最容易的,也是最不会有反对意见的。虽然司马光昨天听到消息,今天就上书说,这是意图消去青苗贷局限于农家的本意,以求进一步盘剥坊廓户的阴谋,但朝堂里,还是嘲笑的声音更大一点——尚幸有司马光这等眼光的聪明人并不多——只是文彦博应该也看透了,不过他位高权重,不会第一个跳出来,但明天多半也会上书。

给低层官吏添支俸禄的这第二条,则已经在筹划之中。都已经过去半个月了,三司那边还没计算出给在京诸司的公吏增加俸禄,到底要耗用多少钱钞。以这个进度来看,要等他们拿出全国四百军州两千余县的数据,怕是要到明年后年了。

至于第三条,就是让王安石觉得该好好磨砺韩冈性子的那一条,也是会将朝局转变为党争的一条。真的说起来,现在只有跟韩冈性子相似的章惇,始终对韩冈赞赏不已。而吕惠卿自己不提,他面前的曾布可是变得很不喜欢那名秦州来的选人。

曾布冷哼了一声,只是他鼻塞得厉害,倒像是在打喷嚏,“他是唯恐天下不乱。相公要压他几年是一点也没错。韩冈此子,可用于外,却不宜立之于朝。年纪轻轻,心机就这么深,日后还了得?”

吕惠卿对韩冈的评价则有另外一份看法:“若是心机真的够深,最后一段话是不会说的。他就是求进太速,反而落了下乘。那天我看相公的神色,可是喜欢得不得了,本是能做相公家的女婿也说不定的。就是他多说了几句,相公才冷了下来。日后用是肯定会大用,相公还让章子厚帮他传了话,但女婿可就做不成了。”

曾布闻言则将脸一板,正色道:“相公家事非我等所宜言。”

“……说得也是。”吕惠卿点了点头,随口应付了一句。转而问道:“那子宣你来此究竟是为何事?”

“还不是韩玉昆出的主意,忙了半个多月了还没忙清。三司也是刚刚把整理后的卷宗呈了过来。吉甫,你猜去年给在京诸司的公吏发的俸禄总计是多少?”

“应该不会多,大部分胥吏都是没俸禄的,”吕惠卿猜度着,“大概只有十几万贯吧?”

“十几万贯?”曾布仰天哈哈笑了两声,将令人震惊的答案爆了出来:“总计三千七百二十四贯又五十六文【注1】!”

虽然早有心理准备,胥吏们的俸禄的确不会多,但吕惠卿听到三千七百这个数字,还是吓了一跳。要知道在中枢的两府诸司中做事的公吏,其数量十倍于官员,但他们拿到手的俸禄竟然不及官员的百分之一!

“怎么这么少?”吕惠卿惊问道。

“在京诸司中吏员近万人,只有其中不到一百老吏领着俸禄,这三千七百余贯,就是给他们的。剩下的绝大多数,名义上没有任何俸禄开销。”

吕惠卿摇着头,“实在太刻薄了,这不是逼人作奸犯科吗?重禄法势在必行!”

虽然厚俸养廉也许只是个美好的愿望,但没有俸禄却绝对养不了廉!人总是要吃饭,要养活妻儿,不给他们发俸禄,他们自然会走歪门邪道去赚钱。荼毒百姓,贪墨官财,胥吏们做的恶事罄竹难书,韩冈前日也是说过,他家差点家破人亡,就是因为奸吏作怪——当然,最后是韩玉昆反过来让那个胥吏家破人亡。

可有韩冈这等心术智计和手段的人才毕竟寥寥无几,绝大多数的百姓都在苦苦忍受胥吏们的欺压,而有奸吏上下其手,高高在上的官人们,也被他们欺瞒哄骗。如果能通过增给俸禄让胥吏们不为奸盗便得以养家糊口,虽然指望他们变成正人君子不可能,情况至少能比现在好上一点。而且这么做,也就有理由对盘剥百姓的险毒胥吏加以重惩。

只是这一条策略的耗费到现在还没有计算出来,不知青苗法和均输法的收入到底能不能支持得了。吕惠卿有种预感,光凭以上两法,再加上还不知道什么时候能见到成效的农田水利法,即使能够支持得住,但其他方面的开支就肯定要压缩了。真的计较起来,至少还得再开辟一两个财源,才能抵得住这个消耗——

吕惠卿沉默的想着:‘也许免役法要提前出台也说不定。’

注1:据《梦溪笔谈》中记载,熙宁三年‘京师诸司岁支吏禄钱三千八百三十四贯二百五十四’。虽然没有熙宁二年的记载,但跟熙宁三年的数据不会相差太远。

