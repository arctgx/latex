\section{第一章 纵谈犹说旧升平(11)}

【前一章的序号错了,应该是纵谈犹说旧升平(十)。】

韩家所在的常乐坊处,近百人气势汹汹的当街涌来,路上的行人车马纷纷避让。

“出了何事?!”有人被推搡到一边,茫茫然的问着。

“你们这是要造反呐!”被人挤垮了摊子的一个老头子怒声喊着。

多少人看着一百多精壮汉子组成的人群,皆是好奇的望着,不知了什么事。

“各位父老,惊扰了。”领头的一名干瘦干瘦的中年汉子站在街口,向四面团团作了个揖,大着嗓门说着,“俺们今日只为判军器监的韩冈那狗官来。照常理,他打他的军器,俺磨俺的米面,两家本不想干。可曾想那韩冈为求功劳,偏要把作坊移到汴河边上抢俺们的位置,将俺们的活路都给断了。可怜俺们家里还有父母浑家孩儿要养活,这一下不是要逼人走绝路吗?不是俺们要闹事,实在是没活路了!!”

但周围却无人受他煽动,恍然之下,纷纷说道,“原来是汴河上的那群磨工啊!想不到他们也有这一天?”

甚至有人认识这位领头的:“周桂这不是找死吗?韩舍人可是好惹的,都能把人送上天了,真真是天上星宿下凡。”

另一人也说着:“他们也是糊涂。韩舍人最得圣眷,宰相都动不了他。真的闹将起来,天子可会饶他们?”

“罚不责众,怕个什么?事情闹得大了,反而是韩舍人倒霉。过去又不是没有例子。杜相公当年沙汰三司吏,闹得有多大?砸进杜府里的砖瓦能砌起两间屋。前两年,王相公还在宣德门挨了一棍子,最后也不过杖责了事。今天的事算个屁啊!”

“在磨坊里做活的都是厢军吧?就算磨坊被撤了,也少不了他们的一份俸禄。”有人狐疑的问着。任谁都知道,裁撤军队的手续,可比要沙汰吏员、工匠要难上不少。就算这里没了活干,其他地方也还会有活等着他们。

“磨坊中的活计从来靠的不是那点死钱,难道你不知道这份差事能落下多少油水?!”心明眼亮的人可不少,“东京城的米麦,甚至茶叶,都是要在汴河上的几十座官营磨坊中走一遭。就算只干没下三五厘的耗费,以东京米麦、茶叶的数量,一年至少也有十几万贯。那些管着磨坊的一个个官员哪一个不是吃得脑满肠肥?最下面的厢兵,一个月差不多也能多分到三五百文。能舍得吗?”

“这般鸟贼,尽日里盘剥百姓。现在韩舍人不让他们盘剥了,就成了仇人了,也不想想那些钱拿着愧不愧?!”

汴河上的官营磨坊在京中有着公愤,送去磨制的米面,总会被克扣掉一部分,他们倒霉只会被叫好。只是说是这么说,却没一个出来主持公道的。都是摆着看好戏的态度,甚至还有一帮市井泼皮聚了过来,准备跟在后面看着有没有混水摸鱼的机会。

周桂见没能煽动得了人,也不再耽搁,一挥手,就领着一群人冲进了韩家所在的小巷。几户邻居只是探出头来,一看巷中摆开的阵势,就砰的一声,将大门给紧紧的关上。

“到了!”领头的周桂在韩家门口停步,一指高高挂在上面的韩府门头,“这里就是韩狗官的家!”

“砸!砸!”一片声的在怒吼着,立刻就有两人提着棍子冲上前来,哐哐的捣起了韩家的大门。

大门一声一声如同敲鼓一般咚咚咚的响着,门框上扑簌簌的向下落着灰。

“姐姐,怎么办?!”

关于将被裁撤的水力磨坊可能会闹事的事,韩冈事前也跟家里说过了,而且在韩冈得到消息的同时,家里也得到了传信。只是临到头来,一想到家里的主心骨现在还在外面,韩云娘就有些心中发慌。

“韩忠!”王旖是大妇,心思还算稳定,叫着家丁里头目的名字,“派了人去兴国坊通知舍人了吗?”

韩忠是韩家真正的心腹,投到了韩冈家里,连姓名都换了,上前道:“回夫人的话,舍人一直都派人盯着的。家里得到消息,舍人那边肯定也得到消息了。”

“你知道舍人是怎么安排的?”周南正问着,就见着一块瓦片嗖的飞了进来,砸在了前院的地上,碎得一片片的。

“都是些泼皮无赖,不成气候。请夫人和三位娘子放心,只凭小人几个,就足够对付他们了。”

韩忠拍着胸脯说着,他身边的几名家丁也都是跃跃欲试。皆是从军中出来的,其中有好些人还担任过韩冈的亲卫,哪里会怕这点小阵仗?别说韩家的家丁,就是听候使唤的婢女,拿起弓来,也不会输给外面的那群在东京城里养得骨头都酥了的厢军。

这时候,聚在韩家外面的人,不知从哪里搬来的一堆砖石,隔着院墙往里面一阵乱丢,噼里啪啦的,砸坏了前院一堆摆设。

一人紧跟在周桂的身后,低声问道:“周二哥,是不是见好就收了?”

“怕什么!两年前的上元节,韩三他岳父在宣德门挨了打,最后又怎么样了?大不了去沧州牢城待两年,等到大赦,就能回京来了。到时候有贵人照应着,要什么肥差没有?!砸!”

周桂指着韩家的院子,狠狠的吼着。机会难得,就算会吃点苦头,但后面可是有泼天的好处在等着他。只是背后忽然两声惨叫,将周桂的吼声完全给盖住。

猛回头,正见七八个家丁装束的汉子,拿着黝黑的铁棍站在了巷口。几个人将两丈多宽的巷道给堵上了。就在他们脚边,有两人做了滚地葫芦,在地上哭着喊着。

这几位都是冷着一张脸,只是站在一起,就隐隐结成了一个阵势,压迫感扑面而来,就算是再迟钝的人,都难感觉得到他们不是简单的角色。

“你们是什么人!?”周桂一声惊问。

领头的韩忠根本没有理会周桂的问话,他领着家丁从后门绕过来,不是与人谈天说地的。上前抬手,毫不留情又是几棒子就招呼在后面等着混水摸鱼的破皮们的孤拐上。一阵钻心的剧痛传来,几人一起抱着小腿,嗷嗷叫着满地乱滚。

做翻了几个挡路的,韩忠等人挺着杆棒一步步上前。前面正想着韩家的宅院里丢着石块的一干人等,终于发现了事情不妙,一个个停了手。但韩忠他们却没有停,手中的棍棒劈头盖脸一阵乱打,不论是什么人,只要挡在面前,就是一棍子下去。

韩家的家丁们前冲后突保持着稳定节奏,互相之间交错掩护,完完全全就是战阵上的功夫。而他们的对手挤成一团,有的要跑,有的留,还有的要反击,没有一个齐心的目标,乱成了一团。

一直向前冲杀了二十步,将三十多人做翻了在地,韩忠一脚将地上滚着爬着的垃圾踹到一边,终于停了步。咚地一声响,酒盏粗细的铁棍就在青石板路面上狠狠一顿,顿时就是几片碎石飞了出来。他指着前面被吓得如同见了老鹰的一群雏鸡,厉声喝着:“爷爷在战阵上杀的西贼也多了,这两年跟着舍人,倒少见了血。吃素吃得让人欺上门来了,真当俺们都做了和尚?哪个先上来让爷爷开了斋!”

“光天化日之下,殴伤人命,到底有没有王法了?!”

“王法?光天化日行劫官人家的府邸,犯王法的是哪一家?”韩忠冷哼着,“爷爷今天心情好,不杀人。只打断你们的狗腿,送你们到开封府去审个究竟!”

“不就七八个人吗?!一人一口唾沫都能淹死他们!”周桂这时无声无息的退到了人群中,大声喊着。

“别躲在后面让别人送死!”韩忠抬起杆棒,指着藏在人群中叫嚣着的周桂,“像你这样的鸟贼,如是在行伍中,早就在背后挨刀了。”

韩忠这一句骂,就像一柄分水刀,将挡在周桂身前的十几人全都分了开来,让他不得不站到前面。

周桂也是个光棍性子,到这这一步,也不再躲闪,走到人前拍着瘦巴巴的胸脯,“爷爷就站在这里,有本事连爷爷也一起打杀了!”回头又冲着一同来的厢兵们,“兄弟们,回去照顾俺家老小,哥哥今天就把这把骨头丢在这里了!”

周桂的这副作派,倒惹起了一阵同仇敌忾的心思,一些后退的人这时又向前走了上来。

只是韩忠没给他更多的机会,更没一句废话,一步冲前,五尺齐眉的铁棍在周桂的膝盖上只那么一捣,卡擦一声脆响,就见着他的关节翻了过来,小腿变得朝前面弯了。

周桂尖叫连声,难以置信的看着向前弯成了九十度的小腿,嘶声竭力的叫着。而他身后的一群人则拼命地往后退,京城安逸了上百年,虽然他们也在兵籍簿上挂着名号,但哪里见识过上来就将人往残废里打的狠角色。

“废物就是废物。”韩忠不屑冲着周桂脸上吐了一口唾沫,“把这几人都给我绑起来,械送开封府,请韩府尹来审一审,究竟是谁在背后撺掇,敢在京城里闹事!”

