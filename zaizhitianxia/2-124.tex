\section{第25章 阡陌纵横期膏粱(三)}

韩冈前生在社会上闯荡多年,见惯了人情世故。人心会变质,虽然现在瞎药对他心悦臣服,俞龙珂见了他也是毕恭毕敬,下面的蕃人甚至视他为神明,但在郭逵等人的权势面前,他们的那一点敬畏之心,转眼就会烟消云散。

而有利益维持的关系却是坚固的。只要有着源源不断的金钱的滋润,韩冈相信蕃人们对自己的敬意,会根深蒂固的保持下去。只是有一条需要注意,韩冈必须得让蕃人们明白,除了他以外,其他人都不能带给他们同样多的利益。这就是为什么韩冈放弃其他同样能给蕃部带来大量收益的手段,而选择了药材这一项。

此事宜快不宜慢,虽然成事至少要一两年的时间,可先得在俞龙珂、瞎药以及张香儿,这三个青渭地区的蕃部大头领的面前画个大饼再说,不然等他们去了秦州,别人还好,俞龙珂肯定会投向郭逵。但眼下空口说白话也不行,先得回去把相关的资料整理出来。

看到郭逵插手缘边安抚司的内事,王韶也没了游玩的兴致。当即叫起了高遵裕,把此事一说,从屋中出来的太后亲叔,脸上便是挂着深冬腊月的严霜。预定中的行程不了了之,众官当即回返古渭。倒是张香儿不知情由,还以为自己哪里慢待了,吓得连连赔不是,韩冈一番好言好语的才把他安抚住。

紧跟着怒发冲冠的两位顶头上司,碎乱而又沉重的马蹄声,就像现在韩冈的心情。真要说起来,还是缘边安抚司先破坏了和郭逵之间的默契,瞒天过海的出兵星罗结部。但郭逵出手撬人墙角,是官场中的大忌,也是任何一个官员都难以容忍的做法。

‘两边都有问题。’韩冈在心中给两边各打五十大板。

王韶和郭逵都想吞下最大的一份蛋糕。郭逵因为他是后来者,所以只求军功。但王韶这边,河湟之事是他首倡,眼下的大好局面,又是他胼手胝足辛苦耕耘的而来。近三年的时间里,王韶所耗心力不足为外人道。单是韩冈认识他的这一年来,王韶已是很明显的苍老了下去。一番心血,他怎会甘心让人拿走最大的那一块蛋糕。

眼下两边的矛盾正在激化中,虽然因为顾忌到后果,都还没有撕破脸的打算,也在极力克制自己的冲动,但最后的结果却是令人难以乐观。韩冈不想插足进去,他无意再为王韶冲锋陷阵,尤其要面对一直很赏识他的郭逵。他为王韶已经做得够多了,眼下还是为自己考虑多一点。

回到古渭,韩冈的第一件事就是找来朱中。朱中既然能向张香儿要药材,对这个行当的了解肯定不少,而且又掌握着疗养院,需要什么药材他也同样明白。另外他又派人去秦州把仇一闻请来,老家伙在秦凤人头熟,地理更熟,哪座山里有什么要,他最是门清。

等韩冈将一切厘清,把公事一一分派出去,回到家中时,已经有着更夫敲着梆子,在城寨中的街道上走着。入冬后,天黑得越来越早,群星已在天穹中闪烁。

十几名亲卫将韩冈护卫在中间,渐渐接近自家的宅子,一个小小的身影藏在门洞中,见到韩冈回来,忙迎上前。

“三哥哥,你回来了。”

“我回来了。”

夜幕下,一个十三四岁的少女倚门而望。纤细的身影柔柔弱弱,让人怜惜。韩冈已经几次让韩云娘不要再到门外迎接。小丫头还不满十四,可就是犟得如同几百万年沉积下来的石头,怎么也不肯答应下来。

进门前,韩冈跺了跺脚,将官靴上沾的泥土都顿在了门外。八九月的时候,因为渭源的事情,韩冈忙得脚不沾地,三过家门而不入,几乎跟大禹一样。这件事让家里知道后,韩冈没少被韩阿李埋怨过,而韩云娘和严素心则更是满眼幽怨。也直到了现在才轻松一些。就是老往地头跑,靴子总是干净不了。

韩冈进屋的时候,韩阿李正在屋中做着针线活,而严素心不在——多半是在厨房中——反倒是冯从义坐在屋中陪韩阿李闲聊,韩冈看他的样子,应该是在等自己回来。

见韩冈进屋,冯从义连忙站起身。而韩阿李则放下手上的针线活,一脸不高兴的说着:“你爹早早的就回家了,三哥你怎么到现在才回来?准备的饭菜都浪费了,还让义哥儿等了这么久,也不知让人回来知会一声。”

“有些急事要忙,一时忘了。”韩冈向冯从义说了声抱歉,冯从义连连摇手说着不敢。韩冈看看内间,问道:“爹在哪里,先睡了?”

“你爹不能跟你比,累了,先去睡了。”韩阿李说着,重新拿起针线。从式样上看,她缝的应是件袍子,也不知是给谁。

韩冈叹了口气:“爹的身子骨也不比年轻时了,娘能不能劝劝爹,让他老人家不要天天下田去?”

韩阿李低着头,手上飞针走线,对韩冈叹道:“你爹就是一条劳碌命,享不了福,闲下来反而会生病……就跟三哥你一样,都想着越忙越好……你也是忙昏头了,也不见你问问义哥儿来家里有什么事?”

韩冈闻声便将视线转过去,冯从义接着韩阿李的话头:“这是上个月的账簿,要让三哥过目一下。”

“算了,这些东西我看着头疼,有娘盯着就行了。”韩冈无意去根究细节,一点点的去查账册。但他也不是直接放手,韩阿李会算账,韩冈家里的生意都是在靠她来做最后的复查。而且现在商行从上到下都建立在韩冈的地位上,冯从义都闹不出什么花样来。

韩冈让冯从义开办的商行叫做顺丰行,与王韶家和高遵裕两家的商行,鼎足而三,仅仅半年就掌控了古渭榷场的超过七成的交易。而且尽管这三家商行在一开始就困扰于比普通人借贷要高出一成的利息,但这几个月的时间,近乎垄断榷场中的交易,却已经足以让他们把钱都还上了,冯从义就是来通报此事。

“那些借了官中的钱确定都还上了?!”韩冈低头算了一下,按照顺丰行的收入,的确可以把半年前的贷款抵消掉。

冯从义立刻点头:“连本带利都还清了……就是人手不足,让许多生意只能眼睁睁地放过去,否则就能更早的把钱都还清。”

“这样啊……”韩冈沉吟着,“护卫可以找蕃人,瞎药那边能派出不少得力人手。至于交易的掌柜,要跟蕃人懂得互敬互谅,不要因为身份而互相诋毁。”韩冈知道,这里有许多人跟城中的蕃人势同水火,但他不想在眼下积极的应对,“至于新任掌柜的关系,可以慢慢的来。眼光放长远一点,一点点把人培养起来,这样的人才,才会有着足够的忠心。”

“三哥的话,小弟记清楚了。”冯从义作出谦虚好学的的模样,其实骨子里还是透着自信。韩冈把药材之事跟冯从义说明了,冯从义只想了想,便说要去调查一番才行。

韩冈不以为异,没有调查就没有发言权,这一句虽然如今的人们并没有听说过,但同样的体会却是许多人都拥有的。冯从义也不例外,这让韩冈觉得很欣慰。

说了一番闲话,冯从义看韩冈也乏了,便起身告退离开。韩冈将其送出院门,只用了半年不到的时间,就把贷款还清,等到明年,开春后商旅重行,剩下的就是净赚,这也算是冯从义的本事了。

在家中住了一夜,三月不知肉味的韩冈把严素心折腾了许久。一点点变得丰润起来的身体,还有光洁细腻的肌肤,让他爱不释手。

第二天,韩冈就从王韶那里听说他要提前去京中诣阙。王厚私下里则跟韩冈透露道,他老子这是进京去唱莲花落的。

王厚调侃自己的老子,但实质上却是一点没错。王韶进京诣阙本来要到年底才去的,现在提前了两个月。一方面是为了带领顺服无比的瞎药三人一起去逛东京;另一方面的原因,王厚抱怨了许多,就是安抚司没有钱了——这年头连地主家都没有余粮——王韶也只能到京城去唱莲花落要钱。

“不过今次这些人当是要赐姓了。”辛苦了许多时日,瞎药终于彻底顺服,连带着俞龙珂和张香儿都要一起进京。他们的成功,韩冈也算上是其中的一半功劳。

“管他赐什么?你听没听说过,听说郭仲通也准备回京?”王厚突然冒出来一句。

“不可能!”韩冈摇着头,“那条传言是假的。”

韩冈在秦州城中的耳目消息比王韶还要强上一筹,州县两边他都有人。尽管韩冈此时官位仍低,但他会为底下人做主的性格,让人投到他门下有着足够的安全感:“郭逵才来么没几天,凡事未见功勋,不可能就这么甩着手回京城去。等着看好了,他肯定还有后手的。”

