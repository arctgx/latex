\section{第19章 萧萧马鸣乱真伪(八)}

饯行宴结束了,来送韩冈出发的人们依照亲疏,在离着东京城不同远近的地方一个个的告辞返回,到最后,离城快有二十里,王旁才最后一个与韩冈辞别。

上午别过含泪的父母和妻儿,韩冈出门时王旁就赶过来相送,等他走到城门口的时候,身边已经有了几十人来相送。相熟的不相熟的都来送他离京,这不外乎是留给未来的人情。在韩冈看来,还不如几个同门师兄弟加上亲戚的送行。

与王旁拱手致礼,韩冈翻身上马,跟随他南下的四名幕僚和一队亲随也全都上了马。

李复、马竺、陈震、周毖,此四人今天早上得到韩冈的通报,明白了他们即将面对的问题,现在正伤着脑筋。而韩冈交给他们的第一个任务,就是想一想下一步该怎么做。

是要只凭眼下的人手坚持进攻?还是等到大军到齐之后再行动?

几名幕僚很意外韩冈会问出这个问题,因为此前韩冈的提醒和教导,都是不断在告诉李复四人,他们的任务是辅佐韩冈处理军中一切琐碎的事务,并不是献计献策。

只是眼下既然被韩冈考校到了,他们当然也不会放过这么好的机会,一个个都想着要在韩冈好好表现一番,如果出色的话,那就能成为韩玉昆倚之为臂助的谋主、策士,而不是他之前提点过的安排军中琐碎事务的属吏。

“履中,你怎么看?”陈震低声的问着李复。

李复想了一想,道:“小弟觉得以龙学的心意,多半还是要打上一场再说。”

“不仅仅是打上一场,”周毖在旁插话,“说不定龙学还存了直捣升龙府的想法。”

李复肃然道:“若是如此,我等身为龙学幕宾必得加以劝谏,用兵当以正奇相合,不可只用奇兵。”

周毖立时反驳:“有三十六峒和广源蛮部相助,打到升龙府下也并不算难事。”

“前一次任用黄金满的广源蛮军是迫不得已的行险,此前龙学也是这么说的。”李复道:“现如今不等大军齐至便贸然深入敌国之中,这个风险有必要冒?”

陈震轻笑道:“以交趾的军势,凭借五千西军精锐加上千五荆南精兵,将正正之旗,临堂堂之阵,也未必不能击破之。”

“用兵岂能靠着‘未必’?”李复厉声质问。

陈震脸色一下涨红,辨道:“龙学若是未战即怯之辈,如何能做到今日的位子上?”

“陷主于危,岂是幕佐当为?!”

“都少说两句。”一直没开口的马竺拦过来,在四人中他的年纪最长,“你们想想,龙学究竟是为何要用我等?出谋划策是一桩,佐理庶务也是一桩,拾遗补缺、劝谏危行还是一桩。各有各的道理,没有对错可争的。与其在这边猜测,还不如先问明白龙学心中的想法再说。龙学要打,我们就做好大军行军出阵的准备。龙学说不打,我们就下去查看军中士气。此事还是龙学与章子厚做主。”

马竺的话是颠扑不破的老成之言,李复、周毖各自收了声,只是互相之间都不搭理。

韩冈不去在意身后幕僚们的争论,他就在马上拱手,向着王旁:“仲元,小弟就此告辞了。不能面辞岳父岳母,也请仲元代为致意。还有小弟家中,也望仲元闲暇时能多看顾一二。”

“玉昆即使不说,愚兄岂能忘记,还请一切放心。”王旁顿了一下,着重强调一般的说着,“有愚兄,更有父亲在,玉昆你一切都可以放心。”

‘若能如此,那就太好了。’韩冈想着。

王安石还在宫中,今天要讨论的议题关系到国家安危,不得不慎重。只是结果可能不会变,都是河北军留于原地,严防契丹南侵。

安南招讨司面临的问题很严重,虽然王旁还受王安石所托,来转告韩冈,说他会尽快将河北、河东的事情给厘清,尽可能快的将剩下的一万多兵马派遣去广西。

但韩冈很清楚,王安石的尽可能,基本上就代表着第二、第三批南下的西军,赶不上这个冬天出战的脚步。

只是心里话不能说,韩冈抬眼道:“这就要多劳岳父和仲元你费心了。”再一拱手,“小弟就此告辞。”

一夹马腹,驱动胯下的坐骑,韩冈不再回头。幕僚也一时收起争议,和随从们紧随在后,紧紧地跟上韩冈的速度。

韩冈望着眼前通往南方的官道,想着的却是身后,‘不知道丰州之战的结果如何?’

……………………

“丰州应该已经打起来了吧……”王舜臣眼望着东北方苍翠的群山。

虽然就在横山北麓,出现了为数数万的党项兵,他们的斥候甚至越过了横山,昨天还与出城巡视的骑兵小队厮杀了一场,不过王舜臣的注意力还是放在了几百里外的河东路上。

“肯定打起来了。”

听到背后传来童贯的声音,王舜臣呆了一呆,才发现自己心中思考的问题,已经不知不觉的说出了口。

回过头,王舜臣看着身量远比自己要高,而有同样壮硕的宦官,“走马探视过疗养院了?”

“去看过了。”童贯不介意去做这样收买人心的举动,应该说是很乐意,“十几个伤病都还精神。病也好、伤也好,想必很快就能疗养康复了。”

“那就好。”王舜臣点头重复着,“那就好。”

童贯见王舜臣关心此事,心中不免疑惑起来:“为何都巡不去探视?”

“拿什么去探视?金银财帛,还是鸡鸭鱼肉?”王舜臣狠狠的说着,“等拿到了足够多的西贼的心肝去探病,那病才容易治得好。”

童贯脸上的表情先是一滞,随即就哈哈大笑起来,“都巡说的是,都巡说得极是!”

王舜臣回头又望着城外,要是没有那些时常来骚扰的党项骑兵,他就能直接加派一队人马,去麟府那边联络,如果能得到两边的通力配合,左军神勇军司这根在大宋立国后不久就一直堵在喉咙里面的毒刺,就可以顺利的给拔出了。

从鄜延路这里进兵,可以直接支援河东路。当年第一次攻打横山,一开始的计划就是以河东路配合鄜延路,在罗兀城东修筑葭芦川等一系列寨堡,将罗兀城这一突出部拉平在新的防线中,只是最后河东军中伏崩溃,不但让防线上出现了一个致命的缺口,也使这一战略安排不了了之。

不过这一战略并没有错,现在换作是王舜臣镇守罗兀,也有着与当时的韩绛、种谔同样的想法,‘太尉那里应该有动静了,总不能一直是静观其变。’

说道曹操曹操就到,王舜臣正这么想着,守着南门的士兵就匆匆来报,说是有贵客临门。罗兀城这边一直以来都是恶客来的居多,所谓的贵客则是更让人更心痒难忍的恶客,都是要拿起弓刀来迎接。

只是看到来人,王舜臣就下了一跳,“十七哥?你怎么来了?”他再望望种朴身后,就只发现了五六个随从,“怎么就带了这点人?”

“嫌少?”种朴瞥着眼笑道,“大队的援兵都在后面。”

“俺哪里是说这事!”王舜臣声音有些急了,“西贼的哨探都跑到罗兀城的南面去了,十七哥好歹也带个百八十骑兵再出来。十七哥你自己看看你身后,才几个人?”

“西贼现在自顾不暇,还不至于两面开战。聪明点的就该去守缘边寨堡,这样即使是再贴近前线,也照样能平安无事。”

鄜延路与麟府二州虽然都在黄河西岸,但两地远隔重山,还有西夏驻扎在弥陀洞的左厢神勇军司两万来人挡在中间,鄜延路这边并不清楚丰州的情况。但只要知道河东开始进攻丰州,就足够让种谔做出决断。

“敢问十七哥,太尉究竟是打算如何行事?”王舜臣问道。

“很简单,就是进兵葭芦川,”种朴立刻说道。

“只是进兵葭芦川?”王舜臣意味深长的笑问道。一家侍候种家多少年了,王舜臣当然知道种谔的想法绝不会是表面上看上去的那么简单。

“当然不是!”种朴不意外韩冈教出来的王舜臣能看破他们所谋划的这一切。“要尽可能的做好伪装。这一战不再夺地,而在夺人。西贼的人财物都已经快见底了,我们要做到就是尽量帮着弄得更严重一点。”

“如此方好。”王舜臣释然点头,“伏击当是能做得。”

做出出兵援助麟府的姿态,于险要之地设下埋伏,到时候就可以等着西贼自动上钩。看种谔的想法,是打算将西夏的左厢神勇军司上下彻底的深埋进地底。

如果能将左厢神勇军司一举扫平,至此之后鄜延和河东便能轻轻松松的联络起来,不再需要向南绕道。不仅如此,得到了左厢神勇军司的地盘,罗兀城孤悬在外的情形就能得到化解,与此同时,麟府军上下也都能得到更好的支援。

这是一个再好不过的时机……只不过还要看一看丰州的情况,才能下定论。

