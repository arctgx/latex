\section{第28章 遥别八桂攀柳枝(上)}

【今天白天忙得一团乱,现在才赶出第一更,待会儿还有第二更。至于说好的补更,只能等到明天了。还请见谅】

‘终于要离开了。’

韩冈收起了王安石的来信。

从落款上看,尽管发出来的时间比调职的诏令要早上三天,但韩冈收到信的时候,则比收到诏令要迟了五天。

举目环顾他并不算熟悉的转运司公廨,他终于要向这片他曾经战斗生活过的的土地告别了。虽然在安排好了交州的发展规划之后,就已经有了离开广西的想法。不过因为多方耽搁,又拖了几个月。

韩冈轻轻一笑,接下来还是不得闲,京西都转运使可不是个简单的职位。

“三哥儿,久等了。”

换了身干净衣服,李信从内间出来。他正巧有事往桂州来,却是恰好碰上韩冈要离开,能送上一程。

李信在交椅上坐下,自家人,也不需讲究什么礼数。端起茶汤来喝了两口,道,“方才看里面的行装都收拾好了,三哥儿你这两天就走?”

“当然,为了处理漕司中手尾,已经耽搁了五天,也该上路了。”韩冈也跟着拿起茶来喝了,“走得快些,还能赶上年节。”

“不用与人交接?”

“副使暂代,就是跟任时中交接,才耽搁了五天。”韩冈啧了一下嘴,冷笑了一声,“这一年多,转运司中的事都是他代管,桂州库中短了四百斤茶和六百多匹绢,他家的门客还敢问我是怎么回事?”

“最后怎么说了?”李信笑着问道,他可不会为韩冈担心。

“还能怎么样,”韩冈带着让人玩味的笑意,“反正账目平了。”

“可怜啊。”李信摇摇头,对任时中有些同情,请了个白痴门客。这个亏空浑赖不到韩冈头上,不是由任时中自己掏腰包补上去,就是设法将账目给做平掉,反正要费不少手脚。

韩冈轻抿着茶水,这交接时的一点小乱子算不得什么,既然任时中已经签了字,剩下的就不干他的事了。放下茶盏,韩冈叹了口气,“章子厚走了,燕达、李宪也回去了,小弟现在这么一走,广西这边可就表哥你一个了。”

李信,“当初在荆南还不是这么过来的。”声音顿了一下,“不过我也不瞒三哥儿你。我在广西实在是习惯不了,虽然没生什么大病,但身上总是觉得不爽利……”

李信对广西气候的抱怨不是第一次了,生长在陕西,却在荆南和广西立功为官,镇守一方,要是能习惯才有鬼了。

“过些日子……”韩冈沉吟了一下,尽管王安石被调走了,但他在枢密院中的人脉还在,只要活动一下,将李信调离广西不是问题,“最多半年,表哥你的调令就该到了。我会跟王副枢和章子厚写信,多半能将表哥你调回陕西,不过说不准会是河东或是河北。收复交州一战,西军的战力天下人都看到了。河东就勉强些,河北那边朝廷要练兵,调过去的可能性也不小。”

“我也明白。”李信点点头,“章副枢也写了信来,说过些日子就调我回北方。总不能让刘仲武一人回去得意。”

原来章惇也写过信。韩冈笑了笑,不以为意,“只是到时候表哥你恐怕也要有一番辛苦了。不知道官军在南方待得有多难,不服气的人恐怕会不少。”

“三哥儿你放心,”李信的笑容中有着百战名将的自信,“为兄不会与人争胜,只是也不会任人小觑……”笑了几声,又看着韩冈,“其实有三哥儿你在,军中大半人多少都会卖个面子,当不会有什么大麻烦。”

“我哪里有这般威风?关西还好说,到处都有熟人,但河北、河东两地,又怎么可能会卖我面子?”

“三哥儿你也别太小瞧自己,你的名声可是天下都传遍了,做哥哥的能有今天,也是靠着三哥儿你。”李信摇头笑着,双眼中神采渐渐的变得迷茫起来,“十年前想都没有想过会有今天,只盼着最后在军中混个几十年,能熬到个指挥使就了不得了。”

“表哥你这话说的,若是没真本事,怎么也做不到现在的位置的。在荆南、邕州和交州,上阵的难道是别人吗?表哥你现在的广西钤辖,是用上阵拼杀挣到的功劳换来的,可跟我无关。”

“不说这些了。”李信笑了笑,伸手过来,拍拍韩冈的肩膀,“等三哥儿你在京西做过一任后,回京做翰林,过个几年再进两府,再过一阵子,可就是宰相了。到时候,也能沾沾三哥儿你的光。”

翰林,两府,宰相。

韩冈摇了摇头,苦笑道,“哪有这般容易。”

就在王安石的信中,已经明说了让他安心在地方上做个几年。做上一任两任转运使,再做个一任两任边州大郡的知州兼经略使,等资望到了,再入京不迟。到时候身入两府担任执政,过个几年重新出外,在重要的州府任职几次,四十多五十岁的时候,就能坐上宰相了——后面的半截,王安石没有说,是韩冈一路推测下来的。

韩冈最不喜欢看到的就是资望二字了。这些年来,要是没有这两个字,他立下的功勋宰相不好说,执政肯定没有问题。可惜就是卡在这两个字上,最后还是只能做着一个转运使和龙图阁学士。

他声望其实没的说,就像方才李信所说,有事无事,军中上下多半都要给他一个面子。就算到了民间,只要报个名字,人流密集的城镇必然有许多人听说过身为药王弟子、星宿下凡,以至于还能让人飞上天的韩龙图——没见天南地北十八路的大小酒店门前,飞着的一个个都是拖着招牌的热气球,除了些城外的野店,早就没人挂太白遗风的杏黄角旗了。

也就是资历不足。他的岳父王安石那是有耐心,厚积而薄发,在地方上仰望三十年,一朝入朝就是翰林,转头就升了参政,没两年就又做了宰相——这还是他几次将到手的相位让与他人的缘故。但韩冈的耐心也比不上王安石,他只是想得到能与付出和成就相当的回报。可惜年龄和资历成了横亘在他眼前,挡住了他更进一步的鸿沟。

“对了。”李信看到韩冈有些郁闷的表情,仿佛忽然间想到了什么,把话题岔开,“方才我在内间看到了三哥儿你要带着走的行装,怎么就几个包裹,是不是少了点?”

“随身带着行李多了,就太榔槺了,赶路也不方便。其他行礼其实也是有的,不过已经让顺丰行的商队一并送去京城了。”韩冈放开了沉郁的心情,笑了起来,凑到近前压低声音对李信道,“虽然里面要送人的礼物都是买的,但让人看几十个箱笼总是不太好。”

李信愣了一下,转而就指着韩冈大笑起来,他的这位表弟还真是会做官。

韩冈倒也不在意被李信笑,只是手段而已,又不是什么伪装。他不是在装清廉,而是他本来就是个清官。

韩冈为官,向不收重礼,在广西也是如此——他又不缺钱,没必要拿自己的名声来换。但当地的土特产还是置办了不少,有些特产,北方根本见不到。

比如桂州的傩面,一套一百多幅,老少男女妍媸胖瘦各不相同。这样的特产,拿到京城,留在家中赏玩很不错,送礼也有面子,不过韩冈是准备送给儿子女儿。

还有铜鼓、羊毫、羽扇,都是桂州的特产。梧州产生铁最好,滕州则有黄岗熟铁,融州人就将梧州生铁和黄岗熟铁,融合起来打造成有名的松纹宝剑。

端州隔得远了,在广东,那里的砚台,韩冈倒是没去要,但有人送了他一方端砚——端溪砚岩并不大,出产的石料,上品为岩石,中品为坑石,下品是黄步石。而岩石,又依出石的位置,分为上岩、中岩和下岩,其中以下岩为佳。韩冈得到的砚台就是下岩出的上品,不过他转手就送给了邕州州学,作为考试第一名的奖品。

砚台韩冈没要,不过墨有不少。容州松树多,产上等好墨,而且十分便宜,好的一块不过百文,普通的论斤卖,一斤才两百钱。在京城,墨价可是要翻好几番,更不用说潘谷等名家造的精品,那都是跟黄金等价,直接送进宫中的,被多少文人争相写诗赞美。但韩冈不是文士,直接让人论斤去买。

至于海边的珍珠、珊瑚等贵重物品,韩冈并不稀罕,但一整套用海螺制成的酒杯,他却是视如珍宝。让人小心的放进箱子里,用稻草和木棉絮填塞好了,才送上船去。

这么些特产,装了几十个箱笼,韩冈嫌随身携带太难看,就让顺丰行连着置办的货物一起送去京城——这就是家里有个商行的好处,从海路出发,这些行礼在路上的运费其实并不贵。

韩冈就这么与李信聊了一夜,表兄弟两人也是很久没有坐在一起谈心了。到了第二天,黄历上是宜出行,宜嫁娶,不宜动土,也终于到了韩冈动身启程的时候了,目的地并不是京西,而是久违的东京城。

