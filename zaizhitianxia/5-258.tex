\section{第28章 官近青云与天通(25)}

赵顼阖起了眼皮,久久的没有一点动静,像是睡着了一般。

向皇后不安起来。从吕公著的失态上看应该是件好事,虽然吕枢密在失态后立刻请了罪,弯腰捡起奏折,但三五下才将奏折捡起来,可见其动摇到了什么样的地步。

只是丈夫现在的反应又让人费解,不知道他的葫芦里到底是卖的什么药。

大概是歇了好一阵后,终于恢复了精力,赵顼重新重新睁开了眼睛。

下平二萧——招。

“官家想要招谁?”向皇后随即追问道,她关切的看着皇帝。她至少能明白,现在丈夫究竟是招谁入宫,就决定了到底是准备怎么安排未来的朝堂大局!

上平十四寒。

‘韩?……韩冈!’向皇后心头一喜,只是赵顼想说的并不是‘韩’,而是‘翰’。

“翰林?”向皇后问着。

赵顼眨了眨眼,两下。

然后又是一个‘去声二十号——诰’和‘下平一先——全’。

向皇后总算是明白了赵顼心意:“是将知制诰的翰林学士全都召来?”

两下。赵顼给了肯定的答复。

一下找来所有掌内制的翰林学士,这是标准的拜相序曲,甚至更高了一格。

向皇后回头来找人。瞥眼看到了吕公著,这位太子太保正垂着手,端端正正的站着,脸色如常,沉毅沈谧,方才的失态仿佛完全不存在。当然,方才托在手上的辞章,也被宽袖掩盖,仿佛不存在了。

多看了吕公著一眼,向皇后便丢下了他,点起宋用臣,派他去玉堂找翰林学士。

……………………

福宁殿中的动作,如同一块巨石投入池塘中,整个皇城都被惊动了。

本来崇政殿上对司马光和御史们的处置已经传出来了,王珪尽管被保住,但他已经没有足够了能力和声望来执掌东府,接下来必然会宣麻拜相。

隐隐躁动早已潜伏在皇城中,多少人预测,半月之内便能见分晓。只是没人料到会这么快,一个时辰都不到,而且还是吕公著自请留对的时候。

不用一刻钟,翰林学士入宫的消息便传到了韩冈耳中。

他也是翰林学士,可惜是不在院的学士。虚衔空名,不加知制诰,不用草诏,不掌内制,称为内翰其实都勉强,当然不可能在这个时候进福宁殿,只能在外面等消息。

“这是大拜除?!”黄裳立刻惊问,“是要任命宰相了!?”

“还能是别的原因吗?”

“内禅?”黄裳刚说出口,自己就摇头否定了。要当真是内禅的话,宰执们应该先一步入宫。

“王禹玉是要罢相了,谁会接手相位?蔡确吗,还是吕枢密?”黄裳问着韩冈。

“官家从玉堂招了几人去?”韩冈却转过去问来报信的小吏。

“三人。在院的内翰全都入宫了。”

韩冈回头对黄裳笑道:“看来的确是大拜除!”他将重音落在了‘大’字上。

得了韩冈的提醒,黄裳模模糊糊的有了点想法,但还是有几分不解,正想细问,却见韩冈站起了身。

从身后门外传来了苏颂的声音,“玉昆,还在衙中啊。”

黄裳连忙起身回头,只见苏颂正跨步进厅,这也是听到学士院锁院消息的。

“玉昆,你觉得如何?”挥退了厅中没眼色的几个小吏,苏颂甫坐下来便问道。

韩冈想了一想,抬眼道:“……大势将定。”

……………………

拜除宰相照规矩是天子御内东门小殿,然后学士院锁院。当这两件事同时出现,皇城内外所有人的耳目都会扩张到最大。

只是现在以赵顼的病情,不可能去内东门。让皇后代行也是一个选择——毕竟已经是垂帘听政了——但赵顼担心皇后不能将自己的心意表达明白,她实在是太缺乏经验。

所有仍在翰林学士院中的翰林学士,便因为这个缘故被招到了福宁殿中。

翰林学士满额是六人,但加知制诰的就没有那么多了。眼下玉堂员额未满,能书诏的更是只有三人,张璪、蒲宗孟和孙洙,三人全都被招进了福宁殿中。

张璪眼下已是翰林学士承旨,作为玉堂第一人,比当值的蒲宗孟还要靠前。

大拜除时,草诏往往五六封,甚至过十封,一人很难完成这么大的工作量。一般都会召集两名翰林学士同上殿,即所谓的双宣学士——冬至的那个晚上,张璪因形势所迫,一人独力写了七份诏书,则是一个不折不扣的特例。

不过三位翰林学士接收的天子第一份谕旨,并不是‘拜’,而是‘罢’。

去声二十二祃——罢。

下平七阳——王。

上平八齐——珪。

罢王珪。

拜相的序幕,却是以罢相拉开,张璪一边让蒲宗孟书诏,一边揣度着赵顼究竟对王珪有多恼火。冬至夜他同样在此殿中,亲眼见证王珪几乎是将天子皇后和太子一家推进了深渊。

之前留王珪是形势使然,可惜在司马光和御史们的折腾下,天子的计算成了无用功。现在不用再保他了,当初的愤怒也就如同池底的淤泥,一并翻了上来。

秦失一鹿,天下共逐之。

张璪的心跳得有些急了。

说起资格,他这位翰林学士承旨,也同样只要一步,便能晋身两府。

……………………

“大势将定?”苏颂问着韩冈,“不知玉昆此话怎讲?”

韩冈冲苏颂笑了笑:“小弟不信子容兄看不出来?”

苏颂不置可否,又反问回去:“玉昆觉得会是什么样的大势?”

韩冈简简单单的回道:“天子觉得能安心的大势。”

苏颂突然凝神专注的看了韩冈好一阵,方才再开口,“玉昆,你之前究竟做了什么?”

“不过是上了三份札子。”韩冈说得轻描淡写,却也不再隐瞒,“三天前是弛千里镜之禁,前天是请求刊行《自然》,昨日则是给先师请谥——这是第二次了,多半能成。”

黄裳听得一头雾水,他和韩冈、苏颂的层次差得太远,根本都不知道两人云山雾绕的再说些什么。但苏颂听得很明白,他神色转为严肃,问韩冈:“玉昆……你当真做好准备了?”

“以前也不是没有过,不过是给天子强压下去了,还要什么准备?何况现在重新起头,既能释天子之疑,也能顺便跟吕宫保掰一掰手腕。”韩冈轻笑着,新党也好,旧党也好,都是对手。对新党在于道统,对旧党那就是为了维护大局,“说实在的,这几天一天一章疏,也不完全是针对吕枢密。”

“是司马君实吗?”苏颂问道。

“当然。”韩冈点头,“旧党赤帜啊,再怎么提防都不为过。”

苏颂为之一笑:“可惜让吕与叔消受了。”

韩冈不知道吕公著会在福宁殿中说些什么,但他的心思并不难猜,他能用上的理由,应该也只有一个。所以韩冈现在和吕公著争夺的便是同一个位置:

——新党的反对者。

新党这个团体,在外靠对新法的认同和附和来聚集官僚,在内则是以新学所代表的未来凝聚人心。

吕公著争在外,韩冈则争在内。

韩冈纵然在新党之中有为数众多的朋友和认同者,但从根子上,他所代表的气学一脉,与新党——确切的说,是坚持新学的新党——是截然分立的不同派系。他有属于自己的班底,有足够的声望,也有实力不弱的后备队伍,只是因为地域的缘故,根基差了不少——关西的进士实在太少了,而气学在文风荟萃的中原和江南,则势力太过薄弱。

只是相对于吕公著代表的旧党,韩冈与新党的交锋,不会损害新法,甚至绝大多数新党成员不会视韩冈为敌,真正与他相争的,只有王安石、吕惠卿等寥寥数人:对天子来说,这就足够了——至于国子监中的学官,他们还提不上筷子,狗肉不上席面。

韩冈屈指轻弹着茶杯,看着绿色的涟漪在盏口中一下下的回荡。

当《自然》杂志正式刊行,气学和新学的道统之争将重新打响,甚至只要公布这个消息,所有人都知道自己要动手了,最迟也不会拖到明年开春。

既然如此,与其拖到年后公布,还不如在这个最紧张的时候放上台面,至少还能额外赚个一石二鸟,甚至一箭三雕、四雕的好处来。

这是他对司马光的防备——韩冈上阵,怎么可能将希望放在皇后一时错口上?那根本是谁都想不到的意外——为了预防司马光上京后引领旧党反扑,他也必须未雨绸缪,早早的做好准备。

不论是司马光老老实实的上殿觐见,然后回洛阳继续修书;还是说他这位太子太师还想搅风搅雨,重新开战,韩冈都会做好反击的计划。即便用不到他的头上,也可以用来对付他人。

其实在皇后垂帘之后,旧党已经很难翻身了。这一点,朝中人人皆知。很长一段时间以来,旧党在朝堂中的作用,就是平衡朝局。但弹劾身居两府多年、且为独相的王珪,却是动摇朝堂平衡的一个良机。一旦这个平衡给打破,旧党的机会就来了。

而司马光果然一如所料,不甘心重返洛阳,探手抓住了眼前唯一的机会。

这样一来,韩冈未雨绸缪的三份奏章的作用便出体现来了。

韩冈三份奏章一上,那就是明摆着跟王安石划清了界限,要重新燃起新学和气学交战的狼烟。当维持住自己孤臣的形象,那么接下来旧党一旦在司马光的引领下展开反扑,那么韩冈就可以毫无顾忌的配合新党进行反击——他可以为道统跟新党闹得翻脸,但若是有人想破坏这些年来辛苦建设的成果,韩冈则绝不会答应。

这一主要是针对旧党赤帜才预先埋下的伏笔,很可惜的没用在本尊身上,中途出了让人啼笑皆非的意外。可如果对吕公著自请留对的目的没有弄错的话,那么将会阴差阳错的着落在了这位枢密使的头上。

其实韩冈也只有六七成的把握,毕竟一名瘫痪病人的心思是很难用常理去揣摩的。对章敦会不会当成自己的羽翼给剪除了,韩冈也一样没办法给出一个明确的回答。

只是他已经尽可能做了他所能做的,不可能更多了,所以韩冈现在剩下的就是等待结果。

茶杯被弹得叮叮作响,杯中茶水也晃得越来越厉害。

浮现在韩冈脸上的笑意充满自信,其实这几日来的争斗,也不过是杯盏中起风浪。真正的大势,就藏在几分奏章中。令人遗憾的是,除了他本人,将不会有人能看透这一点!
