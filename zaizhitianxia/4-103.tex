\section{第18章 青云为履难知足(九)}

盛夏的京城,依然暑热难耐。大街小巷中的酒楼茶肆,同样是热火朝天。

天下时局一向是京城百姓们讨论的焦点,尤其是最近,谈论得就更多了。酒桌边的高谈阔论,酒客们指点江山的模样,仿佛一个个都是两府宰执一般。

李复瞥眼看了一下绘着富贵连枝图案的屏风一眼,薄薄的一面纸面,根本挡不住从隔壁传来的声浪。摇摇头,向坐在对面的范育、吕大临无奈的笑道:“外面都在说着这些事,多少天了,也不见个消停。”

“怎么能不说?”范育笑道,“章惇和玉昆打退了交贼,俘斩万余。罗兀城又是稳稳的控制在官军手中。盘踞丰州的西贼听说今年就只有三分之一的收成,粮草快要用尽了,支持不了两个月。”

“还有江南。”李复又补充着,“那里灾情听说已经有所缓解,今年的收获情况也不算很差,道路上的流民人数大幅度的减少,终于算是撑了过去。”

吕大临叹了一声,“最想不到的是王中正在茂州竟然也赢了。”

王中正自带着熙河路的援军南下茂州,只用了半个月的时间,就轻而易举的踏平了叛乱的蛮部。前后五战,斩首三千余,破寨三十余,降伏的部族有十六家。从这个数字上看,茂州蛮部可谓是元气大伤,十年之内恢复不了。而有十年的时间,朝廷对茂州的控制早就已经稳固,再想举起叛旗,只会死得更惨。

天子一开始点了王中正的将,这不算奇怪。不论王中正到底是有能无能,只要他参与的战事,无一例外都是取得了胜利。横山也好,熙河也好,都印证了这一点。这员福将,天子也不可能视而不见。只是其他几处都是由名臣良将所率领,胜也好、平也好,都不奇怪,而王中正区区一个阉人,只凭福气竟然也能取得如此大的战果,着实让许多人惊讶。

“那也是熙河军精锐的缘故。”范育说道,“赵隆、苗履都是年轻一辈中难得的将才,还有一千上山跑马的吐蕃骑兵,想输给茂州蛮部都难。”

“如今禁军兵强马壮,想必不久之后就能北攻西夏,眼望燕云了。”李复有几分兴奋,作为关学弟子,更作为一名关西人,看到大宋军力强大,心中免不了有几分欢喜。

“富国强兵啊……”吕大临则是一声感慨,“兵是强了,可这国呢?能不能支撑大战的钱粮?”

仅仅用了半年的时间,大宋就从四面烽烟、内外皆困的窘境中走了出来,一夜之间,不论是朝堂还是对于官军的信心膨胀了起来。西夏只能占据着偏僻之地丰州,面对大宋对横山的攻势,甚至连更进一步的反攻都做不到,而契丹人也只是动嘴皮子而已,到底有没有胆量来进攻中国,为西夏撑腰,实情一望可知。

新法推行的目的就是富国强兵。从一开始这就是天子的唯一目标,熙宁以来,这四个字天下人早就是耳熟能详。

因为连年灾异,国库消耗很大,富国暂时还不能说得理直气壮,不过强兵却已经是实打实的现状。军备精良,士卒堪用,也就是说王安石的新法,至少成功了一半。接下来,到底会是收复丰州,还是膺惩交趾,听说朝堂之上依然没有定论。不过更多的议论是能不能两边同时开战。

“玉昆胜得太轻易了。”范育对如今朝堂内外的议论很是不以为然,“千五破十万,斩首俘虏竟然有一万之多。如今外面都在传说,只要朝廷调选一万精兵,就足够剿平交趾、攻下升龙府了。骄兵必败,兵事岂能视同儿戏。”

吕大临与范育是同样的看法,“交趾军是兵疲师老,对南下的官军猝不及防,加之内部有变,黄金满反戈一击。李常杰焉能不败?换做了官军攻入交趾国中,情况就要颠倒过来,一个不好就免不了全军覆没的危险。才出一万兵,未免太过轻敌了。”

“不是有消息说,韩玉昆不日就要抵京了吗?”李复笑道,“这事问他最清楚。先生门下弟子,论起用兵当以他为首,我等倒也不要为他多担心。”

“希望韩玉昆能早点回来。”吕大临抿了抿嘴,“他好歹通一些医术,先生的病还要他来看一看。”

听到吕大临提起张载的病情,范育、李复都沉默了下来。张载在京中讲学一年,在门下聆听授业传道的士子成百上千,正式列入门墙的弟子也为数不少。但就在这一年中,张载的身体也日渐的衰弱。天子派来的御医昨日开出来的药方竟是药性温和的调养方子,根本就不是治病的。究竟是怎么回事,其实弟子们都已是心知肚明。

“先生的病情必当无恙,想必很快就会痊愈的。”过了片刻,范育勉强的笑了一声,转过话题,“之前玉昆南下时走得太急,身边连个幕宾都没有。玉昆前一次来信也说了此事,军中机务乏人参赞,另外邕州州学也缺人照管,最好还是要有几个同门去帮衬着。”

“想必不少人愿意去呢。”吕大临摇头。

李复脸皮一红,其实他也想去。

韩冈眼下在张载弟子中,已经是独占鳌头,在官场中走得最远。从眼前的情况来了看,身入两府只是时间问题。之前韩冈南下时的确走得急了,使得许多有心人没来得及凑过去。当现在他已经成为龙图直学士上京来了,不要他说话,多少人都要抢着来做他的幕僚,就算是南方的瘴疠,也吓不退人。

“如果我不是有差事在身,倒想去南方走一遭。”范育是入京述职,与吕大临和李复不一样,“与叔大概不愿去凑那个热闹,不过邕州州学,的确是乏人主持。今年的进士,用得全是《三经新义》,无论南北学中,都免不了功利之心。也只有岭南、关中之地,进士难得一中,方能放下这一心思。”

“邕州州学……”吕大临皱眉想了一想,问道,“前两天先生还说,玉昆写信来求一篇州学学记,是不是就是这件事?”

“对!”范育点头,“就是为新建的邕州州学来求的。”

“想不到没去求他的岳父,求到先生这边来了。”王安石文名传于天下,就算是张载的弟子,也不好说自己老师的文章能与王安石比肩,关学、新学两家,比的是天人大道,而不是咬文嚼字的章句。

“大道不同嘛,先生已经是答应下来了”

“岭南荒僻之地,当以教化为首。韩玉昆不修州衙,而兴州学,眼光所见长远。”吕大临虽然没有明说出来,当真是有几分心动了。

……………………

天气暑热,骑在马上的韩冈已经是汗流浃背。衣襟的背部,流出来的汗水晒干了之后,接着又被汗水打湿,竟凝出了一层白花花的盐霜来。只不过他一点也没在意头顶上的炎炎烈日,在行人车辆稀少的官道上奔驰着,向着京城飞奔而去。

如果他在过颖昌后,就改为乘船顺水而行,由惠民河入京,倒也不用吃这个苦。只是京城之中,不论公事私事,韩冈都有许多要处置,等不及慢慢的泛水行舟。

半个多月前,苏子元从京中返回,接下了韩冈代理的邕州知州的职务,而韩冈则是被同时抵达诏令召回京城之中。

十分干脆的放下手中的事务,韩冈直接启程上京。在经过桂州的时候,章惇也是送上了一番殷殷嘱咐。两人都清楚,如果要定下攻打交趾,就在他这一次回京了。

这个时候,三十六家溪峒刚刚攻入了交趾境内,从初步传来的消息中可以得知,他们的收获颇丰。光是解救出来的汉人,就有两千多。韩冈许了一人五匹绢作为酬劳,一下子就散出去了一万匹。不过这份钱朝堂上不论是谁,都不敢说花得不值,就算涨个数倍,都会点头承诺下来。

不过韩冈只知道南方的局势。远在邕州,收到的邸报都是一个月前发出来的,韩冈并不清楚,北方的局势眼下究竟是如何发展。

如果要收复丰州,陕西诸部肯定要配合河东的行动,对交趾攻略的影响是肯定了。另外茂州的情况如何,韩冈也不能确定。就算赢了,赵隆、苗履带去的那一批队伍,也不可能调到南方来,必须要加以休整。

如果只能动用京营或是河北军,韩冈宁可不打这一仗,也不会领军出征。他只相信战功累累的西军,而不是几十年没有打仗,已经腐烂变质的京营军和河北军。

眼前的行人、商旅渐渐多了起来,就算是热力惊人的正午,开封城周围依然是行人如织,车马如云。百万人口的大城市,就是辐射出来的余晖,也能让数十里外的城镇,拥有不逊于广西诸州的人口。

拥挤的大道上,韩冈的速度也慢了下来。前面一行车队慢悠悠的向前走着,韩冈一时超不过去,也不得不保持着同样的速度。这一慢,没了迎面而来的凉风,头顶上的烈日就分外炽烈起来,

韩冈心中不耐,随行的伴当连忙上前去,要前面的车队让一让。只是骑着马在车队前领头之人转了过来,是个熟人——赫然是冯从义!

