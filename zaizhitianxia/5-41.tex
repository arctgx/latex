\section{第五章 九州聚铁误错铸(六)}

【补更】

“鸳鸯泺当然不是夏捺钵的地方。”

赵顼脸色阴沉。他找韩冈入宫,可不是来听他幸灾乐祸的。

“陛下不必担忧。”韩冈安抚着天子,“辽人当还没有下定破弃澶渊之盟的决心。”

“何以见得?!”赵顼当即追问。

“用兵贵奇,如果辽人有心毁盟,再起干戈,就不会如此大张声势。何况郭逵已至河北,陛下勿须忧虑。”

韩冈表现出来的轻松,倒是让赵顼心中放心了些许,再联想起郭逵已经去了河北,有名帅坐镇,当可保河北无恙。

郭逵去河北,这个人事安排是韩冈推荐,同时也是郭逵自愿。除了他们两人,朝堂上,还有以王珪为首的宰执们对这项任命全力支持。不过王珪、元绛的支持,多是嫌在朝堂上碍手碍脚,不比韩冈郭逵二人,是真心担心辽人的动向。

韩冈外似轻松,但他心中对耶律乙辛的评价,却也再向上调升了一级。

自从辽主从飞船上摔下来之后,辽国的动向一直很模糊。大宋这边打探到的消息,一个是确认春捺钵一如往年去了鸭子河,第二则是辽国近期没有内乱。

而这段时间的朝野内外所热议的话题,除了迫在眉睫的战争之外,就是有关御史台中的苏轼。辽国动向甚至没有多少人去关心,都认为辽国肯定迟早会陷入内乱,眼下的平静只是各方在为决战做准备而已。

前一天,王安礼还特地为了苏轼来造访韩冈。韩冈不得不说了一通‘始作俑者,其无后乎?若苏轼以诗文得罪,日后还有谁敢做诗词?以言辞罪人,日后谁还敢说话?’的废话来搪塞。

以韩冈的想法,他只关心苏轼最后是不是以诗赋言辞来定罪,如果是其他罪名,他就不会插手,反正别的可能加诸于其身的罪名不至于要了苏轼的性命。但从当下御史台中传出的消息来看,苏轼对于李定等人强加给他的罪名几乎都承认了,也就是说讪谤朝政这一条罪证确凿,连口供都有了,以言辞论罪的结局看来是注定了——这样的情况下,韩冈只能设法保住苏轼的性命。

可是辽国局势的发展出人意料,几乎没人想到耶律乙辛这么快就将国内的形势安定下来了。从这件事上推断,要么就是他的能力的确过人一等,要么就是他请前代辽主龙驭宾天时做好了一切准备……或许兼而有之。

不管怎么说,之前对辽国的判断和预测,全都得废弃了,在讨伐西夏的时候,必须将辽人可能会有的干涉计算进来。至于苏轼,就让他继续在御史台待着吧,暂时都不会有人多余的精力,去治罪苏轼,或是为他奔走呼号。

“以韩卿之见,辽人的夏捺钵到底是怎么一回事?”赵顼向韩冈询问他的看法。

“从地理上说,驻扎在鸳鸯泺的二十万骑辽军不论是南下大同,还是东进燕蓟,路程都不远,也就几天的时间。”

鸳鸯泺的位置大略是位于后世的张家口偏北,韩冈前生曾经去过,对此有所了解。辽人南伐点兵,便多在千里鸳鸯泺,对于这一点,大宋君臣则了解得更深。

“不过以臣观之,辽人这是不甘坐视西夏被灭,故而大张声势。但要说辽人准备南侵,当还不至于如此。如果辽人当真想要支援西夏,只需暗中遣兵数万入夏境,猝不及防之下,官军全军覆没都有可能,并不需要大张旗鼓的将捺钵停驻在鸳鸯泺……耶律乙辛纵然在东京道成功平叛,但其国中人心不服当是难免。一旦他遣军南下与官军交锋,无论胜败,都有身后起火之虞。”

赵顼点了点头,神色中有几分欣慰。

韩冈是反对速攻兴灵的,他的态度至今未变。但从他对辽国的判断上,则可知其品性正直,否则必然是会拿着辽人陈兵鸳鸯泺来恫吓自己,以求改变朝廷对西夏的方针和战略。

“之前吕惠卿就是这么说的……可谓是有识之士,所见略同。”

韩冈眼神变得更为幽暗了一点,看起来吕惠卿这一次是彻底站到了王珪的一边。不过也不足为奇。最近的几个月,手实法在京畿以及京东京西推行的极为顺利,而南方诸路虽有反对的声浪,但政事堂却都强压了下去,作为利益交换,吕惠卿帮王珪说话也是必然的。

“但微臣这仅是常论。”韩冈忽的话锋一转,“一旦西夏灭亡在即,有唇亡齿寒之忧的辽人,又会怎么做,却不便下定论了。”

赵顼看了眼韩冈,声音冷了一点:“韩卿的意思朕明白了,的确应当小心才是。”

韩冈的心是七窍玲珑,赵顼心情变化,哪里感觉不到。什么明白,怕是当自己反对速攻兴灵,拿辽国眼下的动作做文章。

“所谓有备无患,就如之前以郭逵守河北,河东也得加强防备。辽人出兵的几率虽小,但也不可不备。”

赵顼的神色又缓和了一些,“河东路为了防备辽人,出兵一开始就不多。再减一些也不妨事。”

天子只想听到自己想听的,韩冈心中暗叹,‘这可就不好办了。’

看多了史书,多少发生在历史中的事件都在告诉韩冈,战略上的优势,可以因为领导者的愚蠢和贪婪而被抵消,战术上的强势,也会因为后勤等问题而灰飞烟灭。眼下的形势,似乎正要往印证这一点的方向发展。

辽夏两国都还没有动手,仅仅是内部的问题,就让宋军的优势一点点的消磨了下去。回想起当年,河湟之战以及南征之役,要不是都有王安石在朝中支持,绝不可能胜得如此干脆利落。

尤其是当初河湟开边,没有王安石帮着压制住李师中、窦舜卿和向宝的干扰,王韶和韩冈连起步都做不到,哪里能有如今的风光。

可惜如今的两府宰执,没有一个能压制得住各路争功的将帅,反而让矛盾浮上水面,要他们互相配合可就难了。军合力不齐,这样的战争虽不能说必败,但内部消耗太大,必然是让失败的几率增加了许多。

就是天子赵顼也肯定能看到这一点,但韩冈知道,自登基后,没有遭遇过一次惨痛败仗的现实,给了赵顼太多自信。

一切无可阻挡。

元丰二年四月廿一,从河东到熙河,几近四千里的国境线上,三十余万宋军攻入了西夏境内。

自澶渊之盟之后,大宋动员兵力最多、战争范围最广的一场战争,在这一天终于拉开了序幕。

种谔重新踏进了银州城,但他的身后,是精气神不及当初一半的鄜延军,以及三万不听使唤的京营禁军。但依靠兵力上的巨大优势,在出兵之后的半个月,重又顺利的攻到了夏州城下。

李宪自河东出兵,身后的兵力比计划初定时少了整整三十个指挥。不过没人知道,他甚至为此松了一口气。这下粮草的问题轻松了不少。反正从地理位置上看,河东军的作用在六路中是最小的一个。所以他不紧不慢的领军往银夏方向赶过去。不过李宪也不是放弃了军功,他没忘了分兵去攻打沙漠【今毛乌素沙漠】中的绿洲地斤泽。百年前西夏太祖继迁在末路穷途时,几次逃进地斤泽中躲避,最后一举翻身。而这一次,官军不会给他们机会。

高遵裕自环州出兵后,就率领环庆军沿马岭水的上游支流白马川北上,很快就攻破了横山中的最后一道关卡清远军城,进入了横山北麓的西夏境内。接下来就要沿着灵州川穿越瀚海,直取灵州。当然,他也没忘了银夏。正好他手上不缺兵员,分了一万人马向东北挺进,突破了青岗峡的蛤蟆寨,直逼盐州。打算赶在种谔之前,抢下恢复银夏的一半功劳。

泾原路的苗授则老老实实的顺着葫芦河河谷北上,一举攻占了兜岭中的险关磨脐隘,继而又打下了赏移口,西夏的腹地就在眼前。

至于秦凤、熙河两路联军,则是同样顺利。听从了韩冈、以及熙河路众将的建议,王中正将第一阶段的重心放在了兰州上。一路北进,很快便听到了滔滔黄河水声。王舜臣得意的第一个跃马冲进了兰州城。而在此之前,禹臧花麻便已经将庭院打扫干净,静待官军到来。依靠参与进熙河路经济体系后的充沛财力,以及眼前严峻的形势,禹臧花麻利用收买和恫吓的手段,将兴庆府派驻在兰州的三千铁鹞子中的大半人马,收归麾下。官军夺占兰州,兵不血刃。

前线捷报频传,五月中的京城是一片欢声。城中的酒楼茶肆,多少人举杯为官军的高歌猛进而欢呼,诗词、文章一篇接着一篇。

可是韩府之中却有异声。韩冈在灯下问着王旖:“你可知迄今为止,报上来的斩首有多少?”

见妻子茫然摇头,韩冈叹道:“加起来都没有两千啊,坚壁清野、诱敌深入的方略,已经是很明显了。”

从时间上算,走得最快的环庆、泾原两路,应该快要进抵灵州城下了。真正严酷的战斗这时才要开始。

