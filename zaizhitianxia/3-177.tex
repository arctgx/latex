\section{第47章 节礼千钧重(上)}

劳碌命的蓝元震被赵顼点了将,匆匆忙忙的带着人下了城,往兴国坊去了。见着韩冈一人站着,四周却是一圈宴席,赵顼想了想,还是给他赐了席。

等到韩冈辞让两次后,谢恩落座,赵顼便又问道:“韩卿,你先说一说你的板甲。”

“臣遵旨!”韩冈一拱手,朗声道:“军器监所产军器,不论大小,皆定有规格式样。如今监中五十一作,皆秉定规制作,不敢有丝毫依违。故而近年来,军器精良大胜过往,此乃吕参政和曾都承之功也。”

赵顼看了眼吕惠卿,“韩卿说得是。”

吕惠卿欠身一礼,却是连笑都没有。韩冈既然这样说话,分明下面就会有转折。

韩冈端正的跪坐着,向着赵顼:“依监中定制。一副连盔札甲,也就是步人甲,甲叶共计一千八百二十五片。分为披膊、甲身、腿裙鹘尾和兜鍪帘叶四部分。其中甲叶数目、轻重各不相同。披膊八斤三两,需甲叶五百零四,每叶二钱六分;甲身九斤十二两四分,甲叶三百三十二,每叶四钱七分;腿裙十九斤一两五钱五分,甲叶六百七十九,每叶四钱五分;兜鍪帘叶四斤十三两五钱,甲叶三百一十,每叶二钱五分。另有兜鍪杯子眉子重一斤一两,皮线结头等重五斤十二两五钱,总重四十八斤有余。【注1】”

韩冈如数家珍,一个个数字脱口而出,不厌其烦,让天子看到了他对本职工作的熟悉程度。赵顼听得也不自觉点起了头——韩冈上任才多久?

“而要制作这样的一副札甲,所需人工在一百五十个工【注2】。如果是弩手甲或是弓手甲,则会稍少一些。”韩冈望向吕惠卿,“此制乃吕参政所定,应该最为清楚。”

对上天子投过来的视线,吕惠卿有些勉强的点了点头,“韩冈说得没错,弩手甲用工一百二十,弓手甲和长枪甲一百四十,而步人甲用工最多,为一百五十个工。”

一人一天的工作量,标准是十个工。也就是十二到十五个人费上一天,或是一名工匠十天半个月的时间,才能打造出一幅铁质札甲来。而皮制的札甲,也并不比铁札甲省时省力到哪里去,这就是提供给普通禁军士兵的甲胄。若是甲叶更小,制作更为精细的鱼鳞甲,甚至山文甲,则用工更多,更为耗时耗力。

听着韩冈将甲胄的细节娓娓道来,赵顼心中的火气不知怎么的渐渐的开始消散。忍不住问道:“那韩卿你所造的板甲,用工又是多少?”

“微臣调集工匠开始打造板甲,是在得到了锻锤之后。也就在正月十三,即两天前才开始制作。总共用了六名工匠,两天的时间,已经造出了八副出来,正好是一副十五个工。”

‘这不可能!’吕惠卿差点要大叫起来。而冯京、王珪等人也是脸色骤变。吴充甚至惊得将手中的金杯给捏得变形:“十分之一!”

赵顼身子前倾,追问道:“韩卿,此话当真?!”

韩冈拱了拱手:“臣不敢欺瞒陛下。因为微臣挑选的这几位工匠,都是监中年资精深的大工。如果普通的工匠,大概耗时要多上一些,不过绝不会超过二十个工。”

宴席上一时间静默起来。同样数量的原材料,用得人工越少,自然越便宜。如果韩冈说的都是真话,他如此自信也就不足为奇了。

吕惠卿咬着下唇,他本不相信韩冈能打造出合用的甲胄来。以他在军器监的耳目,韩冈的一举一动都瞒他不过。但如果是以试用锻锤的名义,调集五六个工匠封闭实验,韩冈这个判军器监想瞒下几天也不是很难。

这个急就章的成果,到底合不合用,其实还有待证明。但在座的几位宰执,看着韩冈脸上浅淡的微笑,都已经有了失败的预感。

板甲送到了,兴国坊离着宣德门很近,蓝元震往返只用了两刻钟的时间。连带着还带来了两名留守的工匠。

所谓的板甲,当真恰如其名,就只是几块宽大的铁板而已,叮叮当当的被堆在了地上。

赵顼让人拿了一副过来,仔细看着。上面并无装饰,但表面上不知怎么的却是泛着莹莹铜色,摸着很光滑,打磨得很不错。

“微臣的板甲分为四部分,胸甲、背甲,双肩肩甲,还有裙甲。兜鍪还没有来得及造。”韩冈站起身,找来一名身材适中的班直,让两名工匠帮他将板甲穿戴到身上。

班直穿着甲胄,在天子面前转了一个圈。甲胄反射着天上的明月和城头上的火炬,乍看起来比起寻常士卒所穿的铁札甲要漂亮许多,丝毫不逊鱼鳞甲,甚至可以跟明光铠相媲美。不过在每一片甲片的连接处,都是凿了洞用皮绳紧紧的绑起来,一看就是个粗糙的廉价货色。

一众宰辅心中大叫,难怪打造得这般容易,板甲的结构比起札甲实在简单太多了。就是将铁板弯成合适的形状拼凑而起,哪像札甲,要一片片的去磨制、打洞和编织。

赵顼用力按了一下甲片,纹丝不动,问韩冈道:“不知此甲是否坚固?能挡箭矢否?”

“微臣不敢妄言,请陛下命人用神臂弓一试便可。”

赵顼哪能按耐得住,命人取来神臂弓,就在皇城的城墙上实验起来。

“韩玉昆信心十足,想必板甲是远超札甲了。”蔡挺侧着脸对王韶说着。他跟两边都不占,韩冈之事他也是抱着旁观的心思。现在看着情形,韩冈多半要赢了。

王韶微微一笑:“只要与札甲差不多,肯定就能赢了。”

“明修栈道,暗度陈仓。韩玉昆玩的这一手还真是漂亮。”

“附带啊!没听韩冈说只是附带吗?”王韶呵呵笑道,将杯中酒一饮而尽。

生铁很便宜,就算是将生铁反复锻打后得到的熟铁也不贵。铁和皮件等原材料,在甲胄的成本中只占了一小部分,人工费用才是大头。

要造出一副札甲实在太难了,琐碎的程序也太磨人。先得打造一千八百多片不同大小的甲叶,然后经过打札、粗磨、穿孔、错穴并裁札、错稜、精磨等工序。而将甲叶制好以后,还要再用皮革条编缀成整套铠甲。

地方军州中的军器院打造甲胄,一年也不过百来具,人工耗费却是如泼水一般。甲胄成本高昂得即便是富庶无比的大宋,都感觉难以承受。

换成是韩冈拿出来的板甲。则就是胸甲、背甲、左右披膊,还有就四片裙甲几部分,用得铁料虽不会比札甲少太多,但工时一下降到只有十分之一的水平。论起一副甲的造价,则最多只有之前的两成。

若是寻常商家,若是新货比旧时货成本降个一成两成,肯定会说一句:‘这个买卖做得’,若是一下降到五分之一,那就不是轻描淡写的一句‘做得’,而是得拼命的抢着去做。

能在短时间内大批量制造,并且成本只有过往的五分之一,只要板甲能保证跟之前的札甲有着差不多的坚固程度,那韩冈说一句不逊于神臂弓,哪个能否认?!

“依律私藏弩五具者斩,而同样的刑罚,只要私藏甲胄三领就够了。”王韶端起空掉的金杯,示意身后的内侍满上,笑着对蔡挺道:“甲胄可比弩弓要金贵啊!”

踩着脚蹬,将弩弦扳到了牙发上,将一支木羽短矢放进弩槽,张若水举起了神臂弓。

他就是当初拿着神臂弓在赵顼面前做实验的内侍。在七十步外洞穿铁甲,不仅仅试出了神臂弓的威力,也证明了他的射术。时隔数年,他再一次于御前举起神臂弓,不过测试的不再是掌中的重弩,而是作为标靶的甲胄。

在火光下,闪着莹莹精光的板甲就放在五十步外,举起神臂弓,箭矢所指比目标略略向上。调匀了呼吸,手指一扣,不到一尺长的木羽短矢便飞了出去。

当的一声轻响,从五十步外传来。

“如何?”赵顼急问道。

板甲被搬回来,外观丝毫无伤,只是仔细的摸上去,才能发现正中间有个浅浅的凹坑。这已经完全超过了过去的成绩了。

赵顼的手都颤了起来,成本可能只有过去的五分之一,甚至十分之一,而坚固则远远胜之,这是让中国军力更胜西北二虏的发明啊!

周围多少人的眼中透着失望,但赵顼的眼中只有兴奋,这是最好的上元节礼物!

注1:这些数据,参见《宋史·兵志第十一》。另外,古代的一斤为十六两,请参考半斤八两这个成语。

注2:参见朱熹《与曾左司事目箚子》:‘打造步人弓箭手铁甲,一年以三百日为期,两日一副,昨已打造到一百五十副了毕……窃缘上件铁甲计用皮铁匠一万八千,工钱五千二百余贯。’一百五十副铁甲,用工一万八千,平均一副一百二十个工,相当于现在的十二个工作日,仅是人力成本就要三十四贯多。

