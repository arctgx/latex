\section{第26章 仕宦岂为稻粱谋(上)}

“学而时习之,不亦乐乎;有朋自远方来,不亦乐乎;人不知而不愠,不亦君子乎?”

辛苦了半日,韩冈终于可以休息下来。温煦的阳光驱走了冬日的寒意,没有了呼啸而来的北风,坐在室外也不会太过难耐。韩冈便靠坐在一条木质的长椅上,高声诵读着《论语》中的篇章。他半闭着眼,手抚在书页上,其实并没有去看书本,但烂熟于胸的文字,从口中放声而出,并没有一丝滞怠。

韩冈诵读经书,来来去去忙碌着的人们走过他身边时,皆放轻了脚步,不敢打扰到他。甚至其中还有许多,都要冲韩冈躬身行个礼,方才走开。

“什么时候都不忘读书,真不愧是秀才公。”

“听说秀才公每天忙着营里的事不说,夜里都要读书读到近三更。”

“秀才公可是有大学问,连京里来的大夫,还有有名的仇老大夫,都是佩服得五体投地。你想想,孙真人都出来为秀才公治病,不是天上的星宿能请得动吗?”

“别老是秀才公,秀才公。很快就该叫官人了。老都监不是已经把荐章递了上去吗?等过几天,那就是真正的官人了。”

“听说是请秀……韩官人管着秦凤路所有城寨的伤病营。以后好了,得了伤病也不至于再枉死。”

许多人小声议论着韩冈的勤学苦读,还有韩冈即将担任的官职。有羡慕的,却没有嫉妒的,在甘谷城中,但凡见识过伤病新营的人们,都有同样的共识。

他人的议论没有影响到韩冈的诵读。好学,勤学,手不释卷,这是一个很大的优点。韩冈的前身留给他一肚皮的经史,但记忆是会随着时间渐渐消退,必须时常温习。才学是根本,与士大夫们一起闲谈,总不能对经史典籍一窍不通,一个与论语、诗经有关的笑话说出来,别人哈哈大笑,自己却懵然不知,那自家就成笑话了。

韩冈身下的长椅刚刚打造好,还带着新木器特有的味道。椅身正对着南方,可以晒到冬日难得的阳光。这样的长椅,现在在伤病营中有十一条——半月光景,被改作伤病营的甘谷城东南的空营地,已完全变了一副模样。

自从前日张守约将这间空军营让给韩冈打理。韩冈并没有客气,将成纪县来的民伕全数转为护工,指派着城内的工匠和民伕,将伤病新营从内到外改头换面。

营地大门外,还挂着一个甘谷疗养院的牌子。疗养院这个名字是韩冈所起,而题字则是韩冈请张守约亲笔题写,字虽不周正,但此举却体现了韩冈对张守约这位都监兼知城的尊敬。

军营的宿舍,一例都是从一头通到另一头的通铺,只有军官才能例外睡个单人间。虽然时间不多,无法为伤病员打造单独的床榻,但韩冈还是在重新粉刷界地之后,设法用木板竖在通铺上,隔出了单间。十四间大小营房,除去护工的住所外外,总计可以容纳两百三十张床位。伤病员们按照疾病伤患的轻重和类别,被安排在不同的营房中。每一间营房都有数量不等的专职护工,其中重伤重症,甚至会有护工一对一来照料。

营房之外,还有一间濯洗房。濯洗房没有墙壁,只是个棚子,里面的几口大锅不停的冒着热汽,这是用来蒸煮伤病员换下来的床单和衣物,进行消毒。那些床单和衣物,先通过流水清洗掉上面的污物,再经过高温蒸煮,晒干后再发回使用。

所有在营中负责打扫洗濯的,都是伤病员们亲友,还有伤病员本人。韩冈通过教育和辅导——也可以说成是宣传和洗脑——让他们明白互助互利的好处。不用花一文钱,就连能走动的伤兵,都主动出来打扫,保持环境的整洁。

朝南的一面空地,就是韩冈让城内的工匠打造的一溜有靠背的长条椅,等日头好的时候,伤病员们可以坐着晒晒太阳。这之外,他还在营内留下了花坛的位置,准备到春天的时候,再移植些草木过来。同时在计划中,韩冈还打算将营地内的道路改成石子路,而不是一下雨就烂汤的黄土路,反正是伤病营,也不用担心石子路会崴伤战马的四蹄。还有要开挖下水道,用暗沟来排出污物,而不是现在的明沟。

还要做的事情很多,现在仅仅是开了个头。但这座伤病营,或者叫疗养院,已经博来了无数惊叹的目光,也为韩冈博来了一个从九品的武官官职。

“十室之邑,必有忠信如丘者焉,不如丘之好学也。”

读到这里,韩冈合上了书册。不经意间,他已把二十卷论语背了四分之一。

‘经书就是短啊!’

韩冈站起来伸了个懒腰。经典本章传承自上古,字数通常很少,只占需要背诵领悟的很小一部分。但历代以来的注释却千百倍于此。经不通有传,传不通有注、注不通有疏,疏不通还有补注、补疏。要想将古往今来浩如烟海的文章都背下来,再多一条命都不够。连他身体的原主,都只背下来了其中比较重要的一部分。

当然,利用已经背下的文字和自己别出机杼的阐发,在学术水平普遍不高的西北边境,韩冈说不定还能混个贡生,去开封走一走。但如今的进士科举,又与这些经典关系不大,考得是诗词歌赋。没有半点诗才的韩冈,不可能有指望中个大奖。

读书读得累了,韩冈正要回营房巡视一圈,以作休息。一名护工脚步匆匆的小跑着过来,“韩官人,门外有个王大官要入营!”

“王大官?”韩冈愣了一下,心中计较,多半是王韶来了,他认识到王姓官员也就王韶一人。连忙道,“我这就过去。”

韩冈向营地大门走去,暗自冷笑。不管怎么想,王韶都不可能无事跑来甘谷,若是会有什么事,想必就是应该落在自家的身上。真得多谢张守约,他这一举荐,王韶就坐不住了,这买涨不买跌的股民心态,千年前倒也一样有!

不过这对韩冈他也是好事。两家相争,自己待价而沽,总能卖出个好价钱。原本还担心向宝暗中做些手脚,耽误了自家的前程,现在多了经略司管勾机宜文字——相当于后世军区参谋长的高官来举荐,韩冈也不必担心再会有什么波折了。

……………………

“这是伤病营?!”

站在营门门口,王韶有点楞。眼前的这座改名叫疗养院的伤病营,完全颠覆了他过往的认识。没有了普通伤病营中那种死气沉沉的感觉,也没了普通伤病营遍地的污秽。伤病们在营中四处走着坐着,互相谈笑。他们的伤口上都绑着干净的绷带,眼神中也不是如过去那般空洞无物,而是多了名为希望的神采。而一些臂上扎着蓝色布条的役夫,则略显匆忙的打扫庭院,搬运衣物。但看他们的神情,却也没有役夫脸上惯常见的麻木,而是日常生活中才有的平和笑容。

自从担任秦凤路机宜之后,王韶走过军营很多,见识不可谓不广。根据不同的时间,或是不同的将领,军营可以是喧闹的,可以是寂静的,也可以是悲伤的,还可以是愤怒的。但一座干净清爽,甚至带着一点家庭温馨的军营,他却从来没有见识过……

这还是一座聚集了所有伤病的军营吗?这个奇迹韩冈又到底是怎么做到的?!

‘韩冈……韩玉昆……’王韶默念着奇迹之手的名字,‘玉出昆冈。这块璞玉还真是不简单。’

王厚却没有自己的父亲想得那么深,看着脱胎换骨一般的伤病营,只是啧啧的赞了两下,便急急入内,连声的要找韩冈说话。

“不要急!”王韶唤住毛毛躁躁的儿子,眼望前方,“人已经来了!”

远远望着营地大门处王韶、王厚父子俩,以及围在左右的一队护卫,韩冈仍是不徐不急的走着。一派宠辱不惊的气象,将名门弟子的风范淋漓尽致的表现出来。

大概是来回奔忙的缘故,比前次见时,王韶貌似又黑瘦了一分。走到近前,韩冈行礼如仪:“学生韩冈见过机宜。”起身后,又和王厚行了平礼,打了个招呼。一套礼仪做的滴水不漏。

儒家尚礼,此时儿童开蒙入学,第一件事不是认字,而是学礼。吉礼、凶礼、宾礼、家礼,待人接物,言谈举止,其中的礼仪都是要仔细学习。不同的场合,不同的人物,所适用的礼节也都不尽相同,错上一点,便是惹人议论。‘有礼仪之大谓之夏’,这一句不是乱说的。而张载是儒学大家,对于礼法的认识和见解,自然无不精通。韩冈作为他的门生,当然浸淫甚深。平日里表现出来的气度,也是来自于此。

领着王韶父子入营,韩冈一边介绍着周围,一边漫不经意的问道:“机宜和处道兄此来,不知为得何事?”

ps:韩三气定神闲,稳坐钓鱼台,现在轮到王韶反过来求人了。

今天第三更,继续征集红票,收藏。

