\section{第六章 日暮别乡关(上)}

“官人。”

从酒宴上离开,韩冈先去了周南和素心那里,说了两句话后,便到了这边来。听着少女娇柔的呼唤,他微醺的脑中,有了一丝恍惚。忽然觉得眼前露出纯美笑容的少女有些陌生,恍惚过后,才发觉三年前的记忆又重新浮了上来。

韩冈还记得三年前,一丝劫后余生的游魂初次投身到这个陌生世界。刚刚睁开眼时,第一个出现在自己身边的,就是现在坐在床边,向自己展颜而笑的少女。

三年之前,他躺在病床上苟延残喘,在父母出去为了药钱而张罗的时候,就是眼前的少女在悉心照料着自己。

现在韩冈想想,自己当时还真是没心没肺,安心坐享父母的辛劳。虽然是因为初来乍到,与父母还有些疏远的缘故。如今回想起来,心中总是少不了一份愧疚。

但对于韩冈来说,那段与云娘耳鬓厮磨的日子,也同样是值得回味的快乐时光。他当日冲冠一怒,也是为了眼前的少女。

三年间,他在官场上,历经了多少惊涛骇浪,一步步走到了如今的地位。在他的面前,是通衢坦途,升到宰执的地位,为天子牧守万民,也许并不需要太长的时间。但又有谁能想到,在刚刚开头的时候,韩冈所想的,其实只是要保全自己手上刚刚得到的小小幸福。

尚未长成的少女,轻柔的唤着自己‘三哥哥’的声音,就是当年韩冈一番初衷。一时间,他还不想放弃。

韩冈笑了,对着今年即将成为新妇的少女:“还是照旧时一般,叫三哥哥的好!”

少女不解睁大眼睛,疑惑的眼神中甚至有了一丝惶急,不知韩冈为何这么说。

韩冈坐到床边,婚床上的大红被褥填着厚实的棉絮,显得十分松软。轻轻搂过纤细得仿佛稍稍用力就会折断的肩膀。代表少女身份的丫髻,已经换成了妇人的发饰,发髻上还插着珠花、金钗。

韩冈凑近了,嗅着从她身上散出的淡薄清雅的女儿香。

他低声诉说着:“这世上,能这么叫我的,可就云娘你一人。”

云娘转忧为喜,她怎么会拒绝成为韩冈心中唯一的一个,“……三哥哥,三哥哥……”

她一遍又一遍的念着。

在宴席上,都是亲近的自家人,就连高遵裕,都是从冯从义那里,有了亲戚的关系。王舜臣和赵隆都没有劝酒,韩冈喝了几杯后,也只是微醺。但纤柔娇弱的绝色少女,轻声而又亲近的唤着自己,韩冈却不免沉醉了下去。

房中点着两支红烛,上面讨喜的绘着龙凤祥云。烟气不重,还隐隐带着香味。只有京中大户人家才用得上的香烛,是冯从义搜罗了过来,今天送上,也是代表了他的一片心意。

而同放在桌上,一盏银壶,一对银杯,是高遵裕的赠礼。精美绝伦的花式,还有细细雕刻出鳞片的四爪蟒纹,是高遵裕今年从他的侄女那里得到的赐物。

韩冈搂着少女站起来到了桌边,拿起银壶。手腕半转,一缕清泉从装饰成龙口的壶嘴中流出,来自京中的名酒醴泉,倒满了两支酒杯。

跟着韩冈一同拿起酒杯,中间有一条三尺长的红线相连。大概是韩阿李忘了传授这方面的常识。韩云娘捏着酒杯,不知道下一步该怎么做,便仰头望着韩冈。

云娘略凹的眼窝中,浅褐色的双瞳带着水光。宛如两池清潭,似浅实深。一望之下,整个人都要给陷进去。

韩冈深深的对视着:“这是合卺酒,也叫交杯酒,学着我来。”

合卺酒,依照礼制,应该用的是名为‘卺’的葫芦瓢。不过到了此时,不是贵家的嫁娶,就已经没有这么多规矩,在两支银杯下方缠上红丝线已经足矣。

韩冈喝了一半,等少女同样喝下一半后,就跟她交换了手上的酒杯。

同样一饮而尽,云娘不胜酒力,只喝了一杯,呛咳了几声,便是两团红晕飞上面颊。韩冈的手抚上去,光滑细腻的触感中,还有滚烫的热力。

少女白天被开了脸,脸颊上细细汗毛都被用线绞了去。到底是有这一点西域的血统,云娘比素心和周南还要白皙一点的肤色,并不需要擦上太厚的脂粉。淡淡的抹上一层香粉,便已是让人惊艳。

同样是来自京城中的日用品,比起常见的铅粉要好得太多。因为韩冈的告诫,家中的女眷用的都不是含铅的香粉。而且韩冈在医学上的权威性,也让铅粉在陇西城中的梳妆匣内几乎绝迹。亲上去,唇间只有淡淡香气,不用担心会铅中毒。

喝过合卺酒,重新坐回到床边。

知道已经到了最后一步,少女一下变得紧张了起来,心头砰砰的剧烈跳动,身子僵硬的坐得不敢稍动。

抄起纤细的腰肢,将少女搂近了,韩冈吻了过去。唇舌纠缠,一段缠绵悱恻的长吻之后,四唇分了开来。云娘双目迷离,失了神一般,极速喘息着,身子则是瘫软了。

韩冈一件件的将佳人身上的喜服脱下,如同一件精美绝伦的艺术品的娇躯逐渐出现在他的面前。

这一段时间中,云娘都是闭着眼睛,任其施为。纤细柔弱的身材却是瘦不露骨,细腻的肌肤带着珠光一般的色泽。躺下来时,胸口只有微微凸起,但握上去,丰软却能填满掌心。与周南、素心同样的让人迷醉,却又是一个截然不同的类型。

韩冈暗自感谢着上天对他的眷顾,分开了少女圆润的双股,贴了上去。

外面的鸟雀吱吱啾啾的叫了起来,阳光照在窗户上,屋中变得亮堂堂的。

一夜风雨过去,烛泪斑斑,顺着烛台上流了下来,在承托上聚集成一摊鲜红。而床上的一幅白绫,也是被染上了斑斑红泪。

韩冈已经醒了,坐起了身,经年打熬筋骨锻炼出来的健硕胸膛和粗壮的手臂都露在外面,而那幅少女初染的白绫,就在他手中。

云娘也醒了过来,看着自家三哥哥正捏着那幅羞人的白绫,小脸一下都红透了。一把从韩冈手上抢过来,藏在了枕头下。头埋在松软的枕头里,怎么也不肯抬头。

如今的世人多用的虽是木枕、瓷枕,但韩冈却是睡不惯,让人用粟米糠为芯做了睡枕。松松软软的枕头,睡着舒服。但云娘如今用来藏着脸,却成了让鸵鸟藏头露尾的沙土。

“都是夫妻了,还有什么害羞的。”韩冈轻笑道。

听了韩冈的话,云娘勉强转过头来,还是红着脸。

“还疼吗?”

少女点点头,但马上又猛力摇起了头。

“到底疼还是不疼?!”

云娘羞涩不已,拖起被子盖着脸,就在被子下点起了头。她是三女中最为纤弱的一个,初承风雨当然有些娇弱不胜。

韩冈掀开被子,娇嫩纤细的身躯顿时就暴露在阳光下。细腻的肌肤吹弹可破。白玉一般的雪股粉.臀间,还有残沥一般的鲜红。

“三哥哥!”云娘惊叫着。

韩冈起身下床,又回身将被子重新给盖好,“你先歇一歇,过一阵起来去见爹娘……”他凑到少女耳边,调笑得轻声说着:“今晚再继续。”

云娘的脸一下又红了,再次埋头躲进了被子下。

自从收了云娘之后,韩冈的生活又多了一份快乐。没过几日,周南和素心也都各自都从产后恢复了过来,与云娘一起侍候着韩冈。在读书之余,他轮番享受着不同类型的三位佳人的侍奉。偶尔兴致起来,一床四好也是有过那么一两次。

就在韩冈一边读书习文,一边安享红袖添香的快乐的时候,一艘官船正沿着繁忙拥挤的汴河,渐渐驶近的大宋帝国的首都——东京开封。

“终于又回来了。”

一名官人立足船头上,望着迎面而来的一座座如天上飞虹的拱桥,长声而叹。南方士子才有俊雅的容貌,带着一点闽地的口音。三十出头,四十不到的样子,身上的官袍却已经是六七品的绿色。就算在京城中,三十多岁就能成为朝官的也不多见。

视线从衣袍上的深绿色收回,那官人暗叹着。如果没有耽搁了这三年,得赐绯银那是应有之理。哪像现在,当年都不需要自己站起来相迎的年轻人,都已经爬到了自己头上,同样都是一袭绿袍。自己的袍服还是当年天子的恩赐,而那一位可是名正言顺的七品官了。

不过也只是现在而已。他的资序都已经到点了,只要复任之后进了馆阁,转眼就能升上去,倒不用嫉妒年轻人的运气。

眼见着东京城已然在望,随行的老伴当走了上来,问着:“官人,入京后先去哪里?”

那官人考虑了一下,却见着前面的虹桥上站着一群人,正朝着自己所在的这条官船指指点点。

他微笑着站直了一点,双手相持,垂在身前:“不用多想来,来迎接的人已经到了。”

老伴当正要再问,只见着岸边跑来一匹快马,朝着这里喊了起来:“那边的,可是福建泉州吕中允的船。”

官人让伴当叫船夫靠过去,对着岸边来人拱起手:“正是吕惠卿!”

