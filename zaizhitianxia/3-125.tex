\section{第33章 道远难襄理(下)}

韩冈接旨叩拜之后,站起身来时,就已经不再是白马知县,而是提点开封府界诸县镇公事。

据韩冈所知,他现在的这个职位,一般是由正七品的员外郎一级的官员才够资格担任,他岳父当年是任过群牧监判官之后,才又转任此职。而以韩冈自己的预计,他结束了白马知县的任期后,应该是先入朝一两年,再从知军或是下州知州开始外任一任,接着再入朝任职,继而再外放,才能得到相当于上州知州的职位。

如自己入官两载即为朝官,或是王韶出外五年即升执政,又或是曾布、吕惠卿从京官到翰林,也只用三年的那等机缘,其实是可遇而不可求,很难再复制。任官几十年的官员,这等超迁的机会,绝大部分人是碰不到的,运气好的最多也就那么一两次而已。韩冈并不知道他这算是第几次了,但可以肯定,绝大多数的官员对他的经历都少不了羡慕或是嫉妒。

升迁太快其实也是有麻烦的。汉武帝时,方士栾大谎言有不死药可献,武帝大喜,不但封其为五利将军,还将公主嫁给了他,数月之间,就变得炙手可热。但等到一年之后,谎言拆穿,栾大便登时被腰斩于市。眼下天子一下将自己连提数级,可见他对安置河北流民的心情有多迫切,若是不能让其满意,那结果肯定也是不会太妙。

不过韩冈从不怕附带着好处的麻烦,现在赵顼既然肯给,那他就敢拿。

韩冈起身后,蓝元震向着他一礼:“还请提点多多用心,无负天子所望。”

官场称谓,正常的都是选高的来叫,不会有所差错。韩冈原本的知县差遣要远小于本官右正言,所以基本上对此有所了解的人,全都称呼韩冈为正言。而现在韩冈的府界提点要比正言级别高,他自然就又被改称为提点——这官场上的称呼,半点也错不得,否则就要得罪人。

韩冈则回礼道:“请供奉回禀天子,韩冈得陛下重恩,必竭心尽力,善抚流民,使之日后能安然返乡,不至为陛下、两宫之忧。”

蓝元震笑道:“既得提点此言,元震便可安心回宫缴旨了。”

说是这么说,但韩冈接下来肯定要挑时间进京一趟,直接面见天子,陈述自己的应对方案。而且要尽早——

——府界提点的衙门马上就要移到了白马县,虽然这代表着让韩冈全权处理河北流民之事。但也因此,韩冈他也得耗费一段时间来搭建位于白马县的府界提点新衙门。

与此同时,韩冈还要处理好与同僚之间的人际关系——府界提点照规矩都是由两人同时担任——所以他要及早去京城,衙门迁移的事情不能全都交代给另一位提点处理,许多资料、档案、籍簿都是工作上少不了要借助的,而措办公事的人手,也要从东京城的旧衙门中拉出来一批。

事情不少,要操心的地方也多了很多,不过韩冈仍是精神抖擞,他很喜欢这样的挑战。

后院这时送来一大一小两个包裹,韩冈示意下人递给蓝元震身后随行的小黄门。

依照世间惯例,朝臣受诏之后,只要不是贬斥,都要封一封礼金,或是银钱,或是绸绢,来谢过传诏的使节,并不能算是贿赂。韩冈本是要吩咐下人去后院取财物,但自己的这位夫人,的确是贤内助,不待韩冈说出口,就做得妥当。

“多谢提点。”蓝元震收了礼之后,拱手又谢了一声。

韩冈则道:“想必供奉此来,还有相度流民的差事吧?”

蓝元震怔了一下,不意韩冈竟然直接说了出来。有些尴尬的点着头,“……当然。官家也想了解一下提点安抚流民的手段。”

作为天子近臣,蓝元震来白马县,他身上所负担的使命,不仅仅是宣读诏书,理所当然的还有更深入的察访韩冈安置流民的情况。

这应该算是秘密使命,韩冈哪能不知,但他还是很干脆的点出,他为流民做的一切,无不可对人言,无不可让人查探。坦诚一点,就更能显出自己的信心。

他抬手抱拳:“此事供奉还请自便,县中各处供奉可任意查看。韩冈有急务在身,不克作陪。”

“不敢,不敢。”

蓝元震连声谦让。他宣诏之后,就是一个内东头供奉官而已,身上带着管勾皇城司的差遣。韩冈什么身份,蓝元震哪敢让他作陪?

向曾经有过数面之缘的王家二衙内行过礼,蓝元震带着一同来的小黄门,还有一队侍卫告辞离去,根本都不要韩冈安排人手引路。

天使告辞离开,韩冈出门相送。他的两个幕僚则在后面窃窃私语——游醇还在县学中——讨论着这一任命。

“想不到提点晋升如此之快,当真是命数。”魏平真摇头感叹着。随着年纪见长,他对虚无缥缈的命运就越是信之不移,自己给人做了一辈子幕僚,都没能混到一官半职,而韩冈却似乎是毫不费力,时时刻刻都能撞到机缘。

“命数也是提点自己挣来的,换做是你我,都是在路上就咽了气。”相处久了,方兴看的出魏平真在想什么,笑着安抚了一句。又道:“若能提点能将流民一事妥善措置,等旱情消退后,甚至可以再进一步!”

魏平真则是干笑了两声,道:“诏书中并没有说白马县由谁来接手,看来得让侯县丞来代管政事。”

方兴冷笑着:“府界提点的新治所就在白马县中,想必侯敂知道该怎么做。”

官员职位的交接,有两种不同的情况。如果即将离任的官员事先已经定下差遣,正常的手续就是将手中的政务交割给副手代管。若是没有定下差遣,还要到京中守阙,那就得等到新官上任之后再亲自交割。韩冈现在既然已经接下了新的任命,那么白马县中的事务,就得交割给县丞侯敂来处置。而以侯敂的识趣,不会做出蠢事来的。

“提点既然已经接旨,这两天肯定要去京城走一遭。”魏平真说道,“入宫请对自不必说,王相公和孙知府那边,提点也都要去见一面。”

“孙永……”方兴沉吟道,“他是潜邸中人,可年过五旬方得为开封知府。而提点才二十出头,不知他会怎么看提点。”

“孙曼叔为人中平宽和,行事颇正,勿须担心。”魏平真对现任开封知府有所了解,同时也不喜欢方兴这么说人,他举例道,“之前提点动用常平仓存粮,他也没有从中阻挠。”

韩冈和王旁这时正好回来,闻言就笑道:“孙曼叔那边的确不用担心,见一面而已,我又不与其争权,为国尽力,想必仁人君子都不会在这时候扯我后腿。”

韩冈在东京城时曾见过一次孙永。他任白马知县,没有有不去拜见长官的道理。去年见的一面,虽然只是泛泛的尽了一番礼节,寒暄了一阵便告辞了,并无深交,但现任开封知府还是给韩冈留下了不错的印象。

这半年来,韩冈在白马县的一番举措,虽然有天子和王安石做后台,但孙永怎么说也是顶头上司,有资格和充分的理由插手,但他并没有拖后腿,让韩冈一切布置得十分顺利。只是这一件事,韩冈就要承他的人情。

但韩冈的话,却也提醒了两名幕僚另一桩事。方兴皱起眉头,担心道:“既然是提点府界,总不能只管着白马一县。可那些知县不知道会怎么想,说不定其中会有人不乐提点见功。”

魏平真这一回则点头表示同意:“世间君子少而小人多,开封赤畿二十余县,其中妒贤嫉能之辈必不会少。”

王旁一听心惊,连忙对韩冈道:“玉昆,此事不可不虑!”

韩冈则不以为意,“做人做事最忌的就是乱伸手,我也没空将手插进县中事务里去。将天子关心的事情做好就够了。”

韩冈没余暇与开封府中的二十多个知县打擂台,烧火也好,争权也好,并不是眼下的急务。只要将流民营仿照白马县的制度在开封府中建起来,不让大股的流民进抵开封城下,就算完成了赵顼交代的任务。届时就算自己没有因功升迁,坐稳位置也是肯定的。到时候,地方上的知县们,搓扁捏圆全都凭自家的心情了。

“在京库场要抓在手上。”韩冈知道何为重点,“不论粮库还是草场,皆不涉县中事务,要掌控住也容易,而有了粮食,指挥流民就方便了。先顾着府中库场,日后再论其余。”

“恐怕还会有人不知好歹。”魏平真摇摇头。他已是五十岁的人,世间的人和事,他见过的和看过的,不知有多少。韩冈的际遇实在太招人嫉恨,若有机会,想让他跌个灰头土脸的绝对不在少数。那等心怀诡谲之辈,也不会放过眼下的时机。

“这没关系。”韩冈则是咧嘴一笑,笑得温雅醇和,如同此时的春风:“那时候,会让此辈知道我韩玉昆的手段!”

