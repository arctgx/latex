\section{第29章 顿尘回首望天阙(17)}

通往新郑门的州西大街边的李七家酒楼,在东京城七十二家正店中只能算倒数,生意远远比不上邻街的会仙楼,但清静也有清静的好处,如今正处在风尖浪口上的二大王——雍王赵颢——也就是因为这里的清静,不会遇到认识他的闲杂人等,才会过来坐上一坐。

赵颢也是得空跑出来散心的。虽然回去少不得要到保慈宫领一顿骂,但留在宫中就更为憋闷。城东的风流去处是不好去了,容易碰到认识他的人。幸好东京城够大,城东去不了,就到城西来。

赵颢一身便服,让人看不出他本来的身份。不过质地华贵的衣料,挂在腰上的玉佩,还有靴子上银线绣着的花纹,乍看上去就是官宦人家的佳公子。上来招呼点菜的小二,也是唤着他衙内,而不是通常的客官、官人。

在李七家酒楼临街的二楼包厢中,赵颢已经独坐了有一个时辰。可放在桌前的菜肴却都没有动,连银质的筷子也是摆在他进来时的原位上。盛在一盘盘银碟中的冷盘热菜,都是李七家酒楼的大厨精心打造,论口味其实并不输于宫中的御厨,但赵颢连看也不看一眼,只是一杯一杯的喝着李七家酒楼自产的青液酒,望着窗外的大街上发呆。

一壶酒已经喝去了大半,赵颢想灌醉自己,却始终不能如愿。事情就是这么怪,不想喝醉的时候,两杯酒就会倒,想一醉解忧的时候,却是越喝越清醒。

从窗外大街上穿过的一队车马,正向西去。这一队行旅,只有四匹马、两辆车,是个很小的队伍。领头的是个身着青色官袍的官员,骑在马上,背挺得笔直。从他骑马的姿势上看,大概是个离京就任的武臣——文官精于马术的并不算多。虽然不认识,但透过挂在窗户上的竹帘望下去的赵颢,看着那幅背影就有几分生厌。

从楼下的大街上收回目光,赵颢又给自己满上一杯酒。低头望着酒杯中的倒影,他讽刺的笑着。一个亲王,看似位高。但他讨厌谁,却没多少人会在乎。相反地,他看上什么,却始终无法如愿。

一桩青楼中天天能见的争风吃醋的小事,如今却闹得城中沸沸扬扬。就算大哥说要帮着把事情压下去,但这名声上的事哪有这么容易挽回的?教坊司已经不能去了,连个放松的地方都要找个没人认识的去处。这古往今来,有这般憋屈的亲王吗?

“韩冈!”

赵颢念着这个让他成了笑柄的名字,眼神也变得凶戾起来。他早知道周南心中有人,那根本不是秘密,也知道那是个选人,仅仅是不清楚具体的身份。若是士林中有名望的士子倒也罢了,小小的选人赵颢怎么可能放在心上,谁能想到那是个能在天子面前留下名字的选人!才智、胆略都是世间少有。

尽管对韩冈恨之入骨,但周南倾心了他,赵颢也不能自欺欺人的说她选错人了。因为这一桩风流韵事,韩冈的名声,已经在东京城中传开。虽然不是进士这一点让人诟病,不过文官的身份,加之不畏权贵的作为,也能让士林认同了。多少士人都赞着周南是风月班中的魁首,能慧眼识英雄,而他赵颢,就是其中出乖卖丑的反角。

要是当初直接强纳了周南也就没这么多事了……赵颢突然摇头苦笑。那可是个节烈女子,要能被人强纳入房,他堂堂亲王之尊,何须要做着水磨工夫?

赵颢又给自己倒了一杯酒,一口灌下后,把郁闷合着酒气一起吐了出来。

眼神重新变得坚定,赵颢冷笑着,就算丑角又如何?

韩冈薄有功劳是事实,可大宋文武官员数以万计,才能卓异的不可胜数,其中最终能出头的却是寥寥无几。平步青云不仅是要靠才能,还要靠机遇。韩冈连进士都不是,纵然如今得人看重,但将来的路却是会越走越窄。自己可是皇亲,离着九五之位只有一步之遥的皇弟,赵颢不信他日后没有机会!

敲门声突然响起,打破了赵颢自斟自饮的清静。雍王殿下把酒杯重重往桌上一顿,很不痛快的对门外喝着:“不是说过要一个人静一静吗?!”

但敲门声依然在持续,“二大王,是宫里面的消息。”

“是娘娘还是大哥?”赵颢心里尚憋着口气,还没喝痛快,但那两位派来的人却不好怠慢。按奈下不耐烦的心情,道:“让他进来!”

进来的内侍却并不是在保慈宫或是福宁殿中做事的阉官,而是赵颢留在宫中的另一名亲信。他神色有几分慌乱,进来后,就凑到了赵颢的耳边,叽叽咕咕就说了好一通。

赵颢本有几分不耐,但听了内侍赶来急报的消息,他脸色就渐渐铁青起来,怒意在眉峰中汇集,咬紧的牙关嘎嘎作响。

内侍把紧要的消息说完,见着他这副模样,不由自主的退后了半步,小心翼翼试探的问着:“大王……没事吧?”

“事?还能有事吗?!……哈哈哈!”

突的,赵颢爆发起来一阵大笑,笑声中全是疯狂,在李七家酒楼中传递。最后他笑得肚子都痛了,上气不接下气,声音都嘶哑起来,但伏在桌上还是在笑着。

赶来报信的宦官手足无措,上前相劝,却听着二大王断断续续、渐渐低下去的笑声中,却是喃喃自语:“原来……原来如此,原来如此!好个……唐太宗,好一个唐太宗!”

……………………

韩冈莫名的一阵心悸,突然在马上回头。

李小六就跟在他身后,上来问着,“官人,怎么了?”

“没什么!”韩冈狐疑的摇了摇头,收回望着州西大街两侧楼宇的视线,把头转了回来,继续领着小小的队伍向西门进发。

四匹马、两辆车,这就是韩冈去延州上任的队伍。

骑在马上的有三人,韩冈、李小六还有章惇送来的钱明亮,剩下一匹作为备用。两辆车中,周南和墨文乘了一辆,剩下的一辆则是钱明亮的浑家钱阿陈,看守着堆在车厢里的行囊。

韩冈今次是孤身上路,无人远送。东京城中的几个相熟的朋友,章惇现在当是在宫城中,王旁则有着婚礼前的准备工作,路明走得早了,前天跟着王韶一起上路,不然有他扯些闲话,路上的时间也好打发。

不过韩冈倒是不在乎,转头看着身边马车青蓬顶的车厢,有绝色佳丽作伴,这一路行程也寂寞不起来。

熙熙攘攘的商业大街到了尽头,眼前突然开阔,通往南门的御街宽达两百步,犹如广场一般。韩冈正欲横穿御街,就从南薰门方向,过来一队车马,正好快速通过前方。

韩冈一把扯住缰绳,停住坐骑,也阻止了身后的队伍,让那一队车马先过去,不与他们争路。

那一队车马,领头的一人也是穿着青色官服。年纪并不大,二十五六的样子。相貌让韩冈有些眼熟,长得颇为英俊,就是太过消瘦,看起来身体不是很好的样子。那名年轻官人在马上向韩冈遥遥的拱手示意,谢了他的谦让。

韩冈回手洒然一礼,也不多话,就驭马领队而去。

年轻官员的目光追着远去的一行人。擦身而过的韩冈,神光内蕴,看似斯文,却隐含着一股英武迫人的锐气,让他过目难忘。他由衷的感叹道:“不愧是东京,如此人物在南方可是少有得见。”

年轻官员身边跟着一名年纪相当的儒生,他却笑道:“若论人物风采,天下间同辈之人中,能比得上元泽你的可没几个。”

元泽笑了笑:“天下英杰无数,岂止我一人?能在其中有一立足之地,便已是喜出望外了。”

他虽然说着谦抑,但微微扬起的嘴角,却把隐含在胸的傲气丝毫没有遮掩的展露出来。

“元泽可是自谦过甚了……”

元泽摇了摇头,对这个随口而来的奉承并是不很放在心上。马鞭虚虚一挥,再不多话,也领队沿御街向北而去。

擦身而过的官员和车队,并没有给韩冈留下什么印象。只是觉得在哪里见过,不过一时想不起来。想了一阵后,便放弃了。

车队自城西的新郑门离开东京城,驶上了西去的官道。一只素白如玉的纤手掀开了车厢窗户上的帘子,清丽无双的俏脸露了出来,向着身后的城门望去。眼波流光,神情中是数分让人迷醉的落寞。

“舍不得吗?”韩冈在马上弯下腰,问着周南。

周南回过神,仰头对着韩冈,眼中深情如海:“有官人在,即便天涯海角,周南亦是心甘情愿。”

美人恩重,韩冈心中感动。回首东京,望着城墙崔嵬。此次入京,能载美而归,已是不虚此行。至于延州的风风雨雨,他现在也全不放在心上。在亲王面前虎口夺食,韩冈已不惧任何风浪。

任你龙潭虎穴,我也能如履平地!

