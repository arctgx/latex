\section{第31章 停云静听曲中意(九)}

战鼓随着种谔举起的右臂响了起来,将开战的命令传遍全军。

呛啷一声,叶孛麻拔出了腰间的长刀。这是他降顺大宋之后,上京拜见赵顼时得到赏赐,连同身上那一套金光闪闪的盔甲和马鞍后的角弓,都是属于御赐之物。虽然金甲沉重异常,并不适合上阵,但叶孛麻和仁多零丁全都穿戴上了身。

握着腰刀,叶孛麻和仁多零丁向种谔行过礼,便奔向左右两军,那里有他们的儿郎。他们将指挥族中的儿郎们冲向辽人的战阵。

战鼓声裂帛穿云,战斗已经开始了。

最前方的游骑们跟辽人派出的游骑对上了。几十名骑兵游荡在对峙的两军之间。战马交错时,互相交换着刀锋、铁锏和箭矢。

无论宋军、辽军还是党项人的骑兵,都穿着来自军器监的半身胸甲,若不是盔甲外的外袍式样和颜色不同,根本就分辨不出各自属于哪一方。

种建中拿着千里镜,望着战场中央的交锋。千里镜是韩冈所赠,并不是枢密院配发下来,透过镜片,种建中的神色越来越难看。

“只是头下军啊。”放下了千里镜,种建中便是一声叹。

明显的,辽国骑兵的战斗力要在宋军和党项军之上,一开始双方的数量相当,但除了第一回合交手,辽人被宋军随身携带已经上好弦的重弩射下了几个,之后几次对冲,落马的辽军都远少于大宋和党项联军。

“纵然是头下军,现在拿出来的骑手,也不会输给宫分军、皮室军中顶尖的人选。”种谔的神色毫不动摇,个体的战力说明不了什么,整体的实力才是胜负的决定因素。

种谔没打算因人成事。党项人比起对面的辽军实力还输了一成,但加上他手下的精锐就不一样了。种谔也不担心赢了之后,叶孛麻和仁多零丁敢翻脸。不论他们多想自立,现在已经往死里得罪了耶律乙辛,再开罪了大宋,接下来那就是第二次伐夏之役。宋辽两国的大军将会争先恐后的往兴灵这边赶过来。只有死路一条。

对面的战鼓声,穿过并不算宽阔的战场,传入种谔的耳中。

“辽人动了。”

远方的地平线上,那一条由千军万马组成的暗线,就像突然腾起的海浪,变得波涛起伏起来。涛声从地面上传来,数以千万计的骑兵开始随着鼓号声奔涌向前。

对阵的双方皆以骑兵为主。对此有着绝对自信的辽人,才会选择了决战,而不是通常使用的拖延、骚扰最后噬喉一击的战术。

辽人也是想着速战速决。在大宋步卒没有追上来之前,必须先一步击败种谔和他手下的骑兵,否则宋人步骑配合起来,兴灵地区的各家部族没有任何胜利的机会。

种谔看向了侄儿,种建中低了一下头,行过礼,将头盔整理好,拨马返回他的位置上。两个指挥的精锐骑兵就在他身后,静待着最后的号令。

仁多零丁带着一队亲兵赶回了左军阵列。

三万辽军并不是兴灵之地辽人能动员的所有兵力,应该再多个三五千才对。而党项军也可以再挤出五六千骑兵,只是为了防备辽军必然准备下来的偏师,不得不将他们分排在战场外围的几个据点上。

就在年节的时候,也就是前几天,种谔领军赶到了青铜峡口,遇上了叶家和仁多家为首的党项军。这大大出乎了叶孛麻和仁多零丁的意料。而更出乎意料的,是种谔他亲身入帐,硬是说服了仁多零丁和叶孛麻听从号令,双方合兵攻击辽人。

合则两利,现在的兴灵之地之地是为辽人占据,种谔和仁多零丁、叶孛麻有着共同的目标。

对仁多零丁来说,辽人回来得太快,又占着城池,兵力上双方虽相差不远,但仁多零丁自知没办法与辽人拖延下去。

纵然事后会起纷争,可种谔手边才两三千骑兵,又能怎么样?而且这一战若是在种谔的指挥下获胜,在场面上也能说得过去,至少大宋那边还能有个退路,即便只是说是半条。

回到本军之中,仁多保忠来到了仁多零丁的面前。

仁多零丁看着结束整齐的侄儿,关切的嘱咐道:“小心一点,不要让家里的儿郎伤亡太大。”

“知道了。”

“不过一定先要赢。输了就完了。”

“侄儿明白!”仁多保忠的回答更加坚定。

来自中军的战鼓声的节奏加快了,在中军之后,仁多家对面的辽军也开始了冲锋。

“种大帅在催了。”仁多零丁带着冷笑看了中军处高高矗立的帅旗一眼,回头将自己的腰刀交给了侄儿,“去吧,不要耽搁了!”

仁多保忠接过腰刀,高高举起,族中儿郎的应和如山间的呼啸。然后他提缰转身,领头向着敌军迎了上去。

千军万马冲向了战场的中央。

要开始了。

这一场迟来的决战。

……………………

宋辽两国之间紧张局面,从西北传到了京中,又从京中传到了河北。

就是在年节前后,北疆一线的守备也是一点不能放松。

不过广信军这边却是大开校场,在知军李信的检阅下,演武试射,军民同欢,过年的气氛一点也没有因为紧张的局势而冲淡半点。

广信军位于保州的东北角——保州的西侧便是定州——其实就是从保州分割出来的一个军事据点,只有遂城这唯一一座要塞。铜梁门、铁遂城,是当年的名将杨延昭杨六郎驻扎的地方。

广信军的北界,从保州吴泊至安肃军长城口,总共五十里宽,按《武经总要》的说法:‘今北边要害,塘水之外,自保州边吴泊西距长城口,广袤五十里,可以长驱深入,乃中国与匈奴必争之地’。乃是河北千余里塘泊防线中最大也是最为危险的一个缺口。

也正是因为这个缺口存在,广信军才会被分割出来。成为一个独立存在的军事区划。

李信以钤辖任职广信军,算是高职低配。广信军知军应该是兼任都巡检一职,在都巡检上有都监,都监之上才是钤辖。这主要还是李信资历浅薄的缘故。他的寄禄官是正七品的诸司使,而且还有一个遥郡刺史的加衔,头顶上比他官位更高的领军武将也就是三五十人。若是在大战之时,担任更高的职位也不为过,可惜河北几十年的太平年景,一个个论资排辈,好一点的职司多少人在等着,外来的将领根本插不进来。

李信纵然在南疆功绩显赫,可就任在河北,也只能先降两阶任官。不过这两年他表现得很突出,顺利的融入了河北禁军之中,前段时间还因练兵得法,而被特旨减了两年磨勘。

之前广信军守军已经校阅完毕,李信也颁下了赏赐,三千多将士在点将台下按着各自的指挥分散到校场周围。

只有李信着力培养的选锋军还守在点将台下,静静的扶枪而立。这是从麾下六千将卒中挑选出来的四百人。全都是善投善奔、勇猛敢战的健卒。尤其是他们都得了李信亲传的飞矛之术,勇悍冠于三军。

校场中,此时一根长索拉在两根木桩之间,从长索上垂下来几条丝线。而丝线又各系着一枚外圆内方的钱币,只是钱币的质地和重量各自不同,从半两左右的银钱到普通的铁钱都有。

这是悬银试射,不同的悬赏,试射的立脚点也不一样,越远自然越贵重。一名名士兵和围观的百姓轮番上阵,拿着战弓去射那丝线上的钱币。

谁能射中悬在丝线上的钱币,那么那枚钱币就属于那名弓手。射铁钱只需隔十步,银钱则就要在三十步开外了。

而李信又在悬银试射之外,又教练起了标枪。谁能用标枪投中十几丈外地上的银盘,哪个就能将价值更高的银盘揣回家中。

从种世衡流传下来的练兵之法,让清涧城的士兵以善射闻名关西。也让李信麾下的河北士兵,在两年之内便重新恢复了旧日的声威。不仅是他麾下的士兵,李信上任后,推行保甲法不遗余力,他治下的子民,也各个擅长弓箭飞矛。

一声暴起的欢呼响遏行云,一名身高七尺的汉子正在人群中得意的举起手中的长弓。看起来身高体壮,但脸庞十分的年轻,不过二十上下的样子。

“好箭法。”跟随着李信高踞台上的几名将领也拍着手叫好。那个高个子的年轻人方才连珠五箭,射下了五枚银钱,而之前他更是拿着标枪扎中了五只银盘中的三只。一股脑了卷走了十几两银子,算起钱来,也有三四十贯了。普通的禁军军卒,一年也拿不到这么多。

“今天的魁首当是小乙了。”李信侧头对着一名正捻须微笑的老将赞道,“令郎果然不凡。”

一名军校这时匆匆上了高台,附在李信耳边:“钤辖,北面有人来报,析津府那边的辽军南下易州了。”

高台上的气氛顿时紧张起来,李信则是神色不动:“多少人?!”

“三千到五千。可能是真要大动干戈了。”

李信安坐如素,“你去跟张先生说,让他起草给郭帅的急报。再传话给宋贤,让他继续盯着北面。”

回过头来,他平平和和的对一众部将说道:“不要紧,我们继续。”

