\section{第42章 壮心全向笔端含(中)}

“造势的工具?”李诫疑惑的问道。

这下轮到方兴惊讶了,怎么李诫什么都不知道。他父亲李南公沉浮宦海几十年,眼下的局面,就是眼睛瞎了,用鼻子嗅都能嗅出个眉目来,竟然没有给李诫写信说明?

他曾听说李家父子因为李诫专心于杂学所以关系紧张,但看到李南公巴巴的将李诫荐到韩冈面前,一点都不避忌外人议论他是,就知道舔犊之心任谁都免不了。而且眼下李诫在韩冈门下正得用,李南公在情在理都该提点他的儿子几句。可那位转运副使偏偏却没有这么做,实在是让人觉得费解。

不过惊讶归惊讶,方兴也没心思多猜测,“仔细想想就知道了。为了让朝堂同意重启襄汉漕运,龙图的确是拿出了多级船闸,但现在开凿水道、修筑堰坝的全都是唐州的沈知州在忙,龙图修个暂时作为替代的方城轨道就不管其他事了。若是这条路当真在运河畅通之后被弃之不用,这半年来的一番辛苦又是为了什么?投入的那么多钱粮难道都要打水漂不成?”

李诫沉吟了一下,抬眼问道:“是陕西,还是河北?”

见到李诫这么快就反应过来,方兴暗赞了一声,笑道:“多半是在河北。陕西缘边的山太多了,派不上太大用场。而河北那里,就是一马平川,要不然朝廷也不会那么怕契丹骑兵。”

听着方兴的一番话,李诫自嘲的笑着,“原来当真是明修栈道,暗度陈仓啊。”

“不,不能这么说……”方兴摇摇头,“栈道的确是明着修的,但陈仓其实也是明的。”他伸长手臂,拍拍李诫的肩膀,“也就是明仲你还是一心扑在这条轨道上,也不抬头看看外面的变化。方城轨道建成之后,已经有不少人看出了龙图的打算。沈存中就不必说了。襄州的黄知州,汝州的方知州都是心中透亮。还有些个商家,或多或少都猜到了一些。”

李诫心中黯然,长叹了一口气,他的确是个睁眼瞎,天天在自己面前奉承示好的商人,一个两个的心明眼亮,就是自家还蒙着头造车修路建港口。说起来,自己真的还是只适合修桥铺路,官场上的事当真不是普通角色能掺合得了的。

“不要想太多。”方兴看得出李诫的心情变得糟糕起来,“我们既然在龙图门下奔走,听从龙图吩咐就行了,尽自己所能,自会有所收获。至于龙图有何谋算,不是我们该去多想的。”

李诫点点头,苦笑了一声后,不去想这件事了。正如方兴所说,韩冈与他们不一样,地位不同,要面对的对手也不相同,看问题的角度更不会相同。同样的一件事——就比如眼下的方城轨道,在韩冈眼中是一个模样,在他们的眼中又是另外一副模样。

对他们这等只想求一个官身或是指望着能由选人转官的人来说,包括这一条轨道在内的襄汉漕渠,就是实现他们目标的一切。可于韩冈而言,京西的林林总总,不过是向天子和朝堂做个展示,就像是商铺中摆在台面上的样品,很大一部分是给人看的,而不是用的。

只是话说回来,若摆出来的样品本身有问题,肯定是会影响商铺的生意,说重要也的确是很重要。认清事实是当务之急,但妄自菲薄就没必要了。

放下了心结,李诫便拉着方兴,到了他在工坊中日常落脚的小屋。

作为轨道和车辆的监管者,李诫这些日子以来,都是歇在工坊里。

尽管工坊中,在铁、木、营造、机械等方面的饶有长才的大小工匠多达百数,但他们的精力,都放在正经事上。自住的房屋,一例都是简单的木板屋,李诫的小屋也就是胜在不漏风和外面多一圈象征身份的栅栏而已。

李诫让在院中服侍他的老兵将房中打扫一下,又让跟着自己的两名伴当去置办酒菜。将方兴让着坐下,顺手就将房中的一个温酒熬药的红泥小火炉生了起来。

蓝汪汪的火苗在炉膛中跳动着,这是上好的炭火才有的颜色。方兴丝毫不顾形象的将手伸到炉边烤着,“天气变得还真够快的。半个月前还一下热得跟初夏差不多,没想到一转眼的功夫,又变得快要下雪的样子。”

李诫立刻接口道:“中夜清寒,小弟这里正好有一坛京城来的和旨,放在炉子上热过,恰可怯除湿寒……”

“和旨?!可是樊楼所产?”方兴一听就有了兴趣。樊楼为天下第一,樊楼的酒当然也是天下无双。如此贵重的美酒佳酿,方兴过去也没有多少机会亲口品尝。

“正是。难得入手一坛,本来是准备留在纲粮北运之后来庆祝的。不过今天有了兴致,正好共谋一醉。”

一坛子上好的美酒,加上很简单的两个小菜,方兴和李诫二人围着火炉坐着。烫酒用的铜壶架在炉子上,而几支小酒壶则放在大铜壶中。水很快就烧滚了,咕嘟咕嘟的响着,酒香也随着水汽从小酒壶中飘散出来。

李诫等壶中的水滚了一阵,便亲手从大壶中捞起一只银酒壶。给方兴和自己满上,碰过杯,喝了一口之后,李诫舒了一口酒气:“现在小弟总算是明白了,龙图一心想要的是遍及河北的轨道,用来抵抗契丹人。不过,既然是做样子给人看,方城轨道运输时的损耗就不能太大。现在是轨道初运行,多少对眼睛想在鸡蛋里找出骨头来。”

“觉得现在损耗大了?”方兴十分珍惜的小口抿着酒,顺口问道。

“主要是替换的配件耗费太大,”李诫夹了块羊肉放进嘴里嚼着,“轮子轮轴还有路轨都是耗钱的大户,这些天的损耗,若是给人仔细一算,还是蛮吓人的。”

“这事简单。”方兴哈哈笑道,惹得带了几分忧色的李诫惊讶得瞪大眼睛,“只要能把帐目做平,怎么列项都可以随意。”

“随意列项?”李诫疑惑着。他的父亲是转运副使李南公,在财计之事上很有些名气,但他这位衙内只沉迷在机械、营造之类工匠之术上,半点也没有从他父亲那里传习到糊弄上司的手段,“到底是怎么个做法?”

“简单的说,就是将惹人注意的维护成本,打散了分到其他地方。这样即不会耽搁正事,也不会让人有机会攻击龙图和方城轨道。”

李诫想了一下也算是明白了,“说着简单,但做起来似乎挺麻烦的。”

“自会有人去做,你我不必操心,龙图手底下也不缺人。”方兴看得很开,他也不是喜欢攥权的人,“其实还有件事要注意,甚至还要通知让叶县和方城县两家来处置。”

正准备给自己和方兴倒酒的手停了,李诫抬眼问道:“什么事?”

“前面明仲你也说了,接下来,车辆和路轨损耗得会越来越快。自是会有越来越多的纲粮因为大大小小的事故遗落在外。现在都是派人运回来,但接下来处理此事的人手可能会捉襟见肘,来不及运回。一时运不走的纲粮,可是要防着有人哄抢。”

“眼下应该没问题吧,毕竟才开通,就是想做贼,也还没有准备好。”李诫沉吟着,“就怕日后有贼人故意破坏轨道,然后从中取利。”

“的确,愚兄也是这么想的。”方兴点头,“这样的贼人是最可恶。只能用重典来处置,京西这里是重法区,砍头总是不难。”

“那也要先捉到人,否则给贼人跑了,备了侩子手也没用。”

方兴洒然一笑:“其实这些都是小事,只要日常注意,也不会捅出什么大篓子来。有龙图在上压住阵脚,就是宰辅来了,想动摇到京西的局面,也不是那么容易的一件事。”

李诫也振奋了起来,“惊喜事了,河北那边很快就能铺设轨道了,说不定我们都要过去。到时候铺开的摊子肯定比现在要大得多。”

李诫随口一句,却没听到方兴的回答,奇怪的抬眼看过去,却发现方兴正皱着眉头。“不过愚兄总觉得龙图还有另一番谋划,绝不是表面上看到的这一些事。太多人看得出来,反而让人觉得事情有哪里不对劲。”

李诫听着也狐疑了起来,韩冈的计划当真有这么简单?除了他这个不问世事,只顾着督促修路的呆子,连商人都看破了在天南地北都赫赫有名的韩玉昆的心思,怎想都觉得有些不可思议,坟墓里面的吐蕃人、交趾人,应该有许多都睡不安稳:“那究竟是怎么一回事?”

“要是愚兄能想得通透,说不定也能做到龙图学士了。”方兴早就放弃了去猜度韩冈私底下的谋划,“想多也没用,龙图行事一向不是那么容易被人看透的。葫芦里面卖的到底是什么药,得到最后龙图主动拔开塞子才会知道。我们要做的就是做事……然后,喝酒!”

