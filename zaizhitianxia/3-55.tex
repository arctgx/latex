\section{第21章 论学琼林上(上)}

从宫中出来,便已是酉时。而等韩冈回到王韶府上,二更的更鼓都在大街小巷中给敲响。跟着王韶、王厚说了几句今天觐见天子的事,韩冈便自去睡了。

虽然他一向精力充沛,但在朝堂上,与天子对话时一边要斟词酌句,以防错说出一些不该说的话;但另一方面,也必须保证稳定的语速,及时回答天子的征询。要完成这两项要求,自是很伤.精神。韩冈睡到床上的时候,希望日后能早日习惯这样的对话。

而到了第二天,王雱遣人送贴来请韩冈赴宴。午后,韩冈应邀前往清风楼,结束了崇政殿说书的工作的王雱此时正在楼上等着。

“这是怎么回事?怎么这般清净?”

韩冈上来时就有些觉得不对劲,坐下来后才发现,原本喜欢聚集在清风楼上的不第士子们,今天都不见了踪影。

王雱笑了一下:“还不是玉昆你昨天的功劳。”

“都知道了?”韩冈问道,“听叶致远说的?”

“外面早就传遍了。说是昨日在清风楼上,你被驳得差点要辞了进士出身,最后靠了天子遣使方才解围。”

颠倒黑白的一番话传到耳中,韩冈眨了眨眼睛,笑了起来:“是吗?他们是这么说的……”却没有半分动怒的样子。

“玉昆你好像一点也不生气。”王雱在叶涛那里得知了真相,所以对韩冈的反应很是惊讶。

“何必生气!”韩冈摇了摇头,对那等人生气纯属是浪费时间,“难怪今天清风楼上他们都不见人影。”

王雱一声冷笑:“他们哪敢当面与玉昆你对质!”

“当然是不敢的!”

韩冈也同样冷笑着摇头。现在这群儒生,有几人还有孟子虽千万人吾往矣的胆魄?!别说千万人,就是面对他韩冈一个,也根本不会有几人愿意第一个跳出来。都是太过于聪明,只会在背后嚼舌根。临到关头,就会让别人上,而自己在后面等着捡便宜。

王雱和韩冈都有些愤世嫉俗,但也是看透了人心。战乱时代,好勇斗狠那是常事,为了一个目标,多少人前赴后继,那也是不鲜见的。但如今的太平年景持续百年,人心早就软弱了,也只剩陕西等一些战乱不断的边地,民风依然骁勇。

“不提此等事,反正他们什么都做不来。”韩冈问着王雱,“怎么不见仲元?这两次都没有看到他。”

听到韩冈提起弟弟,王雱的脸色顿时被一抹阴云笼罩。虽然很快就恢复正常,但也没有瞒过韩冈的眼神。

看了一眼韩冈,王雱叹了口气,“……此事也不瞒玉昆你……”家中不睦的事,时间长了终究还是瞒不过韩冈这个妹夫,还不如摊开来说,“这段时间,二哥夫妇两人越发的不睦,日夜吵闹,闹得家宅不宁。现在也没心思出来了。”

“天天吵闹……究竟是为何?总的有个缘由吧。”韩冈不是八卦,王旁好歹是亲戚,更是朋友,问上一句是应该的。

“……这是我那侄儿出生后的事,二哥觉得侄儿长得不像自己,所以起了疑心,这样才闹起来的。”

韩冈看了王雱的脸色,就知道其中的情况必然比他说的更为复杂一点。王雱和王旁两兄弟之间的关系,变得如此紧张,不会是因为王旁觉得儿子不像自己,就会闹到这般田地。王家的两兄弟长相皆遗传了父母,王旁才一岁儿子,就算跟王雱相像,也是不该让他起疑心的。

先前问起来的时候,韩冈没想到会是这等事,让他原本想劝一劝的心思,一起都淡了。女婿是外人,岳家的家务事能听不能说,尤其是这等事关名节的闺房事上,更是不好插嘴。

王雱也不想提着方面的话题,喝了两口酒,便问着韩冈:“玉昆今日觐见有半日之久,不知廷对之中说了些什么?”

天子与臣子的私人谈话,按道理说,是不能对外传播的。若是被确认,追究起来就是个罪名,也就是所谓‘臣不密,失其身’。但自家人,就没什么好掩饰的。何况韩冈与天子的对话,在宫廷那个四面透风的大漏勺里,根本也是隐藏不住。

韩冈很干脆的将与天子的对话,主要是关于新法哪方面的,一五一十的转述给王雱。韩冈的一席话,王雱边听边点头,自己的妹夫是在不着痕迹为新法说话呢。虽然不是直接赞美,但弯弯绕的说话,反而会更有效果。

要是韩冈一面倒的说着新法的好话,等于是自毁前程。没有任何他处任官的经验,便说着天下州县皆是乐于新法,天子要会相信才会有鬼。韩冈也只有以这等表面上的持平之论,再用事实为佐证,才会让皇帝信之不疑。

王雱对韩冈对新法的表态,一百分的满意,窃喜自己的父亲没有挑错人。这等人才站到新法一边,日后必然可以派得上大用。只是他的欣喜只保持了片刻。当听到韩冈向天子推荐了张载进经义局,顿时就变了颜色:“玉昆,你怎么如此做?!”

王雱怒气腾起,而韩冈冷然自若:“小弟也只是荐了家师一人而已。既然朝廷设立经义局,要重新注疏经典,以家师的才学、声望,难道不够资格侧身其间?”

“玉昆,你不会不知道经义局是为何而立吧?!”王雱的眼神变得阴沉沉的,他和吕惠卿可是已经确定要进经义局了,哪还会希望有人来跟他打擂台。

“小弟自然知道。”韩冈目光平静如水,毫不退让的与王雱对视着,“但闭门造车是不成的。石渠阁论经,白虎观议礼,孔祭酒撰五经,这都是聚天下贤才之议论,方才得到最后的成果。小弟所学种种皆源自横渠门下,当然不能见其被摒弃于朝堂之外。”

有些事可以妥协、可以退让,但有些事是不能退让、不能妥协的。请张载入经义局,是韩冈乘机向天子提出,尽管他心知成功率并不会太高,但毕竟尚有可能,而不去努力争取一下,可就半分机会都没有了。

不要以为儒家就是温良恭俭让,要真是这般面目,各有一套传承的诸子百家,也不会最后由儒门一统天下。别说百家之间的争斗是刀光剑影,就是儒门内部,也从来都不是和气一团。

正如韩冈提到的孔颖达,他少年成名,在洛阳儒门之会上,舌辩众儒,一举夺魁。但被他压制的宿儒耻居其下,甚至派遣刺客要杀他。若非杨玄感将之保护起来,可就没有流传后世的《五经正义》了。

更别提马融、郑玄这对师徒,同为汉家大儒的两人,他们之间的关系可谓是错综复杂。传言中,甚至有马融在郑玄出师后,怕他日后声名压过自己,欲遣庄客将之追杀的说法。

争名夺利,互不相让,大儒都是难免。而一个学派对另一个学派,更是有着天然的排斥。

王安石作为推行新政的宰相,需要一个稳定的后备人才来源,而不是让国子监尽出一些唱反调的对头。所以有了经义局,重新诠释儒门经典,作为国子监钦定教材,同时成为科举考试的标准答案。

韩冈对此可以理解,但这不代表他能认同。没有海纳百川的气魄,而用行政手段排除异己,作为被排除之列的韩冈当然看不顺眼。

他并不是要跟王家决裂,迟早要闹出来的事情,早一步揭开来,日后才不会产生过大的伤害。同时也要让王安石父子知道,他还是过去的宁折不弯的韩玉昆。

当初在王安石、韩绛两名宰相的重压之下,依然咬定横山难取,最后甚至放弃了已经到手的煌煌之功。如今他也不会因为成立王家的女婿而放弃气学,更不会放弃将后世的科学理论装进儒家这个箩筐里的想法。

在清风楼上不欢而散。第二天,便是朝谢之日。依照故事,状元余中领着四百零八名进士去宫中阁门外,向天子的恩赐而拜谢。

在唐代,进士被取中后,要去中书谒见宰相,一并向主考官谢恩,确立座主和门生的关系。而到了宋代,太祖赵匡胤不喜臣子将朝廷的选拔揽为己功。在设立殿试后,进士们就成了天子门生。要拜谢,当然要向天子拜谢。而且照着旧年的惯例,还要进谢恩银百两,都是由进士们各自出钱凑起来,不过今科被赵顼下诏给免去了。

殿试唱名以来,这还是第一次众进士齐集。韩冈作为四百人中唯一的朝官,前日又被天子单独召对,当然是人人为之侧目,但终究还是没有人敢于第一个跳出来与韩冈过不去。

朝谢之后,进士们各自星散。数日又是一晃而过,这几天中,王家兄弟都没有再来找韩冈,而韩冈却也没有去王家登门拜访,王安石究竟会不会同意让张载进京,而天子的意向又是如何,这都是韩冈想要知道的,不过此事也急不来。真正临到眼前的,还是让新科进士们跨马游街,一齐赶赴东京西城外,三年才有一次的琼林宴。

