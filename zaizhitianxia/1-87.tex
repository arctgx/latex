\section{第36章 不意吴越竟同舟(下)}

【第三更,红票,收藏】

又是喝足了一晚,第二天刘仲武早早的起来,临行前没有丝毫犹豫,跨出门跳上马就走,依然并不打算等着和韩冈一起上路。

在刘仲武的心目中,跟着韩冈一起走,就像脖子上缠着过山风,身子前后群狼随行,屁股下面再骑着头大虫,衣服里还尽是跳蚤那般度日如年。

可这一天夜幕将临时,在郿县【今眉县】的驿馆中,刘仲武怕遇上韩冈,就躲在房中啃着炊饼。但他所要躲避的韩冈,却大模大样的踹门进来,身后李小六领着两名驿站中的军汉,送上了一席酒菜。

韩冈捧着个酒坛,堵在门口放声大笑:“子文兄,今天又是不辞而别,当是要罚酒啊!这坛可是邠州的静照堂,秦凤难得一见的佳酿。有好酒好菜,我们今日不醉不归!”

刘仲武哭丧着脸,又被韩冈逼着痛饮起。刘仲武感觉自己像是掉入的蛛网的飞蠓,怎么挣扎也逃不过韩冈的手掌心。要是他逼着自己明天同行,该怎么办才好?已经躲了两天,还能再躲第三天吗?

酒过三巡,刘仲武喝得忐忑不安,而韩冈又说起话来:“明日韩某要先去横渠镇访友,早早便要启程,便不能与子文兄同行了。”

虽然张载已经入朝任职,张宅中最多也只有几个老家人看守门户。但韩冈上门问候,代表着身为横渠门下的一片心意,传到张载耳中,他能不高兴?给外人听了,也会说韩冈尊师重道。说起来也算是提前借个善缘了。

韩冈笑了笑,歉然又道:“还望子文兄不要见怪。”

刘仲武眼睛都亮了起来,哪里可能会见怪,连连摇头摆手。能甩脱韩冈,他根本是求之不得。自从在七里坪驿站相遇之后,他两天来一直都想把韩冈甩掉,可始终不能如愿。

他所用的这匹赤骝,虽然远比寻常驿马要神骏,全速奔驰起来是普通驿马的两倍还多,但韩冈用的驿马能一日一换,可以不惜马力一直骑在上面。可他刘仲武却通常是骑着跑上半个时辰,便要下来走上半个时辰——如果是连续骑乘,这匹河西良驹要不了两天功夫就会倒毙在路边。

尽管横渠镇本就位于前路上,要去明天的目的地——咸阳——还是得经过横渠,最终都是要跟韩冈碰上面,但只要想到明天终于可以不用四更天就启程,刘仲武已经别无所求。

“官人请自便。”刘仲武眉眼中有着遮掩不住的放松和笑意。

而韩冈的脸上,也是一样的笑容。

韩冈明说要去探访老师,不与刘仲武同行。几天来,刘仲武第一次觉得他可以睡个安心觉,不必再披星戴月的提前上路。第二天一大早,韩冈便起身自往横渠镇去了,而一个时辰之后,刘仲武才打着哈欠,洋洋起身。

迎着冬日的阳光伸个懒腰,刘仲武要来水为爱马清洗了一番,最后气定神闲的跨马上路。没有韩冈在身边,刘仲武终于还是恢复到那位让向宝也得另眼相看的年轻人,行事有条不紊,举止稳重可靠。

……………………

横渠古镇,位于渭水岸边,又离蜀中出关西的斜谷道的出口不远,论地理位置,是关西有名的通衢要地,而商旅往来,更是络绎不绝。若是春夏时节,河水丰盈,无数船只泛舟于渭水之上,从横渠镇边通过。因为就在离横渠不远的斜谷镇,有着大宋最大的内河船场——凤翔斜谷船场,每年利用秦岭的木材,额定打造六百艘纲船,这是大宋所有船场中数量最多的一个。

韩冈一早启程,辰时便抵达横渠镇上。镇内屋舍重重,韩冈左右看看,足有数百家之多,在西北当个县城都够资格。他是第一次来横渠镇,也搞不清张家宅邸位置,便向从身边经过的一名樵夫询问。

“是先生的弟子?”樵夫背上捆着的柴禾有比他的头还要高出三尺,粗手大脚,显是常年劳作,但说起话来却是带着一点书卷气,“先生已经入京了,官人来迟一步。先生家如今只有一对老夫妻在守着。”

“此事韩某已知。不过不论先生在与不在,既然经过横渠镇,总不能过门而不入!”

“说的也是。”韩冈尊师重道,让樵夫点头称道。他看见韩冈主仆的马上捆着大包小包,心知肯定是带着礼物来的。抬手指着韩冈过来的方向:“镇南口迷狐岭下大振谷的那一间独院便是先生的家,岭上就是张老郎中和老封君的坟茔。”

“多谢兄台指点。”

张载祖籍开封,当年其父张迪带着一家人入蜀为官,不幸殁于任上。张载之母带着他和他的弟弟张戬,扶灵回乡。但蜀地距东京路途遥远,他们从斜谷道出蜀入关中后,便用尽了张载之父多年为官的积蓄,却再没一文钱往京城老家去了,只能在横渠镇草草安葬,并定居下来。

张载少年时喜武厌文,当李元昊起兵反叛,他便上书当时的陕西安抚使范仲淹,自请招募关西豪客,去西北收复青唐蕃部。而范仲淹则说‘儒者自有名教可乐,何事于兵’,劝其弃武从文。自此,世间少了一个武将,而多了一名儒学宗师。范仲淹劝学的故事,在世间流传很广,直至千年之后,亦有流传,韩冈小时候也听过这个故事。

就在向阳的那面山坡,樵夫所称的迷狐岭上,便是张载之父的坟茔,做官穷到连回乡安葬的钱都没有,也算是个清官了,也难怪能教出张载这样的儿子。

在张宅之前,韩冈整了整衣冠,带着捧起礼物的李小六走上前,恭恭敬敬的敲响了院门。很快,老旧的院门吱呀一声开了,一位老妇颤巍巍的从门内走出来,打量了一下韩冈,问道:“敢问官人何人?”

韩冈走上前,和声道:“在下韩冈,是先生的弟子,今次入京途径横渠,特来探访。”

………………………

又是一日的奔驰,望着百步外地驿馆,刘仲武犹豫了一下。在路上奔波了一天,他不是不累,但一想到进了驿馆后,说不定还要跟韩冈打上照面,心中却更觉得疲惫。

在街中踌躇了一阵,刘仲武头一抬,盯上身侧的一座高约一丈的彩棚。彩棚之后的楼阁正门上,挂着升平楼字样的匾额。这是一座酒店。

店门前用竹竿和丝帛扎成的迎客彩棚是酒店的标志,秦州两座大酒店——惠丰楼、永平楼——前都设有彩棚。这个风俗还是这几年从京中兴起来的,刘仲武也曾听说东京城中的七十二家正店,家家门口都有彩棚装饰,座座都有三四层楼那么高。而咸阳城里的这座升平楼,门前彩棚只有一丈,只能算是凑数的作品。

刘仲武看升平楼用围墙括起了一座大院子,怕有数亩大小。这么大的一片地,不应是仅仅吃饭喝酒的地方,应该还能住宿。不过在这里住上一夜,他怀里本就不算沉重的钱袋可是要泻肚子了。

费钱就费钱罢,总比跟韩冈撞上要好,刘仲武无可奈何的叹了口气。往京城的这些日来,自来熟的韩冈让他头疼不已。伸手不打笑脸人,韩冈自始至终都没有失礼的地方,又不好真的翻脸,他只能每天都苦捱着。现在想想,还是自己总是住在驿馆里的缘故。

他算是豁了出去,也不想省什么钱了,虽然到了京城中,要打点的地方很多,本想着要省一省的,但跟韩冈走得近了更加不是事。刘仲武心底作了决定,等明天就转从长安道走,拖上一程的时间,与韩冈错上一天,就不必怕再与他照面了。

站在店门处,刘仲武向内一张望。店中客人倒不多,而且并没有个韩冈模样的坐在里面。松了一口气的同时,刘仲武又苦笑起来,现在他几乎都成了受了惊的老鼠,千方百计都要躲着韩冈那只猫走。

抬步跨进店中,一名店小二忙迎了上来,殷勤的问着:“客官,是打尖还是住店?”

“住店!”刘仲武沉声说着,“先弄些好酒好肉的上来,再给洒家弄间干净上房。哦,对了!门口的那匹赤骝是洒家的马,好料尽管上,草料钱自算给你。服侍得好,明天少不得赏赐!”

“客官哪里的话,就算不赏赐,难道小店还敢慢待不成?客官且放一百二十个心,若是饿瘦点皮毛,尽管用鞭子抽小的出气。”店小二的嘴皮子利落,话也说得漂亮,领着满意得点着头的刘仲武进了店中,高声的喊了一句:“住店的一位~~!上房一间~~!”

小二用着唱曲儿的调子,拖长声冲着里面交代了一句,又找了一个跑腿的小子出门牵了刘仲武的马,去店后的马槽安置,这才引着刘仲武上到比较清静的二楼中。

二楼上客人也不多,大小加起来十五六张桌子,只有三分之一坐了人。小二安排了刘仲武坐下,顺手拿着块抹布,将本已经很干净的桌子又擦了两遍,“不知客官想吃些什么。小店的招牌是排蒸荔枝腰子和两熟紫苏鱼,还有上好的锦堂春,再香醇不过,一杯便能醉人。”

“出门在外,也没个什么挑的。就把你们店里的招牌上两道来,再弄盘管饱的好肉,一并烫上两壶锦堂春。”刘仲武也放了开来,既然已经敞开了钱袋,也没必要再节省个什么,好酒好菜便都点上。

“好嘞!”小二应起声来仍带着曲调,向下传菜也仿佛在唱歌,“排蒸荔枝腰子、两熟紫苏鱼各一份,白切羊肉一盘,玉堂春两壶嘞……”回头又道,“客官请少待,小的先下去给客官端点果子上来!”

小二蹬蹬蹬的下楼去了,在楼上服侍的一个小童拎着个大铜壶,过来给刘仲武倒了一杯滚热的茶汤。

茶汤中滚起的热气熏在脸上,双手拢着杯子,温暖的感觉从掌心传遍全身。有热茶没韩冈的地方,让刘仲武坐下来后便不想再站起。他呻吟般的感慨着:“安逸啊……”

这时本是背着楼梯口,独坐在窗边一桌的客人缓缓转过头来,举起酒杯,在刘仲武突的变得又青又红的脸色中放声大笑:“子文兄……真是人生何处不相逢!”

