\section{第18章 青云为履难知足(八)}

“好了。”在外人尽数退下的大厅中,韩冈轻松的笑着,“这下子,李常杰有阵子睡不好觉了。”

“也不知道李常杰还能不能坐在他的位置上。”李信挺了挺腰,六天中他来回奔波,其实很有些累,“一旦溪峒蛮部攻过去,先动的肯定是交趾朝堂。”

“说的也是。”韩冈收起了笑容,“他能杀了先王留下来的顾命,但他不能将宗室都杀了,恐怕有得乱了。”

这几个月,都没有听说李常杰败退回国后,升龙府中有何异变和动荡。韩冈四处搜集情报,也没有听到李常杰被降罪的消息。反倒是李日尊留下的顾命大臣太师李道成病死了。李常杰手段的确是够黑的,但若是官军打过去,甚至只需有些风声传过去,他能不能压得下交趾王族,还真是个迷。

韩冈倒是希望李常杰他能坚持到底,当官军打过富良江的时候,他还在辅国太尉的任上。倒不是因为李常杰失势,报复后的畅快感会差上一些——苏子元也许会这么想,但韩冈不会——而是一旦交趾政变后,将李常杰和李乾德绑了送过来,说不定朝中就会有人撺掇着卷旗收兵。

那可就是个大麻烦了。

不过现在想着这些,与为古人担心差不多,李常杰能否在官军攻过去之前平安无事,那要看他自己的本事了。韩冈现在强压着溪峒蛮部去做的,则是不想让交趾人在征南大军展开攻势前国中平安无事。

大宋与交趾的国界,长达千里。其中山峦起伏,道路众多,大路小道,数以百千计。可如果换作官军过来,能选择的道路则仅仅是有两三条,其中自永平寨入交趾的大道更是必走的一条路。所以交趾人只要防备几条主要道路就够了。

但左右江三十六峒蛮部不同,他们熟悉地理,且本身所在的溪峒就分布在漫长的边境线上。而且为了尽可能多的劫掠到交趾国中的人口财物,不会集中于几条干线上,而是会随着时间的推移,逐步分散开来,形成一个宽阔的攻击面。一旦千里国境处处烽烟,交趾军将会防不胜防,凭着饱受重创的两三万正规军,哪边都别想堵住。而一旦征发起部族军来,一点点消耗的都是升龙府的战争潜力。

“表哥,你待会儿去营中,挑选五十名为人精细干练又能吃苦的士兵,这些家溪峒蛮部里面需要安插些人进去。”韩冈吩咐着李信,“只要差事办得好,功赏之事,我是不会吝啬。”

李信愣了一下,问道:“……是不是要监视这群溪峒蛮?”

“不是监视他们……”韩冈笑了一声,“最多也只是盯着他们,不要伤到了我大宋子民。他们更重要的任务是探察地理,了解当地风土人情。”韩冈又笑了一笑,声音冷了一点,“其实可以把人情去掉,了解风土就够了。贼过如洗,溪峒蛮部杀过去,多半会是鸡犬不留。”

“只要保住大宋子民,我管交趾人死活做什么。”李信哼了一声,转头盯着:“你家里有没有奴役大宋子民。”

“没有。”黄全连忙摇头,“下官家里一直都是靠着交趾和中国的往来市易,若是坏了名声,就没商客敢上门了。就有两个铁匠是汉人,不过都是家父花钱请来的,已经娶了族中的女儿。”

见李信满意的点头,黄全转过来小声的对韩冈道,“龙图,昨日家父来信,说广源州刘纪三人不稳,而且交贼正在打算进犯广源……”

“黄全,不用担心你父亲。黄团练在广源州只要不主动去攻击刘纪三人,他们也不会有胆子先动手。而且只要左州、忠州的事传过去,刘纪等人的只会更怕,没了三人的推波助澜,来自于交趾的压力应该能小一点了。”

黄金满前几天还遣使来报,守在广源州通往交趾境内的几个隘口的驻军增加了许多,而且有进犯广源州的迹象。

此事倒也不出奇。自广源州南向,能直插富良江上游,若是那一段被亲附大宋的势力控制了,宋军就可以轻易跨过富良江,直逼升龙府。交趾上下怎么都不会甘心广源州被黄金满控制在手中。

“但以现下的情况,交趾人是没办法分心的。左州、忠州的消息传过去还好说,一旦三十六峒蛮部全都杀过去,交趾人哪里还会有余力进犯广源?”

得到安抚的黄全退下去了,只剩表兄弟两人的厅中,响起了李信的声音:“三哥儿,朝廷到底打不打算攻打交趾,怎么征南行营的消息到现在都没有?”李信有些发急,“攻打交趾,就算再快也要留下三四个月的时间。广西、交趾的气候也只有冬天好上阵,也就是今年十月到明年二月。要赶上这个日程,征南军到了八月就必须启程了,这样才能在十月前抵达邕州……”

“表哥你放心。”韩冈笑道,“苏伯绪没几天就回来了。到时候不是我就是章子厚,就会被召回京中一趟。只要到了天子面前,征南行营的事就能敲定下来了。到了八月,也就可以连着征南军一起南下。”

“邕州这边先得将招募到两千新兵给练起来。许多事可以交给随行的蛮部土丁,军中的中坚则有南来的西军,可打先锋的差事,只有靠着邕州的兵。”

李信呵呵笑了两声:“正好领军平了忠州左州,可以狠狠的操练了,谅那群新兵也不敢有所怨言!”

从州衙中出来,韩冈终于可以往州学中去了。

新近落成的州学,比起旧时在城隍庙借地的时候要强出不少。有三重院落,楼阁六栋,教室、文庙,连同州学学生的宿舍,大小房间总共有四十间。能在短短时间内建成,一个是韩冈手上不缺人力,另外就是绝大部分的砖石梁柱,来自于城中拆下来的旧料。不过都是十中挑一的上好材料,上了漆、抹了石灰后,也看不出来是旧物。

州学落成,依惯例当有一篇纪念性文字,以便勒石于学中,流传于后世。范仲淹有《邻州建学记》和《饶州新建学记》、王安石有《虔州学记》,都是同样性质的文字——不仅是州学,古来但凡建筑落成,往往都会请名家写下一篇文章。滕子京请范仲淹写的《岳阳楼记》名气更大,欧阳修为韩琦写得《昼锦堂记》同样流传于世。

现在州学里面的学生,都是敬畏着自己是进士科的第九名,而且已经有几部书流传,更是天下名儒张载的得意门生。读书学习时,一个个都是。但自家的事,自家最清楚。实际上他的诗赋文章依然拿不出手,只能转托高人。

该找谁来写好呢?以韩冈的身份,脸皮厚一点,唐宋八大家还在世的几位都能去求来一篇文章。就算苏轼苏辙两兄弟,韩冈也有自信请他们动笔。。

不过其中最好的人选只有王安石一个。只要找自己的岳父来写,多半就是一篇千古流传的名篇。可是韩冈不知道王安石现在有没有心情写,王雱的身体情况已经很不妙了,除非有奇迹,否则也拖不了多久,这时候再劳动自家岳父,未免过分了点。虽然王安石本人不会在意,但韩冈在意。

这件事就再说好了,拖个一年半载也没什么。

进了州学中,士子们看见韩冈,恭恭敬敬的起身,向韩冈行礼,韩冈身上的龙图阁直学士就足够让他们仰望了,天下儒者,有几个能成为学士?只是没人跪拜,文庙之前,没那个官员能让士子弯下膝盖。

邕州文风在广西算得上浓郁,只比桂州稍逊,柳宗元当年贬任柳州时,曾来过邕州见他的族兄柳宽。城外的马退山上,还有一间茅亭。亭外的石碑上刻着他留下来的《马退山茅亭记》。

而苏缄之前在邕州的几年,花了很大力气,设立州学、开辟学田,让邕州州学中的学生,达到了五六十人,广西几个有名儒者,也被请来教授。苏缄本人也是进士,更是有空便来学中。

只是一场大战下来,多名学官死难。现在能站在州学中讲授经文的儒者,韩冈是稳居排第一。作为学官的一名老儒,水平还不到北方贡生的等级。韩冈有心去信京城,问一问有没有人愿意来邕州当学官,

来到讲坛上,韩冈拿起书本,开始向学生们传授经义。

州学之中,为了让学生参加贡举、入京考进士,教授用的教材都是三经新义。这一点,韩冈到现在也没能改变,就算张载在京中讲学已有一年,经义局照样排斥一切不属于新学的理论。

不过私下里的交流,韩冈倒是可以教授一些属于关学的理论。说起来,邕州州学的学生,大部分对韩冈教授的格物致知的道理更感兴趣。从功利的角度上讲,他们想考进士几乎没有机会,但换条路,能有所发明创见的,还是一样能当官。不过更重要的,探索存在于自然中的道理,也自有一种吸引人沉迷的魅力。

也许关学一脉,能在邕州扎根下来也说不定,韩冈想着。

