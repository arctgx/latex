\section{第45章 樊楼春色难留意(二)}

【第一更,求红票,收藏】

刘仲武也看到了五丈河,他晃晃悠悠的走到河边,推开李小六,松开裤带,自顾自的解起手来。一阵哗哗的水声后,他整理着衣服,走回来,反手指着下水道的洞口:“喂,路学究,那就是鬼樊楼吧?”

“没错,就是鬼樊楼。”路明伸着脖子看了一下,点头说着,“也叫无忧洞。多少贼子犯了事后在里面躲过。京师里这些沟渠四通八达,加起来有数百里长,钻进去便没人能找到,多少好人家的小娘子被拖进洞里祸害了!当年的包侍制知开封府的时候,对藏在里面的贼子也没辙。”

“还有这事啊?真的假的?”韩冈倒是给上了一课,来京师前,他从没想到,东京城的下水道设施能有这般完备,甚至可以称为罪犯的基地。

“当然千真万确!”路明以为韩冈不信,分辩道:“不说别的,哪个月京师里没有几户人家的女儿被劫走?有几次,那些贼子失了风,被人撞上,便一溜烟的窜进了沟里。还有传言说,他们就是用这些无忧洞来安置劫来的小娘子,等找到买家就卖出去,从此生不见人,死不见尸。”

“那些被劫的女子难道不会跑?即便在沟渠里跑不出来,等到被卖出去,那时总能跑吧?跑去告官,怎么回不了家?”

“高宅深院里一关,谁能逃得出来!”路明笑了一声,“尚记得仁宗朝有个生性好杀人的宗室,家里的仆婢犯点小错就给他杀了,埋进家宅的地下。多少人家的女儿送进去,就没再出来过,除了死了的,剩下就被关着。她们被一丈多高的围墙围着,消息传不出来。若不是一场暴雨冲塌了围墙,谁知道里面死了近百人?!”

“那最后怎么样了!?”韩冈半信半疑,追问着最后的结果。

路明瞥了韩冈一眼,拖长了声音:“仁宗嘛……”

“居然没杀他?!”韩冈难以置信。

“这算什么?!仁宗朝的宰相陈执中不也是亲手鞭死了一个小丫鬟,紧接着又逼死了两个,到最后,也不过是外放而已……”路明冷哼一声,“要不是当时朝堂上闹得正欢,这件事还扯不出来,陈相公说不得照样做他的相公。死几个下人,朝堂诸公真在乎过?!”

说话间,四人走上了桥头。京城内外,桥梁无数,形制也是五花八门,但其中数量最多的,还是韩冈他们脚下的这种被通称为虹桥的木质拱桥。虹桥既然以虹为名,桥面便是彩虹般的半圆形,这样符合力学原理的外形。使得桥身坚固异常,四五年前,英宗治平年间的一场大洪水,席卷了京师,冲进了宫城和上四军军营,却没有冲垮哪怕一座虹桥。

虹桥的桥面无一例外都很宽阔,基本上都是四丈上下,韩冈他们走上去时,就只占了一条边。不过在白天时,韩冈却是没发觉这一点。那时桥上两侧都给摊贩们占据,近四丈宽的桥面就只在中间留了一条道,供来往的车马行人穿行。

“喂!韩官人,路学究,”刘仲武拍着栏杆,指着桥下的下水道,大笑着:“你们看看,那无忧洞里一点水都没有,也是旱道啊。”

“走旱道好啊,水不湿脚。”

刘仲武在桥上说着胡话,路明也忘了刚才的愤世嫉俗,与他一搭一唱,全然没了形象。看着他们的样子,韩冈打定主意,以后尽量少喝酒。他摇着头,就听着他们东拉西扯的,一路走回到了驿馆中。自明天起,他既不用去流内铨报到,也不用去王安石府守门,可以安安心心的逛一逛东京城。这么想着,韩冈躺到了床上,便呼呼大睡。

但韩冈并没想到,他逛东京城的愿望并没能实现。次日日上三竿,他一觉醒来。刚刚起床洗漱完毕,正准备吃饭,就有人上门来拜访。驿卒在门外通报了,他出厅一看,却见是一个胖乎乎的老头,后面跟着个油头粉面的随从。

“章老员外?”韩冈吃了一惊。昨天他不是请刘仲武和路明喝了一晚上的酒吗?现在大清早就又赶过来,这未免也太殷勤了吧!

再往章俞的身后看去,他的伴当的确像刘仲武所说,是个半男不女的人物,不用说,跟章俞肯定有些暧昧关系。兔子、相公、零号这些都是后世的称呼,韩冈不知道这个时代对断袖分桃的爱好有什么别称,当然,他也不想知道。

章俞对着韩冈拱手行礼:“恩公贵人事忙,小老儿总是错过,今天便特意来得早一点。”

“老员外这话就让韩冈无地自容了。小子即不贵,也不忙。昨日诠试已过,现在只等官诰,却是清闲得紧。”韩冈把章俞往驿馆外厅的楼上请,那里比较清静,回头又对李小六道:“快去把刘官人和路学究请来。”

“昨日小儿回家,也问起恩公……”

韩冈忙打断章俞的话,“恩公二字还请老员外不要再提,韩冈举手之劳,微末之功,实不必如此。老员外唤韩冈本名也就是了。”

章俞连连摇头,唤人本名在此时可是训斥或辱骂时才用的,韩冈的一点自谦之言,他却不能听从:“这样吧,小老儿托大,便唤你一声玉昆。不过玉昆于小老儿有救命之恩,这‘老员外’三个字,小老儿也是担当不起。小老儿行四,玉昆你直称章四便可。”

韩冈哪能这般不知礼,反正如今的习惯都是在姓和排行之后加个‘丈’字,比如范仲淹、司马光排行都是十二,便人称范十二丈,司马十二丈,也有省去排行的,像王安石就直称王丈,“小子还是称老员外为章四丈吧。”

一通关于名讳称呼的谦让仿佛是废话,韩冈心中也是不耐,但古时称呼礼节是人际来往中甚为要紧的一桩事。名正言顺四个字,可不仅仅指的是做事。

章俞与韩冈走到二楼,在窗边相让着坐下。

章俞当先笑道:“听说玉昆昨日已过铨选,只等官诰发下。由布衣得荐入官,一年也没几人,比进士还金贵些,该好生庆祝一番。昨日贺过刘官人,今天就为玉昆贺。”

韩冈推辞着:“在下昨日去王大参府上,大参和编修【章惇】他们有要事相商,在下不敢打扰,等了一阵便回来了,今天说不得还要再去一趟。”

“那也没关系!就改在中午去樊楼好了。虽然比不上夜中热闹,但点花魁时,也不用你争我夺了。”

“去樊楼?!”刘仲武和路明被李小六找上楼来,正好给他听到章俞的话。昨天他喝得太多太猛,今天起床后头疼得厉害。但一听到樊楼二字,刘仲武便立刻感觉不到疼痛了,“昨日韩官人也说今天要去樊楼庆贺一番,正好章老员外来了,那就一同去好了!”

“那真是太巧了。”章俞大笑着站起身,拉起韩冈的手:“事不宜迟,那就一起去。”

被章俞拉着手,虽然是此时的习俗,更亲近的把臂同游也是常见,可韩冈心中还是一阵恶寒。只是看着章俞身后那位伴当,韩冈暗自庆幸他跟自己的形象差得很远,应该不用担心章俞会有什么别的心思。

樊楼春色,天下闻名。即便是韩冈、刘仲武这样来此西北边区的土包子,都是觉得如雷贯耳。樊楼本名为矾楼,又叫白矾楼,已有近百年历史,本是矾业行会的会所。就像同为七十二家正店、位于牛行街的看牛楼酒店,本也是牛贩行会的会所,后来才改为酒楼。矾楼之名在百年间以讹传讹,变成了樊楼。如今听着章俞说,樊楼的新近换主,却有着将其改名的意思。

章俞拉着韩冈一众从城南驿出来,不移时便到了内城东华门外的樊楼前。京师第一楼,或许也是天下第一楼的门面,当然要比秦州的强出百倍。迎客彩棚——京师里称作彩楼欢门的门楼,门楼高宽皆三丈,比城门也差不离了。被七色彩绢结成的绢花所缠绕,花头画竿,醉仙锦旆。

欢门内,是一个横阔三十步的天井,天井周围,便名震天下的樊楼。樊楼建筑由五座两层楼阁组成【注1】,每座楼阁之间,还有拱桥相连,桥面弯弯如虹,就跟汴河上的座座虹桥一般形制。而每座楼阁面朝天井的地方,都有一条走廊。

听章俞介绍,每到夜中,拱桥、走廊上皆是彩灯高悬。楼中的数百妓女,都是浓妆艳抹,站在桥廊之上,以待酒客呼唤。

“自然,那些都是普通妓女,若是红牌便不须如此做作,如是花魁行首,便是达官显贵也要求着来。”章俞笑着,与韩冈一众进了当面的正楼中。

注1:按照《东京梦华录》记载,在宋徽宗的宣和年间,樊楼还有一次改建,由两层改为三层,比皇城城墙还要高出些许,站在西楼的三楼上,可以俯视皇城之中,后来西楼便被禁止对外开放。

