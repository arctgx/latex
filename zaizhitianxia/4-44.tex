\section{第九章 鼙鼓声喧贯中国(四)}

【中午一点前补上下一章。】

古万寨离着邕州不远,沿着左江上溯五六十里,就是古万寨。

此寨是邕州的南方门户,与更南面一点永平寨、太平寨,一起组成了抵挡交趾的南疆防线。

同时古万寨又是处在交通要道上,商旅来往频繁,一直都是富裕之所。自侬智高兵败之后,又经过二十余年的发展,军寨之外,更是形成了一个富裕的大镇。邕州这里一直都有将古万寨升为县治的想法,只是经过了蛮贼的劫掠,寨外近千民居尽数焚毁,生民涂炭,

苏缄站在古万寨的寨墙上,望着寨外的镇子,满眼都是经过火焚后的灰黑色的痕迹。墙倒屋塌,烧成了黑炭的梁柱,孤伶伶的支在灰烬之上。一具具尸骸被放在空地上,无不是被烧得面目全非,让人难以辨认。

在蛮贼来袭之时,此地的镇民大半逃进了寨子里,但还有一部分没能及时离开。蛮贼在寨外大肆屠杀劫掠,寨中守军却一步也不敢踏出寨墙。等到苏缄遣军来援,贼人已经是散诸山野,追之难及。

劫后余生的人们在废墟中寻找亲人的遗骸和残余的家财,而被放在一旁的小孩儿坐在路边哭号。苏缄看得心如刀绞,一个劲的低声念叨着,“此乃吾之过,此乃吾之过!”

“皇城,此非自责之时!”一名身材瘦小的士人走了过来厉声说着,他是苏缄的幕僚,在其幕中时日不短,“轸入鹑尾,位在荆州。这广南两路,亦是荆楚之地。彗兆兵灾,天兆已显!”

这两天,明明白白挂在天顶上的彗星,广西这边都看到了。前日出现在轸宿之中时,也不过将几个星子都比得暗了。但只过了两日,扫帚一般的尾巴就长了一半,不仅遮住了天车中央的长沙星,还将左辖星也给掩了,天车四星的光芒全被彗星给压了下去。

轸即是车,又有悲恸之意,故而轸宿多凶,而彗星更是不必说了,这是凶上加凶。苏缄越看越是心惊胆跳,依照天地分野,邕州所在,却正在轸宿对照的区域。

这名幕僚与苏缄一起站在墙头上,忧心难耐:“新得谍报,交趾和广源州近日已经在召集乡兵,不日即将北犯。而前两日的蛮贼之乱,也当是他们先得了消息。皇城,可要即刻奏请上闻啊!”

苏缄无可奈何的长叹了一口气,越发苍老的容色有着一片苦心不得认可的痛心疾首:“奏疏何曾有用。都在说着主少国疑,妇人当政。二府诸公,几曾正眼看过南方?”

“杀太后,逐顾命,如今在交趾国中垂帘听政的倚兰太后,可不是等闲角色,岂可当成寻常妇人。”幕僚狠狠的咬着牙,“若坏南疆大事,朝堂首当其责!”

苏缄干枯的双手紧紧按着墙头雉堞,手背上青筋凸着,轻颤的双臂,显见心情已是难以自抑。

熙宁五年,李乾德即位,上其父李日尊伪号为圣宗。李日尊的遗诏,是命王后杨氏为太后垂帘听政,太师李道成在外辅佐。但一年之后,新登基的李乾德就以皇太后杨氏阻生母倚兰太妃问政的罪名,将其连同宫人七十六名幽禁于上阳宫,紧接着又勒令一众殉于李日尊墓前,同时又将辅命大臣太师李道成出知于外。转眼之间,掌控朝政之人就成了李乾德的生母倚兰元妃。

这一雷霆手段当然不是七岁小儿能拥有的,而是倚兰太后的功劳。不过其中若没有伪圣宗朝,统领交趾国中大军,被封为辅国太傅、天子义弟的李常杰相助,也是不可能做到。而且在传说中,倚兰太后与这名功勋赫赫的大将,有些不清不白的关系。

倚兰太后出身寒微,是李日尊出巡时,正好看见她采桑而归、倚立兰草之中,悦其色而将之收入后宫,故而才有了倚兰的名号。但她如今却在紫宸殿上,坐于帘幕之后,可知其心术手段,与大宋的庄献太后刘氏不相上下;且又有着能臣‘辅佐’内外,又让人不得不联想起北面的那位曾经统率大军杀入中国的契丹承天太后萧氏。

妇人掌控朝政,野心甚至会比男人还要强,史书多有明载,根本就不需要再多举例。若是以为交趾主少国疑,不敢出战,那可就大错特错。

仅仅是为了要镇服国中异论,倚兰和李常杰就必须夺取一场大胜。再加上刘彝知桂州、掌广西兵马之后,禁绝与交趾市易,交趾国中各部族已然不稳。只要不想这把火烧到自己,交趾太后和那位天子义弟也必须拿大宋开刀。

虽是嘬尔小国,野心从来都是不小的。

苏缄回头望着左江两岸的重峦叠嶂,心中也是大恨,用力的跺着脚:“如何还不防备!”

……………………

天兆每一个晚上都在提醒着京城里人们灾异就在眼前。而朝堂上,则正在争论着这彗星到底算是哪边的问题。

彗星对新党的打击,与去年的旱灾一样有力,王安石只觉得自己的运气当真是坏透了,旱灾连着几年,北边旱罢,南方又旱了起来,如今天上又来了彗星,使得东京城中人人惶惶。

因为天上一颗突如其来的不速之客,天子已经照规矩避殿损膳,又下赦诏,求进言,这对王安石不啻又是一个打击。

“比年以來,灾异数见,山崩地震,旱暵相仍。如今彗出东方,变尤大者。內惟浅昧,敢不惧焉?”

只看诏书中的这几句,王安石就知道天子又在动摇,而在外的元老重臣又要上蹿下跳了。

他也向天子解释了:“晋武帝五年,彗出轸宿,十年,又出轸位,而其在位二十八年,与《乙巳占》所言不合。天道远,当修人事。”但也要天子相信才行。

王旖从娘家回来,心里面也是沉甸甸的。不比父兄对天变毫不顾忌的态度,吴氏和王旖都是为着天上的灾星而忧心忡忡。

回到家里,往内院走,就看见西厢的书房里面正亮着灯,透过窗纸,能看见韩冈正坐在桌前。

王旖走进书房,里面却是一团乱,书架上、地面上,都摊着一本本书,到处乱丢着。严素心领着一个小丫鬟正蹲在地上将书一本本的收起来,见及王旖,立刻起身行礼。

韩冈则是不管不问,放在手边的药汤饮子上冒着热气,应该是刚端来的,只是他动也不动,就对着桌案上放着一页纸皱着眉头。

“官人,怎么了?”王旖进来后,看到书房中仿佛劫后余生的样子,就惊得瞪大了眼睛。本来要对丈夫说的话,一下都忘光了。

“回来了?”韩冈抬头微笑,随手拿起桌上纸页递给王旖。

王旖疑惑的接过来一看,薄薄的纸页墨迹尚新,显然是刚写不久,字也是丈夫的字,不是她以为的信笺。从右到左,一列、一列的排列整齐,条目分明。

打头的一条,是‘始皇七年,辛酉。彗星先出东方,现北方;五月,现西方,十六日’。在这一句后面用小字标着个‘一’。

下一条,‘汉文后元二年,己卯。正月壬寅,天欃夕出西南’。这一句后面则是标着个‘七十八’。

王旖也是这两天才知道,彗星的别称众多,天棓、天欃、天枪、孛星、蓬星这些名词,都是指得彗星。她莫名其妙的问着韩冈:“官人,这是什么?”

韩冈有些疲惫的笑了笑,今天他可是很费了一番精神,去历朝历代的史书中查找他要的资料,“继续往下看就知道了。”

王旖依言低头继续看。收拾好书房的严素心,又把日常养生用的药汤饮子端到韩冈的面前。

第三条是汉昭始元元年乙未,‘汉宦者梁成恢及燕王候星者吴莫如,见蓬星出西方天市垣东门,行过河鼓,入营室中。’

第四条是汉成帝元延元年己酉,‘元延元年七月辛未,有星孛于东井。’

一条条有关彗星的记录,依照年代延续下去,汉、晋、南北朝、隋、唐、五代,直至国朝的太宗端拱二年、英宗治平三年,总共一十八条。每一条都是标着年号、干支,而在结尾处又写着一个数字,最后一条结尾的数字是一三零七。

王旖形状姣好的双眉皱了起来,从头又看了一遍,还是没看出其中有什么门道。“官人?”她张着疑惑的双眼问道。

韩冈啜着药汤,指了一指纸上,“你可以算一算,每一条记录的前后隔了多少年?”

按着年号算间隔时间,除非是对史料融会贯通,否则绝对做不到。可用干支来计算,对后人也许很头疼,但对于已经习惯此中纪年法的人们来说,却倒是不费多少神。王旖默算了一番,竟然发觉相邻两条的间隔,却都是跟韩冈写在各条记录后面的数字相减后的结果一样,而且总在七十六上下。

“这是?”王旖更为疑惑,这是究竟是巧合,还是别有一番原因。侧着头,看着韩冈,等着丈夫的解释。

