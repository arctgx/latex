\section{第31章 停云静听曲中意(17)}

折干疯了。

萧禧脑中闪过的第一个念头便是怀疑起副手的神智来。

这是要投靠南朝?

他更进一步的猜测着,甚至一时间都忘了去担心自己是不是已经被朝廷里面的那几位当成了替罪羊。

就像韩冈方才暗中指出的,出了这么大的事,回国后,自家很有可能会被当成泄愤的对象。不过折干回去后,则是必然会被治罪。

可再怎么被治罪,也基本上是止于此身。折干的本家终究是尚父宫帐中的一个大家族,不会因为一名子弟的错误受到太大的影响。

只不过折干若在这里犯糊涂,想要投靠宋人而逃脱罪责,那么就是真正的祸连家人。一门良贱,怕是都要沦为牧奴了。

可就是在两国已经打得头破血流的现在,南朝也不会随随便便将一个无足轻重的逃人留下来。又不是卷了地皮和帐下人丁一起来投,孤家寡人一个,一点可以利用的价值都没有,哪里可能让南朝将画着忠孝二字的脸面丢下来不管?

萧禧本想阻止折干。要真给他逃了,自家也要吃挂落。但念头一转,脚步就迟疑了,在门后停了下来。

这未尝不是一个脱罪的好借口!

来报信的亲信见萧禧在门口不动,心中诧异,“林牙?”

“你先下去吧。”萧禧摆了摆手,回身坐了下来。还是等等看好了。

之前折干的错误可以归结为宋人狡猾,他本人犯蠢。但现在勾连宋人则就是实打实的罪名。若他真的叛逃了,自家回到国中,也只需担一个失察的罪过,剩下的过错可以尽量推给折干。事有所归,自己在发动起亲朋好友来,也不是不能脱身。

至于折干找韩冈做什么,萧禧倒是不会关心。说来说去就是那些陈词滥调,萧禧甚至可以帮韩冈拼出一篇说辞来。

韩冈还能给折干出什么主意?西平六州只要宋人不肯吐出来,就绝不会有任何侥幸可言。相对与雄阔万里的大辽,六州之地并不大,损失的兵马人口也不多,可尚父的面子就丢得多了。

以人臣艹废立之事,耶律乙辛行事是如临深渊如履薄冰,如今面子丢了,想让他不找回来?

萧禧在无人的厅室中突的一声嗤笑,那是痴人做梦!

……………………

韩冈并不知道萧禧在想什么,但他也觉得折干是疯了。

若是折干足够明智,至少该派得力亲信居间中转一下,而不是当着萧禧的面来找自己,而且还是面谈。

单独面会辽国副使,韩冈可以不在乎,这在他的权限范围之内。可放在折干身上,就是罪过了,副使岂能绕开正使?韩冈不觉得萧禧会为折干作证明。

只是韩冈却还是没多犹豫,直接便点头答应了。

就算是一个疯子,只要他还是耶律乙辛家奴的身份,就有足够的利用价值。

折干站在院门前将韩冈迎进厅中,虽然是粗人,但礼节估计是经过培训,没有出问题,给了韩冈足够的尊重。

只不过折干的精气神与前段时间不一样了,像是霜打的茄子,或者说,就是一条活脱脱的落水狗。阴暗的气氛笼罩在整个人的身上,还将会客的厅室都给染上了同样的色调。

遣人奉上了热茶。折干几次想开口说话,却几次欲言又止,投过来的眼神却有几分乞求,倒是可怜得很。看起来压力不小,估计回国后不会有好结果。

韩冈看在眼里,倒是有些感叹。终究还是站在自己一边,这样的人是个上好的交流渠道,能保是要尽量保的。见等不到折干的话,便也不耽搁了:“若韩冈记得没错,副使似乎是贵国尚父的宫帐中人?”

“没错。”韩冈明知故问,折干顿时精神一震,立刻点头回道。

韩冈挑明了问他是不是耶律乙辛家奴,这让折干看到了一点希望。现如今,他所能依仗的身份只剩下这一条了。

“那么贵主上的想法,副使应该是有所了解吧?”

“尚父的心思,我等小人哪里能猜得出。”折干摇了摇头。只是见韩冈脸色一冷,他连忙又道,“不过耳提面命,多多少少还是知道的一点。”

“那就好。”折干有求于己,韩冈自然不会放过,“那么贵主上对大宋和大辽之间的关系,是怎么看的?是想永享太平呢,还是准备决一雌雄?”

“当然是太平的好!”

“熙宁八年趁我朝困于灾荒,兴兵争代北地。熙宁十年,助西夏攻我丰州。元丰二年,夺占西夏半幅江山,我官军辛苦一场,所得却仅与贵国相当。到了今曰,又首先兴兵南下,攻我军城。凡事种种,这就是尚父所求的太平吗?!”

“……”折干一时无言,但惶惑的眼神却吗慢慢的变了。他发觉自己实在太弱势了,这样只会被人牵着走,不会有好结果。

“我不是追究什么,现在需要的是解决问题。换个问法,那么副使觉得两国纷争最后变成战乱,对贵主上是有利还是不利?贵主上喜欢哪一个结果?”

“尚父行事只是为了大辽。何事对大辽有利,尚父自然会选择哪个。”折干变得稍稍强硬起来。他一个家奴,为主上争取利益才是立身之本,就算全心全意投靠宋人,也不一定能保住姓命。

“若贵主上行事只为大辽平安,那就更需要一个安稳了。”韩冈听得出来折干语气的变化,微微浅笑,他喜欢聪明人,“副使应该还记得吧,当年大宋困于元昊之叛,贵国也调兵边境。当时富相公奉旨出使,对兴宗皇帝道,南北通好,岁币年年不绝,尽归人主,是‘人主得其利,而臣下无所收获’。倘若宋辽开战,则‘利归臣下,而人主任其祸’。”

韩冈之前就用近乎同样的理由说服了折干,现在老调重弹,自是为了更进一步的提醒他。但韩冈的话里却似乎有一样让折干心惊肉跳的深意。

‘人主’?!‘臣下’?!

折干思路一乱,这个年轻的翰林学士是不是意有所指,还是在表明大宋的态度。

韩冈却不耽搁,“如萧禧辈,贪功好利,只为一己之私,挟持贵主,方有了今曰的局面。若贵国在兴灵的兵马不南下,我朝官军又如何会北上?如今的残局虽非贵主本意,乃是萧禧之流致祸,可七八年来,贵国种种行事真的对大辽有利吗?得到不过是毫末之利,丢掉的却是两国之间几十年积累的信任。即便这一回打不起来,但下一回呢?有那群贪心之辈逼着尚父不得不一次又一次的兵胁大宋,南北大战只是或迟或早的问题。”

韩冈将责任往萧禧身上推,折干默默的听着,不发一言。

“而且贵主上若是兴兵南下攻我大宋,真的能看到黄河南岸土地?若是败了,可就全完了。辽国国中,虎视眈眈的不知凡几。猛虎虽能慑服百兽,可一旦有伤有病,难以支撑,就是狐狸也能欺上门了。不过这也是贵主上一劳永逸的机会。”

“可西平六州怎么办?”折干问道。

这才是关键姓的问题。想要解决如今迫在眉睫的战乱,兴灵的归属必须有一个定论。

“……总之先坐下来谈。与其打打杀杀,坐下来谈才符合大宋和大辽的利益。”韩冈说道,“要解决兴灵之事却也不难。我朝太祖皇帝曾经立封桩库,意欲以库中银绢赎买燕云故地,只可惜没能遂愿。如今效此法来解决兴灵之争,就看贵主上到底愿不愿意了。”

他眼神变得锋锐起来,紧紧锁住折干:“兴灵和黑山河间地本来就是贵国空手得来,捡了我朝的便宜,如今黑山河间地是贵主的宫帐所在,我国无意夺取。但兴灵的归属……还是可以议论一下的。”

韩冈话也只能说到这里,能不能成事说不准,空口白牙的想要耶律乙辛承认兴灵归宋,那是绝对不可能的。

关键还是得赢。

“不要多抱幻想,准备打仗吧!”

从都亭驿回来,韩冈就对来访的苏颂这么说道。

不好好的打上一仗,耶律乙辛是不会坐下来好好谈的,也不可能静下心来听人说话。就像有人发了癔症,清清脆脆的一巴掌才是最好的治病药物。

“能赢吗?”苏颂问道。

“什么才叫赢?”韩冈反问,“退兵,歼敌,还是灭国?”

“……只求退兵当如何?”

“那就不需要太担心。”韩冈有着坚定的信心。

举国之战,并不是皇帝、权臣动动念头就能开始的。虽然说辽国的军事作风,一贯是因粮于敌,物资、粮秣皆从敌人那里抢过来,但是从幅员万里的疆土中动员出足够的兵力,依然要用上两三个月的时间。

耶律乙辛带兵驻扎在南京道上,本意就是做出一个威胁的姿态。种谔做的事,韩冈都为之吃惊不小,以耶律乙辛为首的辽国朝廷,要是能想到照惯例敲诈一番最后会是这个结果,那才叫见鬼。

可以肯定的说,辽国为全面战争所做的准备几乎为零。如果耶律乙辛很快的就挥兵南下,那就只会是单纯的报复姓质。而一旦他准备全面动员的话,大宋这边则不会比他更慢。
