\section{第11章 城下马鸣谁与守(16)}

李宪起手在棋盘上放下座子,抬头看了眼对手,疲惫的叹了一口气,“这是第七盘了……”

韩冈坐得端正,精神抖擞的,随即也在两个对角的星位上将两颗白子放下,“连输了四盘的彩头,这一盘一定要回本。”

韩冈的棋艺向来平平,李宪的棋艺比起他来,胜过三四筹还是有的,差不多是让四个子、五个子的差距。但韩冈偏偏要分先,为了不让韩冈输得太难看,李宪每盘棋上都是绞尽了脑汁。

都是寒意已深的暮秋时节,几盘棋下来,李宪的小衣都给汗水湿透了。他几次想要故意输给韩冈,但想不露破绽的输掉,却比小胜一筹更难,磨到最后,却送了韩冈一个六连败。

韩冈执白先行。开局阶段,两人落子如飞,来回二十几手后,棋盘上的大势已经勾勒了出来,韩冈毫无悬念的落了下风,不过他本人并不以为意,反倒更加悠闲,随手在棋面上落了一子:“李瑛此时当已赶到那群阻卜贼的落脚点了。”

李宪低头的看着棋盘。韩冈喜欢乱落子,刚开始两盘,李宪还以为他别有深意,小心提防着。但两盘一过,就将韩冈的底细看得通透。不用再提防,却是得小心不要赢得太多。考虑了片刻,稳当当的落子尖了一手:“就是李瑛的兵力少了点,让人担心。贼人可是他的两三倍”

韩冈的应手更为随意,飘忽不定的在另一端落下:“强盗的目的是财货,不到狗急跳墙之时,不用担心他们会拼命。”

“不过李瑛为人贪功性急,就怕他追敌的时候,为贼军所乘。”

“不是有折家的将种跟着嘛?有郭仲通都看重的人在旁提点,不用担心李瑛会犯迷糊。”韩冈啪的一声,落子后抬头笑道:“所谓用人不疑,既然用了李瑛,还是等着他的好消息好了,勿须操心过度。”

李宪干笑了两声。剿灭阻卜贼寇,李宪本来想亲自出马的,但韩冈既然坐镇在晋宁城中,就没有了他领军的余地。而韩冈更不会领军出外,从没有说要经略使亲自上阵的道理。两人闲来只有下棋。

李宪依然是深思后才应上一手:“李瑛若是能小心一点,击溃贼军当不在话下。正面相对,只要有时间给官军准备,党项的环卫铁骑也罢,契丹的宫分军也罢,都是不在话下的。”

李宪刚落子,韩冈就啪的一声紧接了一招:“只要能击溃贼军,这一仗就赢定了。”

……………………

长刀如林,军阵如山。

当短促的晋腔伴随着刀林倾泻而下,当厚重的军阵顶着奔驰的马群逆冲而上,如同一盆来自数九寒天的冰水,将阻卜人兴奋和狂躁彻底浇熄。

古名陌刀的斩马刀,六尺长、半尺宽,重愈十斤,半为刀柄,半为刀刃。宋军战士们紧握刀柄,劈下刀刃。前方的阻碍,都在尖啸的刀锋掠过之后,一分为二。

一排排雪亮的刀光,卷起了道道血光。

骑手、战马,拥挤在战阵前的一切,皆染上浓浓的血红。

陌刀阵如墙而进,刀转如轮,挡者披靡,人马皆碎。党项人这几年来,早已用生命和血液凝炼成了刻骨铭心的教训。

尽管同样知道斩马刀的可怕,也的确曾经见识过几次斩马刀的挥击,但阻卜人还是缺乏足够的切身体会。没能赶在宋军列阵前进入战斗,党项人基本上都会转身就走,而阻卜人则没有做出这样的决断。

命运就在一瞬间决定。

冲杀在前排的阻卜骑兵们,还没有掀起半点波澜,便被层层刀浪卷得不见踪影。

当呐喊着向前冲击的士兵,将手中的斩马刀挥斩如轮,卷走了敢于挡在前路的敌手,李瑛终于传令后阵和两侧山坡上的弩手们,射出他们在弩槽中等候已久的箭矢。

神臂弓弦铮铮鸣响,千百具弩弓此起彼伏,缀连成一首杀气腾腾的曲乐。以连绵不绝的惨叫声为伴奏,让身在后方的余古赧,从心底里寒气直冒。

仅仅是接战后的片刻时间,冲在最前面的百多人就已经不复存在。只看到一片片刀光无可阻挡的破波斩浪,疾飞的箭矢密如急雨。

侧头看了看二十丈外的老对头,乌八煞白的脸色,余古赧知道应该也同样出现在自己的脸上。

他和乌八都在领军前进的时候,不动声色带领本部的落在了后面。若是对手强势,他们不用担心自己的部众损失太多,若是对手不堪一击,凭着他们手中的实力,也能在战利品中占上最大的一份。这是长久以来的经验,也是他们的特权。其他的部族也都知道这样做的好处,却不敢学着他们的榜样。更弱小的部族,则宁可拼上一拼,否则分配战利品时,永远只有残羹剩饭。

最前方的几个部族已经彻底溃败,却因为后方一时无法顺利撤退。宋军正在乘势掩杀,高高举起的斩马刀,这一次是想将敌人斩尽杀绝。杀气腾腾的态度,让余古赧和乌八当机立断,调转战马,转头就走。

……………………

“李瑛那里该有个结果了。”李宪双手拢着温热的茶盏,感受着传入掌心的热力,看着战火正炽、烽烟处处的盘面,还不忘跟韩冈说着正事,“阻卜人和官军的战阵都是以快打快,没有僵持太久的例子。”

韩冈长考再三,终于落了一子。正要说话,突然眼神一变,望向厅外。“应该是结果来了。”

照壁后的脚步声,随即也传到了李宪的耳中。当一名士兵脚步轻快走上台阶,李宪就知道这肯定是个好消息。

斩首四百余,战马俘获了两百多。

这还仅仅是击溃的结果。若是换成歼灭,还不知道要翻上几番。

所有的人心情都是火热了起来,要是能在这件事上做出点成绩来,不仅仅是站在韩冈面前的位置可以上移几位,甚至有可能去东京城,觐见天子。

李宪对韩冈道:“得盯着贼人逃窜的去向,否则日后还是一个麻烦。”

“沿途各寨堡都遣人带了飞船去。飞船上了天后就可以看得足够远,想潜渡过去,光是运气可是远远不够。”

李宪神色一动,问道:“听说龙图手上还有一个新的发明,能与飞船配合得天衣无缝,让贼人无所遁形?”

“是千里镜吧?”韩冈也同样十分配合,让人将千里镜取来,“此乃天子所赐。乃是东京城中的能工巧匠所打造,并进献给了天子。”

李宪拿着黄铜质地的千里镜啧啧称奇,摆弄来摆弄去,对着厅外的树木看了半天,又举着望远镜看天上的情况。李宪虽然是天子近臣,但他也知道,有些东西韩冈能先得到,他却不够资格。

摆弄了好一阵,李宪方才恋恋不舍的放手:“此乃军国重器,质保一般的原本放大镜,眼镜,显微镜都大量耗用了不多的白水晶,现在又多了一份必不可少的开销。”

“等着水晶玻璃出来吧。到时候,放大镜、眼睛应该就能普及了。”

“广州的蕃商那里的玻璃器皿大半都是透明的。若是能得到透明的水晶玻璃的制法,与透镜有关这些器物,肯定能遍及天下。”

……………………

一直以来,余古赧都是以士兵的多寡来计算对方的战力。

可是在被宋人以劣势兵力大胜之后,却让余古赧绝不敢小觑任何一支宋军的队伍,无论人数多寡。

在他的身后不远处,是一直紧随其后的另外两个部族的军队。他们跟着余古赧,在崇山峻岭一条条岔道中转来转去。

除了他和乌八以外,其他部族基本上全都在突击宋军的过程中,遭受了或多或少的损失。只有他们两人,悄悄的落在后面,看到局势不利,无法击破宋军的阵列,便立刻选择了撤离。

不断逃窜中的队伍中,战马驮着惊慌失措的骑手,很快就到达了马匹的极限。

战马的惨嘶时不时的在余古赧身边响起,一匹匹战马累倒、垮下。但余古赧却毫不吝啬马力,飞快的从一条谷地窜到另一条谷地。而就在这个过程中,一家家部族都选择了远离,设法独自离开葭芦川,而不是跟着最为显眼的余古赧。

离开葭芦川的道路几十条,绝不可能全都堵上。余古赧就是抱着这样的想法,才敢悄悄的穿过宋军用心经营下来的防线。但接下来赶来的十几家部族,却让余古赧的盘算成了空。

前面就是小红崖,看起来十分平静。在一番试探之后,余古赧的前军小心谨慎的走进了小红崖东侧的谷地。而余古赧的后军,此时还在五六里外。

行军的时候,是一军之中最为脆弱的时候。

一批批士卒进入了山谷,领军的余古赧却没有半点体恤的催促着他们加快脚步。如果能顺利的通过小红崖,再疾行二十里,便是能让阻卜人顺利离开葭芦川的出口。

随着进入小红崖谷地中的部众们越来越多,余古赧的心渐渐的提了起来。如果有什么变化,就该是现在。

这个念头还在脑海中旋转,尖利的木笛声就从前方的谷口响起,余古赧二话不说,一拨马头,就往侧面一条山谷冲进去,后面的部众匆匆跟上。

只要还活着,迟早能有回去的机会。余古赧宁可狼奔豕突的逃窜,也不愿意拼上一拼。

性命才是一切。

