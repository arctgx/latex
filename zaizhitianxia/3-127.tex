\section{第34章 雨泽何日及(二)}

“难怪!”

郑侠别出心裁的一手,让韩冈也为之惊叹。

只是‘难怪’二字一出口,王雱和吕惠卿的脸色就都难看了几分。

“玉昆,这不是佩服人的时候!”王雱阴着脸说道。

韩冈却笑道:“不妨事的。”

吕惠卿为人深沉,眨眨眼的功夫就恢复了正常。韩冈的自信让他可以安心,但他不忘提醒:“郑侠献上的那可是图!”

韩冈收起了笑容,正正经经的重复道,“不妨事的。”

韩冈当然明白流民图的作用有多大,栩栩如生的图画远比白纸黑字的奏章更有说服力。当实实在在的图像和空虚的文字摆在一起的时候,哪边更为可信,想必绝大多数人都不会犹豫。

所以吕惠卿和王雱都一下失了方寸并不奇怪,此图一上,原本好不容易才稳定下来的形势完全又都给扭转了回去。

对于这场从熙宁六年延续到熙宁七年,时间长、范围广、受灾民众为数众多的旱灾,最佳的应对,就是当地的知州、知县施政得力,将灾民安抚在治内——当然,这是不可能的。

剩下的招数,就是不能让大股的流民抵达京师,否则京城中略有动荡,反映到朝堂上时,就是一场大地震。

这个道理人人都懂,但如果没有韩冈,王安石就很难有办法应对。因为他手边,除了曾、吕等寥寥数人,在治政的能力和经验上,却也找不到一个合适且可以信赖的人选,总不能让曾布或者吕惠卿出外吧?

同时从品阶上,也只有韩冈最合适。要知道,韩冈的本官品阶,一年前还在吕惠卿和章惇两人之上,只是吕惠卿升翰林学士,而章惇在荆南立功,才又反超了过去。如果将韩冈算进来,新党中的重要成员中,他的官阶排得很靠前,仅次于吕、章,以及背叛出去的曾布。王雱、曾孝宽、吕嘉问等人其实都不如他。

从关系上,韩冈还是王安石的亲女婿,虽然因为荐张载入经义局,两人有了纷争。但韩冈在政治理念上,还是站在新党这一边。而且王安石和韩冈因为经义局之事而有了矛盾,还是一个优势。韩冈出任白马知县,在外界看来,是王安石嫌女婿碍事,所以将他踢出去——尽管其中有很大一部分是事实,当时的确也无人能确定旱灾一定会延续到此时——想看翁婿俩笑话的人很多,故而为韩冈的准备工作争取了不少时间。

韩冈的成功让人喜出望外,不过若是他没有成功的阻挡流民,王安石他们的就得再退而求其次了,于京师城外安稳住流民。而那时候,就要设法钳塞住天子的耳目,不能让他知道流民的惨状。尽管这样做要费些周折,幸而天子不可能出宫视察,两边都是空口白牙的说话,到时候就要拼一下天子到底会相信谁了——失败的例子虽多,但成功的案例也不少。

可谁能想到郑侠会献上一幅流民图?

韩冈没有看到图,不过他能想象得到图上画的是什么。

世人都是相信眼见为实,耳听为虚。赵顼作为天子,没有随意进出宫城的权力。他能做的,仅仅是坐在一成不变的宫室中,从冷冰冰的文字里,了解他的国家现在的情况。

他有耳目,他有密探,皇城司可以清查京城内外之事。可赵顼得到的报告,依然是冷冰冰、毫无感情、且经过修饰的文字。

‘民情忧惶,十九懼死,逃移南北,困苦道路’这些干巴巴的文字如何能触动人心?百姓衣衫褴褛,啃食草木,易子相食的惨状,区区文字能描绘得出?即便有着王安石、苏轼一般的笔力,也不可能让从没有忍饥挨饿、受困受冻的赵顼,体会到无法获得赈济的流民们的困苦。

而一幅绘画水平不要太高的流民图,却肯定让从没有见识过的皇帝感到怵目惊心。

如今流民们的整体情况,其实要比所有弹劾王安石的奏章中所描述的情形要强出不少。可文字和绘画都是艺术的一部分,艺术上的夸张绝不会缺。不论是奏疏还是流民图,想必郑侠在其中夸张的程度不会太轻,否则不至于让赵顼留了王安石到现在。

这个时候,王安石只有两点还算运气。

一是郑侠拿着白马县作为他的论据,第二,他韩冈就在这里。

韩冈因此而胸有成竹。但王雱却不放心。怎么说韩冈也是空口白话,他说白马县安置流民稳妥,能不能让看了流民图的赵顼相信?天子不可能离开宫中,亲自去白马县看个究竟。而当皇帝起疑心时,就算身边的亲近内侍,也不会全盘信任。

所以他再一次提醒妹夫:“那可是图!”

“不妨事的。”韩冈第三次重复着。

……………………

一封用着非法的手段发出去的奏章,惹了朝堂政局的大变。可始作俑者郑侠,却犹在安上门处盯着他的手下兵卒,仿佛什么事都没有发生。

人声,车马声,时时从窗外传进来,郑侠安居在城门边的简陋厅室中,暗自默诵着奏章上的文字。

“如陛下观臣之图,行臣之言,自今已往至于十日不雨,乞斩臣于宣德门外,以正欺君慢天之罪。如少有所济,亦乞正臣越分言事之刑!”

他擅发马递,这罪名是逃不掉的。但如果能让圣聪不再被蒙蔽,使得天子能了解到外界流民的惨状,如他所言,尽废新法,那么十天后还不下雨,就算被处以重刑,他也甘愿接受。

郑侠相信他的奏章和画卷,能对天子有所触动。前日亲自用笔书画的时候,他的心情激荡得都难以自持,手抖着,坏了好几副草稿。流民们的惨状历历在目,想必圣君阅卷之后,也会明了当朝宰辅阻塞言路、不使下情上闻的罪行,以及新法残民之处。

原本城南的流民不过数千,救治虽然不利,可也没怎么饿死人。郑侠本有心上书,但他知道这点流民人数,根本引起不了天子的注意。幸好让他听说了白马县竟有数万流民!

数万啊……这两天过来,说不定就有十万了!竟然将这么多流离失所的河北百姓堵在黄河边上,不让他们到京城来接受赈济,此辈奸佞当真可恨!

郑侠咬着牙,他几乎都能听到无数流民们哭号声压倒了滔滔黄河水。自家身受朝廷俸禄,哪能不为百姓申冤?!

“可恨什么?”

听着声音,郑侠抬头。一见来人,就收起了脸上的痛恨之色,迎客的声音说不上热情:“原来是东美兄。”

来人黎珣黎东美,扁鼻子,一对小眼,下颌突出,硕大的肚腩,却看不见脖子,脸上还疙疙瘩瘩,乍看起来像只蛤蟆。其绰号也是如此,只是黎珣听人如此称呼,却从不生气,是个好脾气的人。所以才能受得了郑侠的硬脾气,被王安石三番四次的遣来说话。

看到黎珣来访,郑侠开始担心,他的奏章到底有没有让天子看见。

郑侠知道自己被王安石看重,要不然前日也不会遣了王雱邀自家入经义局做检讨,又让黎珣三天两头的来寻自己说话,但正因为如此,他就决不能坐视王安石败坏了国政。如今内外皆忧,难道不是宰相之过?!

“不知介夫在恨着什么?”黎珣坐下来笑着问道。

郑侠沉着脸:“只是听说河北流民阻于白马,不得安置。”

“介夫,你这可说错了。”黎珣很惊讶的摇起头,“韩玉昆在白马县,凿水井,开沟渠,设营地,将数万流民都安置的妥妥贴贴。要不是他在此事上建有功劳,天子怎么会将他迁为提点开封府界诸县镇公事?”

“一县之地安抚住数万流民?”郑侠回想起前几天见到王旁时,说到白马县流民多达数万后就突然收口的样子,顿时嗤之以鼻,“笑话,真当世人都是瞎子吗?!白马县可只有两千户人家!”

“介夫,眼见为实啊!”黎珣劝道,“韩冈在关西屡有殊勋,亦多发明,去岁从南方运粮而来的雪橇车不正是他所创,还有水晶阳燧、霹雳炮等物就更不用说了,焉知其人不能安抚流民。”

“关西?”郑侠冷哼一声,“正是此辈贪功邀利,妄开边衅,生民膏血耗于无用之事,才让北狄蠢蠢欲动。素日只见南征北伐,边地诸将皆以胜捷之势,山川之形,绘图而來,却无一人将天下百姓质妻卖女、父子不保、迁移远走、困顿褴褛、拆屋伐桑、争货于市、输官籴米,遑遑不给之状报知于上。”

郑侠一连串的短句如同礌石一般砸了出来,身在王安石门下奔走,黎珣这位熙宁三年的进士却自有其才能。他相貌鄙陋,但口才不差,指了指门外,“不知介夫你说的这些,如今在哪里?”

郑侠闻言便怒上心头,双眉一轩,厉声反问:“难道没有吗?!”

“……难道很多吗?”黎珣悠悠然的同样反问着,“如果这一等苦,生民无人不受,天下早就处处烽烟,你我现下如何还能安坐此处?”

郑侠沉声道:“东美,须知防微杜渐之理。灾患未至时风平浪息,恍若无事,来时便如疾风暴雨,不可复御。流血藉尸,方知丧败,此愚夫之见。贵于圣神者,为其能防患于未然,而转祸为福也!”

他霍然起身,同样一指窗外:“如今之事,正是山雨欲来,藏之未发。不罢弊政,逐奸佞,救补于世,悔之晚矣!”

“罢弊政,逐奸佞?”

“所谓弊政,青苗、免役、保甲、保马是也。所谓奸佞,曾布、吕惠卿、吕嘉问是也……”郑侠恨声道,“如韩冈这等蒙蔽圣聪,诳言欺君之辈,更是决不可留!”

