\section{第30章 肘腋萧墙暮色凉(十)}

【第二章,求红票】

正月的古渭,公事一桩都没有,清闲得要命。虽然这对官员们来说,也算是件好事,但连个好玩的去处都没有,那就让人郁闷了。即便是正月十五上元夜,也只是各家门头上挑两个灯笼,衙门前扎几个一丈高的灯山凑个趣。还有七八具从秦州买来的烟花,摆放在衙门前的空场上,待会儿就要燃放。

作为新成立的通远军的治所,古渭寨不可能还保持原来的名字,有传言说很快就要改名做陇西县了,这是古时的称谓。

风起陇西,听着就要让人敬畏三分的感觉。

“给俺瞪大眼睛,把各处都盯牢了。若今天哪处走水没有及时回报,明天大板子伺候!”

古渭寨主傅勍,可能也是未来的第一任陇西知县,正指派着手下的兵丁,防着上元夜的火情。天下连着放灯三日,古渭也不例外。但冬日天干物燥,这灯一点起来,少不得有火灾。话说回来,若哪年没有烧个几家,那就不是真正的上元节。

此时身在古渭的大小官员,都按着次序坐在衙门大堂中的宴席上,大门敞开着,可以看见广场上的灯山和烟火。只是明显兴致都不高,这等乡僻之地的节庆,说起来,比起秦州这等大去处的寻常日子都不如,也没人有兴趣看着烟火灯山什么的。

幸好还有其他话题供人闲谈。

并不是横山那里已经拉开序幕的大战,而是今年过年后,在城中组织的蹴鞠联赛。

这球赛是韩冈上京前匆匆定下来的。城中分片分厢组成队伍,还有驻军按指挥出人,加上周围的村寨,总计十六支球队,其中有一支还是纳芝临占部的球队。参赛者照例都有赏金,衙门里拿出五十贯,而榷场的各家商户助兴,总计四百多贯的彩头,其中冠军能拿去四成。

有高额奖金勾引人,比赛就显得热闹非凡。单败淘汰制的比赛,通过抽签,排出对阵表。连续八天的比赛,就在昨日落下帷幕。来自骑兵指挥的球队夺下了冠军,披红挂彩的拿走了一百七十多贯财货,还有韩冈特意嘱咐让人打造的高脚银杯。当冠军球队的队正拿着碗口大的银杯倒满酒的时候,所有观众都一齐同声欢呼。

但让古渭城内城外,兴奋的不仅仅是比赛,闲来无事的人们,都是在看球的同时赌起输赢。

傅勍安排下监视火警的人手,坐回自己的位置,插进话来:“哪个不赌?赵隆在赌,苗衙内也在赌,还有王舜臣,他赌得最凶。”

瞅着王舜臣跑去王韶、高遵裕那边去敬酒,傅勍毫无顾忌,他管着古渭内外杂事,就是个包打听,耳目最是灵敏,“王舜臣他先赢后输,蚀光了老本,连借的钱都输光了。债主追到家里来了,把他老娘气得在家里大骂,说是没见过被人追债的官人。拿着门杠,把王舜臣打了一顿。他还不敢动,老老实实的站着挨打。”

除了高遵裕和苗授,现在古渭寨中官品最高的武官就是王舜臣,但他的年纪偏偏是最小的,在座的都知道他改了岁数,好早点入官。也因此,不少人都有三分妒嫉。听到他丢了脸面,兴趣盎然的不止一个。

杨英催问着傅勍:“最后是怎么处置的?”

“还是韩机宜的表弟冯从义帮忙还的债,听说是韩机宜的母亲让他把钱送去的。”

“王大跟韩机宜家关系倒真是不坏,几十贯的帐说帮忙就帮忙了。”

“那是过命的交情啊!”

下面在说球赛,高座在上的王韶和高遵裕也在说着。

“这样下去不行啊……”王韶摇头对高遵裕道。

“停也不好停,张香儿的球队进了前四,回去就摆酒庆祝,还说下次要把头彩拿回去。据说包顺【俞龙珂】、包约【瞎药】那边,下一次比赛也都准备出人来参一脚。”

“不是说球赛,是赌赛。”王韶也听说了王舜臣的事,“王舜臣不自爱,过几日要好生教训。但眼下是哪家在做庄,都欺负到官人家头上了。官府的体面还要不要了?!”

杨英和傅勍正好一起上来给王韶敬酒。听着王韶的话,傅勍摇着头:“真不知道是谁领得头。”

而杨英仗着跟王韶是乡里的关系,插话道:“以下官愚见,不如干脆把庄家拿过来由衙门来坐,居中抽头也是好的。不是说京师中的桑家瓦子、刘家瓦子里的赌赛,都有开封府抽头吗?”

“胡说八道!”高遵裕笑骂道,“哪会有这等事,嫌御史太闲了吗?都是下面的胥吏主持的,衙门里睁一只眼闭一只眼罢了。”

“……其实这样也不错。”杨英笑眯眯的建议着,询问的目光向王韶看过去。

“你们商量着来好了。”王韶站起来,横了杨英、傅勍一眼,跟高遵裕推说身子乏了,就一拂袖子,径自转进来后堂去。

王韶方才就有些火气,现在又突然走了,听口气不太妙的样子。杨英、傅勍都是惶惶不安。老老实实向高遵裕敬过酒,抓来王厚问道:“安抚怎么了?怎么突然生气了?”

知父莫若子,王厚是王韶儿子,对其父的心思了如指掌,低声道:“还不罗兀城的消息闹的。我们在这里观灯谈球,说得都是赌博之事。横山那里却是战鼓隆隆,很快就要大战了。朝廷上什么都是紧着横山来,家严这些天,心里一直都有些烦……你们的心情真的有那么好?”

“说的也是。”傅勍也压低声音,“高安抚过年时去了秦州,前日回来时说,燕达领军去了水洛城,刘昌祚守着甘谷城,秦凤路给鄜延那里打下手,连郭太尉都是闷得发慌,天天在白虎节堂里对着沙盘打转。”

“这也没办法,谁让延州那里是宰相亲自领军……”杨英话出口就知道错了,连忙转过来:“现在韩机宜就在横山,当真是快活极了。”

王厚摇摇头:“你们不知道。韩玉昆接令也不情愿。谁让韩相公连着上了两本,指着要他去。他刚到京里,被王相公召去的时候,家严也在,韩玉昆是当着王相公的面说横山必败,还说如果一定要他去,日后就算横山报功,也别他的名字写进去。”

“韩机宜真是硬脾气。”傅勍咂了咂嘴,突然有些诡异的笑着,“听说韩机宜在京中跟一个花魁打得火热,还跟人争风吃醋起来,是不是有这回事?”

王厚摇摇头。李小六回来后,只跟家里面说了。王厚也是从冯从义那里听到一点:“玉昆是虎口夺食,直接抢了官家弟弟、雍王殿下看上的人。还让天子亲自下旨,把那花魁赐予了玉昆。想想这天下的选人,谁有这么大脸面,让天子送他姬妾?!可就玉昆一人!”

杨英、傅勍大惊小怪的叫起来,惹得周围官员都过来问着详情,关于韩冈在京中的丰功伟绩,扯起来,便是没了休止。

砰砰的几声响,几朵灿烂的烟花爆开在空中,与一轮明月互相辉映。通远军和平安定的熙宁四年上元夜,就在烟花中,继续和平安定下去。

……………………

邠宁广锐都虞侯吴逵从所在监牢尺许见方的窗口中,仰头望着天上一轮明月。噼噼啪啪的鞭炮声随风传来,吹进牢中,却让人心酸不已。

“吴都虞。”一个小心翼翼的声音在身后响起。

吴逵转过身来,脚下的铁链一阵沉闷的响声。守牢的孔目官张小乙正半躬着腰,站在他身后。一摞食盒就在张小乙脚边,带着好酒好肉送了上来。

看着张小乙忙着把酒菜给自己张罗上,吴逵谢了一声:“多谢张孔目。不如坐下来一起吃?”

“不敢,不敢,都虞请慢用,小人就在旁边侍候着。”张小乙点头哈腰,站在旁边连声说着。

吴逵就是吴逵,在环庆军中,名气不小,人望甚高。就算下了狱,也没谁敢招惹他。

关于这一点,张小乙再清楚不过。

半个月前,这庆州大狱中,尚有两个张孔目。他张小乙只是小张孔目,上面还有个积年的老张孔目。现在倒好,就他一个张孔目了。

‘那些赤佬也是能惹的?’

老张孔目也不是拿了不该拿的钱,仅是去讨要惯例的份子钱,不合顺口骂了两句贼配军。当天夜里,就被一刀子被捅在腰上,等天亮后,给收粪的粪头在昌平巷私窠子的后巷里发现时,尸首都冻得梆梆响了。

庆州城内谁他娘的不知道这是广锐军的赤佬干的,但有人敢捅出来吗?

现在大狱里就是把吴逵当祖宗奉着。

张小乙像个小厮一样垂着手站在一边,看着吴逵一手扯下一只熟鹅腿,大口啃着。

吴逵吃得肆心快意,张小乙心里直叫唤:‘押在邠州不好吗?转去延州也成啊!偏偏送来了庆州大狱中押着,不知道广锐军本有两个指挥在庆州吗,不知道邠州宁州的几个指挥的广锐军也给调到庆州来了吗?’

‘管庆州的王相公在衙门中喝酒,半个月不见人影,现在这些赤佬日他鸟的才是爷爷啊!’

张小乙满肚子的埋怨,也不敢说出来,侍候着吴逵扯着熟鹅,就着热酒吃饱喝足,端上了热水洗手,才弯着腰倒退了出去。

听着牢门挂锁的声音,吴逵又抬头从小窗中,望着天上满月

要定他罪的是韩相公,别看现在牢头把自己当爷爷侍奉着,但转过脸来,他怕就是一个刀下鬼了。

带着哗啦哗啦的脚镣声,吴逵慢慢移到窗边,双手攀着手腕粗细的木栏,贪婪的望着挂在天上的银盘。

‘到了明年,这上元夜的月色还能再看到吗?’

