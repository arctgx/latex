\section{第14章 霜蹄追风尝随骠(七)}

半年不见,王厚的形容有些憔悴,一应佩饰全都没有携带,连前年上京时天子特赐的金带,也换成了黑色的牛皮带。与意气风发、衣着鲜亮的王中正有着鲜明的对比。

看到王厚如此,王舜臣神色黯然。

当年提拔起自己,一同为收复河湟而努力的王机宜已经不在人世了。熙宁二年相遇,受其青目,韩冈、赵隆还有他王舜臣,一一被举荐入官。由此拉开拓边西北的序幕。十年之后,大宋甚至都已经收复了甘凉之地,玉门关内,皆为汉土。

这十年间,西夏日趋衰弱,直至今日灭亡,就是从河湟被大宋占据开始,可惜当年以微官上书天子的王韶,没有看到西夏亡国的这一天。

与赵隆一起站起身迎接两位高官,王舜臣低声叹了口气,“王枢密真是可惜了,再过些年都能当相公的。”

“谁说不是呢。”赵隆也陪着叹了口气、

其他王韶提拔起来的官员,或许只是为失去一个靠山而遗憾,但王舜臣和赵隆两人当年跟在王韶身边的时候,经常得到王韶教授兵书战策,以及史书上的战例,乃是有着师生之谊。

王舜臣和赵隆心情沉重,但王中正已经大笑的走过来。在军中久了,王中正一个阉人也有武将的性子,高声冲着王舜臣道:“看看,这不是我们的王破虏吗?!”

王舜臣向王中正躬身一礼:“末将本是罪臣,若非都知青眼,末将还是一介待罪之身,更不会有今天的光荣。”

王中正笑道:“还是王舜臣你本身能领军,换个人也难有这样的功绩。”

王厚也道:“汉唐开西域,自吐蕃破唐,占据甘凉之后,除了昙花一现的归义军,这还是汉人的军队第一次重临玉门关。”

王舜臣上前半步:“枢密的事,在下是在甘州时听说的。还请节哀顺变。”

王厚神色一黯:“多谢了。”

王中正叹道:“襄敏公英年早逝,乃是国家和朝廷的不幸。若有襄敏公坐镇朝中或是陕西,这一战绝不会如此艰难。”

王厚向王中正欠了欠身,“多谢都知之赞,王厚代先君愧受了。”

王中正摆了摆手,“朝堂诸公中,立有不世奇功者唯有襄敏公一人。这一次战局不顺,天子第一个想起来的也是襄敏公。什么样的赞语都是当得起的。”

赵隆恭谨的站在王舜臣身侧,一句话也不多说。王韶刚刚病逝,王厚论理是该回乡居丧三年。但他是边臣,正好又是战火正烈的时候,朝廷照惯例下文夺情。并依旧让他同时主政兰州和熙州,处置军政之事如平日。不过要不是因为王舜臣的缘故,丧父不久的王厚甚至不会参加酒宴。声色之事,更不方便在他的面前多提了,胡女、内媚的话,那是一个字都不能提的。

“好了,不说这些了。”王厚展颜笑道,“今天可是为王景圣你庆功……”说到这里,突然话声就停了下来。

王舜臣看得明白,叹息道:“我这个表字还是王枢密所起。”

王舜臣的表字是王韶帮忙起的,虽说不是多深的寓意,仅是跟着名字来的,但怎么说也是王韶的一番心意。

“好了,别让有功的将士们久等了。”见气氛又沉重起来,王中正望了一圈厅中的将领们,回头对王舜臣笑道,“一战收复甘凉,王景圣你这一次至少一个遥郡。”

王舜臣的心跳漏了一拍,然后又急速的跳动了起来,终于走到这一步了。

在出阵的时候,王舜臣因为犯法被责,夺取了所有的官职,乃是白身。不过打下凉州之后,他就官复原职了。现在变成了玉门关内、整个河西走廊的功绩,正如王中正所说,之后在本官升级的同时,至少一个加个遥郡的虚衔。

节度使、观察使、防御使、团练使、刺史,这五级正任官是所谓的贵官,位居正任官、横行官、诸司使、大使臣、小使臣总计五阶六十余级武将官阶的最高位。

不过贵官,极少授予实际在任上领军的武将。做到三衙管军这个军中最顶级的差遣任上的,通常也只是横行官。正任官全是给皇亲国戚,或是羁縻的边境部族族酋。比如吐蕃赞普董毡,他就是湟州防御使。交趾的李乾德,也有一个节度使的名头。

实际领兵的武将,郭逵至今依然是比节度使低半级的节度留后。剩下的半级,若无大功,很可能要等到死后追授上去。

不过在正任官之外,就有所谓的遥郡官。在低位官阶之外,加一个某某使的虚衔,作为对将领功绩的表章,并不用来计算品阶,也不是俸禄的依据,仅仅是好听而已。在有其他职阶时,同时拥有的使职就是遥郡,若是只有使职,那便是正任官。

高遵裕当年在熙河路立下功劳,从横行官加遥郡的西上阁门使、荣州刺史,直接变成单纯的岷州刺史,就是从遥郡官直接转为正任官。以功劳算绝对不够资格,除了光荣战死、殁于王事,也极少有这样的升迁,只能说是国戚的福利了。

遥郡虽是虚衔,有别于正任官,但也是非宿将、大功不授。不论是得到节度使到刺史的哪一级——其实高位的节度使、观察使和防御使基本上是拿不到的,只可能是团练使和刺史中的一级——日后想升做定额只有二十四人的横班,也就是横行官,都将更容易几分。

与王舜臣一同往席位上走,王中正漫不经心的问道:“听说这一次带回来了一匹天马……”

“啊……”王舜臣怔了一怔,点点头,“那一匹汗血马,乃是当年汉武帝索求不得,使李广利征大宛的天马,非人臣可用。下官得以领军,全是都知提拔,下官这一次回来,正想着请都知将这匹天马进献给天子……”

王中正脸色瞬息间变了一下,但立刻就恢复到正常的笑脸:“这是你的功劳,我不就夺人之美了”

“河东的事听说了吗?”王厚在旁突然说道。

王舜臣松了一口气,连忙问道:“河东那边又出什么事了?难道有三……韩龙图坐镇,还不能压住阵脚?!”

“想哪儿去了?”赵隆摇头解释:“辽人占了兴灵,灭了西夏。韩龙图为防河东被西贼夺取的丰州旧地被辽人占据,便起意收回丰州。”

王厚也道:“韩玉昆要趁势夺下旧丰州,拿瀚海大漠做边防。这是好事。若是做到了,银夏和河东也算是就此安稳了。不过这也是虎口夺食,冬天结束前,不见得能消停的。”

“不过既然河东有韩龙图在,不需要担心太多。”王中正坐在他的主位上,哈哈笑道,“今日当痛饮一番,一贺我王师收复甘凉旧疆,直抵玉门关,凯旋而归,并预祝韩龙图河东功成!”

……………………

韩冈还在麟州,从前线传回来的消息,正不断送到他的手中。

李宪还没到,尽管功赏已经送去晋宁军,但以万人计的步兵想出动,本来就是慢得跟女人化妆一般,没指望能快起来。不过他手上的骑兵,已经有两个指挥先期抵达,算是给韩冈支援。

“能不能派上用场?”韩冈问着刚刚从军营回来的折可适。论起对战马的认识,韩冈门下他是理所当然的第一。

折可适摇着头,“马匹情况都不好。骑乘的马不用提,战马的膘这个秋天也根本就没养起来。今年冬天如果再上阵个几次,肯定会大批倒毙。别说上阵,就是多出去巡逻几次,也吃不消。”

夏天、秋天是战马养膘的时候,这样马匹才能熬过酷寒的冬天。可今天的夏、秋两季,是纯粹的消耗,正常情况的冬天也熬不过去,更不用说要进入战争时期的情况了。而且被先派来的肯定是李宪手中骑兵情况最好的,他们都如此,其他骑兵的情况可想而知。

“之前派去银夏的时候,他们完全是被当成巡卒在使唤,消耗得太大了。”

听了折可适的话,韩冈脸色难看的啧了一下嘴。派骑兵去鄜延路,韩冈不认为自己做得有错,但种谔那边事情做得有些过分了,不是自家的兵,就可了劲的使唤。

韩冈没打算与辽人拼骑兵,真要斗起来,肯定还是以步兵为主。但骑兵连用都不能用,那还真是让人头疼了。

“昨天丰州就已经来报,柳发川和暖泉峰已经出现了辽兵。虽然只是十几骑而已,不过应当是斥候。再过几天,辽人的大军就该出现了。”

“你再去一封信,跟折府州说,若是辽人他们针锋相对,一步也不要退让。记住了,只要战场上不退让,这片土地就占定了!大仗是打不起来的。”

“明白。”折可适忙点头,这还是韩冈一直都坚持的态度。

说是这么说,但韩冈心中却还是忧急不已,心知要尽快将几座关键位置上的城寨修起来,辽人的动作可能会比想象的要快得多,慢了保不准要出什么意外。说实话,他可不想把希望全数寄托在耶律乙辛的政治头脑上!

