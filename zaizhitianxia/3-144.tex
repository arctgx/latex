\section{第38章 心贼何可敌(下)}

沈括一去契丹,没有三四个月回不来,而河东那边,还是继续在谈判。

赵顼咨询元老重臣们的意见,可除了一个支持新法的曾公亮有一说一,说着若是开战之后,如何抵御契丹入侵;韩琦、富弼、文彦博、张方平等人无不是将天子的咨询,当成是攻击新法的机会。

几个老狐狸没有一个明说要弃土,但话里话外都说着契丹兵强马壮,以如今河北饥荒未息,‘若兵连未解,物力殚屈,即金汤不守’。

而王安石却还是拼了命的为赵顼壮着胆子。说‘契丹四分五裂之国,岂能大举以为我害?’,只是其‘方未欲举动,故且当保和尔。’

韩冈从王雱的来信中,听说他岳父仍然不肯放弃,则只能摇头叹息。

天子对契丹的恐惧已经近乎偏执了,王安石要是能说服他,早就说服了,何须等到今日?而能给天子壮胆的那几位,却又都是揣着明白装糊涂,拆台的本事更大一点。

此事想着心烦,韩冈就只专注于他的工作。

尽管与京城只有一百多里,但韩冈在白马一年来,进京的次数屈指可数。做知县时,那是照规矩,州县官不得妄出所辖之地。可到了做府界提点后,还是没有时间多入京城。为了流民安置的任务,他在开封府各县跑来跑去。二十多个县,韩冈全都走遍了,多达八成的乡镇,他也至少去过一次。一个月最多也就在月底时进京面圣一趟,汇报一下工作。

能将几十万流民顺利的安置下来,并且不让他们扰乱地方秩序,决不是坐在衙门里吩咐一下就能轻松解决。也许有人有那个本事,但韩冈的办法就是多走多看。

在京城中,多少只眼睛在阴暗处盯着,一点小乱子就能给放大个十倍二十倍。他可没有富弼在青州时那般的威信,言出不移的权威只处在流民营中。传达到下面去的命令,各县能遵循一半就很了不得了,许多时候,他都只能亲历亲为,盯着看着。

不过随着流民逐渐北返,韩冈现在需要放在流民上的精力越来越少。九月下旬,他移文京府诸县,命他们重新普查在京流民人数。几天之后,王旁将各县的回报汇总,送到了他的手边。

最多时曾经达到五十六万的河北流民,如今只剩六万五千四百一十六人,基本上都是在家乡已经没有土地、没有佃田,不需要急着回乡播种。

其中还是白马县为多,有三万两千余人;其下分别是韦城、胙城两县,旧滑州三县的流民占了总数的八成以上。而其余各县,流民人数超过千人的,只有六个县,剩下的都是三百五百,不足以为患。

九月底的时候,韩冈就带着这个好消息,再一次进了东京城。

上殿奏对,当韩冈言及流民渐退,京府流民只剩六万余人的时候,赵顼也是大喜,连声赞着韩冈公忠体国。只是一番奏对,全都围绕着流民问题,赵顼半句也没问韩冈对于契丹人

韩冈也明白,是他前两次奏对时,给天子留下了强硬派的印象,所以才没有被问。不过韩冈也没心思计较,他就算为此苦口婆心,在天子心中还不见得能落个好,干脆不提。

出了崇政殿,韩冈便往,只是经过中书门前的时候,一个熟悉的声音在背后叫道:“玉昆!”

韩冈回头,竟是久违的章惇。

“原来是子厚兄,好久不见!”

章惇大踏步的走过来,韩冈连忙行礼,脸上笑容,比起前日见到沈括时要真诚得多。

章惇在荆湖数载,将后世的湘南、湘西的数州之地尽数改土归流,设郡置县,一边招募汉人屯田,一边引诱山蛮出山定居。户口总计增加了近十万,使得朝廷对荆湖南路的控制了大为增强。

而韩冈的表兄李信在章惇麾下也大放异彩,李家嫡传的掷矛之术名震,如今已是镇守荆湖南路的兵马都监。因为李信的关系,韩冈与章惇之间的政治同盟越发的紧密,信函往来一直都很密切。

章惇吩咐了身边的伴当一声,让他去中书告假,就与韩冈一起出了皇城,到了州桥边的周家园子找了个僻静的厢房坐下来说话。

等着店中的小二奉茶奉酒上菜之后,章惇一边给韩冈倒酒,一边就责备着:“上个月愚兄就回了京师,想去拜访,你又在白马县那边忙着。上个月月底,听说你回京入觐,愚兄就在樊楼定了酒席,可是左等右等,就不见玉昆你上门来。未免太生分了一点。”

“子厚兄勿怪。”韩冈连连拱手道歉:“小弟是见子厚兄当时正在审着市易务一案,御史天天盯着,不敢上门打扰。”

“在外面让人通传一句,愚兄还能就出来了?难道天子会以为玉昆你来帮曾子宣关说不成?!”

“总不能留人口实。”韩冈辩解了一句,又笑道:“是小弟的错,权且自罚三杯,还望子厚兄见谅。”

在旱灾遍及中原,天子朝堂为此殚精竭虑的时候,市易务一案却并没有停止。只是案子的重心,逐渐转到了曾布是否欺君的事上。八月的时候,章惇一从荆湖回来,就被天子任命为市易司违法事的主审,并让他来根究曾布、吕惠卿何人所言为实。

章惇与吕惠卿关系不恶,当年将他荐到王安石面前的,就有吕惠卿一个。

章惇年轻时犯了不少事,道德名声不算好。当有人举荐章惇时,王安石本不想见他,是吕惠卿帮着说了一句话,让王安石接见了章惇。见面之后,章惇的才能轻而易举的就打动了王安石,就此成为新党的核心成员。而章惇与曾布的交情就不怎么样了,表面和气而已。

故而在章惇的主审下,曾布被贬去江西饶州。而为了平复士林异论,成了祸乱之源的吕嘉问也被请出了京城,去了常州担任知州。

章惇本也是开玩笑,韩冈要自罚,他也就陪着喝了三杯。放下杯子,他正容道:“还要多谢玉昆,今年遣了一批流民往荆湖屯田,帮了愚兄的大忙。”

韩冈摇了摇头:“当时愿意去荆湖的也就是两千多人而已,对子厚兄可是杯水车薪,不值一提……”他说到这里,忽然心中灵光一闪,反过来问道:“子厚兄,你该不会是盯上了剩下的那几万流民吧?”

章惇哈哈大笑:“故所愿也,不敢请耳。”

韩冈则叹道:“熙河路也缺人啊!”

关于剩下的这几万流民如何处置,韩冈有自己想法。都是没有土地束缚的流民,以充实边疆那是最好。本想再等一等,等到十一月的时候,就可以确定剩下的流民无意返乡,那时候再行招募,当能顺利一点。

章惇眯起了眼睛:“听说洮河秋天的时候暴雨成灾,不知有没有大碍。”

韩冈道:“子厚兄你月来在中书,怎么会不知?只是洮河发水,不是渭河,隔着一重分水岭,受灾的多是蕃人,巩州那边可是大丰收。”

洮河在八月的时候发了一次洪水,规模不小,从家中来信,还有朝廷传出来的消息,都说已经闹到了要朝廷救灾赈济的地步。以旧古渭寨,也就是现在的陇西城为中心的巩州,位于渭水之滨。隔着一重髙山的洮河洪水,与巩州毫无关系,棉粮双丰收。

另外洮州的汉人其实也没有怎么受灾,当是旧麦已收、新麦未种,而棉田也收获了,只是毁了些种了白菜、韭菜的菜田,人都事先躲到了附近的寨堡中。但吐蕃人就损失惨重了。宋人在洮州的屯垦区域,如今还是主要分布于狄道城周围,至于其余河谷地带,都是吐蕃部族占据,蓄养牛马牲畜,洪水一来,人跑得了,多少牲畜来不及跑,被冲走了无数。

“如今熙河路的汉人户口已经超过两万户,根基已稳,而荆湖南路诸州县则是新辟之地,山蛮远比汉人要多……”

“趁热打铁不是更好,一场洪水,让熙州空了多少地方。”韩冈笑着反驳道,不肯答应。

“玉昆,总不能独吞吧?”章惇有些急了。

韩冈和章惇都是注重实际的官员,对他们起家之地始终放在心上。六万多河北流民,至少能拉出来三分之一,少说也有四千户。不论迁移到那一路,都是能立刻将一个新辟的州郡安定下来。以两人的性格,当然不可能放过。

韩冈呵呵的笑了笑,退让了一步,“其实流民愿不愿意迁移还是两说,须得由他们自愿,强迫不来,否则御史也不会干看着。到时候,将选择交给他们自己。”

有了韩冈这句,章惇就放下心来,他也清楚,以自己和韩冈的关系,韩冈不会反口。到时候,流民们是去荆湖还是去熙河,就看各人的本事了。

将事情敲定,章惇便与韩冈痛饮起来,只是喝到一半,章惇的一名伴当匆匆赶来,附在章惇耳边说了两句,就见他的脸色顿时变了。

韩冈放下酒杯,沉声问着,“出了何事?”

章惇沉着脸,一个字一个字的从牙缝中挤出话来:“‘敌理屈則忿,卿姑如所欲与之。’”

“这是在说什么?”

章惇怒火阴燃的双眼盯着说了胡话的韩冈,“你说呢?”

