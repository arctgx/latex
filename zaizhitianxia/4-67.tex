\section{第14章 飞度关山望云箔(六)}

‘小人既然已经归顺朝廷,正要赎了过往的罪孽,哪里敢不卖力?’

苏子元在心里重复了一遍。黄金满所说的官话带着浓重的口音,不过苏子元听着一点也不觉得吃力。广源蛮帅的话落在他的耳中,言下之意就是‘投名状小人不会写,但小人知道怎么交。’

只是做了决定之后,竟然如此毅然决然的反戈一击。苏子元算是以切身体会明白了,就算是蛮夷,也并非都是如同刘永一般的废物。

看了看带着赔着笑脸的老蛮帅,此人行事如此果断,如果让他做大,日后说不定就是一个不逊交趾的大患了。不过那也是是日后的事了,眼前的当务之急还是邕州。

远眺着南方,邕州城尚远在地平线下,根本都看不见一点踪迹。“黎生领着一半人马逃了出去,不过六十里的路,想必李常杰今天就能收到官军占据昆仑关的消息。”

“当然,当然。”黄金满脸上堆满讨好的谄笑,“苏皇城已经在邕州守了快两个月,再守个几天也不会有问题。守住了邕州城,肯定是泼天一般的功劳,到时候就能入朝做相公了。”

苏子元越发的佩服起韩冈对蛮人的了解。他让何缮转达的话很直白,没有一点客气。要黄金满自己挑选做哪家的狗,这根本就不是说词,而是过时不候的最后通牒。蛮夷畏威而不怀德,越是强硬,他们就越是听话。黄金满的反应,是对韩冈这番话最好的印证。

韩冈的话的确有效。在苏子元的经验中,如果是跟蛮人谈判,决不能将自己的底线和内情透露出来。但韩冈不但明说自己只有八百兵,还说了自己没把握打下邕州城。但他敢于这么让何缮转述出去,因为他的善意只在邕州城破之前。只要邕州城一落,那就再无所求,黄金满就算站着昆仑关,也没有什么可以拿捏的了。

这是底气的问题,还有对于敌手心思的把握。难怪他年纪轻轻就有那么大的名气,官位甚至压了自己做了几十年官的父亲一头。看他自桂州领八百军南下的决断;在奔波数日后,面对刘永贼军敢于一战的武勇;能看破贼军伎俩的眼力;以及算计人心的才智,苏子元都不得不为之叹服。

长山驿的胜利,可以说是是韩冈一手操纵出来的。随同韩冈南下的时候,苏子元他只是想离父亲更近一点。都没想过凭借着区区八百兵,就能打到邕州——过了昆仑关,就已经是邕州的地界。

到了午后的时候,派出去追击敌军的两百多名广源战士挑着头颅、衣甲,高歌而回。黄金满又遣了儿子黄元率部进驻更南面的金城驿,自己则与苏子元一起返回昆仑关。方才他们得到后方的通报,韩冈这时已经抵达关城中。

金城驿就是在连接邕州、宾州的官道出山的山口上。离着当年狄青尽歼侬智高主力的归仁铺只有二十里。而归仁铺距离邕州城更是只有二十里不到。

尽管苏子元想更进一步的杀到邕州城下,让守城的官军,让父母,让他的兄弟妻儿都知道,救援邕州的官军已经来了,他苏子元也回来了。但他很明白,要想为邕州解围,接下来并不是动刀兵,是要让李常杰惊惧,让围城的交趾兵闻风丧胆,而不是将自己的虚实透露出来。

韩冈和李信在看见何缮领着黄金满派来做人质的儿子黄全之后,就立刻率部动身,从宾州前往昆仑关。

韩冈不怀疑黄金满的诚意。他在交趾人那里能得到什么,在大宋这边又能得到什么。这种最为简单的算术题,就算小孩子都能算得明白。这可不是一和二的区别,而是一与一百的差距。

但李信为防万一,还是先派了一个指挥去接手关防,然后才与韩冈一起进入关城,为此耽搁了一个多时辰。而当他们走进关城的时候,就立刻收到了长山驿大捷的喜报。

黄金满知情识趣,送上的一份大礼让韩冈和李信喜出望外。想不到就在一夜之间,不但昆仑关拿到了手,就连关城南面长山驿的一千交趾兵都不再成为阻碍,李常杰放在邕州北面的防线已不复存在。

并不是哪个倒戈的将领都有这等眼色和胆魄,黄金满作出的表率,让李常杰不可能再信任身边的广源蛮军,一旦两边失和,上下其手的机会可就多了。

“当年班定远身边也只有三十六骑,而我们身边可是有八百精锐。”李信很是兴奋,就连话也变得多了一些。要是能凭着八百人就做出一番事业,将十万贼军惊得狼狈而逃,那可是武将一生的荣耀。

“没有那么简单。”站在昆仑关这座千古名关的关城之上,韩冈望着北面山外的平原,那就是后世地理学上的的南宁盆地。从桂州到邕州,千里之地,就只差了数十里,“要走完最后的几十里,要比之前的九百里要难得多。”

李信沉默了下去,的确,那边可是有着十万大军,再想用着三寸不烂之舌来说降、或是借力打力,难度比起之前都要高了千百倍。

“还有邕州。”从昆仑关返回的何缮嘴里,听到了邕州城最新的战况,韩冈的心中其实只是抱着最后一线希望,“交趾兵前日已经上城,邕州的城防已经毁了,想要阻敌,就只能依靠巷战。可观看过往战例,试问城破之后,守军又能守多久?现在又过了一天,邕州又还能支撑多久?”

李信更加沉默。他的表弟说得没错,邕州城才是最值得担心的。最算换做他来守邕州,在城墙失去作用、城防溃败的情况下,就算仅仅抵抗一天都是难如登天,关键是城内的军心不可能在支持下去。现在就只能祈祷了,让苏缄得以稳定邕州的军心。

红霞满天的时候,苏子元和黄金满都回来了。

城头上飘扬着大宋军旗,让黄金满离着关城百步就下马步行。韩冈则没有慢待这位功臣,他亲自出关迎接。

只有二十多岁的韩冈,还有同样年轻的李信,让黄金满惊讶无比。不过看到苏缄的儿子苏子元对韩冈的尊敬,还有韩冈表现出来的久居上位的气派,他便又为之释然。这位广西转运副使——民间的俗称中都是转运相公——的身份当不会有假。

“李常杰待人刻薄,吝啬无比。”黄金满在向韩冈解释着自己的背叛都是李常杰的错,他知道没人喜欢背叛者,“本来邕州城当是有我等广源军来攻打,可李常杰在昆仑关侥幸胜了官军之后,得人献了云梯车和攻濠洞子。自以为能破城了,就把小人给发遣来镇守昆仑关。”

“朝廷里倒不会有这些事。就算朝廷吝啬,你们能得到的,也会比交趾大方时更多。更不用说,当今天子一向慷慨。这一次交趾入寇广西,天子震怒非常,誓要将李乾德母子捉回汴京城。等拿下交趾之后,也不会有什么大越国了。到时候你们究竟是继续在广源州那片山里生活,还是占了升龙府附近的良田美地,就看你们用不用心了。”

韩冈的承诺让黄金满等一众广源洞主大喜过望。同时黄金满心中也有了几分惊惧,朝廷派来的官员对于交趾的地理,竟然了解得不少。至少知道广源州是山多,而升龙府附近多为平原。

“对了,这一次还抓了几个俘虏,当献俘于运使。”黄金满讨好的说着。

说是几个俘虏,还当真就只有几个。韩冈在宾州,就留了何缮一个活口,而黄金满也仅留了三四个。除了阮平忠这条大鱼,就只有三名剃着光头,穿着僧衣的和尚走进韩冈的视线。

“他们是怎么回事?”韩冈讶异无比,“交趾人上阵还带着和尚?”这是随军牧师的前身?没听过交趾有这个习惯。不嫌晦气吗,若是战死了,正好就可以及时超度?

“运使有所不知。”黄金满知道韩冈是误会了,解释道,“他们其实都是交趾人的奸细。李太……李常杰他打下钦州廉州之后,得到了不少度牒。就按照度牒上的相貌年甲选人,让他们换了僧衣打探军情。这几个都是昨夜带着天军至宾州,还有刘永授首的消息过关来。要去通报……”

黄金满正在为韩冈解释着,就发现年轻的韩运使的脸色一点点的阴郁起来,心中一凛,话声顿时停了。

“速传信回宾州!”韩冈脸色阴沉下来,狠狠的瞪了何缮一眼,“并通知广西各州县……”

南方信佛者极多,富户常常买了度牒,剃度几个僧尼,作为自家子女的替身。而为了免去经常去衙门开具过所的麻烦,许多行商也多有购买度牒傍身。想不到李常杰竟然知道要钻这个空子!就算是战时,只要手持度牒,出入城防,穿越关所,也照样不会有人在意!

何缮面色如土,他竟然忘了把这么重要的事给说出来。

“凡持度牒出入城关者,一律下狱严审,不可走漏一个!”

