\section{第12章 恶客临门不待邀(中)}

析津府又开始下雪了。

燕山山中的道路,已经被积雪填满。而太行山中井陉、军都陉等要道,也有了一尺深的积雪。幸好大辽的战士们不畏严寒和积雪,南京道与中京道、西京道的联络,依然保持着畅通。

耶律乙辛所居的院落中,积雪已经被清除的一干二净。为了防止地面上结冰,甚是还撒上了盐。不过随着又是一场降雪,地面上再一次泛起了白色。

雪片从铅灰色的云层中翩然而落,来来往往的文武官员和吏员在经过院中的时候都加快了脚步。

冬天终于是到了。

隔着一堵墙,屋中却温暖如春,耶律乙辛双手握着茶盏,悠悠然的嗅着袅袅茶香。茶盏中,茶汤匀白,浮沫细细,紧贴着杯壁。

耶律乙辛惊喜道:“竟又咬盏了!”抬眼对面,对拿着茶杓,点汤击拂,一遍念着赞辞的张孝杰笑道,“说道分茶之技,这朝中,当属你张三第一。就是到了南朝,当是也能有一席之地了。”

张孝杰虚虚拱了一手:“尚父谬赞了,下官愧不敢当。听说南朝贩夫走卒都爱分茶斗茶,下官的手艺也只是寻常。”

“太自谦了也不好。”耶律乙辛抿了一口龙团煮出的茶汤,微微皱了下眉。

说实话,看着张孝杰分茶的确有趣,茶杓一击一拂,朦胧间水脉浮动,汤面上就能幻化出各色朦胧的画面出来,或是疏星朗月,或是花鸟虫蛇,宛如雾幻,配合着张孝杰念得诗句,便是雅致二字的极致。

只是满是龙脑香的茶水,还是喝得不习惯。倒不如从少年时就喝惯的砖茶,研末后与奶和盐掺在一起煮,那个才叫至味。

放下茶盏,耶律乙辛看了看紧闭的窗户:“今年的冬捺钵肯定是没办法去了。”

心思剔透的张孝杰同样放下茶杓,带着闲适的笑容回道:“在析津府中过冬,其实也不错。冬天的景致,往年也见过几次。”

耶律乙辛曲指轻叩薄胎瓷盏,名窑所出,清脆如击金石:“希望能在过年前把所有的事都做个了结。至少春捺钵不能耽搁了,鸭子河的头鱼宴不去一趟,女直那边又该有人心不稳了。”

“有尚父的神机妙算在,肯定能如愿以偿。”

“哪里来的神机妙算,三分在人,七分在天啊!”耶律乙辛微微一笑,又琢磨道,“石柳差不多该到兴灵了吧?”

“计算时日,差不多也就在这几天。”张孝杰的笑容讨喜得很,“还是得说尚父是神机妙算,当梁氏兄妹听到耶律都统领军攻入境内、直取兴庆府的时候,他们的脸色想必会很好看。”

耶律乙辛叹了一声:“要是党项那边争气点,我也用不着翻脸不认人,好歹留份人情。”

大辽尚父叹着气,但却掩不住眼中的得意。趁着党项人倾巢而出,举兵南下直取兴庆府,顺道将阴山以南、水草丰美的河间地【河套后套,今五原】收归囊中,这必定是个辉煌的胜利。

自从控制了辽国的权柄之后,耶律乙辛虽然威福自用,但他的地位并不稳固,迫切需要证明自己的能力。尤其是在宋人开始攻打名义上的属国西夏之后,更是需要向国中各个虎视眈眈的贵胄们给一个交代。

岁币也好、土地也好,总得让宋人挤一点出来,全了大辽的面子,给一个台阶下。但宋人既然咬紧牙关,死活不给脸面,那当然就得别出蹊径。不能从宋人那里拿到的,就从党项人那里拿好了。

贺兰山东的丰美土地,契丹人垂涎已久,却始终没能得到。有了兴灵那块肥肉,国中的贵胄们各自都能在党项人身上分一杯羹,自然能换来不少的支持。再加上阴山下的那片草原,只要拿出来,将更是人人趋之若鹜——西夏的黑山威福军司位于又名黑山的阴山之南,横跨黄河两岸,土地肥美,是最上等的牧马地。

耶律乙辛做出决断的时候,梁氏兄妹刚刚领军南下。盐州之役还没有开始,胜负尚未可知。但耶律乙辛一贯都是与其将希望放在别人身上,还不如靠自己。契丹铁骑永远都比铁鹞子更让耶律乙辛等辽国重臣具有信心。

辽国对西夏的支持,是建立在党项人每年进贡的马驼等牲畜上的。这一点,耶律乙辛也无法改变——如果他做了白功,国中的发对派就会乘机兴风作浪,耶律乙辛不会为了党项人将自己陷入险地。而宋人是不需要盘剥党项人的,相反,还能带给他们足够的利益。

谁也不能保证党项人不会在国中无法支撑下去的时候投向宋人。甚至可以确定,他们最终肯定会投靠宋人,至少会在宋辽两国之间争取平衡,由此设法减少每年的贡品。既然如此,还是干脆了当的将兴灵占下来,灭了西夏,免了许多的麻烦事。

“占据兴灵,是兴宗皇帝当年没有能做到的功业。尚父之功,当能直追太祖。”

耶律乙辛没有训斥张孝杰这一不恰当的比喻,仅是哈哈笑了两声:“也是因人成事,没有宋人,想攻打兴庆府,可没那么容易。”

张孝杰陪着笑:“苦恨年年压金线,为他人作嫁衣裳。早知有今日,就该争一下使宋的差事。能在文德殿上看到南朝皇帝的气其败坏或是有苦难言,当是平生乐事。”

耶律乙辛眯起眼,遥想着那个场景,忍不住心中的得意,“不过之前不去取,也是怕党项投向宋人。现在是最佳的时机,有点耐心,还是能等到好机会的。谁说守株待兔不是好办法?!”

看到耶律乙辛心情好,张孝杰更加奉承:“说来还是党项人给尚父玩弄于股掌之上,心神都放在宋人那里,根本都不在北方多加提防。”

耶律乙辛摇摇头:“还是提防了。黑山威福军司三万多兵马,党项人自始至终都没有动过。可惜啊,黑山的那些党项部族,知道认谁做主子比较好。石柳通过黑山的时候,当是不会受到干扰。”他沉吟一下,“不过这么说起来,梁氏和梁乙埋他们也还是大意了,没有防备两万多兵马会决定转投大辽。”

当真以为大辽会全心全意的支援党项人……这不是笑话吗?当年兴宗皇帝和元昊结下的仇怨,才三十年过去,怎么可能那么容易就消解。支持西夏,那是因为有共同的敌人。既然西夏在抵抗宋人的入侵时,损失太大,几乎耗尽了国力,就算有大辽支持也很难再支撑下去,那就干脆占下来。这样总比给宋人占了便宜去要好。

耶律乙辛道:“拿兴灵之地,正好可以酬劳一下奚六部大王回离不之前的支持之功,顺便再在五院部、六院部和国舅诸帐中挑些人过去。有个五万兵马差不多就够了,可以让党项人各安其所。”

“还有黑山的河间地。”张孝杰提醒道。

“至于黑山威福军司的那一片河间地……西京道的皮室军可以迁一部分过去。孤的斡鲁朵也得了天子圣谕,可以设立了,位置就定在那里。”耶律乙辛并不打算将那块地给其他人。

辽国的权臣是可以建立自己的宫卫,比如圣宗皇帝以父事之的韩德让,就有他的斡鲁朵,名为文忠王府。而每个斡鲁朵,既是宿卫的编制,但也是地方区划,下辖州县,设官署,有子民。

耶律乙辛之前刚刚让小皇帝签署了诏令,让他设立一个斡鲁朵,“正户【契丹户】一万户,汉番转户两万三千户,丁口十万。要安置下这么些丁口,土地就不能小。可惜好地方都有人了,原本是准备在东京道占上一块地。不过决定灭掉西夏之后,孤就看中了阴山下的河间地。虽然小了点,但胜在土地肥美,水草丰茂。”

“尚父说得正是。”张孝杰猛点头附和,那一片可是最上等的牧场,与蓟北之野不相上下。耶律乙辛将自己的斡鲁朵设在此处,实力会越来越强。

“若是攻不下兴灵也就罢了,若是攻下了兴灵,过两日,等消息传回来,就传令给萧禧。让他跟宋国皇帝说,西夏之地,两家均分,以瀚海为界!”耶律乙辛有一言定河山的畅快。

“有尚父谋划,兴庆府和灵州肯定是能打下来的。”

耶律乙辛不认为自己的计划会失败。西夏的北方防线,已经因为国势衰弱而被大辽渗透得千疮百孔。统军使耶律石柳所率领了两万契丹铁骑得以直扑兴庆府。兴灵的主力已经被带去攻打盐州,如何能抵挡大辽兵锋?!

由于宋人只攻到灵州城下,兴灵之地眼下的情况要比起银夏肯定要好得多,耶律乙辛向去过西夏的人征询过,灵州之南只占整片平原的一小部分。剩下的土地还有些存底。而且少了一个小朝廷的消耗,结余下来的收获,足够养人了。

“当兴灵成了大辽的疆土,看看宋人到底敢不敢动手来抢!”耶律乙辛绝不相信他们能有那份胆量。

