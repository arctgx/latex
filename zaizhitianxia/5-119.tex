\section{第14章 霜蹄追风尝随骠(三)}

两个月前,章惇重新回到了枢密院中,担任枢密副使的职务。这是战争给他带来的红利。

前方的战争,需要一个稳定的朝堂。但同时,也需要有经指挥过千军万马的统帅、精通兵法的文臣,坐镇在枢密院中。

章惇的机会原本是王韶的,毕竟章惇本人的领军经历全都在南方。不是在荆南、就是在广西。而王韶在西北威名赫赫,又深悉当地局势和一切情弊,远比章惇更为适合在枢密院中辅佐天子。

可惜王韶病死的不是时候。若其不死,枢密使都有机会,有军功又是进士,声望又高,而且还有长达五年的枢密副使的经验。令人遗憾的,他走得太早,否则吕公著都很难再安坐在西府之中。

除了章惇之外,枢密院中的人事并没有发生变动,政事堂中的成员,这几个月来,也没有什么改变。不过等到战争结束,两府之中当都有会有一个大的变动。政事堂和枢密院,不知要换上几张新面孔。

韩冈镇守在河东,稳定了北方局势。又有葭芦川的大捷,加上他之前的功绩早有有进入两府的资格,呼声当是最高的。若再给他收复了丰州,不入西府就说不过去了。

另外还有一个可能的人选——种谔。在立下了赫赫战功之后,种谔有可能跻身枢密院,如同郭逵、狄青等人的例子。只是种谔这个人的性格不受天子喜爱,加上文臣中,对这个总是绕过枢密院撺掇着天子开战,多次侵夺枢密院职权的武将没有半分好感,他想要进西府,阻力很大。

不过这一切最终还是要看天子的想法,尤其是最高层的人事问题,猜测和预计,就跟州桥夜市上摆摊的算命的,永远都是不靠谱。

章惇今天不当值,但战争带来的一系列工作,却都不能耽搁到第二天。

河东那边就不必说了,战功的赏赐、还有收复丰州的钱粮调拨,都是必须在最短时间内解决的急务。银夏的战事结束了,种谔伸出来要钱要粮的手却没有收回去,而且还有许多降兵、伤兵要安置。韩冈和陕西的几位官员,之前各自上表,建议要重夺葫芦河口的应理城,以求利用应理城的地理位置,将秦凤、熙河乃至泾原路屏蔽在后方。这也是当务之急,必须尽快将具体的方略整理出来,署上枢密院的意见,呈交天子决断。

事情很多,等章惇处置完手边的公务,起身回家,都已经星月满天的时候了。

回到家中,早已过饭点,家里人都吃过了。章惇在宫中只吃了午饭,晚饭可还是空着肚子。

匆匆吃晚饭,章惇打算去考校一下两个儿子的功课。等他到了章持章援两人读书所用的书房,却发现两人面前的功课并没有什么进展。

他们两个正抓着一根圆筒状的东西,对着今日的夜空。

拿在儿子章持手中的东西,章惇认识,是将作监制造出来的千里镜。比几个月前的新货,又精良了几分。配合飞船使用,几十里外的贼人都别想偷袭大营。

之前显微镜传世,两个儿子都喜欢拿着看水、看土,看周边手上能派得上用场的东西。眼前的千里镜,其实也差不多。

听到章惇回来的动静,章持、章援都惊得差点跳起来。连忙将手上的东西呈给章惇。

千里镜并不是玩物,用在观察天象上,并不比浑仪浑象要差,或者说,用来配合浑仪其实很不错,翰林天文局这个观测天象,占候祸福的衙门,都上表说要用此来改造局中的旧式浑仪。

而另一个观天的衙门司天监,里面都是颟顸无用的老废官员,连编订的历法都出错,预测的日食也从没准时过。听说每次快要到了预测的日期,司天监里面都要给各处神佛上柱香,求着当天一定要是阴天。对于千里镜的出现,他们倒是一点也没有动作,依然固我,安居如常。

其实朝堂上也尽是些废物,空占着权位,让才智之士不得晋身之阶。

章持、章援看着自己的父亲拿着千里镜,脸色越发得阴沉,互相交换了一个眼神。心中皆暗道,是不是今天的朝堂上出了什么事,惹得自家的父亲要迁怒到他们头上。

章惇掂着黄铜为体的镜筒,“你二人拿着千里镜望天,可望出了祸福休咎?能不能给为父说一说。”

章惇的问题,似乎是在说反话,极力隐藏着心头的怒火。

章持干咳了两声:“祸福休咎,儿子还是不会。只是用来看一看月亮的盈亏变化。”

“还有银河。”章援提议道“用千里镜照过之后,全都是星星,根本不是什么天河。”

“哦?”章惇原本是想训话的,可这一下却被惊住了,忍不住拿起千里镜对着天穹上的繁星望过去。

初冬的望日,天无纤云,月白如昼。

横贯天际的银河,平常看过去时,就是一条雾气弥散的河流,但透过望远镜的镜片,章惇却惊讶的发现,那一丝丝云雾,已经被分解成密密麻麻的星子。想来就是因为星星太多太密,聚在一起,让人误以为是天上的江河。而惯常所见的星辰周围,又多了千万颗不知名的星星,只是太过暗弱,让会人忽略过去。

章惇在天文上见识不算多,没什么研究,不知道过去有没有人发现银河的真面目。不过古往今来描述银河的诗句,都脱不开银河、天河、星汉一类与河流关联的辞藻。

章惇一声慨叹,今日方知银河乃是何物。

他又记起章援的话,将千里镜挪移了一个角度,又对准月亮照过去。

月中阴影,世传是嫦娥的玉兔,吴刚的桂花,乃至有山寺月下寻桂子的诗句,但在千里镜中,桂花也好,兔子也好,从透明的镜片中瞧过去,全都消失不见,仅仅是大大小小、颜色略深的圆圈。

看来看去,章惇总觉得那些圆圈像是井口,又或是坑洞。在广西,大约邕州辖下的羁縻州武笼州【百色】再往北去的地方,章惇就听说过那里的山岭中有几十上百的圆形深坑,最大一个,深阔皆达百丈以上。或许月亮上的圆形阴影就是这样的深坑。

不知韩冈知不知道,以他对天文的了解,也许会知道其中的详情,章惇还记得当年曾经有一次与韩冈谈到天文。据韩冈所说,月亮、大地和太阳,皆是球形,是一个绕着一个,日月的等级,如同祖孙,并不是可以并称的。月亮的光是反射阳光而来,所以盈亏的变化,只是面对太阳和大地角度不同的结果。而所谓的日食和月食,则是三者互相掩映而产生。

韩冈的说法对章惇来说很是新奇,不过后来韩冈便绝口不谈星象天文的话题,章惇见状也就不再追问,反正他对天文星象也没什么兴趣。

放下手中的千里镜,章惇摇摇头。虽然自己对天文兴趣不大,可士大夫中,爱好天文地理的数不胜数。只是数算一关,便挡住了不知多少人的钻研天文的道路。而观测星象的器具,除了一对肉眼之外,无不是只有官府才能造的出来,浑仪、浑象岂是普通人能够打造?但一具能观星的千里镜可是方便得多。再过几年,千里镜多半就跟显微镜一样,在有些身家的士大夫家中普及。那时候,想必每天都会有成百上千的人跟今天的自己一样,望着夜空中的星和月。

只是对章惇来说,地上的事都还没有梳理清楚,那还有心去管天上的事。正如夫子曾言‘未能事人,焉能事鬼?’鬼神之事,与人无尤。天上的事就归于天好了,生活在地上,还是关注人事为上。

‘不过,暂时由着他们来吧。’章惇想着,将千里镜还给章持。

正对着两个战战兢兢的儿子,章惇道:“平日里,你们还是要把心放在功课上,进士才是一切的根基。不过张弛有道,一天到晚读书,也不一定能有多少进步。歇下来时,摆弄一下显微镜什么的倒也不妨事。”

章惇年轻的时候也浪荡过,甚至还偷过同族长辈家的小妾,但轮到自己的儿子身上,他可就是无论如何都不愿看到他们有样学样。平日里两个儿子读书也是辛苦了,拿着显微镜,或是研究一下格物之道放松一下,总比跟着那一干不成材的浮浪子弟,去秦楼楚馆消遣要强。虽说物理、算学、自然什么的都是五经之外的学术,可如今也算是正经的学问了。

章惇又指了指千里镜,正色肃容:“不过这千里镜还是先放一放,星象谶纬向来连在一起说的,私习天文如今虽然查禁得不如国初那般森严了,但真要入罪,最轻都是流放。天子近来御体欠佳,太皇太后沉疴难起,八皇子又夭折了,谶纬的嫌疑,能躲多远就躲多远。”

章持、章援闻言悚然,齐声道:“孩儿明白。”

“所谓法不责众。等过两年,千里镜跟显微镜一样流传开了,再随大流去研习也不迟。”章惇抬眼望着群星密布的夜空,“到时候,若是在这方面有什么不明白的地方,也可以去问韩玉昆,他藏着掖着的,还不知有多少。”

