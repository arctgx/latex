\section{第五章 冥冥冬云幸开霁(五)}

地面上的血迹已经被冻结。

不复一开始的鲜红,而是发黑发紫,深深的浸染到地面砖缝中。

韩绛扫了一眼,便跨了过去,站回他该站的位置。

那摊血迹的主人,不可能再回来了。

宰相班的位置上,现在只剩韩绛和王安石两人。而后面属于参政、枢密的地方,也少了三人——曾布、薛向,以及引兵镇守在宣德门处的郭逵。

仅仅是三个时辰而已。

位于群臣行列顶端的宰执班中,已有四人离开了殿上——三人将永远不会回来,而另一人,下一次再入朝的时候,将会比他原来的位置,更进上一步。

看着韩绛下首处的那个空当,纵然色泽黯淡了下去,却也依然让人怵目惊心。

不过还是有许多人感到安心,没有大搜宫中,也没有驱动兵马,而是选择重开朝会,这是太后与宰辅们发出的一个信号。

虽然李信和王厚已经拿着圣旨,被派出去接管城防,并包围参与叛乱的几位朝臣的宅邸,可朝廷的重心依然是在被中断的典礼上。

重新开始朝会,没有急着追究罪责,更是对惶恐不安的禁卫,以及与叛逆有关连的朝臣们一个安抚。

在营救出向太后与天子之后,禀报了当下宫中朝中的局面,韩冈便建议重开朝会,以安朝中及京中人心。

他的提议,立刻得到了包括太后与众宰执的赞同。

不过王安石建议前往垂拱殿或文德殿御朝,但为向太后拒绝,她要重回大庆殿。

向太后的要求极为坚决,王安石也找不到没有拒绝的理由。

踩在叛贼的尸骸上登上台陛,比任何盛大的仪式,更能证明朝廷的稳固,也更能让太后确认自己手中正紧紧握着权力。

似乎是不一样了。

王安石想着。

经此一变,向太后的表现突然间上了一个台阶。虽然十分正常,但感觉上一时间还是有些难以适应。

照进大庆殿门内的阳光开始偏移,但王安石还感觉不到饥饿。

叛乱。

平叛。

救出太后、皇帝。

在朝臣们的心中,这一段时间仿佛过了很久很久,但群臣重新集结在大庆殿时,其实也仅仅刚过了中午,刚到未时而已。

依然是幼年天子,以及屏风后听政的女性,只是人物不复早间,已变回了原来的两位。

钧容直在殿中奏响宫乐,编钟、玉罄,清脆悠扬,群臣在王中正的赞礼声中,向着天子和太后大礼参拜。

宋用臣和石得一,一个自尽,一个被砍成肉酱。

刘惟简则死了,因为被叛军围捕时反抗剧烈,头上挨了一刀,被救出来后不久便咽了气。大概是听到了太后与天子被救出,叛乱被平息,心中再没有了挂念的缘故。

宫中副都知以上的大貂珰一下少了三人,可以让向皇后信任的更少,只能拉来刚刚被营救出来的王中正。

有时候,运气真的很重要。而对王中正来说,就不是‘有时候’了。

王中正在变乱中没有受到折辱,当他知道宋用臣、石得一伙同蔡确发动叛乱之后,便认了命,即不对抗,但也不合作。

这样的态度从叛乱者的手中,保证了他的性命,也让他现在成了最受太后倚重的内侍,而不是像之前一样,号称宫中兵法第一,地位也最高,还执掌兵权,却不如宋用臣更得亲近。

对于身为天子家奴的内侍来说,来自天子或太后的亲近,比官位更重要。

宫中要大清洗。朝中也要大清洗。太后身边,也有了许多空缺要补充。

王中正贵为观察使,又掌握皇城兵权,这一回有失察之过,但也有不与贼人同流合污的气节。也许会因过错而降职,但来自太后的信任,却是万金难换。

不过王中正清楚,光靠太后的信任是不够的,在朝臣中,也必须有盟友才行。

至于人选,根本不必多想。

多年的交情,以及对对方为人的了解,让王中正只会选择目前并不在宰执班中的那一位。

韩冈在班列中间偏上的位置。

相对于过去都站在最前端的一年多,他现在的位置很靠后。前面还有诸殿阁的学士,与宰执班更是隔得很远。之前他为了接近蔡确,故意装出发怒,还走了许多步,才接近到台陛前。

不过他还站在这里的时候,也就只是今天一天了。

明日再入朝,必然就会回到他应该立足的位置上。

韩冈这一回,绝不会再谦让了。

只有身处宰执班中,才能更好的影响朝堂,才能更早的得到重要的情报。

如果自己没有退出来,好歹能知道蔡确打算废幼主、立新君,却劝说太后失败的消息。

可这一回,苏颂、章敦,这两位韩冈亲近的友人,也倚之为耳目之寄的友人,都没能够及时提供相关的情报。

苏颂对权力看得十分疏淡,加之新近上任不久,对朝堂中的消息并不灵通。这也是无可奈何。

可章敦这边,则是已经有了裂痕,所以反而没有通知。

不,情况远比裂痕更严重.

这不是因为分赃不均而分道扬镳。因利而分,也会因利而合。

可韩冈知道章敦的想法,这是理念之争。非关道统,却一样难以妥协。甚至比起学术上的争端,更为激烈。

有这样的争斗在,两人之间的交情,不知还能维持多久。

而且若自己再谦让,就未免太过虚伪,会联想起王莽的人也会越来越多。

立了这么大的功,就该理直气壮接受提拔和赏赐。

这一回,能够切实得到提拔和赏赐的人数也不多,韩冈就是其中之一,另有一位,则是赏赐必然重逾千金,但能不能得到提拔就得看他是否能够保住性命了。

韩冈起身时,貌似不经意望了殿门一眼,这时候,就只能期待张守约能够吉人天相了,撑过手术后的养病时间。

张守约的手术,以现在的外科学的水平,当然无法开胸治疗。几名御医讨论之后,便直接切开了背部创口的皮肉,将箭簇与箭杆分离,然后小心翼翼的将整支长箭拔了出来。

几乎不能算是手术,只是简单的清理包扎伤口。幸而拔出长箭的创口没有大出血,并没有伤到体内的重要器官。但以张守约的年纪,能不能撑过去,没人能够保证。

此时没有参与到叛乱中来的诸班及宽衣天武,已经全面控制了皇城。绝大多数皇城司的人马,全都被转移到东宫。

不管其中有多少冤枉的,但只要有百分之一的犯罪可能,就不能将他们宽纵起来.

这一点,就像是宰辅们对赵煦的态度。

韩冈希望赵煦能够一直在皇位上,只是他的希望,却难于变成现实。

对于宰辅们来说,他们为什么还要冒那样的风险?有那个必要?

就是可能性只有百分之一,但也不如完全没有的好。

如果是为私利而废天子,当然会被视为权奸。但世人皆曰可废,这就不关宰辅们的事了。

如果霍家没有在另立天子后,变得飞扬跋扈,甚至谋害了皇后许平君,一心念着微时故剑的汉宣帝,恐怕也不会不顾拥立之功。

韩冈等待着,看看宰辅们哪一个会出来对向皇后提议。

在赵煦面前,群臣不可能与太后商量是否要废立天子。

就算其中的大部分都有那份心,也打算那么做,也会另外找个时间,来与向太后讨论这份问题。

只是经过了蔡确之叛,如果有谁开口劝说废立之事,就等于将手上的本钱都推上了赌桌。

一旦太后拒绝,必然会被怀疑成蔡确第二,就不可能再留在朝堂上。

而向太后那边,当哪位宰辅提到行废立之事,也免不了会怀疑,他是否已经做好了比蔡确还要充分的准备。

双方各有顾虑,相互钳制。韩冈觉得短期内,是不可能有人能够放弃胆怯,选择面对。

要提议废去皇帝吗?

章敦心中纠结,他不想做出头鸟,可是在蔡确之后,已经找不得有人愿意去冒这个风险。除了选择自己去冒险,章敦根本就没有其他人选,就算有人选,也不适合去走其他道路的办法。

要是王安石能够率先提议就好了,王安石若能倒戈一击,便能化解皇太后的疑虑,更能让她安心下来。

可是王安石是绝不会这么做的。

他对赵煦的看重,并不因为他失去了经筵官的教职,而发生太多变化。这是移情,王安石对先帝的顾念,成了赵煦身上的护身符。

如果赵煦是无心向学的庸君,王安石对他的看重也会少许多,但现在的赵煦,除了意外弑父一条外,其他各方面,无不是最为出色的幼年天子。

这样的学生,哪一位老师不喜欢?王安石也不可能例外。

废去赵煦,只要王安石还在,就不可能成功。

可只要韩冈在,就算王安石不在,废立天子的谋划,也不可能成功。在韩冈没有改变他本人的想法的情况下,一切改变现状的打算都是痴心妄想。

还不是劝说太后的时候。

