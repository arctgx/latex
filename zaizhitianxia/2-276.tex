\section{第43章 长风绕城遥相对(下)}

“宋人还能坚持多久?”

仁多保忠似是在自言自语。他是今次奉命领军攻打临洮堡的将领,也是仁多家现任族长仁多零丁的侄儿,在家族中被视为仁多零丁之后,有望统领仁多家的呼声最高的人选之一。

禹臧温祓看了仁多家的第二代一眼,这句问话简直是个讽刺。

两人并辔停在官道边的一座小山上,在山脚下的一片空旷土地中,千百名宋夏两国的战士正呐喊着,厮杀着,鲜血染红了土黄色的地表。

横行在阵前的一名宋军猛将,此时正用一支支利箭将一名名西夏战士射落马下。刹那间射出的箭雨超过了一支十人队,而精准到完美的箭术所造成的损失,更是堪比一支百名弓箭手组成的队伍。

西夏军眼下强攻宋军的阵列,但因为这名猛将的存在,使得拥有两倍于敌军的兵力,依然无法对宋军的阵营造成丝毫威胁。

这样的局面下,仁多保忠的话,可以说是盼望,也可以说是诅咒,反正没有一丝现实。

心生不屑,禹臧温祓问道:“看箭术,那当是熙河路近年来声名鹊起的王舜臣。前几日都在他手上吃过亏了。今天有他押阵,还要再攻吗?”

“温祓你说呢?”听出了禹臧温祓的言下之意,仁多保忠反问着。

“我看今天还是算了吧,在对面山中还有青唐部的瞎药藏着。哦,对了,他现在变成了宋人养的吐蕃狗,该叫包约……包巡检了。”

仁多保忠撇了撇嘴,后半截话只有当面说才有意义,包约还不知躲在哪里藏着獠牙,这番话倒像是败犬的狂吠。

率领兰州军的是禹臧温祓——禹臧花麻的亲将,在禹臧家中也是地位不低。但也仅此而已,比起狡猾而又擅长审时度势的花麻,其实并不算差的禹臧温祓,就显得愚蠢了许多,还算是个好对付的。所以当禹臧花麻前几天离开的时候,仁多保忠还暗自庆幸了一阵。

可是现在,仁多保忠却不这么想了。

‘要是禹臧花麻在就好了。’

至少禹臧家现任族长的眼光比起身边的这一位来,要强出不少。

但禹藏花麻本人现在并不在临洮堡下,解决了景思立之后,他就立刻率军回返。兰州城的中心这些天有些乱,禹臧花麻不得不回去坐镇族中,省得不知不觉之间,就被人从族长的宝座给赶下来。

这两年,禹藏家多次出兵皆是无功而返。几次下来,多少年来的积蓄快要耗尽了。虽然半年来,禹臧花麻从原属于木征、瞎吴叱的部落中找回不少,但杯水车薪,赚到的还是没有用出去的多。

身为一族之长,不能给族人带来金银财帛,又不能带来安稳的生活,那他下台,自然是顺理成章、理所当然。禹臧花麻对兰州的统治如今陷入危机之中,也不是什么让人惊讶的一件事。

不论是仁多保忠,还是禹臧温祓,都看对方不顺眼,但合作还要继续。他们都想将临洮堡攻下来,只有开了城,他们此前的付出才能得到应有的回报。

“明着来是不成了,不如派人堵着路,我们回头尽量快一点将临洮堡攻下来。”禹臧温祓再一次提议着。

“派多少?我们两边加起来就一万一,现在伤亡都快有一千了。”

要对付宋军和青唐部蕃军的联手。派得人少,肯定会被他们毫不客气的一口吃掉。派得兵多了,又会减弱攻打临洮堡的力度。

这其实是兵力不足下的两难问题。

“那你说该怎么办?”

禹臧温祓和仁多保忠大眼瞪小眼,却都没有能解决问题的答案。

同样的对话这几天来在他们的口中,不知重复了多少次,就是始终没有商讨出一个结果。被这设寨道旁的宋军硬卡着喉咙,就算攻城,两人都觉得脖子后面的寒毛是竖着的。

两人不是没有想过干脆将临洮堡放到一边,先把宋人的援军给消灭掉。可不但城堡难攻,连小小的营寨也同样难攻。

营中的守将狡猾无比,夜袭、骚扰的战术从来不停。而正面交锋时,区区两千兵力所展现的实力,比起当日景思立身边的两千兵要强出许多。

而且还有青唐部的包约在山间敲着边鼓,像条毒蛇一般择人而噬。此外,临洮堡中的守军竟然敢于出击,昨日甚至害得仁多保忠火烧火燎的赶回去救火。

而今天的这一战是昨日之战的延续,现在看来,应该是没有什么机会了。

“粮草快不够了。”禹臧温祓忽然又叹了起来,“武胜军……宋人现在改名叫熙州了,这里的蕃部一个比一个穷。已经有两三天没有新的补给进营了。”

‘还不是你家族长的功劳!’仁多保忠腹诽着。原本西夏军出征宋国,其粮秣来源要么是靠着攻破宋军的寨堡,通过里面储藏的粮食来补给。要么就是依靠当地的各家蕃部来支持,不过之后要用战利品来回报。

可是现在,临洮堡打不下来,而周围的蕃部早就给禹臧家和青唐部给抢成了白地。眼下大军快要饿肚子的局面,根本是禹臧家造成的结果。

但是为了团结起见,仁多保忠明白有些话还真不能说。

仁多保忠需要一个胜利,有这个需求在,他就不能太过得罪身边的禹臧温祓。

他的叔叔处事一向公正,在仁多保忠和亲生儿子仁多楚清之间,并没有任何偏袒。现在族长之位的继承权,反倒是仁多保忠更为高涨。但如果不能将胜利带回去,他现在的支持率,当然不能保证在现在的位置上。

仁多家是西夏国中最为尊贵、势力也最为强盛的一个部族,仁多家的族长一职,不仅仅是代表着七八万丁口的部族,同时还代表着兴庆府中,仅次于寥寥数人的地位。

仁多保忠决不想放弃这个位置。

而另一边,禹臧温祓也不想多说什么了。

别看现在他们在临洮堡城下打得热火朝天,但实际上,他们不过是一支偏师而已。国中的主力,据禹臧温祓所知,眼下正在西寿保泰军司那一带集结。

虽然温祓并不清楚他们的目标是过柔狼山往秦凤路去,还是过兜岭往泾原路去。但在罗兀城受到了惨重损失的一年之后,国中终于又大举出动兵马,这其实是向国人发布一个的信号。国中已经重新振奋起来,要到宋人那边抢钱抢粮抢女人了。

西夏军势重振,但禹臧温祓现在正在考虑着要不要见好就收。

攻打临洮堡是禹藏花麻定下的计策,但并不是不可更改。要不是看着临洮堡城垣上有多处损伤,加上堡中主将景思立轻易的中伏败亡,温祓并不想,前些日子,他跟着禹臧花麻在攻打临洮堡时,没少吃姚麟的亏。多次攻城所得到的唯一收获,就是进一步确认了宋军在城池攻防战上远超四方蛮夷的实力。

禹臧家这两年来,对外的战事就从来没停过,族中上下都感觉已经快要耗不起了。禹臧温祓这段时间从他的族长那里听到的口气,也是不想再跟宋人拼下去了。并不是禹臧花麻不憎恨宋人,但实在跟他们拼不过、耗不过。

‘财大气粗就是好啊。’

禹臧温祓这么想着。大白高国论起人口来,还不到陕西的四分之一。而区区一个兰州,别说跟西夏本国比,就连木征的势力都比不上。木征没能耗过宋人,据说已经被撵到了露骨山对面。现在兰州想要跟宋人耗,不论是谁提出的这个主意,禹臧温祓都会一巴掌将他们给打醒。

战场上的宋军战鼓突然一声变调,原本结阵以箭雨阻敌的宋军随着鼓声散开了,在一瞬间,就由守势转为攻势。突如其来的反击,让正在战场上奋力进兵的铁鹞子和步跋子猝不及防。只进行了短短时间的抵抗,就全军溃散,败逃而回。

“不好!”禹臧温祓叫道。

“不用担心。”仁多保忠立刻安抚着,“宋人不会追击的,他们是要退军回营。”

正如仁多保忠所言,宋军的确在赶散了西夏军之后,就开始整队后退。

溃散的马军步军停下了脚步,但短时间内,他们不可能重新组织起来。而原本被禹臧温祓和仁多保忠二人放在战场边,随时支援占据的两支三五百人组成的铁鹞子,这时候分别被被宋军和青唐部蕃军的两队骑兵给牵制着,一时难以进入战场之中。

只能眼睁睁地看着宋军退回营地,而后就是一道道的炊烟腾起在宋人的营地中。

“在这样下去,永远都不会有了局。”禹臧温祓因为自己在仁多保忠面前的失态而恼羞成怒,同时也失去了继续下去的信心,“还是退兵吧!”

他不是在征求仁多保忠的意见,他是在预先通知自己的计划。

“且再等几日。”仁多保忠立刻阻止。

“难道还会有援军来?!”禹臧温祓冷笑反问着。

“家叔说了,木征本人依然还在,他还有着翻盘的能力。而且,最有力的援军正在东京城中。”

