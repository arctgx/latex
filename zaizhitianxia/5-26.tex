\section{第三章 时移机转关百虑(12)}

【这是第三更。】

“是不是李资深这个月没人可弹劾了,怕被罚辱台钱……怎么掉到碗里的都当成肉了。”韩冈对过来禀事,顺便通报新闻的下属笑道,“他堂堂新任御史中丞,不在两府中找个人,好歹也得是侍制以上的重臣,怎么挑了个直史馆的知州?”

来禀事的官员,是衙中的勾当公事,四十多岁的选人,几乎没有升上去的可能。不过在衙门中久了,说话、办事也使得力,更会讨好上司。

他闻言便陪笑道:“苏子瞻天下闻名,过去又曾恶了李中丞。李中丞如今用事,自是要先拿名气大、又有旧怨的开刀。”

“怕也是不敢在朝堂里面闹,否则耽搁了伐夏之事,李定他也吃罪不起。”韩冈啧了啧嘴,他可是不怕乱说话。

勾当公事登时就露出了恍然大悟的表情,脑袋连连点着:“龙图之言让下官茅塞顿开,当是如此,当是如此。”

韩冈瞥了他一眼,“你们这些在京师衙门里混老了事的,想不到才有鬼!”

勾当公事连忙道,“小人愚鲁得很,委实没想通。”

李定弹劾苏轼,对京师的官吏们来说,也就是当个聊天的谈资而已。

御史言事定有时限,时限之内如果没有上弹章,那就是不合格,要被罚辱台钱。乌台中人咬人不稀奇,不咬人那才是新闻。

韩冈身上的弹章,数一数能有上百本,而两府中人更是只多不少。被御史中丞盯上,也不是什么稀罕事,谁也没放在心上。

拿起勾当公事送来的公文,韩冈翻了翻,是环庆路发文来给路中的骑兵要马。

不过并不是战斗时的战马,而是平常行军时的骑乘马。经过了几年的茶马互市,陕西缘边五路的骑兵,已经勉强能做到一人双马,或是一马一驴。不过平时多有了缺额,补起来不容易,趁着眼下朝廷要用兵于北的机会,便把手伸出来唱莲花落了。

“寄养在沙苑监的军马,还有四千一百匹吧?”韩冈问道。

“四千一百一十九匹。”

“一千一百匹军马的缺额给环庆路补上。调一千两百匹过去,省得半路死了,还要来打饥荒。”韩冈说着,提笔在公文上写下了自己的意见。

“龙图!”勾当公事惊讶的叫了一声,“给三五百匹就够了!”

韩冈笔没停,随口问道:“为什么?”

勾当公事急着道:“下面的人一贯的狮子大开口,说是要一千一百匹,其实都可以打个折扣的。”

“这是打仗,不是斤斤计较的算账。”韩冈抬起头,脸上不变的微笑,却已经由和煦变得让人心中发寒,他声音轻柔:“宁可多配,不能少配。战时的损耗是平常的十倍都不止。而且配了少了,出了事,前线推卸责任就有地方了。你也是衙中老吏,这点事不应该要人教啊。”

韩冈的话够诛心了,方才还言笑不拘,转眼间把下属吓得脸色发青。

之前韩冈借韩缜的手整顿衙中纲纪,已经给这里的官吏一个警钟,他虽说不想多管事,但若有人将他当成可以糊弄的糊涂官,就别怪他韩冈下手不讲人情了。

“跟外面都说一说,平常倒算了,如今是非常之时,谁敢不长眼睛的乱伸手,下场如何,自己心里应该清楚。”韩冈挥挥手让下属退下。

勾当公事拿了韩冈的批文连忙就退了出去。

韩冈盯着他的背后冷哼了一声,群牧司里的官吏惯会靠山吃山,上百万贯的年均投入、上百万亩的牧监土地,出产的战马连一个马军指挥都配不齐。王安石逼得没办法,才去另起炉灶行保马法。如果真以律法来定罪,这些官吏全杀了或许有冤枉的,隔一个杀一个,肯定有漏网的。

方才此人要真是忠心投靠自己,肯定还会多劝两句,而不是被吓了一下后,就闭嘴不再多言,说不定私底下还要发狠看自己的笑话。

看到环庆路得马如此轻易,过上一段日子,肯定就有其他几路伸手过来要马。这件事也不难预测,谁要是以为他没办法处置,就实在太小瞧他韩玉昆了。

既然韩缜现在忙着枢密院中的差事,群牧司暂时由自己负责,做一天和尚撞一天钟,就得好好整一整。虽不说控制在手里,但也要做到说话算话才是。

而且韩冈静极思动,闲在家中读三苏父子的史论,实在是没什么意思。而儒学上的水平,也不是坐在家中死读书能培养出来的。

想到三苏的史论,韩冈便想起了倒霉的苏轼。仇家李定任了御史中丞,被当成了开门红,一下就被咬上了。

不过话说回来,这件事苏轼本人也有责任。与李定的仇怨,可是他自己惹上身的。

想想当年李定不为生母服丧的一桩公案,挑起来的是反对变法且利益相关的旧党,可将气氛炒热起来的,却是事不关己的苏轼。

好吧,其实他也可是算是旧党中的一员,但毕竟没有什么利益牵扯,也不是言官谏官。当年苏颂任中书舍人,天子要给李定加官,苏颂拒绝草诏,最后被贬官出外,这是有直接关系的,有公事上的牵扯,算不上有多大的仇怨。

但苏轼半点牵连都没有,职位上不搭界,私下里没来往,公事私事都没瓜葛,却偏偏要凑上去,这是主动跟人结怨。

而关于李定隐匿母丧的大不孝一案,韩冈是站在李定那边的。

李定当初被弹劾隐匿生母仇氏之丧,但据李定自称,其父只说仇氏是乳母,而从未说过是生母,加之仇氏在李定幼时就已经离开了李家,李定纵有猜测,也不敢违父命。所以在生母死后,他是以侍养老父的名义,辞官回乡,为生母持丧。

隐匿父母之丧,全都是为了避免丁忧解官,不会有例外。而李定当年虽没有申请丁忧,但他解官回乡是确凿无疑的,朝廷也遣了人去查证,他自称持丧自居三年,是作伪的可能性很小,否则他为什么要辞官?

从逻辑上推理,他受到的攻击并不成立。天子赵顼当年也说‘所以不持心丧者,避解官也。定既解官,何所避而不明言心丧?’

一桩显而易见的事,却因新旧党争,让支持王安石变法的李定备受攻击,都把他当成了对新党的突破口,争相攻击。其中就以没什么瓜葛的苏轼做得最狠,正好当时有个叫朱寿昌的官员,为寻生母,辞官遍寻天下。苏轼便拉着一帮文人去给朱寿昌写诗,而对李定一通嘲讽。

梁子就是这么结下的。现在李定做了御史中丞,找苏轼的麻烦,也不是不能理解。

而且李定的弹劾虽严重,韩冈倒觉得没什么大不了的。

仁宗的时候,进奏院之案,缘起于范吕党争。属于范仲淹一派的苏舜钦以进奏院祠神的名义,卖了院中架阁库旧纸,招了朋友来饮宴。当时席上有人写诗‘醉卧北极遣帝扶,周公孔子驱为奴’,但最后定案时,还是以苏舜钦监守自盗为罪,并未以文字入罪。

而李定对苏轼的攻击,却是集中在他的文字上。苏轼有着文人的一切毛病,爱抱怨,喜欢依靠自己的文采说些酸话,想要从中找到一点对天子的抱怨,以及对国是的攻击,不费吹灰之力。

可这样罗织出来的罪名,能有多大的作用,就完全没办法让人期待了。

你骂过来,我骂过去的,朝堂上很是常见。如今大战在即,朝中要维持稳定,这件案子当不会闹得太大——已经不是新旧党争激烈化的时候了。

也就是苏轼免不了要吃点小苦头。韩冈这两天也分心猜测了一下究竟会是什么样的责罚,究竟是罚铜,还是申斥,又或是降官。

反正也就这些惩罚了,苏轼本来就在外地任官,引罪出外就轮不到他,至于其他的惩罚,最终也只是降官而已,总不可能处罚得太过严重。

可事情的发展出乎韩冈预料。

如果天子想要深究此案,按道理就是该派人去湖州查问详情,但在上元节前夜,韩冈却从属僚那里听说了天子已经责命御史台,派人去提苏轼上京审问。

“这事情做得未免过头了吧?!”

韩冈听说了之后,登时就吃了一惊,这么做未免太过火了。苏轼上京后必然是要进御史台的大狱待审,就算不会对士大夫使用刑具,但御史台想要锻炼成狱,却是一点都不难。

“听说是看了李中丞和舒御史的奏章后,天子震怒,要将苏子瞻提入京城。”

韩冈前两天,先看到了李定的弹章。而在昨日,也看到了舒亶的奏章。一个列了苏轼的四条应当论死罪名,一个则是在苏轼的文集和他再任湖州时所写的《谢上表》中,寻找到了他心怀怨望的证据。

‘这不是文字狱吗?’

虽然苏轼是真的抱怨,但毕竟不是什么罪名,但爆出来的时机不对,天子正是志得意满的时候,耳边却听到了地方官员竟然还有心怀怨望,对新法始终没有好话的例子。

这个时候,天子可不是能容人。

越是才高,在百姓心目中留下的印象就越深刻。一想到苏轼的诗词,能让天下的百姓陷入其中,赵顼就不可能不恨。

“这下事情可闹得大了。”韩冈低声自语。

