\section{第48章 斯人远去道且长(二)}

【第一更,求红票,收藏】

韩冈一觉醒来,阳光已经透过薄薄的窗纸,直照了进来。天色早已大亮,窗外的鸟雀都在吱吱喳喳的叫着。

昨天的酒宴,韩冈难得的醉了一次。虽然不是如路明般的酩酊大醉,但喝到头昏脑胀的感觉,现在醒来后,他便后悔不迭。反倒是刘仲武,前些日子喝酒喝伤了,韩冈记得他昨夜便一反常态,只是浅尝即止。

也是在昨夜的酒宴上,就在韩冈他倒了一杯茶之后,周南就突然间变得亲昵起来。香软的身躯紧贴了上来,韩冈的手肘处还能感受到一阵阵充满弹力的酥软触感。色不迷人人自迷,韩冈一时间头脑都有些晕乎,闹得多喝了两口酒。

如果是劝酒的人别有用心,即便有着西施貂蝉般的容貌,韩冈也会提高警惕,但周南很明显对自家有好感,不然听到自己第二天就要回秦州去,便登时苍白了脸。韩冈虽是才智过人,但对女儿家的心思还是有些糊涂。自己在这位歌舞双绝的花魁行首面前应该没有留下什么好印象,还刁难嘲笑过她,怎么突然之间就莫名其妙的喜欢上自己?

韩冈就着房中的热水,梳洗打理着,最后很麻利的换上了一套适宜旅行的外袍,走到外间。桌上,李小六已经把早饭准备好。

“官人醒了没有?”门外突然响起路明的声音。

“今天就要启程,哪能贪睡?”韩冈放下筷子,问走进门来的路明,“不知路兄有何事?”

“路明是来向官人道别的。”

韩冈对路明的心思有所了解,他每天往外跑,都是为了去打探市价行情,摆明了是要做个商人。只是韩冈觉得路明的计划成功的可能性并不大:“路兄是准备留在东京城?在这里人生地不熟,生意可不好做。”

路明苦笑道:“路某文不成武不就,也只有做个逐利之夫了。不赚些钱,也没脸回乡见人。”

“……若路兄在京城做得不顺,可往秦州一行。虽然秦州的确荒僻,但如今王机宜正要设榷场行市易之事,以路兄之才,当有用武之地。”

韩冈留了句话。因为他并不打算立刻推荐路明去,王韶身边的几个亲信侍卫,有三个是跟自己有关,再推荐人去打理市易之事,王韶心里肯定会闹嘀咕。不过等事情做起来的后,再将路明安插进去,那就没问题了。

路明道了谢,出门去找刘仲武道别,而韩冈看了眼已经变冷的羊肉汤,没兴趣再动筷子。

“晦气!”屋外院中突然一阵喳喳的鸟叫,紧接着传来李小六的声音,“俺今天就要上路,你们这些鸟货却来触人霉头。”

韩冈闻声出门,见着李小六赶着一群乌鸦乱跑。他出言阻止:“别赶。任它们去。”

“怎么,玉昆你喜欢乌鸦?”程颢的声音从院门处传来,与张戬一起进了院中。

韩冈连忙上前行礼,惊喜道:“两位先生怎么来了?”

“给玉昆你饯行啊,”程颢笑得很平和,“这月来,吾等相处甚得,玉昆你要走了,当然要来送一送。”

张戬则看了看院中,重复了程颢刚才的问话:“玉昆你喜欢乌鸦?”

韩冈心思转了一下,便道:“学生倒是不讨厌乌鸦。”

张戬奇道:“玉昆为何有此言?”

“常言道鸦报凶,鹊报喜,但学生觉得,乌鸦此行近忠,而喜鹊却是近谀。”

“鸦近忠,鹊近谀……说得好,说得好!”张戬为之抚掌,笑道:“直言敢谏才是忠臣,只有小人才会满口好话。”

但程颢却是不太喜欢韩冈的说法,韩冈的说法看似一反流俗,但实际上却有媚俗以求清名的成分,“说得虽是有几分道理。但悖于人情并非正道,玉昆你忘了中庸之说了吗?”

韩冈低头:“学生不敢或忘!”

与张戬和程颢又闲谈了一阵,刘仲武和路明也一起过来了,虽然路明打算留在东京,但还是会送韩冈和刘仲武出城,而且韩、刘二人一走,他也得另找地方去住了。李小六对行装做着最后的整理,等到一切准备完毕,已经到了未时。这段时间,除了张戬和程颢,再没有一个人来。

王安石、吕惠卿等人并没有来给韩冈送行,只是提前把赠礼送到了韩冈的房内。当然韩冈也不指望他们来送行,一方面是他们最近事务繁忙,不便请假,而另一方面,就算王安石这个参知政事到不了,几个变法派的主将来给一个选人送行,也够骇人听闻了。

不过韩冈也清楚,王安石、吕惠卿他们不来,恐怕也是有一个部分的原因不太喜欢自己进呈的策略太过尖锐,过于诛心。虽然这些策略他们日后免不了要用,但心里总是有些别扭,所以才有了些疏离。但这正好应了韩冈的希望。

王安石身边缺乏人才和助力,这点事不用说的,要不然他只能选一些正八品、从七品的官员做助手。已经身居高位的官员,没有几个愿意跟从王安石一条路走到底,就如如今的宰相陈升之,他当初可是变法初兴时的主要推手,主管三司置制条例司,但等他登上相位之后,便华丽转身,一转变得反对起新法来。

人才的匮乏,让王安石有了改革科举的心思,也让他不会放过一个可用之才。韩冈知道自己的表现太好了,如果没有他后来的那番建议,凭着他在那天的会谈中前半段的言辞,恐怕在秦州待个两三年,就会王安石找借口调入中枢去。这与韩冈最初与变法派划清界线的计划不符。虽然他如此已经决定加入变法派,也想帮王安石安安稳稳的实现变法,但他觉得还是做个外围成员比较安全。

在韩冈看来,新旧党争的结局短期内必然以王安石胜利而告终,但并不代表他们能一直胜利下去。商鞅也是得意了二十年,最后却被车裂。既然如此,就不能与变法派走得太近,至少不能成为新党的核心成员,所以他才会一咬牙,在王安石他们心中留下心机深沉这个印象的原因。

在韩冈想来,既然王安石此前一直维护着朝局不向党争方向滑落,那他对行事毫无顾忌的人物,就不会什么有好感。韩冈就是通过搜集来的情报,了解到这一点,才会这么去做。而效果也出乎他意料得好,甚至让韩冈预备的许多后续手段都失去了表现的机会。

不用再等人,韩冈领头在城南驿的驿丞那里交接登记过后,一行人便上马启程往城西去了。

韩冈离京的这一天,就在今科礼部试的前一天。不知为何,参加科举的虽然仅仅是几千名来自全国各地的士子,但京中街巷上的气氛却莫名其妙的紧绷着。

韩冈高坐在马上,望着喧闹声比平日减低不少的集市,心中暗道,‘当真是跟高考一样吗?’

韩冈还记得千年后的高考,那三天,每一个城市都是一样紧张,凡事考生优先,如果有那个司机不开眼,在考场门口按下喇叭,他的车子都很有可能被愤怒的考生父母给掀翻掉。

不过科举虽然很被看重,但开封城中并没有专门的贡院。第一次听说这件事时,韩冈是深感意外,难道他前世曾经参观过的鸡笼一般考房,此时还没有出现!?而他得到的回答,是此前每一科的礼部试,泰半是借了太常寺、国子监或武成王庙来充作考场。

行至御街之上,一行人又下马换了步行,没人能在横穿御街时还骑着马。韩冈顺着御街向南望去,在最南面,近着开封城南城门南薰门的地方,是熙宁三年庚戌科的考场,也就是国子监。

就在前几天,韩冈还甚有兴致的去了位于城南的国子监看了一看,只是立刻就被守卫的兵卒给瞪了回来。皇宋的最高学府已经被近千兵卒围了里三层,外三层,水泄不通,但凡有人想靠近,便立刻会被驱走。守备之森严,比起御史台的大狱怕也是差不离了。

按照自本朝定鼎以来,逐步确立的科举制度。每一位主持科举的考官,都在确定了差遣之后,直接去考场住下,周围又用兵将围定,只有耗子能出入,人却不行。这样的制度称为锁院。而今科的主考官王珪等人,早在月前便已定下,到现在已经在国子监内住了快有一个月了。

坐上一个月的监牢,韩冈难以想象这样的憋闷。但从中可知,如今的朝廷对抡才大典究竟有多看重。至少不会像太祖时,为了分出状元谁属,两名殿试排在最前面的考生,脱下外袍,在举行殿试的讲武殿上练起相扑来,倒应了讲武殿之名,最后是文武双全的王嗣宗拿到冠军。

讲武殿上相扑争状元是一桩,还有一桩是关于御街北面的。韩冈又向北望去,那里有一座城门,也就是内城南门朱雀门。

朱雀门的门额是‘朱雀之门’,一直以来都是如此,赵匡胤问赵普,为什么不直截了当的称为‘朱雀门’,赵普回答说是“语助尔”,单纯的助词。赵匡胤便嘲笑道:“之乎者也,助得甚事?”

