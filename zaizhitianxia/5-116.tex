\section{第13章 羽檄飞符遥相系(五)}

【状态上来的第四更。】

竟还是老熟人!

世事无常,还真是巧了。吴逵无声的咧嘴笑了一笑,其实也不能算巧了。熙宁八年前后,军器监出产的斩马刀据说是质量最好的,只有那一年做锋刃的夹钢,才会用上反复折锻的百炼钢,之后军器监就改进了制造工艺,改成了生熟铁糅合起来的团钢。

“那是伪钢,是从南方传来的,跟磁州百炼钢没法儿比。”跟吴逵透露这个秘密的人这么说道。

那是在灵州之役后,各部在战利品中挑选兵器,吴逵发现一个一年前才逃来的前西军队正专门盯着刀铭看,才听到了这一秘密。换了伪钢做锋刃后,质量降低了一些,但成本低得更多,便宜得一造数万柄。

听了他的话,李清旗下的汉军,就专挑熙宁八年的军器来收集,也不局限于斩马刀了。不过这个秘密也没有保留多久,就传了出去,使得一众将领们的亲兵,都换上了熙宁八年的斩马刀。甚至之后以讹传讹,连神臂弓、板甲、腰刀、长枪等不相干的兵器,都特意去挑有‘熙宁八年’和‘判军器监韩’这两个铭记的。

随手将斩马刀轻手轻脚的放到了地上,然后吴逵悄悄的走进火光找不到的阴影中。虽然这柄斩马刀是斩过西夏天子的器物,若是能拿回去,挂在太庙里面都不过分,但现在身在敌营中,这样的长兵带在身上可就是找死了。

方才离着中军大营还有半里地,吴逵便感觉到内部有变,留下随行的亲兵,借着自己身上装束与营中士兵没有什么区别的优势,避过已经无心防守的明哨暗哨,悄然潜入了营中。一刻钟的时间,不仅发现了梁太后和梁乙埋都做了鬼,连梁乙埋带在身边的幕僚——包括那位袁宝臣——也都被斩杀得一干二净。

两位收信人都不可能收信了,吴逵当即选择离开,不过他在离开的时候,选择了秉常被囚禁的御帐方向。本想看看是否是秉常重新出面复辟,却看到了他被斩杀的一幕。

躲在帐篷后的阴影里,吴逵暗暗的摇着头。

梁氏死了,梁乙埋也死了,秉常更是被斩马刀一击毙命。

吴逵都没想到他被派来打听的消息,竟然会这般的耸人听闻。杀了西夏国主和梁氏兄妹的人,根本不可能撑起西夏的大局。几乎都没有太大的动静,西夏国已经不复存在。

吴逵抓抓头,隔着五六丈,御帐前的骚动有愈演愈烈的态势。

一名哨兵不知为何转到了帐篷后,一眼就发现了吴逵:“什么人?”

“你爷爷!”吴逵低低骂了一声,一窜上前。在哨兵的脖子上一勒,那人便没了声息。看着倒在地上的哨兵,吴逵皱眉想了想,便探手从他的腰间取下了报警用的号角。

一声警.号,惊动了整个大营。仅仅存在于御帐中的骚乱,在转眼间便遍及了整个中军大营。

混乱中,嵬名济和他的亲信部下开始受到其他士兵的攻击,无论如何,杀了太后、国相,又杀了皇帝,这样的人,要说他没有野心,谁也不会相信。

趁着一片乱局,吴逵顺利的逃离了大营,片刻工夫,便与留下来观望风色的亲兵会合。

论理是应该回去向李清复命,但考虑过后,吴逵还是决定不与李清打招呼了,出了什么事都说不准。

吴逵从怀里掏出一封早已写好的信,混在李清的两封信里面,一起交给随行的亲兵:“快回去禀报太尉,皇帝、太后和国相都被嵬名济给杀了,包括袁宝臣。现在营中内乱,请他速做准备,我在这里边再查探一下。”

亲兵是李清安排在武贵身边的,但对吴逵并没有多少提防。应了一声,接过信就走,也没去多分辨信为什么多了一封。

等那亲兵走远,吴逵便上了马,向事先约定好的地方奔去。当年离开,逃来西夏的时候,没有想到会在这个情况下离开西夏。

到了地头,十几人便拥了上来。

“都来了?”吴逵环目一扫,自家在结识的十二个兄弟,竟无一脱漏的全都来了此处等他,“当真决定要跟随我走下去。若是你们随军重回大宋,说不定日后都能有个出身。”

“哥哥这般好武艺都不想回去做官,俺们几个也都是没个心思再做官军。回去后还不是照样没个奔头。”

“说得正是。就算是做了官,又能如何?还不是照样要受文官的鸟气?!这盐州不就是如此,要不是那个徐学士,怎么可能这么容易攻破?就是换了俺来守,好歹也能多守个三五天。”

“俺当年犯下的事不小,就是官府那里免了罪,仇家也还在,回去后也不会有好结果,跟着哥哥,省得提心吊胆。”

“当初就是没心思在军中混才出来的,眼下也没个家事拖累,愿意跟着哥哥。”

没有一个人犹豫。

“可是要去更西面!”吴逵提醒道。

“愿随哥哥到天涯海角。”

“听说西面大食的女子都是绿眼睛,金头发,俺去了,可是要先尝尝鲜。”

“俺要两个!”

“大家兄弟,还是先一人一个!”

众兄弟的笑谈中,将西夏军中的小心翼翼抛诸脑后,一转变得豪气干云,吴逵放声大笑:“自从离开大宋,躲到西夏隐姓埋名七八年,想不到还有群兄弟肯跟着我。既然如此,我们十三个兄弟,就去打下一份基业来,天王老子也别想再使唤我们!”

“哥哥说的好,就是天王老子也别想再使唤我们!”

“对了,隐姓埋名是什么意思?”

“都相处这么些年了,还不知道哥哥当年到底犯了什么事?”

一个个疑问随着马铃声渐渐远去,走向风沙吹起的方向。

……………………

“辽军南下?突破了黑山威福军司?”韩冈突然之间听到了这个消息,也是吃了一惊,“是否确认过了?”

“辽军的兵力在三五万之间。要仅仅是几千人,还真不一定能探查得出来。”折可适回道。这是折家送来的情报,对于家里的情报来源,折可适还是很有几分底气的。

韩冈拍拍大腿,对于这个情报,他相信。虽然出乎意料,但却完全合乎情理。

之前给绕进了思维定势,耶律乙辛通过支持西夏来讹诈岁币的举动是为了稳固他的权位,而攻取西夏的土地,不费吹灰之力就做了个得利的渔翁,同样是对耶律乙辛巩固他的的权位有着极大的帮助。

“盐州陷落早了一步。”

“可惜种谔迟了一步。”

“两边应该都在后悔吧。

黄裳等几个幕僚在下面说着。

“龙图,我们该怎么办?”折可适问着,“这么好的机会,不应该错过!”

幕僚们都竖起了耳朵。

有便宜不占那是傻瓜。西夏既然完了,在与辽国定下国界之前,能抢到的都要抢到手。迟了就等着看辽人得意。韩冈在河东,看着西面乒乒乓乓乱打一通,局势变化得让人目瞪口呆,但他闲得就只能发呆。除了一个阻卜人,就没有什么的功劳了。

“说得也是。”韩冈想了一想,“不过太冒险不好,先收回丰州再说。”

“丰州?”几个幕僚齐声问。

韩冈点点头:“丰州!”

丰州有两个,韩冈说的是旧丰州。

丰州、府州、麟州这个云中之地,在唐末便为党项人所控制。府州是折家的老家,而更偏西北的丰州则是王家世袭。当元昊立国,向南在好水川、三川口、定川寨打了三次大捷,向西吞并了河西,向东时,则是将河东路在黄河之西的土地抢走了不少,丰州便是其中之一。

当丰州被攻陷之后,朝廷便从府州割下一块地下来,重新设立了丰州。

不过这个丰州,不再是由王家世袭,而是从朝廷派人来接管。王家尽管死光了嫡系,但旁支还是有不少,真要找人继承还是找得到。朝廷这么做,乃是趁势削藩的用心,断了王氏的根,顺便把折家也削弱了,在朝廷看来这是坏事变好事。

不过现如今,丰州还是在折家的控制中。朝廷派来知州、派来通判,当地驻军的主将和指挥使都是外派来的,但下面的士卒和底层军官,全都是以折家唯命是从,分割不分割其实还是一回事。

旧丰州算是最简单的目标,距离近,而且还是大宋旧有的领土,收回名正言顺。

韩冈抬眼看折可适,笑道:“收复旧丰州,这个想法当是跟令尊一样。到时候,折家可是要为全军先锋才是。”

折可适脸色微变,但有立刻恢复正常,但说话时,变得更为恭敬:“龙图说得是,家严也是想先以丰州为目标,想让下官询问一下龙图的意见。不意龙图已经看破了。”

折家愿意与韩冈合作。

韩冈之前在调遣骑兵相助种谔的时候,没有一点耽搁和延误。给折克行留下了很深的感触。为人正直、行事光明磊落,一如当年他在罗兀城时一样——虽然我不同意你的观点,但我绝不会拖你的后腿。

不管韩冈这么做是真情假意,或是出自什么目的,折克行都觉得这是一个可以深交的文官。就算他是装出来的,日后他若是想要维持这样的形象,那么他就必须继续伪装下去。伪装一辈子,那跟真的还有什么区别?

取得韩冈的认同,由此更进一步拉近关系,是折可适从家中得到的任务。从今天的情况来看的,有些事要坦白一点才好。

