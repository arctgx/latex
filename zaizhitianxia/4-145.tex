\section{第20章 冥冥鬼神有也无(21)}

王安石奉召来到崇政殿的时候,正看见魏国公赵宗谔从殿中出来。

宰相位份最尊,而王安石更是连太皇太后和太后都要给他几分面子的名相。领头的赵宗谔虽然是天子的堂叔,也得避让到一旁,不敢挡着着王安石的路,还得先行拱手致礼:“宗谔见过相公。”

王安石拱手还了一礼,“安石见过魏国公。”

赵宗谔是英宗皇帝的叔伯兄弟,但他的名声不算很好,朝臣们没少找过他的错处,也不敢在崇政殿前跟宰相套近乎,“因皇六子诞,奉官家旨意去祭告太庙。皇命在身,不能延误,宗谔先行告辞,还望相公恕罪。”

“不敢,魏国公请便。”王安石疏远但有礼的回覆。

但凡国事,事无巨细宰相都有权过问,天子的家事也是一般。天子吩咐了赵宗谔什么事,赵宗谔在王安石面前都不敢隐瞒。不过王安石不需要赵宗谔多话,作为宰相,有关祭祀的典礼都是要他点头才能通过。

皇六子的诞生,不仅仅是赵宗谔要去祭告太庙,太常礼院的官员,也要去祭告天地、社稷和几代先皇的陵寝。同时主管婚姻和生育的神灵高禖,也得到了一头牛——也即是太牢——作为祭品。

这一切,都是普通的皇六子不应该享受到的礼仪——如果如今皇宫中不是只有他一位皇子的话。

就在一个多月前,当时天子赵顼唯一的儿子、被封为永国公的皇三子赵俊夭折了。宫中一时间愁云惨雾,天子更是悲不自胜,先是辍朝五日,又连着三天失魂落魄的不去政事堂处理政务。还以用药谬误的罪名,将两名翰林医官副使李永昌、张昭文除名编管,一个去了随州,一个去了唐州,以发泄心头怒火。

那几日,朝堂上都有些乱,陕西、河东的局势虽然初定,但河北对面的契丹人多了许多小动作,一时急报频传。加上南方还有对付交趾的战争,这些事少不了天子来过目。

但朝堂上的隐隐乱流不仅仅是因为国事之故毕竟仁宗立嗣时的乱局,当朝的重臣们基本上都是见证人;而英宗闹出的濮议之争,主要当事人的太皇太后曹氏也还在宫里。

也就在那几天,京城里面到处都是谣言,说宫里面不干净,只要是在宫中生出来的皇子都养不大——仁宗算是最后一个。从仁宗皇帝出生后的那一天开始,已经六十多年过去了,就再没有一个在皇城中出生的男丁能养活成人。虽然谣言荒诞不经,但根本就没法辟谣,事实如此。

幸好没隔多久,也就在一个月前,宫里面的朱才人又给天子添了一个儿子,让皇帝不至于后继无人。

今天是皇第六子的满月,天子遣魏国公赵宗谔祭告太庙、又遣太常礼院的官员祭告于天地、社稷、以及先皇诸陵,这些都是给元子,也就是嫡长子的礼节,但刚刚被赐名为‘傭’的幼儿,已经是天子的第六个儿子了,甚至不是嫡子。而且到底日后能不能保住,不至于夭折,也还是两说。

王安石长叹一口气,要是上下都有个万一,日后会变成什么样的局面可就不知道。

仁宗会选上英宗,是因为他本身就是真宗皇帝的独生子,没有亲兄弟可以接位。但当今天子,可是有两个兄弟,且年长的一个,与他的女婿结怨极深,一个不好,肯定是会拖累到了女儿。要是皇帝有他二女婿的一半能耐就好了,也就在韩冈南下后不久,女儿王旖又给自己添了一个外孙,二十五六的韩冈,已经是有五个儿子一个女儿,一个个健康得很,说起来,天子都是要羡慕的。

听着殿侍们的通传声,王安石跨步进殿。

“王卿。”前些天的满面悲戚已经消失不见,赵顼现在的脸上终于有了笑容,“方才朕命赵宗谔祭告太庙,忘了跟相公先说上一声了。”

“太常礼院即以奉旨祭告天地、社稷和诸陵,也自当祭告太庙。”虽然是应有之理,但其实是侵害了宰相的知情权,不过王安石倒也没有在这等小事上做文章的意思。

王安石没有提出异议,赵顼心情又好了几分:“二十天前,门州大捷,官军轻取交趾边塞。想必此事已经到了富良江边了。”

王安石也听说了门州的捷报,但他觉得还是要小心一点:“也要提防着交趾是在坚壁清野。眼下官军深入交趾数百里,但除了门州,都没有经过一次大战,可见交趾人必有谋算。”

“相公大可放心,章惇为人精细,令婿更是行事缜密。而且还有燕达、李信,都是能征惯战的将领,不至于犯下大错。”赵顼轻笑起来,“想必当能在明年三月之前,将攻进升龙府的捷报给朕送来!”

王安石微不可察的皱了一下眉,这战阵之事,哪有这般容易!

…………………………

李常杰都没想到宋军动作这么快,直接就在富良江边设立了船场,都已经看得见新船了。

他本以为宋人会设法利用海船,钦州的合浦港虽然毁了,但广州番禺港还在。有船在手,运粮、运人,甚至开进富良江来做渡船,都很容易,尤其是前些天永安州失陷之后,李常杰更是如此猜想着。

就在富良江口,他已经给宋人准备好了一个惊喜。即便他们用的是广州走外洋的水手,能与海盗厮杀的亡命之徒。,只要不熟悉富良江口的水道,在江口的伏兵照样能打得他们全军覆没。

可对岸的宋军竟然放弃了简单的手段,反而花费时间打造船只。以章惇、韩冈两人的才智,如果不能在雨季来临之前将船只打造好,他们决不会多费手脚。

“此事到底查实了没有?”李常杰亲自询问着从北岸回来报信的哨探。

哨探磕了一个头,道:“回太尉的话。已经查清楚了。宋军怕被我们发现,将船场藏在的漯河入富良江的河口上面一点的芦荡之后,那里本来就有几个深水塘,是北岸的渔家置船避风的地方。”

“这怎么可能!?”统管水师的主帅阮陶立刻在边上大叫起来,“铁钉、桐油、麻絮、绳索在,这些造船的资材从哪里来的?!船匠又从哪里来?”

“宋人缺这些东西吗?都不占多少地方,从北面运来都方便。更不会缺船工,永安州的船匠有一多半是汉人。”李常杰摇了摇头,这时候对于水师,也不便多加斥责,又问道:“船场中的船只形制如何?”

哨探猛的磕了一下头:“小人无能,宋人的船场守卫森严,都潜不进去。但船场中有不少人,夜里更有不少木排从漯河上游放下来。”

“新砍下的木头能造船?就是房梁、棺材,都应是将木料放个三五年,晾干后才能用吧?”李常杰的幕僚皱着眉头问道。这也算是常识了,汉人也好、交趾人也好,许多人在上了年纪后就开始为棺材寻找上好木料,往往一放就是十几年几十年,没说用新鲜木料。

这次是阮陶帮着解释:“就是用新木头造船,如果只准备用一两个月,就没有什么关系。不过这样的船造不大,造得大了,一下水,船板就会给挤歪掉。”

“也就是说,宋人只能造小船?”李常杰眼睛亮了。

“若是打造五丈以上的大船,等他们将船造好都不知是什么时候了!”阮陶说道。

“宋人不会与我们在江面上硬拼,若是几百艘船连夜渡江,到时候凭着水师的船只恐怕难以抵挡得住。”李常杰的幕僚提醒着,“宋人的能工巧匠手脚可不会慢。看宋人将船只深藏的样子,就知道他们是想将我们打个措手不及,不会与水师在江上决战,”

李常杰的另一位亲信提议道:“不是说宋人的船场藏在芦荡后吗?遣人去那里点个火,将船场一把火都烧掉如何?”

“现在刮的都是北风!”

李常杰几人一起议论着,越发的感觉宋人实在是狡诈。阮陶在叹着气:“明修栈道、暗度陈仓,想不到之前的事,竟是一个骗局。”

就在两天前,当交趾的朝臣们听说对岸的宋军怎么从渔民手中截获几艘不堪用的渔船,他们可是大笑过一阵。只能用坑蒙拐骗的手段来抢劫渔船,宋人的确是技穷了。可现在再一看,竟然完全是伪装,是彻头彻尾的骗局。要不是李常杰多留了一个心眼,遣人仔细去查探,说不定当真给宋人瞒了过去。

“这虚实之道,谁也不能与汉人相提并论。章惇、韩冈都是宋国有数的帅臣,他们的才智都是千万人里才能出一个,”李常杰在章惇、韩冈两人的手里都吃过大亏,遂有空尽力吹捧两人,也显得自己不是因为愚蠢才落败,“他们不可能做蠢事,劫船是假,造船才是真。”

李常杰深吸了一口气,眼神登时锐利起来:“船场决不能留。守卫船场的兵力究竟有多少?”

“回太尉,有四五百人,当是一个指挥。”

阮陶疑惑道:“这守卫不算多啊!”

“一个船场有四五百来守着,已经能完全封住消息了,人再多可就藏不住身。”李常杰双手紧紧握着拳头,狠狠的又重复了一句,“这船场决不能留!”

