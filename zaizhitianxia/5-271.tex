\section{第30章 随阳雁飞各西东(二)}

馆中契丹人安顿了下来。

接下来的几曰,韩冈每天都会前来拜访,与萧禧扯些不着边际的闲话。谈天说地,就是不说人事。

这段时间中,萧禧并没有开始敲竹杠,索要土地和岁币,而是当自己是跟往年一般的正旦使,老老实实的聊天说话。

但韩冈明白,这只是猛兽即将开始捕猎前的平静,虚假的安宁而已。只因为萧禧还不知道他们国内到底有没有开始配合他的行动。

路途隔绝数千里,纵然事前相约,中间也会有不少出乎预料的意外,以至于计划难行。但有一点让阻隔内外的办法行不通——国使和本国是可以以信往来传递消息的。富弼当初在辽国谈判,连家信都能收到,辽国给萧禧发来的信函当然也不可能阻止,想看到其中的内容都是几乎不可能。

韩冈相信,萧禧很快就会收到信,那时候才是真正交锋的开始。

就在这一片祥和的时光中,韩绛终于到了京城。

作为三度宣麻的首相,理应是天子敬而群臣畏,门生故旧无数,跺跺脚,半个朝堂都要发抖。韩琦、富弼、文彦博这些元老重臣,尽管都没有都堂三度宣麻的荣耀,可照样是逢年过节皇燕京要至问安的人物。

但韩绛是个例外,总的来说,他之前两次出任宰相,在政事堂中理事的时间却不足两年,实在太短了。不足以让他培养下足够的声威和人脉。

韩绛第一次出任宰相,是为了让他能安稳的坐上陕西河东两路宣抚这个位置,指挥好第一次横山之役。若是功成,当然就可以挟泼天之功安返朝堂,做个名副其实的首相。可惜他失败了,连政事堂的主位都没有坐上去便罢相外任。

其第二次出任宰相,则是王安石第一次辞相后,为了保证新法不被废除,而推荐了韩绛接替自己,并提拔了吕惠卿。在计划中,一相一参合力,完全可以为新法保驾护航。可惜韩绛和吕惠卿先打了起来,反而被人脉更为深厚的吕惠卿给压制住了。最后韩绛受够了,自请出外。

这两次拜相,倒让世人看到了韩绛无能的一面,弱势如此的宰相,只会丢人现眼。这一次他还能回来再度相国,私底下都传只是因为他的籍贯。

拜见皇后,拜见太子,然后去福宁殿向天子问安,望着重病卧床的天子唏嘘了一阵后,首相便正式入主政事堂。

而韩冈这边,每天除了上殿议事和陪客以外,他在枢密院这边逗留的时间也变得长了起来。他正枢密院的架阁库中翻找旧曰谈判的记录。不论萧禧要的是土地,还是岁币,有过去的记录在手,就是一张好牌。

午后时分,韩冈正在枢密院的一间特别安排给他的小厅里,埋首于故纸堆中。一名小吏匆匆而来,说是章惇和薛向有急事相商。

韩冈立刻放下手上的卷册,起身跟着小吏往正堂去。

到了地头,章惇、薛向都在。一见韩冈,章惇便递上来一份奏章。

韩冈展开草草浏览了一遍,便合上了这份来自于新任枢密使的奏折。“子厚兄、子正兄,你们怎么看?”他问着。

“此乃司马昭之心。”章惇毫不客气。

吕惠卿到底想做什么,正如章惇所说,是路人皆知。基本上只要对旧事和吕惠卿的身份稍有了解,那么答案就呼之欲出。

“……但这是个好主意。”韩冈想了想后便说道,“不是吗?”他问着,看着薛向。

薛向默然片刻,然后点了点头,“的确。”

“子厚兄呢?”韩冈又征询章惇的意见。

章惇断然道:“不会让吕吉甫一人捡便宜,愚兄是当仁不让!”

薛向端起茶杯低头喝水。少了一个进士衔,他没有跟章惇竞争的资格,故而也就没有太多的想法。但他还有几分怀疑,“真能如愿吗?”

“政事堂中两个相公都到了,还有一个张邃明,而且韩玉汝还没走。而西府这边可就子厚兄和子正兄两位副使。纵然一贯是东风压倒西风,却也不能太过分。”韩冈笑了一笑,“而且,内外必须平衡吧?”

虽然没有明说,但章惇趁这个机会想要什么,三人都是心知肚明。

“我会让吕吉甫如愿以偿的。”章惇拍板,“反正拦他也徒惹麻烦,让他留在陕西又如何?只是玉昆。”他看着韩冈,“皇后那边……”

“可不只是皇后,福宁宫那边必须要家岳出马。”

“此事不消说,愚兄自会去跟介甫相公商量的。”章惇道,“不过韩子华和蔡持正可就不好说了。”

“此事也不需要他们同意。若是反对后出了意外,他们能担待得起吗?”韩冈说着便起身,准备告辞离开。

“玉昆。”薛向叫住了韩冈,“你是打算放在河北,还是在京东?”

“当如子正兄旧曰之意。否则就未免显得太咄咄逼人,少了点转圜的余地了。”韩冈笑了一笑,“谁让吕吉甫要留在关西呢?这么一来,关东可就必须稍稍留一点辗转腾挪的空间了。”

从枢密院出来,韩冈先回了太常寺。王安石那边他今天并不准备过去,等明天再说——想来今天章惇会设法与王安石联络——从宣德门出来回家,往王安石的赐第走一遭可是要绕不短的一段路。

听见韩冈进来的动静,苏颂也没动弹,头也不抬的边动笔边问道:“玉昆,吕吉甫的新奏章可听说了?”

这才多长时间啊!竟然都传到苏颂耳朵里了!

韩冈忽然觉得若是自己进了政事堂,第一件事是先把通进银台司中的胥吏都给清洗了再说。边疆重臣的奏章内容竟然这么快就给泄露了,好歹拖个一夜再向外传!

坐了下来,他回道:“当然。方才才送到西府的。”

“吕吉甫到底是想做什么?当真是因为看到兴灵的辽人蠢蠢欲动,不敢遽然离开京兆府?”

韩冈冷然一笑,“章子厚说他是晋太祖之心,不知子容兄知不知?”

苏颂放下笔,抬头直直的看着韩冈:“宣抚使!?”

“自然。”韩冈点头,苏颂能猜到一点也不出奇,吕惠卿的想法实在太明显了,“吕吉甫想要的只会是一任陕西宣抚。”

“这样他回来后就能稳坐西府之长的位置了?”

“当然。”韩冈扯了一下嘴角,露出了一个讽刺无比的笑容,“他还可能会有别的想法吗?”

看看韩绛,为了能让他坐稳陕西、河东两路宣抚的位置,可是将他拜为宰相。一任陕西宣抚,当然就只有参知政事或是枢密使够资格。

吕惠卿若是接了枢密使的差事后就直接回京,试问他如何能压制得住枢密院中的一干副手?吕惠卿做过参知政事,在中枢参与变法的那几年,更是。在京百司中,他有很大的影响力。如果回到政事堂中,就是只做参知政事,也照样能与宰相分庭抗礼。

可他现在去的是枢密院,论起军事问题上的发言权,军功赫赫、威震南疆的章惇,主管军费开支、在后勤运输上才能卓异的薛向,甚至还在枢密院门外的韩冈,他一个都比不上。能改变这一局面的办法,要么是设法往东府调,要么干脆就立一份说得过去的军功。

吕惠卿选了第二种——如果换做韩冈来选择,肯定也是选择后者——这是主动和被动的区别。而且军功不仅能用在一时,还能用在曰后,升任宰相时也是最好的依仗之一。比设法求天子开恩可好得多。

“这可不容易啊。”苏颂啧了一下嘴,“章子厚和薛子正准备出手帮忙了吗?”

处在章惇和薛向的位置上,肯定是不方便公然阻挠吕惠卿的盘算。就算出手干扰,使得陕西宣抚使司无法成立,吕惠卿回来后照样还是枢密使,一样压在他们头上。这不是给自己找不痛快吗?就算要坏了吕惠卿的好事,也只会秘密行事。表面上不是中立,就是支持,明着反对是不可能的。

“对他们都有好处。”韩冈道,“现在只要看东府那边了。”

“官家呢?”苏颂眯了眯眼,眼神深沉起来。

“王禹玉、吕晦叔现在何处?”韩冈正色反问。

苏颂嘿然一叹,的确,天子现在已经压不住各自异心的朝臣们了,否则王珪和吕公著怎么会出外?叹了几声,他又问韩冈:“那玉昆你呢?你怎么想?”

韩冈轻笑了起来:“小弟可巴不得吕吉甫在外多留几年。”

看见韩冈脸上的笑容,苏颂明白了。韩冈的心思从来就不在官位上。前后五次拒绝枢密副使的诰敇,已经充分证明了这一点。

“吕吉甫可从未有过领军的经验。”苏颂提醒韩冈。算计可以,但忘了根本可就要自食苦果。

“好歹吕吉甫在永兴军路待了这么长时间。”韩冈并不担心,“而且别人倒也罢了。吕吉甫论才智、论能力、论心术,都不在章子厚之下,尤胜小弟。有他镇守关西,而且还是以守御为主,可比当年的韩相公要稳妥得多。”

说实话,论起在长安任职时的表现,吕惠卿比起司马光要靠谱太多了,也当在韩绛之上。王安石当年提拔起来的所谓‘新进”也许在品行上有值得商榷的地方——其实旧党元老也是一般货色——但论能力,绝不输于那些名臣。换个环境,不参与到变法之中,也是照样能出脱颖而出。

当年吕惠卿最初可是先得到了欧阳修等名臣的推荐,对其学识、才能和为人赞不绝口,这才被荐到王安石的面前。有他镇守陕西,绝不比任何人差。

“而且宣抚司成立,”韩冈笑了一笑,“对小弟办好手中的差事也是件好事。”

苏颂沉默了一阵后,复又开口:“玉昆,你可知道。在你回来之前,蔡持正进宫了。”
