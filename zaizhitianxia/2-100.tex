\section{第22章 瞒天过海暗遣兵(五)}

【第二更,求红票,收藏】

俞龙珂和瞎药好生的将智缘和韩冈送了出来。他们都是虔心礼佛之人,对主动上门来做法事的东京高僧,千恩万谢也不足以表示他们的感激,就差在脑门上写上顶礼膜拜四个字了。

青唐城外十里处,别过热情的青唐部族长和他的兄弟,在通往古渭寨的道路上,韩冈与智缘并辔而行。

先在吹莽城做了三天法事,又在青唐城做了三天法事。智缘已经好几天没有好好的睡上一觉,但他今天上马时还是精神奕奕,红光满面。在马上还能谈笑风生,头脑的运转也没有一点迟滞。年近五旬,智缘依然如此精力充沛,这让韩冈惊叹不已。

对于僧侣这个职业,韩冈素无好感。如今真正恪守清规戒律的高僧大德寥寥无几,反倒是花和尚多不胜数。喝酒吃肉都算不上什么,逛窑子上青楼、娶妻生子也是寻常,把女人藏在庙中狎1玩,这样的事同样时有耳闻。甚至有个僧人娶了名妓招摇过世,自称是‘没头发1浪子,有家室如来’,世人尤以其貌似豁达而艳羡不已。

韩冈对他们的行为无意作出评判,不守清规也不关他的事。

——可这些僧人的钱是从哪里来的?

一是靠信众捐赠。官宦富户的钱就不提了,捐得虽多,但人数毕竟是少数。吃斋念佛的寻常百姓才是占了大头,辛苦积攒下来的一文文钱,从家中吃穿用度节省下来,尽数捐给寺庙,求个家宅平安,求个来世福报。谁能想到这些钱却变成了逛窑子的花钱?

二是租赁庙中田地。各州各县之中,占地最广,拥有土地最多的地主,往往不是豪门官宦,而是一间间寺院。家族可以在一两代人中兴盛衰落,但占了好位置的名刹,却能延续数十年、数百年。靠着多年的积累,更是靠着信徒不停的捐赠,一间普通的庙宇往往能置办下数十顷、甚至数百顷的田地来。至于大相国寺、白马寺、少林寺这些大丛林,阡陌往往绵延数州数县。

这些田地,僧人并不会去耕种,而是租佃出去。如果仅仅是租佃倒没有什么不对,但佃户的妻子往往会被僧侣强占,人称梵嫂。若是不从就是退佃了事,许多佃户不得不忍气吞声。到后来有些不成器的便是主动把浑家献给,以求个更好的佃田。在江南佛教兴盛之地,这样的情形不胜枚举,世人已经习以为常。

第三就是典当放债。所谓的质库,也就是后世的当铺,便是出自于寺庙,世称长生库。也许一开始还有帮信众临时周转的用意在,但到了如今,已经完全成了一门财源滚滚的大买卖。而放债也是一样,利息与世间平齐,追债时也没几个还会记得慈悲二字。

由于庙产不须缴纳赋税,而僧人也不用服徭役。有了张度牒,再把家中田地店铺挂到寺庙的名下,就可以安安心心的享受没有税赋徭役的幸福生活——许多寺庙都提供这样的服务,并不会乘机吞没产业——这也是为什么一张度牒能卖到三百贯的原因所在。

有心事佛的,没钱剃度。而有钱剃度的,则只是为了做了和尚后的好处。占尽天下便宜,还有着一分道貌岸然的模样,这让韩冈如何能看得顺眼?

不过韩冈对于个人和阶级分得很清楚。僧侣这个阶层已经腐烂透顶,但其中却有不少有真才实学的人物。真定高僧怀丙,以工程技术著称于世,他用两条船从黄河中拉出八匹铁牛的事迹,千年后韩冈都在教科书中学到过,而他修复赵州桥、修复倾斜的木塔,也是在此时传说甚广的故事。他成名在仁宗朝中,如今应该仍尚在人世。

针对智缘这个人,韩冈也同样很欣赏。能放弃在京城的名望和地位,来到古渭这个荒僻之地,为大宋的扩张而尽一份自己的力量,实在是很难得。虽然他所宣称的弘扬佛法只张幌子,本质上还是为了立下一份功绩,籍此取得更高的地位。

但缘边安抚司中,又有谁人不是这样,韩冈不会因此而求全责备,反而多了分认同。他奉王韶命陪着他往青唐部做水陆道场,几天下来,两人谈天说地,刚见面时的一点不快已经不见踪影,

说了一阵闲话,智缘将马身向韩冈凑近了一点,避过俞龙珂和瞎药各自派出的一队护卫的耳目,压低了声音道:“机宜,贫僧这两日观俞龙珂和瞎药兄弟之间似有隔阂,恐有萧墙之乱。若是能从中调解,也许就能让他们对朝廷更加顺服。”

韩冈露出一丝不出所料的笑意,智缘这分明是在试探。不过以智缘的眼力,通过这几天的观察,看出青唐部的两位族酋并没有真正投向朝廷,也是应有之理。他也无意隐瞒伪饰,智缘才智甚高,能算命的眼力更不会差,瞒是瞒不过去。

他便摇了摇头,叹息道:“不瞒大师说,当初若不是利用俞龙珂和瞎药之间的不合,我也不会那般容易就说动了俞龙珂,更不会有后来的古渭大捷。不过两人都是奸狡之辈,互相之间虽有争竞之心,却不会失了法度,有些事他们再想跟兄弟别苗头,都不会去做。”

“原来如此,却是贫僧莽撞了。”智缘对韩冈合十行礼,“多谢机宜将此事相告。”

韩冈并不介意把青唐部和朝廷的真实关系透露给智缘。反正俞龙珂和瞎药已经把青唐部的田籍名簿都献了上去,表面文章做了十足十,任凭智缘有几张嘴也不可能把这件事给扳回来。而他若是将此事散布出去,反而会惹怒举荐他的王安石,如此不智,谅智缘也不会去做。

“大师说得哪里话?既然皆是为了国事,韩冈哪还能瞒着大师?关于河湟之事,只要大师相问,韩冈知无不言,言无不尽。”

“多谢机宜。”智缘又谢了一句,脸色泛起淡淡的喜色,自忖这几日的辛苦没有白费。

韩冈赞着智缘:“大师几日来为国事殚心竭虑,无论是在吹莽城,还是在青唐城,蕃人都已是对大师顶礼膜拜,若大师日后将佛法传遍河湟,可以想见,各家蕃部当会纷纷来投。”

智缘‘阿弥陀佛’的感叹了一声:“却是远远比不上机宜。”

这些天来,智缘对韩冈在两家蕃部受到的尊敬都是看在了眼里。几乎每一个蕃部子民都认识他,都会对他合十行礼,甚至有些人一见到韩冈便跪下来叩拜。就算是俞龙珂和瞎药,还有张香儿,对韩冈也同样是恭谨有加。这不是普通的汉家官人能得到的礼数,智缘在这些蕃人的眼中看到的,是对韩冈的敬仰和崇拜。

智缘已经打听过了,这是因为韩冈传说中的身份,药王孙思邈的弟子,这个名号让人听了就不得不崇敬三分。而韩冈创立的疗养院,救治了数以百计的蕃人,不仅结下了一段段的善缘,也让他的身份更加得到世人的认同。

韩冈完全不通医术这一点,本来应是缺憾,甚至是致命伤,却因为他一直都在否认,反而没了人去在乎、拿着说事,而且这还更增添了他经历的传奇性——不通医术的药王弟子,可比药到病除的名医更为稀奇。

韩冈自己也很清楚这一点,智缘称赞他的时候,他也不过是道了声,“哪里,哪里。”几乎是全盘接受。

随着古渭疗养院的名声日渐扩大,他在吐蕃各族中的名望也日渐加强。单是药王弟子——不,应该是药师王菩萨驾前侍者的身份,就能让他在蕃地通行无阻。

尽管老于世故、精明狡猾的族长们不会因为这个神奇的光环而向韩冈俯首帖耳,但他们终究还是拥有一分敬意,不敢对韩冈有所得罪。假以时日,韩冈有自信凭着他的声望,能说服绝大多数的蕃部投向大宋,并不需要智缘再来多事。

“今次回到古渭,稍作休整,贫僧就可以往星罗结部去了。相信以王安抚和机宜的威名,别羌星罗结当不敢阻挠官军。”

韩冈点了点头,“到时就要劳烦大师了。”

有了这六天的时间,军队、民伕、钱粮、军械,所有的准备都已经到位了。也就在这一两天,从渭源堡派来的信使,将会带来星罗结部骚扰渭源的紧急军情。这份加急情报,能为苗授接下来的行动,给出最好的理由。而他韩冈也将和王韶一起奔赴渭源,在最近处见证他们计划的胜利。智缘虽然求功心切,在今次之事上,也只有做旁观者的份。

青唐到古渭,翻山即至,也就一个多时辰的路。回到阔别三日的古渭寨,王韶立刻向韩冈通报了最新的军情,不过不是渭源的,而是从秦州而来:“西贼日前已经尽起国中大军,号称三十万,由梁乙埋亲领,兵发五路。甘谷昨日已有狼烟,渭州蔡经略也遣急脚,向秦州通报了有西贼万人攻打原州。环庆和鄜延虽还没有消息,不过当是西贼主力所向。今次,西贼是倾巢而出。”

智缘面有惊容,而韩冈则微微一笑,正要说话,一名铺兵满头是汗的被人带进官厅。喘着气禀报道,“安抚,渭源急报。别羌星罗结起兵来犯,还请安抚速速派兵支援。”

