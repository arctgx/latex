\section{第13章 已入苍梧危堞远(下)}

十二名身穿红袍、敞着胸襟的号手,同时鼓起了胸膛,将手中的号角用力吹响。

从牛角军号中传出来的声音悠长嘹亮,有着激荡人心的力量,让听到军号声的人们,胸腔随之一起共鸣。

号角声长长的响了一段,带着悠悠尾声停了下来。片刻之后,又再次吹响。一连响了三遍,在桂州城外的奇山秀水上缭绕不绝。更像是在浸透了油料的柴草中丢了一支火把,城中因战事而阴郁已久的气氛立刻燃烧了起来,

城中百姓纷纷涌出城来,城上城下也都站满了人,人山人海的拥挤,甚至超过了旧年的上元之夜,人人兴奋得无以名状。

“援军来了!”

“王师来了!”

欢呼声中,一面‘章’字大纛当先打起,新任广西经略的名号就此亮出。随即李信的将旗也升了起来,紧接着一面面战旗在船头上展开,在江风中猎猎作响,移上了码头。随着各自的战旗,荆南军中的一千五百名精兵强将衣甲鲜明,一个个的从船舱中鱼贯而出。

身上的甲胄兵器,在下船时就分发好了。而生了病的将士,都是移到了最后的一条船上,不让他们的影响到用来安定人心、震慑交贼的华丽出场。

过千名身穿甲胄、手持刀枪的战士所组成的劲旅,就在漓江边的码头上,炫耀给桂州城的人们。

无数人冲着只有区区一千五百人的队伍欢呼雀跃。多少人朝着北方拜了下来,向派遣援军来拯救他们的天子遥呼万岁。

自从张守节在昆仑关全军覆没之后,桂州城中便一夕三惊。桂州城坐拥十数万军民,却生怕交趾贼军什么时候就杀到了城下。城门一天就只开巳、午、未三个时辰。就算这几日有贼人在光天化日之下,劫掠城外的民居,城中守军也不敢出战,只敢在城头上观望着,任凭贼寇得意的满载而归

号角再一次吹响,排在着严整的队列,跟着跨上马匹的章惇、韩冈和李信,从码头一直往城中走去。

从码头到城中,短短的一段路上,他们收到了无数声欢呼。几乎是第一次受到如此热烈欢迎的士兵们,兴奋得涨红了脸,更加趾高气昂的抬着脚,用力的跺着地面。

这一路行军,远不及后世阅兵式一般的水准,也比不上刚刚从战场上下来的西军,高唱着得胜歌凯旋而回的雄壮豪放。不过上过阵见过血的军队,行军时威风凛凛的模样,也足以震慑桂州城中的十余万官吏军民。

耐下性子,用一场威武的阅兵,安定了广西的军心民心。当章惇和韩冈一起来到州衙偏院的白虎节堂时,便又回到了现实中。

白虎节堂中,刘彝的身影已经不在了。收拾行装,等待章惇有空时与他做了交接,然后北上待罪,才是他的现在能做的事。

众官员中,章惇位份最高,远在仅为司封郎中的转运使李平一之上。他以翰林学士的身份出外,改了龙图阁学士,名义上还是做着了龙图阁直阁的韩冈的顶头上司。

“援军抵达广西的消息肯定已经传出去了,不过还不够。在外面要尽量宣扬,说朝廷已经调集十三万大军星夜来援,刚刚抵达的五千人仅仅是前锋,剩下的将会陆续抵达。”

如果是章惇说得,李平一肯定不敢质疑,但韩冈这位副手的话,他身为转运使就忍不住要说上两句,“贼性狐疑,李常杰听到这个传言,也许反而不会相信了。听说他曾在李佛玛军中用事时,用计活捉了占城国王,也算是略有智数。”

“相不相信随他去好了。这话是说给广源州蛮帅、还有左右江各家溪洞首领们听的。”抵达桂州后的行事方略,章惇和韩冈同行这么多天,早就已经商量好了,“今天经略司就要贴出布告。左右江两岸,胆敢附逆的部族,王师将犁庭扫穴,连根铲除。而先行投效为王师引路者,朝廷则不吝爵赏!”

“这是要分敌众,乱贼心。”韩冈解释了一句,“第一目标始终是交趾,先扫平升龙府,然后再解决敢于附逆的部族,要一步步的来。”

李平一听着目瞪口呆,章惇和韩冈对交趾要灭此朝食的态度让他有了更进一步的联想,“南征行营难道已经建立了?!”

“还没有,不过也快了。天子一怒,伏尸百万。嘬尔南蛮,竟然敢侵攻中国,不平灭其国,焚其王都,如何能当得起天子的雷霆震怒。”

“要为万世开太平,不扫平四荒蛮夷,哪里来的太平。”

李平一眨了眨眼睛,有那么一瞬间,他还以为眼前的两人是沈起和刘彝。

章惇抬眼看了李平一一眼,看透了他的想法:“我与玉昆都有便宜行事之权。”

正准备说话的李平一,顿时就紧紧闭上了嘴。

“邕州怎么办?”李信指着沙盘。

在白虎节堂正中央,有着如今正流行的沙盘——几乎每一个经略司中,如今都少不了沙盘,无论是不是喜欢军事,任何一位经略使看着自己治下的土地,都免不了会有种莫名的畅快。不过广西的地形沙盘制作得很粗糙,远远比不上关西的地形沙盘精细。但城镇、道路、山川的位置大体还是不会错的。

他们现在身在桂州,距离左江之滨的邕州城还有一千多里地。这个距离就算是急行军也要近半个月的时间,而且还要随时提防着敌军可能会有的埋伏,行军速度只会更慢。不可能像在驿站中不断换马,一天能跑四五百里出去。

“可惜不能走水路。”韩冈很是遗憾。通过沙盘上粗糙的表示,可以发现珠江的诸多支流连通着广西的许多军州。从桂州走水路其实也可以抵达邕州,不过是先顺水下行到浔州,然后再沿江上溯,要绕上一千多里的路,有一半的路程还要靠纤夫帮忙。在交趾贼军围困邕州的时候,走水路当然不可能。

“从桂州南下邕州,前半段也可以利用一下水路,不过再往下就要走不少山路,光是一个昆仑关就很麻烦。”章惇不会去祈求李常杰会犯侬智高的错误,想要顺利的杀过昆仑关,要么就是他已经撤退回国,要么就是与占据昆仑关的贼军来上一场血战。

“桂州这些日子应该已经紧急招募了一批新兵。”章惇将视线投向李平一。

“八千名。”李平一报了个数字,又忙补充道,“不过都是拿不惯弓刀的新兵,还算堪用的那些兵马大半都随张守节战殁在昆仑关了,剩下的也就三千一百多名老兵。”

“留着他们下来守桂州,玉昆你和李信先带着一千荆南兵马做先锋去宾州。少待时日,我就领军去与你会合。”

这也是韩冈和章惇视线拟订过的计划。韩冈会先去宾州看看能不能有机会救援邕州,运气好说不定还能将李常杰吓跑。如果不行,就在宾州将随军转运司的准备先做起来。

而章惇先行整顿桂州城中的政务,只有他这位经略使有足够的地位压制住城中。等到桂州安定下来,章惇就会带着路中主力一起南下,以邕州为基地,着手实行反攻交趾的计划。

“敢问运使,宾州粮秣情况如何?”韩冈问得很不客气,可李平一却不敢发作。闹得交趾北侵,刘彝就算了,肯定是完蛋,而他李平一的命运其实掌握在章惇和韩冈的手中。

方才章惇已经表明了对韩冈的支持,也明说了两人都被赋予了便宜行事的权力,李平一也不敢摆着转运使的谱。在朝堂上的话语权,他与当朝宰相的女婿和亲信没办法比。而且从自己的切身利益上,也要将邕州给救下来。

“宾州、象州的都有着一两万石存粮。”

“实数还是账册上的数字。”韩冈咬着紧紧的,帐册上的数字做不得准,就像兵籍簿上的姓名,有多少是从来不存在的幽灵,怎么都说不清的,与实际差得远了。

“宾州去年秋末,我曾经查过宾州的粮库,原本是七万三千石,但实际上则只有两万一千。所以我还参宾州知州一本,现今已经押去京中待罪受审了。”

“所以象州的一两万石也是实数?”

“两边的情况应该差不多,我是从宾州的存粮推测出来的。象州的账簿上是五万七千。”

两州的粮食加起来有三四万担,足以支撑起两万大军的打上两个月的仗了。再多,就要靠后方转运——桂州,甚至荆湖两路。

韩冈从白虎节堂出来,已经是满天星斗。昨日被浓云遮挡的星月,今天则在玉宇澄清的天幕中,闪耀着亘古不变的光辉。南方的群星不同于北方。在开封夜空中清晰可辨的北极星,已经在落在了北面的山后。而南面的夜空中则是有着许多北方人从未见过一次的星辰。

‘可要再撑几天啊!’韩冈的视线从星空中,落到了南面的山岭上。在那群山之后,是应该还在奋力拼杀的苏缄和他的邕州军民。一座孤悬在外的城市,已经在交趾人的优势大军中坚守了一个半月之久。相对于一攻就破的其他城寨,邕州城的坚持不论让谁人来评说,都是令人敬佩不已。

心中的话说出了口,送入了夜风中,“可要再撑几天啊……”

