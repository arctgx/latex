\section{第32章 荣辱凭心无拘执(上)}

春天的清晨,空气清寒。

胯下的坐骑呼哧呼哧的喘着白气,赶着上朝的蔡确在官袍下面套了件丝绵夹袄,又在外面披了件有些陈旧但质地精良的斗篷,但照样冻得手脚冰冷。

“持正,今天到得可早。”

听到声音,看到来人,蔡确在马上腰身弯了下去,与当今的东府第二号人物相互致礼,“身任台谏,不得不早。”

王珪提了一下缰绳,放慢了速度,身边的元随立刻会意的散开来一条缝,蔡确便会意的驭马靠了上去。

待蔡确上来,王珪身子稍稍向后仰了一点:“昨天总算是看不到弹劾韩冈的奏章了。”

王珪说得没头没脑,蔡确却了然一笑:“毕竟正事要紧,总不能为他事耽搁。”

王珪点点头,表示同意:“的确是正事要紧。”又问道:“韩冈当能成事吧?”

“当然,对韩冈岂是难事。襄汉漕运也不要多少,只需要一年百万石而已,正好就是荆湖的粮纲数目,江西江东的上溯走荆襄反而绕路。”蔡确的观点与前日截然不同,“荆湖的粮纲上京,不要再绕道扬州,免了几千里路,省下多少时间,就是中间靠轨道转运,也能省下不少运费。”

王珪感慨道:“所以天子要保着他。”

蔡确失笑:“要是再盯着韩冈,乌台上下都能坏在他手上。”

尽管天子对弹劾韩冈的众官处罚甚重,但愿意飞蛾扑火的监察御史也不会减少多少——本来天子选御史,都是选着愣头青——其实直言敢谏也是个光荣,只要自己弹劾过重臣,日后就是资本,这证明他们忠于职守,不畏强权。

但这样的情况下去,事情就麻烦了——不是韩冈麻烦,而是蔡确这边有麻烦。

万一安排好的人选因为弹劾韩冈出了事,预定计划就全都会被打乱。但硬拦着也不行,蔡确自己也会被当做奸佞牵连进去,为了合情合理让下面的言官们放低调门,蔡确可没少费口舌。

王珪和蔡确同行,路上的官员看到王珪的旗牌,就立刻避让道旁,黑黢黢的凌晨,也看不清执政元随护持下的究竟是一人还是两人。恐怕也没什么人想到,王珪和蔡确之间,还有着私下里的联系。

并辔而行的两人当然不能算旧党,但也不是新党——尽管蔡确本人看着有些像——而是天子偏向哪里,他们就跟着倒向哪里,也许称为帝党更为合适。

相对而言,王珪表现得更为贴近皇帝,对天子惟命是从。蔡确则是会玩些小花样,比如旧时弹劾王安石,比如如今坚持新法,表现出自己独立人格的同时,其实也是在希合上意,让天子感到满意——相对于聪明全都放在了学问上、政治头脑完全是个悲剧的沈括,他的手腕强出不知多少倍。

而韩冈在他们眼中是同类人。与新党若即若离,与旧党千丝万缕,两边都不依附,只讨好天子一人。只要能让如今的至尊满意,地位便是稳如泰山——当然,韩冈讨好天子采取的是累积功劳的方法,这一点,与任何人都不一样。蔡确不觉得自己需要学韩冈,也不认为自己学得来,但只要带来的结果相同,手段是无所谓的。

走了几步,王珪有出声问道:“邓温伯和上官均还是要保大理寺?”

蔡确答非所问:“黄履为人中正敢言。”

王珪点过头,也是跳着说话,“相州一案,失入死罪,陈安民不知自省待罪,反而胆大包天,贿赂法司。文及甫、吴安持,事涉干请,败坏国法,皆当从重。”

“参政之言,正是公论。”

相州一案,是以劫盗杀人的罪名,判了三名案犯死罪。不过依照审刑院之后的复核,这是个错判的案子,两名从犯不当论死。可这时候,从犯皆已被处决,已经来不及挽回了。出了人命,这个错判性质就变得十分严重,参与审讯的官员绝不是罢官能解决的。

当初审理此案的陈安民,他年纪差得有些远的亲姐姐是文彦博的儿子文及甫的生母,同时文及甫又是吴充的女婿。陈安民为了消灾弭祸,一边让当时参与此案的相州发司潘开带钱上京活动,一边则是发动自己的关系,求一个平安。

而这件事,就给蔡确抓到了把柄。相州一案事小,而法司受贿则事大。蔡确想往上走,唯恐事情闹不大,捉了多个有品级的官员进了御史台,文及甫和吴安持都被牵连进来。

御史中丞邓润甫见状则是想大事化小,不想闹得太大,给了天子党同伐异的感觉反而不利于新党,而且蔡确对邓润甫来说也是个威胁,他早想借机打压蔡确一下——台谏官一向并称,以御史中丞为首。蔡确作为谏院之长,头上就只有个表字温伯的邓润甫了。

但邓润甫并不知道,王珪和蔡确之间有了份协议在。

蔡确和王珪两人很简洁的交谈了几句,重申了各自的态度,便立刻分了开来。两人分别担任执政和言官,交情不能好,见面聊个两句就算尽了人情,话说多了,天子那里就难交代了。

且默契早已经形成。王珪想要吴充的位置,而蔡确则盯着邓润甫的位子,合则两利,自然不需要多余的试探。

蔡确离开执政官多达数十人的队伍,很是羡慕的又望了几眼,想想,又敲着马鞍向黑沉沉的西方望过去,脸上带着点笑:‘韩冈该到洛阳了,文彦博最近心情不会好,有的是让他头疼。’

……………………

韩冈已经抵达了洛阳城。

作为京西都转运使,有监察京西一路官员的职责——所以路一级的衙门,经略司、转运司、提点刑狱和提举常平,都被称为监司——官名之后的‘使’更是表明了他代表了天子对京西监察,地位当然要比正常知州要高上一级。

判河南府的文彦博没理会韩冈,这很正常,如果是文彦博迎出城来,韩冈甚至得绕道进城躲着他走。三朝元老、前任宰相、太子太傅、资政殿大学士加上潞国公,这份礼数也只有天子或是两宫有资格接受,连皇后和嫔妃都不够资格,何论韩冈。

但河南府通判不出来,军事判官、节度判官、录事参军不出来,河南知县不出来,甚至连理应有的当地父老出城相迎的场面都没有,事情就做得未免太过分了一点。

韩冈的随行人员一个个怒形于色,他的幕僚方兴,还有十几个投奔到他帐下的同门脸都气得发青,这个下马威给的太黑了。

但韩冈也不能为此向天子抱怨,只要文彦博随便拿个公务繁忙的借口,就能轻易的搪塞过去,天子即便明知是谎言,也照样会帮他这位元老大臣给敷衍一番,一个都治不了罪。就像先前牺牲御史,只为给韩冈一个交代;到这时,就会牺牲韩冈的脸面,给文彦博这个元老一个面子。

幸好也不是完全没有人出城迎接,转运副使李南公,带着转运司中一应属僚,就在离城十里的递铺处等候韩冈的到来。

京西南路、京西北路刚刚合并,但两路的漕司衙门还没有合并起来,都转运使只有韩冈一个,但转运副使一南一北,各有一人。

转运副使李南公出城来迎接韩冈,看到洛阳府县上下都没人出来,心中当即是叫苦不迭。下面的漕司属僚也都变了脸色。

韩冈年纪轻轻就做了都转运使、龙图学士,这份际遇,文彦博都远远比不上,年少气盛,哪里可能会咽下这口闲气。文彦博一点面子都不给韩冈,那韩冈也肯定少不了找文彦博的麻烦,他们夹在中间可是少不了要吃苦头了。

当年知河南府的李中师,与富弼有私仇,就变着法儿的跟富弼过不去,甚至还派了吏员上门去催免行钱。到后来李中师被调走,但被他驱使的属僚和吏员却调不走,不知吃了富家的多少苦。

走得近了,李南公他们便发现韩冈的随从们一个个都是怒发冲冠的样儿,心中便都是咯噔一下,大叫不妙。可是当他们看着人群中间身着紫袍、腰围金带的年轻官员,却一点也不见生气的样子,微笑着不知在说什么,看模样像是在安抚众人。

李南公带领转运司一众官吏到了韩冈近前,齐齐的向韩冈躬身行礼。转运司中官员人数不少,有二十多人,除了留守的两三个,剩下的全都出来迎接。

韩冈早已经下马,向李南公回了礼,让属官们全都起身,便很是亲近的拉着李南公上马同行。

李南公偷眼看这韩冈的脸色,小心翼翼的避开危险的话题,先赞了一番韩冈过去的功业,又道:“京西两路新近合并,衙中诸务正待龙图提点。下官已经将籍簿帐册都整理出来了,等龙图到了之后,就可着手查验。”

“此事不急。想必楚老也清楚,”韩冈亲切的叫着李南公的表字,“天子不以韩冈年轻识浅,特调来执掌京西漕司,只是看韩冈在土木之事上稍有所得,至于漕司中的事务,还须楚老能者多劳。”

