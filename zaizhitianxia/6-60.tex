\section{第七章 烟霞随步正登览(12)}

韩冈家中此时正喜气洋洋。

韩冈先就任枢密副使,再转任参知政事的消息,还不到中午,就传到了韩家。

随即存放在家中的仓库里,韩冈做枢密副使和宣徽使时所用的清凉伞,就被拿了出来。虽然存放的时间并不长,但好几名家丁拿着手巾上上下下擦拭着,唯恐漏下一点灰尘。

更有几名韩家的仆役,迫不及待的将朝廷给重臣亲随配发的朱衣给穿在了身上。

之前韩冈引罪从宣徽北院使的位置上退下来,清凉伞不能打了,而原本跟随在枢密副使和宣徽使前后的亲随,也依制削减,从五十、七十,降到了个位数。

虽然说韩冈这段时间里面,他对朝堂的影响力并未降低,尤其是在平叛之后,更是望隆朝野,但对于韩家的家人来说,还是明确的官职更能让他们扬眉吐气,与有荣焉。

但更多的准备就被王旖制止了,反而紧闭门户,杜门谢客。

韩家不是小门小户,韩冈更是前途无量,就算是就任参知政事,也算不得什么大事。太过轻佻,反而会让人看轻了。

这时候听到消息,赶来送礼的人越来越多,王旖也不想让家里成为东京城中的笑话。

除了厨房里多准备了几道酒菜,就没有特别的庆贺了。

素心已是难得下厨,不过听说了韩冈就任参政,便多做了几道。

而王旖、周南、云娘也换了新衣。对外当然要维持稳中,但家里面庆贺一下却是应有之理。

不过韩冈没有在放衙的时间,到了初更方才进门。

“参政回来了!”

司阍通报的声音比平日大了几倍,顿时惊动了全府上下。

外院的门客、管家、仆役齐齐罗拜于院中,恭喜韩冈得授执政。

等韩冈受过礼,回到后院,王旖已领着妾室和儿女在内院大门处等候多时。

“官人今天怎么这么迟?”王旖接过韩冈的外袍,问着。

“门口太多人,被堵着了。”韩冈笑了笑,又道:“之前先去看了一下张老太尉,所以迟了点。”

虽是初至东府,不过今天放衙,韩冈并没有耽搁时间。但他先去了医院一趟——张守约正在那里休养,

李信的第三个儿子,与张守约的嫡孙女定下了亲事。韩冈与张守约也算有了姻亲。

再有几天张守约就要回家养病了,那时候,已经是参知政事的韩冈,反倒不方便登门探问了,还是赶在出院前最好。

“老太尉怎样?”王旖关切的问道。

“快出院了,情况当然很好。回家慢慢养着,再有些日子就能彻底康复了。”

经过了一次手术,取出了体内的箭矢,张守约顺利的撑过了失血和感染两关,眼下正在逐步康复。

这位老将,虽说年纪大了,可毕竟底子好,眼看着再过些天就能自个儿给站起来了

“就是不能再上阵,也不能再为朝廷办差了。”韩冈叹着气,“年纪不饶人啊。”

不论张守约再如何恢复,也很难重新回到岗位上了。给他治疗的御医,背地里都对张守约家的子弟,以及韩冈和奉旨来探问病情的内侍,说了张守约的情况。不能再日夜镇守宫阙了,必须好生的休养。

张守约自请致仕的奏章前两日便递了上去,向太后则是立刻驳了回来。不过给张守约致仕的封赠都准备好了,等到他能够站起来走动,再上几次请老的奏章,太后就会答应下来。

张守约致仕后的赠与,是太尉兼节度使,而且是节度使排名第一的泰宁军,超越了郭逵刚刚得授的武胜军节度使。

就像国公分大国、中国、小国之类的等级,军镇也有等级之分,从最高泰宁军的到最低的大同军,其地位高下,看朝会上的排位就能清楚了。归德军节度使其实比泰宁军更高,其辖区就在前身宋州的南京应天府,但那时是太祖皇帝曾经的岗位,又是大宋国号的来源,所以从不授人。

张守约因伤而退,所以才会是特旨恩授泰宁军节度使。郭逵虽然功劳不再张守约之下,但临阵受伤本就是要加功,张守约又是致仕封赠,授予泰宁军节度使,郭逵也无话可说。

同时张守约家里,不要说儿子,就是所有的侄儿、孙子,只要还未得到荫补,都被授予了官职。

不过韩冈也看得出来,张守约并不甘心,一如廉颇、赵充国,仍想征战于战场之上。

想起张守约,韩冈的心情就有些沉郁,转过话题,问王旖道:“岳父那边可有什么话?”

王旖闻言脸色一黯,勉强笑道:“二哥哥来过了。”

果然还是拗相公。韩冈对王安石的脾气也是无奈。

王安石的脾气,韩冈也不打算惯着。他有他的目标要实现,不能一直让着王安石。

“官人……”王旖担心的看着韩冈。

“放心吧。为夫也不会与岳父争吵,当面的时候,哪一次不是为夫先让着岳父?”韩冈笑容也微微有些苦涩,“但岳父的脾气也越来越倔,为夫这个女婿不求沾光,也只求能如其他官员一般。”

王旖脸色黯然,一时默默无语,韩冈看了,有点心疼,忙笑道:“真要说起来,岳父可欠着为夫不少人情,要不是看在他送了一个女儿过来,为夫可是连本带利的都要计较的。”

王旖扬起眉,脸上的阴云尽散,嗔道,“你还不计较!”

韩冈笑了,家中和睦才是喜事。不过,也幸好王安石只是平章军国重事。

进房换了一声家居的衣服,妻妾们服侍着韩冈歇了下来。

素心、云娘开始去布置席面,周南去安顿儿女,王旖则拿出一摞求见的名帖,交给韩冈。韩冈看了两眼,便没多少兴趣的放了下来,考虑起接下来的安排。

虽然在中书门下只待了半日,但其中公务,远比枢密院要忙碌得多。而韩冈初来乍到,第一紧要的便是要熟悉公事,另一个则是要把握住一干人事权。

不过并不是什么职位都能拿到手。

至少审官东院、审官西院、流内铨、三班院这四个有关官员注授的衙门,也就是铨曹四选,新党绝不会放弃。

韩冈也还不打算与自己的岳父和一干老朋友翻脸。而且他手中能派上用场的高官就那么几个,又有新旧之争,总不能让范纯仁这样的旧党去管人事。

韩冈估计,自己能到手的只会是一两个重要的卿监,以及五六个不那么重要的下级衙门。加上一直在自己手中厚生司、太医局,也许还能有钦天监,可以控制的衙门也不会太多。

或许自己这个参知政事,将会是主管教科文卫?

只是教育也难说,要是自己敢打国子监的主意,自家的岳父敢跟自己拼命——当然,国子监这种以培养官吏为目的的学府,与后世主管教育的部门完全不是一回事。

但军器监的控制权一定要从新党的手里面给拿回来,这是现在韩冈最想要的位置,若是能再加上将作监就更好了。

拿到这两个位置中的一个,就能安排一下投效自己的重臣。

比如那个最不起眼,才干却明显得要胜过大部分人的王居卿——韩冈没办法满足每一个人的要求,也没打算去酬谢所有人,一切还是以自己的需要为主。

王居卿的都大提举市易司,说实话,实在有些欺负人。

市易司只是三司之下的一个衙门,让一个侍制委屈在这个位置上,一方面说明市易司的重要,但也代表王居卿在新党之中并不受到重视,否则一个侍制有足够多的地方来安排。

而韩冈手中位居高位的人才不多,王居卿就不能浪费。这两天找他见个面,若没有什么问题,韩冈便打算将其调到军器监任职。

不过投给自己一票的那几位,却都不是一个判军器监就能打发得了。

尤其是太后在殿上明说再过半月还有一次廷推之后,想必很有几人心动了。

到底谁会先来?

韩冈端起一杯热饮子,慢慢的抿着。

从票数上看,韩冈这边依然远比不上新党。

虽然说只要能够进入三人之列,就有可能被太后选上。可只要沈括、李承之他们参选,票数立刻就不够了。

而且韩冈也不便强令他的其他支持者,支持李承之、沈括。

范纯仁、孙觉、李常,这几位推举韩冈还有着充足的理由,但应韩冈的要求去推举新党,就纯粹是韩冈本人的私利了。

以这几位的为人,他们又怎么可能会听从韩冈的要求,去给一贯站在新党一边的李承之或沈括投票?

想也知道那是绝对没有希望的,孙觉、李常、范纯仁不啐口水就是涵养好了。

不过下一次的廷推是在半个月之后,在京城中的重臣名单,也会有所改变,说不定到时候,就会跟今天的名单变得截然不同。

到时候,说不定还有一番较量。

不过韩冈很乐于见到又多了一次廷推。

举行的次数越多,习惯得就越快。等所有人都习惯了,也就成为了定制。

向着他的目标,就又迈进了一步。

