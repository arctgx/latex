\section{第29章 坐感岁时歌慷慨(下)}

“这一次回京,还以为天子会想着对西夏开战。交趾都灭了,西军的战力天下人也都看到了,不是说北方的禁军,都已经全数配发铁甲了吗?斩马刀和神臂弓也是几千几千的押送进军库。有这十万虎贲,杀到兴庆府都没问题。兴庆府中还争得不相上下,也差不多该是时候了。”王厚长长的叹了口气,摇头间满是无奈,“爹爹你管着熙河秦凤和泾原,种五管着鄜延、环庆,郭太尉自河东,几方合力,西夏也就能撑个一年半载……不,三五个月而已!”

“熙宁八年的正月板甲局创立。到上个月为止,总计造甲四十一万六千八百余具。斩马刀,十二万两千四百余柄。神臂弓更是有六十多万具。除此之外,飞船,霹雳砲,都是有足够的储备,军中马匹,靠着这些年的茶马互市,光是关西就有了十万余匹,其中战马就有三万。关西和河东的禁军,的确什么都不缺了。但河北军和京营还没有训练完毕,至少还要个两年左右。”王韶顿了一下,声音低了点,“西军太强了,五千灭国。也算是西军一脉的荆南军,则是千五破十万。河北和京营不练起来,谁都不能放心。”

王厚双眉一挑,正要说话,王韶抬手阻止了他,“别忘了,还有契丹人呢。”

王韶对于西夏的政局看得比他儿子要清楚,“梁家根基已深,此前几番大败,反而让他们趁机整顿了国中,秉常背后虽有契丹人支持,也不是那么容易就能赢的。而且秉常才十六七,梁家找个借口还是能拖上几年时间,估计要到他二十岁才会闹出来。不过……”王韶露出了一个笑容,“你若是进宫面圣,还是照样该说什么就说什么,不必有所顾忌。”

“……儿子明白了。”

……………………

远在东京城西南方数百里的地方,一座小城的驿馆中,韩冈正在与来访的监察御史会面。

在韩冈的记忆中,御史一般的都是傲气凌人,就是在宰执面前都只维持最基本的礼节,因为他们是天子用来制衡相权的工具,不需要对宰执们太过敬畏。不过舒亶倒是很是有礼貌。

御史礼数周到,韩冈也不会生生受下,还了一个平礼,到了几句久闻大名,便请了舒亶,在小厅中坐下。

监察御史是风闻奏事,说话不需要有谱,不需要为自己说的话负责,咬上谁谁就倒霉。说句难听话,就是贼咬一口入骨三分,不论有理无理,即便是宰相也得先避位待查。正常的情况下出京的可能不大。现在舒亶跑出来查案,自然不可能是小事,当是想用铁证将某人给钉死

韩冈神色间不见任何异样,与舒亶分宾主坐下来聊着闲话,心中则是揣测着,不知他又盯上了两府中的哪一位了。站在他身后的吕惠卿或是章惇,又是将目标投向谁?

只要不是自己就行了,韩冈想着,他一个都转运使离着东京城远得很,天上乱飞的石头,砸不到他的头上。

“前岁岁中熙河水患,之后家严在信中说,若非有熙州、河州、岷州新辟的四百余顷良田,是岁军需几乎不保。而信道兄与其中出力良多,熙河军民一说起舒管勾,听说是无人不赞。”

“龙图的夸赞舒亶可不敢当。在下在熙河,多得尊翁襄助,且也是给郑提举辅佐而已。”

“这是哪里的话,郑民宪提举营田务不便远离巩州,家严又是老迈,岷州、河州之地,可都是信道兄的功劳。”

韩冈的开场白,骚着了舒亶的痒处。他去熙河路担任营田司的勾当公事,的的确确很卖了份力气,也是他由选人转京官的主要依据。不过若说功劳,还真比不上韩千六那位老农官,只是占了身为进士的便宜。

他瞅着韩冈,年轻的面庞因为久在南方而被晒得黝黑,眉眼和鼻梁有些太过硬朗,但微微笑起来的时候,便显得温和从容、和善可亲。

能做到一阁学士,往往都是四五十岁之后,资历、经验、人脉和才干,再加上天子的青睐,才能有幸得到学士的头衔。如韩冈这样,完全功劳堆起来的,完全是独一无二的特例。

这样的人,在待人接物时没有半点傲气,表现得谦和有礼,让舒亶感到惊讶无比。以他的眼力,并没有发现韩冈的谦逊是伪装而成,而是当真是发自于内心。

要么是韩冈的性格当真平易近人;要么就是他虚伪过人一等;还有就是他已经习惯了眼下的身份,不需用高傲来彰显自己的地位——这在遽得高位的寒门子弟中很少见。且不管是什么理由,韩冈表现出来的态度让人愿意与他交流。

两人又说了一阵拉近关系的闲话。韩冈总是在说着自己在熙河路和南方的见闻,对舒亶出京的缘由则半句不提。

但舒亶有些不耐烦了,“龙图在交州所立功勋,舒亶一直以来都是感佩不已。我等生在东京,却难以想象交州的艰难。”

“上有天子福佑,下有将士用命,中间还有章子厚的指挥之力。”

“此乃百年不见的盖世奇功……不过河湟开边两千里,其功不输收服交州多少。”舒亶感叹着,“河湟开边之后的献俘阙下的大典,在下无缘一见。但为了交州收复的献俘和进献图籍的大典,在下可是从头看到的尾。宣德门城楼上,天子朝臣在上,罪臣在下,周围人山人海,那是再好的丹青圣手都难以描画的场面。”

舒亶啧啧感叹着,韩冈笑呵呵的说着:“若能平定西夏,将梁氏和秉常一起,场面只会更加宏大。”

人心隔肚皮,韩冈前后两段人生,在世上打滚得久了,对舒亶从甫一见面就有几分提防,当然不会随便相信他说的什么话。谁知道自己随口说出来的话,会不会在未来的哪一天变成了他弹劾自己的罪状?只是韩冈的态度热情得很,让人完全看不出来他对舒亶的戒备。

“章子厚如今身列西府之中,如有出战西夏,他可少不了在其中了。”

“韩冈也是一般。此次得授京西都转运,便有重启襄汉漕渠之事。若能荆襄入京的通道打通,日后东京一城就不用全压在汴河之上了。韩冈自入官来,承蒙天子不弃,多委以重任,一点微末之功,也不吝爵赏。此番当皆心尽力,以报天子殊恩。”

舒亶本以为提到章惇,韩冈会有个反应,无论喜怒,他都能跟着说下去。他没料到韩冈根本就不接话茬。

只见韩冈都是东拉西扯,根本就不理会舒亶。到最后百般无奈之下,舒亶也只能选择告辞离开,不敢再跟韩冈闲扯下去。

送了舒亶出了小楼,韩冈返身回了楼上。

“当不是吕参政让他来的吧?”

方兴已经不能算是韩冈的幕僚,在韩冈方才见客的时候,他回去了自己的房间。当舒亶告辞之后,他才从房间里出来。听了韩冈的转述,他觉得有些不可思议。韩冈与吕惠卿的关系并不和睦,甚至有旧怨,这一点,方兴也是知道的。

“吕吉甫没这么糊涂。”韩冈很肯定地说着,只是他本人也没能想明白到底是怎么回事。

舒亶虽然说得隐晦,但也是在劝说韩冈出面支持势头陡然低落的新党。吕惠卿当不会让他这么做,而章惇,则根本不用他代劳。

“既然舒亶已经说出了口,龙图打算怎么做?”方兴问道。

韩冈笑了一笑:“章子厚新立殊勋,怎么都不会动到他头上。”

也就是说,吕惠卿怎么死都没关系。虽然之前方兴已经隐隐觉得事情的确如此,但现在还是为韩冈对吕惠卿的冷淡而感到惊讶。吕惠卿虽然与韩冈没有培养出任何交情,但他毕竟是王安石离任之后,坚持将新法保持下去的首要人选,韩冈这位前相国的女婿,怎么连新法的存续都不放在心上?

对于这件事,看方兴的样子,就知道当是被他误会了。但韩冈无意多加解释。而且他也不认为,新法会有什么危险。若是吴充、吕公著欲废新法,只要让他们看看国库就行了。已经习惯了丰厚的钱粮供给的军队和官吏体系,怎么可能会愿意回到过去,闹出事来,天子都要拿他们来安抚人心。

韩冈拒绝吕惠卿,也只是在确认新法可以安然度过难关的基础上,不想被人用作马前卒罢了。反正他也不怕吕惠卿能将他如何。

这是韩冈的底气。他现在都已经是都转运使、龙图阁学士,做到宰执,除非是有意外,否则就只是时间问题。

而且他现在不能进京为官,是当今天子挡在他的面前。就算讨好了任何一位宰执,是能进两府呢,还是做翰林?都不可能!没好处的事,他疯了才会为人冲锋陷阵。

方兴仔细看着韩冈的神色,知他心意难改。便放弃一般地笑道,“不过京城里面的水还真够浑的,隔着五百里,浪头就扑过来了。”

韩冈笑了一声,“打破了过去的平衡后,要重新找回平衡差不多要一年半载。”

“龙图接下来打算怎么做?”方兴生刚停口,响起了什么,连忙补充,“我是说舒亶那里。”

“那要看他自己打算怎么做了。”韩冈冷淡的说着。

韩冈现在已经够资格拉拢人了。身为一路漕司,只要有他的一份荐书,任何一位选人就能在改官的道路上踏上一大步。

舒亶做为监察御史,虽然不需要再为沉浮于选海而苦恼,但与韩冈有着良好的关系,就意味着日后能得到一个强援,就看看他的心理能不能拐过这个弯了。

