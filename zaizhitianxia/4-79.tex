\section{第15章 焰上云霄思逐寇(十)}

子夜时分,被阴云遮挡的天穹上依然没有一丝星月之光。但如墨染过的夜色中,忽然又多了一条如同星河一般闪亮的灯带。

就被在归仁铺的方向上,一条由无数道火炬组成的光龙亮起,自宋军的营寨中蜿蜒而出。冒着已经若有若无的细细雨丝,迅速的向北奔去,而就在这条光龙离开的同时,宋军在归仁铺的营寨则整个陷入了黑暗之中。

交趾营地骚动了起来,值夜的士兵,宋军这是要逃了?

李常杰从睡梦被唤醒,一听之下,披了外袍就直冲了出来。望着已经与夜色融为一体的归仁铺,还有那条将宋军营中所有的光亮一起带走上路的光龙,李常杰心中满是疑惑。

因为生擒了一名宋人探马,还有活捉了黄金满派去说服刘纪三人的使者,从他们嘴里撬出了归仁铺宋军的底细,与此前的消息对照过后,李常杰放心大胆的又从后方调来万余大军作为臂助。虽然刘纪、申景贵和韦首安仍在推三阻四,但得之宋军底细的他们,反叛的可能性已是微乎其微,由宗亶这位出身广源州的主帅盯着他们已经足够了。

面对只有区区八百兵,却让自己落得丢人现眼地步的宋人,大越国的辅国太尉发下毒誓,要将他们尽数埋葬在归仁铺。只是出乎意外的,宋人的反应竟然如此迅速,迅速到李常杰心中生疑的地步,总是觉得有几分……不,应该说十分可疑。

点着火炬离开,又将归仁铺的灯火全数熄灭。这不是放声宣扬自己要逃离吗?哪有这样的撤退。说不定离开时的火炬光流,只是一个假象而已。宋人依然潜伏在黑暗之中,等待着他李常杰自投罗网。

宋人做得出来。

在从不同角度了解过这一支宋军的行程、经历,李常杰深深体会到了这一点。

与其说统领宋军的韩姓广西转运副使胆大包天,不如说他是一个疯子。从离开桂州,一路直奔南下,中间连片刻休整都没有。尤其是到了宾州之后,行事更是激进。灭刘永、夺昆仑,破长山,最后一举夺下了归仁铺,又在归仁铺设下营寨固守,连停下来歇歇脚的都没有一次。这一次也许也是做着死中求活的打算,试图谋取一个胜利。

不过话说回来,李常杰他当初也是从钦州登陆后,一路攻城拔寨直奔邕州,中间也同样没有休息,而帐下的士卒没有一个抱怨。只要士气高昂,一点疲累根本影响不了战力。不过若是败阵,或是遇上鏖战,这种强催起来的士气,很可能一下就降到谷底。李常杰有过这样刻骨铭心的体会。

宋人那边当也是如此,在连续胜利之后,有着充分的士气,但当他们要面对两万大军的围攻之后,也不可肯再维持着士气。更有可能是一个假象,只是为了让自己犹疑不定而故意。史书中不是有增灶减灶的战例吗?虚虚实实,本就是用兵的法门。

不过要是这么一圈圈的绕下去,事情都没有个了局。

问题归结到最后,就是一个简单的选择:

追,还是不追?

没有犹豫太多时间,李常杰很快就下了决断。

他将两万余大军调来此处,不是为了将宋人给吓走,他不会为此心满意足。不论宋人打的什么主意,他都没有放过他们的意思。

新到的一万兵马不便动用,他们在雨中走了一天,必须休息一夜,只能明天白天再追上来。

至于伏兵的问题,难道宋人会以为自己连个斥候都不派?

只比宋军慢了半个时辰,交趾军也有了动作。

首先出动的是作为斥候的骑兵,向着归仁铺的宋军营地进发。虽然在黑夜中奔驰,很容易因各种意外摔下马来,但就算小跑着,也会比步兵的速度更快一点。

而步兵也在同时出动,一个指挥一个指挥的离开大营,循着不同的路线,向着北面扑过去。

……………………

“李常杰果然出来了。”

遍布荒野之上,是密密麻麻、数也数不清的火光。看着心惊胆跳。

在官道上串联成一线的火炬只不过是一条星河,而归仁铺南面的原野上,则是群星汇聚的天幕。上万交趾兵从营地中杀出,然后在原野上扩散开来,更像猛砸过来的惊涛骇浪,要将大地给掩盖。

夜色遮蔽了交趾兵的身形,只能看见无数火光占据了整个视野。由光织成的洪流汹涌澎湃,比起白昼时的更为慑人心魄。

“人马一多,当真就让人望而生畏。”

听着韩冈仿佛事不关己的评价,李信动了动嘴,若是他手上的兵力再多一点……不过现在想这些事并没有意义,他也只有八百兵而已。

“不过土鸡瓦狗,若是官军再多一点,李常杰如何能猖狂。”黄金满跟在韩冈身边。就算是他的部众正举着火把向北行去,他也只是派了儿子去指挥。作为人质,作为投效的降将,他的态度摆得很正。

“为防被偷袭,李常杰不在官道上集中前进。在天亮以前,他们到底能不能走到寨子外?”

“有人走得快。”韩冈眯起了眼睛。在交趾军掀起的狂涛中,有十几点火光冲在最前面,从速度上看,那是只会是骑兵,看着他们的方向,是直奔归仁铺的大营,“还是先派了人查探,李常杰果然是小心谨慎。”

“这几日连吃败仗,李常杰早是畏官军如虎,哪里还敢不小心谨慎?还有就是运使说的,交趾兵最恨李常杰,军心不稳,若是匆忙间遇到伏击,肯定会溃败。哪里比得上运使得军心?”

韩冈笑着摇了摇头。他说交趾兵最恨李常杰,其实将同样道理用在自己身上,也是一般的适用。经过了这么多天,都没有得到一天休息,恐怕下面的士卒也都对自己有了怨恨,毕竟连续行军作战所引发的疲劳,是任何辛苦的训练都比不上的。

不过士兵中这样的想法,一直都被连续的胜利给压了下去。可如果官军被交趾军围攻在归仁铺中,来自下方的压力就不会像现在这般若有若无。

而且韩冈也不能全心全意的相信黄金满留在昆仑关的部众。官军高歌猛进的时候,只有疯子才会反叛。但若是归仁铺这边形势不妙,李常杰再派人去攻打昆仑关,留在里面的守军不一定能支撑下去。

“先走吧。”韩冈调转马身,“虽然不知道李常杰会怎么做,但如果他追过来,就送他一个终身难忘的教训。”

……………………

用了小半个时辰才走完了短短五里的道路,数十名交趾骑兵接近了归仁铺的宋军营寨。

寨门大开,黑洞洞的营地仿佛一头怪兽张开了巨口。用上万大军都没有打下来的寨子,这时候,只要迈开脚步就能进入。但他们在外面梭巡着,小心翼翼的探头向里面张望,却谁都不敢先进去。

“你!你!还有你!”领头的军校不耐烦了,左手按着腰刀,右手手指一个个点着人,“给我进去仔细查看。”

在军法的威胁下,终于有一小队骑兵蹑手蹑脚的走进了静悄悄的营垒。

里面悄无声息,对敌军的侵入,没有任何反应。斥候们的胆子便放大了一点,举起火炬,就要去检查营帐。只是但他们接近到第一顶帐篷的时候,蹭的一声弦鸣,一支利箭从黑暗中飞出来,扎进了火炬下的一名士兵的颈项中。

一声惨叫传遍营地。

果然有人!

十几名交趾骑兵立刻返身就逃,也不去查看发出惨叫的同伴如今究竟是生是死。而他们逃离的行动随即引发了一场灾难,箭矢密集如雨,从背后直贯而来,将一名名骑兵射杀在逃路上,到最后,只有两人冲出营寨。

在外等候消息的军官完全没有去想营救他的部下,跳上马向来路奔回,同时从腰间摘下号角,用力的吹响。

听到传遍四野的号角声,正在行进中的队伍一个个都改变了方向,以归仁铺大营为目标,开始向中央汇聚。

而仿佛在应和交趾军的来攻,营地内的灯火也重新点亮,从营中飞出来的箭矢也将挡在大门前的交趾骑兵远远的逐开。看着宋军大营的寨门重新合上,交趾军的心中都涌起一股终于识破了敌人阴谋诡计的快感。

……………………

‘被骗了。’

一个时辰后,李常杰平静的外表下,是如同火山岩浆一边翻滚的愤怒。

在外监视的斥候,还在汇报着只有几十名骑兵逃出了。可当他集合了分散开去的队伍,杀到归仁铺的时候,营地中早就空无一人,只有篝火还在燃烧。藏在营中并不是试图扭转战局的伏兵,而仅仅是数目寥寥、用来牵制的骑兵,而自己,竟然傻傻的上了他的当。

宋军在他面前要来就来,要走就走。李常杰只觉得下面的将领投向自己的视线,充满了轻蔑和嘲笑。他在军中用了几十年才树立的战无不胜的形象,被宋人一下下的掘断了地基。

“兵法有云,攻敌之必守。宾州和昆仑关,不信宋人敢放弃。”李常杰要用胜利和杀戮来回复自己的声望。

整个广西一路,真正派得上用场的就只有那一点点可怜的荆南军,而南下邕州的更是只有八百人。在有心防范下,就算桂州再派来援军,他也夷然不惧。

就以昆仑关和宾州作为这一次侵攻的终点!李常杰他要在离开之前,再给宋人一个教训!

