\section{第12章 共道佳节早(九)}

“想不到韩冈当真成了女婿,世事难料,这件事怎么都想不到啊……”曾布为自己倒了一杯酒,与坐在对面的酒友感慨着。

挂在檐角下的一溜灯火在风中闪着,樊楼的五座楼台都被数百盏各色彩灯装饰得流光溢彩。楼上楼下,丝竹悠悠,婉转的歌声,不时传进小小的包厢中。

“其实当年韩玉昆第一次上京时,王相公就很已经很看重他了。要不是阴差阳错,婚事早就定下,何须拖到今日?……要说想不到,还是说这三年,韩玉昆的际遇和功劳却是当初怎么都没想到的。”

章惇回想着三年前,韩冈第一次站到,不过是个刚刚被推荐入官的选人,他那一天在王安石家的表现的确是出类拔萃,但要说能从中推断出韩冈现在的成就,却根本是不可能的。三年时间,走完了他人一辈子的路,在座的几人中,又有谁能想象得到?

“今次平定荆湖山蛮能如此顺利,也多亏了韩玉昆推荐的人选。他的表兄李信果然是豪勇无比,当世罕有一见的猛将。不过更有用的,却是韩玉昆派来的一队医兵。没有他们,荆南山区的瘴气和疾疫早就把官军给打垮了。更别提那十几个部族的投效,有三分之一是靠着给他们族中的贵人医治而带来的。”

荆湖拓土有了阶段性的成果,章惇也赶在腊月前回来了。他可不像王韶那般能耐下性子,可以在关西一待四五年。如今朝中风云变幻,就如福建夏秋时的天候,清晨还是晴空万里,到了傍晚可能就刮起了台风。若是他在外面待得久了,很可能他的位置就会被人所取代。以章惇的想法,明年再用上半年的时间,将荆蛮解决个大半,那时就可挟功回京。

“韩冈的确是个难得一见的人才……可他心思难测,城府太深。行事作为,老成稳重,可出得计策,却又急功近利,完全让人看不明白。招他为婿,恐怕非是相公之福。”

不管是偏见,还是成见,曾布始终对韩冈难有好感。自从当日一见之后,就始终觉得这个年轻人太过危险。就像一包掺了糖的毒药,吃下去很是可口,但说不准什么时候就毒性发作。

心知曾布对韩冈的看法已经根深蒂固,章惇都懒得劝他了。拿着银筷,夹起了一块烤得香酥嫩滑的羊肉,笑道:“不论韩冈他日后如何,现在还不是要仰仗曾学士你的青眼?”

就如章惇所说。尽管礼部试诸考官的名单要到明年正月才公布,但曾布主考官的地位基本上已经可以确定。上一科的主考,就是时任翰林学士的王珪这。如今曾布也是翰林学士,加之今科又是王安石主导的科举改革的第一次上阵,当然不会让主考官的资格落到他人的手中。

曾布冷淡的从鼻中哼了一声。章惇的意思是让他照顾韩冈,可他跟韩冈可没有什么一党同僚的香火之情,就算他是王安石的未来女婿,也别想让他曾布去帮他铺桥修路:“韩冈若真有才学,自会被取中。若其才学不济,他是宰相女婿也一样没有办法。糊名誊抄之后,又有谁找出哪一张卷子是韩冈的?”

“从文笔、文风上来找人,想找到他也不是不可能。”章惇说道。

曾布冷笑一声,反问道:“那苏子瞻当年是怎么变成第二名的?”

嘉佑二年,苏轼的文章被礼部试的主考官欧阳修所看重,但欧阳修以为是自己的弟子曾巩所撰,怕公布后被人说成是徇私,故意将其降了一位。苏轼的省元身份,就这么成了泡影。

以欧阳文忠的眼力,都不能分辨出自己的弟子和苏轼的区别。曾布要想分辨出韩冈的卷子的确更加不可能。

再怎么说,曾巩是欧阳修的学生,而苏轼也早早的跟着父亲和弟弟一起被荐到了欧阳修面前。两人的文章,欧阳修早前都是看过了许多,文风、行笔已经很熟悉。而曾布却没看过多少韩冈的文字,怎么可能从五千份考卷中分辨出来?

章惇也无话可说。虽然可以询问曾布他准备出什么考题,但章惇知道,曾布给出的回答,可能是一记白眼,或是一声冷哼,绝不会给出有用的回答,他干脆就不丢脸了。

曾布明显的是不想帮忙而已,就算不能先行透露考题,只要事先沟通好,让韩冈在文章中留个关节,到了阅卷的时候,一眼就能发现的文章是谁写的。如此的手段世所常见,别说地方上的贡举,就是礼部试中也是有过。只要不是糊涂到将这个作弊之人列到最前面的几名中,放到中间或者最后,谁又能找出不是来?

曾布只是不肯帮忙而已!

但章惇也不是一定偏要帮着韩冈拿到进士的头衔。韩冈就算没有这个身份,以他的才能,要升上去也是很容易,最多进不了政事堂而已。而在章惇对未来的设想中,缺少进士出身资格的韩冈,其实更易于掌控和驱用。

他举起酒杯,与曾布对饮而尽,并不为此再说一句。

当然,今天章惇、曾布坐在一起议论的话题,并不是只围绕着韩冈。朝堂上的局势变动才是章惇、曾布更为关心的问题。

“陈旸叔要回来了。”曾布。曾经先王安石一步升为宰相的陈升之,在地方上任满一届后,也到了安排他下一任工作的时候了、。

“官家能给他什么位置?”章惇推断着,“如今是相公是独相,占着昭文馆大学士。而监修国史和集贤院大学士两个职位,都是空缺着。曾为宰相的陈升之,他回到宰相之位也不是不可能。”

“谁知道呢。说不得定会在外再留三年。”曾布也揣测着,“不过他做宰相,总比冯当世上来要好。”

在政事堂中,始终像是一块堵路石挡在面前的参知政事冯京,当然不受新党一众欢迎。章惇都皱着眉头,“要宣麻,也是王珪先上一步,冯京在中书中的资历,远不如王禹玉深厚。”

“若真的是王珪做宰相,倒是可以放心了。”曾布哈哈笑道,始终只会说着陛下圣明,臣无异议的王珪,在新党中人的眼中,是个极无用的角色。

只是章惇隐隐的却有某种忧虑,晋升之速并不逊于任何人的王珪,真的有这么简单?

……………………

女儿终于嫁了出去,王家上下喜气洋洋。在公事上,宗祀大典也结束了。外面的鞭炮声越来越经常的响起,现在就等着熙宁六年的到来。

因为已经订了亲,韩冈也不可能上门去拜会自己未来的岳家。他依然还是在王韶家,只是中间抽空去了趟种谔府上,与种朴、种建中见了一面。听说了韩冈要娶王安石家女儿,恭喜之余,。

而王雱也抽空与韩冈见了几面,论起对王安石学术理论的理解,自幼听其教诲的王雱,当然是浸淫甚深。靠着他的指点,韩冈对于王学的理解又更深了一步。自然,对即将到来的礼部试也更加有了一份底气。

王安石的学术观点,有一部分是盱江先生李觏的学术理论的改进,比起重视天地大道本源的张载关学、二程洛学两派来,王安石的儒学理论,更追求对现实社会的认识,而少有对格物致知方面关注。

几家学派,几乎是背道而驰,许多地方,跟道佛两家反而更近一些。

但他们,却都算是儒学。

在宋代,儒学就是一个筐。

孙复撰写《春秋尊王发微》,刘敞撰写《七经小传》,两人在书中大改旧时流传下了经典注释,而是以自己心意来解释儒家经典。自此之后,各家学派,各大儒宗,都是别出机杼,将自己学术观念加到儒学这个筐子中。也就是与汉唐儒者‘我注六经’截然相反的‘六经注我’。

流传后世千载的程朱理学,能有几分合乎原始的儒学?孔子若是活在现在这个时代,怕是每一家学派,都不会被他成认为是儒家道统的传承。

韩冈要把物理学、数学、天文学包装进儒家理论里去,当然也是同样往筐里装苹果。张载这个儒学宗师看到之后,仅是觉得有理,能让气学原理在现实中得到印证,便全盘接受了韩冈对格物致知的新解。

其实这就是挂羊头、卖狗肉。

不挂上羊头,狗肉卖不出去。不但卖不出去,还会有人说这狗肉太贱,完全上不得席面。

但挂上了羊头之后,尽管还是有人会说这味道好像不对。可大部分人,却会被便宜的价格,以及还算出色的口感所吸引。等到日月长久,人们都习惯了狗肉,就会觉得羊肉就该是这个味道,真正的羊肉到了面前,反而会被斥为假货。

韩冈便是有这个盘算。只要自己学术能推广出去,日子久了,就会成为正统,人人加以研习。科学体系以儒学的名义建立之后,又有谁能来推翻?

而物理学掺进了儒学中又如何?不过是换了个封皮而已。两者可不是如科学和神学那般不可调和。

儒学是个很宽泛的概念,可以兼收并蓄,可以海纳百川,并没有不可逾越的界限。

就像张载能够重新定义何为儒者,重新定义儒学的本质,二程、朱熹做过,韩冈也同样可以做。

就让后世的学生,为张韩道学而头疼好了!

韩冈乐于一见。

