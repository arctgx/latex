\section{第32章 吴钩终用笑冯唐(九)}

【不好意思,迟了一点。夜里一更照常。】

韩冈与赵瞻顶牛,主持军议的韩绛无意出来缓和气氛,他虽说不上盼望看到这一场面,但现在也不会出头帮着赵瞻。身负君命,却压不下一名选人,丢脸的可是来自东京的这位赵大观。

种谔和燕达见着韩冈、赵瞻之间火花四射,不由得暗中感慨,也只有韩冈这等文官才能不给天子使臣半点面子。换作是他们武夫,对代表天子而来的文官有了哪怕一点不恭顺,这下场就难说了。赵瞻若是因此要治他们的罪,直接就可以报给朝廷,韩绛都不便出面做保。但文官之间的交锋,就看各自背后的实力以及本身是不是占着道理,天子使臣的身份丝毫压不住人。

“韩冈……”赵瞻音调阴冷,代天巡狩的使臣在眉宇间积蕴着雷霆之怒,帐中众将都是噤若寒蝉,眼观鼻、鼻观口,谨守心神,充耳不闻。这不是他们能插话的场合,即便他们的品级都在韩冈之上,可文武之别并不是官品的差距能弥合得了。

韩冈没有半点畏惧,毫不客气的将赵瞻将要迸出的威胁堵在他嘴里:“是否将叛军及其家属流配至河湟,第一先要将之招降,第二也得确定他们再无反意,韩冈现在只不过是提出建议而已,究竟能否得允,还要看天子和两府的决断。郎中若是反对,亦可上书朝中,让天子两府来评判!”

韩冈一句话,看似是就事论事,但实际上等于是一口否定了赵瞻此前在宣抚司拥有的决断之权。按照他的说法,如何处置叛军,都必须征询天子和宰执们的意见。接受身为首相的韩绛的指挥分属应当,而赵瞻越俎代庖的命令,则是毫不合法,绝不当承认。

赵瞻怒不可遏,扭头瞥了上首一眼,正见韩绛仍是一副事不关己的模样,心火便是更旺。在他看来,韩冈现在的发难,当是在背后得到了韩绛的唆使。否则一个微不足道的选人,怎么敢当面驳他的话。

赵瞻并非蠢人,韩绛的态度既然是站在了韩冈的一边,又有可能是幕后的黑手,就不能再闹下去了。他暗地里咬牙,以自己的身份,跟一个小小选人争吵起来,那是自取其辱。心中打定了主意,回去后定是要将韩冈的桀骜不驯报于朝堂,还有他想把叛军依然留在关西的打算,也同样要报上去,让天子和枢密院来问问他,到底是安得什么心!

至于韩绛……等着贬斥州郡吧!

赵瞻不再理会韩冈,转过身,对着韩绛推说身体不适。得允离开后,他便恨恨的甩了一下袖子,再盯了韩冈一眼就转身出帐。

赵瞻走了,军议也没有什么可以再议的,韩绛随口对众将说了几句勉励的话,也便各自散去。

议了半日,什么都没决定下来。最重要的一块肉,还悬在众将校的嘴边。韩冈看着他们出帐时的模样,便是隐隐有着互不相让、针锋相对的情况。看起来为了争夺一个招降的权力,他们也许会用尽手段。

在韩冈看来,除了种谔、燕达这两位不可能出动的副总管,其他将校都有受命的希望。接下来,应该就是他们私下里做文章、找门路,在下一次军议前,抢到一个优势的位置上。

“再等两天,他们差不多就能争出个眉目了。”

军议后,韩绛把韩冈留了下来,除此之外,就只有种谔和燕达。见着韩冈不经意的在看出帐的众将,他便就笑着说道。

韩绛难得的对人和颜悦色,韩冈却也并不惊异,都帮了那么多忙了,怎么可能还板着脸?要不是这些天来帮着韩绛打压赵瞻,他如何会对自己改换了态度。

韩冈摇了摇头,顺着口风说下去:“郭太尉当日能做到的,不代表他们也能做到。争得再厉害,其中真有希望说服叛军的也只有几个。”

争抢劝降一事的将校,目的都是想做郭逵第二,但他们灰头土脸回来的机会也不低。郭逵当年能成功,本身的能力、胆略和人缘摆在那里,并不是他到城中一亮身份,叛军纳头便降的。

“满朝武将,能比得上郭逵的本就不多。也就当年的狄青和种世衡或可稳压他一头。子正和逢辰你二人,比起郭仲通当是还差上一点。”

燕达是郭逵一手提拔起来的,而韩绛方才又说郭逵比不上种世衡。燕达和种谔都是点头颔首,“相公说得正是。”

韩绛突又笑起,“可叹赵大观自恃其能,把郭逵气回长安,否则咸阳早定……现在就得看子正和逢辰你们两人了。”

“末将敢不从命。”两人异口同声。

“玉昆,你当真无意去咸阳城中一行?”韩绛转而又问起,“以玉昆之才,加之如今在军中的声望,当是马到功成……听王文谅说,你跟吴逵当是有一段因缘吧?”

韩冈摇摇头,“下官与吴逵只有数日之交,并不相熟,贸然前去却是难以成功。”

“还是不想争功吧……”

韩冈淡笑不答。他在众将之中的人缘关系,在他表示了无意争夺劝降之后,赫然上了一个台阶,如何还会去自找不快?他转过话头,道:“今次吴逵必死,想必其人亦是自知。想要劝他出降,那是千难万难。所以劝降之事,不在吴逵,而在那三千叛卒!”

…………………………

随着三月的天气越发得温和起来,由西面蕃区东来的道路上,已是雪融冰消。抵达古渭寨——现在已经改名做陇西县——城外榷场的商队也越发的多了起来。

时近傍晚,夕阳西下,红霞映照中,榷场门口的闭市鼓响了起来。一通接着一通的鼓声催促着,榷场中的店面关门打烊;外地来的大小商旅也纷纷收拾了货物,往榷场外的几间兼做住店的货栈去了。而冯从义,也带着两个孔武有力的伴当,从榷场的大门处骑着马离开。

虽然冯从义还很年轻,上唇处还只有茸茸的短须,可在陇西榷场中,他的地位却是很高。见到他骑马要回城,路上看到他的商人,都是隔着老远便打起了招呼。有喊他冯掌柜的,有喊他冯四哥的,当然,更多的便是恭恭敬敬的称呼他一声冯大官人。

因着和韩冈的关系,青唐部的包顺【俞龙珂】、包约【瞎药】两兄弟,有许多买卖都是委托给冯从义主持的顺丰行来措办。不过半年多的时间,不仅是在新成立的通远军已经牢牢的扎下根基,在秦州州城,也已经打下了一片江山。

不过因为韩冈的吩咐,为了不引起他人的议论,冯从义始终保持着低调,只做着批发的生意。在秦州,也仅仅是在秦州河西大街的内巷中盘下了一间小院,并没有在大街上开个门面。顺丰行的名声只在蕃人和商人中比较响亮,基本上在外界,则很少能听到人们关于顺丰行的议论。这一点,与王韶和高遵裕两家的商行完全不同。

冯从义与人打着招呼,一路进了陇西县城。城头上警哨密布,在街上,也是巡城甲骑一队接着一队。

罗兀城的战局虽然离着河湟很远,但对此地的影响依然深远。尤其是广锐军叛乱之后,郭逵和燕达纷纷被调离,缘边诸寨都一下进入了最高戒备状态。

只是最近隐隐的有消息传来,官军撤出罗兀城时,大败西贼追兵,据说是前所未有的大捷。但燕副总管还带着大军在外面,传回来的消息还说在叛军手上吃了个大亏,相信罗兀城大捷的人便没有几个,只有与衙门走得近的,比如冯从义这样的人,才清楚这个消息是千真万确。

进了韩家的大门,把,交给迎上来的下人带去马厩,冯从义整了整衣襟。

堂屋中,韩千六、韩阿李并坐着,另外一个打横坐着的,却是他的表兄李信。

李信穿着官服,装束一新,是明明白白的官人,而不是冯从义这样被人叫的顺口的。

韩阿李一见冯从义,便连声叫道:“义哥儿,还不快来见信哥。”

“二姨,姨父,表哥。”冯从义一个个喊过去,他是收到李信从京城回来的传话,才从榷场回来的,否则他都是住在商行中,过几日才来韩家一趟。

李信起身向表弟回礼,他也是今天才进了陇西城。风尘仆仆,身上的官服还是韩阿李逼着他换来看的。

李信是上个月参加了试射殿廷的考核,得到现在的官身。也许是有补偿的因素在,更有可能是不敢再得罪风头正劲的韩冈,被托付的李信试射殿廷之事,新任三班主簿蔡确很上心,也卖力气,他在三班院中帮了李信不小的忙。甚至还设法说通了来主持考核的枢密院都承旨,在李信参加测试时,加试了一项他所擅长的投枪。

李家嫡传的掷矛之术,是西军中的一绝。在几位考官面前,李信七枪连环而出,将五十步外地一排铁甲挨个洞穿,惊得众人瞠目结舌。是以李信箭术仅为‘中格’的成绩,最后却得到了一个‘绝伦’的评价。与当初跟韩冈同去京城的刘仲武一样,得授三班奉职,比正常的三班借职高上一级。

在冯从义进来的之前,李信正与韩千六夫妇说着他回来时的见闻,等冯从义坐下,李信又继续说起:

“侄儿过长安的时候,鄜延路的官军,刚刚离开延州南下。不过罗兀城大捷,已经传到了长安城中。听说三表弟,在其中立功不小……”

