\section{第30章 臣戍边关觅封侯(二)}

傍晚时分,韩冈辞别了父母和韩云娘,骑着一匹借来的老马,赶在秦州锁城前抵达城下。秦州南门守门的官兵对这名每隔几天就要回家一趟的韩三官人已经很熟悉,不敢怠慢,忙将韩冈放进城来。进了城后,韩冈直奔普修寺而去,这是最近他在城中落脚的地方。

韩冈刚到寺门门口,住持老和尚就带着个小和尚迎了上来,在马前点头哈腰,“三官人!王衙内来了!已经在厢房里等了你很久。”

“多谢师傅!”韩冈下马后拱了拱手,将马缰交给小和尚,自己快步进了寺中。

韩冈如今寄寓在普修寺内,住持和尚对他比以往更加殷勤,将最好的一间客房让给韩冈。尽管秦州离家只有五里不到,隔着一条窄窄的藉水,但韩冈还是选择住在秦州城内,而只是每隔几日才回一次下龙湾的家中。

秦州城门一向关得早开得晚,每日出城入城很不方便,而且王厚、王舜臣和赵隆,还有同样给荐到了王韶的门下,在经略司中听候差遣的李信,也经常来找他。而在王韶和吴衍面前,他也得摆出个随叫随到的姿态。所以借住在普修寺中,比较方便一点。陈举的余党已被一网打尽,就算有些漏网的小鱼小虾,也成不了气候,更不可能有胆子再来行刺,韩冈已不必担心家人的安全。

等到正式为官,挣到了俸禄后,韩冈还准备在城中找间房子,把家安在城里。总不能自家做官了,还要老子和娘种菜卖菜。

可寄寓城中有一桩坏处,就是读书的时间少了不少,每每拿起书本,总会有人来打扰。多少天下来,韩冈拒礼的名声已经传出去,上门送礼少了不少,但王舜臣、赵隆、李信三人隔三差五就带着酒菜过来问候,而王厚更是来得勤快。

“玉昆!喜事啊!大喜事!”甫一见面,王厚就拱着手,笑呵呵的走上来,连声对韩冈道着喜。

韩冈一边往屋里走,一边没好气地道:“上次处道你说的大喜事,是东城布匹李为他的大麻子脸女儿来提亲,再上次是个带儿子的寡妇。今次又是哪家?”

两人熟悉起来后,王厚的本性算是露了出来,就是个诙谐爱开玩笑的性子。前面他说的两次喜事,都是来向韩冈提亲中的极品,却被王厚拉出来当笑话说。可能是在王韶身边太憋闷了,王厚每天晚上都变着法儿的从家里跑出来,找他喝酒聊天,害得韩冈夜里能用来读书的时间都变得寥寥无几。

但王厚是官宦子弟,俗称的衙内,对朝中内外的大小事务,比韩冈了解百倍。多喝了点酒,他的话匣子一打开,说出来的泰半是韩冈闻所未闻的朝野秘闻,还有对朝中新近发生的事务评判——韩冈猜测多半是王韶说给儿子听的——这些对韩冈的用处,可比儒家经典大得多。

只是这次王厚显得很正经,“是真的喜事。刚刚京中来了朝报,令师张横渠朝见天子后,已被擢为太子中允,任崇文院校书。恐怕不久就要大用。”

韩冈一震之下停步回头,惊喜道:“那还真是件喜事!”

张载与王韶是同科进士。相对于王韶因一篇《平戎策》得到重用的情况,张载的升官速度便是按部就班,当然这也与他有很大一部分精力放在教育学生上有关。没想到张载今次进京后,竟然一下升了正八品的朝官,已与王韶的本官相同,又得了馆职,这是大用的标志。

在北宋的官制中,正八品与从八品看似品级只差一级,实则却是有天壤之别。北宋的文官从高到低分为朝官、京官和选人三个部分。其中京官和选人的品级都是从八品到从九品。从称号上看,京官在京中挂名,选人又称幕职官,是地方上的官员,两者名义上相当于后世的国家公务员和地方公务员,等级上并没有高低之别,但实际上却差别极大。

选人占到文官人数的绝大多数,一万多近两万的文官中有近九成一辈子都是选人,时称永沦选海。只有得到五名路一级的高官的举荐——号为五削圆满——,并觐见过天子后,才能升为京官。

一般情况下,内地知县仅有京官可做,后世的七品芝麻官,放在北宋就是个笑话。一县之主,百里之侯,基本上都是从八品,到了正七品,早能担任知州了——都钤辖向宝,是秦凤路武臣中的第二号人物,他的本官皇城使,也是正七品。

宋时官品贵重,第一次为相时的宰执官一般也仅仅四品五品,六品七品也是有的,可不是如满清时那般朱红顶子满眼看、一品大员满天飞。

当京官升到正八品后,就成为了朝官,也叫做升朝官,顾名思义就是能参加朝会、面见天子。想想宫殿才多大,能容多少人?升朝官文武两班加起来,总数也只有千多人。除去大半在外任官的,每次朔望大朝会,得以参加的文武官也不过四五百,张载在中进士十二年后,便已能名列其中,这个速度足以让他的大部分同年们羡慕不已。

而张载的崇文院校书一职,甚至连王韶都要艳羡三分。崇文院又称三馆秘阁,是昭文馆、史馆、集贤院和秘阁的统称,单看此时的宰相都要兼任三馆大学士一职【见第三章注4】,就知道崇文院有多重要。崇文院号为储才之地,进了馆中,便等于是入了升官的快车道,一旦朝堂上职位有阙,首先就会从崇文院等馆职成员里挑选。

作为弟子,老师得到重用当然是件喜事。可对没有关系的王厚来说,却只是个出来喝酒的借口。

“愚兄怎么会骗你!”王厚笑呵呵越过韩冈,先一步进屋。

韩冈也跟着进房,厢房中的桌上已经摆满了酒菜,一个火盆已经燃起,将屋内烤得暖烘烘的。王厚已经坐了下来,正拿起酒坛向个用来热酒的大铜酒壶倒着。

韩冈暗自叹气,有王厚这个酒肉朋友天天来捣乱,根本无法安下心来读书。如今虽不需进士功名就已经能做官,但开卷有益,只有多读书,增长学识,日后在那些千古名臣面前才不会露怯。

王厚可不知道韩冈心中抱怨,他将倒空的酒坛丢到桌子下面,把铜酒壶吊在火盆上热着,坐回来对韩冈笑道:“幸逢喜事,不知玉昆有否佳句以记之?”

“处道兄,你也是知道小弟不善诗赋,就别打趣了。”韩冈叹着气,这不是难为他吗,“但凡吟诗作赋的本事强一点,小弟就去考进士了。”

王厚安慰韩冈道:“但玉昆你通晓经史,擅长政事,这才是正经学问。”

“经传再高,也只能考个明经,进士可就没指望。”

“玉昆你有所不知,”王厚用手指摸了摸火盆上的大酒壶,试着冷热,随口道:“王相公本有意以经义策问替换掉进士科的诗词歌赋,以玉昆之才,当有用武之地。只可惜让苏子瞻给搅和了。”

“什么!”韩冈猛然惊起,“竟有此事?!”

王厚奇道:“玉昆你不知道?哦,对了!这是半年多前的事,你那时正好在病着……就在当时,王相公上书建言,要兴学校、改科举,弃诗赋而用经义。官家可都让二府、两制还有三馆众臣一起议论了,命人人都要上札子。东京城内沸沸扬扬,国子监中人心惶惶,天下都传遍了,你说有没有?!不过最后让苏子瞻的一本奏章否了,此事也便不了了之。”

“是吗?…………”韩冈声音低沉下去,暗自揣测着王安石的用意,此举又会给政局和自己带来什么样的影响?

改科举、兴学校这两条很好理解,就是为了选拔和培养人才——变法的人才。而苏轼会反对,也不难理解,他毕竟是以诗赋出名,也是靠诗赋考上的进士,交好的友人、弟子都是以诗赋见长。屁股决定脑袋,哪个时代都不会变。

韩冈愿意拿脑袋打赌,司马光虽然与王安石互为政敌,但他绝没有在科举改革上与王安石作对过一句。为何?还不是因为他是陕西人——不擅长诗赋文章的陕西进士。只是若想对此事进行更深一步判读,还要把王安石和苏轼的奏章拿到手上才够。

王厚见韩冈突然不说话了,问道:“怎么?还在想诗赋改经义策问的事?”

韩冈抬眼对王厚说道:“我在想王相公为何要改科举。”

“为何?”

“因为人才难得。变法之要,首在得人。而科举抡才便是其中最重要的一条路,如果处道兄你是王相公,你是想看着的是擅长吟诗作赋、却反对变法的进士,还是熟读经史、长于对策的同志?”

“同志?”王厚咀嚼着韩冈用的这个生僻的词汇,笑道:“这个词用得好。《国语》有云:‘同德则同心,同心则同志。’如果愚兄是王相公,当然想用与自己同心同德的人才。王相公在奏疏中本也说了,‘朝廷欲有所为,议论纷然,莫肯承听,此盖朝廷不能一道德之故也’。他兴学校、改科举,当然是为了选拔人才,教育同志,要‘一道德’。只可惜啊……却被否了。”

“谁说给人否了,就不能重提的?今科是不可能了,但三年后的下一科,很有可能就改用经义策问取士!说不定到时小弟也……”韩冈说着说着突然笑了起来,摇摇头:“都已经有官身了,也考不了进士,管日后王相公能不能改,都是跟我无关了。”

ps:好了,文中对北宋官制稍微提了一点,虽然没细说,给大家留个印象也就够了。北宋的品官,高品的很少。品级晋升也要很长时间,不会出现二三十岁便三四品的官员。不过在北宋,五六品也能担任执政,这为年轻官员获得权力,提供了一条路。

反正就一句话,把满清官制留下的印象丢掉,在北宋,差遣和本官是两条线。高品官不一定有高的差遣,而低品官,却可以入居政事堂。至于文武官详细的品级划分,等俺整理好也会贴出来的。

今天第一更,离首页的前十五位咫尺之遥,却始终跨不过去,兄弟们,红票和收藏再给力一点。

