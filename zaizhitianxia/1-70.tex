\section{第31章 马鸣萧萧辞旧岁(上)}

王韶在秦州的府邸并不大,就是两进的小院,比之韩家的宅子也差不多大小。前面住着护卫仆役,后院是主屋。不过也没有必要弄得太大,王韶在秦州任职,只带了次子王厚过来。其他的几个儿子女儿,都留在江州德江的老家。他四年前原配杨氏病逝,续弦徐氏也留在德江,秦州家中只有父子两人,三名侍婢,还有两个配属的老兵充作仆役。

王厚带着满身的酒气冲回家中,正在书房中伏案疾书的王韶便皱起眉头,只是看到韩冈跟在身后,才没有发作起来,教训儿子。

对于韩冈,王韶早没了过去的芥蒂,而是青眼有加。要不然王厚天天去找韩冈喝酒,换作旧时,他早动了家法,打得儿子不敢再乱跑出家门。若不是唯一的女儿才十岁,又早早的许了人家,韩冈就是最好的女婿人选。现在王韶与乡里的亲友书信往来,都要问问亲族中有没有适龄的女儿,好把韩冈与自家用婚姻联系起来。

轻轻叹了口气,王韶在青瓷笔洗中涮了涮毛笔,用厚纸吸干水,挂在笔架上。方才问道:“究竟何事?”

没看到父亲的脸色,王厚兴冲冲的将韩冈的计划一股脑的说了出来。韩冈站在后面,瞧着王韶脸上的神色的变化,却没有发现多少兴奋之情。

“难道机宜早已考虑过?”若在平时,韩冈绝不会这般直接相问,而是会旁敲侧击一番。只是他喝得微醺的时候,被王厚拉到王韶面前,脑袋里还有一点未消的酒意,说话不似平日那般斟字酌句。

“皇佑四年,陕西转运副使范祥于唐时渭州旧址修建古渭寨,至今已有二十余年……”王韶没有回答韩冈的问题,却突然讲起古来,“在这期间,有人提议在古渭开榷场与蕃人互市;也有人提议开办马市,用盐、茶交换战马;更有人想着移兵屯田,将古渭扩寨为城;当然,也不是没有人想要废弃古渭——范祥便是在古渭寨还没有修好之前,便被连番弹章攻击得连贬两级。渭水之滨,城寨二十余,没有一座如古渭寨这般惹人议论。玉昆,你可知这是为何?”

“……地理,历史,人情。”简单的六个字,不是在回答,而是韩冈在整理自己的思路,以便下面能有条理的细细说明。

但王韶一听之下,却是击节称道,“说得对,正是这六个字!看来玉昆你是明白了。渭州自古便是通往西域的要地。汉唐通使西域,多是经由此路。自安史之乱后,渭州便沦于蕃人之手,迄今已有近三百年。将古渭升军,往远里说,意味着朝廷将要重新开拓西域,自近处讲,这是拓土临洮、开边河湟的第一步!……二哥儿,你明白没有?”他却问着儿子。

王厚叹了一口气,他老子都说得这么直白了,哪还能不明白?古渭设军的象征意义太强烈了,原本设寨便惹来多方议论,如果升格为军,朝堂上恐怕便要吵翻天。

“王介甫毕竟不是宰相,而仅是参知政事。”王韶也无奈的叹了口气,大宋国力不比汉唐稍逊,可一旦动起刀兵,却千难万难。纵有班超、马援之才,也架不住朝中有人拼命捣乱。一旦古渭升格为州一级的区划单位,将会代替秦州成为大宋西陲边疆,而将秦州屏蔽在后。从兵备上,理所当然的便要分割输送给秦州的粮饷物资,枢密院中的两位大佬不趁机扯后腿就有鬼了。

“要古渭升军,他事故且不论,单是日常消耗的粮秣,至少必须能自行解决三成以上。玉昆……你可知伏羌城以西,沿着渭河的几个寨子,哪一寨人烟最稠?”

韩冈想了想:“应该是永宁吧……”

永宁寨也在渭河边上,是位于伏羌城和古渭寨中间的一座城寨。离伏羌城四十里,距古渭寨一百四十里,寨中最有名的便是永宁马市,秦州的战马有一半是从这座马市中得来。若论人烟辐辏,古渭寨根本比不上永宁。

“你可知道几年前,范祥重回陕西,又有在古渭设立马市的计划。马市兴盛起来,古渭寨便可逐渐招收户口,最后便可以设县置军。范祥之策当时得到冯京的支持,冯京还上书请增筑古渭城墙。平心而论,一个循序渐进的良策,又得到陕西安抚的支持,应该很容易就能通过。可终究还是没有成功——是给韩稚圭【韩琦】给否了。冯京是富彦国【富弼】的女婿,富韩之间几十年的恩怨,不用我说,想必你也该清楚……一旦关联到西事,事情便不会再那么简单!”

韩冈看得出来王韶的顾虑,将古渭升军,摆明了就要跟李师中翻脸,并逼着朝中给出个说法。这种放手一搏、一翻两瞪眼的赌徒做法让王韶犹豫不决。自己没考虑到王韶的心理,的确有些失败。但他还是觉得该坚持自己的意见:

“机宜到秦州已有一载,期间机宜遍访秦州诸城寨,了解军中情弊,以备日后出兵参考。厚积而薄发,任何时候都少不得。但天子看不到这一点,只知道机宜在秦州已满一年而毫无动静,王相公也许还能体谅机宜是被李经略掣肘,但天子的想法没人能臆测。事到如今,王相公想来肯定是想看到机宜有所动作的。”

“玉昆,难道你还是想升古渭为军?”

韩冈避而不答王厚的问题,“以冈之愚见,任何开拓河湟的策略,必须是惠而不费。若想开拓河湟,必要的人力财力都少不了。可军费有限,横山那边多点,秦州这边就少点。河湟毕竟是偏师,即便收复全土,断的也只是西贼右臂……”

王韶听到这里,微微一笑。断西夏右臂的话还是他在《平戎策》中所说。他点头示意韩冈继续说下去:

“……而横山地势险要,西贼据有横山,便可俯视关中。横山中的蕃部,在西贼军中至少占了三成以上。一旦夺取了横山,党项兵力减少三成,少掉的兵力又会加到我军一方,一增一减,便超过了西贼兵力的一半。

兵源是一桩,粮草又是一桩,而且更重要。七百里瀚海是天险,欲攻灵武【即灵州】粮秣转运是最难的一件事。其实这对党项人也是一样,西贼主力从兴灵【兴庆府和灵州】出击,穿越瀚海运粮根本不可能,全都得依靠横山蕃部的支持,要不然就是攻破我方军寨,夺取存粮。一旦丢了横山,西贼就失去了长期进攻的能力,只能与我隔瀚海对峙。”

王韶听得连连点头,韩冈这些日子的下得苦功不是白费,将王韶手边的舆图与自己心中的后世地图互作印证。对陕西地理的了解,绝对是当世顶尖的水平。

“既然横山如此重要,天子和王相公就不会把更多的资源放在河湟之上。但机宜又要在河湟立功,便不得不动用秦州的资源。在下的想法很简单,如果机宜不能拥有独立的财权,李师中要卡脖子那就太容易了。”

“但也不必急着升古渭为军!先屯田立寨,等户口兵力都充裕了,设军设州也是水到渠成。”

韩冈摇头,虽然按部就班的屯田还是他第一次见到王厚时的见解,但当时只是随口说说,实际上根本不现实:“前日韩某曾与处道说起,为防惹动秦州那些回易商队背后的官员、世家,市易之事要放在屯田之后,以屯田为主,但现在韩某在州中多了解了一点,才发现那是书生之见。”

“嗯?为何?”王厚脑门上转着问号,脸上都是疑惑,但王韶却是露出浅浅的笑意,一副赞许的模样。

“市易只需开头的一笔本金,便可自行支转。但屯田就需要秦凤路源源不断的支持,无论人财物,至少都要两三年的时间。这一点很难做到。不论是谁坐在秦凤路经略安抚使的位置上,都不会支持机宜。”

王厚惊道:“为什么?!”

王韶帮着韩冈回答:“功劳占不到大头,但付账却少不了,哪个愿意?”

王韶有首倡之功,又被钦点来秦州主持实务,如果成功,这么大的一块饼,几乎给他一人吞掉。李师中、向宝岂是蠢人,就是因为要自己出大力气,最后却分不到一杯羹,才不愿支持。要知道,王韶之所以起了拓边河湟的心思,其实还是在蔡挺幕中看了向宝早年的一封奏章的缘故。

王厚恍然大悟,而王韶看着韩冈,心生感慨:“玉昆你真不像是十八岁。”换作是他,就是二十八岁时也没这么多心思。

“此是人之常情。韩冈也只是照常理说上一句,也许真有甘居幕后,不愿居功的贤人。”

“怎么可能有这种人!”王厚摇头,给他人做嫁衣裳,换作是他,他也不干,“所以玉昆你的意思还是用市易?”

“市易也是一般无二,照样还是要从秦州拿到本金……在下的意思是,只要李师中还在秦州,任何事都别想办成。”韩冈提醒着王韶,该翻脸就得翻脸,不能对李师中抱着幻想,“先通过请立古渭军,虽然李师中必然反对,朝中也很难同意,但届时便可退一步申请在渭源或古渭市易和屯田。”

“玉昆你前面也说了吧,李经略肯定会反对的。”

“那就再退一步,从市易或屯田中选一条,再向朝中报请。”

“如果李师中还是反对呢?!”

王厚觉得韩冈可能酒喝多了,说的话有些颠三倒四,前后矛盾。但王韶却放声大笑,笑罢,脸色一转变得冷狠:“那时,天子就该知道是谁是在干扰河湟开边了……”

ps:历朝开疆,以宋代最难,因为将帅们的最大敌人从来不在外,而是在内部。

今天第一更,求红票,收藏。

