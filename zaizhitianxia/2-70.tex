\section{第16章 绮罗传香度良辰(上)  }

【这算是今天第一更。如果家里的网络好不了,晚上的两更可能就没有了。不过前一章已经说过,少更的都会补偿回来,还请放心。】

“玉昆,怎么今天没什么精神?”坐在晚晴楼三楼的雅座中,王厚很是热情,他招呼着韩冈:“来,尝尝这道羊舌签,晚晴楼的招牌菜,迟上一点就只能等第二天了。”

晚晴楼的招牌菜味道的确不错,但这个夜宵可不是韩冈所期望的。王厚忙完了公事,不回家休息,还拉着自己来喝酒,不知该说他精力充沛,还是别的原因。

‘多半是不想一直被他老子盯着。’韩冈的想法算不上是腹诽,只是源于对王厚性格的了解,没哪个儿子喜欢在老子面前乱转的,尤其是王韶这样的父亲,给做儿子的压力实在很大。

王厚难得的能从王韶的压力下脱离几个时辰,整个人兴致高昂,一边劝着韩冈的酒,一边说着:“玉昆,你还记不记得调回京去的李复圭?”

韩冈当然记得。庆州知州兼环庆路经略安抚使李复圭冤杀种詠等三名将佐,前段时间终于被爆了出来,也不知是谁出的手,让整个御史台都上了弹章,上个月月底他便被调回了京中去了。五六两月,秦凤、环庆两路主帅接连更迭,让整个关西军方都有不小的震动。

王厚突然提起他,肯定是有了新消息。韩冈惊问道:“难道说他已经定案了?没这么快吧?!”

朝廷审案的效率有多高,在官场上流传的笑话不止一桩两桩。李复圭这个等级的官员,要审他,必须是御史台、刑部和大理寺三家会审,有时候,天子还会钦点主审人选。单是调和各家法官之间矛盾,少说就要一个月,整个案子不拖个半年,怎么都不可能有结果。

“当然还没有定案,他被召回京去,到现在也不过才一个月。不过李复圭的罪行也是清楚明白得紧。如果不定罪,最后大概是降一官或是两官,到南面的下州做两年知州。如果定罪了,大概是远郡安置,责授节度副使、团练副使之类官职。”王厚用筷子夹了个酿鱼丸,含糊不清的边吃边说,“以李复圭的身份,大概是节度副使。”

“太便宜他了!”韩冈心中有些怒意。冤杀朝廷命官,欺瞒天子,竟然还不一定能定罪。而即使定罪,也不过是个远郡安置的处罚。李复圭作出这等骇人听闻的事来,处罚如此之轻,冤死的种詠等三将都是死不瞑目

安置、编管、羁押,是朝廷对官员的处罚手段,惩罚程度从轻到重。只要不是追夺出身以来文字,也就是削官为民,官员受到的责罚最重也就是软禁程度的羁押。普通的是编管,不得出城,书信要被检查,而最轻的就是安置,只是不能离开所安置的军州乱走动而已。

而且这些被降罪的官员,一般都会被授予节度副使、观察副使、团练副使等戴罪官员专用的官职,虽然不会给他们实际的工作,但有着官职,就可以防着他们被小人所欺,伤了朝廷的体面。韩冈对此都不知该怎么评价了,只能说,这个时代的政府,对文官实在是太好了一点。

“是便宜他了。”王厚说着,“所以他现在还有心情写诗骂人。”

“李复圭作了什么诗?”

王厚停下筷子,又拿起酒杯。韩冈给他杯里倒酒,听他说着:“整首传到秦州的就两句,今天才听到——‘老凤池边蹲不去,饿乌台上噤无声。’”

“饿乌台上?”

乌台是御史台的别称,因为御史台外有片林子,乌鸦莫名其妙的特别多,另外,那些监察御史也是跟乌鸦没两样,一张嘴,就是有人要倒霉。而只看后面的‘噤无声’三个字,就知道这一句,李复圭是在明着骂御史台不作为。

两句诗一起连读,再联想起李复圭被御史们群起而攻的场面,这是他在抱怨御史台只拍苍蝇,不打老虎吗?

“可老凤说得是谁?”韩冈问道。

王厚反问:“‘池边蹲不去’,你说是谁?”

能让李复圭用这种幽怨的口吻说话,而且还是用‘凤’来形容的官员地位不会低,只能在宰执官中去找。再加上一个‘老’字,人选就只剩三个了——七十多岁的首相曾公亮,六十多岁的次相陈升之,以及枢密使文彦博。

只是把‘蹲不去’三个字考虑进来,升任宰执没几年的陈升之肯定要排除。剩下的曾公亮和文彦博两人,则都是实打实的三朝宰臣,从仁宗时就做着宰相。不过,文彦博有起有落,而曾公亮的宰相,却是从仁宗嘉佑六年,历经英宗朝,一直做到了现在。

用着排除法,韩冈得出结论,“是曾老相公?”

“除了他还会是谁?李复圭就是恨着曾相公下令将他夺职,回到京后,才写了这首诗。”

韩冈抿了抿嘴,对李复圭的做法分外不屑。这就是官场上最多见的文人,从不自省,只知怨天尤人。才能没多少,但害人的心术却高明得很。

李复圭的这两句诗,等于点了一根爆竹丢进御史台中,被惊起的那些乌鸦肯定是扑楞楞的满天飞。当然它们不是去回咬已经倒台的死狗李复圭,而是在相位上盘踞太久的曾公亮,那才是能张扬他们名望和刚直的肥羊。

“曾相公怕是要出外了。”韩冈顿了一顿,“就不知王相公会怎么说。”

赵顼启用王安石变革旧制时,韩琦、富弼都先后反对,只有曾公亮为其保驾护航。而且曾公亮的儿子曾孝宽是变法派的中坚,虽不比吕惠卿、曾布、章惇那样亲近,但也是深受王安石信重。

就在去年,王安石的新法在朝中掀起轩然大波,曾公亮虽然没有表态支持,有些情况下还不疼不痒的反对几句,但大部分时候还是保持沉默。以他的首相身份,这已经是最大的支持了。

韩冈不知道王安石会不会因为感念恩情,留下曾公亮。而王厚摇头,“家严说了,王介甫羽翼已成,用不到他护持。他这一去,就是给王相公腾了个位置。对于此事,天子和王相公都会乐见其成。”王厚嘴角的笑容带着讽刺,“也许再过两个月,就是真正的王相公了。”

“曾相公的年岁也太大了一点。”韩冈很平和的说着。

政治上的事本就没有什么人情好讲,而王安石也的确需要一个名正言顺的职位,来掌控变法大局。助役法的施行据说已经迫在眉睫,这条法案关系到民生的方方面面,直接改变了实行千年的徭役制度,不是均输、青苗和农田水利三法案可比,王安石当上宰相,对此法的顺利推行,有着不可估量的作用。

同样是官场中人,王韶对王安石的判断应该不会有错。不过曾公亮可是《武经总要》的主编,这套书总计四十卷,前二十卷是详细描述了军械、阵法、旗号、营垒等方面的军事学专著,后二十卷是汇集了历代战例。韩冈一直都想一睹这本名传千古的军事百科全书的真容,对有能力编纂此书的曾公亮也有几分尊敬。

与王厚继续推杯换盏,当韩冈回到家中时,已经快三更了。今晚他喝的虽不算多,但回来时吹了一阵夜风,酒意也有些上头了,不过还是能走得稳路,不至于摇摇晃晃的要人扶。

进了家门,韩冈让李小六牵着马去马厩,打理好两匹马后,自己去休息。他本人则是直接走进后院,却看着自己的房间正亮着灯。

都这时候了,谁还在里面?韩冈头中醺醺,一时之间,什么都想不起来的。掀帘进屋,

只见严素心正半趴在桌上做着海棠春睡。韩冈脚步一停,沉醉的酒意猛的散去,这时他方才想起今天白天时的事来。

想不到都这时候了,她还在房中等着。韩冈放轻了脚步,静静的走了进去。桌子上除了一盏油灯随着穿堂风忽明忽暗的闪着,还放着一个茶盅。韩冈轻轻的揭开茶盅的盖子,醒酒汤里的陈皮味就传了出来。

在桌边坐了下来,喝着酸甜味的醒酒汤,韩冈看着两尺开外,枕着手臂沉睡中的一张如花俏脸。

严肃心容色秀丽,身材高挑窈窕,本就是个难得的美人。而今天她稍稍画了点妆,大概是知道韩冈不喜石灰抹墙一般的浓妆,只是略略描了眉,抹了口红,并没有像秦州的妓女那样擦着厚粉。但就是这么一点改变,就让她更是眉目如画。

不知是在梦里想起了什么,严素心殷红厚实的小嘴微抿着,修长的双眉也紧皱,显得很伤心的样子,眼角处还带着泪,闪着晕黄的灯光。

韩冈看得怜惜不已。对自己倾心的三名女孩儿,不论是韩云娘,还是严素心,另外还有周南,都是命运多舛的女子。被卖进韩家的云娘还算好,在教坊司中长大的周南虽名为花魁,却不得不在欢场上强颜欢笑,而严素心则更是三个女孩儿中最受命运折磨的一个。

韩冈伸手想拭去她眼角上的泪迹,不城想严素心被他的动作一下惊醒了。她猛的坐直了身子,眼睛睁了开来。几缕散开来的发丝调皮的贴在她的脸颊上,旁边还有着被压后的红痕,可见她睡得已经有了不短的时间。

睁开的大眼睛中有着几许茫然,但眨了几眨之后,严素心终于发现坐在眼前、微笑着的韩冈。一惊之下非同小可,少女啊的一声短促惊叫,身子后仰,就要向后避退过去。却不想她本是坐着,两腿别在桌下,这一动,桌子和人都是摇摇欲坠。

韩冈微微笑着,不慌不忙的伸出双手,一手扶住桌子,一手则老实不客气搂住了她的纤腰。

