\section{第18章 青云为履难知足(十)}

【调整一下每天两章的发布时间。早上的一章在七点,晚上一章则在七八点左右。】

表兄弟在路上见面,两边顿时都吃了一惊。

韩冈睁大了眼睛,“义哥儿,你怎么在这里?”他走的这条路与关西入京的道路不是一条。

冯从义张开口,但不是向韩冈问候,而是回头向车中大声喊了起来:“三姨、姨父,你们看碰了谁了?”

‘三姨?’‘姨父?’

韩冈听着一愣,还没反应过来,车队居中的一辆大车上的车帘一下被拉开,从车中出来的两人,一见韩冈就是又惊又喜,“是三哥儿!真的是三哥儿!”

而看见他们,韩冈同样是又惊又喜,竟是他的父母韩千六和韩阿李。连忙在车前下马跪下,“孩儿拜见爹娘!”

跟着韩冈的伴当们看见是家里的老爷和老夫人,也一个个都连忙滚身下马,就在大道上拜了下来。

居移气、养移体,几年过来,韩千六和韩阿李气象迥然一新,就是穿着朴素的常服,也是一对官宦人家老夫妻的模样。旁边的行人虽多,也都是猜测着这一队是哪家的贵人,没人能猜到只是普通的农官入京。

“怎么瘦了这么多?!”韩阿李下了车,一把拉起儿子,上上下下打量着。黑瘦了不少的韩冈,让她心疼得不得了,“辛苦得都不要命了,是才从广西回来的吧?天南地北的来回跑,亲家公也不照看一下,哪有这样使唤人的!”

“三哥儿是瘦了,不过精神还好。”听着妻子的抱怨,“别在路上,往前面走,不能挡着后面人的道。”

韩冈看看身后,这么一停下来,后面已经给堵起来了。回过头,“爹、娘,还是先上车。这天热得很,在太阳底下晒着不好。”

两边并作一路,韩冈骑着马,跟在父母的车边:“爹、娘,你们怎么这时候上京来了?”

“在陇西做了几年的官,审官东院下了文书,说是任满了,要入京一趟。”

这件事韩冈的确听说了,“不过孩儿听说的是六月啊?”韩冈记得当时他听到这个消息,差点没骂出口,对审官东院的判院恨得直咬牙。韩千六都五十了,竟然让他在天气最热的时候入京城,推迟一两个月又有什么关系。

“因为赵隆那小子,还有苗家的大哥要去领军南方,经略司里面急着要准备粮秣,转运司又要保着仓里的存粮,两边来来回回的,最后用新粮抵数,中间多少事,整整耽搁了一个月。”

几年不见,韩冈的父亲也算是有了一点官员的气派,连说话用词也有了些改变。

“原来如此。”韩冈皱起眉,什么时候熙河经略司和秦凤转运司开始扯皮了。摇摇头,放下这桩心事,“不知茂州赢了没有。”

“赢了啊,过洛阳的时候就听说了。一接战就赢了,斩首有三千多,平了几十个蕃部,一路飞捷进京。”韩千六道,“当初也见过领军的王押班,好像帮了三哥你不少。这一次也见功了,果然还是有本事的。”

又是一个让韩冈发愣的消息。有赵隆、苗履在,加上熙河路的精锐,的确想输都难。不过赢得如此干脆,王中正的运气还真是好到了极点。

把这些事放在一边,韩冈陪着父母一起说着话,“怎么爹爹你上京,绕到了这条路上?”

“是你娘要去嵩山烧香。到了洛阳后就往南走了,绕了个圈子,本来是在密县坐船直接进京,不过到了卢馆镇,正好惠民河前面一段风浪沉了十几条船,堵了起来了,只能上岸换了车子。”

原来是烧香。韩冈正点头,就听韩阿李抱怨着,“你爹死板的很,到了洛阳绕路后,就不肯在用官车官船。其他做官的为娘的也见过,哪有那么多规矩?绕路的钱照付,不会沾官府半点的便宜,偏偏你爹不干。”

“瓜田李下也是麻烦,官船私船只要做得安稳,其实都一样的。”韩刚笑着劝道。韩千六不肯官船私用,韩阿李也知道用了还要付帐。而许多官员则占尽了官府的便宜,甚至借用官船来贩运商货,以避免途中的商税,这等操守还不如自己没读过圣贤书的父母。

韩阿李则狠狠的剜了韩冈一眼,“就偏着你爹。”

韩冈陪着笑:“娘是去了少林寺烧香的?”

“少林寺?你娘又不信禅宗,是嵩山大|法王寺!”韩千六像是想起了什么,“对了,三哥儿你还记得慧信和尚?”

韩冈皱皱眉头,他对佛教没什么好感,尤其是如今的僧人更是奢侈糜烂得让人恨不得再来一次灭佛,除了智缘等少数几个僧人,与和尚们根本不来往:“那是谁啊?”

“就是普修寺道安师傅的徒弟啊,矮矮的、胖胖的那一个。”韩千六似乎是很奇怪儿子竟然不记得当年经常买家里蔬菜的和尚,但韩冈的确是记不得了。都是多少年前的事了,老和尚的印象都有些模糊了,谁还记得个小和尚?

看见韩冈还是想不起来的样子,韩千六摇摇头放弃了,道:“这两年慧信正好在大|法王寺中挂单。他俗家的哥哥就在陇西衙门里做事,寄信回来说了寺中法华院烧香灵验,你娘就记下来了。”

“那娘是上京就是为了烧香喽?”韩冈最奇怪的是这一点,父亲上京是有审官东院的命令,母亲怎么跟着一起上京。

“为娘是来见孙子的!”韩阿李在车里瞪了韩冈一眼,“听说旖姐儿和南娘又怀上了,还有云娘也有了身子,都等了多少年。正好你爹要上京,就跟着一起来了。虽说衙门里面只要你爹上京,没说不能夫妻两个一起进京城的。托三哥你的福,娘现在怎么说也是个郡太君,要上京谁能拦着?”

韩冈神色有些黯然。老夫妻两个留在陇西,唯一的儿子带着妻儿在京城为官。虽然是因为韩千六本人有官职、加上家业都在陇西不便离开的缘故,但韩冈几年也不见父母,的确做得不对:“是孩儿不孝。”

“三哥儿你做官在外,也是没办法的。”韩千六笑着宽慰。

车马一起向前,一家三口就在大路上聊着。

“路中现在怎么样了?”韩冈问起了乡里的情况。

“熙河路哪有什么可说的。”韩阿李摇着头,“户口一年比一年多,田也是越种越多,粮食早不用外路运了。棉田也到处都是,连董毡那边都开始种棉花。也有种油菜的。还有种苜蓿的,用来养马、肥田。加上路中本来就产盐,岷州又有铁。现如今吃穿用什么都不缺。”

“平日里闲下来,市井里面也有百戏、说书消遣,全都是各家从京里请来的。不过最多的还是去看蹴鞠。”韩千六接口说着,“去年巩州联赛是青唐部赢了,顺丰行只是第四。而东街和巡城两队降了级,升上来的都是刚成立才两年。不过今年我们的顺丰行里面来个新人,脚法着实了得,能把头名再抢回来。听义哥儿说,如今京城里面也有蹴鞠联赛了,就跟熙河与秦州一模一样。”韩千六笑道,“过去什么都是学着京城,现在总算有一桩是京城学着我们关西了。”

“蹴鞠联赛,京城?”韩冈再一次感到惊讶,他离京时一点消息都没有听说啊!

“也是今年才开始。”冯从义回头笑道,“早前几年一直都被京中的齐云社一直拦着,好不容易才疏通了关系——也是靠着三哥你的名头。现在我们的棉行是一家,马行也是一家,还有骡马行、茶行、铁器行、金银交引行,再有就是朱家桥、保康门两家瓦子,都是有生意往来的,总共八家各建一支球队,一起参与联赛。各自的球场都备好了,赛程也定下来了,就待过了秋分后开始。”

蹴鞠联赛是韩冈当年在熙河路推行的比赛,如今也是在熙河路最为盛行。就韩冈了解到的消息,熙河路的几个州都成立了类似于后世足协的齐云社。由齐云社主持联赛,汉人、蕃部都组队参加。参赛球队数量最多的巩州联赛,如今都已经分成甲级、乙级,连升降级制度都有了。而这两年秦州也因为参加棉行的豪族发力推广,规则一如韩冈所制定,而不是现在在京中流行的往立在球场中心的风流眼中踢球的形式。就是京城,因为民风的问题,韩冈还以为要好些年才能在开封传播开来。

“关西的蹴鞠见血的时候多,到了京城就怕没人看。”韩冈笑着。

“就是见血才好。软绵绵的都没人看了。”冯从义哈哈笑了起来,“京城里面哪一场相扑不是围着人山人海,越是厮杀得狠了,叫好的人就越多。”

韩冈自重身份,以两府为目标的他,在京城的时候,哪里会去逛街市,更别说去看相扑了。不过相扑受欢迎他是知道的。

这个时代的体育娱乐活动都是太过温和了,就连在两汉,属于练兵之法的蹴鞠,到了宋代之后,就变成比试准头和花哨技巧的游戏,就像踢毽子一样,哪个踢得漂亮,哪个得到的欢呼声就越高。哪里像熙河路,谁敢玩花活,直接一脚就连球带人一起踹飞了。且几年下来,已经自发的形成了战术理论,各队比赛起来都有了章法。

这样的蹴鞠联赛,若是能在京城推广起来,多少能改变一下民风,韩冈很是乐于见到。

