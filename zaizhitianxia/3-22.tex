\section{第十章 大河雪色渺(上)}

长安常平仓的情况其实真的很糟。

程昉只是在出京前,去三司中的度支司,看了一下永兴军路转运司,在年终前送上来的账簿,就知道了他今次接手的任务不会那么简单。

白渠灌区去年因为广锐军叛乱,而差点变成了荒地。为了重新恢复这座关中粮仓,去年和今年,长安各仓中都有大批的粮秣被调往泾阳、高陵诸县,用以赈济灾民,以防户口流失。也就是今年六月,三县的夏粮虽然不比旧时年景,好歹比去年有了点起色,这才让天子和朝堂放下心来。

只是白渠各县今明两年还在免赋期中,朝廷没有田赋可以收取。这样一算,并加上明年的预期,总计三年的白渠灌区的直接损失,一出一入就有百万石之多。如果算上灾荒对周边经济的影响,按照这个时代的计算方法,单位以贯(钱)、石(粮)、匹(布)、两(银)来计点,朝廷的税赋损失,当在两百万以上。

加上因为横山开边而引发的亏空。这两年,永兴军路转运司用着四柱清账法的账簿上,元管、新收、已支、见在四项,‘元管’、‘见在’一年少过一年少,‘新收’连续两年在低位划过,而‘已支’一项上的数目,却是让人触目惊心。

而且更为让人头疼的,明年的亏空依然无法改变。以郭逵为首的关中亲民官们的考绩,那是一个比一个凄惨。郭逵倒也罢了,下等的考绩,对他来说无伤大雅,不会伤筋动骨。

但普通的京朝官,一个下上、下中的考评,磨勘就要延展一年或两年,也就是要想晋升,就必须再多等一两年时间。多少关中官员哭着喊着要调任,把始作俑者的赵瞻恨得要扎他草人的也不知凡几。纷纷上书政事堂,说这根本不管他们的事,完全视广锐军和赵瞻给闹的。

只是华州今次真正糟了灾,毁了屋宅和大部家当的灾民,也不过千多户。长安的情况再差,还不至于连华州的几千流民都养不活。

程昉就很纳闷,为什么他自过了古函谷关之后,便接二连三地在路上看到背井离乡的流民。

就在风雪不断要吹开他裹身斗篷的时候,程昉依然在思考着这个问题。

虽然一名宦官,但程昉身上的任务并不是服侍天子或是宫廷中的哪一位。

这两年,赵顼越发的信赖宦官,不仅仅是让他们作为走马承受,出外探察各地民情。而是将军务、政务上的重要职司,也让宦官们去主持。军事上的王中正、李宪,政务上的程昉,都是现成的例子。

——其实也是新旧两党互相攻击的功劳。

赵顼虽然任用王安石,推行新法,却也不会只听一面之词。可旧党和新党从来都是针锋相对,一个说是,一个说非。一个说左,另一个就偏要说右。这样的情况,让赵顼如何去确认是非曲直?他想要了解真相,唯一能依靠的,也只剩宫中的这群阉人了。

三天前,程昉奉旨出京。一路西行,白天都骑在马上,不停的在驿馆换马,一天便赶出近两百里。就算今早出发时,看着天色不对,也无意耽搁片刻

这两年,程昉一直都在堤上、滩上,风吹日晒的经历不比老农要少。雪下得大了,他也不回头,找那间刚刚过去的客栈,而是继续往前,冒着风雪一路走了十五六里,才在漫天的雪白中,找到了路边上的一处驿站。

在风雪天中,走了一个多时辰,跟着程昉出来的一队神卫军士卒满腹怨言,连两个依例被派来保护程昉的班直护卫,也是一肚子的抱怨。

进了驿站,这些吃够了苦头的赤佬们,便把一肚子个火气发泄到大厅中的百姓们身上。

“滚,别当爷爷的路!”神卫军领队的小校一鞭子将没有及时闪避的老头子抽开,又一把扯住跑过来阻拦的驿丞。鼻尖对着鼻尖,眼对着眼,恶狠狠的说着:“我等奉天子命,护送天使往华州探察灾伤。还不去腾出上房来,耽搁了明日的出行你可担当得起?”

驿丞被瞪得满头虚汗,驿馆厅中更是鸡飞狗跳,已经在厅中打上地铺的七八家百姓奔走躲避,几个幼童被父母扯着,吓得哭喊起来。原来还算安静的大厅内,现在变得一片乱象。

神卫军小校听着看着,觉得闹心,又一把抓着驿丞:“天使再此小住。你还不快将这群闲杂人等,全都赶到外面去?!”

程昉心中大急,下雪天将人——看样子还是离乡的流民——赶出驿馆,这事传扬出去,肯定没他的好果子吃,附近文官们的弹章都能把他被淹没。他连忙叫道,“你们还不住手,不要惊扰百姓!”

但程昉身边的两名班直护卫却拦住他,“都丞。他们只是一片孝心而已。”

程昉的脸色都气得发青,却毫无办法。

今次随行的这些个赤佬,连续几代都在京师军中混迹。各个滑不留手,根本不怕得罪程昉。事情闹得大了,到最后也肯定是程昉倒霉。文官们的板砖只会往宦官头上招呼,谁还会找他们这些蚂蚁虫豸般的小人物麻烦。

只要不是聚众闹事,违逆军令,做的看起来仅仅是仗势欺人的活计,风风雨雨都有程昉这样的大树给挡着。他们这些士兵就最多挨点训斥、罚点俸禄而已。

两个班直看着程昉急怒上火的表情,心头煞是痛快。辛苦了四天,终于出了一口鸟气。再看了程昉一眼,各自冷笑在心中,别当他们军汉平日里任打任骂,就是好招惹的。贼咬一口,都是入木三分。真要捅你一刀子,你又有什么办法?

几个士兵刚刚把占着一张桌子的行商踹走,正回头一起对程昉说着,自己这是在想都丞尽孝心。就见着有人站了出来:“孝心?!……这是什么话,谁教你说的?”

见到有人出头架梁,几个士兵都聚了过来。驿馆里常有官宦出没,但从门外的车马上看,不是高官显宦的规格,最多几个选人或是小使臣而已。三班院里吃香,阙亭之下守骨头的货色。身为班直护卫,隔几日就能见一次天子圣容的人物,却不会把这等人放在眼里。

“我等是奉旨出京!”一个神卫军小卒立刻跳了出来:“你是哪里来……”

种建中直接打断了他的话,自报家门:“本官种建中,家叔现在京中任龙神卫四厢都指挥使。”

除了两个班直外,其他几人的脸色都白了。是种太尉的亲侄儿,响当当的衙内。若是惹恼了他,随便找个借口,就能把他们从禁军发遣到厢军去。将不适任的士卒降入下位军额,这是有先例的,种谔也有这个权力,找几个不长眼的蠢货作伐,真还是轻而易举的事。

县官不如现管,在程昉面前可以滑不留手的软顶着,可他们顶头上司的侄儿种建中却是让他们不敢招惹的存在。

当神卫军的士卒软了下去,两个班直护卫却仍是不同声色,他们是天子近卫,根本不怕有人想跟他们过不去。一人向着种建中道:“种衙内,我等是奉天子诏前往华州。衙内想要阻止吗?”

种建中被当头堵了一下,脾气便要涌上来了。

而此时的程昉,却在看着种建中后面的同伴。与种建中同样高大魁伟的年轻人,并没有出来训斥。程昉知道,并不是他不够资格教训人,而是因为他身份更高。不过班直护卫已经成功的将种建中堵上了嘴,正得意的笑着。

这个时候,坐在一边的年轻人终于有了动作。

“尔等即是天子亲卫,如何还敢在地方上欺凌百姓?可是想败了天子盛德?!”韩冈训斥了两句,矛头一转,却直指程昉:“程都丞!此二人即已配属你之麾下,何以不严加管束,以至于让其再此恣意妄为?”

韩冈颐气使指,训了两句,就训起了程昉。

这里的都是惯看得眉眼高低的滑头,从韩冈的口气以及态度上,可以看出他的架势绝不是种谔的侄儿能比。一时气焰都收了起来,若是程昉顺水推舟,罪名可就落到自己的头上了。

程昉上前与韩冈、种建中见礼:“程昉见过种衙内。程昉见过……”他拖长了声音,等着韩冈报出姓名。

韩冈也不隐瞒,随即报上名讳:“韩冈。”

程昉气息一窒,而周围还没走远的军士们,更是心头一颤。竟然是韩冈。连忙道:“可是收复河湟,一颗仁心救治万民的韩玉昆?”

“不敢当,为国效力,为天子分忧而已。”韩冈拱了拱手,“都水丞的姓名,才是如雷贯耳。”

程昉的名字,韩冈听说过。虽是宦官,却是王安石重用的人物,在治水淤田上有着很出色的能力。农田水利法尽管在新法中并不起眼,但功效却一点不弱于青苗、免役诸法。

程昉为都水丞,统管河北水利深、冀、沧、瀛诸州,也就是原本盐卤黄河河口一带,淤灌出上万顷上等良田。

不似在横山和‘屡立殊勋’的王中正这般引人注意,但程昉在河北的功绩,也让他成为赵顼心中可以重用的人选。今次他上京回禀漳河淤田之事,便被加了个察访华州灾伤的临时差遣,派到了关中来。

