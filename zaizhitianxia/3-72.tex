\section{第25章 闲来居乡里(五)}

窗外天色已经完全黑了,星月挂在天穹上,可太阳虽已经落山了,但气温还是没有降低多少。

在书房中坐了一阵,感觉不到有风吹进来,韩冈和冯从义都觉得坐不住了。从房中出来,到院子中坐下。

命下人端来了用井水冰镇过的香薷饮,跟冯从义一人一杯的喝了两口,韩冈问道:“前日还在东京时,让你定下行会的章程可有了眉目?”

冯从义忙点头,从袖子中掏出几片纸来,他知道韩冈必然要问,事先就带着身上,“已经草拟好了,不过还需要讨论和修改的地方。另外,怡和号的大掌柜成轩,他还有一条意见。”

韩冈接过章程草稿,也不看,问道:“他有什么意见?”

“以成轩的想法,棉花的出产肯定是有行会来包揽,既然不会留给外人,不如一开始就定下为好。也就是在下种时就给付定金,将棉田的出产给下定,而不是采摘下来再买。能早一步拿到钱,田主应该不会不愿,而商行实现将棉花给定下来,各自也能放心得下。”

冯从义说着,看着韩冈的反应,不知他能不能相通其中的关窍。

而韩冈,对此是了解的。

不就是定金预付制度吗?后世有,如今也有。

比如福建的柑橘举世闻名。为了争夺柑橘的采购权,行商们每每都是在春天便来柑橘园,将今年的出产给定下。并不是简单的订购协议,而是直接确定到单株的果树上,选定之后,在树干上系上标志,并给付定金——多少株果树,付相应数目的定金

基本上在三四月间,一片柑橘园就会被几家行商给分包掉。到了收获的时节,行商便各自带人来采摘。付的钱就按照事先签订的协议上来付款。运气好的,自己选的果树大丰收,运气不好,那就是亏大本。

当然,如果是绝收,果园园主也会将定金给予退赔一部分。若是丰收远超预计,行商也会在余款上补足一些——这是为了长久的合作而为之,已经成为惯例。

叶涛的老家在龙泉,他家中就有一座柑橘园,占了两个山头——龙泉是山水多,田地少,九山半水半分田,所以果树种植是龙泉的主要营生——当韩冈在与来自龙泉的叶涛聊天时听说了此事后,立刻就反应过来,这根本就是期货制度的雏形。

不仅仅是柑橘,荔枝、龙眼等南方的贵价水果,生产和销售其实都是如此。所以冯从义一说,韩冈就立刻明白了。

“这个可以考虑。”他点点头,“让他们与田主去商量好了,我这边没问题。”

几家商行在陇西有棉田,但他们从来都不占大头。随着棉纺业的发展,陇西百姓种植棉花会越来越多,所以为了能控制棉布的生产,必须采用这样的手段。否则根本争不过冯从义和他身后的韩冈。如果能做到事先下定,就可以在行会内部对于资源加以分配。

成轩的小心思是有的,不仅是韩冈,冯从义也能看得出来。但冯从义更明白,韩冈的目的是尽快推广棉布,分出去的这些利益他并不在乎。

而且相对的,若是能在棉田播种前,就将田里的出产给卖出去,田主就可以保证稳定的生产。对于农民来说,最困难的时候就是青黄不接的时候,如果事先有一笔定金,可以保证能安心种植,不用担心日后没钱。比起去借便民贷,用定金其实更为安心。

对于事先订立的协议,也许会有人在收获后不肯履行,但行会内部的自律性,能保证其中个别的欺诈行为不至于影响到整体——就算是东京城中那些行首们,在对外时,也会维持下面行会成员的整体利益,而不是涸泽而渔。

“具体的协议他是怎么定的?”韩冈又问着。香薷饮两口喝完,自己摇起了扇子。

“按照前三年的平均亩产决定出价。事先给付一成定金。就算是绝收,定金也只收回一半,如果有出产,只要能超过前一年的八成,不论出产多少,都是按照一开始的协议付账。不到八成的话,有多少按比例给付。”

“给出的条件也太苛刻了,才一成的定金,绝收还要退一半,过八成才付全款,没听过这么苛刻的协议。还是依照浙江柑橘的合同来,不然就免谈!”韩冈决绝的毫无余地。

“免谈什么?”

冯从义正想再说,疑问声从身后传来。韩冈和冯从义忙回头,就见韩千六不知什么时候过来了。

韩冈两人忙起身,让了韩千六坐下。

等儿子和内侄将方才讨论的话题说了一遍后,韩千六摇头:“这样是不好。就是俺,俺也不干。”

“儿子也是这个意思。”韩冈对表弟道:“万事都要站在田主的一边,我们跟那些商号不是一路人。明不明白?”

冯从义重重的点头:“小弟明白,会转告成轩的。”

对儿子的态度,韩千六很满意。虽然顺丰行是他韩家的,若是按照方才的条款,韩家就算在田亩上亏一点,从商行中就能赚回来。但韩千六的想法,还是站在种田人的一边,“种田不容易啊,做商人的张张嘴就将一年的辛苦全吞了,还不肯担风险,哪有这么好的事。”

“爹爹(姨父)说的是。”韩冈和冯从义异口同声。

不过韩千六又感慨着:“正要说苛刻,那还是对棉花的。要是粮食能这般事先下定就好了。”

韩冈和冯从义又同时摇头。

“不可能的。”韩冈对父亲解释道,“只有像柑橘、荔枝,或是棉花,这样可以保证足够利润的作物,才能让人放心的采用预付定金的。除非是绝收,否则就算最后只收获到半数,也是有赚头的。要不然商人们争着付定金做什么?还不是怕钱给别人赚了。而粮食不同,青苗期和收获期的价格差距太大,利润又太小,以预先订购的方法来处理,粮商们三五年内,个个都要破产。”

冯从义附和道:“三表哥说的没错,正是这个道理。”他正说着,突然手一扬,在脖子上重重的拍了一下,“有蚊子。”

“有蚊子?”韩千六立刻道,“三哥去拿花露水来。”

“花露水?”冯从义疑惑的问着。

韩冈依言起身,对着:“是前些日子用烈酒来泡的外用药酒。可以避暑驱蚊,闲极无聊,就给起了个好听的名字。”

去书房,韩冈拿来一个巴掌大的小瓷瓶过来,正常是用来装伤药的,递给了表弟。“前几天,你嫂子已经让人给你家送了六瓶过去,弟妹过来也说好用。”

接过来打开塞子,顿时飘出一股薄荷的香味。冯从义从瓶中倒了一点花露水在手上,抹开来,就是一阵浓烈的薄荷香,然后就是一阵凉意。

“怎么会这般清凉?”冯从义惊讶不已

“酒水都是一般,化气后会吸热。”韩冈解释着。

“这个小弟知道。”冯从义也不是没用过烈酒来清洗伤口,外敷时就是一阵清凉感。不过这个花露水的效果要强上不少。

泡薄荷,加上冰片,虽然两种药材中冰片算是贵的,但还是用得起。记得花露水好像是有冰片和薄荷,可再往下的成分,他就在记忆中找不到了。

冯从义其实不关心其中的原理,他的经济头脑让他几乎在一瞬间就明白了花露水的价值所在。

烈酒的成本高于普通的淡酒,价格更是要高出五六倍,但蕃人还是大批的购买。虽然韩冈拿着阴阳论来吓唬人,可在军中,还是有人喜欢,拼死要喝。烈酒喝下去就是一团火,烧刀子的诨名那是再贴切也不过。大冬天的时候,在外面喝上一口,寒不侵体,浑身都能热起来。

只是烈酒价格再高,也比不上有香味,能避暑驱蚊的花露水。而且用薄荷就有薄荷香,要使用玫瑰呢,用栀子呢,用桂花呢?冯从义双眼泛着黄金的光芒,将小瓷瓶托在掌心,如坠梦中的说着,“表哥,还费力气种棉花作甚?这个花露水可就是一座金山!”

“暂时还不行。”韩冈冷静的摇着头,“原料可是烈酒!”

制作花露水要消耗大量烈酒,单是原材料就很麻烦了。如果没有榷酒制度,韩冈早就想办法让人去造了,就是因为酒水是由官方专卖,他才没有去开花露水的作坊。

酒水一物,官员私家酿一些无所谓——自家喝或是馈赠亲友都可以——但大量出售,就是一桩罪名。不查还好,一旦有人找麻烦,查出来谁都脱不了身。

“赚钱的方法千千万,棉布难道不赚钱?何必用上会留把柄给人的手段?”

“那……”冯从义看了看手上的小瓷瓶,白灿灿的竟散着纯银的光芒,惋惜之心油然而起,“实在是太可惜了。”

韩冈笑道:“作为礼物送人比较好。日后做个人情,比起赚钱更有用处。”

药王弟子家特产解热避暑的花露水,怎么都该价值千金,用来送礼,自然是金贵异常。

“不说这些了。先将眼前事做好。”

