\section{第29章 百虑救灾伤(三)}

云娘坐在马车中,对面是招儿和墨文。而前面几辆车里,周南、素心带着孩子坐了一辆,而主车中,则是有韩家的主母坐镇。跟在车队外,有着二十几名韩家的家丁,加上相府派出来的一众护卫,声势很是不小,行在路上便已是人人注目。

一行人昨日从东京城出来,在半道上歇了一夜,今天一早就继续上路。终于在午后赶到了白马县。摇摇晃晃的马车,让招儿、墨文两个小女孩儿变得昏昏欲睡,头耷拉着。而云娘却毫无困意,为着即将能看到挂念在心上、日思夜盼的韩冈,而雀跃不已。

想到很快就能见到三哥哥,胸中就有一股暖意,甜甜的微笑不知不觉的挂在脸上,也不知时间过得飞快。

一直都在摇摇晃晃的车子突然停下,车厢猛的一定,云娘也从思绪中惊醒过来。两个小丫头也一下被惊醒,揉着眼睛,“云姐姐,是不是到了?”

韩云娘摇摇头,见着招儿要掀开车帘向外看,连忙一手拉住她。虽不知出了何事,但听着车帘外的人声马声,想也知道不能随便向外张望。摆出大姐姐的姿态,提醒着两个妹妹一般的小丫头:“要坐坐好,不要乱摸乱动。失了身份,会惹人笑话的!”

“到了吗?”

听着前面的车夫吆喝声,素心抱着儿子问着对面的周南。

周南先小心的理了理裹着女儿的小斗篷,方才抬起头,听着外面的声音。从城外的空旷,到了城中街巷上的嘈杂,“好像是到了。”

“终于到了。”素心轻声笑了笑,笑容中不无疲惫之意。

她们带着儿女出行,这一路上的确也是累得够呛。一两岁的幼儿出门远行,其实很是犯忌讳,一个不好就会生病,甚至有夭折的风险。

不过韩冈不知是不是太有自信,还是根本没有想到这一点,信中并没有提将孩儿留下给父母照顾。而韩家的父母,甚至连同王旖、素心、周南都对鬼神之说有些迷信,竟然也放心的让两个小孩儿一起跟着出来。

药王弟子的身份,韩冈虽然不承认,但他在治疗上的开创却是世所难及。素心、周南总觉得有这样的父亲,她们的一对儿女也不会出任何问题,便安安心心的坐车东来。而在这几千里的行程中,两个孩子倒真是奇迹一般的一点病症都没有。

搂着沉沉睡着的一对儿女,素心和周南绝美的俏脸上,都是带着一丝期盼。已经到了白马县中,那么很快就能看到那个狠心肠的夫君了。

车轮碌碌,碾过了白马县的大道。

在外面的看到了这一行车队的行人们,开始交头接耳。不过半日的时间,消息早已经给传开了,都知道是如今知县的夫人终于到了。

载着韩家内眷的几辆马车,停在了县衙的偏门外,周围的闲人都被随行的护卫驱散,清出了一块不受窥探的场地。

王旖坐在车中,等着韩冈出来迎接,或是让她熟悉的人过来相迎。但她所听到的却是一个陌生的声音:“魏平真拜见夫人!”

云娘、素心和周南都想早一点见到韩冈,王旖也是一般,如今已经说不上是新婚燕尔,但自从入了韩家门后,就聚少离多,怎么能不挂心?

本想着立刻就能见到丈夫,可没想到却是一个陌生人来迎接。

“官……正言呢?”隔着车帘,王旖问着丈夫的去向。自己都已经到了,昨天也事先传了消息回来,怎么不见韩冈在衙中等候。

听着那个陌生的男声在外面回话道:“回夫人的话,正言今日出城去视察流民营,现在还没有回来。”

王旖知道韩冈现在的确很忙。自己前日刚刚回到东京,大哥就籍故请假,匆匆赶到白马县,与自己的夫君商议要事。作为知县,不但要顾着县中的灾情,还要帮忙参议国家大事,怎么说也算是大宋一千八百知县中的独一份,当然是忙。只是看到兄长和丈夫关系亲和,丈夫还愿意帮着出主意,王旖原本存在心中的担忧也不翼而飞,心情也好了许多。

而王旖也从王雱那里了解到,自己丈夫要处置的事情,不仅仅是他去白马商议的那一桩。现在压在韩冈身上的要务,件件都事关重大,忙得连脚都歇不下来。每天都有半天在外面视察灾情,此外还要整顿保甲,严防流民作乱——大灾一起,盗贼遍地。免不了的事,当然要事先预防着。

所以在家中时,母亲吴氏还千叮咛万嘱咐,到了白马县后,要好生服侍着丈夫,将后院管好,不要让他在外面累着,回到家里还要烦心。

对于丈夫的辛苦,王旖很能体谅。但体谅归体谅,可当真到了县中,却不见丈夫出迎,王旖的心中也不免感到有些委屈:‘哪有忙成这般模样,让一个没见过的幕僚带着仆妇在外面候着的道理!’

魏平真也觉得今天的事让人头疼。他从来没有见过王家的二娘子,作为一个陌生的男子,在没有韩冈出面介绍的情况下,就算以幕僚之亲,也不便先拜见韩家的主母。

在韩冈如今的三个得力幕僚中,魏平真最为老成持重,当然不会做无礼之举。谁也不知道,王家的二娘子是什么脾气,更不清楚韩冈的三位妾室又是什么性子,不小心冲撞了内眷,日后也不好做事。

王旖带上了帷帽,先从车中跳下里的侍女为王旖掀开了车帘,小心的扶着知县夫人从车上下来,在内庭听候使唤的仆妇立刻跪了一地,而魏平真见了王旖掩了面容,松了口气,低下头,半弓起腰来行礼。

“都起来吧!”王旖摆出了主母的架势,又向魏平真行了一礼:“魏先生万福。”

王旖虽然年纪不大,但出身自宰相家的身份,还有在官宦门第养出来的气质,让她一开口就立刻镇得住场面。

大户人家该有的规矩,王旖当然知道。像她这样的名门闺秀,从七八岁开始,家里便开始着力培养各方面的才华。德言容功,为妇四德,这每个大户人家的女儿必须要遵守的铁律,当然都要学着。‘妇德,贞顺也;妇言,辞令也;妇容,婉娩也;妇功,丝麻也。’这四件事,没有哪一家不去逼着女儿用心遵守,否则就会成了世间的笑话。

但更进一步的治理家中内外事的才能,各家各户却不一定能教授得好。在这方面的教育水平如何,官宦人家的底蕴立刻就能从中分辨得出。

王旖只是站着,就自有一份当家主母的气质,没有半点小家子气的寒酸。魏平真也不免点点头,韩冈有这样妻子,就不用担心后院失火了。她下来后,周南、素心和云娘也都跟下了车,同样带着帷帽,不露半点真容。

魏平真引着王旖等人进了县衙,在通往内庭的二门处停了步,再往后,就不是他一个幕僚可以涉足的区域的。恭声又问候了几句,吩咐了此前管着县衙内庭洒扫庶务的两个婆子听候王旖的吩咐,魏平真接着便告辞而出。也省了王旖出口遣人,而伤了感情。

王旖轻轻跨过门槛,走进属于她的一片天地。掀开帷帽,温温和和的一对眸子却有不怒而生的威仪,回头吩咐着仆妇:“你们且各自去做事,一切依着旧例!”

一个个箱笼被搬了进来,男人搬家只要一个包裹,而女人搬家却是大箱小包。这个道理哪里都是一样。素心和周南在家中都有一份事情要做,也听着王旖的指派,做着自己的事。终于有了主心骨的县衙后院,如同终于有了水的水车,终于开始正常的运作了起来。

到了傍晚的霞光占据了半幅天空的时候,韩冈终于回来了。

别过方兴,又问候了魏平真和刚刚从县学回来的游醇,韩冈脚步匆匆的赶回后院。

妻妾儿女今日抵达的这件事,他并不是忙着忙着就给忘了。心中虽然记着要早点回去,但也没想到只是在流民营饶了一圈,就已经到了快入夜的时候了。这还是比较近的流民营。如果等到明年开春灾情不减,其他四五处预定的流民营地一起住满,他要去视察营中情况,一天的工夫还下不来。

久别的妻儿,韩冈哪能没有记挂,经常也是想着。不论是一对可爱的儿女,还是那四名娇妻美妾,哪有不挂念的道理。只是他的时间被许多事给占满了,只能在闲暇的实践中。

在此之前,对于忙忙碌碌对韩冈来说,这个院子不过是个睡觉的房间,加上读书的地方。但看到一盏盏灯火在房中亮起,而灯下的倩影俏生生的等着自己,韩冈的心头有了一阵暖意。位于县衙后方的这个院子,好不容易有了一个家庭的感觉。

微重的脚步引起房中的注意,迎上四张如花俏靥,韩冈微笑着:“我回来了!”

