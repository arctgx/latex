\section{第一章 庙堂纷纷策平戎(六)}

【第二更】

“徐德占就是不能忍下的脾气,若是老实听话倒也罢了,要是下面的兵将对他的吩咐敢稍有违逆,他肯定会杀鸡儆猴。”吕惠卿说着,嘴角露出了一个讽刺的笑容,“即无重名,又无恩信,更无功绩,不靠杀人立威,还能靠什么?韩琦当年都得靠这一招……现在该多想想韩玉昆那边,河北轨道之事要暂时放一放,不知他下面怎么打算。”

“已经确定要停下来了?……从一开始韩冈便在设法拖延出兵,还是在京西的时候就是这么在说,是不是就是因为河北轨道之事?”

“嗯。”吕惠卿点头,“河北轨道缓不济急,又是大耗钱粮,跟用兵西北相抵触,眼下肯定要耽搁几年。韩冈一力反对攻打兴灵,当也是有这个原因在。”

“也可能就此搁置。”吕升卿道,“还记得韩冈当年建言的束水攻沙。王介甫在的时候,也只来得及将外堤修起来。等到王介甫去职,结果就搁置下来了。除非等到日后哪天破堤,或是韩冈秉政,否则都可能一直拖下去。”

吕惠卿笑了一笑。束水攻沙的方略,是韩冈首倡、王安石力推的河防方案,但王安石去职之后,哪位宰执会为韩冈和王安石做嫁衣裳,将他们留下的摊子重新支起来?一番辛苦,最后功劳可是要算到王安石和韩冈头上。黄河大堤现在稳得很,东府的宰相、参政有志一同的拖一拖,天子都没办法。

“但轨道和河堤是两码事。”他收敛了笑容正色说道,“方城轨道的人货运费一月两万贯,抵得上京城市易务的营收。只为了这份收入,河北轨道迟早要建的,何况还有方便调兵的好处在,天子不会让人拖太久的。”

“那韩冈现在也只能等着了,等到官军攻下兴庆府……反正他又不缺功劳,等年纪到了自是能进两府,等上一阵也无所谓。”

吕惠卿没什么表情的端起茶盏,凑到唇边啜了一口,却什么都没有喝到。低头一看,却发现杯中早就空了。

他放下茶盏,站起来,有点烦躁的推开窗,寒风顿时涌了进来。吕升卿打了个寒战,吕惠卿则浑然不觉的站在窗边,望着西侧犹在闪烁的灯火,问道:“正道还在西院?”

正道就是吕惠卿女婿余中的表字。吕升卿闻言点头:“大哥回来前,正道说今天晚上要跟十一他们好好说一说今科考试的要点,多半还在用功……正道是国子监直讲,又是状元,十一他们三兄弟有他指点,一榜进士更有把握了。就算落了榜,去国子监读三年出来也不错。”

吕惠卿神色一缓。

吕氏虽说乃是福建望族,进士多得跟石头一样不值钱——吕惠卿中进士的嘉佑二年榜,同科的兄弟、族兄弟,有德卿、和卿、虞卿、京卿;两年后的嘉佑四年乙亥科,有谅卿、温卿;熙宁三年吕升卿高中;熙宁六年,则是吕惠卿族兄吕乔卿的两个儿子吕阳、吕厚中榜,与韩冈同年;吕乔卿中进士比吕惠卿早,是在庆历二年,与他同科的还有一个吕夏卿,苏颂和王安石与他们是同年——但进士就是进士,能多一个总是好的。

过了年后就是礼部试,吕家今科又有三名子侄上京应考,正住在宅中。吕惠卿和吕升卿的心思都放在手实法上,加上如今的,没多于的精力去照顾他们,干脆托付给余中,

余中是吕惠卿的女婿,与韩冈同榜,而且是状元。这两年都兼了国子监直讲,在太学中为两千四百名太学生讲学,除此之外,还有太常丞的职司。除了韩冈以外,他在同年中算是升得最快的。

“大哥。”吕升卿有一些犹豫的说道,“正道还有件事本是想要跟大哥说的,但正好徐禧来了,就没来得及说。”

“什么事?”吕惠卿关上窗子,坐回来。

“有个外舍生最近公然宣称,太学讲官不公,校试诸生,升补全凭私人喜好。而且讲官赴太学,巳时入,午时便出,疏怠公事。所以正道就想跟大哥提一下,讨个主意。”

吕惠卿听着神色一凛,厉声问道:“这是怎么回事?!”

吕升卿笑道:“只是落选之人心怀嫉恨而已。太学确定了升舍名单,虞蕃不在其中,心怀不甘。不是什么大事,正道只是提了一句。”

吕惠卿可不会相信事情会这么简单,否则没必要郑重其事的说出来,但余中毕竟是自家的女婿,在弟弟面前有些话就不好说,紧皱着眉:“这件事小心一点,御史台中没人不想办一桩大案,扳倒一个宰辅,然后一举成名。想出名想疯了,给他们找到一个机会,肯定要兴大狱,彰显自己的才干。”

“能不能让舒亶他……”

吕惠卿摇头,“别指望。舒亶也是御史!”

吕惠卿从不认为自己有控制御史台的能力,以王安石当年受到的圣眷都做不到,最多也只是能逼着天子二选一而已。乌台中的御史,如果利益相合,他们会站在自己一边,可要说他们会老实听话,自己说什么就做什么,那根本就是做梦。任何一名御史基本上都是各自独立,不会听宰执的话,也不会听御史中丞的话,更别说作为副手的殿中侍御史。

蔡确就是现成的一个好榜样,当初捅了王安石一刀,现在都是翰林学士了,看样子不用多久就能晋身两府。在前途面前,一切都要靠边站。

“当真会到如此地步?”吕升卿苦着脸。

“以防万一而已。”吕惠卿尽量想要做出若无其事的态度,但他的表情却不是这么说。

嫡亲兄长的心情,吕升卿怎么会看不出来,沉声问道:“十一哥兄弟几个怎么办?”

吕惠卿想了一阵,道:“如果十一哥他们三个考不上进士,暂时也不要去国子监,等一年再说。”

吕升卿叹道:“只能暂时如此……但想要学问有所进益,肯定要与别的士子多往来。国子监是绕不过去的。”

“绕不过去就回福建,从福建再考贡生出来。虽说比不上章子厚,但对我吕家子弟来说,进士登科也并非难事。”

“也只能如此了。”吕升卿点头。

瓜田李下的嫌疑一定不能沾,尤其是手实法推行过程中,吕惠卿得罪了太多官绅,露出一点破绽都会成为致命伤。这样的情况下,今科几个应考的族中子弟,能考上进士倒也罢了,若是考不上,又去国子监想混一个下科的贡生资格,肯定会被人拿出来当成弹劾吕惠卿的利器,而且是一击致命的武器。

“好了……”吕惠卿又站起身,心中烦躁,不想再多说什么,“早点回去休息吧,明天还要……”话声一停,他摇摇头,现在进入了年节假期,在正旦之前,已经没有朝会了。吕升卿这样的普通朝官,可以在家好生休息了。

吕升卿识趣,点头起身:“小弟先回去歇着了,大哥也早些安歇吧,明天当还是要进宫的。”

吕升卿离开了,吕惠卿却又坐在书房中。眼下国内国外一片乱,一件件事,都让人头疼不已。尤其太学中的事,让他嗅到一丝危险的感觉,会变成一场大风波也说不定。

跃动的烛光在吕惠卿脸上留下摇晃的阴影。

军事上支持王珪亦无妨,交换来的,也就是手实法的不受干扰。但吕惠卿并不指望王珪会在自己陷入弹劾拉上一把,不踩上一脚便已是万幸。

吕惠卿知道自己的问题所在。两府中从来都是一个求稳的地方,不要太过突出的人,年纪也好,行事也好,都不能与他人差异太多。就是韩冈,治才在朝中亦是顶尖的,一样的投闲置散。

王安石推行新法,自身开罪了无数官绅,与多少旧友反目,为天子做到了富国强兵,到头来照样是出外,如今不到六十,就已经近似于致仕了。自己不过是要施行一部手实法,就要战战兢兢,如履薄冰。

反倒是王珪这样的庸人,却能在朝堂上安居无忧,从无一言违旨,自熙宁初年到现在,一直安安稳稳的坐在东府之中,笑看他人来来去去。只要不做事,就永远都不会犯错!

已经不是熙宁初年了,进入元丰之后,天子的心思更是越来越求稳不求变,吕惠卿如何看不明白。

但他学不来王珪,也不能去学。自己的根基建立在新法之上,就不能改弦更张。既然上了这辆车,成了驭车之人,就必须将车子赶下去,即便前方已是悬崖,亦要坚持到底。

步出书房,抬头向上,仰望星空。半轮明月高挂在幽蓝色的天幕上。月亮不见的另一半,不是消失,而是藏在阴影之中。

吕惠卿望着天上的半月,自嘲的在笑。自己也还身处王安石的阴影中,想要摆脱出去,想要做出一番成就,就不能退缩一步,半步亦不可!

