\section{第15章 前路多坎无须虑(二)}

【差点赶不及今天的第三更。明天的第一更,夜里是赶不出来了,看看明天中午之前能不能完成。不过,明天照样会坚持三更,这一点不会变。还请放心。最后,照例求红票,收藏。】

除去三伏天里越发显得炽烈的阳光和越来越刺耳的蝉鸣不论,六月下旬的秦州城显得十分的平静。白天的街巷上,看不到几个人影。车水马龙中的场面,只有在入夜后才能看到,不幸顶着烈日出行的行人,都是跟着趴在树荫下伸着舌头的老狗一样,挂着脸,叫着好热好热。

而进入六月后,六盘山对面的西贼也出乎意料的安分,大举进攻没有,小股骚扰也没有,连在秦州城门口被抓获的探子也少了许多,好像党项人也受不了这个可能是十年来最热的一个夏天。

至于秦州官场。李师中即将离任,此时已经不大管事了,他现在唯一要做的,就是把手上的帐目整理好,将里面的亏空弥缝住,等待郭逵来交接。

窦舜卿奉旨去了京城,不会再回来。原本横行城中的窦七衙内,他的案子在半个月前被陕西路提点刑狱司衙门给划走了,不再归属秦州管辖。这几天陕西路的宪使就在州衙里借了二堂审案。不论结果如何,定案后,窦七衙内都不可能再回秦州。

前任钤辖向宝拖着病躯,此时应该已经抵达京城。刚刚升任钤辖的张守约,在喝过几天贺酒之后,正在熟悉自己新的工作。因为此前张守约从来没有担任过钤辖一职,诸多庶务让他头痛不已。他身边又还没来得及招揽几个堪用的清客,便找上了韩冈,请他推荐两名深悉厅中故事、并且可以信赖的老吏来帮忙。

韩冈是勾当公事,勉强说起来,也管着胥吏的升迁。经略司中才能干练的胥吏,他都已经了然于胸,而惯于欺瞒上官的狡诈之辈,也是了如指掌。他向张守约推荐了两个,都能满足新任钤辖的要求。

送了两名老吏去见了张守约,面试过后,看得出来他很满意。被张守约留着说了一阵闲话,韩冈起身告辞。李信送了他从钤辖厅中出来,庭院中树荫森森,老槐依旧。但州衙三进东院的两个旧主,一个被他气得中风,一个则被他害得远走,现在暂时就只有张守约一人霸占着。

别过李信,韩冈顺路走到机宜文字的官厅内。赵隆正在门口百无聊赖的坐着,见到他忙站起来问好。韩冈往厅中看去,就见着王厚坐在堆满公文的桌案后,忙着处理王韶丢下的事务。

而王韶本人,韩冈知道,他正在后厅赶着写信,好跟朝廷打饥荒。另外,高遵裕也在做着跟王韶一样的事情——韩冈所出的计策乍看起来并不算好,但等王韶静下心来想过,让他自己拿主意,也只会做出同样的选择。

王韶早前是关心则乱。好不容易将几块挡在路前的石头都踢出去了,刚刚豁然开朗,正想大步往前走的时候,却又飞来一座山挡在面前,他没当场吐血就算心理素质好了,怒火攻心,冲昏头脑也是情理中事。

不比韩冈,并没有将毕生的心血和希望全数灌注进河湟开边事业中,只是顺势而为,说抽手就能下决心抽手的,甚至可以做到旁观者清。王韶在急怒下被蒙了眼睛,他反而看得一清二楚。

王厚忙得头也不抬,只看见他手上的笔在不停的动,一份接一份的批阅着。等走进后厅,里面的王韶同样没有抬头,他正给王安石写私信。王安石的脾气是有名的执拗,要说服他,王韶在写信时就必须很郑重的斟字酌句,以防有一点错漏。他正在聚精会神的检查着,全然没有发现韩冈的到来。

不想打扰王韶,韩冈随即轻手轻脚的退了出来,他低声问着身边的赵隆,“高提举来过吗?”

赵隆点点头:“前面刚来了一趟,跟左正言商量了好一阵子。”

韩冈笑了:“高提举也算是用心了,希望他们能成功。”

为了能赶在郭逵到来之前,将财计之事解决,王韶和高遵裕都是发动了手上所能动用的所有资源。只要钱粮到帐,就算郭逵来了,他所能动用的卡脖子的手段也就剩那么几个了。

王韶身边,现在就只有王厚和赵隆。王舜臣与杨英一起去京城了,去三班院报名,并等他们的官诰。

管着秦凤路经略司架阁库的韩冈,出手帮了王舜臣一个小忙,将他的年龄改成了二十岁。让他一下子就有了就任实职的资格——武臣与进士、明经一样,都是二十岁就能得到差遣——以王舜臣过往积攒下来的功劳,回来后至少能做个寨主。

当然,王韶肯定不会让一个箭术堪与刘昌祚相提并论的猛将,守在寨子里晒太阳。征辟为王舜臣、杨英为僚属的申请已经同时往三班院递出去了,就跟现在的赵隆一样。

见王韶和王厚都忙得不可开交,韩冈也不在厅中多留,直接走了出来。赵隆跟在他身后,到了院中,问道:“三官人,郭太尉到底什么时候才会来秦州?”

“大概要到七月中的样子。”韩冈算了一下。郭逵已经卸下了渭州知州的担子,但他还要去京城走一遭,这一来一回,就算他走得再快,至少也要到七月中,才能来秦州上任。

赵隆听了,一脚踹翻了院中石桌边的一具石墩。一脚之力,就让近百斤的石头咕噜咕噜的滚到了院墙边,“郭太尉半个月后才来,现在就忙成这般模样。等到他到了城门口,真不知会怎么样!”

“到那时反而会轻松下来,倒是赵兄弟你要忙起来了。”韩冈笑着拍了拍赵隆的肩膀,告辞离开。

回到自己的官厅,韩冈舒舒服服的在自己位置上坐了下,武大便端了凉茶上来。半闭着眼睛,啜着甘甜清凉的茶汤,便有着让王厚羡慕不已的自在。与王厚有着鲜明的对比,韩冈身前的桌案,被擦得锃亮,笔墨纸砚摆得整整齐齐,就是没有一份公文放在上面。

勾当公事的工作,韩冈早已是熟能生巧,同时有着官厅中胥吏打下手,他的那一份,早上用上半个时辰就能处理得差不多。而且以他这段时间培养起来的对公事熟悉的程度,就算再面临刚上任是一人做五份工的窘境,韩冈照样有自信一个上午就能全数解决,中午时就可以回家吃饭睡午觉。

而韩冈的另外一份差遣,也同样无事可做。甘谷、古渭两处疗养院的成功,新培养出来的人手,让韩冈有了在秦州城建立第三座疗养院的底气。不过这事需要经过经略使批准,现在李师中把公文都积了一堆,韩冈也懒得找他。等郭逵来了,再请他批一个没在使用的营地也不迟。

六月的后半,韩冈的生活就这么突然的轻松了起来。

每天都是去衙门里把事情做完,再翻一翻过去的公文档案,或是去王韶、高遵裕那里参赞一下计划,等到午后,就可回家去休息。他这般悠闲,便被偶尔晚上会请他出去喝点酒的王厚恨得直磨牙。

王厚再气,也拿韩冈没辙。过去几个月梦寐已久的轻松日子,就在这半个月中终于降临到韩冈的身上,他过得是悠然自在,可以自由的掌握时间,可以系统的把经传重新再研读一遍。

好久没有这么完整的读书用功的时间了,过去的两个月,事情一桩接着一桩,害得韩冈只能零零碎碎的抽空读书。积累下来的一些疑问,还要写信想张载请教。

韩冈从王厚那里听说了,张载因为张戬的缘故,辞去了官职,现在已经回到横渠镇的家中,据说要设立一座书院。韩冈准备等古渭大捷的封赏发下来,就分出一部分财物托人带去给张载。刚开始的时候,他是打着张载的名号才脱颖而出,自保得全。现在以财物回报,确是理所当然。

“进剑者左首,进戈者前其鐏,后其刃,进矛戟者前其镦,进几杖者拂之。效马效羊者右牵之,效犬者左牵之,执禽者左首,饰羔鴈者以缋,受珠玉者以掬,受弓剑者以袂,饮玉爵者弗挥。凡以弓剑苞苴,箪笥问人者,操以受命,如使之容。”

这一天午后,韩家书房中的读书声又按时响起,但从敞开的窗户中传出的声音,却不似前几日那般的清朗流畅,听起来有些拖沓。

真要说起来,九经之中,《礼记》一经最不对他胃口。虽然里面有着中庸、大学等篇章。

但还有十几章,一条条一款款全讲的是礼法,吉礼、凶礼、宾礼,吃饭说话该如何,接人待客该如何,面见天子该如何,规定得极其繁琐,让韩冈看着头晕。只是在科举中,这却是必考的内容。

这《礼记》中记载的古礼其实早就被抛弃了,世间通行的礼仪也是往简单中去。尽管韩冈从张载、程颢那里,都听他们说过要复古礼,王安石这位学术大师,也是喊着要复古,但实际上,周时的立法完全不可能在宋朝重新推行,礼崩乐坏,孔子说过,要复古,圣人也没能做到过。

不过为了一个进士头衔,韩冈就算再没兴趣,都能耐下性子来把礼记背得滚瓜烂熟。如果他现在就有个进士出身,这次古渭大捷的功劳一立,他直接由选人转京官都是可能的。

“进士……”韩冈突然叹起,抬头望着窗外的天空,“留下的时间可不多了!”

