\section{第24章 兵戈虽收战未宁(五)}

作为缘边安抚司的属官,韩冈现在很忙碌。

尽管他有领军出战,在战场上直面敌军的刀剑,但韩冈文官的身份,就决定了他主要的工作还是要在案头上解决。

今次一战,从苗授领军突袭星罗结部开始,到禹臧花麻撤军、苗授和韩冈回到渭源堡为止,跨度加起来也只有七天,比起最早计划中长达一个月的最大期限,缩短了许多。但这七天,消耗的物资并不算少。尤其是军械中的弓弩箭矢等物,几乎一扫而空。还有王韶为了稳守渭源堡,在禹臧花麻围城的那几日,他没少砸钱下去提振士气。这些钱粮物资耗用的帐本,都要韩冈经手、过目、检查、修改,并注明理由。

另外,渭源堡的修筑本是由王君万主持,苗授和王厚监工。但一战之后,短期内渭源当不会再有敌军来袭,为了节省存粮,苗授和王韶商议后便领军回师,带走了大部分的兵力回古渭。而王厚也被王韶派回了古渭,向高遵裕对此战进行通报。而两人丢下的工作,韩冈也不便让王韶这个主帅来处置,只能自己做起了监工。

韩冈的‘监工’并不是拎着皮鞭在工地上巡视,看到不卖力的就上去抽几下——这是手下人的工作——而是监察工程进度和完成质量,从这一点上看,已经跟后世的工程监理没有多少区别了。

苗授和王厚做监理时,是将一千多民伕每百人分作一队,从军中挑选得力人手下到民伕队中去做监工。而韩冈接手后,则是把所有的监工都召回,让民伕自行推选出人望高的领队,各自承包一段工作量相当的工程。每天下工后,计算工程完成的情况,赏勤罚惰。

排名前三的队伍有荤菜加餐,而第一名更是有酒喝。连续两天的第一名,韩冈会发下红色的绸带,让这一队中每个民伕系在胳膊上作为褒奖,而且还附带赏钱奉赠。而每天排在最后的三队则会受到训斥,连下饭的菜肴也是最可怜的咸豆豉,如果有那一队连续两天掉在最后一名,就换掉领队,并对全体加以责罚。

夯土的建筑,修造速度本就快得惊人,而精神上和物质上的双重作用下,让工程进度更是加快了数倍。互相竞争的几支队伍,将因战事而耽搁了的七天时间给补了回来。而且从率领这些民伕的领队中,韩冈甚至还发现了几个能力还不错的人物,准备此间事了之后,将之招揽下来。

利用单纯的竞争之心,还有一点微不足道的开销,就让工程进度快了一倍有余,这笔帐怎么算都划得来。韩冈甚至有余暇,派了人将位于大来谷口的前营地改造成一座大型烽堠。有了这座烽火台看门,日后若有外敌要通过大来谷,渭源堡在第一时间就能得到警讯。

王韶视察过工地后,对韩冈定下的规矩赞不绝口、有会于心。而习惯于旧时不用鞭子就驱赶不动民伕的官吏们,看到了工地上的变化,也是更加敬畏韩冈的手腕,再没有人会怀疑韩冈在官场中的前途。

这两桩大事,还有一些琐碎杂务,韩冈做得都是游刃有余,不费半点心力。也就是繁琐了一点,让他忙里忙外,难以歇下脚来。幸好更为麻烦的功劳计点不由他操劳,而由王韶负责。韩冈是亲自领军出战的当事人,如果他来计算功劳,总会有人担心他偏向自己的下属,做不到让所有人满意。为了争一份功劳,好友翻脸、互相揭短的事情时有发生,也只有作为主帅的王韶才能压得住阵脚。

王韶亲掌功劳簿,韩冈也免不了为他的人向王韶说情,不是别人,而是瞎药。

自三月时的托硕大捷,到现在的九月中,不过半年的时间,围绕着河湟之事,王韶已经领军完成了三次会战。而且都是斩首数百的激战。这在秦凤路过往百年的历史上,也算是罕见的战绩。

不过不同于前两次一面倒的大捷。今次一战,虽然斩首超过六百,但官军这边的损伤,如果把瞎药所部的伤亡计入在内,也是达到了六百余。

王韶对自家伤亡并不是很在意,在他看来,古渭寨驻军的缺额随时可以补充。只要有功劳,什么损失都能弥补得过来。

但瞎药可就苦了,就是因为他最后贪功的缘故,将伤亡数字扩大了近倍。且他损失的都是帐下精锐,一二十年内都不一定能补充起来的。而原本韩冈许诺给他一半的星罗结残部,却被禹臧花麻给收编,在攻打星罗结城时,几乎死得干干净净--不论瞎药还是张香儿,都是没能讨到这个便宜。相对于始终坐守渭源的张香儿,损兵折将的瞎药明显要吃亏得多。

‘那就把武胜军送给他好了。’

赏罚不均只会伤了他人报效之心,在韩冈为瞎药一番分说之后,王韶便很慷慨画了块大饼,一张空头支票就这么递到了瞎药的手上:“巡检深明大义,忠于朝廷。力绝西贼之诱,为王事而用命。日后武胜军还得靠巡检这样的忠臣来戍守。”

对于王韶的空口白牙,瞎药无可奈何,只能低头称谢。他现在就像把家当借给一个骗子的蠢货,明知这个骗子一次次来借钱,只是在空手套白狼,能回本的机率渺不可测,但如果不继续跟进,原本所付出的一切就都要打了水漂。瞎药舍不得他前面的付出,都到了这个地步,己经收手不得,现在他就只能盼着王韶能说话算话了。

送了瞎药出去,韩冈回来劝谏王韶:“安抚,不能就这么打发了瞎药,俞龙珂还在那里看着!”

“此事我当然知道。朝廷的抚恤和赏赐都不会少了他一文。”对于瞎药的处理,王韶早有腹案,“我再为他向天子求个赐姓,不信俞龙珂不眼红。”

韩冈对王韶的处理还算满意,只是最后一句话让他有了些疑惑:“以瞎药的身份,应该得不到国姓吧?”

“如果木征或是董毡来投,多半就有机会。”王韶笑道:“还是让枢密院随便给他找个好一点的姓氏。”

等渭源一切处理完毕,都已经是九月中了。新扩建的渭源堡理所当然的比起过去的形制大了许多,而隔着渭河北面的附堡,也比旧有的渭源堡要大上一圈。主堡接近六百步的城寨的规模,而附堡也有三百步,在其间驻扎下数千近万的大军也是绰绰有余。

将两座堡的处置权留给了王君万,又加派了一队人马进驻渭源附堡和大来谷口烽堠。当韩冈跟随王韶回到古渭后,尚未来得及喘口气,才知道郭逵点名要他向秦州通报今次一战的来龙去脉。

还没在公厅中坐稳,就听到了这个消息,韩冈叹了口气,怨声溢于言表:“郭太尉可真是会体恤人啊……”

“小心一点。”王韶提醒韩冈,“别提功劳,老实说话。”

“下官明白。”

不过比起明日才动身的韩冈,先来的却是北方战事的军情通报,王厚拿着一张纸片,走进韩冈和他的官厅:“董毡竟然抄了后路!梁乙埋这下攻打五路的大军全都退了。”他赞了一句,“这董毡可真是帮忙了。”

“对木征来说,我们是最大的敌人,但对于没有切肤之痛的董毡来说,党项人才是他的对手。不趁西夏国中空虚,还有禹臧家主力尽出的时机,从中沾点便宜,反而不正常了。”

见韩冈说话时也不抬头,王厚好奇的问道:“玉昆你在写什么?”

韩冈与王厚交情匪浅,也不瞒他:“是今次一战的经验总结。”

王厚拿起写好的几页纸,再看看下面压着的一摞文字,“这些都是?”他信手翻了翻,立刻皱起眉头,“怎么么一句好话都没有?”

韩冈笑着摇头:“又不是向上请功的奏折,说那么多好听话作甚。这是为了日后不再犯同样的错误,才总结经验教训。事不过三,连续断过两次后路,今次还想依样画葫芦。”他叹了口气,从茶壶中倒了一杯热茶递给王厚,“现在想想,用计还是太险,想得到的越多,风险就越大。如果先到渭源汇合……?”

“那星罗结城肯定保不住!”

“如果在会合了王舜臣之后,再绕道回师渭源呢?”

王厚立刻道:“禹臧花麻可就要跑了。”

“难道现在他就没跑吗?”韩冈笑了一笑,不以为意,又低头写起自己的总结。

两天后,韩冈他出现在秦州州衙之中,正等着郭逵的接见。只是他抬头数了半天的椽子,也不见郭逵出来。以韩冈的身份,以及他所担任职位,还有他今次负担的任务,竟然会被晾在外厅中,由此可见郭逵心头的怒火实在不小。

韩冈对此并不介怀,郭逵的确被瞒着,没有人告诉他真正的来龙去脉。而他则并不介意向郭逵为自己辩解。

