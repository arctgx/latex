\section{第30章 随阳雁飞各西东(12)}

萧禧称病于都亭驿中,一时间放下了身上的任务,不过来自辽国国内的信函还是照常收到。

检查过信函是否伪造,从密语写就的书信中得到了最新的情报。萧禧将信函递给折干,“尚父已经领兵到了南京道,现在应该已经收到了宋国皇帝中风的消息。尚父用谋鬼神莫测,多半会当真出兵。兴灵那边更是已经下令便宜行事。如此一来,宋人迟早要屈服的。折干,接下来要怎么做,想必就不用说了吧?”

萧禧的态度很明确。韩冈的确强硬,但并不代表他背后的朝廷也会强硬到底。一旦战事开启,在朝堂的压迫下,韩冈的态度终究还是会缓和下来。到时候只要给他一个台阶下,当能成功达成预期的目的。

折干接过信看了一阵,然后默默的收起来。他也知道萧禧的想法,但他心中的想法却跟萧禧不一样。这段时间发生了什么事,他已经原原本本的用密语写了回信,让信使带了回去。如今尚父已经到了南京道,想必很快就能收到这封密信——这件事萧禧当然知道,折干身负的使命中就有监察这一条。折干是耶律乙辛斡鲁朵中的提辖,能领兵的大将,比起亲信,当然是在萧禧之上。

可是不论萧禧犯了多少错,只要结果好,就什么关系都没有。萧禧就是要赌一把,还把折干趁势拉下水,要受责将会一体受责。

折干起身:“韩学士应该快到了,下官这就去做准备。”

“你去吧。”萧禧也不留他,“别忘了将国书带着,让韩冈知道尚父的态度。”

在这几天来一贯的时刻,韩冈准时的出现在了都亭驿。

折干将韩冈延至花厅中,寒暄了数句,便将国书拿了出来。

正准备拆开宣读,韩冈却将手一伸:“不必读了,拿过来便是。”

折干神色微变,但还是依言递给了韩冈。

将这份国书直接上殿呈于宋国皇后,现在看起来已经不可能了。那么请韩冈转交,互相通个气,商量一个两全其美的办法,给双方一个台阶下。大宋不缺钱,只要跟过去一样给大辽这边足够的好处,这次的事也没过去了。至于边境上的一点小摩擦,在白花花的银子和亮闪闪的丝绸面前,完全可以忽略不计。

折干觉得只要宋人能给尚父一个面子,就算一年只有个三五万贯也不会嫌少。

但韩冈将国书拿在手中,手指一动,用丝带缠紧的卷轴转了一个圈,却连看也不看的就放在一边。

轻佻的动作让折干勃然变色,韩冈却只是笑:“这里面的文字,几分真,几分假,你我都知道。尚父想要什么,同样是你我都知道。副使既然在尚父斡鲁朵中任官,那么有些话,到可以你我私下里说一说。”他

轻轻一笑,瞥了厅中的几名辽人侍者一眼,“绝不会让副使辜负了尚父的信任。”

折干脸色数变,一番挣扎之后,终究还是挥退了闲杂人等,让厅中只剩韩冈和折干二人。不过韩冈的笑容,也让他警觉起来:“不知内翰是何意?”

“只为两国交好,尚父想要什么,鄙国就会送上什么。只是要换个双方都能接受的方式。”韩冈无意卖关子,吊人胃口,“想必贵国尚父手上应该掌握了不少商队,一年中当也能有不小的收获吧。”

折干能被派来做副使,并不仅仅是因为他外粗内细,且还忠心于耶律乙辛,更是因为他对宋辽之间的关系也深有了解:“贵国不是一直都对我大辽提防万分,甚至连边境上的榷场出入都要搜检,有多少商人被逼走了?难道贵国打算放开榷场?”

“如果只是单纯的放开榷场,恐怕尚父也会心有疑虑吧?”韩冈笑得更为深沉。

在此时,辽人和宋人不是没有商业交往。相反地,商贸往来其实很频繁。但由于士人控制的大宋朝廷对商人有着根深蒂固的不信任,甚至认为贪于财货的商人会为了钱向辽国泄露国中军情,便一直从各方面施加有形无形的压制——自然,这是冠冕堂皇的理由,虽然也有一定道理,但河北大族的私心更是无处不在——两国间明面上的贸易其实一直发展不起来,只有一支支参与回易的商队在边境上奔走往来。

而且另一方面,辽国也同样对大宋提防万分。倒不是军机情报,而是国中的金银等物大量外流。每年的岁币,五十万银绢中的二十万两白银,往往不要半年就会回到宋人的钱袋里。

所以异族的有识之士,总要喊着废汉礼、复蕃礼。内容相近的口号,西夏喊过,辽国也喊过,汉人的制度和上层生活,的确极有吸引力,但官员们的诗酒风流实在是太过奢侈了,奢侈到就是大宋也是勉力支撑,最后不得不变法。而大宋以外,更是没有哪个国家能支撑得了模仿汉人生活的统治阶层。而且这样的生活也会消磨统治阶层的意志,最后变得糜烂不堪。

辽国一直都采用捺钵制度,让皇帝带着整个朝廷游走四方,而不在某座京城中常驻,其实也有畏惧汉人生活毁坏契丹统治根基的想法在。

折干当然知道这一点,所以他更为不解,甚至有一份好奇心,“学士到底是何意,还请明说。”

韩冈将手一张,伸开五指:“一年五十万贯的收益,不知尚父会不会满意?!”

折干身子猛地一震,手上拿着的茶盏,连茶水都泼了出来。他顾不得烫,连忙问:“学士莫不是在戏耍人?!这可是岁币的一半啊!”

韩冈笑了:“岁币也不过一百万贯而已,其实在大宋和大辽,都不算多。”

五十万银绢,包括二十万两银,三十万匹绢。由于银价对钱一向是一两兑两贯半到三贯,而绢则按照质地不同,一匹从一贯到十贯不等,越便宜的数量越多。故而每年朝廷实际上的付出,平均下来大约相当于一百万贯左右。五十万贯的收益,基本上正好是其一半。

“还请学士细细说来。”

“东京城七十二家大行会,任何一名副行首,一年都能至少上万贯的收入。而以贵国尚父的身份和权势,如果用在商事上,一年的收益,也许一时还比不过五十万匹两银绢的岁币,但要是连一半都做不到,那怎么也不可能。而将岁币从五十万匹两增加到七十五万匹两,这个美梦恐怕连贵国尚父本人都没想过吧?”

韩冈说得并不客气,但折干却听得怦然心动。若是从宋人那里得到的好处能让岁币实质上平添了一半——而且还是专门给耶律乙辛一人——那么回到国中之后,什么事都能抵得过了。这就是功劳。

“但榷场不开,如何能做到这么多?鄙国国中可没有……”说到这里,折干猛然一凛,断然道:“马可不成!”

“当然不是马。大宋一年需要军马数万,想必尚父也绝不会答应。”韩冈不再笑了,而变得言辞诚恳,“不过贵国幅员万里,珍宝特产无数,用以交换鄙国的绢绸瓷器,随便挑一样就可以了。就是只卖长白山中木料,一年也能卖上数十万贯啊。”

之前崇政殿上,在韩冈说出‘朝廷什么都不要做,只要能够默认就够了’这话之后,宰辅们都猜到他打算用边界商贸的收益来安抚耶律乙辛——都坐下来好好做生意了,又怎么会整天想着在边界挑事进而敲诈?

但韩冈想要做的不仅仅是扩大边界商贸往来,更是要帮着辽国开发合适的商业项目,有来有往才能让生意继续做大下去,否则就是单方面的吸血。不要指望耶律乙辛会上这个当。任何一个提议,必须是有足够吸引力。

正如之前所说,辽国对扩大贸易同样有着深深提防。耶律乙辛在才智上,绝对不会输给任何一位明君。他肯定会提防诱惑中隐藏的危机。只是如果是用本国国内的特产来交换,那就是另外一回事了。而且他的利益跟辽国的利益是不一样的,他对辽国的统治并不是名正言顺,必须用更多的好处来交换。

韩冈甚至都不怕给耶律乙辛送兵器送甲胄,因为这些武器的第一目标决不是大宋,而是耶律乙辛在辽国国内的敌人——当然,两府是绝对不会答应的就是了。

折干皱眉想了一阵,他很提防韩冈,但他想不通韩冈的话中有什么阴谋,不过折干知道,这件事不该有他做决断,只要将细节报回去就行了:“敢问学士,具体该怎么做?”

“至于细节,千头万绪,我也没有陶朱公的才华,自然会有人会去求见贵国尚父。到时候,只要副使居中搭个话就够了。”韩冈端起茶喝了一口,“鄙国将会遣人以买马的名义去贵国——想必副使也知道鄙国京中赛马有多风行——不过实际上买马是附带,鄙国并不指望能从贵国那里得到大批的战马,只是借个名义而已。”

用工业品交换原材料和土特产,这是后世最为常见的贸易方式。以货易货也好,用银钱中转也好,个中利润只要想像一下,就能看到那闪烁着的金黄色光芒。

在辽国国中做配合,用的是权势,对耶律乙辛来说,也根本不需要任何成本,几句话就够了,但回报的则是真金白银,自然是得利甚多。在大宋朝廷这里,甚至能跟商人按章抽税,同样有好处。

顺丰行从京城其他商会那边收集来的资料证明,宋辽两国每年的贸易规模总量不会超过两百万贯,这还是已经将估算的回易总额给计算了进来。明面上在缘边各大榷场的交易总额,仅仅是用过瓷器等奢侈品,将岁币中每年二十万两的白银给收回来了,抽到的税也不过几万贯而已。

在韩冈看来,眼下的交易规模实在太小,效率也未免太低,这可是两个拥地万里的大国之间的贸易数量,人口更是世界上分列第一第二。一年预计才两百万贯,这比没生意还丢人。

而且辽国还有很多好处没有开发出来。比如毛皮、东珠、高丽参,甚至海东青——让耶律乙辛去压榨女真人去——就是木材,尤其是上等的大料,北方也是极其稀缺的。若是短距离的海运能够成功的话,从长白山上伐木,顺着水放下来,从鸭绿江口运抵青州,通过济水、梁山泺、五丈河这一条线,一路运到京城。

这是一桩互利互惠的贸易。丝绸、棉布、瓷器,大宋这边多得很,而且扩大生产也容易。耶律乙辛那边只需要出原材料就足够了。韩冈是真心实意为耶律乙辛着想,并不担心他不咬钩。

“当然。”韩冈放下茶盏,“我觉得这件事就不需要劳动重病的萧林牙了,让他安心养病。想必副使应该能直接联络上贵国尚父吧?”

韩冈的劝诱,如同魔鬼的耳语,引动着折干的心。萧禧在熙宁八年做得太过分了,大宋朝廷这边没人看得他顺眼,将他抛到一边去,直接跟耶律乙辛联系,也算是一个小小的报复。

折干面露挣扎之色,但很快就恢复了平静,点点头,“折干明白了。”

耶律乙辛的利益不一定是辽国的利益;萧禧得利,并不代表耶律乙辛也一定得利;同样的道理,正使的目标也不一定是副使的目标。

折干只需要抛弃萧禧,就能在耶律乙辛那里得到他想也想不到的好处来。还有什么好犹豫的?

韩冈微微一笑,算是开了个好头。

接下来出动的将是赛马总会中做副会首的商人,背后是宗室、贵戚和京城世家。因为韩冈主导的缘故,雍秦商会也能顺道厕身其间。灵夏、河东那边都有路走,还可以将河东的折家拉进来。

有好处,大家分。
