\section{第26章 当潮立马夜弯弓(下)}

枢密使吕府的前院灯火通明。

还没有就寝的吕公著这时候领着一家老小,聚集在前院中。在他们的面前,是匆匆出宫的蓝元震。

“小人奉圣谕,招吕宫保入宫。”

“宫保?”吕公著顿时皱起眉头,没有接旨。他身上可没有简称宫保的太子太保这个官衔,更不可能被封为又名公保的太保。

“官家已经书诏册封枢密为太子太保。”蓝元震直言相告。这是他在示好,提前一步告知吕公著,终归是一桩人情。

吕公著却疑心重重,“天子的情况如何?”

“大体还好,已经能借韵书传话。”

“眨眼?”

回想起出宫前韩冈验证天子神智的那一幕,吕公著心中疑云更深了一层。韩冈那时候的表演,该不会是为了现在而做的埋伏吧?

“册封宫保的诏书,还有召洛阳的司马宫师回京的诏书,都是官家通过韵书传达出来的。”

吕公著吓了一跳:“司马十二也被召回了!”

“正是如此,而且还被封作太子太师。以司马宫师和枢密为师保,是官家当着太后、皇后和王相公的面,做的决定。”

吕公著花白的双眉皱得更紧了几分。这份任命突然而来,该不会是宫中有人想收买自己吧?只是同时将司马光招入京中,这个路数怎么想也不对,还任命了司马光做太子太师……

吕公著忽然双目一瞪,该不会是天子向太后献了降表。但他又很难相信这个答案,不,应该说,他觉得这根本就不可能。

“王介甫呢?”吕公著问蓝元震,“是不是太子太傅?”

“不,太子六傅今天只定了司马宫保和枢密两位。”蓝元震顿了顿,“至少小人出宫之前还没有。”

“是这样啊……”吕公著手捻着一缕长须,苦苦思索着前因后果。

蓝元震可是耗不起这个时间了,催促着:“宫保,两府宰执可是都被传诏了。”

蓝元震越急,吕公著就觉得越是可疑。他这时候过来,目的不应该就这么简单。

当年的吕正惠【吕端】为了防止在太宗即位中立下汗马功劳的王继恩,在太宗身后继续搅混水,甚至是直接设计将其人锁在了中书门下的内厅里。

不过帝位传承中,刀光剑影都是平常事。吕公著素来胆大过人,又自命君子,纵是皇城成了龙潭虎穴,他也要闯上一闯。

回头看着儿孙和下人们惶恐担忧的眼神,吕公著大喝道:“尔等紧闭门户,各自回房休息。”

说完便骑着马扬鞭而去。

走上夜色下的御街,南面不远处的州桥夜市依然灿如星海,但北面通向宣德门的一段,则是黯淡了许多,唯有宣德门的城楼上灯火辉煌。

不过在吕公著的这一队前后,都有提着灯笼的一队人马。吕公著眯起眼,前面那一队的灯笼上,韩字很是明显,当时东府的参政韩缜。而后面的一队,从过来的方向上看,则是副手章敦。

吕公著无意跟他们交流什么,队伍中还有阉人在,现在多说两句闲话,日后就有可能成为把柄。

而且吕公著还有事情想不通,都被请来了所有的宰执,难道是想当着宰执们的面公布遗诏?或者是内禅大诏?有必要急成这样吗。

怀着心中的疑问,吕公著和韩缜、章敦前后脚进了皇城,再往前一点,甚至看到了蔡确的背影。

夜幕笼罩的皇城,犹如鬼蜮。班直手中以及高处张挂的一串串灯笼,那些许的光芒,只是更加强烈的凸显了皇城的幽暗深邃。

穿过一重重宫门,两府中剩下的几位执政,陆续抵达福宁殿,王中正和张守约就在外殿中。

一名是身任五品观察使的大貂珰,一名则是三衙管军,都有带御器械的兼差,是今夜领兵镇守皇城的主帅。他们都在外殿里镇守,估计是在防着什么了。

吕公著多看了他们一眼,脚步便落到了最后。深呼吸了两下,定了定心神,吕公著走进了福宁殿的内殿寝殿中。

进了内殿,就在御榻之前,已经是被人围了一重又一重。寝殿是皇帝私人之地,永远都是安静整齐的。可现在的寝殿中挤满了人,吕公著的脸色很是不好看。

最外圈是内侍和宫人,里面一点,则是宿直的王珪、薛向,还有早一步进来的章敦、蔡确和韩缜,除了这几位宰执外,又有韩冈、张璪。而紧贴着御榻,是高太后和向皇后及嫔妃,皇子延安郡王赵佣都在殿中。只是在太后的身后,吕公著还看到了雍王赵颢,这让吕公著心中立刻多了一层阴云。

天子接见外臣,后妃和皇子就该在东间待着,还要拉上一层帘来隔绝内外,怎么一点规矩都没有。要是这样都行,那太后还垂什么帘?!

但他也不好发作,人到得这么齐,分明就是要内禅的步骤,也就是说,赵顼对自己的身体完全失去了信心——如果赵顼还有清醒的意识的话。

天子现在连话都说不出来,除了眨眼,没有别的办法来验证。吕公著无论如何都不会不加测试,便相信方才蓝元震的传话。

王珪拿着韵书,向几位刚刚到场的两府执政解释了怎么通过这本书,来与天子交流。章敦听了几句,抬眼望了望韩冈,见韩冈点了点头,给了一个肯定的答复,他便放心了下来。其他几名执政也在跟相熟的人进行交流,初步确定了天子的神智依然存在。

只有吕公著怀着浓浓的疑心,现在的殿上,除了自己,以及不能说话的赵顼,他不相信任何人。

“陛下请恕臣失礼。”吕公著踏前一步靠近了王珪,同时伸出手,近乎用抢的将他手上的韵书强拿了过来。

他翻着韵书,向赵顼发问:“还请陛下告诉微臣,方才的诏谕封臣为何职?”

赵顼没有生气,他现在表现不出生气的模样,他很熟练地在吕公著的配合下眨着眼睛。

上平一东——宫。

上声十九皓——保。

“那臣的差遣呢?”吕公著再问。只是一个最简单的问题,事先或许做了准备,他当然不能放心。

枢。

使。

吕公著稍稍松了一口气,能用简称来报官职,比起太子太保和枢密使更能确定天子的神智——聪明人往往更会偷懒。但他还是不放心,再次发问:“臣父何名?”

夷。

简。

吕公著将韵书还给王珪,退后两步,跪下请罪:“臣老多疑,有罪。”

赵顼眨着眼:无妨。

两府宰执在列,吕公著又代表其他臣子验证了天子的神智,赵顼便立刻点起了宋用臣。

册。太。子。

从赵顼的枕边拿起张璪在吕公著等执政进宫之前刚刚写好的册皇太子文,宋用臣将之展开,当着重臣们的面,大声诵读。

现在并不是内禅,只是要敲定内禅。先确定皇太子,等皇太子的身份确定,然后再由重臣商议禅让事宜。

大宋的百多年,还没有一次内禅的先例。再往前,也找不到几条故事。吕公著有些担心,如此仓促恐会有失国体。就像现在的寝殿里,乱得没有一点规矩。

所以等赵佣在向皇后的指点下,懵懵懂懂的向他的父皇三跪九叩,接过册书;转过来,又接受了在王珪的带领下的一众重臣的参拜。吕公著便跪下道:“陛下玉体违和,臣乞皇太后权同听政,候陛下康复日依旧。”

视线集中在赵顼的脸上,但天子阖起眼皮,没有动静。

“陛下……”

吕公著还想再说什么,但王珪捧起了韵书,抢在他前面问道:“陛下可是另有心意?”

赵顼睁开眼,眨了两下。

‘是内禅?没用的!’赵颢冷眼看着。

他不在乎内禅,就算皇位现在落到侄儿手中,几年后照样能回来。他至少还有十年时间。尤其是韩冈自寻死路,等到他皇兄一死,韩冈最好也只能到岭南待着,到时候,看看这个小儿谁来保!他恶狠狠的盯着赵佣一眼。

韩冈则是微微一笑,虽然在赵顼迫不及待的招来所有宰执……不,在高太后大发雷霆后,他就已经可以确认自己的赌博已经赢定了,但直到现在,他才彻底放下心来。

不过所有人的注意力都在赵顼的脸上,没有人看见韩冈唇角边那抹如释重负的轻松笑意。

下平七阳——皇。

皇太后。还是皇太子。又或是皇后。

天子到底想说什么,王珪也好、吕公著也好,几乎所有人都在猜测着。

下一个字是去声。

太?

但王珪的声音很快越过了‘太’字所在的去声九泰,赵顼只眨了一下眼,给了否定的答案。王珪的声音,一个韵部一个韵部的向后挪动,直至第二十六韵部,皇帝这才眨了两下眼皮,

刹那间满堂哗然,声浪直冲屋上。

赵颢如同五雷轰顶,脸上不剩一丝血色。

哐的一声巨响,是高太后霍然起身,身后的交椅被带到在地。但高太后丝毫不顾,转身便拂袖而去。

韩冈闭起了眼,有些疲累,今年他是不想再赌博了。

去声二十六宥。‘后’字便在这个韵部中。

缀连前一个字‘皇’,那么就是:

“皇后!?”

“皇后权同听政?!”

“岂有此理!”吕公著暴然而起,一声怒喝!

