\section{第24章 南国万里亦诛除(三)}

【又迟了点,抱歉。】

“怎么又走了?!”

当米彧气喘吁吁的赶到码头上,就看见几艘海船在港口号角的送别中,扬帆北去。趁着春时的南风,一艘艘两三千料的巨舟,片刻之后就变成了海天之际的点点帆影。

船上的几位都是他想方设法要拜见的目标,但自从抵达海门之后,无论米彧如何心急如焚,就看见安靖天南的几位将帅,在交州各地来来去去的到处走动。

章惇和韩冈,从海门到升龙府——如今叫河内寨——与交州诸部订立铜柱之盟,又从河内寨,回到海门,不过刚歇下来没有几天,便领军渡海,自海门返回邕州。

米彧递上去的名帖,根本都没有人理会。他本也不指望能得到章、韩两人的接见,但能跟两家的幕僚或是家人打个照面,熟悉一下,日后打通关节也就容易了许多。

他与章惇是福建的乡里,与韩冈的表弟也算是点头之交,去年冬月他还在京城的时候,因为吉贝布一时挤压,也是冯从义出手帮了他一个忙,凭着这个关系,好歹能拉上一点关系。只是米彧没想到,两边都是没加理会,让他连送钱的地方都没有。

站在码头上,米彧连声叹气,捶胸顿足。来来往往的士兵和苦力,都是拿着瞧疯子的眼神看着他。直到两个巡视码头的士兵看着碍眼,上来赶人的时候,米彧在码头上的表演,才告一段落。

垂头丧气的从港中回返城中,米彧盘算良久。这件事还不能算是全然绝望,至少还有一人可以去打个交道。

章惇、韩冈、燕达和李宪全都返回了邕州,听说是接到了圣旨,要将一干交趾逆贼在邕州城外明正典刑,以祭一年多前,在交贼侵攻中丧生十万亡魂。

而大部分的军队,也跟随着他们陆陆续续的返航。所有的部族洞主也都离开了海门,前往他们的新近得到的领地。

主要的将帅中,只有李信还留在海门。作为权发遣广西钤辖,他要暂时镇守南疆。

李信是韩冈的表兄弟,当然也是冯从义的表兄弟。只是米彧听说李信不喜欢与人结交,不怎么好打交道,加上又是武将,地位远不上文官,在商贸一事上并没有多少发言权。米彧并没有想过去结识他。只是现在没得挑选,只能却求见一面了。

自燕达北返,李信便是交州排名最高的武将,但他并不多出军营,也不会去干扰地方政务,只是检查军中,教训士卒,顺便习练武艺。闲暇时便听从韩冈的吩咐,读些兵法、地理和医药方面的书籍,顺便用着没有什么文采的白话,写写这一战的心得体会。

只要是白天,从海门县城南的军营前经过,都可以看到在营地的校场上,李钤辖正尽心尽力的训练着麾下的士卒。几十人、几百人在校场上,高声喊着号子,依从上官的命令,不断变换着队列、阵法。也有一队队士兵,拿着标枪,向着三四十步外的靶子用力投过去——交州弓弩难用,标枪就是最好的远程兵器。

尽管李信麾下的一千多名广西枪杖手,都是招募组建不过一年的新兵。但他们毕竟是参加了几次大战,并不能算是弱兵,放在两广的军中,从装备、到士气、再到经历,也算是排得上号的精锐了。如果训练得宜,至少十几年之内,这一支军队都能保证水准以上的战斗力。至于再往后,那就不能指望了,毕竟眼下是河北军都在和平中变得稀烂的时代。

李信并不想在广西安身太久,否则时日一长,想回北方就难了。他还是喜欢北方的水土,在南方待的时间虽然长了,但始终难以习惯潮湿多雨的气候。

不过话说回来,李信即便想在广西多待两年,也不是那么容易的事。他在征讨交趾的战事中立功甚多,一直都是作为先锋将冲杀在最前。立下的功劳让李信很难在广西继续流下去——这是他的表弟韩冈亲口所说。

平交一战下来,李信的本官多半能在四十阶的诸司使、使副的漫长道路上,多攀上几级台阶,另外再加上一个遥郡的团练使或是观察使。这在过去,基本上是在军中二三十年的宿将才有的阶级,李信几次大战下来,就全都得到了。

就在七八年前,河湟开边刚刚开始的时候,他和表弟韩冈共同的恩主张守约,也不过是一个从七品的供备库副使,是诸司使、使副中的最低一阶,远不如李信现在的文思副使,更没有遥郡的加衔。只是这几年因为累累功绩,加上宿将的威名,一下就升到军中最高位的三衙管军的位置上。

眼下李信靠着累累战功,本官已经不低,又已经是权发遣广西钤辖了,如果还留在广西,总不能给他一个兵马副总管来做——燕达做到权发遣秦凤兵马副总管的时候,都快四十了,而且还是因为他出身京营的缘故,而李信只比韩冈大了几岁,才三十出头——可若是还做钤辖,从哪里调来将官,有资格压在他的头上?

过些日子,他肯定是要入京,或是转去北方诸路——从地位上,北方缘边诸路的武官,要在南方同阶武官之上,官位也更高。李信过去担任荆南都监,入京参加朝会觐见天子时,在他前面的都是北方的都监。

只是做一天和尚撞一天钟,李信只要还在广西任上,对他的工作就分毫也不打折扣。一千多士兵,一个个被操练的鬼哭狼嚎。要不是他的威望高、名气大、武艺高强,功绩也是让人仰慕,加上都是新兵,没有染上那些兵癞子的恶习,说不定兵变都有可能。

训练了一个上午,李信便一挥手,放了下面的士卒回家去。

每一名士卒,即便是没有家眷的光棍,家中现在都有人帮着洗衣做饭,当然,还有陪夜消遣。李信一说散,急着回家的卒伍们一待李信离开,便做卷堂大散。经过了几个月的战事,区区一个上午的训练,还不至于让他们变得有气无力,做不了想做的事。

不仅仅是下面的小兵有的享受,将校们则依照地位高低,有多有少的得到了一批交趾女婢。官位越高,能挑选得就越早,自然选在身边的一个比一个出色。

李信回到府中的时候,一名青春可人的女侍立刻奉了茶汤上来,又有两名同样颜色出众的女侍帮着脱鞋。将身上的甲胄、兵器卸下,又一名使女进来,说洗澡的热汤已经烧好了,请李信过去。

比起笨手笨脚的亲兵,婢女们的服侍当然要远远过之。李信如今身边的四名婢女,全都是交趾官宦人家出身,虽然算不上是什么绝色,可拿到国中,也算得上是上品了。

洗过澡、更了衣,在简朴的小书房中,李信在桌子上翻到一张名帖。

“米彧?”李信不记得自己有听说过这个名字。

看看题头,只知道是个福建人,是个没有官身的布衣。不过名帖上面竟然说与表弟冯从义有旧,又从京城来,多半是个商人了。再看看附在名帖后的礼单,算不上多贵重,但也不能说是微薄了,也只有商人才会如此。换作是穷措大来拜访,多半就是几首半通不通的诗词。

商人往往富庶过人,民间也早没了对他们的歧视,许多文官武将自己家里就做着买卖。但商人明面上的地位依然不高,四民之中排在末尾,且漂泊江湖之上,不受地域管辖,将一桩桩民生急需的商货低买高卖,从百姓们头上博取利润,总是让许多人看不过眼,正经的官员都不会接见一名商人,而是会让亲信家人去与他说话,居中传递口信。

不过李信便没有那么多想法了。

“让他进来吧。”李信将名帖放起来,吩咐了亲兵一声。最小的表弟,已经有数年不见,只能通过鸿雁传书,怪是想念的。

很快,守在门房中的米彧便被带了进来,行过礼,李信请了他坐下。

看着米彧小心谨慎的斜签着在下首的交椅上坐下,虚虚的只占了半个屁股。李信便让人奉上了茶,问道:“不知兄台从京中来,可是带了我家表弟的信函?”

“小人乃是来往广州和京城的布商,与冯行首素来交好,时常一同痛饮。每每听着他私下里提起韩龙图和李将军。”米彧笑了一笑,“不过小人这一次本没打算来交州。只是在广州听说官军大捷,交贼自食其果,便飞奔而来。”

“哦,原来如此。”李信有些失望,原来并不是带着表弟的书信来。想想,就问道:“兄台最后一次见我那表弟是什么时候?”

“就是在去岁冬月的时候。小人上京,就见到了冯行首。当时冯行首因为向重病的太皇太后进献了西域的珍药,被天子加官一级。不过后来冯行首回头则说,是仗了韩学士和李将军的战功才沾了光。”

