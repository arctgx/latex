\section{第34章 山云迢递若有闻(二)}

‘不要理他。’——韩冈似是信口而言的一句吩咐,使得蔡曚在通远军的地位顿时微妙起来。

蔡曚本人一开始倒是没觉得有什么变化,但他很快就发现,下面的胥吏如今都是当面点头哈腰的听话受教,但转过脸来,就把他的吩咐全都丢在脑后。要不然干脆就是叫苦,就像踢皮毬一样,有志一同的把事情往别人身上推。

就像他让人去架阁库中搬运旧档,那名小吏立刻就回道:“这事不干小人的管,小人也进不去。运判还是找管架阁的那位……要不,小人帮运判找他来?”

唤来管理架阁库的胥吏。五十多岁的老家伙立刻变成了磕头虫,

“没有知军下令,小人不能开门。律条皆在,小人岂敢依违?还望运判体恤小人的苦……”

胥吏砰砰的磕头,声音虽响,却连脑门都不红。

几乎所有的事都是如此,而最让蔡曚愤恨的,就是到了开饭的时候,厨房中的厨子,都推说病了,没称病的做出来的饭菜,蔡曚吃了一口就吐掉了——什么时候盐也能当主菜了!?

外面也有给食吏员的大灶,可蔡曚挂不下脸去吃。只看着对面的韩冈,毫不介意的吃着专供吏员的粗粝饭菜,一边还在批阅着公文。

粗鄙不文!不知礼法!灌园小儿!沐猴而冠!小人得志!

蔡曚的辘辘饥肠,化作了满肚子的愤恨,就是要发作起来。

只是一天之间,蔡曚就用亲身体会明白了什么叫做孤家寡人。

韩冈低头吃饭,但对面蔡曚燃烧在眼中的熊熊怒火他还是能感受到得到。但韩冈毫不介意,这是蔡曚自找的。

差遣是天子授予的,但手上的权力多寡是靠自己争来的。退上一步,对手就会进上一步。韩冈前面稍事退让,蔡曚便得寸进尺。见到蔡曚当真没有合作之意,他便选择了直截了当的翻脸。

只是他一开始,也仅仅是把蔡曚丢下不理而已。但蔡曚却闹着要翻旧档,这件事,明明白白要抄韩冈甚至整个缘边安抚司的老底、寻找罪证用以构陷,不论是真是假,这已经足以韩冈选择了最激烈的对抗。

看着安安分分吃饭的的敌人,蔡曚终究还是忍耐不住,一拍桌子,指名道姓的叫道:“韩冈!”

士人的大名不是让人随便叫的,蔡曚的举动实是无礼之极。韩冈却也不怒,他悠悠闲闲的放下筷子,咽下嘴里的饭菜,喝了口茶权当漱口,才问道:“不知运判有何指教?”

“指教?哪敢对韩官人有所指教?”蔡曚咬着牙冷笑着,“韩官人好大威风,一句话就能让人奔走听命。现在通远军中倒真是只知有你韩冈,却不知王法何在?!”

“若论谨遵王法,运判当不如韩冈。”韩冈口气更冷,“不知在运判心中,天子之命不知比不比得上文相公的命令?”

蔡曚脸色骤变,身子一动,几乎要跳起来,“……胡说八道!血口喷人!”

韩冈叹了口气,又拿起筷子,转头盯着手上的文案,“那就当是韩冈胡言乱语好了,运判不必放在心上。”

恐怕蔡曚千算万算,也想不到蔡延庆对缘边安抚司的支持到了这个地步。不过这也不难想象,韩冈、蔡曚虽然是随军转运使,但如果河湟功成,真正领走应办军需首功的,只会是蔡延庆这位秦凤转运使——虽然都有个转运使的名号,但随军转运使和路分转运使,地位相差不啻千百倍。

虽然是过继,但也曾经做过宰相蔡齐的儿子。只是因为蔡齐有了遗腹子蔡延嗣,为避嫌疑,才解除了父子关系——为争夺遗产,兄长害死年幼的弟弟,此时并不鲜见——蔡延庆把所有的财物留给堂弟,白身离家,此事的确做得洒脱。可若论起人之常情,韩冈不信蔡延庆心中没有芥蒂。若是有了能成为一任宰执的机会,他可能会放弃吗?

这是韩冈为蔡延庆的行为想到的解释,也算是马后炮了。

蔡曚的脸色千变万化,到最后,却是定格在凶厉之上:“韩冈!你区区一个选人,却恃功自傲,蛊惑人心,悖逆无法,要挟上官。你且等本官弹劾便是!”

如果蔡曚的这番言辞,是一个文官用以弹劾武将,那这位武将就会很危险了。可两个文官相争,这点指责又算得了什么?官员指斥,有比这更阴狠的。御史弹劾,有比这更激烈的。而且,当他韩冈不会上书反驳吗?

蔡曚若真的弹劾上去。有人会信吗?也许。但堂堂朝官压不下一个选人,丢脸的会是蔡曚。

“若运判能秉公心,弃私情。韩冈即便受运判弹劾,亦是俯首甘受。”韩冈更是不在意,闲闲的回了一句。

当年的陈举,在成纪县中一手遮天,让几任知县、主簿狼狈而退,现今韩冈在通远军的地位,可比当年的陈举强得太多。外来的蔡曚又能奈他何?

韩冈现在是无暇旁顾,不然凭他在通远军一呼百应的威望,设个局让蔡曚钻进去,栽他一个罪名也是轻而易举。他忙得厉害,无心于多周旋,试探出了蔡曚的倾向,验证了蔡延庆的传话,就直截了当的选择了这个粗暴的手法。

韩冈一开始的退让,现在的强硬,本质都是一个,绝不允许有人在前线开展的情况下,在后方搅风搅雨。韩冈不知蔡曚是怎么被文彦博安排进秦凤转运司的,但他的行为明显会对眼下的战局产生不利的影响。

韩冈要让蔡曚明白还是老老实实的比较好,要想坏事,就要做好被架空的准备。你的地位比我高又如何?没有人听命,就是一个光杆司令。下属架空上官的例子太多了,韩冈即便真的做起来,一点也不显眼——何况,蔡曚还不是自己的上司,朝廷颁下的诏书中,韩冈的名字是在蔡曚之前。排座次的工作,就算是梁山好汉都要费一番心里,何况官场。朝廷的公文,褒贬取决于一字之间,序列的问题就更是官场上的重中之重。

只是韩冈在吃饭时,眉头还是在微微皱着。

蔡曚好歹还是随军转运使,跟韩冈同掌一事,地位关键无比。韩冈把他一时架空很容易,但真正要处置蔡曚,要解决他在工作上的干扰,却是件很麻烦的事,问题一点也不小。

蔡延庆不会出头对付蔡曚,能得他的提醒已经是承了大人情了。而王韶那边,韩冈已经传信过去了。让他和高遵裕要做好准备,赶紧选派得力人手。

渭源、陇西两座兵站,必须要有能力出众、且地位适当的人选掌管,否则必然生乱。照常例,两位随军转运使正是为此而备,但现如今,却成了让人头痛的问题。如果韩冈去渭源,那么陇西怎么办。若是留在陇西,渭源又该如何?韩冈不论在哪边,就等于把另一处,留给蔡曚。除非王韶或是高遵裕有人能坐镇后方——这也是韩冈把事情推给王韶的缘故——蔡曚的事情得尽快解决,否则日后的乱子,那就根本没法收拾了。不论韩冈还是王韶,都承受不起这样的损失。

时间就在韩冈的急切等待中飞速的过去,就像一队队运去渭源堡的粮草,都不会再回来。

蔡曚也从刚开始的愤怒,而变得阴冷起来,他也看出了韩冈的窘境。除非韩冈能一直压着他蔡曚,否则只要离开半步,自己就能随性而为了。到时候,要翻出王韶和韩冈的错来,那就在容易也不过。

就在率领前军的苗授和王舜臣出发后的第六天,前方捷报传回。几匹快马在傍晚冲入了陇西县城,一路高声报捷,带起了一片欢呼。

官军此刻已经突破了大来谷,瞎吴叱在大来谷西面出口设立的寨堡,苗授率领的前锋只用了半日的时间,便一举攻克。王舜臣站在城寨下,身披重甲,单人孤箭,便把一面墙的守军射得抬不起头来,护着苗履率部冲上了城头。

捷报让韩冈欣喜不已,但接下来的情况又让他发愁起来。照计划,下面就是全军突入武胜军,而韩冈要去渭源主持实务,不仅仅保证前线的粮秣供给,同时还要主持修筑大来谷口的寨堡。

第二天,从前线赶回的王厚,解决韩冈的问题。王韶让他带来的话却是让韩冈放下心,直接照计划去渭源堡主持转运等事。

“那蔡曚怎么办?!”韩冈惊问着。

“放心好了,”王厚笑意冷狠,“家严说了,莫当他的刀子不能杀人!”

王厚冷漠的音调中越发的显得杀气腾腾,“如今的机会是家严等了十几年,辛苦了多少个日日夜夜才等到的,如果有人敢于居中干扰,坏了大事,也别怪家严手下无情!”

韩冈全然想不到王韶手段比自己还要激烈百倍,就算不能真的杀了他,可一番重责后,蔡曚就别想在秦凤待了。这也算是个解决的方法,虽然免不了会有一个跋扈的指责,但只要今次能得胜而归,一切阴翳都将烟消云散,魑魅魍魉又岂有在阳光下生存的机会!?

