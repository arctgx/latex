\section{第15章 自是功成藏剑履(12)}

忽忽已是春日。

汾水潺潺,河岸边花开正艳,堤岸上的草木绿意盎然。天空上声声鸟鸣,从南方归来的大雁,排着整齐的队列,掠过太原城继续北上。桃花汛后的汾水边,有着浓得化不开的春色。

韩缜领衔的划界使团,在边界线上的帐篷中,与辽人唇枪舌剑的争辩了几个月,终于可以不用再闻他们身上的马粪味了。一行使节圆满的完成了他们的任务,从胜州南下,经过太原城回返东京。

谈判并不是一帆风顺,辽人从来都不是会好好说话的对象。一边谈,一边打,也是避免不了的。整个冬天,河东北方边境上冲突不断,不仅仅是新得的胜州有战火,就是丰州、府州,乃至雁门关,都有两军探马和巡卒在边界上大打出手的记录,使得谈判席上火花四溅。

上个月局势最紧张的时候,连韩冈都带了三千兵马,动身到麟州坐镇。虽然帅旗插在麟州城头,也有震慑辽人的意思,可更多的还是为了防止辽人发疯,以防万一。

幸而终究还是没有打起来。顶替萧十三镇守西京道的辽人新任主帅,也不敢贸然去正面挑战有韩冈和数万骄兵悍将镇守的河东。

韩缜与辽人谈到最后,也就是争个山头,争条土垄。由此定下来的国境线,基本上就是按照双方的实际控制线来划分。

新界东面一段,也就是胜州这里,实际控制线便是国界,一番大战没能改变,在谈判桌上也别想夺占上几分。而中段则是以瀚海和大漠为界,沙漠中的几个绿洲,依照两家之前的归属不加改变,比如曾经让李继迁躲避其中,rì后得意东山再起的地斤泽,如今也留在大宋手中——由于大漠瀚海大家都没兴趣,中段国界的划定是最早完成的。

至于新界西段,由于太靠近兴灵,尤其是直插兴灵腰肋要害的青铜峡,乃是辽人所必争。最后经过一番争吵之后定下来的约定,青铜峡依然归属于宋人,但自青铜峡峡口向南五十里内,宋人不得修筑任何城寨。

依据澶渊之盟,大宋如果要修补河北、河东边界上的城垣,必须要通知辽国,同时边寨也不得增筑。胜州的军寨也是依据这一条款,在划界条约签订后,便不能再随意增筑。青铜峡不得修筑城寨的条约,则更为苛刻。这就使得官军只能驻守在西南方六十里,黄河河谷要宽阔得多的鸣沙城。那里地势远比不上险要的青铜峡,必须屯驻两倍乃至三倍以上的兵马才能够保证抵挡住辽军的侵攻。同时这个约定,也让迁移到青铜峡南河谷的数万西夏残部,无法安定下来。

不过只要鸣沙城在手,黄河河谷南方的泾原、秦凤,西方的熙河、甘凉,这四路依然不用担心辽人兵马的威胁。这等于是可以将之前秦凤、泾原、熙河三路的守御力量集中在鸣沙城一点上,说起来却也不能算太差了。

这样一份虽不占便宜,却也不吃亏的条约,澶渊之盟以来,这还是第一次。按从东京城中传回来的消息,天子和两府对这份划界条约十分满意

参加了划界使团的官员们为此兴奋莫名,趾高气昂的挺胸叠肚。仿佛大声了一场大战。

可是话说回来,如果将兴灵的损失一并算进来的话,那就不是不吃亏,而是吃了大亏。

“边界是打出来的。没有大刀长枪压阵,笔管子争不来一寸地皮。”折可适低声冷嘲,“就是拥有苏张之才,没有六国和秦国在背后,凭什么能舌辩天下?”

黄裳一听就慌了神,忙看看左右,见没人注意他们这边,才稍稍放心了一点。然后才低声责怪道:“遵正,这话可不能乱说!也不看看场合!”

“这话可是龙图前两天说的。”折可适冲前面努努嘴。

黄裳将视线向前投过去,二十步外,韩冈正与韩缜把臂偕行。两人言笑不拘,看起来关系好的像是兄弟。

“你看龙图现在的样子,会在韩六内翰面前说这番话?”黄裳驳道。

折可适笑道:“小弟也没有在外人面前乱说啊。”

“焉知周围有没有耳聪目明的?”黄裳微怒:“今天还是内翰,等回了京,可就是执政了,若给他知道,你可讨得了好?”

“……执政……”折可适摇头又冷然一笑:“可叹龙图,惶惶之功,竟然还不能在西府中占下一席之地。”

黄裳又急又怒,两眼左转右转的瞟着周围,“少说两句!”

“是。”折可适点头,终于不再说酸话了,转而道:“元厚之出外,韩玉汝降麻,东府中换了新人。不过再过几日,说不定东府又要大变动了。”

黄裳算是放下心来,“谁知道呢,也许吕吉甫这一次还能撑过去。”

折可适翻翻白眼:“手实法乃众矢之的,眼下河清海晏,已经不是去年的时势。要不要继续推行手实法,全得看天子的心意。而且御史台冬天弹劾龙图不果,已经是丢人现眼,这一次,可是咬得狠了,再也不会放的。”

韩缜回去后,基本上确定要入东府了。韩冈在河东都收到消息,他要顶替月前出外的元绛留下来的位置。元绛这名老臣已经年过七旬,此次外任之后,大概就要致仕了。大宋的宰辅,少有在两府任上致仕的,韩琦、富弼都是从宰相的位置上出外,任了两年地方官后告老返乡。

政事堂中,眼下有王珪、吕惠卿和蔡确,再添一个韩缜,就是一相三参。不过吕惠卿最近又有了些麻烦,成了御史台嘴下的新猎物。

手实法一直都是吕惠卿被攻击的要点,之前为了军费,赵顼压下了所有的反对声。不过现在边界新约签署,眼见着和平降临,他会不会被过河拆桥,还真是得两说。

但这话题也不方便再说,黄裳抿起了嘴,抬眼看着前面,不肯接口了。

韩缜与韩冈并肩而行,踏着河畔青青地绿意,边走边说:“这一次划界之议,也多亏了玉昆。没有玉昆的辛苦,”

韩冈摇摇头,自谦道:“韩冈也只是敲敲锣鼓而已,哪里敢在玉汝兄面前说自己辛苦?打压下辽人气焰的可是玉汝兄。”

“真正让辽人哑口无言,争无可争的,还是靠了玉昆。胜州北界的寨堡一完工,辽人就是争无可争了,”

亲自出城给韩缜一行送行,韩冈这名河东帅臣,可是给足了韩缜的面子。而且之前韩缜能在谈判桌上挺起腰板,韩冈在后的助力功不可没。纵然韩冈的年纪惹人忌惮,韩缜也不能免俗,但在情在理也得在韩冈面前留一份人情。

“玉汝兄说反了。”韩冈又笑着将谀辞转送回去,“就是知道玉汝兄一定能成功划界,韩冈才急着修建城寨。否则划界一签,就得跟河北一样,不能再随意修城了。”

一个冬天过来,胜州边境上的城寨全数完工,预定中的防御体系已经成型,并且与府州丰州的北方防线连在了一起。rì后辽人若想南侵,黄河以西的几个军州,互相支援将会十分方便。

不过在持续几个月的繁重劳作中,累死病死的黑山党项超过五百,而受伤以至轻重残疾的,有千人之多。这样的仇恨,肯定是难以抹除,不知要延续多少年。

但韩冈并不是很在意,只要官军有足够的实力,能镇压一切反叛,就算他们的恨得咬牙切齿,还不是得老老实实的听话受命。若是官军没那份实力,即便是眼下老实恭顺的蕃部,照样会起异心。

而且他之所以那么心急,也是因为有澶渊之盟在前,知道划界条约一旦签署之后,再想修筑新的城寨就难了。而眼下各个关键位置上的城寨营垒都已经建立起来,尽管还有许多地方需要补完,但那完全可以放在rì后慢慢来。

离城越来越远,官道两边的酒肆店铺也渐渐的少了,而田垄则远远近近的多了起来。太原城周围水土好,大半种得是麦子,到了三月中,开始拔节的小麦已经有了两尺多高,将狭长的绿sè叶片高高挑起。

二月以来,颇下了几场chūn雨,太原城外的田野上是一幅幅的浓绿。田间地头的麦苗长势喜人,郁郁葱葱。一阵风吹来,麦田中起起伏伏,如同水面兴波。

韩冈看着满目的绿sè麦浪,心中欣喜。他出城给韩缜送行,其实也有顺道视察了本地青苗的想法。

韩缜见韩冈望着道边的麦田,便说道:“看这样子,河东今岁当又是一个丰年。”

韩冈收回视线,笑对韩缜:“说起来还是元丰这个年号起得好,应了天时。”

韩缜抬头望着纤云不见的天空:“元丰之号为天子亲自所起,天子受命于天,能有所映证也在情理之中。”

“加上今年,已经连着三个丰年了……对比起熙宁的后几年,还真是差别大了。”韩冈道,“元丰的三年来,也就去年chūn天河北陕西有些旱情,不过也没持续多久。”

“灾年米贵伤农,丰年米贱亦伤农。玉昆当小心为是。”

“多谢玉汝兄的提点。”

韩缜一笑:“不过有玉昆在倒是不用担心了。”

“当不得玉汝兄的夸赞,纵能平抑粮价,也是占了去岁大战的光。他处不知,为了弥补去年一场大战的亏空,接下来的两三年,常平仓都得要敞开收粮。粮价一时间的确是跌不下来。”

一路走来,已经送到了城南十里。道边的十里亭中已经摆下了饯行宴。

韩冈和韩缜携手进了亭中。席面上的菜肴虽不能算是丰盛,但也都是名厨jīng心制作,用着食盒携来。

韩冈举起已经斟满的酒杯,朗声道:“宋辽两国,在澶渊之盟后,就是七十余年未有大战。不过中国的太平时rì,也就是澶渊之盟后的三十年。等到元昊起兵,西北再无宁rì,几十年烽烟不息,河东、陕西的军民,殁于王事者不知凡几。如今终于灭了西夏,玉汝兄又与辽人签了新约,这太平的rì子,却终于又来了。别的先放一放,且先祝天子千万岁寿,天下太平。”

“玉昆说得正是。”韩缜点头,同举杯,对众人道:“当满饮此杯,共祝天子万寿,天下太平!”

