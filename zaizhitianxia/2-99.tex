\section{第22章 瞒天过海暗遣兵(四)}

一支军队正穿行在横山的峰谷之间。

站在队伍中段,向前望不到头,向后望不到尾。浩浩荡荡,人马数以万计。前军已踏入横山南麓的谷地,而后军犹在云山深处。

西夏国相梁乙埋便身在这支队伍之中。骑着一匹河西骏马,头戴饰着金花的毡帽,套了一身紫花窄袖的圆领长袍,一条金带系在腰间,虽然是汉人,但完全是党项贵人的装束。

梁乙埋是当今西夏太后的弟弟,也即是西夏国主秉常的舅舅。尽管他刻意留起了胡子,但依然遮不住他的年纪。他的姐夫毅宗谅祚,作为景宗元昊的幼子登基时,刚满周岁。做了二十年的兀卒【注1】,因在亲征大顺城的过程中中了一箭,三年前因箭疮不治而驾崩,那时也才不过二十一岁。

虽然梁氏比谅祚年长,但也只大了几岁,今年满打满算也不过三十,而梁乙埋更是只有二十九。这对年轻得过分的姐弟如今掌握着西夏国政,梁氏以太后临朝,而梁乙埋则做着国相。他们的汉人身份,是他们能坐稳两个位置的主因。

换作是其他党项大族就决没有这般好运。野利家、没藏家,这两个分别出过两任皇后的党项豪门,就是因为太过强盛,被元昊和谅祚前后铲除。而梁氏因为汉人的身份,没人会担心他们能谋国篡位,在谅祚死后,反倒因此得到了宗室们的支持,加上豪门各自牵制,也默认了他们的地位。

不过梁氏姐弟并不是就此可以高枕无忧,如果不能满足那些欲壑难填的豪族,梁氏姐弟就坐不稳江山。

西夏国的国计只有一半能靠着自产。剩下的缺额,大部分要依靠宋人的岁赐补足,每年大约二十万贯上下的银绢,对西夏来说是个不容有失的收入。但岁赐往往都要分赐给臣下,并不足以填补亏空,剩余一部分就是要靠劫掠。故而西夏免不了要年年用兵,等财物抢到手,再上书东京求和,照样拿着岁赐。

但自从东朝新君即位之后,这一套招数就越来越难了。梁乙埋叹了口气,脚下虎狼群伺,即便是身居高位,也一样睡不安稳。而面对的敌人越来越强硬,这两年已经陆陆续续吃好几次败仗,尤其是绥德城一役,耗费巨资建立起来的八座寨堡,竟然在一日之间被全数踏破,让他在朝中没少被人冷嘲热讽。

今次梁乙埋领军南下,也是被逼着打起先发制人的主意。原本与他作对手的郭逵被替代陕西宣抚韩绛替代,领军的又是惯来爱冒险的种谔,东人在横山的动作越来越大,这手已经卡到大白高国的脖子上了。再不有所反应,横山难保,银夏怕是也要丢了。

“绥德……”梁乙埋低声念着自己折戟沉沙的地方,宋人有了这座无定河畔的城池,就等于在横山有一个稳固的据点。不但鄜延路的防线大幅向北延伸,同时也震慑了周边的蕃部。据梁乙埋所知,横山南麓已经有越来越多的蕃部与宋人暗通款曲。

横山不容有失,丢了横山,银夏也保不住。没有了银夏,这大夏国的国号还如何能维持下去?所以梁乙埋打定主意,要绥德以北的无定河畔筑城。当初所筑八堡就贴着绥德城,故而被一日攻克。今次再筑城,他便打算离绥德城要远一点。而在绥德城北六十里,有一个适宜筑城的好去处——罗兀。

尽管从南方回来的细作说,宋人也准备在罗兀筑城,但相对于绥德,一下向北跃进六十里的筑城计划实在太过荒谬,宋人过去从来没有这么筑城的先例,梁乙埋觉得韩绛和种谔应该没有疯。

不过罗兀的确是兵家要地,位于唐时抚宁古县之北,一个唤作滴水崖的地方。崖石险峭,高出地面十数丈,原本就有个小寨,作为烽堠之用。梁乙埋去年在绥德建堡的时候,也考虑过此处。不过因为担心他从绥德城下退缩六十里,会惹来国中的议论,便打消了这个念头。谁能想到,最后事情兜兜转转,城寨的位置终究还是定在了罗兀。

只是要想在罗兀筑城,不是那么容易的事。惯用的声东击西是少不了的,不牵制住其他几路的宋军,得到支持的鄜延路,肯定会派出路中主力来破坏筑城的计划。而梁乙埋尽起国中大军,便是要为罗兀城保驾护航。

东朝的关西缘边四路,西侧两路的不易攻打。秦凤有郭逵坐镇,泾原有蔡挺主持。尽管梁乙埋今次领军对外号称三十万,实际也动用了十一万大军,但他决不想去啃硬骨头。秦凤、泾原他都会派偏师牵制,而主力还是放在环庆和鄜延交界处的大顺城上。

梁乙埋曾经在东朝时臣面前自称过国中控弦五十万,但实际上随时可以动用的兵力只有十五六万。所谓的五十万,是把国中从十六到六十的男丁都算上的数字,动员上一次,国力没个一两年都无法恢复。眼下的十一万大军,已是西夏国中大半兵力,即便是兴灵要地,也就只剩三五万兵在防守着。

压在梁乙埋肩膀上的担子沉重得让他都难以支撑,一旦失败,就是万劫不复。在分出了筑城军和几支偏师后,被他带着南下攻打大顺城的,仍然超过了六万。而护翼在他身侧的也是国中最为精锐的环卫铁骑。

兴庆府中,卫翼天子的精锐护卫,分为六班直和铁骑两个部分。

宿卫宫掖的六班直成员,泰半是国中各豪族中擅长弓马的贵胄子弟,既有加强国主与豪族联系的用意,也有作为人质的成分在。总数五千人,除非天子亲征,否则绝不出动。

而环卫天子出行的铁骑,则是从各大监军司的铁鹞子中精挑细选出来。总数三千,分为十部,相当于宋人的十个指挥,在骑兵中最为精锐。跟随元昊南征北讨,战功卓著。今次梁乙埋引兵南侵,他的姐姐让他带出来了五部一千五百骑。

在山道上转过一道弯,出现在前方依然是重重山峦。眼看着盘山道蜿蜒至山谷中,长长的人龙让梁乙埋有些心浮气躁,“罔萌讹,离白豹城还有多远?”

“回相公,还有六十里。”在梁乙埋身后半个马身,一名党项贵族立刻讨好的回答道。

“六十里……”梁乙埋抬头看了看天色,才交午时。到入夜前,应该能赶到白豹城,“不知大顺城那里怎么样了?”

罔萌讹说道:“有哆腊枢密主持,相公当可放心。”

梁乙埋所在的这支队伍,属于出战的中军。而八千铁鹞子,已经作为前锋在昨日就抵达白豹城,今天应该开始分批突破大顺城防线,到其后方烧杀抢掠。

而在东南方向,也同样有一支万人队,赶往金汤城。金汤、白豹都在大顺城的不远处,如同一个钳子,紧紧钳制住宋人的大顺城防线。今天梁乙埋抵达白豹城,明天便能继续南下。有梁乙埋主持,金汤和白豹两城同时出兵,兵锋直指大顺城。等他将鄜延和环庆两路的宋军都吸引过来,罗兀那里就能安然的开始修筑。

山风忽起,夹着灰土劈头盖脸的刮来,迷住了人马的眼睛,也吹得面面军旗猎猎作响。

梁乙埋在山风中,感到了一丝寒意。尽管九月未至,但横山深处已是秋凉。罔萌讹见状连忙递上了一件披风。披风带着翻毛,后面还有坠饰,梁乙埋对这种党项制式的服饰并不喜欢。他每次见到宋人的使臣峨冠博带的装束,满眼都是羡慕。

但梁乙埋很清楚,就算再喜欢汉家的服章礼仪,也不能在外面表现出来。虽然毅宗谅祚早前已经下旨在朝中推行汉家礼仪,但当梁氏姐弟开始主持国政,却立刻又废去汉仪,改用蕃礼——因为他们是汉人。

在西夏国中,一直有都汉化和蕃化两种对立的声音存在。加深汉化,只会削弱党项人的战斗力,就像景宗皇帝【李元昊】早年所说,用牛羊交换无用的丝绸瓷器,徒损国力。但汉人的文明远远超过党项,生活、服饰和娱乐,让每一个党项贵胄都羡慕不已。就算景宗当年一力推行蕃礼蕃仪,但私底下他自己都有穿着汉人的服饰,而毅宗更是对汉物钦慕不已。有两位天子做榜样,下面的贵族无不对宋人的服饰、器物趋之若鹜。

但梁乙埋以汉人统掌朝政,却不能学着去做。元昊、谅祚穿了再多汉人的衣服,也脱不了党项人的内在。但梁氏姐弟的汉人身份,却会让他们必须旗帜鲜明的站在党项一边,如此才不会当作异类。

这还真是累人,梁乙埋想着,但若是能稳固自己的地位,就算茹毛饮血,他也不会在乎。不过当务之急,是打赢眼下的这场战争,鄜延、环庆、泾原、秦凤,甚至河东,他都已经安排妥当。只是突然间,他又脸色一变,想到了自己一个疏忽掉的地方,在秦州更西的地方,还有一个让他心神不安的隐患。

梁乙埋连忙对罔萌讹道:“罔萌讹,你速遣人去找禹臧花麻,传本相之命,让他提防河湟,不得疏忽!”

注1:党项语中对天子的称呼,汉义为青天子。

