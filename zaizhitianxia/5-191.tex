\section{第22章 早趁东风掠马蹄(中)}

以韩冈过往的累累成就,以及他过往一干对手的下场,当他一副胸有成竹的要与人打赌,世上往往会被他的气势压住,没有几人敢于随便应下。

但王旖是韩冈的枕边人。知道韩冈对没把握的事,有时候会更加虚张声势,反倒是有了十足成算后却会装着没什么把握的样子。平常在家里下棋、赶双陆,没少用诈术,早就不会上当了。

“官人又要骗人了。”

“那奴家当真就跟官人赌了?”

王旖和严素心一人一句的笑说道。

“当真要跟为夫赌?”韩冈笑了一声,回头瞥了一眼已经将曲礼打发了,正低头垂手的站在门外没有进来的何矩,“何矩,你说这飞里黄这一场能不能赢?”

何矩听了韩冈的问了,便往房里走了两步进来。

何矩这等大掌事与行里定的都是终身契,在名份上从属于顺丰行,而韩家是顺丰行的大东家,从这个时代的风俗上说,基本上就是韩家的家仆。虽然进不了内院,但韩冈的妻女出来后也不需要回避着他。所以能在韩冈还在外面的时候,领着韩家女眷进包厢安顿。

如果不是这样的身份,大户人家的女眷都要戴上帷帽遮着脸面,否则名声上就有些问题了——当然,不得不挑起一家重担的当家主母可以例外,而且也只局限于官宦豪门,普通人家可没那么多规矩。

“回端明的话,飞里黄其实不差,若是一场场比下去,十二场之内当能攀上乙级。今天这一场,如果路程能减半,赢得必然是飞里黄。而且若不是今天这般一上来就猛冲,其实也有很大的胜算。”何矩斟酌着言辞,尽量两边都不得罪,但王旖嘴微微嘟了一下,还是有些不高兴。

韩冈将自己得意的笑容展露在王旖的面前,看着有几分轻浮的故意笑道:“怎么样,为夫说得没错吧?”

王旖转头开比赛,根本不理会他。

“不过这一局面其实是刻意的,一开始飞里黄的马主就没打算赢。”何矩突然插了一句嘴。

这时候,比赛已经到了后半段。转过一个弯道后,渐渐慢下来的飞里黄和后面追上的几匹赛马快要挤作一团,争抢着内圈,使得比赛进入白热化的阶段。看台上助威掀起阵阵声浪,隔壁的包厢里,也传来了一阵阵毫无顾忌的大声叫喊。

但韩冈和王旖、严素心都猛然回头。“这话怎么说?”王旖问道。

“是要保谁得胜?”类似的战术,韩冈在后世见得多了,听何矩一提,立刻就反应了过来。

何矩没打算放过这一次在韩冈面前表现的机会,“回端明的话,是八号黑风追云。十三号飞里黄先出头带着快跑,领着其他马一起跑,只要其他其他骑手没防备上了当,赛马后半程就同样接不上力气。八号黑风追云跟在后面的大队中,跑得是最轻松的,到了后面就可以冲刺了。”

严素心手上有个事先发来的册子,大略介绍了赛程和每一匹参赛赛马的资料。用活字印刷出而成。从纸质到印刷都很粗糙,远比不上雕版精致。不过胜在快速而且廉价。虽然排字一定要识字,但雕版匠人刻出一部书耗时太多,最后算起工钱来,还是排字工少一点,所以用过就废弃的报纸、广告,或是寺院散给信徒的经文语录多有用上活字印刷。

严素心将册子展开,翻了几页后递给王旖,道:“黑风追云和飞里黄的马主不是一个人啊。”

“本来就不能是一个人,这在赛制上是严禁的。”韩冈说道,“就像一场球赛,踢球的两队不能是同一家的球队。但明面上不行,却也保不住有人暗中做手脚。”

从赛制上,一开始就禁止了同一家的两匹马参加同一场比赛,以防由此来使出保送战术。但实际上这样的违规行为,是禁绝不了的——只要有足够的利益在!

何矩低头回话:“飞里黄和黑风追云虽然马主看着不一样,但实际上还是一家的。这两匹马是四个月前,从凉州一起被人买走。”他用低得只能让韩冈、王旖严素心才能听见的声音说道,“这是行里从凉州传来的消息,是高太尉家。”

韩冈哼了一声,他知道,何矩话中的高太尉就是高遵裕。

自从在伐夏之役失败后,高遵裕就被投闲置散了——也幸好这一场战争胜了,没有被责罚,还是留在了京城中,只是没有事可做。如今看起来,倒是找到乐子了,只是这急功好利的脾气,还是没有改。

“黑风追云要赢了!”韩云娘窗口的栏杆边叫了一声。

韩冈、王旖都转头望着赛场上。

已经是最后一圈,赛场上提示用的红旗也挥了起来,看台上的鼓噪声也陡然间拔高了数倍。十二匹赛马中,唯一的一匹黑马也就在这个时候从混战中脱颖而出,速度逐渐加快,如黑色的旋风一般从外圈赶超上去,转眼间就进入了第一梯队之中,而在前半程领衔的飞里黄则是掉了队,越来越慢。

“苦心积虑啊!”韩冈摇头感慨了一声。

何矩道:“其实三号卷毛青的骑手是顶尖的老手,是濮王府名下,赢面也不小,现在也在前面。”

“濮王府?”韩冈拿着望远镜看了一下,又回头,“……是濮国公?”

“不是,是邺国公。”

韩冈眉头挑了一挑,原来后台是宗室中最喜游乐的邺国公赵宗汉,也是当今天子还在世的近二十个亲叔叔中最小的一位。

英宗的兄弟多,连英宗总共二十八个。这几十年,濮王一系,爵位都在英宗的兄弟们手中传递着。能袭爵的都是兄弟,想要落到下一代去多半还要几十年——最小的赵宗汉,只比神宗大了七八岁而已。

如今执掌濮王府的赵宗晖是赵顼的嫡亲叔叔,身任濮国公、淮康军节度使。想升到郡王,还得几年的功夫。再往上升嗣濮王就得更久了,最后能不能承袭濮王一爵,那还真是难说。

大宋的封爵体系有别于汉唐。亲王就算后人由袭封,也不会立刻封爵,都是得从郡公、国公、郡王一路升上去,很多时候,用个十几二十年升到郡王,到了死后才能再得赠一个亲王封爵。

这样的制度甚至使得仁宗时,宗室中甚至出现几乎无人拥有王爵的局面,让仁宗皇帝不得不加封十位太祖、太宗和秦悼王三兄弟的嫡系为王,免得入太庙时场面太过难看,只是这封爵晋升的制度并没有改变。也不过二三十年功夫,宗室中的王军又少了大半。就是出了英宗和当今天子这一系的濮王府,也没有一个王爵。

赵宗晖是个循规蹈矩的人,能接掌濮王府,除了因为他的排行靠前之外,也是与他的品行有关。但当今天子的叔叔和叔伯兄弟中,也颇有几个好玩闹,最爱声色犬马的。不说别的,光是蹴鞠球队,濮王府一脉就养了三支。这三支球队,因为位置的关系,集中在一个赛区里,前两年每一个赛季都要火花四溅的拼上个好几次,为了一个参加季后赛的资格,争得不可开交。也就今年,在赵宗晖的调解下,其中两个养着蹴鞠球队的国公终于搬了家——否则兄弟情分再过两年都没了,能在京城联赛进季后赛,那就代表着上万贯的收入——赵宗汉就是其中一人。

韩冈笑了一声:“养了一支蹴鞠队还不够吗?连赛马也插一脚进来?”

“赛马从低级往高级晋升,若是一路头名的话,只要六场就足够升到甲级。而手上拥有一匹甲级的赛马,可不比现在手里有着一支季后赛球队一般稍差。”何矩瞅了眼韩家的大女儿,声音忽然压得比此前还要低,“能上场的都是没阉割过的公马,一旦能得了头名,配一次种,可都是几十贯。若是多拿两个头名,就是日后不能跑了,一年三五千贯也照样没问题,又有谁能不动心?”

韩冈摇摇头,这个卖点还是他告诉冯从义的,用不着何矩转述。

赛马联赛不过刚刚兴起,远还没到形成一门产业的时候。也就是赛马总会通过四处放风,硬将种马经济这个概念给炒热了起来。冠军马配一次种就二三十贯,其实是不值的,陇西就没有这般夸张。但架不住京城中富贵人家多,人人往里面挤,自然而然价钱就起来了,就跟后世常见的情况差不多。但也是因为有蹴鞠联赛的例子在前,否则也没那么容易引人上钩。

“邺国公还是为了面子居多。”王旖倒是不喜欢什么都提钱,而且她也不喜欢何矩说的话。

何矩自是不敢跟主母辩,默不吭声的低下头。

但韩冈则道:“话是说的没错,的确是为了面子居多。但邺国公家的三个女儿年纪也到了时候,嫁妆不好办啊。兄弟之间可也不方便借,哪家没女儿待字闺中,都愁着嫁妆怎么办呢。总不能丢了濮王府的脸面。”他笑了一声,也是面子。

宗室也没有资格打掌权的主意,对于地位到了赵宗汉这个等级,钱和面子都很重要。

而就他们在说话间,只听得一阵如同山崩地裂的呼啸,伴随着比赛决出胜负的鼓号,上万人同声而出。韩冈定睛看过去,竟是一匹色泽暗淡的灰马在终点后昂首阔步,而此前正在争夺头名的黑风追云和卷毛青不知在何时,竟然落在了后面。

“黑的马和青的马撞上了。”金娘回头,细声细气的说着。

“世事难料啊!”韩冈大笑。

