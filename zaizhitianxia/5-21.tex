\section{第三章 时移机转关百虑(七)}

大年夜的横渠书院,还有弟子逗留其间。

关中交通不算方便,留居在书院中过年的学生还有三四十人之多。也都是家境贫寒的学生,如同当年的韩冈,没钱回乡,其中有些人甚至连束脩都给不起。

幸好横渠书院名下的田地今年丰收,田租充裕,加上来自四方的捐赠,也支撑得起这些学生的日常食宿。

一顿丰盛的年夜饭吃完,苏昞在院中走了一圈消食,回来后就看见还有几个学生聚在一起,一人端坐在椅上,一人则用手指点着他的额头。旁边的人围着笑,而坐在椅上的那一位不知为何脸胀得通红。

“这是在做什么?”苏昞走过去。

几名学生连忙站成一排,坐着的也站起来了,看起来慌慌张张的。用一根手指抵着人额头的那一位低头回话:“回先生的话,学生几个今日看《桂窗丛谈》,上有重心一篇,说了不少道理。学生愚钝,只看文字难以领会,现在只是准备试验一下。”

苏昞闻言一笑,韩冈的新书他都翻遍了,那一篇也看了。上面说的东西的确很有趣。虽然是日常所见,甚至是每一个人无意识都在做的,可偏偏几千年下来,没有人真正能说出其中的缘由。

“试验的结果怎样?”苏炳问道。

那名学生恭恭敬敬:“书中果然是说的没错。坐在椅子上,身子不前移,不将重心移到脚上,除非能用手支撑,否则就必定站不起来。”

其他几名学生一起配合着点头。

“打赌了吧?”

几个学生脸色都变了,脸上的笑都没了,一个个变得吞吞吐吐起来。

书院中自有规条,除了射柳、投壶这样合乎儒家礼仪的赌赛,其余赌博一概禁止。如今几人明知故犯,又被山长苏昞捉个正着,一顿责罚肯定就免不了了。

苏昞却笑了起来:“今天是新年,是要为师下不为例,还是一以贯之?”

几名学生这下犹豫起来了。下不为例,这件事就算揭过;一以贯之,可就是逃不脱责罚。

还是那名指着同学脑门的学生站了出来,向苏昞躬身道:“先生,此事因张营而起,甘领责罚。不过诸兄乃是受张营牵连,惟愿先生罪责止吾一人。”

张营出来请罪,其他几名学生立刻争先恐后,“先生,此事不是景前一人之过,学生皆有份!”

几个弟子争相请罪,苏昞一时心情大好,笑道:“既然你们知错,也不需重罚了,抄经书好了。纸墨自己去领,将五经都抄写一通下来,上元节前要完成。”

学生们连忙恭声应诺。抄书对他们来说不是什么大不了的,自家读的书全都是亲笔抄写而来。墨和纸又不便宜,许多好书都抄不起。苏昞这是明着责罚,暗地里帮忙呢。

种建中走在雪地中,脚下的雪吱呀作响。

放眼书院内外,满眼都是雪光。

年前的这一场大雪,挡住了种建中回乡的道路。

雪橇车能压在雪地上不陷下去,但拉车的马却做不到。一步一个坑的慢慢向前走,本来能来得及在除夕之前赶回京兆府老宅,眼下却不得不在横渠书院中歇息。

其实原本到了宝鸡就该歇下来了,是种建中觉得应该顺便跟师门联络一下感情。而且横渠书院里面怎么说都是有不少自家同门,总比孤伶伶的在宝鸡县过年好,便又赶了一阵。午后抵达书院,与苏昞和其他学子也是聊了好一通,顺便还祭拜了先圣和张载。

“哥,早点歇息吧,还真的要守岁啊!”种师中站在走廊上,远远地冲种建中喊着。

种建中和种师中两兄弟。种师中是得荫补的官,但他离二十五岁还有几年时间,没资格出来接受实职差遣,只能跟着兄长东奔西跑。

从延州至渭州,又从渭州回京兆府,来回赶了十几天的路,中间只在渭州歇了一天,种师中已经没力气了,再能熬的身子骨也吃不消连日在山川间的奔驰。沿途驿马给他们换了个遍,骨架子都散了。

“彝叔、端孺。”苏昞这时进了客房所在的小院。

“季明兄。”种建中带着弟弟上前行礼。

“还没有歇息?”苏昞说道。

“除夕当是守岁。”种建中笑说着。身后的种师中却低头捂着嘴,打了个哈欠。

“这个时候还奔波在外,彝叔你们兄弟俩也是辛苦。”

种建中叹了口气,请了苏昞进房中坐下:“出站之后,各路难合兵,又不便联络,只能事前先打个商量。”

“若是设立宣抚司,统管整个战局,也许情况会好一点。”

“季明兄,给你说句实话。六路诸将,还没一个指挥过十万大军。包括家中叔伯也是一样。

”种建中道,“而且陕西之地,多山谷、多沟壑,本来就不是能展开大军作战的地方。就算设立宣抚司,到了下面,还是得自行其事。将陕西缘边分作五路,难道是没缘由的吗?实是地势如此,不得已啊。平戎万全阵,河北能布,陕西可是布不开。”

平戎万全阵是当年太宗皇帝亲自设计的阵图,命河北依图布阵,是一套用十余万兵力在平原上布下阔达二十里的战阵。辽人入侵时的确不会往上撞,他们会直接了当的绕过去。

“记得当年韩子华相公领陕西宣抚司的时候,当时光是鄜延路就有十一万大军,全军兵力超过三十万……”

“那是连乡兵、民夫都算进来的数字,真正能上阵厮杀,可堪一战的禁军,一路最多也只有三五万。”有句话种建中还留在肚子里,如果将空额减去,兵力会更少,“不过这一次,如果对西夏开战,厢军、乡兵弓箭手都会上阵,就算不能与铁鹞子厮杀,拿着神臂弓守寨子总不会有问题。”

“其实可以让泾原、秦凤和熙河三路攻打兴灵,环庆、鄜延攻打银夏。两边本来就是秦凤转运司和永兴军路转运司负责支援粮秣,各自合兵也是一个办法。”

种建中摇了摇头:“季明兄,那可是兴庆府!”

苏昞怔了一下,也摇摇头,不说话了。

的确,种建中的拒绝很有道理,那可是西夏国都兴庆府。若能独占头功,就是能吃三辈子的功劳。在大功面前,谁能忍得住?不先打个你死我活,都是军纪良好了。

你要两边各打各的,秦凤转运司支援泾原、秦凤和熙河攻打兴灵,永兴军路转运司负责支持的鄜延路和环庆路攻打银夏,河东再插个花。这个计划的确是不错,但也要鄜延、环庆的将校们答应才行!

西军中的哪个将领不想站在兴庆府的城头上,哪个不想第一个杀进党项人的皇宫?那是西军将校中多少人梦寐以求的美事。敢挡他们的路,嫌自己的日子过得太平淡吗?

苏昞本就是来探问客人,只是顺口说了一下如今的时事。话题接不下去,就又说了几句闲话,便起身告辞。明天种家兄弟还要上路,也不能多打扰。

送了苏昞离开,种建中回来后轻叹了一声。他的确是拒绝了苏昞的意见,但这件事本来就轮不到他们来插嘴,至少也得是种谔和韩冈一级才有资格插话。

不过韩冈现在是什么想法,根本让人猜不透。

王舜臣如今出了事,虽然还不知道内情如何,但他这一级的武将,被捅到天子面前是必然的。

谁也说不准这到底是不是项庄舞剑意在沛公,明着打王舜臣,暗地里则剑指种谔。种建中此番联络泾原诸将,本来是要在渭州七术种谊那里过年的,就是因为王舜臣之事,才匆匆赶回京兆府,只是又在路上因大雪而拖延了行程。

还不知道韩冈听说了此事会怎么想,会不会以为是种家故意放纵王舜臣受到攻击。种建中心中隐隐生忧。

王舜臣出身种家,被视为嫡系,甚至在河湟起家之后,还被招做了种家的女婿。后来被调到鄜延路,就是因为有这层因素在。

在鄜延路中,王舜臣表现得并不算差,只是运气不好,三年前在夏州吃了一个亏,连着几年都没有机会向上爬上一步。

眼下则是由于韩冈对展开攻夏之役的阻挠,使得王舜臣在种家内部有点不受待见。但韩冈与其兄弟相称,过了命的交情。在军中,上上下下也都给他几分面子。

但这一次的利益实在太大了,不论谁能成为先锋,整个功劳的三成都能揽到身上。尤其是鄜延路直面的银夏之地。

在三年前的横山之役中,官军其实已经攻入了银夏,甚至占据了银州,兵锋直指瀚海,只是因为各种各样的缘故,而不得不退回到横山南侧,但在离开前,官军放了一把火,烧了整个银州城,而动手的,正是王舜臣。

三年过去了,银夏至今没有恢复元气,这是个再好捏不过的软柿子,带兵走过去就能捡功劳,谁甘心让别人捡去?

不知道天子会怎么处置王舜臣。自家叔父请求让他戴罪立功的奏章当是已经递上去了,但有多少作用却还难说。

种建中无可奈何的长声一叹,乱七八糟的事怎么就这么多!

