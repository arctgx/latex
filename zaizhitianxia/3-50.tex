\section{第19章 波澜因风起(上)}

三月中的时候,洛阳春光正好。

牡丹花开正艳。

这一富贵雍容的花卉,开遍了洛阳城的城里城外。

寻常的黄花魁、泼墨紫、首案红,处处可见。稀少的一点姚黄、魏紫也能在几大知名园林看到。甚至还有金带围,本是扬州芍药特有的品种,但今年,洛阳牡丹花会上,却又一家花农端出了一本,重瓣色做红紫,而花.芯一圈黄蕊,正如衣着朱紫,腰围金带的宰辅重臣。一时间轰动全城。

扬州的金带围,传言簪花者可为宰相——韩绛守扬州时,金带围花开四朵,王珪、王安石其时正在城中,皆受邀请,唯缺一人。韩绛其时道:今日若有客来访,便邀之共赏。傍晚时,一人来访,却是陈升之,便一同受邀观花。到现在为止,在场的四人已有三名做了宰相,就不知道现任参知政事的王珪,有没有那个运气。

也不知道洛阳的这本金带围牡丹,有没有昭示宰相的能力。

而此时也正是出城踏青的时节。

洛水岸边,一片青布围起的帐次中,丝竹之声徐徐而出。引得来往的游人为之驻足,但隔着春风也吹不开的布帘,还有虎视眈眈的一圈家丁,也只能在外面过一过耳瘾。

闲居在洛阳城中的前任宰相富弼就在帐次之中。

富弼几任宰相,自是富贵无比。家里养的乐班,在洛阳城中,也是极有名气。伴着煦日春风,看着舞姿娉婷,斜倚在软塌上的宰相悠然自得,已经是超脱于滚滚红尘之外,带着几分逸气。

“大人。”帐帘一动,富弼的儿子富绍庭走了进来。

“什么事?”富弼一边问着,一边一挥手,示意乐班退到外面去。

“今科的金榜已经出来了。”

富弼没吭声,这点小事不至于忙着来通知他。必有他事,就等着儿子自己说出来。

“状元唤作余中,宜兴人。榜眼是朱服、邵刚。这三人倒没什么,也的确够资格。只是排在第九、第十的,一个是王安石的女婿,一个是王安石的侄婿。两人竟然同时及第,这件事一传出来,听说东京城中的士子一时群情激愤。”

富家前日被前任河南知府李中师欺负惨了,收免行钱竟然收到了宰相家的头上,富绍庭恨不得咬下王安石的一块肉来。现在听说王安石要倒霉,免不了兴奋莫名。

富弼呵呵笑道:“还以为是状元、榜眼,王介甫的眼界未免小了点。”只是说着便有些觉得不对劲,沉吟了起来,“王介甫什么时候变成了这样的性子了?”

“韩冈、叶涛此二人才学不足,想必王安石也不敢让他们一问鼎甲……”

“韩冈,叶涛?”富弼一下打断了儿子的话,“王家招了他们做女婿?!”

在士林中薄有文名的叶涛倒也罢了。但韩冈乃是在富弼这等重臣中都有着不小的名气。突然听到王家找了他做了女婿,富弼心中不免为之一惊。

“是啊!韩冈是王安石的女婿,叶涛则是王平甫的女婿。他们两个竟然能同时跻身前十,要说王安石没有做手脚,谁能相信?!”

“平甫跟王介甫可不是一条心。”富弼没空去听儿子说废话,一摊手:“卷子呢,现在应该已经送了吧?”

富绍庭连忙从袖子里掏出两片纸来,双手递了过去。

富弼接过来,凝神细看。两篇文章都不长,但他足足看了有两刻钟的功夫。最后,舒手递回给儿子。“这个叶涛,也就第三等的水平。言之无物,写得好看而已。”

果然其中确有情弊,富绍庭猛点头,又问道:“那韩冈呢?”

富弼半眯起眼睛,回忆着方才看到的文字,咀嚼良久。最后,方缓缓道:“他还不错,当得起第九名的位置。”

“大人为何如此说……韩冈的这份卷子比叶涛要差得多啊!”富绍庭惊讶的问着。

富弼瞟了眼不成器的儿子,暗自叹息。

但凡有点眼光的官员,都不会说韩冈的文章不如叶涛。韩冈在文中表现出来的见识和才干,足以让他这等老于事功的宰辅感到惊艳。也就是那些个读书读到傻的措大,才会以为韩冈的文章当不起前十名的资格。而自己的儿子还附和着这种说法,当真糊涂!

收拾心情,富弼摇了摇头:“这份卷子写得好得很,文字稍强一些,就够资格争状元了。”

“……这篇文章真的有这么好?”

富绍庭还是不敢相信,小声问着。他才学再不济,但作为宰相的儿子,文名盖京华的名士也见多了,眼光总是有的。在他看来,韩冈的文字当真是不怎么样。

“司马十二最近在独乐园里挖了个地窖,躲在里面写书。多半还不知道今科的事。你将这文章掩了姓名,去问他,看看他怎么说!”富弼哼了一声,“文笔从来都是末节,平易无错处也就够了,韩冈的这篇策写得恰到好处,根本就不是贡生能写出来的文字!”

富绍庭顿时眼前一亮:“大人的意思是有人为韩冈捉刀?!”

“捉刀?”富弼抬头狠狠瞪了儿子一眼,“韩冈是寻常的贡生吗?看看他在陕西,在熙河做得多少事。卷子中说的那些事,都是他素日里看的、听的、做的、判的,早就明会于心,又何须他人捉刀?!”

富弼训着儿子,忧怒于心。

他这个儿子,连怎么挑人错处都不会。对着刀锋一口咬上去,崩掉牙不说,反手可就会挨上一刀!连个御史都没法儿做,日后真是不知该怎么办了。自己死后,又有谁来保富家家门?!

甜中带糯的江米酒,富弼喝到嘴却是满口发苦。

想想自己的妻弟小山【晏几道】,自从岳父【晏殊】死后,除了喝酒写诗,就做不了一件正经事,好端端的家业转眼就败了,新近作出来到诗词,满眼都是衰亡萧瑟的味道,哪还有半分‘梨花院落溶溶月,柳絮池塘淡淡风’的富贵气象?

而自家的儿子不会做官,连诗词都做不好,也就喝酒的本事能比一比,日后可怎么得了?难道真的要靠着现在正做着参知政事,却跟自己不是一条心的女婿【冯京】吗?

“但韩冈不过弱冠之龄,只是个幸进……”富绍庭还想争辩,但在富弼严厉的眼神中,声音越来越低,渐渐不敢再说。

富弼冷哼一声。

当初说新党尽是新进、幸进,那是说给诸多熬着磨勘一步步向上爬的官员们听的,要引起他们的同仇敌忾之心。但若是当真以为年纪轻轻,能力就会不足,那就是太蠢了——换做是他富弼,还有韩琦、文彦博,哪一个不是步步超迁,磨勘三年并一年,最后一步登天的?有些话说归说,但心里要明白,不能自己都给弄得糊涂起来。

“除非能挑出其中的错,否则就不能说他差!”富弼教训着儿子,“诗赋做得再好,若无治事之才,也不过是进翰林院做待诏的命。而如韩冈这般于军事政事上皆有长才的,日后才有资格入学士院,少说一个边地重臣,甚至宣麻拜相也说不定【注1】。”

父亲给韩冈的评价这么高,让富绍庭重又看了看他的文章。只是看了一阵,还是不觉得有多好,抬头又问着,“以大人看来,这文章中可有何错处?”

“韩冈生长在秦州,在熙河为官三载,所历种种,太平官儿一生也难逢上一次,河湟之事尽在其心中。为父若在政事堂中,那还好说,但现今数年不涉政事,想挑刺都挑不出来。”富弼抬眼瞥着富绍庭,“你若能找出其中错处来,就可以不用跟着为父一直留在洛阳了。”

富绍庭闻之颜色一变,干笑了两声,道:“儿子不成材,还是在家中侍奉大人的好。”

费了半天口水,富弼不耐烦的摆了摆手,让富绍庭离开。

自家的三个儿子中,就没有一个能让他放心的。王安石倒是运气,找了个好女婿……

不过韩冈越是出色,就越是危险,能看出他潜力的不只是王安石和自己。现在要找他错处的人,怕是不会太少了,并不需自己多事。

拿起如意,敲了敲压着席子四角的虎镇,退到外面的乐班家伎便近前来,将方才停下来的歌舞继续下去。

自己都致了仕,只要不被欺上门来,也没什么好多想的,元老重臣的体面天子总是要给上一点,李中师之所以被调任,也就是天子给他富弼面子的缘故。

至于朝堂上勾心斗角的烦心事,让还在做着官的文彦博去头疼好了,

“恋栈不去,活该你头痛!”

春风中,洛水畔,富弼白发银簪,道袍随风,望之有道骨仙风。轻轻击掌,为曲乐伴奏,重又开始欣赏起家妓的妙丽歌舞来。

注1:在宋时,翰林学士院和翰林院是两回事。翰林学士居于学士院中,身为两制官,为‘天子私人’,有草拟诏令之权,是朝廷重臣跃上宰执之位的重要台阶。而翰林院,则是以琴棋书画和诗词歌赋来侍奉天子,官名为待诏,也就是天子豢养的清客而已。

