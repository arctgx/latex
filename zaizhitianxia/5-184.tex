\section{第20章 土中骨石千载迷(13)}

“怎么可能是韩家?”

这下轮到张十九难以置信了。韩琦的名声在民间可是大得很,他的儿孙怎么也不应当转着发掘古墓的念头。

“怎么就不能是韩家?”周小乙冷笑了几声,转头他看着张相,“五哥你应该最清楚,”

豪门大族私底下到底有多龌龊,张相当然是清楚得很,他的买卖也只有从豪门大族身上才能放心大胆的赚到钱,来往得多了,许多消息也就自然而然的钻进了耳朵里。”

“多谢小乙。”张相向周小乙躬身一礼。

周小乙忙摇着手:“我也只是顺道提醒张五哥你罢了。我现在就要出城回洛阳去,迟了恐怕就来不及了。”他顿了一下,又问道:“张五哥你呢,要不要一起回洛阳?”

张相想了一阵,最后还是摇头;“我今天才到相州,累得够呛,打算再多留两天,好生将养一下身子骨。”

听到张相这么说,周小乙也不多劝,拱了拱手,直接就从小巷子中绕了出去,转眼就不见了踪影。

张十九围着张相,又急又怒的问道,“哥哥,现在该怎么办?”

“先去州衙看个究竟,至少罪名都要打探明白。”张相说道,“有些事不去探明明白,光是躲避,有许多事永远都没办法查清楚来龙去脉。”

张相小心谨慎的往州衙去,到底怎么安罪名,他肯定是要当面去看看一看。

州衙的前面,拥挤了数百闲人,都是想知道知州到底想怎么处理这一次的变乱。而州衙边上,便是韩家在城中的大宅,名气响亮的昼锦堂,就在那间大宅中。

张相被堵在了州衙的正门口,正想着要怎么才能挤进去,就看见从北而来的一队车马,分开州衙前的人群进了韩家。也不知是哪家的大官人来了,出来迎客的明显是韩家的子弟,而不是普通的管家。

不过这也不干张相的事,他现在还犹豫着到底是走还是留?

现在走未免太可惜了,一堆堆金银在眼前灿灿发光,就是想走,也挪不开脚步。

干脆与想吃独食的安阳人徐兴徐胡子拉上关系好了,张相这样想着。

洛阳这边的人脉在自己手中。在中间做个周转,尽管不比之前的盘算,但也是一笔不小的进项。而且不必冒风险——性命无忧终归是件好事。

……………………

“多杀几个收赃的贼人,安阳这里也就能安定一点了。”韩忠彦挥手让来报信的仆人退下去,转头便杀气腾腾的跟李清臣说着,“这等贼人,死不足惜!才几天功夫,安阳县这边的古墓全都遭了贼手,往城外走一走,田里面全都是一个个坑。”

李清臣叹道:“还不是王介甫和韩冈翁婿两个闹的,争道统争得地下的先民都不得安生,真是不知让人说什么才好。”

李清臣是韩琦的侄女婿,刚刚从定州任上回来,来相州本是顺道来走亲戚的,谁想到撞上了这一档子事。说有趣倒是有趣,但妻兄韩忠彦就在眼前气急败坏,李清臣也不敢笑出来。

“韩冈的表兄李信也在定州,是个老实人。这两年在定州,从来只在军营里教训士卒,下面的赤佬都给治得服服帖帖,连扰民的事都少了许多。”李清臣说着,“韩冈要是脾性能跟他表兄一样,也没这么多乱子了。”

韩忠彦点点头,身在河北,河北军中的有名将领,他也是都有耳闻。李信被郭逵从南方调来河北,作为一个外人,能很顺利的融入一向排外的河北禁军,又没有同流合污,这份能耐的确出色。当然,最关键的是李信为人老实沉稳,对文臣和读书人都表现得很尊重,所以让人欣赏。

李清臣用话分了韩忠彦的心,转过来则又问:“殷墟甲骨究竟是什么模样?我一路南下,在驿馆中听得人吹得神乎其神,就是没一个靠谱的。”

“要看也容易,我这里正好就有。”韩忠彦提声叫来一名仆役,吩咐道:“将四哥和他的朋友一起请来。”

很快两人就应招而来。

一个便是韩琦的四儿子,韩忠彦的弟弟韩纯彦,另一人年岁与韩纯彦相当,三十出头,但李清臣不认识,不过身材颀长,相貌斯文,看起来很是出众,在李清臣面前自报家门:“历城李格非,见过韩龙图、李博士。”

韩忠彦现在是龙图阁直学士,一般称呼是龙学、直学,但尊称一声龙图也可以,反正韩冈不在此处,也不会让人弄混。

待韩纯彦和李格非与李清臣见过礼,韩忠彦便对李清臣介绍道,“李文叔是熙宁九年的进士,现今在相州州学中任教授,也与我家有旧,不是外人。”

李格非也在旁道:“在下父祖皆出自忠献公门下,曾在陕西和京城任职。”

韩琦做了多少年宰相,在他手下做过官的多了,这样就称是门下,那天子手下就没人了。李清臣知道这不过是贴上门来拉交情的奉承话,也不以为意。

但韩忠彦对这李格非的看重,也是有缘由的,“文叔在金石上,眼光独具,上次我那一具铜鼎,便是由文叔鉴别出来,乃是东周虢国之物。另外两件藏品,则是被他看出了破绽,是奸人伪造。”韩忠彦介绍了两句,又对韩纯彦道,“还不将那几片甲骨拿出来。”

韩纯彦向身后一招手,跟在后面的仆人捧着一个托盘,将几片甲骨递了上来。

韩忠彦说着:“这几片甲骨,跟《龟策列传》和其他几部书中所言无讹,的确是占卜之后刻上卜辞的样子,此处又是殷墟所在,倒有九成九是殷人遗迹。”

李清臣知道,韩忠彦的手上应当还有殷人礼器,所以才能这般确定。不过人臣私藏上古祭礼之器,而且说不定还是为天地鬼神之用,肯定是犯忌讳的,肯定是不能说出来。

李清臣拿起托盘上的银框放大镜来看,但完全认不出上面用刀刻出来的文字究竟是什么意思,只能用点头来掩饰自己的无知。

幸而有李格非在旁解说:“仓颉初作书,依类象形,故而谓之文,其后形声相益,即谓之字。此乃《说文解字》序中所言。象形之文,形声之字,合起来,方是如今的文字。由此可知,越是近于仓颉之时的文字,象形之文越多,而形声之字越少。”

李格非虽然年轻,但一番话说得条理分明,让李清臣的眼神中多了几分赞许。

李格非指着排开在托盘上几片完整的龟甲和骨片,“‘象形者,画成其物’。甲骨之文远比篆书籀文,更为近于图画。多为象形之文,更近于上古。”他点着其中一片骨片上的一个文字,“有些字如果当成图来辨认,还是能揣摩出其本意来。”

李格非的手指指着一个月牙图案,中有一点,李清臣看了几眼,略有几分犹疑的问道:“这是‘月’?”

“应该就是。若能全都辨认出来,殷商时的礼仪,也能从中了解一二了。”李格非慨叹道,“三礼《周礼》、《仪礼》、《礼记》,但其中有多少篇是后人伪作,那就难说了……先圣曾言,郁郁乎文哉吾从周。虽然是想‘从周’,但流传下来的三礼若是为后人所杜撰,哪又该怎么办?只能设法从源流上来找。”

这番话就是气学的韩冈借助殷墟之文来颠覆如今儒门经义的理由,倒是被越来越多的人认同。李清臣摇头笑叹,‘周监于二代’——正好这里就是殷墟。

李清臣也不清楚眼下的局面到底是不是韩冈的初衷,但一切的发展,都使得《字说》乃至《三经新义》,必须要面对殷墟古物的质疑。

气学能不能争得过新学,那是另外一码事,但新学的确是被气学用力的扯了一把下来。按说给新学添堵,不是什么

“相州民风一向淳朴,如今却被闹得四民不安。这几日便要上书天子,把相州的乱象跟天子说一说。”韩忠彦看着李清臣的眼睛,“乡里的农户都只顾在田里挖坑,明年怎么种地?”

“说的也是。我昨天在驿馆中还听人说起,这几日一片有文字的完整龟甲,已经涨到了近一贯。如果不论衣赐,我这个太常博士一个月的料钱也只有二十贯。”李清臣感慨着,“有着卜辞的甲骨,只要挖出来百十片,置宅买田的本钱就有了,百姓哪有不趋之若鹜的?一来二去,民风就这么给败坏了。”

李清臣的话中,隐隐的透着拒绝之意。在他看来,一贯一片的价码是在太高了,由不得人不心动,根本就堵不住。何况一池浑水,漩涡阵阵,事不关己的何必硬往里面趟过去。看热闹就是了。

韩忠彦看着身前的酒杯,他本也不指望李清臣能帮着说话。

十年前,李清臣曾经辅佐韩绛经略横山,攻打罗兀。当此役战败,韩绛贬官出外,而李清臣则是倒戈一击,四处放话诋毁韩绛,以求自全。

这样的人品,据说天子也是鄙薄不已,要不然这些年来,李清臣做为相州韩家的女婿,也不至于一直都沉沦下僚。

韩忠彦将眼中的鄙夷藏起来,看来也只能指望天子了,否则相州的乱象绝难平息,韩家的家风也维持不住了。

