\section{第31章 停云静听曲中意(十)}

战火正炙。

战局始终没有打开,一队一队的骑兵被投入战场,却像是落入了磨盘,转瞬就不见了踪影。时间一点点的过去,冬日昼短,已经是午后时分。宋辽两军已经厮杀了半日之久,这场会战却仍看不到终结的迹象。

种建中手持双刀冲杀在战场中。鲜红的战袍被血染得泛黑,脸颊上的一道长长的伤口,血肉外翻,深可见骨。

呼吸仿佛带着火,灼烧着喉咙、灼烧着气管、灼烧着肺,浑身如同水淋,浸透了汗水。他胯下的战马,口鼻中喷出长长的白气。半日的战斗,种建中已经换过了两匹马,这第三匹看样子也已经快要支持不住了。

一刀将对手的右臂斩下,另一把刀隔开了一支流矢,又低头让过迎面而来的刀锋,但背后一声风响急扑而下,那是铁锏破风的声音。风声猛恶,种建中浑身寒毛竖起,闪避已是不及,埋下头耸起肩膀,将背甲架起,硬生生挨了这一下。

力逾千钧的一击重重的打在了种建中的后背,厚实的背甲被砸得反弯过来,一股腥甜随即涌上喉间,人也一下趴在了马背上。胯下的战马腰背一沉,希律律的惨叫了一声。

种建中紧紧咬着牙关,反手挥起一刀,只听得一声马嘶,眼角的余光便看见追在身后的战马人立而起,将马背上的骑兵抛到了地上。

冲得太快了,口角溢血的种建中直起腰背环顾左右,心中顿时一冷,竟是孤身一人陷入敌军重围之中。跟随着他的一众骑兵,全都没有跟上来。周围的辽兵看着种建中皆是双眼发亮,身上甲胄和战马让种建中如同一颗石块中的宝石般显眼异常。

只是绝望的情绪还没有来得及腾起,下一刻,包围圈一角上突然乱了起来,一队党项骑兵蒙头蒙脑的冲入了战圈。

种建中见机立刻靠了过去。党项话他也能说上两句,加上身上的将军甲式样特殊,这两日见过的人不少,吼了两声,便顺利的融入了这一队党项军中,甚至反带着他们冲散了包围上来的一队辽兵,救下来被困的部下。

战场中央的两方军队已经混做了一团,种建中跻身其中,前一刻还是一举击溃来敌,下一刻,就转被人追杀。时时刻刻都能看见骑兵落马,然后被飞驰的战马重重的踏过去。

没有步兵压制的战场,显得分外惨烈。步调和节奏已经远远脱离了任何一方的控制。

当种建中带着残部撤回来的时候,口鼻带血,身上脸上尽是红色黑色的血渍,分不清是他本人还是从被他斩杀的敌人身上沾上的。他的部下们也是一样人人带伤,个个沾血。一队辽兵追在他们的身后,数百骑纵马狂奔,紧紧咬着不放。看样子是准备趁势攻入种谔和帅旗所在的中军。

“乱我军阵者,皆杀!”

种谔心如铁石,文然不动。即便侄儿狼狈而归,被辽军追在身后,他也只是命令前方列阵的预备队举起手上的神臂弓。

种建中很了解他的叔父,并不敢冲击中军,一见友军就要射击,立刻拨马转向,带着所部残兵从阵前横过,纵然有十几骑转向失败连人带马滚翻在地,却也正好把身后的追兵暴露在了锋矢之下。

箭发如雨,冲在最前的一队追兵在瞬息间灰飞烟灭。

今天一战,宋军的伤亡不在少数,种建中几次领兵冲杀,他带下去的骑兵回来的已经不足一半。

种谔身边也只剩最后一支作为预备队的选锋没有动用了。不过他们也是几次下马列阵,就跟刚才一样,用神臂弓射退了好几支冲到近前的辽军。

种谔是以自身为饵,只留下了一千多预备队,吸引辽军以他为目标。但辽人在箭阵前吃了两次亏之后,便果断改攻向了两翼的叶孛麻和仁多零丁,只留下三千多人马,牵制种谔的中军预备队。不是种建中的回撤让辽人看到了机会,方才不会有人贸贸然直冲向种谔的军阵。

辽人仗着兵多,开战前就派出了几支偏师,不过给提前占据战场外几处战略要点的党项军阻截在东南方。开战后,辽军又派了两个千人队,试图直接绕向种谔的背后。不过黄河在进入贺兰山下之后,河道一分四五,多条平行的径流在兴庆府外穿过肥沃的平原。兴庆府外的两条径流之间,便是今日的决战之地。辽军骑兵要想从战场边缘绕道宋军背后,就会有一段不短的路程穿行在冰结的黄河上,速度不会比步卒更快,被种谔派出的骑兵拿神臂弓射了回去。

种建中此时已顺利的撤到了后方,包扎好了伤口,留下出战的士兵休整,自己换了一匹马后,又回到了种谔身边。战火如荼,等回过气来,他还得再领兵冲锋。

种建中在大纛下远观战场,无论左翼右翼,攻守之间还算是井然有序。只有中军这边,打成了一团浆糊。宋军作战,一贯讲究阵法谨严,可当麾下军队的主力由步卒换成了骑兵,却变得纷乱不堪。

仁多零丁和叶孛麻都支援了种谔一千多兵,都能算得上是精锐,装备齐全。可这批党项人指挥起来却是阻手阻脚,要不是种建中和一众宋军将校在前面的奋战,加上种谔留在手中的底牌押阵,中军这边应该是第一个退败的。不过现在却是右翼的阵线退后太多了。辽军的前部,离叶孛麻的大旗只剩百十步。

“叶孛麻支撑不住了!”种建中的心提了起来。

“不,他还能撑得住。”种谔说道,音调没有一丝改变。

一声长号直冲云霄,从叶家将旗下冲出来了一队具装甲骑,骑手全副武装,连战马的前胸都挂着一块如盾的甲片。只有两百不到的样子,却不费吹灰之力便击溃了迎面而来的辽军。

人马皆着甲的具装甲骑,总是一支骑军中最精锐的部分,叶孛麻分明是将老底都拉了出来。

“高遵裕做得好啊!”种谔夸了一句远在京城的老对头,

党项人也在拼命了。对他们来说,不胜即死。如果这一战赢了,就有讨价还价的余地,万一败了,难道还能退回青铜峡去?那时候,宋人说不定会将他们的头颅送去辽国,请耶律乙辛消火。

在装备上,辽人比起党项人并不占据优势,远远输给宋军。在最精锐的骑兵上,宋军和党项军稍逊一筹。但整体实力却不输连老弱都征发起来的三万头下军。只是仗打到现在,辽人的伤亡可能更大一点,但兵力少了三分之一的大宋党项联军却很难比辽军支撑得更长久。

“看出了些什么没有?”种谔还有心关心侄儿在这一仗中学到了什么。

“骑兵不是这样用的。”种建中摇着头。

百里为期,千里而赴,出入无间,故名离合之兵。骑兵应当觑准敌军的弱点呼啸而来,远飙而去,不应该是在战阵上聚成一团的厮杀。也就是当学习辽人的战法,而不是将骑兵当成步卒来使用。

“战法若学着辽人,打起来那就是有败无胜了。眼下的对阵厮杀,倒是辽人更吃亏一点。不把兵力克扣到两万,耶律余里岂会出来迎战?”种谔话声突地一顿,接着立刻陡然高了起来,“仁多也开始拼了!”

辽军的右翼正在溃退,在叶孛麻打出了最后的底牌之后,仁多零丁也将他手上的具装甲骑都放了出来。来自左翼的喊杀声一下高了十倍,整整三个百人队,抓住了辽军阵线上的一个小小缺口,直冲而上。如热刀切开牛油,迎面的辽军纷纷退避,给狂飙突进的他们让出一条路来。而躲避不及的,便在锋刃之前被斩得粉碎。

大公鼎就在乱军之中,竭力想要维持住战线的稳定。渤海人在辽国中,算是战力最弱的一支。他率领本部对阵宋人左军,支撑到现在已经是超水平发挥了。

“大帅,我去帮把手。两百人就够了!”

“迟了。”方才一瞬间的兴奋消失无踪,望远镜的镜片紧紧压在种谔的眼眶上,攥着镜筒的手青筋浮凸,“我给你五百人也没用。”

耶律余里的中军出动了,奔出去救援大公鼎和他的渤海兵。仁多家具装甲骑的冲击,也如同冲入雪中的马车,速度一点点的慢了下来,很快就被见势不妙的仁多零丁收回去了。

辽军也在收回战场上的军队,混乱的中军在种建中回过气来之前,已经变得泾渭分明。后撤一步的辽军用马弓将追上来的宋人及党项人拒之门外。

“不要追了!”

种谔发布命令。他留下的预备队纷纷上马,上前接应战场中的袍泽们回来。激烈的战事稍稍平息了一点,双方都需要有一个喘气的时间。

这是一场决战是事前难以想像的漫长。不过论起韧性,种谔相信他手下的士兵要远远超过辽军。看看天色,离天黑还有一段时间,重新稳住阵脚之后,还能再继续进攻。

必须要有一个胜负出来!他绝不会答应就这么结束这场决战!种谔决心要赶在吕惠卿的命令到来前打赢这一仗。

