\section{第14章 霜蹄追风尝随骠(六)}

【昨日单位有事,断了更,很是抱歉。今天尽量补回。】

王舜臣意气风发的驭马踏进了兰州城中。

将六千兵马,提封万里,收复故土,一直打到了戈壁滩边,将大宋的战旗,插在了玉门关的城头上。时隔百多年,汉家的战士重新看到了玉门关外的大漠孤烟,这份功绩,足以让任何将领感到自豪。

跟随在王舜臣身后西行的战士,回来只有千人。王舜臣所部尽管作为主力历尽大战,在攻克甘州、沙州的时候,与回鹘人也拼了一场,但在战争中的损失依然微乎其微。绝大多数没有回来的士卒,都是被留下来据守沙州、甘州等新近收复的城市。

不过随行抵达兰州的军队,依然有着六千人之多。多出来的五千人,全都是甘凉诸州的汉蕃部族提供给王舜臣的兵员。

王舜臣在攻下凉州,并继续向西进攻后,生活在河西走廊中的汉人、吐蕃人全都被他调动起来,共同消灭过去几十年骑在他们头上的党项人。

随着他一步步的向西前进,聚集在大宋战旗下的汉蕃两族军队越来越多。到最后,对于跟随王舜臣一同西去的六千战士们来说,这一场平定甘凉之地的战争,完全变成了一场轻松无比的列队行军。他们最大的敌人,是一时难以适应的气候和地理,病倒的不在少数。

可不管怎么说,中国之兵重回玉门关,对于想要恢复汉唐旧日荣光的国家来说,都是值得在史书中大书特书的壮举。

所以王舜臣有足够的资格去嘲笑屡战屡败的东部诸路的同僚,“灵州败了,盐州败了,想不到最后翻盘还是靠了契丹人。”

赵隆也是进兵灵州不果的其中一员,听得有些扎耳朵,只是他素知王舜臣的性格,自家又没去灵州、没去盐州,便没把王舜臣的话当成是对自己的嘲讽,“不能这么说,西贼已经在盐州城下耗尽了气力,种太尉领军进抵盐州,打起来之后,赢得多半还是官军。”

“攻灵州的时候,是说必胜的吧?守盐州的时候,也是说必守的吧?”王舜臣哈哈大笑。

他领军收复了甘凉,从此大宋的疆土便直通西域,正是春风得意的时候。而齐攻灵州的几十万大军,却是接连败绩,最后没让局势落到最差的地步,还是靠了契丹人的力量,怎么让他不笑?

之前王舜臣遭人举报谎报军功,故而被夺了官身。在东面任官的那几年,也是过得甚为不顺。这时候幸灾乐祸乃是人之常情。赵隆也知道这件事,叹了一声:“苗总管是被高太尉拖累了,若是两军齐心协力,灵州城还是能打下来的。盐州也是,京营和鄜延军之间还是有着大问题,否则还是能守得住。”

王舜臣嘿嘿笑了两声,却是一言不发,环目看着宴厅中的陈设。

为了庆贺大军凯旋而归,王中正特意设宴,除了王舜臣所部将校,还有同时跟随王舜臣作战的甘凉汉蕃众部的头领们。在各自的座位上,正兴奋的低声交流。

赵隆也觉得没趣,转过话题,“先有灵州之败,之后又是援救盐州不力,高总管将环庆军糟蹋得不轻,天子也忍不下去了。燕逢辰马上要回陕西,接手环庆军务了。高太尉接下来,大概是到哪个偏远小郡做个知州什么的。”

“燕总管得天子看重啊,平常人比不上的。至于高总管……”王舜臣龇牙笑了一笑,“人家可是皇亲国戚,没必要为他担心。”

赵隆不打算跟王舜臣一起背地里说高遵裕的不是,好歹还有几分交情,“说起来甘凉多半会另设一路,毕竟地盘大了,又隔了一重洪池岭,不可能归属熙河路。”

“其实地盘可以更大的。”王舜臣打了个哈欠,“要不是后面催得急,我还准备往西面再走一走,灭两个小国回来。”

赵隆笑道:“开了沙州和甘州城,已经得了不少好东西了。再灭两个小国,你还搬得回来?”

“些许浮财而已,甘州、沙州都是穷地方,就是有点收获,也要分给下面的将士。”

“叫什么穷啊,也不会跟你抢的。”赵隆冲王舜臣翻个白眼,“只要跟你这一次带回来的那匹浅金色的马送我个十匹八匹……”

如同黄金一般闪闪发亮的毛皮,高挑健美的体格,王舜臣带回来的那匹、神骏得只应该在天上宫阙中奔驰的骏马,方才只看了一眼,赵隆就挪不开了眼睛,就是在兰州城中,都引起了一番骚动。

“十匹八匹!……哥哥你扒了我的皮吧!”王舜臣登时叫起了撞天屈来,“小弟西征两千里,到手的纯种汗血马也就只有这么一匹而已。这匹浮光可是要呈给天子的,我都不敢骑!”

五尺多的肩高,军中许多士卒都没这个高度。宋军战马最低标准是四尺二寸,赵隆最好的一匹坐骑是四尺七寸,在军中都已经是凤毛麟角。高过五尺的好马,连天子都没有。像王舜臣带回来的比寻常战马高了一尺的汗血马,王舜臣说自己不敢骑,也的确没说错——实在太惹人眼了。

赵隆一见之下,眼珠子就挪不开,只要是武将,哪有不喜欢好马的?就是文人,也一样喜欢,拿美人换马的事也不是没听说过。要是王舜臣拥有一匹龙驹的消息传出去,到时候人人伸手,又全都是惹不起的角色,给谁都不方便,甚至还会得罪人,不如送给天子,也能讨个好。

赵隆倒是能体会王舜臣的心情,很是惋惜的叹气,“实在是太可惜了……不过这个浮光的名字是谁起的?”

“是小弟门下的一个幕客起的,说是取自小范老子的《岳阳楼记》中‘皓月千里,浮光跃金’一句。因为皮毛是浅金色的,有点贴近月色。而且还有浮光掠影这个成语。浮光不只是看着神骏,飞奔起来,也如光似电,没有一匹马能比得上。”

“真真是好名字。”赵隆听得心中发痒,很是焦躁的抓着脖子:“什么时候能打到大宛,抓个七八百匹回来!”

“迟早的事。其实一起回来的,还有一批河西良驹,虽然比不上浮光,但也近五尺了。”王舜臣突然眯起了眼睛,压低声线笑道,“除了这些马以外,小弟我还带了还有一批上等的胡女。头发也是如同纯金一般,眼睛则是跟青海一样的蓝。大食的风味绝不同于中国,不过腰和腿可是不得了,而且长相看久了,也会觉得很不错。”

“过去不是没尝过胡女,我眼窝子还没那么浅,秦州的私窠子里就有。”赵隆不屑道,“多换两匹好马就行了。冯四要在熙河路办赛马,这事你也知道的,我这边正却好马呢。”

“私窠子里的婊子怎么能跟在沙州安家的大食商人养的上等家妓比?相貌身段就不说了,她们可都是有内媚的!不要哥哥你可别后悔。”

提起美色,是男人都不会再正经八百的说话,赵隆连眼角都垂了起来,“内媚……那我可却之不恭了。”

“自家兄弟,客气什么?”王舜臣大笑着拍拍赵隆的背,“人和马待会儿小弟让人各送一对去赵大你那边。”

赵隆又谢了一句,笑道:“不过你就这么把人带回去,不怕家里面闹得天翻地覆?”

王舜臣娶得可是种家女,有着背景深厚的娘家,加上已经为王家生了两个儿子,在家中有着足够的发言权。王舜臣不事先说一句,就带着一群胡女回去,挨上一顿乱棒都有可能。

“设个私宅多大的事?”王舜臣嘴巴上说得浑不在意,但半句也不敢提把人带回家。只敢置办外宅。这气势上就小了许多。

“终究还是不敢带回家啊。平常天都不怕的,还怕家里的浑家。”赵隆哪能看不出来王舜臣的心虚,取笑了两句,又道:“要不要给龙图那边也送两个?”

王舜臣嬉笑道:“真送到三哥那里,还不给人怨死。我可不准备日后去做客,给乱棒打出来。我们哥几个自家消受好了……冯四和李二哥那边也备了礼,过几日都会使人送去。”

“听过这一次冯四派去的人帮了不少忙?”

“没错。不是领军西征,都不知道顺丰行的手伸得那么长了。冯四派到凉州的冯远,也是一等一的人才。凉州就不说了,甘州、沙州的豪门大族的名单、甘州回鹘聚集谋乱的消息,都是他通报的。”

“还真是不得了。顺丰行当真是越来越兴旺了。”

“是啊。”尽管王舜臣他每年从顺丰行都能收到大量分红,却没有太多的喜色,“不过太大了也不是好事,招人忌惮,而且也管不过来。就跟兵多不是好事一样。”

“龙图不会看不出来的,要不然也不会让冯四一开始就拉起一帮人一起赚钱,没说把钱都攥在手上。熙河路能有现在这般兴旺,蕃部忙时种田、闲时看球,一个比一个更老实,也是龙图留下的定策,何必瞎担心。”

“……说得也是。龙图的头脑可比我们好得多。”

赵隆笑道,“还不如说说内媚的是吧。不知内媚是怎么样,哪里特别?”

“尝过可就知道是怎么回事……”

王舜臣正要细说大食风情,却见一群人簇拥着两名紫服高官走进厅中。王舜臣脸上形容一变,向赵隆使了个眼色,便换上了一幅公事公办的神情。赵隆回头一看,却是王厚陪着王中正一起进来了。

