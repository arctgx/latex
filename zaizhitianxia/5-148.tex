\section{第15章 自是功成藏剑履(十)}

腊月的太原州衙,工作并不算很多。每年最重要的征收税赋的工作,都集中在夏秋二季,而往往在冬天兴起的工役,今年也因为有了黑山党项的关系,并没有受到战事的影响。

太皇太后丧期虽已过三七,但依然属于国丧之期,尽管民间燃放鞭炮不犯禁令,但由官府主持的一系列年终的仪式,还是不得不宣告暂停。

儒门重礼乐,在韩冈看来十分无谓的仪式,却一向绕不过去。少了这些繁文缛节,他乐得轻松。而且各项祭祀典礼之后,少不了宴会这一环。韩冈本来就不是喜欢饮宴作乐的性子,寇准那般日以继夜的饮酒宴客,实在是学不来。

只是韩冈除了太原知府的责任外,还有河东经略带来的工作。

国界划界的谈判地点已经确定设在在胜州和辽国东胜州之间,位于柳发川大营——最近被天子赐名做靖边寨——以北十里的一处小盆地中,正好是双方控制区重叠的地方。这是韩冈必须要关心的一桩大事。

在胜州之役结束后,边界大体的位置已经确定了下来。但每日巡检的探马之间,大大小小的冲突依然连日不断,也就终于确定了在胜州进行国界谈判之后,这样的纷争才告休止。

韩冈前两天才遣人去了一趟胜州,为韩缜领衔的划界使团的到来提前做好安排。

韩缜也算是倒霉,划界谈判的差事总是要落到他的头上,而且还是在过年的时候。上一次已经是为了天子担了罪名,这一次不论最后的结果如何,清流中肯定照样有人要说怪话。

韩冈将保卫划界使团的工作交给了第一任胜州知州。这位知州并不是从朝中派来,而是自河东军中挑选的一名老将。是韩冈很熟悉的人,也就是日前在代州担任知州的刘舜卿。

刘舜卿是宿将,不需要韩冈多叮嘱什么,自然会将事情办好——尽管他这一年来,一直都是镇守河东北疆,在胜州之役中,并没有出场的余地。

不过据韩冈推测,刘舜卿之所以能领到这个差事,也正是因为他没有参与胜州之役的缘故——胜州之役中,斩首最多的那些个将校,虽然封赏都不缺,但最后他们的差遣,许多都是调到了晋南诸州,而且赶在新年到来前将他们调走。而晋北,代州、胜州、火山军诸军州,都是换成了老成持重的将领,也就折家的老巢府州不方便更动。

当然,朝廷眼中的所谓老成持重,不过是颟顸迟钝的另一种说法。韩冈作为经略使,河东帅臣,在接见这群平均年龄接近六十岁的将领的时候,很是为未来几年的河东防务担心了好一阵。

朝廷这种担心边臣贪功兴兵的心思,韩冈勉强能够理解,但为此换上一群熬资历熬上来的老糊涂,那就完全不知所谓。还不知道守卫河东北疆的官军,会被他们糟践成什么样,当真是以为可以马放南山?

只是韩冈也没打算说什么,一方面,他现在不方便说,另一方面,至少几年内,的确不用担心会有什么大问题。近几年,辽人不可能在边境上有什么大动作,也就是闹些小摩擦罢了。但边境上的小摩擦若是无法坚决的给予回击,到时候,可就是要做好被得寸进尺的准备。韩冈可以很确定的说,天子赵顼肯定没有做好这样的心理准备。到时候,有的乐子看了。

抱着等着日后看热闹的心思,韩冈将边境军州的人事安排放到了脑后,而在太原做好了迎接划界使团的准备,再过几天,韩缜就该到了。但让韩冈感到惊讶的是,冯从义竟然就在这个时候,从东京到了太原这里,比韩缜早了一步。

“都快过年了,怎么不赶着回巩州。”韩冈很是意外,“年底不是关账的时候?不打算回家过年了?”

“三哥放心,小弟明天就回去。”冯从义道,“从太原向西过河,从葭芦川往银夏走,半个月不到就能回巩州了。至于关账,来得及赶上。”

“银夏刚刚收复,路上还不是很平靖,没事冒什么风险?”

冯从义哈哈大笑,“有三哥在,小弟还用担心道路上的安危不成?”

韩冈摇摇头,拿他没办法。不过从葭芦川走,转入无定河,然后再进入黄河谷地,一路虽是刚刚得到的新土地,但由于要跟辽人划界,已经控制得十分严密,走起来还是很方便。以冯从义的身份,加上他身边由广锐军后人和吐蕃人所组成的护卫,也的确不用担心安全问题。

“这么急着来太原是为了什么事?”韩冈问道,他不信冯从义没事会赶在过年前往太原这边跑。

比韩冈还小一点的冯从义,如今乃是关西商界举足轻重的豪商。以顺丰行等几个大商社为核心的雍商集团,不仅控制了国内八成的棉布市场,关西的诸多特产,也全数掌握在他们的手中。雍商比起浙闽的商人更为抱团,在商事上同进共退,就是在京城中,也是好大的声势。冯从义已经不是可以没事乱跑的身份了。

“有件东西想让三哥看上一看。”

冯从义神神秘秘的从袖口中掏出一个小布包,扎扎实实的裹了好几层,打开来,却是一片晶莹透明的镜片。

韩冈拈起镜片,从透镜中看着表弟变得夸张的脸,是凸透镜。从侧面看过去,透明的镜片便现出了墨绿色。

“这是玻璃?”韩冈问道。

玻璃过去俗称药玉、琉璃、颇黎,不过因为韩冈的缘故,玻璃之名已经渐渐成了世人认同的名称。

“三哥眼光如炬。”冯从义表情夸张的赞道。

韩冈没理会表弟的表演,拿着镜片对着桌上的书,测试着放大的效果,一边又问着:“是巩州的工坊产的?”

“怎么可能?!”冯从义摇头叹气,“在巩州的玻璃工坊想要造出透明的玻璃还早得很,只见钱砸下去,就没听个响。是京中的官坊终于造出了透明的玻璃,小弟设法买到了配料的方子,等回巩州,让那群只会造些废料的工匠们好好学一学。”

自从白水晶因为需要制作透镜而价格飞涨之后,能替代白水晶的透明玻璃,便成了研究的重点。眼下千里镜也被发明出来,对透明镜片的需要,更是上了一个台阶。要造出适合打磨而且没有气孔的玻璃镜片并不容易,但原料比起白水晶的比起来可是要便宜上百倍。

韩冈当初主管军器监,曾安排下对这个项目加以研究,投下的经费不在少数。之后几任接手韩冈工作的判军器监,由于种种原因,都选择了将这个项目给延续下去,研究经费也没有削减。累积起来已经有几万贯的投入,并不比修建高炉少到哪里。如今终于有了成绩,也不是多令人惊讶。毕竟透明玻璃在大食商人那里有现成的例子,而官营的药玉作坊也一直都有透明玻璃的出产,只是一直没有弄清楚其中的原理,无法大量生产而已。

从冯从义手上又接过一张纸片,韩冈粗粗一看,就看到上面写着白砂、硼砂、铅等名词,后面还跟着数量。

韩冈举着这张纸片:“买这份配方花了多少钱?”

冯从义比出两根手指,“两百贯。”

“朝廷在白玻璃上投进去的钱,两万贯也不止了。”韩冈摇头感叹着,“两百贯收买一个工匠,就拿到了两万贯的配方。这个生意,做得可真是值。”

冯从义却不觉得有什么问题:“等玻璃生产出来后,还不是要依律交税?朝廷一年拿到的税钱,只会比投入的钱要多。”

韩冈将配方还给冯从义:“光靠一个这么粗糙的配料秘方就想造出透明玻璃,哪有那么简单的事?”

“小弟也知道不容易。光是原料,就很麻烦了,雍地的土性和中原截然不同,就是炼出来的铁都有差别,玻璃岂能例外。但有了方向,相应的改起来也方便。再用个几年、再砸个几万贯,终究能出来的。”冯从义笑道:“而且小弟也没打算我们一家掏钱,好几家都想跟这个风了。”

“一家赚钱太惹人忌惮,的确是该跟棉布一样,不要拿在一人的手中。”韩冈点头表示赞同,接着又道“不过我也希望到时候不要想着一本万利的心思,我还想着什么时候将窗户纸换成透明的玻璃就好了。”

冯从义扭头看看厅中的窗户,咋舌道:“那还不知要多少年了。”

韩冈不心急,“十年二十年三十年我都有耐心等,一步步来就是了。”

冯从义又笑道:“有了玻璃之后,三哥过去所说的水银镜子也可以去造了。这可比玻璃更好赚。”

“说得也是。”

韩冈还记得当初造显微镜的反光镜,上面的锡汞合金接触空气后很快就雾化了。所以他一直想要用玻璃给蒙上表面。也曾拿着水银镜的好处来说服冯从义为此投入资金,这是可以看得见的好处。

