\section{第43章 百里河谷田一顷(下)}

【第三更,求红票,收藏】

窦舜卿的事已经让韩冈的火气发泄得差不多了,不会为李师中推卸责任这点小事生气。他明白李师中理所当然的要推卸责任,还要为前事找借口。他只是想不到李师中会用这种杀敌一千自损八百的战术,即便他的说法为朝堂采信,也少不得一个失察之罪。只是这个罪名可大可小,就看朝堂上有没有人保他。

……但李师中毕竟都是侍制级的高官了。

韩冈对北宋官制渐渐了解,清楚越是高品的清贵官员,越是受到优待。升到侍制,乘用的马鞍上已经可以缝上时称‘金线狨’的金丝猴皮,号为‘狨座’。这等天子近臣,即便降罪,过不了几日就会回复原官,这是仁宗朝留下来的规矩。仁宗皇帝庙号为‘仁’,就是因为他对臣子还有服侍在身边的宫人太好了的缘故,至于百姓嘛,在他统治天下的四十二年里,人丁增长不到一倍,赋税则涨了三倍,从这一点就可以知道了。

李师中即便被治了罪,也不用担心后路,窦舜卿其实也是一般,而王韶不同,他地位太低,只要一步错,便万劫不复,必须要为此辨出个真相来。韩冈与王韶是利益共同体,既然身在东京,没有不为他说话的道理。王安石必须立刻去见,而眼前的两名监察御史,也同样要派上用场:

“两位先生,韩冈不过一个判司簿尉,指证一路副都总管并不够资格。但窦舜卿实是罪在不赦,还请两位先生报于天子,由朝中及早挑选正直大臣,充作特使,去秦州当地查验明白。若王机宜妄言,自当入罪。若窦舜卿欺君,也当一体治罪。”

张戬和程颢心中本有些犹豫,现在中枢两府的宰执们都盯上了王韶,尤其是枢密院中的两位,皆想通过王韶去撼动他背后尚在称病中的王安石。这时逆势而动,非是智者所为,何况无论是从政见上,还是从故旧情分上,他们都没有理由为王安石说话。但如果只是让朝中派出使臣,却没有问题。这本是情理中事!两人都不希望天子和朝堂被地方欺瞒:

“当是要再派人的!”程颢点点头。

……………………

朝臣尽数退去的崇政殿中,赵顼狠狠地丢下一份奏章,紧接着又砸下来另一份。年轻的皇帝为臣子的欺骗而感到愤怒。

“王韶!窦舜卿!”他拍案怒吼。

在群臣面前赵顼要保持着天子的风仪,一直在强忍着怒意。一直等到快到傍晚,商议朝政的外臣尽数退去,繁琐的政务全数处理完毕,赵顼才不用再克制自己——从这一点看来,赵顼算是很尽职的皇帝。

两份截然不同的奏章摆在面前,赵顼不知道哪一份是真是假,但他很清楚,两个人中间必然有一个骗了他。

臣子既然敢说谎,就等于在说他好欺骗。这让赵顼难以忍受。不论是王韶,还是窦舜卿,他将两人放到各自的位置上时,都是考虑再考虑,生怕因为一点疏忽,而造成不可挽回的后果。但正事还没做,两人便斗了起来。李师中自身不正,前后奏报天差地别,却也做不了公正的评判。

从心底里说,赵顼想相信王韶,但他不能冒险,不敢冒险。一个错误的诏令,说不定就会造成一场惨痛得失败,使得边地战局十几年都补救不过来。

可宰执们的声音一面倒的支持窦舜卿,又使赵顼感到惊疑。他有理由怀疑枢密使文彦博、吕公弼,以及御史中丞吕公著三人的用心。万一王韶说得是实话呢?不相信他,可就要失去了一个开疆拓土的机会了。

权衡到最后,赵顼不自觉的又想起王安石。那位称病请辞的参知政事,在过去,总能给他以指点。刘备和诸葛亮是贤君名臣典范,而赵顼也一直都把王安石当成自己的诸葛丞相。

当初,王安石刚刚入朝,曾与赵顼谈起历朝历代的天子,王安石问赵顼最慕谁人?赵顼说是唐太宗。王安石则说,唐太宗何足论,当以尧舜为目标。

虽然王安石现在赌气回家,称病不朝。但赵顼的朝堂上,文武百官,济济一堂。又哪一个比得上王安石?

赵顼想做中兴之君,想踏平西北二虏,想成为真正的天下之主。这样的愿望,这样的想法,没有哪个老臣支持他。只有王安石说可以,说没问题,说一定可以做到。

只要变革法度,只要能坚持下去。

天下和老臣,哪个更重要?

这一瞬间,赵顼完全抛弃了韩琦。不值得为了他,而让大宋的革新大业停下脚步,畏缩不前。朝堂需要的是王安石,不是韩琦。

赵顼唤来李舜举,递给他一份亲手写的诏书:“你再去王安石府上一趟,让王卿家快点回来。他不是气韩琦的奏章吗?朕会把奏章发回中书门下,任他一条条的批驳,刊在堂报上也没问题!让他快点回来!”

……………………

“臣遵旨!”

声音入耳,李舜举点了点头,又叹了口气。

这是他今天第二次来王安石府邸了,而对着躺在病榻上的王安石宣诏更是不知累计了多少次。李舜举当发现自己用十根手指都数不完来王府次数的时候,也不准备脱掉靴子加上脚趾去计算了。

‘都已经逼着官家道歉,真不知道王大参还要赌气道什么时候?’李舜举叹着气,就想收拾东西走人。

等等!李舜举动作突然停顿,方才王安石说了什么?

遵旨?!

他抬眼看着前面王安石的病榻,却见王安石的次子王旁走过来,说道:“近日多劳都知,家父今日病势稍可,已经能起身了。”

李舜举在宫中待了许久,精于察言观色,更是会听话。听出王旁是在赶人,王安石要起床更衣了。虽然这让李舜举的自尊心有点小小的受伤,但只要王安石肯奉召,省得他一跑再跑,难道还有别的奢求吗?

李舜举留下诏书,识趣的告辞:“请转告大参,官家正在崇政殿翘首以待,勿令官家久候。”

“都知放心,家父既然痊愈,当然会尽早入宫谢恩。”

王厚送了李舜举出门,等他回来时,王安石也起来了,刚刚换了一身朝服,头戴长脚幞头,身着紫袍,腰缠御仙花带,带上系着金鱼袋。他称病多日,气色反而好了不少,一副体壮如牛的模样。

天子终于肯服软,又让李舜举传口诏,允许他将韩琦的奏章带去中书,逐条批驳,并用堂报通传天下。天子都做到这一步了,一切目的都已达成,也没必要再继续躺在病榻上装病了。

“大人,你现在要入宫?”王旁追在一边问道,现在已经是申时了,天色已经沉了下来。再过一个多时辰,宫城、皇城就要落锁,现在入宫,时间太赶了,“何必赶在今日?”

“为父是去请罪。当然越早越好!”王安石的脾气虽然犟起来,九头牛都拉不回来,甚至敢于不给皇帝面子,乃是号为拗相公的任务。但他久历宦海,政治头脑还是有的。有来有往才是礼,天子让步了,自己也得有所回报,不能一傲到底。

“把吕吉甫、曾子宣和章子厚一起请来。等为父回来,有事找他们商议。”王安石向外走着,又嘱咐了一句,王旁点头应是。

吕惠卿、曾布、章惇三人都是变法派的主将,王安石的得力助手。他们掌管三司条例司和中书检正公事,这两个机构和职位,都是为了让官品和资历不高的变法派成员能掌控朝廷的财权和政务,而特意量身定制。设立时间还不到两年。依靠两个新机构,变法派在实质上控制了主管天下财计的三司,并能暗中左右着政事堂。

只是王安石称病这么多日子,为防议论,并没有见过吕惠卿、曾布还有章惇这些得力助手,等于断绝了与朝堂的联系——这是此时不成文的潜规则,你可以称病,虽然谁都知道是装的,但没有人会挑明了说出来。不过毫无顾忌的肆意会客,那就是不打自招,欺君的罪名便定了。即便赵顼不治罪,心里肯定芥蒂更深。

另一方面,王安石由于不能去政事堂理事,对地方上的局势也失去了控制,甚至不清楚发展到什么地步。青苗法、均输法和农田利害条约的最新推行情况,他也必须重新掌握。

还有边境上的战局,无论是横山还是秦州,两地的最新变化,王安石也都懵然不知,也就刚刚收到的一封私信,让他心中才稍稍有了一点谱。

政治、经济、军事,仅仅是参知政事的王安石,对大宋政局的影响是全方位的。而他称病不朝所带来的后果,也是全方位的,对此王安石也很清楚。但他相信,只要博得了天子的支持,一切问题都不是问题。

赵顼最终的选择,使变法派没有了后顾之忧。连最老资格、立有异勋的元老大臣韩琦都被天子放弃了,还有谁能阻止变法的进行?

“对了,还有这个。”王安石翻手拿出一张名帖,“你遣人去城南驿,让他明天过来。”

王旁低头看着名帖,上面的名字十分的陌生:“韩冈?”

王安石点点头。夹在名帖中的王韶私信,他已经看过了。近万字的信笺中,除了述说秦州局势,以及新的计划之外,都是对韩冈的夸赞。这让本已经因为举荐之事,而关注起韩冈的王安石更加好奇,越发的想亲眼见上韩冈一面。看看被王韶如此夸赞的年轻人,究竟是个什么样的人物。

“孩儿知道了。”

“等等……”王安石叫住了正要出去的儿子,“还是让他今晚过来。”

王安石是个急性子,不喜欢拖事。另一方面是吕惠卿对秦州发来中书门下,由韩冈编写的伤病营管理暂行条例赞不绝口,直叹是难得的治才,当时他便说要见一见韩冈。今晚王安石有许多近日耽搁下来的事情要与几位助手商讨,其中当然也少不了关于河湟之事,正好叫韩冈过来了解一下,用不着拖到明天了。

王旁愣了一下,虽然不清楚为什么,还是点头应了,自去唤人去城南驿请韩冈。

