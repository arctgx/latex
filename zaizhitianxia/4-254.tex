\section{第43章 庙堂垂衣天宇泰(一)}

【第一更。】

入冬后难得的艳阳天,阳光洒在院中,洒在一株合抱粗的老桂上,也洒在了周南和几个被抱出来晒太阳的孩子身上。

家里的仆妇忙忙碌碌,趁着天好将一床床被褥抱出去晾晒。周南最近又有喜了,才两个月,正是不安稳的时候,不能累着,在府中担的那份事都交给了云娘去做,自己则是安心的养胎。

年纪稍长的三个儿女,都已经开蒙了,每天上午都去外院的西书房跟着西席先生认字。但三个小的才两岁多,被乳母、使女,放在院中嬉闹着。

三个小家伙歪歪倒倒的跑着,追逐着一个充了气的皮球,将柚子一般大小的皮球踢过来踢过去,笑声响彻后院中。一个个脸上都是红扑扑的,长得健康壮实。

家里的三个孩子闹得满头大汗,负责看管他们的周南却是安安静静看着书。成了两个孩子的母亲后,原本就是花中魁首的周南,就如同熟透了的果实,更加艳丽不可方物。温煦的阳光照下来,肌肤如玉一般莹润,一对眸子黝黑晶亮,半个身子斜倚在扶手上,静静的翻着书。娴雅柔美的姿态,就是丹青圣手也难以描画。在旁服侍的使女,都时不时的向周南瞟上一眼,就连女子都忍不住被吸引。

周南看着的还是出自自家的书,不同于应急手册,名为《桂窗丛谈》的笔记在家中更受欢迎。柔软的洪州纸装订起来的抄本,软绵绵的,拿在手中很是舒服。

韩冈的这一部笔记在家里也就几名妻妾事前读过。直到三天前才给沈括送去一部抄本。除了韩家人,没人知道韩冈用功了半年,才将两本书给完成;家里的门客,同样是这几天才知道年轻的龙图学士写了两部新书。

“夫人!”

“娘!”

院中的喧闹突然停了,然后就是一片的问候声。听到声音,周南放下书连忙起身,向着当家主母屈膝行礼:“姐姐来了。”

“快坐。”王旖连忙搀着周南,嗔怪道:“都双身子的人了,动作要轻一点。”

周南展颜微笑:“劳姐姐关心,小妹知道了。”

早有婆子端了交椅过来,周南这才同王旖一起坐下,几个小孩子行过礼后,见王旖没有别的吩咐,就又开始闹了起来。

王旖和周南看得相视而笑,周南道:“五哥儿的伤一好了,就活泼起来了。”

“既然是个哥儿,就该多摔打摔打,一点小苦头都吃不了,日后怎么帮着父兄支撑门楣。前些日子哭成那样,三哥、四哥加起来都没他哭得狠。”

“年纪还小,大了就好。”周南笑道:“二哥就不错,读书习武时都没叫过苦,大哥儿可比不上。”

王旖笑了笑,拿起周南放下的书翻了翻:“怎么又看起来了?”

“姐姐都说了官人写得这本书深不可测,所以想再看一遍。前面囫囵吞枣的,也没看出个眉目来,这一次要细细的读。”

前些日子刚刚拿到新书的时候,周南废寝忘食的用了两天就将一部十卷的笔记通看了一遍,回头就说整部书有意思。王旖则说‘官人的这部《桂窗丛谈》,闲暇时翻一翻也的确是很有些意思,但如果静下心去琢磨,却越琢磨越觉得深不可测。’

周南在《桂窗丛谈》中,看到了天南地北的风土人情,看到了鸡兔同笼的另一种解法,看到了对花鸟蛇虫别出心裁的分类,看到了码头上滑轮省力的原理,看到了点石成金的骗术被拆穿,在她的眼里,这代表着韩冈的博学,还有在格物致知上的成就。但她没想到,王旖对她们的丈夫所写下的这部书,竟然下了深不可测的评语。

刚刚拿到韩冈所撰写的笔记的时候,自家是当做闲书来读。虽然周南是明白自家的丈夫写书都是有一份深意在——就像当年写下《浮力追源》,让人误以为是要造铁船,实际上则是拿出了飞船,同时还促进了甲胄的制造,以及钢铁业的发展——不过周南认为韩冈的想法自己应该都知道了。可王旖却说没那么简单。

以见识论,素心和云娘是远远不如在京城中长大的周南,不过周南也只是在琴棋书画和器乐歌舞上有所擅长,作诗作词能跟一家之主一较高下。说到学识,周南不敢与宰相家的女儿相比,相信了七八分。

拿着丈夫的著作,王旖就手翻着。她在这本书里面看到的是一个庞大的学术体系,涉及到天地自然的方方面面,笔记十卷,只是露在外面的引子,实际蕴含的内容并不是区区十数万字能够囊括。

甚至连冰山一角都算不上,冰山露出水面的还能有十分之一,而韩冈摆出来的只有百分之一——就在《浮力追源》中,韩冈通过水和冰密度的比较,明确的阐述了冰浮在水中的原因,甚至浮出水面的比例。这两年越来越多的人知道看到水面上的浮冰,水底下暗藏的流冰九倍于水上部分。

将本心层层遮掩,就如一道千门万户的迷宫,在里面走起来移步换景,永远只能看到一部分,而不见全貌,就是最后看起来是揭开谜底了,但在没人注意的地方,却是还有几处伏笔潜藏,这才符合她丈夫的为人和性格。

就如轨道。

轨道先使用在码头上,但铁矿的矿山中才是轨道用得最多的地方。天下各大矿山,逐渐推广了轨道的使用,也培养出了一批合格的匠师,为方城山的轨道做好了准备。而方城山的轨道,听说生铁的用量动辄以万斤计,若是没有之前韩冈推动钢铁产量的发展,根本就造不出来。

现在方城山轨道成功投入使用,当河北轨道提上台面之后,国中对钢铁的需求又会上一个台阶,那一座座高炉,就又有了派上大用的地方。

自家夫君做的每一件事,光是拿到台面上的,已经是足以吸引所有人的目光,但再往下发展,却能发现下面还藏着更多也更让人惊讶的东西。

王旖和周南沉默的翻着书,就听得院中扑通一声,韩家的老五在追着皮球的时候一脚踢空,仰天栽倒。

王旖和周南就在旁边看着儿子跌倒了,并不去扶,倒是三哥四哥跑了过去要搀扶。而五哥儿不哭不闹,更不要人扶,一骨碌就爬将起来,跑到他乳母那里摊开小手。乳母忙掏出一粒半透明的冰糖来,看着眼前一只脏脏的小手,就直接给五哥儿塞进嘴里。

“官人说话也促狭。”看到了这一幕,王旖一下笑了,也是韩冈的要求,家里的几个儿子除了刚学走路的时候,跌倒了要扶一把,大一点之后都让他们自己爬起来,哭得再凶都不理会,最多拿块糖来逗着站起来,“记得早前还说呢。教训小孩子,就跟训猫训狗一样,做得对了该夸就夸,该奖就奖,几次下来就知道该怎么做了。”

周南也扑哧笑起来,“当初就是二哥儿最聪明,那时候故意往平地上栽跟头,骗了多少吃的。”

“其实道理是不错。”王旖嘴角翘起微笑着,视线追逐着又开始玩闹的儿子们,“你越是一惊一乍,小孩子哭得就越凶,你不去理会了,反而自己就爬起来了。”

“姐姐说的正是。”周南点着头。正说着,就看到老三也摔倒了,同样是自己爬起来,同样是跑到乳母那里伸手要糖,拿到后就往嘴里塞。

王旖连忙叫着:“三哥儿,糖不能多吃,牙齿坏了可没法儿治。”

周南失声笑道:“真该去问问素心,家里的冰糖还剩多少斤了,不知还够不够他们讨的。”

“上个月从交州送到的有三十斤冰糖,两百斤白糖,三百斤红糖,还有各色蜜饯五百五十斤。到手我就让素心安排人各送两斤蜜饯去给东偏院的那十几位,在北面的方、李二位,也派人送去了,等到过年还要给。至于年礼,走外院的帐,到时候还要跟官人商量。”

韩家内院之中是王旖总掌,几名妾室各管一摊,周南现在养胎,家中事袖手不理,也不多谈此事。转问道:“听说襄州的铺子里面也有白糖和冰糖卖了。”

“这就不知道了,不过当真有卖也是好事。”王旖道,“官人昨儿也说了,派去交州的人都很用心,今年就有交州米在杭州上市了,等到白糖也一并上市,交州就能安定下来了……自家能不能赚到钱倒是小事,开辟了一个产业则是利国利民的大事。”

“官人真是越来越大方了,孔方兄都不放在眼里。”周南虽然是开玩笑,言语间却满是骄傲。

“有了出产,就有了税赋。有了税赋,也就能让禁军在当地驻泊、就食。那一片新疆土就能安定了,不会再有朝臣说什么无用之地空耗钱粮。而官人在这基础上,还能做到公私两便,说到治政之才,官人在朝中也是首屈一指的。”

不是视钱财如粪土,家里的浑家孩子饿得发慌,还能弹琴唱歌的自命清高,君子爱财取之有道,他们的丈夫从来都是为边地开辟一项产业,拉着多少家一起进来,让刚刚攻占的新土由此安稳,而他作为开创者,就只在其中占上一小份而已。

说道视钱财如等闲,这个才是真的。

