\section{第11章 诛心惑神幻真伪(上)}

丢下三具尸体,韩冈回到屋中,换上了另一架上好弦的弩弓,又从桌上拿起一个小布包,快步小屋中出来。他看了看大门处,仍没有什么动静,看起来王五、王九两人还未被惊动的样子。

韩冈方才射杀的三人,都是没能发出一声惨叫便告毙命。这可以说是韩冈的运气,但也是两名守兵的运气,不然他们同样是刘三等人的下场。杀三人是杀,杀五人也是杀,性命攸关,韩冈绝不会手下留情。

韩冈从容不迫的回到三人的尸身旁,先打开小布包,从里面掏了两下,掏出一套引火的火刀火石和火绒来。他看着手掌上的三个小器物,笑得越发的阴冷。韩冈蹲了下来,将手探进刘三的怀里。突然脸色一变,手上一顿,再抽出来时,掌心中却多了一个火折子!

火折子是用白薯藤特制,点燃后吹灭,但火星依然在其中阴燃,要用时只需迎风一晃就能再次燃起。这等特制的引火物能把火种保持一天之久。为什么刘三要随身带着引火的东西,火折子的价格可不便宜!韩冈心中有些觉得不对劲了,连忙搜查了另外两名衙役的怀里。果然,又给他摸出了两个火折子。

此时月色如水,清辉洒满地面,庭院中亮堂堂的,可以很清楚的看见刘三三人腰间都系了个大葫芦。韩冈探手摸了一摸,手上滑腻腻的,像是还未干的血。但他再凑鼻一嗅,却是菜油的味道。

怀中藏火,腰间藏油,刘三三人想做何事不问可知。

“该不会是英雄所见略同罢!”

韩冈只觉得今天遇上了天下间最为荒谬的一桩事,只想狂笑出来。都是想栽赃,却没想到想栽给对方的,竟然是同样的罪名。有什么罪名能比得上火烧军器库?!他和黄大瘤想的都是一般无二!

‘不,不可能是黄德用黄大瘤。’韩冈突然摇头。

黄大瘤决没有这等魄力,也没有这个需要。他有理由杀自己,但绝没能力用上这等过火的手段。如果是烧一点不重要的东西来陷害,用个火折子就够了;三葫芦的油足足有四五斤,用来引火,整间军器库都要烧通了顶。也不可能是陈举想杀自己,以陈举的势力,哪里需要用一间军库为一个穷酸措大陪葬?一句话就能让韩冈死的不明不白。

那刘三死前说的‘陈’又是什么意思?除了陈举还能是谁?

韩冈的脑筋飞速转动,很快一点灵光闪现——如果真正的目标不是他呢?

主使者必是陈举无疑,这点完全可以确定,他人绝没这等胆量和能力。但对付他韩冈应该只是附带,陈举的目标肯定是这座军器库。要烧库房,理由韩冈也能猜个八九不离十。这样的例子,故事中、现实中,还有在他的记忆中,绝不算少。何况,近三十年来,成纪县衙不是烧过三次吗?

纵火焚烧官衙府库,这并非什么骇人听闻的奇事。莫说胥吏放火灭罪证,据韩冈所知,几十年前就连知州放火都是有过的!

知州放火烧去账册毁灭罪证,韩冈都知道的事,在关西也不是秘密。其主角是便是岳阳楼的建造者,范文正公【范仲淹】的好友滕宗亮滕子京。范文正的《岳阳楼记》传之千古,大大的有名。而下令建造岳阳楼的滕子京,在关西也是大大的有名。他在泾州知州的任上,耗用公使钱无数。当事情被揭发,朝中派出监察御史要检查他的公使钱帐册的时候,他也不废话,一把火把账册烧了精光。

‘你不是要帐册吗?诺,那堆灰就是。’

尚幸国朝一向优待士大夫,而仁宗皇帝尤甚。做出了这等事,滕宗亮不但保住了性命,还能继续担任知州,只不过地方换成了岳州罢了。一句‘先天下之忧而忧,后天下之乐而乐’之所以能出现在历史中,也正是因为他的一把火的缘故。

除了滕宗亮这位知州放火外,还有一桩闹得更大的。真宗朝时,八大王赵元俨——也就是民间传说中的八贤王——的侍婢韩氏因为偷了几两金器,为防败露,一把火烧了荣王府不说,火势蔓延,连带着把左藏库、朝元门、崇文院、密阁一起付之一炬。

王府倒也罢了,但崇文院和密阁中,可是珍藏着从唐朝、五代开始,直到宋代的各色孤本珍本的书籍,以及历代诏书、奏疏等重要历史资料,可以说是皇家图书馆兼档案馆。还有左藏库,那是直属于天子的内库,里面是太祖、太宗两代的积蓄,足有数千万贯之多。可就因为几两金子,便一股脑成了灰烬。

至于胥吏放火,那就更多了,不胜枚举。为了掩饰罪行,把证据一把火烧掉的事,在此时常见得算不上话题。宋代的建筑九成九以上都是土木结构,只要一把火,那就是白茫茫的大地真干净,最多最多事先要找个替死鬼顶罪就成了。

如此一想,一切都说通了。作为预定中的替死鬼,韩冈忍不住低低骂了一句:“娘的,真是赶巧了。”

想通了一切,韩冈心如电转,转眼便有了定计。返身回屋,从墙上取下一支号角——这是库房出事时才可吹响的警号——仍旧提着重弩出了门去。只是他刚出门,便止步立定不动。

在韩冈眼前,一盏灯笼从大门处飘了过来,灯笼后面的,正是守门的库兵王五、王九。

王五和王九本是要给放火的刘三几人望风。按照户曹刘书办的说法,纵然军器库遭焚,陈举照样能保住他们。只要把罪名推给倒霉的韩秀才,最多在狱中待上半月,而酬劳足以让他们过上两三年的快活日子。两人的心中都有些不情不愿,可陈举的话他们也不敢不听。今夜王五、王九只得依命行事,但刘三进去了半天,却再也没有动静。两人心中慌得厉害,都觉得有些不对,才打着灯笼过来查看。

可这一看,只吓得两人魂飞魄散。灯笼和明月一起照着地上的三具尸身。刘三等人脸上残留着的惊恐,莫名的传到了王九、王五的心中。而明显是凶手的韩冈,正站在小屋门口从容的看着他们。

韩冈高大的身材如劲松一般挺直,依然是白天时的平和淡定,但站在三具尸身旁边,如何还能是同样的神情?!

“韩三,你做了什么?!”王九纵是大叫着,也驱不散缠绕在心头的寒意。而王五执着灯笼的手,更是不断在抖着。

韩冈冷笑不答,只把号角凑在了唇边。在两人惊骇欲绝的目光中,他使足了气力,将警号用力吹响。不同于内地的城市,每日城内暮鼓敲响后,秦州城的街巷上便开始宵禁。寂静的城市夜晚,一声凄厉的警号击碎了人们的睡梦,许多人纷纷从床上爬起,巡城的甲骑也收缰停步,衙门里值夜的官吏则从房中冲出,多少人竖起耳朵静静聆听,以判断警号声的来处。

号角声一连响了三声,方才缓缓收止,只留着袅袅余音回荡在深秋的寒夜之中。

王九不住的发抖,浑身的热量都给那几声号角吹散,几乎语不成声:“韩三,你知道你做了什么!?”

“看不出来吗?此三人夜入军库,谋图纵火,给我……杀了!”短短的一句话,韩冈却拖得很慢,最后两字又用重音用力吐出。一支上好弦的重弩拿在手中,为他的话助阵。两名库兵只觉得浓浓杀气从韩冈处扑面而来,阴寒刺骨,如坠冰窟。

“胡说,他们……他们……”王五‘他们’了半天,终于想起刘三进来前的说笑:“他们是来请你喝酒的!”

韩冈一声冷笑,连驳斥都不屑:“无故夜入人家者,杀之勿论。何况无故夜入军库?!此三人入库有军令否?!有号牌否?!又身携火种和油水,不知是意欲何为?!”他笑容越发的阴冷,“只可惜了两位王兄弟,倒要为他们一起陪葬!”

“这……这与我们何干?!”王九结结巴巴的说着。

“刘三他们从大门进来,你二人肯定是逃不了同谋之嫌。结伙入军库,不是偷盗,便是放火。而他们人人身携火种火油,除了放火还能作甚?”

韩冈轻轻踏前,落地无声,却如重鼓一击,吓得两人连退数步。韩冈也不看他们,自顾自的绕着刘三三人的尸身踱起步,竟还是读书人特有的方规矩步,自如的仿佛在苦吟诗句。但从他口中出来的,不是吟风赞月的诗词,而是一句句如剑如刀的质问:

“你们想想,若是库中失火,你等库兵真能逃得过罪责?

我肯定是一死百了,但你们呢?

陈举再大,也大不过国法,凭他一个小小的县中押司,能保下你们俩?!

也许他事先跟你二人说过,最多挨上几下军棍,在狱中关上两月就没事了。但他的话真的能信吗?恐怕你们只要住上一晚,恐怕就要被病死了!

杀人灭口,陈举是做不出?!还是想不到?!”

ps:先下手为强,后下手遭殃。同样的计策,先下手的先赢。求红票,收藏。

