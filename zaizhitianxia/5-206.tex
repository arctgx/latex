\section{第23章 弭患销祸知何补(14)}

赵顼的话登时在殿上引起了一阵波澜。 

‘耶律阿果死了?’韩冈也忍不住瞥了赵顼一眼,看来之前的延误,应该就是这个消息造成的。 

御史们并没有继续他们预定好的行动,甚至连朝堂上微启的波澜也没有去理会——连他们也一并被这个消息给怔住了。 

小名阿果,大名耶律延禧的辽国幼主,虽是名义上一个幅员万里的大国之君,可也只比六皇子赵佣大一点而已,什么都不懂的幼童。 

早在耶律乙辛弑旧君、立幼主的消息刚刚传来的时候,所有人都知道,迟早有一天,耶律乙辛是要再下杀手的。 

他可是在将辽宣宗耶律洪基从飞船上推下来之前,就已经杀了废太子耶律浚夫妻。父母之仇不共戴天,耶律乙辛怎么敢让小皇帝活到通人事的时候? 

而且耶律浚和太子妃萧氏一直都没有被追尊——天子的父母不仅没有帝号,而且还是谋逆的罪囚,这件事太过匪夷所思,甚至可以说是荒谬绝伦,传出去连藩属国都要笑掉大牙,但辽国硬是做出来了。从这一点上看,耶律乙辛再次弑君的图谋从来就没有隐瞒过,不论是对内还是对外,连一点掩饰都没打算去做。 

不过耶律乙辛什么时候动手,还是众说纷纭。在所谓的宣宗遗腹子出生后,就有传言,但没有动静。等到辽国在大宋的平夏之役中捡了大便宜后,又有传言,但还是没有动静。谁料到时隔一载,他便悄无声息的将天下人翘首以待的事情给做圆满了。 

赵顼说辽人遣使告哀,那么辽国告哀使肯定是已经抵达了雄州边关,消息是从雄州遣急脚递发急报传回来的。而在此之前,安插在辽国境内的那么多细作却一点消息都没有打探到,或者说打探到了,却没有来得及传回来。 

这件事,要么是耶律乙辛对辽国的控制已经到了天网恢恢疏而不漏的水平上;要么就是他早有准备,幼主一咽气,便按部就班的通知各方,一点也没有耽搁时间。当然,更大的可能还是两者兼而有之。 

且不论如何,弑君之举都是一件骇人听闻的大事,而接连弑君的窃国大盗耶律乙辛,在大宋君臣们眼中,王莽还要输他一筹两筹,宇文护也得屈居其下。 

可以看得出来,赵顼眼下很是兴奋,要不然也不会公然的在朝会上向群臣公示,或许是认为撞上了攻打辽国的大好时机,最少最少也能趁机在辽人手里揩下一点油水来,不会一点便宜也占不到。 

朝会匆匆结束,给赵顼这么一打断,连原本可能正准备跳出来的御史,也不得不暂时鸣金收兵,静待来日——很可能短时间内不会再有机会,因为冬至的祭天大典就在十天后,而皇子出阁,资善堂重开,也就在十天后。 

等到韩冈正式成为资善堂侍讲,再想找韩冈麻烦,不会有任何意义,甚至有可能会导致天子直接出手清洗整个御史台——不能体会主人的心意,只会让主人不痛快,当然就不是条好狗,只有做成狗肉暖锅的下场。 

想到暖锅,韩冈顿时就感到有些饿了。今天的朝会结束的算是早了,但三更天起来,到了现在的辰正三刻,已经是空空如也。又是天寒地冻的时候,若有个热滚滚的狗肉暖锅放在面前,可会是让人想想就垂涎三尺。 

所谓暖锅,其实就是火锅。京城之中,一到冬天,暖锅的生意就好了起来,羊肉、狗肉是最受欢迎的涮锅材料,而各色特制的酱料,更是各大酒楼的不传秘方。只不过这个时代的暖锅,就是在一个小火炉上放个小圆锅,却没有后世常见的中间有个烟囱的紫铜木炭火锅——铜料毕竟不便宜,而韩冈这些年来也没去在食器上留心过。但现在肚中空空,韩冈便越发的怀念起涮羊肉蘸了芝麻酱后的味道。 

回到太常寺,下面早奉上了点饥的点心。衙中多是积年老吏,自然知道每月两次的朔望朝会究竟有多折腾人,又要怎样才能讨上官的欢心。 

满足了口腹之欲,喝着滚热的甘草饮子,狗肉暖锅、羊肉暖锅什么的就给韩冈丢到了一边去了,忙碌起每日都不能耽搁的正经事来。 

太常寺的日常事务,没有花费韩冈太多的时间,大约一刻钟的样子。即便是南郊在即,在圜丘现场检查细节、拾遗补缺的也是太常礼院的工作。而教坊司选拔乐班和跳八侑舞的人选,也直接向中书门下的礼房负责,将太常寺甩到一边。闲着干领俸禄的局面很是可悲。

而太医局和厚生司的事务就多了不少,在经过了对球赛惨案的伤者的救治,两座医院的急救水平整整上了一个台阶,在外伤医疗上,也有了更为响亮的名声。但随着医院的名声渐广,加上韩冈本人和一众御医的名头,使得越来越多的病患选择来医院求治,已经渐渐到了极限。而附属于医院的官药局,占了这份光,生意也越发的兴隆。除去翰林医官和医学生们的门诊津贴,如今每个月依然能给厚生司带来一千多贯的利润。 

不过韩冈暂时并没有打算扩大规模,甚至将见钱眼开的政事堂的要求给顶了回去。一来,两座医院已经抢了很多生意,京城中的悬壶济世的医者,以及大大小小的药局都是要吃饭的,不能尽砸人饭碗。二来,合格的人手不足,韩冈并没有自砸招牌的打算。 

眼下韩冈正考虑将医学规范化,并扩大规模,加强医学生的培养,以满足世人对医疗的需要。 

此时的医学被分为九科——大方脉科、小方脉科、风科、产科、眼科、疮肿科、口齿兼咽喉科、金镞兼书禁科、金镞兼伤折科。 

韩冈打算改变现有的分科方法,使之更加贴近后世。有一小部分原因是后世的医科分类经过了更多的实践验证,说起来应该是要比这个时代的划分更为合理一点,但更多的,还是因为韩冈觉得这样更为习惯,作为统管天下杏林的重臣,又是世所公认的药王弟子,韩冈有资格在这方面随心所欲。 

其中的大方脉科和风科,韩冈准备合并为内科。疮肿科和金簇兼伤折科,则合并为外科。眼科、口齿及咽喉科,合为五官科。又名小儿科的小方脉科依然**为儿科。 

产科当然也不需要变,但由于男女有别,产科的医官能实际上阵的机会并不多。在妇人生产上,稳婆才是最主要的力量,韩冈打算组织京中的稳婆统一参加产科的培训。 

至于最后的书禁科,也就是祝由科,黄帝时与岐黄并称的医科,眼下主要是以烧符水治病,以及各色禁术、咒法,但韩冈经过深入了解后,发现其中也有心理医疗的成分,不能简单的视为迷信,故而暂且留存不动——韩冈很清楚,即便是安慰剂,其实也不是不能治病。 

重新划分医科,更重要的是加强对医学的研究,毕竟韩冈主张开设医院,还是以研究医学、锻炼医术、培养医师为主要目的。内科的研究一时无从说起,不过外科、五官科乃至妇科,人体解剖学都是其中的核心。如何开展这方面的研究,便是韩冈眼下几个要操心的问题之一。 

说到加强医学研究,韩冈对普及自然科学的兴趣更大,也更为迫切。既然报纸已经开始普及,以《蹴鞠快报》为首的各色小报越来越多,那么学术期刊理所当然的也可以提上台面。 

对自然科学有兴趣的士人为数甚多,能潜心研究的同样不少,只看多少宗室、贵胄和衙内在摆弄显微镜和望远镜,就可以了解一二。眼下如果能有一个自然科学协会,有一本附属于协会的期刊,将这些人给组织起来进行专项研究,同时给他们一个互相交流的平台,必然自然科学会有更快的发展,同时,对气学的推广,也会有极大的好处。 

对于此事,韩冈也有了全盘的规划,等待一个合适的时机就会开始运作。短时间内他不可能在朝堂上有何进步,那么当然要将多余的精力和时间放到学术上。资善堂直讲,太子之师,这可是气学最好的护身符。至于新学,就让王安石领着他们去研究甲骨文好了,改变世界将会是自然科学,这一点,绝不会因人心而转移——区区螳臂,如何挡车?韩冈就有这样的信心。 

在公务中忙碌了两个时辰,快到中午的时候,韩冈方才从案牍上抬起头来。堆放在桌案上的一尺多高的公文一扫而空。瞧着被清空的桌面,韩冈也不禁悠闲的伸了个懒腰。 

到了这时候,辽国幼主病夭的消息已经传遍了皇城之中,两府重臣在崇政殿中已经为此事与天子商议了一个上午。 

不过中午会食时,从衙中的几个消息灵通的包打听那里,除了幼主病夭一事以外,韩冈并没有得到了更进一步的详情,跟往日并不一样。看起来从雄州传来的消息,的确就只有这么一点。 

不知这位刚刚驾崩的辽国小皇帝,能从耶律乙辛那里得到什么样的庙号? 

韩冈吃完饭后,一边在院中散着步,一边想着。 

想必耶律乙辛这个必然会名留史册的窃国权奸,肯定不会吝啬一个好听一点的称号。就跟从飞船上摔下来的辽宣宗耶律洪基一样——‘宣’这个字,在谥法中,可是一个很不错的字眼。怎么看,耶律洪基都当不起‘施而不私、善闻周达、诚意见外、圣善周闻’这几条评价。 

不过话说回来,对于耶律乙辛来讲,给他手下的冤魂什么样的名号,都仅仅是妆点门面而已。关心的人不会太多,韩冈也只是当做饭后消失的头脑运动而已。 

这个世上,应当是‘耶律乙辛到底会在什么时候谋朝篡位’的这件事,在意的人要多得多。从穷迭剌之子,到控制大辽的权臣,日后还有可能成为皇帝,这番际遇和经历,说起来,还真是有点让人羡慕呢。 

