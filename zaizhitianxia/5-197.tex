\section{第23章 弭患销祸知何补(五)}

出了崇政殿殿门,张商英脚下不徐不急的矩步方行,一路往御史台的方向走去。

略显轻快的脚步,悄悄透出了新任的殿中侍御史心中的欣喜。

就算是贵如学士、直学,能独对崇政殿的朝官也是为数寥寥。而今天,就在方才,张商英却正是在单独一人向天子奏禀,对蹴鞠联赛聚众过多,乃至成为致乱之源,进行抨击。并且更着重强调了联赛中的公然赌博行为,败坏风俗,有伤教化。

虽说张商英并没有能让天子就此做出永禁蹴鞠联赛,并对有伤风化的指责表示赞同。但今日皇帝肯让他独对,就已经是对他最好的鼓励了。

一抹浅笑忍不住浮上张商英的嘴角。

难道天子事前会不知道他张天觉准备在廷对上说什么?在事先就已经心知肚明的情况下,还让自家单独进了崇政殿,等于是向外贴布告了。

由此一来,韩冈想要安安稳稳的进资善堂,当然又要难上几分——谁让他没事弄出蹴鞠和赛马两项联赛的?是自作孽!

张商英今日说的的确是蹴鞠联赛聚众致乱,可他本质上的目的,还是阻止韩冈晋入资善堂,教授皇子。更确切点,就是不想让他成为未来天子的师傅。否则,成为潜邸之臣的韩冈日后必然将会手挽朝堂大权。新学免不了会被气学取代,新法也会在他的手上变得面目全非

——尽管后面两条,张商英其实是不在乎的,但他相信,肯定有很多人在乎这一点。这对张商英来说,出手的理由已经足够了。

无论是在关中的吕惠卿,还是在金陵的王安石,想必都不会愿意看到这样的结局——即便是以翁婿之亲,当也是忍不下毕生的心血尽付东流。尽管张商英,却知道该怎么利用。

在许多人眼里,韩冈之前能依靠能够在京城中掀起偌大的声浪,完全是因为国子监中的新学一脉实在太不济事。

好吧,说难听点,就是山中无老虎,猴子称大王。

只是韩冈利用一干老臣们对新法的反感,掀起了足够大的声势。使得原本应该仅仅局限于学术上的争论,变成了动摇新法的政治攻势。而天子则在再一次坚持主张新学的立场后,也不得不用侍讲资善堂的机会来安抚韩冈。

要阻止韩冈入资善堂,因蹴鞠比赛而起的惨剧,就是天赐的良机。这么做还能示好新党,张商英没有任何理由放过这么好的机会。

就算失败,张商英也不在乎。做御史的,怕的不是得罪人,而是没办法出名,没办法简在帝心。只要皇帝能记得他,日后终有回报。

走了半刻钟,宫城的南门就在前方。

远远地,张商英正看见一名身穿紫色公服的官员骑着马穿门而过,几名元随跟在他身后,与张商英擦声而过,往宫城内去了。

张商英身为御史,绳纠百官,朝会前在宣德门、文德门监察文武百官有无失仪之举,朝堂上大小官员至少都打过一次照面。

那骑马入宫之人,乍一看先觉得眼熟,走了两步,张商英便一下醒悟,

‘是曹王!’

英宗皇帝和高太后的三儿子,当今天子的三弟,曹王赵頵。

回头又望了望已经在内东门处下马的赵頵,张商英脚步慢了下来,眉头也不禁皱起:这位三大王赶着入宫到底是做什么来的?

联想到昨天就有四五名宗室成员和女眷,进了宫城之中,拜见天子、太后。赵頵今天进宫的目的,差不多也就水落石出了。

‘就算宗室全都来了又能如何?’

张商英大踏步的跨出宫城南门,心底冷笑。要是宗室的说项有用,当初王安石削减宗室爵禄的时候,天子也不会毫不犹豫的就开始推行于世。

这两天,京城中的一众宗室,你来我往的相互串门倒是如同过年时一般,传入张商英耳中的就不只一起两起。

可任凭宗室们合纵连横,也抵不过天子的一句话。

根本没有用的!

难道张商英他会不知道宗室在蹴鞠联赛和赛马联赛中占据了什么样的地位,以及他们对两项联赛的倚赖?

但宗室说得好话越多,私下里联络的越勤,韩冈身上背负的风险就越大。惹起了天子心中的猜忌,韩冈想要出头,将会更难上百倍、千倍。

张商英现在巴不得赵頵入宫后,能帮韩冈多说上几句。这样一来,自家可就能够高枕无忧了。

……………………

一场蹴鞠比赛引发的惨剧,惹来了一群吃腐肉的乌鸦。如同捅了马蜂窝,让一干但有利益受损的宗室们变得躁动不安起来。

两项赛事所吸引来的财富是个天文数字,作为主办方的齐云总社和赛马总社从中抽取的利润甚至不敢公布出来,只是一旦分配给诸多利益相关的参与者之后,拿到个人手上的就不算太夸张了。

对于普通的拿着自己的身份为赛事而奔走的宗室来说,也许一年只有额外的百来贯的好处,但是他们用来养家的俸禄,也不过是这个数目的三四倍而已。从比例上讲,若是失去这一份额外的收入,等于是从他们身上割去了一大块肉,虽不致命,却大伤元气。

即便是两位亲王,天子的亲兄弟,除了俸禄之外,能动用的公使钱一年也仅仅八千贯,还是一半钱一半绢。而蹴鞠总社中的一干副会首,甲级联赛球队的东主,他们每年的门票、广告、分红、奖金等各项收益的总和基本上都在五六千贯以上。赛马总会尽管开办未久,但会首赵世将也已经拿到了多达三千余贯的分红——这个数目,只要不嫌弃相貌和年齿,就是两个三个的进士女婿,也能在黄榜下捉到手了。

所以韩冈可以稳坐钓鱼台,在太常寺一角的小院中,编他的《本草纲目》。因为他知道,有所关联的宗室、贵戚、豪门,都无法坐视御史台继续兴风作浪。

昨天就连王旖都接到了蜀国公主的请帖,邀请她过府一叙。虽说蜀国公主家跟两桩联赛并无瓜葛,不过有资格求到她面前的宗室,并不在少数。

对于此,韩冈让王旖送了回帖,道了歉,说是身体不好要在家里休养一阵。眼下不是掺和进去的时候,韩冈也无意掺和,他想看一看这个利益集团保驾护航的能力,而不是事必亲躬,学着诸葛亮将自己累死在五丈原。

就在中午的时候,韩冈又在手下的属官那里听到了另一桩新闻,“蔡执政家的明老太君昨天入宫拜见太后和皇后了。”

韩冈眉梢一挑,蔡确的老娘也入宫了?还真是热闹。

要不是知道高太后的生辰是在六月,还真是以为是生日到了。向皇后的生日似乎是在十一月——遇上宫中太后、皇后的生日,有着封号的外命妇都要入宫贺寿。韩冈这等外臣,也要为太后或是太皇太后准备寿礼,都得记在心上。

蔡确之母明氏已经六七十岁了,跟宫里面十几二十上下的嫔妃们搭不上话。就算是高太后,也比明氏要小不少。在宫中没有人缘,想入宫也不是那么容易。

而且据说高太后对当朝的一众臣子,可都是看不顺眼。王安石、吕惠卿、章敦,这新党的一干人等就不必说了,他们的家眷入宫,怎么也不可能在高太后那里有个好脸色看。王珪、蔡确、韩缜,这些个宰辅,没一个能入高太后眼帘。

两府中人,也就吕公著还好说些。至于他韩冈,先不说王安石女婿的身份,就是以他过去的所作所为,也不可能有一个好眼色。从表弟冯从义那里引出来的那一层七拐八绕的亲戚关系,连提都提不上。

冯从义的浑家虽是高家的女儿,但毕竟是旁支,其父也就跟高遵裕走得近。好武喜兵的高遵裕在高太后那里从来就不受待见。在高氏一族中,高遵裕是完完全全的特例,其他高家人,可都是老实得很,安安生生的享受着朝廷提供的富贵。

而最重要的,韩冈可是跟高太后最疼爱的次子争过花魁,还顺手在赵颢的名声上抹了一层黑灰。

随着韩冈的大名遍传天下,他与赵颢的旧事也被人宣扬。改了当事人姓名的故事在世间流传甚广,甚至被人写成了杂剧本子,在东十字大街北面有名的勾栏院象棚之中上演过两天——也只有两天,第三天就被禁演了。只是越是**越有人爱看,越是禁演的戏剧,自然想看的人就越多,故而杂剧的剧本传得到处都是。

这样的情况下,尽管赵颢背地里还是恨着他的兄长赵顼居多,但韩冈也不可能在高太后那里受待见,甚至被暗恨也是不消说的。

大约只有那些被赶出京城的老臣,才能得到太后的看重……或许还有苏轼,在保慈宫中,据说时常传唱苏轼的新词。

不过明氏入宫,应该跟这一次的事没有关系。蔡确或是他的亲族应该没有参与进两项联赛之中。从一开始,两家总社拉拢的多是久居京城的皇亲国戚或是豪门世家,而不是经常京中、地方到处调动的朝臣。

只是明氏入宫这一件事,放在外面会让人怎么想,会让人怎么传,那可就是难说了。有心人不会放过这个搅浑水的机会。不是有着一点点误会,也不会有人在韩冈面前扯这等闲话。

不管怎么说,整件事还真是越来越有趣了。韩冈很是期待,这个他培养起来的利益团体,最后到底能有什么样的手段来度过难关。

到了快放衙的时候,一个让韩冈忍俊不禁的消息传到了他的手中,是从今日开审此案的开封府中传来的。

这一日,被传上公堂的证人多达三十余人。三十余证人众口一辞,整件惨剧的肇事者,也就是最初开始动手引发乱事的罪魁祸首,不是别人,是被人踩死的南顺侯——李乾德!

“还真是想不到!”韩冈大笑着,也不知在说谁。

