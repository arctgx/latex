\section{第十章 进退难知走金锣(中)}

“朝中已经议定,丰州必须即刻夺回。”韩冈说道,“一旦契丹人插手进来,就绝不会再坐视官军在西京道边上动刀兵。”

在辽国干涉进来之前,宋夏之间怎么打也没关系,就算占据了银夏,辽人也只能承认现实。而等到辽国的皮室军杀到边境,再想继续开展,就是要做好被捅上一刀的准备。虽然契丹人为西夏人出兵的可能性极小,甚至几乎为零,但朝堂内外都很清楚,天子可不会愿意去冒这个风险。

丰州的地理位置不算好,位于古长城的外围。不论战国秦汉,又或是后世的明代,长城始终是建在易与守备的战略要地上。既然是在长城之外,自然在地理和战略位置上有着不利于守御的一面。

其实在韩冈看来,放弃丰州,稳固横山,进而夺取银夏。从全局上来看,这个交换十分合算的,就算只留下罗兀城,都是笔好买卖。但从政治意义上来说,新党则绝对不会接受。失土之罪,就算拿回更多的土地,也不能功过相抵。横山要保住,而丰州更是要全力夺回。

“所以朝廷议定的策略,是继续向北攻击银州。只要控制了银夏,兴庆府要想与丰州联系上,要么横穿地斤泽所在的大漠,要么沿着黄河绕行,否则就要对占据银夏的官军硬碰硬。”

范育捻着胡须,沉吟了一阵,点点头,“这就是所谓攻其必救,失去了银夏,就是占了丰州又如何?失了青白盐池的池盐,西夏只凭兴灵和沙漠,根本支撑不起国政。眼见银夏或许有失,西贼就肯定要从丰州撤兵。”

吕大临一直沉默的看着地图,这时是第一次开口:“玉昆,西贼攻下丰州,所获粮秣几何?”

“西贼攻下了丰州,大大小小的城寨、村落,加起来几十万石存粮是没问题的。”韩冈苦笑了一声,“所以对西贼来说,以战养战最是划算,只要能打开一个寨子,就是几万兵马一个月的口粮。”

“食敌一钟,当吾二十钟;芑秆一石,当吾二十石。”吕大临摇头叹道,“西贼所行,已是暗合智将之道了。”

“可天下也只有大宋富庶,所以契丹、党项入境时,都能搜刮到大批的粮食财货。如果反过来……”范育探出手指,在地图上比划了一下,“若是官军侵入银夏,或是幽燕、云中,能得到多少粮食补给?”

“中国吃亏就吃亏在这个地方。东南西北的蛮夷虏寇,侵入中国都是为了钱财子女,只要成功,必然满载而归。而反过来,中国大军南征北战,则是大减国力。霍去病北征匈奴,封狼居胥,战马死了多少?!”

“谁让九州之内,但凡能够耕种的膏腴之地差不多都让汉家占了,身居酷寒之地,瘴疠之所,自家的性命也不值得多看重了。但凡有着足够的财税来源,愿意再拼命的倒也不多了。”韩冈伸手指了一下辽国燕山以南的一片地,“所以契丹收了岁币就不再举兵,因为有南京道在。”再指指西夏的银夏、兴灵两块地,“而西贼过去则是年年举兵,因为他们的土地养不活国中之人。”

“玉昆可是在为强贼作辩?”苏昞抬头笑着问道。

韩冈也笑了起来,的确听起来像是强盗的理论,似乎在说汉人不给四方蛮夷活路,“可只要能让这些强盗什么都抢不到,只是白白送命,他们也不会继续做蠢事,必然会俯首称臣。汉唐无不是如此。只可惜在高粱河时功亏一篑,否则如今就不需要再伤神了。”

这又是在说太宗皇帝的错了,不过倒也不犯忌讳,只是未免说得远了。苏昞将话题拉回来,“河东军要提防西京道的辽人,能腾出的兵力不会太多。麟府军在救援丰州时就吃了一个亏,再想凭麟府一路之力收复丰州,恐怕有些难。”

“再难也要收复,不过也不会让麟府军直接冲上去……”并不是什么机密,此时估计也已经传遍了京城,韩冈也不瞒着师长,“午后的时候,中书就移文军器监,让小弟紧急调运一批甲胄和军器过去。”

“从东京运去府州?!”范育惊问道。

“怎么可能,隔了近千里,哪里来得及?”韩冈摇摇头,若是中书敢下这个命令,他能将文书丢回到冯京脸上去,“只能是接力。先从太原武库中,将库存的札甲和神臂弓运去麟府路。而军器监则是负责用板甲来将甲胄的缺额补齐,另外神臂弓的数目也要一起补足。”

吕大临叹了口气:“但愿官军能顺利夺回丰州。”

“游景叔在种子正幕中,彝叔也同在一处,以他们这一次立下的功劳,至少能转两官。”范育也道,“可若是丰州夺不回来,这份功劳很可能就不会授下。”说完,还看了韩冈一眼。

种谔在献捷的奏章中没少说韩冈一系列发明的作用,这份功劳韩冈肯定是跑不了的。但若是丰州拿不回来,夺占罗兀城的功勋也就很难评价了——下面的军卒不会不赏,否则少不了要闹腾一阵。而领军将领的功劳,则可就悬了。若是种谔、游师雄他们没功劳,韩冈也不可能有脸一人领功。

师徒几人又讨论一阵时局军情,韩冈起身从张载家告辞出来,与范育一起离开。张府的门外,这时候还有几个士人。不是刚刚上来递了名帖,就是正准备递名帖求见。

张载如今在京中已经是人所共仰的一代宗师,闲暇时候也少不了有人登门造访。张载则在时间和身体的许可下尽心接纳,丝毫没有崖岸自高的态度。不过今天为了讨论时局,却关起门来不见外客。

“玉昆,日后关西兵事在先生面前还是要少提。”范育与韩冈并肩而行,走了一阵方才这般说道,“京城毕竟不是关中。”

韩冈沉默了一下,点头道:“……小弟明白了。”

张载和他的入室弟子,基本上都是关中人为多,因为近百年来备受党项所苦,他们绝大多数是支持对西夏的战争。但东京不一样。就算关中百姓每隔几年就要为了战事而成为被征调的民夫,就算关中百姓年年受到党项骑兵劫掠,可对东京百万军民来说,差不多可算是另一个世界的事。

只有因此而上涨的物价和税收,才会让他们觉得此时事关自己。不论是关西的战略是攻击还是守御,都是要从他们的身上刮钱过去。京城士林中的舆论也是如此,如果不耗费太多钱粮就能获得胜利,肯定会得到士大夫们的赞许。可一旦影响到自家的生活,那他们就会立刻举起反对的牌子。

张载的名声要紧。他旗帜鲜明的支持战争,肯定会惹来京城军民和士林的反感。而张载又不是愿意说谎和隐藏观点的性格,为了避免落入这样的境地,最好就不要跟张载多说这方面的消息。

韩冈叹了口气:“哪个不想太平?中原人想过着太平日子,难道关西人就不想吗?”

……………………

就在东京城内城外,都将目光放到了陕西和河东的时候,邕州知州苏缄的双眼却是在盯着一举一动。

“交趾太尉宗亶已经领了两千兵抵达广源州了。广源州的部族也全数出动,刘纪、黄金满、申景福、韦首安,他们都被宗亶召了去。”

每报出一个名字,苏缄的脸色就难看了一分。广源州是大宋和交趾之间的缓冲地,过去一直向宋称臣,不过在侬智高之乱后,交趾势力扩张,而宋廷采取了姑息的态度,让交趾将这片产金的地区给控制在手中,连同其中的几个大部族都要向升龙府进贡。

不过交趾对待这些部族一向苛刻,要不是因为断了生计,现如今也不会聚在一起准备与交趾人一同北犯。

“多亏了刘执中【刘彝】。”苏缄仰天惨然一笑,禁绝市易到底害了谁啊!要不是刘彝禁绝与交趾市易,不会有那么多家溪洞蛮部跟随交趾人北犯。

“不过交趾的主力在哪里?宗亶只带来了两千人呐……”苏缄的亲信幕僚很有些疑惑,“如果交趾不出兵领头,侬人诸部绝不会为其火中取栗。”

谁也不比谁傻多少。宋人断了部族中的财源,当然的。可背后的交趾人也不是什么善人。如果一旦在宋军这边吃了大亏,说不定老家就给升龙府派兵出来端掉了。所以交趾必须要率先出兵,出来打头阵,以作证明。

回到后厅,苏缄仍在考虑着此事。交趾即将入寇,但他们主力究竟在何处?

“大爹爹。”

一声清脆的叫喊从下方传来,被打断思路的苏缄低头一看,却是自家孙女笑得灿烂的一张小脸。

看到孙女儿的笑脸,苏缄沉重的心情放松了一点。

“大人,是不是又是交趾的事?”

苏缄的长子苏子元也一起走了出来。他在桂州任司户参军,正好得空来探视。他这一次来,顺便将妻儿,包括苏缄最疼爱的孙女也一起带来了。

先将孙女送回后院,苏缄和儿子坐下来,叹了半日的气,开口道:“交趾即将来犯,你还是早点回任上。”

苏子元有点疑惑:“也不必急在着一时。”

“你不知道。”苏缄端起茶盏,盯着盏中的茶汤,眼底的沉重在波光盈盈的水面中完完全全的映了出来,“再迟就不好走了。”

苏子元皱眉正要说话,一名军校满头大汗的跑了进来,慌慌张张的在门槛出绊了一跤,一骨碌爬起来后也不顾身上的灰,急声叫道:“启禀皇城,钦州急报!交趾数万大军渡海而来,主帅为李常杰,如今已经开始围攻钦州!”

苏缄手上的茶碗落在了地上,一声脆响过后,碎作了千百片。

