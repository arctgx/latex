\section{第26章 鸿信飞报犹觉迟(五)}

【写得慢了,明天七点的一章,只能拖后一点发,】

在广西南部诸州绕了一个圈,当韩冈重临交州的时候,章家商号的海船也已经从交州到泉州跑了个来回了。

这一趟下来,章家足足赚取了四倍的利润,总数达到了十万贯之多,还只是香药仅有一船的缘故。

韩冈能知道这一条航路上的香料到底有多少赚头,还是靠了有人给他透底。这样的利润,就算以韩冈的眼界,也不免要吃上一惊。国内转运贸易,论起赚钱多寡的问题,恐怕就是以这样的一条商路为最。

既然是这样收入丰厚的买卖,韩冈也不指望章家的人能跟章惇一样,在钱财面前,有着足够清醒的头脑。

韩冈已经知趣的放弃了劝说,反正就算章惇也做不到垄断交州的香药贸易,交州毕竟不是章家开的铺子,想怎么买卖,就怎么买卖。

等过上一阵,随着商人来往的越来越频繁,盯上香药贸易的人会越来越多,而利润率也会逐步下降,章家商铺如今的暴利,很快就回成为过眼云烟。若是不能及时抽身而退,而是为了赚取更多,去租用了更多的商船,那么最后血本无归也不是不可能的。

韩冈心里虽是对此有所推断,但见到章恂的时候,却是一点也没有提到关于香药的事。该说的已经说了,再重复也无意义。有时候,就算是好心,别人也不一定领情。

章恂的年纪是比起韩冈要大上一些,但章家的这位十一郎其实也不过刚刚到了而立之年罢了。相貌轮廓与章惇很有几分相似,但缺乏章惇那股子过人的魄力,也没有充斥在举手投足之中的与生俱来的自信。

他在交州等待韩冈,已经有些不耐烦了,毕竟这里形同流放之地,与福建是没办法比的,更不用说东京城。

不过当真见到韩冈的时候,章恂却是恭谨有加的向韩冈行礼,一如他与姓名同出一源的表字公谨一般。

韩冈也不能将章恂的礼数照单收下,侧身避让过,然后换了一礼:“劳公谨久候。”

章恂出身世家,又是多在江湖上行走,待人接物只要有必要,都能做得让人如沐春风。正好韩冈刚刚弄回来一批疍民,他便趁着机会赞美着韩冈的功业,“疍人久不服王化,如今却主动来投,都是玉昆的功劳。”

“哪里。”韩冈摇着头,“若无令兄在前让诸部畏怖,哪有如今的蛮部来投。”

章恂笑着说道:“如果能教会疍民种地,那么把耕种之法传于诸蛮也就不在话下。想必玉昆已是胸有成竹。”

“胸有成竹如何敢当。只是走一步上一步,剩下的就要靠公谨吉言了。”韩冈笑着说,虽然章恂的话只是随口说说罢了,但要是最后的结果是好的,那就太好了。

虽然同是教授不事稼樯的部族耕作,但两个的难度是不一样的。一个是自己辛苦,一个则是手下的奴工辛苦,当然是后者容易,而前者则是很难在短时间内适应。

韩冈要表示亲近,让章恂陪同他去视察安排给疍民的聚居地。其实章恂说得也没错,如果连疍民他们都能开始种地,那么蛮部肯定也不会比他们还差。

富良江快入海的时候,便从一条河道分叉出五六条河道来,分作数条支流入海。总计八百余户疍民,分别居住在三个新建的村庄。分配给他们的土地,正好是在江水分流后,两条分支交夹而成的土地上。这一片地,土地肥沃,又靠着江水,如果疍民们种田水平一时提高不上去,还能在江上捕鱼补贴日常家用。

不过现在看起来情况还不错。疍民们的房子是由州中专门派出了几名善于营造的工匠指点而成。基本上都是一模一样的房子,用着最省的材料,搭建出足够结实的房屋来。

“想不到都是竹子的。”章恂放眼望过去,一家家一户户都占了一座竹楼。同样的只是房子的外形和结构,都是一模一样。远远地望过去,也分不清谁是谁。

“木头容易朽烂,竹子就好一些,而且竹子生长得快,比起木头便宜多了。只是的确是简陋了些。”

“疍民一辈子都生活在水上,有许多东西,在我们看来简陋的很,但比起海上的小船,已经是个在天一个在地。”章恂不介意拍拍韩冈的马屁。

“也不是这么简单。既然疍民在此处居身,就要即刻开始修建堤防,要不然光是洪水、海潮,都会将这几座村子从这片地上给抹掉。”

视察过疍民的村落,韩冈和章恂回到海门。但他们却发现这里的水上巡检,正在强行登船,闹得港中一片混乱。

“这是怎么回事?!”章恂惊问道。

“只是在检查铜禁而已。针对与西洋交易的船只。与夷人交易没问题,但铜钱可不能让他们带出去。”

此时铜禁森严,若是触犯又被捉到了的话,直接就是死罪,根本不管是什么理由。所以船上的商人们一个个脸色如土,虽然他们的生意并不是针对外人,但随身塞着多少用来采买的铜钱,如果官府要较真,人人都逃不过去。

韩冈并不是要拿这些商人怎么样——虽然按照太宗时制定着编策,他们一个个都可以上刑台——韩冈突然派了人来,是为了要重申铜禁。

他既然身在广西之中,忝为一路转运,不能眼睁睁看着大批的铜钱,从交州的海门港流向南洋周边各国。

“搜检真够仔细的。”章恂望着船上人影晃动,由衷的感慨着。

“当然要仔细,如今国中正闹着钱荒,没钱拿出去给外人用了。”

“如果是载着丝绢、瓷器去南洋倒是好了,只要担心风向,其他什么都不要担心。来回倒腾各自也能赚上不少。”章恂这么说话,倒是有三成是在试探。

韩冈摇摇头,算是婉拒了。贸易转运的确能快速发家致富,但韩冈有了更为稳定的。这个时代的海外贸易,对国家的用处并不算大。

韩冈虽然对历史不甚了了,但好歹也了解一点大航海时代的起因。一开始是为了打通土耳其人对东方贸易的垄断,开辟沟通大陆东西两端的交通线。

西方人的目标,一个是中国特产的瓷器、丝绸和茶叶,但南洋地区的胡椒、豆蔻、丁香之类香料,也同样是他们孜孜以求的目标。而且这些香料是冬天腌肉时必不可少的调味料,算是必需品,比起丝绸、瓷器之类在西方观点里的奢侈品,要更为重要。

可是大宋的海外贸易,交换来的没有必需品,也没有硬通货。不是几百年后西方大航海开启的时代,可以通过丝绸、瓷器、茶叶这样的特产——也就是工农业的制成品——换来大量的白银、黄金,对国民经济的好处不言而喻,基本上都是奢侈品。

对于商人来说,只要能赚钱就行了,海贸虽然风险高,可获利也是几杯十几倍的暴利。对官府来说,能从海船上收税也不错了。但对大宋这个拥有上亿人口的世界上最大的经济体来说,则基本上亏本买卖。

在海贸交易中,大宋输出的不仅仅是丝绸瓷器和茶叶,还包括铜钱这样的货币,而且往往是一船船的被运出去。宋钱制作之精美,使得在周边各国都变成了主要的货币。交趾便是如此,而日本、高丽,也同样是如此。

输出之后,

若是流出的是纸币倒好了,但偏偏是硬通货净流出,换回来的是香药、珠宝之类的奢侈品,主要提供给上层使用,于国无益。而市面上的铜钱大量流失,国民经济是不断失血的。

不论是韩冈,还是当今大宋的君臣,对于铜钱流失的危害,都有个清醒的认识。

且这个流失并不是仅仅局限在外国,大宋境内,喜欢屯钱的更是数不胜数。像田鼠一样将手上的钱都埋到地底,这一点最是让人头疼。

钱币要进入流通环节才会发挥应有的作用,朝廷的封桩库倒还好说,里面的钱绢是为了备战备荒用的,但民间珍藏钱币,却是埋进地底去,也不知何时可见天日。

岁币岁赐的支出,并非是小数目。给辽国二十万两白银,给西夏则是七万两——熙宁七年后,给西夏的岁赐就再也没有给过了——只是给予辽国的二十万两,就已经相当于全国一年白银产量的大半了。不过送给辽国的这些白银,基本上在一个月半月的时间里,都通过各色贸易,重新回到了大宋这一边,但铜钱却不是这样,到了异国他乡,就立刻流通起来,根本就没有回来的机会。

韩冈能想到的办法就是使用纸币,利用币值并不稳定的纸币逼迫人们,只能尽可能的将手上的纸币消费或是投资出去。要做到这一点倒不难,但带来的结果只会是滥发纸币,人们最后抛弃这一个国家的货币。

