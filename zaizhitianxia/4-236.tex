\section{第39章 遥观方城青霞举(一)}

【年后事多,更新不正常,少了许多。今天四更补回。第一更。】

被两匹挽马拉着,轨道上的四轮马车行得很平稳,没有寻常道路上的摇晃。只有在马车通过两节木轨的交界处时,前后轮短促的两声响,才会暂时打破车厢中的宁静。

韩冈闭着眼睛,感受着身下有着稳定节奏的咔哒咔哒的声音。类似的声响,他不知有多少年没有听过了,如今落在耳中,一时间便被拉回了旧日。

那还是刚出来做正经事的时候,一个月总有三分之一的时间听着车轮撞击轨道接口的咔哒咔哒的声音入睡。等过了几年,再坐车就没有这种声音了。

不过音色与旧时有些异样,似是在提醒韩冈,已经回不去了。

千年之前的技术水平,想要复制出他记忆中的音色,不知要用上多少年的时间。木轨之间的接缝,也远不及铁轨那么宽。

韩冈急切盼望轨道能变成铁轨,但要以眼下钢铁业的技术水平,想要造出轨道钢,不知要等到猴年马月。韩冈都不求工字钢,普通的宽幅钢条都造不出来,只有铸铁而已——而且即便造出来了,也没人舍得用在道路,而不是用在兵器上。

眼下也只有在木轨上压上铁片,再钉上一层铜皮来防止车辆损坏轨道,日后要修长路,也可以只用铁片来节省成本。

轨道钢啊……韩冈在心中暗叹。

除非天子和朝廷能放开对钢铁业的控制,让民间的资本能渗透进来,否则想要达成自己的目标几乎没有可能。

技术发展可以由官府来引导,但不应该由官府所垄断,尤其是钢铁业、制造业,只有一个垄断的卖家,是成不了产业的。但韩冈受到的干扰太大了,他真正的想法除了自己,说服不了任何人。还要为了迎合现实,不得不将自己的计划改头换面。

韩冈沉浸在思绪中,不说话,同车的沈括、李诫、方兴、沈博毅,还有方城知县,也不便说话。几个人低眉垂眼,都不敢惊扰到韩冈。

不知过了多久,韩冈终于有了动作。眼睛眨了两下,略略直起了腰,还未开口就已经打破了车中的寂静。

“路修得好。”韩冈的口气很是满意的样子,也不解释方才为什么突然就沉默了下去,透过两扇车窗,外面的景物正不断的倒退着,“新修的道路好走归好走,但要如此平稳,也不是那么简单的一件事。”

沈括活动了一下身子,“这马车打造也好。”

韩冈随着沈括的话,扫了眼车中。他说得的确没错。

类似于后世西方马车的形制,而跟此时的马车截然不同,大了许多,也宽敞了许多。光是里面相对设了两排座椅,就让人感到惊讶了,而长长的座椅甚至能让人躺下来。

一辆车中坐了六人,空间却还是显得宽裕得很。这个时代的双轮马车是绝对做不到这般宽敞,也只有四轮马车才能做到。

四轮车并不出奇。在攻城器械中,莫说四个轮子,六个轮子、八个轮子的都有。而眼下在码头和矿山中,用来载货的有轨马车也全都是四个轮子。在轨道上,不需要考虑转向问题,只要想着如何增加载重,四轮远比双轮更为有利。

而且能行驶得如此平稳,不仅仅是路好,也是马车本身工艺精巧。从车架到轮轴,再到轮毂、车轮,都是京中的官坊,穷多年积累而成。无论在材料上,还是在制造工艺上,都代表着这个时代的最高水准。

窗外风景变幻,风从窗口灌进车厢,清凉的,丝毫不见外界的暑热。方城轨道全城六十里路,也不过用了一个时辰而已。

“真够快的。”从车上下来,已经是方城垭口的北端。沈括张望了一下远近的风景,对韩冈笑道:“依律外官不得擅出本界。这里已经是汝州了,被御史抓到,可就少不了一封弹章。”看看弯着腰不敢说话的方城知县,“应该是两封。”

“咬死不认就是了。你不说,我不说,御史怎么会知道?”韩冈开玩笑道:“明天我拉着方静敏也到唐州走一趟,难道他还能出首告你不成?”

“方静敏过来,可是要摆酒庆贺一下。”方静敏是汝州知州,在方城轨道的修筑过程中也出了不少力气,只是不如沈括。

方兴感叹着:“一个时辰六十里,快赶上铺递,寻常铺递也不过一天四百里。”

这个速度与后世当然没法比,但比起这时代的寻常马车来,已经是很快了。除非做好了累死坐骑的准备,否则寻常骑马也是这个速度。

“换作是载货,就不会这么快了。两匹马拉上万斤的货物、七八个人,就只能慢慢走。”早在沈括和韩冈来视察前,李诫就已经测过了时间,“大约只有现在一半的速度。”

这条轨道,载人载货都是合在一起的,货主或是押送货物的人员总要跟着货物走。两个时辰六十里,的确不快,但比运河中的纲船还是要快一些。

“能比船快就好。”即便只有一半的速度,还是能让沈括满意。

“汴河上的纲船额定是六百料到七百料,载重四五万斤,不知在这条轨道上一个车次最多能拉多少?可曾测试过?”韩冈问道。

“增加拉车的挽马的数量,四匹马、六匹马,后面就可以多挂上几节车厢。最合适的还是六匹马,少了马力不足,多了就驱赶不便。六匹马拉四车货、一车客,一次就能抵得上一艘纲船的量。”

李诫的回答证明了他已经对轨道的运输工作进行了多次测试。韩冈和沈括对视一眼,一起点了点头。

原本李诫只是韩冈为了酬谢李南公的帮助,而准备放在唐州分功劳的闲人,但谁也没想到他在工程营造上的能力出类拔萃,在管理上也有出色的表现,逐渐的就让韩冈将整个轨道铺设工程都交给了他来统管。也算是运气了——当然不只是李诫的,也是韩冈和沈括的。

“一趟车四五万斤就只要六匹马,比起太平车不知省了多少。”沈括道,“看起来方城山这里要养不少马了。”

“连同替换的在内,要一百五十匹挽马。”这次回话的是方兴,这个数字早就在心中转了很久,“每年至少还要淘换其中的十分之一,甚至五分之一。”

“拉车是力气活,马匹的食量不会小。”沈括算了一下,“一匹一年至少要十五石粮、三十束草。加起来近万了。”

“还要用榨过油的豆饼来补充力气。不过豆饼不值钱,一万石束的粮草也算不上什么。挽马不比军马,一匹也不过十几贯。加上人工,也不会太多,最多两万贯而已。”方兴笑了一笑,“如果管束不严,一个月给人干没的都不只这个数。”

韩冈点点头,对沈括道:“统管方城轨道,当择人择术,否则就又是肥了一群硕鼠。”

沈括笑笑,不接口。他现在还没打定主意是否要出来管着襄汉发运司。

方兴转了一圈,看着轨道北端的转运港,连仓库都没有修起来,也就是码头给建好了。望着港口中工地,他问着李诫,“要整修完工,还要用多少时间?”

“要到九月了。”李诫回道,“还得开销两个月的钱粮。”

沈括道:“港口只占小头。只为这一条轨道,就花费了不少。”

为修轨道,唐州出钱出人,今年的税赋只在转运司的帐本上走了一圈,钱粮实物直接在州里就截留了,沈括本人自然是再清楚不过。

李诫指了指半尺高的碎石路基,“成本最高的就是路基下的碎石。从山里、河里挖出来,运到方城垭口来铺起,几乎都抵得过一年的运力。”

“这开支可就大了。”沈括皱眉,“只六十里还好,要是长了可就让人头痛了。”

韩冈则道:“不是说一定要铺路基,将路轨直接放在平地上也是可以的。矿山、码头中的轨道,哪有多余的闲钱和时间,还不都是直接铺在地面上?”

“不论是做轨道的,还是做枕木的,都是好木料,多少也值点钱,还不用提轨道上的铜皮。”“沈括注意着韩冈脸上的表情变化,“在矿里和码头上,人来人往,也没人敢打轨道的主意。不过放在野地里,可就两说了。是不是要人沿线盯着?”

韩冈一笑:“肯定要在沿线派人巡守,就跟汴河上要派人看着一样。只是眼下的这六十里轨道,倒也不用太多人手。”

“不过这轨道一修,六十里路沿途都不会停留,垭口中做些茶酒买卖的店家,可都会恨透了这条路。”李诫说着。

方兴冷着脸:“往岭南流放个十几二十人,将伸过来的贼手给杀下去,看看谁还敢犯事!”

“现在说这些也太早了。既然轨道已通,剩下的也就是怎么将秋粮运往京城……”韩冈横目扫过众人,“只有将今年的秋粮运往京城,才代表着襄汉漕运打通,才能让朝廷看到我们的功劳。”

方兴第一个点头,紧接着李诫、沈括他们也跟着点头附和。

这半年来,他们一番辛苦究竟是为了什么?还不是为了功劳和之后的封赏。开漕为国利民,但没有足够的回报,又有谁会分神费力?

喝酒吟诗,日常饮宴多得能被称为酒食地狱,也同样是做官啊,官场中,那样的人更多。

