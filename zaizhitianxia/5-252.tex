\section{第28章 官近青云与天通(19)}

没吃过韩冈的亏吧?

章敦心中的欣喜满载着恶意。当年司马光在殿上,骂王安石,骂吕惠卿,骂曾布,就是没怎么骂他章子厚。这让章敦当年倍感屈辱——他进入新党核心要比吕惠卿和曾布要晚,当时的地位也不算很高。

韩冈从来都不是善茬。关键是他与人相争时,总是先让自己立于不败之地。就算不擅文辞,不识典故,家世又缺乏底蕴,但韩冈通过一桩桩功绩让自己变得无可替代。能作评判的天子、皇后,皆尽左袒,当然是常胜不败。

也就在道统之争上,天子偏袒新党,让韩冈无所施为。可终究还是因为保住皇嗣,不敢把事情给做绝了。

章敦曾听韩冈说过,他当年去京西任职,拜见了富弼,拜见了文彦博,洛阳元老一个都没漏过,却唯独没有见到司马光。

要是司马光早见过韩冈,甚至在他身上吃过亏,如今也该吃一堑长一智了。可惜,根本就没有机会。昨天的会面,据说韩冈完完全全是个守礼晚辈的模样,想必司马光也不会想到韩冈一转眼就能提刀砍上来。

为了保护新法,韩冈可是比任何人都要积极!

转过脸看看张商英等御史,章敦眼神冰冷。韩冈和司马光之间交锋,绝不是为了区区一王珪,如果看不到这一点,那就别想再有出头的机会!

成了韩冈攻击司马光的跳板,张商英已经被踩得晕头转向好半天。不过他决不愿服输,他还可以去攻击韩冈,可来自章敦眼中的森森寒意让张商英不敢再稍动半步。

他是章敦开拓荆湖时提拔起来的,之后犯错被贬,又是得到了章敦的提携。再后来,因为要表现御史的风骨,与章敦逐渐生分。但现在,能救自己的,只有与韩冈情谊深厚的章敦。

张商英终于是确认了,这已不是针对王珪的交锋,而是新旧党争的再起和延续。想到自己竟然被弹劾宰相的金光蒙住了眼,没看到金光后的无底深渊,悔恨如同毒蛇噬咬着心脏。

要彻底站到旧党一边吗?张商英想着。新党这边已经无法立足了。

只是殿上的局势,却让他不敢下此决断。

司马光刚刚出头,甚至仅仅是迂回式的攻击,就已经被警惕性极高的韩冈打得不能翻身。他身后的旧党,又怎么可能例外?

韩冈还不到三十,章敦、吕惠卿、吕嘉问等人也不过四旬出头。新党当年被称为新进,如今十年过去,却全都成为了朝堂中坚。而旧党……张商英看看司马光和吕公著已经白多黑少的须发,这让人怎么对他们有信心?!

张商英犹豫不定,舒亶也犹豫不定,所有站出来的御史,一时间都没有决定是撕破脸皮全然站到旧党一边,还是暂时忍气吞声,企盼不会有太重的处罚。

他们的窘相,全都落到了朝臣们的眼底,幸灾乐祸的笑意也在他们的眼神中交汇。

司马光是新晋的太子太师,而且是天子在病榻上任命的,近似于托孤重臣的身份,绝不会被论以重罪。韩冈指称他是心疾,眼下的结果最多也只是回洛阳养病。

但一应犯错的御史,可就没有这个待遇了。

韩冈攻击的是他们的品德问题,不是论事的对错。一名御史,必须要维护自己的独立性,只向皇帝或是皇帝的代理人负责。

弹劾王珪无所谓对错,即便失败出外,照样能将名声打出去,日后东山再起,只会升得更快。可前后论奏不一,跟着司马光合唱,却是一名御史绝不该做的事。韩冈的弹劾,等于是从根子上断了他们的进路。

乌台监察百官,乃是两府之外,朝中百司数一数二的清要之地。御史们得罪的人不少,惹来的嫉妒也不少。

不少朝臣都在幸灾乐祸的看着殿中的十余位御史,大半个御史台方才都跳出来了,皆在韩冈的攻击范围之内。失去了向皇后的信任,又没有大义傍身,根本就不可能脱身出来。

御史台要大清洗了。

也有些人在看吕公著,旧党赤帜就要成了疯子,不过旧党在两府中的代表却让人纳闷的没有动静。

朝臣们分了心,对于司马光的关注也就少了许多。但韩冈却仍在警惕着那位犹然立于大殿中央的太子太师。

涨红的脸色已经渐渐恢复正常,表情中也找不到了愤怒的成分。当司马光平静无波的视线移过来的时候,韩冈的心中甚至敲响了警钟:

他还没有服输!

想想也是。要是能这么干脆利落就赢了自家岳父的老对头,那还真是小瞧了名传千古的史学大家,更小瞧了自家岳父。

不过韩冈不怀疑自己是否能得到胜利。天子和皇后可以不需要司马光,却不能不需要他韩冈。就像熙宁变法。纵然天下士大夫中多半倾向旧党,甚至地位越高的,反对得就越激烈,让王安石只能选择吕惠卿等新进为助力。可新党照样笑到了最后。国家需要新法,天子需要新党,旧党即便势力再大,根基再深,也只有失败一途。

司马光自然不可能赢了自己。只是杀敌一千,自损八百的结局韩冈不想要,皮洛士式的胜利等于是失败。

“司马卿,还是先下去歇一歇吧,有病得好好养着。”向皇后看着文德殿中已经看不到东西班列的文武群臣,觉得还是将祸乱之源先给清出去比较好。

何况现在司马光受到的刺激太大,若真的在殿上发病,他一生的声名都要丧尽了。让他下去先歇一歇,应该不会错。

这当是常听人说的,要维护重臣的体面。向皇后想了想,自我肯定的点了点头。

殿中又安静了,注意力的焦点回到了司马光的身上。

司马光遽然抬头,愤怒的血色重新充满了他的眼中。

“韩内翰乃是药王弟子,既然说臣病了,那臣当真是病了。”司马光的声音颤抖着,激荡的心境从话声中透出,“熙宁二年新法施行,至今已有十二载。其中连年战火,灾异频频。纵有煌煌之功,可民生之困苦,条条种种实是数不胜数。臣之病,非为己病,实为天下而病……”

他停了一下,轻吐了一口气,仰起的面孔上甚至能看见溢出眼角的泪水,最后,他猛然怒喝出来:“若说臣有病,臣的确已经病了十二年了!!!”

司马光的怒喝在殿中,周围寂静无声。

这是什么?

怨望!

不管司马光说得多么冠冕堂皇,表现得多么悲愤,浓浓的怨意却是溢于言表。是对新法的痛恨!是对天子坚持新法的不满!是要继续坚持党争的宣言!

明明白白的怨望!

可司马光眼下宁可亲口坐实自己的怨望之罪,也不会让心疾、惑疾之类的病症强加在自己的头上。

若是被确定为失心之症,也就没有卷土重来的机会。而现在他所承认的一切,的确可以说是怨望,可是当未来国是更迭,又可以说是思国忧民的表现——就算是现在,当这番话传扬出去后,也肯定能惹来不少同情和敬仰的目光。

而且乍听起来司马光表现得忠心耿耿,忧国忧民,毫无经验的皇后,被其蛊惑也不是不可能的。

这份冷静,倒是印证了韩冈之前的判断,司马光没有服输。甚至还反咬一口——今天韩冈能拿药王弟子的身份来指证他司马光是疯子,那明天又将是谁成为牺牲品?

韩冈今天在殿上做的事到底是什么?

司马光已经说出来了。

是张角的妖言惑众!是赵高的指鹿为马!是李林甫的颠倒黑白!是来俊臣的罗织人罪!

韩冈既有如此手段,朝臣们纵然不是人人自危,也会从此对他提高警惕了。

其实司马光即便证明了怨望,依然无法治罪。以他太子太师的煌煌地位,旧党赤帜的赫赫声威,也只能让他回洛阳养老。尽管司马光对王珪喊打喊杀,但他依然可以仗着与天子共治天下的士大夫的身份,来避免任何加之于其身的罪责。

情况再坏,也不过是一切照旧,司马光回咬一口的结果,却是能将韩冈拖入烂泥塘里。

章敦和苏颂都变了脸色,司马光的反扑在预料之中,不过狠辣却超乎他们想象。

可韩冈神色如常,这又能怎么样?

难道将新党的这一次反扑给打回去,会一点损失都没有?知兵如韩冈,不会这么幼稚。

且更重要的,关键点并不是自己,司马光到现在还是没有想明白啊!

“敢问宫师。”韩冈平和淡定的问道:“王珪之罪当如何论?”

司马光刚刚凝聚起来的悲壮气势顿时就烟消云散,甚至有一瞬间的迟钝,“诛之!”尽管声音依然狠厉,却没有了之前的毅然决然。

“罪名呢?”

司马光气势更低了三分:“奸邪!”

韩冈轻叹一声,摇摇头,却一句话也不再多说了。

还需要他说什么呢?

眼下的关键点是什么?

是对王珪的判罚!

司马光死不认错,咬定了要杀王珪,但他不敢也不能将王珪的罪名一条条列出来。一旦他这么做,即便区区一个大理寺中的法官也能将之一条条的驳回去,无论如何都定不了王珪的罪,最多也只是出外而已!

——在皇帝和皇后的心目中,王珪最该死的地方就是他在定储之事上没有尽到他的责任,可王珪他毕竟开口请求立储,是韩冈、张璪、薛向之后的第四人。

他没有反对立储,而是支持的!这样的作为,甚至无法治罪,只能褒奖!

所以司马光给出的只有空洞的奸邪二字。

如此罪名,还要杀宰相?!

这难道不是心疾最好的证明吗?难道这不是怨望于心,以至于王珪成了出气口的证明吗?

前面听到司马光的悲愤之语,向皇后一时间也不免为之动摇。可现在司马光依然咬定了王珪,却给不出一个让人信服的罪名。这让她又坚定了对司马光的看法!

“记得当年宫师守长安,上书建言边境息兵,京兆【长安】、邠州不必加强防备。而后庆州广锐卒叛乱,叛贼吴逵领兵南下,破庆州,掠邠州,兵锋直指长安城,幸而在罗兀城与西贼交战的王师回返,才将其困在了咸阳。又得韩冈孤身入城说降,方才顺利平叛。只是也让西夏又苟延残喘了多年!”

王珪为相,主张伐夏,虽然有些波折,但西夏终究是灭了。司马光说不要加强长安、邠州的防备,可吴逵叛乱,差点就攻到了长安,解围还是靠韩冈帮了忙。

这是给司马光的最后一击——无能!

说话的,是蔡确!

