\section{第九章 拄剑握槊意未销(11)}

【下一章中午时发。】

王舜臣想要全取河西之地,以他的手上六千兵力是远远不够的。从凉州【今武威】一路往西北走,甘州【张掖】、肃州【酒泉】,瓜州,驻防都要留人,到了沙州【今敦煌】还能有多少兵?

自然要调动起当地的兵源来助阵。就像当年韩冈在广西做的那样,也是一直以来西北用兵的固定模式。

冯远当然明白王舜臣的用心。入华夏则为华夏,入夷狄则为夷狄。王舜臣拿来用的虽是后一句,但关键还是在前一句上。

“如果能为中国效犬马之劳,那也算是入华夏吧?”

“自然。总得一步步的来啊……”王舜臣对冯远笑道,“木征、董毡如今不都穿得跟汉人一样?熙河诸蕃部的族酋们,除了脸皮黑,看穿戴都跟汉人没差别。”

“什么时候他们能读书考进士了,也就跟汉人差不多了。”

“迟早的事。董毡的便宜儿子阿里骨在蕃学里面也是学得有模有样,过些日子给他一个贡生名头,去京中考上一次,中是中不了,可回来后也能暂摄差遣了。”王舜臣摇摇头,“说得远了。如今要做的,是将凉州的汉蕃两家手上的兵全都弄出来,与官军一起打到沙州去。”

“倒也不难,不外乎以利诱之、以势迫之。”冯远对王舜臣道,“小人先去找两家,让他们带头同意就好了。”

“没那么麻烦,召集过来吩咐一下就行了。敢不听话,就拿两家出来杀鸡儆猴,也不费多少手脚。听话的,各州非汉人的户口,全给他们都可以!”

这是前几年南征时的手段,冯远想了想,也觉得这个手段不错,同样能成功。

“不过凉州也要小心。”王舜臣边想边说,“说不定从哪里绕出来一队西贼,抄我们的后路。”

“从哪里?!”冯远很意外。

从凉州径直往东,不用向南绕道兰州,也可以直通黄河之滨的应理【今宁夏中卫】,再往前百十里就是葫芦河口和鸣沙城。那正是苗授和王中正往攻灵州的道路。

除此以外,冯远不记得还有其他道路从兴庆府通凉州,而不用经过官军已经占据的黄河谷道。难道党项人的骑兵能向西穿过贺兰山进入大漠,再向南穿过合黎山抵达凉州不成?

“山里面总有些许小道,而且凉州守军在破城时逃散了不少,也得防着他们。”王舜臣叹道,“既然三哥担心官军会失败,我也不能不防着。”

“说得也是……还是将军考虑周全。”

“不过留下千人也就足够了。剩下的就跟俺去抢地皮、抢钱粮、抢女人、抢好马!”王舜臣说着跳了起来,绕着庭院中的大宛龙驹走了三圈,眼中满是不舍,“这么好的马,可惜不能骑着上阵……干脆献上去好了,省得有人惦记。”

冯远垂下头,将惊讶藏在心底。很少能见到一名武将能压制自己对宝马神兵的喜好,而且王舜臣还是有名的好美酒、好美色,对兵器、战马同样是喜欢珍藏精品。

但王舜臣说得也没错,母马一般是不上阵的,没有阉割过的公马也同样如此,能繁衍更多好马的种子,上阵就太浪费了。既然不能用,留在手上也没意义,还会被其他高官惦记,不如直接献给天子。

“来自大宛的良驹,只要打通河西,迟早还是有机会得到的。”

“所以现在要做的,就是先打到沙洲去。”

冯远接手的任务是布置顺丰行在河西乃至西域的商路,不过在外面暂时挂了王舜臣幕僚的名头。而王舜臣帐下除了冯远,还有三名幕僚。帮忙书写奏折、文章的,查对军中钱谷的,参赞军中机务的,加上冯远一共四人。

不过这几名幕僚的职司之间,分得也不是那么清楚,许多事都是与王舜臣聚在一起议论敲定。每次与他们商议过后,王舜臣都会觉得这样的幕僚才用得放心。自家请来的幕僚总归跟自己一条心,朝廷安排下来的幕职官都只会想着自家的前程。韩三哥一心想要改进的什么参谋制,哪里能让人放心。

当年机宜文字难道不是经略司中的幕职吗?可看看王资政当年,跟李师中、窦舜卿打了多少擂台。还有曾经听他漏过口风、专一规划军略、统掌军令的新衙门,到底是文官还是武官?武官……各路帅臣可都是文臣。文官……那他们跟枢密院争权之余才会做正事,而且上来的只会是会做官的文臣,军事根本指望不上。

将另外三名幕僚招来,还有副将白玉,一起点算清楚了城中的钱粮,差不多足够王舜臣麾下的六千兵马使用上一阵。

有了还算充裕的粮草打底,王舜臣的盘算也就有了实现的可能。他与白玉,以及四名幕僚一番商议,敲定了之后的方略,接着又让幕僚出去暗地里联络了几个亲信,王舜臣便下令击鼓聚将。

鼓声余韵犹存,众将校已经汇聚到王舜臣的面前。他的副手和两名部将,加上各个指挥的指挥使,有老有少,可无一不是身经十数战、乃至百十战的悍勇之辈。

在王舜臣的面前,这些悍勇之辈,却一个个屏声静气,行过礼后,就分了左右站好。资历最老、且是王舜臣副手的秦凤路第六将副将白玉,上前说话,“都军击鼓传唤,此时众将皆已到齐,还请将军令示下。”

王舜臣眯了眯眼,问道:“各部兵将是否已经休整好了?”

下面的将校一个个应声答话,皆道已休整完毕。王舜臣领军顺利的夺下了凉州城,经过了几天的修养,全军上下的士气和体力都恢复了,大部分受伤的士兵也恢复了一定程度的战力,已经可以重新投入战斗。

为了方便王舜臣指挥,划拨给他的六千兵马,总共十五个指挥,却分别来自四个将,又只安排了两名部将来统管,而作为王舜臣副手的白玉,又是有名不爱争功的好脾气。这么一来,白身的王舜臣在指挥上就无人能掣肘,免得内部相争导致无功而返。

“既然休整好了,为何这几日没有人来向本将请战?”王舜臣凌厉的目光扫过众将,“难道想在凉州住个一年半载不成?!”

他站起身,在厅中踱着步子,“要知道,王都知可是领军去攻打灵州,到时候六路合攻兴灵,一举灭亡了西贼,而你们就只夺了一两座城池,日后酒席上夸功耀武,还有你们坐下来的位置?!”

“王耀,你想看到彭孙在你面前炫耀自己砍了多少西贼的首级?”

“徐勋,要是刘仅夸口说自己收了梁乙埋家的女眷,你能拿一个西夏钤辖家的小妾跟他比?”

“穆衍,你的连襟汲光听说是在高总管帐下,你想自家的浑家整日抱怨你没能给他弄个诰命回来?”

王舜臣一个一个的点过去,恨铁不成钢:“再想想封赏,一个凉州的功劳,够几人分的,可还能拿来封妻荫子?……你们啊,难道就想当一辈子指挥使不成?!”

“打到沙州去!不过多走点路而已,但收复了整个河西,绝不会比攻下灵州少上一点半点功劳!那些功劳三十万人分,而河西这里,可就只有十五个指挥。”

“都军,你带着俺们打好了!”一个年轻的指挥使跳了出来,“沿着路向西打过去。”

“对,打到玉门关去!博个封妻荫子。”又有一名中年指挥使站了出来。

两人都是王舜臣的亲信,之前王舜臣就让幕僚联系过两人,在合适的时候捧个场。不过厅中气氛早已被王舜臣煽动了起来,方才被王舜臣点到王耀、徐勋、穆衍等将校,一个个都是士气昂扬,渴求一战。

“打到沙州,打到玉门关!”

“打到沙州!打到玉门关!”

“都军,你下令吧!”

“好!这才是顶天立地的好儿郎。”王舜臣拍手笑道,“不过还有鹰犬可用,不用全部我们自己出手。”

动员了麾下将士,王舜臣便又下令召集了凉州地界内的汉蕃豪门。河西一地,无论是吐蕃部族,还是汉人的大户,都是有私兵,人数还不少。

王舜臣在凉州说一不二,半日之后,他要找的人都到齐了。

前几天,刚刚进凉州城时,几位汉家家主的穿戴跟吐蕃人没有两样,不过这几天全都该回了汉人应有的装束。

王舜臣开门见山:“朝廷命本将收复凉州,如今虽然夺下了凉州,但功劳太少,不够下面的儿郎分的。尔等新近归附,亦是寸功未立。”

蕃部族长、汉家家主们交换眼色,心知肚明这是要他们出兵助战。

王舜臣也懒得骗他们,也不打算征求他们的意见,“所以要你们跟着官军一起出阵。不过本将也不白用你们,按照军中惯例定了个方略,出兵之后,但凡攻下来的村庄、城镇,党项、回鹘的丁口子女尔等可自取。至于府库财物……则是官军的。尔等也可以放心,无论攻城,还是野战,都由官军来解决,用不着尔等动手,尔等只要防着西贼逃窜就可以了。”

厅中一片静寂,基本上没人会相信王舜臣的话,但王舜臣完全不去在意他们眼中的疑虑:“不过丑话说在前面,惟汉人不可动分毫,谁胆敢故犯此禁条,族诛!没有二话。”

王舜臣的威胁实实在在,却没人敢不信。

“好了,有谁不愿去的,尽管可以站出来。”

没有人这么蠢,跳出来给王舜臣机会。

“王将军,可是当真要讲户口分给小人?”有人问道。

“俺们要党项回鹘的人口有屁用!城池、土地占下来,斩首多少就无所谓了!”

又是一阵眼神传递,至少这几句是可信的,如果当真能成事的话,差不多能有个三五千户来各家瓜分。

威逼利诱的手段,王舜臣做得虽粗糙,但他身后的大宋,让人不敢违逆。两天后,汉蕃各族点集了兵马,王舜臣留了千人守城,便一路向西北杀奔过去。

王舜臣在马上前行,千军万马伴在他左右,暗中握着拳头:‘好歹要多挣些功劳,否则日后都要低赵隆他一头了。’

ps:前几章写蒲宗孟忘记了袁绍田丰的故事,引起了一些朋友的议论。但在宋人笔记的记载中,苏轼在省试时杜撰‘杀之三宥之三’的典故,之后被欧阳修询问,他说是典故出自三国志孔融传注中,修了新唐书和新五代史的欧阳修回去还要查书才能确定苏轼是胡扯。苏辙写文,也有连题目出自管子注都想不起来的情况。

不过唐宋八大家偶有疏漏,的确不代表蒲宗孟也一样。尽管他连扬雄写剧秦美新拍王莽马屁的事都忘了——蒲宗孟盛称扬雄之贤,上作而言曰:“扬雄著剧秦美新,不佳也。”——可毕竟在列传中说过他有史才,所以还是修改了一下,免得被朋友说太小瞧翰林学士了。

