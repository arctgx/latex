\section{第28章 夜影憧憧寒光幽(一)}

冬日难得的艳阳天,阳光照得人暖洋洋的。就是天空有些浑浊,被北风激起的黄土灰尘遮得天际朦朦胧胧,如同蒙了一层澄心堂的透色竹纸,泛着暧昧的灰蓝。积雪也被浮灰掩盖,白雪皑皑的山头上变成了一片昏黄,四野里找不到一丝绿意。

已是冬闲时候,乡村里的生活平静而单调。下龙湾村的村民们到了年终,逢着天气好的日子,要么聚众赌博,要么就是在晒谷场上摆下龙门阵,闲扯一番。

韩家的三秀才,是如今村民们最好的谈资。村里的里正李癞子,原本在村民心目中,已经是个惹不得的角色;他的亲家黄大瘤有着如狼似虎般的凶狠,比李癞子还要让村民们恐惧;至于两人的后台,号称一手遮天的押司陈举,跺下脚秦州城就要抖一抖,连历任成纪县尹都要避让三分的奢遮人物,在没多少见识到下龙湾一众乡人眼里,那是天老大、皇帝老二、陈举排老三。

但这些个狠辣角色,在刚刚病好韩家的三哥面前,却是土鸡瓦狗一般。李癞子不合为了三亩地跟韩家起了争执,惹怒了韩三秀才。他一出手便让黄大瘤死无葬身之地,再出手使得陈举家破人亡,甚至给两人都安上了个里通西贼的罪名。

村民们虽是淳朴,却都有着农民式的精明,根本不信陈举、黄大瘤会跟西贼有何联络,都知道这是韩家的三秀才做的手脚,少不得竖起大拇指说声秀才厉害,而等到韩冈要当官的消息传来,又改成了韩三官人本事。每天都有一堆人在晒谷场上,把乱七八糟、不知从哪里来的内幕消息说得口沫横飞,好不热闹。

不过这几日,陈举一案开审,据说十里八乡的村民都涌去了城中,采办年货的同时,顺便看个乐子。下龙湾村的村民们也没例外,倒让村中清净了不少。

陈举的口才了得,又做了三十年胥吏,对法令规条了如指掌,不是靠着诗词歌赋得到官职的儒生可比。在前次的审案中,他几句话就让主审此案的节度推官丢了大脸,让大堂外的看客们大呼过瘾。

但他最大的罪行就是数十万贯的家财,陈举不死,秦州城中涌上来的恶狼,谁也不能安心的分赃。谋叛的罪名,他口才再好也洗脱不去。谋叛在十恶不赦的重罪中排在第三位,仅次于谋反和谋大逆。按刑律是定案即斩,不必等待刑部和大理寺的复审,用此时的说法,唤作‘真犯死罪,决不待时。’

平常的死囚,都是要等到秋后处决,运气好的,其间遇上皇帝大赦天下,便能逃出生天。而韩冈栽给陈举的是‘决不待时’的死罪,定罪之后,便当即拖出去处决——也即是死刑立即执行——连通过京城后台翻盘的机会都不会给他留下。

既然陈举再无可能翻身,韩冈便没兴趣学着村民,跑去看个热闹,若是给人留下行事轻佻,不够稳重的印象,那就得不偿失了。闲暇时不是读书,便是习武。这一日,他拉着表兄李信,找来了王厚、王舜臣和赵隆,一起校验起武艺来。

噌噌弦响,长箭在空中连成一线,仿佛珠链一般,直落三十步外的箭垛,转眼之间,箭垛上便长出了一丛野草花。由稻草扎成的箭垛有水桶桶口一般大小,但王舜臣一口气射出的十二箭,却是密密麻麻的扎在了箭垛中央只有碗口大小的一块地方。

“如何?!”

王舜臣得意的回头,他连续射出十二箭,连大气也没喘一下。以肉眼都跟不上的速度,用着一百二三十斤的力道,还保持着准头,王舜臣的这连珠十二箭,神乎其神,世所罕见。第一次见到这般箭术的王厚看得目瞪口呆,而早有见识的韩冈,也是一阵惊叹。

“李广、养由基也不外如是,当是能与刘子京一教高下了!”王厚摇头叹着,放弃了上场表演的念头。他也是练过箭术,可在王舜臣的衬托下,却连个笑话都算不上。转而问韩冈:“玉昆……你要不要试试?”

“小弟就不献丑了……”韩冈也摇着头。自己病好后,经过仔细调养,拉开一石三斗的战弓轻轻松松;论准头,三十步外的箭垛,也能十中七八。以他如今的气力和射术,放在禁军中的上四军里,都能算是十里挑一的人才,但王舜臣的箭术,当是万中无一。

连珠急射,比起单箭慢射,保持准头的难度不啻十倍。如王舜臣这般,一口气射出十二箭,还能保持着始终如一的精准和力道,韩冈估计即便在拱卫天子的御龙弓箭直中,怕也寻不到能与他一较高下的神箭手。他想着是不是找个机会,向王舜臣学个几招。君子六艺——礼乐诗算御射,自己做不得诗赋,也只能靠其他几项弥补一下。

王厚、韩冈自认不如,王舜臣更加得意,扬着下巴用眼底瞧着李信。赵隆有多少本事他很清楚,就是韩冈的这位表哥有几斤几两,他倒想着探探底。

李信不动声色,走到一边的武器架子前,取下七支投掷用的短矛。转过身,一支一支整齐的插在脚下。只是他对着的方向,并不是箭垛,而是校场另一头的树林。

王厚偏过头,问着韩冈:“玉昆,令外兄要做什么?”

“先家公【外祖父】掷矛之术旧年在凤翔府也是小有名气,阵上斩获不在少数,就不知传下来几成?”

韩冈仔细看着李信的动作,他也没有见识过李信的真正实力。这些天来,他的这位二表哥都保持着军人世家的习惯,早晨起来便打熬筋骨,习练武艺。性格倒不似韩阿李那般火爆,一贯的沉默寡言,韩冈只在小时候见过他两次,记忆早就模糊了。但能在王舜臣的精彩演出之后,还是一副气定神闲的模样,当是有些成算。再看自家使得一手好擀面杖的老娘,可知外公家家学渊源着实深厚,让韩冈对自己的表哥充满信心。

李信从脚下拔起一根短矛,轻轻掂了一掂。没精打采的一双眼睛突然瞪起,精芒四射。一声大喝,他左脚猛然跨出,右臂用力一挥,一道流光直射向树林。

李信的个头在关西算是中等偏下,比身高仅有五尺两寸的王舜臣只高出一指,身材又没有王舜臣那般雄壮,与韩冈比起来都有些瘦弱。不过相貌普普通通、丢进人海里便再也找不着的李信,两条胳膊的气力却是惊人,短矛一掷,竟然发出劲弩离弦的尖啸声。

第一支短矛如流光追影,脱手而出。他右手又向下一探,另一支短矛便出现在掌中。再一声怒吼,第二支短矛紧追前支短矛之后,射向树林。李信一喝一掷,只眨了几眼的时间,插在他脚前的七根短矛便消失无踪。短矛破风呼啸倏起即落,紧随着夺夺几声连响,七支短矛竟然扎在三十多步外的一株白杨上,从上到下排成了一条直线。

“好功夫!”王厚一声大叫,王舜臣也惊得两眼瞪大,不由自主的卸下了自负的表情。

韩冈走上前,抓着插在树上的矛身晃了晃,却动也不动一下,牢牢地钉得死紧。王厚惊奇的咦了一声,也凑上前仔细查看。坚实的白杨树干上,矛尖竟然深深的陷了四五寸下去,难怪晃之不动。王厚又惊又叹地回头看了看神色自若的李信,他灌注在矛身中的这等力道,即便是西夏最为精良的精铁瘊子甲,怕也是一矛掷过去,便能扎出前后两个对穿的洞来。

论箭术李信应该不如王舜臣——话说回来,秦凤路上箭术能比得上王舜臣的,恐怕一个巴掌就能数得完,说不定能与有神箭之称的西路都巡检刘昌祚、也就是方才王厚所说的刘子京一较高下——但李信露得一手,却也不比王舜臣差上一星半点。

王舜臣和李信一番试练,都是顶儿尖的一身好武艺,军中也是难得一见,就只剩下赵隆尚未出手。赵隆也不等催促,大笑着上前。拎起两个二三十斤的石锁,双手一振,石锁便呼呼的上下飞舞起来。

沉重的石锁在赵隆身侧翻飞如蝶,交缠如梭。风声呼吼,扑面而来,势道猛恶,王厚都不禁退了半步。但他看着身边的韩冈纹丝不动,又很不好意思的站了回去。

韩冈是被赵隆震住了。他看赵隆的身形动作,并不是随手耍弄的招式,而是一套汹涌澎湃如长河巨浪的剑舞。两具石锁加起来怕有五十斤重,但在赵隆手中直如同拈着两根绣花针。石锁卷起的道道旋风如雄狮咆哮,可赵隆硬是打出来一股长河浪涌绵绵不绝的感觉,双手上没有千百斤的气力,哪能有这般让人惊心动魄的演出。

结束了一套滔滔长河的剑舞,赵隆将石锁轻轻放在地上,呼吸微微急促,面皮略略泛红。他抱拳笑道:“俺的箭术不行,就只有一把子牛力气,倒是献丑了。见笑!见笑!”

“哪儿的话!?”韩冈笑道:“赵兄弟以石锁为剑,一套剑舞,让我等大开眼界。若这也算是献丑,天下又有几人的武艺能见人?”

看过王舜臣、赵隆和李信的试手,王厚也是喜不自胜。三人的武艺都是一等一的出众,为他生平所仅见。

王舜臣和赵隆已被王韶调到经略司中奔走,王舜臣因功升做三班差使,赵隆也委了殿侍,虽然两人还未有品级,但距流内品官也没多远了,只要稍立功勋,很快就能把他们抬举上去。现在又添了一个李信,而且还是韩冈表兄,更是亲近。日后父亲王韶兵发河湟,有这三名虎将在侧,再加上韩冈的智计谋略,当是又添了几分成算!

ps:高手云集,这是兲之气。

今天第二更,求红票,收藏。

