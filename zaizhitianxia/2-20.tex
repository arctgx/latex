\section{第七章 惊闻东邻风声厉(中)}

【第二更,求红票,收藏】

见高遵裕问起,韩冈便上前一步,躬身行礼:“下官韩冈,拜见提举。”

高遵裕立刻瞪大眼睛,一副吃惊的模样:“韩冈?!你就是韩玉昆?”

看着高遵裕一脸惊容,韩冈忽而想笑,这厮装得未免太过了一点。韩冈这个名字早就传出去了,王韶身边的得力干将,高遵裕来秦州沾光,如何会不打听?还装出这副吃惊的模样,是在拍马屁吗?……但高遵裕拍自己马屁是个好现象。韩冈现在至少有六七成的把握,确定他不是来拆台的。

“贱名有辱清听。”韩冈自谦着。

“久仰大名啊!”高遵裕亲切的拉起韩冈的手,对着王韶笑道:“今次遵裕奉旨来秦州之前,依例入宫陛辞。当时听了官家说起不少关于河湟拓边之事。官家还说子纯你是他由卑官亲自拔擢而起,必不会辜负圣意。吾观天子之意,实是对子纯你寄予厚望啊。”

听着高遵裕说起天子的知遇之恩,王韶眼眶顿时红了,颤声道:“天子厚恩如海,小臣粉身难报。”

高遵裕转头又对韩冈笑道:“而玉昆的名字,官家也是提到了,说子纯至秦州一载,方才荐了你一人,如此谨慎,玉昆必是有大才的。正巧吕吉甫当时也在场,还说起你前日上京的时候救了章子厚之父一命,又不留名而走,让章子厚之父一直追到驿馆里。天子听说后,对你是赞不绝口,说你不输古之侠士,当真难得。”

韩冈低下头去,虽然学不来王韶眼圈说红就红,但声音中却是带了一点感动的颤声:“天子之誉,韩冈愧不敢当。敢不效死,以报天子之恩。”

“河湟之事有子纯亲领,玉昆赞辅,大功告成指日可待。遵裕德才鄙薄,承圣意而来,也不过是为此事拾遗补阙罢了。”

王韶真心的笑了起来,听了高遵裕的这番话,看起来他今次到秦州,当真是来帮忙的,而不是过来捣乱。这让在秦州独力支撑了两年的王韶,心中感动万分。

有了能在天子面前说得上话的高遵裕,李师中、窦舜卿之辈便不足论。如此,还有什么能阻碍他高歌猛进的呢?!

王韶亲热的拉起高遵裕的手臂:“公绰远来必然疲累,还请早点入城歇息。今夜还有宴席为公绰接风洗尘。等明日开始,便要劳动到公绰辛苦了。”

“为国岂敢当称劳?子纯却说得太见外了。”

王韶亲手扶了高遵裕上马,跟韩冈一起随着高家的车队往秦州城里去了。

只是他们后来一番话中却忘了一桩迫在眉睫的大事,等到半个时辰后,王韶陪着高遵裕一齐走进了秦州城,便听到一阵点兵的号角声激荡在城池内外。

“对了,西贼攻打环庆了。”

虽然奉了天子诏的高遵裕今天抵达秦州,但来自环庆的急报,让秦州城里的空气一下紧绷了起来,转移了所有人的注意力。原本应该为高遵裕举行的接风洗尘的宴会没人再提,当天夜里,李师中就匆匆带着两千兵往陇城县去了。

位于藉水、渭水交汇处的陇城县,是秦州真正的枢纽,比起藉水边的秦州城,战略地位更险要十倍。驻兵在陇城县中,可以随时沿渭水西去,支援甘谷城,也可以径直北上,援救环庆路。

每次西贼入侵,秦州城里都会分兵去陇城驻屯,并让主帅坐镇其中,以期能随时出动援救。

不过按道理说,领军出镇陇城的该是身为武臣的兵马副总管或是钤辖。但今次窦舜卿很及时的生了病,躺着病床上,拉着李师中这个兵马都总管的手,涕泪横流的恨着自己今次不能上阵杀敌,然后说着一切都拜托了,把事情一股脑儿的全都丢给了李师中。

至于向宝,他倒是想领军出城,好证明自己还能带兵,但谁也不敢冒这份险。一场中风后,向宝的政治前途在眼下的确是没有了希望,即便他病好,也得去京中一趟,让天子做了确认才会被再次重用。

这一夜,韩冈留在衙门里值守,王韶也留在衙门中,连向宝都让人搀扶了来,坐在他的都钤辖官厅中,只是没多少人理会他。

一队队巡城甲骑的马蹄声在街巷上一夜不停,更夫在城中也转得更急。而城头上,灯火连天接地,守在城上的戍卒比往日多了数倍。各自提着刀枪,一队队的围着城墙绕着圈子。

缘边战事一开,不论是哪一路,全关西都会被惊动。这不是一次两次了,而是年年如此,去年韩冈的两个兄长便死于战事,今年还没过一半,又是十万大军攻环庆。秦州如此紧张也是正常现象。

不过今次却是白紧张了。秦州城中连着战备了七八天,可最后还是风声大,雨点小,攻打环庆的党项人只能算是武装游`行,根本没有打上几场硬仗,便退了回去。韩冈反倒是听说李复圭又派兵去追杀退走的西贼,又攻进了西夏境内。

“这是将功赎罪吧?”韩冈坐在王韶的官厅里,跟王韶说着话。

“李复圭的罪是赎不清的,他多半还是会推到他的手下人身上。”王韶还是对李复圭的人品不屑一顾的态度,“西贼主力应该还是在横山那边,环庆这里说是十万,但能有两万就了不得了。别看李复圭追得欢,这两万人他都对付不了,他绝不敢再硬拼。”

“李复圭这一败,我们秦凤还有绥德城那边,可都要受连累了。”

王韶冷哼一声:“你担心绥德城作甚?绥德城就是个钉子,死死钉着穿越横山的无定河。西贼出横山攻鄜延的道路由此被钉死,而横山诸多蕃部,也被牢牢钉在山中,再不能随西贼倾巢而出,天子对此看得肯定清楚得很。我们还是担心一下自己吧。”

“我们不是有高提举吗?”韩冈笑道。

只是李复圭的失败,还是惊动了京城,很快京中便传来消息,翰林学士韩绛升任枢密副使,出京宣抚陕西。而环庆那边,李复圭让人带兵杀入西夏境内,不敢去动西贼主力,却把边境的几个村子给屠了,拿着老弱妇孺的首级回来充功劳。他这一手,惹得党项人大怒,又带着兵压回了环庆,把李复圭又吓得向临近各路求救。

环庆战事的几次反复,韩冈都懒得提李复圭那个蠢货,反倒是朝廷任命的陕西宣抚使让他起了兴趣,宣抚使之位犹在安抚使之上,而陕西宣抚顾名思义就是能管着关西五路的,“韩绛?”

“就是韩亿韩忠宪的儿子。”大概是以为韩冈没听说过韩绛这个名字,王韶为韩冈解释了一下他的身份。

韩冈笑着摇头:“灵寿韩家,我怎么会不知道。只是韩忠宪八子虽皆为显宦,却没听说哪个带过兵。韩绛名气虽大,但也没听说过他有过领军出战的经历。”

“天子信重,知人善用就够了,也不指望他真的能带兵上阵。”

“陕西宣抚使……”韩冈突然觉得有些事情的确好笑,“韩稚圭当年的位子,现在轮到韩亿的儿子坐了。真是风水轮流转啊……”

韩冈由于姓氏的原因,对于韩琦、韩亿多有了解——倒不是为了攀亲,只是同为韩姓而稍有兴趣——虽然两家是同姓,但关系却不算好。

韩琦和韩亿,两人死敌虽算不上,却也并不和睦。韩琦年轻时曾经把韩亿一脚踢出了政事堂,即所谓的片纸落去四宰执。韩琦是踩在韩亿的头上成的名,当然韩亿和他的几个儿子对韩琦都不会有什么好感。

“对了,玉昆。你可知道韩亿的长子也是叫韩冈?”

“此纲非彼冈,那是纲纪的纲。一为山,一为丝,一个硬,一个软。韩冈虽不才,但胆子可没那位的软。”

王韶哈哈笑着:“说得也是,那位韩纲庆历时知光化军,恣擅威福,御下严苛,可遇上兵变就吓得弃城而逃,这胆子倒真是跟玉昆你不能比。”

这几天王韶很明显的心情变得轻松起来,没事还能跟韩冈开开玩笑。真要论起原因,一个是李师中去了陇城县压阵,窦舜卿又告了病,而向宝现今又没人理会,秦州城内压在王韶身上的压力少了许多,另一个,就是高遵裕的功劳,没事就过来催着王韶做事。对河湟托边的事情,比王韶还要热心得多。

今天他便又转了过来,找着王韶道:“子纯,韩绛也好,李复圭也好,他们打他们的,我们做我们的。总不能环庆、鄜延那边打起来,秦凤这边就不做事吧?你还要在秦州城里待多久?蕃部那里不去多走走,他们少不得会与朝廷离心啊。”

王韶叹着气,“公绰,不是我不想走,实在是走不得。张守约去了京城诣阙,甘谷城群氓无首,如果西贼再次攻来,要调也只能调古渭的刘昌祚。那时候,我都得去古渭压阵!还是再等几天,”

王韶一番推搪,让高遵裕很不高兴的走了。韩冈在旁边看着摇头苦笑。李师中、窦舜卿那般添乱当然不好,但这高遵裕太急切了也让人头疼。

这时一份急报被送了进来,王韶展开一看,脸色为之一变,转而又冷笑起来,他将急报递给韩冈:“李复圭当真把事情全推到他手下身上了。玉昆,你上次提到的种家老四种詠,今次被李复圭栽了罪名,前几天下狱后,已经瘐死在狱中了。”

