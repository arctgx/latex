\section{第23章 铁骑连声压金鼓(九)}

【第一更,求红票,收藏】

三百多骑兵急驰在山谷间,夏日午后暴雨时经常听到隆隆滚雷,在谷地中回荡。蹄声激扬如战鼓,让人血脉为之沸腾。

骑手们因为身上衣袍和甲胄的不同,明显的分作了前后两拨。跟随王舜臣出城一众骑兵,正处在被追杀的狼狈境地。身后马蹄声越来越近,充斥在耳间,伴随着吐蕃话的呵斥声,越发转急起来。

王舜臣押在队尾。前面是他所统率的几十名汉家骑兵,而背后,就是被他的盯梢战术弄得火冒三丈的吐蕃蕃骑。

自禹臧花麻撤退后,王舜臣便率领手下仅有的不到百名的骑兵,追踪着撤走的吐蕃人。按照他的战术,敌退则追,敌回则退,始终保持着百步以内的距离。而且他们在追击的过程中,一边用着硬弩向前攒射,一边高声叫骂和嘲笑。

尽管过程中并没有造成什么伤亡,但以王舜臣为首的这一群苍蝇,他们的精神攻击,已经成功的让禹臧家的队伍延缓了撤离的脚步。在这期间,吐蕃人几次派兵来驱赶,都被王舜臣躲了过去,但等到他们回到队伍中,牛皮糖一般的汉人骑兵马上又跟了上去。到最后,忍耐不住的吐蕃人终于派出了三百名精锐骑兵,气势汹汹向王舜臣他们反扑回来,誓要把他们追杀到底。

极速的奔驰中,迎面而来的狂风在耳畔呼啸,但王舜臣已经能模模糊糊的听到身后追兵的喘气声。最多还有三十步的距离,便会被追上。在追逐狂奔之中,吐蕃人射来的箭矢漫天飞舞,却没有一支能命中他们的目标,不是远远的飞脱,就是被他身上的甲胄、还有搭在马身上的防箭毛毡给挡住。

头顶上突然铛的一声响,一支长箭射中了王舜臣的头盔。一阵冲力传来,他的脑袋便是向前一低。紧跟着,从背心处又感受到几次微不可察的冲击。

王舜臣的身体因为驭马狂奔而变得火热起来,唯有心头保持着一片被冰冻过后的冷静。察觉到身后的敌人已经近得足以瞄准好自己,他有着临战前的紧张和兴奋,却完全没有半点恐惧的之心。

双手手持马弓,急促的呼吸逐渐调匀,双腿紧紧夹着马腹,身体随着胯下坐骑起伏不定,但拿着两尺短弓的双臂,却慢慢稳定下来。王舜臣的呼吸越来越平稳,而眼中的神采也是越发的闪亮。

王舜臣轻拨弓弦,他在骑射中的射击精准度要比步射时差上许多,但如果瞄准的是战马的话,却也照样能百发百中。双腿夹.紧坐骑,王舜臣突然拧身便射,一箭离弦而出,无巧不巧的扎进了追得最近的一匹战马的鼻子内侧。

如果仅仅射中了身体和头面,从六斗上下的马弓射出来的箭矢,只能给皮厚肉糙的战马带上一点皮外伤,让战马受到一点很快就能恢复的惊吓。但射中了鼻中最为敏感的嫩肉,情况那就截然不同。中箭的战马惨嘶声中人力而起,把马背上的骑手掀翻在地,甚至还路上团团转着,将后面的同伴给阻挡。

虽然通向大来谷的道路至少有着三丈宽,但这匹伤马在队伍的最前方发了疯般的乱窜,追击中的队形顿时连锁般的乱成了一团。王舜臣的这一箭,就像把柴束丢进河堤缺口,试图挡住河水在决口处奔涌,却没想到竟然真的成功。

趁此良机,王舜臣瞬间勒马止步。踩着马镫在马背上站了起来,双手中的马弓在眨眼间,已经换成了步射用的长弓。有了还算稳定的立足点,王舜臣再一次展露了他冠绝三军、出神入化的射术。

受命追杀王舜臣一行的吐蕃军官,正催着手下人将那匹发了狂的战马弄开,一支利箭便从张开的口中射入,箭头射穿了软腭,顶上了颈椎,雁翎翎尾摩挲着双唇,把他的咆哮堵在了喉间。吐蕃军官眼神中满是难以置信,他抬起颤抖着的双手,想拔出嘴里突然多出来的异物,但转瞬间,他就从马背上翻倒了下去。

还没有等周围的吐蕃人反应过来,弓弦再次鸣响,王舜臣竭尽全力,一口气连续射出了十一箭。穿颈、破喉、钻心,爆发般的射击,让王舜臣的双手差点都麻痹,但一箭箭无不命中要害,一片惨叫声过后,让他又多收获了十一份战绩。

三军可夺帅。

当作为全军的箭头,追在最前的一队人被王舜臣一人斩灭,而原本逃窜中的汉家骑兵又兜转了回来。两方对峙山谷中,尤拥有着数倍兵力的吐蕃人却反而是弱小胆怯的一方。

不过王舜臣对于冲击数倍于己的敌阵还有些犹豫,而吐蕃人也是因为顾忌着被少数敌军给逼退,而进退两难。

两边都是犹豫不决,看起来最后的结果当是失去战意之后,各自掉头回返。但烟尘飙起,地面在颤动,从星罗城的方向传来的动静,却成了压倒骆驼的最后一根稻草。一众蕃骑终于退了,追着他们的主力而去,慌张的仿佛在逃命。

“是援军!”

“是援军来了!”

麾下骑兵们的欢呼声中,王舜臣终于明白,禹臧花麻究竟是为何而匆匆撤退。

几刻钟后,王舜臣迎向了领兵来援的主帅。

“王舜臣拜见机宜!”他在韩冈马前躬身行礼,端端正正的摆出了下属拜见上官的态度。

“今次王兄弟你做得好啊。”韩冈跳下马,搀着王舜臣,笑意盈盈的夸奖着:“苦守孤城,最后还能有胆气出来追击,军中可是少有人能比得上你。而且若没有王兄弟你坚守星罗结城,禹臧花麻就能全力攻击渭源堡。如果情况变成了那样,也许堡子最后能保住,但守着营垒的苗都巡那里,可能就要出事了。这一战的关键,可是靠着王兄弟你的奋战!”

“多谢三哥夸赞。俺也只是运气而已”王舜臣把韩冈的夸奖照单全收,仰着头笑得开怀尽兴,

在这个时代战场上,将领对战局的掌控有很大一部分得依靠猜测和推算,而战事的成败,甚至更多的还要倚重于运气,王舜臣说他是运气倒也没错。韩冈是从星罗结城赶过来的,虽然仓促,但该问的他一点也没有少问,王舜臣如何守的城池,韩冈已经了如指掌,若非自己到得及时,说不定城就会给攻破了。但王舜臣在这段过程里所表现出来的能力和才干,却是当得起韩冈的赞许。

“三哥,现在我们该怎么做?”王舜臣问着韩冈,他现在还沉浸在一人射落十二骑的兴奋中,“要不要追上去,好歹从禹臧花麻身上咬一块下来。”

韩冈则保持着冷静,“牵制住禹臧花麻就够了,不让他们走得太快,等渭源堡的援军来了再说。”

瞎药这时从前方转了回来。前面韩冈汇合了王舜臣后,便命他向前去追踪禹臧花麻,要尽量拖延他撤退的速度。对于韩冈的吩咐,瞎药现在是如奉纶音,不敢有半点违抗,都精心尽力的去完成。

现在他从哨探口中听到了到了禹臧花麻的消息,就立刻他恭恭敬敬的跑过来,对韩冈道:“启禀机宜,禹臧花麻已经在大来谷口处停了下来。好像是要构筑营垒的样子。”

“在大来谷口筑垒?!”

韩冈和王舜臣顿时都吃了一惊。禹臧花麻又不是汉人,他是吐蕃人。蕃人的营垒都是以脆弱著称,吐蕃人也不例外。都是一冲即破,毫无守御的价值。不比宋军,在军事工程方面的能力独步于世,造出的营垒,比起一般蕃人小城都要坚固得多。禹臧花麻临时修造营垒,而且还是位于大来谷口,如果不是突然变成蠢货,那么就是他别有一番心思。不过不论禹臧花麻的本意为何,他的这番行动,分明是在邀请韩冈去攻打他。

韩冈首先冷静下来,下令道:“盯住大来谷,如果他们是真的要筑堡,立刻回来通知我。”

瞎药领命退下了。王舜臣扯了扯韩冈,冲着又跳上马的瞎药背影呶呶嘴,问道:“三哥,这是怎么回事?怎么变得老实听话起来了?”

“人总时会变的。”韩冈看了对自己恭谨有加的瞎药。瞎药现在被自己所慑服,在短时间内,他心中的阴影不可能消退,在自家面前都会俯首帖耳、老实听命。

有着瞎药做耳目,韩冈对前面的事了如指掌。

禹臧花麻已经停止了撤退,他若是想就此行军急退,逃肯定是能逃掉,但他在族中和国中的威望。可就要一落千丈。就算禹臧花麻再如何想回师,也必须占到点便宜后,才能安心的回返。

“这是禹臧花麻在将军。他已经将了我一军了,现在竟然想着还要引诱我去上钩。”韩冈明白,如果自己不去迎战的话,禹臧花麻就能对他的族人们说这是宋人在害怕,不敢应战,然后大摇大摆的撤离。

韩冈不会让禹臧花麻的盘算得逞,他要让今次入侵的贼人付出足够的代价。

