\section{第八章 朔吹号寒欲争锋(三)}

虽然说夹袋里缺人,韩冈只能安排几个亲信进中书,做为自己奔走的吏员,韩冈也不担心他会被手下的僚属们给架空了。

各房检正、提点,大半是蔡确和曾布提拔上来的,将他们从人人称羡的堂官,送去开封府大狱与叛逆作伴,也只消韩冈说上一句。

韩冈不屑杀鸡儆猴,可若是有人想要试一试他新官上任的三把火有多旺,他也不介意拿人来试一试火候。

现在谁都知道韩冈若是脾气拧起来,就又是一个拗相公,再有了太后的支持,王安石贵为平章军国也没辙。

而中书五房检正公事张安国,就不太方便下手。其乃是王安石的门下客,诗文往来,韩冈都与他见过好几次。几年前他便是是刑房检正,之后出外做过一任通判回来,就坐到了中书五房检正公事的位置上。

之前张安国领着一队同僚来拜见韩冈,态度谦恭倒是谦恭,就是不知道他们心里怎么想。

气学的学子若是能早点大批入仕,韩冈就能轻松一点了。可惜正常的进士出身的官员,没有十余年的磨练,很难晋升京朝官。除非是才干突出,才能在十年之内走进朝堂。

选人、京官阶级的官员,其实也能充任各房检正,但那就不能叫做检正公事,而只能称为习学公事。

只是更加让韩冈感到无奈的是,他手上连充当习学公事的合格人才也几乎没有。

不过韩冈也不是太过担心。

既然他已经成为了参知政事,很快就会有人来投靠了。

吕惠卿、章敦、曾布,将蔡确、吕嘉问、李定、曾孝宽、李承之也算进来,这一干人,都可以算是新党的核心成员,或曾经是。

但他们没有一个是王安石的学生。曾布跟王安石的关系近一点,因为王安国的夫人就是曾布的妹妹。而吕惠卿、章敦、李承之等人,都是他准备变法之后,由朋友推荐到他手边的。

尤其是蔡确,当年韩冈第一次上京拜见王安石的时候,蔡确还没见过王安石,等韩冈第二次上京,蔡确才有机会跻身到王安石的身边。

而王安石当年在金陵教授的学生,基本上都是在治平、熙宁年间考中进士,如今最多也只有十余年的资历,想要进入朝堂高层,至少再有十年时间——他们中大部分人的进步速度,甚至远远比不上蔡京。

既然新党都是如此,韩冈也不会强求提拔起来的助手都是气学的成员。

反正只要认真做事,韩冈也不在乎他们到底是哪一派的出身。

“大参。”

配属给韩冈的堂后官在厅外通报了一声,就领着两名吏员进了公厅。

后面两人的手上,又是高高一摞需要处理和批阅的公文。

韩冈放下了手上的卷宗。他在处理私活上,耽搁了太多的时间,也难怪这位堂后官会过来提醒。

虽然韩冈也不是没有在处理着政务,但案头上的宗卷不见减少多少,

与天下所有的衙门一样,中书门下内的官员数量并不多,为数更多的是听候差遣的堂吏。

六百余名堂吏,由中书各房的堂后官管理。堂后官们的直接上司,是提点五房公事。堂后官和提点往往是吏员出身,正好与士大夫担任的中书五房检正公事相对应。

大宋四百军州,能够入流的吏员,平均每年只有二三十人。而中书门下的这些堂后官,就占去了吏员晋升官员的大部分份额。出身于吏员的行列,很多都是几代传承,就是韩冈也必须依赖他们来处理手上的公务。

但凡吏员要想对付新上任的官员,最简单的办法就是将公务全都堆上来,不论是积年的案子,还是新出的案子,都混在一起呈上来。先一棒子将人给打晕。让其望而生畏之后,吏员们就能上下其手了。总之就是下马威。

只是能够走进政事堂的都是从三万官僚、两千进士之中搏杀上来的英杰,早见惯了人事。或许有各种各样的毛病,但中书堂吏想要欺瞒、整治宰辅们,有没有胆子另说,成功率就低得可怜。

所以论起手脚干净,遵纪守法,中书堂吏跟其他衙门的胥吏没什么区别,但比起谨言慎行,中书门下的胥吏们却是其他衙门所比不上的。

韩冈也是初来乍到,尽管公务多得让他一时间差点手忙脚乱,但他并没有感受到吏员们的恶意。而且这些公文,也会先经过各房的检正官。

各房都要处理各自的一摊事务,也要对各项公事给出自己的意见,并不是按照发来的原样,全都堆在宰相和参知政事的案头。

刚刚送来的这些公事,不怎么重要的,直接就可以从韩冈手上打发出去。绝大多数的公文,韩冈只消看两眼,画上一圈或一勾就可以丢到一边。

这就跟韩冈接见官僚的情况相似。每天被引入政事堂中拜谒宰相、参政的官员,数以百十计。宰辅们平均接见每个人要是超过五分钟,今天就别做事了。所以基本上都是说上两句就送客。

不论那些官员为了拜见宰执,事前准备得多充分,也不论那些奏章在书写时,耗费了多少精神,几易其稿,在宰辅们这里,很多时候,都是不值得多看一眼,多问一句。

但有些重要的奏章,可就需要写上处理意见,然后进呈给太后。方才韩冈刚刚仔细看过的一份奏折,说沂州雪灾,冻伤人畜无数,急需朝廷赈济。韩冈写明了可以交由京东东路转运使来处置,但沂州必须及时上报伤亡情况,并在回忆了京东东路漕司和沂州的库存之后,又建议太后下拨两百本度牒,给予沂州使用。

而最为紧要的公事,则必须在东府所有宰执的手中走上一边,集中所有人的意见然后呈交上去——韩冈现在正在看的一封就是,事关黄河堤防,再小都是大事。

都提举河防工役的程昉上表奏闻,黄河下游内黄段北堤今冬整修时出现大面积的坍塌,可能之前修筑时偷工减料的结果,急待朝廷处置。

韩绛和张璪都表示由都水监派人去查看究竟,到底是过去修筑时的遗患,还是这一回整修不力造成的损坏。韩冈想了一想,提起笔,建议太后派人去现场体量——在这里,他与韩绛、张璪有着相同的意见——只是没提议派哪里的人。

韩冈一份份的公文、章疏看过去,不知时间流逝,有些昏天黑地的感觉。

其实他若要偷懒,也有的是办法。

简单的随手批阅,困难的就要带回去找幕僚一起处理。若是困难又紧急的公务,就要看情况,或是拉着几位同僚一起商议,或是干脆以奏论不明为由打发回去,让人重写。

否则以韩绛的精力,哪里能处理得了这么多公事?只是为了尽快上手现在的位置,韩冈才会不厌其烦的悉心检视奏章。

韩冈好歹是从一县之地爬上来的,县中、州中、京府中的位置都做过,路中监司更是转运使、安抚使、制置使的名号都挂过,那时候处理的公务,与现在比起来,不过是数量多寡,以及范围大小的问题。

但韩冈将桌案上的公事处理了一多半之后,韩冈的堂后官又进来通知他了,“大参,时间差不多了,相公和张参政都要往前面去了。”

自然,这位堂后官的身后还是有着两名捧着公文、奏章的胥吏,进来将这些公文放下,然后拿走了韩冈已经处理好的那部分。

看了看桌案上又堆起来了的公文,韩冈抬眼看着厅外,对面院墙在太阳下的影子,已经向着东北方,被拉得老长。

“都这时候了。”

韩冈长身而起,虽然手边的公文很重要,而且若不能及时处理,等他回来,说不定连桌子都看不见了,但他现在要处理的事情更为重要。

——作为统掌天下权柄的宰辅,中书门下的宰相和参知政事们,手中最需要处理的公务,主要还是人事。除了一些公事需要相互通个气,更重要的是有些职位,需要他们共同拟定就任人选。

不过当他走出门前,顺势回望了一眼,韩冈顿时无比怀念起他当初在同群牧使和判太常寺时的清闲了。那时候,他每天最多也只需要花上一盏茶的时间来处理公事!

韩绛和张璪都比韩冈早到一步,早早的就来到了宰辅共同议事的后厅。而检正中书五房的张安国也在其中。韩冈进门后,看见两人先至,便告了个罪,然后在自己的位置上坐了下来。

“玉昆昨日只是过了庭参,今天才算是初次处置堂中公事,可还习惯了?”

“还早。前些日子实在太清闲了,得过些天才能习惯下来。还得请子华相公和邃明兄多担待一阵了。”

韩绛的问话倚老卖老,韩冈也不介意。欠了一份人情是小事,关键是辈份和年纪差太远。说了又问:“今天有那几处需要除人?”

