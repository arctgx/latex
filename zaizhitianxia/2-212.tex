\section{第32章 吴钩终用笑冯唐(19)}

最后一批流放通远的罪囚过境秦州。新任的秦州知州,前陕西都转运使沈起,便遣了衙中僚属来帮着韩冈,将罪囚在城外的空营中安顿下来。

而随着雄武军节度判官一起到来的,还有阔别了许久的王厚。

看见王厚,韩冈又惊又喜,“为何处道兄会在秦州?”

“愚兄是来迎玉昆你的!”王厚笑道。他看了一眼韩冈身后的近三百名罪囚,还有数倍于此的他们的家人,“还有这最后一批流配通远的囚犯。”

好友多日不见,韩冈和王厚有一肚子的话要说,韩冈更是有许多话要问,不过节度判官就在旁边,韩冈在情在理也要先招呼好他。

当初州中的节判吴衍,于韩冈的大恩,不过由于在王韶和李师中之间站错了队,早已离开了秦州。现在的节判谢蕴,韩冈并不熟悉,与其寒暄了几句之后,本想就此送他离开营地。谁想谢蕴在走出营地大门,辞别时却道,“在下出城前,沈经略曾有言。若玉昆今日有闲,可往州衙一叙。如果旅途疲累,那就罢了,可等过后再说。”

话虽如此,但韩冈可不会不识趣,自高自大的让沈起等待。他拱手应道:“大府有招,韩冈哪敢不允。眼下正是有闲,当随节判同去城中。”

一个称呼经略,一个道着大府,对沈起几个官职头衔的取用,便体现了韩冈和谢蕴之间立场的不同。

王厚听着心中快意,韩冈对河湟之事的独占之欲,可不必他父亲要差了,“在下也随之一起入城好了,到时就在衙门外等着,等玉昆你出来后,正好去晚晴楼逛一逛。”

谢蕴脸色微变,却也不好阻止——王厚根本不归他管——而且王厚到了州衙门外,沈起也拉不下脸让他真个等在外面。

韩冈安排好随行的军队和囚犯,又遣人通知了周南一声,便跟王厚一起,随着谢蕴往秦州城去了。

沈起的大名韩冈早有耳闻。是朝中不多的会做事的能臣。他在长江口的海门县任知县的时候,曾经为了让沿海百姓不受海潮之苦,主持修筑了海堤百里。

韩冈在大宋官场上混迹逾年,很清楚以知县的身份能掌握的资源究竟有多少。用微薄的资源而修筑起百里海堤,以此时工程技术水准,沈起在政事上的手腕不言而喻——州中、路中应该没有给他多少支持,否则,功劳就不会算在沈起头上。

韩绛担任陕西宣抚使,为了能更好的保证前线的粮秣军需的供给,便将政务水平出众的沈起找来,让他做了陕西都转运使。而此次横山攻略,在后勤上,沈起领导的陕西转运司,没有给前线的大军添过一点麻烦,以此可见沈起的手段。

庆州兵变之后,在郭逵紧急被调任长安的时候,喜欢谈论兵事、在朝中也有知兵之名的沈起,由于正好身在陕西,所以被天子和政事堂给挑中,让他来镇守秦州重镇。

沈起对河湟开边有什么想法,现在还没人知道。但看他赶着招见韩冈,恐怕还是存了一点心思。

“不指望沈大府能对河湟开边有何助益,只要他不插手通远军中内事便可。”在谢蕴身后,王厚脸上挂着温文尔雅的笑意,但轻声道出的言辞却是冷峻无比。

“横山已败,关西也只剩下河湟了。天子如何会让人干扰?而且我还听说,秦凤转运司年内就会从陕西那里划分出来,到时候,秦州如何还能再拿钱粮干扰开边之事?”

当初韩冈在游师雄那里听说的仅仅是传言,但前日,他收到了章惇的私信,在信中却是已经确定了秦凤转运司的设立,等收过了秋税就开始组建转运司的衙门。

王厚明显也听说了此事,笑了笑:“好歹沈大府还是秦凤经略使。”

“经略使由谁做都无关紧要,等到要出兵的时候,缘边安抚司也可以变成正式的安抚司!”

“想不到玉昆也看出来了……家严也是如此说的。”

横山攻略功败垂成,能让赵顼扬眉吐气的也只剩下河湟这处偏师。有了天子的支持,三千叛军才能这么容易的被流放到通远军去。而凭借天子的支持,以通远军为核心,从秦凤安抚使路辖下,划分一个新的安抚司出来,也不是什么不可思议的一件事。

不论王韶,还是韩冈,都是看出了天子急不可耐的性子,心中底气十分的充足。

可能是看出了韩冈不会亲附沈起,谢蕴纵马在前,与后面的韩冈和王厚渐渐拉开了距离。

见谢蕴离得远了,王厚也不用刻意压低声音,“除去了驻军,原本通远军辖下的汉儿,也就只有五千一百余户,这还是把古渭……陇西县东面的永宁寨等十一处城寨的百姓,一起给算进来的结果。只论陇西到渭源这一条线,其实才一千三百户,七千余口。”

“被流放来的叛军总计可是有两千四百二十六户。”

——三千叛军中有兄弟、父子,所以户数少于人数,而陕西的一户人口往往能超过十人,寻常也有五六人,故而被流放到通远军的罪囚多达一万六千余名。

韩冈笑得得意,若非如此,他何苦要想方设法把这些叛军弄到手?要想化夷为汉,没有足够数量的汉人作为核心,怎么可能成功?

为了充实通远军的人力和物力,政事堂是把原古渭寨以东的十一处城寨都划归了通远军,其中就有以马市而闻名的永宁寨。所以通远军的户口还勉强能让人看得过去,如果只有古渭和渭源,一本册子就能所有汉户登记完毕了。可即便如此,通远军的户口还是不多,现在一下多了两千四百户,等于增加了全通远军户口的一半,或是陇西县【古渭】以西地区的两倍。

对于这些叛军,韩冈可是从来没打算把他们当成罪囚来使唤,都是看作了能充实通远军的重要的人力资源。两千四百户,在边地已经是一个县的编制。而且还都是有过战斗经验的精锐。即便过去是叛军,但在众羌环绕的河湟地区,不依附官府,可没他们的活路。韩冈也不怕他们有什么变乱。

“已经到了陇西的罪囚,安抚是怎么处置的?”韩冈问着王厚。

他可不希望他辛辛苦苦才要来的人手,被人糊里糊涂的全都弄废掉。虽然王韶和高遵裕应该不会做蠢事,但不问一下,韩冈也不放心。

“玉昆你放心……”王厚像是知道韩冈在担心什么,笑道,“愚兄离开陇西的时候,才到了第一批,两百多户。就放在古渭……陇西县城边上。剩下的几批则是会一点点向西排过去,住在沿河的护田堡中。至于玉昆你亲自押来的最后一批,家严和高钤辖准备安置在渭源。”

“这样最好!”韩冈对王韶、高遵裕的安排很满意,“这些人多有官身。能在西军中为将校,手上没点本事是坐不稳位子的。这几天我看了,他们的确是各个武勇了得,没有一个弱者。如果好生对待,让他们的戴罪立功,渭源将稳如泰山!”

韩冈见到了沈起。新任的秦州知州还安排下宴席来款待韩冈,还邀请了王厚,从他在宴席上的态度,看起来沈起的确有心于河湟。

但沈起不是郭逵。韩冈可以信任郭逵的指挥才能,甚至希望开战时,由郭逵统领大军——这比王韶成为主帅更为稳妥。不过,若是换作沈起,同为一介文臣,韩冈还不如去相信王韶的能力。

对于沈起在宴席上的试探,韩冈装着傻,哈哈笑着推了过去,这些烦心事还是交给王韶和高遵裕去处理。

第二天清早,大队离营启程。又用了七天的时间,韩冈一行最终抵达了目的地,回到通远军中。虽然与他离开时,城池和区划名字全都变了。但出现在山坡下的那座不算高峻的城墙,在韩冈眼中还是那么的亲切。

看着坐落在谷地中的城池,自韩冈以下的上千人,都不顾头顶上的炎炎烈日,全都在山路上加快了脚步。在道路上奔波劳碌了半个月,而且还是炎热的夏日,就算只是初夏,也已是让人已经迫不及待地想结束这段艰难的旅程。

从山路上下来,离着陇西县城还有很远的距离,一道尘烟出现在前方的道路上,一对快马迎了上来,却道是奉命来迎接韩机宜。

随着队伍的不断前进,一对又一对报信的快马冲到了韩冈面前,高声通报,皆道是奉命迎接韩机宜凯旋。到最后,离着县城还有三四里,两面大纛终于并排着出现尘头中。

王、高。

一见到两面大旗,韩冈立刻翻身下马,迈开脚步,迎了上前。

在王韶和高遵裕的马前,韩冈拜倒与途:“哪里能劳动两位安抚相迎,韩冈受宠若惊!”

王韶和高遵裕也立刻下马,并肩上前,把韩冈从尘土中扶起,高声笑道:“玉昆,这是你应得的!”

是的,这是韩冈应得的。

