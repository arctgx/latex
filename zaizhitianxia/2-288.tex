\section{第47章 百战功成朝天阙(下)}

已是盛夏时节。

七月的正午,太阳炽烈得仿佛能点起树枝。从早上起,就一点风都没有,热得连知了声都没了。鸡蛋落到地面上,立刻就能被烤熟。

京城中,除了要准备参加贡举的士子还会在呼朋唤友,其他地方都一派平静。前日因为熙河路护送木征上京面圣的轰动场面,也渐渐从士民们的话题上消失。现在的东京城中百姓们,除了羡慕之外,都在等着要看一看朝廷会如何安排今次的功臣。

此时,秦凤路的德顺军那里的战事也平静了下来。

德顺军的战事早在王韶回师后,就已经结束了。赶在调去熙河路的秦凤、泾原两路精锐回军之前,党项人从笼竿城下及时撤围。他们攻打了整整一个月,却也没有破开城池,西夏攻城手段之低劣仿佛在这一战中得到了印证。不过从真实的情况来说,是党项人对笼竿城围而不攻,在仁多零丁的率领下,他们打下了笼竿城外围的几个寨子,顺手赚了一笔。

现在整个关西安静得都让人觉得有些异常。吐蕃、党项,都老老实实的守在老巢之中,没有一个乱动弹的。虽然不知道他们暗地里有没有在打什么鬼主意,也不清楚他们是不是已经决定在秋后来打草谷。但至少可以确定一件事,他们眼下都没有了继续进攻的力量。

因为罗兀之战的损耗过大,西夏国力至今未复。而湟州的董毡,在宋人的兵锋之下也似乎吓破了胆,已经同意归顺朝廷。在河湟名气极大的智缘大师,现在已经成了董毡的座上宾。

会仙楼后.庭中的荷塘中,荷花盈盈,遍布池中的粉红花瓣被阳光直射着,反而更添了荷塘的三分颜色。而楼中一角,正有一个小小的房间凭栏而亡,正好能将会仙楼后.庭的风光尽收眼底。但房间中的两人都无意观看风景。一个四十岁上下的中年人,三缕长须垂下,看起来很有几分威严。另外一个是才三十不到的青年,神采飞扬,眼神灼灼。

“听说是韩冈提议,要让董毡的儿子阿里骨成为首个进入熙州蕃学的学生?”中年问着。熙河现在是一个蕃部接一个蕃部归顺。若能以董毡之子阿里骨入蕃学,必然能让河湟一带的所有蕃部的全数归顺,“但这不是人质吗?”

年轻人回答道:“阿里骨不是董毡的亲儿子,只是他的正妻带来的。董毡的儿子年纪都不大,但阿里骨却成年了。让他离开湟州,董毡必然会有几分香火情给我们。”

中年人搭着胳膊,“要镇住董毡,就是他的亲儿子也没用。只不过在必要的时候,阿里骨的身份也能派上些用场。”

“阿里骨若真的入了蕃学,肯定会引起一番议论。如果他能上京,怎么都能得到一份赏赐,一个官身。”

“蕃人得官容易,得到赏赐的机会却很少。”中年人道,“董毡的这个便宜儿子就算入京,当下也不会有太多好处。单是赏赐熙河、秦凤和泾原三路的参战将士,就要上百万贯。国库现在虽已充盈了,但也没多少提供给一个蕃人。”

“不世之功,当还以稀世之赏。上百万贯的赏赐又算什么。因为他的功劳,本来就是右司郎中的王韶,现在已经是升了右谏议大夫。”

中年人摇了摇头:“这不算厚赏!”

年轻人神秘的笑着:“等入京后,就知道他的赏赐厚不厚了。蔡子政【蔡挺】可是在西府中等着他呢!”

“枢密副使?!”听到这个消息,中年人立刻凑前了一点。

“同时又荫补了两个儿子的官,现在他排在前面的四个儿子都有了官身。押送木征上京的次子王厚,现在都是大使臣了——正八品的内殿承制。想想宰执家的儿子,他们得荫补也不过是正九品的太常寺太祝,京官而已。”

“王韶的这个儿子一直都跟着他,几年来立了不少的功劳,又赶上天子高兴,赠官也是等闲。”中年人听出了年轻官人背后的一丝嫉妒,举杯喝酒,遮住了嘴角的笑意。又问道:“那高遵裕呢?”

“改了岷州刺史。”

“岷州刺史?!他原来就是荣州刺史吧?”中年人奇怪的问着,怎么是平级转迁。疑惑中,脑中灵光一闪:“难道……!”

年轻官人点着头:“正是那个难道,高遵裕西上閣门使的本官的确是落职了。”

“那他不就是正任官了?!”

高遵裕原是荣州刺史,尽管与现在的同是刺史。不过不算品级,也不是正官,而是遥郡官,即是所谓的美官,只是好听的加衔而已。甚至一些老资格正七品的宫苑诸使,连横班都没入,照样能得个观察使、团练使的遥郡加衔。而正任官有多贵重,端看英宗皇帝就知道了,他正式成为储君前,虽然仁宗早已属意于他,也不过才是正任官第四级的团练使,比高遵裕现在只高一级。

高遵裕原本的荣州刺史,因为尚有西上閣门使的寄禄官在,所以仅是遥郡官,但他现在作为本官的西上閣门使被落职,那改封的正五品岷州刺史便成为了他新的寄禄官,也就是计算品级和俸禄的本官。

“西上閣门使是横班倒数第二级,现在他跳到正任刺史上,一下跳了五六级啊!”中年人为高遵裕加官进爵的速度感慨着。

小使臣,大使臣,宫苑诸使,横班,然后才是正任官,这是武将的本官官阶的迁转顺序。高遵裕原本站在横班的倒数第二阶上,地位已经很高了,还在当年在秦凤路任职的向宝之上。但已经身处如此高位,竟然还能一跳五六级,未免太惊人了一点。

“……多半还是靠了太后……”年轻官人消息灵通得仿佛能知道东京城的任何一个角落发生的事情,“听说前些日子因为市易法的事,官家顶撞了一下太后,现在回过头来就是给高遵裕加了正任刺史。”

中年摇了摇头,宫廷之事能不说就不说,虽然此处可算是私密,但毕竟还是公开场合,说不定隔墙有耳。

“高遵裕都成了正任官……那韩冈呢?抗旨矫诏的事都做下来,硬是保了河州半个月,不至于坏了河湟大局。现在应当少不了他的赏赐吧……”

“赏赐是有,从太子中允升到了国子监博士——只是若是他能有出身,那就是太常博士了——种菜园的韩冈之父,也得到了加官,说是指挥屯田有力。不过诏书中还命韩冈随着王韶上京诣阙,但却给他给堆了。”

中年听到最后一句却是皱眉不解:“该不会抗旨抗上了瘾……”

“韩冈是要去秦州参加举试,早就上请锁厅了,现在没时间上京。”

“这话说的,面见天子说不定能给赐个进士头衔。”中年人半开玩笑,从他轻松的口气看,也是不当真的。

年轻官人却一下当了真,顿时就变得严肃起来:“就算宰相的亲弟弟,要想被赐进士出身,好歹也要有几十卷的文章,韩冈有什么?沙盘、军棋、医药、还有争战、转运,这功劳算算倒是不少,但哪个能配上进士的?……进士科为国抡才,讲究的是一个‘文’字。就算天子要赐他进士,也得先过了御史一关,还得要学士院那里不封驳。”

中年人为着年轻官人的激动又笑了。不过他说的也有道理,王安国靠了五十卷文章得到一个进士头衔,私下里都没少人说怪话。韩冈一卷文章都没有,凭什么生受一个进士,有功劳,赐钱、赐物、加官便是,金榜题名的光荣,的确是拓边蛮荒所不能比的,更不能代换。

年轻人道:“韩冈三年忠勤王事,从布衣而入朝官,这是他应得的。可狄斑儿去了一趟广南回来,也没听说天子因为他平了侬智高之乱,给他一个进士出身。韩冈又如何够资格?”

“所以韩玉昆没有来京城,直接锁厅,准备八月去秦州。”中年人说着,“秦凤路中有心考进士的官员也没几个,韩冈本身还是有些才学,听说他当初入官时,在流内铨被人使了绊子,但考的墨义十道却全都对了……好歹一个贡生总能考到。”

“他能得王韶荐,自然也是有才学的。但这三年来,他又有多少时间攻读诗书经传?无暇读书,又岂能中上一个进士?!”年轻官人却把左手拇指中指一圈,其余三指一翘,摆出了兰花指的样儿。“若是进士这般好中,陕西诸路能一科才出那么两三个?”

“这话还是不要说了,说不定韩冈今科就能成为一个进士。”

“那就要看他的运气了。”年轻人不屑的笑着。

“是运气和耐心。”中年人为之更正,“韩冈的表现我看过几次,在年轻人中,的确是难得一见。”

年轻人又变得不服气了,“那就看看今次韩冈是否真的有运气和耐心!”

