\section{第15章 焰上云霄思逐寇(二)}

“射!”

随着一声大吼,是一声拖长调的号音,而紧接着就是噌噌弦鸣。一片飞蝗自宋军阵列上腾起,掠过七八十步的距离,一头扎进对面的敌军阵中。

列阵于归仁铺前的交趾士兵还没有反应过来,就被神臂弓射出来的箭矢扎成了一只只刺猬。声声惨叫扬起,还算整齐的战线上,一下就多了许多缺口。

尽管邕州潮湿尤胜荆南,但在潭州武库中保养得宜的这批神臂弓,还能将威力保持一定时间。而从对面射来的那等威力、射程,都只能用可怜来形容的弓箭,却根本都落不到宋军阵间。

双方远程武器的差距如此悬殊,登时就使得交趾将领失去了对射的勇气。同样是号角声响起,交趾军中一群步卒舞着藤牌杀了过来。虽然一起来攻的还有着两千蛮军,但他们被分在两翼,拖在后面。凸出在前的只有衣甲鲜明的宋军。

重弩上弦缓慢,只要顶住两轮,就能杀到宋军的面前,交趾军的军官们大声鼓舞着士卒们的士气。

“自作聪明!”韩冈、李信同时冷笑。不需要翻译,他们也知道对面的军官们在说什么。要是靠着单薄的盾牌就能防住神臂弓,这件武器也不会被称为军国重器。

战鼓按着节拍,受过训练的宋军弓弩手们,在鼓点声中,整齐的上弦搭箭,然后激射出去。三棱木羽劲矢离弦而出,如同撕开一张薄纸一般,很轻易的就穿透了藤牌。随着藤牌兵接近宋军军阵,弓弩的威力飞速增加,不光是穿透藤牌,同时也穿透了他们的身躯。

神臂弓上弦的速度也远远超过交趾将领的预计,踩着神臂弓前端名为干蹬的铁环,比起踏着弓臂要容易用力许多。宋军弩手三排轮换,连绵射击一点也不留下缝隙。

交趾军好不容易鼓足勇气的反击,在连续不断地射击面前土崩瓦解,只是宋军阵线在射击中踏前一步,就让交趾兵坚持不住,纷纷返身逃窜。

这里是当年狄青大败侬智高的战场。侬智高的主力就是在此处被赫赫有名的狄武襄用着他带来的八百骑蕃部骑兵来回冲垮,彻底覆灭了侬智高刚刚成立不久的大南国。

今天一方仍是身着红袍的宋军,而另一方,则已变成了更加靠着南面的交趾军。交趾兵多,宋军兵少,而且宋军还将两千广源蛮军分左右拖在后面保护侧翼,真正与交趾对垒的就只有八百将士,但宋军的敌人依然完全不是对手。

本来李常杰只放了一个指挥,又加入了从长山驿逃回来的败兵,还有刚刚从后方调来的三千步军和四百骑兵。只要他们能拖延一阵,就可以等到邕州城下的援军。可偏偏一点时间也拖延不下来。刚刚抵达三千援军,匆匆忙忙的列阵之后,当头就被一棒打昏,甚至连反击都组织不起来。

一直守在一旁的四百骑兵终于出动了,他们再不动作,宋军就能把他们的步兵如同兔子一样赶得满地乱窜。

但交趾人根本就不会使用骑兵。别说跟天下闻名的契丹铁骑相比,就是党项或是吐蕃,都有天壤之别。乱哄哄的冲到结阵前行的宋军面前,想凭着勇力冲杀——这是他们过去面对南方的占城军时,经常使用的战术——可宋军仅仅是凛然不动的用神臂弓一次齐射,就让他们人仰马翻。紧接着,李信亲领的选锋出阵,一声大喝,将掷矛纷纷投向混乱中的骑兵。

交趾军的骑兵是军中至宝,都有着甲胄护体,神臂弓射出的弩箭好歹也被牛皮甲抵挡了一部分的杀伤,可在沉重数十倍的掷矛面前,一层皮甲甚至连纸都不如。

从天而落的长枪粗暴的破开衣甲,连人带马一起扎成了肉串。战马的惨嘶响彻战场,而马背上的骑手早就被重矛夺取了性命。尚未接战就损失了四分之一,冲向敌阵的道路还被堵上前面落马的自己人所阻挡。而令人畏惧的掷矛还在眼前等候,无奈之下他们也只能调转马身,向来处退下去。

“真是胡来!”韩冈摇着头,骑兵岂是这般用的。要是他手上有四百骑兵,早就让他们冲到李常杰大营边去耀武扬威一番,放几把火,让十万大军夜不能寐。

可有了骑兵的缓冲,被压得节节倒退的交趾步兵,终于在单薄的用栅栏围起来的驿站外,重新稳下了阵脚。只是人人脸青唇白,惊魂难定,用着恐惧的目光看着对面的宋军。一次接战,就将宋军的锋锐表现得淋漓尽致。

“看来还要再来一次!”李信对韩冈说着。

韩冈眯着眼晴看着对面,“再来一次,就差不多了。不过他们应该不敢过来了,只能我们攻过去。”

“末将遵命!”李信手一招,掌旗官将他的将旗拔起,向着前方大步迈去。

鼓号响出一个变调,脚步声响成一片,军阵向前徐徐移动。

“如何?”韩冈驱动坐骑,随军前行,一边还问着另一侧的同伴。

“天军神威,我等蛮夷远远不及。”黄金满就在马上弯腰致礼,宋军表现出来的战斗力,也让他看得心旌动摇,难怪刘永甫一接战就被杀的全军覆没,也难怪眼前的这位韩运使敢于领着他们杀向拥有百倍大军的邕州。

韩冈看得出来这位广源蛮帅已经心服口服,不无自豪的笑了一笑。

他把黄金满留在身边,也防着一些意外发生。八百士兵从昆仑关杀出来,没有后援,只有两千名刚刚来投的广源蛮军,韩冈总要留心一下,表示大方也得选对时候。

视线重新回到对面。官军再一次用神臂弓奏响杀戮的乐章,在密集的箭雨下,交趾人的战线依然勉力维持着,但处处透着虚怯。看到交趾人的样子,韩冈知道,归仁铺的这一战,算是赢了。

论起两边的实际战力,其实不会有这么大。可交趾一方气虚胆弱,而他这边则士气正盛,两边的差距一下就拉大了。冷兵器的战争中,士气的因素很重,韩冈早已明白这一点,才敢于领军出昆仑关。

在他的判断中,李常杰很可能派来一万上下的援军来镇守归仁铺。所以他一路急进,要趁交趾援军立足未稳,打他们一个措手不及。一千官军为骨干,加上两千士气正盛的蛮军,也足以压倒匆忙而来军心浮动的敌军

他哪里会想到李常杰竟然只派来了三千多援军。这一战若是不能轻松胜出,那是真实枉费了自己的一番心机了。

交趾军的阵列在宋军的咄咄攻势下,已经难以保持战线,现在只是靠着更多的人数,勉强压住阵脚。

“对面已经坚持不住了。”韩冈点了黄金满的将,“黄洞主,让你的兵杀过去吧。”

黄金满就等着这句话,抱拳应声:“小人遵命!”

一声声有别于宋军中的号角长音,在拖后的两翼压阵的黄金满率领的两千蛮军,立刻闻声出动。广源蛮军没有宋军的阵型整齐,但呀呀怪叫的冲杀上前,就是给了正在坚持中的交趾军最后一击。

旗帜散了满地,甲胄兵器也都一起抛弃,交趾兵溃散而逃,在归仁铺外的原野上,如同一群被猎手追逐的野兔,迈开双腿奋力逃窜。

让黄金满分出一部去追击,剩下的人则打扫着战场。

虽然远涉千里,八百荆南将士一个个都很累了,但因为宾州城下的大捷,以及黄金满轻取长山驿的缘故,让他们对韩冈的信心十足。打扫着战场时,他们也都是喜笑颜开。

他们只付出了微薄的代价,就在宾州城外解决了同样人数的广源蛮军。而投降官军的广源蛮又同样轻取长山驿的交趾兵,这么一算,他们虽然只有八百兵,就足以对抗数万交趾兵。而且李常杰背后还有为数更多的广源蛮军,就算上阵他都要提心吊胆。到时候,说不定都不用自己动手,贼人就会败了。

一战大捷,苏子元心中的沉重也没消解多少,他远眺着邕州城的方向。到了白天,见不到城中的火焰,但更加醒目的浓烟,深深的烙在相隔三十里的苏子元眼中,清晰可辨。

“要不要继续追击?”李信看了苏子元一眼后,转身问着韩冈。

“让黄金满追出一里就够了。现在是要尽快让刘永他们还有所有交趾兵得知官军到了,只要能做到这一点,就不需要太过于急进。”

韩冈没有冲昏头脑,要去直奔交趾本阵。他不是急进,而是想要表现得急进。也是展示出自己的咄咄逼人,就越能让李常杰感到危险。只要虚张声势的表现出大军已至,一并将交趾和广源两军都震慑住,就能逼得他们从邕州城下撤退。

“今天就在归仁铺休息。”韩冈下令道。

“交趾人不会夜袭?”

“我们越是张扬,交趾人就越是没有这个胆子。不过的确要防着就是了。让骑兵再辛苦点,巡视周围。看看邕州到归仁铺的这一片坦途,李常杰怎么夜袭我军?”

