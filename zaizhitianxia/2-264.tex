\section{第38章 一夜惊涛撼孤城(下)}

王舜臣说不清围着香子城的究竟有多少敌军。但他知道,他手上这区区七百人,应该不及对方的一个零头。如果再加上可能存在的伏击援军的贼人,说不定有五六千之多。

无数火炬环绕小城,浩如星海。号角声,呼号声,一声一声的在城外的猝然响起。小小的城堡就像浪涛中的一叶扁舟。随着风势在海上飘摇,随时都有可能被越来越汹涌的海浪给吞没。

“都巡,怎么办?”一名小校颤声问着,周围的一群军校,望着城外的敌军军势,也都苍白起了一张脸。

“怕个鸟!”王舜臣骂着两股战战的部将们,“这是虚张声势懂不懂!要外面的贼人真有本事,直接就攻上来了,点什么火把,嫌着家里油多吗?”

“就是要把你们这群废物吓得发慌,他们才敢攻城!”

“打起精神来,别像娘们一样,听到鬼叫就脚软。守过这一夜,援军就来了!”

“韩机宜为什么要让本将在香子城住上几日?还不是早算到了贼人会来偷袭!一切早就有所准备,就是为了要杀这群蕃狗个片甲不留!”

王舜臣一阵大吼,声传内外,暂时安抚了军心。

他怎么也算得上是熙河军中声威赫赫的名将。虽然改动后的年纪依然不免让人议论。但王舜臣军中独步的箭术,这两年没少在众军面前表演过。在蕃部中都出了名的神射,让他的名望早就安扎在下面的士兵心中,一番严词训斥,也都能听进心去!

王舜臣的话也算有道理,听说了经略司的韩机宜早有准备,也都暂且放下心来。

军心安稳,王舜臣本人也暗地里松下了一口气,当即下令,“把所有的神臂弓都搬出来,还有弩箭。快!上城的每个人都分给我发上!”

听到命令,立刻就有人出去执行他的命令。

香子城别的不多,就是兵械多。单是神臂弓有五千多张,都是今天送到香子城,准备拿到前线大营去替换的。硬弩的寿命有限,往往发射几十次就坏了。通常一场拖延的稍长的大战下来,要更换的弩弓就是成千上万。

这五千张神臂弓就是王舜臣守住城池的本钱。

‘只要挡住贼军的第一波……’王舜臣这样想着的时候,城外的敌军已经有了动作,数以千计的士兵如洪水一般涌来。

不愧是香子城的旧主,对于城防上的缺点了如指掌。最多的兵力集中在城墙有着破损的一面,如果在白天,就能看到这面城墙上,曾经坍塌过又经过修补后的痕迹。

城头上,站上了六百人,只留了一百在城中,随时支援出现危机的地方。

只有三百余步的小城,如果按照宋军的标准,只能称为堡。六百人并肩站着,已经将一周城池全部站满。每一个人的脚边的都叠放着一排上好了弦的弩弓,火光和月光的照耀下,可以看到弩弓前端的圆形脚蹬。

敌军越来越近,王舜臣传达他的第一道命令:“将火炬熄掉!”

命令在城头上传递了出去,而城墙上的火光也由近及远,一个接一个都熄灭了。瞬息间,陷入黑暗中的香子城城墙,城上城下都是一阵不适应。

吐蕃人无法再看清城头上守军的位置,但守城的宋军们很快就发现,他们现在可以更为清晰的看着来袭敌军的身影。

王舜臣继续下令,“听本将的号令,再行射击。不待号令而先行射敌者,皆斩!”

严厉的命令让守军忍住去射击先头部队已经出现在城下的敌军,躲避时不时飞上城头的利箭,眼睁睁看着他们举起了攻城用的长梯——不过一丈多高的城墙,能架上城头的梯子,根本不需要费多少手工。

敌军在城下越聚越多,一架架长梯搭上城墙。

近在眼前的蕃人嚎叫声传入耳中,王舜臣看着出现在身前雉堞上的长梯上缘,不时的颤动代表正有人利用其来攀上城墙。

“击鼓!”王舜臣终于下令,“把所有的箭矢都给我射下去!”

战鼓声将王舜臣的号令传递到城中的每个角落。

而与此同时,一面丑恶的怪脸出现在城头上,王舜臣两名亲兵,立刻张开弓,将两支长箭怒射进他的眼眶。一声惨叫过后,他们拿起脚边张开的神臂弓,冲到墙边,从箭孔中狠狠地扣下牙发,弓弦颤动中,顺利的收获到两声嘶嚎。丢下射空的硬弩,他们又一同换上了另一支张好的神臂弓。

同样的情景,发生在每一寸城墙上。五千张神臂弓分发下去,一人就有七八张弩,射空了弩箭,又换了弓来。数千箭矢密如急雨,集中在极短的时间中一下迸发出来。这是韩冈当年在军器库中杀人起家时的手段,借用在战场上时,效果却更是绝佳。犹如冰雹雷霆洗过城下,转瞬间便是死伤一片。

没有任何阻碍的冲到城边,没有哪个吐蕃士兵会想到面对的将是如此猛烈的攻击。正幻想着破城后的大肆劫掠,死亡的箭雨就降落到他们的头上。

城下拥挤的人群一下静了下来,死寂中只有一声声哀嚎传出。吐蕃军的攻势猛然间的一顿,连冲锋时的呼喝声都低落了不少。冲在最前面的都是军中的精锐,一下伤亡大半,全军的士气几乎都被一下打光。

一击破敌,转眼之间就将形势逆转,城头上顿时陷入一片狂喜的境地。

欢呼声中,王舜臣依然保持着冷静,未来名将的素质展现在战斗之中,“继续!快把神臂弓都张开!……还有,快把搭城头的梯子都收上来!”

看着士兵们一个个坐下来给神臂弓重新上弦,年轻的都巡检的脸上终于有了一点喜色。望着城外犹然灿烂的星火,冷笑着:“看你们还能玩上几次!?”

……………………

半轮下弦月已经向西低垂,再过不久就该天亮了。

韩冈派出的骚扰队伍,在山中取得了不小的成功。尽管蕃人的将领看起来是个沉稳的性子,没有被出城时的动作所扰乱。但随即射出去的一支支火箭,虽无法在潮湿的林间燃起山火,却成功的惹起了战马的惊慌。

潜出城去查探敌情的斥候,都说城外的吐蕃人被惊扰得坐立不安。也不必他们说,只看满山亮起的火把,韩冈就能知道他派去山中的那两百多人,给了吐蕃军多大的刺激。

一夜未眠,加上此前的长途行军,到了天亮之后,这群吐蕃骑兵还能拥有多少战力?

韩冈有些得意的轻笑了两声,一群没有精神的骑兵,就算是普通步兵也能轻而易举的将之解决。再说战马如果不能养足精神,出问题的可能性会更大。牲畜怎么也做不到像人一样,有着凭借毅力克服困难的能力。

刘源在韩冈身后笑问着:“这就是节目?”

“如何?”韩冈反问着。

“明早当能必胜!”刘源沉声说着,若是连千人左右的疲军都赢不了,珂诺堡内的数倍守军都可以去自尽了。

他的声音顿了顿,神色凝重的补充道:“眼下就得看香子城了!”

“嗯!”韩冈的声音也有些沉重,可随即便展颜一笑,“应当无事。有王兄弟在,他应当能守得住!”

只是这时候,突然听到了一片欢呼之声,是随着风,从吐蕃人立足的那一片火光中传递而来。

韩冈疑惑的回头与刘源对视一眼,这声欢呼为他们两人的好心情抹上了一层阴影。

究竟是出了何事?

韩冈有些坐立不安的等待着答案的揭晓。

天渐渐的亮了,珂诺堡中的守军在韩冈的命令下,开始整队集合,准备出战。

而就在这时候,一支吐蕃骑兵穿过了淡淡的晨雾,来到了珂诺堡城下。领头的骑兵手上高高挑着一面旗帜。形制不是吐蕃人的惯用的式样,而是汉军的角旗。城头上的众人看得清清楚楚,那分明是田琼所率领的骑兵指挥的战旗!

且在战旗的顶端,挂着一枚首级,凝固起来的血水模糊了相貌,但依然扣在首级上的头盔,让城墙上的人们辨认出了首级所代表的身份。

“田琼!”

“是田指挥!”

“田琼死了?!”

城头上一片惊声,人人脸色发白。田琼昨日是奉命去救援香子城的,现在他都战死了,他麾下的四百骑兵必无幸理。没有援军支援,甚至是有可能亲眼看着援军覆没在城下,那香子城怎么样了?

不止一个人在问着:“难道香子城破了?!”

“田琼是香子城中的人吗?”韩冈厉声反问过去,凌厉的眼神堵住了每一个人的嘴。

“田琼败了,必然是败在蕃人的伏击中的。要是真的是香子城被攻破了,拿过来的就该是王舜臣的将旗!”

韩冈的话稍稍安定了军心,他回头喝令众军:“击鼓,出城迎战。杀光这群蕃骑,为田指挥报仇!”

一下声色俱厉,“把首级给我夺回来!”

