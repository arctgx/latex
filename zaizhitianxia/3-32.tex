\section{第12章 共道佳节早(六)}

跟在王雱兄弟之后,明显的是一个女子。

带了帷帽,垂下来的薄纱遮挡了面容。紧紧裹着半新不旧的狐皮斗篷,将身材遮掩的同时,却把窄窄的肩头勾勒出来。

韩冈这下算是给王家的两兄弟吓到了,就算他再没眼力,也能猜得出这一位跟着王雱兄弟的女子究竟是何方神圣。

所以他才在心中大骂着,这在开什么玩笑!

韩冈倒不是为自己担心,而是未出阁的闺秀与非是亲眷的男子私会,这件事一旦传扬出去,宰相家女儿的名声就完蛋了!

所谓‘内外各处,男女异群;莫窥外壁,莫出外庭。出必掩面,窥必藏形。’

通传世间的礼法,良家女子不能随便见与男子相见。以司马光的说法,女子到了订婚之后,父亲就不能进入她的房间,姐妹出嫁后回娘家时,弟兄不能坐在她旁边。

就算是开放的唐朝,李林甫让女儿自己挑女婿,也是把候选人招到家中去,让女儿在屏风后面挑。没有说让女儿到家门外,当面相夫婿的。就这样李林甫还被人嘲笑其是寒门素户,不知礼法。

而宋代,民风比唐朝更要保守十倍,对未婚少女的约束也更为森严。不比唐时,能穿着男装,带着个婢女就往外跑的。

也不是说宋代的女人就是大门不出二门不迈,该宽松的地方还是比较宽松的。东京城中,如大相国寺等让人烧香拜佛的寺院,在佛像之前叩拜行礼的多有女子。

曾布的夫人魏氏,也常常抛头露面的参加甚至组织诗社诗会,不是全女性的诗会,而是皆为官员和士子列席的高水准的诗社。魏夫人甚至连闺名小字都跟着诗词传到外面来了,但世间的评价还是很高。

但世间对未嫁女和已嫁女的道德要求却是完全不同的。小家碧玉倒也罢了,都要帮衬家中做事,只要不是跟着外面的男子打情骂俏,抛头露面一点不会影响她们的名声。

可名门闺秀就不一样了。平常外出,都是坐着马车,有家中仆婢在外护持。春来去城外踏青,还要用帐幕给围起来。若是能大着胆子来私会尚未定亲的男性,这种离经叛道的行事,必然要受到世人的指指点点。

幸好王家两个儿子不算糊涂,一起跟着上来,当然算不上是私会,但传扬出去,也不是好事。

韩冈一怔之后,便疾退了两步,将王家的三名子女让进屋来。让王家的女儿在门外待得越久,暴露的可能性就越大。

王雱进了门后,便笑呵呵的对韩冈拱手道:“玉昆大名闻之久矣,却始终缘吝一面。咸阳平叛、河湟开边,玉昆种种行事,让王雱渴慕已甚。日日思君不得,岂料今日终得相见。”

“不敢,元泽兄的大名才是让韩冈如雷贯耳。朝廷支持河湟开边,也有元泽兄的先见之明。”

韩冈先向着王雱回礼,自谦了几句。然后又亲近的跟王旁说了些别来无恙的闲话。

两个兄弟各自跟韩冈见过礼,那名女子便来到了韩冈的面前。

“此是舍妹。”王雱只用了简短的四个字,向韩冈引荐他们身后之女,不知算不算是某种程度上的掩耳盗铃。

王安石的二女儿向着韩冈福了一福,道了声:“公子万福。”声音清雅悦耳。

韩冈则回了一揖,并不多问。

既然王家兄弟都不想多说什么,韩冈不会跟他们过不去。跟着一起装装样子也无伤大雅,且更能让他们安心。就将三人请了坐下。

他邀人的动作自然无比,王雱让人来定的位置,现在倒成了韩冈作为主人来宴客。

韩冈、王雱和王旁三人围桌而坐,而王家女儿则是坐到了略远一点的墙壁下的座位上。

王家兄弟对此视而不见,而韩冈更是识趣的保持沉默——终有图穷匕见的时候,他倒要看看究竟要玩什么花样。

厢房中一时静了下来,隔壁的喧哗声便重又吵闹了起来,对朝堂大事指指点点,朝中最近的人事变动,还有新法的推行情况。言辞之中,也少不了士子们特有的狂妄,指点江山起来,比起宰相的气魄还要大上三分,只有那名余中,还算是稳重。

“这群狂生……”王雱摇了摇头,看他的样子,很是受不了隔壁士子们的胡言乱语,“方才的一番毁谤之词,玉昆可都听到了?”

韩冈心平气和的为之一笑:“新进遽得高位,哪有不受嫉恨的?此是寻常之事,韩冈早就学会对此不放在心上。左右也只是图个嘴上痛快而已,让他们说说也无妨。”

“玉昆你太放纵他们了。”王旁不知生者哪门子的气:“这等爱嚼舌根的小人,就该从重处置。圣人诛少正卯,可没有说放上一放的。”

说起少正卯,韩冈倒是要为孔子辩护一番:“先圣诛少正卯,不见于经典之中,乃是荀卿臆造之语,污谤圣人千年。岂可信以为真?”

关于孔子有没有以五恶之罪诛杀少正卯,世说纷纭。比较早的《左传》、《国语》中都没有记载,最早出现的时候,是荀子说出来的,之后便流传开来。连《史记孔子世家》中都录入了进去,一直被人信之不疑。

只是如今宋人疑古,对此事便多有评述和翻案,韩冈此言算不上特别。但王雱听了摇头:“不论是否确有其事,征诛的手段,该用的时候还是该用的。”

王安石的学术推崇孟子,并不赞同荀卿的观点。不过王安石当年的上神宗皇帝万言书,却有这么一段‘故古之人欲有所为,未尝不先之以征诛而后得其意’。

以王安石早年的这一份奏疏中所表明的他的观点,凡事要想成功,就必须先将反对者给清除。这一个观点,却是从荀子之学而来。

王安石世间流传的著作,韩冈基本上都看过。这一份著名于世的万言书,韩冈当然不会没有熟读。

“只是政事归政事,寻常聊天都要管着,日后可就是道路以目的结局了。”韩冈放得开,那群贡生骂到自己头上,最有力的回击就是考上进士,日后晋升宰执,压在他们的头顶上。

……………………

如同隐形人一般,坐在一边,王旖静静的听着两位兄长和韩冈侃侃而谈。

王旖已经到了十九岁,这个年纪也不出嫁,外面的风言风语也便多了起来。渐渐的,原本活泼的少女也变得愈渐沉默寡言,每日里除了读书习字,就是陪着母亲做做女红,说些闲话。

对于韩冈,王旖不能说不好奇,当年还想着见见能一怒拔剑的侠客究竟是什么模样。只是如今的好奇心已经渐渐淡了,加上前日的父亲托人向韩冈提亲,却被对方找了个借口敷衍了过去。

尽管父母对这门亲事还抱着希望,但王旖却能从韩冈的拖延中,看到对方的不情愿,以及隐藏于心的抗拒。这是与生而来的直觉,与眼光无关。

前日王雱来问她对婚事的意见,王旖没有说别的,只是说想当面见韩冈一次。虽然当时被一口拒绝,但做大哥的,终究还是拗不过妹妹,不得不点头答应了下来。

也是他们不觉得韩冈会拒绝了这份亲事才会答应,否则怎么也不可能点头。

但王旖有些话想说。

因为妹妹在此,王雱两兄弟不便于此久留,又聊了一阵,便拱手相辞。王旖站了起来,却没有跟着往门外走。“小妹有话要对韩公子说,还请大哥、二哥在外稍留一步。”

非礼勿言,单独与男子交谈,更不合礼法。王旖的这句话,王雱两兄弟事前都没有从妹妹的嘴里听到过。一听之下,皆是脸色一变。

王旁连忙要阻止,但王雱想了一想之后,却点了点头,“那愚兄就在外面少待。”拉着王旁出去了。

等着王家兄弟出门,韩冈便转去对王旖道:“小娘子若有话,请直说。”

王旖走到韩冈面前:“小女子年近双十。年岁既长,又是蒲柳之姿,不堪为君奉箕帚。此生惟愿侍奉父母,别无他求。”

韩冈却没有生气,她还没有把话说完,理由也没问,要发火也使得先听完再发火也不迟。

“可是韩冈有甚鄙薄之处,不堪小娘子青眼?”

“韩公子之才世人皆知,小女子岂有不满之处?但大姐误嫁吴家,让父母日夜烦忧。但公子处事,难与家君一同。若日后政见不合,不能父母安心,便是小女子的不孝了。”

王家二女儿一番话可谓是坦率了,让韩冈为之惊讶。心中顾虑着父母,也算是有孝心。但她是怎么知道自己会与王安石不是一条路上的人?

心中有些怀疑,但韩冈无意多问,更无意自辩。政治上很少有人能始终如一,分道扬镳倒是常见。

“此事却小娘子错了。”韩冈正正经经的与王旖说道,“内外有别,政事岂预家事?吴枢密家内外不分,那是他们的错,若是以为韩冈也是如此,可就不对了。”

