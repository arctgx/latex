\section{第37章 长安道左逢奇士(中)}

【第二更,求红票,收藏】

见韩冈肯开金口,税吏们知道事情终于过去,齐齐松下一口气来。

“还不是绥德城闹得。”山羊胡子跳将起来,牵着韩冈的马缰向前走,一边指使手下将那个胖子蜀商放掉,一边仰着头小心回话,“一年上百万的钱粮砸下去,也听不到个响。京城那边说要给钱给粮,却都是打着折扣,还要我们关中填亏空。偏偏陕西钱粮不足,转运相公没办法,只有多多收取商税了。今天是京兆府,过几天陕西路都要查得严了。转运相公明明白白说的,无论哪路神仙,不把税钱缴足,都不得放过去。天可怜见,俺们这些抽税的平常也没个好处,上缴的税钱短了少了还要挨板子,现在大过年的又被派出来吃风,家里的浑家小子都在等着回去过上元节。可有什么办法?转运相公说话,谁敢不听?小人也是没辙啊!在风地里受足了冻,看着满天满地都是白的,脑袋僵了,眼睛也昏了,不意得罪了官人。幸好官人宰相度量,不与小人计较……”

山羊胡子倒是会说话,一句句的连珠炮比王舜臣的箭飞得还密,他这一大通抱怨,倒是翻来覆去的把苦水都倒尽了,就算韩冈心中还有怨气,也不好向他身上撒。不过韩冈也知道,这是山羊胡子欺他年轻,不知做税吏的油水何在。要是税吏真的这么苦,何不回乡种田?

韩冈也不戳穿他,却想着陕西转运司下的这个命令。如今陕西转运副使陈绎,听说他精通刑名之术,曾平反了不少冤狱,除此之外,韩冈便对他一无所知。但既然精通刑名,理所当然的便是了通世情,直透人心。如果这样的人出手,后面自然暗藏深意。

陈绎把抽税声势闹得这么大,但在大过年的时候,又能抽到多少商税?而且怕是没几天一片怨声会传到京城里去。这是叫穷啊!韩冈心道,陈绎这么做,很有可能是在逼着朝廷快点拨钱下来。只是他再往深里一层去想,更有可能是在借力打力,利用关中的民情舆论,去阻挠横山战略的实行。

而区区的绥德城那一块,砸进去的钱粮竟然有百万之多,也让韩冈吃惊。看起来种谔在那里的动静并不小。也难怪李师中能气定神闲地拒绝王韶在渭源筑城的提议。陕西的预算有限,转运司不会另外支钱。王韶再有本事,也难在陕西转运司的库房里把筑城的钱粮给挖出来。

韩冈皱了下眉,看起来自己到京城去,又多了个任务。

当然!韩冈低头看了看在他马前殷勤的牵着缰绳的山羊胡子。陕西转运司会把手伸到过往的官员身上,理由应该不仅仅是为了叫穷、生事,阻挠开拓横山。另一方面,如今的文武官员也的的确确的都钻到了钱眼里去了。

韩冈都听说过有些官员会在上京时夹带着土产商货,以求贩运之利。而在他上京前,也的确有几家商行想请他一起出发。因为王厚貌似无意的提点了一句,让韩冈对此心中警觉,拒绝了那几家商行的无事殷勤。

东京是为国都,有百万人口,上万官僚。人多了,钱也多了,商业随之繁盛,四方财货无不汇聚至京城。将各地土产转运至京城贩卖,是一桩包赚不亏的买卖。而笑贫不笑娼的世风,使得官员也不以经商为耻。往往都分派家人、亲族去经营商事,并利用自己的官身,来躲避各州税卡。

按照朝廷颁布的律条,地方上的商税分为驻税和过税两种。顾名思义,驻税就是商品在本地销售缴纳的税金,即是营业税,而过税经过税卡时缴纳的税金,即是关税。驻税为三厘,即百分之三,而过税则是二厘。

这个税收额度看似很轻,但过税不是交过一次便高枕无忧,而是经过一个军州,便要交上一次——这是一般情况——有的军州,往往会多加税卡。一般来说,运程超过千里,计入税金,再把运费加上,运输成本就要超过货物原价——这还是指得是水路。陆路走上三四百里,售价就要翻倍才不会亏本。

所有世间有种说法,叫做百里不贩樵,千里不贩籴——超过百里,卖柴禾便赚不到钱,超过千里,卖米也就赚不到钱。运费和税金,是遏制商业发展的最大的主因。

为了规避这两项开支,最简单的就是利用官府的运输渠道。许多官员进京时会带上地方土产,而且还借用官船来运货,便是为了把运费和税金全都省掉。

韩冈甚为鄙视那等庸官,自家赤膊上阵,只会弄坏自己的名声。要赚钱,手段多的是啊。只要有可信的人手,一年几千贯根本不成问题。

山羊胡子帮着韩冈牵了一段马,税卡也过去了,孝心也表现过了。韩冈不为已甚,正打算示意山羊胡子回去了事,自己和刘仲武一起继续上路。但刚刚离开的税卡处,突然又传来一阵喧闹声。一个有些尖锐的声音大叫着:“吾乃邠州贡生,尔等拦住去路,是欲何为?!”

一口儒生的酸话让韩冈好奇的回头,只见天边飞来一座小山,正正压在税卡之前,却是方才看到的那头可怜的骡子到了。

山羊胡子看着韩冈回头,以为他想帮着那位邠州贡生。也难怪他会这么想,自古文人相轻,但读书人却总是见不得同样的读书人受到小人欺辱。“官人,小人就去把他放过来。”

“不搜检了?”韩冈并不知他方才回头一眼,让山羊胡子以为他想帮着邠州贡生一把,有些惊讶税吏们怎么好说话起来。

山羊胡子以为韩冈在说反话,忙陪笑着:“官人既然要帮着邠州来的秀才,小人哪敢再搜检?”

我什么时候说过要帮他的?

山羊胡子又看了看税卡那里,回过头,苦恼的跟韩冈叹起气来:“官人,这事有些难办呐。若是平常,俺们倒也睁一只眼闭一只眼的放过去了。好歹是个贡生,说不定今次就考个进士出来。但眼下不行啊,转运相公都发了狠,他这么一座山也似的包裹,能过了一关、二关,过不了三关、四关。出不了百里,铁定的会被拦下来……”突然,他话声一顿,像是灵光一闪,“有了!官人请等等。”

丢下一句话,蹬蹬蹬的跑了回去。山羊胡子自说自话,让韩冈有些郁闷。他不说话,只看那山羊胡子怎么做。可结果,让韩冈吃了一惊。

山羊胡子真的会做人,他把邠州贡生拉到一边说了两句,不知说了什么,贡生顿时就不闹腾了。很快两人便一起向韩冈这边走来。而贡生的骡子,是连着包裹都被留下,可原本属于胖子蜀商的三头骡子中的一头,却改被贡生拉在手里。

这是三一均摊啊!韩冈摇头笑叹着,三头骡子,还了胖蜀商一头,税吏们笑纳一头,贡生则换了一头。行了,除了蜀商吃亏以外,所有人都满意了!而胖子蜀商险死还生,也不敢有所怨言。

能吏啊!当真是能吏!

贡生随着山羊胡子走了过来,韩冈依礼下马相迎。

那贡生差不多有四五十岁的样子,长得有些干瘦,胡子不知是根本没长,还是为了装年轻而刮了去,脸上干干净净,可这样一来,千丘万壑般的皱纹却也暴露了出来。看上去,比刘希奭还像个阉人。

他身上套了件罩风的袍子,不知多长时间没有清洗,黑得发亮,已经看不出原来的颜色。他在韩冈身前躬身行礼,谦卑的说着:“后学晚生路明,草字明德,邠州人氏,见过官人。”

看着比自己年长至少一倍的中年,在自己面前自称后学晚生,虽然是世间的惯例,韩冈的心理还是觉得有些别扭。

韩冈心中有些奇怪,“省试是在二月中,如今正月已经过去了一半。路兄现在才入京,不怕赶不上举试?”

地方上的解试在去年八月就结束了,一般的情况下,得中贡生的士子都会选择在九月、十月的时候入京赶考。他们都要在东京住上三四个月,直到次年二月中的礼部试和三月初的殿试为止。这一方面是要习惯京城的水土,省得在考试时弄坏身子,另一方面也可以结交四方士子,增广见闻,并切磋学问。

而路明直到现在才入京,将考试时间卡得将将好,若不是看到他举止透着酸气,韩冈定会将路明视为伪造证据的骗子。

路明扬起脖子,自傲的说着:“晚生腹中才学尽有,今次入京就是要做进士的。岂会如那般庸人,进个京城便心惊胆战?”

这货还真是敢说,真有才学也不至于蹉跎到四五十岁。韩冈有心想探探他的底,便问道:“以路兄才学,邠州的解试当是轻而易举。”

路明哈哈笑道,“晚生去考,岂有不过的道理,过往哪次不是易如反掌?”

路明如此一答,韩冈心中就有数了。为了确认,他又试探的问了一句:“京中风土异于秦川,若是抵京后不休养一阵,怕是会水土不服。路兄就不担心有何意外?”

“晚生京城去得多了,岂会水土不服!?”

路明这两句话终于透了底,‘原来是个免解贡生。’

所谓免解贡生,是指经过了多次解试合格,进京后却屡考不中的士子,让他们可以不必再参加地方上的解试,而直接进京参加科举。其实这与特奏名进士是一个条件,不过是为了安抚那些不肯放弃考取正牌进士的士子,省得他们一怒投往敌国——主要还是西夏。

因为陕西各州的解试远远比东南各路要容易许多,连续考中的贡生多不胜数,特奏名也好,免解贡生也好,主要都是陕西人。这两样制度本也是朝廷拿出块骨头来安抚陕西士子人心的。

