\section{第38章 岂与群蚁争毫芒(五)}

自汝州南下,穿越方城垭口,直抵荆襄。虽然襄汉漕渠没有打通,但千百年来,这条路都是沟通南北的一条极为重要的通道。从中原至荆湖,都得走这条路,无论春夏秋冬,路上的行旅永远不见少。

不过如今正值炎夏,为了避开太阳升起后的暑热,道上的车马旅人都会选在大清早动身。

启程时,天还是黑的。先披星戴月一个时辰,再顶着晨光一个时辰,地面便会烫得马蹄都不敢停步,只能歇到路边的避阳处,一直得歇到傍晚才能再次起身。

而这也给了路边茶棚、酒店带了来让人欣喜的收益。能出外远行,无论是为了何事,都少有人会穷到坐在树荫下拿着草帽扇凉,而舍不得掏出几个铜钱,买上一盅凉茶、一碗淡酒。

开在方城垭口南端的一间脚店,即卖茶又卖酒,不过是间草屋,门外还支了个棚子,里外七八张桌。但自从襄汉漕运的工程开工之后,生意好得让店主做梦都在笑,只恨不得一年有四个夏天,十二个六月或是七月。

晚上有下了工的厢兵和工匠来买酒,白天门前则停满了商旅的车马。装钱的木盒子一天就能装满,叮叮当当的脆响总是不停地响起,店主时不时的就掐上自己一下,确认自己是不是在做梦。

不仅仅是店主如坠梦境,就是老走这条路的商人也对草棚中的客人人数感到惊讶。一个坐在墙角的老家伙,就在跟着他的晚辈在感叹:“换作是庆历年间,这个时候店里可不会坐上这么多人。谁人敢走夜路啊?被劫的商客,一个月好歹也有十来回,有的连脑袋一起被劫了。多少人宁可被晒得中暑,也不敢拿性命去贪些早晚的凉风。”

老头子可能是耳背,说话的声音很大,不仅是他的晚辈,店里面的人可都听到了。店家连连点头称是,他还认得这位走了三十多年方城道的熟客。

“老丈说得正是。也是如今太平盛世,道上无贼,换作是十几二十年前,不结成大队,谁敢在夜里单身行路?”一名长得干瘪的商人接着口,洛阳雅音标准得很,但尖尖的胡子,削瘦的双颊,让他看着活脱脱一只山羊。

太平盛世?有些人嘴角就翘了起来,但没人会在这个场合将自己心里话给说出来,闷头喝茶喝酒。

“还是保甲法的功劳。”与前一名像是山羊投胎的瘦商人有着明显的对比,一个挺起的肚子让他身上的衣服比常人要多耗上三尺布的胖商人,则赞赏新法中的一条,“之前没有保甲,捕盗得靠县里的弓手,想想他们有几个会与贼人拼命?也就是有了保甲之后,就算来了一伙盗匪,在乡里面就给射死了,拿了去县里州里请赏。淮左郭七都听说过吧?熙宁八年在淮南的时候,俺可是亲眼看见一个庄子的保丁把他活捉了送到县里去。他领着二十几个马贼横行淮泗十来年,就在小村子里翻了船。手下给杀了精光,自个儿没几天就给处了磔刑,四分五裂的吊在泗州的城门口。”

“保甲法为什么能捉贼?就是把人当贼防着!”有一个中年人明显是喝多了,红着脸大声道:“俺去年回乡里走亲戚,坐下来还没来得及上茶呢,保正就溜过来问了,上查三代,下查子孙,就差问生辰八字了。问得那么细,俺还以为他家里有要嫁人的女儿想便宜俺。”

他的话说得有几分刻薄,倒引得店中一阵呵呵轻笑。

“有犯知而不告者,依连坐法处罚;强盗在保居留三日者,邻居不知情亦科罚。凡有行止不明之人,本保亦须觉察收捕送官。保正也要为自家着想。”坐在另一桌的一名书生冷笑着说道。

这名书生不过二十多岁,但他并不是单独出行,而是一大家子三四十口。仆人在外面看着车子,女眷也留在树荫下的车上,而在店里休息的七八人,全都是读书人打扮。领头的老者五十多岁的样子,而这名书生,看年纪相貌应该是老者的子侄辈。

书生看模样就是读书人,一大家子的气质都是如此,应该是书香门第,但他们穿着上却普通得很,几乎都是布衣,就连看起来辈份最尊的老头子,也是一身式样朴素的靛蓝色细麻布裁制的衣袍,脚下也不是官靴,而是鞋子。但偏偏外面停着的两辆车马,都有着唐州衙门的印记,应该是在前面的驿站刚刚换过。

除了这一家子之外,店里的全都是走南闯北的商人,或许其中有几位识不得几个大字,但其中的每一个,都有着一双靠着走走南闯北的经验而磨练出来的敏锐眼力,该看的都看到了。

胖商人的声音变得恭谨起来,“衙内果然好见识,小人等可想不到那么多。”

“衙内可当不起,叫声秀才也就行了。”书生看看另一桌的老者,笑道:“家严也不过有个教化的差事而已。”

“教书先生?看着不像啊……”胖商人纳闷了一下,随即醒悟,“啊,俺知道了。莫不是县里、州里的教授吧?”

县学、州学里的教授、博士之类的学官不算正牌子的官员。尽管吃着朝廷的禄米,用着官府的车马,但这些职位都是安排给那些考不中进士的特奏名,没有品级,也就是不入流。张出招牌来,也没人会怕,几名商人也坐得安稳。

不过奉承话还要说:“官家降诏办学。如今县里办县学、州里办州学,学校起了不少,就是缺个能教书育人的先生。看令尊的模样,才学必然极好的,到了州中,少不得能教出几个进士出来。”

年轻的书生听了便是一笑,这么粗鄙的奉承他并没有放在心上。而那老者看见儿子隐了身份,与商人们聊着天,眉头就有些皱。他不喜欢说谎,但要他大张旗鼓的表明身份也不觉得有必要,干脆就坐着不说话,只喝茶,让晚辈去招呼。

老者其实也有些体会,新法虽然不合人意,但也不是全无用处。保甲法劳民伤财是一桩,坏了边州的乡兵之法也是一桩,但在平靖地方、编户齐民上,比过去要强了不少。

比起仁宗的后半段天下盗贼风起的惨状,如今道路上已经是安靖了许多。仁宗时的盗贼,许多都是百姓的身份,只是穿州过县做上一票,然后拿着赃物回家享受一阵,这样的贼人总是最难剿的。

而保甲法实行之后,天下各路的农民都要赶在冬天农闲时操演军事,一个百户人家的村庄,少说也有两百多保丁,有了保护自己的能力。且通过编订保甲,官府对乡村的控制力也上了一个台阶,忙时务农、闲时为盗的许多贼人,连逃都没逃掉。

一辆有轨马车沿着轨道呼啸而来,距离草庐只有几十步。老者抬起头来,双眼紧紧追随着马车消失的地方。

另一边的胖商人也是伸着脖子直盯着满载着充作路基卵石的马车,方才他们已经经过了正在忙碌中的工地,两头并进的轨道,还差十里左右,就能汇合在一处。

“太平车能载五六千斤,却需马骡十数。这跑在轨道上的马车,前后四节,载货上万斤,就只需两匹驽马。”他回头看看自家的车马,长叹了一声,“省得太多了。”

老者身边的另一名读书人低声说道:“难怪韩冈敢接下襄汉漕渠的这个差事,只要有了轨道,直接就可以跳过方城垭口这一段难关。可笑天下的矿山、港口都已经修上了轨道,就没人想到用来修做官道,还得韩冈自己来说。若是有一人想到,韩冈也不能独占其功。”

“不知端叔如何看韩冈?”老者声音同样的低,但他们称呼当今京西都转运使时的口吻,其实已经暴露他的身份。

应该是以‘端叔’为表字的年轻人,不说韩冈的功劳,却道:“父母居于陇右,贼虏在侧。其为独子,却任官中原。他事不论,只孝道一事,便不可取。”

老者点点头,这话说的是不错的。

只听那端叔又低声道:“文正公为人至孝,韩冈单就此事上便去之甚远,他事更远有不及。”

自立国至今,能被称为文正的可就那么几个,眼下能与话配得上的,只有一个范仲淹。老者的身份自然呼之欲出——新任信阳军知军范纯仁。

范家以忠孝传家。范仲淹二岁而孤,其母改嫁后将其带到朱家,改名朱说。等到范仲淹成年考中进士后,又改了回去,而他之后又为其继父请求赠官。

到了范纯仁这一代,范家的几个儿子同样是孝顺。范纯佑、范纯仁等人,都是一直随侍在父母身侧,直到范仲淹去世后,范纯仁才出来做官。而且在做官的同时,范纯仁还在照顾着他的长兄范纯佑。范纯佑有心疾,疾作则数人不能治。范纯仁为了照顾他,推辞了好几次提拔。

端叔若是称赞自己,范纯仁不会乐受,但称赞范仲淹,范纯仁自然不会拒绝。

“不过韩冈乃是当世奇才,”在孝道上,范纯仁不值韩冈所为;但他对韩冈的能力则评价很高,“眼下的有轨马车便是一桩。在关西、在京城、在广西,军政二事都让人只能自叹不如。因为罗兀城之事,他在环庆军中,名声也是极高。端本你在鄜延,应该更清楚。”

端本,或者说范纯仁的弟子李之仪——他表字端本——在鄜延路任职多年,当然了解韩冈在鄜延军中的人望,同时也了解韩冈的人脉关系:“韩冈与种谊之子种建中份属同门,与种谔之子种朴同样交情深厚……”

范纯仁笑容有些发苦,而后就长叹了一口气。

他是反战的,所以跟鼓吹对西夏开战的种家翻了脸。自从横山一役后,西夏两年来都不敢再有任何动作。范纯仁只希望这样的太平日子能持续下去,就算持续不了,也不该由大宋这边主动打破,为三两人功名利禄之心,而遽兴兵事,对国家、对百姓绝非好事。

范纯仁反对开战,李之仪是他的弟子,便在鄜延路反对战事。现在范纯仁调到了信阳军,而李之仪更是被贬去了辰州,一同南下。

范纯仁歉然:“只为此事,倒是连累了端叔。”

李之仪洒然一笑:“只缘国事,何谈连累。”

又是一列有轨马车满载着筑路材料飞驰而过。范纯仁低头喝着乡里的粗茶,李之仪的洒脱让他很是欣赏,至于韩冈,范纯仁只想着与之会面时,该怎么劝说于他。

若能说服韩冈,阻止战事,当能多上一份助力。

