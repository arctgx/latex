\section{第28章 官近青云与天通(23)}

“不意今日又见王曾。”

走下台阶,章敦冷冷的说了一句。

在他身侧的韩冈则回道:“谁是丁谓?”

两人对视一眼,呵呵各自冷笑。

吕公著究竟是在想什么,在他跳出来之后,宰辅们哪有看不透的?

蔡确、韩缜沉着脸。章敦笑中则带着隐忧。只有薛向,如无事人一般——没有进士的身份,反而就不需要想得太多。

仁宗初年,宰相丁谓当权,与内侍雷允恭相为表里,把持国政。参政王曾为除丁谓,砌词留对,与章献太后密议,一举扳倒了这位权相。

自此之后,一旦有哪名重臣在拜见天子后主动请求留下来奏对,那么在世人眼中,他的意图只会是针对同列。从权谋上讲,也失去了动手的突然性,反而打草惊蛇。

故而便逐渐成了官场上的一项禁忌,基本上很少再出现这样的作法。

“如果只是针对小弟的话,那倒是没什么关系。”韩冈淡然说着。

章敦看着前路:“也只是对玉昆你而言。”

“的确如此。”韩冈仰头喟叹。章敦与自己走得实在太近了,不免会受到牵连。

韩冈回头看看夕阳下的福宁殿,吕公著到底会说什么,其实完全可以想象得到。

即便不是在殿中旁听,吕公著也不会有其他的说法。

……………………

当蔡确、韩缜等人全数离开,只留下吕公著一名执政的福宁殿,又陷入了沉寂之中。

赵顼躺着,向皇后坐着,而吕公著则稳稳地站着,赐坐也没有理会。

帮赵顼掖好了被角,趁势整理了心情,向皇后抬头看着吕公著,沉声问道:“不知枢密自请留对,究竟是为了何事?”

吕公著深深的一躬身:“为了皇宋基业。”

臣子们大言诳君的手段,向皇后经历得不多,但她对吕公著即有成见,听到这话时便自然而然的有了戒心,“枢密何出此言?!”

“臣观今日朝堂,已是隐忧潜伏。王安石有威望,门生子弟遍布朝堂;韩冈有重名,得人心,世人敬仰。如今翁婿二人同列朝堂,相互配合无间,长此以往,皇宋基业恐有不稳。”

带着沉沉杀机的话语出口,殿中更加静了三分。从西南方照过来的阳光映不进殿中,只能将南面的窗棱染上一层如血的红光。

“过去也不是没有过。”向皇后越看吕公著越不顺眼,立刻道,“吾虽是妇人,也知道晏相公和富相公翁婿二人曾同列一朝。”

“那是富弼曾说晏殊奸邪!”吕公著抬起眼,一对白眉下的双眼利如刀剑,“今日在殿上,司马光的确多有错处,但昨日,韩冈在席上端茶递酒,岂是重臣所为?!”

向皇后张口结舌,难道要说韩冈是王安石的女婿,谨守晚辈的本分,所以才会端茶递酒?!可这不正印证了吕公著的话?

“陛下。”吕公著语气沉沉,“臣非是论韩冈之品性。试玉要烧三日满,辨材须待七年期,现在是看不清的。”

向皇后一下气白了脸,白居易这首诗实在太有名了,指着吕公著的手都在颤:“周公恐惧流言日,王莽谦恭未篡时。这两句,枢密何不明说?!”

“臣只为皇宋基业,非是为一己之私攻劾王、韩翁婿。”

“好个非为一己之私,”向皇后气得笑了起来,“冬至夜吾母子性命几乎不保的时候,不知吕枢密在哪里?!”

“殿下看重韩冈,或有其因由。”吕公著毫不动摇,皇后的否定他不在意,关键还是在赵顼身上,皇后越是偏袒韩冈,皇帝就会越担心:“但韩冈未及而立便名声广布,世人视之若神。今日殿上论司马光有心疾,又有几人不信?殿下当也是信了吧?”

向皇后立刻道:“司马光强要杀王珪,岂非心疾?”

“那一众御史呢,他们不也一样要杀王珪?”吕公著反问。

“他们受了蛊惑而已。”

吕公著神色一肃:“受人蛊惑,已是罢官去职,那么蛊惑人心之辈,如何不论之于法?!”

向皇后的口才哪里能跟老辣圆熟的吕公著相提并论,登时就被堵住了。优待司马光的决定,还是刚刚在崇政殿上做出来的。

吕公著也不继续与向皇后辩驳,他看着沉静的躺着的赵顼,“韩冈名重当世,王安石威望尤髙。章敦蔡确为其爪牙,韩缜、薛向唯唯诺诺,若翁婿二人同在政府,日后谁人可制?”吕公著跪了下来,再拜叩首,“陛下,非臣疑韩冈和王安石。但两人身处嫌疑之地,只为两人着想,也得让他们避嫌才是!就算或有顾虑,也得剪其羽翼,以防不测。”

司马光虽然失败了,但对吕公著来说,一切才刚刚开始。

因人成事,这样的想法,他从来没有过。

低头整理着丈夫的被褥,向皇后藉机稍稍冷静下来。抬起头来,她猝然质问着吕公著:“韩冈如今只为不掌诰的内翰,王相公更是五日方才一朝,不及远甚。枢密是不是看到王珪去职,想争一争宰相的位子?!”

“殿下此言,是在疑臣。”吕公著面不改色,向皇后的举动在他眼里实在幼稚得可笑。他掏了一下袖袋,抽出了一份奏章来。吕公著双手托着奏章举过头,朗声对赵顼道:“臣之辞表便在这里。臣非恋权,旧年臣于王安石亦有举荐之德,若能如韩绛、陈升之一般附和变法,宰相之位何足论?今日之言,非为权柄,乃是臣为皇宋基业的一片赤心!”

……………………

城南驿,司马光所居住的小院紧闭的门扉打开了,司马康将刑恕送了出来。

虽然是送客,但司马康的脸色阴沉得像是送葬。

刑恕也是一脸沉重,却仍好言安慰着司马康:“先生是太齤子太师,多年来始终简在帝心,是天子垂危时想要托孤的重臣。虽说今日受辱于小人,皇后又为奸佞蛊惑,但无论如何,不还是给了先生一个体面吗?”

“体面?”司马康脸色却更加阴沉:“就是那些赐物吗?”

刑恕叹了一声,摇摇头,拍了拍司马康的肩膀,却也不在多劝了。

都到了现在这般田地,还能怎么样?

刑恕瞥了一眼稀疏的花木对面躲躲闪闪向此处张望的数个身影,转头又望向不远处的另一重院落。那重院落也是大门紧闭。

王安石这段时间在城南驿的作息习惯很稳定,此时乃是午后时分,他一般是不见客的。但王安石应该已经是知道了朝会上发生的一切。

刑恕冷笑了一声,不知道那位平章军国重事究竟是怎么看待他的那位女婿的?

昨日席上谦和有礼,今日殿上便翻脸无情。就算是亲如翁婿,恐怕也是适应不了吧?

但私谊归私谊,国事归国事。当年王安石能为变法事与多少好友割席断交,今天若是知道司马光大败亏输,当是击节叫好的为多。

唉……摇摇头,又是一声长叹,刑恕别过司马康,向驿馆外走去。

司马光的颓态,他方才看得分明。踌躇满志的跨进文德殿,结果却是丢盔弃甲,一败涂地。失去了唯一的机会,有生之年当再难入朝,如何不颓唐?

不过刑恕并不认为这是司马光能力不足,实乃天数耳。

司马光选择的时机和手段,不可谓不妙。在极为有限的时间内,已经是做到了极致。就算是刑恕现在再来回想,也觉得司马光借弹劾王珪来张起沉寂已久的旧党声势,并宣告自己重回朝堂,从任何角度来看,都是最为上佳的选择。

尤其是在御史台已经群起而攻的时候,抢先一步对王珪给出决定性的一击,不但能借助已有的声势,也让御史台根本没有办法调转枪头,只能追随在后。

让整个御史台为王前驱,难道还有其他更好的手段吗?

可惜还是失败了。

时也命也!

留中也好,拒谏也好,反驳也好,皇后可能的反应,司马光肯定都做了预测。而其他臣僚,无论是韩冈、章敦,还是蔡确、韩缜,包括下面的御史,以及一干有发言权的重臣,也定然都做好了针对性的计划。

在朝会上发难,本就是背水一搏,贯通史学的君实先生,不可能糊涂到不做筹划便仓促上阵。

可天时不在此处,皇后的那一句‘依卿所奏’,比什么样的反驳都有用。

谁能想得到?!

刑恕又是一叹。在廊道上擦身而过的一名官员,便随即浮上一抹幸灾乐祸的笑意。

冷淡的瞥了此人一眼,记下了相貌,刑恕继续向前。

幸好还有机会。

从这段时间,皇后对王珪的保护来看,天子很明显的是要维持朝堂稳定,异论相搅的宗旨绝不会随意更动。

既然如此,也不用担心对新党的攻击,会有太坏的结果。

司马光若是能将王珪扳倒那自然是最好,旧党肯定气势大张。若是做不到,对吕公著来说,机会同样到了。

宰相和执政之间,有着天壤之别。以刑恕所知,吕公著现在的唯一所想,就是光大门扉。而要想维持吕家的家门不堕,与其委曲求全的去迎合新党,还不如争上一步,争一个宰相之位出来。

宰相之门,即便韩冈日后当权,也不便有所轻动。韩冈就算将吕家恨之入骨,也得为他韩家着想——始作俑者,其无后乎?

吕公著若是能成为宰相,只要不糊涂到去沮坏新法,只要隔三差五唱唱反调,至少在天子大行之前,地位将会毫不动摇。

至于之后如何,更不用担心……王珪可都是被放过了!难道还能重开岭南路不成?

一旦吕公著如愿做了宰相,父子两相国,届时以吕门之贵,日后与天家结亲也不是可能。家门长保不衰,吕公著当真就能如愿以偿。

宰相门下客。

刑恕冷笑一声,似是不屑,却犹有几分自得。

不枉自己奔走之劳。

……………………

福宁殿中,向皇后仍阴着脸,气愤填膺,说不出话来。

而吕公著的气质越发纯粹,平和淡定,不见喜愠。

这是吕公著在表态。

表明与王安石决不妥协的姿态。

代表洛阳老臣的司马光今日折戟沉沙,旧党声势大挫,那么新党必然气焰大涨。这样的情况下,天子定然需要一位坚定的反对者留于朝堂。

除了他吕公著以外,还有谁人可选?就算什么都不做,也会稳当当的保住现在的权位。

可是吕公著还想更进一步。宰相的地位在枢密使之上。枢密使执掌军事,而宰相则是军政无所不统。

眼下王珪出外乃是必然。即便今天已经将所有弹章全部驳回,王珪也必须知趣的出外——这样还能留一个情面,若是还不知趣,那就没什么人请可讲了。

当王珪离开,空悬下来的宰相之位,在两府中以资历论,吕公著自问不作第二人想。其余人不是资格不够,就是进入两府的时间太短。

只有唯有一点,就是他是旧党。如此一来,即便是新党中资历浅薄如蔡确,中立的唯唯诺诺如韩缜,也有了跟自己竞争的资格。今日在殿上蔡确会跳出来,正是为了一个宰相之位。

吕公著无意改弦更张——即便他这么做了,坏了名声后,结果只会更差——那么能做的就只有一条:便是更加坚定的反对新法。一个保持为国事而不惜自身的旧党,与一名新党中人同掌大政,就是天子唯一的选择。

至于新法的稳定,在有王安石做着平章军国重事的时候,天子并不用担心太多。

这么多年了,又发生了这么多事,吕公著已经没有了与新法争竞的精力,他现在只想保着家门长久。他静静的等候着,结果究竟如何,就看天子的反应了。

躺在病榻上的赵顼终于有了动作,他的眼皮眨了起来。

一下,两下。

然后是第三下。

