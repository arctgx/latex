\section{第22章 早趁东风掠马蹄(上)}

韩冈对这个曲礼很有些兴趣,主要是因为他的籍贯。

倒不是有什么同乡之谊,韩冈之所以这么些年来,而是因为密州胶西下的板桥镇。

胶西板桥是新设了市舶司的地方。其所在位置大略就是后世的胶州湾。

自国初时,便有泛海浮舟的商人来往此处。到了熙宁八年,元绛奉诏出使高丽,大宋与高丽有了正式的外交往来之后,胶西板桥也越来越繁荣,最终使得朝廷决定在此处设立市舶司。

如今密州市舶司所管辖的胶西板桥港直接面对数以百千计的海商,乃至高丽和东瀛的商人,依靠对海船抽解和博买,这两年密州市舶司,都能上缴五六万贯的净收入,占到了南北各大市舶司总收入的十分之一,除此之外,还有大量的药材、皮草和战马被市舶司抽解和博买。这些北地特产,都是比钱更有价值。

从朝廷的收入来看,板桥港是如今的北方第一大港,规模远远超过位于胶东半岛北部,当年还没有衰落的登州和莱州两港。

这两座港口,本来是面对辽国、高丽和日本商人的主要商港。可在国初与辽国征战不休的那段时间里,因为有可能会被辽人的奸细由此处混入,故而被勒令禁止对外通商。

在韩冈眼里,这是个极其愚蠢的决定。封锁的结果,并不能改变河北边境处处烽烟、细作遍地的局面,而是直接导致了两座港口的衰败。

百年之后的现在,当年的禁令虽废弛已久,与辽人的商贸往来也不再是让朝廷忌惮的禁区,但元气大伤的两座商港已经被胶东半岛南部的胶西板桥港所取代——与高丽日本的联系,争不过密州胶西,而对辽国的通商,也无法与陆路抗衡,想起死回生也只能使镜花水月。

不过这件事也不是很重要。后世胶州湾在海运上的地位,本就是要超过登莱两地的,如今不过是走在正确的道路上。在韩冈眼里,只要海上贸易能更加繁荣就足够了,他可不是太关心到底是哪里繁荣。登莱也好,胶西也好,哪边的海贸兴旺都可以。

海洋的重要性不需要多说。大航海时代所带来的推动力,使得西方文明彻底从中世纪的黑暗中走出来。

只是大宋这个时代的顶级帝国,一切都能自给自足,对外征服的欲望很小,但沿海各路在近海水运上依然有很大的发展空间。如今两广各州至福建、两浙的近海运力,每年都是在大幅攀升,这一点从顺丰行从交州发回来的报告中,能清楚的看到。

赵顼也是知道海运的好处的,据韩冈从王安石那里听说,变法之初,议论起如何增加朝廷岁入,市舶司的商税也被当成一桩重要的议题,赵顼就曾经说过‘东南利用之大,舶商亦居其一。若钱、刘窃据浙、广,内足自富,外足抗中国者,亦由笼海商得法。’

何况对于大宋天子和朝廷而言,他们不会介意拥有一支强大的海军。不说可以轻而易举的压制高丽、日本,或是南洋,就是辽国,也会因为地理位置的关系,而被牵制到其在东京道和南京道的兵力。

从山海关到锦州的那一条路,全程都在渤海沿岸通过。如果不从辽西走廊走,辽国的南京道想要跟东京道联系上,就必须绕道燕山北侧的中京道,要多走上一两千里路,辽国的东京辽阳府,则是因为辽河的缘故,直接受到渤海水军的威胁。至于桑干河边的南京西京府,千石的船只更可以载着大军直接进抵城下。

控制了渤海,就是占据了一个战略性的制高点,让辽人不得不加强两路的守备,在战略层面上落入下风。

尽管到了冬日,渤海少不了要封冻,辽人厉兵秣马的时候,渤海水军无从发挥。但如果当真要设立渤海水师,目的就是进攻,而不是防守。以直逼东京辽阳府和南京析津府的战略攻势,来遏制辽人胆大妄为的躁动。

试想一下,就算是辽国的骑兵在某个冬天能突破河北前线的三关之地,但来年春夏,宋军的战士就能反攻向辽阳或是析津。大宋官军不再是一面倒的闷守,而是能做到深入敌境、攻击辽人心腹要害的反击。

如果能够在河北轨道修建的同时,组建渤海水师,大宋与辽人之间的攻守之势,将会就此完全逆转。

就算不用打仗,攻势和守势之间,国力消耗的差别也是高达数倍。一旦大宋能反过来以咄咄逼人的姿态压制辽人,以辽国的国力,支撑不了太久。

当然,现在将水军主力放在登州,等于是挑战辽人的神经,若是边境上的辽人做出个威胁的姿态,耶律乙辛再派个使臣来质问,朝廷里面随时都有可能来个友邦惊诧,将事情给搅黄掉。

可若是将水师的主力暂且驻扎在胶东半岛南面的胶西板桥,就没什么可担心了。辽人就算明知那是针对聊过,也不好。韩冈确信赵顼绝不会拒绝一个压迫辽人的机会,只要能找到合适的人选将整个提案捅上来。

韩冈虽还没有晋身宰执的行列,但早已开始放眼天下,如今只是先布局,落下几处闲子,可一旦等到了适当的时机,发动起来当能有让人惊喜的结果。

慢悠悠的走进包厢,除了在壁角站着的乳娘和侍女,只有严素心投来疑问的眼神。王旖、云娘和三个小儿女都是聚精会神的望着在前方跑道上奔驰着的十余匹骏马,浑没在意走进来的韩冈。

在看台上千万人的助威声中,石炭残渣铺起碾实的跑道,参加比赛的十二匹赛马纵蹄狂奔。如风驰电掣,转瞬间百十步的距离便一晃而过。

在在比赛开始前,赛马就回提前进入马栏。待比赛开始的号炮声响,马栏前的栅栏便会齐齐打开,而后一众赛马便从栏中奔出。

今天的第一场比赛,全都是新参赛的马匹,都没有什么名气。远远比不上这段时间正当红的青骓和掠影——这是模仿天子的那一匹浮光而起的名号,据说有着大宛天马血统。但看台上的此起彼伏不间歇止的喧嚣化作声浪扑进厢房中,却让人感觉不到这些赛马的默默无名。

韩冈走到栏杆边,也不坐下来,站着凭栏而望。就看到其中一匹高大神骏的河西马,一马当先,将其他赛马远远的抛在了身后,看模样就是一举夺冠的架势。

从袖口中拿出一个千里镜——这是何矩给他的,待到离开后就会还回去,绝不会带进城中——韩冈用千里镜看着赛场上飞奔在最前面的那一支,号衣上的红色十三很是显眼。

“飞里黄是赢定了!”王旖终于抬起头来,脸颊上有着因兴奋而来的潮红。

“不,他输定了。”韩冈摇头反驳。

“为什么这么说?”王旖和严素心齐声问道。

韩云娘盯着外面的比赛,根本没有注意到包厢内的对话,而王旖和严素心都是一头雾水,眼下赛程过半,但十三号飞里黄依然是排在第一。

“这可是长达六里的赛程,要绕场三周,一开始跑得太快,后面就会慢下来。”韩冈轻笑着解释,“十一号飞里黄的骑手是个新人,一上来就领跑。三号和八号这两匹马,体格不比十一号差,现在虽然混在众人之中,但他们肯定是准备将气力留在后半程发挥出来。”

仿佛是在配合韩冈的话,片刻之前还遥遥领先的十三号飞里黄,这时候已经跑得越来越慢,七八个马身的差距,也在转眼之间缩小了一半。

王旖和严素心看着直发愣,只听到韩冈继续在说:“从一开始就硬拼的那是蠢货。一场比赛要合理分配体力才有可能赢下来。若是赢了今天的这一场,就有资格参加更高一级的比赛。可若是以为这个原因,就只顾往前跑,那肯定是会被淘汰出局。”

在这场比赛中出场的所有赛马,都属于丁等一级,只能参加所谓的垫场赛。但赛马采取的是积分制,随着赛马成绩一步步的提高,积分越来越多,就可以一步步的向上升级。等级越高的比赛,奖金就越多。甲级的赛马,只要参赛,就是最后一名也有数量丰厚的奖金可拿。当然,若是成绩一直不好,也是会降级的,没有哪家马主会乐意一直拿最后一名的奖金。

“官人你怎么知道这些的?”王旖从来都没见过韩冈对赌马有什么兴趣,怎么看起来这么熟悉其中情弊。

“现学现卖而已。”韩冈解释道:“要知道,在赌场上庄家是不会输的。”

“庄家?”王旖楞然,立刻追问,“难道这赛马事先就被人定好名次了?”

“倒不是操纵比赛,但谁有实力,谁没实力,只要拿到资料,在比赛前预测个大概出来并非难事。有这本事的,也不止是总社中的成员。”韩冈坐了下来,冲着妻妾笑道,“要不要打个赌,今天的这第一场,买十三号这一对赢的,当不会有几个。”

