\section{第23章 铁骑连声压金鼓(三)}

【求月票,收藏】

当互为死敌的两方在出乎预料的时间和地点近距离接触的时候,无论是韩冈这边还是对面的党项人,作出的反应都完全相同。

不待韩冈命令,他手下的亲卫纷纷抽刀出鞘,卫队中最高大的三人,齐齐抢前两步,用自己的身体将韩冈挡在身后。而其他卫兵,则一下分散开来,围成了一个圆阵,连周围的青唐部族民也一起提防起来。在瞎药居城中见到党项人的踪影,传递进脑海中的第一个念头,就是今天是自投罗网、误入虎穴了。

惊讶的眼神闪出了凶戾的光芒,对面的党项人也几乎在同时把刀剑抽出。不论是抓了还是斩了一个宋人的官员,换到手的军功足以让他们这等小卒混上一个好官职,紧盯着韩冈的他们,就苍鹰见到了猎物。他们没有像韩冈的卫队一般,围成圆阵,把需要护卫的重要人物围在中间,而是头领在前的突击阵型。

看到党项人摆出的阵势,韩冈转头看着瞎药居住的主屋,如果他判断得没错的话,这些党项人的头领当是就在屋中与瞎药会谈。

双方隔着十步左右的距离对峙着,空气凝重得如同一根绷紧的弓弦。没人会怀疑,只要场中有一点异动,一场惨烈的厮杀就要展开。在杀机凝聚的战场边缘,青唐部的吐蕃人比两边的人数加起来都多,但没一个说得上话的主事者出头,让他们只能在一边干着急。

“机宜!”韩冈卫队的队正是个三十左右的老成汉子,不算聪明,武艺只能算中上,但他对韩冈把他提拔在身边感激颇深,故而忠心耿耿。他一边挺刀与对面的党项人,一面压低声音对身侧的韩冈道:“这里不能留了,俺们护送你冲出去。”

韩冈轻轻敲着挂在腰上的剑鞘,危在旦夕的紧张气氛没有干扰到他头脑的灵敏。插在华美的银边黑漆剑鞘中的不是装饰性的长剑,是一把良工打造的直刀。锋快无比的刀刃能轻而易举的斩断手腕粗的树干,乃是高遵裕前日送给韩冈的礼物。

不过若是在瞎药成了敌人的情况下,韩冈不觉得凭着这把刀,还有他手下的卫队能把他安全护送出去。如果瞎药还没有投靠到西夏一方,成为大宋的敌人,那他也没有必要把刀拔出来。

“别在人家家里打打闹闹,像什么样子?把刀都收起来!”韩冈下的命令让手下的亲兵为之楞然,但韩冈没有在意他们的惊讶,而是将身子转了个方向,面向主宅大门:“在主人家面前,不要让人说我们不懂礼数!”

韩冈的话一字不露的传入耳中,瞎药却站在大门前纹丝不动。听说韩冈来了,他立刻就找个借口从野利征那里脱身。只是当他快步从屋中迎出来时,却发现韩冈竟然已经出现在宅院的门前,与野利征的部属面对了面。这一惊,让他脑袋顿时都懵了一下。

震惊过后,就是一阵狂怒充斥胸臆。瞎药带着杀意的眼神,如刀枪一般戳向陪同韩冈的一名军头,‘怎么让两边见了面?!’只是当瞎药看清楚,究竟是谁人把韩冈引得跟党项人碰面的时候,他的眼神突然间就更加凶狠起来。

韩冈从瞎药的脸色中看出了一点名堂。回头瞧了瞧把他迎进来的那名吐蕃人。看来前面自己是想错了,并不是他在青唐部中的人望有多高,而应该是瞎药用错了人

——‘俞龙珂的手段也不差啊。’他暗自思忖着。

眼前的情况让韩冈也有些头疼。以他的经验来说,如果在无意中碰上了他人的隐私,如果不想跟人翻脸的话,最好的做法是当作什么都没看见,什么也没听见,给对方一个台阶下,这样至少可以在当面含糊过去。但这个经验,对于现在他所面对的局面,却又派不上用场。韩冈正想着解决的办法,注视着他的瞳孔却一下收紧。

从瞎药出来的地方,又走出来一人。穿着西夏的官服的中年蕃人,带着浓重口音的汉家官话,却不会让人误听:“原来是有贵客上门啊!”

瞎药被身后的声音惊了一下,身子又僵住了。他没想到,留下陪客的两个亲信竟然让野利征就这么走了出来。

野利征出来后,第一眼就看到了韩冈一众,暗道自己果然没有听错。他身份特殊,瞎药让手下的人把他安稳住,但他要走出来,就算是瞎药在场也阻拦不住。他走上前去,立刻就被他的部众被保护起来。隔着七八步的距离,与韩冈面对着面。

瞧着眼前在自己的城内对峙的双方,瞎药眼中凶光大盛,可转眼间便又深藏下去。他本想着在宋夏两边走着平衡,争取更多的利益。就像他一向瞧不起的兄长俞龙珂那样,在大宋、西夏、木征以及董毡四家之间来回摇荡,这样的做法,仿佛是在鸡蛋上跳舞,可十几年来,俞龙珂却一点也没出过差错。

如今轮到他自己来独立处置外事,却一下就变成了王见王的死局。瞎药明白,摆在他面前的选择只剩一条,不管是韩冈,还是野利征,总得挑上一边。两边的后台虽然都不是他能招惹得起,但事到如今,却也没有别的办法了,总要得罪一方。

韩冈打量着西夏人的使者,而对面也是同样投来审视的目光。

“韩机宜。”智缘的声音自他身后响起,低低的仅有韩冈一人能听见,“可记得徐令之子?”

‘徐令’这两个字所能容纳的含义实在太宽泛了,可能是人名,也可能是官名,还有可能是某个同音的词藻——韩冈并不擅长猜谜,对一些典故也不甚了了,正常情况下他是猜不到智缘究竟在说谁。

不过依照眼下的局面,智缘会提到哪一位名人,韩冈即便是用脚趾头去想也能想得到。而从结论倒推回去,徐令究竟是哪一位,那就很容易能找到答案了。

曾做过徐县县令的班彪,有着一对撰写史书的儿女,有着一个擅长辞赋的皇妃妹妹。但最重要的,是他还生了一个更为出色、千年以来始终受人赞颂的小儿子。

投笔从戎、远行万里、扬威西域的班定远,让千年以来的汉家士子,不吝用最热情的诗句去赞美。班超出使西域,在鄯善国中,以麾下三十六人夜袭匈奴使节,斩首而归,逼得鄯善王投向了中国。

在要招揽的对象的居城中,与敌国来的使臣狭路相逢,无论是韩冈还是智缘——不,只要稍稍读过史书——都会第一个想到班超这个名字。

智缘多读史书,作为一名侍奉佛祖的出家人,敢于来河湟争取边功,他的性子也与班超相仿佛。

只是韩冈比智缘要冷静得多,其中关节想得更为清楚。这里可是关西,直通着西域。作为关中出了名的英雄人物,班超的名字和事迹流传甚广,就算是蕃人,也只要稍有见识也都说出个门道来。想要夜袭党项使者,也得看瞎药答不答应。

韩冈摇摇头:“学不来的……”脸上浮现出的浅淡笑容中,有着让人无从揣摩的深意,“怎么也学不来。”

智缘的眼神黯淡下去,而野利征的视线却锐利起来。

野利征在武艺上毫无长处,身材又不高大,刀枪弓马都是平平,唯独听觉上的敏锐胜人一筹。站在七八步外,虽然没听到韩冈身后的和尚说了什么,但韩冈的回话他却听清楚了。

在野利征看来,汉人最大的缺点,就是把所有的异族都看作是毫无头脑的蛮人。若他们党项人真如汉人们说得那样愚蠢,当年景宗皇帝【李元昊】也不可能把东朝派到关西的主帅耍得团团转。

野利征一向自负头脑,当他发现了韩冈身后的那个和尚在说话时,也不把盯着他的视线挪开,便心知那秃驴是在说着自己。再配合上韩冈的回答,他头脑中便灵光一闪,明白了他们到底再说什么。班超的故事野利征也是听说过的,汉人要向西开边收复故土,总是少不了要提到班超。

以三十六人就在一国之都中斩杀敌国的使者,野利征也挺佩服班超这样的英雄。对比起韩冈的怯弱,更是让他心生不屑。韩冈名气老大,却是个没胆子的主,也就是汉人才会把这样的书生当作宝贝,真的遇到事的时候,就见到真正的模样了。

野利征今次唯一的任务,就是把瞎药招揽过来。本来他只完成了最低程度的工作,让瞎药不去掺和渭源堡的战事。不过现在既然有宋官来找青唐部族长的弟弟,又正巧正面撞上,这对禹臧花麻交待下来的任务,实在是求之不得的好事。只要杀了韩冈,瞎药还能往哪里去?

想到这里,野利征随即上前几步,用笑容迎上韩冈的双眼,像老友见面一般打着招呼、行起汉礼,心中则是一片杀机:

‘想做班超?我也一般儿想做啊!’

