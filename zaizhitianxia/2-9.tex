\section{第三章 收兵止戈留余恨(上)}

【第三更,求红票,收藏。】

要对付的敌人被解决了,领军出征的主帅因昏倒而无法视事,誓师出征也便成了个笑话,只能不了了之。

秦凤都钤辖这时躺在床榻上,双眼紧闭,嘴却微张,从嘴角不停的流着口水出来。向宝的幕僚一齐聚在房内,而韩冈则是坐在床边。

“韩抚勾,钤辖怎么了?”

韩冈才抬起头,一群人便紧张的围了上来。

韩冈一脸沉重,沉默的摇了摇头,如果穿上全套的手术服,再把个口罩挂在耳边,就活脱脱一个从手术室出来的一个手术失败的主刀大夫。韩冈不好把幸灾乐祸的表情露出来,但他真的想说一句节哀顺变。

以韩冈的气度,他当然能做到一笑泯恩仇。比如他现在,就可以笑着站到向宝床边,说,过去的事就算了吧。

韩冈晃了晃脑袋,收起了胡思乱想。向宝这些天给他的压力着实不小,让他连自残的招数都考虑过了,现在看着向宝成了废人一般的躺在床榻上苟延残喘,韩冈没大声笑出来,不是因为他道德水准高,而是知道他此时站的地方还不适合笑。

“在下不通医术,才疏学浅,无法确诊。各位还是赶紧为钤辖请个医术更好的郎中来。”韩冈摇着头,而这个说法,也是他一直以来所坚持的。但韩冈现在给人的感觉,却不是他不能确认,而是确认了不好说。

向宝的几个亲信幕僚互相看了几眼,眼中有着藏不住的忧虑,他们都看出了韩冈的言不由衷,而且向宝的病症,只要稍通医理,便不难看出。

“韩抚勾,还是说实话吧,钤辖到底是什么病?”

韩冈犹豫了一下,又回头看看张嘴流涎的向宝,摇头叹了口气,道:“风疾。”

韩冈不懂医术,但中风这个病还是能看得出来,他前世的亲戚中,很有几个中过风。在韩冈看来,向宝的这副模样,多半是脑袋里爆了血管,中了风。

向宝平日锻炼得是好,但他饮食从来都是酒肉不断,又是年过四十,身体没些隐患是不可能的。如果是正常的情况下,也许这些隐患要到二三十年后才会爆发出来,但方才向宝的心情在山巅和渊谷间的剧烈变化,却是将身体里的炸弹提前引爆。

“韩抚勾,你能确定?”有人还抱着一丝希望。

“能看出钤辖病症的,应该不止是我吧?”韩冈毫不犹豫的打碎他们的侥幸之心。“向钤辖这样的情况,得赶紧送回秦州,这里缺医少药,拖久了对钤辖毫无益处。”

“韩冈,你不是号称神医弟子吗?!”

“我向来只通治术,不通医术,这一点,我想各位应该都知道的。至于什么神医弟子,那些都是谣传。”

韩冈说着,却见向宝的幕僚都是恨恨的看着自己。方才他们也许希望自己能妙手回春,故而还有些恭谨。现在看到他没法救回向宝,眼神便都不对了。在场的哪一个想不通王韶为什么会抢向宝的生意,还不是因为这个站在他们面前摇头说‘没救了’的韩冈。

该不会给乱刀砍死吧?韩冈心知这样下去情况会对自己很不利,立刻道:“幸好向钤辖还可挽救……”

“怎么说?!”十几张嘴一齐追问。

“幸好向钤辖也不过才四十出头,年富力强,风疾也伤不了根本。调养一阵,只需一年半载,也就能回复旧观,倒不必太过担心。”

好歹得给人一点希望,不然他们在绝望下,不知会做出什么事来。

只是看着口水沾湿了方枕的向宝,韩冈觉得这个希望其实很渺茫,连带着他幸灾乐祸的心渐渐的都淡了去。不管怎么说,向宝的政治前途算是完了。风疾是重症,向宝越迟醒来,就代表他的病症越重,以向宝现在昏迷的时间,他即便醒来,怕也再难射箭练枪,说不定连下床行走,都将是一桩吃力的事情。

一个偏瘫风疾的将领,并不会受到朝廷的欢迎,天子也许会同情他,但不可能用他。不管过去向宝有多少雄心壮志,他已经没有机会在表现了。

从向宝的病房出来,韩冈在永宁寨中走着。由于方才发生的事,永宁寨内外已被紧急封锁起来。而寨中的士兵,除了值日在外的,都被约束在军营里,使得平日熙熙攘攘的永宁寨,倒显得空旷而少人气,全然不见赫赫有名的永宁马市的热闹。

永宁马市是陕西最大的马匹交易中心,每年朝廷通过永宁马市,用茶绢等特产购买到的军马几达数千匹之多。韩冈有心好好见识一下永宁马市的风采。只不过春夏时分,马市的一般都不算红火,只有到了秋高马肥的时候,才会有大批的好马骏马出现。

王韶也打算在古渭开办马市,想通过大量购入的战马,来博取天子的认可。不过王韶的打算被三司和枢密院同时反对,说离得蕃部太近,马市的安全难以得到保障。

不过现在,王韶他领着几个收服下来的蕃部,一起把托硕部个剿了。他的话语权应该增加了不少,再提设立古渭马市的提议,应该能得到天子的认同了。

在经略司通过军议后,王韶却自行其是,抢到向宝的头里去。他这么做算是违反了组织程序,违反了官场规则,同时让多少士兵失去了争夺功劳的机会,肯定要被人记恨上。但王韶也通过这件事,表现出了自己对蕃部的控制力和影响力,在天子和王安石面前的加分,那是不必说的。

其实说起来,对王韶的成功,意外的不仅仅是向宝,韩冈本人也是很惊讶。韩冈当日与王韶商议,其实也不过当作备选而已,但王韶最后竟然给他做到的,这结果比韩冈想象的还要好上一点。

能在三五日之内调集七部联军,一举击破托硕部,将其族长生俘。要知道,王韶根本调动不了古渭寨的三千兵,他没哪个权利。王韶最多也只能请动在蕃部中威望甚高的刘昌祚,帮他说几句话。

韩冈现在想想,可能是他太低估王韶这两年在蕃部中结下的善缘和人情了。调集七部联军,而且用来筹划的时间又那么短……

韩冈突然停步,王韶找来的蕃部数目好像太多了点,这么短的时间,若说没有外力相助,怎么也不可能完成。

看起来刘昌祚也是彻底站到了王韶那一边去了。

……………………

到了中午时分,向宝终于清醒过来。但也仅仅是意识清醒,他的身体依然不能动弹。

醒来后,当他回忆起半日前在校场中发生了何事,他下达的第一条命令,便是,“给我杀了王韶!给我杀了韩冈!”

向宝的门客僚属面面相觑,若是在战场上,还能报个失足落马或是中了流矢,但现在还在永宁寨中,如何还能动手?就算想找个借口治韩冈的罪,也得向宝自己能起来再说。何论他还要杀王韶!

“看来钤辖对韩冈误会很深啊。”韩冈叹着气,走进向宝的卧房,“不过,不管有什么误会,等钤辖病愈之后,都能有解决之道,就是现在不能再动气了,这对身体恢复并不会好。”

看着韩冈进来,向宝益发作怒,口齿不清的吼着:“你们还愣着什么,还不杀了他!”

没人听他的,没有一个人动弹。从现在开始,不会再有人听他说话。一个风瘫的将领并不为军中所需要,也不会为幕僚所礼重。如果在他身体健康的时候,他的命令也许会得到实行,即便是让他们去杀一个朝廷命官,说不定都有人亲自敢做。但如今向宝的情况变了,他的健康状况已经让他难以维持过去的权威。

韩冈也只把向宝的怒火当成耳旁风,他拉着向宝最为信任的一个门客道:“钤辖能自行醒来,这是件好事。日后经过一段时间的养病,应该还能恢复。只是不能再生气了,若下一次再发病,钤辖当真就没救了。”

门客点着头,回头看看仍不住咒骂的向宝,哀声叹气。韩冈在房中站了一站,便告辞出来。向宝骂起人来,中气十足,复原的可能性不算小,只是他肯定再也带不了兵了。

可怜吗……韩冈可是一点也不同情向宝。只看向宝一醒过来,就对自己喊打喊杀,就知道他没有半点反省之心。

‘从来都是你跟我过不去,我何曾欠过你!’韩冈心中恨恨的想着。

若不是与王韶商议的釜底抽薪,过两天躺在床上等死的就是他韩冈自己了。向宝纵然不敢耍手段杀一位文官,但找个借口给自己几十军棍,他却是敢做。杖责可轻可重,端看心情如何。如果换了自己,向宝自然是往重里下手。十几军棍打下去,任你壮比犍牛,也是要成废人。

两军争战,本就没有仁义道德可言。韩冈与向宝相悖如参商,相恶如敌国。之间的关系没有化解的可能,既然这样,至他于死地,看着他成为残废,韩冈确是半点心理负担都没有。

不知道这件事传到秦州,李师中他们的表情又会如何?

韩冈现在心中有些想看看那时候的秦凤经略的反应,应该比向宝在点将台上的晕倒还要让人痛快。

