\section{第十章 却惭横刀问戎昭(15)}

【第二更】

盐州城外,漫卷的黄沙之中,是一声声有气无力的号子。奔走于城墙上的身影,在沙尘中忽隐忽现。起起落落的木桩,慢悠悠的将黄土一点点的夯实在城头。

“这要拖到哪一天?当初修罗兀城也没见慢到这一步。”高永能正全副武装,盔甲都穿上了身,在甲胄外还罩了一身官袍,透过风沙,看着忙碌中的城防工地。

不远处,一名监工在民夫们中间来回转着,时不时就是一鞭子上去。被鞭打的民夫速度快了一点,但等监工绕过去,就又慢了下来。

“李转运提拔了一群泼皮,只管杀人、鞭打,也不见城墙修得更快。”高永能看着这一幕,抱怨着。,

“少说两句吧。力气用在西贼身上。”高永能的身边,老将曲珍没什么精神的搭着话。

两人一同守候在盐州城的南门外,在他们身后,几名偏裨将佐则是扶刀肃立,动也不动。

为了加固、加高城防,盐州城正在大兴土木。现如今,周围十里的城池四壁正同时开工,以求加快进度。

城壕已经拓宽了一倍,掘出来的泥土全都成了城墙的一部分。等到城墙完工,再将引走的水流引回来,盐州城便能拥有一道三丈宽的护城河,若能深掘出几个泉眼,

但大兴土木的另外一面,便是遍地的沙尘黄土。掘出来的黄土,被秋风一吹,就成了漫天的沙尘。一阵清风过去,城里城外就登时多了一层黄沙。

高永能挂着脸望着南门外的工地,只要一张口,带着咸味的沙土,就直往嘴里钻。要是往日,将口罩一戴也就没事了,可现在是在恭迎徐学士,不恭敬的举动,还是能免则免,省得被记恨上。高永能是不怕徐禧拿自己开刀,徐禧还不够资格,但自家的子侄亲信不少,得防着被牵连。

这股子郁气,既然不能在徐禧面前发泄,也就只能累着高永能身边的曲珍的双耳,“盐州本地征发了一万多民夫,从环庆又是送来一万,怎么两万人一齐上阵,这城防才完成了一半?难道要等下雪时再完工。”

曲珍叹了一声,说到了心里正烦的事,最终还是搭了腔:“民夫少一点,给他们口粮多一点,也许还能快上几分。”

制盐是一项消耗大量人工的产业。故而盐州的人口在银夏之地,是超过宥州、夏州的大城。但盐州城的规模,却并不是。盐州之所以能成为西夏的财源,靠的是城北十里之外的盐池。城池本身无足轻重,在这里修筑高大的城墙并没有太多的意义。

蛮夷不擅生产,青白盐池的盐丁大半是汉人。而且日后盐池重启,还要靠盐丁们卖力,对他们不能过于苛待。连同他们的家人,全都得养起来,这就是一万多的人口。即便将党项人全数撤走,整个盐州的人口还是几近三万。

这些盐州百姓在官军夺城之后就逃散了一部分,剩下的在官军的驱使下,投入到了筑城的劳作中。算是以工代赈,不仅仅是让他们吃饱,而且还得有多余的份量,让他们拿回去养活家人。

除此之外为了尽快将增筑的工程完工,徐禧又从后方调集了一万民夫。想要在四十天之内,将工程全部结束。

但民夫和他们的家人加起来有四万,在盐州驻守的官军近三万,东面一点的宥州还有一万大军,光是为了给八万人——另外还有六千多战马——补充粮草,就让环庆路伤透了脑筋。战事已经持续几个月下来了,陕西的民力几乎都耗用殆尽,经常有补给不上的时候。军粮无人敢克扣,所以减少的只能是民夫们的口粮。吃不饱饭,当然也做不了力气活,逃亡的民夫一天比一天更多。

便因如此,预计四十天完工的城防,到了一个月的时候,才完成了一半,至少还要一个月。高永能怎么看都不觉得

“盐州这个地方,筑个什么鬼城?!党项人又不会攻城,两丈半和四丈有多少区别。神臂弓往下射就是了。兵精粮足,就是草就的军营都能守,有城墙的大城有什么不能守的。我只要四千本部,将京营的那群白痴都调回去,守住盐州的把握,我还能多上两成。”高永能愤愤不平的说着,向右手边瞪了一眼过去。

就在十几丈外,除了高永能和曲珍这一群西军的将校,也有一群身着武官服色的汉子,高高低低二十七八人,也在窃窃私语,不知说些什么。那是来自东京开封的一众京营将校,盐州城中,两万京营将士便是他们的属下。这一群人占着从南门延伸出来的道路的正中央,明显比站在路边上高、曲二人所领的西军更加得势。

西军和京营两边泾渭分明,互相之间连话也没有多一句。相对于跟在曲珍、高永能身后的几名校尉各自静默肃立,京营那边的声音就大了许多。主将们说话不足为奇,下面的军官也都在窃窃私语。这在军纪森严的陕西禁军中,是不可想象的。

这就是徐禧要用来抵挡西贼决死反扑的主力。

看到他们,再想想徐禧,曲珍和高永能都对近在眼前的战事,悲观至极。都想找个由头离开盐州,不在这里担惊受怕。被连累得一世英名尽丧怎么想都不会甘心。

就是保住盐州的局面又如何,统帅之功是徐禧的,军功的大头是兵力更多的京营的,自家不但没多少功劳可领,还要冒着大风险的拼死拼活。对于点名让自己留下来的那位,曲珍、高永能可是厌烦透了。

他们也是在军中几十年,与不少文臣打过交道,名震天下的夏悚、范仲淹,少年得志的韩琦、韩冈,各色人等都都见识过。但如徐禧这般不靠谱又惹人厌的顶头上司,还真没见过几个。

说起来两人都宁可放弃盐州的功劳,也要离得徐禧远远的,可事情的发展并不以他们的想法为转移。领军镇守盐州,陪着一群京城来的衙内兵,一起等着西贼攻上门来,还有比这个更憋屈的吗?

高永能狠狠的啐了一口,将心头的不屑、愤懑连同嘴里的沙子一起啐了出来,“一群废物,在金明池里踢球不就好了,跑来争什么功劳。也不扯开裤子低头看一看,软得都站不起来的鸟货,还想上阵跟人厮拼。”

南方的远处尘头大起,一小堆作为先导的游骑已经快要到了近前。

曲珍和高永能不约而同的闭上了嘴,

片刻之后,徐禧和千名骑兵就到了盐州城下。

西军、京营两边的将校一齐迎上去,向徐禧行礼。

半月不见的徐禧依然是自信满满,看到城防的进度,虽然变了一下脸色,但立刻就又浮现起自信的微笑:“本来还担心最近的风沙太烈会阻碍筑城,但现在看看,还是比预计的要好。”他朗声向众将宣示:“故善战者,致人而不致于人。官军占据了盐州,西贼就得拿性命来拼。穿越瀚海而来,人困马乏,粮秣又难以补充,只消能守上半个月,西贼将不战自溃。就算他们能咬牙坚持,从环州、夏州来的援军,也能让他们有去无回!”

……………………

狂风卷着沙尘,劈头盖脸的迎面砸来。种朴披着连帽斗篷,又用口罩蒙着口鼻,低着头,沉默的驾驭坐骑向前行进。在他身边,四百余名的骑兵,正踩着前人留下的脚印,步步向前。

依照朝廷的命令,一旦西贼举兵攻打盐州,屯兵夏州的鄜延军是要出兵救援的。尽管他父亲另有盘算,但在第一时间把握到西贼的动向,同时保证沿途的安全,照样是免不了的。

种谔治军严明,种朴身为他的儿子也没有多少优待,被派出来巡视无定河北面的荒漠,以防西贼偏师埋伏于此,等待伏击援军的机会。

在风沙中行军,仿佛是盲人瞎马,眼前是黄蒙蒙的一片,除了脚下的一小片地,什么也看不见。幸好种朴身边有着精心挑选的识途老马,在这一片土地往来了几十年的向导,知道在荒漠中哪里有水源的存在。就算有黄沙遮蔽视线,也能准确的指引着种朴的这一队人马,往前方的水源地暂时落脚休息,避一避风沙。

种朴一行,一心想赶去前方的水源地暂避风沙。但这一场沙尘却在他们的行进中,莫名其妙的消失无踪,转眼之间,眼前不再是昏黄一片,抬头便可见到澄蓝的天空。

可这时候,种朴和他麾下的四百余名骑兵,却没有了抬头望天的余暇。就在他们的侧面,出现了一支军队,观其前进的方向,也是荒漠中的那一个绿洲。

两军相隔仅有一里,快马转瞬可至。以战马的速度,可算是近在咫尺。方才是因为风沙阻隔了耳目,现在风沙一停,两边几乎同时发现了对方。

“那是哪一家的兵马?!”种朴大惊,眯起眼睛神色紧张的望着对面。

