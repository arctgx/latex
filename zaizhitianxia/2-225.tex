\section{第34章 山云迢递若有闻(六)}

韩冈签发命令,将渭源堡中的大半民伕,转移到野人关和庆平堡中。

尤三石也接到了命令,带着他麾下的保丁,便要往城外去。只是走到营寨门口,脚步却停了下来,身后的保丁也都一片低声的叫道,“刘指挥!”

坐在营门内侧的空场边,乡农一般打扮的中年人,竟是尤三石早前所在的那个指挥的指挥使刘源。

而后保丁们又是一片声在响:“陈虞侯!”“胡都头!”“张都头!”

除了指挥使刘源,聚在营门一角的,竟然一个个都是过去广锐军中的将校。或站或坐,皆在闷着头做着自己事。

尤三石曾听说曾经统帅三千广锐叛军的将校们,都被安置在陇西县城外,被牢牢的监视着,想不到今次也被征召了起来。看到曾经指挥过自己的将校,尤三石下意识的就要单膝跪倒,但立刻又想起了现在已经不是广锐军中的时候了,身子却僵住了。

见着一个眼熟的家伙冲自己半躬了腰,却又不跪下去,刘源抬了抬眼皮,“做你自己的事去,傻站着做什么?”

尤三石叉手行礼,提着弓刀,忙着带队出城。跟着尤三石的一群前广锐军士卒,也都是先行过礼,然后才出城而去。

为了救援吴逵,广锐军能一呼百应,便是因为官兵之间的关系要远胜他军。别的不论,单说吃空饷的情况,平常关西军中都是两成,只有广锐军才不过一成。即便是广锐番号烟消云散的现在,旧时的关系依然还留有残迹。

坐在一块石碾子上,刘源手提大斧,拿着磨刀石慢慢的将斧刃一点点抛光。在他旁边,有的人在给长弓换弦,有的人在擦着刀。虽然已经从马军变成了步军,从将校变成了罪囚,但武艺还是留在了身上。

韩冈远远的望着这一角落中的动静。两百多旧时将校气息沉稳如山,气定神闲的模样,与普通军士给他的感觉,便是截然不同。

西军不是京营禁军,也不是河北禁军,多年战乱,使得西军上下皆以武艺量人。随便拉出来个小卒,都能开八斗弓,三石弩。而将校们,尤其是指挥两三个十人队的十将到管辖五百人的指挥使,这一阶层的军官,基本上各个都是弓马娴熟、武艺精强。且能在属于骑兵部队的广锐军中立足,发号施令的将校,更是没有一个会是弱者。在韩冈看来,这可是比各路选锋更为精锐的战力。

蔚然一笑,他转身回厅。

没有近三百名由前广锐军的将校组成的队伍压阵,韩冈如何敢把出发地的渭源堡留着只剩不到千人。就在半年前,可是有着罗兀城的先例在,看到抚宁堡被夺占,他怎么可能会不提防吐蕃人偷袭渭源。

韩冈不知道会不会有人来偷袭渭源,但他翻看过往战例,将帅的侥幸心理是大军败阵的主因。他并不认为吐蕃人能大胆到来偷袭渭源,但只要有一丝可能,他还是决定把这群叛军将校都征调了上来。不论他们有没有派上用场,光只是存在,就足以让渭源堡守得稳如泰山,也能让自己放下心来。

而相对的,韩冈为此付出的代价,就是不可预知的风险。并不是说这些将校还会有心反叛,而日后很有可能会有人拿这件事来攻击韩冈任用叛贼——叛军中的军官和士兵,在天子眼里是两回事。一方是预谋有份的叛贼,而另一方基本上就是遭受蛊惑、逼不得已的可怜之人——

韩冈调用叛军士卒组成的保丁为民伕,无可厚非,甚至在一些人眼里,这是叛军们应该受得苦。可把叛军军官聚合为兵,这份责任他担在身上,一旦败事,便是一桩逃不过罪责。

韩冈不怕承担责任,利益和风险他都已经衡量过了,如果有罪责临身,他甘于承受。但如果有事发生,比如现在冲进来的急报,却就是他的先见之明了。

“瞎吴叱的胆子什么时候变得这么大了?”

韩冈没有一丝惊讶,只是在冷笑。

渭源堡中战鼓擂起,王中正在慌乱中,匆匆上了城墙,找到了挺立城头的韩冈。

王中正本是准备要回陇西,只是途径渭源。他亲身跟随王韶进了临洮城,功劳已经挣足,下面就是返回安全的陇西城,等着他的任务结束,功劳到手。

王韶也希望王中正能回陇西,他前面命蔡曚来临洮报道,可秦凤转运判官不肯听命。王韶并不指望王中正会插手进他和枢密院的博弈中,但只要蔡曚能当着他的面,把自己的命令再次拒绝,那也就足够了。蔡曚不从号令的行为落在奉旨监军的王中正眼里,王韶将其下狱,就是名正言顺。如果蔡曚顾忌王中正而接令,那就更好。

王中正也知道王韶的用意,顺手就把事接了下来,这样可以名正言顺的回陇西。只是他的运气算不上好,才刚刚想在渭源休息一夜,便在床上听到战鼓催动。

在震耳欲聋,不断激荡着的鼓声之中,王中正凑到韩冈耳边,大声叫着:“韩机宜,这怎么回事?!”

韩冈微笑回头,“都知,看来你得在渭源堡留上两天了……有贼偷袭渭源!”

鼓声阵阵。刘源等一众将校已经列队,韩冈此时正站在他们的面前。

视线扫过这一众叛将,他们的神色恍若无事,只有眼神中时不时的闪过热切的光芒。

韩冈:“诸君旧日皆是军中柱石,阴差阳错才变成了今天的情况。再想披挂领军,那是不可能了。但你们的儿孙还是有机会的,只要他们不受牵累。是否能为子孙脱去贼名,就看诸君的奋战。”

众人之中,刘源是官位最高的指挥使之一,而他又是指挥使中年纪最长的一人,一众便是以他为首。他躬身向韩冈道:“韩机宜,我等多承你的救命之恩,全家亦是有机宜你,才方得保全。今次既然贼军来袭,机宜有用到我处,我等岂有坐视之理。无有他话,只有效死而已!”

一个许诺,一个承诺,刘源掌中大斧随之一转,便带着一众将校,走到栅栏边,直面来敌。

韩冈重新回到城头上,吐蕃人的旗号已经出现在渭源堡外。

由于临时囤放军资粮秣,前日又驻扎了大军,在堡垒外侧,增筑了一圈栅栏。括起来的空地,便成了营寨和仓囤。区区千人不到的守军,其中还有两百在渭水对岸的北堡中,要防守曾经驻扎过万军的营地,其实是杯水车薪。而营寨之外,浩浩荡荡,却差不多两千多吐蕃骑兵。

算过了兵力对比的差距,王中正浑身的冷汗都冒了出来,“韩机宜,不点烽火吗?”

“区区贼军,何止于此?”

点燃烽火是向东通报给朝廷,根本无济于事。向西招援的信使则已经派出,还不如看看怎么将对手解决。

吐蕃人来势汹汹,到了渭源堡外,根本不事休整。主力稍停,而三百多前锋便直奔南侧的寨门而来。

刘源领军正在此处。三百多蕃骑冲杀渐近,坚实的大地都在颤动。以他们来势之猛恶,看起来十分脆弱的栅栏,说不定能一举冲破。

比来敌数目略少的前广锐将校们,则是看不出半点慌乱。无人号令,各自张弓搭箭,蕃骑尚未冲到营栅前,一阵箭雨便离弦而出。

这些都是怎样的高手。

王舜臣的连珠箭术,韩冈看到了;刘昌祚的巨弓重箭,韩冈也看到了。近三百将校,无一不是精于弓马,仅是转眼之间,就把当先冲来的蕃骑射落了一片,人仰马翻,飞扬的尘土之中,只有惨嘶悲鸣传出,甚至没能让他们靠近栅栏。

前军顿挫,后续的骑兵立刻收缰止步。最后只剩十几二十骑,一时收拾不住,在箭雨中冲到了营栅边。

刘源不知何时已翻出了栅栏外,一弓腰就杀进了这队蕃骑之中。人马纷乱,刘源一时间消失了踪影。当他再出现时,却不知怎么就窜上了一匹战马,原本拿在手上的重斧,已变作一杆大枪在挥舞。长枪吞吐,转瞬间,就把左近的几名蕃骑都扎下马来。

“此人武勇当不逊旧年的郭遵、张玉!”

城头上,看着刘源大发神威,将来袭蕃骑一个个挑下马来,王中正乍舌不已。

韩冈玄然一叹:“可惜他是个罪囚。”

王中正神色微变,转头看向韩冈,眼神深沉,“韩机宜你是要保他的功劳……”

“不。”韩冈摇了摇头,不可能的事他不会去指望,“身为朝廷命臣,附贼做反,能饶了他的性命,已是天子恩德。最多是免其过往罪衍,让他的子孙不受他的拖累。”

“这倒没问题。”王中正神色一松,虽然要看三代,但还是没人太在意。张得一为贝州反贼王则写,他的两个兄弟照样做官。他认同了韩冈的说法,“天恩浩荡,若此辈有心改过,当无不允之理。”

前广锐军的将校们,犹在奋战之中。

刘源挥舞着长枪,抢下了十几匹战马,加上一开始骑手被射下来、战马还没来得及逃回去的。转眼就是三十几人翻出营栅,跳上马去。

杀人夺马做得行云流水,王中正在上面都看得目瞪口呆。

可毕竟这一队宋军人少,瞎吴叱和结吴延征也没想过会太顺利,单是发现渭源堡中兵力不足的情况,就已经很鼓舞他们了。

号角重新响起,刚刚正在修整中的蕃骑纷纷起步,冲着似是脆弱的营地,杀奔而来。

