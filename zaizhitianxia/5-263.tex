\section{第29章 浮生迫岁期行旅(三)}

路明心里有些急,早早的就站到了船头,望着虹桥,望着城门,望着码头,等着船到地头。

他这一路东行,路上在潼关道耽搁了,整整三天。而出发时,也因故迟了两天。幸好从洛阳过来后,河上有些积雪,正好能坐船,好歹也挣回了一天半天的时间。 

说是船,其实就是船型的雪橇。 

更明确的说,是在用来载人的小型河船下面安上两条雪橇板,是一种只在开封周围的运河中使用的水陆两用的交通工具。 

如今雪橇车已经在北方普及,甚至还有发展,比如根据底盘上安装的是冰刀还是雪橇板,便分为在冰上行驶的冰橇和雪上行驶的雪橇两种。汴水冰上积雪未化,厚厚的近一尺深,正是雪橇车的用武之地。 

在京城附近,也只有冬季封口的汴河等运河中,才会有稳定连贯且平滑的冰面,不像有流水的自然河道,冰层凝结缓慢,而且往往多有坑洞和起伏,雪橇、冰橇在上面行驶容易损坏,而且拉车的驴马等牲畜也容易伤到蹄子。 

六匹骡子拉着船,在雪面上滑行,穿过城门,穿过两重虹桥,直抵雍秦商会专属的码头上。

路明踩着踏板下了船,码头上的龙门吊带着个网兜垂了下来,船上的一点货物给丢进了网兜里,很快便给卸了下来。路明的伴当整理着这些礼物,码头上的管事已经小跑了过来。 

路明心中焦躁,但仍耐着性子与管事扯了两句闲话,就这几句话时间,一辆马车驶了过来。看了看天色,时间尚早,章敦也好、韩冈也好,都不会有时间。带着一名伴当直接上了车,路明吩咐着车夫:“先去会馆。” 

至于他带来的货物,管事早就熟练地安排了另外的一辆装货的马车,跟在后面。 

在马车上,路明一直皱着眉。 

他之前已经在洛阳的雍秦会馆听到了一些消息。有关天子发病的消息,有关皇后垂帘的消息,有关十天前大拜除的消息。 

皇帝中风,皇后垂帘,这当然是让人惊讶的大事。但这十几二十年来,仁宗驾崩,英宗驾崩,曹太皇垂帘,以路明的年纪,其实已经习惯了。他关心的是自己的两个后台的境遇,当听说韩冈在冬之夜的表现后,对京城的局势,也就放心下来了。 

但接下来的大拜除中,几个已经消失在记忆中的姓名重新出现在人们面前,甚至是让人想象不到的人事安排。旧党彻底倒台,新党大兴,但他的恩主却没有动,韩冈则是接连拒绝了参知政事和枢密副使的任命。不过在洛阳的会馆中,他也听人说,是因为官家硬是让王相公和洛阳的大程先生做太子师,惹火了新任翰林和资政两学士的小韩学士。 

只是不管怎么说,京城的局面都是不用担心。后台地位稳固,路明在洛阳其实没少受人羡慕,只是他心里还压着一件事,沉甸甸,让人笑不出来。 

到了雍秦会馆,他便遣人去章府递个帖子。晚上就可以直接上门去见章敦了,他是章家门下客,这就是跟官员不同的地方。就是不知道现在入住章家方不方便,只能先在会馆里等一等。 

京师的雍秦会馆,成了在京的关西商人,以及不少陕西籍在京官员的聚集地。谈天说地,顺便开拓一下人脉。对于商人和官员来说,他们的互补性其实是很强的。这一个集会场所,很是受到欢迎。 

在会馆中,路明大大小小也算是个名人,有当年西太一宫题诗的谣言,更多的还是他背后的章敦和韩冈。一见到他,便有不少人围上来,强拉着他入席说话。路明左推右让喝了两杯酒,方才脱身。这么一闹,他也不敢在会馆中久待,梳洗过后,换了身衣服,带上礼物径直去了章府。 

在章府内坐了半日,放衙的章敦终于是回来了。 

在路明眼里,章敦的气色不差,并没有因为在大拜除的时候没有收获而失望气馁。而在章敦的那一对利眼中,路明的神色就很不对了。 

甚至连寒暄都没有两句,分宾主坐下后,章敦就问,“明德,是关西那边出事了?” 

“是青铜峡。”路明沉声道。 

自宋辽分割西夏后,一年多的时间过去了,契丹人已经完成迁移工作。黑山河间地自不必说,那已经是耶律乙辛的斡鲁朵所在地。而给他当成战利品分给麾下部族的兴灵,也逐渐有了越来越多的契丹人。 

“抵达兴灵的契丹、渤海和奚族等部族,已经达到了四万帐。汉人还好说,党项人都看不到踪迹了。”路明说着他打听来的消息。 

“青铜峡那边呢?”章敦心急的追问。 

“形势不妙!” 

黄河穿过青铜峡流入贺兰山以东的兴灵地区,在两国和议之后,青铜峡河谷,全都是从兴灵撤出来的党项部族。 

一方面党项人对辽人恨之入骨,叶家和仁多家都极为敌视契丹人。但另一方面,党项人欺软怕硬是有名的。当越来越多的受到辽人的压迫后,他们更可能投向辽人,然后配合契丹铁骑向南劫掠大宋。 

“自峡口以南五十里,不得修建城寨。如此一来,青铜峡的党项各部永远都不可能定下心来。” 

这自然是很危险的局面。章敦紧锁着眉头。基本上从渤海,一直到西域,宋辽两国的万里疆界中,便以这一段防线最为薄弱,是明摆着的突破口。 

所以在青铜峡河谷南端的鸣沙城——距离北端峡口近六十里的地方——囤积了整整六千禁军。而其后方的应理城,同样还驻扎了一个将五千西军精锐。 

“刘仲武怎么说?”章敦问着他在军中的心腹,也是现如今的环庆路都钤辖,鸣沙城城主。 

路明立刻从袖子里掏出一封信,上面的火漆印痕宛然。 

章敦接过信,也懒得拿刀子拆信,直接就将封口给撕开。抽出厚厚一摞信纸,哗哗哗的一目十行扫了一遍,脸色更是阴云如晦 

路明担心的看着章敦。章敦则又从头到尾的看了三五遍,方才放下信纸。 

“光从环庆路发来的奏折上看不到这些详情。想不到局势已经败坏如许。当时要跟玉昆好生商议一下。” 

章敦叹着。随即拿起笔,匆匆写了个帖子,正想交给一名亲随,但又收回了手。转对路明道:“不急这一天。明天当面请韩玉昆来家里喝酒。” 

虽是这么说,但章敦眼中忧色难解。萧禧即将进京,辽人肯定会配合他行动——否则耶律乙辛就不会让他做正旦使。一旦青铜峡中的三万帐党项部族被契丹人给驱动了,那么环庆路可就要面临一场大战了。 

若是天子没有中风,那只能说是送功劳来了,上上下下都会摩拳擦掌。可如今女主临朝,实在是不能让人安心,而且还是没有任何军事经验的皇后。一旦辽人收到这个消息,第一件事就是立刻增兵,甚至亲自上阵,而不是用党项人做代理。 

“不是说官家还能理事吗?”路明安慰着章敦,也是在试探。 

章敦下意识的摇摇头,用眼皮理事,控制一下两府人事已经是了不得了,如何应对得了边境一天数十封急报。而且眼下病重的天子虽然还有意识,可估计他也不会放心让名帅大将领兵于外。 

这么些年来,章敦也看透了,这一位皇帝……猜忌心实在太重。 

如若辽人当真想要再讹诈一回,这一次说不定就是熙宁时河东弃土的翻版。 

当年辽人胁迫于外,元老恐吓于内,这位皇帝就开始逼着前面谈判的韩缜签字割地。 

已经离开朝堂的一干元老,如韩琦、张方平、文彦博,一个个要皇帝念在宋辽两国百年通好的份上含辱忍垢,又说与辽人交战必败。反正他们都不在朝中,说话不嫌腰疼。台上的宰执,无论新党旧党却是拼了命要拦,割地后坏的是他们的名声。负责与辽人谈判的韩缜也是咬牙不许,不肯坏名声。 

可天子的决心下得很快,直接绕过两府,给韩缜去了一封密信,‘疆界事,朕访问文彦博、曾公亮,皆以为南北通好百年,两地生灵得以休息,有所求请,当且随宜应副。朝廷已许,而卿犹固执不可,万一北人生事,卿家族可保否?’ 

皇帝都拿家族来相要挟了,韩缜哪里敢再硬挺着,直接就老老实实的按照辽人的要求划界了——有所求请,当且随宜应付嘛。 

若是这一回还玩这一手,那还真是难办了。 

左思右想,玻璃灯罩内的烛花爆了又爆,章敦终于放弃了,有些事,徒在家中苦恼也无济于事。 

“明德你这一次回关中,玉昆那边有托你做什么吗?”章敦随口问着。 

路明点点头,“倒是有的。韩资政着小人去搜集大食的书籍,说是要借鉴一下。还委托小人去找个通译。” 

章敦对韩冈这个爱好忍不住要皱眉:“译经润文使那是宰相的差事,他放着枢密副使不做,倒是操起宰相的心了。” 

路明也不知韩冈的想法,“其实还有些种子。冯四那边也在搜集。就在巩州,韩家的庄子上新近用玻璃做屋顶,搭起了一间暖棚,说是冬天种菜。” 

京城这边,有温泉蔬菜,有窑洞里的暖房蔬菜,反季节的菜蔬数目不少,章敦不以为意,也就用玻璃来做让人惊讶了一点。但他章家如今也在建玻璃工坊,玻璃的原料到底是什么,他倒是一清二楚。真不是什么奢侈的东西。 

算了,他叹了一声,等明天跟韩冈商议了再说。 

