\section{第43章 庙堂垂衣天宇泰(19)}

“依律,人子过继之后,与生父母再无瓜葛。但英宗皇帝是怎么做的?”富弼忽然剔起的眉眼,显示他对十几年前的旧事,依然余怒未消,“仁宗大奠,梓宫之前,英宗称病不至,天子不可能没看到;太皇太后对两府哭诉英宗不孝,天子不可能没听到;英宗要追尊生父濮安懿王为皇考,当着天子面做的;韩琦使人灌醉太皇太后,伪传懿旨,同意追英宗生父为皇考,天子虽然不曾亲眼见证,肯定也有耳闻!”

富绍庭默然,自己的父亲以当年之事为恨,他是一直都知道的。

“没有儿子,帝统旁落,绝嗣的后果,天子决不会愿意看到。仁宗晚年,与曹太皇夜坐对哭,是因为什么?绝嗣啊!而韩冈名望再高,还能造反不成?总有挡着他的人。”富弼一个劲的摇头,嘿嘿冷笑,“前事历历在目。天子想要这样的孝子贤孙?!皇后想要这样的孝子贤孙?!……只要能帮他保住儿子,韩冈做得错失再多,名望再高,皇帝一根寒毛都不会动他。”

富绍庭只觉得体内的水分都化作冷汗流光了,整个人都变得麻木。低头恭维道:“也只有大人能看的通透。”

富弼得意的扬起胡须:“皇佑、治平年间的宰辅也没几个了,当年的事,台上的有几人亲眼见证?御史台那些毛头小子当时还不知在哪里窝着。也只有王珪,当初做着翰林学士的……为父敢打赌,这一次,他这位三旨相公只是冷眼在看,一句话都没有多说。”当世硕果仅存的两位三朝宰辅中的一人冷哼了一声,“御史台中当真是一代不如一代了,两只眼珠子只知道看着皇帝,一心只想踩人头上跳上去。都不想想后宫里面,要保住韩冈的有多少?事关皇嗣,后妃们劝一句比御史说一百句都管用。”

咂了咂嘴,富弼突然又挂下了脸:“韩冈肯定也是看明白了。至少看透了大半,所以才敢将种痘法的来龙去脉全都和盘托出,有恃无恐……现在后生小子,还真是……”

富绍庭脑袋在发懵。

富弼和所有老年人一样碎着嘴感慨了一阵,突又问道:“记得当年韩冈跟雍王争夺花魁的事吧?”

富绍庭点点头,怎么可能不记得?这可是传遍了天下的风流轶事!据说在南方还有人编成了说书的段子,不过改了人名、朝代罢了。在这些故事中,那位与穷措大抢花魁的亲王,都是可笑的反角。

“那为父问你,将为父、文彦博、韩冈摆在天子面前,你认为天子要托孤时会选谁?”

富绍庭整个人更是怔住了,空张着嘴,如金鱼一般无声的一张一合,不知该说什么。

只听着富弼朗声总结:“在皇子成年之前,天子绝不会动韩冈的,只会将他留在京中,保扶皇子!等过个十几年,如今的怒意,又哪还会留存到那时?早就一笑了之了。”

又过了两天,从京城送来了一份邸报,富弼拿着一看,顿时哈哈大笑而起。

“看看为父是怎么说的,”老头子都有了小孩子的得意,“病急乱投医,只要是根稻草,天子都会抓着不放,何独韩冈。”

富绍庭接过邸报,前两条无关紧要,第三条就是以尽死保赵氏孤儿事,以程婴为成信侯,公孙杵臼封忠智侯,立庙祭祀之。

他摇头叹着,还真是病急乱投医。

……………………

“这是病急乱投医吧?”方兴抬眼问道。

“当然不是。”韩冈斟酌了一下,“好吧,应该是不全是。”他笑了起来,“这吴处厚还真是妙人。”

“‘臣尝读史记,考赵氏废兴本末,当屠岸贾之难,程婴、公孙杵臼尽死以全赵孤.宋有天下,二人忠义未见褒表,宜访其墓域,建为其祠。’”李诫笑着,“这样当真能保佑皇嗣?”

方兴和李诫都上京来了,虽然种痘法在京城中掀起的轩然大波掩盖了襄汉漕运的成就,但他们的功绩是实打实的。另外李德新也被急调入京,向天子、太皇太后、皇太后,皇后以及贤妃验证种痘免疫法的效果,现在并不在驿馆中。

韩冈收起笑容,一声轻叹:“天子是想将整件事给打住,不想再听人闹腾了。”

此前逼得天子将弹劾自己的御史黄廉、何正臣贬斥出外,韩冈就成了御史台的眼中钉。这些监察百官的乌台言臣,哪个是忍气吞声的主儿?宰相开罪他们,都会被恶狠狠的咬上一口,何论韩冈,同仇敌忾的继续上书弹劾。反正紧咬着韩冈肯定能得个铁骨铮铮的评价,就算出外过两年就能回京来,他们可不会怕事。

不过赵顼做了多年的皇帝,也知道如何应对这些有恃无恐、喜欢博取直名的御史。他突然之间将仅是区区一名选人的吴处厚的奏章批复下来,要为程婴和公孙杵臼立庙祭祀。有一半就是想表明自己的态度,让御史台偃旗息鼓。这样的暗示,比起明面上的训斥,更能让御史们听话。

而另一半,则是当真想给皇嗣多加一分保险。舔犊之心人皆有之,能保着唯一的儿子,就算只多百分之一的可能,赵顼也不会放过。也就是花点钱买个心安,说不定真是因为保护赵氏孤儿的两名忠义之士不得血食供奉,所以赵家的皇嗣始终保不住。六十多年了,没有一名在皇宫中出生的皇子长大成人,的确是给人一种受到诅咒的感觉。

“谏议,该进宫了吧?”方兴看看外面的天色,提醒韩冈。

韩冈皱眉道:“不说不要这么称呼吗?”

李诫依言换了称呼,“龙图,差不多到进宫时候了。”

韩冈本是正六品的右司郎中,因为有学士衔,再上一阶不是五品的卿监,而是一下跳到从四品的右谏议大夫。以韩冈的年纪,未免太开玩笑了。谏议大夫是能担任执政的最低一级官阶。但凡臣僚,升任执政时,如果本官官阶不到谏议大夫,都会直升此阶,吕惠卿当年便是如此。可有功不能不赏,爵位要靠军功;散官阶则不足以褒奖;已是龙图阁学士,不可能让韩冈再往殿学士上去,也只能晋升他的本官——右谏议大夫。

谏议大夫是从四品,正常官员想靠磨勘,至少得要穷数十年之功方能晋升上来,所以绝大部分宰辅,第一次进中枢,都是跳级上来的。如韩冈这般,依靠世所难匹的功劳,将磨勘二字甩在身后,十年之内升到从四品,如今算是独一份。

不过韩冈还在等着他下一份的差遣,京西转运使的差事很快就该卸下了,就不知道下一步会在哪里。而今天入宫要讨论的事情,也许关系到他接下来的差事。

进了宫中,抵达崇政殿,却发现东府的三名宰执,王珪、吕惠卿和元绛都在。

“韩卿,你来得正好。”赵顼脸上温文笑意,完全看不出他心中对韩冈的芥蒂,“推行免役法的差事,朕与三位相公商量了,准备交给太医局,想听听你的意见。”

“太医局?”韩冈摇头,那群给圈养起来的御医不杀人就万幸了,哪里还能指望他们主持救灾防疫的工作。何况他们的职司和这个并不搭界,若真有此意,知制诰们肯定会兴高采烈地封驳回来,打他韩冈的脸同时,也向皇帝证明自己不是干吃饭的。

“救灾防疫非关医事,正如草台厮扑与战阵厮杀之别。太医局的医官一次救一人,而防疫则救万人。如是归入太医局,当灾疫一起,一介医官如何能驱使灾民迁移,如何能制止官吏主持的赈济工作,这些都不是区区医官该操心的。”

“以卿之意当归入何处?”

“以臣愚见,在朝,当新设一司,归于中书。在路,应由常平提举司监察。在州县,自有亲民官监理。”

“新设一司?”赵顼沉吟了一下,“也不无道理,不知韩卿打算起什么名字?”

“卫生司。守卫众生之司。”

韩冈前面在京西转运司设立卫生防疫局,名声都出去了,自是当顺理成章的推广开来。

“卫生?”赵顼摇了摇头,“不合古意。”

‘不合古意’?韩冈脑中灵光一闪,那个传言该不会是真的吧?他在京城的这几天,听说赵顼正准备改易官制,重行三省六部制,本来以为是谣言,现在看来说不定是真的。

吕惠卿插话道:“《尚书·大禹谟》中有‘正德、利用、厚生’之语。所谓‘德惟善政,政在养民’。陛下此举为千古善政,养民亿万。名为厚生,理所当然。”

“厚生?”赵顼念了两遍,觉得还不错,“就叫厚生司好了,依韩卿的意见,安排在中书辖下。至于主官……”赵顼看着韩冈,“不知韩卿有何推荐?”

中书的人事,怎么轮得到自己来插话?瞥了一眼三位宰执,看着神色个个面露微笑,但心中怎么想,就不知道了。

韩冈躬身推却,“臣任官多在外,对朝堂贤才一概不知,不敢妄言。厚生之事事关重大,想必陛下和三位相公、参政,能有更合适的人选。”

