\section{第17章 家事可断百事轻(下)}

【逛了一个晚上回来就码字,到现在赶出新一章。继续求红票,收藏。】

凤翔府旧名岐州,两个名字皆出自于凤鸣岐山这个典故,从周文王在岐山中听见凤凰清鸣,到此时已有三千年。而凤翔府历经变迁,却始终是关西重镇,在安史之乱中,凤翔还被定为大唐西京,唐肃宗也曾驻跸于此。

而凤翔府也不愧是凤凰来翔之地,城中处处可见一株株高大的梧桐树。凤非梧桐不栖,非醴泉不饮,至高至洁。凤翔人就是取了此意,才在城中遍植梧桐。如今正是盛夏,郁郁葱葱的梧桐树,如一具具伞盖,为行人遮挡着炽烈的阳光,让城中处处得见荫凉。

不过韩冈不是来凤翔府旅游寻古的,他前日在张守约处听说了李信也出了事,便向王韶告了假,连夜赶来凤翔府的府治天兴县。

前几天,听说舅父被打的事情时,韩冈并没有放在心上,完全交给了李信去处理。本以为以李信的能力,能把这件事处理得妥妥贴贴。谁想到他会凤翔府后,竟然把事情闹得大了——虽然这也没什么,韩冈一向喜欢把事情往大里闹,但这么做的前提是必须保证自己的绝对安全,可不是把自家送进大狱。

对于自家表兄,韩冈很是看重,以李信的才能,如果机缘到了的话,日后必然能在军中大放光彩,能成为自己的得力臂助。韩冈不可能坐视他在狱中受苦。

从跟着李信去凤翔的军汉嘴里,韩冈了解了事情的大概。他的四姨已经在去年年初的时候病逝,而他的四姨父早就是因为风疾瘫痪在床多年,上个月也过世了。只是知道了这两点,下面的情节韩冈不用听人说,自己就能推断得出来。

而那名军汉也证明了韩冈的推断,自韩冈的冯家姨夫瘫痪之后,几个原配所生的儿子便控制了冯家内外,等到四姨病死,韩冈的表弟冯从义便立刻被赶出了家门。而且他们做得最绝的就是买通了冯家的族人,将四姨的名字从族谱上划去,也就不再是明媒正娶的正妻,而成了妾室。

对于此事,韩冈的舅舅本是不知,他四姨自出嫁后就跟家里联系很少,到了他外公过世后更是断了联络——说起来,韩冈四姨自己也是有问题,结了亲后,怎么能不与娘家多走动。弄得连死信都没有娘家人听说。若不是韩冈到舅舅听到自己的四妹夫过世的消息,在没接到丧贴的情况下,主动上门去拜祭,还不会知道此事。

从这件事上看,韩冈的舅舅会跟冯家起冲突就不足为奇了。而且冯家在理亏的情况下,竟然敢将自家舅舅打伤,这肆无忌惮的胆子,还当真不小。而李信回到家中,看到老子身上裹着伤,就上门去冯家讨个说法,最后言语不合,李信把冯家的人一顿好打,韩冈的三个便宜表哥都挨了几下。打完人后,李信直接去县衙自首,后来就被押进了狱中

韩冈从来都是他欺人,却忍不下被人欺。冯家将事情做得这么绝,他当然没有一笑了之的好脾气。区区一个豪强,就算有什么奢遮靠山,他也是半点不惧。若是不能让冯家受到应得的惩罚,就枉费了他将陈举灭门的时候,被人扣上的破家绝嗣的诨号。

坐在长兴县衙前的茶馆中,韩冈从袖口里掏出一张名帖来,交给李小六。

“小六,你去将这份拜帖送进县衙里,交给一位慕容主簿,就说同门末学韩冈,正在衙门外的茶馆中静候。”

李小六不多问,接了拜帖就出去了。为了不引起他人注意,韩冈并没有穿着官袍,只套着见普通的士人襕衫。茶馆主人虽然对韩冈这个陌生脸孔很有兴趣,看着他骑过来的马匹也是难得的神骏,但并不知道韩冈到身份,也只是多看了几眼,让小二将他点的清热凉汤送上去,并没有赶着上来谄媚。

韩冈则是隔着窗棂望着县衙,看着李小六跟守门的衙役说了几句,就等在衙门外的影子下,等着里面传出话来。

天兴主簿慕容武,是韩冈在张载门下的师兄,只是韩冈投师时,他就已经考中明经了。不过当两年前,张载受邀在武功县绿野亭讲学的时候,慕容武正好来探望过一次,跟师弟们也混了个脸熟。

虽然此后并没有联系,但自从韩冈在去京城的时候,遇到了种师道,便着意要跟张载门下的其他弟子取得联系。只要人在关西,不论在哪路为官,韩冈现在都了解得很清楚。这么好的资源不利用,那实在是天大的浪费。

今次韩冈来凤翔的第一目的是救李信出狱,在与舅父见面前,他便先打算见一见慕容鹉,将事情的来龙去脉再问个清楚,最好能将李信保出来,一起回去见舅父。

韩冈在茶馆中独坐,慢慢品着饮子,不过这家店里所卖的清热饮子的味道,与严素心比起来差了不少。只是韩冈不喜浪费,口中又干,便是坚持一口口的喝完。

刚刚把小二唤来,给自己续了一杯,韩冈便远远的看见一名身穿青袍、留着一把长须的官员,在李小六的引路下,急匆匆往茶馆这里走来。

韩冈放下茶盏,在茶馆主人和小二两对警惕白食客的眼神注视下,走到门前。

“可是玉昆贤弟!”慕容武远远的叫着韩冈的字。

韩冈则是深深一揖:“韩冈见过思文兄。”

慕容武两步抢上前来,先回了一礼,直起腰后把定韩冈的手臂,笑容满面:“这年来,玉昆已是名震关西,连愚兄身在凤翔也是如雷贯耳。前些日子游景叔【游师雄】、薛景庸【薛昌朝】写信来,一齐提起了玉昆。都说如今先生门下,又多了一位少年贤才。”

“诸位兄长谬赞了,韩冈愧不敢当。”

韩冈与慕容武谦让着,一起走进茶馆中。本来还担心着韩冈是来吃白食的店主和小二,现在都换上了一幅笑模样,

两人又谦让了一番后,方一齐坐下。等店家奉上最上等的茶汤,慕容武便问道:“玉昆此来凤翔,是不是为了令舅和令表兄之事?”

对于慕容武类似于未卜先知一般的先见,韩冈毫不奇怪,自家舅父和表哥在吃亏的时候,不可能不把自己拉出来做大旗。不过他还是装出一副惊讶的样子,尽量捧得慕容武高兴一点,“思文兄果然才智过人,小弟还没说竟然已经猜到了!”

慕容武果不其然,一下变得得意起来,笑着道:“令舅和令表兄都提到过玉昆你的身份,愚兄在这府城中还算是耳聪目明,此事很快传入愚兄耳中。听说了他们与玉昆你的关系,愚兄便跟管狱的孔目提过了,让他多看顾令表兄一点。”

韩冈连忙避席,对着慕容鹉拱手道谢。

慕容武则把韩冈拉回来,佯怒道:“玉昆你这说哪里的话,既然是份属同门,就没有坐看的道理。你再如此,愚兄可是要回去了。”

韩冈也不当真,又好生谢了几句,才又坐下说话。

韩冈对慕容武道:“今次小弟来凤翔,的确是听说了家表兄锒铛入狱,而匆忙赶来。家舅年事已高,却受辱于晚辈。家表兄一言不合,挥拳伤人,也是出于一片纯孝。现在家舅卧病在床,日日思子而不得,不知思文兄能否让小弟将家表兄保出来,以慰家舅念儿之心。”

韩冈睁眼说着瞎话,慕容武则是一副唏嘘作态,为李信父子的不幸叹了几声,又道:“其实这倒不是问题。说实话,令舅在凤翔军中名气不小,玉昆你的外祖父亦是甚有声名,而令表兄又是在秦凤钤辖帐下行走,再加上玉昆你的名气,不看僧面看佛面,虽然府中的刘节推说是要打,府里的衙役都没敢下重手……”

“请稍等,思文兄。”韩冈连忙把慕容武叫停,吃惊地问道:“这事怎么已经闹到府里去了,不是该由县中处置?”

“冯家在县中闹过一次,由于令舅和令表兄皆不属长兴县管辖,县中不好处置,何知县就推到府中去了。不过玉昆你也不用担心,虽然令表兄的确出手伤人,但冯家的人都没有重伤,而且又是为父出头,谁也不会为难他。待会儿玉昆你和愚兄一起去府里,在陈通判、刘节推面前说上几好句,自然也就放人了。”

听到这话,韩冈便又是连声道谢。

慕容武则掀开杯盖,慢慢喝了一口茶汤,问道:“既然那两位真的是玉昆你的舅父和表兄,那冯德坤……”

韩冈随即接口:“是小弟四姨之夫。”

“玉昆,”慕容武神色郑重起来,放下茶盏,向韩冈说着,“据愚兄所知,令四姨初至冯家时,只说是妾室,虽然后来被扶正,但因为冯德坤风瘫之后,她不许原配所生的冯家三子拜见亲父,又被冯德坤找族中耆长为证,将其休了去,只是令四姨当夜就暴病而亡,所以丧葬时,还是按照妾室之礼。至于令四姨所生冯从义,因其母之事,与三位兄长不合,故而与去年便离家,至今未归。所以令舅和令表兄打上门来,冯家的人也觉得冤枉。”

‘这算什么?!’韩冈愣住了,怎么两边说得完全不一样,这算是罗生门吗?

