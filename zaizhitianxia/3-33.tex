\section{第12章 共道佳节早(七)}

一般来说,新党中人若被人指称政治立场会与王安石不同,基本上总是会设法掩饰或是辩解。

但韩冈没做丝毫辩解,而是就着王旖的话来回复,完全不介意当着王家女儿的面,承认日后跟王安石分道扬镳的可能。

王旖了不以为异。在她的了解中,韩冈就是这样的人。

前一次来京城,他更是当着王安石的面,说横山必败,若是一定要他去,纵有功劳也不要算他的。而最后,韩冈也的确是言出即行,所有的功劳全都推掉了。带回了罗兀守军,平息了广锐叛乱,这样殊勋,即便是做到宰相,都能作为功劳。事后,她父亲回来后还在叹着,这等言出不移的男子,世间已经很少见了。

能有如此品性的夫婿,当然是足以让人自豪。但父亲旧年的多少好友,如今都跟他成为死敌。难道这些人中,就没有一个慎严自守的君子?以韩冈这等性格,若是真的嫁了他,一旦跟父兄对立起来,她又能如何自处?而更重要的,是那时父母又将多么自责和伤心?

只是这么一想,王旖动摇的心思就又坚定起来。向着韩冈再福了一福:“小女子不知君子之风,还请公子恕罪。但……”

韩冈立刻毫无风度的打断了她的话:“我当年给王相公出谋划策,乃是相信王相公必能扫除百年积弊,外御敌辱,内安万民。只要王相公今后能一心为国为民,韩冈又如何会背其而去。难道小娘子对相公没有信心?……还是说对韩冈没有信心?”

王旖的言辞一滞,真要比口才,深居闺门之中的她,怎么可能是韩冈的对手。一时进退两难,点头不是,摇头也不是。

不成气候的对手张口结舌,韩冈便是微微一笑。

原本他考虑到底娶不娶王家的女儿,完全是用着功利的心思在思考。不像对云娘,是混和了亲情和怜爱;不像对素心,因为就在身边,而渐渐变得亲近起来;更不像是周南,因为她坚贞不移的一片痴心真情,而让韩冈也被感动,进而喜欢上了他。

但与进入韩家家门被韩冈收入房中之前,就已经与他熟悉起来的三女不同,在正妻不可能于成亲前相见的情况下,韩冈也只能用功利的想法去判断婚姻是否对自己有利,而不是去考虑结婚对象本身如何?

韩冈因为政治因素而犹豫不决,更因不想自己的名声有上一点污损,而不愿意在考试前被王家牵连上。但现在不一样,王家的二女儿站到了自己的面前,是个活生生的有自己想法的人,而不是过去存在于韩冈心中的宰相之女这种单纯的一个名号。

王旖不顾有损名节,来解除与自己的婚姻约定,不是考虑自己,更多的是想着父母。韩冈阅人甚多,知道她并没有在说谎。否则前面自己做了一番澄清之后,就不会太过坚持。

这样的性格,韩冈很欣赏。有胆识,但还有着单纯的心思。如果娶了这样的,应该不会闹得家中。虽然以韩冈的性子,不可能一眼就喜欢上她,但看得顺眼,有着几分好感,在这个时代,就已经很难得了。

就算帏帽之下长相不堪,也没什么关系。娶妻在德不在色,有家中的三位绝色佳丽,韩冈也已经觉得足够了。既然感觉合得来,还是早点抓住好了,谁知道拖下去还会更好的选择?

“关于这么亲事,韩冈并无拒绝的想法。只是不想被人说成是趋炎附势之徒,才会一拖至今。但现在看看,韩冈的确是太过自私,耽误了小娘子的青春韶华。即是如此,韩冈责无旁贷,明日便请人上门给个明确的回复好了。”

王旖终于从张口结舌的状况下脱离出来,娇柔的声音在震惊中一下提高:“公子!这怎么可以……啊!”

韩冈没等她说完,就抬手一下掀开了帏帽。可爱的惊叫声中,王家次女的真容就展现在他眼前。虽然不能算惊艳,跟家中的周南、素心、云娘比起来,都有着差距。但这位出自江南水乡的女儿家,眉如远山,眼如秋水,轮廓中有一份来自于江南山水间特有的纤秀。

而因为韩冈的无礼之举,双目圆瞪而惊呆了的可爱模样,让韩冈看着很心动。他是花丛老手,毫不客气的一把搂着纤细柔软的腰肢,动作很快在张开的红润小嘴上亲了一下。

放开手,为王旖重新带好帏帽,凑近了在她耳边轻声却坚决的说道:“韩冈虽不如相公能从一而终,但也是会真心待人,既然承诺下来,就绝不会相负!”

王旖痴痴呆呆的站着,被亲到双唇热得发烫,方才被结实的双臂强硬的搂在怀里的感觉,还有那股远远不同于己的气息,让她整个人都陷在极度的混乱之中,也不知道该怎么做出反应。

‘他怎么能这么做!?他怎么敢这么做!?’

韩冈的行为完全出乎她的意料,难道把她当成了外面卖笑的欢场女子,就像他收入房中的那个花魁?但要生气,可最后韩冈说的那两句,却又是真情流露,让王旖难以腾起怒气。

就这么愣愣看着韩冈转身推门而出。

这边的事情算是解决了,搂也搂过了,亲也亲过了,以这位大家闺秀的性格,以现在世间的风气,当也不可能再坚拒。对于一生只见过亲戚中的男性的名门闺秀来说,遇上一个还算不错又为家人认同的男子,她们本来就没有什么抵抗力的。

虽然这么做的确浮浪了一点,也显得心机过重,但这是韩冈短时间内,所能动用的最好的办法。不然王旖一力坚辞,王安石那边恐怕也不得不拒绝了。只看王旖能说服她的两个兄长同来见自己,就知道她在家中,对她自己的事务有着一定发言权。

韩冈推门而出,便立刻看到王雱两兄弟跟隔壁的贡生们面对面的站着。难怪王家小娘子的惊叫时,他们没有反应,害韩冈还担了一份心。

王雱在京城也算是个名人了,认识他的人不少。隔壁的一个在国子监读书的贡生喝多了酒,准备出门放松一下时,正好一眼看到他站在走廊上,接着便是一声惊叫。

惊叫之后,贡生们一个个都出来了。一开始还笑着,但是王雱的身份在他们之中传开,顿时人人都变得脸色灰白。当着宰相之子的面,议论起新法来,基本上可算是最糟的局面上。有好几个人回想起自己方才的一番醉话,双腿都直发颤。但也不得不壮着胆子上来跟王雱兄弟见礼。

这时韩冈正好推门出来。

听到动静,王雱转头过来:“玉昆?说完了?”

“玉昆!”王雱的称呼又是惹来一声惊叫。

看着王雱带着笑的眼神,韩冈摇头叹气,王家的大衙内这是故意在拉他下水。

“在下韩冈。”

原本因为遇上了王雱,七八个贡生都已经变了脸色。现在韩冈一下出现,方才说着澄清天下的两三人,完完全全的都傻了眼。谁能想得到韩冈和王雱竟然坐在了一起。

看不看得起韩冈的出身是一回事,畏不畏惧他这个天下最年轻的朝官那是另一回事。就算韩冈这次考不上进士,不代表以后也考不上;也不代表天子不会看在他的功劳之上,赐他一个进士出身。弱冠之龄的朝官,谁能说的准他日后能走到哪一步?万一这个仇被记下,说不定就是结上一辈子的怨。

韩冈走到众人面前,环目一扫,问道:“哪位是讳中的余兄?”

众人先是面面相觑,然后一个二十五六的士子站了出来,旁边一人相貌与他有几分相似,但又比他大了一点,应该是他的兄长。

他向着韩冈一揖到地:“学生余中,不知韩博士有何指教?”

不为方才的言辞道歉,多半还是存有侥幸之心,韩冈就对着余中拱手一揖:“指教不敢当,同为贡生,余兄就不必如此多礼。”后又微笑道,“方才多承余兄回护之言,韩冈铭感五内,必深记在心。”

韩冈这就是坏心眼了。当着贡生们的面,不但坦陈了自己已经听到了他们的对话,还直接把余中从众人中摘出去。这下子,余中反倒要成了众矢之的。

——韩冈的话分明表示今天的事他已经记下了,众人中就只有余中可以除外。见到余中被韩冈放过,而自己就要提心吊胆,能忍住嫉妒之心的肯定有,但也肯定不会在眼下的这群人中。会反过来恨起韩冈的当然也会有,可是余中毕竟离得近。

王旁很是奇怪的望着韩冈,感觉韩冈怎么跟前面说的不一样,刚刚说过不在意,怎么又记恨起了这些贡生?但王雱却是微微冷笑,他算是看明白了,韩冈这一下子,余中肯定要跟这群人生分了。

而余中本人倒是没有反应过来,还以为韩冈是好意,喜色上面,连声说着不敢当。

只这一下子,韩冈、王雱都把他的为人给看透了。

余中的名字,韩冈依稀听说过。方才因为王雱、王旖兄妹的来访,被打了岔,他一时没想起来。但现在却想起了王厚曾经跟他提过这个名字。五千多上京的贡生中,有的有名,有的无名,但基本上每一科能挤进前十名的进士,在考前都已经表现出足够的才华,声名大噪。

比如韩冈面前的这位来自常州的贡生余中,就是士子中甚为出名的一个。所以方才这些贡生才会说他不用考虑主考官是谁,有去争夺状元的资格。

韩冈自上京后,完全没有去打听今科贡生中有那些名望甚高的士子,不管怎么看,他韩冈都肯定是最有名的一个。当然,也是所有知名的贡生中,最不被看好的一个——要是真有才学,早就去考进士了,何必跟着王韶出生入死?基本上,士林间的舆论都是这么看韩冈的。

余中虽然才高,但看起来不是个刚直的性格,见着韩冈表示亲近和感谢,就把同伴都忘在了脑后,这个人品还真是让人摇头。

撞上了不该撞上的人,说了几句场面话,贡生们都是急匆匆地离开了,而余中也没有拖得太久,也跟着兄长一起告辞了。但王雱和韩冈都记下了他,虽然他人品可能不怎么样,可有才就够了。

那边的事了,剩下的就是这边的了。

王旁心急,问着韩冈,“玉昆,怎么样了?”

韩冈深深一揖:“方才已经跟令妹谈过了,明日韩冈会托人上门提亲,还望两位兄长能在相公面前美言几句。”

听着韩冈的话,王雱两人都愣住了,怎么跟韩冈单独说了两句话,就让他完全变了个说法。视线越过韩冈,透进房中。仍站在屋内的俏丽身影,依然带着帏帽、裹着披风,王雱兄弟一时间,都觉得自己的妹妹变得高深莫测了起来。

