\section{第15章 焰上云霄思逐寇(七)}

接近午时的时候,从南面传来了咚咚的战鼓声。

低沉猛烈的鼓点,告知在望楼上的韩冈等人,李常杰他并没有对峙等待的打算,现在就要开始进攻。

伴随的鼓声,交趾军举着李常杰的大纛缓缓走入了战场,走进了韩冈等人的视线中,一步步的接近了归仁铺大营。

在离着归仁铺只有一里的地方,交趾军连同鼓声一起停了下来,但这个停顿十分短促。调整了一下阵型,鼓声重新响起。交趾中军大纛纹丝不动,可只有四分之一的队伍留了下来,围着李常杰的将旗作为预备队,而将大多数的兵力都入了战场,向着归仁铺大营碾压而来。

“看起来李常杰是知道了。”韩冈抿了抿嘴,苦笑着。虽说有了心理准备,但侥幸之心也是免不了的。当看到自己当真言出成谶,总是有些不舒坦。

这几天,苏子元除了在公务上说话,就极少开口,看见李常杰的主帅大纛,仇恨的视线从眼底迸出——话说回来,这两天任谁也都没聊天的想法——不过看见韩冈,他提醒式的发问:“交趾贼军会不会分兵绕过归仁铺大营大营,直攻昆仑关?”

“若是李常杰当真分兵,我高兴还来不及。分出三千,到了晚上就好夜袭了。分出五千,我敢出寨与之对阵。”韩冈当真盼望李常杰犯糊涂,“昆仑关也不是没有人驻守,就算只是广源军,也能守得住十天半个月,交趾人的残忍是有名的,破了城后会做什么。要想正面攻破关城,交趾人还没那个能耐。而且就算占了昆仑关,对现在的交趾人根本没有多大的意义。”

可惜李常杰用得是无懈可击的正攻法,主力直取归仁铺大营,逼着宋军硬碰硬。只有两支各四五百人的偏师,也就是两个指挥的兵力,开始走起了弧线,看起来是要绕道归仁铺大营的后方,打算用来封锁归仁铺和昆仑关之间的交通。

韩冈低头看着下方的营地,无论是官军,还是广源军,在面对交趾军进攻的时候,都没有胆怯和畏缩的迹象。军心还算稳定。

“运使,是守寨还是出战?”李信问着韩冈的意见。

韩冈反问回去:“你的意思呢?”

“先守寨。”李信道,“贼军弓弩少,又不善攻坚,有着营垒护翼,正好可以射个痛快!”

“黄洞主,你觉得该如何?”韩冈转头问着黄金满。

“……守寨。”黄金满犹豫了一下,回答着韩冈的问题,“已经过了午时,夜间不便进攻,再过两个时辰李常杰就要撤回去了。”

“伯绪你说呢?”韩冈再问苏子元。

“守寨。”苏子元抬头看天,“今天北面有厚云层积,湿气比前两日都重,傍晚可能会下雨。”

三人理由各自不一,答案则如出一辙,且正是韩冈所想,也正是之前商议的计划,“好!那今天就坚守营寨。帮交趾贼军好好回忆一下,他们在邕州城下顿兵两月的经历!”

“李信,黄金满,你二人下去统领本部,依既定方略行事。”

“末将遵命。”“小人遵命。”

不论是韩冈,还是李信、苏子元,都是只依靠八百官军作为核心战力,没把黄金满的数千蛮兵看得太重。不过用他们来做单纯的防守,或是胜负已定时的追击,还是能派些用场。

战鼓就在中军大帐前擂响。以不逊于交趾军的声势,让营中的三千将士听着号令前往自己应在的位置,等待着敌人的到来。

李常杰眯起眼晴,遥遥眺望着给了他太多惊讶的对手。从他离营出战,到现在逼近大营,留给他们的这么长时间中,面对数倍大军的攻势,宋人选择了坚守而不是撤离。这个选择怎么都有些让人纳闷,外无必救之军,内无必守之城。如果没有援军,固守远远比不上城寨的营垒,完全是件最愚蠢的行为。宋人会有这般愚蠢吗?还是说来援的宋军不止那八百人?

李常杰不知韩冈是别无选择,只能让自己来吸引交趾军目光的磁石。而同样的,李常杰也是没有别的选择,才会只率领一万多人前来攻打归仁铺。

可以这么说,双方都因为各自的原因而被绑着手脚。

李常杰要提防广源蛮军在背后生事,还要担心眼前的八百宋军不是宋人全部的南下人马。坐拥数万大军,能带出来的就只有一万多兵。

而韩冈为了不让邕州百姓遭受屠戮,面对李常杰的攻势只能选择硬顶,数日之内无法退回昆仑关去。若是他能放得下邕州,只要回到昆仑关,将大宋的战旗往关城上一挂,谅李常杰也没胆子再来攻打一次关城。

于归仁铺处发生的又一场大战,在双方没有多少选择的情况下,终于展开。

人马上万、无边无岸。在战鼓的催促下,近万交趾战士如同夏日雨云扩散,浩浩荡荡的占据了两军之间的战场,一步步的向宋军大营掩杀过去。一排身高体壮的士兵举着巨大的木盾走在最前面,这是防备神臂弓最好的武器,同时也是铺平寨前一道壕沟的工具。随着越来越接近营寨,他们的速度也逐渐加快。

“要射击吗?”苏子元问着。

韩冈摇头:“再等等!到了寨墙外再说。”

因为湿气深重的缘故,不论是广源蛮军还是交趾军,弓弩都不算多。像宋军上阵时人人皆是弓弩手的情况,在南方的战场上并不多见。

近万交趾军只有两三千名弓手在绕着寨墙牵制射击,而剩下的士兵也同样分散开来,围绕着寨墙,怒吼着、狂嗥着,从东南西三面开始同时围攻归仁铺军寨。试图利用人数上的绝对优势,争取用最短的时间,在第一次攻击中就将营寨给攻破,而不是像在邕州城下慢慢的耗尽气力。

“想不到交趾贼军如此勇猛,幸好他们的象军在邕州城下都损失掉了。”韩冈还是从黄金满那里听说的此事,“要不然一群大象冲过来,肯定要手忙脚乱一番。”

“有神臂弓在,就算是大象也一样能射杀。”苏子元急色问着韩冈,“运使,还不射击!?”

“还要再等一下。”韩冈还要等。他的兵力不足,也需要在最短的时间发挥最大的杀伤,让交趾兵为之胆寒。

宋军所设立的营寨外墙并不是一条圆滑的直线或是弧线,而是如同锯齿一般的前后凹凸,突出于外的部分如同城墙的马面,长而密。陕西修筑城寨,或是西军设立需要固守的营垒,外墙都是用着类似的布局。

交趾人不知道这样设立寨墙的用意,他们只为自己没有任何阻碍的冲到在营垒外而感到庆幸。冲在最前面的士兵,将手上的木盾放倒下来,架在壕沟之上。没能来得及掘深掘宽的壕沟紧贴着寨墙,只有一点落脚的地方,如果没有木盾压在壕沟上,根本就站不住脚。

两脚踏着木盾,就在寨墙下方,一群特意被挑选出来的精锐士兵,拿着大斧劈砍起并不结实的栅栏。丁丁斧声,木屑横飞,栅栏不断的摇晃着,看起来转眼就能将眼前最后一道阻碍给拔除。而寨内没有任何反应,像是被营外的交趾弓手们给压制住了一般。只有老于战事的少数人,清楚这样的沉寂有哪里不对,设法给自己寻找一个隐蔽的地方。拥在最前面的交趾士兵则是欣喜欲狂,只要再有片刻时间,他们就能冲入宋军的营寨中。

不过他们的欢乐也就到这里,随着望楼上的旗帜变幻,徐缓的鼓点节奏顿时为之一变,号角声也同时响起。

“射!”

上百名军官同时发出号令。响成一片的弦声中,弓箭、弩箭,交织而下,将自己心中积蓄的怒气注入箭矢之中,向着最近处的敌军攒射过去。

用着壕沟掘出来的泥土,在锯齿状的寨墙突出部的内侧,修起了高出两尺的台地。弓弩手们就站在台地上张弓搭箭,让寨墙下的交趾兵,不论站在何处,左右两边都会受到射击。

而每隔十几息,紧随着一道尖利的木笛声后,寨墙外侧的敌军之中,就是一片密集的箭雨落下,一下就清扫出一片空白。

归仁铺位于官道边,是供行人车马休息的去处,其地势当然是不会选择坡地或是台地。当宋军修建营寨时,就略略偏了一点,将后面的一座约莫一丈来高的矮坡一起括了进来。

特意挑选出来的两百名神臂弓手聚集在矮坡上,居高临下,他们手上的远程弓弩能覆盖东南西三面的敌军。就算交趾兵是用着木盾做阻挡,等他们接近到足够的距离,也不能帮后面跟进的队伍遮挡弓箭。

重弩独有的噌噌射击声,有节奏的响着。

都头被射死了!

指使也死了!

许多交趾兵就在用着土话疯狂大喊着,混乱中他们完全组织不起来攻势。

最贴近寨墙边的广源蛮兵疑惑的听着交趾人的叫喊,黄金满则惊讶的回望着台地上的神臂弓手们,他们难道都是盯着军官在射击?

的确是盯着军官。这两百名神臂弓手的任务不仅仅是帮着寨墙危急的区域解围,而且还包括了狙击隐藏在敌军军阵中的指挥官。将中底层的军官射杀,是摧毁敌军战斗意志的最好手段。

交趾军的第一波攻势,只维持了片刻就宣告失败,寨前的交趾兵纷纷逃散,在军寨前留下了数百死伤。呻吟和惨叫回响在寨墙外,听在寨内守军们的耳中,就是最动听的吟唱。

望楼上的韩冈负手笑道:“官军最擅长的不是攻,而是守。交趾军以短击长,这是在自找苦吃。”

