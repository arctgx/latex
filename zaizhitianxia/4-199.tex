\section{第31章 九重自是进退地(13)}

梅花已谢,桃杏正浓,当春风在洛阳城中舞起来的时候,一张短短的信笺摆在富弼的面前。

御制的粉笺销金纸上,只有寥寥数行草字。观其内容,也不过是设了一个诗酒之会,以耆英为名,邀请富弼与会。

类似的请帖富弼年年都能收到,作为前任宰相,国中有数的元老重臣,地位和身份都让他成为最受欢迎的宾客。但富弼点头答应的时候寥寥无几,很多次都是看过一遍后,就让儿子富绍庭写了婉拒的回帖。

不过这一次,发起之人却是富弼的老朋友,新近来洛阳上任不久的文彦博——‘凡所谓慕于乐天者,以其志趣高逸也,奚必数与地之袭焉。’说是要承袭白居易白乐天当年退居洛阳,设九老会悠游林下的志趣,于今日设耆英会。

“文宽夫当真有雅兴,五老会聚了,同甲会开了,今天终于想起来找为父了。”富弼将文彦博的信望身前的几案上一丢,抬头望着肃立在身前的儿子,考试一般的问着,“你说,他是在想什么?”

富绍庭张开口,吭吭哧哧了半天,却是说不出话来。他的老父既然如此相问,就代表文彦博的举动必有其深意,只是他想不明白,这深意究竟在何处。

过了好一阵,方才没有什么自信的说着:“五老会有范景仁【范镇】、张仲巽【张宗益】、张昌言【张问】、史子熙【史炤】,同甲会有司马伯康【司马旦】、程伯温【程珦】和席君从【席汝言】,皆是反对新法的老臣,在西京广有声望,或许有心合众人之力,打动天子。”

“都被人从东京赶出来了,西京中的声望又算个什么?要打动天子早就打动了。”自家的儿子才仅中人,勉强做个守牗之犬,绝非是龙虎之辈,听到回答的富弼连失望的力气都没有,瞥了眼苦思冥想得脸色涨红的富绍庭:“文宽夫是初来乍到,找些人来壮声威,打算跟为父分庭抗礼来着。”

富绍庭有些吃惊,感觉难以置信。但富弼却是对文彦博的为人了解甚深,并不觉得自己是冤枉了文彦博。

在文彦博来洛阳之前,他富弼绝对是西京老臣中的第一人,但文彦博一来,第一第二就要争个高下了。

富弼冷笑着。他都在洛阳几年了,却没玩过这一出。寻常也有诗会,却从没想过要弄出个名目来。

也就文彦博有意思,到任后就招了几个致仕的老臣来做五老会、同甲会,洛阳有点声望的耆老旧臣一个个都被他邀请,就是把他富彦国给落在外面。直到人都请遍了,方才再携胜势来邀请自己。

“五老会请的范景仁、张仲巽、张昌言、史子熙,皆在洛阳住得久了。前两天的同甲会,又请了司马十二的兄长、二程的老子,那席君从倒算是添头。”富弼一个个数来,“如今要办耆英会,就变成了尚齿不尚官。以齿序论,前面请的那几位,都得以为父为长,人情也送了来,人望也得了来。这一套做得面面俱到滴水不漏,还不愧是文宽夫。”

最后他扬起胡须哈哈大笑,“‘西都旧士女,白首伫瞻公’,天子逐人不遗余力,‘身在洛阳心魏阙,愿倾丹恳上公车’,文宽夫和诗时也都这么说酸话了,你说他还会指望能卷土重来?”

‘西都旧士女,白首伫瞻公’;‘身在洛阳心魏阙,愿倾丹恳上公车。’富绍庭并非孤陋寡闻之辈,这两句分别出自于文彦博去年转调西京河南府,离京辞行时,天子的赠诗和他本人的回赠。

两首诗看着是君臣相得,天子恭维文彦博是‘四纪忠劳著,三朝闻望隆。’,西京之人翘首以待,而文彦博的诗中用‘康时有志才终短,报国无功术已疏’表示自己的的谦虚,又用‘身在洛阳’两句,表达对天子的依依不舍。

可只要往深里一想,就是天子等不及的在赶人,而文彦博则是满心不情愿的吐酸水。于唱和之间,也能看得出文彦博的一颗心还放在朝堂上。

眼下在洛阳城中布宴席,设诗会,白居易的九老会是珠玉在前,但文彦博学来,却有让人有效颦之感。

听出父亲话中全然不掩讥讽之意,把文彦博的一点小心思刨开来晾在太阳底下晒着,富绍庭小心翼翼的问着,“大人是不是想要推掉?”

“推掉?为什么要推掉?”富弼一拍卧榻,反问着儿子,“当然得去!难得春暖花开的好时节,为父也不知能再过上几次了,怎么能放过?不过得请他文宽夫过来,这耆英会的第一回,就在家里的园子里开。这两日正好漪岚亭畔桃杏花开正艳,又有杨柳随风,却是个观花饮酒的好时节。”他拍拍腿脚,“这条腿走不了远路,还是在家里方便。”

富弼说完,抬头再瞅瞅儿子,富绍庭正忙不迭的点头称是。老宰相无奈的叹了一口气,连察言观色都如此迟钝,入了朝堂定然会被人欺,也就是胜在老实,不会欺凌族人,守着家业还成。

心中满腔的遗憾和落寞,富弼他提声道,“还不去唤人拿纸笔来,为父要写回帖。”

……………………

文彦博于去年年底被调来洛阳,判河南府兼西京留守。

从接到这个任命开始,文彦博就知道当今天子是不会再招他回朝任职了。

早在当今天子即位时起,文彦博就反对任何开疆拓土的战争。蛮荒之地得之无用,还要空耗钱粮。败且不论,只要一点微不足道的胜利,都能让王安石稳固他的权位。故而文彦博看任何一位有志开拓的臣子都不顺眼。

但如今官军连战连捷,在南方已经灭掉了交趾,收服了西南夷,在北方也逼得西夏喘不过气来,让辽国都忌惮不已。

士林和朝堂中,宣扬平灭西夏,收复燕云的潜流已经渐次形成主流,甚至如今世间新近流传开来的诗文中,偏向于好武用兵,鼓吹汉唐武功的也越来越多。

如今的情况下,像文彦博这样的反战者,是不可能继续留在北京,执掌大名府,参与河北的一应防务。只要他还在大名府,就是重整河北军力的最大的绊脚石。

调往河东、陕西是不可能的,那同样是个阻碍,而以文彦博的身份,则也不能调往南方,因为那更是贬斥,又会惹起一番波澜,所以西京这个养老地就是最好的选择。

天子的心思,文彦博把握得很好。但要让他去迎合天子的想法,文彦博却是宁死也不干。就是被调任西京,他也绝不打算后悔。宁可找些他当年担任宰相时,都没拿正眼看的老家伙,再加上几个元老重臣,一起来凑个热闹,写几首诗句,博个诗酒风流的名声,也绝不向王安石、吕惠卿之辈低头。

富弼的回帖到得很快,自称足上旧疾发作,不便随意外出,所以恳请将耆英会第一回的会场设在富家的花园里。

文彦博将富弼的回帖看了两三遍后,终于放了下来,对着儿子文及甫笑道:“只可惜不是七八月,听说富彦国家有独立凌霄花,不附他木而独立成树。如今正值初春,饱不了眼福了。”

文及甫附和着说道:“儿子日前去富家时看过了,天下凌霄花皆是附树而生,只有富家园中的凌霄花,高达数寻,独立成树,实是难得一见。”

文彦博听了之后,眉毛动了一下,要是有个能问一答十的儿子在身边就好了。

大宋以孝治国,通常都是鼓励儿子留在父母身边照顾,也愿意为此提供协助。就如王旁跟着王安石南下江宁一般,文彦博、富弼都留了一两个儿子在身边,去世的韩琦也是一般。但跟在身边的儿子无一例外的都是平庸之辈,自家的儿子更是明证。

“凌霄花是小事。”文彦博已经忘了方才自己的说得话,“富彦国愿意赴会,这是难得的大事。有为父和富彦国,当人人愿意与会。”文彦博又叹了一声:“司马君实其实也是个好人选,就是还不到花甲之龄。想请他也无名目。”

“司马君实的书应该已经校订到了晚唐,想必他很快就能结束。”文及甫没话找话,“听说韩冈要来京西了,想来程伯淳、程正叔必是欣喜欲狂。”

“韩冈!”文彦博不喜欢听到这个名字,但儿子文及甫说得却并没有错。

虽然韩冈是张载的私淑弟子,但在程家,韩冈一样是持弟子礼。逢年过节礼数从来没断过,更别说当年在家门前雪地里站了整整一个时辰,尊师重道之处早已是天下知名。程颢程颐当然乐于看到自己做了转运使的学生来京西任职。

“就等着他过来了。”文彦博温温和和的笑道,他对韩冈有种莫名其妙的敌视。对韩冈任职京西也有所准备,如果有机会,他不会放过。

