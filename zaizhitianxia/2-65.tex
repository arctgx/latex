\section{第15章 前路多坎无须虑(一)}

【想不到年后竟然这么忙,一月份在上午就能将一章码出来,现在拖到快下班的时候了。今天第二更,求红票,收藏。】

郭逵要来的消息半天之内传遍了秦州内外。

对于郭逵的到来,民间的反应很正面。毕竟是声名煊赫的宿将,有他来镇守秦州,会让人安心不少,至少今年秋天,党项人当是闹不出大乱子来了。

韩千六在晚饭时跟儿子说着闲话,也喜滋滋的提起郭逵要来的消息,“三哥,是不是郭太尉要来秦州了?都说他看人极准,料事如神的。有他在,秦州可就安稳了。”

“郭太尉他哥哥郭巡检,三哥他外公当年是亲眼见过的。骑着一匹五尺多高的河西马,手上的两只铁简都有十几斤重。”韩阿李出身武家,军中旧事比韩冈还门清。

“当年李元昊攻打延州,三哥外公随军赶去救援,路上正好看见郭巡检跟着刘太尉也往延州赶。不过刘太尉他们走得太快,连夜路都敢走,最后就在三川口出了事。三哥外公也是运气,他们一千多人已经连夜赶了百十里,最后都没力气走路了。刘太尉就没看上眼,没把他们一起夹裹上,不然也一般儿要折在三川口。”

“郭遵的确可惜。”韩冈喝着汤,很随意的评价着。

郭逵的长兄郭遵,是军中有名的猛将,名副其实的万人敌,只是跟随刘平战死在三川口。据说在最后一战中,郭遵手持铁简在西贼阵中杀了个三进三出,敲碎了数百名党项人的天灵盖,不过寡不敌众,最后坐骑被绊住,遂战死在阵上。

“郭太尉比他哥哥强。郭太尉是做过相公的,郭巡检却只是匹……匹……匹,三哥,匹什么的?”

“匹夫之勇?”

“对,就是匹夫之勇!跟郭太尉没法儿比。”

韩冈父母的心情,代表了大部分民众的想法。而官场中的反应就有点五花八门。等待郭逵来交接的李师中幸灾乐祸,普通官员则是隔岸观火,而王韶、高遵裕则被激得跳脚。

白天的时候,听说了郭逵要来,高遵裕气急败坏:“郭逵真要来了,我们还有站的地方吗,看看他在鄜延怎么挤兑种五的?!”

王韶眉峰紧锁:“就算天子看不到这一层,王相公总该能想明白,怎么能让郭逵来秦州?!”

郭逵可不是李师中、窦舜卿、向宝那等货色,李、窦、向三人加在一起,都比不上他。郭逵是做过枢密院同签书的,货真价实的一任执政,如今大宋百万军中,只有他有这个资历,地位稳坐第一。他要给王韶弄点乱子,那就真的什么事都别想做了。

“郭仲通是雄武军节度留后,秦州的节度军额便是雄武军,说起来,秦州就是他的本镇。天子是不是看到这一点就把他调过来的?”

“玉昆!都这时候了,你还说风凉话?!”王韶气急了,差点都要拍桌子。

韩冈歉然的笑了一下,他没想到王韶现在心里躁得连个冷笑话都不想听了。在他看来调郭逵来秦州绝然不会是天子的失误,也决不会仅仅是为了稳定秦州军中,王安石那边肯定有着更深的考量。

王安石本人的政治头脑不说,他身边的几个助手都是明白人,没有一个差的,怎么可能想不到郭逵来秦州的后果。既然王安石考虑过郭逵在秦州将会造成的变数,还坚持将他调来,就代表在王安石他们眼中,有着比河湟开边更为重要的利益。

“大概是横山那里要有大动作了。”韩冈这回说得很正经。

联想起年初时去京城时,从种建中那里听说的郭逵与种谔之间的紧张关系,还有前次绥德大捷,郭逵启用燕达、弃用种谔的事实。“很明显的,就是某人嫌郭逵在鄜延有些碍眼碍事,想把他踢远点。”

听了韩冈的分析,王韶终于冷静下来,“玉昆你说的某人是韩绛吧?”

高遵裕心中则是依然郁闷不已,“郭逵哪里不能放?调哪里都比调到秦州要好。”

“谁让秦州正好出了事,需要个重臣来镇守。”王韶无奈的叹着,“有空位怎么能不补。”

高遵裕郁闷不已,闲扯了几句,就直接回家休息去了。

等高遵裕一走,王韶便问韩冈道:“玉昆,你有什么主意?”

“下官觉得还是先往好处想,不过机宜你也可以在给王相公的信里多抱怨两句。以王相公的性格,应该会给点补偿的。”有些话在高遵裕面前不好说,私下里说一下就没关系了,就像王韶和王安石的书信往来,其实朝廷有规定是不允许边臣与宰辅私下里联络。

‘这算什么主意?!’王韶总觉得韩冈并没把郭逵的事放在心上。

“能要到什么补偿?!古渭大捷的封赏都不会给足,何谈补偿?”他悻悻然说着。

两次大捷时间离得太近,无论王韶还是韩冈都不可能才隔着两个月的时间,就又给提升个几级。最后得到的封赏,肯定要打个折,多半是用财帛之类的赏赐,或是对父母的封赠,来代替官职的晋升。

但会哭的孩子有奶吃。你不多叫唤两下,谁知道你的尾巴被踩到了?

韩冈依然坚持己见,“下官觉得还是多给王相公写两封信,等回去后,下官也会给章子厚去信。修造渭源堡的钱粮,市易和屯田的本金,还有古渭建军的提案,都提上一提。就算我们这边漫天要价,他们那里落地还钱也行。这亏不能吃得不明不白。”

韩冈很轻松的说着,他现在还是抱着乐观的态度。郭逵是做过执政的宿将,声威赫赫,名震中外,这一点的确是事实。但韩琦、富弼之辈,哪一个不更胜一筹,还不是都离开了京城。如果郭逵真的敢于沮坏河湟开边,天子和王安石会放过他吗?

何况要评价一个人,要察其言,观其行,郭逵还没来秦州,怎么能贸贸然的下结论。抱着对抗的心思去迎接郭逵,也许本来能搞好的关系也会变得糟糕。

…………………………

“郭逵答应去秦州了。”

赵顼放下手上的一本奏章,对王安石说着。郭逵接受了新的任命,将奏章递了上来,同意去秦州,而放弃延州知州一职。

当然,赵顼也不认为郭逵敢拒绝。文官如果有事不想做,可以直接推掉,但武臣就不行,他们唯一能辞的,只有升官封赏,如果是平调职司他们还推辞,那就是跋扈之行。

“王卿,郭逵到秦州后,是不是要叮嘱几句,让他多看顾一下王韶?”

“依臣之见,还是让郭逵守稳秦州便可,河湟的事让王韶独力处理。多说一句,以郭逵的心性,或许就要跟王韶起龃龉了。”

赵顼叹了口气,紧皱的眉头上尽是疲惫:“关西的几位帅臣,也只有蔡挺让人省心。”

“蔡挺在渭州除旧弊,定新规,将关西四路中,军力最弱的一路打造得固若金汤。有他镇守泾原,鄜延路的侧翼就可以放心了。”

蔡挺在渭州推行的将兵法改变了宋军过去大小相制,难以指挥的弊病,很对王安石的胃口。在王安石的计划中,等到朝廷钱粮充足,就可以动手改革军制,将兵法、保甲法和保马法这三项有关军事制度的法令,都已经进入筹备阶段。

“郭逵之才不在蔡挺之下,名望尤高,可就是事多。若不是他跟韩绛不合,也用不着把他调去秦州。”赵顼又在叹着,“只希望他能如王卿你所说,与王韶争胜负,而不是互相拆台。”

王安石知道以郭逵大权独揽的性格,以及身为前任执政和节度留后的地位,他去了秦州,很有可能就要跟王韶为河湟开边的领导权起冲突。

但秦州军中地位最高的三人一下子全都走了。为了稳定秦州军心,除了郭逵,一时之间他和赵顼都找不到更好的人选了,即便是泾原路经略安抚使蔡挺也不够资格,而他们一开始准备在半年后用来替代窦舜卿这个过渡人物的韩缜更是远远不够。

另外还有一点,就是郭逵与李师中、窦舜卿他们不同,他是全力主张开边之策,就算他和王韶相争,也不至于会耽搁正事——以上都是王安石说给赵顼听的理由。

而实际上,王安石不过是两害相权取其轻罢了。虽然秦州连传捷报,但河湟作为偏师的地位并没有被改变,横山的战略地位远远高于河湟。

郭逵当初任鄜延路经略安抚使,与种谔争位,几乎将种谔挤兑得无法在鄜延路立足。如今韩绛任陕西宣抚使,重用种谔为主帅,因而让郭逵大为不满。为了不让郭逵干扰到现在由陕西宣抚使韩绛主持的战略规划,必须将其调走。却又不能将他调离关西,郭逵本身的资历、能力和威望在军中犹如定海神针,万一韩绛那里有个万一,有他在,至少还能稳定住关中的局势。

而王安石为郭逵选择的地方,就是正好需要重臣去镇守的秦州。不过为了让王韶能安心做事,不至于给郭逵压得太惨,章惇帮着出了一招。

王安石对赵顼道:“陛下。古渭大捷之功,已得王中正查验,皆为实情,并无虚妄。由此可见王韶之才非区区机宜可屈。数月前,王韶曾上书奏请于升古渭为军,以便统一兵权、事权,更为名正言顺的招揽蕃人投效朝廷……”

前次张守约入觐,也是有过同样的请求,但赵顼仍有些犹豫,“直接在古渭建军,是不是有些仓促了。”

“那就先围着古渭寨划出一块地来,设立秦凤缘边安抚使司,由王韶担任安抚使,先给他一个署理秦州西陲军政的名义。等到一年半载之后,稍见事功,再将古渭升为军不迟。”

