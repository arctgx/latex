\section{第30章 众论何曾一(四)}

【这两天工作上有些事,回来后只赶出了这一章。下一更要到明天中午,还请各位书友见谅。】

接下来的两天,韩冈以游玩的名义带着王旁出城。不过如今乃是数九寒冬,而且还是大旱之下的冬天,连冬日最值得欣赏的雪景也没有。所谓的游玩,自然而然的也就变成了探视民情。

王旁随着韩冈去了城外的流民营,还看到了指挥流民开凿深井的井十六。又去了黄河边,见识过了冬日的黄河,以及护卫河边的千里长堤。

浅浅的只剩河床中心一段的黄河,让王旁对如今旱情有着最直观的认识。而黄河滩涂上,数之不尽的蝗虫卵更是让他感到心悸。反倒是再次回到流民营,营中的流民们各个看着气色都不算很差,并不似他在脑中描绘出来的骨瘦如柴的流民形象。

流民们知道他们现在的安定究竟是谁的功劳,在道边对着韩冈恭敬行礼。

视线从跪拜下来的流民们身上扫过,王旁扭头对韩冈笑道:“玉昆你的功劳不小啊!”

“拯危济困,义之所在,也是小弟的分内之事。”韩冈正色道:“如果救治不当,可都是我这个亲民官的责任。一县不治,县官有责。一州不治,州官有责。一国不治,那可就是岳父的责任了。”

王旁听了脸色微变,“玉昆,这是天灾啊!你该不会也要说什么天人感应吧!?”

“天变不足畏。我也是从来不信这一套。但灾后的应对却是政府推脱不了的责任。”韩冈抬手推了推刚刚夯筑起来的简易窝棚,的确还算结实,赞了负责夯筑的流民两句。回头继续对王旁道:“狗彘食人食而不知检,涂有饿莩而不知发。人死,则曰:‘非我也,岁也。’是何异於刺人而杀之,曰:‘非我也,兵也。’”

丰收之年,浪费口粮圈养牲畜而不知囤积,大灾之时,路有饿殍而不去发仓救治。等人死后,却说:“不是我的责任,是年景不好。”这何异于以刀剑杀人后,推卸责任道:“人不是我杀的,是刀剑杀的。”

孟轲见梁惠王时说得这番话,王旁自然不会不记得。

以孟轲的观点,救治百姓本来就是官府的责任,救治不了便是官吏的过错,责任无可推卸。怪罪到年景上,就跟杀人者怪罪凶器一般,这当然是大错特错,无论去哪里都说不过去。作为思孟学派的传承,不论是关学还是王学,都是有着同样的看法。

他点着头道:“不意玉昆你对先贤之言,已是在身体力行了。”

“小弟可当不起仲元兄的赞。”韩冈半开玩笑的说着,“真的遇到灾情的时候,该推卸责任还是会推卸的,就算是小弟也不会愿意将天灾造成的损失全都架在自己身上。”

“玉昆说笑了。”韩冈为了安顿好流民,救治灾伤,究竟付出了多少心血,王旁这两天都看在了眼里。要是韩冈是随意推卸责任的人,根本不需要做这么多。其中有许多其实应当由开封府来主持,而不是韩冈这位知县。

“拯危济困,视民如伤,眼前的百姓都是得玉昆你之力方得安定下来。实是功德无量啊……”

韩冈摇摇头:“只是小弟不过是安排着一千多流民就已经忙碌如此。等到开春后,河冰化尽,成千上万的流民渡河南来。到时候,光靠一县之力怎么也忙不过来了。”

“开封府……”王旁只说了一句就自己给否定了,这当然不可能。开封知府治理京城内还来不及,哪有多余精力像韩冈一般奔忙。如果只是简简单单的救灾,流民们绝不可能有着现在如自己所看到的这般平稳生活。“不知玉昆你可有什么手段?”

“没有。这要朝堂上下一心,可不是小弟一个人能解决的。”韩冈望着南面东京城的方向,冷笑着,现在朝中君臣怕是还没有将注意力放在救灾上呢。

……………………

为了到底如何处置这群与自己有亲戚关系的奸商,赵顼这几天几乎都快忘了如今还在延续的旱灾。

三十七名深陷诏狱的奸商,个个罪无可恕。视如今的灾情为赚钱的时机,动摇国本以逞私欲。大宋是他赵顼的,赵顼当然不可能坐视这等。王安石的霹雳手段,赵顼心中也是觉得痛快不已。

但是人抓起来后,麻烦也随之而来。将三十七人全都杀了当然痛快,但这一干粮商们与自家实在勾连得太紧密,牵一发而动全身。将他们下狱,是以造成民乱为借口,当时无人敢插言。如今京中安定下来,来求情的便越来越多。甚至嗣濮王,也就他的亲伯父都来为其中一名粮商求情,这个面子他怎么也不好不给。

只是放了其中一个,剩下的必然不可能再重责,否则人心难服。但就此放过更不可能,明着下诏肯定会被打回来,宰相、执政都不可签署。而暗中命令开封府和御史台在会审时松一下手,就不知道会有几个士大夫点头。许多时候,士大夫们对自己的原则,比天子的命令更为看重。

一直到文彦博的奏章送到眼前,赵顼才惊醒过来,比起已经抄家下狱的粮商一案,如今的灾情,才更要他加以关注。

判大名府的文彦博,在奏章中说着大名府外已有近十万流民聚集,而北京的常平仓经过了几个月来的散发,已经难以支撑,亟待京中调粮补充。而且文彦博的口气很大,一下就要了六十万石。

前任宰相和枢密使的奏章,直接就能呈到赵顼的案头上。而赵顼也说过,若是有关河北灾情的奏章,不得耽搁,要直接呈递给他。当这份奏章送来的时候,赵顼正在经筵上。王雱和吕惠卿两位侍讲正为天子说着‘官不私亲,法不遗爱’的道理。

两人都是舌灿如花,引经据典的将法家的理论,用儒家的道理来包装,说得赵顼连连点头。只是到了河北急报进来,王雱和吕惠卿便不得不停了口。

赵顼接过奏章看了之后,眉头就紧紧的皱了起来:“黄河上雪橇车可不好走,水路不通啊!”

雪橇车在冻透底的汴河上好走,可黄河冰层下的水流却从来没有停过。赵顼岂会在这等事上冒险?万一运粮的车子陷到河底去,到时后哭都不哭出来。但雪橇车有个好处,就是冬天汴河的纲运自此不会再停运了。

从送进宫中来的一辆样车上,赵顼也明白了这一无轮车的优势在哪里,即便冰雪厚积,雪橇车也能如履平地。不论在民生上,还是在军事上,都是一件难得的利器。可叹要不是今次的大灾,说不定就埋没在关西的崇山峻岭以及政事堂的故纸堆里了。

“吕卿、王卿,要将六十万石粮食尽快运到大名,可有什么办法?”问着,赵顼就将文彦博的奏章中的要求一起告知了王、吕二人。

王雱听了之后,立刻说到:“开封、大名,两京相隔五百里。从京城运粮到大名去,只有陆路可行。可五百里转运,路上损耗不计其数,恐怕也难以救急。依微臣之见,不如将送到黄河边的旧滑州三县,让流民南下就食。可以节省下运粮北京时在路上损耗的大半。”

赵顼摇摇头:“一路南下,恐怕在路上会有许多流民难以支撑。”

“如果是被迫南下,流民、官府无所准备,当然会如此。不过如果有沿途州县提前做好准备,那就不会有太大的问题。‘河内凶,则移其民于河东,移其粟于河内;河东凶亦然’,梁惠王能做,以陛下之仁德如何做不得?”

在王雱看来,今冬的灾情是没救了。到了正月还一场雪未下,田地里的麦子已经难以挽回。就算补种春麦,能守到秋时的也不会有太多。而且文彦博还是判大名府,有他在,就算送粮过去,河北流民也肯定要南下。

即然河北流民南下开封的未来无法改变,那最好的处置办法就是将流民们控制在自己的手中,以防有人乘机为奸。流民多也好,少也好,不让他们乱起来,那就没有任何问题。

由于此前的成功,王雱对于控制民意的好处已经食髓知味。而且来到开封等赈济的流民即便有个十万八万,只要老老实实的待着等大灾过去,天子也不会太担忧——不将其惨状之间看在眼里,对于身居九重的皇帝来说,就仅是个数字而已。

赵顼没有想得如王雱那般深,但他也觉得能将流民提前控制住是一件好事,不过他仍是摇头,“还是不妥。”

吕惠卿一言未发,只看着王雱的表演。在他看来,王雱的盘算太不现实——说是滑州的三县,其实应当就是韩冈所在地白马县——离着东京城实在太近了一点。

想想寇准,当年他费了多少力气才将真宗皇帝弄过河去?如果滑州还在,流民潴留在白马县,天子不会太担心。但现在滑州已经并入开封府,流民过了黄河就是进入了东京地界,天子怎么可能会答应?!

