\section{第16章 千里拒人亦扬名(中)}

“刘城主呢?”韩冈问的是伏羌知城——世间俗称知城、知寨为城主、寨主——伏羌城内乱成这样,再怎么说他也该出来弹压一下。

“今天一早,刘城主就带了两个指挥去了安远寨,好歹把谷内的蕃人给镇住。”

“那副城主呢?”

赵隆不屑的鼻中一哼:“溜须拍马上来的,他的话谁会理?”

韩冈摇头暗叹,难怪城门口检查的那么松懈,城中连个主心骨都没了,谁还会认真值守?人才果真是难得,能作为定海神针的将领,秦州也不多。少个张守约,固守秦州西北边防的甘谷城、连同周围一片防线全都人心惶惶。少了刘安,伏羌城也是乱了套。不过人才越少,自己出头便越是容易,鹤立鸡群,如何不显眼?不醒目?

韩冈一边想着,这时车队前方的街道中突然乱了起来,十几匹满载着货物的驮马突然从横街冲出,将前面的行人赶得鸡飞狗走,把车队前行的道路也顺便堵上了。

看着一片混乱的前路,赵隆骂道:“直娘贼,真的乱了,连去达隆堡回易的商队都逃回来了。”

回易就是走私,虽然在西北边境,除了几个官办榷场外,宋廷严禁宋人与党项人有贸易往来。但实际上,来往宋夏之间的商旅数不胜数,尤其以贩私盐最为多见。西夏拥有西北最为优良的盐产地,青白盐池出产的细盐,没有卤水的苦味,口感犹在解州盐池的解盐之上,价格又因为没有官府从中盘剥而十分低廉,所以极受西北百姓的欢迎。

能在敌对两国之间游走交易,虽然这些商人们看起来都是普普通通,但各自的背景都不可小觑。在边境走私的商队,没有点势力早给人吃得连骨头也不剩了。不过,如眼前这只马队这般嚣张的,却也不多见。

走私商队中的一位三十上下、瘦得如一根蔫黄瓜的中年人,正颐气使指的指挥下面的仆役驱赶挡在马队前的行人。他穿着普通的绸缎衣服,又走在驮马边上,应该一样也是个仆役,不过是等级高点罢了。只是宰相门前七品官,看瘦子狂妄的模样,也许已经能抵得上八九品了。

“赵敢勇,你知道他们是哪一家的?”韩冈问道。

赵隆冷笑一声:“都钤辖家的人,每月来往个三五趟,怎么会不认识!?”

“都钤辖?向宝?”韩冈再问。

“还能有谁?”赵隆没好气地答道:“秦凤就这么一个都钤辖!”

“难怪!”韩冈、王舜臣异口同声。

兵马都钤辖向宝,按序列是秦凤路军中的第三号人物。一个经略安抚路,地位最高的是经略安抚使,因为他同时还兼任一路兵马都总管,也就是军政和军令一把抓,基本上都是由文臣担任。而他之下,便是实际领兵的副都总管,而副都总管之下,便是兵马钤辖——若是钤辖资历老,前面便可缀个‘都’字,正如向宝。再往下,还有路都监——知甘谷城的张守约,便是秦凤路兵马都监。

除了经略安抚使外,下面三个都是武臣,互相之间级别有高低,但却无隶属关系,各自领兵驻扎于不同地点。可以分庭抗礼,大小相制,同听命于文臣经略。真要评判他们哪个说话更管用,还是要看他们的威望和功绩。

前任秦凤路副都总管杨文广刚刚调任,继任的副都总管是个没什么本事和战功,不过是在京营禁军中靠熬资历熬到点,韩冈连他的名字都不知道——恐怕秦州中知道他名字的也没几个——现在论起秦州军中真正说话管用的,还属都钤辖向宝。

前面乱了一阵,向家的回易马队改往韩冈他们这边过来。王舜臣忙提醒韩冈道:“惹不起的,权让一让吧!”

韩冈点了点头,也不想节外生枝,便下令让民伕们将骡车赶到一边去,让他们一让。

向家马队走过韩冈一众身边,那个瘦子突然停下脚步。问着靠在车上的王舜臣,“你们是哪一家的?”

赵隆在旁代答道:“是奉命由成纪往甘谷运军需的。”

瘦子冷哼一声,阴阳怪气道:“这么多人押送一点酒水,也不嫌麻烦,都能让人躺在车上躲懒了。”

王舜臣脸色数变,有一瞬间韩冈还担心他会出手给瘦子一下,但到最后,他硬是咽下了这口气,从车上下来,老实站好。除了一位重伤员,其他受了伤的民伕也依次下来,排队站好。一位正名军将,一个民伕,除非想自杀,如何敢去得罪已能被尊称太尉的向宝?就算是种谔来了也保不住他们。

瘦子见王舜臣等人从车上下来,倨傲的横了一眼,一副理所当然的模样。他的视线从众人身上扫过,来回几遍,最终一指韩冈,“就你了!”转过头,又对跟在身后的几个伴当道:“你们从这里拖三辆骡车走,赶紧去西门把剩下的货都装起来,九老爷正在那里等着。”

瘦子仗着有向宝做后台,也不信会被拒绝,颐气使指,完全视韩冈、王舜臣为无物。等几个伴当应了,才又转回来,对王舜臣道:“如果甘谷城有人问起,就说是向太尉家借了人车去,到了秦州就放还。若还有问,去向府找俺向荣贵。俺给他个交待!”

冷眼看着向荣贵自说自话,现在又看到几个向家的仆役要把车上装的绸缎往地上丢,韩冈终于忍不住了:

“等等!”

“怎么?!”向荣贵一眼瞪了过来。他到现在为止,仍把王舜臣视作众人的头领,跟方才赵隆一样,将韩冈当成了赶车的民伕。

“你要总要给韩某一个交待罢!”韩冈声音比眼神更冷,他一个向府的仆役凭什么能给人一个交待?到了甘谷城,不见了人,不见了货,有一百个理由让韩冈他生不如死,向荣贵会为他说半句话?扯什么蛋呐!

“这可是要送到甘谷城的军资!”韩冈强调道。

“向爷也没动你军资,只要你的车子而已!”向荣贵脸上怒意渐显,他只是觉得韩冈看着比那些民伕顺眼,才挑了他出来,“你这狗才,别不识抬举!若不是临时短了人手,向爷也不会当街拉人!”

王舜臣一把扯住似要发作的韩冈,今日一场厮杀,战后又得救治,他对韩冈已是敬重有加,如何愿看到韩秀才自蹈死路?却强扭着自己的暴躁脾气,向向荣贵卑颜笑道:“这厮脾气不好,官人换一个罢!”

“换什么换?!向爷说是他,那就是他!”向荣贵指着韩冈,瞪起他的那对白多黑少的小眼睛,狠狠道:“莫废话,跟着向爷走。别不识好歹,这也是救你的命。看着你个子高大,抗肩舆正合适!”

“给我滚!”韩冈一声大喝,中气十足,震得整条街都响起回声。不知何时,他已气得脸色泛青,双唇都在发抖,一副怒发冲冠的模样,“不过一个在钤辖府中奔走争竞的走狗,也敢奴事士子?!就算你家主子向宝过来,他也不敢!”

街市上,韩冈这一吼,吸引了所有人的目光。不论是王舜臣还是赵隆,又或是向荣贵,都被韩冈这突如其来的吼声给震住了。

死死盯着向荣贵,韩冈甚至觉得光凭语言无法表达出他的怒火,翻手摘下强弓,弯弓搭箭,一箭便向他射过去。

“秀才不可!”王舜臣在旁看得大惊失色,连忙抢上去要拦着。只是韩冈手脚太快,让他眼睁睁地看着那支长箭射飞了戴在向荣贵头上的毡帽。

王舜臣惊魂初定,暗自庆幸韩冈的箭术并不算好,隔着两三步都没能把人射中。要是真给他闹出人命,肯定要抵命。只是他一见韩冈手再次伸向了身后的箭囊,心脏又猛的大跳了几下,差点从喉咙口蹦出来,一步冲前,和赵隆两人一起将韩冈死死抱住,在韩冈耳边大叫道:

“韩秀才,你疯了?!射死了他你也要没命啊!”

“士可杀!不可辱!”韩冈拼命挣扎,咬牙切齿,看起来只想再给向荣贵一箭,“他这厮辱我太甚,竟欲以士子为畜!某为横渠弟子,受此之辱,日后又何面目去见师长同窗!”

赵隆给吓得不住的念佛,直念叨着:“阿弥陀佛,真的疯了!阿弥陀佛,真的疯了!”

王舜臣则苍白着脸,一边抱定韩冈不敢丝毫放松,一边对吓呆了的向荣贵吼道,“还不快走!”

“你给俺等着!”被吓得魂飞魄散的向荣贵丢下一句话,把马队丢下,连滚带爬的跑了。

向荣贵一走,韩冈立刻停止了挣扎,神色也突然间平和下来。挣脱开王舜臣和赵隆的双手,很淡定的整理起衣服。

王舜臣与赵隆面面相觑,周围看客指指点点,韩冈则是神色自若。

“秀才!”赵隆算是怕了韩冈这个疯子,说话也是小心翼翼,“你们还是快走罢!连夜去甘谷……”

“往甘谷夜路怎么走?”韩冈摇头,“今天是月末,夜里连月亮都没有,怎么走夜路?”

“可向荣贵马上要带人来了!”王舜臣也在旁帮忙劝着。

“他不是要韩某等着吗?我就在这里等!”

ps:新的高潮来了,红票和收藏的高潮也要来啊。今天第一更。

