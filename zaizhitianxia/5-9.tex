\section{第一章 庙堂纷纷策平戎(九)}

【第二更】

宫中派使臣来传诏到底是为了什么,群牧司中官吏自然不会有人猜不到。

这两天为了西北二虏之事,韩冈没少被请到崇政殿上去。厅中的吏员们对此都是见怪不怪了,只是觉得今天似乎是早了点,但再一想,已经没有朝会耽搁时间了,早一点也不足为奇。

有关辽夏两国都陷入了内乱的消息,早已传遍了民间。不仅仅是朝堂上吵得热火朝天,连京城中都因此而沸腾起来。京城百万军民没有不兴奋的,人人都看到了一口气解决边患的机会。

现如今,到底要不要攻打西夏,只要到街上的茶社中坐上一个时辰,至少能听到七八场议论。

不过不像朝堂之上,几乎是速战速决的方案一边倒受人支持的局面。由于韩冈的声望,至少在民间,速攻论和缓攻论的比例算是一半对一半。打还是要打,只是要不要直取兴灵还有争论,民间两派势均力敌,吵得沸反盈天,一时间取代了季后赛的总冠军谁属,成为最热门的话题,让茶社酒馆的掌柜和东家们喜笑颜开。

群牧司中的大小官吏,如今也在打赌,赌到底是主张速攻兴灵的王丞相得偿所愿,还是坚持缓进的韩龙图棋高一着。衙门中消息灵通,使得押在韩冈这边的赌金少得可怜,只有喜欢冷门的几人在韩冈身上下了大赌注。

韩冈并不知道有人将发财的希望寄托在自己的身上,他听到禀报之后,就整理了一下身上的穿戴,到了院中听候内侍传达天子的口谕。

年纪不大的小黄门尖着嗓子,在韩冈的面前抑扬顿挫:“天子有命,均国公和淑寿公主今日种痘,着韩冈即刻入宫。”

天子的口谕一出,不仅韩冈愣了,就是旁边作陪的司中官吏也都愣了。不过,官吏们很快就反应过来,皆是露出了一幅果然如此的神色。这是理所当然的,要是自家能请动韩冈这尊大神,也会照样想着能在儿女种痘的现场,能有药王弟子坐镇。可惜能使唤得了天下闻名的韩龙图,也只有当今天子。

韩冈却在苦笑。自己的名声在外,赵顼下此诏,也只是为了求个安心而已。舔犊之心当然值得感动,韩冈也能理解,不过他却不能随随便便的答应下来。

众目睽睽之下,韩冈继承了他岳父留下来的传统,“外臣岂可入内宫?种痘之事已有厚生司主持,韩冈岂能越俎代庖。此诏韩冈不敢受。”

韩冈直截了当的拒绝。周围旁观的群牧司官员都是倒抽一口凉气,为韩冈的胆量吃惊不小。

韩冈头低着,眼睛看着地面,表现了足够的谦逊,但他的腰背是挺直的,绝不会为天子的乱命而动摇。

他是朝中有数的重臣,更是天下知名的儒者,不是天子家奴,怎么能往内宫乱跑。而且韩冈一直都不承认自己通晓医术,又不是厚生司中人,已经完全种痘之事交托出去了,遽然插手其中,名不正言不顺。

不过身为士大夫的臣子们究竟是什么德性,赵顼也很清楚。韩冈也是其中的一员,从来就不能指望他们能屈己适人。

前来宣召的内侍早有准备,立刻道,“龙图,种痘的地点安排在崇政后殿,并非内宫,而且宰相亦在。”

韩冈这就不能拒绝了,崇政后殿是他经常去的,且王珪亦在殿中,韩冈也就不用担心职权问题。

其实也是韩冈想看到的结果。作为臣子,总不能让天子太难看。他前面的话已经透露了自己的想法,给赵顼留了余地。等宣诏的内侍回宫复命的时候,天子应该知道怎么做。

不过韩冈没有想到,赵顼竟然事先都预计到了自己的可能会有的反应,让自己没有半点拒绝的余地。

韩冈心中暗暗冷笑,当今天子,在这些小事上表现得还是挺聪明的。

“臣遵旨。”韩冈领命。

周围官吏们紧绷的神经都松弛了下来,甚至能听到他们同时吁了口气。能当面看到大臣落天子脸面的机会并不多,对于低品的小官,以及更为卑微的胥吏们来说,实在是很挑战他们的神经。方才韩冈硬顶着天子的口谕,随之而来的紧张感和压迫感,让许多人都喘不过气来。

而且天子竟然事先考虑到了韩冈可能的拒绝,特地为他安排地点和陪客——在群牧司官吏们的眼中,种痘之事上,王珪的存在完全是韩冈的陪衬——天子对韩冈的信重由此可见一斑。羡慕嫉妒的眼神,在庭院中飞来飞去,就是不离韩冈的左右。

内侍明显的也松了口气。他们这等传诏的天使,最怕的就是碰上犯了倔脾气的大臣。运气不好时,就要来来回回跑上好几趟。而且回禀天子的时候,还要战战兢兢的担心会不会被迁怒。身为天子家奴,一个不好,就是万劫不复的结果,可比不上士大夫们的自在,能放开来说话做事。

天子就在崇政殿等候,而天子仅存的一对儿女也正要进行种痘,韩冈也不多说废话,让人牵来自己的马匹,出了衙门就往崇政殿中去。

韩冈抵达崇政殿后殿的时候,正如内侍所说,他发现王珪就在其中,而且赵顼也在,当然,更少不了来为皇子、公主种牛痘的厚生司中人,判厚生司的安焘带领李德新为首的几个痘医,就站在殿门内侧。

派出去的使臣久久不至,赵顼正等得有几分不耐烦,看到韩冈终于到了,他紧绷的脸松弛了下来:“韩卿,你可终于到了。”

“微臣叩见陛下。”

韩冈在赵旭面前行礼如仪,肚子里则腹诽着,希望日后不要让自己每次都来做了压宅的镇物——如果天子还能继续生养的话。

“有了韩卿来了,朕就可以放心了。”赵顼笑道:“种痘法乃韩卿你所献,夺天地造化。有韩卿在侧……”他看看王珪,“还有宰相,朕也就能放心了。”

韩冈和王珪连声谦虚,说了些相互捧拍的废话,高高在上的天子已经忍耐不住了,提醒道:“该开始了吧。”

没人反对。王珪和韩冈都想早点结束。

赵顼随即就派了人去传话。

片刻之后,皇六子均国公赵佣从偏殿被抱出来了。

拥有亿万人口的世界第一大国的第一继承人,被乳母抱在怀里,旁边一个老嬷嬷小心看护着,还有宫女、内侍,十几个人围在左右。

赵佣穿得鼓鼓囊囊的,看不出身材胖瘦,但脸颊没有富贵人家小孩子的丰满,而且脸色也少了幼儿应有的红润。眼睛睁得大大的,前后左右的张望着殿中。看看赵顼,又看看王珪和韩冈,然后从安焘他们身上一个个看过来。

熙宁九年腊月初八出生,刚满两周岁,按虚岁算则是三岁。从来没有出过内宫,到了崇政殿就是一幅好奇的模样。

除了均国公赵佣之外,同时出来的还有稍大一点的淑寿公主,粉雕玉琢的小女孩子,也跟她的弟弟一样对内宫之外的世界很好奇,不过还知道要先向赵顼问安,挣扎着下地来行了礼。

赵顼对仅存的一对儿女很是疼爱,看着儿女的神色完全是一副慈父的模样,在朝堂上是完全见不到的。

转过来,赵顼就催着快动手。

先上阵的是淑寿公主,才五六岁的小女孩,以厚生司众医官的经验,更本不算是什么难事。但李德新明显的紧张,想拿起宫中提供的银针,手指抖着,用了两次才抓了起来。

而当他拈着银针,当场用火和酒精消过毒,凑近到淑寿公主身边。一对乌溜溜的大眼睛就盯着李德新手上的银针,还没凑近到手臂上,淑寿公主立刻就放声大哭起来。小小的拳头挥舞着,就是不让李德新拿着银针的手靠近。

李德新急了,连忙催着服侍淑寿公主的宫女,“抓着手,抓着手,不抓住手就种不了痘。”

乳母一下将淑寿公主给抱紧了,又有一名宫女抓着手,但哪里敢用力,几次都被淑寿公主挣脱了出来。

四五岁的小女孩儿哭得更凶,嘶声力竭。父皇,父皇的叫着赵顼。

赵顼在旁边听得一幅想捂耳朵的表情。回过头来又瞪着李德新几人,恨他们怎么闹得跟生手一样,就是怕吓到女儿,不便骂出来。

李德新额头上豆大的汗珠直冒,后面的随从忙着用干净的手巾帮他擦拭。就是面对亲王家的儿女,长公主家的儿子,都没有这么大的压力。

小孩子哭闹听得多了,就没见过不哭的。让人扯住手臂,直接施针施药,根本不费事。几千几万人都做下来了,却偏偏栽在了公主手上。

“陛下。”王珪站出来了,“还是快一点得好。公主穿得单薄,手又露在外面,受了风邪可就不好了。”

赵顼忙着点点头,亲自动手抱住了女儿,让李德新快点下针种痘。

银针划破了白皙细嫩的皮肤,鲜红的血流了出来。赵顼瞧得心疼的,抬头怒瞪着李德新,催着他快一点。李德新汗水一个劲的直冒,淑寿公主哭得声音几乎震破了殿上的琉璃瓦,但终于还是完成了。

乳母抱着淑寿公主,赵顼就在旁边哄着,许了玩具、许了糖,许了菓子,好半天才让女儿抽抽嗒嗒的不再号啕大哭了。

终于将淑寿公主安顿好了,李德新已经是一幅快虚脱的样子,但还有更大的难关等着他。

