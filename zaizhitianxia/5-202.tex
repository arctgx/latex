\section{第23章 弭患销祸知何补(十)}

轻巧的马车碾过天波门前的青石路。

王旖的所乘的四轮马车从曾经是天波杨府的孝严寺门前经过,在天波门前稍停了一下,查验过身份,便直入宫中。

这辆马车可以直接进入皇宫之中,被招入大内外命妇也不方便在宫城中下车行走。当然,车子并不是走的皇城南面的宣德门或是左掖门、右掖门,而是得从西侧的天波门入宫。只要不是节庆或是喜丧之礼,城西的外命妇入宫多是走这条路。而城东,就是东华门,禁中采买外物都集中在此门外,市面之繁华在东京城中也是顶尖的。

入了禁中,王旖便立刻下车,在皇后派来的内侍引领下,一路往坤宁殿过去。

到了在坤宁殿前,远远的就看见一名宫装的嫔妃带着五六名宫女和内侍从殿侧的寝阁出来,而后快步离开。

‘是刑婕妤!’

王旖时常入宫,见到的也多是向皇后和生了六皇子和淑寿公主的朱妃,其他嫔妃偶尔也能见到,只有刑氏从来不在其中。

这个邢妃,就是因为痘疮而死的七皇子的生母,至今尤深恨韩冈没有及时进献种痘的方子。虽说远远地看到了人,王旖连提都不提。

“东莱郡君求见。”

通传声从殿外一路传进殿中,而宣她入内的懿旨转眼又传了出来。

王旖跨步进殿,被引到东寝阁中。

向皇后和朱贤妃在座,但王旖的眼角却在第一时间瞥到了皇后手边小几上的香精匣子。

那是自家的出产,而且是价值最高的商货之一。

韩家名下的诸多作坊,王旖作为主母,多有了解。织造、玻璃、香精、制糖,都是如同聚宝盆一般的产业。但财产太多也是有问题的,唯一值得庆幸的是香精工坊在陇西并不是独一份。

皇后拿出此物,是不是有什么用意?王旖在行礼的时候,心中一片纷乱,不由得暗自念叨:‘要是那个冤家在身边就好了。’

……………………

韩冈此时也在皇城之中,不是太常寺,而是崇政殿。

在做了判太常寺,主管厚生司、太医局,又担负起编纂药典的差事后,韩冈难得有被召上崇政殿问对的机会。不过这一次天子赵顼所关心的也不是韩冈手上的差事,仅仅问了两句有关《本草纲目》的进度。便将话题转到了东京城中这两天最热门的事情上。

“依韩卿之意,此事当如何处置?”赵顼问着韩冈。他想听一听韩冈的看法,也想看看韩冈的才能。

“此事事归开封府,宰执可论,台谏可议,却非臣可以妄言。”韩冈推脱着,“且蹴鞠赛制出自于臣,臣亦当避嫌才是。”

赵顼摇了摇头:“朕知此事与韩卿无关,勿须讳言,可放胆直言。”

这样的鬼话,也只有鬼才会相信,韩冈腹诽着。不过他也正在等赵顼的这句话。

“臣遵旨。”韩冈早已是胸有成竹,行过一礼后便开口说道,“据开封府的奏覆,肇事之人乃是南顺侯李乾德,其人又已自食其果,就是追究罪责,也无从着手。补偿一众苦主,并设法防止悲剧重演,才是当务之急。”

韩冈说的话,算是陈词滥调,几乎所有人都明白这个道理。

从当今的刑律上看,球场外的惨剧根本无从定罪,至少不是故意杀人。可即便是过失杀人,又怎么才能将那些将人踩踏致死的凶手们绳之于法?那样成都的混乱,恐怕有几百上千人之多。

但出了意外死了人,必然要有个原因,也必须有人出来负责。若能将责任推到死人身上,这对大家都是好事。怨有所归,只看这四个字,就可以知道转嫁责任从来都是最简单有效的办法。将罪名推给的李乾德,虽然手段下作了一点,但从大局上,对绝大多数人都有好处。

此外,赵顼最近不是被人怀疑是李乾德之死的幕后指使吗?现在洗清了冤情,岂不是皆大欢喜——自然,这一句是不能说出来了的。

赵顼略皱眉:“开封府的断案不一定是正确无讹,南顺侯无法为自己辩解,只能任人说。不过南顺侯府已经递了状子上来,要朕为其洗冤。”

“既然南顺侯府有争议,可交由御史台复审。”韩冈很干脆的说道。

赵顼微微一笑:“交给御史台就够了?不用大理寺和审刑院?”

“日前的惨剧是因争吵而生乱,非是有心人兴风作浪。即便是南顺侯引发,也不能算是罪名。此一事不涉律条,不当动用到大理寺和审刑院。而御史台有风闻奏事之权,朝廷诸事亦皆得与闻。纵使非关律法,也有资格复审。”

赵顼挺意外,御史台一直都想在韩冈身上找回面子,韩冈主张将权力交给御史台,岂不是自往虎口中钻?

只不过,赵顼多想了一想也就明白了,这是将御史台架在火上烤。市井中的舆论已经完全将李乾德当成了罪魁祸首,甚至是凶手,若是御史台偏向他,等若是一口气得罪了所有人,千夫所指,无疾而终,御史台中的官员没几个能抵挡得了这样的风暴。

御史台跟所有衙门都不对付。韩冈确信,只要赵顼不明确表态反对,那么政事堂只会站在两大会社的一边。

据韩冈所知,蔡确是肯定支持蹴鞠的。蔡确的弟弟蔡硕的内兄姓明,是蔡确之母的堂侄,与蔡确兄弟是姑表亲。福州的蹴鞠联赛,明氏在其中有着很重的份量。

要知道一旦朝廷禁蹴鞠,禁令的范围就不会局限于京中。而天下各军州,能参与控制齐云社和蹴鞠联赛的无不是巨室世家,满满的利益在眼前,怎么可能允许有人虎口夺食?

瞧得出韩冈胸有成竹,赵顼忍不住带着点恶意的问道:“如今的蹴鞠联赛乃是韩卿当年所创,如今一场球赛,便能聚万人之众,不知韩卿对此有何看法?”

韩冈略皱眉头:“臣当年提倡的蹴鞠,不过是军中戏,希望汉蕃两部能消弭隔阂。也因为是军中戏,所以更重拼杀和争锋,便依从马球改了许多规则。会变成如今的这番局面,也是臣事先所没有料到的。”

赵顼不想听韩冈的辩解,他开门见山的问道:“依韩卿之见,蹴鞠联赛是该继续办下去,还是就此停办。”

韩冈思忖了片刻,缓缓的开口:“记得种谔之父,其镇守清涧城时,曾经在山头修有一庙。不过此庙地势甚高,到了最后,竟还有一根主梁没有架上去。”

韩冈突然说起了故事,赵顼并没有打断他,而是专心的聆听。战国策上的那些说客,甚至儒门的经籍之中,以古讽今,或是借用寓言来说服他们的目标,都是很多见的。韩冈也不过是拾人牙慧。

看来这就是韩冈的目的。赵顼想着,很有耐心的听着韩冈继续说道。

“为了能尽快将房梁上好,种世衡使人传播消息,说是要在黄道吉日举办一场相扑大赛,以庆贺寺庙落成,召集清涧城内城外数以万计的百姓与会。到了约定好的日期,满城百姓都到齐,种世衡便催促说,快点将房梁与上去,好让比赛能顺利开始比。本来要花费上百人的劳力和为数众多的钱粮,但种世衡一句话,便让数以百计的百姓一齐出手,将房梁一举运了上去,庙宇一蹴而就。这一切,仅仅是为了看一场相扑而已。”

种世衡的故事,韩冈说的不是很有趣,但种世衡的头脑却已经明明白白的展示给了赵顼。当今天子点头赞许:“种世衡的才智,纵使放在国初,也能跻身第一流。”

“此事世人盛赞种世衡之智,但从清涧城军民的角度来考虑,为什么一场相扑便能聚集成千上万的人手,使得原本要耗用大量人工的梁柱,轻而易举的架上了房顶?”

赵顼似乎是明白了一点:“韩卿的意思是?”

“乃是因为世人的需要。在劳作和饮食之余,世人还是要有些打发时间的去处,明世人之心,察世人所求,故而种世衡的谋算能够成功。”

“……韩卿的意思是大禹治水,堵不如疏。”

韩冈点点头:“陛下明鉴。既然百姓喜闻乐见,何必严禁。又非淫祀、啸聚,只是如同庙会一样的球赛而已。能进场看球,必是有闲有钱之人,也不至于需要担心有心人能拥众作乱。”

“说得的确有理。不过球赛上的赌博,实在是有伤朝廷体面,易为世人所笑。”赵顼的问题,如同在考试。

韩冈幸而早有准备:“蹴鞠、赛马,本是军中练兵之法,若能专款专用,用在保甲之事上,当无人可以议论。”

赌博,在后世被律法禁止得更严,但国家坐庄开赌,将赌金的利润用在正当的地方,却是理直气壮,也没有什么人能非议。

赵顼沉默了下去,手指按着眉心。以韩冈对他的了解,应当是心动了。

通过保甲训练民兵,是加强国家军力的重要手段,但为此花费的钱粮亦是个大数目,地方上也多有怨言。就赵顼所知,保甲法推行有年,但只在北方各路多多少少有一点成果,而在南方早已是流于形式,冬日各保甲保丁作训,全都是糊弄过去。

若能别开财源,将开支给补足,至少将蹴鞠和赛马的赌金税收的使用设为定制,那么对保甲制度的巩固必然是个绝大的助力。

更何况眼下在两项联赛中流转的金钱,可是一个天文数字。在其中分润到的,从开封府衙中的官吏,到数以百计的大小宗室,都是。这还仅仅是开封,天下四百军州,开办蹴鞠联赛的占到其中的一半以上。

禁了开封府的联赛,全国各军州的联赛也肯定一并禁了,若是青苗贷那般有补于朝廷的法令还好说,但禁了蹴鞠联赛,对朝廷可是没有半点好处,反而会让宗室更加依赖国库里面的财富。

已经不是变法时的你死我活,有必要闹得人心不安?何况还有钱的问题。

赵顼在登基后就觉得这些亲戚对朝廷财计是个巨大的负累,让王安石制定宗室法,将朝廷发给钱粮的人数大幅减少。但剩下的宗室,在国计而言,依然是个巨大的负担。

而且对那些宗室来说,身处那个位置上,该花的钱不能少,光靠朝廷发下来的俸禄和偶尔的赏赐,永远都是不够的。天家的体面也要照顾,不靠外财,难道还能从内库里想招数吗?

赵顼已经有了决断,只是在他的脸上看不到半点端倪。
