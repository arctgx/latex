\section{第11章 五月鸣蜩闻羌曲(五)}

【第三更,求红票,收藏】

韩冈是个乐观的人。一直以来,他都对自己充满了信心。自信凭借自己的才智和能力,无论前路有何阻碍,他都能一剑斩开。即便斩不开,也能设法绕过去。

但他的思考方向,却是一贯的偏向阴暗面。凡事都会先往最坏的方向去考虑,总是不惮于从最卑劣的角度去揣测人心。

而事实,往往证明了他这种做法的正确性。

当听到仇一闻说他徒儿的这桩案子牵连到秦凤路副都总管窦舜卿,这位与王韶一派互相攻击的死敌,韩冈便一下提高了警惕。

是阴谋,还是巧合?

韩冈无意去头痛事实为何,他只会去往阴谋的方向去思考,去准备。

他有理由怀疑这是窦舜卿针对他的阴谋。在王韶身边为之奔走、有时又会出点计策的助手,创立疗养院帮王韶收拢秦凤军心的得力干将,韩冈的这个身份已经为秦州官场所公认。

窦舜卿想要打击王韶已经不是一天两天,那个万顷变一顷,一顷变没有的弥天大谎,就是他的得意杰作。可是如今,窦舜卿自己心中也应该清楚,在王韶已经立下军功的情况下,他针对王韶的计划是越来越难以成事。

既然如此,就得换个方向。

如果不能动得了本人,那就从他身边人下手。反变法派怎么对付的王安石,窦舜卿他们也会怎么对付王韶,而且李师中都已经做出了榜样——虽然他的阴谋为王韶和韩冈所破坏,还让反王韶的阵营折了向宝这个大将。但对付韩冈,终究要比对付王韶要容易……

心中思量迅如电闪,韩冈脸上的笑意丝毫未有的收敛,但眼底的寒芒却愈发的锋锐摄人起来。

“仇老,你这可是欺负小子年轻啊……”韩冈笑吟吟的说着,但说的话却毫不客气。

仇一闻知道自己在说事的时候玩了一点狡狯,但他也不在意韩冈现在的心情,“老头子不是怕韩官人你听到窦副总管的名字就退缩吗,就跟老头子此前找过的那几个没胆的家伙一样……”他盯着韩冈,“韩官人,你前面都答应了,现在该不会说不干吧?”

“这可难说。”韩冈的笑容渐次收敛起来,眼中寒意更盛,“只许仇老你诳我,就不许我反口吗?既然要跟窦副总管打交道,这事我可是还要再想想。”

仇一闻沉默了下去,眉间沉郁渐次凝起。韩冈喝着凉茶,似无所觉。两人都不说话,厅中一时静了下来,窗外的蝉鸣越发的变得聒噪。赤日炎炎,掠过小厅的穿堂风都是热烘烘的。

韩冈的视线漫无目标的在厅外游走,透过竹帘,院中的地面都在反射着阳光,白晃晃的眩眼。

一开始韩冈听说患儿的家属把医生送进大狱,韩冈就觉得有哪里不对劲。因为不符合如今的实际情况。但如果是窦舜卿想借着仇一闻的手把自己拖下水,那就能说得通了。不过死得的是窦舜卿家的重孙子……窦解应该才二十出头吧,就有三个儿子了?!韩冈晃了晃脑袋,没心思去赞叹窦解少年时的惊人战绩。

以上的猜测也有可能是把窦舜卿他们想得太聪明或者是太阴险了一点,说不定今次的事故真的是意外而已。不过,一旦韩冈为仇一闻的党项弟子出头,那么就算早前窦舜卿没有这个意思,但他身边的人,也会提醒他把西夏、党项郎中和韩冈,用一根绳子拴起来。韩冈曾经给陈举一党栽了个西贼奸细的罪名,他可不想弄出个现世报的笑话。

仇一闻是个好人,在秦凤路上做了几十年的医生,不知救治了多少人。但他的声望斗不过窦舜卿的权位,所以他来找韩冈帮忙。但从自身安全上讲,韩冈他不可能去帮他,去帮他找窦舜卿说话,把他的党项弟子从大狱中摘出来。

韩冈若是这么做了,不是递了把刀给窦舜卿,就是自己把脖子伸到绞索里——两者的分别端看今次的事件是否是窦舜卿的阴谋——结果都是找死。

但韩冈也不想就此得罪仇老郎中。他看着仇一闻的脸色,已经冰冷如寒冬子夜。如果自己真的说个不字,他多半就会掉头就走,再也不会给自己什么好脸色。这对韩冈维持在秦凤军中的声望很不妙。

毕竟韩冈在甘谷疗养院中,得到仇一闻的帮助很多。而且他手下的一众以朱中为首的医师,也是受到仇老郎中不少指点。而韩冈的名声也是仇一闻先帮忙捧起来的。

受人恩德总得回报。韩冈当然不会自己跳进窦舜卿的陷阱中去,但他还是有着变通的办法。

“仇老。”韩冈重新挑起话头,仇一闻头转了过来,脸色还是难看。

“在下从来都不喜欢被人诓骗,若是平常有人如此戏弄于我,我可是掉头就走。不过这也是仇老你第一次求我办事,在情在理,我也不能拒绝。这事我会帮着你想办法的。”

听韩冈说到这里,仇一闻脸上开始晴转多云。

韩冈继续道:“窦副总管位高权重,我区区一个从九品跟他攀不上交情。不过在王机宜和高提举面前,我还是能说得上话。通过他们跟窦副总管讨个人情,只要窦副总管为自己的重孙气得不是太厉害,应该就不会有问题了。”

仇一闻已是变得喜上眉梢,没口的谢着韩冈。一直看着他反应的韩冈心情为之一松,看起来仇一闻并没有参与到窦舜卿可能的阴谋中去。

“今天仇老你奔波劳苦,暂且歇息一天,等明日,就请仇老你和小子一起回秦州。想来这件案子不会这么快就判下来,就算判了也要等大理寺批下来,在入秋后才会动手,我们还有点时间。”

韩冈把事情丢给王韶和高遵裕,让王韶和高遵裕他们去跟窦舜卿打交道,而将自家从陷阱中摘出去……不过要先在王韶和高遵裕面前做个预防,省得他们以为自己是祸水东引。

仇一闻听了韩冈的话去休息了,韩冈则是忙碌起来,因为比他原定的计划要提前了几天离开,他不得不将忙着安排着疗养院中的一应事务。接下来,一宿无话。

次日一大清早,韩冈就和仇一闻一起启程返回秦州。作为寨中地位最高的文官,就有这个好处,不用理会比他高品的两位武臣的话,可以自行决定行止。

韩冈骑马,仇老郎中坐车。也不避白天暑热,韩冈和仇一闻从清晨到入夜,都奔波在路上。几天后,到了陇城县,他们便如愿以偿的赶上了王韶一行。

“玉昆,你怎么来了?”韩冈被引进王韶的房间,房间的主人便惊讶的问着他。在计划中,韩冈至少要等到古渭疗养院中的蕃部轻伤员大部分痊愈后才会回返。

“因为有件紧急事务要想机宜你禀报?”

王韶清楚韩冈不是会一惊一乍的性格,他回来得这么急,那当是一件大事了:“什么急事?”王韶追问着。

韩冈便把仇一闻的党项弟子被窦舜卿下狱的事情,原原本本的向王韶说了一通。

王韶随即陷入沉思,韩冈的行动已经明确的向他做出了暗示,他很容易就看穿了韩冈到底想说些什么。

他的言下之意,让王韶觉得匪夷所思,窦舜卿至于用这个策略吗。“玉昆,你这是不是误会了?”

“不知窦副总管说秦州只有荒田一顷四十七亩,是不是误会?”韩冈立刻反问。

不管是真是假,先把罪名栽给窦舜卿再说,不然怎么请得动王、高二位?若无必要,王韶和高遵裕都不愿跟窦舜卿打交道。但看到窦舜卿都欺上门来了,他们却没有不还手的道理。正好窦副总管本有前科,不由得王韶不信。

王韶沉吟着,过了一阵,他问道:“玉昆,你有什么想法?”

“窦舜卿这是挖坑陷人。只要我不踩上去就行了。”韩冈接着话锋一转,“但仇老曾有助于我,此事虽小,我却不能不报。所以想请机宜跟高提举说一声,请他出面把仇老的那个弟子救出来。”

韩冈知恩图报的想法,王韶倒是很赞赏。而且窦舜卿能害他王韶,能害韩冈,却不能害了高遵裕。让高遵裕出面,窦舜卿也只能干瞪眼。

王韶随即将高遵裕请来,把事情的来龙去脉以及韩冈的分析跟他一说,高遵裕毫不怀疑的相信了。窦舜卿曾经陷害过王韶,高遵裕也道这事他做得出来。

太后的叔叔沉吟着,自家的事老是被人阻着让他很是心烦:“总是让窦舜卿之辈算计来算计去,也不是个事。虽说兵来将挡,水来土掩,也不惧他半分。但有千日做贼的故事,却没有千日防贼的道理。照我说,还不如辛苦玉昆一次……”

韩冈的脸色为之一变,心道‘该不会……’

果然,就听高遵裕道,“……将计就计,让窦舜卿自食苦果。”

‘麻烦了。’韩冈暗自叫苦。王韶和高遵裕可能的反应他都有预测过,将计就计反过来害窦舜卿一下,也是可能性之一。而且还很高,因为王韶和韩冈此前对付向宝的手段,也可以归入将计就计的这一类。

但韩冈可不喜欢这一手。

高遵裕看到了韩冈的脸色,他笑道:“玉昆你是不用担心的。有你此前的功劳,天子不会相信窦舜卿的话。窦舜卿想做的,也不过是把你弄进大狱,好好的教训一番。等回秦州,你就住进我家去,有我保着,看他怎么抓人。”

“仇老已经七十多了,可吃不住牢狱之灾。”

“跟你一样,我也会保他的。”高遵裕答应得很快,但韩冈在他脸上没看到半点诚意。

