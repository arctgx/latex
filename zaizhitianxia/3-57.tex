\section{第21章 论学琼林上(下)}

【凌晨两点写完后,纵横网络出了问题,只能现在发了】

韩冈脚步一顿,眉头也不由自主的一皱。想不到在宫宴之上,竟然被人连名带姓的叫着。

这可不是千年之后,呼名道姓正常无比,亲近的更直接唤名,而略去姓。在这个时代,平辈之间直接叫名讳,那就是在骂人——为什么‘名’之后要加个‘讳’字,就是忌讳的意!长辈能唤小辈名字,但也不是常有的事,基本上是责骂时才用。地位高者亦同此理。

王安石、王韶从来都是称呼韩冈的字,就是天子也道一声韩卿。‘韩冈’二字,说实话,还是他自称的时候比较多。来人直接叫着韩冈的名讳,的确在礼法允许的范围内,但从称呼中就可以知道他并没有带着善意。

来人四十出头的年纪,方面大耳,留着三缕长须,甚有威严。腰缠金丝缠成的御仙花带,却没有配鱼袋——不论金鱼袋还是银鱼袋都没有。韩冈知道,这不是他的职位低,而是官阶高到学士一级之后,出行就不配鱼袋了,只配金腰带。直到升做两府,才御仙花带和金鱼袋一起佩戴,称为‘重金’。

在韩冈身后奉酒的小吏,低声在就他身后提醒——这也是他们的工作之一——“此是杨翰林,讳绘的便是。”

其实不用提醒,韩冈在殿试上已经见过一次,跟着王安石之后,为二甲、三甲唱名的正是翰林学士杨绘。

杨绘曾任宝文阁侍制,后升上来做了翰林学士。以文名著称于朝,不合于新党。除此以外,韩冈对他就没有多少了解了。但杨绘的口气如此之冲,想来也简单。不敢再王安石面前犯冲,在韩冈面前展示一下风骨,也算是划清立场了。

‘这就是做王相公女婿的结果。’韩冈避开席面,上前半步相迎。杨绘无礼,他却不能无礼:“韩冈拜见学士。”

弯腰起来,只见杨绘拱了一下手作为回礼,韩冈神情不动,但眼神又冷了三分:“不知学士又何指教?”

“也无他事。”杨绘这时倒换了一副和颜悦色上来:“酒宴过后,就要颁赠天子的御制诗,不知韩玉昆你可有什么不方便的地方?”

依故事,琼林宴上,天子都会以御制诗一首赐予众进士,而为了感谢天子所赐,进士们都得和诗一首,呈与天子。

杨绘来见韩冈,周围的进士都被惊动了起来。而一听到杨绘的询问,更是各个嘴角抿着笑意,竖起了耳朵。

同为朝官,一直呼名唤姓未免太过分,杨绘也不便这么做。但他直接问着韩冈接下来能不能作诗,这就是当众打脸了。韩冈不通诗赋可是有名的。如果一般情况下,韩冈不是笑着咽下这口气,就是设法将话题转到对自己有利的地方再反击回去——谁叫他不可能没有准备的情况下,信笔挥洒出一篇能让人看得过去的作品。

不过琼林宴上要做诗,是从唐朝曲江宴上传下来的规矩,韩冈自是有所预料。为此,他已经看过了历年来的琼林诗作,臣子和诗中所常用的辞藻都背了一肚子,只要不是一些险韵,都有办法应付过去。

韩冈不善诗词,只是相对而言。自知缺点在何处,有三年的时间却不去想办法弥补,他也没那么蠢。这三年来,他写了多少公文?笔力早就练出来了!要知道公文也是讲究着文笔。韩冈的缺乏文采,是跟那些能高中进士的儒生相比,并不是说他一点诗都不会做——之所以一直对外宣称自己不擅诗赋,是给自己一条退路,但事情逼到头上,反咬或是跳墙的本事,他都有。

韩冈一开始的底气也是如此,但还是有人为他担心。前日王韶带回来的一句话,让韩冈事先知道考题。今天颁下的御制诗,当然不可能是今天早上赵顼才匆匆写下的,都是提前了几日准备好,且不是军情机密,很容易就打探得出来——谁也不会想到,这件事还会有人作弊。

所以昨天韩冈都是用着这个韵脚,苦思了一天,做了几首诗。修改了一番后让王韶评鉴,也点头道勉强能说得过去——琼林诗作,本来就是那么回事,非是王、苏这一级的大才,任谁也难写出好的来。

故而韩冈回答杨绘时,便是底气十足,仍带着谦逊的微笑,回答却没有半点迟疑:“韩冈虽不通诗赋,但故事如此,自当敷衍一篇出来搪塞一下。”

“敷衍,搪塞?!”杨绘语气变得激动起来,厉声质问:“韩冈,你受天子重恩,难道天子的御制诗,你就不能用心去和上一篇?!”

“这……的确是韩冈失言了。”

韩冈自承不是,双眉去又皱上几分。他自知这算是失言,但杨绘抓着这一点来攻击,已经近于文字狱,未免太过分了一点。官场上虽不讲究着一团和气,这么旗帜鲜明的为难自己,究竟是要做给谁看的?

见到韩冈皱眉不语,杨绘笑得更加得意,称呼也越发亲近:“如此倒也罢了,相信是玉昆你无心之过。只是近日听闻,玉昆你前日在清风楼上,被一众士子抢白得要辞了进士,这可就有失朝廷体面了。”

依照这两日传开来的谣言,韩冈在清风楼上被人逼的要辞去进士及第。这等无稽之谈,不少人都是摇头不信,怎么可能有人会丢下进士头衔不要。但他们嗤之以鼻的同时,依然还是将这流言传播出去,当成了可以嚼舌根的好谈资。事不关己,当然是乐于传一传韩冈的笑话。

所以杨绘把这事当众揭出来时,周围诸人都想看看韩冈是怎么为自己辩解的。

“谣言止于智者。”韩冈不急不怒,气定神闲的回复着,不再多说第二句。

周围听见的都将视线转回到杨绘脸上,看着翰林学士的脸色立刻就难看起来。没有加上‘相信以学士的才智,当不会相信此等谣言。’这样类似的话语,以用来缓和那一句咄咄逼人的口吻,韩冈这明明白白的就是在骂人。

“事关朝廷体面,不得不多问一句。”杨绘的声音冷着。

“若败坏朝廷名声,自有有司追问。如此等谣言,只是坏了韩冈一人名声而已。韩冈都不在意,学士又何必记挂在心上。”韩冈微笑道,明摆着在说‘关你屁事’。且更进一步反驳杨绘:“谣言无稽,当弃而不顾才是。即相问,便已是一失。韩冈斗胆,还望学士深思!”

他说着,拱手行礼。不知到杨绘是怎么想的,心里有什么盘算。从自己的这边来考量,还是直接翻脸比较容易解决问题。反正眼下杨绘狗嘴吐不出象牙来,都不会有好话。

“玉昆说的有理,杨绘当会深思。”杨绘笑得温和,“玉昆口才深得横渠张子厚的教诲啊。尚记得张子厚在洛阳坐虎皮论易,一番讲学,无人能比啊!”

这下轮到韩冈脸色变了,眼神中也终于多了一分怒色。说起这个时代真正能让韩冈敬佩的,人数其实极少,也不过三五人而已,但张载绝对是其中之一。如果杨绘只是跟自己过不去,斗几句嘴倒也罢了,比嘴皮子他绝不会吃亏。现在竟然攻击到张载身上,韩冈就不能容忍了。

“家师穷究性命天理,日渐日新。其学上承先圣道统,直探天人大道;下授弟子经义、治事之术。韩冈不得家师教诲,哪会有效命于天子的才能?!”

韩冈为张载而愤怒。可张载的名声,毕竟只局限于关中,在东京无高官名臣为其宣扬,韩冈一番话说出来后,多有人不以为然。

“天人大道?”杨绘呵呵一笑,“晋人确有言‘名教出于自然’,不过却逐渐沦于玄想,日后败坏名教,儒门沉沦百年,便是这等人的功劳,只盼张子厚不会重蹈覆辙!”

韩冈针锋相对:“格物之道讲究着以实证之,可与玄想全然不同。学士妄加评判,却是沦于臆测了。”

“臆测?!”

杨绘放声大笑。韩冈在天子面前要把张载塞进经义局,这件事,已经在重臣中传开来。杨绘本意是要看王安石的笑话,女婿造反的事并不常见。但现在,他倒不介意帮王安石一个忙。

笑声中,韩冈已是平静如初。半眯起的眼皮,遮住了双瞳中开始算计着人心的神采。他向来自控,方才的怒气只是短短一瞬,现在已经可以定下心想一想该怎么顺势而为了。

格物之说现在还局限在洛阳和横渠,并没有广泛的流传开来。趁此机会,韩冈倒也有心将之退而广之,进而帮张载铺平道路。而物理学最关键的就是以实证之,而寻常人却是对世界有着各种各样的错误认识,韩冈并不缺乏对付杨绘的手段。

只要把杨绘吊上钩,就足以让格物之说传于天下。

韩冈冷眼看着笑得放纵的杨绘,嘴角凝出一丝讥讽的笑意。就让他的名声……成为科学进步的第一个牺牲品好了!

