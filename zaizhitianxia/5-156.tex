\section{第17章 往来城府志不移(三)}

准备了近一个月,太原府的夏收终于开始了。

天上的太阳火辣辣得能将人烫伤,跟三伏天相比,也差不了多少了。不过这样的天气,不论是收割还是翻晒都是个好时候。

太原府的万顷麦田,在十天的时间内差不多都收割干净了。当金黄sè的小麦被摊开暴晒在阳光下,夏税的工作,也随之展开。

关系到一天的太原府衙门中的官员和胥吏们,都是忙得脚不沾地。不仅仅是各地州县上缴的籍簿,还有韩冈的吩咐,都让他们没空停住脚步。

“府库帐册怎么到现在还没做好?打算拖到什么时候?!”

韩冈在前院催促的声音都能传到后院中,等他回到后面,王旖就有几分不解的过来问着他:“官人怎么这么急?整理帐籍簿册,也不是一rì两rì就能完事的,好歹也要几个月。”

韩冈摇摇头:“不是为夫急。是我们很快要回京城了,这里的事当然得做个了结。将个烂摊子交给后任,我可丢不起这个人,而且有什么事可说不清楚。”

王旖疑惑的眨着眼:“官人怎么知道要回京了?是不是有什么消息。”

韩冈右手握拳,轻笑道,“消息倒没有,可若是调令不来,那为夫就再上几份奏章好了。神木寨的石炭,不止可以用来烧砖。完全可以跟保德军的铁矿配合上。若是保德军开铁场,规模和产量不会比京城和徐州小到哪里去。”

王旖惊讶得瞪圆了眼睛:“官人,上一次的奏章难道就是为了能让天子将你召回京城去?”

“还不知道有没有用。所以才要继续上奏开辟保德军铁场。”

听了那么多,王旖终于是明白了,韩冈是把握到了天子的心思,才敢这会这样行事。

不过韩冈提议在保德军设立铁场,可不仅仅是为了让天子担心自家再夺立功勋。河东需要一个稳定的钢铁来源,而太原府的许多富户都需要钢铁。保德军是眼下河东路能拿出来的最好的铁矿了,只要朝廷能同意设立保德铁场,规模肯定会在短时间内扩充起来。

保德军【今山西保德县】有铁矿,在过去是为河东边寨的弓弩院提供制造箭头的铁料,也是边州百姓铁器的来源。产铁量虽然少,但实际上的储量应该不会小。而且对比起如今全国上下总计也不超过十万吨的钢铁产量,以及河东的实际需求,再小储量的铁矿也是够用的。

府州缺铁,可朝廷对府州的供给只有打制成兵器的成品,而且枢密院每年遣人查验数量来,远比其他的地方要严格得多。折家想要铁料,只能去买铁锅铁铲。所以折家很希望,能有一个。保德军和府州就隔着一条黄河,如果保德军开办铁场,那么以折家在河东北方的影响力,暗中细水长流,一年弄到上万斤的铁料并不是什么大问题。

当然,韩冈这边只管提议建设铁场,是否能成功,韩冈并不能打包票,而且失败了也不打算再提。至于朝廷同意了韩冈的建议之后,折家怎么从保德军弄到钢铁,那是他们自家的事了。

比起折家,韩冈宁可去关心轨道的结构问题。

到底什么时候能将铁轨造出来?就是没有工字轨,仅仅一根铁条,都比现在用的硬木要好,河北轨道拖到现在,在韩冈看来,并不仅仅是因为对西夏的战争问题。

韩冈仰头看着被太阳映得发白的天空,“调令多半就在这两个月,若是回京的话,就要顶着这个太阳了。云娘和南娘又有了身子,家里面还有几个小的。这一路上的车马劳顿,累着了可不好。”

周南和韩云娘皆有身孕,韩冈都已经有六个儿子,而他现在年轻得很,想要追上九子八婿的郭子仪,看起来也不是那么难。如果周南和韩云娘两人最后生的都是儿子的话,等大一点,确定不会夭折了,就过继过去承宗祧,还了父母的心愿。

“还是到时候在说。”王旖说道,“若当真有诏令回京,肯定就有办法来解决。”

天气一天比一天热,收获和征税的工作渐渐地到了尾声,的确是超过往年三四成的大丰收。稻谷满仓囤的情形,出现在各县汇报中的次数,可不是一次两次。使得因为战事而亏空的常平仓,也因此而顺利的得到了补充。

在夏收和夏税征收的这一个月,太原府的粮价并没有大的波动。常平仓在稳定粮价上,起了关键的作用。

当太原府元丰三年的夏税征收终于有了一个圆满的结果,天子也那边也对韩冈的两封奏章有了想法,将他调离太原,调回京城的迹象也渐渐多了起来。

正如韩冈所希望的,他的新岗位在夏收结束的终于定下来了。不再担任太原知府和河东路经略使,而是回返京师,改判太常寺,兼提举太医局、厚生司。

这就是韩冈的新差事。

“三哥哥,太常寺是做什么的?”安排前来宣召的中使去了寅宾馆住下,刚刚回到后院的韩冈被韩云娘扯着袖子问道。

“问你南娘姐姐。她比我清楚得多。”韩冈笑着打发了韩云娘过去。

周南离开教坊多年,早年的伤痕已经淡忘得差不多了,听到韩冈的话,旧rì的教坊司花魁淡然一笑,摇头道:“太常寺中,奴家只知道一个教坊。其他相熟的衙门,如太乐局、鼓吹局,都归了太常礼院管。至于再多的,可就不是奴家能知道了。”

“就跟你南娘姐姐说的那样,判太常寺就是个没什么事的闲差,正经事都给太常礼院拿去了,不归太常寺管。”韩冈打了个哈欠,从严素心手里接过了一碗冰镇百合绿豆汤,“判了太常寺这下回了京城,倒是又能过上一阵清闲rì子了。”用勺子挖了一勺绿豆沙,至今嘴里,有点含糊说着:“倒还真是没想到。”

“官人不是事先就知道要回京城了吗?怎么还说想不到。”王旖笑着问。

“我可不知道会是这个差事。也不可能猜得到还能多了个殿学士的名号。”

太常寺是九寺之首,统管朝堂礼乐、祭祀,太医局也是其属。其主官太常卿,地位近于六部尚,高于宗正寺外的其余七寺,有‘尚里行’的说法。可依照如今的官制,六部九寺的职事早就被瓜分殆尽,太常寺也不例外。朝廷的礼乐制度和仪式,有太常礼院管辖。名义上礼院尚归太常寺辖下,实则专达于上,不受太常寺制约。而熙宁九年的时候,太医局也从太常寺中独立出来。

眼下太常寺剩下的基本上就是一个空壳子,只管着社稷二坛、武成五庙等京城的祭祀场所,以及一群荫补官中的斋郎、挽郎——祭祀上打下手,葬礼上执灵杖,这是他们的工作。此外,还有一个教坊,京城中的伶人,jì女都在太常寺的管辖之下。

不过礼乐是儒门的核心之一,故而太常寺这个职位清重而位尊,可以安置高官。说句难听话,就是养老的地方,跟韩冈之前的同群牧使一个类型。

但在差遣的改变之外,韩冈还得到了一个新的职名——

端明殿学士。

且原本的龙图阁学士的职名并没有被削去,也就是说,韩冈现在是端明殿学士兼龙图阁学士。

在品阶上,殿学士在阁学士之上。从今rì开始,若有人要用职名称呼韩冈,那就应该是端明而不是龙图了。

一般来说,端明殿学士是翰林学士资历较深者的加职,或是翰林学士非因罪离任后的赠予。司马光现如今正是端明殿学士,他是在翰林学士任上反对变法,最终出外任职,之后改授端明殿学士。而韩冈援引的是则王韶的旧例。当年王韶奉旨拓边河湟,在攻下河州之后,他的职名便成了端明殿学士兼龙图阁学士。

端明殿学士的授予,可以算是对韩冈的补偿了,不然从太原知府兼河东经略使的任上,转任判太常寺,其中贬责的味道太重,并非是待遇功臣之法。

不过赵顼也是顺便通过这项任命,更加明确的表明他并不打算让韩冈拥有与地位相当的差遣。否则端明殿学士就应该改成翰林学士。

以韩冈如今在儒林的声名,也当得起翰林学士的头衔。只要不加知制诰的名号,也不用担心在起草诏一事上会丢人现眼。可赵顼却宁可用更髙一级的职名来安置韩冈,也不让他有机会插手参与朝政。这样一来,当然就没有机会走进两府之中。

但换个角度来想,天子这是在用更高的官位、职名、勋阶来安抚韩冈。

韩冈很明白,换做普通的臣子,天子若是看不顺眼,或是想磨练一番,直接打发到远僻州郡去好了。也就是向世间传授牛痘的自己,能让天子不得不投鼠忌器。

这不仅仅是功劳立得太多的缘故,提举太医局和厚生司的兼差说明了一切。

‘正合我意。’韩冈想着。

