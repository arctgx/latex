\section{第十章 千秋邈矣变新腔(七)}

只闻国家缺贤,未闻朝廷缺官。

韩冈言辞尖刻,却自有其道理。

“冗官、冗兵、冗费,三冗之患,从仁宗时就开始说,可至今仍未能得到解决。尤其是冗官,虽愚暗鄙猥人莫齿之,而三年一迁,坐至卿丞郎者,历历皆是。崇文院本是待贤之地,天子储才之所,但如今贤者不得其任,颟顸愚顽之辈却充斥其间,究其因,还是冗官为患。”

“蹇周辅为官,所任多有建树。先帝亦曾赞其‘精敏可属事’。”

“不过为一李逢案尔。”

韩冈不屑一顾。王安石当年因为李士宁那个假道士,差点被这桩案子给牵扯进去,现在却拿着这桩案子来为蹇周辅张目。

他看了一下屏风,他相信向太后不会记不得前两年弄得朝野沸腾、却牵强无比的那桩太祖子孙谋反案。不过他再看看王安石,老泰山却在发怔,该不会只知道这句评价,却不知道其来由吧?不过以蹇周辅与王安石之间地位的差距,王安石能记得这个人,估计也就是一两句的评价,和几桩事例。在细节上,不可能比得上有所准备的自己。

“赵世居、李逢皆是手无缚鸡之力,纵有心做反,三五内侍,便能将其生擒。先帝只不过是心知患在萧墙之内,却有顾虑不能发作,只能以赵世居、李逢作伐,以震慑贼子不轨之心。”

韩冈话中指的是谁,自不用多说。其实当年赵世居、李逢谋反案,也不过李逢的一些言辞戳到了赵顼的痛处,天子恼羞成怒故而大办。但如今正好能够前后呼应,却说得通。

向太后深有感触,点头道,“参政说得是。”

尽管当初她的丈夫到底是为了什么理由才大开杀戒,向太后并不知道。但她还记得,那一阵子,以及之后的一段时间,入宫来的宗室妻女无不拘谨了许多,平日能说几个笑话的,都噤口不言,唯恐行差步错。赵世居、李逢的这桩案子,的确有震慑宗室的作用。

韩冈紧接着说下去:“而蹇周辅奉旨断案,只是在希合上意,故而事后才会有‘精敏可属事’之语。此辈安可称贤?”

王安石一时沉默,让韩冈确认了自己的猜测。王安石或许了解蹇周辅,但他并不了解当初蹇周辅是因何得到这个评价——当时的王安石,韩冈记得他还是在金陵。

正所谓知己知彼,百战不殆。蹇周辅不过是年纪大了急着卖身,王安石为了党争,听了一面之词就匆匆赶来,又怎么能够与有所准备的自己相比?

“更何况,黄裳在河东所立功勋,蹇周辅又如何能比得上?难道先帝对逢迎之辈随口一句称赞,比不上切切实实的军功?”韩冈几句反问,随即又‘啊’的一声叫,“对了,蹇周辅亦曾招降廖恩。昔年廖恩领数十盗贼为患福建,州郡不能制,蹇周辅受命为福建转运副使,出面招降了廖恩。”

韩冈边说,边用眼角盯着王安石的反应。不过分心归分心,嘴上吐字的速度却一点不慢,不给王安石接口反驳,

“但廖恩降伏,乃是闻说王中正已领兵南下,畏其宿将威名,故而王中正领天兵一到,便立刻拿着蹇周辅颁出的招降文书来投降了。蹇周辅能招致其降顺,不过是狐假虎威。试想周辅不过区区一文士,素无声威,更无军功,如何能让扰乱一路的巨寇闻风丧胆?还不是因为廖恩害怕刚刚平了茂州的王中正,想见好就收,若蹇周辅当真有才干,何不为民除此獠,反倒招安其人?现如今,福建倒是在传唱,要做官,杀人放火受招安,使贼人不畏王法,正是蹇周辅所致!且南兵本不习战,故而让廖恩得以逞凶,换作是在北方,县尉领十几二十土兵弓手便可将其生擒。数十盗匪为患,比得了入寇河东的北虏大军?”

韩冈的话如同连珠炮一般,王安石几乎给他气得发晕。

王安石瞪着自家的女婿,不说自己不知道廖恩之事,就是知道,他再糊涂,也不会拿着南方盗匪与辽国大军相提并论。偏偏这个好女婿将这话栽到自己的头上,一句紧接着一句,丝毫不给插话的机会,直到将这桶脏水泼完为止,这才停了下来。

王安石用深呼吸压下来心中的愤怒,冷声反驳:“论功业,黄裳对外,蹇周辅在内,内外虽有别,却同为天子效力,各自竭尽全力,如何分高下?论行迹,黄裳是辅佐之劳,蹇周辅却是独任之功,黄裳又岂能说是在蹇周辅之上?何况今日又是在说何事?能否通过制科,若是以功业论高下,又何须考试?黄裳过去的功劳,朝廷又难道没有赏赐?”

如果是在才学有一定水平的先帝赵顼面前,王安石完全可以引经据典,当初他就是这样凭借对经史的熟悉说服了赵顼。但面对韩冈和太后——尤其是太后——时,一些引经据典的手法,完全排不上用场。向太后的水准只比寻常妇人好一点,韩冈与人辩论则更是多用事实说话——其实从这一点中,完全可以看得出韩冈对经典的态度,不屑一顾。

不过王安石也是会学习的,同样不给韩冈反驳的机会,“黄裳的功劳,朝廷赏赐了。黄裳的才识,朝廷也承认了。得官不过三载便为太常博士,是靠磨勘而来?其进士出身,又是哪一科考出来的?朝廷与太后待黄裳不薄,如今难道还要因为已赏之功,再给他一个制科出身不成?黄裳考的是制科,而蹇周辅正是考官,如何判,蹇周辅说了算。礼部试的结果,就是天子,也更动不得,阁试的结果,参知政事也罢、平章军国也罢,也都更改不了。蹇周辅是尽其职守,有功无过!”

一口气说下来,王安石已经开始喘气了,他的年纪摆在那里,远不如韩冈有长力。

见王安石一口气接不上来,韩冈便自自然然的接了过去:“方才臣也说了,此事只能将错就错。黄裳纵使受了委屈,这件事上,也必须维护朝廷的威信。这是臣的意见,想必黄裳也能体谅。若王平章忘记了……”韩冈转过去面对王安石,“那韩冈还可以再重申一遍,事关朝廷威信,黄裳被黜落这件事,不可改易!”

韩冈再一次重复他的观点,并不是为了黄裳被黜落,而是针对考题上的错误。这让向太后看在眼里,怎么看也比王安石一心偏袒蹇周辅的态度要强。

“但蹇周辅等人无知,制科上用错考题,难道不该问罪?”韩冈对蹇周辅紧咬不放,“若要说只有通过阁试,才能算得上是军谋宏远材任边寄,臣无话可说。但臣可以明说,蹇周辅所出的那些题目,臣最多也只能做出其中一半,肯定过不了阁试。若蹇周辅没错,那臣便是眼光短浅不堪任边寄了?臣是否得将历年来出典边郡所受封赠都还回去?”

“封赠因功而来,又不是看出身!”王安石一声冷喝,“韩冈你贵为参知政事,怎可将朝廷封赠当成儿戏?须知制科为大科,待遇犹在进士之上。想要得到制科出身,又怎么能不经更加严苛的考试?黄裳想做边臣简单,也不需要制科出身,他已经得太后赐予进士出身,又已是太常博士,完全可以去边郡任知县,若其间有功于国,晋升之速,又岂在制科出身之下?”

“平章弄错了,黄裳的考试不是严苛,而是错误吧。”韩冈根本不理会王安石的问题,抓住其中一点来回答,“凭蹇周辅所出六题,能找出一个边臣来。朝廷为何要将进士科与明法科分别考试?不正是因为对臣子的要求不同,题目必须不同的缘故?”

“明法科出身,在进用上远比进士科要低。而朝廷给军谋宏远材任边寄这一科的待遇,可是比贤良方正能直言极谏和才识兼茂明于体用两科要低?”

“既然阁试题目都一样,那制科为何要分作十科,何不作一科来考?”

“只为刷去才识不足、滥竽充数之辈。到了御试中,自会分科来考。就如礼部试,亦是刷落才识不足之辈。黄裳若是才学兼优,必不致于累科不中。”

“韩冈倒记得蹇周辅也是累科不中呢,倒是熬进了崇文院。”韩冈刺了王安石一下,又道,“不知在平章看来何为才识?明经义?还是能治事?如曾孝宽、吕嘉问之辈,何时中过进士?而阁试中的四位考官,也不是都是进士出身,赵彦若便是荫补。敢问他们的才识如何?”

赵彦若以明史著称于朝,也是因此被选入三馆秘阁,但他的确不是进士,而是荫补出身。曾孝宽、吕嘉问就更不必说了。

王安石眼神如同数九寒天的河水,在冻结的冰面下亦是一片冰寒,韩冈果然是在针对这几名考官,早就有所准备。蹇周辅,赵彦若,他们的底细韩冈一清二楚。王安石都不了解,韩冈却了如指掌,除了他早有预谋,哪里还有别的解释。
