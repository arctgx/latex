\section{第34章 山云迢递若有闻(五)}

这几天,发运来渭源的粮秣和军资有些乱。数量并不短少,但物品清单的书写,明显跟韩冈之前制定的规范完全不同。照定例,所有的物资清单都必须经过随军转运使的签押,蔡曚也没有放弃这个权力。在韩冈离开陇西之后,下面的官吏也夺不走这项权力

可蔡曚把韩冈已经确定的成法丢在一边,随性书写,清单上一点条理都没有。让韩冈接收时,计点起来很是头疼。虽然下面有人说蔡曚是故意所为,但韩冈觉得,能把一张出库单都弄出问题来,这纯粹是蔡曚的自尊心高过他的能力。

如今俊杰才士遍地,可官场上总有人不能胜任的情况,蔡曚怕就是其中的一例。韩冈现在好像有些了解为什么文彦博要把蔡曚给塞进来了。

‘还是让他早点去临洮报道吧。’韩冈捏着鼻梁,希望这套旧时学过的技法,能让他酸痛的双眼恢复清明。

王韶说要拿蔡曚试刀,可他终究还是不敢拿前线将士的肚皮冒风险。故而还是下令让蔡曚前往临洮城主持转运事务,干脆放在眼皮底下监视起来。

韩冈支持这道命令。如果蔡曚抗命不去,便可以直接办了他。如果他听命去了临洮,韩冈就正好能够整个后勤方面事务,而不是跟蔡曚分段包干,让自己为他拾遗补缺。

秦州征调的三千民伕再过两日就该到陇西了。又多了几千张嘴,可没时间再为蔡曚擦屁股了……

“机宜!出事了!”一名信使大声叫着,冲进了韩冈占据的堡中中厅,“昨天出发的那一队在白石山下面被伏击了!”

韩冈脸上一下褪去了血色,但他仍尽力保持震惊,训斥着,“慌什么,不过一队而已,想乱了军心吗?!”

信使一听,连忙跪下请罪。

韩冈瞪了他两眼,这才问道:“究竟损失了多少?!”

“伤了二十余人,死了九个。军资大约损失了两成多。”

“……还好!还好!”韩冈放下心来,靠上了交椅椅背。手压了压心口,这一惊一乍的,心脏都有些吃不消。所有发往前线的粮草,都是有一定冗余度的,并不可能将将好就是前线大军日常需要的那么多。眼下损失的四分之一,还在韩冈承受范围之内。

“贼人多少,又杀了他们几人?”心情稍稍放松下来,韩冈又问道。

“总共五百多贼人,被杀了一百多个。。”

‘扯淡!’韩冈差点没骂出声,他屈指用力一扣桌子,怒声道:“要是能杀了一百多名贼人,还能损失那么多粮秣?真当蕃人的胆子都是铁打的不成?!死战不退的,有这能耐,怎么把临洮城都丢了?……把斩首数报给我!”

信使不敢再夸大,小心翼翼地回话:“实打实的是六个。里面有两个本是活捉的,不过伤重死了。”

‘这还差不多。’韩冈点了点头。

六个对九个,在被偷袭的情况下,这兵力损失的交换比不能说是吃亏。偷袭辎重队的吐蕃人是主动退出战场,从情理上说他们应该还有一些阵亡。

“伏击辎重队的是木征的兵,还是瞎吴叱的兵?”

“都不是。”信使摇摇头:“被俘获的贼人说是禹臧家。”

“禹臧花麻?!”

韩冈皱起眉头,这还真是出乎意料。禹臧家是西夏的臣子,他替瞎吴叱出头,是受了兴庆府的命令,还是延续去年渭源之战时的默契?

韩冈一时想不通。不过不管是谁出手,这次辎重队被伏击,代表着通往临洮的粮道不再安全。吐蕃人随时可能会再来,可能是禹臧花麻,也有可能是木征兄弟。

有句成语叫做食髓知味,吐蕃人占了一个便宜,总不会就此跑掉,洗手不干的。老虎一旦吃过人后,也都会把人放进菜单中。吐蕃肯定会再来阻断粮道。韩冈想了一阵,他现在能做的,就是把运输队的规模尽量扩大,并加派护卫。

吐蕃人不可能排出比五百骑规模更大的队伍,不然就会被临洮城的官军给缀上,如果有相应的准备和足够的兵力,足以让他们无功而返。可是如果吐蕃人改弦更张,改成小股的骚扰就让人很头疼了。

要尽早将吐蕃扫平,便必须克服军粮补给上的困难,任何可能的问题都要考虑到。

韩冈走到沙盘前,默默思忖着。有关武胜军的地形沙盘早就制作完毕,尽管比较粗浅,也足以用来制订作战方案,以及武胜军防御体系的规划。

在规划中,不但临洮城要增筑,还要在临洮南北各修南堡、北堡,堵住临洮城所在的这一段洮水河谷。比起单独一座城池,完整的防御体系更为关键。

只是这就需要大量的民伕,可若在民伕的转移过程中,被蕃贼突袭,多半要出乱子。

‘看来得下决心了。’

王厚这时听到消息,匆匆赶来。正看着韩冈对着沙盘喃喃念叨着。

“从渭源堡到庆平堡,也就是鸟鼠山这一段,还是比较安全的。”一支长木棍在沙盘上晃动,韩冈低声自语,“但再往西去,一直通到临洮的剩下的七十里路,间途岔道众多,让人防不胜防……”

盯着沙盘,韩冈咬着下唇,过了不知多久,最终有了决断。

兵站!还是要设立兵站!

辎重队在兵站和兵站之间运输,各家分管一段。再在沿途的几处战略要地设立寨堡,护翼辎重队。从渭源堡到临洮城,一百多里粮道,以野人关、庆平堡为核心设立兵站,将粮道分作三段。三四十里一处兵站,正常情况下,半天便能走完,就算被贼人骚扰,也能保证在日落前抵达下一个兵站。

而且设立兵站后,每支辎重队都只负责二三十里的行程。在兵站做交接。这样他们能熟悉起道路,了解哪一个地点有危险,那一段安全。不过会一直紧张着,变得容易疲劳。

韩冈此前一直不能下定决心,因为兵站制度,需要驻防的兵力要比正常情况多上不少,以便保护凭空多出来的几处军需要地。但现在看来,还是势在必行。

“先将粮道稳下来。”韩冈回头对王厚说着,他还是注意到了王厚的到来,“请处道兄去临洮代小弟向机宜面禀,在野人关、庆平堡两处,各屯一个指挥的骑兵和五百步卒。渭源堡的这里也会把负责辎重转运的民伕,分到野人关和庆平堡两处,让他们各负责一段转运。虽然行程上要慢上一天,但安全性能提高不少。”

韩冈决心把这条粮道变成一个难以下嘴的刺猬,不论蕃人来的是大队小队,都别想在这里占上半点便宜。

“那渭源堡怎么办?”王厚也清楚,渭源堡现在精锐尽去了临洮,再分走了大半广锐军出身的民伕,可就过于空虚了。

“不用担心。”韩冈的视线放回沙盘上,“渭源堡安全得很。”

……………………

“从这里转过抹邦山,可以直通渭源。”结吴延征举着马鞭,遥遥指着南方。

瞎吴叱顺着马鞭的方向看过去,没精打采的说着,“这事谁不知道?”

结吴延征是木征的弟弟,也是瞎吴叱的弟弟,继承了瞎吴叱在岷州北部的地盘。今次听说宋人攻打武胜军,便连忙带兵来救援——武胜军一失,他跟河州的联络就要断了大半。

只是当结吴延征赶来的时候,王韶都已进了临洮城。就只看到一个灰心丧意的瞎吴叱。他看了看颓丧的兄长,冷笑道:“宋人对此好像并不知道。”

来往于西域和大宋的商队习惯了穿越鸟鼠山,但这并不意味着临洮和渭源之间没有其他的通路。除了偏北侧的鸟鼠山之外,还有一条南线,道路没有鸟鼠山这般崎岖,让车辆难行,只不过要向南绕个圈子。【注1】

对于以马和骆驼为脚力的商队来说,不能让车走的路,并不代表不能让马队走。反而多出来的七八十里路,让商队都失去了兴趣。这条南线,要通过抹邦山,而现在攻下了临洮的宋军,还没有足够的兵力占据这片地域。

瞎吴叱眼神终于变得锐利起来,听着弟弟继续解说:“据哨探回报,宋人现今护翼粮道的军队,都是放在野人关和大来谷中,他们在渭源城完全没有防备。……因为没人能冲得破临洮、野人关和庆平堡这三道防线。所以宋人变得自高自大,根本不去防备我们的反击。禹臧花麻虽是狡诈无比,居心叵测,但他也跟三哥你约好出了兵。听说他已经派人去阻截宋人的辎重队,让宋人把大军派出去守护粮道。那时候……”

“我们便可以去偷袭渭源堡!”瞎吴叱兴奋起来,“宋人能偷袭下野人关,我们也能仿其故智,去把渭源堡打下来!”

结吴延征厉声狠笑,“到时候看看王韶还能不能在临洮城中安坐!”

注1:这是如今的316国道路线。

