\section{第32章 吴钩终用笑冯唐(十)}

“三哥儿又立功了?”韩阿李立刻兴奋地问道。

“三表弟很有名,在军中。在长安。还有在京城也是。听说在罗兀城。救了不少人。满驿馆都听到有人说他。”李信就算做了官,还是不善言辞,说起话来也是一句一句慢悠悠的,韩阿李听着开心,却也心急。

等着李信终于说完,韩阿李转头就吩咐韩千六,“明天去找厚哥儿问一问。三哥儿立了功,从罗兀城回延州了,衙门里应该也能收到消息。”

前段时间,听说了韩冈被调去陕西宣抚司。韩阿李隔三差五就让人打听鄜延那里的消息,一段时间下来后,倒把罗兀城、绥德城这些地名说得琅琅上口,熟得不能再熟。

再三叮嘱过丈夫,韩阿李就又半是开心,半是感叹的说着:“三哥儿是越来越了不得了,过去怎么都想不到……”

冯从义笑道:“是啊,前次有个商人从京中来。一说起三表哥,就翘大拇哥,说是敢跟亲王争风,最后还惊动了官家来成全,立国以来还是头一遭。”

韩阿李听得兴致更高:“官家圣明,明断是非,所以能做天子!”言下之意就是跟儿子争花魁的赵颢,便只能当个破落亲王。

韩千六的胆子不如他浑家,叹着气:“只盼三哥不要给什么花魁迷昏了头,把家里的事都给忘了。”

前些天李小六带了韩冈的口信回来,从他嘴里听说京城里发生的那些事事。抢了亲王看上的花魁,让天子下诏成全,韩千六老实了一辈子,过去只觉得自己的儿子越来越有能耐,可现在却是越来越让他心惊胆跳起来。

“家里的云娘、素心,哪个不是一等一的人才,偏偏去京里还招惹什么花魁?”韩千六唉声叹气着,过去他见个班头就要心惊胆战,现在靠着儿子的关系,遇上太后的叔叔也能说几个笑话;他种了一辈子菜地,如今靠着农事上的本事,管着千百顷官地,也算是扬眉吐气了;可儿子偏偏跟亲王抢起了女人,想想韩千六的脑袋就要一阵发昏,“今天得罪的亲王,那可是太后的嫡亲儿子,官家的亲弟弟,这日后该怎么得了?”

“怎么了?怕什么?”韩阿李冷眼瞧过去,“三哥儿就是这么本事!人品、人才、相貌,哪样不好?人家周小娘子放着好好的亲王不要,为三哥守节,多难得的女孩儿家?小六回来都说,东京城上上下下都是说三哥的好,雍王的不是,惹得官家都要下旨成全,你这韩菜园还怕个什么?!”

韩千六争辩着:“俺是担心……”

“担心什么?!”韩阿李回头往堂屋后面看了一眼,明白了,“要是三哥敢偏心,我是不饶他。但三哥也不是负心的人,你瞎担心个什么?!”

韩阿李一阵抢白,韩千六被堵得说不出话来。多少年夫妻都是这样,他也不生气,端起茶喝着,不说话了。

韩阿李又道:“三哥年纪小,风流点没什么,就是给韩家早点添个后才是真的。你们说是不是啊?……”她冲着后面喊了一声。过了一阵,韩云娘和严素心就脸红红的端了待客茶汤、菓子出来。李信、冯从义都是自家的至亲,她们女眷也不用避。只是方才在外面听着说起韩冈找的花魁,不便出来,只好等在门后面。

上了茶,严素心和韩云娘又躲回到后院的厨房去。靠着门框,韩云娘幽幽的问着严素心,“素心姐姐,三哥哥会不会忘了我们……”小脸上有着夜色投下的忧愁,“是东京城里的花魁啊……我们怎么比得上?”

“周家妹妹的长相和性子,你不是问了小六多少次了。怎么还担心?”

严素心笑了笑,但笑容有些勉强。韩云娘是从小在韩家长大,再如何都是韩冈身边最亲近的人,但自己就不一样了,想到这里,她一时心乱如麻,乱哄哄的就像锅中滚水,混乱的思绪浮起又沉下,也是幽幽一叹,“不知官人什么时候才能回来……”

……………………

韩冈此时却是在工匠营中。

才一天的功夫,工匠营的作头何忠,就带着他的手下把韩冈说的新型投石车拿了出来。速度这般快,自然不会是从新打造,只是把旧的行砲车改造而已。去掉了绳索,改钉上一个斗框,在里面装上石头。

何忠向韩冈和游师雄介绍着:“这是七稍砲所改,如果是用人手来抛石,二十斤重的石弹能抛到六十步外。”

投石车上的抛竿,一般都称之为‘稍’,但为了在抛竿的柔韧性和坚固度中取得平衡,抛竿一般都是用几条木杆合并起来,一条杆称为一稍,有三稍、有五稍,最多的便是七稍。

“试过没有?”游师雄问着。

“没试过哪敢请官人过来查验?”何忠憨憨笑了笑,“已经试过了好几次。”又一指砲车所对方向,“诺,石弹还在那里!”

游师雄望了过去,才三十步到四十步的距离,“好像近了点?!”他犹疑的问着。

“官人放心,这只是试砲而已。”何忠说着:“旧的行砲车并不合用,肯定是要重新打造。现在只是先试一试这种方法成不成!”

“现在再试一试。”韩冈急着看成果,催着何忠来。

何忠一声令下,七八个工匠一起忙碌了起来。他们的动作很快,拉下抛竿,向竿后的网兜中放入球形的石弹。转眼之间,被拉下来的抛竿向上一翘,石弹从网兜中被抛出后,在半空中划过了一道弧线,砰的一声响,落到了四十多步外的地方,向前滚动了十几步后,停了下来。

“还是不算远!”游师雄摇着头,四十步别说跟八牛弩比,就是神臂弓也比不上,根本就是普通弓箭的射程,但他更吃惊于这投石车的简单易用,过去的七稍的行砲车,好歹也要七八十人服侍,“这人手用得实在是少!”

“少多了!”何忠强调道,又说道,“石弹抛得近,是因为前面斗框轻。斗框里放进去石块的越多,石弹飞出去的距离就越远,放得少,自然就抛得近。”

“怎么不多放一点?”游师雄连忙追问。

“这斗框吃不住。”何忠他拍了拍身边的投石车,“等过两日,新的行砲车打造出来后,将前面的斗框跟抛竿榫合在一起,就可以多装些石块进去,肯定能抛得更远,六十步绝对没问题。”

“那就好!”听了何忠的解释,游师雄释然了。

韩冈对何忠的工作也很满意,赞了两句后,对游师雄道,“其实确定了框子内石块的重量,以及石弹的重量后,再结合起抛竿两臂的长短,最后能将石弹投出多远,那是可以通过算式计算出来的。只要有了算式,想把石弹投到哪里,就能把石弹投到哪里。”

游师雄问道:“还是玉昆你‘以数达理’的说法?”

韩冈点着头:“君子六艺,礼、乐、射、御、书、数,‘数’能名列六艺,岂是只用来计算钱谷的?天文地理何处用不到一个数字。圣人之为,自有深意。雪花六出,桃花五瓣,总是有其缘由。大者如日月东升西落,千年不变,万载不移,必有其理蕴于其中,所以日月之食,钦天监便能计算得出。小处就如这行砲车,也是有其道理的,亦可计算得来。”

韩冈转过头来问着工匠营的作头:“何忠,你在工匠营中有不少年了吧?昨天我说的话,不知在工匠营里有没有地方能用得上?”

“小人在工匠营里做事已经有三十多年了。”何忠对韩冈崇敬不已,都把他当作了鲁班一般的人物来看待:“可韩官人说的道理,我们干了一辈子的工匠都没有想通。但昨日只是听了韩官人一番话,却一下都明白了。谁能想到一根撬棍都有这么多道理?天天都见识着,就是没去深思。唉……所以小人只能做个工匠,官人才是官人。”

“听了一句便能领悟,足见何忠你其实早已把握到了其中的精妙。有句话叫做技近乎道。一门技艺到了极处,也便能看到大道了。何忠你做了几十年的工匠,道理早已存在你心中,只是你没有察觉,仅是一层窗户纸没有捅破而已。”

游师雄听着觉得韩冈的比喻挺新鲜,笑问道:“今次是捅破了窗户纸?”

韩冈转过来问何忠:“何忠,你觉得呢?”

何忠用力的点头。

三天后,何忠带着一众手下,日以继夜,终于打造出了第一具新型的投石车。在斗框中填满了砖石,试砲时一砲将二十斤的石弹砸到了七十步外。按照何忠的说法,如果给他更多的时间,更好的木料,再用精铁打造出其中几处关键部件,他完全可以造出将五十斤的石弹投出百步以上的砲车来。

已经回到了泾阳帅府行辕,韩绛还是在几个时辰后就收到了新型砲车成功的消息,放下笔,由衷的感慨着:‘这个韩玉昆的确是不简单。’

