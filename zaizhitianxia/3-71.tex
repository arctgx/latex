\section{第25章 闲来居乡里(四)}

为了避开四更天就起来上路,在午时之前,赶到了三十里铺,离着陇西城,也就剩三十里地。

看着天上的炎炎烈日,不论是一马当先的冯从义,还是成轩、刘广汉等几名来自秦州几大商号的主事者,都决定在这里歇上两个时辰。

铺,是军中驿传歇脚的地方。因为不是正经的驿站,不能换马,所以只有步递的铺兵才会在此停留。

三十里铺仅仅是一个挡风遮雨的棚子,商人们进来后,连着护卫二十多人,将这件长条棚子挤得满满当当。看着挤得不像样,护卫们自觉的都蹲到树荫下,将棚子留给主人。

十几辆车,几十匹马停在铺外,冯从义正在太阳下吩咐着下人,好生照管马匹。

坐在荫凉处,看着冯从义在外面忙碌。刘记的少东家刘广汉用力的摇着折扇,额头上的汗水涔涔往下淌着,低声抱怨着:“上次那一位纳妾,我们眼巴巴的上门送礼,都没带见一面。现在一句话,又要屁颠颠的跑过去。照我说,还不如另起山头!”

坐在身边的富态中年成轩,是怡和号的大掌柜,他摇着头,知道刘广汉只是在图个嘴皮子痛快。不过看在两家的关系上,还是低声劝道:“少说两句吧。脱不开的,也不看看韩家在陇西的势力。”

怡和号和刘记两家都是秦州的大商号,身后的家族也是代代有人做官,互相之间还有着姻亲。关系走得近,说起话来也没有太多的顾忌。

“韩家在陇西扎根才三年吧……”

“一年也一样,广锐军那群叛贼,还有青唐部的蕃人,哪一家他说的话没有份量?”

“广锐军叛贼倒罢了,哪有蕃人用钱买不过来的?”

成轩摇着头,他知道刘广汉是在嘴硬,蕃人最是难打交道的,一句话说不好就翻脸了。广锐军要承韩冈的人情,难道蕃人就不要?!这几年吐蕃贵人生了病不都是往疗养院里送,那是救命的恩德。若是哪家商行得罪了韩冈,他的一句话,就能让那一家的商队在蕃区寸步难行。

“别忘了,棉花采摘时耗用人手最多,没人支持根本拿地里的棉花没办法,更别说,大部分棉田都在韩家手上。而且就算有办法将棉花收上来,要是库房里失火出事又怎么办?你以为他不敢下黑手吗?”

得了提醒,刘广汉想起韩冈的那个让人畏惧的匪号,却仍是不服气,“难道就顺丰行说什么,我们就做什么?!”

“所以要去看一看。”成轩坐直了身子,望着西面,“看看韩官人的心胸如何,太贪心的人可都走不远。自己吃着肉,也得明白骨头要留给身边的人。若是连口汤都不分,哪个会跟着他?日后也不会有前途的。”

“仅仅是啃骨头喝汤吗?”

“若能细水长流,少赚一点也无所谓,银山哪如银水?”成轩笑道:“先慢慢来,时间长得很,谁也不知道几年后会有什么事。”

冯从义这时安顿好外面,走进来了。瞥了一眼坐在一角低声交谈的成轩和刘广汉,再看看其他几家商行的主事者。这一次棉纺上的谈判,几家都各自有着心思。只是最关键的种植和采摘,大部分都控制在自家手里,甚至是纺纱也是一样,实在不行甚至可以直接换个合作对象。要不是自家的三表哥想要早一步将棉布推广出去,就根本没有这些商行的机会。

歇了两个时辰,一群人东拉西扯的聊着天。看着日影西移,阳光也不再那般炽烈,准备上路继续行程。却听着东面的一片蹄声过来,几家商行的护卫们立刻紧张起来,纷纷拿起了朴刀和杆棒。

只是当一队吐蕃骑手来到近前,却都放心了下来。马身上拴着的一只只兔子、狐狸和山鸡,还有一头豹子被绑在一匹无人骑乘的空马上。还有两名鹰隼站在骑手肩膊上左右顾盼。就知道,这是一队打猎归来的队伍。

既然不干自家事,便都放松了下来。可这一队骑手越过三十里铺时,却停了马。只见领头的骑手拨马回头,操着口音浓重的官话:“这不是顺丰行的冯东主吗?!”

说话的人二十多岁,身高肩宽,有几分英武之气。冯从义一见,便连忙上前,用着吐蕃话跟他交谈起来——当初韩冈将与蕃部的交涉工作丢给冯从义之后,他只用了两个月就学的字正腔圆,一点都不带磕巴。

说了一阵,冯从义回身让伴当从车上捧了两匹上品的绸缎来,而那名骑手则将那头豹子作了回礼,学着汉人的礼仪拱了拱手,然后重新上路,一阵风的跑远。

冯从义让人将豹子抬上车,回来对好奇的众人道:“那位是阿里骨,湟州董毡的儿子,如今正在蕃学中。”

“是便宜儿子吧?”刘广汉笑道,又望望渐低的尘烟,眯起眼,“这一人,必要时可是能派得上大用的。”

便宜儿子也是儿子,董毡亲生的二子皆年幼,如果有外力扶持。阿里骨也可以坐上吐蕃赞普的位置。该怎么做,就要视情况而定。不过这件事所有人都知道,得意洋洋的说出来,可不是什么聪明的举动。

在晚间的时候,赶在城门落锁之前,冯从义一行人终于进了陇西城。冯从义并不是直接到韩府,而是将他们带到自己家中安顿了下来。

一番梳洗之后,让管家好好招待客人,冯从义先一步去韩家拜见姨父姨妈,当然更重要的是要见韩冈。

韩家现在一团喜气。韩冈的大女儿已经能开口说话了,正含含糊糊的叫着爹娘。

韩冈抱着女儿,哄着她不停叫自己,笑容中一点也不见在官场让人畏惧的锋锐。白居易六个月能识之无。不过那是少有的特例。十个月的时候,能开口说话,已经很不错了。

“这一去东京,可真够长的。金娘都会说话了。”

冯从义从怀里掏出了两个佛像吊坠。来自于和田的羊脂白玉,被京城的名匠雕凿得精致无比。小指指节大小的吊坠,连下面的莲花座上的莲瓣都一片片的清晰可辨。

周南生的女儿长得玉雪可爱,眼睛乌溜溜的看着冯从义掏出来的小玉佛像。而素心生的韩家长子却是老老实实的,不哭不闹,在一边睡觉。

等到冯从义跟父母行礼问安之后,韩冈引着表弟到了书房。

坐下来寒暄了两句,韩冈便直接问道:“设立棉布行会的想法,他们是否都支持?”

贩牛的有牛行,贩马的有马行,卖肉的有肉行,甚至收粪的都有粪行,三百六十行,行行有行会,只要做着生意,都要归属于一家行会。每一家行会,基本上都控制着一个州,甚至周围几个州的商贸往来,而各行各业中最大的行会,全都是在东京城中。

这些行会不仅仅是掌控着东京街面上的店铺,许多时候都控制着整条产业链。从生产,到运输,再到销售,都是融为一体。比如布行,从蚕茧收购,缫丝、纺织、印染,等各个作坊,都是紧密联系在一起,互相之间的关系是盘根错节。

虽然东京城中把持商业流通的行首们被市易务强力打压,靠着行政手段夺取了流通渠道的控制权,但行会的势力依然广大。来自于陇西的棉布,只能在东京城的布匹铺中少量销售,想要扩大销售范围,不但难以得到布行行首们的支持,还会因为占据旧有的上品绸缎的市场空间,而受到布行的压制,这一点其实已经得到证实。

东京是天下中心,流行的风潮都从东京向全国扩散。如果不能得到东京的市场,就没办法辐射向全国。东京布行靠着这个优势,要将手插进棉花的种植和纺织上来。这是韩冈所不能答应的。要打破这条产业链对布匹市场的控制,只有独立出来,自成一套体系。

只是冯从义从东京回来,几番考量之后,有了另一个想法:“其实吉贝布,是黎人对棉布的称呼,只有来自琼崖的棉布,才能称为吉贝布。以小弟的想法,不如将棉布说成是吐蕃人的特产,设立专营蕃货的行会,与旧有的布行不冲突。”

“和气生财吗?”韩冈笑道,看破了冯从义的心思。

他原来准备甩开布行,自行其是设立棉布行会,与旧有的布行打擂台的用意很是明显。冯从义要将换成了蕃货行会,其实就是要缓和这个矛盾。可尽管披在外面的皮可以换,本质上的利益之争却不会改变。

“但有用吗?”韩冈问道。

“至少不会显得太针锋相对,如果这样对付我们,他们那边算是理亏。”冯从义对此考虑了很多:“而且还可以将其他蕃货一起包括进来,一起挂着蕃人的牌子,也会省去许多麻烦。”

世人都知道蕃人难以打交道,就算看上了其中的利润,会起意抢夺的也不会太多,的确能省去一些麻烦。

“那好,就按你说的办。”韩冈点头。冯从义能有自己的看法,而不是一味的听从,这是他所乐意见到的。只要自己提出要求,就能给出回答,这才是合格的部下。

见到自己的意见终于得到了韩冈的首肯,冯从义很是高兴。停了停,又问道:“……三哥,要不要拨冗见一下他们?”

“不见!”韩冈一口否决。不会见他们这些商人。结交溷类,对自己的名声有损无益。通过冯从义作为中间人,才是正确的做法。讨论行会之事,让冯从义去处理就够了,讨价还价的事,自己没必要掺和进去。

“让他们去看棉田,已经安排人招待他们了。”

