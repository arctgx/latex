\section{第10章 弹铗鸣鞘破中宸(下)}

【第一更,求红票,收藏】

自元旦时古渭寨中的一面之缘,半年后韩冈终于再一次见到了给他留下深刻印象的青唐部族长之弟。

瞎药套着一身鱼鳞铠甲,头盔已经摘了下来,左手按着剑柄,稳稳的坐在一匹高大雄壮的河西骏马上。隔着战团,远远的与他的兄长对峙着。

太阳落山已经有半个时辰了,东面的天空已经有星月在闪烁,而西边的最后一点晚霞却正照在瞎药的身上,擦了油后的铁甲锃锃反射着红光,鲜红的披风和身侧的将旗在风中呼啦啦对舞着。

一颗死不瞑目的首级,挂在瞎药将旗的旗杆上,带着红色上翻帽檐的精铁头盔证明着首级的身份。俞龙珂带着八百甲骑在山岭中跋山涉水了两天,到最后,最大的战果却在瞎药的旗子上挂着。

渭水边的厮杀还在继续。已经彻底崩坏的董裕军,与一天前的身份掉了个个儿,成了被屠戮追杀的对象。瞎药带来的士卒大约也是七八百人的样子,都在右臂上缠了白布作为记认。他们举着刀枪,毫不留情的将还在顽抗的敌人一一砍杀。

临死前的惨叫没有一刻不在响起,刀枪入肉的闷声也没有一刻停歇。鲜活的肉体在刀枪中变成不动的尸块,鲜红的液体在战场上肆意流淌,血流漂杵不再是空洞的形容词,空气中弥漫起的血腥味,让胆小者反胃作呕,让勇猛的战士更加疯狂。

溃散了的军队只是被宰杀的羔羊,即便有哪位勇士想扭转眼前的危局,就地组织反击,也会立刻被从四面八方射过来的支支利箭给洞穿了身躯。

就凭着微薄的兵力,却能把近四千人的董裕中军打得全军溃散,让后军不战而逃,瞎药之前的指挥功不可没,而他交好董裕继而又反手一刀的心机,更是让人击节赞叹。

而俞龙珂这边的八百甲骑,却不等青唐部族长的命令,直接动手跟着自己的同族兄弟一起剿杀起残余的敌人。惨败的士兵中,不断有人绝望的跳入渭水。不是希图借助夏日湍急的河水逃出生天,而是仅仅是想躲避青唐部战士们的杀戮。

战斗即将进入尾声,两支同源的军队合流在一处。指挥着两支军队的领导者,终于面对面的站在了一起。

相比起一直黑着脸,直到瞎药走过来时才换上一幅笑容的俞龙珂,瞎药的嘴唇边一直浮着自信的笑意。兄弟两人在马上互相拥抱,用着吐蕃话交换着问候。看着他们脸上亲切的微笑,没人会怀疑他们兄弟之间真挚的情谊。

韩冈远远的躲在战团之外,为防流箭,他下了马,靠在一棵大树边。冷笑的看着不远处,那对面和心不和的兄弟聚在一起在交流着感情。

韩冈的护卫围成了一个大圈,守卫他的安全,防止有人杀红了眼,把他们当成了战功,也防着董裕的残兵想从这里逃出生天。

王舜臣一直都骑在马上,提着弓在外圈巡视。他用着四支直贯入脑的利箭说明此路不通,又以射穿脚背提醒两个蠢货,不要弄错了敌人。觉着应该不会再有不开眼的蠢货来冲撞韩冈,王舜臣也下了马,向圈子中走过来。

听到王舜臣走过来的动静,韩冈从俞龙珂兄弟身上收回视线,回头对着王舜臣笑了笑,问道:“怎么不继续练练手?多好的机会啊。今天多斩下几个首级,赶明儿也好向上报功。有我在看着,俞龙珂和瞎药都不敢抢你的功劳。”

王舜臣看着韩冈一如往日般平和沉静的笑容,突然间觉得陌生起来,仿佛是第一次认识面前的这个人。他踌躇了一阵,最终还是一咬牙,沉声问道:“三哥,你是不是事先知道瞎药会抢在前面偷袭董裕?”

韩冈挑了挑眉毛,对于王舜臣问出的这个问题有些惊讶,他笑道:“这些天我可是一直都在你旁边的,要是我想跟瞎药联系,也只能派王兄弟你去啊。”

王舜臣没有笑,“三哥你说的话俺都记得,这一路上,瞎药的事三哥你可提了不少次。而且三哥你前日还跟俞龙珂说过,行军之事不用着急,可以稳一点,上路后,又没对行军之事说上半句……如果是这两天多催促一下俞龙珂,今天我们是能赶在瞎药头里的。”

听着王舜臣的话,韩冈开始回想这两天自己到底说过了些什么,发现自己在不知不觉间,好像真的说了不少不该说的话。他自嘲的笑了笑:“看来我的口风还真是不严。”

王舜臣顿时大惊,脸色陡然变了,“难道三哥你真的……!”

“你想到哪里去了?!”韩冈皱着眉摆手道,“根本不是你想得那样,我怎么可能联系得上瞎药,你不是都跟着旁边?”

“那……”

“这只是很简单的推测。我把我自己代入到瞎药这个身份上——如果我是瞎药,我会怎么做?”

“呃……所以三哥你才能事先猜得出瞎药会来?”王舜臣还有些半信半疑,不,看他的眼神,应该是有八成不信。

这样可不好,韩冈想着。

“没错!”他却点着头,正色说道,“王兄弟,我也不瞒你。我的行事风格,真要算起来,跟瞎药也差不离。都是精于算计,总会选个对自己和身边的人最有利的一条道路去走。你想想,我过去是不是都是这样行事的?”

韩冈说得很直率。王舜臣对他很了解,装着老实人的模样根本没用。而用谎言瞒过,只会让他离心,实话实说才是正确的选择。以王舜臣跟自己的关系,只要对他推心置腹,就不虞他会跟自己疏离。

王舜臣低头回忆起他过去所了解的韩冈,想着想着,便发现好像真的是跟韩冈说得一样。

“今年年节时,来古渭寨送年礼的瞎药给我的印象极深,尤其他那对桀骜不驯的眼神,怎么看都不是甘居人下之辈。我要推断瞎药的行事,也只会把他往狡猾多智的方向去考量。”韩冈见王舜臣低头思考,又趁热打铁的说道,“瞎药今次做得正如我所料,把董裕、俞龙珂都算计了进来,而且做得很完美,一点破绽都没露出来。现在他斩了董裕,在青唐部和青渭的声望已经凌驾于俞龙珂之上,也许再过一阵子,说不定青唐部的族长便要换人了。”

王舜臣听着韩冈的话,先是点头,但想了一想,便又提出了一个疑问:“但三哥你也没必要帮着瞎药,若是早点拆穿,或是一直催着俞龙珂快点走,瞎药根本不能成事。”

在王舜臣看来,虽然俞龙珂心意坚定,不会什么都听韩冈。但韩冈只要一个劲的催他快点行军,俞龙珂总得给韩冈一个面子,这样算下来,至少能比现在早到一个时辰,而他们与瞎药的差距也就在哨探来回的一个时辰之间。

“一切都建筑在猜测上,我怎么跟俞龙珂说?”韩冈摇头笑道。“而且我为什么要帮俞龙珂?如果他斩了董裕,为七部报了仇,青渭诸多蕃部必然会亲附于他,声望和实力都足以抗衡木征,可以反过来压制古渭寨,便更加不会听从朝廷之命,这只会给河湟之事添置障碍。”

“所以三哥你才阴助瞎药?”

韩冈抬头看看已经亲热的携起手,在战场上并肩走着俞龙珂、瞎药两兄弟。冷笑道:“今日之后,在青渭一带,瞎药必将兴起,而俞龙珂势力转弱。俞龙珂要想保住现在的权威,只有给自己找个好后台。”他又叹了口气,“这也是没办法的事。要知道,当我们前日抵达青唐城的时候,七部残破已不可避免。既然如此,也只能为机宜找个新手下了。”

“三哥你真……真是……”王舜臣真是了半天,也不知用什么词来形容韩冈的才智才好。现在他心中,除了佩服,就只剩惊叹。

“不要把我想得太聪明。”韩冈再一次摇起头,“上面的一切都是纯粹的猜测,我不可能按着猜想去做事。所以我今次做的,也只是稍稍拖延下俞龙珂进兵的速度。反正只要赶得及抄董裕的后路,走慢一点也不会有关系。瞎药之事只能算是个惊喜罢了。”

如果有,当然好,如果没有,其实也无所谓。韩冈对瞎药的行动,其实是抱着的是旁观者的心态,并没有太过在意。就算俞龙珂独霸青渭又如何?在大宋面前,也不过是只蝼蚁而已。只要朝廷支持王韶,凭着实力照样能压服那时的俞龙珂。

一直以来韩冈并没兴趣对小小的青唐部用什么离间或是二虎竞食之类的计策。太麻烦不说,也没那个必要。今次不过因为是顺水推舟,却也无所谓,左右是举手之劳,动动嘴皮子而已,并不会累着自己。若是吃力点,韩冈可没兴趣。

也正如韩冈方才所说,如今青唐部应该算是分裂了,而通过此战,瞎药在青渭的威名恐怕已经超过了俞龙珂。青唐部的族长如果想保住他现在的位置,也只有投靠大宋,投靠王韶。

“还有!”想起王韶,韩冈不得不提醒王舜臣,“今日我所说的都要保密,传扬出去,我可是会有些麻烦。”

“三哥放心,”王舜臣不闻情由,用力点头,“俺绝不会对外说半个字!”

