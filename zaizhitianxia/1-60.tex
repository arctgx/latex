\section{第28章 夜影憧憧寒光幽(二)}

一番演武之后,韩冈领着一众友人回家休息。不再是几个月前的村口草庐,而是一座前后两进的宅院,这是韩家的老宅。韩冈受了举荐,王韶、吴衍和张守约三名举主知他家中境况贫寒,便各自赠银以助行色。韩冈并不客气,很洒脱的收了,只道了声谢,丝毫没有感激涕零的样子。他的这种不为财帛所动的态度,反而让王韶三人更加看重。拿着收到的银钱,韩冈将家宅赎回,时隔半年之后,韩家重又搬回了熟悉的地方。

进了家门,几人进去拜见过韩冈的父母——韩冈、王厚交情非同一般,有通家之好,王舜臣、赵隆也是一样,韩阿李也不须回避他们——围坐在韩冈的厢房内,韩云娘上过茶后,端了盘果子零嘴,也退了出去。

“玉昆,你这家中还是少人服侍啊……”王厚打量着有些年头的旧屋,造的还算坚固,就是显得太寒酸,“令尊令堂身前不能没人,一个小养娘怎么照顾得来?你都是官人了,还是再收几个仆役婢女跟前使唤才是。难道这些日子没人来投效?”

“有!”韩冈点点头,他现在跟范进中举没两样,多少人听说他要做官了,赶上来送钱送物,还有的就是自己卖身为奴,想投到韩家里听候使唤。“不过小弟都给拒了。”投身官家为奴的,多是乡里的破落户,这样的人来投效,求得就是仗着身后大树的树荫作威作福。韩冈怕还没做官,就被一群恶仆毁了自己的名声。

韩冈此举坐实了他视钱财如粪土的名声,但王厚觉得他做得过火了点,“玉昆,崖岸自高并非德行,和光同尘才是正理。送上门的田地都不要,本都是你自家的东西……”

“都典卖出去了,怎么还会是我家的东西?”

王厚说的是李癞子的事。下龙湾村的里正运气的确很糟。前面靠着陈举提携,好不容易用了过半家产从黄德用案中脱了罪,现在又被卷入了陈举一案。尽管与陈举关系疏远,但只要有点牵连,便少不得被州衙里派出来的衙役敲打,李癞子家仅剩的一点家财又流水般的用了出去。

河湾菜田本是韩家之物,消息灵通的衙役没一个人敢打主意。李癞子上门想把菜田还回来,求得韩冈高抬贵手,开口说句好话。只是韩冈没肯要:“何况因那几亩田地死了多少人?土里都透着血,如此不祥之物,拿回来也会贻害家人,小弟也不想要了。”

现在回想起来,一切的起因都是因为藉水河湾边的区区三亩菜田。黄大瘤死不瞑目,而陈举很快就要千刀万剐。如果再加上末星部的近千帐的蕃民,因着三亩菜田,血流成河,人头滚滚落地。仿佛一个浸透了血腥的黑色笑话。

“……说的也是,那块地的确不吉利。这世上有钱哪里买不到好地?等李癞子完蛋,就看哪个蠢货会盘下来!”

“赶尽杀绝的事小弟做不出来,还请处道你帮忙在州衙里说一声,放李癞子一马吧……”

王厚惊起:“玉昆!李癞子虽非罪魁,却是祸首。一切事都是因他而起,你竟然还要饶过他?!东郭先生可做不得!”

“小弟已与家严家慈商议过了,都是乡中邻里,并非陈举之流,没必要把他往绝路上赶。”韩冈神色间温文淳厚,标准的秉持仁恕之道的正人君子模样。

这些日子,李癞子天天求上门来,好话陪了不少,头也磕了许多。

韩千六对那块田地感情很深,又是老好人一个,便想收下地,让儿子帮李癞子说句话。但韩阿李心中怨气不解,根本不肯答应,地宁可不要,人绝不能饶,她骂着韩千六:“看你那点眼界!李癞子害得俺家差点家破人亡。如果没三哥儿在外面拼命,全家都死绝了,李癞子会到坟头上哭一声吗?!过去典给他的地,就放在他家那里,俺也不要他送回来。该是多少就是多少,俺们拿着大钱去赎,不占他一文钱便宜!”

而韩冈比他老子还好说话,却是不要地,人也要放过去。他劝着父母:“李癞子也害不了人了。一条死狗,何必穷追猛打,传出去对孩儿的名声也不好。”

宽恕是强者的权力,如果韩冈在被人步步紧逼、性命攸关的时候,说什么仁恕,那是完全是个笑话,陈举、刘显、李癞子之辈,多半会哈哈大笑一阵,把他当成白痴。但现在韩冈居高临下,放过李癞子一马,便是气量如海的宽容。

对于一个儒生来说,名声是最重要的,睚眦必报这个词从来不是对个人品德的好修饰。世所言‘量小非君子,无度不丈夫’,过人的度量和不拘于旧怨的洒脱,对提高自己在世人眼中的评价很有好处。

最关键的一点,就是比起向宝这只在阴暗处敛耳伏躯的大虫来,李癞子根本连屁都不是,没有任何害人的能力。既然留着他一条命,对自己毫无伤害、无伤大雅,还能向世人证明自己的宽容和大度,又何乐而不为?相反地,如果李癞子还拥有能伤人毒牙利爪,韩冈绝对会把他连皮带骨一起拆散掉的。

韩冈籍此说服了父母,但他不想用这个理由来说服王厚。个人形象的树立有着很深的技巧,在甘谷城中,韩冈已经表现出了过人的德行,现在他更需要要塑造的是自己的才智和谋略。

“陈举有一个儿子脱逃在外,黄大瘤也有两个儿子,他们现在都不知所踪。虽然我不担心他们能把我怎么样,但家中父母小弟怎么能安心得下?总不能请王兄弟或是赵兄弟两个日夜来守着吧?外兄也是要大用的,不可能守在家中不动。自来只有千日做贼,没有千日防贼的道理。不看着陈家余孽被一网打尽,我怎么也不能安心。”

“这跟李癞子有什么关系?”赵隆茫然的问着,而王舜臣露出了深思的神情。

王厚替韩冈解释:“李癞子是黄德用的姻亲,又因为黄、陈两案倾家荡产,如果不饶他,他说不定会狗急跳墙……玉昆,你是不是这样想的。”

王舜臣觉得难以置信:“陈缉那几个贼逃囚的胆子应该没这么大吧?打三哥的主意,这是杀官造反啊……”

“早就是死罪了,就算杀官造反,还能在砍下首级之后,再弄活过来砍上第二次?他们没什么好怕的,一定会来!”韩冈很肯定。

还要多谢李信,他的这位二表哥从凤翔府护送着韩家父母会秦州,在路上便发现了有人鬼鬼祟祟的在后跟踪。不过他只埋在心底,没有说出来。一直到了与韩冈见面后,才说给了韩冈一人听。而黄大瘤两个儿子的相貌特征,韩冈又怎么会不了解?黄家兄弟既然跟踪着从凤翔府回来,他们在打什么主意,不用想也知道。

“若不是为了对付陈家余孽,我何必买回旧宅?田园生活虽好,但为官之后,必然要将家搬到城中。为何多此一举?还不是为了要引出陈举余党。城中人多,说不准哪里就会捅出一把匕首,防都没处防。但下龙湾村里就不一样了,乡里乡亲没有不熟悉的,生面孔根本进不了村,要想打探我家的消息,只能靠着村里的人……除了李癞子,陈缉又能依靠谁?”

韩冈的声音沉稳中充满自信,十分的有说服力。王厚信了八成,王舜臣和赵隆则根本不会去怀疑韩冈的判断。至于李信,始终都是一种表情,没人知道他在想些什么。

“韩三官人……韩三官人……”从后门处,突然传来小孩子的唤门声。

李信过去开了门,带进来的是李癞子才十三岁的小儿子李小六。一进厢房,就跪下来给在座的几人磕了头,起来后道:“俺爹有急事要俺带话给官人:陈举的二儿子陈缉,如今已经收买了一伙强人——头领唤作过山风的便是——说是总共有一百多贼人,要向官人报杀父毁家之仇,时间就是今夜。现在逆贼黄二带着一名喽罗守在小人家里,俺爹脱身不得,所以让小人来急报官人。”

李癞子的幺子年岁虽小,却口齿伶俐,在场的几人都听清楚了。王舜臣、赵隆投向韩冈的眼神中有着三分惊讶七分崇拜,王厚也是惊诧莫名,韩冈的预言才出口就得到印证,哪能不让他们震惊。

“一百多?”李信第一次开口,只有短短三个字,声音沙哑得像把锉刀。

韩冈摇头,秦州道上哪可能有这等人数的强盗团伙,光靠打劫为生可养不活这么多人:“四五十人都不可能。魏武帝下赤壁,还号称八十万呢。一百多……哼,秦州的哪伙强贼有这个数目?!最多二十人,再多,早就给剿了。”

“玉昆……贼人数目先摆一边!”自相识以来,王厚不知多少次从韩冈身上收获到惊讶,从为人,到眼光,再到能力,但以今天的庙算为最,他脸上写满了不可思议:“你唤愚兄和王、赵两位过来演武,难道是事先就已经算到了陈缉今夜会来?!”

韩冈笑而不答,事实就是最好的答案。

ps:好戏锣响,厮杀的场面要开始了。

今天第三更,红票还是不给力啊,还请各位兄弟,多多支持cuslaa。

