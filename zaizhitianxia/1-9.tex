\section{第六章 气贯文武与世争(上)}

旭日初升,红霞灿烂如锦。秋风萧瑟,黄叶漫山如席。

下龙湾的秋日清晨,由浓浓的红黄两色交织,天光山色,如同画里。村外藉水川流不息,水声中添了几许寒意。

在藉水边的一块空地上,只听得嗡得一声弦响。一支长箭离弦而出,正中二十步外稻草扎成的靶心。在一尺大小的圆形箭靶上,还高高低低插了六支长箭,都是围着靶心,没有偏离太多。

一轮射罢,箭箭中的,韩冈专心致志地脸上,也便带出了一点微笑。垂下持弓的双手,连喘了几口大气。站在一旁的韩云娘连忙跑过来,拿着条葱绿色汗巾,踮起脚抬着手,擦去韩冈额头上的汗渍。

襦裙袖口宽松,小丫头手一抬,便褪到了肘后,半截莹润如玉的皓腕就在韩冈眼前晃着,淡淡的暖香从袖中飘出。她身子只及韩冈的胸口,整整矮了一个头还多,抬手擦着韩冈头上的汗,整个身子都不得不贴上来。隔着几层薄薄的衣裳,感受着贴入怀中的酥软温香,韩冈心底忍不住有些燥热,更有着一份促狭之心,双臂一合,韩云娘呀的一声可爱的惊叫,被他搂在怀里。

“三哥哥不要……”

韩云娘娇羞不胜,双臂无力推拒着。纤柔绵软的娇躯在怀中扭动,韩冈心火一时大盛,正想进一步动作,一阵人声却远远传来。小丫头似迎还拒的挣扎突的变得剧烈起来,身在屋外,韩冈不敢用强,手一松,韩云娘忙跳到一边,嘟起嘴,扭头看向另一面,不肯再过来。

小丫头气呼呼的,脸色殷红如旭日映照,耳朵热得发烫。韩冈轻笑了两声,又抬起掌中长弓,不敢再去撩她。

韩冈现在所用的长弓,并不是旧时自用、由嫡亲二哥所赠的一石三斗的硬弓,而是他老子韩千六旧年收藏的七斗猎弓。而且由于收藏日久,保养不当,这猎弓的力道大约只剩四五斗的样子。以他如今的气力,也能轻易拉开。

这段时间以来,每天清晨,韩冈便开始拉弓射箭。不仅仅是因为要仿效前身的行事,以防自己的身份败露,更是为了要早日恢复健康的身体,而在加强锻炼。

这个时代没有抗生素、没有现代医疗,一点病症就能要人命。韩冈劫后重生,对自家性命看得更重了几分。好不容易得来的第二条命,他一门心思要加强锻炼,虽不可能百病不侵,但至少也要多活几年。

走上前摘下插在靶上的长箭,韩冈又站回射击的位置上。弓弦有节奏的振颤着,一支支长箭准确的飞向靶中。这些天的练习并没有白费,命中率比一开始时大大增加。烙在身体上的记忆正在慢慢恢复,不论是射箭的姿势,还是拉弦用力的指法,韩冈都比起初强了许多。

日上三竿,韩冈已是汗透重衣。起床梳洗后就开始的锻炼,也差不多到了结束的时候。用力射出最后一箭,在靶心又留下一个深凹,他和小丫头一起收拾好弓矢,沿着河堤向家中走去。

在藉水岸边举目远眺,秦州城在北面重重山峦的映衬下,是微不足道的渺小,但实际上,秦州城墙的厚重巍峨,是为西北边陲之冠。自来到这个时代之后,韩冈还没有去过咫尺之外的城池,但他对秦州的了解比天天去城中的父母可多得多。

秦州隶属于秦凤路。其路因秦州和凤州而得名。韩冈前世的地理学得还算不错,又走南闯北多年,全国各地的重要城市可算是门儿清,但对宋代的地理名词却还是是摸不着头脑。秦州、凤州都是很陌生的名词——他只依稀记得陕西有个凤翔县,却与位于秦州东南的凤翔府同名——不过秦州又名天水郡,而且治下还有一个天水县,这个地名看多了三国的韩冈却是如雷贯耳。

以韩冈的地理常识来看周围地形,秦州州城一带,包括小小的下龙湾村都是处于藉水河谷中。至于南北两边的山峦,北面唤作长山的应是属于六盘山,南面便是千百年来从未改换名号的秦岭。而贺方熟悉的天水县则还在秦岭之南,位于嘉陵江的源头上。可以说千年间的地理完全变了,因为二十一世纪的天水应是在秦岭北麓的,也许正是在如今秦州城的位置上——韩冈虽是猜测,但事实也正是如此。

天水在后世属于甘肃,但如今的秦州却是属于秦凤路。而秦州也不仅仅隶属于秦凤,同时也是治所位于京兆府【即长安】的陕西路的辖区。看似让人头晕,但实际上坐在秦州城中的是秦凤路经略安抚使,而在京兆府内的,则是陕西路转运使。虽然都是名为路,其实一个是经略安抚使路,一个是转运使路,按着后世的说法,这是军区和省的差别。

东西走向的横山和天都山是宋夏两国的分界线。而陕西延边地带,又被从横山和天都山向两侧延伸出来的南北走向的余脉所分割。被分割出来的各块地区之间由于山势阻隔,难以互相支援,并统一指挥。为了更好的对抗西夏的党项铁骑,宋廷便以南北走向的分水岭作为边界,将陕西从东到西分成了鄜延、泾原、环庆、秦凤四个经略安抚使路,以独立处理军事。但代表地方政事辖区的陕西转运使路尽管一直有动议要将其一分为二,以利监察地方政务、并安排粮饷转运,却至今未有变动。

回到家中,韩千六今日有事先进了城去,韩阿李则烧好了一锅热水候着。韩冈锻炼了回来,浑身是汗。为防风邪侵体【即感冒】,他每天都要在锻炼后用热水擦洗一番。病愈后近一个月的修养,韩冈的身体虽未恢复旧观,可脱掉外袍后,也不再是骨瘦如柴的模样。

身在家里,小丫头也不再羞怯——主要还是习惯了的缘故——不需韩冈自己动手,她便主动上前拿着热毛巾帮忙擦洗。揩干后,最后还帮着换了身干爽的衣服,把韩冈服侍得妥妥贴贴。只是正因为身在家中,顾忌着父母,这时候反过来倒是韩冈不敢有所动作。

运动之后,用热水擦洗一番,韩冈一身舒畅。靠坐在书桌边的交椅上,看着韩云娘在房中忙来忙去,心中不禁涌起一番温情。韩冈可以说是爱上了如今这种腐败的生活。千年之后,就算是国中的达官显贵,怕是也很难得到一个可爱的少女如此全心全意的照顾。

半个月的时间,说长不长,说短不短。韩冈每日里读书射箭,重生后,原本一些模糊淡忘掉的学问重新被回忆巩固,而下一步该如何进行,他也有了初步的计划。

韩冈铺开书册,打算按着计划开始今天的功课。韩阿李这时端着碗羊肉汤和块炊饼走了进来,韩千六大清早就出去了,韩阿李独身一人也不能去山中采山货,就留在了家中等韩千六回来再去。

将韩冈今天的早饭放在桌上,看着铺满在桌面上的书卷,韩阿李有些觉得奇怪,自家的三儿子往日最喜欢【和谐万岁】吟诗作词,才十五六岁就积了上百首下来。怎么现在病好了这么些日子,就只顾着读书?

“三哥儿,怎么这些日子只见你读书练箭,却不作诗了?”

韩冈愣了一下,马上又笑了起来:“当年学问不精,所以也不觉得自己诗词写得差。但孩儿自投到横渠先生门下后,才知道什么是井底之蛙。比起诸多同窗学友,论诗才,孩儿是远远不如。”

“哦……”韩阿李的声音中透着些许失望。三哥儿一向是她最疼爱的儿子,从来都是可以向邻里亲友夸耀的骄傲,直指望他能光宗耀祖。没想到去了外面游学了两年,回来却说自己远不如人。

韩冈见状,忙向母亲解释道:“不过论起经义大道,孩儿还是不错的,先生也多次夸奖孩儿。经义是最正经的学问,诗词歌赋都比不过的。”

听儿子这么一说,韩阿李顿时喜上眉梢:“张先生是天上的星宿,他说的不会有错!三哥儿你要听张先生的,好好读书,日后考上进士,也可光宗耀祖。”

韩冈称是受教,目送韩阿李笑着出房。这也是父母之心,听着孩子自称自赞的话,只会为之高兴,都不会怀疑半分。不过韩阿李所说的,也是他身体的原主十几年来的心愿。前任一门心思都放在读书做官上,连带着自己可能受了影响,不过,更有可能是如今的韩冈,对权势对富贵的那种发自内心的渴望。继承了这个时代流行的学术常识,又拥有千年后的知识,韩冈比起前任更有自信,也更有野心。

可韩冈纵然有两个时代的学识,想考个进士一样还是水中捞月。进士科考的主要是诗词歌赋,兼及一点策问经义。韩冈很有自知之明,他前身的诗才本已是惨不忍睹,自家继承后更是尤差三分,想去考个进士完全不现实,恐怕连通过州里的发解试都有难度。

ps:本章中有一长段说明文字,虽然有些无聊,还请各位仔细看一下。要了解宋代行政区划,首要的便是要分清安抚使路和转运使路的区别,不分清这两点,看后面的文章就会很容易糊涂。

ps之ps:谁能告诉俺,欢|吟究竟是哪里触犯和谐之光了?

今天第三更,照例求红票,收藏。

