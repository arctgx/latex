\section{第一章 纵谈犹说旧升平(七)}

【比预定的时间要晚了一些,对不住各位了】

周南的生日是哪一天,韩冈当然记得。

每年春雨淅淅沥沥的时候,就是她的生日。虽然已经记不清父母和家人,但周南还是记得在她被没入教坊之前,生日时淅淅沥沥的雨声。

今天韩冈本也是记得的,还想要给周南庆贺一下,礼物也准备好了,只是没想到被吕惠卿给耽误了,又因为这几桩事弄得一时忘了。

“那礼物准备好了没有?”韩云娘很好奇的问着。

走到后院的回廊上,韩冈收起了伞,将之倚在二门外。捏了一下小巧的琼鼻,眨着一对好奇的大眼睛的韩云娘立刻变得眼泪汪汪起来。

韩冈笑着:“等明天早上问你南娘姐姐吧。”

这个小家庭中的成员,都还没到祝寿的年纪。遇到生日,也不会大事操办,以防折了寿数。就是韩冈过生日,也不过是一碗长寿面,还有妻妾们亲手裁制的衣服和鞋子而已。

但韩冈每年都没忘记给心上的妻妾一个惊喜。有时是一对晶莹剔透的耳坠,有时是一支雕工精美的步摇。善于茶艺的素心韩冈为他找来了两只御赐的龙团;尚未脱了孩子心性,喜欢些小玩具的云娘,是一套活灵活现的泥塑人像;曾经以匕首定情的周南,则是来自东瀛的短刀;两个月前王旖生日的时候,韩冈甚至还送了一支亲手做的枣木簪,虽然王旖口中嗔怪,但之后就一直作为发簪带在头上。

这点讨好女孩子的小手段,韩冈并不缺少经验。不管是哪个时代,女孩子总是要哄着、宠着,韩冈虽然忙于政事,但经营家中的手段也不会给荒疏掉。

王旖、周南和素心都在房中等着韩冈。三个小孩子早就困了,被乳母带回去睡下了。只有她们手上绣着花样,有一句没一句的说着,偶尔从一旁的使女手中接过热茶喝上一口。没有男主人在的房间,纵然人气不少,也是让人觉得缺了些什么。

只是一到韩冈回来,家里的气氛就不一样了。众女都站了起来。王旖问道:“官人,可曾吃过了?”

她不去问韩冈到底去了哪里,只关心着丈夫是否饿着肚子。韩冈在云娘和素心的服侍下脱了身上的外袍,周南从里间拿来了换洗的衣服。

韩冈已经习惯了有人服侍自己穿衣脱衣,一边抬着手,一边跟王旖说着话:“就没打算在吕吉甫家叨扰,前面在军器监中吃了一点垫了肚子。”

“吕吉甫?”王旖很是讶异的歪着头。她知道除了王韶和蔡挺以外,丈夫如今跟宰执们的关系没一个好的,吕惠卿也是一样。

王旖疑惑起来时的习惯很是可爱,头略略歪着,眼睛也争得大了一点,显着有几分稚气。

“还记得李士宁吗?”

韩冈在交椅坐下来,抬起脚让她们把脚上的官靴给脱掉,抬手接过素心递来的祛寒的热汤饮子。韩冈啜了一口,带了点紫苏味道,浑身也暖了起来。只要下雨,不论季节,严素心都不会忘记在小药炉上炖上一罐。

“记得。”王旖点点头:“还记得官人好像不喜欢他的,都没怎么说过话……他出了何事?”

“给牵进了一桩案子中,吕吉甫怕牵连到岳父身上,所以邀为夫过府商议此事。”

“官人!”王旖一下变得紧张起来,抓着韩冈的手,“爹爹不会有事吧?”

“你不想想岳父什么身份?绝不会有事的。”反手抓着王旖细嫩的小手,韩冈轻笑着,“吕吉甫也只是为了以防万一,怕有人趁机使坏。不过只要有天子在,岳父可以高枕无忧。”

韩冈如此说,王旖便放心了下来,展颜笑起:“那就好”美目流转,看了看周南,又道:“我们姐妹已经给南娘妹妹贺过寿,下面就是看官人了。”

王旖招呼着云娘和素心。自幼受着三从四德的教诲,她这位大妇毫无小户人家的挟忿含酸,盈盈的举步离开:“今天南娘是寿星,官人可要好好陪着。”

厅中一下只剩韩冈和周南,连周南身边的墨文都不知跑到了哪里去了。

韩冈摊开手,周南将小手放了上来,轻轻攥住,一起往周南的房里去。

“今天收了多少礼物?”

周南摇摇头,嘟着嘴:“还没有收到官人的。”

白皙腴美的酥胸鼓鼓的顶着衣襟,山岭沟壑的风光,在韩冈双眼所处的高处能尽收眼底。但到了腰后就向内收了进去,可收到极致,又夸张的涨了出来,因练舞而变得挺翘又充满弹性的臀股,每每让韩冈爱不释手。已是一枚熟透的水蜜.桃,到了单独相处的时候却是一幅孩子气。

到了周南的房间中,蜡烛已经被点上了,外面罩了银红色的纱罩。

韩冈在桌边坐下来,揽过周南。

周南依顺的靠在丈夫怀里,低头看着亮在自己面前的匕首。

这是韩冈前面从书房中拿过来的,准备了十几天,今天正好可以送出来了。

刀鞘刀柄并没有什么装饰,猪婆龙皮鞣制的皮鞘虽然并不便宜,但单纯染上一层黑色,也就看不出有什么特别。就像韩冈的为人,锋芒、光彩全在自己身上,从来不在服饰上做文章。

周南捏着刀柄,向外轻轻一抽。

短短的匕首上,是如同层层浪涌的纹理,而不是能映日月的晶莹铮亮。一道道黑白纹路,细密交叠。深如夜空,浅如晨雪,五六寸长的刀面如同一幅浓缩过的水墨山川。

“这是松纹?”周南有些惊奇的看着。她只听说过世上有所谓的松纹剑,上面有着一道道花纹。纤长的手指探上去就想试一试刀锋。

“小心!”韩冈连忙抓住傻乎乎的小手,“这是来自于大食的镔铁,吹毛断发的。”

是真正的大马士革钢,不是后世骗人的赝品。这个时代,想打造出与大马士革钢相似的花纹来,也许在技术上有那个可能。但造假者还没有这个见识,大马士革的名气远远不如日本,假造日本刀更为赚钱。虽然世上也有所谓松纹剑、雪化刀,但并没有多少人将钢上碎乱的纹理,当成是名剑名刀的卖点和标志

韩冈从周南手中将匕首拿过来,在桌角一划,一片木片就削了下来。

“好快!”周南小声的惊呼着。将匕首接下来,战战兢兢的拿着。

烈性子的她,因为韩冈当年所赠的定情信物,而变得喜欢起了匕首这等危险的玩具。但她并不是自己收集,只是将韩冈几年来送的刀匕视若珍宝的一一珍藏起来。不是因为刀匕本身,而是因为韩冈。

“也是监里最近要在钢铁冶炼上下功夫,为夫就让人找来了天南海北的名刀名剑,还有各色铁器,其中就有一柄大食镔铁刀。为夫看着喜欢,就另外向提供镔铁刀的大食商人买下了这柄短匕。就是刀柄刀鞘太俗,让人给换了。不过这匕首太过锋利,可不能乱动。”

韩冈拿起刀鞘,就着手将锋锐给收了起来。

周南仰靠在韩冈的怀里,将匕首贴在心口:“奴奴只会藏起来。”

韩冈坐着,周南站着。

韩冈的脸正对着周南如膏脂般腴白腻滑的酥胸,呼吸的热气穿过薄纱裁成的一层亵衣,直透了进去。小妇人一下就情动起来,用力抱紧了韩冈,在耳边呢喃着,听不清在说着什么,只有一股股温软的气息呵着耳朵。

柔软细腻到了极致的肌肤,过了哺乳期也没有消减回去,无视地心引力的骄傲的挺立着,充满肉.欲的弹性。虽然舞蹈时甚至会感觉像是个累赘,但看着丈夫爱不释手的揉捏着,微微的痛楚中,就是涨满胸臆的欣喜,还有一阵阵让人变得湿润起来的酥痒酸麻。

双手向下环住不堪一折的纤纤腰肢,虽然是丰腴的身子,又生过了孩子,但腰身却依然犹如少女时一般。韩冈为了自己着想,没少鼓励妻妾日常多活动身子,踢着气毬,荡着秋千,也学着跳舞,身材一个个都保持得很好。

廊外传来细碎的脚步声,墨文提着食屉走进了房中。跟在周南身边的小丫鬟如今也年满十六,男女之事不会陌生。看着韩冈和周南的动作,小脸就红了起来。不过对于贴身的丫鬟,夫妻之间的私房事都是不须要避忌的,平日里也见得多了。只是在午夜梦回之时,那个羞人的地方往往都会让她难堪的潮湿。涨红着脸,在桌上摆下了几碟精致的小菜,还有一壶温过的水酒。

韩冈拿着筷子,夹了一块鸡,不同于严素心管理的厨房一向的味道,“是南娘你做的?”

周南点了点头:“好吃吗?”

难怪素心并没有依着往常自己迟归时,及时的端上来的加餐夜宵。官员在外赴宴,很少有埋头痛吃的时候,酒喝得一肚子,菜肴没动几下都是常事,回来后总要再吃上一点。

烛光下,佳人如玉。

水酒虽然清淡,但一杯下肚,就已经给佳人玉色的面颊上,添上一抹酡然醉红。

