\section{第一章 纵谈犹说旧升平(四)}

直到暮色降临,韩冈方从宫中出来,赵顼并没有立刻应允将军器监中几个重要的制造局迁到东京城外去的提议。他必须听取中书的意见。

赵顼的犹豫,不仅仅是担心板甲、斩马刀,以及韩冈信誓旦旦会比如今的畜力锻锤更强三分的水力锻锤的制造工艺会泄露出去,同时也担心撤销官营的水力磨坊、改以铁器作坊会影响太多人的生计。

苏颂与韩冈并行而出,摇头轻叹:“汴河上的官营水磨水碾,每天的出产全都供给东京城百万军民,不可能随意撤销,若无替代,京城之中必然生乱。”

虽然方才在殿上没能即时说服赵顼,也有一部分原因是因为苏颂的反对,但韩冈并不没有因此而对苏颂有所反感。单纯就事论事的意见,他还不至于没那么个气度去听取,但他也绝不认同苏颂的说法:

“没有水磨、水碾,可以用风磨、风碾,即便没有风磨、风碾,也可以用上畜力。这门生意的收入,对于商人绝不算少,想必他们也会趋之若鹜。可在官府来说,一年二十万贯的营收,则是微不足道。朝廷为了区区二十万贯,平均每年就要往汴河中多投入差不多五六十万贯的清淤费用。而若是改以铁器作坊,虽不说能将清淤费用省下来,至少能把帐目给作平掉。。”

苏颂瞥眼看了一下韩冈,眼中不掩对这位年轻后生的欣赏,说话、行事都让人感到舒服,方才在殿上争执时,也没有出现此时朝堂争锋,不论事,而直接攻击对方人品的做法。苏颂为人厚朴,很是欣赏这样的年轻人。

只是他也同样不会就此同意韩冈的观点:“帐不是这么算的,民以食为天,将百万军民的口中之食转经商人,其中的情弊想必玉昆比老夫更为熟悉,难道就不怕会重蹈旧日粮商覆辙?”

韩冈不与苏颂争了,说服一个权知应天府也没有意义,无奈的叹了一声:“还是因为黄河水泥沙太多。放进汴河的水越多,造成的淤积就会越厉害。如果不是这个原因,使得汴口不能敞开,又何必让水磨与水碓争夺地盘。”

汴河在京畿一段的来水,全都靠着黄河来提供。但黄河水一碗水半碗沙,汴河又是人工河,水势平缓,放水进来越多,淤积的泥沙当然会越多。

汴河若要通航,只要保证六尺水深就足够了,并不需要多开汴口河闸。但为了驱动水力磨坊,却要时常开启,使得汴渠中有足够的流水。因此造成的大量泥沙淤积,就要耗用更多的人力来清理。从收入上来看,当然是得不偿失。

“黄河水清非百年不可见其功,这话可是玉昆你说的,怎么现在又作无谓之叹?”

苏颂知道韩冈去年曾提出了束水攻沙的治河方略,并指出黄河的泥沙多来自于关西,要想解决黄河泥沙,除非能让关西从此草木丰茂,现在为黄河泥沙叹气,倒是让他有些觉得好笑。

韩冈笑了一笑,摇头不语,与苏颂做口舌之争没什么意思。

两人一起沉默的向宫门外走着。走了一阵,已经出了文德门,宫墙就在眼前,苏颂忽然问起,“若是设置铁器作坊,可是要改以专利?”

韩冈摇头:“不会,军器倒也罢了,民用铁器怎么可能让官府专利?从成本和品质上来说,民间打造的铁器绝对争不过官营,没必要下个禁令,徒惹起朝野议论。”

在韩冈看来,如今的朝廷有个很坏的毛病,那就是专利。

此时的‘专利’二字,并非后世的含意,而是字面意义上的专享其利,指的是垄断。官府如果准备要对某个行业垄断,就会对民间的商业行为进行禁榷——也就是禁止民间商人对这些商品进行交易。

盐业这等从汉代开始,就给朝廷收归国有的生意不算,酒麴、香药、白矾,铜、铅、锡等能造钱的金属,乃至如今川陕的茶马贸易,都是由官府专营,只有不多的一部分有民间插足的余地。

而且官府专营的手段也足够恶劣,并不是靠着规模和技术,而是靠着行政禁令。比如河北的矾业,过去向来是民营,有几个大家族因此而成为豪富。但当官府见到其中之利,插手矾业生产之后,却因为生产等各方面的原因,争不过民营的作坊。主持官营作坊的官员,便上书请求对矾业禁榷,由官府专利。

不过这等将商业利益一口独吞的毛病,并不是新法推行才开始的。这是传承了晚唐五代时各个藩镇的习惯。那时候,为了养兵,每一国、每一个藩镇都少不了开设店铺、作坊。只要是赚钱的买卖,那就什么都做,绝不仅仅限于盐、铁二物。几百年来,官府经商早就成了习惯。

多少旧党都在指责新法是在与民争利,可只要去看看厢军中,有多少指挥的名字是酒店务、车船务,就知道铜臭之气早就弥漫在大宋皇城的殿宇之中了。

其实铁也是专营的,从西汉桑弘羊开始,铁矿的开采和营销绝大部分时候都是由官府来控制。不过眼下铁器的制造,尤其是民生用具,其实朝廷放得很开,经营铁器的大商家各地都有,朝廷只是将矿山和锻冶给垄断了而已。

“铁器并不是白矾。”韩冈继续对苏颂解释着,“白矾官营与私营的作坊工艺相同,经验还要输上一筹两筹,当然比不过私家作坊。但现在官中打造铁器,换做了机械锻锤后,已经远远胜过民间。”

“军器监中的各色锻锤,难道不会给民间的作坊偷学过去?”苏颂质疑道。

“哪有那么容易!?”韩冈哈哈大笑,但心中却是在说着‘正是吾之所欲’。

通过官府的技术优势,来逼迫民营铁器作坊改进制造工艺,强行推动大宋的钢铁制造业的发展,进而带动整条产业链,这是韩冈希望能看到的未来。

纵使韩冈的期盼,会有各种各样的原因不能顺利展开。可只要官营铁坊开始打造民间铁器,铁制农具的大批量生产将是顺理成章,不会有半点阻碍。到时候农具的价格大幅度降低,也会促进农业生产,给国家带来极大的利益。

铁与血是国家之本,西方名相俾斯麦的话,韩冈有着深刻的体会和认同。

只不过这个道理,韩冈没办法当着天子的面说出来——对于机械制造技术,朝廷看得很紧,唯恐会被敌人偷学了去。韩冈自知无法说服赵顼将各种机械公布于众。即便要民间要制造出来也不是什么难事,天子也不可能会答应的。

苏颂见到韩冈如此自信,心里暗叹一声,也不欲再多言。

回头看看笼罩暮色中的宫室,一座座殿宇顶端的琉璃瓦在夕阳下,泛着的赤金色光泽。厚重的色调,有着难以以言语描述的庄严,暮鼓此时正好响起,沉重的鼓音带着回响,更增添了宫廷的。

苏缄此时还留在崇政殿中受着天子的询问,想必正在说着交趾和邕州之事。他的这位堂叔,还有些地方要借重韩冈的军器监,想了一想,便有忍不住提醒了一句:“玉昆,还是要小心。许多事,并没有你想象的那么简单。”

韩冈拱手一礼,“学士放心,韩冈会小心行事。”

砸人饭碗怎么可能没有反弹?但制铁工艺的进步,使得军器监的铁匠有一多半失去了职位。为了安置这些多余出来的工匠,也就只能委屈一下的汴河上官营水磨工坊的从业人员了。

出了宫,辞别了苏颂,韩冈本准备去军器监中看了一下情况,就直接回家。只是刚到军器监,还没坐稳,吕惠卿就派了人带了正式的信笺,来邀请他过府一叙。

身在官场,许多事就身不由己。而且从吕惠卿的短笺中,韩冈也看到一丝让他视而不见的消息,也只能放弃与家人坐在一起吃饭的计划,而先往吕惠卿的参政府上行去。

这个时候,吕惠卿和吕升卿正在府中等着韩冈的到来。

吕升卿的脸上,此时有着浓浓的不情愿。作为一国副相的弟弟,他已经很少有这样的神情:“此事当真要靠着韩冈?!”

吕惠卿不喜欢弟弟的说法,端起茶盏的手用上了一点气力,手背上青筋浮凸了出来,“他是王介甫的女婿,轮不到他置身事外。”

“韩冈可是从来都是喜欢站干岸的,一门心思就是格物致知。之前也是……”

“韩冈没这么糊涂,”吕惠卿用力的说着,“用雪橇车运粮的主意究竟是谁出的?而安抚河北流民又是谁做的?别看韩冈看上去始终不肯归附,但真正遇上会动摇到王介甫的时候,他可比谁都卖力。”

吕惠卿虽然说得煞有介事,可吕升卿总觉得自己的兄长似乎是在隐瞒着什么,给出的理由虽然充分,但完全不合吕惠卿的性格。

“李逢案当真会牵连到王介甫身上?”

“不是会不会,而是已经牵连上了。知会江宁已经来不及,这个时候不通知韩玉昆这位宰相家的东床快婿,难道还要让我一人出面去顶着吗?”

兄弟俩正说话间,门外急声来报,说起居舍人韩冈已到。

“快请!”吕惠卿说着站起身来,步出厅门,降阶相迎。

