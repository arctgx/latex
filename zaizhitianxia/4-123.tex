\section{第19章 萧萧马鸣乱真伪(九)}

【晚上有推不掉的应酬,回来时就已经是半夜了,很抱歉今天晚上只有一更。明天中午时会将下一章补上。】

已经是丰州。

用了两天的时间,穿过了几座被党项人刻意放弃的寨堡和烽火台,顺着河流形成的谷道,丰州城已是遥遥在望。

‘再退下去就是到了丰州城下。’折可适想着。

想来党项人应该不会太过相信他们自己的守城能力,在官军的攻击下,丰州城能守住三天就是奇迹了。与其在城池攻防战上决定战事的胜负,不如在野外决出个高下——这是指他们没有其他阴谋诡计的情况,在丰州的这一片地,也都是如同鄜延路一般的千丘万壑,一路行来,经过的支谷岔道无数,而折可适派出去的游骑,着重搜查的就是这些地方——但不管怎么说,战斗也的确是越来越近了。

抬起略感沉重的双脚,折可适维稳当当的走在通往丰州城的山道上。为了节省战马的脚力,除去了散在外面的斥候游骑,宋军的骑兵都是牵着马在走路。

脚步声连绵起伏,不徐不疾,有着稳定的节奏,蕴含着一股紧紧绷起的张力。两千多马步军,簇拥着折可适的将旗,前后拉出来一条长达一里多的人龙。不过整支队伍前后衔接的紧密,折可适又特意多派了人手去盯着前后左右,并不虞被敌军突袭。

正在这么想着的时候,一声尖厉刺耳的木笛警哨猝然响起。这声警哨响得十分突兀,原本就绷紧了神经的宋军将士,一下就握紧了手上的刀枪。

折可适翻身上马,踩着马镫转头看着声音传来的方向,那是一条并不算宽阔的支谷。一名骑兵拼命的从支谷中狂奔了出来,而立于支谷谷口前端高处的两名哨兵,用着更大的气力急促的吹着口中的木笛。

折可适所率领的作为前锋的两千兵马,刚刚好就从这条支谷谷口前穿过。如果当真给藏身在谷中的敌军杀出来,正在行军的队伍登时就会被拦腰冲断。

“全军止步!”折可适一声吼。令行禁止,军令从前传到后,只是走了几步的时间,正在逶迤向前的大军顿时停止了前进的脚步,“陈四,折成康,你二人速领本部上山!封助谷口。”

“诺!”被点到的两名指挥使回头一声大吼,“儿郎们,跟着俺上去!”

身处在谷口前后两个指挥的七百多名士兵,立刻就随着他们的指挥使了支谷谷口两侧的山坡。

“前后各自列阵!遇上贼军就给我坚守住。”折可适又同时向前军、后军派出了传令亲兵。

领头在前的一个指挥、殿后一个指挥,在听到号令之后,立刻列阵站定,将手上的重弩上弦,弩弓上闪烁着银光的箭簇。他们前后各守一边,虽然是分散了折可适手上本就不多的军力,但防着前方和后方受到夹击,则更为重要。

大地震撼了起来,一些细碎的沙砾从山坡上滑下,沉闷的重音自远处传到脚下,折可适面前的战马,正不安的转着耳朵。

折可适打了个响鞭,一纵坐骑,回头赶到了谷口处。一望谷中,他便回头大喊道:“李铁脚,你好了没有?”

不过就这么片刻的时间,分镇山道上前后两端的两个指挥已经列下阵势,陈四和折成康也率部攀上了并不高峻的山坡,一群士兵正手忙脚乱的给重弩上弦。

最后一队,最接近折可适的将旗,也是折家最精锐的指挥之一,总计接近五百人的队伍,也已经排了作战阵型,直面谷口,正急匆匆给自己套上原本让骡马背着的甲胄。

简易型的板甲,只是由数片铁板组成,穿戴在身上也容易,事先练习的士兵两个人互相帮忙,转眼便已经整装完毕——只要不慌乱,这些临战前的准备,折家的子弟兵们一般都能很迅速完成。

将折可适竖立在谷口外的将旗掩护在身后,四百六十多名步兵分作数列站在谷口,横过来挡住了从支谷中出来的道路。一具具银闪闪的铁甲缀连在一起,仿佛一座坚实无比的壁垒,午后阳光的照耀下,一条流光组成的银龙就在铁甲壁垒上游动。

“好了!”当最后一名士兵开始敲着胸甲在自己位置上站定,李铁脚向着折可适吼了回去,但他的声音刚刚出口,一下就消没在敌军的蹄声中。

也就是一瞬间的事,党项人引以为豪的铁鹞子从蜿蜒曲折的谷地深处转了出来。原本只是在山谷间回荡的万马奔腾,瞬间之后就在耳边鸣响。他们来得很快,从哨探发出警报到他们冲到谷口就只用了半刻钟的时间,如同黄河怒涛一般的蹄音从谷底奔流而出。数以百计的党项骑兵乌压压的一片,淹没了整条谷地。

潮水向着堤岸涌来,封堵在前方的宋军一身上下的全副铁甲,的确让人一望便知是难得的精锐,但薄弱的战列、寥寥无几的数目,却让党项骑兵们更加兴奋的踢着马腹。防守阵线人数太少,在来势汹涌的铁骑冲击之下,也应当是犹如鸡蛋壳一般脆弱。

已经严阵以待的宋军,迎接即将到来的冲击。

身穿着板甲,堵在谷口的指挥就是折家的子弟兵。上溯数代都是生活在府州,其中任何一人都能与折可适拉上千丝万缕的关联。手持沉重的斩马刀,全长近五尺,刀身长达三尺有余的重型兵器,紧紧的握在他们的双手间。随着呼吸,宽长的锋刃轻轻晃动,但他们脸上的神情却没有半点动摇。

“射!”

两声号令,自谷口两侧的山坡上同时传出,神臂弓发射时独特的弦鸣穿插在奔雷般的马蹄声中。高高低低站在山坡上的弩手齐齐扣下了牙发,但下落的箭矢只保持了应有的威力,却远没有组成箭阵时的整齐,也失去了力遏敌军的能力。

只是在山坡上威力十足却分外凌乱的射击下,最前面的几十名铁鹞子还是一个接一个的滚翻在地。尽管紧随在后的骑兵越过他们继续前进,但这一多达千余骑的铁鹞子的冲锋势头,已经一点点的被压制下来。

只要慢下来就已经足够了。

随军同行的战鼓就在折可适的耳边敲响,精赤着上身的鼓手,用粗壮如树干的手腕抡起鼓槌,稳稳的敲打着鼓皮,一点点的加快了速度。随着渐渐激昂起来的鼓音,迎着越来越近的敌骑反冲而上,雪亮的斩马刀一时高举如林。

“杀!”

来自于敌对双方,数百人同时暴喝出声,不同的词汇,却能将同一个念头吼叫出来。

铁鹞子挥舞着手中铁鞭和长刀,但面对着面的宋军,却是毫不畏惧的用斩马刀自空中猛力劈下。

带着呼啸全力挥砍下来的锋锐,比起同时挥击下来的铁鞭长刀更快一步砍在目标身上。人的惨叫和马的嘶鸣同时响起,横挡在谷口的宋军阵线依然严整,而对面的党项骑兵则只剩下混乱。

犀利的长刀,让阵前的铁鹞子完全无法抵抗,似乎是几百年前隋唐陌刀阵的再现,一次合击,就将冲到阵前的铁鹞子化为满地的肉块。满目的鲜红仿佛一幅用错了颜色的泼墨山水,抹在了地面上。浓郁得会让人作呕的血腥气出现在战场,可是这样的血腥气却能让造成这一切的一方更加血脉沸腾。

若是在往常,仅仅这样的一击,就能彻底击碎党项军的攻势,可紧跟在后的铁鹞子们似乎没有看到前面的这一幕,依然前赴后继。

一股让折可适的脖子后面的寒毛都倒竖起来的危机感,顿时从心中涌起,凉意由前胸传到后背。“李铁脚!”他突然叫道,“动手快一点,不要站着不动,给我反压回去!”

在千百人同时发出震撼灵魂的吼声的时候,折可适的号令完全穿不进前方混乱的声场,只是随即变了节奏的战鼓,却顺利的让正承受着冲击的官军开始了前进。

一步。

两步。

三步。

斩马刀逆势向前,汹涌的洪流在刀下变成了血红色。

刀起刀落,锋刃上的寒芒在起落间依然闪亮,仿佛张开利齿的巨兽,在吞噬着挡在去路上的猎物。

仍欲奋勇冲锋的铁鹞子被无可阻挡的陌刀阵反压了回去。这时自宋军的来路上,又是一声警哨猝然响起,传入折可适的耳中。

“果然来了!”回望来路,折可适喃喃出声。

当时从小道绕行而来,有三四百名骑兵直奔宋军后阵。奔驰之速,尤胜从支谷中冲杀出来的过千主力。

折可适甩手将自己的腰刀,递给自己的亲卫,“传本将军令,临阵非令不得退,妄退者力斩!”

接下腰刀的亲卫眼中满是疑惑,李铁脚那里已经反压回去了,断后的已经列阵,哪里还需要使用督战队的时候,

折可适却知道,眼下必须要用了。既然眼前已经出现了差不多两千名骑兵,那么藏在他们背后的只会更多。摆明了要歼灭他这一支前锋的党项人,绝不会仅仅只安排下两支队伍,再从山间窜出四五队都有可能。

