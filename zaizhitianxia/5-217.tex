\section{第24章 缭垣斜压紫云低(九)}

茫然看着祝酒的金杯从手中滑落,赵顼一时间不明白到底发生了什么,心中满是困惑。

但在赵顼身边担任宿卫和引导的王中正和石得一两人的眼里,天子的脸在陡然间变得僵硬,变得怪异而扭曲,最后定格在一种让两人毛骨悚然的神情。

“官家。”石得一抢前一步,弯腰捡起金杯,凑近了观察赵顼的神色。

只是石得一这一看,顿时就是分开八块顶阳骨,一盆冰水浇下来,从头顶冷到了脚跟。只觉得整条脊梁骨都像是变成了冰柱一般。正捡拾起金杯的手也像是抽了筋,刚刚拿起来的金杯,又砰地一声落到地。

赵顼眼神中透着惶惑,为什么眼前变得模糊起来,为什么有人就在身边说话,却听不清他们到底的再说些什么。

大宋天子的嘴张了开来,双唇哆嗦着,想说些什么,却是一个字也说不出来,仅仅是在喉间发出暗哑的咕哝。

耳边有如蚊蝇环绕,抢到近前的两人似乎是石得一和王中正,但眼前就像是蒙了一层纱,也分辨不清到底是是不是他们。

难道是中风?!

赵顼渐渐变得浑浊的头脑中,却有一道灵光闪过,终于想明白了到底出了什么事。只是赵顼宁可自己没有想明白。

眼前的视野忽然歪斜,赵顼并没有感觉到自己失去了平衡,可越来越近的地面清楚地告诉他,自己的确是摔倒了。

当赵顼从御榻翻倒的时候,殿下的朝臣们终于觉察到大事不妙。并不是金杯脱手的小小意外,而是很可能是要人性命的重症。

殿一时间没了杂音,文武百官连大气也不敢喘,只是紧张的望着台陛的天子。

还能将郊祀后的宫宴主持下去吗?

心中的恐惧如同潮水一般涌来,想将他埋入黑暗之中。赵顼的意识拼命的挣扎着。可他的挣扎,就像是陷入了蛛网的飞虫,完全没有达到应有的目的。赵顼并不是在一瞬间就失去意识,而是清晰感受到自己的身体已经无法控制,在明白了自己到底出了什么事的情况下,意识才一点点的开始模糊起来,只有对死亡恐惧留存。

被王中正扶住的天子,看模样已经不可能继续方才的任务。王中正和石得一对视一眼,对方的想法都已经了然于心。

“扶官家回内殿。”石得一说道。而不论是王中正,还是其他服侍在侧的内侍,完全没有反对的意见。

赵顼被搀扶进去的那一刻,让所有在场的官员都感到风雨欲来的危机感,极浓极重。不止一人将视线投向赵顼的两个亲弟弟。赵頵倒也罢了,跟其他望着内殿的官员差不多的反应。赵颢低头看着眼前的桌面,动也不动一下。可任谁也知道,他心里面还不知如何敲锣打鼓,兴奋得无以名状。

宴会怎么办?

天子还没有让皇子出来奉酒,预定中的程序没有完成,那么请皇子出阁读的奏章到底要不要递去?已经有很多人开始犹豫了。

如果皇帝还能恢复,肯定不会有人起异心。但赵顼的病可不是感冒发烧那般轻易,几乎是无药可救的,让人们没有了太多的顾忌。

手足麻痹,口不能言,这是典型的中风症状。

在赵家的前五位天子中,因风疾而不能理事的不是一个两个。真宗、仁宗、英宗,都是风疾而沉疴不起。

大庆殿中的文武百官里面,深悉医理的至少有十分之一,具备些许基本的医学常识的则能有一半。而什么是中风,几乎每一个人都有这个见识。

天子的离席,不仅仅是给朝堂蒙一层阴影那么简单了。

很多人还记得,就在几年前,当今的皇帝似乎曾经有过一次疑似中风的发病。一次中风还不一定致命,但两次、三次中风,可就跟一道道走过鬼门关一样,鲜有能撑过去的。

宴会的主人离开了,剩下的客人全都陷入了。这个时候,宰相应该站出来收拾局面了。章敦盯着斜对面的王珪,打着眼神催促王珪。但王珪根本就不跟其他人对眼,只顾伸长脖子望着通往内殿的小门。

章敦狠狠地咬紧牙。不能挺身而出,稳定局面,这还配做宰相吗?换做是自己,绝对不会做出这样的蠢事。

不知过了多久,内殿中终于有了动静,王中正匆匆从殿中出来,站到台陛下,“太后有旨,着王珪主席。”

王中正再没有别的话,王珪起身领命。有了吩咐,他就敢做事了。

只有王珪的主持,自然不可能让延安郡王赵佣出来面见朝臣。只用了小半个时辰,一场耗费巨大人数众多的宫宴便匆匆结束,臣子们从大庆殿中鱼贯而出。

但解散了宫宴,却并不是所有人都离开了皇城。原本宫宴结束后官员们就该四散返家——在开封,冬至日是一年中仅次于正旦的大节日,就算皇帝也不便耽搁臣僚们想早点回家与家人相聚的心思——可是今天却有许多人因为各种各样的理由想要等个结果而滞留在皇城中。

皇城中官员们的神色,完美的诠释了什么叫做人心惶惶。天子到底能不能撑过去,可是事关他们命运和前途的关键。

韩冈也没有离开皇城,而是直接返回了太常寺。太医局中的几名医官都已经被召去了福宁殿,为天子诊治。要有什么消息,这里是消息灵通仅次于两府的地方。而且韩冈相信,他肯定会被召进内宫,在太常寺这边等着最合适。

从架抽出一部有关生物学的科普读物的手稿,韩冈气定神闲的校对起来。中风不是心脏病,就算一病不起,至少也有两三天的时间做缓冲,总会有办法让局面不至于落到最坏的地步。

也正如韩冈所预料,刚刚坐下来没半个时辰,宫内派了人出来,请韩冈入宫中。来人是赵顼身边的内侍,虽然名字不清楚,但相貌很面熟,这也让韩冈多放了点心。

