\section{第18章 向来问道渺多岐(二)}

“这就是生物树?” 

苏颂惊讶抬头看着张挂在正厅中的图画,不仅是因为第一次见到用图示来分类的手法,也更因为韩冈的分类条目别出心裁,太过有新意。 

在诏书发出去的半个月后,苏颂便抵达了京城——这也是亳州距开封不过数百里的缘故——并来了《本草纲目》编修局中报到就任。而韩冈也不得不再一次向他人解释生物分类学的基本概念。 

动物植物两株树,每一株树从下向上都分出多支枝桠,而每一支枝桠也是不断的分岔再分岔。 

主要的枝桠是门,次一级的纲,再往下,便是目、科、属、种。 

植物树上的主枝,是种子植物门,蕨类植物门,苔藓植物门、藻类植物门。动物树上则是脊椎动物门,节肢动物门,软体动物门,环节动物门,原生动物门。 

韩冈编订的分类跟后世的并不完全一致,但与这个时代对生物的了解相适应,也更容易解释。只要先把框架搭起来,日后修改那是日后的事。 

而苏颂有些瞠目结舌。将两幅画从墙上拿下来看了之后,上面分出来的枝杈怕不有数百上千,未免太详细了一点。不过再看小字,其实写了字的枝杈在其中只占了一小部分,大多数还是空白,等着填空。 

苏颂仔细看着两幅图。他在动物树最上面的一条小枝上发现了猩猩两个字,沿着这条小枝回溯上去,便是猿属,猿种,回溯就是灵长目,在灵长目这条枝桠上,有猴,有狨,有狒狒等一条条分岔,而灵长目再回溯,则是哺乳纲,哺乳纲向上,便是脊椎动物门。在脊椎动物门的分支中尚有全是鸟雀的禽纲,聚集了蛇蜥的爬行纲,蛙类的两栖纲,以及鱼纲。 

这些纲目的命名,让人一见之下,就能会然于心——也就无足的蛇为主的纲,怎么起名做爬行纲让人费解。 

再看植物树,也同样是清晰明白。 

这绝不可能是韩冈一时兴起的答案,肯定是积累了多少年后才积累起来的成果。韩冈还不到三十啊,这些积累究竟是从哪里来的?难道当真是天授不成。那样可就是跟圣人一般了——圣人生而知之,贤人都少不了要向人学习。 

“就像书籍编目,经史子集只是大范围。想要能够详检,必须就必须分得更细一层。就拿史部来说,断代的《汉书》等诸朝国史;编年的《春秋》诸传,以及《资治通鉴》;国别体的《战国策》……《三国志》其实也可以算是国别体。”韩冈打着比喻,向苏颂解释着他的分类如此详细的缘故,“再譬如地理,路、州、县、乡,一层层下来,将幅员万里的大宋,划分的一清二楚。划分得越细,方剂中,一些药材的替代使用也就方便了许多。” 

“玉昆,这个道理愚兄也是明白。但如此分类,总得有个缘故,有个由头。为何要这样分,这样分类的道理是如何来的。而且药材不仅仅是草木虫鸟鱼兽,也有金、土、水之属,丹砂、水银、无根水,这些又如何归类?”苏颂跟韩冈交情匪浅,说起话来也不需要避忌,可以放心直言。 

药材有生物和矿物之分,不过还是以草木为主,所以有本草之名。这是没话说的。但到底要怎么分,以什么规则来分,就是韩冈要在《本草纲目》中解释的。而韩冈也算是胸有成竹。 

“动物、植物的划分,生物树的由来,不过是对草木虫兽本质特征的归纳和分门别类,比如被子植物门下面的单子叶纲和双子叶纲,看看种子就可以明白了。麦、稻、蜀黍【高粱】,吃到嘴里都是一粒一粒的,发芽时,也是单片叶出来。而豆菽,一粒便是两瓣。而这个柑橘的种子,拨开外皮,也是两瓣。”韩冈就在桌上,将一个温州柑橘剥开,弄了一颗种子出来,分开来给苏颂看,“这样的种子发芽时,便是这两瓣子叶先出来……其实只要将黄豆和稻子泡在水里,一看就知道了。” 

韩冈喝了口茶,润了润喉咙,见苏颂凝神细听,便又继续说道,“至于金、土、水之属,也有元素论在。比如绿矾,那是铁属。胆矾,则是铜属。所以胆水炼铜后,得到就是绿矾水。至于丹砂,乃是水银属,炼制水银,便少不了丹砂。而用硫磺兑水银,又能生成丹砂,可见其实质上是硫汞齐。” 

韩冈是想将生物学暂时纳入其中,将药材的原材料给分门别类。不过顺便将化学的元素论掺入其中,也是一桩好事。 

苏颂沉吟了许久之后,轻轻点了点头。但很快他又质疑起来:“只是玉昆你将动物、植物以门纲目科属种六个等级来划分,一层层的分类下去,是不是太多了一点?天地万物,就算只将其中一成给编目考订下来,都不是几十年就能完成的事。玉昆,这么做未免有些贪大了。” 

“分其类属,明其源流,使世人不至为谬误所惑,这是韩冈的本意。不过《本草纲目》是药典,也只需将已经运用在方剂中的药材给分类。至于其他的动物、植物和矿物的分类,得等日后慢慢来,韩冈并没有打算一次就能尽百年之功。那样未免太自大了,韩冈自知非是圣贤,做不到这一点。” 

具体的细分类,韩冈虽然头疼,但只要将规则定下,也就足够了。来自于后世的记忆虽然都是粗浅,但那也是数百年无数人心血的结晶。其中的道理,只要解说明白,说服大部分人绝不会有问题。韩冈要做的就是提出原则,展示范例,剩下的就让后人去补充。而《本草纲目》这部药典,正是韩冈要展示的范例。 

苏颂垂着眼,细细想着韩冈的这一番话。 

韩冈说的话,苏颂当然明白。但韩冈的行事作风他更明白,拿到表面上的,永远只是冰山一角。就像他在浮力追源中所说的,浮冰藏在水面下的部分,占到了九成。 

韩冈真正的用心,绝不仅仅是编纂药典这么简单。一石二鸟、三鸟都是在他的计算之中,板甲、飞船就是最好的例子。 

苏颂抬起眼,瞅着三尺外那恬淡平和的微笑,却想着在这一微笑之下,到底藏了多少心机。 

…………………… 

“韩玉昆所谋甚大?”杨时眉心紧皱,“敢问先生此言何解?” 

窗外夜风习习,已是近秋时节,白天的暑热被夜风一扫而空,不再像半个月前一样,到了夜间,也依然闷热难耐。 

秋天终于到了啊。 

程颢从窗外的婆娑树影上将视线收了回来,看着房中的游酢、杨时、谢良佐、吕大临四人。游、杨、谢三人要么紧锁着眉,要么一脸疑惑,都想不透韩冈,只有吕大临板着脸,一语不发。 

“与叔最是了解韩玉昆脾性。”程颢引着吕大临说话,“想必是了然于胸了。” 

“吴郡陆玑的《诗疏》。”吕大临惜字如金。 

简称《诗疏》的《毛诗草木鸟兽虫鱼疏》,出自东晋乌程令陆玑之手,乃是研习《诗经》的主要注疏之一,专门针对《诗经》中提到的动植物进行注解。杨时和谢良佐好歹也是贯通五经的儒者,自是早已研习通透,但他们却不明白吕大临此言何意,与韩冈的图谋又有何干。只有游酢身子一震,像是受到了启发,想到了答案。 

将众弟子的神色收入眼中,程颢呵的轻声一笑,看了看似乎已经明白过来的游酢。游酢随即会意,对杨时和谢良佐道:“不知中立、显道是否读过韩玉昆的《桂窗丛谈》。” 

“当然。”虽然是对立学派的著作,但也只有去研习,才能揪出其中的破绽加以驳斥。 

“那其中的‘螟蛉之子’一条呢?” 

“啊!”游酢出言点破,杨时和谢良佐顿时恍然。 

杨时一捶掌心,“原来如此!” 

谢良佐也失声惊道:“好个韩冈!” 

吕大临沉着脸:“韩冈的心思一贯的深沉难测,不等到他揭开谜底,很难看得清他的全部用意。不过从过去他的行事上,倒也能猜个五六分出来。诗经中,论及草木一百一十四种,鸟兽虫鱼六十种,螟蛉和蜾蠃可仅仅是其中之二!” 

吕大临声音沉甸甸的压着人的五脏六腑,韩冈一贯的喜欢釜底抽薪,起意编修药典,也算是他惯用的手段。 

“王介甫这一回进《字说》,其中当多有其婿之力。韩玉昆将格物致知的手段发挥到淋漓尽致,这一次也不会例外。”一直默不作声,盘膝静坐榻上的程颐忽然开口,“但根本还是《易》。《诗》、《书》虽重,但论天地之本源,天道之理,毕竟都比不上《易》。” 

为了应对越来越激烈的学派之争,二程这一回已经将他们对《周易》的诠释编纂成书,名为《易传》。可是要与新学、气学,一争高下。 

