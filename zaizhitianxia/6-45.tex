\section{第六章 见说崇山放四凶(11)}

王厚拖着疲惫的脚步,从皇城中走了出来。

穿过宣德门那深长幽暗的门洞,阳光洒下的时候,他不由的眯起了眼。

一天多没睡,连吃饭喝水也只是抽空,当骤得大任的王厚全心全力完成了任务,并像太后进行了禀报之后,剩下的就只是疲惫。

困倦难当,连头脑也变得迟钝起来,思绪仿佛落进了泥潭,全力挣扎也改变不了越来越吃力的结局。只有空空如也的肚皮,还能清晰明亮的发出饥饿的声音。

“二郎,要回去?”

牵着马过来的是服侍王厚多年的亲随,等到王厚终于出门,便立刻迎了上来

“……回哪里去?”

王厚用力揉着额头,然后反问。

“二郎,可有想去的地方?”

王厚正在考虑,不过还没等他得到结论,就有一群人涌了上来。

‘上阁!’

‘皇城!’

都是在称呼王厚,不过其中一半和另一半并不一样。不过不管怎么称呼,都一样是王厚。

西上阁门使,提举皇城司,并不怎么符合官制,但为了酬奖王厚的功劳,同时当时更多的也是为了让王厚能更名正言顺的统领皇城司的成员,让他们戴罪立功,宰辅们没有人对此表示反对。

不过这并不是让王厚在做阁门使的同时管理皇城司,仅仅是让他就任皇城司的主官。

阁门使即是实职的差遣,也是武官序列中的一个阶级。

王厚原本是要就任阁门使,但本官阶级依然还是在正七品的诸司使一级,可现在因为宫变一案中的功劳,却变成了就任提举皇城司,也就是说,随着王厚就任皇城司,他的西上阁门使从实职差遣变成了官阶。

尽管听起来乱得让人摸不着头脑。但这只意味着一件事,就是王厚直接跨进了横班,成为了大宋百万军中仅有二十位的高阶将领中的一员,最顶层的三衙管军就在身前不远。而以王厚的年龄、功绩、背景,他日后晋升三衙管军也不在话下。

就因为王厚前途无量,赶上来奉承的官员便争先恐后。只是王厚此时头昏得不行,肚子也饿得难受,几句话甩开了这帮人,便快马离开,转了几条街巷,在一僻静的小巷中停了下来。

王厚就在马上脱了官袍,借了一名亲卫身上的衣袍和帽子,打发了这人拿着官服先回去,他本人则带着剩下的几个亲随出了巷口,在路对面找了家酒店坐了下来。

点了酒菜,王厚刚拿起筷子,就听见隔邻的桌上有人高谈阔论。细细一听,不仅是这一桌,就连周围的几桌所议论的,都是昨日的大庆宫变。

从宫变当日开始,持续了一天一夜的搜捕,到了第二天才宣告结束。

并不是没有漏网之鱼,不过比虾米大不了多少的小鱼,就算是跑了也无足轻重。而且开封府又开出了海捕文书,其中的绝大部分,都很难逃出开封府的地界。

也就是到了这时候,有关宫变的细节方才在京城中传播开来。但真相混淆在谣言中,传得漫天飞。不过有一点不会变,第一,宰相在大庆殿上被干掉了,第二,解决他的是韩冈。

韩冈的名气本来就无人不知,这一回再次扬名。可换来的不是顶礼膜拜,而是市井中兴致高昂的高谈阔论。

也许在上层是攸关生死,韩冈是死中求活。其一骨朵击毙蔡确,虽有武力的成分,但更多的还是其眼光和决断力的体现。可是到了下面,他如何做翻了蔡确,倒成了百姓们关心的重点。

王厚从来没有想过在大庆殿上的惊险一幕,最后能变成喜剧或是武戏。

当他听到旁边有人在说,韩冈拿着一柄先帝临终前秘密赐下的金骨朵,上打昏君,下打奸臣。一锤击毙想要谋反的奸相,又逼退了想要篡夺侄儿皇位的奸王,还有偏心又老糊涂的太皇太后,便连酒杯都放下了,就竖着耳朵听人说书。

“小韩相公那两条胳膊可是有千斤之力,力能扛鼎,一把扯定了那奸相,一锤下去那就是红的白的一起迸了出来。虽说奸相被小韩相公一锤砸碎了脑壳,但班直都不甘心,他们人多势众,小韩相公就一个人。殊不知小韩相公那是上界大仙转世,身具神威。只一声大喝,便吓得数万皇城内的班直都惊破了胆。吓趴下都有一大批,大庆殿里从逆的那些禁卫,一个个都吓得屎尿横流,臭气熏天。”

王厚听得直摇头,这编造得实在是太离谱了。但他却依然安坐不动,听着边上的乐子。

“小韩相公那是何等人?在考进士之前,在关西是打遍了八百里渭水上下无敌手,又认识了一群兄弟,喝过酒,烧过黄纸,斩过鸡头,要不然故去的王枢密会千金礼聘小韩相公做军师?一是小韩相公文武双全,又通医道,二是小韩相公能打的兄弟多。飞矛的李将军,连珠神射的王团练,还有那个赵……赵……赵将军,都是了不得的高手。”

王厚低着头,差点没将酒杯给咬下一块来。忍住笑是在是太难了,就是牙齿咬着银杯,呼呼的笑得身子直抖。

“小韩相公就在殿上将衣服脱了,那刺青如锦缎,从胸口延到背后,殿上上上下下那都是看得眼呆。说时迟那时快啊,小韩相公一把抢上前,拿住奸相就做了个跌法,将那奸相摔在了地上做马趴。一脚踩定了奸相,这才挥起金骨朵,把那奸相打了个三千桃花开。”

王厚用手压着胸口,都快喘不上气来。这是喜欢相扑争交的,相扑那是打架前先脱衣,光着膀子只裹一条兜裆布,所以女相扑在京城中那么受欢迎,韩冈在打杀蔡确前先脱衣,这不是相扑是什么?

“难道不知韩相公的外公那是西北有数的名将?曾与狄公并肩杀贼。家传的飞矛之术,那可是飞将军李广传下来的……别插话,俺难道会不知飞将那是箭术如神,连珠箭如纸上贴花,一贴接一贴,旁人想插上一贴都插不上。”

“只是飞将军的有个儿子名唤李敢,不幸在阵上伤了一条胳膊,不能再使箭。所以便苦研飞矛,这日夜苦练,本又有远射的天分,终于给他练成了,从此跟着冠军侯南征北战,立下了赫赫功名,还封了侯……什么,李敢是冠军侯杀的?别胡扯,那姓司马的就会胡说。前回从洛阳来了一个司马缸,挖了地洞在里面写书,又在殿上一通乱说话,被小韩相公一眼就看出了他其实是发了疯!”

“说到哪儿了?……啊对了,李侯练成了飞矛之术后,就一代传一代,就这么流传了下来。一直传到了小韩相公的表兄李将军手上。这李将军有个名号,唤作小飞将,可不就是这么来的。”

“想那小飞将那是何等英雄人物?一杆飞矛,杀得西贼和南蛮子哭爹喊娘,就是跟辽狗厮杀起来,也没落多少下风。”

“只可惜这等秘技是传子不传女,所以小韩相公都没能学到,否则一飞锤砸碎那奸相的狗头岂不省事?还要冲过去打。”

“而且你们可知道那飞将军的箭术传给了何人?……没错,就是新近平了西域的王团练!王团练那靶上插花可比绣花快上千百倍,眨眨眼的功夫就用箭在靶子上钉出了一朵花来了。所以他们才会在一起出来辅佐小韩相公,这就是缘定千秋,传遗百代。”

这又是讲古的,水平远超周围。王厚听得兴起,肚子也不饿了,却是笑疼了。捂着肚子,趴在桌上,他倒是想看看韩冈听到这些传言后,会什么什么样的表情。

……………………

“一声喝退数万班直?”

听到家中妻妾的转述,韩冈好悬没大笑出声。

现在那些谣言散布者,都是在过过嘴皮子上的瘾,扯淡的时候也没必要保证真实性。但离谱得也未免太过分了一点。

要真是有数万班直,不要他们造反,三司的吕嘉问就要先造反了。

天下有官品的文武官加一起才多少?四万多点,五万少点。

宫中班直禁卫的俸禄,可不比入流的文武官差到哪里。若是这样高薪资高福利高待遇的班直有个三五万,朝臣就要去喝西北风。

还有那李信、王舜臣缘出一系,更是让韩冈笑得没了形象。终于是知道天波杨府的媳妇是怎么一个个披挂上阵的了。

不过外面一说起殿上事,都少不了那支涂金铁骨朵参与。不论哪个段子,都会绘声绘色说一通金骨朵怎么敲碎了奸相脑壳。

要是能拿回来就好了。韩冈想着。

如果韩冈能拿回骨朵,再在上签个名,再写上‘元佑元年二月丁丑,格毙蔡逆于大庆殿上’,包管日后价值连城,若能让太后也顺手签个字,变成了御赐之物,那就更有历史意义了。

到了韩冈这样的地位,这样的身家,寻常的古物珍玩都不会放在眼里。而韩冈本人,尽管连珍惜的古董珍器也不放在心上,但想到能给后人留一个传奇般的国宝,也免不了会暗快于心。

只可惜铁骨朵是宫中御龙骨朵子直的武器,不是可以拿出宫的纪念品,韩冈也没好意思收在自己的身上带出宫去。不过真要说起来,就是光明正大拿在手中,韩冈照样能够大摇大摆的出城,没人敢拦着他。

终究还是脸皮薄,没能把事情做出来,让‘上打昏君,下打奸臣’的铁骨朵遗失在宫中。韩冈对此是深表遗憾。

笑话传遍了城中,但朝堂上则是正经八百的开始讨论如何封赏有功群臣。

尽管还没有最终结果,不过韩冈已经确定要晋封国公,并不是曾经坚辞不就的莱国公,而是齐国公。跳过小国、中国,直接晋封大国国公。不为宰相,便为国公,而且是大国国公,这在过去几乎找不到先例。

而王安石则是要在楚国公之外,再加一个国公头衔,是为两国国公。要不是大宋开国以来,臣子没有生封郡王的旧例,王安石应该能够更近一步的。

而后章敦,苏颂,张璪等人都有封赐。这些将会在几天内讨论出最后的结果,然后公诸于众。

看起来已经是收拾后事,可朝堂中人人皆知这只是暴风雨袭来前的平静——只因为韩冈有关如何选择宰辅人选的提议。

王安石和两府宰执都对韩冈的提议没有异议。一下子将拟定宰执人选的权力交给下面的大臣,韩冈的提议,不论哪位宰辅反对,都会成为天下所有侍制以上的官员们憎恨的目标。

所以朝堂上乱成了一锅粥。韩冈在家里却坐得稳如泰山。

不论外面掀起多大的风浪,韩冈也没有改变他的态度。依然四平八稳,仿佛什么都跟他没有关系。

冬季快要过去了,春天已经离之不远。

晴日的午后,没有实职在身的韩冈过得悠闲自在。在后花园假山上的小亭中,一边晒着太阳,一边仔细检查着儿女们的功课

韩钟、韩钲,在韩冈面前毕恭毕敬,静静的等着韩冈对他们功课的评价。而金娘则在不远处,拿着千里镜一样的筒状东西,眼睛贴着其中一头,往里面看进去。

“大姐儿,别玩万花筒了,该学刺绣了!”周南难得板起脸,教训着女儿不要在玩了。

金娘仿佛没听见,依然拿着

“多玩一玩也没什么,小孩子,玩心重。”

听到韩冈这么说,金娘反而不再玩了。嘟着嘴,放下了万花筒。

韩冈笑着让女儿出去学习女红,随手拿起了万花筒。

这是家里才送来的玩具,韩冈之前都没注意。

里面呈三角形放了三块长条形的玻璃银镜,银镜内侧是一些彩色的碎琉璃和云母片。对着阳光的时候。

彩色玻璃还没有确定的配方,但大一点的玻璃工坊都在加以研发,在烧熔的原料中掺入各种矿石粉,试图造出彩色的玻璃来。

万花筒的外观很精致,但更有吸引力的地方,是不断变化永远不会重复的图案。虽然里面的彩色碎片只有十余片,但只要手腕轻轻转动镜筒,就能看到五彩斑斓,繁复又对称的图案。

“官人!”刚刚送走了女儿去学刺绣,周南回头就看见韩冈拿起了万花筒在玩。顿时心中就堵了一口气,“你这让家里的孩子看到了会怎想!”

韩冈随手就放下了,不过仅仅是看了几眼,就已经看到了足够多的东西,这让他对关西制造业的进步十分满意。

虽说万花筒是小孩子的玩具,不过能用玻璃银镜造出这样的玩具,也证明了雍秦地区手工业的水平。什么时候能够造出人工的动力源,那基本上就是工业革命的开始。

“官人还是多想想,方才不是有人回来报称李中丞又去何处走亲访友了。就知道丢下个烂摊子让人收拾,也不想想该怎么做。”周南没好气的说着。

云娘笑道:“现在这样也好啊,等三哥哥做了相公就没那么悠闲了。”

“做相公?那可就难了,得慢慢等。”韩冈摊摊手,“为夫现在连两府都难入。如果今天廷上推举的话,为夫多半会输,做不了头名,甚至可能成不了候选人。”

“……那官人为什么还要献策?”严素心不明白了,“就是直接推辞,太后也不会多生气的。”

“是官人还是不想进两府?”周南问道。

陪伴韩冈多年,周南素知丈夫对清凉伞并不是很放在心上。真正关心的还是气学。推辞东西两府执政的位置,已经不是一次两次。可一遇上道统之争,却分毫不让,皇帝也好,宰相也好,都那他没辙。

若说丈夫这一次为了气学的未来,放弃了唾手可得的宰执之位,周南是一点也不觉得奇怪。

王旖也觉得这时候最好不要进去,俸禄又不会多多少,家里还整天不得安生。又不是误了这一次就再也进不了两府,何苦每次都吃苦受累,韩冈若能做个晏殊一般的太平宰相,那才是王旖最期盼的。

“风尖浪口上,总是要提心吊胆,还不如不做。”

韩冈笑道:“这点风浪,小船会翻,大船可不会。”

“官人方才不是说没人推举,所以选不上吗?”王旖奇怪地问道。

“廷推可是在半月之后!”

“这就不会出岔子了?”

“当然。”韩冈用力的点头。

王旖更加迷惑起来:“为什么?”

半个月时间,难道韩冈还能有什么手段来扭转?可是以她对丈夫的了解,韩冈肯定不会像吕嘉问、李定一般四处奔走,寻找支持者。这样一来,半个月的时间,有资格推举的还是那些人,又怎么可能会有多少变化?

王旖全然不明白,就连周南、素心和云娘也是一脸迷惑的望着丈夫。

韩冈回手指了指自己,问道:“为夫是什么人?”

妻妾们都听得出,韩冈是在询问,而不是自负的反诘。

周南歪了歪头,笑道:“当然官人啊。”

韩冈打了个哈哈:“话不错……不对题。”

“万家生佛,当世师表?”

“如果不笑着说就显得更有诚意了……”韩冈笑了一下,然后摇头:“不是。”

严素心问道:“……是最得太后信任的?”

话问出口她就知道错了。韩冈得到太后的信任,是一以贯之,并不是说半个月后就会有何改变。

而且这个信任在韩家并不是很受欢迎,毕竟这又是一个姓韩的。

所以韩冈还是摇头,“不是。”

王旖不打算猜,直接问道:“官人,到底是什么?”

“是啊,三哥哥,是什么啊?”云娘推着韩冈手臂,催他不要再卖关子。

韩冈微微笑,“为夫是北人。”