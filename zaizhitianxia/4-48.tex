\section{第十章 进退难知走金锣(上)}

重夺罗兀城的兴奋不过数日,紧接着就是当头一棒向着赵顼的脑门上挥来。

丰州失陷完全出乎所有人的意料之外,没人想到西夏敢这么赌上一把。

丰州陷落,得到了充分补给的党项人军势大振,同在黄河西岸的麟府二州如今都有陷落的危险。而且还要提防着契丹,谁也不知道他们会不会趁火打劫。

这是谁的责任?

几乎也是惯例了,当这个噩耗传入京中之后,朝堂上的大臣们,不是想着该如何应对眼前的局势,而是追究责任。

欲要追究守臣失土之罪,但知州高遵路已经战死疆场,连同下面的将校三十七人,还有近两千守军,一同殉国。与高遵裕一样,高遵路也是太后的亲叔叔,既然他已经以身殉国,再加罪也未免太不合人情了。

板子当然首先是要落在府州知州折克柔身上,不管怎么说,他也有失察敌情的罪名。只是也不能深责,朝廷还要靠他收复丰州。

麟府丰三州是折家的地盘,其中居于核心地位的府州,开国百年来全是折家人担任知州。想想韩琦,他三判相州就被说成是朝廷莫大的恩典,而折家盘踞云中之地上百年,却已经被习以为常——在许多宋人的眼中,府州折家那是当地的土官,而不是朝廷派遣的流官。

禁军、义勇和弓箭手加起来接近两万人的麟府军,说极端点就是折家的私军,折家家主对他们的的影响力不在朝廷之下。这在大宋国中,也算是独一份。说到将门中的种家姚家,那都是根基浅薄,跟盘踞麟府一带上百年的折家没法儿比的。

但也是因为这个原因,麟府军换装的序列总是排在最后。神臂弓都没有配足,配发床子弩的记录还是在庆历二年,嵬名元昊领军攻打河东的时候,更别说板甲、斩马刀、飞船这些军器监出产的新玩具,连个样品都没有发过去一件。

为了夺回丰州,这些军器要紧急调拨,河东的兵马也得做好支援的准备。但此时崇政殿中,依然不是在讨论此事。

“此乃陛下误信人言之故!”吴充当初就反对对西夏开战,现在得了罗兀,却丢了丰州,更是让他抓到了把柄。对赵顼一点也不客气,“自熙宁五年息兵以来,陕西、河东三年不见战事,秉常亦自恭顺。陛下误信种谔狂言,兴兵侵夏。须知犬入穷巷,其必反噬。先有秦凤遭袭,西贼破数寨而归,继而又有丰州被攻占。得一孤城,却失一州之地,当可谓之得不偿失。臣请陛下召回大军,调回种谔,以论其罪。”

这一次的战事,天子不顾他这位枢密使的反对,而强行让鄜延路出兵,这枢密使做得还有什么意思?文彦博当年就能将夺下绥德的种谔丢到随州四年,他吴充也不会输人。若以为到了这时候,他还会恋栈权位,不敢直言,就未免太小瞧他吴充了。

赵顼的脸青一阵白一阵。吴充戳到了他心里的伤口上,但他还不能发作,否则有损声名。外面的士人从来都不会留口德,即便是皇帝也一样。

“丰州之事与种谔无关!”

赵顼出言袒护种谔,将吴充的指责堵了回去。他还要灭亡西夏,种谔这样善战的将领,肯定不能少。

吴充心下冷笑,也不言语了。想息事宁人哪有这般容易?御史台的言官们现在应当都在写弹章了,自从侬智高之乱后,国朝再也没有失陷过一座州城。这可是几十年来的第一遭,总得有人出来负责。

“西贼力弱,若尽起河东之军,丰州指日可复。而种谔携胜势溯无定河北上,兵胁银夏。西贼必首尾难顾。”冯京几句话平复了赵顼的坏情绪,只是赵顼刚刚点了一下头,冯京就话锋突然一转:“只不过,万一西贼将丰州献与契丹,如之奈何?”

赵顼脸色更为苍白,若丰州当真落入契丹手中,就如羊入虎口,哪还有夺回来的机会。一时心乱如麻,好半天方才问道:“蔡确现在到了哪里?”

冯京回道:“蔡确只走六日,此时应当还没有到雄州。”

“发金牌急脚,命其兼程而行!”

“陛下!万万不可!”几名宰辅闻言心中大急,齐声阻拦,这事哪里能做得?一时间,两边都忘了党派之分。

王安石连忙道:“越是危殆之时,越是得戒急戒躁。若是被北朝觑透了虚实,必生觊觎之心。北人之欲壑,岂是区区五十万银绢能填?届时必生事端。”

“陛下只需遣人将此事告知蔡确便可。”韩绛也道:“他只是通报攻取罗兀的国信使,丰州之事与其无关。即便辽人索求金银土地,自会遣使来,也轮不到他说话。”

辽国肯定不会想看见灭掉西夏,一旦西夏求到辽主面前,甚至按照冯京所说,将丰州送给辽国。辽国君臣如何会放过这个机会,即便会将丰州送还,也肯定要连皮带骨的狠狠斩上一刀。

“就依韩卿之言。”赵顼点着头。接着又惶惶然的问道,“但眼下河东、陕西两地之事,又该如何处置?”

“如今正值冬日,北方必是大雪封路,交通往来不便。丰州陷落的消息,一时也传不到辽主的耳中,当尽速遣兵夺回丰州。而鄜延路也当继续被上,攻打银夏。不论银夏得与不得,当能令丰州贼军不敢一意坚守。”吕惠卿声音停了一下,“要在辽国出手干涉此事之前!”

这就是有底气和没底气的差别。

只要辽国不插手进来,崇政殿中的君臣并不担心西夏,张玉在甘谷城,种谔在罗兀城,一攻一防两次大捷,都说明了宋军的战力已经远胜西夏。可一对上辽国,谁也不敢说必胜,甚至连作战的信心都没有,连同赵顼也一样。

只能选择躲避。

赵顼静静的闭上眼睛,心头沉甸甸的。都已经八年了,他登基已有八年,可登基时所发的宏愿,依然是镜中水月。究竟什么时候能让他不用再顾忌北虏,出兵北收燕云?

……………………

“朝廷肯定要顾忌北虏的反应。”

“西贼攻打丰州就是为了将辽人拖进这场战事中来,现在肯定已经遣人去通知辽国……不过辽人会趁机勒索,当不会出兵掺和。”

桌上摊开一幅潦草的地图,宋、辽、夏三国尽绘在图上。张载站在桌前,韩冈、苏昞、范育、吕大临这几位得意弟子都在桌边,看着地图议论时局。

张载门下弟子,少有只会说着仁义道德的腐儒,他们的目标都是真正贯通六艺的儒者。为万世开太平并不是指穷兵黩武,但也少不了涉及兵事。即便是吕大临、苏昞这样专注于经义、礼法的儒者,也对诸多兵书倒背如流。

“玉昆说得没错。”苏昞低头看着地图上丰州的所在,虽然很是模糊,但至少大体的位置没有错,“西夏女主外戚当道,国力日渐衰弱。甘谷城下野战参拜,继而又被种子正轻取罗兀城,以西夏现在的困境,也只能求救于契丹。”

“罗兀城是不是西贼故意没有加以防备?”范育问着。

“罗兀城的陷落,其实当也是在党项人的意料之外。”韩冈想了一想,说道,“若是一开始就明知罗兀难守,就算想装个样子,也不会放上几千铁鹞子。那可都是精锐,单是俘获的战马就有整整一千三百匹,是鄜延路如今战马总数的一成半!”

“说得有理。种子正的确是个将才。”苏昞抬头冲韩冈笑了笑,“也有玉昆的功劳在。”

“丰州旧属契丹。太祖开宝二年,其守将千牛卫将军王甲举城来归。不过当时的丰州,其实是在屈野川【今乌兰木伦河】东。归于中国后,便与折家一样,由王甲的子孙世代传承。只是到了庆历元年,被元昊领军夺占,时任知州的王甲曾孙王馀庆战死,之后就再也没有夺回来。现在的丰州,是嘉佑七年,于府州萝泊川掌地复建为州,也就是将旧属府州的古长城以北的地方都划了过去。”

张载对丰州的掌故侃侃而谈。在韩冈的记忆里,当年求学时,张载也在教学中表现了他对陕西、河东的山川地理和历史变迁了如指掌。现在依然能娓娓道来,可见他旧年在这方面到底下了多少功夫。旧年献兵策于范仲淹,也是有所依仗。

“中分府州,重设丰州,其中当也有削弱折家的用意在。”韩冈道。

“初时麟府,有王家分庭抗礼。自丰州陷落后,便是折家一家独大。”张载说到这里便停了口。这等用来制衡臣子的手段,出自于法家,兼有法术势中的术、势二道,他这等纯儒自视看不过眼。避过此事,问韩冈道:“玉昆,朝廷是否已经决定要将丰州夺回?”

“就是今天上午崇政殿中刚定下的。”韩冈点了点头,“丰州肯定要夺回,否则西贼将此州送给辽国,将辽人引进来,那样可就麻烦了。”

