\section{第48章 浮云蔽日光(下)}

【今天就只有一章了。明天开始,便是新的一卷。】

这一天的崇政殿外,弥漫着一股怪异的气氛。

无论班直还是内侍,都无心守卫殿门,甚至都不顾规矩,低声的交头接耳。

飞天遁地的故事只出现在传说之中,许真君的拔宅飞升更是人人都要羡慕,只是都知道这等美事轮不到自己。可今日偏偏出了异事,军器监竟然送了一个篮子上天了。装在篮子里的东西很好笑,是一头猪。但猪能飞上天,人当然也可以。

过去在宫中的传言中,韩冈只是一个年轻有为的官员,最多可以说一句前途不可限量。但此时在守殿班直和内侍们的眼里,跨进殿中的韩舍人他身上,却镀上了浓浓的一层神秘色彩,让人不禁联想起,他一直以来矢口否认的药王弟子这个身份。

韩冈走进了殿中,他们都竖着耳朵听着殿中的动静。

“韩卿!”赵顼略显急促的声音从殿中传出来,“军器监中可是有造能够飞天的船只?”

“确有此事,臣命名为飞船。”韩冈给了一个肯定地回答。但他轻描淡写的语气似乎在说着此事不值一提,“不过此前仅仅试验了三五回,只敢装上禽畜,还没到载人上天的时候。臣本准备等能送人上天之后,再来禀报陛下。”

竟然是真的!

韩冈答得如此爽快,反而让赵顼一时之间不知该说什么好。他本是在听着枢密使吴充有关河北禁军改编的汇报,没想到半途中,权知开封府的韩缜匆匆忙忙的求见,一问之下,竟然是军器监送了一头猪上了天,惹起了京中的大骚动。

猪飞上了天,这话乍听起来很好笑,但细细想来,就让人笑不出来了,甚至让赵顼感觉有些毛骨悚然。

这可是飞天啊!

一边的吴充,也是觉得韩冈的行事越发的不合正道,方才他就灌输给了赵顼不少危言耸听的话。一等韩冈承认,便站了出来,厉声喝问:“韩冈,你好好的板甲不去打造。却做这等神怪之物,致使京城骚然……”

“少见故而多怪。虫鸟皆能飞天,也不见有人惊讶。”韩冈毫不客气的打断吴充的指责,“若是一个月下来,天天都能见到飞船上天,也就不会有人再多看一眼。上元节的灯会,年年万人空巷,观者目眩神迷。可若是一年三百六十天,京中日日有灯会,京城百姓还会有兴致吗?……习惯之后,也就只是平常而已。”他很是不屑的一笑,“柳河东《黔驴》一篇,想来吴枢密必定听说过。虎之畏驴,乃因其不知驴。待其知驴之底细,那驴也便成了虎的腹中之食。只要日后京中天天可见,明其底细,也就不会再有今日之事。”

吴充脸色气得发青,赵顼却没有关心。他性急的问道:“韩卿,你到底是怎么让船上的天?”

韩冈冲着天子欠身一礼:“臣对此已在《浮力追源》有过说明,此与铁船同理。只要整体的密度小于水,铁船便能浮于水。若想浮于空气,只要比空气轻就行了。飞船之所以能飘在空中,就是因为其整体要比空气轻。”

“气难道有轻重之别?”赵顼追问着。

“空气无形而有质,乃物也。其既为物,自有轻重。热气则轻,冷气则重……”

“一派胡言!”这下轮到吴充打断韩冈的话:“人间四月芳菲尽,山寺桃花始盛开。越是高处就越是寒冷,何曾见过高处反比低处热的?”

“正是因为高处不胜寒,故而热气会往寒处行,此乃天道循环,阴阳互补之理。若非如此,飞船何能飞天?”韩冈微笑着:“而且热气上浮寻常即可见,只因吴枢密是君子,故而不知。”

吴充知道韩冈绝无好话,正待发作,赵顼则抢前一步,好奇的发问:“韩卿此话何解?”

“礼记有云:君子远庖厨。吴枢密仁人君子,故而不知厨中之事。而韩冈不才,则是略有所闻。即便是厨中烧火的粗实女婢,也是知道热气是往上走的,否则烟囱何不往地底修?”韩冈语带讥讽的反扎了吴充一记。

吴充没想到韩冈口舌不饶人,脸色更加阴沉:“不论飞船之理如何简单。可世人多愚,日后必会有妖人以此为仗,用来煽惑世人。”

“若是不知情由,飞船确是会让人有些惊讶。不过论其本源,也只是俯仰可见的寻常之物。韩冈亦仅是根究其理,进而推而广之。所谓格物致知就是如此。人皆有知,只要教化得力,必然让妖言无所遁形。如果今日韩冈拿出一艘铁船,不知世人可会惊讶?”

吴充就是等着韩冈这句争辩,立刻追逼道:“若当真能教化万方,飞船当会遍及天下。”他转身对着赵顼:“臣恐日后天下城垣便从此无用,就连皇城也要任人出入了。”

此话一入耳,赵顼便不自在的在座位扭动了一下身子,若是贼人从天而降,城垣的确无所施用。

韩冈却仿佛听到了一个很好笑的笑话:“枢密是在说有人能从比开宝寺铁塔还要高的地方跳下来潜入宫中?既有此能,五丈宫墙亦难挡。”

“难道飞船只能上,不能下?”

“飞船之大,犹如屋舍殿阁一般。悬于空中,或许会忽视。但若是降下来,只要眼不瞎,何人看不见?再说了,飞船随风而行,如蒲公英一样,无风不动,乃是随波逐流之物。可不是如同行人车马,想往哪里去就往哪里去!”

韩冈和吴充斗起了嘴,赵顼听得不耐烦了。说了半天,都没说到他关心的事上。提声打断了两人的争辩:“韩卿!这飞船一物,究竟本于何处?格于何物?”

韩冈先为此前的失礼告了罪,然后回道:“回陛下,就是市井中常见的孔明灯。只不过放大了百倍,从只能带着蜡烛的灯盏,变成了能载人的飞船罢了。”

“孔明灯?!”赵顼惊诧无比,他可是从小就看着孔明灯点着升空,从来都没想过能由此造出可载人的飞船来。

“正是孔明灯,燃烛便能浮空,便是因为灯中空气受热之故。”韩冈瞥了面色发黑的吴充一眼,“只要当场看一下飞船的构造,也就能瞧出其中的门道了。世间之事往往亦是如此,看似鬼神莫测,一旦说破,其实一文不值。”

赵顼不意韩冈说得这般轻巧:“韩卿,铁船不过是浮水而已,飞船可是能飞天啊……”

“不知陛下何有此言?要说原理,飞船仅是对浮力的运用。要说本源,就是一个略大一些的孔明灯罢了。说到用处,能做的也不过是能代替巢车,远观敌阵。远比不上铁船,能带动与钢铁有关的军民器物制造水平的整体发展。”韩冈顿了一顿,“……臣也不敢欺瞒陛下,打造飞船之本意,多为光大气学之说,格物之理。若不是这飞船还有着一点可以顶替巢车【注1】的功用,臣甚至都不敢拿军器监的名义来做。”

韩冈脸上的困惑让赵顼不禁自问,自己是不是太大惊小怪了一点。不过就是能让人飞天而已,算不得什么大不了的……

……这怎么可能?!!!

这可是能让人飞上天的工具,赵顼也只在做梦是才幻想过的事情,神仙方能为之。现在有人在自己面前说,这根本算不了什么?那为什么几千年来,就没一个人想到过!?

赵顼半眯起眼睛,缓缓说着:“韩卿太过自谦了。”

韩冈摇摇头:“这就是格物致知!并非世人才智不及,只是没去想而已。风吹草动,叶落花开,虽是寻常,却自有至理在其中。只要不是视之为常,一眼带过,去根究其理,必然会有所得。”

“韩卿所言确是至理……”

军器监所造的飞船在京城中引起的轰动,远在之前铁船、板甲之上。连韩绛、冯京等宰执,都在震撼中一时无语。

原本位于汴河边库区中的飞船基地,也给移到了兴国坊中。而外界一片沸腾,韩冈却根本就不当一回事,这样的态度下,让他在世人眼中变得莫测高深起来。

尽管士林中的评论有着不少杂音,可韩冈在御前廷对时,已经明明白白说着飞船模仿的是孔明灯。这么简单的道理,不知多少人听说之后,在暗恨自己为什么没有想到。

同时飞船的成功,让许多印书房连夜加印起原本是手抄本的《浮力追源》来。而张载的关学,终于在韩冈的极力推动下,走到了京城这座舞台上。横渠四句教,也在京城士子口耳相传中,传播开来。

为天地立心,为生民立命,为往圣继绝学,为万世开太平。

这四句话的气魄宏大,道尽了儒门子弟应有的作为。一时之间,横渠张载的名望压倒了诸多名儒,而成了士林之中,最受敬仰的几人之一。

当然,此时最为吸引京城、乃至整个京畿目光的,还是在兴国坊中全力整修的飞船。

二月中旬,也就在骚动后的半个月,‘载人飞船’在万众瞩目下,于金明池畔飞上了天空。周全,这位在河湟丢失了右手的老兵,也成了这个世界上第一位踏足虚空的凡人。

抬头望着虚悬在近百丈的高处,被波澜不兴的微风吹向湖面的热气球。欢声雷动中,韩冈冷淡的笑着。

只是离着他的目标,又向前迈进了一步。

注1:古代战场上用来登高望远的车辆。

第三卷:‘六三之卷——开封风云’完。

请期待下一卷:‘六.四之卷——南国金鼓’。

