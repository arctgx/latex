\section{第二章 天危欲倾何敬恭(八)}

“你先下去。”

石得一挥挥手,让手下先出去。

韩冈和王厚是为了大图书馆选址才出门的,这点很容易确定。

虽然比同宰辅的韩冈亲自去查看地势,这一点听起来就是一个笑话。但他还拉着王厚,看起来更多是好友多年未见,遂把臂同游而已。与二大王、三大王没有什么关系,仅仅路过。

不过作为一名在宫中服侍几代天子、太后的大貂珰,石得一很清楚作为天子家奴,要做到什么样的地步,才能算得上是称职。

主子没有考虑到的,他们必须考虑到,主子已经考虑到的,他们则要考虑得更周全。就算可能性微乎其微,也必须一并考虑进来。

韩冈或许只是闲来无事,与友同游京师,可也有可能是带着王厚来认认门。

王厚做了好些年的边臣,功劳苦劳都有,还有一个留下无数人脉的父亲,若是问对称旨,留在京师任职根本不是问题。

韩冈也有可能看到这一点,顺便就将王厚推荐给太后。他现在又没了差事,想举荐谁都不会犯忌讳。

若韩冈当真推荐王厚,会不会跟二大王有关?

二大王上蹿下跳的确让人恶心,但苍蝇不叮没缝的蛋,若不是有了一个现成的臭鸡蛋,他还是得在院子里继续疯下去。

拿起一张写满字的纸片,石得一眯起眼睛,嘴角泛着冷笑。

许久,方才放了下来。

他揭开暖炉,然后除了那一张之外,桌上的其余字纸一张张的都丢了进去。

火焰一下窜了起来,火星子飞入空中,石得一定定的看着赤红色的火苗,心里却在想:房子里面多了多少炭毒?

有些事,要么迎上去,要么就踢开来,犹犹豫豫,最后怎么都落不着好。

待房中的火光渐弱,石得一收起那张纸,然后起身离开,出了门,便往大行皇帝的灵堂过去。皇太后这些日子就住在在那里。

夜渐深,石得一快步穿过一条条回廊。换了白色流苏的玻璃灯笼高高挂起,映着廊中敞亮。遥遥的,就见到宋用臣守在殿门外。

“太后可还安歇了?”石得一上前询问。

“还没有。”宋用臣神色木讷,反问宋用臣:“可有事?”

石得一暗暗叹了一口气,这几日宋用臣都是这副有气没力的模样,跟之前的意气风发可是差得远了。他这辈子的荣华富贵看来都要化作流水了。

石得一心里想着,行了一礼:“石得一有事须奏禀太后。”

……………………

上元节近在眼前。

不过今天大宋诸路,千万城镇,都不会有上元灯会。

韩冈家里几个小子原本盼了一整年出去能看灯会,听说今天的上元节不放灯了,一个个都没精打采起来。

他们的坏心情,还是从不能放鞭炮烟火的正旦开始的,一直持续到了现在。

小孩子的坏心情没什么关系,但京城中的许多商家受到的影响更大。

上元节不是一年中最重要的节日,却必然是最热闹的节日。许多京城里面的商家,就等着上元节的五天灯会期中,好好的做上一笔,

不仅热热闹闹的联赛被停办,正旦萧条冷落,就连最热闹的上元节也没有了,也许能占上一年利润三分之一的收入化为泡影,哭天喊地的多不胜数。很是有些人在抱怨先帝死也不挑个时候。

上元节主要影响的还是小商家和摊贩,中等水平的商家,不会因为损失一两次节日收入便陷入困境。可京城中多少以歌舞曲乐醉人的酒楼,整整三个多月没收入。弄得开封府上下都要叫苦不迭。

联赛的抽成也好,酒楼背后的教坊收入也好,以及各处瓦子这样的娱乐场所的分账,这些都是官府的财政保障:厢军之中就有一个酒店务,跑堂、收账和做菜的可都是兵,经营产业,也是各地衙门提高收入的手段——这是从五代的藩镇传承下来的惯例。

尽管如此吗,如果没有之前的一场大火,开封府的财政还能支撑得住,不就三个月嘛,之前曹太皇上仙,也不是没经历过,但石炭场大火,把开封府的底裤都烧通了。

沈括找到韩冈这边,千求万请:“玉昆,只能靠你了。”

沈括苦着脸。他之前去找蔡确,蔡确直接摊手给他看,犒赏三军的钱和绢,都要七拼八凑,哪里有闲钱给开封府支应?开封府的各项产业没了抽成的确不假,但市易务的收入也降了近三成。朝廷也等着钱用。

沈括左转右转想不出招来,转头就只能来找韩冈想办法。

“这是蔡相公的事吧?”韩冈都不知该笑还是该气,换作是脾气硬一点的开封知府,借着那场大火,怎么也能从蔡确那里挤出十几二十万贯来。可惜沈括脾气太软了,都不敢跟蔡确强讨,“存中兄,管朝廷钱粮支出的是中书门下,是三司,不是皇宋大图书馆。我现在也要唱莲花落的好不好?”

“这事沈括也知道,不是来问玉昆你要钱,只是来讨个主意。”

“主意?我也变不出钱来。现在只能熬过去吧?”韩冈气得笑了,“野地里的蛇啊,熊啊,到了没食物的冬天,都会找个洞钻进去,睡上几个月的觉,这叫冬眠。消耗少了,就能多熬一阵子了。”

“能省的可都省了。就是不能省的地方太多了。别的不提,那石炭场大火后留下的千户灾民,总得好生的安置,也需要给些补偿,好让他们重置家宅。否则冬天里冻死几个,有伤太后和天子的仁德。而且还有别的支出……”

“还有什么……”韩冈问道。

“宫中出来的那批宫人。他们都要安排到敇建的寺观去。”

赵顼驾崩的那一夜,值守在福宁殿中的一应人等,基本上都已经被清出了宫中。正好赵顼死了,没了服侍的对象,他们安排到哪里都不成问题。宫女愿意回家的都发遣回家,嫁人也好,出家也好,与朝廷没有关系了。但那些内侍,还有几个无家可归的宫女,就只能先安排到敇建的寺观中,等到山陵修好,他们中的大部分都会被安排去守山陵。

纵然有些可怜,赵顼之死也不怪他们,但谁也不敢保证他们会不会日后因为走投无路而选择走极端,宫里面总不可能继续重用他们。出于原则,韩冈尽力保住了他们的性命,但还让他们留在宫中,就不算是原则了。

“照惯例,他们的衣食要开封府给,发遣金也是转由开封府给。”

韩冈听得直皱眉,“宫里面的人,宫里面发遣出来,当然是宫里面出钱,开封府掺和什么?”

沈括苦笑:“有惯例、故事,总不能当不存在。”

终究还是沈括性子软,畏惧宰相。否则这样毫无道理的惯例,平常时倒也罢了,现在根本就拿不出钱来安置,就应该直接踢回去。

沈括不是没有果决的时候,在天子丧期,照例是不得行大辟,也就是不能处决犯人,而沈括在处置趁火灾而犯法的盗贼时,是当日处决,也不在乎犯了忌讳。

如果只是约束人的法令条款,沈括还真的有那个胆子,但换作是地位比他高的人,沈括的胆子就大不起来了。

当然,这也是沈括负责任的表现,否则开封府没钱又关他什么事?发不出口俸,给不出修缮金,甚至安置费都没有,沈括最聪明的作法,就是直接跟下面的人说没钱,让他们去蔡确家门口讨钱去。最蠢的就是明知没办法解决,却将所有事情给担待下来。

沈括选了最蠢的办法,可韩冈却不能置身事外。

“罢了。”韩冈叹道,“也不是存中兄你的责任。可是要我去跟蔡持正讨个人情?”

“若能如此,那就太好了。多谢玉昆相助。”

沈括说着,起身向韩冈行了一礼。

韩冈侧身避过,“免了,存中兄,我这也不是白帮你。”

“玉昆你还有什么事要沈括去办?”

“请存中兄帮忙给大图书馆再挑一个好点的位址。要地势高、地面宽敞、往来便利,离开封图书馆不要太近的。”

“这事容易。”沈括一口应承,“明天就将开封府名下的产业都列出来,玉昆你自己来挑。若是玉昆你嫌麻烦,开封府这边先帮你过一过筛,发现合适的位置就通知你。若是开封府下面的产业里找不到合适的,其他家宅和店面,都会帮玉昆你去寻找。”

韩冈的要求多多,可沈括也是知道投桃报李。韩冈既然肯帮忙与蔡确关说,他当然得去努力解决韩冈的问题。

“多劳了。”韩冈举手致谢,总之,有了沈括全力相助,想找到一个合适的馆址就容易了许多。

沈括说了阵闲话,告辞离开。作为知开封府,他每天要处理的公文超过百封,没有多少时间可以耽搁在外面。

韩冈则不管那么多,他现在可是清闲的很,能帮帮沈括也是应该的。

也不知最近蔡确从铸币局那边挖了多少好处,韩冈既然要帮沈括解决财政上的漏洞,当然也少不了要先调查一番,到底要怎么从政事堂那边挖出钱来。也得费上一番思量。

