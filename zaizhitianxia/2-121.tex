\section{第24章 兵戈虽收战未宁(八)}

听着折可适的分析,种建中陷入沉思。

由于跟韩冈打过交道,这段时间又听说过韩冈的不少事迹,种建中静下心来想想,倒真的觉得他的这位同门师兄弟的确做得出来。

斩杀敌国使节,如果是在本国国内做下的,肯定是要被御史弹劾。两国相争、不斩来使,几千年来传下的规则,让朝廷丢不起这个脸——过去就算跟西北二虏打得最猛的时候,也从没为难过两国的来使。不过换在是吐蕃蕃部中,斩杀来撬墙角的西贼使臣,却是直追班超的功业。

“如果真的是玉昆做的,那……”种建中话刚说了一半,白虎节堂的大门一下打开。陕西宣抚司中的一众参军、将佐从堂中鱼贯而出,绯色、绿色、青色的官服一片片的晃着人眼,鄜延路的与军务有关的官员都到了。种建中和折可适所等候的种谔、折继世两人,亦随众人而出。

种建中和折可适都站起身,准备上去迎接。只是折可适的脸突然绷了起来,低声怒吼道:“王文谅那厮怎么进的白虎节堂?!”

他的一双略显细小的眼睛盯住的是一个四十多岁的蕃人。当结束了军议的众官从白虎节堂出来后,关系好的都走在一起,关系疏远的也会打个招呼再离开,唯有这个被折可适唤作王文谅的蕃人,孤伶伶地走着,没有人理睬他。

种建中看着王文谅,也吃了一惊:“真的王文谅……他怎么够资格进去的?!”

折可适脸色铁青着,双手紧紧握拳,眼底的怒火好似能融金铄石:“不过是没藏讹庞的家奴而已,逃到这里也不过是个左侍禁,他怎么配进白虎节堂的?!”

“大概是敢拼敢杀吧,加上他又能言善辩……不然怎么能得韩宣抚的欢心。”

王文谅本是没藏讹庞家奴。而没藏讹庞是曾经的西夏权臣,也是前任国主谅祚之母的兄长。没藏家是党项大族,当年煽动李元昊长子宁令哥弑父,是他主谋。而把自家外甥、不到一岁的谅祚抬到国主之位,也是他的手段。

只是没藏讹庞太过跋扈,渐渐长大的谅祚对其心生不满,而原本能弥合两人之间矛盾的没藏太后,又因与她所私通的僧侣宝保吃多已一起去贺兰山游猎,而被二十几个吐蕃盗匪所杀。少了靠山的没藏讹庞依然跋扈,甚至把自己的女儿强嫁给谅祚。所以他的结局就跟历史上所有架空天子、谋朝篡位的权臣一样,最后被谅祚下令灭族,王文谅就是在那时逃了出来,投靠了大宋。

——当时,如今的梁太后还是没藏讹庞的儿媳妇,不过她与谅祚私通,给没藏讹庞的儿子编制了许多绿帽子。而当没藏讹庞因为谅祚越来越自有主张、不再听话,受了绿帽儿子的撺掇,打算杀了他换一个新主时,也是梁氏向谅祚通报,使得谅祚能够先下手为强。

靠着这份功劳,梁氏成了西夏王后,而梁乙埋也就攀着妹妹的裙带,一路上窜,直至如今成为西夏国相。王文谅虽然逃了出来,但他的家人全都陷在了兴庆府,他与梁氏之间有着血海深仇,打起仗来就跟拼命三郎一般,这就是他为什么得韩绛欢心的缘故。。

一般来说,蕃将手上的兵员往往都是自己族人,不会拿去跟敌人硬拼,但王文谅是从西夏投奔而来,本就是孓然一身,所掌握的兵力统统是调配到他手底下的外人,上阵时便分外卖力,毫不顾惜底下人的性命。正是由于在战场上与众不同的表现,王文谅得到韩绛的赏识。

只是这样的赏识,是建立在王文谅挥霍帐下士卒性命的基础上的,韩绛每每拿着王文谅的做法,来逼手下的蕃将。世镇麟府的折家也是蕃将中的一份子,手中的精锐就是不到三千的族中私兵,打仗虽然拼命,却做不到王文谅的程度,所以没少被韩绛骂过。

就因为韩绛几番训斥,刚刚过去的西贼全线南侵,折家也的确拼了命。一仗下来,折可适便少了两个兄弟,一个叔父。如今折家上下对韩绛不敢有所怨恨,却把王文谅恨到了骨头里。

折可适死盯住王文谅,从他身子里透出来的杀意,让种建中都打了个寒颤。只是王文谅走了几步,节堂中却奔出一名小吏,喊住了他,两人一起返身走了回去。

连走了出去,都不忘把他叫回来,种建中都觉得韩绛对王文谅实在宠信得过了头。不过种谔、折继世已经走了过来,种建中也无暇去多想。问好行礼后,折继世就带着自己的侄孙急急的走了。而种建中也跟着种谔,往府衙外走去。

种建中追在叔父的身后,像小学生般提着问题:“五叔,今次是不是把罗兀城的事给定下来了?!”

种谔边走边道:“此乃军国大事,岂会谋于众人?今天没提这一条,等私下里再去拜访韩宣抚述说此事。”

“那今天说的什么?”种建中好奇的问着。

“划拨在王文谅手下的蕃骑战马不足,一千五百人还不到八百匹马,需要紧急调派。”

“从哪里调派?沙苑监这水平,今年能出一百匹就不错了。”秦州那边靠着市易弄到马匹不难,但弄到合格的战马却比登天还难。

“谁手上有马,就从哪里调……”

种建中闻言便浑身一震,脚步不由得停了下来,这是要夺汉兵的马给蕃人,“谁想出的这个馊主意?!”种谔还是沉着脸一直往前走,种建中忙追上去,“五叔!这怎么行?”

“谁的骑术更高?汉人还是蕃人?”种谔一直往前走,“汉军有弓弩就够了,与其不上不下的被西贼的铁鹞子砍,还不如让给蕃人。”

种建中难以置信的望着种谔,他很清楚为了让麾下的骑兵们都拥有足够的战马,种谔过去究竟费了多少心力,他紧追在种谔的身后:“五叔,你真的是这般想的?”

种谔大步往前走,却不回头,“废话忒多!回去跟十七说,让他先做好准备。今次一定要把罗兀给抢回来。”种谔的声音低了下去,低到种建中都听不清,“不能再输给秦州了!”

………………

韩冈正坐在古渭寨的架阁库中,翻着薄薄的档案。过去二十年来留下的记录,只占满了半面墙壁。卷宗的数目少得连普通的县城都比不上,就是落满了灰尘。连最常被人调用的田籍,也是一样都灰蒙蒙的。

展开屯田的一个成果,就是要备办的田籍和五等丁产簿比过去多了数倍,需要调集人手来编修。韩冈翻着过去的档案,盘算着着是趁此机会将古渭寨辖下所有户口的簿册一起重修,还是只编修新移民的部分。

李小六从门口探进头来,“机宜,王衙内来了!”

韩冈把手上的鱼鳞册一丢,看得久了,正想找个机会歇一歇。刚出了架阁库,走到外面的公厅中,王厚就已经跨进门来。

“疯掉了!”他连声摇头叹息,他是刚刚从王韶那里回来,“当真是疯掉了。”

韩冈把王厚引着坐下来,问道:“谁疯了?没头没脑的。”

“还有谁,宣抚司的韩相公呗!”王厚没好气说着。不等韩冈问,便把韩绛欲夺汉军的战马交给蕃人的事,原原本本说了一通。

“真的假的?”韩冈的第一个念头就是怀疑此事的真实性,实在太不可思议了。

“这事关西都传遍了。据说被点上骑军都是哭着不肯把坐骑送给蕃人,却给韩宣抚硬是抢了去。”王厚直摇头,感叹道:“真是疯了!”

韩冈也跟着王厚一起摇头,“韩宣抚做得太过了一点。哪能为了蕃人,伤了自家人的心。”

“谁让王文谅上阵不顾生死,得了韩宣抚的欢心呢!”王厚冷笑着。

“……有没有说韩宣抚动得哪里的骑兵?”

“已经拿了环庆的广锐军先开刀了。”李小六端上茶来,王厚端起茶盏,就不顾烫嘴的喝了两大口,“接下来不知要摊到那一路,看这样子,迟早要轮到秦州头上。”

“广锐军……”韩冈眉头皱了起来。

广锐军隶属于侍卫亲军司下面的马军司,在大宋禁军的骑兵部队中并不算是上位军额,比不上龙卫、云骑、骁武这些一干骑军,但也算得上是历史久远的精锐了。辖下共有四十二个指挥。不过广锐军的这四十二指挥分布得很散,从太原、并州,到秦州,都有广锐骑兵驻扎——名为一军,其实是各自为政,只听枢密院和本路州调遣。

这也就是为什么从范仲淹开始,蔡挺、王安石等有心于西事的臣僚,都要推行将兵法的缘故。同属一军的军队,竟然分散得天南海北,本该是一军之首的都指挥使就成了个笑话,根本指挥不了手下的兵将。基本上,大宋禁厢两军,无论马军步军的哪一个军额,情况泰半如此。就如今次出战渭源,王韶所动用的十个指挥,便总共来自于七个军。

“环庆军中本就因为李复圭胡乱杀人,搞得人心不稳。韩子华再这么欺压下去,环庆迟早会闹出乱子。”韩冈话声冷澈,像是在预言,透着浓浓的不祥味道。

