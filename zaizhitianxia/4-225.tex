\section{第37章 蒿目黄尘顾世事(上)}

过了端午,到了五月中的时候,京西的气温已不再是春夏交替时节的忽髙忽低,而是已经稳定在了高位。

但也就是在这个时节,三千厢兵陆续被调来汝州、唐州的交界处,正式拉开了打通襄汉漕运的工程。韩冈也暂时离开了襄阳,驻节于方城县。

由于工期并不算紧,厢兵们的工作时间都定在早晚清凉的时候。相比其他工程来,这算得上是十分轻松的工役了。除此之外,还有充分的后勤准备,加上足够诱人的奖励制度,让韩冈预备好的杀一儆百的最后手段,并没能拿出来亮一亮相,连工程进度也比计划中要快上不少。

开工已经七天了,轨道路线的地基顺利的从垭口两端,向中心处延伸。一天总额高达百贯的奖金,悬在六十支工程队,总计三千厢军的头上,让他们忘记了疲劳,在有限的工作实践中拼命的卖着力气。

韩冈刚刚从工地上巡视回来。

今天早上他所视察的几段路线上,厢兵们纷纷要求延长每日的工作时间。相比起一个月才四五百文的军饷,只要进度和工程质量的综合评分每天能排在前十,每一名厢兵至少都能分到五十文的奖金。若是排在第一,就是整整五百文。穷惯了的厢兵,有几个不愿意发上一笔小财?

但韩冈拒绝了厢兵们的要求,让他们稳着一点,将工程质量再提高一级。

脸上带着满意的笑容,韩冈回到临时居所。才下马,勾当公事的方兴就递了一份刚刚送达的公文过来。

“是什么事?”韩冈漫不经意的回身,拍拍一大早就背着他巡视了几十里地的坐骑。

方兴回道:“朝廷颁布《手实法》,要路中对治下各州县加以督促。”

“手实法?!”韩冈脸色猛然一变,扯过方兴递来的公文,只看了两眼便骂了起来:“吕吉甫这是疯了不成?!”

“龙图!”方兴咳嗽了一声,又看看左右,提醒韩冈注意言辞。

韩冈倒给方兴小心翼翼的姿态,弄得有些好笑。不过背后骂人的确不好,换了个说法:“吕吉甫这次做得岔了。”照样还是否定。

韩冈很早以前就听说过了手实法这个名词,这是章惇跟他提起过的。就像青苗、免役、市易诸法都不能算是王安石原创一样,手实法其实也是源自于前代,为唐代宰相元稹任知州时所首创。

当初王安石初次罢相,吕惠卿升任参知政事后,就开始筹划推行手实法,以厘定天下人户家产情况——主要是田地,也包括房产、牲畜,以及桑麻等经济作物——制定五等丁产簿,并藉此来征税。这的确可以算是方田均税法的一个变体,但比起方田均税法来说,却更为宽松。

虽说基本上皆是通过确定各家各户所拥有的土地数目,以便征收税赋——确切点说方田均税法是征收田赋的依据,而依手实法所制定的五等丁产簿则主要是抑配便民贷和征收免行钱的依据——但方田均税法,是由官府派人下乡丈量,而手实法则是由百姓自报数目。

不过当初王安石复职太快,让吕惠卿还没来得及推行,就不得不宣告中止。但眼下王安石、冯京、吴充一个接一个罢相,政事堂中只剩王珪一个对天子唯命是从的宰相,再加上元绛这位年纪虽长、但在政事堂中资历浅薄的新人,吕惠卿的当初的计划也就随之提上台面。

只是韩冈出京时此事还一点动静都没有,现在才几个月的功夫,这一项新法案,就已经在皇城中走完了一整套的流程,颁行于天下。吕惠卿的手脚的确是够快的。

可惜的是,这一项法案,在韩冈看来,实在是不合时宜。

“龙图也觉得着手实法不合人意?”方兴问道。

韩冈叹了一声:“变法的时机早就过去了,吕惠卿如今逆势而行,乃是事倍功半、得不偿失。”

“不过说起来,这其实也只是对方田均税法的补充罢了……如果能把最后一条去掉的话。”

“……能去得掉?”韩冈笑着问道。

经过了两年的延误,新近出台的手实法对比起旧法来,似乎是修改了一些不太适当的条款,涂脂抹粉了一番,很有几项体恤百姓的条款——比如说‘屋宅分有无蕃息立等,凡居钱五当蕃息之钱一’这一条,说白了就是自住的屋宅和自耕的田地在计价时,只抵作同等大小的出租屋宅和佃田的五分之一,自耕农可因此而得惠。

但为了保证这条法案能带来足够的财政收入,基本内容却并没有什么变化。其中最关键的一条,就是如果禀报不实,允许他人举报,这些被瞒报的财物将分出三分之一与告发者——愿意自觉自愿的缴税的民众当然不多,愿意公开家产的人们同样没有多少,所以为了防止有人弄虚作假,吕惠卿便在手实法中加了这一条。

如果没有这一惩罚条款,只令百姓自报田产数目的手实法,不会有任何实际意义,可有了这一条之后,任谁都能看得出来,手实法的推行会因此而困难百倍。

如果只从法令上来看,在韩冈眼里,手实法的确算是良法,可算是超前的财产申报制度——因为要申报家产的不仅仅是民户,也包括世称形势户的官户——但这项法案,一旦施行起来,却会引发很大的问题。

以钱财来鼓励告发隐财,日后告发之风便会无穷无尽。从儒家教化上讲,这是有伤风化之法。而说句难听话,真正的大户势力盘根错节,根本没多少人敢去告发,就是告发了也不一定有用,反而是家底稍稍殷实的普通富户,没有什么势力、家产却让人垂涎,最后就会成为最大的受害者——官吏们的道德水准,根本不能指望,而从利益的角度上说,官吏又怎么会跟自己的家产过不去?——只要中户因此而破产成风,用来攻击吕惠卿、废除手实法的藉口也就有了。

但也不是说手实法完全不可能实现,换在是十年前,甚至是六年前免役法初行,国用窘迫的时候,即便反对声再大,天子咬着牙都能推行下去,有什么问题,完全可以在施行中加以解决,只要能让天子在国库中听到叮当作响的铜钱声就行了。

可惜的是,眼下已经是元丰元年而不是熙宁元年。

吕惠卿从来都是个有想法有心气的人,自从他辅佐王安石考订新法开始,便是如此。这样的人、这样的性格,当然不会甘心一直身处在王安石留下的阴影之下,他现在做的事,就是在设法摆脱王安石的阴影。拿自己这些年里积攒下来的的政治资本,来博取更多的权力以及更响亮的名声。

“吕参政应当是权衡过利弊。但这是赌博吧?他可没有王相公当年的声望……不成功便成仁,太冒险了。”方兴对吕惠卿这一次的行事很不以为然。

“他肯定是想明白了。”韩冈从不怀疑吕惠卿的智商,但下这么大的赌注,却也同样是他难以认同的,“赌博嘛,没有说百分百能赢的,做一件事,任谁也不能说自己肯定能成功。只是有些不值当啊。”

如果天子能够全力支持吕惠卿,韩冈也不在意得罪一下京西的官宦人家,反正他们有钱,多出点也是应当的。但若是赵顼做不到十年前支持王安石一般的成为吕惠卿的支柱,韩冈疯了才会为随时准备退却的赵顼和一心想摆脱王安石的吕惠卿出生入死。

韩冈完全不看好吕惠卿。熙宁六年的市易法就闹得京城一团乱,王安石的政治资本大幅衰减,到了一年后终于辞了相位——王安石初次辞相的原因很多,可这一件事肯定是其中最关键的一条——而如今手实法更胜市易法一筹,吕惠卿却没有王安石的资本雄厚,眼下他是赌天子要保着自己,不可能有十足的把握。

韩冈叹了一口气之后,就把这件事丢到了一边去。但过了两天,李诫从正在亲自测量船闸、堰坝修筑位置的沈括那里回来,说了些公事,却又提起了手实法来。

“此事你怎么看?”韩冈反问着。

李诫犹豫了一下,摇摇头,“朝廷不缺钱。去年京西北路的税赋、便民、市易、免行,林林总总加起来也有六百余万贯石匹两,比京西北路税赋更多的路州为数都不少……”

韩冈摇头,打断了李诫的话,冷笑道:“朝廷的褡裢里面永远都有个补不好的窟窿……谁会嫌钱多?你能收多少钱上来,就有人有本事用多少钱出去。即便不用,堆在库房里,让天子看着也舒心……不曾闻‘五季失图,猃狁孔炽’?”

李诫点点头,当今天子登基之初就模仿当年太祖皇帝之志,在宫中建了三十二间库房,自作诗句‘五季失图,猃狁孔炽。艺祖造邦,思有惩艾。爰设内府,基以募士。曾孙保之,敢忘厥志。’,以此来作为库房的编号。打算用这里面的钱,作为军费,灭掉西夏,收复燕云。

——太祖皇帝赵匡胤当年设封桩库,便是存了用库中财物换回燕云失土的打算。如果契丹人敬酒不吃,那封桩库中的这笔钱就会转作北伐的军费,变成一杯赠给契丹人的罚酒。可惜太宗两次北伐皆败北,而真宗皇帝则是失土没得回,就只拿到了一张澶渊之盟,最后拿着这笔钱去封禅修庙,。

李诫听得明白,但他疑惑了起来:“龙图是同意手实法?”

韩冈摇摇头:“只是眼下朝廷的税赋收入也不算少,靠手实法多征的钱钞,并不是朝廷急需。”

韩冈不介意在李诫面前议论朝廷成法。李诫会在他面前提起手实法,其实是帮他老子来问的——韩冈专注于襄汉漕运,京西漕司中的大小事务,大半都是由一南一北两位副使、两名判官来处置。

韩冈是转运使,在职责上有督促治下州县推行朝廷新规的职责。如今朝廷颁布手实法,韩冈也有义务在京西路将之推广。这些年来,诸多新法一部部的颁布于世,每次总有几位漕司主官被调职或降罪,这都是因为督促新法不利的缘故。所以韩冈现在的这番话也是说给李南公听的。

对于手实法,韩冈的态度很明确,他无意去监察督促,但也不会出手干扰,想必李诫应该是明白了。

