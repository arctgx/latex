\section{第11章 五月鸣蜩闻羌曲(七)}

【第二更,求红票,收藏】

“啪!”的一声脆响。窦解唇角的狰狞笑意还未收起,便被窦舜卿的一巴掌给打歪了嘴。他捂着右脸,瞪大眼睛,不敢置信的望着自己的祖父。

窦舜卿狠狠收回手,又剑指指着窦解鼻子,怒声喝骂:“小畜生,你这是给人当刀使还不知道!要是能这么容易就把灌园小儿弄进大狱,向宝能不做?他给王韶、韩冈欺了多少次,可他直接动了韩冈一下?他是武将。我也武将。可那灌园小儿可是文官!”

他一个武将把文官关进大狱?!是嫌御史台里的那些乌鸦太清闲了吗?

国朝左武右文,文官斩武将天经地义,若是反过来,武将囚了文官,那就是通了马蜂窝。那时候,文官们可不会管什么党争政争了,压制武将的跋扈才是大节。

狄青领兵平侬智高,归入他帐下的文臣数违军令,狄武襄都不敢动一下。窦舜卿虽自视甚高,也不觉得自己能跟当时领军在外的狄青比权势。

窦舜卿斜睨着自己的孙子,看着这小畜生,心头就是一阵火发。随随便便就听信人言,也不好好想想,当真要害了全家,“说!到底是谁把这些话教给你的?”

看着祖父须发怒张,窦解给吓得脸色发青,嗫嚅道:“……是个叫王启年的小吏。”

“小吏?!骗鬼去!”窦舜卿霍的站起身来,抬脚就把孙子踹得老远,也只有在这时候,他才表现出了一名武将的灵活身手,“都这时候了,你还敢骗我!”

窦解吓得更是厉害,一翻身,端端正正的在地下跪着,涕泪横流的哭喊道:“真的是王启年,真的是王启年,孙儿不敢欺骗爷爷!”

窦舜卿看着孙子的神情不似作伪,心知应该说得是实话,他不耐烦的叱骂道:“从今天开始,不许你出门半步。若敢违命,看我不打断你的两条腿!”接着又重重的一拍石桌,一声暴喝“滚!”

窦解连滚带爬的瘸着腿出去了,窦舜卿余怒未消,他在石桌上端起一碗凉透了的香薷饮子,正待要喝,却想起来两名给他打扇的婢女从头到尾看到了方才的这场好戏。

窦舜卿回过头,冰冷的眼神扫过。两名婢女还算聪明,连忙跪下,身子微微颤抖着等待着他的发落。

“……方才的事不许说出去,否则拿家法杖死尔等。”窦舜卿威胁了两句之后,一挥手,“你们下去!”

婢女忙叩头谢了窦舜卿的恩典,站起身急急地出去了。

院中只剩窦舜卿一人。午后的阳光热辣辣的射在地面上,热浪滚滚,暑气逼人。没了身后扇来的凉风,短短片刻,窦副总管已是汗流浃背,而他的心情更是烦躁。

他的这个孙儿也不知受了谁的撺掇,竟然在他面前出这等馊主意。说是一个小吏的建议,这窦舜卿可半点不信。一个小吏哪有此等心术,肯定是受了谁人的指派,来诓自家的孙儿。

窦舜卿心不在焉的一口口喝着冰凉的香薷饮子,就算喝干了,也没有发觉。端着茶盏靠在嘴边,他心中却在计较着。站在王启年背后的,究竟向宝还是李师中?

现在秦州城内,跟王韶结下解不开的怨仇的,除了他们两个也不会有别人了。

他们打得也真是好算盘,让自己出头跟王韶再斗上一场,他们却站在后面看热闹,捡便宜。

想让我出头为你们火中取栗?窦舜卿眯起了眼,眼角纹路深深。

那个灌园小儿已经立下了这么多的功劳,就算他误用了西贼奸细,也不过斥责两句,罚个半月一月的俸也就过去了。怎么也治不了重罪,最多是在狱中关个两天就了不得了。

而且指称没有治好自家重孙的党项郎中就是西贼奸细,这件事在秦州处理掉并没问题。但若是闹大了,让王韶和高遵裕把事情原原本本的传到京中,却会变成一个笑话,怕是会惹怒天子。

不过窦舜卿转过来一想,如果不是让他来动手,这个计划其实也不差。因为本来的目的就不是把韩冈治罪,而是把他治死。

韩冈看着高大健壮,但听说他半年多前才得过一场大病,躺在床上也是半年,元气不是这么好回复的。把韩冈弄进大狱,只要把他关个几天也就够了。狱中动点手脚,出来就只剩半条命,活不了几天。

换作是李师中,当能名正言顺的将其弄进狱中。

窦舜卿想了想,觉得把这事转给李师中也不错。正好试探一下他。就看着秦凤经略使是不是幕后的主使了,如果不是,他应当对这个计策感兴趣的。

……………………

王九和周宁毕恭毕敬的垂手站在韩冈面前,腰背谦卑的微微弯着。经过了这么多事,韩冈在秦州的威名日盛,两人在他面前不敢有丝毫不恭。

尤其是今次听说他领命说服青唐部的蕃人出战,斩首一千一百多级,凭借如此的战功,眼前的这位韩官人,肯定又要加官进爵。早早的抱上的粗腿眼见着越发的粗壮起来,王九和周宁的心中也是兴奋不已。

他们的想法都在脸上写着,韩冈也都看在眼里。既然两人都已经打定主意在自家门下作牛作马,就没必要跟他们说废话,韩冈直接问道:“尔等可知近日窦副总管家将一个郎中送进了大狱?”

“这事小人知道。”王九和周宁一齐开口。

“知道就好!”韩冈满意的点了点头,两人果然在州衙中有些关系,“你们就把你们知道的一个个说来。”

“窦家这件事做得不地道。”这次周宁抢先一步,“窦七衙内的不过死了个幺儿子,就把郎中绑着送进了衙门里。说是要告他妄改方药,诈取钱财,听说还硬是要将那个郎中绞了,祭窦副总管的重孙子。”

“现在秦州城里的人也都说窦家实在太跋扈了一点,哪个郎中能拍胸脯说自己没医死人过?真有这本事,也能做第二个孙真人了。俺浑家这些年一共生过三个,就一个小二活下来了,俺也没说把郎中拉去衙门里报官。”

“其实这就是窦七衙内要出一口气。自窦副总管来到秦州,窦七衙内在街市上横行霸道,已经闹出不少事来,有他爷爷在,秦州城中也没人敢惹他。

今次他幺儿重病,先请的几个郎中知道窦七的为人,全都不敢下针开方,摇着头就走了。偏偏就那个郎中不知进退,开了药,也施了针,可是窦家的幺儿还是死了。

正好这个背时的郎中还是个党项人,跟秦州城里的其他郎中都没什么来往,说绑了也就绑了,也没人愿为他出头。”

“啊,对了!”周宁突然叫了起来,他想起了一件事,“这位党项郎中据说是仇老的弟子,靠着仇老的面子,所以他的医馆才能在秦州城中开张。”

“我问得不是这些。”听着两人说了一通,韩冈摇了摇头。他想知道的不是这些传在外面的留言,而是藏在内里的隐情和伎俩,“你们可知最近有谁去狱中见了他?”

王九和周宁对视一眼,一起朝韩冈摇头,“这个却是不知。”

周宁这次又抢先一步,他对韩冈道:“请官人给小人两个时辰,小人很快就给官人打听回来”

“俺一个时辰就够了。”王九像是在跟周宁竞价,一下就把价钱喊低了一半。

“小人其实也只要一个时辰!”

“好了。”韩冈不耐烦的说着,“你们一起去!快点把事给问回来。还有……要小心一点。”

两人会意,一齐开口道:“官人放心,小人绝不会说是官人要小人来查问的。”

周宁和王九急着走了,各自去发动他们的关系,为韩冈打听消息。

“仇老怎么样了?”韩冈回头问着。韩云娘便从小厅的侧门走进来。方才厅中有外人,小丫头也不便抛头露面。

“仇老爷子已经睡下了。”韩云娘答着话,手上则是端着一杯解暑的酸梅汤,递给韩冈,“这是素心姐姐做的,用井水冰过了。她现在正在厨房里,说是三哥哥你奔波劳累好些日子,要为三哥哥做一些补身子的菜。”

韩冈眉头挑了一下,这都叫起姐姐妹妹了?看起来严素心和韩云娘的关系已经处得很不错的样子。

笑着接过茶盏,立刻从指尖处流过一丝冰凉。素色的瓷面上凝着一片细细的水珠,还没喝下去就解了韩冈一身的烦热。揭开盖子,喝下一口酸酸甜甜的汤水,冰澈的清爽感觉从喉间一直传进腹中。

韩冈长长的叹了一口气,只觉得还是在家的好。只恨总是有人不肯让他清闲下来。

见着韩冈刚刚回家,就忙着把人招来问话,忙得不可开交的模样。韩云娘很乖巧的走到韩冈身边,蹲下来帮他捶着腿,扬起小脸问着:“三哥哥,出了什么事?”

韩冈抬手轻抚着云娘的头,发丝柔柔细细,像是在摸着一只可爱的小猫,他轻轻笑着:“没什么,只是一些跳梁小丑不肯下台,想强留在台上多翻上一阵子罢了。”

