\section{第41章 顺风解缆破晴岚(下)}

“何必这么着急?”吕升卿还是难以理解,“征讨西夏不过刚有个风声,种谔也刚刚上书,天子还没有点头呢。”

“王禹玉一直做着三旨相公,别的不说,揣摩圣意上,有谁能比的上他?”

吕升卿默然。王珪三旨相公的外号已经叫开了,请圣旨、领圣旨、已得圣旨,来来回回就是这三句话。一切秉承圣意,完全没有自己的主张。

这样的宰相根本是不合格的,但在天子那边却是很讨人喜欢。

跟孝子贤孙一般听话受教的臣子,哪位皇帝会不喜欢?尤其是如今的天子,已经做了近十二年的皇帝了,越来越喜欢大权独揽。通过更迭宰辅,将朝堂稳稳的控制在手中。现今的朝臣中,又有哪个有韩琦、王安石当年做宰相时的权威?他提拔王珪上来,就是为了能让政事堂能听命行事,不会唱反调。

过了片刻,吕升卿又疑惑开口道:“……让韩冈去河北,他就去不了陕西了?两件事有先有后吧。陕西那里少说还要一年的时间筹集粮秣军资。”

“的确,西北开战,应该在一年或两年后。当年为了争夺横山,韩子华【韩绛】主持陕西宣抚司用了近两年时间进行筹划。虽说如今国力昌盛,三年前,重夺横山甚至连统辖诸路大军的宣抚司都没有成立,但现在要想攻取兴灵,剿灭西夏,却少不了还要用上至少一年的时间来筹备,加上为了加强对辽国的防备,会为等待河北轨道大体完工再拖上半年。基本上就是一年半的时间。”

吕惠卿似乎是详细的计算过,“至于河北筑路,从京西那里耗用的时间上看,差不多也要两年。一年半也只勉强够他去完成河北轨道的主体,但想要河北、陕西两头都插上一脚,占到便宜,”吕惠卿一摇头,“绝不可能!只要韩冈接下去河北的差事,他就不可能来得及赶回陕西!”

“万一韩冈能在一年半之内完工呢?”吕升卿质疑道,“他在京西已经做熟手了,手下也有一批能做事的幕僚。”

“即便韩冈有本事用一年半解决河北之事。”吕惠卿笑了一下,“这几乎不可能,光是勘察地理、确定路线、筹备物资,就至少要半年。只是打个比方,若他当真能在一年半之内完成,他也一样去不了陕西。”

吕升卿惊讶,“为何?”

“战争不是儿戏,临阵换将那是自取败绩的愚行。看天子的心意,也许这一次,不会成立总括全军的宣抚司。”吕惠卿微皱着眉,“已经不是熙宁三年四年,西军从二十多年前几次毁灭性的惨败中,刚刚恢复了元气的时候了,如今的西北各路,都是战功累累的骄兵悍将。”他哼了一声,“就是我去了,也难说能坐稳宣抚大帅的位置。”

吕升卿听得出来,他的兄长对自己在军事上的发言权太低而有所不满,但他聪明的闭紧了嘴,并不搭话。

说了两句心头不痛快的事,吕惠卿回到正题,“虽说这一次不一定会有个掌旗的,但经略各路的帅臣都有筹备的责任在。如果韩冈不能在一开始就参与进筹划工作中,等到旌旗西指的那一天,他也不可能被临时调往陕西去担任主帅——不去种树,却想着去摘桃子,决然没有这个道理!更别说韩冈本人外示谦和、实则高傲,又顾忌着受人议论,就是天子要他去,他都不会答应。”

吕升卿低头想了一阵,的确是这个道理,不过这得有个前提,“万一王禹玉一定要韩冈去陕西怎么办?”

吕惠卿笑了,他扳起了手指:“如今两府加起来只有六人。我这里不用多说。元绛赶在致仕前进了两府,能多荫补几个子孙、门客,已经是心满意足了,不会有心跟人争什么。也就是王珪一心希合上意在推动攻打西夏。”

“西府那里,吕公著巴不得有事能拖一下天子攻取西夏的盘算。章惇不会反对我,当然,他也不会明确支持,他还要顾及跟韩冈的交情。郭逵肯定是想着去陕西,但他若是去了,就必然要设立宣抚司——他一个武将,天子能放心?两府中也没可能会同意,御史们更是乐得有个好靶子了。既然自己去不了,郭逵就不会支持韩冈去帮种谔,两边可是有旧怨,多半会推荐赵禼和燕达。”吕惠卿冷笑一声,“你看着吧,就是王珪也不会愿意看到韩冈在战场上得意……天子也会顾虑着韩冈在陕西立下大功后,还怎么挡着他,不让他进两府。”

吕升卿默默的听着,不停地点头。他有种感觉,吕惠卿这样不厌其烦的一番话说下来,与其说是向他解说朝堂上的局势,还不如说是他的兄长正在通过向人倾诉来整理思路。

相比起一心推动对西夏开战的王珪,吕惠卿这一年来的心思都放在手实法上,他要登上相位,就必须有所成就。推动战争,他争不过努力向天子靠拢的王珪,自己的立足点在哪里,吕惠卿比谁都清楚。

一边是向西夏开战,一边则是纵贯千里的大工程,两桩大事同时进行,对人力、物力……最关键的是对财力上的需求,至少要比现在的财政支出多上一成到两成——也就是一千万贯上下。

钱从哪里来?

吕惠卿得意的轻弹手指,自然是要靠推行新法。

向兄长告辞出来,走了几步,吕升卿回头张望,房中的吕惠卿这时又投入到工作中去。

今天的一番深谈,从头到尾,他就没见兄长怀疑过韩冈所主持的襄汉漕渠工程能否取得成功。

稍稍回想了一下,吕升卿悚然而惊,不仅仅他的兄长没有怀疑,甚至整个朝堂都没有怀疑。

对于韩冈的提议,从一开始,朝堂上就缺乏质疑的声音。除了寥寥几个不开眼的新晋御史,其他人都是采取冷眼旁观的态度,尤其是两府,在无论大事小事,都少不了争执的现在,竟然有志一同的一点障碍都没有给韩冈设置。

韩冈不是没有政敌,他可是年纪轻轻就升了学士。要知道,进了两府才一个直学士的所在多有,凭什么三十不到就是一阁学士了。嫉妒他的朝臣没有一千,也有八百。但这么多人,硬是没有一个敢于出言质疑襄汉漕渠能否打通,以及打通后,是否能达到计划中的目标。或许他们心中犹有疑虑,但没一个敢于说出来,应该都是打着最后看了结果再说话的想法。

走了两步,吕升卿长叹了一口气,这不难理解。

这样的做法,看着稳重,其实就体现了他们心虚胆怯,在下意识里,已经默认了韩冈在治事的权威,以及他说到做到的能力。

如此顾忌韩冈的原因,吕升卿能体会得到。

聪明人不会栽在同一个坑里。从熙宁二年年底韩冈得官,到如今还不到十年。区区十年间,已经不知多少人多少次在韩冈身上吃了亏丢了脸,其中不乏文彦博这样的元老重臣,也不缺韩绛、吴充、冯京这等当朝宰辅,亲耳听闻的、亲眼看见的、亲身经历的,一干老臣、重臣都不敢正面再招惹这位灌园子了。

唯一能做的,就是将韩冈摒于京城之外——这也不是他们自己的打算,而是在附和天子的心意。如果天子什么时候想要召回韩冈,又有几人敢于站出来反对的?

以士大夫的脾气,就是正面与天子顶上都是不怕的,那是涨脸面、涨名气的好事。大部分的重臣,都是从御史起家,嘴皮子上的功夫,没人能与他们相比。

可吕升卿偏偏记得,韩冈最喜欢说的就是实事求是、以实证之,但凡有人被他拉倒是非真伪的辩驳中去的,没有一个不是大丢其脸,杨绘现在还在南方做着知州呢。

吕升卿一点都不觉得那时候有人会敢与冒着丢人现眼的风险跳出来。

不过那多半要等到十年后了,吕升卿觉得轻松了一点,只要天子还顾忌着韩冈的年纪,他就不可能入朝为官。

就在吕氏兄弟议论着朝局的时候,一人之下万人之上的大宋宰相王珪王禹玉,正在灯下展开一封书信。

这是种谔写来的密信,除了一开头的问候和感谢之外,大部分内容都是说着对夏开战的布置和计划。但王珪看了之后,就一脸不快的随手放在了一边——种谔在里面隐晦的提到了韩冈,并希望朝堂能将他派去陕西。有韩冈在,他们就能安心征战,而不用担心后方。

先不说区区一个武将,竟然敢干涉边帅的人事安排,单是他提到的人选,就让王珪有着几分不喜,放到天子那里,恐怕也不会干脆的点头答应。

如果让韩冈去了陕西,以他的能力,以及在西军中的声望,加上跟他配合的种谔,不出意外的话,必然能拿到头一份的功劳。那时候,就是天子不情愿,也得给他一个枢密副使做做。

但陕西那里并不是非韩冈不可,稍稍逊色一点,但也足够派上用场的人才,还是有很多的选择。

熙宁八年攻略横山的时候,韩冈还在军器监里打铁呢。不照样打得党项人狼狈而逃,逼得契丹人只敢束手观望?

有了板甲、斩马刀、飞船、霹雳砲,还有经过多年征战的名将锐卒,区区一个韩冈,多他不多,少他不少,不过是锦上添花而已。既然天子担心着几十年后的朝局,做臣子的也得为君分忧才是。

何况王珪也不想看到不到自己一半年纪的后生晚辈,站到紧贴着自家背后的位置上,只为了自家的心情着想,也得让韩冈在外面多留上几年。

十年吧,王珪算了算,那时候差不多也该致仕了,不用担心再看到韩冈那张太过年轻让人气急上火的脸。

也就在同一天,韩冈抱着方才哭得嘶声力竭,现在倦极而眠的五儿子,对身边妻妾笑道,“半年吧,半年后应该就能回京了。”淡泊的笑容中有着毫不动摇的自信,“没人能拦得住,天子不会理会的。”

