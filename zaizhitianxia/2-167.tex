\section{第30章 肘腋萧墙暮色凉(二)}

日出之时,晨钟回荡在无定河上。

新年的第一道辉光,从东侧的山头上洒向了绥德城中。

不过熙宁四年元旦的绥德城,没有鞭炮,没有喧闹,只有整装待发的两万将士,只有冲霄而起的浩荡战意。

绥德城中的校场,容纳不了太多的军队。即将出战的两万大军,都聚集在北门外的空场上。临时搭建起来的点将台上,种谔正主持者出战前的仪式。每一位将领都肃穆以待,他们都明白,这一战事关国运,将会是宋夏两国攻守易势的标志。一旦夺占并守住了罗兀城,西夏的灭亡就指日可待。

种建中仰望着自己高台上的叔父,种谔正手持御赐长剑,将祭旗的黑牛牛耳割下。

如果今次功成,当初狄青、郭逵所担任过的位置,他的五叔也将有资格坐上去。种家将的名声将会在京城中闪耀,而当年祖父的遗憾,也将就此弥补。

种建中现在是种谔帐下的机宜文字。他这个官职只是临时性的,不是各路帅府中的正式职位。在他的身边,种朴、折可适这几个年轻的武官,也都担任了军中机宜一职——实际领军他们还不够资格,但这些年轻的将门子弟的素质,却是军中难得的人才。故而被任命为机宜,以便参赞军务。

“可是今次只带了三日粮草。还有随行的民伕……”折可适回头看了一眼,在城中,还有上万民伕即将跟着他们一起出发。三万张嘴,如果要靠人力来转运,他低声对身边的种建中道,“太尉下令他们多带筑城用的工具,而口粮,也只带了三天的份量。”

“不必多虑,岂不闻取用于国,因粮于敌……横山有粮!肯定有粮!”

种建中对自家门下的谍报深具信心,几十年来,种家能立足于西军诸多将门之中,叔伯辈战功不断,除了本身的才华之外,也多亏了当年祖父种世衡断断续续镇守清涧城近十年,在蕃人中所留下来的人脉和关系。

知己知彼,百战不殆。这句老话,从来都是颠扑不破的道理。

种谔将注入了牛血的烈酒一饮而尽,拔剑上指。旌旗招展,万胜的呼喝伴随着沉重的鼓声一齐响起。

熙宁四年的正月初一。

就在天下亿万兆民庆贺新年的时候,种谔率领步骑两万,兵出绥德,沿着凝固的无定川,向北急进。

……………………

正月二日。

罗兀城的守将都罗让正为了新年的到来,而纵酒狂欢。

西夏名义上向大宋称臣,作为称臣的标志,其国中所用历法便是需遵从大宋国中通行的历法。每年秋后,新年历由钦天监计算审定,呈与天子,继而颁行天下,而大宋的属国也就在这时候得赐新历。

对天文学水平不高的党项人来说,让他们自行推算历法,实在有些吃力,用大宋的反倒方便。要不然,以他们敢于自定年号,隔三差五就来打饥荒的胆子,也不会给宋国君臣留什么脸面。

都罗让虽是党项豪族都罗家的子弟,但他御下一向甚宽,自个儿喝酒没趣,便把守在堡中的两百多人,一起都拉来了喝酒唱歌,城中的空地上,点着一堆堆火,火上都架着一口剥制好的羊,转着圈烤着。熬出来的羊油滴在火上,滋滋作响,而一股焦香传遍小城之中。

不是没人提醒都罗让最近的绥德城那里有异动,需要严加防守。但都罗让他想党项人要过年,汉人人也照样要过年。辛苦了一年了,哪边都要轻松一下,哪有大过年的的出兵打仗。

横山对大夏的价值,还有无定川的重要性,镇守在此处的都罗让当然不会不知道。不过山对面的银州城就有大军屯守,他的叔父,都枢密都罗马尾就在银州城中。若有军情,旦夕可至,都罗让哪有什么好担心的。

他从除夕开始,带着堡中守军,醒了就喝,喝倒就睡,到现在已经三天了,酒库中的存货,竟然还有三分之一剩下。

守着这座孤零零的城堡没别的好处,就是从来往的回易商队中,私下抽取的过路钱多。其中也多有用酒、绢之类的商品,来充抵过路费的。用来存放兵器的仓库,现在都被酒水、丝绢给占了去。都罗让拍着圆滚滚的肚子,他在罗兀不过守灵三四个月,腰带已经就松了半尺多。

“真是个好地方啊!”他由衷的叹着。

一声凄厉的号角声,倏然响起,把都罗让的感叹全然掩盖。

“出了什么事?!”

罗兀守将昏昏沉沉的被人强扶上了城头,就看着城下的河谷中,在宋军的红色战旗引领下,数百名骑兵已经将围住了罗兀城的城门,而南方远处的谷地更是被灰黄色的烟尘所掩盖,不知有多少兵马正向罗兀城赶来。

都罗让目瞪口呆,被酒精淹没的脑中全是空白:“这……这……这怎么可能?!”

来袭的宋军用行动回答了都罗让这个愚蠢的问题。百多名骑兵冲至城下,直接下马,开始用着箭雨扫射城头。箭雨令人惊叹的精准,把城头上的守军压制得抬不起头来。而剩下的骑兵就在他们的掩护下,竟然从马背上卸下了一截截事先打造好的构件,转眼就架起了十来具云梯。

这数百名骑兵,都是鄜延路军中挑选出来的精锐,人人弓马娴熟,骁勇敢战,号为选锋。不仅是鄜延路,其余诸路也都至少有一个指挥的选锋精锐,作为主帅最为倚重、用来改变战局的队伍。种谔一开始就把他们放出来,便是为了能一举夺城。当云梯组好,选锋们就呐喊着,抬着这些云梯,直冲城下而来。

“城主!”守卫罗兀城的党项士兵们叫着都罗让,盼着他能有个主张。

而都罗让只剩下一个念头,“快放烽火!快放烽火!”

他不断重复着这句话。宋军来的出奇不意,守城的工具全都没有准备,对于守住城池,都罗让已经不报希望。唯一的期盼,就是他的叔父,能为他报仇。

当第一名宋军选锋攀上了罗兀城头,战事的结局已经宣告注定。

等不来城主都罗让的命令,绝望的守军自行发起了的抵抗,但在不断涌上城头的宋军选锋的刀光剑影中,他们节节败退,根本无力抗衡。当城门被夺占、打开,守在城外的宋军便一拥而入,开始镇压城中剩下的抵抗

不过半个时辰的时间,被视为横山中枢的罗兀城,轻而易举地就被种谔领军攻下。

在西夏人眼中,罗兀城只是银州防线的一部分,虽然重要,但因为银州就在山外,急行军半日即至,无需在罗兀驻屯大军。当初梁乙埋筑罗兀城,也是打着以此处为前沿防线的念头。

不过在宋人看来,横山南侧的罗兀,远比北侧的银州更为紧要。控制了罗兀,就能与党项人平分横山,而以党项人对横山蕃部的压榨,一旦宋夏双方都在横山中拥有了核心据点,横山蕃部彻底倒向大宋,将是必然。

着眼点不同,对罗兀城的处置也完全不同。西夏国相所命人修筑城寨,只有两百步周长。而种谔夺下罗兀后,接下来为了抵御西夏人的反扑,将要扩建罗兀城却阔达千步。而且罗兀城不能成为孤城,附属于罗兀,以其为核心的防御体系也要同时修起。在预定的计划中,就有两座城寨要同时修造,以保护从绥德到罗兀的交通线。

“这只是开始而已。”种建中随军踏入城中时,这样想着。

他被分配下来的工作是计算罗兀城中的存粮。正如事先侦查所得到的消息,西夏人的罗兀城,最多也不过两百步的周长,但其中粮草竟然堆积成山。

需要仰头才能看到全貌的一座座粮囤,足足让今次出动的两万步骑加上上万民伕吃上两个多月。种建中不由得暗叹横山诸部当真是胆小如鼠。被西夏人欺负到这等境地,竟然还没有半点反抗。不过这对于大宋来说,却是一件好事。党项人在此横征暴敛,而大宋以宽和相待,不出半年,此处蕃部将彻底归心。

当然,前提条件是得先把驻守在银州的西夏援军击溃。

就在刚刚进逼到罗兀城下的时候,种谔就已经派出部将吕真,率其本部千人为斥候,前往北方山口处侦查敌军的动向。

等到午后时分,斥候赶来回报。银州方向,西贼已经出兵赶来救援。旗号是西贼驻守银州的都枢密都罗马尾,并有参政、钤辖旗号十数面。

“其先锋已至山中的立赏坪,半个时辰后,即将抵达马户川!”斥候在帐中急声禀报。

立赏坪就在罗兀城和银州之间的山口下,如果再算进哨探赶来回报所消耗的时间,西贼的援军现在已经翻过山了。

‘来得好快!’种建中闻言心惊,与身边的折可适交换了一个惊讶的眼神,当真是来得太快了!

