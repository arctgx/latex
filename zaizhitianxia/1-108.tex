\section{第42章 诡谋暗计何曾伤(一)}

【第二更,求红票,收藏】

东京城内外,大小酒店、食肆、铺子,有数以千计之多。但能被东京城百万士民口耳相传的,只有七十二家正店。其中有的是官营,有的是民营,有的原是行会会馆,也有的本是豪门旧宅,来历五花八门,但名气却都是一般儿的传遍天下。

位于东京内城新门里的会仙楼正店,虽然比不上樊楼的富贵奢华,也比不上清风楼的店面广大,更比不上御街边的张家园子和状元楼的地势绝佳。但会仙楼有个优点,便是闹中取静,尤其是后院的诸多雅间,都以幽静隐秘而著称。

坐在会仙楼的楼上靠北临窗的座位,不但可以纵览汴河胜景,还可以望见北面不远处,隔着一座虹桥,就在汴河对岸的开封府衙。只是很少会有贵客来选择在楼上用餐,二楼三楼的桌位,日常多半是被开封府的低层官吏所占据。在后院的花园中,被假山、树木、小桥、池塘,还有几条蜿蜒曲折的长廊所分割出来的座座雅间,才是会仙楼中最为受到欢迎的地方。

流内铨令丞刘易,近几年来,还是第一次走进会仙楼的后院。虽然他也是个官人,而且还是京官。但在物价腾贵的东京城中,他一个从八品大理寺丞的些微俸禄,想养活全家十几张嘴,还要应付不时来打秋风的乡人,早已是捉襟见肘。

与平常百姓幻想的官人们的富贵生活不同,刘易这样的青袍小京官,他最为常见的待客方式,就仅仅是在路边的小酒肆中胡乱吃上一顿。即便这样,他的钱囊一个月也经受不起几次消磨——留京城,大不易。

被一位知客在前引着,刘易穿廊过户。他看着前面知客所穿的衣服,竟然不比微服而出的自己差上多少。尽管刘易穿得不是质地优良的公服,但身上现在的这一件用也是不错的料子。可区区一个仆役,竟然能跟他这位官人相比!

在廊道上左绕右绕,最后刘易在客的带领下,终于走进了一间门额上龙飞凤舞的写着忘归莲华四个草字的小厅中。厅门内,迎面便是是一张四扇屏的荷花屏风。四张荷花姿态各异,有含苞欲放,也有花开正艳,还有残荷独枝,中间偏右的一幅上,一支亭亭独立的半开花瓣上似有似无的还带着点点水意,当是出自名家手笔。

绕过屏风,就看见长着一张方面大耳,面白留须,模样甚有威严的中年男子在窗边坐着。将人引到,知客便退了出去。进退间不发一言。没有不呼自来、筵前歌唱的打酒坐妓女;也没有腰系青花布手巾,为客人换汤斟酒,俗称焌糟的妇人;更没有一拨儿插科打诨、博取赏钱的厮波闲汉,一切都保持着尽可能的安静,便是这间会仙楼后院的最大特点。

刘易走上前,躬身向中年人行礼:“下官拜见侍制。”

中年人指了指旁边的一张桌子:“坐!”

刘易看过去,桌上早已摆满了冷碟和果子。注碗、盘盏、果菜碟、水菜碗,大小十几件,还有两人座前的酒盏、酒壶、筷子,无一不是闪闪发亮的银器,加起来不啻百十两之多。

东京城中,只有七十二家正店才有这般豪阔的财力,寻常的脚店和小酒肆,即便想做的奢华一点,用的器皿也得到正店来借。

两人落座,很快一盘盘热菜也端了上来,每一道依然是用着银碟盛着,特制的银碟下,还有着阴燃火炭的托底,以保证菜肴不会很快冷去。

端菜来去还是悄无声息,知客最后在屏风处站了一站,见两位客人没有其他吩咐,便躬身退出门去。小心的将门掩好,厅中就只剩下刘易和中年侍制两人。

只有午夜时分,山中寺观才有的寂静降临在厅内,厅外的杂音一点也没透进来。小厅以莲为名,窗棱、桌案、梁椽,乃至杯盘碗碟,处处都打着莲花的记号。就连在窗下燃着的熟铜火盆,也是一朵完整的千叶莲花。袅袅香烟同样自荷花花苞形制的青铜香炉中丝丝缕缕的升起,在厅中扩散开。一股淡淡绵香在鼻尖传递,香味清而醇,不似寻常薰香的浓烈,正是应了这间荷厅的特色。

刘易无意多看,厅中死一般的寂静让他坐得很不自在,他陪着小心,问道:“不知侍制唤下官来此,为得何事?”

中年人第二次开口,说得话多了一点:“……近日可有一名秦州新选人来流内铨递家状注官?”他顿了顿,又加了一句,“有天子亲下特旨的。你可知道?”

刘易当然知道。天子亲下特旨,为年岁不到的选人派定差遣,这还是新条贯颁布后的第一次。身为流内铨令丞,哪有不知道的道理,“是不是韩冈?”

“没错,正是他!”

“不知侍制想要他如何?”刘易还明白,韩冈已经被定了差遣,如果要帮他只要在旁边看着就行了,既然侍制提及他,只可能是使坏。

“两天后,安排他参加铨试。”中年人的要求很简单。

刘易吃惊的猛摇头,这怎可能做到:“铨试是为了定差遣,但他本已有了天子特旨,差遣早定下了。秦凤路经略司勾当公事,兼理路中伤病事宜。根本不需要再参加铨试啊……”

中年人身子略略前倾,只一动,在刘易眼里就如山岳倾颓,迎头压来,只觉得沉沉的有些难以喘息。就听中年人问道:“韩冈……他有没有出身?”

刘易老实的摇头回答:“没有!他只是个靠举荐得官的布衣而已。”

“无出身者注官候阙,难道不是必须要参加铨试吗?”中年人轻轻笑了几声,有着一点偷了空后的得意,“朝廷即有条贯在,依律而行便可。汝等尽忠职守,天子还能说不是不成?”

“……下官明白!”刘易略一思忖,便点头称是,对面的人说得的确没错。他笑道:“请侍制放心,下官自然会好生料理韩……对了!”刘易的眉头又一下皱起,“新官铨叙,陈判铨肯定会在场。下官从何下手?”

中年人脸上的微笑书写着自信,轻轻点着酒杯的手指,让一圈圈波纹在银边装饰的液面上回荡,好像就是在说着一切尽在掌握中,“你们的判流内铨事,那一天不会留在衙门里。在京百司,每天都要轮上两人上殿廷对,奏报司中大小事务。两天后,正好轮到陈襄和度支司的左仲通上殿。”

“原来如此!”刘易点着头,他这时才醒悟过来,眼前的这位侍制本就是管着殿廷轮对的次序的,“既然陈判铨不在,要安排起来就方便多了。侍制请放心,有下官,再加上程禹,包管让韩冈过不了铨试这一关。”

中年人轻轻点头,很细微的动作,就让刘易喜出望外。

刘易抬手为中年人斟酒,随口笑着问道:“只是下官在想,韩冈不过区区一个从九品选人,为何要与他为难。仅仅是铨试,又不是进士举,即便今次不过,官身照样还在,也不过是要等个一年半载再轮来考差遣。大费周章的,不知……是为了……”

刘易的声音越来越小,眼前之人突的变得冰寒的眼神让他感到畏缩。宛如被撬开了八片顶阳骨,一桶夹着冰块的河水当头浇下,浑身从骨子里都瑟瑟发寒。他立刻低头认错,“下官多嘴了!”

可透过这冷如高山玄穹的一眼,刘易已经看透了面前的宝文阁侍制的真实用心。剑锋所指,并不在韩冈,而是在王安石!

对,没错!正是王安石。韩冈虽是由王韶、吴衍和张守约三人共同推荐,但亲自请了天子的特旨,赐了差遣的,却是王安石。只要能在铨试上证明韩冈才学能力并不合格,就等于是在说天子无识人之明。而天子多半便会把这笔账算在了王安石的身上。

若在过去,天子并不会把这等小事放在心上,但如今以王安石所面临的境地,刘易相信,他的倒台只要再压上几根稻草。韩冈也许只是一步闲棋,但闲棋多了,即便以参知政事的权柄,也是承受不住这样的分量。

中年人这时站起身,丢下一句“好自为之!”,便抬步出了门去。

刘易手忙脚乱的陪着站起,却识趣的并不将之送出门。就站在屏风边,看着中年人并不宽厚的背影消失在门外。人已经远远的走了,藏在心底的八个字才缓缓出口:

“项庄舞剑,意在沛公!”

“管他呢!”又发了一阵呆,刘易毫不在意冷笑一声,韩冈又不是他亲戚,王安石也不是他举主。何况让他这么做的,又是得仰着脖子才能看到的宝文阁侍制。听话受教,自然会有好处,如果不听话……刘易可不想去偏远小郡做官。

只是他一个小小的京官,竟然能把手插进高层的争斗中。即便只是轻轻的搭了一下,推了一把,保不住什么时候就会被碾得粉身碎骨,但这种撬动朝局的感觉,却让他心醉神迷!

拿起酒壶,刘易给自己满满的倒上了一杯会仙春靡,又直接用手抓一条玉板鲊丢进嘴里。自他进了忘归莲华厅后,并没见到那一位动过筷子哪怕一下。现在他走了,一桌的上品宴席,便全便宜了自己。

尝着佳肴,品着名酒,刘易快活的哼着小曲。有酒今朝醉,无酒亦自眠。想那么多作甚,好好的犒赏一下自己才是真!

