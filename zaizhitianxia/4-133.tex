\section{第20章 冥冥鬼神有也无(九)}

所谓文法,就是成文的制度、规条,是让一个国家正常运转的基石。

如今国中通行的刑统、疏律是文法;禹贡中的‘五百里甸服、五百里侯服’也是文法。

契丹、党项建立辽国、夏国,都是从设立文法——创造成文的官制、法律开始。

一旦拥有了文法,就代表一个蛮荒部族,变成了拥有了秩序的国家,从野蛮走向文明。

对于中原王朝来说,单纯的蛮部带来的威胁,最多也只是骚乱而已,仅仅癣癞之疾。但有了文法之后,一个新兴的国家可以不断吸收周围的部族和人民,扩张自己的势力,对于中国的威胁,往往要大上几十倍、几百倍。

从历史也好,从现实也好,明证处处可见。

当初吐蕃赞普唃厮罗正是在青唐王城订立文法,让宋廷一直深以为忧,直到唃厮罗父子相攻,这才放心下来。

而熙宁初年,朝堂上关于是否要开拓河湟的争论,其中赞成派最重要的理由,就是董毡、木征开始在河湟订立文法,可能会让青唐地区变成下一个西夏。

有了辽国、西夏两国持续带来的威胁,大宋对于周边的部族极为警惕,毁灭交趾的秩序,将混乱带回交州,让交趾从一个有着文明的国家,变成蛮部聚居的土地,对大宋百无一害。朝堂上或许有杂音,但在天子和两府之中,没人能拿着仁义二字来责难韩冈的行事。

这个道理两个幕僚都明白,但李复还是很难适应韩冈的改变。在京城的时候,韩冈虽然不是满口仁义道德,但也不会将数十万人处以肉刑的事轻描淡写的不当一回事。温文尔雅的讨论经传上的文字,但到了广西之后,张口就是血淋淋的话语。

“抚有蛮夷,奄征南海,以属诸夏。”李复已经忘了韩冈的身份,只把他当成了书院中正在辩论经义的同窗,“交趾朝堂上下的确是罪不可恕,而交趾百姓何辜,何不以仁恕之道教化之,日后以为大宋子民?”

“仁者,人也。”苏子元冷硬的声音从门外传入,随即邕州知州踏进房来,“化外蛮夷,无异于禽兽之属,岂能与华夏子民一视同仁?交贼入寇,三州生灵涂炭,十万大军,家家户户皆有出兵。事涉谋乱,本就是要株连九族。只用刖刑处置男丁,已是仁德无比了。”

苏家阖门死难,连同五万邕州百姓同遭兵焚,苏子元眼中的恨意滔天,连忙跳起来迎接的李复、陈震甚至不敢直视,只能低头行礼。苏子元代表邕州百姓要交趾血债血偿,谁能反驳?

“抚有蛮夷,前提是恭顺。若是不顺,自是雷霆万钧。”韩冈也在配合苏子元,“交贼犯顺,上下同罪,判罪也自当一同。”

“只是若行此法,恐交人顽抗到底,不肯降伏。”

“不妨事的。”咧嘴笑起来的韩冈在油灯的暗弱光芒下,露出的几颗白牙森森的泛着寒光,“比起化夷为汉可要容易多了。为天下开太平,刀剑总是先上的。”他不介意教一教初出茅庐的年轻人,什么叫做现实,“教化二字,光用笔写可不够。”

征服一个已经成型的国家,扭转一个偏离本源文明、已经拥有自己特色的文明,至少要穷三五十年之功方得小成。在这段时间中,不能选派错误的官员、不能执行错误的政令,一直都要小心谨慎的对待,并不断加强与本土的联络,直到两代人之后,当地的百姓重新成为诸夏的一员,这样差不多才能安定下来。

但这样去做太麻烦了,很可能到了半途就一切辛苦都灰飞烟灭。甚至不需要亲手去做,只要谋划一下,想象一下,就会知道这么做有多么麻烦了。就如修补一座房梁都坏了半截的破房子,不如拆了重造一样,直接动手清理,就要容易得多。

解决交趾百年之患,最简单直接的办法,将已经脱离旧轨的交趾打回蛮荒,继而再将诸夏文明带回来。对于贪财嗜利的蛮部,把他们捆在大宋身上,并不是多难的一件事,韩冈也自有一番打算。

似乎是要缓和一下气氛,韩冈笑了一下,“自然,现在讨论如何处置交趾,未免太早了一点,日后有的是时间。在这之前,还是得先打进升龙府再说。不要羊还没杀,就讨论该放什么调料。”他看着苏子元,“伯绪此来,也是要与我说及此事吧?”

“当然。”苏子元点头,韩冈回到邕州之后,第一件事就是号令三十六峒蛮部,这段时间有许多事还没来得及韩冈商议。

陈震看看有点呆愣的李复,站起身来,“龙学、使君即有要事商议,我等先行告辞。”

“无妨,”韩冈摇摇头,“都留下来听一听吧。燕达就快到了,尔等即为我辟为椽属,军中之事,还是多听一听为好。”

……………………

燕达稳当当的踏着船板,走上了桂州的土地。

漓江两岸秀丽无双的山水,让看惯了关西厚重粗犷的山峦的燕达,也不由心醉神迷。有别于深深呼吸一口有别于北方的湿润清新的空气,他觉得自己在一瞬间喜欢上了这个山清水秀的地方。

只是恍惚仅仅维持了一瞬间,燕达立刻就回过头来,望着船上船下面目憔悴的麾下将士,皱紧了眉头。

就在十月中旬的时候,五千西军将士终于在安南行营兵马副总管燕达的率领下,抵达了广西桂州。不过他们下船的地方,离着桂州城尚远,并没有直接泊入桂州城边的码头。

燕达带着南下的秦凤路和泾原路的十四个指挥,其中有八个步军指挥,六个马军指挥,总计五千三百人。他们经过了长途跋涉,而且是长时间的水路,有许多士兵生了病,绝大部分都是憔悴无比,莫说上阵,就连行军的都难以胜任。

为了防止他们的出现,影响到桂州乃至广西的民心士气。章惇在离着桂州城二十里的地方,给他们安排了落脚的兵营。一切做得仿佛是要对外隐藏这一批援军。

尽管消息根本隐藏不了,就算桂州城中只会洗菜做饭料理家务的妇人都知道大军已经抵达桂州。但在民间的传言中,都只是以为经略章相公是为了偷袭交趾,才特意隐瞒了他们的到来,而没什么人猜测是水土不服的缘故。

“水土不服?……笑话!”

“也不看看现在做着转运相公是哪一位?那可是药师王菩萨的大弟子啊!”

虽说传言经过几千里的传播,已经变得十分离谱,但韩冈在医疗事务的权威,依然得到所有人的肯定。不过为了保证五千西军能早日康复,章惇已经事先派出了手上所有的医官,在营地中守候着。领头的医官雷简,是从关西战场上下来的熟人,在这五千人中也颇有威信,现如今正在照料五千将士的健康。

将对安南行营主力能否及时康复的担忧放在一边,关于这十四个指挥实际上到底有多少兵力,章惇现在则更加放在心上。只是他不能去亲自去计点,还是在第一时间向燕达询问。

“章帅尽可放心。”燕达一点也不隐瞒,连己方兵力的数目都瞒着章惇这位主帅,只会给之后的战事添乱子,“末将所率部众,总共四千六百余人,战马一千两百余匹。这十四个指挥,是末将亲自挑选,是两路之中最精锐的指挥。”

章惇很是满意,脸上多了点真切的笑容。兵籍上的五千三,能实有四千六,这个比例可以算是很高了,也可以证明燕达所言非虚。

“只是末将担心战马。”燕达的一张凶恶得能吓哭小孩子的脸正发着苦,“战马一千两百,皆是出自熙河,不习南方气候。现在是人有医,却没有畜医。时日一长,恐怕会有所损伤。”

“这事逢辰你大可放心。”章惇自信的笑道,“等你们身体调养好了,南下的时候,本帅可以给你们多配上两千匹马骡。”

“当真!?”燕达惊问着,立刻又注意到了自己的言辞,连忙道歉:“末将不是怀疑章帅,只是没想到做得如此周全。”

章惇不以为意:“广西牛多,马骡也不缺。边上就是大理国,出自大理的马匹虽然不必上河西马的高俊,但惯于在山地上行走,也能吃得了苦。”

“难道说广西有马市?”燕达更加惊讶,外国的马匹要想进入中国来,只有通过互市,但他南下时可没有听说

“这是苏忠勇和苏伯绪父子两代辛苦的结果。”章惇轻声叹了一口气,苏缄和苏子元都为了开辟马市而奔波劳累过,虽然之后有章惇、韩冈联名上本才建立了马市,但功劳还是苏家父子的,两人都没有掠人之美的想法,“邕州的横山寨,还有宜州,两处马市虽说只开了三个月,可到手的马匹就有四千之多。”

“四千?!”听到有这么多马,燕达的声音顿时就提高了一截,搓着手,兴奋地说着:“有这么多马匹,接下来的一万三千兵马南下后,也足以补充上了。”

章惇看了兴起的燕达两眼,黯然一声叹:“看来逢辰你是没有听说……因为契丹在丰州暗助西夏,不会再有援兵来了。安南行营能动用的主力就只有你带着南下的五千兵马和前次随我南下的一千五百荆南军!以及桂州、邕州刚刚训练出来的五千新军。”

