\section{第14章 飞度关山望云箔(五)}

“邕州那里有动静没有。”阮平忠昨天晚上累得够呛,手下献上来的新货被他折腾了一宿,现在腰酸背疼的,从驿馆的房间里面走出来双脚都打着晃,“城破了没有。”

“没有,还没有消息来。”阮平忠的副将黎生摇摇头,对着自己的顶头上司道,“不过也快了,昨天就说已经上了城,说不定现在都拿下苏缄的首级了。”

“邕州若是当真被打下,肯定要屠城。为了这座城,耽搁了多少时间,死了多少人。李太尉可不敢压着下面的人。”阮平忠坐了下来,叹了口气,他们与邕州大营的联系,仅仅是一天互相通报一次,隔了几十里地,联络也很不方便,“可惜赶不上了。要是现在被调去给邕州最后一击,不知能落下多少好处。”

“那边都杀红了眼,谁肯让我们摘桃子。照我说还是早点破城最好,就能解了这个倒霉的差事了。”

“也不算差了。这里闲归闲,要命的事也少。要是摊着了攻城,下面还不知要死多少。没听说武胜军、飞捷军都给拼光了吗?”

“哈,说得也是。”

阮平忠和黎生领着一千人占据了昆仑关后的长山驿,离着北面的昆仑关有二十多里的路程,距离南面的邕州城则有六十里。

李常杰不会全心全意的相信广源州的蛮帅们。黄金满受命把守着昆仑关,提防着北方的来敌。而阮平忠和黎生的任务就是监视着黄金满——以作为昆仑关援军的名义。

被调来看守黄金满,一开始阮平忠和黎生两人都以为是倒运的差事。邕州的富庶是有名的,里面尽是金银财帛,打进去后,人人都能分到。而自家只能,等着上面的那点微薄的赏赐,这让怎么会甘心?

不过当战报一天天从邕州城下传了过来,两人都越来越庆幸自己的好运。

“塞翁失马、焉知非福。当真要多谢李太尉和宗太尉的抬举了。”阮平忠上过学、读过经书,尽管交趾国中的进士也考不中,引用一两句成语倒不在话下。

肚子咕咕叫了两声,阮平忠用力一拍桌子,冲着门外吼道:“人呢?!死哪儿去了,本将军都起来了,不晓得端茶端饭上来。”

片刻之后,一名使女就慌慌张张端着饭菜茶汤走进来。自打进广西,他们一路抢来不少民女,姿色好的留在身边,差得送进军营,而能看得过去的,就被逼着服侍着仇人。

那名使女进来之后,一见到阮平忠阴沉着脸,就浑身发抖。走到阮平忠身边,连手上端着的托盘都在上下颤着,“奴……奴婢万死,请将……将军恕罪。”

“怕什么。”阮平忠笑眯眯的说着,“小心服侍怎么会责罚你?”

“是……是。”使女面色如土。她可是亲眼看见眼前这个看似和善的交趾将军,是怎么虐杀了前面一位不小心犯了错的同伴。

她双手颤着端上茶。越是要小心,却越是犯了错。脚下没站稳,一杯茶就泼在了地上,几滴茶水溅上了阮平忠的靴子。

阮平忠低头看了看,眼睛就瞪得如同铜铃一般,一句话也不多说抬腿就是一脚飞踹。身材矮小的少女咚的一声就一头撞在墙上,昏了过去。

阮平忠站起身,要上去再来几下。他最近闲得没事,心中也时常烦躁,都是靠着杀人来恢复心情。

黎生一手拦住他:“不要浪费。”

“……拖到营里面给那些小子去。”阮平忠想了一想,就挥了挥手,让外面的侍卫将昏倒的使女拖出去。坐下来后,又变得气定神闲,仿佛什么事都没发生过,“不知道黄金满那边怎么样了。”

“那条老狗就缩在关里呢,哪里有什么动静?要是换做我们守着昆仑关就好了。”阮平忠的副将变得有些不忿气,“听说刘永他竟然跑去打宾州了,虽然比不上邕州,但好歹也是块肉啊。”

“不说没打下城池吗?”阮平忠从来都看不起广源州的那群蛮子,“不过谅他也不敢打,看到邕州打得这副惨状,看到宾州城,哪里敢硬攻城了。”

“就算是村子,也少不了有些财物。就算没有财物,好歹也有人口。”

“我们在长山驿守着,刘永敢不分我们一杯羹?”阮平忠冷笑起来,“就是刘纪来了,也照样得按规矩来。也不想想李太尉会帮谁?”

黎生连连点着头,刘永在宾州肯定收获不少,到时候要他个三五成,也不算欺负他,“到时候挑几个好货色,也好带回家里去。”

两人正说得开心,突然间外面就是一片叫声响了起来。阮平忠和黎生猛地站起身,一名士兵就冲了进来,“杀……杀……杀过来了,昆仑关败了,宋人杀过来了!”

“什么?!”阮、黎二将大惊失色,立刻冲出了驿站,驿站外的营地现在乱作一团。而从昆仑关的方向上,正可以看见满坑满谷的广源蛮军,正一窝蜂的逃了过来,乱得不像样子,连旗帜和盔甲都丢了。等蛮军来得近了,就看见逃在最前面的几张熟悉的面孔。

“黄元?到底是怎么回事,你爹人呢!?”阮平忠留下黎生整顿营中秩序,自己则又惊又怒的冲上前,“昆仑关怎么失的守?!”

黄金满的儿子没理会阮平忠的发问,只大吼一声,“动手!”随即就是一铁鞭照头挥来。

阮平忠甚至没有想明白是怎么回事,只凭直觉就翻身滚下马。卡擦一声脆响,黄元的铁鞭没能将阮平忠的脑袋打成碎瓢,可还是击中了肩膀,将披甲的交趾将军肩骨打得粉碎。阮平忠在地上翻滚着,正要跳起来逃开,立刻被黄元身边的蛮兵扑了上去,绑了个结实。

就在同时,原本一群败军,纷纷冲进了营地中,用着交趾话在营中见人就杀,又乱吼乱叫:“奉大宋天子命,讨贼逐寇。”

“十万天兵已至昆仑关!”

这些所谓败兵,其实都是黄金满挑选的精锐。作战少有讲究阵法号令,往往不是正规军的对手,但在乱战之时,他们的武勇却不是惊慌失措中的交趾兵能比得上的。

一见阮平忠被打落马下,黎生在第一时间就逃了出去,他很清楚在这等乱军之中,不可能再收拾起兵马来。没了两名领军的将领,失去了军中的主心骨,什么士气都没有了,一千交趾兵连像样反击都组织不起来,如同散开的鸭子一般,被他们向来看不起的广源蛮兵拿着刀枪一路追杀下去。

昆仑关很容易被绕过去,经常被前后夹击。一千交趾兵不驻守在昆仑关上,而是守着后路,军事上也说得过去。但李常杰让阮平忠守着关后二十里的长山驿,更多的还是应该有着监视黄金满所部的的任务。

而且执行这个任务,如果与目标近在咫尺,很容易会引起双方的冲突。交趾军维持的二十里的距离,也是为了让自己的差事能够顺利的完成。只是这二十里的距离,就让韩冈派去的说客有了可乘之机。也让黄金满得以从从容容的拟定计划,统领麾下三千兵马反戈一击。

一个时辰后,成了阶下之囚的阮平忠,捆成一个粽子丢在长山驿的庭院中,黄金满在苏子元身边说着,“可惜逃掉了黎生。”

而黄金满手底下的士兵,正收拾着满地的尸骸。驻守长山驿的交趾兵跑了一多半,落在广源蛮军手上的不论死活则都给砍了脑袋夏利,又从交趾军营中救出了六十多名掳掠而来的女子,还有一堆没能带走的财物。

一战又是近四百斩首,不说朝廷发给的赏赐必然丰厚,就是能杀一下一直压在他们头上的交趾人的气焰,黄金满手下的洞主蛮将们都是满心欢喜。不过他们在苏子元面前倒不敢流露出半点自满的模样。

黄金满在广源州一众的洞主蛮帅中算得上是稳重的一个。要不然也不会刘永跑出去大抢特抢,而他还约束着自己的部众,不让他们出去分一杯羹。一开始黄金满的部众们,私下里都有人说他胆小如鼠,听说了宋军已经抵达桂州就吓得如同受了惊的兔子,钻在洞里不肯出来——从桂州到邕州千里之遥,宋军哪里会来得这么快!

可当他们在苏子元和何缮两人确认了刘永全军覆没的消息后,才知道自己有多幸运。刘纪家一千多战士,被八百官军杀得干干净净,就只换回了对面的四条性命。面对如此骁勇之师,有关墙护着还能勉强自保,要是撤退的时候,给吊在身后,那可就只有全军覆没一条路了。

对官军的畏惧存在心中,一群洞主将斩获的首级都献了上来,讨好的簇拥在苏子元身边。

苏子元眯起眼睛看着头颅堆起的几座小丘,满意的点着头,“洞主的忠勇,本官是看到了。必然将此战报与韩转运,书呈桂州和朝廷。尔等且等着赏赐好了,天子绝不会吝惜。”

“小人既然已经归顺朝廷,正要赎了过往的罪孽,哪里敢不卖力?”

