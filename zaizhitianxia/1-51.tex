\section{第24章 自有良策救万千(下)}

“土性松软,用来固定伤处,怕是不会太牢。”仇一闻突然说道,在他看来,韩冈的理论并非没有破绽。军营中,跌打损伤都是最为常见的伤患。很多仅是普通的骨折,只因为正骨后护理不当,导致骨骼生长错位,变成了终身的残疾。就算是岐黄老手的仇一闻,也改变不了如此现状。

韩冈瞥了仇老军医一眼,道:“我说得土,不是地上的泥土,而是石膏。”

金木水火土这五行,只是大的分类,下面还有细分。金银铜铁锡五金,属金类。杨柳榆槐松,是木类。如石膏这等无法冶炼等矿物,都是算在土类中。石膏此时与后世不同,很少作为建筑材料使用,平常人们用的只有石灰。石膏的用处,反倒是在药材上多一点。石膏性寒,有解热毒、清热病的功效。

所以雷简诘问道:“石膏大寒之物,用于骨伤,有何根据?”

“石膏是外用,并非内服。而且欲用石膏治骨伤,必须先将其煅烧后化为粉末,去其寒性。再用水调和成泥状,糊于已经用柳木绑扎好的伤处,最后用麻布扎紧。煅烧过的石膏遇水便凝,坚实如石,根本不怕骨头再次错位。柳木板、石膏粉还有清水,分属木土水,也就是说,要想将骨伤养好,须得同时有水、土、木滋养。”

韩冈辩才无碍,雷简和仇一闻已是无话可说,反倒是越想越有道理。医官讲究的是药性,药理。而跌打郎中则是治好就行,对两边所用的措辞并不一样,韩冈都是对症下药。而仇、雷两人,也确实被他唬得一愣一愣,虽说不上崇拜,但投向韩冈的视线却都有了几分敬意。

齐隽也傻了眼,一真一假的两只眼睛同样的呆滞,他怎么也想不到韩冈竟然还会医术——好吧,其实这他有所预计,但比雷简、仇一闻还强,那就完完全全出乎他的意料。这下子该拿韩冈怎么办?看韩冈在伤病营中的威风,想暗地里下手都是没用,说不定还要把自己搭进去。

“韩秀才果然医术高明,佩服,佩服!”听着韩冈说得鞭辟入里,仇一闻并不吝啬自己的夸奖。

可韩冈却摇头道:“韩某没有学过医术,望闻问切,在下一窍不通,下针开方,在下也是一点不懂。韩某方才所说的,不过是拾人牙慧,转述而已,不敢居功。”

“转述的是谁人之言?”雷简和仇一闻同时追问道。韩冈所转述的道理发前人所未发,医术当是了得。

“一个游方道士……那是今年五月的事了,韩某正在渭州游学于子厚【张载字子厚】先生门下。”韩冈微微扬起头,目光迷离,似是在回忆,但实际上却是在飞快地编织谎言,“刚过端午的时候,子厚先生受朝中吕学士【即时任翰林学士的吕公著】推荐,要入朝任官,韩某本欲随行,不曾想却接到家中的书信。”

听到这里,众人对韩冈肃然起敬,而齐隽几乎要破口大骂,韩冈竟是受到了翰林学士吕公著推荐的张载的弟子,赫赫有名的横渠先生的亲传!难怪陈举送来的厚礼那般的沉重,人家的身份贵重啊!该死的陈举,竟然要让他陷韩冈于死地,若是真做出来,横渠先生岂肯干休?韩冈的同学们岂肯干休?

‘你不仁,也莫怪我不义。’齐隽前面还认为是韩冈行了大运,捡了便宜,现在想来,行了运的也许是他自己。

齐隽对陈举恨不得寝皮食肉,想着该如何报复。这边,韩冈仍在叙述着自己的神奇遭遇,

“你们也知道,四月正是西贼入寇秦州的时候——”他笑了一笑,笑容显得有些惨淡。

“那信里……”周宁问着,韩冈的家事内情,民伕中都有所传言,能猜到信中大概说得是什么。

“信中说得便是韩某两位兄长皆没于王事,要我赶回家去奔丧。”韩冈长长得的叹了一口气,“当时我冒雨往家赶,没想到因此受了风寒,到了半路便病倒在路边的山神庙里。”

“秀才真是好命,逆旅得病,稍有不慎,就是一条人命。”仇一闻对道路边的小庙都很熟悉,知道里面常常会有些半路得病,死在庙中的旅客。

“是啊,的确命好。韩某当时独自躺在山神庙中,身下连个草窠子也没有。山神庙还漏雨,人就泡在水里。躺了半日,已是人事不知,命悬一线。”韩冈说起故事来,七情上面,只看他的表情,却如真的一般,“没想到正巧一个道士进来。”

“那道人一丸药就让韩某发了汗,转眼病就退了一多半去。”韩冈深情的缅怀起并不存在的人物,“他照料了韩某两日,期间谈了不少有关医术话题,也包括骨折的事。当他走得时候,还让韩某再躺一天,否则还会再病起。他的嘱咐,韩某虽信却无法遵守,毕竟奔丧事急。只觉得有了点气力,就又强撑着往家中赶去。不想病势复发,进门就倒了,差点儿就没命了。直直在床上躺到了一个多月前才能下地……”

“这个道士究竟是什么人?姓甚名谁?”雷简急问道。

韩冈气定神闲的为自己圆谎,“那道士当是闲云野鹤一般的人物。名讳倒没说,只知道姓孙!”

王君万为寻找雷简和仇一闻,踏入了伤病营,正正听到韩冈的最后一句。站在人群背后,王殿侍插言问道:“谁姓孙?”

没有人回答他,雷简、仇一闻还有齐隽都直愣愣的看着韩冈,说不出半句话来。

……………………

半日后,韩冈已经站在了甘谷城衙的后厅里。他只用了‘孙道人’三个字,就让韩冈这个名字直接传到了秦凤路兵马都监兼甘谷城主的耳中。

须发花白的张守约正坐在厅堂内,王君万和一众官吏罗列其左右。

“你就是遇仙的韩冈?”甘谷城主开门见山的问道。

“遇仙?”在秦凤路都监面前,韩冈双唇微张,神色茫然,“这是从何说起?”

张守约眼睛一转,如屋外凛冽北风一般冰冷的视线就落到了王君万的身上。王君万惊问韩冈:“韩冈,你不是说过遇到了前朝的名医圣手孙真人【孙思邈】吗?怎么又改口了!?”

“韩某几曾说过?!”韩冈也是又惊又怒的模样,“我只是说过,当初救了在下一条性命的道士姓孙,如此而已。这与药王孙真人又有何干?孙真人生在唐初,距今几百年,如今岂会在世?韩某圣教弟子,不语怪力乱神!”

当早前韩冈将编的谎话中,救了自己一命的道士说成是姓孙的时候,他就已经对随之而来的传言有了心理准备,这也是他希望发生的情况之一——药王孙思邈孙真人在关中名声赫赫,几百年来,有关他的传说数不胜数,至今未绝——而结果也如韩冈所预料,甘谷城主张守约因为韩冈在伤病营的表现,更因为遇仙的传言,而将他招到了面前。

“你!”王君万踏前一步,怒意难遏。

“好了,吵什么!”张守约一喝斥退王君万,又转对韩冈道:“听说韩秀才你并不懂医术,这样也能救人?”

“在下在伤病营中用的是治术,而非医术。不闻群牧监要知养马放牧,也不闻司农寺须会种地耕田。何须懂医术?又非致命伤,能活到现在,如何不能活到未来。只需精心照料,又有几人会枉死?如今伤病营中,多少人已在康复中,正是明证。”

王君万不火了,性急的问着:“不知韩秀才你有多少把握,把俺的儿郎们都救回来?俺这里还有十几个亲近兄弟在家养着。”

“韩某不敢保证个个都能痊愈,但能确定,绝对要比过去少枉死许多。照顾病患,不是施针下药,重要的是用心!”韩冈有绝对的自信。他的信心同样来自于伤兵救护,不是别人,正是后世的传奇护士南丁格尔。

十九世纪的战场上,伤兵的死亡率并没有因为科学进步而下降,始终都保持在三成到五成的水平上,不是因为医药,而是因为用心与否。当英法俄土在克里米亚开战,南丁格尔带着护士队来到战地医院,没有高超的医术,没有神奇的药物,只凭着精心的护理,提灯女神就让伤兵在战地医院的死亡率降到了个位数。这是仁心带来的奇迹,也是韩冈打算复制到甘谷城伤病营的前景。

这不是王君万期待的答案,但能有这个回答,他已经很满意了。回过身,他代替韩冈向张守约请求道,“都监,不如就让韩秀才领了伤病营吧!雷大夫和仇郎中都听他的。”

“韩冈,若老夫将伤病营……不,将甘谷城内所有的伤病都交给你,你能不能照料得过来?”

“不闻万人敌是真的要上阵砍上一万人,韩某要照料人,也不必每一个都亲自动手!”

韩冈的回答有些狂妄,厅中的一应官吏都听着不快,但张守约并不以为忤,有才气的年轻人若无一点傲气,那就反而奇怪了。而且韩冈还是不顾危险、连夜赶入甘谷城的唯一一支队伍,这份人情张守约也是记着的。

“那就这样罢!”张守约最后拍板,“将城东南的那座营地空出来,把所有的伤病都转过去。齐隽,韩秀才要什么,你就给什么!嗯……钱和兵器例外!”

“诺。”齐隽毫不犹豫地应声答诺,现在韩冈才是他需要结纳的人物。至于陈举……他是谁?

“韩冈,甘谷城中的伤病都交给你了,望你勤勤谨谨,毋负众军之望。”

“都监放心,学生明白!”韩冈谦卑的躬下腰,低下去的脸上却是大愿得逞的笑容。

ps:好了,这就是韩冈的手段,不需要医术,只需要一点仁心和卫生常识便足以。

今天第三更,求红票,收藏。

