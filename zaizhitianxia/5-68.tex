\section{第九章 拄剑握槊意未销(16)}

【第三更】

“韩卿此去太原府。并州之政,河东之兵,朕尽托付于卿家。御寇抚民此等事,有卿家在,朕可高枕无忧。若遇军情紧急,不暇上禀,卿家可便宜行事。”

“为陛下分忧,臣之职也。臣冈谨受命。”

崇政殿中,韩冈与赵顼交流着没什么实际意义的废话。

猜测终于成为现实,韩冈不能不往争权夺利的方面去想。但话说回来,韩冈也不需要谦虚,他坐镇并州太原府,没有任何可以供人指摘的地方。

眼下无论是赵顼,还是两府宰臣,都不相信辽人会攻打河东,硬碰以雁门、瓶形【今平型关】二寨为主体的寨堡防线。

西陉、胡谷、雁门、土墱、大石、茹越、麻谷、瓶形等沿着边界排开的大小十五座军寨,以及数以百计与之配套的烽燧和堡垒,将代州这个探入辽国西京道的突出部,从西、北、东三个方向,牢牢的守护了起来。

但河东路的地理位置却是最为关键,向西压制西夏,向东可援助河北,同时向北还能牵制辽军,当郭逵、王韶等长于军事的重臣不在朝中的情况下,韩冈可以说是是朝廷眼下能拿得出来的最佳人选。

吕惠卿的目光在韩冈身上打着转。

之前吕惠卿受命出面与韩冈协商——要不然韩冈拒了诏命,学着他岳父的样儿,事情就让人哭笑不得了,必须要事先沟通——本以为要费上一番口舌,孰料他竟然很痛快的接下了去河东的差使。

以韩冈的脾性,从来都是宁折不弯。即便这一次缘国事不得不相从,事后竟然连一点反击的动作都没有,除非这正合韩冈的本意,否则决然说不通。

吕惠卿不意韩冈如此好说话,但沉下心来仔细想想,倒是找到了一大堆韩冈要去河东的理由,就是不便当面详询究竟,确定自己猜测的对还是错。

韩冈再拜起身,时隔半年之后,将再一次离开京城,接下了前往太原、担任一路帅臣的诏命。

太原府是次府,在编制上,高于州、军、监,仅次开封、河南、大名、归德等大都督府。而河东路在二十多经略安抚使路中,序列也十分靠前。就是宰相、执政出外,坐上这个位置,也不能算是薄待。

不过出外就是出外,离开天下的政治、经济、文化的核心,无论如何都不是任何一位重臣心甘情愿的选择。因为回返之时很可能是遥遥无期。

韩冈离两府只差一步,但年龄和资历的问题始终跨不过那道坎。他出外任官,到没那么多不情愿,但在宰辅们眼中,那就是一个碍眼的家伙终于离开了。

只有王珪对韩冈的离开满腹怨言,不是他喜欢韩冈,而是吕惠卿将无人可制。

依照惯例,一州知州就任,都要朝会上走过一道陛辞的程序。而一路帅臣,更是要在天子面前经过问对,确认能够适任之后才能上任,过去也有问对让天子过于满意,而留在朝中就任要职的例子。

但韩冈就没那么多麻烦了。

他的能力不需要质疑,让他去太原,是为了解决当务之急。赵顼在崇政殿议事之后,将他单独留对只是为了听一下他到了河东之后,将怎样处理辽国和西夏的问题……。

“在解决西夏之前,中国无力分心与契丹为敌。”

韩冈开门见山的评论,让赵顼顿时就挂下脸来,但转而就是苦笑。要是韩冈说的不对,就没必要让他去太原了。

“韩卿之言甚是。”赵顼叹息点头。

郭逵正在河北整训士卒,最后能有多少成绩,也是难说得很。

智者有百年远见,愚人只能看到眼前。郭逵还算不上智者,却也决不是愚人,他至少是个聪明人,做事前会先为自己搭好台阶。

郭逵到河北后,没两天就上了一本奏章,批评当地禁军、厢军、保甲训练不足,不堪校阅,空有兵甲而已。而到了灵州兵败的消息向各路秘密传达之后,昨天郭逵递上来的奏本,调门一下又提高了许多,声称如果不能加强训练,河北缓急间将无兵可用——没有一支能派得上用场!

这份奏报让赵顼陷入了慌乱之中,就是宰执们也都是神色忧愁,没人想起出言安慰天子。

如果郭逵所言为实,那么河北军的情况的确堪忧。如果郭逵所言夸张成分居多,却也同样证明他对抵御辽人缺乏足够的信心,否则何须为自己找退路。

郭逵的奏章,也让韩冈的发言多了几分底气:“中国有足够的能力同时打上三场局部战争,臣几年前参与南征之役的时候,横山和西南都有战事,最后是轻松取得了胜利。但同时展开两场全面战争,以大宋之力还是差了一点”

局部和全面,赵旭觉得韩冈的用词很有点新鲜,但细想一下,却很恰当。

顾名思义,所谓局部战争,就是之需要动用一路两路的兵力、财税,最多再动用一部分精锐就能解决的战争,即便失败,与国家的损失也不会太大。而全面战争,最少也要动用数路人马,以朝廷数载财税为本金,才能打得起的战争。

在官军和交趾打得如火如荼的时候,朝廷对横山和西南夷又同时出兵,当时朝中虽然紧张,却也没有如临大敌、战战兢兢的紧张情绪。但如今在平夏之役战局不顺的情况下,辽国的动作,让赵顼还有多少朝臣、百姓夜不能寐。

“如果辽人犯境,韩卿是打算……”赵顼想了想,觉得姑息这两个字不太合适,选了一个褒义词,“卧薪尝胆?”

韩冈摇头:“边境之安不是求来的,而是争来的。若真宗皇帝没有亲征澶州,而是巡幸蜀中、金陵,岂有澶渊之盟?”

“澶渊之盟不过是城下之盟。”赵顼低喃着。

当今的大宋天子念兹在兹的便是洗雪旧辱。让他堂堂天下之主,与偏鄙蛮夷做亲戚,这样的澶渊之盟绝对是耻辱的一部分。华夏之君,纵不能做天可汗,也不当做鞑虏国母的侄儿、侄孙。

见赵顼听到澶渊之盟就有几分不自在,韩冈毫不客气,“至少要强于巡幸南方。七十年澶渊之盟,朝廷复出的银绢不足三千万匹两,换算成钱,也不过六千万贯而已。……现在的这场平夏之役,已经花掉的费用早已超过千万贯,如果继续打下去,直到西夏支撑不住,再加上战后的封赏,以及对亡族的抚恤,至少还需要两倍于此的付出。”

“如果是能够确定胜利,这样大的投入没有任何问题,但兵事总是伴随风险,一旦输了,就是血本无归。”

韩冈这般说,赵顼沉默着。

“灭国一劳永逸。做不到,那就退一步,坚守边地,让贼寇劳而无功。若还做不到,那就用银绢来买平安,至少要比贼军入寇,国中城乡毁坏,损耗国力要强。虚名岂如实利?”

换作是过去,韩冈会对澶渊之盟看不上眼,但现在更进一步的认清现实了。给钱没什么,只要不变成付账付习惯了就可以了。

若能花钱买来辽国对西夏的不闻不问,岁币再增加一倍都无所谓,反正一旦灭了西夏,几年后辽国就会成为下一个目标,百万贯的岁币,找个借口就能赖掉。

可惜耶律乙辛不会那么蠢,钓饵会吃掉,鱼钩则会直截了当的打回来。

“岁币是缓兵之策,用钱买来十年生聚十年教训的时间,以图将来。只是澶渊之盟订立之后,国中就变得习于安逸,诚可惜哉。若是能厉兵秣马,纵不能观兵临潢府,也不至于会有元昊之叛。”

“事已至此,无可奈何。”赵顼沉重的叹息声不像是一个拥有万邦的君王

接下来的时间韩冈在崇政殿中,将自己抵达的河东后,将如何抵御辽人的想法,向赵顼做了个简短地回报。

这恐怕是赵顼唯一担心的,就是韩冈为人太过刚硬,刺激得辽人放弃一切,主动南下。但韩冈之前说的一番话,倒是让赵顼放下了一点心。至少不会比郭逵差了。

接过了太原知府的差遣,韩冈又征辟了三名门人充作为椽属,黄裳也是其中之一,加上十几名幕僚门客,出镇河东的团队算是组建完成了。

与此同时,数千里之外的夏州城中,一番争论正如火如荼。

刚刚从赵顼手上得到一封密诏的徐禧强硬无比:“盐州决不可弃!”

“盐州守不住的。”种谔的声音中有着浓浓的疲惫。

“种太尉。”徐禧并不忌讳让人听出话声中的恶意,“你守不住并不意味他人守不住。而且你到底是守不住还是不想守?”

种谔面沉如水。李宪早就跑了,直接跑去守弥陀洞。也就他最倒霉,只能留下来镇守夏州,日日听徐禧的骚扰。

“五叔。”等种谔大步从主帐中走出来,守在门口的种建中就冲着种谔问道,“徐德占还是要守盐州?”

“当然。”种谔眼下并不想多谈这个问题,大步往自己的洞中去。

“徐禧怎么调动驻守延州的兵力?鄜延路的兵将,没人会听他的。”

“他要是没有在军中找到足够的助力力,他也不会选择这个时间发难。”

“……该不会是京营吧?”

“除了那几位还会有谁?”

“不能安排些事给他们去做?”

“拦着他们立功?”种谔摇摇头,“这可是不共戴天之仇!”

种建中跳了起来,“我要写信给韩玉昆。”

“别忘了,”种谔提醒着,“吕惠卿与徐禧有姻亲!”

