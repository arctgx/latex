\section{第28章 官近青云与天通(15)}

开席之后,韩冈拿着酒壶给王安石等人倒酒劝酒,标标准准的晚辈的姿态。王安石视同寻常,吕公著和司马光也坐得稳稳的,韩维在韩冈给自己斟酒时,微微皱着眉,但最终还是什么话也没说,待斟满酒后,自自然然的举起杯子,与王安石对饮。

但陪席的王旁和司马康就很不自在了,韩冈在给王安石他们倒过酒后,也不忘将他们也一并照顾到,但王旁和司马康就不能大剌剌的坐着了,总要站起来。幸而韩冈身上的衣袍,已经借了身材相近的王安石的旧衣,倒是不显得那么扎眼了。

就在司马光与一干旧友和晚辈相会,‘把酒言欢’的时候,城南驿已经给他安排好了住处。司马光的几个随行伴当,拿了行李,全都安顿了下来。

在城南驿驿丞周至的安排下,司马光和王安石两家下榻的院落还是尽可能的离得远了一点,但专供重臣的几间上院几乎都在一处,说起来也只隔了三重院落而已。

很简单的一席酒宴过后,自不会有秉烛夜谈的闲心,司马光和王安石、韩冈翁婿一并送了吕公著和韩维两人离开,又在后园中分手辞别,各自回各自的住处。

到了下榻的小院中,疲惫不堪的司马光先进了房休息。司马康则先是去吩咐下人,留下值夜的人手后就可去安歇。回过来,又亲自端了一杯消食的热茶进了正屋。

司马光坐在灯下,正沉默着,眼神漫无目标的落在墙上的一幅俗气无比的富贵牡丹上。接过了儿子端上来的茶,不知多久之后,他忽而一声叹:“王介甫老了。”

“嗯,的确是老了。”司马康陪着话,附和道。

十几年前,王安石初至京师的时候,还与司马家常来常往,司马康见了他不知多少次,不过几个月后两家就翻脸了。与当日相比,如今的王安石当然是老了。

不过司马康知道他的父亲不是在说王安石的形容相貌,而是王安石的心态老了。已经没有了当年为新法,与诸多老友大战三百回合的锐气,言谈间只论旧日交往。今天的王安石只戴着一软脚幞头,穿了一身士人襕衫,而且是洗旧了的青色布袍,乍看起来就是乡里常见的一循循老儒的模样。不见锋锐,多了几分和蔼可亲,只有一张黑脸如故。

“听闻是官家在病榻上亲自任了王介甫为平章军国重事,而不是宰相。”司马康说道:“大概已经是看得出来他无心于朝堂了。”

其实司马康的意思应该是反过来,王安石因为做了平章军国重事而心灰意冷,只是总不能批评天子,而且他相信父亲应该能听明白。

不过司马光没接口,过了半晌,才又开口:“吕晦叔不服老。”

司马康点点头,“吕三丈护卫正道,壮心犹在。”

他今天也看出来了,现任的枢密使,与自家的父亲和王安石同为东宫三师的太子太保吕公著,现在依然是斗志犹存,犹有翻天覆地的打算。要不然也不会一听到消息,就急匆匆的和韩维一同登门造访,这当然是为了和自家父亲联手,以壮声势。

依照正常的情况,官员回京一般都会先遣亲信提前一步来通知,也好让亲友做好准备,甚至出城相迎。而自家父亲为了避免麻烦,不但兼程而行,也根本没有通知任何人,直接就进了城,直到去了宣德门消息才传开——撞上王安石只是意外——这一做法,其实已经将心意表现得很明显了,但吕公著依然迫不及待的来了。要说他没有其他意图,又有几人会相信?

至于韩维,司马康的眼睛不瞎,明显是被吕公著拉过来的。在几人中,今天他的话是最少的一个。就是执壶侍宴、尽量不做干扰的韩冈,都比他多说了两句。

韩维在许州【今许昌】,是出了名的悠闲。司马康在洛阳都听说了,甚至比富弼当年判大名府时还自在。

春暖花开的日子,只要天气晴好,他就出许州城,泛舟西湖之上。或在湖畔的展江亭中,邀请一二过路的官员,更多的还是士子,不问相识与否,只要看得顺眼,满九人便开席。吟诗作对,观赏歌舞,直至夕阳西下。在洛阳的程颢、程颐前两年都被邀请去许州过。

至于衙中公务,自然就是交托给属吏处置,谁也不敢让贵为资政殿学士的判许州劳累到身子骨。

且如今是皇后垂帘,而不是对新党成见极深的高太后,显然现在韩维跟吕公著是两个想法,跟自家父亲更是不是同路人了。

司马康想着,他看着司马光,不知父亲怎么评价这旧日老友中的最后一位。

但司马光直接跳过了韩维,“难怪程正叔这么喜欢韩冈。”

司马康眨了眨眼,愣住了。

司马光话说得直白,他也听得明白,但他却想不明白。

程颢倒也罢了,性格宽和,口不臧否人物。而程颐待人则严厉得多,一向不苟言笑,对人更是少有奖誉。但对于韩冈,程颐的评价极高。韩冈立雪程门,程颐一直说他在敬字上做得最好,明师道之尊。就算因道统之争而分歧明显,也只是就事论事,从不听闻批评韩冈品行。而且他和程颢对韩冈的欣赏也影响到了门下弟子身上,司马康也听说了,已是同门的吕大临,还不如韩冈得程门弟子推重。


但司马康知道,西京城中的一干元老中,富弼对韩冈的评价最高,‘此子宰相器’是富弼亲口对儿孙说的;‘让他出一头地’,已经致仕的富弼都没好意思对外提。而文彦博在韩冈身上吃的亏最多——旧日在朝中的事不说,几年前韩冈任职京西,司马康是亲眼看着文彦博是怎么被只有他三分之一年纪的韩冈堵得狼狈不堪,颜面落尽,那时的韩冈也是如今天一般谦退——更不会小看其人。可自家父亲说起他人对韩冈的评价,偏偏就提起了仅为一介布衣的程颐。

不过知子莫若父,知父亦莫如子,司马康想了一阵,影影约约的也摸到了父亲的想法。“王介甫和韩玉昆虽为翁婿,但在儒门道统上却是针锋相对。张载在世时,便已争执不下,这两年更是愈演愈烈,连天子都被卷了进来。药典、殷墟是韩冈针对新学而下手,而千里镜的禁令更是天子左袒新学,打压气学的明证。”

司马光却对儿子的话没有什么反应,也不知听入耳了没有。呷了一口已经变得温温的茶水,道:“没有韩冈,垂帘的应是太后。”

司马康闻言立刻紧张了起来,仔细观察着父亲脸上的表情。

这是惋惜,还是单纯在陈述?

司马光的心中是在惋惜,在大致了解了冬至日的那一夜发生的一切后,他才知道,距离自己平生大愿的实现,竟然只差了那么一步。

仅仅是因为一个人,一句话!

只是木已成舟,司马光无意追叹,惋惜却是免不了的。

以今日京城中的局面。王安石越是摆着怀念旧日情谊的作态,就越是不方便翻脸。而作为几乎是同一等级的重臣的韩冈,在三人面前做了半日的晚辈,更是配合得天衣无缝。看似谦退,但实际上却是以退为进。

四名旧友相会,后生晚辈在旁服侍,与公事之争全然无关。就算想翻脸,也得顾及自己的身份和形象。如果说是正邪不两立,还能不假言辞,直接割席断交。但王安石和韩冈的私德和名声,让人并不方便以此借口。给王安石和韩冈这对翁婿一搭一档的拿捏着,今天在席面上完全被压制住了。不过是闲聊和吃饭而已,看似平静无波,但很明显的是王、韩占据了主动。

幸而眼下时局的关键还是在向皇后身上。

向家是外戚,向皇后本人经常接触的又多是宗室的家眷,对新法的感观不会太好——就像曹太皇、高太后,之所以会厌弃新法,那是因为耳边全都是抨击新法扰民的声音,怎么可能还会对新法有好感?但若是自己一至京城便呼朋唤友,摆明了要动摇朝局,那么向皇后那里肯定是要平添恶感。将心比心,在向皇后和她背后还躺在病榻上的天子心目中,稳定当是压倒一切。

幸好城南驿中还有个王安石,要不然刚一到京师,便与吕晦叔、韩秉国相会的消息给传出去,那么立刻就会在向皇后心中留下一个要找麻烦的印象。

吕公著人老成精,不会看不到这一点,他的打算很复杂,司马光明白吕夷简的三子绝不是所谓的纯臣,保守家门不堕当才是吕公著的第一目标。

吕公著另有算计,韩维百事不理,司马光想要将与民争利的恶法给掀翻,却从他们身上看不到希望。

不过与王安石和韩冈今日把酒言欢,还是有个好处,司马光轻声道:“明天当能越次上殿了。”
