\section{第46章 正言意堂堂(下)}

【又迟了,不过下面还有一章】

赵顼在宣德门上,坐立不定,心急的等着韩冈。

虽然以赵顼近十年天子的政治智慧,隐隐的也觉得今年军器监大张旗鼓的将铁船扎成彩灯这件事有哪里不对。可韩冈过去给他带来了那么多惊喜,现在又处在在其本人强行争取来的判军器监的位置上,想来也不会没有让人喜出望外的好消息。《浮力追源》都写出来,怎么想,韩冈都应该是胸有成竹的。

身旁的嫔妃都不怎么搭理,城下的一片灯海更无心观赏。丝竹乐声中,臣子们之间的交流,赵顼也没去注意。只是不断的看着上城的楼梯口,心急难耐的等着韩冈前来。

过了半个多时辰,终于等到了蓝元震脚步匆匆的走上来缴旨,“官家,韩冈已经在城下,等着官家传召。”

赵顼连忙急道,“快宣他上殿觐见。”不经意间,连话都说错了。

韩冈走上了宣德门,瞥了一眼城下,城中繁星百万,果然是难得的美景。走到天子面前,大礼参拜。

“平身!”赵顼急着将韩冈唤起,“韩卿,今夜朕观城下,只见军器监的灯山作了船型。想必铁船一事,卿家多半已有眉目。不知还需要多少时间?”

韩冈一躬身:“回陛下,大约要十五到二十年。”

赵顼闻之一惊,甚至以为自己听错了。再看韩冈,他的态度坦坦荡荡,一点也没有愧疚、畏缩。肯定是听错了,赵顼心想。“多少时间?”他重又问了一遍。

“十五到二十年!”

韩冈的语气一点也没有波动,咬字清楚,让赵顼终于听清了。

‘什么?!’

一片压得低低的喧哗声,从天子亲设的宴席上响了起来。冯京手上的酒杯差点都没拿稳,韩绛、王珪脸上的表情也呆滞了,吴充扶着桌案就要跳起来,十五到二十年,真亏韩冈敢说!

韩冈欺君四个字尚在几个重臣嘴边,天子脸色丕变,就听到韩冈继续说道:“生铁性脆,熟铁性柔,必须得用刚柔和济的精钢来制作龙骨、船肋,正如房梁、庭柱必须得用坚实性刚的大木才行。而如今精钢稀少,必须要改进制钢之法;精钢难以煅炼,想要得到造型合适的龙骨、船肋,又要改进铸造之法;船板、甲板,虽然不需要精钢,可需要大幅的铁板,这就需要新的锻造手段;铁遇湿则锈,船行水上,必须还要有防锈之术,需要找出铁生锈的原理,才能加以应对。而且昔时造船都是木料,要改以铁制,即便是几十年的老船匠,也要从头学起,这亦是难处。细细算来,十几二十年,是一切顺利的结果,需要朝廷不断投入人力物力。中间如有波折,甚至三五十年都有可能。”

韩冈的一段话,平和得如同春来的湖水,不起半点波澜。可这番话却如当头一盆冰水,冷得就像是刚从金水河中舀上来一样,一下就把赵顼满腔的兴奋一下都给冻得萎缩了下来。而熊熊怒火,则开始在心中燃烧。

只是他还抱着一丝希望,而韩冈平静无波的神情中一点也不见愧色,说不定还别有隐情:“韩卿,宣德门外的铁船彩灯,难道是有人背着你做的?”

“回陛下的话。今年监中的铁船彩灯,经过了微臣准许。”韩冈一肩将责任给担着。其中的内情,他全当作不存在,并不准备向赵旭诉苦。韩冈是判军器监,一监之长,被小人作祟的事,说出来也不成体统。

赵顼感觉被臣子戏耍了一回,方才的迫不及待现在看来竟然如此可笑,胸中的怒焰腾腾而起,费了好大力气方才被他强行按捺下来。此时赵顼怒极反笑,声音一下温和了许多:“难道韩卿打算将那艘灯船,十几二十年,甚至三五十年,年年都放在宣德门外?!”

吴充终于拍案而起,随着天子一同厉声质问:“韩冈!难道你要天子为一艘铁船等上十几二十年?!就算黄河改道,只要朝廷肯调集人夫,拨给钱粮,导归正途也就是一年而已。持续十几二十年调拨钱粮,黄河大堤都能跟开宝寺铁塔一般髙了。”

“陛下!”王珪也站了起来,“韩冈欺君,当论其罪以重处!”

“非也。”韩冈摇了摇头,没理会吴充和王珪,对着赵顼道:“陛下误会了。铁船乃是军器中之集大成者。要想打造出能在水上疾驶,矢石不可伤,油火不可焚的铁船来,的确需要持续十几年甚至几十年,不断的在锻造、冶炼上投入人力物力,方才有可能造得出来。但锻造、冶炼上的每一分进步,就有一分用处,正所谓日渐日新,并不是一定等到几十年后才能建功。”

“哦?不知韩冈你所说日渐日新的又是什么?”冯京慢慢的开口,“是否是你拿一百贯悬赏来的用来舂米的锻锤?”

“此亦是其中之一。舂米的脚踏锤改造而来的锻锤,比双手抡锤更为稳定,打造兵甲也更为容易。要知道,许多地方还都是用杵舂米,远远比不上脚踏锤舂米来的迅速。”

“好一个日渐日新……”赵顼一个字一个字的说着,韩冈还真是给了他‘惊喜’。

看着赵顼脸色已经黑得锅底一样,韩冈知道时机差不多了,欲扬先抑的手段用得过头可不好。抢在天子发作之前,他拱手一礼:“好叫陛下得知。就如现在,虽然军器监中仅仅是有了几架合用的锻锤,但用这几具锻锤刚刚打造出来的军器,已经不输于神臂弓了。”

韩冈声音刚落,便是满堂大哗。

赵顼闻之变色,而冯京、王珪等人更是冷笑不已。

不输于神臂弓?!

韩冈到底知不知道他在说什么?!!

神臂弓是真正的军国重器——射程最大能及三百步,七十步外洞穿铁甲,一两百神臂弓手聚集列阵,发矢便密集如雨。从射程、威力到发射速度,都远胜过去的重弩。是禁军面对北虏铁骑时恃之以自保的利器,更是如今朝堂对禁军彻底压倒党项骑兵的信心之所在。

跟神臂弓不相上下?这个牛是吹不得的!

韩冈虽然不是信口开河之辈,但他到了军器监才几日?怎么可能一下就变出什么花样来?

听着他前面说得锻锤,那他能拿出来的多半就是铁甲。但军器监五十一作中,与铁甲有关的有铁甲、钉钗、铁身、纲甲、柔甲、错磨、鳞子、钉头牟等八作,韩冈若是调集这么多作坊同治一事,吕惠卿难道会不知道?

吕惠卿就坐在下首处,看他脸上没来得及藏起的讶异,他是真的不知道!

更别说一领铠甲没有几十天时间造不出来,多少道工序,初上任的韩冈哪有这个时间?

而且铁甲要怎么跟神臂弓比?

若是韩冈不能证明他造出的铁甲胜过神臂弓,那就是明明白白的欺君!而在一众重臣面前欺君,钉死的罪名,就算赵顼都难以帮他挽回。

冯京依然面沉如水,但早已是喜上心头。

即便韩冈拿出来的铁甲,比过去的甲胄要强上一些。但不同的器物,本来就不好相比,只能让人凭着感觉来。只要他一口咬定了说不如神臂弓,韩冈又能怎么样?

满座朱紫,真正会支持韩冈的,也只有王韶一个。

韩绛、王珪、吕惠卿、吴充、蔡挺,有哪一个会站出来坚持到底的支持韩冈?都不会!

这个罪名韩冈担不起,更洗不掉!

王韶安坐在座位上,冷眼看着城楼夜宴上的乱象。若说到对韩冈的了解,将韩冈从布衣举荐入官的王韶当然远在众人之上。

几年来的交往,让王韶很清楚,韩冈就是有个喜欢使用苏张之术的坏毛病,在言辞间设下陷阱,不知陷了多少人进去。但他越是玩弄言语上的技巧,就越代表他胸有成竹。若是没有把握,怎么可能会如此说话?

只是王韶还是有些担心,毕竟睁眼说瞎话的可能不是没有——这不是指韩冈,而是指冯京、吴充等人。若是韩冈拿出来的东西,他们硬说不好,韩冈不是没有聪明自误,反被天子降罪的可能。枢密副使的视线一扫宴上诸人,心头甚至有些发寒。韩绛、冯京、王珪、吕惠卿、吴充,韩冈的敌人未免太多了一点。

臣子们各有心思,而天子也是。赵顼沉默的看着韩冈半晌,左看右看,还是不能确定韩冈到底是有着底气,还是在装佯。最后他放弃了猜测,狐疑的看着韩冈:“韩卿,此话当真?”

韩冈的举止依然沉毅稳重,冯京、吴充这几名宰执的攻击,仿佛如流水过石,一点也没引起他心中的波动,确确实实的宰相气度:“微臣本想过了上元,将其他几事一并奏上。不过今日陛下既然垂问,微臣现在便去取了来,呈于陛下御览。”

“到底是什么?朕使人帮你去兴国坊拿。”赵顼没心情再等待。

韩冈很简洁的吐出两个字:“板甲。”

