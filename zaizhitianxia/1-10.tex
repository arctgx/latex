\section{第六章 气贯文武与世争(下)}

而且据韩冈所知,通过解试后的士子,称为贡生,也可称为举人。但与后世的举人不同,这不是一种终身通用的资历,而是一次性的资格。这次通过解试,去京中考进士不中,那三年后如若想再考进士,还得先参加解试并通过,否则照样没有贡生资格,去不了京中。

除非朝廷能改诗赋取士为经义策问取士,否则韩冈便无望一个进士。尽管如此,韩冈也从没有动过抄袭后世诗词的打算。没有底蕴就别骗人,你可以欺骗一时,却不可能欺骗一世。诗词歌赋是统称,不是抄两句歪诗就够的。

就算靠两首诗词换了点名声,到时有人请去赴宴,去还是不去?此时的宴席都要作诗助兴,一个剽窃者能在酒席上就做出应景的诗句?

这个时代文人的社交活动主要就是参加诗会。韩冈的记忆中就有七八次的经历。诗会上作诗,要分韵限韵,指物为诗。诗还要合情合景,不能海阔天空的乱来。韩冈不认为自己能达到被限定了韵脚,看着风景、器物,就能诌出一首好诗的水平。还有几人联句,押着韵脚,你一句我一句,将一首长诗敷演出来。这样的联句诗,不但韩冈的记忆中有,在红楼梦等古代小说中,也多有提及。

只有一两首上品,其余诗作皆是平平,在诗会上的表现甚至让人难以入目,差距如此反而会惹人疑窦。若本来就是八十多分的水平,一下考个满分,还能说是进步了。但本来只有二三十分的水准,得个一百分,哪个会相信?!

韩冈的前生留下的记忆中有诸多名家文集——虽然细节聊聊,但目录还是有的——其中诗词只占了小部分,除此之外,有表、有章、有传、有记、有论,还有赋、状、书等文体,不是局限于诗词两事。真要冒充个文学大家,各种文体都得涉猎。总不能只会诌两句诗词,赋不会写,表不会写,传记也不会写罢?

你可以找个借口说不再作诗,但日后找你写行状,写墓志铭,写事记的总不会少,外人可以不理,亲朋好友难道还能推吗?这时又该怎么蒙骗过去?事实上,没有点真材实料谁能蒙混上几十年?!

人心险恶,而文人尤甚。江淹仅是文字稍稍退步,就被嘲笑成江郎才尽。如果诗才忽高忽低,只有几首好诗出场,有可能不被人说成剽窃吗?

而且会做诗不代表会做官,历代重臣,有文名的极少极少。李白、杜甫都是一辈子潦倒,何必跑上去添个自己的名字。而且要当官,也不只进士一条路。陕西的进士一向不多,但当官的并不少,并不是非要考进士不可。

除了进士科外,朝廷还设有还有明经科等科目的举试,以选拔人才。韩冈的经义水平不错,明经科的难度又不高,有‘三十老明经,五十少进士’的说法——三十岁考上明经已经算老了,五十岁考上进士却还算年轻。前身留下的底子还在,韩冈自问只要辛苦几年,拿一个明经下来肯定要比进士容易得多。

即便不想参加考试,韩冈还有受人举荐而得官一途,这也是他信心的来源。西北战事频频,对人才的渴求远高于其他的地区。韩冈如今习练箭术,也是为了博个功名。只要比武夫有文才,比文人有武力,再凭借自己的头脑口才,混个出身真的不算难。

二十多年前,李元昊举起叛宋大旗,党项骑兵在西北纵横无忌。当时的北宋,已经三十余年不闻金鼓,朝中无人可用。范仲淹、韩琦等名臣,陆续从朝中来到西北,将陕西局势安定下来。这期间,多少关西英才都借势得荐,入朝为官。又有多少军中小卒趁势而起,一跃登天。

韩冈的老师张载,本也可能是其中的一分子。张载当时曾上书范仲淹,打算收复青唐吐蕃,作为攻打党项人的偏师。后来因范仲淹的劝告,张载才弃武从文去考了进士,并开始授徒讲学。可他自始至终都没忘了教授弟子兵法战策的学问,在如今大宋的各个儒家学派中,张载的关中学派【简称关学】是最为重视兵法的一脉。

张载三年前在京兆府的郡学中讲学,两年前为签书渭州军事判官,辅佐环庆路经略安抚使蔡挺处置军事,闲暇时也为诸徒授业,去岁又应邀在武功县绿野亭聚徒讲学。也许在中原横渠先生名气尚不算大,但在关西他却是德高望重,关西士子对其闻风景从。

韩冈忽然自嘲而笑,说来说去,还是要靠自己的老师。曾拜张载为师,的确是自家的运气。不论哪个时代,出身名师,又有同窗守望相助,博取名望自当比其他的人要容易许多。张载这位老师是他此时最大的依仗,理所当然的韩冈必须去更深入的了解张载的理论。也就是基于这个理由,最近这段时间韩冈有很大一部分精力,放在整理温习当初在张载身边听讲时留下的笔记上。

‘虚空即气。’‘气之为物,散入无形,适得吾体。聚为有象,不失吾常’‘太虚不能无气,气不能不聚为万物,万物不能不散为太虚’

这张载对天地自然的看法,世界以气为核心,天地万物皆由气而生。把‘气’替换成物质,‘太虚’替换成宇宙,可以看出张载的理论根源是唯物的,

‘气块然太虚,升降飞扬,未尝止息。’

此是‘运动绝对性’的另一种表达方法。

‘聚亦吾体,散亦吾体,知死而不亡者,可与言性矣。’

好罢,这一句根本就是物质不灭论——死也罢,活也罢,肉体不会随着死去而消失——所以叫做‘死而不亡’。

除了这些之外,韩冈还从笔记上一些张载所说的残章断句中看到了量变转向质变的理论,虽然张载将之称为‘渐化’和‘著变’。还有与对立统一有关的辩证法的雏形——‘一物两体……此天之所参。’

虽然张载的言论可谓是诘屈聱牙,不似后世说得那般简单明晰,可韩冈并不会因此而轻忽视之。因为张载的气学理论,跟韩冈所秉持的哲学理论有许多共通之处。只要换个说法,甚至可以把原子论、元素论、辩证法等后世的自然科学理论改头换面的融合进去。而且这些属于自然哲学范畴的理论,是经过千百年无数人的验证,其严谨性远高于气学理论,又能通过实验加以验证——也即是符合儒家格物致知的教导。

将后世的自然科学理论打包成气学,是个很有趣的想法,韩冈觉得其中很有成功的可能。一旦成功,不但张载留名青史的不将仅仅是简单的四句豪言,他的气学理论同样将会流传后世。而韩冈梦寐已久的权力和地位也将会随之而来。

韩冈这几天闲暇之余便是设定计划表,给自己划定了时限,打算花上半年时间,将这一包容在气学中的新理论编写出来。对于创造一个新理论来说,这个时间不算长,可以说是很短,但对韩冈已经足够。因为他的打算并不是创造一门学术取代气学,而是用自己已经明了的理论去弥补气学的不足。同时还要留着进步的空间,以供日后逐渐改进。

超前时代半步是天才,超前一步,那就是疯子。韩冈没有挑战整个社会的狂妄,他不是唐吉珂德。他的目标是能保护自己和家人的权位,仅此而已,并不贪心。唯有这一点,他不会为任何事所动摇。

一个能自圆其说的系统,要按步骤慢慢来,不可能一蹴而就。同时,这也是给自己逐步提升名望的机会。同时逐渐提升的名望,便能给自己带来自己想要的权位。权位的提升又能反过来推动学说的推广。学术和权位,两者是互相促进。没有权势的辅助,一门学说想要散布开去,都是要几十年上百年的功夫。

韩冈对历史不甚了解,但也知道理学在历史上的地位。作为理学始祖的程颢、程颐,却正是自己老师的表侄——去年自家还见过程颐一面,那是个用严肃死板包装起来的让人生厌的中年人,挑剔苛刻的目光,让每一个张载的学生都战战兢兢,唯恐哪处失礼丢了老师的颜面——可就算到了南宋的朱熹那里,理学也没能一家独大,甚至还因政治原因被禁止过。

只恨自己当年在火车上闲来无事翻看朱熹的传记,并没有深入的去了解其中的细节,见到关于理学的章节便跳过去,反而对朱熹收尼姑、扒儿媳的八卦关注甚多。这就叫有钱难买早知道,韩冈现在可谓是悔不当初。

静下心来,韩冈埋首伏案,细心钻研。等到他稍有成果,书信往来也好,直接去见面也好,新的理论只要能引起张载的兴趣。自己在关中士林的名望,也便奠定了第一步。

PS:张载被朱熹尊为理学五子之一,与他的表侄程颢、程颐,以及二程之师周敦颐,好友邵雍并称。但张载创立的气学体系偏近于唯物主义,而与比较唯心的理学完全背离。这就是北宋各家学派的道统之争,不但将敌对的学派斩草除根,还要移花接木,将之夺取过来。

在北宋,学术之争与战争并无二致,你死我活。

今天第一更,求红票、收藏。

