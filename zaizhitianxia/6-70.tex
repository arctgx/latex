\section{第八章 朔吹号寒欲争锋(十)}

太后的询问充满了偏见,甚至恶意。

这时候没人会认为太后现在的问话,是源自于她的体贴。

看起来新党的存在,已经让太后感到不耐烦了。

这样的苗头却没有让章敦一丝一毫的畏惧,一口咬定:“蒲宗孟正是知贡举的人选!”

决定知贡举的人选,论理并非是枢密使应该涉足的领域。不过但凡由天子决定的位置,无论东西两府都有建言的权力。

“不过蒲宗孟一人知贡举恐有疏失,臣再举河北都转运使李承之,与蒲宗孟并权知贡举。”章敦继续道。

不是权同知贡举,而是并权知贡举,也就是地位相当,不分高下。否则以李承之的身份,不可能屈居蒲宗孟之下,去就任地位整整低上一大级的权同知贡举。

不过李承之的情况,在场的都清楚;两天前他将手中的选票投给了谁,也没人会忘记。

王安石顿时成了众人注目的焦点,对章敦的提议没有任何表示反对的意思,站在那里,丝毫不见有说话的意思。

李承之在文学上没有足够的名声,而且又是投靠了韩冈的叛徒。正常的情况下,王安石无论如何都不可能同意让一名并非新党的成员拿到知贡举这样关键性的位置。但相对于人数多达数百的进士,两府的一个位置更为关键。而且李承之还有蒲宗孟牵制,而下面的考官更不可能找到新党以外的色彩,知道该如何选择。

只要能再干掉韩党的一票,枢密副使这个位置,就脱不了新党之手。

这是太过明显的兑子,尤其还是在太后否决了张璪的提议之后,向太后不可能看不明白。

只是章敦有恃无恐,韩冈手上没有人,但就算他苦于无人可用,也绝不敢将旧党拉回来。

那些老家伙,别看现在一个个委曲求全的模样,将韩冈当成了救命稻草来重视,等他们重新得志,能把他和他的气学,连皮带骨头一起都吃掉。

韩冈是借助新党当权的形势,才会让旧党来投。一旦没有了新党,他根本压制不住那群老家伙。章敦确信,为了避免鸠占鹊巢,在许多安排上,韩冈必须配合,乃至忍让。

“韩……相公,”太后的声音打了个磕绊,“王平章,还有诸位卿家,可有意见?”

“此议甚佳。”王安石当先表示同意。

韩绛没有立刻开口,停了一下,而后问道:“两人并知贡举,此事可有先例?”

“近年来绝无。”张璪摇头。

不过这是助攻,章敦随即便说道:“臣曾记得太祖太宗时,曾多有诏令,以多人知贡举、权知贡举。”

这是当然的。

当年制度未定,连状元都可以是武英殿上靠相扑夺来,诏书上没有分清知贡举、同知贡举的区别,没有写明初考官、覆考官、编排官之类的各项负责人,只是笼统的提一下某官等几人知贡举,这样的情况是有的。但究竟是谁为首,只要看哪一位在诏书上排名最高就可以了。

不过韩冈倒是初次听章敦亲口说,要以太祖太宗时旧例为法。这变法来变法去,说是要上追三代,却又倒回去了。

韩冈盯着章敦,“太祖、太宗之时,国家初定,制度多有阙陋,安可以之为法?殿试,太祖设之。考官即受命,便赴贡院锁院,太宗时新制。编排官、弥封官,真宗时方设。不知枢密今日以太祖太宗时故事为法,荐举二人并知贡举,礼部试中诸多制度,是否也是恢复到太祖太宗时?”

只听韩冈和章敦的对话,都无法确定哪个才是新党。

一种怪异感从王安石的胸口中腾起。这就是党争。

尽管他一向否认有党,但章敦和韩冈现在的表现,却分明昭示了党争的存在。

党争之中,并不讲究什么道理、原则,是非对错全都丢到一边,一切都只看胜负。王安石当年与旧党相对抗,因为旧党众人恶毒的攻击,许多原本都看不顺眼的人和事,他也不得不坚持下来。

“参政的意思是……?”

韩冈摇头:“先例是先例,可以依循则依循,不能依循则另创新制,以顺应时势,所以先帝当年变易祖宗之法。章敦推荐李承之与蒲宗孟并知贡举,臣无异议。但李承之现为河北都转运使,其知贡举,河北漕司却需人主持。”

李承之在政见上与韩冈相似,本人也是才具卓异,韩冈希望他能够留在朝中帮助自己。本来韩冈就准备为其谋取朝中适合他的位置,现在经过知贡举中转一下,就更加容易了一点。

抢在所有人发言之前,韩冈又道:“宝文阁待制、右司郎中李常本是进京待选,却因病滞留京师。近日终于痊可,已能上殿。其人才干久已闻名朝中,河北漕司若由其主持,当可无忧。”

韩冈话音悠悠而落,可一时间无人能有所反应。

他将握有一票在手的李常推荐到河北都转运使的位置上,加上李承之就任知贡举,一下损失了两票,这等于是向天下昭告,放弃了下一次的推举。

向太后一时间也惊讶得说不出话来。

如果换一个形势下,韩冈这么做,就可以说是引用私人、培植党羽。但现如今,一干重臣都没人愿意离开京师,韩冈此举可谓是公忠体国的表现了。

之前章敦等人脸皮都不要了,就是为了要削减支持韩冈的票数——这一点,她如何看不出来。可现在韩冈却很干脆的将自己的支持者安排出京,一点也不为枢密副使这个位置,为王、章党徒侵占而感到担忧,也避免朝堂因争执而陷入动荡。

十余年前,刚刚开始变法的时候,新旧两党党争激烈,尽管丈夫始终坚持着推行新法,可回到寝宫后,每天每夜都长吁短叹,为朝臣不能体谅国势艰难而夙夜叹息。这些旧事,当年向太后便记忆深刻,现如今在脑海中仍历久弥新。

当确认知贡举不能拥有推举宰辅的权力,章敦便立刻设法将李承之推入贡院之中。而韩冈不但答应下来,还更加干脆的将李常都打发了出去。

在太后的眼中,这就是为臣之道上的差距。

既然韩冈为朝廷着想而举荐李常,她没有理由不支持。

“既然有参政推荐,想必李常定能胜任河北转运一职。”

太后点头,那么只要李承之和李常同意,十余日后的廷推就不会再有波折。

章敦对于韩冈的决定,并不感到惊讶。

若是将旧党中人放入朝中,做出有悖于方今国是的举动,韩冈也不免受其牵累,归根到底,在所谓的韩党变成气党之前,韩冈身边的人,都是各具异心,与他并非同心同德的同志,重用不得。

只是韩冈能如此拉下脸来过河拆桥,倒是让人有些吃惊。

近午时分,开宝寺附近,急促的马蹄声一路传来,穿过开宝寺正门,在贡院之前猝然消失。

百来名班直护卫,前后护送着一群官员下马走进了贡院。

待最后一人没进门中,贡院的大门立刻被合上,门后随即一声响,门闩被放下了,而门前的两支铜环也扣上了一只巨型的铜锁,被牢牢锁紧。来自宫中的禁卫,以及开封府派来的士卒,又团团围定了贡院的门户。

这一刻,来自天下各路、参加礼部试的五千余名士子,全都明白了,今科考官的名单拖到今天,总算是出台了。

“玉昆,十几天后的廷推,当真什么都不想了?”

“选谁上来做枢密副使?真的没人能选上啊。”

韩冈轻摇着头。

他与苏颂正在回公廨的路上,与其他宰执相隔甚远,可以放心谈论。

“但也没必要将李常也推出京城去,留在京师也可以吧?”

“不行的。之前殿上廷推时,韩冈多蒙范、李、孙三位推举。但与其说他们是支持韩冈,不如说是反对家岳。若留其在朝中,必定会干扰国是。于国何益?于民何益?”

给韩冈投票的三名旧党成员,范纯仁是加急入京,李常称病,硬拖着不走,只有孙觉是按时回京。虽然是时间上有些参差,但基本上可以说是同时。

三位旧党全是在外就任州郡,一方面,能看得出新党完全控制了朝堂,另一方面,便是旧党还有很强的底蕴,否则不会随便回来几位侍制,里面就有三位是旧党。

不过外任的官员,在京城中不可能逗留太久。正常来说,半个月之内,包括在河北担任都转运使的李承之,四人全都得离开京师。但要是他们各自告病,辞不就任,完全可以拖到推举之后。

知广州的陈绎,已经上表称病。之前李常就是听说了要廷推,所以才称病不肯接受朝廷的安排,所以之后朝廷就算要敲打谁,棒子也不会先打到他头上。

如果几位外任的支持者都留在京师,以韩冈升任参知政事后所握有的权力,再从近处招一名旧党重臣进京就能拿到足够的票数,推举一名同党进入枢密院。

不过韩冈根本就没有打算去为这三位求取官职的打算,最多是将他们安排得离京城更近一点,职位更高一点。

并不是韩冈忘恩负义,而是现在需要确认的是究竟谁求谁?
