\section{第17章 往来城府志不移(五)}

游酢并不是打算指责韩冈的人品,只是想说他的才智和城府。但看到两位同门都误会了,也不方便辩解。

“有韩冈主持,纵然张横渠仙去,但气学也是日渐昌盛,他回京之后,就算有公事耽搁,也必然能有所开创。”谢良佐岔开了话题,叹了一声:“对手日增,时不我待啊。”

杨时没有半点担心:“气学其实自顾不暇。天人之论,犹如鸿沟一般,韩玉昆跨不过、补不上。其实就是上元节宣德门外的灯山,看着光鲜炫目,实则就是竹皮薄纸糊起来的,一戳就破,一烧就着。要不是因为这一点,吕与叔如何会转投而来?

在杨时看来,别看现在气学给其他学派带来了巨大的压力,不过就未来的发展来说,气学的敌人就是其本身。如果没有一个完整自洽的体系,任何一门学派都是很难传承和发扬的——尤其是在竞争者如此之多的情况下。

气学最大的问题就是自然和天人之论割裂极为严重。承认天子受命于天,这是气学圭臬《西铭》中阐述的观点,但这一点是决然不可能从张载的气之一元说中得到证明,而韩冈主张的自然之道更是让这个裂痕变得更深更大了。

“韩冈对此避而不论,可躲能躲到什么时候?这是一个大关节,避不得、让不得。要么就是天子不再受命于天,要么韩冈就得承认他的自然之道有错。”

游酢却觉得事情没有那么简单,以韩冈的心术才智,不可能坐视这样巨大的破绽不去弥补。何况张载诸多门人,也不可能就这样放着不管。

程门自号道学,眼下的第一大敌是控制了士子们晋身之阶的新学,但远期则必然是气学。韩冈用心长远,日后等他身登相位,自然会想方设法让气学成为国子监中教授学生的课本,让其成为天下的显学。

就如手上这只千里镜。韩冈一直以来对天文星象只有只言片语,最多也仅仅是提及过日月星辰乃是由气而生的宣夜说。但千里镜的出现,让人们可以细观天穹,对日月星辰能够有着更加深入的了解。

组成显微镜和千里镜的两种透镜都是他所创,而且还阐明了原理。明其理,故而才有了显微镜和千里镜。

系辞曰形而上者谓之道,形而下者谓之器。

依气学之说,透镜折射光线的原理就是形而上的道,是从世间实物中归纳出来的道理,而千里镜、显微镜,就是这个道理重新反馈到世间的结果,是形而下的器。

道和器是一体的,若只求形而上,那就是无根之木无源之水,空谈而已。而只注重形而下的器,不注重归纳其中的道理,那就只是个庸夫而已。

气学,或者说韩冈,一直都在主张经世济用、明体达用、学以致用,不同的词汇有着相近的含义。任何道理和学问都必须能用到实际上。秉承的是安定先生胡瑗的理念,在横渠书院,诸多弟子都要兼习经义和治事,水利、兵法、钱粮、刑名,在钻研经义之外,都要在其中选出两项来学习。

对系辞这一句话的诠释,便是气学的一个大关窍。

但程门之中,对这一释义完全无法认同。杨时道:“正如吕与叔所说,韩冈终究还是所学不正,一应建树都是旁枝末节,须知道理性命才是根本。”

“但越是浅近,越是能引人就学。显微镜和千里镜,在洛阳城的官宦子弟中都蔚然成风。”谢良佐叹道,“下里巴人,和者数千,阳春白雪,和者数十,等到‘引商刻羽,杂以流征’,那就只有三数人能和得上了。”

“仰之弥高,钻之弥坚,瞻之在前,忽焉在后。圣人之学,颜子【颜回】亦觉艰难。浅近易学的那是少正卯。”

说归这么说,但其实程门中的每一个人都能从韩冈身上感受到了巨大的压力。韩冈的声望,来自于一桩桩功绩的累积,他的威信,来自于一名名百姓受到的恩惠。名望越重,说话的份量也就越重,他所主张的理念,愿意去学习的人也就越多。

韩冈编写的蒙书,在关中的蒙学中已经开始推广。教人识字、明义的有三字经,数算的有算术,讲述天地万物的有自然,从头到脚全都是气学的影子。等到这些小学生们长大成人,还会有多少人能接受其他学派的观点?

新学靠着王安石的权威,成了朝廷主张的显学。就算其他各家学派,想要去考进士,都必须学习三经新义。但新学如今的地位,靠得还是新党的地位,当朝政不再由新党来掌控,新学当然也就被断根了。

而气学,上有韩冈护持,下有关中蒙学不断培养出识字,加上横渠书院中出来的士子,由于有治事之材,只要运气不差,入官之后,肯定要比只通经义和诗赋的官员更受重用。

如果要与气学一较高下,就必须尽快了。否则等气学声势大起,就会变得跟如今的新学一般,压制所有的学派。而且以气学如今深植根基的做法,一旦盘踞下来,便再难动摇。

“不用担心。”谢良佐走到游酢身边,“且不说气学如此声势,必惹得新党视其为眼中钉。就是只凭我程门一脉,日后约期辩经,也定然能拿回一场大捷来。”

……………………

江宁府的夏天一直都是以炎热著称,不过城外钟山边上,有着徐徐山风,倒也不是那么难耐。

王安石坐在道边的一方青石上,面前一副棋盘,对坐一名道士,两头干瘦的老驴在旁边啃着青草,一株老槐荫荫如盖,为他和弈棋的对手遮挡着火辣辣的阳光。

山风徐来,卷走了炎炎暑气。王安石一身道袍,对面的又是一个老道,两人都是木簪芒鞋,身上看不到任何饰品,看起来就是两个普通的道人——应该说是穷道士——在路边下棋。

山林下的道路时有行人往来,从他们的身边经过,最多也就瞥上一眼两眼,都没人注意到坐在道边石头上的,有一人是曾经执掌天下政务、权势赫赫的名相。

“前些天怎么不见相公出来?可是贵体有恙?”李叔时在棋盘上落了一子,随口问道。

王安石专注着棋盘上的黑子白子,漫不经心的回道:“病倒没有,困于文牍而已。”

李叔时抬起头:“是相公这几年在写的那本书?”王安石这几年一直在琢磨着训诂字义,这一点李叔时与其下棋聊天时多多少少也听了一些。

“已定名做《字说》。”王安石点了点头,随手落了一子。

其实《字说》这个书名王安石很早就确定下来了,脱胎于《说文解字》,在跟亲友交流的时候,因为尚未成书,却是没有公开的将书名附上。依照书名来看虽说是解字,但内容却多为训诂,又兼论音韵,儒门小学中的文字、音韵、训诂三个门类却占全了。不过小学本是一体,皆是经学之本,提到其中一个,就少不了带出其他两个。

早在英宗仍在位时,王安石就开始撰写本书,到了一年前才有了初稿。他将初稿分抄了寄给几个功底深厚的亲友,让他们品鉴指正。他人的回信皆说好,可就是二女婿最不客气,直接就说是刻舟求剑。可也多亏了韩冈那个好女婿,让王安石对《字说》几处不合人意的地方也做了些修改。这一回《字说》一出,新学的根基也就稳下来了。

李叔时闻言拱了拱手,“哦!那可真是可喜可贺!相公才学冠绝当世。《字说》一出,先儒传注当让出一头地了。”

“岂是欲与先贤争列?不过是为了正本清源罢了。”王安石道,“先王患天下后世失其法,故三岁一同。同者,所以一道德也。”

李叔时能与王安石做棋友,见识自不差。听到隐含杀机的‘一道德’三个字,眼前便是一片金戈铁马,耳畔也仿佛有鼓角齐鸣。这部书果真是为了压制一干儒门别传。

王安石和李叔时边聊边下棋,太阳在天空中一点点的移了位,渐渐的落在了王安石的身上。

见王安石大半个身子都笼罩在依然炽烈的阳光下,而他带在身边才十岁出头的小伴当又蹲在地上看蚂蚁,李叔时咳嗽了一声,提议道:“相公,不如换个地方吧。”

王安石安坐于青石之上,不动如山,毫不在意,“由他去,来生转世做牛,须得日头里耕田。”见李叔时有些迟疑,催促道,“快下啊,别耽搁,老夫这盘可是要赢了。”

竹林沙沙作响,一阵清风从林中,吹散了身周的热浪,苏昞听着林中传出的自然音韵,心中一片平安喜乐。

就在书院的一角,来自书院左近镇子上的小学生们正在高声念诵着三字经。童稚之声,让人听了也能会心一笑。

关中一地已经有大半蒙学开始采用三字经和韩冈的算学、自然两部蒙书来教授学生。以十万计的蒙童,就算人才是百里挑一,也是以千来计算——这就是气学的未来。

对于韩冈的计划,苏昞很是钦佩。愿意花时间来培植根基,眼光望着十几年几十年之后,这样的耐心很少出现在年轻人身上。年长者有耐心却缺乏时间。而韩冈,时间、耐性和才学都不缺,日后光大气学一门,必然是他。

与此同时,炎阳高照的暑热中,一队车马抵达了东京城的西门。

戴着遮阳的斗笠,身着别无外饰、适合散热的宽大袍服,韩冈仰头望着高耸的城垣,时隔一年,重又看到了东京城的城墙,但之前的心境并无改变。此处虽是不见蛮夷铁骑,但亦是用武之地。

韩冈家的千金兴奋的从马车中探出头来:“爹爹,爹爹,到京城了?”

“是啊。”韩冈屈指一弹女儿小脑门,“到京城了。”

“爹爹欺负人。”金娘捂着头,眼泪汪汪的嘟着嘴坐回马车里了。

被女儿的娇憨逗得心怀大畅,韩冈回头望着深深的门洞之后那宽敞笔直的大道,轻声道:“我又回来了。” 

