\section{第11章 立雪程门外(下) }

跟着程家管家和程颐,韩冈被请到正厅。

一名六十多近七十的老者已经等在厅中,当然是程家家长。程珦精神矍铄,相貌清癯俊雅,程颢、程颐都遗传了他相貌。穿着一身道服,没有带帽,满头银发都用一根木簪簪上,一尺多长雪白胡须,一看就是个仙风道骨一般的人物。

扶着程珦的,是个八九岁的男童。韩冈记得他是程颢的幼子端本,旧年在京城中见过的。当时的韩冈还是很受程家子弟喜爱,上门拜访时,都被他们围着。时隔三年,端本也长到八岁了,见到韩冈,也还能记得他。

张载的大表兄,程颢的父亲,对于以张载、程颢弟子的名义来访问成家的韩冈来说,自是辈分极尊。韩冈很干脆的跪下来行礼,以晚辈的身份恭恭敬敬磕头问安。

行了礼之后,韩冈与程家通家之好的关系基本上就定了下来。

到了程珦面前,说话就不会再争着天人大道,而是一团和气的聊起天来。程珦对自己的两个表弟也是很挂念,问了不少张载、张戬的近况。

而韩冈也终于知道了为何不见程颢。程颢管着西京竹木监,今天因雪事去了北邙山下的治所,要到三数天后才能赶回。回想起方才程颐回复邵雍的邀请,分明是显而易见的拒绝。

韩冈不能在洛阳久留,最多耽搁两天的时间,程颢也是见不到了。但程珦程颐还在,等下人奉上茶,很随意的聊着天。

韩冈以医道名世,宋儒往往习医,对养生很是看重,程珦便问着韩冈一些有关医术方面的话题。而当韩冈亲口澄清了所谓药王弟子的身份之后,程颐便也投入了谈话之中。

韩冈依然自陈不通医术,但他于疗养院中几年浸淫,见识广博,说起医事也能侃侃而谈。不知不觉的,就说起了高遵裕家小妾难产。

“学生家也有一对儿女,离乡刚刚出生的。当初二侍妾有孕时,学生担心着产难,也是考虑了许久,后来在看到火钳时忽有所感,都是钳物而已。正好高公绰所宠难产,便请稳婆主持,试了一下,倒也建功了。”韩冈笑了一笑,“也算是格物之道,推而广之的运用吧……”

韩冈本以为程颐会因此事事涉妇人,而心有不喜。岂知程颐对此毫不介意,甚至大加赞赏:“药石之事虽是小术,但‘仁’在其中。产钳一物,若能免去天下妇人之产难,善莫大焉。须知学者治学,必先识仁。仁者,浑然与物同体。义、礼、知、信皆为仁。玉昆此举,亦是大仁。”

虽说程颐的性子让人难以亲近,毕竟还是大儒,识见远过常人,并不受世人偏见影响。何况韩冈雪中立于门前的态度,极让程颐满意,前段时间对韩冈的一点看法,早就不知踪影。

当然,韩冈发明产钳一事,在已经流传开来的熙河、秦凤两路也没有被人另眼看待。换做是普通人,自然会有问题。但他怎么说都是药王弟子,插手妇产,也没什么人会觉得不对,而会说做得很对。

世上许多事,有人能做,有人不能做。要看身份,要看人。若有人抱着‘和尚摸得,我就摸不得’的想法,名声尽毁都是轻的。东施效颦,从来都是极具现实性的道理。

看着侃侃而谈的韩冈,程珦难掩眼中的欣赏。身份才学名望品行皆是难得,而且还跟自家关系匪浅。家世浅薄的这一条,在程珦看来去,却是韩冈的一条优点。

程珦向来识人。当年程珦请濂溪先生周敦颐做两个儿子老师的时候,周敦颐还是一个监狱中的小官。但就是这个到了熙宁年间依然不算知名的狱官,将二程引上了追求天人至道的道路上。

第一眼见到韩冈,稳重有礼的举止,就让程珦有了三分喜欢。再与其交谈了一阵,对他更是看重起来,前途的确是不可限量。想想自家的孙女儿,也曾在膝前念着‘玉昆哥哥’的好,这让程珦动了点招孙女婿的心思。

教书育人的确能声名广布,可就算名气再大,在这个世道上也很难攀上一门好亲。泰山先生孙福,家世清贫,穷到四十岁才有弟子将自家妹妹嫁给他。同在洛阳城中的邵雍,也是穷了半辈子,到了四十岁后才有了家室。

程家女儿的婚姻有些高不成低不就。论身份,他们也是官宦世家,诗书传家的书香门第。只是几代以来虽是代代为官,但也没一个能身居高位:曾祖程希振是虞部员外郎;祖父程遹卒于黄陂知县的任上;程珦做了几十年的知州,就是不能升上去;至于其余程氏族人,为官者甚众,但同样没有能成为高官显宦的。在大宋官场上,是十分常见的中层官员家族。

这样的家族,屡代簪缨的大族不会与他们联姻,一般就是和同等或是稍低一些的门第结亲。但二程是什么身份?当世大儒,一代宗师,与富弼、吕公著来往频繁。与宰执高官走得近了,眼界随之高涨,女婿当然要三挑四选。

只是出色的弟子往往早有婚姻,向他们求学的士子虽众,可能入他们眼帘的,往往都是二三十岁之后,早就娶妻生子了。

孙女儿年岁渐长,程珦为此挂心了很久,终于碰上了一个好的,自是不能放过。

“听闻玉昆你二兄皆没于王事,只有家中双亲。你留在熙河任职时倒也好办,但如今河湟功成,考上进士后,当会出外为官。不知玉昆你日后处置?”

韩冈也为此伤过脑筋,“家严如今在熙河监理屯田事,家业也尽在西北,学生的确不便奉双亲同至任上。如今也只能盼望考中进士后,还能回关西任官。”

实在不行,还有冯从义这个表弟呢……不过这话就不必说了。

程珦张了张口,正待要说下文,程颐却抢前一步,“忠孝二道不可偏废,玉昆若能回关西,一方面能为过守边,一方面又能奉养父母,的确是件两全其美的好事。”

程珦不快瞪了儿子一眼,想了一想,却也顺势将话题绕开去,不再提起后话。

……………………

“二哥,你方才拦着为父却是为何?”等设宴席款待了韩冈,将之送走之后,程珦回头便问着儿子:“二十九娘快到年纪了,难道不要挂心起来?还是说,你觉得韩冈这个人选有什么不合适的地方?”

“父亲大人有所不知。前两年就听说韩玉昆已经跟王韶的外侄女结亲,怕是今科考试之后,就要成亲了。即便没有此事,韩玉昆前面不也说了吗,他已经有了一儿一女,难道让二十九娘嫁过去就为他带儿女?”

程颐重礼。韩冈未婚即纳妾,不合礼法。为着名妓闹出的那摊子事,程颐也是难以认同。作为弟子晚辈,韩冈的品性才学无可挑剔,让程颐很是欣赏。但要家中最受疼爱的二十九娘嫁给他,他就不可能点头。

韩冈未婚便有儿有女之事,程珦并没有放在心上,此事如今很常见。倒是韩冈和王韶的外侄女订了亲,让程珦颇感失望,不甘心的又追问着:“他与王韶家结亲之事可是确实?”

“子厚表叔曾经写信过来,提起河湟之事时,顺便提了一句。”程颐道,“韩冈这个佳弟子,子厚表叔难道不想要,他家也是有女儿的。可惜两年前问话的时候,韩冈已经跟王家定亲了。”

“原来是这样啊……”程珦微感失望的点点头,幸好方才没有问。结没结亲的问题,贸贸然的直接问出来,就未免太冒失了一点。

“何况二十六娘的嫁妆也是问题,韩冈这样的女婿,我们家可给不起嫁妆。”程颐又道。

韩玉昆身份不低,选上这样的女婿,光是嫁妆便给不起。

为什么进士那么受看重,以至于榜下捉婿。就是因为进士升官容易,顺利的话,十几年就能侧身朝堂之中,而韩冈,他已经是朝官了!

京城富户要找一个进士女婿,如今的嫁妆都要给到五千、一万。而韩冈的朝官身份,使得要攀上他这门亲事,少说也要上万贯的财货田产。

就算韩冈本人是个不在乎财物的性子,随手就将封赏送了大半给张载。但程家也得担心女儿嫁到了韩家后,会不会嫁妆给得少了,会受人欺负和鄙视。

世风沦落,人心不古。

新妇在夫家是否会受到重视,端看嫁妆给的份量。送得少了,直接休掉的都有。就算是王公贵戚,要嫁女儿的时候,也得想方设法凑出三十六个箱笼,带上百来亩脂粉田来。

对此,一干有识之士无不为此扼腕叹息,大加抨击。可到了现实中,换做程颐程颢要嫁女儿,他们也不敢拿着女儿的幸福做赌注。

要怪,就怪程颢没能在京城时早问上一步,那时候把亲事定下来就方便多了。但三年前鄂娘才十岁出头,怎么也不可能找上已经年届十八的韩冈。

程珦叹了一口气,自嘲的摇了摇头。毕竟孙女明年才十三,根本不用着急的。

