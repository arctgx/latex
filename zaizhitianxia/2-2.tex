\section{第一章 一入宦海难得闲(二)}

【第二更,求红票,收藏】

儿子的话,韩阿李听了就不高兴,送信的人可是她选的:“小王货郎来往凤翔秦州二十多年了,如果再算上他老子老王货郎,父子两人在秦州和凤翔两头跑加起来快五十年,给人带的信,只要人还在肯定能送到。多少年的信用在了,他们不会说谎!何况给你舅舅、二姨的信都送到了,说给你四姨的信也送到了,难道还会有假?都说读书读多了心眼就变多,还真是一点都没错!三哥儿你也是越变越滑头了……还是原来书呆子的那样好!”

韩云娘一下捂住嘴,猛的低下头,肩膀一抖一抖的暗笑着。

韩冈被骂得无可奈何:“娘说的是!”

“你看你,滑头了了不是!?什么‘娘说得是’!分明就是再说‘娘说得不是’!”

韩阿李这么一说,韩冈说是也不行,说不是也不行。他求助看看自家老子,韩千六却是一辈子听惯浑家骂了,安之若素的夹着小菜,照常吃饭。‘算了,三十六计走为上好了。’韩冈想定,三两口把早饭吃了,也不顾韩阿李还是不痛快,道了声孩儿走了,便到外院左厢后的马厩牵了自己马出来。

原本这些琐事都是李小六负责,但昨天韩冈放了他的假,让他回家探视父母,现在韩冈只能亲历亲为。

韩冈牵着马,韩云娘从后院小碎步的跑过来,依依不舍的送了韩冈出门。韩冈骑上马,走了老远后,回过头,还能看着小丫头倚门望着。

韩冈家离州衙不远,出了家门前的小巷,向左一拐,一百多步外就是州衙大门,同时也是秦凤经略司衙门。按说这么近的距离走路就可以了,养匹马在家还浪费草料钱。但官员的身份让韩冈必须骑马。若是看着一个同僚身穿官袍在大街上赶路,任凭哪个官员都要摇头,说他有失身份。

转眼就到了衙门前,韩冈收缰下马,守在门前的一群老兵中走了一个出来,将韩冈的马从小门牵到州衙里的马厩里去养着。在厢军和禁军中都有降等的制度,想衙门前的这些老兵,都是没有了战斗能力,无法胜任更高强度的工作,被从军中刷下来,最后领着半俸,在衙门里或是官员家又或是官办的寺庙里,做点杂事。

韩冈正要进门,突然背后传来一声唤:“前面那不是韩官人吗?!”

听到那个声音,韩冈先皱了下眉头,然后回头笑道:“是元兄啊……”

来人是韩冈入京三个月里的变化之一,唤作元瓘,现在是王韶身边的幕宾。元瓘是个还俗僧,是王韶的乡人。新近还俗不久,戴着帽子下面,是才两寸多长的头发。小眼睛,招风耳,蒜头鼻子,脸上总是油光光,相貌甚有特色。

元瓘赶到近前,身上衣物熏得浓香就直冲着韩冈的鼻子。韩冈侧过身子,率先往里走,省得自家被荼毒,嘴里还带着话:“元兄今天来得早啊……”

“机宜今天可是有要事要找小人商议,不得不来啊。”元瓘装着不情愿的样子,实际上却是在炫耀自家在王韶面前受到的重用。

韩冈不怎么喜欢元瓘,倒不是因为这个还俗僧总抱着在王韶面前争宠的心态,对自己莫名其妙的有着竞争心理。只是单纯嫌他总是衣服薰上浓的能毁掉人鼻子的香味,一副自诩风流的模样,这让韩冈总是觉得跟某个他感觉很恶心的家伙的嘴脸很像,但偏偏韩冈却是想不起来究竟像哪一个。

不过王韶倒是赞过元瓘精通书算,有货殖之术。韩冈看王韶的意思,大概是想让元瓘负责市易之事,如果一顷四十七亩的事争出个眉目,不但屯田可行,市易也可以乘机浮上台面——王、窦的万顷和一顷之争,争得不再是田地多寡,而是朝堂的信任到底是哪一边,这实质上已经成了王韶和李师中秦州两个派别的政治争斗。

一旦王韶的说法被承认,那他的其他策略也就同时得到了施行许可,将稳稳地把持住开拓河湟的控制权。至于李师中、窦舜卿,还有向宝,都不可能再留在秦州。反过来,王韶若是失败,他也在秦州待不住了。

韩冈一边想着事,一边有一句没一句的与元瓘扯着闲话。在走过第二道门后,韩冈拱手道别,如释重负的往左转去。而元瓘则看着韩冈的背影冷哼一声,继续往前走。王韶的公厅在州衙第三进的西厅,而韩冈却是在第二进。

元瓘不痛快的哼哼声,韩冈虽然背着身,还是听得很清楚。温文有礼的向迎面走过来的同僚打了个招呼,韩冈心中觉得莫名其妙,这元瓘的敌对意识到底怎么来得。难道他以为在王韶面前表现得好,就能压倒自己,混个更高的官位出来?

笑话!

他跟王韶是什么关系?说是政治同盟是有些勉强,但说是助手,王韶却从不敢把自己呼来喝去——自己并非是从王家门客这个身份上推举出来的,在人格和身份上是平等的,而元瓘是什么……走卒而已!

真是莫名其妙!韩冈摇着头,往自己的公厅走去。

经略安抚司,管得是一路军事,又名帅司。所以衙中的公务都是跟军事有关。军队、堡垒、补给、道路、情报、器械,这些是经略使要考虑的军务,必须面面具到。

大的战略规划,虽是由天子和两府决定,但也会征求经略司意见,更多的时候还是由经略司提议而天子两府审批。战略规划的实行,掌中军的自然又是兼任兵马都总管的经略使,下面各部则有副总管、钤辖、都监分担,出谋划策的是机宜、参军、参议这些幕僚,至于勾当公事,也就是韩冈的工作,便是最为繁琐的庶务。

虽然批奏并不归勾当公事处理,但要按类分发到各曹各司,然后将各曹各司处理好的公文收集起来,检查过后再转发给原主,算是承上启下的部门。经略使和经略司中的其他高官交代下来的事情,如果分不清是由哪个分司接手,也是勾当公事处理。除此之外,一些其他曹司不管的琐碎杂务,也是勾当公事的任务之一。

韩冈在这间有些阴暗破旧的房间里,做了有十天了,感觉下来他的这个工作,是类似于办公厅主任之类的职务,每天要面对的公文要按堆来计算。

幸好自己不是一个人,这是韩冈第一天走进这间屋子时的想法,同为勾当公事,还有另外四名选人。这在诸路中,也只有关西诸路才能享受到的庞大编制,若是在两浙、江东那边,经略司中,通常只会有一个管勾公事。而现在的想法则是,日他鸟的,都这么些天了,李师中你怎么还不动手?!

摆在韩冈,而其他四人,这些天有两个告了病假,有两个各自被李师中和向宝调去处理另外的要务去了,整个勾当公事的公厅中,就剩韩冈一人来承担原属于五人的工作。

这样的独角戏,自韩冈走进州衙的第三天便已经开始,到现在七天过去了,还没有结束的迹象。官厅中的公事,基本上都是胥吏处理,而后才有官员查看是否有问题。即便五名勾当公事只剩一人,只要肯放手,韩冈照样可以喝着热茶,弄两本诗集来读。

但韩冈看起来不放心别人的样子,他手下的胥吏把事情做好后,他都要重新检查一遍,找出一点错来,就会丢回去让人重做。七天来一点疏失也没有出现,处理得游刃有余。不过任谁都知道永不出错是不可能的,不少人都在想他如此勤力,迟早要累昏头,而韩冈本人只希望李师中也能这么想。

在门口,韩冈将脸板起,大步跨进房中。房内,十几名从属于勾当公事的胥吏已经在侯着。领头的一个叫王启年,在衙中待了十几年了。据说本是个市井无赖,后来不知从哪里诈了一笔钱来,送给当时秦州通判小妾的表弟,进了秦州州衙里做吏员。他在衙门中日子久了,也颇有些手段,收服了几个兄弟,在衙门里干起来奉承上官,盘剥百姓的生意。

见到韩冈进来,王启年便领头上来行礼。只是他的动作都有些慢慢吞吞,连带着跟在他后面的十几人也是一副黏黏糊糊,不情不愿的样子。

看着他们这疲沓模样,韩冈脸色更加深沉下去,冷声道:“王启年,你们没吃饭不成?!”

“小人不敢。”王启年回了一句,动作稍微快了一点。

韩冈冷眼看了他一下,便坐到自己的位置上。这些天,韩冈始终板着脸,一点笑模样都没有。衙门中,每一个胥吏都知道,新上任的勾当公事是个心狠手辣之辈,城里有名的陈押司跟他过不去,被他反手就杀了个绝户。

一开始时,王启年他们也是战战兢兢。只是看着其他四名勾当公事相继找借口避事,从中嗅出了什么味道,又暗中得了他人的吩咐,渐渐开始挑战韩冈的权威。当然,这是一步步来的,到了现在,也不过是行礼时拖沓一点,做事再慢上一点,弄得太大,他们也怕惹毛了这个看起来性格颇为阴狠的韩三。

只是韩冈尽是板着脸,在公务上又挑剔得要命,让王启年他们心中都很不痛快,私下里都说,就算没有人吩咐,也要让这个菜园子见识一下衙前虎的手段。

