\section{第14章 霜蹄追风尝随骠(一)}

当西夏的灭亡已成定局,西北的这一场高潮迭起、每每出人意表、峰回路转的战争,终究还是到了尾声。

无论辽国还是大宋,都还没有做好全面战争的准备,眼下的当务之急,是将西夏的遗产分割,然后吞并、消化。

韩冈和折家夺取丰州的计划,就是为了争抢西夏的遗产。

“西夏亡国,亏光了老本,这是不消说的。而大宋损失虽大,但能得到银夏和甘凉,好歹还不算折了老本。不过耶律乙辛的便宜就占得大了,让人恨啊!”

听了折可适的话,黄裳不屑的说道:“灭了西夏,占了兴灵,的确是神来之笔。但兴灵之地的党项人人数尤众,契丹人想要据有其地,不知要死上多少人。”

折可适呵呵笑道:“所以小弟说是耶律乙辛占便宜,而不是大辽。何况辽国的损失,又关耶律乙辛何事?阻卜人的势力被削弱了——西阻卜也算是阻卜的一员——过去曾经大败辽国,留下不少血仇的西夏也灭亡了;大宋一番辛苦,可树上最大的一颗果子,就给耶律乙辛不费吹灰之力的轻易摘走,让大宋丢人现眼。名有了,利有了,仇报了,顺便还让敌手丢了大脸,耶律乙辛的地位将会如日中天。”

韩冈本是在看着公文,听见两人的对话,抬头道:“以耶律乙辛的行事,当是会将那些偏向他,却又不完全听命的势力安排到兴灵去。在剿杀党项部族的过程中,逐步消耗他们的实力。他不会吃亏的。”

好吧,其实这是韩冈的想法,换做他来做,肯定会这么去做。

黄裳摇摇头,犹有不屑:“想不到耶律乙辛眼光狭隘如此。”

“记得佩六国相印的苏秦吗?他为什么送张仪去秦国。人与国家的利益不可能是一致的。莫说权臣,就是皇帝,不也有隋炀、商纣吗?”

虽然韩冈认为杨广的名声有一多半是多亏了硬要看自己起居注的那一位,帝辛也是得多谢武王、周公,乃至春秋时代的百家诸子常年不懈的诽毁,但这时候就没必要标新立异了。

韩冈深有感触的叹着,“耶律乙辛是权臣,不是皇帝。在辽国的未来和自己的权位之间,你说他会选择哪一个。而且在他眼里,多半是自认为只有巩固了自己的权位,辽国才有未来。”

“这样的人其实不少啊……最近不是有个徐禧?还有朝堂上的相公、参政呢。”折可适毫无顾忌,嬉笑不拘。

黄裳更是书生意气,也不会认为骂几句朝堂诸公有什么大不了的。跟着折可适一起骂起了王相公和吕参政。

韩冈看了折可适一眼,又低头下去看公文。

折家这一次的心思不小,多半也是被辽人刺激到的。

旧丰州也好,新丰州也好,其实能算是折家势力辐射范围。

丰州的第一代是王甲,王家的家主,归附大宋、修筑丰州城都是他的决断。与折家联姻也同样是他的决定。其子、同时也是丰州第二任知州王承美,便娶了折家女为妻,第三代的王文玉得喊折御勋和折御卿为舅舅。而折御卿、折御勋还有一个妹妹,嫁给了杨业。

对于折可适来说,折御卿和折御勋两位曾祖父、曾叔祖父,隔得虽不算远,但也是几十年前的人了,逢年过节倒还能记得上柱香,呈上碗麦饭。但平日里,极少挂在嘴边。

只是折家在云中之地势力扩张和根基深植,泰半是折御勋和折御卿的功劳。云中大族,基本上都跟折家有或近或远的亲缘关系。这些关系交织成的一张大网,使得折家在云中之地屹立不倒。

不过若是辽人占了旧丰州,并向南收取了所有大漠以东的西夏土地,将云中这个突出部半包围起来。折家倾覆的危机将近在眼前,再严密的关系网,也挡不了马刀一击。折家是不得不拼命。

韩冈已经传书李宪,这件事需要他的协助。

几天后,被韩冈使人飞马传书召回太原的李宪,终于到了,比韩冈预计的要早了三天。

行程匆匆的李宪一脸的焦急,一见到韩冈,劈头就道:“龙图,旧丰州夺不得,夺不得呀!旧日丰州的辖地,一多半在辽人手上。强行要取,那是要出大乱子的。”

韩冈愣了一下神后,顿时哈哈大笑起来:“都知误会了。丰州和丰州是不一样的。这一次我要占的,是丰州城——被元昊占去的丰州城。韩冈再是糊涂,也不可能这个时候去从契丹人手中抢地盘。”

听了韩冈的话,李宪也怔住了。就像说起开封,有的时候指的是东京城,五十里城墙括起来的那一小块地方,但有的时候,则是指的整个京畿路。丰州城的确也可以说成是丰州。

过了一阵,他整个人松下了一口气,软软的坐在了交椅上,自嘲道:“真是急糊涂了,一路上只顾着赶路,都没好好想一想。以龙图之智,的确不会如此糊涂。”

折可适和黄裳在旁抿着嘴,想笑却不敢笑出声。韩冈笑道:“也是都知心忧国事之故。”

李宪的误会的确是大了。

同样的一个地名,古今的位置许多时候都有着极大的区别。比如渭州,古渭州就是现在的巩州,原名古渭寨,真正位于渭水之滨;现在的渭州,却是在泾原路,泾渭分明的泾水从州中流过。再比如榆林,到现在为止,榆林都在黄河前套的东端,而千年后的榆林,却转到了银州城附近,远隔千里之遥。

古丰州远在黄河北,地处九原,是秦时从匈奴人手中夺来的,到了唐时,还是有名的受降城所在。之后被辽太祖耶律阿保机领军占据,至今尤归于西京所辖。这古丰州是不用想的,十年之内都不会有机会。

而旧丰州虽在黄河南,可跨度极大。丰州王家本是党项藏才族出身,在宋初,黑山南北皆是藏才族三十六部的集聚地。等到王家的家主王甲举部众内附,受命立城建州,甚至有居于黄河北侧的藏才族余部投附。不过那都是属于羁縻性质,就像广西的邕州占地之广,甚至堪比一路之地,左右江地区全都属于邕州,但下面尽是羁縻州。

旧丰州向北去的辖区,曾经跨越黄河,在黑山之下,应算是河套平原中的前套地区。但在契丹人势力扩张,加大了对西北的控制之后,韩冈也不指望能去占这个便宜。

韩冈将李宪邀请到白虎节堂中,在一幅新做的沙盘边,指着上面的一个城池标志:“我所想要的是丰州城,和附近的一小片核心地区,并不是丰州全境。”

李宪看着沙盘,缓缓的点着头。

旧丰州城的位置,在府州西北二百里——这里的府州,指的是府州城——是王甲内附后,贴近宋境的位置修建的城池,位于黄河支流屈野川边。

收复旧丰州城,因为是在契丹人嘴边抢食的缘故,难度虽然不低,但比起夺取兴灵,可就简单多了。新丰州的位置,是划了府州的萝泊川掌地复建,本就在府州城西北百二十里。尽管府州和新旧丰州并不是在一条线上——新旧丰州是正东正西的位置——可从新丰州再往西去百十里,并不算多。最关键的是,屈野川向南去,流经的是麟州的州治新秦。直接从麟州北上,比从新丰州向西要方便许多。

李宪专注的看着沙盘,过了一阵,他抬起头,紧紧盯着韩冈:“龙图的目的,当不在丰州,而是麟州、晋宁军,乃至银夏之地。”

“看了沙盘,的确直观不少吧?”韩冈笑了起来,李宪说得正是,他和折家的目的都不仅仅局限于旧丰州,“重夺旧丰州,目的在于屈野川、及其支流浊轮川【今乌兰木伦河】流域。控制了这两条河流,等于关上了辽人从西京道南下的大门。将大漠【今毛乌素沙漠】以东的西夏国土,也就是麟州、晋宁军以西,银州、弥陀洞以北的大片土地,一同收归大宋。”

从麟州沿着屈野川上溯,大约走上两百里那就是旧丰州的所在,再向西偏一点,是一片有水草有树木的地方,在后世乃是以鄂尔多斯为名。

韩冈这段时间一直在揣摩黄河西侧的地理,与已经越来越模糊的记忆相印证。榆林的位置,基本上可以确定在如今的弥陀洞和银州附近。从弥陀洞北上屈野川,不用翻山越岭,没有太崎岖的地形。这条路本来就是关中连通黄河以北的九原【包头】的主要道路。韩冈虽然已经记不清了,不过想来后世也当是修了沿着榆林北上,直通鄂尔多斯的道路。

在韩冈的示意下,黄裳拿着根白木小棍指着沙盘解说:“官军重夺旧丰州之后,便拉平了防线,使得麟府丰的云中之地,甚至可以直接得到银夏驻军的支持,而在防御上,西面是毫无人烟的大漠,要走上一千多里地,才能抵达兴灵。或者说,想从兴灵来攻,得走过一千多里的沙漠。”

手指粗细的尺许木棍沿着黄色的沙漠边界划了一条弧线,“若是给辽人抢先一步,河东路就不仅仅要防着北面,西面也会是警.号阵阵,连同银夏,也同样会陷入随时会被攻打的危机中。若是官军能抢先夺占,日后只要加强北方的防御就够了,辽人虽然可从沙漠中绕道,但消耗之大,便绝不可能循此路驱动大军。”

黄裳收起木杖,“守御旧丰州,对钱粮的消耗不在少数。不过比起银夏、河东都要加强防守的情况,则要节省大半。而且晋宁、麟州之西,屈野川之南,弥陀洞之北的这一大片土地,可以算是不错的草场,能放养大批的马匹。若以做买卖的话来说,是一本万利。”

沿着河谷移动兵马,远比翻山越岭要容易得多。旧丰州跟麟州之间,可以通过屈野川河谷来运送兵员。收复丰州及其周边的屈野川、浊轮川流域,在此设立一条寨堡防线,便能让河东西侧、银夏之北,得到一大片缓冲地,同时也是不错的牧马地。

控制了屈野川、浊轮川,便封住了辽人南向的大门。西夏的土地,以大漠为分界线,以北以西属于契丹,而以东以南属于大宋。这是眼下能争取到的最好的结果了。

李宪回身,对韩冈拱手一揖:“龙图传唤的心意,李宪已经明白。李宪愿与龙图同上奏本,收取丰州旧地。”

