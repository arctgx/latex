\section{第14章 飞度关山望云箔(一)}

在迁江县【今迁江镇】过了江,就是位于群山中的一块盆地。只有一座座小山包在平地里突兀的竖起。如果在北方,这么一片肥沃的土地,至少能养活十万人口。

奇异的地理,让官兵们好奇的看着周围。只有苏子元,他两只眼睛直勾勾的望着前方。离着宾州还有三十里,而到了宾州,距离邕州就只差一座昆仑关了。

在抵达桂州城之后,两个指挥的荆南军,只休整了一夜之后,就开始在韩冈和李信的带领下向南进发。而苏子元自请作为向导随军南下,士行以孝字为上,理所当然的就得到了章惇和韩冈的准许。

两个指挥,加上李信从自己的麾下带出一个都,总共八百四十人。除此之外,桂州补助给韩冈、李信一行的,就只有一队帮他们拖着甲胄辎重的骡马。

接近千人的队伍行进在平坦的官道上,只有刷刷的脚步声响着。

身后一阵蹄声接近,回头看过去,是在后压阵的李信赶了上来。

“运使,差不多该歇一下。”李信一板一眼,对苏子元身边的韩冈说着。就算是韩冈的表兄,但在人前,他也只称呼官职。

苏子元很早就听说过这位新一代的名将,号称掷矛之术独步军中,殿前演武时,天子都拍案叫绝。其人在关西、荆南的战场上斩首无数。据说曾于一战之中,连杀七位山蛮族酋。

战功显赫、被天子看重,还有个宰相家的女婿、日后极有可能进政事堂的表弟。这样的将领,苏子元本以为他会是恃功自傲的狂夫。谁想到竟是个沉默寡言的性子,而且从不卖弄与韩冈的关系。在人前对韩冈的称呼就是一桩例子。不过,苏子元也听韩冈提起过。关西名将种谔的子侄,上阵时同样是喊着他大帅、太尉的时候居多。

论起行军打仗,李信是专家,韩冈点点头:“就休息一刻钟。”

李信一声令下,除了守卫远近的十几名斥候,所有士兵都在官道上直接坐了下来。武器就都放在手边,随时可以起身迎战。

韩冈也下了马,亲兵帮他拿了张小交椅坐着。唯有苏缄的长子,坐下来又站起来。

“在担心邕州吗?”

听到韩冈这么问道。苏子元张了张嘴,想解释一下,却又不知该如何解释。他当然担心邕州的家人,但这个心思放在向导上,就难以取信于人了。

不过韩冈没有为难苏子元:“……如果贼人已经攻破了邕州,就没必要再封锁着消息,放出来才能震慑人心。”

道理是没错,但也只是安慰性的话语。邕州已经连着几天没有斥候传回军情,南下的一路上,听到的消息都是自相矛盾。唯一清楚的就是贼军打造的攻城器械被苏缄烧光,战败被俘的官军中有人投靠了交趾,再往后就一片空白了。苏子元心里怎么可能踏实得起来?

“运使,到了宾州之后,下官愿去领一队人马,去昆仑关查探军情。”

“不行。”韩冈十分干脆肯定的拒绝,“打探军情自有斥候,不需要军判亲自出马。”看到苏子元急了起来,他又安慰起来:“伯绪你大可放心,我与章子厚奉旨南下,不是为了将贼军礼送出境的。”

苏子元点点头,终于坐了下来,只是垂着头,一句话也不说。

知道苏缄的儿子心急如焚,韩冈估摸着快到一刻钟的时候就站了起来。

这是一个信号,李信和一众将校也都一起站起身,催促着下面的士卒收拾一下,准备继续赶路。

“歇息了也差不多,就别再风地里坐着。”看到士卒们的动作有点慢,韩冈的声音放大了一些,“前面本官已经派人先去宾州准备了,到了宾州城,就有热水热饭,可以好生的歇息一夜!”

“诺!”

士兵们齐声答诺的声音一下变得朝气蓬勃。也难怪他们能提起精神,吃饭时能吃上热饭热菜,行军后能用热水泡一泡脚,就是苏子元听得都心动了。

韩冈能如此重视这等寻常看不起眼的琐碎小事,苏子元暗道,难怪能落下如此大的名头。只是准备起来繁琐一些,却能最大程度的消去士兵们的不满。八百将士跟随韩冈南下,在连续多日的行军中,依然保持着高昂的士气,这个手法.功不可没。

整队之后,大军又重新进发。但没走多远,派到前面探路的游骑,一人疾奔而回。而跟着他一起回来的还有两名骑手,其中一人,还是韩冈早早派出去的。

韩冈一早就派出了跟随他南下的亲信韩廉,带着一队人马作为信使,通知沿途州县做好迎接大军的准备,也是负责鸣锣开道。每到一处州县,就立刻派出人手到下一座州县去安排好食宿。现在回来的就是他在迁江县派往宾州的其中一名信使,只是他的身后跟着个陌生的士兵。

“启禀运使,宾州城正被交趾贼军围困。”信使指了指身后,“他就是宾州派出来求援的。韩殿侍正带人盯着贼人,命小的回来禀报运使。”

那名精悍的军士虽然惊讶于韩冈的年轻,但他还是看得出韩冈的地位在众人中是最高的,跪下来从怀里掏出一份封了火漆的信函,高举双手呈给韩冈,“小人黄安,奉了宾州赵知州的命,出来往桂州求援,想不到运使已经领兵到了。还请运使尽速出兵,杀光那群狗贼。”

求援的信函指明是给广西经略司的,韩冈不便拆看。不过他将信函给苏子元,让他验了封皮上的火漆、签押和印信。就见桂州军判点点头,证明是真货。

确定了来人的身份,韩冈也不需要再看必然满是夸大之言的求援急报,“围城的贼军到底有多少人?”

“有一千多兵马。”

“领军的使交趾军,还是广源州的蛮部?”

“……装束很乱,似乎是蛮部。”

“他们到底攻城了没有?”

“刚过来时他们杀到城下,要宾州开城投降。不过赵知州说官军就要来了,砍了两个密谋献城的奸细。他们见城门不开,也不敢攻城。就在城外的庄子上烧杀。”黄安猛磕了一个头,抬起头来,额头和眼圈都红了,“他们来得太快,许多百姓都没能来得及逃进城中。运使,再不去救,他们可都要被杀光了!”

“运使。在侬智高之乱后,广西各州的城池都加高增修一遍。宾州城防不差,一千多人肯定攻不下来。如果内外配合,当能将他们聚歼在宾州城下。”

苏子元这是在敲边鼓,韩冈笑了一笑,髙喝一声,“李信!”

“末将在!”李信踏前一步,“请运使吩咐!”

“你去问问下面,哪个愿意拿到南下的第一功?!”

“末将愿意!”

“小人愿往!”

“职部愿往!”

韩冈询问军情的时候,几个将佐都竖着耳朵,一听韩冈要派人做先锋,立刻跳出来抢着要第一个出阵。军心可用,韩冈对苏子元笑道,“邕州尚远,就先拿那千名蛮贼祭刀!”

………………

宾州城外浓烟滚滚,来袭的蛮贼已经分散开来,在各个村庄中疯狂杀戮劫掠。而城中守军紧闭四门,全然不敢出击,坐视贼人在城外肆虐。眼睁睁的看着贼人将抓来的男女丁口用绳索绑了,准备带回去驱使奴役。

统领这群强盗的头领刘永坐在一座村庄最大的一间屋子中。身边围了几个相貌姣好的女子,怀里还搂着一个。她们战战兢兢的服侍着刘永,丝毫也不敢怠慢一点。张开口,就有人送了菜,抬起手,就有人将酒杯奉上来。

下面的头领,一个个也都是如此享受。抢劫得来的财物女子,让他们兴奋得一杯一杯的灌下美酒。

只是宴会并没有开得太久,一名探马带着紧急军情赶了回来。

“什么,宋人的援军来了?!已经到了三十里外?”刘永将怀里的女人甩手推倒一边,一下站了起身,浑身的酒意都醒了,“来了多少?”

“有八九百,肯定不到一千。”

“才八九百,当是先锋吧……”

就算仅是出自溪洞的广源蛮军,但刘永和他的兄长广源州蛮部的大首领刘纪,一向号称知兵,家里藏着兵书,寨子里也养着汉家的读书人。这次出兵,也让他们当着参谋。

“何学究,你看如何?”刘永问着离着自己最近的留着一把山羊胡子的老学究,这是他的谋主,出得主意让他们得以满载而归。

何学究放开了搂在怀里的女人,捻着胡须:“从宋军所处的位置上看,他们当时就在今天清早从迁江出发的,方才探马打探得离宾州三十里,而现在可能只有二十里了。这路走得未免太快了一点。从迁江到宾州,行军可不是一天该走完的路程,肯定是救邕州心切……”

刘永听出了何学究的言下之意:“你的意思是?”

“我们这边派了探马,想必宋人也不会不派探马,方才就有回报说周围有骑马的探子。以在下看来,不如先收拢兵力,看看宋人下面会怎做?如果他们停下来休整,我们就带着这些男女回昆仑关。如果他们敢来救援……”何学究抖着山羊胡子,哼哼的阴笑了两声,“正好可以弄到些趁手的兵器,也可让李常杰不敢小觑洞主。”

刘永对何学究言听计从,立刻召集起散在周围的人马,并派出探马。半个时辰之后,派出去的七名探马就回来了一人,背上还插了一支箭,血流了满身。报说宋军已经到了十里之外后,就昏倒在地上。

“好!”何学究一声大叫,“不过一个时辰,就赶了二十里。虽然宾州北面的这一段不是什么山路,可跑得这么快,哪里还有气力打仗?!想不到领军来援的宋将竟然这般愚蠢。”他跳起来对刘永道,“二洞主,先派主力带着捉到的生口回昆仑关,我们只领两百精锐躲在这村子里面,外面再生些烟火做遮挡。宋人必然是要去救人的,只要他们追过去,就可以从背后杀得他们措手不及!”

刘永听明白了,咧开大嘴喜道:“到时候,前面再回来……”

何学究得意的笑起:“正好可以杀光这群宋军!”

