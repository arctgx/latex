\section{第27章 舒心放意行所愿(下)}

时间一点一滴的过去。

虽然还没到开门的时间,但来到宣德门前的文武官员们却越来越多。朝臣们都是想早一步得到最新的消息,远比往日早上一个时辰,便又官员聚集在宣德门前。

仅仅是低声细语的议论,汇集起来后也成了一股嗡嗡如同蜂鸣的声浪。

“元长。”蔡京的身边也有人聒噪着,同科的友人赵挺之正强压着兴奋低声说着,“你听说了没有?”

“是二大王逗留宫中未出?小弟已经知道了。”

“岂止是二大王逗留宫中未出!”赵挺之一下拉高了声音,但立刻又警觉的望望左右,小声道:“两府诸公可全都给召入了宫中。就在两个时辰前,赶在下半夜。”

蔡京当然知道,他还派了家人守在御街前。不过他却略略瞪大了眼睛:“竟有此事?!”

“骗你作甚?!”赵挺之转着脖子努努嘴,“没看现在都在议论吗,恐怕不知道的加上元长你,都不会超过十人。”

“若是圣躬不豫,可不是现在的情形。京城里早就响钟了。”

一年前,太皇太后上仙,京城中各家寺院的铜钟铁钟同时敲响。若是天子龙驭宾天,当然不可能会是现在这般安静。

“就怕不是啊!”赵挺之摇摇头,“今天的早朝,还不知道是谁从殿后出来呢。”

看似忧心忡忡,但赵挺之的双眼中却掩饰不住兴奋之情。

这也正是蔡京现在的心情。帝位的转移,必然会给朝堂大规模的震荡,不知会有多少重臣从高处落下,留下一个个空位待人填补。对于沉寂下僚的低品朝臣们来说,却是最好的机会,只要把握住了,就有一飞冲天的可能。

宣德门五道城门中最外侧一道突然有了动静,随着门后的声响,被拉开了一条缝隙。门前的朝臣中,立刻就引发一阵骚动。

蔡京心中惊讶,晨钟还没敲,应该还没到开门的时间。

随即两队由七八名内侍和班直组成的队伍,各有一名身穿紫袍、背负黄绫包裹的黄门领队,从宣德门的侧门中出来。

是传达圣谕的内侍。

“不知是给谁的。”赵挺之盯着两名紫服黄门背后黄绫包裹。

蔡京拉着他退后了两部,“还是先让一让吧。”

朝臣们给他们让出了道路。

两队人马一前一后,并没有放马疾行,而是亮着嗓子髙喝着:“两府宰执同请,延安郡王已为皇太子,皇后权同听政。”

门前广场在一瞬间就变得安静了,随即却又化作更大的声浪爆发了出来。

‘皇后垂帘,这怎么可能?!’

不是没有人怀疑自己的耳朵,但当身边的人全都发出疑惑的质问,却反而得到了证明。

“太后尚在,皇后如何能垂帘?!”赵挺之惊怒道,“祖宗法度呢?!”

“祖宗法度……”蔡京叹了一声,仰头望着宣德门城楼上的一盏盏犹在闪耀的灯火,“熙宁、元丰十三年,不知是在做什么?”

现在皇后权同听政,请太后垂帘的奏折就怎么也不能上了。幸好请封孙思邈的奏折还能派上用场。

‘有备无患终归是没错的。’蔡京安心的想着。只是他再一摸袖袋,脸上却一下泛起古怪的表情。

“怎么了,元长?”赵挺之问道。

“不,没事。”

蔡京摇摇头,心里却发了慌。方才陡然间吃了一惊,现在都忘了请太后垂帘同听政和请封孙思邈的两份奏章,各放在在哪一只袖袋里了。

宣德门已开,人流涌动,蔡京暗暗叫糟,这时候,任何有异于常人的的动作,都会引起他人的注意。身前身后都是人,万一打开来给人瞥见了内容,而又不是自己能递上去的那一份,那么就有些危险了。

‘早知道就不那么急了。’蔡京后悔着。

他是想在朝会上直接将札子递上去的,这样才能让太后或是皇帝记住自己的名字。否则从通进银台司、中书门下这么绕上一圈,那就不知会有多少份同样的札子一并送到天子的案头上。那时候,天子或是太后只会关心谁没有进札子,而不是谁进了札子。

环目望望左右,就是附近,蔡京发现就有几个摸着袖口,神情呆滞,与自己相仿佛的朝官。

人太多了,现在可不能拿出来确认。但蔡京还是大着胆子,往袖口里掏,这个机会一辈子也不一定能撞上几次,哪里能放过。只是他刚刚将一封札子掏出来,却突然间被人撞了一下。

猝不及防,蔡京的手指不由得一松,啪嗒一声,手上奏折落到了地上。

蔡京心头大惊,立刻驻足弯腰,想将落地的奏折给捡起来。可刚刚弯腰,他的手就停了下来。与他同时弯腰的还有一人,落在地上的奏折也多了一份。

奏折都是同样的外皮,大小形制完全相同,落在地上的位置靠在一起,根本分不出到底是谁人的。

蔡京望望对方,四十上下的年纪,身着朱衣,正是并肩而行的赵挺之。赵挺之此时如同是受到惊吓一般,发白的脸色很是难看。

蔡京歉然一笑,却很自然的将两本奏折一并拿了起来,接着又更加自然的翻了一翻。当即就见到赵挺之脸色骤变。

虽然仅仅是一瞥而已,但请立太子和请太后垂帘的内容已经尽收眼底。

将其中的一份奏折递了过去,蔡京笑道:“这是正夫兄的。”

……………………

朝臣们还未到,但一班宰执们已经等在了文德殿前的东阁中。

虽然几名宰执几乎是一夜没合眼,但看起来精神并不算,甚至还有些亢奋的味道。

不过往日押班时,都会好声好气的与身边同僚聊上几句的王相公,今天却是板着脸,坐在了自己的位置上,并不与人搭话,倒像是又多了一个吕公著一般。

昨夜没有宿直宫禁的几名执政,都想知道上半夜发生的一切。王珪看起来没有好心情,更不可能向他打听他到底犯了什么错,而张璪先回了玉堂,韩冈又被留在了寝宫中,薛向就成了唯一的选择。

韩缜想从薛向这边了解一点内情,但蔡确却抢先一步邻着薛向坐了下来,让韩缜只能转回去找个位子做下。

蔡确就着热茶吃着糕点,漫不经意的问着:“师正,王相公今天怎么变了一个人?”

薛向的心情有些沉重,在昨夜,他的表现并不是很好,只是相对于王珪要强了不少。说实在的,他对王珪实在是有些怨恨,要不是想等着王珪这名宰相的决定,也不至于落到了窘境中。

心情不快,自然就有些尖刻,更无意代王珪隐瞒:“持正当知道魏文正吧?”

“魏征?”魏征的追谥是文贞,但仁宗名为赵祯。为避讳,文贞这个谥号便成了文正,蔡确挺纳闷,“这跟魏玄成有何瓜葛?”

“王相公只是想做魏征。”

蔡确正端着茶盏的手抖了一下,不知是该笑还是该摇头,他低声:“王禹玉和魏征?此比可是有些不伦不类。”

“的确不伦不类。”薛向轻声地笑了笑,偷眼看了王珪一下,声音便比蔡确放得更低,“魏征在隐太子身侧为谋主,至太宗朝中却成了诤臣,不知哪一边才是他的本心。”

蔡确神色骤然一变,声调也随之一沉:“此比不伦不类!”

“的确如此。”

近乎重复的对话,意义却已截然不同。

……………………

韩冈迟了一步。

他是端明殿学士。但在皇后和宫里的嫔妃们眼中,他还有一个药王弟子的身份,更是皇太子赵佣的师长。比起太子太师、太子太傅、太子太保这样如同虚名的东宫三师来,韩冈的这个资善堂侍讲更为亲近。

比起早早的就出了寝宫的宰执们,韩冈直到早朝快要开始的时候才在王中正的相送下,从寝殿里出来。仅仅比皇后和太子早上一步离开。

不过为了避免显得太过惹眼,他自福宁宫出来后,并没有与宰执们一样,直接自文德殿后出来,而是特意快步绕了个圈子,跟其他官员一般走文德门,从后进入朝官们的行列中。

站在文德门处弹压百官的是新进的监察御史里行,对自己的工作正是最为热情的时候。当他看到身着紫袍金带的年轻官员匆匆而来,就开口催促,并打算将这个人给记下来,以待后用。只是当他看清楚这位金紫重臣的长相后,正欲出口的催促,却化作一声尖叫:“韩冈!”

韩冈步进门中,听到这名并不认识的御史,连名道姓的叫着自己,并没有发怒。点了点头,回了他一个微笑:“正是韩冈。”

虽然韩冈并不打算太过惹眼,但那声惊叫已经惊动了很多人,分别在东阁和西阁下,排成队列的两班朝臣,从后往前,渐渐的静了下来。

一名名朝官循声回头,盯着顺着东班队列,从后往前、徐步而行的殿阁双学士。

纵然是全不知情,但本应是太后秉政,却变成了皇后垂帘,要说韩冈在其中没有起到作用,任谁都不会相信。而且在立场上,太后和留宿宫中的雍王对有可能保住太子和皇帝的韩冈不会有好感,可皇后却会将他视为救星一般看待。

从今往后,至少在太子成人前,这名灌园之子的地位将会不可动摇。

有人目光炙热,有人神色冷淡,更多的是羡慕、嫉妒,却没有一人能够无视。

天子垂危,太子监国,身为药王弟子的韩冈肯定是最大的受益者。太子也要靠他来佑护。

宰执们也回头,目光复杂的看着韩冈很自然的走进了班列之中。

“玉昆,你可是来迟了。”排在后一位的苏颂冲他低声道。

韩冈微微一笑:“不为晚也。”

