\section{第18章 青云为履难知足(16)}

【周日有事要外出,晚上的一更大概会到凌晨。还请各位书友见谅。】

韩冈的声音渐渐在梁柱间消没不见。

没人能想到韩冈只是为了从关西调兵,就要将更戍法提上台面。

所谓更戍,就是禁军逐年更换驻军的地点,一般都是从京城至边州轮戍。这是太祖皇帝赵匡胤定下来的规矩。

五代之时,节度使执掌地方军政大权,大股小股的军队各自占据一块地盘,收取税赋供养自己。财权自有,当然也就可以不听朝堂的命令。有鉴于此,赵匡胤做出来的应对就是两个方法,一个是将不私军,另一个就是更戍。让将领与军队脱离关系,同时让军队与地方再无瓜葛。

自此之后,天下再不复五代时的混乱。只是随着承平的时间越来越长,两条祖宗之法的带来的负面效果也越来越大。

将不私军,让领军的将领放在如何为自己找一个好职位,而不是努力训练士卒。

更戍法则随着禁军数量的急剧膨胀,让国家财力无法再支撑下去,逐渐的,以驻泊、就食为名目的禁军就越来越多。到了如今,早就不复实行多年——就算一直喊着复祖宗之法的旧党,也不见几人要恢复更戍法。

“若行更戍,钱粮哪里来,更戍禁军一年就有半年在路上。又有多少时间在边州营寨中守御?”吴充质问着。

“没说要全部轮戍,在河北四路中,各选一将或两将出来。京营之中,也选取万五到两万上下的兵员,去陕西戍守两年。这样的调动,钱粮消耗并不大,又可以为京营、河北训练出可堪一战的军力。”

韩冈对赵顼道,“臣之前亦曾建言,将军中因战事而伤残的将校士卒集结起来,用来整训士卒。一部在关西以战代练,一部则留在驻地,由老卒教之以战,双管齐下,数年之间便能在河北和京营中,为陛下训练出一批能克敌制胜的精兵强将来。”

赵顼轻轻点头,就算更戍法短时间内不可能复行于世,韩冈给出的变通之法,却是可以快速推行。

如果仅仅是西军一家强势,做皇帝的哪会不担心?肯定要掺水掺沙子。现在韩冈把话撂下来了,赵顼当然想让自己的几十万禁军都变成如同西军一般的虎狼之师——他还打算光复灵夏、收复燕云呢!

韩冈的一番话,也将冯京方才的攻击化解于无形。他在西军中人缘好、与年轻有为的将领们关系好又怎么样,他现在表现出来的可是大公无私的态度,为整个大宋考虑,并不仅仅看着西军一家的利益。

“再说调军南下之事。”韩冈前面的一番话,已经让他控制住了殿中议论方向的主导权,“对于河北军来说,调去南方和调去关西,虽然同样是调动,但区别不小。南方有瘴疠,还要进攻严阵以待的交趾军,而去关西只要守城就可以了,并不用担心疾疫。如果去询问河北军上下,想必绝大多数会选择去关西。此乃上驷、下驷之法。关西守城易,南方攻贼难。以西军南下,再将河北军西调,虽然中间绕了一道,但比起直接调动河北军要更为合适。”

韩冈一通话说完,露出诚挚的笑容问着吴充:“不知吴枢密觉得还有什么不妥的地方?”

吴充脸上的表情没有一丝动摇,只是脖子下变得血红的瘤子偷偷泄露了他心中的愤怒。

但紧跟着开口质疑不是吴充,而是站在吴充下首的王韶:“河北情况难知,不过若是让京营禁军更戍关西,军中或有怨言,甚至会让京畿不稳。”

王韶的话让众人为之楞然,怎么他突然向韩冈唱起了反调?

“所以这种风气要加以扭转,既然吃着朝廷俸料,哪有不为朝廷效命的道理?”

韩冈一句回答,殿中众人这下都看出来了,王韶是在帮韩冈提前堵漏。‘配合得可真是好。’吴充眼神冷然。

只见亲自提拔韩冈的副枢密使又说道:“万一有奸人心怀诡谲,暗中散布谣言,恐致兵乱。此事不可不虑。”

“副枢此言韩冈不敢苟同。士卒抗命,自有军法绳纠,奸人致乱,也有朝廷律法在。如果担心京中不稳,加上一条也可以,不愿去关西戍守的,可自请降入厢军听候使唤。”韩冈冷笑了一声,“总不能拿着禁军的俸钱,却不做禁军该做的事吧?不知副枢认为韩冈说得对与不对?”

王韶微微一笑,不再说话了。他与韩冈一搭一唱,提前一步就堵上了有人心怀鬼胎暗中作祟的可能。

“若有不从号令或妖言惑众者,自当严惩不贷!”吕惠卿这时问道,“不过从河北调兵至关西便需要一个月,再从关西调兵到岭南,就算后半程一路舟楫,也需要近两个月,那时候就已经是冬月末了。抵达邕州之后,还要再休整半个月。不知是否来得及赶上出兵的时间。”

“调兵南下,也不是将一路兵马都调空,不过三四十个指挥的事,分摊在关西缘边诸路的城寨之中。即便全数调离,短时间内也不会对关西防线有所影响。如果是分批调动,就更不会有问题。”

韩冈说着自己的方案,“可先从秦凤、泾原两路,点选五千精锐自宝鸡南下,由嘉陵江可一路乘船直抵广西,只要四十天足矣。等京营河北的援军抵达关西,第二批、第三批则陆续南下。先期抵达的精锐,可以先扫荡交趾边境,待到全师抵达后,便可一举攻向升龙府。”

韩冈只要能调动西军南下,至于之后的事,他就不管了。反正重行更戍法说到底只是一句口号,为自己的计划做背书而已,成事最好,不成也罢。当初就是钱粮不足、将骄士惰才停下来的。就算只是部分精锐轮戍,没个一两年的准备也是不可能的。

宰执们都离开了,只有韩冈被留了下来。

更戍法是否重行于世,当然不会在武英殿的偏殿中就有个结果,但韩冈目标已经达成。赵顼已经严令枢密院在两天内选定南下的第一批人马,并且明说此事由王韶负责。有王韶在,想必秦凤、泾原两路的精锐指挥,都会被点选出来,韩冈也不需要担心会有人给他滥竽充数。

赵顼绕着刚刚制作完成的南海周边地形图,闷不作声的踱着步子。

韩冈垂着眼帘,双手交握的收于袖中,等着赵顼的发话。

“韩卿,”过了不知多久,赵顼终于开口,“邕州之事,多亏了有卿家在,否则不知会让南国群蛮小觑中国多久。”

“当是陛下的圣德庇佑,又多亏了苏子元为人忠孝,否则如何能一举说服黄金满?若无黄金满的五千兵马,区区八百荆南精兵,对邕州只是杯水车薪而已。而若无黄金满的倒戈,李常杰也不会分兵监视其余三家广源蛮军。”韩冈很认真的说着,“邕州一战,官军都是与自缚手脚的李常杰作战,故而胜得轻易。如果出战的全都换成官军,要想击破入侵的多达八万人的交趾和广源联军,至少要一万精锐方能足用。”

韩冈的这番话,在他之前的奏疏中也说明过了,现在在天子面前,还是保持同样的态度。

赵顼很欣赏韩冈的这一点。从不夸大自己的功绩,往往还会推功于他人,甚至视功赏于粪土。也许就是无欲则刚的缘故,这样的韩冈偏偏就能屡立功勋。而有着这样的性格,韩冈也就不需要欺君罔上,凡事都能实话实说。

走了两步,赵顼又忽然叹了起来,“王雱是可惜了。有才有能,年纪又少,朕还准备日后大用。”

听到赵顼提起王雱,韩冈便是神色一黯。

赵顼回头看看韩冈,笑道:“韩卿也是年轻,比朕还要小上两岁。如今已判漕司,日后如韩太师例,刚过而立便身入两府,当也是不难。”

“臣出身寒门,乃是陛下简拔于草莽。若无陛下不次重用,臣岂有今日。”

赵顼淡然一笑。能特旨晋韩冈入官,一直都是赵顼心中的得意事。若是让韩冈直接去考进士,说不定现在连举人都难得一榜。而少了韩冈,许多事可就会是另外一幅模样了。

“韩卿可知朕一生所望?”赵顼按着放着沙盘的台面,俯视着属于他的山峦河流,自问自答,“是观兵兴灵,是收复燕云!二虏窃据中国之地,年年索要钱粮,朕忝为天下之主,却要日日受其羞辱!”

韩冈在赵顼身后躬身行礼,语调激昂:“臣虽不才,愿随陛下扫平四夷,恢复汉唐旧疆,让天下百姓安享太平盛世!”

从殿中出来时,已经是日上华灯。赵顼在武英殿的偏殿中,看着属于他的天下的时候,不由自主的说了许多雄心壮志。

收复灵夏,收复燕云,恢复汉唐旧疆,不用再给付岁币,不用再认契丹那门穷亲戚。作为天子,他有这份梦想很正常,至于能不能做得到,那是另一个问题。

不过眼下国势复振,军力大兴,离着赵顼的梦想的确是近了。

自幼追求的目标越来越近,赵顼自然不会让人或事去阻挡。所有挡在他富国强兵的梦想之前的障碍,都会被他清除掉。如果吴充在这关头上耽搁的话,他的枢密使也做到头了。

不过赵顼的话中,也似有另外一番深意。

韩冈抬头望着天上的一轮新月,自己的确是太年轻了。只不过……这又如何?!

