\section{第34章 山云迢递若有闻(12)}

出兵已经半个多月了,因为宋人据城以待,让禹臧花麻无从下嘴,而不得不走上了与宋人对峙的选择。另外,为了打击宋人的持久力,他更是派出了大队战士,去骚扰宋人的辎重队。

前几天,还有很好的消息传来。自家人在宋人的粮道上,直接做了一次剪径的小贼,抢来的物资让所有人都羡慕三分。

在一举成功的情况下,禹臧花麻盼着还有第二次、第三次,可到了今天,他所派出去的小队竟然被宋人大半歼灭。

“已经有好几队没能来得及逃回来了。宋人的骑兵在道上来回巡视,辎重队又都是捡着天光最好的时候上路,日头未落就入了军寨,缓急间下不得手!”

禹臧花麻他很明白自己的身份,不可能为了帮助木征、瞎吴叱去火中取栗,而派出太多的士兵。眼下的十几支队伍都是他想尽方法挤出来、压出来的,损失太重,他回去后也不好交代,“那就把他们都调回来。……我们继续拖着就是了,宋人绝对耗不过我们。”

“可粮草怎么办,左近诸部都有些推三阻四了。”

“等瞎吴叱回来,让他去跟武胜军这里的部族去谈,要不然就别怪我翻脸。”

禹臧花麻想借着自己手上的兵卒,为自己取得一个合适的报偿,压榨起瞎吴叱来,他已经是得心应手。不过半日之后,便有哨探匆匆传回话来:“有传言说,瞎吴叱和结吴延征已经打下了渭源堡!”

“什么?渭源堡?!”禹臧花麻乍闻消息,先是摇头不信。可很快就暗自思忖起来,难怪瞎吴叱来过一趟后就不见了,原来去了渭源堡。

“宋人的旗号呢?”他追问着。

“宋人的旗号都在城头上,好像还多了几面。”

‘嗯……’禹臧花麻沉吟着,听起来宋人真的是败了,不得不从临洮撤军。

“要不要追击?”一名部将问着。

禹臧花麻思前想后,“再等等,等木征他先动!”

可一天过去了,木征那边始终没有动静。

而这一天中,宋人已经把斥候游骑的巡视范围扩大了一倍,人数增加了不少。使得禹臧花麻派出的哨探,很难接近。而有一人,传回的消息说,道路上有很多宋军,有向东去的,也有向西来的。

情况看起来已经很明显,城中的宋人的确是在悄悄的潜离临洮,而为了掩饰这一点,王韶正拼命在外进行伪装——所谓宋军在道上东来西去,自然是障眼法而已,东撤的宋军必然要比西来的多上许多,几个来回后,临洮宋军就撤光了。

但禹臧花麻就像一只狐狸,性格狡诈、为人反复是一桩,而多疑也是他的性格之一。

虽然现在谈听到了每一条消息都是指向宋人撤军,可禹臧花麻总觉得有哪里不妥当。又想了一阵,便点起一名可靠的部众,“去联络木征,说我跟他明天一起行动,夹击宋人。”

信使走了,有人为禹臧花麻的决定而感到不安,“花麻,你真的要……”

“说说而已!又不是真做。”禹臧花麻背信弃义的回答,毫无半点愧色。

到了第二天,预定的时间已经过了,可木征并没有强渡洮水,而禹臧花麻自是从一开始就没有半点南下的意思,两边的战线依然静悄悄。

“木征为什么不来?”禹臧花麻疑惑的问着,全然没想到自己也是选择了观望。

他的疑问,在半日后被新的消息所解释。瞎吴叱被擒、结吴叱腊被杀,两千精骑被打得灰飞烟灭。

‘原来如此!’禹臧花麻似是看破了宋人的用心,

他厉声叫嚣着:“我们要跟王韶耗下去!……看宋人如何能整修得起临洮这座破城!”

……………………

木征和禹臧花麻久无动静,王韶和高遵裕皆知他们多半已经是看破了己方的计策。

“看来禹臧花麻不肯上当啊……还真是白费功夫!”高遵裕的话音有些自嘲,又隐隐多了几分难以掩饰的怨气,“不比韩玉昆,在渭源守株待兔,却当真有兔子一头撞上来。”

这本是高遵裕提出的计策,王韶并没有反对。尽管在他看来骗到人的可能性不大,不过在城中闷守,还不如让下面的士卒活动活动筋骨。

现在计划果然没能成功,高遵裕很有些失望的样子,可对王韶来说,却是能成最好,成不了为无所谓,就当练练脚力好了。

高遵裕很挂不住脸。他让下面的几千将士来来回回白跑了好几个圈子,却是连点苦劳都没能给人挣下,下面的赤佬们哪会有好话说?他在军中也有耳目。近日听说渭源屡屡见功,临洮城的将校士卒本都有些心浮气躁,现在因为自己让他们白跑了腿,私下里的怪话让高遵裕听了之后,得用力捏着虎口,才能把心头的怒气给压下去。

想出这个计划的人其实并不是高遵裕,而是他八杆子打不着的一个远亲,人称高学究,是个考不上进士和明经的村学究。听说了高遵裕到了秦凤,便跑来求个出身。高遵裕可怜他,才让他入幕中做了宾客。但他在幕中凡事都是眼高手低,好不容易出个主意,竟也是无用功。

对于让自己在麾下军中的丢了大脸的高学究,高遵裕此时分外的不待见他,直接吩咐亲兵,让他把高学究领去下面军中,还传话道:“多出巡几次,当能建功立业。”

高遵裕的满腔邪火,王韶看着神色淡然。他的这个副手在军中丢点脸,对他并不是坏事。不过见着高遵裕怒意难遏,还是出言安抚:“公绰少安毋躁,眼下的情况,禹臧花麻也坐不久了。”

高遵裕皱着眉反问:“……怎么说?”

“禹臧花麻出兵,他的军粮供给当是大半由武胜军这里的蕃部提供。可眼下少了瞎吴叱,武胜军这里又有几家蕃部会对投靠了党项的禹臧家服气的?”

王韶不愧知人善任的名声,一眼看破了武胜军未来的走响。

“木征不会让蕃部给禹臧花麻提供军粮?!”高遵裕沉声说着。木征和禹臧花麻虽不是一个路数,但唇亡齿寒的道理,他们肯定是也是懂得,木征当不会让禹臧花麻被饿跑。

“如果瞎吴叱出面说服他们不要听木征的话呢?”

“……瞎吴叱肯干吗?”

王韶嘴角一点点的挑起,笑容中带着让人不寒而栗的凶煞之气:“那就由不得他了!”

……………………

在渭源堡的随军医院中做完了手术,瞎吴叱已经脸色苍白在病床上躺了三天,犹在昏睡着,只有偶尔才会醒来片刻。两名一同被俘的亲卫一直守着他,韩冈并没有为难他们。

不过当韩冈派军医来为瞎吴叱处理伤口时,这两名亲卫就一下跳了起来,差点将在他们眼中,准备暗害瞎吴叱的军医给掐死,直到听到了韩冈之名后,方才做到了边上。

瞎吴叱被踩断的右臂已消失无踪,只有一圈圈被绑紧的绷带和浓烈的药味。如果打开绷带,可以看到创口是直接用火烙过,创面上一片炭黑,这是如今最好的解决截肢创口溃烂的手段。

粉碎性骨折不是这个时代的外科医生能够治疗的,即便在后世,当上臂臂骨被踩成碎片,又拖延了一天的时间,医生能为患者做的,多半也只剩截肢了。以瞎吴叱的伤势,能保住性命已经是万幸,韩冈说他运气,那是半点没错。

眼下,只要瞎吴叱再继续能撑过未来的几天时间,他的小命多半就算保住了。

另一个好运的王君万,仍在率领已经增加到两千上下的蕃人,在山野间搜寻残敌的踪迹。他虽说是捡了刘源的便宜,但一个活生生的瞎吴叱,就能抵得过任何人的战功。

将瞎吴叱送来的那一部蕃人,韩冈直接就从库中搬了两百匹丝绢提前赏给了他们。当汇聚在营中的几家蕃部,看到了这十几名蕃人的战马全都被高高堆起的丝绢沉甸甸的压着的时候,所有人都疯狂了。立刻向韩冈请命,准备杀入山野之中,漫山遍野的去搜寻剩下的敌军。

榜样的力量是无穷的。

韩冈几乎可以确定,很快就会有越来越多的蕃人和首级送到他的面前。

在夺下了临洮城后,已经过去了近十天的时间。从秦州征调起来的第一批民伕,现在都已上路,很快就会抵达渭源,继而向西,为修筑城池而努力。而回到陇西城中的王中正,也通知说他已经把蔡曚逼着过来。

这样的情况下,瞎吴叱的苏醒便并没有带来太大的问题,韩冈也不是很关心。可是因为王韶紧急传令,让他依此而为,让韩冈在瞎吴叱再一次醒来的时候,来到他的床前。

“瞎吴叱……”

听到有人叫着自己的名字,瞎吴叱目光仍然涣散,视线的焦点过了很长的一段时间才落到了韩冈的脸上。一见床前之人的相貌装束,他双瞳一下收紧,“……你是……”

韩冈居高临下:“韩冈。”

