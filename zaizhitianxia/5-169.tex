\section{第19章 此际风生翻离坎(上)}

“怎么会是张商英?”

蔡卞记得上次跟蔡京提起有人在城东的十三间楼楼上用千里镜偷窥甜水巷内动静的时候,蔡京对此很是有兴趣,本以为他会为此上书,但怎么一转眼间就变成了张商英出手。

蔡京瞥了兄弟一眼,“怎么就不能是张商英?”

“十三间楼的事他怎么知道的?”蔡卞疑惑道。

蔡京唇角一抹暧昧难明的笑意:“提议往十三间楼吃酒是舒亶。”

蔡卞眉峰骤起,不甘心的问道:“哥哥你就一点没瓜葛?”

“有瓜葛可就麻烦了。”蔡京摇晃着手中的酒盏,“这件事现在沾不得边,那可是大麻烦。”

张商英上本言及京城中多有人以千里镜窥人阴私并观星犯禁,这一份奏章,的确正如蔡京所说,在京城中一时间惹起了不小的风波。

夜观星象那是干犯朝廷禁令,至少一个流刑,严峻者斩,而窥人**则涉及到个人的品德问题,同时两项罪名加上来,谁手上拥有千里镜,都得掂量一下,因而而胆战心惊的为数不少,气急败坏的也为数不少。

当苏颂听到了这个消息,当即就找到韩冈:“玉昆,你看这事怎么办?”

“子容兄何出此言?朝廷自有禁令在,不当私习天文,这张商英他说得没错吧?”正在检查搜集来的药材的韩冈很是疑惑的抬起眼,又看看在外面哗哗下着的暴雨,便吩咐小吏给苏颂送杯热茶来。

苏颂一向最喜天文,最近正用着千里镜统观天象,张商英的奏章仿佛是在他的心窝里捅上一刀,正一肚子恼火,只是碍于士大夫的习气没有立即发作。眼下见到韩冈一幅事不关己的样子,当真是沉下了脸来,一下就迁怒到韩冈身上:“玉昆,难道你想置身事外?”

“什么叫想置身事外?”韩冈笑了,苏颂一向沉稳,是标准的老牌士大夫,如今急怒到这般田地,多少年也难得见一次,“我本来就是在事外。发明千里镜的不是我,发明显微镜的也不是我。我献上的只有两种透镜而已,”他指了指眼睛,“都是带在眼睛上的。张商英的奏章,自是与韩冈无关。”

“玉昆!”苏颂黑着脸,“你是在说笑吗?”

韩冈笑了一笑,抬手亲身给苏颂倒茶,“子容兄且息怒,你当真认为朝廷会以此事穷究不成?”

“那可说不准。”虽是这么说着,但神色却缓了下来。

私习天文的确是罪名没错,但天文上的禁条,那是防止有人以谶纬之术迷惑世人,防止有人藉此来叛乱。

所谓刑不上大夫,士大夫夜观天象的成百上千,在天子面前议论天文的不知凡几,苏颂他本人和沈括两人就不说了,王安石当年在赵顼面前也没少议论过天象,谁敢拿这一条太宗朝颁布的禁令来将士大夫们治罪?

真不知道张商英是疯了还是傻了。愣头青到了这个地步?也不看看现在在玩显微镜和望远镜的究竟是些什么人。最多几个倒霉鬼被拉出来当做典型罢了。

不过韩冈似乎说得极少。

“……玉昆,除了行星绕日以外,从来都不见你多言天文之事,当是估计到会有今天……”送茶进来的小吏,让苏颂的话顿了一顿,等到人出去,才又接着道,“显微镜、千里镜,想必你也是知而不作?”

韩冈笑而不答,根本就没必要开口回答。

苏颂哼了一声,低头端起茶杯,喝了两口后似乎消了火,问道:“那么玉昆对绕在木星周围的小星当是不感兴趣吧……似乎有四颗的样子。”

韩冈闻言,终究有了点反应。虽然前些日子在得知沈括发觉了土星光环之后,就知道木星的前四颗卫星迟早有一天会被发觉,但来之于苏颂,还是让人惊讶不小。

他失声而笑:“与对月亮一样……”

“原来如此,果然便如月之于地。”苏颂放声大笑:“那可要起个好名字了。”

“沈存中近日也来信说他在土星周围发觉了一圈圆环,使得土星看起来像个草帽。”韩冈悄然笑道:“他手上的千里镜,肯定是最好的那一个等级。”

“他也没耽搁啊。”苏颂感叹了一声,又骄傲起来,“不过土星外的那一圈环,其实我也看到了,跟木星外的小星一样都不难发觉。”

韩冈点点头,“可惜沈存中性子偏软,这一次的事张扬出去,他一时间肯定是不敢再在千里镜上用心了。”

“背后论人可不好。”苏颂笑道。

“那下次当着沈存中的面劝他好了。”

苏颂笑了一声,又叹了起来:“本来还想向天子建议用千里镜来改进浑天仪,但现在给张商英这么一闹,事情可就难办了。”

严令禁止对天文的研究,这是祖宗时留下来的法度,由此形成了北宋天文学和历法学严峻落后,以至不及辽国。苏颂出使辽国时,曾经因为大宋天文官对冬至的推算差了辽人一日,犯了大错,虽然他砌词搪塞过去,但大宋在天文和历法学上不及辽国,却是无法否认的现实。

“与其费神改造浑天仪,还不如先将千里镜造得更大一点。”韩冈道,“何况我可是主张宣夜说的。”

“宣夜说……这个说法偏得很,其书早亡,只在《晋书》中留了一笔,也不知道是怎么翻出来的。”

宣夜说宣称‘日月众星,自然浮生于虚空之中,其行其上,皆须气焉’。本来就只在《晋书》中出现过一次,而且是在很少有人愿意去研究的《天文志》中。

士大夫的藏书都是有限的,能看到的书也是极有限的,远远比不上后世能轻易获得的资源。张载为了寻求天地至理,才从《晋书·天文志》中里找出这个观点,寻常的士人恐怕最多也就知道宣夜说这三个字,根本不会深入了解。

“但宣夜说比之盖天和浑天,不是更近于现实?”韩冈反问。

“那倒是。”苏颂点头,观天多年,尤其是开始用千里镜来观星后,更是确定浑天和盖天两说与现实不合。“不过要改过来,可不是那么容易,尤其是现在这样的情况。”

“那就再等几年好了,反正是迟早的事。”韩冈看起来毫不挂心,“我现在倒想看看,张商英这一次张开口,到底会咬到谁人?”

……………………

秋日难得一见的暴雨下了两日后刚停歇,东京城中的河渠中,都涌动着滚滚洪流。开封府正在计点在暴雨中毁损的房屋,不过对于朝臣们来说,由于天子特旨,倒成了难得的休息日。韩冈也趁机拜访了章敦。

张商英的这一次上书,对韩冈本人并不会带来什么危害,就算是想罗织罪名,也得看看天子那里是什么想法。

但眼下最大的问题,是韩冈能够借千里镜的发明权撇清自己,但世人眼中,他与千里镜还是脱不开关系的。由此一来赵顼对偏重于自然的气学的态度,肯定又会更添上一分成见。

而且私习天文,不仅仅是谶纬异说能带来危害。天的本质被观察得越透彻,皇帝的天之元子这张画皮,也就越难披得上身子。这是动摇天子统治基础的要挟,赵顼这个皇帝到底有没有觉察到这一点,韩冈也没有多少把握。

所以张商英在这方面做文章,韩冈很想知道他到底是出于什么意图。

由于之前的事,韩冈觉得张商英的头脑有几分问题,但他也清楚张商英会这么做肯定是有其追求的利益。

对章敦,韩冈间接了当的问着:“张商英到底是怎么回事?”

“他是想争一下殿中侍御史,眼下刚好有一个空缺。”枢密副使的耳目自是要比判太常寺灵通,一句话便点出了张商英的意图,“之前他在玉昆你身上折戟沉沙,这一回便是想找补回来。”

“就为这个?!”

“殿中侍御史已经是难得的清贵要职了!玉昆你不能拿自己来比他人!”章敦无奈的叹着气,“要是再没成绩,张商英在乌台中也做不长久了。他这一回上书,多半是看到天子让人在经筵上宣讲《字说》,揣测圣意的结果。”章敦想了想,又提示韩冈,“玉昆,你最好得注意一点,天子最近说不定会下旨禁了民间持有千里镜。”

“只是如此?”

“难道玉昆你还想看到谁被下狱论法?”

韩冈笑了一笑,问道:“不知私藏千里镜,论刑是依照甲胄呢,还是依照重弩?”

“玉昆,这可不是玩笑。军国器本是当禁。到如今才禁了民间私藏,已经算晚了。”

“禁是禁不了,朝廷光是要分配军中就是成千上万,私家要制造也简单。何况越是禁,越是让人趋之若鹜,难道还能为此抄家搜检箱笼不成?……不过风声起来,肯定是一盆冷水就是了。”

章敦见韩冈脸上不见一分担心的神色,讶异的问道:“玉昆你倒是一点不在意。”

“因为在意不来啊。”韩冈笑道。也没有什么好在意的,心里说着。

