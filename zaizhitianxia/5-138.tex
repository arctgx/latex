\section{第14章 霜蹄追风尝随骠(22)}

折克仁咬着牙狞笑着。他耳朵上被辽人射出的缺口,又开始阵阵发痒。

当辽军开始越过营栅的时候,他们失败的命运便已经决定了。而发现了营栅和城墙之间一片陷马坑还不知道撤退,更是在自家的棺材上亲手钉上了钉子。

折克仁方才就是自豪的看着自己前段时间一番辛勤劳作,带来了丰厚的回报。

在半个月前,柳发川大营最外围的防线还在这一片陷马坑的后方,但等到正式修筑大营,陷马坑的北侧又修了一圈栅栏。辽人怎么也不可能想到,在营地中还会有这么多陷阱。最前面的至少百多名精锐骑兵,都在这片陷坑中折戟沉沙。

因为陷马坑的缘故,依然坚持向城中进攻的辽军骑兵,避免不了的慢了下来,而队列之间的间隔也缩短到不复存在,也便成了霹雳砲下,最好对付的牺牲品。

虽然辽人比起预计的要谨慎,进入营中的兵马,不及全军三分之一。不过辽军的主帅缺乏决断,没有在第一时间下令撤离,在霹雳砲发射之后再想走,就得付出巨大的代价。

高高飘扬在空中的飞船指引着霹雳砲投射的方向。设在营中几个制高点的上的十一架霹雳砲,并不是使用能摧城毁垣的重型石弹,而是将一包包碎石子投掷向敌军最拥挤的地方。

用绳袋包起来的碎石,或是在空中解体,继而洒落下来,或是重重的砸在地上,然后向四面八方迸射开去。无论人、马,都在如雨点般飞溅的弹雨中,被砸得遍体鳞伤。

且宋军预备下来的招待,除了石子之外,还有一个个装着延州石油的燃烧陶罐。一旦落到地上,便是一圈火焰撒开。

除了霹雳砲,由神臂弓射出的箭矢,也增添着城中辽军的混乱。相对于声势浩大、声光效果一流的霹雳砲,神臂弓虽默不做声,可收割下来的性命一点也不比霹雳砲少到哪里去。

从高处望下去,可以发现冲入城寨中的辽军已经一片混乱。号角声此起彼伏的响着,但完全看不到有秩序的行动。如同一群没头苍蝇一般,乱哄哄的躲避着头上飞来的石弹和火雨。凭着这群东撞西撞的苍蝇,想要冲出去,或是跟犹在苦力营中挣扎的叛乱者会合,完全是痴心妄想。

而城外此时也终于有了动静。谷地两侧峰峦,在几个呼吸之间,亮起了无数星火,遍布了山林,如同两条星河落入人间。

那里正是辽师后军所在。

事先埋伏在山岭中只有两支各三百人的弓手,人数不算多,随便一条小谷地就能藏起来。辽人派出来的斥候,不可能做到将所有能藏兵的谷地全都搜索一遍,要瞒过他们并不困难。但这加起来只有六百人的士兵,用来迟滞甚至阻截敌军的逃窜却是足够了。而折可大也正是带人去与伏兵会合,去攻击城外的辽军。

站在城中的高处,折克仁远眺着城外的动静。

借着天上的半轮明月,折克仁同样的混乱出现在辽军阵后。且不说来自于两侧山头上的射击。单是确认落入陷阱,就能让大多数辽人失去继续作战的勇气。

留在营栅外的辽军主力,在伏兵的攻击下,丢下了被困在营垒中的同袍手足大败而逃。而攻击他们的那一部人马,则紧咬不放,追了上去。

折克仁一见之下便变了颜色,忙招来两名亲兵,吩咐道:“快去追大郎,跟他说穷寇勿追。”

人是派出去了,可到了半夜时分,领军追击辽人的折可大才转回来。头盔拿在手中,皱着眉头边走边看。

“头盔怎么了?”折克仁问道。

折可大啐了一口,“中了几箭,把盔缨给掉了。”

折克仁再看那头盔,果然上面的红缨不见了。脸顿时就挂了下来:“不是让你小心点?!冲那么前做什么?”

“这套盔甲配面具的,就眼睛留条缝,没什么好怕的。”

“我是说这盔甲!”折克仁沉着脸,“这还是三伯当年从庞相公手上得的赏赐,千金难买,再过几年,恐怕连修都没处修了。”

折可大叫了起来:“十六叔只担心头盔?!”

“担心你是白担心!”折克仁没好气的哼了一声,“千叮咛万嘱咐不要往前冲,上了阵全都忘光了!”

折可大摸着脑袋尴尬的哈哈笑着,不敢回嘴。他的盔甲还是旧式的山文甲,正如折克仁所说,是从上代传承下来的,防御力比起当今制式的板甲要强些——虽说将领们的盔甲如今也是量身订造,不过折可大的官位还不到那个地步。

折克仁又哼了一声,问道:“跑了多少?!”

“一大半。”折可大脸色也变得不好看起来,“契丹人不好对付。”

折可大他领军追击辽人。不过辽军在逃窜之余,还不忘留下一支殿后的军队。就是那区区三五百骑,硬是将折可大带出去的两千兵马堵在路上近一个时辰,让辽军得以顺利远遁。

折克仁听了折可大的解释,叹了一口气:“契丹人的精气神果然不是西贼可比。日后镇守边陲,有得头疼。”

留下了上千具尸体,来犯的辽人狼狈的逃回了武清军。而苦力营中的叛乱,没有得到外援的支持,也顺利的被镇压了下去。

在天亮后,得知辽人惨败而退,萧海里立刻被残余的党项人给出卖,所有混入苦力营中的契丹人在火并中全数被斩杀,无一得脱。至于营中的黑山党项苦力,仅仅剩下之前的六成。不过相对于之后的工程量,这个数目也差不多足够了。

一场大战终于是结束了。

这个‘大’字不能说是很恰当,从规模上只能说得上是勉强,从防御战的角度来看,战果倒是很不少。可实际上的战斗,却完全称不上激烈,一切变化尽在预计中。当辽人主动跳下来的时候,让人无法有太多的成就感。不过看着累积起来的辽军首级,还有一堆旗帜、鼓号,折克仁还是掩不住脸上的笑意。

折可大随着折克仁从苦力营中出来。已经被镇压的党项苦力营并没有什么好看的,经过一阵好杀,又将挑起乱事的罪责归咎到契丹人身上,剩下的黑山党项都老实了不少。到完工前,应该不会再有胆子反乱了。

营外的火场,尚有袅袅余烟。夜中一场大火,是阻止苦力脱逃和引诱契丹军上钩的关键,不过囤积起来的草料、木料都被焚烧一空。一堆堆从神木寨采来的石炭,昨夜也是烧得火光接天。不过拨开表面的灰烬,下面却是烟熏火烤过后的炭块。

捻起一块黑得发亮的石炭碎片,折可大问着折克仁:“这个还能用吧?”

石炭堆得很是紧密,烧起来时,焰火连天,但烧了半夜之后火头便逐渐变小,很快又被两侧高坡上融化了的雪水给熄灭。倒是还留下了不少残余。

“烧一烧就知道了。”折克仁道,“说起来跟如今炼铁用的焦炭一样,都是闷烧过的。说不定还能用。”

“那就试试看能不能炼铁,要是能用的话,府州的铁匠铺倒是方便了。”

……………………

“耶律总管怎么了?”

从暖泉峰下回来的萧敌里,还没进大营就收到了兵败柳发川的消息。

“耶律总管不从枢密号令,不听忠言,妄自攻打宋军营寨。不但没能成功将做内应的萧海里救回来,还损兵折将,伤亡惨重。”

萧敌里和萧海里名字虽相近,但关系隔得可就远了,官位差得更远。但萧海里到底在哪里,萧敌里还是很清楚的——进驻东胜州的大军南下,就是为了呼应潜入宋人营垒中的那一队人马。但那应该仅仅是呼应,无论如何都不该变成攻打宋军营寨的结果。

萧海里陷在宋人的营寨中,最终也没能脱身。而本来仅仅是奉命伪做威胁宋军大营,呼应宋人营中萧海里起事的耶律罗汉奴,却妄自攻打宋军营垒,最后落到损兵折将。

大多数辽军尚未被宋军围困,一见战局不妙,便顶着风暴一般的箭雨,仗着快马和夜色逃之夭夭。可突入营栅的那一千多人,都没有来得及逃出生天。甚至连耶律罗汉奴本人也是重伤而归。

这个消息让萧敌里心头压下了一块巨石。

萧十三肯定不会放过耶律罗汉奴,他之前也的确是当着众将的面,叮嘱过耶律罗汉奴不要攻打宋军营寨,做做样子让萧海里可以乘势起事就可以了。

当时人人都认为这是萧十三对不情愿驻兵武清军的耶律罗汉奴的妥协,可谁能想到在出兵之后,耶律罗汉奴却一改旧意,去进攻宋军营寨了。

从听到的消息中可以得知,是耶律罗汉奴看到宋军营垒中起火生乱,有心抓住这个时机好一举建功,可惜的是,那根本就是宋人的陷阱。

这一下兵败的罪名全都得由耶律罗汉奴承担,就是煽动宋营中的黑山党项叛乱失败,也同样得由耶律罗汉奴承担。甚至萧十三还能说,他本有从宋人手中夺回旧丰州的计策,可全被冒进的南院大王亲弟给破坏了。

想到这里,萧敌里忽而遍体身寒,难道萧十三他是算准了耶律罗汉奴的性格,才派他去呼应萧海里不成?!

