\section{第42章 诡谋暗计何曾伤(二)}

【第三更,继续征集红票,收藏】

朝堂上的局势依然还处在僵持中。

由于司马光草拟的一份诏书,气得王安石上章自辩,逼得赵顼亲下手诏认错——‘诏中二语,失之详阅,今览之甚愧’——但赵顼的手诏无用,王安石依然称病不朝,一份份奏章都是求着要出外。而赵顼,也不厌其烦地下诏慰留。很快三天过去了,王安石和赵顼之间辞章和诏书往来了多次,也的确跑细了传诏的御药院都知李舜举的双腿。

也不知是不是为了激王安石出山,还是因为王安石的执拗性子让天子有了逆反心理。赵顼最近还下诏要提拔司马光为枢密副使,一张清凉伞【注1】不知多少人眼巴巴的抢着要,可司马光却拒绝了这个晋升执政的机会。

这样的情况下,韩冈往王安石递的门贴自然不会有回音。而他往流内铨呈了家状,也被告知要等上几日——对此,韩冈并不惊讶,官僚机构若是行动迅速反而奇怪了。

身在富丽甲天下的煌煌巨城之中,韩冈不是没有想过抽空逛一下东京。只不过到了东京城后,他正事还没办成一件,无论是王安石还是流内铨,让他没有那个闲心思。何况天寒地冻,万物衰败,也不是逛街的好时候。

现在韩冈每天就只是在路过时大相国寺后门往里面张望一下,顺便在路上看看御街两边有名的千步御廊,或是望一下相当于后世的游乐场、有着各式杂技、曲艺的桑家瓦子。还有最引起他兴趣的,便是天下之重心,东京之中心——大宋皇宫。而韩冈每天都要去报到的流内铨就在宫城内。

这几天,韩冈都是上午去流内铨,午后到王安石府,在两个地方报个到,顺便听个消息,有时还会想想秦州的事。

临出来时,王韶已经准备上书朝中,用一万顷未垦荒地,来为自己的在古渭建军,并屯田渭河两岸的计划背书。

那一份奏章,最多只会比自己出行迟两天。传递专折的急脚递的速度,一日一夜至少四百里,却要比韩冈来东京要快上三倍以上。如果中间不耽搁,按时间算,朝堂的回复早在自己抵达东京前,就应该回到了秦州。说不定王韶的第二份奏章,此时也已经送进了通进银台司中。

应该不会有问题,毕竟李师中自己都这么说过。韩冈放心的不再去想此事,需要关心的还是自己的事情。

除了流内铨和王安石府,以及考虑秦州之事外,一天剩下的时间,韩冈都是去张戬和程颢的府邸拜访。当然不是闲谈,而是求学。由于探明了张戬和程颢的政治倾向,韩冈便很小心的不去打听如今朝堂政局方面的消息,只是对经义上的疑难问题详加询问。

而程颢和张戬,尤其是程颢,对韩冈的好学很是喜欢,不厌其烦地向他解说释疑——监察御史的工作并不繁忙,尤其是现在新法近乎停顿的时候。张戬和程颢都多了许多时间。师者,传道授业解惑者也,程颢在这方面,做得十足十。他热心的教导,让韩冈心中都不免有些愧疚。

韩冈对儒家经义的求学,从本心上可以算得上功利。他的人生观世界观价值观早已成型,根深蒂固,极难动摇。他对儒家经典的学习,只是想将后世的学术理论融合进来。连韩冈自己都没发觉,由于自负于千年时光的差距,即便在求学中,他也免不了带着居高临下的态度看待此时的儒家学者。

但韩冈通过与程颢的来往,发现他学术宗师的地位并不是靠后世吹捧得来。程颢对一些新观点的理解很快,也没有死板守旧的顽固。韩冈的一些新奇观点,尤其是从算学的角度去解释格物致知的道理,程颢也觉得这样的想法很有意思,并细加追询。

当然,韩冈和程颢对于气在理先还是理在气先的问题,还是有着不同意见——这是门派之别。无论如何,韩冈都很难从唯物主义者转化为唯心主义。对于此,程颢都不禁摇头叹着韩冈在天地本源上的看法比张载还要偏激。

又是一天过去,韩冈从程颢家吃了晚饭回来。今天听了一天的春秋谷梁,被塞了一脑子的‘为尊者讳,敌不讳败,为亲者讳,败不讳敌’,到现在还在晕着。刚进门,驿丞迎来上来,递上来一封信,“韩官人,傍晚的时候流内铨遣人送来这封信,并说通知官人你后日铨选,让你切记,不要忘了。”

“铨试?”韩冈谢过了驿丞,疑惑着打开信封,打开一看,果然是盖了流内铨印章的公文,通知他两天后去参加铨选考试。

‘见鬼了,差遣不是定了吗,怎么还要考?’韩冈一肚子的纳闷,有官身无差遣的选人要参加铨选,但他的职司已经挂在了秦凤经略司中,还是天子亲下特旨,怎么又来了?而且上午他就在流内铨衙门中,怎么没人跟他提上一句?现在还派人送了信到驿馆,这是进士才有的排场啊。

韩冈总觉得哪里不对劲。只是既然流内铨有了这样的命令,他一个还未得官的从九品选人,却没有拒绝和申辩的余地。王安石现在不见外客,更找不到他出头,如今即便不愿,也得去流内铨走一遭。

路明放弃了科举,现在不知在盘算些什么,这些天每天都是早早的便跑出去,入夜后方才回来。而刘仲武去了三班院也还没回来。韩冈坐在驿馆外厅中,又叫了一份饭菜,方才在程颢家做客,他没好意思多吃,只能回到驿馆再补一顿——这几天也都是如此,反倒是李小六,一直跟着韩冈在外跑的他,都是在张戬和程颢家的厨房吃饭,反倒能吃得肚儿溜圆。

不过在驿馆里也有在驿馆里的好处,韩冈吃完加餐后,也不立刻回房去。就坐在外厅一角,低头喝着饭后养胃的香薷饮,一边竖着耳朵,听着周围的谈话。

城南驿中都是官人,闲聊起来话题当然离不开最近引起朝堂动荡的一桩桩大事。

“王介甫的辞章已经上到第几道了?他是不是铁了心要走?”

“走个鬼啊!也不想想官家会不会放人!”

“那可不一定,还没听说过十几封辞章上去,官家还不准的?”

“世上什么最重要?是钱啊!官家没钱,王介甫却能赚钱,这叫一拍即合。韩相公,司马君实,那是要官家节衣缩食,拍的起来?!合的起来?!”

韩冈这几天在外厅中听到的议论,都不认为王安石会真的辞职,更不会认为赵顼能同意。不同于上面的那些因为争权夺利而蒙了眼的朱紫高官,城南驿中的这等消息灵通的低品官员,因为站在圈外,反而看得更清楚。

朝堂离不开王安石,就算韩琦都动摇不了!

“但官家让司马君实草诏,去慰留王介甫,却是做岔了!”

“没错!没错!王介甫本是以退为进,可却被司马君实当头一棒,敇文写得那叫一个妙啊!”

“‘士夫沸腾,黎民骚动,乃欲委还事任,退取便安。卿之私谋,固为无憾,朕之所望,将以委谁?’你看看这话说的!”

“所以司马十二是翰林学士。你我只得混吃等死。”

哈哈一阵哄堂大笑。

韩冈也觉得赵顼让司马光去挽留政敌,实在有些没头脑。只是司马光是翰林学士带知制诰,朝中的重臣任免,都是通过翰林学士起草的。赵顼大概是看了司马光正好在眼前,而过去王、马二人又是好友,所以找他来写。但以现在司马光和王安石的关系,赵顼命他起草慰留诏书,他会怎么做根本不必多想。

司马十二的文才虽不如王安石,但毕竟是写出资治通鉴的人物。字寓褒贬的本事那是不必提的,文字上做点手脚,足以让王安石的假辞职变成真辞职。

在韩冈看来,这司马光也的确够阴。这人做的,表面上是带着嗔怪的语气在挽留,但实际上就是在挑起赵顼的怒火。

……当然,也有可能是韩冈他以小人之心度君子之腹也说不定。司马光真的是想用这种的言辞,来挽留王安石!

不过王安石回应,却表明了他是跟韩冈一个看法。而赵顼的道歉认错,也是证实了天子对司马光起草的这份诏书的理解。

厅中众人还在议论,而韩冈喝完了香薷饮,已经打算回房去了。这时,刘仲武走了进来。跟韩冈天天去流内铨一样,他也是天天往三班院跑,每天回来,如不是城外斜阳霞满西天的傍晚,便是华灯闪烁群星璀璨的深夜。

只不过前两日刘仲武回来时,脚步沉重,脸色也是一般无二的沉重,自然是没有好消息。但今天却是步履轻快,笑容也爬上了脸。

韩冈问道:“子文兄,你试射殿廷的时间定下来了?”

刘仲武笑呵呵的说道:“托官人福,就定在后天。有十几个人一起,俺也看了他们,除了一个河东来的汉子,没一个成气候的。”

“在下也是后天铨试。到时却是要与子文兄一块儿上考场了。”韩冈的笑容看不出方才的半点忧虑,却半开玩笑的恭喜刘仲武道:“在下先预祝子文兄能旗开得胜,凯旋归来。”

“承蒙吉言,也望官人能簪花而回。”刘仲武并不知道韩冈本不需要铨选,听说韩冈跟他一样收到消息,也为他感到高兴,同样开着玩笑的祝福,把韩冈当作要考进士的贡生。

韩冈笑着拱了拱手:“多谢,多谢。”

第二天,刘仲武留在驿馆内蓄养精神,而韩冈则先去流内铨确认消息,又到王安石府走了一趟,最后还是去了小甜水巷旁的程张两家,行程与前几日没有区别。只是当天夜里为了能养足精神,早早的便睡下了。

一觉醒来,便是决定韩冈一生命运的日子到了。

注1:按照宋朝惯例,官员中只有宰执才能被赐张清凉伞。

