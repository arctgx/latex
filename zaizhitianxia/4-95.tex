\section{第18章 青云为履难知足(一)}

王舜臣站在罗兀城头上。

北方的群山,秋天时漫山黄叶,冬时则是白雪皑皑,到了冰融雪消的现在,是起伏如水波的青翠欲滴的嫩绿。

可惜王舜臣不是文人,对于边塞的风物,心中都没有半点感怀伤物的触动。就算是文人,如果是日复一日的看着北方的山岭,看着山岭中随着季节变幻着颜色,恐怕也不会有多少作诗作词的心情。

王舜臣只觉得自己运气实在是背透了。

去年种谔攻打罗兀城的时候,他领军守着侧翼的抚宁城,连条党项骑兵的马尾巴都没见到。等到现在不好动手了,说不定要将罗兀城换回丰州的时候,守罗兀城的这个吃苦的活计,就落到了他的头上。这一守,就是一个冬天。

想当初他在熙河路,一直是王副枢身边的得力干将,什么时候不是杀在最前面,什么时候不是放在最重要的位置上?可当他回到了种五郎的麾下,却没有了当初在王韶麾下的风光。也只有领军翻越横山,攻打北麓银夏的时候,才被人想起。

几声嘎嘎的鸣叫从天上传下来,王舜臣面容呆滞的抬起头,一队排成人字行的大雁自他的头顶飞过。想不到这群扁毛畜生回来得倒是早,王舜臣一下有了精神。只是看清了雁群所在的高度,就又变得没精打采起来。

现如今他领军枯守在罗兀城中,张弓搭箭的机会也只有出城狩猎的时候才有。春天里的兔子倒是满山在跑,就是皮毛没用,连肉也不好吃,王舜臣本也不在乎皮毛鲜肉,只是想练练箭术,就这样还被人给劝了。

前两天突然跑过来的智缘和尚的弟子——当日只是跟在智缘身后的一个普通和尚,现在已经是紫衣大师了。也不知道他放着好端端的僧录司里的僧官不做,却往横山这边乱跑究竟是为了什么。不过因为是过去的老熟人,所以王舜臣还特意设宴好生接待了他。

之后两人闲聊起来,那秃驴先劝了自己不要拿着生灵练箭,春季万物生发,正是繁衍生殖的时候,此时杀生不祥。后面又说他现在的心情叫什么来着的,王舜臣皱眉想了一想,对了,叫‘苦逼’!

的确是苦逼。

前面他王舜臣都已经打到银州的边上了,不过是差了一口气,没一下打下来罢了。只要再多给几个指挥,将粮饷备足,用十天半个月做好准备,自己转头就能将银州打下来。

拿着银州换丰州不好吗?退个鸟兵啊!

还是因为没有将银州打下来,不但在种谔那里被鄜延路一众军头们私下里嘲笑,枢密院也下文斥责,夺了他一级官阶,又退回到供备库副使去了。而且到现在也只能守着突前太多的罗兀城。

不带这么欺负人的……

早知道就留在熙河路……不对,应该是想办法去南方。

自从王韶上调京城、而韩冈也进京任官之后,王舜臣已经在都巡检的任上做了有三年多了。从熙河路到鄜延路,都是做着都巡检,也不见往上升一升。

而调去南面的李信却是早早升做了都监。昨天从延州传来了最新的邸报,邕州大捷,他的韩三哥肯定要升官,李信也肯定要升官,要是自己也跟着去了,肯定少不了一份功劳。

李信一开始被张守约看上了,调到身边做亲兵,比自己要迟了近一年才得官。可如今荆南、广南两战下来,都是天下有数的名将了,说不定什么时候,就做了一路副总管了。

这际遇当真是说不准,并不是先走一步的,能一直走在最前面。

在城头上发了半天愣,王舜臣算是完成了一天的巡城工作,转身下了城后,就直接就晃到了城南角落里的妓馆去。

只要有钱赚,商人也好,妓女也好,都不会缺的。别说延州、绥德这样的大城,就是这座刚刚收复不就的罗兀城,里面塞了两千多有力没出使、有钱没处花的精壮汉子后,跑来赚份快钱的妓女为数不少,,总能保持着三五十人数量。

对于边地军城中出现的这些妓户,军中上下都是睁一只眼闭一只眼的。又没个战事,不能阴阳调和,士卒们郁积在身体里的火气,可就要往上跑了。城中的几家私窠子里面,

不过王舜臣身为主将,不便跟下面的士兵进出同一条路,另有好一点、当然也是贵一点的去处。

“……左牵黄,右擎苍,锦帽貂裘,千骑卷平冈……”

刚刚走进巷中一座三进的院子,就听着从里面传出了歌声。虽然唱得不算多好,但声音是少女的清越,比起城中几座私窠子里的那些老妓来,只听声音就要好了不知多少。

“……为报倾城随太守,亲射虎,看孙郎!”

这歌词王舜臣听着感觉不差,只是用着江城子的温婉调子,唱起来词和曲完全不配。也不让人通报,王舜臣直接跨进房中。

一名十七八岁,相貌只能算是清秀的小妓正引吭高歌:“酒酣胸胆尚开张,鬓微霜,又何妨……”

只是王舜臣进门,歌声立刻就停了,房中的几人都将视线投了过来。

唱曲儿是一个,另一个则是三十多岁,虽说徐娘半老却是风韵犹存,人称李四娘,另外还有两名左右服侍的小婢。而身在房中的客人,则是天子放在罗兀城的耳目,延州的几位走马承受中的一人。下面缺了物件,他来妓院,也就听着小曲儿,做不了其他的事。

“哦,是都巡来了……”

随着歌声停了,那名走马大步走过来相迎。身高体健,黝黑的面容甚为英武,颌下还长着十几根胡子,如果不知道身份,他和王舜臣两人站在一起,说不定他还更像一名大将。

王舜臣跟随种谔出兵的时候见过这名走马一面,同时做了随军的监军。到了罗兀城后,接触了就更多了。为人、性格都颇让人看得上眼,看起来也是个会接交人的四海性子,也不知是得罪了哪路神仙,巴巴挣到了随军的机会,却跟自己一样,被放到了罗兀城来。

“供奉今天也是有雅兴?”

“也是闲来无事,故而到此一游。”

王舜臣与延州走马已经是很熟了,也不多客气,互相行了礼,就一起坐下来。李四娘让人很快的送了酒菜上来。

王舜臣低头一看杯中的酒水,立刻就皱起眉来:“童供奉来了,怎么不上好酒,上次的玉照白呢?”

“都巡有所不知,从延州带来的几坛早就没了。现在的酒在罗兀城中,已经是一等一的好酒了。”李四娘轻蹙着眉头叹气,话声中听来多有几分委屈。

“四娘可说错了,”延州走马大笑出声,“罗兀城也有好酒的,就看都巡舍不舍得了。”

李四娘几人一起看向王舜臣,而王舜臣则反问着:“此话怎么说?”

走马笑道:“昨天进城的几辆车马,难道不是从熙河路来的?”

王舜臣摇头叹了口气,也笑道:“就知道瞒不过供奉……来人,将昨天放在地窖里的烧刀子拿一坛过来。”

守在门外的亲兵在门外应了一声,脚步哒哒的就走得远了。

“可是韩舍人所创的烧刀子?”小妓好奇的问道,“不是说那等烈酒阳气过重,饮则伤身吗?”

“怕什么,提刀上阵不照样要拼命,喝点烈酒又算什么?!”王舜臣满不在乎。

走马承受道:“韩舍人学究天人,他的吩咐还是要听着。还是得少喝几杯。”

王舜臣点点头,对身高体健的走马道,“说起来前些天三哥的信里提过供奉,说曾跟供奉里见过好几次。”

“韩舍人竟然还记得童贯?!”做了延州走马承受的童贯心中满是惊喜,声音都变得尖细了起来。

“怎么不记得?”王舜臣笑道:“供奉不是好几次都是带着喜报去见三哥的吗?”

童贯喜不自禁的连连点头,“这是童贯的运气。”

说了两句闲话,王舜臣转头问着小妓:“方才唱的谁做的曲子?怎么没听过。”

“是知密州的苏子瞻苏修撰。”

苏子瞻……王舜臣脑中回想着,他似乎听说过这个名字,好象是在诗词歌赋和文章上的名气很大。不过他跟韩冈交好,韩冈诗赋都不上手,所以王舜臣也不认为会做诗词歌赋有什么了不起。作词做得再强也不过是个柳屯田,能如欧阳公那般,诗赋出色,做官也能做到执政的,多少年也不见得能出一个。

“都巡怎么不知道。”对王舜臣的平静,小妓似乎很惊讶,“‘十年生死两茫茫’可曾听过?去岁一出,就遍传天下。”说着一对眸子变得闪亮亮,满是憧憬。

什么‘生死两茫茫’,王舜臣更是不知道了,神色中就有了不愉。

李四娘虽说已是三十多岁的老妓,不比年轻时受欢迎,但待人接物、察言观色的能力倒是越见老辣,一见王舜臣似乎有些恼火了,连忙笑道,“苏修撰与韩舍人可是有些来往,都巡当还记得韩舍人家的花魁……”

“啊!”王舜臣一拍桌子,终于想起来了,这件事他哪里会忘掉,那可是得罪了天子的亲弟弟啊,“原来是他。想不到他诗词还做得。”

