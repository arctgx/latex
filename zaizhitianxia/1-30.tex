\section{第15章 三箭出奇绝后患(上)}

“应该就是今天了吧?”

“就是今天!”

淡淡的檀香缠绕在鼻端,不过空气中弥漫的则更多的是满桌佳肴的香气。只是坐在厢房中的两人哑谜般的对话并不应景,每个字中都透着浓烈的杀机。

秦州城中素斋做得最好的天宁寺的香火,虽比不上妙胜院【今南廓寺】这样在鸿胪寺左右街僧录司【注1】挂上名的大丛林,但胜在清雅,有闹中取静的味道,又拥有一座名气甚大的菊园,每逢入秋,秦州城的达官贵人们多喜来此处赏菊喝酒。

不仅如今已经入冬,素斋在西北的冬天并不受欢迎,来到天宁院的官人们几乎绝迹,只有喜欢口腹之欲的陈举常常来光顾,施舍的香油钱亦不在少数。

陈举用勺子舀了块酿豆腐吞入口中,半眯着眼享受起在嘴里扩散开来的滑腻细软的美味。天宁寺的豆腐细嫩的异乎寻常,还没有平常豆腐犯苦的卤水味,这是天宁寺的独门秘方,没人知道究竟是怎么做出来的,是让陈举百吃不厌的一道菜肴。

刘显坐在陈举对面,他的碗筷都还没有动过:“按着行程,如果没有拖延的话,韩冈现在应该已经出了夕阳镇,往裴峡谷去了。”

“不知末星部能不能成功……”

刘显轻松的笑道:“去埋伏的都是十里挑一的精锐,韩冈手下不过三十多民伕,又有薛廿八和董超做内应。就算王舜臣是个能打的,被几倍的精兵一围,他一人又能抵得多少事?”

以末星部的实力,八九百兵也勉强能动员得出来。但这么多人一起出动动静太大,为了防止走漏风声,百人便是极限。从近千人中精挑细选出来的百名精锐,怎么可能会输给不到半数的民伕?!

“也得防着万一啊……”与蕃人打得交道越多,陈举就越是明白他们不能深信,怎么都要防着一手。

“有齐独眼在,就算能到甘谷,韩冈也绝逃不过一死。算时间,今天小七也该到了甘谷,有他知会着齐独眼,押司何须忧心。”

陈举慢慢的点了点头,对于自己安排的记记杀招,他相信韩冈不可能都躲过去,只要中了一个,他必死无疑,唯一担心的就是他半路跑掉,“韩冈的父母逃到了凤翔府去,说不定他也会逃。”

陈举说着放下筷子,拿起酒杯,刘显见了忙提起酒壶给陈举满上,笑道:“四郎也是在凤翔呢……如果韩冈潜逃,他的父母肯定要下狱,四郎正好可以插上一把手。”

“他把官做好就够了。斩草除根我自会安排人去做!”

陈举是个吏员,祖孙三代在成纪县衙中作威作福。如此权势,陈举当然想传给儿子。他总共生了八个儿子,但活下来的就只有三个——在此时,无论民间还是皇家,幼儿夭折率都是超过一半,很少有韩家那样三个儿子有养到成年——

陈举的幺子今年刚满八岁,而老二、老四则都已成年。他的次子陈缉如今也在成纪县衙之中做事,前些时候领了差事往京兆府办事去了。至于四子陈络,陈举很早就决定不让他留在成纪县中与长子打擂台,而是花钱为他捐了一个官身,如今是在凤翔府下面的县里做着监酒税的小官。

陈举为儿子买来的官身称为进纳官。虽然进纳官在官场上多受人鄙视,很难升得上去,可有了一个官身,能减了税赋,免了差役,行事也方便一些。就如陈举已经病死了的二弟,也曾经捐过一个官,帮着家里减去赋税。

“只要韩冈死了,只要他一家死绝,谅也没人再敢来捋押司你的虎须。”

陈举一仰脖,将水酒一饮而尽。放下酒杯,眯起的眼中杀气腾腾,攥紧右手的力道几乎要将酒杯捏碎。

自从军器库一案之后,他在成纪县中的威信大落。他过去使人办事,从来不会有二话;但如今,有许多都是被拖着的。

这是谁害的?

是韩冈!

为了填窟窿、弥补后患,他几万贯花了出去,家中现钱一下全没了,商号差点周转不过来,接连卖了几片好地和宅院才弥补了亏空。

这是谁害的?

是韩冈!

财不露白,但多少官吏看着眼红,每天晚上他都是辗转反侧到三更天后,才朦朦胧胧的睡过去,往往还在噩梦中一身冷汗的醒来。

这是谁害得?

还是韩冈!

韩冈不死,如何心安?

“只要韩冈死了!”陈举恶狠狠地说着。

是的,只要韩冈死了……

……………………

“要本官帮你家押司杀了成纪县来的衙前?……这韩冈是哪里来的人物?究竟是怎么得罪了陈举?”

甘谷城的公厅中,一名身着青袍的中年官员带着一丝玩味的语气出言问着。齐独眼——这是中年官员的绰号,齐隽才是他的本名。齐隽两只眼睛都睁着,左右双眼分不出孰真孰假,只是在他左眼中还能找到一点慈悲,而右眼里就只剩下冷漠和无情【注2】。

甘谷城监理库房大小事务的管勾官——扒皮抽筋齐独眼,在秦州也是鼎鼎大名。落到他手上的衙前从没有一个能安安生生的回家复命,都是倾家荡产,才能喂饱这头磨牙吮血的独眼恶狼。看他不顺眼的人很多,据说秦凤兵马都监兼甘谷知城的张守约也一样,但齐隽只跟衙前过不去,从不在军资上动手脚,本身又属于文官,张守约也没理由找他麻烦。

在齐隽面前,一个风尘仆仆的高壮青年低头回着话:“回官人,押司今次让小的来甘谷拜会官人,就只让小的带了这么一句话。”

齐隽迷起眼睛,声音冷了下去,“黎清,这是你家押司求人的态度?”

“押司说了,官人与他是兄弟一般的至亲,要小的在官人面前小心伺候着。只是押司没吩咐的事,小的也不敢乱说。”黎清的态度恭恭敬敬,却拒绝得毫无余地。

齐隽冷哼一声,知道在黎清嘴里问不出什么来。能让陈举派出来,肯定深得信重,黎清这等干仆必定都是家生子,至少从父母开始就是在陈家做事,这样的身份,当然不会随随便便泄露主子的隐秘。

他信手拿起黎清送到自己案头上的一个沉甸甸的盒子,打开了一条缝瞟了一眼,嘴角似笑非笑的扯动了一下,右眼中的冷漠当即褪去了不少,声音也和气了起来:“如今甘谷情势不妙,亏你也能进得城来。”

“为了押司奔走,一点小事算不得什么。”黎清低头轻声说着。

“小事?!”齐隽哈哈笑了两声,笑声很干,很快就收止。看起来有些忧心的样子,“已经不小了……”

“管勾……”一名胥吏突然出现在门外。

“怎么了?”齐隽问道。

“启禀管勾,上个月陇城县来的那名衙前死了,从伤病营抬了回来,还请管勾先查验了,好拿去烧掉。”

“才死啊,还真是能拖……”齐隽摇着头,似是不满的样子。他说着就走到门外,黎清也跟了上去。

就在院子中,摊着一具青年男子的尸体,一张芦席就铺在下面,显是就是用着芦席裹着进来的。也许是因为冬天的缘故,尸体并没有腐烂,但莫名而来的浓浓尸臭却传遍整个院子。透过裹在尸身上的破碎凌乱的布料,能看到下面几乎没有一块完整的皮肤,或青红、或紫黑,触目惊心,甚为可怖。

尸体的面部如鼻子、耳朵还有面颊上,缺了不少皮肉,甚至能看到下面的骨头,黎清猜着可能是给老鼠啃了去,而且看这些缺口处都有血渍凝成的紫黑色,甚至应是人还活着的时候就被老鼠咬的。

“喏,这就是上个月从陇城来甘谷的衙前。”齐隽用着一块熏香后的手巾捂着口鼻,一手还指着向黎清介绍着尸体的身份,“这个给脸不要脸的腌臜泼皮,押运路上弄了多少亏空下来。让他弥缝上,他却死咬着不肯答应。本官也懒怠与他废话,先敲断了腿,直接丢到伤病营中去。”

他抬脚踢了踢尸体,把尸身两条腿上的伤口露了出来。那里已经被老鼠啃了个干净,白森森的骨头只挂了点血丝在上面,“若是在夏天,伤口生了蛆几天就能咽气,不过如今入了冬,竟让他拖了半个月去,害本官等了那么长时间。”

齐隽的口气平淡得如同弄死了一只鸡、一条狗,混没把人命放在眼里,黎清听着心生寒气。他也是在陈举手下老做事的,凶悍狠戾的人物见过不少,但齐隽这般身体力行着众生平等的性子,他毕生也只在陈举身上见过。

齐隽挥挥手,示意下面的人将尸体抬出去,回过身对黎清道:“如今甘谷城出去也难,你且在这里等两天,只要韩冈到了,那就是煮熟的鸭子,别想跑出锅去!”

黎清木讷的脸上多了点笑意,跪倒磕头,大礼致谢:“多谢齐官人!”

注1:鸿胪寺属于三省六部九寺中的九寺之一,是古代国家中枢部门。归于其下的左右街僧录司则是统管天下寺院僧尼的机构。

注2:据《南村辍耕录》所载,宋时“杭州张存,幼患一目,时称张瞎子,忽遇巧匠,为之安一磁眼障蔽于上,人皆不能辨其伪。”由此可见,在宋时已经出现了瓷质义眼。

ps:敌人一个接一个跳出来,韩冈的性命危如累卵,欲观后事如何,请看下回分解。

第三更,求红票,收藏。

