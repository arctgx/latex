\section{第19章 萧萧马鸣乱真伪(六)}

从宋用臣的手上,王安石将奏报接过来匆匆看了一遍,抬起头,正对上赵顼惶急的眼神:“陛下,这是走马承受的奏报,并不是郭逵的。郭逵能发急脚递,比走马奏报要更快一步。既然走马传了奏报,为何河东经略司没有传信回来?”

赵顼沉吟了一下,“王卿的意思是说郭逵不认为那些是辽军骑兵?”

“至少是没有确定。”王安石很肯定的说着,“是否有辽骑在丰州,此事郭逵尚未探明,怎么敢妄报于朝堂,只能等查探明白再行奏上。”

“那依王卿所见,丰州的辽军是真还是假?”

“辽国嫁了公主与秉常,此前又曾意欲强逼陛下以罗兀、绥德交换丰州。西夏持之以为依仗。但要说辽国会为西夏火中取栗,却是难说。辽人嗜利,我有每年五十万银绢与辽人,而西夏国势日蹙,又从何处得来钱财,交予辽人?”

“可契丹骑兵又是从何而来?”赵顼心中疑惑难解,“党项人也不会刻意准备一批骑兵以充今日之用。”

王安石想了一阵,道:“西夏邻接辽国上京道。想那上京道中的阻卜等部,习俗类与契丹、装束类似契丹,如若冒充契丹人,也只需略略改一下装束。”

王安石如此断言,韩冈先是一阵讶异,随即便心中了然。王安石这是在安慰天子,其实心里面并不如嘴上这般确信。不过就算是错了,以他的身份地位也能抵得下来。

在韩冈看来,丰州的辽国骑兵不论是真是假,郭逵既然没有派人回来,究其本意还是准备杀过去,先试一试成色如何。

韩冈对郭逵所率领的河东军深具信心,虽然此前在熙宁四年的横山会战中,就是因为河东军的当先崩溃,才导致了整个战局的逆转。不过那是韩绛胡乱下令的结果,实际上的河东军战斗力并不弱。北面是辽国西京道,而西面又是西夏,同时还是同时肩负支援河北、关西的任务,河东路中禁军和乡兵都能算得上是出色,至少比起久未上阵的河北军要强。

如果河东军在此战中表现得足够出色,而且还击败了契丹军,也可让天子和朝堂对官军更多一点信心。不用像现在这样,必须要王安石来劝慰。

只是王安石能挥霍着自己的政治资本来安慰天子,但韩冈却不能,且河东的事务,并不处在他熟悉的范畴,他也没必要多话,干脆闭口不言。

不过事情不是那么简单,赵顼不会放过韩冈这么好的参谋对象。他的才智早已得到证明,眼光也同样超人一等,正所谓识见过人。既然韩冈就在殿中,当然少不了询问一声。

“韩卿,你道丰州的契丹军真伪如何?”

韩冈偏头看了一眼宋用臣,躬身行礼道:“此事臣未明就里,不敢妄言。”

赵顼会意,道:“宋用臣。”

今天在赵顼身边值日的内侍,连忙将王安石刚刚交还回来的奏报,转过来又递给韩冈。

韩冈匆匆看了一遍,又揣摩了一下其中的措辞,也算是对郭逵的心思了解了个大概,“臣的看法一如丞相,辽人贪好财帛,西贼穷寇,当不致为其奔走卖命,总有些许,也是逐利而已,不会死战。且河东走马既然回报此事,郭逵如何不知?只是不敢妄下定论。然郭逵宿将,既知契丹骑兵可能援夏,必然会有所准备,陛下可以无忧。只不过京城之中也需要早作准备,不能等到最后才匆匆忙忙调集大军。”

韩冈长长的一段话,其实都是些不落口实的闲言赘语,没有太多意义。但他说到最后一句,却怔了一下,仿佛想到了什么。虽然很快就恢复正常,但脸色还是变了一点。

赵顼和王安石都没有看出韩冈脸色的变化,都是在想着韩冈的一番话。哪里会想得到,他现在的心中正在破口大骂。

这份奏报来的太不是时候了,只一下子,河北军就不能动了!

虽然天子和朝廷不会下诏让丰州前线的大军返回——从时间上看,此时河东军的前锋应该已经攻进了丰州境内,很有可能已经开始接战,两军纠缠的过程中,一旦撤退,结果就是惨败,根本撤退不得——但让河北军在防备辽军南下的同时,准备救援河东,这都是必须会颁下的诏令。而没有河东军的填补,第二批第三批的西军就不可能南下广西,也就是说,安南行营能依靠的只有刚刚入蜀,准备顺水直放邕州的那五千兵马。

这仗可没法儿打了。韩冈想想,心中便又暗暗摇头。已经是不能不打了,箭在弦上不得不发。一旦在邕州待得日子长了,西军兵将罹患疾病的几率会越来越大,而士气也会低落得厉害,到最后就连三十六峒蛮部、以及广源州甚至脚趾国内有心投效的部族,都会犹豫起来,甚至再倒回去。那时候,再想动手可就难了。

从殿中出来,已是将及黄昏。王安石还留于殿中,与赵顼讨论着之后的应对——想来不外乎镇之以静之类的计划,还有加强河北、河东防御的方案。虽然还没有涉及南调的安南行营所部,但等到两府八座都到齐了,却是不可能不提的。

‘也就在明天了。’韩冈想着。为防京城骚动,即便是有关契丹军的消息,但今天的夜里是不会紧急召集宰执议事,一切都会等到明天。而到了明天,就在崇政殿为此而争论的时候,自己则是已经整装南下,出了南面的城门了。

如果朝堂上的结果一如自己所料,那么安南行营还想继续任务,就必须凭着眼下的兵力,去踏破升龙府的城门。

已经南下的五千,加上一千五百的荆南军,以及刚刚招募编组的广西新军——除去驻扎的兵力,能调动出战的在三四千左右——总计一万人。这就是安南行营能动用的全部军队了。

恐怕唯一的好处就是自己的工作轻松了,一万兵马的粮秣转运,要比之前的工作量少去大半,而需要征调的民夫也少了许多。

当然,这一切的前提是继续进攻交趾,也许当自己南下的时候,朝廷的一封诏令,就让安南经略招讨司和安南行营立刻停止一切动作,就此偃旗息鼓。

韩冈回头望着巍峨的皇城,冷哼一声,他可不会让这件事发生。

回到家中,只有严素心出来迎接,王旖尚在孕中,周南、云娘都是刚刚分娩,连床都不能下。

就在堂屋中,贺礼都堆了整整一屋子。家里的下人正对照着礼单和礼物,并一一登记造册。看他们的速度,今天晚上就别想睡觉了。

转运使、龙图直学士加上对未来宰执的期许,让韩冈在京城中受到超乎寻常的看重。平日经常宾客盈门,幸好今天他因为明天就要启程而闭门谢客,否则还不知道会有多少人登门拜访。

到了后院,韩冈先去换了衣服问候父母,三名儿女都在他们的房中。家里尽是产妇孕妇,不能让他们这几位小祖宗去打扰。

出来后就去韩云娘和周南的房间。周南是二胎,生得很顺利。而云娘则是头胎,稍稍有了些波折,不过也算是顺产。

房间中并没有太多人,人气一杂,对产妇和新生儿都不利。周南正在睡着,脸色有些苍白,韩冈问了问旁边的墨文,知道之前起来喝了些药粥,也放心了一点。包在襁褓中的三儿子脸还是皱皱的,并没有睁开眼睛。

从周南房中出来,韩冈去看了韩云娘。却是醒着的,正抱着刚刚生下来的幼子。形容间还有着少女的稚嫩,但看着儿子的时候,韩云娘的眼神已经是一名真正的母亲。

“三哥哥!”听到韩冈进门的动静,韩云娘瞬息间浮现在脸上的喜悦如同鲜花般绽放。

“今天可还好一点了?”韩冈在床边坐下,搂着纤瘦的肩膀看着她怀里的儿子。六斤重的小子,比他的几个兄姐都要重,生出来时让韩云娘吃足了苦头,差点就用上了产钳。

就在韩冈的怀中乖顺的点着头,“已经好多了。就是身上都是汗,现在还是难受。”

产妇产后调养,韩冈不算了解,但还是听说过一些,比如今的风俗许多地方更为合理。之前妻妾们的几次生产,也都进行了印证,如何照顾产妇,韩家也算是有经验了。

“过两天就能洗澡了,再稍微忍耐两天。不过只能淋浴,不能盆浴。还要注意不要让你们娘子受风。”韩冈吩咐着服侍韩云娘的下人。

家中也有设有淋浴房,不过因为是租来的官产,不好随意改建,只能依旧使用旧式的淋浴设施。只是特意为了家里的孕妇们布置了一下,变得保暖遮风。

其实用来烧开水、提供洗浴的简易锅炉,则早已在各地疗养院中推广,最近上四军的军营也已经设置。说起来就是将外面公共浴室所用的锅灶炉膛稍加改造而已,甜水巷的浴室院提供的服务并不比后世的洗浴稍逊。

说了阵话,见怀里的佳人犯了困,韩冈扶着她躺了下来。云娘张着眼眶微凹的秀美双眸,眼中尽是不舍的情丝,“三哥哥,你明天就要走了?”

“是明天才走,今天还在家里。”韩冈的手在云娘细腻的脸颊上轻轻拂过,“乖,先睡会儿,起来后再跟三哥哥说话。”

云娘依顺的闭上了眼睛,却抓着韩冈的袖子没有放开手。韩冈笑着坐好,握住云娘伸出来的纤细的小手,看着她渐渐睡去。

