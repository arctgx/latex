\section{第47章 百战功成朝天阙(中)}

【最怕双休日,总是有事耽搁码字。今天的第二更要到一点以后,各位兄弟不要等了,等明天再看吧。】

何去何从?

韩冈神色变得微妙起来,王韶这话问得很有意思。

他下一步的走向,早就已经确定,王韶不会不知道。韩冈他早早的就跟人说过了,河州之战结束后,接下来就是锁厅参加科举,混一个进士头衔出来。

王韶是要走,但韩冈走得只会更早。八月在秦凤路中的锁厅试得到贡生资格,明年——也就是熙宁六年——的二月参加科举,接着是发榜、然后金明池赐宴,之后审官东院才会重新决定他的任官地点——选人的任官由流内铨处理,而韩冈已经是朝官,当归入审官东院治下。

就算会被安排回熙河,也要等到这一套程序走完之后。如果没中进士,同样也要等到发榜之后。可以确定的是,至少到明年四月以前,他都不会再回熙河。

——除非要他放弃参加科举。

“朝廷用人之法的确是有待商榷……明年的举试之后,韩冈若还有重回熙河的机会,自当设法让接替之人不至于败坏国事。”

韩冈与王韶关系紧密,云山雾绕的话,他不会拿出来糊弄人,而是很明确的告诉王韶,‘如果是要我放弃科举,那就不要说了。’

放弃明年的科举,放弃他唯一可能得到进士头衔的机会,韩冈是绝不会答应。

刚刚改换的考试科目,将所有擅长诗赋的士子,拉到了与韩冈水平相当的同一条起跑线上,甚至更低。熙宁六年这一科中,原本会属于南方士子的进士名额,也将会大幅度的偏向更擅长经义的北方士人,当然,也更适合在经义策问上用心了三年之久的韩冈。如果拖到了熙宁九年,他就要跟已经适应了新科目的贡生们竞争,折戟沉沙的可能将会千百倍的增加。

同时这一科的考官,必然是新党中坚。章惇最近要出外,但曾布,还有即将结束丁忧的吕惠卿,都有可能成为主考官中的一人。以他与新党的关系,得到考题虽不现实,但大方向却能确定。而且跟新党众臣结交的过程中,他更可以让吕惠卿、曾布来熟悉自己的文风、思路……以及用词习惯。

但下一科就不一定了。韩冈没有把握到四年后,新党还能留在台上——变法最终是失败的,从他所知的历史中可以确定——若他不能成为进士,就没有机会干预朝局,更不可能改变新党失败的命运。

韩冈有足够的自知之明,他可不是章惇,想考进士就能考中进士。除去不搭边的地利,若是没有天时、人和的帮助,韩冈自问没有机会能跨马游街。

‘我不可能放弃的!’

王韶看到了韩冈眼神中的坚定,情知是难以说服。

换作是他本人,恐怕也是两难的选择。如果仅仅是要成为朝中重臣,以韩冈的才能,有没有一个进士头衔并不重要。但日后要想在宰执班中得到一个位置,进士出身就会很关键了。

而从眼下的情况看,韩冈成为宰执的机会很大——他年龄上的优势实在太大了。为了日后的前途着想,韩冈拒绝的理由当然十分的充分。

叹了口气,变得默然不语。

王韶担心来接任的官员会坏事,希望韩冈能放弃科举。韩冈虽然拒绝的毫无余地,但他也不想让王韶太难堪,也觉得至少要点醒一下把河湟看得太重的王韶。

“今次经略翻越露骨山,近四十天渺无音讯。不知经略可知为何朝廷是直接下令河州撤军,而不是选调得力人选,来暂任熙河经略一职……以保住河州?”

韩冈的问题,王韶如何会想不明白,这是官场上的通病:“如果来人只是保着河州,功劳最后多还是我的,日后有人提及河湟,之会先想起我。不过若是丢了河州后,再有人领兵攻下来,功劳可就是他自己的了。朝中诸公都在等河州陷落,谁又会为我来自蹈险地……”

他说到这里,突然觉得不对劲了,抬眼一瞪韩冈,一双眸子突然变得锋锐如枪。

韩冈不动声色:“巩州如今已经能自给自足,马市中一年还有上千匹马的收入——前两年都是一年增长一倍——今年如果没有这次的大战,多半就能涨到两千。狄道城有南关堡、北关堡护持,北关堡以北,还有临洮堡、结河川堡,这数堡之间,乃是洮水中段最好的一段河谷地,最少也能容纳上万户屯垦。还有岷州的钱监,年初的时候就已经开始出钱了。”

王韶双眉越凑越近,韩冈的口气分明就是在说,只要保着巩州、熙州核心的洮水河谷,还有拥有钱监和铁矿的岷州,至于其他地方,丢了也无所谓——包括刚刚打下来的河州、洮州。

“……玉昆,你可知这几年来,我在河湟之地,付出了多少心血?”王韶的声音中,竭力压抑着自己的怒意。

“韩冈久随经略。经略在熙河用心之深,韩冈看得很清楚……但大势如此,正如洪水破堤,还是不要顶着潮头为上。”

韩冈的性格更偏重于理性,对于螳臂挡车的行为,丝毫没有兴趣。如飞蛾扑火一般,向熙河蜂拥而来的热情现在根本堵不住——参加了河湟拓边的官员们的升官速度实在太快了。

王韶就不提了,韩冈从布衣升朝官则更是一个奇迹。要知道,仁宗皇佑年间的进士到现在还有一大半没有转官,英宗的进士转官的人数还要少,更别提当今天子即位后的进士了。熙宁三年的进士,除了状元叶祖洽一开始就被授予京官,后面的二、三名榜眼也要一任后才有机会,至少两年,也就是今年才能转官——而且必须有着很好的表现,路中监司又有高官推荐。

而韩冈也是熙宁三年得官,才两年过去,现在就已经是太子中允了。并且攻取河州的功劳还没计入,一旦最后论功,就算有人拿着他的年龄和资历说话,就算他并没有追击木征的功劳,至少也要连升两级。

文人最大的毛病就是看不清自己,而喜欢贬低别人。韩冈知道,认为自己比一个灌园小儿要强的,也不知有多少。就算在张载的门下,也有不少人都只是嫉妒着韩冈的好运,而看不起他的才学——游师雄和种建中在给韩冈的信中,都遮遮掩掩的提到了此事。

既然如此,就让他们来试试看好了。是骡子是马,拿出来遛遛。当成果换成了功劳,那就已经成了过去,只要保住其中的核心利益,至于其他,由着让人去败家。真的闹大了,坏了国事,反而就是自己的机会了。

这个道理和手段,王韶不是想不到——韩冈一说,他就明白了——但是他关心太甚,不比韩冈这般能放得开。

“玉昆,你……”

看着王韶要驳斥,韩冈立刻抢先一步追加了一句:“如果经略去问处道,他的回答当也是跟韩冈一样。”

“二哥也是……”

见着韩冈平静如水的神情,王韶知道,他不会在这个问题上骗人。知子莫若父,儿子王厚的性格王韶也明白,想来当是跟韩冈一个想法。

摇了摇头,看来自己真的老了。

………………

从王韶那里告辞出来,大堂中的酒宴仍未停息,看起来要闹到通宵达旦的样子。

避过两个出来吹风,歪歪倒倒站不直腰的醉鬼,韩冈往自己的小院中走去。

跟王韶的对话还在脑海中回想着,反复想了两遍,自问没有会让王韶与自家翻脸的地方。要骂也是先骂他的儿子去。王韶没回来的这段时间,王厚和韩冈的往来信件中,都已经准备好应对河州撤军后的局面,当时就在说只要保住巩州、岷州和熙州的洮水河谷,其他任由朝廷来人折腾。

不过其实那只是最坏的情况,如今河湟之地,在木征就擒后,就只剩个董毡。且董毡已是孤掌难鸣,即便联络党项人,也无力对抗已经在河湟拥有了巨大优势的宋军。即便换个好大喜功的主帅,也不过吃点亏,丢个一两个寨子而已,大势是改不了的。

回到自己的小院,韩冈先向着东侧望了一望,只有两盏孤灯挂在不远处的另一座院子门口。那是蔡曚落脚的地方,虽不是故意安排的,但冤家对头住的对门,的确很有些黑色幽默的味道。

这个废物,将后勤弄得一团糟,河州的苗授和二姚兄弟都跳脚了,若不是韩冈安排在珂诺堡中的一些存粮,他们就只能靠剥削河州蕃部来过活。

现在王韶回到了狄道城中,蔡曚便乘势称了病,他造成的混乱还没有带来太严重的后果,最算责罚也不会太重,多半还是被调离秦凤转运司。如果丢人现眼的事不算,说起运气,蔡曚也不算差了。

韩冈幸灾乐祸的笑了一笑,就把此人彻底丢到了脑后。推门进院,在摆放着一部部书卷的桌前坐下,重新又开始了今日被耽搁的功课。

读书,习文,韩冈的精力全都放在了即将开始的科举上。

眼下就该等京中的消息传回来了。

