\section{第42章 皇祚思无疆(下)}

放下喝空了的瓷盅,赵顼接过王中正奉上来的丝巾擦了擦唇角边的药液。

从玉辂下来,他就感觉着身体不适,只是喝过了随行御医所开的药汤之后,发了汗,感觉才好了一些。

王中正忧心的看着在烛光下,脸色依然显得苍白的皇帝,“官家,要不要再诏两名御医来看一看?”

“刘方明已经是随驾而来的最好的御医了。回城招人来,又会致乱,还是罢了。”

王中正小心的说着:“朝臣中应该也是有几个通医术的。”

“没一个能比得上刘方明。”说道通医术的臣子,赵顼就立刻想起了韩冈,“韩冈虽然深通医理,但对望闻问切、施针问药却是一窍不通。”

“可惜了那么好的仙缘。”王中正深感惋惜。

“韩冈可从来没有承认。”赵顼其实也是有些怀疑。只看韩冈的年纪,就知道他在医理医道上的见解和手段不可能是自己闭门造车出来的。但如果是得人传授,到底是从哪里学到却是一个谜,路边破庙的孙姓道士,又精擅医术,怎么想都不可能与孙思邈没有干系。“王中正,你曾与韩冈共事过多次,可有提及此事?”

王中正陪着笑:“微臣在韩冈嘴里听到也是一般。不过臣在关西还听到了一些说法。说是韩冈的确是遇上了孙真人,但当孙真人问他愿意做一人医还是万人医,他选了后一项。从此能设疗养院救治万人,能有产钳救产难,却再也学不会半点医术。”

“无稽之谈。”赵顼虽是这么说着,却也觉得有几分符合了事实。

“官家。”另一位随行内侍李舜举走过来,“该去大次了。”

赵顼略一颔首,便站起身要举步离开寝殿。

“官家,那要不要将怀炉带着?”王中正跟在后面低声问道。

赵顼摇摇头,王中正是一片忠心,但却是不可能的。在朝廷大典上,一切都必须依照礼制。随身的饰品、器物,不可多,不可少,绝不能有半点差池。就算坐在玉辂,都不能在脚边放着,何谈随身携带怀炉。即便天子也行不得快意事。

大次,就是按设在祭天圜丘前的帐幕,供天子更衣休息所用。而重臣们所使用的帐幕,则成为小次。

不过赵顼是没有办法休息的。他要穿着绛纱袍,戴着通天冠离开行宫,然后在大次中换上祭天的衮冕。半个时辰的时间,往往就在整理衣物和装束的过程中,飞快过去。

帐幕外,乐声伴随着脚步声响起,这是陪祭的官员们开始站位。

赵顼此时已经身着十二章衣,上有日、月、星辰、山、龙、华虫,下有宗彝﹑藻﹑火﹑粉米﹑黼﹑黻——总计十二道图案,将天地万物穿戴在身上。头戴十二旒冕,十二条五色丝线串成的珠串,就垂在眼前。

赵顼深吸着气,平复心中纷乱的情绪。已经在坛所练习过多次,之前分别在熙宁元年和四年,也有过两次正式的郊祀。但他依然有些紧张,一次失误就是关系到之后的三年,更是会影响到他在国中朝中的威望,一点差错都不能有。

听着熟悉的乐曲,赵顼判断着最后高潮的开始时间——还有半刻钟。

韩冈强忍住要打哈欠的冲动,但他还是有些困。昨夜抵达青城后,他根本就没有睡,也没有哪位臣子能安心的睡得下来。祭天大典是从子正之后就开始,那么一点点休息时间,最多也只能供官员们闭目养神而已。

他所在的房间,安安静静。六位左右正言,都在闭目养神。官员为了拉关系,为日后铺平道路,三五日不睡,都没有什么问题。不过房中的韩冈却是个最大的问题。

韩冈一口气开罪了两位宰相,做足了孤臣的姿态。天子也许会喜欢,但他的结果很可能就是出外。这样的情况下,没人敢跟他走得太近。如果没有几天前的事,韩冈在这群人中必然是众星捧月,但眼下,却是只有平平常常的几句寒暄——官场之上就是这么现实。

不过房内的寂静很快就被打破了,几名太仆寺中的吏员,一间间的开始请人出来。韩冈随着自己所属的队列,站到了预定的位置上。在今天的仪式上,主角是天子,配角、龙套是那些有职司在身的礼官,至于普通的官员,乐班,舞班,周围的士兵,都只能算是壁花。

圜丘被内外三重矮墙给,这三道围墙被称为壝。每道壝墙间隔二十五步。天子的大次就设在外壝。又有两排火炬,从大次一直延伸到圜丘前。

天时已至,百乐齐鸣,乐班齐声高歌:‘在国南方,时维就阳。以祈帝祉,式致民康。豆笾鼎俎,金石丝簧。礼行乐奏,皇祚无疆。’

随着歌声,赵顼手持白玉圭,从大次中走出来。一步,一步,走近上下四层的圜丘。

走到圜丘祭坛下,乐班高唱的歌曲又一变:‘步武舒迟,升坛肃祗。其容允若,于礼攸宜。’此是伴随天子登坛的《隆安》之歌。

踩着歌词和节拍,赵顼举步走上祭坛。

从昊天上帝,到众星星主,总共六百八十七位神祇,祂们的神位在圜丘上,按照层级高低上下排列。最上方的一层,有昊天上帝,有皇地祇,还有陪祀的太祖皇帝。下面则是五方天帝,日月星辰,二十八宿等神主。

圜丘正南方的这一级级台阶,在此时,只有赵顼的双脚能踏上去。

因为他是皇帝。

书曰:‘乃命重黎,绝地天通,罔有降格。’

孔传曰:‘重即羲,黎即和。尧命羲和世掌天地四时之官,使人神不扰,各得其序,是谓绝地天通。’——帝尧任命羲、和世代执掌天地四时之官,使人间与神明互不干扰,各守其序。自尧之后,天神无有降地,地只不至于天,明不相干,至中唯有人皇。

前有三皇,后有五帝。当始皇将皇、帝的称号融二者为一,理论上,其在人间的地位,就是唯一能够沟通天地的神明,亦是使人间不受天地干扰的至尊。

韩冈远远地望着圜丘祭坛,等待天子祭拜祭坛最上方三座神主。

尽管因为长达数月的准备,还有为时七日的典礼流程,使得从祭的官员、将校都是有些懈怠,也都从心底里感到疲惫。但到了天子踏上圜丘台阶的那一刻,懈怠和疲惫从围绕圜丘的数万人的脸上、身上顿时不见。

随着天子踏上圜丘,仿佛天地神明的注意力都集中到此处。在这座祭天之所,多少人宁神静气,随着乐曲,轻轻动着嘴唇,一起默默的哼唱着大典韶乐。

这就是宗教仪式的感染力,除了极少数人,无人不沉浸在肃穆庄严的气氛中,就连韩冈自己,也差一点沉没下去。

儒门道统敬鬼神而远之,但礼天地、敬祖先,就是华夏一脉的信仰,而将皇帝和上天联系起来,更是儒门的重要成分。

但凡天灾人祸,或是祥瑞吉兆,都是上天对天子和朝堂治政的评价。天人感应之说,虽然识者嗤之以鼻,但毕竟已经深入人心千多年。若是逢上大灾大疫,即便智者,也免不了会疑惑和动摇起来。

不击败——最少也要动摇——环绕在皇帝身周的光环,韩冈希望看到的一切,就绝不可能实现。

这是要跟着数千年来积累起来的风俗、惯例和人心来较量,韩冈孓然一身,却要想改变这一切,可谓是螳臂当车、自不量力。

但他还是打算要去做,否则,他来到这个世界又是为了做什么?!做个优秀的宰相,侍奉天子,然后在青史中留下一个名字就算完事了?

韩冈可不会这么认为。

一个稳定的中枢是必要的,可一个被神圣光环笼罩的皇权却是不需要存在的。

只有摘下了天子身上的神秘面纱,去除了被加之于天子身上的神性,韩冈才有机会实现他的愿望。

双眼盯着天子在圜丘顶上的一举一动。不过,韩冈还无意上火刑架。

所以到现在为止,他都没有将望远镜和显微镜给拿出来——尽管已经有了凸透镜,有了凹透镜,但他就是耐着性子等着天子或是其他某个人,在不经意的时候,将两片不同类型的镜片交叠在一起。

韩冈对此很有耐心,无论是放大用的凸透镜,还是作为近视镜片的凹透镜,都已经在官宦人家常见,民间的工匠也有人开始仿制——白水晶的价格虽然长了不少,但照样有人用得起——两种镜片开始普及,望远镜的出现是迟早的事。

到时候,肯定会有人对着天上日月星辰,拿起望远镜观察着。

接下里就算韩冈什么都不做,几百年后,天文学的发展也会将天子从神明一点点的拉到了凡尘中。

但这实在太慢了,韩冈依然有着在保护自己的同时,将皇权掘土断根的手段。

一切都会一步步过来,就像此时天子登上圜丘祭坛,一步步的来!

