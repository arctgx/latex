\section{第41章 辞章一封乱都堂(四)}

【第一更,求红票,收藏】

赵顼浮在脸上的带着冷意的笑容,仿佛方才文彦博的翻版。大宋天子一瞬间成熟了不少,眼神中还残留的一点天真褪去了。视线从群臣身上划过,每一个人顿时发觉,从皇帝那里传来的压力不知不觉的已经大了许多。君臣都沉默着,巍巍崇政殿,像是又要潜入沉默的深海中。

“陛下!臣已将慰留诏书拟好,还请陛下御览。”

司马光出声,打破了僵持的安静。他双手捧着刚刚草拟好的诏书,微欠着身走上前。走过文彦博身边时,司马光脚步稍重了一点——他是在提醒。

文彦博自庆历七年【西元1047年】便入居政事堂,朝堂故事哪有不熟悉的道理?可他偏偏催着天子把王介甫赶出朝廷,却一点也不顾及王介甫的脸面,连惯例故事都不管了。这样真能如愿?不,这反而会惹起皇帝的反感!

自仁宗朝以来,侍制以上官员请郡,除了因为在建储之事上开罪了英宗皇帝的蔡襄,哪个不是下诏慰留几次,方才批准?!王安石弄出的新法虽是祸乱国政,但本心非是为己。此事天子心知。即便要将其罢去,心中也免不了有愧疚之心,他的辞章岂会一请而允?!

司马光为文彦博的失态叹气,他这叫关心则乱!文彦博向来是以稳重,多谋著称朝堂。总角之时,便知道用水将树洞里的球浮出来。跟自己一样,小小年纪便广有名声。但现在看看他,不该说的说了,不该做的做了。等天子回过味来,心里又会怎么想?不,看天子的模样,他已经明白了过来。有些事不该说透,不能说透,却偏偏给说透,这叫弄巧成拙!

“司马卿,快把诏书拿过来。”

司马光将拟好的诏书双手呈上,让一个随侍的小黄门将诏书拿去,展开在赵顼眼前。

“……今士夫沸腾,黎民骚动,乃欲委还事任,退取便安。卿之私谋,固为无憾,朕之所望,将以委谁?”

赵顼默念着,不自觉的微微点头。因为一点委屈,便丢下政事不理,还称病要出京,对于王安石的做法,赵顼心中其实还有些抱怨的。‘现在士大夫议论沸腾,百姓骚动,你却要辞去职务,自取安宁。卿家为己所图,固然无憾,但朕的期望,又该委托给谁?’司马光这一段,当真是写进了自己的心里。

拿过朱笔,签字画押,盖上印。赵顼将诏书递给身边的近臣,“传与王安石。他再病着,朕就要派太医去了。”

……………………

作为参知政事,王安石现在的府邸照例是御赐之物。有花园,有楼阁,是东京城中数得着的大宅院。但在宅院中生活起居的人却很少。

王安石没有娶过妾,身边也没有什么通房丫头,仅有一位陪了自己几十年的老妻吴氏。在众臣中,除了司马光,再无他人如王安石一般。平常在身边听候使唤的,只有一位老仆。在家中奔走的,不过十几个男女。

王安石与吴氏总共生过三个儿子,三个女儿,但一儿一女幼年夭折,儿子女儿都只剩下两人。

长子王雱自幼聪颖,十余岁便能做策论洋洋数万言,三年前考中进士,又回乡娶了金溪萧家的女儿,如今人尚在南方为官。

次子王旁远不如他大哥聪慧,性子又有些古怪——其实这也不难理解,父兄太过出色,这做小儿子压力便会很大——考进士是没可能了,王安石想着日后还是为他求一个荫补,安排着娶门好亲,平平安安的过个日子。

大女儿已经嫁了人,是当年在群牧司任官时的同僚吴充的儿子吴安持。如今吴充已经做到了三司使,一国计相,儿女亲家同居高位。不过吴充对变法之事向来不置可否,看意思也是否定的居多,旧日的好友,如今的亲家,也是渐渐分道扬镳的模样。

长子长女都不在身边,大弟王安国去了西京任国子监教授,王安礼,王安上两个弟弟,一在河东,一在江南,兄弟几人分居天南海北。陪在王安石夫妇一起住在这间宅邸的亲人,就只剩两个儿女。

时已近晚,王安石在书房中等着消息,他并不知赵顼最后会做出什么决定,但今天之内,慰留诏书应该会来。不论是天子同意他的请辞,还是不同意,照着旧例,都不会一请而允,都会来回几次。就像天子登基,对皇位必须要三辞三让一样。如果变法就此而止,辞章往返两三次后就会放人了,如果天子还想继续变法,真心留己,五辞、六辞之后,都不会答应。

一本孟子拿在手中,字里行间满是王安石旧日做的注解。孟子的理论向来为他所秉承,又别有阐发。作为当代屈指可数的学术大家,王安石前些年在金陵教书育人时,都是以孟子为中心。只是他今天没有心情看书,本身又是个急躁性子,把书翻得哗哗作响,几个时辰了,一个字都没看进去。

书房门开了,不是王安石等的消息,而是夫人吴氏走了进来,脸色阴阴的:“二姐刚刚回来了。”

“哦!”王安石随口应了一声,二女儿今天去探望她嫁出去的姐姐,这件事他也是知道的。

“……说大姐儿最近在吴家过得很不好。”

王安石放下书,面沉了下来:“出了什么事?”

一听问,吴氏顿时爆发出来:“还不是你闹得!都是你弄得新法,舅姑都给她脸色看,连姑爷也吵了几次!”

“……是吗?”

王安石声音干干的。他和吴充过去同为群牧判官,情谊甚笃,故而结为儿女亲家。可没想到因为新法之事,他与吴充越走越远,旧时的情谊不再,反而连累了自家女儿。

“大姐那里让二姐儿经常去看看,若是有闲,带小九回家来住两天也行。”

女儿都嫁出去了,她婆家的家务事王安石也不知该如何处理,也只能让女儿回来住两天散散心,正海也可以把外孙带来。他都已经五十了,平日也在忧虑不知什么时候才能抱上孙子。脱去号为拗相公的外衣,其实王安石也只是一个普通的老人。

“饭还没好吗?”王安石不想再听这些烦心事,催着开饭。

吴氏恨恨地盯着王安石。她知道必须在吃饭前把话说清楚,等到开始吃饭,他就又会去想事情,面前放的菜不论多难吃,王安石都会一口口的吃下去。甚至不需用菜,就算是鱼食,她的这位夫君也会毫无感觉到一颗一颗的吞进肚子里去,吃完了都不会发现——这是他跟着仁宗皇帝一起钓鱼时做出的事。听说仁宗皇帝认为是装出来的,心怀伪诈,可自家的夫君自己最清楚,他那性子,哪里会演戏?!实实在在的糊涂!

吴氏柔声说着:“老爷,就是回家住两天,终究仍是要回去的。还是把姑爷换个差事吧,离了京城就行。”

“吾已称病,说不定等几日也是要离京。怎么换?”

王安石的推脱之言,终于惹怒了吴氏,一拍桌子:“王獾郎!大姐是我身上掉下来的肉,你不心疼,我心疼!”

纵然这里并没有外人,但被夫人叫着自己的小名,王安石还是觉得有些尴尬,顾左右而言他:“大哥儿那里有没有来信?”

吴氏脸一背,就不去理他。

王安石看得苦恼,他并不惧内,雅善诗赋的吴氏也一直都是自己的贤内助。但这两年,不知为何自家夫人的脾气慢慢变得古怪了起来,往往因为一点小事发火。但好歹是糟糠夫妻,让一让也没什么觉得丢脸。

书房门忽然被敲响,王安石的老仆在门外响起:“介甫相公,中使来了!是御药院的李都知。”

王安石如释重负,立刻躺回书房内的床榻上,吴氏恨恨地哼了几声,最后还是坐到了床边。装病就有个装病的样子。虽然他的称病谁都知道是假,但一点表面文章都不作,却是在找御史弹劾。

李舜举进来时,王安石已经躺在床上,吴夫人在旁服侍着。只是王介甫一点病容都没有,很健康的样子。李舜举习以为常,拉开圣旨便开始读起来——在称病的臣子家宣旨,不会要让躺着病榻上的臣子起来跪下,这是顾全着大臣体面,也是天子体恤臣子的表现。

在病榻前,李舜举抑扬顿挫的读完诏书。一如预料,并没有回应。李舜举做了多年的宣诏使臣,很清楚是怎么回事。今次为了将王安石请出山,不走个四五趟,跑细了双腿,也不会有个结果。不过想想过去,至少今次不用再为了宣召而追进厕所了。

只是他放下诏书,却发现王安石的脸色,不知何时已是铁青一片。他小心翼翼地照规矩提醒着:“大参,还请接旨。”

“这是司马君实写的?!”王安石厉声问着。如果将诏书拿到眼前,只看笔迹,他便能知道是不是出自自己旧友的手笔,但这旨意他如何能接!?

李舜举方才一读诏书就知道不对了,在他看来王安石发怒也是情理之事,他点头答道:“的确是司马内翰的手笔。”

“司马十二好文采啊!”王安石气得双手之颤,直直坐了起来,也不装病了。‘士夫沸腾,黎民骚动’,这分明是在逼他辞职!‘卿之私谋,固为无憾,朕之所望,将以委谁’,这十六个字,更是诛心之至!天子看了对自己的看法又会如何?

“……都知请回吧。”王安石强忍着怒气。

李舜举见状,也不敢触王安石的霉头,立刻告辞离开。但走之前还不忘说一句:“官家可是真心诚意的等大参回来。”

李舜举走后,王安石翻身下床,铺纸磨墨,在书桌前奋笔疾书,司马光的话,他要一句句的驳回去!

