\section{第八章 战鼓尤酣忽已终(上)}

【一刻钟后再一章。】

当第一缕晨曦划开笼罩在贺兰山下兴灵平原的黑暗,寂静了一夜的营地又重现活跃了起来。在为点卯而敲响的军鼓中,宋军的将士们迎来了决战之日。

激越的战鼓声传遍四野、铺天盖地,环庆、泾原两路宋军列队而出,汇入灵州城南门前的原野上。指挥使们骑着马在阵前来回奔驰,都头、队正在队列中高喊,一同呵斥着手脚笨拙、影响队形的士兵,让浩浩荡荡的阵势逐渐成型。

就在军阵之后,是四十具各色攻城器械。

纵然党项人费尽心力的去砍伐树木,去烧毁民居,却也不可能将原野上的每一寸土地都给检查一遍。木料最终还是找到了一些,以党项工匠的能力,这些木料就只能当做劈柴,在宋军的工匠们手中,却依然用捆扎、镶嵌等工艺拼凑齐了一架架攻城器械。

半个月的时间,总数二十多架霹雳砲,以及十五具云梯都打造完毕,整齐的排列在阵势之后。矢石、战将,双管齐下,在足以攻破灵州城防,而且高遵裕还有其他手段破城,即便此事不成,他照样能走进灵州。

两路联军的数千骑兵,早已在提前出营。他们分成十余部,铁蹄连声,游荡在军阵周围,五里之内尽可以看见他们的身影,护翼阵势的顺利展开。

一座三丈有余的高台矗立在中军本阵中,战鼓在高台下擂动,赤裸着上身的鼓手挥动鼓槌时,块垒分明的肌肉上青筋根根迸起。他已是汗流浃背,但激荡的鼓音依然充满了力量。

高台正中是一张交椅,一面鲜红如血的大纛竖在交椅之后。此时只有一名掌旗官和两名护旗的小校立足台上,交椅空悬,争得着它的主人走上高台。

两艘飞船悬浮在高台上方近三十丈的高空。飞船的吊篮中,两名瘦削的士兵正忙碌着。他们是主帅的眼睛,有‘远见’之名,这些天来,多次发现了准备偷袭大营的敌军,并将灵州城中的战略要点看了个通透。此时他们的双眼扫视战场远近,时不时便抛下一个装着城中守军最新动向的竹筒,及时提醒下方的主将。

又是一个竹筒从天而降,一名高大健壮的亲兵右手在空中一挥,便将竹筒攥在掌心。随即他转过身,恭恭敬敬的将竹筒呈送至灵州城下十万官军的主帅手中。

环庆副总管高遵裕看过收在竹筒中的纸条。“城中西贼守将上来了。”他的声音低沉而缓慢。

鼓号的喧嚣声中,嵬名阿吴一行人出现在城头上。西夏国的战旗在敌楼上升起,以西夏文字书就的‘嵬名’将旗随风招展。

宋军来势汹汹,西夏国中其他部族皆可降,唯独嵬名家不能降。嵬名阿吴受命领军镇守灵州,在灵州成为决定大白高国生死关键的时候,也只有王族值得信任。但这样的态度,却也让其他部族的态度有所动摇。

一支支号角被吹响一面面不同颜色和花纹的旗帜在城墙上飘舞,城中数万守军涌上城头,汇聚在不同的旗帜下,举起手中的弓刀,用党项语高呼着胜利。

万众共一呼,其声响遏行云,高遵裕却语带嘲讽,“鸡鸣犬吠,不过如此!”

在数万党项战士的注视下,在万千宋军将士的等待下,宋军主帅高遵裕一身戎装,头戴金盔,扶着御赐宝剑,稳稳的走上高台。

千万人的视线集中在他的身上,高遵裕的心中涌起无以名状的兴奋。领军征战十余载,大小战事经历过百余起,但只有眼下,才是他最为光辉灿烂的时刻。

享受着众人注目的愉悦,高遵裕抽出匣中宝剑,遥遥指向前方城头上守军将领:“拿下此城,城中女子财货由尔等自取!”他深深吸了一口气,狂吼了出来,“封妻荫子,只在今日!”

高遵裕一向不喜无谓的鼓动,他只让人看到实实在在的好处。长剑指向城头,十数名亲兵拿着铁皮话筒,将他的话传遍军中。

数万宋军战士随即以刀击盾,以枪顿地,同声呼喝,如山崩,如海啸。比起被围在灵州城中,只能靠嚎叫壮胆的西贼,将心中贪婪和渴望呼喊出来的宋军,更为气冲斗牛。

每天的口粮已经减了三成又如何?过两天就剩一半又能怎么样?缴获的牛羊都吃光了也没什么大不了的!

打进了灵州城,要什么没有!?

没有再多的闲言赘语,高遵裕的长剑挥下,战鼓声节奏随之一变。整齐的步伐,为鼓声伴奏。军阵的士卒踩着鼓点整列向前,当城头上的箭雨砸在最前一列的橹盾上的时候,他们方才停住了脚步,反击的箭矢也立刻向城头激射过去。

宋军的战术十分简单,利用神臂弓组成的箭阵,强行压制城上守军的反击。高度上的优势对远程兵器是个很大的加成,但党项人的弓弩与神臂弓相比,这份差距已经不是区区四丈的城墙所能弥补。

密集的箭矢转眼间便压制住了城上的弓箭手,密密麻麻的木羽短矢深深的钉进了墙头,一面面战旗在箭雨中被撕扯成一丝一缕,箭矢如同春时飞蝗、夏日急雨,劈头盖脸的洗过城墙,甚至没有一人敢于抬起头。

霹雳砲一直紧随着军阵,当军阵停下脚步,它们则从阵列的缝隙中继续向前,直至与军阵的最前列平齐。由于材质的问题,高遵裕手上的霹雳砲,射程要比标准的七稍砲要缩短近三分之一,只有放在城上西贼箭矢可及的位置上,才能发挥出最大的威力。

云梯也被推上去了。每一座云梯下面和后方,都有几十多名手持长枪和巨斧的精锐,他们是选锋,当城头上的,就该轮到他们冲锋。

云梯前方的城壕已经不复存在。四丈多宽的护城河,如今都已被沙石填平。围城的每一天,环庆军都会出阵,直逼灵州城下,用神臂弓压制城头守军,趁势不断填塞灵州城壕,如今已经有长达四十步的河道被沙石填埋上,而且还在不断延伸中。

没有什么可以阻止自己的脚步,当十几枚人头大小的石弹同时弹上半空,然后在黄土夯筑的城墙上撞击出一个个满是裂痕的凹坑,高遵裕对此确信无疑。

……………………

“仁多零丁还要磨蹭到什么时候!”

一个年轻的党项将领在敌楼中愤怒着,他是嵬名家的新生代,只有他才方便在嵬名阿吴面前如此说话。

每一刻都有宋人的石弹撞上城墙,建造在城墙上的敌楼随之颤动,一蓬蓬灰尘从承尘和房梁上来洒下,人人都是灰头土脸。

嵬名阿吴抬起眼,慢吞吞的道:“慌什么?宋人还没有上城呢……等他们把云梯推上来,就不会再丢石头了。”

见自家的子侄还要辩解,嵬名阿吴皱眉道:“别小瞧了仁多零丁,他不会贻误战机的。”

嵬名家的年轻人不敢再辩,退下去站定。

眼下就只有相信仁多零丁,嵬名阿吴心中暗叹。只是箭矢和石弹,不过是才开场而已,当云梯推上来,短兵相接,才是正戏开锣的时候。

到时候他能挡住宋人多久,嵬名阿吴完全没把握。那时仁多零丁的援助若还是没有到,灵州城很可能就保不住了——尽管眼下六路入侵的宋军仅仅到了两路而已。

嵬名荣领军向西,一路骚扰。硬生生的用巨大的伤亡,将王中正的进兵速度迟滞下来。加之从兰州往灵州,比起泾原、环庆,路程长了两倍都不止,三天前才抵达应理【今中卫】,离葫芦河口还有一段距离。而种谔和李宪困于地理和天时,加上连通银夏和兴灵道路上的几处绿洲水源都被破坏,更是到现在都没有能渡过瀚海。

只有环庆、泾原两军十万人马,却依然能轻易压制拥有七万守军、八万丁口的灵州城,甚至连出城骚扰都做不好。明明是铁鹞子的骑术更高一筹,但宋军骑兵依靠身上的铁甲却占了上风。军力衰弱的现状,让嵬名阿吴胆战心惊。

如果宋国的皇帝选择了慢慢放血消耗的战法,大白高国最多十年就要灭国,没有任何挽回的机会。幸好宋国皇帝选错了道路,眼下还有拼死一搏的可能。

现在就等仁多零丁那边的消息传来了。颤抖的敌楼中,嵬名阿吴静静的等待着。

……………………

仁多零丁已经很有些年头没有全副武装了。二十多斤的精铁瘊子甲穿戴在身上,感觉比过去要沉重了许多。

他所率领的队伍离开灵州战场有二十里,分散在几座临近的村庄中。从理论上说,宋人的飞船应该能看到他们。可就算是天空中飞舞的猎鹰,也分辨不清二十里外的细小人物,何况视力虽佳,却也没有脱离人类范畴的宋人远见。

“宋人已经开始攻城了,不知道灵州能坚守多久。”

“两个时辰就够了。”

“宋人会不会发现?”

“就是发现了又如何?当他们发现,就已经来不及跑了。”

仁多零丁与叶孛麻交换着只言片语,也同样在等待着。

