\section{第19章 萧萧马鸣乱真伪(一)}

呼啸在在无定河谷中的风,这些天来渐渐的已失去了夏天时的燥热。不过山头上的颜色依然还是郁郁葱葱的深绿,并没有随着季节的转移,而一下改变了颜色。

身处直面敌锋的前沿城寨中已有一载,童贯也逐渐习惯了陕北的春夏秋冬。自从到了罗兀城之后,时不时的被王舜臣叫上一起骑马出猎也是成了惯例。

“秦凤路、泾原路要南下的兵据说已经到宝鸡了。”

穿行在谷道中,两侧山林森森,骑在马上,童贯与身边罗兀守将说着闲话。

王舜臣打了个哈欠,“走得倒挺快,过两天就能进汉中吧……是邸报上的消息?”他没精打采的问着。山间的空气清新无比,但没有猎物出现在眼前,他就怎么也打不起精神来。

“不是。上个月的邸报要到月中才能送来,这个月的更要到下个月的十五六,没那么快。”只从邸报中,就能看出罗兀城的偏院,童贯道,“是今天早上绥德那边来的信。”

“哦。”王舜臣点点头,监军的宦官之间都有消息往来这事他是知道的。

童贯接下去说着:“鄜延路的兵力据说也要动,不知道会是哪个指挥能被选上。”

王舜臣咂着嘴,眼睛扫着山林中。随行的骑兵前后护持,林中也有士卒拿着棍子乱扫,试图惊动草木深处的鸟兽。只是已经好半天了,也不见有什么值得他拉弓的猎物。

“罗兀城肯定不会动,种太尉也不可能答应将罗兀城的关西汉子换成河北军。”

童贯偏头看了看王舜臣,虽然看不清藏在刺猬般虬髯中的一张脸上的表情,可罗兀城主的话声中,已经没有了前几日的火气。

得知河北军要配属到关西来,王舜臣当即就大骂了一通。又不是本乡本贯,哪里能相信河北军会帮他们关西拼命守城?只是当他跟着又听说是韩冈的提议后,立刻就不言语了。

“听说李都知也要跟着南下,这下终于可以有机会立功了。”

童贯脸上多了一份笑容:“是啊,总算有机会上阵了。”

安南经略招讨司成立,主帅和两名副帅都是朝臣,统领数万兵马,在天子的想法中,肯定需要有人盯着。作为监军,随军南下的是童贯的师傅李宪。

他们这些宦官,在宫中也照样要读书、上学,有的时候还能旁听一干大儒的讲课,也能接受三衙之中的军事训练,许多人都是文武皆备,绝不下于寻常的士大夫。而在熙宁年间的这一批宦官中,李宪也是号称兵事第一。童贯作为李宪的弟子,也是一直都想在军事上有所成就。

不过李宪的运气一直不佳,横山、河湟的功劳都被王中正给拣去了,现在又一战定茂州,已经是赫赫有名的内侍中的大将。要不是王中正现在在蜀中的群山间脱不开身,安南道经略招讨司中,肯定少不了他的位置。

王舜臣认识王中正,王中正的能力他也了解。名气老大,已经有人将他与旧时宦官中的名将秦翰相比的王都知,天上总是将功劳掉到他的头上,运气当真不错。

不知何时,王舜臣已经是长弓在握。信手拉弦,嗡的一声,夹在指间的箭矢化为一道流光。随即在二十多步外,一只刚刚从树梢上被惊起的山雀从空中掉了下来,扑的一声摔在了道路上。

“好箭法!”童贯拍手叫绝。隔着二三十步的距离,一箭就射中天上的活物,而且还是骑在马背上,这份准头,在军中也是极难得的。

“哈……”王舜臣叹了一声,“可惜不能党项人来试一试箭。”

在军中赫赫有名的连珠神箭,一年过来就只能用来射鸟,王舜臣也只能叹息自己的时运不济。

前面的士兵捡了王舜臣射中的猎物回来,双手呈上。

手指粗细的箭矢射穿了只有半个拳头大小的山雀,附在箭上的力道一下几乎将这只小小的猎物给扯碎。如果是在纵马奔驰的时候,王舜臣的准头就会大打折扣。不过现在是信步由缰,慢悠悠的在山道间行走,拿着马弓,便也是一射一个准。

王舜臣拿过来看了两眼,甩甩手,将箭杆上的一滩黏.腻的血肉甩掉,将精制的长箭收回箭囊。

他对童贯笑着:“好歹要弄只兔子回去做汤。要不然山鸡也行,总不能空手回去。”

忽然间,就在山林中帮着驱赶猎物的士兵忽然大喊了起来。王舜臣一下剔起了双眼,没精打采的慵懒一扫而空,如同长刀出鞘一般变得锋锐犀利,而前后左右的骑兵,也都是一下换了副模样,手挽长弓变得警惕起来。

“怎么了?!”童贯觉得周围的气氛一下就改变了。

王舜臣没回答,翻身下马。‘有奸细’的吼声这时才从山林中响起。

手中换了张力道更大、适合步射的战弓,王舜臣一声吼叫拉弦如满月,瞬间便是两支长箭向着山林深处中飞了进去。

一声惨叫在箭矢落处响起。过了片刻,一个人就被两名士兵拽着胳膊拖了下来,两条腿的腿弯处都被利箭扎了个对穿,不能走动,只能被拖着。

“都巡好箭术。”童贯由衷地说着。

人就藏在树林中,就算是在跑动时,绝大部分身子还是会被草木遮住,但王舜臣射出的两支箭矢依然准确的扎穿了贼人的两条腿。

拖到近前,贼人被扯着头发拉了起来。三十四十的样子,装束是汉人的打扮,看着像名樵夫。不过当他呻吟着向王舜臣说自己是良民的时候,王舜臣阴狠的笑着:“罗兀城出城砍柴的人,不会有往北走的,本将几次为此下了严令,如果有人违反,视同通敌,射死勿论。”

“果然是西贼的哨探。”童贯低头亲自看过了贼人的双手,上面的茧痕完全是常年拉弓留下来的,抬起头,他厉声喝问,“今天城外巡检的是谁当值?!”

“……是罗都头。”一个士兵犹犹豫豫的回答着。

“都巡?”童贯转头问着王舜臣的意见。

“一个两个哨探,就是要摸到城头下都容易,更别说藏在山林里面了,只要不是大队人马就不用在意。”王舜臣将哨探踢起来,“绑起来拖回去细细审问。”

“要回去了?”童贯问着。

“嗯。”王舜臣点头应着,跳上马,提缰调转马头,“贼人都摸到罗兀城边上了,要早点回去做个应对。”

他们现在的位置是无定河谷分出来的一条岔道,虽然周围山上草木茂盛,但离罗兀城实际只有三四里。

尽管外貌粗豪,王舜臣行事其实向来小心。出来射猎,从来不会走得太远。放出去的耳目都占着制高点,监视着周围,不虞被大股敌军包围还懵然不知。如果只是小队人马,他手上的长弓也就能难得的开一次荤。

“哨探都跑到了罗兀城边上,银州的兵马肯定又多了。”走在回程的路上,王舜臣回头看了看被绑成粽子的哨探,“丰州是郭太尉亲自领军上阵,西夏人硬打是打不过,只能从旁边来找补了”

“不知道是不是准备佯攻?”

“如果我们表现得弱一点,佯攻也会变成真打。”王舜臣咧嘴一笑。

罗兀城是控制横山的关键,丢了罗兀,就是丢了横山。王舜臣现在把守的罗兀城如同一根骨头卡在西夏人的喉间,就算西夏在环庆、泾原攻城掠地,也远远弥补不上失去横山带来的损失,若有夺下罗兀城的机会,党项人绝不会放弃。

“不过佯攻也好、真打也好,援兵从绥德过来,快马只要一天而已,输不了的。”他继续说着。

童贯笑道:“若党项人当真攻来,都巡也就可以大展身手了。”

“那当然,总不能让李信、赵隆他们笑话俺只能在这里射鸟!”赵隆一举定了茂州,李信前面在邕州立的大功。而且安南行营成立之后,他肯定是先锋将。两人现在立下的功劳都是王舜臣没办法比的。但王舜臣绝不会甘心认输,“打南方的蛮子,哪比得上砍西贼痛快。要是南面的手脚慢一点,我们说不定都能冲进兴庆府了。”

“辽国多半不会坐视……毕竟是自家女婿。”

“西贼已经不行了,没钱没粮怎么也打不了仗。就算背后有契丹撑腰,光靠一口气也撑不了多久。”王舜臣猛然大笑了起来。“契丹也是个穷鬼,辽主还有那个大名鼎鼎的魏王,就算梁太后张开大腿,他们也不会给她一个铜板的。”

王舜臣的话粗得很,但说得确实在理。与王舜臣同守一城,童贯早已一清二楚,王舜臣决不是外表一般的只知冲杀的猛将,眼光手段都是一流,除了好酒贪杯以外,就没有别的毛病。

毕竟是跟在种谔、韩冈身边多年,耳提面命的历练出来的,日后少不了也是坐镇一方的大将。童贯这般想着,看向王舜臣的视线也越发的热切,这里是他飞黄腾达的根基!

