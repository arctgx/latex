\section{第八章 朔吹号寒欲争锋(四)}

“这一回需要堂除的不算多,就两只手。 ”张璪笑说着。

张安国则拿起手上的资料回答韩冈问题,“开封的中牟知县,京西的襄州知州,两浙路的明州知州、杭州通判……”

韩冈用心听着。张安国念出来的这些官阙,与他收到目录没有什么差别。

在决定选人等级文官的流内铨门外,有所谓阙亭。但凡州郡申报衙署中有官阙,流内铨便会张榜公布,这是避免奸猾部吏倒卖官阙。而宰辅们手中的官阙名录,记录着阙额。每天都会发送到各位宰辅的手中,待他们进行安排,也就是所谓的堂除。

堂除,就是需要经过政事堂直接授予的职位。如果只是知州、知县,照常理是应该交给审官东院来拟定人选。可是自开国时起,比较重要的知州、通判、知县的职位,便被政事堂直接控制,究竟安排谁去就任,全得要当时的宰辅来发落。

随着时间的过去,堂除的范围也越来越大。每一任宰相,都会自觉或不自觉的去干涉审官东院——或其前身审官院——的工作。当这些干涉成功后,往往就会成为定制,从此这个职位便成了政事堂的所有物——而以政事堂与审官东院的上下级关系,宰辅想要伸手,鲜有不能成功的例子。

现如今平均每天,都要有五到十个州县亲民官,或路中监司属官,或在京百司主官,需要宰辅们来决定。

韩冈昨天用了半个晚上的时间,将堂除的范围背了下来。大概几乎所有的望县,上州以上,全都是堂除的范围。

而宰辅们的手上,还有一份眼下在都中候阙的官员名单。

只是他们并不需要按照候阙的官员名单来安排,那些正在任上、有后台的官员往往不满任便能迁转,而没有后台的官员,在京城中待两三个月等官阙都是常事。

一般来说,当宰辅们将这些官阙派定之后,每隔五日就会要进呈给天子过目。

“玉昆……可有心仪的人选?”

待张安国念完,张璪问着韩冈。

韩冈轻呷了口茶汤,道:“韩冈初入中书门下,朝中贤士仅知一二。过去也只在关西、岭外、京西、河东等处任职过,对本路的官吏贤与不肖有所了解,州县繁剧清省与否略有心得,可其他地方就不怎么熟悉了。”

张璪闻言,与韩绛对视了一眼,脸上稍稍有了些许笑意。

他们不怕韩冈提条件,就怕韩冈不提。以他在太后心目中的地位,什么位置都要抢,那可就让人头疼了。王安石能以一新进的参知政事,逼得政事堂其余宰辅无处立足,正是因为当时赵顼全心全意的信任。

“除了京西,无不是边境要地,亟需身兼文武的贤能治理州郡。”张璪顾谓韩绛,笑叹道:“除了玉昆,还真想不到有谁能衡量此等贤才。”

韩绛则苦笑着摇摇头,颇是无奈,“……玉昆在京西主持工役,从南到北都走遍了,说起当地人情地理,我等是比不上玉昆。”

张璪笑容不改,“子华相公说得是,的确不如玉昆。”

韩冈点点头,笑道:“子华相公、邃明兄谬赞了。其他去处,可就要劳烦两位。”

就韩冈所知,这一年多来,堂除的候选名单,大半都是由蔡确、曾布先行拟定,再与韩绛、张璪商议,并进呈给向太后。不过据说向太后有打算改变这一点,在宫中也有些流言传出来,或许这也是蔡确决定叛乱的主因……至少是之一。

如今蔡确、曾布败事,张璪肯定想趁此良机,扩张自己的权力范围。灵寿韩家累世簪缨,韩绛也有的是亲朋好友和亲朋好友的亲朋好友、以及他们的子弟需要安排。

韩冈要与他们相争,不会急在此时。正如韩冈对韩绛所说,他对朝中官员的了解太少,就任的地域也不算多,对遍及天南地北的职位,很难挑选出合适的人选。万一任人不当,韩冈必须要付连带责任。韩冈可不想隔三差五便被罚铜,钱是小事,但太丢脸,也会损伤个人的威信。

在韩冈看来,与其只能在大饼上舔一舔,还不如先切上一块独占下来,小归小,却是能够稳稳的吃进肚子里去的,谁也争抢不得。

几位宰辅的对话,张安国权当没听到。

正常宰辅权力分账,不会如此**裸,一般也就是在人事安排上,通过对一处处官阙进行不断的提议、争论、妥协,最后划定各自的权利范围——是默认,而不会明示。

而韩冈直接将话给挑明了,的确省了不少时间和误会,不过也太直白了一点,仿佛武夫的脾气。韩绛这样的老派人明显的不习惯,倒是张璪,反应过来后,还不忘讨价还价一番。

韩冈通过廷推进入中书门下,在太后的支持下显得气势汹汹。现在韩冈只要求几处边角地——京西也就是个添头——其实是退让。他的要求,跟他得到的票数比例相当,差不多四分之一。

韩绛的xìng格厚重,韩冈也是给韩绛多一点尊重,这是他应得的回报。之前政事堂中的政务,多决于蔡确,现在让韩绛舒舒心,也是应当的——他还能在政事堂中待多久?

不过说到底,还是韩冈在政事堂中没有底蕴,当年他若是接受了韩绛的举荐,担任检正中书五房公事,对中书门下有些熟悉就方便的多了。至少能在那些堂吏和堂后官中间打下基础,下面有了根基,做事就不会两眼一抹黑。不过以当年的情况,韩冈拒绝中书,就任判军器监,对他是最为有利的选择。

当然,韩冈在现阶段,也没必要去与韩绛、张璪争夺更多的职位。他还没那么多人需要安排,争来了也无用。现在要做的是稳扎稳打,抓住几个关键xìng的位置就够了。

在这其中,军器监是韩冈必须要拿下的地方。

那是王居卿的位置。

从韩冈之前所了解到的王居卿一贯以来的表现,他应该很适合这个位置。就算王居卿不称职,韩冈在军器监中也有足够的基础,让军器监能够稳定运作。

而且王居卿是侍制,目标又是军器监,韩冈要拿到这个位置就更容易了。

卿监以下的官员任免,都是政事堂的职权范围。可涉及到侍制及侍制以上官员的位置,那就是天子的权柄了,宰辅们只有建议之权。仅仅是天子的安排不当时,宰辅们也能表示反对,到时候,就看皇帝和宰辅哪个撑不住先退让。

军器监从地位上说,属于卿监一级,宰辅们可以直接任命,事后报备。但由于这个衙门在朝堂中地位特殊,历任判监都是得了天子钦命,由此成为故事,加之往往由侍制官担任,宰辅们也只有推荐之权。

要说服太后那边很简单,此外只要跟韩绛、张璪通个气就够了。两人不反对,又有太后支持,王安石就算想要反对,也无能为力。

唯一值得担心的是王居卿给人翻出什么黑历史来。如果他过去犯下什么大错,给人揭了出来,韩冈也救不了他。

韩冈不信王居卿在给吕嘉问背后一刀前,没有半点准备。

可是不做事,就不会犯错。这不论是在哪个时代都是通病。越是事务繁剧的衙门,越容易出事。越是清要的衙门,就绝不容易犯错。

三馆崇文院中的校理、检讨、编修、校书们,怎么都不会出事,又因为身居储才之地,升迁往往极快。而三司衙门中的官员,往往难有全身而退。就是贵为宰辅,也时不时会被罚上几斤铜。

想到这里,韩冈又想起司马光当年议论黄河改流的事。吕公著反对任命司马光的论述实在太典型了:“朝廷遣光相视董非所以褒崇近职、待遇儒臣也。”

因为过于典型,韩冈一想起,做实事和说话的区别,就会想起这一桩。日后若是也写私人笔记,记录一些朝堂中的经历,韩冈不会忘记将这一桩给记录进去的。

王居卿是靠做实事做到了侍制,他犯下的过错,只会比人多,不会比人少。

韩冈不担心他在推荐王居卿担任判军器监之前爆出来,只担心在王居卿就任之后,给人揭出来。

那时候,就又是一团乱。

从这方面来看,御史台中就需要安排一个人了。通过做翰林学士的沈括,不难做到这一点。不需要正直的,只需要听话的。也不要他能够攻击谁,只要其存在于御史台中,远比直接上场攻击更有威慑力。

至少能让人想要攻击王居卿时能投鼠忌器。

除了王居卿之外,韩冈还有其他的担心,黄裳的制科考试能不能通过也是需要考虑的一件事。

如果只是给他安排一个位置,倒是很好说。但这毕竟是比进士科还要高一个等级,又名大科的制科考试,开国以来,能通过的还没超过五十人。刷掉黄裳,没人能说不是。

不过在宰辅会议时,没有太多时间给韩冈思考问题,还有铨曹四选进呈上来的最近一期的人事调整名单。

韩绛将这份名单压在手上,对韩冈道:“已经开春了,西域的雪也要开始化了。组建西域都护府的事不能再拖延。”