\section{第二章 一物万家欢(下)}

听着房内的哭声,高遵裕引颈相望。

很快,产房的大门打开,徐老稳婆的媳妇从门中出来走出来。对他福了一福,“恭喜总管,是个公子。且多亏了韩官人的计策,眼下是母子平安。”

高遵裕长舒了一口气,放下了心头大石。正了正衣冠,回头对韩冈道:“为了这小儿,闹得阖府上下鸡飞狗跳,倒让玉昆见笑了。”

“情关至亲,乃是人之常情。要是今日笑了总管,等到明日,韩冈还不要让总管笑上两回?”

哈哈哈,高遵裕一阵大笑,“玉昆还这么会说话。”

笑罢,神色郑重起来:“今日多劳玉昆,若非玉昆之策,今天就不是喜事而是丧事了……唉,不愧是药王弟子。”

韩冈很无奈的摇着头:“跟药王无关,下官也无缘见过孙真人。只是用上了一点格物致知的道理,要夹东西不都是用钳子?既然要将孩儿弄出来,用钳子作为外力也是最简单顺手的……”

“哪有这般简单。”徐老稳婆在旁不满韩冈的谦虚,“韩官人使人打造的产钳,老婆子做了几十年都没能想到,韩官人却是一句话的功夫就有了主张。。这产钳日后不知能救下多少条性命,老婆子这里要为她们拜谢了。”白发苍苍的老婆子说着,跪下来就要向韩冈行礼。

韩冈连忙搀起老稳婆,“这可使不得……”

“玉昆,这事你当得起!”

韩冈摇了摇头:“今日只是利用外力,毕竟不如自然之道。用上产钳之后,母子二人多少会有些问题,说不定日后还有些后患。韩冈哪里敢居功,还望总管能够恕罪呢!”

韩冈的话,让高遵裕疑惑不解。等他看到抱出来的儿子,才知道韩冈为什么会这么说。

因为胎儿是用钳子夹着头颅出来,现在头面上就有印痕。听韩冈的口气,日后也许还会有后患,而银钳探入体内,产妇的身子肯定也伤到了一点。但徐老稳婆就在旁边赞不绝口,对韩冈几乎要顶礼膜拜,让高遵裕的一点犹疑不翼而飞。

仔细想想,原来是母子都保不住,现在好歹全活了下来。至于后遗症什么的,救人的时候也顾不得许多,命保住就成了。就像战场上的伤兵,必要时为了保命,还得把手脚给锯掉,事后又有谁能抱怨?

新得麟儿,宠妾无恙,高遵裕心情大好。只要是今次有份功劳的,都是一份厚赏。两个稳婆都是加倍赏赐,除了银钱外,还有二十匹红罗彩绢,都是数倍于惯例的给稳婆的报酬。而那位银匠——他姓刘,最后就是他的作品排上了用场——更是高遵裕直接就给赏了五十两银。

而韩冈也对高遵裕道:“这个刘银匠做事有谱,虽急而不乱——那几个匠人也不多想想,不经打磨的器物如何能用?——如果做其他事也是这般,倒也值得抬举他一下。”

“听他的口音是蜀人。”高遵裕微微一笑,“蜀地的银匠果然不同一般。”

“……也得与汉高同姓方可。”

两人对视一眼,又是哈哈大笑起来——就在几十年前,大宋正有一位来自蜀地,做了太后前夫的刘姓银匠。不过眼下的这位刘银匠,倒不可能会有一个能让天子都看上的浑家。

韩冈向高遵裕告辞,“既然此间已然无事,韩冈就不敢再打扰总管,先行告辞了。明日再来恭贺总管喜获麟儿。”

高遵裕点头:“也好。等家中少安,我就让徐婆子和她的儿媳妇去你玉昆府上。”

回到家中,韩冈先去见了父母,韩千六已经出去了,韩阿李正在家。

看到儿子回来,韩阿李就问道:“三哥,总管家的明珠怎么样了?”

“母子平安。”韩冈没问是从哪里听到的消息,他知道自家老娘的耳目从来都不弱,“赶明儿就要准备些礼物送过去了。”

“恩,应当如此。”韩阿李点了点头,“过几日连着就有素心、南娘和云娘的三件事,总管那边多半也是要连送三次礼过来。我们的这一份就不能轻了,省得有人说我们韩家不知礼数。”

“全凭娘来处置。”韩阿李知人情,韩冈也没什么要补充的。这两年来,家中的人情往来,都是韩阿李来掌管。只是在摸不准对方身份地位的时候,才会征求一下韩冈的意见。

正想回书房继续今天的功课,韩阿李却叫住他:“三哥,你先别走。听说明珠今日难产,是三哥你出了主意,满城里找工匠。娘倒不明白了,别的倒也罢了,你什么时候学得这产科的事?”

“孩儿何曾学过产科?这全是深研格物致知之术的结果。世间儒者只知死读书,有几个能知道天下万事的道理,其实都在圣人之言中。只要肯多看多思多想,医卜星相等小道,闻一知十也不是什么难事。”

“你就尽管扯吧。”韩阿李了解儿子,听了就知道有一半是在胡扯,“当初你拖着不肯纳云娘,也是满口的道理。”

“孩儿还真是冤枉。高总管家的明珠都快二十了,今次生产差点都一尸两命,要是过去孩儿太早收了云娘,她吃的苦头只会更甚。”

“好了,好了,说一句你顶一句。要不是看在你一直对云娘好的份上,早打断了你的腿。”韩阿李佯怒着,把韩冈赶了出去。她也拿儿子没办法,本来只是随便抱怨上一句,没想到这顺口的话都能顶回来。

韩阿李想着该送何等礼给高家,才算不失礼数,而韩冈,则自回自己居住的偏院。

素心和周南都在正屋中等着韩冈。眼下不敢让她们动针线——做女红很是费神——各自拿着一卷载着唐传奇的《异闻录》在翻着。

自家的良人一大早就被高家的人拉着跑了出去,又让人打听到是明珠难产,两女心里当然有些害怕。严素心翻到《莺莺传》后,许久都不翻上一页。而周南却把《李娃传》翻来覆去的看着,也不知看进去多少。

终于等到韩冈回来的动静,都同时放下了手上的书卷,“官人,明珠姐姐怎么样了?”

“没事,没事……平安得很。”韩冈走过去,将周南和严素心一左一右拉倒自己的腿上坐下。压在膝头上的重量比起过去沉了不少,但丰腴的弹性,却也更上一层楼。

两手按在高高挺起的肚子上,感受着自家儿女在里面的动作,韩冈把今天自己的功劳毫不掩饰的都说了出来。他知道,这个时候,就要安着两女的心,不能让她们感到害怕。

而周南和严素心在吃惊之余,却也当真安心了不少。有着这样的夫君,那真是想出事都难。

第二天,徐稳婆就被高遵裕送了过来,她的媳妇则暂时留在高府,看护刚刚生产过的明珠,还要两日才能过来。

见到韩冈,老婆子就又要躬身下拜。韩冈让人将她搀扶起,“徐婆婆不需多礼,过两日还要劳烦到你。”

“官人放心,老婆子当是要尽心尽力,不会疏忽半点。”

依了韩冈的吩咐,徐稳婆被领到客房去安住。老婆子被人带出去的时候,尤念叨着药王弟子、药王弟子。韩冈听着一笑,今日之后,他这个身份就越发的坐得稳当了。而这等民间的传言也用不着堵,只要在官场上的公开场合加以否定就行了。

头仰靠着椅背,韩冈。他既然来到这个时代,免不了要将千年后的事物提前到此时。一般来说,肯定是想到火枪大炮。但韩冈不这样想。火枪大炮要有,但不用急,大宋一时半会儿倒不了,等他到了合适的位置上,自然会拿出来,将功劳名望都赚足了。

韩冈现在要的是声望,缺的也是声望。

声望这东西,看不见,摸不着,却是实实在在的。造出火枪大炮,百姓们不知道有何意义,一开始只能让少数人能知道其中的功效。但一个药王弟子,加上神乎其神的事迹,就能让韩冈这个名字传遍天下人的耳朵里。而医学上的名望,也不会惹起朝廷忌惮,韩冈并不要担心,人人皆知韩玉昆后,会给他带来什么不利的影响。

闰七月的月底,周南先有了动静,因为保养得宜,而且又是因为自幼习舞,体质很好,很快就顺产了一个女儿。尽管不是儿子,但韩冈岂会在乎?抱着女儿爱不释手的样子,倒让有些担惊受怕的周南心情好了起来。

只过了两天,严素心也生了,这次却是个儿子。

数日之间,儿女皆备。韩父韩母喜不自禁,而韩冈也是兴奋得几乎难以自抑。而各方听到韩冈有了子嗣,陆续送上门来的贺礼,堆得堂屋中都站不住脚。

韩云娘忙碌之中,也为周南、素心感到欣喜。但看到被众星捧月围着的两女,她的心中也不免一阵落寞。悄悄的出了拥挤的堂屋,刚刚拐过屋角,却被人一把揽过。

一头撞进结实的胸膛,少女惊骇欲叫。但一抬头,映入眼帘的却是心中最挂念的那个人的微笑,“三哥哥?!”

韩冈搂着纤巧的柔嫩娇躯,温声说道:“等我得了贡生回来,就将你迎进门。”

韩云娘轻轻嗯了一声,将头埋在韩冈的怀里。

