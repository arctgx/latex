\section{第42章 望断南山雁北飞(下)}

巩州陇西。

景思立兵败的消息刚刚传了回来,因为被严密封锁,作为大后方的陇西城中尚未出现混乱,但位于城中心的衙门里却已经是人心浮动。

实际主持河州一役后勤供给的秦凤转运使蔡延庆,正在考虑着是不是要立刻出兵救援。

陇西城和渭源堡中还有两千人马,狄道那里已经移文来说要将其调动。而且巩州还有没有动员起来的弓箭手,加上归顺的几大蕃部,三五天之内怎么也能拉出一两万兵马来。要挽救眼前的危局,兵力还是足够用的。

只是蔡延庆刚刚说出自己的想法,就立刻获得了一声异议,“不可!万万不可!”

跳出来反对的,是转运判官蔡曚。去年临洮一役,蔡曚兼任随军转运时在韩冈和王韶手上吃过了教训,半年多来老老实实的做人。但这些天,随着王韶领军翻越露骨山而失去了音信,他又重新活跃起来。而等到今天景思立兵败的消息传到陇西,他更是一下变得像雨后的青蛙一般欢蹦乱跳:

“调兵乃是经略司之权,转运司岂可侵夺之,此事万万不可!”

“事有经权之分,眼下的情况也顾不得那些规矩了,日后我上表请罪便是。”

若是转运司中事,蔡延庆可以轻而易举地将蔡曚的势头压下去,但现在说到越权调动兵马,他觉得最好还是要将之说服。

蔡曚的脑袋摇得像个拨浪鼓,连声说着‘万万不可、万万不可’,更威胁着蔡延庆::“若是运使一意孤行,下官可是要上书弹劾的!”

蔡延庆脸色阴沉下来,若是真的让蔡曚给自己泼上一身脏水,京中再有人趁机在天子面前进谗言,那他还真是有理说不清了。想了想,他转过去问安静的坐在一旁的王厚,“处道,你是熙河路中人,这件事你说该怎么办?”

蔡延庆是在征求王厚的支持,如果有王韶的儿子出面,征发兵马能够省上不少事,而在天子面前,也有敷衍得过去的借口

但王厚沉默着,没有如蔡延庆所愿,即时开口回答。

他在犹豫,一旦同意了蔡延庆插手军务,等于开了个恶劣的先例,日后别人将会怎么看待熙河经略司。而且最关键的是,目下还没有到山穷水尽的时候。

临洮堡那边的情况的确很危急,损兵折将的惨状,自王韶到秦州之后就从来没有出现过。现在一万多兵马远在河州,而居中的熙州被党项人攻打而危在旦夕,一个不好就会出现前方大军全军覆没的惨状。

说起来,的确是该出兵救援。

可是到现在为止,韩冈都没有移文过来,说要调动陇西城中的兵马。也就是说,至少在韩冈看来,他所暂代的熙河经略司,依然还能控制着眼下的局势,不需要调动兵马,也不需要征发民伕、蕃军,更不需要外人插手进来!

如果自己附和了蔡延庆,他该怎么对韩冈说。等到追击木征的大军凯旋归来,又怎么该见父亲。

而这边蔡延庆见着王厚犹豫不决的闭口不言,心中不痛快的催促道,“处道,狄道那里已经移文要调兵了,此事已是犹豫不得!”

听到催促,王厚闭上了眼睛。深吸一口气,重新睁开双眼的时候,他已下定了决心:“狄道的移文,下官先前也看过了。”本来就是给王厚的,“但这只是王都知和沈中允的意见,上面并没有韩机宜的签押!……家严在领军南下时,将经略司中之事,尽数托付给韩机宜,由他代掌印信。眼下没有他的签押,调令就是一张废纸,何谈出兵?”

蔡延庆闻言脸彻底黑了下去,心底的怒火毫不掩饰的外露出来:“处道,现在可不是讲究门户之见的时候了。你可想看着你父一生心血,最后落到功败垂成的结果?”

王厚则是更加坚定的摇头回应,“临洮堡不会有失,而家严回来时,河湟也依然会稳如泰山。现在当是镇之以静,不要让巩州上下陷入慌乱的境地。”

他说着,就站起身,向蔡延庆拱手行礼:“还请运使稍待时日。”

王厚旗帜鲜明的反对,蔡延庆瞪了许久,也拿他没有办法。虽然王厚的官位不高,但他的身份太过敏感。即便蔡延庆强命下面征发,下面有人想凑趣的呼应,也得掂量掂量王韶回来后的结果。

蔡曚得意起来,“运使,这事还是请朝旨的为好!”哈哈笑了两声,“眼下王、高二位久无音信,熙河经略司只靠着一个黄口孺子来撑场面,还是早点禀报朝中,选派得力之人来河湟!”

王厚冷下脸:“家严只是没消息而已,别真当他回不来了!”

从蔡延庆那里告辞出门,王厚心中郁郁难解。临别时,蔡延庆看过来的眼神,直如一块巨石沉甸甸的压在他的心头。原本他很被蔡延庆所看重,但这一下,两人的关系已经彻底冷淡了下来。

其实蔡延庆做得是对的,国事为重,权限之事当然得先抛到一边。为国而无暇谋身,蔡延庆的作为的确让人敬佩。

但韩冈的应对应该也是对的,他没有下令调动各处兵马,只是带着两千人去临洮堡,就是要维系熙河路的稳定。还没到最危急的关头,贸然调兵、征发,只会让巩州、乃至整个熙河路陷入一片混乱。一旦乱势成型,就很难再镇压下去。恐怕十数日后,就是中使带着命令河州前线撤军的诏令过来。

与蔡曚的龌龊心思不同,蔡延庆和韩冈的决断没有对错之分,只是立场不同而已。之间的取舍,让人难以决断。

王厚仰头向天,他之所以拒绝了蔡延庆,是因为他相信韩冈肯定能够将眼前的乱局处理妥当。

一阵清亮的鸣叫从天际传来,晴空之下,一行鸿雁正排着整齐的队列向北方飞去。鸿雁传书,王厚也盼着自己的话能传到韩冈那里去:“玉昆,一切都要看你了!”

……………………

就在景思立兵败身死的消息传到陇西城的同时,同样的消息,也让镇守在河州城中的苗授,连忙派人将姚兕姚麟都招了回来。

看过了韩冈让人送来的手书,姚兕依然他那张招牌的棺材脸,而姚麟则是失声而笑:“一切如故!……韩玉昆还真是敢说啊!河州城内外,兵马一万三四,他这一句可就都要让我们把这么多条性命交到他的手里了。”

苗授板着脸,不言不语,任凭姚麟说着。

姚兕咳嗽一声,先堵住了兄弟的话,这才问着苗授:“苗兄,你说现在该怎么办?退军,还是坚守?”

“该怎么做,就怎么做!”苗授回答着,“既然韩玉昆说一切如故,那贤昆仲就继续去清剿河州蕃部,而在下,也继续镇守这河州城。”

姚麟双眼一下瞪起,眉头挑起的角度凝着他心中的怒意:“苗都监,河州城这里可是有着近一万四千条人命!包括你我!”

“除非珂诺堡有失,狄道城失陷。不然我们的退路就是安安稳稳的,贤昆仲何须担心?”

“何须担心?”姚麟嘲弄的笑容,“苗都监,这临洮堡的情况在下是再熟悉不过了。有一段城墙外侧塌了一半,在下奉命与景思立交接时,还是没有给修起来。在禹臧花麻攻打临洮堡的那些天里,城墙上不知有多少暗伤。说不准什么时候就会垮下来。这样的寨堡,都监你说能守住吗?”

“……韩玉昆已经领军赶过去了,就算最差的结果也能保住结河川堡。”

苗授并不是很喜欢韩冈,但他信任韩冈,信任韩冈的能力。几年来韩冈的作为,让苗授相信他能维持住河州的安全。围在临洮堡外的西贼刚刚全歼了景思立所部,气焰正盛,但苗授就是相信韩冈有能力不让他们干扰到河州前线。

姚兕、姚麟都是外路将领,他们该挣的功劳也挣足了,就算熙河功败垂成,最后的罪责也压不倒他们两人头上。但苗授不同,他其实是王韶、高遵裕南下后,经略司中的最高官员。只是韩冈是文臣,能力又值得信任,所以王韶才将职权让韩冈带掌——虽然只是经略司中庶务。

但同在一个监司中,苗授与韩冈已是一荣俱荣,一损俱损。从目前的情况看,苗授只有选择支持韩冈:“……请二位放心。”

“北面的临洮堡危在旦夕,南下的三千军又是生死不明,苗都监,这样的局面你让我们怎么安心得下?”姚麟厉声质问。

苗授话声不徐不急,目光坚定异常:“王经略和高总管肯定能回来!”

姚麟嗤笑一声,正要出言讽刺几句,姚兕拦住了他。姚家兄弟中的长兄正色对苗授道:“苗兄,一旦结河川堡被围困,粮道就断了。自康乐寨同珂诺堡的山路,支撑不起一万四千人的粮草补给……所以在下只看临洮堡的结果。如果临洮堡失陷,为了帐下的几千儿郎,我兄弟俩肯定要撤退了。就算日后受到责罚,也比兵败身死要强。还望苗兄勿怪!”

苗授略作深思,最后点头,“……也罢,就以临洮堡为据!”

点头的同时,心中则在说着:‘韩冈,不要辜负了王经略的信任啊!’

