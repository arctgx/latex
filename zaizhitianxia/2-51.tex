\section{第12章 平生心曲谁为伸(五)}

【承诺中的最后一更,从明天起转为一天两更,时间分别是中午十二点,和晚上七点。另外,春节几天,俺会尽量保证一天双更,但更新时间不能确定。也许间或会少更几次,但肯定会在后面补上,请各位相信俺的人品。】

韩冈明说要跟窦舜卿过不去,给他找些麻烦。高遵裕和王韶想了一想,各自都默许了,但他们却没问韩冈到底要怎么做。

高遵裕是不想掺和,韩冈成功那是最好,窦舜卿也的确让他很是心烦;若是韩冈失败了,自己事先不知,也可以撇清干系。但要是多问了一句,说不定会就被韩冈趁机拖下水。

王韶则是对韩冈深有了解,知道他行事看似大胆无忌,实则稳重得很,若无把握,绝不冒险。而且高遵裕在这便,就算问了,他也不可能会和盘托出。

辞了高、王二人,韩冈回到勾当公事的官厅。他的四个同僚都不在,有两个是因为暑热故而告假在家,剩下两个今早韩冈还见着,现在却不知去哪里了。

而看到韩冈回来,官厅中的胥吏们纷纷上来行礼,态度明显恭敬了许多,不是过去的畏惧,而是真心诚意的敬服。

王启年曾经领着厅中公人跟韩冈过不去,而他在其他几个勾当公事面前则是曲意奉承。但今次王启年被窦舜卿杖死,他所奉承过的官人们连个屁都没放,就只有韩冈一个人冲到兵马副总管那里闹了一通,为王启年出头。跟着谁人比较让人安心,那是不言自喻的。

韩冈刚在自己位置上坐下,一名小吏就赶着上来,为他端上一盏用井水镇过的冷香饮子,陪着笑道:“抚勾在外被太阳晒得热了,这等饮子最能消暑解渴,抚勾喝两口消消暑。”

韩冈点了点头,接过茶盏。突觉身后又是一阵凉风送来,回头一看,另外一人正拿扇子给自己扇着风,也是堆出一副笑脸。

这两位都是王启年的跟班,过去是尽拍着另外一位跟着李师中的勾当公事的马屁,却很少搭理自己。今日韩冈倒是第一次受到这等待遇。

享受着习习凉风,韩冈喝了两口冷香饮。这等用草果、橘皮等药材烹煮出来的解暑汤味道的确不坏。放下茶盏,他问道:“今天厅中可有何公事急等处置?”

管理厅中文牍的文书走过来,半躬着腰,恭谨的说着,“抚勾你且安坐,小的们把事情理个头绪出来,就拿来给抚勾你批阅。”

韩冈还记得自己刚来的时候,就是这位文书,把厚厚几叠公文堆满了他的桌案,让韩冈他连个放手的地方都没有。摆在他面前的全都是繁芜琐碎鸡毛蒜皮的小事,但又不能不处理,韩冈费尽了心力,又从架阁库中查阅先例故事,对照着批奏,到了夜中方才处理完毕。现在倒是一反前态,帮自己进行预处理。

韩冈轻颔首,道了一句:“劳烦了。”

这位文书便是一副受宠若惊的模样,连声说着‘不敢,不敢’,转回去忙起了公务。

低头又啜了口涩中微甜的冷香饮子,韩冈微微浅笑。厅中胥吏对他改变态度也是意料中事,这也是他事先的计划。他今次刷了窦舜卿的颜面,也算是卖力了,不弄个一石数鸟、一举多得的收获怎么行?

经过今天一事,韩冈至少在勾当公事厅的胥吏中,有了说一不二的分量,而在整个州衙数百吏员中,他也是结下了个善缘。好歹是为了属下公吏跟副都总管顶牛的人物,秦州的官员中,没一个有他这等胆量,也没一个会有他这样的做法。

正在给韩冈打扇的姓蔡,给他端茶递水的姓武。

韩冈闲得无事,便随口问着他们,“蔡三,武大,尔等可知王启年家中境况如何?”

个头长得很正常,就称呼让韩冈觉得很好笑的武大立刻回道:“回抚勾的话。王八哥家中境况算是不错,也没二老要养,养活婆娘孩子就够了。他老子早死,他娘给他二哥养着。旧年跟两个哥哥分家产时分到了不少东西。家中现有一个结缡五年的浑家。生了一儿一女,大的是女儿,三岁。小的才半年。”

对于王启年家中的情况,韩冈已经事先了解过了,知道武大没说谎。他叹了一口气,道:“家里的顶梁柱走了,孤儿寡母的,日子过得也艰难。你们以前与王启年走得近,能帮衬便帮衬一下。而且他就剩个才半岁的儿子,打主意的不会少,小心不要让人蒙了他的家产去。”

“抚勾放心,小人理会得,小人理会得。”蔡三、武大连连点头。又笑起拍着韩冈的马屁:“抚勾当真是仁厚绝伦,不愧是孙真人……”

说到这里,话声就停了。两人惶惶不安,他们都知道韩冈不喜欢提这码事,从来都是绝口不认的。

“算了,下次注意。”韩冈宽厚的笑了一下,把手上的空茶盏推过去,“冷香饮子还有吗,再给我倒一杯来。”

………………

入夜后,普修寺中后院中,一株枝叶苍劲的老松正散发着一阵阵松脂的清香。韩冈坐在树下的一张石桌边,身边王舜臣打横陪着,下首处却是又黑又矮的王九坐着。

普修寺近着县衙,也近着韩家,主持也跟韩家关系匪浅,而且在夏天,这里十分清凉而又清净,韩冈是特意选了这个地方,来商量一些重要的事情。

石桌上摆着一些酒菜,香味随风飘散开来,但韩冈没动筷子的意思。

“消息都散出去了吗?”他拿着酒杯轻轻摇晃,漫不经意的问着。筛过的佳酿清澈如水,一轮皎洁的明月在酒杯中随着晃动聚来散去。

“官人放心,已经都散出去了。”

在韩冈面前,王九向来恭谨得很,一面石墩,他只斜签着坐了小半边。听到韩冈问话,就立刻站起来躬身回答。

王九和王五是亲眼见着韩冈是怎么从一个被逼着来服衙前役的穷酸措大,变成如今的韩官人的。韩冈翻云覆雨的手段,让两人从心底里感到畏惧。

吃喝起来向来不让人的王舜臣也没有碰菜,韩冈不喜坏人法度,他来寺中吃饭,不论酒菜都是素的。但王舜臣是喜欢大鱼大肉,根本吃不惯眼前一桌的清淡口味。

他现在反倒是对韩冈和王九的话感到兴趣,“三哥,你们到底在说什么?”

“我这是因势利导,顺水推舟。”韩冈不明不白的说了一句,算不上是回答。但他无意再多解释,“王启年为窦舜卿出谋划策,陷害与我,他是死不足惜。但他毕竟最后投了我,他的家人我却一定要保住。”

王舜臣闻言惊道:“窦舜卿难道要……”

韩冈摇头道:“不能是窦舜卿,要窦解才行。”他拿起酒壶,给自己斟满酒,“一定要窦解才行。”

韩冈说得没头没脑,王舜臣茫然起来,而王九心领神会:“官人放心。窦副总管位高权重,消息不容易传入他的耳中,但窦七衙内就不同了,他的几个亲近伴当都是能带上话的。。”

韩冈满意的点头,又提醒了一句:“该怎么把事情传到窦七的伴当耳中,不需要本官多说吧?”

王九嘿嘿笑道:“官人你放一百个心,俺当然不会当面明说。”

王舜臣越听越迷糊,听起来像是针对窦舜卿孙子的一桩阴谋,但他却想不通韩冈将会怎么做,他现在让王九做得事又是什么意思。

“三哥,你们到底在说什么?!”王舜臣又一次问道。

“在说怎么对付窦舜卿……他的孙子。”韩冈开了个小玩笑,接着他就正经起来,“虽然今次一战之后,王机宜的地位稳固,再无人能动摇,而且窦舜卿和李师中肯定要被调任。但窦舜卿总是跟本官过不去,不能就这么放着他大摇大摆的走,总得让他吃点苦头。当然……”韩冈笑了一声,“窦舜卿地位太高,本官顶撞他一下不难,但真的要跟他撕拼起来,还是有些难度。”

“所以三哥你就找窦七衙内的不是?”

“没错。”韩冈很干脆的承认道,“如果给我半年时间,就算是窦舜卿我也能让它变成向宝那个模样。但窦副总管很快就要走了,以他的年纪,日后也回不了秦州。一时之间,也只能拿他的孙子出点气了……”韩冈转过来对王九道,“一切我都安排妥当,现在就担心王九你那里出篓子。”

“官人安心等着看就好,左右小人也只是暗地里在市井中传两句谣言,怎么都不会有事的。”

韩冈听得满意,随即点了点头。王九是地头蛇,在市井中联系又多,酒桌上装作不经意的说上两句,很快就能把消息传开,到最后,也不会有人能查出究竟是谁起的头。

这么简单的事,王九自然不会推脱。但他并不知道,韩冈方才说的话其实是半真半假。

比如说窦舜卿快要离开秦州这件事,就是为了安王九的心才说出来的。人心隔肚皮,谁也不知道王九会不会起异心,韩冈不会自大到认为自己怎么说,王九他们就会怎么做。

韩冈心里明白,王九他们听话受教,是因为这么做能给他们带来利益,同时也是因为畏惧自己的手段。凭借着两点,韩冈一声令下,他们就把王启年给查了个通透。但要让他们跟着自己去与窦舜卿面对面的死斗,韩冈就不能保证王九等人不会转头去向窦舜卿告密。

“好了。”韩冈笑着劝过王九几杯酒,对他道:“你就先回去吧。把此事办妥当,日后我少不得保你个好位置。”

韩冈的保证现在就是金字招牌,他说过的话几乎都已经实现,王九千恩万谢的从后门离开了。

一等王九出门,王舜臣立刻问道:“三哥,你真正要对付的是窦舜卿吧?”

韩冈哈哈一笑,脸色阴冷下来:“还用说吗,这不是理所当然的!”

