\section{第19章 萧萧马鸣乱真伪(二)}

轻轻掖好了盖在王旖身上的被子,韩冈直起腰来。

眼前因怀孕而变得圆润起来的面容,此时正深深的陷入睡梦之中。呼吸轻轻细细的有着稳定的节奏,可见睡得很沉。只是双眉之间,仍然与清醒时一样,有着深深的纹路,不知在睡梦中又看到了什么。

王雱走了,连同他的未亡人和遗孤。在王雱的灵柩被搬上船之后,就会一路顺水南下,直放江宁。王家虽说出自江西临川,不过王安石的父母都是安葬在江宁,加上王安石已经过世的两名兄长都是在江宁入土,王安石的这一支,基本上已经从临川乡里分了出来。

早间送了王雱的灵柩上船回来,还不到中午王旖就坚持不住了。虽然她还在勉力支撑,想要装着若无其事的样子,可她这些天连伤心带疲劳,已经是消耗光了所有的体力,喝了点严素心做的热热的浓汤垫饥,就给韩冈扶上了床躺下歇息。

即将临盆的孕妇,本来就该避着大喜大悲,还有劳累。可偏偏遇上了王雱的事,根本就没有办法。也幸好王旖身体底子好,这两年也调养得宜,要不然当真会出大问题。

睡梦中,王旖嘴唇动了动,韩冈没有听清妻子咕哝着什么,但她的脸上出现了晶莹的水光。方才在汴水岸边的伤心,现在似乎还留在眼角。

韩冈心中怜意大起,抬手拭去妻子面颊上温热的泪水,抬手将被褥又整理了一下,再吩咐过房中的几名贴身使女和婆子好生照看着,这才轻手轻脚的出了房间。

韩千六和韩阿李此时还在正屋中,和周南、素心、云娘她们说着闲话。

之前审官东院给韩千六安排下一个秦州竹木务的官职,是从监察秦岭上的竹木砍伐运输的差事,算是又有油水的肥差。不过这不合韩千六的胃口,家都安在了陇西,没兴趣回秦州,直接推说老病,辞官不就——虽说曾想回陇西任旧职,但也不便自己提出来,否则就会引起不好的联想。

既然眼下审官东院的安排不合人意,夫妻两个就准备着在京城好好的逛上一阵,然后就回巩州陇西养老。

韩冈进门时看着父母,暗自叹着气。老夫妻俩好不容易上京一遭,自己这个做儿子却是东奔西跑,不说日常侍奉在身边,甚至连陪着他们去逛一逛东京城都做不到。虽是有着客观原因,但他心中依然免不了有些歉疚。

吃过午饭,一家人坐在了一起,两个孕妇得到了最大的看护。

似乎不久之前,韩云娘还是个楚楚可怜的小女孩,可现在她抚着自己挺起的肚子,脸上满是母性的光辉。她本是纤细的体格,生育时未免会有些让人担心,不过在孕期中很注意没有滋补过头,应该能撑过最危险的头胎。而二胎的周南,则是能让人多放心一些。旧年的花中魁首,做了母亲之后,一点也没减去曾经的颜色,反而增添了成熟的魅力。

看着身边的如花美眷,韩冈争强好胜的心思忽然间就淡了许多。听着父母妻妾的闲话,心中一片平安喜乐。比起捷报传来时的激荡在心中的喜悦,别有一番滋味。

“三哥哥。”云娘对韩冈说着,“昨天旖姐姐也说了。爹娘就在京城歇上两月再走。等过两个月,我们就陪着爹娘一起回陇西去。”素心和周南都点着头。

周南、云娘差不多都到产期了,王旖也就迟上不到一个月。有长辈照看着,一家人都能安心。而等她们坐完月子,正好可以跟韩千六、韩阿李回乡。

“不成,那时候也不过刚刚生产过,哪能随便出远门!”韩阿李摇着头,“就算你们吃得住,小娃儿又怎么吃得消?就在京中安心的住着,三哥儿很快就能回来。”

见韩冈开口要说话,韩阿李又道:“三哥儿你也别担心,之前几年不是好好的吗?也别担心俺们。义哥儿的浑家性子好,为人又孝顺,虽然是太后家的人,也没有一点脾气。有她在身边,不比你差多少。”

以三纲五常为圭臬的时代,都讲究着以孝治国。‘父母在,不远游。’就算不得不出外,如果父母身边没有兄弟照顾,那么至少要将妻子留下来服侍。

就像王安石,王雱刚刚去世,因为身边不能没有儿子孝顺,故而王旁被特旨调回京中,前两天刚刚进了粮料院为官。也因此,为王雱扶灵回江宁安葬的也是王安国而不是王旁。

韩冈此前的做法,虽然不像‘不为父母丧而请丁忧’那般干犯律法,但真要计较起来,还是会被人说闲话的。不过韩冈之所以能放心出来做官,并且还把妻儿都带出来,只留着父母孤身在陇西,也是因为有冯从义这位表弟在。

韩千六、韩阿李都把这位外甥当成亲儿子看待,冯从义虽然经常要出外奔波,但他的妻室都是留在陇西。

“倒是三哥儿你,去了广西之后一切要小心。上阵时,也不要再冲在前头。”这些年来,韩冈参战的次数多了,每每都能大胜而归。但要说韩千六和韩阿李能放心的下来,这是怎么也不可能的,“也别怪为娘多说,你做了几年官,上阵的时候也太多了。人家哪个做文官不是在衙门里呆着?就你充能,文曲星去抢武曲星的买卖。”

“娘放心。”韩冈赔着笑,“孩儿都是一路转运使了,都被人称作转运相公了,哪里还会再上阵?!只是运筹帷幄而已。”

“还说!”韩阿李一瞪眼,“外面都说你去邕州,以千五破十万。才一千五百人,你是怎么着运什么幄的?!”

韩冈干笑两声,没敢再辩。亲娘抢白的时候,最好闭上嘴,他这个做儿子的经验虽然不如父亲,也是知道的。

说了一通话,韩千六与韩阿李带着孙儿孙女进里屋午睡去了。

说了一阵话,周南和云娘也都累了,韩冈亲自一个个将她们送回房中歇息。周南和云娘几乎是同时怀上,产期最多再过几日就到了。韩冈在京城中还能留上几天,如果能亲眼看着她们生产,去广西的时候也能放心一点。

韩冈和素心走在家中的廊道上。这是韩冈回来后,难得两人独处的时光。

“这些日子累着素心你了。”韩冈说着。

严素心轻轻的摇了摇头,“奴家不累。”

她小碎步跟着韩冈往书房中去。

从侧面望着脸颊变得削瘦起来的丈夫,高挺的鼻梁此时越发的显得挺直。看着丈夫削瘦之后,更加明晰坚硬的下颌线条,都有些揪心的疼,到底有没有好好吃饭,如果是自己在身边,肯定会安排好日常饮食。回京后,又是一直忙着,没有好好调养。严素心想着,今天晚上要做些什么菜,给良人好好补一补。

只是看着韩冈到了书房,就吩咐下人去准备衣服,她一下醒过神来,惊讶的问道,“官人要出去?”

韩冈点点头,他下面的确还有一个宴席要赴。

纤细的手指轻轻勾住了韩冈袖口,仰着头,吞吞吐吐的,“能不能不要去。”

韩冈轻声叹了口气,如果只是官场上的无谓应酬,这个时候他当然会婉辞。但这是要宴请几个关学一脉的师兄弟,都是准备招揽入幕中,韩冈这个请客的主人怎么可能缺席。

“不能不去啊……”

韩冈自从做官之后,身边一直都缺乏幕僚辅佐。虽然不像包拯那样平居无私书,故人、亲党皆绝之,但也是一般的缺少幕僚、宾客。地位低的时候无所谓,可随着地位渐高,就如一路转运使的工作,单凭一人不可能面面俱到的处理妥当。

这是韩冈本身的个人情况造成的。幕僚少,是因为他职位的缘故,最需要幕僚处置公务的亲民官,他并没有做过多久。韩冈在熙河是幕职;在京中,又是任了监司;这其中就只做过一年多的地方官,缺乏幕僚辅佐也不奇怪。此外一般的高官显宦身边得力的幕僚,基本上都是从选人、京官开始,一路跟着十几年、几十年下来,其中运气好的,被推荐为官。哪里像韩冈,六七年间连着跳级,才投到门下的幕僚,不过一年就立了大功一起得官,寻常就算是宰相家的幕僚也不会有这份运气。

前面南下,走得太急,没能从京中的同窗中招来几个幕僚——不过就算当时是找了来,以邕州大战的功劳,多半也都是一起得官,韩冈身边也照样无人——现在有了点时间,在再次南下之前,他当然得将身边的短板补齐。

“不过不会太久……宴上要说正经事。”韩冈搂着美人儿厨娘的香肩,“夜里等为夫回来。”

严素心耳朵一下就烧了起来,就算嫁给韩冈这么久,她还是不能习惯丈夫突然而来的亲昵。只是要挣扎,又舍不得离开丈夫坚实的臂膀,垂着头,红得发烫的面颊只敢对着地面,过了半天才轻轻嗯了一声。

