\section{第九章 纵行潼关道(下)}

天色如晦,厚重的阴云几乎压到了中条山诸峰的顶上。

风也刮了起来。冬月的寒风如刀,浃肌透骨,在黄河边的潼关道上肆虐。

转眼之间,种建中便已是手足冰冷。他搓了搓手,对掌心呵了口热气,转头对着身边并辔而行的同伴道:“玉昆,看起来是要下雪了。”

种建中的话刚出口,韩冈脸上就感觉到了一点冰凉。仰头望着天空,玉屑一般的碎雪已经从云层中洒落,“不是要下,而是已经下了。”

漫天的雪珠,种建中也看到了,立刻道:“离前面的驿站还有五六里,得赶紧快点走了!”他回头,对着身后的一队随行车马吼着,“再加把劲,早点赶到驿馆中,有热酒招呼!”

一行人的行速立刻加快,挥鞭驭马,向着前面的驿站赶过去。

前日在长安驿馆中,遇到一年多不见的种建中,的确是个惊喜。本来韩冈以为种建中现在当是在京中苦读,准备来年的考试。谁想到投宿驿馆时,竟然当面撞上。

在去年横山之役结束后,种建中和种朴就跟着转调京中任职的种谔,一同去了东京城。种建中本人在京营之中也有了一份差事。不过,他为了参加明法科考试,今年六月后锁了厅。

种建中本也是准备着在京中读书,给韩冈的信中也是这般写的。但因为关中地震,便被种谔打发了回乡,看看老宅有没有在地震中受到损害。

前日碰面后,说起种谔的这个安排,种建中就有几分悻悻然的神色。这样看起来,可能是对于自家侄儿跑去考明法,种谔的心中有些不高兴的缘故。

在韩冈看来,种建中若是考得进士倒也罢了,能考中进士,就算是将门世家肯定也会大肆庆祝。但种建中却考得是明法,日后连转官都有难度,还不如留在军中。

但种建中心意已定,却也没法劝。韩冈提了个头,见到他不想多言,便也罢了。一起上京,正好做个同伴。不过韩冈、种建中的同伴不仅仅是只有对方,另外还有一人。

行不过三里,风雪已是劈头盖脸,有越下越大的架势。韩冈自叹命苦,总是轮到在冬天进京,每次都要遇上这么一场雪。

这时一骑远远的从前方奔来,隔着老远就喊了起来,“韩三哥!十九哥!快一点呐,俺已经在前面的驿馆订下了酒菜和房间了!”

这是种建中的弟弟种师中,今年才十五岁,今次跟着种建中一起进京。

听到种师中这个名字,韩冈就想起了种师道。可惜种家现在查无此人,不知是不是日后改了名。

今人改名也很常见,或是犯了讳,或是嫌着不吉利,很轻易地就可以将名字给改了。前任宰相陈升之,本名为旭,升之乃是表字。如今改用旧字为名,却是为了避今上的讳。

韩冈看了看已经跑过来的种师中。十五六斗少年郎正袖着双手,骑在马上连缰绳都不握,纯凭脚力控马。只论骑术身手,到也有几分后世名将的谱。

也许他就是日后的种师道吧……

不移时就已经到了驿馆处。这是潼关中道的小驿馆,只有两重院落。因为时近腊月,潼关道上行人甚多,此时已经是人满为患。但韩冈和种建中都有官身,连着种师中,他身上都有一道荫补来的官诰。三人拿到一间上房,都没费什么口舌。还是韩冈无意以势压人。要不然以他的朝官身份,能把随行伴当都安排了单间。

让伴当上去整理房间,韩冈和种家兄弟在正厅中打算找个位子坐下来。只是厅中满满堂堂,有几十百姓坐着蹲着。不似行商商队那般以青壮为主,而是老弱妇孺一大家子。粗粗看过去,在不大的正厅中,竟有七八家之多。

“是流民。”种建中凑过来低声说道,“华州的。”

韩冈点了点头。

自从走上潼关道,这一路过来,看到了不少华州流民。他们也不是穷的叮当响,绝大部分都还有个包裹,在驿馆中,还能有个座位。在驿馆院中,还有好几架小推车的,上路时,孩儿坐在上面,包裹家当放在另一边。

韩冈三人进厅,原本占着一桌的客人,便被驿丞请开。韩冈看了看起身离桌的五人,有老有少,有男有女,正是一家。

韩冈招了招手,当家的老头子变过来了。

“小老儿孙福,拜见两位官人。”

老头儿黑黑瘦瘦,在韩冈和种建中面前毕恭毕敬的。前面驿丞的态度,已经说明几人的身份。

“尔等可都是华州人氏?”种建中问着。

孙福恭声回道:“回官人的话,这里的八户人家都是从华州来的。”

“老丈先请坐下来说。”韩冈和气起来,便是没有半分架子。等老头儿诚惶诚恐的坐下后,很和气的问着,“地震山崩已经是两个月前的事了,怎么还会出来?”

见着韩冈没有摆出官威,孙福的胆子大了一点,叹起气来:“实在等不到官府的救济,不然谁还愿意离乡背井。”

“为何不去京兆府?”韩冈问着。

潼关道三百里,一路走到洛阳不知会累到其中多少人。而向西去长安,就只有两天的脚程。远近有别,为什么会选择一条远离家乡的路

孙福长叹了一口气:“官人如何不知,如今的长安城已经没粮可放了。”

韩冈听了一惊,“这事你是从何得知?难道已经去了京兆府不成?”

“小老儿没去长安,也是上路时听人说的。”看着韩冈可能不信,孙福又急道,“华州都是在这么说,从乡里出来的,就没一家去长安。”

韩冈与种建中交换了一个眼色,的确,他们在长安并没有看到流民扎堆的情况。

又问了几句闲话,孙福就很识趣的告辞。

等他起身离开,韩冈便皱起眉头:“长安怎么会没粮了?今年关中又没有遭灾?”

“欺上瞒下的事可还少了?那个地方的粮囤不养了一群耗子?!”种建中愤世嫉俗的说了两句,却又沉吟起来,“但这是长安啊,怎么会先没粮……会不会是为了明年便民贷的本金,所以不肯开仓?”

“不至于的。郭太尉不会如此不智!”

虽然种家跟郭逵关系不睦,但种建中也承认,郭逵怎么都不可能糊涂到为了,而不出手援助华州灾民。

那么,长安无粮的消息又是从哪里传出来的?要知道长安的粮仓数量,是为关中之最。

照着司农寺制定的便民贷款的条例,常平仓再怎么向外放贷,最少都要保证三成上下的仓储。就像是后世的银行准备金,不会全部都砸出去。加之如果放贷数量不足,还有抑配——也就是强行让富户来借贷——这一手段,基本上只要不是碰到席卷一路的大灾,便民贷款可以说是旱涝保收,并且常平仓依然能保证一路民生不至于有大的危险。

秦凤路的确是与关中分家了没错,但韩冈一年多前,就在陕西宣抚司待过,至少知道一点长安这边的永兴军路转运司的情况。白渠灌区的歉收,虽然使永兴军路这两年军备不振,无力用兵,可也不会让灾民饿着肚子。

“从长安过来,没有看到流民。可见这消息的传播效率之高,让所有的华州流民都往东去,而不是往西行……无头流言能一下驱动了所有人,若说是无人在后兴风作浪,未免有些不合常理。”

不过若真的有人传递谣言,驱使流民前往关东,那他们胆子未免就太大了一点。

“现任的京兆尹不是郭逵吗,谁能在他面前玩花样?”种建中拿着韩冈方才的话来反问。

“所以想不通啊,山崩看似厉害,但华州的灾其实并算不重,只要用心一点,华州本州都能自行解决。”

今次的地震其实并不算很厉害,少华山阜头峰崩塌,也是日积月累的结果。之前的几十年,有过多次落石伤人毁屋的记录,能迁走的几乎都迁走了。

韩冈一路行来,可以看得出,道上流民的人数很少。如果是有心人在后使坏,按理说不可能影响到新党的地位,只不过,出了潼关道后,那一边,可就是洛阳河南府了。

韩冈沉吟着,种建中、种师中安安静静的坐在一边,没有打扰他的意思。

想来想去还是无法确认,抬头自嘲得笑了笑。也许是平日里勾心斗角太久了,总是免不了要往人心险恶的方面去想。他对望过来的种建中道,“也许当真是长安的常平仓已经缺粮了。……”脸色又沉重起来,“不过那样的话,关中可就危险了。”

不同于用谣言煽动起来的流民,只需要及时派人在函谷关口安抚住就能解决,若是长安城的常平仓空了,来年开春后的便民贷成了笑话不说,关中都将陷入危机中。

作为关中核心之地的常平仓都空了,难以想象永兴军路转运司辖下的其他军州,那些地方仓囤会是什么样的情况!

而且关中因为要提防着党项人的侵袭,对粮囤的检查一向最为严密。换作是京东、京西,或是江南诸路,那些没有军备压力的地方,也许会更糟糕。

韩冈现在不知道,哪一个猜测会是真相。可不管是哪一项是真的,对新党来说,都会有些麻烦。而且最麻烦的是两者皆为真。京兆府常平仓的确无粮,而别有用心的消息散布者也确实存在。

那样的情况,恐怕身为宰相的王安石都要好一番头疼了。

