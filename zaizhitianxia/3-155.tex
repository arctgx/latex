\section{第41章 礼天祈民康(二)}

“此事还不一定。”韩冈摇了摇头。

他知道孙永到底在说什么,不是孙永做得累了想要走人,而是他韩冈在开封府衙中待不久了。

韩冈过去能熟知朝中之事,不光是靠了王安石和王雱的来信,也有王韶的帮忙。枢密副使通风报信,韩冈的耳目照样能直上朝堂。

韩绛举荐他韩冈为中书五房检正公事的消息,也就两天的时间,便传到了韩冈的耳朵里。

对于此事,韩冈并不准备瞒着孙永——他和王韶的关系,朝堂中谁会不知道!?

因为罗兀城之事,韩冈其实并不怎么喜欢韩绛。不过到了咸阳城破,叛军出降后的那段时间,韩绛却是很配合的将三千多广锐叛军,很妥善的一批批的送到了熙河路。

以韩绛当时的权力,他将这些叛军全数处决了都没有任何问题——环庆路经略安抚使王广渊,当时可是一点证据都不要,就杀了两千多据说有叛乱迹象的士卒——但韩绛却是遵守着诺言,让熙河路得到了如今支撑路中汉人势力的一个极重要的支柱。

就是靠着广锐军这点残部,韩冈在河湟拓边的过程中屡立战功,不论是在渭源堡,还是在珂诺堡,韩冈指挥的几番大战最后能得胜,几乎都是广锐军的功劳。从这一点上,韩冈就要多谢韩绛。

韩绛现在的举荐,并没有摆出施恩望报的态度,而似乎是一片忠心的为国考量,韩冈说不得就要承他的人情。

另外,韩绛并不仅仅推荐韩冈为中书都检正,甚至隔了一天,就加了一笔,又荐了韩冈为判军器监。这不合规矩,但王安石过去这样荐过曾布、也同样荐过吕惠卿,有先例在,韩绛依样画葫芦的举荐韩冈,当然也是一点问题也没有。

对于韩绛对韩冈的举荐,吕惠卿能反对吗?

他不能。

除了在年龄上做文章以外,吕惠卿找不出任何理由来拒绝韩冈。不论从功绩、还是能力、又或是官阶,韩冈都不逊于甚至要胜过当年担任中书都检正的吕惠卿。同时,韩冈对于新党有恩、有亲,世人都看在眼里。吕惠卿可以不加以举荐,但当韩绛推荐了韩冈之后,他则不能加以反对。

冯京、王珪有反对吗?其实也没有。

冯京、王珪这一相、一参,多半是乐得要看韩绛和吕惠卿打擂台,坐视新党自行分裂。新党分裂,朝堂上必乱,韩、吕这一鹤一蚌让天子感到失望,到时候,当然是渔翁得利。

所以这项任命,在中书和崇政殿之间的一套流程走得很快。天子批红、宰辅签押、御宝一盖,最多再过两天,韩冈的新任命就要下来了。

“难道玉昆你不愿意?”孙永追问,意味深长的笑道:“难道认为韩子华的举荐不妥?”

韩冈抿了抿嘴,“也不能这么说。韩相公的举荐,韩冈当然是铭感五内。只是愧不敢当啊!”

孙永呵呵笑了笑,低头喝了口茶,“玉昆你任此职若有愧,何人敢说无愧。”

韩冈沉默了下来,不是在想韩绛的举荐,而是在猜度着孙永的心思。

对于这一项举荐,尤其是举荐人的身份,韩冈说惊讶也惊讶:韩绛没跟他打招呼就将他给推荐了上去,让韩冈觉得有哪里不对劲。

不过要说有多惊讶,也还不至于到惊骇莫名的程度,前两天听说此事之后,他也只是啧啧嘴就过去了,眼皮都没有跳的。

论起能力,朝中能坐稳中书都检正这个位置的绝不止韩冈一个,而论地位,论声望,论功绩,也都有着复数的人选。但将数者合一,真正细论起来,正担任着府界提点的韩冈却是排在最前面。

韩绛推荐韩冈,这一封荐书,这一个人选,从各方面来说,都是是无懈可击、无可挑剔的。

但其后的用意,也是人人都看得明白。不仅是韩冈,他的三位幕僚,加上王旁,都是一眼就看了出来,韩绛这是要跟吕惠卿争夺对新党的控制权了。

毕竟是宰相,韩绛怎么都不会愿意看着吕惠卿把持朝政。天子注重新法,所以多加采纳吕惠卿的意见,但他韩绛也是支持新法的,难道他不能取代吕惠卿吗?!他可是宰相!

韩绛这点小心思,根本是不瞒人的,说不定天子赵顼都能看得明白。

只是孙永为何提及此事,难道是投靠了韩绛?这个念头一起,韩冈心中立刻给否定了,孙永是潜邸旧臣,背后是天子,没有必要投效任何人。可是韩绛的兄弟韩维也是潜邸旧臣,与孙永当有一番交情在。若是韩维居中搭桥,也不是没有可能。

孙永却饶有兴味的看着韩冈的沉默,年轻人少有三思而后行的,能思虑周全的并不太多,但韩冈却做得很好。不过顾虑得太多却也不是什么好事。

韩冈很快则抬起头来,正视着孙永:“吃苦受累了一年多,大府方才所叹,韩冈也是深有同感。而中书事务之繁剧,并不在开封府之下,韩冈想着能先清闲个几日。”

孙永一下惊道:“难道玉昆你打算出外?”

“下官不敢欺瞒大府,升官如何不愿?但中书五房检正公事,韩冈自知不能胜任。但那判军器监一职,下官自问还是有些把握,不会愧对天子。”

孙永是韩冈的上司,赵顼打算调动韩冈的时候,照常理也要征询孙永的意见,以及要听取孙永对韩冈的评价。这是应有之理,韩冈现在对孙永将心中的想法说出来,也是有着让他代为传递的心意在。这等事不足为奇,想必孙永也能明白。

孙永听的确明白了。韩冈这是不想给韩绛当打手,也不想变成新党分裂的开端。所以打算辞一职,受一职。留在京城中,但不会跳进漩涡里。

“这样也好,玉昆这一年忙得事情也多,稍稍清闲上一段时间,也不算是坏事。”

“多谢大府垂顾。”韩冈拱手说道。

雪越下越大,铺天盖地的压下来,下来两三个时辰都没有见到停歇的迹象。韩冈和孙永不得不在青城行宫中逗留一晚。当然,作为臣子,两人不能在殿阁中居住。这一天晚上,他们和一众随从都给安排在了宫门内的房间——这也是郊天大典开始之后,普通官员居住的地方。

遣了人冒雪回城去报信,并为明天的朝会请假,韩冈和孙永就住了下来。一整夜听着狂风呼号,被风鼓动的暴雪不断敲打着门窗,寒风从门缝窗中透进来,让孙永、韩冈不约而同的想着回去后就安排人手,整修青城行宫的驻地。

到了第二天午后,下了一天的暴雪方才宣告收止。地上的积雪厚达三四尺之多,孙永看着堵上了殿门的雪层,差点就要哀声叹气起来。

不过他也知道叹气没用,急着要会城去,点起人手来清扫道上积雪。这件事情不能拖,越拖越是麻烦。而且暴雪之后,城中民居都少不了会有坍塌,砸死住户的情况每年都没有少过,这些事,都要他这位开封知府来调动、来处置。

看着孙永在行宫正门口急得团团转,来回左右的踱着步子,每走几步就要望着行宫外看上两眼,韩冈不由得就有些觉得好笑。最后忍不住出言安慰道,“大府放心,城中此时肯定也在急着,想必很快就有人来接我们了。”

也的确正如韩冈所言,大约一刻钟之后,从北面东京城的方向,的确来了三辆马车。两匹马在前面拉着,后面的车斗下装得不是车轮,而是两根长长的木条。

见着城中的下属,找了雪橇车来接自己,孙永紧绷的神情终于放松了下来。

与韩冈一起坐上同一辆车,前面一声皮鞭响过,雪橇便在雪地上顺滑的开始行驶起来,没有寻常马车的摇晃,也没有寻常马车吱吱呀呀的轮轴转动声,平稳而平静。

坐在安安静静的车厢中,车厢下方只有橇板碾过雪层的丝丝微声,孙永神情忽然一动,问着坐在对面的韩冈:“玉昆,你是不是还有什么藏着掖着的什物没有拿出来?要不然为何只要做着判军器监?”

孙永越想越是这个道理,但凡官员,无不喜欢清要之职。不做事、干拿钱、对朝廷大事又能指手画脚的职位,那是人人喜爱。而那等事务繁剧的职位,就没人喜欢去做。

可不论是中书五房检正公事,还是判军器监,其实都是忙碌而不得清闲的职位。韩冈虽然说着要闲职,但他接下判军器监的职位,从情理上是不想参合政事堂中的纷争。不过理由要是这么简单,也未免太小瞧了如今名震天下的韩玉昆了。

韩冈抬眼看着孙永,见这位开封知府盯着自己不肯放过,叹了一声道:“韩冈承袭横渠先生之教,研习格物致知之说,的确甚有心得。判军器监虽非合意,但也是与韩冈所学有些瓜葛,若能执掌其事,当不会让天子失望。”

