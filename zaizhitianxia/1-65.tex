\section{第29章 君意开疆雪旧耻(下)}

赵顼、王安石君臣两人的对话就这么一直持续着,从西北边事,一直说到江南纲运。只有文彦博会瞅准时机主动出头来攻击王安石,曾公亮、陈升之等人则如同土石木偶般站在一边。如果不是赵顼偶尔会向他们询问一些问题,几位宰执官怕是要沦落成纯粹的装饰物。

王安石自任参知政事以来,虽然还没升任宰相,但由于赵顼的信任,中书权柄已尽在他手。政事堂中的宰相执政本有五人,宰相富弼、曾公亮,参知政事王安石、赵抃、唐介。不过曾公亮老迈不理政事,富弼因与王安石政见不合而告病不出,赵抃能力不及,总是在叫苦,唐介则与王安石几次君前辩争不过,气聚于胸,发疽而死,唯有年富力强的王安石生气勃勃,独力处理着所有的政务。故此世间便有了‘生老病死苦’的笑话——王安石生、曾公亮老、富弼病、唐介死、赵抃苦。现今政事堂中又换了几人,但王安石执掌中书大权的情况依然不变。

崇政殿中的奏对一直持续到近午,需要君臣商议的政事处理得差不多。沉默得跟块石头没两样的首相曾公亮终于开口:“已近午时,臣等不敢耽搁陛下进膳,臣等告退!”

首相发话,殿中重臣便齐齐告退。赵顼也不留他们,只犹豫了一下,对王安石道:“王卿,你且暂留一步。”

王安石依言停步,其他宰执照样出殿离开。自王安石从江宁入朝之后,单独奏对的情况太多了,多到无人感到惊讶的地步。

王安石站在殿中,等着赵顼说话。赵顼从御桌上的一摞奏章中,抽出做了记号的三本来,着站在身边小黄门将之递给王安石。

王安石展开一看,却是昨日他签书过后,随着其他重要奏章转给赵顼过目的三封荐书——秦凤路管勾机宜文字王韶、雄武军【秦州】节度判官吴衍,同举荐秦州成纪县布衣韩冈入官,为秦凤路经略安抚司勾当公事,兼理路中伤病事宜,而秦凤路都监张守约也同样举荐韩冈,不过只有后一项。

王安石只看了几眼便抬起头,他知道赵顼想说些什么。

“王卿,你说说王韶这年来到底做了些什么?!”赵顼的声音中透着隐隐怒意。

关西的主战方向进展顺利,但预期中的侧翼,却没有什么动静。王韶去了秦凤一年,如今给出的成绩却是一份荐书!赵顼是顾忌着一直对王韶青眼有加、大力支持的王安石的脸面,所以方才才没有当众斥责,但现在还是要说出来:

“王韶三人所荐的韩冈才不过十八岁,连个出身都没有。难道要朕给一个从九品选人下特旨不成?秦州就没有其他人才了吗!?”

年龄不到,不得任实职,这是朝中通行多年的任官制度。除非是有功名在身——如进士、明经等科——不然为官者未及二十五岁,虽可以有官身,但却不得拥有差遣。也就是挂个官名,领些俸禄,却不能出来做事。

大宋开国百年,对臣子越发的厚待,高品的文臣武臣都可以荫补子孙,宰相和执政的子弟,往往才十来岁甚至八九岁就能得官。如果给这些乳臭未干的小孩子实职去做事,国家政事便要出大乱子。所以过去有定例,进士、明经及武臣以弱冠【二十岁】为限,荫补以二十五岁为限,低于此不得任实职。除非有多人同时举荐,否则就必须等到年限,才会有差遣。

可如今荫补得官的越来越多,身为官宦子弟,找几个父辈的亲友同时举荐也很容易,所以旧有的任官制度已是名存实亡。有鉴于此,王安石出手对任官法做了调整。依然还是以二十岁和二十五岁为界,过此才能得到实职差遣。如果要想例外,却不再是多人举荐就能成功,而是惟有请天子亲下特旨。

这条法令是刚刚修订,尚未颁布天下,王韶、张守约等人不知其中缘由,将才十八岁的韩冈荐了上去。依旧例,有三人同荐,年未弱冠的韩冈完全可以担任实职。但按照如今的规条,韩冈如果得不到赵顼特旨,纵能有个官身,却不可能得到差遣。

对于国中的大部分官员来说,干拿钱、不做事的生活,其实也不差。士大夫们都喜欢诉讼简、物产丰的州县,如果要天天审案、还弄不到一点油水,那做官还有什么意思,却是人人都避之不及。但韩冈不能出来做事,那王韶、张守约举荐他又有什么意义?

王安石对此看得很明白,所以才把王韶等人的荐章递了上来,请求天子的特旨。若非如此,这三份荐章根本不用递到赵顼眼前,依朝中制度,低品官员的任用本不需要天子过目,政事堂直接就可以处理,韩冈才一个从九品,哪要劳动到赵顼烦心!?

天子躁怒,对许多臣子来说,就是雷霆压顶,可王安石神色如常。他是秉持着疑人不用,用人不疑的态度。王韶在西北河湟的前景被他看好,同时赵顼也一样给予很大的希望。虽然因为宋夏两国正因绥德城的归属,在横山东段的无定河流域随时可能爆发大战,需要的粮饷资材都是个天文数字。朝中已无法给秦凤、给王韶太多的物质支援,但至少在人事上,王安石准备尽量满足王韶的要求。

“韩冈虽年少,然其才卓异。如果他是世家子弟,或可谓其中有情弊。但臣见王韶荐章,只云其为灌园之后,不闻有何家世。且此次举荐韩冈,不仅有王韶,还有雄武军节度判官吴衍,以及秦凤都监张守约,一名灌园之后,能同时得到他们三人的荐举,不可能是靠溜须拍马而来。”

“王韶在荐章中也曾有说,韩冈押运辎重,于峡道遇贼,亲斩不用命者二人,驱使民伕抗敌,大败数倍蕃贼,斩首三十余,其勇武可知。在甘谷城,不待命而救治伤病数百,其仁德可知。在秦州,又破西贼内应之奸谋,其智计可知。韩冈虽是年少,但行为已有大臣气度,陛下不可以年幼轻之。”

王安石如今正得圣眷,赵顼将之视为师长。不管有多怒,往往都会被王安石说服。他略作沉吟,最后点头同意道:“那就依王卿之言。不过是个从九品,许了王韶也无妨。”

“陛下圣明。”

王安石脸上闪过一丝喜色。王韶与李师中向来面和心不和,同时又因为提举秦凤蕃部事务侵占了都钤辖向宝的职权范围,而与其龃龉甚深。有李师中和向宝压着,敢与王韶结交的秦凤官员,只有聊聊数人。一年以来,王韶在秦凤的工作完全没有进展,也便是因为这个原因。不过如今王韶他能让节判吴衍以及都监张守约同时举荐一人,可见他在秦州的局面终于打开。

王安石不知韩冈的底细,还以为吴衍和张守约的举荐是因为王韶而来,从已有的信息来推导,得出这样的结论很正常,不过韩冈本身也肯定有点能力,否则王韶绝不至于推荐他。

如今天下官多阙少,往往是三四个官争一个位子。选人入京待选,都必须在流内铨候阙【等候职位差遣的空缺】,而新晋选人,更是必须去流内铨缴三代家状。同时还有时间限制,必须在四季的第一个月,也就是元月、四月、七月和十月这四个月的十五日以前在流内铨登记,才能排得上号。不然,就得等下一个季度了。

王韶在秦凤路已满一载,从来都没有举荐他人,由此便知他行事有多谨慎。可现在对韩冈,他不但荐了官身,还把差遣都给定下了,可见王韶对十八岁的韩冈信心有多足,或者说,他对韩冈的才能有多渴求。

通过王韶的奏章中,王安石倒是对韩冈有了点兴趣。一个出身贫寒的士子,通过不懈的努力,发挥自己的才能,最后得到高官的认可。类似的故事在世间流传得很多,远的不说,自幼丧父的范仲淹,画荻习字的欧阳修,都有过这样的经历。但他们获得名声,靠的是诗词歌赋和文章,不是像韩冈,靠的是勇武、才智以及胆略……还有仁心。

对王安石来说,诗词歌赋不足为凭——尽管他已是当世最顶尖的文学大家之一——大宋需要的是秀才,而不是学究。有才能、有冲劲的年轻人那是越多越好。即便韩冈只有十八岁,只要多了几年经历,在地方、京城做过几任,未必不能成为栋梁之才。

这段时间以来,曾经举荐过王安石的那些老臣、友人逐步走向了他的对立面,现在他最喜欢任用的就是有冲劲的年轻人。王安石所着意提拔的吕惠卿、曾布、章惇以及王韶等人,在官场中其实年纪都不算大,任官都不过十年出头。泛着腐臭味的祖宗之法,许多人在宦海沉浮多年后都已经习以为常,如果没有年轻人来冲击一番,这个大宋朝只会渐渐的腐烂下去,直到灭亡。王安石的那份吹响变法号角的《百年无事扎子》,说得便是此事。

大宋百年无事,那下一个百年呢……又会如何?!

ps:宋代有制度在,不会让那些才七八岁的荫补官入职,而韩三年纪不到,只能求个特旨了。好了,铺垫章节结束,视角转回到韩三身上,不要以为他领了便当。

今天第二更。

