\section{第20章 冥冥鬼神有也无(12)}

“玉昆,想不到市井之中把愚兄说得如此不堪,当真是羡慕你啊。”

当市井中突然冒起的流言,传到章惇、韩冈两人耳中的时候,两人对坐在一张坐榻上,正悠闲的下着围棋。从盘面上看,章惇明显占优,几块棋连在一块儿,韩冈的棋面则是支离破碎,不成样子。算起还棋头的子,至少就要多上四五个。

韩冈低头看着棋盘,注意力全都放在棋上,专注的眼神似乎在说交趾人的离间计根本不值一提:“交趾小儿技穷了。”

章惇抚掌大笑:“看来他们当真没了招数,竟然想使出离间计来。”

“没错!”韩冈从棋盘上抬起头来,重重一拍坐榻,“交贼技穷,只能用上离间计。但传得这么快,想必定是有些细作潜藏于邕州城中。”

章惇沉吟着点点头:“该让苏伯绪好好的查上一查了。”

妒贤嫉能,虽是恶评,其实也算是人之常情。寻常人看见在自己在意的地方比自己要强的人,都会少不了有那么一点嫉妒。能心胸宽广得毫不介意,几乎是百中无一。但不能控制自己心中嫉妒,让情绪左右自己的行事,那就是庸人了。

章惇绝不是庸人。他是南征主帅,不论韩冈、燕达、李信立了何等功劳,他都是能拿到最大的一份,只要赢了此战、打进升龙府,必定有个枢密副使的位置。他没必要去表现自己,就算智胜韩冈,勇胜燕达,武艺犹在李信之上,可所有的事难道还能都靠他自己来做?还不得依靠韩冈等人。

摆正了心态,确定了自己的位置,章惇根本就不会将谣言放在心上,何况他也有足够的自信。若是没有这份自信,他又怎么可能考中进士之后,又去重新再考一次?那可不比中个状元容易。

低头看看棋盘,韩冈方才重重的拍了一下坐榻,让满盘的棋子全都移了位。章惇拿着手上的扇子敲了敲棋盘,“玉昆,这局该怎么算?”

韩冈哎了一声,瞧着乱掉的棋面一脸遗憾的摇起了头:“本来还想翻盘的,这下看来只能做和论了。”

章惇盯着棋盘半天,抬头又瞅瞅韩冈:“玉昆,你这棋品倒是跟你岳父一模一样啊。”

韩冈哈哈笑了两声,权当做没听到。前面连输两盘,而且都是十几个子以上的大败,连输了做彩头的十几瓶家里送来驱散蚊虫的香精,再输一盘可就没有了。

章惇一拂棋盘,将棋子收进棋盒,笑问道:“彩头不会浑赖吧?”

“愿赌服输,输了多少,韩冈当然就会认多少。”韩冈毫无愧色的说着,“转头就将香精给子厚兄送过去。”

赢了韩冈不少贵重的香精,章惇也不贪,不逼着韩冈再来一盘,只笑道:“玉昆你家的香精倒是好东西,比起你旧时送来的花露更好。不但身上添香,还能避着蚊虫,只可惜太少了一点。”

“若不是南下,小弟可不想在身上擦着这些东西。”韩冈摇头感慨着,“本来都是作为药物的,现如今全都变成了喝的、用的。全都偏了初衷。”

加了薄荷、冰片的香精是夏天驱虫用的,浸了桂花或是蔷薇的花露则是女性的化妆品。若是问价格,却是既无价也无市,根本就不对外出售。韩家在自家庄子里的作坊生产出来,只是平常用来赠送亲朋好友罢了。如韩冈这次南下,章惇、李信的份都有带上。

这也是冯从义的计划。先低调的生产几年的时间,也逐渐将玉露香精的名声打起来,之后再拿出去贩卖,如此才能赚到大钱——这个时代也是认名牌的。若是随随便便就推出去,名气还未确立,工艺技术什么的早就会被有心人偷个干净了。远比不上冯从义的计划,让顺丰行出产的香精走名牌高价路线,就算日后技术被人学了去,也只能做个山寨。

章惇本想着问一问香精、玉露的方子,但想了一想之后,还是放弃询问。这方子多半是韩冈家日后传家的宝物,能保证几代富贵的,自己贸然相询,说不定会闹得不愉快起来,还是不冒这个险为好。

与韩冈一起,一枚枚收拾起棋子,章惇就忍不住倒着肚子里的牢骚:“丰州有郭逵领军,加上又是多达六万的重兵。这雷霆之势,不是等闲可比。一月不到便大事抵定,还多了四百皮室军饶头,当真是好福气。我们可是正式的经略招讨司和行营,却别说六万重兵,连两万都没有。援军就更不指望了,能将甲胄、兵器都给我备齐就好了。”

“现在还多说什么呢?看情形都要跟辽国撕破脸了,北方正是风声鹤唳,还怎么指望援军?”韩冈就在棋盘上将黑白棋子分开,直接就将白棋一起扫进棋盒,“好不容易合计出来一个更戍法,只因为契丹人一动,一下子就成了没影的事了。”

“更戍法日后还是有用的,天下禁军练兵都是少不了的。就不知道日后是河北军去关西练兵,还是关西去河北练兵。”

“或许是各练各的兵,整个北方都没有一块清静的地方。”

章惇盖上盖,将装满黑子的棋盒往桌上一放:“这不都是玉昆你的功劳?”

“子厚兄说笑了,何干小弟之事?”韩冈也将棋盒放好,“小弟可没有这个本事。”

“六十万带甲,任谁听了都心惊胆战。不是玉昆你的功劳?”

甲胄对于军队的意义极其重要。有甲护身和无甲护身,士兵们的士气和胆量有着远远的差别。而世间通行的律法中,也体现出了这一点。私下里藏刀藏枪都不算什么,但要是收藏个两三套甲胄,脖子就可以送到刀斧下了。六十万铁甲大军,大宋周边的诸多国家加在一起,也不一定能有五分之一。这就是韩冈的功劳。

不过韩冈可不会承认是他的发明让辽国心生畏怯,进而支持起西夏来,“这可算不上,是中国国势渐长,才逐渐有了压倒契丹的势头。”

“那加上飞船、霹雳砲和雪橇车呢?”章惇神色一动,似乎想起了什么,“对了,前两天工匠已经带着飞船都到了桂州,有了飞船,上阵时,也能派些用场。加上工匠们的手艺,攻打升龙府也更容易了。”

“十几名工匠病了一多半,还要在桂州疗养几天。”这些工匠都是韩冈特意从京中要来的,但他们南下之后还是水土不服,只能用上几日加以调养,“等他们休息好了之后,就召他们南下过来,希望他们能赶上出兵。”

章惇正要说话,这时门外的亲兵提声通禀:“学士、龙学,归化州的侬智会到了,正在外堂求见。”

广西经略的表情为之一变:“侬智高的兄弟来了,玉昆,可要一起去会他一会?”

侬智会是侬智高的亲兄弟,如今却是执掌着边境的归化州。

说起侬智高之叛,若说错,当时的朝廷肯定也有错。侬智高一开始是向交趾的称臣,因为受到交趾盘剥,意欲归附大宋。但朝廷因为侬智高是交趾的臣子,担心收他内属会惹来边衅,同时又担心给了侬智高官职后,他会藉此收服左右江诸峒,最后冒出第二个交趾来,便几次三番的加以拒绝,而不是设法从中挑动侬智高与交趾为敌,以夷制夷。

由于朝廷不肯接纳,在交趾那里侬智高受到的压力也越来越大,最后他也就干脆了当的举起叛旗。这一打,便将大宋在两广的老底给揭破了出来,最后还是靠着狄青从关西率军来扫平。

侬智高兵败,侬智会逃离邕州,占据了临近广源州的古勿洞作为据点。当如今的天子登基后,侬智会重新上表归附,这时的朝廷不似仁宗,很干脆的答应下来,将古勿洞升格为归化州。虽然归化州不属于左右江三十六峒,但势力不下于任何一个三十六峒中的大部族。

之前三十六峒在韩冈的支持下攻入交趾恣意劫掠,大发横财。而归化州正面就是广源州,让侬智会无从着手,根本就打不到交趾境内。想来他现在是听说了准备,赶来分一杯羹的。

要见一个羁縻州的知州,不可能让章惇、韩冈两名主帅同时出马。韩冈摇了摇头,将收买人心的机会让给章惇,“小弟去疗养院探望一下李宪,他的病差不多也该好了。”

提起监军的李宪,作为主帅的章惇就是一阵不舒服。身边有着一个监视自己的阉人,任何一位主帅都不会喜欢。只是他心中就算是希望李宪在疗养院中,一直到他打下升龙府才将病养好,但他总不能就这么说出口。

只能在脸上浮起虚伪的笑容,“那就请玉昆代愚兄去问候李承受,愚兄这就去见一见侬智高的亲兄弟。”说罢,章惇便长身而起,“等到过两天广源州那里的消息传来,我们也该动手了!”

