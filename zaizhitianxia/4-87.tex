\section{第15章 焰上云霄思逐寇(18)}

韩冈领军从归仁铺撤回昆仑关的消息,章惇前天已经收到了。之前韩冈领军直扑邕州的军报曾差点让他跳脚,看到韩冈返回昆仑关的消息,他仍依旧为正在邕州的官军提心吊胆——李常杰并没有撤退,反而领军直逼昆仑关——直到今天再一次得到了‘今日大雨,关城平安’的军报,这才让章惇放下心来。

连着下几天雨,围城的交趾军如何还能保持着士气?若当地的雨水再继续多下几天,别说撤军了,还要提防着会不会被韩冈领军追杀千里。让麾下的士兵连续多日的在雨水中摸爬滚打,领过军的章惇知道这是多么艰难的一件任务,只要主帅稍稍一个疏忽,没有将下面的士兵稳定住,就能掀起一场兵变来。

所以章惇能放下心来。李常杰已经在邕州城下攻了两个月,刚刚攻下来就挨了韩冈的几下闷棍。领军攻打昆仑关的时候,又遇上了连日雨水,换作是他章惇,也只能想着该如何体面安全的将军队撤回去了。

另外韩冈率部在一进一退的过程中并没有受到多少伤亡,也让章惇松了一大口气,若是韩冈贪功冒进让队伍有所折损,他就有的是口水仗要跟吴充的枢密院打了。

不过韩冈传回来的还有一条噩耗,邕州城确定已经被攻破。这条消息也让章惇也有些黯然神伤,一路紧赶慢赶,终究还是没能来得及救下苏缄和邕州满城的百姓。城中官吏生死不明,也不清楚城内的百姓又有多少逃过交趾军的屠刀……

“能将昆仑关夺回,能降伏广源蛮军,能领军破敌制胜,韩冈的才华当真少有人能及。他年纪轻轻,已经有了这样的功劳,日后必定前途无量。”广西转运使李平一在章惇面前笑着,对自己副手所立下的功劳推崇备至。

章惇瞥了李平一一眼,这位转运使在粮秣安排的上不见有何才华,连挑拨离间的本事都一塌糊涂。神容如常:“韩玉昆文武皆备,本就被天子所看重。且他一向善知进退,立功倒是不在话下。”

不论是官场还是战场上,过去的韩冈给章惇留下的绝大多数都是勇猛直进的印象。许多时候韩冈都表现得强硬无比,对上天子、对上宰相,对上他一个个顶头上司,皆是宁折不弯。

但现在回想起来,韩冈似乎并没有因为强硬的态度而吃过大亏。有好几次都是,但不久之后就因为强硬坚定的态度而得到了更高的评价。比如他对横山一役的看法,再比如他去军器监的行动,一开始时都是开罪了宰相,但事后的结果无不证明了他的眼光和手段。

李平一没在章惇的脸上看到想看的表情,略感失望的说道,“不知李常杰会不会硬要打下昆仑关?”

“李常杰的想法让人难以揣摩,不过他会不会硬攻昆仑关是一回事,能不能打下来则是另外一回事。”章惇对李平一的问题给了一个毫不含糊的回答,“交趾军兵疲师老,守住昆仑关倒也不难。”

这两天收到的军报,都在说昆仑关那边连着在下雨。章惇问了熟悉邕州气候的官吏,知道每天从二月开始,广西——尤其是邕州——雨水就多了起来。这样的气候中,不但雨水多,而且岭外两路让人闻风丧胆的疾疫也多了起来。交趾兵再习惯南方的气候,也照样还是人,恐怕也不可能在拥挤的军营里,被雨水泡着,还能保持着一点疾病都没有。他们那边不可能会有疗养院,更不会有药王弟子。

不管怎么说,章惇此前派过去的援军这时候按照行程的话,差不多也该到了。只要韩冈手上有了一千五百名荆南军的精锐,加上黄金满手下的蛮军,要将天时地利人和三项都不占的李常杰打回老巢去,并不需要他花费太多的气力。

章惇安安心心让人端茶上水,拿着些闲话敷衍着李平一,到了这个时候,就只要等着南面传回捷报了。希望韩冈还能给他一个惊喜!

……………………

又一夜过去。

在这一夜中,李常杰没有让城中的守军有着休息的空间和时间。

夜色是最好的保护色,趁着夜色,垒土上城,是当初攻打邕州时得到的经验之谈。只要一夜辛苦,就能将土堆垒到城上,就算是疲惫不堪的交趾兵,也从身体里鼓起了最后一份力量,拼着性命的将一包包土运到城头下。

城头上弓弩连绵,向着一名名上冲上来的交趾士兵射下密集如飞蝗的箭矢。可就算如此,没了力道的弓弩并没有太多的用处。而失去了一直以为臂助的弓弩,宋军也不能找到更好的阻止敌军垒土攻城的办法。只有不断投下石块和檑木,充作防御的手段,尽管砸伤砸死的敌军不少,也拖延了攻城一方的部分时间,但对于交趾兵来说,这些从城头上砸下来的城防武器,反而是最好的修筑土台的材料。

通往城头上的土台,就这么一点点的累积起来。快要天亮的时候,一条斜斜而上的土坡,就仅仅差了最后的五尺髙,便能与城墙的雉堞平齐。只要是身高略高一点的士兵站在土台上,可以直接看见城墙上的动静。

到了这个高度,就不用再堆土了,只需要架上木板,直接靠在城头上,土台上的士兵也能顺顺当当的攻上了城头。

“大局定矣!”李常杰一声畅快淋漓的大吼,苦熬了多少时日,又让他费了多少心血,现在终于到了结束一切的时候。

锵的一声拔出宝剑,遥遥指向风雨飘摇中的昆仑关,“给本帅攻进昆仑关去!先登者,一等功、一等赏、官阶七资三转!夺敌大旗者,二等功、一等赏、官阶五资二转!能斩下宋军主帅首级者,为此战头功,即以团练之职赠之!封妻荫子,就在今日!”

随着李常杰颁功布赏,一阵山呼海应的声浪随之从交趾人的阵地上升了起来,一直冲到了昆仑关关城的上空。

等候已久的交趾兵们顶着长长的木板,向着关城冲杀过来。只要再有半刻钟,他们的脚步就能踏上昆仑关的城头。

吱呀呀的声响,紧闭许久的关城的城门终于开了。交趾人不知道他们究竟要打算做什么,在胜负已定的情况下,就算派军出城来抵抗,也一样无济于事。

一队交趾军向着逐渐开启的城门杀了过来,一名看起来孔武有力的小军官举着巨大的木盾,顶在最前头。只要防住了威力大减的神臂弓,撞开挡在前路上的守军,就能直接进城中去了,先登之功也许抢不了,但抢下一个率先入城的功劳,机会可就在眼前!

小军官低着头卖力的向前冲过去,无论是箭矢还是刀枪,都不可能穿透他手上的巨盾。离着城门越来越近,而城门打开的的缝隙也越来越大。已经冲到了门口,只差一步就能冲进关城了。

但下一刻,小军官就感觉自己好像飞起来。宽阔厚实的木盾突然中分裂开,连同他本人都一起裂开。身子轻飘飘的,仿佛浑身上下都没有重量。最后他看到的是一柄手提车轮一般大小的精钢巨斧,还有将粗长的斧柄抓在手中的一名身高七尺的巨汉。而巨汉的背后,就在关城之中,是一群身着宋军衣甲的士兵。

一斧头将打头的交趾军官砍成两半,巨汉大步跨出门来,一步就踏进了交趾兵的行列中。冲着心中惊颤不已的敌人怒吼一声,巨汉将沉重的精钢大斧横着一抡。原本是扫腰的招数,落在矮小的交趾兵身上,就成了瞄准脑袋脖子挥过去。如同虎吼一般的呼啸声中,一道划着圆弧的斧光劈开了脖子,斩裂了脸庞,让五六个交趾兵一下倒飞出去。

满脸横肉,乱蓬蓬的胡须歪七扭八的往横里长着。身上披挂着鱼鳞铁甲,只一看就知道至少有二三十斤的份量,可他挥斧下砍横劈,却没有受到半点影响。

手起斧落,闪烁着精光的大斧被巨汉挥动得如同一道龙卷,将被他冲进的人群卷入死亡地带之中。而从城门中杀出来的宋军将士,同样手持大斧、身着甲胄,跟着一起在关城下冲杀起来。

他们的冲锋势不可挡,不过一眨眼的功夫,城门附近的交趾兵已经被砍杀一空,不是躺在地上,就是逃离了城下。被劈翻在地的交趾兵还有几个没有当场气绝,捂着伤口发出一声声凄惨的哀鸣。

巨汉低头冷淡的看了最近的一名交趾伤兵一眼,随即就是重重的一脚跺在他的胸口上。听见脚底的呻吟声转成一声短促的惨叫,卡擦的骨裂声从脚底传上来,他更加兴致高昂,冲着南方的贼军阵地挥起巨斧,放声大笑:

“歇了一天,正好给爷爷活动活动筋骨!”大手一挥,带着身后一群虎狼,“这破敌斩将的头功,爷爷先预定下了!”

