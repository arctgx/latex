\section{第14章 卧薪三载终逢春(中)}

【今天的第三更。明天的第一更,夜里会赶出来的,请各位书友明天起来后再看。求红票,收藏。】

接近入夜时分,韩冈和王中正一行回到渭源堡中。王韶正负手站在大厅中,低头看着一幅方方正正的木盘。

王中正随之看过去。此物说是盆景,但无草无木,更无怪石。却有房屋有围墙,在六尺见方的底面上,一上一下的布置着,像是两座具体而微的宅院……不,王中正再仔细看过,根本不是宅院,而是两座寨堡。

“这是那处的沙盘?”王中正问道。还在京城的时候,他在武英殿中亲眼见识过赵顼命人打造的几十块沙盘。虽然眼前的这一块与他见过的不太一样,但应该是同一类东西。

“新渭源堡。”王韶答道。

韩冈发明的沙盘让王韶触类旁通,他来渭源的目的就是要为新堡选址,并决定大小范围和式样。为了能更直观的进行确认,他找来木匠打造了新堡的实物模型。

“现在的渭源堡,只能起着哨探的用处,不过是个略大一点的烽火台而已。前次董裕在渭源堡外长驱直入,堡中却无兵可以断其归路。”韩冈接口为王中正解释,“在渭源修造新堡,囤积粮秣,驻扎大军,就是将防线前伸至鸟鼠山下。而古渭一带则可以安心的展开屯垦。”

王中正又低头看了一阵沙盘,在沙盘一角有着标志东西南北的十字箭头,边上还有确定距离的比例尺。对于沙盘上的学问,为了能在赵顼面前说上话,宫中的宦官没有不学的,王中正也懂得如何利用比例尺来换算实际距离。

沙盘上的两座寨堡,一东一西的相隔大约半里布置着,而渭水流经西堡南侧,却从东堡北侧经过。王中正奇怪的问道:“为何这两座新堡离得这么远,又隔着渭水?”

“渭源堡孤悬于外,并设两座、分据渭水两岸,中设绳桥或浮桥连接两岸,便可成犄角之势,能自护得全。而半里之地,一百八十步的距离,也算不上远。”王韶指了指位于北岸的西侧寨堡,苦笑了一下,“其实若是能建在河水的正对面当然是最好,但在渭水北岸,最近的一处适宜筑堡的地方却是这里,没得他处可选。”

王中正皱眉问道,“若是渭水泛滥怎么办?洪流之下,桥梁难行,那两堡间的犄角之势就成不了了吧?”

“都知考虑得的确周全。”韩冈先赞了一句,“不过洪水泛滥之时,多是暴雨之后,地面泥泞,贼人也难以进攻。”

“原来如此。”王中正点着头,喃喃的念了几句。最后抬头笑道:“却是吾多问了。”

王中正对渭源堡问得多了点,王韶听着就觉得有些问题。带着疑问的眼神投向韩冈,韩冈随即心领神会的轻轻点了点头。

果然如此!王韶精神便是一振:“都知能亲来渭源,可见对军国之事也是放在心上的。可比窦副总管强多了。无论是向钤辖还是窦副总管,自上任以来一次也没到过渭源堡。而李经略,也是对扩建渭源堡毫无兴致,压了不知多少文书。”

“官家对河湟之事始终放在心上,无论渭源还是古渭,都是经常挂在嘴边。吾既然到了秦州,自当来渭源一趟,返京后也好有话回禀官家。以官家对河湟之事的重视,事无巨细怕是都要问到。”王中正撇清似的说了两句,但话里话外都是透着他本人对开边之事的关注。

“唉!”王韶一声长叹,对着东面拱手叹息,眼中几乎要流下泪来:“天子如此看重,三年来王韶只有些许微功可报天子恩德,实在是羞愧难当,羞愧难当啊!”

“朝臣中伤于内,帅府沮坏于外,左正言还能连番大捷,何谈难报天子?”王中正见状,忙劝着王韶:“若左正言此话传出去,不知有多少人要无地自容了。”

看着两人声情并茂的演出,韩冈站在旁边没有说什么。王中正的心意已经透露出来,而王韶的目的也已经达到了。

王中正有心于边事,王韶老于世故,王中正只多问了两句,他就看了出来,又从韩冈那里确认了,自然不会轻易放过。他很想把王中正这位大貂珰拉进来,不仅为了更好的得到天子的支持,更是为了对抗高遵裕。

王韶一直都希望有一个能在天子面前说上话的助力,高遵裕是太后亲叔,天子舅公,当然可以算得上。但高遵裕这个人本身的性格,却是贪功过甚,让王韶心中忌惮。说不准那天他的位置就给高遵裕给挤掉了。

所以王中正一来,王韶就盯上了他。为了与天子联系得更紧密,王韶不介意把一个支持开边之策的宦官拉来当监军。以宦官为监军,唐宋皆有。如走马承受一职,甚至可以直接参与到地方上的事务。而在地方上领兵、修河的宦官为数也不少。

此时的士大夫,对阉人极端歧视,有事无事就要敲打他们一番。但对阉人参与到政事军事中来,却是习以为常,需要时说上几句,不需要时就任凭阉人在地方上领兵任官。而韩冈却正好相反,他不歧视阉人,却不习惯阉宦参与国政。

故而韩冈对王韶的想法不置可否,在心底里,还是反对居多。在他想来,王中正可不一定会与着王韶一条心,说不准会跟高遵裕打成一片,而且王中正本人的品行也成问题。只是他心里的想法并不打算说出来,因为对高遵裕,韩冈心中也有所顾忌。两害相权,也难说孰重孰轻。

陪了王中正用过晚饭,送了他去休息。王韶拉着韩冈和王厚又站到沙盘旁。他想听听韩冈的意见。

“玉昆,你觉得两堡如此布置是否妥当?”

“如果钱粮和人手足够的话,能造得更大一点就好了。”这是韩冈的回答。

韩冈对军寨建筑其实并不了解,他只知道城墙越高越厚,里面存放的粮秣军械越多,这城寨就越是难以攻克。但他更清楚,修造任何工程,第一个要考虑的都是预算问题,接下来则是人手问题,至于建造成什么模样,都是要受这两条左右。

“哪来的多余钱粮?超过五百步的寨子是不用想了!若是钱粮足够,直接渭源堡扩建成千步城不是更好?!何必弄什么犄角之势,在对岸再造一座堡?古渭寨、甘谷城都没有,还不是安安生生的。”

韩冈的话,引爆了王韶藏在心底的炸弹,他拍着沙盘边上,大声骂道:“政事堂也是好笑,我跟他们要钱修城,他们倒好,让二哥带回两百份空白度牒来。也不想想这里是秦州,不是京城,有几人会拿两三百贯来买一张度牒的?!还说是值五万贯,要能卖出一半价钱,我都要烧香念佛了!”

王韶的抱怨自有其道理。

因为有一张度牒,可以免人丁税,可以不用路引过所就能游走天下,想弄一张来护身的商人数不胜数。而且有的富户要保子嗣平安,也需要一张度牒来剃度一个替身。

所以度牒就相当于有价证券,能卖上不低的价钱。有时候,地方上有灾荒,朝中拿不出钱来救济,就发下度牒充当灾款。另一方面,真正吃斋念佛的僧侣,却有许多因为买不起一张度牒来剃度,而只能终身当个沙弥。

不过度牒的价格就跟有价证券一样,有着波动性。有时高有时低,有的地方高,有的地方低。如京城、江南这些富庶之处,往往能卖高价,两百贯、三百贯都卖过。但在秦州,王韶刚刚让人问过价,一开始报得是一百二十贯一份,但当听说了王韶手上有两百份度牒,啪,当即就跌倒九十。

政事堂发下两百张度牒当作五万贯来拨款,但实际上却只能卖出不到两万贯,这让王韶如何不气?这种东西,还不好找人硬摊派,只能一张张发卖出去。

王韶骂了一阵,也就停了。事已至此,也只能向朝中将此事说明,并继续要钱要粮——用不到两万贯来筑寨堡,在秦州城边上还好说,但换到离秦州三百多里的渭源,单是征发起来的民伕所需的粮草,在路中转运的消耗就能吃掉一半去。

“再能要到两三万贯就好了。”王厚为他老子端来一杯凉茶消气,王韶心气平和了下来。他还是有些自信,凭借他现在在天子心中的地位,再要到两三万不成问题。

韩冈低头看着沙盘模型:“若能再多个两三万贯,照着图样,将现在的渭源堡扩建一番,再在对岸新建一座,勉强也够了。届时在两边各放上一个指挥。有三四百人足以将堡子守住。”

王厚在旁插话道:“禁军一个指挥才有三四百,厢军可没有。”

“怎么也不可能放厢军来戍守的!”韩冈摇头,提高的音调中满是不屑,“就是招乡兵弓箭手来此受田戍守,都比放厢军的好。”

按照编制,一个指挥一般是五百人上下。但这只是兵籍上的数字,减去吃空饷的比例,和一些不堪上阵、但后有靠山的老弱,一个指挥真正可以投入战斗的也就三百多人

——这里指的是普通的禁军,若是厢军,则一半是空额,剩下的一半又多半在官员家奔走听命。他们的战力甚至还不如关西的乡兵。若韩冈当初押运军饷去甘谷城,随行的不是当过弓箭手的民伕,而是厢军,他说不定早早的就跑路了。

