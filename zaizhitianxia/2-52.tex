\section{第12章 平生心曲谁为伸(六)}

【第一更,求红票,收藏】

在落日之后,秦州城终于清凉了下来。夏日的夜色中,有明月,星光,还有阵阵凉爽的山风。而不似前段时间,就算是子夜,还是让人烦闷不已的燥热。

这两天,紧跟着古渭大捷,党项人也在甘谷城下被刘昌祚击退,李师中率军回镇,秦州城内的紧张气氛缓和了许多。同时因为入夜后气温更为凉爽,白天门可罗雀的店铺,日落后却是顾客盈门,城中几家行会遂联名向李师中请命,希望能在入夜后也照常开门,他们暗中给又几个能说得上话的关键人物送了些礼。故而前日李师中回城后,就照着旧年的故事,顺势下令将初更就开始的宵禁推迟了一个时辰。

这一日,正好韩云娘跟严素心商量着要扯几匹布给家中做几件秋衣,韩冈也是闲来无事,不想成天埋在书堆或是阴谋诡计之中,就带上李小六跟她们一起出们逛街。严素心拉着招儿,一行五人在吃完晚饭后,慢悠悠的散着步,到了秦州城中最繁华的河西大街上。

大街之上,行人如织。

为了招揽顾客,两边的店铺都是在门头上高高的挂起一串灯笼,映得街面灯火通明。不仅店铺,就来拿街边摊贩,也在摊头上刮着各色有趣的彩灯,唱着成调成套的吆喝,来吸引游人的耳目。

韩冈在路边缓缓走着,他没兴趣逛铺子街摊,可见到几个出色的美人,也不介意多看两眼。但他看来看去,最出色的还是前面拉着严招儿的云娘,还有跟在他身后一步,亦步亦趋的严素心。韩冈长得高大,器宇轩昂。相貌虽只算得上不错,但神采自蕴的气质却是难得一见。

他穿着文士襕衫,以方领矩步,行于街市之上。澹泊闲雅的气度如同鹤立鸡群,引得街上的不少女子都看了过来,有几个贵家的闺秀,用小团扇遮了脸,偷眼看着韩冈。当然也有大胆的,李小六这个伴当就被人拉住了好几次,向他问着韩冈的身份。不过李小六伶俐得很,全都给打发掉了。

可李小六追上来后,却嘟嘟囔囔的向韩冈抱怨着:“官人,你以后还是别出来了。要出来也该穿着官袍,也好把人给镇着。你现在这样子,多少人家要抢你做女婿。你看看,俺的袖子都给扯破了。”

对于李小六的抱怨,韩云娘和严素心觉得很有趣,用手捂着嘴,呼呼的暗笑着。韩冈对此也有些无奈,谁能想到天气热人,这人也变热了。北宋风气比唐时当然是严谨了许多,但比起明清还是很开放的。

在此时寡妇改嫁是常见的事,反倒是守节守上几十年的情况却很少。就算是官宦人家,也是守满三年便自离去,而平常百姓,多是守个一年半载就改嫁。甚至像韩冈的大嫂,自他大哥战死之后,才两个月功夫就带着嫁妆回了娘家,很快就另嫁了人家。

而出门上街的良家女子也很多。就如严素心她这个做厨娘的,不可能在家里等着卖菜的上门兜售,肯定是要出门,有时还要到河西大街的蕃商开的货栈去买些孜然、胡椒之类的调味料,向锅碗瓢盆,针头线脑之类的日用杂物也是一样要出门采办,要操持家务的小家碧玉大率皆是如此。

而大家闺秀们也并不是二门不迈、大门不出。踏青赏花,探亲访友,或是姐妹淘在一起组织诗会的事情,韩冈就听过不少。而且就在秦州城中,便有几家闺秀组织了这么一个诗社,一个月、半个月就聚上一次。听说其中有李师中的女儿,也有几个土著豪门家的闺秀,最近还加入了窦舜卿的女儿和孙女——据称老当益壮的窦副总管的女儿比孙女还要小上一岁。

几次诗会一开,闺秀们的诗作也陆续流传了出来,被好事之徒拿着四处宣传。前些天就有人拿着咏荷花的一卷诗集,到了衙门中来让韩冈和他的几个同僚品评。韩冈一览之下是赞不绝口:“墨黑、纸白,装订的功夫也是一等一。还有这是谁人誊抄,字写得当真不错,难得!难得!”

官厅中的众人闻言无不掩口而笑,而把诗集带来的好事之徒则是悻悻而去。韩冈他在这件事上虽是不留口德,但那些个名门闺秀的作品也的确是难以入目。除了李师中家的女儿写的两首还算通顺,其他的甚至有些连平仄都没对上,完全是拿着华丽的词藻堆砌,削足适履式的求着对仗工整,风格学着西昆体,却不及杨亿、刘筠等人之万一,真还不如韩冈自己写的水平。

不过相对于天天要出门买菜的严素心,韩云娘就很少出门。走在街市她就变得很活泼,牵着招儿的手,在各家的摊子上好奇的看着。

韩冈掏钱给她们买了不少零嘴,韩云娘跟着招儿一人拿着一串用糖水煮过的林檎果,另一只手还拎着几个荷叶包,里面是水鹅梨、小瑶李子、闵水荔枝膏什么的,说是要带回去给韩阿李。

看在两个小女孩儿在前面脚步轻快的从一个铺子转到另一个摊子,跟在后面的慢慢踱步的韩冈的心情也轻松了起来。虽然正准备对窦舜卿动手,但也不妨碍他出来逛一逛街市。

不过今天的正事还是买做秋衣的布匹,在大街上逛了一阵子,韩冈五人随便找了一家绸缎铺走了进去。

“韩官人?!”

刚进门,迎面便被人叫破了身份。抬眼看过去,却见着一个胖子站在店铺中的柜台后。圆滚滚的身子,圆滚滚的脸,鼻头都是圆圆的。腮帮子都被肥油充满,把五官挤得嘟在了一起。但职业性的笑容十分的很和气,还有着一份恰到好处的谦卑。韩冈看到这份笑容便心道,能得迎宾待客之三昧,这胖子至少也该是个掌柜。

“真的是韩官人!”胖子很轻巧的绕过摆满绸缎布匹的柜台,惊喜的走到韩冈面前打躬作揖。

绸缎铺的掌柜能叫出自己的名字,韩冈挺惊讶的,问道:“你认识本官?”

“哪能不认识呢?”绸缎铺的胖掌柜直起腰来谄笑着,“韩官人的名字在秦州早就传遍了,又有谁人不知?小人也是前日有幸一睹丰颜。”

大概是好说话的性子,胖掌柜在韩冈这个官人面前也不露怯,嘴皮子飞快地动着:“韩官人今日带着家眷来,是不是要买些什么?小人这店铺虽不算大,但里面的货色却都是顶尖上好的料子,蜀地的锦,扬州的绢,定州的丝,和州的麻,天南海北的织物小店都有,秦州城中的其他铺子可都没小店这般齐全。”

韩冈点了点头,却没答话。胖掌柜很乖觉的跟在后面,也闭上了嘴。

严素心和韩云娘这时已经走到店铺里面,由个学徒陪着,在翻着几疋素色隐莲纹的绸缎。关西的丝绢率是黄丝,就算染过后,做出来的衣服颜色都不正。

两女在绸缎中挑三拣四,一匹匹的对比着看过,争论着花色和颜色的好坏。女儿家买东西向来是慢,韩冈也是有经验和体会,耐下性子等着她们。只是闲着无事,顺便也在铺子里左右看着。

虽然胖掌柜自谦的说着店铺不大,但这间绸缎铺的门面其实不算小,而且还是位于城中最繁华的河西大街上,单是这铺面本身就值上不少,何况店中的这些绫罗绸缎,也是价值高昂。

韩冈转了一圈,却停步在单独的一座柜台前。柜台上,也堆着十几匹各色花样的布匹,但跟店中的其他布料却完全两样。

“这可不是绸子吧?”韩冈捏着一角提起来,指尖搓揉了一下,厚实柔软。没有丝绸的细滑,也不似麻布的粗糙,分明是棉布的感觉。

胖掌柜瞧着韩冈看货,立刻笑成了一朵花,走过来大赞道:“韩官人好眼光,当然不是绸子。这可是琼州黎人所织的吉贝布!”

“吉贝?是木棉吧?”

“对!对!就是木棉布。”见韩冈识货,胖掌柜猛点头,“不过叫吉贝布不是讨个好口彩嘛?想着这吉贝布,从琼州飘洋过海,再运来秦州,可是万里迢迢,一路险阻……”胖掌柜摇头晃脑,背着不知是谁人写得广告词,说得是一套一套。

韩冈听得好笑:“吉贝是琼州黎人口中的木棉,可不是什么好口彩。”

北宋的棉花,还被称为木棉,主要的种植地是两广和海南,还有蜀中和大理,据说西域和甘凉一带也有。此时黄道婆还没有出生,汉家的织物向以丝麻为主,棉花种植稀少,使得黎人织布的技艺反在汉人之上,弄得棉布的名字都学着黎人。

韩冈指着这匹布问着胖掌柜:“这木棉布多少钱?”

胖掌柜作出很大方爽快的样子,“官人若是真心想要,俺就直接给官人送到府上去,至于价钱,看着给就是了。”

“到底多少?”韩冈不为胖掌柜这样的推销手法所动,问着他实在的价格。

胖掌柜低头做个谢罪的模样,然后伸出双手比出了五和三的手势,“惯常报的是七千文足,实价则是五千三百一匹。”

