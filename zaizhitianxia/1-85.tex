\section{第36章 不意吴越竟同舟(上)}

[有朋友说俺ps剧透用得太多了,俺从善如流,以后无特殊情况就不用了。但红票和收藏还是要的。另外还有朋友说辞行说了五章太长了一点,虽然是没错,但这是必要的铺垫,里面出现的人物和情节都会在后文出现,总不能让他们突然冒出来,并不是在灌水。只稍作解释,下面请看正文。]

渭河岸边,陇山脚下,正是秦州通往凤翔府宝鸡县的两百余里官道所在。沿着渭水河谷向关中腹地而去的官道,曲折绵长,冰结的渭水如一条玉带,穿行于陇山群峰之间。夜色将临,夕阳已经落到了山后,只能从白雪皑皑的山巅上,看到一点反射过来的落日余辉。

踏着渐临的暮色,在这段官道的中段,一处年久失修的驿站前,韩冈吁的一声,勒停了马匹。李小六紧随在韩冈身后,几乎滚着下马,狼狈的坐在地上呼哧呼哧的喘着大气。小孩子气力短,骑在马上奔波了几个时辰便吃不消了。

当日韩冈押队从秦州往甘谷去,才走了三十里到了陇城县便停下来休息,这是因为再往西北去的第二程六十里的山路并不好走。而从秦州往京城去,一千七八百里路,骑马总计不过十九程。按此计算,第二天入夜时就得抵达宝鸡县,所以第一天,便是整整一百三十里路。

渭水是北面陇州和南面凤州的界河,自出秦州地界,在陇州和凤州交界的山谷中穿行二百里后,流入凤翔府境内。位于渭水北岸的官道从地理位置上看,应该属于陇州,但由于陇山阻隔的关系,陇州无法直接进行管辖,实际上是被秦州和凤翔府两家各管一半,各自派出巡检在路上维持治安。

驿站的位置依山傍河,接天连地,山河有龙蛇之相。此地风水甚好,埋下棺木,便能旺家。因而这座合口驿站,破落得像座老坟边的旧祠堂,韩冈却也是一点也不奇怪。

如果是在京城中,安顿辽国和西夏使臣的都亭驿和都亭西驿,那便是雕栏画栋,重楼叠翠,比秦州的州衙还要气派三分。不过既然是山沟子里的驿站,设施便简单了很多。这座名为七里坪的驿站,房顶上的积雪中能看到茅草挺立,而后院的一侧厢房,甚至塌了半边都放在那里没有打理。

‘或许真的是祠堂改得。’韩冈想着。

甫进驿站,一名在驿站中打下手的驿卒老兵就迎了上来,张口便道:“敢问官人,可是要住店?”

‘什么时候驿站改客栈了?!’

韩冈听着老兵的招呼,微微吃了一惊。只看老兵上来迎客的动作话语熟极而流,便知道驿站充作客栈的时日不算短了,而且院落中停满了卸了牲口的车子,看起来在驿站中落脚的队伍也不少的样子。

韩冈没住过驿站,不清楚这里将驿站兼做酒店,是不是个特例,但秦州城中最为有名的惠丰楼便是官办的酒楼,从这一点来看,驿站兼营客栈业务,说不定是这个时代的普遍情况——就如后世的单位招待所,也照样对外开放。

收起惊讶,韩冈从怀中掏出驿券,冲着老兵扬了一下:“驿丞何在?本官受命入京,要在此处住上一夜。”

见韩冈拿出盖着朱红大印的驿券,老兵的神色顿时恭敬起来。忙入内唤了驿丞出来。七里坪驿站的驿丞大约四十多岁,圆滚滚的肚子有着宰相的份量,看来驿站中的油水不是一般的充足。

韩冈将驿券递了过去。六寸长、两寸宽的纸条上面,有着他的身份年龄、相貌特征,以及入京的时限,最重要的是一颗鲜红的秦凤经略司官印。驿丞仔细验过,点头哈腰请了韩冈进了驿馆。李小六聪明伶俐,不待吩咐,牵起两匹马,跟着老兵到院后的马厩中安顿。

韩冈进了驿站厅中,看起来与普通的脚店也差不多的样子,也卖酒,也卖肉。此时正是饭点,三三两两客人散座在厅中。韩冈环目一扫,眉头便不由自主的皱了起来。吵闹点无所谓,但环境污糟得比伤病营还超过几分,那就让他难以忍受了。

他摇了摇头,这间驿站建立起来后,到底打没打扫过一次?!

在门口停步,韩冈回头对驿丞道:“先找间上房,饭菜给我端到房中。”

驿丞在韩冈面前陪着小心,“回官人,官人到得不巧,年后进京的官人们也多,馆里的两间上房都给占了。”

“一间上房都腾不出来?!”韩冈脸色微沉,只看眼前的一地久未清扫的污秽,普通的房间不用指望会比大厅好上多少。

“回官人的话,委实没有了……”驿丞被韩冈瞪了一眼,背后一阵发凉,想不到这位年轻的韩官人不过十九岁,就有了不怒自威的气势。他主持驿站数十年,见识过的官员数以千计,心知如韩冈这般年轻气盛的官人,即便官位不高,最好也不要去违逆。他苦苦想了半天,有些犹豫地试探的问着:“官人你看这样成不成?今天正有一个要去京中的刘官人,也是秦州来的。官人若不嫌弃,与那位刘官人并一间屋如何?”

“刘……?”韩冈沉吟起来,这怕是熟人,“你带本官去看看。”

驿丞指着厅中角落,一个健壮背影正凭桌而坐:“刘官人就在那里!”

韩冈眉毛抬了抬,果然是刘仲武没错。

去京城的官道,一程一程的都有定数,驿站的安排便是由此而来。刘仲武不可能说一口气跑个两百里,再在荒郊野地找户民家休息。他既然和韩冈都是同一天从秦州出发,那么在落脚的时候碰上,也是理所当然。

韩冈本想着逼驿丞给腾出间上房来,但看到向宝大力提携的刘仲武,忽然觉得让向宝不痛快也不错。他走到刘仲武面前,拱手微笑:“在下韩冈,见过刘兄。”

桌上酒肉俱全,刘仲武正挥着筷子大快朵颐。韩冈冷不丁的走到面前,他眼睛瞪得溜圆,一下惊得跳起,刚吞下去的肉正好卡在喉咙里。

“韩……咳咳咳!”刘仲武用力捶着胸口,驿丞忙过来帮他捶着背。韩冈将桌上的酒壶递过去,刘仲武一把抢过来,揭开壶盖,仰着脖子咕嘟咕嘟地如同灌蟋蟀一样灌了下去。好半天他才回过气来,直喘着,“韩官人,怎么是你?”

韩冈脸上笑容不改,再次拱手行礼:“韩冈方才冒失了,惊扰到刘兄,还望恕罪。”

刘仲武赶忙跳起回礼,弯腰至地。韩冈如今在秦州风头正劲,即便他不自报家门,刘仲武一眼便能认出他来,要不然也不会差点被噎死。以韩冈和他举主王韶,与自家恩主向宝之间的恩怨,刘仲武根本不想跟他有任何瓜葛。

只是韩冈是已经有了官身的文臣,而他还要到京中去参加测试,地位有天壤之别,前面韩冈过来时,他已经失礼。韩冈礼貌周全是品德高致,刘仲武又哪里敢大剌剌的坐着妄自尊大,即便因向宝的缘故在,也大不过礼法去:“小人不才,让官人见笑。……不知官人有何指教?”

韩冈看了下驿丞,驿丞识趣的上前:“韩官人来得迟了,馆里的清净上房都已有人占了。小人心想二位官人都是秦州来的,不知今夜可否挤上一挤?权变一二?”

刘仲武看了看韩冈,韩冈微笑不语。再看看驿丞,犹在那里打躬作揖。

一时间,刘仲武进退两难。

向宝赠他以美人,又荐举他入京,而且为他饯行时,都钤辖还厚赠金银以壮行色。如此深恩,粉身碎骨去报答还来不及,他又怎么能恩将仇报?

但韩冈就在他面前直说要分半间屋子住,礼数一点不缺,刘仲武又没有办法跟他翻脸。韩冈本人的才干不提,他身后还有王韶、张守约,又是横渠先生的弟子,向宝都要忍气吞声的主,自己得罪他作甚?躲着走才是正理。

刘仲武不打算与韩冈争屋,退让道:“韩官人既然要住下来,那就住小人的厢房好了。小人就在厅里找几张桌子并一下,胡乱躺上一晚也无妨。”

“这如何使得?!”韩冈连连摇着头,既然刘仲武给他面子,当然要还回去,“凡事都要讲究个先来后到,客随主便。刘兄比韩某先至,前一步定了房间,算是主人。韩某后至为客,这世上哪有客人把主人赶出去的道理!?”

“韩官人在此,小人坐都没资格坐,何来先入为主的说法。韩官人尽管住,小人哪里都能凑合。”

“韩某一来便占了刘兄的厢房,传扬出去,别人不知是刘兄谦恭,倒会让人说我韩冈得志猖狂。”

不论是争房,还是让房,在驿馆里做了二十年的七里坪驿丞都见多了,“两位官人不必谦让,刘官人定下来的屋子分得内外间,等小人将床铺铺上去,各自一间,都能睡得安稳。”

“那自然最好,就这么办!”韩冈拍板决断,没给刘仲武反对的机会。转过来又对刘仲武道:“多谢刘兄分屋与韩某落脚。刘兄大名震秦凤,韩某钦慕已久。相逢便是有缘,今日偶遇,当醉饮一场方休。”

刘仲武欲推辞,却被韩冈强拉着。韩冈拉人上船的手段早就历练出来,他岂是对手。几句话便噎得刘仲武点头答应。他既然不敢翻了面皮,掀了桌子,也只能硬起头皮,苦着脸,与韩冈一起好生的喝了一顿酒。

四十文一斤的玉春霖在西北已是上品,刘仲武一年也喝不到三五次。可他今次喝得全不知滋味,只觉得今生没喝过这般难下肚的水酒,就跟喝着鸩药一般。

被韩冈扯着一杯杯的灌下去,刘仲武一个晚上都没坐安稳,仿佛屁股上有针在扎——跟韩冈把酒言欢,传到向钤辖耳中,哪会有好下场!?但韩冈一直拉着他,直喝到驿馆里的半坛存酒底儿干,方才罢休。

