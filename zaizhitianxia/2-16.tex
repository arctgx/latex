\section{第五章 平蛮克戎指掌上(四)}

【第一更,求红票,收藏】

秦州州衙最后一进的院落一角,是知州的书房。不同性格的知州,书房中的布置也便不尽相同。而最近的这任知州,他的书房里总是少不了各色笔墨画具。就在书房的墙壁上,挂着一幅幅装裱精美的工笔画,无不是出自书房主人的手笔。只是最近的这段时间,书房的主人放弃了绘画的爱好,而是埋首于公文中。

“想不到是沈起,他来有什么用,和稀泥吗?!”

李师中冷笑一声,把自己正在看着的一封公文甩手丢在桌案上。只是他手上用的力气大了点,文书在桌面上转了半圈,啪的一声滑落到了地上。就听着秦凤经略的声音在书房中响着,对着他的幕僚说道:

“让沈起来重新体量秦州荒地,根本是个笑话。沈兴宗他向来看重清议,没胆量站在王韶哪一边。但他本人又是个知进退的人物,不会与辅臣过不去。他那个性子,到最后肯定是和个稀泥,想着两边都不得罪。翔卿你看着吧,沈起最后肯定会说,秦州荒田既不是王韶所说的万顷,也不是窦舜卿、李若愚说的一亩都没有,而是在两三千顷上下。他若是不这么讲,我把脑袋输给你!”李师中平常就是一张大嘴,在私底下,更是口舌无忌。

“现在重要的不是这件事吧?!”

姚飞摇着头,他要李师中的脑袋作甚。把李师中丢下的公文捡了起来,他说道:“沈起怎么样都好,天子连亲信侍臣的话都不信,还派了沈都转运再来秦州走一趟,天子的偏向已经不言自谕。”

“王韶团聚七家蕃部,灭了托硕部一事,已经深得圣眷,这我看得出来。但这是王韶的本事?!”李师中想起王韶当日在军议上的模样,完全不似作伪。而王韶最后突然一改初衷,跑去古渭,却是在他探望过称病的韩冈之后的事了,“韩冈才是运筹帷幄之人。”

“是与不是并不重要,韩冈才智再高也不过一个从九品,真正有威胁的时候,要到十几年后了。现在王韶才是相公你要在意的。”姚飞尽着他作为幕僚的责任,向李师中提着自己意见,“向宝中风,近日必然去职。新任钤辖少不得在关西选调,若是让张守约升上来,王韶更加难治。相公还是早做打算,在临路挑一个合适的人选,向上请命。”

李师中没有即时回答,而是犹豫了一阵,最后吞吞吐吐的问道,“翔卿你说……天子究竟有多看重王韶?”

李师中后悔了!

多少年的交情,姚飞一眼就看得出来李师中是后悔了。这也难怪,李师中错估了天子的决心,以为王安石根本无法与韩琦、文彦博等人较量。所以他一直站在王韶的对立面,但眼下的这种情况,却是李师中始料未及。

姚飞摇着头,一针见血的指出李师中的想法不切实际:“现在再去结好王韶已经来不及了。而且王韶此人性格独断,绝不喜欢与人分功。再有两天,高遵裕就要到秦州了,到时王韶说不定会被他赶出秦州城,河湟之事,也就与他无关了。”

“对了,还有高遵裕!”李师中先点了点头,又摇了摇头,“先是内臣,现在又是外戚,如今的官家怎么尽用着这些人?”

姚飞不接口,想了想便将话题转开:“对了,这两天王韶不知在做些什么,让韩冈给他家里一口气弄了近百斤蜜蜡。”

“蜜蜡?近百斤?王韶这是想做蜡烛来卖吗?”

“这就不知道了。”姚飞摇摇头,也无意去深究,把李师中的注意力引开就够了。

……………………

蜡烛比油料要贵,故而世间多用油灯。能用得起蜡烛的人家,家底都是一个比一个殷实。

韩冈平日在家读书,到了晚上便不是用得蜡烛,而是点起油灯。不仅是韩冈,王韶平常也是一样节省。不过他们提供给田计制作沙盘的蜜蜡,却是一用几十斤,一点也不觉得心疼。

田计重新制作更加精细的沙盘模型,用去四天时间,蜜蜡总计费去了近百斤。无论王韶王厚,还是韩冈,都为了这块沙盘耗尽了心神和精力。

韩冈在这段时间里,通过沙盘的制作,使得他对等高线地图的认识加深了不少。一开始制作沙盘,只是对着旧制的简陋舆图来模仿,从那种地图上,分不清山势高低及河道流转,都得靠王韶王厚通过记忆一点点的加以修正。

而现在画上粗浅的等高线地图,线条细密的地方山势陡峭,线条稀疏的地方地势平缓,打造沙盘起来,一下方便了许多。同时关于这些认知,连王韶、王厚都已经了如指掌。另外还有地图的比例尺,也是一样被韩冈提出,而后被采用。不过比例尺的问题,也是王韶王厚的估算。为了把沙盘长宽的缩小比例确定,王韶还让韩冈去了架阁库,把前些年绘制的地理舆图给翻出来,重新按照比例关系,将之复制对照。

“想不到制作沙盘还有这种窍门在。虽然等高线图乍看上去眼晕,但习惯了后,就能一眼看出地势变化。山岭河谷一目了然。”王厚半开玩笑半认真的逼问着韩冈,“玉昆,你老实说,到底是在哪里学来的?”

“学?我自己都不知道的事,处道你让我怎么说?”韩冈摇着头,“只是灵光乍现罢了。”

田计经过了四天来的辛苦,胡须变得乱蓬蓬的,头发也同样散乱,眼珠子中尽是血丝。他声音沙哑,仿佛锉刀一般,“韩官人灵光乍现得妙。日后再做沙盘,有了等高线图和比例尺,可就简单多了。”

“但事前就要把地图画好,比例尺量好,这准备工作要做的地方就很繁琐了。”

韩冈谦虚着,站在新制的沙盘前。这块沙盘不再是三尺方圆,而是接近一丈的大小,由纵五横五总计二十五块沙盘拼组而成。将王韶家的主厅,堵了个严严实实。

真要说起来,这副沙盘并不正规,与实际也有许多差距。就韩冈的记忆力,他甚至还发现某个地方少了几处山头,而另外一处,则多了一条支流河谷。但韩冈对此也不能肯定,他这仅仅只是凭着记忆而已,并非精心绘制的准确地图。

通过这些天的辛劳,韩冈是明白制作沙盘到底有多辛苦了。日后这些事,还是交给专业人士去做,自家只要加以审核就足够了。而眼前的这副已经做好的沙盘,因为是给皇帝看的,上面蕴含的信息已经绰绰有余。多一个山头,少一个山头都无所谓。

“也算是大功告成!”王厚也是累得精疲力尽,但他心中很兴奋,再过几天他就要压着俘虏去东京面圣,这样的荣耀不是因为他的父亲,而是有着他自己的一份功劳。

王韶则是没多话,默默的回到自己的房中补眠,他也是同样的辛苦。而且王韶的年纪摆在这里,不比韩冈、王厚他们能熬夜。

王厚半俯着身子,看着沙盘,上面的河流树木、荒漠山林,都是用着不同颜色的木屑表示出来,这也是韩冈的意见。

王厚再一次赞叹了田计的手艺杰出,另外又道:“田员外,帮我做几个小泥人,好放在这副沙盘上。”

“做什么?”这是韩冈在问。

“充当各城各寨的守军。”王厚眨了眨眼睛,对着韩冈笑道,“愚兄过去有闲时,总喜欢看着舆图指点江山。不过旧日的舆图看着就乱得很,也没个什么用场。不想这几天,有了沙盘出来,过去梦寐以求也难以做好的事,如今却是轻而易举。”

田计动作很麻利,一切都是熟工,三下五除二,就是一批十几个泥质兵人,摆在王厚的面前。这些泥兵人姿态各异,惟妙惟肖,有的骑马,有的步行,简简单单的几刀,却把军中男儿的气概雕了出来。

王厚轻轻拿起一个小兵,放在沙盘中秦州城的位置上,“秦州有兵近六千,分属十四个指挥,其中骑兵两个指挥,剩下的都是步卒。”

他紧接着又拿起另外一个兵人,放在甘谷城的城防处上,“这是甘谷城的兵。甘谷城总计有八个指挥,两千五步卒,四百骑兵。”

第三个兵人放在水洛城,“水洛城中有兵两千,五个指挥。”

第四个兵人放在古渭寨,“这里守着两千步卒,另外最近又多了三个指挥的蕃落骑兵。”

看着王厚在沙盘上,做着有些幼稚的游戏,韩冈突然醒觉。军用沙盘的真正用途,不是拿给天子看,也不是用来攻击政敌,而是在开战前,进行战事得失成败的计算,并且对战术计划拾遗补缺。

看起来自己的真是有些糊涂了,连沙盘最大的用处都忘了利用。有了沙盘,也不用在战前烤乌龟壳来判断吉凶了——虽然是殷商时的事了,但在此时,为将帅者还是要学着算命的技术。在武经总要中,专门有一章在说该如何占卜胜利。

“处道兄。”韩冈上前一步,“这沙盘不是这么用的。”

