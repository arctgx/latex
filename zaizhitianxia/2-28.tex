\section{第九章 长戈如林起纷纷(一)}

【第一更,求红票,收藏】

预定中的献俘仪式给枢密使文彦博给搅了。

据文彦博所说,托硕部其实不过秦州边境的一个小小的蕃部,丁口即少,兵力亦自不盛。王韶领着几个蕃部击败了托硕部,纵然是连族长也俘获了,其实也不过是些微不足道的功劳。这样也敢押至京城来献俘,实在有失朝廷体面。想当年,曹玮在秦州,他所消灭的大蕃部有几十上百,而如托硕部一般的,更是车载斗量,却也不见他一次又一次的献俘陛前。

文彦博的这番话,让王厚心中愤愤不平。即便他因为参赞军务、押送战俘、以及献上沙盘、军棋等事,被天子赐予了三班借职的品官,又跟着张守约一起,被越次招入宫中面圣,王厚的心中,还是有犹有余怒。

但文彦博拿着曹玮来跟王韶比较,就是王韶亲至,也只能低头受教,道一声‘文枢密说得正是’。

曹玮曹宝臣,是开国名将曹彬之子,也是如今曹太皇的亲叔。他是真宗朝时镇守关西的第一名将,名震西陲。听到他的名字,无论党项吐蕃,小儿也不敢夜啼。别看现如今党项、吐蕃闹得如此欢腾。当年在曹玮面前,李元昊的老子李德明,吐蕃赞普唃厮罗,都是老实做人,哪个敢轻举妄动?——早给他杀胆寒了。后来若是曹玮不死,有他虎威镇着,李元昊绝然不敢做反。

可是这等英雄人物,也只会出现在开国之初的时代。放到现在,又有哪位将领能比得上曹玮的一根脚趾头?即便是狄青狄武襄,他升任枢密使,也不过是灭掉了一个在广西叛乱的侬智高,何德何能跟曹玮相提并论?而狄青之后,国朝武功日衰,王韶今次斩首六百,败敌逾万的功劳,已经算得上当今天子即位以来,仅次于围绕着绥德城的两次大战,而能排在前三的大功了。

崇政殿外,王厚突然低头轻咳了两声,掩去心中突然腾起的尴尬。不过这个大军万人是董裕和托硕部自己说的,不是王韶瞎编出来。自家老子在奏章中说今次败敌逾万,也不能算是欺君,而且六百首级可是实实在在的。

王厚的咳嗽声,引来几道不满的目光,他连忙低下头,不敢再惹起周围注意。

王厚的周围戒备森森,护翼天子的班直护卫皆是重甲持戈——其实也不是戈,而是一条条长柄骨朵——身材则是一个比一个高大。王厚五尺六寸的身量不算矮了,但在他们面前却硬是低了一头去,让他自卑不已。即便是韩冈来了,站在他们中间,也都只能算是中等偏下。

王厚听说宫中的班直,有许多都是世代相传,自太祖的时候就开始在宫中应付差使。而他们娶妻也往往都是刻意挑着身材高大的女子,这样一代代传下来,一个个都是六尺有余。几十条大汉并肩站着,就像一根根庭柱笔直的撑着天空,气势煞是迫人。

今天早早的吃过午饭,在张守约的提点下,连口水也没敢喝,王厚进宫在崇政殿外等着觐见。到现在也不知等了多久,他站得腰酸腿疼,却还没有个消息。不过王厚前面的张守约,花白的头发在长脚幞头下露了出来,已经都是花甲之年,站了那么久却仍是一动不动。而环绕着崇政殿周围的班直侍卫们也是一动不动。

这么多人围着皇城的中心站着,动也不动,连一声咳嗽都没有,王厚都感觉着静得吓人,仅有的声音还是不远处,从崇政殿内传出来的,另外……就是风声。

可能由于周围都是高近十丈的殿阁,风在殿阁间穿梭,呼呼的刮得甚急,使得穿着厚重朝服的王厚,一点也不觉得热。感受着寂静中清凉,王厚突然想起来,自他进了皇城后,却是连一声蝉鸣都没听到。今年天气热得早,京城中的树上早早的就有知了在吵,但偏偏在宫城中一声都没听到。

‘还真是奇怪,难道是天子之威,能够远驱蛇虫?’

王厚胡思乱想着,心中的想法可算得上是不敬天子。这时一阵凉风突然迎面吹来,王厚将头抬起一点,用余光看过去,只见崇政殿紧闭许久的殿门终于打开了,七八人陆续从殿中走了出来。出来的人皆是衣着朱紫,显是身份极高。王厚忙把头垂得更低了一点,不敢有丝毫不恭。王厚也不知他们究竟是宰执中的哪几位,但个个位高权重却是不用说的。不过如果文彦博在里面,王厚却希望他能在哪里踩滑了脚,跌上一跤。

只看着一条条红色和紫色的朝服下摆从眼前穿过,黑面木底的官靴踩着地板夺夺的一串响声渐次远去,崇政殿里终于空了下来。

‘终于能进崇政殿了。’

王厚抖擞精神,等着天子的传唤。可是出乎他的意料,天子的传诏并没有立刻出来。又等了大概半个时辰的样子,才有一名小黄门走了出来,将张守约和王厚叫进了崇政殿中。

王厚还是第一次觐见天子,连宫城也是第一次进来。关于崇政殿的一点常识,还是从王韶那里听来。

当举步跨入大宋帝国的中心地带,从亮处走进暗里,周围的光线随之一暗,王厚的心中便是一阵发虚。他跟着张守约亦步亦趋,唯恐哪里的礼节出了错,被站在内殿外的阁门使说成君前失仪。

在王厚入京前,韩冈还跟他开玩笑的说过。当见了天子后,不知他是战战兢兢,汗不得出,还是战战惶惶,汗出如浆。当时王厚撇着嘴,拍着胸脯说自己当是气定神闲,能闲庭信步。但现在,王厚连自己到底是出汗还是没出汗都弄不清了,鼻子里嗅到的薰香让他的脑袋更是发晕,耳朵里嗡嗡直响,使他根本听不明白天子驾前的宦官究竟再说什么,只知道当跟着张守约行动,学着他的动作,这样才不会出问题。

而就在这一段度日如年的时间,王厚心里却莫名其妙的蹦出了与韩冈的对话。他这时候才举手认输,在天子面前气定神闲的本事,果然不是没经验的人能拥有的。

张守约则是很淡定。他年轻时曾经镇守过广南西路,担任走马承受一职。当其时,狄青狄武襄刚刚平定了侬智高之乱,当地民心未定,乱军时有出没。当时的仁宗皇帝对广西局势甚为忧心,故而张守约便能两年四诣阙,每次入觐,都会被天子留下来说话,问着广西的现状,同时征求他对处理南方边事的意见。

而英宗,还有现在的年轻官家,张守约也都是见过的,心中更没什么负担和压力。进殿后,就按着礼节一板一眼的向天子行礼,经验丰富的老将给身后的年轻人,做出了最好的榜样。

跟着张守约三跪九叩,王厚就算站起后,也是深深的低垂着头,做足了恭谨的态度。对于崇政殿内部布置他不敢多看,不远处天子的御案他不敢多看,而天子本身,王厚当然更是不敢贸然看上一眼。只是他一拜一起之间,眼角的余光却瞥到挡在连通后殿的通道前的一扇屏风。

那扇屏风上没有花样,没有纹饰,底色只是普通的下过重矾的白绢。但屏风面上,却密密的写了不少字。白纸黑字,醒目无比,而且都是三字一段,两字一隔——皆是人名。

那一扇就是传说中的屏风,王厚从他父亲那里听说过,能被写在这扇屏风上面的名字,都是曾经给天子留下深刻印象的小臣。上面的每一个名字,皆尽是天子亲手所书。等待日后有机会,便可以从其上简拔。

无论哪朝哪代,除非是不理事的昏君,或是为臣下反制的有名无实的君主,所有的皇帝都免不了要日理万机。开国以来的历任天子,也不会例外。他们每天要批奏的奏章数以百计,奏章上提到的名字则更是近于千数。而且文官选人转为京官,武官小使臣晋升大使臣,也都必须要觐见天子。每隔几天他们就会编为一队,引见给皇帝。

几百人上千人的名字就这么日复一日的在皇帝面前晃着,即便他们有再好的记性都背不下来、跟不上去,除了十几二十个重臣,还有在身边服侍自己的内侍,剩下名字一年也不一定能出现一次,天子哪可能记住?往往就会记错人和事,张冠李戴的情况也时常发生。

所以为了防止遗漏人才,崇政殿中便有了这扇屏风。但凡在奏事和觐见上给皇帝留下了好印象的小臣,无论是外臣还是内侍,天子都会提笔在屏风上记下来。据传言,不仅仅在崇政殿里有一座记名屏风,在天子寝宫福宁殿中,也有一座同样的屏风——这是为了天子无论何时想起,便能随手记下

王厚虽然对记名屏风很有兴趣,但在觐见天子时,紧张的心情本也不会让他太过在意。只是王厚方才叩拜之间,视线不经意的扫过屏风。视力出众的他,却是亲眼看见就在屏风靠右的一侧,有个名字单独起了一行,那两个字让王厚分外眼熟——

——韩冈。

