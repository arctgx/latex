\section{第12章 共道佳节早(二)}

韩冈是在城门处与种建中兄弟分了手。同行半个多月,互相之间的交情又深厚了几分。与他们定下了过几日去种谔府上拜访的约定后,韩冈便动身前往王韶府上。

崇仁坊的王枢密宅倒是好找,几个月前宣德门前的献俘大典,让王韶的名号传遍了东京城。韩冈只让伴当对新郑门的租马人问了两句,那位四十多岁的老开封就很热心的给韩冈一行三人指点了一番。

到了王韶家的门前,新科枢密副使府邸前的街巷,也跟前两次上京时,韩冈在王安石家门前看到的情况一样。尽管数量上无法比较,但拥挤着大批等待接见的官员那是不会变的。

‘炙手可热啊……’韩冈暗自感慨着。从偏鄙小臣一步登天,王韶如今可是如今大宋朝中,最让人羡慕的角色。

以王韶的年纪和功劳,只要不犯错,命再长一点,日后升任枢密使乃是板上钉钉的事。如果再遇上北方边境起风波,需要重臣坐镇东府,王韶甚至有望一探宰相之位。要知道,韩琦也罢,富弼也罢,他们升任宰相时,所立下的功绩都远远不如王韶。

拥挤在王韶家门前的这些来干谒的官员,就算一时不能被提拔,可为了日后的前途着想,现在也要在王韶面前混个脸熟。

韩冈停在人群外,看着门外的这么多官员,王韶肯定是在家的。也不多话,直接遣了伴当上去叫门,以他跟王家的关系,递门帖什么的反而就生分了。

见着新来的年轻书生,下了马后,派了伴当去找王府的司阍。四周的官员都暗笑着,这个小子糊涂,哪有到执政家门前不亲自送门状的?

惹怒了守门的司阍,把门状放到最下面,那就不知道什么时候才能被接见了。再看韩冈没有穿着官服,摇头的更是多了。

就在韩冈边上的一位官员踱了两步过来。他凑近了,对韩冈道:“这位秀才,你可是做岔了。王副枢家的大门,怎么能不自己去敲?”

韩冈正眼看过去,这一位四十多岁,身上的官服上带着油斑,恐怕有一年没换了。听口音当是江西人,跟王韶平日里不自觉的带出来的乡音很是相似。

见着这位应该是久迁不调的老选人目光灼灼的盯着自己,韩冈心中透亮。这哪里是好心的提点,根本是在试探自己的身份。

韩冈拱了拱手,算是道谢,“多谢尊兄提点,却是不妨事的。”

果然,见到韩冈如此态度,这一位的神色立刻就亲热了起来,“难不成兄台是王家的戚里?!”

“倒也不是。”韩冈摇了摇头。

来自江西的老选人心下一齐,正要再问上两句,王府门前忽然一片骚动声。

抬眼望过去,就见着王家门前的两个司阍,年长的一个如尾巴被烧着的兔子一般一下蹿进了府中,另一个则是挤过拥挤的人群,两步就在韩冈面前跪了下来,“小人拜见机宜。”

围观的众人齐齐一惊,这位不懂礼数的年纪人竟然是个官人。再听着王家看门人对韩冈的称呼,其中几个脑筋转得快的,立刻就反应了过来。

“是韩冈韩玉昆!”

“是熙河经略司做机宜的韩冈!”

“推了上京的献俘大典,锁厅考试的。”

“想不到是他!”

韩冈的身份暴露众目睽睽之下,一片哗然之声猝然响起。韩冈全当没听到,他微笑着将王家的司阍,此前也是王韶的亲兵扶了起来,“早锁厅了,不是机宜啦。”

“是!是!”司阍头点得如小鸡啄米,为韩冈在前面引路:“机宜……官人且跟小人来,枢密和二郎听到官人到了,肯定欢喜得紧。”

韩冈冲着身边发着愣的老选人拱了拱手,便跟着司阍走进了王府中。

王厚这时正奉着父命过来迎接,见到韩冈,欣喜难耐。一边喊着,“玉昆,你总算到了!”,一边就拉着韩冈去见王韶。

先是畅叙一番离情,王韶便拉着韩冈,向他引见了自己的家人。除了次子王厚,王韶还有另外几个儿子,除了长子出去了之外,其他六个,都向韩冈一一介绍过,连同妻女都跟韩冈见过礼,全然没有把韩冈当外人避着。

一番纷扰之后,王韶、王厚和韩冈一起进了书房。说了几句别后的近况,王韶问着韩冈:“玉昆,你今天入城,可有去中书?”

“都锁厅了,进京难道还要去中书报到?”韩冈不解的反问着。

“这倒不是。”王韶向着韩冈解释,“但玉昆你不一样,你是简在帝心啊!天子若是知道你上京了,肯定要召见你的。但你也不去中书露个面,天子何从得知?”

韩冈摇摇头,“老是拒绝天子的诏令不太好,还是等考完后再去上书请对。”

“……玉昆,你难道不想诣阙?!”王厚惊问着,“你不想做官了?!”

“怎么会?一睹清光,聆听德音,做臣子的哪有不愿的?但礼部试之前就不好见。若是考前见了天子,未免会有瓜田李下的嫌疑。韩冈的名声倒没什么,若是让人误会天子处事不公那就不好了。做臣子的,岂能让天子受此污名。”

韩冈如此说着,他的话语中,听起来隐隐的有这股刚正严毅的傲气。

如此义正词严的忠良之语,王韶一听,却哈哈大笑了起来。王厚也在笑着,指着韩冈:“玉昆,你这终南隐士的手段,怎么做到朝中来了?”

韩冈先板着脸,却撑不住也笑了。自己的脾性王韶、王厚都清楚得很,丝毫瞒不得他们。

其实这是最朴素的饥渴营销法。

天子一直想见韩冈,却是阴差阳错,始终不能如愿。现在已经时熙宁五年,该诣阙的朝官们早轮换了一遍。如今满朝文武,天子没见过面的恐怕也就韩冈一人了。最年轻的朝官,又是屡立功勋,天子对于韩冈的期待之心,那是显而易见的。

只是对于韩冈来说,既然吊胃口已经吊到了现在,那就干脆把皇帝的胃口再吊到进士科举时也没关系。保不准他在礼部试上出了点差错,天子一句‘怎么不见韩冈’,就把他又拉回来了。要是先见过面,天子已经给了恩赏,礼部试时,再出手的可能性就要低上不少。

韩冈看似对即将到来的礼部试胸有成竹,但他其实还是战战兢兢,千方百计的想办法一点点的积累自己的成功率。就算是再微小的助力,韩冈都会设法维持住。为了一个进士资格,就算是盘外招,只要有效,他都会用上。

“那王相公那边,玉昆要不要去见一见?”王厚又问着韩冈。

韩冈大摇其头:“天子都拒了,怎么能去见宰相?一切等考完试再说。”

“考完?……”王韶沉吟了一下,便单刀直入的问着,“玉昆,你到底准不准备与王介甫家结亲?”

韩冈笑了:“如今韩冈儿女皆有,家慈也不再催着了。”

韩冈文不对题的这句话,王韶听得明白。要想跟他结亲,不用去陇西找韩冈的父母,直接找他本人就行。

“玉昆,既然令尊令堂都不会干涉,此事当是得由你来决断。不过在我看来,你还是与王介甫结亲得好。”

“韩冈也没说不结亲啊……只是要到考完之后,再给个明确的回复!”

王韶摇摇头,“还是早点确定得好。可以先给个准信,等考上进士就娶。”

韩冈看出了王韶的态度与几个月前有了些变化,皱起眉来,“最近可是有什么大事?”

“最近东西二府加上御史台为了点小事,闹得不可开交。我这边也不得不与王介甫闹了一次。”

王韶也不瞒着韩冈,将这些天来,朝中发生的大事,向韩冈详细的说了一遍。

“让东府和御史台都吃了一个大亏,吴冲卿还真是本事!”听完王韶的一番叙述,韩冈啧了啧嘴,对吴充的手段很有些佩服。能跟王安石做亲家,果然也不是省油的灯。一句没头没脑的流言就扭转了局势,难怪天子能放心的把文彦博请出朝堂。

“不过王相公就这么吃了个哑巴亏不成?”韩冈问着。据他所知,王安石的脾气可没这么好,不会左边挨了一巴掌,就把右脸送上去的。

“可不就这样吃了!”王厚挑了一下眉,冷笑着。

韩冈看看王厚,又望望王韶,眯起眼笑了起来:“这样情况下,枢密还要让韩冈娶王相公家的女儿?”

“要不是我这边没有好人选,怕日后变成冤家,怎么都不会让给王介甫的。”王韶很坦率地说着,“天子年岁渐长,王介甫不可能再像熙宁初年的时候那般得圣眷。他的宰相之位恐怕也做不了几年了。只是除了个别的法令外,新法还是会被天子推行下去。玉昆你也不用有什么避忌!”

“避忌?”韩冈呵呵笑道,“除非王家的女儿性格不堪,那才要避忌。如果三从俱守,四德皆备,韩冈哪又不愿的道理?!”

