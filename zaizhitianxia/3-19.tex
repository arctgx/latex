\section{第九章 纵行潼关道(上)}

韩冈在横渠镇上盘亘了三日,期间多次与张载等人讨论他所提倡的格物之道。而他关于日月星辰的观点,甚至也已经广布到普通的学生之中。

其中有人有会于心,有人全盘接受,可也有不以为然的,更有吕大临这般严厉驳斥的。

吕大临的口才在张载门下应该算是很突出的了,引经据典的本事韩冈也望尘莫及。第一天夜中的讨论,韩冈试图用自己将力学原理和儒学词汇结合起来的解释,来向张载等人阐述后世的经典力学。而吕大临的几句话,就一把抓住韩冈言辞中的漏洞,压得他差点败阵。

一个是韩冈本人水平不够,闭门造车、勉强拧合出来的东西,当然不可能像后世文字都经过千锤百炼后的定律那般完善。同时也是韩冈本人状态不好,连续赶了几天的路,本就是累着,熬起夜来,虽不至于说胡话,但脑筋转得就比平常慢上了一点,当然不是养精蓄锐的吕大临的对手。

艾萨克爵士不是那么好当的,《自然哲学的数学原理》这本书韩冈听说过,却不是他的水平能写出来。现在的情况,是韩冈可以通过日常现象来推导出结论,却无法用数学精确的描述。韩冈的空口白话,加上并不完备的词句,那一晚的辩论,当然显得有些苍白。能将他的观点顺利传达,就已经是他过去与张载书信往来后的结果。

而到了第二天起来,韩冈回头一想,却是大骂自己糊涂。物理之道本就不是口舌之争——摆事实,讲道理,实验才是第一。光用嘴说并不直观,以实验证明自己的理论,比吵上三五年都管用。

靠着已经冰结起来河渠,还有几个小物件,韩冈很轻易的就粗略的证明了第一和第三定律。而需要精确测量和计算的第二定律,虽然一时无法证明,可已经用实验证明前两项定律,也足以让围观众人连带着也相信了八分,甚至更为难测的万有引力之说,竟也有人信之不移。

吕大临的驳斥依然严厉,可在事实面前就让人难以信服了。证明自己的观点,只要其中能有一事让人信服,其他观点也能让人连带着相信。韩冈用的手段近乎于此。可惜这只是辩论术,而不是科学的论证方法。

但赢了就是赢了,韩冈也算是松了一口气,他此前绝没想到,自己的一番心血,竟然会有这么多漏洞。要不是这几条定律有着天然的正确性,以及可以用实验来证实,自己可是要丢大脸了。

不过吕大临的驳斥,对韩冈来说不是没有好处。他连番攻击,让韩冈注意到了自家理论中的漏洞。不仅仅是可以将这几条定律更加完善,而且对于之后即将面临的批驳,有了心理准备,更可以做好反击的准备。

只是吕大忠、范育等人,在几番激烈的讨论过后,神色间却都隐隐的有些忧色。这样的态度,让韩冈觉得有些纳闷,便登门请教。这一问方才知道,他们是在担心士林中的议论。

尽管有识之士都能看出这一套格物之说对于儒学压倒释老两家的意义何在。但有识之士毕竟是少数,而喜欢找碴、贬低对方的文人,却是车载斗量。

张载本来就是说着‘民,吾同胞’,在士林中,隐隐有人讥刺他已近墨家之流。现在韩冈的一番实验,却是墨家更为接近。这就不免让人担心起世间的议论。墨子要世人兼爱,视之为兄弟姊妹,孟子驳斥为无父无母之论。与墨家相合,这个罪名,关学当不起。

另外,万有引力之说,直捣天人感应的根本腹心。吕大忠曾半开玩笑说,如果此事确认,日后国史中的天文志就要大改,而钦天监怕是也要头疼了。而且太宗曾有诏令,禁止私下妄习天文。虽然如今已是法禁宽松,被人抛到脑后。可真的要有人根究起来,也是一桩麻烦的事情。

但韩冈也是出于无奈。

汉儒唐儒在传习经义时,很少论及宇宙天地,至少比起如今的各个学派,要少上许多。现在不论是关学、理学,还是王安石的淮南学派,当头第一桩说得便是天。先论宇宙自然,其次才及人,而不是前代儒者那般,以人世为主——这也是跟佛老相对抗的结果。为了能配合如今的风潮,为了能吸引张载等人的注意,也为了能将物理顺利融入关学之中,万有引力是必须加上去的一条。

故且不管这么多了。

毕竟忧虑的只是吕大防等弟子,而张载本人,却是丝毫不在意。他一心根究大道,哪还在乎这点凡俗小事?

在横渠书院中几天的叨扰,韩冈大有所得。但看看行程紧迫,也不得不向张载辞行。

张载没有挽留韩冈,只是写了几封信让韩冈顺道带给关东的亲友,并出面为他饯行。

今科举试,横渠门下去京城参加科举的并不少,而出自陕西的士子那就更多了。张载在饯行宴上不忘嘱咐着韩冈:“今次上京,不仅仅是考试,也是结交四方友人的时候。玉昆你才智眼界学问皆远过常人,唯一可虑的就是你的骄心。年纪轻轻就身居高位,是好事,也是坏事。切莫崖岸自高,要平等待人!”

张载的谆谆教诲,殷勤嘱咐,让韩冈感动不已,当场拜谢下来:“多谢先生指教。”

见韩冈诚心实意,张载也很是满意,特地指了几个今科参加考试的学生,让韩冈有空便去拜访、结交。

韩冈点头答应了下来,又笑道:“其实还有好几个。种建中,就是种太尉的那个侄儿,他今次也上京赶考。”

离乡的前两日,韩冈还收到了种建中的一封信。上面说他今科也要去京城参加考试。想来他会住在担任龙神卫四厢都指挥使的种谔府上,到了东京之后,应该很容易就找到他。倒不像张载前面提到的几个,诺大的东京城,百万人口之众,没一点明确的线索,根本找不到人。

听到韩冈提起种建中,张载沉吟了一下。

“是字彝叔的吧?”他还记得种建中这个学生。种谔的侄儿这一身份不提,几次春来射柳,总是排第一的弟子,印象总不会不深,“他的学问还有待磨练,怎么这么早就去了?”

“彝叔考得不是进士,而是明法一科。”韩冈为种建中解释道,“他本来就已经有官身了,不过他还是想转为文官,需要考个出身。”。

旧时科举,进士考诗赋,明经靠经义。现在进士也考起了经义,理所当然科目中便再无明经,而是改成了明法,考律令断案。这也是王安石为了让刑名专业化而进行科举改革——因为不熟悉律令,被胥吏所欺的官员数不胜数。

尽管选人转京官,一般都是要考断案和律令,以防止新进京官担任知县一级的亲民官时,无法胜任这等重要的职位。不过条贯虽好,却架不住当事者不去遵守。

审官东院一般不会再这一项考试上卡人——选人能转官,背后无一例外都站着路一级的高官显宦,没事谁敢得罪他们——最后转官出来的官员,还是要被衙门中的胥吏欺瞒。

王安石想改变这样的现状,所以便有了明法科。

只是虽说进士科改以经义取士,对陕西等北方士子来说,是个利好的举措。但明经科取消,以明法科代替,对北方士子而言,却是不折不扣的坏消息。

“明法科。”张载摇头叹了口气,“玉昆你去考进士,今科上榜的应该能见到不少同乡。只是……”

韩冈知道张载想说什么,接过话头道:“只是如果将明经科也算进来的话,论起整体取士的数量,今科能进学的陕西士子很有可能会减少不少。”

世人皆知,论起经义,北方士子与南方士子的差距,要远远小于诗赋。可轮到刑名之道上,北方人仍是远远比不上南方。

相对于向来对衙门远避为宜的北人,南方人就不怎么怕去衙门里打官司。尤其是江西人,好讼那是天下闻名的。市井中一点鸡毛蒜皮的小事,就会拉拉扯扯的到衙门中要求评理,让县官们不胜其扰。

而且江西乡里村学中,教授的课本往往不是《论语》,而是《邓思贤》这样的教人如何打官司的律讼书。靠着风土人情的熏陶,江西连十岁小儿都能在衙门上侃侃而谈,让县官下不了台来。

“南人好讼,北人难及。好讼之地,其民往往好辩。遇事偶不合,便执之而喋喋不休,必欲使人雌伏而甘心。”张载边说边摇头。

韩冈记得张载貌似并没有在江西任过职,而且看他老师的神色,似是意有所指……听起来,多半是在说王安石。

王安石的确有这个毛病,早两年,天子和他意见相左时,都是天子败下阵来。

但张载并不是在指责王安石,而应是想起了旧事在感叹而已。既然没有明言,韩冈便半开玩笑的说:“如切如磋,如琢如磨。如能在江西好生切磋琢磨一番,天下州县都能去了。”

韩冈歪用诗经里的文字,让张载为之一笑。

他这个弟子的确会说话,而且不是圆滑油滑的那种,言辞行事中,年轻人的锐气并不缺。张载不由得想起当年去向范仲淹上书时的自己。

但这个学生,可比自家当年强多了。

一番酒后,韩冈向张载行过礼,便出门上马,告辞远去。

路边田地,阡陌纵横如井字。世间多有赞着周时井田,复古之说,二程、安石皆有言及,但众家学派,也只有张载将之践行。

重实证,轻言语,这便是关学的根基。

