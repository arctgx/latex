\section{第35章 把盏相辞东行去(五)}

韩冈一行抵达东门时,王韶和吴衍还没到,却见到了另外一拨送行的队伍,正是刘仲武。这位得了向宝青眼的年轻军官,被一群人簇拥着,依依而别。向宝没有出来送行,但他还是派了一个亲信。两拨人马都挤在城门内外,靠得很近,但互相之间连个招呼都不打,完全视而不见。

“要不要跟他一路走?互相也好说个话。”王厚开着玩笑,声音大了点,刘仲武好像听到了,头动了一下,又立刻转了回去。

韩冈洒然笑着:“我是无所谓,但他怕是不干。不闻向钤辖气量有多大,跟我走在一起,回来后,刘仲武有的是小鞋穿。你看,果然先走了!”

刘仲武走得貌似急了点,仿佛在逃跑,送他出行的大队朋友中有十几个跟着他一起上路,他们都是跟刘仲武关系特别好的亲友,按习俗都是送个五六里,七八里,九十里才会回转。而韩冈这边,王厚也在十里铺那儿准备好了酒席。

黯然销魂者,唯别而已矣。古时交通不便,一别之后,再见便难知时日。但这对韩冈并不适用,现在在场的都是年轻人,春秋正盛,而且韩冈只是去京城打个转,很快就要回来。也没有十里相送的惆怅,而是预祝韩冈一路顺风的欢快。

一片喝道声从城中远远的传到了城门口,韩冈一众循声望去,只见旗牌之后,王韶与吴衍并辔同行,正往城门这里过来,而行在他们身边的,竟然是秦凤路走马承受刘希奭。

‘想不到他也来了!’

……………………

秦凤经略使的书桌,已经被一幅八尺长、四尺宽的熟宣所占满。用明矾蜡过的上等宣造,衬在幽沉黯哑的漆工桌面上。纸面中的楼台亭阁、花石人物,为工笔素描,各个鲜明无比,惟妙惟肖。

李师中一身青布道服,发髻上只插了根木簪,单看上去就像一个普通的老乡儒。他站在桌前,手执兔毫笔,盯着画面聚精会神。书房中的火炭烧得并不旺,但李师中的额头上却细细密密的尽是汗水。一旁磨墨添水的书童,屏声静气,墨块研磨间,不敢发出丝毫声响。

一幅《菊酒忘归图》,李师中从动笔开始,到如今已经超过了三个月。一遍稿,二遍描,刚开始的一个月虽然事忙,却很快的画完了大半。但自从……自从……好吧,李师中承认,自从韩冈这个名字传入耳中,乱七八糟的事便一桩接着一桩。在自己还没有觉察到的时候,本已经被他打压了近一年的王韶,竟然在收了韩冈为门生之后,转守为攻,不但连络起张守约和吴衍,甚至还在年节前直奔古渭,自己哪有心情再画下去……

不需通报,姚飞径直走进李师中的书房,先横了磨墨的书童一眼,示意他离开,而后低声向秦凤经略禀报他刚刚得到的消息。

亲信门客的声音入耳,李师中低头仍看着画卷,头也没有抬上一下。片刻之后,方将画笔饱蘸了浓墨,在画卷上添了几笔,寥寥数笔,又是一名憨态可掬的醉客跃然纸上。放下手中兔毫,他才回头笑道:“韩冈今天上路,这不是早就知道的事?不说这个了,翔卿,你来看看,这画还有哪里须改的?”

姚飞轻轻叹了口气,也许李师中认为自己掩藏的很好,但他早已看出来,对那位才二十出头的士子,秦凤经略暗地里实则颇为忌惮。要不然,他也不会在韩冈进京的这一天,心情突然变得好起来。看来自己是要坏了李经略的好心情了:“禀侍制【注1】,刘希奭也去送行了。”

李师中脸色顿时一沉,本来轻松写意的脸上一下阴云密布,可停了一下,他转而又满不在乎的笑了起来,“走马承受又如何?不就是通着天嘛!想想种谔,他夺绥德是得了天子的密旨,依旨而行。文宽夫【文彦博】还不是逼着官家,把种谔贬到了随州待了两年,连传递密旨的高遵裕也被踢到了乾州做都监,最近才迁到西京去。”

真要斗起来,李师中半点不惧刘希奭。刘希奭背后的皇帝虽是天下至尊,但也并不是不可违逆,只要分出个是非对错,皇帝也不能随意而行,“朝中有君子在,有诤臣在,即便天子也做不得快意事,何况区区一个走马承受!”

“相公!还请慎言!”作为李师中的亲信幕宾,姚飞其实很头疼他所辅佐的秦凤经略安抚使的一张嘴。许多话心里明白就行了,说出来作甚?!不过若不是李师中心情激荡,也不会一下子冒出这么多话来。

李师中长于政事,兼通兵事,历任地方都能留下不错的成绩。姚飞几十年来辅佐过多名高官,大小官员见过成百上千,这么多人中,李师中的手腕算是一等一的,绝对是能力出众的官员。

只是李师中十五岁便敢上书议论朝政,入仕后,从没歇过他的一张嘴。在天子驾前,在宰辅面前,自吹自擂的情况多不胜数。李师中在朝野中留下的印象就是个好放大言的能臣。

姚飞每每为李师中叹息,就因为他爱乱说话,经常与当朝宰臣相龃龉,往往因为言辞而被黜落。若非如此,资历足够,功绩足够,年纪也到了的李师中,怎么会始终与宰执无缘?他升到侍从已经快二十年了,经略使也做过了几任,就差最后一步始终跨不过去!

“就怕韩冈去见了王大参,有他为王韶奔走连络,不知会在秦州搅起多大风雨。”

“王安石?”李师中不快的冷哼一声,“他能做什么?外臣中,韩稚圭【韩琦】反变法,富彦国【富弼】反变法,文宽夫【文彦博】一样反变法。宫里面,太皇太后、太后,哪个支持变法?王安石如今祸乱朝纲,闹得天下沸腾,坐不住他的位子的。我老早就说过,王安石一对眸子黑少白多,甚似王敦,迟早乱天下。”

“相公说的是!”姚飞清楚李师中很早以前便与王安石打过交道,只是两人甚不相和。确切的说,是李师中看王安石不顺眼。以至于早在两人刚刚入仕的时候,李师中便说过王安石迟早会乱天下。

这并不是什么秘密。

二十年前,包拯担任参知政事的消息流传开来,世间多有人言,‘朝廷自此多事矣’——包拯自身甚正,所以也要求他的同僚们与他一样端正,所谓严于律己,严于待人,做御史时,一份份弹章谏章,让朝堂同列苦不堪言,连仁宗皇帝都被喷过一脸口水——这样的人升任大参,当然让人担心他会闹得朝中鸡飞狗跳。不过李师中则说,“包公何能为,今鄞县王安石者,眼多白,甚似王敦,他日乱天下,必斯人也。”

其实类似的话,在朝野中不甚枚举。不说别的,富弼、文彦博哪个没被这样骂过,而相三帝、立二主的韩琦,被人弹劾说他有悖逆之心的奏章叠起来能跟他一样高。都是图个嘴皮子痛快,一千条也不一定有一条能对上,只是李师中恰巧说中了而已。

“可韩冈毕竟是官家亲下特旨授予差遣的,他的名字,官家总会留个印象。”

李师中依然不在意的样子:“官家记着又如何,昭陵【仁宗】不知道我的名字?厚陵【英宗,注2】不记得李师中这三个字?如今的官家会不清楚秦州知州、秦凤经略是谁?!皇帝心里记着人多呢!虞舜放四凶,你说虞舜记不记得四凶【注3】的名号?!”

李师中的声音不自觉的变得有些尖利,姚飞看得出他失态了。

本来无出身的文官,在二十五岁之前非特旨不得任实职的新条令,是在李师中后悔没有反对王韶三人的荐书时,突然递到面前的。当日李师中心情便好了不少,他面前的这张画有四分之一是在那一天晚上赶出来的。可到了第二天,政事堂和审官院批准韩冈为官的回复便送到了李师中的案头,里面还夹了赵顼的特旨。那一天,秦州州衙里奔走的胥吏便为韩冈吃了大苦,竟有十二个人挨了杖责。

“行了,我都知道了。”李师中最后平平淡淡的说了一句,代表他打算结束这次并不愉快的对话。

姚飞很识趣,告辞了就准备离开。李师中突然叫了一声:“翔卿,等一下!”

姚飞回过身来:“不知经略有何吩咐?”

李师中犹豫了一下,问道:“架阁中的……”

李师中欲言又止,姚飞却心领神会,立刻回道:“机宜前次的奏章王韶已经看过了。”

秦凤经略脸色稍霁,点点头,带上了一丝微冷的笑意,“看过就好!”

他低下头,心神重新沉浸在画卷之中。姚飞走出门去,望空摇头叹息。杀敌一千自损八百,这样的计策用着也是无奈。

注1:宋代的文官,尤其是八品的升朝官以上,身上的头衔不仅仅有本官、差遣,许多还会被授予馆职,标志文学高选,并非实职。如李师中,此时他的差遣是秦州知州兼秦凤路经略安抚使,本官是正六品的右司郎中,而馆职则是天章阁侍制。一般来说,因为宋代重文的关系,除了有上下级从属关系,其他情况下多以馆职来称呼。在如包拯,他在宋代通称为包侍制,就是因为他曾为天章阁侍制。至于包龙图,则是明代以后的事了——而且这是错误的称呼,因为包拯仅是龙图阁直学士,而非大学士,不够资格以龙图为后缀,只能被称为直龙或直阁。

注2:昭陵是仁宗陵寝永昭陵的简称,厚陵是英宗陵寝永厚陵的简称,此时士人的习惯,常常用陵寝的名称来称呼先帝。

注3:出自《尚书•尧典》,舜继承尧让出的帝位后,将原本是尧臣的共工、欢兜、三苗、鲧四人或流放,或诛杀。此四人便被称为四凶。鲧,是禹的父亲。

ps:因为李师中的天章阁侍制,顺便提一下北宋的官衔种类。

前面提到的本官和差遣,大家应该了解了一点。但北宋的官号除了这两项以外,还有其他几个职位系统:散官阶,这是定服色,也就是官袍的颜色用的,除此之外别无他用,继承自唐代;馆职,这是文学备选,一般京朝官中的少数人才有;爵位,公侯伯子男,不用解释;另外还有功臣,有功臣封号,便可入国史了;勋号,虚衔,无职事,无俸禄,只有个品级。

举个欧阳修的例子,做过参知政事、官场沉浮四十年的他,致仕前在亳州的头衔是:推诚保德崇仁翊戴功臣【功臣号】、观文殿学士【馆职】、特进【散官阶,正二品】、刑部尚书【本官,从二品】、知亳州【差遣】、上柱国【勋号,正二品】、乐安郡开国公【爵位】、食邑三千八百户、食实封一千户欧阳修。

今天第三更,求红票,收藏

