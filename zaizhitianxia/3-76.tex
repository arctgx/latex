\section{第26章 任官古渡西(一)}

九月的京城,离着孟冬十月不远,但头顶上的艳阳依然高照。虽然热力不比夏日,但干燥尤甚。韩冈入京已经有半个月了,这段时间里,他别说没看到一滴雨落下来,甚至没看到一个阴天。

“又旱起来了!”

在相府的书房中,韩冈与他的岳父和大舅子坐在一起,为这个干燥的深秋而苦恼着。

京畿和河北,今年夏天整整干了两个月。直到六月下旬之后,方才有所好转。而在关西,今年夏天的雨水虽说也少,但还不至于到了滴雨未落的地步。但也不能确定说今年秋冬雨雪还会丰沛,关中的湿气,也多来自于东面。东部若是继续干旱下去,关西的情况也不会好。

‘幸好出关后,看起来今年情况不对,就通知了家中多囤粮以防万一。’韩冈心中想着。

“不知道什么时候才会下雨,如果拖到今年冬天,情况就会不妙了。很快就要下种了,再没降雨,明年不知道会有几分收成。”韩冈对王安石和王雱说着。

“中书过两日又要去祈雨了。六月两次祈雨,倒是下了一些,六月底的时候,还去了东郊登坛谢雨。”王安石道,“七八两月雨水的情况都不错,跟往年比起来,也不差太多。”

“这事小婿也知道。”韩冈点头,“但去年河北蝗灾很严重,今年四月又闹过一次,七月时,更是从契丹的南京道那里来了一片飞蝗,这情况不对啊。”

“……玉昆你知道得怎么这么清楚?!”王安石都有些惊讶了,韩冈才分明才到京城没多久。

韩冈叹道:“外面都传遍了,只要在酒楼中一坐下来,不需要多打听就能知道。”他再叹一口气,问着王安石,“河北的常平仓怕是没有多少了。”

“三年耕,有一年之积;九年耕,方有三年之储。连续两年灾荒,河北的情况已经很糟了。”王安石心情也变得低落,同样叹着气。不过,很快就振奋起来,“值得庆幸的是,现在还没见到流民,河北的常平仓,还是支持住了。只要今冬明春雨雪依时,就可以安心下来。”

说是这么说,精神看起来也很好,但王安石眼中的忧心忡忡,却是瞒不了人。河北不是新党的地盘,每一项新法,推行的最为艰难的便是在河北,尤其是便民贷。

其实这跟民风也有关系。北方的百姓都不喜欢借贷,许多时候,宁可典卖家当,也不会跟人借钱。韩冈的父母就是最典型的例子,只要家中还有产业,宁可卖产业,也不愿借下子孙几辈子都还不下来的高利贷。

而南方民风奢侈,对商业也不像北方有所歧视,金钱往来也是很平常。所以对于借贷便没有太多的心结。但这样的性子,多有还不清欠账而破产的情况出现。

河北便民贷的推行情况,在官吏、民风的相互影响下,在全国是倒着数的。

因为借贷少,所以河北常平仓不会因为大部借出出而无力救灾的情况。但百姓之所以要借便民贷,本也是为了救荒之用。本质上都是一样的情况。

而且遇到大灾,朝廷也不会逼迫灾民还贷。维持国中稳定是统治者的第一目标,只要向上通报,基本上都能得到减免或展期。不会像地方上的富户,将债务人的妻儿都可以逼着去卖掉还钱。

不过朝廷也不会因此而亏本,这也跟保险差不多。如果遇上大灾,对于那个地方的保险公司来说,是肯定是赔。但放到全国,总体上还是有赚,便民贷其实也是一般,除非是遇上了全国性的灾情,否则从总体上不会亏本。

韩冈摇摇头,这事想得偏了。

“说得偏了。”王雱也转开话题,“今天找玉昆你,是想商量一下,玉昆你的差遣。”

“不是军器监吧?”韩冈反问着。

韩冈抵京后的第一次入宫,天子话里话外都想将自己安排去军器监,盼着他能再拿出一件与霹雳炮相仿的兵器。而且以韩冈在治事上的手段,可以帮着整顿各地军备生产。

但内定的判军器监吕惠卿不干,“诸司之中,正官为朝官,而副职则为京官或选人。韩冈乃是太常博士,又有贴职在身。若自此为定制,恐为不美。”

吕惠卿看似是在说让韩冈来做副手实在是太委屈了,但实际上是在说,判军器监这个位置他不会让出去的——虽然一个字也没有提到,但一切尽在不言中。

而韩冈那边,去军器监,他是愿意。但他决不愿意为人打下手。他若是在任上有所发明,功劳算是谁的?让给别人,韩冈可不干。若是枕戈待旦的时刻,大局观还算不错的韩冈不会讲究功劳谁属。但眼下事情又不急,内忧比外患更让人头疼的时候,有必要将自己稳抓在手的功劳送人吗?

两边都不干,这件事也就黄了。强扭的瓜不甜,想着将韩冈安排去军器监,那是让他去做事的,不是让他去赌气。

“不是军器监。”王安石摇着头,“一旦河北今冬灾情不减,必然会有大批流民渡河南下。需要得力之人将他们给堵住,决不能放流民进入东京城之中!”

“要小婿去河北?”韩冈面作难色,“以小婿资序,只够任通判的。上面若是坐着个知州,可是什么事都做不了。”

“不是通判……”王雱在旁摇头,“是知县!”

韩冈先是一呆,转而便笑了起来:“是白马?还是酸枣?”

与聪明人说话就是轻松。王安石和王雱对视一眼,脸上也带起了一点笑意,韩冈脑筋转得的确是快。

此事当然也不难猜。

若是要自己以通判资序担任普通的知县,除非是要翻脸,才会如此安排。别说他韩冈,就是将新党中任何一个朝官,放到河北的哪个地方任知县,谁还会管你什么流民?帮你办事,为朝廷分忧,还得憋屈着来!世上哪有这等道理?更别说权力越大,能做的事也就越多,明摆着就是贬斥。

不过属于通判资序的知县职位也是有的。就像后世的直辖市,下面的区县都算是地市级。大宋的四座京城,下面的县治由于属于赤县或畿县,能担任知县的便都是通判资序。

大宋的县,也分三六九等。赤、畿、望、紧、上、中、下,按照重要性和户口多寡依次排下来。其中赤县只有两县,将东京一分为二的祥符和开封。畿县就多了,东京开封、西京洛阳、南京应天、北京大名,属于四座京城的县治,除祥符、开封外,都是畿县。

王安石会安排给自己的,当然不会脱出这些地方。

而且说起要想安置河北流民,就必须是在渡口边。黄河上的大渡口,就那么十几处。关中的风陵渡不提了。西京洛阳府有白波【孟津】。东京开封府,则是延津与白马津,大名府有马陵渡。再往上或是往下,当然还有,只是就跟风陵渡一样不搭界,就不用一一例举。

在这其中,只有大名府的马陵渡,北面是卫州的延津,以及位于安利军对岸的白马津这三处,才会有河北流民。白波渡由于离得远了,又直面河东,不可能会有。此外大名府由于是文彦博坐镇,王安石也不会让自己去跟他顶牛。一旦闹起来,就会如汪辅之的例子,将他这个小臣调任他处。

用着最简单的消去法,韩冈得出结论自然不难。

“是白马县。”王安石跟韩冈摊牌。

延津属于酸枣县,而白马津就在白马县中。酸枣县一直都属于东京,白马县则原属滑州。不过在去年,郑州和滑州都撤州置县,归入了开封府管辖,属于滑州的白马县,当然也成了畿县。

“也是前日得了玉昆你的提醒,回去后考虑了一番后的结果。”王雱道,“以玉昆你的治才,守在白马渡边,才能让人放心下来。”

“愧不敢当。”韩冈温文尔雅的笑容很是谦逊,心里却是在冷笑。

‘根本是在扯淡……’最多三成是王雱说的原因,七成当是怕他在《三经新义》成书前再来捣乱。如今自己正得圣眷,天子时常见招。说不定哪天就封了一个崇政殿说书的经筵官,或是同修起居注什么的,可以天天进宫面圣。到时候,随便在天子耳边吹上几句风,说不定经义局中又要多生变故。

而王雱看着韩冈的笑容,心知以自己的这个妹婿的才智,当是已经猜到了真正的原因。

韩玉昆就是不肯跟自家父子一条心,总要想着他的格物之说。回京后的两次面圣,都没忘了跟天子提及。要不是这样,自己的父亲也不会点头将他给派出去。以韩冈的才智,以及他的治政水平,让人难以舍弃,新党中能比得上他的又有几人?不是因为他不肯顺服,何必这般浪费人才。

“其实,玉昆你缺乏的就是资历。只要在白马知县的位置上待上一年两年,做完这一任,回来后,就可去在京诸司中任正职了”王安石安抚着韩冈。

“岳父说得是,小婿明白。”

韩冈点头受教。对于这项任命,其实很符合他的心意。早一步经过第二任通判这一道关,将基础夯实,并不是坏事。拥有了知州资序后,不论是在朝中任诸监司的主官,还是外放任职,选择的余地都大了许多,而且头上也没有碍手碍脚的婆婆了。

就是因为是两全之举,所以才会给自己这个位置,若是对自己没有好处,王安石也不会拿出来破坏翁婿之间的感情。

韩冈起身,拱手致礼:“小婿必不负所托。”

