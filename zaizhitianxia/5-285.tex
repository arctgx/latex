\section{第30章 随阳雁飞各西东(16)}

就在叶孛麻决定出兵的同时,相隔不过二十里,仁多家已经在召集帐下的部众。

就在之前的两天,征召令一封接着一封的发了出去。若是西夏国还在的时候,要想等到所有人都得到征召,至少需要半个月的时间。可现在,半天就够了。

人人皆知,这是要开战了。两天的时间,所有接到征召令的部族和仁多家的长老们,都亲率兵马赶到了黄河边的仁多家居城。

但仁多零丁并没有接见他们,在他的帐中,正有一名装束与党项人迥异的辽人,与他分庭抗礼的对坐。

萧佛奴在青铜峡中已经有一个月了,之前更是奔走一年,而今天,已经到了收获的时节。

仁多零丁正摸着手腕上的一串佛珠串,每一颗珠子都是光润圆滑,是真正的东珠。他老脸上绽出了一个敦厚的笑容,“多谢统军相赠,如此贵重的宝物,零丁真是受不起。”

“这是尚父所赐。尚父听说了总管好佛,就特意从天子的赐物中选了这一件,是由敝国几名大德加持过,有消灾免难、厚积福德之能。”

仁多零丁将佛珠串抓得更紧,口气也更加谦卑:“尚父的恩德,零丁铭记在心。”

仁多零丁谢过之后,就没别的多余话了。

见他不再像前些日子那样催着要能打造霹雳炮的工匠,萧佛奴也算松了口气。虽然他已经写了信回去,但他其实也不指望能从后方得到工匠。这些都是宝贝,谁都不肯放手的。

不过萧佛奴估计仁多族中应该也有合用的工匠,要不然也不会松了口。之前的坚持不过是在讨价还价,在自己拿出了尚父的亲笔信和诏令敇书之后,就不需要再费口舌了。

可是讨价还价也需要本钱。

若是原本的大白高国,能征用的丁壮几近百万,十万精兵举手可及,那样的国家就是大辽、大宋都得正眼相看。

但现如今,旧日的西夏已经不存在了!剩下的一点余孽要么托庇于宋人,寄身于青铜峡中;要么就还留在兴灵,与大辽做牛做马。

没钱没粮,没有田地、没有牧场、没有产业,宋人将他们视为眼中钉,从不会信任一星半点,甚至在他们背后修城筑堡。前有狼、后有虎,中间的羊连根草都没有,这根本就没活路了。但党项人不甘心就死,必然会拼死一搏。正是看到了这一局面,所以萧佛奴才会主动申请到青铜峡中说降。

“只要打下了鸣沙城,宋人就无能为力了。到时候,你们是我大辽的臣民,你们占下的土地当也是大辽的,试问宋人敢不敢与大辽拼上一拼?”

“统军说得正是。”听着外面的动静,仁多零丁站起身,“事不宜迟,还请统军随零丁上点将台。”

帐前的点将台乃是新修,一丈多高的位置,让人可以清楚的观察到汇集到校场中的所有的士兵。

一面大纛插在台上,两名旗手在风中牢牢把定了旗杆。金白色的旗面在风中卷动,绣在旗面上的西夏文字,那是党项人旧日所用的旗帜,而不是宋国的赐物。

萧佛奴在台下靠后的地方站定了脚,并没有跟上去,仁多零丁的儿子仁多楚清也便陪着他一起站在这里

萧佛奴喜欢这个位置,比临时堆起的点将台要低上半丈,但比起点将台前的数万人众则高得多,同样可以居高临下的俯视着人群如蚁。

萧佛奴眯缝着眼睛,很是享受这个位置给他带来的愉悦。

台上的仁多零丁虽为万人瞩目,号令一出,族中精兵皆从其命,但真正的控制者却是在阴暗处的自己。背后操纵一切的快感便由然而生。

台前人山人海。

只看充满了视野的人群,萧佛奴便知道,仁多家的精壮,以及依附于仁多的其他几个小族的精壮,几乎都来了这里。

这些人皆是收到了仁多零丁的征召令,都是知道仁多零丁近日就要出兵。叛离宋国,攻打鸣沙城的传言早在一个月前就传遍了青铜峡中,即将面对已经严阵以待的宋人,即将面对刚刚整修完毕的鸣沙城,最后甚至不知有几人能活下来,但他们还是义无返顾的都来了。

萧佛奴抬头看看东西南北,两山夹持的谷道比起青铜峡峡口处当然要宽阔得多,但对于大小十七族、多达四五万帐的党项人来说,还是太过狭小了。仅仅是一年多而已,这群劫后余生的党项余部已经在这里牢笼里消磨掉了所有的耐心。

仁多零丁一人站在最前面,护卫们离得他很远。河谷中的风很大,吹动着金白色的大纛猎猎作响,也让他的声音只能使最接近台前的几百人听见。但下面有足够多的人帮他传话。

仁多零丁的侄儿仁多洗忠在人群中正跟另一族的好友察哥并肩站着,都在等待着仁多零丁。

“真的要打鸣沙城了?”察哥低声问着。

“不能不出兵了。”仁多洗忠回应道,“再拖下去,明天春天要死一半的人。”

提起这番话,察哥也不由神色一黯,春天的确是过不下去了。

“都抬起头!看看四周!”仁多零丁的第一句话就让所有人都为之一愣。正在说话的仁多洗忠和察哥停了口。甚至真的有一小部分人依言抬头望着天空。

“睁大眼睛好好看看!”仁多零丁大声喊着,“到底看到了什么?”

“是山,全都是山!”他自问自答的揭开答案,“在抬头就是山的山沟里,我们已经住了一年多了!”

人群中开始有了的反应。

“这是怎样的一年啊。”仁多零丁叹息着,“一年的时间,仁多家就只有三百小儿出生,若是在过去,三四倍总有可能。”

“何值三四倍!”人群中的反应渐渐激动了起来,“再多也能有啊!”

萧佛奴轻轻点头,虽然是个老懵懂,好歹还有点水平,知道怎么煽动人。这一下,肯定有大半人愿意跟他去冲一下宋人的坚城了。说不定,还真能给他攻下来。

“还记得贺兰山吗?再过几个月,山头雪水就要淌下来。”仁多零丁开始描绘旧日的美好时光。

“还记得贺兰池吗?九十九眼泉水有多么甘甜。”

“还记得五台山寺吗?多少人去拜祭过里面的卧佛。”

“还记得七级渠吗?灌溉了多少良田。”

“还记得诓保大陷谷吗?谷中放养的山羊烤起来可是天下间最好的烤肉。”

“还记得大小白羊谷吗?每年的这个时候,北面就要从这里运马过来了。”

仁多零丁一句句的大喊着,原本还冷静着的族长和长老们也开始无法在遏制自己的激动,甚至有许多人都哭了出来。那些可都是他们过去最熟悉的地方。

萧佛奴越听越是不对味道,刚想说些什么,却突然被人架住了。

刚想回头,一团麻布便塞进了嘴里,身子也给牢牢抓住连动都不能动。

萧佛奴亡魂直冒,这是要反水吗?

耳边传来噗噗几声轻响,眼角余光望过去,他的两名伴当被人从背后捅上了肾门,喉咙被粗壮的胳膊环扣住,喉间咯咯作响,却发不出声来。过了片刻,放开手时,便软塌塌的倒在了地上。

高台上的仁多零丁完全没去在乎身后的动静,他嘶声力竭:“但这些,现在都不是我们的了!”

手腕上的佛珠串在激动中被他一把扯断,珍贵的东珠叮叮当当的落了满地,他回头,“将那贼人给我押上来!”

仁多楚清得令,立刻就押着挣挫不休的萧佛奴上前。

仁多零丁指着方才还是座上宾的萧佛奴:“这一年多来,辽贼百般欺压,时常纵马过界,杀伤我族中子弟数以百计。而这贼子现在竟然还要唆使我等为他们卖命去攻打宋人?!岂不知我们最恨的可就是你们契丹人啊!当真会以为怕了你们辽贼不成?”

回过头来,吃斋念佛的慈眉善目早就变得杀气腾腾。他抽出腰间长匕,劈胸就搠进了萧佛奴的心口。

萧佛奴拼命的挣扎,但仍只能眼睁睁的看着刀子没进自家的心口处。胸前一凉,来自兴灵的特使眼中神采便渐渐涣散消失,下身处一阵臭气冒了出来。

仁多楚清将手一松,萧佛奴的尸身砰的一声落地。用力踢了一脚,仁多楚清狞笑着抄起斧头:“腌臜的蠢货,真当你外公给你赔了几天笑脸是讨好你吗?今天便宜你,给你个痛快!”

仁多保忠带着一溜血光,顺势抽出了长匕。掌心抹着刀身上残留的血渍,便转手抹在旗杆和鼓皮上。鲜红的血印,充满了震撼力,台下寂静无声,数万双眼睛望着台上的仁多零丁。

仁多家的老族长反手将腰刀一下插在地上,沾满了鲜血的左手将儿子递上来的首级高高举起:“辽贼夺我故土,使我不得痛饮贺兰山池的雪水。宋人故是仇敌,但辽贼背盟偷袭则尤为可恨!今日辽宋相争,辽贼尽在韦州城下,兴灵正是空虚。就以此贼首级为证,敢问我党项男儿,可敢随我杀回贺兰山下!”

……可敢随我杀回贺兰山下……

