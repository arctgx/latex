\section{第22章 汉唐旧疆终克复(下)}

自从丰州和鄜延的战事,以大宋官军的胜利而宣告结束之后。京城之中,越来越多的人在议论着官军会何时起兵攻打兴灵,剿灭西夏。更有一众人等,已经开始在想着收复燕云来了。

尽管朝廷没有公布,也没有承认,在丰州的党项军中,隐藏着一队辽国最为精锐的皮室铁骑。而这一队皮室军,却为官军轻易剿灭,官军甚至连损伤都没有多少。

越来越多的证据表明,皇宋官军已经拥有了压倒西北二虏的强势。如今在民间,大半百姓一说起此事,都有着一股扬眉吐气的舒畅。

不过如今议论得更多的,还是正在鏖战之中的交趾战事。

从门州大捷的捷报飞传京城之后,远在数千里之外的战事,便成了京城人现下最热门的话题。而前两天,官军已经攻打到了富良江边,与退守升龙府的交贼隔江而望的捷报,更是让京城中一下沸腾起来。

只差一步就能灭国了,从太宗皇帝平灭北汉之后,百多年来,大宋都是以内守为主,哪里还有举兵灭国的记录。

章惇、韩冈、燕达、李信这些将帅的过往功绩,都给人拿出来在嘴里嚼着。人人都在说着,有着如此战绩煊赫的名帅良将,拿下升龙府当不在话下。

每一间酒楼茶肆,都能看到一帮子闲人,在高谈阔论着官军该如何打过富良江,一个个仿佛都成了运筹帷幄的谋士,各种靠谱和不靠谱的议论在酒桌茶桌上飞来飞去。

“想不到还有说要让飞船送人过江……当真是可笑了。”蔡京摇着手上的酒盏,“富良江岂是这般容易过的?”

“元长这话说的,官军都已经打到了富良江边,以章子厚和韩玉昆的心思,他们肯定是要杀去升龙府,将交趾王拿来京城。”与其对饮的上官均摇摇头,“再不好过,还能比得上黄河长江?交趾现在可不是多余的季节。”

“不是水势如何?而是有交贼拦着。”

“官军对上交贼,都是以一当百,前番可是有先例的。”

蔡京放下酒杯,正色道:“愚民无知,怎么连彦衡兄也糊涂起来了。韩冈之前能在邕州大胜,那是因为交贼当时已是师老兵疲,再加上官军出其不意、掩其不备,方才一举功成。可如今南讨交趾,整整耽搁了一年,交贼早有所备。”

“若当真有所准备,怎么官军这么快就已经打到了富良江边了?”上官均反驳道。

“彦衡兄,可知何为坚壁清野?”蔡京说道,“自从官军攻入交趾境内,算得上大战的,只有一个门州……”

“元长!”上官均的脸上满是难以认同,“难道忘了前天传回来的捷报?”

蔡京微微笑了起来:“门州一战,格毙的交贼主帅是乾德亲叔,人称洪真太子。而前两天的捷报,说是大败数万交贼,斩杀的主帅又是何人?只不过一个州官罢了!交趾有多大?一个州中就能点起数万兵马?不过是吹嘘而已。实际上,门州之后,官军再没有与交贼主力交手过。”

上官均声音便是一滞。

蔡京继续说下去:“以现在的情势看来,交贼的主力全在富良江对岸。如果小弟是李常杰,便会将江上的船只全都毁掉或是收到南岸去。包括蛮部在内,总计十万兵马要吃要喝,交贼坚壁清野后,粮食哪里来?而且还要设法过江,又要耽搁多少时间。”

“过江哪里会那么难!”

蔡京叹道:“交趾能渡海攻打钦、廉,他们的水师不会是摆设。”

“论工匠手艺,交趾如何能与中国相比。官军打造的战船,绝不是交趾人能抵挡得了!”

蔡京哈哈大笑起来,“造船哪里有可能这么容易!新伐下来的木料,要用三五年来阴干,没有木料,没有铁钉、没有桐油、没有丝麻絮料,哪里能造得出战船来?”

见上官均还是不服气,他抛出了最为有力的一个证据,“想必章、韩二位招讨在交趾如何处置当地人丁的手段,兄应该听说了吧?”

上官均板着脸:“交贼掳掠汉人为奴,让中国之人为其做牛做马,也该有此报。”

如今官方的宣传口径,就是依照安南经略招讨司的奏议,将交趾人在钦州、廉州、邕州的罪行,以及被掳去交趾的百姓所受到苦难加以宣扬,以维持复仇的正义性,明明白白的说是要以直报怨。只是砍掉脚趾,已经是很宽宏大量了,而且动手的还是蛮部,官军只是作壁上观而已,怎么说都没有错。

“将叛贼魁首论以国法,但古往今来哪有问罪百姓的道理?都是胁从不问。外面都说章子厚、韩玉昆这么做着实痛快,可仔细想想,若是能将交趾百姓安抚,让他们成为皇宋子民,哪里会放弃?就是做不到,才会选择放手,让蛮部来清洗。不过这驱虎吞狼之计,一个不好就是养虎为患,故而又有了刖刑一策。”

蔡京叹了口气,“为了日后南方安定,章、韩两位,算得上是殚思竭虑了。可这些手段里面,能看得出他们有把握攻下升龙府吗”

上官均一时无从辩起。

“虽不敢说升龙府肯定打不下来,但多半很难,一个不好,还能让交贼扭转战局。现在也只能指望章子厚、韩玉昆能见好就收,不要让官军尽数折在富良江畔。”

蔡京重新拿起了酒盏。

如果升龙府当真能打下来,这二人的气运和手段,那就太过于惊人了。不过这世上,哪有这么多好运的事。一国之都要是一万人马就攻下来,官军都能打到辽国上都临潢府去了。

他正这样想着,就听见楼下一片蹄声响过,几名骑手接二连三的从道路奔向宫城,人人高高举起宣扬捷报的露布,从酒楼下飞驰而过,沿着他们经过的大街突然间暄腾起来,多少人在奔走呼喊,“官军攻下了升龙府!”

“官军攻下了升龙府!”

……………………

赵顼正在患得患失之中,章惇、韩冈能以万人之军势如破竹的攻入交趾境内,他当然是欣喜欲狂。但危机也在其中,这一战实在是顺利的过了头,顺利得让赵顼都在怀疑交趾人有什么阴谋诡计在酝酿。

王安石这几日脸上的忧色也是越来越重,一直都在为交趾的战局担忧。毕竟真正堪用的兵力只有西军和荆南军总计不到七千人。而且攻入交趾境内,只有寥寥数战,李常杰、宗亶这样知名的将帅都没有出动,怎么看是在故意吸引官军深入。

天子、宰相都是同样的想法,而两府之中的其余宰执也都不看好章惇和韩冈的冒进。交趾人坚壁清野的行动做得太明显了,门州之后再无大战。身为宰辅,他们得到的消息远比外界要详尽,如何看不出来交趾人的计划?

只是没人能下定论,说官军一定打不下升龙府。就连吴充也是一样,他在韩冈身上马前失蹄的次数太多了,只能揪着章惇、韩冈的行事来批判和弹劾。

毕竟那不是一个两个的问题,而是几十万人一齐受刑,可以说是史无前例,比起屠城还要骇人听闻。虽然招讨司设法将此事交由让蛮部来做,而其中隐藏的用心也得到了天子的认可,但吴充实在是难以忍受,不过该说的早就说了,天子不理会也没有办法。

今天要议论的也只是到底要不要下诏让章惇、韩冈两人相机行事,不要硬攻。不过从时间上算,诏令传达过去后,若能过江肯定已经过江了,若过不了江,那么以章、韩二人的才智,多半也会选择及时撤军,等到主力援军到达之后,再次出兵交趾——就在二十天前,预定中的第二批四千援军已经出发南下,而第三批则是因为河北局势依然严峻,而要再等上一阵。

只是这件事并没有讨论得起来,刚刚上京来的元老——知应天府、兼宣徽使的张方平,上殿之后,只说了几句公事,便立刻抨击起了安南经略招讨司的两名主帅来:“章惇韩冈在交趾倒行逆施,不施仁义,仇怨将百年难解!日后交趾不顺,举兵犯境,二人岂能无罪?!”

赵顼心中不愉,脸色一沉,“难道交贼在钦、廉、邕三州大肆屠戮,这样的仇怨只要三年五载就能化解了?张卿岂不闻虽远必诛四个字?交趾兴兵十万犯境,家家户户皆有出兵,论以国法,谋叛者株连三族,即以交趾论,其国中何人无辜?”

李乾德是得到大宋册封的郡王,率土之滨的说法,更是不能否认。交趾是大宋的属国,李乾德是大宋的臣子,交趾百姓也要受到大宋的管辖,如果附逆反叛,以宋律论罪,当然不能说有错。

“如果当真是虽远必诛倒是好了。”张方平摇头叹道,“数十万刑余之人,必会对皇宋恨之入骨,所以臣才会说着怨恨会延续百年。”

只要杀光了,便不会有事,但手尾不净,仇恨便会代代流传。没人能想到张方平的意思竟然是这样,连赵顼都愣了。

张方平板着脸,神色更加严肃,“招讨司行事如此残虐,将交趾男丁尽数施以刖刑,这岂能吓阻交人反抗,只会更增添他们的坚守之心,想要强行打过富良江去,官军损伤必众,也难见功成。且用兵万里之外,民夫转运困苦,为中国计、为百姓计,还是尽早下令撤军为是!”

吕惠卿心中冷笑起来,说过了半天,果然还是这个目的。

“陛下!陛下!安南招讨捷报,官军已破升龙府!乾德出降,李常杰畏罪自尽……”拿着捷报便兴冲冲的冲进殿上来的石得一忽的愣了,为什么前面转回头来的张宣徽脸皮红得发紫,眼神就想要吃了自己一般?

