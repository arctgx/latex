\section{第九章 拄剑握槊意未销(五)}

【算是第三更,补前天的缺更】

韩冈恍恍惚惚的从睡梦中被吵醒,却仍闭着眼睛不想动。

就那点工钱,哪有半夜扰人清静的道理。将怀中的娇柔温软的身子搂紧,韩冈换了个姿势继续睡。

但门外的声音更急了,“龙图,龙图,童供奉已经到了正厅,有天子的口谕!”

“三哥哥,”今夜陪着韩冈的云娘已经惊醒过来,挣开搂着自己的手臂,撑起身子:“会是什么事!?”

韩冈打了个哈欠,稍稍清醒了一点,冷哼了一声:“半夜鸡叫从来都不会有好事!”

云娘这下心中更乱了,“三哥哥,怎么办?”

听着云娘的声音中颤抖的都带上了哭腔,韩冈笑着拍了拍怀中娇躯:“别自己吓自己,真要有什么事牵累到我,就不是童贯来了。”他坐起身,“来,帮我换衣服。”

云娘听了,忙披了件外袍就下床帮韩冈换上公服,偷眼看着韩冈的神色,一幅担惊受怕的样子。

韩冈让云娘服侍着穿了衣服,从房中走了出来,这时候家中的灯全都亮了,王旖、周南和素心也都给闹醒了,自房中出来。

一见韩冈,王旖急忙上前,抓着韩冈衣襟:“官人,夜里中使怎么来了?”

周南和素心也跟着一起上前,神色惶然。

仆婢们也是人心惶惶,不知出了何等大事,让天子夜里派人来府中。

韩冈暗叹一声,难怪说伴君如伴虎。自家的性命操纵于一人之后,夜里来传句话,就让人寝食不安。

“小事而已,不要乱,一切如常。”韩冈放声说道。

他倒是老神在在,到底什么缘故,韩冈心中也有底,多半是败了。就是不知道他们是怎么败的。究竟粮草不济,还是别的原因,希望不要是契丹人插足进来。不过自己的猜测对与不对,答案很快就会揭晓。

稍稍安抚下家人,韩冈跨步走进正厅之中。

“龙图!龙图!”童贯正在厅中急得团团转,一见到韩冈后就急匆匆的说道,“天子有召,命龙图速速入宫议事。”

韩冈看了童贯一眼,一句话都没多问,回过身,就在童贯的注视下转身向后去,“供奉回去后禀明天子,说臣韩冈睡了,有事明天再说。”

究竟发生了什么事,看这位在崇政殿里听候使唤的宦官的表情就知道了。

果然还是败了。尽管一如所料,但韩冈的心中却没有半分得意。

也许看到王珪和他门下走狗们的苦脸心情能畅快一些,但这一次败阵,不知多少将士战死或重伤,怎么也幸灾乐祸不起来。却只想回去睡上一觉,将烦心事都忘掉。

看到韩冈当真就要回去睡觉,童贯的眼神由焦急转为惊愕,大惊失色的在韩冈身后尖叫道:“龙图!是天子有召!”

韩冈回过身,宁宁定定的问道:“可是天子不豫?”

童贯摇头,虽然天子没吩咐他说明原委,但提前泄露给韩冈是没问题的:“不是,是……”

“难道是太皇太后有恙?”韩冈又问,打断了童贯的回答。

“不是,是……”

“是辽人打到大名府了?!”

“不是,是……”

“那还有什么大事值得天子半夜招臣子入宫?!”韩冈一声断喝,第三次打断了童贯的回话,“你且回去报与天子,既然无甚大事,等明日朝会后,在崇政殿中商量也不迟。”

“龙图,是高苗二帅在灵州城下战败了!”童贯的声音冷静了下来,他已经听明白了,但他还是提醒韩冈,“相公、执政那里都派人去传召了!”

“王禹玉是当朝宰相,吕晦叔、吕吉甫、元厚之,皆是国之重鼎,岂会糊涂到连夜入宫!?嫌京中太安稳了不成?”

韩冈说着,示意管家给童贯递了个比平常丰厚得多的红包,送了他出去。自己则转身往后院去,对紧张惶恐的家人道,“没事了,回去睡觉。”

心情不好,这时候他什么都不想理会。

“官人,当真没事?”周南扯着韩冈衣袖,不让他走。她们在后面也听到了前面的对话,韩冈直接将天子派来的中使赶回去了,这比方才听到中使半夜上门,还让人担心。

韩冈握了握周南的小手:“放心,吕公著和吕惠卿绝不会入宫的,元绛惯看风色,说不准也不会去。有他们在前面顶着,我有什么好怕的。”

周南、素心和云娘回头看王旖。她是宰相家的女儿,当知韩冈所言真伪。

王旖对此等事当然是耳闻目睹了不少,点了点头,“当年爹爹做过很多次,不会有事的。”她怅然一叹,“想不到当真败了。爹爹和兄长夙夜忧劳,官人费尽心血,竟然会是这个结果。”

“有人不心疼辛苦挣来的家当,偏要往赌场跑,这又有什么办法?”韩冈理了一下公服的襟口,“回去睡觉,管他什么事,都给我明天再说!”

……………………

赵顼不知自己在这里做了多久,似乎才眨了眨眼,又仿佛已经是一年半载。

他脑中一团乱麻,什么都没想,也不知道该想些什么。

烛光闪烁着,一明一暗,让赵顼只觉得眼睛发花。殿中班直和内侍们的眼神怎么看怎么不对劲,是不是在嘲笑自己的失败?

“将灯都灭了。”他烦躁的呵斥着。

没人敢在天子气头上违逆或拖延,忙将殿中的三十六根手臂粗细的龙凤香烛一支支的吹灭。

黑暗降临,赵顼这才觉得了安全了些。不用看到他人眼中的嘲讽,不用再装出一副平静庄严如同木像土偶的表情。

什么都不用想,或许那就是一场无稽的噩梦,只要灯火再亮起,一切就会恢复正常。

“官家……”

“官家。”

“官家!”

石得一的声音一次比一次更响亮,将崇政殿后殿中虚假的寂静击碎。

“……什么事?!”赵顼随口应道。

“官家,王相公到了!”石得一连忙说道。

黑暗中,赵顼驱动停转的头脑,仿佛拔出匣中生锈的铁剑,吃力、迟缓,但最终还是想起了王珪为何入宫。

原来不是梦啊……

赵顼用力压着心口,将突如其来的一阵心悸给压下去。

从后殿来到灯火通明的前殿,王珪已经到了。叩拜一番,赵顼便给王珪赐了座,君臣二人同坐下来,相顾无言。

赵顼不想说话,王珪也不知该说什么好,都没想到高遵裕和苗授都打到灵州城下,竟然还会失败、还能失败。

王珪是第一个到的,但第二人始终未至。

不过派去召吕公著的内侍无功而返。

“官家,奴婢奉旨传诏枢密使吕公著。吕枢密回复道,深夜入宫,恐惊动京城百姓,不敢奉旨。”

“哦,是吗?”赵顼低低的应了一声,这是预料之中的回答。

又等了片刻,派去召吕惠卿的黄门回来了,紧接着是元绛的。

“官家,吕参政说宰执非宿卫,无夜入宫城之理。”

“官家,元参政说宰执连夜入宫,恐致谣言,有事明日再议不迟。”

除了王贵以外,执政们一个一个都给了否定的答案。赵顼忍不住了,起身绕着御桌打起转来。

吕惠卿没到,吕公著没到,两人都拒绝了在夜中入宫,元绛也没有到,他是老狐狸了,知道夜中入宫只会生乱。

郭逵在定州、薛向在洛阳。两府宰执六人,眼下就只有王珪一人站在崇政殿中,与绕着御案直转圈的赵顼大眼瞪小眼。

王珪这下算是知道什么叫树倒猢狲散,吕公著、吕惠卿不来是情理中事,但元绛不来却意味着他放弃了与自己的联手合作,见风使舵的能人啊!

“官家,童贯回来了。”

赵顼停住脚,抬起头,真正精通兵事的专家到了。

“宣。”

童贯低头小碎步的进了殿中,眼角余光一扫左右,就只看见王珪一人在殿中。

宰执们的府邸就靠着宫城不远,比起同群牧使的宅子要近得多,看起来韩冈说得没大错,其他执政都拒绝夜入禁宫,就王相公一个人到了。

国之重鼎,这个词谁当得起,谁当不起,可就是一目了然了。

赵顼看到童贯也是孤身一人回返,终于出离愤怒了:“韩冈也不来?!”

“官家,奴婢奉旨传谕龙图阁学士韩冈,韩龙图说,无甚大事,并非急务,等明日朝会后,在崇政殿中商量也不迟。”

“‘无甚大事,并非急务。’你就没跟他说灵州兵败了!?”赵顼心头腾起一股邪火,从头到尾就反对激进的韩冈,这时应该很得意吧。

童贯低声道:“韩冈只是问奴婢,是否是陛下不豫,是否是太皇太后有恙,是否是辽人打到了黄河边。如果都不是,那就是‘无甚大事’!不值得连夜入大内。”

“好!好!好!”赵顼脸上的笑容比哭还难看,“全都不愿夜入宫城,不愧都是一心为国为民的纯臣!不愧都是纯臣啊!!”

“陛下!”王珪这时猛然抬头,“高、庙二人告退,只是小挫,并非全局失败!还有秦凤、熙河的兵马,也还有鄜延、河东的精锐,还有反败为胜的机会!”他嘶声力竭。

