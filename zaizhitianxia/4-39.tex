\section{第八章 欲谋旧地重兴兵(中)}

“小乙你的武勇,天下也是有名的,用你作先锋,路中无人能说半句。”

王舜臣心头如烧得一团炭火,种朴的几句称赞如同扇过来的清风,让火势烧得更旺,“俺今天就去绥德,整顿兵马、教训士卒。只要五郎一声令下,俺就往西贼占据的罗兀城杀过去。”

“不急。还得先去见了毋经略,领了将令再说。”

虽然已经定下来这一次的横山攻略是由种谔来领军,但现在新上任的毋沆才是王舜臣名正言顺的顶头上司。而且按照如今的循例,一路之中的几位统军大将——钤辖、都监、都巡检,都是各自独立,甚至可以顶撞兵马副总管的将令。只要他们老老实实的听从作为文官的路中主帅的吩咐,没人能给他们打上违抗军令的罪名。

“俺明白,俺明白。”王舜臣摸着头,自嘲的笑着,的确是心急了。

“这一次对横山的攻势一定要稳,必须将军械钱粮都筹划好,兵将也要整顿,差不多还要有三四个月的时间,到了秋冬的时候,正好可以面对面的较量一番。小乙你也需要时间去将第七将的兵马给接收下来……”种朴更进一步的想王舜臣说明,“这段时间,延州的北方同样是要靠你来镇守。别我们还没有出战,就当头输了一阵。一旦吃了这么一个大亏,想要再挽回,可就不是那么容易的事了。”

种诂在环庆,种谊在泾原,都能给在鄜延路作为主攻方向的种谔以帮助。虽然没有设立宣抚司,配合上看似有问题,但种家几个兄弟如今都在临敌的第一线上,种谔出战,几个兄弟哪有可能不帮手?种家可是将宝压在了横山上,好不容易重又到手的机会,一点差错也不能出。

“俺知道了。”王舜臣收起笑容,变得严肃起来,“五郎、十七哥,你们放一百个心,俺肯定会将几件事都做好。”

种谔满意的点点头,种朴则是笑道,“有小乙你这句话,哪里还有不放心的?”

王舜臣也呵呵笑了两声,又谦虚了几句。

“对了,俺听人说,今次攻取横山,韩三哥会来鄜延,管着全军的粮秣和医药。是不是有这么一回事?”王舜臣问着他想问了很久的问题。

种谔沉吟了一下,道:“韩玉昆知兵,不是站在沙盘前指手画脚的那种,是当真会带兵治军。他入官后我就一直看好他,只是没想到他升得能有那么快!再过几年,就能过来做经略使兼兵马总管了。”

听到韩冈受到称赞,王舜臣也觉得与有荣焉。当年在押送粮草的过程中结下的过命交情,如今更是密不可分:“当年十七哥写信来的时候,就说过了。所以说五郎慧眼识人,就跟老太尉一样好眼力。”

种朴在旁道:“王大你看看这书架,父亲翻看韩玉昆的书,可不比看兵书、史书的时候要少。”

王舜臣顺着种朴的手指看过去。在种谔书房的墙上,挂着大大小小的长短兵器,刀枪剑戟都不缺,一看就知道是武将的书房。不过让书房名副其实的书架也是有的。

但书架上的书册也是以兵书居多。孙、吴二子的兵书自不必说,三韬六略、唐李问对、尉缭子、司马法,乃至阴符、握奇,甚至还有武经总要中的几卷,只是大多数都落着灰,仅有少数的十几卷被翻得页边发毛,其中就有韩冈的疗养院制度和浮力追源。

不是种谔不喜读书——在靠着另一堵墙壁的书架上摆着的一卷卷史书,都是干干净净,能看得出时常被人翻阅——而是种谔懒得多看那些嚼着舌头、说些弯弯绕绕酸话的兵书。

他一向认为兵书要直接浅显,不能以辞害意,宁失于繁,勿失于简,学着文人讲究着文法,那就不是兵书了,给秀才们拿去玩着运筹帷幄的游戏好了。真正阵上厮杀,绝不是孙子兵法中简简单单的十三篇,就是武经总要中,说得也是少了。

所以种谔欣赏韩冈。韩冈所写的那部关于军中伤病治疗养护的章程,如果放在给文人看的兵书中,多半就是善抚士卒四个字一笔带过,多的也就用三五段话,说说食水医药等事。由谁能像韩冈一般,将军中医疗之事,掰碎了、揉开来,不厌其繁的将小到洗手、吐痰的事都细细写来?

“不过军中讲究的就是说一不二,韩玉昆当真来了,可能屈居人下?”种谔摇着头,“所以这番流言当不得真。”

种朴也道:“韩玉昆肯定不会来的。不设宣抚司,鄜延路哪里能安排得下他?”

王舜臣皱着眉:“永兴军路转运司不是正好可以派得上用场吗?做转运副使,韩三哥也足够资格了。”

王舜臣其实说的没错。在没有陕西宣抚司的情况下,想要让韩冈来管着大军的粮秣转运和伤病医疗,也只有在永兴军转运司中做文章,一个转运副使少不了他的。

“可若是韩冈做了永兴军路的转运副使,当他来主管军中粮秣后,到时候谁能压得了他?”种朴不介意在王舜臣面前说出这些掏心窝子的话。以他对王舜臣的了解,知道这位自幼跟在自己身后的旧日伴当,绝不会是私下里揭人短长的长舌阴险之辈。

王舜臣欲言又止,他清楚种谔的性格,也清楚韩冈的为人,都是对自己充满自信,能够独掌一面就绝不会给人做副手的脾性。若当真聚在一起,说不定还真的争个高下出来。

见王舜臣无话可说,种谔也就不需要再多解释。

他当然希望麾下能军心稳定,敢战堪战。前几年经过横山、咸阳、河湟多少事,在西军中名声响亮的韩冈,就是最好的随军转运的人选。再加上这一年来,韩冈在军器监的诸多发明,至少在西军之中,没人能反对这个提案。但若是韩冈有可能会动摇到他的权威,种谔就绝不会欢迎。

横山一役,种谔不可能,也不愿意让人在自己身边指手画脚——军中岂能有二帅!这是原则性的问题!

站起身,种谔出门转向偏院,只丢下一句:“跟我来。”

王舜臣和种朴老老实实的跟着起身。“这是去哪里?”王舜臣侧脸问着种朴。

种朴低声回答:“白虎节堂。”

……………………

就在种谔在白虎节堂的沙盘跟前,向王舜臣解说自己的收复横山的方略时。兴庆府中,也在讨论着迫在眉睫的战争。

梁氏兄妹,梁乙埋的儿子梁乙逋,宗室大将嵬名阿吴,外姓豪族们的头领仁多零丁,还有十几个文武重臣齐聚紫宸殿。事关国运,殿上的气氛则显得更为紧张。

“又是种谔。”

一提到这个名字,不仅仅是说话的梁乙逋,就连殿上的其他臣僚都感到牙疼。这些年来,每次宋人在横山挑起事端,都是由种谔起头。前些日子一听说他回鄜延路来了,每个人都知道横山又要开战了。

“祥佑军司发来急报,宋军随时可能北侵,请求立刻加派援军。”

“肯定要派,但到底要派多少?”

“至少一万!”

“横山蕃部几年前就毁了一半,派过去一万,他们的口粮从哪里拉过来?”

“难道就不能我们这边先动手,只能等着宋军来攻吗?再过两个月可就是秋天了,正好起兵。”

“那宋人就有理由将契丹的责难顶回去了。”

“管他怎么想。只要我们赢了,辽人不会逼我们大夏。若是没能如愿,待到宋军北攻横山,契丹还能坐视不救?”

“什么都要靠契丹。当年我跟着景宗皇帝,可是契丹、宋人都打过,何曾怕过他们!?”

“时过境迁,宋人不一样了。”

“是你胆子太小……”

“吵什么?!”外臣中,威望最隆的仁多零丁,睁开有点迷迷瞪瞪的昏花老眼,双目一扫之中却有如电光掠过,“还至少有两三个月的时间,宋人才能一切准备就绪。用不着太着急,稳着一点。”

仁多零丁发威之后,人多嘴杂的紫宸殿上又重新恢复了理性。一直保持沉默的梁乙埋和高居在殿上的梁太后使了个眼色,对仁多零丁的威望有了几分忌惮。

“宋人大张旗鼓,会不会声东击西。兰州禹臧家,这两年生意做得越发得大了,禹臧花麻都恨不得认王韶、高遵裕做亲爹。”

“派人去兰州盯着,再在朝中给禹臧花麻找个位置……让他入京做枢密副使,不信他会不愿意。”

“那只狐狸怎么可能会来兴庆府?只要诏令一下,他少不了就会称病说自己快死了。上表请老他说不定都能干得出来。”

“总不能坐视他投向宋人吧?”

“禹臧花麻不会那么容易下决定,而且以种谔的性格,他会同意声东击西的策略,为人做嫁衣吗?”

“话是这么说,但总不能不防着吧?”

“那就再多派细作过去打探。消息探明再动手也不迟。眼下关键还是在横山。”

