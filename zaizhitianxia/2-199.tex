\section{第32章 吴钩终用笑冯唐(六)}

围城日久,军中生病的有不少。而且还有许多因为前次的大火而烧伤的士卒,在千年之后都是难以医治的重度烧伤,在此时更是无药可医,这些天来都已经陆续病死。现在还能躺在病床上呼吸的,却都是一些轻伤员。

不过在一般的伤病营里面,轻伤员能否痊愈还是要看运气。幸好疗养院中的医工,都是燕达从秦州带来的好手,做得也很不错,病房中整洁清爽,病人也都得到了妥善的治疗。

这些人基本上韩冈都认识,他们见到韩冈进来,便是又惊又喜的上来磕头。领头的还是韩冈的熟人,老军医仇一闻的弟子李德明。

李德明给韩冈行过礼,起身后道:“早听说机宜到了宣抚司中,一直都盼着机宜来,现在终于给盼到了。”

“仇老近来可好?”

“家师身体好得很,最近还是常常出去到各处军寨去。”

“仇老年岁也不小了,该歇下来享享清福了。”韩冈摇摇头,“你这个做徒弟的也该劝一劝。”

李德明笑道:“家师都说自己是劳碌命,闲下来反而会生病。”

韩冈笑着摇摇头,的确是有这种人。又问了问蕃军的事,基本上就是他逼着王文谅出战,现在王文谅战死,隶属于他的蕃军也受了重创。有不少伤员,韩冈自己不便去,便让李德明派了几名得力手下去了蕃军的伤病营照看。

韩冈在病房里转了一圈,他的名声响亮,听说韩玉昆来了,伤兵们的精神顿时好了许多。李德明都笑说,要是韩冈天天来走一趟,不用施针用药,伤病自己都能痊愈了。

韩冈笑骂两句,刚刚坐下来想歇歇脚。一阵斧刨绳锯的声音就一个劲的往他的耳朵里钻。“怎么这么吵?”

“后面就是工匠营,现在天天在打造攻城的战具,白天一直在吵着,只有晚上才能歇下来。”

“疗养院从来都是选得清净的地方,怎么安排着跟工匠营做邻居,这让人怎么养病?”韩冈听着锯木的吱吱呀呀的声音,就觉得碜得慌,就像旧时听到指甲划过黑板的声音,浑身发毛,怎么都坐不安稳。对李德明道:“这里你先照看着,我过去看看。”

韩冈起身就往后面的工匠营去,看看是不是能让他们安静一点,不成想却见到了游师雄。

“景叔兄,怎么在这里?”

“这里是愚兄在管……倒是玉昆你为何过来?”

“疗养院就在前面,听到声音就过来看一看。”

韩冈没明说,游师雄却是会意一笑,歉然道:“惊扰到玉昆了。”

韩冈哪能跟游师雄计较,说了声没事,便在游师雄的陪同下参观起工匠营来。

经过近一个月的赶工,攻城用的战具已经打造得七七八八。登高望远的巢车、攀城用的云梯车、过濠河的壕桥、还有用来挖掘地道的头车,一辆辆的停放在工坊中,被游师雄不厌其烦地向韩冈一一介绍,最后两人的脚步停在了一辆投石用的行砲车前。

韩冈也算是久历战阵,最近还在重兵围困中的罗兀城待了不短的时间,守城的武器见过不少。不过由于从没有参与过攻城战,自然攻城的战具就只见识过寥寥几种。

这个时代的投石车韩冈还是第一次见,他现在所看到的这具被称为行砲车的攻城战具,并不是他前世记忆中的那种投石车,除了抛竿不变以外,样式简直是天差地远。尤其是抛竿前部,一条条垂下来了几十根绳索,而不是绑着石块或者重物。

“怎么拖着这么多绳子下来?”韩冈好奇的问着游师雄。

游师雄抬手扯着绳子,向下用力一拉,穿在横梁上的抛竿另一头便被拉得挑了起来。他对着韩冈笑道:“这些绳索要三十人同时拉扯,才能把石头抛出去,力气小一点都不行。”

韩冈听着纳闷:“怎么是用人力?!”

“不用人还能用什么,总不能用牛和马吧!?”游师雄哈哈笑了两声,“牛、马可不会那么听话。”

“小弟不是这个意思。”韩冈摇着头,“用人向下拉扯绳索,是为了让抛竿翘起以便把石块抛出。也就是说,只要有个向下的力量,好把抛竿的后端翘起,是不是用人来拉,本质都是一样。”

游师雄听出了一点意思:“玉昆你有什么想法?……”

“小弟是在想,如果不用人力来拉,而是绑一块巨石或是其他重物呢?!”韩冈拿着树枝,在有着一层浮土的地面上几笔画出了他记忆中的投石车的外形。

游师雄一边听着韩冈解说,皱着眉对着草图看了半天,猛抬头,“何忠呢,把他找来!”

转眼之间,被唤作何忠的一名老工匠,就被找了过来。

“何忠是工匠营里的作头,这里的事都由他管。”游师雄向韩冈介绍了一下,便让韩冈向何忠细细解释。

老工匠听了一阵,又开口询问了几句,便开始点头,回过身对游师雄道:“大概能成。”

有了专家的认同,游师雄开始为韩冈的博学惊叹着:“想不到玉昆你对这军械之事也这般了解。”

“不!”韩冈立刻摇头,“小弟完全不懂,不然怎么会连行砲车的绳索作何之用都不知道?”

游师雄疑惑着:“那为何玉昆你能……”

“只是透析其理而已。”韩冈立刻接口说道。笑了一笑,他又道:“不知景叔兄见过杆秤没有……为何一条有着提绳的木杆,加上一个秤砣,就能称出东西的重量?还有撬棍,为何一人之力,便能撬起一块千斤巨石?”

游师雄更加疑惑:“这与砲车有何关联?”

“因为道理是一样的。”

韩冈随手拿了一根木杆过来,将力学上最基本的杠杆原理和公式,向着游师雄娓娓道来。后世的物理,与现实关系紧密,一点简单的实验就能验证。

单纯做一个的能臣,韩冈并不甘心。学术上,他也有独树一帜的想法。他一直都有意把后世的科学理论与儒学结合起来,这其中最为直观、且容易验证的,便是力学的几条原理。

韩冈已经用撬棍和刚刚让人找来的杆秤说得游师雄连连点头,并在纸上把力臂和力的公式写了出来,最后总结道:“不论是投石车、还是杆秤,又或是最简单的撬杆,外形、用处无一相同,可本理唯一……都是力与力臂的平衡之理。”

游师雄思考了很长一段时间,终于有所领悟,正色向韩冈拱手致谢:“多谢玉昆点悟,如今愚兄方知,砲车与杆秤用得竟是同一个道理……”想了想,又疑惑问道,“不知玉昆为何能想到这一点?”

韩冈笑道:“岂不闻修身、齐家、治国、平天下的前面还有四条。”

游师雄考中进士的学问,《礼记·大学》中的重要纲目当然不会不知,张口便道:“‘古之欲明明德于天下者,先治其国。欲治其国者,先齐其家,欲齐其家者,先修其身。欲修其身者,先正其心。欲正其心者,先诚其意。欲诚其意者,先致其知。致知在格物。’玉昆你是在说‘格物、致知、诚意、正心’这八个字吧?”

“小弟想说的是格物致知。”

“这是‘格物’?”游师雄指了指纸面上的公式,这与他背过的注释完全不同。

“正是!”韩冈点了点头:“不过是先生的‘格物’,而不是郑、孔的‘格物’。”

韩冈对于这四个字的认识,主要是参照了程颢曾经给他讲解过的释义。程颢对格物致知的认识,与此前世间通行的说法完全不同。其中的关键是‘格’这个字作何解释——《大学》中并没有注解,只能靠后世的儒者自己诠释——

如今通行于世的儒家经典的注释,一个是来自于东汉大儒郑玄的注疏,另一个便是唐时大儒孔颖达的诠释。他们都是把‘格’说成是‘来’的意思,就是说知善深则来善物,知恶深则来恶物,教诲人要行善事,方有善物而来。

而韩冈从程颢那里学到的却近于后世的说法,所谓的格物,就是穷究事物之理。张载对于格物说得不多,但他的学说在这方面,也跟程颢相差不大。

张载的关学崇孟,二程的洛学也同样崇孟,都属于思孟学派的源流,自认继承了孟子的道统。对于出自曾参的《大学》自然要深加研究,而不是像汉唐时,只是泛泛而言。

游师雄一拍脑门:“原来如此。愚兄离开先生门下久矣,先生的教诲久未聆听,不意已经荒疏到了这个地步。还是玉昆你有幸,能跟着先生整两年,聆听大道本源之说。”

“万物皆有其理,故而名之为‘道’。先生功参造化、直透大道,韩冈甚至难望其项背。不得不别走蹊径,故而便有了‘以数达理’的想法。”韩冈自负的笑了笑,“道家有三千大道之说,观我圣教,道理虽一,然旁艺亦可近大道!”

“好个旁艺亦可近大道!”忽然身后传来鼓掌的声音。

韩冈、游师雄立刻回头,赫然是韩绛带着种谔、燕达站在了门口处。

半日不见,不知发生了何事,颓唐已久的韩相公目光重新锐利起来,还来到了前线视察。他走进来,看着韩冈在纸上写的一条简单明了的算式,还有新型投石车的结构草图,摇头赞了许久。

“想不到玉昆你不仅仅是用事之才,在学问上却也是自出机杼。”韩绛并不算精研学术的儒者,对于如今的学派之争只是旁观而已。不过方才他在外面听韩冈说得深入浅出,用着最为简洁的算式,便把投石车的原理说了个通通透透,让他也不得不为之惊叹,“只让玉昆管勾伤病事,确是委屈了。”

韩冈连声谦让:“韩冈愧不敢当!”

他是站在前人的肩膀上,并非自己的功劳。前世学到的定理、公式,看似简单,实则是来自于千万人、千百年的积累,然后才由一人研究而出。韩冈虽然是要将其揽为己功,却还不至于自以为是,把韩绛的赞许照单全收。

“就按玉昆你说的来好了。”韩绛更在意的还是投石车,“这投石车先试做两架,如果合用,当奏之于天子。”

