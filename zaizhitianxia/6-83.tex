\section{第十章 千秋邈矣变新腔(五)}

稍稍安抚了几位同僚,蹇周辅安安稳稳的坐了下来。

泰山崩于前而色不变,麋鹿兴于左而目不瞬。

这是苏洵所说的为将之道。所谓‘为将之道,当先治心。泰山崩于前而色不变,麋鹿兴于左而目不瞬,然后可以制利害,可以待敌’。

尽管在韩冈、章敦的影响下,使得在枢密院和武学之中,对苏洵这种书生之见嗤之以鼻——为将之道,首在庙算,在于战前的粮秣、兵备和训练,在于知己知彼,在与他们本身的专业素质,至于临战时的指挥,让三军安危系于一人的性格上,这却是在军事上要极力避免的情况。

但作为诸人之首的蹇周辅表现得如此沉稳,仿佛三军有胆,让其他几位考官都安定下来。

很可能即将面对暴怒的韩冈,蹇周辅浑然不惧。

新党、韩党越是对立,他越是安全。

既然自己判黄裳落榜,主因就是因为黄裳的观点是气学而非新学,那么韩冈为此来非难的时候,王安石就必须维护自己。

这不以个人想法为转移。

在两家相争的情况下,选择一边倒虽然有彻底开罪另一方的风险,但也必然会得到这一方最有力的维护。

只要王安石还想让新学站在官学的位置上,否则他一避让,气学可就要趁势而起了。像自己这种公开坚持新学的官员,事后若被韩冈打击报复,王安石怎么去维持新党的人心?

而且韩冈如果要为黄裳张目,他怎么面对韩绛、张璪?政事堂中的另外两位所推荐的人选。同时王安石、章敦所举荐的两人悉数落榜,反而不便袖手旁观。

重要的是,黄裳已经在阁试上被黜落,韩冈不论有多充分的理由,都不可能推翻考官们的结论。抡才大典之所以为世人所重,正是因为即便贵为宰辅,也不可能干涉入选的名单。若韩冈今天开了头,日后莫说制科,进士科也会成为宰辅逞其所欲的场所,有识之士,哪个不惧?

韩冈要将黄裳推上去,硬是改掉已然确定的结果,必将惹来众怒。难道他不想要他的名声了吗?纵使太后会左袒韩冈,也不可能全凭自己的喜好做事。

蹇周辅笑着提起笔,经过了这一次阁试,自己如此旗帜鲜明的坚持新法,出判外路的日子已是指日可待,一张狨座也近在眼前——事关推举宰辅的选票,那种立场模糊不清的官员,又岂能比得上自己?

“磻翁,政事堂那边又派人来了。”

一名同僚推门进来,脸色苍白,后面跟进来的两人,表情中同样透着怯意。

“什么事?”

蹇周辅放下了笔,语气平静无波。

“说是堂中的三位相公、参政请我等过去,有事相询。”

领头的一人说着,身子微冇微的发抖。纵然是在蹇周辅的坚持下赌了一把,希望以此求名,得到一干新党大佬的看重,但韩冈的愤怒传来的如此快速,还是让他浑身发冷。

蹇周辅轻轻的皱了皱眉。

阁试结束了,结果也呈交上去,作为考官的职责已经交卸,他们的身冇份又恢复到三馆秘阁中的普通官员——昭文馆、史馆、集贤院和秘阁所组成的三馆秘阁——而宰相韩绛,正是昭文馆大学士兼监修国史。

顶头上司相邀,过去还是不过去?

蹇周辅长身而起,神色淡然,“既然相公、参政有招,当然要去。韩相公和张参政推荐的两位都过了,幸好韩参政推荐的人没过,否则真当避嫌了。”

蹇周辅的话,让其他三人立刻安定下来。说得也是,有韩绛、张璪两人在,韩冈怎么去质问自己对考题答案的判定?让韩绛和张璪推荐的两人通过考试,正是为了面对现在的境地。

蹇周辅笑了起来,他正要将这件事的声势闹大一点,让王安石不得不出面。来自政事堂的邀请,正可谓是瞌睡时捡到枕头,他一看左右,“又不是龙潭虎穴,有什么好怕的?而且这件事,哪边占着理这还用说吗?”

随即举步。自家年纪都一大把了,此时不争,更待何时?

大丈夫生不能五鼎食,死亦当五鼎烹。

日已暮,又何必在意日后。

……………………

片刻之后,蹇周辅及其他三位考官都被领进了政事堂的正厅中。

厅内,宰相韩绛在正中,参政韩冈、张璪分据左右,仿佛三堂会审。但看到三人的表情,就知道这一场三堂会审的主审究竟是谁。

不过看见韩绛、张璪都在,蹇周辅登时安心下来。

韩冈若要改变结果,只能去找太后。有太后支持,他才能将黄裳给捞回来。

韩绛、张璪现在都在这里,他凭什么能够让自己屈服?他们所推荐的两人都通过了阁试,韩冈要发落自己,他们怎么可能容忍?

“史馆修撰蹇周辅拜见韩相公,张参政,韩参政。”蹇周辅与三名同僚,一一向韩绛、张璪、韩冈行礼。

待三名宰辅回礼之后,蹇周辅不卑不亢,“不知相公、参政招我等来此,可是有事吩咐。”

“有关今科制科的阁试,有事想要问一问。”韩绛说道。

蹇周辅躬了躬身:“请相公垂询。”

韩绛没有发问,偏头对韩冈道:“玉昆。”

韩冈点了点头,转过来盯住四人:“今科制科阁试,各位所出考题,我与子华相公、邃明参政都看过了。六道题分别出自《书》、《唐书》、《晋书》、《墨子》、《春秋公羊传注疏》与《周官新义》,题目出得不差……”

蹇周辅立刻欠身一礼:“多谢参政夸赞。”

韩冈哈的笑了起来,像是听到了很好笑的笑话,呵呵几声之后,笑容猛的一收,“只是有一个问题……请问军谋宏远材任边寄科的考题在哪里?!”

方才在政事堂中听到韩冈质问,韩绛、张璪的第一反应,就是他在说什么胡话。但是立刻,他们就明白了韩冈的用心。

不是针对三馆秘阁的考官对黄裳试卷的判定,而是直接去质疑他们在出题阶段就犯下的错误。

文章之高下,其实是没有标准可言。

写得再好,也有批评者,写得再差,也不一定没有欣赏之人。

视角不一,观点不一,立场不一,学识不一,经历不一,对文章的评价当然也不会一样。

以《春秋》之经典,却也有将之视为断烂朝报的;以《汉书》之精妙,也有说其是‘排死节,否正直’的。

所谓文无第一,正是这个道理。

韩冈若是去争黄裳的文章高下,少不得拿已经通过的三人文章作比较。拿到殿上争论,太后多半会偏袒韩冈,但如此一来,又置业已通过的三人于何地?韩绛和张璪推荐的两人,可都是在通过者之列。

但正是因为韩冈改去质问考题,这才让两人没有在政事堂中便跟韩冈争执起来,而是选择了站在一边,看韩冈接下来打算怎么做。

听见韩冈的质问,蹇周辅脸上的微笑和心上的轻松顿时不见踪影,戒惧之心腾起:“周辅不明参政何出此言?”

“不明白?那我再问清楚一点。黄裳这一回考的是到底是制科下的那冇一科?是贤良方正能直言极谏科,才识兼茂明于体用科?还是军谋宏远材任边寄科?看各位给出的考题,题目全都一样,却是看不出来这三科到底有什么区别?”

蹇周辅头脑一蒙,心口也猛地一抽紧,难怪韩冈会让韩绛、张璪在这里,根本就是有恃无恐。

“过去朝廷从未开过军谋宏远材任边寄科,如今开科试人,依从制科旧例。曾经授人如贤良方正能直言极谏,才识兼茂明于体用,以及茂材异等等科,阁试皆是一般,并无二致。”

“对,没错。”韩冈点头,“军谋宏远材任边寄一科,过去从未开科,黄裳乃是第一人。”

军谋宏远材任边寄科过去根本就没人参加过,自然也就从来没有为此开科。

制举虽名为十科,但开国以来,真正开科取士的也就其中的三科,军谋宏远材任边寄科并不名列其中。

且不论是贤良方正能直言极谏,还是才识兼茂明于体用,又或是茂材异等,首先讲究的都是考生的学识,贤良、才识、茂才,全都是与才学相关,也就是对经义的理解,而军谋宏远材任边寄就完全与经义无关了。

“但其中区别,尔等三馆秘阁中人应该明白。今年正是大比之年,进士科之后,就是明法科,明法科之后还有特奏名。在明法科考试上问政事,在特奏名试六论,在礼部试上问受赃当如何判,考官有过无过?”

蹇周辅凛然道:“特奏名第一,不过与判司簿尉,明法科出身,亦只与刑法官。二者出身,只能逐阶而升,而进士出身,却能够隔阶晋身,三者岂能相提并论?唯制科十科,不论哪一科,都是堪比状元的制科出身,自当一视同仁。”

“蹇修撰可是以为我等可欺?”韩冈挑起双眉,“我说得是用事,你说的却是磨勘。军谋宏远材任边寄科要取中的是能够领军镇戍边地的帅臣,而非是宿儒、谏官、词臣,这跟朝廷待遇有何关联?”

“学识不足,安可入制科,何况黄裳屡试不第,侥幸得授进士出身。”蹇周辅身侧的一名考官抗声说道。

自入堂来,便被韩冈屡屡责难,纵然畏惧韩冈的权势,也忍不住这口气。

“你是集贤校理赵彦若吧?”韩冈瞥了他一眼,年纪也有五十的样子,哈哈冷笑起来:“赵校理这话说得倒是有意思。你忘了你身边的这位蹇修撰,名为周辅,字称磻翁,这名字可是从垂钓磻溪岸畔的姜太公身上得来的。”

蹇周辅脸上一阵青红。拿人名做戏,最是恶劣。除非是关系极亲近的朋友,否则这样的话形同侮辱。

韩冈冷冷的哼了一声,盯住比自己年长许多的赵彦若:“吕尚【姜太公】身兼文武之道,让他去考制科如何?将武庙中供奉的古今名将,从主祭吕尚、配享张良两人开始,一路排下来,正殿十哲、两庑供奉【注1】,总共七十二人,让他们去考经义,考今天的题目,有哪个能考过的,尔等说一个出来就行。”

赵彦若当即反驳:“敢问参政,黄裳可是在武臣之列?得中之后是否转为武资?”

“不知诸葛亮领军北伐时官居何职?”

“参政忘了何为军师将军!?”

“若是忘了,如何会问?直至唐时,士人出将入相亦是等闲,彼等用心于文武之道,何曾留意于章句?”

“韩参政。”蹇周辅又大声叫了起来,“黄裳与其余人等同为制科,考题岂能有别?其科目虽与人不同,区别亦当是在御试中。周辅既得太后诏为主考,位虽卑,考评各人高下亦是微臣职分,纵使有权臣干涉,周辅也不敢辜负太后的信任。之前周辅已将结果呈与太后,若参政对吾等四人评判有异议,可与太后去说。”

不能再留了。蹇周辅心道。

与韩冈一番辩论已经足够让这件事张扬出去了,在这里与韩冈辩论,虽然能激起同仇敌忾之意,但时间一长,见韩绛、张璪全无相助,皆站在韩冈一边,其余三人里面包管有人会软了脚。

“韩相公,张参政,韩参政,恕吾等有公事在身,不能久留。若再无吩咐,周辅要告退了。”

“玉昆?”做了半日配角的韩绛望了望韩冈,“可是问明白了。”

“一切清楚了,今日可知为何多有贤人远离朝堂。”韩冈摇摇头,冲蹇周辅四人一挥手:“尔等下去吧。”

蹇周辅拱手一礼,与其他三人退了出去。

出门后,微笑随即浮了上来。这件事,如愿以偿的闹大了。

即便韩冈能让韩绛、张璪袖手旁观,但韩冈若为了黄裳被黜落而去找太后,王安石自会来助阵。

王安石、章敦两人推荐的考生全都被黜落,这样一来,王安石和章敦的保护也就免了偏袒之讥,更可体现他们的公心,与为此发难的韩冈正好做个对比。

自己已经为新学冲锋陷阵了,没必要对面的主帅还要自己这个先锋去对付。

“玉昆。”

待蹇周辅四人退了下去,张璪回头看着韩冈。

韩绛和张璪对韩冈的做法也不怎么理解,将人招过来一阵大骂,又能济得什么事?直接去找太后才是正经。

“玉昆,你方才的一番话的确有道理,但蹇周辅最后所说的话也不是没理由。玉昆你想想,朝廷选士,首要还是在公平上。黄裳诚为贤才,但既然其名次已定,又岂能轻改?若是太后再给黄裳上殿的机会,世人不知情由,听说之后岂会不疑玉昆你仗势欺人?换作玉昆你在三馆秘阁中,受命担任这一次阁试的主考,你该怎么出考题?”

张璪好言好语的劝说着韩冈。

换作张璪自己处在三馆秘阁的位置上,也会为此而头疼,总不能给黄裳单独出一份考卷。

题目不同,难易程度自当有别。纵是在出题人自己看来已经做了足够公平,但在他人眼中却绝对不会这么想。

只要有谁没有通过,而与他科目不同的考生却通过了,肯定会质疑那边的题目出得简单了,而给自己的题目出得难了。尤其是黄裳这样处在风尖浪口上的考生,参加的科目又只有他一人被推荐,只要他通过了,崇文院的诸位考官,必定会被一阵质疑的浪潮给吞没。

若是黄裳没有通过,而其他人通过了,又会换作黄裳——更重要的是他背后的韩冈——来怀疑考官是不是想要故意将黄裳给黜落。

这是两难境地。

与其遇上那样的情况,还不如一视同仁,让所有人做同样的考题。

张璪的说法是人之常情,在无法尽善尽美的情况下,表面上最公平的办法,便是减少议论的最好手段。

“朝廷的储才之地,难道养的都是一群畏首畏尾、不肯尽忠职守的蠹虫?”韩冈冷然,张璪说的道理他当然都懂,后世全国一张卷和分省考试争论从来都没有停止过,但现在他又怎么能让蹇周辅的私心得逞,“这等只知自全、不敢任事的蠹虫,也敢来评判何人是贤良方正能直言极谏?!”

说起来,经过了方才的一番对话,韩冈已经确定是蹇周辅这几位考官最后刻意将黄裳给刷落了。以蹇周辅熬到老的年资,他现在只缺一个晋身重臣的机会,帮黄裳一把没有什么用,但换个方向,王安石就得为他出头了。

韩冈如此坚定,让韩绛、张璪没有了劝说的余地。不过他不是多问废话,韩冈的态度,必须得到确认。

张璪轻声叹着,“既然玉昆心意已定,张璪也不好多说。说起来黄裳的确是人才,若不能用在合意之处,的确是浪费了。”

“玉不琢,不成器。玉昆你虽少有才华,也是几经磨难方成大器。黄裳几多波折,也不一定是坏事。孟轲之言,想必玉昆你也清楚。”

“相公的好意,韩冈明白。不过黄裳此番为人黜落,究其缘由,却是被韩冈所拖累。”韩冈起身,对韩绛、张璪道:“韩冈有事须外出,要先行一步。”

“玉昆,去哪里?”韩绛叫住了韩冈。

“韩冈要去求见太后。”韩冈丝毫不隐瞒自己的去向,“还有几件有关代州的事情需要与太后说。”

代州知州章楶是韩冈之前就任河东制置使时的助手,代州的军政布置皆是韩冈离任前所安排的,有关代州的一应事务,韩绛、张璪都不会多说一句。不仅仅是代州,整个河东北部郡县军政,全都是在韩冈的管辖范围之内,其一言可决。

但韩冈现在去求见太后,又怎么会是为了代州之事,至少不仅仅因为代州。

“玉昆,还是小心为是。”韩绛语重心长的提醒道:“正如你之前所说,蹇周辅即是故意黜落黄裳,又岂会没有别的准备?”

“相公放心,必不让小人得意。”

韩冈说完,冲张璪点点头,随即便匆匆离开了政事堂。

目送韩冈的背影离开,韩绛、张璪对视一眼,一起摇头叹息。

之前新党与韩党的交锋,仅仅是暗流汹涌,不过是在选举时略有凸显。但这一回,韩冈为了黄裳被黜落一事去求见太后,却是亲手拉开了党争的大幕。

朝廷自此多事了。

……………………

崇政殿前,王安石脚步匆匆。

尽管没有安排他入对,但看到王安石阴沉如锅底的脸色,沿途的内侍、侍卫谁敢拦他?只能纷纷分出人手,去向崇政殿的太后报信。

直到快到殿门口,才有一名内侍拦住了王安石,杨戬张开双臂拦在了王安石的面前:“平章,平章,还请停一停,还请停一停。”

王安石双目一瞪,“杨戬,你敢拦我?!”

积年宰辅,当朝元老,一怒之威,竟将杨戬身后的禁卫全都刷的一下给吓到了两边。

杨戬却没有躲开,但整个人也吓得僵硬了。

王安石冷冷看了他一眼,就欲从其身侧绕过。

杨戬终于从僵直复原,噗的一下跪到,手却往侧伸去,扯住王安石官袍的袖角不肯放手,大声叫道:“小人不敢拦着平章禀报国事,太后也不会拒见太后,但这毕竟是小人的职分,请平章稍待,马上就会有人来请平章入内。”

王安石将长袖一拂,一声断喝:“放手!”

杨戬几乎都要瘫了,这可是刚刚领头平叛的平章军国重事,纵使不管事,可宫中又有哪个不怕他。但杨戬的右手,却死活不敢放手。

一名小黄门赶着出了殿来,大声叫道:“太后有旨,宣平章觐见。”

杨戬终于放了手,却也不起来,就在地上叩头如捣蒜,口称死罪不止。

王安石却懒得理他,整理了一下被拉偏了的衣袍,随即走近了崇政殿中。

无视同在殿内的韩冈,王安石向着屏风后的太后行礼。

待王安石行过礼,太后立刻问道:“平章求见,可是有何急务?”

纵然女婿就在殿内,王安石连瞥都不瞥他一眼,“禀太后,臣为阁试而来。制科御试人选已定,岂可变动。黄裳被黜落,是其学问不佳。蹇周辅知阁试,有功无罪!”

来自屏风后的声音充满了惊讶:“平章在说什么?”

注1:北宋武庙与文庙相对,总共祭祀古今名将七十二人。

武庙神主:吕尚(姜子牙)

一、配享主殿:张良

二、十哲:管仲、孙武、乐毅、诸葛亮、李绩并西向;田穣苴、范蠡、韩信、李靖、郭子仪并东向;

三、东庑供奉:白起、孙膑、廉颇、李牧、曹参、周勃、李广、霍去病、邓禹、冯异、吴汉、马援、皇甫嵩、邓艾、张飞、吕蒙、陆抗、杜预、陶侃、慕容恪、宇文宪、韦孝宽、杨素、贺若弼、李孝恭、苏定方、王孝杰、王晙、李光弼并西向;

四、西庑供奉:吴起、田单、赵奢、王翦、彭越、周亚夫、卫青、赵充国、寇恂、贾复、耿弇、段颎、张辽、关羽、周瑜、陆逊、羊祜、王浚、谢玄、王猛、王镇恶、斛律光、王僧辩、于谨、吴明彻、韩擒虎、史万岁、尉迟敬德、裴行俭、张仁亶、郭元振、李晟并东向。
