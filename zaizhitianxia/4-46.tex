\section{第九章 鼙鼓声喧贯中国(六)}

【下一章还是中午一点前。】

断断续续下了两天的雪终于停了,而穿行在山谷中的寒风一直就没有停歇过。

山风将地面上的细雪刮起,呼呼打着旋儿,化作一团团白色的幽影,随风奔出百十步,然后又忽的一下,在一片白色的雪地中散得无影无踪。

宋军营寨前沿的栅栏,离着驻扎着上万人的罗兀城及城下军寨只有半里地。厚达一尺的积雪覆盖着其中的地表,两军之间的空旷地带,并没有什么人去踩踏,是近乎完整的一片雪白。

唯有靠着西北山壁的一处雪迹凌乱,还有着一处处深褐色的印痕,那是昨夜过来偷营的两百西贼精锐,被守候已久的宋军万箭射杀后的残迹——他们被砍下的头颅,此时正一个个被高高的悬在栅栏上,空洞的眼神正对着将他们派来送死的地方。

正常的时候,这仅仅只有半里宽的分界线,双方都应该放出几十名精锐的游骑梭巡其中,看着对方哪里有了疏忽,就趁势如兀鹫一般叼上一口,也好提振自家的士气,打压对方的军心。只是在厚厚积雪之中,什么样的游骑都会变成慢吞吞的乌龟,运气不好的就是被人围杀在雪原中,哪一边都没心思放出自家的精锐来送死。

只是仅仅半里的距离,不过一百八十步,如果用神臂弓向着罗兀城的方向仰天射上一箭,六寸的木羽短矢只要不被风卷走,多半就能落到城头。也的确有士兵这么做,明着欺负党项人缺乏能远射的重弩,时不时的射上一下,尽管落下来的箭矢已经伤不了人,但也能在守军中造成一点小混乱。

当然,能随意使用神臂弓,而不是必须等待军令才给扣动牙发的士兵,在罗兀城下的两万宋军中,也只有区区一个指挥的选锋军。

鄜延路的选锋,平日里身穿红袄,腰扎锦带,跨着的腰刀都是良工打造的稀罕货,在延州城中的时候,看人都是用鼻孔的。现在聚集在栅栏后的他们,红袄锦带依然穿在身上,不过外面则套了一身区别于普通士兵的精制板甲,有着护臂护腿,上面泛着漂亮的铜色,身后还配了一条鲜红如血的短披风,戴在头顶的铁盔上,一团红缨骄傲的高高挑起。

这群选锋只是寻常的坐着站着,却已是威风凛凛。其中一名小卒表现出来的骄悍,就能普通禁军中以勇力著称的军官不相上下。如果再将放在身侧的七尺陌刀一举,可一击斩杀战马的刀刃上幽蓝的寒芒映着雪光,顿时就是一阵凛冽杀气扑面而来。这是自一路数万大军中精挑细选出来,用以攻城拔寨的一群勇士,也是种谔手中用来改变战局的一支力量。

不过他们现在无所事事,除了寥寥几人闲来无事的向着罗兀城头或是城下军寨射上两下,多半还是看着身后一片空地。

就在空地中央,种朴抬起了头:“风似乎大了点,这个天气能上去吗?”

“今天的风还不算太大,没有关系的。”种建中冲着前面吆喝了一声,“把绳子栓紧点,钎子也往地里多钉两寸!不要人还没进去,就飞上天了!”

就在选锋军士卒注视的方向,也就是在种建中和种朴的面前,一堆篝火正熊熊燃烧。从火堆中穿出来的滚滚的热浪将周围的寒意尽数驱散,也将营中清理之后,还残留的些许积雪也一起化尽。

而更多的热气,则是被灌进了一个巨大的气囊中。就在上千只眼睛的注视下,气囊一点点的鼓胀了起来,向上浮动。正在加热之中的仅仅是一艘飞船,但旁边还有两艘飞船停着。过一阵子,这些飞船就会交替的浮上天空。一艘眼看着就要落下来,下一艘就立刻升上了天去,可以持续的坚实着敌情。

飞船的气囊外侧都用五色绘了兽面,一个个活灵活现。张着血盆大口,瞪着眼睛,面目狰狞,直欲择人而噬。

军器监提供给鄜延路的飞船无一例外都是素色的,毫无纹饰。韩冈也无心弄些花里胡哨的纹路来做装饰。

不过飞船一落到种谔手中后,就立刻改头换面。在种谔看来,把飞船仅仅当成巢车来使用,实在是过于浪费了。飞在天上的鬼怪,且受到官军指挥,这可是能将西贼的士气给一下打到底的利器。

用了小半个时辰来加热,气囊终于整个腾空,由几十条绳索连在一起的篮筐,被扯得离开了地面。不过被缆绳系在地上,离地也只有几寸而已。

就在统领飞船事务的军校催促下,一名精瘦干练的士兵随即手脚灵活的爬进了篮筐。

种建中有点不甘心的看着飞船篮筐中的背影。他的身材健硕,乃是个伟丈夫,虽然身上并无一丝赘肉,可体重依然超过太多。当初在京城时,种建中因为要保持文官的稳重和体面,眼馋的看着一个个武将跳上飞船,然后升空上天。等他到了延州之后,闲下来第一件事,就立刻乘上飞船上天,饱饱的看了几日延州城的风景。可眼下是战时,一点重量浪费的余地都没有,种建中即使有再多的心思,也只能干瞪眼。

篮筐的正中央安着一个四尺多高的燃油炉子。斥候一翻进篮筐,就立刻将火生了起来。窜起的火焰,维持着气囊中热空气的温度。虽然不能直接凭着这具油炉将飞船升起来,但已经可以将飞船的滞空时间延长一倍。

在飞船统领的指挥下,缆绳渐渐的松了。挂在篮筐外的沙袋,一个接着一个的被解开,每丢下一个沙袋,飞船就向上蹿上一截。就像被放在了台阶前的青蛙,一阶一阶的蹦着。而注视着飞船的人们,也随之一点点的抬起了头。

巨大的气囊就这么拖着篮筐和篮筐里面的宋军斥候,就这么蹦上了天空。自近三十丈的高处,俯视着离之不远的城池和军寨,将城寨中的一举一动尽收眼底。

当兽首飞船在宋军阵列中腾起的时候,罗兀城的城头上很明显的惊起了一阵骚乱。就算党项高层中有那么几人从细作那里,听说过宋人手中有了能飞天的船只,也不可能有机会当面看过实物,更别提下面什么都不知道的士卒。

旧年狄青因为相貌英俊如女子,故而上阵时就在脸上带着一副青铜鬼面,披头散发的冲锋陷阵,让当面的西夏军见之胆寒。当年带着青铜鬼面的猛将如今早已不在人世,但现在飞在天上的恶兽可比当年的狄青更为可怖。

统率城中兵马的嵬名阿埋望着天上张着血盆巨口的怪兽,手脚冰凉。他当初就知道摊到这个职位不会有好结果,几次想调回兴庆府却都被堵了回来。这两年没少烧香,想不到还有

“这是什么怪物?!”他尖叫着。

“这是飞船。”嵬名阿埋身边的幕僚提醒道。

“这就是飞船?”嵬名阿埋心定了一些,他自从来到罗兀城后,为了自己的小名,着力打听宋人那边的动静,飞船这个名词还是听说过的。

幕僚用力的点头:“蹈于虚空,乘风而行,这肯定就是飞船!”

“传令下去,让士卒不要惊慌,只是宋人造得军器罢了!”嵬名阿埋立刻下着命令。只是他让下面不要惊慌,可他自己的声音都在颤着。宋人都能飞上天了,那城墙还有什么用,转眼就能飞到自己的头顶上,“对了,要射下来,让人快点将那怪……飞船射下来!”

罗兀城城头上的西夏弓手瞄准了天上的飞船,纷纷射了过去。只是这完全是浪费箭矢的无谋之举,没有什么弓弩能射到比十几层的佛塔还要高出许多的飞船。仰着脖子看着都累,箭矢飞到半空就纷纷落了下去。如果硬要说的话,也只有宋军倚仗守城利器的床子弩,才有可能将天上的飞船给射下来。

趁着罗兀城中的混乱,宋军将一架架霹雳砲推上了阵前。六十余架巨型的投石车在城下寨前摆出的浩浩荡荡的阵势,让城头上的守军看寒了胆。种谔的将旗也提了上来,牢牢的插进了雪地中。两万宋军将士全都高声欢呼,伴着猝然响起的战鼓,如雷霆、如海啸,引得附近山岭中的积雪崩塌下来。

一个竹筒这时从飞船上丢了下来,啪嗒一声落地。立刻有人上前去捡了起来,转身送到了种谔的面前。抽出竹筒中的纸条,种谔展开一看,就抬头望着罗兀城中,“城里的西贼居然没有动静。”

“还是不敢出来拼命。嵬名家的人是不是给梁氏兄妹吓破胆了?”种朴冷哼了一声。

六十架霹雳砲就停在寨子外,而寨中都没有派人出去守卫,就算明知道会有陷阱,也该出来拼个运气,看看能不能烧掉几架,怎么连一点动静都没有?

“他们肯定会出来的。”种谔没有再多话,找来亲卫吩咐两句,让他跑去传令,就立于大旗下静静地等待着。

