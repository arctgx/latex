\section{第十章 千秋邈矣变新腔(14)}

“……参政可是想要让棉布代替绢帛做军袍?”

见韩冈说得言辞凿凿,向太后沉默了片刻,才再一次向他确认。

“正是。”韩冈很肯定的点头,“棉布比丝麻更厚重,比丝麻更细密,比丝麻更加耐用,比起绢帛做成的军袍,棉布军袍更为持久耐用,”

“但朝廷要将下发诸军的成衣和衣料换成棉布,那可要数百万匹之多!”

大宋的马步禁军,无论是冬衣,还是春衣,除了脚上的麻鞋之外,其余从里到外,几乎是绢绸所制【注1】。

三衙中的上位禁军,自都虞候以下至军士,每年得到的制作军袍的衣料,多的有白绢三十匹,少的也有绢或油绸六匹,另外还有丝绵、麻布和裁衣钱。中下位禁军,每年多也有绸绢六匹,丝绵十二两和随衣钱三千。同时厢军也都有衣料下发,春冬两季衣料加起来也有两匹到四匹。

天下禁军、厢军加起来有百万之数,每年耗用在他们身上的绢绸多至五六百万匹,如果都改成棉布,得将天下的棉布产量都搭进去才差不多。

“没有必要将所有的衣料都换成棉布,当下棉布的产量也支撑不了,只要先以棉布代替做披袄和外衫的那部分衣料,就近供给产地附近的禁军,那还是可以支持的。至于其他各处,可以先从棉絮开始,看看军中的反应——用棉絮代替一部分丝绵,棉絮、丝绵各半。”

“棉絮倒是可以。”

丝绵就是没有纺成线的蚕丝,如今下发给军中,是作为冬衣的里料。换成棉絮,依然可以做冬衣里料。

世所共知,棉絮和丝绵同样保暖,而棉布如今是奢侈品,至少是稀少的贵价货,以棉絮代替丝绵,不用担心军中有人反对。而且军中的丝绵通常是烂茧坏茧抽出来的丝絮,也远比不上正常轧花清理过的棉絮。

“棉布就近供给,那就只有西军了,参政,是不是?”

太后的声音中依然带着迟疑。就是太后也知道,韩冈的提议,对棉布商决不是一桩好事。

不管从哪个角度来看,朝廷从来就不是一个好买家。

看到一个行业有暴利,如果转运贩售没有太多麻烦,又易于控制,朝廷往往就插足进入,手中拿着的武器名叫‘专利’。

盐铁的专利就不说了,从汉时起就在朝廷手中,汉昭帝时桑弘羊与贤良文学们争议盐铁专卖的《盐铁论》,是如今儒生们的必修课。

盐铁之外,比如矾业,比如酒业,再比如茶业,如今都是给朝廷包下去了。

若是那些难以控制的行业,比如海外贸易,就是和买。但凡海外商船抵达各地港口,市舶司便会先从其中征收两成的税额,再将其中有利润的海外商品以平价强制收购一部分,剩下的才允许发卖。

而民间生产出来的丝绢,朝廷除了惯常的税收之外,也常常以和买的方式,以低于市价的价格从百姓们手中强制收购。在全国各个丝织品产地,和买已经成了一项税收,压在当地每一户百姓头上。

不过上有政策,下有对策,那些需要按时被和买丝绢的百姓,都会设法用最少的生丝,来织出长度宽度都合乎规制的绢帛,然后再敷上厚厚的粉,让重量也能够达到标准。这样一来,只需要给来和买的胥吏递上一点贿赂就能过关。而这种由朝廷买下来的丝绢,由于太过单薄,不能裁剪成衣,又很容易损坏,各地仓库常常会爆出一次姓有上万匹丝绢因朽烂被废弃的消息。

可尽管和买制度弊病丛生,新法推行有年,却也没能将之改变。朝廷总有办法来将损失转嫁出去。

既然就算朽烂了的粮食都能当成口粮配发给军中——曾经有过官府拨发军粮皆是黑米而惹起兵变的旧事,所谓的黑米,就是烂掉的大米,至于此事如何解决,则很简单,官府将下拨军粮改成黑米、白米各半就解决了——那么轻薄不堪使用的丝绢照样能发下去当做军饷,只要还没有彻底朽烂。

而新法推行的目的,本也不是为了百姓,而是为了富国强兵。和买制度没有干扰到国库收入,而废除和买丝绢反而会让国库收入锐减,新党当然没有动力去改变。虽说百姓和士兵受到损失,可百姓纵使被盘剥也还能过得下去,而另一边又多是厢军,闹不出什么乱子,谁也都不会去在乎。

和买的弊病尽人皆知,所以韩冈此刻在崇政殿上,推荐棉布代替绢帛成为军服的衣料,而且他家里就是陇右最大的棉布商,棉行行会使得韩家与熙河、秦凤各大世家紧密的联系了起来,另外还有熙河路山中的蕃部,以及甘凉路上开始种植棉花的汉蕃各部,但在殿中所有人看来,韩冈绝不会是在推销自家的商品。

跟朝廷做生意,这是疯了才会有的想法。想也知道,一旦棉布成为朝廷指定的军服材料,朝廷是绝对不会以市价收购,而绝对会选择和买。就算是贵为参知政事的韩冈,也不可能让朝廷放弃和买的方式。

就算一时间可以在和买的价格上做些文章,让当地百姓可以得利,但时间一长,一代代官吏从中上下其手,侵占越来越多,加之物价的改变,会让这和买之制成为百姓脖子上的又一道枷锁。

韩冈这是不想回乡了吗?一旦太后接纳了韩冈的提议,棉行中的成员,都要把他恨到骨头里。

还是说他有别的想法?

上至王安石、韩绛,下至刚刚得授枢密副使一职的曾孝宽,都觉得韩冈不会作茧自缚。

“当然有西军。”韩冈一口应承下来,“关西苦寒,朔风一起,寒意侵骨,丝麻织物一向不能御风,而军中所发衣料更是如此,将外袍衣料换成棉布,也可让戍守在外的将士得以安度寒冬。”

韩冈稍稍停了一下,然后在众人的注视下,继续道,

“另外,据臣所知,如今江南也开始种棉织布了。”

原来如此,章惇恍然,这是为了排除竞争对手吗?

不过章惇立刻又疑惑起来,说起周边军旅,关西的禁军数量,可不是江南能够相提并论。关西的军力极盛时接近四十万,占去了全国兵力的三分之一,如今再怎么削减,其中禁军也不会少于十五万,而江南诸路的禁军,加起来也没三万人马。

杀敌一千,自损五千,兑子也没有这般兑的。

不过章惇转念一想,又想到了缘由。

当西军都开始穿戴棉布衣袍,一贯自视极高、看不起外路土包子的京营禁军又如何甘愿穿一身廉价的丝绢让人笑?到时候闹起来,朝廷为了安抚他们,必定要从江南和买——关西的棉布已经提供给了西军,当然不可能再冲他们下手。

当京营禁军这班赤佬都穿戴上了棉布军袍,恐怕朝廷中的官员,也会要求朝廷将下发的丝绢换成棉布……

不……不是恐怕,应该是肯定。

章惇心中对自己说着,官员们的德行,作为西班之首的他最清楚不过。

京城中的大小官员,文武两班和宗室、内侍加起来近万,他们每年需要赐予的衣料,同样是一笔大数目,而且他们对于衣料的要求更高。当他们开始请求朝廷赐予棉布衣料,朝廷将目光转向西北的时候,韩冈就完全可以以关西百姓无力支撑而为之坚拒。

任谁都知道,西北穷而东南富,西北叫穷,人人肯信,而东南叫苦,得到的只会是讥笑。

章惇啧啧暗叹,韩冈这未雨绸缪的心思,可是盘算得够深远的。

韩冈也的确不过是未雨绸缪罢了。

一方面,以棉纺产业曰渐扩大的现状,迟早会成为朝廷征税与和买的对象,既然是迟早的事,与其到时候与人喋喋于朝堂之上,还不如现在将整件事控制在自己手中。

另一方面,也是为了东南方向上的竞争者。

对于来自于东南的竞争,雍秦商会中的核心行会——棉行的成员们都有足够的心理准备,但他们的心理准备,依旧远远比不上即将面临的威胁。

衣被天下四个字,从字面上就可以了解到成为棉花成为江南主要的经济作物之后,将会对世人的穿戴产生什么样的影响,而且那还是单指松江一府,也就是如今的秀州【今上海、嘉兴】北部区域的生产能力。

注1:宋代的军装分为春冬两式,马军、步军各自式样不同,每年有时会下发成衣,更多时候就下发布料和丝绵里料,着官兵依照体例自行裁剪制作。

据仁宗天圣七年大理寺裁定的诸军衣装供给标准的规定:

春衣:

马军七事:皂绸衫、白绢汗衫、白绢夹裤、紫罗头巾、绯绢勒帛、白绢衬衣、麻鞋;

步军七事:皂绸衫、白绢汗衫、白绢夹裤、紫罗头巾、蓝黄搭膊、白绢衬衣、麻鞋。

冬衣:

马军七事:皂绸绵披袄、黄绢绵袄子、白绢绵袜头裤、白绢夹袜头裤、紫罗头巾、绯绢勒帛、麻鞋。

步军六事:皂绸绵披袄、黄绢绵袄子、白绢绵袜头裤、紫罗头巾、蓝内搭膊、麻鞋。

以上是‘不系军号’军服,并不标示部队番号。

另外还有‘系军号’军服,如捧曰、天武等军的绯绸衫子,神卫、渤海等军的紫绸衫子,龙卫、吐浑等军的紫施衫子,而御前班直,更有锦袄子、褙子和皂罗珍珠头巾作为‘系军号法物’。

