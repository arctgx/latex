\section{第23章 弭患销祸知何补(三)}

前一天在京齤城中发生的大齤事,第二天一大早,避免不了的就会在上朝的朝官中传播开来。

十七人死亡,一百多轻重伤,是几十年来伤亡最大的人为事故。对于死伤者的同情,在被灯笼照亮的朝官们脸上基本看不到多少,但这一事件,会引发什么样的后果,已经在人群中惹起了一阵议论。尤其是其中还死了一个贵胄,更是让这番议论热烈了三分。

许多人观察着御史们的神色,想看看他们最终会咬到谁为止。仅仅是倒霉的两支球队,还是要敲打一下齐云总社,顺便将赛马总社一起带进来,又或是将一直想要弹劾却始终没能成功的韩冈列为攻击的目标。甚至有可能开封府都脱不开干系,渎职和坐视的罪名,很容易加到几任开封知府的头上。

不过正在宣德门前的两名御史,一如既往的黑着脸,不苟言笑,看不出什么眉目来。都说包拯脸黑,所以是真御史。但包拯做御史,人所共服。眼下的御史台越来越不成器,还是一般模样,倒是猪鼻子插葱——装象的感觉了。只是虽然看不出来,但不会放过这个机会,应当是可以确认的。

或许,又会有乐子看了。

在御史之外,还有一个疑问:

“谁是南顺侯?”

有些见识的官员,听到这个名号就知道绝不是正经的封爵,多半是来自南方的降臣。可是对天下四百军州、两千县监了若指掌的毕竟是少数人,能确定大宋没有一个南顺县的朝官,在现在的宣德门前并不是很多。

幸好有见识的人在人群中还是有的:

“不就是交趾的僭主吗?当年在交州称王称霸,自号大越皇帝,还犯我疆界,屠我子民,不过天兵一至,也就灰飞烟灭了。”

问话的人听了却悚然一惊,“交趾的,该不会是……”

只是他半句话才出口,立刻就又紧张的闭上了嘴。而周围的众人,先是一头雾水,但看清他脸上的惊容后,却也没花多久就反应过来,先后警觉的将话题转开。

只要在朝廷里面做官的,不会不记得国朝之初,有个在生日的时候被赐了牵机毒的南唐违命侯,还有在六十岁寿诞的时候突然暴毙的吴越国钱邓王。生日忌日并在一处,给后人省了一重麻烦。太宗皇帝的体贴,世人都是一清二楚的。

李乾德于乱中被践踏致死,说起来是个意外。只是官场之中,人人都少不了多个心眼,要让他们相信这个意外仅仅只是意外,那还真是有些难度。如果整件事不是意外的话,那么天子在其中扮演了什么样的角色,就让人实在不敢再往深里去想。

尽管一时间无人再敢公然议论这一件事,可整件事已经传遍了皇城之中。当不需要参与日朝的韩冈抵达太常寺衙门的时候,一下就成了众目汇聚的焦点。

恍若无事的走进衙中,照常处理日常事务,韩冈的神色上并没有一丝异样。下面的官吏互相之间乱抛眼色,却没有一个人敢站出来问上韩冈一句。直到苏颂处理完成了光禄寺中的事务,来到太常寺这边时,才有了问向韩冈的第一个问题:“玉昆,昨天出了事的是棉行的球队吧?”

“出事的是看球的看客,两边的球队都安然无恙。”韩冈摇着头:“死了十七人,近两百的轻重伤,城西医院忙了一夜,要不是在筋骨外伤上有的翰林医官和医生全都到了,这一回就不止十七人了。真不知道怎么能闹起来?看球赛能看到斗殴闹事的地步,这个风气不好好整治一下,日后只会变得更恶劣。”

韩冈看起来坦率得不得了,苏颂才问上一句,就像竹筒倒豆子一般将心中的想法全都倒了出来。

苏颂坐了下来:“玉昆你的意思要严惩?”

“杀人偿命,伤人重惩,十七个人的性命岂能就此罢休?那位南顺侯倒也罢了,但剩下的十六人,无辜枉死,总得给个交代。”

苏颂大概听明白了韩冈的意思。既然要依律追究元凶,那么理所当然的,球队也就能置身事外了。而从律条上来说,的确是不关两支球队的事。看球的球迷犯下的罪即便再重,也牵连不到球队身上,而且事发地点据说还是在球场外,依照哪一条刑律,也不能将罪名安到两支球队身上。顶多是追凶时,带人过堂而已。

以两支球队中的成员在京齤城中的名气,就是过堂,也不能将他们一并下狱。而开封府中的官吏,在蹴鞠联赛上得到的好处数目甚多,更不会在球员身上玩敲骨吸髓的那一套,必然是轻松脱罪——只要御史台不插手的话。

苏颂相信韩冈也能想到这一点,也不多言。转而问道:“这一次的死伤怎么会这么重,到底是怎么回事?过去从来都没有过。”

韩冈叹了一声:“若是外路的州县,一场比赛不过聚集三五千多人,也就是草市、庙会而已,纵生乱,也不会有大的伤亡——京齤城之外,也就东岳庙会等寥寥数事能聚万人之中。但京齤城军民百万,一场比赛往往万人。这方面,必须设法弥补。亡羊补牢,为时未晚。前日的惨剧,不应该再发生了……”

“玉昆你打算怎么做?”

“什么都不能做啊。”韩冈摇头,浮现在脸上的笑容平平淡淡,“该怎么处置,是废是改,那得由天子、政事堂和开封府发落,非韩冈所宜言。”

“就不担心株连到两支球队和齐云总社?”苏颂很是有兴致的问道。

“终究还是开封府的事。有钱醇老【钱藻】在,想必肇事之人无法逍遥法外,而无辜之人,也不至于蒙受不白之冤。”韩冈事不关己的说着,他丢开手上的笔,笑着对苏颂道:“这一回厚生司、太医局和医院也算是练兵了。日后再有天灾人祸,有了经验也免得临上阵会手忙脚乱。”

韩冈摆明车马,绝不会公然插手此事。并非职司相关,他可没打算站出来干预。想来有不少人盼着他跟开封府闹起来,韩冈如何会让他们如愿以偿?他现在只管手边的差事,这件事根本就不需要他强出头。想拿十七人的性命

“看来玉昆是胸有成竹了。”

苏颂明白韩冈的为人,不管面临什么样的局面,还没有亲身较量一番,便宣告认输,绝对不会他的性格。若不是有绝对把握,绝不会坐到一边冷眼旁观。

与韩冈有关此事的对话到此为止,苏颂知道自己只需要等着看后续发展,便能知道韩冈的底气何在。而这一切来得很快,到了第二天,齐云总社公布处罚决定的消息便传遍了京齤城。

棉行喜乐丰队和福庆坊福庆队两队,罚分二十分,在地区常规赛进入后半段之后,这么大的罚分,使得两队实质上退出了季后赛名额的争夺。并各罚款五百贯,作为医疗费用和抚恤费用。在齐云总社发出的声明中,虽然两队并非肇事者,但必须为球迷负起连带责任。

除此之外,在惨剧头七的那一天,齐云总社将会礼聘僧道做一番水陆道场,为十七条冤魂祈求冥福,并求佛祖道祖保佑,让伤病之人能早日康复。同时为了避免惨剧重演,齐云总社也会要讨论如何能对球场进行允许范围内的改造。而最重要的一条,就是从即日起,京齤城的蹴鞠联赛将会暂停一个月,等待朝廷的处置。

苏颂也不禁对这一以退为进的手段激赏再三。

这一下子,气毬便被踢到朝廷那一边,‘老实守矩’的齐云总社通过这一招,轻易的就凝聚了混乱的人心。当齐云总社摆出了老实听教的态度,对朝廷来说,已经不方便加以重惩。因为在总社背后,有着以宗室、贵戚、豪商所组成的团体,更有着几十万京齤城百姓的支持。只要人心稳固,朝堂想做出不利的判罚,也会有着极大的阻力,甚至难以成功。

在这一过程中,韩冈甚至什么都不需要去做。

……………………

“这样就行了吗?端明?”作为韩冈的亲,何矩心中依然打着小鼓,两只眼睛上密布血丝,显然一夜没有阖眼。为了说服总社中的那群老顽固,何矩费了不少的心力。

“足够了。”

韩冈漫不经意的点着头。这就是他事先的吩咐,态度要端正。犯了错不要紧,要紧的是是不是已经有过正式的赔礼道歉。将场面上的事做漂亮了,就会使得齐云总社一下就摆脱了朝野两方面的围剿,摆脱了被动的局面。

“要记住了,不要等着上面的决定。”韩冈再一次叮嘱着。

这件事上要争取民心和士林中的舆论,就必须提早一步将可能成为攻击目标的弱点给消除。已经做到了这一步,若是御史台穷追不放,朝野内外的同情心,只会落在齐云总社身上。

“比赛重开要等朝廷的吩咐,不过十六名受害者入土安葬,总社的会首和两队的队头,还是要去上柱香,吊祭一番才是。”不能遗人把柄,韩冈的态度十分坚定,“赛马总社那边也要配合齐云总社,兔死狐悲的道理要多提醒两遍。”

韩冈在此事上的嘱咐也到此为止。在编纂药典并潜移默化的推广气学这个大课题面前,眼下的那点麻烦,只是枝节而已,不值得深究。

眼下韩冈就是想通过这一桩意外,好生的看一看以蹴鞠联赛为脉络所组成利益集团,到底能有能耐。

