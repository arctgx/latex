\section{第46章 世情如水与天违(下) }

【今天实在是对不住各位书友了,先补上一章,下一章明天补上。】

大殿上一片寂静。

疯狗一般咬着王安石和新党中人的唐坰,也如被雷劈了一般,变得张口结舌起来。

“石得一……你再说一遍!”赵顼的手颤起来,有些恍惚,一时间竟不敢相信期盼多年的心愿就这么简单的成功了。

文德殿中的几百名文武官员,也都是如陷梦境,怀疑着自己的耳朵。不过有的是噩梦,有得则是美梦。

石得一在殿门口向里面爬了几步,扯着嗓子叫了起来,“启禀官家,熙河路派来的信使就在宫外!露布飞捷,东京城……不,从长安到东京,天下人都知道了。熙河大捷,王韶在关西拓土两千里,生擒木征,收复蕃部无数!”

若在平日,石得一如此行事,必然会被御史弹劾有失朝仪。‘官家’二字,也是私下的场合才会用到的称呼。但现在哪个御史还有这份闲心?

赵顼都差一点就坐不住要站起来。他向前探着身子,更进一步的追问道:“露布飞捷?!就在门外?!”

“启禀官家。”皇城司提举抬着头叫道,“就在宣德门外!”

“奏报呢?”

“应当送去了崇政殿!”

“重赏!”赵顼重回御榻上,如释重负的笑了起来,“重赏!从熙州到东京,这几千里路上,所有传递捷报的急脚皆授以重赏,钱十千,绢四匹!”

“臣遵旨!”石得一叩头领旨,尽管这并不是他的职司。

百官大起居是朝廷的重要典礼,严禁外事干扰。而文德殿也与大庆殿一样,是礼仪性质的殿阁,并不处理政事。就算是紧急军情,也应该送到崇政殿中。

不过送进通进银台司的奏报,都是要经过皇城一侧的安上门,而皇城司的作用不仅仅是打探京中民间情报,同时也是管理者皇城内外的门户安全。熙河路露布飞捷的信使刚刚抵达,石得一就收到了消息。

正常的军情传递程序是急脚递或是马递将四方奏报送到通进银台司,然后再从通进银台司送往中书,中书再转往崇政殿。区区一个皇城司提举根本不能插手其中,更是犯了大忌的一件事——如果石得一能将银台司转发到奏报都控制起来,那就等于出现一个能把持皇城内外联络的权阉了。

但石得一仍是不顾一切将捷报直接送到了文德殿上。他敢如此行事,并不是被胜利的消息冲昏了头脑。因为他听到了唐坰在殿上揪着王安石弹劾的消息,明白这是对王安石示好的良机,更是能博得天子好感的最佳机会。

一点为了天子而犯的小过,就算惹来了御史们的弹劾,也只会让天子心中多了一分亏欠,日后反而会变本加厉的给补回来。现在的这位宫廷的主人,与真宗、仁宗同样都有这一个毛病。

石得一爬起来躬身退出门外,赵顼这时坐不住了,竟站起来在御座前来回走着。来回踱了几圈,又坐下来,忍不住的呵呵笑着。

没有人会在这时候打断赵顼的兴头,更没有人会跳出来说只是熙河路一面之词、要先派人确认明白了再说。

这个等级的捷报,本就不会有人敢于伪报。如果公开表示自己的怀疑,日后被证实真实性后,那就是丢人现眼。

熙河路的大捷既然已经确定,唐坰之前对王韶、韩冈的一番攻击,也就成了放屁。连带着他对王安石的弹劾,也一起成了笑话,就算其中有值得下手的地方,又还有谁会在此时此地,跟一直以来都站在王韶背后的王安石过不去?

王安石黝黑的面孔在被唐坰当面弹劾后,就一直阴沉着,现在也终于放松了下来。这时候,谁还能再指责他?王安石从陛前返回大殿中央,只留着唐坰孤零零的站在原地。

唐坰失魂落魄,冯京和吴充也是板着脸,往回走的王安石都看在眼里。只是竟然连王珪都是脸色难看,却是出乎意料之外,这还真是让人惊讶。

一直以来,王珪可都是以天子的意志为依归。正常情形下,他肯定是是第一个跳出来恭喜天子的,而不是发呆的站着。

但这个疑惑只在脑中一闪而过,王安石现在也是兴奋莫名。朝堂上的局面因为一次捷报而逆转,他依稀记得之前有过一次,那一次甚至是将文彦博差得气得中风。

不过前次是意外,捷报到得凑巧。而今次的石得一,却是故意选在这个时候来报喜信的。王安石明白石得一的用心,但还是对皇城司提举有了一点感激,因为石得一的确是在最合适的时机将捷报送来。

殿中数百人的视线都在跟着王安石的脚步,看着他走到自己的班列处,看着他回身,看着他冲着赵顼一揖到底。

然后朗声说道:“木征降伏,董毡已是独木难支。一战拓土两千里,真宗以来,边功以此为首。今日臣为陛下贺,臣为皇宋贺!”

宰相领头,群臣一个个都反应了过来。皆深揖下去,跟着王安石一齐恭喜赵顼,“臣为陛下贺,臣为皇宋贺!”

声震大殿内外的恭贺声中,赵顼放声长笑。一个多月来郁结在胸的闷气,终于舒发开来。而几年来的殷勤期待,也终于等到了开花结果的一天。

恭贺之声结束了下来,笑声也终于停歇。赵顼望着王安石,望着几年来在风风雨雨之中,一直支撑朝局的宰相。刚刚上京时的意气风发,但到了如今,已经是两鬓添霜。

皇帝的心中感慨万千,“四年了,整整四年了。这四年来,没有相公的一力主张,没有相公的鼎力支持,哪会有今日的胜果。熙河大捷,虽是数万将士奋力报国的结果,但在朝中,却尽是卿家之力。”

王安石有些羞愧,黝黑的面皮微微泛红。今次河州退军,他也是投了赞成票的。若不是王韶及时回来,差点就造成了不可挽回的结果。

他连声自谦:“王韶是陛下信而用之,高遵裕亦是陛下亲自点选,而韩冈更是陛下简拔于草莽之间。熙河诸将官,皆是靠了陛下的识人之明.何预臣事?陛下之赞,臣愧不敢当。”

赵顼微微翘起了嘴角,王安石的话正说到了他的得意之处。王韶是他看了《平戎策》之后,一手提拔起来的。高遵裕也是他给王韶钦点的副手,而韩冈更是他亲自授以差遣,不然,尚未弱冠的少年人又怎么有资格去边地立功。

不过之前王安石对熙河的一力支持,还有新法对于开边之事的帮助,赵顼都看在眼里,“没有王卿,岂有今日之胜?!相公不必再推脱了。”

大宋天子一时兴起,就从腰间解下了随身所系的白玉腰带。极细的金线编织成的的腰带外侧,镶着一片片椭圆形的羊脂白玉。浮雕出五爪天龙的金质钩环上,镶着一粒粒宝石珍珠。单是做工,就价值千金。而其中代表的意义,更是重如千钧。

赵顼拿着玉带递给了身边的李舜举,“就将此带赐予相公。”

王安石连忙跪倒推辞。这份赏赐实在太重。天子亲佩的御带,岂是臣子能用的?

但赵顼正在兴头上,根本阻止不得。王安石三番四次的推脱,但赵顼是五次六次的强要王安石接下。

最后王安石推辞不掉,放在跪谢之后,勉强接受的此带。

看着王安石腰环玉带的模样,赵顼满意度点了点头,“日后上朝时相公定要佩上此带。相公有了玉带……还有王韶,还有高遵裕……恩,还有韩冈!”

“王韶、高遵裕领军追击木征后,没有韩冈主持,莫说河州,就是熙州都能沦陷了。撤兵的诏令,换作胆小畏事的,也怕就当场接下了。那样王韶连回来的路都没了,哪还会有今日的大捷?”

自言自语了一阵,赵顼站起身,“今天到此为止,都各自归班吧!”

说完,他转从殿后离开。他急着要回崇政殿,去看送到他御案前的捷报。

众臣恭送了天子离去,从吴充开始,一名名大臣都过来向王安石表示自己的恭贺之意。王珪和冯京脸上都挂着笑容,也跟在吴充之后,上来恭喜过得到御赐玉带的王安石。

一番纷扰之后,王安石当先离开,他也要去崇政殿与天子商议如何处理河湟的捷报,其他朝臣也陆续离开了今日朝会一波三折的文德殿。从皇帝到小臣,好像所有人都忘记了殿上还有一个唐坰在站着。

章惇出殿之前回头一望陛前孤零零的声影,前面逼得当朝宰相下不了台的殿中侍御史,现在却轮到他自己下不了台了。看着倒是痛快,但要是唐坰这厮羞恼之下,一头撞向庭柱,那可就有些败人兴。他在门口停了一下脚步,提醒一句站在门边的御史中丞邓绾。

跨步出殿,从阴暗的殿中,走到炽烈的阳光下。眼睛一时适应不了阳光,而章惇心中的感觉,也觉得好像今次在文德殿中呆了很久很久。他一生几十年的经历,说到峰回路转、出人意料,当以今日之事为最。

前面王珪正慢慢向崇政殿走去,口中的喃喃自语,竟随着风飘进了章惇的耳朵里:“时也,命也。”

章惇双眼眯了起来:‘这是何意?’

