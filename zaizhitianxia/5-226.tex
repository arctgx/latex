\section{第25章 晚来萧萧雨兼风(中)}

虽然在王珪念出司马二字时就已经想到了会是这个名字,但听到了赵顼点出了西京留守、判西京御史台的全名之后,向皇后还是不敢相信自己的耳朵。

“官家,可是要招司马光入京?!”她凑近了赵顼耳边,声音中隐隐透着心中的惶急。

赵顼眨了两下眼,没有一点拖泥带水,给了皇后肯定的答复。

向皇后攥着汗巾不说话了。

不仅是韩冈,或是向皇后,相信王珪、薛向他们,都会觉得赵顼肯定会找王安石入宫,甚至第三度宣麻拜相,托孤于他——王安石能在郊祀大典前赶到京城,不论是什么原因将他从金陵城招来,在世人看来,可以说是冥冥中自有天意。顺天应人,这应该是常理。但赵顼偏偏选择了司马光。

薛向从牙缝里挤出的声音微不可闻,只有站在侧后方,又闭着眼睛的韩冈听见了,“异论……”

异论相搅?

不过韩冈不这么认为,都这时候,还玩什么帝王心术?

赵顼病得连话都说不出来,以中风的普遍情况,他这样子一年半载都拖不过去。既然能清醒到召回司马光,就不会自大得认为自己能牵制住高太后。

要异论相搅,也要皇帝或是垂帘听政的太后有这个手腕才行。难道赵顼有自信拖着病体施展权术,还是说他相信他的母亲能有执中而行的政治头脑。

高太后对新党成见极深,这件事朝臣们人人皆知。她一旦上台,又有旧党在朝,那么当旧党攻击新党的时候,她会偏向哪一边?而旧党攻击新党的理由,自然是拿着新法施行中的弊端说事。

党同伐异,就算新法做得好的地方,旧党也不会承认。因人废事的场面,千年后有,此时当然也有。不是韩冈小瞧人,兼容并蓄的胸襟,不是什么人都能有的。

不对。

韩冈心中一动,睁开眼,眼角的余光左撇右撇,看看高太后,再看看雍王,脸色都难看得紧。

能身列两府,就算没有才干,政治眼光不会缺少。而薛向,不但才干不缺,论起嗅觉和眼光,韩冈并不认为自己能胜过他。高太后和雍王都是当事人,他们的感觉也应该不会错。

思路转了个弯。

韩冈算是明白了,自己的思路果然是钻进了牛角尖。

的确是异论相搅。

大概在赵顼看来,王安石压不住高太后,即便王安石压得住高太后,但后宫是在高太后手中,作为外臣的王安石,保不住赵佣。

既然如此,新法也好,旧法也好,最后搅成什么样,现在的皇帝都不在乎,只要保住儿子。

“陛下,可是要由中书门下下堂札?”王珪问道。

由政事堂下文调司马光进京,声势会小一点。这也是在试探赵顼的心意,到底是怎么一个想法。

韩冈集中了注意力,再一次盯住赵顼的眼皮。

去声。

十八啸。

诏。

诏书。

是要以诏书来招司马光进京。

韩冈抬头向上,长长的呼了一口气,郁结在心的愤懑却怎么吐不出来。

站在不同的位置,看问题的角度便截然不同,得出的答案也绝不一样。眼前的这一幕,就是又一次绝好的证明。

旧党要上台了。

新法危在旦夕。

吕公著虽是做了几年的枢密使,但他的作用仅仅是掺和而已,不让新党独据朝堂,国是依然是新法。这一点,从来没有变动过。

可旧党赤帜司马光被招入京城,还是天子清醒后的第一封诏书,近乎遗诏托孤的态度来对待旧党,那么新法和旧法之间的交锋将不可避免。

何况还有高太后在。

当然,这也等于是断了太后示恩旧党的机会,贬去旧党的是赵顼,现在重新启用他们的还是赵顼,而且以托孤的形势,不愁他们不为赵佣卖命,而不至于将感激和忠诚献给太后。

皇帝这是宁可放手让朝堂乱起来,也要力保延安郡王的安稳。

只是世间明眼人所在多有,司马光更是其中的佼佼者,能有几分机会让他入彀?一成,还是半成,甚至可能会更低。

不过,赵顼的做法,其实已经钳制住了旧党。

因为世人只会看到赵顼托孤的举动,不会去深思其中的用心,也不可能有机会了解。这是用士林和民心来压迫司马光等一众旧党,让他们不敢逾越雷池一步。

旧党可都是自命君子啊……他们敢不要脸吗?

先伤己,再伤敌,钳制上下,好狠的一招。

“翰林不在这里。”高太后抬头问王珪道:“玉堂那边今夜有谁留守?”

王珪停了一下,偷眼先看了赵顼一眼,这才低下头去,“回太后,是张璪。”

高太后点起身边的亲信内侍,“陈衍,去宣张璪来福宁殿。”

陈衍立刻领旨离开了——垂帘听政的太后的谕旨,是可以叫做圣旨的。有慈圣光献曹后的旧例在,招翰林学士夜入福宁殿那是一点问题都没有。只是皇后绕在手上的汗巾,又被缠紧了一圈。

今晚的赵顼似乎精神很好,努力的要将所有的事都安排妥当。当陈衍离开,他又开始眨起眼睛。王珪翻着韵书,一个字一个字翻译,声音却渐渐不成语调。

司马光。

吕公著。

为师保。

赵顼艰难的眨着眼睛,用了半刻钟,将九个字的圣谕传递出来。

韩冈掌心中满是汗水,之前的猜测居然还是有错。

不是留着新党和旧党在朝中厮杀,而是毫不犹豫的选择了旧党,站在了旧党的一边。

“官家,要以司马光和吕公著为师保?!”

高太后的声音尖利,听起来却让人感觉隐藏着几许怒意。可惜韩冈从侧面看不清高太后的表情,不过雍王脸色的变化,在韩冈的角度,却能尽收眼底。有那么一瞬,一直都用余光关注着他的韩冈,在赵颢的脸上,发现了一闪即逝的冷笑。

赵顼的眼皮眨了两下。

没有多,没有少,依然稳定。

这是在作交易,或者说,是妥协。跟太后做交易,向太后妥协。

韩冈都开始佩服起赵顼了。壮士断腕的刚烈,竟然在从来没有吃过苦的皇帝身上见到了。毕生的心血和成果,轻而易举的便放弃。这份狠决,韩冈真的没有见过几人做到过。

赵佣的年纪太小了,又没有其他兄弟,一旦他出了事,赵颢必然接位——有东汉旧事在前,不可能幼主夭折之后,再立一幼主,朝堂上下都会有忌讳。

所以赵顼才要想太后妥协,让高太后折腾就折腾朝堂,新法施行了这么多年,在地方上根深蒂固,旧法想要推行,只会一个麻烦接一个麻烦,到最后,高太后也不会有太多的精力来跟他的儿子过不去了。

反正高太后上台后有七八成的可能在旧党的帮助下,清光朝堂上的新党,更是会毫不犹豫的废除新法。既然如此,还不如就先卖个好,不要给太后留下麻烦。

等几年一过,赵佣成人,那就没有太后的事了。那时候,再恢复新法也不为难事。看起来是妥协退让,甚至是服输,但还是为了将来东山再起。

母子之间,算计到这一步,也难怪高太后会变了声音,而赵颢的冷笑也就能理解了——赵顼没有考虑到他母亲的性格啊。

韩冈再去看王珪和薛向,已经是变得面无表情的两人,看起来一样也都了然于心了。

不过有一点让韩冈觉得纳闷,他和两位宰执能想得通透,是因为他们在朝堂上的经验。但高太后能想明白,以她过去表现出来的性格,却让人觉得应该不可能想得透。何况她今晚还留了儿子在宫中,换作是曹太皇在她的位置上,决不至于这么做。

那么,只有一个可能了。高太后现在已经是将自己放到了垂帘听政的位置上,那么从这一角度去思考问题,而且还是从结果上逆推原因,就不那么难了。另一方面,赵顼毕竟是儿子,做事和思考方式的规律,做母亲的想明白不是难事。

赵顼闭上眼睛,看起来在翰林学士入觐前,并没有更多的吩咐了。该说的都已经说了,该表明的都已经表明了,十几年的心血,在今夜被他完全放弃,视若敝履一般的丢到了一旁去。

在儿子继承皇位,和毕生的心血之间,赵顼毫不犹豫的选择了前者。将他赵顼的血脉传下去,这样新法才有未来。

想得明白,做得更是痛快。

第一次,韩冈佩服起赵顼的手段,但他还是无法接受。

“太后,官家,张璪已奉旨在殿外听宣。”陈衍匆匆进殿,向着太后跪倒。

高太后提声道:“宣其入殿。”

陈衍立刻起身回头,提声道:“宣张璪进殿。”

当高太后开始垂帘听政,那么赵顼再也没有一言九鼎的权力了。

不出意料,韩冈在赵顼的双眼中找到了一丝失落,除非他能重新开口说话,而且要清楚、流利,否则,权力将不会回到他的手中。

以眼下的状态,赵顼的政治生命,正在渐渐终结。当内禅诏书下达之后,作为统御天下的天子,才三十出头的赵顼,将不再存在。

