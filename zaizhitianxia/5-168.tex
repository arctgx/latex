\section{第18章 向来问道渺多岐(五)}

韩冈编药典,果然还是为了气学。

虽说是事先有所预料,但当真确认了消息之后,杨时还是免不了要重重的叹上一口气。

韩冈在药典中推介自出机杼的分类学,指正经书中的谬误。这件事在儒林之中,掀起了一阵惊涛骇浪,,兴起的风波,并没有局限在东京城中,很快便向四面八方传播开去。

洛阳距离东京极近,得到消息也就隔了区区三五日,对早就在推测韩冈用心的程门弟子来说,正好是映证了他们之前的猜测。

“昨日伯淳先生赴司马君实之邀,据说正是为了此事,”谢良佐道,“不知道伯淳先生和司马君实准备怎么驳斥韩冈的谬论。”

“不能与韩冈就这些细节论辩。”杨时摇头道,“真要与他辩驳,那就落入他陷阱了。”

“不知中立何出此言?”谢良佐疑惑的问着。

“敢问显道,气学中的关节是什么?”杨时反问。

“……若是韩冈的这一脉,当是格物致知。”

气学之中也有分歧,韩冈这一脉与转投而来的吕大临便不是一个路数。程门弟子对气学之中的纷争,看得也是比较清楚的。

“没错,正是格物致知。韩冈一直主张的格物致知,与伯淳、正叔两位先生所言相异,专注于自然中的细微之物,由小证大,道自器中出。虽说格局小了,但其胜在浅近,能让世人一试便知。如此一来,你当真要与其辩驳,就必须在器上取证据,否则就无法取信于世人。”

“……也不尽然。”想了一阵后,谢良佐摇摇头,“自汉至唐,经书释义本多歧,到如今都是各说各的。要想分出个是非对错,要么从细微出来,要么就是放大了看。在细枝末节上,小弟承认的确是无法跟韩冈相比的。但韩冈一条一条的考订诸经中的每一句话,其实还是落入了汉唐诸儒章句之学的窠臼,只是看着手法有些不同而已,本质是一样的。”

“显道不是明白了这个道理吗?就是因为这样才难办啊。”杨时笑了起来,“经典千年传承,经多人传抄,加之年久散逸,总有错漏之处,韩冈就是盯着这些错处做文章。加之先儒以己意解圣人之论,也是多有错处,今儒对此说得太多了,韩冈从中着手,也是想以此来宣扬他的气学。”

“但论经书,须观其大略,察其要旨,寻究章句,并非正道!”

“自是如此。可世人眼光短浅者甚多,有几个能一眼看到圣人的本意,经书中的要旨?

谢良佐长声一叹,一时不知该说什么才好。

韩冈直叱《礼记》触犯了程学的逆鳞。《礼记》中的《大学》、《中庸》两篇,是程颢程颐都很看重的篇章,其代表的是孔子、曾参、子思和孟子这一脉道统传承。程学一贯以承袭孟子道统为目的,韩冈如此作为,等于是在柴禾堆里丢了一把火,不烧起来才有鬼。

虽然气学也在说《中庸》,而格物致知四个字更是出自于《大学》之中。韩冈攻击《礼记》,看似是自伐根基,但杨时和谢良佐都清楚,韩冈这次对《礼记》下手,本质上还是在争夺道统,并不是自己跟自己过不去。

《大学》和《中庸》虽是出自曾参和子思,可将之收集起来,编订入《礼记》的,却是西汉的戴圣。韩冈的目的就是将《小戴礼记》中属于戴圣的部分给剔除出去,明了圣贤之本心,或者更确切点的说,一旦他向世人证明了《礼记》中的错误,并将之归咎于戴圣,那么接下来,他就可以将这本经书中所有不合气学要旨的章节和条目,说成是戴圣篡改,直接删改了事。

“经文也好,注疏也好,到时候是去是留,全都得看他的喜好了。”杨时音调沉沉的说着。韩冈的胆魄和手段,实在是让他惊惧。如此行事,可见韩冈根本就没有将先儒放在眼里,一切以自我为中心。

谢良佐摇着头:“都说要一改汉唐旧风,但做到韩冈这样的,可是少见。就是王介甫,都不至于如此肆无忌惮。”

杨时冷冷笑了起来:“王介甫一直都在说要‘一道德,同风俗’,使学者归一。他也不可能忍得下韩冈对《礼记》起干戈。”

‘一道德,同风俗’这六个字出自于《礼记·王制》,王安石曾经说过《礼记》中多有后人伪作的篇章,但一道德的理论来源,依然要从被列入九经之一的《礼记》中取得。

韩冈想控制《礼记》的诠释权,甚至删改权,当然也是新学所不能忍受的。何况还有《诗经》一事,也是新学和气学矛盾的焦点。杨时在新学上的水平最高,不过他对新学的钻研,目的还是在其中寻找错处,了解的越深,批评起来当然也越能一针见血。

只是眼下的当务之急,已经从新学变成了韩冈的气学。釜底抽薪,不外如是。

杨时的一对眸子变得越发的深沉:“一旦给韩冈将药典编成,不知会捡拾出《诗经》和《礼记》中的多少错漏出来。正叔先生如今一改旧意,与伯淳先生同修《易传》,便是为了防着日后。”

程颐过去曾经与杨时说过,不要写书,易分心,自是于道有害——‘勿好著书,著书则多言,多言则害道’。但如今气学与新学的纷争愈演愈烈,王安石和韩冈翁婿二人为道统相争,程学也不能在外旁观,儒门道统,那是绝不能让人的。

“张横渠旧年亦曾说,正叔先生‘深明易道,吾所弗及’。《周易》乃是儒门根本,顺性命之理,通幽明之故,尽事物之情,而示开物成务之道也。韩冈所学浅近,在《周易》上,可是差了许多。”

儒门一切的根基,还是在《周易》上。文王拘而演《周易》,八卦出自伏羲,但《周易》中的六十四卦的卦辞和三百八十四爻的爻辞,则是出自文王和周公——‘西伯盖即位五十年。其囚前里,盖益易之八卦为六十四卦’。

而文王和周公,是儒门崇拜的圣人,孔子最敬的便是周公,其为《易》作传,得《十翼》。所以《周易》其实是分成《易经》和《易传》两个部分,但不论哪个部分,都是儒门的根源。对圣人本意的疑问,都能在《周易》中得到解释。

“穷天地之理,明圣人之道。皆在《易》中,待先生《易传》一成,世间诸多异论,便一无所惧!”

……………………

韩冈最近很忙,《本草纲目》的编修他当然不能放手,但厚生司和太医局中的事务,同样也不能丢下不管。

京城中,一东一西的两座医院——本来从疗养院改称医馆,是为了能有所区分,不过又容易和驿馆、医官这两个词相混淆,干脆就改名成了医院,这样对韩冈来说也方便——已经修建得差不多了。

门诊部、住院部、药房,一众屋舍都已经整修完工,并粉刷一新,里面的医疗设施,也全都准备妥当,连备用的物品,也都分门别类的存放在库房之中,并登记造册。

医院中的人手也准备得差不多了,医官和医生总计七十八名,其中的十几名医官,将会在两座医院中轮班出诊,而剩下的医生——太医局生——则是被一分为二,各自常驻在其中的一座医院中。从之前的疗养院继承下来的护工,有一百六十余人,也是按照医院的不同分成两个部分。还有医院中的杂务,

为了更好地管理医院,韩冈从厚生司中挑选了两名老成的官员,在医院中担任院长。不过又为此订立了条例,只有对医院的庶务和财务有管辖之权。人事权,确切的说翰林医官、太医局生们的任免权,还是在天子和太医局手中。

这一个是为了防止日后医院成了院长一言堂,另一方面,也是免得与太医局和翰林医官院起纷争,韩冈虽是提举厚生司,但也兼提举太医局,一碗水要端平才是。

在门诊部的大门处,连医官门诊的牌子也挂了出来,虽然每十天才会在医院中坐诊一天,但给天子和皇亲国戚治病的御医们将会出诊的消息早就传遍了京城,每天都有人上门来询问医院何时才会开张,都想让翰林医官们来看病。就算排不到御医的门诊号,让太医局生来医治,也比江湖游医要强。

如今东京城中多少病家,就等九月初一,医院正式开张了。

不过这些事,对韩冈来说,也只是日常的工作而已。真正挂在他心上的,这些事还算不上。

韩冈很清楚,随着自己图穷匕见,儒门各家学派,必然会有所反应。如果仅仅是争于学术,韩冈倒不担心,他能应付得了。但反击的手段,可不一定会局限在学术上。

马融遣家奴追杀弟子郑玄的典故,可是彩虹流传。隋唐时的大儒孔颖达,也同样遭受过同列宿儒的刺杀。眼下虽不至于会有人敢去刺杀韩冈,但其他的手段一样能给气学当头一棒。

“千里镜流传于世,致人窥探**,又有私习天文者,干犯朝廷禁令……三哥,你难道准备为此上本?”

“不是愚兄。”冲着蔡卞,蔡京微微一笑,“是张商英。”

