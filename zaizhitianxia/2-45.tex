\section{第11章 五月鸣蜩闻羌曲(八)}

【第三更,求红票,收藏。又迟了一点,真不好意思】

王启年战战兢兢的跪着,头也不敢稍抬。可背上依然传来一阵沉甸甸的压力,被秦凤路兵马副都总管盯着,就像有一块千钧巨石压着,让他连呼吸都艰难了起来。

见着王启年心惊胆战的模样,窦舜卿则是益发的不信给自己家的七哥出主意的会是这样胆小如鼠的小人物,他身后肯定是有指使者!

窦舜卿慢吞吞的喝着茶,让王启年跪了好一阵。他才放下茶盏,慢悠悠的说道:“你倒是好胆!”

王启年将脸贴在地板上,连声说道:“小人不敢,小人不敢。”

王启年的胆子有时大,有时小,端得要看情况和面对的是谁。在对百姓敲骨伐髓、以及钻官府空子的事情上,他是胆大包天,而在动动手指头就能送他归西,而且根本不须担心罪名的窦舜卿面前,王启年则是胆怯如鸡。

不过到了这时候,他还是不明白,窦舜卿找他究竟为了什么?

今早他去衙门时,被龙干桥边的郭铁嘴叫住,说他今天印堂发黑,必有灾厄。王启年听了,就一脚踹翻了算命摊。但现在他后悔了,早知有这档子事,就该耐下性子问问该怎么禳解才是。

“你给我家七哥出的倒是个好主意。”窦舜卿的声音依旧慢吞吞的,却说得王启年一愣,难道是为他前日为窦解出谋划策,对付韩冈的事?

窦副总管说完上面两句,猛然间一拍桌,怒声喝问:“说……究竟是谁指派你来的?!李师中还是向宝?!”

王启年几乎被吓破了胆。哪有什么人指派!

窦七衙内看韩冈不顺眼,自己不愿动手,却找他这等小人物作伐。王启年也不愿动手,但窦七衙内总是催他,最后他被逼得实在没办法,正好看到被关入狱中的党项郎中,还有去大狱探他的仇一闻,顺便又联想起韩冈和仇一闻之间的关系,才随口出了个主意。

“没有,没人指派小人。全是小人自个儿想出来的。”王启年头摇得跟拨浪鼓一般,若是说他出的计策是受人指使,那他接近窦解就是别有用心,心怀鬼胎,而不是单纯的出了个馊主意,保不准窦舜卿或是窦解就会因此杀他泄愤。

“你认为本帅会信?”窦舜卿冷笑一声,又提醒王启年,“别随口说一个人出来,现在还跟王韶过不去的,城里可就那么几个。”

王启年头脑都乱成了一团浆糊,这到底什么跟什么啊?真是冤枉。没有别的选择,他也不敢冒险,“小人出得馊主意,实在该死。但要说小人受人指派诓骗七衙内,小人也没那个胆子。”

说完,便砰砰砰的磕着响头,为救自己小命,他磕得煞是诚心,没两下,脑门上就见了红。

窦舜卿眼皮也不动一下,不论王启年怎么推脱,他其实已经认定他是受人指派,而且必然是李师中和向宝中的一人。不过既然王启年是李师中或是向宝的手下,就不好做得太过分,要不然,以窦舜卿的脾气,直接把王启年给杖毙在堂下。

“算了,本帅也不逼你了。”窦舜卿送了口,“本帅只问你一句话,是不是李师中?”

王启年猛摇头,这罪名,他怎么也不敢栽到李师中的头上。

窦舜卿坐了回去,仰头看着顶上的房梁,“原来是向宝啊……难怪。”声音越来越低。

而王启年却是越发的心惊肉跳,

怎么都给认定了?难道今天当真要归位。

……………………

半个时辰后,王启年晃晃悠悠的从窦府里被赶了出来。走出窦府大门,市井喧闹伴随着热浪迎面而来,让他明白自己还活着。不过连王启年他自己,都弄不清为什么窦副总管没有杀他,而且还赏了他一饼银子。怕不有三四两中,拿去金银铺中,好歹能换回十足贯的大钱。

抬手摸了摸脖子,还是完整的。王启年长舒了一口气,虽然今次吃了一番惊吓,而且到现在还是糊里糊涂,但在窦舜卿面前混了个脸熟,又得了赏赐,好歹也算是靠山了。这番惊吓,吃得也不算亏本。

“王大哥!王大哥!”

王启年出了窦府所在的大街,正要回自己家去,却听到身后有人叫自己。回头一看,却是在成纪县衙中做事的王五。算是熟人,却没什么交情,而且听说他还是因为韩冈才被调到县衙中做事的,王启年现在还不想跟他打交道。

不过王五转眼间已经跑到他的面前,王启年也只能堆起笑脸:“怎么是王五兄弟,今天不用当值吗?”

王五却不听王启年在问什么,拉起他的手:“今天有贵人在前面请王大哥,还请王大哥赏脸。”

“什么贵人?”

“王大哥去了就知道了。”王五说着,就硬拉王启年往路边的一家酒店走。

没头没脑的王启年怎敢去,跺着脚往后退,却有撞到一人,回头一看,却是他更熟悉的王九。

王九上来架住王启年,笑着道:“王大兄弟,还是去了再说。”

王启年几乎是被两人押解进了酒店。夏日的午后,小酒店中生意并不好,只有一桌有人。他看过去,两个站着的伴当,也是成纪县衙的衙役,而且还是同族兄弟——周宁和周凤。客位上的是机宜王韶的随从杨英,而坐在主位上的却是他熟得不能再熟悉的人:

“韩抚勾!”王启年惊道。

刚才还在窦舜卿府中说起韩冈,自己又是出了要害他的主意。现在见到本人,心中免不了就有些发虚。但一想到自家身后已经有了窦舜卿这座三山五岳一般的硬靠山,他的胆气就壮了很多。

王启年主动上前行礼:“不知韩抚勾唤小人过来,究竟是有何训示?”

“究竟是为了什么,王启年,你自己心中应该最清楚!至少不是请你喝酒来着。”韩冈说得很直接,听到王启年被叫入窦府,他没心思再云山雾绕的试探。

“看抚勾说得,小人还真是不清楚。”

王启年抬起头,毫不退让的跟韩冈对瞪着。他在窦舜卿面前吓得瑟瑟而斗,那是因为小命给人攥在手上,但从九品可不像窦舜卿那样,杖死吏员也可以若无其事。

韩冈虽然凶名外著,但在光天化日之下,又是在酒店中,他也没什么好怕的。真的有事,躲到窦府里去就行了,何况这个灌园小儿又没几天好蹦达了。

韩冈看着王启年胆气甚壮的模样,心中一片雪亮。他冷笑着,右手搭在桌上,中指轻轻的扣着,哒哒的单调声响中,他缓缓说道:“西门李成衣家产争夺案;刘十五杀人案;宗孝坊纵火案;熙宁元年元月雪灾所耗赈灾款项的账簿……王启年,这些年你把架阁库中的卷宗卖掉了多少,烧掉了多少,又瞒下了多少,要不要我一件件的数给你?”

王启年听着韩冈一件件的数着他过去做下的好事,听到一件,身子便抖上一下,脸色也是灰白了下去。心中一阵发慌,灌园小儿什么时候把这些事给翻出来了?只是听到最后,他却不抖了,笑了起来:“这些事牵扯甚多,抚勾你还是要慎重啊。”

“所以当本官把这些事揭开来时,你多半会在狱中被个土口袋压上个一夜半夜,上不了公堂。”

王启年摇头,摇得很慢,却很坚定:“小人什么都不知道!”

“窦舜卿保不了你。”韩冈瞪着王启年,冰冷的说着。见着王启年不为所动,表情遂软了下来,摇头叹道:“算了。本官知道你嘴上有门闩,什么都不会说的。”

王启年闻言,笑意便爬上了脸,冲着韩冈作揖:“那小人可以走了吗?”

“走?”韩冈脸色一冷,喝道:“架住他!”

王启年还没反应过来,旁边的四个县衙衙役一起动手,将他牢牢架住。虽然不是专管捕盗的快手,但王五他们也颇学了两招,摁住手脚,让王启年一动也动不得。

“韩冈,你这是做什么?!”王启年脸色煞白,用力挣了又挣,连礼节也不顾了。心中发慌,难道郭铁嘴今早说得灾厄,是印证在现在,而不是窦府中。

“既然你嘴上不肯说,我直接问你的心好了。”韩冈走到王启年身边,盯着他慌张的眼神:“你知道吗,平常的时候,心跳脉搏都是很平缓的。不过一旦说谎,心跳就会快上一点,而脉搏也会变化。嘴能说谎,但心却是说不了慌。”

王启年心慌了,嘴却是硬着:“胡说八道。”

韩冈伸手搭上王启年的右腕,“本官可是不是在胡说,你忘了我是什么身份?”

王启年的脸色变了,连旁边的几个人都是一副恍然的模样,“原来如此!”杨英在旁边点着头。

韩冈三根手指搭在王启年的手腕上,做着把脉的动作,开始提问:“昨天你见过窦七衙内没有?”

“有又如何?!”王启年厉声瞪眼。

“不要说话!”韩冈一皱眉,“我只问你的心就够了。”他又对王九道,“如果他再乱叫,就堵上他的嘴。”

王九点头应了,韩冈再次发问:“方才你是不是见了窦副总管?”

王启年扭过头,不搭理。

韩冈却不管他,仍是一个问题接一个问题的问着,都是些寻常问题,有的他心中有答案,有的他也不知道答案。

王启年一直闭口不言,问题听得多了,身体和神经也渐渐松懈下来。韩冈看在眼里,眼神突的一变,唯一要问的问题厉声问出了口,“利用关在大狱的那位郎中来害我,窦副总管已经打定主意了吧?!”

王启年身子猛然一颤。他这一动,不但是韩冈,连其他人都知道了真相了。

“好狗胆!”杨英拍案大骂。王五周宁他们手上也是一阵用力,勒得王启年龇牙咧嘴。

“看来是真的了。”韩冈嘿嘿冷笑。

