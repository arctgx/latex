\section{第26章 任官古渡西(四)}

接风宴上,前任知县凌庄拼死拼活的将上首位强按着韩冈做了。又带着县丞、县尉,殷勤的劝着他的酒。一场宴席下来,对韩冈表现得比亲娘老子还恭敬。而韩冈的三名幕僚,也一样被请到了堂中的席上坐下,好生的接受了一夜的招待。

到了三更天,方才回到驿馆。

进了房中,原本看着有些醉意的韩冈一下变得清醒起来,双眼清亮有神,与三名幕僚坐下来,喝着下面端上来的茶。

方兴坐下后就摇着折扇冷笑起来:“礼下于人,必有所求。凌知县今日的一番作态,看起来不像是奉承正言的样子,多半帐目上有些问题,心里虚着。”

游醇冷笑一声:“但凡作奸犯科之人,哪有不心虚的情况!”

“就看正言是否要一查到底了?”魏平真问着韩冈。

如果库中亏空严重,跟帐目对不上号,谁也不会蠢到接手。若是糊里糊涂接下来,到了转运使司来人查帐的时候,哭都来不及。拖上几日不交接,若在地方,州中就要派人下来了。白马县属于京城,开封府一旦派人来,事情可就更为麻烦了。

大宋官吏多有贪腐之辈,官库也是亏空的居多,但即便如此,世间极少有新任官员不肯接任的情况出现。基本上在交接之前,官员都会将帐目作平掉,相信凌庄下面也有人来处理帐册。不管是用帐目合库存,还是用库存来合帐目,只要两样能对得起来,韩冈就没打算追根究底的打算。

被三人一起盯着,韩冈啜了两口没什么滋味的茶水,抬头道:“只要帐目对得上就可以了。”

方兴、魏平真心领神会的微笑点头,但游醇却是在迟疑着。

韩冈看了游醇一眼,便多解释了一句,“真要穷究到底。保不准库房就要被放把火。里面都是民脂民膏,被烧掉后,苦得还是百姓。”

一宿无话。

抵达白马县的第二天,韩冈婉辞了县丞县尉的盛情邀请,与方兴、游醇在驿馆中聊着天。而精于帐目的魏平真,则带着韩冈家里的帐房去库中对账。

魏平真查得很仔细,便民贷的存底都一张张的对着数字,凌庄则派了人过来打下手,领着几个胥吏端茶递水。可到了中午,正要吃饭的时候,魏平真却将帐册一推,“天色已晚,明天再来看看。”

说完也不收拾桌子,就和帐房一起直接起身掉头离开。

虽然外面的日头正在正南方的天顶上挂着,但凌庄的幕宾和几名胥吏都不敢拦着他们。送了魏平真两人离开,回头来一看,几本帐册摊开来的页面上,都是做过手脚,却没有将尾巴收拾干净的。虽然很隐晦,但破绽就是破绽。

凌庄和诸立各自接到通知之后,顿时明白了韩冈的心意——要么将亏空给补上,要么就快点将这本帐给做圆了。

韩冈的态度算是很好了,但凌庄却是心头有火。那点错处,在一般的检查下只会被忽略过去,没人会计较的。但一旦叫了真,要弥补起来却很麻烦,不是在账本上改个数字就可以的,官库那边也要补上差额,少说也要近万贯。说起来,要不是差得太多,当初直接就将亏空补上了,也不会留下什么破绽。

对韩冈的审核严苛,他恨得牙痒痒的。一万贯,说多不多,说少不少,放火烧屋不值当,还不一定能成,但给出去又是肉疼。想着没办法,过来陪小心,试探着韩冈的心意,“正言年少有为,少待时日,必可至公卿……”

韩冈笑容淳和:“韩冈能以弱冠之龄,屡见拔擢,这都是天子的恩德。韩冈粉身碎骨亦是难报啊……”

凌庄没话可说了,韩冈的这段拒绝没有给他留下任何空子,根本就不容他将重要的话说出口。看来是用钱收买不来了。想想也是,才二十多岁的朝官,又得天子看重,绝不会为了点钱财,而坏了自己的名声。

东拉西扯的说了阵废话,起身告辞离开。凌庄阴沉着脸出来,回头冲着驿馆冷笑:“现钟打不了,不信边鼓都没得敲!”

凌知县这番发狠的结果,当天晚间韩冈就已经知道了——他竟是遣人悄悄的给韩冈的三名幕僚都送去了一份礼。

“他们都收下了?”韩冈问着来报信的伴当。

伴当摇摇头,“游先生没有收,但方先生和魏先生都收了。”

“我知道了!”韩冈没有生气。

都是读了十几年、几十年圣贤书的,不去考进士而来给自家做幕僚,难道是为国为民?笑话!一个是挣钱糊口,另外就是早一步进入官场与人结交,日后好被推荐为官。

既然要靠着这些幕僚来做事,韩冈能堵着不让他们收钱吗?按着如今的规矩,幕僚们只要不越线,都得睁一只眼闭一只眼。韩冈也只希望他们明白谁是他们的东主,不要光想着拿钱,却把最为重要的关键给忘了——并不是没有幕僚为一己之私害了东主的故事。

不过韩冈更为清楚,只要自己不懈怠,凡事盯紧一点,就不虞一干幕僚坏了自家的名声。他可不会那等知会写诗作文的士子,可以任人欺瞒,在衙门中用心做了三年实务,经历的则更多,有什么情弊他不知道的。。

韩冈只可惜自家亲戚少,能派得上用场的两人,一个在荆湖战场已是威名煊赫的青年名将,另一个则是执掌着一家在关西很有些名气的商号。若身边有一两个得力的亲眷,有些事让他们来做,比起用着外人更为可靠。内外相制才是御下之道,韩冈当然不会蠢到任人唯亲,但也不会觉得在有着亲亲相隐的这一条法律的宋代,外人会比自家人更为自己着想。

到了吃饭的时候,三名幕僚都过来韩冈这边。

一进门,方兴就拱着手:“承蒙正言匡助,方兴今日可是发了一笔横财。”

方仲永的族弟很是洒脱,一点也不遮掩自己收了前任知县贿赂的事情。

魏平真也跟着笑道:“两锭三十两重金花银,凌知县可真是大方。”

在市面上,金银并不能当作钱钞来使用,必须要通过金银铺来兑换成钱币。但用来送礼,却是比沉重的铜钱更为多见。只是现如今的银价,一两能抵一千七八百文,以七百八十文一贯来算,也不过是两贯半。三个六十两,加起来连五百贯都不到,相比起万贯的亏空,凌庄的确是够‘大方’的。

这点小钱,方兴、魏平真不屑归不屑,但都很干脆的收下了。既然韩冈没有将凌庄赶尽杀绝的想法,那他们将贿赂收下,其实也是在安凌庄的心,正符合韩冈的心意。

不过,这等曲里拐弯的想法,程颢门下的游醇却没有:“怎么可以这样?!”

“节夫,其实不妨事的。”韩冈连忙道,他可不想自己的三位幕僚变成互相拆台的情况,“凌庄既然送来,就可以径自收下。我本无意刁难,但不便直说,你们收下才能让他安心。何况只是普通的人情往来,与公事无关,何须挂怀于心?”

游醇却摇着头,一脸不以为然。只是见韩冈如此说,才不再多言。

他对韩冈很是敬重,并不是因为韩冈的官位,而是韩冈的为人。在洛阳时,听说韩冈去岁上京应考,为了求见程颐程颢,竟在程家门前的雪地中站了一个多时辰,这件事,已经在洛阳城中传遍了。名满天下的韩玉昆,还能如此尊师重道,实在是让游醇敬佩不已。

一起吃饭的时候,方兴和魏平真似是毫无芥蒂,但韩冈知道,他们跟游醇肯定是合不来了。

等到夜中,韩冈招来亲信伴当,吩咐着,“天气冷了,从箱子里拿四匹棉布、二十两银子,给三位先生送去,让他们换身冬衣。另外再给游先生多送六十两银子去,说是我的一番心意。”

一口气送了数百贯出去,韩冈却没有多少心疼,这是应该做的人情,总不能眼睁睁地看着幕僚收了贿赂,自己却不做一点表示。尤其是游醇,虽然不通人情,但这番正直的做法,更是得加以奖励。

伴当应声出去了。韩冈坐在书桌前,考虑着该怎么安排自己的这几个幕僚。

魏平真年纪大了,对钱财看得重,但为人老成,做事稳妥,经验更是丰富,日后可以多多依仗。

游醇年纪与自己相当,又少经历,真要做幕僚,其实排不上用场。不过他的学问还可以,文名更是与他的弟弟游酢一起,在少年时就传遍乡中。可以推荐他去做县里学官,如今王安石兴学校,州里县里都建有学校,可以安排游醇教导白马县的士子,想必他也愿意。

至于方兴,治政上的能力暂时没见到,可诗文水准不错——能与王雱交好,水平自然不会太差。要他做事可能有些麻烦,平常谈天说地还是不错的,就当是身边养个清客好了。等上任后,有足够的时间去看他擅长何事。

