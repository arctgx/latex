\section{第一章 庙堂纷纷策平戎(二)}

从宫中出来时,已是华灯初上,无数星辰在天空上闪烁。

骑在马上往家里走,韩冈还在想着方才崇政殿中的争论。

两府中的六个人,王珪、元绛和薛向是支持攻取兴灵,吕惠卿的态度暧昧,但只要王珪肯做出妥协,以支持手实法交换吕惠卿的支持,没有任何难度。只有吕公著和郭逵跟自己的看法相同,希望能稳一点,不要太过于急躁。

从人数比例上说,速攻一派占了绝对优势,而赵顼,也明显偏向前者。如果算进朝堂上的普通朝臣,持缓攻态度的声音,完全可以忽略不计了。

在韩冈看来,以如今大宋的国力,即刻出兵攻打西夏,直取兴灵,成功的可能性是六成到七成,在灭国之战上,这个几率已经算是高了。韩冈当初领军南下,也不是全然有把握,事先估算的几率其实也差不多就在六七成的样子。

但另一方面,采用韩冈的策略,先取银夏、兰州,不急着攻打兴灵。通过经济和政治手段,用一年到两年的时间,徐徐削弱西夏的实力。官军如泰山压顶得强势,很有可能会让西夏国中分裂的两派矛盾缓和而一致对外,但缓上一缓,通过各种手段挑拨,却能让矛盾就此爆发出来。

散其心,分其众,以大宋强大的综合国力将西夏击垮,成功几率近乎百分之百,完全不要冒任何风险。单纯只用军事来较量则不然——政治、经济、军事三方面,就以军事,西夏和大宋差距最小,利用地理上的优势,西夏甚至有击败官军的可能。

两相比较,自然是后者更稳妥。只是有人担心不抓住这个机会,会让辽夏两国缓过气来,故而赵顼听不进去……这就没办法了。

韩冈也不能保证辽国和西夏不会很快的平定内乱,他只是能保证以大宋国力可以压制经过内乱的辽夏两国。厚植国力,是压倒对手最简单有效的方法。可惜的是,天子不认同。

只能在旁边看着了,希望不至于落到任何人都不愿意看到的起步。

回到家中,工匠们已经离开了,带走了工具,但材料还在院中。

韩冈让人挑着灯笼看照壁上的图案。

只见左下角上已经镶了一颗五角星。火光下,分不清颜色,但深色调的瓷片,应该都是红色的。但问题是不仅仅是左下角,而是照壁的四个角全都用同样色调的碎瓷片拼上了同样的图案。

拖来的碎瓷片,红色的多有货真价实的钧窑红瓷,放到后世,就是碎片,也是价比黄金。一堆黄金做角落处的装饰,的确是够奢华的。可这已经是将韩冈的本意彻底给改了,让他去哪里再找三个皇帝从飞船上摔下来?

“官人!”王旖在内院听到动静,就迎了出来。见到韩冈正盯着照壁上的图案,便笑着说道“官人画的图奴家看着觉得很合眼,就让人在四个角上,都镶了一个,不知合不合官人的意?”

王旖仰头看着韩冈,摇晃的烛火映在深黑色的一对眸子中,闪耀如星光般璀璨。

韩冈低头,在她耳边轻笑道:“娘子真是为夫的贤内助。”

王旖听了,横了韩冈一眼,转身就进里屋去了。脚步轻快,丢下一句话:“官人还是早点换了衣服,都在等你吃饭呢。”

韩冈是听说耶律洪基死因之后一时兴起,才在角落里画上一颗星星。虽然可以百分之一千的肯定那是耶律乙辛下的手,但怎么说韩冈都占了一份功劳——汉时跋扈将军梁冀毒杀质帝,那块肉饼也是名留青史的。但他并没有打算向世人宣告什么,只是打算暗地里得意的回味一番。

可这个时代,名人和天上星星都有瓜葛,人亡星陨,所以世间才传说韩琦故世,大星陨于庭。韩冈一听说耶律洪基坠亡的消息,就在照壁上特意加上一个五角类似于星星的图案,怎么看都是他在炫耀自己战绩。

这样当然有问题。王旖也知道这么做不合适,韩冈一走,就赶紧让人添了三颗星。反正韩冈只是在一角画了颗星星,并没有吩咐说不能在其他角落镶上一颗星星。

韩冈虽是不在意这等小事,但王旖能帮着考虑周全,当然是难得的贤内助。

韩冈回了后院,换了衣服,先去看了周南。

周南养病的房间是专门设置的一个别院——给病人另外安排独立的住处在大户人家很常见——韩冈进去的时候,正是周南吃药的时候。

周南正皱着眉,苦着俏脸,看着碗里黑色的药汁,可抬头就看见韩冈进门,她立刻露出了美得让人心悸的笑容,“官人!”

韩冈坐到床边,让周南靠在自己的怀里。没有梳理的一头长发如瀑般披散下来,笼在白色的小衣上。被褥向下拖了一点,小衣下高高挺起的两团丰软顿时露在了外面,将将掩着有了规模的腹部。

韩冈将被褥向上提了提,盖住了她身子,免得受凉。柔声问道:“怎么样了,头还晕吗?”

周南摇摇头,靠在韩冈的怀里很是舒服,“已经好多了。”

她回到家后歇了两日,又请了御医开了两服调养的药,气色看起来已经好了不少。不过御医也说了,动了胎气没那没容易就好,还要养上一阵。

韩冈伸手从使女手中端了药,还热着。用勺子舀了一口,凑在了周南的唇边。

周南仰起脸,看到的是温和渊深的一对眼睛,在眼中看见的是宠溺和关爱。

依顺的张开口,喝了下去。“好苦。”周南顿时轻声叫着,脆弱的像个孩子一般。

“现在苦一点,等病好了就甜了。”

韩冈鼻子嗅了嗅,房间内在药味中还带着一股鲜香。病房里面又两个炉子,一个是小药炉,另一个则是取暖用的火炉。火炉上架着一口蒸锅,“锅子里面热的是鸡汤吧?”韩冈问着。

使女回道:“是鸡茸粟米粥,用的是鸡汤炖的。”

“等喝了药,就喝点鸡汤,正好去苦味。”

“嗯。”周南娇憨点头,乖乖的喝药。

服侍了周南喝了药,又让她喝了点鸡茸粟米粥,说了几句话,扶着她躺下来休息了。帮周南盖好被子,韩冈示意站在一旁的使女好生的侍候,然后悄声的走了出去。

到了正屋,王旖她们已经等了很久了。

王旖低声问道:“南娘妹妹可还好?”

“没事,不要紧。”韩冈说道。

“爹爹!”活泼可爱的小丫头跳了出来,乌溜溜的大眼睛灵动至极,举着双手让韩冈抱。

韩冈俯身将女儿抱起。他有五个儿子,就一个宝贝女儿,当然最受疼爱。“今天有没有听话?”他笑着问。

“金娘最听话了,一百个大字早上就写好了。”小丫头叫道。她趴在韩冈耳边神神秘秘的说,“爹爹。听说辽国的皇帝从飞船上掉下来了。春锦和秋罗,都说是爹爹做的。”

“当然是胡说。爹爹坐在京城中,手可够不到辽国去。”韩冈伸出右手,“你看,爹爹的手就这么长,站在这里连门都够不到。”

房中的人撑不住都笑了起来,金娘也知道韩冈是在开玩笑,扭着身子不高兴。韩冈宠溺的拍拍女儿的头,把她放下道:“好了,别耽搁了,吃饭。”

金娘乖乖的坐在桌边,一家人坐在一起吃饭不提。

吃过饭,老三老四老五三个儿子被抱进去睡觉了。韩冈叫了三位儿女一起回到书房,坐下来考校他们的功课。

韩冈坐在高靠背的交椅上,问着面前站成一排的儿女:“三字经可背熟了?”

三个小孩子一起用力点头,“都背熟了。”

“九九口诀呢?”

“也背熟了。”

“那好,一个个来,背给爹爹听。”

虽然并附注释的三字经才刚刚交付印书坊刻印,但原本早就抄了几遍,给韩家的子女去学习了。不过几百字而已,小孩子记性又好,半个月时间,全都已经背熟。

“钟哥儿,你先来。”韩冈点了老大的名。

韩冈过去忙于公事,很少有空闲顾及子女。几个孩子对他这个父亲都有几分畏惧。也就是韩冈比较宠唯一的一个女儿,所以金娘才跟他亲近。现在则是有空了,肯定是要多关心下儿子。

站在父亲面前,小韩钟有些紧张。韩冈的书房平常是不让他们进出,不论是在京城的旧宅,还是在京西,都是如此。站在父亲的书房中,对面就是家中人人敬畏的父亲。

“人……人之初,性本善,性相近,习相远……”

韩钟略带颤抖的声音,从韩冈亲笔定下的三字经开头背起,一句一句,渐渐就镇定了。全篇一口气背了下来,中间错了两个字,但又立刻改正了。

韩冈听得很满意。读书要先能背,然后能写,接下来还要理解,最后就是应用。第一步算是成功了。

接下来是九九乘法口诀。不过刚要背,司阍就送了一份帖子来,说是人就在门房候着。

龙图学士家的大门可不好进,没有些关系,少说也都等上三五日。尤其到了晚上,对游宴毫无兴趣的韩冈如今都是闭门谢客,司阍也知道这一点,门状一般都不会收。

韩冈看了看名帖上的落款,算是知道为什么家里看大门的司阍会收了名帖来禀报——是郭逵的儿子郭忠孝。

韩冈跟郭忠孝过去在秦州的时候见过几次面,虽然是武将之子,荫补的也是武职,但他还是二程的门人,看起来是要走文官的路线。不过韩冈没听说郭逵的儿子考中进士,多半还是个荫补官而已。

韩冈将名帖一收,吩咐道:“带他去偏厅。”司阍离开,韩冈就对儿女道:“今天爹爹有事,就散了。”

“是,爹爹!”老二韩钲叫得比谁都开心。

韩冈瞪了儿子一眼:“别指望逃过去,明天继续!”

