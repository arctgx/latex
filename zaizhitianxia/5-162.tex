\section{第17章 往来城府志不移(九)}

不过,太常寺的一众穷官吏也好,更胜台官的伶官丁仙现也好,韩冈都没兴趣与他们多纠缠。丁仙现肆意妄言的时候,是在变法之初,到了苏轼都被拘入台狱的如今,谅他也不敢再乱说什么。

被迎进衙中正厅后,照规矩点卯认人,说了几句场面话,就让下面的官员各自散去。

若是在地方上,正任主官就任,还得有一场在衙门正堂中办的接风宴,可是在京中的衙门里,而且是在皇城中,便没有这等规矩了。

当然,接风宴一般也是有的,只是得在外面的酒楼中,也不能动用公使钱。当几名下属的官员出言邀请的时候,韩冈直接就推辞了,当即就看到下面有几人明显的松了一口气。

除了那点数目可怜的俸禄以外,外快就只有依靠辖下的社稷诸坛、武成诸庙。一干坛庙,又不是佛寺,祭品本少,却还有三班院的人一并来分账——三班吃香这句俗语的来历——在其中分润到手的,每个人也不过是一星半点而已。虽然有个油水丰厚的教坊司,可也只能干看着,沾不上手。只要亲戚稍多,平常家里都吃不饱,哪有闲钱出来给上司接风洗尘的?

教坊司虽在太常寺辖下,但教坊中人与士林和官宦来往密切,太常寺对其并没有多少控制力——乐籍的管辖权都在开封府中,周南当年脱籍,状子也是往开封府而不是太常寺递的。

韩冈无意对此现状有所改变,甚至可以说,他不愿与教坊有何瓜葛,乃至整个太常寺。

方才在廷对上,韩冈了解到了一点赵顼的想法,而且他的计划也是在厚生司和太医局,没必要将精力放在太常寺这里。

厚生司现在归于中书门下辖下,等接手之后,免不了要与政事堂的宰辅们打交道。而太医局原属于太常寺,前几年才分离出来,以选派医官和教养学生为主,又有选派医生出诊在京诸军和国子监、武学,并不是全部只为皇亲国戚和官宦服务,这就是为什么韩冈可以向赵顼要求设立医院的缘故。此外还有奉旨赴灾区治病送药的工作,只是现在已经归入了厚生司。

想一想,如今在官制上还真是有混乱。赵顼既然让他主管厚生司和太医局,正好顺便将两个衙门给结合起来。

疗养院的制度已经确立多年,以军中为多。但医师坐馆的数量依然稀少,京城中的情况还好说,有太医局生来填补空缺,而外路就没那么好的运气了。要改变这一切,厚生司和太医局必须要起到更大的作用。

此外医药政令归于翰林医官院,这就让韩冈对于卫生医药方面的管辖权缺了很重要的一块拼图。不过以韩冈声名,要将这个管辖权争过来也不是什么难事。

太常寺众僚属全都散去,韩冈抬头看着经年未修的堂间梁柱,红漆斑驳,甚至能看到里面遇水朽烂发黑的底色,当真是破落得让人叹息,这可是九寺之首的太常寺啊!

但韩冈心中也就只有感叹而已,回去后要写札子,明天还要去厚生司,再过一日,则是太医局,这一下,可有的忙了,哪有闲心管太常寺房屋的油漆。

……………………

韩冈抵京,在京中本就是惹人注目的一桩事。

在过去,每当韩冈履新或抵京诣阙,总是少不了会有惊动京城甚至震撼天下的创举,飞船、板甲、牛痘,无不是如此,使得京城中人都很是期待韩冈这一回能有什么新的动作。

今日韩冈入对后,整个皇城中的各个衙门都竖起了耳朵,想知道韩冈在天子面前到底说了些什么。等到韩冈要编纂药典,对全城百姓的医馆消息传出来,便是一片哗然。

“史馆这边还有人打了赌,”黄昏放衙归家后,蔡卞还跟蔡京说着今天的事,“王正仲【王存】说是新的免疫法或是医书,曾子固【曾巩】则是说多半是重订朝廷礼典。”

“两边都擦了点边,但也不能算是猜中。应该是庄家通吃……”蔡京问着堂弟,“这一回有庄家吗?”

蔡卞摇摇头:“归入公帐,日后馆中置酒,由这里出钱。”

“元度你呢?”

蔡卞苦笑了一下,叹道:“小弟押了半贯在王正仲那一边,也一并归了公……哥哥你这里就没有人打赌的?”“御史台上下在韩冈手上吃的亏不止一次,对这个名字有忌讳,没什么动静。”蔡京嘴角翘了一下,也不知在嘲笑谁,“不过说起来,台中其实也有遣人去打探消息,要是韩冈当真献了新的免疫法,不论是什么病的,没人会念着旧恶,硬是拒之门外。”

蔡京已经从厚生司中调任,去辽国走了一遭之后,只用了很短的时间,就进了御史台,做了监察御史。在中书门下任职的经历,以及在厚生司和出使辽国的功劳,让蔡京成了如今京城官场中小有名气的新贵。

蔡卞轻呷了一口杯中的百合凉汤,由于是冰镇过的,口感分外的清凉。冬季赐炭,夏季赐冰。这是大宋在京官员们所能享受到的福利。但这样的福利,官品越低,享受的理所当然的就越少。蔡卞如今只是个史馆校勘,离正式的三馆馆职还差了一点,同时在国子监的差事同样品阶不高,仅是个刚刚转官的京官而已。尚幸他与蔡京同住一宅,倒是能享受到朝官的待遇。

“可惜哥哥已经不在厚生司中,否则编纂药典,也少不了哥哥。”

“因人成事,纵有功,在世人眼中,也尽是韩冈的。岂是愚兄所愿?”蔡京摇头笑着。

他在厚生司中得到的功劳,多有人说是占了韩冈的便宜。虽然蔡京并不在乎这一点,只要能升官,还怕别人议论?笑骂由他,好官我自为之。邓绾的名言可是说进了蔡京的心里。

但论起好官,厚生司如何比得上御史台?说起清要之职,那可是以谏官为首。

“不过韩冈倒是个聪明人。”蔡京又说着:“等药典编成之日,多半就是他入两府之时。现在天子一直压着,不让他入两府,还不是因为他的年资不足的缘故?”

“《太平御览》穷两府三馆七年之功,《武经总要》则是五年,《资治通鉴》至今未成,可都十二年了。韩冈要编订的药典,是繁是简尚且不明,就是他所说的重列纲目,还不知是什么样。今天连《本草》都贬了,若是药典有甚差池,可少不了丢人现眼。想必韩冈会精雕细凿,如此一来,说不定十年亦难见功。”

“且等着看就是了,过几日他肯定要上札子,到时候不就一清二楚了?”蔡京没兴趣去猜,道:“明日愚兄在棉行楼中置酒,元度你来不来?”

蔡卞摇摇头,“有哥哥出面就足够了。”

他的脾胃一向弱,一饮酒回去后少不了要上吐下泻,甚至病个几日,要是学着蔡京日日饮宴,那还真是要人命了。但奇怪的是,他在家里就是吃些冰镇过的冷食,倒是一点没事。

不过蔡京这样好宴客的行事风格,让蔡卞有些担心,“哥哥,你现在已经是乌台中人,宴请外官恐会惹人议论。”

“不妨事。”蔡京打个哈哈,“没说做了台官,就该跟亲朋好友割席断交的,人情往来,又如何少得了?”

见蔡卞还想劝,蔡京又换了话题,“说起来之前介甫相公进《字说》,为新学大张旗鼓,愚兄本以为韩冈为气学会针锋相对,却没想到上朝后就变成了药典和医馆。看起来他还是不敢与他岳父硬顶着来。”

“《字说》乃是介甫相公多年心血所得,韩冈就算想要争上一争,也得用上几年的时间去写文章,可不是张张嘴就能驳得了的。”蔡卞是王安石的弟子,曾经前往江宁向王安石求学,否则也不会被留在国子监中任职,“他改在医典上下功夫,大概是打着避实就虚的打算……诗赋就不用提了,只论起经义,韩冈也不是介甫相公的对手,最多也就跟洛阳的二程打打擂台。”

“韩冈所学,偏近自然。对经义的解释,自然是要差了介甫相公一筹。只是京城的衙内中如今流行显微镜和千里镜,前几日去汴水边的张家酒店,还听见吕晦叔【吕公著】家的十三衙内在隔壁的厢房中高谈阔论,说他用千里镜看土星,发现土星是扁的,像是边上多了一圈帽檐,给人一通嘲笑。”

“是个眼睛不好的。”蔡卞笑了,道:“不过小弟倒是听说,拿着千里镜从十三间楼往甜水巷里偷窥的更多。”

蔡京闻言眼神就变了:“是哪一家的在这么做?”

蔡卞觉得蔡京的语气有异,看了他一眼,随即便明白了,“哥哥可是要准备上本了?”

蔡京点了点头。新入乌台,第一个人选实在是很伤脑筋,必然是要找个合适的对象,好来一个开门红。不过蔡京也知道,刚刚上任的官员最好不能弹劾,尤其是天子亲自任命的人,那是跟自己过不去。不过若是做得好,五六年内,晋身侍制倒也不是难事。

随即又是冷然一笑:“也要先看看够不够资格。”

