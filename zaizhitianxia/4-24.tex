\section{第四章 岂料虎啸返山陵(三)}

【不好意思,迟了一点。】

“乾健坤顺,二气合而万物通;君明臣良,一德同而百度正……”

夜色下的宫掖,森森如晦。外廷中的一栋栋殿阁,仿佛一只只怪兽蜷伏在黑暗中,仅有点点微光于其中亮起,勉强多了一点人气。

只有今夜的内东门小殿,此时是亮如白昼。数以百计的班直护卫将小殿内外护住,围得水泄不通。

天子的御驾就在此处。

每逢大拜除,学士院锁院,而天子也会驾临内东门小殿。每每到了这个时候,东京城中文武官员都会将目光投向这座禁中略偏东向、紧邻学士院玉堂的小殿中。屏神静气,等待着从里面传出来的声音。

章惇已经从小几旁站了起来,草拟的诏令文稿拿在手中,朗声诵于天子。视一榜进士如探囊取物,章惇的才名不在历科状元之下,只是开头的两句,就让赵顼轻轻点了点头。

……眷予元老,时乃真儒。若砺与舟,世莫先于汝作;有哀及绣,人伫久于公归。……

“想不到王介甫当真要回来了。”

冯京今天没有上殿,但他作为宰相,耳目还是足够的灵通。升任宰相的半年多,让他的权势又扩大了许多。这对于他总掌朝政是好事,但王安石回来后,情况却要反过来了——王安石能容他做参知政事,却不会容他做宰相。

“谁让他招了个好女婿!想不到韩冈还有这个手段。”

蔡确他昨天的确是向韩冈通报了消息,可那只是为了结个善缘,省得日后难看。却没有想到韩冈竟然直接就把王安石举了上来。他不是要将张载请进京城来吗?怎么就敢打起让王安石复相的主意。以王安石的倔脾气,就算受了韩冈如许大的人情,也别想他会点头让关学在京中传播。

“不是韩冈,是韩绛!”韩冈算什么,没有韩绛自请留对,赵顼即便想要起用王安石,也决不会这么快的下定决心。

“都是姓韩的。”蔡确笑叹了一声,从眼下的情况看,日后朝堂上绝不会少了韩姓的宰辅,“不过韩稚圭自开春后就重病卧床,已经快不行了。昨天相州那边就来了人,看样子拖不了多久了。文宽夫、富彦国、曾明仲也都老了,说不准什么时候就配飨宗庙。等他们一去,王介甫可就是元老了。”

冯京紧紧抿起嘴,沉默了下去。王安石能成为元老,不知他冯京日后能不能成为一言一行都能牵动朝局的元老重臣?

……越升冢席之崇,播告路朝之听:推诚保德崇仁翊戴功臣、观文殿大学士、特进、行吏部尚书、知江宁府、上柱国、太原郡开国公、食邑四千六百户、食实封一千两百户、王安石信厚而简重,敦大而高明。潜于神心,驰天人之极挚;尊厥德性,溯道义之深源。……

“御驾已临内东门小殿,肯定是大拜除了。”韩冈在二更天的时候也得到了消息,此前的猜测终于得到了确认,他的岳父当是要复相了,“今天不知会有多少人睡不着觉。”

王旖肯定是睡不着觉中的一个,双眉依然轻蹙:“会不会是别人?”

“吴充?吕惠卿?都不可能啊!”韩冈笑道:“韩绛留对难道是为了推荐他们两个吗?天子对岳父一向信重。若是寻常时候,也许会不觉得。但到了朝局僵持不下的时候,这个信任的用处就出来了。”

元老重臣的价值就在这里,一任宰辅的资历能让一名官员成为朝廷柱石,越是局势动荡的时候,他们受到的期待也就越大。

王旖点头,勉强的笑了一笑。毕竟还是至亲,韩冈说得再是信心十足,王旖也照样要担上一份心。尤其是王旁,他可是被牵连进了谋反案中。

“明日去宣德门外,看了榜文便知端的。只要天子有意让岳父复相,就绝不会允许有人动仲元一根寒毛!”

……延登捷才,裨参魁柄。傅经以谋王体,考古而起治功。训齐多方,新美万事。而则许国,予惟知人。谗波稽天,孰斧斨之敢鈌;忠气贯日,虽金石而自开……

两部经传新义的改稿就堆在书桌上,吕惠卿昨天本准备着今晚就将最后的修改给润色一番,但他现在却无心动笔。吕升卿、吕和卿也都在书房中,只有在外任官的吕温卿不在。

等了半天,不见吕惠卿开口,吕和卿忍不住提起话头:“王安石又要回来了,朝堂上的局面又要有一个大变动。”

“无妨。”吕惠卿似乎并没有感染到两个弟弟心中的焦躁,轻笑道:“介甫相公回来后,正好可以将手实法推行下去——已经耽搁得太久了。”

吕升卿哪里会信,手实法已经连实施的细则都编定好了,可王安石回来之后,难道还会推行吗?就算推行了,也不会再是他兄长的功劳,而是王安石的。

“都是韩绛,竟然自请留对!”吕升卿狠狠的说着。

“是韩冈!没有韩玉昆,事情可不会这么顺利。介甫相公挑的这个女婿,可是挑得再合适不过了。”吕惠卿为之更正,笑意盈盈,对韩冈赞赏有加,只是眼中,却难掩刻骨的憎恨。

……向厌机衡之繁,出宣屏翰之寄,遽周岁历,殊拂师瞻。……

近日朝局的嬗变,可以说给了赵顼一个教训。异论相搅如果操作不好,就是干扰到朝政施行的党争。一边倒不行,但两边势均力敌,也同样是个灾难。

对比起眼前的现实,赵顼还是怀念过去的岁,每每回想起王安石主持朝政的时候,虽然反对声始终不绝于耳,但朝局总能稳定下来有着王安石做主心骨,任何的问题都能解决。而不是像现在,朝堂上一团乱,两派互相攻击,却没有一个能将对方压制。——这其实也是人之常情,人总是习惯性的美化过去,总会觉得过去比较好。

抬手摸了一下嘴角的燎泡,赵顼忽然发现,在决定了将王安石重新请回朝堂之后,突然间就感觉不到疼痛了。

‘果然还是将王安石调回来省心。’

……宜还冠于宰司,以大厘于邦采,兼华上馆,衍食本封。载更功号之隆,用侈台符之峻。……

章惇庆幸自己站对了位置,要不然等王安石回来后,他肯定是要靠边站了。

只是他到现在还不清楚,韩绛和韩冈事先有没有串通过,要不然为什么这么重大的决定,韩绛竟然只在殿上就做了出来,而且还选择了惹人议论的自请留对。

回去后要问一问韩玉昆,就不知道他会不会说实话。

几句话就让天子动心,再有了当朝宰辅的支持,轻而易举的就改变了眼前的僵局。章惇不觉得韩冈在这方面的才智超过自己多少,却很佩服他的决断,毫无顾忌的抬了那尊大佛出来,一下就镇压住了朝堂。

风向要变了。

……於戏!制天下之动,尔惟枢柅;通天下之志,尔惟蓍龟。系国重轻于乃身,驱民仁寿于当代。往服朕命,图成厥终。……

韩绛今天是破釜成舟,既然冯京、吕惠卿让自己在政事堂中只能做个押班、盖印的闲差,还不如干脆让王安石回来。一拍两散,大家都别玩。

不论最后能不能,王安石都要领自己的人情。而前面他举荐自己两次为相的人情,也算是还了大半。

韩绛冷笑,将银质的酒杯捏得格格作响,半年多来积攒的怨气绝对不浅。原本他是踌躇满志,想有一番作为,却没想到就这么一事无成的度过去了。

没说的,韩绛从来都不是好脾性的人,这份怨恨,他会十倍还之。

……可特授依前行吏部尚书、同中书门下平章事、昭文馆大学士、兼译经润文使……

宰相兼任译经润文使是唐时传下来的规矩,如今的首相——昭文相——都会兼任此职。

章惇是第一次为天子撰写拜相大诏,但他作为翰林学士,该知道的规矩也都加以了解过了。过去的诏令都是要编纂成册,章惇在担任知制诰之后,不知翻阅了多少遍。不但可以熟悉朝廷故事,在撰写诏令的时候也不会有所错漏——免得贻人笑柄,甚至因此而被降罪。

“等等。”赵顼打断章惇的诵读,“别忘了,要加食邑一千户。”

“臣已遵谕旨,加食邑一千户,食实封四百户。”

宰臣、亲王、枢密使每次加食邑,都是一千户,而实封则率为四百户。

“嗯。”赵顼点了点头,章惇这位新科翰林对朝廷故事的熟悉,让他很是满意,“另外再改赐推忠协谋同德佐理功臣。”

“惟命。”

宰相的功臣封号基本上都少不了推忠、协谋、同德、佐理八个字,韩绛、冯京身上都有,章惇其实也已经写上去了。而加食邑、加实封的事,一开始赵顼也吩咐过了。要不然章惇前面也不会写上‘衍食本封’‘更功号之隆’几个字。

将草稿诵读了一遍,又交给李舜举呈于御览。待天子点头认可,章惇便展开白麻纸,端端正正的誊写起来,旁边的小黄门专心致志的帮着磨墨。

一笔一画,召唤王安石入京为相的诏书,一行行的在纸面上显现出来。

