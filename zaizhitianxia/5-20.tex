\section{第三章 时移机转关百虑(六)}

【昨天晚上断网,今天白天又在外面,现在两更补上,这是第二更。待会儿还有一更。】

吕大钧点点头:“所以韩冈等了十年,直到在广西发现了牛痘,才命人去验证。功效确凿无疑之后,方才公诸于世。那么愚兄再问与叔你……”

“二哥!”吕大临直接打断了吕大钧的问话,“换作是小弟,当是发现不了牛痘之事,不用谈什么公诸于世了!小弟论才论能,的确都不如韩冈,这一点,小弟无意否认!”

“只是向道之心绝不输人?”吕大钧轻声一笑,就像吕大临知道他想问什么一样,他也知道自己的兄弟想说什么。吕大钧收起笑容,正色问道,“那韩冈是为了什么才将人痘和牛痘之术说得那么明白?只说牛痘难道不是可以免去结怨天子的危殆?而且韩冈运气还不好,直接撞上了七皇子建国公因痘疮而死。换做是与叔你,会说得这么明白吗?”

随着吕大钧的问题,院中陷入了沉寂,只有身后酒宴正是热火朝天的大厅,传来阵阵荒腔走板的小调,端着酒菜的仆役从门中鱼贯而入,而捧着空菜碟和酒壶的仆人则鱼贯而出。

吕大钧皱着眉向身后看了一眼,拉着兄弟往僻静的地方走去。吕大临沉默的随着吕大钧的步伐,久久不能回答。

吕大钧也不等吕大临的回答了,他边走边说:“有望宰执,却近乎于放弃了未来晋身两府的机会,宁可开罪天子,也要推广他的大道。韩冈向道之心,不比与叔你稍差!”

“二哥此言差矣!”吕大临绝不会承认自己跟韩冈有哪里相似,站定了:“小弟自知学问浅薄,如今乃是求道,而韩冈则是要将自己旁门之术,直接标榜为大道、正道!”他的声音因愤怒而大了起来,“韩冈之学,只得一偏。他的笔记,二哥你不是也看过了吗,里面有几句涉及经义?!”

韩冈前些日子遣人将他的新书《桂窗丛谈》送到横渠书院苏昞处,书院中的学子当时是人人传抄。一个月的时间,虽不能说在关中士林传扬开了,但以吕大钧的身份,手上拿到一份抄本却不足为奇。

吕大钧知道,吕大临手中也有一份抄本。他瞥了弟弟一眼,无月的朔日,只有黯淡的灯光,看不出吕大临脸上的表情。

“见过人家盖屋建宅吗?”此时两人已经站在了院墙边,吕大钧指着一丈高的墙壁,“总是先要将地面给夯实了,然后才会立柱架梁、砌砖夯土。数丈高的楼阁,都是从地基开始。韩冈也是一般。他从身边事说起,螟蛉义子的谬误、浮力的原理、彩虹的真相,乃至牛痘的发现,一点一滴都是围绕着‘格物致知’四个字而来。看着不涉大道,可都是在为他的学术夯筑地基,等到有一天,韩冈正式开始涉及天人大道,那便是水到渠成,无物再可阻挡!”

“也要他能做到!”听到兄长对韩冈所作所为的推测,吕大临毫不动摇,“在经义上,他还差得远!”

“日渐日新,以韩冈之材,难道还不能学吗?!”吕大钧质问道:“韩冈不及而立。至少有三十年,甚至四十年、五十年的时间,去补充,去完善,最后去宣讲他的气学。你若是有心坚持自己的大道,日后必然会有几十年的时间与他相争,这个准备,你做好了没有?!”

吕大临眼神凝定如钢,无所畏惧的与吕大钧对视着,一字一顿:“自反而缩,虽千万人吾往矣!!”

“愚兄不是要阻拦你。在正叔先生门下,愚兄也所得甚多。闻道有先后,达者即为师。正叔先生即是达者,愚兄虽是年长,却是远远不如,所以正叔先生讲学时,也是洗耳恭听,最后深有所得。”吕大钧顿了一顿,“而韩冈年虽少,但在格物致知四个字上,亦是达者,试问与叔你,在此一节上有他看得透吗?”

吕大临张口欲辩,却被吕大钧给打断了,“与叔你既然认为韩冈所学不正,那就得想办法去驳斥他!但在此之前,你必须认清你的对手,去好好想一想你的对手的长处,去深入了解过他的观点……甚至去学习他的道、他的术,而不是一味的排斥。排斥韩冈的所言种种,并不代表你就赢了,只会让人认为你浅薄!”

吕大钧的一番话如同狂风骤雨般劈头盖脸砸向吕大临,而吕大临的神色则是愈见冷漠,却没有任何屈服的神色。

吕大钧都有点口干舌燥了,但他依然坚持:“如果你有秦始皇的本事,能焚书坑儒倒也罢了。可你压不了韩冈,相反的,韩冈日后还能轻易压倒你。等他坐上宰相的位置,如今正当红的新学,不是被韩氏气学所顶替,就是两者并行。到时候,你站在那里?”他叹了一声,“韩冈当日致书关中,将与叔你写的行状一番宣扬。几封信一出,气学门下顿时同仇敌忾,一下就被他凝聚住了人心。现在关中士林,人人都知道,韩冈是气学赤帜,日后必能承袭子厚先生之教,为气学光大门楣。故而人心不散,门庭犹在。而你现在,又有什么?”

“韩冈用心不正!”吕大临如同一头倔驴,完全听不进去。

“哦,是吗?……”吕大钧说了这么多,却说不动自己的弟弟,一时间都有些心灰意冷,“‘向道之心从无一日而绝’,看来是我听错了!”

“二哥!”吕大临悲愤的叫道。

“话说出口了,可谓是掷地有声,但你真的做到了吗?不论韩冈的用心,他的学问是实实在在的。”吕大钧双眉挑起,怒声质问着吕大临:“先圣问礼于老聃,问乐于苌弘,问官于郯子,学琴于师襄。此四子,无一人可及先圣,先圣尤躬问而学之。韩冈若学无所长,能有现在声望?能有现在的地位?能有如此多的功劳和实绩?不论是非好赖,一概贬低,你这是向道的做法!?”

“韩冈那并不是道啊!……”吕大临也是委屈无比。

吕大钧却更怒:“韩冈有事例为凭据,日后他说话,必然有人虔信不疑。你呢,到时候你拿什么证据来证明自己,跟韩冈辩论?就是先圣,也要笔削春秋!”他恨铁不成钢,“好好想想吧!”

吕大钧说罢,拂袖而去,只留下了吕大临孤伶伶的站在寒夜中。

吕大临并不认为自己错了,大道本就不在那些细枝末节上。韩冈自己曾经都说过那是旁艺。自己也并不是否定韩冈的才能和成就,只是认为他表现出来的那一部分成绩仅仅只是术和技而已,离着自然大道有着很远的一段距离。

吕大临只是没想到自家的兄长竟然认为自己都是妒贤嫉能。他心中一阵阵的抽痛,牙关死死咬紧,几乎要迸出血来。

“所谓好学者,不迁怒,不贰过。与叔……当自省。”

从夜色中,悠悠传来一句话,是程颐的声音。

“先生!”吕大临连忙回头。

不远处的院墙下,一扇小门吱呀打开。一个略嫌削瘦的身影从门处走了过来,正是方才自称不胜酒力、提前退席的程颐。

程颐本来是准备在年节前回洛阳的,可是一听到牛痘传世,便立刻做出了在关西在留上一年的决定。

他的看法跟吕大钧相同,韩冈是放弃了自己的前途,冒着巨大的风险来宣扬自己的道。凭借着牛痘在天下万邦的推广,韩冈对格物致知的释义,以及与其紧密联系的气学,都因此而更进一步的发扬光大。

韩冈苦心如此,可比辞官授徒更要艰难上十分。不仅是要承受着天子的压力,还要靠自己为整个学派保驾护航。

任何一们学派,没有高官显宦的襄助,想授徒传世,那是极困难的。

泰山孙复,安定胡瑗,徂徕石介,全都是靠当时的宰执重臣在背后支持,才能国子监中立足。而盱江李觏,因为无人在朝中匡助,现在他的传人已经寻之不见,只有一部分观点被王安石所吸收。

张载若无韩冈,气学出不了关中。而二程年纪不大时便广有声名,那是有洛阳诸位元老重臣一力推重的缘故。

韩冈现在没有后台帮他宣讲他的学术,只能自己亲历亲为,而气学门墙,还得靠他来支持。一人身兼两职,却还要咬牙支撑,甚至不惜为此开罪天子。

这样的坚持,有着压倒性的力量!

程颐作为旁观者,看着也是不免要感慨许久。

“和叔说的是不错的。求学不论高下。和叔立乡约,任道担当,其风力甚劲。与此事上,吾亦要向和叔请教。”

程颐的气度让吕大临感佩不已,但对韩冈的看法,他依然不改!道之所在,虽千万人吾往矣。韩冈曲解大义,如何能容忍?!

程颐只当没看到吕大临脸上的倔强,继续说道:“先圣求学四方,礼乐官制皆得授于人,也曾说过吾不如老农、吾不如老圃,但有一条大关节却始终没有动摇——”

看了一下侧耳恭听的吕大临,程颐铿锵有力的说道:“大道不曾改!”

