\section{第一章 坐忘渭水岸(上)}

“已是百战功成,想不到还是缘悭一面。”赵顼抬手推开当面的数支柳条,“朕是皇帝,可想见一次臣子却是这么难。好个韩冈,为个解试,竟然连上京诣阙的机会都推了。面承清光难道还比不过一个贡生资格?”

王韶心中一惊,抬头向前望了一眼,倒是没在天子的侧脸上看到有何不快的神情。

御苑之中草木森森,冠盖如伞,遮挡了午后的艳阳。

江南可采莲,莲叶何田田。

虽非江南,但一道白玉栏杆围起了数亩的小湖,莲叶覆盖了半个湖面,清风徐徐,碧绿的荷叶竟也波浪起伏起来。

君臣二人行走在清风之中,赵顼继续说着:“说起朕自登基以来,自问可算是勤政。没有见过一面的朝官,除了广南两路的几个知州以外,也就韩冈一人了。”

“韩冈为人刚直,不愿受非份之赏。”

“他的脾性,朕也知道。”赵顼点了点头,道:“横山纵胜,亦不愿居功受赏。当着宰相的面如此说话,世间当真没有几个。拯危阻敌,孤身平叛,这样的功劳都放下了,更是只有一人。”

“也有小人说韩冈如此是沽名钓誉。”

“那就多给朕几个同样沽名钓誉的……朕手边正缺这样的人呢。”赵顼笑笑,带着王韶走到了一座小桥上,手扶栏杆,“朕虽是看重韩冈,不过若他与卿家一同上京,朕最多也只能给他一个参加礼部试的资格。非是朕吝啬,实是韩冈功绩虽著,可文名不彰。一个进士出身虽不算多重,但也不便赐于他。惹来议论,更对他日后立于朝中不利。朕可是想着将来要大用他的,若是有了污名,那可就不好办了。”

王韶看着身前削瘦背影,心中一惊。虽然他早知赵顼对韩冈很是看重,但听到这番话,还是心中惊讶不已。但赵顼的话,也是王韶对韩冈的看法:“以韩冈之才,一榜进士当是易如反掌。如若是诗赋以取士,或许还有待商榷。但论起经义策问,他已是出类拔萃。其人之才,不仅仅是治政用兵。”

“其实若有治政用兵的经济,学问稍逊其实也无妨。就如薛向,他没有一个出身,但还不是做到了一路监司,乃至现在的三司使?熙河所用,在朝中,也多得薛向悉力营办。”赵顼顿了一下,“就是没出身,也是一样能为朝中重臣。”

‘但以韩玉昆的年纪和官品,他怎么可能只想着一路之地,三司之职,而不想着身列宰执班中?’王韶暗自想着,却没有说出来。

“还记得王卿五年前献上的平戎策。”赵顼回转身,同时也转过了话题,“‘夏人比年攻青唐,不能克,万一克之,必并兵南向,大掠秦、渭之间,牧马于兰、会,断古渭境,尽服南山生羌,西筑武胜,遣兵时掠洮、河,则陇、蜀诸郡当尽惊扰’。”

想不到赵顼竟然还能记得当年献上的《平戎策》中的内容,但时过境迁,“如今陛下已经不用担心了。”王韶微一躬身,充满骄傲的对赵顼说道。

“乃是卿家之力。”赵顼赞许的点着头,“‘西夏可取。欲取西夏,当先复河湟,则夏人有腹背受敌之忧。’如今木征就擒,董毡亦将降伏。断西贼右臂之势已成,就不知何日才能直捣腹心……”

“陛下……”王韶脸色微变,急忙道:“河州大战虽胜,但如今秦凤仓囤已然一空,熙河也须休养生息数载才能自给自足,实在不是向灵夏用兵的时候。”

“这朕也知道,灭国之战非是等闲。朕也不会急于一时,多少还有几年的准备。”赵顼凭栏而望,落在一瓣残荷上的视线,看着的却是数千里外的金戈铁马,“二十万不成,六十万难道还不行吗?”

战国时西秦灭楚之战。始皇征询老将王翦,若以他为将,灭楚须兵几何。王翦的回答是六十万。这个数字,几乎是秦国的举国之兵。所以始皇,用了另一个只要二十万兵的将领。但用兵不是购物,价廉者中选。楚国是百足之虫死而不僵,二十万秦军伐楚,便是大败而归。最后还是按照了王翦的要求,动员了整整六十万,方才灭亡了楚国。

因为新法顺利推行的缘故,赵顼对大宋的国力有着足够的信心。国库中的仓储,已经不复赵顼刚刚登基时,让他手脚冰凉的空旷。只要再等几年,就能筹备起足够平灭西夏的人力物力和财力。

“朕今年不过才二十有三,几年时间,还是等得起……届时,也少不了要用到卿家的时候。”

王韶深深一弯腰:“臣当效死。”

“效死就不必了,朕还等着卿家如今次一般,让朕能在紫宸殿上受群臣朝贺呢……”

君臣二人继续在荷塘边漫步。赵顼居前,听着王韶说着些河湟的奇闻异事,不时还追问着两句。

李舜举这时匆匆而来,神色凝重的向赵顼递上了一份奏报。

赵顼接过来展开一看,神色也变得沉重起来,眉头微蹙,轻声自语:“天下文才十斗,不意今日又少了一斗。”

王韶在后面看不到这份奏报上说的到底是什么,但从赵顼自言自语中,也能猜出个大概。天下文才十斗,能独占八斗的是三国时的曹子建【曹植】。而大宋国运昌盛,文运大兴,才子大家,车载斗量,再无人能独占天下文采大半。而能当得起十一之数的,也就寥寥数人。稍作思量,其人身份便是呼之欲出。

‘欧阳九风流顿尽。’王韶心中一叹,不无悲凉。他中进士是在嘉佑二年,也正是欧阳修主考的那一科。若非欧阳修一改当时流行的险怪艰涩的文风,他说不定还中不了那个进士。

“醉翁亭中不见醉翁矣……”赵顼也黯然一叹,将奏报递回给李舜举:“赠故太子少师欧阳修为太子太师,馈赏依宰相制。命太常礼院定其谥号。至于荫补等事,待遗表至,再论!”

……………………

此时知太常礼院的是由布衣入官的常秩。欧阳修旧时与常秩最善,曾几次三番的举荐于他。虽然后来,因种种事端而疏离。但人去恩仇尽,过去的事,也没必要再提。

常秩坐在公厅之中,太常礼院中的众官坐在下首,听着草拟谥号的太常博士李清臣道:“太师一生,于教化治道为最多。下官按谥法,道德博闻曰文,当谥之以‘文’。”

“文……”以文臣来说,谥号中得了这个字,已经是了不得褒奖了。常秩想了想,问道:“过往谥‘文’者,是为何人?”

李清臣早已命人查过资料,答道:“国朝谥‘文’者,杨亿一人而已。唐时谥‘文’者,则有韩愈、李翱、权德舆、孙狄。”

“韩退之倒也罢了。但杨亿、李翱、权德舆、孙狄之辈,如何比得了欧阳永叔?”

“不当用‘文’字吗?”被人否定,李清臣心头不快,“敢问知院欲谥之何字?”

“永叔为天下文宗,‘文’之一字,当仁不让,不可改易!然永叔平生好谏诤,所谓‘智质有理’,当加一‘献’字,为‘文献’。”

“文献迭犯庙谥,不可用!”李清臣立刻否定道。

“若献字不可用,则加一‘忠’字,为文忠。”常秩似是早有定见,前面被否定掉便立刻提出了另一个方案,“永叔尝参天下政事,曾进言仁宗,乞早日下诏立皇子,使有明名定分,以安人心。及英宗大行,今上即皇帝位。永叔两预定策之谋,有安定社稷之功。又曾和裕内外,周旋于两宫间,迄于英宗之视政。按谥法,‘危身奉上’为忠。”

“且永叔天性正直,心诚洞达。为人明白无所欺隐,不肯曲意顺俗,以自求稳便安好。论列是非曲直,分别贤愚不肖,从不避人之怨诽诅疾。忘身履危,以为朝廷立事。‘廉方公正’为忠,这四个字,永叔也是当得起的。”

“谥者,行之迹也;号者,表之功也。永叔一生,道德博闻,危身奉上,廉方公正,这都是有的。谥永叔为文忠,不知诸位意下如何?”

常秩是欧阳修旧友,说得又甚为有理,众官点头之余,都看向了李清臣。李清臣起身行礼,“不改于文而加之以忠,议者之尽也。清臣其敢不从!”

欧阳修一代大家,如今天下文士,多以其为宗。不过他虽为三朝重臣,但一生却从没有站对过一次,最后落得一身谤言,声名丧尽。僻居远州数载,直至今日,才又回到世人的心中。

当教坊司的花魁们,开始唱起‘清晨帘幕卷轻霜。呵手试梅妆。都缘自有离恨,故画作远山长’的时候,王安石也听到太常礼院给欧阳修定下的谥号,为故友长叹之后,也不免黯然,“今日永叔得谥文忠,不知后人如何谥我……”

曾布道:“相公匡扶今上,一扫大宋数世积弊。百年之后,何愁不得美谥?!”

“算了!”王安石洒脱的笑道,“死后万事皆空。授以何谥,那是他人之事。吾辈论事,只在今生!”

