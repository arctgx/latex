\section{第36章 万众袭远似火焚(11)}

王厚给沈括一家安排的住处,是衙门附近的一个小院。形制并不大,但还算干净。

沈括进院看过后,感觉还算满意。这才陪着小心的将一直坐在碧油小车里的浑家请了出来。

续弦张氏,是他恩主张刍的女儿。治平四年【1067年】发妻叶氏病逝。两年后,也就是熙宁二年,曾经是沈括上司的张刍将女儿嫁给了他。才二十岁,嫁给沈括也只有三年。老夫少妻,又是年轻貌美,沈括对张氏可是千依百顺。

张氏蹙着眉头,在院中转了一圈,却是一言不发,走进了正房中。

一直跟在后面的沈括稍稍松了口气,虽然看着不满意,可至少张氏没有反对住进这件院子。不然就让他在这里难做人了。

过了一阵,张氏的贴身小婢出来,却对沈括道:“夫人说累了,要先歇一歇。请官人自去处置正事,不必挂念家里。”

沈括连连点头,“我这就去衙门里。”

安顿下张氏,沈括便匆匆赶往州衙,接手随军转运的工作。

沈括听说过韩冈的名号。对于韩冈这个每多发明的年轻人,沈括的兴趣很浓。先通过透析砲车的原理,进而发明霹雳砲这样的军国利器,这一点,沈括也是有些佩服的。还有军棋、沙盘等物,才三两年的功夫,已经遍及天下。不拘于经传文字,想来也算是同道中人了。

韩冈亲笔所写的兵站制度的文稿拿在手中,随意翻了一翻。一点也不像是二十岁的年轻人能写得出来的。听说刚入官时,就已经写过一部有关疗养院的制度,连王相公都赞不绝口。不能以年龄轻忽视之。

但沈括可不会全盘照着韩冈的规划而来,虽然这一套制度看着完备,可也并不是没有改动的余地。

不加以更动一二,另有开创,如何能显出他的手段?!

“来人!”沈括指派起手下地胥吏,“速将帐册都搬来!”

……………………

韩冈抵达最前线的工地时,景思立已经率部把营盘当道扎好。

就在秦凤军营地的后方一里处,两千余名民伕正在一片周长约六百余步的工地上忙碌着。

这座寨堡被王韶命名为临洮,也就是将狄道城的旧名,移花接木到这座位于洮水河谷北方前沿的寨堡上。

修筑临洮堡,是为了抵御北方来敌。而在临洮堡后方十五里的河川交汇处,另有一座兼做兵站的寨子正在修建中——熙河经略司登记的名字是结河堡,以流经堡侧,汇入洮水的结河川而得名。

两座城寨一立,通往香子城和珂诺堡的河谷道的安全就得到了保证。而且大宋对洮水河谷的控制,也随之向北——也即是下游——推进了四十里。

一队队民伕喊着号子,夯筑、挖掘,在工地上忙忙碌碌。行动间有条不紊,却仅仅指挥者得力之故。被征发起来的民伕,基本上都有修筑工事的经验,宋人在关西修筑堡垒的工程从来都没有停过,哪一家的壮丁隔个两三年,就会有一次夯土挖坑的活摊到头上。

这跟后世许多出身农村的建筑工人相似,农忙时在家务农,闲时就会出来做工——当然还是有区别,一个是拿钱的,一个则是白工。

临洮堡工程进度很快。

护翼堡外的壕沟已初具规模,而矩形的城墙也已经打好了地基。春天的营垒修筑工程,比冬天要轻松许多。单是取土一项,就能省下不少人工。冻得如同钢铁一般的土地,不知弄伤多少民伕的双手,而换作是解冻后的大地,轻轻松松就能将地里的黄土给铲起。

韩冈巡视过营地,又抬头看了看两侧的山头高地。那里有几个原木搭起的高台,是最简单的哨口,用来监视是否有敌军来袭。等到临洮堡完工之后,就会将那几处高台改用黄土夯筑起来,作为烽堠使用。

从工程进度方面来看,景思立做得还不坏。并没有之前王韶、韩冈担心他会因为心怀芥蒂,而对于营造修筑上的工作不加关心的情况出现。

景思立听说韩冈到了,很快便赶了过来。略叙寒温,韩冈遂问起最新的情况。

“发现了蕃人游骑的踪迹?”韩冈听了几句,就立刻问道,“都监可知是哪一部的蕃骑?”

“派出去的哨探也只是远远的看到了。”景思立有些惋惜的说着:“没能捉个活口来,弄不清是哪一部的。”

韩冈略感失望,兰州禹臧家、乃至他们背后西夏的反应是重中之重,不能确认,就不能合理有效的应对。但在景思立面前,他也不便将心中的想法说出来,省得景思立会认为自己是在抱怨。

“就当作是禹臧家的人吧,”韩冈轻笑道,“熙州北方,也只有他们才会不厌其烦的来窥伺我官军。”

景思立哈哈笑了两声,“包约也是这么说的。”

“包约他人呢?”韩冈问着。

按说包约这位青唐部的二当家,应该正带着他的族人在此处与北方的禹臧家对峙中。怎么只有景思立过来,他却不到?

“左近又有一家蕃部不稳,今天早上他就率军赶过去了。”

韩冈闻言,摇头失笑:“什么不稳!就是压榨得过了头,熙州北方的蕃部被他这群青唐部的人祸害惨了。”

“包约做的事,禹臧花麻也在做,而且做得更过火。”景思立冷笑着,“这群蕃人,就该好好的磨上一磨。”

“现在没时间教训他,等收拾完木征,肯定要让包约他收敛一点。……正好可以让这一带的蕃部归心我大宋。”

以夷制夷,然后居中调解,并保证各方势力可以互相制衡,这都是汉人千年来用得不能再熟的伎俩。景思立也不以为怪,早就知道的事,没有熙河经略司的纵容,包约何来这个胆子?

“王经略什么时候开始攻打河州?”景思立将包约的事丢到了一边,问着韩冈。

“必须等临洮堡和后方的结河堡都完工,粮道稳定下来,才好一鼓作气的继续前进。”

韩冈指了指这一侧的工地,“临洮堡的修筑进度,在下方才看过了,大约还要半个月的时间就能完工。一旦临洮堡完工,都监你就可以去珂诺堡与王经略他们汇合了。”

“怎么,不让我在这里守着禹臧家的人?”景思立语带嘲讽的冷笑起来。王韶的心思他怎么看不出来,被发遣到这里来守备,一心想建立功业的景思立早就忍得一肚子的郁闷。

韩冈略低了低头,并不与景思立争辩:“王经略打算在河州城决战,少不了都监的助力。”

前面王韶指挥熙河军先声夺人,通过攻夺珂诺堡,占据了战略优势。等稳定了后方的运输线,下面就要全力攻往河州城。要应对木征的人马,以及董毡可能派来的援军,光是王韶手上的几千熙河军肯定是不敷使用,少不得要把秦凤、泾原两路的援军提上去。

比起泾原军来,秦凤军王韶看着还是亲近一些,好歹他也曾是秦凤经略安抚司的机宜文字,做了好几年的事。而且韩冈在秦凤军中的人缘,也不是等闲可比。所以景思立手上的兵,王韶肯定要用上,不会一直放在后面吃灰。反倒是姚兕姚麟的人,要分派的去处很多,反而无法集合起来使用。

“不过还要提防禹臧花麻。不论是前年的渭源之役、还是去年的临洮之战,盘踞在兰州的这个吐蕃族的叛逆,都有出兵相助。今次肯定也少不了要插上一脚。”

韩冈正说着,几声尖利的号角从山头上的高台传来,急促中带着一丝惶急。很快,一直派在前面的斥候也狂奔而回,越过前方的营垒,在景思立的面前跳下马,急声禀报道:“都监、机宜,北面有一军来袭。都是骑兵,看声势当在三千上下。”

景思立看看韩冈。现在的熙州北部,能派出三千骑兵的势力,只有禹臧一家。就算是横行霸道的包约,要守着的地方太多,也只能带出两千不到的骑兵。

韩冈无奈的摇摇头,这还真不能算是他的乌鸦嘴,“该来的总归会来。”

烟尘漫天,随着震天撼地的蹄声席卷而来。

韩冈和景思立已经赶回到前方的营寨中。近万大军在跨河而立的营中严阵以待。

有大军驻屯的营寨在前面守着,一条栅栏更是将整条洮水谷地拦腰截断,后面的工地依然照常开工——就是不知道他们的效率还能剩下多少。

千军万马奔腾而至,气焰冲天。这一军蕃骑一直冲到秦凤军的营寨外,隔着百来步,方才停了下来。

“两千。”

韩冈也算是上过几次战场,但如景思立这般一眼就能数出敌军多寡的本事他还没能练出来。而景思立这等老将的另一桩本事,也是韩冈望尘莫及,

“看不出贼人的战意。多半不会打,只是来骚扰而已。”

