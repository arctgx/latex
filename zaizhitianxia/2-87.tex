\section{第20章 心念不改意难平(六)}

【第一更,求月票,收藏】

秦州城头上没有什么好风景,东面一条大道直通陇城,背后是人烟辐辏的城市,南北两面青绿色的山峦已经让人看得厌烦。

藉水在城南不远处流过,河水泛着浑浊的黄色,藉水河源处树木茂密,水土完好,河里的泥沙也不知是从哪条支流从山沟里冲下来那么多黄土。

都是韩冈看惯了的风景,早已没了兴致。今天的天气又是个‘秋老虎’,太阳才升到半空,就已经展示出堪比三伏时的热度。黄土夯筑而成的墙体被晒得滚烫。比呼吸还要轻微的山风根本缓解不了城头上如地狱般的酷热。

郭逵对酷暑似无所觉,扶着雉堞,向四处远望。

韩冈站在后面,已经热得汗流浃背,回头看看已经散入城中的官员们,他心中羡慕不已。回头看着郭逵宽厚的背影,韩冈弄不清他到底在想什么。

说是要谈谈话,但现在却一句话也不说。如果说是要挖墙角,又不是很像——前面郭逵说得那些攀交情的话,显得太没有水准,一点也不含蓄,有失他郭太尉的身份,反而让人觉得有些假。

可总不会真的是站在城头上看风景,欣赏一下秦州的美丽风光吧……

韩冈想了一阵,放弃继续伤脑筋了。若是郭逵想故弄玄虚,自己就奉陪到底好了,反正自己的年纪轻,就看谁的体力更好一点。

“玉昆。”郭逵突然出了声。

韩冈精神一振,“下官在。”

“你对河湟之事看法如何?!”郭逵的问题突如其来,简单直接得如羚羊挂角,无迹可寻。

韩冈却是胸有成竹,慨言道:“河湟不定,克复西夏便是水中捞月。”

郭逵听得一奇,拓边河湟仅是偏师,其重要性完全比不上横山,这是朝野共同的看法。韩冈之言别出心裁,让郭逵觉得很新鲜。问道:“河湟当只是偏师,‘断西贼右臂’可是王子纯在《平戎策》中说的。不知玉昆所言,又有何凭据?”

韩冈自有一套解释:“自鄜延向北越横山,便是银州、夏州。而西贼巢穴却是在兴灵。光是夺取了银夏,并不足以剿灭西虏。银夏与兴灵间有七百里瀚海。韩海之中少有水草,渡瀚海攻贼。恐怕尚未见敌,便已是自行溃灭。”

“这跟河湟又有什么关系?”

“河湟的北面,过了六盘山,就离兴灵没多远了,而且并不需要渡过瀚海。而且蜀道不止一条,经由岷水、洮水转运亦是一条要道。若能攻下河州熙州,蜀地的粮秣钱饷就能直接运入关中,不需要经过陈仓道。而秦凤一带,需要的粮草物资,也可以由蜀地运出一部分,而不是必须从东面调来。

另外,收复河湟蕃部后,就有了足够的蕃军可以驱用,有粮有兵,便可翻越六盘山直捣敌巢。日后朝廷讨贼,先以河东、鄜延、环庆攻银夏,秦凤、河湟牵制贼军。若西贼不救银夏,西贼依之为命脉青白盐池就会落入我手。若救援银夏,西贼南面必然空虚,秦凤、河湟届时就能趁虚而入。”

“……这是王子纯的想法?”

“王安抚正按着《平戎策》上的计划,来主持军事。托硕、古渭虽有巧合的一面,但都是计划中的一环。”

韩冈答非所问,而他的回答是在向郭逵说明王韶在开边事上的作用,还有自己的立场。

韩冈委婉的表明立场,让郭逵沉默了下去,又转回身看起了风景。而韩冈对自己必须在两人中选边,心中有些无可奈何。

相处了几个月后,他对王韶的了解已经很深。王韶是绝对不会让出河湟开边的主导权的!拓土之功在开国之初也许不算什么,以曹彬平灭南唐的功劳,甚至也不能换来一个枢密使。但在如今,却足以让一名小臣籍此挤进宰执中的行列——王韶的心气一直很高。

任何人想在这方面打主意,必然会引发王韶的疯狂反扑。高遵裕就是清楚这一点,才甘心做着王韶的副手,并不试图取王韶而代之。因为在天子心目中,高家的舅公远远比不上王韶,绝不会支持高遵裕的野心。

而郭逵甘心做绿叶吗?他平过荆湖山蛮,他孤身降伏了保州叛乱,在关西更是屡有战功,眼光精准闻名朝中,但他却缺乏狄青在昆仑关大破侬智高那样光彩夺目的战例。

…………韩冈的思路突然一顿,

狄青?!……

而这时,郭逵再次开口:“王子纯的《平戎策》,本帅也看过,的确难得。朝中少有人能把关西局势说的如此透彻。”

“不过王安抚也说过,《平戎策》并非他凭空而来,也是有其源流。家师早年就有开拓河湟的心思,而关西军中不少人都有同样的想法,好像太尉也是提过的。太尉当年在关西,能与狄武襄和种仲平【种世衡】并称,也是……”

“玉昆你这是说瞎话了。”郭逵当即打断韩冈的话,显然韩冈这等没有技术含量的马屁并不受他欢迎,“当年关西最有名的是狄汉臣【狄青字】和种世衡。范公向朝中举荐的十几名武臣中,他们两人是排在最前的。”他自嘲一笑,“可没本帅什么事!”

韩冈若有所思,郭逵称呼狄青的字,而直接叫着种世衡的名。看来郭逵跟种世衡有旧怨难道不是谣言。难怪他一直跟种谔过不去,想不到还有这层原因在。

不过郭逵能提到狄青就够了,他故意用着拙劣的手段拍着郭逵的马屁,就是要引他提到狄青。有狄青的前车之鉴在,相信郭逵会收敛一点。

这么想着,韩冈的话题便不离狄青:“狄武襄以行伍入朝堂,身居枢密一职。能与他相比的,也只有太尉了。”

“狄汉臣以朝议而去职,因忧惧而早亡。名将不得善终,让天子不止一次的对着我等感叹。”

大概是因为韩冈并不是进士的关系,郭逵为狄青叫屈起来便没有什么顾忌。不过他的语气里却还有些愤愤不平的感觉,不知是不是因为赵顼认为他郭逵不如狄青。

狄青也的确是冤,不过,这个时代的武臣有几个不冤的?在文臣当道的年代,武夫妄想跟文臣一较高下,或是动了文臣的奶酪,从来只有死路一条。

成功的将对话的主导权从郭逵手中抢过来,韩冈便不会再还给郭逵。他问道:“听说狄武襄之子现今也在延州。”

“是汉臣家的三哥狄詠!”郭逵也没有注意到韩冈的用心,“汉臣的儿女不少,可惜没有几个出色的。多是承了汉臣的好相貌,却没传下他的胆略和武艺。他家的大哥早夭,现在也就老二、老三还能让人入眼,其他却都不成。”

“不是听说他屡有战功吗?都已经升到了都监了。”

“狄三也是靠着父荫,天子追缅汉臣,所以他也跟着沾光。当年狄汉臣平侬智高后,他就是阁门祗侯了。可现在十几年过去了,他已经年过而立,也不过立了些微功劳,却也不算什么,不能跟玉昆你相比。”

韩冈自谦道:“当年侬智高之乱,狄三都监可是跟随狄武襄一起去得广南,岂是下官可比。”

“他有什么功劳?有功是狄汉臣,还有他带去的将士!”郭逵低头望着城墙脚下的一处军营,正在出操的数百士兵,整齐的队列和雄壮的口号让他捻须微笑。“狄汉臣为了对付侬智高,从关西带去了一千蕃落骑兵。但玉昆你可知最后还剩多少?”

“多少?”

郭逵沉声说道:“不足四一!”

“就剩了两百多人?!”韩冈本不觉得这些蕃人到了广西还能囫囵个儿的回去,但死了七成还多,却着实让他吃了一惊。

“战死得很少,多是病殁。到广南就病倒了十分之一,等开战时只有八成上阵。返程时仅有半数,回到关西就只剩四分之一了。南方瘴疠之地,北人不习水土,苦寒之地的蕃人更是病得多了。”

郭逵叹了口气,转过头来盯着韩冈:“军中防疫是门大学问。想玉昆你也读过兵书,军中扎营率有定规,各部之间都会隔着甚远,严禁互相串访走动,不容半点差池。一为防敌防火防奸细,第二,就是防着疫病。”

韩冈开始明白郭逵为什么看重自己了,“太尉的意思是……”

“玉昆你的功劳虽多,临危受命也好,说服蕃人也好,在本帅看来只能算是不错而已。但你所创立的疗养院,还有你编修的条例,本帅却是要为之击节叫好。”

郭逵身为统领大军南征北战的主帅,对军中医疗的看重是他几十年军旅经验的总结,即便是韩冈自己,也不会如他这般重视。

“玉昆你虽是缘边安抚司管勾机宜等事,但你也兼理着秦凤路伤病事。这两者,希望你能权衡好,不可偏废。秦州疗养院的事本帅已经有所准备,需要什么尽管提。只要玉昆你做得好,本帅不会吝于举荐。”

