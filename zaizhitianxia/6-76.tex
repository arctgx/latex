\section{第九章 旧日孤灯映寒窗(上)}

周围人流如织,却安静得听不到几句人声。

大多数人都在念念有词,低着头,只看着脚下。

开宝寺的铁塔下,这样的场面并不鲜见。每到正月初一、四月初八、腊月初八等节日,开封府中有数的大丛林,总是会这般人头涌涌,却又安静的只有唱经呗诵的声音。

不过,这并不是佛诞日或元日进香。

皇宋三年方得一次的抡才大典——进士科礼部试,终于在今天开始了。

数十步之外,贡院的大门敞开,汹涌的人流正慢慢的汇入贡院之中。

间中有几声来自于贡院守卫的呵斥,但反而更显得人流安静得异常。

远在贡院前街两端的街口处,开封府便设下了鹿角栅栏。所有送考之人,全都给拦在了外面,能走进这条街的,要么是应考的贡生,要么就是官员,至少得有着身份证明才能通过。

黄裳并非第一次站在科场外,但作为旁观者还是第一次。

原来身处在数千人中,完全没有感觉到有这般安静。当时只顾着回忆自己事前写好的猜题文章,走了几步又去想会不会再次落榜,到了门前,就收拾心情,完全不去看周围的情形。

每一科上京应考的数千贡生,仅仅是天下间数百万读书人的一小部分。从数千人中脱颖而出,成为三四百名进士中的一员,说比例,比不上百里挑一的州中解试,但这是与天下间数以百万士人中的佼佼者同场竞争,难度自是又上了一层。

所以在当时,黄裳的心中只有紧张,身在人群中,只能感觉到自己的渺小。

不像现在,已经处在人流之外。

跳出三界之外,不在红尘之中,这才叫超脱。

而自己,是超脱了。

站在开宝寺的牌楼下,黄裳看着一名名装束各异的士人从他的面前走过。

有年轻的,也有年长的。老的能须发花白,年幼的就只有十七八。

黄裳刚刚看见一名只有十三四的贡生走过去,不知是天生个矮加娃娃脸,还是当真只有这个岁数。不过有别于周围同伴的紧张和小心,那位贡生倒是显得趾高气昂,意气风发,大概当真是初生牛犊不怕虎。

黄裳无声的笑了起来,曾几何时,他也是如此意气风发。

十七岁第一次州中应举,便高中前三,当时以为一榜进士是囊中之物,唾手可得,但十余年下来,却颗粒无收,纵然一次次的州中解试都能名列前茅,但一到京师,便铩羽而归。

如果是关西、河东等处士子倒也罢了,州中头名到了京中能列名榜末已是侥幸,但自家乡里是福建路南剑州,天下各路应举之难无如福建,而福建应举之难则无如南剑,多少乡中远在自己之后的士人,都陆陆续续考中了进士,而自家却依然只能一次次的遗恨科场,这让他情何以堪?

直到游学到任官襄州的族兄那里,遇上了韩冈为止。黄裳选择了仿效韩冈,先为幕僚立功得官,有了官身再去应考。

换了心境,也许原本在科场上拥堵在心中的才学,便能够发挥出来的。

其实也算是畏难而退了。

不过黄裳当日拜入韩冈门下的时候,决然没有想到,自己甚至能够跳过礼部试和从来无缘一见的殿试,直接拿到进士资格。

黄裳已经由太后钦赐进士出身,与眼前的这些犹在贡院门前紧张得发不出声的贡生,已经不在一个层面上了。

但黄裳的心情反而更为紧绷。

新进士张榜,琼林苑赐宴,接下来就是轮到他上殿了。

不过想要拿到上殿参加御试的机会,还要经过三馆馆阁成员的考核,也就是所谓的阁试——这才是最大的难关。

本朝自开国以来,通过制科的士人数量都没超过五十人。

而本朝的进士有多少了,一万、两万,还是三万?黄裳估计从没有人数过,但绝对是通过制科人数的数十倍——这还是包括开国之初的几十年,进士科平均每科只有十几二十人通过的情况。

就是现如今的朝堂中,有着进士头衔的,至少两千人,占据了朝官的绝大多数,同时也是地方各级亲民官的主体。而当今还在朝中的制科出身官员不过两手之数,前日还刚刚少了一个,贬了一个。

为什么制科多年来就那么几十人能够通过?主要就是阁试一关刷去了太多滥竽充数之辈,那是远比礼部试更为严格的考核。

否则到了御前,几句好话一说,说不定就能让天子晕头转向,加之上表举荐的重臣,也多半在殿上,配合着搭个腔,一个制科出身的资格就轻松到手。

可以想见,阁试的题目必然是往难里出,出的简单了。让太多人通过,岂不是伤了崇文院的名声?三馆秘阁中的成员,想来也必是以无人通过为荣,以放人过关为耻。

依靠恩主提前拿到的进士出身,万一连阁试都通不过,黄裳可没脸再去见韩冈。

不过黄裳若是没有些自信,就不会到开宝寺这边来。

来此目送贡生,可以说是感慨,也可以说是怀念。

因为这一切已经与他再无关系。

这段时间以来,黄裳对经史典籍以及历代注疏的攻读,远比旧时更认真了十倍。半年多下来,自觉学问又精深了一层。若是回去考进士,也许也能一争前十。

黄裳的嘴微微抿了起来,与眼神一般的坚毅。

此番赶考,是为了成功,不是为了再一次的失败。

……………………

“黄裳!”

走在身边的张驯突然叫了一声。

声音刚出口就给他压低了,但宗泽听到了,向周围看过去,立刻就在开宝寺的牌楼下找到了目标。

宗泽多看了两眼,也终于将人给认出来了。

的确是韩冈那位有名的幕僚。

“他来这里做什么?”

张驯的口气有着难以压抑的愤怒。

马上就要参加礼部试的贡生,看到一名刚刚从太后手中混到了一个进士资格的幸运儿,的确是该愤怒的。

宗泽同样有些不解,黄裳转眼就要去参加制科考试了,却为何在今天跑到开宝寺这边来?

“当不会是为了上香。”宗泽不知道黄裳是不是来这里看贡生入考场,但想来总不会是去开宝寺上香的,“去二圣庙会更灵验一点。”

“谁管他那么多。”张驯带着怒气,“子夏子路会庇佑这种幸进之辈?”

宗泽微微一笑,对张驯的攻击保持了沉默。

要说功劳,黄裳两次在河东辅佐韩冈的表现,的确远不如韩冈当年在熙河辅佐王韶的表现更加耀眼

当年韩冈可是在王韶、高遵裕两位上司追击蕃军残部,独立支撑一路军政,不仅仅击退了乘机来犯的西贼,还接连挡回了两道要求撤军的圣旨,平复了河湟拓边功亏一篑的危机。

拥有那样的功劳,先帝都没有赐予韩冈一个进士出身,而黄裳的功劳仅止于辅佐,却轻易的拿到了。

在士林中,对这种投机取巧的做法,很多人都十分反感。

都是先羡慕,再嫉妒,然后恨之入骨。就跟现在的张驯一般。

但要说幸进,那就过分了。再怎么说,黄裳都是在边疆立过功的,不是在国子监中指点江山的士人能比。而且能在南剑州拔贡,黄裳本人的水平也足够当得起一个进士出身,只是过去欠缺一些运气,现在老天假韩冈之手将运气还给他,这也是酬劳黄裳旧日的辛苦。

宗泽与张驯在人群中缓缓前进,由于要搜检衣物内外,贡生的数量又太多,在贡院门口形成了拥堵。

好半天,两名考生也仅仅前进了十几步。

张驯板着脸,已经安静了好一阵,突然间又压低声音迸出了话来,“他礼部试都过不了,阁试肯定不能过!”

宗泽想不到都走过去了,张驯仍是耿耿于怀。

“黄裳阁试肯定过不去。”张驯再一次重复道,“那可比礼部试难得多。”

阁试当然难。

宗泽也很清楚,就连张驯这种自视极高的人,即便说要参加制科,却也只会是说说而已——左右找不到能推荐他的重臣,说到做不到,也可以推到宰辅有眼无珠上。

事实上,能有通过阁试水平的,一帮进士里面也不一定有一个半个,加上运气,或许能有一个。

阁试的题目很简单,就是六篇论。

不过题目的范围很大,遍及以九经、兼经、正史,旁及武经七书、《国语》及诸子,在正文之外,群经亦兼取注疏,这个范围要远远超过礼部试。

在这六题之中,三题出自正文、三题出自注疏,考生在阐述论点之前,必须先指出论题的出处,并须全引论题的上下文,这样才能称为‘通’,也就是合格。

但题目绝不会那么简单就是书中的原文,而是有明暗之分。直接引用书中一二句,或稍变换句之一二字为题,称为明数;颠倒书之句读、窜伏首尾而为题,则为暗数。

这种将原文扭曲的暗数,就是专门用来刷落考生的题目。虽说依照规定,六题明暗相参,暗数多不过半,但也绝不会少于一半,而想要通过阁试,至少要有四个‘通’才行。

这可比礼部试要难多了。

若黄裳不能通过阁试,便失去了御试的机会。而只有参加御试,韩冈这位参知政事才有机会干预结果。

