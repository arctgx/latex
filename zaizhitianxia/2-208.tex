\section{第32章 吴钩终用笑冯唐(15)}

赵顼进来的时候,文彦博正等得心浮气躁。

一部分是最近枢密院和王安石主持的中书门下,在争夺三班院的控制权的事情上落了下风,吃了一个闷亏;但主要的还是因为如今京城中流传的有关分割陕西路的传言。

政治流言是每一个大国首都最大的特色,无论古今中外,概莫能外。开封作为大宋京城,一国的政治中心,自然也不会例外。

无数人的生活都跟朝堂上的变局息息相关,几万对眼睛时时刻刻都盯着宫中、朝中。对于天子和宫廷来说,他们的生活根本没有隐私可言。今天早上发生的事,下午就能传遍京城;夜中发生的事,到了第二天上午,路边卖凉汤的婆子都能摇着扇子说出个道道来。

仁宗皇帝玩一龙二凤的游戏,上朝时多打了个哈欠,就立刻被言官们群起而攻,逼着他把两个心爱的美人送去道观出家;如今的高太后和曹太皇,因为英宗皇帝纳妃的事吵了两句,第二天桑家瓦子里的说书人,就有段子扯起了隋文帝和独孤皇后的故事。

天子当然不想自己夜中敦伦的事都被人拿出来当话题,要是隔绝内外消息的手段,能像宫墙一样,把宫内发生的秘密全数拦在宫中,生活上当能轻松许多。但身居高位的宰执们,一旦看到宫中有这等阻断内外的迹象,立马就能蹦起五尺高。不把危险的苗头打下去,把执行的人踢出去,他们是不会罢休的——没有了宫中的消息,御史们也会少了一半的工作,为了自己,他们也会彻底的站在宰执们一边。

当年仁宗皇帝重病,文彦博、富弼他们可是想方设法地改变旧时规矩,留宿在宫中,甚至一步步的进了天子的寝殿。美其名曰,不得让妇寺之辈隔绝中外。这时候,可就没人讲祖宗之法了。

不过,东京城中的流言实在太多,靠谱的很少,尤其人们传谣的时候,往往偏向于惊悚怪奇或是风流韵事。所以御史们也只是风闻奏事,让他们事事去追查个究竟,就不要做事了。手上掌握着更为有效的信息渠道的宰执们,更是不会对耸人听闻的谣言一惊一乍。

只是今次文彦博听到的传言不同以往,并非是毫无实据。分割陕西转运使路,很早以前就人有上书过了。

原本的秦凤路是经略安抚使路,属于军事方面。现今传言中,要从陕西路划分出来的秦凤路,则是转运使路。负责粮秣运送,控制着财权。若是当真设立秦凤转运使路,很明显就是为了河湟战略的大举行动做准备,就像为了攻取横山,而设立陕西、河东宣抚司一样。

从道路交通上说,陕西一路过于庞大。为了能利于指挥,旧有的陕西经略使路被一分为五——分为鄜延、环庆、泾原、秦凤和永兴军路;转运使路一分为二也是很正常的。

在行政上也不难做到,大宋的路一级的编制换得频繁,河北、两浙都没少动过,只需朝旨一封而已。多了一个路一级的监司,官场上也必然受到欢迎,如今朝堂上是僧多粥少,一下多了几十个位子,对官僚们来说当然是件好事。

虽然是传言,可却有着很强的现实性。能一针见血指出横山攻略失败后,朝廷在陕西战略转移的动向,必然有人在背后操纵。同时以文彦博对赵顼的了解,如果有人如此上书,他多半就会点头答应。

文彦博心中不停声的骂着,‘横山一场乱局刚刚平息下来,陕西一路正是要休养生息的时候,又开始打着西面的吐蕃人的主意。总得让人喘口气吧?!’

在空旷寂静的崇政殿中等了不知多久,终于听到从殿后小门后传来的一片脚步声,天子驾临的通传之声,也随之而来。

大宋的枢密使屈膝跪倒,低着头,挑起眼皮,用余光迎着几个熟悉的身影走进殿内,其中穿着红袍的瘦削男子走到了御案后,坐了下来。

天子落座,文彦博随即叩拜下去,行礼如仪。

平身过后,看着文彦博站起身,赵顼不忘给老臣赐坐。但文彦博直挺挺的站着,把赵顼的好意推了个一干二净。

赵顼叹了口气,皇帝不好做,大臣给他脸色看也是常事,他都习惯了。不再强求文彦博落座,直接问道,“文卿此时求见,不知有何要务?”

“臣是为了西事而来!”文彦博朗声说着,分割陕西路尚是传言,他当然不会拿出来说,只能够旁敲侧击:“吴逵之事至今悬而未决。叛军降伏多日,可罪魁依然未擒。臣请陛下降旨关中,各州各县严加防范,巡检司巡查道路津梁,绘影海捕,悬赏吴逵。”

“自当如此,韩绛奏文亦是如此说,且已经做了。”

虽然前几天就知道吴逵下落不明,但经过了十天的搜索而不获,陕西宣抚司最终放弃了。今天传了消息回来,韩绛、燕达皆为此上表请罪,并禀明已经下文在陕西路绘影海捕,请朝廷予以追认。与文彦博所说并无不同。

只是赵顼心中不无疑惑,吴逵虽是兵变罪魁,需要海捕的要犯,但也不至于让枢密使急着进宫来。难道文彦博紧急求见就是为了说这些?

当然不可能,文彦博后面还跟着话:“吴逵久在军旅,深悉个中内情。臣请陛下即刻下旨,陕西缘边四路之城寨、要隘、营垒、馆驿,皆须重新检查防备,各部驻军则提前更戍,旗号暗记亦须加以更换,以防其人投奔党项,泄露军情机密。”

“……此事韩绛也已经在奏文中说过了,朕也准了。”

两番建议都成了马后炮,文彦博神色不变,前次在朝堂上差点中风晕倒后,他的心理素质反而变得更加出色。他继续说着:“吴逵领广锐军叛乱,祸乱关中。广锐之名已是不祥。请陛下下旨,裁撤广锐军,销毁旗号文牍,将未叛之余部,并入他部马军。”

“……关于此事,韩绛也说了,朕同样准了……韩绛的奏文还说,请朝廷尽速在陕西推行保甲法,各乡各村结为保甲,严防盗贼、逃人和奸细!韩绛甚至还为环庆及泾阳等三县请命,免了今年的税赋……这几条,朕都允了。”

赵顼一叠声的把韩绛奏疏中的内容都说了出来。他做了这么些年皇帝,阅人甚广,臣子的言谈举止中有什么用意,许多时候他都能看得出来。文彦博现在还拿老眼光看他,把他的年轻当作好糊弄,未免太小瞧人,也是欺人太甚了。赵顼盯住文彦博——若有什么话,现在也该说了。

被赵顼一阵抢白,文彦博依然平静自若。但现在他也明白,不能再玩弄言辞上的游戏。跳过了过于冗长的开场白,他直接进入正题:“陛下。三千广锐叛卒虽因被困咸阳城中,势不得已而降伏。但贼心难改,一旦他们脱离绝境,未必不会再叛。且吴逵潜逃在外,亦有可能与其相勾连,此事防不甚防……”

“文卿你的意思是?”

“三千叛军祸乱关中,如何还能将其留在陕西?当尽数流放广南,以防其与吴逵勾连。另外叛军余属贷其死罪已是宽大,若依陕西宣抚司之言,与叛军同流通远军,岂是对兵变的惩处?当悉配为奴,以儆效尤!”

文彦博杀气腾腾,赵顼却是叹了口气,“至于此事,韩绛在奏文中也说了。”

文枢密脸色微变,只听赵顼道:“承诺之事不可轻改,否则朝廷言而无信,必生变乱。且吴逵生死不明,若其当真潜逃,留其叛党在关西,也好作为诱饵。暗中监视众叛将,如果吴逵死不悔改,犹有叛逆之心,前去联络他们,届时便可一网成擒。”

赵顼不知道韩绛什么时候变得这么条理分明,面面俱到,这与他之前的奏章风格截然不同,不知道是不是换了起草奏章的幕宾。但韩绛的奏章宛如先见之明一般的与文彦博针锋相对,一条条的抢在文彦博的前面,让文枢密使的一番盘算全部落了空。如此巧合,让赵顼也不禁哑然失笑,原本郁闷已极的心情,现在稍稍好了那么一点。

文彦博的用心,赵顼已然知晓。

得到了文彦博那么多的提示,加上近两天皇城司的密奏,赵顼对文彦博为何而来,心中有数。

项庄舞剑、意在沛公。明着说的是对吴逵叛军的处置,实则却是在杯葛另外一桩要事。

赵顼慢悠悠的对文彦博说着,口气像是在征询他的意见:“文卿,最近朝中有人上书,但言陕西转运司事务剧繁,倍于他路。历任转运使,一任任满,也难将各军州走遍。若是西贼同寇多路,更是难以支撑。请朝廷将陕西路一分为二,以便指挥调动……此事京中亦有传言,不知文卿事先听说过没有,对此又有何看法?”

“此事……万万不可!”

文彦博毅然决然,硬到极致的口吻,没有一丝通融的余地。

