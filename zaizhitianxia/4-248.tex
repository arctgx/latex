\section{第41章 顺风解缆破晴岚(上)}

【一次次食言而肥,发现自己是越来越胖了,不管怎么说,下一更肯定会赶出来,请各位明天早上看。】

九月底的方城县山阳港,此时熙熙攘攘。

京西商人们的嗅觉,跟其他路州的商人一样敏锐。当开始修建轨道时,就已经涌来了第一批商人,等到轨道修好后,又来了第二批,不过前两批都是小商家,只是来看看作为襄汉漕渠一个组成部分的方城轨道,本身有没有油水可捞。

但今天过来的,都是京西各州府的大商号,甚至还有荆湖、蜀中的商人,听说了襄汉漕渠开通在即的消息,匆匆赶来,要亲自查证传言是否虚妄。

事实证明了传言。三架高大如楼宇一般的龙门吊已经让人惊叹不已,数以百计的商人站在龙门吊旁边,亲眼见证了,只用七八个人操作的吊车,轻轻松松就将一船船货物转送到岸上等待载货的货车上。

其效率远远超过人力运输数十倍,商人们仰望着龙门吊的眼神,仿佛是在看着大雄宝殿上的金身如来,几乎要顶礼膜拜。如果天下的港口中都能用上着龙门吊,那能为他们省下多少钱?!木质的吊车在他们的眼力仿佛闪耀着动人的金光。

也有一些个商人望着行驶在轨道上的马车,打听着有价值的信息。

“昨天在路上跑了个来回。一趟十五文钱,六十里路一个半时辰就过去了,回来也是十五文。一百多里路,往常都是坐着马车赶上一天的,坐有轨马车,一个上午多一点就完事了。”一名操着京西口音的本地商人向几个刚刚赶到的外地商人介绍着。

“十五文。”旁边的一位五十多岁、看着有些斯文气的老行商,慢慢点头,“这个价挺便宜的。”

“前段时间是试车,所以便宜,据说等到正式开通,就是二十五文一个人了,身上带的行礼还不能超过二十斤,再多就要加钱。至于货物,交运费的同时,照样要征税。”本地商人更正道。

“怎么没看到送人的马车,”一个长得肥肥白白,很是富态,满身绫罗绸缎的胖子转着头打量周围,“那些车子应该都是载货的吧?”

“难道你们还不知道?”本地商人的惊讶很是夸张,“转运司有六十万石纲粮要在冬月前运到京城,差事赶得这么紧,方城轨道从今天开始,就不再载客了。”

“说笑吧?”听到这句话的胖商人吃惊不已,“襄汉漕渠就是为了纲运而开,这点俺倒是知道的,以韩龙图的才具,想来也不会有问题。但一个月六十万石?这怎么可能!一年十二个月,可就是七百二十万石。天下纲粮才多少,六百万!都没听说超过七百万的。”

跟他通行的另外一位中年商人也不信——他们都是操着:“小韩龙图是不是犯糊涂了。水运一个月六十万石倒一点不出奇,想那汴河也没有多宽。但这陆运能有六十万石可就不得了。一天两万石,可就是两百万斤!”

“其实有一个半月的时间。”老行商多知道一点事,“京城水运封航一般是在冬月中旬。六十万石是分作四十多天运完。”

“那一天也要一百三十万斤的样子!”那个胖子说道,“还能跟汴水差不了太多!”

“没看到吗?”来自本地的商人指着前后五节的有轨马车,“这样的一节车比太平车还要大一半,一辆太平车能载五六千斤,你们说这五节车又能载多少?一刻钟发上一车,轻轻松松就能完成。”

胖商人和他的同伴眉头还是皱着的,掐着手指算了一通,还是觉得不对,抬起头,眼中满是疑惑。

老行商也心算了一番,“不对啊,就算一刻钟一班,一天最多也不过二十七八趟。”

“怎么可能才二十七八趟车,”本地商人摇头失笑:“看清楚点,这轨道,夜里也能走车的。”

“夜里也能走车?!”来自外地的三名商人齐齐惊道,转过去盯着一尺来高的轨道。

“看来肯定是不会有问题了。”老行商喃喃的说着,望着轨道的眼神有些复杂。

“为了这一次的纲运,转运司还特意给每辆车加配了挽马。都准备到这一步了,怎么可能会有问题?”本地商人很是自豪的说道。

码头边,十匹挽马前后排着队,被安顿在了一列已经装好纲粮的马车前。而几名小吏正拿着红色的彩绸准备挂在马前。在边上,又有人摆出了香案,供了三牲,香炉也摆出来了。还有一串串鞭炮,

“今天是第一趟运纲粮的马车要上路,算是方城轨道正式开通,待会儿还有官人们要来。州里的,县里的,转运司中的都要来,据说韩龙图也会来。”本地商人压低了声音,“听说还把州城里教坊司的官妓都调来了,为了这第一车上京的纲粮,可是会热闹得很。”

“还真够热闹的。”胖商人伸着脖子,向里面张望着。

“韩龙图看来的确是有信心啊。”老行商长舒了一口气,声音低了点,“好像什么时候看到都是如此呢。”

“什么?”旁边的胖子没听清楚。

“没什么。”老行商摇摇头,也跟着向众人环绕的中心地带张望去。

为了加快转运速度,方城轨道这里,甚至还给每辆车多配了四匹马,总共十匹挽马一起拉动沉重的货车。虽然性价比上比只配六匹马时要低上一些,但从单位时间的运输总量上,肯定是要高出许多。

不过是再多三五百匹挽马而已。韩冈身为转运使,要弄到一两千匹战马很难,但要调集一两千匹挽马可是轻而易举。只要能在封冻前完成这一次的运输任务,成本什么的,完全可以丢到一边去。

之所以要将时间卡得这么紧,一定要在封冻前运完六十万石纲粮,而不是干脆放到明年开春再说,就是为了表现出轨道出众的运载能力。三个月六十万石,平平无奇,当年为了打压京城的粮价,可是一个月不到就通过雪橇车弄来了二十万石。

但一个半月不到的时间,就运送六十万石纲粮,这个成果就能让人目瞪口呆了,接近于汴河的运力,那就是给有轨马车最好的广告。

在外奔波的行商没有几个蠢人,虽然更深层的用意,他们限于信息不全的缘故,无法推测出更多。但当他们看到十匹挽马拉着几万斤的粮食行驶在方城山中的时候,不会想不到,这样的一条能比得上汴水运力的陆上交通线,对于商业流通能有多大的促进。

“其实这一次京西各大商号都派人来了,都想看看轨道到底能不能成事。”本地商人指了指站在前面的一个胖子——比这边的胖商人还要圆上一圈,“前面的那一位是专从蜀中贩药材的庆余堂的胡掌柜,这两天,他来回坐了八趟车。看他现在的模样,若是这一次当真能在冰封前将六十万石的纲粮运抵京城,他们家的药材日后只会走这条线了。”

胖子眼神深沉:“如果当真成事……恐怕三五年内,轨道和有轨马车,就会遍及天下各路。”

“不知运费是怎么个算法?”老行商问到了一个关键性的问题。

胖子和他的同伴都竖起了耳朵,静听回答。

本地商人明显是打听过了:“听转运司传出来的说法,从鄂州或是江陵发往京城的货物,连运费带过税的总花费,要保证在从扬州发往京城的同色货物的三分之二以下,尽量让荆湖、蜀中的货物,不会比走汴水更贵。”

“当真?!”胖商人和他的同伴又是异口同声。

“这还能有假?”本地商人对两人的质疑有些不满,“你们看着就是了。”

老行商眯起眼睛:“如果税费加起来当真只有汴水的三分之二,算上节省下来的时间,还有长江上的一段开销,恐怕荆湖和蜀中的商货,日后只会走襄汉线了。”

“那不是当然的!?”胖商人兴奋的搓着手,“蜀中运到京城的货物,不论是药材还是绸缎或是花果,运费一律比本钱都髙。若是能剩下个三分之一……不对,从蜀中运出来费用的加上,能少五分之一。运一万贯的商货,运费就省了两千贯……”

他举着两根胡萝卜似的手指,比划来比划去,下巴上的赘肉直抖着,盯着停在码头边,被人围起来的那列有轨马车,两眼直发光,仿佛上面满载着的不是纲粮,而是一枚枚闪闪发亮的簇新铜钱。

贪婪的眼神狠狠盯着有轨马车好一阵,转回来,看着为他们解说了半天的本地商人,“啊,对了,还不知道老兄贵姓。”

“不敢当,免贵姓王,周吴郑王的王,做些针头线脑的小买卖。敢问兄台贵姓?”

“小姓李,木子李。这位是在下表兄,与老兄同姓。现在是在荆湖贩米,但襄汉漕运既然开了,也有打算去京城走走。”胖商人作揖道,“今天可是多承老兄相告,帮了大忙了。”他看了有轨马车的方向一眼,“看起来还有些时间,小弟做东,找个干净的酒家喝上一巡,不知老兄可否赏光。”

王商人推脱了两句,就点头答应下来:“在下就厚颜叨扰了。”

李姓的胖商人转过来又看向老行商,“不知老丈高姓,可愿同去小酌。”

老行商拱了拱手,道:“老头儿姓路,路明。相逢即是有缘,得君相邀,不当推辞。同去,同去!”

