\section{第九章 旧日孤灯映寒窗(下)}

也不知是因为什么原因,礼部试这一日的公务,莫名的比前几天少了许多。

到了中午的时候,韩冈的午餐端进来的时候,至少可以直接放在桌案上了。

作为参知政事,每个月有三十五贯的餐钱,比不上做宣徽使时的五十贯多。但去掉每个月休沐的那几天,平均一贯两百钱一顿饭,只要韩冈想吃,正常的一二十道菜都不会有问题——开封的酒楼,只要不是天南地北的特产,酒菜的价格都不贵。

只是韩冈吃饭,相对于他的身份还是清简得很,普通的两菜一汤,饭里都是添了些许糙米、杂粮,并非碾了又碾的精米,以吃完为上。到了他这个地位,更注意的是养生,对暴饮暴食敬谢不敏,烈酒更是涓滴不沾。

不过韩冈另有一重身份,尽管从来不会施针开药,可在养生上说什么都会有人信,见韩冈如此饮食,才几天功夫,韩绛、张璪都开始学着韩冈这样吃饭了,还让韩冈院中的厨房传了一份菜单过去。

韩冈对此也只能是付之一笑。

随便吃完了饭,喝着消食的饮子,他顺手抽出一部新送来的韵书,慢慢翻看起来。

不过韩冈看的并不是常见的《礼部韵略》,书册单薄了许多,但里面的文字也印刷得细密了许多。

《礼部韵略》类似于后世字典,全部文字的顺序,则是根据韵部来进行编排,也就是以韵母为主的排列方法。诗词歌赋是否押韵,必须以《韵略》为凭。换在朝廷还是以诗赋取士的年代,每一次进士科开考,考生们都会得到一部刚刚印好的《韵书》作为诗赋的标准。

韩冈手中的韵书,比起《礼部韵略》,多了部首编排查字,在句读上也学习《自然》等气学书籍,加了标点符号,还有着释义,并列出了以其为词首的常用词。

只是在声韵上,还是以韵母为顺序,比起后世以声母为顺序的字典,依然有着很大的区别。

这是来自横渠书院新编的《常用字字书》——不敢以‘典’为名,只能名为字书。在很大程度上,接受了韩冈提议,不过改不了旧日韵书的印象,所以有了这个四不像。

在韩冈看来,这部书并不合格,还需要经过多次修改。若是这部字书当真能达到,韩冈记忆中那部几乎每名学生都拥有的袖珍小字典的水平,恐怕今科考试的士子们,都少不了会人手一本。

当然,今天贡院中的考试,既不会有字书,也不会有韵略。

今科考试的时间,比往年稍迟了一点。

九年前的这个时候,韩冈已经走出了贡院的考场,等待着曾布、吕惠卿等人批阅的结果。

当时韩冈颇用了些盘外招,费了不少的心思,这才与来自天下各路的一众贡生,站在一条起跑线上。

最后通过礼部试时,不上不下,一点也不引人注目。

想到当年参加的考试,韩冈也就一并想起了一同上京赶考的旧年同窗慕容武。

慕容武已经是凤州通判。但仅仅是第一任的通判资序,想要成为韩冈的助力,还差得远。

其能力也算不上太出众,能很快的升上来,还是因为他在郿县知县的任上兢兢业业的缘故。

因为张载及其父、其弟的坟茔就在郿县,所以郿县的几个官职就是气学的自留地。从知县开始,县丞、县尉、主簿,都是气学门人。县学中的教谕,也是一样是气学门人——只要韩冈还在一日,他的面子足以抵得过区区一县的几个职位——而张载的独子张因,正在横渠书院中读书。

在那座规模越来越大的书院中,常年有着超过三百名士人在内学习,在易于出行的春秋二季,学生的数量更是能够膨胀到一两千人之多。

韩冈眼下正建议横渠书院模仿国子监的制度,再稍稍加以改变,分成初中高三级,以对应不同水准的学生。

至于老师,这两年就从没少过五十人。大部分是留在书院中的气学弟子,一小部分是资深的学生兼任,加上时不时特邀名儒来书院中宣讲,让书院的影响力越来越大。

从规模上,目前横渠书院仅次于国子监,是为天下第一书院。

同时横渠书院由于不断得到捐赠,在郿县及其周边各县,横渠书院有超过四十顷的田地,已经成了凤翔府最大的地主之一。在其名下,还有十一座风磨坊,每年的收入不在少数。另外书院还将院中师生们编纂的各色书籍交托印书馆印制发售,还能得到一部分分红。

有了这些收入补贴,不仅能够让书院中寒门士子不用忍饥挨饿,可以安心读书,也让书院更有吸引力。

每次看到书院的变化和发展,韩冈都不禁感叹,他的师兄苏昞,作为书院山长的确是劳苦功高。

横渠书院是韩冈计划中极为重要的一环,代表着气学的未来。

但数学、物理学和化学等方面的进步,才是韩冈对横渠书院的期待,这不光是人多就可以的。

如果是对外,在不能用笔和嘴来说服敌人的时候,只要用上大炮就没有问题了。

火炮的威力会让一切反对声平息,如果做不到,那就代表威力还不够,需要口径更大、炮弹更重、射程更远的火炮。

而在对内时,大炮也是学术之争上的凭据,是证明气学优点的证据。要想压倒对手,同样需要口径更大、炮弹更重、射程更远的火炮,以证明气学的功用。

经世济用。

气学想要扩大影响力,成为一门显学,离不开这四个字。

自家寒窗苦读的辛劳,仿佛就在昨日。而现在已经要指导学生们攻读的方向。

时时都在关注着横渠书院内部一举一动的韩冈,知道他的根据地虽然很缓慢,但的确是向着他想要看到的方向在前进。

书院中的数百上千名士子,日夜苦读的内容,并不局限在科举的项目中。

尽管这一科,包括下一科,再下一科,从礼部试出来的新科进士里面,不会有多少气学弟子的身影,但日后朝堂之上,气学弟子必然会因为他们的才干而走上高位。

而且以韩冈现如今的地位,还有日后几十年盘踞朝堂的时间,也绝不会是白白看着新学垄断着进士资格。

就比如明天就要开始的阁试,韩冈就不会坐视新党刁难他要重用的人。

韩冈不知道黄裳对阁试有多少把握。

在心理上,尽管黄裳在他面前从来不会表现出慌张和不安,但在面临如此重要的关头前,黄裳不可能不紧张。

可是在学问上,韩冈还是愿意相信黄裳的自信。

也许前世记忆中的状元,就像韩冈的进士第九,是天子直接从榜尾提上来一个样。黄裳的状元也有可能是当时的皇帝看着顺眼,所以在礼部试和殿试上的名次并不高的情况下,行使了特权的结果。

不过连续多科南剑州解试位居前列的实力是毋庸置疑的,考不中进士也只是运气。或是状态不好,或是题目不对。

黄裳想要通过阁试,最后就只是题目的问题。

在很大程度上,考生们的命运就决定在考官身上,一方面是考官出的题目是否在自己准备范围内,另一方面,自己辛苦完成的文章能不能得到欣赏,决定权也全都在考官们的手中。

阁试考试的范围,九经、诸史、武经、诸子,加上注疏的内容,文字数量就是数以百万计,不可能有多少人能够将注疏都一股脑的背下来,他们能够做的,是记住其中绝大多数的关键内容,以及经义本来的要旨,剩下的就看会不会运气不好,撞上自己记不得出处和内容的考题。

就算是苏轼、苏辙,他们能通过阁试,都有考官没有刻意刁难的因素在。渊博如欧阳修,都能对苏轼杜撰的典故不敢轻下结论,苏轼、苏辙难道能比欧阳修强出许多?

韩冈不知道黄裳会遇到什么样的题目,也不想知道。怎么出题才能让王安石满意,又不开罪自己,这是崇文院中人需要考虑的。韩冈要看的只是结果。

制科不会像进士科举一般的锁院,但必要的隔离还是少不了。出了题之后的几日,参与出题的几名三馆秘阁成员,都要被约束在秘阁之中,直到开始考试为止。不过这样的制度,远比礼部试要容易钻空子许多。一众考官,更加容易受到场外因素的影响。

但韩冈要的只是一个公平的机会,并非特意的照顾。

党争虽然已经是个现实性的问题,可韩冈并不觉得要不择手段的去体现党同伐异四个字。

如果在没有人下绊子的情况下,黄裳不能通过阁试,那就是他自己的问题,韩冈也不会因为他被黜落,而针对那些考官下手。

拿起一张夹在《常用字字书》中的纸片,韩冈看了一阵,最后摇头一笑,随手便丢进了盛满水的笔洗中。

草草写了几行字的纸片只有巴掌大,在笔洗内很快就湿透了。韩冈再拿着笔杆搅了一搅,便烂做了一团,再也看不清上面的字迹。

这样就行了。

韩冈想着。
