\section{第30章 众论何曾一(八)}

二月春风似剪刀。

道旁、河边的柳树枝条,在变得温和起来的阳光下有了融融嫩绿。片片新叶随着新生的柳条于微风中,如丝一般飞舞。于柳树一样,杨树、槐树等树木也都在春风中

只是深植于土中的树木能顺利发芽,但更浅一些的花草却是与地里的庄稼一样枯黄干萎。除了一株株生出嫩绿枝叶的乔木外,茫茫大地之上难见春色,二月的暖风带起的不是春意,不是花草香,而是劈头盖脸的沙尘。

天是灰蒙蒙的,泛着让人感觉着压抑烦闷的土黄色。抬头向上,高悬在天顶的太阳都在灰蒙蒙的云翳中变得有些模糊。

叮叮的铃铛声中,一行马队从灰蒙蒙的雾气中走出来。在视线恶劣的天候下,马队走得很慢。队伍中人人披着斗篷,甚至其中有几个还戴着口罩。

口罩本是韩冈所创的疗养院中医生动手术时所用。去年,当曾经在关西得到韩冈教诲的太医局医官雷简,奉旨在东京城中开始设立了疗养院,医护制度也随着他一起传到了京城。而疗养院中所用的器具,不知是何时已经在京中流传开来,其中就包括口罩。

在灰尘弥漫的日子里,东京城的大街小巷中,已经可以不时的看到戴着口罩,匆匆而过的身影。而在城外的道路上,骑在马背上的骑手戴着口罩的比例则更高——避尘的帷帽在高速疾驰时,很容易被吹飞,远不如口罩实用。而且一般的男子也很少喜欢戴着帷帽这等女人多用的玩意儿。

不过曾布没有带口罩,他不习惯在嘴上罩了几层细麻布的感觉。侧头避过迎面来的灰土,他开口问道:“究竟还有多久才到白马县?”

紧跟在后面的从人拍马上前:“回学士的话,刚刚过了界碑,现在已经是白马县境内了。”

吕惠卿抬手将口罩扯下半截,笑道:“子宣何须心急?仲元方才也说了,最多两个时辰就能看到县城了。”

王旁低头骑在马上,保持着沉默。倒不是因为跟在两名当世难得一见的俊杰身边,给他的压力很大。而是他昨夜没有睡好,今天上路后就没有精神。

曾布和吕惠卿奉旨出京,和王旁同时出发。不过曾吕二人是去河北相度市易、并察访灾情。而王旁是要去白马县,仅仅是顺道同行而已。

从京中往黄河这边走,沿途几县的情况都很糟。京畿一代的土地一向肥沃,但眼下看到的情况却不能不让人担心。麦田中完全看不到绿色,只有与大地一样的灰黄。可以看到有许多农夫,愁眉苦脸的挑着水在田头间走着,也有已经在田头站定,拿着瓢向地里泼水。只是用水桶挑水浇灌田地,根本杯水车薪,干裂的土地就向渴极了的喉咙,水一泼下去,眨眨眼就不见踪影了。

不过到了白马县这一段后,路边的田地干旱如前,但百姓们取水浇田却是很方便。很多都是上下摇着一根木杆,然后不断的有水流出来,虽然出水不多,但胜在细水长流,不像木桶下井提水,慢悠悠的才有一桶水上来。

但也不尽是从井中直接提水的,也有些田地并不靠着水井。可那些田地,也都能看到一队队农夫从远处挑着水过来,将一桶桶水放在田头,守在田头一群老弱便就着桶中的水,同时开始浇灌着一块地。一瓢一瓢的不断的将水泼洒到地里,很快就将这片田地给浇透,然后就改去浇灌另一片田地。

从田间阡陌上竖着的的界碑可以看出,几片田并不是一家。可那一些浇田的男女老幼却不分你我,一视同仁的浇灌着田地。如果仅是一片地如此,还可以说是当地百姓自发组织起来互助。但随着逐渐接近白马县城,吕惠卿和曾布所看到的每一片地,都是多少人一起出来同时给一片地里浇水。

“韩玉昆治事之材的确让人惊讶。”吕惠卿做过地方官,知道组织百姓互相帮助有多么麻烦:“能上任七天就将三十年的积案断明白,才智之士果然是不一样。”

吕惠卿知道曾布不喜韩冈的行事风格,但他在曾布面前却不会为此少赞半句。

吕惠卿戴着口罩还如此多话,让曾布微微皱了皱眉,然后只顾着看着田间地头的农事,却半个字也不回。

王旁却在旁则有些骄傲地说着:“眼下还没有利用起风力,如果能将风车安到水井上,以风汲水,就可以直接让水从沟渠中流进地里,如此一来就不需要这么多老弱出来操劳了。”

王旁靠着父亲和兄长,在京城中找到了两名能够打造风车的木匠,现在就跟在队伍中。其中一人还是国初名匠俞皓的四世孙,乃是祖传的木匠手艺。

想那俞皓,担任过朝廷的都料匠,世称俞都料。有着三卷《木经》传世,是如今的木匠打造楼台宝塔的必备书籍,在大宋的匠师中,乃是公输般一流的人物,甚至有人直接就说他是鲁班转世。

京城中,高达三十六丈、于庆历年间被焚毁的开宝寺木塔,就是俞皓一手督造。当年开宝寺木塔修起来时,向着西北倾斜。人问其故,俞皓说京城多西北风,现在虽然向西北倾斜,但百年之内就会给吹正过来。而这座塔被焚毁时,塔身则已经被吹正,且离着建起的时候,却正好一百年。

曾吕二人都知道韩冈的打算,也知道今次王旁带了什么出来。为了解决旱情,如韩冈一般费尽心力的知县当真是不多见,为了浇灌田地,一口气在县中开了上百口井的传言,在京城中也能听到。从宫中传出来的消息,天子赵顼对此还多有褒扬,赞着韩冈公忠体国,堪为亲民官之表率。

远远地看到一队人从前面迎过来,只看队列,也算是严整。一名身穿绿袍的官员一马当先,王旁眼尖,一看到来人就扬起了手:“是玉昆来了!”

迎客的韩冈,还有作为客人的曾布、吕惠卿还有王旁,互相见礼过后,就一起往着县城中去。

韩冈总觉得曾布和吕惠卿突然间一起被派出来有些不对劲。对于天子的这项任命,他有一点不好的想法。两人是王安石最重要的助手,现在一齐遣出在外,京城中的王安石身边可就是孤木难支。想想如今正在朝堂上纠缠的事,说不准就是赵顼为了保下粮商们,先从王安石身边削了人手。或许还有可能,是想让王安石和他的同党看一看他们治下的河北是什么样,好让王安石自己辞相……

韩冈这般想着,又暗暗的摇了摇头。也许是自己太过于阴谋论了,也许只是天子赵顼单纯的信任曾布和吕惠卿,认为他们能将事实不折不扣的汇报上来。

曾、吕二位要过境的消息,前两天就传到了白马县,故而今天韩冈一大早就出城来迎接——中间也顺道看了一下沿途几个村子抗旱的情况——无论是临时派遣的察访使,还是惯例的路中监司巡视地方,都会派人事先通知途经州县。如果没有通知,突然冒出来一个官人,查验真伪都难。

韩冈一路上与三人说着话,感觉曾布与吕惠卿之间的关系有些微妙。但韩冈也能理解,两位如今地位渐髙,瑜亮之争肯定是免不了的——尽管东京城中的桑家瓦子说三分的先儿很有名气,但韩冈只是在第一次上京是去听了一回,也没听出个门道——不知道这时候三气周瑜的段子有没有出现。但既然日后苏轼写词赞过周瑜,多半还没有流传。

说实在的,韩冈有时也有恶作剧的心思,想着提前将一干名篇,用着匿名的手段在寺庙或是一些名胜之地写上去。虽然他对于那些名篇都已经记太不清,但重要的词句还是记得很牢固。只要提前写出来,如今在杭州快要任满的那一位可要吃个闷亏。不过想想还是算了,苏轼这几年都在外面,也算是吃了苦头,没必要再落井下石。对于这位留名千古的文豪,韩冈还是保持着一分敬意。

骑着马,很快就看到白马县的城头,而在城池之前,就是一座刚刚搭建起来、被一圈土墙围起的流民营。

吕惠卿在马上直起腰,向营地中望了一阵,回头过来道:“听说玉昆已经在县中设立了四五处流民营。有此布置,想必河北流民南来后,介甫相公也能安心了。”

韩冈正待谦虚,曾布却道:“河北流民数以万计,不知玉昆你有没有足够的准备。”

“流民之事暂时还不必担心。”

“看来玉昆当真是胸有成竹了!”吕惠卿笑道。

“呵呵。”韩冈自嘲的笑了两声,“不是相信自己,而是相信黄河。”

曾布和吕惠卿闻言皆是噗哧一笑:“原来如此。”

王旁疑惑不解,但看着曾布、吕惠卿一听就明白,也不好意思将自己的迟钝

韩冈瞥了一时没有反应过来的王旁一眼,回望着前方叹道,“现在的黄河已经开始解冻,冰面开裂甚多。原本冰上的道路三天前开始就不能再通行,但河上想要走船至少还要半个月的时间。差不多要到二月下旬之后,才是流民大举南下的开始。”说着,韩冈再看了看曾布和吕惠卿,“学士和检正要想过河在白马渡是不可能了,要向东北绕道过去。”

“当然。”吕惠卿点了点头,“路程本来就是这般定的。”

