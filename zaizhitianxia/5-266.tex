\section{第29章 浮生迫岁期行旅(六)}

司马光在殿上口口声声要杀王珪,而且坚持不改。若是当真给他成功了,洛阳的一干元老可就要人人自危了。虽说整件事是他的运气不好,撞上了没经验的皇后,可在富弼这边看来,司马光还是太过分了一点。

“那儿子这就去安排礼物和人手了。”富绍庭应诺,抬头后随口又笑道:“明日司马君实回来。过几日吕晦叔当解职出外,不知道会不会也被调来洛阳。到时候,又要准备一份礼物了。”

“还是别来的好。”富弼脸色忽的一沉,“有一个文宽夫已经够多了。”

富绍庭愣了一下,回过神来才明白,原来当年的旧怨一直还在父亲的心中耿耿于怀。

连忙告辞离开还政堂,富绍庭才长舒一口气,他父亲跟吕夷简之间的怨恨,甚至比对韩琦的芥蒂还要深个三五分。

当年辽人兵胁河北河东,富弼奉命使辽,仁宗皇帝在殿上一条条的将谈判内容吩咐下来,宰相吕夷简在侧旁听,也参议了许多。可之后政事堂开出的国书中内容却与殿上的商议内容截然不同。幸亏富弼存了小心,离城后就开了国书看,一见不对,当即掉头回宫找场子。在仁宗皇帝面前,大骂吕夷简要害死自己,以私心坏国事。但仁宗不愧那个‘仁’字,在中间打圆场和稀泥,大事化小小事化了,换了国书就将富弼打法上路了。

因为这件事,富弼对吕夷简衔之入骨。对吕公著、吕公弼兄弟,平日里笑呵呵的,往来礼数不会少,还说不少不记恩仇的好听话,但眼下看过来,却是半分亲近也没有。

富绍庭暗叹了一声,旧党元老们几十年的官做下来,之间恩怨甚多。要不是有个王安石,大张旗鼓的提携新进,逼得他们不得不合力。哪里会笑嘻嘻的坐在一起,早就撕破脸皮了。当年司马光跟着欧阳修、带着御史台,将张方平揪着往死里打,现在还不是书信往来。

记得去年司马光会合六七耆老,开真率会,会于名园古寺之中。果实不过三品,肴馔不过五品。一切以简俭为上,挺符合司马光的性格。但文彦博偏要凑热闹,一日带着几席酒菜直抵会场,司马光不好赶人,但之后司马光说了什么?‘吾不合放此人入来。’这是富绍庭听楚建中提起的,也不知有没有传到文彦博的耳朵里。

富绍庭自知才智不高,父亲富弼对自己的要求只是谨守门户四个字。但对于洛阳的一干元老宿旧,就在近处看得久了,也知道天底下的乌鸦都是一般颜色。

不过他立刻就不敢再想了,再往下想过去,可是把自己老子都绕进来了。

但富绍庭也不能不多担一份心,如今有心人闹得谣言四起,弄到最后,别把富家也给绕进来!

他有些担心的向东南方望去,是不是将还在嵩阳书院的侄儿叫回来,年轻人可是最容易受到煽动了。

……………………

嵩阳书院。

创立在北魏年间的这间书院,因为靠近洛阳,自然而然的就成了旧党培养新生代的地点。

二程自不必说,司马光也常来此授徒,吕公著当年也曾在开讲过。文彦博、富弼以大笔的资金支持,两家的子弟也有来此入学的。

对于新党,自然是恨之入骨,对于新法,也是众口一辞。

眼下旧党大挫,在嵩阳书院里,就像火星落入了柴堆之中。

“自真宗以来,南人进士渐多,北方进士则越来越少!”

“关西不用说,灌园子的进士第九,几十年来已经是最高了,而且还是得天子赐。司马十二的名次跟他差不多。可怜了,其他人有入一甲二甲的吗?!”

“开封府解试入选比例虽高,可其中又有多少是移籍冒籍的?”

“所以眼下恶果便在此处,南人盘踞朝堂,而正人不得与列!”

“奸佞当道,蒙蔽圣聪!”

 “什么蒙蔽圣聪?就是给王安石那个奸佞给囚禁了!……”

嵩阳学院中的大厅中,越来越多的学生为新党的得势而愤怒着。

前段时间,冬至夜的消息传来后,在书院中,对韩冈的作为颇多人予以赞赏,毕竟太后、雍王那种迫不及待等着天子驾崩的心思,实在是表露无疑了。母不慈,弟不恭,能只用皇后垂帘,而不彰显其罪,已经是天子孝悌的表现了。

可当司马光、吕公著在同一天内倒台,立刻就有很多人开始抨击韩冈,不过还有不少人站在韩冈这边——主要是一干洛阳元老家的子弟。他们跟寒门出身的同学不同,司马光要杀宰相,已经触犯到他们自己的安危了。

而且韩冈的质问,连司马光都回答不上来的问题,还有几人能拿着刑律给其定罪。

硬说王珪之奸罪该论死,怎么也说不通的。三旨相公的绰号,代表他一切都以神宗的意志为依归,这是过去人人嘲笑的,他若是有什么错处,说句难听的,天子都逃不过去。唯一能批评的,就是他为人不正,不能尽到宰相的本分。

难道要说请立太子上他没有尽到宰相的本分?可迟了一点不能算是罪名,做和没做是性质问题,而迟和早只是顺序有别。若请立太子也是罪名,那么还能批评擎天保驾的韩三吗?

除了一部分人以外,其他人都对此沉默了。

只是台上尽数新党,而旧党一个不留,还是在许多人心中压下了一团火。当几条新的流言不知从哪里传出来后,顿时就引爆了局面。

“吕相公不肯与奸人合作,所以被赶出了朝堂。如今朝中豺狼当道,正人皆尽出外!”

“灌园子沽名钓誉,辞参知政事,辞枢密副使,但谁人不知他是王安石帐下走马狗?!”

“吕惠卿、曾布、章敦,群小汇聚,天子为其所囚,试问天下正人可能坐视!?”

吕大临在旁听着直摇头,与游酢一同从喧闹的厅中出来。

“先生那里会不会有事?!”吕大临有些担心。

“师道之严,谁人敢于触犯?”游酢虽然这么说,但还是担着心。与吕大临一同到了后方小院,发现一切如常,这才松了一口气。

程颢程颐在内,两名学生进厅后,先行了礼。

“现在外面流言汹汹,伯淳先生还要去京城吗?”吕大临问着程颢。

“当然要去。”程颐抢着便说,“论断是非,岂能从与流言?大兄不亲眼去看一看,从何得知真伪?”

“流言是一桩事,但资善堂中,有王安石和韩冈在列,先生纵有满腹才华,身怀正道,也恐难施展。”吕大临很担心,在如今的流言下,程颢接下了这个位置,等于是公开说站在了新党一边,成了众矢之的。没看现在司马光的弟子已经发了疯吗!

程颐眉目一挑:“自反而缩,虽千万人吾往矣。与叔,纵有千难万阻,又岂有畏难避道的道理?!”

吕大临欲言又止。游酢暗暗摇头,这时候还说什么,大程先生都已经领了旨了。

五天前,诏书就送到了洛阳程府。以程颢为资善堂说书,同时还在三馆中安排了一个秘阁校理的差事——不是加衔的贴职,而是真正要做事的馆职。

为太子师,又是清贵之位,如何能放弃?这可是道学跳出洛阳,走向全国的难得机遇。

游酢是福建人,对于方才厅中的地域之争听得就不舒服。而且他的兄长游醇还是韩冈的门客,被举荐上去为官。之后便脱离了福建千军万马过独木桥的解试,由锁厅试顺利入解,在元丰二年考中了进士,对本身并非贵门的游家来说,恩德甚重。

原本就韩冈一人侍讲资善堂,现在却加上了王安石和程颢,皇帝打压韩冈的想法,其实是很明显的。纵然批准了三份奏章中的两份,又修改了针对千里镜的禁令,也不过是找平衡罢了。

吕惠卿与韩绛失和,曾布还是新党的一员叛将,明眼人都看得出来,皇帝还是在异论相搅,不过是换成了新党内部加上气学的韩冈。

游酢道,“韩玉昆一心想光大气学,只看其三疏,便知其心,终究不是跟王介甫是一路人。先生入资善堂,他不至有所不敬。”

“子厚先生的气学,早就给他带入歧途了。”吕大临冷然道,“他争的岂是横渠之学,乃是他一家之学!”

游酢无奈一笑,韩、吕之间的恩怨,他可不敢掺和。

“先生!先生!他们……他们……”一名程颢的学生上气不接下气的跑了进来,惊慌失措的样子让程颐看得直皱眉。但游酢觉得不对劲了。

“不要急,慢慢说!”程颢道,他也知道事情变得更糟,但慌慌张张就未免太过失态了。

那学生喘了几口气,正要说出来发生了什么,门外又冲进一名学生,大叫道:“先生,先生,他们要去京城叩阙上书!”

这一下,即是程颢程颐都没办法安坐了!

“是谁在煽动?!”

“是邵子文!”


