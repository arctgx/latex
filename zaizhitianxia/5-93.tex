\section{第11章 城下马鸣谁与守(五)}

“终于来了。”

随着盐州城中的守军主帅一声满怀喜悦的轻叹,数以千计的骑兵,出现在从瀚海的方向上,直奔盐州城而来。

人马汹汹,犹如洪流,掀起黄沙滚滚,直漫天际。

千军万马在荒原上的奔驰,震动着大地。传到城头,高永能在雉堞上,就看到一些细小的沙砾随之震颤。

“西贼来势汹汹啊……”徐禧携城中众将登敌楼远眺城外敌军,指着里许外的一队队缓缓逼近的敌骑,纵声大笑:“不意其前来送死竟然这般心急。”

高永能只扯了下嘴角,几个来自开封东京的将领则是嘿嘿的陪着笑了起来。

在他耳边,是主帅自信满满的豪言:“西贼摆开如此阵势,换作是十年前,烽火能一直传到京城中。不过放在今天,也是群土鸡瓦狗,只能吓唬一下黄口小儿。”

来自京师一众将校与主帅一样的兴奋。

“学士说得正是。西贼竟敢来攻盐州城,岂不知徐学士就在城中。”

“有徐学士在,当杀他们一个片甲不留!”

高永能冷眼旁观,附和和反驳的心思都没有半点。

西贼并不是直接就往盐州城撞上来的。从昨日开始,以白池堡、乌池堡为主盐州外围的一干据点,便全数陷落,无一留存。逃回来的守军只有小半,剩下的则都是没于军阵中。

早在开战前,高永能就提议过撤回外围据点的守军。但徐禧却不同意,他振振有词,缺乏外围据点的支持,一座城池再坚固也只是孤城。

这的确是兵家正论,但得看看具体的情况。这些据点驻屯的兵力数目太少,在战局上发挥不了什么作用,留下游骑在外作为耳目就够了。

而且虽说驻防在外的兵力不足以用改变战局,可作为人头来砍的话,就未免太多了一点。尽管放在外面的每一处兵力都不多,但十几处加起来,也足足有一千人!

昨日徐禧对此就只说了一句,‘此辈固我寨防,临阵不屈,待凯旋回京后,当为其请功追封。’

而高永能则更愿意他们活着。

高永能远观敌方军势,除了几支数目不到两百,却行动灵活的骑兵往城下奔来,其他的西贼到了两里外就停下来了。在他们的驻足之地,有尘云盘绕,显然是在不断的调兵遣将。

悬在敌楼上方的飞船,正不断传下西贼后方的人马调动。一张张简短的纸条顺着细绳从吊篮上滑下来,在城头上的徐禧和诸将手中传递。

高永能看过最新的一份敌情,西贼开始分兵去堵住从盐州城延伸出来的其他道路。他眉头皱了一下,就上前向徐禧提议:“学士,西贼初来乍到,立足未稳,却又分兵想堵上其余几门的道路,正是用兵之时。还请学士选调城中精锐,出城冲杀一番,给西贼迎头一击。”

徐禧回头瞧了高永能一眼:“当以堂堂之师临堂堂之阵,岂不闻王师不鼓不成列。”

高永能看了看城下,心想干脆从这里跳下去算了,死得干净点,省得最后憋屈死。

他望向曲珍,用眼神求援。可年过古稀的老将,这时候沉默得像一棵树一样,树皮一般粗糙的脸上,连一丝表情都看不到。

曲珍前两天还在劝徐禧,不要留在盐州,身为主帅,坐镇后方就足够了,否则事有万一,连个督促援救的都没有。

这是曲珍不想徐禧在前方碍手碍脚所找的借口。在曲珍看来,如果是他来领军,如果没有这个扯后腿的徐禧,保住盐州至少还不能算是梦想。

曾经在京中做过三衙管军、担任过神龙卫四厢都指挥使的曲珍,只要徐禧不在,就能自然而然的接收盐州防务,可惜徐禧偏偏不肯回去。

徐禧在官场中多年,曲珍想取得前线指挥权的想法,他洞若观火。只是在他眼中,这是曲珍妄图与他争功的明证。所以徐禧反过来咬文嚼字的嘲讽道,‘曲侯老将,何怯邪?’说曲珍找的借口,却显得他胆小如鼠,何须惧怕西贼。

想来曲珍一刀将徐禧砍死的心也不缺,高永能想着。徐禧说什么堂堂之师的蠢话,可就是把自己和曲珍看成一派,故意来堵自己的嘴。

京营的将领看笑话,都是人精,哪能不知道,徐禧这是在故意敲打曲珍和高永能。但除了他们之外,却有一人觉得徐禧的对话不对劲:“学士。舜举服侍天子,多曾听天子说起用兵当奇正相辅……”

就在今天早上才冲进盐州的天子特使,这时候也在城头上。李舜举拿天子做大旗,徐禧也不能把他当做曲珍来对待。

“都知放心,若无狡计可用,正面相抗,西贼如何能胜我官军?”徐禧远望城外敌军,“而且西贼远道而来,定然最为提防官军,这时候出阵,必然是无功而返。得等他们松懈下来。”

不愧是说服了天子和参政的口才!

高永能心口被气得疼。他祖上是从马姓改了宗的吗?还是说名字里面有个括字?真不知道皇帝和吕大参怎么会信用这么不靠谱的措大!

城中三万将士坐视只有三分之一的敌军围城,这个士气怎么办?

但李舜举似乎被说服了,点点头,又安安静静的站着。高永能就只在喉头里咕哝了一下,没有将话说出声来。

李舜举除了忠心,并没有什么其他方面的才能。天子将他派来盐州,名义上是体量军事,实际上应该有在关键时阻止徐禧的任务,拥有拉住徐禧笼头的权力。只是他没有运用这份权力的能力。

在世人的眼中,李舜举远不及永远都是在福星照耀下的好运的王中正,也不及号称内侍知兵第一的李宪,相比起蓝元震、石得一、宋用臣这一干大貂珰,李舜举的能力都还差一点。

只是作为一名内侍,忠心就是最大的长处。比起其他身居高品的宦官,李舜举永远都比他人更加接近天子。别人兼程赶路,都是一日走上两日的定程,但李舜举却是一日走上三程甚至四成的路,只用了九天就赶到了盐州,忠心王命可见一斑。就是能力不足,胆略欠佳,却让徐禧更加得意猖狂。

来袭的党项军已经在五里地外开始扎营了,徐禧还带着将校在远观军势。

一直沉默着的曲珍,这时候转身就往城下去,高永能一见,便追了上去,在背后叫了一声,“太尉。”

曲珍回过身来,“你那边粮食够吃多少?”他直接了当的问着。

高永能愣了一下,然后答道:“……杀了马也就二十天。太尉你那里呢?”

“一样。”曲珍很简洁的回答,没心情多说一个字。

在阻卜骑兵出现之后,党项兵发盐州的战略目标得到确认,盐州城除了加紧运送粮草,也开始疏散多余的民夫。但在不断出没的阻卜人的骚扰下,粮食储备并没有达到预期的数量。

而且前面为了加快筑城的速度,调集了三四万民夫同时开工,现在听说党项人将至,就赶着将他们都发遣了回去。但在最后的一段时间,为了让他们加急赶工,饭都是让民夫们敞开来吃,粮食还能剩多少?

高永能所说的二十天,包括了他麾下五千兵马一开始就私留下来的一部分存粮,加上盐州城明面上分派给他的粮食储备,再配合上战马等牲畜作为补充,最后计算出来的时间就是二十天。

二十天,对于一场战役来说,其实不算短了。

城池攻守,打个一年半载的的确有,但绝不是在西北。党项人拼不起消耗,三五日攻不下来,通常转身就能走了。而宋军要攻城,手段则多如牛毛,党项人基本上也防不住。

但放在盐州这里,曲珍和高永能都知道,很可能会出现一个特例。事关银夏之地的得失与否,党项人会咬着牙打下去。如果能比党项人拖上更多时间的话,倒也能捱得过去。但他们既然气势汹汹的来了,想必是做好了准备。

高永能叹了一声:“这仗可怎么打?环庆路、泾原路都指望不了,难道要等种谔来救援吗?”

“也要种五愿意!”

高永能点点头:“在出兵之前,西贼不会不考虑援军的问题。恐怕他们有充足的把握。”

曲珍的眼中满是冷漠,声音更冷:“盐州城中的粮食多寡,西贼多半已经计算清楚了。”

“前两天徐学士还说了,吴起领军,上下饮食起居如一。能与卒伍同饮食、同起居,方可为将!”也就从那一天开始,徐禧每天就只吃两个炊饼,早上吃了一个,剩下一个放在怀中,到了晚上吃。在徐禧的带动下,所有的将校都是两个炊饼垫肚。高永能摸摸自己的肚子:“换做我是兵,倒想要一个天天吃山珍海味、不过也能让下面的兵将一起吃饱的主帅!”

纸上谈兵。对兵法只知其一不知其二。光会做做样子。这些批评曲珍都懒得说,转过身,往城下走。

高永能在后面问道:“团练要回去歇着?”

曲珍头也不回:“徐学士不是说,要以堂堂之兵,临堂堂之阵吗?老夫去筹备他说的堂堂之兵去。”

几步下城,上了马就往本部所在军营的方向去,转眼就去了远了。

高永能回头看看敌楼,又看看曲珍的背影,最后叹了一口气,摸着咕咕叫的肚子,回头往敌楼走去。

