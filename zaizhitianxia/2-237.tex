\section{第35章 重峦千障望余雪(二)}

对于韩冈能把一群叛军指挥得奋死拼杀,赵顼是赞赏不已,但对这些叛军的赏赐,却让朝廷伤透了脑筋。

“可以厚加优抚,至于官职,那是决不能封!”王安石作为宰相,拍板定案。

对于王安石的这项决定,文彦博也没什么好说的。以刘源为首的广锐旧卒表现出来的战力,已经让朝堂诸公都感觉着棘手,绝不会让他们复官,否则他们再起叛心,谁都担不起这个责任,只能选择用田宅钱钞来满足他们。

“可照秦凤缘边安抚司的提议,赦了有功之人的过往罪由,让他们的子孙后代可以入军中博一个官职,只是必须留在通远军,不得回迁。”冯京作为参知政事,也站出来表现自己的存在。

风姿秀挺的金毛鼠,与脖子上生了个肉瘤的吴充站在一起,有着鲜明的对比。倒是上首的王珪,相貌并不必冯京差上多少。

“只是一旦赦了罪之后,恐怕他们都不会再如今次一般用命了。”

赵顼的忧虑,一众臣僚没一个接口。这群叛军,用一次已经够麻烦了,谁还敢用第二次?!

臣子们的沉默,让赵顼心中不快,微微皱起眉头。

曾布闪出班来,他跟章惇站在班列最后,官职紧要的两人有资格走进崇政殿,但更多的时候,还是站在最后做个合格的盆景。但有机会说话,曾布决不会放过:

“臣有一事,禀明陛下:王韶、高遵裕近日具本上闻:武胜军已经攻夺,临洮城也即将修筑完成,两人拜请朝廷赐予嘉名,以彰皇宋声威。”

曾布的话,让赵顼来了精神,为新征服的土地赐名,这是他喜欢做的事。略作思忖,他便道:“武胜军赐名镇洮军,临洮复旧名为狄道。”

曾布躬身领旨,武胜和临洮这两个名字便成为了过去。

“由谁来镇守镇洮军?”赵顼问了一个关键的问题。

“王韶举荐的是韩冈!”

“韩冈?!”文彦博脸色都变了。

冯京也心生不悦:“镇守镇洮,他一介选人哪里够资格?!”

“敢问冯参政,韩冈不够资格,那谁够资格?!”王安石还没来得及说话,最下面的章惇就已经在厉声反驳。

他走出来,向过天子行礼,侧身直叱冯京:“韩冈功绩早已足够。霹雳砲数建功勋,疗养院救治无数,沙盘、军棋,更是行遍天下。此外,河湟数次大捷,韩冈皆有殊勋。横山虽败,可韩冈功绩难掩。本职的医治伤病,无一丝可挑剔;其在罗兀、咸阳,功劳又有谁人可比?再论他今次镇守渭源,斩首过千,贼将一擒一斩,同时还让临洮前线数万人的吃穿用度没有一分匮乏。

换作是他人,只要有其中任何一桩功劳,都足以保升朝官了。章惇斗胆,敢问冯参政,参政前次反对韩冈转官,今次又说他不够资格担任镇洮知军,那就请参政说一个有韩冈一半功劳的选人出来吧!推举一个有韩冈一半功勋的京朝官来知镇洮军好了!”

章惇声色俱厉,句句质问,且不等冯京措辞反驳,又转身对赵顼道,“陛下,韩冈才具过人,功劳迭出。在河湟又是名声、恩信远播于蕃部之中,有他来镇守镇洮军,陛下当可高枕无忧,而通远,也可以安心休养生息,以待明年开春。”

赵顼连连点头,章惇的话说到了他的心坎上,他转过视线,用询问的眼神望着他的宰相。

王安石会意低头:“这也是王韶的举荐。”

王韶举荐韩冈的用意,王安石心知肚明。若是韩冈还是保持在现在的官位上,那根本不够资格在更大规模的会战中担任要职。就算今次的攻略武胜,他担任随军转运使,朝廷也是又安排一个蔡曚来同理一职,这项任命就差点坏了大事。

韩冈的地位如果不能快速提高,明年的决战河州,他如何能坐得上随军转运使的位置。河湟一次次大捷,引来的贪婪目光,不止一个两个。到了真正决战的时候,就算天子和王安石都压不下要来分一份功劳的群臣。

王韶其实不介意分一点功劳给他们。但这些人中,有几个会如王中正一般老实?要是来的是自作主张,骄横跋扈之辈,他哪有那么多精力去压制。万一派来的人不合用,那可要坏了大事了。王韶自知不能将他的这一亩三分地都用篱笆锁牢了,但他至少要保证韩冈能主持随军转运之事,否则他即使出战在外,也要担心着身后会不会出乱子。

王安石收到的信中,王韶已经把他心中的打算说得明明白白,一定要保证韩冈的晋升。不仅仅是晋升京朝官那么简单,连资序也要超迁,否则枢密院有绝对的权力来否决日后决战时,韩冈担任随军转运使的任命,而御史台也会出手干涉——别以为那些御史们心胸有多广。

王安石出头支持韩冈,王韶作为眼下赵顼最为看重的边臣,他们两人共同的意见,赵顼怎么会反驳?何况韩冈本就是他很早就看好的臣子。韩冈入官都是他特旨批准,由布衣亲自拔擢。韩冈表现得越出色,就越体现了他赵顼的用人眼光——这两年来,韩冈已经给他长了很多脸了。

“既是如此,那就……”

“陛下!”见天子就要点头,冯京急声反对,二十岁就转官担任边地要职,这实在太夸张了:“韩冈齿序太少,年资太浅。区区弱冠之龄,入官亦仅两载,遽加升用,对其亦非好事。且这个先例留存下来,日后必有奸猾之辈加以利用。”

曾布出班道:“韩冈德才兼有,功绩少有人及。敢问冯参政,不知甘罗拜相,去病领军,他们那时年齿几何?”

“甘罗、霍去病皆是早夭之辈。少年得意,后事难终。”枢密副使吴充也同样反对对韩冈的任命,这么多次了,吴充早看出了赵顼对韩冈的赏识,他不会跟天子硬顶,直接在下面使绊子就行了。而曾布的话,给了他机会:“陛下,韩冈人才难得,还望不要奖誉太甚,以防其早夭!”

见着赵顼犹豫起来,文彦博赞赏的看了吴充一眼,立刻上前添砖加瓦:“再如旧时杨亿,少以神童荐于太宗驾前,才华横溢,太宗、真宗皆信用有加。惜其寿数,却仅仅三纪又一年而已。”

杨亿杨大年是太宗、真宗两朝时,在朝中任官有名的神童才子,连名相寇准都很赏识他,可他就只活到了三十七岁便病死了。

赵顼对韩冈很是赏识,他当然不想让韩冈年纪轻轻就出了意外,一二十年后,韩冈少说也是安定边疆的名臣,若是做得好,前途更是不可限量。

听了吴充和文彦博的话,他想想也是,过往少年得意的臣子,少有寿终正寝的,反倒早夭的居多。

“恩赏不公,可是朝廷幸事?!”章惇竭力为韩冈辩驳,“以韩冈之功绩才能,竟迁延于选海之中。这三五日一上殿的选人,又有哪一个还有脸面转于京官?!”

章惇的话,赵顼也觉得有理。那位始终没能谋面的年轻官员,朝廷实在亏欠他很多。

天子左右为难,王安石其实也担心韩冈擢升太速,会有什么不测。天变不可畏的说法,那是韩琦的总结,并不是王安石亲口所说。其实在他心中,对宿命论的一些观点也有些认同。

只是韩冈不能不赏,正如章惇所言,这么多功劳还只是选人,朝廷日后如何激励士民忠心国事。所以只能折中:“就算不能做知军,权发遣通判也是可以的。转个京官,当是无妨。知军一职让人兼着就是了,高遵裕、苗授都行。”

“韩冈资序仍是不足。”文彦博直言否决王安石的意见,“即便韩冈转为京官,要想任职通判,前面还有两任知县要过。”

资序是决定京朝官任职高低的重要依据。正常的情况下的京朝官,都是两任知县资序轮满,才能擢为通判。两任通判资序轮满,才能担任知州。自然,政事堂、枢密院,三司等中枢机构中的一系列职司,也是按着知县、通判、知州等资序来划分高下。

比如中书各房检正,就是第二任通判资序,也就是担任过一任通判,或是相当于一任通判的差遣,才有资格任职,要不然就得加个‘权’或‘权发遣’。

这是为了防止年轻的官员经验不足而任职高官设立的制度,只是渐渐变成了论资排辈的工具,到了仁宗后期,甚至变成了无论官员的贤愚不肖,都是各自按年甲资历轮候,这也是官僚社会的通病。

为什么王安石提拔吕惠卿、曾布、章惇等人后,会被人诟病不已?就是因为他乱了朝堂上的资序。让资历不够的年轻官员,一下跃居高位。让那些熬足了年纪的颟邗老官,心头愤恨难耐。也让那些老派人物,觉得乱了规矩。

可赵顼终于烦了,“此非密院之事,文卿家就不要多说了。”他直接让文彦博闭嘴。

文彦博白眉一轩,顿时怒容满面,赵顼这话实在太不给他脸面。他立刻抗声道:“那河湟之地,设立经略安抚司之事,臣还能不能说?!”

