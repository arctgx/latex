\section{第五章 圣贤需承传人荐(上)}

吕大防是旧党。韩冈是新党。

但两人坐在一起,却没有什么尴尬。

吕大防虽是铁杆的旧党,但他并不是那种逢新法必反的人,对其中诸法也都有所保留——话说回来,出自关中的士子,对于富国强兵的渴望不是河北京畿的士大夫可比,新法之中虽有惹起他们反感的一部分条令,但对将兵法、免役法等能整军强兵、解民困厄的法度,基本上都是持欢迎的态度——所以韩冈对吕大忠的公正,还是很有几分好感。

而在吕大防看来,韩冈尊师重道,事事为关学张目,甚至不惜与王安石冲突,是正人君子所为。而他帮着王安石度过几次难关的举动,也是作为臣子、作为士大夫该做的,并不是为了迎合权臣而做出的残民之举,当然也是有着一份好感。

互相看得顺眼,就不会有太多的龃龉。而且还有谋划张载入京的事要让两人一起操心。

吕大防性喜简朴,又是因为刚刚结束了守制而入京守阙,韩冈也没有在樊楼等大酒楼铺张设宴,而是就在家里设了便宴,吃着严素心精心制作的小菜,两人坐下来慢慢说话。

喝了几杯酒,各自说说河东和京城的传闻,又对最近的一些热门话题评述一通。气氛融洽起来,韩冈便切入了正题:“家岳将至京城,韩冈便不宜再留于朝堂。过些日子,就回自请出外。”

吕大防点了点头,这是应有之理。翁婿不便同居朝堂之上,要顾及着瓜田李下之嫌。当年晏殊、富弼这对翁婿同在中枢的情形,如今很难复制。除非韩冈也学着富弼,指斥王安石是奸臣。

但这自请出外的奏章其实只要上过就可以了,只要表明了态度,御史就不好再拿此事做文章。到时候只需天子留人,臣子也就可以顺水推舟的留下来——当然,不能忘了,还要隔三差五上一个请郡的奏章,作为补充。拖个半年,没有问题。

“不过在这之前,韩冈还有个心愿未了。但凡治学,不入京城,便不为天下所重……”韩冈说到这里话声一顿。

吕大防心领神会。他亦推重气学,当然希望张载能入京讲学,只是有新党在,肯定是没戏,当初韩冈不是已经碰了一次壁了吗?

“奈何令岳。”他摇了摇头。

“无妨。家岳那里,韩冈从无亏负,不惧问罪。但对子厚先生却是有愧于心,居于朝堂有年,仍不能使先生入京讲学。”

韩冈答非所问,只是向吕大防表明了自己的决心。他不仅仅是王安石的女婿,也是张载的弟子,身负这两个身份,与其小心的在两者之间守着平衡,还不大道阔步,按照自己的心意去做。只要自己的份量足够,王安石也得捏着鼻子承认结果,张载也不会对枝节之事太过于放在心上。

韩冈让吕大防不要顾忌,有事他肯定会为张载担待着,吕大防也就安心下来。韩冈写信邀请他来时,他就考虑过该如何让张载入京讲学。想来想去,还是得采取一个变通的办法:

“去岁郊天大典,仪制多有错漏。近日听闻天子对此有所不满,欲加以更易之。子厚先生谙熟周时仪制。玉昆你我齐荐,入太常礼院当是不难。”

儒门重礼,但凡大儒无不是精通礼法。仁为体,礼为用,这是儒学的根基之一。

张载的确精通礼法,尤其是以复古为己任,对周礼的研究可说是登堂入室,无论是仪式还是制度,从上到下都早已融会贯通。但韩冈希望张载入京是来讲学的,不是到东京来给人议论谥号的。

“圣人夏礼能言、殷礼能言,杞宋不足征,文献不足故也。”韩冈想了一阵之后,摇头表示反对,“如今礼院所用《开宝通礼》,乃本于《开元礼》而损益之。先生至太常礼院,必欲有所更易。然礼院之中,人事繁芜,言出多头。四季祭星主,其太牢、少牢之争,亦迁延数载未有定论。先生岂有一展长才的余地?事既不可为,就不免会有西归之念。”

这就不好办了。

张载一旦纠缠于俗务,尤其是太常礼院中的官员无不是深悉礼法的宿儒,而院中吏员也几乎都是对礼制仪式浸淫甚深的积年老吏。张载去了礼院之后,如果要恢复古制,必定会受到阻挠甚至攻击。身体本来就不好的张载,怎么可能有多余的精力去与他们一一争论。

而且如今礼院的工作,主要是主持各级祭典的仪式,同时也有审定臣子的谥号,另外甚至是民间上请朝廷册封的神灵该是第几等爵也算是管辖范围。在韩冈看来,实在是算不得什么大事,如果都是为了这些事来争吵,就太过于浪费张载的声望。

而且韩冈还有句话没有明说出来,但想必吕大防能听明白。

——嫉妒之心人皆有之,以如今张载逐渐响亮起来的声望,必然会有许多人以折辱、驳倒他为荣。国子监讲学,韩冈绝不担心,以张载的水平,绝不会逊于当年的胡瑗。但到了礼院的地盘上,许多事可就说不准了。

韩冈对张载其实敬重有加,而且另外还包含了一份私心在,他怎么可能会愿意看到张载被俗务所缠,失去了进京的本意。

吕大防的意见被韩冈很直接的拒绝,他并没有生气:“不知玉昆可有良策。”从韩冈的态度上看,他应该是有办法的。

“良策算不上,只是过两日,就要明着上本荐先生入国子监讲学。”

“明着……?”吕大防的声音中多了几分犹疑。虽然因为安置流民数十万,加之一系列的发明,韩冈在朝堂上的话语权已远非两年前新中进士时可比,但他要推荐张载入国子监,需要翻过的山却也并没有在这两年间降低多少,“难道玉昆你能说服吕惠卿?还是已经说服了令岳?”

“不,都没有。”韩冈摇了摇头,“该反对的肯定会反对。只是当轴诸公中,肯定还是有人会支持的。”

王安石还有一个月才能抵达京师,在这之前,都还是有点机会。而且就算王安石到了京师,也不是全无可能。想看到翁婿两个打擂台的,绝不止一个两个。硬要说起来,冯京、吴充等人都有可能成为此事的助力。

吕大防闭起了眼睛,沉默了好一阵,猛然睁开,神光锋锐:“玉昆,你可是要我去拜谒冯当世、吴冲卿?”

“韩冈曾听闻,微仲兄与王禹玉向日有旧。”韩冈微微一笑。只要可堪一用,他都会利用上,就算是王珪、冯京、吴充这样的政敌也无所谓,而且敌意有时候也不是全无好处。

吕大防方才已经考虑过了,也不再多犹豫,“愚兄只能去跟王禹玉请托齐荐子厚,却不能论及他事。”

“韩冈素知微仲兄为人,不敢多有请托,也不敢用诡计亵渎师长。也就是请微仲兄向王禹玉提上一句。”

吕大防是个方正的性子,韩冈并不指望吕大防能用离间王安石、韩冈这对翁婿为理由,去说服冯京、吴充他们。但在王珪这位熟人面前顺口提上一句,想必吕大防也不会固执于自己的性格。

“既如此,愚兄也不敢推托,此亦是愚兄分内事。”吕大防举起酒杯,以酒为约,与韩冈对饮而尽。

将此事定下,韩冈和吕大防都放下了心事,开怀畅饮,一边海阔天空的聊着,一边喝酒吃菜。

吕大防身高七尺,比韩冈还要高出近一个头,就算是端端正正坐在座位上,就已经很有压迫感,方才见面时,巨大的身躯,更是让韩冈感觉有些压力。现在放开肚子,吃喝起来也比韩冈远胜,转眼桌上几盘菜就不见了踪影。

韩冈连忙让人上酒上菜,吕大防则道:“在边州,粗食劣酒也不是没有尝过,京师的美酒佳肴也一样吃了。口腹之欲不可放纵,好坏都是由他。食不厌精脍不厌细,那是祭礼,愚兄寻常在家中吃饭喝酒,都是以简朴为上。”

年近五十的吕大防与只有自己一半年纪的韩冈称兄道弟并没有半点不快,辈份这个东西与年纪无关,韩冈本就是吕大防三个兄弟的同窗。再说以韩冈如今的声望也当得起与吕大防平起平坐,

“存天理、灭人欲,此乃正道。微仲兄之言,韩冈也有会于心。”

礼记中有一段叫做‘人化物也者,灭天理而穷人欲者也。’韩冈说的这六个字是反过来用。吕大防听了觉得甚有道理,点着头重复的念了好几次,赞赏不已:“饮食是天理,穷于口腹之欢,那就是人欲。‘知好色、慕少艾’是天理,贪纵床笫则是人欲。挣钱养家是天理,宝于财货则是人欲。守中即是理,穷极则是欲……能体会出这六个字,玉昆你也算是明理入道了。”

“韩冈可当不起。”韩冈笑着道,“洛阳的正叔先生曾在信中解释‘人心惟危,道心惟微’这八个字,说‘人心私欲,故危殆。道心天理,故精微。灭私欲则天理明矣。’所谓存天理、灭人欲这六个字,原本于此。”另外更多的是朱熹的功劳,不过这就不能说了。

‘洛阳……’吕大防知道,韩冈在程颢、程颐面前也是自称弟子,算是承袭两家之教,当日他立雪程门外的事迹,也早已遍传天下。“是否要将洛阳二程也一起推荐?”吕大防问道。

韩冈踌躇半响,最后摇摇头,“……力分则弱,还是先荐了子厚先生。”

