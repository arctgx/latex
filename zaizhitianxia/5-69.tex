\section{第十章 却惭横刀问戎昭(一)}

军情紧急,留给韩冈整顿行装的时间只有三天。

韩冈也没有耽搁,将一应准备做好,移交了公务,辞别了家人,三天后,上殿陛辞,随即启程离京。

京城之中对韩冈出任河东路经略使的反应趋向正面,眼下有足够能力和威望镇守河东的,也就那么寥寥数人,不论怎么算,韩冈都是其中之一。

“镇守河东,寻常时随便哪位侍制都能适任。不过如今的局面,除了郭逵、王韶以及章惇之外,也只有韩冈了。赵禼、熊本都还差一点。”

城西的刘楼之上,刚刚结束了一任通判、回京诣阙的赵挺之也与同伴议论着最近的时事。

“韩三去了河东,好歹夜里能睡得稳一点。”

强渊明凭栏俯视着楼下的汴水,河水潺潺,乃是从西水门而来。

就在昨天,韩冈一行数十人,便从此门出城,先沿着汴水抵达黄河,然后渡河北上太原。

“恐怕你强隐季还是睡不稳。”

熟悉的声音在房外的廊道上响起。刘楼在七十二家正店中排名倒数,也不是没有缘由,房内对话的声音,竟然能传到门外去。

赵挺之和强渊明并没有因此恼火,而是笑着起身相迎。房门向内推开,蔡京徐步跨进门来。

“元长你可终于到了。”强渊明畅快的大笑道,“迟了这么久,还以为你不来了。”

“你强隐季倒也罢了,逐日看得脸熟,正夫兄可是难得回京一趟,如何能不来?”蔡京向着赵挺之拱手一揖,“还没恭喜正夫兄喜得贵子。日后公侯万代,福泽绵长。”

“多承元长吉言。”赵挺之连忙回礼。

“元长你尽会吊人胃口。”强渊明与蔡京、赵挺之是同年,情谊甚笃,也不在意什么礼节,一把扯住蔡京,“你前面说的话到底为何意?”

“是不是哪里又出了事?”赵挺之也紧张的问道。

蔡京左右各瞥了两人一眼,也不卖关子,直言道:“王韶病卒了。”

“……王韶死了?!”赵挺之和强渊明同时惊叫。

“嗯。”蔡京点了点头,“王韶自出外后不久,便生了病。腹生疽痈,逐渐肌肤溃烂,药石难救,最后听说是洞见五脏而死。”

“洞见五脏……”赵挺之干咽了口唾沫,那该是什么样的惨状。

强渊明也是脸色泛白,一时间都不知道该说什么好。

蔡京大马金刀的坐了下来,拿了一只银杯过来,给自己倒酒,“当地的走马承受遣急脚递将消息传递上京,小弟也是在中书门下兼了差才听说的,他的遗表则还要过上一阵才能抵达京城。”他将杯中酒一饮而尽,长舒一口酒气:“临战失大将,乃是不祥之兆,而失却帅臣呢?”

强渊明、赵挺之震惊之余,又满是惋惜。

论起兵事,王韶是是实打实的文臣中兵法第一,连韩冈都是出自于其门下,章惇比他也少了一份老辣。眼下临战,天子能放韩冈出外,只是因为已经下旨召王韶进京。纵然此前一直传说王韶抱病,可所有人都觉得,但不至于会就这么简简单单的病死。

“蔡子正才过世不久……”强渊明苦笑着坐了下来。

赵挺之也跟着坐下来叹息道:“王子纯、蔡子正两人一去,擅长用兵的两府帅臣,如今就只剩一个章惇了。”顿了一下,他又道,“郭逵其实也不差,但他终究是武将!”

“元长。”强渊明欠身问蔡京,“你说天子会不会降诏将韩三召回来?”

“韩三都离京北上了,哪里还可能将他召回来?”蔡京笑了一声,“如果是三天前,倒还有可能另遣他人去河东。可都陛辞了,又将他召回,好像朝廷离了他就办不了事了。哪位宰辅愿意丢这个脸?”

“说得也是。”强渊明一笑,又坐直了身子,“今天一并请了元度【蔡卞】,可惜他写回执推了。元长你没从元度哪里听说什么?”

“还能什么,太学案!”蔡京猛然间拔高了声音,“太学案罪名是在推荐免去解试和礼部试的上舍生、内舍生时,挟情私取。这等于是制举舞弊,拿几人首级出来警戒后人,也不是不可能。”

强渊明摇着头:“余状元都被拘入御史台,要是因罪夺了告身,可就是开国以来的第一遭。”

“此事小弟也听说了。”赵挺之也道,“没想到会闹得这么大,也难怪元度要闭门谢客。”

蔡卞因为曾经求学于王安石的门下,是新学一脉的嫡系,故而才几年的时间,就在国子监中做了直讲。

自从三舍法确立,太学扩招,国子监中的学官人数日渐增多,基本上都是新学一脉。在他们的教导下,新学一脉不断壮大。现如今,国子监中的直讲、讲书、助教,一个个被牵扯进太学案中,眼下就只剩蔡卞等寥寥数人独撑大局。多数牵连进太学案中的学官,多半逃不离贬斥出外的,严重的甚至会追毁出身以来文字,而接替他们位置的学官,自是不会是新学中人。

“吕参政不是有消息说很快就要宣麻了吗?怎么还让太学案的声势闹得这么大?”

“李定要自清,不可能手下留情。舒亶想立名,只会往重里拷问。其实更多的还是苏轼的缘故,要不是天子特恩开释,让御史台脸面无光,也不至于急着在太学案上挽回颜面。”蔡京哈哈一笑,“纵使李定、舒亶都偏向新法,但他们要为自己考虑,吕参政就是成了吕相公,也一样压不住阵脚。”

……………………

时间一天天的过去,从西北传回来的消息,也越来越压抑。

环庆高遵裕、泾原苗授,两人放弃韦州,率残部撤回境内。统领秦凤、熙河两路的王中正由于独木难支,亦借道葫芦河率师回返。李宪领河东军离开银夏,在弥陀洞驻扎下来。

两个月前声势浩荡的六路齐发,在灵州城下的一场溃败之后,已经烟消云散。此时就只剩下鄜延路在竭力维护着朝廷的脸面。种谔率领的官军盘踞银夏之地,看模样,似是要与党项的铁鹞子一决生死。只是他本人竟然已经回镇银州,这份反差让人分外觉得纳闷。

河东、河北两路的气氛则是越发的凝重,辽人虽然还没有动作,但谁都知道这等于是张弓搭箭,虽是平和,但私下里暗流汹涌。如果不小心行事,很有可能就会遭到党项人的反击。

由于西北两处的局势越来越紧张,吸引了所有人的目光。继续高歌猛进,已经攻下甘州的王舜臣,他的功绩在京城中没有掀起一丝涟漪。纵使他能光复河西,但在辽人可能南下的压力下,说不定转眼就会被西夏夺占回去。

但七月上旬的天下时局,是异样的平静。

西夏没有动作,辽国同样也没有动作。战争在这段时间里,似乎已经不复存在。

一直到了七月十一,河东、河北同时来报,辽主的宫帐已经离开了鸳鸯泺,开始向南京道的方向进发。

辽主七月迁捺钵至秋山行猎,九月至燕京体察南京军政,这样的出巡路线过去是经常出现的。可放到现在,味道就变了。

这很有可能是战争的开始。但也有人认为这是耶律乙辛在虚张声势,只是想从朝廷手中敲诈出更多的岁币而已。

不过并没有人敢于明确的站出来说明耶律乙辛绝不会举兵南下。作出判断很容易,但对自己的判断确信无疑也不难,难就难在将自己的身家性命全都压上去。

没人敢赌上一把,天子从此一夕三惊。连同东京城中也是一般。

几日后,辽国派来的使臣便在雄州叩关,声称是奉了辽国新帝以及尚父耶律乙辛之命,前来劝说南朝收兵。

并非是恭祝天子和两宫太后生辰,也并非是共贺年节,临时加派的使节,必须得到天子的许可之后,才能被允许进入内地。

在得知辽国使臣的身份之后,赵顼和每一位宰执,都有将其人拒之门外的打算。

那是个老朋友,乃是大宋君臣都很熟悉的萧禧。

不过辽国新君名为延禧,为了避讳,萧禧改以表字为名,改名萧海里。只是在东京城这边,依然习惯性的用着他的旧名。

当年萧禧硬是逼得赵顼割让了代北之地,外面甚至传言说一口气让了七百里,让赵顼生了好一阵子的闷气。如今萧禧复至,不用想就知道,必然是耶律乙辛想借助他丰富讹诈的经验。

以现如今的天下局势,不可能将辽国使臣拒之门外,表现出刻骨的敌意,但太过于纵容,也会显得畏怯,反而会让萧禧这个贪婪之辈得寸进尺。

还没等商量好该怎么应对,在一次宴席上,酒醉之后的萧禧透露了国书中的内容——当然是故意的——雄州的守将用金牌加急将辽人索要的条款传到京城。

很简单,就两条。

但每一条都让赵顼听得火冒三丈:

第一,从大辽属国西夏撤军。第二,岁币增加十万两银,十万匹绢。此外,还有个顺带的要求,将种痘法传授于大辽。

如若不从,请会猎于中原。

