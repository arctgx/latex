\section{第18章 青云为履难知足(三)}

近万颗首级,只有十几二十人点验,到了入夜的时候,才查验了大半。

本以为会就此停手,到了明天再来,可李舜举却让人点起火炬,继续点验下去。他在宫中的大貂珰里面,就属他做事一板一眼,天子交代下来的任命从不打折扣。

这是怕去睡了之后,夜里韩冈他们在其中做手脚吧?

李信就在韩冈身侧冷哼了一声,心中有股被羞辱的恼火。

韩冈倒是无所谓:“既然已经做了,就该做到底。事情才做了一半就放手,还不如一开始就不要做。李都知这么做并没有错。”

直到半夜,终于全部点清了。李舜举和他的随从们在点验过程中,发现了大约三五百枚的首级,不能确定是真是假。数量不少,不过从比例上看,则完全可以忽略。

韩冈和李信等人的战功确定了,李舜举的脸上多了些笑容。

歇息了半夜,到了第二天,李舜举来找韩冈,说是要去军营一观。

李舜举这一次南下所奉密旨,就算不说出来,韩冈等人也能猜到了。是要将邕州内情外情都查探得明白,回京后禀明天子。

苏子元和李信的眼里,都藏着喜色。要不是有出兵交趾的打算,李舜举何须查探得如此仔细。

刚刚进了军营中,李舜举就闻到了一股浓烈燃烧的艾草烟味。

“这时候就烧艾草了?”李舜举有些惊讶,离着端午还有些日子呢。

“端午燃艾草,就是为了避五毒。中原气候四季分明,只有到了端午之后,蛇虫才会猖獗起来。不过广西这里一年四季都是蛇虫满地,艾草要时常点着。”韩冈领着李舜举往营中走,“许多疾疫都是来自与蚊虫叮咬,就如疟疾,绝大多数病患往往都是在蚊虫最盛的时候开始发病。”

韩冈将疟疾定义为蚊虫叮咬后的病症,如果他是普通的官员,少不了会受到质疑,但从韩冈嘴巴里说出来,医官们奉若圣谕,士卒们信之不移,李舜举也是连连点头。权威的好处完全体现了出来,只是少了对权威的质疑,对医学乃至科学发展绝不能说是一件好事。

用艾草薰蚊,虽然连人也一起熏得头昏脑胀,但被蚊虫叮咬的确少了许多。就连军营的营房,韩冈都是让人用了纱帐来封住门窗。在军营和疗养院中,李舜举内内外外都仔细的看了一遍。转过来又出了城,城外数百顷田地里的占城稻,马上就要开始收割了。

“这些都是官田?”李舜举望着丰收在即的稻田,惊讶的瞪大了眼睛。这才多少日子啊?

“不是官田。一部分是无主田地,一部分则有主。不过今年兵乱,形势特殊,这一季的占城稻收获,是统种统收,收来后由官府加以分派。到了明年开始就是各家种各家的了……以今年的情况,还能再种收两季。”

李舜举多少懂一点农事:“这地力支撑得起吗?”

“邕州现在是人少地多,第二季是在城西的一片地里栽种。这两天已经开始翻耕下种。再过一两天那边结束,这里就可以开镰。”韩冈笑了一笑,“也就今年一年,收了三季之后,百姓的口粮和常平仓也都够了。”

“原来如此。”李舜举笑道,“龙学理政果然是世所难及。”

“都知过奖了。”韩冈沉吟了一下,道:“另外还有两千战俘的事,要请都知禀明圣上。”

李舜举转身过来:“龙学请讲。”

“交趾贼从邕、钦、廉三州掳掠了大批人口,回去后驭之为奴。我们这边也不能白白的养着这批战俘。邕州的官田都需要人手,还请都知为此禀明天子。”

中国一向以华夏自称,以仁孝治天下。若是虐待战俘和降人,将领都少不了会受到弹劾。一般来说,愿意留下来的就留下来,不愿留的,则是交换回去。韩冈驱用战俘的做法并不多见。。

“自当如此。龙学之言,舜举回京后,必如实报于官家。”李舜举倒不在乎,邕州死伤太惨了,他过来时看得都是怵目惊心,驱用战俘耕作连报复都算不上。叹了口气,“有龙学在广西,官家当可无忧矣。”

韩冈谦虚了几句,李舜举分开左右,神色一时郑重无比,“天子口谕,韩冈接旨。”

韩冈此前估摸着李舜举当是带了密旨来,没想到还带了口谕,“臣恭聆圣谕。”

“岭南山岭众多,瘴疠遍地,官军驱使不易,行军用兵当以峒丁为先导。用峒丁之法,先须得实利,然后可以使人,不可以甘言虚辞责其顺命。如关陇点教蕃兵,使轻罪可决,重罪可诛,违西夏则其祸远,违帅臣则其祸速,合于兵法畏我不畏敌之义。苟无实利,则欲责其顺命也难矣。”

韩冈暗自点头,这一段说得正合他心意。

“今卿可选募精劲土人一二千,择枭将领之,以胁峒丁,谕以大兵将至,从我者赏,不从者杀。若果不从,即诛三两族。兵威既立,先胁右江,然后胁左江。此等既归顺,则攻刘纪巢穴不难也。”

韩冈心中更是一喜。看起来章惇和自己的计划已经得到了天子的默许,先用手上的兵力,解决掉广源州。等到大军前来,再直攻交趾。不过这话应该是跟章惇说吧?他才是广西经略使。

“章惇好作崖岸,不通下情,将佐莫敢言,卿当为言之,毋得轻敌。”

原来如此。

赵顼并不是庸主,只是心性的问题,让他不能算是一个出色的帝王。论起才智并不差,该如何驱用蕃人,也心知肚明。绝不会说什么以德服人之类的书呆子气的蠢话。‘从我者赏,不从者杀!’这才是最好的手段。

韩冈行礼如仪:“臣必谨遵圣谕,请陛下放心。”

李舜举传达了圣旨,又查看过了邕州的情况,便启程返京,而新任邕州知州的苏子元则随行北上,邕州则由韩冈代管。

他走后不久,黄金满也率军回广源州老家去了,只把在邕州做了巡检的黄全留了下来。章惇从桂州军器库中弄了五百套皮甲给了他,加上一批上好的刀枪。有了这些军器,以及之前分给黄金满的一批战利品,他的都巡检的架子就撑起来了。

交趾军势大衰,而且广西这边韩冈和章惇也四处放话,明着说要对交趾人在广西犯下的一切罪行展开清算。谁敢在这时候有所妄动,杀鸡儆猴的刀子就要直接砍过去。黄金满他回广源州,刘纪三人都不敢拿他怎么样,甚至还要担心他什么时候领着宋军攻过来。

“当务之急还是编练新军。”李信对韩冈说着,天子的态度很明确,交趾那边肯定要报复,而在报复交趾之前,首先就要先将广源州拿下:“朝廷给了两千人的名额,总体保证八成的数量,核心的两个指挥则要满编。”

“不!”韩冈对吃空饷的情况一时深恶痛绝,尤其是自己要带领的兵力,绝不想看到明着一千人,实际只有七八百的情况,“要满编,甚至可以超编一部分。”

“只是军中将校……”

“只要压下去一年就够了,后面的情况如何不用去管,有着灭国的功劳在前面的吊着,就算再贪也能忍下。”韩冈眯起了眼睛,森然一笑,“正好可以一试军法!”

李信沉默的点头,他的表弟恐怕当真会拿人头来试刀。

看见李信沉默,韩冈笑了起来,“其实只要将话说开来了,聪明人还是很多的,看着封妻荫子的大功在眼前,谁还会贪那点蝇头小利?”

随着天气渐渐炎热起来,供给邕州的物资慢慢多了。家里面写给韩冈的私信也一起发来。这算是给他这样外任边疆的臣子们的福利了,不仅是边臣,出使外邦的使节也能通过驿站系统收到家信。

当年富弼临危受命出使辽国的时候,只要收到家信都直接就火给烧了,说是徒乱人意。富弼将他为国忘家的强硬态度摆出来,倒是给辽人看得居多。辽人见了他这番作派,心知不可欺辱,对岁币的讹诈也不得不收敛了一些。

邮政其实是个赚钱的买卖,如果能分割一部分出来为民间服务,其实能赚到不少的钱。当然,前提是押送邮件、货物的驿卒,有足够的职业道德,而不是靠山吃山、靠水吃水。

从信中,韩冈知道了自己的父亲很快就要上京诣阙。同时也知道了连王旖也一起怀孕的消息。另外王雱的病情也让他担心。此外,还有王安石的信,里面说了朝堂上对攻打交趾的争论,虽然至今没有确定,但很有可能只会征发一两万的军力。

一两万就一两万,作战的方式可以模仿之前的战斗。以官军为核心,同时将蛮部都汇合起来,一起攻打升龙府。

交趾的首都,当就是后世的河内,所谓的富良江则是红河,只控制着富良江下游地带的平原抵达,而山区的蛮部的确听从号令,但不会跟交趾一起共存亡。

韩冈想着,还是摇了摇头。说是一两万,真正能到多少,还要画个问号。而且究竟是从哪里调兵,更是要好好推敲。具体的情况,还是要等苏子元回来,轮到自己上京的时候再说。

