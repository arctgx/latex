\section{第17章 夜顾茅庐访遗贤(下)}

“请问韩秀才可在?在下德安王厚,夜来拜会,还望不吝一面!”

一声突如其来的唤门声,打断了厢房中正喝得热火朝天的气氛。王舜臣使劲晃了晃有点发沉的脑袋,只觉得从门外传入耳中的声音有些奇怪:“是不是方才来找秀才的小官人?怎么是南方的口音?德安是在南面的哪个路?”

“德安?是江西罢?”韩冈前世跑过长江南北,也去过庐山,九江、德安都熟悉。二十一世纪的德安属于江西省,却不知道北宋的德安是不是也归于江南西路。

“江西人?!”赵隆本被一下惊醒,听说是江西人后,却放松下来:“那就不是了。”

“什么不是?”王舜臣问道。

赵隆笑道:“伏羌城少见南人,本还以为是这些天在伏羌城附近跑进跑出的王机宜家的人。不过王机宜出身江州,那是江东的地儿。”

“江州?!”韩冈醉意全无。九江古称就是江州,看过水浒的他如何会不知道?!“德安就在江州!”

赵隆喝进肚子里的酒都化作汗水冒出来了:“真的是王机宜?!”

“王机宜?”韩冈急问道,他还没有没听说过什么王机宜,跟节判吴衍的交谈中,也没从他嘴里听到过‘王机宜’三个字。

“就是上书天子要并吞青唐,拓边河湟的那位王机宜!”刚到秦州不过半年多的王舜臣,比土生土长的韩冈对秦州内外更为熟悉:“他得了官家的赏识,被派到秦州来,名为帅司【经略安抚司简称】的管勾机宜文字,管得却是所有与蕃部有关的事情。那摊子事本该是经略相公和钤辖府一起管,现今给王机宜夺了去,两家都不高兴。”

韩冈将脑中的两份记忆互做对比,很快确定了青唐的位置。那大概是后世的青海湖东部地区。而河湟,则是河州和湟水,位于甘肃青海交界的临夏、和政一带。在唐朝时,处于与吐蕃王国交锋的第一线。唐玄宗后,逐步被吐蕃占据。而在吐蕃王国分裂后,仍被吐蕃残部所控制。在此时,则是泛指了青海东北、甘肃东南的一大片被吐蕃控制的地区,也称之为熙河——即以熙州、河州为主的区域。

那位王机宜既然有心为大宋开拓边疆,自然是求贤若渴,若能得到他的赏识,受荐举而得官,也是不在话下。如此良机,韩冈不会白白放过。

“王机宜叫什么名字?”韩冈又急急追问。

“王韶!”

‘王韶?’韩冈觉得有些耳熟,却记不起究竟是因为两个记忆中的哪一个而觉得耳熟。

“请问韩秀才可在?!”从门外传进来的声音高了几分,显是王厚等得有些不耐烦了。

“来了!”韩冈起身,理了理皱成一团的衣服,上前开门,一名二十上下,英俊瘦削的年轻人便出现在他的眼前。

“韩秀才?”王厚瞪大了眼睛。若不是同样的一副高大身材,他便完全无法将眼前这位满身酒气的破落户,与傍晚通衢上义正辞严的韩秀才联系在一起。就连让王厚印象深刻的挺眉秀眼,也因酒意而变得涣散无神。

“正是韩冈!”韩冈却半眯起眼,因酒意而涣散的眼神重又锐利起来,他先拱手行礼道:“官人既是有事找韩某,不如先进屋说话!”

王厚向屋中张望了两眼,犹豫着不肯进屋。他连跑两趟,又在门外等了许久,本是用汉昭烈三顾茅庐的旧事来安慰自己。现在只见偏厢中乌烟瘴气,桌面上杯盘狼藉,两名军汉面红耳赤,哪里愿意进屋去说话,连带着对韩冈也是失望已极。

“兄台可能喝酒?”看出王厚的犹豫,韩冈突如其来的问道。

王厚一愣,不知该如何回答。心想怕是要请自己喝酒。如此腌臜污秽的地方,王厚哪肯干,只想找个由头推脱掉。

韩冈笑道:“秦州的水虽不如江南水甘甜,但酿出的酒却别有一番滋味。风土不同,人情不同,水酒的滋味也自不同,不亲历一番,也说不出孰高孰低。王官人你说是也不是?”

韩冈的一番话听在王厚耳中,似是别有深意。他犹豫再三,还是勉强跨入门里。

王舜臣和赵隆这时已经将桌子收拾干净,见王厚进来,便要告辞离开。

韩冈拦住他们,让他们坐下继续喝酒:“哪有来一个客人,却赶走两个客人的道理。王军将和赵敢勇还是坐下来说话,想来王官人也不会介意。”

韩冈率性而为,也不问王厚愿意不愿意。王舜臣和赵隆现在都以韩冈马首是瞻,也知道韩冈不会害他们,也不多话,径直坐了下来。

王厚在屋中站着,进退两难,最后一咬牙也拉过一张交椅坐下。心想:既然进来了,坐一坐也无妨。顶多话不投机,提前告辞便是。至少现在,韩冈特立独行的款待,让王厚觉得韩秀才还是有点能耐,否则也不会有这样的脾气。

王厚坐下了,韩冈也跟着坐下,心中得意而笑。根据他过去的经验,把人骗来是最难的,而把人留下却很简单。

韩冈是故意慢待王厚,与其毕恭毕敬,还不如简傲一点,至少让王厚不敢轻慢,也多一点敬畏。依照世间的认识,越是有才之辈,越是盛气凌人,王厚他应该能习惯。反正看王官人见到自己后的神色,对自家的评价应是落到了谷底,已经低得不能再低,只要表现得出色点,升上去一点便是净赚。

也不问王厚来此的目的,韩冈直接找过一只干净的酒碗,为王厚满上,又说道:“庐山险秀,又近着江州,王兄德安人氏,真是好福气。‘日照香炉生紫烟,遥看瀑布挂前川,飞流直下三千尺,疑似银河落九天。’李青莲妙笔生花,每次一读此诗,便让人对庐山神往不已。”

韩冈顿了一顿,王厚正想要开口插话。不成想韩冈又抢先一步,继续道:“德安与庐山近在咫尺,又与千里彭蠡【今鄱阳湖】比邻而居,万里长江也在附近奔流不息。湖映山色,江水滔滔,如此胜地,世所罕有。若有机缘,还真是想去上一次。”

“江南是比关西要富庶。”王舜臣随口带了一句,他酒意上涌,也不顾王厚的身份了,“江州水土养人,据说那里的小娘子也比关西的水灵。”

“江南水乡出美女嘛!”韩冈随着身边醉汉的口气笑说了一句,话锋又是一变,“不过……江州是人间胜地,却不是建功立业的地方!”

被韩冈带起了心思,王厚重重的点了点头,又想说话,不想王舜臣已被韩冈的最后一句说得豪气顿起:“秀才说得正是!要想立功,还要看我关西!”

韩冈却摇头,“治军必先足食,足食必先养民。关西水土已远不如汉唐时的富庶,一场大战便能让各路的粮储耗光。没粮没饷,光靠关外输送,空耗民力,朝中也难支持。”

“秀才说得是。”王舜臣立马接口道,“俺还在延州的时候,吃过关东运来的麦子,也吃过蜀中的稻米,不过还是关中的谷子【注1】好吃。”

一番对话几乎变成了韩冈和王舜臣的一搭一唱,王厚几次要开口,都没找到机会。

韩冈又道:“所以只有一个办法能解决这个问题!”

“什么?”王、赵二人问道。

“屯……田……!”

“还有市易!”王厚终于能插上话了,他急急地说着话,仿佛要从嘴里迸出来,“在渭源开办榷场【注2】,不但能抽取税入,还能顺便收些租佃,不用劳烦国中转运。更能让青唐诸多蕃部亲附大宋,实是一举多得。”

听到这话,韩冈心中一喜:‘终于套出底了。’

一直故意不让人开口说话的机会,让他压着闷着,等到瞅准时机再稍稍放松,便会如王厚这般不由自主的将心底所想都暴露出来。韩冈他化用了一些自己所知的常识,又融入了一点不算出奇的见解,只通过话语的组织,把准了王厚的脉,就轻而易举地套出了王韶的计划。

渭源就是渭水的源头,犹在伏羌城上游近三百里,已经深入被青唐吐蕃窃据的土地。看起来,在渭源开办供蕃汉交易的榷场,便是王韶收服青唐、开拓河湟的第一步计划。

既然已经了解了一点对方的底细,再因势利导,或反驳,或赞同,把对话的主导权掌握在手中,骗过眼前的毛头小子,太容易不过!

“没错!王兄说得正是!有钱有粮,方可出兵打仗。”韩冈先附和了王厚一句,却又言辞恳切的说道:“不过两件事都是要大费周折。须得缓缓而行,不可希图一蹴而就。”

“是啊!”赵隆忙点着头,“来往边境有多少家回易商队,还有他们身后的官人们,都是不想开榷场,会妨碍到他们赚钱。”

注1:南方的谷子是稻,而北方的谷子通常指的是小米,也就是粟。

注2:榷场,就是市场、集市。通常特指边境地带,与外人交易的场所。

ps:今天第一更。求红票,收藏。

推荐一下三水的九霄天帝,书号69321:

这是一个吞云吐雾、炼气修行者为尊的世界

这是一个强者如林的世界,其中强者肆意妄为,弱者逆来顺受。

少年方兴,与炼气修行无缘,郁郁不得志。

一具水中女尸,让他的穿越者灵魂在沉睡百年之后终于觉醒,从此他踏上了成为至高者的永恒之路。

