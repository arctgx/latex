\section{第九章 拄剑握槊意未销(12)}

向东进军灵州的王中正现在可没有王舜臣所想那么轻松,就在鸣沙城北方不远,离峡口【青铜峡】只有三十多里的地方,秦凤、熙河联军受到了西夏铁鹞子的夜袭。

王中正所部自从过了天都山之后,一直都有一支多达万骑的铁鹞子在阻挠他们的前进。他们不分昼夜的拼死突袭,给宋军造成了不小的伤亡。

到底是为什么让这些党项骑兵——而且是绝对的精锐——奋死拼搏,就是王中正也能猜出个八九不离十。全军上下因此都想急着突破他们的阻碍,但王中正本人的才能有限,还有一应蕃军出工不出力,使得秦凤、熙河联军一路走得步履维艰。

但就在过了鸣沙城后,之前骚扰、阻挠他们的铁鹞子突然间就撤走了。当这支党项骑兵在的时候,人人恨其碍事。但当他们离开,自王中正以下却人人失落,皆以为灵州城已破,这支骑兵或许是被调回兴庆府,或许就干脆逃命去也。

灵州既然已破,只能赶得及攻打兴庆府。当时王中正曾想加急赶去兴庆府,但粮草没有跟上来。而通往兴庆府的前路,被党项人和泾原军两番清洗,肯定不可能找得到粮草,而且刚刚攻下灵州的高遵裕和苗授都肯定无力继续去攻打兴庆府,所以停下来等了一天也没什么关系。

而这一等,等来的就是铁鹞子的夜袭。

不得不说,从王中正开始,所有人都给骗了,但运气却硬是站在王中正一边。如果当时没有因为粮草问题,当即赶去灵州与泾原军和环庆军会合,全军覆没都有可能。可偏偏宋军在原地停留了一天,却反过来让党项人误会了,以为宋军已经看破了他们的骗局。

仅仅是夜袭的话,铁鹞子发挥出来的实力,还是奈何不了宋人的营垒——王中正一向胆小,对营垒的防御,一直放在首位——并且在刺猬一般的营垒防线上,碰得头破血流。如果天亮后,宋军能出寨反击的话,说不定还有机会弄出个大捷来。

可惜的是,宋军这边因为环庆、泾原两军的惨败而士气大落。看到一枚枚袍泽的首级,以及身着板甲,在马上用斩马刀挑起一个个头盔的铁鹞子,许多人都无心作战,在指挥上出了不少篓子。

为了保护粮草,宋军不得不出寨维持粮道安全,这就给了铁鹞子冲锋陷阵的机会。但结阵后的宋军,就算士气衰落,也照样能让铁鹞子吃足苦头。

最后黄河河畔的这一仗,彻彻底底的变成了一笔烂账。

三天时间,双方打得昏天黑地,损失和斩获两边都计算不清了,不是伤亡数量有多大,而是乱得无法统计。而局势,依然是未分胜负的平局。

历经鏖战,现如今的赵隆,决没有王舜臣想象中的自满。

他现在连话都说不出来了。

昨日的战斗中,他杀得一时兴起,将捂在脸上的护面给摘了下来,指挥着麾下的士卒。不意当即脸上就中了一石头,是泼喜军用旋风砲射出来的飞石。还好距离隔得远,石子的威力已经不大了,没伤到骨头,但腮帮子还是肿了起来。敷了化血化瘀的药,又用细麻布裹了脸,发出的声音含含糊糊,让人很难听得清。

这一仗下来,将领中,伤员绝不止赵隆一人,统领一部蕃军的青谊结鬼章都战死了,其余诸部,也都吃了不小的亏。其实也是吐蕃人不习军令的缘故,如果是官军单独列阵,情况还能好些。

不过铁鹞子的损失也不小。每一面旗帜下的军队,三天下来,明显单薄了不少。

西贼大军的突袭突如其来,结果能打成平局,运气算是很好了。

王中正也为自己的运气也感到庆幸不已:“幸好行程耽搁了一些,要不然可就彻底完了。”

刘昌祚点了点头:“嗯,运气好。”

“要是没有因为粮草耽搁,堵路的西贼走后,我们至少能走上五十里路,全军穿过峡口【青铜峡】。”

“嗯。”刘昌祚没什么兴致的回应道。

“过了峡口,就是兴灵。届时人心松懈,结果决不是现在的样子。”

“哼……”

“不过若是攻得再快一点,早几天打到灵州城下,或许能挡住西贼在河渠上做手脚。”

若是在往常,赵隆能开口说话,还能回应主帅两句,帮他化解紧张情绪。但现在赵隆只能在帐中坐着,几乎可以算是王中正在自言自语的壮胆,刘昌祚只是哼哼哈哈的发个声。

刘昌祚运气不好,没跟对人,加上随着资历,性子越发高傲,没哪个主帅喜欢他。而且不知道为什么,到了殿上觐见天子的时候,明明腹中锦绣,可偏偏倒不出来,几次上京诣阙,都没有在天子面前落个好字。

以至于天子在战前还特意下诏说,‘刘昌祚奏请多不中理,虑难当一道帅领。’让刘昌祚听命于王中正。

赵隆,他也可算是一时名将了,南征北讨的经历都有了,但年纪和资历差了刘昌祚老远,他跟刘昌祚交流时,且待理不理的态度也只能咽下一口气。但王中正是主帅,表面上还是很是平静,但私底下还不知将刘昌祚恨到什么样了。

不过刘昌祚的确能打仗,党项人几次攻击都给他领众轻易击退。王中正也没蠢到临阵夺其兵权的地步。

但眼下帐中的气氛实在不太妙,赵隆叫了一名亲兵进来,自己在他耳边尽可能用最大的音量来说话,然后让他传达出去:“西贼应该打不下去了。”

起头一句话,就让王中正一下提起了精神,“当真?!”

“粮草。”刘昌祚低声道,只有自己能听见。

帮赵隆传话的亲兵果然道:“西贼只会比我们更缺粮。他们沿着黄河过来的这条路,是苗帅的泾原军走过的,加上之前那段纠缠,恐怕窖藏的存粮全都给挖出来吃空了。难道还能有余力从后方转运不成?他们可是一向不擅长运粮。”

要不是之前在龛谷川发现的御庄存粮,要不是泾原路的补给,要不是吐蕃蕃军将躲进山中的党项部族像挖耗子洞一样一家家搜了出来,被耽搁了这么多时间,王中正所统领的这一军早就因为粮尽而退兵了。

王中正一下兴奋起来:“是不是再拖几日,西贼就得退兵?!”

“韦州。”刘昌祚又低声插了一句。

这下王中正却听到了,疑惑道:“韦州?”

赵隆瞥了刘昌祚一眼,让亲兵转述给王中正:“正是韦州。泾原、环庆两路惨败,只会沿灵州川退往韦州方向。但韦州能不能保得住,却是说不准。万一保不住的话,西贼是能绕道我们背后的。”

王中正脸挂了下来,没人敢将自己的身家性命放在一群残兵败将身上。

就是王中正再不知兵,也知道赵隆来跟他说这番话的意思。受困于粮草的党项人,多半已经派兵去攻击韦州,以图能绕道自家身后。必须要退兵了。

他看看赵隆,又瞅瞅刘昌祚:“谁来殿后?”

没人殿后,敌前撤军就是个笑话,但殿后又是个危险的活计,九死一生或许夸张,但生死一半一半却一点不夸张。

赵隆是提议者,当然是有了心理准备,正要站起身,刘昌祚却抢先一步:“末将愿领军殿后。”

……………………

对鄜延河东两军的诏令,已经发了出去。

基本上跟韩冈的建议差不多,命种谔和李宪收兵,稳住银州、夏州,和鄜延、河东两军之间的。但话没有说死,临机处断之权还是给了前线的将帅。

不过为了制衡种谔,体量军事的徐禧还从天子那里得到了一份拥有更大权限的密诏。对此韩冈是明确反对的,吕公著、吕惠卿同样反对,可密诏的风声虽然听到了,但没有公开的诏令,只要天子不承认其存在,任谁也没办法再说话。

当然,政事堂和枢密院可以联袂下一封堂札,宣布没有两府诸公签押的诏令,就是一张废纸。但这么做,对天子实在是太过针锋相对,谁也不愿意出这个头来提议。

很让人头疼的问题,不过也算是一个惯例了,抱怨几句,也只能放在一边。还有更多的正经事要去做。

前方的战况,是所有人都殷殷期盼的消息。尤其是王中正所统领的秦凤、泾原两军的情况,更是重中之重,如果王中正失败了,种谔也就只能回到横山脚下。如果没有失败,那么就有彻底夺占银夏的机会,甚至反败为胜的可能。

这一可能性,人人都想把握到。但王珪甚至比起天子来还要紧张三分。

而就在宋国国中的注意力都集中到了银夏之地上时,远在鸳鸯泺的大辽太师兼太傅,终于有了动作,率部抵达了大同府。摆出了随时可以南侵的姿态。

天下局面由此而兴波澜,一日一变,变动得太厉害,就是韩冈,也无法算计得清楚,耶律乙辛到底是盘算个什么。

难道先嫌宋辽夏三国之间的力量消长,还不够乱吗?

