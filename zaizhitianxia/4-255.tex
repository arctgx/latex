\section{第43章 庙堂垂衣天宇泰(二)}

【第二更】

“说是公私两便,不事聚敛,但在棉布和白糖两个新行当都占去了一部分,加上四表叔又是个堪比陶朱的,家里面的产业如今也不小了。如今家里面开销虽大,可花的还没有赚的十分之一。”

“也是给儿女预备的。”

‘儿女?……’

王旖上下打量着周南。才两个月,还不显怀。褙子里面是一件略显宽松的浅葱色对襟小袄,胸口高挺,腰身纤细,小腹也是平平的,完全看不出是怀了身孕的样子。

“这一次也不见害喜。要不是总觉得困,请了人来问诊,还真不知道是喜脉。”周南憧憬道,“安安静静的,多半是丫头。”

“再生个女儿才好。家里儿子一堆,天天吵得头疼,女儿就那么一个。”王旖说是这么说,但她瞧着几个小子闹腾的样子,却是带着笑,“官人都说了,嫁女儿不会心疼嫁妆。”

两人正说着话,忽然就听到前院传来一阵咚咚响的升堂鼓。

“今天官人审案?”周南惊讶的问着,转运司衙门的升堂鼓平时可是少听到。

王旖点点头,回答了周南的疑问,“早上出去的时候,的确说有一桩分产的案子要审。”

“好端端的怎么想起来审案的,往常不是发回州里的吗?”

“我也是这么问的,官人说老是将案子转发、退回不太好,总得留个一件两件的下来,妆点一下门面。”

转运使的工作,除了保证地方税赋顺利运进京城之外,还有监察领下军州施政和财务的职责。正常的转运使,一年通常有半年在外巡历州县。而韩冈仗着他有打通襄汉漕运的工作,就只巡视过漕运沿线的军州,远上一点的军州,几乎都没有去过。

而除了监察,而一些案子也会越过州县递到转运使衙门中,转运使也有审理权。不过韩冈一般不会接。会闹到路中监司的案子,要么是大户人家的争产案——家底薄一点,在州县里就倾家荡产了;要么就是无头、积年的刑事案——基本上都事关人命,否则也不会让人闹到路中。

如果是事关人命的刑事案,那是属于帅司和宪司的差事,韩冈会移牒经略安抚司或是提点刑狱司。

几个路一级的监司都不在同一个地方——如京西南路安抚司在邓州,由邓州知州兼任,北路则在许州——这是为避免诸司聚集一城,最后权力为一人控制。但公文往来就麻烦了。幸好这等不服州县判状的刑事案,韩冈才遇到两起,也没费他多少时间。

而争产案则稍多一些——就跟后世一样,民事案件比起刑事案件要多得多。基本上是兄弟争产的为多,也有女婿与幼子争产,继母与嫡子争产。在孔方兄面前,孝悌什么的也就甩到九霄云外去了——韩冈基本上都是发回州县,他不是很待见兄弟姊妹之间为钱闹成冤家的事。而且以世间的风气,能将案子打到路中,两边基本上都不会是安分守己的人。

不过转运司毕竟不是负责断案的衙门,一年来撞到韩冈手上的案子也就这么十几宗,都是上述的两类案子,没有一宗例外。韩冈也没有收审其中任何一桩。也就是今天,想审一桩出来应付一下。

“应该挑的是一桩简单的案子吧?只是应付差事。”

“能从县里打到州中,又能从州中折腾到路上,事情总归不是那么简单。”

只不过才说了两句,又是一片喧腾在前院响起。连三个小家伙都忘了踢球,疑惑的望着声音传来的方向。隔得有些远了,听不太清在喊些什么。但参杂在喊声中的退堂鼓,王旖和周南还是都听出了节奏。

“怎么这么快?”周南疑惑着,“升堂鼓才敲过啊。”

“肯定是结案了,而且判得妙。若是不合人意,衙门外的百姓只会私下里传言,不敢这般喧哗。”

“结案了!结案了!”一个婆子从前面过来,啧啧称着奇:“龙图果然是天上的星宿下凡。闹得郢州州里县里都不安生的案子,到了龙图手上,竟然就这么结案了。”

“怎么审的?!”王旖心中有几分好奇。

那婆子到了主母近前,眉飞色舞的说道:“龙图开了大堂的中门在审。拿着郢州的判状来问王家兄弟哪里不满意。哥哥说分给弟弟的那一份多,弟弟说哥哥的那一份更多。龙图问了两遍,都不改口,就做主将分给弟弟的给了哥哥,又将哥哥的那一份给了弟弟,这下两边都如愿以偿了,实在是有苦都说不出。外面的都在说龙图判得妙。”

周南听了,掩口就笑了:“官人这判得倒爽快。”

王旖皱着眉:“好像过去有过类似的案子,不过好象是兄弟争房产。”

“小妹倒没听说。”周南还是忍俊不禁,“不过官人如此断案,倒是促狭了,真不知那两兄弟听到判词后是什么脸色。”

“促狭?为夫判得可是再正经不过。”刚刚才将案子给结了,韩冈竟然就抄着手回后院来了。王旖和周南起身行礼,满院的仆妇都低了半截,三个孩儿上前喊着爹爹,韩冈一一应过,坐下来喝着下人奉上来的热茶,“类似的案子过去可不止一件两件,也算是最好断的案子了。为夫这边是觉得总将案子退回州县不太好,干脆挑个简单点的来审。却不知郢州是怎么弄的,竟然审不下来。”

“要是郢州的州官能跟官人比,当也能做转运使了。”王旖随口奉承着韩冈,见丈夫的视线在院中梭巡,像是在找什么,又解释道,“素心和云娘正在对帐,还要一阵子。”

“对帐?”韩冈沉吟了一下,点点头,“也快到冬至了,的确该先将帐先清一下。”顿了顿,问道,“今年府里没有什么大项支出吧?”

“还真没有,”王旖说道,“不是在京城里面,人情往来少了许多。又没有添丁进口,没几处需要花大钱的地方。虽说是多了一班门客,但也没用上多少。但进项却不少,比起去年竟翻了一倍。”

王旖说到这里,就有些犹疑,韩冈笑道:“如今熙河路一年的税赋加起来快比得上秦州了,朝廷一声令,拉出十万蕃军也不在话下,交州的情况只会比熙河更好。两边既然发展起来了,顺丰行的家底自然是水涨船高。”

“熙河路都能拉出十万蕃军了?”周南咋舌不已,“官人领兵攻交趾,满打满算,也不过动用了不到两万的官军。有这十万蕃军,还不得将西夏都给攻下来。”

“熙河可比交州难多了,十万蕃军当真点集起来,人吃马嚼,路中的那点存粮连十天半个月都捱不过去。交州那是自己维系粮草,调了再多的兵将,也不用惦记肚子能否填满。交州七十二家蛮部念着过去的好处,巴不得对外开战。”

在交州,分出去的那七十二家蛮部,耕地做工的都是交趾奴隶,还有家里的女人,男人是不做事的,整天都是跨着弓刀,转悠来转悠去。

这些都是朝廷养的恶狗,官军留下来的威慑力让他们只敢对着外面龇牙咧嘴,如果附近的真腊或是敢有半点不顺,又或是他们中间有哪人有异心,交州知州只要一句话,就能把他们放出去杀人放火抢一票。

“不管怎么说,交州和熙河现在可都是不用担心了。”周南笑道:“官人接下来肯定是要推动铺设轨道了?比起开河更方便。”

“没那么简单。”韩冈叹道,“人想中进士要十年苦读寒窗,树要成栋梁需百年雨露风霜。轨道从发明到应用,至今也不过四五年的时间,不论是矿山还是港口,都是短途。方城山中的轨道也不过才六十里,已经算是长的了。就这么区区六十里,不论从车辆和轨道本身,还是在运输的调度和维修上,都出现了许多始料不及的问题。”

他对着专心在听的两女说道:“如果是千八百里,暴露出来的问题只会更多。为了解决这些问题,需要进一步投入大量的人力财力去研究、去应对。……而且还不单单是轨道本身的问题,与地方州县之间权益的分配更是重头戏,也不知要争上多久。如果没有人在京中为轨道主张,半途而废都有可能。为夫可是想着能惠及万民,一番心血总不能付之流水。”

王旖和周南头点得都有些沉重,韩冈的这一次的谋划,没有瞒着她们。只有回到朝堂,才能保证下一步的计划顺利进行。

“两本书已经写成了,接下来就是要付梓。准备献给天子,让地方官员参考着如何应灾防疫的《肘后备要》,不用印,工工整整的抄写一份送上去给天子就行了。而《桂窗丛谈》则是要印个几百部出来,分送亲朋好友,借他们之口,将声势扩大开来。”韩冈在阳光下眯起眼,微笑着,“伏龙山那里的消息现在也该传到襄州了,再过几天,就可以把黄庸请到家里来,借助他的手,在襄州城中推广。”

“这么一来,官人可就坐实了星宿下凡的身份了。”王旖笑道,“还是说药王弟子比较好?”

“世人多愚,凡事总是联系到神怪上,但为夫巴巴写书,把整件事的来龙去脉说个明白,就是怕给那一等巫婆神汉给利用上。为夫不信神佛,虽然有些事用眼下所知的理论的确解释不通,但将理由归结到神佛之上,空长个脑袋是做什么用的呢?”他冷哼一声,“什么瘟神、痘神,迟早要一扫而空。有钱拜那等土偶木雕,还不如拿出来施粥散药,做些好事!”

