\section{第30章 随阳雁飞各西东(四)}

黄昏之后,章敦轻车简从,悄然来到王安石的府邸。

其实执政私访平章府上,还是挺犯忌讳的一件事。就是还在两府门外的韩冈,经常拜访王安石或是章敦都不是那么方便。

但今天之事对章敦而言至关重要,他想要征得王安石的支持,就必须亲自登门造访。而不是靠韩冈或是其他人,甚至几封书信能解决问题。

“子厚你今日造访,可是为了吕吉甫的那一份奏章?”王安石喝着紫苏饮,开门见山。到了他这个地位,在政事上无欲无求,也不需要跟晚辈弯弯绕绕了。

“正是。”章敦不以为异,王安石本就是急脾气,“眼下兴灵辽人蠢蠢欲动,吉甫忧心国事,想要一知兵良臣为其后,也好安心进京。不过仅仅是一名知兵的良臣,章敦觉得尚远远不够。银夏种谔、环庆赵禼、泾原熊本,互不统属,其力三分。若只是万余辽师,数万党项,各守一方倒也不惧。可万一辽人遽兴大军,或有被各个击破的危险。”

王安石沉吟了片刻,又问道:“不知子厚作何想?”

“以敦之愚见,当可如熙宁旧例,设陕西宣抚司,以宣抚使统括西北大局。”

“就是以吉甫为宣抚使喽?……”吕惠卿的想法,王安石看到奏章就知道了,而章敦的建议也不出所料。“薛子正和玉昆是什么想法?”他问着,单刀直入的问题一如性格般迅急。

王安石可不信章敦事前会没有跟韩冈、薛向两人商量过。让吕惠卿如愿以偿的留在陕西,这可不是小事。章敦不征得他们两人的同意——尤其是影响力极大的韩冈——就算得到了自家的同意,也照样有被坏事的可能。何况在吕惠卿一事上,他们三人当是有着相同的利害关系。

“玉昆说了,吕吉甫若能坐镇在陕西,他这边跟萧禧聊起来也就容易了许多。”

宣抚司的成立有着很大的象征意义。这一点,王安石、章敦、韩冈他们明白,对面的辽人也同样明白。一旦陕西宣抚司成立,就会成为韩冈手上与辽人讨价还价的筹码,是大宋宁为玉碎不为瓦全的证明。压制起萧禧,当然会容易不少。

王安石神色一动:“萧禧难道已经有所索求了吗?”

“何须他开口?只看兴灵方向上辽人的异动,就知道耶律乙辛的想法。何况他派来的这位正旦使还姓萧名禧。”章敦声音变得高亢了一点,“辽人欲壑难填,耶律乙辛更是要借中国的财力来稳定自己的权势。只要两边消息一通,确认了边境上辽人的动向,萧禧便立刻会开口索要岁币和土地!”

王安石摇了摇头,理由还是稍嫌牵强了一点,“兴灵的辽人迁移过来不过一年多,还没有这个实力。”

“相公,银夏路可是有个种谔!”章敦急忙又道,他自问找了个好借口,“没有一个宣抚使,如何能拉住这匹劣马的笼头?”

种谔的脾气和性格,王安石也有所了解。好歹是天下排在前几的名将了。章敦说的话,还真是一个好理由。挡回辽人的贪欲,这是现在朝廷想要做到的。但打得过分了,将战事扩大,却也不是朝廷希望看到的。这样一来,好战的种谔就是一个由糖块包着的毒药,吃在嘴里很甜,但外面的糖块没了,毒药可就出来了。只是临阵换将,却更是一个糟糕的选择。

想了一想,王安石道:“薛向呢?他怎么看?”

“薛子正也是觉得这样比较好。”章敦回道:“而且他还建议说最好在关东修一条轨道,与关西的宣抚司相配合,共同来压迫辽人。”

“轨道?”王安石眨了一下眼,今天的‘惊喜’还真是一个接着一个,“……铁轨还是木轨?

“如今都已经是铁轨了。”章敦放松的笑道,“京城的码头上全都是铁轨车。”

“不是说铁轨长了就有问题吗?”王安石对铁轨依稀有些印象。

“那是之前的事了,是因为不知道热.胀冷缩这个道理的缘故。”章敦解释道,“凡物遇热而胀,遇冷而缩,铜铁五金之物尤其明显。京城的码头上一开始,一段段铁轨都是靠得很近,如同木轨一样。但之后换季时,铁轨不是两头相接挤在一起,就是缝隙扩大。从此之后,每一段铁轨和铁轨之间,都会留下空隙。具体的间隔,都有经过验证。”他又苦笑了一下,“这个道理玉昆似乎事前就知道了,但他偏偏不说,直到出了事,才诱导人去探究其原因。”

王安石深有感触的点了点头,以他对韩冈的认识,这种事自家的女婿多半做得出来。

几声喟叹,他又道:“河北铺设轨道,陕西设立宣抚司,的确能让辽人知道中国决不妥协的决心。”

“相公误会了。”章敦急忙更正道,“不是河北。”

“不是河北?!”

“关西设宣抚司,河北再开始修建轨道,未免显得咄咄逼人,万一辽人以为朝廷准备开战,反而就没了转圜的余地。”章敦基本上就是将韩冈之前说的话转述出来,“而在京东沿着汴河铺设轨道,配合起关西的宣抚司来,对辽人依然是警告,却不会显得过于锋锐。而且多了一条宿州至东京的铁轨,对补充汴河运力不足也是一件好事。同时河北轨道太长,几近千里,而宿州至东京不过六百里,于途需要跨越的河川少且窄,也简单了许多。等有了五百里以上轨道的经验,下一步才是近千里的河北轨道和从京兆府到京城的轨道。”

章敦一口气说完,王安石只是点头,却不言可否。

他心中暗叹,章敦、薛向和韩冈其实都已经有了自己的盘算,却是正好利用吕惠卿的私心来为己谋划。喝了口茶,润了润喉咙,王安石沉声问道:“子厚,你可知道蔡持正午后入宫时说了什么吗?”

“知道一点,但并不详细。”章敦点头,这是他出宫前听到的消息,只有一句话,要想了解更为详细的内情,就要等到明天了。不过这一句话已经足够,“蔡持正希望让成都府路的蔡延庆改判京兆,这个人选并不差。经历也好,能力也好,都是上上之选,如果仅仅是稳定京兆府的话,也是足够了。不过蔡延庆帅长安,绝比不上吕吉甫任宣抚使更能压得住阵脚。枢密使兼宣抚使能控制得了环庆、泾原和银夏的兵马,而区区一个永兴军路经略使,则远远不够资格。而且陕西宣抚司成立,也能警告辽人,中国已有防备。正如弦高献牛酒于秦师,甚至可以不战而屈人之兵。”

章敦说了很多,但王安石仍是半点不信。

他可是在京内京外做了几十年的官,朝堂和地方的政务、刑名、军事、人事,哪有他不熟悉的?如今虽没了与后生晚辈周旋的精力,心境也远不及过往,但这并不代表他的眼力退化了。

王安石不信韩冈会担心争不过一个契丹人,也不信韩冈会在与萧禧的交锋中落下风,他太了解自家的女婿了,前面章敦帮他说的理由仅仅是借口。自家女婿应该只是单纯的希望参与编纂了《三经新义》的吕惠卿在外面待久一点,在自己和程颢进入资善堂后,京城里再多一个能拉下脸皮来坏事的吕惠卿,对气学的压力就未免太大了一些。

同样的道理,其他人皆是有自己的盘算。拿出来的理由,看起来再怎么冠冕堂皇,或是听起来如何如何的推心置腹,其实都不过是借口罢了。

王安石看得很清楚。花白的双眉微垂,昏黄的老眼中投出来的目光,却比年轻人更为犀利。

蔡确阻止吕惠卿留在陕西,是不希望多上一名东府的同列——若是吕惠卿在关西立了大功,不是不可能被晋升为第三位宰相。韩子华已经老迈,但吕惠卿英气勃勃,正当盛年。一旦他立功后做了宰相,凭借过去在朝堂中留下来的人脉,经历太少的蔡确会被他完全压倒。

吕惠卿是想多立功勋,以便回京后能压得住章敦、薛向,乃至游走于外、但影响力犹有过之的韩冈。如果机缘来了,功劳再大一点,甚至可以直入宰相班。这可比现在就回来,被章敦、薛向、韩冈三人联手挤兑得没处站要好。而且御史台人事更迭剧烈,吕惠卿旧日能联络得上的御史全数出外,一旦与其他早早便在京中的宰执们争执起来,局面将会极为被动,这当也是他不选择立刻入京的缘故。

薛向的盘算则多出于私利。在六路发运司中,薛向的势力盘根错节,纵然离任已久,依然可以借助旧日的人脉遥遥控制。对其他发运司、转运司,他也同样有着不小的人脉。不过在轨道出现后,在水路转运以外又多了一个同样便利的选择。以薛向的眼光和见识,多半是看到了轨道大兴的趋势不可避免,为了在一开始就获得对轨道运输衙门的影响力,便主动开始寻求操纵轨道修筑工程的机会。吕惠卿的奏章给了他这样的一个机会,所以他拿自己的支持跟章敦、韩冈做了交换。

至于章敦……王安石暗自冷笑了一下。

西府之长长期在外,枢密院中必须有知兵的名臣主掌内事。但缺一个进士出身的薛向;尤在河北镇守的武将郭逵,都不可能入选,章敦是眼下唯一的人选。但东府现有两名宰相,枢密使吕惠卿一旦留在陕西出任宣抚使,若是只有两名枢密副使出掌枢密院,只会导致轻重失伦,内外失衡的局面。就天子和皇后而言,必须提拔一位地位相当,至少相近的主官。

判枢密院事、枢密使、知枢密院事、枢密副使、同知枢密院事、签书枢密院事、同签书枢密院事,前三个为正任,后四个为副贰,西府执政的名号这样按地位高低一级级的排下来,时任枢密副使的章敦,他盯上的是与枢密使近乎平级的知枢密院事,在吕惠卿不在的时候以知枢密院事来主管西府内部事务。

吕惠卿的奏章传来也不过半日。只在半日之间,这些后生晚辈——好吧,与自己年岁相仿佛的薛向可以不计入在内——就已经做好了各自的盘算,甚至联合了起来。王安石也不免兴起一股长江后浪推前浪的感慨。

但看着面前神色诚挚谦恭的章敦,王安石脑中又不由得冒出一句老杜的诗来:

君看随阳雁,各有稻粱谋。

