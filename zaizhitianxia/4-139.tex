\section{第20章 冥冥鬼神有也无(15)}

【晚上补上昨天欠的一更。】

最后的会议在入夜后就宣告结束。

第二天一大清早,天色仍是暗淡的时候,一声嘹亮的号角声划破了长空。

徐百祥从昏睡中惊醒,从设在七尺多高、只有半尺见方的小窗中,响亮的号角声传了进来。

‘是出什么事了?’他想着。

‘难道是交趾人又打回来了?’他又进一步的幻想着。

舔了舔不剩一颗牙齿,而带着一股血腥味道的牙槽,徐百祥想站起来。但他只是稍稍移动一下,挂在身上的锁链便是晃动不休,在狭小的狱室中,发出哗啦哗啦的刺耳警告。

“徐百祥,你想做什么?!”

昏黄的灯光下,就坐在对面的两名狱卒,厉声断喝的就是他们其中一人。两人正瞪着眼睛,盯着徐百祥的一举一动,手上还紧紧攥着铁尺,必要时可以一下将这位罪囚给废掉行动的能力。

在徐百祥被押入牢狱的时候,经略章相公曾经吩咐下来,不许让这名罪魁祸首死了,无论是怎么是自杀、他杀还是因病瘐死,监管他的狱卒们一律从重论处。继而韩冈、苏子元等几个治理广西的重臣,都如此吩咐下来,监管徐百祥的狱卒们,都是日日提心吊胆的监视着徐百祥的一举一动。

他满嘴的牙齿都被强行拔掉,省得他咬舌自尽,双手用了三十斤的重枷牢牢锁住,而双脚也都拷上最为沉重的镣铐,任何行动都会被轮班住在同一间牢房中的守卫们盯着。从早到晚、又由暮达旦,一天十二个时辰,每一个时刻总会有两个人、四只眼睛盯着他,不会放过他的任何一个可疑的动作。

监狱中没人知道为什么几个相公、府君都如此重视这名逆臣贼子,只知道他是被交趾人抛弃的狗,背主之后,又被新主人抛弃的狗。应该早早的杀了,让邕州百姓痛快一番,也好让他们回去祭拜交趾人之手的家人。不过今天他们终于知道留着徐百祥的性命到底是为了何事?

“大帅传徐百祥!”

在狱中孔目的带领下,两名身材健硕伟岸的军汉,来到徐百祥所在的牢房前,提高了嗓门向里面喝着。

“什么?”两名狱卒一见是顶头上司带着人来了,连忙起身,惊讶的问道:“大帅要传徐百祥?外面不是正准备出兵吗?”

“出兵哪能不见血?大帅正要拿这名狗贼誓师祭旗!”一名军汉喝着,“养了这么多日,可不就是为了今天!?”

徐百祥一听之下,还抱着一丝幻想的他,顿时沦入完全的黑暗,拼了性命的开始挣扎起来。他不甘心就这么死了,他还要做知县、作知州,做掌控一国的权臣。

一名狱卒立刻转身蹿回牢房中,对着徐百祥的脊椎骨抬手就是一铁尺,“狗贼,终于等到今天了!”

沉重的铁尺落在脊背上,正是捕快捉贼的手段,正在拼命挣扎的徐百祥顿时就瘫软了下来,身子都在一阵冲击中麻木掉了,被人从牢狱中直接拖了出去。只是两只眼睛瞪得大大的,满是不甘心。

“此人正该千刀万剐!”

“想学张元、吴昊,也得先把眼睛长囫囵了。狗眼瞎着,连投主子都投到了一个贼身上!”

“做狗的,被人当死狗丢下,活该有今日。”

高台上,几名将校评论着刚刚上场的主客,在高台之下,是即将南下的近万马步禁军,排出了一个个整齐的阵势,等待章惇的检阅。就在他们周围,还有数以万计的邕州百姓——今天,他们几乎是倾城而动,就是为了一见血海深仇的死敌的结局。

经略招讨司选定的誓师出兵的地点,不是在城外的校场,也不是在城中的衙前,是在祭祀苏缄和一众在邕州一役中殒身殉国的文武官员,乃至士卒、胥吏的忠勇祠中。同时在忠勇祠中的后殿里,几面墙都是髙入房梁的石板,上面刻着密密麻麻的人名,都是从一名名活下来的邕州百姓那里,搜集到的冤魂名单。

数以万计的人命,都是因为一人的野心而惨遭屠戮;好好的一座城池,就是因为一人的贪欲而陷入火海。

愤怒的吼叫从四面八方传过来,不像来自北方的官兵们对徐百祥事不关己的评论,来自于邕州数万百姓是单纯的愤怒和憎恨。多少个家庭不复存在,留下来的人们只能追忆着往昔的幸福时光。

“剐了他!”

“剐了他!”

“剐了他!”

无数人的怒吼着,两名刽子手,将徐百祥高高绑在木台之上,咬在口中的是一柄巴掌大小的匕首。

徐百祥万分恐惧的瞪着眼,看着两名刽子手嘴角边上的寒光闪动。双脚已经没有任何气力,浑身都软.掉了,淅淅沥沥的水迹在股间洒落,一股异臭在处刑台上弥漫开来。

刽子手脸上泛起作呕的表情,撇开眼睛。然后就拿着小小的匕首,慢条斯理的开始向下一片片的削着徐百祥身上的皮肉。每一刀下去,木台上的徐百祥就是一阵嘶声力竭的惨叫,每一条肉被削下来,就立刻被人送去到邕州百姓那里展示。

骚动出现在百姓们的行列中,不知多少人在哭喊着涌过来,要亲口一尝仇人的血肉。

“此为后来者之戒!”章惇冷冷一喝。

“太便宜他了。”

韩冈看了两眼就转移了视线,他对折磨一个该死的囚犯,没有什么兴趣。韩冈只在乎最后的结果,泄愤式的行为并不合他的性格。能让邕州百姓一舒旧恨,让军民同心,凌迟也好,斩首也好,杀牛祭旗也罢,杀人祭旗也罢,都不过是一个形式。只要能振奋起士气,什么手段都可以,没有什么差别。

天上的太阳有几分黯淡,已是冬月,刮起来的风更也冷了一些。风很是干燥,其中并不带多少水意。这样的日子,在广西实在太少,只有短短三四个月。而这短暂时间,如今已经过去了一个月。

可是要尽快了!

……………………

自从到了门州,李洪真才发现自己上了大当。

难怪自己主动申请调往门州领军的时候,李常杰也没有怎么犹豫就答应了下来,一点都不担心自己会投向宋人,再利用宋人的军力来争夺交趾王位。

李洪真想不到宋人竟然要将全数大越裹得男丁处以肉刑,而且他们还不是自己做,反是让过去数月之中,与交趾结下了血海深仇的三十六峒蛮部来做。

本来李洪真从宋人唆使三十六峒蛮部的手段中,能看得出来宋国并没有打算夺占大越国的土地,只是用杀戮来报复此前李常杰在邕州、钦州和廉州的杀戮。要不然也不会任由蛮人在大越国境内胡来。抓到了这一点,李洪真就有充分的把握去说服宋人的主帅——也就是章惇和韩冈——将李常杰、倚兰以及乾德一起卖掉,至少能保住富良江南岸的国土。

可事情完全出乎李洪真的意料之外,宋人竟然会用上如此狠辣的手段,甚至连消息都不加以掩饰。门州已经传遍,想必过上十数日之后,就能遍及大越国中的每一个角落。

宋人如此做,国中对宋军的抵抗肯定会变得激烈起来,但他们的激烈抵抗却持续不了多久,只要宋军的攻势超过了一个限度,在必死和失去脚趾之间,人们就肯定会选择后者。

好死不如赖活,少了两小块带骨头的肉而已,又不会送命,只是不能再上阵打仗了,做些农活还是可以的。眼而下的大越国中,内忧外困,愿意从军打仗的人数可以说是寥寥无几。

而且没人会怀疑宋人的承诺,只是砍下脚趾,并不是要伤害性命。

若是说投降后就能好吃好睡,不会受到欺凌,恐怕没人会相信;说施了肉刑之后,就能保住性命,却没人会怀疑,既然都已经砍了脚趾了,再杀了自己又有什么意义,还不如留下来使唤。这个道理能想得通的人不少。

早前李洪真的父亲李佛玛修建宫室之后,大举招募宫女和内侍。虽然要想做个安安稳稳的内侍,有一件东西必须抛弃,只要是真正的男人,都不愿抛弃的东西!可为了得到一口饱饭吃,就在升龙府中竟然连自阉的都有。想来只要宋军攻到门州城下,他麾下的守军肯定会有不少人为保住性命,能主动自残肢体。

“好狠毒的手段!”

李洪真咬牙切齿。但发狠过后,便又心慌意乱起来。凭着门州的兵力,他怎么抵挡宋军的攻势。慌乱的心思一直持续着,直到南下的宋军,越过国境出现在他的面前。

“只有两千?”李洪真惊讶的问着。

“只有两千!”赶来禀报的守将点头肯定。

李洪真立刻匆匆的亲自上了城楼,向着城下望过去,在城外列阵的宋军数量,竟然当真只有两千上下。

‘难道宋人就只有这么多兵?’李洪真惊讶的撑起双眼,宋军不但人数微薄,而且连座军营都不准备,难道想一战就攻下门州。

‘绝不可能!’门州可是最重要的关隘,哪里可能一战而下,李洪真正在这么想着的时候,突然看见宋军阵营后出现的异物,眼睑都要裂开来了,城头上多少人难以置信的呻吟出声:“那……那究竟是什么啊?!”

