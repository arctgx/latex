\section{第六章 见说崇山放四凶(11)}

王厚刚刚被章辟光送了出来。

离开开封府衙后,王厚便向朱雀门进发。

上百骑兵行走在夜色中,在光线照不到的角落里,也有窥伺的目光在闪动。但看到王厚一行,他们就躲藏得更隐秘了几分,甚至连目光也隐去了。

王厚看见了,却没有捕捉他们的意思。

这些都是各家派出来打探消息的仆役,当初王韶还在京中任枢密副使的时候,遇上朝局动荡,也没少派家丁出去监视道路,打探消息——由于出身军中,他们表现还相当不错。

抓这些耳目,平白得罪人,就是送进去也会被放出来。而破坏约定俗成的惯例,在京城中可就要被视为异己,受到抵制甚至攻击。

而且王厚还想早些跟李信通个气,有些事不能依靠亲信来传话,面对面的交谈最为安全,不能在浪费时间。

这已是王厚今天第三次押送人犯至开封府。其中还有些是犯官的家眷,一路上哭哭啼啼让人好生心烦,真不如第一次跟章辟光一起押送蔡家叔侄,直接堵上了嘴。

说起来还是有了功名敢下手。王厚要顾忌文官们的想法,但同为进士的章辟光完全不在意,之后送到开封府,沈括那边

一开始的蔡京已确定下狱,听章辟光的口气,这两天就处理了他。

狱中料理犯人的各色手段源远流长,博大精深,只是王厚并不好学,也就没细问了。章辟光想要表现,就让他表现好了。

跟在蔡京之后,一批接着一批的逆党被送进开封府。押来的人犯一多,原本面积并不算小的开封府狱,就变得拥挤起来。

原本只惯了三五名犯人的牢房,一下塞进了十几人,别说躺下来睡了,就是站着也嫌挤。

这样的混乱中,一两个犯人出点意外,发些急症,真不是什么大事。

沈括那个胆子,不敢下手帮韩冈,但也不敢坏事。有章辟光在中间下手,蔡京逃不了。

出来时章辟光,给了他一个机会。没有韩冈的支持,沈括就别想入两府,这节骨眼上,怎么能犹豫呢?

不过王厚倒没觉得自己之前没有抢先将蔡京解决有什么大不了,相比起大庆殿上抢了武器杀出来对韩冈的帮助,这也算不上是个事。只是犹豫了一下,给章辟光抢了先去。

但最丰厚的奖赏,在此之前就已经确定了,全然不需要的多担心。

马蹄声得得响着,蹄铁敲击着青石路面,几十匹骏马踏出的节奏交织相融,如雨打芭蕉般的清脆爽利,仿佛王厚的心情。

只是没有过久,王厚拉起缰绳停了下来。

一支巡夜的小队,正押着三人从前面过来,王厚的亲随见状,便迎上去询问究竟。

“怎么回事?”

待亲随回来,王厚就问道。

“回皇城的话,他们是犯了夜禁。”

“夜禁?今天还有人敢犯夜禁。”

现在可不是白天,可以光明正大的走在御街上,更别说今天如此特殊了。

一干人躲在阴暗处可以当没看见,但鬼鬼祟祟的想要横穿御街,被抓到就没有放过的道理了。

王厚瞟了三人几眼,其中一人穿着最为华丽,与两名仆役装束的汉子截然不同,明显是做主人的。不过长得膘肥体壮,满脸横肉,倒像个土财主。

整个人被困得结结实实,双手被绑在背后,嘴上也勒了一圈,喊不出话来。只是靠近了,往后就从他身上嗅到了浓浓的一股酒味,还有桂花香,也不知是在哪里蹭了一身的香粉。

天子丧期之中,天下禁乐,京师的时间尤其长,可这一位明显就是喝了花酒回来,又正好给巡夜的撞上了,当然不能放过。

官府的棒子不打勤的,不打懒的,专打不长眼的。

“夜半不归,看着就不是好人。”王厚笑道。

王厚看了醉鬼几眼,没什么兴趣的摆了摆手:“罢了,送他到开封府吃几天黄粱糙饭就好,还能减减膘。这身板再胖下去,到了祭春就该挨宰了。”

王厚说了个好笑话,手底下一群人哈哈的陪着大笑起来。

却听见前面有人一声呵斥,“是何人在御街上喧哗?”

笑声猛然一窒,王厚抬起头望过去,迎面过来的一队人马。

近了之后才看清楚,那并不是巡夜的兵卒,而是为重臣开道的亲随。

王厚顿时就皱起眉来。

万一是哪家脾气不好的文臣,这就又是一封弹章背上身——这个日子,可不是能放声大笑的时候。

不过等他看清了灯笼上的字号,神色就放松了,拍马迎了上去,“可是东莱韩府?”

“啊,是处道啊。”

……………………

韩冈从内东门小殿离开时,已经夜上三更。

拜除王安石为平章的诏书已经写好,就待天亮发出去。

而韩冈所提议的选举,费了点周折,则也拟定了诏书和细则,这还要与宰辅们进行讨论。

太后同意了,平章军国重事也同意了,仅剩的宰相和参政,也就是韩绛和张璪两人,也不可能同时否定太后和王安石的意见。

不过其中也做了一些补充,尤其确认了两府中,不同位置上的候选者的范围。

比如宰相这个位置,两制官是不可能一下就坐上去的,必须是现任的执政,或离任的宰执才有资格。而枢密使,参知政事可做、枢密副使也可以升任,枢密副使转任参知政事也十分常见,都不会单纯由两制以上官来参加选举。

所以暂定下来的,没有担任过宰执的两制以上官,只能为枢密副使。只有翰林学士中资历最老的翰林学士承旨才有资格,与枢密副使,和曾为执政的重臣参加参知政事的选举。至于宰相和枢密使,就没有两制官出场的空地了,只有现任和前任的宰执才能作为候选人。

至于专供有功名将的签书枢密院事,因为是另一个体系,本身也没有实权,则另当别论,并不计入选举的范围之中。

讨论完这些琐碎的细则,时间已经不早,送了疲惫不堪的太后回寝殿,韩冈也再一次从宫中出来。

太后又忘了让韩冈留在宫中,而没有进入两府的韩冈,也没打算在宫中过上一夜。宿直的是诸位宰辅。

王安石倒是留下来了,他是新任平章。

在一起从内东门小殿中出来之后,王安石并没有向多问什么,包括韩冈的动机,以及这个想法的来源。而是直接去了其他宰辅们落脚的地方。

韩冈正准备回家舒舒服服的睡上一觉,没想到在路上碰见了王厚。

……………………

听到王厚的声音,韩冈挺惊讶。

以王厚的性格,不应该这么轻浮,半夜里在御街上大声说笑。

“处道?”韩冈惊问。

“真的是玉昆你。”看见韩冈当真在人群中,王厚上前说着,“不是才入宫去?怎么就又出来了?”

“没什么好奇怪的吧。难道还能住在宫里面不成?”韩冈笑着说。

“两府可都没出来。”王厚回头往宣德门的方向张望了一下,转回来就压低了声音,“方才进去的也不只玉昆你,可出来的就玉昆你一人。”

王厚领军巡视城中,知道韩冈和王安石入宫也不足为奇。

韩冈道:“家岳已再任平章,所以留在宫中。”

“……那玉昆你呢?!”王厚愣了一下,然后问,“难道还要辞了再接?”

“不是。”韩冈摇头,“暂时不会有诏书。”

王厚的脸色变了:“今天这么大的功劳,还进不了两府,日后谁还跟叛逆拼命?”

心情急躁之下,连声音都变了腔调。

韩冈则笑道:“两边有关系吗?”

“玉昆,是不是因为蔡京?!”王厚厉声道,“你还不知道吧,蔡京已经下狱了,府中的章判官会处理好的。”

不从贼者有功。若是蔡京援引这一条,说起来的确能脱罪。可谁帮他说话?

王厚相信章辟光的能耐,更相信他的胆子。只冲着韩冈的面子,这位章判官可就巴结上来了,相信他绝对有哪个胆子搏一搏。

“没事,不是蔡京。”韩冈笑着摇头,“蔡京不算什么。是我的建议。”

“玉昆?!”王厚一声压低嗓门的怒叫,差点就忍耐不住。

只是看了看左右,他还是按捺了下来。调转马头,护送韩冈回去。

与韩冈并辔而行的时候,王厚小声问:“到底是怎么回事?”

韩冈没有藏着掖着,而是很坦率的将自己在殿上的提议都转告给了王厚。

“平白添这番周折做什么?”王厚难以理解,“若是太后亲自选定……玉昆,你是不是不想进两府?!”

木秀于林之类的话,王厚不想再多说,韩冈肯定知道这一点,而且他也从来没有为此而避让过。

但这个提议对他有什么好处?

最高兴的会是谁?

反正绝对不是韩冈。

韩冈的用心,王厚不明白,殿上的宋用臣也不明白,太后当然也不明白,但王安石应该是明白了。

韩冈也从来没指望他们能明白自己,只要能够跟着自己走就好了,

韩冈需要支持者,但他的根基是最浅的。

在京的侍制以上官有选举权,韩冈在其中能不能进前三,可真是一点也说不准。

如果是京城百万军民来推举,不会有第二种结果。就是扩大到升朝官这个层面,由在京的七八百朝官一级的官员进行廷推,结果也必然是韩冈排在前面。

但选举权现在是集中到除去宰辅后的二十三人手中,选举的条件则只会是利益的交换,只会是党同伐异。

而韩冈,他的突出,反而会在地位相近的人群中惹来反感。且论起利益关系,他与其余重臣之间的关系实在是太浅了。既非新党,又非旧党,韩冈一直以来都刻意表现出来的独立性,让他在朝中的重臣中,几乎寻找不到助力。

这样的自知之明,韩冈还是有的。

他从来不会认为那些在官场中打了几十年滚的老油条,能放弃自己已有的立场,转而支持自己。

谁会选韩冈?

