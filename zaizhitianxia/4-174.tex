\section{第26章 鸿信飞报犹觉迟(六)}

这个时代的朝廷,在信用上没有多少值得一提的地方,倒时在朝令夕改上,很有些口碑。

钱法一变再变,陕西是否通行铁钱的来回摇摆,都是一桩桩例子。为此倾家荡产的商人为数不少。

想要推行纸币,也要看看这里是不是蜀中。

蜀中因为缺铜,而外地的铜钱又不易运进去,所以一直以来都是铁钱区,而铁钱又重,不易携带,所以才有了交子的出现——这是商人们自发形成的,而后才被官府给看上。换作是其他地方,多半是宁可使用沉重的铜钱,也不会去用让人无法相信的纸币。

不过话说回来,以官府垄断的食盐为本所发行的盐钞盐引,倒是可以暂代纸币的用处。韩冈旧年在陕西,从他手上发出去的政府开支,有许多都是以盐钞的形式出现的。

陕西自来多边患,官府运粮耗费太大,为了省事,便有了‘入中’之法。商人从外地运粮上前线,而官府就给他们盐钞作为酬劳,让他们去解州盐池换盐,不想要盐的,也可以去京兆府或是东京的钞场去换钱。

纸币就是国家的信用凭证,只要盐钞可以按照面值用来交换生活必需品的食盐这等实物,就不用担心贬值的问题。而世间的商业交易时将盐钞当做钱来用,也已经并不是很稀罕了。

就算眼下盐钞也有滥发,但只要还有盐可以兑换,便不是什么大问题。毕竟准备金和发行的货币量,并不需要一比一,而是可以超发,只需保持畅通无阻的兑换途径,便不需要担心。

而且一张盐钞能交换上百斤盐,价值为六贯,商人们带在身上很方便,但普通百姓哪个也不会用,就算出了问题,影响的只是商人,最多也只会引发动荡,却不会造成国家的动乱。

韩冈转头看了章恂一眼,他还在专注的盯着在船上搜检的兵卒。章家的货船很平静。但另外一艘船的甲板上有些乱,看起来是查到了什么。

如果自己的想法说出来,章家的十一公子恐怕会在肚子里开骂了。不过韩冈却也不会当真认为盐钞出事无关紧要。

再怎么说,他家里也有个关西数得着的大商号,挂在帐中的盐钞少说也有二三十万贯,加上关西与顺丰行为盟的大商号,至少上百万贯攥在手心。以后用得着他们的地方多得是,这是他手上重要的工具,韩冈怎么也不可能看着盐钞变成废纸。

在码头上看了一阵,章恂家的商船已经扬帆起航。

章家走得是国内的航路,别说章恂他这位东主,就是下面的船老大和水手们,也都是即便一文钱也会想着在交州换成丁香、象牙,回到福建就能翻上几倍,谁也不会在船上放沉重又占地方的铜钱来。

章恂对韩冈笑道:“交州是出去的多,进来得少。蕃商多是去广州、泉州、杭州……还有京东的胶西板桥贩货,运钱出海也是在那几处为多。刚刚开埠的海门,不会有人敢干犯钱禁。”

但章恂话声刚落,从另外一条船上下来的士兵向港中的巡检报告了什么,而那名巡检则又是一脸慌张的跑来向韩冈来汇报。

“私运了多少钱?”韩冈对港镇巡检的慌张觉得有些好笑。

这名巡检当初在军中也是颇立了点功劳,最先冲上升龙府东门城头的也有他一份,怎么做了巡检后,就变得这般不稳重了。

“回龙图,不是钱。”巡检的脸色都白了,结结巴巴的流了一身的汗,“是六十三领的铠甲,还有四百多条长枪、一百三十柄刀。”

“甲胄?!”章恂在旁也变了颜色,刀枪倒罢了,民间私藏甚多,在刚刚经历过战事的交州更不足为奇,但这甲胄可不得了,三副甲胄就能将人送到斩首台上了,何况这是六十三领。

韩冈的脸色也沉了下来,问道:“是什么甲?”

“皮甲,交趾的。”巡检小声答道。

章恂松了口气,至少不是板甲。刚刚结束战争,散落在民间的甲胄也多,倒也不足为奇,不至于这么惊慌吧。他想着,忽然心中一凛,‘该不会出自府库吧?’

韩冈眼睛眯了起来,“可是问明白了来自何处?”

“听船上的人供述,是从河内寨外面收来的。”

章恂长吁了一口气,万一出自府库,知州李丰可是难辞其咎。

韩冈转过来对他笑了笑,那是看透一切的笑容,“缴获的甲胄都是有数的,点验过后存放在交州的府库中,没那么容易偷出来。倒是各部手中多多少少都有些战利品。”

章恂点点头,就见韩冈有继续问着巡检:“这一干甲胄完好的有多少,残破的又有多少?”

“大半都有些损伤,不过都不严重。”

“这艘船来海门几次了?”

巡检犹豫了一下,咬牙答道:“……这次已经是第三次。是准备运往三佛齐的詹卑城。”

‘难怪。’章恂心道。去往异国的海船本应是检查的重点,但到了第三次才搜检出来,前两次还不知给他们运出了多少去。

韩冈想了想,便吩咐道:“去通知你们的李知州,这是交州内部的事。”再看了一眼惶惶不安的巡检,笑道,“能抓到就是有功,过去的事不要多担心。”

得了韩冈这一句,巡检如释重负,连忙跪下行礼:“多谢龙图,多谢龙图!”起身后就赶紧回去让人通知城中的知州李丰。

“玉昆。”章恂犹犹豫豫的开口,私运兵器出海,知州李丰少不了要被牵累受罚,这是章恂所不想看见的,“你看这事……”

“这是好事嘛。”韩冈一句打断了章恂准备说出口的话,“诸部卖出手上的兵甲,好的肯定留着,只有破损的才会卖出来,但诸部手中的甲胄兵器减少,那都是好事。”

韩冈愿意帮忙保着李丰,自是章恂所愿。但竟然说这是好事,这让他惊讶的指着港中的那艘已经被几十名士兵控制的海船,“那这一艘船……”

“已经查出来了。”韩冈喟叹着。

如果没查出来,睁一只眼闭一只眼的就放过去了,眼下既然已经给查了出来,哪里还可能放过?朝廷的法度任谁也不能在明面上违反,韩冈也绝不会开这个口。

“那该怎么处置?”章恂又问道。

“这是交州的事。”韩冈摇摇头,转身上马。回头看看被拦在港中的那艘船,连监察港中的巡检都没打点好,便敢走私甲胄兵器,这根本是自寻死路!

李丰很快就到了港中,用了半天的时间,到了晚间,他便过来向韩冈禀报这一桩案子的来龙去脉。

“这艘船的船主刘武儿是广州人氏,一直以来都是往来三佛齐和广州,都是以香药和丝绸茶叶瓷器为主,与三佛齐王交好。因为最近国中有战事,所以要买一批军器。刘武儿受命后便来交州,向诸部搜求闲置不用的兵甲。”

“可曾审得确实无误?”韩冈问道。

“上下的口供都一样。”李丰说道,“而且听海上传言,三佛齐国最近的确在与丹眉流交战。而且船中还有一个自称是三佛齐的大臣,唤作群陀毕罗的,连三佛齐对中国历年朝贡的事,都能说得明明白白。”

“以你之见,当如何处置?”韩冈问着李丰。

李丰犹豫了一下,说道:“南海诸国以三佛齐最为恭顺,今年的贡使就是在广州登岸,就半年前的事,据说三佛齐国王还被天子封为了保顺慕化大将军。”

“南海诸国以三佛齐最为强盛。”韩冈摇摇头,他从不认为一个国家会对另一个国家心甘情愿的臣服,“现在恭顺不代表以后恭顺,四边诸国只有一直衰弱下去,才是大宋之福。想必谁也不想看到海外再出一个西夏或是交趾吧?”

多少向大宋朝贡的小国,他们所谓的恭顺全都是为了利益。如果没了利益,谁会无缘无故的向着千万里之外的中国皇帝俯首称臣?作为一国之君,在自己国家中称孤道寡难道不好吗?偏偏要接受一个万里之遥的国家赠予的官职?全都是利益!

韩冈说得是正论,李丰也难以反对。韩冈偏了偏头,问着坐在下首的一人:“行之,你这个海门知县也别光坐着,说说当如何处置?”

海门知县是韩冈的幕僚马竺,在只有一座县城的交州,也算是州中排在前面的官员了。

韩冈他身边的幕僚换得甚勤,只要立一次功劳,幕僚们便能从中得到封官的恩赏。当初跟随他的游醇三人,一个不落的得了官。而这一次跟随他南下的四名幕僚,也全都因功得到了官封。

不过马竺现在在厅中也只有旁听的份,直到韩冈问起来,他才出言道:“刘武儿私运甲兵,数目极大,肯定要依律处置,这点事没话说的。但南洋诸国以三佛齐最为恭顺,其国的大臣也不好就此论其死罪。以下官之见,刘武儿一干罪囚,当由交州依律处断,而群陀毕罗则先将其禁足,报于京城,待天子圣裁。”

凡事往上推,这是官僚的做法。虽说不能为错,但如果不能在奏章中提出自己的意见,那也别想受到上面的重视。

韩冈摇头道:“到了大宋的地头,就要受大宋律法的管,该怎么审就怎么审,至于会不会赦免,那是天子和两府的事,这边依律行事就够了。”对着意欲争辩的李丰,还有欲言又止的马竺,“既然主君是皇宋之臣,那下面的臣子当然也是。身为皇宋子民,那就别想在《刑统》下例外!”

