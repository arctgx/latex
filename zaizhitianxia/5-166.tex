\section{第18章 向来问道渺多岐(三)}

七月流火。

南方星空接近地平线处的大火心宿二,那猩红的色泽在天幕上闪耀着不吉的光芒。在无月的夜晚,天上的星辰仿佛亮了许多,平常被月光所掩盖的黯淡星辰,这时候,也一个个的出现在星空中。

苏颂在得到千里镜的这一年里,早养成了夜半观星的习惯,与同僚的交际往来,减少到最低限度上。透过千里镜观察着千万甚至亿万里外的星辰,寻找星辰轨迹变化的规律,这是苏颂如今最大的乐趣。

从韩冈书房敞开的窗户中,依然可以看到天上的万点繁星,多宝格上,也有着几架千里镜和显微镜,但苏颂却将注意力放在房内,放在韩冈说的话上。

不比在太常寺衙门里那样需要防人耳目,在私家的书房中,出己之口,入人之耳,就可以畅所直言。

韩冈图穷匕见,一点点的将自己的真实目的坦诚的告知给苏颂:“古人并不是一定是对的。比如螟蛉义子的谬误,如今是改了,但腐草化萤的谬误,千百年来却无一人指正。”

烛光下,韩冈拿出了一个杯盏大小的透明玻璃瓶。瓶中有湿土,有草叶,而在草叶上还趴着几只飞虫,再仔细一点看,还能看到瓶底中,还有几只毛虫状的黑色爬虫。

若在平日里,苏颂多会嘲笑一下韩冈的奢侈,拿着价值十几贯的玻璃瓶养虫子。但眼下他便无余暇去做这样的闲事,韩冈既然说腐草化萤是谬误,那么瓶中的自然是萤火虫。

接过韩冈一并递过来的放大镜,苏颂郑重仔细的观察起瓶中的飞虫和爬虫来。这可是非同小可的话题,就跟当初韩冈指出螟蛉之子的错误一样有着极其重大的意义。

螟蛉之子的典故出自于《诗经·小雅·小宛》——螟蛉有子,蜾蠃负之。韩冈当初在《桂窗丛谈》中详细的阐述了蜾蠃为幼虫捕食螟蛉的过程,看起来不过是纠正了一个常识上的谬误,实际上,却是将过去所有对诗经的释义,硬生生的捅了一刀。

有许多人想驳斥韩冈,但越来越多的人通过实证,证明了韩冈的正确。圣人是不会错的,所以错的便是释义。从最早的毛诗郑笺,到如今各家学派,每一家都是将《小宛》中这一句解释成蜾蠃收螟蛉为义子。而韩冈便证明了这一条释义的错误。

在辩论中,只要揪住言辞上的一项错漏不放,全力攻之,往往便能让对手丢盔弃甲、溃不成军。而当一部注疏中,出现了问题——哪怕只有那么一点——就完全可以由此来推及其余,质疑其他诸多释义的可信性。

韩冈就是这么做的,而他也的确让无人敢在他面前谈论《诗经》的传注。有一点,必须要知道,作为新学的根本《三经新义》中,可就有一本注疏《诗经》的《诗义》。

眼下腐草化萤一节,出于《礼记》,见于《月令》。一旦韩冈将之证明是错误,那么接下来他去质疑《礼记》的正确性,也就是顺理成章。

在‘螟蛉有子,蜾蠃负之’的前面,尚有一句‘中原有菽,庶民采之’——中原庶民采食菽豆——那么由此意来引申,‘螟蛉有子,蜾蠃负之’的本意,就是蜾蠃捕捉螟蛉之子而已。只要将‘负’另外给个吃或者储存的释义就行了。

但《礼记·月令》中的条目,就完全没办法用另一种释义来搪塞了。要么是韩冈错,要么就是《礼记》错了。

吹熄了房中的灯火,韩冈拿出来的小瓶中的萤火虫,便在黑暗中开始闪烁出微微的萤光。瓶底的几个毛虫状的爬虫也开始闪起了萤光。

“下面的也是萤火虫?”苏颂惊讶起来,他本以为小小的爬虫是萤火虫是食物。

“这是萤火虫还没有化蛹的幼虫。不过子容兄你也看到了,就是幼虫也一样能发光。”

韩冈向苏颂解释着。顺手将瓶盖给打开。感受到了外界新鲜的空气,几点萤光立刻飞出瓶中,在房中轻盈的飞舞着,但残留在瓶中的草叶上,仍有极其微弱,却又可以辨认清楚的萤火。

“这是萤火虫的卵,同样在发光。”韩冈将瓶子举在半空中,让苏颂的视线得以与虫卵的萤光平齐,“萤产卵于草中,从卵,到若虫,蛹,再到成虫,都可以发光。其变态类似于蚕。所以蚕与萤共属于昆虫纲。”

韩冈将玻璃瓶递到苏颂手中,重新点起蜡烛,让他拿着放大镜仔细查看。

“六足、身躯由环节组成,通常有头胸腹三部分,成虫头上有触角和复眼——什么叫复眼,用显微镜一看就知道了,或者不用显微镜,直接看看蜻蜓——多数有翅,成长时多有从卵到若虫再到成虫的变态。这是昆虫纲成员的特征。蚕、萤、蚊、蝇、蜻蜓、蝴蝶、飞蛾,都附和其中绝大部分,故而皆属于昆虫纲。而蜘蛛、蜈蚣,同样有环节,但由于足多,与昆虫相异,各自别立一纲,蛛形纲、多足纲,同属与节肢动物门。”韩冈指了一下生物树上的相应枝桠,“虾、蟹其实也被在下归于节肢动物门,只是同样另属一纲——甲壳纲。”

韩冈说得很详细,苏颂拿着玻璃瓶,在手中转着,沉吟不语。

“同时有生有死,但动物、植物生死繁衍截然不同。也不可能互相转化。萤火虫从生到死,只要仔细观察,便能一清二楚,其他昆虫无不如此。也就是说,只要明了昆虫本质,就知道所谓腐草化萤根本就是完完全全、彻头彻尾的缪谈。”

“是《礼记》有错。”苏颂语调沉郁的说着。

有卵,有幼虫,还有成虫,一切都跟蚕类似,这是由事实证明的观点,比起腐草化萤说,当然更为可信。

但《礼记》毕竟是经书!对儒者来说,质疑经书,甚至更进一步说经书有错,可是要越过极大的心理障碍。也幸好《礼记》非是孔子手笔,而是西汉小戴所编纂,故而名曰《小戴礼记》。若是议论起《论语》,无论如何,苏颂都过不了心理这一关。

“在下一直都在说格物致知,而不是格书致知,那是因为书中多有错谬,要求于真,本于实。腐草化萤乃是《礼记》中的错谬之处。小戴四十九篇,其中多有伪传,由此可证。其《周礼》并称三礼,更是大错特错。”

圣人是不会错的,那么一旦文章错的,肯定就不是出自圣人的传授——虽然这条逻辑链,其大前提从本质上是错的,但在眼下的这个时代,圣人永远正确,却是人人信之不移的事实。

《小戴礼记》四十九篇中有礼制、礼仪,并解释仪礼,记录孔子和弟子等的问答。戴圣做的,仅仅是编纂。而他编纂的四十九篇中,哪些是真,那些是伪,其实是难以分辨。当东汉大儒郑玄为其做了注解之后,《礼记》的真伪便无人去怀疑了,在唐时更是被列入九经,直到韩冈出现。

韩冈盯着苏颂的手。苏颂正下意识的转动着手上的玻璃瓶,透明的瓶子咕噜咕噜的打着转,折射出来的火光,不停地晃动。以重礼守礼的儒门中人的标准,这样失态的行为,是不应该有的。苏颂的心在动摇,韩冈编纂医典,也许就是为了将所有经书中与草木土石鸟兽有关的篇章,拿出来考证一番,以验明真伪。

“以韩冈一点愚见,《礼记》之中,也就《大学》、《中庸》等数篇,得了圣人本意。”

这是要将《礼记》从九经中踢出去啊!苏颂的手一紧,死死攥住了并不算大的瓶子。他从韩冈的话中,甚至隐隐听出他有打算将《礼记》从经史子集四部之中的经部中给剔除出去。……‘原来,这才是他的目的。’

韩冈双眉轻挑,这就是自然科学在经学上的作用!

他在向苏颂解说着个人见解的时候,心中隐隐藏着一分激动。不论是儒家还是佛家、道家,甚至是西方的神学,都不可能与天地自然分割开来,避免不了的要对自然界的现象描述、总结,解释和加以说明,这是必不可少的根基。

但没有科学的研究方法为指引,对自然现象进行总结归纳时避免不了的会有诸多谬误,所以在后世的西方,科学能划破中世纪的黑暗,也就在情理之中。而眼下,韩冈一步步的将经学的画皮撕开,驳斥过往的释义,甚至是抢占解释权,当然也并非难事。

不要在自己不熟悉的领域,跟专家辩论。反过来说,想要辩论获胜,就要将话题引入自己熟悉而对手不熟悉的领域。早在韩冈开始抢夺格物致知的诠释权的时候开始,他便是这么去做,至今没有改变,也不会去改变。

当韩冈能将名列儒门九经之一的经典都进退由心,那么他在儒门的地位将不言而喻,气学在儒学中的地位也将自然而然的确立,无可动摇。

苏颂抬眼看着韩冈,温润醇和的眼眸,却闪着坚定如石、无可动摇的光芒。这样的年轻人啊,难怪他对天子的压制根本毫不在意,区区爵禄,又岂能约束得了一心放在学问上的儒者。

大概韩冈是以配飨文庙为目的吧,以功臣配飨太庙,并不是一项能吸引所有人的光荣。引导后人,传习大道,或许才是最诱人的荣耀!

