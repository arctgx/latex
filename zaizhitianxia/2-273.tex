\section{第42章 望断南山雁北飞(上)}

得到了韩冈命令,王舜臣用最快的速度,将他的本部兵马动员起来。

一个半时辰后,韩冈和王舜臣就率领着两千步卒离开了狄道城,向着北面的临洮堡匆匆赶去。

而就在一个时辰前,刘源则以奉命领着他的人出发了。一百多广锐将校,骑着一匹,又牵着一匹,从吐蕃人手上抢来了战马终于有了用武之地。

曾经被木征选中偷袭宋军后路的战士都是族中精锐,战马也是十里挑一,当日要不是他们一路奔波,来不及休息,也不会刚刚接战就一败涂地,让宋军捡了个大便宜。

奔驰在北向的官道之上,刘源犹不时的挥动手上的马鞭。出来之前,韩冈对他的吩咐是‘查清路上的伏兵’‘确认临洮堡的安危’,除此之外,没有更多的叮嘱。

一名老于战事的宿将,只需要接收命令,具体怎么做并不需要太多的嘱咐。

在结河川堡丢下了精疲力竭的战马,换上了空跑前半程的另外一匹战马,刘源领军更加小心的向前路探去。

趁着夜色,在山间行走。身边有着熟悉地理的吐蕃人引路——包约和他麾下的部族一直都在这片土地上,刘源一到结河川堡,就立刻联络上了他——刘源带着几个精明强干的手下,很顺利的抵达了临洮堡外

前面已经能看到了禹臧家的旗号,但就在同时,还有一面大旗落入了刘源的眼中。刘源认得那面旗帜代表的意义,那是西夏军中不多的几个让人觉得棘手的将领,或者是他的族人:“仁多……”

……………………

大军急急而行,到了结河川堡之后,终于停了下来。这里是往河州去的转运中枢,同时也是距离临洮堡最近的一个寨堡。在往北去,可就是危机四伏,不能再用前面行军速度来赶路。而两千人马急行军几十里路,也必须歇上一夜。

设立兵站的最大一桩好处,就是过路的队伍都能得到稳妥的食宿安排。不用韩冈操心,自有人为这两千军准备下了热腾腾的饭菜和床铺。

王舜臣去盯着他的兵,而站在韩冈的面前,则是回来的刘源,还有青唐部包约派来的亲兵。

从刘源口中听到了侦查来的情报,韩冈一声冷笑,“原是打着围城打援主意啊!”

刘源点了点头,“党项人的确是分作了两部。一部在攻打临洮堡,一部则是向南准备守在路边的险要之处。”

这是党项人的老手段了。若是为了救援被围困的友军而跑得太快,就会一头撞进陷阱里去。不过很多时候,出问题的并不是领军的将领,而是将领后面、指派他们的文官。尽管明知前面是陷阱,还是被威逼恐吓的催着上路,最后也真的走上黄泉路了。

值得庆幸的是,韩冈不是这样的官员。二十出头的年纪,却有着四五十岁老人的谨慎。前些天,王韶硬是要追击木征的时候,刘源就听说韩冈苦劝了好久也没个结果。眼下王韶音讯全无,使得各路将领自行其是,景思立败亡也是因为王韶不在的缘故,使得一片大好的河湟形势重又陷入了困境,这不得不说是王韶的决断造成的结果。

刘源一边想着,继续说着:“不过跟包巡检的人打了几仗,西贼又退回去不少,至少目前到临洮堡的十里之外还是安全的。”

韩冈看向包约派来的亲信,那个年轻人低头,“族长正在守着,所以不能前来拜见机宜,小人出来时,还再三叮嘱小人,要向机宜请罪。”

“忠心国事,何罪之有。”韩冈哈哈笑着,心道包约也越来越会做人了。

笑过之后,他问道:“围攻临洮堡的有多少人?”

“大约一万上下,但不是之前攻打临洮堡的禹臧家兵马。小人看到的旗号,不仅仅禹臧家出兵了,连仁多家也带着他的铁鹞子出来了,而且小人还在敌阵中看到了骆驼,很有可能是泼喜军。”刘源顿了顿,“虽然他们人少,可都是精锐。”

“仁多……是仁多零丁吗?”

“希望不是他,而只是他的族人。”

韩冈微皱着眉头,这个西夏老将他听说过,但事迹不甚了解,不过既然刘源都郑重其事,肯定不是个简单人物。

……………………

在结河川堡休息了一夜之后,韩冈统领的两千宋军在大道上继续前进。

宋军稳稳地推进着,让准备趁势进攻的党项人没有下手的机会。同时在山中一直维持着战线的包约所部,也让党项人感到十分得棘手。就算西夏人想埋伏,也得瞒过包约的耳目再说。

大概是放弃了远袭宋军的打算,韩冈终于抵达了临洮堡的五里开外。站在路边的山坡顶上,已经能看到城头上的旗号。

韩冈眯起眼睛,远远向北眺望着。而王舜臣也站在他身边,一起望向临洮堡去,“王存还真是有一手,竟然能守住这座破城。好像之前就被禹臧花麻弄坏了,还没来得及修好吧?”

韩冈也是由衷的点头,他原本都准备退守后方的结河川堡,甚至做好了固守北关堡的预备,但想不到王存依然稳守着临洮堡不失,这就让韩冈有了将局势重新稳定下来的信心。

“三哥,下面怎么办?”王舜臣摩拳擦掌,等着韩冈一声令下,就立刻杀往临洮堡。

“就在这里扎营!”

“……什么?!”王韶差点要蹦起来。

韩冈望着远山下的城池,踩了踩脚下的泥土,重复道:“就在这里扎营!”

“呃……啊!”王舜臣恍然大悟,“我明白了,是不是要做个幌子,趁党项人不备,在夜中进兵?”

“这是什么话,我什么时候这么说过?”韩冈瞟了王舜臣一眼,无奈的摇了摇头,“自作聪明!”

“难道是要休整一夜,明天一鼓作气?”旁边的刘源插着话,“但这未免太近了一点。”

“不,要扎下硬寨!准备好多留些日子。”

王舜臣这下急了,“临洮堡可是快要被攻破了!”

“破不了的。既然我们已经到了这里,临洮堡就肯定破不了!”韩冈口气坚定,“西贼要顾忌着我们这两千人马,他们就不敢全力攻城。”

“但临洮堡中的粮食怎么办?”刘源在旁边插话问着。

“少了景思立两千兵马,临洮堡的存粮能吃上一个月,就算断粮了,也有马骡和……能吃!……张巡守了睢阳守多久?”韩冈说得冷酷,但也是事实。人马少了一半,堡中的粮草就自然更为充沛了,而且又有牲畜,怎么都不会饿着。

“这样就能帮临洮堡解围吗?”王舜臣问道。

“当然!不需要去撞西贼的陷阱,也不需要跟西贼决战,我们只要让西贼无法专心攻城,那就足以将为临洮堡解围,只要让王存知道我们到了就行了。”

别以为存在舰队造出来是为了浪费钢铁,也别以为他韩冈顿兵不进,是为了在外面看热闹。单是‘存在’就已经足以让党项人不敢全力攻城。若是让他等到机会,也有随时刺出致命一击的准备。

“可是……”王舜臣仍然想说着些什么。

“我们已经败不起了!”韩冈终于变得声色俱厉,眼中怒意让王舜臣和刘源看得心悸。

难道他不想将围在临洮堡外的西贼大军,像羊一样赶得满山乱跑?但眼下的局面,根本容不得随性而为。

河州的兵绝不能动,兵站中护卫粮道安全的兵马同样不能轻动。韩冈现在带来的两千人马,就是眼下熙河经略司仅有的机动力量。韩冈现在就是靠着王舜臣本部的两千兵马,加上不知能不能派上用场的包约,维持着王韶留下来的局面不至于崩溃。这两千人还有三四千没什么大用的蕃军,就像挡在大堤缺口处的沙包,一旦沙包没了,洪水就会立刻冲向堤坝之后。

“你们可曾想过,要是我们败了,熙河路的局势还有挽回的余地吗?河州还能保得住吗?!”

韩冈厉声反问着,王舜臣欲言又止,看上去还是有些不服气。但韩冈的话已经说到这个地步,他也不敢再有什么异议。而刘源年纪已长,行事要稳重得多,更不会有二话。

“……就依三哥所说。”王舜臣最后勉强说着。

韩冈叹了口气,这个决定让人心服的确不容易,而且要维持住现在的局面,敌人也不单是在眼前。

他看看站在一边的包约,一直都没有说话,也不知这个家伙心中在想些什么。但想来他应该是支持自己的——拿族人跟党项人硬拼,他肯定是不会愿意。不过要让他号令周边蕃部,让西贼得不到粮食补给,那就不会有二话。自然,韩冈也不会给他这么简单的工作,谨守通往后方的大路,让西贼不能去骚扰后方的结河川堡,也是包约必须完成的任务。

“不要急。”韩冈转回来和声说着,“先等着,西贼肯定会露出破绽。那时才是出兵的机会。”

想了想,他又道:“还是要做好准备,把营地扎得牢固一点。西贼破不了临洮堡,肯定会转头进攻我们。”

