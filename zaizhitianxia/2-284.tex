\section{第46章 世情如水与天违(中)}

见到王安石,门前众官纷纷向道路两边退避过去,恭迎宰相骑马进宫。而曾布,章惇和王雱则停了下来,他们可不够资格在宫中骑马。

王安石骑马入内,而王雱三人下马,随着众官一起进宫。

今日是百官大起居的日子,天子驾临文德殿,接受群臣朝拜。

众官进宫后,通过文德门,就在文德殿外的东西阁门处列队。王安石立于最前,而只是朝官最后一级的王雱,则站在班列的末端。

王雱正静等着文德殿的大门打开,参知政事冯京就从他的眼前仰首而过,目不斜视。而枢密使吴充紧跟着在后面,这两位今天到得都算迟了。

眼角余光瞥着自家妹婿的父亲挺着脖子上的瘤子从身边过去,王雱心知,要想说服天子,就必须驳倒执掌西府的吴充,还有参政的冯京。虽然从父亲那里得不到助力,但王雱还是想到了崇政殿后,再试上一试——他并不是父亲说什么,自己就做什么的那般乖顺的儿子,总有着自己的想法。

冷笑一声。

一个是宰相,一个是枢密使,王安石和吴充这对亲家可谓是把持大宋的军政大权。不过现在吴充可是明摆着跟王安石走不到同一条道上,新法之事没有少反对过,而今次撺掇天子撤军河州,也是他所主持。

越是反对王安石,天子就越是能安心,只要行事稳定在天子容许的底线上,吴充的地位就会越来越是稳固,他接任枢密使后的一番作为,充分证明了这一点。

只是吴充事事与新法摆出势不两立的姿态,其中有几分是因为他偏着旧党,有几分是怕被人拿着他与王家的姻亲关系而逼他引避,王雱倒是很想弄个究竟。

阁门使吟唱般的赞词响了起来,高大的殿门毫无声息的被推开。在编钟玉罄的韶乐中,文武百官排着队,小碎步的走进文德殿中。

御史中丞邓绾还是照三独坐的规矩,以一张小交椅坐在殿中西南面的门后。而殿中侍御史则分列在殿中后端的两个角落中。但两位殿中侍御史其中的一位,现在去了河州。所以知谏院的唐坰代替了吕大防的位置,站到了殿堂一角。

王雱随班走进殿中,一眼瞥过去,唐坰的身影让他不禁皱了一下眉。

唐坰曾经依附过王安石,为了得到举荐,还说过要斩韩琦、富弼的首级来推行新法。虽然是个狂生,但他是曾公亮的亲戚,本身又有文名,所以才被王安石荐为御史。

不过不论是王安石,还是王雱,都不喜欢这个疯狗一般的家伙。荐为御史后,就再没有荐他更进一步的想法。唐坰小肚鸡肠,已经多次在公开场合口吐怨言。所以当他升任知谏院后,应该照规矩晋升本官官阶的,但就给王安石押了下来,以正八品的太子中允知谏院,这还是立国以来的第一遭。

王雱听说这些日子以来,唐坰已经上书二十多道,全是议论如今的时事,将新法从上到下批了个遍。不过全是无用,都被天子留中了。

但这种疯狗,也只有一张嘴皮子厉害,汪汪叫着狠而已。

王雱将心神从唐坰身上收回,他没多余的心思去想着疯狗的事,他还有正事要做。

……………………

百官大起居,是礼仪性质的朝会。并没有多少事情需要赘言。赵顼只要如常例坐在御榻上,按部就班的完成被重复了千百遍的程序。

大宋天子端坐着,身形纹丝不动,但脚尖不停的移来移去,分明在说着心中的不耐烦。

他还要考虑如何处置韩冈的问题。昨日崇政殿中的一番争执,冯京提及河湟时,并没有将横山之事拖出来当例子。要是引起天子的逆反之心,事情反而会多生枝节,只是明着说要依律治韩冈抗旨矫诏之罪。

赵顼绝不想将处置韩冈,在他看来,最多申斥一句便可了事,治罪那就不必了。怎么看韩冈都是忧心于国事,无暇谋身,说是贪功就未免太过,韩冈当初在罗兀撤军和咸阳平叛之后,可是推了多少功劳,分开来,足够好几个选人转官了。

赵顼都想好了,如果今天冯京再提起处置韩冈的事。他就用一句‘将功赎罪’给打回去。前日韩冈在罗兀、在咸阳,立下的多少功劳都没有封赏,今次就以此抵数好了。怎么都能抵得过的!

赵顼不想治罪韩冈。就如他前面所说,有功不赏,有过便是大加责罚,这让外面的臣民如何看他?他赵顼岂是如此刻薄之君。身为大宋天子,宽宏的器量绝不能少,公平赏罚才是御下之道。

大宋天子一边想着朝会完结后崇政殿中的要处理的政事,一边在御座上等着一整套无聊的流程结束。,这是上百年延续下来的规则,赵顼自登基以来,已经经历数百次,从无一点意外。但今天却破了例,赵顼从没想过,在百官大起居上,竟然出现弹劾宰相这一桩奇事。

知谏院的唐坰,拿着长长的奏章就站在离赵顼只有七八步的地方,王安石也同样站在御座前。唐坰方才一句”陛下前犹敢如此,在外可知!“,逼着王安石走到御座前,听着他的弹劾。

偌大的殿堂中别无声息,连乐班的韶乐都停了下来,只有唐坰兴奋的声音在回响:“安石专作威福,曾布等表里擅权,天下但知惮安石威权,不复知有陛下。吴充、冯京知而不敢言。王珪曲事安石,无异厮仆!”

王珪听得低下头去,似有惭色,冯京与西班中的吴充对视一眼,眼中都有着一点疑惑,他们只是‘知而不敢言’,一向秉持圣意的王珪却成了厮仆——‘这是谁的主意?’

“元绛、薛向、陈绎,安石颐指气使,无异家奴。张琥、李定为安石爪牙,台官张商英乃安石鹰犬。逆意者虽贤为不肖,附己者虽不肖为贤。”

唐坰继续高声读着手上的奏折,将新党众臣一个个拿出来叱骂。

赵顼听得按耐不住,几次命他住口。但唐坰却半步不让,丝毫不理会天子的金口玉言。侍臣卫士,人人为之大惊失色,却都不敢上前去,将唐坰拖出宫去。

以无可阻挡的气势骂完新党众官,唐坰话头一转,又直指横山和河湟。连同天子赵顼的一番作为,全被说成是好大喜功,而王安石知而不谏,是李林甫、卢杞之辈。

冯京低下头去,吴充垂眼顶着空无一字的笏板,宰执们竟无一人上前阻拦。王雱按奈心头火,狠狠的看过去,东西两班的最前面,只有王珪在望着唐坰。

‘这是唐坰一个人的反扑?’疯到这种程度,反而让人不敢相信了。但冯京、吴充岂会如此不智?王雱只觉得走进了一团迷雾,根本想不通一个究竟来。

而唐坰疯狂的行为还在继续。

一条条的念着给王安石拟定的罪状,唐坰的脸上都泛起了红晕。尤其是说到了最近的河州惨败,他的声音更是响亮把屋瓦都能震下来。

没办法,王韶、高遵裕生死不明,景思立则是明明白白的全军覆没。失踪一个经略、一个总管,死了一个都监。说句难听话,河潢的战局到了朝堂之中,已经变得跟三川口、好水川还有定川砦一样了。甚至还有有过之——

“几十年来,官军外战败阵所在多有,可何曾战殁过一个经略安抚使?!”

“王韶只是一时断了音信,并不是战殁……”

王安石被唐坰弹劾着,不敢自辩,只能低头听着。而赵顼都感觉到唐坰的口水溅到了脸上,又被骂着好大喜功,坐立不安,一时忍不住,便开口出言辩解。

终于引动天子的话头,唐坰的眼神都亮了,他正等着呢。手中的奏折一收,更响亮的声音直冲着赵顼而去:“王韶失踪已经一月有余!道路再如何艰险,也不该这么长的时间毫无音信。分明是贪功之故,以至于全军覆没。王韶、高遵裕死不足惜,却连累了数千将士,这番罪过他百死莫赎!”

赵顼阴沉着一张脸,好好的一场朝会被搅成了菜市口。朝廷大臣撒泼骂街,传到外面,他这天子的脸面如何还能留着。

“还有那韩冈,”提及此人,唐坰就怒不可遏,二十岁就成了于己平起平坐的朝官,屡立功勋,天子垂青,世人赞颂,还从亲王手上抢了一个花魁,这天理何在!“出身鄙俚,不学无术。侥幸得功,立身于朝堂之侧。不知报天子深恩,而贪功妄进,致使景思立败亡。其罪不在王韶之下,当斩其首以谢亡人!”

赵顼求援的视线扫过殿上,但众臣中竟然没有一个能站出来帮忙的。不论是被弹劾指责的,还是没有弹劾的,都是低着头去。突然看见执掌皇城司、控制着宫庭门卫的石得一就在殿门外踌躇不前。赵顼看到他,仿佛看到了救星,“石得一,何事?!”

石得一滚着进来,跪在进门后不到一丈的地方。

冯京、吴充都暗暗摇着头,‘这能拖几刻?’被天子打断了说话的唐坰更是心头怒起,拧起眉,就要将败坏国事的宦官也一起骂进去,“王中正交接韩冈,抗旨矫诏,大坏国事……”

只是石得一的高声禀报,文武百官们却听着更为清楚:“启奏陛下,宫外有捷报传至。熙河露布飞捷,王韶已复洮州,生擒木征!”

ps:修改了一下。

