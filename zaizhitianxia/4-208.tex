\section{第33章 物外自闲人自忙(四)}

吕大临正在前往洛阳城的路上,身边跟着一名三十上下、笑得温文尔雅的士子。

“快到洛阳了。”那名士子就在马上直起腰,向着远方张望着。

吕大临扫了那名士子一眼:“和叔何须心急,洛阳城也跑不了。”

“刑恕还想去拜访一下几位先生,不知今天剩下的时间还够不够了。能早一步进城,就尽量早上一步。”

吕大临哦了一声,给自己的坐骑一鞭子,有他做了榜样,在前面带领着,前进的速度顿时快了几分。

刑恕从后面赶上来,笑着就在马背上给吕大临拱了拱手。吕大临摇摇头,表示自己只是顺便而已,不算什么帮忙。

吕大临其实不怎么喜欢刑恕,尽管刑恕算是他的同学。

从平时的言行上看,刑恕似乎也是个实诚君子,而且人缘甚好,在洛阳城中,到处都有朋友。同时他还是司马光和程颢的弟子,又曾游走于吕公著的门下,还听过张载在京城时的讲学。甚至他的名字当年都传到了王安石的耳中,据说王安石曾经想用他,但刑恕理都没理,这个态度,让洛阳城中的旧党重臣们对他更加看重。

从身份上说,刑恕算是旧党新一代中的骨干,如果新党得势,势必要大用的。但吕大临几次与其说话,总觉得他有哪里不对劲的地方,似乎不是表面上这么简单。或许是个人偏见,可对吕大临来说,与刑恕同从嵩阳书院往洛阳去,区区两天的行程,的确比起礼部试发榜前还要难捱。

吕大临将自己的心思藏得很好,刑恕似乎也没看出来,依然毫无觉察的与吕大临谈笑自若,一直延续到洛阳的城门下。

“公休!那不是公休吗?”进了城之后,吕大临正想找个借口跟刑恕分开,刑恕却一脸惊喜的冲着前面的一名骑着马的青年叫了起来,还不忘指着人,回头跟吕大临介绍,“那是君实先生之子,表字公休,单名一个康字。”

司马康听到身后传来的声音,回头就看到了刑恕。他这边才停下来,刑恕已经拉着吕大临过来见司马光的儿子。

互相介绍了姓名和身份之后,司马康主动向吕大临拱手行礼,“久仰大名,钦慕已久,今日方得一见。”

司马康说他久仰吕大临的大名并不是空话。当年一人一句,将横渠四句教敷衍出来的吕大钧、苏昞、范育、韩冈,被合称为张门四弟子,随着张载入京,横渠四句教和四人的名望也同时传播开来。

吕大钧跟随张载最久,苏昞、范育都参与编写了关学的典籍,而韩冈在四人中虽是最为年轻,但他算是从关学中分支出来的格物一派的开创者,加上又是有望身登宰执,却是四人中声名最为煊赫的一位。

不过吕大临也是张门的杰出弟子之一,与他的两名兄长同归张载门下。司马康曾经听他父亲提起过,吕大临是蓝田吕家唯一没去考进士的子弟。

论才学,吕大临考中进士应当不难,他的几个兄长都是由进士得官,但吕大临却放弃了科举,而转由荫补,自谓是‘不敢掩祖宗之德。’

官宦人家的子弟,只有能力考进士,都不会选择走荫补这条路,荫补上升的通道只有一条缝,远比不上进士的通衢大道。可吕大临偏偏选了这条难走的路,甚至都没去守阙,而是跟随在张载身边问道,司马光对此很赞赏。但司马康今天过来一见,只觉得吕大临依稀就是一个就是个脾气和性格都古板的儒生。

“公休怎么你今天出来了,可是通鉴告一段落了?”刑恕笑问着。

“是韩冈。”司马康说了一句,之后又想到两人刚刚进城,应该不知这两天的变化,“和叔和与叔刚刚进城,恐怕还不知道吧……韩冈两天前已经到了洛阳,但他到洛阳的时候,河南府衙没有一个人去为其接风。”

“什么,没去接人?!”吕大临和刑恕闻言都吃了一惊。

司马康点点头,“所以今天韩冈就直接进去了州衙。”

“这么快就兴师问罪了?”刑恕啧啧感叹,“韩冈果然‘器量’过人啊!”

吕大临愣了一下之后,才反应过来,原来刑恕说韩冈是为了私怨而登河南府衙的大门。吕大临不喜欢韩冈,对韩冈用格物致知将关学带偏掉,他对此有着一份成见。但韩冈受到批评,吕大临心中却是没有欣喜;“还不知道事实如何,不当匆忙下结论。”

刑恕笑了一笑,“与叔说得有理,应当先等等看。”

……………………

韩冈此事正不急不躁的换着一身新近做好的官袍。

紫袍金带,腰悬金鱼,踩着厚底官靴,重臣的风采一点也不输人。

周南和云娘为他整理着衣角和方心曲领,素心去了小厨房,而王旖正不厌其烦的叮嘱着韩冈去见文彦博时一定要小心。

“河南府又不是龙潭虎穴,怕他做什么?!”

“官人!”

王旖很不高兴的叫了一句,韩冈随即改口:“为夫知道了,的确要小心。文潞公今天设鸿门宴,以掷杯为号,从屏风后转出五百刀斧手来。”

王旖狠狠剜了韩冈一眼,有时候她的丈夫就喜欢说些无聊的笑话。

韩冈其实并没有将文彦博太放在心上,天子都不知见过多少次,区区一个前任宰相也算不了什么。在外人看来,韩冈可谓是气势汹汹,前日刚刚受辱,第三天便找上了门来。

但文彦博并没有严阵以待,韩冈报复得越凶狠,他的未来就越是一片黑暗。

不过在韩冈来说,只是礼仪性的拜访,是转运使对西京留守的拜访。足足六七十人的队伍,鸣锣开道,从转运使衙直奔河南府衙,有不少闲人悄悄的跟在后面。

进门,入厅,接下来韩冈就见到了文彦博。

文彦博正冷笑着,韩冈迫不及待的到来,也让他变得期待,如果韩冈想要清查账簿,文彦博会让他如愿以偿,但之后他文宽夫可不会留半分口德,几份奏章都准备好了。

不过对于这样的期待,韩冈没有满足的义务。再拜起身,韩冈就在冷笑中的文彦博的邀请下,坐下来说话。

只聊了几句,文彦博就变得纳闷起来,这是朝会吗,有监察御史盯着还是怎么的?韩冈说话惜字如金,仿佛在斟字酌句。年纪轻轻,就犹如一颗河水中浸泡多年的卵石,看似圆滑,内里却是坚硬无比。说话、行事都时一板一眼。从见面行礼,到了之后的交谈,都能让文彦博感觉到这一点。

只寒暄了两句,话题就移到了正事上:“韩冈受命于天子,来京西主持开凿漕渠。只是钱粮有所不足,届时可能会需要河南府开仓相济。”

“有了天子诏命,老夫自是不会耽搁。”文彦博在推脱。

“得潞公此言,韩冈就放心了。”韩冈说着就站了起身,文彦博疑惑的看着他。

韩冈笑容冷淡,他没有与文彦博结交的意思,也没有缓和关系的打算,只是保持着对老臣的礼貌,这是为了自己,而不是为了尊重文彦博,他跟文彦博没有话说,“河南府中事务繁忙,韩冈不敢多扰,就此告辞。”

韩冈走得甚为干脆,一句话都不多说。他已经将礼数做得周全了,一切都当做应付差事,之前两边计算时间,他与文彦博见面只用了区区一刻钟而已。

韩冈告辞之后,文彦博还有些发愣,这算是什么事?上门来就是为了打个招呼?可几十年的经验很快就让他想明白了,韩冈此来就是为了打个招呼,文彦博的心情顿时就恶劣起来,咬牙切齿的发狠道:“好个韩冈!”

……………………

“潞国公的脾气还真不小,韩冈上门还没半刻就被他赶出来了!”

洛阳城中,今天不知多少人再等着文彦博和韩冈摆明车马后面对面的硬碰,富家这边也不例外。登门拜访富家的邵雍之子邵伯温,正在富弼和富绍庭的面前,眉飞色舞的议论着今天发生在州衙中的好戏,“照我说,就该让韩冈去查账,眼下即便查出了错来,也能说是韩冈在借机报复,逐人实在是浪费了难得的良机。”

“子文你说错了。韩冈并不是上门要查河南府的账,他也没打算查河南府的账。”富弼的第三子富绍隆走了进来,“漕司那边,昨天有人向韩冈提议要查河南府的账,韩冈问了一句上一次查账是什么时候,又问了一句,下一次查账应该是什么时候。然后就什么事都没有了。”

富弼听着都是一愣:“那今天韩冈上门难道是真的是为了礼数而拜访文宽夫?”

“好像正是如此。”

“那潞国公赶他作甚?”邵伯温不相信韩冈能有这么好的器量。

“是韩冈自己离开的。他到了河南府,说了几句场面话,潞公都还没来得及点汤,他就直接起身告辞。”

富弼和富绍庭这时候终于都明白了。富绍庭感叹了一声:“想不到韩冈的脾性竟然如此执拗。”

“不是他执拗,而是他行事有其礼、有其节。”富弼很是有几分欣赏韩刚今天的作为,“如今已经很难见到这样性子的后生晚辈了。”

因为韩冈所秉持的原则,他在抵达洛阳的第三天去拜访了文彦博;也是因为他秉持的原则,韩冈无意采用不合情理的手段去找文彦博的错处;但同样是因为原则的关系,他根本就无意与文彦博缓和关系,短短一刻钟的拜访,已经证明了他与文彦博的嫌隙有多深。

但不能说韩冈又错。从头到尾韩冈都没有一点失礼,从礼数上挑不出毛病来。总不能因为他在文彦博那里待得时间很短,就说他有错。

富弼一声叹:“文宽夫丢大脸了!”

