\section{第五章 冥冥冬云幸开霁(三)}

刑恕曾经见识过御史台中,怎么处置不肯交待罪行的人犯。

在脸上一张张的贴上黄表纸,然后喷上水或者醋,让人犯在濒死的窒息中失去一切反抗心。

要不然就是整个人手脚被捆上一圈圈的绳索,偏偏绳索上还被倒上了一盆水,收紧后的绳子能将手脚勒得发紫发黑,再丢在冬天的风地里,一时半刻,就能送去大半条命。

不过御史台中有一点好,对犯官是不动刑的。在提供的饮食上掺些污物,或是在牢狱外处刑人犯,让惨叫传进牢房,就算得上是逼供的手段了。

现在即便刑恕已被认定是蔡确从犯,谋反的党羽,可也并没有给他绑上绳索,更没有上任何刑具,只是将他约束在大庆殿的偏殿中。曾布、薛向则是在正殿中,苏轼更是在另一头,虽然同为犯了不赦之罪的重刑犯,还是依照官职分出了等级。

外面有十几名军士在看守,殿内则只有刑恕一人,以及蔡确的尸体。

殿宇内空旷无比,却让刑恕几乎都要喘不过气来,仿佛有巨石压在胸口上。

他胸中憋闷欲裂,仿佛每喘上一口气,胸口上的巨石就会落下一分。

同样的窒息感,使得刑恕的双眼早没了之前的灵活,口才更没有施展的余地,只是在苟延残喘。

殿中寂静无声,外面看守的声音传进来后,就放大了许多。

“……肯定是凌迟啦,斩首都是恩典。”

“两府的几位相公可都是发了誓,不诛从党。”

“兵不厌诈嘛。谋反能怎么饶?”

“这可说不定。相公们怕是都不想落一个食言自肥的名声。”

守在殿外的并不是御龙四直的成员,而是金枪班,他们并没有参与到政变中,能够用看热闹的口气谈论宰辅们是否会践行诺言。

大庆殿上喧哗,平时就是重罪,若是议论不该议论的政事,更是不会轻饶。

若是在平日,纵然贵为班直,但在进士眼中,依然是赤佬。有谁胆敢对士大夫无礼,结果都会很凄惨。莫说大声喧哗,就是低声私语,被御史看见听见后,也少不了一顿教训。

可作为蔡确党羽,刑恕现在连捂住耳朵的力气都没有了。

“这里面空着做什么?为什么里面一个人不放?想想就知道了。”

“是……”

声音突然间就低了下去。

是啊,为什么韩绛刻意下令让金枪班的禁卫在外看守,里面却不留人?而王安石和其他宰辅都默认了。

金枪班里面是有聪明人呢。

刑恕抬头看了看离地数丈的房梁,又将殿中的柱子一根根数过去。

韩绛是希望自己能够将这个机会给利用上吧。

‘不要给其他人再添麻烦了!’

在张璪离开时,向后投过来的一瞥,仿佛就是在这么说着。

大庆典上,由韩冈领头,宰辅们当面宣誓,只诛首恶,从者不问。靠了这一句,稳定了殿中班直之心,让他们尽数叛离。

明明可以做个功臣,享受一切可以享受的待遇。却因为胆怯,现在却要担心宰辅们是否会说话不算话,被秋后算帐。

刑恕已经没力气去嘲笑他们的愚蠢。

但作为从犯,正可以借着这一条免去一死。只要宰辅们不肯舍了面皮,太后也必须让上一步。

只是谋反的从犯又岂能这么简单的就逃出生天?前两年的赵世居谋反案,那几个只是说了几句好听话,甚至只是送了两本星图谶纬书籍的天文官,在地府里也会大喊冤枉。

所以刑恕现在的待遇,就是解决两难境地的办法。

外面陡然间一阵喧嚣。

好像稍远的地方,有许多人在吵嚷些什么。

刑恕一下便站了起来,紧张得听着外面的动静。

一丝侥幸从心中腾起,仿佛在海中沉浮时,在前方发现了一块木板。

宰辅们都去迎接太后和天子,这边除了一个郭逵,就没有别的重臣。

说不定,还有扭转时局的机会。

可喧哗声很快就平息了,殿外的议论则继续传进来,在梁柱间旋绕。

“韦都虞死了!”

“咬舌自尽唉。”

“前日看见他时,还真想不到会有今天的事!”

殿门外一阵唏嘘感叹。

殿门突然被推开,刑恕就看见有几个人从门缝中向他这边张望了一下,转眼就又关起来了。

韦四清死了。

自尽。

这一位御龙直的都虞候是宋用臣联络上的。在保扶太皇太后的这件事上,他出了大力。昨夜的改天换日,有他一份。

昨日刑恕在蔡确身边还见过韦四清。方才在隔壁的正殿中,他更是亲眼看见了李信用一柄飞剑,打碎了最后的机会。

当时韦四清还活着,现在就已经命归黄泉。

刚才向殿里张望的这几人,是不是很失望?

自己硬是厚着脸皮还活着。

刑恕嘴角抽了一下,却挤不出一个笑容来。

怎么就这么败了?

刑恕到现在都难以相信自己看到的一切。

蔡确会答应铤而走险,刑恕在其中起了很大的作用。而他能起那么大的作用,与他了解蔡确的心思分不开关系。

蔡确之父蔡黄裳,曾为陈州幕职,其时前相陈执中出判陈州,以其不堪任事,勒令其致仕。以至于蔡家流寓陈州,全家的生计都陷入了困境。直到蔡确中了进士,才扭转了如此窘境。

宰相随口一句,便让蔡黄裳丢官罢职,以至郁郁而终。蔡确对陈执中的憎恨,是父仇不共戴天。所以前几年的陈世儒弑母案,便是蔡确力主将陈执中的独生子给处以极刑。

而蔡确对权力的渴望,也同样发轫于旧年的经历。一想到十年之后宰相之位不保,甚至不是十年,当蔡确劝说太后失败,他的位置就已经动摇了。王安石和韩冈会将他当成出头椽子,用力的打压下去。

要让蔡确相信局势会向最坏的方向发展,刑恕根本就没费什么力气,一切的根源完全出自蔡确自身的恐惧。

但将这份恐惧发掘出来,则正是刑恕之力。

刑恕在程颢门下,一向备受看重,在洛阳诸元老那边,也极受重视。

他一向自诩日后当能步上青云之路,三十登朝堂,四十而望公辅,五十岁,就该是相公了。

可是自从那位年纪比刑恕还要小许多的半个同窗出现后,刑恕对未来的规划,就像是笑话一样。

随着功劳的积累,官位的晋升,就是西京元老之中,都没人再将韩冈当做年轻晚辈来看待。

不论是官场、学术还是人望,刑恕无一事能与他相提并论。甚至做一做比较的想法,泄露出来,都会惹来一阵嘲笑。

幸好从蔡京开始,韩冈在官场上就一路下坡,到了炭毒案中,韩冈错误的选择,让他过去积累下了的功劳都摇摇欲坠。

这一回的事变,并非刑恕引发,除了在蔡确耳边推波助澜,剩下的只是居中联络而已。

不过刑恕很早就考虑过了这个对他最为有利。

光靠蔡确,终不过是一个走狗。

路上的野狗时常能见,几乎都是丧家之犬。

刑恕从来都没想过将自己的未来绑在蔡确的官靴上。

刑恕很清楚自己的份量,蔡确之所以要用自己,也是看在了自己背后的关系。

真正的能让他功成名就的,是存亡续绝的功劳。

刑恕想要的是挽救旧党。

蔡确、曾布和薛向撑不起大局,太皇太后上台后必然要引洛阳元老入朝。

一旦太皇太后能够垂帘,压在旧党头上的这个天,给彻底给翻了过来。

早在蔡确决定放弃向太后的几天前,刑恕就已经在想象他日后回到洛阳,会在元老们中得到什么样的待遇。

但韩冈用骨朵挥出的一记猛击,不仅击碎了蔡确的天灵盖,也将他刑恕的幻想,给砸得粉碎。

殿中的光线一下就有了变化,殿门不知被谁推开了,又有人向内张望。

很快,从门缝中传来了一句话,“胆子倒是够大的。”

“还指望能活吗?”

殿门砰地一声又关上了。

外面的班直都在盼着他自尽,但刑恕不甘心。

就算以后一辈子都是罪囚,但好死总不如赖活。

不管怎么说,刑恕觉得性命比一切都要重要。

只要能活着,就有希望。

现在,他只能指望东西两府的宰执们,能够信守诺言了。

能不能逃过一命,就看宰辅们能不能让想太后承认他们的许诺。

……………………

“听凭吾处置?”

听到韩冈的话,向皇后静静的站了起来。从面前的宰辅脸上逐个看过去,最后,又落到了韩冈的脸上:

“一个是先帝之叔,另一个是先帝之母。韩卿家,那你说该如何处置?”

“有刑律,有故事。”

“嗯?”向皇后轻轻的鼻音问着。

韩冈低头:“赵颢依律当论死。立斩于宣德门外。太皇太后依春秋故事,不当问。让臣来断此案,便是这个结果。不过太后若觉不如意,听凭处分。”

“让吾来处分?……”向皇后轻笑,“吾若是当真处分了太皇太后,日后怎么见先帝?就按照韩卿家说的办吧。”

