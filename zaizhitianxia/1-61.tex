\section{第28章 夜影憧憧寒光幽(三)}

夜色正明,一轮半月挂于树梢之上。群星璀璨,北辰在北方群岭山巅上闪耀,而最为明亮的天狼星,则高悬于天顶处。自古天狼主征伐,每逢秋冬,当天狼星出现于天穹正中,便是北方边疆号角战鼓齐齐响起的时候。在天狼的注视下,千百年来,汉家儿郎与北方游牧民族之间有过多少征战杀伐。在今夜寒风中,天狼高悬,平静的小村内外都充满了杀机。

冬夜冰寒,呼出的白气转眼便凝在了唇须上。潜伏在下龙湾村村外的树林中已超过了两个时辰,锐利如刀的夜风穿过林间,带起鬼哭狼嚎一般的啸叫。陈缉虽然用皮裘丝棉将自己包裹得像个粽子,耳朵和鼻子还是冻得生疼。手脚发木变僵,都已经感觉不到上下二十根指头的存在。

黄家老大在陈缉的身后瑟瑟发抖,冻出的清水鼻涕都黏在上唇的胡须上,白花花的一片。他没有陈缉那么好的装备,穿着的羊皮袄在滴水成冰的寒夜里显得太过单薄。他抱着膀子,用力跺着脚,踩着地上的树枝噼里啪啦响着。

陈缉冻得没气力去训斥黄家老大,但一声冷哼在他身侧响起,带着不快和怒意。黄大闻声悚然而立,不敢再动弹一下,树林中重又恢复了寂静。

陈缉的身侧,是一个中等个头的干瘦汉子,四十多岁的年纪,有着一张愁眉苦脸、满是皱纹的老脸,半驮着背,显得有些老迈。但他在穿过树林的猎猎寒风中,竟纹丝不动,仿佛感觉不到半点寒意。方才他一声冷哼,便让黄大老老实实的静声肃立,这是过山风在秦凤道上横行无忌几十年的积威。

在外侧,陈缉招来的帮手,还有过山风的麾下喽罗,高高低低近三十人,都在等待着最后的命令。

“过头领。已经两更天了。”陈缉焦急的催促着中年汉子,却不敢用更强硬的口吻。

没人知道过山风的真实姓名,就连他手下的了喽罗据说也不清楚。陈缉也只知道他身前这名黑瘦干枯、长得很不起眼的汉子,身后跟着上百条冤魂。落草二十多年来,官府几次三番要清剿,都无功而返。除此之外,便一无所知。

过山风望着半里外的村庄,看不到半点灯火,夜色下,仅是一团模糊的黑影,的确没有防备的样子。“张兄弟,你仇人的家宅不会弄错吧?可别带错了路。”

“绝不会错!”陈缉给了肯定的答复,去联络李癞子的两人已经回来了一个,并把好消息带了回来。就是李癞子太胆小,死活不肯出门,不得不让他女婿黄二盯着他。

“那好,张兄弟,我们走吧!”过山风收起了小心谨慎,带着手下杀向夜色中的下龙湾。

陈缉点了点头,跟着过山风一齐起步。他不敢让自己的身份泄漏,遂化名姓张,连目标韩冈的底细也是糊弄了一番过去。凡事都讲究个‘势’字。树倒猢狲散,陈家完蛋了,没了陈家的势力做后盾,他也不过是个绘影海捕的逃囚。真的暴露了身份,过山风难道还没有黑吃黑的胆子?过山风这个绰号,得的不是没有来由。

……………………

“李癞子家的两个贼人,刚刚走了一个,就剩一个了,李二哥正在盯着他。”二更天的时候,王舜臣赶回来报信。他和李信方才受命护送着李癞子的幺子回家,韩冈不会轻易相信一个曾经的仇人,王舜臣和李信送人回家是幌子,真正的任务是确认消息的真伪。

“王兄弟,你再去李癞子家,知会二哥把那个贼人杀了。李癞子既然投了我,我便要保着他的命,别让人伤了他。”王舜臣匆匆的又走了,下龙湾村并不大,李癞子的新家离着韩家又不远,来来去去都很方便。

韩冈和王厚站在门外,虽然风很冷,但即将到来的战斗让两个年轻人热血沸腾。韩冈压低声音,在战斗开始前,他不想惊动父母:“看来贼人很快就要到了!这些贼子必须一网打尽,否则日后卷土重来,又是麻烦的事。”

王厚没有任何上阵的经验,他看着指挥若定的韩冈,有着一丝羡慕,“玉昆……可有良策?”

“良策算不上,不过是引进来关门打狗。”

秦州的村子都是有边墙的,下龙湾也不例外。虽然不算牢固,也不高峻,仅有六尺出头,身手好一点的轻轻松松就能翻过去。可村中有许多房舍是以边墙为家中茅房或院落的墙壁。这就决定了贼人想要逃出村,就只有几条大路可选,不然就必须先冲入人家,才能逃出去。

‘一旦他们这么做,就会陷入人民群众的汪洋大海之中。’

王舜臣、赵隆和李信三人,万人敌也许还称不上,但都是以一当百的高手。不过实际战斗和演武不同,敌人水平也不差,夜中厮杀,说不准就会出些意外。韩冈哪能舍得,当然得为他们多拉些帮手,“这里是关西,关西男儿岂会甘受贼寇摆布?只要有人挺身而出,便能号召起全村老少群起而攻!”即便不能指望村民动手,也可以利用他们分散贼人的注意力。

……………………

陈缉和过山风一伙没有任何阻碍的潜入了村中,都是做惯了盗贼,穿过被打开的村寨围墙大门,连看门狗都没有惊动。顺着打听明白的道路,摸向韩家的宅院。一切顺利的超乎想象,正当陈缉以为胜利在即,马上就能手刃仇雠的时候,一声大吼,划破了冬夜的宁静,也打碎了他的幻想。

“有贼入村!各家谨守门户!”

随着韩冈一声吼,村中的几十条看门狗各自狂吠起来,一盏盏灯亮了,人声动荡,从村中的各家各户传出。

陈缉脸色剧变,难道是哪里走漏了风声。经历丰富的过山风仍保持着镇定,在他二十多年劫掠生涯中,失了风的经历从来不少:“快!冲过去,砍了人就走!”

一人这时从路口岔道上转了出来,矮小却宽厚的身影堵在前方。月光没能照出他的面容,神情都隐藏在黑暗中,只能看见一支搭在长弓上的箭头,闪烁着月色清辉。

“此路不通。”略显低沉的声音,有着沉甸甸的压迫感。

过山风哈哈大笑,恶声道:“就凭你一张弓,也敢堵着爷爷的路?!”

跟着过山风的都是落草几年乃至十几年的悍匪,劫掠地方都已记不清多少回多少次,杀起人来如杀鸡屠狗一般毫不在意。陇城县的几任知县都在他们身上吃过苦头,还重伤过一个县尉,死伤了不少衙役土兵,何况区区一人?!

只有十多步的距离,箭术再好,又能射到几个?村里道路众多,在狭窄的村道上,弓箭根本施展不开。所以过山风今夜率人入村,都是人手两把长短兵,根本没带着累赘碍事的长弓箭囊。

“杀了他!”过山风一声令下,一群喽罗应声上前。都是习惯厮杀的老手,前冲时身形放低,左手护住面门,持刀的右手挡在心口,就算手臂上中个一两箭,也死了不了人。

嗡的一声响,弓弦动了,但这弦声却长得过分,余音不绝于耳。陈缉听在耳中,觉着有些恍惚,这是一箭?很快他便知道了——不是一箭,是七箭!

急速颤动的弓弦仿佛变成的虚幻,连绵不绝的嗡嗡弦鸣中,一支支长箭激射而出。十几步的距离不过冲到一半,最前面的七个喽罗便全数栽倒,各自捂着小腹在地上惨叫翻滚。射不到头,射不到胸,能射的要害就只剩下小腹了。王舜臣减少了连珠箭的数目,却让准头翻倍的提高,七箭无一落空,让跟在后面的贼寇不敢再上前。

“你是何人?”过山风又惊又怒。这等高手秦凤路中也没几人,怎么会平地里冒出来?

“王舜臣!”一声尖叫从过山风身后传来。王舜臣的连珠箭术早有盛名,陈缉不认识王舜臣的人,却听说过他的箭。韩冈身边的神箭手还会有谁?只有王舜臣!

“是陈缉吧?……”王舜臣悠悠然问着,双手一动,又是一支长箭出现在弓臂上。一轮速射,王舜臣的手臂也有些酸麻,暂时还射不出第二轮,但方才他造成的杀伤,让眼前的敌人不敢轻举妄动。

“中!!!”

狂野的吼叫卷起一阵烈风,两具石锁从王舜臣两侧呼啸而过,飞向拥在一起的贼人。两名悍匪躲避不及,被正正撞在了胸口。惊心动魄的骨骼碎裂声中,两团血雾喷薄而出,两个人一起嗖的倒飞出去。肋骨成了碎片,胸口完全瘪了下去,还在空中的时候,心肺都被震碎的他们就已经成了尸体。连着撞倒了身后的几名同伴,砰砰两声落在地上,不再动弹。

赵隆高壮如熊的身影自黑暗中浮现,出现在王舜臣的身边。甩出两具石锁后,拿在他手上的是两支亮晶晶的六棱熟铜简。酒盏粗细,比普通的铁简重上一倍还多,被紧紧地攥在手中。赵隆轻轻转了转手腕,便是一阵凶恶的破风声。

眼前只有两人,而手下还有近二十个,该怎么办?

陈缉一瞬间作出了决定——逃!

他转身便逃!

ps:几位高手严阵以待,韩三坐镇指挥,谈笑间强虏灰飞烟灭(字没错哦)。

今天第一更,求红票,收藏

