\section{第八章 太平调声传烽烟(三)}

【第三更,求红票,收藏】

天色渐渐暗了下来,晚霞如一副巨大的红色绸绢,在天地间批洒开。映得露骨山头上的无数白石一片亮红,仿佛炉膛中燃烧着的石炭。

董裕骑着马,顿足在露骨山南侧的山道上,远远眺望着南方。

天气炎炎,即便是太阳落山后,山风仍带了一丝暑气——如果是汉人,也许会觉得很舒服,但董裕身为高原上的吐蕃子民,却是分外耐不了热。

他身上的皮裘脱了一半,露出了半边坚实如铁的胸膛。腰间的五彩系带松松的系着,半幅披肩搭在肩头,用的是最上等的绢绸,在落日的余辉中闪闪发亮。

在董裕的右臂上,系着个三寸大小的圆盘形饰物。上面缀着一颗颗圆润如珠、名为瑟瑟的碧色宝石。这是吐蕃赞普一系才能佩戴的标志,代表着臂饰主人拥有继承自松赞干布的血脉。如果是普通的部族族长,臂饰就只是单纯的金银之物。

而董裕能配上这件臂饰,便是因为他是前任赞普唃厮罗的亲孙,现任赞普董毡的侄儿。同时也是河州蕃部的第二号人物,仅次于他的兄长木征。

他立马于高高的山道上。隔着一重矮丘,在南方极远处的一点淡淡星火,是来自于宋国最西处的寨堡——渭源堡——的光芒。不过渭源堡并没有驻扎多少宋军,历年来,吐蕃勇士若要东去,根本都不用理会渭源堡中的守兵。

董裕本也没把渭源堡放在心上,一直以来他总是很自大的带着他的兵从渭源堡前通过,去找他的岳父说话。这样的自大,直到他今次被王韶带着七家背叛了吐蕃的部落,从身后狠狠地捅了一刀后,才烟消云散。

董裕摸了摸右脸脸颊上刚刚长出来的粉红色的新肉,嘴角抽动了一下,绽出一个狰狞无比的笑容。眼底阴寒森森如电,那是饿虎在夜色下,盯着猎物时闪烁的幽幽寒光。

尽管已经过去了几个月,但中箭的那一刻,董裕仍牢牢地记在心间。他从没见过那样迅疾的箭术,也就是一个呼吸那么短暂的时间,堵在逃路之上,迎面而来的那名宋人,竟然一口气射了十多箭。当时董裕竭力的避开了其中的一半,又靠着他身穿的硬甲挡住了剩下的一半,但最后还是漏了一箭,扎在了他的脸上,箭头甚至杠到了牙齿,硬砸了他两颗大牙下来。

“王舜臣……”

念着这个名字,董裕又觉得他的伤疤开始发痒了。在那一战之后,他设法打听到了那名宋军将领的名字。就跟留在他脸上的这道永远也不可能消褪掉的伤疤一样,董裕心中的恨意在他斩下王舜臣的首级前也绝不可能会消失。

“董裕,还在想托硕部的事?”一个有些苍老的声音在身后响起,董裕连忙回头。

一个光溜溜的脑袋先映入他的眼中,继而才是穿了一身肮脏的僧袍,皮肤黝黑,满脸皱纹的老和尚。

董裕赶紧下马,冲着老和尚行礼:“师尊,你来了。”

“嗯。还算赶得及。”老和尚应声说着。

能让河州一带仅次于木征的大首领董裕恭敬有加的,在西北的蕃落中已经没有几人。但眼前的这名蕃僧结吴叱腊却绝对是其中之一。结吴叱腊是河湟一带有名的吐蕃族僧侣,不过他的有名是来自于他手上的兵力,这名老和尚,吃斋念佛的时候少,杀人放火的时候多,根本不知道什么是慈悲。而他今次与董裕会面,也不是为了弘扬佛法。

刚寒暄了两句闲话,董毡便急着问道,“师尊,不知你找的那几家来了没有?”

“你放心,他们很快都会到的。”结吴叱腊安抚着董裕焦躁的心情,“等他们来了,便可以好好商议着下面要做的事了。”

“我只是想再会一会在我脸上射了这一箭的宋人。”董裕平静的声调中透着浓浓的恨意。一时忍不住又去摸着伤口。距离那一战,已经过去了好几个月,但这块伤疤却仍时不时在发痒:“想不到汉人中又出了不输刘昌祚一般的好汉,等今次事成,我要他的头割下来当酒碗。”

“董裕!”这时山道上传来一声吼,毫不客气的叫着董裕的名字。

董裕和结吴叱腊同时望了下去。一个高大健硕的吐蕃汉子沿着山道骑马奔上来。可能是嫌热,他把帽子脱了,也是秃秃的一颗光头,穿着僧袍,而与结吴叱腊不同的,是他留着一捧大胡子,乱糟糟的在山风中飞舞。

“是康遵啊,你终于来了。”董裕遥遥高声喊回去。

“结吴上师有命,哪敢耽搁。”

被唤作康遵的蕃僧,骑着马直冲董裕和结吴叱腊的近前。马蹄飞舞,溅起了无数尘土碎石,董裕和结吴叱腊脸色不变,就看着高大的河西战马带着沉重的蹄声正面冲来。

当康遵一人一骑离着董裕、结吴只剩五六步的时候,见着惊不动他们,方才用力一扯缰绳。胯下坐骑被勒得人立而起,跳着向前蹦了几步,紧紧擦着董裕的肩膀冲了过去。

“董裕,看来你的胆子还在嘛!”康遵跳下马,哈哈笑着。

“康遵星罗结!要不要比试一下,看看我的刀在不在?”董裕冷冷的说道,带着伤疤的右脸扭曲的抽动了一下,眼中又泛起了杀机。

康遵星罗结,星罗结部的族长。也没见人给他剃度受戒过,但他总是做着僧侣打扮。他完全不理会董裕的愤怒,毫不客气的说着。

“董裕,你今次带了这么多兵过来,难道是想报你前日在青渭结下的仇?”康遵星罗结并不惧怕董裕和他身后的木征,他手上的实力足够他自保,说起话的口气都跟董裕平起平坐,“你要做赞普,我可以帮个手。但要是说去打古渭寨,为托硕部报仇雪恨,抱歉,我不奉陪。我星罗结部人丁一向不旺,经不起这等折腾。”

康遵星罗结的一番话,让董裕红褐色的一张脸,一下变得血红。只是转眼间,他却是笑意堆上脸,“我打古渭寨做什么。嫌家里孩儿死得还不够多吗?”

董裕咧嘴笑着,脸上的那条狰狞的伤疤,也没影响到他的笑容,“我今次要对付的是跟着王韶一起攻打托硕部的那七个部落。宋人就罢了,既然是吐蕃人,还敢在我背后捅刀子,那是绝饶不了他们。我家哥哥今次让我带了六百人过来,都是家里最勇武的孩儿,按汉人的说法,是个顶个的好汉。正是要报那一箭之仇”

“你从木征那里借了兵来?”康遵星罗结捻着胡子,歪嘴笑着,“你下了不少血本啊。”

木征董裕两人虽然是亲兄弟,但早早的就已经分了家。各自过各自的,连部众都分了。上次在托硕部中损失的其实都是董裕的部众,而木征根本就是在河州看热闹。而董裕今次从木征手上借来了六百族中精锐,就跟康遵星罗结说得一样,可是下了不少血本。

“今次出战,我本身就领着三千兵,又有我家哥哥借的六百精锐,另外还有四五百人,都是托硕部逃出来的人,人人悍不畏死,想着要报仇。”

康遵算了一下,“那就有四千兵了。”

“我的四千兵,还有康遵你的两千儿郎,另外师尊还找了其他几家部族,加起来也有两千兵。”

“八千人?”

“对!”董裕用力点了点头,“总计八千大军,对外可以号称两万人马。我等明日我就让人把话传出去,我董裕今次就是要报托硕部之仇。若是纳芝临占、党令征他们七部能早早的来到我马前跪下请罪,我还能饶了他们,若是胆敢拒我大兵,不肯降服,我必灭他们全族!”

听了董裕的话,康遵星罗结嘿嘿冷笑:“我知道你是看着刘昌祚带着他的两千兵去了甘谷城,急切间赶不回来,才敢如此放言。但古渭周围可是青唐部的地盘,你要在这里把事闹大,你看俞龙珂会不会答应?”

董裕摇头:“今次青唐部绝不会插手,俞龙珂也不会乐意看到汉人在青渭耀武扬威。”

康遵星罗结哈哈大笑,笑声一落,脸色又冷了下来:“俞龙珂是条狡猾的狐狸,没人能揣测得清他的想法。如果董裕你只是凭着猜度说他会站在一边看热闹,我是不会出兵助你的。”

董裕犹豫了一下,看了一直沉默着的结吴叱腊一眼,决定还是透露一点消息,“瞎药会带兵来!俞龙珂这些年胆子越来越小,像只山鸡一样受不得惊吓。青唐部中,声援他弟弟瞎药的声音越来越大。有瞎药牵制,俞龙珂抽不出身来。”

康遵星罗结闻言又是放声大笑,“这事何不早说,作甚遮着掩着。青唐部既然出不了手,那还有什么好担心的。”

星罗结部的假和尚换上了一幅市侩的笑脸,“我从家里带兵过来,也是冒着风险,若是不能拿些好东西回去,家里都要挨饿。董裕你说说,你打算分我多少?”

结吴叱腊代董裕回答:“墀松德赞在时,长安城任我吐蕃大军进出。唐帝没钱酬谢我们帮他平息叛乱,还把长安城当作了酬劳。今次只要康遵你肯用心,可以任你挑两个部族做报酬。”

