\section{第31章 九重自是进退地(一)}

【祝各位书友新年快乐。】

“辽国的废太子死了?!”大宋皇帝紧紧捏着御榻一边的扶手,双手直颤着。

石得一恭声道:“听说是在临潢府拘押之地暴病而亡。”

“好!好!好!”

赵顼再也无法在御榻上维持着天子的形象,听到这个消息,他怎么可能还能安坐如素。

赵顼想要大叫一声,以发泄心中的兴奋之情!

这当真是如有天助啊!

辽主如今只有一个皇孙,而这个皇孙的杀父仇人又是当今的权臣,且耶律洪基年纪也过了四十五。

辽国诸帝,也就辽圣宗过了花甲之年,六十一岁驾崩。辽太祖耶律阿保机活到五十多,其余几位皇帝,有三十多的,也有四十多的,反正没有一个是长寿的,由此来推断,这耶律洪基当也没有几年可活了。

到时候,主少国疑,又是权臣当道,而且君臣之间还是不共戴天之仇,辽国内乱可想而知,那是指日可待。

赵顼兴奋之下,一时都忘了自己这边的情况其实也差不多。太庙中的几位,可是没有一个活到六十的,而且以他的父亲最为短命。太祖、太宗、真宗、仁宗,都是过了五十,唯有英宗,只活到了三十六岁。

不过赵顼这时候不会去想败兴的事,他的思绪千回百转,已经从辽国几年后可能会发生的内乱,转到了西夏国中,两三年年内必然会发生的内乱上。

虽是母子之亲,但权力却是分毫不能让人。西夏国母梁氏与其兄梁乙埋把持朝政多年,西夏国势日蹙,国中多有怨声。据派去西夏的密探回报,西夏国中各地,多有人盼望梁氏能早日撤帘,然后让有着契丹人在背后支持的秉常亲政,以挽回如今山河沦丧的局面。

只是梁氏得到了仁多家等几个异姓大族的支持,才压制住了王族嵬名家。但这样的局面是不稳定的,秉常一岁大过一岁,梁氏压不下他几年了——多少人都有着同样的预测,兴庆府中的变乱,最多也只有两三年了。

契丹内乱,西夏内乱,而他赵顼只要保证着大宋国中的稳定,一旦时机到来,便能点集百万兵马,一举平灭西夏,继而收复燕云失土,甚至可以一路打到临潢府,乃至狼居胥山。

章惇能如马援一般在交趾标铜立柱,难道他的泱泱大宋,就没有一个能如霍去病的名将?!他早就清点过自己口袋里的诸多将帅,其中的任何一个,只要有着运气和时间,加上无穷无尽的国力支持,到最后,都能完成霍去病的功业。

想要实现自幼年便有的梦想,也只要再等上几年了!

……………………

为了春捺钵,也是为了迎接头鱼宴,在浩浩荡荡的十数万大军护卫下的大辽朝堂,已经离开了冬捺钵的所在地广平淀,开始向北方混同江【松花江】畔的鸭子河泺前进。

大军每天都要前进几十里,离着上京临潢府越来越近,过去之后再向东北走,就是混同江。原本在临潢府中,还有一个与皇帝关系紧密的囚犯,不仅年轻,而且身份尊贵。但这名囚犯,他暴病身亡的消息,已经于一个月前传到了广平淀,临潢府之外的流放地中,已经看不到大辽前任太子的身影。

就在一个月前,还有许多年轻人甚至天真的以为只要对耶律乙辛认输服软,他的攻势就可以到此为止。但废太子的暴卒打碎了诸多幻想,也给辽国的朝堂带来了一股难以遏制的暗流。

多少王公贵戚听到此事之后,背后都有一道凉意划过,继而一阵怒火便熊熊燃起。

耶律乙辛实在是太过肆无忌惮,当今天子的独生子已经被废去太子身份,又以拘押上京了,到最后竟然连性命都保不住。有凶焰正炽的耶律乙辛,那谁还有能耐保住自己的小命?

但他们的这点怒火,却如同草原上的兔子,只冒出来个头,就在窜遍全身的危机感中给缩回了洞去。

死了亲生儿子的都不说话,他们越俎代庖又是何必?!

对于唯一的儿子突然暴毙,耶律洪基没有太多的疑问。犯了重罪,心惊胆颤之下,很容易毁了身子骨,继而生病暴卒。

但作为一名父亲,耶律洪基却也免不了要伤心。再怎么说都是儿子,而且还是好不容易才得来的独子,从小养到大,最是疼爱不过,虽然由于种种原因,让他废去了太子之位。但他身下的这个位置迟早是他儿子的,这一点毋庸置疑。

当日,又是跟废掉耶律浚太子之位时的情况一样,耶律洪基整整七天无心游猎,一直待在帐中。

从耶律乙辛的脸上,根本看不出来有任何异样,尽管他就是废太子耶律浚暴卒的元凶,但当他听到传回来的捷报之后,没有丝毫欣喜,也不见如释重负的神态,什么反应都没有,就像是死了一个陌生人一般。

自从他亲手设计,将皇后萧观音陷于死地之后,他就已经不能回头了。当自己与耶律浚两人之间,只有一人能活的情况下,耶律乙辛绝不会选择牺牲自己。

他不担心耶律洪基会对耶律浚的死而迁怒自己,权位越高,对亲情的看重就越少。

大辽国中,拥有相同血脉的人们互相厮杀的情况太多了。当年的承天太后,可也是对她的亲姐姐也照样狠得下手。父子叔侄兄弟姊妹,最为亲近的血缘关系在遇上了权力之后,连坨马粪都算不上。马粪干了之后还能烧,这父子之亲,也不过是让人多留点泪,心情差个几天罢了。

除非日后自己被证明在此事上有欺君之罪,否则就不用有多余的担心。

“太师。”突然赶到耶律乙辛帐中的萧得里底,脸上有着几分抹不开的紧张,“皇帝想要召太子妃来此询问!”

“我已经派人去了。”耶律乙辛早就得到了消息,他这位权臣在捺钵中布置下来的耳目,怎么都不可能输给萧得里底,“她见不到天子。”他的声音和表情同样冰冷。

萧得里底先是愣了一下,继而便放心了下来,但很快又有了一份隐忧,“先是太子,接着又是太子妃,会不会惹起疑心?”

“难道让她见了皇帝,就不会惹起疑心了?”耶律乙辛反问着,见萧得里底愣住,他冷道:“两害相权取其轻!”

在路上死了的危害小,当真让人到了天子面前,进而引发的反应,可是毁灭性的灾难。

看见萧得里底苦思冥想,耶律乙辛道,“最好还是多想想该怎么对付宋人。听说南朝六十万禁军,已经是全数配装铁甲了。”

关于这个问题,的确已经在辽国高层传播开了。尽管有人嗤之以鼻,表示自己绝对不信,但也有许多则是在看到天上飞船之后,才又全盘接受了这种说法。只是这样一来,对宋人的畏惧之情,也随之弥散开来。

萧得里底就是其中一人,只是他的位置特殊,所以被重点照顾,“如果列阵而战,再强的骑兵,也别想攻破身着铁甲,手持神臂弓的南朝禁军步卒。”

“要比谁家马多、骑兵多,这还当真不是什么难事,也不会输。可是要比谁家铁多,甲胄多,那就真的是不能跟宋人相比。”

“宋人幸好只是工匠手艺出色,要是连上阵厮杀的武勇都一般出色,那大辽可就危险了。”

“哪里可能?”也许个别人能两者皆备,但放在一军之上,能有这等素质,基本上就是凤毛麟角,不可能成军的。想要成为大辽一国威胁,好歹也要有个两三万再说。

耶律乙辛在心中自我安慰着,至少他不担心南面突然间冒出一支手持锻锤的军队来,一边拿着锤头砸人,一边为其他队伍修补兵器,打造各色军器。

不过不管怎么说,配装铁甲消耗的财力物力和人力,只有大宋一国能够做到。辽国要拼了老命,才能勉强做出四五千套来;西夏就是想拼了性命,都不够那个资格。

大宋的根基一天稳过一天,要想对抗拥有数千万户口的大国,也只有同样等级的大国,区区西夏,根本不在话下。而眼下的天下形势,大宋的国力已经远远压倒辽国,辽国如果不能联合西夏,同样也是无能为力。

只是西夏国内如今的形势有些不妙,耶律乙辛很清楚的知道了这一点,如果没有外力干涉,梁氏当能一举控制住嵬名家和秉常。不过现在就变成了两强对峙的局面,想要分出个胜负,却是难如登天,看来还得自己插手。

护送着辽国君臣,正在前往鸭子河泺的浩浩大军,于除夕前夜抵达了上京临潢府,并就此停了下来,正旦之会,依制当在上京城中的宫殿中举行。等到正旦朝会结束,辽国君臣才会重新踏上行程。

由于宋辽两国历法不同的缘故,万里之外的东京城,戊午之年的元旦要比辽国提前了整整一天。

就在辽国军民欢度除夕的时候,皇宋元丰元年的正旦大朝会也终于开始了。

