\section{第36章 万众袭远似火焚(一)}

和煦的春风,吹绿了江南,吹绿了京东,吹绿了河北,也吹绿了西北边陲的大地。

阳光还是像冬天一样黯淡,经过了连续半个月的晴天,积雪也才刚刚化到一半。融融嫩绿从半遮半掩的雪层下冒出头来,雪水淙淙,渭水两侧的河滩田地上仿佛变成了癞痢头,白一块,绿一块。斑驳的田地看起来很是难以入眼,可如果深悉农事的人来看,那他的视线就能穿越时间,看到了未来的丰收。

一支浩浩荡荡的大军行进在渭水边田亩中的大道上,人马足足有万人之多。足足有三丈宽的官道,在数万只脚和蹄踏上后,立刻显得拥挤不堪。幸好事先有分了前中后三军,前后阵的距离超过了两里。长长的长蛇阵,虽说等于是对敌军的邀请,但在行军时便能稍微放松起来,让将校官兵们走起路,也能变得轻快许多。

前军转过了前面的弯道,队伍被山峦所阻挡,已经看不见了。身处中军之内,景思立望着同样隐入天际的广袤田野,沉吟着。

一场战略性的决战,是任何一名有着进取之心的将领都梦寐以求的战争。比起在边地紧锁防线,候着不知何时会攻过来的党项人。还不如主动出击,先在党项人的肋部插上一刀。

景家在关西多年,与西夏的仇怨早结得深了,景思立也想早一点看到党项人的末日。

他的父亲景泰是旧年的关西名将,而且是考中了进士后,投笔从戎的名将。因为景泰久历边陲,在关西军中人脉极深,而且他还是卒于秦州任上,在担任秦州知州、秦凤兵马都总管时病死。这让朝廷都要,给了景思立几兄弟均增以荫封。而景思立的兄长景思忠,则是殉国于西南夷的战斗中。因而景思立再一次得到荫补。

一门忠烈,让景思立年纪轻轻就担任起边地的知寨。靠着父兄的荫蔽起身,与郭逵有几分相像。而后景思立更是得了韩绛的赏识,又擢了权摄保安军事。他在大顺城立下了不小的功劳,眼下就坐上了知德顺军、兼秦凤都监的位置——德顺军属于秦凤路,在秦州的东北面。今次来自秦凤路的援军,便是以他为首。

景思立能够成为知军,也算是军政皆通。看到巩州的一片片麦田长势喜人,心中是暗暗称赞。只看田地中麦苗的长势,就知道熙河经略司在巩州没有少下功夫。

而且巩州还有棉田。景思立久在缘边守卫,与吐蕃、党项回易的生意,都少不了他家的商队一份,对于商界中的消息,景思立也不会如同隔山一般毫无所闻。秦州的诸多商行和他们背后的家族,如今据说都有心去巩州开荒种棉。棉布的利润人人心动,比起天下都有出产的丝绢来,木棉布、吉贝布,这等名字不同但本质同一的稀缺织物,至少能保证家族十几二十年的稳定收入。

景思立深悉王韶秉持朝廷的心思,要把河湟之地稳稳的拿到手中,而不是变成又一个由蛮夷统治,只是名义上从属大宋的羁縻州。王韶在巩州的一番辛苦,甚至连叛军都接收了下来,都是为了能将河湟之地重新抓在朝廷手中。

景思立来之前就已经隐隐听说了传言。王韶前日去秦州,跟蔡延庆商讨今次决战的细节的时候,曾说再过三年,巩州不但粮食和衣料能满足自身守军的大半需求,而且一旦岷州的铁矿和钱监开辟,连军饷也能解决一半以上的问题。

本来秦凤军中的议论,都是以为王韶这是夸大之词,至少故意耍了一个心眼——三年后,河湟多半就能平定下来,那是熙河各军州并不需要驻屯太多官军——可现在看这眼前的这片田地,景思立已经信了八分。

“巩州今年的收获当是比去年要好……王存,你说呢?”景思立回头问着身侧的一名将佐。

王存是景思立的部将,听到询问,便道:“那是肯定得。听说巩州的官田,都是韩玉昆之父主持开垦种植。因为他田种的好,天子都特别赠了官职。这务农都务出官来了。”

“做工的难道就没有官身吗?献了神臂弓的李定,他现在也是个不大不小的官了。更别提那些入粟买爵的商人了。士农工商,真想做官,都是做得的。”

景思立和王存正在说话,前军派人赶来回报,“启禀都监,前面熙河路的韩机宜来迎接了。”

“韩冈来了?”景思立心头一惊,离陇西城还有十几里呢。他不敢多耽搁,吩咐了王存镇守中军,连忙打马上前。

景思立第一次在近处见到韩冈。对于这位在马背上腰挺背直的年轻人,景思立绝不会因为年龄而轻忽视之。

一从看到了疗养院的效果之后,景思立就觉得他的确是个人才。何况如韩冈这个名字早已是如雷贯耳,在关西大得惊人。不但在关西诸路的军中人望甚高,据说天子和宰相都是对他每多圜护,看得很重。

见到韩冈离城十几里来相迎,景思立的自尊心得到了最大的满足。但他也不敢妄自尊大,韩冈现在的身份并非他能够傲视。

看见韩冈一行,景思立远远的就提声打着招呼:“可是韩机宜?”

“在下韩冈,见过景都监!”韩冈也是隔着老远就回着话。到了近前,他更是对景思立下马行礼

“不敢,思立久闻韩机宜之名,今日一见,方知盛名之下故无虚士。”景思立不愧是进士家的子嗣,说起套话来,也是文绉绉的。

看着文气甚重的景思立,韩冈就想起了王厚。他们两人都是深悉兵法的进士的儿子,都是已经或准备在军事上有所收获的武臣。也许景思立的现在,就是王厚的未来。

只是王韶至今也没有转为武将,依然还是文职的身份,甚至还有一个侍制头衔,在这一点,他就不如景泰做得干脆。

韩冈与景思立寒暄了一阵,便上马与他并辔而行。

景思立是第一支抵达熙河的外路援军。今次从关西各地,来到熙河路的实际战力,总计将达到了破纪录的三万人。

当初攻打罗兀时,种谔带去的兵马也才两万。从这兵力的数量上看,。可要安排下三万人的饮食,同时还要照料胃口比起三万大军还要多上许多的万匹战马,韩冈这些天在累得一身疲惫后,有时都会觉得王韶好不容易才为他争来的随军转运使。还不如在巩州做个安安心心的通判。

景思立自从军后,积极的领军上阵,多有功勋,又能主持着缘边重要军州的军政大事。他能压倒毛遂自荐的刘昌祚,得以统领秦凤援军,并不是仅仅靠着张守约对他的赏识,以及传言中沈起对刘昌祚的不满——那位名震汉蕃的神箭实在是跟文臣合不来,韩冈对此都有所耳闻——而是他真的有这份本事。

景思立和韩冈说着闲话,话题不知不觉的就转到了眼前这望不到尽头的田野之上,“看到了这一片田垄,才知道巩州不是得来无由。”

“还是人手少,要是能再添些人丁就好了。只是现在的情况,又有几人愿意来熙州、巩州屯田?”韩冈叹着气。“左近都是吐蕃人,就是为着后代考虑,只要有钱还是在秦州买房置地。到熙河路来,就是纯粹的枕戈待旦。在种田的同时,还要随时准备战斗,没有哪个普通百姓能有如此胆识。现如今巩州的安稳还是靠着广锐军那群叛逆。”

“广锐军也算是难得的精锐了。不然区区三千人,也闹不出这么大的声势。”景思立与吴逵也见过面,对其人的武艺、将略也十分欣赏,谁知竟然会变成现在的局面。

韩冈没有景思立的感慨,似是无意的说道:“听说今年天下厢军就要全数撤并,所有的旧时军额都将改换。”

景思立看了韩冈一眼,熙河路的机宜文字这话说得好像太直白了一点:“厢军也有要上阵的,而且里里外外的事务也少不了他们。”

韩冈呵呵笑道:“校阅厢军的主意不敢打的,那些不校阅厢军除了在官营的酒楼里跑堂、还有在官宦家中跑腿之外,其实也能派上些别的用场。”

“比如屯田?”景思立试探的问着。

“正如屯田。”韩冈举起马鞭,遥遥指着一周山峦河川,“巩州如今已经名副其实的被巩固,经略司的一句话,无论汉人蕃人都得站起来听着。不过鸟鼠山的对面,可就不是跟巩州一样的情况。大部还是在木征手中的洮西姑且不说。熙州北面的蕃部几乎被扫平了,不论是饿死,还是战死,其实那里已经都没有多少能再站起来的蕃人。而熙州南部靠着岷州,去年经略司就已经把岷州定下,现在有傅勍和王惟新坐镇岷州、熙南,那里的蕃部还都算老实。”

景思立听了心头一阵疑惑,情况听起来不是很不错嘛。

就听韩冈继续道:“只是蕃人不可信,没有汉人为根本,别看现在恭顺,一旦朝廷松上一口气,他们转眼就能反脸过来。”

“所以要屯田。”

景思立明白了韩冈为什么把兴趣突然转到了厢军身上,但对还是第一次正式交谈的自己说起此事,未免太冒失了一点……他心中猛然一惊,瞪大眼睛看着韩冈挂在脸上的那个很自然、却又仿佛透彻一切的微笑,

‘难道自己有心留在熙河的心思被看破了?!’

