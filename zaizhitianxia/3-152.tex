\section{第40章 帝乡尘云迷(五)}

寒冬终于到了。

连着两场寒流,京畿河北普降瑞雪,整整下了两天一夜,白马县的街道上积雪多达三尺之厚。

路上不见了寻常的车马,不过却跑起了雪橇车。站在路边,能看到车子一辆接着一辆,长长的木条压着积雪一滑而过。才一年的时间,韩冈当初的发明,竟然已经在京畿普及开来。虽然拉着雪橇的牛马走得也吃力,但有车子能载货,比起往年冬天,一到雪后,商业交通便完全中断的情况要好上许多。

府界提点司衙门的后院的池塘,此时也冻透了底。韩冈让工匠打凿了一具小小的冰橇,丢给了府中的小孩子们去玩着。正好是今天是个大晴天,后院中不但晒满了被褥,家里的女眷和侍女都出来看着小孩子玩在一起。

周南、素心给他生的一对儿女算虚岁已经三岁了,正是活泼好动的年纪。加上还有王旁的儿子。三个小家伙,就在后院中堆雪人,打雪仗,然后坐着冰橇在池塘上乱跑。

踏着那满园的乱琼碎玉,三张小脸冻得红扑扑的,一边跑着跳着,一边又笑又叫。

小孩子的尖叫和欢笑声透过书房支起的窗户传了进来,吵得房中说话都听不清楚。正在跟韩冈说着事的王旁不由得皱起了眉头,话也停了。

韩冈站起来抬手将支着窗户的木撑拨开,向下开的窗户啪嗒一声合了起来,回头对着王旁笑了笑:“的确是吵得慌。”

王旁摇摇头:“玉昆你也太宠他们,该管一管了。”

韩冈倒是无所谓,小孩子就该活泼一点,闹腾就闹腾好了,没什么大不了的:“就让他们闹着吧。我家和你家的三个加起来还不到十岁,再大点自然就好了。”

“你也是闲的。”王旁对妹夫倒没有客气,“雪橇、冰橇,用的是地方,都是军国之器,你却拿来给小孩子玩。”

韩冈呵呵笑了笑,最近他真的比较清闲,刑狱都审核了一边,大赦的名单也呈递上去了。剩下的那些流民们,韩冈的上书也已经得到天子的回复,同意他的提议,在流民中招募人手去熙河路实边,或是去荆湖南路屯田。韩冈派了人下去询问,但真正要行动,还要等到明年过了年节之后。

“难道不见小孩骑着竹马、拿着木刀吗?玩具和用具本就一类,小时候玩过,长大了也不会生疏。”韩冈对着窗外指了指,“你家的大哥儿,可比我家的儿子有精神多了。”

王旁摇摇头,对韩冈的话没有太大反应。不过这个态度已经让韩冈很满意了。

小别胜新婚这句话还是挺有道理。分别了几个月之后,王旁夫妻之间关系也算缓和了许多。至少王旁现在在表面上看不出来有什么异样了,心中的芥蒂虽然不清楚到底消除了没有,但他对妻儿的态度比过去好了不少。

在韩冈这边,没再有过去岳母吴氏写信来时,抱怨着家中鸡犬不宁的事情发生,虽然不能用和睦幸福来形容,可至少能做到字面意义上的相敬如宾了。

王旁之前怀疑儿子不是亲生,只是疑心病而已,谁也不能说儿子一定要像老子。且庞氏是大户人家出身,就算叫韩冈来看,她也的确是个规规矩矩的大家闺秀。当初又是在相府之中,有多少双眼睛盯着,怎么可能有机会会闹出什么丑事来?王旁的疑心是没来由的,王旖私下里都跟韩冈说过好几次,为她的二嫂打抱不平。

不过若不是王旖的缘故,韩冈也不会掺和进他人的家事中。至少在千年之后,就算是亲戚朋友,也是要保持着一定的距离,相类似于王旁的事情,很难插手其中。

只是因为王旖,韩冈才插手此事。虽然有违他做人行事的习惯,可如今得到了一个还算不错的结果,也就无所谓了。

窗户关上之后,房间中就登时安静了不少。

衙门里的公事没几句话就说完了,话题就转到了朝堂大事上。王旁接着之前的话题:“韩绛已经到了京城,不知道政事堂中他到底能不能给控制住大局。”

“这要看他的本事了。”

韩冈不怎么看好韩绛。韩绛在横山的表现,在韩冈眼中,可是不合格。而从口气中,不免将心意的带了出来。

王旁也听了出来,道:“看来多半还是由吕惠卿来掌控新法,韩绛只是居中把着大纛。”

“那也说不准,韩绛和吕惠卿恐怕不会如杜正献【杜衍】和范文正【范仲淹】那般和睦。”韩冈不信韩绛能甘心在政事堂中做一个摆设。

韩冈在京中与吕惠卿的交谈内容,回来后并没有对任何人说。不过他的态度,却已经让他的几个幕僚,甚至王旁都看出了来:“那样的情况也不算坏,玉昆你说呢?”

“究竟最后会如何,现在还不能确定。说这些还为时过早,等着看吧……很快就能见分晓。”韩冈说了几句没有内容的空话,就意欲敷衍过去。

王旁笑了笑:“愚兄倒是觉得玉昆你最好还是能担任中书检正一职,以你之材,当能不让吕惠卿、曾布、章惇之辈专美于前。”

韩冈知道为什么王旁会这么说。有吕惠卿、曾布在前作为例证,中书检正很明显就是一个飞速晋升的台阶,如果当真坐了上去,只要事情办得好,蹿升起来也就转眼间事。

就像御史中丞、翰林学士以及三司使那般,是晋升政事堂和枢密院的捷径,坐在这几个位置上,有不少人是直接晋身宰执的。

但韩冈则不在意的笑道:“中书检正倒也不一定要争,我坐上去也不可能如曾、吕二人那般直升翰林、三司。说起朝中职位,我倒想着能去管着军器监,当能得心应手。”

过去的布局,现在差不多到了该收线的时候。种下去的树,也改去捡果子了。眼下正是去担任军器监的好时机。

正好赵顼自己将脸送给契丹人打,地也割了,脸也丢了。现在转回来,肯定是咬指噬心,不是后悔,就是愤恨,肯定想要在军事将脸面找回来——关于这一点,真宗皇帝就是一个最好的例子。

签订澶渊之盟后,真宗皇帝就因为签订了城下之盟,自感在天下臣民面前丢了脸,千方百计想要挽回。日后的伪造天书,封禅泰山,大修上清感应宫,种种让后人啼笑皆非的闹剧,全是因为因澶渊之盟的心病而来。最后闹得连他的皇后章献刘后都看不过去,将伪造的天书丢进了他的棺材里一起埋了起来,省得让后人笑话。

赵顼从性格上,与真宗皇帝很有相似,他既然丢了脸,肯定要找回场子。而天书、封禅等事,真宗都已经做过了,那么赵顼也不可能再来仿效一遍,那样更是丢脸。所以只有军事!用煌煌武功,将脸面挽回。

军器监这个位置现在可是一个宝地,绝不比中书检正差到哪里。

王旁没有韩冈想得这么深,但他也知道,韩冈的确适合担任这个位置:“玉昆旧年就造出了霹雳炮,军棋沙盘也是玉昆你的发明,你去了军器监,打造良弓劲弩,铁甲精兵,当能压倒吕吉甫一头去。”

韩冈摇摇头,他对此自有主张。

韩冈去军器监可不是要改进武器——或者说不会立刻动手——谁说管着军器制造,就一定要学着吕惠卿的样子,去打造兵器的?

毕竟吕惠卿才刚刚卸任,而曾孝宽也只是同提举而已。如果他一上来就改动吕惠卿已见成效的法度,反倒落了下乘,在外人眼里,他就是一个意欲贬低前任功绩、彰显自己才能的小人了。萧规曹随,被世人赞许千年,有曹参先例在前,韩冈不会甘做小人。

就连制造火器,都要暂时放一放。先得将理论拿出来,然后再以实证之。这个顺序,不能错!

格物致知四个字,因为韩冈的缘故,现在被关学所抢注。他既然已为世人打开了一扇窗户,如果其他学派要驳斥他的理论,就必须给世间万物的运动变化一个合理的解释。而韩冈有着后世的记忆,虽然粗浅,但靠着那些经过千万人千锤百炼的理论,总比让他与人辩论儒学要容易得多。

既然已经将科学与关学拉上了关系,下面韩冈便可以没有太多顾忌的将科学理论拿出来。至于两者的联系,让关学的成员来想法设法的解释,来为他辩驳,并不再需要他亲历亲为了。

韩冈大略的将想法说给了王旁听,没有细说,只是说要在格物致知上多下功夫,在军器监中用于实处。王旁也只能苦笑,想不到韩冈在他的父兄南下之后,毫不耽搁的又要讲关学推上台面。

“那愚兄就拭目以待了。”王旁叹了口气,他可不是王安石和王雱,对此也没办法。顿了一顿,又道,“难怪玉昆你不肯跟吕惠卿有瓜葛,原来是因为这件事。”

