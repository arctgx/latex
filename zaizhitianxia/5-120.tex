\section{第14章 霜蹄追风尝随骠(四)}

当朝廷批准收复丰州的令旨传到太原府的时候,韩冈已经在前往麟州的路上。

没能多陪一陪妻妾儿女,韩冈心中带着深深的愧疚。不过这个时代的高层官员,调动起来十分频繁,能与家人团聚在一起的时候都不多,不独韩冈一人。

一千多河东骑兵,护卫着韩冈一行,快马行进在屈野川边。

一路上经过了好几个的村子,都是得到了走在前面的一队人马的通知,村中的耆老迎出来,在路边叩拜迎接。

韩冈正赶路,没去理会他们。直接的从村口过去,也不让下面的人去骚扰村中。每一次韩冈这么一晃而过,却都能在那些乡民的脸上,看到如释重负的表情。想来这些乡民也是不会愿意接待朝廷的官员,大概是跟畏惧蝗虫的感觉一样的。

秋后雨水少,上千匹钉了蹄铁的战马刨着黄土路面,走在队列中间的韩冈,便吃了一路的灰。

道路一侧的河水水位降得很厉害。许多地方,河面距离两边的河岸有两三丈之高。在几处水流平缓、河面宽阔的地段,暴露出来的河床也远比流动的水面要多出不少。

“龙图,”黄裳驭马凑近了韩冈的身边,“方才经过的那个村子,田里面的情况好像不太妙。”

韩冈也正想着这事,闻言就叹了一声,“怵目惊心啊……我看那些田里的麦苗长势,实在是怵目惊心。一亩地能有一百五六十斤收成就了不得了——这要明天开春后田地料理的好——等到,磨成面,就更剩不下多少。”

自从渡过黄河之后,这一路上韩冈都没有看到苗情好的田地。地里的绿色稀稀落落,麦苗出得一点也不整齐。要说原因是雨水稀少,可太原自入秋后,雨水一样少于往年,但太原的苗情就不错。韩冈出来时着意看了,田里面基本上都是齐刷刷的绿色,可不像麟州这里,就跟瘌痢头一般。

若是在太原出现这样的情况,给韩冈知道后,下面的知县就有得苦头吃了,他可不是眼中揉得进沙子的主。可惜韩冈能管麟州的兵马,却管不了麟州的政事,地方上的事务,他插手不得。

“该做的正事不做,明年麟州肯定又要向朝廷打饥荒了。也罢,这样征发民夫倒方便了。冬天的时候去修造寨堡,好歹能节省一点家里的存粮,对麟州的百姓说不定还是件好事。”

边界上的土地,并不是占下来就算数的。必须要在屈野川流域修筑起一系列的寨堡,组成一条足够牢固的防线,能够抵挡住契丹兵马的侵袭,这样才算是得到了一块真正属于大宋的土地。

可这么一来,就必须大规模征发民夫。韩冈在太原的时候让人计算过,差不多要动员五万以上的人力,才能保证在明年开春前,完成初步稳固的防线。而要形成陕西缘边四路那样水准的寨堡防御体系,更是要穷三五年之功。

只是这一切得看朝廷那边能给多少支援了。大战刚刚结束,随便哪个地方都是伸手要钱,功赏、抚恤、新修、重建,上千万贯砸下去,也就能听个响。能分配到河东这边的还真不知有多少。除了希望朝廷那边还能挤出钱来,就得靠河东本路今后几年的财税。但麟州的情况,如果发生在所有边境军州,问题可就大了。

“明年的税赋就不指望了,只盼百姓们的口粮能保证。”韩冈很是无奈的叹气道,“都这时候了,想补种也来不及,再过半月,黄河上都要有冰了。”

“龙图有所不知,今年天候偏暖,黄河在麟州的这一段要冻上还早。”折可适也凑了过来,他只听见了韩冈的后半句,笑着更正韩冈的话:“看现在这样子,差不多得到十月底,也就屈野川这里,再有一阵北风,差不多就该上冻了。”

话声顿了一下,折可适扯了一下缰绳,不让自己的坐骑超过韩冈的马头,接着又道:“其实天暖也是好事。黑山下的那一段,现在多半就已经结了冰。不过河面肯定还没冻结实。全是流冰的河道,行不了船,能耽搁辽人不少时间。”

“能多耽搁一天是一天,只要西京道的主力不回来,辽人放在边界上的那点人马,倒也不用担心。”话题被折可适带偏了,韩冈也没心思转回去,“有半个月的时间,李宪的人马就该到了,也免得在路上受冻。。”

黄裳是南方人,提起北方的冬天就有几分畏色,“军中的冬衣是不是也该发了?”

“今年下发军中的丝绵和布料还没运来……西北这一战打得也真够呛,连中原都短了运货的牲畜。”

韩冈说着,瞥了眼黄裳。给他韩冈做幕僚,黄裳还有其他门客都不缺衣服穿,每季都有四套新衣换着穿,冬衣也在这个月一起找裁缝做了,他身上正穿着呢,就是怕冷而已。

“听说熙河路这两年都是用棉花代替丝绵?”折可适问道。

韩冈点点头:“棉花比丝绵便宜,而且丝绵一次就发给二两,够什么用?一件像样的冬衣都做不了。同样的钱,两斤棉花也能买了。甘凉的田地,都适合种棉花。到时候跟熙河路一样,教蕃部种棉花去,等他们习惯了种棉织造,就脱不开大宋的控制了。”

“棉花是好东西。丝绢终究还是太轻薄,且能养蚕纺纱的地方也不多,不及棉花易于种植,又易纺纱织布。”

“不知换成棉布衣料裁成的官服穿着暖不暖和,勉仲兄其实可以去做一套。”折可适拿着黄裳打趣。

黄裳抿嘴一笑,却也不争锋相对。

一场战争从春天打到冬天,尽管远没有达成预期的目的,但该升官的升官,该发财的照样发财。功劳总归是有的。

就如在河东,一个葭芦川大捷,不仅仅让十几个勇猛无畏的军官晋身品官行列。参与谋划的黄裳和折可适也得到了韩冈的举荐。折可适本有就官职,加官一两级不成问题。而黄裳在有了官身之后,下一科,贡生的资格就好考多了。

韩冈和两位幕僚正说着话,前方就传来一道急促的马蹄声。很快,一人被带到了韩冈的面前。

兴奋的年轻人神色间充满了兴奋:“龙图,是这折府州的捷报!浊轮砦的西贼已开城投降,官军一部随即便进驻暖泉峰!”

折可适脸上顿时泛起了喜色。这一次主战的都是家中的主力兵,若是损失的多了,十年内都别想恢复元气。幸好守寨的西贼没有硬拼。

屈野川的干流是从西北流向东南,近源头的地方就是旧丰州。而屈野川还有一条从北方辽国流过来的支流浊轮川,当年宋人和如今的党项,都设立了浊轮砦来挡住这条能让辽人南下的道路。其北侧的暖泉峰,又是浊轮川边、边界上的一处战略要地,是辽人南下的必经之地。麟府军能占据此处,等于堵上了这条路。

“折府州呢?”韩冈将欣喜藏在心底,问道。

“折府州正自领主力,驻屯在子河汊上。子河汊小寨的四百西贼守军也的投降了,也没怎么打。还是照样启程,明日应就能到了旧丰州城。”

“好!”韩冈瞅着折可适笑道,“这可真是谋定而后动啊。浊轮砦和子河汊小寨的西贼守军能兵至即降,肯定是府州那边事先下了功夫。”

“也是西贼都被契丹入侵惊破了胆,才会主动投附官军。”折可适并没有否认韩冈的话。

在河东主力未至的时候,折克行麾下的麟府军便是平定地方的主力,同时还肩负堵塞辽人的由此叩关的借口。到现在为止,折克行做得算是不错了。

屈野川干流的上游,原是辽夏两国之间边界,最近处距离辽国的东胜州只有四十里。这里的一片土地,辽人也经常来往。折家的先祖折御卿曾经在子河汊大败辽人,斩首数以千计。国初时,这个战果也是诸多大捷中的一流水准。

韩冈转头对折可适道:“百年前,令曾祖领军子河汊大败辽人,不过之后就给西贼夺走,如今终于又回来了。”

“是在下先曾叔祖!”说起先人的丰功伟绩,折可适顿时挺起胸膛,倍感自豪,“当年的子河汊一场大捷,让北虏至今不敢南窥。”吹嘘了一句之后,折可适又道,“不仅仅是子河汊,有了子河汊的土地,等于又多了一处养马的好地。”

“子河汊那是河谷吧?”黄裳疑惑道。给韩冈做了这么久的幕僚,最基本的概念还是有的,“养马不是要高寒之地?要有长山大谷,美草、甘泉和旷地。子河汊那片地不算大吧?”

折可适道:“这几条,子河汊虽不能说应对得上,但都能沾点边。国初之时,军中战马以府州为最,靠得就是子河汊中的善种。其次才是环、庆、秦、渭的西马,骨格虽稍大,但蹄薄多病,不如府州马善奔。不过自从丰州陷落之后,府州的马种便一代不如一代。”

黄裳笑道:“甘凉如今已收复。陕西边地的西马日后全都能换成河西种,甚至大食种,府州马就是重夺子河汊,那也是比不上。”

折可适摇摇头:“早就不指望了。自从有了蹄铁,西马蹄薄的毛病就不算什么了,体格又胜出,府州马如何能比得上?”

“莫要多话了。”受到接报后,韩冈意气风发,“加快速度,赶往麟州城!”

