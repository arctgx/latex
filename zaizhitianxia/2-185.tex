\section{第31章 战鼓将擂缘败至(三)}

解冻未久的泾水哗哗的流淌着,难得清澈的河水带着高山融雪的冰寒。一支数千人马组成的军队,就在泾水旁的官道上迤逦南行。

春风吹绿了泾水两侧山峦,初春的风景,美不胜收。以泾水河谷为中轴的环庆路,每年到了冰雪溶解的时候,都会跟世间的其他州县一样,陷入春天的忙碌之中。

但今日的谷地中,却是寂静一片。应当开犁播种的田地,却是渺无人烟。这支大军经过的地方,连村落上都没有一道炊烟——不论是蕃人,还是汉民,都已经得到了叛军南下的消息。在这支军队尚未到来的时候,便纷纷带着家当逃入了山间。

吴逵骑着他的爱马,提着他惯用的铁枪,沉默的走在在大军中。周围的士卒也都是与吴逵一样沉默,整支队伍带着怪诞的氛围。但有许多人都背着硕大的包裹,那里面全是从庆州城中抢来的财物。

虽然跟着身边的都是叛军,但照样有着队列和号令。而且由于吴逵坚持的缘故,广锐军的旗帜依然被高高的举着,一丈多高、红底黑缘的大纛,就在前广锐都虞侯的身边,被一名掌旗官牢牢把定在手中,指引着大军前进的方向。

几个月的牢狱生涯并没有影响到吴逵的健康,相反地,因为好吃好睡,他反而还长胖了一些。

跟随着吴逵多达数千人的队伍,有骑兵,有步卒,虽然主力仍是广锐军,但还有其他军额的人马参加了进来。他们都是常年受到欺压,心头一股怨气积蓄良久,当有人举旗一呼,便群起响应。

前几天,有消息说韩绛要来庆州,敦促庆州出兵牵制围攻罗兀的西贼,城中就有传言说韩绛来了之后,要斩吴逵祭旗。对于出战的畏惧,对于韩绛偏袒蕃人的怨恨,加之吴逵在广锐军中威望极高,这就是叛乱的开始。

吴逵现在都在纳闷,王文谅那蕃狗到底靠了什么让韩绛对他言听计从。

当听到了兵变中,一片声要救自己的声音,吴逵就知道,不论做出什么决定,他都是死定了。在叛军的救援下出狱,朝廷要杀他,硬留在狱中不出去,朝廷还是一样要杀他——或者好一点,让他自尽。

终归是一个结果,没有家室之累的吴逵,也没有其他选择了。

张玉不在,姚兕也不在,除了一个林广,庆州已经没有一个能让吴逵看得起的将领。而且把他救出牢狱的几千兄弟,也不能就此放手。

被旧属从狱中救出后,放开了一切的吴逵,立刻带人将庆州南城不肯一起兵变的驻军歼灭,把知庆州兼环庆经略的王广渊吓得躲到了北城去,继而又强攻北城,逼得王广渊趁夜逃出了庆州。

吴逵在庆州城中留了三天,看似危险,但依仗庆州城的优势,轻易击败了几处星夜赶来平叛的官军,让他对叛军的控制上升了好几台阶。将粮草、兵械备足,同时将杂乱不一的叛军整编,变成能听从指挥的军队。这虽然是叛乱中死中求活的无奈之举,但也是吴逵作为一名合格将领的明证。

整编了叛军后,吴逵主动离开了庆州城,开始南下。缘边四路兵多将多,寨堡也多。留在庆州只是等死而已。

东面的鄜延路正纠缠于罗兀城的攻守之中,但只要韩绛一声令下,拼着一点损失,集合了两路精锐的大军,就随时能从绥德经过大顺城直扑过来。

而西面的泾原路,别的都还好说,兵将都不算出色,就是经略使蔡挺让人心生畏惧,原本是缘边四路中最弱的一路,但就是因为有了蔡挺,使得西贼的主攻方向都避开了泾原。

北面投夏人,吴逵从没有想过。唯一的选择是南方,虽然他不知靠着手上的兵力能在进剿的官军攻击下支撑多久,但吴逵并不甘心就这么去死。现在的三千人只有三分之一拥有战马,只要能在长安附近把马匹配足,稍加磨练,就是一支精锐。

“都虞,前面快到安定了!”一名只有十五六岁的士兵骑着马从前面过来,向吴逵禀报道,“解指挥说安定城中马多,问都虞你要不要打?”

安定县是宁州的治所,过了安定,下面就是邠州。而领着刚刚整编过的前军指挥的解吉,则是吴逵的亲信,也是将他救出大狱的首领。

吴逵想了一想,摇头道:“邠宁之间的白骥镇同样有马,防御却弱得多。你去与解吉说,让他速领本部直取白骥,为全军抵达做好准备。”

少年躬身应诺,又打着马向前跑去了。

指派下属,运筹谋算,吴逵脑中一阵恍惚,仿佛让他回到了旧时一般。若是能时间能重来该有多好,可惜了他几十年来辛辛苦苦的才挣来的都虞侯。

吴逵突的又悲愤的大笑起来,现在都这副田地了,还想什么过去?

“反正都是死路一条,拼一个够本就行!”吴逵恨恨的用力攥紧了手上的铁枪。他现在只想把声势闹大点,闹得越厉害,处事不公、让他落到现在这般田地的韩相公,就越坐不安稳。

“还有那王文谅!总得让朝廷杀了那厮!”

吴逵狂笑的神态恍若厉鬼,就算做了鬼下地狱,也要把那厮给拖下去。

……………………

晨光未露,夜雾犹在。

早春凌晨时的清寒中,罗兀城的南门悄悄的打开。趁着天亮前的黑暗,一支多达千骑的护卫队,护送上百辆马车悄悄的离开了罗兀城。

人衔枚,马裹蹄,连车轴上都抹上了厚厚的猪油,行动看似悄无声息。但所有人都知道,在城外各处的高地上,都有着一对对锐利的眼睛,盯着城门口的一星半点的动静。

他们是诱饵,是今次计划中关键的一环。

战鼓响了起来,

这个计划无论张玉还是高永能都是点头同意的。在城头上望着这一支骑兵队伍的离去,韩冈的思绪回到了昨天主帐中军议之时。

当韩冈说出了自己的意见,向众人问着‘如果一支有车有马的队伍突然悄悄的离开罗兀向南去,落在党项人眼里会是什么情况?’的时候。众人正在考虑,张玉却立刻眉飞色舞起来,毫无形象的拍着大腿,叫好道:“这一招好,正愁不能跟西贼好好拼上一把!”

得到他的提醒,想通了的幕僚们也一下兴奋起来。经过了这些日子在罗兀城的战事,城中没有一个将领害怕与党项人对阵。反而是愁着不能给党项人一个痛快。

一名四十多岁的中年幕僚也跟着叫起来:“对!趁此机会把西贼骗出来打一仗,让他们不敢追击!”

“先派出一队假的,将西贼骗出来,等阴了他们一招后,再让正主离开!”

“先悄悄从南门出城。然后等西贼出动追击后,我们就立刻出城做拖延。”种朴出着主意,他坏笑着,“要骗人,就骗到底,让西贼信以为真。”

有了种朴带头,一个接着一个诱敌上钩的计划被提了出来。人人眼睛发光,要趁此良机给围城在外、却始终不肯硬拼一场的西贼一个好看。

高永能和张玉听着这些主意,都是暗自点头,而韩冈也任由他们发挥。这些在战场上骗人入彀的本事,自然要专业人士来完成。韩冈只管出题,答案就不需要他来想了,坐等结果而已。

他只想着等送梁乙埋一个狠狠的教训,到了后面正式放弃罗兀城的时候,前次吃得亏,党项人当是还记忆犹新,只要他们稍稍犹豫,他自是能跟着大队扬长而去。等回到延州,那就是有仇报仇,有怨报怨。

一切就如计划中的一般顺利,这一干骑兵和车队在黎明前悄然离开,在两刻钟之后,让城北数里外的西夏军营地,彻底的沸腾了起来。

东方的天空此时渐渐有了一线微光,深黯的夜幕化为了瑰丽的紫罗兰色。

先是一队五六百人的骑兵奔驰出营,从旗号上看,赫然是最为精锐环卫铁骑,继而又是三四千骑铁鹞子飞驰而出。他们一前一后远远的绕过罗兀城,千军万马的蹄声撼动天壤之间,大地隆隆作响,看其汹汹去势,就是要追击离开的那队车马的模样。

“战鼓!”

张玉一声暴喝,五六十岁的老将中气十足,声震城池内外。

城头上的战鼓随之响起,鼓音震荡,压倒了西夏铁骑的撼地之声。

南面的城门中开,守候已久的城中守军从门中鱼贯而出,在战鼓声中离开了城池的护卫。两个指挥的大宋骑兵,先一步拦在了数千铁骑之前,并不硬拼,仅是稍做阻挡。

这片刻的拖延,让交战的大宋骑兵在瞬间减少了十分之一的兵力,而他们的牺牲则让出城的七千余名步卒乘机组成了战阵,利用所处位置上的优势,先一步堵住了无定河谷南下的去路。

狭窄的河谷通路只有半里多的宽度,被宋军战阵和罗兀城分去了大半空间的狭小战场上,遭到堵截的数千党项骑兵选择了开战。

终于可以好好决一次胜负了!

