\section{第三章 时移机转关百虑(五)}

又是一年新至。

关中的冬夜滴水成冰。从喧闹的大厅中出来,顿时一阵寒气侵体,吕大临不禁打了个寒颤,将身上的斗篷裹紧了些,但头脑却一下清醒了很多。

抬眼望着东方,还是沉黑的,不过已经是后半夜了,四更天,应该很快就要天亮了。

回头看看厅堂中一个个酒兴正浓的亲戚,吕大临无奈的摇了摇头,若是只有自家兄弟在,决不至于如此。

不过这也是没奈何的事。

蓝田吕氏是关中的著姓豪门,一到年节之时,族中各房亲戚能赶回来的都会回来祭祖,吕姓子弟就多达百数,加上妻妾、仆婢,就有上千人之多。在这段时间里,祖宗留下来的庄子上,比起集市还要热闹。

“怎么也出来了?”先一步站在院中的人回过头来。

“酒喝多了,闹得慌,二哥不也先出来了。”吕大临道,看看吕大钧左右,又问,“正叔先生呢?他不是跟二哥一起出来?”

“正叔先生先回去休息了。”吕大钧朝院子的西侧扭头看了一眼,程颐入关中讲学,一整年都没有回洛阳,今年年节也没有回去,在学生们都返乡后,被盛情邀请住进了吕家的老宅中,甚至连年夜饭,也被请上了正席。

吕大钧走近了几步,与兄弟并肩站着:“愚兄是出来避酒的,再过两日就要去延州了,没心情多喝。”

“延州……真的要开战了?”

“这还能有假?”吕大钧道,“如今西夏内乱,国母囚子,大好时机如何能放过。”

吕大钧是在一个月前接到了永兴军路转运副使的任命,过了年后就要去上任。接到任命书的时候,还并不知道西夏国母梁氏囚禁了儿子的消息,而且种谔的提案在朝堂上早被拖延了下来,仅仅是个普通的任命。但如今西夏内乱的消息传来,吕大钧自知战争已经不可避免,当他上任之后,随军转运的差事,少不了要占上一份。

“李稷可不好应付。”

“宁逢黑杀,莫逢稷、察……”吕大钧略带玩味的笑着,“李长卿的确行事苛暴。不过也算不上什么大碍,纵然他是转运使,愚兄也不惧他。”

“……辽国可是大碍。”吕大临沉默了一下说道,“嫁了公主给秉常,当不会坐视西虏被灭!”

身在关中,这段时间又住在乡间的庄子上,吕氏兄弟还没有收到更让人振奋的消息,但这不代表吕大钧会对辽国有多畏惧,“为了救援西夏,辽国能派出多少兵马?派得少了,连同兴庆府一并攻下。派得多了,官军就守住银夏。若是辽国全力相助……”他嗤笑了一声,“不用动手,党项人就会跟契丹援军拼命——西夏国中可供给不起辽国的多少兵马。要担心,也就担心辽人会去攻打河北,围魏救赵……不过朝堂上,虽说王韶、章惇都已出外,但知兵的重臣还有郭逵和韩玉昆在,当不至于在此事上有疏漏。”

听到兄长提到韩冈,吕大临突然间就陷入沉默。

吕大钧看了的弟弟一眼,心知肚明,叹道:“还有心结?”

吕大临的嘴紧抿了起来,他又怎么可能没有心结。因为韩冈的缘故,吕大临如今在关中学者中名声坏了不少。韩冈将几封信向关中一送,登时掀起了轩然大波,质问的信函如雪片般飞来,有一些脾性暴烈的同门,甚至直接与他割席断交了,同时也让程颐在关中讲学变得艰难无比。

“……小弟向道之心,从无一日而绝。”吕大临过了好半天才沉沉的说着,“子厚先生仙逝,小弟无处求学问道,一时怅然若失。幸而有伯淳、正叔两位先生,才又得了指点和传授。二哥你也是知道的,小弟在伯淳、正叔先生面前,何曾说过气学一句不是?子厚先生没有传授的地方倒也罢了,只要子厚先生传授过,小弟何曾背弃?!”

吕大临说得有些激动,吕大钧暗暗的摇了摇头。

吕大临的确是受了许多委屈,但那篇行状写得更是有问题。‘尽弃其学而学焉’,不论是真是伪,所谓‘为尊者讳、为长者讳’,忘了这八个字,又怎么让人看得不怒?

韩冈又是对张载敬重无比的弟子,尊师重道天下知名,看到自家兄弟如此辱没先师,没直接拔剑斩过去,已经是好脾性了。

但这些话也不好说,吕大钧轻叹一声,而后问道:“与叔你可知道韩冈现在是什么职位?”

吕大临眼神转冷,声音也平静下来:“至少还不是宰执。”

“是右谏议大夫、同群牧使!”吕大钧着重强调道,“比大哥都髙,愚兄更比不了。”

“纵使做到宰相,我不惧他一分半点。”吕大临声音更冷。

“愚兄不是这个意思。”吕大钧无奈的摇摇头,自家的兄弟对韩冈成见已深,要改变果然不容易,“熙宁三年,他帮王韶稳定了巩州,阵斩来袭的吐蕃主帅;熙宁四年,他在鄜延路保住了罗兀城的数万大军;也是同年,他亲入咸阳城,说降了广锐叛军;熙宁五年,河湟开边,他的功劳仅在王韶、高遵裕之下,甚至在王、高两位主帅追击敌寇生死不明的时候,连挡两道圣旨,保住战果不失,没有落到罗兀城之败的境地;熙宁六年,他中进士就不说了;七年,天下大旱,韩冈在白马县安置河北近百万流民,无一冻馁而死,在河北民间,他的名声好得如同万家生佛一般,当初洛阳就有被调来筑堤的河北流民,求着要韩冈去提举工役!”

吕大钧说到这里,又看了弟弟一眼,见他板着脸在听,继续道:“换作是其他人,有他这几年的功绩,进两府已经是足够了。可韩冈呢,连侍制都没坐上。接下来他在军器监任上,成就非凡。飞船就不说了,光是板甲,就让朝廷只用了不到七百万贯的花费,便给六十万禁军都配上了铁甲。同样的花销,在过去,能有十万套铁甲就不错了!而且还远远不如板甲的坚实耐用。此外,在冶炼锻造上的用心,又让铁器大行于世。可谓是为官一任,遗德深远1”

吕大临紧绷着脸,目光毫不偏移的投射在吕大钧的脸上,看不出他内心有什么变化。

“紧接着,就是南征之役。”吕大钧嗓音又转向低沉,“灭亡交趾这千乘之国,代价却微小得连西北一场败仗都比不上。当年侬智高为交趾所迫,又为国朝所不容,故而起兵反乱,狄青领军将之剿灭,便做上了枢密使。章惇做了枢密副使,韩冈功业不下于他,领军救邕州是他的功劳,大败李常杰也是他的功劳,策动诸部齐攻交趾同样是他的功劳,而令南下西军不然疾疫依然还是他的功劳,此四事,奠定了交州大捷的基础,但韩冈最后只得到了一个龙图阁学士,甚至不得不出外任官。”

见吕大临依然没有什么的反应,吕大钧叹了口气,“之后韩冈在京西的功业就不用愚兄再多说了,无论是襄汉漕运,还是轨道的运用,都对中国有着难以估量的作用,日后天下都要受其功,这也是足以晋身两府的资本。可韩冈却被投闲置散了。”

“这是他年龄资历不够,非是朝廷赏罚不公。”吕大临终于开口。

“你也知道韩冈是年资不够,而不是能力功绩不足。”吕大钧笑了一笑,问道,“那你可知道,韩冈如今年齿几何?”

吕大临再次静默下来,看着兄长,等着他的后文。

“二十八岁。”吕大钧心中又叹一口气,继续说道:“依韩忠献【韩琦】的先例,韩冈也就只要再等七八年的时间,什么都不用做,便能晋身两府。就是运气、时机都差点,再有十一二年,到他四十岁的时候怎么都该进去了。可韩冈却偏偏多做了一手,将种痘法公诸于世,不免让人有画蛇添足、节外生枝之感。换作是与叔你,会像他一样做吗?”

吕大临头昂了起来,毫无犹豫的应声道:“那是当然的,即有此等良法,公诸天下乃是义不容辞,小弟绝不会敝帚自珍!”

“说得好!”吕大钧点头赞许,他看得出来他的兄弟是言出由衷,“孔曰成仁,孟曰取义,在仁在义,都不能将种痘法敝帚自珍。”只是他顿了一下后,就又一笑,“所以韩玉昆将之公诸于众。但他不仅仅是公诸于众,而是将牛痘、人痘的事说得那么细,让人知道他早在十年前就得到了人痘的方子。韩冈将人痘之术瞒了整十年,其间天下无数幼子夭折于痘疮之疾,甚至天子也不例外,由此不免结怨于天子和世人。愚兄再问一句,换作是与叔你,会不会在得知人痘之事后,就此公诸于众?”

吕大临面现挣扎之色,脸色一息数变,最后吃力的摇头,“不会……只看人痘,已经近于巫蛊之术,绝不敢用!”

