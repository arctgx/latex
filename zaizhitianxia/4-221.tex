\section{第36章 可能与世作津梁(二)}

【自动更新没弄好,两章都耽误了,】

韩冈和沈括的话题,还是局限在如今的任务上:“汝州的旧渠,我一路过来的时候,用了两天仔细看了一遍。情况也不错,与唐州一样通水通航,也就是过方城垭口的地方处断掉。”

“所以说襄汉漕运就只有一个问题,就是方城垭口。打通此处,整条道路便畅通了”

“规划要做好。”韩冈沉声说着,“筑路的工匠大约是五月的时候能到。调集唐州、汝州和邓州三州厢军三千人参与工役,在冬天之前,漕运便能开通了。”

“轨道应该不会这么难修吧?”沈括惊讶的问道,“才六十余里,来回两条线也就一百二、一百三。”

韩冈道:“轨道两端的港口,光是用来拉货的挽马,少说也要两百匹。还有调度、车辆,都需要时间来训练。”

“原来如此。”沈括连连点头,对韩冈笑道:“还是玉昆你考虑得周全,沈括的确是欠考虑了。”

“存中兄只是忙得没去多想罢了。”韩冈摇摇头。

沈括是故意装傻,这么一个聪明人,又是为了他自己的前途,怎么可能不前前后后盘算个通透。不过他要装傻奉承自己就让他做好了,拆穿了说不定就留下芥蒂了。

“关于如何打通方城垭口,在下以已经有些想法。其实只要设坝拦水,将沙河水位提升三倍。那方城垭口那一段就能减少一半的人工。”

“但难度不小,且船只过大坝也是一桩难事,多级船闸如何跟大坝连起来,不是那么容易的事。”

“比起之前的困境,还是简单多了。大不了再旁开一条河,就像灵渠一般。而灵渠的斗门提水,最大的错误就是斗门和斗门之间隔得距离太长,灵渠便是因为这个原因,一个时辰也不过提水半尺,斗门间距如果只有两三艘船那么长,转眼就能将水位提起来。玉昆你创设的多级船闸,比起斗门有用得多。”

韩冈摇了摇头,他不是乐观主义者,也不是悲观主义者,他是极端现实的人:“要先建起来再说,如今是图纸上的推测,实际上还不知道能不能成功。”韩冈顿了一下,打个巴掌,要立刻给块糖吃,古今中外都是这般做事:“只要襄汉漕运打通,日后可能会依六路发运司【汴水】和三门白波发运司【黄河】例,在襄汉漕渠上也设立一发运司。”

“国之命脉,自然不能归于地方。”沈括眼神中闪着兴奋,“此事若成,可是相当于修了半条汴河的功劳。”

‘半条汴河吗?’韩冈淡然一笑。

沈括虽然是当时罕有人能及的大才,但襄汉漕渠实在太耀眼了,让他没有去在意对物流运输意义更大的一项发明。

可对于韩冈来说,哪一个更有意义,根本不用多想。只要轨道在襄汉漕运上发挥足够的功效,之前仅仅用在港口和矿山中的轨道,就会从此在国中推广开来。相比起勾连四方的官道来,如今的轨道,修筑、维持和使用的费用都要小上许多,而运力和运费的对比,也是轨道更为优胜。

物流是工商业发展的关键,相比起开凿耗时耗力的运河来,轨道对物流促进要还是会更大一点,而韩冈的远期规划,都少不了一个稳定的物流体系。

不过眼下最大的问题,还是一个稳定的朝堂。

“不知存中听说了没有,吴冲卿已经外放去扬州了。”喝了几杯酒,韩冈漫不经意的跟沈括提到最近朝堂上的人事变动。

沈括点点头,示意自己已经听说了。宰相的交替,对全国都有影响,王珪就任、吴充去职,这两个消息没几天就传到了沈括的耳中了。

他实在不知自己该摆出什么样的表情来,虽然他向吴充投书示好,反被吴充给卖了,但为此公开的幸灾乐祸实在不太好,在自家里乐一乐就可以了,可要是自己感叹遗憾一番,也未免太做作了些。

而且往深里去想,这也不算什么好消息。自己刚刚叛出新党,天子就利用相位的转移,向天下昭告他主张新法的心意绝未动摇。从这一件事上,沈括知道,短时间内想再回京城是不可能了。

沈括喝了一杯酒:“连着换了几个宰相了,朝中政局如同乱麻,说不定介甫相公有可能被天子下诏起复,以稳定朝纲。”

韩冈深深瞥了沈括一眼,嘴角带着一丝略嫌讥讽的微笑,没接口。

沈括脸皮红了一下,很是有些尴尬,话出口后他就知道自己失言了。在王安石离任后,捅了免役法一刀的就是他,而且之前大赞免役法,让此法推行全国的也是他。沈括嘴张了张,一时间就变得不知该说什么好,讷讷难言。

韩冈不为己甚,摇了摇头,叹道:“现在已经不是熙宁八年了。”

登基已经有十一年了,作为天子,赵顼已经有了足够的权威来控制朝堂,而国内外眼下还算稳定的局势,也让赵顼可以放手调整他的政府。

这样的情况下,他又何必将王安石请回来,两任宰相还好说,一旦三度宣麻,王安石的地位和声望就会打了滚的往上走,而赵顼在世人心目中的评价,恐怕就会变成了一个无法控制局面的无能皇帝吧——试问当今的天子会甘心吗?

只要对朝局稍有了解,就都该清楚,除非出了大事,否则王安石已经是回不来了。

沈括叹了口气,“若有介甫相公在,朝堂上必不至有此动荡。”

韩冈哂笑着:“有空说这个,还不如想想下一名参知政事究竟会是谁?政事堂中如今只有一相一参。人手太少,轮值都不够,东府肯定要进人。”

“……这要看天子的想法了。”沈括想了一想后说了一句废话。

韩冈点点头,却是深有感触:“的确是得看天子的想法……不过眼下能让天子满意的却不多啊。”

沈括又瞅了瞅了韩冈,声音微沉:“可惜了王子纯。”

韩冈一笑,很是无奈。

一般来说,在眼下宰相一再更替的情况下,政事堂中,需要再添一位能够久任的参知政事,以维护政事堂的稳定。

能担任参知政事的人选很多,基本上过了直学士一级,资历和地位就差不多了——韩冈是特例,得排除在外——比如翰林、三司、御史中丞、开封知府,或是在京外任职的一些资历合格的官员。当然,也有可能从枢密院调任——东府比起西府,一向是要高上半级,枢密副使转任参知政事,更是合情合理。而从资历来说,在枢密院中坐了五年的王韶,已经有足够的资本。

其实从一开始,没人能想到王韶可以在枢密院中待上五年。就连王韶他本人,都认为会在两三年之后离京出外,去西北的某一路,做个经略安抚使,过几年重新入京。来回几次,枢密使能做、参知政事也能做了。

可这五年间,大宋军事上的不断胜利,让天子一时不愿将稳定运作的西府大加变动,王韶也就得以跟吴充、蔡挺一起,将枢密院最高层的几个位子,把持到了去年。

到了如今,王韶已经不需要再去边州培养他的资历,就在朝堂上的他,进入政事堂,稳定如今的朝局,当然是顺理成章。而从他的政治派别上,也是天子如今所喜的中立派。这一切,都使得他在诸多合格人选中,排在最前面。

——如果他没被人弹劾的话。

就在韩冈离开洛阳的前一日,从京城传来消息,蔡确在以相州一案将宰相吴充、御史中丞邓润甫一起干掉之后,作为新近上任的御史中丞,又上表弹劾王韶滥任乡党、援引失当,乃是国之大蠹,要处之而后快。

但凡御史中丞上任后,基本上都要在两府中找个靶子练一练手,同时也是以此来向乌台中的下属,证明自己的能力。

不过王韶被蔡确咬上,这个原因只占很小的一部分,更多的,还是东府中的那几个位子。

韩冈无奈的摇摇头,王韶的确荐用乡党的时候比较多,这份弹章不能说是污蔑。但卡着这个时机上表弹劾——而且罪名不是虚构——不论最后王韶到底会被怎么处置,他离参知政事的职位,肯定已经远了许多,短时间内是不大可能从西府跳槽到东府了,而是很可能被踢到外面去。

“不就一柄清凉伞,至于吗?!”韩冈又叹了一口气,为了挤身东府,脸皮都撕下来了。至于与蔡确合谋的究竟是谁,他也不想去多想了。

沈括的笑容忽然变得有些发干发涩。

韩冈说的倒轻巧,一柄只有宰执官才能得到的清凉伞,多少人求了一辈子都没能求到手。

开国以来东西两府的宰执加起来才多少,有没有超过两百?!答案多半是否定的,也就一百出点头而已。

只是百多年来,天下文武官员总数又有多少?累积起来数以十万计。千分之一、万分之一的几率才能得到珍物,在韩冈的话里仿佛就当成了路边摊上卖的油纸伞一样。

哪里有这么简单!!

可沈括望着韩冈过于年轻的侧脸,也就以未及而立的年纪,便升任一路都转运使、龙图阁学士的韩冈,才有资格这么说。

沈括回想自己,当初清凉伞对自己来说,其实已经是触手可及了,如果没有当初的那件恨事,说不定这一次的朝堂变局,自家就能从其中挖到最大的一块黄金。

沈括咬着下唇,名为悔恨的毒蛇在他的心灵最深处徘徊不去。

