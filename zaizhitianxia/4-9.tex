\section{第一章 纵谈犹说旧升平(九)}

【赶在了六点前,勉强还能算是下午吧。还有上一章的更新,最后闹出了笑话,现在已经改正过来了,多谢各位书友的指正。】

周全虎虎生风的大步跨了出去,步履间又多了几分自信。

韩冈在后面看着他走出去,唇角上就带了点笑意。

会待人,才能用得好人。周全之前是韩家的下人,这个身份以如今的习俗,就算他做了官之后,也不会有所改变。但这件事,各自心里有数就行了,没有必要一天到晚的提醒着。毁家灭族的怨恨往往就是在一个不经意的态度中种下的,韩冈在这方面一向很是小心。

“去板甲局,把臧樟找来。”

韩冈将门外的小吏叫进来,吩咐他出去找人,又想回周全的事。

周全如今在军器监中耳目通灵,算是个很有用的亲信。不过也仅此而已,到了军器监之外,就没办法再帮忙了。韩冈想想,发现自己的个人势力还是太过于浅薄,身边连个出主意的人都没有。章惇和王韶只是盟友而已,王旁尚在白马县的府界提点衙门中任职——否则说不定还算是一个可信的助手——换句话来说,他其实一直都是在孤身奋战。

之前的三位幕僚,已经各散东西。方兴和魏平真得官之后,都外放了州县担任幕职——他们能这么快就有了官阙,也是韩冈活动的结果——只是没有一个出身,他们在官场中,正常情况下其实都走不了太远。所以游醇还是准备考进士,去了国子监读书,准备迎接今年的贡举,现在也就住在国子监中的宿舍里。

如果不出意外的话,日后这三人都能成为助力,但眼下韩冈还是需要几个能派得上用场的助手,尽管监司之职,不需要清客来辅助公务,但他日后迟早要外放的,什么没几个清客和幕宾,在州县中做事也是麻烦。

叹了一口气,韩冈还是盼着关学一脉的同窗能来投奔于他。他已经为此在给张载的书信中专门提过了,希望能给他推荐几位合适的人选。

臧樟很快就到了,由于韩冈的举荐,原大炉作的作头成了板甲局的同管勾官,另一位管勾官是由内侍兼任,负责将局中事务及时通禀天子,就跟斩马刀局的情况一样。见到监中多出来的阉人,韩冈都不知该说什么好。赵顼管得实在是太宽泛,这是天子该做的吗?幸好几位被派来做监军的内侍都很聪明,没敢在他面前乱来,而是老老实实的等着分功劳,否则韩冈肯定是忍耐不住。

“舍人。”臧樟进来后先行了礼,“不知舍人唤下官来此,有何吩咐?”

“只是有些事要询问一下管勾。”臧樟在军器监中的地位不低,要不是他的官身是靠着打铁得到的,就算接替白彰留下的军器监丞的职位,也不会在监中引起异议。对于这样的一位技术官僚,韩冈都是保持着几分敬意,“板甲局筹备完毕,板甲也开始按照预定目标每天出产。我昨日面圣时,已就此向天子禀报过,天子也说这事的确做得好。”

看了眼脸上泛起喜色的老工匠,“不过有些事想必管勾你也听说了,若是板甲局的作坊过些日子迁往京城之外,不知你能不能安排妥当?”

臧樟有些迟疑,“其他倒没什么大问题,人也好、作坊也好,迁过去就迁过去了,只要有份活干,哪里不是生活。再说,离着京城也不远。就是生铁的事,如果迁到水边,肯定就是日夜不会熄火。那时候,作坊中取用的生铁能不能供得上来?”

“徐州的生铁应该没有问题,实在不行还有相州和磁州。”

“利国监的铁矿就那么大,徐州能送来的生铁数目可能凑不上。相州和磁州从矿坑到水路的距离要远过利国监,用得又是石炭,成本太高,质地也不好。”臧樟说道,“而且去年天下铁课才五百万斤啊,连英宗皇帝的时候都不如,那时可还有八百万斤!”

“治平年间的铁冶可有‘私人承买’?现在各地矿上的冶户不都是改成了官府抽分。铁课少一点很正常,但总产量还是是增加的。”

正如臧樟所说,如今全国的‘铁课’总数每年是五百万斤——这里的‘课’是课税——比起英宗时的八百万斤少了近半。但这是因为朝廷对于铁冶管理制度进行了改变的缘故。各地的矿监依然还是官府控制,但最底层的开采和冶炼渐渐的都变成了私人承包制——‘召百姓采取,自备物料烹炼,十分为率,官收二分,其八分许坑户自便货卖’出产以官二民八来抽取——也就是生产出来的生铁官府抽两成当做税收,剩下就让坑户自己贩卖。

从工业化生产的角度来说,将矿石冶炼交由私人承包,其实是种倒退。可从管理上来看,将最为繁琐的采掘和冶炼外包出去,却是省了朝廷的许多人工,也能吸引更多的人去从事冶炼这个行业。比起旧时的冶户受到官府欺压,而户口不断流失的情况,要强了上不少。

“而且矿山遍地都是,只要有人,就能开采出来。单是徐州利国监的出产,其实可以远比现在要多。”

虽说如今的开采技术主要还是以浅层矿藏为主,但要满足举国上下对铁制品的需求,却已经足够了。一幅全套铁甲不过二三十斤,一百万套,才几万吨铁而已,实际上一年的需要不过五分之一,一年有二十万套就足够了,即便加上日后要生产的铁器,也不过需要五万吨。

当然,以现有的技术条件,一年五万吨铁,其冶炼、锻造的难度肯定远远超过后世,但只是作为原材料的矿石、煤炭,想要开采出足够的数量来,还是没有问题的,只要增加一点效率就可以了。徐州后世有名的利国铁矿,韩冈又不是没听说过,只是他现在才知道‘利国’二字来自于此时。那个产量有个零头就够了。

“但也要有人啊。利国监十六个矿坑,一年下来也不过几千万斤矿……”

“若本官记得没错的话,矿石从矿坑运出来,基本上都是用肩挑背扛的吧?”

臧樟点了点头:“矿上哪里有好路,到处都颠簸得厉害,别说马车用不了,就是独轮车都用不长,只能用人力来。”

韩冈抽出一张纸递下去,这是他用炭笔画的轨道和有轨马车的图样,后面详详细细的用蝇头小楷写了上千字的说明。

臧樟低头一看,顿时就疑惑的皱起眉头:“这是……”

“这是我准备在矿山上用的有轨马车,原理与雪橇车差不多。可以用在矿山处,也可以用在码头上。应该会比用普通的马车要好许多。”韩冈吩咐着臧樟,“论起监中的匠人,你比我熟悉得多。回去推荐几个合用的人手上来,看看能不能将这轨道和有轨马车给造出来。到时候用在五丈河码头到监中的道路上,也省得用太平车来回转运生铁了。”

五丈河是运来徐州的生铁的水路,每天都有船只停靠在军器监的码头上,但码头离着兴国坊虽说不远,但生铁、石炭等原材料的转运照样很是麻烦。韩冈早就有心铺设铁路,虽然还不可能用铁,但用硬木为轨应该不会有问题。

“另外在监中,在轮轴轮毂的方面要加以悬赏。有轨马车需要一个更为稳定的轮轴和轮毂,木质也可以,但若是能用钢铸、铁铸那就更好了。”

“下官明白。”臧樟没有二话的就点头,有板甲和飞船在前,韩冈不论说要造什么,在军器监中都不会造成疑议。

臧樟下去了,韩冈敲了敲桌子,又翻了翻随身携带的小册子,想起来还有炭火也是一桩亟需要解决的事。日后徐州利国监的铁矿石产量上来了,木炭的数量恐怕就不够用了。但改用煤炭,则炼铁质地不佳。

韩冈记得后世炼铁都是焦炭,不知是不是就是因为直接用煤炭会有什么问题。不过以他现在的地位,命人炼焦也不会多麻烦,就当成烧木炭好了。每个地方的煤炭都要试一试,看看哪个地方的煤出产的焦炭更合适炼铁,到时候通过水路转运到徐州去。

想到这里,韩冈忽然怔了一下,他记得徐州附近似乎也是有煤的,而且后世的苏北皖北——也就是如今的两淮——是遍地煤矿,靠着小煤窑发家的朋友,韩冈旧年也认识几个。中国石油虽然不多,但就是煤多,千年前后都是一样。

其实采掘也好,冶炼也好,这些都不能算是他的分内事,如果板甲局因为生铁不够而不能提供足够的产品,责任算不到他韩冈头上。各地的铁监自成系统,又不归他韩冈管辖,这是三司中的盐铁司的差事。

不知道如今的三司使元绛,会不会因为自己插手这方面的事务而心生不满。不过韩冈想了一下也就放到了一边去了,如果元绛当真因为职权被侵犯而与自己过不去,赵顼可不一定会袒护这位三司使。

