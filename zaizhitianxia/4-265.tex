\section{第43章 庙堂垂衣天宇泰(12)}

秋去冬来,又是到了岁末时分。

进入十月之后,横渠镇上连下了几场雪,气温也陡然而降,苏昞的书房中升起了火炉。一个红铜水壶架在炉子上,水汽蒸腾,给干燥的室内空气添了几分湿润。飘散着墨香和书香的书房中,一边读书,一边喝着热茶,日子过得惬意的很。

苏昞刚刚辞去了官职,留在学生少了一多半的横渠书院,一边担任山长维持气学门庭,一边则为张载留下的《正蒙》做注解。

在张载远去京城之后,横渠书院就立刻变得门庭冷落起来,只有少数学生仍在书院中学习。书院靠着留下来的几名弟子,给后辈传授课业。而等到张载去世之后,横渠书院更加萧瑟,原本跟随张载去京城的弟子,纷纷返回故里,却没有多少人回归书院。

最近又有洛阳程颐入关中讲学,有蓝田吕氏的幼子一力推重,关中士子多有投奔其门下。说起来,如果没有横渠行状之事,程颐入关中讲学,苏昞肯定是准备去听讲的。但韩冈一封信来,他苏昞跟吕家一下变得冷淡了很多,横渠书院也立刻就对程颐关上了大门。

但兵临城下,关上城门的结果,只会让敌军可以放手扫荡城外的乡野。失去了乡野的支持,城池也难以维持下去。

苏昞本来认为气学已经走入了衰败的结局,自己的努力只是拖延时间,尽一份心意而已。苏昞不认为自己或是其他师兄弟能有回天之力。任何一家学派,如果没有一个传承学术的核心人物,光是高官的支持是不够的。韩冈所学偏重自然之道,轻重失伦,纵然位高权重,日后也有问鼎两府的机缘,但要保住气学门庭还是远远不够。

不过从昨天开始,他就不再这么想了。苏昞微微笑着,仿佛压在心头的千钧巨石终于卸下了一般。

跟随着苏昞的老仆敲门进来,瞅着坐在桌前悠然的啜着茶汤的苏昞,心中有着掩不去的疑惑。自家的主人不知是遇上了什么喜事,一年多来,始终缠绕在眉宇间的忧色,似乎就是从昨日收到一份京西的包裹开始。

不过心中猜疑归猜疑,该禀报的话却不敢耽搁:“老爷,游运判和慕容知县来了。”

苏昞一下站起,“游景叔和慕容思文一起来了?”

“是,两位官人正在外面。”

慕容武是苏昞连夜派人请来横渠镇的,但游师雄竟然与慕容武联袂来访,倒是出乎苏昞的意料之外。

整了整衣袍,苏昞出门见客。

游师雄和慕容武正在外厅。

游师雄是上京诣阙经过横渠镇,而慕容武……他现在是郿县知县,横渠镇正是其辖下。

能将慕容武安排在郿县做知县,自然是韩冈。韩冈想要放个人在郿县照看横渠书院和张载的遗孀遗孤,也就是张张嘴的事。不过一个上县知县而已,还位于秦凤,审官东院和政事堂哪边都不会驳韩冈的面子。他不是为自己,而是为老师,就是捅到天子那里,都是没问题的。

当初韩冈跟王珪打招呼,王珪没有半点推脱的就将郿县的原任知县给安排去了江南一油水丰厚的望县,将慕容武调了过去。

而游师雄担任秦凤转运判官,分管熙河路的粮秣转运,这其中韩冈也出了一份力。

坐在书院待客的小厅中,游师雄正与慕容武聊着他前日拜访巩州陇西韩家庄的见闻:“韩家家中使唤的下人,大半来自于陇西。河湟之战后,一些落下残疾的老兵,带着全家投到韩冈门下。愚兄每次去韩家的庄上,看到的壮年男子,多多少少都有些残病。”

“真正没病没伤的在军中都能有个好前程,谁会投到他人门下做走马狗?”慕容武叹了一句,“记得当年韩玉昆还上书请求以老兵为教导,训练新兵,天子也是批了。如今却听说到了河北,塞进去的全是有关系混进去捞军饷的,有本事的真没几个,原本准备安排伤残老兵的缺,更是全被人占了去。更戍法更是只见雷声不见雨点,喊着要复行,就是不见有谁调来。”

“就是当初交趾那边赢得太快了,本来是为了添补南征大军留下的空缺,用了不到一万西军就赢了,哪里还需要调兵来补空当。陕西不急着要人,河北那边当然更是能推则推,能拖则拖。更别说王相公正好去位,接手的宰辅,谁还会为玉昆的提议辛苦去,让玉昆占个首倡之功的便宜?”游师雄摇摇头,不想说让人泄气的话题了,“投奔到韩家的老兵,他们的子女受到的待遇都不错,韩家在庄子上设立家学,让他们白天读书,早晚习武,并没有当成奴仆来驱用。”

慕容武笑道:“韩家家门新起,若不能收拢人心,日后也长久不了,毕竟是在熙河,少了贴心的助力,可是争不过那些山上、海边的豺狼虎豹。”

“韩家在河湟六州,土地总数超过了三万亩,还有各色作坊十余家,陇西城中的铺面也有不少。”游师雄身为分管熙河的转运判官,对当地几家大户的经济情况了解得十分深入,“区区数载便富甲一方,看起来是准备在熙河路扎下根基,开枝散叶了。”

“明摆着的事。”慕容武早就看透了,“王资政将他儿子留在熙河,就有分立家门的打算,韩家如今守着熙河,似乎也有仿效种家,转为将门的意思。”

慕容武说着并没有多少鄙夷之色。在西北,说起弃文从军,歧视当然有。但西北中进士不易,换个手段保住家门,也不是太稀奇,世人见得多了。

文官转为将门的,不止种家一个例子。当年战死在河湟一役中的景思立,一门五兄弟都在军中,其中三人殉国,而他们的父亲景泰,就是进士出身,后来才转的武职。

身在西北,想成为书香门第,难度比起文风浓郁的江南来不啻百倍。而且风险太大,只要有一代做不了高官,家门就会衰落,一旦出不了进士,家业就是树倒猢狲散。但若是转为将门,除了上阵拼杀的牺牲,保住家门却不是难事。种世衡镇守清涧城十九年,为家族夯筑好了的根基,打下了一片基业,才会让种家成为如今名声最为响亮的将门世家。

不过游师雄却是摇摇头,“愚兄倒不是说着这个意思。虽然看着根基浅,韩冈比不上王资政,高家就更不用比了。但韩家在熙河路的份量绝不在王家、高家之下。玉昆的那位表亲,不是普通的人物啊!”

“冯从义还是李信?”慕容武确认道,“玉昆的两个表亲可都不简单。”

“自然是顺丰行的冯大掌柜。他这两年奔走各方,从雍秦一地的豪商们手里,都化缘募来了不少钱钞,准备在天下各地设立雍州会所。不以生意行当区别,只以地域划分远近。照顾雍州——也就是秦凤路出来的商人、士子还有文武官员。”游师雄感慨的叹了口气,“这三位表兄弟都是异数,玉昆从文,李信从武,冯从义从商,三人在各自的那一片天地都是出类拔萃——玉昆当然更出色点——韩家家系倒也罢了,其父除了农事上其他地方都很普通,但他母家却是怎么看都觉得不简单。”

三位表兄弟中,韩冈当然是主心骨,但从李信和冯从义的表现上,也不能说他们占了韩冈多少的光。没有本事,做不了那么大的事。

“想不到运判和父母官一起来了。”苏昞一声笑,走进了厅中,打断了两人的谈话。

慕容武对苏昞的好心情惊讶得抬了抬眉毛,在他上任之后,几次见面都没见苏昞心情这般好过。“小弟是在先生的庄子上遇上景叔兄的。”解释了一句,心中则是讶异不已。

游师雄跟着对苏昞道:“小弟上京路过横渠,正好去探望一下师弟,没想到就碰上了思文。”

三人行了礼,各自坐下来。

寒暄了两句,苏昞问道,“先生庄子上的情况怎么样?”

慕容武端起粗瓷茶盏暖着手,回道:“小弟方才在先生的庄子里里外外看了一遍,房屋的情况都不错,就是后院的柴房给雪压塌了,已经吩咐人去重修。”抿了口热茶,他对苏昞笑道,“今天可真够冷的,昨天下雪时躲在房里烤火还不觉得,只觉得风雅。今天一出来,还没走两步,这骨髓都快要给冻住了。”

“多劳思文了。”苏昞点点头,又问“今年的租子都收上来了吧?有没有人抗租的?”

租地的农户不全是老实巴交的庄稼汉,也有奸猾的,主家软上一点,佃农反过来就能骑在头上。许多时候,田租都要催上几遍才能到手。寡妇幼子加上没有一个家族支撑,很容易受人欺凌,恶奴欺主的事,时常都能听到。

“郿县中的哪个也不敢赖。”郿县知县笑了一笑,“其他的州县,小弟也提前写信过去了,各家都帮忙盯着,已经交齐了大半。不仅仅是小弟,景叔兄也在帮忙照看着。”

