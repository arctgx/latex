\section{第22章 声入云霄息烽烟(上)}

“这就是安远寨?”越过一条架在甘谷水支流上的短桥,韩冈望着出现在前方的寨堡,有些不相信眼睛。

王舜臣知道,每一个第一次看到安远寨的人,差不多都会有韩冈现在的反应,他笑道,“五百步寨,九百步城,安远寨可是实打实的五百步。”

“南北只一步,东西二四九,加起来的确五百步,这样的规划也叫寨?!”

当然,韩冈是夸张了一点。寨子再如何也不会建成一条线的样子。不过安远寨的确是南北窄,东西宽。整座寨子从南到北大约五六十步,而东西长度则是南北宽的三倍,近似于一个扁扁的长方形。寨墙从西侧山头延伸下来,一直拖到甘谷水的河滩旁,将官道正好拦住。

“这样的寨子可不好防守……”安远寨东面是甘谷水,南面是支流,两水就在安远寨东南角五十步外汇合,可做城壕之用,但党项人如要攻来,却是只会从北面。

“三哥你可说错了。”王舜臣难得的能有教训韩冈的机会,他笑着解释道:“安远寨不能从外面看,进到里面就知道了。外面看着是一体,其实分作上下两寨。山上的一段是上寨,谷底的一段则是下寨。下寨是易破,但想攻下上寨可就难了——地势且不说,里面有好几口二十丈深的水井,足足费了半年才挖成,从不干涸,一点都不怕敌军断水。”

“原来如此!”韩冈点头受教。想想也是,打了多少年仗,修了几百上千的寨堡,宋人要还是会浪费人力物力去修一个无法防守的寨子,那就是笑话了。安远寨修成如今的形制,自然有它的道理在,不是自己随意一眼就能评判的。

说着,一行人已到了寨子前,验过关防,又经过了远比伏羌城细致十倍的检查,韩冈和车队终于被放进了寨中。

正如王舜臣所说,安远寨是个被一分为二的寨子。两寨之间的隔断并不低于外围寨墙的高度和厚度。西侧的上寨随坡而上,东侧的下寨则地势平坦。下寨中,是营地和衙门,而上寨则安置了军库、粮囤,刁斗森严数倍于下寨。

此时的安远寨人声沸腾,周长五百步的寨子,不知挤进了多少军民。连接南北门的主道上人头涌涌,韩冈的车队被挤得寸步难行。

“不知现在寨中有多少人?”韩冈再回头看看,大书了‘劉’字的红色将旗正高高飘在寨墙上,“伏羌城的一千兵,不至于把安远寨挤成这般模样。”

“还有达隆堡的人。秦州参与回易的商队,有三分之一是去达隆堡做买卖。”

达隆堡在安远寨的西面,顺着安远寨南的甘谷支流向西七十里就是达隆堡——得名自居住于其地附近蕃部隆中部,即抵达隆中的意思——而沿着寨东的甘谷主流向北三十里则是甘谷城。

“向家的商队也是从达隆堡回来的罢?”韩冈尚记得赵隆说过的话,“昨日向家便在伏羌城了,这些人今天才到安远寨。”

王舜臣冷冷笑道:“谁能跟都钤辖家比耳目消息?”

他又问韩冈:“三哥,下面是继续往甘谷城去,还是留在安远寨这里?”

韩冈没有直接回答,反问道:“为何伏羌刘知城不带兵继续北进甘谷?”

“安远寨属于伏羌城防区,刘知城守在这里没有问题。但甘谷城是张老都监在管,不得军令,哪个敢任意越界?”

王舜臣出身武家,自出了娘胎就在军营里打混,对军中的情弊却是一切门清,他嘿嘿冷笑,道:“其实这也是借口,已是军情紧急,刘知城带兵驰援甘谷,李相公都不会说话,反而要奖赏。现在顿兵安远寨,只是求个安稳,不多做,就不会犯错。刘知城留在安远,甘谷城失陷便与他无关,可只要他北出安远寨,往甘谷城走上一步,就代表他已经出兵援救甘谷城。一旦没能救下,便要一体受罚。”

他叹了一口气:“俺们武人升官难呐,拼了命才升得几级。但贬官却是容易,犯点事便是三五级的往下掉。一次追贬十几级,从崇仪使降到效用士的也不是没有过。不奉上命,哪个愿自投险地?”

“哪边都一样啊……”韩冈也感慨着,做得多,错得就多,不如老老实实等着上命。千年前,千年后,哪个时代的官僚都是一般德性。人性不变,人情亦不变……也幸好如此,否则他也难在此地混出头来。

“那我们怎么办?”王舜臣问道,“是继续去甘谷,还是暂且留在安远?”

韩冈沉吟起来。

不即时去甘谷,先留在安远寨等消息,借口都是现成的,而且最多一两天就能有个结果,这样也安全一点。何况他现在在街上,正看到了几支在伏羌城曾见过的、预备要去甘谷的辎重队伍,都没有往北去的打算。罚不责众,大家都一样,谁都没话说。就算陈举要找麻烦,吴衍也好、王厚也好,都有足够的理由帮他开解。

想到陈举,韩冈嘴角扯动,露出一丝轻蔑的笑意——如今他得罪了向宝,却与王韶的衙内交好,又有裴峡谷中一战的功绩,名声必然能直达经略使李师中的案头上。不论李师中对他的感观如何,却不会容忍胥吏欺辱一位已有重名的士子。数日前,陈举对他来说还是一手遮天的奢遮人物,如今,却已不在话下。

再回到去与不去的问题上。如果按照预定行程准时抵达甘谷,的确要冒风险,可得到的回报一样丰厚。甘谷城危,众将皆退缩,无一人敢援。但此时,一名衙前带着三十余人押着军资抵达甘谷城,这是再光彩不过的演出。同时还能得到秦凤路第三号武将张守约的看重,正好可以把向家可能有的攻击给堵回去。

思绪停在这里,韩冈自嘲的笑了。都到了安远寨,只差三十里,如何不拼到底?与其把解救自己的希望寄托在吴衍、王厚身上,不如通过自己的努力,让向宝、陈举之辈,不敢动自己分毫!

他猛抬头,望北方。渐渐西斜的阳光下,狼烟依旧滚滚。他再回头,数十道信任的目光正等待他的决断。

哈哈一笑,韩冈转身率先前行,“走!去甘谷!”

……………………

夜色如墨。

行走在朔日的夜空下,周围没有半点灯火。除了民伕们手中的火炬照亮了一点周围的地面,让队伍不至于走到官道外,就再无一点亮过星光的光源。

深一脚,浅一脚的在不算很平整的官道上前进,一路行来,一众民伕都被韩冈所慑服,对他的决定没有太多的怨言,也不敢有所怨言。

在出安远寨时被监门官挡了一阵,辎重队的行进速度比预计的要慢了快两个时辰。原本酉时【下午五点到七点】前就该抵达甘谷城,但现在已经近戌时【晚上七点到九点】,却还没有看到甘谷城的影子。

入夜后,山谷间的寒风更加凛冽,不住往衣襟里灌去。躺在车上,身子转眼就会变得僵冷如冰,连伤员们都不得不下车走路,好让自己暖和一点。

王舜臣吸了吸鼻子,向着走在身边的爱马靠了靠。寒风吹得久了,身子都变得麻木,心底暗骂着监守安远寨北门的监门官,却没气力骂出声来。不过他右手依然有力的握着战弓,谷内的心波三族都有不稳的迹象,入甘谷后,只要出了城寨,他便握紧了长弓。就算因为受伤不得不改用左手控弦,王舜臣依然有自信将箭囊中的长箭,尽数射入拦道贼人的要害。

韩冈走在王舜臣的身后,山谷两侧的山峰,挡住了大半幅夜空,只能看到长长的一条夜色。宋代的夜晚不比千年之后,在他出生地时代,即便无星无月的子夜,天空中依然泛着地面灯火映出的亮光。但此时,除了黯淡的火炬和寥落的星子,天地间再无一丝微光,那是最为纯粹的浓黑。

随着队列前行,身前的浓黯不断被火炬驱散,而身后却又被四周涌来的黑暗所掩盖。脚步和车轴的吱呀声,单调的回荡在谷地中,如影随形。就像整个世界,就只剩下他们这一行人。只有偶尔随风传来的两声夜枭尖利的啸叫,让他们了解到还有其他生灵存在于身边。

从安远到甘谷,不过三十里的道路,到底还要走多久?!

木然的低头看着被火光照亮的前路,韩冈一步一步向前走去。前路一片黑沉,走了不知多久,却仍没有抵达甘谷,他的心情也逐渐低沉下去。黑暗中,原本被压下去的情绪如同从河底的淤泥中翻出,搅得他的心绪一片浑浊。

韩冈总忍不住胡思乱想,自己在安远寨作出的决断是否正确,甘谷城是否还留在大宋的手中,甚至还会想起到凤翔府舅舅家避难的父母和韩云娘,每一次,尽管理智一直在告诉他不会有问题,但他总是不由自主的要往最坏的情况去想。

ps:离着甘谷越来越近,韩冈的这段旅程即将结束。但黑暗中,依然有阴影存在。

今天第三更,征集红票,收藏。

