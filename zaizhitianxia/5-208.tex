\section{第23章 弭患销祸知何补(16)}

韩冈已经退下去了,既然不支持对辽用兵,赵顼也没什么跟他好说的了。

双手撑着沙盘的边框,赵顼青白的脸色阴阴如晦,略薄的双唇紧紧抿着,盯着沙盘上起起伏伏的地形,许久没有说话。

寻遍朝堂,两府宰执和知兵的重臣竟然没有一个支持他的。朝臣们一盆盆冷水泼上来,大宋天子的心情要是能好得起来,那才叫有鬼。

韩冈出得主意是不错,等辽国内乱,跟耶律乙辛比寿数,凭借着近二十岁的年龄差距,迟早能等到大辽尚父的死讯。但赵顼就是不甘心啊,这样的比法,乌龟倒是比大虫、狮子都要强了。

万一耶律乙辛能活到八十又该如何是好?那不是要等到猴年马月去。

等待时机?机会是要靠自己去拿,而不是靠天上掉下来。

如果边境平安,外无援手,就算是日日夜夜想耶律乙辛死,辽国之中忠于旧主的一帮人也只能隐忍不发。可若是大宋摆出攻辽的姿态,甚至不用动手,辽国国内也肯定会有人受到鼓舞,甚至起兵。

赵顼就不信,耶律阿保机的子孙做了那么久的皇帝,就没有几个忠心于他血脉的忠臣。

何况耶律乙辛也不是吃素的山羊,那是吃人的老虎。等个几年,说不定就要举兵南下侵攻了——这等事,他不是做不出来。

耶律乙辛的权位并不稳固,为了镇压人心,一场场的的胜利,以及胜利后的战利品是他稳固地位必不可少的手段。拿大宋做垫脚石,耶律乙辛在过去的几年里,已经做过不止一次了。

辽人占据了黑山河间地,兴灵之地也落到了他们的手中。从河北到河东,再到兴灵,辽国在长达万里的三个区域上与大宋接壤。

这么长的边界线,利于攻而不利于守,谁保持攻势,谁就能占据优势。

赵顼前段时间将横山和横山以北银夏等的新辟疆土,并为宁夏一路,可是打着继续收复兴灵的打算。若是想继续维持守势,根本就不需要这么做,直接维持鄜延、环庆、泾原、秦凤四路分段防守的局面不就好了吗。

先发制人,后发者制于人。

这个道理赵顼不信他的宰辅们不明白,以韩冈在军事上的眼光和见识,更是不可能不明白。可他们偏偏都选了静待旁观。似乎攻灭了西夏的胜利,已经让他们的锐气消磨殆尽了。

地位高了,就不想拼命,只想保住眼下的权位,或许还有其他的理由,但怎么说都是畏辽人如虎的怯意更多一点。

想不到找个一心想要收复燕云的重臣就这么难。赵顼盯着沙盘,视线的焦点却不知落到了哪里。

没有宰辅们的支持,就算他想有什么动作,也全都施展不出来。而且所有人都不支持,难道还能将他们全都替换了不成?

且就是想换人,时间上已经来不及了,就算只调动其中两三人,换上他心仪的人选,也少不了要一两个月的时间,才能将局面稳定下来。那时候时机早过,怎么都追不回来。

赵顼憋了一口闷气在胸膛中,只觉得心口一阵阵的发慌,头也隐隐作痛。

“官家。已经快到用晚膳的时候了,若是官家还另有事,是不是派人去太后那里通报一声?”宋用臣小声的提醒着赵顼。在李舜举战殁在盐州城之后,就数宋用臣跟在赵顼身边的时间最多。

赵顼摇摇头,直起腰,沉默的向殿后走去。

每天的晨昏定省,赵顼从不会忘掉。除非有大事耽搁,他早晚都要去太后那里走一遭问候一声。每隔几天,赵顼还会去陪着太后一起吃饭,以表孝心。无事破例,反倒会让人胡乱猜测。

黄昏的时候,保慈宫中比一天的其他时候都要热闹,除了赵顼,皇后向氏也带着淑寿和赵佣来向太后请安。

“父皇!”

见到赵顼,待他问候过高太后,一对儿女便上来行礼问候。

看到儿女们满是稚气的笑脸,赵顼心中的阴云一时散尽。

赵佣比寻常的这个年纪的男孩子要瘦小一点,脸色也苍白,看起来就有些不足之症。远远不及他身边的姐姐那般康健。不过性格沉静,也不似他这个年纪的小孩子那般毛躁好动。

赵佣这时候穿戴得整整齐齐,瘦小的身子却套着一身宽袍大袖,罩着貂蝉笼巾的七梁进贤冠戴在头上,完全是正式场合上的一套仪服。

“今天学得怎么样?”赵顼坐下来问着儿子。出阁读在即,再过几日就要从内宫中出来,初次亮相在朝臣们面前,由不得赵顼不担心。

“方才给祖母看过了,”赵佣抬头朗声说着,“祖母说好。”

“是吗?”赵顼故作不信,“是祖母疼你,才这么说的?”

赵佣不敢反驳,有点可怜的望着高太后。

“是不错。”高太后说道。

“还不再演一遍,让你父皇看看。”向皇后则催促着赵佣。

赵佣站到了内厅的正中央,一板一眼的将这几天教习内容表演给赵顼看。

揖拜,恭立,奉酒,退座,动箸,起身,进退有据,一丝不苟。每一步都依从礼法,将宴上的礼节掌握到这般水平,已经没有什么再需要学的了

当赵佣最后欠身而起,下垂的双手自然收拢在小腹处,下垂的宽袖纹丝不动,整个人静静的肃立在面前,赵顼也不禁点头而笑,“看来当真是学通了。”

向皇后一把搂过赵佣,笑着道:“这孩儿就是聪明,什么都是一学就会。”

赵顼微笑着点头,这样他就放心了。

赵顼并不打算让赵佣参与祭天,以赵佣的身子骨,吹上半个时辰的冷风,最轻也要大病一场。不过之后的宫宴,是必须要上场的。

对于一个才五岁的孩子来说,宫宴这等正式场合,一套礼节也是很折磨人的。如果在宫宴上闹了笑话,在朝臣们的心目中留下不习礼法的印象,日后想要再挽回过来,可就不知要费上多少气力。若是被有心人拿去散播,更是不利于日后接掌这个国家。

幸好赵佣的表现还不错,只要在宴会上不紧张的话,应该就不会有什么大问题了。

其实赵顼也不想主持这个南郊祭天。一整套繁琐漫长以至于结束后让人半个月都缓不过气来的的仪式不提,光是每次郊天结束后,从国库里面拿出来的三五百万贯用来犒赏百官、诸军的财物,想想都是让人心疼不已。

——一百万贯的财帛,已经可以养上整整两万禁军精锐一整年了。而三五百万贯足以打上一场大战,为大宋自边境的蛮夷手中开拓一州数县之地;或是为一百个指挥的步军官兵准备上全套甲胄、兵械;也足够宫里两三年的日常开销了。

即便不谈钱,又有谁愿意在冬天里吹上一整日的冷风?更休提还要斋戒多日;来回都要端坐在寒风嗖嗖的玉辂之上;到地头后,又要换上几次衣裳,然后独自登上同样寒风嗖嗖的圜丘,进行初献、亚献、终献等一套持续几个时辰的仪式,而那张黑羊皮所制的大裘,可是一点也不挡风。

郊祀祭天,一次两次还是兴致高昂,为绝地天通的资格而兴奋不已,但三番四次后,可就纯粹是个避之唯恐不及的苦差事了。

只是这几年风调雨顺,国泰民安,又灭亡了强敌西夏。不祭谢天恩,如何说得过去?赵顼就算是想偷懒,找个借口赖掉,朝臣也不会答应,民间也免不了会有些让人匪夷所思的谣言出来。

如果这时候有个规模很大的灾害,比如以熙宁年号的十年中的后几年时所出现的大灾,倒是可以以心念万民的理由,将祭天之事给暂停。可赵顼就算丧心病狂,也不敢在心里盼望出现这样的灾难。何况熙宁七年的时候,赵顼也并没有终止祭祀上苍,那时候,他一心倒是求上天和祖宗保佑,早点将那场遍及全国的大旱给结束掉。

怠政,是国事糜烂的先兆。唐玄宗殷鉴不远,赵顼无论如何都不会做这样的蠢事。他还没到那个年纪,何况还有收复燕云的最终目标在。

总不能将这个责任留给儿孙?赵顼瞥了儿子一眼。

只是一套礼节下来,就已经累得赵佣微喘,额头上薄薄的出了一层汗,被皇后向氏抱在怀里,一张小脸也泛起了红晕,赵顼一声轻叹,“要做个好皇帝,也不是那么容易。”

虽说是坐拥万里疆域,统治亿万生民,但大庆殿上的御榻,坐上去可不是那么舒服,许多事也并不是能够随心所欲的。

赵佣似懂非懂,张大着眼睛望着他的父皇。

见气氛沉闷起来,高太后开口道:“官家,用膳,别耽搁了。”

太后的吩咐改变了殿内的气氛,宫人们立刻忙碌了起来。

在保慈宫进了晚膳,赵顼先行告退。从殿中出来,他问着身后的宋用臣,“今日政事堂谁当值?”

“回官家,是韩维。”

“去跟他说,待辽国告哀使至东京,该怎么做,就依循故事,用不着再多上禀了。”
