\section{第32章 吴钩终用笑冯唐(18)}

韩冈拒绝接受封赏的消息,传到京中之后,当即引起了一番风波。他是跟赵颢争风吃醋过的名人,在京城和朝堂上的名气比他的官职要大得多。一听到他推辞了丰厚的封赏,旧党说他知廉耻,不敢无功受禄,而新党则说他是为人重义,不愿独自受赏。可隐隐的,也有人说他是沽名钓誉。

赵顼也纳闷,拿着李宪的回书,问着王安石:“王卿,韩冈这是在为人打抱不平吗?”

当日与韩冈的对话王安石还记得,但他也没想到,韩冈竟然能言出必行。

凡事皆是有所得必有所失。横山一役,消耗了关中多年的积蓄,虽然斩首超过此前十年的总和,但还是没有达到最初的目的。功败垂成,光是把罪名推到一个叛臣的身上,就此轻轻揭过,实在说不过去。而且在功败垂成之后,宣抚司上下一人都未被治罪,说起来已经是足够宽大,再大加封赏,那究竟谁要为此事负责?

如此责难,王安石都辩不过文彦博。保住了领军众将,让韩绛事先洗脱罪责,已经做得太多了。他也得为日后考虑。留下了一个坏的先例,就会给后人留下钻空子的机会,任何一项制度都是这样一点一滴的败坏的。一个看起来说的过去的借口,就能让所有人脱罪,还要送上封赏,怎么想都会遗留后患。王安石当时思来想去,还是决定稍作退让。

不过轮到韩冈身上,情况就不一样了。他的功劳,文彦博都不能睁着眼说没有,跟游师雄一样,都是例外中的例外。而韩冈躬身践行,更是少有的事。王安石在听到长安的回信的时候,也吓了一跳。

“韩冈早前入京时,曾与臣言及罗兀难守,不愿去韩绛幕中。又曾道如果定要他去陕西宣抚司,败且不论,即便是胜了,封赏的诏书中也不要写上他的名字。臣当时只以为是,仍是强要他去了延州。后闻韩冈至韩绛帐下,在罗兀城中多有谋划。更是以为他已改弦更张,没想到还是如此强项。”

“竟有此事?”赵顼心头一震,很难得的大吃一惊。

想不到韩冈事前也这么不看好横山之事,甚至还说出了这样强硬的话。而王安石在韩冈说了这些话时,还逼着他去,更是硬到了极点。换作是他赵顼,肯定就此放过了。

‘真不愧是拗相公。’赵顼想着,‘外号当真不会起错!’

“此事千真万确。”曾布在后面为王安石作证,“当时臣等亦在旁听闻。韩冈的确是一心放在河湟之上,极力推辞前去横山。”

章惇冷淡的瞥了曾布一眼,立刻接口道:“不过韩冈并没有因私心坏国事,若非有他出力,罗兀、咸阳,皆要多生枝节。”

赵顼闻言,沉吟了一下,慢慢点头。章惇说得没错,换作是别人,不私下里捣乱就已经是阿弥陀佛了,有几人能像韩冈一样,为自己并不看好的工作而卖力,甚至在其中立了大功的?

对于这样的臣子,赵顼觉得要多加褒奖才是。

而且此前韩冈有很多功劳都没有被录入,一个不论在河湟还是在横山,都出了死力的臣子,到现在还是一介选人,赵顼一直都觉得对他都有所亏欠。

“朝廷岂有有功不赏的道理!?”赵顼说着。

若是普通的臣子作出这等近于沽名钓誉的手段,他干脆就不会去理会。他们要求名,就给他们名好了。求仁得仁嘛,当真朝廷要求着给他们封赏不成?但韩冈不同,他功劳实在太大了,人品上赵顼也信得过。

正如章惇所言,虽然韩冈反对横山之策,却没有以私心坏国事。无论韩绛还是种谔,还有张玉、赵禼,都赞他忠勤敢勇,智术过人。近日刚刚献上来的霹雳车,也是他所发明——霹雳车这个名字,还是赵顼所起。

如此多的功劳,加上诸多重臣的推荐,还有他本人的才华,莫说京官,升做朝官都绰绰有余。在赵顼眼里,韩冈除了年轻,没有别的缺点。连心性都是极好的,重义守信,刚直不阿,不为爵禄所动,这在近来赵顼做见到的臣僚,很少有人能比得上。

这样的臣子如何不重用?要加以重赏!赵顼这么想着,打算再发一次诏书过去,“以发明霹雳砲的名义如何?”

但王安石却摇头,“以韩冈的脾性,臣恐怕就算强逼着也不会接受!”

变通就是妥协,韩冈要是接受,少不得会受到嘲讽,韩冈也不会这么软弱。而敢跟亲王争风,脾气不硬那就有鬼了。

“韩冈真的是不想要封赏?!”

“以臣看来,是千真万确!”

赵顼头疼起来:“那该如何处置?”

“韩冈既然要辞让封赏,如其所愿即可。是否有为宣抚司众官打抱不平的意思,则可以不去理会。”王安石提着自己的处理意见,“以韩冈之才,回到河湟,不愁无功可立。”

“这样不太好。”赵顼摇摇头。一件事归一件事,立了功如何能不赏?回河湟立功,到时自然会依功封赏。而眼下,在陕西宣抚司的功劳,也同样要赏赐,这才是朝廷待臣之道。

“可韩冈不会接受。”王安石还记得韩冈那对尖锐锋利的眉眼,沉甸甸的眼神,就跟自己一样,都是不为外物所动的强硬性格。

君臣二人都在犯难。

章惇站了出来,“臣闻韩冈之父韩千六,虽是一介老圃,但精于农事,在通远军屯田一事多有功绩,王韶、高遵裕皆有所言。”

赵顼想了想,这也算是个变通的办法。就是韩冈官位太卑,如果他已经是朝官了,直接封妻荫子、封赠父母,处理起来很方便。不像现在,必须绕着来,“那就给韩千六赠一官。”

“得官不可无功!”曾布劝着赵顼不要太急,“不若等六月开镰,若军屯田亩果真有所收获,赠官便可名正言顺。”

赵顼沉吟一阵,点了点头,可终究还是难以释然,这非是优待功臣之道。但韩冈强硬如此,他也不能逼着来。本来赵顼还想见一见韩冈,但现在正风尖浪口之上,他不想让韩冈成为众矢之的,还得先放一放,只能再等机会了

——河湟那里也该快上一点了。

……………………

韩冈一心一意要在河湟立下功勋,把送上门的封赏,都给推辞了。这也算是对宣抚司同僚们的一个交待,现在他们对韩冈也变得亲热了许多,而不是像过往,只有武将才跟韩冈关系好。

本来游师雄也是想跟着韩冈一起来推辞封赏的,但被韩冈劝住了。韩冈他是名正言顺的宣抚司中属吏,但游师雄立功的时候是邠州军事判官,后来是被燕达征辟,再后来,才是到了宣抚司中做事。但到了现在,也还没有一个正式的编制。既然如此,又何必凑这个热闹。选人转官不容易,游师雄跟自家的情况又不一样,推了不一定会再有机会。

游师雄最终接受了韩冈的劝告,两天后,就启程去了东京城。选人转官的数量,一年大约在一百多一点。一旦转官之后,就是有资格成为亲民官——最低的也是知县。人选的合格与否直接关系到地方百姓,故而大宋历任天子都是极为看重转官一事。每一位转官的选人,都要诣阙上殿,由天子亲自评审一番。韩冈放弃了自己的资格,游师雄也只能自己独自上路。

但韩冈也不复早前的轻松。陕西宣抚司的名号尚未撤销,但帅府众官则都已经给撤去了职位。倒霉的回京城流内铨门口阙亭守着,等着张榜公布新的实缺位置;而运气好的,早早定下了职位,各自上任去了。

韩冈是从河湟临时调来,本来的职位并没有被撤消。他还是缘边安抚司的机宜文字,以及秦凤路管勾伤病事,另外,新来的诏书上又加了他一个通远军签书军事判官的职位。同时,随之而来的还有命他押送最后一批叛卒前往通远军的命令。

投降的三千叛卒,早已经分批前往通远。而他们的家属,也已经随之前往。这是早在咸阳刚刚攻破不久,朝廷便已经下发了同意的诏书,并指明由燕达负责,韩冈监管。

按照韩冈的建议,燕达同意让叛军家属与他们犯罪的子弟随行。这等于是给叛军们安排个累赘,就算半路想跑,也带着家中老弱也不方便逃跑。同时燕达还下令,在叛军中实行了连坐制,五户一队,只要少了一人,便是全队受到惩罚。

就这样三千叛军分作十批,一批批的离开了渭水北岸,知道现在,只剩下最后一批,主要有前任将校所组成的队伍。不过过去的职位早已成了陈年旧事,现在他们的身份只有一个,那就是流亡河湟的罪犯。

有人监视,有人压阵,韩冈又派出了最后十几名护工一起随行。天气热了,以防疾疫。

韩冈的威望甚隆,也注意不让押送叛军的士兵,骚扰这些罪囚的家人。在路上,没有半点风波。经过了近十日的缓慢行程。到了五月初的时候,韩冈终于看到了阔别已久的秦州州城。

