\section{第13章 赳赳铁骑寒贼胆(中)}

三更时分,万籁俱寂,而书房中仍燃着幽幽烛火。陈举犹未入眠,正与刘显隔案对坐。桌上摆着的两盏尤冒着滚滚热气的紫苏和气饮,清淡悠然的香药味随着蒸汽弥散在书房中。宋人喜饮茶,更喜欢名为饮子的药汤。陈举便最喜的便是在入夜后,喝上一盏浓浓的紫苏饮,视天候的变化增减汤中的辅料,用以滋补养身,近五十的年纪,还能有着一头黑发,也都是日常调养得宜之故。

“都安排好了?”陈举郑重其事的问着刘显,慈眉善目的一张脸透着阴狠。上一次他这般谨慎计划,是六年前要对付一个进士出身的主簿,再上一次,则是十一年前的成纪知县,如今他要害的是一个什么都没有的穷措大,但陈举的表情,如临深渊,如履薄冰,却比对上两个进士还要紧张三分。

“押司放心!今次让薛廿八和董超跟着韩三去。他们两个都是武艺高强,又对押司你忠心耿耿。一路两百里,总能找到机会料理了他。”说罢,刘显谦卑的看着陈举,“不知押司意下如何?”

陈举举着碗喝了一口滚热的紫苏饮,挑起眼问道:“没了?”

刘显楞了一下,小声问道:“……难道押司觉得薛廿八和董超两人对付不了韩三?”

“对付韩三?”陈举带着疑问的口气慢慢说着。脸色猛然突变,甩手用力一砸,哐当一声,紫苏饮在空中泼洒开,天青色的薄胎瓷碗在地上碎成了千百片,刘显从椅上被吓得跳了起来。

“你还敢小瞧韩冈?!”陈举眉头缠绕一股子戾气,指着刘显的鼻子厉声骂道:“看看你前面支的招,那猴崽子上当了没有?!他比鬼都精!两人顶个屁用,他能让王五、王九帮他杀刘三,难道就不能收服薛廿八和董超?!”

刘显被骂得抬不起头来。今天白天让陈举跟韩冈示好,就是他这个狗腿军师出的主意。只要韩冈敢为自己申诉,少不了被打上十几记杀威棒。以刚病愈的那个痨病鬼的身子骨,三五棒也就死了。能把韩冈打死在县衙中,日后谁还敢捋陈押司的虎须?没想到韩冈却一口应承了下来,什么伎俩都没用了,总不能这样还打,韩措大也是有后台的。

陈举骂了半天才停,厌憎看着百无一用的户曹书办,也不指望他的主意了,道:“末星部那里派人去知会一声,让他们动手。韩冈这一队才三十多人,末星部应该能对付得了。”

刘显有些迟疑:“拦道劫路……末星部怕是不敢动官中的财货!”

“那他们今年冬天就给我冻着。一滴酒、一匹布、一两棉花都别想从我这里买到!”陈举赚钱可不仅仅靠着鱼肉乡里,他家的商号暗地里掌控了好几家蕃部的交易权,这才是他随随便便就能拿出几万贯的主因。他冷哼了一声:“前年他们能做下,今年难道就不能做了?”

“知道了!”刘显低声应下。秦州的蕃部多有靠劫道来赚外快的,虽然很少有部族敢动官货,但商旅被劫的不在少数,末星部也不例外。但官货和私货有时不一定能分得清,就像末星部,他们前年就误劫了军资,惹起了好大一通乱子来,是因为没有留下活口才逃过了追查。只是没能逃过陈举的眼睛,成了他捏在手中的把柄。

陈举屈指叩了叩桌子,凶厉之色在眼中闪过,光是一个末星部他并不觉得有多保险,兔子还有蹬鹰的时候,狮子搏兔也不是十拿九稳:“再送封信去甘谷,跟管库的齐独眼说一声。万一末星部缩了卵,我们还有后手。”

一般来说,押运粮秣军资中最让衙前们头疼的,不是艰险曲折的道路,而是抵达目的地后,接收点验押运物资的监库官吏。如果说从秦州到甘谷在崇山峻岭中穿梭的四日行程,有如潼关之险、蜀道之难,那甘谷城的监理库帐的管勾官齐独眼就如黄泉前的鬼门关一般。

多少衙前押运了粮秣军资抵达甘谷之后,都要在齐独眼手中被血淋淋的剥上一层皮去,如果老老实实交钱免灾,那也就罢了,若是推三阻四,少不得要吃几顿杀威棒。陈举跟齐独眼交情匪浅,狼狈为奸的事情没有少做过,请他出手对付韩冈,也就是一句话的事。

“齐独眼太贪了,不大出血根本使唤不动他。”刘显替陈举心疼着钱钞,齐独眼之贪,名震秦凤,若不是他买来的后台牢靠,早就被弹劾下去,要请他出手,不是百来贯就能打发的。“可今次又不是一定要他出手,末星部的那一关韩冈根本过不去,只是为防不测才要劳动到他。”

“这笔钱省不得,宁可到最后成了画蛇添足,也不能让韩冈逃出生天去!”

如今的局势,陈举不会吝惜家产,虽然他能把韩冈弄去押运军资,但他的身家、他的弱点已经暴露的光天化日之下。只有始作俑者的韩冈死了,表面上跟自己毫无瓜葛的死了,才能让那些隐藏在黑暗中的猛兽们,收回他们的贪婪目光。

韩冈必须死!

……………………

两天后,熙宁二年十月廿八,天上铅云密布,空中寒风凛冽,今年冬天的第一场雪眼见着就要落下,无论从天气还是黄历来说,都是不宜出行的时候。但韩冈却没有按照历书自由行动的权力。

从县衙拿到通关文书,再查收了押运的银绢酒水和载货的车辆,韩冈跟赶来送行的韩千六依依道别。而韩冈的母亲韩阿李,已经带着小丫头在城外等着,等韩千六送走了儿子后,就一起去投靠韩冈在凤翔府做都头的舅舅,过了年后再回来。

韩冈的外公过去也是个都头,好水川一战,宋将任福及其麾下全军覆没后,他曾被紧急调往笼竿城驻守。与被同时征发到笼竿城的韩冈祖父结识,最后将女儿许配给韩千六做媳妇。有韩冈的舅舅这位两代在军中的老军头保护,至少安全上不用担心。

目送韩千六离城,韩冈开始了自己衙前生涯的第二项差事。

随行的有三十七名赶着骡车的民伕,他们都是乡里的三等和四等户,服的是夫役,与韩冈服的衙前役类型不同,但同样的辛苦和危险。除此之外,还有两名跟韩冈一起来押运军资的长行——军中的普通士兵都唤作长行——一个姓薛,族中排行二十八,人称薛廿八,一个大名唤作董超,都是常年在县衙中跑腿的角色。不过以韩冈看来,这两名军汉都是从骨子里透着阴狠凶戾的人物,绝不是好相与的。

‘夜里睡觉要小心了,要不干脆先下手为强。’韩冈心里盘算着,到底哪一种策略更安稳一些。他心中已是喊打喊杀,视线中也不免带上了一点杀意,如刀一般在两人的脸上划着,反倒将薛廿八和董超看得浑身不自在,最后忍无可忍,狠狠的瞪了回来。

‘还是杀了吧!’经过了那一夜,韩冈早不把人命放在眼里。只要觉得有必要,杀杀人放放火也没什么不敢做的。而他也不缺暗地里害人的手段,摸了摸藏在怀中的一个小包,不得不说,军器库真是个好地方,什么东西都有。

缴送甘谷的军资已经如数捆扎上骡车,银绢和酒水都不是占地方的东西,这些个骡车运载的数量,足以让驻扎在甘谷城里的三四千名官兵快活的过到腊月中。三十七名民伕俯首帖耳的站在车子旁边。韩冈一头头牲畜、一辆辆车子亲自检查过,确认骡子是否健康,车子上的东西是否都扎得足够结实。吴衍答应派来的人到现在还没到,韩冈费尽脑汁的想要再拖一些时间。

“韩秀才,该上路了。”董超不耐烦的催促着韩冈,薛廿八在旁拿着水火棍乓乓的捣着地面,也是等不及的样子。他们知道韩冈是在磨时间,等下去说不定事情会有什么变局。

可韩冈是一行的头领,要上路,须得等待他的命令,韩冈不肯动,他们还能架着他走?——在城中,还做不得这等事。当然,若是路上军资有所折损,罪名也是韩冈担着,得照数描赔。衙前役最苦的地方其实就在这里,因此而破家荡产的数不胜数。

‘上你娘的路!’韩冈心中暗骂,没好气的回头看了两人一眼:“磨刀不误砍柴功,你们急什么?”

等一切检验完毕,已经过去了一个多时辰。韩冈抬头看了看天色,天上的阴云越发的厚重起来,再不走,怕是到了半路上就要冒雪前进了。

“韩秀才,这下该走了罢。”

韩冈慢慢的拖时间,董超、薛廿八和一众民伕早就不耐烦的坐下来等着。见韩冈终于将最后一辆车检查好,两人站起身又一次催促着。

“天光甚好,也不用太着急。”韩冈睁着眼睛,说着瞎话。

“好个屁!韩措大你是鸟书看多了,眼珠子发昏……”董超跳起就张口开骂。

韩冈瞥眼过去,眼神锋锐如刀:“我说天光好,那就是天光好。军令在我,莫道韩某不敢杀你,以正军令!”

ps:双方都有算计,最后看谁能成功。今天第一更,求红票,收藏。

