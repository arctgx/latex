\section{第二章 凡物偏能动世情(二)}

汴河是从大宋的心脏延伸出来的主动脉,水上舟船不绝,而河岸边,也是一座码头接着一座码头,尤其是京城附近,码头、船只,更是数不胜数。不同的货物,都是从不同的码头卸下来,送到不同的仓库中去,各自互不干扰。

韩冈和苏颂二人从官船的码头走了没多远,前面就又是一座码头。不过这是卸货的去处。上百名搬运工踩着晃悠悠的船板,来回于船舱和地面。从一艘艘满载的货船中,将一个个沉重的坛子扛在肩头,搬下船来。就在码头边上,一辆辆马车顺着路停着,同样有着一群搬运工,往返于码头和车旁,将坛子转运上车。待车斗装满之后,马车便向着仓库或是城中疾驰而去。

每一个坛子,都是用着黄泥封口,外面捆扎稻草或是麦草。而在这座码头上,空气中弥漫着一股子酱香味道,但也参杂着一阵阵刺鼻的醋酸味。

苏颂指着码头上的坛坛罐罐:“这是供应京中醯醢的码头,京城内外的百万军民,日常所用的醯醢便大多从这座码头上运下来。”

醯就是醋,醢就是酱,转运酱醋的码头上,当然会留下这两种的味道。

韩冈笑着道:“只可惜不是酒水,否则就能闻到美酒的香味了。”

苏颂没有笑,问道:“听说玉昆你在军器监中,准备打造用在码头上的有轨马车?”

“正是!”韩冈点点头。

如今军器监要做什么,京城中至少有一半人在看着,都想看看韩冈会不会拿出与飞船相媲美的东西来。韩冈让人去打造的有轨马车,当然就一下子在京城中传扬开来。但军器监中严守着机密,外界尚无人能知晓内情。尤其是‘有轨’二字作何解,更是众说纷纭。

苏颂也没能想明白:“轨,车辙也。有车自然就该有轨,不知玉昆你的有轨马车究竟是什么样?”

“有轨马车是雪橇车衍生而来,重点在于路而不是车。修好了轨道,让车在轨道上行驶。”

韩冈说得有些含糊,但苏颂并没有细问,另外问道:“那玉昆你打算将有轨马车用在何处?”

“可以用在码头上,也可以用在矿山中,以货运为主。”

苏颂指着码头:“这样的码头也能用?”

“当然。”韩冈点着头:“已经在监中试过了,再过上几日,就可以用在五丈河的军器监码头上。”

车轮早已经给铸造出来了。不过不是用的铁而是青铜,而且还是外圆内方,外圆就是韩冈画出来的火车车轮模样,但中间是个方孔,将作为车轴的硬木两端削成方形插进去,正好可以卡住。外面还有一个‘辖’来卡住车轮,不让其从车轴上脱落。整个车轮并不大,只有普通的碗口大小,但卡在轨道上却没有问题。

以这样的轮轴为核心,组装起来的有轨马车,只能说凑活着用,而不能说好,并没有达到韩冈的要求。从技术含量上,甚至还不如如今的马车,只是取着制造简便而已。但实验下来的结果,却已经很让人觉得惊艳了。

的确比起用普通的马车更为方便,而且是两匹马一拉就是四辆车,加起来足足有六千斤。从两匹挽马轻轻松松向前昂首阔步的情况来看,应该可以拉得更多。只是铸造出来的车轮仅是十六个,组装出来的马车也就只有四辆——这也是与此时惯见的马车不一样的地方:京城之中大部分的车辆都是两轮,只有少部分才是四轮。

另外军器监中的工匠还设计出了两种轨道,一是按照韩刚的设计模式,用硬木打造轨道,然后将特殊式样的轮子放到轨道上。另外还有种想法,就是在路面上直接挖出两条平行的坑道,让普通马车就在坑道中行驶。这样只要维持住坑道的完好,车辆就不会受到破损的路面的影响。

“不知玉昆你是否还记得半个月前的事吗?”苏颂问着。

韩冈知道苏颂提的是哪一件事:“所以这一次只准备在兴国坊中使用,还有就是在徐州利国监的矿山中。”

“能挡着别人去学吗?”苏颂不会让韩冈轻易糊弄过去。世上的聪明人实在太多了,只要有利可图,他们学习能力足以让人瞠目结舌。

“这毕竟是好事,能省下大量的挽马,也能腾出更多的人力去做更多的事。”韩冈脸上带着淡漠的微笑。工业化进程每一步的脚印中,全都是手工业者的尸体。韩冈带来的几项技术进步,也同样免不了要造成一批人失去工作。此事难以避免,韩冈也无意为了避免此事,而延缓技术进步的速度。

“许多事的确是好事,但好事不一定能带来好结果。”苏颂不是在反对,他只是在阐释一个事实。

韩冈笑容不改,可微微扯开的唇角中,还是多了一点苦涩:“此事韩冈已经深有体会了。”

并不是因为一众磨坊兵去他家闹事,而是事情后来的发展。

半个月前的倏忽而起、倏忽而平的风波,虽然在外已经平定,但在朝中却变成了巨浪。因为韩冈的缘故,军器监的锻造作坊要顶替官营水力磨坊的位置。上百磨坊兵进城来要将事情闹大,可转眼就是领头的被打断了腿,械送进了开封府,而剩下被鼓动起来的参与者也就一哄而散。韩冈表现出来的强硬姿态,让幕后之人也不得不收手。

本来事情当会就此而止,也就有人因此而上书痛斥韩冈一番。但吕惠卿却出头支持韩冈,并声言此事绝非等闲,是扇摇军士为乱,一定要揪出幕后的黑手,明正典刑。吕惠卿摆明了要穷究到底的态度,让新党中人也一起上台大合唱。这一下子,就变成了是韩冈与吕惠卿联手掀起一场让朝中动荡的风暴来。

这件事韩冈也是有所预料,因为韩冈了解吕惠卿的为人和他现在的处境。

如今在政事堂内,吕惠卿虽然在政务上一直受到赵顼的支持,但冯京、王珪的阻碍太大,而韩绛也是明里暗里都跟着他争夺新党控制权,半年多下来,手中的势力虽然增长,却远远不如之前的预期。吕惠卿在政事堂中憋屈已久,早就在等着一个出手的机会。虽然眼下并不是什么大事,但许多事其实也只差一个借口而已。就像当年权相吕夷简清理范仲淹的势力,用的借口就是贩卖官中故纸用以饮宴。

只是吕惠卿这一下顺水推舟,推得也太欢了。韩冈虽是有这心理准备,也乐见其成,但真正看在眼里,也仍不住要暗骂上两句。他倒是占了大便宜,反倒连累了自己。不管怎么说,吕惠卿要起大狱,揪出幕后黑手的做法,绝对不会是为了韩冈出上一口气。韩冈还不能确定吕惠卿到底打算要将谁卷下来,要攀诬不难,但要攀诬到宰执官们的身上,可不是那么容易的事。

而另一方面,韩冈关于在军中设立教导队来教训士卒的提议,朝堂上还在斗着嘴。

政事堂两相两参中,韩绛、吕惠卿都表示支持,冯京则是极力反对,王珪则是左右不帮。枢密院吴充没有表态,王韶支持,而蔡挺反对。再下面的官员,也是各有各的议论。

但当有人拿着韩冈在殿上的只言片语,说将兵法是个错误,不该汰撤老弱时,就惹起了新党一方的反弹。要不是赵顼控制得宜,话题说不定就会变成了争论新法上。

人多力不齐;国事不可谋与众人。许多老话,是放诸四海而皆准。吴充将这件事推到台前来,让两制以上的重臣来合议,是个再聪明不过的手段。

这种情况,也正印证了苏颂之言的正确。

韩冈沉默了一阵,忽而又问道:“不知学士觉得韩冈的提议是否合适?”

苏颂是即将去南京上任的官员,朝堂上的讨论并没有参与进去,但韩冈想听一听他的看法。不管怎么说,苏颂的眼光和见识,韩冈经过一段时间的来往,也已经有了很深的认识。

“将为一军之胆,但历经战事的老卒,则是筋骨。没见过血的新兵的确远不如老卒。”苏颂虽然没有多手军事上的经验,但他对军队有个清醒地认识,“兵贵精而不贵多,所谓的精,不仅仅是练,也在于战。”

韩冈默默点头,但他清楚,一般人如此说话,后面肯定要跟着转折。

也的确不出他所料,苏颂的确转折了:“但伤残的士卒任职教导,能否让新兵心服口服?”

这也是反对者的理由,军中以勇力为上,若是肢体残障,难以表现出弓马枪棒上的精妙,又如何能让士卒信服?韩冈的提案并不涉及政治站队的问题,朝堂中的反对者,比如蔡挺,也是因为觉得残病士卒难以镇住军中,才选择反对的立场。

“若是改成以立过功劳的老卒组成教导队,并配以武艺、功劳皆出众的小使臣带领,而不限于残病士卒呢?”韩冈问道。

