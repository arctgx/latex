\section{第十章 却惭横刀问戎昭(七)}

【不好意思,迟了一些。不过今天肯定三更,这是第一更。】

为了报复之前雁门寨新铺被辽人烧杀一事,代州知州刘舜卿扣下了两家辽国商人。

这件事可以算是捅了马蜂窝。

被扣下的两名辽国商人并不是契丹人,而是汉人。天下万邦,世所公认,最擅长工农二事的只有汉人。便是行商天下,汉人也不输回鹘。在辽国,基本上都是汉人出来做贸易,尤其是宋辽两国之间的贸易往来,全都是由汉人把持。

不过这一干汉商的背后,就有许多契丹贵族的影子。刘舜卿扣下来的两位商人,如果加上他们的商队,人数就多达八十五。他们要带回去的货物,大半是辽国稀缺的贵价货,而且还有许多香精、宝器、佛像等奢侈品,明显就是提供给辽国的达官贵人们的货物。

紧邻代州的辽国朔州知州说不定就有股份在里面,或许还有几家贵胄豪门占了一份。并不是韩冈胡乱猜测,而是从询问出来的口供中就提到了不少达官显贵的名号。不过由于有可能是他们吹嘘,以求脱罪,只能半信半疑。

韩冈抵达代州的消息并没有瞒着人,所以当韩冈从东面的繁峙县回来的时候,朔州知州派来的使节已经到了代州,而且是点名要见韩冈。

一名身穿绿袍的官员,在雁门县衙中接待了这名使者。他们的会面,也就是谈判。唇枪舌剑,乃是免不了的。

“吕兴、晁安究竟有何罪过?!任意拘禁无罪行商,难道这就是大宋做事的规矩?”

“吕兴、晁安二人被拘,乃是其人涉案的缘故,非为他事。既然在大宋境内犯大宋刑律,当然也就要按大宋的规矩来办。”

“自澶渊之盟后,大宋大辽交好七十余年。还望录事转告韩经略、刘知州,两国的情谊得来不易,不要因细故而坏旧谊。”

“杀伤十数人,烧毁房屋六间,难道这就是大辽与人交好时惯做的事?邵祥不才,见识浅薄,不意大辽有这等流俗。”

“前日已经向贵国通报,雁门寨新铺乃是盗匪所为!我朔州萧知州已遣人追查。吕兴、晁安二人乃是正经行商,往来边界十余年,岂会与盗匪相勾连?”

辽国使者极力反对将雁门寨新铺一案和两家商人联系起来,而自名邵祥的绿袍官员则是一口咬定两人涉案。

“吕兴、晁安二人名为行商,却行事诡秘,其属多有窥伺机要之举,已经在狱中审问得实。现本县怀疑其与雁门寨新铺一案牵涉颇深,人证俱全,口供犹在,岂是污蔑?”

“既然是拘入狱中审问,要什么口供没有!?”

“邵祥不知贵国如何断案,不过大宋国中断案,非奸狡滑黠之辈,少有动用大刑的时候。雁门县中断案一向公正清明,如果新铺劫案当真与其无关,州里、县里都不会冤枉他们。更不会逼其认罪。”

“人在狱中,怎么说都由你们?”

“在下所言真伪,到了两人开释之后便可知端的。而且为何贵国能如此肯定吕兴、晁安与劫案无关?不是尚无那群盗匪的详情吗?”

“十几年的行商,几万贯的身家,如何会跟盗贼沆瀣一气。”

“或许不是盗匪也说不定!……若是贵国能尽早雁门寨新铺的凶手绳之于法,移送鄙县,待问明的确与吕兴、晁安二人无关,肯定会尽快将此二人放回。兄台与在下同为录事,当是明白做幕职的苦处。只要兄台能促成朔州尽快将当初的盗贼捕获移交,在下保证让二人立刻脱罪,不让兄台来回往返受累。”

一番商谈无果,辽国的使者大怒而回。而韩冈、刘舜卿一干主事者对此并没有太放在心上。本来就是要表现得强硬一点,辽人的反应,也在预料之中。

翻阅着特意安排人手记录下来的对话,韩冈笑着对刘舜卿道:“这邵祥做得不错,刑房录事可算是屈才了。”

韩冈根本就不见朔州派来问罪的使节。就算有个正经的官职,但区区一个录事参军,根本就没资格拜见一路经略。刘舜卿则是怕会惹来一身麻烦,也不见他,丢给了雁门县——代州的州治就是雁门县——而雁门县的官员们更是妙人,知县推县丞、县丞推县尉、主簿,县尉和主簿找不到其他官员来推了,商议一下之后,就交给了下面的录事——比押司低一级,略高于书办的吏员——最后出面接待辽国使者的便是雁门县刑房录事。

大宋和辽国之间的外交向来是采用对等原则,对方派来的使者,正常情况都是由平级的官员来接待。如果资格不够,往往就暂时赐予一个平级的官阶。在过去的几十年里,经常有借紫——提前赐予三品服章——的情况出现。因为这个惯例,雁门县刑房录事穿上了一身绿袍,假借了一个同录事参军的名头,简称正好便是录事。

这件事说来有些可笑,不过从结果上来说,邵祥表现得很不错。就像韩冈所说,一个录事的吏职的确是屈才了,以他的口才,以及胆量——破坏宋辽两国的盟约,这可不是小罪名,即是知州都不愿担在身上——应当放在更合适的位置上,才不至于浪费人才。

听到韩冈的褒奖,雁门知县连忙在下面附和:“邵祥一向行事稳妥,这些年来,县中刑房极少有差错。”

“是不是荐他一个官身?”刘舜卿提议道,“也好让他继续与辽人打交道。”

“也好。”韩冈点头道:“就先让他负责对辽人的交涉,如果办得好的话,朝廷也不会舍不得一份判司簿尉的爵禄。”

这就不是领俸禄的官员那么简单了,而是有品级的官!县学里的学长、教谕,说他们是官,也的确是官,也领俸禄,但他们都是流外官,没有品级。想要晋身流内,可不是那么容易的事,进士释褐授官,也不过是判司簿尉。雁门县中,有品级的官员也就是知县、县丞、县尉和主簿四人。

邵祥此前仅仅是个吏员而已,连不入流的官都不是——刘舜卿本意就是举荐他一个流外官——而韩冈一句话,却将他抬举到流内品官的行列。虽然还有个前提条件,但韩冈此前已经将底限画了出来,只要顺着这条线走,怎么也不可能将事情办砸了。

这番话传到外面,肯定会惹来多少羡慕嫉妒的目光,就是在率为官员的厅中,也是引来了几声感慨。

这件事议论两句,就放到了一边。仅仅是花絮而已,还有更重要的正事,否则,代州的一众文武官员,不会大半聚于州衙厅中。

韩冈问刘舜卿,“边境各寨是不是都安排好了吗?”

“已经安排好了。雁门山、屋山和恒山的那些寨子外围的军铺、烽燧,都加派人手。最有可能被辽人犯界的土墱寨、西陉寨,伏兵都安排下了。”

“再传话给各寨,让他们再小心一点,不要钓鱼不成,反给鱼拖下水。”

眼下边境的局势如同绷紧的一根弦,随时可能被剪断。就在三天前,代州、乃至宁华军、岢岚军、火山军,韩冈全都遣人通知了,让他们加强防备——那已经是韩冈上任后,第二次传令缘边各军州。如果算上他上任前,朝廷的诏敇和孙永的军令,已经是半年来的第五次。

刘舜卿低头道:“末将明白。”

韩冈和刘舜卿都不会认为辽人会咽下这口气。他们要是这么容易就善罢甘休,也不会成为从唐时开始,就困扰中国的边患。也不会认为他们会只动嘴皮子,辽人手中的马刀总是随时准备挥下。接下来,少不了会有小股兵马犯界。韩冈要刘舜卿做的,就是迎头痛击,打得他们回去.舔伤口。

至于朔州派来的使者,只不过是个无足轻重的角色,要不然也不会一级推一级,最后轮到一名胥吏接待他,简直都是笑话了。韩冈和刘舜卿身边哪里找不到更合适的人?

只因为他们都知道,要想跟辽人好好说一说话,是用刀枪打出来的,不是靠嘴皮子辩出来的。根本没必要搭理所谓的使者。

韩冈并不想挑起宋辽之间的战争。饭要一口口吃,事要一点点做,不先解决西夏,反而在伐夏之役的同时,另外再开辟一个战场,少不了要伤筋动骨。

但越是不想挑起战争,就越要表现出自己不惜一战的强硬。要是让辽人看出自己这边的顾虑,想讨价还价都难了。

萧十三、乃至他身后的耶律乙辛,同样害怕战火,一旦被逼得出兵,亲自领军还是坐镇国中,想必耶律乙辛都下不了决心。出战军队又该如何编成,同样会让耶律乙辛伤透脑筋。

麻杆打狼两头怕。

不敢否认已经订立的边界条约,将犯界烧杀的罪行推给并不存在的盗匪,辽国的态度其实已经放软了。

这样的情况下,强硬以待才是最为正确的做法。等到拼过一下之后,让辽人明白自己这边的决心,才有可能迎来人所共盼的安定局面。

