\section{第17章 籍籍人言何所图(上)}

萧禧发现仅仅数日之间,宋国君臣的态度竟然一下颠倒过来。

就在前日,他上殿谒见南朝天子时,高谈阔论,舌辩众臣,依靠着大辽的威势,只差一步就让年轻的皇帝同意放弃罗兀城,换回党项人手中的丰州。

今天本想着再加把力,让自己领到手的这份差事有个圆满的结果。可是几曾想到,当自己的上殿之后,还没有提起罗兀、丰州的事,南朝皇帝直接拒绝了,声称要用武力讨回丰州。不仅是皇帝改变了态度,连前日保持着沉默的枢密使吴充,都声称不接受这样的交换。

为了将宋人伸向西夏的手打回去,萧禧已经在东京城待了不少时日,下了许多功夫。在他看来,南朝天子外强中干,畏大辽如虎,只要自己保持强硬,他迟早要屈服。

前一次争夺代北之地,本意就是想将五十万银绢的基础上,再加上五万、十万。至于国境究竟是在分水岭上,还是分水岭下,根本就不放在心上。可惜到最后,仅仅是涨了点脸面,都没捞到半点实际上的好处。

如果是当初李元昊的。听说了南朝和西夏之间纠缠不清的纷争后,他受命调解此事时可是摩拳擦掌,准备靠着此一事,为大辽再挣来一笔岁币,哪里会想到几乎就在一日之间,一切全都变了样。

“究竟是怎么回事?”萧禧的副手耶律引吉也要一样满心疑惑。

“已经派人去打探了。”

“是不是南朝的元老重臣们上书说了什么?前几天南朝皇帝不是下诏求直言吗?”

“绝不可能!”萧禧轻蔑的冷哼一声:“那些货色只会咬着王安石。”

已经死了的韩琦是累败之帅,还活着的富弼是纳款之臣,文彦博也不过是平了个占了州城的教徒,在大辽根本都称不上是叛乱,哪年没有几起?这些所谓名臣元老,虽然国中有人赞着,但萧禧一个都看不上眼。

只有王安石,当他为相之后,宋军的战力是实实在在的在提高。南朝河北、河东禁军整编的情况,一直都派有人盯着。其中的变化,让国中上下都心生警惕,就连将国政甩手丢给魏王的天子,也一样对此多加追询。

兵贵精不贵多,南朝禁军从七八十万人,降到现在的六十万,是将身上的赘肉减去。严格起来的制度,让不合格的士兵从上位军额降到下位军中。不断淘汰不合格的士兵,将处于同一州中分属不同军额的指挥编为一将,明其指挥,统一号令,过去让宋国禁军束手束脚的制度,一点点的被废除,其战力只会比过去更高。所以这两年才有了国界之争,大辽君臣有志一同,要尽早将南朝咄咄逼人的势头给压下去。

原本是成功了,但现在却又失败了。这究竟是怎么回事?一个时辰后,答案带回来了。

“邕州大捷?!”萧禧更加疑惑,“前几天不是已经报了大捷?”

“前面还是昆仑关大捷,再前面还有宾州大捷。大捷来,大捷去,就是不见交趾人撤军,也没见救回邕州,怎么又来个大捷?”耶律引吉脑中的雾水也变得更重了。

自从章惇、韩冈这两位新党干将领军南下之后,萧禧就一直等着南面的消息。这是他谈判时的好材料,宋人越是内忧外困,就越无法坚持保着罗兀城。

不过章、韩两人抵达广西后,获胜的消息,就一个接着一个的传回俩。今天斩首一千,明天又斩首一千,后天夺回了关隘,再过一天又说降了蛮军,可到了最后,一直都没有一个确定的胜利,哪有这样的大捷?

萧禧和耶律引吉都不相信这个答案。

派出去的探子则竭力分辨自己没有说谎:“的确是差了一步没能来得及救下邕州,但传回来的捷报上说,转运副使韩冈已经将十万交趾军打了回去,俘斩过万。”

“俘斩过万?”萧禧算了一下,讶色浮于面上,“这样交趾军的伤亡要超过一半了!”

“带去的援军不是说才一千五?能大败十万交趾兵,还俘斩过万?!”萧禧的副手头摇得跟拨浪鼓一般,“就是十万条狗,也不是这么好杀的。”

“首级不好作假,南朝肯定会派人去清点。”

“清点就能证实了吗?谎报军功难道还少?”耶律引吉瞥了萧禧一眼,意有所指的说着,“只要能瞒得住,扯什么谎不敢?”

萧禧只当什么都没听到,不去理会因为太过亲附太子而被魏王耶律乙辛踢到南朝来的副手过于露骨的发言。魏王和太子之间的事不是他能掺合的,他这时候也不想公开站到任何一边,“如果是别人那就罢了,领军的可是韩冈!”

对,就是如同一颗新星在南朝官场上升起,身上拥有一道道光环的韩冈,“以他过去的行事风格,不可能弄虚作假。就算要谎报军功,也不需要弄出这样骇人听闻的战绩。”战场上什么事都能发生,尽管这一战怎么看都透着怪异,可要一口咬定绝不可能,萧禧还不至于如此强断。“当是收到这份战报,南朝君臣才变得如此强硬。”

韩冈的事迹,耶律引吉与萧禧一样已经深入的了解过了,“如果真的就危险了。宋军一千五百大胜十万交趾军,而且用得还是荆南兵,并没有动用到最精锐的西军。”

韩冈这个名字很早就已经传到了萧禧的耳中,他还细心的派人去调查了一番。毕竟是宋国年青一代数一数二的人物,说不定未来几十年都要跟他打着交道。

但萧禧真正对这个名字戒备起来的时候,是在飞船出现之后。当听说宋国有神物能直上九天的时候,萧禧的心都凉了。可是亲眼看到实物之后,再读过发明者所阐明的原理,才发现其原理竟如此简单,只是千年来无人想到——而韩冈却偏偏想到了。

萧禧戒备就是韩冈想前人所未想、思前人所未思的才能,这样的人,对于敌国太过于危险。至于国中一个劲来信追询的飞船,萧禧倒不是很在意了。

用着同样原理制造的孔明灯,是孩童的玩物。与飞船形制相似,只是不能载人的热气球,南朝城市中的酒楼门前都有在飘着,东京城中更是到处都是,下面拖着做招牌的条幅,甚至大一点的新店开张,就是两只热气球飞起来,下面的条幅是一对看得眼熟的对联——生意兴隆通四海,财源茂盛达三江。

而载人上天的飞船,不过是个特别点的军器而已,买一个一点也不难。再精贵的东西,只要给钱,南朝的商人们就敢卖。宋人将神臂弓视为珍宝,但都已经打造了十几万几十万出来了,当真以为大辽弄不到手吗?只是不便模仿、也不需要模仿罢了。

放弃仿造神臂弓,不但因为承受不了过高的成本,也因为马背上的契丹人,永远不可能、也不需要像宋人一样,聚集在一起,排列出阵势,用弓弩来抵抗敌军的侵袭。大辽有自己的战术,有自己的作战手段,并藉此成为天下的霸主,压得宋人不敢抬头,完全不需要邯郸学步。

所以宋人当成至宝的飞船,在萧禧眼中,也不过是个能飞天——不,应该说是能让人飘上空中的物件而已,看明白了就不足为奇。如果没有绳索系着,就只能在天上随波逐流的东西,对以骑兵为主力的大辽没有多少用处。不能跟着骑兵一起行动,

飞船望远防敌的能力,也只能配合着行军缓慢,辎重如山的宋军。大辽的骑兵只要过了一万,在行军时,为了取得足够的补给,往往就会散出一个超过百里的正面,外围还有扩散得更远的远探拦子马,拥有这样严密且范围广大的防线,根本不需要难以运输的飞船随行。

只有板甲……韩冈发明的板甲,不但名气很大——听坊间传言他还故意设了陷阱,让几个跟他过不去的宰执灰头土脸——而且也能跟大辽的骑兵配合得上。就是听说需要太多的机械才能打制。为了打造板甲,南朝甚至将军器监的作坊搬到了城外去。这让看到宋国几代天子如同田鼠一般将好东西尽往窝里藏的脾性,因而讪笑不已的萧禧感到惊讶无比。

从韩冈身上,萧禧看到了危险。南朝的确富庶,是大辽的百倍。但打仗起来,贫富与否就没有太大的关系了。穷了才会拼命,富人就不会了,千金之子坐不垂堂说的就是这个道理。

可宋人在军事上转变的方向,让萧禧心中有着一份隐忧。只是阵上厮杀,大辽绝不畏惧任何人,可若是变成了谁砸钱多谁就赢的话,那就大辽太过不利了。

其实军器监中所有的发明都是配合着宋军的特点:

神臂弓让宋人的箭阵更为犀利,只要一个指挥的弓弩手结成箭阵,想要冲破过去,必须要消耗光他们手上的箭矢。

斩马刀和板甲能让宋军步兵军阵变成坚不可摧的移动堡垒,六十万禁军全数铁甲,手持斩马刀列阵而行,这是大辽君臣的噩梦。

飞船更是守城、守寨的利器,几艘飞船在天上飘着,一对对警惕的眼睛从上俯视着地面,分兵合击、直贯敌背的战术就将化为泡影。

还有用于转运的轨道,能飞速打造铁器的锻锤……一系列的发明,让南朝有可能依靠着远超任何国家的财力,将军队打造成了一个无敌于天下的怪物。

“要想想办法了……”耶律引吉声音沉甸甸的。

而同样的念头,也在萧禧脑中转着,使得想想办法了,“不能任凭宋人这样走下去。”

