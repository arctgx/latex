\section{第15章 自是功成藏剑履(四)}

“木秀于林,风必摧之。堆出于岸,流必湍之。性高于人,众必非之。区区党项,杀之又何妨。岂不闻论至德者,不和于俗;成大功者,不谋于众。韩冈所为如何能说是错?”

蔡京举起酒杯,笑问着隔着火炉对坐的强渊明。

太皇太后刚刚上仙,尚未除服。酒馆茶社等去处,蔡京和强渊明两位官员是不能去的。就在蔡京家的后厅中,他两人围着一个小火炉,喝着滚烫的热酒。一旦议论起时事,便离不了韩冈这档子事。

“是不能说错,但也不能说对。这件事本来就不该做得那么过分,杀个一万也就够了。”强渊明笑道,“韩冈虽然名高位重,根基毕竟还是太浅。出身贫贱,非是阀阅之家。一旦天子不保他,就全是落井下石的,连个助阵的都没有。”

河东军上报的战绩,在御史台中引发铺天盖地的攻击。而天子似乎也没有保他的意思。蔡确在殿上给天子出得主意,看似要保韩冈,但实际上是将韩冈和河东军分开来,且明着确认了御史们对韩冈的弹劾有功无罪。

天子的申饬密诏已经在千百官僚的注目中连夜出了东京城北上太原。今天就赶着太皇太后的丧事,御史台之前还在观望的其他御史已经开始穷追猛打,而许多想博一个出身的官员,也一窝蜂的一拥而上。

韩冈之前若是被治罪,河东军都要乱了。正如吕惠卿在庆寿宫偏殿中暗示的诛心之言,两万斩首将韩冈与河东军上下都绑在了一起。但变成了如今的局面,韩冈本人却是再难利用河东军相助。

“不过小弟方才从外面过来,听到了不少议论。”强渊明继续说着,“街头巷尾,乃至国子监,对这一次御史台做下的事皆是大骂居多,没一个说他们好话的。”

蔡京了然笑道:“种痘法推行有年,其功效人人可见。胜州妄杀的党项才两万人,天下四百军州,被救下来的幼童却不啻百万。得韩冈恩惠,自然是站在韩冈一边。”

天子不就是怕着这个恩惠吗?

蔡京和强渊明对视一笑,没有说出口,却各自心领神会。

“说起种痘法,不仅惠泽大宋百姓,就连辽国也是感恩戴德。”蔡京转开了话题,说起他出使辽国时的见闻:“辽国的南院大王耶律奴哥前面四个儿子都是因痘疮而夭折。其第五子还在襁褓间,耶律奴哥担心他会得痘疮,日夜都无法安眠。去其府上种痘的时候,千恩万谢,送了珍玩什物无数,说是终于能保住这份家业了。”

“元长你去了一趟辽国,燕京城中贵胄家的好处怕是拿遍了吧?”强渊明笑说着,双手捧着巨大的两升银酒壶举了一举。

酒壶上的海东青是辽国银器上常见的图样,与宋人的富贵连枝、福禄寿一类的花样,差别一眼就能看出来。而能装两升酒的银酒壶,在辽国常见,但大宋这里却少有这般粗犷的式样。

蔡京哈哈一声笑,“都是捡了韩玉昆的便宜。”

他去了辽国一趟,礼物倒不算什么,更重要的是多了一份资历。且不说这次回来就叙功晋升,得了直史官的贴职,就是御史台,也已经在向他招手。只要名望再大一点,能在天子心中的印象再深刻一点,走上终南捷径,将是顺理成章。

在析津府的时候,因为领着一队医官传授种痘法,在辽人贵胄中还颇受尊重,只是没能得到辽国小皇帝种痘的机会,不过耶律乙辛倒是见过几次。所以回来之后,蔡京还被天子特旨召见,详细询问与耶律乙辛见面时的一言一行。

蔡京并不觉得耶律乙辛近期内会对中国有何觊觎之心。若是他笑呵呵的谈着两国夙日之盟、旧时之好,那倒是要提防上三分。但在析津府中的两个月,大辽尚父一直冷眼相待,始终都是冷遇,那还有什么可担心的?尤其是在辽人与兴庆府占了大便宜之后,更是不用担心拒绝增加岁币会惹怒辽人。

蔡京是在因韩冈而设立的厚生司中任职,而得到了去辽国的机会。现在不忘本,对他的名声很有好处。反正他人微言轻,说多少好话也帮不了韩冈。只要注意不触犯上面的忌讳,多说点其实无妨——韩冈虽然进速,说不定还要十年蹉跎。到时候,未必不能与其一争高下。

两人推杯换盏,说着闲话,忽然一阵喧哗从外面传来。

蔡京放下酒盏,疑惑的看着外面:“又是哪里出了事?”

蔡京好热闹,租的院子靠近街市,平日入夜后,街市上的声音也是不绝于耳。但如今是国丧之期,市面上一下清静了许多,蔡京和强渊明喝了半天的酒,也没有听到什么杂音。

强渊明也停杯不动,担心的道:“可不要是走了水,昨天惠德坊才烧了一半。”

蔡京一听,心中顿时发了急,忙招了外面的元随进来,让他出去打探详情。

元随下去后不久便回来了,向蔡京禀报:“直史,是河东捷报,刚刚从前街上飞捷而过,说是官军在胜州大败辽人。”

“辽人?!”强渊明惊得从座位上跳了起来,“怎么跟辽人动了手?”

蔡京也坐不住了,“速去通进银台司打探详情!”

……………………

“你们先下去吧!”章惇刚进属于他的庭院,就把院中的从人全都赶了出去。在除了他以外,没有第二个人的公厅中坐下,章惇便长吁短叹起来。

章惇这两天脾气见长,让衙中属吏都不敢接近。不仅是为韩冈无罪而受责,更有兔死狐悲的危机感。

蔡确出的主意看似是帮韩冈,其实就是硬生生坐实韩冈的罪名。天子密诏降罪,难道他还能公开上表反驳天子的话?只能捏着鼻子认下,或者就是干脆辞官。

而且更大的问题是天子的态度。韩冈在官场中十年了,不论是什么人,只要在官场中久了,肯定少不了过错,就是他本人没错,亲朋故旧总能挑出错来。现在天子摆明了不保韩冈,那么从韩冈身上、从他的亲朋故旧身上,都是能挑出刺来。

铺开信纸,就着映进西窗中的余晖,章惇提笔给韩冈写信。

天子想要打压韩冈,这一点,相信韩冈本人也知道,既然如此,怎么能给天子这个机会?

韩冈就是太糊涂!

章惇一贯的提笔万言,一边写字,一边分心到韩冈身上。

不论韩冈存了什么想法,都没必要拿着自己的前途为国家去消弭可能存在的祸患。有些事能做,有些事做不得,也不看看官家领不领情!

危身奉上是为忠,但韩冈的危身奉上,不但给了人攻击的把柄,坏了自己的名声,还得不到天子的认同。

公而忘私、国而忘家也不是这么做的。

章惇只想叹气。当年在广西,与韩冈共事的时候,也从来都没见他犯这样的错,怎么如今换到了河东,就变得这般糊涂起来,当真让人觉得纳闷……

给韩冈写信的笔突然间停了下来,章惇疑惑的抬起头,他越是深思,便越是觉得这件事做得不像是韩冈的手笔。作风也不像是韩冈的为人。

莫不是在自污吧?章惇突然想到。但随即又给他自己否定了,韩冈的直脾气,可不会如此。而且他有心光耀儒门气学,更不会让自己的身上占到难以洗脱的污点。

韩冈的品性算是刚正,但从来不是殒身而不恤的性子。以他的才智,就是再糊涂,也不会将自己往火坑里推。身为天下知名的儒者,主张凡事秉仁心,尊礼法,执中道。以中正之道明体达用,眼下的情况却是他走了极端。

难道河东前线有什么事没有报上来?

章惇疑惑着,想着是不是派人去河东走一趟。亲眼看一看韩冈是不是故意这么做。

“枢密,枢密。”来自耳畔急促的呼唤,让章惇回过神来。

“什么事?”章惇带着被打扰的怒意。

“枢密,河东路经略司露布飞捷入京师,说是大胜辽人!”

辽人……还大胜?

章惇楞然片刻,忽又失声笑了起来。笑声渐渐变大,让下面的官吏一头雾水。

抓住了辽人不甘吃亏的性子,硬是借由此事,甚至还顺便将边防城寨给修建了起来,还不惊扰边境的百姓。如此治政、谋算、用兵,便是朝堂中,也是一等一的水平。

辽人犯界,黑山党项乘势作乱,河东军一番苦战,斩首数千,让辽军惨败而归。这件事不就证明了之前韩冈对黑山党项的屠戮乃是先见之明?如此一来,朝廷如何还能以杀良之名,治罪于他,乃至河东军上下?

纵然与辽国之间还有一份澶渊之盟,韩冈将捷报一路宣扬说起来并不合适,但从他和河东军的角度讲,越是宣扬得广,那就越是安全。

在韩冈新近送来的捷报面前,刚刚做出的决议,已经成了一个笑话。御史台对韩冈的弹劾,韩冈可以一句句的驳回来。

