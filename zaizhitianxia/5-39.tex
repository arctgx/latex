\section{第五章 九州聚铁误错铸(四)}

【第二更。】

为什么梁氏身为汉人,却要反对汉化?还不是因为玩不起。胡化的政权节省开支,要学着宋人立文法、修宫室、定次序、备礼仪,转眼就能将国库给抽空掉。俗话说发财立品,家穷就别指望能有什么礼仪规矩——李清虽是夏臣,但他作为汉人的自豪感却是根深蒂固的,谁让华夏与蛮夷的区别,就是蛮夷自身都无法否认。

且这两年由于连连败阵,军事上更是一点收入都没有,只能靠与宋人的回易所得勉强维持国计——青白盐池所产的池盐不知向宋人那边走私了多少出去——加之要应付辽人贪婪的胃口,整个国家都已是处在苟延残喘的阶段。

“宋人若是再过一个月才能出兵,想要攻到灵州城下,就要顶着烈日过瀚海,即便是秦凤、熙河可以顺着黄河走,也是一样要长途跋涉。酷暑难捱,等他们到了灵州城下,不会剩下多少气力。但要是宋军到了秋凉之后才动身,对我们来说,更是一桩美事。届时战马已肥,我军在灵州养精蓄锐,而宋人出兵数月,则是疲惫不堪,正是以逸待劳,可以轻易取胜。”

梁乙埋的声音传入李清的耳中,让他警醒过来这里还是在朝堂之上。

可是听到梁乙埋自欺欺人的一番话,李清的心中只有冷笑。国势所限,即便这一仗赢了,缺乏根基的西夏除非能在宋人那里咬下一块肉来,否则绝对支撑不了多久。

李清没有与西夏偕亡的打算。

如果西夏能支撑下去,他会继续做着忠臣。但万一形势不妙,有亡国之危,他可不会死硬到底。不管怎么说他都是国中汉人的核心,尤其在景询这样的汉家文臣接连在,让手挽兵权的李清更加受到拥护。

梁乙埋并不知道李清的心思,可就算知道也不会觉得有什么值得惊奇的。首鼠两端的部族和臣子,已经数不胜数,不缺李清一个。

没人能保证眼下站在朝堂上的仁多零丁和叶孛麻,他们两家会忠心到底。反倒是梁氏和嵬名氏,一个是后族、一个是王族,投靠宋人完全没有好处,说不定哪天就被杀了满门良贱,以防太祖继迁复兴家族之事重演。

眼下已经定下的基本战略就是放手让宋人杀到灵州城下,设法断其粮道。按道理说,利用宋军分兵出击的机会,各个击破才是最上之策,诱敌深入其实已经算得上是断臂求生。但没人有这个信心,能连续击败宋人的主力,甚至彻底击败其中一路都没有把握。以宋军这些年表现出来的战斗力,如果战事发生在横山附近,最好的结果也仅仅是残胜。只有让宋军经过长途跋涉之后,利用地利不断削减他们战斗力,才能让大夏看到胜利的机会。

坚壁清野诱敌深入是很简单的策略,但如果做得好的话,还是能一举逆转战局,甚至能一举全歼来袭的宋军,让全师出动的宋国西军就此一蹶不振,如同当年接连遭逢三川口、好水川和定川寨三场惨败后的宋国一般。到时候,就又可以通过索要岁币岁赐来维持国用。

但这个战略要怎么做到却是更为关键的问题。要考验西夏君臣的执行能力,同时还有灵州的守御能力。

不过这两件事,得实际做起来才知道成不成,现在说什么都没用。

“北面的消息什么时候到?”将前些日子定下的战略又重复了一遍,仁多零丁问着梁乙埋。

“还是那句话,春夏要养马,到了秋天才能南下。”梁乙埋摇摇头,振作起来:“不过本来就没有将希望都放在北面,这一仗想要赢主要还是得靠自己。种谔退军是天助,天不欲亡我大白高国!否则诸路此时随鄜延一同合攻,想要抵挡住可就难了。”

仁多零丁年纪大了,越发的相信冥冥中有所谓的气运,点头表示同意:“虽说东朝越发强盛,之前也连番胜我王师,但这一次,明显的是过于冒进,让数十万大军自蹈险境。宋人将骄士惰,我大白高国时来运转的时候到了!”

结束了军议,从殿中出来,仁多保忠撇着嘴,跟在伯父身后。

在方才在殿上不敢多言语,但在他看来,用了一个多时辰的议事,都是说了一堆废话,该怎么样还是怎么样,依然是坚壁清野,诱敌深入,在灵州城下决战,什么都没有变动。

“不知道兀卒怎么样了。”仁多保忠回头看了一看,视线越过紫宸殿。西夏王宫不大,其后隔了两座殿宇就是国主秉常现在被囚禁的寝宫,“听说这些天,常能看到有人从甘露殿中被抬出来。”

“还是让他留在宫里生儿子吧。没看嵬名家的人,都没一个帮他说话吗?”仁多零丁毫不关心那个愚蠢的皇帝。

在他看来,嵬名家的人当真是一代不如一代,从太祖继迁开始,就是每况愈下,景宗元昊虽是自立称帝,但在仁多零丁眼中,为政手段却比他的父亲太宗德明差的太远。从辽国、到吐蕃、再到宋国,周围邻居全都打了一遍,弄得四面皆敌,还把从祖父、父亲手上继承下来的财富都消耗一空,空得了一个皇帝称号,连死因都是个笑话,抢了儿媳,被儿子削掉鼻子失血过多而死。如此可笑,让他们这些臣子都抬不起头来。

走在宫掖之中,仁多零丁不知道自己还能在这座宫殿中走上多少次。

眼下就得看这一次能不能撑下去了,如果灵州不保,仁多零丁绝不会让仁多家跟嵬名家同生共死。

在大夏陈王、枢密使的身份之前,他更是仁多家的族长。对于党项人来说,家族才是一切。至于自称鲜卑拓跋氏后人、死活都要攀个富亲戚的嵬名家,他仁多零丁管他们去死!

……………………

仲春时节,东京城中梧桐、杨柳等落叶树上的新叶,已是一片浓绿。往城外去踏青的游人也渐渐稀落了下来。再过上两个月,差不多就到了富贵人家去城外别业避暑的时候。

最后一批出征的京营禁军也在父母妻儿的送别下,雄纠纠气昂昂的开赴陕西前线。如果不是亲眼看到,很难以想像,即将上阵的军队能士气高昂到每一个士兵都是昂首阔步。在韩冈的印象中,即便常年在血水中打滚的骄悍老卒,在上阵前都会变得比平时有些异样。

韩冈在京城任官的时间,加起来说长不长说短不短,足够他看清楚京营禁军的实际水准。他们的实际战力和表现出来的自信,完全不成比例。

在赵顼强令种谔回师的整件事中,京营禁军上下肯定也是出了一把力。本来就是为了捞取功劳而上阵的他们,谁敢干扰到他们的计划,就是不死不休的结果。

韩冈已经无心去管鄜延路的那点破事。强令种谔回师,加深了各路将帅之间的隔阂。这一下子争抢功劳的戏码给拉到了明面上来了。以为将种谔拉回来,就能杀住争功之风,这完全是一厢情愿。眼下种谔多半连配属到他麾下的七个将三万余京营禁军都别想控制住了。

他现在只希望上了战场之后,京营禁军还能保持一半争功的气势,这样至少不会拖西军的后脚。

韩冈现在在群牧司中大部分的事务都属于战前的预备,随着季节从初春一步步向暮春走去,他手上的工作也变得轻松了许多。更多的时间和精力,都能放在他自己的私事上。而且周南也快要生产了,而素心在隔了六年之后,也终于又有了喜。这让韩冈的心情也变得好了不少。

不过他的家中,有人比他更关心如今的时事。

韩冈在书房中检查着一块新磨的凸透镜片。透明晶莹的晶体将透镜对面的书架扭曲了形状送入了韩冈的眼中。虽然是一名不出名的新匠师,但手艺甚至胜过了老工匠。韩冈估计换上这枚镜片之后,能将他的显微镜的放大倍数增加到一百倍。

不过要配合这片镜片,现有的显微镜却得重新打造一遍外框。韩冈正想着要怎么设计,王舜臣却突然跑来扰人清静,“三哥,王中正到京城了。”

“我知道。”韩冈将镜片小心的放进一个填满了棉花的小盒子里,转过身来指了指放在墙角的一张圆凳,示意王舜臣坐下说话。

“三哥,要不要去见他一面?”王舜臣坐下来后,试探的问着。

“他是带御器械,又是御药院都知,又久在京外,这些天肯定会进宫侍奉天子。而且多少只眼睛盯着他。”韩冈摇摇头,“我私底下不好与他有联系。”

王舜臣皱起了眉头,韩冈不去,有许多事怎么跟王中正协调好。

“我不能见他,不代表你不能见。”韩冈笑道,“你跟王中正又是熟识,想起复走他的门路,倒也说得过去。”

“这……”王舜臣犹豫了一下,抓耳挠腮的道:“没说好的事,贸贸然上门,方不方便?”

