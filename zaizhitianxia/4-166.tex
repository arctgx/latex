\section{第25章 山水留连住多时(上)}

已经是八月了。

北方秋色渐浓,夏日时的高温,也散去了许多。

而攻灭交趾,献俘阙下。百年来前所未有的灭国之功,给京城带来的狂热,到了此时,已经随着渐起的秋风告一段落。

曾经的交趾国,如今成了广西路辖下的交州。拥有七十四个羁縻州,四座军寨,以及一个县的交州,几乎可以算得上是一路之地。

基本上来说,交州的南蛮人数,是汉人数量的三五十倍还多。不过可以确定,因为畏惧官军的赫赫声威,至少十年之内,他们必然是对中国最为恭顺的边州。

分裂了百余年的交趾重新回归中原王朝的统治,而亡国之君李乾德,于献俘阙下之后,便被转封为安南郡公,并由天子赠予了一座宅邸,与其母倚兰一起要在京城养老——尽管他还不到十岁。

而交趾的朝臣们,大半沦为溪洞诸蛮的,剩下的,有一部分死在了忠勇祠前,只有少数幸运儿,与交趾国的太后、国王一起上京来,得到了一个微不足道的官职,同时也有了一份养家糊口的俸禄。

这些人,看似凄惨,其实也算是罪有应得,如果没有他们在幕后的推波助澜,交趾入寇其实很好有可能不发生。

交趾君臣如此,直接领导这了这一场灭国之战的章惇,则如愿升任枢密副使,自此进入了执政的行列。至于官阶、封爵、职名,还有金银财帛,这些林林总总的赏赐实在是难以计数。只是没有太多的实际意义了。

另外还有主将燕达,因为对交趾的军功,他的现在已经是稳坐在三衙管军的位置。燕达出身京营,又有着边功,本身还是屡屡得到天子越次拔擢,日后代替郭逵成为军中代表人物,首屈一指的武将,也是不在话下。

——当然,郭逵本人是绝不会甘心被年轻人超过去的,他可也是新近击败了丰州的党项人,一同将前来捡便宜的契丹人也一并踢了出去。

辽人猖狂了许多年,如今受到挫折,却不知道该怎么去做。到底是开战,还是忍耐,辽人自己都陷入了两难之中。这样的武功,在过去的一百年中,没有哪一名帅臣有资格说一句不算什么。

李宪也一并得到了奖赏,随着交趾覆亡,他在宫中的地位也是水涨船髙,已经接近了他的老对头王中正。在还不清楚到底能不能解决交趾人的情况下,他自请南下,也算是赌对了一把。接下来就该是在北方建功立业,自此成为留名青史的名宦。

韩冈同样有重赏,差遣没有变,还是广西转运使,不过已经是正经八百的龙图阁学士加上食实封的爵位,而官阶也升到了正六品。此外,父母、兄弟,妻妾都有封赠,五个儿子全都得到了荫补。

要知道,正常的州官只能在致仕和遗表中,为自己的子孙挣个一官半职。要想像韩冈家里一般,襁褓中的幼子都吃着朝廷俸禄,至少得做到宰相才有可能。

朝廷的封赏之丰厚,让人无话可说。就连被留在广西继续任官的韩冈,对此都没有抱怨什么。

但有人抱怨,只是与战事无关,而是为了大宋国中的安靖。

于年初结束了战争之后,熙宁十年到目前为止的大部分时间,都显得平静无比。不过河北和陕西又是遭了灾,依然还是旱。从熙宁五年开始,国中的灾异一个接着一个,水旱连连,想逃都没处逃,民间受损无数。

这样的灾情,在援救的同时,已经不只有一个人,在考虑着是不是该改个年号了。

在使用着熙宁这个年号的十年里,虽然对外战争一直都是大捷接着大捷,眼看着就能将西夏灭亡,将辽国击败,收复兴灵和燕云。

但这十年中,国中老是受灾。洪灾、旱灾和蝗灾,彗星、地震还有山崩,接二连三的灾异,总是让人觉得是不是这个年号哪里犯了冲,所以触了霉头。所以尽早改一个意头吉利一点的年号,也好迎来几个风调雨顺的好年景。尽管这样的想法很是无稽,但实际上也是无奈之下的企盼。

而王安石现在却并没有在考虑着更改年号之类的事务,他眼下连宫中都有几天没有去了。去年送走了长子,今年又走了弟弟王安国,王安石颓丧不已,他的亲眷已经不剩多少了。

王旖换了一身素白的衣服,在内间帮着接待亲友家的女眷,脚都没有停下来的时候。几天下来,连伤心带疲惫,脸色变得有些憔悴,眼圈下也是两抹疲劳过度的黛色。

不过头七过后,这一份差事,也算是告一段落。与母亲和大姐一起返回相府,王旖在摇晃的车厢中昏昏欲睡,累的够呛。

等回到家中,却见到两名二十上下的年轻人,正从王安石的书房中千恩万谢的出来。

“是侯叔献家的两个儿子。”

王旖不清楚侯叔献的家人,王旁却是认识他们。侯叔献在的时候,也有过一番往来。

“来找爹爹的,究竟是有什么样的事?”王旖略带好奇的问道。

“多半是来道谢的。”

侯叔献早死,年初时因染疾而一命呜呼。在他死后,他的续弦不安于室,还在丧期就开始勾勾搭搭的,很是坏了侯叔献的名声。侯叔献的两个儿子偷偷告到了王安石这边来——他们不敢告官,以子论母,不论有理无理,都是死罪——王安石因为旧年开河之事,对侯叔献有一份愧疚,直接就将侯叔献的未亡人断回了娘家。

外面都说侯叔献是死后休妻,但侯叔献的儿子对王安石感激涕零,若非王安石,他们不知还要受多少辱。所以还特地过来,向王安石道谢。

王旖和王旁联袂进了书房中,王安石正在看着桌上的一本装订粗糙的小册子,里面应该就是他们要找的这一本。

“金陵陈迹老莓苔,南北游人自往来,最忆春风石城坞,家家桃杏过墙开。”王旖瞥了一眼,知道这是熙宁六年,变法受到最多攻击的时候,王安石所写的绝句。这时候拿出来,却更为应景,“爹爹难道是打算要辞相了?”

王安石摇摇头,却没有吭声。但王旖说得并没有错,他的确是还有辞相南归的打算了。

如今朝中的大事小事上,天子独断专行的倾向越来越严重。王安石在政事上的许多意见,有很多都没有被采纳。尤其是人事安排,但凡倾向

这样的态度,让王安石平添了一分归意。

翻翻自己在京执政的这些年所写的诗词,从意气风发,到如今的无奈思归,完整的展示了他几年来的劲旅,身心皆是为此而疲惫不堪。

‘丈夫出处非无意,猿鹤从来不自知。’,这是王安石放弃了在江宁的生活,终于在当今天子的征召下上京任官时,对友人劝谏的回复。

那时的意气风发,在十年的执政过程中,已是荡然无存;而踌躇满志的心境,也消磨殆尽。

今日若以元日为题,却不会再有‘爆竹声中一岁除’‘总把新桃换旧符’的慷慨激昂。

王安石已经厌倦了朝堂上的争斗,早就开始想着放开一切,辞任返回江宁。

就是如今住在家中的二女儿,让王安石不知该怎么办。自己若是辞相,女儿又该去哪里住?不可能回旧宅住下来,没有一个主心骨,这样的全是女子的宅院,麻烦事最多。

韩冈还留在广西,因为年幼的子女需要照顾,同时也经不起车船劳顿,王旖她们也不能去广西与丈夫团聚。虽然女儿什么没有说,但王安石是知道王旖希望韩冈能回到京师,若是不成,至少可以北面一点。

也许在自己辞相之前,当设法将女婿韩冈从广西调回来。京师应当不可能了,但更近一点的地方,应该不算很难。

翻手将自己的诗文小集收了起来,王安石坐着又发起呆来,没有与上来收拾书桌的女儿的说话。

若在几年前,王安石连发呆的时间都不会有,往来不断的访客能让他的书房始终保持着客满的状态,总是热闹非凡。

而眼下随着吕惠卿和章惇的先后成为执政,王安石的书房虽然不能说是自此门可罗雀,但宾客人数大减,却是不争的事实。

宰执之间,为防结党之议,私下里都是尽量少有往来。吕惠卿升任参知政事之后,几年来,上门拜访的次数屈指可数。而章惇进了西府后,也没来过几次。

没了吕惠卿和章惇,王安石身边其实还有些幕僚和助手,但他们的地位不高,能力也不强,能起到的作用很是有限。

幸好已经不是变法制度风雨飘摇的那些年,因为各方的势力已经在眼下取得初步的平衡,而新法的成效也是有目共睹。王安石不必担心自己离开后,会对新法的事业产生什么样的负面影响。

只是该怎么离开,在何时离开,这些都还真是要让人破费思量。

