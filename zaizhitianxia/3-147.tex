\section{第39章 苦心难成事(下)}

熙宁七年十月初五,王安石卸下同中书门下平章事并监修国史的身份,出知江宁府。

而本官从礼部侍郎连晋九级,被擢为礼部尚书,以资政殿大学士的身份成为了前任宰相。

王安石独相数载,他如今辞位,宰相之位不能空悬,必然得有人出来接替。

所有人都望着学士院。不论是开封、洛阳,还是大名、相州,也都是在屏声静气,等着天子的御驾来到内东门小殿。

依照多少年来的惯例,每当朝堂大拜除之时,不论是宣麻拜相,还是准备册封太子,天子的御驾都会驾临内东门小殿,在殿中向翰林学士口述自己的旨意。同时负责草诏的翰林学士所居的学士院都要锁院,以防消息走漏。

东京城的大街小巷,早在王安石开始递上辞章的时候,就开始讨论究竟是谁来接手这个一人之下、万人之上,礼绝百僚、群臣避道的位置。

“是冯当世【冯京】?还是王禹玉【王珪】?又或是吴冲卿【吴充】?”

秦失其鹿,天下共逐之。当王安石放弃了他的宰相之位,政事堂和崇文馆里的最高位置就此虚悬,朝中的两位参知政事,还有一名枢密使,皆有资格问鼎此位。

一人反问:“陈旸叔【陈升之】曾任宰相,他在枢密院的位置还在吴冲卿之上。怎么他不能做?”

“也有可能是洛阳、大名的那几位。北虏虎视眈眈,国中板荡,必须要有元老重臣来镇守朝局。”

“要是韩、富、文等人回来,新法可就完了。”这是幸灾乐祸的声音。

“谁支持新法,天子会让谁上来。谁能让朝廷财计稳定,天子会用谁。冯、王、吴、陈,还有几位元老,可有一个支持新法,他们上来之后,又有谁能有办法弥补朝廷亏空?如果不能,那多余的支出又要从哪里削减?废掉新法的亏空,少说都要一两千万贯,当年要有人有这个本事,也不会是王介甫上台来……当真以为新法能废不成?!”

有人在樊楼之中如此说道,闻者纷纷嗤之以鼻,以为狂生。王安石都下台了,新党如何还能盘踞在朝堂之中。想想范仲淹,他一离开朝堂去了陕西,吕夷简就立刻开始反扑,最后将新政一党一网打尽。

但结果很快就出来,就在天子准了王安石的辞章之后的第二天夜中,御驾来到了内东门小殿,学士院的大门紧锁,玉堂周围被着甲持戈的班直护卫,围得水泄不通。

到了第二天的清晨,在宣德门处张榜而出的白麻纸上写就的名字,既不在如今的政事堂内,也不在西府枢密院中,更不是远在西京、北京的一干元老重臣,而是知河阳府韩绛。

曾为首相,却因横山攻略的失败而失去相位的韩绛韩子华,终于在沉寂了数年之后,从朝堂之外杀了回来。

此份诏书,大大出乎世人意料,使得东京城中的议论,一时没有了声息。

紧接着执政的班列中,也添了一人。翰林学士吕惠卿升任参知政事,本为从七品右正言的本官官阶,也因这项任命,自动迁转为从四品的右谏议大夫——参知政事一职,六品七品都能担任,而一旦升任之后,本官就会立刻升迁到从四品这一级上。

连续两项任命,给了所有正在因王安石的辞相而兴奋的旧党们当头一棒,天子依然主张变法,依然还是支持新法,依然要让新党居于九重之上。

将自己的心意昭示所有朝臣之后,赵顼重又驾临内东门小殿,学士院锁院如昨。那一天,政事堂中再添了一名宰相。这名宰相是从政事堂中升任而来,不过不是王珪,而是冯京。

赵顼无意让韩绛独相,做了天子七八年,异论相搅的手段他越用越是娴熟。

始终支持新法的韩绛,对新法表面上态度暧昧、而实际则一直反对的冯京,这两人相互牵制,天子也就可以稳稳地控制着朝堂。

“大事上一塌糊涂,也就在小事里做点文章。做了这么些年皇帝,想不到就学到了这么一点东西。”

白马县的提点司衙门,韩冈独坐在书房中冷笑着。因为对契丹的讹诈,吓得割地求和,他对赵顼的看法变得很多,越发的瞧不起。还没有兵临城下,就吓得这般模样,日后还能指望他北收燕云吗?难怪会有靖康之耻,赵家的子孙,看来都是一路货色!

但对赵顼的鄙视,他只会藏在心底,日后做事说话,他将会做得更加聪明。对天子的为人越是了解,韩冈也越能在知道该说什么、该做什么。

十月下旬,已经是天寒地冻,汴河水运停驶,而冰上的运输因为河冰尚未完全冻结,尚没有开始。

冬至将至,祭天大典上,天子依照惯例要大赦天下。韩冈作为府界提点,他的任务则是清查京府各县的刑狱,审核开封府中大赦的名单。

十天来,他已经跑了开封府东侧的好几个县,将狱中一干轻罪囚犯的名单连着判词都大略的看了一遍,其中有不少冤枉的,只不过因为他们都在大赦之列,韩冈就没有当场给指出来,只是暗暗记了一份名单,以用来日后清查。

陈留县的汴河码头便,韩冈半眯着昨夜熬了半宿、发干发涩的眼睛,对身边的王旁叹道:“谳狱清明四个字说着简单,做起来还真是难。”

王旁同样熬了一夜,眼中同样都是密布红丝,如同兔子一样。他听到韩冈的话,回头笑道:“县中的那些冤案,玉昆你不都是一眼就看出了破绽?你的眼光可比得上包孝肃,不让汉时于定国。”

“冬月请治谳,饮酒益精明。汉时宰相于曼倩【于定国】饮酒愈多,断狱愈明。纵然案情错综复杂,判断起来亦是举重若轻。于公之姿,仰之弥高,钻之弥坚,我可是远有不及。而包孝肃的清正刚直,更不是我能比的。”

“也差不了多少了。没看到这些天经过的几个县,那些知县都是战战兢兢的?将冤狱的文牍分开来摆,玉昆你尽管一句话都没说,他们心里还能不明白?!”

王旁一边说,一边却伸着脖子向北张望。

韩冈见及于此,笑着劝慰道:“岳父岳母应该快到了,不用太着急。”

韩冈他是府界提点,能在开封府内到处跑着。他出来清查各县刑狱,正好撞上王安石离京前往江宁府,理所当然的要出来送上一程。他回头看看身后幕帘深垂的马车,王旖抱着才刚刚满月的儿子就在车中。

王旁随着韩冈,在提点司做得正是得意的时候,并不打算跟着父母一起南下江宁,所以今天是跟着妹妹一起来给王安石送行。

不过王雱则是要一起南下,虽然辞了侍讲一职,但他还在经义局中有一个位置。

王安石照旧提举经义局,这也是天子赵顼依然主张变法的明证之一。王安石、王雱,还有王安石特旨请来的熙宁六年的状元余中,他们将在江宁府继续编订三经新义,为朝廷取士给出一部答案明确的教科书来。

而且天子对于王安石还是有着一份感情,昭命王安石出入如二府之仪,大朝会列入宰相班列。所以从北面远处,远远的看到了一行穿着红色元随服饰的旗牌手,韩冈就知道他的岳父来了。

王安石带着老妻吴氏,还有王雱一家——王旁的妻子庞氏则是已经到了白马县——以及几十个仆役婢女,这就是宰相南下的全部人数。外面的一群护送他南下的队伍,到了江宁府,以他的性子差不多就要慢慢解散了。

见到韩冈带着女儿、外孙来相送,王安石夫妻喜出望外。

王安石见着韩冈,半句不谈朝堂政事,只是开开心心的逗着外孙。吴氏则是抹着泪水,与二女儿在一边说着话。

只有王雱拉着韩冈和弟弟在一边说话:“天子要富国强兵,此意不会轻更。玉昆、二哥还是用心做事,不必担忧后事。”

韩冈点着头,这是应有之理。

王雱回望京师,长叹道:“只望天子能知耻而后勇,日后不再有今日之事。”

韩冈同样叹道:“就怕物极而反,日后变得一意进取而不知守中之道,而执政则推波助澜。”

说是一个时代结束了未免夸张了点,但说如今的朝局将会从明确走向未知,则是可以确定。

王安石名垂朝野,德隆望重,有他在,新党不论遇到多少风浪,终究还是能保持着基本的稳定,能压制着。而如今的韩绛,他虽是宰相之尊,但他在新党中的发言权却不如吕惠卿。

而以吕惠卿——不,应该说以所有继承人的心思——都不会将前任的政策全盘接受下来,萧规曹随的度量,韩冈不觉得吕惠卿会有,而新官上任三把火的想法,应该正在吕惠卿脑中转着。

“终究不会大的更改,如今诸法,绝大多数吕吉甫当年都有参与审定,并不全然是曾布的功劳。”王安石微笑着,终于为此说了一句。

送别千里,终有尽时。韩冈夫妻一路送了王安石二十多里,终于停了下来。

驻足于汴河之滨,目送着前任宰相一行车马,向着南方辘辘远去。

