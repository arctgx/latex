\section{第19章 萧萧马鸣乱真伪(七)}

“元长,这些年可是一切安好?”

强渊明刚刚抵达京城,第一件事就是宴请现任流内铨主簿的蔡京。

蔡京举起酒杯,“隐季,想不到数载不见,风采依旧。”

“远不如元长你能纵马而行。”强渊明的话中隐隐带着一分羡慕,还有嫉妒,只是旧日的交情,让他倒不用顾及言辞太多。

两人是同时考中的进士,强渊明如今还在选海中沉浮,但蔡京已经是京官了。自熙宁三年考中进士之后,蔡京只做了六年选人便顺利转官,这个速度其实已经算是骑着马在跑了。

“愚兄运道好而已,遇上了熊伯通。”蔡京微微一笑,俊逸的笑容让刚刚走进来准备唱曲儿的妓女一下看得呆住了。

这些年,大宋一直都是在开疆拓土之中,在三个方向上都取得了成功。

一个是西北由王韶主导的河湟拓边,这是最见成效的典范;一个是南方,章惇掌管的荆南,不仅平定了荆南山蛮,由此锻炼出来的兵员还轻松无比将十万交趾大军给打了回去;最后一个就是在西南,由熊本熊伯通主持的针对西南夷的开拓行动。

相比起王韶和章惇,熊本在名气和官位上都要低上一些,功绩也是逊色。不过蔡京却要承熊本的人情,他在舒州团练推官的任上,得了来巡察地方的熊本的赞许,言其学行纯茂,不久之后就脱离了夔州路那个苦地方,回到了京中。现在蔡京是流内铨主簿。尽管职位依然不高,但所在的位置极其重要,是管着选人差遣的职司。

强渊明要请蔡京,自然在共叙旧谊的同时,还有另外一份想法。如今流内铨的阙亭外,一众待选官员对于蔡京这位主簿,虽不能说点头哈腰,却也是恭谨有加。而那等不请自来,在桌前打酒坐、唱曲儿的妓女,也在拿着眼睛瞟着蔡京——尽管是另外一个原因。

长得稍微有那点不如人意、还在选海中打滚的强渊明,心中一时无聊得紧,向着楼下的街道左右看着。忽然就看见一队骑手拥着一位穿戴着三品服饰的官员,跨马从楼下而过。旁边另有一人并辔随行,定睛看过去,却是强渊明曾经远远的见过几面,是王安石家的衙内王旁。

王雱刚过世不久,王旁眼下还在服中。但他现在却替下了麻衣而换上了吉服,这究竟是为了为什么原因,只看他所陪伴的一群人,理由不问可知。且年纪轻轻便身着紫袍犀带、腰间系着金鱼袋,这自然不会是他人,朝中的文臣中数来数去也就只有这么一个异数。

“元长,那一位可是韩玉昆?”强渊明指着楼下的紫袍显贵。

蔡京一低头,在看见了王安石次子的同时,旁边的一人同样映入他的眼帘。旧年在西太一宫中擦身而过的记忆一下又清晰了起来,“正是他!正是韩玉昆!”

让守在包厢外的小厮,拿钱请了唱曲儿的妓女出去,蔡京道:“他这是要回广西了。”

“关于安南之事,元长应该也听说了罢?”强渊明凑近了,声音也压得低了。

蔡京点了点头,端起酒杯润了润喉咙。契丹助战丰州,这是今天早朝之后从宫里传出来的。契丹既然不稳,河北军就不能轻动。

强渊明笑道:“韩玉昆算计来算计去,就没算到契丹人竟然会出兵援助西夏。眼下北方诸路的兵力都不能随意调动,能南下的西军,压下就只有五千。”

“千五便能破十万,五千难道不能扫平升龙府?”蔡京反问着。

“那是交趾军已经在邕州城下做了疲兵,功劳是坚守邕州的苏缄和说服广源蛮帅的苏子元,并不全然是韩冈之能。只有区区五千西军,加上荆南军的一千五百,剩下的都是些滥竽充数之辈了。”

“隐季你觉得韩玉昆做不到?”蔡京慢悠悠的反问着。

强渊明张了张嘴,却没敢说不行。经过了这么些年,又发生了那么多事,现在再没人能小瞧韩冈,“直学士、转运使,要是做不到,朝廷还真是白提拔他了。”

蔡京向楼外望去,韩冈和王旁已经走远了。

韩冈的年纪比蔡京小上五岁。可韩冈如今已经是直学士。蔡京自知他想要攀到那个位置上,即便有着过人一等的运道,差不多也要十几二十年的功夫,说不嫉妒那是不可能的。不过蔡京是个极为现实的人,并没有想过与韩冈过不去的打算,也绝不会将自己心中的嫉脱口而出。

强渊明知道蔡京的脾性,也不再多说什么,转过去问道,“韩冈之父前几天不是说已经抵达京城,是不是由元长你管着?”

蔡京淡然一笑:“听说熙河六州,这几年新辟的田地有万顷之多,开辟的沟渠加起来长达千里,灌溉了数十万亩田地。虽说其中必然有所夸大,但熙河路自给自足已经有两三年了,这点却做不了假。岷州的滔山监在造铁钱,钱粮如今都能在路中自备。且熙河路也不是光有钱粮,那里的特产,想必隐季你也是知道的,”

“河西吉贝。”强渊明如何不知。

“这都是韩冈之父的功劳,熙河路各家靠了这吉贝布赚了不知多少钱去,王韶、高遵裕、韩冈可都掺和在里面。”蔡京叹道,“现如今听说京城几个行会要学着熙河路措办蹴鞠联赛,领头的棉行脱手就是几万贯砸下来,买了场子、买了房子。熙河的富户往少里说都是数十万贯的身家。”

强渊明明白了蔡京想说些什么:“想不到韩冈父子不仅官运亨通,而且财运也一样亨通。”

“开辟田亩百万亩,大小沟渠千余里,韩谦益当然能从选人转为京官,只是最近听闻他辞了审官东院的新任命,自称老病不能任事,打算告老归乡。”

“有着韩冈这个儿子,回家做个老封翁,自是要比在外风吹雨淋强。”

“开田地,兴沟洫,若是福建也能做得如熙河一般容易就好了。”蔡京不知怎么突然间就感慨了起来。

“元长你说的是木兰陂吧?”身为蔡京的知交,强渊明如何猜不到蔡京现在在说哪一桩事,“钱四娘自尽了、林从世也败光了家业,不知道元长你说动的李宏到底能不能将木兰陂给修起来?”

蔡京是福建兴化军仙游县人。兴化军有木兰山,一条河流从木兰山中流出,称为大目溪,也称木兰溪。这木兰溪只有数百里长,却因发源自高处,而溪流湍急。山上一场大雨落下,木兰溪就会暴涨而导致洪水泛滥。到了海潮起时,海水又会上侵,沿着木兰溪上溯到接近仙游县城,将河道两边的田地一起变成盐卤之地。

春夏有洪水,秋来有海潮,兴华军的灾难一年年的延续下来,多少人想为此修起一条海堤,拦住海水;修起堰坝,使洪水不再泛滥——这就是为何要修木兰陂的原因。

蔡京也想修木兰陂,此前蔡襄在泉州修洛阳万安桥,就是给了他一个启发。

此前修陂两次失败,并不是全然无功,蔡京从他收到的李宏的来信中知悉,李宏和他的助手冯智日已经找出了前两次失败的关键,且为此而选择了另外一处溪流浅缓、海潮难至且河底由石块组成的位置。

只是木兰陂的修造实在太难,此前两次都是毁于一旦,李宏是蔡京深沉恳切的给请来的,随身带着七八万贯作为资金,只是一年不到的时间,这些钱已经都给用光了。

“木兰溪水势太急,此前钱四娘、林从世两次失败,都是没有将堰坝给筑好,最后被水给冲垮。李宏前日写信来。信中说他已经与几名大工匠合计出了该如何修筑堰坝,而不至于被冲毁。只是要将堰坝筑起来,至少要用到千斤巨石三万到五万块,单是从山里采来一块运到木兰溪边,就是一贯多。仅仅是买这些石料,就花光了李宏手上所有的钱钞。修堰坝还要人工,修堤也要人工,此外还要开沟渠、洗盐卤,至少还要六七十万贯。”

“六七十万贯?!”强渊明吃了一惊,这可是个大数目,“元长你可能再筹到?”

“难啊!”蔡京摇头叹气,“愚兄昨日已经上本,要请天子给木兰陂一个名分,这样才好说动更多的人来。”

相比起仙游蔡氏的其他几房,蔡京的这一脉只能算是寻常的富户,出不了多少钱。不过蔡京在家乡颇有盛名,在钱塘尉的任上,就说动了李宏出头继续修造木兰陂。不过眼下钱没了,要向将这件事继续熬下去,只能靠着乡里。

“不过也不能指望太多……”强渊明也跟着叹了起来。六七十万贯,足以让几家肯出钱的富户倾家荡产。

“其实说多也不算多。据说在熙河路中的蹴鞠联赛上,各家为争一个头名,就连蕃部都是一掷千万,下注赌胜的则为数更多。”蔡京眼眸的颜色几乎都变成了铜钱色的深黯,“如果能从中抽头,想必官中的收入绝不会少。”

强渊明眨了眨眼睛:“……元长你是打算在福建开蹴鞠联赛?!”

蔡京洒然一笑,露出的一排白牙似乎在闪光:“不觉得有用吗?试想熙河开田、开渠,给流民牛马农具的钱是从哪里来的?”

“诱人赌博,有伤风化,御史……”

强渊明话到嘴边,看见蔡京似笑非笑,顿时恍然,蔡京他如何会自己上本?他那张嘴皮子能将死的说活过来,诳来一个傻瓜为控制蹴鞠联赛从中牟利来鼓吹,最后坏了事,也能将自己摘出去。不过此事想必会有不少人感兴趣,再如京城中的几家瓦子,里面斗鸡斗狗、相扑摔跤,可是有着开封府的背景。

‘或许,还真能给他成功!’

