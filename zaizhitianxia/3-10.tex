\section{第四章 秋来暮色寒(下)}

快到九月的开封城,正是一年中少有的好时节。

此时的气温不高不低,阳光柔柔和和,瓦蓝瓦蓝的天空上,几朵白云更映出来蓝天的高广。

没有春天的浮灰沙尘,也没有夏日的酷热难耐,滴水成冰的寒冬更是不能相比。

王旁不想将闲暇时光变成义务劳动,这样的日子,随便找些人喝酒纵乐都很容易。但他身后还跟着自家的妹妹,总不能呼朋唤友的去喝酒。只能陪着很少能出府的王旖,在开封城中的几条最为繁华的商业街中,留下了自己的脚步。

“这位官人,可要一支糖渍林檎?糖料和果子都是最上等的。”

不知什么时候王旁和王旖已经站到了买者糖渍菓子的小贩身前。

看了看后面向自己连连点头的妹妹。就听见王旖正在劝说着他:“二哥,给二嫂也带一支回去吧……二嫂现在正好喜欢酸酸甜甜的东西”

应了一声,王旁让小贩将之处理好,准备包起来带走。

王旁现在很闲,所以才能陪着妹妹逛街。

父亲和大哥日日忙于公务,虽然自己在父亲成了宰相后,得了荫补,也有个京官官身——正九品的太常寺太祝。但还跟自家的妹妹一般清闲。

荫补官要到二十五岁才能出来受差遣,王旁还要闲上两年。要想提前出来做事,要么考上一个进士,要么就是要得到了天子的特旨。

岂能人人都如韩玉昆?而跟轻松考上进士的大哥相比,王旁则更是自惭形秽。

王厚回头看了自家妹妹一眼,要是韩冈当真成了自家的妹夫,再加上一个大哥,那他在家中,还真是没处站了。

小贩将王厚要的东西递了过来,“四支糖渍林檎串,该收官人二十四文钱!”

王旁正往袖口掏钱袋的手一下定住了:“原来不是四文一支吗?”虽然这点小钱他不可能在乎,但被人欺骗,他可不干。

“再过几天林檎果就改成官卖了,这价格当然要涨上去。”小贩理直气壮。

“菓子怎么可能改成官卖?”王旁摇头不信,“就算要改官卖,可市易法还没有推行呢,怎么物价就涨得这么厉害?”

“这事小人哪里知道?!”小贩不快的说着,“小人只知道腌渍用的糖贵了,这林檎果也贵了。小人也要吃饭,也要养家,只能随行就市涨上了一涨了!”

“这是怎么回事?”王旁纳闷起来。

“还不是王相公闹的。”坐在一旁的一个闲汉突然插话进来,“把个青苗贷掩耳盗铃的改个便民贷的名字,这个笑话就不提了。闹个保甲法,乡中到了冬天都不见一个消停。现在又是市易法,钱全给官府赚了去,还给不给我们小民活路啦!”

闲汉身边,他的一个同伴立刻捂上他的嘴:“小心一点,有皇城司的人!”

王旁回头与王旖对视一眼,兄妹两人的脸色都已经变得苍白。

道路以目、民怨沸腾,诸如此类的成语,走马灯一般的在二人的脑中流过。

“怎么会变成这样?!”

……………………

同样的问题也出现在赵顼的心中,市易法还没有正式开始实行,就已经让京城为之骚然,如果正式开始推行,情况又会变得怎样?

据吕嘉问所言,市易法的目的虽然是聚敛不假,但抢夺的是各大行会行首们攥在手中的定价和发卖的权力。市易法的推行,将会把豪商的利益转移到官府手中,并不会影响普通百姓和小商贩的生活。

如此,才得到赵顼的首肯。

可是眼下事情的发展,却已经偏出了吕嘉问事前向天子作出的预计。

依照冯京的说法,这是民间听说了官府要将所有商货都买走,使得京中人心惶惶,所以货价一涨再涨。

而王安石那边的吕嘉问则说,这是京城的奸商们为了不让市易法推行,故意散布谣言,致使市井恐慌。也就是说,现在是各方行首正在串联起来,一起抬高物价,以煽动民怨来对抗朝廷即将实施的市易法。

但吕嘉问的这个指责太过于诛心,赵顼都不敢去相信。

一旦他相信了这个指责,下旨让开封府和御史台去穷治。势必会变成牵连几十家甚至上百家京中豪商的大案。而豪商跟宗亲的联姻,赵顼一清二楚,如果他真的如此下旨,几千宗室,差不多都要到他的面前哭丧。

“王卿,你说着市易法推行还是不推行?”

“箭在弦上,焉能不发?”

王雱虽然回答得痛快,但他仍是为着市易法之事而头疼。

市易法提出已经快有一年了,但为了能够顺利推行此法,王安石让人进行了几次三番的考验。卷宗来回反复。但始终没能达成一致。虽然已经确定到了十月就正式开始推行——这也是天子的恩德。因为十月过后,天气转寒,汴河上就要堵口,大宗货物自此还能再赚上半年的钱。明年开春之后,才会变成钱财向衙门中流去。

但终究还是闹出事来了。原本因为河湟大捷而带来的光环,如今已经散去。朝臣们现在都知道,这段时间以来,为了市易法一事,太皇太后和太后都是三番四次的劝过天子。

赵顼咬着牙,对祖母和母亲的要求,绝不松口。

而王雱也知道,只要一步后退,那就步步后退。

仁宗当年三司衙门之中,冗官多,而冗吏更多。宰相杜衍奉旨沙汰三司吏,但听说了此事的三司吏,立刻扩散谣言,将沙汰胥吏的范围一举扩大,一下惹怒了许多衙门中的吏员。这些胥吏群起而攻,最后逼得杜衍在京中坐不住,只能自请出外。在王雱看来,杜衍若是一意孤行下去,将领头的抓起来严加处置,也不会落到出外的结局。

杜衍的结果,让人引以为戒。

大概是知道从王雱口中得不到没有偏向的执中之论,赵顼也就无意追问下去。而是随口问道,“京中解试的情况如何了?”

“过几日就该张榜了。无论是在开封府监考的张商英、蒲宗孟。还是在国子监监考的张琥、李定,他们是现在都在连夜批阅考卷,不会误了发榜的时间。”

张商英、钱藻等五人考试开封府举人,張琥、李定等六人考试国子监举人。以考生到贡生的录取比例而言,开封府和国子监跟陕西都差不多,远远要强于浙江、福建的军州那百分之一、两百分之一。

“秦凤路的解试也当有结果了。”赵顼却叹了一口气。

王雱顿了一下,以路为主体的考试,那就是锁厅试了。他反问道:“……陛下想说的可是韩冈?”

“恩。朕还想看看韩冈到底有多少的才学。”赵顼点了点头,却又笑道:“韩冈好象是一直都不肯承认是药王弟子,但现在他连救治妇人难产的手段都拿出来了,孙思邈徒弟的这个身份,怕是要坐实了。”

王雱的惊讶写在脸上:“竟有此事?!”

“高遵裕的妾室前日生产时产难,一夜不出,要不是韩冈让人造了产钳,钳出了腹中小儿,多半就是一尸两命的结果。走马承受传回的月报中有提及此事。高遵裕发回的私信中,也是说了一通。不会有假的!”

“……臣闻孙思邈所著《千金方》中,就有妇人科三卷。既然研习医术,小儿和妇人两科,自是不能避过。”

管接生的那是三姑六婆中的稳婆。听赵顼的口气,他在此事中还是很欣赏韩冈,但王雱却不喜欢。虽然帮着韩冈说话,但王雱却总觉得韩冈做得过头了。

“救了两条性命。救人一命,胜造七级浮屠。他所发明的产钳若是能救天下难产的孕妇,就不知胜过了多少座佛塔。”

赵顼沙沙的踏着落叶,在一片红黄之色的小树林中漫步者。作为天子,他看重的自然是人丁户口。一个产钳,就能拯救无数条危难中的性命,等于是保住了原本会损失掉的人口。

若说起妇人的秽体五漏之身,以产妇为最,民间对此都有避讳。但医者父母心,扁鹊,华佗,这些上古名医的传说中,也都有救治难产妇人的故事。

韩冈的所作所为,赵顼自是乐见。

……………………

秋色渐浓,望着不远处山中的黄栌和枫树,已经让人感觉到了浓浓的深秋带来的寒意。

韩冈此时已经回到陇西的家中。并带着他已经成为今科贡生的证明。

靠着这份文书,韩冈直接就能在官府的驿站里得到一般官员等级的照顾。而当他前往大宋的中心时,也同样能得到一般的礼遇。

韩冈并不在乎这一些明面上的优待。过去他吃得苦头从来不少,恨的是权力被人分走,而有没有礼遇反而不重要。

他现在心急的是另外一桩事。

为了考试,韩冈已经习惯于不见外客,但周围的人众却一直他等下来。

韩冈中了贡生回来,这就意味他终于等到了期待已久的日子了。

