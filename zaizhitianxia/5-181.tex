\section{第20章 土中骨石千载迷(十)}
暮色降临,韩家的正厅中灯火通明,一向不喜游宴的韩冈难得的设宴待客,虽然宴席上没有伎乐,却也足够热闹。

韩冈多年来京内京外任职多处,推荐了不少官员出来,而在韩冈府上,也养着十几名门客,加上气学的门人弟子,为数更是众多。不过能当得起韩冈设宴接风洗尘的,也就是寥寥数人,黄裳便是其中之一。
在韩冈自河东任上调任太常寺之后,黄裳也辞了在河东的差遣,不过他并没有立刻跟着韩冈回京城,而是先回了家乡一趟。到了快入冬的时候,才回到京城。韩冈一向看重黄裳,待到他入京,便摆下宴席,为其接风洗尘。

酒宴之后,韩冈又在房中招待了黄裳,端着茶,坐下来说话。

半年不见,黄裳黑瘦了一点。从河东到福建,再从福建进京,奔波万里,外形上有这样的变化也正常。不过看着精神得很。

“这一次勉仲进京,是不是一直待到两年后的进士科?”韩冈问着黄裳。

黄裳点点头,道:“其实只有一年半了。离解试更是只有一年。时不我待啊。”

“说得也是,的确没多久了。”看来黄裳在考试前,是不准备候阙出来做事了,要专心致志的准备科举,韩冈笑道,“不过以勉仲之材,厚积薄发,今科定然是能高中的。”

“多谢龙图吉言。”黄裳低头谢了韩冈。

坐着喝了杯茶,韩冈问着黄裳:“勉仲这一次回乡,一路上所见福建和江南今秋的收成如何?”

“今年风调雨顺,又是丰年,各路皆是稻谷满仓。就是福建,只靠广西海运来的六十万石稻米,一路的在粮食上的亏空也弥补上了,此乃端明之功。”黄裳先说了两句好话,“不过就担心谷贱伤农,今年各处的常平仓已经都收满了,明年若还是丰收,粮价肯定要大跌了……其实今年江南的酒价已经跌了三成还多。”

“三成?怎么这么多?”

酿酒靠的是粮食,荒年粮食少,酒****,丰年粮食多,酒价跌,这是正常的。但丰年喝酒的人也多,这样的年景,酒价一下跌下来三成,这个数目未免就多了些。

韩冈也有些头疼,明年要还是丰年,粮价必然是要跌的。最好的办法,是兴修水利或是交通等工役,消耗一部分钱粮,以稳定明年的粮价。税赋收上来就是该花的,要是学着文景之治,粮食烂在仓库里,串钱的索子一并朽烂,那就太过浪费。以现在的存储水平,四五年后的稻米早就发黑霉烂了,保证有三年之积就已经足够了。

只是这个问题,只能让天子和政事堂去头疼了,韩冈处在现在的位置上,却是连一句话都插不上,没资格去干预,正经是将现在的工作做好才是。

黄裳也知道韩冈现在的职位在这些事上插不上嘴,也不再多提,道:“上京过金陵的时候,黄裳顺道拜见了介甫相公一面,也带了信回来。”

韩冈前面已经听说了黄裳去了半山园,黄裳是韩冈的门客,从河东南下时,韩冈顺便就托他给王安石带了信和礼物。不过主要还是将黄裳介绍给王安石。通过顺丰行和自家的人手,韩冈与王安石之间的信函,基本上两个月就能联系上一次,用不着借外人之手来通信。但他没想到黄裳回程的时候又去了半山园拜访了一趟。

“家岳说了什么?”

“介甫相公只是与黄裳谈了些解字上的话题。”黄裳回道。

“如何?”

“介甫相公这几年佛经读得多了……”黄裳摇摇头,“解字又多不合古意。”

韩冈神色一动:“《字说》和殷墟之事,勉仲你是不是已经听说了?”

“在南京的驿馆中听说了。”黄裳沉声道,“端明编纂《药典》,正好收到相州的甲骨,真乃是天意了。”

“时运而已。”韩冈笑了一笑,将家中留存的几块甲骨拿出来展示给黄裳,“更多的还在编修局中,勉仲若有雅兴,可以往编修局一行……就在太常寺中。”

黄裳现在已经是以气学门徒自居,拿着甲骨文眯着眼睛看了好一阵,才放了下来。对韩冈道:“不是端明,真不会有几人能注意到。有些见识的士大夫,又有谁会去检视药材。”顿了一下,又道“听说已经有不少元老上请天子早日决定发掘殷墟,”

“上的人是不少,不过天子还没有下定决心。”

请求发掘殷墟的老臣越来越多了施行新法的优点,在西夏灭亡之后,已经为天下大多数士人所认同,更让天子坚定了百倍的信心。由此一来,想动摇新法,完全不切实际。已经远离朝堂十余年的一干老臣,根本不可能有多少机会来攻击新法。若是老调重弹,说什么民怨,这几年的天下各路大丰收,也能让他们的老脸都丢尽。韩冈眼下给予他们的机会,可以说是多年来唯一的机会,就是只为一泄旧怨,他们也不会放过,而且又不是反对新法,天子也是无可奈何。

所以黄裳笑道:““再拖也拖不了多久的。”

“的确拖不了多久,再过几天,消息遍传天下,恐怕长安、洛阳的盗墓贼全都要往安阳去了。”韩冈忽的低声笑道。

“那韩忠献岂不是难以安生了?”

“应该不至于。”再大的胆子也不至于有人敢在太岁头上动土。韩琦才死几年?朝廷和后代都有人看着,“不过也不排除韩家拿此事当借口来反对发掘殷墟……毕竟那是在安阳。”

“韩忠献家会反对?”

“韩家的产业半相州,当然不会愿意看到朝廷在他家的田地里面挖坑。毕竟那是殷墟,不是一座两座的古墓,而是两千年前的一座都城。一旦朝廷决定发掘殷墟,韩家的损失将不在少数。”

韩琦出身安阳,又相三朝、立二帝,原本官员不得在本籍任官的规矩,都为他破例了四次。等到韩琦在判相州的任上病逝,接任的相州知州姓韩名正彦,正是韩琦的侄儿——之所以没让儿子来接任,那是因为要守孝三载的缘故——对于韩琦一家,几任天子都是给足了面子。

相州田地有三成——而且是最好的那三成——是韩家的,相州各县的店铺有一半跟韩琦家脱不开关系,不过这些产业大部分用了诡名寄产的手段,寄托在了他人的名下,所以看起来不是那么扎眼。只是这等情报,根本不用费神去查,到相州的酒楼茶肆坐一坐,随便打听一下就能知道了。韩冈本来以为韩琦的儿子、女婿会来找自己,但这些天下来,一直都没有动静。

韩冈继续道:“若是发掘殷墟,韩忠献家多半是反对的。一旦韩家上表说此举惊扰先人,天子或许会顺水推舟也说不定。对韩家来说,佃租的损失还是小事,万一有人首告韩家私藏殷商天子祭器,那就是黄泥落裤裆,罪名就算能洗脱,至少也要脱一层皮。而且树大有枯枝,相州韩家家大业大,人口繁多,不肖子孙不在少数,若是出了一个贪财好利的,能将一族上下千百口人都拖累进去。”

虽然韩家的反对声几乎是必然的,但韩冈对此并不在意。

韩家人丁旺盛,虽然相州那么多的产业都是韩家的,但上上下下靠着韩家吃饭的人也是个极为庞大的数目,又要维持着韩家的体面,每年的租税、贸易和放贷等收入,只能说勉强够用。韩家子弟要享受,做些不正当的买卖,也是免不了的。

当真以为安阳地里的那些古董千百年来都没有人发现?那是笑话。没被注意的是甲骨文,殷商铜器和陶器,早几百年就给挖出来了不少。韩冈派去相州的人,在搜集到占卜的甲骨之余,还收购了两件殷商青铜礼器,便是明证。

尽管顺丰行与韩琦家下面的商行没有什么来往,但雍商之中,与之作买卖的还是有那么两三家。透过他们,韩琦家的一些情况,韩冈了解得不少——也不仅是相州韩家,国内的一干豪门中有五六成的家底,韩冈都能做到心中有数,比起皇帝和官府都要清楚。

只要抓好了这个问题,就是韩琦复生也没办法解决,随着地里面掘出来的礼器越来越多,给予天子和相州韩家的回旋余地就越小,迟早的要对韩冈进行妥协。

一座城池中能发现的器物,成千上万,数也数不清出。相州韩家也不可能遮掩的住,随着时间推移,殷墟的名声将会越来越大,那时候发掘出来的殷墟遗物将会越来越多,这样的情况下,天子也没办法完全由着自己的心意来压制既成事实,并不一定需要司母戊大方鼎一样的证据。

不过若是当真能从地下将上千斤重的礼器给发掘出来,届时天下都会轰动,别说《字说》,就是天子也得低头。那可是比传国玉玺更古老的器物,放在太庙或是南郊祭天的场合,天子也是脸上有光。

在见证人遍及天下的时候,事实是无法抹杀的,天子的权力对此也无法施为。

