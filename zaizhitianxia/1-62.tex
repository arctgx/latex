\section{第28章 夜影憧憧寒光幽(四)}

赵隆和王舜臣都是在秦州城出了名的猛人。但不是亲眼看见,陈缉怎么也想不到,两人的武技竟然可怕这样的地步。才一接阵,辛辛苦苦找来的帮手瞬间就给他们杀了三分之一去,那可是横行秦州十几年的过山风的手下啊!有这样的两人守在韩冈身边,何谈报仇雪恨?!

留得青山在,不怕没柴烧,陈缉当机立断,而他的手下在黄家老大的带领下,紧追身后,一阵狼奔豕突。陈缉跑了两步,突然横里闪进一条巷道中。幸亏躲避得快,他刚刚闪身,一道流光就擦着他的耳尖飞过。尖啸声刺痛了陈缉的耳膜,而身后一声接一声的凄厉惨叫,让他根本不敢回顾。

竟然还有一人!

陈缉肝胆俱寒,听着身后接连不断的惨叫声,不知名的那人厮杀起来,竟然不比王舜臣和赵隆稍差。韩冈一个刚当上官的措大,哪儿来的那么多高手听他驱使?!身边跟着这些个与护翼天子的班直侍卫,都不相上下的好汉,韩冈所在下龙湾就跟龙潭虎穴一般,早知如此,他陈缉怎么会自投死路!

陈缉心中大恨,情报上的失误,让他只能像条狗一样的夹尾而逃!

陈缉逃了,陈缉的手下也逃了,可过山风还犹豫在上前拼命和逃跑的两难选择间。

铮铮弦鸣,又是两箭从后面的黑暗处射了出来。过山风吐气开声,腰刀用力一荡,格开了箭矢。身子却猛地一震,一支突如其来的长箭已经穿进了他的腰间。过山风一声怒吼,腰刀甩手砸向王舜臣和赵隆,自己捂着创口,转向另外一条路,向村口逃去。

“是谁的箭?”王厚垂手执弓,扭头问着韩冈。过山风中箭,而箭矢是他们两人同时射出,王厚没看清那一箭是谁的功劳。

韩冈叹了口气:“是王兄弟的。”他和王厚射出的两箭都被过山风格飞了,命中的一箭,是王舜臣射出来的。比起王舜臣,他和王厚的箭术还是差得太远。

‘王舜臣?!’王厚心中暗惊,他根本就没看到王舜臣动过手臂!

头领跑了,残存的贼寇跟着一起逃窜。韩冈又是一声大喝:“快追!莫要让几个小贼逃了!”

各家院门被打开,几个胆大的村人拿着家用的猎弓和长矛探出头来。贼人在哪?区区几个小贼,关西汉子可不会放在心上。

……………………

猎物低着头拼命的奔逃,猎手紧紧追在身后,这是陈缉最喜欢的狩猎运动。每到秋冬,他都会带着养在庄上的几条罗江犬,去山里狩猎,兔子,麂子还有山鸡,运气好时,还能撞上了冬眠的熊窝,扒下熊皮做件大衣。而更让他兴奋的游戏,是用得罪陈家的活人扮演的猎物,提着两条腿的猎物首级,让陈缉有着百战功成的成就感。

但今夜是陈缉第一次扮演着猎物的角色,惊慌失措得仿佛一只被十几条猎狗一起追逐的兔子。他终于体会到被追逐着的猎物心中那股绝望,完全没有希望和前路的深沉黑暗。

追逐声越来越响,陈缉奔逃中回头一望,身后火炬熊熊,几十道闪耀的火头映得雪地一片红光。自己孤伶伶跑在一片雪白的土地上,带出来的十几个手下,还有过山风一伙,都不见了踪影,只有黄家老大紧紧跟在身后。

怎么会这样?!

李癞子也是今天午后才得到消息,韩冈怎么会事先找来王舜臣和赵隆?难道他能掐会算不成?陈缉一边跑,一边胡思乱想。

对了!他只要能逃到村子东北的树林中就安全了,夜里不会有人敢追入林中!等到了白天,他早就能远走高飞。日后再聚集人手,来报今日之仇……

一声暴喝声震四野,若有若无的尖啸滑入耳内。陈缉还沉浸在日后复仇的幻想,没反应过来,一声死前的嘶喊声便在身后响起。他胆战心惊的侧头回望,一直紧跟着自己的黄大已扑到在地,一动不动,没有任何生息。背上一根短矛如战旗般骄傲的竖着,凛凛的向四周散发着杀气。

比凛冽的夜风还要冷上千百倍的冰寒从脚心直通头顶,把陈缉的五脏六腑一齐冻结。差一点的弓都射不到的距离上,用手抛出的标枪竟然能一击毙敌,这是何等的神技!

逃!逃!逃!

陈缉不敢再回头,用力迈开已无知觉的双腿,拼命的向前方逃去。他已经无法再去考虑逃路的方向,恐惧完全控制了他的心脏。心底只剩一个念头,那就是逃!

乾坤一掷,便将近五十步外地逃敌扎死在地上,跟着从村中杀出来的乡民一阵惊呼赞叹,但李信依然面无表情。他看着陈缉独自奔逃的背影,没有再追上前。

一阵狂风掠起,扎在李信头上的英雄巾在风中狂飞乱舞。赵隆骑着他那匹老马从李信身边一冲而过。马颈之下,一团黑影摇晃着,一股浓烈的腥气散入风中。李信动了动鼻子,这是他熟悉的味道——是被熟铜简敲碎了天灵盖后流出的脑浆,再混着血水的味道。

‘是过山风?’

李信猜测着。能让赵隆紧紧拴在身边的,只有陈缉和过山风两人的首级,黄家兄弟都不够资格。何况黄家老大躺在前面,而黄家老二又是在李癞子家被他解决的。黄二本是李家的女婿,却给老丈人卖给了韩冈,李信方才一枪扎死他的时候,黄二眼中都是茫然不解。

雪夜奔马,其实再危险不过。隐藏在雪地下的坑洞,就是一个个陷阱。漫无止境的雪原上,不知隐藏了多少杀机。一不小心,便会折断马蹄,顺便摔断骑手的脖子。但赵隆全不在意,他胯下的那匹老马仿佛有着透视雪地之下的魔力,在奔驰中时不时的跳起又落下,避开一个个隐蔽陷阱。

马背颠簸得如同惊涛骇浪中的一叶扁舟,可骑在马上的赵隆,就只用双腿夹着马腹,便稳稳的钉在马鞍上。他双手紧握铜简,双眼如鹰隼般锐利,毫不犹豫地追逐着陈缉的身影。

越追越近……

越追越近……

陈缉还在不停的跑着,身上的每一分气力都送到双腿,沉重的皮裘外套被他一件件丢弃,没了这些御寒的衣物,他就算能逃进树林,寒风会代替追兵,让他一样逃不过死亡的追袭。只是陈缉已经考虑不了任何事情,头脑中的只剩一个逃。

但赵隆已追到了身边,他无意把功劳丢给上天。雄壮的身子踩着马镫站起,摇摇晃晃,仿佛一头熊与老马在表演马戏。摇摇晃晃的身子没有影响赵隆的动作,他瞄准陈缉的肩膀,用力挥下了铜简……

韩冈站在家门口,他的父母惊醒后又被他劝入家中,由韩云娘陪着,依然有些坐卧不宁。王舜臣守在韩冈身侧,几十个被惊起的村民聚在左右,立了功劳的李癞子在韩冈面前点头哈腰,谦卑的笑着。而家门前的道路上,整整齐齐摆着十几具尸体,王厚蹲着那里点验着数目。

大局已定。

不费吹灰之力。

比预计的更为顺利。

李信回来了,带回了黄大尸体。赵隆也回来了,他的鞍前横架着半死不活的陈缉。

“恭喜玉昆!”王厚站起来向韩冈拱手称贺,“贼首皆已擒斩。陈缉、黄家兄弟都在此处,陈举的余党全都完了。再加上过山风这个添头,都是玉昆你运筹帷幄之功啊!”

“岂是我一人之功。”韩冈笑着谦虚,“没有众家兄弟奋命,我也不过是个纸上谈兵的措大罢了。”

“玉昆莫自谦。若无你提前找了我们几个过来,又哪有今夜的痛快!?”

韩冈淡淡一笑,又谦虚了几句,但王厚说的并没有错,正确的情报决定了战局的成败,这的确是他的功劳。

虽然韩冈猜不到陈缉行动的准确时间,但陈家老四这几天就要从凤翔府押来,他不信陈缉会放着亲兄弟不救。又想杀自己,又想救兄弟,那么时间安排就要大费思量。考虑到两件事的难易程度,比起可能造成大量人员损失的劫囚,还是把更容易的诛杀仇人放在前面更合适。

还有另外一件更重要的因素,秦州是西北边境,而凤翔府在秦州的东面。先杀韩冈,再去劫囚,可以顺势向东,逃亡内地。但先去劫囚,再杀韩冈,即便成功,当所有通往内地的道路都被封锁,到时往哪里逃?西北的蕃部?那是找死。向南去蜀中?冬天翻越积雪的秦岭更是找死。难道还能留在秦州?

韩冈相信陈举的儿子不是蠢人,当能算到这一步。所以陈缉如果要动手,也只会在这两天。一方早有准备,一方却是自说自话,被仇恨蒙蔽了双眼,有着现在这样的结局,又有什么好惊奇?

从近两个月前的飞将庙中一场闹剧开始,一连串的风波终于有了了局,最后的一点余波在这里已经平息,韩冈仰望天空,长长的舒了一口气,白色的气息带着积压在心底的一切不安和忧虑,在夜空散去……

五日后,陈举谋叛之案定罪。主犯陈举凌迟于市,其二子陈缉、陈络并斩,妻女悉没于官,从犯刘显以下或斩或绞或流,无一人得脱。一日之间,菜市口上,处决竟达十一人之多。刑求之多,株连之广,秦州五十年来,以此案为最。

当日,李师中亲自监刑,王韶列坐,秦州城中的大小官员几乎都到齐了。刑台周围人山人海,如同社日一般热闹。

随着李师中一声令下,儿孙尽数被擒,失去了所有希望的陈举,如条死狗一般被拖到了架子上,顿时掀起了一阵声浪。

可导演了这一切的韩冈,却安坐在普救寺的厢房中,喧腾透窗而来,却也压不住琅琅书声:“多闻阙疑,慎言其余,则寡尤。多见阙殆,慎行其余,则寡悔。言寡尤,行寡悔,禄在其中矣。”

ps:陈举终于族灭。韩三接下来要面对的对手,并不止向宝一人。

今天第二更,求红票,收藏。

