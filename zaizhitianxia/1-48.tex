\section{第23章 谁言金疮必枉死(下)}

“韩冈?”徐疤脸扭头看了看黎清,又转了回来,“你叫韩冈?”

“在下正是。”

徐疤脸再次面向屋外,黎清震惊的表情像是凝固的瓷像,没有任何改变。徐疤脸看着奇怪,指着他问韩冈:“是你的熟人?”

“不,从来没见过!”韩冈说得是实话,但他轻易的就能推断得出这名青年的身份。青年看到自己的反应,还有听到自己名字后,齐独眼仿佛看到扒光了毛的鸭子在天上乱飞的表情,韩冈若还不能将事情推测个八九不离十,就太对不起自己的头脑了。

一阵泡过热水澡后的轻松感传遍全身,韩冈心头如释重负。自出秦州以来,遮在头顶上的阴云终于散去了大半。陈举能动用的手段到这里应该就用尽了。回执在手,齐独眼已经失去了对付自己的最为有效的武器。纵然他在甘谷城还有一点小势力,可要想如愿整死自己,再难找到名正言顺的借口。只要还在甘谷,自家的人生安全,就不需要再担心。

……………………

辛苦了数日,一切终于有了了局。韩冈站在街中,心中却有些茫然。他带着手下的民伕将军资运送到齐疤脸指定的位置后,民伕们已经被安排去了夫役营。韩冈也是同样在夫役营中有个床位。现在手上拿到了回执,去夫役营睡上一觉,等到明天就可以启程回家……

可这是最差的选择!

回到家后又能做什么,陈举也许会被王韶干掉,但更有可能安然无恙:对付根基深厚的陈举,就算是经略司机宜也要安排筹划,征得经略使李师中的同意,这肯定需要时间。那时怎么办,去接受第三桩差事,还是托庇于王韶?韩冈都不愿意!

无论从野心、骄傲,还是对自己安全的考量,短时间内他必须留在甘谷,同时还要为自己开辟一条晋身之路!

甘谷城中的大街上,惯常的宵禁已经消失,欢呼胜利的军民依然在街上纵酒狂歌。一队往南面去的报捷使节,被他们堵在了城门处,强拉着喝下一碗祝捷酒。担惊受怕了多日,终于可以解放一下,就算是张守约也不愿在这时候再强调军纪。

韩冈淡漠的站在街中心,看起来分外显眼。一名醉汉一手拎只酒壶,一手拿个酒杯,晃到了韩冈的面前:“兄弟!怎么傻站着?老都监带着两千兵就杀退了一万多西贼,今天可是大喜的日子!来,喝一杯。”

“两千退一万……一将功成万骨枯,是这个理吧?”韩冈声音低沉,暗夜中,幽暗的双眸更为深邃。

“啊?”醉汉被韩冈的眼神吓到,不由自主的离了他一步。

韩冈呵呵笑了两声,冲汉子拱了拱手,挤开拥挤的人群,大步往夫役营走去。

“疯……疯子!”醉汉望着韩冈的背影摇摇头,又歪歪倒倒拉着别人喝酒去了。

甘谷城的夫役营在甘谷城西北角,韩冈费了一阵工夫才走到。入了营,找到自家的队伍。王舜臣去了城衙还没回来,除了他以外,所有的人都聚集在夫役营中分配给韩冈的营房中。

韩冈一进屋,朱中急忙迎了上来,神色惶急,“秀才公,方才城衙来人了,说是要重修甘谷城防,张老都监下令把来甘谷的民伕都截下来,我们就是第一批。秀才公,你看这怎生是好?!”

朱中一开口,三十多个民伕都围了过来,盼着韩冈给他们拿个主意。大冬天的,又要夯土干活,少不得丢掉半条命,运气差点,这一百多斤就要交待了。

“俺们拼死拼活赶到甘谷,不是为了做苦力啊。”人群中不知是谁低低的抱怨着。

“就是,就是。”

“莫慌,我自有主意,保管你们不会吃苦。”韩冈威望极高,他一说话,众人便安静下来。他心中则是在大笑:‘这真是天助我也’。

安抚下人众,他径自找到了几名伤员,“你们收拾一下,等王军将回来,跟我去伤病营。”

“去伤病营?”

“甘谷城的伤病营有军医驻留,你们的伤还要找大夫看一看。听说太医局派来秦州的医官总共才四个。秦州城里有两人,外面的城寨只有鸡川寨和甘谷城这两座最前线的城寨才各有一个医官。你们的伤口都要重新处理一下,有京里来的大夫诊治,比急就章的包扎肯定要强上不少。”

“三哥!没哪个随军大夫会给民伕治病!”王舜臣与韩冈前后脚进屋来,正好听到韩冈的话,“伤病营就连着化人场、乱葬岗,进去染了疾疫,几天就会没命。”

此时军中已经有了医院的雏形,都把病人安置在一个地方,以便医治。不过为了治病的方便只是个借口,主要还是担心伤病员的哀嚎,会影响到军心。因为由太医局派出来的医官,通常只为官吏们服务,并不会惠及民伕和士卒。

所有的士兵、民伕得病后,都是苦挨着,最多也只能得到几个亲近好友的照顾。由于那些亲近好友也得按日出工、巡检,病人和伤员得到的照料也是时有时无,多半还是等死。

见王舜臣糊里糊涂的一进门就拆自己的台,韩冈立马瞪了他一眼,这事难道他不知道?就是没有医生才好啊!

王舜臣被这么一瞪,脖子便是一缩,不知自己犯了什么错。

韩冈走过他身边,扯着他往外走:“先去伤病营看看再说,万一有着医官,也好让他诊治一下。如果如王兄弟你所说,没有大夫给人诊治,那就更要去看看!”

带着几名伤员到了城南伤病营。不同于外界的喧闹喜庆,破败的营地阴森寂静。营房内看不到一个医官,只有上百名伤卒面容呆滞的躺卧在几间营房的通铺上。充斥于耳中的尽是伤病员的哀声,空气中弥漫着一股腐臭的味道。

遍地是脓血和污物,还有老鼠和蟑螂的尸体,可以看出,甚至自冬天开始前,伤病营就完全没有打扫过。正如王舜臣所说,这座伤病营,直通的是化人场和乱葬岗。只站在其中,韩冈就觉得自己寿命便已缩短了许多。

四个有伤的民伕惶惶不安,向韩冈恳求道,“秀才公,不能把俺们留在这里。俺们又没大碍,能赶车,能走路,带俺们回去罢!这里都是救不回来的死人……”

“谁说的?”韩冈声音莫名提高,打断了四人的话,“只要用心照顾,除了伤太重的,又有谁救不回来?!”

韩冈的声音惊动了苟延残喘的伤兵们,他们一个个抬起头来,望着莫名其妙来到营中的几个陌生人,眼中都是疑问:

他们到底想做什么?

韩冈挺直了腰杆,迎上数百道疑惑的视线,音量又大了数分,“谁说在这里是等死!”

……………………

“韩三哥,你真的要留在这鬼地方?”

王舜臣已经在伤病营中待了一夜,他看着韩冈找来了民伕,指挥着他们和伤员们的同伴一起清理营房,换洗被单,又一个一个的给伤员们清理伤口。但他还是弄不清韩冈为什么要这么多事。

“这是王兄弟你第三遍问这句话了!”韩冈头也不回,专心致志的给一名肩头中箭的伤员更换绷带,一夜过来,伤员们的眼神已经变了,疑惑虽不减,却多了许多感激,“我的回答还是一样。既然让韩某看到了,我又如何能走得心安理得?”

无视周围伤员怒目瞪来的视线,王舜臣仍苦口劝着韩冈:“这真不是三哥你的差事啊!”

“仁者爱人,此是儒门之教。救人一命胜造七级浮屠,这是佛家之语。无论儒家、佛家、道家,都有讲一个仁字。眼看着这些伤员重病待死,如何不救?与差事又有何干?”韩冈回头,一夜未睡的他脸上露出了一抹略显疲惫的笑容:“必先助人而人助之。你出力帮他人,他人日后也会帮你!”

韩冈不避污秽,亲手用盐水给伤员清洗干净伤口,撒上一些放在营房中、不知有效无效的金疮药,再用干净的细麻布小心的包扎上,“凡事但求一个仁心,至于别的什么,倒没必要去计较了。”

韩冈留给王舜臣的印象是果决狠厉的性子,才智过人的头脑,喝酒时豪爽大气,被人羞辱时脾气则会变得很暴躁。但一直以来,王舜臣都没想过,韩冈竟然还有一颗仁爱起来就有些婆婆妈妈的娘们儿心——用文人的话说,就是妇人之仁。

王舜臣不知这样形容韩冈到底对不对,但在他想来,等先回去交了差事,再来救人也不迟啊!能救些伤病的军汉是好事,王舜臣也被韩冈救治过,当然不会觉得救人是坏事,可何苦把自己搭进去。

他不是没猜测过,韩冈如此是不是有着另外一层用意在,可王舜臣左想右想,也想不通透。他烦躁的抓着头,在营房中随着韩冈转来转去,尽管在职事上与韩冈全无瓜葛,但王舜臣觉得韩冈不走,他也不该走,却不得不在这里心烦意乱的等着韩冈回心转意,打道回府。

又给一名伤兵换过绷带,韩冈直起身子,反手捶了捶腰。一夜过去,他弯着腰给伤员换绷带不知多少次,又在营中走来走去,腰腿几乎都没感觉了。回头一看,王舜臣竟然还跟在身后。

“王兄弟,你还是先回秦州复命去,留在这里耽误事啊……”

王舜臣很坚定的摇摇头,“一起来的,当然要一起走。俺岂是那般没义气的人?”

韩冈见状,扯着王舜臣走到门外,“王兄弟,不是为兄不想走,实是走不得。王机宜要对付陈举还要一些时日,现在回去,那是正撞在枪尖上……”

“三哥欺我!你何曾惧过陈举半分?!”王舜臣不是没想过韩冈不肯回秦州,是为了要躲着陈举。但这一路过来,看韩冈的表现,反过来还差不多。

“跟陈举斗,我的确不惧。但陈举毕竟势大,跟他斗我是在刀尖走路,保不准什么时候就会挨上一刀,夜里也难睡安稳。”

王舜臣看着韩冈满眼的血丝:“在甘谷城就能睡安稳了?”

“我现在就睡,你说有没有人能在这里谋害我?”韩冈一句反问得王舜臣哑口无言,又道:“你回去后,先去拜会王处道。有他引荐,王机宜必然会信用于你……”

“就像前日王衙内引荐三哥你?王机宜的那般信用,俺可没力气搭理!”

“别犯浑!你跟我不同,我的功名在甘谷,你的前路却在秦州!若我所料不差,你和赵子渐,王机宜肯定都会重用!”韩冈的声音严厉起来,有种不容拒绝的威严。

王舜臣是武夫,王韶身边正缺得力人手,而且通过王舜臣还能结交到吴衍,王韶肯定不会放过的。至于自己,王韶不是不想用——韩冈也看得出来——只不过王机宜要先给个巴掌,才会塞颗枣过来。韩冈对巴掌没兴趣,那颗枣子自得另外找地方拿。

王舜臣虽然不笨,但人情世故上绝比不了活了两辈子的韩冈,他抓着头:“俺怎么想不明白。”

“日后便知,现在说了就不灵了。听我的,你回去了自然知晓。”

ps:虽然王韶吝惜一个官职,但韩冈照样能凭着自己的才能打开个出路来。

今天第三更,求红票,收藏。

