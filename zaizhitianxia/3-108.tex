\section{第30章 众论何曾一(一)}

韩冈一觉醒来,头还有些酒后的昏沉。

睁开眼睛,一张熟悉的俏脸就在眼前。浅褐色的双瞳透着浓浓的情意:“三哥哥,你醒了。”

紧接着艳冠群芳的面容也出现在视线中。旧日教坊司中的花魁今天为了新年精心装扮过,薄施脂粉,唇朱眉翠,一见就让人迷醉。

昨夜除夕,一家人都在正屋中守岁,但出去看人了鞭炮烟花回来坐下来没多久,韩冈就昏昏沉沉的睡了过去。一觉醒来,天色都已经大亮。

韩冈坐起身子,看看身上的衣服,都已经被换过了,摇摇头自嘲的笑道:“糊里糊涂都到了新年了。”

周南笑着:“官人沉得很,倒让我们姐妹累了好半天。”

“南娘姐姐说的没错,三哥哥还不给换,费了好多力气。”云娘带着嗔意娇声说着,似是抱怨。

“怎么平日夜中不嫌我沉?”韩冈调侃着。

周南、云娘脸一下变得发烫,韩冈厚着脸皮能说出这等荤话,她们脸皮却薄得很,根本应付不了。

笑了一笑,侧过脸,就在自己身边王旖沉沉睡着。韩家的主母现在有了身孕,就在过年的前两天刚刚被诊断出来的。孕妇不耐熬夜,早早的就睡了,现在也没有醒。

王旖怀了孕,云娘那边韩冈也是在一直努力着。至于周南和素心两女,韩冈与她们度夜时都是算着安全期,尽量错开时间。用着的是最粗陋的避孕法,却是奇迹一般的没有出任何意外。虽然如今当真是多子多福,韩冈也希望能多有几个儿女、。但连续生子太耗元气,韩冈觉得她们还是歇个两年再说。

蹬蹬的几声脚步向,严素心亲自端着早餐进了屋来:“官人,醒了没有。”

周南、云娘立刻起身帮着放下托盘,韩冈笑道:“早就醒了!”

说着从榻上下来,王旖也被他的动作给惊醒了,迷迷糊糊的睁开眼睛,问着是什么时候了。

韩冈回身将被子给她盖好:“早着呢,多睡一会儿。”

“官人才要多歇上一歇才是,昨天到了晚上才从城外回来。”王旖的话中有些幽怨,更多的是心疼,韩冈作为知县,实在是太忙了一点。

“城外已经安排好了,这几天还是能好好的歇上一歇的。”

在京的官员要参加元旦大朝会,韩冈身在地方,就没有那么多麻烦事。印也封了,事也没了。将对于旱情的忧虑放在一边,照规矩享受着年假。

话是这么说,但赶在年节前,还是有了一批流民渡河而来。为了安排他们住下,韩冈也是辛苦了两天。因为是正好是年节前的两日,人人盼着回家过年。韩冈知道自己要不以身作则,即便有他的声威压着,也必然是人人懈怠,最后这几百流民中多半会有人冻饿而死。

韩冈也有想过先任由手下的吏员懈怠,等出了事,自己正好可以趁机再整顿一番。以便到了春来最关键的时候,不至于有人敢于疏忽大意。但这也只是想一想而已。韩冈虽然早就是满手血腥,并不在意人命,但牺牲无辜之人的事他却是要尽可能地避免,这是原则性的问题,韩冈一向认为做人要有最基本的底限,不会去触动和突破。

两个奶妈这时抱着奎官和金娘过来给韩冈拜年。小孩子长得也快,一年多的时间,一儿一女都开始学说话、学走路了。不过除了叫人,其他话还是没学会。

大儿子叫了韩冈一声,就闭上眼睛继续睡了。而活泼的金娘则精力充沛得很,喊着爹爹,张着小手要韩冈来抱。

韩冈探手将女儿抱过来,小脸粉嫩,很开心的笑着。从年头上算,自己在这个时代已经经历了六年,而算真实的时间,也有四年多了。欣喜的看着女儿的笑脸,韩冈忽而发觉,自己好像已经彻底地融入了这个时代。

“给李家叔叔的信也要早点写好。过两天,叔叔派来的亲随就要回荆南去了。”王旖提醒着丈夫。

“嗯。”韩冈点了点头,“回礼也要准备好。”

就在年节前,李信写了信来,问候了韩冈这位表弟。李信在荆湖战场上表现突出,在章惇麾下屡立功绩。李家嫡传的掷矛之术,在荆蛮中的头目将领中所向披靡。短短时间,李信就已经在荆蛮部族之中立下了赫赫声威。

武将升官的速度从来都是能让文官悲愤不已,李信在荆南打了一年半的仗,期间得了章惇多次力荐和请功,本官就已经一再跃升为从七品的供备库副使,虽然是四十阶诸司使副中的最末一阶,但也已经代表李信成为了大宋为数不多的中层将领中的一员。现在他在荆南做着都巡检,日后凭着战功,继续晋升也是情理中事。

对于李信的连续升迁,韩冈从心底里为他感到高兴。没有家世上的背景,要营造出家族在地方上的势力其实很耗时间,现在多了一个善战的表兄李信,韩家在关西的地位会更快稳固起来。

在家中轻轻松松的度过了四天。到了初五,便是立春。

立春劝农,皇帝籍田,官吏鞭牛,向上天祈求今年的农事平安。此乃是农业社会一年中最为紧要的大事。从宫中到州县,上至天子,下至小吏,都不能随意逃席。韩冈作为一县之长,百里之侯,当然也少不了要上阵。

立春的这一天清早,一头用泥塑起,涂了彩绘的春牛便已经摆放在县衙前,旁边的还有泥塑的农夫和农具。

当晨曦的阳光从东面的城墙上刚刚露出头来的时候,韩冈身穿朝服,带领着县中官吏,自正门步出县衙。当他看到衙门前的几具泥胎雕像,仿佛一瞬间又回到了四年前。

熙宁二年的腊月廿一,比年节早了十天到来的立春。当时就要启程的韩冈,在秦州旁观着李师中带领一众官员举着五彩棒鞭打春牛。而如今,四年后的今天,他韩冈则亲自上阵。

摆在自己面前的泥塑春牛,其手艺水平,远不如韩冈当年所见那般活灵活现、惟妙惟肖,显得生硬无比。能与鄜州田家嫡传相媲美的高手,当然不是随随便便能找得到。

所谓时过境迁,当年在秦州制作春牛的工匠田计,现在靠着为天子制作沙盘,早就有了一个官身。而曾与自己并肩站着的王厚、王舜臣等人,如今天各一方,却都已是年轻一辈的佼佼者。

读了请游醇所作的祭文,在香烛上点火烧了,韩冈接着拿起五色丝缠起的彩棒,绕了春牛一圈,然后在臀后虚虚抽了三下,这就算是礼成。

下面的县丞、县尉、监镇、监税等县中官员则紧接着上来,排着队绕圈挥鞭。

在这过程中,一队乐班吹吹打打,奏着欢快的曲子,不过周围围观的人群中,气氛则是越来越紧绷,仿佛夏日已经占了半幅天空的雷云,下一刻就会有狂风暴雨、雷霆闪电。

今年鞭牛祭春的围观者男女老少数百上千。在外围,还有商贩挤在人群中,贩卖着他们货栏中的泥塑小春牛。但挤在最前面的则各个都是精悍健壮,摩拳擦掌两眼盯着春牛,灼灼的似乎发着饿狼望羊的绿光。

韩冈看着便是暗叹一声,越是灾伤之年,百姓对祭祀也就越是虔诚。为了争夺一块来自于春牛的泥土,使得家中田地今年能有个好收成,让灾害不至于延续一年,恐怕他们都会将吃奶的力气全都使了出来。

当最后一名官员鞭牛之后,赞礼官高声宣布。乐班的伴奏,也在猛地飙起的高音中嘎然而止。

随即轰然一声响,围着春牛的上百群狼一拥而上,如同长河浪起,顿时掩盖了五彩斑斓的泥牛。无数支手臂常常探出,将一匹与真牛大小相仿佛的泥塑春牛碎尸万段,分抢了个干净。一眨眼的功夫,春牛不见踪影,而原本用来祭祀的场地,则已经变成了多人乱斗的角斗场。

鞭牛之后的场面,与韩冈四年前见到的也没有多少去区别,而且更疯狂。一开始还是争抢着能致田地丰收的春牛泥块,但到了后面,有些人火气上来后,都忘记了一开始的目的,而当真跟对手厮打起来。虽然不在典礼的节目表之内,但也是每年惯例要上演的压轴好戏。观者如堵,叫好声不绝于耳。

不过这样一场殴斗不会延续,一见其中有人见血,一群县中听候使唤的弓手便同样一拥而上,将仍在争抢厮打中的壮汉们驱散开,而将场中受伤的汉子抬了出来,没大碍的训了两句让其回家,而伤筋动骨的则是有着来自于疗养院,听命随侍在一边的跌打医生来治疗。

年年都会发生的事,衙役、弓手们都知道该如何应付。只是今年特别激烈,事后得到消息说有十几人骨折,倍于往年。

争夺春牛,代表着立春仪式的结束。都已经是立春,从历法上,冬天已经过去。而这个十几年来应该是最冷的冬天,京畿这边却是一场雪也没下。

旱灾依然还在延续,艰难才刚刚开始。

