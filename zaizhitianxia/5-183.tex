\section{第20章 土中骨石千载迷(12)}

从洛阳到相州,用了十天的时间。

相州的治所位于安阳,这座城池位于沟通南北的要道之上,南北城门处向来最为热闹。如今因为殷墟之事,更是热闹了几分。

从马车上下来,张相只觉得自己的骨头都快要被颠散了。转了转脖子,竟然嘎嘎的响了两声,毕竟是岁月不饶人。四十多岁,还在外面奔波,身子骨自然是吃不消。

走进面前的客栈,张相直接报了姓名。客栈掌柜脸上的笑容更加谦和,颧骨上的肉都堆了起来:“可是京城集古轩的张掌柜?令仆已经定下了一间上房,就在院后。令仆十九哥刚刚出去,还没有回来。”

张相点点头,被人领着进了后院。

这间客栈,档次在酒楼和脚店之间,南来北往的行商住得最多。往后院的上房去,路上遇到的几个全都是商人的打扮。

不过最后擦身而过的两人,张相感觉到,他们有着跟自己相似的味道。

‘鼻子真是一个比一个灵。’

从两名汉子的背后收回视线,只消看了几眼,张相已经确认了他们的身份。

张相一贯自称是集古轩的二掌柜,来相州收货,而实际上,集古轩这块牌子天南地北都有人挂,再俗烂不过,想查底细,没个一年半载都查不出来。

张相知道,做他这行买卖的最重要的就是一个合情合理的身份,行走在乡野之中,四出收购来历不明的古董,若没有能说的过去的身份,直接被捉到官府里的几率,甚至比贩私盐的都要高,被黑吃黑的可能就更高了。

只是风险大归大,利润则只会更大。像他这样的古董贩子,最喜欢的便是历朝历代的古城旧都。长安、洛阳这两座千古名都就不必说了,相州安阳的名气,这一个月下来,在张相的这一行中,可就要直追长安、洛阳了。

当日在开封城外的板桥镇上听到了传闻,张相当即便遣了族弟快马赶到安阳打前站。自己回洛阳将手上的大小事务处理完毕,也带着钱钞赶来安阳。

张相所入住的这一档次的客栈,全都靠着城门。他事先先期来此的族弟张十九约定好在南门东首第一家订房,如果客满就往下顺延。所以一进城中,张相直接就找到了地方。

只是张十九现在出去了,人并不在房中。推门进房,空荡荡的,没有什么贵重的陈设,但打扫的还算干净。

领路的小子退出去了,让随行的伴当去整理行装,张相随手展开放在桌上一张蹴鞠小报——深秋近冬的时节,正是各州的蹴鞠联赛如火如荼的时候——只是他看了两眼,就丢到了一边。

相州这边的蹴鞠联赛是韩家的人在背后主持,说热闹也热闹,但终究还是不如京城和洛阳。东京、西京的达官贵人多,又讲究个脸面,就算操纵比赛结果,也不敢做得明目张胆,使得赌客也信任这样的比赛。但相州这边是一家独大,只看小报上一场场比赛的结果,张相就知道,里面肯定有鬼。这样鬼才会下场去赌。

张相要等的人,并没有让他等候太久。小半个时辰后,一个精瘦精瘦的后生推门捡来,手脚细长,举止利落,看起来十分干练。正是张相先派来相州安阳的张十九。

一见张相,张十九便道:“哥哥来得迟了。”

“十九,你这话怎么说的?”

“甲骨的价钱涨到天上去了。方才小弟去外面走了一遭,乡下的甲骨,只要品相好的,都已经涨到了一贯一片,字多的还要加钱。只敢先买下两片。”

张十九从怀里掏出一个小布包,一层层的打开,五六层后,才是两片刻了几列小字,发黄泛白的龟甲,

捧着龟甲到长相面前,他叹了一口气,“真得多谢小韩学士,要不是他揭了底,这龙骨就只能卖出骨头价,哪里能像现在这样一片就能值一贯?一个月前,恐怕都不会有人能想到,骨头上的字有这么值钱。”

 “你也不想想现在有多少家同行来安阳收货?”张相说着,接过龟甲,也不用手拿,还是用布包托着。他方才还看到两个,想来着相州城中,跟他做着同样买卖的同行,绝不会太少。

将龟甲小心的放在桌上,张相仔仔细细的看着,还从怀里掏出一个放大镜,照着上面似字似画的甲骨文。

张十九在旁说着:“但现在涨得太快了。下面的村里都是各家私藏,硬是不肯就这么买,还想等着涨得更高一点。照这势头,再过一个月,恐怕价钱能涨到十倍都不止。”

张相拿着放大镜,眼神专注,随口应着张十九的话:“等再过一个月,假货就多了。价钱不一定能比现在还高。”

张相一边说话,一边细致入微的审视着两片价值高昂的龟甲。过了半日,伴当已经将行礼收拾好,张相才抬起头:“原来这就是殷商古文,难怪几千年都没人注意。看到东西才知道是为什么。”

将两片龟甲收起来,张相站起身,对张十九道:“先到外面转转,探探风声再说。”

“哥哥一路过来辛苦,也不多歇一歇?”张十九问道。

“正经事要紧。”张相表面上若无其事,其实心中可急得很,他这一次出来,可是带上了不少家当,是决不能亏本的。

张相一行三人,只将一点贵重的细软带在身上,径直出了客栈。

三人刚刚跨出客栈,迎面就是当当当的一阵锣响,一伙人敲着两面破锣,从南门鱼贯进城,一下就吸引了数百人夹道围观。

张相三人驻足观望。

从穿着打扮上,这一伙人都是乡里的农民。不过一个个提着棍棒,拿着长叉,敞着前胸的衣襟,多半是保甲中的保丁的身份。

在这伙人之中,还有两个人,被四马攒蹄绑在杠子上,扛着进了城门。跟乡里面打到人熊、大虫时一个待遇。

“又是哪家不开眼的贼寇被生擒了?”张相远远望着,笑着道,“这可是河北的保甲!”

保甲法推行有年,过境劫掠的贼寇往往就被保甲给捉了,使得地方上的治安渐渐的好了不少,尤其是河北山西这些民风强悍的地方,贼人的下场十分凄惨。旧年仁宗时,强人穿府过县,‘一伙多过一伙’的情况已经不复存在。路上的商旅和行人,也比旧年多了许多。

张十九挤进人群去打探消息,过了片刻又脸色发白的挤了回来,“哥哥,不好了,是大名府的刘豹子失了风,说是掘人坟墓给捉到了。”

张相脸色也变了,刘豹子那可是江湖上有名的古董贩子,怎么就给人当盗墓贼给打了?张相没听说过他什么时候客串过摸金校尉。

踮起脚,仔仔细细盯着杠子上的两个倒霉鬼一阵,张相就更加疑惑起来,“我怎么没看到刘豹子?”

“给保丁当场打死了,首级就挂在前面。人死了,样子就全变了。但脸上那块烫出来的花斑,不是刘豹子还会是谁?”

张相再往前看,一行人已经往州衙的方向走远了。他皱着眉头,视线追着人跟了一阵,最后摇摇头,终究还是不愿相信。

“刘豹子做这买卖做了三十年,你几曾听过他亲自下手的?君子不立危墙之下,好歹也有十几万贯的身家了,去年又纳了一个小妾,身娇肉贵,疯了才上阵。要不是这一回是新出来的买卖,他肯定是守在大名府,根本都不会往相州这里过来。”

盗墓贼就跟贩私盐的一样,都是将脑袋悬在腰带上,而且名声更坏。但刘豹子只管收货,就是遭报应也是做贼的先遭殃。

“或许是刘豹子多半是心里急。”张十九猜测着,“乡里的村夫一个个粗手笨脚,那些龟甲骨片,劲道用的大了点可就碎了,一铲子下去能有多少。又不是拿来做药,碎了照样能派上用场……”

“再急也不会亲自上阵的。”张相不相信,“刘豹子那人,我打过好几次交道,从来不冒风险。”

张十九几乎都要赌咒发誓,可张相仍是半信半疑。

突然两人的背后一声唤:“这不是张五哥?”

张相闻声回头,就看到一个相熟的面孔。还没有等他反应过来,就被人横拉竖拽,扯到后面的巷子里。

“周小乙,你这是作甚?”张相挣脱开来,护着衣襟怒声质问着。他认识此人,也是在做一门生意的同行,也有几分交情。只是刘豹子似乎出了事,让张相不敢相信任何外人,听到身后张十九和伴当急忙追了上来的声音,才让张相安心了许多。

周小乙压低了声音,急道,“刘豹子那个精细人都失了风,张五哥你怎么还这么不小心?!”

“当真是刘豹子?!”张相回头看了张十九一眼。

“不是他还能有谁?脑袋都给人砍下来了!尸首也不知丢哪儿去了,亏他攒下了几十万的身家,最后连个全尸都没有。”

“说是掘人坟墓。”

“哪里是掘人坟墓?跟人争食给栽了罪名。”周小乙愤怒的握着拳头,“没见过下手这么绝的人……张五哥,相州可是不能呆了,有人要通吃下这一盘买卖。”

“杀了刘豹子究竟是谁?”

“徐兴徐大胡子,他可是正经八百的安阳本地人。”周小乙说道:“张五哥你别说徐胡子的手下没见过你,那几个就在人群中盯着。也不知多少人被盯住了。徐胡子他是打定主意要将外人都给赶出相州。”

“徐胡子哪儿来的这么大胆子,是谁给他撑腰!?”

“徐胡子背后是韩家的人!”周小乙又颓然一叹:“相州这里的买卖只要韩家想要,就肯定是韩家的,外地人争不来。”

