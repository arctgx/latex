\section{第18章 青云为履难知足(七)}

李信领军凯旋。

用了六天的时间奔袭三百里,攻破了两家溪峒,将不服号令的蛮夷彻底毁灭。

韩冈如此下令并不是他狠毒,在需要站队的时刻,还想首鼠两端,站在一边等着捡便宜,韩冈绝不可能留着这样的人在背后。

顺我者生,逆我者亡。

天子让李舜举传达的口谕就是这个意思。

一支支高举的长枪枪杆上,一枚枚首级随着两脚的交替前进而晃动着,凝固在脸上的表情只有恐惧。从城外回来的大军,每一人的枪杆上都有一枚首级,而最后进城的一队驮马背上,用筐子装的首级只会更多。

几天前还与自己并称为左右江三十六溪峒的左州、忠州两家的壮丁,现在已经挂在了官军的旗杆和枪尖上。在枪尖上晃动不已的一枚枚首级,让不得不跟着韩冈出来迎接大军凯旋的一众溪峒之主,看得不寒而栗。

昨日还将宋国的美酒喝得畅快,只以为是准备要用酒肉来示好,好让他们多发兵。哪里会想到,就在这几天的时间中,官军已经直接攻下了两个左州、忠州溪峒。

心中的惊骇不由自主的化为言语流了出来,对韩冈的恐惧让一众洞主们低低私语。

“恐怕转运相公一开始就是打着这个主意。”

“那是肯定的!”

“要知道留着我们喝酒吃肉是为了今天,我早就回去了。”

“你敢走。不怕官军转天就堵到你家思陵州的门口去”

“……说不定转运相公根本就没派人去忠州、左州。”

“……那卫福和侬章额还真是太冤了,什么都没做,杀星就上门了。”

“今天能杀卫、侬两家,明天就能杀到我们头上……还能忍吗?!”

“不能忍就去死好了,转运相公不就在那边吗?”

“择朵!你跟卫福有亲,我们可没有。想要去找死,你自个儿去,别拖着我们。”

“最好还是几家联合起来,若是哪天官军打过来,互相之间也好有个照应。”

“一家家的离得那么远,谁救得了谁?难道还能一辈子防着吗?”

“都少说几句,这时候还说什么!不把这位小韩相公服侍舒坦了,等着做下一个卫福、侬章额吗?!”

“怎么服侍,谁家嫌人多粮多?跟着官军去打交趾,我们肯定死在第一个,然后官军才会上来跟交趾打。”

“一个眼下就要死,一个至少能拖后几个月,选谁还要想吗?”

对洞主们的窃窃私语,韩冈恍若未闻。闲言碎语他根本不需要去在意,只要他们听话就行了。

凯旋仪式用了一个时辰宣告结束,进献战果、论功行赏。在邕州大战过去了三个多月之后,收缩在邕州城附近的官军,终于有了实质性的行动。对此战的胜利,邕州百姓欢声雷动。

洞主们一个个面色如土,如果这时候再违逆韩冈的心意,官军接下来的目标,肯定就会轮到自己。

出战的大军回营休整,韩冈已经安排下了酒宴,他让人从宾州运来的酒水,大部分还是为凯旋的军队所准备。

待观礼的百姓都散去,韩冈返身往衙门里走。在韩冈身后,邕州通判低声问着昨夜提前带着忠州、左州两家洞主首级回来的韩廉,“到底卫福、侬章额是因为什么没来?

“何须多问?忠州有六百汉人,左州有一千多,大多是最近刚刚掳掠而来,吃了不少苦。”李信在旁接口,脸上有着淡淡的不忍,以及浓浓的憎恨,“当时一时义愤,就把两家住在主峒中的男丁都杀光,就究竟是什么原因也没必要问了。”

轻描淡写的说着两家被拘束起来的汉人‘吃了不少苦’,但实际情况,肯定只会比李信说出来的更惨,说是一时义愤,恐怕愤怒更多一点。

“只杀了男丁?”邕州通判问着。

“两家的妇孺倒是没动,总不能做事做得跟蛮夷一样。”

“只是破了主峒吧?”韩冈在前面问着。

“嗯。”李信点头,“时间仓促,只来得及攻下两家的主峒。不过赶来救援的援军,也一气杀退了几部。”

左州、忠州两家是左江有数的大溪峒,一座主峒,下面还有好几处、甚至十几处小峒。李信领军速攻,攻下的当然也只会是两家的主峒,在附属的小峒中,两家少说还有数千近万的人丁。

“是不是要将忠州、左州领下的溪峒都扫平掉?”韩廉兴奋的问着。

“早间本官说过,从今往后左右江三十六峒没有左州、忠州两家。剩下的手尾就让后面的洞主们去处理,投名状先得给我交上来!”韩冈回头,“斩草要除根!”

汉人善生聚,不论做工务农都远比蛮夷出色,辽国、西夏攻进中国的时候,做得最多的也就是劫掠人口,驭汉人为奴,为他们提供税赋。四方蛮夷,也都将汉人当成是肥羊一般。交趾就是一个现成的例子,而最近在蜀中茂州闹起来的蛮部叛乱,也是因为当地筑城,让蛮部无处劫掠吗,又担心起汉人报复的缘故。而且这样的事,千年之后也不少。

对付这样的强盗,最好的办法就是将贼手剁下,让犯罪的成本高昂得无人能承受得起。只有用杀才能止杀。只有用更血腥的手段报复回去,才能遏制蛮夷对汉人的窥探。

——这就叫‘以直报怨!’

这是临时州衙主厅中的第三次会议,莅会的还是韩冈为首的邕州官员,以及左右江三十六溪峒的洞主。

“古万寨、太平寨、永平寨,本官接下来要重新修复这三座军寨。”韩冈旧话重提,但说话的语气和内容已经完全变了,而在他面前,洞主们甚至都不敢再坐着,全都老老实实的站着,“不过交趾人多半不甘心官军重回左江,必然会遣军来骚扰,所以必须先发制人。”

交趾军吃了这么大的亏,据说死了几万人,哪里还有可能回来骚扰?!但在场的洞主们没人敢指出韩冈话中的错处,韩冈就算说太阳是方的,他们也只会点头道——转运相公说得没错,我天天日出时都能看见太阳的四条边!

韩冈环目一扫,一个个俯首帖耳的洞主让他满意的点头:“本官需要尔等攻入交趾境内,其国中的子女财帛任尔等自取,能拿到多少,都看你们自己的本事……”

“韩相公,”一个战战兢兢的声音响了起来,是站在后排的一名洞主,“我等身处右江,离着交趾实在有些远了……”只是他见到韩冈森寒的眼神挪过来后,立刻慌了起来,“小人肯定是要派兵去的!至少一千!就……就是粮食接济不上,如果去交趾国中没有搜到存粮,恐会耽搁运使的吩咐。”

“关于这一点,本官也想过了。不论是古万三寨,还是交趾边境,都是在左江这一边。右江的溪峒如果要派兵的,的确很不方便。”韩冈说着,他并不是御下苛刻的人,也是讲道理的,“但你们可以向左江的溪峒借粮,只要之后等你们搜到战利品后,将欠账都还回去,那就没问题了。另外不要忘了,还要付上利息。如果有人怕日后撕掳不清,本官也可以为两边做个中人,必不会让人翻脸不认。”

左右江附近的各大溪峒家底都不少,左右江就是这一片地区的黄金水路,在这两条江河中占有一席之地,没有哪家部族会发达不了。多了也许没有,但借贷个一两千石粮食,倒算不上什么大问题。

“不论在交趾国中得到什么,都是你们的,官府不会要你们一分一文。只有一点要记住……”韩冈一下变得声色俱厉,“不过如果是汉人,就必须给我送到邕州来!本官会按人数给付钱粮为赏。如果救回来的汉儿数目多的话,本官也会上报朝廷,无论官职、还是财帛,都不会吝啬。”

“相公放心,我等绝然不敢冒犯上国百姓。”厅中的蛮人们一起向韩冈作着保证。

“如果交趾军来袭,小人肯定会拼死抵御,不过万一战事不顺,也许会难以抵挡……”又有人有着疑问,“不知相公能不能派一支官军为我等做依仗,只要能一挫交趾兵锋便可。”

“不会让你们与交趾军硬拼,遇上交贼的时候直接回师就可以了。”韩冈当然不会让人失望,:“若是交趾军追来,本官自会遣兵对付。”

免掉了后顾之忧,在场的洞主们也稍稍放心了下来。韩冈等了一下,见没有人再有事要发问,便说道:“好了。左州和忠州的主峒都已经被攻破,就是还有几十个小峒没有清理干净。本官不打算留着他们,谁打下来就是谁的!”

虽然顾忌着脸面,没有人接口,但有好几名洞主的眼神燃起了熊熊的火焰。

“日后官军南下攻打交趾。尔等只要愿意,也可以一同随行。只要在战场上出了力,都会加以封赏。最后会视功绩多少,交趾的人口、财富,甚至土地,都有参与分配到资格!”韩冈推波助澜的一笑,“是想在山沟里做一辈子洞主,还是让子孙在交趾号令州县,全在尔等一念之间。”

