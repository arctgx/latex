\section{第22章 明道华觜崖(一)}

韩冈暗暗下了决定,而杨绘笑罢之后,则道:“就当我是臆测好了。就不知韩玉昆你在张子厚的教导下,究竟从实物中格出了什么理来?”

“名教出于自然,这句话其实不算错,只是晋人在这之后走上了错路。天生万物,天理便在万物之中。为什么冰会浮在水面上?为什么虹作七色?为什么将轻重二物同时从高处落下,却能同时落地?寻常都能见到的事物,道理便在其中。只要拿着身边之物一桩桩去格,个中道理集合起来,便能一步步近于大道!”

韩冈絮絮说了一通,杨绘却一下揪住了其中的一句话,急声追问:“轻重二物同时从高处丢下,能同时落地?!”

“自然。”

韩冈回答的干脆,

“难道说将鸿毛与石块同时丢下,会同时落地?”杨绘仿佛卡着仇人脖子似的,揪着韩冈话语中的错处,“怎么韩玉昆你所看到的寻常,怎么与我等看到的寻常完全不同?”

一阵轻声窃笑在人群中传开,杨绘说的正合他们的想法。

韩冈立刻回道:“羽毛受风,所以会慢下来,这跟轻重无关。同一张纸,平着落地和团起来落地,快慢是不同的,这就是受风的缘故。如果是同不受风,一颗十斤重的铁球和一颗一斤重的铁球同时从开宝寺铁塔上丢下,却肯定是一起落地的。”

杨绘皱起眉来,想了一想,却与周围人众一起摇头,“……胡说八道!十斤和一斤怎么可能会一样。”

“不去看过,便妄下判断,所以说学士是臆测。”韩冈笑容如春风一般和煦:“要是学士不相信,不如择日去开宝寺铁塔上一试便知。”

杨绘见韩冈胸有成竹,眉头皱得更深,眉心的皱纹变成了一个川字。有心想否定,却是怕最后错了,自己丢脸。但是他怎么想,都绝不这根本不可能。心念急转之中,忽然想到韩冈不仅是张载的弟子,听说他更是孙思邈孙真人的私淑弟子,会不会……

韩冈却见杨绘退缩下来,不敢回答,更加得意的笑着:“学士既然不敢去开宝寺铁塔一作验证,那也就罢了,韩冈也不敢强求。”

听了韩冈话,原本还在犹豫间的杨绘,却一下冷笑起来:“既然韩玉昆你一作验证,我确想见识一下。赌上一把如何?”

“赌?”韩冈自信的点头道,“有何不敢!”

“出了何事?”在旁冷眼看了许久的曾布,终于走了过来。

曾布是压宴官,尽管现在宴会上的规矩已经解放了开来,可以尽情欢庆。但维持宴会上的欢快气氛,也是必须的。若是闹出不愉快的事来,给御史盯上,各自都不好过。

前面看着杨绘过去找韩冈,明显的没带好意,曾布没有动,哪边吃亏对他都一样。杨绘地位高,口才好,而韩冈的口才绝不输于他,心性、才智更是贾诩一流的人物。只要不闹开来,看看他们两人演出的戏码也不错。

不仅仅是曾布,吕惠卿等几个考官,以及其他学士都在外面看乐子。不过现在闹到要开赌,就必须上来看一看了。

“不是什么大事。”对曾布的询问,韩冈拱手回道,“只是一尽酒兴的小赌而已。”

“要赌什么?”曾布明知故问。

“韩冈与学士要赌一赌,将一颗十斤重铁球和一颗一斤重的铁球一起从开宝寺铁塔上丢下来,是先后落地,还是同时落地。”

“元素【杨绘字】,是这样吗?”曾布反过来问杨绘。

“不!”杨绘却摇头否定,双眼盯着神色疑惑起来的韩冈,冷笑着:“既然韩玉昆你说这是理,那只要是高处,在哪边都一样吧?不一定要在开宝寺铁塔上。铁塔可以,繁塔可以,甚至这边的华觜冈……”杨绘回手指了指东南面,越过殿门,能看见半里之外,在琼林苑东南角,建有一座高台的山冈,下临一汪清池,“应该也可以吧?!更不需要铁球了,那物件不好找。石锁啊,秤砣都一样,只要一个十斤、一个一斤就行……韩玉昆,你说是也不是?!”

‘可惜了比萨斜塔的实验。不能向伽利略来致敬了。’虽然鱼儿上了钩,韩冈还是感到一丝遗憾,也没有及时回答。

韩冈似乎是在犹豫的迟钝,落在杨绘眼中,便让他眉眼一挑。眼神一下锐利起来,露出了看破了一切的笑容:“怎么?除了开宝寺铁塔,其他地方就不行吗?还是说只能用铁球?”

“……当然不是,都可以。”

韩冈的回答似乎有些勉强,连笑容都收了起来。周围众人都觉得他心虚胆怯,纷纷窃窃私语起来。

“好!”杨绘哈哈大笑,“即使如此,本官就跟韩玉昆你赌了!……择日不如撞日,就在今天琼林苑中决个对错来。”杨绘也不敢拖时间,要速战速决才是。万一韩冈有什么术法,弄什么狡狯,到时候可就要干瞪眼了。琼林宴上,不拘俗礼,借用一下琼林苑中的楼台,不会有什么问题。旧年也常常有进士登华觜冈临风赋诗,“还望玉昆你不要临场退缩才是!”

韩冈还没有回答,吕惠卿就凑了上来,笑道:“既然是赌,总得有个彩头吧?”

杨绘看了眼吕惠卿,又瞅瞅韩冈,暗自忖道,韩冈要荐张载入经义局,果然把内定中提举经义局的吕惠卿给得罪了。

“不如就罚酒三杯好了。”杨绘提议道。

酒席上的赌斗,没人会在乎彩头的,关键是面子。谁被罚喝了酒,可就是当众丢人现眼。

“最好还得即席赋诗一首,以记今日之事。”

吕惠卿又追加上来的提议,更是坐实了杨绘心中的想法。韩冈看了吕惠卿一眼,脸色木然,不知在想什么。

周围众人中,知道韩冈举荐张载的,也是了然于心,皆道吕惠卿够狠,这一下,韩冈别想再留在东京城,说不定连王安石的女儿都没脸娶了。

至于绝大部分的新科进士,见着新党中坚明着拆王安石女婿的台,却是变得狐疑起来。

吕惠卿神色夷然不变,他过来帮腔,却不在乎别人是怎么想。

韩冈在经义局中横插一杠,吕惠卿当日听了后便是冷笑不已。谁都知道经义局是做什么的,真正有心争夺儒门道统的学派,哪一个愿意将这个位置相让?要不是王安石现在占着宰相的位置,旧党的一封封奏章,足以将设立经义局的主张送到故纸堆里去。

但从吕惠卿的角度看来,韩冈这一次做得十分聪明。通过举荐张载入经义局,在不伤新法的前提下,向天子表明了自己的独立性,而且还让天子觉得他顾念旧情、不忘根本,为人正直。这一感观,足以铺平韩冈之后的仕途道路。

而在王安石那边,韩冈一心支持关学这点的确让人恼火,只不过韩冈再怎么样也是王家的女婿,也不可能当真翻脸。更别说他至少是站在新法一边说话。其实这样也就够了——畏于权势而尽弃其学的女婿,王安石也不可能看得上眼。

“你买谁赢?”曾布低声问着吕惠卿。

“就跟子宣你一样。”吕惠卿笑了一声,看着人群中杨绘哈哈笑着,与笑容浅淡的韩冈,一团和气的将赌注定了下来。

曾布也在瞅着杨绘脸上自信的笑容,摇了摇头:“杨元素糊涂了,白活了四十多年。也不仔细想想韩冈一段话是怎么说的,见了钩子就往上咬,团鱼都没他咬得快!”

“杨元素是聪明人。但聪明人往往就会自作聪明,把事情往复杂里去想。”吕惠卿侧过脸,对曾布道,“何况他也不可能如你我一般,深悉韩冈的为人心术,吃亏上当也是免不了的。”

聪明人对自己都是有着绝对的信心。看到韩冈胜券在握、胸有成竹的样子,杨绘绝不会去怀疑自己想法的正确性,而只会将韩冈的信心来源往阴谋诡计方面去考虑。既然是这样,韩冈只要多提两句开宝寺铁塔,他就必然会想歪掉。

吕惠卿方才似是站在杨绘一边,其实是在帮韩冈着阴杨绘。韩冈瞥过来的一眼中带着笑意,分明也是看透了吕惠卿的用心。

杨绘不合于新党,但至少现在还没有对新法攻击的太厉害,所以还能做着翰林学士,否则早就给赶出朝堂去了。不过吕惠卿知道,两年前,天子曾有意让杨绘任御史中丞,不过给人给挡了。外面传说是王安石,但实际上却是文彦博。这是杨绘攻击新党不利的缘故。不知杨绘本人知不知道。

但现在杨绘身上的压力很大。翰林学士再进一步不是执政,就是御史中丞。所以杨绘一直都不肯明确的出来攻击新党。而旧党那一边,作为杨绘的政治后台,却不会让他的态度继续暧昧下去。

今天杨绘来找韩冈麻烦,吕惠卿只要联系起他现在的处境,就能想明白究竟是怎么回事。

‘不敢动狮子,却想来打虱子。小心虱子不是虱子,而是狮子!’吕惠卿冷笑着,心中转着绕口令。

