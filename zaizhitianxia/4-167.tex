\section{第25章 山水留连住多时(中)}

【自动更新出错了,真不好意思。】

岭南两路,一向被北方视为蛮荒之地。

瘴疠横行,蛇虫遍野,举目多为荒野,不宜常人居住。

而事实也正是如此,广西、广东的任何一个军州,不论是户口还是税赋,都难以与北方略大一点的县相比。唯有广州是个特例。

这是天下排名前三的大港,信风到来的时节,每天进出港口的商船数以百计。只要站在港口的码头上,一天之内,就能看到行驶在七海之上的形制各不相同的海船——有桅杆高挑、骨肋坚实的广船,有两头高耸、船尾饰有彩绘的福船,有平底多桅杆的沙船,有船首尖翘、两侧绘有一对眼睛的鸟船,更有来自于西方,张着三角形风帆的船只。

这一艘艘,满载着各地珍奇而来,又满载着贵重的货物而去。每一艘离港、入港的船上,都有着价值几万、十几万、甚至几十万贯的货物。

不过广州城中,聚集了最多财富的地方,却不是桅杆林立的港口,而是有着几十家金银彩帛铺聚集的东门大街。

南来北往的商人们,因为铜钱、铁钱沉重,为了携带方便往往都是带着金银或是彩帛之类的‘轻货’。等他们到了地头,都必须将这些轻货在金银铺中兑换成铜钱才能使用。而随着金银兑换业务的发展,许多商铺的本金越来越足,渐渐的都做起了放贷、典当的买卖。

一栋栋雄壮的屋宇沿着厚重的青石板所铺就的大街延伸开去,广阔的门庭在高墙壁垒之间显得幽暗深邃,冷漠的将穷人拒之门外。这里的每一条砖缝都闪烁着金光,沉重的马车在石板路上磨出的车辙里,都藏着叮当作响的铜钱。

每天都有数十万贯的资金在东门大街上流动,一次简单的交易都是几千近万。到了每年冬夏,信风渐起,一年中船只进出港中最多的时候,更会窜到上百万贯的水平。

除了汴京城中,同样是金银彩帛交易聚集的界身巷让人只能仰望之外,就算是泉州、杭州两个同样、甚至更胜一筹的繁华商港,东门大街诸多金银铺的东主和掌柜们,也都是不服气的,‘那些都不成气候!’

东门大街旁的酒楼,只为金银铺的东主、掌柜还有他们的客户们服务,价钱当然是最贵的,同时也是最好。几十万的生意都在推杯换盏中完成,拿着嵌了宝石的银杯为交易成功而碰杯,轻描淡写的吐出的数字后面,多是缀个万字。

从汴京传出来的风俗,两只热气球带着招牌飞在天上。三层高的楼宇,就是放到京城都不会丢脸。菜单上,竹鼠、山鳖、鸧鹳、蝙蝠、蛤蚧、蝗虫、蜂房,只要是能下肚的,越是珍奇之物,就越是受到欢迎。

山珍海味摆了一桌,对坐的就两个人,一人带着嘲笑的口气:“前两天往京里贩棉布的米二,竟来找我贷个五万贯。这点钱,往年说借也就借了,喝杯茶的事。可现在谁不知道这些年棉布的生意越来越难做,他在家乡欠了几万贯的债瞒得再隐秘,也躲不过我家的耳目。他之前在港中倒是有条船,但船上装的是什么吗……竟然是牛!”

“要赚钱,耳朵可不能只放在广州、福建,交州那也是个宝地。”听到这番话,屏风之后的另一人,得意的压低声音向同伴炫耀着,“米二贩牛,就是为了搭上了广西小韩龙图的线。前些日子鄙号的人,可是亲眼在海门看到他从李钤辖的门中走出来的——李钤辖是什么人,小韩龙图的亲表哥——打通了这条路,只要有小韩龙图说句话,他下一次从交州回来,至少能带上一船的香药。昨天我借了十万贯给他,五分的利!”最后还不无遗憾,“只可惜这样的买卖也就一两次,等他有了本钱后,就不会再借了。”

冒着遇上台风的风险,米彧抵达海门港的时候,已经是八月底。

他这一次,特地从泉州随船带来了一船农具,如今交州的蛮部都是铸兵为犁,亟需大量的农具来维系生产,而作为转运使的小韩龙图眼下最关注的也就是交州的农业生产,米彧看准了这一点,带了农具回来,不为赚钱,只为卖好。

因为运送耕牛去贩卖,米彧被人耻笑,回到乡里还要被逼债,连父母兄弟都不搭理他。但能藉此与韩冈搭上了关系,投再多本钱也不嫌多,转眼就能赚回来,衣锦还乡都是一趟船的事。

通过半年紧张的建设,海门港已经是初见规模。

烧制的简易水泥,从码头到道路再到屋舍,到处都有使用。来不及烧砖、凿石,但大量水泥的运用,让城中几条主要街道,看起来并不比铺了砖石的道路稍差。

道路两旁,以刺桐为行道木,到了开花时节,便会是如同泉州一般,到处是艳红如火的花朵。道路的设计者还设计了排水的暗沟,如果是普通的雨水不会淹没道路,稍大一点的也会很快引到海中。

另外海门港有个特别的地方,就是从码头通往仓库区的道路,并不是普通供车马行驶的道路,而是沿着汴河两岸正流行的轨道。硬木打造的木轨一直延伸到城中的库区。

货物下船后就送上架在轨道上的货车,几千斤的商货,只要两匹挽马来拉着就过了。在对应的库房中卸下货,空车则顺着另外一条线再从库区又绕回来。回环式的物流交通,让进出两条线上的车辆互不干扰,形成了一个稳定迅捷的通道。不仅能运货,还能送人,省去了大量的人力物力。

从还在船上的时候开始,一直到走进港口,米彧都没能将嘴合拢。两个月前这里还是一片工地的模样,大半道路都还没完全竣工,到处都能见到污泥和脏水,但现在出现在他面前的,却是一座干净整洁、井然有序的港镇。

虽然船只还不多,可在米彧看来,就算数量再多十倍,这座港口应该也能井井有条的容纳下来——从一开始,对海门全局的设计,估计就是以明州、台州的中等港口为目标,同时还留下了扩展的空间,达到广州、杭州的规模也不成问题。

迎接米彧的是顺丰行特地从关西调来开拓新局面的掌柜,姓王,单名一个清字。

王清的模样五大三粗,双手骨节粗大,显得十分有力,不似商人倒像是一名军汉。不过这也不能说错,他的确原本就是吃过兵粮,耳后还有刺字。只是几年前报了病从军中退出来,投到了韩家的门下。靠着能写会算,加上一些精明干练的本事,成了顺丰行中主管一地的掌柜。

王清有着关西军汉的豪爽,但也绝不缺精明,只是笑起来就是满面憨厚,让人看不出半点狡诈。迎上来时,热情无比的米二哥长、米二哥短的打着招呼,天知道两人就只有数面之缘而已。

王清亲自来接米彧,与其说是米彧的面子,还不如说是船上货物的面子——从负责上船临检的监镇,将消息传到王清手中,只用了一刻钟,这个速度着实让人叹为观止。

王清与米彧一起上了轨道马车,往城中行去,一路上说着些有趣无趣的闲话。

寒暄了几句,米彧便问起了韩冈和李信两人的现状。

王清有些许遗憾的说着,“米二哥你来得不巧啊,龙图刚刚去了钦州,这一次要从邕州绕上一圈后,才会回交州来,少说也要二十天——万一居中有事要去桂州,那就是两个月了。至于钤辖,倒是在交州,不过现在去了河内寨,去看看那边的寨子到底建得怎么样了。”

“那还真是不巧。”米彧心中满是失望,却竭力不然自己表现到脸上来。

他为了能讨好韩冈,可是在船上带了沉重又占地方且卖不出价的农具来,正常的海贸谁会这么做,这完全是奢侈浪费的行为。本以为能藉此见上韩冈一面,谁知道这般不巧。

“米二哥放心。”王清亲昵的拍拍他的肩膀,“现在交州南北七十二家,没一家不缺农具,就派了人在邕州、钦州和廉州到处找着。如果他们听说了米二哥你带了这么多农具来,哪家不要讨好你?交好了这七十二家,米二哥你在交州便是一路畅通,没见到龙图和钤辖,其实也无大碍。”

“在他们那里的讨好,哪里比得上龙图的随口一句。”米彧摇头。奉承话说着,但心里则是欣喜非常。

韩冈、李信毕竟会离开,也许就在不久之后,而七十二家部族——这个数目其实也只是叫着顺口,实际是七十四个羁縻州——却是会在这里一代代的繁衍下去。结交了蛮部,对于他的生意有百利而无一害。作为商人,都是和气生财,上层路线要打通,下层路线也同样要保证,上下都讨好,这样便才能做得长久。

“米二哥,你可知在你之前,先一步进港的船上,带的是谁人吗?”王清忽而问道。

“是谁?”

“是章七相公的亲弟弟!”王清笑道,“他和米二哥你都是从泉州过来的。”

米彧脸上多了几分讶色,但更多的还是事不关己的淡漠,毫不在意的反问了一声:“啊,是吗?……想不到就是前后脚啊。”

