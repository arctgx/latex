\section{第47章 节礼千钧重(中)}

赵顼一脸迷醉的摸着光滑如锦的板甲,仿佛在抚摸着一名绝色佳丽。

韩冈在后面轻笑。如果拿着现在的札甲实验,五十步的距离上其实也几乎不可能洞穿。毕竟当年御前试验神臂弓时,是在军器监成立前,甲胄质量低劣的情况之下——但凡做实验,实验组和对照组应该放在同样的条件下,韩冈这一次是占了大便宜。

但韩冈会说破吗?自然不会。

其实也是工匠们的功劳,韩冈心中明白。

要将铁板打制成带着弧度的防箭式样,也的确要费一番手脚。不过几位被点到的老工匠都是聪明人,看到板甲制作之简便,又不是没听说过韩冈的大名,一早就已经确定了将他们圈起来打造的新式甲胄,会被新来的韩舍人给献上去。

为了能在天子面前讨个好,几位大工匠卖足了气力日夜赶工,使尽了浑身解数。平常打造时偷空减料的一些程序,这一次可都用上了,质量远比平常的札甲要好得多。

在将甲片埋在髙热炭火中闷烤了一夜之后,用了最短的时间来处理后续,最后将之抛光。甚至还找来胆矾水,稍稍浸了那么一浸,镀上了薄薄的一层铜色——韩冈这才知道,原来这个时代,已经在用胆矾水来炼铜了——虽然在地上磨一磨,甚至拿指甲擦一擦,说不定就会露出底下的铁质,但眼下看着的确有几分高档品的感觉——只要不看式样。

靶子被重新摆了回去。

韩冈对着张若水道:“可以再近一点!”

他脸上的微笑告诉了在场所有人他信心十足,靶子依言被拖回来一点。

四十五步。这一次是一个略深一点的小坑。

“再近一点!”

四十步。板甲正面终于给射穿了,不过箭簇没有整个穿过去,仅仅扎出个小洞。

“再多靠近些!”韩冈似乎是不耐烦的在说着。

这一下,靶子一下被拉近了许多,三十步。

随着扣下牙发,弩弦一下张直,笃的一声闷响,箭矢洞穿胸前的甲片。不过拿过来一看,箭簇也给毁了。纵然着甲的士兵受伤,伤得也不会太重。

随着挂在靶子上的铁甲一步步的被拖得越来越近,冯京、吴充等人脸上的表情也越来越精彩。一直等着到了三十步,看见箭矢终于穿透了进去,憋在胸口的一口气才吐了出来,但脸色已经很难看了。

“此物果然远胜札甲。”赵顼走上前去,笑得合不拢嘴。回头提高了嗓门,“不输神臂弓!”

吕惠卿张开口打算说些什么,不过想了想之后,还是将上下嘴唇又闭了起来。就算能证明军器监现在的札甲不输于板甲,也没什么意义了,反而会让人看低了自己。

能在四十步的距离上防住神臂弓的射击,那么只有六七斗的骑弓,站在面前都别想射穿掉。即便现在的札甲不输板甲又如何?这等粗制滥造的铁板拼凑起来的甲胄对于禁军步卒来说,已经绰绰有余。

“再取骨朵和一幅鱼鳞甲来。”韩冈高声说道。

试过了防箭的能力,韩冈又要展示一下板甲抵挡钝器伤害的能力。这一回,他可不担心当场做对比了。

“不用试了。”赵顼提声阻止了。

逢到节庆,便到宫中来表演的杂耍百戏,那些个玩胸口碎大石的汉子,敢在心口上放块青石砖,受大锤砸;却没一个会只垫一层被褥的。

为什么如今骑兵手上所使用的短兵,铁鞭铜简要远远多于刀剑?就是因为钝器的力道能穿透到铁甲背后的身体上。而韩冈拿出来的板甲,一看就知道远比柔软的札甲更能防住钝器伤害。

照现在这样再试验下去,几位宰辅的脸面都不好看——冯京、王珪和吴充脸上的表情,赵顼兴奋之余,还是看得很清楚。

重新坐回了宴会中的席位,最下首的韩冈腰背挺得更直,意气风发,而上首的宰辅们则一个个都沉默起来了。

只有王韶偏帮着韩冈,开口打破沉寂:“这板甲果然不在札甲之下,就不知成本是多少?”

“这样的一副板甲,连人工带材料,初步估算当不会超过十五贯,多半在十贯上下。即以十五贯计,将六十万禁军尽数换装,就只要九百万贯。分作三年的话,一年只要三百万贯就够了。”

十五贯!而且还是最多!

尽管一众君臣,已经确定板甲要比札甲便宜许多,但当真听到这个数字时,还是免不了要吃上一惊。

寻常的札甲三四十贯绝对是少不了。可韩冈的板甲,比旧制札甲更为坚实也更便宜。

而即便给禁军全体换装的九百万贯,真要咬着牙,一年也能拿得出来。如果按韩冈所言,分作三年四年轮班换装,就更算不得什么了。

这可是给朝廷省了大钱了。

要知道,即便是在现在,经过精简的五十八万禁军,也不是全数都有铁甲可以装备。赵顼一直想要做到的兵利甲坚,已经做好了往甲胄制造中陆续投入数千万贯的打算。板甲一出,至少省下了一半的投入。而锻打技术的进步,同样可以用到其他兵器上。比如如今正在大量制造的斩马刀,也有好几道工序能用得到锻锤。同样能节省大量人工。

看过了板甲,赵顼也无心再观灯饮酒:“韩卿,你回去后尽快定下板甲的式样规格,上报于朕。”

韩冈起身:“臣遵旨……臣恳请陛下,于监中成立板甲局,调集各作匠人,专门打造板甲。以免受到外物所扰。”

“可!”赵顼点头。韩冈打算设立板甲局,是为了避免板甲的制造,受到军器监中其他势力的干扰。就像为了制造斩马刀,也专门成立了斩马刀局一样。人们对变化的抵触心有多强,赵顼如何会不清楚,暗箭难防啊。

“板甲局可依斩马刀局例,于局中设令、丞各一。”赵顼继续说道,“还有今次打造板甲的六名匠师,你可将他们的姓名,与请设板甲局札子,及板甲式样规格一并呈上。此等有功之辈,朕必不吝厚赏。”

“臣遵旨。”韩冈躬身领命。

“至于铁船……”赵顼沉吟了一下,“板甲为朕省下不少钱粮,拿出一部分也不妨,朕还希望能看到更多不逊于板甲的发明。”

……………………

又喝过两巡酒,上元之宴宣告收场,天子摆驾回了后宫。

随着净鞭声响过,御街之上的彩灯灯山登时全数熄灭。原本城中最为明亮的去处,转瞬就陷入了黑暗之中。就像是一片浮云,漂过来遮住了天穹一角的群星。

韩冈随着人流,一起下了城头,送了天子,方才散了班次。

吕惠卿走过来,说了两句闲话,很有风度的恭喜了韩冈。而冯京等人则早就各自上马,走了一干二净。

王韶留在最后,等吕惠卿也离开了,才走过了:“玉昆,陪着我走一走。”

“敢不从命!”

韩冈说着,随了王韶在出了宣德门,骑上了马。

连串的马蹄声得得的响着,王韶说道:“今天做得不错,冯当世、吴冲卿有苦难言。蔡持正也是说你是明修栈道、暗度陈仓,兵法用得好。”

微皱起双眉的韩冈似乎是有点纳闷:“蔡副枢这话是怎么说的?”

“《浮力追源》都写出来的,人人都以为玉昆你要造铁船,宣扬格物之说,想不到一转就变成了打造板甲。”

“格物致知那可是韩冈毕生所愿,怎么都不会放弃的。”韩冈解释着,“板甲仅仅是附带而已,终究还是要造铁船的。”

“但铁船不是非有一二十年之功就造不出来吗?今夜之事传出去,玉昆你必然名望更高一层,但于格物之说上,可就一点也占不到好处啊。”王韶眯着眼睛瞥了韩冈一眼,“还是说玉昆你还有什么谋算没拿出来?”

“说要造铁船的不是我啊!”韩冈呼呼的笑了起来,“这些天来,枢密可曾见到韩冈有一字上请?!”

“……那铁船彩灯到底是怎么回事?”王韶沉默了一下便又问道。

“不过有人想为难在下而已,唆使了两个胆子大的出头。不过我也不打算深究究竟是谁在背后捣鬼,过两天就推荐那两人到春州、雷州的弓弩院去,就说他们铁船彩灯造得好!想必宰相、参政们当不会反对。”

王韶微微一愣,转又呵呵笑了两声,问道:“……如果中书驳回来,玉昆你是不是准备递到御前去?”

韩冈笑而不语。

他无意深究幕后黑手,甚至还在御前将责任一起担下——尽管这个责任对他来说根本无所谓——但不代表韩冈愿意咽下这口气。现在不趁此时,挑两个不长眼的出来杀鸡儆猴,试问如何镇得住军器监中?

王韶瞥着韩冈凝在嘴角、微显冷酷的笑意,暗叹了一声。

官场上的俗谚:‘春、循、梅、新,与死为邻;高、雷、廉、化,说着就怕。’虽说岭南是贬斥之所,但北方人去了琼崖【海南】,反而不一定有事,倒是稍北的春、循、梅、新、髙、雷、廉、化这八州,瘴疠遍地,去者往往无北返之望。

但韩冈下手如此之狠,也怪那两人自己,如果不是韩冈有后手,必然逃不了谪官出外的结局。

“自食苦果,怨不得他人。不过这事放在一边,”王韶眼神犀利,盯着韩冈,他可还没老糊涂呢,能被人随便把话岔开还不知觉,“玉昆,你到底还有何盘算?”

韩冈粲然一笑:“就让韩冈稍稍卖给关子如何?反正很快就会知道的。”

王韶再盯了韩冈一眼,也不再追问,摇头笑了一笑:“那我就看看玉昆你还能带来什么惊喜了……”

