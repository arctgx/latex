\section{第一章 一入宦海难得闲(四)}

【第一更,求红票,收藏】

韩冈当日说的话尚掷地有声,王厚当天午后,就跟着王韶去了古渭寨——王韶名义上是去确认最近已经有大战迹象的硕托、隆博二部的动向,而他的本意则是对李师中、窦舜卿、再加上个向宝三人的得意嘴脸,来个眼不见为净。王.克臣和李若愚那两人的证词已经早早到了东京城,与其心惊胆战的等着发落,还不如继续做事省得自己胡思乱想。

等到了十天后,当王厚跟着父亲在古渭寨转了一圈,发现硕托隆博两家当真要打起来后,再赶回到秦州,走进勾当公事厅时,便看到了一群小吏聚在一起,把韩冈的桌案堵了个严实。

王厚走近两步,就听见韩冈在里面一一发落着,房子漏了、地板坏了,韩冈已经让一个木工专门等着为各曹司服务;想调出架阁库存档,须呈上主官亲笔;家里分派的老兵手脚不稳,韩冈答应为他们调换;马厩最近用得草料不好,害得马都瘦了——

“请回复刘参议,衙中马房最近所用刍豆都是上等,两个马夫也同样勤力,其他马匹皆养得膘肥体壮,只有参议的一匹马变瘦,当不是马厩的问题,在下会帮参议找个马医来的。”

韩冈就这么一个一个的把人打发走,后面又不断有人进来,而他手上的公文批改检查却没有停过。在韩冈身边的一个食盘里,放了碗益气补中的香薷饮子,就看着他在说话之余,时不时端起来喝两口,看起来仍是游刃有余的模样。

等着围住韩冈的人群稍少,王厚才怒意深重的走上前:“这是怎么回事,怎么还是玉昆你一人在做事?!其他四个人呢,空领俸禄不成?!”

“处道你回来了?”韩冈抬起头,立刻就要起身相迎。

王厚却不理这么多,拉着韩冈又坐下,道:“玉昆你前日不是说不能再一个人做五份工了,怎么现在还是没变?!”

“没办法。”韩冈摊开手,很无奈的模样:“另外四位抚勾,两位告病在家,两位奔走在外。这几天还是只有小弟一人。若是有人回来,只要一天,小弟就往甘谷城去视察疗养院之事了。”

“那两个痨病鬼究竟得了什么病,多少天还没好?!要不要准备身素衣服给他们送行!?”

“处道兄误会了。”韩冈笑着,一边指了指手上公文上的一处,对旁边的一个小吏说了声‘这边错了,赶快去改’,转过头来,一边又解释道,“前些天是相抚勾、小刘抚勾生病,大刘抚勾和曹老抚勾奉命出外办事,这几天,则是大刘抚勾、曹老抚勾生了病,相抚勾和小刘抚勾出外……”

“这有什么区别?!”王厚怒道。

“当然没有任何区别。”韩冈说得很干脆。

前七天是甲乙生病,丙丁出外,后七天是丙丁生病,甲乙出外,窦舜卿和李师中这摆明是要跟自己过不去,只是这种手法很幼稚,也太保守,不符合韩冈对两人的认识,但韩冈对窦、李手法的评价,不会解决自己现在的处境。

韩冈的差遣虽然是勾当公事,但还有一桩是兼管路中伤病事宜,完全可以以后一桩为借口,把管勾公事的活计给推掉。就像王韶虽然是经略司机宜文字,但他基本上不做机宜文字方面的事务,而是处理他的兼差,提举秦凤西路蕃部事宜,并提举秦州屯田、市易。

在王韶的计划中,韩冈作为他的助手跟着他跑,而韩冈的打算也是先跟王韶在秦凤西部缘边各寨堡走一圈,然后在古渭寨建立疗养院,为下一步打基础。但当王韶和韩冈想做自己的正事时,李师中和窦舜卿却先下手为强,让韩冈一时之间离不得官厅。

韩冈清楚这并不是他们真正的杀招,李师中和窦舜卿也不是要对付自己……很明显的,他们目的不是为了自己,而是自家身后的王韶。既然要对付王韶,他们的手段就不会那么简单。现在不过是先挑挑刺而已,真的动起手来,就会一锤定音。

‘可是要定音,不是已经定了吗?’韩冈还是想不透,一万顷变成一顷四十七亩,而一顷四十七亩变成零,王.克臣和李若愚的结论传到京城,如果王安石不保他的话,王韶只有丢官去职一个结局。这一招已经够狠了,再画蛇添足也不会更增添整垮王韶的几率。

“玉昆!”

韩冈在沉思中被王厚一声惊醒,抬头一看,王启年站在自己面前,又呈上来一大摞公文。

韩冈看了看公文的厚度,问道:“就这么多,没少吧?”

衙门中的胥吏,最常用的欺瞒上官的做法就是将一些有关碍的卷宗藏起,使得一些案件失去证据,而胜负颠倒;也有更胆大的,干脆私刻了大印,模仿长官画押,自己做了知州、知县,去给那些他们受到贿赂的案件判状。

不过,韩冈的这个勾当公事厅只是个转发和检查机构,厅内胥吏隐藏公文,对韩冈的影响并不大。他也只是多口问一句。

王启年很恭敬的回答道,“回官人的话,就这么多。”他的姿态,竟比七天前老实恭顺了许多。

这种姿态的转换,里面是否拥有诚意,韩冈全然持否定的态度,只是没有表现出来。他对王启年一直保持着冷漠,指了指桌案:“你就放在这里。”

王启年依言放下一叠公文,躬身退下。见他退开后,王厚就在韩冈耳边低声说道:“玉昆,你要小心一点,他不是好人。”

“多谢处道提醒。”韩冈点头谢道,虽然这些他早就打听到了,不过王厚的关心,是必须要感谢的。“小弟知道,他过去跟陈举走得很近。”

王启年是市井无赖出身,又素无品行,身上还背着命案,但他在经略司衙门中说话够份量,跟陈举走得近也是情理之中,另外还有一种说法,就是王启年十几年前能进经略司,还是陈举的功劳。

陈举垮台,他在秦州城中各处衙门的眼线耳目却都还在。虽然韩冈可以确信,他们没有帮陈举报仇雪恨的意思。但究竟是哪些人,他却要做到心里有数。这种想法很早就有,韩冈也着力打听,王启年的名号也是他在去京城前就听说过了。

王厚则是听得糊涂,“玉昆,我说他不是好人,是我前些日子看见他跟窦解走在一起,去逛了惠民桥后的私窠子。”

“窦解?是窦家的哪一位?”这下轮到韩冈糊涂起来,他一时间想不起来这个人物究竟是何方神圣。

王厚提醒道:“是玉昆你去京城的前一天,在惠丰楼上与刘走马喝酒时,遇上的那一个,窦家老七,窦解。”

“啊!”得到提示,韩冈恍然,“原来就是那个涂脂抹粉的!”

“对!就是他。王启年就是领着他去了惠民桥后。”

“王启年陪着窦解去逛惠民桥后,这事处道兄怎么知道的?该不会也去逛了吧?”

韩冈看似毫不在意的开着玩笑,心中却在惊奇,王启年竟然会跟着窦解那个三世祖?

……………………

就在当天夜中,白天被韩冈和王厚所提及的王启年和窦解两人,正躲在惠民桥后的一家上等的娼馆中,窦解抱着个艳娼,上下摩挲着——虽说娼妓并称,但实际上妓是卖艺,而娼才是卖身——而王启年站在他身边低声说着话:

“想不到韩抚勾还真是能撑,都半个多月了,还是稳稳的滴水不漏。在州衙里面,可是有不少人在赞着他的手腕过人。”

窦解的脸色顿时就像挂了层霜,右手便在一团丰盈中用力一捏,惹来一声竭力忍住的痛叫。窦解一脚把那艳娼踢走。当房内只剩他和王启年两个人时,他狠声道:“那是谁也没有认真对付他!家祖本是想先从那灌园小儿下手,再去对付王韶,这事还跟李经略商量过。只不过现在王韶都成了过街老鼠,马上就要丢官去职了。家祖就没心思去动那灌园小儿,才让他得意到现在。”

“小人也听说过,经略相公私底下都想把灌园小儿千刀万剐。”王启年眼睛转了转,诈了窦解一句。

窦解的心里藏不了秘密,听王启年一说,便点头道:“谁说不是,上次李师中和家祖见面,他可是明说韩冈是王韶的爪牙,必先废掉不可。”

“照小人说,李经略只想着扳倒王机宜,至于韩冈不过是条虫子,想捏死就捏死,他当然不会放在心上。不过韩三前次太过欺辱衙内,还是一把捏死他比较痛快!”

窦解突然觉得王启年他太热心了一点,“王启年,你跟灌园小儿有什么仇?”

王启年心中一跳,忙赔笑道:“小人不也是为衙内生气嘛。灌园小儿身上的粪臭都没洗干净,哪比得上衙内这等世家子弟。他欺凌衙内,任谁看到,心里都会生气!”

“说的也是!”窦解点着头,“说得好,说得好。”

王启年心中暗暗冷笑,窦家的这个衙内,真是够蠢的。不过也幸好他够蠢,才会这么听自己的话。挑拨了窦解出头,动手的只要不是自己,韩三就算能脱难,日后报复也到不了自己头上。

想起韩冈,他心中就恨。他这些年省吃俭用才结余下两千多贯,都投在陈家的质库里吃利息,想等着过些年老退之后,就可以拿这些钱回乡买个大宅和十几顷田,做个富家翁。谁想到,韩三那灾星一动,什么都没了……

王启年心中正在恨着韩冈毁了他的大宅、田地,耳中却传入了让他大惊失色的一句话。

“既然你为我生气,那你就把韩冈往死里掐。你们做胥吏的,不是很有手段吗,实在不行,把架阁库烧掉也行,那里正好是他管。烧了后,他肯定要吃罪。”窦解不聪明,所以他会把所有的事都推给其他人做,并认为他人为自己做事是天经地义。他为自己的妙计哈哈大笑,一见王启年没有及时点头答应,便又生气起来,“怎么……你不愿意?!”

王启年却是目瞪口呆,许久都说不出话来。。

更正公告:记忆果然不靠谱,前面信手写下来‘王.克臣、李若愚两个阉宦’,回头一想,宋廷怎么会为一件事同时派出两个宦官?重新查了一下,其实王.克臣不是宦官,而是开封府判官,而李若愚才是。

