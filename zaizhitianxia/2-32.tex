\section{第九章 长戈如林起纷纷(五)}

【第二更,求红票,收藏】

韩冈突然叫破了青唐部众人隐藏在心中的秘密。厅中忽然静了下来,薪炭时不时在火盆中噼啪作响,沉重的呼吸声在厅内回荡。

青唐部的首酋们被韩冈的视线一个个扫过,仿佛被狼盯上的兔子,很不自在的在座位上扭着身子,低下头避过他过于锋利的目光。

只是当韩冈将眼光重新投到青唐部族长身上时,却是为之一怔。

俞龙珂神色太自然,心平气和的模样,仿佛韩冈方才只是在说,他的弟弟踢翻了别人家的马桶,捣坏树上的鸟巢那样微不足道的小事。

瞎药肯定与董裕有所瓜葛,从首酋们的表现中能看得出,决不会有错。但韩冈叫破此事,就如天外飞来的一剑,首酋们的反应才是正常的,而俞龙珂却没有任何变化。如果他能把心情掩饰得这么好,自然他的城府也不可能是普通的水准。

既然如此,俞龙珂又怎会被瞎药所欺?城府、心机、才智,以及自我控制的意志,都是有着千丝万缕的联系,很少有人会只有其中一项出色,而其他几条是瘸腿。俞龙珂何能例外?

韩冈突然警觉起来,他方才有些太小瞧俞龙珂了。青唐部的族长若是连他的弟弟都对付不了,凭什么能在宋、夏、木征、董毡四家之间玩着平衡游戏?

宁可把对手想得聪明一点,总比被人扮猪吃老虎强。

韩冈心中的一番变化只是在一闪之间。俞龙珂缓缓开口,却是推搪之言:“不知官人为何提到我家那个不成材的弟弟?”

韩冈笑了起来,话锋试探着俞龙珂:“本官听说令弟瞎药平素里都有一番振作之心,希望能光大青唐部。今次董裕入侵青渭,不知令弟会不会出兵对付董裕,亦或是等董裕满载而归,再去接收七部空下来的地盘?”

韩冈的一番话说得不算委婉,但对付蕃人,不得不直接一点,若是把官场上绕着弯儿说话的习惯带过来,人家还不一定能听得懂。不过韩冈也没有直指董裕敢侵犯青渭,是因为早与瞎药有所联系,那样就是撕破脸的说法,会让俞龙珂下不了台。

俞龙珂脸色却突然一变,让人吃惊的叫起苦来,“官人有所不知,我家的那个弟弟自幼不听管教,我这个做哥哥都拿他没办法。如今也分了家,各自过各自的。今次正是我的这个弟弟被董裕引诱,让我难以出手相助。不是我不想帮着赵官家啊,实在是我那个弟弟……唉!”俞龙珂摇头叹息,毫不介意的把已经被韩冈看穿的底牌丢了出来。

‘脸变得真快,果然不好对付。’

韩冈看着俞龙珂七情上面的表演,抛弃底牌的决断,发觉前面自己的推断都是太自我了,根本没有从俞龙珂方面的利益去考虑。他前面是觉得已经把利害关系都说清楚了,俞龙珂怎么也该表示一下。但自俞龙珂的角度来看,自己大概都是说着些空话而已,没有点实质。

大概因为王韶的事有些昏了头,要冷静,韩冈提醒着自己。

虽然他猜到了瞎药给董裕说动,但这只是误打误撞,而且也不是俞龙珂拒绝出兵的真正理由……不,俞龙珂他肯定心动了,不然不会开始叫苦,他现在不答应,只是他想要的更多——俞龙珂的一个承诺不是买不到,只是韩冈的价钱出得还不够高。

但韩冈并没有出价的权力,王韶也不会给他这个权力。韩冈能动用的,只有对青唐部未来的许诺,希图籍此来打动俞龙珂:“不知俞族长有没有听说过千金市马骨的故事?”

俞龙珂茫然摇头,他官话说得好,但对汉人的历史了解却没多少,当然不会知道,但他知道这必然是韩冈做说客的手段,“是用千两黄金买马骨头?”他满不在意的问道。

“这是一千多年前的事了。”韩冈站在厅中,给俞龙珂和青唐部的首酋们讲起了故事,“当时中原四分五裂,共有七个大国互相征战,都想着一统天下。在中原东北,也就是如今辽国所据有的地方,有一个燕国。这燕国不比现在的契丹,是个兵力微薄的小国,但他们的国君却又想着统治天下,所以想着对外招揽人才。”

“可堪用的人才不是那么好找,所以燕国国君向自己的一位老臣征求意见。那位老臣便说,大王不如把高官厚禄都给我,既然我这等庸才都能身居高位,那自认超过我的贤良,当然会来投奔大王。”

“这跟千金买马骨的有什么关系?”俞龙珂突然插话,他听得有些不耐烦了。

俞龙珂的反应在意料之中,韩冈正是要磨磨他的性子,他微笑着继续说道,“因为那位老臣跟燕国国君也说了个故事:过去有位国主想要买一匹千里马,他派人拿着千两黄金去买,但买回来的却是一堆马骨头,国主要治使者的罪,使者却说世人看着大王既然愿以千金市千里马骨,那自然愿意用更多的钱来买活生生的千里马,还请大王稍等一段时间,自然会有人来卖。果不其然,没两个月就有人带了三四匹千里马来售卖。

燕国国君由此被老臣说服,给他极丰厚的赏赐,并筑起了一座黄金台来安置天下贤才。而天下人才果真都纷纷来投,燕国由此而强盛。”

俞龙珂听完故事,皱着眉问道:“官人是想把我青唐比作马骨?只要青唐部能投靠大宋,就会像着燕国的那位老臣一样,被高官厚禄的的赏赐?”

“不!”韩冈摇头否定,“七部才是马骨头,而青唐部以及河湟诸部则是千里马。今次本官来向族长求援,就是想让河湟诸部看一看,只要亲附皇宋,我们绝不会把他们抛弃!”

“官人有所不知,我家的弟弟暗中助着董裕,青唐部内有许多人也向着我那个弟弟。而且现在部中的钱粮又不足,不是我不想出兵,实在是出不了兵啊……”

俞龙珂跟方才一样,依然叫着穷、叹着苦,为青唐部和自己的窘境摇头叹息,仿佛一个穷人在向自己的富亲戚叹着今年的年关过不去了,伸出双手求着援助。

他静等着韩冈的回答,他当真心动了。虽然今次董裕搅乱了青渭的局势,但也给了青唐部混水摸鱼的机会。而且令他想不到的是,宋人竟然又为七部求上门来,这样的好事其实俞龙珂期盼已久。不过既然宋人要青唐部出兵,怎么也得给点实在的,光是空口说白话如何能引人出动,即便是钓鱼也得在钩子上刮饵吧。

俞龙珂还记得少年时,跟随父亲去青唐王城拜见赞普,在湟水边看到了渔民,为了捕捉自青海逆流游进湟水里的那些一人多长的湟鱼,他们可是把大条大条的羊肉挂上钩子。

‘钱、粮、土地、官职,你能给什么,我就要什么。既然你有求于我,那我就不会客气。’俞龙珂坐得安安稳稳,他不愁韩冈不答应。

韩冈悠悠叹了口气,摇了摇头,很是无奈:“既然如此,那本官也只好告辞了。”他说罢,行过礼,转身就往外走。

虽然韩冈很想说服俞龙珂,援救附宋七部。但他前面的一番言辞可能是给俞龙珂和青唐部留下一个错误的印象,好像他是非救七部不可。

这可是大错特错!

前面韩冈也说过了,如果救不了亲宋七部的话,那也是没办法的事,等刘昌祚回来就给他们报仇好了。

韩冈不徐不急的往厅外走去,厅中鸦雀无声,只有他的脚步踩在没有拼接好的地板上吱呀作响。

既然卑躬屈膝的求你,你都不肯答应,那我便掉头就走。想趁机喊高价,笑话,我有必要为了七部的死活毁了自己在国中的名声?——若是在请援之事上许诺太多,事后王韶高遵裕必然反口不提,而他韩冈也肯定要受到责罚,官场上说不定还会留下一个韩三哭虏廷的笑话。

七部安危事关朝廷脸面,这样的话不过是说说而已,说客的口吻罢了。也许七部覆灭会影响到王韶的声望,但终究不会有多大的干系,毕竟王韶背后的靠山是大宋。

而七部蕃人的死活,更是与韩冈毫无瓜葛。就算见了王韶,一句‘韩冈有负所托’也就过去了。王韶难道还能治他的罪不成?

韩冈走得干脆无比,毫不拖泥带水,一点迟疑也没有。

俞龙珂本料韩冈是故作姿态,安心坐着,等着他回头。可韩冈出了厅门,出了宅院大门。继而听着外面了来报,古渭来的韩官人已经骑上马,带着随从要出城去了。

十几个首酋齐齐望着俞龙珂,他们都没提防韩冈如此果断,说放下就放下。如果韩冈负气而走,那就当真把人得罪狠了。王韶聚七部灭托硕的事历历在目,得罪了他派来的说客,对青唐部可不会是件好事。

俞龙珂还在犹豫,他还是想赌韩冈是在装模作样,但又一名亲信跑了回来,“秉族长,韩官人已经出了城门了!”

俞龙珂脸色大变,当真是把人给气走了,他连忙道:“韩官人奔波了一夜,哪能就这么走了,快请他回来好生歇息,省得外面说我青唐部不懂待客……”

“不!”俞龙珂猛的跳起,推开报信的亲信,来不及穿鞋就直接跑出门去,他要亲自把韩冈请回来。

