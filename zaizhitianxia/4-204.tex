\section{第32章 荣辱凭心无拘执(下)}

一宿无话,到了第二天清早,韩冈便升堂理事。

转运使的工作是‘总一路利权归上,兼纠察官员以临郡’,也就是处置一路租税、军储,以供邦国之用、地方之费,同时还有监察地方官员的职责。

其实说起来,转运使就是个劳碌命。一年至少有半年要在外面巡历州县,审核帐册、检察积储,对路中的官员依功过进行奖赏或处罚,做得尤其出色或是特别不堪的,更是要上书朝廷,加以推荐或申请贬斥。而衙署中的庶务,基本上是由留驻于治所的转运副使和转运判官代劳。

故而韩冈第二天的工作,基本上就是走过场,也就是引见一下衙中官吏,并对漕司衙门辖下的仓储情况有一个大概的认识,并没有花费韩冈太多的时间。

初来乍到,衙署中的公事并不会一起堆到韩冈的面前,否则就是挑衅了。不过该让他过目的也有,方兴做为韩冈的助手,先行给出几个参考意见,最后交给韩冈处断。还有一些小事,韩冈门下的幕僚也可以给挑起来。作为都转运使,他本来就只要统筹全局就够了。

熟悉了一下衙门中的人和事,韩冈就准备退堂,他昨天已经派人通知了程府,说自己今天要来拜侯,没有必要多耽搁时间。

列于堂上的官员按着官品高低,从低到高依次告退。一名四十多岁的中年官员这时忽然站了出来,向韩冈拱手行礼:“运使,下官有一事容禀。”

韩冈看了一下这位突然出声的官员,是转运司的管勾帐司,姓孙名霖,在转运司中主管财计,算是仅次于转运副使和转运判官的属僚。

孙霖当着众人的面请求留下来说话,这与崇政殿君臣议事后,宰执中的某一人自请留对一样犯忌讳——谁也不知道他这么做,是为了构陷谁人——当然,这也是表明亲附韩冈的立场最为简单直接的手段。

可惜的是,韩冈并无意用破坏规则的手段来开拓班底,既然文彦博正摆明车马的跟他过不去,多少只眼睛盯着自己,凡事就要做得光明磊落。

“可与李副使有关?”韩冈问道。

李南公闻言身子一震,孙霖连忙摇头,“没有。”

“那就好。”韩冈点头道,“楚老请留步。本官初来乍到,衙署之中尚显生疏,孙帐司所欲陈达之下情,可能还要劳烦楚老。”

在孙霖站出来后,李南公和其他官员都想着要尽快离开,尽管他们都不直孙霖的为人,但走得慢了,更是危险。谁也没想到,韩冈竟会出言将李南公留下来。

李南公愣了一下,转身犹豫的瞥了韩冈一眼,又看了看孙霖,遂走回来,在自己原来的位置前站定。

看到转运副使和和韩冈的亲信方兴一起旁听,孙霖脸色变得有些难看,但韩冈不让他拖延,“好了,孙帐司,现在有什么事可以直说了。”

孙霖犹豫了一下,最后一咬牙,“下官前日点检路中账籍,却见河南府中公使钱开支数目,与前账不合,其中似有隐情。且河南北关诸仓仓储之数,两账亦有所不同。下官恳请运使将之彻查。”

“……上一次检查河南府库帐籍是什么时候?”想了一想,韩冈问道。

孙霖一愣,不知韩冈为何问出这个问题的时候,李南公就在旁插话道:“是在去年的九月。”

“才四个月。”韩冈沉吟了一下,“依序下一次检查河南府库帐籍又该是在何时?”

“应该是在明年仲春。”李南公已经明白了韩冈的心思,回答得飞快。

“明年……”韩冈一笑,转头看看提议的孙霖,“还有问题吗?”

孙霖脸色发白,连连摇头。虽然李南公说的顺序其实是京西北路的顺序,而韩冈担任的是京西路都转运使,南北两路合并,这个巡查的顺序是应该加以调整的,从洛阳起头理所当然,他既然如此提议,自是有所准备。但韩冈的态度已经十分明确了,他哪里还敢有问题?

韩冈并不知道洛阳这里的府库到底亏空了多少,但有问题是肯定的。任何账目,都不可能挑不出错,无论前世后世都是一样。韩冈如果抱着找茬的心思去查洛阳府库的账,必定能找出一堆错来。

可眼下的情况,就是他查出实实在在的亏空来,报上去都是他韩冈借机报复文彦博的失礼,无论有理无理,落在世人眼中都是他的器量偏狭,日后有的被人说道。

就是眼下应当轮到查验河南府的府库帐籍,韩冈也会设法拖个一年半载,何况洛阳帐籍库存刚刚经过点验不久,他又怎么会追上去赶着要查?韩冈可不会拿自己的名声去交换河南府账目的明白清楚,想想自己岳父最后受到的待遇,韩冈还没有对皇帝忠心到那等地步。

韩冈如何会做这种蠢事?!

明年开春再查也不晚,到时候文彦博填补不上亏空,韩冈自有应对。

示意孙霖可以走了,再让方兴送了李南公离开,韩冈起身穿过大堂后方的小门,向后院走去。

孙霖这话说的并不是时候,要不然,韩冈也不会如此对待这位来输诚的官员。他也希望能及早在转运司中收服一两个可供驱使的手下。可惜他犯了大错,韩冈的报复心并没有他想象中的那么强烈,更确切一点,是韩冈不想让外人认为自己的报复心强烈,只能拿孙霖来做个靶子。

对此,韩冈也不觉得可惜,愚蠢的同伴比敌人更为危险。不懂得察言观色,只知道奉承上官。这样的一位官员,如果能留在敌方的阵营还好说,自己若是将他留在身边,只会害人害己。

其实如果没有文彦博弄出的这档子事,韩冈的确是打算从洛阳先查起,查过之后,两年内就可以少往洛阳来。他其实有想法,将转运司的治所暂时移到汝州或是唐州——洛阳毕竟离着渠道太远,往来并不方便。

虽说方城那里有沈括先行勘察,将设计蓝图和沙盘模型先弄起来,眼下是开春,什么工役都不可能开始,但韩冈接下来的半年多时间,还要检查襄汉漕渠沿途的各州各县,将需要调动来清理河道、修造轨道的物资、资金和人力,给筹备起来,时间也是很紧张。而且一旦开工,韩冈就必须去当地坐镇,谁也不知道只靠沈括一人,工役的现场到底会出什么事情来。

……………………

程家眼下还很平静。

尽管前一天韩冈刚刚抵达洛阳便遣人送来了诸多礼物,但在程颢、程颐眼中,都不如韩冈在拜帖上写下学生韩冈顿首再拜的字样。

——韩冈对于程家的尊敬才是真正的礼物。

同时河南府官员在文彦博的影响下,都没有出城去迎接新上任的京西转运使,这一件事,也传到了程家。

程颢程颐都对此不以为然,文彦博这么做,说难听点,就是小肚鸡肠,不像是宰相所为。就算其中有几个陷阱,可只要韩冈一切做得光明磊落,所谓陷阱对他来说,就是大道坦途一般。

不过二程与文彦博也有几分香火情在。前两天文彦博还请了他们的父亲去参加同甲会,给足了两人面子。

程家在勋贵遍地的洛阳城中,只能算是寒门素户。二程的父亲程珦,做了一辈子的官,还比不上韩冈的三五年。如今致仕在家,也只有一把年纪可以称道。也就是靠了程颢、程颐的声望,让他可以与文彦博一起饮宴终日。

“其实当年子厚表叔在洛阳讲学的时候,也是靠了镇守河南府的文潞公。”程颢开口与兄弟程颐议论着,“不管怎么说,文潞公对子厚表叔有着一份宣名举荐的恩德在,今天还是得劝一下玉昆,让他不要与之争执。这样即能还了子厚表叔旧时的举荐之恩,对玉昆本人来说,也是同样不损声名。”

与总是带着温文笑意的兄长不同,程颐永远都是板着一张脸,严肃无比:“文潞公做得差了,韩玉昆则不能跟着一起错。他来了之后,当时要劝诫一番。”

程颢点点头,韩冈怎么说都是他的学生,不能看着他做错事。与文彦博这位元老为了点面子斗起来,世人只会说韩冈有错,年轻的官员在老臣面前,本也没有面子一说,不敬老,那就是错!

已经过了中午。韩冈昨日约定上门拜访的时间是在午后。程颐放下了手中的书,看看外面的阳光:“韩玉昆快要到了吧?”

程颢摇摇头:“还没听到喝道声呢!毕竟是第一天升堂,也不知要耽搁多久。”程家并不富裕,宅院狭小,坐在家中的后院,都能听见前面大街上来往宅邸的货郎唱着的太平歌。

但程家的司阍这时匆匆的从前院跑了过来,连一向提醒他的礼仪规矩都忘了一干二净。在大程小程两位名儒面前直喘着气:“小韩龙图到了,正在外面求见。”

程颢程颐也没能料到,韩冈并没有穿着紫袍金带,也没有带着他的旗牌,而是换了便服,轻车简从的抵达了程府门前。

