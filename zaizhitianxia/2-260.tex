\section{第37章 青山声碎觑后影(三)}

珂诺堡。

要调两千人来,到手却只有一点。这样雁过拔毛、拦腰斩一半的手段,让韩冈想起了那些向下分派赈济灾款的小吏。

沈括克扣军粮倒也罢了,怎么克扣起人力来了?

但禹臧家两万大军压境,也让韩冈能稍稍体谅沈括的压力。

姚麟的急报就是从珂诺堡这边传递去前线。不过两万兵马,坐拥险地坚城的姚麟并不是镇压不住,但临洮堡城下的两万禹臧军之后的意义,却是让所有人都忍不住要皱起眉头。

禹臧家能出动的兵力是有限的,今次出现的两万兵马就算是其中有很大一部分是亲附的其他蕃部人马,但对于禹臧家来说,也已经是竭尽全力了。之前禹臧花麻从来没有如此不顾后方过。禹臧家与兴庆府的微妙关系,熙河经略司早探听得明白。禹臧花麻能放心的倾巢而出,不用说,已经从梁氏兄妹那里得到了足够的保证。

而且禹臧花麻和木征之间的联系也紧密得很,宋军的主力刚刚抵达河州城下,禹臧军就出现在洮水之滨。看起来在宋军的庞大压力下,禹臧家和吐蕃王家,兰州和兴庆府,其固有的矛盾已经趋于弥合。

姚麟能不能守得住,韩冈不担心,就怕党项人再插一脚下来。

韩冈收起掩在心头的忧虑,问着奉命前来的前广锐军指挥使,“刘源,伤势怎么样了?”

刘源躬身行礼:“回机宜的话,小人的伤不碍事!”

王舜臣与刘源前后脚进来,大摇大摆的坐在了韩冈身边,“俺肚皮差点都被破开照样吃喝跑跳,身上多一两个洞而已,屁大的事。”

“罪囚不敢与都巡相提并论。”刘源半弓腰,声音生硬。

王舜臣眼眉一跳,就要发作。韩冈轻咳了一声,冷淡的横了他一眼。让熙河都巡检干笑了两声,又安坐了下来。

“禹臧家也是让人头疼。”韩冈屈指敲了敲桌子,“最怕党项人也来凑热闹,偏偏梁乙埋真的可能要出洞来了。”

“……难道此前没有准备?”刘源轻声问着。

韩冈摇了摇头,“准备和预案都有,只是不想用到而已。”

至今为止,秦凤、泾原都没有总动员,派往熙河来的虽是精锐,但本路中的守备依然足以进行一场大规模的战事。一旦真的遇上党项来袭,沈起、蔡挺都会下令出兵的。

另外熙河路本身也只出动了不到八千的兵马,实际上在各个兵站、寨堡都有足够的守备兵员。且因为农事的原因,屯田的各保甲也没有全数征发。如果党项人当真来袭,熙河路本身也足以抵挡,甚至击败之。

只是一旦与党项人开战,消耗的钱粮将是一个个让人心惊肉跳的天文数字。河湟开边并不是击败木征就结束的,接下来还有震慑蕃部、清理余党;修筑道路、寨堡;移民、屯田、市易等一系列的工作,若是没有后方的钱粮支持,所有的规划都要落空。至少要耽搁上一年的时间。

“还是盼着王经略那里早日取胜为是。”韩冈真心企盼着。

“迟个两天其实也不打紧啊,”王舜臣半开玩笑半认真的说着,“等俺伤再养好些再说。”

韩冈又瞪了王舜臣一眼,“这边也有些事要你来帮个忙。”他对凑近上来的王舜臣说道,“香子城这两日都有上报,说北方的山间发现了好几批吐蕃人的哨探,请求我这里多派人手过去以防万一。”

“香子城?”王舜臣惊讶的问着,“怎么不是珂诺堡?!”

“是啊?”韩冈阴冷的笑着,口气让人听起来却不知是不是在诘问,“河州也有小路通珂诺堡,为什么吐蕃人的哨探尽是在香子城出没,而不在珂诺堡周围打探?”

“难道想调虎离山?!”王舜臣立刻反应过来。刘源心有同感,在旁点着头。

“不知道!”韩冈却摇着头,“人心隔肚皮,木征的想法,我们坐在几十里地外怎么可能臆测的准?……如果想深一层,万一这是木征故意要让我们这么去想呢?那该怎么办?”

“这?……”王舜臣和刘源都有些不知该说什么好。事情若是如此反反复复一层层想下去,就没有个终结了。

“不过若是我等没有余力,也只能在两种可能里挑上一个来防备,但眼下可以不一样。”韩冈胸有成竹。虽然只是多了一千来人,但他手上可以打出的牌却多了一倍,“管他有什么计策,都防着就是了。珂诺堡是要地,难道香子城就不重要了?”

王舜臣听出了眉目,问道:“难道要小弟去?”

刘源则在同时上前半步,动作像是在毛遂自荐。

“若王兄弟你不去,我就想自己去香子城看一看了。”韩冈话声一顿,看着王舜臣和刘源,“刘源,你在珂诺堡待命,随时准备出发。王兄弟,现在你先去香子城看个究竟,过两天如果没有问题的话,你再去河州城下的经略那里报到!”

……………………

临洮堡外。

禹臧花麻望着并不高峻的临洮堡,正犹豫着是不是要让他麾下的士兵作出攻城的准备。

虽然吐蕃人不擅攻城,但他正面对的临洮堡也并不是什么坚城。周长不到五百步的堡垒,挤进三四千人很是勉强,要说什么防御体系,那根本是个幻想。

刚刚筑好的城池其实很是脆弱,夯筑得再结实,其实还是因为含着大量水分而在外力的作用下显得容易松塌。只有过了几年后,墙体逐渐风干,才会变得越来越坚硬。

毕竟不是所有的城池都像赫连勃勃命人筑统万城那样,让士兵用铁椎来验证城墙的质量。椎进一寸杀工匠,椎不进一寸则杀士兵。这样的统万城,历尽千年而不倒。如此高标准的工程要求,新近完工的临洮堡可做不到。

姚麟也知道情况会这样,才率两千主力在堡下结阵,隔着濠河与禹臧家的两万大军对峙着。城头城下都有战士手持硬弩严阵以待,攻来的敌军会受到上下的两重打击。

温祓望着两里之外的敌军很久,这才转头问着沉默了同样时间的禹臧花麻,“是攻城,还是照着原计划行事?”

“结河川那里听说也已经筑起了堡垒,绕不过去。”禹臧花麻的声音中有着悔恨和遗憾,“我还是太小瞧宋人的筑城能力了。”

“刚刚修起来的结河川堡不会太结实,宋人的兵力现在也当是大半留在这里。”

“后面还有临洮……不,是狄道,这里才是临洮。”禹臧花麻的话透着他对宋军情报的深刻了解,“狄道城里肯定有兵,也许会被前后夹击。”他望着远近山色,一个月前的融融嫩绿,已经渐渐化为深色,“瞎药【包约】那个被汉人养起的狗也在附近。”

禹臧花麻扬起马鞭,遥遥一指临洮堡,“就攻这座城!”

……………………

河州城下。

周围厮杀声震耳欲聋,每一刻都有临死前的尖号传入耳中。长箭四处横飞,马蹄声碎乱,浓重的血腥味让人如同浸身血海。身处在战场之上,身后战鼓一刻也没有停歇,但赵隆却停了下来,皱眉看着左手中的铁简。

不知击碎了多少敌军头盔和下面的头颅,也不知敲断了多少条手臂和肋骨,赵隆所用的铁简上亮晶晶的一片,原本留在上面的斑斑铁锈也全都被磨掉了。

可这柄重达六七斤斤、握把处径圆近寸的四棱重兵器,现在在中段,竟然已经弯了一个很明显的弧度出来。

是前面夯死那个穿着党项瘊子甲的吐蕃将领时坏的,还是撞上那名同样拿着铁简的吐蕃勇士交击时坏的?

赵隆正皱眉想着,身后一阵急促的蹄声接近,还有周围亲兵们的惊叫。

听着风声,更不回头。赵隆身子一侧,反手猛力一挥。一支枪尖从左肋外刺空穿过,而左手的铁简却正正敲到了实处。

咚的一声闷响。熟悉的反震并没有传入掌心,只觉得手上一轻,半弯的铁简啪的断裂开来,但被敲中的头盔也彻底瘪了下去。

带着嫌恶的侧头看了一眼想偷袭他的蕃人,赵隆就见到一颗眼球带着鲜红的筋肉悠悠在黑洞洞的眼眶上晃着,从瘪掉的头盔接缝中,混着血液的灰白色浆体缓缓流下。

这幅画面只是一闪而过,转眼间,那具尸体就被胯下的战马高速的带着向前奔驰而去。

就在支流河谷的谷口前,失去了冲力的两支马军正处在一片混战之中,让每一个习练兵法的将校都会为之摇头叹息的混乱中,神力惊人的赵隆,他身边已经没有多少蕃人敢于近前。

从马背上射过来的长箭,如果落点不是无甲的要害,赵隆根本就不去理会。若是奔着面门而来,随手就用右手的铁简给挥开。战马披着一条防箭的毛毡,头面处也罩着一层皮套,连同战马上的全身甲胄的赵隆,这一人一骑仿佛鬼神一般让人畏惧。

用半截断简砸下了一名蕃军,赵隆换上了长枪,提枪上指,他提声大喝。

吼声传遍了四野,没人听清他的吼着什么,但由他起头,重新奔驰起来的一队熙河选锋已经在横扫战场。

