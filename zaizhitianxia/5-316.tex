\section{第31章 停云静听曲中意(23)}

“特免死了。”耶律俨在析津府城中的一间不起眼的小屋中低声说道。

同在屋中的几人都没有太大的反应,甚至连惋惜和感慨都看不到,脸上只有麻木。

冬捺钵布置在析津府城外,耶律乙辛和他所立的幼帝都在城外的帐篷里面住着。析津府虽贵为南京,其实并没有南朝那般为一路核心重镇的作用,皇帝和权臣无事都不会进城居住。

不过这样一来,城中的守卫也就理所当然的松弛,这让许多不得不在阴影下做事的人们,有了更多的活动空间。

“这是第几个了?”萧信义拨了一下火盆中的炭火,骤然跳动起来的火苗映得众人脸上忽明忽暗。

“早就数不清了。”“耶律俨叹着气,“我大辽立国百多年,想不到到了如今,忠臣孝子连个立足的地方都没有了!”

表字特免的萧兀纳在辽国朝中很有威望,与耶律乙辛向来不合,不过在耶律乙辛弑君之后,这些大臣或贬或逐,然后一个个的死个干净。萧兀纳不是第一个,也不会是最后一个。

“不过现在国人都是畏其权势,不得不俯首低头。不过只要有一次惨败,胡睹衮那老贼就别想坐得安稳了!”耶律那也冷然说道。

“西平六州能算吧?”坐在角落里面的刘伸兴奋了起来,“失土之败,百年来何曾有过。”

屋中唯一的汉臣,在整个南京道的汉官尽数投入耶律乙辛怀抱的时候,一个汉臣还能忠心于先帝,却是极为难得。

“那个不能算。”耶律俨摇头说道,“迁移到西平六州的都不是他的心腹人,穷迭刺的儿子将他们丢到那里,是为他的宫分看守门户。丢了西平六州,不过死了些不算听话的狗,大门外的狗屋给占了罢了。宋人敢沿着黄河去攻打黑山吗?还是说河东的宋军敢北上?!”

“遂城那边不是也败了一场。”刘伸在朝中一直都是以忠义正直而著称,一提到叛逆的惨败便难以遏制心中的兴奋,“打得真是好啊!”

“的确是吃了小亏。但抢回来的人口早就抵得过了。”萧信义泼上一盆冷水。

“不能这么算吧?”刘伸争辩道,“听说还是与宋人在城外野战时惨败的,菩萨保和敌古烈两人手中兵力还比遂城的兵马多。”

“听说没用,拿回来的是真金白银和活蹦乱跳的生口。”耶律俨叹气道,“只要主力不损,谁能奈何得了他?”

“……要不要跟宋人联络一下?”刘伸迟疑了一下,然后又小声说道。

“军行在外,就是胡睹衮那贼子也不可能知道大军当夜走哪条路,住在哪里。怎么跟宋人说?若仅仅是大军出动或是别的消息,宋人自己就能打探到了。”耶律那也哼了一声,他叔父是前任的北院大王,在在座的人中,他的地位最高,“谁能保证宋人不会一转眼就将我等卖给胡睹衮?汉……南蛮子我可信不过!”

“虽然不好办,可终究是一个办法。”刘伸坚持道,“胡睹衮手中可以作为依仗的军队不过数万,剩下的都是不得不听命行事。黑山下的宫帐、南京道的汉军,上京道宫分军一部,还有西京的皮室军一部。若是能除掉这几部胡睹衮的心腹兵马,到时候只要有人能站出来,必能一呼百应。”

他看看左右,“难道还要等到他篡位不成?向他三跪九叩,山呼万岁?”

都到了这个时候,国中大部分人都在等着耶律乙辛他篡位了。或许还要经过一个禅让仪式,不过有与没有基本上都一样,现在的这位年幼的新帝,虽说声称是宣宗皇帝的遗腹子,所以继承了侄儿的位置,可辽国国中没人认为这会是耶律洪基的种。从伪帝手中接过皇位,自然是个笑话。只是一旦他当真篡位,地位必然会比现在更加稳固。

房中一时无言,每一个人都沉默了下去。

“哦?……倒是长进了不少了。”

半日后,耶律乙辛在城外的大帐中放声大笑。一阵畅快的笑声之后,他又冷下脸来,“想不到宋人的手都伸到了南京转运司了。”

“下官事前也不知晓,也多赖了尚父的洪福,让这个奸细自己跳出来。”耶律俨低垂着头,不敢稍抬。

“这件事就交给宰相来处置吧。”耶律乙辛想了想,

“下官明白。”张孝杰上前拱了拱手,笑答道,“尚父请放心,必会给南人一个惊喜。”

耶律乙辛又又对耶律俨道,“你继续在里面打探,若能将这群贼子一网打尽,我必不吝赏赐。”

耶律俨这名细作躬身退了出去。

“怎么会变成了这个局面?”

耶律乙辛轻声一叹,如果能事先知道,他绝不会去破坏好不容易达成的和约。可惜现在后悔已经迟了,丢了贺兰山下的西平六州,那些先帝的‘孤臣孽子’便蠢蠢欲动,说不得,只好拿起屠刀了。

“虽说没预料到宋人会这般强硬,但幸好事先也有所准备,多亏了尚父的深谋远虑。”张孝杰讨好的笑着。

耶律乙辛摇摇头,却没有半点笑意,“等结果出来了再说吧。”

“那个……”张孝杰又犹豫的问道,“萧敌古烈和耶律菩萨保该如何处置?”

耶律乙辛想了一阵,最后一摆手:“罢了。让他们戴罪立功吧……他们送回来的银绢子女都分赏下去,也免得有人啰嗦。”

“是。”张孝杰领命。收了分赏下来的战利品,自然不会有人要将两人重重治罪了。毕竟是耶律乙辛看重的人才,能保自然是尽量保。

“再去信跟他们说,没事别往宋军军阵上冲,教训了多少次,都白教训了。”耶律乙辛怒意上涌,“一群记打不记吃的夯货!”

……………………

郭逵正忙碌着,距离受到遂城捷报已经两天了,突入河北境内的辽军受到了强烈的阻击,攻势并没有太大的进展。受害的村落乡镇虽多,但终究没有让其深入河北内地。

这几日,宋辽两军在保州、定州大小百十仗,有败有胜,但几座城池依然安然无恙,而雄州、霸州那边,辽军更是没能突破三关防线。

辽军的攻势远比预计的要软弱,事前预测辽军并没有做好大规模战争的准备,现在看来是正确的推断。仓促进兵,自然不会有太好的结果。没有充分的准备,就想打进河北内地,那可是个彻头彻尾的笑话。而且据细作回报,耶律乙辛自开战后就坐镇在析津府,半步不敢南下。

郭逵忍不住都要冷笑,耶律乙辛太过小瞧人了。真宗时辽人犯界,可是太后和皇帝亲征的,这才打到了澶州。耶律乙辛担心背后,不敢出动,这前面的兵马又怎么可能有太多的信心?

河北的军队纵然再糜烂,囤积在边境上的兵马也不是可以轻辱的。任何一座军州武库中的兵器甲胄,都能装备上万人马。保甲法更是训练了乡中丁壮。一旦朝廷召集忠义乡兵,边境上转眼就能多出十万兵马,岂是旧年可比?

眼下当然还用不着召集乡兵,只要河东兵马照计划从太行山方向,抵达了真定府,直接就能配合保州、定州的守军将南侵的辽军给歼灭。

郭逵嘴角抽了一下,就是不知道到底是什么时候能到了,按理说早该来了……

门外的脚步声由远及近,郭忠义匆匆踏进房中,兴冲冲的说道:“大人,河东的兵马到了!”

郭逵神色一缓,轻松的微笑出现在脸上。河东的援兵比想象的来的还要迟,迟了最少两天。不过终究是来了:“终于是到了。”

“多谢韩学士在河东斩首数万的功劳啊!”郭忠义撇了一下嘴,“他任用的将帅全都给调走了,补上来的都是颟顸无用之辈,只迟了两天,还算是好的了。”

“别乱说!”郭逵冷喝一声。次子聪明外露,喜欢招摇,让他很是担心。远比不上长子那般省心。

郭逵的长子郭忠孝,旧年曾参赞军务,为机宜文字,但此时并不在大名府。郭逵将他留在了东京,让他好好读书准备考进士。之前已经失败了两回,这一回据说是很有把握了。军功再多,也比不上一个进士及第,若是郭忠孝能金榜题名,一点军功又算得什么?

而且对辽的这一战,郭逵也没有完全的把握。万一这一战打得不好,帅司行辕中的大小官吏都会受牵连,鸡蛋不能放在一个篮子里,郭逵更不会将一枚天鹅蛋放进破篮子里。

不过现在看来,次子也可以挣一份军功了。

站起身,郭逵笑道:“现在我们就看一看,怎么让客人都留下来吧。”

……………………

“那就是雁门关?”黑夜中,一人远眺着群山间一道黑黯幽沉却有着数百灯火妆点的暗影。

另一人在他身边低声道:“正是雁门关。”

“看来路没白走。”前一人低声笑了,又是一声轻喝,“把雁门关攻下来,为尚父祝寿!”

两人起身而行。

紧随在两人的身后,陆陆续续从山林中蹿出的身影,竟有数百上千之多。一支突然出现的军队,就这么在夜幕中,向北直行而去。
