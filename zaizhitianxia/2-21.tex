\section{第七章 惊闻东邻风声厉(下)}

【第三更,求月票,收藏。中午的章节名写错了,那一章是第七章(中),这一章才是(下)。】

韩冈的脸色也变了,连忙接过王韶递过来的急报。低头匆匆看过,死的竟然不仅仅是种詠这个庆州东路监押。

王韶捻着手上的笔杆,眼神深沉:“钤辖李信、都巡检刘甫违节制,斩。都监郭贵,迁延不进,流,种詠是东路监押,也被瘐死在狱中。一路钤辖都给他杀了,李复圭的胆子还真是包了天去。”

“那是他有事想遮掩,才顾忌不了这么多。种詠被瘐死,怕也是他暗中下得黑手。”韩冈拆穿了李复圭的用意,便沉默了下去,双手紧紧握拳,许久之后,才长长的吐了口郁气,“李复圭做得太过分了。”

虽然他与种詠只是在长安道上匆匆一会,话都没说几句,没什么交情,但他跟种建中和种朴是一见如故,也算是自家人了。而且当日种詠也是一副意气风发,正欲为国建功的模样,谁想到转眼之间就是阴阳相隔,韩冈也免不了有些兔死狐悲的感慨。

“争功诿过的事,世间难道还少吗?”王韶脸上的笑容越发的冰冷刺骨,“想想窦舜卿,他前几日病得多及时!?……不过李复圭也的确够狠,把人都杀了灭口,这回谁能知道这一战到底是怎么一回事?究竟是李信、种詠他们不听节制,违反命令,还是他胡乱下令,令得战线崩溃?”

李信……这个被斩的庆州监押的名字,让韩冈想起来自己的表哥来。这个名字重复得还真是不吉利。

韩冈现在都有些庆幸,李师中只是添乱而已,而李复圭这等货色,却是功劳要独吞,过错却要推干净,而且真出了事,还不给人对质的机会,直接杀人灭口——真是够黑的。

“错误都是下面的,功劳都是自己的,李复圭杀了李信、刘甫,流了郭贵,顺便把种詠丢进狱中暗中害了,他倒是把自己都摘得干干净净,就只要负个管束不严的责任。”韩冈不能不佩服,王韶看人的确有一手,“他当真是没担待,机宜看得真是准……”

“也不是我看得准,谁不知道李复圭这厮从来都是没担当,他过去……”王韶吐了半句后,却把话咽了回去,摇摇头,又不说了,反而对韩冈道:“玉昆,你以后做官也得小心点。”

“多些机宜提醒,若真遇上了这样的长官,我会尽量绕着走的。”

王韶笑了起来:“我是说李复圭算是果断了,见事机不妙便杀人灭口。玉昆你平日行事也是果决无比,就是日后可别变得跟他一个模样。”

“……论起下手快,我只会在李复圭之上。但说起没担当,他的本事我怎么也学不来。”韩冈脸色悻悻,真不知道王韶平日究竟是怎么看自己的,才会说出这番话来。

王韶见韩冈神色不渝,笑着安抚道:“我也是担心玉昆你的性子。只是这么一说,玉昆你也别放在心上。”

‘我要真的把罪名栽给人,也不会做得这么难看。总得把人卖了还帮我数钱才是。’韩冈咳嗽两声,把话题转开:“庆州的钤辖,监押,都巡检等一众将佐不是被杀就是被流,庆州那边如今怕是没人敢带兵了。”

“李复圭一口气杀了这么些将领,一两年内,庆州军心都别想稳下来。环庆是缘边四路的中段,秦凤也好,鄜延也好,还有泾原,接下来都要被庆州拖累了。”

韩冈点点头,同意王韶的判断。说实话,无论宋夏,两边都是奸细一个接一个的往对面派,对面有个风吹草动往往都瞒不过去,庆州如今给李复圭搅得天翻地覆,党项人不钻空子才有鬼,“日后西贼很可能会拿庆州做突破口。无定河被绥德城堵上了,甘谷这边又建了城,如今党项人南下,最好走的就是环庆路的马岭水这条路了。”

“那就看新任的韩宣抚会怎么处置了。他身边不会缺参谋,我们能看出来的,他当然也能看出来。我们现在可没空替他人担心。”王韶一转变得忧心起来,高遵裕总是来催促,虽然能体现出他对河湟之事的支持,但也是一个不好的苗头,“高公绰那边也不能一直搪塞下去,不然迟早会出问题。”

韩冈当然能看得出来,高遵裕的耐心也是有限的。若是不能给他一个满意的交代,说不定他就会和李师中去合作。

“就不知钱粮什么时候会有着落。”韩冈心里其实跟王韶一样急,但有些事心急也没用,“屯田要人要粮,市易要钱要物。李师中拿着这些卡脖子也不是一天两天了,以前难做,现在还是难做,高遵裕真的想早点见功,不是来催我们,而是去找李师中要钱……”

“对了,机宜!”提到高遵裕,韩冈就是灵光一闪,他向王韶建议道,“能不能让高提举想想办法。实在不行让他跟官家叫几声穷,也许能从内藏库里挖点钱出来。以高公绰如今的急切,跟他说一声,说不定转眼就能帮着把钱粮都筹备好。”

“你能保证转运司和李师中不雁过拔毛?”王韶反问了一句,却立刻又摇头苦笑道:“算了,就算给飘没个五成,好歹还能落下一半来。二哥今次去京城,也是要钱要物,我本也是只想着能有一半拿到手就不错了。”

王韶派儿子去京城,还有个任务就是要钱。没有钱粮,王韶怎么开拓河湟。就像后世机关里,控制不了财权的领导,说话都没人理会。

“高遵裕的事我来处理,不管他从哪里想办法,我只想看到真金白银。什么时候钱物能到帐,什么时候就可以开始做正事了。元瓘现在在外面跑,已经联络了不少商户,一等榷场开启,市易之事立刻就能运作起来。”

听王韶这么一说,韩冈这时才知道为什么这几天都没看到元瓘那个还俗僧。

王韶又道:“王舜臣那边就有玉昆给他说一声,他跟种家情谊匪浅,种詠出了事,总得跟他提上一提。”

“此事不必机宜说,我也准备请他今晚到家里喝顿酒了。”韩冈叹了一声,“说真的,这事还真难开口。”

结束了一天的差事,韩冈回到家中,便让李小六去请了王舜臣过来喝酒。

王舜臣跟韩冈是一起上过阵,出生入死的交情。但自韩冈从京城回来,事务繁芜,两人就没有坐在一起好好喝过酒。今天听了韩冈的邀请,王舜臣便很高兴过来做客,还带着一篮子白杏做礼。

王舜臣到了韩家后,先拜见了韩冈的父母,然后在小厅中坐了下来。严素心精心地弄了一桌酒菜,两人一坐定,便一道道的端了上来。

王舜臣夹了块油泼兔,丢进嘴里嚼着,含糊不清的笑着:“三哥你家这个厨娘请得好,人长得俊俏,菜也做得比酒楼都好,该不会当日在药房外见到她的时候,就存了心思吧?”

韩冈看着王舜臣哈哈的开玩笑的样子,心中不忍。犹豫再三,还是将种詠之事跟他说了。

王舜臣自幼跟着种朴做伴当,种家上上下下没有不认识的,感情也很深厚。当他听到种詠被李复圭害死,就一声怒吼,一拳砸坏了韩冈面前的桌案。

碗碟丁玲桄榔的碎了一地,韩云娘在外面听到声音,忙赶了进来。看到王舜臣面目狰狞,拳头上都是鲜血的模样,吓得捂住了小嘴,差点叫了起来。

韩冈挥挥手,示意小丫头出去。严素心这时端菜上来,见到王舜臣这般模样也吓了一跳,放下菜,回头就端了一盆净水过来帮忙处理伤口。

而王舜臣这边,就见着他狠狠地骂着,“李复圭那狗官,犯在爷爷手里,直接就割了他鸟……头。”

王舜臣的声音到后面,都变得哽咽起来。不断用手抹着脸,不想让别人见到他哭的模样。

韩冈知道王舜臣的心情不好,等严素心把他的伤口处理好了,便把他引到书房坐定,让严素心端了凉茶上来,坐下来慢慢劝解。

可王舜臣却什么话都听不进去,红着眼狠狠地说着:“大郎、二郎、五郎他们不会看着四郎就这么白白死了,这个仇肯定要报!”

韩冈暗暗摇头,现在种家担心自己还来不及,还是先自救再说吧。

自来都是贼咬一口,入骨三分。李复圭给参战的一众将佐都栽了不听节制的罪名,当事人全都死了,就一个是流放,这件公案可以说已经定案了。即便是种诂、种谔,都没法给种詠他们翻案。

谁叫李复圭是文官!别看现在王韶骂着李复圭,一旦种家要为种詠申冤,他绝不会站在种家的一边,最多也是两不相帮。

而种詠的罪名既然定下,一旦有人想攻击种家,都会拿种詠出来说事。无论是种诂、种谊还是种谔,如今都得考虑着自保的问题。

在韩冈看来,种家将想洗脱李复圭栽给种家的罪名,就不得不拼命了。不多上阵杀贼,在天子心中,种家将就会始终跟不听节制,致使官军大败的种詠联系在一起。

韩冈知道最后会是什么结果,也不好劝王舜臣放宽心,最后只能道:“天道好还,报应不爽。李复圭的所作所为,日后总有回报他的时候。”

