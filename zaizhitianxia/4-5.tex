\section{第一章 纵谈犹说旧升平(五)}

韩冈被领进吕府的花厅之中,吕惠卿以参知政事的身份降阶相迎。

人在家中,吕惠卿也不会穿着紫袍金带,而是简简单单的道服荆簪。立于阶下,风仪绝世。非是相貌,而是清雅淡泊的气度让人一见便心生钦慕。也就是。今之贤人,

见及于此,韩冈连忙快步上前,“韩冈拜见参政。”

吕惠卿亦是快行两步,将拜下去的韩冈一下扶起,有几分嗔怪的说着:“玉昆,礼法岂为我辈所备?”

“韩冈可不敢当。”韩冈谦虚了一句。说着又向一起迎出来的吕升卿行礼问候。

等三人将表面上的礼节尽到,互相之间的寒暄说得也是到位。韩冈与吕惠卿一起携手走近厅中,仿佛两人之间一点芥蒂都没有,完全是情谊深厚的至交。

坐了下来,待吕府的下人送上了茶汤,韩冈这才收起了客套,直言问道:“参政的信笺,韩冈已经看到了,不知李逢一案,究竟有何急状,竟惹得参政漏夜招韩冈过府?”

吕惠卿叹了一声,正容道:“玉昆,你可知道此案又牵连出了何人?”

韩冈看了看一边端端正正、一言不发的坐着的吕升卿,再瞅瞅吕惠卿,心如电转,试探的问道:“该不会是家岳吧?”反正绝不可能是自己,他一个三代务农的灌园子,在官场上可没那么多能够株连的关系。

“玉昆果然一猜便中。”吕惠卿了不以为异,他都这个态度了,韩冈猜不出来才怪了。

“究竟是何人?!”韩冈有些纳闷。

王安石与赵世居毫无瓜葛,而李逢……他是曾任秦州知州的滕甫的内兄,与范仲淹也有亲戚关系,就是跟王安石拉不上钩。要是能查出关联,早就传出来了。

吕惠卿没有卖关子的想法,若是做了反而有失他参知政事的身份:“是李士宁!”

“……那个假道士?”

韩冈不动声色,口吻中还语带戏谑,可是心中已然明了,这件事的确会有些麻烦。因为那位李士宁,是王安石家的座上宾。据说身怀异术,也会写诗,所以能在京城中的士大夫里颇吃得开。

在熙宁初年王安石还没有进京之前,就已经与其有过一段交往,王安石还为他写过诗。而等到王安石为相,李士宁还在相府之中住过半年,与王雱兄弟也有点交情。而韩冈不喜佛道二教,本身又不会写诗,虽然见过李士宁的面,当初与王旖成婚时也收了他的礼物,却根本就没怎么搭理过他。

不过也仅仅是麻烦。在韩冈想来,光凭一个李士宁,此案很难将王安石也拖下水。吕惠卿未免有些大惊小怪了。

“假道士?”吕升卿出言反驳,似乎是在彰显自己的存在感,“玉昆,李士宁可是有着度牒的!”

韩冈失声笑道:“所谓度牒,片纸而已。拿着两三百贯买了度牒,可就当真能成为佛门弟子,老聃传人?”

之前他无意与苏颂争辩。不过在眼下的场合,在言辞上,他则不愿落上半点下风,得磨到吕惠卿将他的真实目的给说出来。

见到弟弟和韩冈斗起嘴来,吕惠卿则是悠悠然的喝起了茶,停了一阵,才慢慢地说道:“李士宁是否是假道士故且不谈,但他与介甫相公却是脱不开干系。审案的沈存中是个软性子,而范百禄是范镇的侄子。恐怕有伤。”

“即便李士宁当真涉案,不还有邓文约在。由他主持,何须担心?”

韩冈说的似乎是傻话。在座的三人都清楚,在王安石和天子之间,邓绾会选择谁那是不需要多问的。邓绾这位曾经放言‘笑骂从汝,好官须我为之’的御史中丞,之前一直紧随王安石,是因为天子希望新法不受干扰。

有件事必须要清楚,御史的任命与宰相全然无关,是御史中丞、侍御史和翰林学士共同举荐,其主要目的就是为了限制相权。邓绾能做到御史中丞的位置上,不是因为他亲附新党,而是他亲附新党这件事让天子满意。

吕升卿呼呼笑了起来,“邓文约可不会为介甫相公说上半句好话。”

但吕惠卿绝不会认为韩冈的问话之中含着傻气。当韩冈将视线投过来,他便慢条斯理的端起茶盏,“李士宁涉案,如其确系叛国大罪,当依法.论断。”

韩冈微微一笑:“家岳最重法度,必不会为私谊而坏国法,更不会包庇叛国重罪。”

“有玉昆的话,那我就放心了。”

“参政当比韩冈更为熟悉家岳,有参政在,家岳在江宁也可以安心了。”

李士宁一案,很难动到王安石身上。无论如何,这一案仅仅是赵顼的发泄之举,而不是改变朝堂政局的风向标,如果当真被牵扯到前任宰相的头上,如今声势浩大的李逢、赵世居谋反案,都会嘎然而止。韩冈对此心知肚明,难道吕惠卿会不明白?

吕惠卿急着找他过来说一段废话,这是在以协商、妥协的姿态来表明态度,缓和两人之间紧绷的现状,改变过去疏远得近乎于敌对的行为。至于王安石因李士宁被牵涉进谋反案,仅仅是个借口,韩冈都无意细问,只是笑道:“不知冯相公会不会想趁势掀起一番波澜来。”

“这是肯定的。不过天子聪明英睿,不会偏听偏信。”

与聪明人说话当然让人轻松,只是韩冈反应太快,也让吕惠卿心生忌惮。自家的兄弟此事还是懵懵懂懂,吕惠卿虽然也不愿将自己的退让,给弟弟看出来——同样也是这个道理,他并没有请章惇同来——但吕惠卿也是免不了有着恨铁不成钢的叹息。

在因为之前招揽不成而两人变得生疏之后,吕惠卿终于决定调整对韩冈态度。就像吕惠卿不能将章惇当成自己的门下走卒来使唤一般,以韩冈如今的成就,加上天子的信任,也足以当得起政治盟友这个身份。

虽然对过去之事心中犹有芥蒂,可韩冈既然表现出了足够的实力,那么就没必要再纠缠于旧怨。携起手来,眼望未来那才是最好的做法。无论如何,对于双方来说,对方都不是亟需击败的敌人。

“但天子对冯相公始终信任有加。”韩冈说着,“许多事,天子都会咨询冯相公的意见。”

“有王禹玉在,冯当世怎么能比得过他?”

“说起天子信重,东府之中,无人能及参政。”

“玉昆你何曾稍逊。”吕惠卿笑道:“尊师张子厚,能教出你这位佳弟子。子厚与我份属同年,当年在新科进士之中就已博通经义,深悉礼法而著名。”

“只恨韩冈所学不能及先生之万一。”

吕惠卿抿了一口茶:“去岁郊天大典,礼仪上有多处不尽人意,天子有意将宫中礼乐重新修订。”

韩冈叹了口气:“只恨家师如今多病,教书传道之余,已无力多涉其余。否则考订礼格,必能让天子满意,士林信服。”

“听说冯当世可是格物致知四个字听着就头疼。”吕惠卿半开玩笑的说着。

韩冈笑道:“冯相公这些日子倒并没有在军器监的奏事上有所刁难。”

之前冯京、吴充与自家为敌,是因为他露出了破绽,给两位宰辅看到了机会——更确切点说,他们以为他韩冈露出了破绽。但眼下,既然自己以《浮力追源》一时名满天下,在上深受天子信任,在下也已经稳稳的控制住了军器监的局势,无懈可击。冯京、吴充两人,都不会蠢到再将目标放在自己身上,而只会是在政事堂中试图把持大权的吕惠卿。

吕惠卿微皱了一下眉,话锋一转:“如今诸法皆备,但丁籍产簿已经多年未有修订。若无五等丁产簿为凭,赋税难以收取,而任何法令也都难以实行。只是眼下一旦修订,定会有人作伪,使得乡宦得利,而小民遭受刻薄之苦。”

“可是手实法?”韩冈早就听说吕惠卿想要做什么。

两人间的话题兜兜转转,终于说到了正题上。不过这样才对,作为政治盟友,尽管高下依然有别,但两方之间的关系是靠了利益交换来维系,而不是赏赐和奉承的关系,只是看起来倒像是市井贩夫之间的讨价还价——虽然本质也的确是一样。

“如果让百姓自报,必然会有人行奸……参政是不是准备奖励首告之人?”

“自然。”吕惠卿轻飘飘的回到,毫不在意这句话所代表的意义。

韩冈忽然觉得,吕惠卿是不是在摆脱王安石的阴影上走得太远了一点。虽然吕惠卿方才已经表明了为了维护王安石会不遗余力的态度,但眼下,他明确的说出要推行新的法案,韩冈免不了要怀疑起他到底有多少是厌倦了王安石得力助手这个身份。

“奈何世人贪利者为多。”

“朝中自会遣人去各路监察,清理其中弊端。”

“参政,可是有市易法在前。”韩冈提醒着吕惠卿,手实法可是与市易法一样,都是要耗费大量政治资源的法案。

吕惠卿双眼盯着韩冈,眼神一下变得犀利起来:“……陛下是支持新法的。”

