\section{第19章 萧萧马鸣乱真伪(十)}

李铁脚哈哈狂笑。

骑兵追杀步兵所在多有,可步卒追着骑兵砍杀的情况,却是难得一见。

李铁脚麾下的五百甲兵,身上的铁甲、陌刀足足有三四十斤重,只能快步行走,连跑步都做不到,可他们现在却偏偏是在追砍着转身逃窜的铁鹞子。

一口气攻过来的铁鹞子为数太多,尽管其中分了批次,也在两侧留下足够回转的空当。可两侧的山上站着宋军的弓弩手。当前阵没有冲破宋军防线,准备转回为后阵空下位置的时候,却被乱箭阻住。而后面又急速涌来,顿时就堵在了山口附近。而山上的宋军弩弓手们,这时候也聪明的纷纷转移目标,向拖在后方的铁鹞子射击,拦住了他们的退路。

进退两难的党项骑兵,如同放在砧板上的鱼,被乱刀剁砍着,血液飞溅起来,让谷中泛起了暗暗的红。

就在李铁脚指挥麾下的陌刀手斩杀着铁鹞子,宋军的后阵也迎来了党项骑兵的冲击。

按说党项人已经吃足了神臂弓的苦头,不会傻乎乎直奔严阵以待的宋军杀过去。可后阵的阵型乍看上去并不齐整,仿佛是仓促之间才成型的队列,看到这一幕,这一队骑兵便毫不犹豫的直冲而上。

看着敌军越冲越近,已经进入了射程范围,可把守后阵的指挥使仍然没有下令射击。而是让他麾下的战士们端着手上的重弩,继续在等着机会。一直等到冲到了只剩五十步的距离,最多三五次呼吸就能杀到眼前。这位指挥使才向下一挥手。

第一排开始射击,箭矢齐刷刷的飞了出去,立刻就人仰马翻,让五六名最前面的骑兵重伤倒地。当后续的骑兵从前方的阻碍中脱离,想要继续前进,立刻迎来了第二排的箭雨。紧接着就是第三排,第四排。这一个四百多人的指挥被分做了前后五排,让弩手们在射光箭矢之前绝不会停下张弓搭箭的动作。而为了给后方空出前进射击的位置,这一个指挥的阵型也的确看上去是有些不整齐。

突前、殿后加上护卫折可适的中军,这三个指挥都是麟府军中的精锐,只有跑上山的两个指挥稍差一点。不过能被选上成为前锋军中的一员,也不是普通的队伍。

李铁脚正带着追砍着埋伏在支谷中的铁鹞子,山坡上两个指挥的弩手正帮着他阻拦敌军的逃窜;偷袭后阵的一队骑兵,则被神臂弓射得连头都抬不起来,只能选择返身逃回。

折可适这时已是在望着道路的前方。就在后阵刚刚响起弓弦声的时候,派往前方探路的游骑回来了。只回来了一人,在肩膊上还插着一支雁翎箭。

就在提着陌刀追看铁鹞子的将士们,在谷中髙呼着胜利的时候,前方又出现了一队骑兵,大约五六百骑。在欢呼声响起之后,快速的冲刺一下便缓了下来,继而慢悠悠的停止了前进的步伐。

这本是完美的三面夹击,三路攻击之间只差了片刻。不论以哪一国的标准,都可是说是同时了。只是那两路败得太快,如果计算交战的时间,也就是一个回合的样子,几乎是甫一接阵便被击败,让三面夹击蜕变成了各个击破。

但折可适的心情没有放松下来,再看到最后一支骑兵的同时反而一下抽紧了。

与之前的铁鹞子相比,最后出现的这一队骑兵给人的感觉完全不同。不是说装束、战具和骑乘的马匹,就是连前进的队形,马蹄声的节奏,都与党项骑兵有着如同府州、丰州之间一般遥远的距离。

出现在眼前的这一支队伍,他们的消息早就已经传开来,攻入丰州的宋军上上下下都做好了碰面甚至交手的心理准备,只是突然之间遇上,还是有些难以适应。

“是契丹……”阵列中突然一声惊叫,但立刻就断了声音。

折可适麾下的将士们刚刚击败了铁鹞子的兴奋,就在片刻之间冷了下来。与百多步外的敌军骑兵互相望着,场中一片死寂,气氛如同绷紧的弦,似乎下一刻就会迎来石破天惊的爆发。

折可适举起了手,随即,把守前阵的弩手们将已经张好弦的神臂弓举了起来,空气仿佛凝固了。

折可适心如电转。山谷中的士兵们已经得到了消息,在他传过去的命令下正在做着调整。虽然此前短暂却激烈的战斗消耗了他麾下军卒们大量的体力,可就算那一队契丹骑兵攻过来,只要他肯付出代价,折可适有自信能将这一个对手彻底击破。

但对面的骑兵并没有跟着铁鹞子一样试图直接冲击宋军阵列,而是犹如狡猾的豺狼一般远远的观察着战阵。

呜的一声唿哨响起,一支只有十几骑的小队突然向前猛冲,飞动的马蹄像是要直攻上来。当宋军弩手们的食指即将扣下扳机,那十几名骑兵却准确的踩在神臂弓有效射程之外,停了下来,并不再接近,而是兜转了回去。

‘这是在试探!’折可适眼神更加凝重。

只过了片刻,又是一队冲了上前,还是冲到一半就转回,只是这一次回转的地方又近了几步。

当契丹骑兵第三次冲来,又在九十步左右的地方回转,身在箭阵中的弩手们稍稍起了一点骚动。

折可适咬了咬牙。这样一次冲击,只要再来几次,下面的士兵恐怕就会忍不住收紧自己的手指。

他看看支谷之中,铁鹞子已经远远逃离,陌刀阵也退回了,在谷口内侧重新列队,山坡上的两个指挥则同时做好了调整。至于后阵的士兵,完全不关心背后发生了什么,一心一意压制着对面的骑兵,将他们越射越远。

折可适低头吩咐了一句,让传令兵立刻上前。

契丹骑兵又将那套把戏重复了两次,每一次引起了更大的骚动。如果是党项骑兵来,没人会在乎。偏偏换作是契丹人做来,就硬是能让军心有些许动摇。

就在这时,一支由神臂弓射出的木羽短矢,嗖的一声急速的穿过了七十步的距离,准准的就要射中一名正对着过来的敌骑。可那名骑兵只将手上的铁锏毫不在意的随手一挥,当的一声,飞过来的短矢便落在了地上。

但一箭之后,豺狼一般的敌人却不再上前。对峙了一阵,就只见战旗摇动,转头就窜进了另外一条谷道,消没在山路上。

折可适憋得胸口发痛的一口气,这才长长的舒了出来,声音大得如同在高喝。不,并不是他发出的声音大,而是他身边的兵将都在同时吐气。

‘果然还是不一样。’不只是一个人在这么想着。

“是前面斥候说的那一队契丹人吧?”李铁脚凑了过来,他的神色间已经没有了方才将来袭的铁鹞子砍得大败而逃的得意和轻狂。

这最后一队敌骑,来的雷声大、雨点小,没有动手就跑了。可偏偏没人觉得这一队骑兵是因胆怯而退,也不会认为他们会就此不再出现。

“幸好我们赢得快。”折可适紧绷着脸,依然没有松弛下来,“若是在我们与铁鹞子纠缠的时候,这一队骑兵突然杀出来……”

听到折可适的话,李铁脚只在脑中想了一想就打了个寒颤,难怪铁鹞子会不管不顾的直冲军阵,就是为了给契丹骑兵开道的。要是自己手脚慢了一步,这一战当真不知道会怎么样了。

“看他们的气势,该不会是辽主斡鲁朵的宫分军吧?”他轻声问着。

折可适翻了翻白眼,“辽主的御帐亲军怎么可能随便出来?天子不亲征,班直什么时候离开京城过?”

“可是……”李铁脚舔了舔嘴唇,他也是大胆的,并不是害怕对手,而是直觉着那一支骑兵不好对付,“怎么看都不是一般的骑兵啊,还是说契丹的骑兵都有这般气势?”

折可适、李铁脚,都是惯于上阵,厮杀也不知有过多少场。对手到底能不能打,从行动上就能看出个端倪来,那几百名骑兵绝对是远超刚刚打过交道的铁鹞子。要是契丹国中随便拉出一支骑兵就能有着如此威势,大宋这边一年只付了五十万银绢的岁币,就得了几十年的太平,那还真是一笔大赚的买卖。

“当然不可能。”折可适死也不会相信,契丹国中的普通兵马,就能有着自家三千子弟兵一般的水平。

“可既然不是辽主的斡鲁朵,那也不可能会是其他斡鲁朵下面的兵。”李铁脚说道。

“嗯,”折可适表示同意,“也不会是镇守陵寝的宫分军。”

大宋是每一个皇帝登基,就会建起一座楼阁,存放他本人御书、御制文集、各种典籍、图画、宝瑞之物,如太宗的龙图阁、真宗的天章阁、仁宗的宝文阁。

辽国则是出一个皇帝就设置一个斡鲁朵,其中挑选出来的精兵就作为他本人的宫卫。当辽帝驾崩之后,皇陵附近都会建一座州城,叫做‘奉陵邑’,专门安置皇帝生前的斡鲁朵,以奉陵寝。此外有几位皇后、甚至秉政的权臣也有自己的斡鲁朵。属于斡鲁朵的宫卫军,当然也都是精锐,只是他们不会离开自己服侍的皇帝,而镇守陵寝的旧日宫卫,除了轮班宿卫当今天子外,也不可能轻易调动。

“那就是只会是皮室军了。”李铁脚声音低沉起来。

折可适点头:“当是皮室军!”

皮室军也曾是辽太祖辽太宗时的御帐亲军,南征北战立下了赫赫声威,在大宋北方的军中同样是如雷贯耳。不过这些年来,已经成了镇守边地的核心主力,宿卫之职交给了宫分军。

除此之外,契丹国中再下一等的军额也有许多。属于五院、六院、乙室、奚部的部族军,还有汉军、渤海军等治下民族组成的军队,诸多贵族名下的头下军,以及属国、属部的军队。以武立国的辽国国中,各色名号的军队多如牛毛,他们才是契丹国中主力。

可要说给这等杂兵能给刚刚取胜的宋军带来这般大压力,折可适是绝不会承认的,“上京道和西京道中,契丹总是有些精锐的。去年争代北之地,好像就有皮室军调到西京道来。”

“多半就是他们!”李铁脚一声大叫。

“……也只是猜测而已。”折可适哼哼的冷笑了两声,“不过只要打上一场,抓两个俘虏就能一清二楚了。”

就在说话间,折可适派出的亲兵们,已经将这一战的战果全都点算了出来。斩获的敌军首级总计两百不到一点,而自家的损失并不多,只有三十余人伤亡。如果在十年前,绝对是个辉煌的大捷,可以露布飞捷一路奔驰上京,但到了十年后的今天,也只是个说得过去的胜利。

“下面怎么办?”李铁脚和几个指挥使们聚在折可适的身边,询问着下一步该如何走。

“丰州城在望,还说怎么办?”折可适反问,“契丹的这一队骑兵,若是驻扎在丰州城或是军寨中,也许能待得久一点。但他们要是埋伏在荒郊野外,能超过五天就有鬼了。”他张开五指,比出自己的右手。

众将点头,折可适说得当然都没有错。战马的食量几乎是骑手的十倍,出战的时候,怎么也不可能为他们的战马将食料全都准备,也只有留在城寨中,以丰州州城和周边诸寨堡的仓储还能勉强供给得上。

“难道要与契丹人耗下去不成?”

“不,他们如果当真要为西夏出一份力,必定会再出来。”折可适用力一挥手,“把话传回去就行了,我们去丰州城下扎营!”

