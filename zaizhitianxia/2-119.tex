\section{第24章 兵戈虽收战未宁(六)}

秦州的州衙还是韩琦在的时候翻修的,二十年过来,已经一点点破败了下去。屋角、檐头无不透着时光留下的痕迹。韩冈枯坐在外院的偏厅中,抬头看着头顶上脱了漆的房梁,静待着郭逵派人来通传。只是等了许久,等得茶都凉了,也不见有人过来。

韩冈已经很长时间没有受到这种待遇了,上一次被晾在一旁没人理会,还是在王安石的府邸上。而眼下在秦州,韩冈的名声让他在任何一处都能成为座上宾。只是以郭逵的身份和地位,把他晾在一边,出口怨气,韩冈也只能一笑了之。

而且郭逵发怒,也不是毫无来由。缘边安抚司把所有事都瞒着秦州,身为秦凤路经略安抚使,王韶、高遵裕的顶头上司,郭逵当然火大。虽然把偷袭星罗结部的计划,用扩建渭源堡伪装起来,可是其中的破绽显而易见,尤其禹臧花麻从中横插一杠后,让郭逵这等在军队中、官场中打滚了几十年的老军头,一眼就看破了王韶从中玩得那些花活,这些事根本就瞒不过他的眼睛。

这世上的任何一位长官,对于像王韶、高遵裕这样自作主张、又瞒骗自己的下属,都不可能有好脸色。韩冈以己度人,对郭逵的怒气也能理解。只不过冷板凳坐了久了,他心里对郭逵的小心眼也免不了有了点看法。

幸好韩冈的养气功夫虽比不上那些儒林宗师,但喜怒不形于色的本事还是有的。冷掉的茶水没有再动,整整过去了一个时辰,韩冈在厅中端端正正的坐着,脸色毫无愠色。

忽然从厅外的院中传来一阵喧闹,韩冈细听了一下,却是秦凤副总管燕达到了。据韩冈所知,燕达这段时间坐镇在陇城县,以便可以随时支援甘谷城,或是东边的泾原路。当韩冈入城时还没听到他的消息,可能是刚刚从陇城县回来。

今次梁乙埋南下,动用了举国之兵,齐攻包括河东路在内的缘边五路。是宋夏两国之间,近十年以来规模最大的一场会战。相对于围绕着横山的主基调,缘边安抚司和禹臧部之间,纠缠于渭源和星罗结城的战斗,连伴奏都算不上,只能算是背景声。

连秦凤路的注意力都没放在战事激烈的渭源堡,钤辖张守约领兵驻扎水洛城,时刻准备援助泾原路。而都监刘昌祚则镇守在甘谷城,也跟党项人打了一仗。燕达又坐镇在两人背后的陇城县,随时可以支援两边。不过最后论起战功,却还是以王、高两人手上的首级数为最,而损失的兵力,也同样是缘边安抚司最多。

大概又是半个时辰的样子,静了一阵的院中,重又喧腾起来。当是郭逵结束了和副手的面会,将燕达送出了主厅。只不过燕达没有就此离开,脚步声从院中接近过来,转眼秦凤路副都总管的一张能吓坏小孩子的丑脸,就出现在偏厅门外。

韩冈一见,便站起身来,上前行礼:“韩冈拜见副总管。”

如果在外面,叫燕达一声总管也无不可,但此时身处经略司中,郭逵就在附近,韩冈老老实实的加了个‘副’字,燕达也不会因此而恼火。

燕达跨步进门,扶起韩冈,笑道:“玉昆今次可是立了大功了。”

这句话入耳,韩冈便是心神一凛,该不是他杀了西夏使节的事爆了出来?这件事虽然在缘边安抚司和蕃人中,都不是什么秘密,可是由于种种原因,让韩冈心有顾忌,故而对外都声称是瞎药所杀,连战报上都是这样写的。如果事实真相被揭发出来,就又是一个欺瞒长官的罪名。他连忙自谦道:“下官愧不敢当。”

燕达一边的嘴角抽动了一下,也许是在笑,但透着讽刺的味道。他并没有在此事上纠缠,而是跟韩冈一起在厅中分宾主坐下。秦州军方第二人的燕达坐进厅中,对郭逵察言观色而慢待韩冈的厅中小吏,终于记起了他们的工作究竟包括哪些内容,热腾腾的茶水和菓子,眨眼间就换了新的上来。

“玉昆可知今次梁乙埋是因何而退?”燕达没理会小吏们的殷勤,而是单刀直入的问着韩冈,这种直接爽快的性格让人不以为侮。

韩冈想了想,用了最稳妥、也是流传最广的回答:“只听说是被董毡逼退的。”

说归如此说,韩冈对于此事决计不信,只是随大流而已,而燕达则是哈哈笑了一阵:“玉昆,这是说给外人听的,要真的当了真,那就是个笑话了。区区董毡的两万余人,只是借势出兵,又不敢深入兴灵腹地,如何能逼退梁乙埋?”

“不知是因何故?”韩冈问道。

燕达没有回答,反问了一句:“有关罗兀筑城的传言,不知玉昆你听没听说过?”

韩冈点了点头,关于韩绛和种谔要修罗兀城的消息,早就传遍了关西军中。顺着无定河一跃数十里,紧贴着银州筑城,这么冒风险的策略,让韩冈都不免为之心惊。尽管,可风险实在太大了,西夏人绝不会坐视。

韩冈猛然一惊:“难道给梁乙埋抢了先机?!”

燕达慢慢点头,他已经说得够明白了,韩冈能推测得到也在情理之中:“梁乙埋今次出征,用得是声东击西之策。他入驻金汤城,主攻大顺城和附近的军寨。这一下子,把关西四路的兵力都吸引了过去,全都去支援环庆路,倒把鄜延路的无定河给忘了。事先谁也没能料到,梁乙埋的目的竟然放在罗兀。”他叹了口气,叹息声中有着无限的感慨,要知道,燕达之前可是在鄜延待了不短的时间,“现在罗兀已经给梁乙埋修起来了,虽然只是个不大的寨子,但有银州在背后支撑,要想攻下此地,基本上已经是不可能了。”

韩绛和种谔对他们的计划没有保密,连秦州这里都听说了,无孔不入的党项探子不可能打听不到,而罗兀的地理位置又极关键,梁乙埋即便不会相信这个胆大到近乎荒谬的计划,但提前做个防备,对一国宰相来说,也是举手之劳。

‘难道今次梁乙埋撤军,是因为已经把罗兀筑好了的缘故?’

这个问题,韩冈本想追问,却没有问出来,因为他已经想到答案了。

凡事有因必有果,有果必有因。但因果之间,并不是一一对应的关系。梁乙埋退兵的这个结果所对应的原因,不可能是简单的一条。既有董毡抄截后路的因素在,也有大顺诸寨久攻不破的缘故,另一方面,罗兀成功修筑,自此横山也可以安泰一点,也让梁乙埋失去了战斗之心。三个原因各有道理,最后结合起来,梁乙埋就只剩下退兵一个选择。

只是还有件事让韩冈感到疑惑。他对此事并不了解,但他经历得多了,也知道以党项人的能力,在军事工程上创造不出奇迹:“以西贼筑城的本事,在这么短的时间,能把罗兀城给修筑成什么模样?”

燕达摇了摇头:“这就不知道了,消息还没从鄜延传过来。不过想来头疼的该是韩宣抚还有种谔才是。”

燕达倒是不避嫌疑,这些私底下对亲信才会说的话都说给韩冈听。韩冈感觉得到,这位副总管对自己好像抱着不小的善意。

只是这就让韩冈有些奇怪,他根本就跟燕达根本扯不上关系。燕达的副都总管一职,是枢密院与政事堂斗争的产物,据说有文彦博一力主张,而他韩冈则正好相反,有关他的任命都会被文彦博反对。对燕达来说,文彦博对他的知遇之恩,还在郭逵之上。就算有郭逵从中转圜,燕达也不该跟自己太亲近,何况郭逵现在还不待见自己。

燕达没看出来韩冈在想什么,他还有个问题要问韩冈:“不知玉昆对屯田之事有什么看法?”

“不过‘势在必行’四个字而已。”

“好个势在必行!”燕达笑道,“渭州的蔡子正,也就是环庆路的经略安抚使,前几天才发文来叫过苦。自渭州至古渭,斗米两百钱,是原价的十倍,剩下的的都是随军转运之事。”

秦州耗用军粮,本就是难以自足。不足的部分,一般都是由关中来补充,走的是渭水一线,自凤翔府而来。不过前些日子,鄜延、环庆有警,物资皆支援前线,已无库存。想了半天,最后就从渭州囤仓调拨了一部分军粮运到古渭,不过这一条路,要翻越陇山,这运费冲抵进米价里,不翻个一两番,那就有鬼了。

“如果能在当地能解决一部分,运费就能节省下不少。”

燕达的想法廖无新意。他要怎么做,韩冈也都明白。将荒地分包给个人,收获的粮食留下口粮和种粮后,由官府收买。而这些人本身,也负担着上阵迎敌的任务。这样的做法类似于隋唐府兵,不过在如今,也只是个专门的屯田兵而已。

燕达想说得就是这一条,“要加快屯田!”

