\section{第20章 冥冥鬼神有也无(六)}

“不知道丰州的情况怎么样了。”

离开了灵渠,在漓水上泛舟而下,望着久违的桂州城,韩冈突然想起了远隔数千里的丰州。丰州一役的成败决定了广西到底能不能得到足够的支持,韩冈自南下之后,一路上都记挂于心。

“到桂州了?”

身后响起了脚步声。腮帮子都瘦了下去的李复和陈震两人摇摇晃晃,上了船首的甲板。

大概是没听到韩冈的话,他们的四只眼睛直勾勾的盯着前方远处的城池。在一瞬间垮下来的双肩,都是一副如释重负的没用样子,就差大喊着‘这一路终于走完了’。

过了方城之后,韩冈一众就换了官船。一路南下都是乘舟而行,因为难得的顺风顺水,该走上二十多程的水路,只用了十三天就走完了,比起去年还要去潭州带兵南下时,要快了许多。不过除了韩冈以外,其他人对一帆风顺的行程都是不是很喜欢,脸色也不是很好看。

水上受风时行速如同奔马,尤其是在泛舟洞庭之上的时候,竟然遇到了一次狂风,虽然他们所乘官船并没有倾覆,对老走水上的船工们来说,也算不上什么大事,但船上绝大部分的关西人都是吓得魂飞胆丧。甚至有好几人在这一趟旅途中都病倒了,直到进入平稳的灵渠之后,才有了些起色。

韩冈看着一个个上了船头的下属,摇了摇头,不是为了他们,而是担心起转道从蜀中南下的五千西军兵马。他们也是一路乘舟,到了桂州的结果,不会比现在船上这一群人好到哪里去。。

官船在码头上停了下来,韩冈早在过灵渠的时候,就通过马递传信桂州。章惇早早的就遣了人来码头上迎接韩冈。另外代替韩冈管理署中事务的转运副使任时中也亲率僚属来迎接。任时中还知道这一路水上舟行给满船的关西人带来多大的折磨,一起派来的车子有五六辆。

先让李复等人上了车,没事人一样的韩冈和几个护卫接着骑上了马,向着久违的桂州城进发。

韩冈先回的是转运司的衙门,本想着梳洗之后再去拜访章惇,却没想到章惇竟然亲自到了转运衙门中等着他了。

韩冈先上前行礼,笑道:“一别数月,子厚兄可还安好。”

章惇回礼后就拉着韩冈进了堂中,驱开闲人,方才摇头道:“好什么,玉昆你南下时应该听说了吧,丰州来了契丹人。”

韩冈点了点头,脸上的微笑带着苦涩:“当然!”

章惇喟然一声长叹:“这一下子,真的就只有五千西军兵马了。”通过马递传到他手中的消息,比起乘船的韩冈只早上一步,也就在前一天到了桂州。拆开来一看,章惇好悬没有将书桌给掀翻掉,“契丹人当真是有本事,只不过是三五百的数目,一下就牵制住了北面诸路数十万的兵马不能轻动。”

“只能看着郭逵的本事了,如果他能尽速解决丰州之事,也许还能多一点兵力南下。到时候我们也能轻松一点。”

“那可不一定。”章惇的脸色依然入挂严霜:“有件事玉昆你大概还不知道。”

“什么?”韩冈问道。他看着章惇的模样,心中已经有了不祥的预感。只是若有大事,方才进城的时候,任时中应该对他说才是。

章惇没有卖关子,很干脆地说了出来:“交趾前日已经献上了降表。”

“是吗,那还真是不知道。”韩冈的瞳孔在一瞬间放大,只是混乱的心情瞬间后又平复了下来,更想着对这件事章惇保密得还真好,竟然连转运副使都不知道。

韩冈过人的自制能力,并没有让章惇在意太多,叹道:“交趾派来的使臣现在就在钦州。我已经将他们的降表和奏疏一起送去了京城,就不知道朝堂上会如何”

韩冈皱着眉头:“降表中的内容如何?”

“空口说白话而已,满篇全是辩解之词。只是说交还掠走的百姓,日后依时入贡,并割让广源州。”

“即是这样的降表,子厚兄你还担心什么?”

“你说为什么?”章惇直接反问,看着韩冈愣了一下之后,就无言以对的样子。脸色似笑非笑:“对了,他们还找到了罪魁祸首。”

“不是家岳了?”韩冈还记得交趾入侵时,散发的檄文中将罪名归咎于谁人。

“当然不是,是徐百祥。”章惇冲着皱起眉头、苦苦思索的韩冈道,“是个不第的秀才,据交趾降表中所言,就是徐百祥写信愿做内应,劝说他们出兵的。”

“好本事啊。”韩冈心头怒极,反而失声笑了起来,“一个不第秀才就能让他们出兵攻打邕州。要是我说上一句,是不是能让他们打到辽国的辽阳府去?!”

章惇恨声说道:“玉昆你也莫说气话。就是这一个徐百祥献上了囊土攻城之计。要不然,邕州城也不会这么容易就给攻破。”

“原来就是他!”一道青气在韩冈脸上闪过,提起拳头在交椅边的几案上重重的反手一捶:“此人当千刀万剐!”

韩冈的怒喝随着木头折断的声音一起爆发了出来,好端端的几案竟然给他一拳打做两段,几案上的杯盏也碎了一地。

章惇心中一惊,他一向知道韩冈的武勇,在文臣中绝对是排在前几的。想不到他在一怒之下,竟能一拳打坏了上好花木打造的桌子。

守在门外的护卫奔了进来,章惇挥了手让他们出去,转过来关切的问道,“玉昆,你的手还好吧?”

韩冈揉了揉发痛的骨节,摇摇头:“没事。”又急问道,“此人可还捉到了?!”

“已经下了大狱,好生的养着,日夜都有人盯在他身边,绝不会在明正典刑之前让他死的。”章惇说得咬牙切齿,他对这名汉奸也同样是恨之入骨。

“等朝廷的令旨来,就在忠勇祠前生剐了他!”韩冈恨恨不已。摇了摇头,收了脾气,又说回正事:“朝堂上有家岳在,而且以天子的脾性,这样的降表肯定是看不上眼的,怕就怕其他几位宰执会从中作祟。”

“只能尽快发兵。”章惇说着自己的打算,“等秦凤泾原两路的五千兵马一到,就要开始南下,在这之前,先让邕州的荆南军往边境的永平寨去。”

韩冈的眉心又写出一个川字来:“他们也是一路乘船,至少要休整半个月的时间,否则根本恢复不了元气。”

“愚兄也写了奏疏,拖上两月不成问题,只是再长恐怕就难了。”

“有两个月就够了。”韩冈笑了起来,“既然朝廷虽然只给了五千兵马,但兵械军器都是按照总数来的,很快就会运送到岭南来。有这些军国之器,先打出个大捷出来!”

“接下来就可以坐看北方的局势变化了。”章惇的心情也轻松了一些。有一个大捷在手,看着即将灭掉交趾,天子就算挤也会挤出人来的,“不知丰州现在的情况如何,该有个结果了。”

……………………

结果当然好得不得了。

尽管党项人的主力在嵬名阿吴的率领下主动放弃了丰州城,但郭逵还是打出了一个斩首两千的大捷来,其中就包括了四百余来自西京道皮室军的契丹铁骑。

而在丰州大捷中立下赫赫功劳的折可适,已经被天子招入宫中,亲自询问他立功的过程。

“当臣听到哨探们的报警之后,推车的民夫全都立刻离开了官道,躲进了官道两侧的山林之中,并将独轮手推车丢了满地。”

赵顼打断了折可适的叙述:“这是陷阱吧?”

“陛下英明。”折可适点头:“推车的民夫,其实都是禁军所扮,身上还带着神臂弓,进入山林之中,就立刻集结起来封堵皮室军的退路。而车中也都是易燃的干草,里面还藏着硫磺、砒霜、狼毒、巴豆等发烟的毒物,专门用来薰马的。”

“接下来是怎么做的?”赵顼兴致高昂,连着崇政殿中的宰执,甚至包括了内侍、班直也都在专注的聆听着。

能一口气全歼四百皮室军,比得上四千铁鹞子了。毕竟是护翼大辽天子的皮室军,与大宋的上四军、西夏的环卫铁骑一个等级的精锐。

“当先出现的是一队铁鹞子,臣指挥着伪装成辎重护卫的本部挡在了他们面前。这一次,萧药师奴还是照着老办法,铁鹞子先攻吸引我军的注意力,而皮室军从后突击。只要配合得好,倒是让他们成功过两次,即便失败,皮室军也能跑掉。但这一次,臣却是等候已久。”

真正说起来,事后折可适也反复推敲。那一队皮室军,应该是被他们的友军给陷害了。否则以之前率领皮室军的主将萧药师奴的脾气,应该是见势不妙拔腿就走的,怎么可能会被从后赶来的援军咬住脱不了身。以皮室军去留由己,从不在意同伴的行为,他们在战场上被党项人在背后捅上一刀,一点也不奇怪。

但这话是不可能对天子和宰辅们说,折可适用着清朗的声音,将已经经过修饰的过程在崇政殿上娓娓道来。

