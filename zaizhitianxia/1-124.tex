\section{第45章 樊楼春色难留意(四)}

【第三更,求红票,收藏】

一阵吼声过后,苍老的歌声停了,胡琴声也没了踪影。那位不知名的老者是有感而发,但被人莫名其妙干扰到,心情一转,这曲子当然是怎么也唱不下去了。

而韩冈这边,也没了听曲唱曲的兴致。大牌的玉堂秀收了琵琶告辞离开,而周南就带着一阵香风,坐到了韩冈的身边。同时章俞又命福泉找进来几个歌妓,陪在身边。刘仲武和路明都仔细看过,心里也怀着期待,但这其中却并无一人能比得上周南。

而韩冈对坐在身边的美人全没放在心上,心里都在想着自己在西太一宫中题的这首小令。他本以为要过些日子才会传唱开来,反正自己那时都回秦州了,与己再无瓜葛,谁想到才几天工夫,就在樊楼中听到了。韩冈并不想靠文名诗才出头,这剽窃之事无意去做,反正只要自己不承认,谁也不会知道是自己做的……除了路明——想到这里,韩冈望过去,却只见路明低头盯着酒杯,也不知在想个什么。

韩冈看了一眼便收回目光,心中则不免有些惊疑。周南一颗心玲珑剔透,隐约估摸到了一点。便凑到韩冈耳边,吐气如兰,“官人喜欢这首小令?这是最近才题在西太一宫壁上的,就跟王相公的两首六言题在一起。就是没有题名,也不知是谁人之作。不过有人说道,是一位来自关西的老贡生所作。”

啪啪两声轻响,却是路明的筷子掉了。听说留在西太一宫壁上的小令没有书款提名,而且最后反而着落在自己的头上。他抬起头震惊的看向韩冈,这实在出乎他的想象。

被路明吃惊的盯着,韩冈神色自如。右手敲着桌面,打着拍子,重复着刚才听到的曲子,哼着有些走调的歌声。他自得其乐的哼了一阵,便又笑道:“当真是绝品,难怪传得如此之快。王大参的两首六言已经让西太一宫蓬荜生辉,这一首再写上墙去,只论文采风流,大相国寺也得瞠乎其后。”

周南轻蹙眉头,有些疑惑的看着韩冈谈笑风生。

虽然这位韩官人不像她过去遇到的那些的读书人,总是纠缠不清,要么自吹自擂,要么就是炫耀着自己浅薄的才学,让一向讨厌这些厌物的周南感觉十分轻松。但韩冈没有过来殷勤的奉承,或是竭力的表现自己,也让周南感到很奇怪,甚至有些不服气。

寻常外地州县来的士子,到了樊楼之中,免不了目迷五色,神魂颠倒。看到了像自家这样花魁行首,更是会前后失据,犯下许多蠢事,往往就成了在姐妹间传播的笑料。但身边的这位韩官人到好,除了刚见面时表现出一点惊艳之情外,一直都有些心不在焉。

周南能感觉得出来,韩冈应该对自己有好感,但那种好感也仅局限于泛泛的欣赏,完全没有动心的模样。绝不像平常见到的男子那般,看到自己时总是充满着贪欲的目光。

不知为何,周南突然生起气来,眼中含嗔,银牙咬着下唇,不服气自己被忽视。声音也便冲了一点:“官人年少有为,春风得意,怎么喜欢这首曲子?”

“说不上喜欢,只是此曲令人叹为观止,觉得好而已。”韩冈突然扭头深深的盯了周南一眼,如愿的看着少女双颊微晕的把视线闪躲开去,可一闪之后,她却又狠狠的瞪了回来。

见着宜嗔宜喜的俏脸上悄然带起的薄怒,韩冈只是笑了笑。便又立刻正色沉声:“韩冈自少文武兼修,亦有班马之志,如今正是男儿立功之时,却不会有悲风伤秋的余裕,也不会有‘断肠人在天涯’的感慨。”

“那官人到底喜欢什么样的曲子?”周南仰着头,看着韩冈。长长的双睫一颤一颤的眨着,睁大的一双秀目中还带着小女孩儿的稚气。

‘演技真好。’韩冈不禁暗赞。知道周南是在装模作样,他便有了点恶作剧的心思:“关西的得胜歌不知小娘子能否唱来?”

明白韩冈是存心刁难,可周南她半点不惧。关西得胜歌在京中也有传唱,尤其是教坊司,都会让所属的歌妓学上几首,好在接待关西来的将领时,表现上一番。她得意的横过韩冈一眼,悄悄的又哼了一声,也不知从哪里找来两块红牙板,清唱起来:

攻书学剑能几何?争如沙塞骋偻罗!手执绿沉枪似铁,明月,龙泉三尺崭新磨。

堪羡昔时军伍,谩夸儒士德能多。四塞忽闻狼烟起,问儒士:谁人敢去定风波?

如果让殊乏文采的韩冈去形容,他会把周南的嗓音比作黄莺一般,悠扬婉转,正能撩动听众的心弦,仿佛天籁。如果她唱的是婉约小词的话,多少人都会沉醉下去。‘寒蝉凄切’让人悲,‘东郊向晓’让人喜,喜怒哀乐,全在她歌喉之间。

只是今次换作了传唱自盛唐时的得胜歌,周南声音中的缺点便完全暴露了出来。太过柔美的嗓音缺乏刚劲力量,叮咚脆响的红牙板更远比不上战鼓激昂,两厢相加,便完全毁了一首让人热血沸腾的好词。

刘仲武方才又多喝了两杯眉寿,脑袋又是晕乎起来,他肆无忌惮的嘲笑着:“这是女儿家唱给情郎的吧?若是俺们关西男儿阵前战后唱起来都是这个味道,党项人笑死的会比较快!”

韩冈也是一阵大笑,摆着手让周南不要唱下去了,“这一首不是小娘子唱得来的。‘谁人敢去定风波’,当是以铜琵琶,铁绰板,以关西丈二大汉唱来。如周小娘子这般,年才十七八,手持红牙板,也就只能唱得‘杨柳岸,晓风残月’。”

如果说刘仲武的嘲笑像是一记正拳,那么韩冈的评价便是如利刃透骨而入,丝毫不留口德。周南眼眶都红了,紧抿着嘴,硬是不肯哭出来,已经有些规模的胸口急速起伏着。

见周南气苦欲哭,韩冈发现方才自己做得实在有些没风度,才十七岁的小姑娘,欺负她也得不到什么成就感。“韩冈失言了,若有什么得罪的地方,还请周小娘子恕罪。”

“谁稀罕你道歉。”周南最后一跺脚,转身就冲了出去,犹如一朵彩云冉冉而出。

厅中一片寂静,客人和妓女,都坐在一桌上,互相看看,都不知该说什么好。

章俞这时哈哈大笑,笑声打碎了厅中的尴尬:“自来都是求着花魁来,今日把花魁给气走,玉昆你可是独一份。”

路明也跟着笑道:“不过韩官人也说得没错,关西得胜歌有十几二十首,却没有一首是能唱得出来的。”

韩冈心中的歉疚转瞬即逝,他说的可没有一句假话。想到得胜歌,韩冈现在便又回想起镌刻在心底的那一幕:“我上一次听到得胜歌。还是两个月前,秦凤张都监以两千破万人,大败西贼,凯旋而还的时候。灯火如星河,歌声冲霄汉。关西男儿的豪迈自歌中而出,不是女子可比。”

“官人说得好!”刘仲武抚掌大笑,韩冈正说到他心底里去了。

气氛重新热络起来,章俞又叫了一个上等妓女来陪着韩冈,不过还是远远不及被气走的周南。喝酒,行令,划拳,不一会儿,酒席上的热闹又高了许多。

一顿酒喝了不短的时间,最后因为韩冈晚间尚有要事,方才作罢。

互相道别后,两拨人各自回住处。返家的返家,回驿馆的回驿馆。只是刘仲武喝得太多,韩冈让李小六雇了辆车,直接运回去,而他则是和路明租了两匹马,往回走。走在回驿馆的路上,路明问道:“韩官人,为何不在诗后题名?!那可是难得一见的佳作。”

韩冈没喝多少酒,而且他方才喝的和旨又是以清淡著称。头脑清楚的很,“我也有话要问路兄,为何你方才不提出来?”

韩冈这么一反问,路明脸上的疑惑之色不见了,却露出了一副‘果然如此’的表情。

“‘小桥流水’,这一句说的是秋天——深秋。冬天黄河都结冰,何况小桥下的溪流?”

‘所以这首小令说的不是我,韩官人你也不可能是这首小令的作者,二十岁春风得意,怎可能有四五十岁的悲叹?’这几句,路明咽在了肚子里,没有说出来。

路明才学并不出众,甚至还不如韩冈。但即便是以他的这点学问,却在冷静下来之后,一眼便看出诗中的破绽,查明韩冈的谎言。

“路兄果然心明眼亮,”韩冈笑赞道,他承认道,“作者的确不是我,人可欺,天难欺,所以我也不能夺为己有。不过既然世间皆穿此诗是一关西老贡生所为,路兄何不干脆认下来?”

韩冈说完,便紧盯着路明的反应,看着这位三十年不中的老贡生脸上的神色如走马灯的变幻。到最后,路明放弃了的叹着气:“官人不是说了吗,人可欺,天难欺。这事路明也做不来。何况在下就这点学问,说是我做的,谁又会信?”

韩冈点了点头,收敛了心中的杀意。他虽然不打算窃取文名,但这首《天净沙》他也不想让人偷去。若路明受了自己这么多人情后,还敢夺己之物,他可不是心慈手软之辈。不过路明能做出正确的决断,不为一时之利所诱,日后有机会倒是可以帮上他一把。他说道:“前日在西太一宫的一番话,是韩冈信口而出,非有恶意,还望路兄勿怪。”

“虽然官人你是信口之言,但那当头棒喝对小人的意义,却没有任何区别……断肠人在天涯……断肠人在天涯!”路明喃喃的反复念叨,仍是深有感触,他问着韩冈:“不知这首小令,官人究竟是从何处看来?”

韩冈咧起嘴笑了:“路边上。”

