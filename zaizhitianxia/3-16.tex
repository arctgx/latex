\section{第七章 儒统渊源远(下)}

【这一章还真难写,不知不觉,又到了快三点了。】

可能是今年的最后一次讲习,今天横渠书院中的气氛就有些不同于往日,连聚在正堂大厅中的学生也比平常多出了不少。

过了今日,书院中的大部分学生各自都要回乡,只有少部分缺乏回家路费的才会留下来。而张载最出色的几个弟子,蓝田三吕中的在外任官的吕大忠和吕大钧也恰好在这个时候来拜访横渠书院,吕氏三兄弟同聚一堂,这样的情形已经很少见了。想来今日的宣讲,将会是一个大课题。

李复很期待他的老师今天会讲些什么,身边站着算是父执辈的范育,并不敢乱动弹。不过同在横渠门下,当聆听讲学时,李复便是跟范育平起平坐的,并不用执晚辈之礼。

范育是邠州三水人【今旬邑县】,本人年纪已经过了三旬,早早就中了进士,也是很早就追随张载的弟子之一。这两年他一直在外任官。今年他请了假,回来省亲,顺便就到了新修起来的书院中来听讲。这半个月,他都在书院之中。在接受张载讲学的同时,也一并了教授师弟。

范育的父亲范祥,在关西名气很大。陕西如今所用的钞盐法,便是由其所创。省运费,得实利,一出一入,陕西因此而多增数十万贯的盐税。同时范祥还是河湟开边最早的倡议者之一,并在没有得到朝廷同意的情况下强行修筑了古渭寨。今日河湟功成,起点就是古渭,范祥的功劳不可磨灭。他的这份功绩在一年前,熙州之战后,被生前好友向天子提了出来,让范祥得到了追赠,连带着范育的幼弟也得了一个赠官。

相对而言,李复的资格就很浅了。皇佑四年出生,此时不过二十出头。这个年纪在张载的弟子中,只能算是小字辈。不过在他同龄人之中的,可有最近声名鹊起的韩冈。同为横渠弟子,听说韩冈的累累功勋,李复觉得也算是与有容焉。

‘三吕都来了,范世叔也到了。’李复咂了下嘴,心中所想不由得冒出口,“韩玉昆若是能来就好了,真想见见他呢……”

范育一笑,接口道:“前日上京的慕容思文,不是说今次韩玉昆也会去考进士吗?理应会来。”

“但要是再迟点,小侄可就要先回乡……”

李复突的话声一顿,站在前面的吕大临不知什么时候回过头来,瞪着私下里说小话的两人。

李复立刻闭嘴低头。他家跟范家是世交,范育又是再平和不过的性子,两人算是忘年之交。但三吕中最年幼的吕大临一直跟着张载,连官也不去做,日常督促师弟们功课的就是他,让李复很是敬畏。倒是范育,平和的微笑着冲吕大临点了点头,算是致歉。

吕大临颔首为礼,又转回头端正站好。李复方才的声音传到了他的耳中,吕大临不喜欢韩冈的理论,认为他并没有遵循先生的教导,反而走偏了路。尤其是从游师雄那边传来的‘旁艺也能进大道’的说法,实在太过狂妄,让他听了很是不喜。

正想着的时候,张载已经出来了。五十多岁的当世大儒,因为常年苦思天人大道,心力耗用过甚,气色并不太好。但他走起路来,却是规行矩步,儒者气象就蕴含在举手投足之间。

年纪最长的吕大忠领头,近百名弟子群起而拜。张载等他们拜过起身,便回了一礼,又当先坐下。

等学生们全都在蒲团上做好,张载没有宣布今日开讲的课目,而是开门见山的问道:

“何者为儒?”

何者为儒!张载的这个问题很大,好像很空泛,却是有着深意,近百个学生都是沉吟不语。

按照说文解字的说法:儒,柔也,术士之称。在孔子之前,儒者是一个阶层,有治国平邦之术的,是为儒也。到了圣人横空出世,儒学独树一帜,成为一个春秋战国时的显学。儒这个字,就成了一家所用。而到了汉武帝罢黜百家、独尊儒术之后,儒就成了士子的代名词。

不过在这个场合,张载所要的答案,当然不是这个。在座的学生,也没人会拿着说文解字来回答老师的问题。

李复资格虽浅,但胆子却是极大的。吕大防、吕大钧两个大师兄还没说话,他就当先站起来,提声道:“‘祖述尧舜,宪章文武,宗师仲尼’者为儒。”

此三句的前两句出自中庸,说的是孔子。但带上后一句,就变成了是班固在《汉书艺文志》中的说法。李复觉得,所谓的儒基本上就是这个道理。

但张载却是给了李复当头一棒,他摇头,“班固之言,只得一偏。”

李复愣了一下,呐呐的问道:“不知先生之意为何?”

张载没有即时解答,而反问众弟子:“儒者当有何为?”

此言一出,不少人就明白了,张载对此已经说得太多。

“为天地立心者为儒!”吕大钧当先起身,“天地本无心。其仁也,鼓万物而已,不与圣人同忧。传习圣道,便是以己心合天心,大其心,以为天地而立!”

张载满意的点点头,“此一也。”

此一句,是关学的根本大节。吕大钧这位首徒,其实是张载的同年友。与其说是弟子,不如说是师友。多年来共同揣摩儒家大道,自家的学术,他最为通透。吕大钧能第一个说得出来,也是情理中事。

头一句一出,第二句便紧跟着出来。

“应为生民立命【注1】!”

苏昞站了起来,他是邠州武功人。他在张载门下传习日久,自然也能轻易的总结出这一句,“民,吾同胞;物,吾与也。为儒者,奉天子而理天下,应为生民立命。”

“此一也。”张载点了点头。

吕大钧和苏昞说得很完美,将张载的天人合道之说已经归纳得大半,西铭一篇,根基就在这两句上。但众弟子见张载的态度,明白这个问题并没有结束。

“须为往圣继绝学!”

前面的吕大钧和苏昞为张载的众弟子归纳出了关学大纲的前两句,半刻的静默之后,在厅堂一角,又有人续上一句。

众人看过去,却是范育。

“汉儒崇章句,唐儒耽佛老。不知天地之大,孜孜于章句之间,惑溺于外道之中,而孔孟之道不之传也。须为往圣继绝学。传习圣人之学,承袭儒门道统!”

范育朗声说着,旁边的李复崇拜的抬头望着他。

张载带出了一点笑意,鼓励的也对范育点了点头,“此亦一也。”

近百弟子与李复一样,崇慕的看着吕、苏、范三人。能在这个场合让张载满意,说明他们已经可以继承关学的衣钵。禅宗六祖慧能,一个扫地僧,可不就是靠着一句‘本来无一物,何处染尘埃’,压倒了禅门大弟子神秀的‘时时勤拂拭,莫使惹尘埃’,从而在五祖弘忍手中得了禅宗的道统?

阐明大道,数句足矣。

但张载还是有些不满意。他看看在座的一众弟子,心中暗叹,思孟源流的修身齐家治国平天下,还没有在前面的三句中总结出来。他少年习弓马,读兵书,其门下亦多有素习兵事者。儒门六艺,御射二术侧身其中,试问儒者如何不谈兵?为生民立命也不是靠着‘民胞物与’四个字就够的。

“犹未足也……”张载慢慢摇头。

堂中一片安静。

接受过张载教诲的弟子们,其实都隐隐知道张载的心意。但他们却无法组织出一句,能与前三句相抗衡的心得出来。

为天地立心。

为生民立命。

为往圣继绝学。

张载鼓励弟子要‘大其心’,不是自谓高过一切的狂妄,而是以己心合天地之道,所谓‘义命合一存乎理,仁智合一存乎圣,动静合一存乎神,阴阳合一存乎道,性与天道合一存乎诚’。现在出来的这三句,已经说透了儒者当如何立于天地之间,如何对待生民,如何传承道统。

只是,现在所剩下的最后提纲挈领的一句,又该是什么?

众人苦思冥想,观其神色间,或有所得,但却没有一个能成句的。

今日先生谆谆教诲,诱导众学生将他所传授的道理总结归纳,关学的纲领就在四句当中。前面已得三句,吕大临认为总结全篇的最后一句,就当留于自己。而他也觉得这一句已经呼之欲出,只是却在嘴边顿住。呼吸都有些艰难起来,仿佛被蒙在布袋中一般憋闷。他咯咯的咬着牙,就是憋不出一个字来。

“当为万世开太平!”

不知从何处传来的声音,如同清风吹散了吕大临心头的憋闷,晨钟暮鼓一般让他恍然过来。

“对!就是这个!”

声音一出口,吕大临便一下惊觉,‘最后一句是谁说的!?’

疑惑尚在头脑中转着,就在堂内聚集的众弟子身后,也即是大门之外,依然是方才的那个清朗而沉稳的声音:

“为儒者,当为万世开太平!”

注1:世传的横渠四句有两个版本。‘为生民立命’这一句,另作‘为生民立道’。本文取前者,流传较广,同时押韵。

