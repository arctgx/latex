\section{第39章 铜戈斑斑足堪用(上)}

鼓声之中,城门大开,披甲持戈的战士从门中鱼贯而出。

城外的蕃骑一见便随之而动。青谊结鬼章本就是为了激怒城中守军,才如此高调的举起田琼的首级。如愿的见到宋军出城,便立刻挥兵往敞开的城门处冲杀而来。

“射!”

韩冈短促的命令调动着鼓点的转换,城头上的守军随即弓弩连发。在神臂弓的攒射下,城下列阵的官军前方,有了一道难以逾越的天堑。

青谊结鬼章呵斥连连,但几支被派遣上去的骑兵无不是动作迟缓,躲过一波箭雨后,还没有冲到敌军阵前,就立刻迎来了第二波洗礼。而胯下的战马也在青谊结鬼章呵斥全军的过程中,突然前蹄一软,差点就把他给摔下马来。

发现麾下骑兵的战马都与自己的坐骑一样,都失去了冲击力,青谊结鬼章的脸色变了。这才发现在一天近乎不眠不休的活动后,积累的疲劳使得战马的体力已经几乎见底,甚至连冲锋的姿态都难以保持。

宋军就在吐蕃骑兵的眼前结阵,沉重的脚步声伴着锃锃的弦鸣,一步步的逼近上来。

现在只要吐蕃军稍稍靠近箭阵,劈面而来的便是一波箭雨。虽然昨夜吃了大亏,可都能说是宋军狡猾。他一直都抱着宋人胆怯文弱的心理,即便经过昨夜之事后有着改弦更张的想法,也根本没有料到,城中的敌人看到了他亮出的战果后,还有出城决战的胆量。

在不断前进的宋军的逼迫下,吐蕃人不得不节节后退。尽管他们的表现出来的战斗力,已经超过韩冈的预期,甚至有两次反击抓准了箭阵前行时的破绽,差点搅乱了宋军的阵列。但这样出色的表现,只是昙花一现而已,随即就被宋军更为激烈的反扑给掩盖。疲劳在战况不利的情况下,是千百倍的涌向心头,让吐蕃军的反击越来越无力。

而两支从珂诺堡侧门绕出去的队伍,此时已经潜藏到山中,试图绕行到吐蕃军背后,将之包围起来。

青谊结鬼章束手无策,期盼之前与田琼首级一起传到手里的消息中所说的援军,能早一步赶来。只要能攻破香子城,解放下来的大军就会赶来攻打珂诺堡,这个想法是让鬼章部族长在当前不利的战况下,还犹豫是否继续作战的关键。

战鼓继续擂动,韩冈此时已经到了城下。他手上以一千广锐军为后盾,加上一干原本就留守于城中禁军,也不怕吐蕃人能玩出什么花样。

胜利就在眼前,得来的轻易无比,宋军上上下下都是有着轻松的心态。现在韩冈就盼着派出去的两支偏师能早一步到位,将眼前的贼军全歼于城下。

一阵轻微得近乎微不可察的震动不知从何处传来,沉陷在血腥之中的人们没有发现异常,而战马则已经感受到了危机就快要抵达身边。

刘源低头看着自己的坐骑不安的转着耳朵,想着究竟发生了什么事,而很快,他猛然抬起头,踩着马镫站了起来,铁青着脸望着南方的远处。

刘源的动作惊动了韩冈,想着同样方向望过去,一抹尘烟闪过了远处山头间的缝隙,落入他的眼帘。

只是有千军万马的狂奔,才能从谷地中掀起宛如春日沙暴一般的烟尘。韩冈的眼神一下锐利起来,而刘源更是一屁股坐回马鞍,低声叫道:“是吐蕃贼军!”

不用刘源的惊叫提醒,眼前战场上这群吐蕃人变得激烈起来的喊杀声,已经说出了答案。而韩冈也很清楚,就算是王韶已经攻下了河州城,也不可能在这时候抽调出千名以上的骑兵。

越来越多的士兵发现了前方的异变,同一个问题不断闪现在他们的脑海间:‘难道香子城破了?!’

韩冈掌心被汗水湿透,是回守珂诺堡;还是连同即将到来的援军一起击破,伺机夺回香子城,两个选择在脑中一闪而过。

“好像有些不对!”刘源原本发青的脸色中突然透出一点疑惑。

“怎么?”韩冈立刻问道,“来的不是吐蕃人?”

“不……蹄声很乱,不像是获胜后该有的声音。”刘源纳闷的皱着眉头,“难道他们没打下香子城?”

听声望气,据说是武经总要中规定武将应该具备的能力。韩冈他是文官,对此无需强求。而刘源则是各方面都很出色的武将,连传说中的文王六壬神课、黄道十二命宫吉凶占法都有所钻研,只是运气一向不佳而已。所以韩冈现在要做的,就是决定要不要相信专家的判断。

不需要再多的信息来支撑判断,韩冈选择相信刘源。就算刘源的判断是错的,他也要让自己麾下的队伍去相信。

“收缩阵型!加强进攻”韩冈毫不犹豫地下着命令,“刘源,让你的人准备!”

“不撤退?”韩冈全盘采纳他的推测,连刘源本人都吓了一跳,“机宜,有可能我弄错了!”

“提议在你,取信在我。责任我来担。”韩冈语气静如止水,“多上三五千骑兵又能如何?打到底。吐蕃蕃贼绝不是官军的对手。”

……………………

自从接战以来,青谊结鬼章就明白自己犯了一个很大的错误,甚至是难以挽回的错误。

他高估了自己麾下骑兵的战斗力,同时低估了宋人的实力。从昨夜到今日,几番受挫之后,他终于知道了为什么木征会在宋人的进攻下节节败退。

现在战马的脚力已经出现了问题,而宋军又摆出了意图全歼的样子。但从后方传来了急报,拯救了鬼章部的族长。

青谊结鬼章派去香子城方向,等待援军、同时以防被两头堵截的哨探,终于向他报告了援军抵达的消息。得到这个好消息,军中士气大振,鼓起余勇,又跟毫不退让的宋军杀个难解难分。

只是片刻之后,当所谓的援军终于出现在眼前,青谊结鬼章却发现了情况有所不对,这的模样完全不是胜利的队伍应有的外象。而率领他们的将领栗颇,却是当初木征任命的主将。

青谊结鬼章一下变得又惊又怒:“你们没有攻下香子城?!”

“你也没有攻下珂诺堡!”栗颇立刻反驳道。

“一千人攻不下珂诺堡有什么奇怪的?而你有近五千人都没攻下……怎么就剩三千了!?”青谊结鬼章的脸色更是难看。

“在路上散了一些。”栗颇一派不想多说的模样,又反诘道,“但你不是让珂诺堡的援军冲出来了?整整一个指挥的骑兵。要不是我提前派人埋伏道边,就让他们冲进香子城中了。”

“别告诉我你昨夜的战果就是那一个指挥的宋军骑兵!”

“……是王韶的援军来得太快。”

“有你的,能埋伏珂诺堡的援军,去伏击不了河州的援军。木征是瞎了眼了,怎么选了你这个名将!”

“青谊结!”栗颇恼羞成怒,几乎要翻脸,作为木征麾下大将,他跟青谊结鬼章没有半点交情。“再说这些废话,就能把珂诺堡攻下来了?!”

鬼章部的族长一时无语。

栗颇乘胜追击:“一荣俱荣,一损俱损。香子城不用提了,河州城也不知怎样,不在这里扳回来,下一个可就是你鬼章部了!”

“你待怎么办?”青谊结鬼章怒声问着,要不是因为唇亡齿寒的关系,他早就回族中去了。

栗颇看着前方喊杀声犹然回荡在山谷间的战线,问道:“城里的宋军都出来了?”

“至少出来了大半。”珂诺堡中怎么也放不下五六千人,三千四千已经顶头了,“堡中最多还有千人。”

“只有千人的话,我们从暗道进去,他们根本堵不住。……青谊结,你缠住这群宋军,别让他们退回去。待我休息半个时辰,便助你把这群宋人给灭掉。”栗颇恨恨的咬着牙,“珂诺堡必须得攻下来!”

但宋军丝毫没有后撤的迹象。就算吐蕃人的三千援军逐渐汇集过来,兵力已经超过镇守珂诺堡的宋军,但韩冈依然没有回撤的打算。

敌阵中的号角变得刚劲起来,战事也随之更为激烈。韩冈望着吐蕃人的动向,问着刘源道:“他们像是要缠着我们的样子。”

“大概是防着我们撤回珂诺堡中。他们是想等援军休整过后,在城下歼灭我们……机宜,要退兵吗?”

“这个时候,只能进,不能退!”韩冈虽不算知兵,但他足够了解人心。任何策略都要人来完成,现在这种情况下,麾下的将士都赌上了最后一口气,全然忘了蕃贼的援军。可一旦后退半步,让他们的这一股气势消退,败势任何人都止不住!

“进攻!”他一把抽出腰刀递给亲兵,“你去监阵,妄退者悉斩!”

“刘源!”韩冈再叫起身边最后的一张牌,“轮到你了!知道该做什么吧?”

刘源一拱手,用力吐出了两个字:“杀贼!”

