\section{第14章 飞度关山望云箔(七)}

幸好三名假扮僧侣的交趾奸细在昆仑关被捉住了。从对阮平忠的审讯,还有黄金满的陈述中,韩冈可以确定,官军大败刘永的消息并没有传出去。

奸细就算披着一层和尚的外衣,也不敢在白天往昆仑关去。三人是分头在白天出城,夜里则一起赶往昆仑关报信,故而都被骑马上路的苏子元、何缮超越过去。

但现在黄金满反正的消息已经传出去了,聪明的人都会选择走小道绕过昆仑关。只要有心,两三天后,韩冈手上的底细就能出现在李常杰的面前。

黄金满看着韩冈脸色审问过奸细和俘虏之后,神色依然沉郁。上前小心的提议道:“运使,要不要派兵去守着其他小道。”

“嗯。”韩冈点头,“这事你速去办。点选得力人手,尽可能的将绕过昆仑关的小道都封锁起来。只要是准备穿越小道往邕州去的,一律格杀勿论。”韩冈杀气腾腾,眼下这种情况,能多拖延上一天,离交趾撤军就近了一分。

韩冈并不是怕自己底细,只是不能太早。只要李常杰得到黄金满反叛的消息,他就的整顿行装准备撤退。等他开始动身南返后,即便得知了韩冈手上只有八百官兵,再想回头就不是那么容易了。韩冈也只需要之间两条消息之间,有那么几天的间隔,为邕州城争得一线生机。

“亡羊补牢,为时未晚。”韩冈回头,苏子元发青的脸色落在他眼中,“交趾奸细的腿脚,不会有那么快。”

黄金满得令后就立刻点了兵将出发,他要镇守昆仑关,周围主要道路都派人查看过一遍,本来就有人监视,现在在多派人手封锁截杀,也不需要忙着查看地图。

“也算是诚心效顺了。”李信很满意黄金满的做事态度,低声对韩冈道,“这次事了,他少不得一个蛮部巡检。”

一任巡检,就代表着会被纳入朝廷的正式编制,同时发给俸禄。而不是仅仅是那等赠给四方蛮夷首领,属于虚名的刺史、团练使之类听着好听的官职。

听话且有能力的下属,除了一些嫉妒心强且没自信的人以外,基本上没有哪个上司会不喜欢,韩冈对黄金满也很满意。要是换做首鼠两端的人镇守昆仑关,他哪里有这么般容易站到昆仑关的城头上。

现在就不知道宾州能不能及时堵上了这个漏洞。韩冈手中只有八百兵,想堵住各条道路,也只能依靠黄金满。

“不知韩廉什么时候能回来。”李信问道。

“明天吧。从昆仑关往邕州去,一来一回也要不短的时间。”刚刚抵达昆仑关,韩冈就已经派遣斥候去打探邕州城的消息了,趁着夜色,看一看邕州城的情况。如果条件允许,再放几把火通知城中军民。虽然起不到多少骚扰作用,更不可能与邕州城联络上,但他们的出现可以给交趾人一些压力,“希望他们能带回好消息。”

从正堂出来,迎接韩冈三人的是满天星斗。

苏子元抬头望着天上的群星,“已经是二月了,再过些日子,广西这里的雨水就会逐渐多起来,一直要到秋后,才会渐渐稀少。”

“下了雨,仗可就没法儿打了,只能收兵回家。”

苏子元叹着,“可惜邕州的冬天很少下雨,要不然李常杰也不能围攻邕州这么长时间。”

韩冈点着头,不论有没有攻下邕州城,一旦雨水连绵,就算是交趾人也得撤军。数万大军聚集在温热潮湿的环境中,疾疫是免不了的事,李常杰也不可能改变这个现实,而自己也一样。

接下来的半年,为避雨水,打不了什么仗。就算要反攻交趾国内,也要等到秋后才能开始动手。

黄金满这时安排好了人手,小跑着回来向韩冈三人一一禀报。对他的安排,并不熟悉地理的韩冈也没有什么问题可以问,不过苏子元也没有说话,应该不会有问题。

赞了这位广源蛮帅之后,韩冈又问道:“黄金满,现在跟随在李常杰身边的广源蛮帅还有几人?一共多少人马?”

“当初受了李常杰的蒙骗,一起北上的大首领,连着小人一起,一共有四人,其中以刘纪为首。但在邕州城下被神臂弓射杀了一个申景福,他的部众就被其弟申景贵接掌。在小人被派来镇守昆仑关的时候,申景贵加上刘纪和韦首安三家,兵马总共还有两万左右。不过现在大概只有一万五六。”

“邕州城防坚固,他们伤亡惨重这是肯定的。”韩冈道,“不知你能不能派人与他们联络起来?就算不能让他们攻击李常杰,能让他们早点撤回老家也是好的。”

黄金满这一次没有即刻回答,犹豫了一阵,“李常杰在归仁铺也放了一个指挥,听说了小人反正后,少不了会调遣兵马出来封锁消息。小人派人去通知那三家倒是没问题,就怕送不到他们手上,反而给李常杰给截住。”

韩冈对黄金满的回答并不意外。他很清楚,黄金满既然投靠了过来,便少不了会有担心被人分了功劳的心思在,对于招揽其他三家蛮帅不会太过用心,这件事,应当是由自己私下里派人去做,最多也只需要黄金满派一个介绍人就够了,而不是要让他来负责。

苏子元在旁边则皱着眉,他是不满韩冈将此事交给黄金满,同时也对黄金满的推脱有些恼火,“运使,这件事不如由下官去好了。如果仅是黄洞主的亲信,想必难以取信刘纪三人。只有下官带了运使的亲笔信去了,他们才会相信朝廷的诚意。”

“不行,伯绪你去不得。”这一次,韩冈拒绝得没有任何余地。

“刘纪当还不曾知道刘永之事,运使不必为此担心!”苏子元争辩着。

“不关刘纪刘永的事。”韩冈并不是因为刘纪刘永两兄弟的感情多好,才反对苏子元的行动,要说服独守昆仑关的黄金满只是件轻而易举的小事,但正如黄金满所说,要说服身边的三位蛮帅,可是要先抵达邕州城下才行,“昨夜你来昆仑关不会有危险。但去邕州城下就不同了,一路上风险太大。”

苏缄一家都在邕州城中,只跑出了苏子元一个,这种失败的可能性接近七八成的任务,韩冈不可能交给他来做。邕州城眼看着已经保不住了,苏子元再出事,日后苏家这一支就连个上坟的都没有了。

“运使……”苏子元算是明白了韩冈的心意,言辞恳切的说着,“事君在忠,事父在孝。下官去说服刘纪三人来投,对东京城中的天子是忠,对邕州城中的父母是孝。若是畏死而不去,那苏子元岂不是不忠不孝之人?!”

“不行!本来我就没打算当真能说服刘纪三人。”韩冈根本就不跟苏子元再辩,转身对黄金满道,“你派人去见刘纪三人,抓住也好,抓不住也好,对我们都是有利无害。所以你也不要选派亲信,稍微精明干练点就够了。”

说起来韩冈还更希望派去的信使被抓住,离间计比起收买、说服等手段来来,要容易生效得多。他不信李常杰能有多大方,在黄金满倒戈之后,还能安心的让广源蛮军守在自己的身边。敌军将至,身边的盟友又不稳,聪明人都知道这时候该撤退了。

“我只需要分裂交趾、广源联军就够了,至于是否会倒戈一击,我也不指望他们能学到黄洞主的一成半成。”韩冈冲着难以释然的苏子元笑了一笑,“就让李常杰和刘纪他们自相猜忌好了。”

“运使果然是智计超凡。料想李家小儿必然心生疑忌,到时候两边不合,他不想走也得走了。”黄金满满口谀词,拍过韩冈的马屁,转身又去安排人手。

韩冈看了苏子元一眼,“伯绪,你先去休息吧。昨天你辛苦了一夜,连带着今天,你可是两天没睡了。”

苏子元心情正郁结,也不想多说什么,低头行礼:“下官告退。”

韩冈望着苏子元有点虚无的背影走进城楼中,叹了一声,“眼下也只能做到这一步了。”

“也只能如此了。”李信也同样叹着,“都得看邕州到底能不能守住。”

“这件事,也只能说是尽人事,听天命。”苏子元不在,韩冈就没有必要再隐瞒自己心中的想法。他手上的资源太少了。要想凭借武力为邕州解围,至少要十倍的兵力。他一向喜欢以势压人,使用计策不过兵蹙将微时的无奈之举。

“三哥儿,我一直都想问了,这一次你对救下邕州城,到底有几分把握?”

“一成……不!”韩冈想了想,又摇头,“可能只有百分之一。”从一开始他就没有报太大的希望,但只要不是零,那就不能视而不见。放着不理,就算有九成的把握都会变为零;而尽全力去争取,百分之一的机会,也有可能变成百分之百,“总不能眼睁睁的坐视交趾人屠戮邕州百姓。”

