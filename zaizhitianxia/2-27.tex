\section{第八章 太平调声传烽烟(六)}

【第三更,求红票,收藏】

“五万?”

王韶下马后的第一句话,便是反嘲式的疑问,配着随后而来的冷笑,显而易见的表明了他心中的怀疑和不屑。

脸色突变的高遵裕被王韶的冷笑,笑得心情平复下来。他侧头看看韩冈,还不到二十岁的青年,竟然也是一副不为所动、冷静从容的模样。

高遵裕心头突然一阵火大,自己的定力竟然还不如一个黄口孺子,不过转而他又释然,这还是他尚未熟悉当地情况的关系。

“传说是五万,实际上能有多少?”高遵裕其实也不是不通兵事,方才只是猝不及防被吓了一跳,现在冷静下来,却也看出了传言的无稽。

王韶将马交给寨中迎上来的士卒,自己与高遵裕向寨中走,边走边道:“我是不知道今次木征、董裕招来的兵到底有多少,但当初董裕带了四百精锐,加上托硕部的两千多兵,可就是敢号称一万大军的。”

高遵裕算了算宣扬夸大的数字和实际兵力的比例,脸色又是一变,“那木征既然号称五万的话,今次来攻古渭的怕是也有一万多兵。”

这个数目让高遵裕心提了起来,要知道古渭寨如今的兵力,在刘昌祚带了两千去,可就只剩千人。

王韶的声音一如往日的平静:“但当时在青渭流传的谣言,却是传说董裕带了一万河州精锐来助战,连同托硕部的兵力,加起来总共两万人,而古渭寨派出去的探子有一多半回来后就跟我说,董裕托硕联军的数目超过三万。”

“这……”高遵裕终于知道传言有多么不靠谱了,“那到底会有多少。”

“不会超过一万,大概七八千上下。再多了,木征家的粮食也会支撑不住——而且木征还要留兵在家,防着他的叔叔。”王韶对自己的判断很有自信,他回头问韩冈:“玉昆,你说呢?”

“下官决计不信木征会有胆子来攻打古渭。”韩冈不正面回答王韶的问题,老是附和,却也显不出自己的能耐。

“怎么说?”王韶为之停步,就在古渭寨的正门处,等着韩冈的回答。

“木征的性格,其实应该跟俞龙珂差不多,都是小富即安,割据一地便心满意足。要不然他也不会容忍他的叔叔董毡做着赞普——再怎么说,木征都是唃厮罗的长孙,虽然其父瞎征与唃厮罗反目,但他承继吐蕃赞普的资格却还是在的。可木征其人虽然有野心,过去也做了不少小动作,但他却始终不敢跨出最后一步,自立为王。”

高遵裕点头赞着韩冈,“玉昆果然对蕃人知之甚深,这勾当公事一职倒真的没给错人。”

“多些提举夸赞。”韩冈谦声谢过高遵裕的夸奖,他站在在寨门前说话,一行人就将古渭寨正门堵上,内外为之阻隔,但韩冈却不管这么多,犹定住脚继续说着:“既然木征是这样的性格,他又怎么会敢明目张胆的过来攻击古渭?!就算他胜了,也得不到什么好处,若是他败了,周边蕃部想把他取而代之的不知有多少,更何况人在南方青唐王城的其叔董毡,也不会放过这么好的机会。”

大部分人的性格其实都是如此。但凡有了一点成就,心中所想的就是保全眼下的一切,就算他还有更进一步的野心,但他也不会愿意去为了遥不可及的目标,而去冒不可测的风险。俗语说的‘千金之子,坐不垂堂’,便是这个道理。

“那今次来攻的究竟是谁?”高遵裕追问着,凡事总得有个领头的吧,韩冈说木征不会来,那今次领着近万人来报复的,又会是谁?

“董裕!”韩冈回答着他的问题。

“只是董裕?!”

“木征和董裕早早就分了家,上次被打得落荒而逃的也是董裕。木征根本就没吃亏,损失又不是他的,木征又何必为了董裕的事而火中取栗?蕃部不似汉人,即便是亲兄弟之间也不会有多少生死相系、荣辱与共的想法。木征父祖之间的争斗,还有其父与董毡兄弟相争,都是明证。”

韩冈的一番话说得高遵裕连连点头。“子纯,你看玉昆说得有没有道理?”这下轮到高遵裕征求王韶的意见。

韩冈的推断,王韶其实也在一直在想着,也觉得有道理,“应该是董裕。木征的确不会来,他没必要冒险。不过董裕哪儿来的这么大的胆子?而以他的身份,又怎么可能召集到其他蕃部来帮他出兵复仇?——他可不是木征。”

高遵裕这时发现他们把城门的道路给堵起来了,忙向里走了几步,把城门口让了出来。他和王韶又重新向寨中走,高遵裕也揣测着董裕为何能找来这么多帮手。董裕不是他的叔叔董毡,手上也没多少部众,能挤出两三千三四千就不错了,而今次来攻古渭的兵力数目,就算没有一万,也有七八千之多,这么多人,光凭董裕的威望,不是短时间就能召集得到的。

“该不会董裕把是隆博部给卖了吧?”隆博部是当日的罪魁祸首之一,与托硕部的战争也是从他们手上开始的,董裕如有机会,当然愿意把隆博部的所有权分给他人做礼物。

“隆博部早就给托硕部吞吃的一干二净。”韩冈摇着头,他不知不觉的就把说话的主动权拿到了手中,“当初机宜领着七部合攻托硕,就是因为他们攻打隆博部时做得太过分,杀人、劫掠,把隆博部洗劫得一干二净,甚至还动到了运去古渭的军粮。”

当日王韶领军攻打托硕,是从渭源掩杀过来。而在此之前,隆博部早就被有着董裕支援的托硕部给打残了,丁口、牲畜和财物都被抢走。而王韶击败托硕部后,所有的战利品则是给纳芝临占部为首的七家部落分掉,在这其中隆博部一点便宜都没占到,没有挽回任何损失,相反地,还被七部强要他们出兵的费用。如今隆博部是穷困潦倒,每况愈下,眼见着就要分崩离析了。

“西北的蕃部都是无利不起早,杀一头骨瘦如柴的羊,骨头有得啃,肉可没处吃。董裕可没本事就靠着穷困潦倒的隆博部把人骗来。”

“那董裕用的是哪里的财物来勾引人?”高遵裕问着。

“纳芝临占部,以及所有跟从机宜扫荡托硕部的各个部族。董裕肯定是把主意打到了他们的身上。”韩冈说完,瞅了瞅王韶。而王韶则是鼓励性的点了点头,显然他也抱着同样的看法。

“为什么不是古渭寨?”高遵裕刨根问底的追问道。

他环视寨中,寨内的街市空空荡荡,别提车队,就连行人都难见一个。而城头上的守兵也是站得稀稀落落,刘昌祚将寨中兵力带走了三分之二的后果,就是使得古渭寨只能做到最基本的守御,勉强守稳城头。比起董裕手上的兵力,城中守军可能就只有他们的十分之一多一点。

以董裕手上的蕃人的兵力,高遵裕觉得他们已经足以打下古渭寨。而古渭寨中的钱粮、军器,可是很大一笔财富,不是灭了七部的缴获能比得上的。

韩冈向高遵裕详细解释他的理由:“因为两件事难易程度不同。要挽回颜面和前次的损失,董裕从纳芝临占部为首的七家蕃部身上就能做到。而攻打古渭寨,打不打得下来姑且两说。即便打下来了,他可就是捅了马蜂窝,必惹得天子震怒,立刻就要面对全力反击的西军。董裕区区一个蕃人首领,怎么可能有能力当得起了天子的愤怒?!”

高遵裕还在想着韩冈的话。这时候,留守寨中的副城主已经闻声相迎。比起雄阔豪勇而又心思慎密的刘昌祚来,他的这个副手在气象上就差了许多。点头哈腰的把王韶、高遵裕请进了衙门中。服侍着几人在寨中官厅里坐下,他又在嘘寒问暖的前后跑着。

有些不耐烦的让他在一边坐下,高遵裕问起了最为关键的一个问题,他同时问着王韶和韩冈,“那我们该怎么做?”

‘我们该怎么做?’这个问题问得好,不管对手是怎么样的人物,也不管他们有什么盘算和计划,最后的关键还是在自己身上。

可不论王韶、韩冈再怎么信心十足,眼光再如何锐利,都不能改变古渭寨中只剩一千兵的事实。他们想要援救以纳芝临占为首的亲宋七部加起来也只有四千多。而他们需要面对的将是至少七千以上的吐蕃蕃兵。

最关键的,是今次王韶再也不可能重复前次对付托硕部时的奇兵突出,从后方偷袭董裕。吃过一次亏的董裕只会小心再小心,根本不可能再给王韶得意的机会。

可王韶和韩冈还有底气,毕竟在青渭一带并不是只有他们一家,另一个主人却也不是任人欺凌的主儿。

“俞龙珂会眼睁睁的看着董裕扫荡青渭吗?”

青唐部的实力也许不及木征,但光凭董裕和他引诱来的乌合之众,却也不可能吓到青唐部的大首领。

“得去找俞龙珂说说话了。”王韶轻快的语气说得就像是要去串门。

