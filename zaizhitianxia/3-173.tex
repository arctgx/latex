\section{第45章 成事百千扰(下)}

“想不到韩冈连龙骨、船肋都知道,他还真是关西人吗?”吕惠卿回头对弟弟吕升卿笑了一声,回头再问趁着夜色,来府中报信的军器监丞:“用钢铸龙骨仅仅是贵吗?”

“不仅是贵,而且也没那么多好钢,磁州一年也不过那点分量。大炉作也没有这个能耐。龙骨、船肋耗用的钢料实在太多了。”白彰的口气很确定:“下官虽然没见识过如何造船,但总算见识过修船。几年前朝廷重修,就用了军器监的人。”

“修御舟?是黄怀信主持的吧?”吕升卿对此还有些印象,“当时是将御舟拖到金明池边叫大澳的池子里,把船用木桩架离了水,在架子上修船。后周显德年间的辟金明池时就造的观水军交战的御舟,一百多年了,这还是第一次修,换了多少朽烂的船板下来。”吕升卿啧啧着嘴,“除了里面的架子,几乎都换了,跟打造新的一样费时费工。”

“说书说得是。修船的铁钉全都是小金作打造的,当时还没军器监呢,下官也还在三司胄案衙门里听候差遣。”

军器监成立之前,下面的作坊主要都属于三司胄案,不过现在胄案已经给撤销,统管军器制造的就只有军器监一家。这其实就是吕惠卿一手推动的。

白彰继续向吕家两兄弟介绍道:“龙骨、船肋就像房子的大梁、椽子,用得材料决不能节省,好歹要几千斤钢料。一柄斩马刀也只要二两钢,一艘铁船的龙骨和船肋如果都用上钢料,几乎是斩马刀局半年的花销!”

白彰听说了韩冈要用钢料铸龙骨就哈哈大笑了一场,现在在参政府中提及此事时,依然忍不住要笑,“不当家不知柴米油盐贵,韩舍人实在太心急了,三五艘铁船就用掉天下武备一年的钢料,桑家瓦子变戏法的张宝儿能无中生有、望空采花。韩舍人如果当真要用钢料来造船,下官就只能去求张宝儿了。”

“韩冈说用铁直接铸船不行,当真是不行吗?”

“如果想要一次铸成,注定造不了大船,几千斤的铁佛铁钟铁鼎好铸,十几万斤的船那可谁都没办法。下官也打听了,凤翔斜谷船场,一艘六百料、七百料的纲船,所用的木料就要上万斤。换成铁,三五万斤少不了的,再大一点的船,那就要十万斤往上了……天下没人有这本事!”

“蒲津渡【位于今山西永济】上的铁牛一头也有十几万斤,怎么不能铸?”

吕升卿走过黄河蒲津渡上的浮桥,拴着蒲津浮桥的八头铁牛,连着下面的底座,平均一座十几万斤也都是有的。如今的铸造工艺不会比唐时逊色多少,怎么就铸不成?

“说书,铁牛那可是实心的,而船是空心。说道空心,鼎也空心,但鼎身多厚?船身最多可也就只能有一寸厚,否则肯定会沉。韩舍人也是这般说的,还说了如何换算。说是铁船要想浮在水上,其自重必须要轻于排开的水。”

“说得有理,做起事来却不成。”吕升卿哈哈笑道:“一向以为韩玉昆是做事的人,治才了得,没想到换到了军器监,却是连出笑话。”

吕惠卿没跟着弟弟一起嘲笑韩冈,他犹记得当年在王安石府,刚刚得到官身的韩冈在王安石面前侃侃而谈的场面。小瞧对手,从来都不会有好结果。

“你前面不是说韩冈准备打造铁板吗?”他问着白彰。

“若是打算学着木船那般,想把铁打成船板也难。”白彰摇着头,“抡锤子可不知捶到熙宁几年去。下官听说关西岷州的滔山监。在铸钱的同时,也打造军器。他们在锻造甲页和刀剑时,用的就是江西景德镇破碎瓷石的水碓。比人力要省,只是冬天没水的时候就不行了。韩舍人也说了水碓的事,但东京城里的河水,几乎都是开辟出来的沟渠,水流极缓,根本用不了水碓。所以已经悬赏百贯,征求用畜力或人力的锻锤。”

吕升卿还是忍不住要笑:“临时抱佛脚,就不知有几分用了。”

“未必没有成效。在白马县帮他开井的那一个井师,不是已经授了官了吗?钱是小事,但如果有人念着一个官身,肯定会为此尽心尽力。”吕惠卿板着脸说道,“还有帮着天子打造沙盘的田计,他可是捏泥人的出身,照样被韩冈荐了做了官,如今挂名在枢密院中。”

“此辈亦能为官……”吕升卿的口气有着说不出的讽刺。

“有功于国,鸡鸣狗盗之辈亦可用!”

这些年来,吕惠卿被那些只有嘴皮子的政敌恶心透了,越发的认同起魏武帝的用人策略。

而从神臂弓开始,但凡能献上军国之器的,朝廷都不会吝于一份俸禄。田计得官理所应当,而来自于蜀地的凿井法,一年来也在韩冈着力推广下,在京畿传开了。旱涝保收四个字,引得多少村子凑钱凿取深井,打造提水的风车。那井师也是帮着救了几十万流民的!吕惠卿并不会可惜赏赐给他的官身。

“即便能有人献上锻锤,也不知何时能将铁板打造好,而且龙骨、船肋的事没有解决。”白彰在兴国坊已经有二十多年了,很清楚一项新技术推广起来有多难,“造船并不容易,就算是木船也要从几大船场调匠师入京。想让他们习惯用钢铁来打造船只,并不是短时间就能见成效的。”

韩冈打算造出的铁船,需要调集大量的工匠,需要耗费巨量的人力、物力。最关键的,还得要有足够的时间——这个结论就是吕惠卿想听到的。只要韩冈不给他惹事,吕惠卿乐得他在军器监造他的船,花个十年八年都没关系。

“韩玉昆既然要造铁船,就让他造好了,我这边也会全力支持他的。在造船之事上,监中上下都依他号令,不得懈怠或拖延。”吕惠卿慢速低沉的语调,使他的命令让人不敢违抗。

白彰连忙抱拳:“下官遵命,请大参放心。”

……………………

另一个夜晚,另一个府邸。

冯京对着垂手弓腰站在面前的青衣官员笑着,“吕吉甫倒是好心啊,竟然在造船上全力支持韩冈。”

“吕参政只不过是想让韩舍……韩冈无暇顾及他事而已,并非真的好心。”

“所以说他是太好心了,民脂民膏是这样用的吗?”冯京的眼神冰寒。

青衣官员点头哈腰:“相公说的是。”

“听说韩冈悬赏了百贯来征求什么锻锤?”冯京问起了另一件事。

青衣官员失声笑道:“其实就是一个舂米的锤子改的,韩冈还照样赏了他五十贯。”

“这是千金市马骨!”冯京冷笑了一声,韩冈的伎俩并不出奇。喝了两口热茶,他慢慢的问出了关键的一句:“军器监的花灯准备得怎么样了?”

“相公放心,肯定能赶在上元节前做好!”

……………………

已经是正月十二,离着上元节只有两天。

韩冈这两日心情很不错。

他在军器监的数千工匠中,为新式锻锤而悬赏。只用了五天,就有了回报。

最简单的一种锻锤,是用脚踩的,就是农家用来舂米的那种,只是将石臼改成铁砧罢了。用着简单的杠杆原理,长长杠杆,短的一段是落脚的踏板,而长的一端拴了个五六十斤的锤头。人站在踏板上,上下踩动,就能将锻锤驱动起来。尽管看起来的确很可笑,但还是比抡大锤要方便得多!尤其是落点不会偏离,十分的稳定,即便是新手也能使用。

另外还有几具锻锤,则更像是真正的机械。也有用脚踏的锻锤,不过一人就可以操作和使用,竟然用了连杆,仿佛是纺机的变形。另外还有两具利用畜力的,都是利用绳索或是皮带传动,带起两百多斤的锤头在一人高的地方落下。

那等舂米型锻锤的结构简单到可笑,而其他几具锻锤结构也同样并不复杂,但效果显著。脚踏锤力道较轻,却可以用来打造精细的部件。而畜力的锻锤,将一块五六斤的熟铁锭,捶打成甲页一般薄的铁板,则只用了吃顿饭的功夫而已。

这也是没有水力锻锤的替代方法,如果利用水力,一眨眼的功夫就是一锤落下。韩冈挂在书房中的佩刀,就是出自于滔山监的铁匠营中,真正经过百次反复折打的百炼钢刀——水力锻锤有两种,一种力道重而慢,一种轻而块。两种锻锤各有各的用处。景德镇瓷器的原料供应,也全靠重锤破碎瓷石,小锤细锤成粉。

只要鱼饵足够大,鱼就能游得足够快。在韩冈看来,吕惠卿、曾孝宽实在太过于浪费军器监这个宝库了。这几千天下最出类拔萃的工匠,他们只需要一个方向性的指引和一块足够大的肥肉,就能爆发出让人惊叹不已的力量。

技术早就到位,只要换个思路。

韩冈的心情很好,今天就随着曾孝宽一起,来看着准备用在上元节灯会上的紧急赶制而成的彩灯灯山。

军器监彩灯的造型是一艘单桅帆船,真船一般大小。用着薄木片赶制而成。沿着船帮挂了一圈小灯,高高挑起的桅杆上,也吊了十几个大灯笼。而船帆,上面挂了数百个小灯笼。外面涂成了红褐色,如同铁锈一般。看着就是个世人心中铁船的模型。

白彰挺着胸脯,带着实际负责此事的官员,站在铁船彩灯前。向着两位判军器监,露出了自信的笑容。

可看到今年军器监的彩灯造型,曾孝宽脸色突变,却是又惊又怒的望着白彰。而韩冈,则是很亲热的拍了拍白彰的肩膀,笑得如同船上的灯火一般绚烂:“做得不错啊!”

