\section{第34章 山云迢递若有闻(四)}

“王韶攻下了临洮?!怎么这么快的!瞎吴叱呢,他守了几天?”

兰州通往武胜军的山道上,禹臧花麻勒停了战马。刚刚从前方奔回来的信使,让他脸色骤变。随着禹臧花麻的停下,一眼望不到头的队伍也随之止步。

两百里外赶回来的哨探,浑身上下都是尘土,不论是人马,在寒风中,身上都是热腾腾的直冒着白气。他在喘息的间隙向着禹臧花麻禀报着详情,“宋人是在三天前攻下的临洮,但在这之前,瞎吴叱就已经弃城而逃。现在他的大帐已经到了洮水西岸,将东面都让给了宋人。”

“瞎吴叱跑得好快。”禹臧花麻一肚子的不屑,张口便骂,“指望他多撑两日都不成。木征的这个弟弟还真是废物一个。难怪他老子争不过董毡,连个赞普都当不上……”

“木征没有出手?”禹臧花麻身边的一位亲将问着。

哨探摇头:“没有。”

“花麻,现在怎么办?”亲将紧张的征询着禹臧花麻,“回兰州吗?”

“温祓你说什么胡话?!”禹臧花麻回过头来狠瞪了一眼,“刀子出了鞘,不见血能回来吗?就算趁火打劫,在武胜军抢上一把都比直接回去的要好!”

“洗劫武胜军?!”温祓差点就要失声叫起,他立刻贴近了禹臧家的族长,急急的劝道:“花麻!这事可不能做啊!惹怒了木征,说不定他会把宋人引往兰州来!”

“我有这么下令吗?!”

禹臧花麻很不耐烦的说着,他只是打个比方而已。他当然不会这么做,要是惹起木征几兄弟的同仇敌忾就麻烦了。要是他们引来宋人,禹臧家可撑不住。

临洮往北小三百里便是兰州,若是木征在王韶的压力下降伏宋人,兰州可就要直面三路夹击了——改了名的古渭往北,也是有小道能通兰州。尽管那条小道长达四百里,道路亦是崎岖,但要是当宋人和木征自武胜、河州出兵的同时,再派出一支偏师,那兰州的情况就很危险了。如果到时候董毡也不甘寂寞,又从西攻来,禹臧家可就不仅仅是危险,而是将会灰飞烟灭。

“现在宋人在做什么?”禹臧花麻转过脸来又问道。

“他们好像要修城。把临洮城重修一遍。”

“花麻!不能让他们安安心心的将城修起来!”温祓立刻叫起,“临洮城一旦被修好,以宋人的守御,没人能打得下来。过上半年,周围的蕃部都会投过去。”

“慌什么……”禹臧花麻颇沉得住气,他能坐上族长的位子,也就是因为他越到关键的时候,性子越稳,“援救瞎吴叱没能来得及,但宋人要把临洮城重修起来,留给我们的时间,少说也还有两个月,不用慌。”

他想了想,道:“权且联系一下木征吧,现在不想跟他斗了,宋人来了大家都没好日子过。还有出兵的粮草要让瞎吴叱掏出来,得跟他也联系一下。”

“国中呢?要不要再去信?”

禹臧花麻早就传信给梁乙埋,但他并不指望兴庆府能派援军来。半年前在横山的会战,伤了国中元气,说是夺下了罗兀城,但伤亡如此之众,梁乙埋根本交待不过去——所以他拿刀子交待了。对于梁氏兄妹的决断,禹臧花麻还是很佩服的。

尽管不指望援兵能来,温祓的提议,禹臧花麻却还是点头,“要,怎么不要?你去写一封奏折,给我来签押。”

温祓会写党项文字,帮禹臧花麻写奏折也是常事,笔墨纸砚都随身带着。他点头答应了,就要找个干净地方写字。

“等等!”禹臧花麻却叫住他,又追加了一句,“弄只兔子来,好写血书!”

……………………

韩冈现今已经在渭源堡中。尽管他还担心着陇西城中会不会出乱子,但他现在注意力已经都被向临洮城转运粮秣的事情给占满了。

他越是看着战报,越是觉得今次的任务实在不易。

有过千年之后的记忆,韩冈对攻城拔寨的兴趣不如如今的将领,对歼灭敌人的数量则是很放在心上。横山攻略尽管失败了,可消灭的敌军都是精锐,党项人元气大伤。西夏的恢复力又远远不如大宋,从这个意义上说,这一战还是赚了。

可把话题说回到今次这场战事上,王韶拿到的斩首究竟有多少?在捷报中没说。韩冈估计他也是不好意思说。

如果是在仁宗或是英宗的时候,三十、五十的斩首,也算是功劳了,至少一路都监拿出来时不会脸红。可是放到现在,一场场大捷接连不断,每隔几个月,就是几百上千的斩首。将朝中上下的胃口都撑大了,眼光也抬高了,斩首不过五百都不好意思对外面宣扬。

今次在庆平堡、野人关和临洮城的几次战斗中的斩获,怕是加起来也只有两三百出头。而瞎吴叱好歹是木征的嫡亲兄弟,本部人众的数目绝不会少,两三千的战力还是能拉得出来。如果再添上亲附众部,上万甲兵总是有的。这一对比,就能明显的发现,王韶、高遵裕根本就没有伤到瞎吴叱的元气。

顺利攻克了临洮城的确是好事,可留下来的麻烦不小。韩冈宁愿连番大战,以上千伤亡为代价,将瞎吴叱和木征的军队一起扫平。论起野战的能力,韩冈对集合两路精锐的官军有着极大的信心,可要是在河湟的崇山峻岭之间,追逐着四散奔逃的吐蕃人,他的底气就不是那么充足了。

杀人盈野才是正道。

不杀得木征胆寒,如何能慑服他以及藏在西北的董毡。

韩冈叹了一口气,现在想这些也没有意义。还是早点把粮草给前面运送上去,出战诸军离开庆平堡时,携带的干粮只有七天的份,而现在已经过去两天了。

运送第一批粮草的队伍已经整装待发,是以马骡等牲畜为主的驮队,运送的人力都是军籍。但接下来运送辎重的人手,却不便再使用军中人力,只能在地方上征调。

王韶、高遵裕亲笔签发的调令事先便留给了韩冈,盖了缘边安抚司大印的文字已经印版印刷了出来,马上便要送去通远军的各个村寨。

韩冈不知蔡延庆什么时候能把筑城的民伕送来,广锐军的叛卒如今都在他的指挥之下,但他们的人数只够用来运送粮草,何况这些人都是堪战的精锐,拿去夯土实在浪费了一点,用来诱敌反而用处更大一点。

他再三检查着要分发下去的令文,以防有文字错漏,以至本意全非。类似于传单的令文上并没有油墨香,能涂在铅字上的油墨现在还没有出现,只是普通的墨汁。但字迹工整,且大印上的文字也是清晰可辨,不愧是雕版的产物。

韩冈现在还没精力往活字印刷术上去费精神。如今活字印刷是有,但通常都是寺庙中用来印经文和谒语,在美观和质量上,无法跟平常卖的书册相比——而且对于印书坊来说,活字印刷用的木活字很快就会损坏,而一套好的印版却能留给子孙传承,曾经有过两兄弟为了争夺一套老杜诗经印版的继承权,而打起了官司。以现在的技术条件,哪一项印刷手段更为合适,不言而喻。

“把令文都发下去,每个保甲都要传到。”韩冈将传单递回给等候命令的胥吏,“前面已经下过文,都该准备好了。传语各保保正,今次之事不许有任何推脱,否则勿怪军法无情!”

……………………

尤三石又检查了一遍绑扎的腰带和绑腿,这是他在军中十几年来养成的习惯。

在同一间屋中,他的浑家就坐在一边,正为尤三石整理着行装。垂下来的发丝遮住了脸上的表情,只能听到她幽幽叹着:“又要上阵了。”

“这是韩机宜的命令。”尤三石强调着。

广锐三千叛军都是靠着韩冈才逃了一条性命,全家老小也都是韩冈给保下来的。犯了千刀万剐的死罪,被招降后居然都不是被流放岭南等死,而仅仅是是变成了屯田的屯丁。照样能吃饱穿暖,全家人还在身边,比起在广锐军的时候还舒坦些。

现在韩冈下文征调各保甲出人服徭役,有些人不知好歹,腹诽不已。但大多数叛军士卒还是很淳朴的,知恩图报的心思都有着。

尤三石是一任保正,是由同一村中的叛军士兵们推举而出。当初广锐叛军归降后,被决定流放通远,所有有衔头的军官全数被放在陇西城边安置。而近二十处叛军的村寨中的保正、甲头,都是自行推举出来,皆深得人心,能肩负起重任。

背起行囊,提起弓刀,在妻儿的眼泪中告别而出。尤三石所在的保甲出动了一百三十多名精壮的汉子,连同渭水之滨的数十家寨堡,总计两千余保丁齐聚渭源堡。

在咸阳城投降的半年之后,广锐军重新集合。

