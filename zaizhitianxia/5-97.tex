\section{第11章 城下马鸣谁与守(九)}

【还有一更,拼着不睡觉也会赶出来。请明天早上看。】

“高字将旗,是高永能亲自领军出阵!”

当前去打探敌情的游骑回来禀报,李清才知道是高永能先出动打头阵了。

回头望一望,跟随他出阵的四千多将士,只有一千不到的是骑兵,剩下的都是步卒。而对面高永能带着三千本部,从人数上是己方占优,而从战力上,当是高永能更强一点。

随着两军两军逐渐接近,对面的宋军,李清看得越来越清楚。打头的是两个都的步卒,银枪、银盔、银甲,只有盔缨鲜红。这分明是鄜延路的选锋!

高永能所部已经离城一里多,到了城池和己方大营之间的中间线,却不见城中有人出来配合。李清暗自庆幸,幸好对面只有四百不到的骑兵,这样还能拼上一拼。

高永能压阵前行,肚子里骂声从出城前到现在都快没有停止。那位徐学士要堂堂正正的大胜,却不肯答应曲珍提议的夜袭。要是昨夜就遣兵骚扰,好歹也能让对面的敌人声势再弱上两分。

出城一里,对面的西贼身上的装束也看得清楚了,就是汉人的模样。

“是撞令郎。”高永能惊讶着:“西贼那里竟然还有汉军?!”

他锐利如同鹰隼一般的视线在敌军军势上扫过。排得很乱的阵型,并没有经过严格的训练,前面几排有铁甲,后面绝大多数则是普通的服装。散在外围护翼军阵的骑兵不到一千,看起来也不是精锐。

到了两军相距两百步的时候,李清就指挥着撞令郎就停下脚步,开始拿着神臂弓射击宋军。但宋军却是举起盾牌,在箭雨中稳步前行。

箭矢在铁甲和盾牌上叮当作响,骑兵也不停地在宋军阵边飞驰而过,但无论怎样,也无法迟滞他们的脚步。

等到两军的前列相距五六十步的时候,宋军阵中的神臂弓齐齐发射。整齐密集的箭矢,立刻就压制住了对面的弩手们。铺天盖地的飞蝗箭雨,撞令郎之前的射击不可与之同日而语。

而就在压制住对面的射手的情况下,两百名银盔银甲的选锋军,从阵中冲出。只眨几下眼的功夫,五十多步的距离就缩短为零。一杆杆银枪在暴喝声中,齐齐扎了出去。

几乎是在一瞬间,李清就发现他排在最前面一排的带甲士兵被一扫而空。而紧随其后的也都立不住阵脚,开始溃散。

‘还真不得了。’李清庆幸自己没有将真正的精锐放到阵前,今日只是试探,他可舍不得丢下太多本钱。

前方溃散的波动已经传递到李清身边的中军,他侧头望了望西面,低声自语:“该动手了吧?”

几乎就在同时,东面的前军主营号角声响彻战场,引来了无数目光。

随即大地忽然震颤,沉郁的雷音从地面响起,一支铁鹞子从大营中杀了出来。

李清放松的一笑:“叶孛麻终于肯出兵了!”

这一彪人马分作三股,奔驰在中央的一股蹄声最重,气势最猛,胸前是银光闪烁的板甲,头上也是宋军样式的头盔,手上攥着铁枪,马鞍后插着一对熟铁锏,胯下的坐骑前半身也蒙了一层防箭的皮质具装——在禁军全数铁甲化的现在,牛皮有许多都改制成马铠——这样全副武装的具装甲骑完全是建筑在灵州之役的大捷上。

拥有战甲的重骑兵直冲高字将旗。而左右两股则是没有佩甲的轻骑兵,他们没有往宋军的阵列冲过去,而是向城外宋军和城门之间穿插过去。竟是想将高永能所部全都留下来。

徐禧在城头上正为高永能大败敌军而兴奋,却没想到会有铁鹞子突然插入战场。

“杜靖,王含,你二人速做准备!”

曲珍凝视着对手的大旗,西夏金白色的战旗一面面的在战场上奔驰,铁鹞子们如同群狼,兴奋的冲向目标,要将被他们围困的猎物撕得粉碎。

想吃掉高永能的人马?哪有那么容易!他又看了慌乱中的徐禧一眼,怎么能不防着西贼出兵支援?

出阵的铁鹞子看似铺天盖地,其实因为是骑兵的缘故,奔驰起来不得不散开,实际上只有一千多骑。

高永能往敌军骑兵攻过来的方向瞥了一眼,面上不见半分惊容。手中旌旗一展,号角声气,原本外向凸前的战阵随着旗号向内一收,已经将敌阵冲散的选锋军随即退回到阵中心。

具装甲骑冲着军阵踏着震天的蹄声直奔而来。护翼军阵侧面的两个指挥,听从号令将手上的斩马刀齐齐亮出。雪亮的刀光直指前方,万军辟易,后方更有神臂弓手严阵以待。

面对宋军的防线,重骑兵们并没有继续冲击,而是提缰在阵前轻轻一转,从直奔改为阵前横过,手上早已上好弦的神臂弓向着宋军军阵释放出一片箭雨,随即远去,只留下了三两个不幸被宋军射落的骑手,躺在地面上。

而这一支宋军之前的敌人,这时候却飞快的后撤,在一转眼的功夫,就远离了百步。在撤退的过程中,阵型反比方才前进时更为严整,让高永能对他们的主将刮目相看。

‘骑兵倒是不弱,但这边的步兵退得太快。想不到都这时候了还想保存实力?难道因为是汉人的缘故,不肯真的为党项人拼命?’

依靠叶孛麻的铁鹞子对敌阵的压制,李清顺利的带着他麾下汉军退回了营地。宋军也随即转向,向后回撤。叶孛麻的骑兵在旁骚扰,却没有动摇他们分毫。

“算是全身而退,当真是侥幸。”回到营中,李清下马后就长叹了一口气,“想不到宋军的选锋越发得犀利了。”

“城里面还有几万拖后腿的兵马,太尉倒是不用太担心。”

李清闻言,唇角就多了点笑意。看不起京中的那些驴粪蛋.子,一向认为老子天下第一,西军的臭脾气在武贵的身上表现得淋漓尽致。

“这一次算是试探出了宋军的一点底细。”李清说道,“西军的确善战,但主帅实在是不成气候。今天宋人的用兵,脱节得太厉害。换作是差一点的军队,肯定是保不住了。”

“高君举的阵法,能赢他的没几个。”武贵低声的说了一句,又对李清道:“这一下可以安心攻打盐州城了。”

“还得夏州和环州那边不要有人来援救。”李清想了想,就又笑道:“不过无论是种谔还是高遵裕,都不会愿意看到徐禧立功,但他们能拖延的时间是有限的。来自朝廷的敇令,他们决不敢违背。”

“环庆军还好说,惨败之后,元气大伤,军心也散了。派个万八千铁鹞子,去守着櫜驼口,环庆军肯定过不来。但种谔那边……”武贵抬眼看李清。

“白池堡那里多半已经派兵绕过盐州城去柳泊岭、铁门关和左村泽了,攻下那几处的寨堡不成问题。”李清道,“至于从夏州往盐州来的道路,国相应当不会忘掉遣兵去牵制。”

“阻卜人不可信!”武贵立刻道。

“当然不是依靠阻卜人。”李清一笑,又道,“你去安排今天出战的儿郎好好歇一歇。我这边得去跟叶孛麻好好商量一下,明天可就不能像今天这么难看了。”

一夜很快又过去了,这一次,是城外首先敲响了战鼓。

徐禧率领众将,随着敌方的战鼓,登上了盐州城头。西面远处,尘云翻滚,不知有多少人马出没在黄尘中。

“想不到经过昨日一战,西贼竟然还是没有接受教训,是否要死光了才肯罢休?”徐禧远眺敌军,冷笑了两声。猛然一拍雉堞,指着西面,“谁愿出城迎战,为我灭此顽寇?!”

京营的四名将领,王含、杜靖、符明举、朱沛同时出列,“末将愿往。”

高永能也跟着踏前一步,“末将亦愿往。”

高永能此举,立刻惹来四名京营将领的怒视。徐禧见他还想争一个出兵的机会,立刻就笑道:“前面君举你已经挫敌锋锐,拿了头功,今天就好生休息一下。总不能一直让你的第六将出战,让其他两万多人在城中坐享其成,这对第六将的四千将士也是不公的。”

四十多岁、短髯修得整齐、相貌端正的杜靖在旁接口:“学士所言正是,也该我等出点力气了。坐食城中,实在无颜以对天子和学士的厚望。”

“也不能让西军一直累着。”

“总该我京营上一次阵了。”

几个京营将领同声反对让高永能再次出战。高永能的脸色越发的沉了下去。

昨日面对的是西贼中战力倒数的汉军撞令郎,而且仅仅是前军的前锋,看样子还是出来试探高下的。就这样还只是小胜一仗,连斩获都没多少。就此认定西贼不堪一战?可是要吃大亏的!

党项人以骑兵为主力的铁鹞子,昨天并没有尽全力。应该是战马在经过瀚海之后,体力还没有回复,致使错失良机。但歇了一天之后,战马状态恢复,不是昨日可比。

