\section{第18章 弃财从义何需名(上)}

【新的一更,求红票,收藏】

两种截然不同的说法,让韩冈无法确定真相究竟如何。人都是从自己所处的角度看待问题,自然不可能客观真实,但差别这么大,肯定有一方说了谎。以韩冈的才智,不会认为自家人说的一定就是对的,但也不会全盘相信慕容武转述的供词。

仅有的两条能确定的,就是四姨在冯家的正妻地位不受承认,如果这一点被采信,韩冈舅舅打就是白挨了,正妻的娘家人是亲家,而小妾的娘家人则是毫无关系的外人。另外,就是冯家内部有财产之争,韩冈的表弟冯从义,应是被迫离开家的,他的三个哥哥施手段赶走了他,看眼下的情况很难分得到家产。

只是韩冈还不清楚冯家三子如此作为,究竟是为了报复在冯德坤在重病时受到的屈辱;还是捏造了事实,以便能多分一分家产。而这些事,不经过仔细调查,很难做出判断。

可难道要他去找证人,一家家的询问过去不成?

想到这里,韩冈突然笑了。他来凤翔是来做明辨是非、秉公直断的青天大老爷的吗?

当然不是!

他是来帮自家表兄脱罪,帮自家舅舅出气的。李信被关是事实,舅舅被殴是事实,四姨暴毙是事实,还有他的表弟冯从义被从家中赶走也是事实。单是这四件事,让他找起冯家的麻烦来,没有半点心里负担,理由也足够了。

但清官难断家务事,真要磨起来,单是家产析断的案子就能打上几年、十几年。韩冈还见过为了一间祖屋,兄弟两人争了三十年的案子。跟冯家在衙门中慢慢耗,他哪有那个时间!郭逵很快就要到秦州了,而缘边安抚司的工作他也不能丢下太久,两三天内就要回秦州去。留给他的时间很少,韩冈希望最好能速战速决。

隔着桌子,韩冈脸上表情的变化尽入慕容武的眼底。从传言中,慕容武听说过好几桩韩冈出名的事迹。他的这个小师弟,绝不是温良恭俭让的性格,欺上他家门去的没一个有好下场。落魄的时候都敢在一路都钤辖脸上甩耳光,在关西江湖上据说挺有名气的疏财仗义的陈押司,给他弄得灭门绝嗣。何况他现在已经是官身,让慕容武不禁可怜起惹上了他的冯家。

而且韩冈正参与着河湟开边之事,是王韶的得力心腹,深受看重。前段时间,王中正奉旨往秦州,新晋的秦凤钤辖张守约同行,凤翔府就在他们的必经之道上。

韩冈受到的封赏,慕容武都在款待两人的宴席上都听说了。入官还不到半年,就得到晋升,让慕容武羡慕不已。同时他还知道,韩冈在古渭大捷中,是出了大力的,等过一阵古渭大捷的封赏再下来,他很有可能再晋升几阶。

张载本身文武双全,儒学、兵事皆有所长。他的弟子中,文武分界便十分明显。有以蓝田三吕为首的偏于文事礼法的弟子,也有如游师雄那样虽然考上进士,但依然重武好兵的弟子。至于韩冈,明显就是跟后者相似。能力偏向武事,性格也是直截了当,从不退缩。这样的性子助他得到王韶的青睐,也让他敢于孤身深入蕃部——韩冈奉王韶的将令,夜入虏帐,说服青唐部族酋的经历,已经传遍了整个关西。

这样的人物,日后的前途不可限量,根本就不是普通人能得罪得起的。慕容武庆幸他是自己的同门,也是早早的就有结交的心思。今日韩冈自行送上门来,慕容武求之不得,也正中他下怀。

韩冈不知道慕容武心中在想些什么,但坐在桌子对面的这位师兄,想跟自己结个善缘的心思从他脸上的表情中就能看得出端倪。

“多谢思文兄将个中内情说与小弟。”韩冈先谢过慕容武透露出来的情报,而后正色道:“不过正如思文兄方才所说,先外祖和家舅在凤翔军中多年,其位虽卑,却广有声名。向以名节自守,亦是自珍家门,断乎不会将女儿送与他人做妾。”

“啊……啊,玉昆说得有理!”慕容武稍楞了一下,连忙点头,“冯家当是为了洗脱罪名,才会如此宣扬。”

慕容武的附和有些勉强,韩冈的说法其实一点道理都没有。

军汉这个群体,包括没有官身的小军头,基本上是穷困的多,富裕的少。除非是龙卫、神卫、捧日、天武这样的上位禁军,尚能做到粮饷充足、待遇优厚,而那些下位禁军,还有更惨的厢军,只要家中人口稍多一点,或是有点恶习,一点俸禄登时就能耗个干干净净,供养不了一家老小。在平日里多有出来做些小买卖的,也有些不成器的帮浑家拉皮.条,而把女儿嫁给富豪做妾,还算是很有体面的事了。

而韩冈好歹做了好几个月事务最为繁冗的勾当公事,对军中弊政尤为直观,当然一切门清。外公把四姨嫁出去的时候,自家老娘早就嫁到了秦州,连大哥也生了,对凤翔府娘家的事其实不甚了了。现在清楚一切来龙去脉的,只有自家的舅舅。他这不过是向慕容武表明自己的态度和立场,不指望慕容武会相信,却希望他能相应的做个表态。

慕容武的反应不算好,也不算糟,只不过他不会站到冯家的哪一边的事,韩冈可以确定。所以他现在就可以直截了当的询问:“敢问思文兄,方才是所说的刘节推跟冯家是什么关系?”

节推是节度推官的简称,而推官,管得就是断案。前面慕容武说,凤翔府的刘节推在断李信的案子时,要重责于他。以李信的身份和后台,加上又是自首,一般情况下不至于如此轻率,冯家当是在中间推了一把手。韩冈想要问明白其中的关联,以便针对着做些准备。

慕容武意味深长的笑了笑,说道:“冯家在长兴县是大族,令表弟所在的十六房更是豪富,故而与凤翔上下的官人们有些来往。”

“原来如此,多谢思文兄为小弟解惑。”韩冈点头谢道。慕容武的言下之意,冯家跟刘姓的节度推官只是金钱往来,并没有更深的关系。

那这事就好办了。韩冈不用头疼要跟哪个官员打擂台了。他在凤翔人生地不熟,若是跟这里的哪个官员斗起来,强龙压不了地头蛇不说,说不定还会落个虎落平阳的境地。而且刘节推只是收了钱才帮忙,当是不会为了钱,而当面跟他韩冈过不去——不需要担心贪污受贿的官员会有什么操守。

“玉昆说哪里的话,几句话而已,又是极亲切的师兄弟,不值得这般多礼。”慕容武笑了两声。

韩冈再谢了一句,又重提旧话:“家舅现在家中卧床,苦盼着家表兄得脱牢狱之灾,不知思文兄能否襄助小弟一臂之力。”

“此事极易,请玉昆随愚兄来,先去拜访一下陈通判。”

以韩冈的身份,为李信作保很容易。在慕容武的带领下,他没有去跟节度推官扯皮,而是直接去见了凤翔府的陈通判。慕容武与这位陈通判有些交情,而陈通判一见到韩冈,就是一副很欣赏的态度,没说几句,就追问起韩冈婚配与否,当听说韩冈已经跟王韶的内侄女定了亲,他眼中的失望也显而易见。不过失望归失望,韩冈求他的事,他没二话就答应了。韩冈拿着陈通判的亲笔手书,到了大狱中,顺顺利利的就将李信保了出来。

在大狱外,韩冈好好的打量了一下自己的表兄,除了衣服破烂烂一点,的确吃多少苦头的样子,走路也是稳稳当当。府里的衙役的确给了面子,或者说,自己的凶威让凤翔府的衙役都感到胆寒。

接下来……韩冈站在大狱门外,想着,就是该去拜访一下自己的舅舅了。

………………

同一时刻,在凤翔城西的一座占地甚广的大宅正厅中,三个年龄不一,但相貌又几分相似的中年、青年正在厅中坐着。容貌很是普通,但脸上或多或少都有些被殴打过的瘀痕,当然他们就是韩冈的三位便宜表兄弟,冯从礼,冯从孝,冯从仁。现在他们的脸上都是一副愁眉苦脸的模样。

老成持重的冯从礼摇头叹着:“想不到李家的小子这么就被放出来了。竟然请了县里的慕容主簿做中人,在陈通判那里说几句好话就放了人。这下事情可不好办了。”

冯从仁年轻一些,脾气也略显急躁,他叫道:“我们又没错,都是那个贱婢作下的事。她要不是老想着把家产多搂给老四,好好的生意不做,谁会做这等事!?就算那姓韩的是官人又如何,俺们可是真的被打了。”

“那赤佬打上门来,我们连还手都没有,怎么也不理亏!”冯从孝也是憋气,谁能想到那女人的娘家,会突然冒出个做官的外甥来。听说还很有名气,做下了不少大事,心狠手辣得狠。不过他说对上李信的时候没有还手,也是往自家脸上贴金,当时十几个家丁一齐上,都被一个人打得落花流水,若不是他下手轻,可不只是这点皮肉伤。

