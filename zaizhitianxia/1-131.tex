\section{第47章 不知惶惶何所诱(中)}

【第一更,求红票,收藏】

“密县?”路明探过头来,吃惊道,“这不是京东东路的上县吗?官人怎么会被授予上县的县尉?”

“上县?……原来如此。”

韩冈转眼便会意过来,这是王安石给得报酬吗?未免也太小气了一点。不过韩冈挺欢迎这样的改变,“上县县尉的俸禄可比下县要高不少,没人会嫌俸禄多。”

“上县易下县,可不仅仅是俸禄多一点这么简单。”韩冈的身后传来一个莫名耳熟的声音。

韩冈闻声回头一看,便站起来行礼,“原来是刘令丞!”竟然是前些日子在铨试时给韩冈添乱而不果的流内铨令丞刘易。

刘易笑嘻嘻的过来,拱手道:“玉昆贤弟,久违了。”

贤弟?我们的关系有这么好?韩冈算是佩服刘易这样的低层官员的脸皮厚度了。虽然这样的人物并不罕见,但事有反常必为妖,刘易主动过来搭话,肯定有其原因。

刘易在韩冈一桌的空位上坐下,故示亲近的笑道:“向日一见,便知是玉昆贤弟是贤良之才。如今得王相公青眼,鹏程万里也是指日可待。”

“不知令丞此言何意?”韩冈问着。

“玉昆何必故作不知。”刘易见方才韩冈的神色一点变化都没有,哪里肯信他对此一无所知,“王相公亲自自中书下令,将玉昆的本官定为密县。上县簿尉晋初等职官,是两任四考,而无出身的下县簿尉,就至少要三任六考,也就是六年后,才能晋升。而且以王相公对玉昆你的看重,只怕三五任之内,就能转官了。”

原来如此。只是韩冈觉得让刘易有这种一百八十度转变的理由还不足,光凭王安石将自己的本官提了两级,刘易就改换门庭,这实在太可笑了。即便想搭上新党的船,也不该找尚无半点声名的自己。

究竟为了什么,刘易很快为韩冈解惑:“今天吕吉甫致书天子,但言近日朝堂诸公,往往斥青苗贷为害民之法,为一正此法利民之本心,奏请改青苗贷之名为利民低息贷,而青苗法也同时改名做利民低息贷款法。”

韩冈笑了,等了半个月,新党终于忍不住开始有动作了。虽然将青苗贷改换名头,是他出给王安石的几条策略中,最为简单易行的一条,而且是忌讳最小的一条,其他条款并无动静。但既然新党既然已经采用他的计策,那么当这个策略有了成功的回报后,接下来的几条,怕也是会陆续施行。

在刘易眼中,韩冈唇边若有若无的微笑,是一切了如指掌的自信。他心中暗喜,看来自己果然猜得没错。这名从九品的选人,当已经入了王安石的眼界,是参与核心策略的资格,说不得日后就会跟吕惠卿等人一样,数年间便会飞黄腾达。

既然自己办事不力,开罪了过去的后台,都有消息说自己最近可能会被迁到荆湖南路哪一个偏僻军州任司理参军,那换个门庭也是理所当然的。以刘易如今的窘境,即便是根稻草,他也要抱上去,韩冈虽然官卑,却也是刘易缓急间能找到的唯一助力。。

……………………

与刘易随便扯了几句,韩冈把他打发走了。刘易巴结自己的原因,韩冈到现在都无法确认,但他隐藏在笑容中忧虑,能看出来不似作伪。

只是韩冈没兴趣应付他,自己拿到了告身,他这趟来东京的行程也就到了尾声。连朝堂局势究竟怎么变化,韩冈也不想再理会,何况一个毫无节操的流内铨令丞?

秦州的事大概是解决了。与新党斗得越厉害,旧党众臣就越没有余暇去找王韶的麻烦。韩冈前些天还在驿馆听见秦州的宜垦荒地是一顷还是一万顷的争论。但今天,当韩冈回到城南驿中时,他所听到的讨论,无一例外都是与青苗法易名有关。

“青苗贷改名便民低息贷款?王介甫这是出的什么昏招?”

“改个名字就有用了?”

“犯官改了名字重新考进士的都有,这法令改个名字,说不定骂的人就会少一点了。”

“胡扯,改个名字不过是换汤不换药,本质还不那些东西。”

“你们不知道,这是三命僧化成支的招。前日夜里王大参亲自把化成请到宅中,请他发了文王六壬,算出了青苗贷的名字不吉。所以王大参才赶着改名。”

“林十七,你也别扯了,一个和尚不念经礼佛,却去当瞎儿先生,他说的话,能有几分是真?”

“不知司马君实会怎么说!”

“大概会笑……”

城南驿的外厅中一时成了菜市场,韩冈听了几句,便转身上楼。消息刚刚传开,少有几个靠谱的。但听着他们的话,他给王安石支的这一招的用心还没人看透。不过等过上几日,新党接下来的手段一个个开始实行,王安石的用意,自然很快就能传播开来。

只是自己提议的计策,却在口耳相传中变成了三命僧化成的招数,韩冈只觉得有些好笑。三命僧化成在东京城名气极大,以能断人三生休咎而闻名。他住在大相国寺的偏院中,每日宾客盈门,高官显宦从来不少,连王公宰臣家的家人都在老老实实的排队,请他推算个运数。

韩冈对此则秉持着孔夫子的态度,敬鬼神而远之,不语怪力乱神,只在外面看了两眼,就掉头离开。

回到房间后,韩冈让李小六打理了一下行装,这两天就该回秦州了,东西要先整理一下。而韩冈,则整了整衣服,往小甜水巷的方向去了。

这些天,张戬和程颢都挺忙,攻击新法的工作让他们忙得脚不沾地。因为程张二人的忙碌,韩冈已经有两天没有去拜访,如今就要返乡,韩冈当然要再见上他们一面。

僻静的后巷中,韩冈推开偏门,自行走进程家的院子。程颢、张戬都把他当子侄看待,他在两家进出自如,并不需叩门等人通报。

“玉昆哥哥。”韩冈刚走进院中,一个小女孩的清脆嗓音便传进他耳中。

韩冈循声看过去,一个十一二岁的小女孩从通后院的小门中走了出来。小女孩儿绑着双丫髻,长得雪玉可爱,一双透着天真的大眼睛,皮肤如初雪一般白净。大概是天气尚有些冷的缘故,小脸上还泛着红晕。

“是二十九娘啊……”韩冈冲小女孩笑了笑,一点也不避讳。

小女孩儿是程颢的女儿,族中排行二十九,今年才不过十一岁。是程颢在鄂州任官时所生,故起名作鄂娘【注1】。以任官之地,为子女取名,是很常见的事。司马光便是在其父司马池在光州光山县任知县时所生,其名就由此而来。

小女孩很懂礼貌,儒学宗师家的家教也的确出色,程鄂娘行礼、问好做得一板一眼。并不似老学究打躬作揖的那样礼节繁琐得惹人厌,而是平添了一分可爱,更有着大家闺秀的娴雅,可以想见她几年后的出色。

韩冈回了半礼后,就见着小女孩儿小碎步跑到身边,仰头问着:“玉昆哥哥怎么这两天都没来?”

“先生事忙,不便打扰。”韩冈低头看着程鄂娘带着稚气的一张小脸,如同山中潭水一般清澈的双瞳,就想起了远在秦州的韩云娘,不知她现在怎么样了。韩冈暗暗一叹,收起纷乱的心绪,他又问道:“先生呢?今天还忙不忙?”

程鄂娘很认真的点头答道:“爹爹刚刚回来,和表叔公在书房里。”说着,她又歪着头想了想,“表叔公心情很不好呢。”

小耳报神跟韩冈很亲近,程家张家的几个子女也都跟韩冈很亲近。程颢张戬治家严谨,对子女的管教十分严格,平常吃用都是从简朴中来,玩具什么的更是少有。而韩冈因为经常在程张两家蹭饭,有些不好意思,便在逛大相国寺时,买了几件小什物送给两家的孩子,程鄂娘手腕上的辟邪桃核串,就是韩冈送的。

韩冈是一片好意,张戬程颢也不好说什么。也因此,程张两家的子女们,看到韩冈便是哥哥长,哥哥短。

又哄了小女孩几句话,韩冈便走进程颢的书房。书房内张戬沉着脸,使得气氛有些凝重。

“两位先生,韩冈来了。”韩冈上前行礼。心知两位监察御史应该是听说了王安石今天的动作。他们不同于城南驿中的闲官们,变法派的一举一动他们都会往深里去想,所以心情看起来有些糟糕的样子。

“玉昆来了。”程颢抬头招呼了一声,张戬则闷着头不说话。

虽然韩冈心知张戬阴沉的原因,但还是得装作糊涂的问一下。他用询问的眼神望着程颢,程颢了然一笑:“玉昆,可听说过今天朝堂上的一桩大事?”

“听说了,方才驿馆中一群人正说着这件事。利民低息贷款是吧?”韩冈点点头,直言道:“这是好事啊。”

“什么?!”张戬难以置信的抬头看着韩冈,他的这个学生怎么会支持青苗法?他怒道:“与民争利这是好事?朝廷放债这是好事?!”

韩冈不以为然。管子设女闾,以皮肉钱九合诸侯匡复周室,圣人不还是说‘微管仲,吾披发而左衽’。不过这些话韩冈不好说出口,那样就真的要吵起来了。

注1:程颢在史料中留下记载的女儿有两人。年长的未留名——只云程氏孝女,而年幼的幼年早夭,在她墓志铭上记载名叫澶娘——是程颢在澶州任官时所生,时间是在熙宁四年。故而从程澶娘的名字反推回,得到了程鄂娘这个名字,也算是杜撰了。

