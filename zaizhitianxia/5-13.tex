\section{第二章 牲牢郊祀可有穷(中)}

【写得很慢,还请见谅。】

张孝杰被赐姓耶律,早就改了姓氏。但萧十三心情不好的时候,照样指名道姓的称呼他张孝杰。

张孝杰和萧十三都是耶律乙辛的亲信,可萧十三不喜张孝杰,张孝杰也反感萧十三,两人的关系,只比两只盯上了同一块骨头的饿狗好一点。

听到张孝杰更正,且又笑得一幅阴森诡谲的样子,萧十三张口就要一阵嘲讽,但他瞪过去的眼神,却在下一刻就变得诚惶诚恐,连忙和萧得里特站了起来,躬身向被张孝杰引领进来一人行礼:“太傅。”

耶律乙辛跨入殿内,礼官和内侍一同迎了上来。

大辽郑王、太师兼太傅、尚书令、北院枢密使、只差一个尚父头衔的耶律乙辛很是温和从容的向殿中众人点头致礼,而后恭恭敬敬的在辽宣宗耶律洪基的灵柩前磕头上香。专心致志处,完全是一幅忠臣贤德的模样。

对先帝的一番礼仪结束,耶律乙辛在几名亲信的簇拥下来到偏殿,坐下来喝着茶。

宫中所用的茶汤是南朝所赠,是上等的龙凤团茶,但萧得里特和萧十三饮不知味。

等了片刻,见耶律乙辛悠然的品着茶,张孝杰也是一副模样,萧十三忍不住开口问道:“太傅深夜来殿中,可是有什么吩咐?”

耶律乙辛抬抬眼,“方才是在说奚六部的事吧?”

萧得里特陪着小心的回复道:“萧谢家奴始终没有消息,所以就有些担心奚族不稳。”

“暂时不用再担心奚六部。”耶律乙辛说着,将茶盏放一边,皱眉头:“这茶没存好,走了气,可惜了这南朝御用的好茶。”

张孝杰立刻起身对外吩咐道,“去给太傅换盏好茶来。”

而萧得里特听了前半段则是眼睛一亮,立刻追问道:“难道谢家奴已经来拜见太傅了?”

萧十三也是恍然,难怪耶律乙辛心情会那么好,笑问道:“谢家奴是自己来的,还是派的儿子来?怎么一点动静都没有。”

“谢家奴没来拜见太傅,也没派人来拜见太傅。”张孝杰转身回来,卖关子似的停下来扫了萧得里特和萧十三一眼,见两人都望了过来,便略感得意的揭开谜底:“他去拜见宣宗皇帝了。”

萧得里特和萧十三先都是怔了一下,然后才将脑筋转过来。

“谢家奴死了?!”“他病死了?!”两人同时开口追问。

“谢家奴死了。”耶律乙辛叹道,“遗表刚刚送到。”

死了而不是病死。刚才萧得里特是惊讶中的询问,用词不谨不足为奇,但耶律乙辛的回答却不该如此。

萧得里特和萧十三眼中尽是狐疑,只要在官场中,对于上司的一言一行,都是揣摩再揣摩的,与智愚无关。

“他不是告病吗?”张孝杰看出了两人的疑惑,笑道,“这下病死也是顺理成章。”

没有张孝杰的话,萧得里特和萧十三还仅仅是怀疑,但现在几乎可以确定了。

“遗表里荐的是谁?”“由谁承继?”两人又是同时发问。

“自然是谢家奴的长子回离不。”张孝杰回道。

“回离不为人如何?过去没怎么听说过他。好像没什么名气。”老奚王谢家奴是有名的狡猾,但萧得里特对老奚王的儿子们并不了解,“他的弟弟观音奴倒是名气很响,前几年五国部之乱,他所领的那一部战功排在前面。”

“庸懦、贪心之辈。谢家奴本来不打算传位给他的……但现在既然暴病而卒,也就只能让回离不。”

暴病而卒还能上遗表……已经可以将几乎二字给去掉了——肯定是耶律乙辛下得手。

萧十三和萧得里特眼中透着惊惧,他们皆知耶律乙辛手上还藏着其他棋子,却没想到那些棋子的手段这般厉害。而萧十三更是瞪着张孝杰脸上的微笑,眼中闪起了凶光。自己都被蒙在鼓里,张孝杰却知道,难道太傅当真更信任这个汉人?

“既然说回离不为人贪婪,那他会不会被人收买过去?”萧得里特又问道。

“以回离不的性子,只会等,只会看,就算到了最后,真有叛军杀出来,甚至与官军杀得两败俱伤,他也不敢当真举起叛旗,出来占这个便宜。”张孝杰冷笑着,看起来很是熟悉回离不其人,他转头恭敬的望着耶律乙辛,“要不是他是个无能的庸碌之辈,太傅如何会让他当上奚六部大王?他的几个兄弟都比他强得多。”

萧十三和萧得里特已经没有任何问题了,都到了这一步,还有什么问题可说?

老奚王去世,诸子争位,回离不这位庸懦的长子是最不被看好的一个,但得到了耶律乙辛的支持,最后坐上这个位置的,却偏偏就是回离不。

大概耶律乙辛是想这么对世人说吧。萧得里特想着。而谢家奴的暴毙,则是警告另外一些人的。

萧谢家奴之死,最大的得益者不是别人,正是耶律乙辛。

奚六部大王能控制和影响的头下军、部族军不在少数,尤其是在变乱之时,凭着奚王的身份,可以驱用大批的奚族战士。

而且奚族的势力在中京道最强,中京道虽是五京道中地域最狭、户口最少的一道,但胜在占了‘中’字,拥有绝佳的地理位置。

耶律乙辛控制了南京道和上京道的核心地区,如今再得到了中京道。控制区就是从南到北连通起来了,将大辽的中轴地带彻底掌握在了手中。而形势不稳甚至有可能会有反叛的西京道和东京道则远远地被分割隔离,一旦有变,则可以各个击破。

在震惊过耶律乙辛的手段之后,萧得里特和萧十三满心都是兴奋,得到了中京道,形势便稳定下来了,全胜可期。

在害死了皇后萧观音之后,他们在朝中的日子一天比一天难捱,跟随着耶律乙辛,看似烈焰烹油、鲜花着锦、权势大涨,但危机就在身边,随时可能万劫不复。害死了太子,害死了太子妃,也不过是将危机延后而已,就算耶律洪基龙驭宾天,也没能松上一口气,决战的时候到了。直到方才听到了谢家奴的死讯,中京道和奚族六部都由此稳定之后,心中的巨石才算落了下来。

张孝杰看着耶律乙辛:“谢家奴之死即如天助。宋人肯定会以为我北朝国中必有内乱,当会趁此良机攻打西夏。只要届时出兵扶持西夏,必能给南朝君臣一个惊喜。”

耶律乙辛点了点头。萧得里特和萧十三都是眼中一亮,的确说得没错,“到时候不论是出兵西夏,还是干脆出兵河北,都能让宋人措手不及。”

“当然是出兵河北,去西夏尽是荒漠,哪有南朝的河北富庶!”

“到时候陈兵黄河之滨,甚至打到开封城下,两百万、三百万的岁币亦不在话下。”

“什么岁币,夺了宋人江山,不全都是大辽的。还可以亲眼见识一下江南风月。”

两人越说越兴奋,而张孝杰对耶律乙辛一拱手:“能随太傅骥尾,征服南朝,名垂青史,实在是孝杰三生有幸。”

耶律乙辛微微一笑,萧得里特和萧十三得到提醒,先在心中骂了一句马屁精,转过来也开始对着耶律乙辛满口谀词。

两名内侍这时进来换茶,正在说话的三人都停了口,殿中一时间变得安静下来。

耶律乙辛端起新换上的茶盏,嗅了一下散出来热气,不置可否,但终于是微不可察的点了点头。关注着他脸色变化的张孝杰,神色也放松下来,示意两名内侍出去。

耶律乙辛啜了两口茶水,突然没头没脑问道:“听没听说过种痘?”

萧十三一脸茫然,萧得里特没听明白耶律乙辛想说什么,很是莫名其妙的问道:“是黑豆、黄豆的种豆?”

耶律乙辛想起了昨日自己第一次听说种痘时的反应,笑了一笑:“痘疮的痘。”

萧得里特愣住了,接着寒从脚起,瞬息间传遍了全身,“该不会是散布痘疮杀人吧,谁有这种手段?!哪里传的?!”

耶律乙辛脸色也变了一下,皱眉低语:“能治人多半就能用来杀人,或许真有这个能力……”又抬起眼,笑道:“不过现在传出来的种痘,只是预防痘疮,种过之后,一辈子就不会得痘疮了。”

萧十三和萧得里特都摇摇头,这些天来,他们的精力都放在如何积蓄兵力,拉拢实力派上,哪有心听这等没来历的传言。再看看张孝杰,脸上的表情也是茫茫然,他这些日子,跟萧十三和萧得里特一样,操心着怎么为耶律乙辛拉拢支持者,没有多余的时间和精力打听无稽的谣言。

但耶律乙辛不会无缘无故的出言相问,当是有来由的。

“是哪里的谣传?”萧十三问道。

萧得里特也在问,“是最近才传出来的吗?”

“是不是从南朝哪里?”张孝杰最后一个发问,他感觉种痘应该算是医术的范畴,而论起医术,当是南朝更出色一些,医书多、名医也多。每年使节往来,南朝总是赠送大批的丸剂、散剂等成药,往往被用来赏赐臣下。

“的确是从南朝传来的。”耶律乙辛说道,“献上此术的,名唤韩冈。”

