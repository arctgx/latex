\section{第二章 凡物偏能动世情(五)}

吴充根本就不将韩冈当一回事,口舌之争吃亏又如何,慢慢钉死他晋身宰执的机会,就已经足够了。

“若说韩冈在白马县安置流民的时候,以当时所见,他日后甚至能有五六成机会做到宰执;但到了现在再来看,韩冈倒有七八成进不了东西二府。”

对着已经开始在点头的儿子,吴充笑着韩冈的愚行:“王介甫辞相后,他竟然一口气开罪了韩子华、冯当世和吕惠卿这三位宰执,论起得罪人的本事,也只有祢衡能比一比了。”

吴安持无话可说,他的父亲的确是一针见血说到了韩冈的缺陷上。

“王介甫再是因为新法得罪了多少豪门世家,让多少旧友与其反目成仇,可他至少在担任参知政事之前,没有让人看出了他的真面目。天下人都将他视为能拯救朝政困局的大贤来期待,朝堂上下,除了寥寥数人之外,无不是在期盼着他上京,入主政事堂。

而韩冈就实在是太过高傲,宰执他不亲附,士人他不结交,诗文水平连浅薄二字都不够资格形容,官员中的聚会根本就没办法去参加。唯一擅长的就是机关巧器之学,只是美其名曰格物致知,将公输般与先圣拉上关系。”

“韩冈好歹也是造出飞船,公输般恐有不及。”吴安持轻声提醒着父亲。

“《浮力追源》的确说透了飞船的原理,现在是人人都会造了。等到日后给辽人、夏人学去,你再看看他会受到多少封弹章!”吴充冷哼一声,“好了,你早点回去睡,记住为父说的话,不要与韩冈结交,省得日后受牵累。”

“儿子知道了,父亲大人也请早点安歇。”吴安持老老实实点头,行了礼,就下去了。

吴充重新往内间走去。

方才说的一切,这倒也不是他针对韩冈的原因。吴充只是看得不顺眼直接开口说而已,区区一个没有多少前途的起居舍人,他一任枢密使根本没有必要顾忌。

韩冈拿着过去的功绩和发明,在天子面前有着足够的影响力。但凡他说的话,天子能信上七八分,说不定什么时候,他就将王安石给弄回来了——这样能动摇天子心意的小臣,换作哪位宰执过来,都不会看得顺眼。

只是冯京此前苦心积虑在军器监做下的那些龌龊之举,反而成就了韩冈的名声。表面上捧着,暗地里做手脚,冯京做的蠢事,吴充可不会去学着来。韩冈的确是才智过人,对付他即便机关算尽,也免不了要落入陷阱,倒是直接出手打压,韩冈也只能老老实实的用嘴皮子辩驳。

寻常臣子,以韩冈的才能、功绩和声望,根本不可能只有这么低的官位,他现在只是因为太过年轻之故,说起来,的确有不少人在猜测,韩冈要熬到多少岁才会晋身政事堂,都将他视为未来的宰执。

但这样的臣僚,也让人心生忌惮。随着韩冈年岁见长,将他视为威胁的就会越多。到时候,不论是谁上台,都会设法阻止他进入政事堂,甚至阻止他上京城。他声名越盛,两府中的宰执就越是要压他。

天子的宠信绝不可能保持很久,韩冈对天子的影响力也不会保持太长时间,而士大夫之中的关系和人缘,却是时日久长。无人可以依仗,无人可以用为奥援,只有寥寥数人为友,试问韩冈能在官场上走多远?

吴充一点都不会担心。

……………………

同天节还有两天就快要到了,但赵顼却是越来越不想上朝,好不容易熬过了朝会,又在崇政殿议事上,被搅得昏头涨脑。

朝廷如今变得泾渭分明,好好的朝会和议事,最后都不免变成了臣僚们或阴或阳的互相讥刺和弹劾。为了两桩没有最后确定审判结果的案子,朝堂的重臣们已是撕破脸来攻击。而两桩案子究竟的实情如何,他们都不再关心。

赵世居、李逢谋逆案,如今已经牵扯进了数以百计的士人。

因为李逢本来就是士人,而赵世居交游甚广,也与许多士大夫书信往来。从他们家中抄出来的书信,有许多让赵顼耳熟能详的名字,甚至有些人都是经常见的。甚至连自己的四弟嘉王赵頵也被牵扯了进来——他曾经请求将参与进谋反案中的医官刘育,任命为嘉王府中的医药袛应。

但赵顼决不想就此停止,他已经忍了很久了。从开始削减宗室的待遇时,就一直在忍着宗室对他的攻击。以《宗室法》将一大批远支解除宗室的身份,让他们失去任官的机会。转运法、市易法,都在宗室身上割肉。试问他的亲戚们怎么可能会甘心。在去年的那场大旱中,上蹿下跳的人实在太多,他听到的嘲讽和讥笑也实在太多,但赵顼他堂堂天子,在当时却只能干咽下这口气。

赵世居绝对饶不得!赵顼要确定,这个朝堂之中,没人能动摇、敢动摇他的皇位。只是从另一位谋反案的参与者李士宁身上,使得王安石也被牵扯了进这一桩案子,让赵顼烦心不已。

而另外一桩案子,也就是官营水磨坊的厢军士卒团聚起来,冲击韩家宅邸的案子,已经变成了厢军士兵无事啸聚、谋图不轨。吕惠卿的极力主张,终于让赵顼也觉得被煽动起来的厢军士卒,也的确是一个危险的信号,不能轻易的放过。而且这桩案子更是从一开始的指使者,在御史台中牵扯到了宰相冯京的身上。

两桩案子,让两府之中的宰执们变得更加对立,之间的矛盾也更加锐利。此事的朝堂上,宰辅们也只记得互相攻击,让朝廷政事运转也渐次缓慢下来。

赵顼决定要尽早将两件案子给处理干净,将两帮涉案人员以重惩,给予天下一个足够的警示。不过,赵顼叹了一口气,这也要臣子们配合才行。依他的才智,如何会看不出现在两边揪着这两桩案子,其本意到底是为了什么!

开封知府韩缜这时站了出来,前面的宰执官们为了两件案子争吵了一通,现在终于都累得没有气力了,让他也可以站出来说些话。

“陛下,前夜观音院外码头上失火,现有多家绸缎商联名具状,控诉力工袁十二等二十余人于码头上纵火行凶。”

“纵火?”赵顼最不喜欢的就是听到这两个字,东京城这样的大城市,最怕的就是火灾,一旦烧起来,就是接连数坊一起陷入火海,“回去速速断明此案,将涉案之人依律严惩!”

“启禀陛下,臣昨日已经问过了几人。”韩缜冲着崇政殿上至尊的座位躬身道:“袁十二等人是因为码头上安置了轨道和有轨马车,不再需要太多的力工而被雇主解雇。袁十二等人力工与之争执良久,最后一气之下,方有了这一次的泄愤之举!”

“轨道!有轨马车!这些不都是军器监的东西吗?!”赵顼一直都在关心着军器监,韩冈现在的精力放在那边,他当然很清楚。

韩缜也很清楚:“轨道和有轨马车,如今尚在军器监中打造之中。但这个消息,京中已然传遍。想必绸缎商们也是因为听说了此事之后,才会收买了军器监的工匠,将轨道和有轨马车的式样给偷了出来。”

原来事情做得滴水不漏的能臣,这段时间怎么尽出纰漏?!赵顼越来越是头疼了,真的是什么都不顺心。

“速召韩冈觐见!”带着怒意,赵顼吩咐了人下去。

宰执们还没有走,甚至准备来个第二回合。没想到韩冈又出了事,要上殿来了。吴充捧笏而立,低垂下去的头在冷笑着,韩冈这是自己在错蠢事,怨不得他人了,就看看他能怎么为自己辩解。

技术扩散绝对是一件好事!

韩冈还记得某本科普书中,曾经看到有一位十七还是十八世纪的科学家,为了得到贵人们的资助,而在家门前竖起了一个气压计,打了一个十分有效的广告。

他韩冈也是在打广告,如今七十二家正店门外的热气球,还有在金明池时不时飘起来的飞船,都是他韩玉昆打出去的广告。而结果也很好,浸铜法和火车的原型都顺利的流传了出去,也让人看到了相信他韩冈的好处。

日积月累,日后不论他准备做什么,想必都会有人趋之若鹜,去学着实验、去试图仿造,得到成品后,就去大力推广,说不定他还有机会看到蒸汽机的出现,只要他为此说上一句。

至于一些后患,就完全没有必要顾忌太多,或者说带来的损失,要远远小于获得的回报。这个回报,是他在天下人心目中的地位,虽然眼前无用,但韩冈想着的就是日后。

力工失业那又如何?他们占了天下人的比例多少?占了东京城中的人口多少比例?有多少人会去在意?事不干己,世人也只会当成一件轶事来流传,而主角还是他韩冈。

不过力工们也不是没有去处,只要有把子气力,哪边都不会饿死。虽然前天夜里的那场火灾,在童贯领旨到来前,韩冈就已经听说了。但他绝不会在意,被烧掉了轨道的绸缎商们也不会在意,只要有利可图,那些奸商们肯定会做出更加刻薄的举动,让轨道推广的更加迅速。

心中这么想着,走进崇政殿中,韩冈一点也没有担心。

