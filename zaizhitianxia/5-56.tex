\section{第九章 拄剑握槊意未销(四)}

“官军围城已达半月,西贼竟不敢应战,可知灵州光复已是指日可待。”

“高遵裕用心,苗授也同样用心,能有现在的结果,全是他们用命王事的结果。”

“不仅仅是高苗二人,王中正也同样用心,他沿着黄河走,一路过关斩将,种谔、李宪如今虽被挡在瀚海,但之前也多有功绩。别说他们这几位主帅,就是那个戴罪立功的王舜臣,不也是已经打到了凉州城下?”

“王卿家说得甚是,诸路都是高歌猛进,西夏已经是日暮途穷了。”

“以如今官军的威势,最多再有半月,王中正必然能赶到灵州城下,到时候,就算灵州城还没破,又怎么挡得住二十万官军的合击?!”

崇政殿中,韩冈正板着脸听着赵顼、王珪君臣二人如同梦呓的一搭一唱。

用了半个月都没有攻下灵州城,还能指望一个月后攻下吗?粮道还能维系多久?士气还能保持多久?

至今为止,一场规模以上的会战都没有,党项人打着什么样的主意难道还用多想。在他们的底牌翻出来之前,根本就不该多做幻想,但天子和王珪偏偏都看不到这一点。不是才智、眼光不够,而是他们下意识的将所有与危险有关的征兆和念头都忽略过去的缘故。

“种谔迁延不进,着实可恶。但王中正当是快到灵州了,想必能助高遵裕一臂之力。”

“种谔终究还是平定了银夏,李宪也是保护了粮道,还是得加以褒奖。”

“说得是,王卿家说得甚是,”赵顼大笑着连连点头,这几天,他的嘴角都笑出纹路来了。他转头看到了韩冈的身上,眯眼笑道:“韩卿,这一次你可是要输了。灵州眼看着可就要打下来了”

“如果臣错一次,官军就能赢一回的话,臣倒甘愿多错几次。”韩冈见赵顼嘴角又要得意的翘起,话锋一转,“不过环庆、泾原围攻灵州半月,而西贼竟不出兵援救,必有奸计,还望陛下下诏让其慎重。”

“韩卿还是多虑了……”赵顼一摆手,满不在意,“说不定现在已经打下来了,过两天消息就能到京城!”

“老成稳重是好事,但须知过犹不及。”王珪摆着架子教训起韩冈这个后生晚辈来,“且韩冈你与他们共事多年,对高遵裕和苗授应该了解甚深,难道他们是轻敌躁动的人?他们一样是军功显赫的名将啊。”

韩冈没有附和,却也没有反驳。这时候就没必要多说什么,等结果来就能知道了。

凡事都往好处想,这是军事中最大的忌讳。事情总是会往最坏的一面发展,韩冈两生几十年的经历,对此深有体会。

但自己的区区一个同群牧使总是被请上崇政殿,是想听自己唱反调,还是想看到自己最后预言失败,然后灰头土脸的样子,还真是说不准。只是看了看赵顼和王珪脸上得意的笑意,自己总是往人心险恶的方向去想的习惯,也不能算是错了。

翰林学士蒲宗孟今日当值,在殿上将嘉奖众将帅的诏令一挥而就。赵顼和王珪看过一遍后,便签押盖章。

诏令一封封的发出去,韩冈和蒲宗孟从殿中出来。王珪没有离开,他还要留在殿中与赵顼预先庆贺西夏将亡,韩冈甚至还听说王珪私下里已经让太常礼院去筹备告祭太庙的仪式。

蒲宗孟与韩冈并肩走着,走了一阵后突然笑道:“玉昆还是这般强项。看到玉昆,就想到舒国公了。”

“传正谬赞了,韩冈还差得甚远。”韩冈谦虚了一句。

他倒是没想到,蒲宗孟竟然语带讽刺的提起新近被封为舒国公的王安石。拗相公三个字,可不是什么好词,骂人的话。他好歹也是新党,什么时候跑到王珪那里去了?

不过仔细想想,倒还真没什么好意外的。

平定西夏的功劳极大,十个交趾加起来都比不上——当初为了一个罗兀城,都是由宰相韩绛统领——加之成功率又高,不跑过去分一杯羹,难道像自己一般跟天子顶着来不成?

王珪一脉这些天气焰极盛,其本人还要装出一副宠辱不惊、胜败无碍的宰相气度来,但他门下的走卒却是趾高气昂。蒲宗孟眼下也可算是一例了。

韩冈如今已是宠辱不惊,毫不在意与蒲宗孟一路谈笑。到了他们这个位置上,当面骂阵就太失身份了,心中记着就好。

转到文德殿前,权御史中丞、兼判司农寺的李定迎面而来,见到韩冈和蒲宗孟并肩而来,远远地就打招呼行礼。韩冈和蒲宗孟连忙上前回礼。

蒲宗孟看看李定,“资深可是要去崇政殿求对?”

“正是。不知现在天子是否还在崇政殿中?”

“天子正在与王相公说话。”

今天早朝时,韩冈还见到了李定。当时李定就在文德殿的东阁处向人称赞苏轼,说他是大才,几十年前所作诗文都能记得一清二楚,不过李定周围就没人敢接这个口。

三人又寒暄了两句,就相互告辞各自去做正事。都是朝中高官,就算心中不合,面上也要做出和气相处的模样来。

“李资深倒还真是忙,这时候了还赶着请对。”

“如今接连大案,御史中丞自然免不了劳心劳力。”

“接连大案四个字说的好。”蒲宗孟呵呵一笑,在学士院的后门前停步,“还望御史台不要食髓知味啊。”

辞别蒲宗孟,韩冈独自往群牧司衙门走去。回想李定脚步匆匆的样子,多半是如今落在御史台手中的几桩大案又有什么新进展了。

两府之中,下一个又会是谁倒霉?

韩冈扳扳手指,突然发现这个人选似乎并不存在,除掉已经被牵连的,驻守边地的,剩下的两府宰辅都跟王珪走得近——吕公著、吕惠卿各自麻烦缠身,郭逵在河北防备辽人,元绛、薛向,眼下都是偏向王珪。

迎合圣意的王珪和他的党羽不用说,就如今风传很有可能在近期入东府的蔡确,他明面上与王珪来往不多,却也实实在在的帝党,与王珪一条阵线——不过话说回来,一切听命于天子的臣子,似乎也不能叫做党。

因为陈世儒一案,吕公著成了摆设,枢密使依然做着,但他在军事上的发言权还不如做副使的薛向。也许这一战过后,他就要退位让贤了。

吕惠卿那里也出问题了,太学受贿案,把他的女婿余中一并牵扯进去。而且被牵扯进去的学正、直讲、教授等学官越来越多,眼见着就要变成大案的样子——不,应该说已经变成大案了。

如今王安石以三经新义为核心的理论,是天子钦定的标准,太学和国子监中的学官是发扬新学的中坚,他们如今一个个被押进台狱,在所谓贪渎之案的包装下,却有浓浓的政治意味。

说到政治对刑案的影响,韩冈倒是想起苏轼还依然被关在台狱中。御史台这些日子以来,都在兴奋的翻着他与人来往的信函,其中针对新法,或攻击或隐射的言论一条条都罗列出来,呈与天子。司马光、范镇、张方平、钱藻、陈襄、刘攽、李常、孙觉等旧党的中坚和成员,像地瓜串一般连藤带蔓的被牵连了进去。行事不谨,口舌招尤,连亲朋好友一起祸害了,这是苏轼的本事。

探究案件本身其实没有任何意义,贪渎也好、讪谤也好,牵涉朝堂高层的任何一桩案子都跟政治牵扯不清。当年吕夷简穷究苏舜钦擅卖故纸饮宴一案,难道是为了朝廷的纲纪着想?

韩冈感觉现在朝堂上的风向,似乎就是要大清洗的样子,新党、旧党可能都要因为两桩案子元气大伤,以天子的圣意为依归的帝党,却正是春风得意的时候。

可是他们还能得意的多久?

带着一如既往的温文微笑,走进群牧司衙门的大门,向纷纷上来行礼的属僚一一回应,韩冈心中是对赵顼及王珪一党的冷嘲:也就在这几天了。

韩冈对于这一仗胜率的估算,从一开始时的七成以上,到开战前已经变成了六成。等到种谔被强行召回后、耶律乙辛驻兵鸳鸯泺,就连一半都难以维持。随着高遵裕和苗授在灵州城下日久,胜率也在不断降低,现在韩冈再来评估,就只剩三分之一。

衙中如今已经没有什么的急务要处置了——李稷那边不再拿战马找借口。种谔在瀚海东侧止步,加之李宪清理了骚扰粮道的西贼骑兵,让鄜延路的粮秣转运工作变得稍稍轻松了一点。韩冈也因此变得清闲无比。

在衙中用了两刻钟处置公事,然后用一个下午进行休息,然后到了散值的时间,听着鼓声响,不当值的韩冈就起身回家。

回到家中,照常更衣、吃饭,跟妻妾聊了几句闲话,顺便还看了看儿女的功课,又去书房中读了一阵书,依时上床睡觉,与往日没有区别。

等到半夜,外院的司阍叫着内院的门,然后将韩冈从睡梦中唤醒,使女的声音都在颤抖:“宫里面的童供奉来了,说是奉旨而来!”

