\section{第23章 弭患销祸知何补(15)}

到了午后,苏颂照例来到编修局。

就算现在接下了执掌钦天监的差事来重新修订历法,苏颂还是会到太常寺这边的本草纲目编修局中来。钦天监中人浮于事,乌烟瘴气,倒是跟韩冈一起讨论,还算是轻松一点。

但今天看到韩冈,苏颂却惊讶道:“出了这么大的事,还以为玉昆会被一并召去崇政殿问对……”

这话换作是别人来说,可就是很明显的讽刺,不过韩冈熟知苏颂为人,倒也不会误会,且也并不在意,反而笑道:“过去或许可以,如今怎么可能?!”

私下里向韩冈征询专家的意见,和公开让韩冈参与到天子与宰执们的议事,是截然不同的两个概念。如果说在韩冈官位并不算高的时候,还不至于太过在意这等细节问题,那么随着韩冈地位日高,尤其是眼下即将成为太子师,即便是小事,也必须注意起来。赵顼眼下肯定是不想给人以韩冈能够干预军国重事的误解。

“……说得也是。”苏颂点点头,这样的道理很容易想得明白,“但天子终究还是少不了要来征询玉昆你的意见。”

“召不召见其实都一样。”韩冈说道:“反正两府之中,应当不会有人糊涂到要在这个时候打辽国的主意。”

“怎么,玉昆你是反对攻打辽国?”苏颂笑问道。

“辽人早有准备,这个便宜可不好占。”韩冈可不信苏颂想不到,“为什么耶律乙辛会选在这个时候弑君?他自己选择的时机,必然是对他最为有利——至少在耶律乙辛,和他麾下的一众逆贼眼中都是如此。眼下是仲冬时节,北方积雪深重,而幽燕只会更甚。河北河东都无法出兵,辎重也跟不上去——十万人以上的大战,雪橇车的运力只能是凑数——想要用兵北境,至少要过上三个月,等仲春雪化之后方可。而对于辽人来说,冬天却是最好的时节。”

韩冈音调又低沉下来:“这是对外而言,对内,耶律乙辛选择这个时间也不可谓不妙。之前他不费吹灰之力,便夺下半个西夏,还灭除了西境阻卜一部,他在其国中声威当是一时无两。一年的时间,新土已固,前些天从河东传回来的消息,其麾下斡鲁朵的人马已陆续抵达黑山,其中精骑上万,工匠亦以千计。他对辽国朝堂的控制想必也更加严密,这不正是他谋逆的最好时机?”

“其实也有可能是辽国幼主当真因病而亡吧?才不过五六岁,这个年纪病夭的不在少数。牛痘也防不了所有的病。”

“话是没错,但料敌从宽,凡事还是往坏里去想。”韩冈呵呵笑了起来:“过去上阵那么多次,不论是遇到什么意外,只要往坏处想准没错。”

苏颂没有跟着笑,神色变得更加严肃了一点。韩冈的话像是在说笑,但只要多想想他过往的经历,这条经验肯定不知是付出了多少代价、受过多少挫折、遭逢多少逆境后才换来的。

“看来这一回天子当是不能如愿了。”苏颂长声叹道。

“这不是明摆着的事?”韩冈摇了摇头,“高粱河之败。虽说是败在用兵仓促,攻下太原后,不作休整便直取幽州,但实质上,便是败在小觑了辽人。万乘之国,岂是可以轻忽视之?以楚国之衰,灭楚亦要六十万秦军。”

任谁看到今天朝会上赵顼的神色,都能知道他打得什么主意,但现实不以人心而转移。在平夏之役后,大宋朝廷并没有为攻打辽国做好准备,精兵强将依然放在河东路,及新设的甘凉路、宁夏二路上。在战略上采取的是防守为主,力争早一步消化夺来的土地。想短时间内从守势转为攻势,以眼下东西两府的执政能力,只能是幻想而已。

“开战是不行了。不过如果能学着辽人故伎,在边界上大张旗鼓,并遣雄辩之士往辽国一行,趁机夺回一部分割让出去的土地,或是逼其削减岁币,那也是一桩美事。”

“虚言恫吓并无意义,辽人的虚实,大宋这边看得很清楚,但大宋的虚实,辽人也一样能看得出来。过去受辽人之欺,那是形势所迫,畏辽之心在国中又根深蒂固。可在辽国就不是这样了,若是朝廷学辽人故伎,恐怕叫嚣着起兵越界厮杀的人能逼着耶律乙辛立刻南侵。”

苏颂听了韩冈的一番话,静默片刻后,忽而一笑:“看来是我多虑了。”

“子容兄有顾虑也是应当的。”韩冈满不在意的笑道。

苏颂问这么多,其实是想确定韩冈的立场——正如苏颂一开始时所说,当世知兵的朝臣也就那么几位,在军事上天子肯定是要征询韩冈的意见,若韩冈全力支持对辽动武,以他说话的份量,不是没有可能让皇帝一意孤行。幸好韩冈的回答却是没有任何可以指摘的地方,倒让苏颂觉得自己的确是想得太多了。

韩冈也是明白这一点,才会不厌其烦的将自己的心意向苏颂详加解释。苏颂的为人并不好战,若是朝廷打算对辽动武,他肯定是会坚决反对,所以有些话,说明白了比较好。

将一些事说清楚了,韩冈和苏颂又投入到编纂药典的工作中。只是没过片刻,一名内侍来到编修局的小院中,说是天子有召,命韩冈上殿觐见。

交换了一个果不其然的眼神,韩冈辞过苏颂,便跟随内侍入宫。只是赵顼接见韩冈的地方,不是在崇政殿,而是武英殿中。

在武英殿内,并没有两府重臣的身影。事先已经猜测到的局面,韩冈当然不会觉得意外。

赵顼背着手站在一幅巨大的沙盘后,低头俯视沙盘的一张脸上看不出喜怒。不过他的这个表情,已经说明了之前赵顼在宰执们那里得到的答案。恐怕没有一人支持对辽作战——即便是王珪、蔡确那般听话的臣子,也不会跟着赵顼发疯——都没有蠢到家。

在韩冈看来,除非耶律乙辛突然暴毙,否则几年内,大宋不会有任何机会,所能做的只有观望和等待。不过在观望和等待之间,还是有许多事可以做的。比如加大对科技的投入,比如修好贯通河北的轨道。

贯通河北的轨道,是宋辽交战时,大宋立于不败之地的关键。这几年下来,赵顼比起韩冈更加关注轨道上的技术进步,在天子的督促下,能工巧匠的智慧如同泉水般迸发出来。

运用在京城的汴河水运码头上的铁轨,以及轨道车辆上的铁制四轮底盘,这一系列的发明和运用,完美的延续着韩冈在离任前定下的技术发展路线上。

眼下铁轨已经在汴河边的码头上普及,新型的铁轮比起木质的轮子也的确更适合在轨道上奔驰,钢制的轮轴也出现在军器监的铁场中。

所以韩冈一句句的问着赵顼,“臣敢问陛下,钱粮是否备足,军械是否整齐,军心是否可用,听说与辽国交战,民心是否稳定,朝堂上是否为此做好了战火连绵十余载的准备?”

赵顼的脸色一点点的阴沉了下去,韩冈的质问比起宰执们的反对更让他觉得羞恼:“难道仅仅收复燕云就要用上十几年?!”

赵顼反问的声音都有些变了,但韩冈毫无惧色:“辽国乃万乘大国,百万精兵。即便不是灭国之战,仅仅是为了燕云,也得两三次数十万人马以上的大决战,十万级的会战七八次,几千几万的战斗那更是得数十上百。没有十余年的时间累积胜果,如何能成功?”

“这是怎么算的?!”赵顼沉着脸,阴声问道。

韩冈侃侃而谈:“只要将过去平灭西夏的情况代换过来就行了。为了灭亡西夏,只从熙宁三年、四年的第一次横山之役开始计算:平夏之役用兵三十余万,民夫百万,这是规模最大的决战。其次的会战,有前后两次横山之役;断西贼右臂的河湟之役;熙宁十年的复夺丰州和葭芦川两战由于是相互配合,加在一起也能一并算进来。动用十万人马的会战就是四次。再往下的战斗,大大小小每年都没有断过。西夏穷兵黩武,但兵力也不及辽人五分之一。户口大约只有十分之一——即便只算燕云,丁口最多也只能达到一半的样子。以此来计算,重夺燕云便要做好差不多数量两倍以上会战次数的准备。”

赵顼皱着眉,不说话。他没想到韩冈是这么计算出来的。只是赵顼也不是对军事一窍不通,韩冈的话虽然偏驳,但南京析津府和西京大同府,想要收复燕云之地的两个核心城市,两场大规模的决战的确不能少。而在燕山诸山口、榆关【今山海关附近】,以及奉圣州【今张囘家囘口】、胜州,乃至兴、灵也少不了几场大战。这么一算,韩冈的计算倒是一点不差。

而韩冈在继续阐述他的观点:“这些战役,决战绝不能败,一败便无可挽回,会战败上一次,就要付出几倍的努力,而更小规模的战斗,也必须胜多败少,以求不断消耗辽人的军力。”

对于攻辽的计划,韩冈一向不支持从河北出兵。以河北平原的地形,对辽人的骑兵实在是太有利了。倒是以河东的地形,能充分发挥宋军步卒的作战优势。而且禁军中最为精锐的西军,更是能够充分发挥他们在峰谷之间追杀西夏人的实力。可如果将他们安排在河北平原上,在战术上恐怕会很不适应。

但耶律乙辛将他的斡鲁朵放在黑山下的河套平原,不仅仅是贪图那里的土地肥沃,必然也有地理战略上的考量。当黑山下有了一支多达两万的精锐骑兵坐镇,不论是西北侧的阻卜人,还是东南方向上的西京道,都在其兵锋攻击范围之内。

宋军从河东出兵,想要打下大同,收复云中之地,比起几年前,难度要高了许多。肯定是一场大规模的决战,用来决定云中诸郡的归属。

“难道只有这样才能赢?!”赵顼不忿的怒叫着,“耶律乙辛接连弑君,难道辽人就无忠义之心?!”

“陛下!”韩冈提声道:“昔之善战者,先为不可胜,以待敌之可胜。不可胜在己,可胜在敌。若大宋攻辽,谁能保证辽人不会有同仇敌忾之心?与其期待耶律乙辛众叛亲离,还不如做好辽国上下一心的准备。若是辽人当真并力拮抗,也一样能胜。若是辽人心不齐,那便是锦上添花的美事。”

赵顼默然良久,垂着头看着河北的沙盘,最后心中的坚持化作长叹了一声,“韩卿是坚决反对对辽用兵”

“陛下明鉴。魏武平冀州,袁熙袁尚北逃辽东。魏武并没有派兵去攻打公孙氏,反而驻兵不进。可二袁的首级,却自动送到。”

赵顼的声音和缓了一点:“魏武灭袁,跟如今有何处相似?”

“庙堂之谋,运用之妙,存乎一心。缓而胜急,本质上是一样的。耶律乙辛在弑君之前,已秉政二囘十囘年,如今年过五旬,待其病死,甚至只需病重,无力控制朝政,辽国必然生乱。快则数载,多不过一二囘十囘年而已。陛下如今也不过三旬,至其时春秋正盛,国势亦当倍于当下,何愁不能一举灭辽?”
