\section{第二章 摇红烛影忆平生(下)}

自此以后,宋夏之间的边境上,就没有一年听不到金鼓号角之声。关西的百姓,不是被征发起来充当民伕,就是直接从军披挂上阵。韩冈的父亲和大哥都曾充过民伕,运粮去前线,又或是去边境筑城。而韩冈的二哥,则在年满十六岁后,投了军中。他从军后屡上战阵,数年间多次受伤,因功混上了一个名为左十将的没品级的小军官当当。

一家养了三个儿子,一个务农,一个从军,一个读书,各自都有出息,韩家在村中也算是让人羡慕的家庭。可到了今年,一切却变了样。

今年四月初,西夏军又一次南侵,十余万军全力攻打秦州。韩冈二哥再度披挂上阵,而韩冈在家务农的大哥也被临时征召。可两人一去,就再也没有回来。韩冈在外跟随张载学习了两年,端午刚过,便被一封十万火急的家书唤回。

尚记得当时韩冈从外地求学的地方日夜兼程赶回家中奔丧,在半路上就因淋了雨受风发病。强撑着病体到了家中,便一病不起。那时还是五月中天气正热的时节。如今贺方身上已经盖上两床厚被,还感觉着有些浑身发寒,不仅因为身体虚弱,也因为天气的确转凉了。推算时日,恐怕已经是入秋的八九月。

因为一场肺病而倒在床上三四个月,贺方用切身体会感受到千年之后的社会究竟有哪些优势。在贺方如今所处的时代,人命轻如鸿毛,无论是战争还是疾病,就能让一个健壮的年轻人轻而易举地丢掉性命,绝不是能让人一笑而过的。

而一场病灾也让韩家从一个小康之家变成了破落户。家里的两进宅院应是卖掉了——否则贺方现在所在的房间,就不会跟韩冈留下的记忆对不上号——上百亩的田地也卖掉了,仅剩下的三亩菜园还被人日夜惦记着,贺方听到了田地买主李癞子和父母的对话,却不知最后的结果如何,韩家仅剩的三亩多地是不是也被卖了出去。

想及此事,贺方心中便是一团火焰在熊熊燃烧。因为自己的缘故而让家中被人趁火打劫,不论是贺方还是韩冈,都因此郁愤于胸。

“天道好还,报应不爽。落井下石的事情可以做,但日后被人捅刀子,也不要喊冤……”这是贺方的一位前辈在酒后对他说过的话,那是他们刚刚出席过另一位同事追悼会后的感慨。躺在殡仪馆透明棺材里的同事,还有他一张无论怎么化妆也修补不过来的、被砍得支离破碎的脸,让贺方受到了极大的刺激。那天之后,贺方便放弃了那份来钱快的工作,而找了份正正经经的事去做。之后的为人处世上,他总是要多收着几分,凡事从来不会做绝。

前辈的那番话,贺方印象很深,用在现下也正合适,‘天道好还,既然你敢趁火打劫,也别怪我给你来个报应了。’贺方是个恩怨分明且记仇的脾性,他自心中立誓,这报应当由自己来出手。

不过千年之前并非全然让人失望,就在床榻的另一侧,一名身材纤巧的少女正半趴在床边打着盹。从贺方的这个角度瞧过去,看不到少女的相貌,只能看见她被灯火染上一层柔光的如云秀发,听见柔柔细细的弄得贺方耳朵有些发痒的呼吸声。从少女的单薄身形来看,最多十一二岁的样子,而实际上,她也正是刚满十二岁。贺方第一次醒来,一声‘三哥哥’就是出自于少女的口中。

尽管她称韩冈为‘三哥哥’,但少女并不是韩家的女儿。根据韩冈的记忆,少女名叫云娘,是韩家的养娘,乃蕃人出身。四年前西夏国主嵬名谅祚亲领大军南下攻打秦州,延边亲宋的熟蕃被灭了许多,又被赶跑了许多。当时秦州道上兵荒马乱,年纪尚幼的云娘便被人贩子趁乱拐出来,卖给了韩家,也自随了韩姓。

所谓养娘,贺方从字面上去理解是养女的意思,不过这是宋代对婢女的另一种说法。至于韩云娘唤韩冈作三哥哥,也不出奇。在古代,家养的婢女,只要服侍的主家没有官身,把老爷太太唤作爹娘,把少爷叫哥哥,是很常见的事。而贺方至少看过金瓶梅,也并不是很惊讶这些。

韩冈在病榻上半昏半醒的这些日子,主要都是由韩云娘照顾着。才十二岁的少女将病人服侍得妥妥贴贴,连后世大型医院都很难完全避免的褥疮也没生一处。韩冈习以为常,但夺舍转生的贺方却知道这有多难得。心怀感激,贺方勉力抬起手,打算理理韩云娘铺散在被褥上的秀发。很轻微的动作,却惹得少女从睡梦中惊醒。

“三哥哥?……”

少女犹在半睡半醒间,眼睛迷迷糊糊,声音也是软绵绵的,带着些稚气的口齿不清。只是她一抬头,贺方便陡然觉得眼前一亮。在韩冈留下来的记忆中,他两年多前离家游学时,韩云娘只是一个还没长开的黄毛丫头。但如今在贺方眼里,十二岁的少女却着实让他惊艳。

可能是在床边趴了太久的缘故,象征少女身份的双丫髻已散了半边,半幅秀发飞瀑般坠了下来,晕黄的灯火映在发丝上,一如最上品的绸缎般闪亮。俏靥被秀发半掩,给稚气未脱的瓜子小脸平添了几分妩媚。

红润的小嘴微张,小巧的鼻梁挺直,双眉弯弯如月,眼廓则略略有些下凹。可能是带了一点点西域血统——回鹘商队在秦州常来常往,蕃人又不如汉人那般讲究贞洁,所以在秦州有西域血统的蕃人却也并不算少——五官深刻明晰的相貌并不符合此时的审美观念,但韩云娘若是走在千年后的大街上,不知会惹来多少憧憬的目光。

从睡梦中惊醒,韩云娘困顿的揉着眼睛。等她放下手,正正与贺方满是惊艳赞叹的视线对上。

“三哥哥!……”小丫头捂着小嘴瞪大眼睛的吃惊样子惹人怜爱。前日她看见她的三哥哥在昏睡了许久之后终于有清醒的迹象,这几天她得空便趴在床边,与韩母交替看护着,盼着着韩冈再次醒来。

这半个月来,每位从秦州城里重金请来问诊的大夫,在诊断的最后都摇头叹气说她的三哥哥没救了——好几个大夫都说过从没有人能重病卧床四个月,最后昏迷不醒半月有余,还能再救回来的——但韩云娘小小的心里仍抱着一丝希望不肯放弃,每日都尽心尽力的为韩冈换衣擦洗,得空便向天上的四方神灵祝祷。

小丫头的心思很单纯,她既是韩家的养娘,当然要尽心尽力。何况在韩家,待她最好的便也是韩冈。天可怜见,多少天的辛苦终于没有白费,想到这,韩云娘鼻子一阵发酸,晶莹的泪珠一滴滴的滑下脸颊。

扶在床边,韩云娘抽抽噎噎的哭个不停,几个月来的疲累和不安都随着泪水涌了出来,她紧紧攥着被角,“三哥哥,你可醒过来了……”

泪滴闪着灯火,仿佛一颗颗水晶珠子从小丫头的双颊落下,贺方有些心疼伸出手,想擦去她脸上的泪水。小丫头被贺方的动作惊了一下,却没避让,任由贺方有些笨拙的帮她拭去泪水。这时她也不哭了,不知从哪里掏出一块汗巾,擦擦眼泪,小丫头便要站起,“对了,我去唤爹爹娘娘起来。”

“让爹娘睡着罢,他们也累了。”贺方探手过去攥住韩云娘的手,把她拉近了。感受着掌心处的腻滑如脂,纤细的手腕似乎轻轻用力就会折断。看着她清减了许多的小脸,贺方柔声说着:“这些日子辛苦你了。看看,瘦了这么多……”

小手被紧紧攥住,彼此呼吸相闻,韩云娘只觉得脸热得发烫,如果换作是白天,没有摇曳的火光映照,她脸上的羞涩红晕一下就会被发现。她不知道三哥哥为何不像过去那般谨严守礼,让自己手脚都不知放在哪里是好。

扭捏了一阵,韩云娘突然掩着小嘴轻呼了一声,“呀,忘了把灯熄了,费了这么多油!”说着就又撑着贺方的身体想站身起来。

“不用急。让灯点着就是了,烧完了自己会灭。”小丫头的花样,老于世故的贺方哪能看不出。他促狭的将手握紧,不让她顺势抽走。

韩云娘轻轻地又扯了几下,见贺方不肯松手,也就不动弹了,静静的坐在床边,秀丽纤巧宛如夜昙绽放。只是被贺方目光灼灼的盯着,小丫头头越垂越低。没被握住的右手在下面轻捻着腰间丝带,盯着什么纹路都没有的被面,像是想看出一朵花出来。

厢房中的两人一坐一卧,视线虽不相交,双手却是紧紧相连。灯花时不时的噼啪一声作响,却更增添了一份静谧。灯下看美人,使人不觉沉醉。握着少女纤细的小手,看着她娇羞动人的模样,贺方只觉得心中平安喜乐。虽然已经无房无田,但有个小萝莉作伴,他突然间觉得如果能来到宋代,倒也不错……

