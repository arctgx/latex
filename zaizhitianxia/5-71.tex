\section{第十章 却惭横刀问戎昭(三)}

【实在很对不起各位书友,又断更一天。两个多月来,一直都是断断续续,也不能按时更新,说补更又不能做到,人品丢光了。不过为期八十天的学习终于结束了,昨天吃过散伙饭,今天又开过结业报告会,算是功德圆满。从下个月开始,本书的更新基本上就能恢复正常了。】

太原新城中的主要街巷,极少有横平竖直、贯通四方的十字大街,就是州衙所在,也一样是断头的丁字路。这样的城市格局,打巷战不错,但防守城壁时,却是给守军的往来调度平添了许多麻烦。

另外韩冈估计赵光义还有另外一重算计,规划失序的城池,对于城市的发展也是一个巨大的阻碍。也许太宗皇帝认为这样建起的城市,已经不可能再发展到旧日的规模。

赵光义的行为,如果让韩冈来评价,评语只有极其愚蠢四个字——绝对是个极其愚蠢的做法!

当晋地回归中国,太原便是开封的北方门户。废弃有山河之固、城壕之险的旧日坚城,迁到汾河东北重建新城,一旦北敌南侵,想要攻下来可就容易了许多。

韩冈就算再不了解历史,几十年后,天下半壁沦丧他还是记得的。从河东、陕西和中原的地理战略来看,太原若不失守,陕西可以高枕无忧。河东、陕西皆在,金人即便占据了河北,最多也只能南下劫掠一番,不可能顺利的占领中原。

不过话说回来,摊上几个昏君,有多少天险也是无用。对于家国存亡,太原城建在哪里并不是至关重要的一件事。

日后如果不是那位擅长书画的道君皇帝登基,大宋的江山想败掉并不是那么容易的一件事。

……但这些事放在现在来考虑还是太早了一点。

韩冈忽然自嘲的笑了起来,让厅中的门客和胥吏茫然不知是何意。

韩冈摇摇头,翻看着桌案上垒得老高的公文。全都是从治下各县、监送来的,政务、军事、刑名、财计,亟待他这位知府来处置的大小事务林林总总百余件,这还仅仅是一天的份量——尽管这其中有一部分是因为夏税入库对帐以及处在战争中的缘故,但就是太平时日,至少也有一半的数量。

看到这堆满桌面的公文,就能看得出太宗皇帝当初的决定有多愚蠢。

即便赵光义对屡屡成为叛臣据点的太原深恶痛绝,可地理位置上的优越性,让太原的发展不因人心而转移。

从开国时被摧残得只剩数万丁口的破落之地,经过了百多年的发展,今日的太原已经是一个拥有十五万户,人口总数大约在七八十万上下的大州府。

几近百万人口的州府,除东京城外的其余四京,也不过如此。所以就在去年,天子赵顼不得不下诏,将并州升格为次府,复名太原府。而韩冈,算是大宋开国以来的第三任太原知府——第一任是潘美,杨家将中潘仁美的原型,他之后就毁太原改并州了;第二任是韩冈的前任孙永,第三任便是韩冈。

韩冈对此倒没多少感到荣幸的心情,就像刚刚度过假期的学生,对现在的繁忙甚至有几分不习惯,怀念起在群牧司的轻松日子来。

也是自家幕僚和门客的不能提供太多帮助的问题。韩冈一边飞快的批阅着公文,一边苦笑着。自家一向是实事做得多,下面的人立功得官的机会便也多,其得到一份告身的几率甚至比得上宰执官门下的幕宾。有点能力的在自己一任过后,全都得了推荐,最差的也能做个学官,领取朝廷俸禄。剩下来的就逊色许多,办事都谈不上牢靠。

还有一批新人,全都是气学弟子,一时间根本派不上用场,只能先安排他们看一看城防、仓库,然后让他们在衙门里体验一下生活,得慢慢培养起来。而且得时刻帮他们盯着,以防被胥吏所欺,否则自家也要被牵累。

也是关学底蕴太过浅薄,横渠先生的大名名传天下士林也不过数载,之前仅仅是在关中享有盛誉而已。而关中士人黄榜留名的寥寥无几,为了能让他们有晋身机会,韩冈也只能将身边的位置给他们腾出来。等再过些年,传习气学的士人不再以关中人氏为主,那时候,韩冈也就能省心了。

不过事有两面,自己培养出来的幕僚,总能更让人放心一点,等他们上手之后也很容易配合起来。

又是忙到三更天,韩冈终于将所有的公文都批阅完毕。揉着酸涩发胀的双眼,韩冈听着外面更鼓声,上床睡觉。枕在藤编的枕头上,韩冈还在回想着今天批阅的那些公文,会不会有什么问题。

直到四更的更鼓响,韩冈猛然一惊,最多也只有两个时辰能睡了。再这样下去,就得学司马光了,弄根圆木做枕头,什么时候睡滑下来,惊醒了,那就继续去做事。

叹了一口气,韩冈强行清空脑中的念头,数着羊让自己睡了过去。

第二天清晨,过了五更,韩冈就在鸡鸣声中醒了过来。尽管只睡了两个时辰多一点,但冲了冷水澡后,又是精神奕奕,这就是年轻的好处。只是当韩冈走进公厅,无奈的发现,在他的桌案上,各方公文又垒起了城墙。

今天并不是开公堂的日子。在桌后坐下来,韩冈有点发呆瞅着堆起来的文书。等到用一两个月熟悉了太原府内外,就能知道许多事该往哪里推了,至少现在,他还需要通过这些公文来了解太原府,还不能交给下面的官员。

过了卯正,李宪前来报道,韩冈得以趁机从公文地狱中抽身出来。两人寒暄了一番,又待韩冈问候了李宪的家人,终于说起了正事。

“弥陀洞及葭芦川四寨,现有河东兵马四万,粮草转运殊为不易,而且在如今的局势下也没有必要。是否方便先调三万回来?在弥陀洞留下六千骑兵,四千步卒,已经足以在必要时援助夏州,并守住弥陀洞!”

韩冈与李宪有商有量,这是他给面子,李宪自不敢拿大,点头道:“只要龙图下令,末将便依令而行。”

李宪如此识趣,韩冈脸上的微笑又多了一分:“其实都知身上的差遣是经制河东兵马,并不一定要留在弥陀洞。如今辽人在大同蠢蠢欲动,若是都知能坐镇代州,韩冈也就放心北方了。”

韩冈并不认为北方能打起来,但将李宪丢到代州,自家便可趁机插手黄河西面的军务。否则隔了李宪一层,总归有许多不方便的地方。

“龙图有名,李宪岂敢不从。”李宪早就想从西北战场脱身了,韩冈等于是给了他一把下台的梯子,哪里会不愿意,只是有些事他需要交代一下,“不过,鄜延路体量军事兼计议边事的徐禧徐直阁,他前日曾致信末将,想要调用一万兵马……”

“做什么用?”韩冈心中警惕起来,眼神一下变得凌厉。

“龙图没听说吗,”李宪总觉得种谔应该向韩冈通气才是,至少有韩冈在,说不定还能压制住徐禧的盲动,“徐直阁想要守住盐州,将整个银夏之地全都占据下来。”

韩冈的心猛地一跳,右掌一下攥紧了扶手。

得速遣人去种谔那里询问明白!韩冈有几分恼火的想着。如果徐禧真的犯蠢,怎么也得设法阻止。

韩冈心中忧急,但他说出口的话,只是轻描淡写的一句:“不意徐禧竟然如此贪功。”

李宪审视的目光扫着韩冈,却见韩冈神色依然如常,喜怒不形于色,倒还真是一点破绽不露。“许是看着瀚海难渡,所以有恃无恐。仅仅是两路兵败,其他几路都是顺利的撤了回来,相对而言,西夏的损失要大得多,这样的情况下,只要继续消耗下去,西夏必定支撑不住。”

韩冈心情更坏一分。自己倒是多次说过瀚海难渡,对宋夏两国都是一样,只要占据银夏,就能逐渐困死西夏。说不定这番话倒是不幸成为了徐禧佐证。要是徐禧拿着这番话来为自己的行动做依据,那还真是让人心头不痛快到极点。

而且徐禧跟吕惠卿的关系又亲近,通过吕惠卿,得到天子的认同,应当不是难事。

如今王珪因为灵州兵败,圣眷大衰,吕惠卿则是正得意的时候,也许再过一阵就能宣麻拜相。有他的支持,徐禧若当真要力保盐州,多半当真能给他如愿以偿。

“不知以都知看来,徐德占能有几分成算?”韩冈问道,他想知道李宪的想法。

“这就是跟当初攻打灵州的情况一样,不是赢不了,而是不值得去冒那个风险。”李宪坦诚的说道,“现在大宋也冒不起风险了。”

李宪和韩冈的观点可以说很相似。经过灵州之败后,绝大多数人都认同了郭逵和韩冈的预见,徐禧那是例外。如果不是李宪说谎的话,那就是徐禧想立功想疯了。

不过在灵州兵败、万马齐喑的情况下,如果当真给他守住了盐州,将西贼挡在瀚海中,一举成名是不消说的。但在这样的情况下要冒的风险,绝不比之前攻打灵州小多少。

虽然不在职权范围之内,但韩冈既然听说了,便是责无旁贷,肯定是要阻止。不过……韩冈十分隐晦的瞥了李宪一眼,关键先要确定徐禧是不是当真犯蠢了。自家不是御史,可以风闻奏事,听着风就是雨,那就要让人笑话自己太不稳重了。

其实韩冈有几分疑惑,按理说种谔那个脾气,就是徐禧手上有密诏,他也不会善罢甘休的。不过也说不准,种谔毕竟是武将,要跟名声不小的徐禧相争,恐怕心里还是有些虚。

