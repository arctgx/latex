\section{第七章 烟霞随步正登览(三)}

“家叔祖吩咐景贤,如果侍制觉得推举宰辅一事有悖祖宗之法,那就什么话都不用说了。”

“难道郑公也觉得变法好?”

“家叔祖说了,潞国公曾经有一句话说得很好。”

范纯仁思忖了一下:“……为与士大夫治天下,非与百姓治天下?”

“正是!士大夫与天子共治天下,何事不可预?”

这句话文彦博敢说,富弼当然也敢说。要不是嫌拾人牙慧,很多人并不介意多说个十遍八遍。

“这样啊。”范纯仁点了点头,“那如果纯仁决定参与推举,那郑公又有如何吩咐?”

富景贤顿了一顿,看了范纯仁一眼,沉声道:“请侍制推举韩冈!”

富弼对韩冈的欣赏,范纯仁很早就知道了。

主要还是当初韩冈在白马县,救治流民百万。富弼得知后便在家中说王安石为国抡才尽找些奸险之辈,为自家招婿倒是多长了几只眼睛。富弼次子富绍京曾经写信给范纯仁,将这件事当笑话说了一遍。

不过范纯仁对此评价也是深以为然。虽然说韩冈的卖力使得新党安然度过了危机,但百万流民的安危,远重于朝堂政争,若是流民救治不当,整个京畿之地都要陷入大乱,孰重孰轻,不可能不明白。

只是富弼如此明确的表态要支持韩冈,但韩冈本人会怎么做?

这么多年来,他对新党的帮助无人能否认。尤其是在军事上,没有对外战事上的成功,新党的根基不会这么牢固,而以富国强兵为名的新法,也会失去推行的正当性。这一切,韩冈在其中功不可没,他可能放弃之前的一切,转投到旧党的怀抱?

“不知贤侄如何看新法?”范纯仁问道。

当初王安石推行新法,派去洛阳的新任河南知府叫富弼家缴纳免行钱。钱是小事,但脸面丢大了。但那一位是吕夷简的女婿,与富弼早就结下了梁子。他上门让富弼家交免行钱,几分为公,几分为私,那是不必多说的。但富家对新法的态度,在李中师以权谋私之前,就已经是没有半点好感了。富弼从宰相的位置上退下来,正是因为王安石进入了政事堂。如今多少年过去了,但积怨却不可能那么简单就消除。

富景贤的心情却是一松,范纯仁既然这么问了,也就是代表他已经意动。

“新法有其害,亦有其利,其攫取民利之本意,景贤一向不喜,但在役法上,却是要胜过旧日的差役。”

过去的差役法,由于残民过苛,一直为人所诟病,纵使司马光也曾上表要改革役法。但新旧两党分裂朝堂之后,还能坚持旧日态度的,却就只剩那么几个了。但从实际情况来看,只要不昧着良心,孰优孰劣一目了然。

富景贤继续说道:“而且如今新法推行日久,民情惯熟,若遽然再改,就如当初以新法变旧法,百姓不宜再受如此苦……这也是家叔祖的教诲,不知侍制如何看。”

“贤侄回去后,请上覆郑公,纯仁的想法与郑公一般。”

富景贤深深低头:“景贤明白了。”

……………………

“包绶?”乍听韩冈提起一个陌生的名字,王厚疑惑的眨了眨眼睛,“是包约、包顺的人?”

包约、包顺都是曾经让王韶、韩冈和王厚绞尽脑汁去对付的蕃部大首领的名字,原名自不是如此,只是因为仰慕传说中的包拯包侍制,自归顺后便请求朝廷赐予他们包姓。

“不是。”韩冈摇头,“不过也有些瓜葛就是了。”

“什么瓜葛?”

“他是包孝肃的儿子,这不是瓜葛吗?”韩冈笑了,“……而且也是潞国公家的新女婿。”

“包孝肃都是多少年前的人了,怎么他儿子才被文潞公招了做女婿?”

“是续弦。”

“潞国公把女儿嫁过去当续弦?!”王厚惊问道。

如文彦博这样宰相、枢密全都做过的身份,把女儿嫁出去却不是元配,可谓是有失体统。正常来说,最多也只会是嫁出去的女儿早亡,将小女儿嫁过去做续弦,维持过往的姻亲,也可以保证外孙的安全。

即如欧阳修先以薛奎薛简肃长女为妻,丧妻后又娶了薛奎的幼女。所以同为薛奎女婿的连襟王拱辰就写诗取笑道,‘旧女婿为新女婿,大姨夫做小姨夫’。刘敞也拿他说笑话,说是先弄大蛇,在弄小蛇,当然,这里的蛇是‘虚以委蛇’中的那个音——姨。

“不过包绶的年纪比你我都小,包孝肃过世时才五岁。听说是长嫂崔氏抚养成人。所以当初文潞公还特地上表,要为崔氏请封。”

王厚拿着包绶的名帖翻来覆去的看了几遍,“字不错……只是递了名帖来?”

“已经足够了。”韩冈道,“我说过的……潞国公从不服老。”

王厚点了点头,但又道:“就文潞公一位?西京的其他元老呢?”

“还有郑国公。”

韩冈从厚厚的一摞名帖中中找出一封来,王厚看了一眼上面的姓名,“富景贤?”

“郑国公的侄孙。不过听说因为郑公三子无子嗣,郑公准备为其将景贤过继来,跟亲孙子没区别。”

听到韩冈如此说,王厚心中惊异不已。韩冈与富弼议亲虽只是刚起个头,但能知道这些富家内部的隐秘事,韩冈私下里与富家的联系可见一斑。而且从这些事来看,富弼对韩冈的欣赏也是显而易见的。

“愚兄听说富郑公对玉昆你一向都很看重,现在看来是真的了。玉昆你到底是哪里得了郑国公如此青睐?”

韩冈哈哈笑道:“因为郑公与我都不擅诗赋吧?”

王厚为之莞尔。

昔年科举以诗赋取士,富弼若不是转从制科出身,一辈子都做不到宰相。之后富弼被招试馆职,仁宗皇帝还特地将原本应该考核诗赋水平的考试,改成了策论。

但若说富弼是因为韩冈也不擅长诗赋而对他另眼相看,那绝对是一个笑话,不如说两人的经历极为相似。

中制科入仕十三年而为枢密副使,是富弼。而特旨得官十二年后任西府副贰,则是韩冈。

“恐怕还有性格。郑国公敢对天子说伊尹之事臣能为之,而玉昆你,就干脆是当殿杀宰相了。”

韩冈摇头不语。他与富弼的性格还有些区别。

仁宗时,群盗犯高邮,知高邮军晁仲约无力御敌,便要求城中富民出金帛,具牛酒,出城相款待,请盗贼们高抬贵手,去他处抢劫去。之后此事曝光,对这位无能的晁仲约,富弼要杀之以为后人之戒,而范仲淹则表示反对。事后还对富弼说,‘轻导人主以诛戮臣下,他日手滑,虽吾辈亦未敢自保。’富弼则始终不以为然。

从韩冈的角度来讲,以公事论,晁仲约当然该死,但韩冈并不是朝廷的代表,也没有坐在御榻上,没有必要为王法的威信担心。换做他当年处在范仲淹的位置上,也只会将晁仲约远远的打发出去。就像这一次对待叛逆,能够免除一死的,就尽量保住他们的性命。

“这一位也是来递门贴的?”王厚又多看了几眼名帖,然后摇头,“字不如包绶。”

“不,昨天他已经来过了。他这一回入京,是为了迎接范文正公的儿子。”

“……是范纯仁?”

“正是范尧夫。”

这个时代,以尧舜为名为字的士人多如牛毛。这边有个范尧夫,而洛阳过去还有个邵尧夫。

这一位算是旧党之中,没有什么瑕疵的。司马光对新法的反对最为激烈,所以他才是赤帜。而范纯仁虽非赤帜,但刚正严毅之处,也让新党头疼了很久。

王厚隐隐记得将要入觐的侍制中有这个名字,但时间要差上几日,“他不是来不及了吗?”

“郑国公既然这么说,就可能有把握。”

“说的也是。但这一位范尧夫,玉昆你过去有没有见过他?”

“当然有过。只是谈不来。现在几年过去,说不定会好些……不管怎么说,都是文正公之后,我横渠门下得有一份敬意才合适。”

范仲淹于张载有劝学之德,说起来韩冈与范家也算是有一段渊源。当初范纯仁贬官京西,曾经特地绕路,去见过时任京西都转运使的韩冈一次。那一次会面,不能说是很愉快,两个对自己的道路坚定不移的人,道路又相背离,无论如何都不可能合得来。

“仅仅敬意恐怕不够呢。”王厚道。

“君子和而不同。总是有相和的地方。”

韩冈从来都不是新党的一份子。若说让王安石头疼的次数,韩冈不比任何人稍逊。

新学、新法、新党,这是三位一体。再过几年,世人忘了旧法,那在台上的就都会是新党了。

韩冈与旧党,完全可以求同存异。在旧党元老已经无法翻身,而新人又难以出头,甚至因为刑恕而要翻船的现在,韩冈成了他们的救命稻草。

而且韩冈一旦秉政,他肯定会学新党一样,从科举上着手来提拔人才。能多一个出头的门路,北方人都会趋之若鹜。

