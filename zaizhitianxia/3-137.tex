\section{第36章 望河异论希(三)}

【不好意思,有事迟了一点。不过夜里还有一更。】

老汉见着郑侠没反应,也不气馁,反过来又对行商道:“也就是最近的事,东京城里面有个看城门的官,上书说如今的旱灾蝗灾全是新法不是,要官家废新法,赶了王相公走。其实这事倒也罢了,不论是哪家法度,好坏都要交税。但他千不该万不该,就不该骗天子说废了新法就能下雨,这倒好,小韩知县一见天子,就戳破了他的谎。

这官儿也该死,骗天子还不够,还说小韩知县不是,在白马县害了几万流民。想想,这是多大笑话?!人家流民都要为小韩知县设长生牌位了,竟然还有人睁眼说瞎话,说害了流民。现在听说天子明察秋毫,将他下狱治罪!……这就叫活该!”

卖茶老汉说得口沫横飞,老公人过来扯了扯郑侠,“官人,还是上路吧。”

郑侠纹丝不动,脸上看不出喜怒,只是拿着汤碗的手轻轻抖着,他要听着这老汉的下面怎么说。

“小韩知县自从来了白马县,天天都没歇过脚。为着河北的流民,小韩知县跑瘦多少匹马?为了应付这场大旱,县里打出的多少口井?现在架着风车的几十口深井,全都叫韩令井,从早到晚的提着水出来,以后几十年都不用怕旱灾了。小老儿这卖茶汤的水,就是从几十丈深的韩令井里提出来的!比起原来的井水好了不知多少,过去白矾一个月就要用上一斤,现在一钱都不用了!想想能为小老儿升了多少棺材本啊!”

郑侠面无表情的坐着,心中则是如同八月十八的钱塘江,惊涛骇浪不停地翻涌。

他从卖茶老汉身上能看得出来,白马县的百姓是当真将韩冈顶礼膜拜。

难道说自己真的误会了韩冈?

不!

郑侠在心中立刻否定。

王安石在熙宁之前,还不是负了三十年的重望?王莽在篡汉之前,也不是人人夸赞?韩冈现在的表现,也不过是他岳父当年的翻版,等他日后得志,天下必受其人所苦。

想到这里,郑侠容色一肃。

天下正受新法所苦,韩冈却不加以阻止,反而助纣为虐。他在白马县做得再好,也不过是小恩小惠而已!

再看一眼说得口沫横飞的茶棚老汉,眼中不无怜悯。乡愚识见不足,眼光不及长远,所以才会被奸佞所欺。

歇也歇够了,郑侠就准备会了钞后就动身,忽然就听到一片蹄声,从北面的官道上奔来一队人马。

远远地一见来人,郑侠身子就僵住了,而茶棚老汉伸着脖子张望了一下:“哎呦,是王相公家的二衙内!”

“王相公家的二衙内?”河北行商闻言一惊,随着望了过去:“相公家的衙内怎么来了这里?”

“王二衙内也是好人,给小韩知县打下手,县里面的井水、沟渠都是他督办的。现在县里面的几十个换米点,小韩知县也是天天派人来督察。前两天,也就是王二衙内来的。由他盯着,你说谁敢克扣半点?”

河北行商沉吟着点头:“这么说来,王二衙内也是个好官。”

“王相公也是好心办坏事,给下面的人蒙骗了,听说小韩知县也劝过。想想当初小韩知县来白马,外面不都说是翁婿两个吵架的缘故?”

茶棚老汉和行商这边说着话,王旁就在换米点下了马,主持换米的胥吏迎上去点头哈腰,而排着队的乡民们也同样一起行礼,一片声的问好。

王旁的随从也跟着下马,有几个是负责保卫的,眼睛四处瞟着,一眼发现了停在茶棚外的驿马和马车。属于驿馆的马匹和马车,很容易分辨出来。

官员过境,于礼就要接待。那人忙去了王旁身边说了一句,王旁立刻就走了过来,到了茶棚外问道:“是哪一家官人要北上过河?”

郑侠默不作声的站起身来。

站在太阳底下,茶棚下阴凉处的人和物就有些模糊,王旁眯着眼睛看过来,瞅了好几眼才看清了是郑侠。惊叫道:“郑介夫?!”

郑侠躬身一礼,向过去的老相识很疏冷的说道:“衙内,郑侠这厢有礼了。只是戴罪之身,不便与衙内相见。”

王旁张了张口,正要说话。就听着茶棚下面又蹦起一人,“你就是那个胡说小韩知县害了流民的犯官?!”

茶棚老汉一下跳将起来,拿起蒲葵扇往外挥着:“去、去、去,不收你茶钱了,小老儿这破茶棚待不下郑官人你这尊大佛!”

“不得无礼!”王旁和老公人连忙一起叫道。

茶棚老汉则梗着脖子:“二衙内,你们官场上的事小老儿是不知道,但说小韩知县坏话的,小老儿可侍候不起。也别说小老儿无礼,郑官人过境的消息传出去,看看会有多少人有礼!”

郑侠脸色发白,王旁尴尬得不知如何是好。而河北行商则是唯恐天下不乱的在后面拍着手:“公道自在人心,还是乡野之中有义民!”

……………………

一个时辰后,王旁已经到了黄河边的大堤下,正看见高耸的堤坝上高高矮矮的站了一群人。身材高大的韩冈在其中最是显眼。

将马交给迎上来的随员,王旁疾步上了大堤,与正向韩冈汇报工作的方兴打过招呼,径直来到韩冈身旁,问着:“玉昆,你猜我方才见到了谁了?”

韩冈望着远处的工地没有动弹,漫不经意的回道:“郑侠?”

“呃……”王旁愣了一楞,转又醒悟:“是大哥的信?”

“除了元泽,还能从哪里听来的?”韩冈回头笑道。王雱前两天就写信来说了郑侠的事。编管恩州的判决,信上也写了。

要往恩州去,当然要经过白马县。虽然也可以从濮阳那里过河,但郑侠可是被押解着的罪官,有何等道理能让他绕道而行,浪费公帑?

“玉昆,我已经在县里的驿馆中将他夫妻俩安顿下来了。”王旁说着,又试探的提议道,“要不要去见他一见?”

方兴一听顿时来了精神,凑上来笑道:“提点,最好还是见他一面。待以重礼,厚给程仪,在外面也能博个不计前嫌的美名!”

韩冈瞥了方兴一眼,他脸上的笑容,怎么看怎么像是奸笑。

“见什么?相逢一笑泯恩仇吗?”韩冈摇摇头。他并没有打落水狗的心思,却也没兴趣表现一下所谓的宽宏大量,“事出无谓,何须如此。好生在驿馆里着,明日礼送出境就是了。”

尽管外面都在说郑侠心怀诡诈,欺君罔上,但身为当事人的韩冈并不会这么认为,那场雨应该只是巧合而已,郑侠没那个本事预测。

且从王安石父子三人的口中,韩冈也稍稍了解了郑侠的为人。即便出了这一档子事,王雱两兄弟都没有改变对郑侠的评价;同样的,相比起叛离的曾布,王安石对郑侠也没有什么恨意,毕竟郑侠对新法的态度始终如一,更何况郑侠已经自食苦果。

对于郑侠,韩冈无意揣着幸灾乐祸的心思去故作姿态,那样有失身份。而且就算能蒙过外面的人,但能蒙过郑侠本人吗?万一他一气之下跳进黄河怎么办?——韩冈很珍惜自己的名声,可不愿看到这样的事发生。同时韩冈也没有与其结交的心思,这等君子最大的毛病就是固执,去见他没得找气受。敬而远之,才是最好的选择。

既然韩冈不肯去见面郑侠,王旁也没办法,方兴也只能收了心思。随着韩冈一起,望着周围的工地。

入夏之后,黄河的水量依然不丰,只是在河床中心地带流淌,南北两边空出的河滩比起河面还要宽得多。就在黄河南面的这片河滩边缘,数以万计的民夫如同蚂蚁一般覆盖了高耸的堤坝。

单是白马县这边的百里堤防,韩冈就动用了上万名从流民中征召的民夫,将大堤加高夯实。丁壮上堤做活,而家中老小则是出外捕捉蝗虫换米。对于许多家庭来说,一天下来,还能结余个二三十文钱来,如果能持续两三个月,对于背井离乡的流民们来说,就能存下一笔度过荒年的资金了。

远远近近的号子声在河面上回荡,一根根木桩被提起,然后又重重的落下,大堤就在一记一记的夯筑下,变得逐渐坚固起来。

方兴指着工地道:“今天上堤的民夫,总计一百四十六组一万零四百二十一人。告病的有九十六人,加上昨日受伤送医的十七人,与疗养院报上来的人数能对得上。另外报了逃逸的有四人,姓名也已经报上来了,在下已经遣人去了四人所在保甲追查。”

方兴跟在韩冈身边半年多了,知道韩冈很在意施行中的细节,汇报起来,就是不厌其烦的说着数字。韩冈多次说过,所谓的‘重其大略,不暇细务’,这是对外面说着好听的。真正做事,从细节上就能看出来是否用心。

方兴用了大概有一刻钟,加上王旁上来之前的半刻钟,才将今天要汇报的工作捡着关键的地方,向韩冈都说了一遍。

