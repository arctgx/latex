\section{第41章 礼天祈民康(八)}

赵顼愣了一阵神后,忽然警醒过来。宰相是朝廷的脸面,不能让小臣冒犯。

“韩卿,此言不妥。毕竟不是一回事。”他口气倒是回护韩冈。

“微臣知错。”韩冈半转身对冯京一礼:“的确是韩冈失言,还望冯相公见谅。”

韩冈道歉的态度虽然礼数都到了,可落在赵顼眼中,却是有点硬邦邦的,看上去似有几分不服气的样子。

赵顼回想起了当日韩冈曾要郑侠到白马为官,亲眼见一见他为了安置流民所作的一切。完全是年轻气盛,受不得委屈的模样。韩冈少年得志,从来没有受过挫折,忽然之间受了污蔑,有此情状也是难怪。

不过冯京也的确做得不像个宰相,赵顼如何看不出来以冯京的私心。从冯京的角度来讲,韩冈最好离着政事堂远远的,现在倒也是如愿了。

赵顼双眼半眯了起来,宰相如此,难怪韩冈对中书都检正的任命避之唯恐不及。的确是要畏难啊,这可比安置流民难多了。

韩冈低头道歉,冯京则回以宽厚一笑:“无妨,无妨,不过是一时失言而已。”

宰相气度的冯京,此时恨不得生食了韩冈的肉。他没想到韩冈竟然如此毫无气度的当面讥讽他这位当朝宰相,而且还是在天子面前。但韩冈的话,硬是推敲起来,却还不能算是罪名,只能说是比喻不当,所以躬身一礼就算是道歉了!

可天子已经生疑。

同样是疑心。韩冈让天子起疑,不过是日后仕途坎坷一点。可宰相若是让天子起疑,那等于是宰相之位的基础受到了动摇。任何行动和言辞,都会引起天子狐疑的目光。

这让冯京怎么不恨!

从殿中退出来的时候,已是暮色深沉,只有西面的天空还带着一点残存的血红。

“多承相公推重,韩冈方能得偿所愿。”韩冈拱手一礼。无论如何,方才冯京都是举荐了他为判军器监,这句客套话,是他必须要说的。

“望你无负天子,用心任事。”

冯京套话回了一句,也不等韩冈回话,便一拂袖袍,转身而去。虽然步履依然保持着宰相沉稳,但他的这个态度,显是已经气急败坏。

“相公放心,韩冈理会得。”韩冈于冯京身后再行一礼,将礼数做得周全。

但这一下,他与冯京可算是正式撕破了脸,差不多可以等着下面的御史出头来弹劾了。

当然,一两个月之内不可能,皇帝对今日之事肯定还是记忆犹新,必然会有所怀疑。但三五个月之后,多半事情就会来了。而韩冈拒绝了韩绛、拒绝了吕惠卿,使得他在朝堂上孤立无援,到时候就只能靠着天子的信任。但天子许多时候是争不过臣子的,宰相做几个月就出外的可能并不大。既然冯京几个月后不会离任,肯定就是韩冈要吃亏。

不过,燕雀安知鸿鹄之志——确切点说,是燕雀安知鸿鹄之能!

有个三五个月时间,差不多就已经足够了。

冯京领头而行,韩冈不便超过他,故意走得稍慢,转过廊道,冯京便已经远远的走到了前面去。

看着前面宰相修长的背影,韩冈冷冷一笑。

‘无负天子’,冯京的最后一句话可是半带着威胁。

想及于此,韩冈的笑容多了几分讥讽。

天子的看法从来都不足为恃!王安石在熙宁初年,于赵顼乃是如师如长,言出无不依从,但不过五六年的功夫,这份宠信便不复存在,最后便黯然离京。

打铁要靠自身硬。韩冈很早就明确了这一点。

王安石养望的手段,韩冈学不来。而且王安石三十年的积累,不过几年就消磨干净,这前车之鉴,更是让韩冈不会去学。

王安石声望大落的原因很简单,他的人望是建立在士大夫阶层之中,由朝中的一干重臣常年加以延誉而来。不论是富弼还是吕公著,又或是文彦博,都曾赞许过他,当时期待王安石的盛况,甚至到了‘士大夫恨不识其面,朝廷尝欲授以美官,惟患其不肯就’的程度。

只是当王安石开始推行新法,原本对他赞誉有加的友人,便一个个背他而去。孤立无援的王安石只能违反朝堂循例,开始大加起用年轻的官员,却也惹来更多议论。如此一来,他在士林中的人望,当然会如同一级级瀑布缀成河道的山间溪流般一跌再跌。

而韩冈很清楚,如果他要想达成自己的目标,他的声望就必须建立在更为稳固的基础之上。

目送着冯京进了政事堂的宫院,韩冈转往宫门处走去。现在想这些也有点远了,不管日后怎么说,眼下也算是稍稍出了一口气。方才殿上的对话,肯定会传出去,而觉得冯京碍眼的,绝不止韩冈一人。

回到城南驿馆,刚刚歇下来没多久,便有客来访。韩冈一看名帖,竟是章惇,他连忙出去,迎了章惇进来。

“直院要见韩冈,片纸即可招至,哪能劳动玉趾?”韩冈开着玩笑的说着。

章惇前日刚刚升的知制诰、直学士院,虽然还不是翰林学士,但也已经跻身玉堂,离着学士之位只差一点了。

“片纸?天子的诏书又下了几道?”章惇笑着反问。

与韩冈说笑了两句,相邀了坐下,方正色问道:“玉昆,你当真无意任中书都检正?”

韩冈摊摊手:“两相两参各有谋算,中书之中漩涡潜藏,贸然深入其中,哪会有生路?”

去中书门下做五房检正公事,这并不是难,而是烂!中书之中一滩烂事,韩冈他不愿插手,想必章惇他也明白。

章惇当然明白,但有一点他更清楚:“那为何冯当世、王禹玉都怕玉昆你入中书?韩子华又盼你入中书?”

“实是诸位相公太看得起韩冈了。”韩冈轻描淡写的顶回去。

“玉昆,你的理由恐不止于此。”章惇追根究底。

“剩下的理由何须韩冈说出口,难道直院还不知道?”

章惇无奈的叹了口气,他怎么会不知道。格物之说,乃是韩冈素来所重。只为了能推动其在京中传播,韩冈都跟他的岳父差点翻脸。章惇很清楚在王安石这块巨石去了江南之后,韩冈打算要做些什么。

只是韩冈去了军器监,开始宣扬格物之说,到时候,同判经义局的吕惠卿还是要头疼。

如果韩冈当真受了韩绛的,那对吕惠卿来说就是腹心之疾。但眼下他得了判军器监的任命,在吕惠卿看来,那就是心病改脑病,都是让人睡觉都睡不安稳的。

他为着吕惠卿笑叹道:“吕吉甫这个参知政事做得殊是无味,总是不得安生。”

韩冈冷哼一声:“镇宅之物一去,屋中岂能干净得起来。要想镇住朝堂,得看他自己的本事了”

章惇闻言失声而笑,笑意中带着讽刺。

韩绛、冯京、吕惠卿,加上韩冈,在中书五房检正公事以及判军器监这两个职位上,各有各的算盘。

现在看来,韩冈算是遂了心愿,冯京虽然也是达成同样的目的,却是在这一过程中跟韩冈撕破了脸——这其实对韩冈不蹚浑水的本意来说,已经算是失败了——而韩绛不如意,吕惠卿则更是要头疼。站干岸的王珪心思当如冯京差不多,只是没有与韩冈交恶。

这还真是乱!

“君看随阳雁,各有稻粱谋。各有谋算,却没一个称心如意的。”

韩冈闻言,慨然一叹,“同在局中,概莫能外,又有谁人能超脱出去?”

章惇闻言微微一笑,各人都有各人的心思,难道他章子厚没有?只是他的心思与韩冈并不冲突。

章惇虽然与吕惠卿有些交情,如今也算是在辅佐其掌控新党,但从年龄和地位上说,两人之间是有竞争的,吕惠卿不可能不提防于他。而与韩冈年纪的差距,让章惇完全不必担心十年之内,两人会产生职位上的冲突。更别说两人之间的互相支持一直都没有断过,互为政治盟友的关系,可比与吕惠卿要亲近得多。

“吕吉甫近日又举荐两位崇政殿说书,其中有什么打算,想必不需要愚兄说了。”章惇说道。

吕惠卿的想法,韩冈怎会不清楚:“吕大参终究还要顾忌着家岳。不过这个人选私心太重,天子不会看不出来。如今可不是熙宁初年,再想靠着区区两位经筵官,在天子面前为新法说话,已是水中捞月,不见得会有多少成效。”

天子为帝日久,也越发的老练,掌控朝堂的手段日渐娴熟。吕惠卿效法王安石,以沈季长和吕升卿为崇政殿说书,这一做法,章惇也是不以为然。但他今天不是来听韩冈的嘲讽的:“好了,玉昆,别的愚兄就不多说了。今天愚兄来此本意只是要问你一件事。”

“还请直院明示。”韩冈明知故问。

章惇眼神一下变得尖利起来,仿佛要看透韩冈的内心,语调深沉:“到了军器监之后,你到底打算怎么做?!”

韩冈粲然一笑:“当然是萧规曹随!”

