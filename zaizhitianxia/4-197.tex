\section{第31章 九重自是进退地(11)}

一叶落而知天下秋,而京城中的官员甚至不用看到叶子落,就能知道秋天已经来了。

蔡确领衔弹劾,而吴充更是将沈括议论役法不便的文字,送到了天子案头,惹得天子震怒。几乎是在一日之间,原本炙手可热的三司使宅邸,一下变得门可罗雀。而有可能接任翰林学士和三司使的几个热门人选,则是转眼就宾客盈门。人人都在等着天子将贬斥沈括的诏书发下来。

韩冈亲自登门造访时,就没看到了沈家宅邸前停着一辆马车、牛车。马也只有寥寥数匹,孤伶伶的一排系马石,全数空空荡荡,只被占用了两根而已。换在往日,恐怕来迟一些的,甚至都别想挤进去。

“阴森森的,这三司宅当真是不吉利。”韩冈带出来的一个亲信家人打了个寒战之后,就咕哝着。

韩冈听着一笑,从这些年三司使的下场来看,的确是不怎么吉利。

三司使的宅邸是几年前薛向任职时新修的,门前是河,宅后则是大社,从风水上,于宅中住户不利——这也是最近沈括犯了事,京城中流传出来的谣言。

不过这个谣言倒有几分可信,薛向本人没有住进这座宅子,但他之后的曾布、元绛、邓绾都住进来过,也都无一例外的在三司使位置上落得灰头土脸,最后引罪出外,而眼下就轮到沈括了。

韩冈下了马,立刻就有伴当拿着他名帖去了宅门前,找沈家的司阍递过去。

沈括家的司阍人没精打采的守在大门处。眼前只要支个筐子,就能用来捉麻雀的空场,让他知道了什么叫落差。就在前几天,他一天还能挣到一贯半贯的额外收入,现在能有个十文八文就不错了。

终于看到一队人马过来,只是这群人中,既没有打着旗牌,也没见哪人穿着官袍,似乎是领头的高大年轻人,则是一身儒士的青布襕衫,就像是个普通的士子——除了身边的人多了一些。

只是当他懒洋洋没什么精神的从伴当手中接过韩冈的名帖,还没看名帖上的姓名,就从对方嘴里听到了京西都转运使、龙图阁学士的头衔,一下便惊得跳了起来。瞪大眼睛再看韩冈,就不是随从多一点的秀才,而是微服私访的高官气派。

司阍急匆匆跑了进去,过了片刻,中门大开,沈括直接迎了出来。

“存中兄!”韩冈看到沈括的时候也被吓了一跳。

原本沈括留着半尺多长的三缕长须,有着士大夫的清逸。但现在出现在韩冈面前的沈存中,飘逸的长须却是少了一撮,秃掉的地方黑黑的像是颗指间大小的黑痣,从色泽上看,是抹了养伤的药膏。

但一看到沈括脸上尴尬的表情,韩冈便收起了惊讶,当做什么都没看见,“存中兄,别来无恙。”

沈括一声苦笑,“如今这番局面,岂曰无恙?”他亲热的拉起韩冈的手,“玉昆今天登门,愚兄实在是没想到,还请家中说话。”

韩冈毫不犹豫,随着沈括走进去。

在客厅中分了宾主做下,寒暄了几句之后。沈括小心的问起韩冈来意,“愚兄眼下的情况,玉昆也当知道了。不知今日登门究竟是为了何事?若是叙旧,愚兄已承了玉昆的情。”

“叙旧也是一件。役法实行经年,或有不如意的地方,若能更正,也是一桩美事,只是存中兄行事未免有些孟浪了。”看着沈括脸色有些难堪,韩冈瞥眼看了一下横挡在厅中上首的屏风,笑道:“韩冈也不是来登门问罪的,想必存中兄也知道如今韩冈身上背了什么差事。”

沈括点点头,“玉昆主持襄汉漕渠一事,想必是胸有成竹了。以玉昆之才,天子和朝廷当可静候佳音。”

“小弟愧不敢当。”韩冈自谦着,“存中兄大才名重于世,天文地理、河工水利,无所不通……”

沈括听着摇头,“有玉昆在,愚兄哪里当得起这几句,自愧不如,自愧不如。”

韩冈看着沈括颓然的模样,也不捧他了。应酬式的笑容收敛了起来,正色说道:“韩冈想说句冒犯的话,还望存中兄海涵。”

待沈括点点头,说了句‘但说无妨’,韩冈就继续说道:“依眼下的情况,存中兄只能是外放了,难以再留居京城。”

客厅的屏风后,突然传出一声很细微的冷哼声,沈括尴尬,韩冈只当没听到:“不过出外就郡,也有两个选择。一个是待罪,一个则是立功,敢问存中兄愿意选择哪一项?”

沈括也是绝顶聪明的人,韩冈这么一提,他也就想得明白:“玉昆是想要愚兄去京西?”

“襄汉漕渠,韩冈独力难支。但若有存中兄相助,此一事当无所滞碍。”韩冈笑了一笑,“襄汉漕渠一旦功成,便是第二条汴河。即便过去有再大的过错,也足以抵得过了。”

沈括沉吟着,韩冈的提议对他来说的确是很有吸引力,将功赎罪,怎么都能抵得过了。这一次,屏风后就没声音了。

沈括想了半天,韩冈静静地等他答复。只是最后,沈括瞥了一眼屏风,却说道:“玉昆难得造访,愚兄家中也有些粗茶淡酒相待,还请稍等,待愚兄去吩咐一下。”

韩冈一笑,知道沈括此事不敢擅专,需要进去问一下。不过想必他的那个河东狮,即便性格再暴虐,也不会蠢到毁掉沈括卷土重来的机会。点点头,“也好,久未与存中兄共饮,今日当共谋一醉。”

沈括这边是敲定了,韩冈在沈家喝了一顿酒后,得到了一个肯定的回答。剩下的问题就是要说服天子。

这件事当然不难,天子也当是想早日看到襄汉漕路打通。早一天开辟一条沟通南北的新道路,那开封的安全也就多上一分。

沈括从才能上说还是很出色的,朝中也是知名。前两年丈量汴京到泗州的地势高下差别,就是沈括领头测量,最后测出来的结果是十九丈四尺八寸六分。此外他任职地方的时候,在水利上多有创建,万顷良田都是由他所开辟。如果有他辅助韩冈,当然是强强联合,把握更多上一分。

韩冈回去后的第二天,就将已经写好的表章递了上去,希望朝廷能安排一名擅长水利和土木工役的官员,去汝州或是唐州——方城山便是两座州郡的界山——虽然韩冈在奏章中并没有点沈括的名,但如今朝中最擅长水利和土木工役的官员,除了韩冈之外究竟是谁,自是不言而喻。

赵顼考虑良久之后,就将降罪诏书上的宣州改成了唐州,同意让沈括去京西戴罪立功。但韩冈找了沈括助阵的这件事,还是出乎世人意料许多。尤其是沈括反复无常的墙头草模样,现在是新党旧党都不待见,人厌鬼憎,朝中顿时一片哗然。

韩冈上书请求将沈括外放唐州,其实是帮了他一个大忙。只要襄汉漕渠功成,凭着这份功劳,也能赎了旧过,说不定还有找头。

私下里,王韶也询问过缘由,韩冈则是尽可能诚实的回答了。

他有自己对未来的计划,并不打算在京西耗费太多的时间。

襄汉漕渠历史上虽然没有开通,但故道皆在,只要稍加疏浚便可。唯有穿过方城垭口的那一段要深挖,至少六七丈深。从土方量说,在这个时代基本上是个天文数字。更别提万一下面都是石块,那就更是让人无能为力。

韩冈打算通过轨道来跳过这道难关。但他既然说过要重新开凿襄汉漕渠,那么方城垭口的那一段的渠道,也不能就此置之不问,否则也少不了有人鸡蛋里面挑骨头。

所以韩冈需要一个接手之人。他本人只要能保证通过轨道达到百万石的运力,那么他承接的这个任务就算是成功了。接下来,继续挖掘方城垭口的河渠的任务,韩冈就可以交给汝州、唐州接手,不需要他这位京西都转运使继续为此殚思竭虑。

不过王韶没有将韩冈的话外传,所以第二天又引来了另一人来质问。

“玉昆,你可知道蔡持正昨天在御史台中与人说什么,”隔了一日,章惇便为此事找了过来,对于韩冈事先没通气,他着实有些不痛快。

“蔡确?”韩冈知道这一次是蔡确领头弹劾沈括。比起蔡确看风色选站位的本事,沈括的确差的太远。蔡确当年对王安石反戈一击,仕途却没有受到多少挫折,如今眼看着就能升御史中丞了,而沈括,却是狼狈离开。

“他说了什么?”韩冈问道。

“‘都说舒公好放生,每日就市买活鱼,想不到韩玉昆也学着放生了。’”章惇学着蔡确的腔调,“可不要落到水里,连个水花都上不来。”

韩冈闻言,神色一动,“家岳确定要晋舒国公了?”

韩冈顾左右而言他,章惇无可奈何的摇了摇头,只得将此事放过,沈括怎么说也是有才华的,韩冈得他襄助,襄汉漕运的把握又多了一分,“难道玉昆你还不知道,介甫相公辞江宁府,就宫观使的辞章,已经上到第三份了——第一封刚到江宁两天就上了。昨天已经议定,天子也同意了,介甫相公江宁落职,改集禧观使,过两日等太常礼院那里将制书做好,就会颁诏。”

