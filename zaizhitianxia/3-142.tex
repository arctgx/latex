\section{第38章 心贼何可敌(上)}

赵顼这一年来,用切身体会了解到了什么叫做祸不单行。

旱灾、蝗灾、粮荒、流民,这是环环相扣,有一有二就有三的,也许并不足为奇,但契丹却是趁此时机,向他勒索土地。

赵顼推行新法是为了富国强兵,可到了内忧外患一齐而至的时候,他却发现实行了几年的新法,竟然不能让他的国家平平安安的度过这一场危机。

席卷全国的大旱刚刚过去,留下的后患还没有收尾,而契丹人的贪婪在使节一次次南下中暴露无遗。

宰相王安石一个劲的要让他强硬以待,无须畏惧。可如今的时局,赵顼他怎么强硬得起来?

河北流民在道,而最为充裕的开封常平仓也逐渐枯竭,而朝廷还要负担着流民的生计一直到明年夏收。试问这样的情况下,大宋如何能经得起一次大战?

若是契丹入侵,朝廷无法救济河北流民,事情就会变得如同富弼所言,四方凶徒,观望之人,‘谓国家方事外虞,其力不能制我,遂相啸聚,蜂猬而起。’

到时候,他的国家覆亡可就在眼前。

这段时间,赵顼夙夜忧叹,难以入寐,身体一点点的消瘦下去。

但越是如此,他就越不会放手政事,每天不看到奏章,赵顼就难以安心下来。

正好元老之一的张方平回到京城,要转任南京应天府,依例当进宫入对。

张方平虽然不如韩琦、富弼和文彦博的地位,但也是仁宗朝就做了翰林学士,又做过参知政事的前任执政。而且在英宗病重,欲立赵顼为皇太子时,正是他从英宗手上拿到了御笔手书,算是有定策之功,元老二字也算当得起。

张方平在殿上再拜起身,虽已近七旬,须发皆白,仍是精神矍铄。

赵顼先赐了座,等张方平谢过坐下,方道:“卿家在陈州,理民有方,安民有术,走马多有言及。”

“不敢。臣老迈无能,不能为陛下分忧。”张方平抬头看着赵顼,叹道:“陛下可是瘦了。”

赵顼心中一暖,也只有这等老臣才会关心自己,笑道:“卿家的身体却是康健。”

“乃是陛下圣德庇佑。”

君臣寒暄了几句,赵顼问道:“素闻卿家明西事。契丹欲与西夏为婚,不知卿家以为如何?”

张方平道:“陛下勿须多虑,契丹旧年曾与董毡联姻,又何曾胁及西夏。西北二虏,凌逼中国,并不在婚姻,而在其兵强马壮。”

赵顼沉吟了一阵,问道:“庆历以来之事,卿家知之否?元昊初臣,当日又何以待之?”

张方平低头回道:“臣时为学士,誓诏封册,皆出臣手。”

“卿家其时已为学士,可谓旧德矣。”赵顼感慨一阵,道:“如今之事,朝中众说纷纭。卿家元老,身历三朝,当为朕解惑。”

“不知两府诸公如何说?”张方平抬头问道。

赵顼犹犹豫豫的道:“但言契丹君昏臣黯,国势衰弱,且苦于内乱。其不来便罢,若其南来,当可一战而胜!”

张方平嘴角微抽,露出一丝似笑非笑的神色,他在天子的话语中,听出了很浓的犹疑:“陛下可知百年来,宋与契丹交锋几何?胜负几何?两府八公可曾禀明陛下?”

赵顼闻言一愣,这事可都没人跟他说过,也从没有细细数过,“卿可为朕说来!”

张方平面容整肃,厉声而道:“凡与契丹大小八十一战,惟张齐贤太原之战,才一胜耳!”

赵顼脸色发白,难以置信的问道:“仅有一胜?!”

“若非如此,何来澶渊之盟?”张方平反诘道:“契丹太后、天子、宰相领军深入宋境,顿兵于澶州城下,其后路又有王超领二十万兵马堵截,遂城、梁门皆有良将控扼,为何以寇准之胆略识见,还不促真宗与之决战?”

张方平喟然长叹,语气沉重的说道:“兵虽众而力难敌,不足以胜之也。”

赵顼默然不语,细细想来,的确是这个道理。

见着赵顼已经动摇,张方平步步进逼:“故事历历在目,和与战,陛下以为孰事为便?”

赵顼难以决断,他当然愿意以和为贵。可如果真的如了契丹人之愿,他这个天子如何还有脸面见人。勉强回道:“用兵虽不便,可委曲求全亦非善策。”

“臣愿陛下以太祖为法。”张方平语气沉重:“太祖不用兵于远,如灵夏、河西,皆因酋豪盘踞,遂许之世袭;环州董遵诲、西山郭进、关南李汉超,皆厚加禄赐,且宽其文法。诸将财力即丰,太祖之命便俯首遵循,不复五代故事。其时间谍精审,官吏将士皆用命,故而能以十五万禁军,而当百万之用。及至太宗谋取燕蓟之地,又内迁李彝兴【李元昊先祖】、冯晖,朝廷便自此而为边事所扰。真宗澶渊之战,与契丹为盟,至今人不识兵革。三朝之事如此,望陛下鉴之。”

赵顼听着张方平侃侃而谈,并不知道里面给掺了多少私货,只觉得张方平说得甚为有理,而且越听越是有道理。

心中的想法不由自主的在脸上流露了出来,张方平一见,便趁热打铁:“如今两府、边臣,皆言不惜一战。其人之言,只为一己之私,乃欲以天下于一掷。事成而不见利多,不成则诒以后患,陛下切不可听!”

赵顼颓然的闭起眼睛,旋又睁开,“昨日沈括进京入觐,所言称旨,朕已命他去枢密院查阅故牍旧档,望他能查明过往,也可让朝廷以理服人,让北人愧而自退。”

赵顼虽然没有明说,但心中意向已经确定。

张方平低下头,“陛下圣明。”

……………………

王雱无功而返,见过妹妹之后,次日一早便离开了白马县。

他没能说服韩冈,但也没有多少郁愤,心中只有无奈。

天子畏敌如虎,虽然韩冈没有明言,可对此的腹诽,王雱也是心知肚明的。如果能够挽回——就如流民图案一样——王雱相信韩冈会为此而努力——他的这个妹夫之前的奏疏,王雱也从父亲那里听说了,其中的言辞极是激烈,吓得天子不敢让他去河东。

只可惜韩冈也自叹无能为力。相比起年龄,韩冈丰富得让人难以置信的经验和经历,让他的话比起王雱更有说服力。王雱眼下得不到他的支持,别说说服天子,就是说服父亲也难以做到。

而且也正如韩冈所言,退一步海阔天空。既然未来还有入相的机会,何必恋栈不去?避过眼前的危机,让天子独力承担

看看立国以来的历代宰相,两次、三次为相的数不胜数。韩琦是三进政事堂,文彦博做过宰相,又做枢密使,而富弼也同样是两次为相。上上下下根本不出奇。能在相位上一坐十来年的,扳着手指也数不出来。

王安石今年才五十三,这个年纪对于宰相来说,其实还很年轻,在两府中的政治生涯才刚刚开始。现在退下去,过两年朝中局势动荡的时候,又能重新回到政事堂中。等两次三次为相,元老重臣的身份也就有了。

送了王雱回来,韩冈也在想着今次之事。

其实王安石的下台,早已有了心理准备,否则韩冈也不会这么容易就能让王雱放弃。换作是熙宁初年,王安石的话,天子怎么会完全听不进去?王安石在天子那里的信赖基础,已经不足以支撑他做宰相了。

眼下的关键还是在如何新法的存续上。

韩冈并不认为王安石的下台会导致新法被废。如今的财政问题是无解的,除了王安石,没人能给赵顼一个有用的回答。韩冈虽有自己一番想法,但要施行起来,却也得慢慢来,绝无可能一蹴而就。

但也不是说新法就稳如泰山。王安石下台后,很有可能新法就会被废除或部分废除,然后天子看着情况不对,再来恢复。

凡事没有不经挫折便能成功的道理,只有来回反复,让赵顼吃点苦头,他才会坚定对新法维护。

昨夜从王雱口中,韩冈听说了他的岳父,在旱灾闹得最厉害的那段时间的想法。当时相位不稳,已经有出外的准备,王安石有心推荐韩绛代为宰相,并让吕惠卿进入政事堂。

韩冈对此其实并不是很赞同。让冯京、王珪继任不好吗?让他们尽管废新法去,将朝政弄的一团乱,到时候,王安石再来收拾手尾。

不过王安石的性格肯定不会干,就是说给王雱听,他也肯定会一下蹦起来。所以韩冈将这话藏在了心底,没说出来。

回到房中,王旖在床榻上半靠半坐着,精神已经好了许多:“大哥已经走了吗?”

韩冈点点头,坐到床边,将拖下来的被子好生的给盖好。

王旖小心的看着韩冈的脸色:“大哥这次来,是不是有什么要事”

王旖正是坐月子的时候,不能累着、冻着,稍有不慎,就会落下病根。

韩冈让她躺回去,笑道:“没事,没事,你多睡一会儿,养好身体才是,这些事就不用太操心了。”他叹了口气,“这等事,我也不想去烦了。”

