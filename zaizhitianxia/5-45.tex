\section{第七章 苍原军锋薄战垒(一)}

李稷寒着脸回到后方的营地,因为粮草不济,他在种谔那里讨了个没趣。

坐下来还没等人奉上茶汤,就拍着交椅发作道:“章楶呢?他转运判官做得好啊,该送到的粮食拖到现在都没有到,真当我不能斩他的首级不成?!”

一名亲兵小声的提醒李稷:“运使,章运判方才已经押粮草进了营。”

李稷脸色微微一变,不甘心的又问道:“多少?运到的有多少?”

“听说是五千石,具体数目小人不敢细问。”

“才五千石,够吃几天?”李稷冷哼一声,却也没再多说什么。

章楶出身浦城章家,族叔章得象是宰相,族弟章惇是执政,族侄章衡是状元郎,可是当世赫赫有名的大族,可不是任人欺辱的寒门。没有充分的理由,根本不能动他分毫。

等到解暑的凉汤送上,李稷喝了一口,随即又提声喝问:“吕副使呢?”

吕大钧是李稷的副手,但他对眼下的局面也是束手无策。

夏州离得太远了,提供给种谔的粮草,两成在罗兀、五成在绥德,剩下的则在延州。就是从罗兀城运过去,都有两百里之遥。绥德的粮食要运到罗兀,延州的粮食运到绥德,而从关中来的粮草则是汇集延州。这些都要转运司操劳,组织民夫转运,让李稷伤透了脑筋。

‘得想个办法才是。’李稷想着,‘看样子这一战的结果或许有变也说不定,这时候得先留条后路。’

……………………

由于东京城和前线的路途遥遥,最新送抵京城的军情,随着各路的不同,与实际时间有五天到十五天不等的差距。

当韩冈同时收到官军攻下兰州、夏州的消息后,并没有染上半点朝野内外弥漫着的兴奋。

兰州的情况乃是预料之中,时间也没有耽搁,甚至比韩冈预计的还要快了一点。

但种谔那边明显就有问题了。与一个月前,种谔率领鄜延军进兵的速度相比,一旦刨去当初在弥陀洞上耽搁的时间,前后所花费的时日竟然一模一样。

——有一点是绝不能忘掉的。在夏州之前,银州、石州,所有的城池都已经被攻破了,所有的敌军也都被清洗过了。这样的情况下,单纯的行军竟然依然与一边作战一边行军时有着一样速度,怎么想都觉得其中肯定有哪里不对劲。

究竟是种谔失去了锐气?还是京营禁军成了拖累?韩冈没有千里眼,但他知道,多半是兼而有之。而韩冈更清楚,如果光是这两个原因还好说,最糟的情况是后方粮草供给不上,因此才拖慢了官军前进的脚步。

而当韩冈看到永兴军路转运使兼鄜延路经略司随军转运使李稷向朝廷发来的急报,声称陕西天气暑热,牲畜死亡太多的时候,不无感慨的发现,最糟的情况已经发生了,而且统管转运的主官分明已经对此失去了信心。

这根本就是开始为了失败而在推卸责任了!现在于天子面前做了报备,等到当真失败的时候,便能藉此脱身了……或许脱身不了,不过至少罪名能推卸一部分给负责牲畜调配的群牧司,由此而减轻一点罪责。

韩冈可不会容忍有人往自己身上泼脏水。他跟李稷不熟,可不会为这位明显能力不足的转运使多担待一点。

就当着天子的面,韩冈毫不客气的拆穿了李稷的用心:“看来李稷是没有把握能为鄜延路十万兵马及时送上粮秣,为自全而寻求退路了。”

“韩卿何有此言?”赵顼很是不快的皱起眉,李稷不过是在抱怨而已,怎么韩冈就像被踩到尾巴的猫一般,一下跳到老高。

“陛下明察。”韩冈持笏向赵顼一礼,李稷都知道要留一条后路了,他可不会犯糊涂:“臣在战前调配各路军马。在诸路之中,提供给鄜延路的军马是最多的。而且从永兴军路征发的牲畜,分给鄜延路的数量也是最多的。现在各路还没有叫苦,鄜延路却第一个叫了起来,除此之外,臣实在是想不出其他理由了。”

赵顼沉着脸不说话,韩冈进一步说道:“同州沙苑监,如今还有三千匹种马,京兆府各县中也还能调集千余匹马驼,只要陛下应允,臣可以保证李稷上报死了多少牲畜,就给他补上多少,并多加两成。这样一来,如果再有粮草不济之事,此罪当与群牧司无关。”

这都是官场上见多的把戏,纸面上的言辞都是表面文章,藏在深处的算计到底是怎么回事,也不只是韩冈一人看出来,想来李稷也不会赌其他人都是瞎子。想来他多半是认为群牧司没办法填上这个漏洞,所以才有恃无恐。

只是他错估了韩冈的能力,更是误判了韩冈的脾气。而且韩冈可是自始至终都是反对激进,李稷的做法等于是将刀子送到韩冈的手中。

却之不恭!

韩冈不求赵顼现在相信,也不是为战后推卸责任做打算,他是在设法动摇赵顼的决心。

由于粮秣的问题,想必各路进兵的速度都受到了影响,现在的局势还来得及挽回。否则一旦官军抵达灵州城下,要么全胜,要么就是全败,不会再有第三种结局了。

“韩卿。”赵顼语声徐缓,凝视着韩冈的眼神充满威严,“三千种马价值以百万贯计,不是等闲之物可比。”

“种马易得,胜机难觅。若是因为牲畜不足而贻误战机,朝廷的损失会更大。”

韩冈这是在挤兑天子,一点顾忌都没有。官军越是高歌猛进,他的心就越是抽紧一分。

党项人设在灵州的陷阱任谁都能看得出来,但赵顼认为党项人的计策只是垂死挣扎,不会有任何作用。可在韩冈的眼中,如今的局势已经到了一翻两瞪眼的时候,成与不成就在灵州。官军越接近灵州,西夏翻盘的机会就越大。

李稷现在说牲畜多病死,便是为了推脱粮草供给不上的责任。而能影响粮道的,不仅仅是组织上的问题,还有虎视眈眈的党项人,他们想反败为胜都想疯了,诱敌深入的计划不就是为了拉长粮道以便下手吗?

“当年以绥德城为出发地,向北攻取罗兀,仅仅不到百里的距离,便已经给了党项人足够的空间来截断官军后路,如今一跃千里,难道其间就没有让西贼下手的余地?”韩冈提高嗓门,“除非官军能顺利的攻下灵州。否则这一仗必败无疑!”

韩冈对西军很有感情,相对的,由于过去的往来,西军上下也对他很有好感。加之疗养院等事,以及他母家出身军中的身份、两个兄长也算是战死疆场。文臣之中,韩冈对西军的影响力算是最大的一个。

已经看到迫在眉睫的危机,韩冈无法说服自己坐视,然后等自己的预言成立。

赵顼脸色变得难看了,没有人喜欢乌鸦嘴,万一说出来成了真怎么办?

唯一在殿上的宰辅王珪,觉得这是韩冈在嘴硬不肯认输,他在旁笑道:“官军有板甲、有斩马刀、有神臂弓、有飞船、有霹雳砲,灵州不足为虑。”

韩冈被堵了一下,这里面大部分还是他的发明。韩冈寒着脸:“可惜没有粮食。军器皆是外物,食、水才是肚中货。没有吃的、没有喝的,纵有板甲也穿戴不了。”

赵顼这些天来派了人去暗查群牧司。知道韩冈对于前方的要求,都是不折不扣的完成,没有一点从中阻挠的想法。

韩冈行事清正,赵顼对此很是欣赏。但这并不代表他欣赏韩冈对战局的悲观看法。

“韩卿,六路至今都没有一路声称缺粮。纵有些许延误,很快就能运送上去。”

“因粮于敌已经不可能,只凭现有的运输能力,鄜延、河东的军粮,支撑不到灵州城下。环庆、泾原、秦凤、熙河的情况也差不多。”韩冈双手紧紧攥着笏板,“骡马牲畜之事,群牧司可照应得全,但六路三十余万官军的粮秣供给,没有一家能照应得全。告急的文书不会太久。”

韩冈对种谔很是了解。以种谔的为人,一旦军粮不济,绝不会蠢到强赖下去,肯定要设法寻求保全自己。只要青山在,不怕没柴烧。像他们这等宿将,对危机的嗅觉是最灵敏的。一见时机不妙,在战场上,是设法领军后撤,在官场上,便是设法将责任往外推。抱怨粮草不济,耽搁军事的奏章这两天就该送到京城了。

赵顼叹了口气,发现自己招韩冈上殿觐见是自己跟自己过不去,这是何苦来由?

结束了短暂的接见,韩冈随即离殿。王珪留下独对。他笑着对赵顼道:“不是一家人,不禁一家门。看到韩冈,就想到他的岳父了。”

赵顼点了点头,韩冈执拗起来,的确不比王安石稍差。笑了起来,“过个二十年,就又是一个拗相公了。”

不过,赵顼很快就笑不出来了。

当天夜里,河东军的运粮队遭袭的消息传到了京城。河东路第四将副将訾虎战死,押送粮草的千名将士和三千人夫死伤泰半,大量的牲畜车辆损毁,而运送的近三万石束粮草全数被焚。不过李宪在请罪的同时,也向朝廷提议借用鄜延路的粮食,以防河东军断粮。

赵顼没有不批复的道理,朱批时唉声叹气,想起了韩冈的话,又赶紧派人去督促前线的粮草转运。

只是时局变化得很快,好消息则紧随其后。

熙河路方向攻下了卓啰城,拔掉了卓啰和南军司,接下来王中正便依照预定方案帅主力向东,王舜臣领偏师西行。

泾原路的苗授攻克鸣沙城,环庆路的高遵裕攻下韦州,紧接着两军都开始向灵州挺进。将鄜延路甩到了身后。

