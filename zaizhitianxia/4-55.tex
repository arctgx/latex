\section{第11章 安得良策援南土(五)}

韩冈坐在桌前。

青色纱罩笼着烛光,照得室内一片晕黄,偶尔噼啪一声的烛花轻爆,就会让他映到墙上的身影一阵摇晃

外面的更鼓声提醒着他已经快要到三更了,可是韩冈现在还没有困意。明天应该能听到结果了,就不知道天子会不会如他和章惇所愿。

其实章惇并不是个合适的人选,王韶才是。王韶若是愿意,只要他向天子申请,决无不允之理。而当他当上主帅,怎么调兵都容易解决。

韩冈前日曾跟王韶商议过,希望王韶能主动申请南征的帅位,但王韶摇了头。

王韶并不想去南方。就算平灭交趾能让他更进一步,他更愿意凭着对西夏的战功,成为枢密使。

王韶看似激进,献奇策、用奇兵、立奇勋,世称三奇枢使,但他其实是保守的性格。旧年在陕西游访数载,将河湟边地的地理民情都了解通透之后,才献上了平戎策。而担任秦凤路机宜时,也是深入了解的本路人事,提拔了一干合用的手下,才正式开始拓边的进程。河湟开边在军事上的顺利,其实也是跟他夯实了用兵的基础分不开关系。

现在要他立刻统领大军去全然陌生的南疆,王韶并不愿冒这个风险,而且统领荆南军队他更是没把握,尽管有李信和刘仲武。“如果以三年为限,倒是可以走一趟。”

“邕州哪里能等到三年,三个月都等不了。”韩冈记得昨天他是这么对王韶说的。

而王韶则是很冷淡的回答:“邕州不需要救,也救不了。能及时救援邕州的军州只有桂州和广州。即便是荆湖南路的潭州,从收到朝廷诏命到发兵抵达邕州,差不多也要一个月的时间。算上消息来回京中传递的耽搁。自邕州被围,到潭州援军抵达,最快也要两个月以上——差不多七十天!

“以交趾国力,能支持出战的十万大军的粮草供给?我也查过了钦州、廉州还有邕州以南几个军寨的存粮数目,也支持不了十万兵力多久。交趾发兵越多,围城的时间就会越短,怎么算绝不会超过两个月。从潭州派去的援军抵达,交趾人早就退了。”

“也有可能是号称十万而已。”韩冈反驳道。

“如果十万仅是号称,实际上只有两三万,那邕州城则根本不需要担心。就算三五万,只要在攻城时损失个三四千,也就得退军了,以邕州的城防杀不了三四千贼军?交趾人什么时候善于攻城过?”

王韶说得是兵家正论,韩冈无从驳起,这些都是他知道的,根本没办法否认。但他总觉得关系到拥有数万子民的城市安危,只要有百分之一的可能,就要做出百分之一百的努力。

王韶知道韩冈不服气,耐下性子说道:“就如当年平侬智高之乱,皇佑四年九月廿一狄青领命为帅,十月初八方才上殿陛辞。等到他领军抵达昆仑关的时候,都已经是皇佑五年的正月了。这还是算快的,朝中无人作梗,而西北两处有澶渊之盟、庆历和议,都不会阻碍调兵。换做现在,你想想能否来得及?”

“以邕州的情况,要么就是数日之内就被攻克,要么就是交趾人支持不住、撤围而去。说交趾贼军能围攻邕州城数十日,甚至几个月,几乎没有这个可能。……玉昆。你一向沉稳,怎么今次如此焦躁?”王韶不解的问道。

“不是急躁,而是只觉得该这么做。”韩冈如此回答。

他的态度是受到了后世对交趾的成见的影响,而苏缄这位忘年交,也是韩冈想尽快出兵援救的原因。听到了交趾北犯的消息后,韩冈十分后悔没有听信苏缄的警告,没有尽力帮他说服朝堂上下。

南方潮湿多雨,八百具神臂弓可坚持不了多久,就算神臂弓用得筋角等物并不多,但也同样容易受到湿气的影响。如果邕州失陷,韩冈不觉得自己会没有责任。

只是王韶的态度,让韩冈难以如愿。而这等事,也是勉强不来,韩冈也没办法劝得王韶改变想法,正如王韶对他说的:“就没有想过仓促出兵进而失败的情况?不是所有人都能如侬智高那般好欺负的。”

甚至是郭逵,恐怕也不一定会愿意。有狄青的先例在,武将升得越高就越是危险。胜则不能加功,败则不免责罚,他何必去吃那个苦头?

只是韩冈救援邕州的心意坚定,所以他与章惇一拍即合,尽快调动荆南驻军南下救援。灭国的事先放到一边,先击退交趾人,确定广西的安全,才是首先要考虑的。

近三更的时候,在妻妾的催促下,韩冈正准备就寝,天子的传唤也送到了他的家中。

从前来传达口谕的内侍的口中,听到了天子到底招了哪几位入宫,韩冈脸色大变。摇着头,骑上马,随着前面并不相熟的内侍,在夜色中匆匆赶往宫中。

快要抵达右掖门的时候,就见到前面一队只有十来人的队伍,正在城门前,上去一看,正是章惇。

回头见到韩冈,章惇就忍不住抱怨起来。

“天子怎么这般糊涂?!”章惇都口不择言了,也不管两名内侍就走在前面,“宰执是什么身份,连夜招入宫中。明日京中说不得就要遍传谣言!”

“我也是这么说的。”韩冈摇了摇头,叹了口气,也是觉得赵顼实在犯了糊涂。

他和章惇白天时都是认为赵顼会绕过两府直接下旨。两府八公如今分成三派,一派旧党、一派新党、一派看热闹的,朝堂上要能争出个结果就有鬼了。而天子的性子急,章惇和韩冈在白天时又给了他另外一个选择,纵然可能会导致一系列的后患,但眼下毕竟没有发生,所以直接采纳两人的意见可能性很大。

但两人绝没想到竟然会被连夜唤入宫中。而且不仅他们,连同宰执们都一起叫来了。从内侍嘴里听到这个消息后,韩冈一直在摇头。出了这等事,明天还不知会被风传成什么样,再是军情紧急都不能这般行事!

招小臣漏夜入宫没什么,就算是翰林学士、中书舍人这样的两制官也一样说得过去。可宰相执政就不同了,身荷国鼎之重,一举一动都影响着朝堂大局,多少双眼睛看着他们。一见宰执们匆匆忙忙入宫,还能会有什么样的联想?说得难听点,现在雍王赵颢恐怕已经被人从床上叫起来,竖着耳朵听着宫里的动静呢。

在外面听了通传,两人趋步进殿。两府八公,王安石、韩绛、吴充、吕惠卿没到,而冯京、王珪、蔡挺、王韶则是都到了。而同时被传召的燕达也到了。

宰执们的府邸离着宫城很近,这个时候又不可能再出去饮宴,应该都在家里。现在没到的,多半就是直接挡回了天子的口谕。韩冈心道,看来谁是真宰相,这时候就能分得出来了。

赵顼脸色正难看,当王安石第一个将他派去的内侍赶回来时,他就知道这件事做错了。但事情已经如此,有没有补救的余地。反正谣言毕竟是谣言,只要没有后续的事实跟进,京城中就是又再大的风浪也能平复下来。

看到韩冈和章惇两人进来,赵顼也不多说闲言赘语,直接道:“章卿,韩卿。关于援救邕州一事,需要调动多少潭州驻军。”

“陛下,此事万万不可!”冯京当即站出来反对,“万一江南变乱,敢问如何收拾残局?”

王珪也道:“陛下,江南可是根本,若无江南纲粮,京中亦要闹起灾荒。”

冯京、王珪的阻挠在意料之中,甚至不能说他们是以私心坏国事。如今的情况,没有一条是万全之策,章惇和韩冈的提议,也是冒着风险。相对于两人,冯京他们的提议则更为稳妥——只要将邕州丢在一边。

而现在没有王安石、没有吕惠卿,王韶因为不支持韩冈的提议,也不会帮着说话,只有韩冈和章惇孤军奋战。

“只要赈济及时,安抚有方,何惧流民作乱?难道相公认为江南有多少不肖之官,会惹得百姓揭竿而起!?”

这等水平的反驳,冯京根本不放在眼里:“两年前的河北流民,靠的都是江南的纲粮。如今江南灾荒,赈济灾民的粮食又要依靠哪里?南征大军南下,消耗的粮草又要从哪里征调?”

“够了!”

赵顼厉声呵斥道。又变成了白天时两边争执不下的场面了。见到天子发怒,冯京和章惇便低头请罪。

“章卿、韩卿,调动潭州驻军救援广西,到底需要多少兵马?”

“只要两千人足矣!”章惇答道。

“两千兵马?!”赵顼惊讶的问道。

“只需两千!”韩冈为惊讶的天子解释道,“譬如斩马刀,其锋一刀可断马首。可近六斤重的长刀并不是皆为精钢所制,而仅是以钢为锋刃,刀身仍为锻铁,也就是夹钢法所打造。用兵亦是如此,精锐为锋刃,寻常的军卒才是主力。以千名历经战事的精锐开路,而桂州守军从后掩杀,交趾贼军不堪一击。”

冯京冷笑起来:“只是区区两千援军,不及桂州驻军的三成。领军将校又是资望浅薄,如何会被帅臣看重,当做锋刃使用?交趾敢侵攻中国,刘彝当引罪避职。桂州知州尚未选定。既然章惇信誓旦旦,臣请以章惇为桂州知州,统领广西军事,平靖路中。”

“章卿?”赵顼将视线转了过来,冯京的提议倒算不错。。

章惇愣住了,他是要做南征的主帅,不是广西经略。但他是个敢于拼命的性格,这时候拒绝了,南征之事可就跟他再无瓜葛。答应下来,日后还有机会,“臣领旨!”

见章惇答应了去桂州,赵顼又想起了韩冈的职位,“韩卿。冯卿、王卿、吴卿同荐你为广西转运副使,不知你可愿意就任此职?”

韩冈同样是怔了一下,怎么是广西转运副使?脑中这么一转,立刻就反应过来。心底冷笑,小心偷鸡不成蚀把米。上前道:“臣既食朝廷俸禄,自当为陛下分忧。”

