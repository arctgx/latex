\section{第15章 焰上云霄思逐寇(九)}

自韩冈领军南下之后,章惇便投入了忙碌的工作之中。

属于一州之长的事务本来就是千头万绪,而且还有经略安抚司上的责任。即便章惇可以将交接时清查账目的工作丢给下面的门客去处理,但更多军政两方面上的事务,还是得靠他来亲历亲为。

也幸好接手桂州的人是章惇,换上一个能力稍差一点的,还不知会耽搁多少事,闹出什么样的乱子来。

眼下章惇手上的最重要的一件工作就是招募新兵,以弥补之前在昆仑关因张守节而全军覆没的三千人。那三千人并不是普通的三千广西军,是从桂州、融州、柳州、昭州四州驻军挑选出来的精锐,空饷的情况比普通军额要好得多,实打实的兵力超过两千。

若是这两千兵能进入邕州城,朝廷上下、包括章惇也能安心一点,也不必刚到桂州,就让韩冈领军南下;若是他们能守住昆仑关,桂州这边也不用一夕三惊。可惜托付给了张守节那个蠢货。

如果领军的不叫张守节,而唤作张守约——换成关西赫赫有名的老将——就根本不会有全军覆没的可能。这是韩冈此前跟他说过的话。

哪个不希望领军是威名煊赫的宿将,可广西这个地方,若有一个两个能领军上阵的,当年的侬智高之乱,也不用狄武襄南下。

除此之外,要给付新军的军器辎重也得着手准备,韩冈亟需的箭矢等辎重也同样得赶紧发出去——唯有粮草倒是不用担心,宾州良田万顷,州中的粮食产量在广西排在前三,仅次于面积要大上数倍的桂州和邕州,其仓中存粮足以支撑万余大军一年的食用。

又是批了一夜的公文,章惇只是小睡片刻,就在窗外的鸟鸣声中醒了过来。在开封,冬天能听到的鸟叫就只有乌鸦和麻雀。

揉了揉又胀又痛的额头,章惇从房中走出来,偌大的庭院空荡荡的,仆役婢女只有寥寥几个,且都是他一到任有人送过来的,并不是随行南下。府中人手不足,更多的琐事还是从州里调了老兵来服侍。

不过章惇也不需要什么服侍,按照之前的约定,他很快就要领军南下了。尤其是收到韩冈南下的这几日加急发送的军情后,更不敢耽搁时间——他走得实在太快了,从逐日传回的军情中,韩冈的行程他了如指掌,让章惇不得不担心韩冈心急中会出意外。要知道,苏缄的苏子元可是跟着一起南下的。

“宾州的消息该传回来了。”章惇走到前院的公厅中,自己手上最为得力的幕僚已经坐在了里面处理文字。

伏在文案上的童迁抬起头,“论理说应该是今天,可千里迢迢,路上说不准会在哪里被耽搁了。不过要是当真不到,今天给朝廷的奏报就又难下笔了。”

章惇摇头苦笑了一下。天子让他将邕州的战事一天一上报,可昆仑关被交趾人堵上,什么消息都传不回来。每天写给天子的奏折都让他绞尽脑汁,必须有新的内容,但也不能将没影的事胡乱说。

坐下来,听候使唤的老兵奉上了茶汤和菓子。章惇吃了一点垫饥,“韩玉昆他手上的八百兵能当三五千广西军用。他抵达宾州后,昆仑关以北的象、柳数州就能安稳下来。”

“桂州同样也就安稳了。”

正说着,重重的脚步声从廊外接近,一名胥吏冲到厅门前,声音中带着狂喜,“经略,韩运使那里的消息到了,说是在宾州大捷!斩首近千!”

“宾州?!”

“斩首近千?!”

章惇与童迁惊得都跳了起来,斩首近千?这是跟几万敌军打得仗?韩冈身边的才八百人呐!

连忙让人将奏报拿来,看了韩冈在里面详述的经过,章惇也算明白了,这斩首千人的大捷究竟怎么来的。

“竟然一个都没跑掉。”章惇放下奏报,摇着头,“贼军也太贪心了,如果刘永放下掳掠来的人口立刻逃走的话,兼程而来的官军也追之不及。”

童迁点点头:“的确是贪心之故。否则官军难有如此大胜。”

“就算是运气,毕竟也是大捷!总算有个好消息,”章惇哈哈笑着,脸上泛着光彩,“也可让天子安心一点了。”

童迁没有笑意:“……就怕韩玉昆犯了同样的错。”

章惇收起了笑容。的确,大喜之后就有隐忧。

这个胜利实在太过轻易,在提振军心士气的同时,也免不了让韩冈等人有了骄横轻慢之心。加上从韩冈的军报上,还附有通过俘虏而得知的最新的邕州军情。邕州眼下危在旦夕,有苏子元在,韩冈下一个目标肯定就是昆仑关。

苏子元要急着救苏缄,就怕在他的撺掇下,韩冈太过激进。就不知道韩冈能不能压得住苏子元……不对,韩冈本来就是打算全力救援邕州,年轻气盛的,又逢大捷,说不定转头就去攻打昆仑关了。

“要给韩玉昆写信去。”章惇说着,就拿出了纸笔。

“来不及了。”童迁摇头,“在路上一来一回,中间差不多要一旬的时间,肯定是来不及了。”

“也说不准。”章惇道,“万一韩玉昆打算在宾州休整,却被苏子元连日在耳边催促动了,这封信说不定正好能镇得住。”

急急草就了一封书信,让人即刻南下送给韩冈在。等信使走出厅外,章惇这才有空闲想起要将这份捷报送往京城去,只是发了急脚递的时候,他脸上满是忧色,“希望接下来的消息不会让天子忧心忡忡。”

只是到了入夜时分,新的一份捷报又跟着来了。

“官军收复了昆仑关?!”

“苏子元连夜赶赴昆仑关,说服了守将黄金满?”

这个消息一传回来,章惇这下知道自己丢了人,但他没时间后悔前面的莽撞,心里面有着更加不妙的预感。

对韩冈来说,得到了昆仑关,他的功劳已经拿得足够多了。但苏子元立此大功,在韩冈面前说话的份量大增,说不定就能撺掇韩冈继续。而且韩冈与苏缄交好,以他的性子说不定当真会冒险。

‘这下要糟了!’章惇和童迁心中都在这么想着。

而第二天的信报,更是坐实了这一点。韩冈不仅仅是重新得到了昆仑关,同时更让黄金满攻下了交趾军驻屯的长山驿,开始向邕州挺进。

为了连续几场战事的胜利,城内城外一片欢欣鼓舞,衙门里面的官吏一个个喜笑颜开。萧条多日的酒楼重新高朋满座,人人都为着韩冈和苏子元叫好。

只有章惇坐不住了,韩冈越是高歌猛进,坏事的可能性也就越高。他等不及正在组建中的新军,立刻派出了手上仅剩的荆南军,让他们即刻南下。

只希望他们抵达的时候,昆仑关还在官军手中。

…………………………

交趾兵最恨的就是李常杰!

苏子元还在想着韩冈的话。

虽然乍听起来有些难以相信,但细细想来,这话说的并没有错。

顿兵邕州城下两个月,军中伤亡惨重,说交趾兵恨苏缄,那是当然的,但苏缄此时怕是已经战死了。长山驿和归仁铺连着两场大败,伤亡同样不少,说交趾兵恨韩冈,也肯定少不了,可恐怕他们连韩冈的名字还不知道。恨宋人,目标太广。恨黄金满,只是个蛮帅而已,都不是主力目标

倒是李常杰,已经攻破了邕州城,不让下面的人舒心畅意的劫掠;宋人的援军已经到了,但又不让他们撤退;攻打营寨不克,明明是块石头还硬要啃下去,死的伤的都是身边的人,就算赢了也没有什么好处。要说交趾兵没有怨气,这可能吗?

但没人会指望交趾军会反抗李常杰的命令。不可能兵变的,只不过消极怠工却是人之常情。

李常杰不会不清楚这一点,为了提振士气,大概是许诺只要解决归仁铺这里的官军,就返回国中,或是拿出此前劫掠而来的财物大加赏赐。

对于李常杰的想法,官军这边可没人会打算去满足。

贼军已经从邕州城中撤出来了,城内的百姓也得以逃离邕州,从时间上算,这时候当已经逃出了大半。

而且李常杰又将主力调来归仁铺这里。留下来的军队尽管可以重新冲进邕州,但李常杰能允许广源蛮军占了这个便宜?!他手下还在与官军拼命呢,后方广源军却在大发其财,不怕闹出兵变来?只能互相监视着。

不论是输是赢,交趾军已经来不及回去掠城了。

目的既然已经达到,现在官军需要做的就是撤退!

望着营中一片忙忙碌碌的身影,苏子元对韩冈叹道:“这下又要用到黄金满了。他再立了功下去,正牌子的刺史都能做了。”

“为朝廷出生入死,天子又怎么会薄待他?如果这一次再能立功,广源刺史他是当定了。”韩冈笑了一笑,“谁让他忠心耿耿呢?不提拔他又能提拔谁?”

“其实广源州上下都愿意做大宋的忠臣。朝廷的赏赐从不吝啬。前些年广源州曾经掘出一块人头大的黄金,被李日尊强行索要了过去。如果这块黄金是献给朝廷的,肯定能得回更为丰厚的赐物。当年若是允许侬智高朝贡,让他有余财安抚部众,如何会起事叛乱。这一次要不是因为刘彝做得混账事,广源州怎么可能跟着交趾人一起反叛?”苏子元的声音一下高涨起来,“广源州不动,李常杰如何能打到邕州!?”

“就是这个道理。”韩冈安慰似得拍了拍苏子元的肩膀,“这一次就是要给黄金满表示忠心的机会。”

