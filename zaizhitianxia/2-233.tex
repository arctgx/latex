\section{第34章 山云迢递若有闻(14)}

有了带路党,对于征服者来说,的确是件让人舒心顺意的快事。

尤其是瞎吴叱这样在被征服者中,有一定威望的带路党,更是。瞎吴叱虽然是被木征支持着在武胜军立足,但他的身份才是他立足武胜的根本。现在有他来出头让武胜军各家蕃部不要给禹臧家供给粮草,还让招揽他们投靠大宋。虽然一时间还没有哪家蕃部当真归附朝廷,但至少都是犹豫了起来,将提供给禹臧花麻的粮食都停了下来。

武胜军的蕃部,有不少曾经跟着董裕进攻过过去的古渭寨、如今的陇西城。但在董裕兵败身死之后,都是无意再于宋军对抗。但都因为怕宋人,日后被人清算,盼着有人先出头。现在既然瞎吴叱站了出来,而木征和禹臧花麻拥有近两倍的兵力,仍不敢攻打临洮城中宋军,看起来宋人控制武胜军也成了定局。那么投靠宋人,也没有什么好担心的了。青唐部的先例摆着,肯定比在木征或是禹臧家的控制下,要好上许多。

第一批民伕已经踏着冰雪,抵达了临洮。跟着他们一起去临洮前线的,还有大批的军用物资和粮秣,加上大批腌制过的马肉。

——韩冈最近将缴获的伤马、死马都让人处理了,把马肉一条条的分割腌制,连同内脏和骨头都一点不浪费的全数都一起变成了士兵和民伕们碗中的肉汤。

通远军最大的出产,不是粮食、不是马匹,而是盐。青唐部、纳芝临占部都是靠着盐井而撑起了家底。韩冈一开口,就一文钱不花的就从青唐部弄来了大批的粗盐。将上万斤马肉腌制后,自己留了小半,大部分都送去了临洮。

之后从临洮传回来的消息,王韶和高遵裕都挺高兴的,一点荤腥的刺激和吸引,这让士兵和民伕们会更加卖力。不过临洮那边有些得寸进尺,让韩冈设法多送一些酒水上去,尤其是他给疗养院准备的烧酒,更是直接被点名。

韩冈看到盖着缘边安抚司大印的命令后,摇头叹了口气,转手将这封命令发去了陇西——只有陇西才有烈酒。

现在在陇西主持转运工作的是王厚。在蔡曚被召去了临洮后,他乘势主管陇西转运,情况比起蔡曚插手时要好上了许多。毕竟跟韩冈一起共事许久,处断公事的手法也互相交流学习。而且王厚对手下的胥吏了如指掌,知道何人擅长何事,分派起工作来,不会浪费他们的能力。

韩冈不仅仅负责粮秣转运的工作,他现在还要主持庆平堡的修筑。从调集来的民伕总计有一万人,大半将会放在临洮城的增筑工程上,然后还有扼守临洮城南北两条道路的辅堡。

但王韶仍是设法分给了韩冈两千人,让他先把庆平堡增筑完成,继而再改建野人关。设立兵站已见事功,无论王韶和高遵裕都乐意将兵站制度保持下去,自然要加强庆平堡和野人关的守卫。

天气一日日的冷下去,而庆平堡的建筑则是一天天升起来。

韩冈远眺极西。在洮水对岸,木征始终不敢过河,而缺粮的禹臧花麻,更是干脆的派人抢劫起不再给自己提供粮草的蕃部,惹得更多蕃部开始投向大宋。

随着临洮城逐渐完工,到了那个时候,木征和禹臧花麻他们还能支持多久?

……………………

望着对岸的临洮城,木征发着怔,已经有半个时辰没有动弹上一下了。

面前的洮水并不宽阔,但水量充足,木征想过河,但他始终找不到一个合适的时机。

再过一月,等洮水彻底冻结后,他手上的兵马当能安然过河。可眼下洮水上的冰层太过薄弱,想要渡河,得靠船只或皮筏。在眼下宋人对洮水严防死守的情况下,则根本没有半点机会。

但再等一个月,宋人对临洮城的扩建恐怕就已经结束了,届时就算过了河,他也拿坚城毫无办法。

木征颓然叹了口气,只能说宋人选择的时机实在太好了,行动又太过迅快,让他来不及反应——‘不!’木征摇了摇头,其实他有时间反应的,但他当时并没有想到,他的两个弟弟会胆大妄为到跑去攻击渭源堡。不然有瞎吴叱和结吴延征牵制,以两千部众足以调遣起武胜、岷州的上百家部族,聚起两三万人马,那样的情况下,他要过河其实并不难。

‘实在太蠢了,宋人怎么可能会那么容易对付!’木征在听闻噩耗之后,已经不知是第几次在痛心疾首。这让他本是安坐钓鱼台的心思,变成了望洋兴叹。

——十丈之水犹如千里之遥。

蹄声从身后传来,周围的亲卫一齐循声望去,只见一名骑兵从西面的营地飞驰而来。那名骑兵冲到近前,跳下马,几步走上来附在木征耳边,低声说了几句。

“竟然找上门来了?”木征闻言后一阵惊讶,但他也没有耽搁,回身跳上马,皮鞭连挥,急速回营。

回到自家主帐,吩咐了从人出去将等候已久的客人请进来。很快一阵风掀开帐帘,一名年纪犹不到三十的年轻人走了进来。眼睛不大,但精悍无比。

“禹臧花麻?”木征安坐不动,抬眼望着禹臧家的年轻族长。

年轻人没有半点退让,抬了抬眉毛,反问道:“木征?”

比自己小了一辈直接叫着名字,木征微感不快,但还是示意禹臧花麻坐下来说话。

禹臧花麻大模大样的做了下来。他禹臧家能背弃本族,投靠党项人,当然不会对什么赞普血脉放在心上。

禹臧家作为吐蕃的叛逆,当年李元昊举兵入侵河湟,他们跟着党项人在这片土地上没少造杀孽,血债累累,至今未有还清。木征经历过当年的战乱,对禹臧家的现任族长没有多余的话,奉茶寒暄一概欠奉,直接问道:“禹臧花麻,你来做什么?!”

“只是想跟你说一句‘合则两利,分则两败’而已。”

“你后面不是有党项人吗?何必担心宋人?”

木征并不是在拒绝,而是要试探一下禹臧花麻的底线,同时更是要在谈判中占据主动,如果他真的还会因为当年旧恨而影响到现在的判断力,那就根本不会把禹臧花麻请进来。

“难道木征你打算一家与宋人拼杀到底,你那叔叔当是不会跟你一条心吧?”禹臧花麻直戳木征的痛处,以他的眼光,木征在战略地理上的劣势,他一目了然,“河州位置关键,是在河湟之地正中央,宋人不会放过这块地盘。而董毡的青唐王城可就不用担心了,宋人怎么都不会在灭掉党项人的时候,再分神去青海湟水那边。”

木征神色冷淡,“武胜向北就是兰州,你说宋人是先打我河州呢,还是先攻你兰州……尤其是现在董家的那一对兄妹,在兴庆府杀得血流成河的时候。”

“是,你说的没错。宋人想要攻打大夏,当然不会放过兰州。”禹臧花麻并不介意承认自己的弱点,“如果不是因为兰州位置太过重要,宋人肯定不会留给我禹臧家来控制,我投了宋人那又如何?”

“所以你来求我?”

“我不想在宋人的指挥下低头哈腰,难道木征你就很愿意?所以说我们是同病相怜!只有携起手来,与宋人对抗。”

木征在禹臧花麻的话语中听到一丝诚意,问道:“你打算怎么做?”

“正面是打不过的。”禹臧花麻眉峰微皱,“倒不是赢不了,可杀敌一千自损八百,我们也耗不过宋人,他们的人实在太多了。”

“像你之前做的那样,断宋人粮道?你现在成功过几次?”

禹臧花麻避而不答:“把武胜军让给宋人如何?如果宋人在武胜军驻守三千人马,一年要消耗三到五万石粮草,一万兵马,那就是十万到十五万石。留得兵马越多,要转运来得粮草就会越多。”禹臧家与宋人时常交战,对宋军的粮草转运,禹臧花麻有着很直观很明晰的认识,“而且要运送一石粮食到临洮,在道路上就要损耗至少两石三石的粮食,宋人即使财大气粗,又能在武胜军支撑多久?”

木征一点都不考虑的摇着头:“现在我可使唤不动武胜军的各家蕃部,有我那个不成材的弟弟帮忙,武胜诸部现在可不会听我的话。没有他们掩护,抄截宋人粮道根本不可能!”

“那就杀光他们!”禹臧花麻笑容如春风,半点不见杀气,木征回绝的这么快,其实就是证明他早就考虑过这个手段,“一家一家的杀,一部一部的灭……看看宋人会不会为他们报仇?杀光胆大的,剩下都是胆小的。”

木征眯起眼,冷声道:“禹臧,你是不是跟着党项人太久了?杀起我之族人,杀得很痛快吧?”

“营门外的几个首级那又是谁的?”禹臧花麻笑得更为开怀,反手指了指帐外,“洮水以西还有几个不听话的部族?论起下手之狠、之快,小子可是拍马不如。”

木征脸上的神情丝毫不变,只是将双手交叠在一起。他这对干干净净的一双手上……其实满是血腥!

