\section{第34章 山云迢递若有闻(11)}

【下午晚上有事,第一更提前,夜里的第二更可能会迟一点。】

自从半个时辰前,接到了从渭源传来急报,驻扎在临洮城中的宋军营地,一下变得紧张起来。

隔着洮水,隶属于木征的一万多人马,驻扎在对岸的洮水之西。而在北面的二十里外,禹臧家的数千军队,也扎下了营盘。虽然不知道他们之间有没有联系,但从他们两家所处的位置上看,这兵凌临洮的两相夹攻之势,可是明摆着的现实。

三方对峙,身处漩涡之中的宋军,却没有半点畏惧。如果禹臧和木征群起来攻,那正是他们迫切以求的乐事。反倒是眼下的对峙,才是让人头痛。王韶、高遵裕一众将帅,都在绞尽脑汁的想方设法,要把两家贼人都引来攻城。

可是就在临洮众官将,都把注意放在木征和禹臧花麻身上的时候,哪个都没料到,他们竟然还有余力,打起了渭源堡的主意。

——如果事情仅止于此,情况还算不上糟糕,渭源堡本有足够的兵力。可偏偏因为之前禹臧花麻派兵出来抄截粮道,使得韩冈不得不在连接渭源、临洮的要道上设立兵站,不但从临洮请调了接近两千人马,同时也调走了渭源堡中的大半守备力量——在韩冈传回来的急报中,已经明确的说明了堡中的守军,就算把民伕加进来也不足千人。

围绕着帐中的巨幅沙盘,帐中的气氛仿佛夏日暴雨前的空气,一时阴郁无比。

“不会是别人,只会是瞎吴叱。潜过洮西侦查的斥候回来都说,没看到他的旗号。”一名幕僚用着肯定的语气说着。

“可能还有岷州的结吴延征,从地理上看,他跟瞎吴叱在洮西汇合的可能性很高。”另一名幕僚不甘示弱,也尽力表现着自己的才智。

王韶则是死死的咬着牙关,他没能料想得到,瞎吴叱、结吴延征这两个几乎被遗忘的弱小势力,竟然有可能改变整个战局。

高遵裕盯着沙盘看了半日,突然抬头怒道,“王舜臣和赵隆呢?!怎么还没到!”

他话音未落,赵隆这时大踏步的走进帐来,身上已是结束整齐,甲胄俨然。军中定例,介胄不拜。他便只是抱拳行礼,“王安抚,高安抚,职部选锋已经整装待发,只待军令。”

王舜臣也随之走了进来,同样穿戴好了盔甲,头上的血红色的盔缨随着他沉重的步子前后舞动,“安抚,末将所部也已准备完毕,还请两位安抚下令。”

“好!”王韶点了点头,“赵隆!你率选锋,速回渭源,一路不许耽搁。到渭源后,视战况你可自行决断。”

赵隆再一拱手:“末将尊令!”

“王舜臣,你率部南向往抹邦山去,打下两处渡头,堵上贼军后路。”

王舜臣也躬身接令。

见两将都领了军令,王韶拿起了朱漆的令箭就要丢下去。

可就在这时,帐外守门的亲兵进来通报,“安抚,渭源堡又派信使来了。”

王韶脸色微变,令箭拿在手中,连忙道:“快让他进来。”

高遵裕的脸色也变了,声音都在抖着:“子纯,会不会……”

“不会!有韩玉昆在,当不至于此。他再差也能招来几百蕃兵助守,兵力不会相差太大!”王韶又紧咬起牙,渭源决不能有失。

此时帐帘一动,一名矮个矫健的军卒被领了进来。

帐中之人都盯着他,却惊讶的发现这么被领进帐来的信使,脸上竟然带着完全没有掩饰的喜色。

“什么?!大捷?”

“还斩了结吴延征?!”

“竟是那群广锐叛将?”

只听了信使的几句话,主帐中一下喧腾起来,王舜臣和赵隆都不顾尊卑,跳起来追问。

再次向信使确认了胜利的消息之后,王韶紧绷的神色放松了下来,韩冈果然不需要让人担心。他的指挥之才还是其次,其大胆任用的广锐叛将,比预计的还要出色许多,证明了韩冈眼光的出色。通远军收留他们,果然没有做错。。

王韶长吁了一口气,扭头对高遵裕舒心的笑道,“想不到广锐军竟然精悍如此。三百破两千,虽是夜袭,说起也没多少人能做到。这胆色、这武勇,真是难得……实在是可惜了。”

王韶有些为这些叛将感到遗憾,以他们表现出来的战斗力,即便除去了对将功赎罪的渴望,也是足够惊人的。即便是在西军中,也算得上是精锐了。

“谁让他们叛乱的?要不然何止于此?”高遵裕摇了摇头,“不过这事有些难办,今次他们立下的功劳可不小。”

任用曾经的叛军,只要能建功,主事者不会受到指责。但封赏起来就很头疼了,谁也不敢再重用他们为将。但赏罚不均,又肯定会惹起广锐军卒的愤怒。若是将其再行逼反,不论是谁决定的此事,他们的政敌都不会放过这个机会。

“让两府去头疼好了,我们该怎么报就怎么报。”

王韶却是毫不犹豫的把麻烦事全都推给上面,这根本不是他们该关心的事。

他霍然而起,将原来就已经拿在手中的令箭投了下去,丢在了赵隆的眼前。韩冈努力营造起来的胜势,他不可能轻易的放过,“赵隆!还是照先前计划,你率部南下,将洮水上的渡头给我堵上。蕃人残兵如果聚合起来,肯定还是要走渡头……我把选锋都交给你,决不能让他们顺利过河!”

赵隆拾起令箭,抱拳行礼:“末将接令。”

他直起腰,又是大踏步的转身出帐,带起一阵旋风。

王舜臣有些急了,连忙道:“那末将呢?”

“用不着你了!……韩玉昆手上的兵力足够。没听到吗,他把蕃人都弄来了。虽说这些蕃人都是一团散沙,但漫山遍野的捉蕃贼,倒比官军更熟练。”

王韶哈哈笑着,王舜臣失落的神色看在眼里,“禹臧花麻在北,木征在西。现在被击败的,只是瞎吴叱和结吴延征这样的弱敌,后面有的你立功的机会。”

……………………

瞎吴叱躺在草窠里,脸色蜡黄着,双眼紧闭。

他的右臂歪曲成一个可怖的角度,正常情况下,胳膊只有一处能弯折的关节,而瞎吴叱的右手上臂,却是向外弯着。捆扎伤口的麻布上,斑斑血渍正在一点点的扩大。麻布之下,还能看到一处尖锐的突起。如果对外伤稍有了解,便能看得出来,那是骨折后,穿刺出肌肉所造成的痕迹。

这是瞎吴叱从马背上摔下来后受的伤。并不是摔伤,而是踩踏。他自幼骑在马上,就算落马也能在掉落下来的一瞬间保护好自己,但面对身后冲过来的战马那就没办法了。仅是右臂被沉重的马蹄踩上,而不是头部和躯干这等要害,这已经算是佛祖保佑的好运了。

但瞎吴叱无力庆幸这样的好运,右臂受了重创,血在一夜之间流了不少,现在甚至开始发烧了。

瞎吴叱的身边,只剩下十几人,沉默着,不知该做什么为好。黑夜中的慌乱,把他们这一群亲卫全都冲散了开去。最后只有十几人护着瞎吴叱,一直把他拖到了山上。可也就到此为止,瞎吴叱的伤势使得他们行动不便,而紧追而来的宋人,又找来了此地的蕃部来搜寻逃散的部众。

一名亲兵缩着脖子,从灌木丛中向下张望着,这一段时间,他们亲眼看到了十几队附宋蕃军,在他们藏身的山坳附近扫过。十几名亲兵都是很后悔,前面一次转移的时候,不该落下了瞎吴叱的镶了宝石的头盔。这份物证,就像落在了地上的蜜糖,立刻引来了一地蚂蚁。

一阵呼叫声从下面的山坡传来,好像是有人发现了他们之前留下的痕迹。更多敌军随之聚了过来,在更大的范围中展开了搜索。

见势不妙,留下几人抵挡,两名亲兵抬起瞎吴叱就向深山里跑去。

但没走多久,他们的脚步就突然停了下来,不知何时,前方的去路,竟然已经围起了十几名附宋蕃部的部众。

盯着瞎吴叱三人,一众蕃人的眼神中尽是凶光,对于斩首和俘虏,赏赐虽有高下,也差之不远,若是为了那么一点差价,而选择了俘虏,一旦给人跑掉了,那可就折了大本。

‘还是脑袋好!’

从这二三十个蕃人的眼睛里,瞎吴叱明明白白的看到了他们的想法,在昏昏沉沉的,他厉声尖叫起来:“我是瞎吴叱!是赞普家的人!”

“瞎吴叱……”

听到瞎吴叱的身份,一众蕃人眼神中的杀意顿时全都消失了。从松赞干布传下来的赞普血脉,对吐蕃人来说,是不能随意折辱的。当然,他们也不会把瞎吴叱给放了,这关系到让他们的部族过上好几个肥年的丰厚赏赐。

用着木棍和毛毡做成了担架,把瞎吴叱给抬了出去。半日后,生擒瞎吴叱的消息传到了韩冈的耳中。正在点算斩首数目的帐中官吏,都停下了手来,紧接着就是一片欢呼声暴起。

“算他命好。”

韩冈没有主语的一句话,让随侍在侧的刘源有些摸不着头脑。不知他是在说王君万,还是在说瞎吴叱。

‘可能兼而有之吧。’

