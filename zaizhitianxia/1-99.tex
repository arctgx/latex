\section{第39章 太一宫深斜阳落(三)}

【今天第二章,求红票,收藏。】

转过身,向偏殿内里走去,庭院中的声音渐渐听不到了。路明也跟了上来,他其实还想再听着,但韩冈走了,他也自知不便单独留下。虽然本身从不承认,但他心中实则对进士已然绝望,要不然也不会领着韩冈东逛西游,就只在太一像磕个头求个心安。

韩冈走在偏殿中,迎面过来一人。其人修长挺拔,相貌亦是出奇的英俊,风流倜傥,举世无俦。韩冈近来见过的人中,王厚算得上是英俊了,王君万比王厚还强上数分,但与此人一比,可都比下去了。那人与韩冈擦肩而过,见韩冈看着他,便微笑着轻轻点头,又很自然的走了过去。

“真是难得的风流人物!”韩冈赞了一句。

“韩官人亦自不输他。”路明拍着马屁。

韩冈摇摇头,笑道:“自家事自己清楚。”

英俊青年从韩冈进偏殿的小门出去,走上廊道,坐在院中赏梅观雪饮酒赋诗的几个士子一下鼓噪起来。

大嗓门当先响起:“蔡元长,你来迟了!”

“在下看到赵正夫你留下的口信,可半点没耽搁。”

“我说的没错吧,元长他最喜游宴,听到消息就会来的。”福建口音也跟着说道。

“强抒仲,就你话多。”

“怎么不见元度?”

“七舍弟在房中读书,不肯出来。”

“是上次回去吐怕了吧?”

“说真的,你们两兄弟的脾性差得太多。元度是怕见人,怕赴宴,喝了酒水茶水回去就要吐,而你蔡元长听着要开宴,就巴巴的赶来。也不看再过几日便要入贡院了。”

“上官彦衡,这话是也坐在这里的你说的?!”

韩冈并不知道,与他擦身而过的是千古留名的蔡京,日后的蔡太师。他此时在西太一宫中的偏殿转着圈,视线在墙壁上流连。不出意料,偏殿中有着跟李广庙一样的题诗白壁,用石灰粉刷得雪白,都是让来此游玩的骚人墨客留下墨宝所用。不过西太一宫与李广庙有别的地方,是这几片墙上不仅墨迹斑斓,诗词数以千计,将整面墙的下半部都遮了去,还有好几处被一块块青纱给笼罩上,不知是因为什么缘故。

路明看见韩冈盯着一幅幅青纱,笑着解释道:“能被青纱罩上的诗词,不是出自名家之手,便是由高官显宦写下。以青纱笼之,以表尊崇之意。”他环视着殿中的四面墙,突又感叹起时光的流逝,“比起前次来时,好像被罩起的又多了许多。”

“原来如此!”韩冈点点头,走上前去,揭开离他最近的一块青纱。随即便‘咦’了一声,立定不动。

青纱之后,既非五言七言的绝句律诗,亦非可容传唱的长短句,而是两首少见的六言。字如斜风细雨,虽然不合近体,但自有一番神韵藏于其中。

“柳叶鸣蜩绿暗,荷花落月红酣,三十六陂春水,白首想见江南。”

扬州三十六陂的名气可大得很,韩冈都听说过。再看看偏殿外的鱼池,池畔枯柳、池中残荷,若在夏日来此一游,必有江南风景再现眼前。难怪此诗的作者由此心生感慨。他追忆起江南风景如信手拈来,想必在江南的时间肯定不短。

白乐天有多首《忆江南》,韩冈也是耳熟能详。他只觉得眼前的这首‘白首想见江南’,词句朴实,别无华饰,但诗情诗感,却并不逊于白居易的‘风景旧曾谙’。作者对江南风情的追忆沉凝在字里行间。让他一读之下,不胜心向往之。

‘难怪能用青纱罩上,这等水准,无论唐宋都是顶尖的。’

韩冈啧啧赞了半天,又吟起旁边的另一首,同样的六言绝句,同样的字体,当时出自同样的一人,

“三十年前此地,父兄持我东西,今日重来白首,欲寻陈迹都迷。”

吟念之声在殿中回响,一股沧海桑田物是人非的悲凉顿时涌上心头,韩冈即便再不知诗,但最基本的好坏还能作出评判。诗言情,两首六言,各二十四字。前一首感慨远游离乡,后一首悲叹旧日难再。漂泊在外多年的垂老文官的形象,便在心中鲜活起来。

韩冈摇头感慨,不愧是开封,可比李广庙里满眼的连‘到此一游’都不如的诗词强得太多了。等到他会秦州,找几个小工,弄点石灰过去,好好把李广庙的内壁刷上一遍,那等污眼的东西,还是不要留得好。

“啊!”路明突然叫了起来。

“怎么了?”心神被叫声从两首绝妙好词中惊出,韩冈转头很不高兴的问着。

却看见路明的手指着诗词最后的题款如筛糠般抖着,神色都如被雷劈过一般。

“临川王……”韩冈顺着过去一看,也差点失声叫起,但马上醒觉,声音又立刻低了下去,“……临川王安石!”

竟然是王安石的诗作!一国执政的大作,就这么写在墙壁上,被一张碧纱帐护着!

韩冈再回头仔细看着两首诗的字迹,方才没注意,但现在一看,的确是王安石的手笔。王安石性子急,所以字体都是如斜风细雨一般,而画押签名,最后的‘石’字也是随手一划,乍看上去像是个‘歹’字。韩冈在王韶那里看过了几封王安石的私信,王厚还对王安石签名画押的字体说过几个笑话,他对此印象很深。

自从来到这个时代,一说起王安石,耳中便充斥着变法变法变法,让他全然忘了,人家可是唐宋八大家之一啊!

韩冈又回过来将两首诗读了一遍,两遍,三遍,赞叹声便不绝于口。

不愧是唐宋八大家中的一员。唐宋八大家中,韩愈的地位最为特殊,在文学上,他是古文运动的先驱者。而在儒学上,他是宋学诸多流派的发轫。唐时佛道昌盛,儒学没落,而韩愈横空出世,重振儒门,广大圣教。韩冈在张载门下,同学之间但凡提到韩愈,多以韩子称之。

而王安石不比韩愈稍差,论文采,但看着两首诗就够了,何况还有‘春风又绿江南岸’和‘唯有暗香来’,论地位,比起终官吏部侍郎的韩愈,王安石此时的地位可要高得多。至于同入八大家之列的三苏、曾巩,此时远远不如王安石,只是盛有文名,这样的人,大宋开国一百多年,从来没少过。也就如今在外任官的欧阳修能跟王安石比一比。

就在墙边,横着的几张桌案上都放着笔墨。这是为了在宫祠中游逛的骚人墨客兴致起来时,能提笔就写而准备的。王安石的诗作旁,一面墙上周围尽是与他相和的六言,其中多是次韵,也就是与王安石的两首诗用着同一个韵脚。韩冈一扫而过,却没一个能入眼的。写诗是真情流露,但和诗就是凑趣了,和诗写得比原诗好的,真的很少见。

韩冈看着看着,突然有了点恶作剧的心理,他记忆中正有一首可以用一用。自己从来都不擅长诗赋,即便想剽窃,肚里也寻不到多少货,而且若是剽窃的诗词太好,反而会暴露——穷人乍富,任谁都会怀疑钱的来历——但也有的诗作,虽无华彩,朴实平易,但因为是有感而发,反而有着打动人心的力量。那样的诗词,即便自己写出来也不会惹人议论。

韩冈走到桌边,往石砚台中倒了点水,拈起墨块慢慢的磨了起来。路明站在旁边看着。他年轻时也是自负才学,兴致起时便提笔写诗,还自以为出色,费了大量时间辛辛苦苦的修改编纂起来。只是到了如今,早没了那等心情。

磨好了,韩冈拿起笔,在砚台中饱蘸了浓墨,站在白壁前。初次题壁,韩冈的心中却没有半点怯意,写的并不是自己的东西,丢脸也不怕,而且以他要写的诗句,也不至于会丢脸。抬起笔,运了运气,他便在雪白的墙上挥毫泼墨起来。

“枯藤老树昏鸦?”

首句入眼,路明便是一奇,怎么不是次韵和诗?

韩冈提笔换行,第二句随手写就,“小桥流水人家。”

路明轻轻点了点头,两句连起来一读,便有了点味道。

韩冈手笔不停,“古道西风瘦马……”

三句一出,尽管只是九个词连缀,可深秋残冬的苍凉之感已油然而起,万物凋零的西北秋冬被刻画的入木三分。路明静静的等着韩冈的最后一句。王安石的‘白首想见江南’,前三句说景,最后一句才是全诗诗眼所在,韩冈虽然不是用的其诗之韵,但诗句的结构却是一模一样,最后一句当是提振全诗的关键。

韩冈一气呵成,六个字又出现在墙上,“断肠人在天涯!”

墙壁上从右到左,竖排着写了四句。全诗写毕,韩冈退后一步,提着笔,纵览全诗。王安石的诗,韵自难相和。但韩冈模仿着同样的结构,将马致远的《天净沙》删了一句,如果不看平仄、韵脚,可以算是配合得上。

