\section{第40章 败敌逐远山林深(下)}

前日军议之后,王韶和高遵裕就带着三千名从军中挑选出来的精锐,向南往露骨山进发。

苗授坐镇河州大营,整备营地,并派人去清理河州城中的废墟。

景思立去代替姚麟,姚兕在他兄弟到来之前,就已经开始清理周边的蕃部。必须要打得他们派出人质来投降,既然已经占据了河州城,就不可能会容忍木征的余党继续安坐在河州的关键地带。

韩冈则是同王中正一起回到了熙州。

跟随着他,在珂诺堡立下大功的广锐军,此时重又分作了两部。刘源带着将校跟在韩冈身边,仿佛一支卫队,而士兵们则还是留在珂诺堡,继续负责往香子城那一段的粮秣输送。

就在狄道城的城门处,韩冈第一次见到了沈括。

名垂青史的古代科学家,此是看起来也只是个寻常的士大夫。而且可能是因为工作忙碌的原因,还有有些不修边幅的地方。韩冈没有镜子,不过想来赶了一天路的自己,也应该是同样的狼狈。

沈括与王中正见礼之后,来到了韩冈的面前。

同样的职位、同样的官品,韩冈因为资历上的差距,先一步向沈括行礼:“在下韩冈,见过沈兄。”

沈括躬身回礼,“沈括久闻韩玉昆之名,今日得见,果然一如传言之中。”

说着惯常初次相见的寒暄,韩冈对沈括的第一印象也只是平常而已。

但能够准确地把握住自己在兵站制度上的关键,将之全盘接收下来,却又在不重要却显眼的地方加以变动,在向人表明自己的能力不输同僚的同时,还能让转运之事稳定的运作。

沈括这些日子所表现出来的能力,让人没有话说。能看出韩冈调动广锐军卒的用意不难,但敢于从中拦腰斩上一半,并将叛军推上关键的地方,这个胆量也是不一般的——虽然可能是有韩冈在前,他自己仿效者的缘故在。

陪着王中正和韩冈进城,沈括问起了河州那里的情况。王韶的安排,韩冈不信沈括不清楚,但现在作为寒暄用的闲话,说说也是无妨。

但沈括的话很快就带上了一点责难,“怎么就让王经略率军翻越露骨山去了?”

王中正和韩冈的心中,同时就有些不痛快。谁说他们没有劝过?前几日就应该知道的事,现在何必多提。

韩冈出头说话,“木征非除不可,否则其人一日尚在,河州就是一日不稳。经略既然作出决断,我等则自当领命相从。”他很自信的笑了一笑,“以经略用兵之能,当能马到成功。”

“王经略当是因为有玉昆你在,所以才能放心的去追击木征。”

“在下觉得还是各司其职的好。在下身为机宜文字,职司不过是参赞、辅佐、建言而已。转运已是外务,何论领兵?!要是经略犹在,韩冈可是愿意轻松一点。”

韩冈的话似是无心,但听在沈括耳中,这也是在暗示他,在经略司中的事务上没有他插话的余地。

从眼角瞥了一下身边之人太过年轻的侧脸,沈括一下沉默起来。

韩冈陪着安静的走了两步,忽又问道:“对了,在下有一事想问一下存中兄。前日存中兄移文说临洮堡在禹臧军的攻击下,有所损伤。不知轻重如何?”

“幸无大碍,只是外墙崩塌了半壁而已……眼下当是已经修理完毕了。”沈括见韩冈有意缓和气氛,也便顺水推舟,“临洮堡那里打得很是激烈,差一点都破城了。要不是听到了河州大捷的消息,说不定都没法再守下去。”

“有姚君瑞【姚麟】在前奋战,又有存中兄在后支援,临洮堡怎么都不会有失。”

沈括看了韩冈一眼,猜测着他是不是在讥讽自己抢夺广锐叛军的事。只是在韩冈的脸上,他只看到了真心诚意的笑容。将疑惑和猜忌藏于心底,沈括叹道:“临洮战事之烈,超乎想象。城壁毁损都不说了,连姚君瑞脸上都中了一箭……也幸亏是姚君瑞,他被箭射中之后,虽然是血流披面,但仍是谈笑自若。这等定力,才将军心给镇了下来。那个领军的禹臧花麻,在蕃人中也算是难得的将才了。”

韩冈附和着点头说道:“禹臧花麻奸猾无比,又是难得的将才,他的确是不好对付!”

本来韩冈已经准备趁着禹臧花麻没有收到木征兵败的消息,设法调集大军阴他一招。若是能解决掉一部分禹臧家的精锐部队,日后肯定对攻取兰州有利。可是禹臧家的族长,耳目比想象中的要灵通许多。韩冈刚刚将珂诺堡中的驻军调出两千,就听说他已经领军北撤了。

一路走到到了县廨,进了官厅之后,三人又闲聊了一阵,不过很快就散了。除了闲人王中正之外,韩冈和沈括都是忙人,恨不得一天有三十六个时辰的那种。

到了当日午后,韩冈在翻阅公文时收到了一个消息:“包约回来了!正在外面求见。”

包约算是自备干粮的友军。熙州北部的土地和蕃部都归属于他,他当然要自己出手去拿,没有道理说要由大宋为他出兵出粮。他帮着姚麟守住了临洮堡,也算是一桩功劳。

韩冈看了看包约递进来的门贴,还有上面的写得四平八稳的名字,不由失笑。包顺、包约两兄弟,自从归顺一来,几年之中也算学到了一点官场上的规矩。

“让他进来。”

韩冈有用得到包约的地方……而且他现在也只能使唤得动包约。

就算王韶让他代管经略司中事务,但实际上的调兵指挥之权,怎么都不可能转移到韩冈手上。景思立和二姚都不会搭理他的命令,而王韶安排苗授驻守河州城,让韩冈回到熙州,也是不想出现韩冈、苗授争夺经略司话语权的事情。

韩冈现在能指挥的军力,除了包约为首的这些个蕃人,也就是受命被征召而起的民伕了。幸好广锐军还在,他们的实力远在普通的禁军之上,而身份却还是一介乡兵弓箭手。

韩冈摸了摸这几日下颌上长出的胡须,王舜臣的本部也当能算是一个。

宋朝左武右文,各个经略司中,除了经略使以外,其他的武将,都是大小相制,互不统属。钤辖、都监、都巡检之间,并没有明确的隶属关系,有的时候,甚至兵马副总管也是一样管不到下面的领军将领。王舜臣只要愿意,完全可以不理会苗授的命令。而他跟韩冈的关系,却必然会对韩冈的话言听计从。

韩冈叹了口气,他现在就只有靠着这些人,来解决可能会碰上的问题。

王韶要攻下洮州,差不多要两个月的时间。谁也说不准这段时间中,会不会有什么变故。河州、熙州、岷州,这些地方都会可能会出现问题,而兴庆府的梁氏兄妹、兰州的禹臧花麻,青唐王城的董毡,他们会不会在这段时间中再有什么行动,谁也不能拍着胸脯说没有。

还有河州的木征余党,他们正像毒蛇一样蜷在窝里,随时有可能出来咬人一口……最可怕的还是木征,要是王韶到了露骨山对面的洮州,而木征却又不知从哪里钻了出来,那乐子可就大了。

一直以来,韩冈都是跟在主帅的身后,作为副手或参谋来行事。虽说他在镇守后方时,也算是乾纲独断,挥斥方遒。但实际上,从大范围来讲,他依然还是从没有独立指挥全局的经验。

眼下王韶领军远征,高遵裕有随行而去。自己受命担起了整个熙河路的责任,韩冈顿时感到了肩头上的压力千百倍的增强。可是换个角度来说,这也是难得的经验和机会。

翻越露骨山的决定,已经向朝廷发送了过去。想必在收到王韶的奏章后,天子应该会后悔给了他便宜行事的权力。

只是怎么看都是危险的行动,但危机中的确是有成功。

邓艾冒险穿过阴平小道,攻灭蜀汉。历史上这个成功的战例,就是一个明证。

就不知王韶那里究竟顺不顺利了。

……………………

露骨山山高林密,草木深深,阳光下依然显得阴森的森林,犹如吞吃一切的怪兽。最高峰的积雪,就像是白骨一般森森然,让人一见,便浑身上下就能感到一股寒意。

三千名汉家儿郎驻足停步。王韶站在进山的道路前,仰头望着山巅。

王韶并不准备走上木征相同的一条道路。穿越这座山脉,还有其他的通道。王韶可不想走到南面的山口时,就看到了前面出现了等候已久的伏兵。

虽然木征肯定料不到他会追击,但春时翻越深山密林,已经是一件很冒险的举动,王韶并不打算为自己增添更多的危险。

没有多余的话,王韶跳下马,当先踩上湿滑泥泞的山路。

跟随着熙河经略,在当地向导的引领下,三千宋军终于踏进了数百年没有汉家甲士涉足的山岭之中。

