\section{第43章 庙堂垂衣天宇泰(九)}

从京西的驿站系统精挑细选出来的车夫望空一挥鞭,啪的一声脆响,满载着沉重的纲粮,一列有轨马车缓缓的起步,离开山阳港,向北方的另一座港口行去。

方兴目送着这列马车远去,“只望今天发的车能一路顺畅,到了山阳港,我们手上的麻烦事就能少一半了。”

李诫点头:“要是像昨天就麻烦了。”

昨天夜中,一列满载着纲粮十五里后,一段路轨不知何时被碾压错位。这列有轨马车没有提防直接碾了上去,连车带马一起从轨道上摔了下去。

车夫出了事,而负责押运的四人幸运的只受了点皮外伤。处理损坏的马车,大家都有经验,而处理损坏的轨道,也都有预案准备着。

在方城轨道的中段,设有一个维修点,一人解开一匹没有受伤的挽马,架上自带鞍鞯,就赶过去报警。又有一人返回原路,在百步外的路边的立木上,悬起了从上到下一串五盏灯笼,这是事先预定好的告急信号,让后车看见之后能紧急停车。剩下的两人一个救助车夫,另一个则拿起了弓箭,紧张得提防起黑暗中可能会出现的敌人。

负责在维修点值夜的官员,先向山阳和山阴两港派了人去通报,接着派出三名工匠,一个骑着马、两个赶着车,带着十几个士兵,赶到路轨损坏地点。靠着灯笼和火把的微光,紧张的投入了维修工作之中。等到他们将损毁的轨道修好的时候,天色都已经蒙蒙亮了,整条轨道中断了有两个时辰。

方兴叹了口气:“要不是预案做得好,夜里必定会有个大乱子。”

“凡事预则立,不预则废。”李诫说着,“有了事先编订的预案,事情处理起来也方便了许多。”

“其实还是经验少的缘故,多来两次就不会这么手忙脚乱了。”方兴笑道,“毕竟轨道问世不过数载,现在能安排好已经是很不容易了。”

“现在轨道比起春天翻浆的官道好多了。开封往北去的那些官道,冬天冻得跟铁铸的一样,可春天一放暖,看着好端端的大道,车轮过去就是一条沟,还冒着泥浆水,修都没法儿修。拖到了夏天,路上全是一条条水沟,积水能有一尺深,里面一群群蝌蚪,还蹦跶着青蛙、蛤蟆。还有路上那一个个冒出来的泥浆坑,虽说看着浅,但正要踏上去,保不准能将头顶都淹了。”

李诫拿着轨道做对比,抱怨了一通北方的官道,方兴微笑的听着。等到李诫话声听了,他凑近了一点。

“没听说吗?”方兴偏偏头,低声问道。

“听说什么?”李诫一头雾水,没头没脑的问话让人觉得莫名其妙。

“襄州的事。”方兴左右看看,发现周围的吏员、随从,看着模样都是专心的做着事,却都朝着自个儿这边竖起了耳朵。

拉着李诫走到僻静的地方,方兴轻声的将自己听到的传言说给了李诫听。

“种痘,说笑吧?”李诫听了之后,就哈哈的笑说着,“这种流言根本就信不得。贝州王则起事前还有降妖伏魔的名头,还是弥勒佛,最后就是千刀万剐。这肯定是有人故意传出来骗愚夫愚妇的,岂能信以为真。”

共事了近半年,李诫与方兴多多少少的也有了份交情在,说话也少些避忌。襄汉漕运功成在望,旧时在家中被近亲戚里都小觑,可如今李诫依靠一己之长,已经快要得到让人称羡的回报。现在在韩冈幕中,没有了开始时的谨小慎微,倒是越来越挥洒自如了。

方兴却是没笑,“如果是平白无故传出来的话,倒是可以不放在心上。可你也不想想坐镇襄州的是哪一位?”

“当真是龙图?……”李诫心中充满了疑惑,皱着眉头,“怎么连个信都没传出来?这么大的事,龙图好歹也给通知你我一声,也方便你我做出应对。”

方兴其实也是纳闷不已:“说起来我俩都在唐州这里坐着,但邵彦明【邵清】、田诚伯【田腴】那边就住在漕司衙门里,怎么连个气都不通?要是当真有这回事,他们再怎么样也托人送条口信来。”

“不是说他们受了龙图的托,在编什么《三字经》吗?”李诫抱怨着,“都多少日子了,到现在都还没有成书。”

方兴摇着头:“虽说是蒙书,但好歹挂个‘经’字,做得差了,可是惹人笑。邵清、田腴岂会愿意遗人笑柄?再说了,他们都是出身横渠门墙,但名气不大,学问也不是那么的出众,要想将气学的塞进蒙书中,头悬梁锥刺股都是在所难免,哪有心思顾及其余?”

“说他们也没用,各有各的差事要忙。”李诫将话题扯回了流言上:“如果此事确凿无疑,而且的确能有效用,龙图在朝中的地位可就是没人能动摇了。”

“是啊,到时候不管龙图愿不愿意承认,这药王弟子的身份肯定是洗不脱了。”方兴笑了笑,跟着却板起了脸,“其实这件事对龙图而言,即是好事,也是坏事。”

“坏事?”李诫皱眉道:“怎么可能是坏事?种痘法一出,龙图的子孙可就能安享富贵,世世受到崇敬。”

“正是这个原因!”方兴一百桌子,提声叫道,“龙图的功劳够多了。要不是年龄的问题,做宰相都绰绰有余。现在多一个种痘,又能挣来什么?以龙图在民间的声望,早已经是世所传扬的星宿下凡了。再得了人心,别说药王弟子了,他要转过头来做药王都能做的。你想想,天子能不担心?”

“宰相肚里都能撑船了,官家岂会如此小肚鸡肠?”李诫反驳了两句,看着方兴摇头暗笑的表情,就顿了一下。想了想,换了更合理的理由:“官家才两个儿子,有了种痘之术,至少不用担心痘疮了。保佑皇嗣,这是天大的功劳。”

“所以我才会说,这是好事也是坏事。祸兮,福之所倚,福兮,祸之所伏……”方兴话说到一半,突然摇头自嘲而笑:“其实现在说得也是多了。不抓到兔子,光烧水也做不了饭。整件事还没个眉目,我们就在这里胡思乱想的,至少等到事情确定之后才说不迟。”

李诫也笑了。不过一条谣言而已,两人争得口沫横飞,一点意义都没有。“等着看好了,到底是真是假,应该很快就能见分晓了。”

方兴、李诫都是忙人,也没有太多时间闲聊,分了手后,各自去做正事。到了黄昏的时候,两人才又重新面对面的坐在一起吃饭,顺便要总结、商议一下今天和明天的工作。

两人刚坐下来,带了一摞籍簿正要说话,方兴的从人却敲门进来了,“管勾,龙图的信。”

“襄州来信了?”方兴神色一动,立刻摊开了手。

随从手上拿着两封信,递给了方兴一封,另一封则一伸手,递到了李诫的面前。

“给我的?”李诫疑惑着,接了过来,落款也是韩冈。

两人将信拆开,飞快的浏览了一遍。除了鼓励和褒奖两人在漕运之事上付出的辛劳,剩下的说的就是有关种痘的事项。

疑惑得到了解释,谣言得到了证实。韩冈整件事的来龙去脉都在信上说了一通,尽管早有了心理准备,之前又讨论得激烈,但方兴和李诫真正从韩干手上得到确认之后,还是惊异不已。

两人看看自己收到的信中没有什么私密的内容,便又互相交换了看了,两封信内容都差不多,说得几乎都是同样的事,韩冈没有厚此薄彼。

李诫心潮起伏,脸上是激动地红晕:“连同从叔伯家的兄弟姊妹,小弟这一辈中,在痘疮下的夭折就有四人。如果龙图种痘之术当真能见奇效……”李诫忽然抿紧了嘴,眼睛用力眨着。过了片刻,放声道:“这是泽被苍生啊!……”

“的确是泽被苍生,但问题比估计的更严重了。唉……”收起信,方兴却是摇头叹了一口气,他在兴奋之后,却陷入了忧虑当中,“真想不通龙图究竟是打得什么主意。龙图既然有此术在手,为何不及早报与朝廷,就是孙真人传下的种痘法,也不是没有变通的办法。拖了十年才献上去,天子会怎么看?要是我,要么一开始就献上去,要么干脆就不献了,或是献上去后,不要说是十年前得到的方子,只说最近在医学上略有所得。怎么能将这等会惹怒天子的详情和盘托出。”

“龙图不敢掩故人旧德,也不敢谎言欺君,以诚事上,是我等之表率。”李诫挠着下巴,“而且龙图仁心爱人,怎会愿意眼睁睁见着有人因自己而死,所以才没有将旧法献上去。”

方兴摇着头:“天子统御万邦,愿意一死示忠心的数不胜数,哪里找不出人来制熟苗?龙图这件事,可是做得岔了。”

