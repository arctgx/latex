\section{第七章 都中久居何日去(四)}

不知是谁在天子面前给自己上眼药,韩冈有些恼怒的想着。

这等言辞,以种谔的政治智慧都不会说出口,说了就是跟他韩冈结仇了。

什么陕西军中都盼着他去做随军转运,镇守后路。攻取横山的事八字还没有一撇,哪里来的人说。还不是有人暗中使坏,只要一个不好,就会惹起天子的忌惮,最少也会留下个恶劣的印象,现在种个种子,日后碰到合适的时机就会发芽了。

幸好自己之前在熙河路的定位是辅助者,只在转运和伤病救护上做文章,而上京后在的两个职位上的行事,更是加深了这一点的印象。如果自己是亲自统帅过大军,而不是零散的几次镇守后路的战斗,那再受众军拥戴,可就是很麻烦了。

“敢问这话是谁说的?”韩冈脸都板起来了,这种要命的事面前,他不介意放开自己心头的怒意。

“延州走马。”

王安石报了个出乎意料的答案,韩冈听了就是一怔。

“不要想太多了,玉昆,”王雱笑道,“你在鄜延路军中的名声可是好得很。”

韩冈点头回以一笑,可心中仍难以释然。不论是真心还是假意,军队对自己的好感被报上来决不是好事。而且身在朝堂,如何让人不能多想。延州走马……回去倒要查查他的底,看看究竟是怎么一回事。

“玉昆,你是不是不看好这一次的战事?”

在军事上,新党一边其实拿不出多少人才,王韶根本与王安石不是一条心,进了枢密院后,与新党的联系就只剩韩冈了。章惇也只是在荆南耀武扬威一番,靠得还是西军的将领为核心。说起来,也就韩冈有过对垒西贼和吐蕃蕃部的经验。而且他当初说横山不能成事,竟也当真失败了。

“如今的形势,比起熙宁三年四年的时候,已是强出百倍。无论将卒、军械,皆是远胜旧日。西贼则是日渐衰弱。当年就只是功亏一篑,如今要不是与契丹联姻,西贼也就是如釜中游鱼,只待王师扫平。”

“哦!”王安石面现喜色,“当真能胜?”

韩冈摇了摇头:“战事从来都是说不准的。无论事前做了多少准备,拥有多少把握,一点小小的失误,就能全赔进去。不过……”他又笑了一笑,大宋真正具有压倒性优势的并不是军力啊,“以西贼的国力其实完全无法与中国抗衡,只要将帅不贪功,步步为营,逐步进逼,就算一时无法取胜,西贼也会支持不下去的。”

“原来如此。”王安石点点头,又笑道,“世说玉昆你用兵沉稳,果然没错。”

韩冈没有笑:“只是唯一让人担心的就是契丹啊!辽主和魏王乙辛会给我们多少时间?”

赵顼当真能抵挡得住辽人的压力吗?韩冈抱着深深的疑问。

……………………

西夏到底会不会亡?

韩冈从相府中回来后,就一直在思考此事。

现在是熙宁八年,而不是熙宁三年。由于自己的存在,历史的发展已偏离了他所知道的方向。这一次的横山攻略究竟能不能成功,韩冈无法再如几年前那样确定。如果自己参与其中,尽力襄助的话,很有可能会见证历史——韩冈对自己有着足够的信心,无论是判断还是能力。

不过他对这个答案的追求,并不算迫切。离着开战还有不短的时间,也还不到他离开军器监的时候。

大宋要开始的一场战争,绝不是上面的天子宰相拍拍脑袋,下一份诏书就可以。尤其是面对西夏这等拥有数十万兵马的万乘之国,正常情况下,都要有着至少半年的筹划期,用来确定统帅将领、筹备粮秣军资、点集兵马器械,否则谁也不敢轻言出战。

——当然这只是对宋人而言如此。对于西夏、辽国来说,钱粮二物只要攻入宋境,要多少就有多少,所要耗费的仅仅是派信使传令和集结军队的时间而已。

现在陕西宣抚司还没有设立,主帅人选也没有定下,想要观兵横山,还有很长的一段时间,这也是世人共通的判断。

不过事情的发展,完全出乎世人以及韩冈的意料。接下来的发展并不是陕西宣抚司成立,而是朝廷降诏,将泾州知州毋沆任命为延州知州,原任延州知州赵禼则是转调庆州,兼任环庆路经略使。

“不会吧?!是不是听错了。”在军器监中听到这个消息的时候,韩冈第一反应是怀疑起消息的真实性。

“舍人,小人亲耳听到的,绝不会有错。”来报信的韩孝在韩冈面前赌咒发誓。

“是吗,那做得不错。”韩冈挥挥手,示意韩孝下去。

在只剩一人的公厅中,韩冈的手指无意识的敲着桌案,‘看来是没有宣抚使了。’

毋沆曾经是赵禼前一任的延州知州,只是他当时仅仅是过渡,做了一个月就被赵禼替掉了。另外他是吕大防儿女亲家的这一件事,韩冈也曾有听闻。

毋沆曾经担任过陕西转运副使,能力也是有的。但他在军事上的才能,世间却没有多少传说。如今他竟然卷土重来,顶掉了赵禼,这个任命只证明了一件事,就是朝廷不希望有人给种谔对鄜延军的指挥,而毋沆唯一的价值就是凭着过往的经验,做好种谔的后勤工作。

但鄜延路绝不可能以一路之力对抗西夏,鄜延路终究还是需要隔邻的环庆路和河东路帮助。没有更高层的协调,怎么让两路在合适的时机出手,而不是争功诿过、拖延战机?

是不是为了欺骗西夏人故意放出来的幌子?韩冈不禁这么猜想。

就像长平之战,白起为秦军主帅的消息一直被隐藏到赵军覆灭之时。要不然这个里里外外都笼罩着让人疑惑的迷雾的任命,怎么会通过政事堂和枢密院的?

但这个猜测完全不可能,大宋不是秦国,朝堂上的事没有这么玩的,那个漏勺一般的崇政殿,哪里能将秘密守住。

“这下玉昆你不就可以不用去延州了?”当天晚上,韩冈与王雱见面的时候,王雱就这么笑着跟韩冈说道。

韩冈抽了一下嘴角,算是在笑:“说得也是。”

他韩玉昆已经是第二任知州资序,让他给宣抚使打下手没问题,一任宣抚判官可是能与路分监司中的转运使、提刑使一较高下的职位。但给经略使打下手,难道还让他去做机宜文字?

被毋沆顶掉的前任延州知州——赵禼赵公才,他熙宁四年权发遣知延州的时候,本官是正七品右司谏,贴职是直龙图阁,熙宁五年本官晋升,跟韩冈现在一样。让他去给毋沆做副手,朝廷也不会开这等玩笑。

“要不是玉昆你年纪太少,其实延州知州你也能权发遣一下。”

若当真以他为延州知州,那必然要立陕西宣抚司了,就跟当年赵禼兼任宣抚判官和延州知州一样,否则区区一个七品文臣如何镇得住当地的武将。韩冈摇头叹道:“资望差得太多,我可压不住种五,还有那一干骄兵悍将。”

王雱哈哈大笑:“玉昆你可是在说胡话了,文武之间哪有比资望的?就是种谔桀骜不驯,你有天子之命在身,指派他行事,难道他还敢不从?”

这是此时的通病,韩冈也不与王雱争。笑道:“不过在文臣中,小弟也是没法儿与人比辈份的。”

“愚兄也还不是一样?”

如今朝堂上进士出身的臣僚按辈分来算,文彦博、富弼、张方平,加上最近重病不起的韩琦这些六七十岁的老臣算是一辈,皆是在仁宗中期崭露头角,后期执掌朝政,到了如今,早都是说话掷地有声的元老重臣了。但他们也已经是老的老、退的退、死的死,很快就要退出历史舞台。

接下来,在庆历、皇佑【1040前后】年间进入官场的王安石、王珪、冯京、吴充、司马光这一拨人,则又是一辈。五十上下的他们,陆陆续续占据了朝堂上的最高位置,如今新旧两党的争锋,就是以他们为核心而展开。

再往下,嘉佑年间【1056前后】入官场的算是现在的第三代,其中吕惠卿走得最高,下面的曾布、章惇、苏轼、苏辙,乃至张载、程颢都属于这一辈。高的能做到参知政事和御史中丞,运气不好的,还在选人中打转,但大部分都进入了京朝官一级,是中低层官员的中坚。

最后就是在英宗和当今天子的这几年得中进士的官员,有前途,但还没有足够的表现,只能期待日后。至于韩冈,实乃异数。比他早一科的,与他同一科的,绝大多数还在选海中沉浮,不知要到何时才能五削圆满、得以转官。当然,坐在韩冈对面的王雱,也是另外一个异数。

与王雱坐在一起聊了一个晚上,这个任命究竟是什么用意,韩冈也从王雱那里了解到了,让种谔能统管全局的另一个目的,是为了能堵上辽国的嘴。

“终究还是要顾及辽人。如果北朝遣使质问起来,也好说一点。……仅是边地之争,没看到只动了鄜延路一家嘛?”王雱的笑容中藏着浓浓的讽刺。

韩冈摇头苦笑。从酒楼中出来,与王雱道别后骑上马向家里走,银河横跨深蓝色的天幕,千万颗星辰如宝石一般闪耀璀璨。

这时一道流星划破北方的天际,在许多人的眼中留下一道光影。为韩冈牵着马的韩孝在前面咕哝道:“不知又是哪里死人了。”

“胡说八道。”韩冈笑骂着。

不过第二天夜里,一个消息撼动了整个东京城

——相州韩琦薨逝!

