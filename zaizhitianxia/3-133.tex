\section{第35章 甘霖润万事(中)}

韩冈就算远在京城,但他依然关心着熙河之事,毕竟他的根基在那里。

如今在熙河路实行的方略,基本上就是王韶和他一手制定,从大体规划到施行细则,无处没有韩冈的一番心血浸透其中。

韩冈也知道,任何一名愿意去熙河路任职的官员,都是有着一番雄心壮志,决不会甘心被遮掩在前任的阴影下。肯定会千方百计的要做出一番事业,以彰显自己的才干,博取不世之功,不让王韶专美于前,此事韩冈能够理解。但若是有人恣意妄为,为求功绩,破坏了如今熙河路的安定局势,他则绝对不会放过。

只是王韶、韩冈在京中,高遵裕最近又调往泾原路任兵马总管,当年的熙河路的几位主官,只有苗授升任了正六品的横行官——西上閣门使后,做着河州知州一职。

虎豹离山,新搬来的猴子有些想法不足为奇。

经略使蔡延庆为人沉稳,老于宦事,他能收得住手,耐得下性子从小处着手,若是换作当年的秦州知州沈起,必然大刀阔斧,设法挑起事端。去年年中,沈起自请调往广西桂州。这段时间,连韩冈都听说了他在桂州教训士卒,整备战船,磨刀霍霍的不知要拿谁开刀——多半是交趾——没有任何战略上的考量,为求一己之功而妄开边衅,韩冈也只能庆幸他去祸害交趾人了。

不过在熙河路,蔡延庆之下还有一干人不甘寂寞,希望能弄出些事来让他们立功。王韶今天担心的就是这些人。

枢密院有王韶坐镇,要处置有关熙河路的军情事务,吴充都绕不过他去。故而王韶拜托韩冈做的,就是不要让中书门下这边出漏子。尤其是改变已经卓有成效的制度,更是要从根子上直接给断掉。

“枢密放心,韩冈回去后,便向家岳陈述利害,不让人坏了熙河路的大好局面。”韩冈向着王韶做着保证。

王韶点了点头,喜道:“只要中书能持之如一,熙河路中也翻不起浪来。”

王舜臣、赵隆现在是熙河路的中坚将领,各自分镇一方,王厚月前从狄道知县任上直接转了熙州兵马都监,坐镇熙河路的中枢,随时可以支援岷州或是河州。军中的下层将校,当年亦无不是在王韶麾下听命,而各州县的吏员和底层官员,也同样是与当年熙河路的几位主官都能拉上关系。

只要这些根基还掌握在手上,京城两府又支持路中稳定,熙河路的主官不论换了谁来做,王韶和韩冈都能稳得住阵脚。

一番酒后,看着雨势渐小,韩冈就借了王韶府上的马车,径直往王安石府上过去。而王韶也有事要做。今天既然下了那么大的雨,化解了几分旱情,他升为执政中的一员,肯定要入宫拜贺,也差不多是时候了。

坐着摇摇晃晃的马车上,一路上尽是百姓的欢呼声,冒着雨,就在大街上拍手叫着。

听着外面的声音,韩冈心中也被感染上了一分欣喜。只是冷静下来后,又开始想着要如何说服自己的岳父。

以王安石的性格,他在治政上,不会顾念什么翁婿之情。但在延和殿廷对之后,他欠了自己一个大人情不说,连自己在新党中的发言权也是水涨船高。一旦说起熙河之事,相信王安石不会也不能忽略自己的意见。不论是谁想要在熙河路设牧马监,韩冈都能让他收了歪心思去。

抵达相府时,天色已晚,而雨势则已稍歇。韩冈径自进了府中,就只有王雱在。韩冈一问,才知道他的岳父果然也跟王韶一样去了宫中,先贺今日之雨,而后再奏请天子明日照旧例,至大庆殿贺生辰。

这个生日,赵顼原本是不准备过的。大旱当头,还耗费民脂民膏的为己庆寿,不但不能彰显朝廷声威,反而会让入贺的万邦使节看轻了,也少不得朝臣和民众的议论。可偏巧赶在生日的前一天下了雨,上天有了吉兆,王安石当然要领头上表,明日依旧例在大庆殿为天子贺寿。

进宫上表要耽搁些时间,韩冈坐下来等着王安石回来。

听了韩冈的来意,王雱便道:“既然玉昆你说熙河牧监不当行,那就是不当行,难道大人还能不相信你?”

王雱的回答不出意料,韩冈笑道:“怎么也要向岳父陈述一番。”

“玉昆你就是想得太多……对了。”王雱像是想起了什么,“有一件事要问问玉昆你。”

“何事?”

“不知玉昆你觉得浚川杷是否堪用?”王雱问着。

“浚川杷?”韩冈模模糊糊的似乎在哪里听说过,只是一下想不起来。

王雱见到韩冈对此不甚了了,忙找出了一份公文来,上面还附着很粗糙的草图。

韩冈看着看着,就皱起眉头来。

所谓的浚川杷,就是一个巨大的铁耙子。因为黄河淤积泥沙之故,有人向王安石献策,打造巨大的铁耙,挂在船后在河底扒泥,将河底淤积起来的泥沙扒松了,然后让水冲走。这样河床就不会一年年的抬高。

王雱盯着韩冈的神色变化,问着:“玉昆,你看此物如何?”

通过雪橇车一物,加上霹雳砲,放大镜等发明,韩冈在机关巧器方面已经是权威。王雱要问一问他的意见,而韩冈的回答是摇头:“此事断不可为!”

“为何?”王雱诧异的问道,“此事已经有了成例。”

成例?!

韩冈终于想起自己什么时候听说过此事了,就是去年方兴当笑话说起的,提举大名府界金堤范子渊——也就是治河的大臣——在黄河分流的二股河上,征发了几十艘船,在河上来回拖着一个大耙子,说是要松土浚河。

这根本是笑话,希合上意的人太多了,王安石既然喜欢浚川杷,下面自然敢不顾事实的来附和。

其实没有实际见到疏浚河流的场面,说此事不可行,不是正确的做法。但韩冈可以确定,没有流传到后世的治河手段,多半就是不可行的。

韩冈组织了一下言辞,反问着王雱:“敢问元泽,关中亦有黄河,为何不见长安要年年增高堤坝?”

“当是水势缓急不同,泥沙不沉之故。”这个道理王雱很清楚,“浚川杷的用处就是扒松河底泥沙,让水流将之带入海中。”

“此乃缘木求鱼。黄河之水,一碗水半碗泥,到了秋时,更是八分沙两分水。今天将泥土掘松,明天就能再淤积上。难道要日日施行不成?这要耗费多少人工?!”

“那玉昆你有何良策?!”

黄河治水的故事韩冈听得太多了,后世行之有效的方案他也能粗浅得说个大概,现在终于有机会在他人面前提起,“很简单。顺势而为之。既然黄河之水能将泥沙带来,也能将泥沙带走——也就是束水攻沙!”

“束水攻沙?”

“大堤之内再筑一堤,强行让黄河水流加速,是泥沙不得淤积。而河水被内堤拘束,自然要深切河槽。河槽日深,也就相当于大堤日高,常此以往,河堤自然稳固。纵有洪水来袭,也是在内堤之中流淌,而且还会冲刷去更多的泥沙。就算洪水过大,以至于漫过河槽,外侧还有外堤括起的宽滩来分洪。到时候,泥沙在宽滩上淤积,也就相当于增加了内堤的高度。”

韩冈此言别出一格,又随手拿过笔墨纸张画着剖面图,让王雱为之深思。

见着王雱皱眉思索,韩冈更进一步说道:“设内外双堤,堤防可固。堤防既固,则水不泛滥而自然归于河槽。河水既归于河槽,则不能上溢必然下刷。沙之所以涤,渠之所以深,河之所以导而入海,皆相因而至。”

一直以来,治理黄河水患所用的方案都是分水势,通过将洪水分流而减缓水势,使得黄河不至于破堤。

“但分水愈众,水势愈缓。水势愈缓,泥沙则沉积愈多。泥沙沉积愈多,则河槽愈髙。一年年反复累积,到了如今就已经变成屋上行舟。如此治水,只会越来越难,而黄河破堤也会越来越频繁。”

王雱想了一阵,觉得韩冈说得极是有理,但又不敢就此点头,却道:“这还要让愚兄多想一想,也得跟父亲商量一下。”

韩冈笑道:“其实这仅是治标之术,泥沙大半入海之后,日积月累,也有沧海桑田之虞,到时候,说不定河水还会因海潮而倒灌回来。”

“治本呢?”

声音从身后传来。

韩冈和王雱惊得一下起身。方才一个说、一个听,都聚精会神,竟没有注意到王安石什么时候回来了。

王安石做了下来:“玉昆,你继续说,治本的方法是什么?”

“只要让黄河不再携带泥沙就可以了。黄河水一清,河槽就不会年年上涨,而是不断的冲刷下陷,自是不会再泛滥。而黄河水中泥沙,皆来自于关中、关西。再往上,则终年清澈如泉。究其故,还是关中关西的不毛土山太多,一有雨水,便泥水同下,汇入河中。若是山上有草木覆盖,山间流水便会清澈许多。”

水土流失的道理,其实不要韩冈说,这个时代的对水利稍有关心的士人,都能有个朦朦胧胧的意识在。韩冈只是这么一说,王安石父子便都点起头来。

“关西、关中两地皆是黄土堆积成山,欲使山上有草木覆盖,非积百年之功不可为。而近日小婿曾听闻,朝中有人提议,从熙河路伐木以治宫室。此事万万不可。如今熙河路草木丰茂,河水泥沙量少。若是山中树木采伐一空,河中泥沙便会加倍增多,届时黄河必然更加难治。”

