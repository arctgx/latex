\section{第22章 明道华觜崖(四)}

伽利略设计出来的逻辑链,简单的只有三个环节,但却牢固得如同钢铁打造的一般。

没有人能提出异议,即便是杨绘,一时间也组织不起反驳的词句——想要胡搅蛮缠,弄混了水,也得先把前前后后想得通透。

而赵顼没有给他时间,想了一阵,觉得真是这番推论当真挑不出错来,“果然是这个道理。”点了点头,对韩冈笑道,“这格物之说,果然有些意思。”

天子出言,不但确认了韩冈的胜利,更将格物摆上了台面。

韩冈低头谢过天子的赞许。可虽然胜了,他倒也没有得意洋洋起来。他是站在巨人的肩膀上而已,并不代表自己的高度。对于那等在无尽的迷雾中,只靠着自己的双手而开辟出一条光辉道路的伟人,韩冈只有抬起头,高山仰止的资格。

而韩冈这种平静,却让天子更提高了对他的评价——宠辱不惊,从来都是一个难得的优点。

不过话说回来,赢了就是赢了。眼下的胜利,为韩冈接下来的要做的事,奠定了坚实的基础,可谓是鸠占鹊巢的第一步终于走了出来,将科学包装上儒学的外皮,终于有了点成果。

虽然孔子若是能看到格物致知竟被如此解释,肯定要大摇其头,叱骂不已。但二程开创的理学,张载开创的关学,也与孔子的原始儒学根本不是一回事。反正都是一偏,偏向科学一边,韩冈觉得对华夏的未来更有好处。

赵顼拍着栏杆,迎着大道对面金明池上吹来的风,慢慢的点头赞着:“张载说虚空即气,想不到他已经将气和风精研到这般地步。气动成风,物落受风,最后不受风而轻重落速皆同,这么些事,他竟然都通过格物给格出来了。”

韩冈摇头,虽然有些冒犯,但他必须更正赵顼的错误观点:“启禀陛下,并非如此。”

“……那是什么?”

“虚空之气,乃是元始之气,万物之本源,无形无质,万物由其凝成。而气动成风的气,却是万物之一。虽说看不见摸不着,但呼吸皆气也,风一吹,很容易就能感触得到。”

其实气学最大的麻烦就在这里。张载说的虚空即气,指的是万物本源。但这个气,却又与空气重复了。同样一个字,却分别有两种不同的定义,解释起来很是麻烦。

赵顼听着也觉得麻烦,半懂不懂的,只能点头而已。

“臣有一事,要请教韩冈。”杨绘这时忽然开口。

在一旁得空,杨绘终于想出了一点破绽。虽然他打不断那条牢如精钢的逻辑链条,却能指出前面实验中不对劲的地方。他可不愿自认失败,就算死到临头,也要挣扎一下。

赵顼看看杨绘,想了一想,点头示意韩冈上前。虽然杨绘今天丢了大脸,连带着翰林学士都没了光彩,但皇帝不会不给自己的侍臣面子。

韩冈便道:“还请学士指教。”

杨绘走上前来,盯着韩冈:“前面你曾说过,鸿毛所以落得慢,乃是迎面受风的缘故。想那堵门石下落时也受风,所以慢了下来。而秤砣有石块在前面挡着风,却会落得快了,就压着堵门石上。”

“学士的意思是说,迎面受风被阻的堵门石,比起不受风的秤砣落得要慢?”韩冈问道。

“正是!”好不容易捉到的破绽,杨绘乃是一口咬住。

“这风可真大!三十斤的石头就能吹飞起来。”韩冈笑了,杨绘已经是黔驴技穷,下面可就是要穷追猛打,“学士可是想岔了。须知迎风面越大,受的力就越大。”

他视线移转,对人群之后,手上拢着一把折扇的余中道,“状元郎,可否借一下扇子一用。”

余中先是一愣,然后立刻上前来,先对天子一礼,又将扇子递给韩冈,“当然可以。”

韩冈眼下大占上风,当然是结好的时候。

韩冈接过扇子,手便是一沉。余中的扇子,扇面是纯白重绢,正面一幅泼墨山水,山水神秀凝于方寸之间,一看就是名家所作。背后题了一首小词,龙飞凤舞,亦是佳作。扇坠是指节大小的羊脂玉,而扇骨则是乌檀木。余中竟然用上这样精美的折扇,韩冈需要再确定一下:“有些太贵重了。”

余中知道韩冈要用扇子来做个验证,很洒脱的道:“但用无妨。”

“多谢!”

余中退后,就见着韩冈将扇子平展开来,一松手,就轻飘飘的落下去。又将其捡起来竖着一放手,啪嗒一声落地。两次下落,扇子都是展开的,但姿态一变,下落速度登时就变了。

这个小实验一做,旁听的都明白了韩冈的意思。同样一柄折扇,都是张开的,但迎风面不同,下落的速度也不一样。

只听韩冈道:“堵门石比起秤砣要大得多,受得风阻也大得多。当真两物分开来后,只要能同时放下。敢问学士,这结果会怎样?”

杨绘瞠目结舌,按韩冈的说法,甚至可能秤砣比石块落得还快!

赵顼也吃惊起来,原本从直觉上,他觉得越重的东西下落越快,后来从韩冈这来知道应是一样快,这还勉强能接受。可现在好了,反而是轻的下落得更快!

韩冈这是明着欺负杨绘。他说的其实是刻意忽略石头和秤砣本身的质量差距,而将问题不知不觉的转移到单纯的迎风面积和阻力的问题上。

如果杨绘能想透其中的破绽,韩冈可以趁机将质量和力之间关系的定义举出来,加上前面的实验和理论,牛顿力学第二定律就可正式在公开场合亮相。可惜杨绘却没有,沉湎于诗词歌赋和经义文章,在科学方面的思考能力还是不够水准,很容易就被弄糊涂了。

韩冈也没有失望。一次灌输的太多,反而会出问题。简简单单的一个实验,加上一条完美的思辨论述,已经足以将杨绘等反对者打得溃不成军。事理皆在眼前,容不得他嘴硬。

至于之后的事,可以慢慢来。总有聪明人能想透,到时再反驳,便可以将格物物理上的争论一波波的炒热!

这才是正道,比起去横渠书院单纯扯着勉强糅合起来的理论,而被吕大临问得近乎张口结舌的情况,接受了教训的韩冈,终于找到正确的路子。

现在韩冈要趁热打铁,将风阻的存在和影响,用事实再次确认:“正如人骑马,骑的越快,迎面而来的风就越大。这是就要低头弯腰,缩小迎风面,以减小阻力。”

“林管勾,还请再找两个一样的秤砣来。”

林深河前面押错了宝,现在哪敢再违抗韩冈,不移时,便让人找了两个一模一样的秤砣来。

韩冈没动手,只是让人拿出一块丝巾,将四个角用细绳扎在其中一个秤砣顶部的孔洞上,绑成了降落伞的形状。

在天子的目光中,韩冈将两个秤砣同时丢下去。其中一个扑通一声落水。另一个则是靠绸巾兜着风,慢悠悠的落下去了。

等着天子的视线从降落伞上转回来,韩冈道:“绑了绸巾的秤砣比起另一个秤砣还要重一点,但就是落得慢。同样的实验,千万人都可以做,就是学士亦可以私下里做,都能得到同样的结论:落物的速度与轻重无关,只与阻力大小有关。”

他说话间不忘带一下杨绘,提醒人们,翰林学士杨元素的赌帐尚没有还。

“不知此有何用。”杨绘冷然问着,他本来打定主意不开口了,防着继续丢脸,但韩冈挑衅似的带上他一句。他却不能继续做哑巴了:“就算知道是落物速度与轻重无关,只与阻力多少有关。敢问此一条又有何用?”

在没有亚里士多德的两千年权威压制,这个实验的意义,当然不如伽利略如同惊雷一般劈开中世纪的迷雾那般振聋发聩。但只要引起天子的兴趣,就已经够了。一旦赵顼对此有了兴趣,而要把格物致知的理论塞进经义局的新编教材中,便容易了许多。

——毕竟韩冈还有三棱镜分光实验,帕斯卡的木桶实验,甚至用来表现大气压力的虹吸管等一系列实验没有出手呢。初中物理上,看似简单的一系列实验,却是多少大智慧者才智的结晶,韩冈若是拿出来,嘴硬如杨绘的会问一句有什么用,但更多人却会去思考其中的原理。

不过现在杨绘就在眼前,还是要应付他一下。

韩冈的回答并不是语言,而是动作。似是无奈的摇了摇头,看杨绘的眼光也是居高临下,如同大人看着赌气不肯服输的小孩子。叹了一口气:“既然学士如此说,那也就罢了,权当无此赌注好了。”

“你!”

杨绘一下气得脸皮发紫,眼睛都红了。韩冈若跟他争辩,总有破绽出来。但韩冈竟然不肯辩论!

韩冈才不会跟杨绘斗嘴。姿态越高,杨绘就越丢脸。文人之间的争辩,尤其是这等已经是一方赌上一口气的意气之争,就像是后世论坛上争吵,根本不可能说服对方。只要设法能逼着对方失态,那就是赢了。

杨绘耍了赖皮,这对韩冈更是好事,真的在天子面前斗起气来,还会被咬上一个尊卑不分的罪名。而现在,丢尽脸的只有杨绘一个。

赵顼对杨绘的态度也有些不满,好歹也是翰林学士,怎么能这般无赖?可也因为杨绘是翰林学士,朝廷重臣的脸面要给他留着。

“今日之事已了,朕先行回宫。今次的琼林宴,也别拖得太久!”说着便自顾自的下去了。

恭送过天子圣驾,韩冈看看杨绘,又看看曾布。天子既然让琼林宴不要拖得太久,当然也就得听着,应该算是告一段落了。天子提前结束琼林宴,可见是不想让杨绘继续受气。但他离开时并不拉着杨绘,更可知并没有太过照顾杨绘的想法。

看着脸色一阵发青,一幅快要吐血模样的杨元素。韩冈知道,他今天的胜利乃是实实在在。想来张载,离着经义局已经不远了。

