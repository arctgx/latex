\section{第43章 庙堂垂衣天宇泰(15)}

钱乙是翰林医官,是世所公认的专治小儿科的名医。在京东行医数十年,声名达与京畿,才会被天子使人招入京城。

当今天子的儿子生一个死一个,英宗在位时间太短也就算了,但仁宗也是一般,自真宗朝后,没有一个皇嗣在宫中出生并养大,这根本就不是病症的问题了。钱乙做了多少年儿科名医,豪门富户走得多了,兄弟阋墙的戏码看得也多了。正常的情况下,哪有可能几十年来一个劲的死儿子?

传说宫中阴气深重,有历代无数生不出儿子的嫔妃郁郁而终后出来作祟。在钱乙看来,作祟的情况有,但绝不是与什么鬼神之说有关。另外天子本身体质就虚弱,偏好的女性有多是身轻如燕的类型,生出来的子嗣身体能好就有鬼了。

钱乙最怕的就是遇上体质虚弱的幼儿,太容易生病,而且治不好。胎里带出来的病,根本就不是药石能根治的。若是遇上了疾疫,身体健康的幼子能保住性命,但体质虚弱的根本撑不住。建国公的痘疮,就是最好的例子,才下了两贴药,施了一回针,就过不用再麻烦他了。接着,就听说了种痘法。

今天一大早,宫中都在传说整个御史台大半都在上书弹劾献上了种痘法的韩龙图。

当然不是以种痘为名,有的说京西转运司的账目有错,耗用钱粮过多;有的则说韩冈本人贪渎,家中在熙河路有田三百顷;有的说韩冈在广西借势牟利;还有的说韩冈所学不正,更有的说韩冈欺世盗名。基本上就是痛打落水狗,趁着天子深恨韩冈的机会,踩一下让他们又羡又妒的龙图阁学士。

私下里的说法,无论人痘还是牛痘,既然学到手了,都该献上去,决定种痘法用于不用的当是天子,韩冈有什么资格代天子决定?!

这叫什么?

活脱脱的升米恩斗米仇的小人嘴脸。

明明已经是子孙受惠,却还要说为什么不早点来。出身民间的钱乙实在是不能习惯这样一切以自我为中心的想法。他也曾经遇上过这样有理说不通的病家,吃过同样的苦。

韩冈通过十年的时间,将一个要死人的旧方,改成更为优良的新方。作为一位名医,钱乙当然知道找到一个对症的药方有多难。种痘之法闻所未闻,全无先例可循,要改进更是难上加难,从韩冈的奏章中,钱乙看到了千辛万苦的汗水,韩冈在广西为国事忙里忙外,还要分心医道。其中的用心之处,可不是几行字就能描述的出来的。

在天子明显带着猜疑之心来询问时,钱乙选择了不带倾向的公正回复:“药王孙真人的《千金要方》和《千金翼方》,微臣旧年熟读多遍,前几天更曾特意翻看过,都没有发现能确定是种痘法的条目。”

“难道不是从孙思邈哪里学来的?”

听到天子疑惑的声音,钱乙低头:“此非臣能所知。”

钱乙还能说什么?

论理他是该多谢韩冈的。他小小一个翰林医官,还是因为天家屡丧皇嗣才被召入京城,但第一次出手,就没能救回建国公。虽然因为是痘疮,不大可能会被治罪,不过已经是很尴尬了。而韩冈因为种痘法成为众矢之的,引开了世人的注意,无形中帮了他钱乙的大忙。

加上对韩冈如今的境遇的同情,以及同仇敌忾之心,钱乙当然想帮韩冈多说两句好话。但他只是翰林医官而已,不是翰林学士。属于翰林院,而不是翰林学士院,两个字的差别,决定了两边地位的截然不同。

不能说的再多了。

三十不到的龙图阁学士太受人嫉妒,此前由于天子的关照,能一直被保护着,不受嫉恨所扰,但现在圣眷不再,又有谁能阻挡韩冈受到攻击?恐怕天子也是快慰于心。

钱乙不敢再想下去了,这样的想法,对天子太过不敬。将堂堂一国之君想得小肚鸡肠,总归不太好。

钱乙腹诽着,但绝不敢宣之于口,帮人可以,但将自己搭进去可就不好了。

至于韩冈,钱乙爱莫能助,只能看他的运气了。那根本不是区区翰林医官所能涉足的领域。

……………………

韩冈不是未卜先知的半仙,当然不可能预料得到皇七子建国公正好赶在自己的奏章前一天,因痘疮而病卒。

当韩冈收到这个消息的时候,愣了好半天,动荡起伏的心绪最后化为一抹苦笑,出现了在十年宦海沉浮已经变得温和惇厚的面容上。

这事是不是该叫做谋事在人、成事在天?

韩冈想了想,觉得这个比喻并不是很确切,世事难料四个字倒是更贴切点。

往死里得罪了天子这件事,怎么都是没料到的。

韩冈也是父亲,如果自家的六个儿女中哪一个出了事,他肯定绝对不会原谅有能力相助,偏偏却没有出手的人。反倒是自己出事,倒还能一笑了之。

再冷静的人,关系到亲生的子女,也会将理智抛到九霄云外。何况如今的皇帝子嗣艰难,十一二个子女,只有三个活下来,而现在更只剩两人了。虽然道理上自己做得并没有错,但恨一个人,从来都不可能抱着客观的态度。

幸好自己现在是在京西,离得远了,赵顼的恨意一时还传不过来。而过些日子,应该就能冷静下来了。

转运司衙门的偏厅中,韩冈的幕僚们失魂落魄。他们跟随在韩冈身边,大多数都是看在大宋最年轻的学士光辉灿烂的前途上,眼见着襄汉漕运功成,又献上了种痘之术,本想着能一人得道鸡犬升天,谁成想一个惹,这下子功劳成了罪过,

可以想象,远在唐州的李诫和方兴听到这个消息后,将六十万石纲粮成功运抵京城的喜悦最多也只能剩下一成半成。

失望之下的抱怨自然免不了:“要是龙图早年能将人痘献上去就好了。”

韩冈叹了口气,到也不怪他们:“为什么你们会相信人痘之术一定管用?那是因为有牛痘作证明。还有我之前在陕西、在广西立下的微末之功为凭证。若是随便一名陌生的路人出来说他有种痘免疫之术,敢问诸位是相信他,并将此术献与天子,用人命来换取他口中的痘苗;还是完全不信,将他打出去?有几人会选择前者,几人选择后者?

十年之前,我将种痘之术献上去,有没有人相信还是两说——区区一个选人,不过是跌打损伤上有点手段罢了,谁会信?换作是我也不会相信啊——即便是信了,又会怎样?给皇子皇女的痘苗,肯定要多种上几轮,而且为了保险起见,最好还是通过小儿来制作。”

韩冈一扫脸色发白的一众幕僚,笑容冰冷,“这个世上,竖刁易牙也是有的,多半会有人愿意拿着自己的儿孙来换取富贵。但他们做下的事,罪孽也少不了我一份,哪里能忍心。”

有人信了三分,但还有人眼中闪着狐疑。

韩冈进一步道:“学而不思则罔,思而不学则殆。不知诸位在求学过程中,师长的教导是信之无疑,完全不去思考?还是仔细揣摩,将之领会通透,并加以验证?……肯定是后者吧?

可要用人命来换的人痘之术,又是毫无来由,有几人能坚持验证到最后一个轮回?过去也不见先例,与其说是医术,被人说是巫术的可能性更大。只要开了头,死了人,巫蛊之术的罪名可就让人想栽就栽,那是要掉脑袋的。”

“只要隐去得授人痘,说成是在广西发现的免疫牛痘不就行了?”

“那与欺世盗名何异?!种痘之术我只是改进而已,岂能据之为己功?”韩冈板下脸厉声喝问。

几个幕僚被叱问得面面相觑,哪有这种说法。

他们都不是十几二十岁的毛头小子,早看透了世情,哪里会相信这种场面话。能坐到韩冈这个位置上的,怎么会有如此幼稚的想法。这可是只用了十年就从一介布衣杀到龙图阁学士的异数。

“再说,为臣者又岂可欺君?”

韩冈义正辞严,堵的人无话可说。可从不欺君的臣子,不是一百个里都出不了一个,而是一百年都不一定出一个。

高居庙堂之上的皇帝出宫的机会一年也没几次,对外的信息掌握全都得靠奏章和密报,哪个做臣子的会在奏章中老老实实的说真相、只说真相、说全部的真相?总会有点偏向、有所取舍。就如如包拯包孝肃,也没少说危言耸听的话。

但谁敢光明正大的说不可欺君这句话是放屁?

一众幕僚依然愁眉苦脸。韩冈的自辩的确有道理,但一个丧子的父亲,还有宫中的皇后,七皇子的母亲刑婉仪,以及过去子女因痘疮而夭折的嫔妃,他们在悲恸下,是不会讲道理的。

“你们放心,”韩冈却在笑,“天子是明君。”

