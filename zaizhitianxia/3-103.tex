\section{第29章 百虑救灾伤(八)}

东京城外,靠着汴河边上的镇子,其实也是一等一的繁华之地。车船脚店,逆旅客舍,各色的商铺鳞次栉比,不啻万家,人来人往并不逊于城内多少。

往年的这个时候,都是各家各户出来采办年货的高峰,不仅附近的百姓蜂拥而来,就连住在城中的人们,也因为城外的物价便宜而出城来采购。可如今两个月的大旱,带动了物价高涨,没有余钱的人们哪有出来逛街的心思,年节前的气氛半点也无。

一座原本位于河上虹桥边,每天都是热热闹闹的小酒馆,也是生意大落。如今虽然有客人上门,但点都是最便宜的酒菜,用着满腹牢骚充当祝酒辞,弄得酒馆中的气氛阴郁无比。

“这年月,真真是让人没法儿过了!”一个中年汉子小小的喝了碗中的半口酒,带着酒意哀叹着。

邻桌的一个瘦瘦的后生咚的放下碗,怒意冲天:“就是王相公弄个幺蛾子的新法,才惹来了如今的大灾。天灾倒也罢了,怎么连常平仓都舍不得开?真要等着粮价高了再卖吗?还让不让人活了!?”

“阿弥陀佛,天灾人祸。”坐在门边,一个僧人也跟着长叹。光光的头皮泛着青光,短短的发茬有一两分长。

一直没精打采的掌柜在柜台后抬起头来,问着和尚:“师傅,前几天河西的李家员外不是刚给你捐了三十斤香油吗,你还叹个什么气?”

“阿弥陀佛。”那僧人双手合十:“和尚不能光喝油,也要吃饭的。”

中年汉子听了就道:“要是俺也能多喝点香油,饭倒也可以少吃两口了。”

“可是油也贵了!”掌柜唉声叹气起来,“才两个月的功夫,涨了一倍还带个拐弯。灯都点不起,菜上也放不起油了。下次师傅你来店里,也顺便带点油过来。”

“难怪这两天菜这么难吃……”中年汉子丢下了筷子,“连酒都没有滋味,到底掺了多少水?!”

掌柜听着一下急了:“天地良心!俺出来做生意几十年了,从来没在酒菜上克扣过半点……”

正说着,门前人影一晃,一人突然咕咚一声撞进门来,却是在门槛上绊了一下,滚着进来的。

“这不是李四吗?”中年汉子低着头,看着地上的滚地葫芦:“怎么慌成这样?是不是要躲你家的婆娘?”

瘦高的后生也认识来人,带着促狭的笑容道:“四哥放心,等四嫂过来的时候,我们不会说你在这边的,只说你去找东门下的小春红了!”

“说你娘的胡话呢!”被人拿着自己把柄打趣,李四骂骂咧咧的从地上爬起来,大声道:“河上有车!有马车在汴河上走!”

先是一瞬间的静场,然后哄堂大笑在小酒馆爆发出来。瘦高的年轻后生捂着肚皮,用力敲着桌子哈哈大笑:“四哥,你这才叫说胡话!”

李四急了:“骗你们作甚?几十辆车在冰上跑着呢……”

“阿弥陀佛。”僧人又是合掌低头,口宣佛号:“车非车,马非马,李施主,一切皆是梦幻泡影……”

“施你娘的主,和尚,我没钱给你骗!”李四又骂了一句,对着店中众人发急道:“这是真的!说谎的死全家!”

仿佛就是在为李四作证,小酒馆的门外一群人向着汴河的方向跑了过去,隐隐约约还传来‘马车’‘赶车’什么的。

中年汉子和瘦高后生对视一眼,就跟着李四从小酒馆中跑了出去,与方才的那群人一起蜂拥上了虹桥。僧人看看一下没了人的小酒馆,则摸摸光头,抓着念珠也跟着出去了。

这几位都是老主顾,掌柜不怕他们跑了,吩咐了跑堂的小子看店,也便出门看个热闹。他往虹桥上走,心中还有些纳闷:

汴河不是黄河。车马在冬天踏冰过黄河不奇怪,但马车在有桥的汴河上跑是从来没有过的……还几十辆?汴河上的桥有百十座呢!一辆车能分上两座三座,还别提汴河两边的大堤,比黄河的河堤可要陡多了,马车怎么下去?

酒馆掌柜挂着疑惑,一路上了虹桥。

一座木头搭起拱桥弯弯如虹,横跨在宽阔的汴河之上。这就是汴河在东京这一段上最为有名的虹桥。为了跨越汴河,而不影响河中带着帆的船只,汴河上的桥梁都是建成了拱桥的式样,越近东京城,拱桥的式样就越特别。坐船沿着汴河北上,只要看到一桥如虹,就该知道东京城到了。

宽达数丈的桥面两侧现在挤满了人,河道两边的大堤上,也聚集了一片观众,差不多上千人都在短短的时间内聚集了起来,低头看着河面上。

双目一扫,掌柜找到了他的几个客人,从他们那边挤了进去,向下一望,当真就看见一辆马车从桥下掠过,转眼往北去了。很快,就又是一辆过去。

酒馆掌柜在汴河边开店几十年,见过的马车也多了。但今天在河面上跑的这些马车的形制,他却从来没有见识过。拖着车子的只有三匹马——不,掌柜发现刚刚由过去的一辆,两边拉车的竟是骡子,只有中间是马——而载货的车斗竟然多达五节,如同蜈蚣一般拖在后面。马车车斗都没有轮子,只在下面装了两根狭长的木条。木条在两头翘起,长长的露了出来。

“这叫什么车?”掌柜身边,瘦高的后生低声的自言自语。

没人能回答他。

不时的,还有这样的一列列马车从南边驶过来,一路往富国仓而去。绝大多数都是拖了五节车斗在后面。每一节车斗上米袋高高堆起。这样的车斗载货就算不多,但四五节加起来,至少也有百来石了。

“这样的一列车怕不有上百石。”中年汉子将掌柜心里话说了出来。

“你没看到那一辆。”李四指着正在远去的一列车,“看到没有,竟然船都拖上来了!”

掌柜和中年汉子顺着李四的手指定睛一看,登时都吃了一惊。拖在那辆马车车后的根本就不是车斗。

一列列马车已经过去了不少,掌柜也能看得出,拖在挽马后的车斗只是临时拼凑起来的。并不完全一样,有大有小、有宽有窄,式样五花八门,与整齐划一的纲船截然不同。不过李四指的那一列车拖在第一节的车斗,却也实在太过特别,竟然是由船改造的。只是在普通小船下面架了支脚,钉了长长的两根木条。

掌柜和中年汉子目瞪口呆:“竟然船也上来了。”

李四现在在飞快的掐着手指。口中念念有词,他是在算着这冰上马车的运力。作为码头上的工头,冬天有了活计,那可是好事。但究竟有多少活,当然要算上一算。

一列车大约一百石。而在他上桥的这段时间就已经过去了七八辆。如果今天都是如此,算起来一天差不多能有两百列粮车抵京。那就是两万石。

一天两万,十天二十万,一个月那就是六十万石了。而正常一年六百万石的纲运,分到二月到十月的九个月中,平均一个月也不过六十多万石的样子。虽然说汴河的运力,朝廷的纲船只占了其中的一小半,大部分还是给民船占着。可冬天汴河冰上的运力,能有通航时一小半,就已经是让人目瞪口呆的一件事了。

“一个冬天,运上来百万石也不过等闲啊。”掌柜也算了出来,同样张着嘴合不拢。

中年汉子啧啧称叹:“可比太平车强多了,用太平车一个冬天绝对拉不了百万石上京。更别说用来拉车的牲畜就少了许多,路上的耗费还少。”

北方多见的太平车,能载五六千斤,是一等一的大货车。不过这等货车,要十几匹牛马牲畜来拉着,而且不只是吊在前面,车后面还要栓两匹,下坡时用来反着拉,省得一下冲下坡去。

瘦高后生摇头反驳道:“水面上可比路上要平得多,太平车上来后,也能少用不少牲口。”

中年汉子嗤笑着:“太平车怎么拖?也不看看冰上有多滑!车轮在地面上滚得顺,可在冰上能滚得起来?肯定是四面打滑!”

瘦高后生辩不过中年汉子,皱眉不解:“这些车子没轮子,不易向两边打滑也就算了,可那些挽马怎么在冰上走的这么稳当?”

这时从堤岸上围观的人群众,一个年轻人被挤了下去。双脚刚刚踩到冰面上,就咚的一下栽了个大跟头。后脑勺着地,要不是带着皮帽子,脑壳都能瘪掉一块。

汴河河面上的冰层有多滑,这下所有人都看在了眼里,故而也更加疑惑起来,“想想马蹄才多大,又是硬梆梆的容易打滑。人都跌倒了,可那一匹匹挽马怎么一点也不滑脚?”

“想那么多做什么?这就是雪橇车,王相公当真从南面将运粮食上来了!”掌柜这是终于记起前两天听过的消息。双手合掌,与身边的和尚一起阿弥陀佛、阿弥陀佛的反复念着,“这一下子,粮价可是要大跌了!”

