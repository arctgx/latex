\section{第15章 自是功成藏剑履(八)}

虽然之前在城门口就已经确认了任务多半已然失败,但在进入府衙,童贯终究还是抱着一丝希望。只是韩冈的一句话,让童贯的心情直入谷底。

“龙图的奏表上得好快。”童贯脸上挤出的笑容,比哭还难看。

“黄门此言何意?”韩冈一幅什么都不知道的样子问道。

童贯见状无奈,只能挑明了说:“之前龙图历次捷报,斩首两万余。此事本是国家之福,不过虑及杀伤太多,有伤天和,官家心伤太皇太后上仙,故有是诏。”

韩冈站起身:“还望黄门转奉天子,天子仁德爱民,出于天性,臣韩冈未能体察圣心,实是羞愧难当。不过如今查明并未与辽人勾结的黑山党项二十七部六千六百余人皆已安顿下来。请陛下放心。”

童贯估计这就是韩冈在回奏中的话,也就是顺口而已,给犯了错的天子一个脸面。也不知道天子看到韩冈的奏报会是什么样的反应,反正从臣子嘴里说出这样的话,就是他这个崇政殿中服侍天子的黄门,也见得多了,天子恐怕更是看得心中生厌。

但童贯此来不过是传递天子的心意,韩冈说什么都不是他该评价的:“小人回去后,当会将龙图之言禀明天子。”

“那就劳烦黄门了。”韩冈重又坐下来,“黄门此来只为此事?”

童贯的话已经说得很直白了,就是为挽回之前皇帝错误的密诏所造成的后果。而且多半是没有截下之前发出来的这封密诏,才不得不登门造访。就不知道天子还有没有其他吩咐。

“……”童贯愣了一下。

“不知黄门出京前,天子可有其他喻示?”韩冈更直接的问道。

童贯陷入了沉默,心中狐疑。韩冈的问题当真就是表面的意思?

韩冈军政两事皆有长才,入居两府之后的表现应当不会比任何人逊色。但提拔他若是会有损朝纲,天子也不会觉得浪费这个人才有什么关系。

童贯很清楚天子的心意,两府的职位是为了辅佐天子治国,不是给功臣的赏赐。只要韩冈的年纪问题会引起后患,就不可能让韩冈担任枢密副使。

不过韩冈晋身两府依然是迟早之事,这一次不行,过几年他过而立,到了韩忠献公当年晋身两府的年纪,也就没有如今这般惹人顾忌了。

反观自己,天子交托的任务没能完成,对于一名品位还不算高的内侍来说,是灾难一般的结果。可不比那些高品的文臣,犯了事,过两年就能回来。在宫中,可没人会给第二次机会。他的师傅李宪恐怕也会干脆了当的放弃他——毕竟只是徒弟,而不是养子。

从今往后,一辈子最多也只能在针线、大小金之类的宫苑作坊中打转,最后去敇建的道观或佛寺终老。这对于一心想追求更高位置的童贯来说,不啻是生不如死的噩耗。

但是,如果有韩冈这个和王中正、李宪都有交情的重臣助言,情况却是会变成两样。

“龙图……”童贯舔了舔嘴唇,喉咙有些发干。

“黄门先喝口茶。”韩冈微笑着,“这是炒青的山茶,口味有别龙团,却也不算很差。”

……………………

形势大逆转,御史台中的乌鸦们为自己的前途担忧,最近安静了不少,这让所有除言官以外的臣僚都觉得很是舒心,章惇也不例外。

打发了张商英派来送信的家人回去,章惇冷笑一声,却把刚刚收到的信丢到了一边去。他可不想理会那个只会坏事的家伙。

拥有风闻奏事之权的御史台论奏不实虽不需受什么责罚,但如果被弹劾的臣子反击,逼天子做出个选择,下场却也不会太好。自来弹劾宰相执政失败的言官,多半会驱逐出朝堂,虽说过几年就能回来,往往还能升职,不过比起能成功将宰执弹劾的那些御史,如韩琦那般,肯定是远远不如了。

张商英当年弹劾枢密院中不法之事,并要求枢密院听从政事堂的调遣,一下子就捅了马蜂窝,惹得枢密院诸辅臣同时缴了印,要天子给个说法。这样的情况是不论对错的,张商英因而出外,甚至被贬为一个监酒税的小官。

张商英是章惇举荐起来的,几年前急功好利,将枢密院整个得罪,给王安石带来了巨大的麻烦。如今章惇想方设法又将他从监酒税的位置上拉回来,不成想他又根本不通报,参与上表弹劾韩冈,想在韩冈身上挣回之前耽搁的时间。

章惇心底里对此很是愤怒。他不求张商英能听从自己的命令,但也不要添乱才是。当年在平定荆南时,结识了仅仅是个小官的张商英,因为其口才和识见让人激赏,所以才加以推荐。谁想到却是个坑人的货色,早知道就让他在酒糟里打一辈子滚好了。

“枢密。”

一名家丁走到书房门外,敲门进来。

“什么事?”

“通进银台司消息,河东那边又有奏表到了。”

章惇神色一动,追问道:“奏报的内容是什么?”

那名家丁摇摇头:“听说是实封的密奏,不知道里面到底写的是什么。”

“……看来天子的密诏没有来得及追回。”章惇低声自语,挥手让家丁出去。

河东的奏表在时间上很是让人奇怪。不过应该是韩冈对之前密诏的回复,否则就不该在这个时候通过上奏表,而且是密封起来的实封状。

章惇越来越看不懂韩冈的行事了,他的目的究竟是什么?!着实让人费解。

……………………

赵顼没有想到童贯竟然没有追上,而且还让韩冈提前一步上了请罪的奏表。

之前所谓的密诏并没有瞒着人,韩冈的密奏又如何能瞒得过去世人的耳目?这一次在世人面前,他可就是扮演了一个糊涂皇帝的角色。

赵顼一直以为韩冈是辅政利国的能臣,日后的宰相之才,但没想到他也是个越来越棘手的麻烦,早知道就不让他去河东了。

赵顼面无表情的看着韩冈的请罪书。

上面甚至连辩解也没有几句,基本上是密诏上怎么说,他就怎么回复。不过文采焕然,应当不是韩冈本人的手笔。当年韩冈在殿试时的文章,赵顼还记着,那个完全是地方官对当地政事的奏报。

不过这篇文字写得漂亮,反倒让赵顼看的上火。要是韩冈亲笔所写的那种,那还能见到真心。幕僚代笔,自己誊抄一遍,怎么看都是在应付故事。

赵顼现在面临的问题就是论功行赏。尽管给辽人做了渔翁,在天下人面前丢人现眼,但夺下来的土地依然可以算是一个胜利。朝廷需要为这个胜利付出的报酬,也是远远超过之前任何一次战争。韩冈的问题只是其中一部分而已。只不过因为御史台将事情闹得太大,才成为众人关注的焦点。

不过相对于士兵的赏赐,将帅们的功赏其实不需要太头疼,只要能拉下脸来,赖账也没什么关系。而底下的士卒若不能给出让人满意的功赏,那些个赤佬可就是会立刻翻脸闹事——还是人数多寡的关系。

当年太宗攻克太原,灭亡北汉,之后便挥兵直取辽国南京道,就是因为功赏不至,以至于在燕京城下功亏一篑,惨败于高粱河畔。

而太祖时,曹彬领军攻克南唐。开战前,太祖皇帝承诺的功赏是使相——节度使兼枢密使。不过等曹彬得胜归来,太祖给出的赏赐则是五十万钱——五百贯。

太祖皇帝过河拆桥的行事手段是否合乎人情暂且放到一边,开疆辟土的奖赏最低能到哪一步,也算是有了一个依据。

以曹彬为标准,韩冈的功劳实在算不得什么。不过一个不毛之地的胜州,户口不及南唐千分之一,土地也只有百一之数。如果是开国之时,以曹彬为标准,即是往高里算,五贯十贯也就能打发了。

当然,赵顼不可能这么苛待功臣。开国时的手段,不可能使用在如今。但即便不能用在如今,可有了这一条旧例打底,权柄过重的职位完全可以拒绝授予。

还有当年狄青平侬智高,回朝后被晋为枢密使。本来有许多朝臣援引曹彬旧例,来否决这项任命。但仁宗皇帝坚持授予这个职位。可成为文臣眼中钉的狄青最后的结果却是让人叹息。

有正反两条先例,赵顼想要做事就方便了许多。只要韩冈不能入西府,种谔也不可能有机会,王中正更是可以随随便便就打发掉。

赵顼叹了一口气。既想维护朝纲,又想维持一个公正慷慨的名声,天底下哪有那样的好事。还是舍了点面皮方便做事。

给立有功勋的将帅多些金银财帛,升几个无关紧要的职位,也就打发了。

不为执政,本官升到谏议大夫就到顶了。韩冈的差遣和本官都没有晋升的余地。不过其他用来搪塞的名号多得是。勋、位、爵、食邑、检校官、馆阁贴职,应有尽有,想要找个打发人的名号,实在再容易不过。

给他什么职位呢?赵顼想着。

按理说可以先给韩冈一个枢密副使的任命,等他照常例拒绝之后,直接转封其他职位。但赵顼还真怕韩冈会不顾颜面的一口应下,就像当年王安石,一直不肯入朝,当自家任命他为翰林学士时,却是一口应承下来。

翁婿两人都是有心做事的人,不是那等沽名钓誉,喜好故示清高之辈。赵顼可无法保证韩冈会不会依常例拒绝诏命……还是直接点好。

