\section{第三章 时移机转关百虑(一)}

腊月三十的这一天,空气中到处都是一股浓浓的硫磺味。

鞭炮声响彻云霄,从腊月二十三开始,一天比一天更为响亮。到了除夕,更是不绝于耳,自清晨一直响到了午后。

韩府的后花园中,韩冈三个大一点的儿女踏着雪,在地上乱跑。前一日刚刚结束的一场暴雪,厚厚的积了有一尺深。前面的院落都已经清干净了,只是后花园却没有让人去清理。

几个小孩子又叫又笑的乱跑一气,互相砸着雪球,园中的积雪被踩得一片碎玉乱琼。

而三个小一点的,也已经能下地走路了,跌跌撞撞的在雪地里爬几步走几步,周围一圈的乳母、丫鬟围着,拍着手引着他们走。

周南抱着小手炉坐在凉亭中,亭内点着两个火炉,石凳上铺着羊皮垫,倒是不见一点寒气。猩红的斗篷紧紧裹着身子,在领口上镶的一圈厚厚的上等狐皮,毛茸茸的狐裘掩着变得稍稍圆润的面颊,笑看着孩子们在雪地中的玩闹。

“小心一点。”韩云娘吩咐着服侍的使女婆子,“把哥儿姐儿都盯好了,别让他们往雪地里扑。指不定雪下面是什么。”

一个个都恭声应诺。

周南捂着嘴笑道:“云娘也大了,就是两年前还是会一起闹呢。”

严素心已经做了今天除夕宴的准备,陪着周南坐在亭中,说道:“等玩过后要让他们好好洗个热水澡,喝点驱寒的热汤,把寒气给散掉。”

周南叹起起来:“都跟皮猴子一样,几个哥儿倒也罢了,金娘再这样下去怎么得了!”

“官人都常说,小孩儿跑跑跳跳是好事。病恹恹的才头疼呢。”严素心朝东侧的一栋小楼努努嘴,“姐姐前两天去宫中随班探问太皇太后病情,之后就被朱贤妃给拉着问了好一通育儿经。均国公就是种了痘,还是一样让人担心。”

后花园中唯一的一座小楼里,孩子们的欢叫传了进来。

小楼原名小琼楼,不过韩冈感觉着恶俗,连同被起名做听雨阁的池畔水阁的匾额,被韩冈一起丢进了后院角落里,置放杂物的房间。两栋建筑,韩冈却连新的名字也懒得起,干脆就空在哪里。

府中的正屋正在重修中,后花园的小楼就成了韩冈暂时的落脚地,书房也移到了此处。

听到窗外笑声,韩冈也在欣慰的笑着,说着和严素心一样的话:“有精神是好事,病恹恹的可就糟了。”

“官人,今年的收支还听不听了!”坐在韩冈对面,王旖心浮气躁起来。

自来都是男主外女主内,家里的吃穿用度等日常开销全是王旖领着周南她们三人在管。年终是关账的时候,虽然韩冈没有要求,但王旖总是会将一年的家计收支,拿着账本一笔笔的向韩冈说上一番。

可韩冈很不耐烦听这些。一边翻阅着沈括刚刚送来的一部笔记——这是受了韩冈的影响而出现的新书——一边喝着温过的甜米酒,躺在白木靠椅上,很是闲适。懒洋洋的几乎要打哈欠:“你看着就是了,也不是什么大数目。”

进入腊月之后,王旖身上的事情就多了起来,置办年货、新衣,还要准备送人的年礼,安排家中仆役。抽着空余的间隙,辛辛苦苦的好不容易将账本一式两份的誊写好,韩冈却是一幅无心多问的表情。

王旖本来就累得够呛,再看着韩冈懒怠的模样气就不打一处来:“官人,奴家是妇人,眼界窄,不比官人在衙门中,眼里过的数字全是几十万、上百万。家里一年一万七八千贯的花用,可不敢说‘不是什么大数目’!”

见到妻子生气了,韩冈将酒盏和书都放下,欠起身去拉她的手,赔笑道:“怎么就发起火来了?为夫听就是了。”

王旖手一抽,依然板着脸:“官人,奴家哪里敢发火。知道家里是豪富,顺丰行和庄子上一年出息都是十万二十万贯,一两万的这点小钱官人看不上眼也是该的。”

顺丰行送来的账,还有家中在陇西庄子上的出产,算是外账。由韩冈所掌握,韩家的家底全在外账上。王旖手上的账,则是内账。只记录家里的日常用度,和一些小项目的支出,比如这一次整修府中屋舍,预算是两千贯,就是走王旖手上的账。王旖恪守着本分,从不多问韩冈关于外账的事情,都是韩冈主动相告。

“从顺丰行送来的岁用钱就是两万贯。却还仅仅是可以分到手上的红利的十分之一,剩下的都暂存在商行中。陇西庄子上的收入,也有十余万贯。不看外账,也不知道家里豪富如此,我这本内账,实在是可笑了……”

王旖说是可笑,可脸上一点都没有笑。

韩冈很纳闷,怎么就突然发火了?他心里算算时间,还不到日子,无明火不该是这个时候有啊。

但想想这几天,王旖为了清帐、年礼,都忙到三更,大概也知道了为什么。伸手将王旖强拉到怀里,轻轻拍着背,“好了,好了,是为夫不是。你把账本放这里,为夫待会儿细细看。下午就好好歇一歇。”

王旖心里正生着气,见丈夫这样糊弄人,就挣扎着要起来。韩冈却揽住了她的纤腰,任凭如何挣动也不松手。

“韩玉昆!”王旖又急又怒的叫着。

韩冈却笑眯眯的看着王旖生气的样子,半点也不怕。还故意偏偏头,往窗口看看。

王旖身子随即颤了一下,她这个主母要面子,声音传到外面,给儿女和下人听到,日后就别做人了。不敢再出声,但咬着下唇,挣得却更厉害。

韩冈在王旖耳边说着软话,手却一点不动摇。他两条胳膊能拉石五强弓,王旖百般挣挫不开。

终究还是力气小,却抵不过韩冈的腕力,挣扎了半天,王旖已经是气喘吁吁的,头发都散了。最后狠狠的在韩冈腰间扭了一把,瞪了两眼后,任凭丈夫搂着,不再动弹。

王旖一时平静了,韩冈也不敢再闹。妻子脸皮薄,气得哭了,连着几天就没好脸色看了。

“其实不必算得这么细,”韩冈轻抚着妻子的脊背,看似纤细的身子,其实摸起来却没有骨节嶙峋的突兀感,触手之处充满弹性,“如今家中控制的田地、工坊,顺丰行下的店面、商路,加上在雍商中的地位,有形无形的资产,价值少说也在千万贯以上。就是放在江南,我韩家也是最顶尖的富豪之一!”

王旖却没有跟着韩冈一起得意,从韩冈怀里撑起身子,冷静的说道:“官人,不觉得太多了吗?为官才几年啊,就千万贯的家当。”

“难道还怕钱多烧手?”韩冈哈哈笑了两声,见王旖一对黑白分明的眸子沉静的看着自己,就收起笑容,“这是为夫开创了两个新产业的结果,可不是靠盘剥百姓来的。这钱,为夫拿的一点也不亏心。”

“产业?”王旖疑惑着。

“光靠收受贿赂,强买强卖,货殖回易,一辈子也就是几十万贯而已。所以太祖皇帝说的好,措大眼孔小,十万贯便塞破屋子矣。”韩冈没有直接回答问题,反而正色问道,“为夫可是那等眼孔小的措大?”

王旖摇摇头,她的丈夫当然不会那样的人,其实她的私心里一直为自己的父亲和丈夫感到自豪。但她对韩冈说的还是不明白。

韩冈微微带笑:“陇西棉布、交州白糖,在为夫之前,这两样特产都不存在。这两个产业,因为夫而生,也因为夫而兴,如今行销天下,备受欢迎。靠着天南地北的两个产业为核心,顺丰行才能发展的这么快,雍秦商人和陇右蕃部才会以为夫马首是瞻。这就是家里为什么能在数年间积攒下价值千万贯的这份家当的原因所在。……与他人做同样的营生,就要跟他人争夺固有的利源,闹得你死我活亦不足为奇;而独创一门营生,可以将争权夺利的力气都花在正业上,不但赚得轻松,而且占得分量也多。”

韩冈说得兴起,搂着妻子坐起来:“那些个士大夫,一个两个喊着不与民争利,可实际上呢,自己还不是买田置地?司马十二说过,天下的财富就那么多,官多一点,民就会少一点。可他买田置地,多得了这份利之后,那不就肯定有人少了这一份利?这不也是与民争利吗?”

王旖在韩冈怀里动了一下,换了个舒服点的姿势。对韩冈的话想了一想,道:“……田是拿钱买的。”

韩冈一笑:“田能生利,若不是因为急用,有多少人会主动卖田?而且官僚买田,很少会去买下田,而是盯着好田,许多时候,甚至坑蒙拐骗无所不用其极。难道这不是争利?”

王旖皱着眉,觉得韩冈说得的确有几分道理,但也觉得还是有些问题,就是不知哪里有问题。

韩冈放声道:“自来都是兴利为上,争利为下。司马君实之辈,不知兴利,只知道说着不该争利。朝廷要用事,百姓要富足,这都是要靠兴利而来。司马君实说天下财富有数,就那么多不够分,朝廷富了,民就要穷。话说得不错,可将这个道理推到民间中呢?只能有人富,有人穷。那么别说做到天下大同了,就连小康都做不到。身为圣门子弟,治国、平天下,就不去好好想想该怎么解决吗?”

