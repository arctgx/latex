\section{第30章 肘腋萧墙暮色凉(11)}

【第一更】

韩冈在抚宁堡工地待了两天,也只帮着折继世把基本的医治救护的制度整备起来,对于其他事务,他并没有插嘴,在他看来,抚宁堡的情况已是无药可救了。对在二月之前,完成只有罗兀城十分之一工程量的抚宁堡,韩冈抱着很深的悲观态度,能把城墙完成,就已经是谢天谢地了。

返回罗兀后,韩冈倒是发现这里的情况要好上不少。西城无门,而其余三座城门已经完工,城墙的墙体初具规模,而城墙外的壕河也已然完成了差不多,城内的建筑物,也有了雏形。从形制来看,西夏人所修建的旧堡,将成为核心的内城,而现在所修筑的城墙,则是外城。两重城壁护卫起来的城池,加上优越的地理条件,在陕西缘边诸多军城中,也算得上是屈指可数的坚城了。

骑着马,韩冈向着号子声传至天际的城中行去。

就在外城的南门处,纷纷乱乱的一大群驮马和两轮小车停放着,把城门都堵了起来。马背和车厢上的货物都高高堆起,韩冈离开前,罗兀这里可没这么些车马。而以罗兀城中的粮秣储备,暂时还是用不到绥德城往这里运送粮草。

“去问问怎么回事?”韩冈让护卫自己的亲兵去问个究竟。

等亲兵回来时,不是带着回话,却是带着种建中过来了。

种建中方才大概在城门口处理这群车马辎重,得到韩冈回来的消息,便立刻骑着马飞快的迎了出来。见面后也不说其他的话,只喜笑颜开的连声赞着:“玉昆你的主意果然有用!只公布了筑城进度,又用包干法赏赐做事最为得力的一队,士气立刻大振。三天的工数,两天就完成了。看起来,在月底前肯定能完工。”

韩冈倒是没有传染上种建中的兴奋,点了点头,一副理所当然的模样。却道:“抚宁堡那里……”

种建中用力一摆手,直接打断了韩冈的话,“抚宁堡那里只能草就,来不及全数完工,这点也已经知道了。只要城墙没有问题就行了,再无其他要求。至于驻军,家叔已经说了,先留一个指挥在堡中。预定的另外三千人,则暂时驻扎在抚宁堡西南十五里外的细浮图城,如果西贼分兵攻打抚宁,直接从细浮图城出兵救援,不会有任何问题。”

细浮图城在抚宁西南十五里,因为城中有一座小塔,因此而得名——佛塔的梵音就是浮屠(浮图)。韩冈听着就觉得有些不对,要是细浮图城能护住罗兀城后方的交通线,筑抚宁堡做什么。而且,把战略要地当作前出的据点,反而后方重要性略逊的城寨驻扎大军,

“这不是跟前面西夏人在罗兀、银州的兵力安排一样吗?!”韩冈惊问着。

“怎么会一样,抚宁堡现在可是有罗兀城在顶着!”种建中毫无半点担心的样子,摇着头,像是在笑韩冈想得太多,“西夏人守罗兀时,要是南面有座大城顶着,罗兀城怎么也丢不了的。”

种建中的轻松,让韩冈更为惊讶:“抚宁堡可是当着几处谷口,道路众多。只要西贼费点力气,从北面都是能绕过罗兀,直接进逼抚宁堡!”

“那时候,绥德军向北,罗兀军向南,细浮图城再出兵,把西贼聚歼在抚宁堡下,这么大的功劳,可是让人迫不及待了。”种建中大力拍着韩冈的背,笑着:“唉唉,玉昆你就是爱杞人忧天!早就对着沙盘合计过不知多少次了,预定中的方案也有了准备,不会有事的……”

他捻着下巴上的几茎短须,“还是玉昆你的功劳,要是只看着地图,定出来的计划都是简略得很,只能靠随机应变。但有了沙盘后,军情、地理一目了然,各种情况的应对方案不费力气就出来了。放一百个心好了!没有万一的!”

种建中都这样说了,韩冈也不便再多言,正好走到城门边,韩冈就转过话头,问着这队车马是做什么来的。听了种建中的解释,方知道是陕西转运判官李南公亲自押送一批物资从绥德来了,他押送的当然不是粮草,而是守城时所用的各色军资。

守城的兵械也来了,大战前的准备工作一步步的完成,而战火也是越来越近了。韩冈恍惚间几乎都能听到,来自横山北侧的荒原上,呜咽的号角,还有那铺天盖地、如同夏日郁雷的马蹄撼地之声。

随着种建中进了城,韩冈忽然觉着城中的民伕好像少了不少,至少少了三成,连驻军的营帐也不见了许多。

“彝叔!怎么城里的民伕少了许多,军队也少了……”

种建中把韩冈往建在滴水崖上的内城领去,答道:“罗兀城的城墙已经筑到了一丈高,已经有一定的防备能力了,不需要两万大军蹲在外面守着,留上八千就足够了。”

“他们人呢?”

“一队去北面的山口,进筑赏逋岭寨,守着马户川和立赏坪。”种建中在通往内城的坡道上停下脚步,越过下方的外城,指了指无定河斜对面的山谷,大约两里外的地方,“看那里,另一队就在那边,”

韩冈顺着种建中的手势望过去。两里外的景物已经很模糊了,又是藏在山谷中,他过来时没有在意,但现在被种建中一指,就立刻发现那边也是摊开了一处工地。

“永乐川?”

“对,就是永乐川堡!”种建中点点头。

韩冈眯起眼眺望着。那条山谷是无定河支流永乐川的出口,从地势来看,在那里建座寨子,的确可以与罗兀城成犄角之势。这新筑的永乐川、赏逋岭二寨当皆是罗兀防线的组成部分,看起来罗兀城的守御能力的确是越来越稳固了。

“也是多亏了玉昆你,本来永乐川、赏逋岭只计算着时间,只够草草立两座小寨。但现在,当是能按着形制,筑正式的寨堡了。”

韩冈被赞的都有些麻木了,谦虚了两句,低头看看下方的工地。又有一点疑问浮上心头:“不过就是修两座寨堡,也用不着分那么多兵出去吧?”

“剩下的去接应河东军了。河东那边拖了快半个月,到现在都没消息。五叔前几天就已经传书延州,请韩相公赶紧催一下。有了河东出兵,罗兀城当会更为稳固。”

韩冈拍拍脑门,事情一忙都忘得一干二净。攻取罗兀并不是鄜延路一家的事。陕西缘边四路,还有河东路,都是要动手的。要不然,韩绛也不会兼着陕西、河东宣抚使的名头。

河东,顾名思义就是黄河以东,就是在几字型的黄河东侧的那一竖的东面。不过大宋的河东路在黄河以西,也是有着一块地盘。那就是以麟州府州为中心的河东西北战区,在宋室建立以前,是如今的麟府折家的控制区。

河东与西夏的交界是平行于黄河的南北纵向,而陕西与西夏的分野则是以横山为主的东西横向。在陕西与河东的西夏边境交汇处,那一横一竖形成的直角所在的区域,如同一根楔子割断了河东与鄜延路之间的联系,就是与银州并为西夏国西南防御核心的神勇左厢军司。

攻占罗兀的直接目的是横山,夺取横山的意义则在于银夏。而在夺取罗兀的同时,鄜延路与河东路一齐进兵,也就可以把神勇左厢军司这根楔子,给连根拔掉。一旦给宋人打通了麟府和鄜延的交通线,将两地连成一线,罗兀防线完固,银夏地区将唾手可得。

就在预先的计划中,河东路也要出兵筑城,来巩固罗兀防线。鄜延路这边是罗兀、抚宁、永乐川、赏逋岭诸城寨。而属于河东一方的则是荒堆三泉、吐浑川、开光岭、葭芦川这四座寨堡。一旦这些寨堡修起,牢固的罗兀防线将能把鄜延、河东之间的交通线稳定下来。

“不过河东那里可能会有些难度。前些日子,银州都打成了这般模样,都罗马尾一败再败,左厢神勇军司硬是一个兵都没出动。”种建中望着东北方被雪色掩盖的层峦叠嶂,“为了提防西贼安排在左厢神勇军司的两万军,一开始都是提心吊胆的等着,连夜里都不敢合眼。现在出兵去接应,也是为了能更顺利一点。”

说话间,种建中和韩冈已经进了内城。把韩冈送到主帐外,种建中笑着道:“好了,五叔正在等玉昆你的回话,我就先下去处置今天送来的东西了。”

目送种建中离开,韩冈在帐外通了名,立刻就被招了进去。

三天一晃而过,罗兀城的城墙顺利的增高中,而韩冈手上的工作也很稳定,病人和自残的现象都少了许多,民伕们皆是急着完工好早点回。看起来一切都很顺利。

但就在正月二十的这一天傍晚,一队骑兵冲进了罗兀城。很快,种谔的亲兵四散而出,召集来城中诸官。坐在大帐中的种谔,面含隐怒,咬牙切齿的样子仿佛要吃人一般。

众人心中都有了不好的预感,果然,等了一阵后,种谔终于说出了一个噩耗:

“河东那里败了!”

