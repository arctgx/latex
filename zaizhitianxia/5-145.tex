\section{第15章 自是功成藏剑履(七)}

太原城的城门就在眼前,行人车马将城门前的道路都堵得拥挤不堪。

童贯失望的叹了一声气,终于垂下了手中的马鞭,也终于不再用靴子后跟踢着马腹。

快跑中的坐骑,慢慢的放缓了速度。一个多时辰前才换的驿马,这时候已经是满身是汗,呼哧呼哧的从鼻中喷着长长的白气。

“黄门,不追了?”童贯的两名从班直中点出来的随从也跟着慢了下来,凑过来问道。

“还追得上吗?”童贯没好气的回道。

他自奉诏追回之前密诏,出宫后便一路急追,皆是兼程而行。但前面的那一位却也是双快腿,一心想将天子的吩咐办妥当了,一路上将沿途驿站的好马全都挑走。这两天一夜的迟误,就变成了长江黄河一般的天堑,童贯一路追到了太原城外,竟也没能赶上派出去的中使。

“先进城吧。”童贯呆呆的望了太原城的南门半天,无奈的又叹了一声,翻身下马。回头看了一眼还在马上发愣的两名班直,低喝道:“不要太惹眼。”

得了童贯点醒,两人也立刻从马背上滚翻了下来。跟着童贯一起牵着马,往城门走去。

可能是战事刚刚结束没多久的缘故,太原城城门处的管理依然严格,行人车马都被仔细检验。一眼看过去,都没看到有人骑在马上入城出城。童贯不想惹起太多关注,下了马后,又示意一名班直拿出他自己的号牌去通关。

但就在童贯正在城门处等候着回应,一名铺兵装束的骑手却在城门口跟守门官说了几句,也不下马,便直直的便冲了出来。

童贯的视线一直追着那铺兵直到再也看不见,顺利的进了城门后,走了两步,突然跌脚失声,“哎呀,不好。”

“黄门,怎么了?”两名班直忙凑了过来。

童贯声音沉了下去:“方才过去的是马递。当是韩冈的回奏!”

“不会吧?”两名班直回头看了看城门,满面疑惑:“黄门是怎么知道的?”

童贯反问:“胜州大捷之后,河东还有什么地方要动用马递至御前的?”

动用驿马的马递直通通进银台司,是可以绕过两府,直上御前的驿传手段。寻常情况下妄自动用马递,可是要受罚的。

“会不会北面的辽人又有什么动作了。刚在韩龙图手上吃了亏,辽人肯定会大举报复的。”一人猜测着。

“当真是辽人举兵报复,那就该是急脚递!”童贯指了下城门处的守卫:“方才看到他们亮金牌了吗?”

两名班直对视了一眼,都摇了摇头,的确没有看到。带着紧急军情的急脚递。把金牌一亮,马都不停直接就从城门冲过去了,怎么可能还会在城门口磨蹭,跟人说两句才走。

“太原城中,能动用马递的只有知太原兼经略使的韩龙图。这时候动用马递,倒有五六成的可能是韩龙图上表谢罪或是自辩。天子下的是密诏,用步递发回去,肯定绕不过两府。”

童贯的一番解释,合情合理,两名随从有了几分信服。一名班直又问道,“黄门,那现在怎么办?要不要回头拦着?”

两人都很清楚童贯身上的任务。没有拦住密诏,就已经是办事不力,再让韩冈的回复传到京城去,天子那边可就是不知是办事不力那么简单了。

“拦?拦马递这不是找死吗?!天子能用金牌召回密诏,边臣的奏报,你能召回还是我能召回,马递上路后,边臣本人都不能再拿回来啊!”

童贯喘了一口气,满肚子怨气。幸好出来前多问了一句,要是没追上该怎么办?

“先去一趟府衙吧。”

……………………

已经将谢罪表遣马递送了回去,亲笔为韩冈起草奏章的黄裳依然难以释然。

“龙图何必这么快就上谢罪表,朝廷收到胜州大捷的消息,肯定会明白之前的错误。”

“既然收到了天子的密诏,无论如何都必须有所回应,岂能耽搁拖延?”

是否及时回复天子的内降,这是态度问题。至于这个回复会不会让天子感到难堪,韩冈可没兴趣关心。

到了他这个地位的文官,只要把表面文章做圆满了,也就没有什么好怕的。天子的心情好坏,从来不是真正的士大夫放在第一位要考虑的。

“那龙图也不该将罪责全都揽于一身。”

虽然韩冈也是无意收留太多的黑山党项,可要不是折克行和李宪两人手下将校贪图斩首,也不至于杀得那么狠。而且修筑边地营寨的黑山党项之所以能被煽动,也是因为做工时,被过分催逼,以至于生不如死的缘故,否则以他们跟辽人的血海深仇,也不至于反去配合辽人。

“军令是我下的,自不能让罪名推到别人身上。”韩冈转头问道,“勉仲,你看我是争功诿过的人吗?”

“黄裳失言了。”黄裳低头表示歉意,想了想,又问,“……龙图,那此事要不要知会折府州?”

“这有什么好说的?”韩冈笑着摇头:“被天子密诏叱责,又不是多光彩的事。”

“不是……”

黄裳想要解释自己的意思,韩冈却又摇了摇头,“若是想要折家欠下人情债那就更不必了。天子既然只以密诏降责,本就只罪于我一人的意思,并没有打算否认这一战战功的打算。既如此,何必再与人说?”

黄裳赧然,韩冈的意思他听明白了。以君恩为己恩,这是臣子的大忌。这个便宜,的确不能占。即为密诏,泄露给李宪当然不行,就是折克行也一样。

韩冈的心中盘算没有他说得这么光明正大,只是不想落了下乘而已。反正李宪肯定很快就能从京城宫中得到消息。折家在京城中也肯定有耳目通风报信,没必要枉做小人。

见韩冈没有再多的吩咐,黄裳便告辞离开。

韩冈看着他的背影摇摇头。黄裳游学四方十几年,决不是没有眼色的人。韩冈只让黄裳帮忙起草奏章,都没有将折可适招来,心意早就表明了,他可不信黄裳看不出来。不过奉承人的水平还有待磨练,实在有些粗糙。

见外面没有什么事,韩冈起身入内,往书房去。

家里面这两天无论是谁都是愁眉不展,让韩冈觉得有些烦。王旖四女皆道伴君如伴虎,谁知道皇帝会不会因为一时之气,找个借口对自家的夫婿加以惩处。

不过在了解当今天子的为人性格之后,韩冈觉得倒是没什么好担心的。

没有太宗赵光义的阴狠果断,也不及真宗赵恒能做到自欺欺人。现在的这个皇帝,本来就是太在意外界评价的性格。

更重要的是韩冈本人也不是可以任凭搓扁捏圆的软柿子,咯手得很。整件事上没有犯过半点错,想找借口都难。且经过一百多年的养士,士大夫的阶层能对天子产生足够的牵制。就是皇帝也做不得快意事。

以功劳算,这些年来的功绩,早已足够抵消资历上的欠缺,并将自家顶入两府之中。这一次出镇河东,没有出过一次纰漏,就算有,也都立刻弥补了。

本以为阻力只在皇帝那里,两府诸臣应该都该学聪明了,不当主动表态。只是没想到御史台中的成员,会有那么多人将自己当成眼中钉,当成刷声望的工具。在赵顼的密诏中,看到他隐约透露的这些细节,还真是出乎意料。

既然如此,就必须要做个选择。

……在官职和夙愿之间做出一个选择。

对于韩冈来说,做出这样的选择根本不需要犹豫。本来来河东也只是一个意外,依照之前的想法,也没必要急着入两府。但这一次的功劳,总得换来一些实质性的回报。

回到书房,韩冈喝茶看书。给王安石的第二封回信已经写好了,进一步阐明了对王安石寄来的新书的看法。稍稍有些激烈,没有留上翁婿间的几分情面。学术之争上,也没什么岳父、女婿,该争就得争到底。

虽然最近几年斗争的目标是程学,但有机会,给新学下几个绊子,韩冈也不会犹豫。而且能在学术之辩上给新学一个难堪,也是气学大涨声望的机会。不过以王安石的学问,要从他的书中挑错,还要得人认同,也不是那么容易的一桩事,甚至可以说很难。韩冈一向用的是扬长避短的手段,但这一次,可没有那么容易。

坐下来仔细检查刚刚写好的信,斟字酌句的推敲着,尽可能的不留下给人挑出错处的地方。心神很快便沉浸了下去,将朝廷、皇帝这一干烦心事丢到一边去。

只是韩冈在书房中没有坐上多久,家里的下人来报:“龙图,外面有一个姓童的黄门求见。”

“求见?”韩冈放下笔。又是带着密诏,所以怕引人瞩目吧?姓童的话,多半是童贯了。而且童贯跟自家有过往来,被派来太原见自己,多半也是想利用这份香火情。

不过不管是什么原因,反正总算还是来了……只慢了一步啊。

韩冈轻笑了一声,“快请。”

