\section{第11章 城下马鸣谁与守(六)}

盐州城西北二十里的白池堡,入夜后依然人声鼎沸,堡内堡外,篝火多如繁星。攻伐盐州的大军,其中军主力眼下正驻扎于此。

白池堡原本是用作护卫盐池,位置就在盐池旁,就连空气中似乎都带着咸味,不过食水却没有问题,是甘甜清澈的泉水,与盐池中的卤水截然不同。

西北的盐池,不论是宋人手中的解州盐池,还是西夏这里的青白盐池,全都用的是晒盐法。将卤水引入盐畦中,然后依靠日光暴晒,不过最后还要用清水冲刷一遍,这样才能洗去苦味。不论是解州盐池还是青白盐池,都是有着一道清澈的甘泉才能成事。白池堡外的清泉,用来煮茶倒也是正好。

不过现在叶孛麻现在没空坐下来喝茶,在灵州之战中,表现得光彩夺目的老将,就在堡外上马,带着本部亲兵飞驰向盐州城的方向。

叶孛麻抵达盐州城下的前军营地时,已经入夜,营地内外篝火星星点点。留守营地的主帅仁多澣正在大门外守候。

焦急的等叶孛麻下马,仁多澣劈头就问:“太后和国相怎么说?”

“要我俩稳固寨防,等明日主力抵达后别作商议。”叶孛麻笑道,“太后亲自领军,不就是为了盐州城?用不着我们打头阵拼死拼活。”

仁多澣神色放松了些许,“要是当真如此,倒是一桩好事。”

“总该让梁家和嵬名家有地方出点风头,太后和国相就是这么想的。在灵州城,我们外姓的都为他们两家拼了命,也该他们出来做点事了。”

仁多澣咧嘴一笑。还真不知道在白池堡中,太后和梁乙埋是怎么被人挤兑。

这一次攻打盐州,梁太后亲自领军,就连被囚禁起来的儿子秉常也一并带了出来。班直、环卫,能有一定战力的军队全都随行,只留了梁乙逋和嵬名阿吴镇守兴庆府。

太后领军上阵,倒也不算很稀奇。辽国、西夏过去都有过先例。盐州是事关国运的一战,败,则西夏不存,仅靠兴灵之地,最多也只能苟延残喘几年。胜,大白高国才有延续下去的可能。

之前在灵州,外姓将领光彩夺目,硬生生的翻了盘。如今同样是事关国运的一战,梁氏总不能安坐在兴庆府中,梁家和嵬名家要想继续统治西夏,必须出来立下功劳。

梁家的权势一直都是建立在嵬名家的支持上。他们的兀卒太过于亲近辽人,触动了太多人的利益,就是在嵬名家中,对秉常心中生怨的也不只一个两个。要不然梁氏兄妹囚禁天子,也不会得到宗室们的支持。

“想必太后这时候最想看到的就是兀卒早点把她的孙子生出来。”

“这也要兀卒自己愿意才行。”

不管怎么样,梁太后总希望自己的血脉能延续下去,但有了孙子之后,儿子倒也可以丢掉了。当今大白高国的皇帝,或许会在某间戒备森严的庙宇中度过余生。

在大营门前,借着熊熊的篝火,叶孛麻看清草草修毕的寨防。

前面在白池堡时,修筑外围供中军驻扎的营垒时,就已经很艰难了。而前军营地的情况更差。

“盐州城外的树都被砍光了吧……”叶孛麻啧了舌头。

仁多澣指着用树枝勉强钉起的营门:“能找到修栅栏的树枝就算很不错了,营门才让人头疼……都是这一仗打的,盐州周围好砍伐的木料,早就被清扫光了。”

宋军之前攻打盐州及其周边寨堡的时候,费了一番手脚,能修建营栅和制造攻城器械的树木就没多少剩下的,等到宋人开始增筑城防,对木材的需求又上了一个台阶。

叶孛麻左右看看并不牢靠的寨防,忽而问道:“宋人会不会夜袭?”

“来也好,不来也好,都给他预备着。”仁多澣遥望灯火通明的盐州城头:“能玩出来的花样就那么几个,一辈子打猎,还能给雁啄了眼睛去?……先进营吧,等着他们来。”

叶孛麻呵呵的两声笑,与仁多澣并肩进了营中。

跟随在仁多澣身后的亲兵,在夜晚身上也穿戴着盔甲,是宋人独有的板甲。一路往中军大帐去,叶孛麻就听见身后咣咣的铁甲声响。叶孛麻的亲兵倒没穿,但他们也有,等上阵时就会套上。

眼下在外面的普通士卒,就算是征发起来的骑兵,上阵时也多由甲胄可以穿戴。灵州城一战,缴获的盔甲数以万计,南朝将卒都是丢盔弃甲而逃,唯恐身上带得东西多了,逃起来耽搁时间。这一番收获,让铁鹞子变得名副其实。

叶孛麻很早就听说了宋人改进了他们的制甲工艺,打造的成本、耗用的时间,无不大幅下降,甚至据说只有之前的十分之一。其主导者就是在西夏也是声名煊赫的韩冈。

三数年内,六十万东朝禁军全数配发铁甲。这一事实,在这些年压得西夏国中对此有所了解的显贵们喘不过气来。

幸好光是有神兵利器也不是肯定能在战阵中得胜,正面无可拮抗,但合用有效的策略,就让大白高国的战士将铁甲从宋人身上剥了下来。

现在几乎所有的西夏将领们的亲兵,如今都是穿戴着宋军指挥使及其以下的军官们的装备——灵州城下的缴获中,其数量仅次于士兵们的简易板甲——由于这些甲胄都是按照官职的不同等级,镶上不同的饰物,外观甚为精美,穿上使得亲兵们的面貌焕然一新。

仁多澣的亲兵穿戴得都是都头一级的全身铠。铁甲浸了铜,微微泛着赤红。不像卒伍们的装具,只有覆盖前胸后背的甲片,以及保护下半身的几片裙甲。而是肩部、臂部以及腿部都有配件。头盔盔缨上方还竖起一根四寸长的小棍,棍上黏着面三角形的小角旗。

据叶孛麻所了解,宋军都头们的站位,是列阵时的标准。他站在那里,他下面的士兵就会跟到哪里,这是宋军军令中严格要求执行的条款。都头们头盔上的角旗,就是让下面的士兵们知道,该跟着谁,谁才是统领百人的头目。

这样的一套盔甲,到了党项贵胄们的手中,用来代表亲兵身份也是一样的有用。

在宋军中,更高一层的甲胄都是给将领们量身订做了,件件都是价值千金,鎏金、鎏银的不在少数。叶孛麻拿回家的几件顶级战利品,连衬里都是用着熊皮,只是对他来说并不算合身。

同样听着身后的甲胄作响,仁多澣对叶孛麻笑道:“过去可从来没想过,能给族里的儿郎都配上铁甲,灵州一战,可也算是不亏了。”

叶孛麻摇了摇头:“有什么好高兴,今天夺了三万套,明天宋人就能打造个六万套出来。六十万禁军三年全部换上铁甲,一年能产二十万套铁甲。如今丢了三万套,也就让赵官家肉疼心疼,都不带伤筋动骨的。”

他转头对仁多澣苦笑,“越是跟宋人厮杀,就越是明白他们有多财大气粗。一年造甲数十万,丢了多少,十倍补回。怎么跟宋人比,就是灵州那样的胜仗再来一次都完了。一场水放下来,灵州城外的田全毁了,明年的口粮还不知在哪里。”叶孛麻仰天叹息:“过去是肥羊,如今就是山猪。一样的咬一口肥油四溢,可他们长獠牙了。想咬下一块肉,自己还不知要出多少血。”

仁多澣闻言默然,走了几步又勉强笑道:“好歹还有些肉咬回来,比前几次输得本钱都光了要好。不是还俘获了一干工匠吗,没有这些匠人,谁有把握能攻下盐州城?”

“可是没铁匠……宋人都不给营中配铁匠了!”

灵州城下一战,俘虏甚多,足足有两千之众,其中光是能制造攻城器具的工匠就有四个,被派来服侍飞船的工匠一人,只是没有铁匠——在过去,宋军不论是出阵还是守在寨中,总会有一两名铁匠用来修复兵器或是甲胄,但现在的宋军如果兵器坏了,直接换一个新的;甲胄出了问题,换个配件。宝贵的人力不耗费在修修补补上。

这是标准的财大气粗。

“……不管怎么说好歹还要拼一下。”仁多澣说着,人在死前还要挣扎一番,何况一个拥有百万人丁的大国,“前面宋军攻到灵州城下时,是命悬一线,整个吊在半空中,就一根手指搭在悬崖上,就这样还爬了回来。如今的局面纵然也是危在旦夕,但总比几个月前要强得多。这一战若胜,一二十年内,东朝绝不敢再西顾,不争一争如何能甘心。”

“说得也是。”叶孛麻的神色也缓了下来:“要是宋人退守银州、夏州,他们倒有九成胜算,无论如何我都是不会来的。现在他们贪心的想保盐州,好歹是对半开了。”

“东朝的皇帝总是胡乱选主帅。什么徐禧,过去听都没听说过。这一回要是胜了,耶律乙辛多半忍不住要领军南下,到时候就是有十几二十年的平安了。”

