\section{第33章 物外自闲人自忙(五)}

“韩龙图给文相公赶出来了?!这还当真不得了。”

“真的假的?”

“俺家隔壁的刘小乙的小舅子的亲叔叔,当时可是亲眼看见韩龙图黑着脸从府衙中出来。”

“刘小乙的小舅子的亲叔叔?……武大不是卖炊饼的吗?他怎么看见的?!”

“衙门里的人难道就不吃炊饼?”

本来河南府的官员没有出城迎接新任的都转运使的这一件事,就已经成了洛阳城中议论最多的新闻,最受关注的焦点,人人都想看着整件事下一步会怎么发展。而更进一步的发展,也的确符合了人们的期待。

韩冈隔了一天就登门造访,一刻钟之后,便从府衙中出来。如果是低品的小官来拜见,文彦博能留他一刻钟说话,已经是很给面子了,但韩冈是都转运使,有着紫金鱼袋的龙图阁学士,得到这样的待遇,实在是让人无法想象。

如今洛阳城中的传言,当然不会担任转运使的小韩学士登门造访,与文相公共商国是之后,便起身告辞。这样一点戏剧性都没有,当然也不会有人有兴趣,这说法很快就淹没在了流言的大潮中,转眼就没了踪影。

现如今是韩冈被文彦博赶了出来的谣言甚嚣尘上。不仅有人拍着胸脯说自己隔邻的小舅子的亲叔叔,当时正在府衙之外做买卖,亲眼看见韩龙图铁青着脸从衙门中出来,甚至当时文彦博和韩冈在厅中是怎么开始争执,怎么又吵得不可开交的故事都给编出来。

最离谱的传说是最后文相公摔了杯子,一群刀斧手——不对,是一群衙役蜂拥而出,将韩龙图请了出去。整套故事编得有鼻子有眼,证人从河南府的衙役到街上的路人,再到韩冈身边的随从,一个个罗列出来,让人不得不信。

而且与此同时,韩冈拒绝了司中属僚的提议,并不准备即时点验河南府库账籍的说法不知从何处也传了开来。

韩冈善抚士民,为人也算是仁厚,在洛阳是人人都知道的——当年一万多河北流民被派来整修洛阳段的黄河大堤,一下就跑了三千人,剩下的则是哭着喊着要朝廷派韩冈来主持工役。至于文相公,那是有名的行事果决。所以这其中的是是非非,洛阳士民都有自己的判断。

也因此,拥有激烈冲突的戏剧性的谣言,比实际上的真相传播得更广。反正洛阳城中除了当事人,还有少数两边都能得到准确消息的耳目灵通之辈,基本上都是从口耳相传中听说了这个消息。

在许多人看来,府漕两家算是结下了死仇,接下来事情会怎么发展,洛阳城中多少人都当做一场难得的好戏在期待着,但远在东京的赵顼绝不想看到这一幕。

之前得知河南府上下都没有出迎韩冈,赵顼就已经很恼火了。韩冈身上的都转运使,从这个‘使’字就该知道,他代表着天子——至少如今在名义上还是如此——作为天子之使,文彦博这位老臣却倚老卖老,赵顼哪能不怒。

此事倒也罢了,只要不会干扰到正经事,赵顼不是不能优容。但才隔了两天,韩冈登门拜访府衙,才一刻钟就出来了,当韩冈是河南府的知县吗?韩冈哪一次上殿谒见,他赵顼不都是留了至少一个时辰来君臣问对,文彦博倒好,竟比他堂堂天子架子都大。

赵顼也不是没有怀疑过这件事的真实性,毕竟文彦博是老臣了,在官场上几十年,登上相位都过去三十多年了,怎么想也不该犯上这样的错误。可赵顼昨天、今天收到的来自洛阳的奏报中,几乎每一封都说到了此事,只是说法上有点混乱。

西京洛阳河南府离着东京城并不远,加上整件事闹得声势太大,也就是事情发生的第二天,当今的大宋天子就在崇政殿中看到了洛阳走马承受和各级拥有密奏之权的官员的禀报。

远在在东京城中的赵顼耳目十分灵通,能通过走马承受等各种途径得到准确消息。但问题是他得到的消息中,并不只有真实无误的信息,而是真真假假混合在一起。当今的大宋天子并不清楚,在这么多的相互抵触的消息中,哪一条才是真相。

朝廷安排在洛阳的走马承受有好几人,但他们都是风闻奏事,传回来的消息多而杂,加上七八名当地朝臣的密奏,其中有不少述说自相矛盾的,真相需要赵顼自己来挖掘——基本上摆在赵顼案头上的情报往往都是如此,就像当年熙河路的宜垦荒地到底是只有一顷还是一万顷,王韶、韩冈和李师中、窦舜卿吵了有半年,朝廷派去确认的使臣睁着眼睛说瞎话,最后还是靠了沙盘,同时有了实际的产出才确认下来。

回到洛阳现如今发生的事情上,传来的消息可以归结为两类,一类就是文彦博先无礼后再无礼,说了两句就点汤送客,赵顼得到的奏报中大部分都是如此说;另外寥寥数份奏报,则是说并无此事,韩冈是拜访河南府衙后自行离开,在这其中还有指责韩冈本人意欲报复,急不可耐的要检查洛阳府库——这份奏折,赵顼看了就丢了,韩冈在他的奏章中没有说过文彦博半句不是,而转运副使李南公也是有密奏之权。

韩冈在抵达洛阳的第二天,就写了奏折回来,说了自己初步的计划,上面是明明白白的写了他准备先去点验汝州、唐州的钱粮,至于原京西北路诸州的监察工作,韩冈则说打算暂时交由转运判官按照先前的次序来处理,希望赵顼能够批准。在奏章中,他半句没提河南府上下没有出迎的情形。而这一次事情闹得沸沸扬扬,韩冈同样什么都没有说,连封密奏都没有,看样子是想要息事宁人。

该相信谁,赵顼大体也能判断出来。韩冈过去的奏章全都是就事论事,除了评价自己的属僚才具、德行是否可堪任用,他从来没有指责过任何人。至于文彦博……过去就没少指责过韩冈,而今次阻止属僚出迎韩冈,没有一封奏报上否认,帮着文彦博说话的也仅仅是忽略不提而已。

赵顼很是恼火,他派韩冈去做正事,不是为了跟人斗气。他知道文彦博不喜韩冈,但保持着宰相的气度难道很难吗?偏偏还出了此事,难怪文及甫敢写信为贿赂大理寺的犯官干请。

赵顼不禁怀疑起自己,是不是最近安排两府的人事,有些做得过头了。宰相和枢密使皆是反对变法的一派,虽然自己只是想缓和一下熙宁的十年间,矛盾重重地朝堂气氛,但现在看来,恐怕是让人误会了自己的想法……

赵顼的眼神变得冷厉起来。抬起手,从准备留中不发的一叠奏章找出了一本,翻开来,这是弹劾吴充之子吴安持的奏章。拿着朱笔提起几个字,放到了另外一叠准备转回中书的奏章上——这下应该不会有人误会了!

东京城的事定了下来,但洛阳的事就有些棘手了。文彦博是老臣,三朝元老,甚至还是拥立他父亲英宗为皇嗣的功臣之一,于情于理都不能不给他一个体面——也许这就是文彦博有恃无恐的原因——尽管赵顼已经心有定见,但此事还是得先派人去确认一下。

“李舜举。”赵顼回头叫着就侍立在身侧的御药院都知,“你明天去洛阳一趟。”

李舜举犹豫了一下,没有即刻应承接旨,而是问道:“敢问官家,为了何事去洛阳?”

赵顼听着一愣,但立刻就反应过来。这件事不能明着查,否则就变成两派互相攻击的战场,闹到最后,真相什么的都不会有人去关心了,都成斗鸡一般将对手赶下台去。赵顼暂时还不想为此事闹得太大,否则襄汉漕渠之事必然会受到干扰,必须得找个合适的名目。

大宋皇帝想了一阵,几个借口都不合适,而李舜举木桩一般的站着,也不帮着想,从他的本心上来看,并不想去洛阳,往漩涡里跳。

“官家。”站在下方的一名身材高大粗壮的宦官突然出声。

“童贯?”赵顼疑惑的瞥了一眼这两年来在崇政殿上一直都十分老实听话的近侍,“你有什么话要说?”

童贯忙低头弯腰,道:“郑国公生辰将至。”

赵顼眼睛一亮,他都忘了,原来富弼的生日就要到了。依照多少年来的惯例故事,宰相或前宰相的生辰,天子都要赐物以示荣宠。富弼、文彦博、王安石的生日时都能得到赵顼的赏赐。

比如文彦博,他四十二岁做宰相,至今已有三十余年。生日是定例能收到的四只涂金镌花银盆已经累积到一百多了。赵顼听说他过生日的时候,就将这一百多具涂金镌花银盆罗列在堂上,让来祝寿的宾客艳羡不已。

眼下富弼的生日既然快要到了,那么派中使去洛阳自是光明正大,顺便去问一下如今洛阳的时事,也不会惹起太大的风浪。

“李……”赵顼本想就此吩咐李舜举,但刚开口就停住了,反而叫着方才在殿下出了个好主意的内侍:“童贯,富弼生辰将至,这一次,就由你去一趟洛阳,待朕问个好。”

