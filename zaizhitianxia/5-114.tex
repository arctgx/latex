\section{第13章 羽檄飞符遥相系(三)}

【第二更】

武贵双臂环抱,铁塔般的站在大营门前,十来个平日里最亲近的兄弟,与他一样都在大营门外站着。牢牢的堵住了大营营地。

营地之外,还有两队骑兵一前一后的绕营栅巡视,这是李信手下最为精锐、也最听他吩咐的泼喜军。李信出营前,吩咐了他们两边同心协力,好生看守好营地,以防有人趁势脱逃。

武贵在门前站了有一阵子了,但他站在大营门前,却没有人敢去说个笑话或是上前挑衅。看着他的人不少,但眼睛里面都透着深深的畏惧。

方才李清不在营中,的确就有几个军官想把手下的人拉走,直接去投种諤,来个先到先得。不过给武贵带了十几个兄弟硬是将他们拦在了营中。

汉军中都知道武贵的武艺不成,没人将他放在眼中。但武贵持枪挎弓,半柱香的时间,便用一杆神出鬼没的长枪,接连挑下了七八个以勇力闻名军中的军官,无人是他一合之将。掌中长弓,更射落了每一个向他叫嚣之人的头上盔缨。这时候,人们才知道,武贵过去几年将自己的功夫藏得有多深,纵然是放在大宋军中,都应是万里挑一的高手。

而他同伴们手中的一张张神臂弓,也让人心惊胆战。战争已经结束,无人有拼死一战的勇气。跟着顶头上司投降是投降,跟着李太尉投降也同样是投降,谁会去为别人争这口闲气?!

连同武贵在内,不过十三个人,便将三四批总共四五百人给堵在了辕门内,让李清布置在外围的两百名泼喜军,完全没了用武之地。

李清回来时,对营中发生的事懵然不知。但他领军多年,营中的气氛有异,进营之后,没几步就感觉出来了。拉着武贵便问,“我不在的时候,营中可有人闹事?!”

武贵立刻摇头:“没有,平安无事。有太尉的虎威在,宵小岂敢近营一步?”

李清眉头一皱,转着眼环视周围一圈。看过每一个人脸上的表情,他心中顿时了然,却也不戳破,笑道:“如此最好。武兄弟的才干过人,能安抚兵将。日后入了大宋军中,说不得还要依靠武兄弟你来辅佐。”

武贵拱手一礼:“多谢太尉抬爱,武贵铭感五内。”

李清哈哈一笑,拍了拍武贵的肩膀,“不要多礼了,我还要谢你才是!”

现在李清已经将武贵当成心腹来使用,扯着他便往帐中走:“你可知道今天商议的结果?”

武贵摇摇头,“小人愚鲁,哪里可能猜得到?”

“西夏已经亡国了!”

李清劈头的一句话,便让已经有心理准备的武贵都吓了一跳,“亡国了?!”

“嗯。仁多零丁和叶孛麻带着外姓诸将要自立,说是党项、鲜卑从此分家,自个儿抱团投向大宋。”李清回头看着武贵,“你说说,西夏是不是亡了?”

武贵消化了一下这个消息,说起来李清派人向种谔请降的事他也知道,种諤的回复他同样知道,仁多零丁等人能这么快做出决定,细细一想却也不足为奇。

他随着李清进了主帐,皱眉问道:“……太尉的想法呢?”

“仁多零丁和叶孛麻想找我一起商议,抱成团跟种谔和大宋朝廷打交道。”

武贵摇起了头:“跟他们走得太近,不太方便。他们人多,我们人少。他们是党项,我们是汉人。”

“我也是这么想的。”李清在几案后盘膝坐下,示意武贵也坐下说话,“不过我这么急着赶回来,更是为了镇住这四千人。但这四千汉军,相比起嵬名家或是仁多零丁他们来说,还是太少了一点。需要找个能互相提携的外援。”

武贵眼神闪动:“太尉的意思是?”

“汉臣,文官。”李清慢慢的吐出两个词、四个字。

西夏国中的汉臣,只有不多的在军中领兵,其他基本上都是梁乙埋提拔上来的官员,在西夏这个小朝廷中充任文官。实力弱小的文官系统平常很不起眼,而且在秉常亲政的那两年,受到了很大的打击,至今没有恢复元气。但如果能联络上,眼下却是莫大的臂助。

李清找出一张纸,提起笔,匆匆写了一封短信,给武贵看过后,装入信封收好,“这是给国相……给梁乙埋的信。”

接着从几案上的一本书中抽出一封信来,一并交给武贵,“我之前已经给梁乙埋身边的袁宝臣写好了信。待会儿,你去给梁乙埋送信,然后私下里将这一封信给袁宝臣。”

“小人明白。”武贵将给梁乙埋的信收在怀里,却把给袁宝臣的信藏在了脚后的绑腿中。

李清想了想,又道:“光是信还不够。”

说着从外面叫了个亲兵,让他带上一包金银,跟随武贵一起去。绝大多数的汉人文官,只要一封信提两句,就能让他们投过来。但要想与他们深交,还是附上点财货比较好。

李清本想再多派点人跟着。但考虑了一下后,还是作罢。武贵刚帮了他一个大忙,派一个人跟着,只是个卖苦力的,派得人多了,倒让他心中生了嫌隙。

武贵接了令,转身就往外走,今夜事情紧急,延误片刻,局势就能起变化,耽搁不得。

出了营帐,便看见一个人守在帐篷边上,再仔细看看,在火炬照不到的地方,还站着有十来个人,全都是方才协助武贵镇压营中异动的兄弟。

武贵留下亲兵,上前几步,“出了什么事?”

“哥哥。”领头一个低声急问道:“这一下子,当真是要投官军了?”

武贵点头,“没错。”

“哥哥你也打算回大宋去?!”那人追问道。

武贵又摇摇头,“都从大宋那里出来了,再回去做什么?当年犯下的事,到现在还没了结呢!”

“哥哥果然也是这么想!”十几人一齐喜道,其中一人道:“哥哥,我们一起出去吧。凭我们兄弟十三人的本事,哪里混不出头来。何苦再去受那份气?!”

武贵叹道:“我这两年受了李太尉不少恩德,就这么走了,也显得我太没义气。等这最后一桩事,帮他办妥了,我才好离开。”他从十二个兄弟的脸上一一看过去,“若兄弟们有心与我吴逵结伴去闯一闯,就到八里外西山脚下的第一座递铺外侯着,天亮之前,我会去那里的。”

没听出武贵自称姓名时细微的变化,十几个汉子齐声叫道,“哥哥,可是说好了,不要骗我们。”

武贵,不,应该是吴逵,他沉声道:“大丈夫一言九鼎,既然说出了口,定不会食言!”

……………………

嵬名秉常自从被囚禁以来,便被封死了对外联络的信道。但一些最基本的情报,比如灵州之役的胜利,盐州之战的胜利,都是知道的,也不可能不知道。

当盐州被收复,千万人的欢呼声传到秉常的耳中,他的反应是愤怒的摔掉了手上的茶盏。但从昨日开始,气氛又变了一个样。身为阶下囚,对周围人的态度十分敏感的西夏国主,立刻就察觉到局势当是有所变化。

‘究竟是怎么回事?’大夏天子脑中转着疑问。都已经是打下盐州的第二天了,照常理,应该列队入城才是,夸功耀武得及时来做,否则就失了提振士气的好机会,但他侧耳细听,却没有听到任何动静。

他这一天,几次想出去看一看究竟,却都在御帐门口被人拦了下来。帐外的守卫,都是他的母亲和舅舅亲自选定,全都只对太后和国相唯命是从。

怒火中烧的在帐中发了半日的闷气,嵬名秉常终于恍恍惚惚的在铺了羊皮的软榻上睡了下去。

不过他没能睡得太久,很快就在睡梦中感觉到营帐里有了异样动静,让他猛然间惊醒过来,坐起了身。

帐帘刚刚放下,眼前模模糊糊的有个人影应当是才走进来。黑暗中,看不清他的身份,但随即一点红光亮起,将帐中的几根蜡烛依次点燃。跳动的烛光照亮御帐,黑暗中的身影便暴露在嵬名秉常的眼前。

出现在御帐中的不是这两年来所熟悉的任何人,而是一个关系略远的宗室,在叔祖嵬名浪遇之后,统领嵬名家的主力。

“嵬名济!?”西夏国主又惊又怒的叫着这个令他切齿痛恨的名字。要不是他的支持,他母亲也不会这么容易就将自己囚禁。

“正是嵬名济。”嵬名秉常只听到哗哗的甲叶声响,嵬名济就在自己的面前跪了下来:“微臣叩见陛下。”

秉常心中惊疑不定,嵬名济这时候来见自己,完全不合常理。但他的心中也有一丝希冀,就是因为不合理,才让他有了希望,“你夜里过来,就只是为了叩见朕?”

“不是。”嵬名济垂着头,沉着声:“还请陛下节哀。梁乙埋狼子野心,试图夺权篡位,刺杀了太后。”

“……你说什么?舅舅杀了我母后?”秉常不可思议的瞪大眼睛,他根本不可能相信,怎么能用这么正经的语调说笑话。

