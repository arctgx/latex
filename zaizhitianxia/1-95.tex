\section{第38章 逆旅徐行雪未休(五)}

【新年新气象,祝大家新年快乐。今天第一更,请把红票当红包发过来】

韩冈抱拳回礼:“王兄弟于在下有救命之恩,又一同历经艰险,乃是刎颈之交。他的信中既然言及在下,也免不了夸赞过头了一点。”

“哪有的事!玉昆太自谦了。”种建中很亲热拍着韩冈到肩膀,重复着,“玉昆你实在太自谦了!”

种建中看看与韩冈一桌的同伴,路明仍惊魂未定,种建中过去拱拱手,“兄台,方才对不住了。”又冲刘仲武一抱拳,打了个招呼。回头来对韩冈道:“玉昆,先生已入京师,我们同门兄弟各自星散,如今是难得一见。难得相见啊……不如拼作一桌坐着谈吧。”

“那是最好!”韩冈很干脆的点头。唤来驿卒,将两张桌子拼在一起。重新上了酒菜,六个人便坐在了一起。

种建中向韩冈介绍着与他一起的中年人:“这是小弟四伯,正任着庆州东路监押,如今缘边无事,便告了假出来。”

种建中的四伯与种建中和种朴都有着几分相似,就是气势更加沉稳,韩冈行了一礼:“韩冈见过种监押。”

种四则拱手相回,吐出两个字:“种詠。”其人惜字如金,看起来种詠比起李信还要沉默寡言。

种建中心中有些奇怪,韩冈行的礼节比他四伯种詠要更重一点,这是也许因为韩冈与自己是同学,但说话却不是晚辈见长辈的口吻,而且韩冈还在驿馆里占一张桌子。难道他已经得了官身?!种建中压下心中惊异,试探的问着:“不知玉昆今次来京兆府,所为何事?”

韩冈直言道:“从秦州来的,准备进京去。”

“赶考?”种朴话刚出口便摇摇头,“这时候赶考早迟了。”

韩冈瞥了路明一眼。“是去流内铨应个卯。”他淡然说着,“新近受了秦凤路的王机宜荐举,在经略司中奔走。”

如自己猜测中的一样,韩冈竟然已经得到了官职,种建中惊讶之余,也为韩冈感到高兴。他斟了满酒,向韩冈敬道:“玉昆,恭喜你得荐入官,实在是羡煞我等!”

韩冈举起杯:“不敢当,小弟只是先走一步。以彝叔之才,得官是易如反掌。日后必能后来居上,名位当远在小弟之上。”

两人对饮了一杯,一同坐下。韩冈问道:“彝叔你呢,来京兆府又是何事?”

“刚从南山老宅回来。今年是先祖父二十五年忌辰,家父和几个叔伯都从外地回来了,昨天才刚刚散掉。”

“那前些日子,缘边几路的名将岂不是少了一半?”韩冈半开玩笑的恭维了一句。

“玉昆说笑了。”种建中和种朴哈哈大笑,连有些严肃的种詠,也免不了脸上带起了一丝笑意。

种世衡儿子生得多,自身立得功劳也多,他的八个儿子都受了荫补,分散在陕西各地为官。

如今在关西,种家将威名赫赫。最响亮的,便是夺占绥德,如今正在前线参与横山战略的种谔种五郎。而鄜延种家如今的家主,老大种诂少年时不肯为官,把荫封都推给了兄弟,宁可学着叔祖隐君种放的样儿,隐居在终南山中,时称小隐君,后来因为一桩种家的恨事,不得不出山,如今是原州知州。而老二种诊,此时则是环州知州。

绥德是边塞,原州是边塞,环州也是边塞。种谔在鄜延、种诂在泾原、种诊在环庆,种家兄弟中名气最大的三人都是在对抗西夏的最前线上奋战,故而时称三种。

种詠的功绩名气皆差了一等,但也是庆州东路监押,还是濒临前沿。至于其他三个种家兄弟,也一样是领兵在外。鄜延种家,在关西将门中,算是稳坐在头把交椅上,远远压倒曲、姚、田等其他将门世家。

“不过绥德那里最近走得开吗?”韩冈问着,“不是听说最近西贼在那里又有什么大动作了?”

种建中眯起眼睛,笑道:“玉昆你这是代秦凤路的王机宜问的?”

“河湟那边的事连彝叔你都知道了?”

“同在陕西,横山要打,河湟那里也要打,怎么会不知道?”种建中笑着解释道,“小弟最近在五伯帐下学着做事,也算是历练一下。”笑声一收,脸色也微沉了下来,“就是最近清闲了许多。”

“是因为郭宣徽?”郭逵与种谔的恩怨,在关西从来不是秘密,或者说官场上的纠葛,永远也不可能是秘密。前面种建中只提王韶,却不提李师中,摆明了对秦凤官场同样也了解甚深。

“还是叫他郭太尉吧。”种朴不爽的心情比种建中还要明显。种十九只是种谔的侄儿,而种十七可是种谔的亲儿子。

韩冈听着生疑,按民间习惯,高级将领都能尊称一下太尉。但在官场上,便不会如此。

“难道郭仲通又升官了?”问出口的是路明,他并不像韩冈那般说起话来都要思前想后,想问便直接问起来。

种朴看了路明一眼,又看看刘昌祚,方才光顾着跟韩冈说话,却忘了问候一下他的同伴。他起身道了声不是:“方才失礼了。还没问过二位的高姓大名。”

刘仲武和路明连忙起身。鄜延种家威震关西,两人都不敢怠慢。通了名,互相敬了几杯酒,一番纷扰后又重新坐了下来。路明又提起方才的话题:“郭仲通是不是又升了官?”

郭仲通就是郭逵的表字,他做过陕西宣抚,做过枢密院同签书,做过宣徽南院使,还有个检校太保的衔头,在大宋百万军中,算是头一号的人物。再升官,还能升到哪里?

“升做检校太尉!所以现在是郭太尉了!”种朴悻悻然的说着,检校官十九阶,都是给高官的荣誉加衔,而检校太尉是第二阶,上面只剩检校太师一职,比起检校太保要高两阶,标准的加官晋爵,“天子甚至颁下手诏,‘渊谋秘略,悉中事机。有臣如此,朕无西顾之忧矣。’”

天子下手诏嘉奖,这可是了不得的荣誉。韩冈问道:“是因为看透了西贼打算用塞门、安远二废寨交换绥德的阴谋?”

“还有隐了诏书,没有让绥德城被火给烧了。”种建中很直爽,不会因为不喜郭逵,而不提郭逵在绥德之事上的功劳。

种谔奉密旨兴兵夺取绥德,惹怒了执掌兵事的枢密院。种谔本人被贬斥随州,而传递密旨的高遵裕也被左迁。枢密使文彦博甚至在朝野中大造舆论,以绥德地理位置不利防守为由,蛊惑赵顼下诏焚毁绥德。这一切,都是因为天子密旨侵犯了枢密院的职权,文彦博无法攻击天子,便只能打压种谔。烧了绥德城,种谔便是劳而无功,天子赵顼则是小小的丢了把脸,吃过教训后,想必不会他不会再绕过枢密院,而给前方将领颁下密旨。

但郭逵此时正好调任鄜延,诏书到了他这边,便传递不下去了。郭逵将诏书藏起,反而上书力谏绝不可放弃绥德城。比起枢密院中如文彦博这样最擅勾心斗角的文臣,宿将郭逵对绥德的评价当然更为有力,赵顼追回诏书,绥德城便也因此留在宋人之手。

韩冈叹着:“加官晋爵,又得天子手诏,郭太尉当真是炙手可热。”

“如此下去,五弟在鄜延恐怕再无立足之地。”种詠则忧心冲冲的说着。

而停了一阵,种建中心情却变好了不少,笑着说道:“玉昆,别幸灾乐祸。郭仲通可不止升个太尉,本官也改地方了。”

改地方了?韩冈听着便愣了一下。

郭逵是正任的静难军节度留后,标准的正四品,本官再上一级,就只剩从二品的节度使一阶【注1】。但节度使一般是退职的宰相,或是亲近的宗室、外戚才能获得的位置。武将一般得等到死后追赠或是致仕加赏才会又机会染指。要不然,就要立下让世人无话可说的战功,譬如南征北战立下汗马功劳的狄青那般,而郭逵还不够资格。

就像州县有望紧上中下之分,节度军额也有高下之别。比如北宋几十个节度军额中,最高位的是归德军,过去的宋州,如今的南京应天府【今河南商丘】。当然,这个军额绝不会给人,因为这是太祖赵匡胤曾经的位置,而大宋国号也是来自于此,应天府之名同样来自于此。

而郭逵的静难军是邠州,就是路明的老家,并不是重要的节度军额。为了酬奖郭逵的功劳,将他的静难军节度留后移到位置更高的节度军额也是应该的。

注1:依照北宋的武官官制,武臣第一阶是节度使,第二阶是节度留后,前者是从二品,后者是正四品,但两者之间,被没有正三品、从三品这两个品阶的官职,而节度使往上,也没有正一品,从一品两阶官职。节度留后往下,便跳过从四品,为正五品的观察使。再下,是皆为从五品的防御使、团练使和刺史。以上正任诸使号为贵官,同一朝中,领军武将能得到贵官的,只有屈指可数的数人。

