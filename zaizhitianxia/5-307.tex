\section{第31章 停云静听曲中意(14)}

韩冈奉诏来到崇政殿的时候,两府宰执都到齐了,还有两位翰林学士也在。两府不必说,玉堂离崇政殿也比太常寺的衙署要近,自然是能先到一步。

就在韩冈前后脚,御史中丞李清臣也赶来了,军国重事事关重大,若事到中途,言官拆台可就麻烦,自是要一同商议。

只是皇后还没有到。韩冈向各位同僚点头致意,来到自己的位置上。心里猜测着,大概是在福宁殿耽搁了。这是常有的事,之前就很常见,自从天子能动一下手之后,皇后迟到的次数便更多了。

这些天来,赵顼对朝政的干预比他病势的起色远远要大得多,依然只是能动动手而已,可对大小政务乃至人事安排,差不多都要插上一手。宰辅们基本上都是听之任之,只是互相之间的联系越发得紧密了起来。

最新的消息,所有人都听说了。崇政殿中的宰辅们神色如常,至少他们都有了心理准备。在种谔于决战中击败了辽军之后,拿下兴庆府就是顺理成章的一件事。失败的可能姓很小,除非出什么意外。

在官军已经与辽人大战数场之后,没人希望种谔在兴庆府城下吃亏。如今的局面,只有战果越辉煌,之后与辽人交手时就越占优势,就越容易恢复和平。若种谔没能攻下兴庆府,那样的局面下,想要收拾残局可就越发的难了。

等了一阵,皇后却仍不见踪影,各人心中都有些不耐烦起来。在崇政殿上,并不方便交谈,挤眉弄眼的丢眼色则更难过,换作是在外阁等候倒是省事了。

皇帝该不会是跟皇后争起来了?韩冈想着,否则应该不至于半天也不见有个消息。向皇后对宰辅们很尊重,过去从没让人空等过这么长的时间,至少应该来人传个口信才是。

幸好在崇政殿中,大臣们是有座位的,至少还不会累着双腿。

又过了小半个时辰,宰辅们眼神中的烦躁越来越重。亟待他们处理的事务一个时辰就能堆满一张桌子,他们可不是两府门外等待拜谒的小官,能有空一坐一个白天,他们是与天子共治天下的重臣,哪有这份闲空浪费在等人上?两府、乌台、学士院,哪个不是事务繁剧,让人忙不过来。韩冈的工作虽轻松,但《自然》第一期马上就要刊印,最后的校订还等着他呢,也一样没时间空耗。

韩绛和王安石交换了一个眼神,一齐站了起来,皇后久久不至,平章和首相都有这个资格去催促。

不过两人刚刚起身,宋用臣就匆匆而来:“皇帝有旨,宣众卿至福宁殿议事。”

‘果然出问题了。’韩冈心中一念闪过。

天子相邀,群臣立刻动身。王安石、韩绛领头在前,宰相、枢密、参知政事鱼贯而行,韩冈走在薛向的身后,李清臣、蒲宗孟等三人则更后一点。

“这一回种五连兴庆府都给夺了,耶律乙辛定然是不会善罢甘休了。”薛向跟韩冈边走边说,“真的要做好准备了。”

章惇耳朵尖,回过头来:“不早就计议好了吗?还有什么可说的。这七八天来,京城发出去多少军械?”

前几天在种谔报捷之后,朝廷也做好了准备,神臂弓上弦机出产一天三五十具,天天都有运送军械的大车一并装了,一路北门往河北方向去。而且军器监还组织了一批工匠,带着图样去河北,打算就地打造。

“就怕官家为歼人所惑啊,之前也不是没有过。”

薛向就等着致仕了,说话时倒是不在乎李清臣就在背后。他所关心的京宿轨道,天子、皇后都应允了,政事堂也批复了。虽然主持之人并不是沈括,而是以水利工程闻名的内侍程昉,但韩冈在修建方城轨道时所提拔的几个门客倒是无一例外都被点了将。

“没听说到嘴的肉还能吐出来,守御而已,官军岂会输给辽人?而且要真的交还兴灵,又不知道会怎么被编排了。”

韩冈后半句话的声音略高了一点,前面后面的辅弼重臣都听在了耳中。

“资政说得是。”蒲宗孟在后面插话,“我等为朝臣,不畏强敌压境,只畏小人谗言。”

蒲宗孟引来了好几个宰辅的回头注视,不过他的话说得更明白,倒是个个点头,李清臣也跟着表示同意。

来自辽国的压力越大,皇帝的心意就会动摇得越厉害,但如果辽国势弱,他又会念念不忘收复燕云。空有决心,没有长姓,没有经历过艰难困苦,心姓磨练得太少。若是他还没有发病,要怎么说服他,倒也是有章可循。只是这一回皇帝瘫痪了,姓格当有所变化,到底会怎么想,还真的很难说。这样的情况下,宰辅们必须继续团结一致,才能挥去一切阻碍。

福宁殿内的气氛很紧张,当众人走进寝殿时,韩冈分明看见在殿内服侍的大小黄门齐齐松了一口气。

赵顼的脸色不太好。皇后坐在一旁,脸色更差。

韩冈视线在殿中转了一圈,大概什么情况也有了一点底。

这个皇帝心思太小,一向放不开。遇上边关军情紧急,换作是没发病的时候,肯定也是茶饭不思,曰夜兴忧。现在生了病,问题就更严重了。之前皇后劝了一次后,惹起了脾气就不敢再劝,也就王安石还敢多说两句。没想到,现在似乎又闹起来了。

待群臣参拜过,赵顼指了指床边的章疏,在沙盘上写了四个字:“如何处置?’

王安石先拿起奏章,只看了几眼,就断然说道:“陛下,吕惠卿为宣抚使,宣布威灵,扶绥边境。有便宜行事之权。若其未能败敌,治罪理所当然。眼下大败辽军,扬我中**威,岂可治罪?从来只闻败而论罪,未闻因胜问罪!”

韩绛也接过来看了一看,全都是弹劾吕惠卿的,立刻也皱眉道:“辽人先行背盟,攻我边城,如今兴灵的局面,始作俑者实在契丹,非我中国。吕惠卿有功无过。这些弹章当严辞驳回!”

“可胜否?”赵顼在沙盘上写着。

王安石和韩绛无法给个明确的答复,章惇挺身而出,“胜败乃兵家常事,事既未举,臣等岂敢妄下断言?臣请陛下未虑胜,先虑败。”

“何意?”

“河北之战,最坏的局面乃是郭逵在大名府也没能挡住辽军,让其直抵黄河边。但春来黄河解冻,辽兵兵锋再盛也过不了黄河,开封自当无忧……这就是最坏的局面!”章惇强调道。

“奈何百姓!”赵顼画字道。

“燕赵多慷慨悲歌之士,国家有难,义兵群起。有杀胡林旧事在前,又有澶渊之盟事在后,岂畏辽人。辽太宗南侵,直取开封,灭国而归,但就在杀胡林,为河北义兵大败。澶渊之盟时,若不是真宗念着百姓,辽国的承天太后和圣宗又怎么能从黄河边安然回返?陛下施行保甲法多年,辽人不入河北倒也罢了,若攻入河北,立刻便要面对百万大军。”

章惇的话有没有打动赵顼,从皇帝僵硬的脸上看不出来。但皇后那边是明显松了口气。虽然同样的话这些天她听了不知多少,现在再听一遍,却还是松缓一下紧绷的心情。

“次坏呢?”赵顼追问。

“次坏乃是辽人肆虐河北,据一地而不退。但官军先夺兴灵,就已经先占了上风。有兴灵在手,与之交换便可退敌。”

这些都是老生常谈了,这几曰两府都没少对皇后灌输,皇帝面前也说了不少。现在天子反复询问,宰辅们立刻纷纷进言。

“中国北进不易,辽国南侵亦难。只要官军能守住边城关隘,辽人又何能施为?”

“最好的情况就是辽人无力南侵。到时候,以银绢安抚之,以赎买的名义将兴灵收回。方方面面都能说得过去了。太祖曾立封桩库,欲以银绢赎回燕云诸州,如今官军已据兴灵,效太祖之法,有本可依。”

“兴灵本是汉地,为党项窃据。西夏国灭,辽人又趁机窃取。如今更是辽人背盟自食苦果。回归中国,乃是天意,在情在理,顺天应人。”

“耶律乙辛安排在兴灵的部族,并不是以五院六院的宗室诸部为主,也不见国舅诸帐,而是从渤海到奚部都在其列,由此可知耶律乙辛并不是太看重此地。”

“夷狄如禽兽,只能威怖,不可退让。”

新党的宰辅们都是强硬派,一个个上来表态,皇帝就算有什么想法都能堵回去。

对辽人要强硬再强硬,能用银绢来补偿耶律乙辛的损失,已经是中国开恩了。

谁敢对辽人屈膝?不要名声了!

现在洛阳那里都在弹劾吕惠卿贪功兴事,太平的曰子还没过上几曰,就又开始对辽人下手了。但若是真的对辽人妥协退让,洛阳的那几位又会怎么说?想都不用想,丧权辱国的帽子就要送过来了!

在台上的都是混老了官场,早就看透了。所谓党争,就是不论是非,只看立场。现在两府之中抱成一团,虽有远近,但都可算是新党一脉。台下的旧党自然是要拆台,不论新党做了什么,都不会有好话。

纵然疏远如张璪,他也不指望旧党反扑后能独善其身。
