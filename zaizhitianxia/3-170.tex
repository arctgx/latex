\section{第44章 岂惧足履霜(下)}

腊月底的时候,白天多了鞭炮声,夜空中则渐渐多了烟火的五彩斑斓。无论白天和黑夜,人们脸上的笑容和空气中的硫磺味则是一起增加了。

昨日送过了灶神,家中已经给装饰得喜气洋洋。

各个衙门此时都开始放假了,军器监也不例外。从上到下,除了监库的军卒和官吏,都一起放了年假。不过工匠们都住在兴国坊中,家人也一起受着严密的监视,即便放了假后,也不能随意出外走动。

韩冈却是成了大忙人,但凡有些交情的,这些天都过来上门拜访,连王韶都遣了人过来打听,问着铁船之事。韩冈则只承认理论上可行性,却没有说一定能造的出来。

尽管如此,新上任的判军器监准备打造铁船的消息,依然在京城中甚嚣尘上。他不承认,那是谨慎,但如今忽然流传开来的手抄本上,可是白纸黑字的写明了铁船浮水的原理。即便是说给些乡愚听,最多费点口水就让他们明白了,很浅显的道理,证据也随处可见,只要将瓷碗丢进水里就能了解,过去却没有人去为之深思,并加以推演。

这就是格物致知的运用,大道至简,却在百姓素日所见之处。

这个年节,东京城上上下下,都在期待着铁船的出现。

“三哥哥,当真要造铁船?”韩云娘给韩冈磨墨的时候,突然就问道。

上下一色的鹅黄色襦裙,外面套了一件夹了棉的半臂,纤细的腰身则给巴掌宽的腰带衬托了出来。

韩冈放下笔,抬手亲昵的刮了她一下鼻子,笑道:“怎么家里面也在传了?”

韩云娘秀目含嗔的横了韩冈一眼,才说道:“外面都在打听,隔壁陈员外家的李娘子今天也来打听。”

云娘所说的员外,不是外面烂大街的、店铺招呼客人时所称呼的员外,而是货真价实的虞部员外郎,品阶是与韩冈平级的正七品,管着在京库务的陈燊。

陈燊与韩冈做了邻居,抬头不见低头见,见面时互相都会打个招呼,只是并不亲近。不过两家的女眷走动得倒是频繁,且还是陈燊的夫人主动贴上来的。这等夫人外交的手段,也让心明眼亮的韩冈在妻妾面前,为之笑叹过几次。

“那你们怎么说的。”韩冈问道。

“姐姐吩咐了下去,对外面都说不知道。”云娘直接称呼为姐姐的就是王旖。至于对周南和素心,则是喊着南娘姐姐和素心姐姐。“姐姐今天回李娘子,也说妇道人家只知家中事,外事不问。”

“这事做得好。”韩冈听着就说好。王旖的吩咐的确也不算差了,家中之事都能帮韩冈考虑着,省得他处理外事的同时,还要烦心家中给他捅娄子。

“其实姐姐回头也跟我们说了,家里的人都忠心得很,都没人会向外说家里面的事,她也只是多提上一句。”

韩冈现在家中所用的仆婢,虽然还算不上家生子,但基本上都出自于关西,是从投奔到韩家名下的庄客家中带出来的。想要收买他们,可没那么容易。仅有两名老仆是从开封雇佣,不是让他们干活,而是教着韩家的仆人们符合京中官场习俗的礼节。另外还有一个老宫人,仁宗时曾在宫中做了二十年,韩冈雇了她作为教导韩家使女们的教习。

其实如果不是韩冈在熙河路的地位,要想招揽一两百户庄客,根本不是几年之间就能完成的。往往都要一两代人,或是二三十年时间来积累。也只有韩冈,在大战中立下了赫赫功名,而后直接接受了一批残疾的军汉,连带着他们的家人都投奔到韩家名下。还有当初护卫他的一干亲卫,也有四分之一从军中退了出来。同时这也是靠了熙河路是新辟之地,韩家能大起庄园,同时将庄子周围的土地都纳入名下,换作是国中腹地,想买个百十亩连成一片的土地都难。

“多说一句就对了。”韩冈则是对王旖的做法大为赞许,“纲纪都是一步步败坏的,耳提面命才能让人时刻小心。如果太过于放心,迟早会出乱子。”

“来,磨墨!”他向云娘一招手,“今天得将名帖都写完,过年还要送人呢。”

到了午间的时候,外出了两日的冯从义回来了。

进门后,韩冈就问道:“都让人准备好了?”

冯从义点着头:“表哥放心,都已经准备妥当。城外西面的那间库房,安排人住下一点也没问题。如今年节,附近的几乎都空着,不怕走漏风声。最多一个月,飞船肯定能造出来。”

韩冈将热气球起名做飞船,就是要确定腾飞的原理来自于大气给予的浮力,是飞在天上的船,道理如一,只是外在不同——理一而分殊。

“这件事关键是保密。”韩冈叮嘱着,停了一下,更进一步的明确说道:“在试飞前一定要保密!”

“表哥放心。”冯从义拍着胸脯道:“选的不都是自家的庄客,嘴巴哪敢不严?决不会对外泄露半点!而且小弟也会去盯着,绝不至于有差错。”

韩冈一贯的厚赏重罚,仆婢的家人在庄子上都有一分优待,但相对的,如果犯了错,惩处也绝不会轻。不是肉刑,那样太粗率,也违反律法,而是单纯的株连。如有重过,绝不仅仅是个人受到责骂或是罚没月例,直接就会连累家人。

韩冈还算是好的,真正让人害怕的还是那些以军法治家中的士大夫,比如王韶,他对仆婢的管束就以号令森严著称。而吕惠卿,也是有名的治家严谨。无论有没有过军旅经验,文臣们都喜欢用着军中的做法,动私刑,杖杀仆婢的事时有耳闻。

而顺丰行在京城的店铺中,也有几人来自于关西,被安插在紧要的位置上,监督着京城的雇员。加上护卫着,都是得用的干才。可以守着秘密,又能帮韩冈将事情做好。

“对了,表哥。”冯从义凑近前来,很有些紧张,“如果当真造出飞船的时候,他们会不会给关押起来,就像军器监的那些工匠一般?”

韩冈微微一笑,浅淡的笑容却能安抚人心,“原理都出来了,还有谁学不会的?飞上天的东西谁都能造,没看到外面的挂着的一排灯笼吗?”

对韩冈将孔明灯当成普通的彩灯,一排挂在栏杆上的行为,冯从义实在不知该说什么好。别人家的灯笼是向下吊着的,而韩家的灯笼却是向上吊着。不过看着倒挺漂亮。就不知道来拜访韩家的谁有这个见识,能从这里看出些端倪来。

韩冈又问着表弟:“义哥儿,若此次飞船当真成功,要不要为兄将你的功劳报上去,也可以一并受赏。”

“多谢表哥,不过还是行商更适合小弟。”

冯从义有着足够的自知之明,靠着高家的关系和韩冈的支持,他已经有了一个官身。即便因功受赏,也不会有什么区别。而受赏过后,说不定就要一辈子管着制造飞船的事,冯从义怎么可能会愿意?

“只是飞船当真有那么大的用处?”冯从义不怀疑飞船能不能成功,韩冈将一番道理说得简单明了,再透彻不过,而且还有孔明灯在外面飘着。若当真飞起来,肯定轰动天下。但韩冈想靠飞船得到的,冯从义却没有把握。

韩冈笑了,点了点头,道:“你还是到时候看吧。”

对韩冈来说,铁船和热气球两个都有那就最好,可以从多方面证明浮力原理的正确性。

不过铁船要见功,难度很大,焊接的问题就不说了,变通的办法总是有的。可要想造船,耗用的人力、物力都不是小数目,至少在军器监内他要做到如臂使指才行。可惜韩冈做不到。谁让吕惠卿现在是参知政事,县官不如现管这句话,在如今的军器监内可行不通。判军器监怎么跟执政比。而且还有一个曾孝宽在,他若反对韩冈的命令,韩冈也别无他法,只能将官司打到御前去——这样他韩玉昆就是个笑话了。

制造热气球的难度,则要比铁船小得多。再怎么说,都是在拿破仑时代之前就出现的东西。制造起来不会要求大量的人力物力,成本绝对要比铁船要低。从技术角度来讲,这个时代也完全可以胜任——并不要后世经过改进的热气球,能环球航行的那种。只要能载着人浮上天空,飘上一两刻钟就足够了。

舟船古已有之,即便不是木头而是该由钢铁打造,给人的震撼仅仅是一时的。在没有出现蒸汽机的时代,铁船即不如木船灵活,又不如木船物美低廉,肯定会有人说,造此无用之物,是在浪费公帑。

但飞天之梦,又有谁实现过?!

比起铁船,热气球的出现给人的震撼可是要大上千百倍!

而且热气球一出,空气的物质性便可以由此得证。

虚空即气——不,确切的说‘气’更应该写作‘炁’——这个概念,将会深入人心,张载进京的道路也由此铺平。

至于韩冈本人,在士林中有《浮力追源》张本,而在民间,他身上的光环则更会添上光彩夺目的一圈。

一举多得……

……一本万利!

