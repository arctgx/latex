\section{第29章 浮生迫岁期行旅(九)}

腊月深寒,黄河已经彻底冻结。

黄河冰面上,用木板和稻草席铺起了一条道路。行人车马络绎不绝在这条道路上络绎不绝。

不过当辽国正旦使的队伍开始踏上河上道路,南来北往的行旅便全都给赶得远远的。

萧禧每次来南朝,本意都只是想敲笔钱回去,增加个十万八万的岁币,就像南朝仁宗时因困于西夏立国而不得不增加岁币那样。但谁知道熙宁八年的时候,左敲敲右敲敲,梆梆梆的一阵竹杠敲过去,最后就变成了割让土地。

辽人不缺地,只缺钱。弄回些地皮,也就涨涨面子。不过这对萧禧倒是都一样,土地也好,银绢也好,不论从宋人那里捞回什么,都是他的功劳。

宋人对萧禧是戒惧,而对副使折干则是鄙夷。粗鲁的北方蛮子,当然不会被自命天朝上国的子民放在眼里。

宋人隐隐中透出来的鄙夷,让随行的副使折干脾气变得火爆起来:“南人就会装模作样,看不顺眼就说,说不通就砍,明明看不顺眼还陪着笑,是要讨赏钱吗?!”

折干一通火,让周围的宋人又离得他远了一点。他这个粗暴的脾气,倒是符合宋人对辽人的认定,南下以来,监视他的视线已经不一开始少了许多。

萧禧权当没看到,若这名奚族人的心思跟他的外表一样粗鲁,尚父就绝不会让他南下。

不过宋人离得远了,倒是方便他说话,与折干并辔而行,萧禧低声道:“胡里改是告哀使,算他的行程,这时候就该回来了,可知路上为何没碰见他们?”

“……当然是宋人在其中捣鬼。”折干冷哼一声,“多半是从另一条路回去了!”

萧禧紧接着问道:“那宋人是为了什么才这么做?”

“要能打听得到就好了。”折干叹了一声。

他装粗装到自己都想吐,但还是没能让宋人派来服侍左右的仆役松懈一点。这些人一个盯着一个,从来不落单,根本就别想打探得到半点口风。

每天到了驿馆,外面少说也会站着三五百以守卫为名而派来的精锐禁军。消息递不进来,也传不出去。完全是两眼一抹黑啊!

“其实到了东京,自然就知道了。”萧禧很放松,“而且这是欲盖弥彰,遮遮掩掩的,不就证明了有什么事害怕我们知道?”

“怕就怕上了殿后还不清楚出了什么事!宋人做得出来。”

“那就直接多要点好了。”萧禧咧开嘴,常年吮骨吸髓的牙齿白森森的,“看看宋人的应对,也就能知道他们有多心虚了。”

……………………

苏颂正在埋首于公案之上。

这些日子以来,虽然朝中各种各样的事一个接着一个,但《本草纲目》的编纂工作并没有被耽搁,说起来真正办事的还是下面的编修们,韩冈和他苏颂更多的工作只是在审核。

不过苏颂手上的笔就没有停过,才半日功夫,呈交上来的草稿,已经被改得面目全非。

韩冈的生物树还是太粗陋了,要想天下千万物种门纲目科这样排下来,不知要穷几百年之功。眼下只能针对药材,而且还是错漏百出。

不过《自然》期刊即将发布,可以合天下之智。韩冈准备给所有的物种都配以发现者的姓名,以其为正式的学名。也就是在物种之名后,加上发现者的姓名作为后缀。

比如苏颂,若是发现了家里飞进来几只很特别的山雀——比如说脸是紫的——只要能将其习性和特别之处给确定,并证明这是一个物种共有的特征,可以遗传,便能初步将之命名为紫脸山雀·苏颂,简称苏氏山雀。同样的道理,韩冈在家中后院发现了一片特别的蕨菜,将其独特性和遗传性加以证明后,也可以叫做韩氏蕨菜。

以名诱之,想必天下士子中将会有不少人趋之若鹜。当然,为了防止有人随便找根草就说是新物种,接下来还有认证的环节,要给出批量的标本,给出发现地等一系列的证明,并交由权威认证。暂时是《本草纲目》编修局,但等到《自然》名气大了起来,就可以组织一个研究姓质的会社。虽然看起来很粗糙,实际上也因为是草创而无成规,但终究会逐渐进步的。

不过这个会社真正组建起来后,就不会再局限于区区药材或是生物了。苏颂就有打算在其中组建一个以天文观测的分社。就他所知,韩冈也有这方面的打算。

正提笔修改着文字,外面忽然来报,说是宫里派了小黄门来传韩学士。

苏颂抬头看看对面空着的位置,这可还真不巧,“让他进来。”

“苏学士。”面对在朝中名望高峻的苏颂,小黄门恭恭敬敬,甚至战战兢兢,“小人奉皇后懿旨,招韩学士上殿议事。”

“玉昆他去了都亭驿。”

“都亭驿?”小黄门的脸就耷拉了下来,那可是要跑到皇城外找人了。万一中间走岔了,或是韩冈根本就是寻个借口出去,还不知要到哪里去找。

“方才政事堂传信过来。说是辽使今曰晨间已过黄河,明天就便至京城。所以方才玉昆就去了一趟都亭驿,看看里面的准备得怎么样了。”苏颂略略解释了一句。

“小人知道了。多谢学士相告。”小黄门急着找人,向苏颂行了礼后,就跑着走了。

苏颂却感觉有些奇怪,韩冈上午就在崇政殿,现在又派人来传,难道出了什么事。

……………………

韩冈倒是就在都亭驿。

馆伴使顾名思义就是在馆中陪伴客人,在情在理,也得先来一趟驿馆。熟悉一下馆中的官员和规矩,也省得沟通不畅,出了意外。

当杨戬找来的时候,他正听着都亭驿中官吏的报告。不过朝廷的事要紧,听了懿旨后便立刻起身。

到了殿中,除了一个避位的韩缜,其余宰执们都在。而向皇后想问的,是泾原路和环庆路。那边一直没有更进一步的消息,让皇后很担心,想要问一问韩冈的意见。

皇后这惊弓之鸟的感觉,跟熙宁八年时的天子差不多,不过好歹比那时的皇帝更容易安抚。

“泾原路和环庆路那边没有消息,的确让人担心,但毕竟有良将坐镇,当不须担忧。反倒是银夏路……”

皇后的担心全无来由,让他哭笑不得。韩冈倒是觉得值得担心的另有其人。

“种谔不就在银夏?”向皇后疑惑的问道。种谔可是闻名万邦的名将帅,不比郭逵差。

“……臣是怕他功高而骄,对辽人不加提防。”

被韩冈这么一说,向皇后立刻就担心了起来。

说种谔功高盖世肯定过誉,可以种谔历年来的军功,除去开国的那一批名将外,基本上也就在三五人之列了,可以跟狄青、郭逵争一争头名。这样的良将,若是以功高自矜,小瞧了辽人,的确是让人担心。

“那就由政事堂下堂札命其谨慎行事。”向皇后吩咐蔡确道。

韩冈暗暗松了一口气,好歹糊弄过去了。他担心种谔,不是担心他守不住银夏,而是担心他又想立功。韩冈太了解种谔,都打了多少年交道了,这时候他多半又转着主意想要从辽人身上挣一份军功了。

青铜峡蠢蠢欲动的党项人,搔扰韦州的契丹人,在这里看是危机,但在种谔眼中,却是实实在在的机遇。

种谔是个天生好战的疯子,也许这么说会很过分,但若是没有战争,他多半会活不下去。若是换个时代,他多半会高喊着‘诸君!我喜欢战争!我很喜欢战争!我非常喜欢战争!’,而带着手下的将士席卷每一处战场。

虽是被笼头约束着,却是没有一刻不想挣脱束缚。当年在平夏之役前,就有人说过种谔不死、战事不止,如今这番话依然可以贴在种谔的脑门上。

但这话不好说给别人听,韩冈也只能埋在心里。

“种谔的侄儿好像就是盐州知州吧?”向皇后又问。她模模糊糊有些印象,这几天她看了不少地方上的人事安排。

章惇点点头:“种建中现在是权发遣盐州知州,银夏西路都巡检。盐州驻扎了一个将三千兵马,新近又加固了城防,不虞辽人侵袭。”

“这种建中能力如何?”向皇后问道。她担心种建中是靠了种谔才有了这个位置。

“良将之才,而且还是张文诚的弟子。”

向皇后也想起来了,前几天,好像就有说过。她看向韩冈。

韩冈点了点头:“种家诸子,种谔为首,种诂、种谊亦是良将,其余兄弟同样深悉兵法,而下一代的种朴和种建中,皆是号为将种,在过去也屡立功勋。”韩冈道,“不过种朴这一代,也就只有种建中,再加一个种师中,余子皆碌碌。比起种谔那一代,还是要差了不少。”

殿上人都听得出来,韩冈这是在帮种家说话,要是种家将的第三代还是人才辈出,那可就让人担心了。

不过种家是韩冈在西军中的基本盘,殿中众宰执都知道这一点,没人想跟韩冈无缘无故结仇。何况他们对种家还的确不了解。

“种谔、种建中毕竟是武将,见识或有不明。”章惇帮着将话题从种家身上引开,“吕枢密之前任职陕西数年,等他上京后,殿下可以向他征询。”

向皇后点了点头。

从行程上,青州的韩绛这时候应该动身了,吕惠卿也多半已经收到了诏命,而更远一点的曾布,则是还要几曰。

要等他们全数进京,恐怕要到明年了。

