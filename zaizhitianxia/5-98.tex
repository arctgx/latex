\section{第11章 城下马鸣谁与守(十)}

【第三更。】

高永能向曲珍的方向望过去,却见曲珍眼观鼻、鼻观口,仿佛石雕一般。之前徐禧几次拒绝曲珍的建议,使得两人的关系越来越恶劣。昨日领军出战的是高永能而不是曲珍,也是因为这个原因。

高永能见状,也只能暗叹一声。没有曲珍的支持,说什么也没用。收敛起眼神,强耐下性子,向徐禧抱拳:“末将遵命。”

“杜靖!朱沛!你二人率本部自西门出战。王含你去北门,符明举你率部往南门。随时准备出城支援。”

半个时辰之后,杜靖和朱沛两将所部已经抵达西门内侧。八千余将士从门边一直排到道路上,人人披甲,手持长兵,阵列严整。

尤其是最前面的两个指挥,同样手持银枪,身上的盔甲擦得闪亮,形制比起鄜延选锋更为精良,内领还是鲜红的锦袄。排开的阵势横平竖直,犹如一块被精工凿制的碑石。

站在阵前,杜靖和朱沛都是一身金甲,披风血红,身侧一面大旗扬起。

杜靖意气风发的对高永能道:“鄜延选锋世所罕匹,勇武当为西军之冠。不过我八百银枪效节,却也决不输人。历年天子观兵,银枪效节军可从来都没落到前五之外。”

在京城中,练兵一向是勤快的,三天两头就要上校场校阅。这不是枕戈待旦,随时准备上阵的勤快,而是因为天子经常会到场观兵,若是在校阅时表现不好,主将的前程也就没了。

练兵的精力都放在装束和阵法上,论起军容军貌,还有阵列运转,西北的土包子,当然无法能跟京中相比。但上阵的本事,是骡子是马,拖出去遛遛就知道了。从来没有上过阵,忽然之间见了血,又被强敌紧逼,高永能还没见过能不崩溃的。

面带微笑的腹诽着,高永能回头看了曲珍一眼。

出现在曲珍眸子中冷漠的眼神让他悚然一惊。没有一点情绪外露,完全是事不关己,就像看着一群倒闭路边的无名尸体。

高永能眼神一动,不意却发现李舜举也在面色凝重的观察着曲珍的反应。还没等他想明白,却见李舜举的视线转了方向,向自己看了过来。

高永能随即垂下眼帘,面无表情的看着自己的脚尖,‘这个阉货,或许不是那么愚蠢。’

……………………

韩冈皱眉看着晋宁军和府州传来急报。虽然并不是盐州那里的情报,但依然不是什么好消息。

“那群阻卜鞑子全都抢疯了。”韩冈面对有关阻卜骑兵的报告,最后也只能得出这样的结论。

阻卜强盗一直都在银夏一带以及河东与西夏的边境活动,这是所有人都知道的。但随着在阻卜人活动范围之内的地区都提高了皆备,受到的攻击便越来越少,而战果也是在几个、十几个的累积着,逐渐超过了一百。

或许再过一阵,就能将他们全数逐走,或是消灭。许多人都是这么在想。

可是就在这几天,大约七八百人左右的阻卜骑兵,竟然设法穿过了葭芦川几处寨堡之间的缺口,往黄河这边抢过来了。六天之内,十七个村落受到攻击,其中三个村寨被攻破,百姓伤亡过千数。

韩冈之前将河东的骑兵调了大半出去,帮助种谔稳定夏州连通盐州的道路,之后朝廷也对此进行了追认。由于并不认为北方的边界需要用到太多骑兵,韩冈的安排,也没有引起太多的反对。

但阻卜人一来,就立刻让掌握在韩冈、李宪手中的骑兵兵力捉襟见肘来。对于以劫掠为目的、机动性极强的轻骑兵,要想追上他们,要么是用同样的轻骑兵追截,要么就是用多倍的步卒合围——当然,最好的办法是两者皆备。

但在缺乏足够骑兵的情况下,镇守在黄河西岸的李宪,现在只能想方设法指挥步卒的围追堵截。而对之前调走骑兵的议论,就一下多了起来。

“这不正合三哥的意。”冯从义笑道,“如此一来,谁还能说三哥对陕西支持不利?河东的骑兵都送给了种谔,闹得追击阻卜骑兵都没了人手。西军和西贼斗了上百年,两边的细作成千上万,河东骑兵进抵夏州,葭芦川一线出现缺口,想必就是西贼传给阻卜人的。”

如果没有百姓的伤亡,能给葭芦川各寨一个教训,使他们提高警惕,倒是韩冈乐意见到的。可现在的情形,让他如何能有好心情。

对于远道而来的表弟的猜测,韩冈既不承认,也不否认,只叹道:“现在只希望他们会贪心到想过黄河劫掠,到时候,就能在几个渡口边,将他们一网打尽了。”

“阻卜在苦寒贫瘠的草原上,一直都被契丹人压榨,如今终于有了出头的机会,哪里敢不尽力?也不是为了党项人,全都为了他们自己,肯定想多抢一些回去。”冯从义笑了一声又道,“不过但凡有点头脑,就不会转着东渡黄河的念头。想来河东腹地抢上一把,别说能不能抢到还是问题,就是抢得心满意足,想回去时也是被堵在黄河边上。”

“义哥儿你行商多年,耳目比愚兄灵通。可知阻卜人的详情……比如部族、头领什么的?”阻卜扰乱河东,韩冈想着要是有熟悉阻卜各部的人那就好了,情报可是关键。

冯从义摇摇头:“没有听说过太多,之前只有偶尔有只言片语传来……阻卜隔得实在太远了,没生意可做,所以一直没有想过去打探他们的消息。这一次小弟还是在听说阻卜南下后,才特意找人询问,但也仅是知道契丹人在草原中央驻屯了数万本族大军,又设立了西北招讨司和阻卜大王府,隔绝东西南北,强行将阻卜分作北、东、西三部。如今南下的就是西阻卜。”

他停了一下,拿起茶杯喝了一口茶水润润喉咙,接着又道:“这一次领军南下的阻卜首领,应该是把绝大多数男丁都带出来了。如果能给他们一个狠的,西阻卜多半会被北阻卜吞并。”

“那耶律乙辛肯定是哭都哭不出来了。”韩冈哈哈一笑,若是能通过云中之地,直接向草原上输出军用资源就好了。加强了与上京道,想必能让耶律乙辛焦头烂额。不过那是后话,可不是现在该考虑的问题。

冯从义点点头:“若在草原上为契丹树立一强敌,的确是大宋之福。北阻卜这些年来一直叛降不定,其下诸部基本上已经统合为一,其部族长名为磨古斯,据称其有枭雄之志。”

韩冈有几分惊异的瞥了冯从义一眼,他口中说是对阻卜不了解,但现在拖去朝堂上,充当一个熟悉北虏内情的专家,多半能蒙不少人。

“究竟是从谁人那里听来的?!”韩冈立刻追问道,“是哪一家的行商?!”

韩冈的急躁,让冯从义笑了起来:“三哥难道忘了,小弟这一次可是从京城来的。”

韩冈先楞了一下,而后灵光闪过,失声叫道:“……枢密院?!”

“自然。”冯从义笑道,“西军这些年往西贼那里派去的奸细数不胜数,而朝廷往契丹人那里派去的细作可是更多。前些年契丹东北的五国部女直叛乱,没几个月,王介甫相公的奏章上就写出来了。不是细作的功劳,还能是谁?”

韩冈不介意从枢密院那里多了解一下敌情,但打铁要靠自身硬,至少要先有击败阻卜人的成绩,这样才方便谈判。他将冯从义找来,也有一部分原因是希望从冯从义这里了解一部分阻卜人的风土人情。

但现在就不好说了,韩冈转移话题:“一旦盐州兵败,契丹必定会趁虚而入,届时银州、夏州亦保全。但官军如今已经收复了沙州。前锋更是抵达了古玉门关。可只要凉州的后路不稳,甘凉之地就不能算是夺回来。”他一声长叹,“放弃了应理城【今中卫】是最大的错误。”

应理城附近就是葫芦河和黄河的交汇处,有道路直通凉州,溯黄河而上可往熙河路的兰州,顺葫芦河往下游去则就是泾原路的原州,往秦凤路的德顺军也有好几条道路

党项军占据了应理城,据有葫芦河口,居于内线,可以四面出击。秦凤、熙河乃至泾原路皆受其威胁,西贼的铁鹞子甚至可以奔袭凉州。而官军占据应理城,接下来熙河、秦凤以及泾原路,便都成为了内地。原本是绵长的防线,但现在只要守住一个点就够了。

冯从义对地理也有所了解,想想的确是如韩冈的所言,“应理城必须拿回来。”

兄弟两人正在说着话,一名亲兵匆匆走近厅中,给韩冈带来一封短笺,看封皮上的落款,是来自种谔。

在夏州和太原之间,韩冈安排了一条驿传的线路,盐州的消息,不论是种谔派出去的斥候还是徐禧派出来的信使,传到夏州都只要一天,而从夏州再传到太原,则只要五天。

韩冈打开密信只一看,瞬息间就变了颜色。冯从义的一双眼睛,清楚的看见韩冈的手在颤抖。

过了好一阵,韩冈才打破沉默,用尽可能平淡的语气对冯从义道:“这是夏州传来的消息,是五天前发出的,说得是六天前的事。西贼围城,城中守军出战,但在城下惨败,王含战死,符明举、朱沛重伤,出战的士卒伤亡近半。”

冯从义也同样脸色大变:“盐州要丢了?!”

