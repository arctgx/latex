\section{第11章 城下马鸣谁与守(15)}

这是一座位于山沟中的村庄。两边高山夹持,中央有一条山涧流淌。村子就坐落在河边。山上山林茂密,而山沟中的平地全都被开垦出来。整齐的田畦中引来了水流,使得村中的田地,大半是上等的水浇地。

在过去,这是个平静富足的村落。村中数百百姓安居,出产又算得上丰富,饮食起居在河东也算是不错了。

村里百姓对此很是满足,他们乐于平静的度过每一天。但这一日,不大的村子里,正弥漫着浓浓的血腥味。

村庄外侧的围墙,有好几处破损,一丈高的土墙完全的垮塌下来,碎石和土块落了满地。靠着围墙缺口的房屋,只剩下焦黑的房梁和椽子,以及几堵熏黑的土墙。

走在路中的,除了阻卜人和马匹之外,完全看不到村庄主人们的身影,只有地上的一滩滩血迹证明他们曾经为自己的家园而奋战过。而村外的田地、林地中,全是三五成群的阻卜骑兵,数以千计。

偶尔村里的房屋会有几声短促的惨叫响起,但立刻又安静下来。

等到夜色降临,一堆堆篝火围绕在村庄周围,这座富足安康的村落就变得如同人间鬼域一般令人恐惧。只有饮酒后得意的狂笑,还有时不时响起的悲鸣。

村子里最大的一间屋子中,余古赧从围着火盆的一张张脸上看过来,视线转了一圈,看到的面孔有十七个。

与余古赧同时南下的部族,总计二十八家,九千三百余骑。西阻卜诸部几乎是全数动员,没有一家不出兵。而现在围在屋中——包括余古赧在内的——就是其中十八家的族长或是领军的首领。

按道理是不该这样的,事前的约定也并没有让各家部族挤在一起。分散开来进行劫掠,从宋人的河东路边境,到曾经属于党项的银夏之地,全都是阻卜人的猎场。而葭芦川,应该是属于余古赧的拔思母部和三个附庸部族的。

“怎么都往这里来了?!”一直跟着余古赧的一个小部族的族长首先发难,“不是说好每家各占一片吗?”

但另外一个后到的族长拿着刀,用刀尖拨了一下火盆中的木炭:“屯秃古斯,你跟着余古赧都吃了肉了,好歹也分我们一口汤才是。现在就剩这边还没下过手,不往这边来,还往哪里去?!”

余古赧狠狠的拧着眉。

就是草原上的狼群,惯常见的都是十几匹、几十匹一群,没听说上百匹、上千匹的。当真有上千匹狼聚在一处,没几天就能饿死一大半——百十里方圆的一片草原,最多也只能养活两三百匹狼而已。

南下的部族大半聚集在葭芦川边不过几十里方圆的一片山地,一路上都没有受到像样的阻截,说是宋人胆怯,也要余古赧敢信!

屯秃古斯看了余古赧一眼,转头又道:“挤在这里的人太多了,宋人杀过来,怎么跑得掉?”

那名族长闻言立刻大笑起来:“屯秃古斯,怎么胆子变得跟鹧鸪一样小。怕什么,我们这里可是有五千兵马!”

“乌八,葭芦川这里的村子,可够五千兵马分的?!”屯秃古斯怒问道。

“谁说要打村子的,有点志气啊。现在我们已经合兵一处,目标就该放高一点!”乌八用刀鞘重重的一敲地板:“再向东去不远,便是宋国的一座大城,听说叫做晋宁军。里面几千户人家,金银绸缎美人数也数不清。”

就算打下几十座村子,其收获也肯定比不上攻下一座城池。宋人城市的富庶,早就在阻卜人的口耳相传中,传到了每一个的心里。若是能攻下一座城池,城中的财货兵器,可谓是应有尽有,要什么没有。带回去立刻就能招募草原上小部落,将手下的部众扩大上数倍。

“干了!”

“肯定不能放过。”

“要尽量快一点。等宋人反应过来可就来不及了。”

“家里面还缺两个工匠。要在晋宁城里面找全了。”

“两个工匠算什么?想要的话,分你二十个!”

余古赧看着一群开始兴奋起来的族长和头领们,头就疼起来了。

他虽然是率领西阻卜各部南下的部族长,可就跟头狼一样,若是不能给下面带了足够的好处。掉过脸来,就能给下面的各部族分食掉。

之前与宋军交锋过几次,虽然他们身上的装备让人眼馋,可实在是太硬了。如果给宋军时间布开阵势,就算想攻破他们的阵势,也得费上许多精神,甚至有可能得不偿失,或是丢盔弃甲的风险。

幸好他们的骑兵不行,战局不利,转身就能跑,两条腿的总是追不过四条腿。可一旦攻城的话,那可就两说了。城里面的守军会攻出来,或是背后杀出一群伏兵援兵来。

余古赧拿着腰刀敲了敲地板,打断了热火朝天的讨论:“乌八,你可知道宋人步兵的厉害?城池攻不下来怎么办?党项人为了攻下盐州城,摆出了十几万人马,我们这里可就五千。胡斯里和厄不吕他们还都在西面几百里外。我们这点人别说攻下城池,要是宋人在我们准备攻城的时候围过来怎么办?”

乌八跟余古赧针锋相对:“这里全是山,这么多山沟宋人怎么堵得过来?若是宋人的城池不好打,想逃也不是难事。”

“是啊,何必要硬拼?”乌八家的跟班也在配合着说话,“宋人的城池就在眼前,总要试一试再说。万一打下来就是净赚,若是看着难打,走就是了。”

谁也不会与宋人硬拼。

之前跟宋军的一支骑兵硬拼过一次,仅次于余古赧和乌八两家的扎剌部,立刻变成了垫底的一支。前两天刚刚在探路的时候,撞上另一支宋军骑兵,全军覆没。

有扎剌部的前车为鉴,已经没有人会糊涂到拿自己部众儿郎与宋人硬抗到底。而且现在哪一人的毡袋中不是揣满了金银铜器、绸缎布匹?回去后立刻就能给妻儿绫罗绸缎的穿戴起来。谁还愿意拼命?

可人人都是这个心思,还打得下前面的晋宁城吗?还有什么必要试探的?!

报警的号角声一下打破了屋中热火朝天的讨论。凄厉的号角声一声声回荡在村庄两侧的山间,拥挤在村内村外的阻卜骑兵,如同被惊起的麻雀,哄哄的一团乱。

一名斥候已经到了余古赧等首领们的面前,“族长,东南的山谷里发现宋军,正向我们这里杀过来!”

“宋军有多少人马?”余古赧紧张的问道。

“两千多人,差不多都是步兵,只有很少的骑兵。”

“距离呢?”

“就在五里外。”

乌八立刻叫了起来:“整整两千人啊,都到了五里外才发现。你们的眼睛是瞎了?!”

余古赧的脸色更加阴沉。斥候咕哝着,为自己辩解:“这里的山谷太多了……”

“怎么办?”一名小组长开口问道。

“当然是打。不过才两千步兵!”乌八很是不屑的又叫道:“我们这里可是有五千兵马!”

余古赧甚有决断:“乌八你在村子里守一阵,我领兵绕道宋军的后面,到时候你我前后夹击。”

乌八的眼中疑云浮现:“为什么不是余古赧你在村中守着,我绕去宋军的后方?”

余古赧这下当真是怒火上涌,。握紧了手中钢刀,与乌八怒目而视。两人之前的气氛一触即发,似乎只要再有一点火星,他们就要火并起来。

“还在这里吵什么?!”一个胡子全白的老头儿这时候用力跺了跺脚,“有这个时间早就杀过去了。在山谷里宋人又排不开阵势,怕他们作甚?!”

白胡子老头显然有几分人望,立刻就有人上来将余古赧和乌八分了开来。

大敌当前,余古赧和乌八两人也没有再争吵的心情。各自就坡下驴,互相瞪了一眼之后,就扭头分开。

各部的族长和首领立刻冲出房屋,各自上马赶去村外他们部众休息的地方。叫起麾下的人马,更没有什么计划,直接向宋军出现的方向冲了过去。

……………………

数以千计的骑兵在山谷中飞驰,骇人心魄的重音早就传到了宋军将士们的耳中。

派出去探路的宋军斥候,也带着敌情回到了主将的身边。

“阻卜人是疯了吧?”领军的李瑛惊讶莫名,“骑兵竟然在山谷中往我军阵里冲?当他们是伏兵吗?”

被派来押阵做监军的折可适,同样很是不可思议的摊开手,“或许真的疯了。”

大地震颤仿佛底下当真有地龙在翻滚,这是千军万马的奔驰。

山坡上裸露在外的土石也在扑簌簌的向下落着,不是骑兵奔驰的震动,而是宋军步卒在登上两侧的山坡。

“斩马刀!”李瑛一声号令,前排的步卒全都将斩马刀持在手中。

“神臂弓!”李瑛又是一挥旗,山谷中立刻传出一片上弦的声音。

“狠狠的打,一个都不要放过!”

