\section{第九章 拄剑握槊意未销(14)}

【真是不好意思,前天在兄弟婚礼上帮忙,昨天又陪从外地赶来出席婚宴的几个好朋友一天。欠下的章节尽量补上。这是今天第一更】

今天的崇政殿议事,并没有做出什么决议。在辽人彻底撕破脸之前,暂时只有以不变应万变的想法。

王珪领着众臣向天子行过礼,当其他人开始退出崇政殿的时候,他却是站着没有动。

一直以来,在廷议结束后,赵顼时常单独留下王珪说上两句。

对于如何治国,赵顼有许多想法,不过这些想法许多时候很难在廷议上通过,或者要大费一番口舌。但如果有宰相的相助,根本不需要与群臣辩论,只要让宰相去传达事实就足够了。

以三旨相公为名,王珪将任务完成的很好,是个合格的传话人。

但今天的情况与往常不同,赵顼叫了另一人:“吕卿,你且留一下。”

吕惠卿的脚步顿住了,低头躬身领命,藏起了脸上的表情。

王珪也同样适时的低下头去,让每一道试探的目光都撞到了他的长脚幞头上。

等到他们两人重新抬起头来,已完全看不出脸上有一丝异样。

殿上的每一个人都想到会有这一刻,但没人料到会这么快。就在今天的廷对之后,被留下来问对的便已不再是宰相王珪,而且还是参知政事吕惠卿。不过这件事,虽在意料之外,却在情理之中,毕竟有着明确的态度,表示支持继续战争的宰执,除了王珪以外,就数吕惠卿了。

在内外稳定的情况下,以君命为依归的王珪,理所当然的受到天子的看重。但灵州之败,显示出王珪并不足以平复危局,他所受到的圣眷因而明显减弱。而性格坚定,如今依然选择支持战争,同时还坚持着手实法、能为国库继续增加收入的吕惠卿,自然而然的成了赵顼倚重的对象。

至于韩冈,在这个节骨眼上被留下独对,就是升任执政的先兆——最近由于韩冈都与宰辅们一起入崇政殿议事,他即将晋身两府的传言越来越多,只为平息谣言考虑,赵顼就不会这么做。至少在现下,还看不出天子有这个打算。

但吕惠卿留对的政治意义同样深重。

也许王珪独相的现状保持不了多久了,同样的想法出现在每一位步出崇政殿的重臣心中。

久违的独对,吕惠卿知道自己的机会终于来了。他强忍着兴奋,在天子面前阐述着自己的建议:“时局变易,并州之守,孙永已难符其任,陛下宜速选调贤能,镇守河东,以待辽人。”

“孙永……”赵顼微皱眉,认真考虑着吕惠卿的建议。

……………………

自出崇政殿,王珪的步速就较往常略快,吕公著依然是沉稳如一,宰相和枢密使一前一后的走着。元绛和韩冈则落在后面。

元绛只比韩冈略前半步,边走边侧首:“今日殿上议事,多亏了玉昆你的谏言,否则光是进入大同府的两万辽人,就能让京城内外人心惶惶。”

“仅是泛泛之谈的附和而已,远比不上吕吉甫识见深刻。”

韩冈想看一看元绛的反应,但浸淫官场日久的元绛,他的表情和话语,完全没有透露出任何对韩冈有价值的信息。

他平和淡定的走在回廊上,向韩冈诉说着自己的观点:“河东乃北方攻守之枢,孙曼才却当不起勾连东西,通南阻北的重任。河东路的守臣还是得早日决定下来。”

“此事非韩冈所能置喙。”韩冈不想在朝廷人事上与这位政事堂中的老狐狸交流,这不是他该说的,元绛看似交浅言深,但他表现出来的态度却依然模糊不清,“边路帅臣之任,当是大参与相公议定,报与天子处断。以天子之英睿,大参和其余诸公的见识,想必能有让人信服的决定。”

韩冈拒人千里——尽管他也认为孙永早就该滚蛋了。

从耶律乙辛帅二十万辽师抵达鸳鸯泺时开始,替换并州太原府的守臣一事,就已经摆上了台面。至今没有一个定论,只是因为时任知府的孙永是天子的潜邸旧臣,在赵顼仍是颍王的时候,孙永便是其椽属。

也因如此,尽管孙永一直都是反对开疆辟土的一派,王韶旧年上平戎策,时为秦州知州的孙永大加反对,但他一直都能坐在重要的岗位上——秦州、谏院、军器监,全都是能立功受赏的位置。纵使一时因罪失意,也很快能被天子特恩起用。

但在辽人摆出举兵南向的姿态,开始调遣精锐南下大同的危急时刻,孙永的才具和政见,放在太原知府、河东路经略安抚使、河东路兵马都总管这三个位置上,便如同猴子拉大车,完全匹配不上。

元绛并不介意韩冈的冷淡——至少表面上完全看不出来:“若河东能如河北一般,有贤臣名将坐镇,京中当可高枕无忧。”他侧脸瞥了韩冈一眼,“……想必吕吉甫也是这般想法。”

这不是废话吗?!

元绛都能想到的事,走在前面的两位会想不到?还是说他韩冈会想不到?

京城中的两府宰执,眼下只有两位旗帜鲜明的要继续将战争进行到底。

其中王珪因为兵败灵州,需要他韩冈的支持。但吕惠卿却没有灵州之败的拖累,反而就不需要了——崇政殿中,不需要有两个在军事方面有裁断权的臣子。

吕惠卿趁此良机,设法让自己出外也是必然。

尤其是王韶的病情已经在京城中传扬开,吕惠卿只会忌讳身体太过康健的韩玉昆,而不会太在意据说已经病倒不能动的王子纯。

“听说王子纯的病势不轻?”元绛向韩冈刺探着王韶的病情。

“何处有此传言?”韩冈装糊涂,要是自己点头确认,王韶的病却好了,那就是耽搁了他的上进,“王资政文武兼备,习武养气从不偏废,就是抱恙,也不过伤风感冒而已。”

“那就好。”元绛捋着长须,微笑点头,一副仁人长者的态度:“有王子纯在,他不论是坐镇晋地,还是留镇大梁,都能让人高枕无忧。”

“大参所言正是。”韩冈略嫌冷淡的回了一句,终于让元绛选择了沉默。

只要王中正和种谔都能将麾下大军顺利回撤,这一战的主动权将重新掌握在大宋的手中——韩冈对此深信不疑。

就如出拳攻人,都要先将拳头收回来蓄力。之前无论是高苗二人灵州兵败,还是种谔、李宪顿足于瀚海之滨,都是力道使尽的缘故,后勤补给线已经拉到了极限,军心士气也给消耗一空。

如果将攻出去的兵力收回来,占据几个战略要地,以河西、银夏两地的归属为诱饵,强逼西夏过来争夺。以逸待劳的结果,绝对会让铁鹞子讨不了好去。

从宫中出来,韩冈就想着,自己现在的位置和参与的事务隔得有些远。在军事问题上的权威所支撑起来的发言权,对自己的好处并不大。

不在其位不谋其政,这句话说的其实很有道理。

侵夺他官事权,自然会惹来仇怨。就如韩冈本人,也是难以容忍有人侵占自己的职权。

而韩冈现在可是将手伸进了宰辅们的自留地,尽管他始终自觉的约束自己,尽量就事论事,不掺和其他领域的议论,但想要宰执们对自己有多少善意,那也是绝不可能。除非他能真正的进入两府之中,否则他在崇政殿中的存在,便如白羊群中的黑羊一样刺眼。

韩冈从来没有想过要做斗犬,跟谁都要斗一斗。他在廷议已经尽量低调,但天子的征询顺序,总是将自己放在最后,弄得好像他韩冈才是拍板定案的人一样。

元绛为什么能隐隐指出吕惠卿会设法将韩冈支去河东。还不是因为元绛本人深有感触,不是他体会到吕惠卿的心思,而是借着吕惠卿为幌子,说他自己的心里话。

韩冈同样也是早就对宰辅们有着极高的警惕之心,才能立刻反应过来。

所以韩冈之前跟几位宰执都有着或大或小的言语交锋——反正讨不了好,还不如在天子面前做个孤臣——即便一时顶撞了天子,但等赵顼冷静下来,至少不会留下多少坏印象。

但事情做得太过火也不好。暂时韩冈不想再跟宰执们有什么冲突,尤其是从今天开始,吕惠卿和王珪之间很快就会有一场风暴即将爆发,站在他们中间,极有可能会被牵累到。

韩冈这一次设法挤进京城,本意是想继承张载传下来的衣钵,在京中宣讲气学,不意却被西北的战事给耽搁了。事前谁能想到耶律乙辛下手如此干脆,惹得天下局势大变?

如果不能宣讲气学,在内在外,韩冈都不在乎。在外还好一些,尚能借助军功,多提拔几位本门弟子。因为种痘法的传扬天下,气学在当世,其实已经可以算得上是一门显学了,归于门墙之列的弟子,并不在少数。

眼下朝中还有太学一案,不知什么时候才能审结。现如今被牵连进去的官员,基本上都是新党未来的中坚。如果从重论处,就是当年苏舜钦一案的翻版,新学大挫可以预期。等到自己回来,留下的真空,正好能让气学一脉插足进去。

不过这还是想得远了,吕惠卿到底能不能让天子点头同意让自己去河东?这还是一个问题。

