\section{第28章 官近青云与天通(22)}

王中正对游师雄很了解,之前的伐夏之役,在秦凤路转运司的游师雄与他接触很多。 

“横渠门下一向文武双全。当年范文正守陕西,横渠先生便上书要取河湟为助力,可谓是远见卓识。后来兴学授徒,也多谈兵事。韩学士算是其中最拔尖的一个,游师雄却也是一流的人才。去岁伐夏,游师雄与王襄敏的次子王厚同为随军转运,多有功勋。微臣的那点功劳,也多亏了游师雄和王厚在后襄助之力。” 

向皇后对王中正的回答很满意。知己知彼,百战不殆。她至少还知道这两句。王中正虽然推重游师雄,但从方才的几段话中,也能看得出王中正对陕西的文官武将们的了解。 

她暗暗点头。难怪能领军南征北讨,号为禁中第一名将,秦翰也要瞠乎其后,这不是没有来由的。 

提起朱笔,在身边那面空白的屏风上写下了游师雄的名字,向皇后回过来又问王中正:“游师雄现在何处任官?” 

王中正发了一下怔,一般来说天子若是这么问,就肯定是想要提拔这个人了。只是游师雄现在可都是重臣一级了。 

“现下游师雄身在甘凉路。以右司谏、直宝文阁权发遣凉州,并领甘凉经略使兼兵马都总管二职。”王中正低头回道,他怕向皇后脸上挂不住,“甘凉乃是新复之地,自吐蕃大兴后,三百年不受中国管辖,至归义军兴起亦只能羁縻而已。必得能臣守之。游师雄在关西夙有威望,又有能力,功绩即显,故而破夏之后半年,官家便不问资序,将之破格提拔。” 

王中正的话有点啰嗦,向皇后听着感觉挺怪的。偏头想了想,觉得自己是明白了。这是王中正规劝自己不要立刻提拔游师雄,以免甘凉路不稳。 

她有着些许遗憾,感慨着,“想不到都是一路帅臣了。” 

垂帘以来,心思全都放在了朝堂上,连一路帅臣的姓名都没时间去了解,向皇后想想,觉得自己实在是浪费了太多时间在司马光这等人身上。 

再仔细想想,帅司、漕司、仓司、判司,天下各路四大监司的使臣,更是有大半不知道到底是谁。就算听说过姓名,也不知道他们过去有何功劳和过失,更不清楚他们的能力如何。而在各路监司之下,还有四百军州,两千多县,镇子更是无数。 

治国之难,她现在算是领会到了。 

“这也是官家的提拔。”王中正微微松了一口气,道:“官家乃是明主,故而用人都能各尽其分,用其所长。” 

‘所以要镇之以静?’向皇后狐疑的看看王中正,不知道是不是他是不是又在劝谏。 

心中怨怼之意油然而生。虽说这也是自家丈夫的意见,连王珪都用‘使功不如使过’的理由放过了。可司马光之辈,却是想趁着自己还没有熟悉国事,直接欺上头来了。 

王中正没感觉向皇后心思的变化,接着道:“广锐军被困咸阳,犹自作困兽之斗。幸而韩学士孤身入城,说降叛军。罪魁吴逵自焚,只是尸骸难以辨认,所以并没有报功。剩下的叛军活下来的近三千人,连同全家老小,全都被发配去了熙河路。这也是受了韩学士之请,说是杀降有伤陛下盛德。” 

“韩学士仁心。”向皇后由衷的说道。 

“的确如此。贝州和保州都有降军事后被刑,只有广锐军这边被保下来了。” 

石得一在背后抬眼看房梁。皇后和王中正倒是忘了韩冈在河东,将南归的黑山党项杀得只剩数千人,拿了两万三万的斩首,交趾人更是只有八只脚趾。 

王中正却说得正是兴致高昂的时候,“在河湟开边时,因广锐军乃是西军中数一数二的精锐,也被派上了战场。立了不少功勋,赎了过往之罪,但官家也只赐了金银田土,并没有给其官职。而且在河湟之役中,广锐军领头的将校死得七七八八,不致为患了。” 

“难怪韩学士能未及而立,便已近宰执。”向皇后深有感触,“十年前才做官就立了这么大功劳,怎么也当得起了。” 

王中正更正道:“圣人误会了,韩学士在横山和招降两事上,并没有受功赏,全都辞了。” 

“这话怎么说?” 

“因为在被韩大观征辟的时候,韩学士明着对王相公说罗兀难守、横山必败,若是一定要他去,有功劳也别算他一份!” 

向皇后惊诧莫名:“韩学士竟然这么说!” 

“可不就是这么说的?”王中正摇摇头,“但王相公也厉害,却硬是将韩学士派去了韩大观的帐下。说不要功劳那是你的事,朝廷要你做的事,照样还是要去做!” 

向皇后听了更是觉得匪夷所思,竟然还有这样逼人上路的做法,在朝堂的人事安排上,若是被任命官员不愿去做,怎么都不会强迫的。王安石就不怕韩冈怠工? 

王中正叹着气,“所以说拗相公当真是名副其实,就是韩学士撞上了,也是一点办法也没有。” 

王中正稍稍开了一个小玩笑,见皇后抿了抿嘴,像是想笑又不方便笑的模样,心中便更是轻松了几分。 

“不过韩学士难得的地方就在这里。他去了延州,直接就往最危险的罗兀城去了,一点也没有拖延。到了罗兀城,对伤病们尽心尽力,面对西贼,也是用心辅佐张、高两位主帅。要知道一旦赢了,韩学士可就是最吃亏的一人,没功劳还要受人笑,但韩学士完全没有计较。要不是广锐军叛乱,罗兀一役当真就给官军赢下来了。” 

向皇后前面已经知道韩冈在横山的赫赫功绩,却想象不到韩冈是不顾受人嗤笑的结果上为国事尽心尽力。 

王中正轻声喟叹:“微臣当年曾听官家说过,‘言罗兀难守,事前不止一人。但仍尽心尽力,惟韩冈一人而已。’备称韩学士为人甚正。” 

行事如此光明磊落,再想起冬至之夜,韩冈面对太后的义正辞严,向皇后却一点都不惊讶了。 

…………………… 

午后崇政殿再坐,韩冈和除王珪外的众宰执前后脚都到了。 

殿脚有一个判起居注,殿中站着一个御史中丞李定,加知制诰的翰林学士蒲宗孟则也在一旁候着,准备书诏。 

而司马光,则不见踪影。前面韩冈经过殿外东阁时,也没看到司马光在里面等候。他的结果自是不言而喻。 

对于今天朝会上发生的一切,宰辅们很快就给出了一个处罚决定。 

蔡确、章敦全都支持速办严办,韩缜、薛向表示谨慎的支持,吕公著继续保持沉默,其他人包括韩冈都没资格说话,更不会跳出来表示反对,就这么顺顺当当的敲定了下来。 

司马光入觐,照规矩赐物赐药。不论他受与不受,朝廷还是给他一个体面。但之后,就让他回洛阳,绝不留他。韩冈本来还想让他去殷墟,但现在已经是不现实了。 

至于御史台对王珪的弹章,则全都驳回。之前在殿上附和司马光的御史,一体下诏叱责,并解职外放。 

御史中丞李定没有为他的手下辩解,应声接了下来。出了这么大的事,他台长之位,已经是做到了头,今天回去,就该上表自请出外了。 

被蔡京出来弹劾的有失朝仪的吕公著、蔡确、章敦和韩冈等人,向皇后打算不问。但韩冈、章敦和蔡确自请罪,逼得吕公著一同低头,便一体罚俸半月。也没什么争执的,谁还会为这点小事废口舌? 

至于剩下的亟待处理的政事,则一切按照流程走。奏章上有不明白的地方,几名宰辅按照各自的主管范围向皇后详加解释。作为天子私人的两位翰林学士,韩冈和蒲宗孟,也一并受到咨询。 

由于向皇后对政务的生疏,处理起来比赵顼在时要慢得多,但也没有拖到第二天去,快黄昏的时候,总算是结束了。 

从政务中歇了下来,殿上重臣们各自喝着皇后赐下的茶汤。章敦向韩冈使了个眼色,韩冈会意,微微颔首。眼神一转,看了看吕公著。 

王珪避位待罪,今天接下来自是照旧由吕公著领头入福宁殿探视天子。 

天子病重卧床,宰辅们除了轮值宿卫以外,还要入福宁殿问疾,探视天子病情,以防有人隔绝中外——这是当年富弼和文彦博在仁宗发病时挣来的权力,一直延续了下来。韩冈则是身份不同,则是一日一入宫,与宰辅们同行。 

因为司马光之事,皇后现在应该不会受吕公著蛊惑,说什么都没用。如果吕公著真有什么想法,入殿问疾是他必须要把握的机会。 

到了福宁殿寝殿中,赵顼已经被唤醒了。睁着眼睛,等着宰辅们来此。 

吕公著当头,依照几天来的惯例向赵顼问安,拿着韵书确认了神智,安慰了几句,便领着同僚向天子告退。他们不耽搁,赵顼也没留客。 

众人再拜起身,一个个倒退两步,就要转身出寝殿。但应该和其他宰辅一并退出寝殿的吕公著却没动身,他向着赵顼行了一礼:“陛下,臣有言欲奏禀,乞留对。” 

果然如此! 

韩冈算是松了一口气,吕公著的回击总算是来了,比起他一直隐而不发要好不少。 

但吕公著到底想说什么,却是让人要多想一想,一时捉摸不透。 

