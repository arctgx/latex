\section{第30章 随阳雁飞各西东(20)}

大公鼎赶到中军大帐的时候,西平六州都管耶律余里正在愣愣的望着西北方,旁边的左详稳奚乌也也在陪着他一起发愣。

被护卫在营寨最中心的中军大帐本是营中最忙碌的一个区域,但现在大帐附近的百多人却仿佛都被冻结住了,僵硬矗立在夜幕下。

“可是耀德城出事了?!”大公鼎连喘气都顾不上,跳下马就直扑耶律余里的身边。

“不知道!”耶律余里没说话,奚乌也在旁边摇头,“连个报信的人都没来!刚刚才派了拦子马去打探了。”

大公鼎脸色更形难看:“那都管找我来又是为了什么?!”

“是种谔!”耶律余里转过头来,一直洋溢在脸上的自信不见了,双唇抖着,“方才斥候回报,种谔已经领军从盐州出来了。”

大公鼎仿佛被劈面打了一拳,双脚猛的一软,幸好有儿子左右扶住,才没有一下摔倒。

以种谔出兵的消息佐证,耀德城绝不可能有任何侥幸了!

奚乌也一脸的茫然无措。

围城的这么多天,他们一直都在盼望银夏军能快一点出兵。眼下种谔当真出兵了,却没有人还会想着再去跟他打上一仗了。

大公鼎挣扎起来,抓住耶律余里的手,“种谔现在到了哪里?”

“已经走了一半的路了。”

大公鼎的手无力的落了下来,最后一点希望也破灭了。

是了,为了能引银夏军,以免他们见势不妙就逃回盐州城去,这边派出去的斥候与宋军的游骑虚晃两招后,就退了回来,让出了一百多里地。

大辽的远探拦子马可是当时闻名的精锐,要不是想一口吞掉宋人援军,如何会输给飞驰时连缰绳都不敢松开的宋国骑兵?!

大辽国的骑兵绝不会畏惧与宋人野战。

阵列不战,这是大辽对阵宋军时的铁律,但不能列阵的宋军则就是大辽铁骑屠戮的对象,而作为援兵的宋军偏偏不可能随时列阵。一百里以上的路程,来去如风的骑兵足以将必须不断前进的宋军给拖垮——即便领军的是宋人之中最为骁勇的名将,也是一样。

若是一天前听到这个消息,估计有一多半将领能大笑起来,但现在没人能笑得出来了。

来自盐州的宋军距离溥乐城只有一百里多一点,即是以步兵的速度,也只要两天。而银夏那边,应该是绝不缺乏骑兵。

一名名驻扎在其他营地中的将领们都赶来了。

人人脸上都写满了惶惶不安。

本是他们打算要在从盐州到溥乐城的两百里瀚海路上,给宋人一个血淋淋的教训,谁能想得到宋人不去援救溥乐城,而直接烧了耀德城?

他们竟然敢攻打大辽的城池!?

“尚父不会饶过那群南蛮子的!!”一人大叫道。

“先想想怎么退吧。这仗打不了了!”另一人直接一盆冷水。

怎么退?回西平六州——也就是兴灵——的道路是沿着灵州川的六百里,失去了耀德城粮秣的情况下,即便可以抛下党项人,如何能让剩下的过万骑兵安然回到家里去?

“营中还有三天的粮秣。”大公鼎说道。

众将各自面面相觑。

一天两百里吗?人吃得消,马可吃不消!

“还是先派人回耀德城救火!”大公鼎提议道,“能救下一点是一点。”

这个提议没人反对。

“那要多带点人手,以防万一。”一名契丹奚族说道。

耶律余里摇头:“多了就麻烦了!”

“党项人也没多少了,敢闹事杀了就是!”

“谁说是党项人了!”耶律余里怒吼着,右手用力捶地。

一下所有人都明白了过来。

“谁回去?”奚乌也的声音有些低。

回去救援耀德城,就有可能一头撞上宋军的伏兵。深夜之中,只要一个不小心就有全军覆没的危险。

可这件事放在下面的人眼里,却是这一部丢下其他人先退了。不论是谁领兵先走,人走得越是多,就越是让剩下的人感觉自己被丢下来殿后。

甚至现在就在帐中,都会有人肚子里转着小心思。谁敢保证赶回耀德城的那一部,吓退了宋人之后,不会拿着粮草拔腿就往北去!若是剩下的粮草只剩一点,鬼才会给其他人留上一口。而且从盐州出来的银夏军主力就在身后盯着,万一被咬住了,不死也要脱层皮。

帐中一下就没声音了,半天也不见人吭声。

大公鼎口中上火,胃突然间疼得厉害。

这一年多来,各家也没少争过草场、田地。他们的军队是头下军,是由契丹、奚族、渤海等部族私兵所组成。占优势的时候人人争先,可如今战局一变,那就是人各异心了。

只是大公鼎也不会糊涂到自己跳出来说为大军殿后,让耶律余里或是奚乌也带着主力回师耀德城。都是自家的儿郎,如何让他们为契丹、奚人去死?

“不如等过两个时辰,快天亮时再走。快到耀德城时正好天亮,也不用怕宋人的伏兵。”大公鼎想了半天,提出了一个不是办法的办法,“这样就算只有一千兵马回去,也足够了。”

“那耀德城的粮草呢?!”一名奚族的部将怒气冲冲,“就丢着不管了!?”

虽然耀德城的火势正旺,但城中的粮仓也不是挤在一起,不一定会一下全都烧光。能早一点回去,就有很大可能能多救出一份来。那些粮草可是各家这一年来辛辛苦苦积攒下来了的军粮,烧光了,明年夏收前再想出征,就要给肚子上的腰带多勒紧几分!而且是连人带马!

大公鼎阴着脸,望奚乌也,那是他手下的人。可奚乌也低着头,盯着地面。

“报!”一名亲兵冲了进来。

耶律余里很不耐烦瞪着他:“何事?!”

“溥乐城的骑兵出城了!”

奚乌也终于不再沉默,他惶然叫道:“种朴这是要拖住我们!”

“多少骑?!”

“看不清,应该不到一千。”

已经足够了。

溥乐城中的骑兵数量其实都有数,五六百基本上都是全部了,现在应该是倾巢而出。这个数目已经足够上半日,甚至让殿后的后军被宋军咬上,吃掉。

这一回,更加没有人敢留下来为全军殿后了。

“报!!”又是一名信使冲进了大帐,歪歪倒倒的,差点将大帐给撞翻。

“怎么了?!”耶律余里怒吼声更大了一分。

“党项人攻下顺州了!”

耶律余里顿时僵住了,奚乌也却跳了起来。

“西平六州的党项丁口不全在这里!?那群老弱怎么可能能占了顺州?!”奚乌也劈手揪起那名信使,牛一般瞪起的双眼血红一片,顺州可是他的头下军州!

“来攻的是青铜峡的党项人!”信使几乎要哭出来吗,“城内的党项人开了城,是里应外合啊!”

帐中陡然间没了声音,所有人的视线都投向了耶律余里。

唆使青铜峡的党项人攻打鸣沙城是耶律余里的主意,领军攻打溥乐城也是耶律余里的主意。更确切一点,是耶律余里身后的耶律乙辛的主意。西平六州中,耶律乙辛派来镇住这一飞地的亲信,正是耶律余里!

宋人已经跟过去不一样了。

每一个人都震惊于宋人的行事作风已经完全看不到过去的影子了。

烧了耀德城的粮草,甚至意图尽灭南下的大军,背后又唆使被挑动的党项人反攻入兴灵,这一整套伎俩,很明显的的是要吞下兴灵。

这是过去的宋人绝不敢做的。

高粱河之败的百余年来,就只有一个曾经反攻入大辽境内的杨延昭。澶渊之盟后,更是一个都没有。

但现在宋人敢了。

静默中,耶律余里站了起来。

“什么时候,宋人和党项人都敢欺负到我大辽的头上了?!”怒火烧红了耶律余里的双眼,冲着每一个人大声吼:“究竟是什么时候啊!??”

“这一仗打得太大意了,该顾着身后的。”有人小声的说着。

“是吗?”耶律余里剔起眼,一对环眼圆瞪,充满压迫力的视线从众将脸上扫过。每个人都低下了头,但没人否认。最后他点头,看着众将,一下一下的点头,“说得没错,是我不对!我认罚!”

这里谁敢罚你?

肚子里面的话没人敢说出来。但接下来耶律余里做的事却吓住了每一个人。

契丹人藏在骨子里的那一股凶戾之气爆发了出来,耶律余里将左手小指放进嘴里,瞪起眼,在众目睽睽之下,牙关猛然一合,两股血水从嘴角飚出,竟是硬生生的将手指给咬断了!

大公鼎等人看得寒毛直竖,被耶律余里眼中的凶戾给慑住了,不敢言,不敢动。就看着耶律余里扬起脖子,将嘴里的血肉给硬吞了下去。

张开满口鲜红的一张嘴,耶律余里的话中犹如阴风袭来,“这一战是我的错,就拿这根手指认罚了!有没有人觉得不够?!”

没人敢搭腔。

重重的冷哼一声,耶律余里举着少了根手指的左手,龇起血淋淋的两排牙齿,“就以此指为誓,我要把那群党项贼都吊在西平府的城头上!”

抽出刀,将帐帘一刀劈开,跨出大帐,耶律余里举着刀回头怒吼:

“还坐着干什么?!都随我杀回去!!西狗想找死,回去杀了!宋狗敢过来,回头杀了!谁敢挡在前面,就杀了谁!!!直娘贼的,全都给我起来!!!”

