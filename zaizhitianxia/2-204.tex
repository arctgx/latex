\section{第32章 吴钩终用笑冯唐(11)}

随着环绕着咸阳城的围墙和壕沟大体建成,围城的官军在城外终于有了动作。

吴逵对此早有准备,听到城外传来的鼓声,也只是下令一队骑兵做好出城的准备,然后默然的提起铁枪,走上城头。

但出乎于吴逵的意料,官军并不是来全力攻城,仅仅是在东门和南门外排下军阵。而在城池的东南角,离城墙不过五十步的地方,八具行砲车一字排开。

很明显,堵东门和南门外的官军,是为了防止叛军骑兵出城摧毁这八具行砲车,才列阵以待。

砲车的威力,吴逵曾经亲眼见识过。当几十斤中的石弹、泥弹从天而降,就没没有命中,其呼啸而来的声势都能把敌军给吓跨。如果有几十架砲车同时集中于城墙一点,很容易就能在城头上清理出一片空地来

可是,排在他眼前的砲车的数量,未免太少了一点。

吴逵看得出来,官军摆出的架势并不是要攻城,但排出这几具砲车又要做些什么?

围着八具砲车忙碌的士卒,总计才百多人的样子,平均到一架砲车上,不过聊聊十几人。

而据吴逵所知,就算是小型的三稍砲,也要二十多人来拉索,而如城外这八具砲车的大小,定然是七稍砲无疑。没有三五十人一齐用力,砲弹怎么抛出去?

而且行砲车最大的问题是准头不行。几十人拉纤一般的扯着稍杆,前一次的出力和后一次的出力,几乎没有保持原样的情况。上一次命中目标,但下一次就能偏到三五十步外去。同时为了使砲手拉索时的行动如一,还要对他们加以训练,耗费大量的时间。所以行砲车在战场的使用上,完全比不上以八牛弩为首的床弩普及。

只是吴逵总觉得有哪里不对劲,正要下令这一段城墙上的守兵立刻瞄准城下射击,就见着官军的投石车已经有了反应。

完全没有任何人拉扯,被压下去的稍杆却猛然扬起。如同抡圆了手臂,八个小小的黑点从城外的阵地上飞了起来,划过几道完美的抛物线,越过了五十步的距离,轰然数声巨响,猛然砸到了城墙上。

直到在震颤的城头站稳脚跟,吴逵仍旧难以置信的望着城外的八具砲车。几条细小却深长的裂缝,就在他的脚下延伸出去。就在身边,十数名叛军士卒,被溅起的碎石砸得头破血流,而其中一名运气最差的,头颅处已经成了一团血泥。

没等吴逵回过神来,就看到稍杆再一次扬起,石弹从稍杆的尖端飞速而出,依然如前次一般,准确的命中了咸阳城的城墙。

吴逵扶着雉堞,茫然自语:“怎么可能这么准?!”

………………

“五轮四十发二十五中……”韩冈听着砲车命中率的即时回报,当即责问起来,“怎么准头这么低?”

“不低了。”游师雄收回了眺望城头的视线,“都超过六成了!”

“区区五十步的距离,才六成的命中率,放在哪里的都说不过去。不论是神臂弓还是八牛弩,都比这要强得多!”

游师雄愣了一下,“……玉昆,你应该没看过早前的行砲车投石吧?”

“几次上阵,都没有轮到行砲车出场的机会。”

只是在韩冈想来,砲车的射程已经事先在工匠营里计算和试验过了,配重也已经确定。不过是换了个发射场地而已,在五十步的距离上,不求百发百中,百分之八十的命中率应当有!

游师雄摇了摇头,“玉昆你莫要求全责备。这新型砲车,无论从威力、准头还是速度上,都比过去强了十倍不止。说实话,本来以为十发之中,能有四发命中城墙,就已是喜出望外了。”

“是这样吗……”韩冈仍是难以释然,他现在再一次确认,还是火炮更好一些。

就在韩冈和游师雄说话的时候,砲车仍在一刻不停的投射着,向着城墙把一枚枚重逾二十斤的石弹抛向城头。由于发射速度快得惊人,事先准备的四百砲弹,没用一个时辰,便已经全部投射了出去。而在耗尽所有的石弹之前,一刻不停的被轰击着的咸阳城东南角的城墙,则终于垮了半幅下来。

在城下官军的欢呼声中,尘埃落定。原本宽阔得可容四马并行的城墙,现在大约有十余丈的墙体,其外侧已然崩塌了下去,只剩下大约一丈宽的单薄残垣,阻断城内城外。

如果能继续攻击下去,这一段城墙被摧毁也是转眼间事。但砲弹告罄,且一个时辰不停的发射,八具投石车也坏了一半。

“已经很好了。”何忠对韩冈和游师雄说着,“几十人同时拉索,力道、方向都不稳,许多砲车投个七八次便散了架。哪像这几具砲车,一连投了四五十次,才坏了一半。而且今天夜里修一下,明天还能上阵。”

“这么快?!”游师雄惊讶的问着。

“容易坏的中轴、稍杆,都另外做了预备,换上去就行了。今天坏的四具,除了一具是支架断了,不便修理。其他都是稍杆和中轴坏了,修起来很方便。”

游师雄对何忠的话赞赏不已,不愧是在工匠营中的老人,做事果然妥当得很。

何忠带着八具砲车退了下去整修。游师雄对韩冈笑道:“如过明天再来一次,咸阳城怕是转眼就能破了。”

“但我看贼军的损伤并不大……”

“嗯。”游师雄点点头,“是不大。但今天的成果已经足够吓坏他们……现在当是派人入城说降的大好时机。”

……………………

“都虞,官军那边派人来了。”

“官军……”

听到亲兵的通禀,吴逵叹了口气。曾几何时,他也是官军中的一员,他麾下的三千人也同样是官军。但眼下,他们身上却脱不了一个贼名了。

而官军的行动,也不出他之所料。早间的砲击显然是震慑,所以并没有趁着城墙坏损而展开攻城。只是拥有如此威力的武器,而不用以配合攻城,看起来韩相公并不想有太大的伤亡——这一点,当是可以利用一下。

被派来劝降的陆渊,是环庆路的都监,也是吴逵的同僚,两人之间有着十几年的交情。

两人相见后,唏嘘了一阵,回忆了一下旧日情谊。接着,显得有些急不可耐的陆渊,便开始劝说吴逵开城投降。

听到陆渊开出的条件,吴逵惊讶不已,“只是流放而已?!”

“的确只是流放。而且不是南方,还是在关西!”

“……真是多谢韩相公的仁心了。”吴逵冷笑一声,嘲讽一般的咧开嘴。周围一起旁听的叛将则都是阴沉下脸去。他们跟吴逵一样,都绝不相信韩绛会这么宽大。

韩绛是什么人,他们再清楚不过。要不是韩相公,如何会变成今天的这个局面?要是条件苛刻一点,他们反而信了,去南方的烟瘴地,他们也是有着心理准备。可陆渊开出的条件,宽大得让人难以想象,乱了关西一场,竟然还能留在关西?

真当他们好骗不成?!一众叛将顿时眼露凶光。

“这是真的!”陆渊急忙解释,“是宣抚司管勾伤病的韩玉昆提出来的。他请了韩相公的钧令,只要开城投降,不伤城中百姓,便可以全家流去河湟开边屯田。”

“韩玉昆?”听到陆渊提起韩冈,吴逵的脸色顿时变了,急问道:“是秦凤的那一位?!”

“正是前段时间,与你同行长安的韩玉昆。”

听到陆渊能知道自己与韩冈同行的事,吴逵当即便信了三分。几日的同行,加上一起对付过王文谅,他对韩冈的印象很好。而且韩冈的名声在军中也好得很。以韩玉昆救死扶伤的仁德,陆渊说是他提议饶了三千叛军的性命,这番话当不会有假。想了想,吴逵又问道:“那小弟呢?也是流放不成?”

“也是一般!”

吴逵叹了一口气,又哈哈大笑起来,“四哥,你也别诳我了,我死罪是定的。是否投降,不过是战死和凌迟的区别罢了。”

陆渊的话,让吴逵对他前面的承诺重新怀疑起来。他一抬手,阻住陆渊的辩解,继续道:“现如今王文谅也杀了,韩相公转眼就要罢官去职,我吴逵受的委屈也算是报了差不多,这条性命丢了其实也无所谓。但下面的兄弟是为了我才走上这条绝路的。他们只是被逼无奈,并非有心反叛朝廷。我吴逵虽然是个叛贼,这义气二字还是懂的。就算死,要为这些兄弟争出一条活路来。”

吴逵说得动情,边上的叛将人人感动不已,甚至有人叫起,“都虞,我们不降了……要死一起死!”

“别乱说话!”吴逵回头骂了一句,又对陆渊道:“陆四哥,不是小弟不信你,实在是不敢拿三千兄弟的性命冒风险。还请四哥回去,请韩相公派个说话能算数的过来。只要事情确凿无疑,我这一军当即便降!”

