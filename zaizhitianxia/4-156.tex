\section{第22章 汉唐旧疆终克复(中)}

四人敲定了毁弃升龙府的计划,接下来的具体的实施方案,那就是行营参军们的工作了。

燕达、李宪告辞离开,章惇站起身活动了一下肩膊,对韩冈笑道:“接下来的事,我们也就能轻松一点了。”

“子厚兄,”韩冈失声笑了起来,那是自嘲的苦笑:“要不要听现在军中到底有多少生病的?”

章惇脸上的笑容收敛了起来,天天都要报给他的数字,哪里还要韩冈来提醒,“海边的气候会好一点,要尽快将主力移驻海门。”

南方的山林河湖,多有瘴疠疾疫,不过到了海边上,就会好上许多。海门在唐代能成为交趾叛乱之后,替代交州城【升龙府】成为安南都护府的新治所,一个是因为交通,另一个当也是因为从北方调来的平叛军更能适应海边的气候。

韩冈轻叹一口气:“军中现在可是有多少人盼着直接回邕州,回桂州,甚至直接回朝,得让他们的心先定下来。”

“不能将海门修建起来,这一战就只能说是未尽全功。加上也还不到班师回朝时候,与其在邕州桂州待上半年,还不如就在海门休整。”

章惇现在是全心全意的要支持海门开港。一旦海门设立港口,展开海贸,第一个受惠的就是福建。章惇以他平灭交趾的功劳,在这其中的发言权必然最大,他的家族能得到的利益当然绝不会少。

章惇的态度,让韩冈暗喜于心。

交情只能管一时,而利益则能一直维持下去——只要共同的利益能继续存在。韩冈当年笼络王韶和高遵裕,也就是用了如此手段。而且交趾能带来的利益,绝不仅仅局限于木材和粮食。韩冈心中还有着一番另外一个方案。

有功领、有财发,只要能做到这一点,无论是谁与他韩冈共事,就算对自己的才能感到嫉妒,靠着共同的利益最后还是能同归一路。

走出临时的议事厅,李宪回头望望正在商议着什么的两位主帅,双眉在沉思中皱起,却又很快的摇摇头,像是放弃了继续想下去。

这段时间以来,他是越来越惊讶于章惇和韩冈之间配合的默契。

两个皆是性格、棱角皆不缺乏的俊才,共事在一处,就算不争功,碰撞和摩擦也必然免不了。

天子亲授密旨的时候,曾经担心。章惇与韩冈的交情甚好,这一件事,知道的人不少。但无论多好的交情——即便是几十年的老友——一旦共事,因为性格、观点和治政的手段有别,最后反目成仇的例子,那是数不胜数。王安石就是最好的例子。

但眼下两人配合的却是很好。从韩冈领军在邕州城下击败李常杰开始,便是如此。直到现在,依然还是这样。

章惇是主,韩冈是副。

韩冈的谋划没有章惇的全力支持,根本无法实现。换作一个妒贤嫉能的主帅,到最后恐怕就会变成两帅相争,最终一事无成的局面。

自信心很重要,李宪是这样想的。无论韩冈表现出多么惊人的才能、手段,章惇都对自己的才能充满信心。就是有着这样充分的自信,他才能让韩冈放手施为。

像韩冈这样才能卓异、又是年轻有为的英才,章惇能信而用之,换个说法,这就是宰相气度。

章惇决不是简单的人物。李宪很明白这一点。但他的人品不被天子甚至王安石看好,李宪也知道这一点。

在王安石的重要助手中,章惇升官的坎坷,远比吕惠卿和曾布要多。换作是曾吕二人,哪里需要像章惇这样,要去荆南冒一次风险,才能晋身两制。也不需要像章惇这样,要领军剿灭一个国家,才能有机会进入两府,他们要做的就是留在京城,辅佐王安石而已——曾布若不是犯糊涂,这时候也能进政府了。

当然,韩冈升官的难度则更高,年龄的问题让他的多少功劳最后只能换到打了折扣的封赏。也不知之后天子会怎么安排他的职位。或许会让他留任在广西也说不定。

但两人最后能达到的高度,李宪还是有数的。

历经军政二事才出头的官员,他们的根基远远要强过一直在京中任职的官员。资历、经验积累起来的权威,都是日后进出两府,或是临危受命担任要职的前提。

李宪自感来到广西、继而深入交趾之后,与他们还是亲近得少了。他虽是中官,一般来说只要服侍好天子、太后就足够了,但若是在外朝没有几个能支撑自己的盟友,日后想有所成就,那也是休想。

统一了思想,接下来当然就是展开工作。

燕达要负责督促蛮部,毁弃掉升龙府。拆毁城墙、废弃房屋,城中的交趾国人则分派给有功的部族。

毁掉了城墙,即便日后因为地理和道路上的优势,升龙府或许会再次复兴,那也不能再是一座拥有坚固城防的坚城。

而李宪则是要跟着章惇、韩冈率领主力前去海门。不过在动身之前,李宪却是来找章、韩二人,并不是有什么要紧事,而是有个提议。

听了李宪的提议,韩冈是小吃一惊。这个主意也亏李宪想得出来,明明是个阉人啊,竟然能想到阴阳调和的问题。

李宪他竟然拿着军中这些日子管不住下半身犯下的事为理由,提议从蛮部手中收回部分素质上佳的交趾女,一人分配一个,让他们带回老家去。

李宪的提议看似有些荒诞,不过这也是好事,这正证明了李宪已经放开了想象力,一切以中国的利益为依归,不再把战争单纯的当成打仗、降伏了。

对于这个提议,章惇当然不会反对。可以安抚军心的手段,从来只会嫌少,不会嫌多。

“南下以来,军中上下都是,一个个都是正当年,而且眼下又是疾病多发,若能有个提振士气的方略,用一用也是无妨。”

韩冈对杀人放火并不放在心上,但奸.淫捋掠的事,却是一直都要求军中严加管束。章惇也是如此,这可以算是儒生的洁癖。

但身在军中,到了广西之后几个月,都见不到几个女人,下面的将校士卒憋得够厉害。交趾国已经完了,让参战的将校士卒有地方放松一下,并不是坏事。

“可若是将校们以其为妻,那又该怎么办?”韩冈发问。

以官军的兵力,如果按需分配的话,至少能一下刮走上万名交趾女子。如果是明媒正娶,交趾人就都成了官军的亲家,到时候各家亲戚一攀,留在蛮部手中的交趾人也就不剩多少了。

“只能为奴为婢,不能为正妻,违者严惩之。”章惇冷笑着,“等他们回到关西旧地,隔着万里关山,也没什么关系了。”

……………………

倚兰带着自己的俩个儿子坐在一辆马车上,前后都是围满护卫,已经到了该上京的时候了。

抬眼望着远方的升龙府。数以千计的人丁,甚至一头头大象,正在城墙边的工地上,为着共同的目标而努力着。

他们竟是要拆毁升龙府的城墙!

交趾多雨,当年为了筑起这道城墙,不知费了多少气力,如今却要一举尽废。

倚兰泪水盈盈,亡国之人,已经没有愤怒的权力,除了哭泣就什么也有没了。

如果问如今,正在蛮部的皮鞭和棍棒下做着苦力的交趾国人,他们最恨的是谁?第一个就是李常杰,第二个就是倚兰和他的儿子乾德。

大宋要灭亡交趾,就绝不会允许他们成为人心所向的目标。反倒是杀生成仁的洪真太子,在人们的心目中地位就变得高大起来。

在随军北去的交趾君臣频频回头,难以割舍的望着家乡,而官军的主帅和幕僚们正在商议着该如何维系大宋对交州的控制。

除了韩冈开海的计划,还有人想到了其他各式各样的方案。

大部分被否定,但有一项是绝对没有问题的,只是韩冈不怎么喜欢。

“标铜立柱?!”韩冈嫌恶的微皱起眉头,这是钱多了烧得慌。

汉故伏波将军马援是怎么在交趾下的手,这标铜立柱的典故韩冈当然知晓,“但将几万斤的铜柱立在交州,只会遭贼惦记,如今可不是汉时,人心不古啊!”

当真不如立一座京观实在。

不过韩冈在交趾,可是不想杀生太多。只是砍脚趾,不过是施肉刑而已,面上刺字可也算是肉刑,同为肉刑,没有谁更仁慈的说法

而章惇是肯定希望标铜立柱,这是千古留名的盛德之事,他哪里可能放过这个机会,韩冈看得出着一点,也不好说自己反对,左右也不是什么大事,花钱而已,由着他们玩好了。

海门即将开始修建,而升龙府就要成为废墟,不过真正的变化还是在朝堂之上。天子、两府在得到了安南行营靠着区区五千关西援军,一举灭亡了交趾的捷报后,对于朝堂上的变局,又会起到什么样的作用?

