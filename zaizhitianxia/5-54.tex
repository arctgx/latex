\section{第九章 拄剑握槊意未销(二)}

为了能达成目的,人总是能变得明察秋毫起来,为自己找到合适的借口。不过徐禧的眼光也不差,这番话说得种谔和李宪一时间都难以反驳,甚至隐隐的都有些心动。

不过面临的种种难题,却让两人都不敢接口。秋凉时分,差不多就要到八月底九月初,十几万兵马的吃穿用度,要通过数百里危机四伏的道路来运送,试问后方还能坚持多久?征发的可都是民夫!而且大军在外时间长了,军心士气都是问题,

可当种谔和李宪看到了危机的时候,徐禧他却看到机会。

被堵在瀚海东侧的这些天,他心浮气躁的整夜整夜的睡不着觉,甚至后悔来了鄜延路这里,要是去了环庆路该有多好?

灭国之功,那是什么样的功劳?!

韩冈区区一灌园竖子,连诗词都做不得,可靠着军功随身,要不是有年纪挡着他,东西两府是摆在他面前任他挑选,但眼下也就只差最后一步了。还有王韶、章惇,两人一个接一个做了枢密副使。

眼下这一战最大的好处是完全没有文官统军,各路主帅都是武职,升官也是往三衙管军的序列去升,他这个担任体量军事的文官,与其他人没有冲突,只要能插手进去,就是仅仅推上一把,也能在通往宰执之路上,跨上一大步。

祸兮福之所倚,福兮祸之所伏。

高遵裕和苗授的失败,其实不一定全然是坏事。

李宪无奈的摇头,打仗不能光看到好的一面,兵还是同样的兵,为什么会再而衰三而竭,气势弱了,士气衰了,想赢就只能是做梦了,“耶律乙辛到了鸳鸯泺后就没动静了。现在西贼好不容易赢了,肯定会派人去联络他。”

“讹诈而已!”徐禧神采飞扬,说到文采、辩才,朝中比得上他的还真没有几个,“难道他当真敢于挥师南下?想要耶律乙辛脑袋的人,在他身后的可比他面前的要多……”他上身前倾,着重语气强调道:“而且是多得多!”

“就是他分个三五万出来,都让人吃不消。粮道太长了。”李宪愁眉苦脸,面前的这一位应该先去看看沙盘,“钱粮都难以转运,要保住无定河的粮道不是那么容易。”

“盐州南下,可以直通环庆。”徐禧立刻回道,“想想盐州是哪一路拿下来的。”

当环庆、泾原两路失败的消息传来之后,徐禧只觉得念头通达,思维也变得十分敏锐。

李宪面现难色,“要环庆路运粮给我们,这件事怎么跟高公绰说?”

“灵州一败,高遵裕还能留在鄜延路吗?何况盐州本就是环庆军收复的,还有环庆的五千兵马在,如果官军能守住盐州,日后攻下兴灵后,他也能分润到一点功劳,可以将功赎罪嘛。人同此心,难道说李都知就甘心于区区三百斩首的功劳?”徐禧微微一笑,这样舌辩群儒的感觉真的很好——尽管他辩论的对象一个是阉人,一个是武夫。胜之不武啊!他得意的想着。

李宪脸色发黑。他不知道该怎么说,硬顶天子派来的监军,文官可以、武将偶尔也可以,但他这个宦官,就不怎么方便了。

何况自从出兵以来,除了跟骚扰粮道的西贼打了一场,拿了三百斩首之外,别的功劳就什么也没有了。

从李宪的角度来说,他当然是希望能有所成就的,但眼下的情况并不是可以说稳拿稳的占据银夏,甚至可以说机会和高遵裕苗授战败前官军攻取灵州的几率完全相反,西贼远远强于官军。

李宪没办法,不得不又向种谔使眼色,让他出头来推脱。

种谔权当没看到。

他不是李宪,鄜延路的兵力远远超过河东,如果守住银夏的计划能成功的话,最大的一份功劳必然是他的。

依种谔的本心,自然是希望能领军直驱兴灵,独占全功。战前他对这一战的胜利没有任何怀疑,光靠鄜延路的兵马就足够了——要知道,在这之前,辽夏两国内部同时爆发内乱的消息,不仅仅诱使大宋起兵伐夏,同时也让西夏国内的部族开始分崩离析,只要攻得够快,一脚就能将这座破房子给踢垮。

可自己出兵后又硬生生被叫了回去,不仅仅将官军内部的分裂暴露出来,也给了梁氏兄妹收拾内部的时间,同时还让西夏国中部族看到了辽国对他们的支持——耶律乙辛清扫叛乱的速度实在太快了,一眨眼的功夫就安定了国中,领军到了鸳鸯泺。

高遵裕和苗授之前的失败,不仅仅是因为决堤放水的缘故。如果速攻的话,何须半月围城?就是放水,也可以找个高地避水,等水退了就再攻过去。以西夏人当时的准备情况,恐怕连地窖里的存粮都不会来得及掘出,而田中的麦子也正式收获的时候,根本不用担心粮草和士气的问题。

种谔恨得牙疼。朝堂中就是一群白痴!

反对速攻的韩冈那时候还知道决不能撤军——这才是真正懂得兵事的文臣——可天子、宰相为了让所有人平分还不存在的伐夏之功,竟硬是将他逼回去。

种谔沉默着,李宪的眼睛都要抽筋了,都不见他出言反对。徐禧看在眼里,心中得意的暗自欣喜。

他不信种谔不想翻盘?!种家的老五可是出兵西夏的首倡者之一。外面都有传言说种谔不死,边乱不已。一旦种谔就此认命,事后算账,罪责包管跑不掉他的。

徐禧趁热打铁道,“子正,须知眼下官军并不是全师败绩,环庆、泾原两路的损失说不定也不会太大。这一战官军只不过是小挫而已。兵虽然少了,但粮草相对的也就多了起来。而且别看髙、苗二帅惨败,西贼也是杀敌一千自损八百。加之坚壁清野的战策,从来都是先伤己后伤人,西贼的损耗只会在官军之上。”

徐禧口舌无碍,“即便与中国损失相当,但以中国之人财物,岂是偏鄙小国可比?富家翁丢个千八百贯也不会伤筋动骨,换做一个中户,可就要倾家荡产、卖儿卖女了。”

种谔觉得自己可能的确是太小瞧人了,徐禧不完全是军事白痴,好歹也是赵括、马谡一级,口才足以打动人。

“的确官军比西贼更能撑得住,不过前提是要将银夏守住。可眼下士气都消磨光了,银夏之地本来就是利攻不利守,如果有环庆路与我摆开犄角之势,倒不是不能顶住,但眼下鄜延路独力难支,如果西贼猝然来攻,不是硬拼就能赢的。”

种谔不理脸色难看起来的徐禧,“现如今环庆新败,西贼气势正盛,如果他们不攻西夏,反而南下去攻打环庆路,围魏救赵的话,又该如何?万一王中正失败的话,鄜延、河东就要独抗西贼,就算我们有信心,可天子和宰辅们会答应吗?”

种谔不可能甘心放弃银夏之地,但他绝不会坚守盐州和宥州,后者粮道太长,前者是环庆军打下来的,坚守和夺占相比,功劳能有三分之一就不错了。

至少要退回到夏州,乃至夏州和银州之间的石州,这样才是最稳妥的办法。而且兵力不能堆放太多,只能算是前出的据点。无定河畔、紧邻宋夏边境的银州,以及连接河东、鄜延两路的弥陀洞才是必须要保住的战略要点。

“行军、作战时的粮草消耗要远远超过大军驻扎下来的时候,而随着粮道的延伸,路中的损耗也会大大增加,运送到前线军中的粮草数量不断降低。试问陕西的储备够不够支撑到冬天?

今年五月的收获季,由于陕西民夫的大量征发,据说让今年的收成降低整整一成还多。今年陕西的税赋可能会为此被减免,储备吃空了,到时候粮草怎么办?

而且收获时节征发民夫运粮,到了初冬的耕种时节还要征发民夫运粮,能不能让天子和朝堂同意?之前没有问题,但现在高公绰和苗授之都已经失败了!只能先放弃靠近瀚海的盐州、宥州,保住夏州、石州不失。”

种谔一个借口接一个借口,但他的目的只是要挡住徐禧罢了。

种家的子弟可不是一点小挫,就会甘心认输的主。就像如今流行于陕西的蹴鞠比赛里的说法,眼下不过是上半场结束,还有下半场没开锣呢。只要能请动朝廷让自己指挥全局,只凭鄜延、河东两路的实力,依然能压倒西贼。

而且徐禧说的也没有错,西贼眼下虽然击败了高遵裕和苗授,但被打到了腹心之地,还要决堤放水,实际上的损失比起大宋这里要严重得多。眼下就只要坚持下去,最后的胜利就像道边垂下来的梨子,探手就能摘到了。

徐禧算是确认了种谔的态度。同样是保银夏,但要保住的范围还是有所差别。

“子正的心意,徐禧明白了。”徐禧微笑道,“你我虽有分歧,但保住眼下的成果的想法别无二致,不如先搁置争议,将想法相通的地方奏上朝堂,免得有小人先行动手的,撺掇了天子退兵,到时候可就是难以挽回了。”

徐禧暂时不想跟种谔争了。先得让朝堂同意保住银夏,至于保住哪些地方,等圣旨颁下来在扯皮也不是不行,种谔胆子小,他手下可是有胆大的。

种谔也正要这个回答,“学士之言甚佳,种谔岂敢不从?……子范,你意下如何?”

徐禧、种谔同时望向李宪,河东路的力量,眼下缺不得。李宪的助言,也不能缺少。

李宪沉思着,徐禧的方案太过自以为是,但种谔的打算,倒还算稳妥。他微一欠身,“不才愿附骥尾。”

