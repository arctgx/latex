\section{第25章 阡陌纵横期膏粱(四)}

“司马光最近又写了三份奏章,《谏西征疏》、《乞罢修复内城壁楼橹及器械状疏》和《乞不添屯军马疏》,对河湟、横山二事横加阻挠,调他去关中看来是错了!”

“司马光到了京兆府后,不修战备,不厘军务,只顾着写文章。韩子华在延州剑指罗兀,若是得不到京兆府的支援,横山局势必然糜烂。如果司马光不能改弦更张,就必须把他调走才行。”

“换谁?”

“把郭逵调任京兆府如何?”

“恐给关中平添一分变数。”

“郭逵在秦州就没有干扰过缘边安抚司一星半点,可见他是吃过教训后,便洗心革面了。回到关中,只要能配合延州,韩子华也不会再说他什么。”

王安石一边回忆着今早发生在中书制置条例司中的一番争论,一边亦步亦趋的跟在天子赵顼身后。

十月下旬,京师南郊的皇家苑囿玉津园,满园的菊花已是凋零殆尽,而腊梅却还未到绽放时节,枫树、黄栌的红叶现在大半都落在了地上。园中放养的那些来自南方的珍禽异兽,如狮子、大象、孔雀,现在都在暖房里闭着中原严冬的风寒,也不能放到外面来,让驾幸此园赵顼看个热闹。

不过赵顼到玉津园也不是来看狮子大象的。最近一段时间,他在宫中待着憋闷,他的奶奶和母亲,也就是太皇太后曹氏和太后高氏两人,一直都没停过对变法之事的抨击,让赵顼实在有些难以忍受。趁着今日天气甚好,便在结束了朝会之后,到玉津园中散散心。

可是就算散心,一向勤政的赵顼也不会把政事放在一边,王安石今天就跟在他身后。一众宰辅中,也只有王安石有此恩遇。

最近陈升之因母丧而丁忧去位,如果在英宗朝以前,宰辅丁忧,当是会在一两个月之内就夺情起复,不需要庐墓守制。但自前几年富弼在宰相任上丁忧,推辞了夺情诏书,为亡母守孝三年后,就再也没有哪个宰执愿意冒被言官抨击、士林鄙视的风险。今次就算赵顼想要夺情,陈升之宥于士林清议,当也不会点头答应。

至于首相曾公亮,他经过了一番惯例的挽留和坚辞的戏码后,已经在两个月前卸了职司,到京城外找地方养老去了。次相陈升之今次丁忧守制,也就是说,如今的政事堂中,宰相的位置全都空了下来。

虽然赵顼还没有御内东门小殿,招翰林学士锁院草制,但王安石和韩绛两人升任宰相早已是定局,板上钉钉的事。尤其是王安石,要不是他谦让,以他的身份早在去年就该玉堂宣麻、金殿拜相了。如今韩绛领军在外,他的宰相之位只是为让他能更加稳固的掌握关西的军队,真正的宰相其实只有王安石一人。

君臣二人踏着落叶,在枫树林中慢慢走着。班直侍卫们都围在林外,将整座林子给封锁起来。赵顼和王安石都没有说话,静谧的小树林的深处,只有靴底踩断枯枝才会发出轻微的劈啪声。在这异常安静的树林中,时间和空气仿佛都被凝固。

沉默了走了一阵,赵顼终于出声:“王卿,王韶他们何时会到京城?”

赵顼这是在明知故问,王安石知道年轻的天子这些天来,对王韶的行程一直都放在心上,什么时候走到哪里,他都很清楚,现在只是开场白而已:“王韶当是在这几天就到了。”

“人既然都快到,关于渭源之战的赏格怎么还没定下来?”

“此为枢密院所辖事务,陛下可召文彦博来询问。不过枢密院至今尤要治韩冈、王舜臣用兵不力之罪,赏格也便难以订立。”

“因为缘边安抚司前后加起来总计接近千名的伤亡?”赵顼停住了脚步,回头对王安石叹道:“这一战,战马也的确折损得得太多了。”

王安石默然,渭源一役连战死带病死的战马超过了三百匹,如果加上蕃人的,则接近一千匹。

“比秦凤、泾原两路今次的损失加起来都多!”赵顼说起战马的损失,就是一副痛心疾首的样子——因为大宋军中的战马实在太少了。

有马的称作骑兵,没马的唤作步兵。可是在如今的大宋,就算是骑兵,也不一定有马。‘天下应在马凡十五万三千六百有奇’,这是去年枢密院连同群牧监一起统计上来的数字。也就是说这十五万三千六百匹马,是如今大宋军中的在籍军马总数量——包括了驮马、驿马和战马。而以驮马、驿马及战马之间的数量对比,一般是在三比一左右,也就是说真正可以上阵冲杀的战马大约是在四万多。

这些战马基本上都分布在河北、京中和陕西、河东,尤以关西缘边四路为多。其中分配到秦凤路的战马为五千。

但是就跟登记在兵籍簿上的人数和实际的兵力之间,有着极大差别的情况一样。秦凤路写在纸面上的战马数量,其实也跟真实数目有着很远的距离。明面上的五千骑兵,实际上仅有四千余人,其中拥有战马的,则更是降到了三千多。

除了秦州城中的两个指挥接近满编,其余驻扎在各个边境城寨的骑兵指挥,基本上只有六成到八成不等的兵力。而且这还是在年年战事不断、兵员空额不多的秦凤路,如果是在河北、中原等地,情况其实会更糟。

赵顼只是对军中的空额稍有了解,看到今次在渭源的骑兵损失,就已经心疼得不得了。而在地方任官三十年,在群牧监也做了几年判官的王安石,对军中弊端,比赵顼肤浅的认识可是深刻十倍。

——陕西河东的实际兵力,可以按兵籍簿上的八成算;京中、河北则得按六成计;蜀中、荆湖能动用的军队,大概是实际数量的四五成;至于江南,直接当作没有比较好,那里的军队做小买卖的本事比拉弓射箭要强,在官宦门下奔走的时间比拿着刀枪的时候要多。而战马的情况也是与人一样。

除了战事不断的陕西河东以外,大宋其他地方的军队早就烂透了。在军中势力盘根错节的将帅,把大笔的军费花在自家的宅院里。占据了每年国家财政支出八成的军费,就这么让大大小小的军痞给分块吃掉了。有多少用在了兵备上?

王安石为王韶辩解道:“如果王韶建功,顺着熙河而来的战马,能把所有的亏空损失都填满。”

“可汉儿的确不如蕃人堪战。托硕、古渭两次大捷,王韶动用的都是蕃人,损伤少的可怜,而今次对上的禹臧花麻,让王韶动用了缘边安抚司的军队。最后的结果是其他人只是被迫退而已,虽为大捷,但损伤比起之前两次,可是要大得太多。这样看来韩绛在延州做得还是有原因的,虽然强取了庆州广锐军的战马,但蕃人有了马后,就是如虎添翼。”

王安石一时不知该如何说才好。对于陕西宣抚司内部的事务,他不好插手干涉。而且韩绛其实是代王安石去的陕西。就在去年,因郭逵对横山的战略与种谔相争,还有朝中对新法的攻击,使得王安石曾有了自请出外去陕西的念头。

当年庆历新政的失败,有很重要的一个因素就是主持新政的范仲淹,因三川口之败,而离开京师去陕西代替范雍任陕西宣抚使。当时王安石若是去了陕西,新法也很有可能就此夭折,韩绛对王安石的恩情甚多。在情在理,王安石都不便在陕西军务上干涉太多,反而要为他鸣锣开道。

‘也不知横山那里能给出什么答案。’王安石心里想着。

韩绛和种谔在罗兀城上的失算给了宣抚司上下当头一棒,韩绛现在的做法,很明显现在是在拯救横山的危局。相较于横山,河湟的地位就不那么高了。

如果在十年前,或是二十年前,当党项人倾巢而出,关西四路没有被攻下一座重要的城寨就已经是个可喜可贺的胜利。

可如今,大宋的国力日盛,对于仅仅是逼退敌人的胜利,再算不得什么功劳。就像今次的渭源之战,让禹臧花麻狼狈而走,虽然因为对付的敌人不同,而难度则更高,只是跟前两次大捷的战果比起来,感觉上还是黯淡了许多,赏格怎么也高不起来,对此不满意的人也很多——不仅仅只有天子一人。

至少韩绛是不满意的。从他这段时间的几份奏章上可以看得出来。他对秦凤路不能全力支援横山颇有微词。他现在一门心思都放在罗兀城上,靠着他的宣抚使身份从各路征调粮秣,通过了近一年的积累,韩绛在关西已经有了不低的名望。关西诸路的大概是为了求一个耳根清静,也都答应了他的调及。

王安石重又跟着再次安静下来的年轻天子在树林中走了起来,‘管不了那么多了……’

就跟他全心全意的放在新法的施行上,看不见其他的东西一样,韩绛的双眼现在应该只能看见罗兀城的背影。军功让人垂涎。一旦功成回朝,他就将是名副其实的宰相,一人之下万人之上,这让人如何不疯狂?!

王安石也只能选择坐视,而无法插手其中。

等到到了午后,王安石方才回到政事堂中,一桩奏章正被放在他的案头上,奏章上的贴纸说明了来历,是韩绛的文字。

“又来要什么?”王安石微微一笑,展开奏章看了一眼,只是调用一个从八品的选人,不算什么大事。但等王安石匆匆浏览了一遍后,脸色却突然变了,“韩冈迁调延州,管勾鄜延伤病事?!”

墙角竟然挖到了王韶脚底下!

