\section{第30章 随阳雁飞各西东(23)}

一年多前只能望空而叹的灵州城,现在就在种谔的面前。

灵州城并不高峻,也不算雄伟,隔了五里地远眺过去,不过是原野上的一小团阴影。就是到了近前,应当也不需要将头仰得多高。

可这座城池对种谔,乃至整个大宋的意义都绝不一般。

在立国之初,灵武节度使冯继业归降,灵州便孤悬在外。在咸平五年【1002年】为李继迁所夺,知州裴济死难。此后八十年,贺兰山下的这片土地,便成了党项人不断发起南侵的策源地。年年岁岁,岁岁年年,陕西的子民都在烽火和号角声中度过。

重夺灵州,恢复兴灵,灭亡西夏,这是数代大宋天子的夙愿,也是无数西军将士的夙愿。

如今西夏早已灭亡,剩下的也就只剩灵州。

一股冲动要让种谔下令全军攻城,将灵州一举夺占,但他立刻就冷静下来。

比起去年年中时,攻到灵州城下的高遵裕和苗授所率领的十万人马,种谔如今带在身边的只有可怜的两千骑兵。甚至没有民夫的支持,粮草的运输只能依赖搜罗了整个银夏路才得到的四百余辆四轮马车——这主要是从青白盐池运盐用的——只能是勉强支应。

据探马回报,城中的守军早早便将灵州的城门紧闭,区区两千骑兵,不可能攻下这座城池。

这一认知,让种谔心中沉甸甸的。早在北上兴灵的道路上,他撞上了不少辽人游骑,大军北上的情报早就泄露了出去,辽人自然是早已知晓。现在辽军主力并没有等在灵州城下,而是不见踪影,除非能得到他们和党项人的准确情报,否则种谔怎么也不可能放得下心来。

种谔麾下的将士们正在打造今夜的营地,利用了灵州城南被废弃的一座旧日卫堡,倒是很快就将营地搭建了起来。

补全了围墙,修好了箭楼,一顶顶帐篷出现在营地中,安营扎寨的工作只用了一个时辰。

当两队亲兵开始巡视营中内外,种谔神色中仍不见缓和,眉头皱着,显是心事重重。

“大帅。种建中回来了!”年轻的盐州知州出现在种谔面前,抱拳行礼。风尘仆仆的一张脸,眉眼间都凝聚着兴奋。

种谔此时早已换上一副轻松平和的神色。他望了望卫堡下,进入营地的马匹和牲畜远比他出去时多得多。笑容更加轻松了一点:“这一趟收获不少啊!”

抵达灵州后,种谔做的第一件事是派出斥候,追询敌踪;第二件事便是出兵抄掠,就地取食,不然就是坐吃山空,无奈退兵:随行马车携带的干粮干肉支撑不了太久,后方也无法安然运送更多的粮食。

“在山坳里撞上了一个小部落,总计斩获了六百多腔羊,全都赶回来了。草料三囤,干豆和麦子有七八百石,已经留了人手看着了,还请大帅派人去运回来!”丰收而归的种建中禀报战果时中气十足,轻兵而出,最重要的就是抄掠到足够的食物,“还有马和骆驼,加起来也有一百三十多匹!这一路上折损的马匹不少,这一下子能补上一些亏空了。”

种建中身上的血腥气浓得化不开,罩在盔甲外的外袍上,桃花瓣一般的血迹星星点点。种谔眼尖,就在堡下的种建中的坐骑马鞍后,还挂着四颗男子的首级。左右各二,与插着铁锏的皮袋紧紧贴在一起。

种谔没对种建中带回来的战利品多关心半点,“可有辽军主力的消息?”

种建中摇了摇头,声音低了许多,“没有。大半是牧奴,几个看起来有些地位的也说不清楚。只知道耶律余里回来后,就立刻领军往西面去了。不知道是西北的兴庆府还是西南的青铜峡口。此外,留在灵州城中的守军似乎并不多,据说不到一千。”

种谔眼神阴沉,在敌人的土地上,情报远比食物重要。连辽军主力的动向都在抓不到,结果可能会很糟。深呼吸了一下,整理了烦乱的心绪,他又问道:“有没有损失呢?”

种建中脸亮了起来:“就折了一个兄弟,还有五个受伤的。一个重伤,其他四个全是轻伤,包扎一下就能再上阵。”

种谔点点头,神色松缓了一点点,“将伤亡的儿郎送去医工那里。马和骆驼交给杨勇。至于羊,全都分下去。跟杨勇说,随车带来的酒也都一起分下去。让儿郎们过一个好年!”

“诺!”

种建中抱拳行礼,便转身大步离去。

种建中下去了,种谔依然站在卫堡的最高处,在暮色中远眺着灵州城。远在地平线上的城池看起来小巧精致,似乎张开手就能攥在掌心中。

种谔不顾全军覆没的风险,沿着灵州川北上,可不就为了这座城池?只要打下了灵州,兴灵将一举平定。

种谔低头看着下方。灰黄色的地面上,是一层细细的黄沙。传说中丰腴堪比江南的兴灵之地,眼下是满目黄尘。细细的黄沙随着风被卷起,犹如沙漠一般荒凉。

去年高遵裕在灵州城下因党项人掘了河渠而惨败。从渠中流出来的黄河水,淹没了灵州城外的良田。由于西夏紧接着就灭国,接着又被不擅营治的辽人占据,灵州城下的田地完全没有回复的迹象。也不知河渠的决口,到底是被堵上了,还是因为冬天水枯的缘故,没有流出来,反正水退之后,剩下的就是黄沙。

此时种谔领军驻扎的卫堡,是灵州城外不多的高地之一。去年党项人破堤放水,一时水漫荒原,奔涌而出的黄河水淹到的三十里外。种谔从南面过来时,残留在地面上的水痕是很明显的。洪水推动沙砾,在大地上画出了好几条波浪起伏、两边都望不到尽头的平行线,环庆、泾原两军应该有不少官兵退到此处。方才修寨防时,就找出不少遗骸残兵。

白骨森森,铁锈斑斑,看见一层薄薄沙土下的袍泽遗物,不少士兵都泪水盈眶。种谔命人好生处置了,也不禁惨然于心。

都是高遵裕做的孽。不过也是兴灵的位置太过偏僻,孤悬在荒漠和高山之间,远离陕西的核心之地。攻进来很难,攻进来后想安然而退就更难。这一次若不是确定了青铜峡中以仁多、叶两家为首的党项部族已经攻入了兴灵,种谔也不会如此冒险。

耶律余里到底在何处?

种谔苦苦思索。若他是耶律余里,绝不会分兵守灵州。要么就全军坐守灵州,先将官军击败,要么就去攻打党项人。

耶律余里不能算是名将,不过一中庸之才,只是再蠢的将领也该知道在面对大敌时分兵乃是取死之道,若是让宋军与党项人合流的话,更是脖子上套了绳索后往悬崖下跳。如果耶律余里没蠢到家的话,灵州这边他必有布置。

种谔望着沙地,心有所感。一路上他从来没有掩饰过形迹,北上数百里,辽人早就知道有一支宋军追在身后。耶律余里能做的选择只有寥寥几个,种谔都做好了应对。

欢呼声在种谔的脚下响起,瞬息间传遍营中。种建中带回来的收获,让两千多将士欢呼雀跃。

年节时出征,虽然有种谔这位深得军心的名将统帅,士气也是有所折损。幸好种建中弄了一批鲜肉回来,免得种谔下令分解死掉的马肉了。

篝火熊熊。

种建中虽有一个文官出身,但现在他的模样却让人根本看不出来。内袍扎在腰间,赤裸着精壮的上半身,丝毫不畏深夜的寒风。

他拿着把精钢匕首,一刀捅在肥羊的脖子上,鲜血立刻咕嘟嘟的冒了出来。种建中紧紧揪住拼死挣扎的肥羊,身上的腱子肉一块块的鼓起,让打下手的亲兵拿着头盔接了血,撒了一把盐进去,就放在一边。待会儿凝固了,与羊脑、下水一同炖煮,味道可是鲜美无比。

营地中四处飘起了肉香。油汪汪的烤羊肉,配上泡了干饼的羊杂碎汤,再加上热腾腾的酒,这个身在异乡的除夕之夜,倒也算得上是惬意了。

种谔拿着酒碗,走过一堆堆篝火。一群群士兵跳起来,诚惶诚恐的接受种谔的敬酒。麾下的将士,种谔认识不少,有许多都能叫出名来,喊着名字,拍着肩膀,再对饮过一口酒,换来的就是效死之心。

“大帅!我有急事禀报大帅!!”

绑着两名俘虏,一队斥候突然间出现在了营地外。惊到了正在欢庆中的宴会。

为种谔提着酒壶的种建中极为惊讶,甚至都愣住了。派出去打探消息的斥候回来了,这当然是好事。终于捉了两个生口,而且还是带着重要情报的生口。甚至可以说是喜事。

但种谔派出去的斥候,基本上都是他所看重并准备提拔的底层军官。沉着稳重是必备的素质,就算打探到了什么重要的敌情,也不敢一进大营就开始嚷嚷。

“说!”种谔平平静静.甚至当着所有人的面。

“辽人在七级渠,正准备决堤放水!”

种谔拿着酒碗的手轻轻一颤,立刻又稳定了下来,“不妨事,掩不到我们!”

