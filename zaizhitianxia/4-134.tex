\section{第20章 冥冥鬼神有也无(十)}

已经进入了冬天,十一月的关西早已经连黄河都冻上了,雪层也会就此覆盖荒凉的黄土高坡,但广西的冬天一点也不见寒冷,流水潺潺、草木青青,甚至还有新芽野花点缀在路旁,仿佛北方的阳春三月。

和煦的阳光从高广无垠的碧空中散射下来,没有阴湿的空气,没有扰人的蚊蝇,一时天高云淡。连到了广西之后,始终难以适应的西军将士,这一下子都精神了起来。

“相公在信中怎么说?”章惇问着与他并辔而行的韩冈。

在两人的身前身后,是数以千计的西军精锐。他们是早上从归仁铺出发,到现在已经走了快二十里,而韩冈则是天还没亮的时候,亲自去归仁铺迎接大军的到来,顺便还从铺兵的手上收到了王安石和家人寄来的信笺。

韩冈一手揽着缰绳,一手将刚刚收到的私信递给身边的章惇,“只是说天子看到了降表就丢到了地上,说交趾欺人太甚,定要打破升龙府,将交趾君臣全数拘上京去问罪。又说让我们不要着急,谋划妥当再行出兵。”

章惇一目十行的看了王安石写给韩冈的信函,跟韩冈说的并无二致。抬头与韩冈对视一眼,两人一齐摇头苦笑了起来,

王安石就是这么说才有问题。如果朝堂上一片平稳,没有任何的反对之声,他根本就不该多提什么‘谋划妥当再行出兵’。肯定是有人建言天子纳下交趾的降表,就此偃旗息鼓,所以王安石才担心章惇、韩冈心急,急着去攻打交趾,以至于犯下难以挽回的错误。

会如此劝谏天子的是究竟哪几位,韩冈也能猜得出来。不过天子的反应让两人可以松下一口气,至少两三个月内不用担心后方有问题。

“关键是军中该怎么办?”

章惇的帅旗就高高举在两人的马前,被暖风拂起的旗尾撩到了韩冈的耳边。

身前身后的士兵们,正昂首挺胸的高举着旗帜和刀枪走在官道上。一个个雄纠纠、气昂昂,睥睨当世,顾盼自豪。已经完全不见了一个月前,笼罩在全军上下的病恹恹的模样。

但这五千西军将士他们是为了击败交趾而来,为了封妻荫子的功劳而来。只是这一束挂在一众驴子眼前的鲜嫩多.汁的草料,仅仅是针对武将而言。对于最下层的士兵来说,就算没有与交趾战斗过,也不过是少一点赏赐罢了。如果朝廷接受了降表,他们不仅是更早一步离开广西这个鬼地方,还能免去了去更南方的交趾受罪。

现在的这副气派只是因为没有听说交趾献上降表,只要目标投降的消息在他们中间传开了,求战的氛围肯定大打折扣。

“兵不厌诈,瞒是肯定瞒不过,只能砌词骗过去了。”章惇一下又摇起了头,“也不能说是骗,交趾人本也不是真心投降。只有自缚出城才叫投降,只肯献上降表那就是假的,只是想混过去而已。”

“光这些可不够。”韩冈并不觉得有多少说服力。在这个时代,献上降表已经可以算是投降了。即便拿出太祖皇帝的卧榻之论,也不能压住所有的异论。

“那就只有升龙府了。”章惇毫不犹豫将交趾王都丢出来当做赏赐。。

‘开城大掠吗?’韩冈只能选择摇头,“一旦放开来劫掠,我们身边还能剩几人?到时候若是城中一个反击,我们可就麻烦了。隋炀帝二征高句丽,隋军已经渡海打进了平壤城,就是因为大军散开来劫掠,才被打个全军覆没。”

“玉昆可有良策?”

“良策倒是没有,不过周毖有个建议。他家里面开了几间质库,在财计上有些长才。”韩冈现在是尽量的考验和锻炼他的几名幕僚,能交给他们的工作就尽量依靠他们,也经常让其出谋划策。关于如何保证军中士气,韩冈虽然没有向幕僚们询问过,但关于如何划分战利品,才能让军中上下能基本上都认同,倒是当做课题考过李复、周毖等四人。

章惇对韩冈的这位幕僚有点印象,前段时间被韩冈留在桂州检查漕司账目的,“他怎么说?”章惇问道。

“按周毖的提议,最好是事先约定好如何分账,将校的、士兵的,事先定好规矩,等到开城后,斩获全都拿到手后照比例划分,不仅仅是官库,所有的收获都如此。”

章惇皱眉想了一想,要保证入城搜刮民财的军队没有私心有些麻烦,即便派人搜身也免不了私藏,但这个策略已经勉强算是可堪一用,“不算最好,也只能这样了。是五五还是四六?”

“官中难道不要占一份?一点不沾,下面的士兵反而不会相信,四三三才对!”韩冈很熟练的说着:“士兵们占四成,将校三成,最后三成没入官中。”

两名强盗首领讨论着如何分赃,脸上也是一点也不见有半分愧色,因为他们觉得自己理直气壮。他们准备做的,就跟交趾人在邕、钦、廉一样。只是讨论之后,互相一看,又都摇头苦笑,脸皮都还不够厚,至少心中也不可能将这等事当成理所当然。

就像要将所有交趾男丁处刑,韩冈、章惇都是心意已定,不会更改,也不会有半点犹豫,但整件事,他们也是准备尽量交给三十六峒蛮部去做,因为那将是蛮部的奴隶。

“粮秣是否都已经齐备?”章惇转移话题,避过继续讨论让他们尴尬的问题。

“不可能不齐备吧,丰州可是帮了大忙。”韩冈笑着,语气中带着讽刺的辛辣,“人马少了一多半,吃喝当然也少得多了。我这个随军转运,可是再轻松不过。”

章惇也跟着笑了一声,“安南行营何尝不是。”抬头远望走在队伍最前的燕达,“我和燕逢辰,可算是领军最少的总管和副总管。”

整个安南行营下辖的官军兵力,加起来才一万两千人,其中近一半还是新兵,派不上多少用场。唯一的好处就是在兵力减少的同时,后勤上的压力大大减轻。韩冈并不需要为三十六峒蛮部和广源军筹划粮草,只要顾着自己人就够了,现在反倒是军中使用的牲畜,比人吃的要多得多。

“药材也备齐了。”韩冈又补充道,“防暑、避瘴的药物一车一车的从北方运来,而且最多的还是用来薰衣驱蚊的艾草。本来是足够给三万人一年支用,现在才一万出头,用上两三年都没问题。”

章惇点着头,突然就直起了腰,“该进城了。”

邕州城快要到了,避让到道路两边的行人,也渐渐的多了起来。大部分人的眼神都是在好奇中带着兴奋,看着传闻已久的精锐之师,只是隐隐的也有着些畏惧。

韩冈抬头看了看天色。冬日的邕州,天上看不到一丝云翳。已经有半个多月没有下雨,不过没多少灰尘,阳光也是正好。

旌旗招展,五千马步禁军排列整齐行进在官道上。

就在进城之前,他们全都换上了晶晶闪亮的铁甲。在桂州休整的二十天里,关西将士们闲下来,就是打磨身上的甲胄、和手中的刀枪,磨得不见一点锈色之后,再抹上薄薄一层防锈的牛油。现在无论刀枪还是甲胄,都在阳光下,闪烁着慑人的寒光。

三十六峒诸部、只要还在邕州的洞主们,此时全都聚集在城外,迎接大军的到来。

数以千计全身铁甲的精兵,让一个个将皮甲藏在家里当成宝贝的蛮部洞主,看得心惊胆寒。

“那么多铁甲……”

“竟然人人都有!”

“听说北方的六十万禁军,全都有铁甲。只有南方的官军怕铁甲生锈,才只配了皮甲。”

“三十年前,狄相公来平侬智高的时候,也没说人人都有铁甲。”

“都过去三十年,当时生的小子,连孙子都能有了!”

“你们难道没听说?外面可都传遍了。现在官军穿得甲胄,是转运韩相公所造,比旧时铁甲容易打造一百倍,所以能一造数十万领。年纪轻轻都做到相公,都是攒功劳来的。”

同时一柄柄高举在手中的斩马刀,也让洞主们心头寒气直冒。

“一柄刀就用那么多铁。少说也能打造十条长枪!”

“没看到锋刃吗?看颜色就知道哪里是铁,根本是精钢啊!”

“难怪说他们比荆南军强上十倍,全都是钱堆出来的了。”

“也只有朝廷才这般有钱,换作是交趾,穷得跟猴子一样。”

可随着官军越来越近,慢慢的就没有人说话了,只看着炫花了双眼的铁甲,还有一柄柄似乎能连人带马一齐斩断的长刀,每一名洞主的身子都在颤抖。

千军万马整列行军,脚步声渐渐的汇合一个声音,如同夏日午后深黑色的雷云,沉沉的压向所有人的心头。

原本的一千五百从荆南调来的军队,已经让十万交趾兵大败而逃,现在又来了五千据说比起荆南军强上十倍的西军,灭掉交趾岂不是易如反掌?

幸好投了官军!这样的想法充斥在每一名蛮部洞主的心中。

