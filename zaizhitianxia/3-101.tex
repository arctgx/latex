\section{第29章 百虑救灾伤(六)}

京城中的米店,门面通常不大,只是进深颇深,以便于存放粮食。在门面处,一边都挂出一溜木牌,上面写着当下的粮价。同时在亮出来的样品上,也会插个价格牌。在商行中少有的明码标价的传统,使得顾客们不要进门,就能一目了然的看到现在的行情。

不过这个传统,许多时候也让进门来的客人们感到痛苦。红漆涂标的百三十文米价,高高挂在最醒目的位置上,红灿灿的,不但刺眼,更是伤心。准备买米回家的男女老少们来到米店前,抬眼看着标价木牌,无不是摇着头,却又无可奈何的走进店中来。

原本人们来米店买米买面,或是其他杂粮,基本上都是一次买一斗的为多。一般挎在臂弯里的专用的米篮子,一次正好装一斗米。只是现在,从米店里出来的百姓,他们手中的篮子通常都只装个半满。而经常让一次几石、十几石的将米面送到家里的官员和富户,如今的订购量也比过去少了很多——买不起的原因只占一小部分,更多的还是粮店囤积惜售的缘故。

粮食的飞涨带动了其他商品的同时上涨。以羊肉、猪肉、鸡鸭为主的肉类,价格同样翻番,菜蔬、零食无不是跟着粮价一涨再涨。同时日用品的售价,也在一片恐慌中,飞到了天上去。从熙宁六年的十月开始,到现在两个月下来,普通百姓的生活费用几乎是翻了一番。

且涨价的还不仅仅是关系着百姓生活的商品。在城中租马租车的费用,在车马行的协调下,以草料大涨的名义,统一涨了三成。至于酒楼食肆,教坊妓院,也毫无例外是大涨特涨。

七十二家正店,三千脚店,开封府中的这一干酒楼食肆,大部分已经变得门可罗雀,甚至有许多都早早的放了雇工们的年假,省得开张一日就亏上一日。在如今市面愈加的萧条,就算一些坚持开张的大酒楼,看到一个进来的客人都跟看到亲戚来访一样殷勤。而那些依然常来常往的老客户,更是将他们顶在了脑门上,当成了祖宗来供奉。

“换作过去,燕四哪会将吴楼的锦夜白一次拿出来这么多陪席?”

高扬摇了摇手上的酒杯,将杯中清澈如水的佳酿亮了给坐在对面的酒友看着。东京粮行的九位行首之一,同时如今带动全城物价大涨的元凶,对于现今百姓们的困境,却是笑得风清云淡。

“人总要吃饭的。”同为粮行行首的金平,则是回以更为寒冷的笑意。

高扬他家差不多可以改姓赵。他亲娘是县主;浑家算是他表妹,当然也是县主;而他被儿子娶的媳妇还是县主。另外还有个做进士的妹夫,虽然官位不高,但终究还是一个进士,如今也是京官了。而金平家的情况也是差不多,同样是赵家的女婿——东京城中,大一点的行会的行首们,不跟宗室攀上亲,混到一个官身,那行首的位置都别想坐稳。

“这两个月来,东京城内外可是怨声载道!”高扬悠然自得的笑着,“王相公的十八代都是一代代的被骂上去了!”

“就算王相公再如何能耐,也做不安稳了。更别说还在黄河中闹出那么个大笑话。”金平神色间透着狠厉,“前天我浑家循例进宫问安,已经跟太皇太后和太后都说了如今的情况。回来后说两宫听得忧形于色,太后甚至还痛骂了王安石。如今天子内外交困,王相公可是在政事堂中坐不了几天了。”

高扬轻轻点了点头。这几年来,他们这群人被新法死死压着,每一条法令出来几乎都是在割他们的肉。王安石为了给国库搂钱,尽在他们这些商人们身上打主意。跟宗室结下的姻亲,王安石竟然一点都不在意。均输法、市易法,这两条法令就像两把斧头,一左一右,一前一后的将他们这一干豪商们的老底给贴地砍了去,一点也不顾天家的情面。

幸好王安石倒行逆施的行径,现在连老天都看不过眼了,去年山崩、今年蝗旱,明年的灾情只会更大。王安石领衔的新党即便再有本事,也是难为无米之炊。

高扬举杯与金平对饮,一口干了之后拿着块丝巾擦了擦嘴,道:“今天早上,方十五那边提议说要将粮价再涨上去一点,如果能涨到一百五十文,王相公怕是拖不过明年元月。”

“不急,先放出风声去,而我们这边再收紧一点。离着年节还有半个月,腊月廿三送了灶神之后再涨价,效果会更好。先要逼着他动用常平仓出来。”金平恶狠狠的说着:“现在常平仓还没有动,外面还有人幻想着王相公尚有底气。等到常平仓一开,是个人就该知道王安石那边已经支撑不住了。如果明年灾情延续,谁还能指望常平仓拿出粮食来救灾?东京百万军民心中意乱,明年的粮价完全可以会涨得更高一点。”

“还是老哥想的周全!”高扬拍手大赞,站起身殷勤的为金平斟酒,“此事一成,不知多少人要感谢老哥呢!”

金平闻言自负的笑了笑,又道:“就算救得了眼前疮,可是到了明年,浑身可都会烂掉的。看王相公还有什么招数!”

只要是明眼人,都能看得出来,如今东京城内的问题并不是粮荒。京畿、河北的灾情是在夏收之后,而两浙的旱灾,也没有影响到南方供给京城的六百万石纲运。

只是延续秋冬两季的大旱已经搅乱了人心,使得高扬、金平这一干粮商们可以趁机上下其手。而且怨有所归,高扬、金平他们根本都不用担心自己的安全。

凭栏下望,正是东京城的南大门——南薰门。

南薰门与大内相对,一条南北向的御街直通内城。当年宫中大殿新起,太祖赵匡胤让人将宫门全数打开,立于宣德门处,可以一直看到大庆殿中的御榻上。太祖皇帝由此而言:‘此如我心,少有邪曲,人皆见之。’而外城的南薰门与内城的朱雀门、皇城的宣德门在同一条直线上,其实眼力若是有鹰一般的水准的话,也可以从南薰门一直看到大庆殿上。

正因为这此门直通宫城,以忌讳之故,寻常士庶殡葬车舆皆不得由南薰门进出。不过有个好笑的地方,带着晦气的棺材不给走,但更脏一点的猪可以走。不知是何时留下来的旧例故事,民间所用生猪——宫中只吃羊,不吃猪、牛——必须从此门进入京城,不得走其他城门。每天由此入京的生猪都有成千上万头之多。

哼哼唧唧的声音从楼下的大街传了上来,数百头猪被牧猪人赶着,顺着道路一路往城里走去。这些猪都是在城外交割过,已经属于肉行,现在送去给东京城中的各家肉铺屠宰,再从肉铺送进千家万户。

“肉行的生意也淡了,换作是去年,我们在这里坐了这些时间,好歹过去七八群猪。”

“徐仲正最近的日子可是难过。麦麸、米糠都在涨价,看明年还有谁人吃猪。”

高扬、金平两人对视一眼,幸灾乐祸的笑意从眼底传到了脸上,一同仰头哈哈大笑起来。

畅快淋漓的大笑声回荡在空旷寂静的酒楼中,百无聊赖的坐在柜台前发楞的掌柜燕四抬头看了一眼,然后又狠狠的向地上啐了一口。老主顾是要奉承,但不代表他不知道是谁将如今的粮价抬得如此之高。

高扬、金平还有其他粮行中的行首们,经常到他的酒楼中来小聚。半年前,他们还是唉声叹气,不时的还在包厢中大骂王安石,但这两个月来,他们脸上的得意越来越浓,也让燕四越发的看他们不顺眼。

粮行众人将快乐建筑在别人身上,燕四无所谓,最多叹上一口气,转过头去还是赚自己的钱。但若是建筑在自己的身上,燕四可没有佛祖一般的好脾气。

‘生儿子没屁.眼!’‘死后下油锅!’‘被米袋压死算了!’

在谦卑迎客的笑容中,吴楼大掌柜的肚子里,满是恶毒的诅咒。

一阵急促的马蹄声从门外传来,在门前停下。燕四立刻惊喜的抬起头,可等来人一进门,他又无力的垂下头去。吴楼的掌柜认识来人,乃是粮行中人,是高扬手下的亲信。

不待他相问,燕四向上指了指,道:“都在老位置上,直接上去好了!”

高扬亲信也不过话,连拱手都没有,大步就窜上了楼去。高扬家好歹也是跟宗室联姻的大户人家,对下人的要求也多,平日里不会这般无礼。燕四看着心奇,心道不知是哪边出了事,才会这样的着急。

片刻之后,楼梯上蹬蹬蹬的一阵响,高扬、金品两个大行首慌慌张张地从楼上下来,一个两个脸上的得意全都不见了踪影。请客的高扬跟燕四说了句“过两日来会钞”,就这么火烧房子一般的跑了出去。

见着他们的背影消失在门口,燕四一阵发楞,“到底出了什么事?”

