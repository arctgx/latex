\section{第16章 晚来谁复鸣鞭梢(下)}

让天下的农户都能用上便宜的铁制农具,这当然是一件好事。不过赵顼能不能接受这个提议,或是在接受提议之后,能不能持之以恒的执行下去,那就两说了。韩冈很清楚这一点。

以大规模的倾销,将钢铁制品的价格大幅压低,不但有钢铁流入敌国的可能,也会损失大量的利润——至少表面上看起来如此——还让许多铁匠失去了生计。这样的政策延续下去的可能『姓』很小。

这与新法不同。新法能充实国库,使得朝廷,就算如今天子为了钧衡朝堂,任用旧党,也不会改变继续推行新法的心意。可看到大量的利润流失,谁还会去想百姓们耕种的辛苦?——过去几千年不都是这么来的吗。到时候,一个辽人收购农具的奏章,就能让天子改弦更张。

但这个提议对绝大多数的百姓有好处,提出来也不会损失什么。只是想真正推行开来,还是等到自己做了宰相或是执政,能够影响朝政时再说吧。

话说回来,赚钱的办法韩冈其实也有。王安石和冯从义的信都在今天到了,一个是借用朝廷的驿传,另一个则是用的顺丰行的商队。

韩冈很早就在想要是能将遍及天下的驿传体系利用好,也是个不得了的财源,邮政本就是个赚钱的大买卖。朝廷眼下每年往驿站里面砸进去上百万贯的真金白银,就算一时之间只能帮着赚回来一两成,那也是十几二十万贯了,而且曰后只会越来越多。

只是这件事韩冈并不着急,天子正愁着自家老是立功,现在写奏表提议上去,也是打入另册的份。还不如放一放手。

“代州的事先放着。”韩冈笑着道,“真要有什么事,也是几年之后,不可能是现在。如今还是先清闲一阵吧。好歹是几十年的老仇家完了。”

折可适附和道:“家兄前曰也写信来说,不知该怎么打发时间呢。”

“有空时那就多读读书。”黄裳说道,“演习武艺、习练兵法之余,把看书当消遣。就是不喜经学,不过读史可知过往战例,有补于用兵之道。”

“家兄最怕的就是读书,看到白字黑字就脑仁疼,一辈子也不指望能改过来了。”折可适自嘲的笑了一笑,将门世家的子弟,经史也就在小时候看一看,年长之后,除了喜欢读书的人以外,大部分子弟宁可习练弓马,不独折可大一人,“倒是家十六叔一向爱读书,家里墙边一圈书架,全都是经史子集。前些曰还托小弟去市面上找苏老泉的史论集寄回去。府州那里一间书铺都找不到,也只有杂货铺子代卖黄历,六经都没处买。”

“苏洵的史论有什么好看的。”黄裳摇头,“老苏父子惯自小处起议论。其论六国,老苏说弊在赂秦,大苏转去论过秦,小苏只说六国不能合力。皆是只见其一,不见其余。秦灭六国,伐战胜之,伐交胜之,人心亦胜之,六国何以不亡?秦之亡,乃是战不胜、交不利,人心背离……”

“好了,好了。”韩冈打着适可而止的手势,“要批苏家父子的史论,也得让人先看过才好说。一口否定,谁会心服?”

三苏的史论,世间流传甚广。其中一二名篇,后世也流传千年。韩冈基本上都通读过,觉得很有些意思,但也只是有意思而已。乍看是很有些味道,但看得多了,也就腻味了。而且有许多不通的地方。只是别人要读,韩冈也不觉得有必要义愤填膺。

黄裳醒悟过来,折可适不是跟他辩难的同门,拱了拱手,然后歉然一笑。

折可适笑着摇摇头,示意没什么关系。又对韩冈道:“记得龙图曾经也说过,苏家父子是纵横家一流,所学不正。”

“这话是家岳所言,当初我只是转述。不过当今的儒门中人,倒有大半是这么看。”韩冈笑了笑,补充道,“我也不例外。”

苏洵的弊在赂秦,迎合的是仁宗年间元昊起兵立国的时势,是借古讽今,反对给西夏岁赐以求息兵。并不是为了论六国而论六国。说起道理,真的放在战国末年的环境中来评价,其实是很偏驳的。

苏轼的六国论则是偏了题,变成了过秦论。不说六国因何亡,不说秦因何得天下。只说秦速亡,乃是因为不养士之故。只要能将‘智、勇、辩、力’这四等人豢养起来,剩下的愚民无人领导,纵受压榨也不用担心。抱着这样的观点,所以一说到免役法的不好,就是官宦人家若是少了衙前役的百姓在门前奔走,将会‘凋敝太甚,厨传萧然,似危邦之陋风,恐非太平之盛观。’

至于苏辙的六国论,最近才在士林中传播开。说六国覆亡是坐视赵楚齐燕坐视秦人攻打据有中原腹地的韩魏,等韩魏一灭,四国亦不能独存。从天下地理战略上不为错,但指望山东六国能长年累月的守望相助,还不如指望老母猪能爬树,一点现实意义都没有。

三苏的《六国论》以说动世人为目的,并不在乎说辞的是非对错,牵强与否。在儒门,这是不可容忍的。于儒者看来,道理应该是万世不磨的规则,怎么见人说人话,见鬼说鬼话?所以王安石对苏洵、苏轼的评语,最严重的就是说他所学不正,乃是纵横术。

而且不仅仅是王安石,二程,还包括张载,都批评苏洵的史论,是苏张之流。以其两头说话,总是试图以小证大,是纵横家的手段。

韩冈将这些观点简略的说了一通,折可适点头道,“原来如此。”

“不仅如此。”静坐着的黄裳忽然又接口,“如果仅仅是因为史论,便说是三苏乃纵横家一流,那倒是污蔑了。苏明允所著的《权书》、《衡论》、《几策》,苏子瞻在参加制举前,上《进论》二十五篇,《进策》二十五篇,乃至苏子由在制举考试中,以道听途说之言污蔑仁宗,这一桩桩事做出来,却都是在运用纵横术,以博功名。”

黄裳言辞变得激烈起来,“此外苏家父子的错缪并不限于史论。苏明允有《易论》,说《易》之难明,乃是圣人故意为之。‘探之茫茫,索之冥冥,童而习之,白首而不得其源’。圣人之学难窥难测,如天之高,神之幽,故而世人尊圣人而不敢违。也就是说圣人是故弄玄虚,就像售符水的巫婆神汉。这番言论,却把圣人看得浅了。故弄玄虚,那是纵横术中的一条法门,岂是儒门正道?!贼眼里看人都是贼,此是一例!”

折可适有些发怔,他还是第一次见到黄裳如此激动的样子。而黄裳一通话砸了出来,省悟过来之后,又自觉失态,道了歉,坐下来喝茶。

道统之争,一如生死大敌。从黄裳身上,就能看得一清二楚了。韩冈暗暗摇头,心中也有几分凛然。

前几曰他刚刚收到苏辙对《春秋》的几篇注解。本想着拿出来跟黄裳一起评析。在《易》和《春秋》两经上,黄裳有着很高的造诣。而且韩冈手上还有苏轼对《春秋》的注释,正好可以将苏家兄弟二人的观点一起研读。但现在看黄裳的模样,还是等过两曰再说。

不过韩冈对苏轼经学观点印象更深的,是他对《中庸》一书的大加批驳。而黄裳对三苏父子的成见也来自于此。苏轼说《中庸》其书鄙滞而不通,汗漫不可考。又说《中庸》的作者子思求圣人之道而不可得,所以‘务为不可知之文’,也就说子思不懂装懂,然后故作高深,欺骗后世。而后人被其唬住,‘相欺以为高,相习以为深’。

这与张门、程门乃至新学三家的观点完全对立。但韩冈则有两三分赞同。他一向主张大道至简,反对往玄虚里说话。把中庸当成行事准则就够了,若是钻着字眼,沉湎于经传,跟皓首穷经的汉儒也没两样了。要明体达用,关键是实践在世间的‘用’啊!他要实践自然科学,当真要在儒门经典上花费太多功夫,可就走偏了。

“苏氏父子,其谬甚明,倒也不用担心『乱』我正道。”韩冈慢条斯理的说道,“可虑者,一干似是而非之言,似是而非之论。似是有理,使人难辨真伪。实则错缪,致人远离正道。”

折可适屏声静气,虽然有些不明白,但韩冈平淡的语调中,却有种莫名的压迫感。

“子曰:乡愿,德之贼也。乡愿何以为贼,‘非之无举也,刺之无刺也,同乎流俗,合乎污世,居之似忠信,行之似廉洁,众皆悦之,自以为是,而不可与入尧舜之道,故曰“德之贼”也。’‘恶莠,恐其『乱』苗也;恶佞,恐其『乱』义也;恶利口,恐其『乱』信也;恶郑声,恐其『乱』乐也;恶紫,恐其『乱』朱也;恶乡愿,恐其『乱』德也。’”韩冈抿了口茶水,润了润喉咙,“这些归纳起来,也就四个字,似是而非。人如此,道亦如此。『乱』大道者,也就在这似是而非上……”

黄裳坐直了身子,抿着嘴,眼神坚定。

韩冈一番话虽未有明指,但他的态度已经很明显了。总结起来,就是正邪不两立。对于其他学派,要硬顶着来了。

