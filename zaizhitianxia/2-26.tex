\section{第八章 太平调声传烽烟(五)}

【第二更,求红票,收藏】

见王韶和高遵裕这么快就从沈起那里出来,没有被留饭。韩冈心知,看起来他们谈得并不投机,或者说,陕西都转运使被外人看到自己的狼狈样子后,心情不好,让王、高二人不得久留。

“怎么回事?”王韶见着王舜臣当街扭着一人,旁边还有一群观众围着,便问韩冈。高遵裕也皱眉看着街口的一片乱象。

“只是抓了个持刀行窃的小贼。”韩冈向两人解释道,“王兄弟正要把他送去见官。”

“哦。”王韶对这等小事完全没兴趣,他对高遵裕道,“地方一乱,作奸犯科的贼子就多起来了。”

“让傅勍别轻饶了他。敢在城寨里持刀行劫,必得狠狠治罪,杀一儆百,省得西贼的奸细趁机作乱。”高遵裕的这番话不是对韩冈说的。他带在身后的伴当听了后,跑到王舜臣身边,说了两句,便一齐押着小偷往城衙去了。

小偷被硬拖着走了,他的挣扎只引来了王舜臣的铁拳。韩冈对他并不同情,被抓包后竟然动起刀子。既然有杀人的念头,那被打死也是活该。

倒是失主冯从义,韩冈却是回头又看了看,那个年轻后生正跟着王舜臣一起去了城衙,虽然喜欢跟衙门打交道的人不多,但被王舜臣盯着,又是得人相助,他不敢也不能跑。

‘应该不是。’韩冈暗暗摇着头。

冯从义跟他四姨家的表弟同名同姓,但韩冈四姨嫁的是凤翔府的富贵人家,怎么想她的儿子也不可能跑到三阳寨来。而且看这位冯从义的打扮,却是有点穷酸相,衣服都是旧的,而且补过,自然不会是他的表弟。

王韶、高遵裕已经在前面走了,韩冈快走了几步,紧跟上前,就听着两人说着方才见沈起时的事。

“沈兴宗还真是可笑,天子让他体量秦州荒田,他却到甘谷城走一圈就算把事做完了,古渭、渭源都不去,李若愚上次来也没有他这般懒怠。”

“我看沈起的意思好像是要把甘谷的三四千顷田算进来。渭源、古渭的几千顷他不看,但把甘谷之内的四千顷一加进来,子纯你说的秦州万顷荒田也不能算错了。”

虽然王韶说秦州荒田的范围是从渭源一直到秦州州城所在的成纪县,这三百里河谷中有宜垦荒地万顷,其中膏腴之地有千顷。但荒地主要是集中在渭源和古渭两处,渭水自伏羌城以下,由于地理位置比较安全,汉人们多来屯垦,田地被荒废并不多。

而沈起却只到了伏羌城,便往甘谷去。渭水河谷中的荒田他不看,却盯着甘谷之内的田地。沈起的盘算,王韶看得很清楚:“他是想两不得罪,打算拿甘谷内的田地糊弄过去。”

韩冈在后面听的没头没脑,但他拿着王韶、高遵裕的对话想了一下,也稍稍明白了沈起的打算。

李师中、窦舜卿说王韶所奏非实,渭水两岸并没有万顷荒田。按沈起的意思,他大概会说,李、窦二位说得不错,他沿着渭水走了一段,的确没看到一亩荒田。但王韶说军粮可以自行解决一部分,这话也不差,甘谷里就有几千顷地,足以支撑河湟开边的行动。

“真是打得如意算盘,也不看看枢密院肯不肯让他两边迎风站。”王韶对沈起这种明目张胆和稀泥的做法很不满,也想看着他被枢密院的文彦博怎么骂回来。

“这事就不提了,天子之才乃有天授,圣聪岂会为奸人所蒙?不管李师中、窦舜卿有何奸谋,也不管沈起打算如何推诿,官家总能看得一清二楚,查个水落石出。”

高遵裕不想提什么荒田的事。以他对天子心思的了解,即便王韶真的被降罪,也不可能被调离秦州——前面七部攻托硕一役,已经证明了王韶行事的卓有成效——只会被降职而已,而那时,领导河湟拓边的可就是他高遵裕了,王韶就只是个助手。

这样的结果对高遵裕最为有利,他虽然不能为此推波助澜,但也是乐见其成。他现在只担心一件事:“只是沈兴宗今天刚从甘谷城逃回来,却是说了半天也没说清楚甘谷城现在的情况到底如何。子纯,你看甘谷城今次不会有事吧?”

“甘谷城怎么可能会有事?”王韶觉得高遵裕的担心很无稽,“刘昌祚在甘谷城内威信未立,可能不敢出城作战。但甘谷城的城防,以西贼的攻城水准,不用个五六万人轮番上阵,根本不可能打得下来。西贼今次也不可能蠢得去攻城,只会用主力牵制住甘谷城里的刘昌祚,再派小队人马杀入谷中放火抢粮。”

高遵裕点了点头,王韶说得的确在理,他回头又问韩冈:“玉昆,你觉得呢?”

韩冈即便心中有异议,也不可能说出来。何况王韶的话是他凭着在秦州多年经验的推断,当然不会有什么错失。故而韩冈点头,“机宜说得正是韩冈想说的。”

回到驻地,王韶和高遵裕命人上了饭,吃完后都各自回房休息。而很快,王舜臣也回来了。

“都解决了?”韩冈问着他。

“还有什么好说的。现在寨中都乱做了一片,傅寨主正在火头上,那个小贼撞上来,他当然不会轻饶。”王舜臣一屁股坐下,桌上的饭菜还是韩冈帮他留的。王舜臣扒了几口,又道:“不过也不会真的杀他,毕竟罪不至死,听傅寨主的意思,是打上几十杖,刺配流放了事。”

“傅勍倒是心仁。”韩冈笑了一笑。换作是其他寨堡的守臣,直接就是拖出去砍了。把头挂在寨门前悬着,省得寨中再乱下去。而傅勍倒好,就是在气头上也不信手杀人。

王舜臣也赞着傅勍的为人:“傅寨主人不错,本还要拉着俺和高企喝酒,只是想着明天一早就要上路,还要赶着回来回话,才推掉的。”

“傅勍的确人不错,就是贪杯了一点,不然以他的资历品阶,何至于只能担任个寨主。你以后也要注意点,不要贪杯误事。”

韩冈由于担任着勾当公事一职,又是随时能进架阁库翻看资料档案,秦凤路上大大小小近百名文武官员,早给他了解得七七八八。

比如三阳寨的寨主傅勍,他的经历韩冈便是一清二楚。傅勍在军中的资历不比刘昌祚稍差,过去也颇立过一点战功,本官也升做了正九品的三班奉职。

但就是因为他贪杯好喝酒的缘故,坏了事,很吃过几次挂落。尚幸傅勍在秦凤军中人缘不错,不少人帮他说好话,所以官职没有被降,就是没人再敢给他好差遣。本是能担任缘边大军寨的资格,现在沦落得却只能镇守一个五百步的小寨。

“三哥放心,俺碰到要做正事的时候,从来不乱喝酒……对了,三哥你认识那个被偷钱袋的冯从义?怎么一听到他的名字就追问他?”王舜臣突然想了起来,又问着韩冈。

“只是他姓名与我的一个亲戚相同,所以多问了两句。”韩冈信口答了,又问道,“那个冯从义是哪里人氏,来三阳寨做什么营生?”

“他说他是凤翔人氏,到三阳寨是跟着家里的亲戚来做买卖的。”

“凤翔?!”韩冈一惊,一下站起来,急问着:“人呢,他现在在哪里?”

“不知道。”王舜臣摇了摇头,对韩冈的惊讶有些茫然不知措,“应该还在寨中吧。现在天色晚了,也不可能出寨去……三哥,怎么了,他是什么人?”

“我有个没见过面的表弟,就是叫做冯从义,是我四姨的所生。”韩冈对王舜臣也不隐瞒,“王兄弟你知道的,我外公家就是在凤翔府,李二表哥也是凤翔府过来。那位冯表弟同样在凤翔府。既然今天的这个冯从义是凤翔府人,说不定真的是我的表弟。”

王舜臣一听之下便跳了起来,急着道:“我去找他。”

韩冈看了看外面,天色都已经全黑了。他想了一想,摇了摇头,笑道:“算了,就算今次错过,日后也不是见不到他的。何况他也不一定真是我的表弟,若是误会了反就是个笑话了。今天天色已晚,还是早点休息,明天还要赶路呢。”

一夜过去,三阳寨内乱势依旧。傅勍没有杀人立威,下手不狠,当然震慑不了寨中宵小。寨中十字主街上时不时因为碰着撞着而引起一番争吵,这让高遵裕和王韶对赶过来送行的傅勍没有什么好脸色。

韩冈为傅勍感到可惜,‘送上门的好机会不去把握,本人又乏决断,也难怪始终升不上去,日后再被降职,也怨不得人了。’

韩冈跟着王韶、高遵裕一起上路,也不去想着他的那位可能擦身而过的表弟。不一日,当他们赶到古渭寨,一个噩耗正等着他们:

“木征、董裕已经尽起大军,意欲为前日托硕部一事报仇雪恨。河州、青渭各部齐齐响应,已经超过了五万人马!”

