\section{第46章 正言意堂堂(中)}

“看着天子模样,怕是就在等着韩玉昆的好消息。铁船啊,试问木舟如何能抵挡?当能横行水上!”

案上的御酒清澈如水,将天上的一轮圆月和冯京得意的笑脸,一齐映在杯中。这是难得的一箭双雕的机会。御酒绵香,后劲十足,冯京此时正醉意上涌。

韩冈初来乍到,在军器监中孤立无助。看到铁船彩灯,就算想放把火说成是意外,也找不到人去听命行事——已经坏了一次,上元节前的两天,不知多少人日夜守着。想到韩冈只能在旁边干着急,看着彩灯被拖到宣德门,冯京便忍不住心中的快意。

上首的韩绛低头看着酒杯:“韩冈素来稳重,不意今次行事如此轻佻。真不知是谁教出来的。”

韩绛似是意有所指,冯京却不会承认,让他去猜好了:“韩玉昆要光大关学门墙,传播格物之说。将宝全都压在了铁船上,虽然的确急躁了些。但年轻人,心急也是难免的。”

韩绛哼了一声,不再言语。

王珪则笑道,“心急也无妨,只要能见功就好。”

冯京哈哈笑道:“以韩冈的品性,向来是有的放矢,想必已经胸有成竹了,倒也不必为他担心。”

吕惠卿听着,暗自一叹,都是明眼人,都在怀疑甚至确定是冯京做了手脚。其实这也是因为冯京今夜为了钉死韩冈的罪名而说的那些话,让他无法隐瞒自己的动作。

冯京是有恃无恐,不管怎么说,他都是没有罪过的。难道还能为军器监的灯山立案不成?

韩冈除非能尽快拿出铁船,否则身上的污名已经洗不掉了,即便知道冯京下的手又如何?而他吕惠卿即便想自证清白,也没有办法,谁让他是前任的判军器监,任谁都会怀疑其中有他一份功劳。

铁船造不出来,至少几年内绝不可能。不论韩冈是承认还是否定,都会坏了名声,失去天子的信重。没了这两样,要将他赶出京城,再容易不过。

‘韩冈毕竟是太心急了。’

正如韩绛方才所说,韩冈还没有造出铁船,就已经为了宣扬格物之说,先行写下《浮力追源》,在天下传得沸沸扬扬。不论谁看了那本书,都会觉得韩冈去军器监就是为了打造铁船。

但这个做法其实是个轻佻之举——更是自取其辱。只要轻轻在后一推,将此事给定下来。一旦韩冈不能尽快造出铁船,看着他不顺眼的士林中人,可不会留丝毫口德。

‘自找的!’

可吕惠卿觉得自己被卷进来却是无妄之灾。

两相两参,吕惠卿排名最后。资历比不过王珪,地位比不过韩、冯,但在中书中,他的发言权还是最大的。不过这一次,他真的是被冯京害苦了。

深深的盯了冯京一眼,这笔帐,吕惠卿他是记下了。

至于韩冈,吕惠卿倒也管不了了,只能送他四个字——自食其果。不论是苦的,还是甜的,都是韩冈他自己种下的。

……………………

今天是上元节,不过韩家仅仅是摆酒置宴,自家人在一次聚着,并没有出去赏灯。韩冈在御街上应过卯,也就直接转回来,不凑那个热闹。

越是热闹的节日,京城中就越乱。尤其是拐卖人口的人贩子,这时候最是猖獗,而且他们最喜欢的就是富贵人家的儿女。身上的饰物还有本人,都能卖上高价。每年都有听说哪家官员的子女被拐走的消息。韩冈就是准备等到正月十八,稍显清静的时候再一起出去观灯。

“人无害虎意,虎有伤人心。退也退了,让也让了。怎么都没想到,吕吉甫竟然还是忍不住跳了出来。”韩冈轻拍桌案,和着乐曲的节拍。住在街对面的天章阁陈侍制,请了一队乐班来家,丝竹之声缭绕于周围的街巷之中。

与韩冈在家中后院中对饮的冯从义轻声问道:“当真是吕参政?”

韩冈沉默了一瞬间。当时看到曾孝宽慌乱的样子,让他也不能确认。不过吕惠卿的嫌疑也的确最大,白彰是谁的人,军器监中哪个不知道?只是韩冈并不在乎究竟是谁主使,已经是赢家了,何必在乎自作聪明的输家是谁?

“不过这手段倒是出人意表,让人叹为观止。”韩冈几天来,一直都为这逼他上烤架的手段拍案叫绝,“灯山坏了一次后,加急赶工了六天才打造出了新彩灯,赶在上元灯会的前两天才看到。拆又不能拆,改又不及改,只剩两天的时间,做什么都来不及了……上元灯会,热闹的是观灯,不是造灯。哪家监司的主官都不会将彩灯放在心上,全都是丢给下面人来负责。这还真是钻了个空子,防不胜防啊。”

冯从义悠然长叹:“可惜就要回关西,看不到吕参政偷鸡不成蚀把米的表情了。”

叹过,又呵呵的笑了起来。天下闻名的俊才,又是执政一级的高官,却是机关算尽也奈何不了他的表兄,冯从义当然想笑。

只可惜冯从义他是顺丰行的大掌柜,不能离开关西太久,过了正月十五就要回去了。不过在此之前,韩冈让他安排在城西仓库的那组人,已经给安顿下来了,物资也准备充足。只要汴口还没开,那一片以布商为主的仓库就足够清静。

韩冈摩挲着酒杯上的纹路,抬头望月:“就等着能载人的飞船出来了,眼下的只能算是玩具。”

“两只鸡果然还是太轻了点。”听了韩冈的说话,冯从义忍不住又笑了起来。笑了几声,又惋惜的说着,“若是飞起来的时间再长一点就好了。”

仅仅是载重加起来不到十斤的实验性热气球,在过年的那几天,已经给造了出来。的确离了地,不过用一根绳子拴牢了,并没有飞高。这个热气球有着极为简单的结构,就是气囊和装着鸡的竹篮。气囊是绸子里面糊了纸,被一张渔网罩着,渔网下面拴着只竹篮。甚至连加热都是在地面上,等热气冷了就落回了地面,漂浮的时间总共也只有一刻钟的时间。

可韩冈已经很满意了:“不要贪心。能飞起来就是成功。”

冯从义点着头附和道:“表哥说的是,别的都是假的,只有飞船飞起来才是。”

“其实名分也很重要。我已经将他们几个都暂时转入了军器监中,只要飞船造出来,就是军器监的功劳,不至于惹人闲话。”

韩冈虽然新上任的判军器监,但要把几个亲信安插进监中也不是什么难事,更是在情理之中。哪位官员上任,身边不带几个得力的人手?而且韩冈还不是以权谋私的抢占重要的职位,或是一些油水丰厚的差事,仅仅是给了个吏员的身份,年后半个月都没有到任,这就更是不会惹起军器监内部的反对,甚至是注意。

在正月的一轮满月的照耀下,韩冈和表弟一起喝着热酒,有一句没一句的聊着天。成功就在眼前,心情也便放松得很。内间,两人的妻妾也在一起聊着天,欢声笑语不时的从帘中传出来。

只是到了二更天的时候,门外突然有了动静,先是一阵嘈杂的马蹄声,而后急促的敲门声从大门外传到了后院之中。连着女眷都惊动了,纷纷从内间出来。王旖惊疑不定的问着韩冈,“究竟出了何事?”

开门请了来人进来,却是韩冈的老熟人蓝元震。

尖着嗓子,皇城司同提举兼御药院都知的蓝元震传达赵顼的口谕:“圣上口谕,着起居舍人、判军器监兼直龙图阁韩冈,即刻入宫觐见。”

韩冈领旨行礼后,早已有了经验的韩家家人,便给蓝元震和随行之人送上了应有的谢礼。

蓝元震谢了韩冈的礼,上前半步,小声道:“看到军器监今年摆出来的灯船,官家欣喜不已。冯相公和王、吕二参政,都奏禀官家,召舍人入宫相问。”

“原来如此,多谢都知。”韩冈会意点头,脸上没有半分异样,“还请都知少待,且等韩冈更衣。”

向蓝元震告了罪后,韩冈走进房中。

冯从义脸色惶急:“怎么来的这么快?表哥,要是在君前坐实了要造铁船,就算之后造出了飞船,也会有麻烦的。”

“放心,我不会就此应承。而铁船也不是完全是幌子。凡事若是没有后手,当轴诸公最差也不过是降职远调而已。而我,恐怕早就死在秦州的山中成了道边枯骨。论到做事,我可比冯相公和王、吕二参政用心得多。”

“当真?!”

冯从义还是很慌。从韩冈的话中,他已经知道对手是谁了。虽然韩冈信心十足,但对手毕竟是一相两参,而韩绛的态度也暧昧不明。在政事堂中,韩冈已是举目皆敌!

“纵为宰执又如何?他们的眼界实在太小了,争来争去又有何意义?”换了朱色官袍,佩了银鱼袋,韩冈举步舒缓的走出来:“以为我韩冈仅仅是为了功名二字,才来军器监的吗?”他冷笑一声,“李义山【李商隐】的两句诗,送给朝堂诸公却是正合适!”

“什么?”

“不知腐鼠成滋味,猜意鹓雏竟未休!”

