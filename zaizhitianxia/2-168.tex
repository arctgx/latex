\section{第30章 肘腋萧墙暮色凉(三)}

【第二更】

只要稍通兵事,就知道银州守军的出兵速度有多么惊人。

这不可能是他们事先预计到宋军将会在正月初的时候出兵罗兀,否则罗兀城也不会这么容易的就攻打下来——从绥德一举突进六十里,在大宋这边,都被人看成是疯话,传了许久,除了一些关系人和耳目灵通的官员外,也没多少人真个相信。

从银州西贼的反应来看,他们自看到罗兀城上的烽火、收到罗兀被袭的消息,到开始出兵,最多也只用了半天时间来进行调集兵马、整备装具的工作。这个出兵的速度,快得让每一个宋军将领惊叹,心道难怪罗兀城中没有驻屯多少兵力,也没有扩建——有银州的支撑本也就足够了。

如果今次不是出其不意的攻下罗兀,只要守军能守住城池半天到一天的时间,那从银州赶来的援军,就能轻易的把来袭的宋军击溃在罗兀城下。

种建中暗自庆幸,幸好为了夺下罗兀城,事先没有少做手脚,堆满仓库的酒水,可是种家的回易商队不断奉上的礼物。

种谔此时已经在城中主帐内发下令箭,“高永能!你率本部三千轻兵前去马户川,务必将都罗马尾先行截停,本帅领中军,随后便至!”

种谔的副将高声应诺,双手接过令箭。很快,高字将旗就在三千步骑的簇拥下,向北疾驰而去。

高永能先去堵截来援敌军,罗兀城这边,随行的民伕已经有两千多人先期抵达,被分散到预定的地方,围绕着罗兀城,开始挖掘土地,修筑营地——通过精准描绘的地形沙盘,种谔早已确定如何扩建罗兀城。包括敌军随时有可能突破前沿防线的情况都已经预计。现今首要的目标是依照扩建城池的规划,加紧建好初步的城防,使之可以成为暂时屯军并防守的营寨。

军势争分夺秒,民伕们不需要催逼,在被冻结的土地上,高喊着号子,用力挥动着手中的铁镐,加紧修筑起防御工事。而士兵们有一半与民伕一起开工,剩下的则并没有参与到修建营地的工作上,而是在蓄养体力,等待种谔的号令传来,随时前去支援北去的高永能。

种谔在主帐中飞快地踱着步子,原罗兀守将的首级也没兴趣看上一眼,用脚把大帐的直径丈量了一遍又一遍。一边等待着前方传回来的消息,一边催促着加快营寨修造的速度。

半日后,营寨外围的防御工事已经初见其功,种谔留下了一部兵马守卫,并继续加强防线,而自己就点集了兵马,准备北去支援高永能。但这时候,一队骑兵却高居着旗帜,从北方鼓噪而来。

并不是西夏的士兵,而是高永能带去的人。他们在营地前高呼着万胜,把胜利的喜报通报给每一个人。

高永能竟然已经在马户川击败了正欲过河的西夏援军!

据称来援敌军多达万余,高永能以三千破万骑,斩首百余——其中斩首当是实数,而万名援军可能是夸大其词,据种谔所知,银州城中的常备兵力也不过万人。都罗马尾不可能全数带出,虽然他还可以征发部族兵力,但以都罗马尾出兵的速度,当不会有时间让他去发动周围的部落。

在种谔的估算中,与高永能交战的敌军,大概能有六七千人。而能用三千步骑,击败两倍的纯骑兵部队,并且还能斩获百多首级,这个胜果的价值,其实跟高永能回禀的捷报没有什么区别。

“不过如此!不过如此!”

种谔在营中哈哈大笑。这段时间以来,他身上承担着的压力实在太大了,不但天子、韩绛和朝堂都在看着他的行动,下面的士兵,周围的同僚,也都在盯着他。相信他的人给他压力,而否定他的人,也给了他压力,如果罗兀城不克,他种谔再想翻身,可就不知要等到猴年马月。

种谔此前在韩绛面前一直都是胸有成竹的态度,但心底里始终有着一分不安,这也是人之常情。幸好今次一战功成,只要接下来能守住罗兀,那他种五在军中的地位,将不可动摇。

而到了入夜时分,斥候传回了最新的消息。都罗马尾刚刚在立赏坪扎下阵脚,结寨自守。

“还想等机会?……找三件女人衣服给都罗马尾送过去,他若敢战,明天就在立赏坪决战。若不敢,就干干脆脆的穿着女人衣服回银州去。”种谔不给敌军主将留下丝毫颜面,他现在正希望西夏国都枢密在大怒之下,会同意出阵决战。

不过种谔也不会太过疏忽大意,他叫来负责外围侦查的部将,“吕真,你率本部人马仔细盯着都罗马尾,有何异动就立刻回报,不得有误!”

胡乱的假寐了一阵,当次日四更天的时候,种谔等不到都罗马尾的反应,正准备再派人去试探。吕真派回来的斥候,又传达了更为让人吃惊的捷报——方才在山口处的立赏坪,刮起了一阵狂风,吕真派出去的斥候只是随着风叫了几声,党项人就大喊着“汉兵来了!”,而后便溃不成军的逃窜回银州去了。

虽然并不认为都罗马尾有击败自己的能力,但看到让自己战战兢兢、严防死守、如临大敌的对手,竟然因为一场山口处常见的狂风,还有几声凑趣的叫喊,就全军溃散。除了能联想到风声鹤唳的前秦苻坚,种谔对西夏军战斗力的判断,又打了一个更大的折扣。

“完全是惊弓之鸟嘛……”种朴也拿着酒杯,对堂弟笑道:“西贼已经完了,连镇守银州要郡的主帅都是这副德行,其他地区的守臣也好不到哪里去。光复兴灵,灭亡西夏,恐怕也就在数年中了!”

……………………

正月初十的时候,韩冈终于抵达了延州。

从京城到这座边地重镇,韩冈一行走了有半个月。当除夕的鞭炮声响起来的时候,他正在河中府的驿馆之中。密集的鞭炮,让那一日韩冈想起,他已经连续两年没有在家过年了。如果算上前身在外求学的时间,那就还要翻一倍,有四年之多。

不过先托了王韶,而后又派了李小六带了消息回去,家中的父母应该不至于太担心自己。就是不知道素心和云娘两人,听说了周南的事后,会有什么反应。韩冈只希望他让李小六给两女带去的礼物,能让她们不至于吃醋得太厉害。

这份担心,一直持续到他抵达延州城。韩冈有时在想,女人多了的确麻烦。如果能像当世的士大夫一样,把姬妾只当作娱乐的工具,就没那么多要操心的事了。可若是他真的这么做了,也不会让三女为他而倾心。

在城外,望着延州河山,韩冈却是有种沧海桑田的感触。

他不是第一次来到延安,虽然时间跨度上有些问题,建筑没有一点千年后的影子,幸而山峦河川的位置却没有大的改变。宝塔山、延河等名胜,都能找到此时相对应的地方。

在延州的城门处,韩冈让钱明亮向守城的士兵亮出了自己的身份。

见着守卫城门的军官听到自己的名字后,一下变得恭谨起来的姿态,韩冈半惊半喜的发现,自己的名声竟然已经在离秦州有千里之遥的延州传开了。

当韩冈一行车马穿过城门,驶入城中。一个士兵问着守门官,“那官人究竟是谁啊?哪儿来的那么大架子?”

“你耳朵怎么长的,难道这些天来都没听说韩相公要请孙真人的弟子来延州吗?那还是韩相公连上两本,亲自向官家求来的!”

“啊!就是韩……”

“闭嘴,那名字也是你能乱叫的!?韩相公都不一定会直接叫他的名讳。”

‘怎么可能!’两人的对话被风送了过来,韩冈自嘲的笑了一笑,下层的百姓会把谣言当一回事,可对于韩绛这等位极人臣的宰相来说,自己就仅仅是个选人罢了,只不过稍有能耐而已。

离着上元还有数日,正月未过,这年节也不算过去。可延州城中的鞭炮声却是稀稀落落,比韩冈经过的几个县城还要冷清。当他走进延州城中的时候,正看到一队队的民伕,被一些骑兵们押着,从北门陆续出城。

已经开始了……

韩冈早已经听说了种谔在罗兀城胜利的消息,而且他就在路上,看到了露布飞捷的急脚递。骑着快马的信使,在马身后张着长长的布幔,上面写满了今次罗兀城的捷报。从一个城镇,到另一个城镇,把胜利的消息如同风吹起的蒲公英,不断的传播出去,一直传到东京城中。

夺占罗兀的顺利,早在预料之中,接下来要面对的局面,才是决定最后胜利的关键。韩冈依然保持着早前的看法,始终不看好横山攻略的最终结果。

先去了驿馆,将周南等人安顿下来。韩冈便独自前往帅府,向守门的小吏递上了名帖。

小吏好像也是听过韩冈的名字,不敢怠慢,并没有摆出宰相门前七品官的态度,而是忙进去通报。韩冈在门厅候着,一人大步走进来,竟是与他有过一面之缘的熟人——王文谅。

