\section{第20章 敌如潮来意尤坚(下)}

战势如同跷跷板,一方气势下落,另一方气势便会相应上升。王君万正在回撤途中,鼓号声便从西夏阵营中响起。两支千人左右的铁鹞子从中军分了出来,一左一右,包抄向宋军的侧翼。

张守约瞪着呐喊着冲杀而来的西贼,再看看短时间内,已经无力再次冲阵的骑兵,冷哼一声,直接翻身下马。丢下头盔,听其当啷落地。解开披风,任其随风而去。甘谷城的老将卸下了披膊,甩掉了甲胄,将内袍扎在腰间,露出上半身伤痕交错的如铁肌肤。张守约健壮不输少年的身体半裸在寒风中,却无半点瑟缩。他几步上前,一手排开将旗下猛击战鼓的鼓手,手持一对鼓槌,抡圆双臂,狠狠的敲响了大鼓。

咚咚!咚咚!

鼓声震天,主帅亲手敲响的战鼓震动了全军,士气顿时大振。合着节奏,刀盾手以刀击盾,枪矛手用枪尾捣着地面。

万胜!

万胜!

这是两千将士不屈的高呼!这是汉家儿郎对胜利的渴望!

张守约双臂一荡,鼓槌节奏转急,进军鼓点响起。他麾下一千五百多步兵,便应着鼓点,结阵上前。一排排刀枪直指前方,抵住铁鹞子的冲击,后阵的弩弓随着鼓点一波一波的撒出箭雨,让西贼难以寸进。

大宋步兵虽然单人战力远不如契丹、党项这些蛮夷。可一旦摆下箭阵,便是万军辟易,纵然是契丹铁骑也要绕道闪避。不击堂堂之阵,就算是党项人也清楚这一点,两支侧击的骑兵停止前进,缓缓退到宋军的射程范围之外,来回游窜,不敢贸然前冲。

箭落如雨,不住的散落在两军阵中。西夏军无法突破宋军的防线,但宋军也无法击破西夏军的阻截,战事一时胶着起来。

……………………

自出伏羌城之后,辎重车队顺着官道一路北行。两侧的山势渐渐高起,其实已算是六盘山的余脉。

山谷间的甘谷水上游出自于温泉。温泉在这个时代被称之为汤,有温泉的山被称为汤山,因而甘谷又名为汤谷。河道两侧,良田处处。甘谷谷地的万顷良田都被这条河水滋润着。六十里长的谷地出产丰茂,举目望去,满眼尽是一方方田地收割后焚烧秸秆的深黑痕迹,不负甘谷之名。

只是甘谷水毕竟是黄土高原上的河流,如今入冬后雨水稀少,水流清澈无比。但到了夏日雨季,据说一场暴雨过后,浑浊汹涌的滔滔洪水能将整个谷地都淹起,水退之后,到处是半人多高的巨石,连谷底都能被削下一层去。甘谷水边的官道就是在河道西岸上,有许多路段,堤岸和河面的差距甚至高达近十丈,由此可见洪水冲刷的威力。

越过一处缓坡,官道低了下去,只高出河面两丈多。看着河水潺潺,清浅如同山涧溪流,韩冈心中一动,唤停了车队的行进,和王舜臣从官道下到河滩边。他蹲下身去,伸手试了一试。当即倒抽一口凉气,

“好冰!”

初冬的河水尚未上冻,但温度已经跟冰块没有两样。探手入水,一道冰寒就直透囟门,韩冈顿时觉得连半边身子都冻住了。就着冰寒的河水,他洗了洗脸,却怕弄坏肚子没敢喝下去。

韩冈身边,王舜臣满不在乎的跪在地上,用手掬着河水咕嘟咕嘟地喝了几大口,乱蓬蓬的胡须都淅淅沥沥向下滴着水。抬起袖子胡乱在脸上擦了一擦,动作豪放不羁。喝完水,他长舒一口气,突然仰天骂道:“日他娘的,一肚子的鸟气到现在才消。”

韩冈拍了拍王舜臣的肩膀,他知道王舜臣因何事不痛快,能为自己生气,这朋友交的就没问题。“何必呢……举荐一事要你情我愿才行,既然我不入王机宜的眼界,那也就罢了。”

王舜臣啧了一下嘴,心中还是不痛快,在他看来王家父子实在有些不靠谱:“王衙内说得好好的,王机宜也到了城门口。扯了两句就放着三哥你出城,连好话都不说。这不是耍人吗?没见过这等鸟事!”

“王处道是王处道,王机宜是王机宜,不能混为一谈。一起喝了一夜的酒,处道的为人,王兄弟你也该有点数。他当是真心诚意想举荐于我,只是不得王机宜的认同罢了,不然王机宜何须把处道先遣走?”

“王机宜也忒没眼光了……”王舜臣神色悻悻然,踩着松塌的土石几下跳上河岸。他们这些军汉,对于出生入死的情谊最为看重。一起上过阵那就是过命的交情。在裴峡谷,他与韩冈联手退敌。韩冈的为人、气度还有手段,他敬佩有加。而且还有十九哥种建中这一层关系在,王舜臣很是盼着韩冈能得官,日后即便不提携自己,有个相熟的官人,也是件光彩的事。

韩冈跟在后面,借着王舜臣的力也上了堤岸,“王机宜有没有眼光那是他的事,我只要他能帮着解决掉陈举便心满意足了,否则我何苦把缴获的首级和兵器丢给王处道?”他说得很坦白,朋友相处,重在推心置腹。就算不能推心置腹,也要作出与朋友无话不谈的样子,“只要没了陈举,我在秦州便能安安稳稳的读书。凭我韩冈之才,日后得官也不需要他来举荐。”

“说的也是!凭三哥你的才气,日后是要考进士的,哪里要靠他来举荐……”

王舜臣点头说着,韩冈的本事他是见着的,可比他过去见过的一些文官强得多。但韩冈这时不知为何突然来回张望,神色也变得严肃起来。

“韩三哥,怎么了?”

“你不觉得有些不对劲吗?……好像太安静了点!”

韩冈心中有些收紧,方才在路上走着还不觉得,但现在一停下来,就发现现在传入耳中的,除了哗哗作响的河水,就只剩有一声没一声的寒号鸟鸣。

“嗯。”王舜臣也看出谷中不对劲的地方了,他自幼便在军中打混,对危险的直觉也是异乎寻常,“谷中的蕃部怎么都不见了!”

甘谷本是蕃部筚篥族的地盘,但因为躲避战火,筚篥族十几年前举族南迁,移去秦岭之中居住。留下的谷地被更加彪悍的心波三族给占据。心波三族名义上是三家,其实就是靠着联姻聚合起来的一个部族。他们一直都是在宋夏两国间游走,即有亲附宋军与西贼厮杀的时候,也有跟着党项人出谷南侵,在汉儿们身上分上一杯羹的时候。

尽管心波三族因为反复不定在关西结怨甚多,但他们一旦势弱,也是能放下身段装起孙子来,让大宋难以下定剿杀的决心。不过心波三族这种墙头草的生活,到去年甘谷城落成后,便宣告结束。连接西夏的通道被封死,他们只能做起大宋的顺民。

秦州的蕃部已不是逐水草而居的游牧民族,他们虽然很少有修造房屋的习惯,但一样开垦田地进行耕作。聚居在甘谷谷地中的心波三族,据说拥有四千帐幕,按照汉家的计算方法,就是有四千户人家,是秦州数百蕃部中排得上号的大族,轻而易举就能组织起一支大军。

总计四千帐落的蕃人,被甘谷城和伏羌城南北包夹,不得不老老实实在谷中垦荒种植。但韩冈他们一路走来,却都看不到吐蕃人的帐幕,他们究竟去了哪里?韩冈和王舜臣对视一眼,去哪里不重要,伏羌城里去避难的更多,关键是他们接下来想做什么。

向北方眺望而去,山巅之上,从远到近一道道笔直而上的浓烟散入云霄,甘谷城危急的消息终究还是遮瞒不住,沿着在甘谷谷地中的烽火信道直传而来。

“如果甘谷城破……不知那些鸟贼会选哪一边?”王舜臣抬眼盯着散布在两侧山巅的道道狼烟。他并不认为心波三族敢去围攻甘谷城,这些吐蕃部族若是有这种胆子,早就被灭了。他们就只敢趁西贼来袭时浑水摸鱼沾点便宜,绝没胆子正面与西军对抗。现在可能是躲进甘谷两侧的支谷深处,等待甘谷那边分出个结果。

“这还用说吗?”韩冈冷笑。非我族类,其心必异。这些蕃部夷人,如果不能将其教化,化夷为汉,他们对大宋在西北边陲的统治就是一颗颗毒瘤。每逢党项入侵,跟着其助纣为虐的蕃部从来都不少。如果是强硬一点的边将驻守,还能拿几家作伐,杀鸡儆猴一番。但若是碰到了大范老子【范雍】一般的软弱文官,就会任着蕃人在关西嚣张跋扈。

韩冈突然跳上身边的骡车,高高的站在车斗上,向着手下的民伕高声喊话:“最后一程了,大伙儿再加把劲,午时若赶到安远寨,入夜前就能躺在甘谷城的床铺上!”

三十多张嘴齐齐答应,咕噜噜的车轮节奏重新响起,比之前快了许多。辛苦了四天,中途还打了一仗,民伕们都在盼着结束的时候。

韩冈又从车上跳下,走回王舜臣的身边,笑道:“不管怎么说,现在就只能看张老都监的了。”

ps:前方战事正烈,后方隐忧丛生,甘谷城已经成立一个危险的漩涡,韩冈正带着队伍向漩涡中走去。不过富贵险中求,不冒点风险,如何能成功。

今天第一更,照规矩求红票,收藏。

