\section{第14章 霜蹄追风尝随骠(九)}

“折家欺人太甚!”

“韩冈竟然视我大辽软弱可欺!”

“枢密,要好好教训一下南朝!”

“把屈野川的宋狗都赶走,那是大辽的地!”

“要打破府州!”

“杀到太原去!”

萧十三冷眼看着下面叫嚣着要报复回去的将领们,一个个面红耳赤,咬牙切齿。

只是萧十三的心中,却没有那没多的愤慨——其实这群人中,心中的怒气恐怕还没有表现出来的十分之一——只有对利弊得失的计算。

当宋人将手伸到东胜州南面预定要占据的丰州旧地,萧十三就知道冲突是迟早的事,但这一次冲突的起因未免太过无稽了。

探查敌情的斥候离开时的随手一箭,竟无巧不巧的伤了名宋将一只耳朵,然后那名宋将便悍然领兵越境烧了两间巡铺。之后派出去探查宋人修筑进度的斥候,都被神臂弓乱弓攒射,三人受伤,其中一人伤重,多半是活不了了。而武清军那边也报复回去,据说斩伤了几十名宋兵,排除吹嘘的成分,萧十三估计是打柴落单的宋军士兵.运气不好被撞上了,不知道死了几个。

对于宋人这般强硬的态度,让辽人很不习惯,这也是为什么一众将领们都叫嚣着对屈野川的宋人动手。

可宋辽两国之间打了几十年的仗,伤亡上百万人。就算澶渊之盟之后,边境上也经常有你烧我的巡铺,你杀我的子民,最后一直闹到两国天子那里的情况。可为了半只耳朵,闹出那么大的动静,日后回想起来,就不知道是该哭还是该笑了。

不过辽夏两国之间,并没有订立精确的边界,不过习惯性的国界还是有的。但西夏灭亡,留下的国土被宋人乘隙占据。只要没有正式的国界文书,大辽这边完全可以不承认。真正决定国界的,是双方投入军力的多寡和最后战事的结果。

“你们当真要为半只耳朵开战?”还是有人比较冷静。

“不是为那半只耳朵!是为大辽的土地。再迟了,宋人可就将整条屈野川占下来了。”

“屈野川边的寨堡才开始修,要修好还早得很。河东、河北的那一圈寨堡,不都是几十年的修造,才有了今天的规模?”

“就是千步城,让宋人去修,一两个月就能修好了。还几十年……有个十年时间,东胜州边境的山上就全都是宋人的寨子了。”

“就是要趁现在还没有修好,大虫小的时候,一脚就能踢死,等到长成之后,又有谁敢去惹?”

“也不是当真要打,只要陈兵武清军,逼宋人从屈野川撤军。”

“韩冈要是那么容易吓倒,哪里能做到河东经略使的位置上?”

“而且现在主力都在黑山河间地,要调兵回来,至少要到半个月之后。”

萧十三给下面的人吵得头疼。

从他的角度来说,跟宋人大打出手不会带来什么好处,息事宁人才是最好的结果。最近朝廷用兵的重心是在兴灵和黑山河间地,还有蠢蠢欲动的北阻卜。加上为了明年鸭子河边会集女真诸部的春捺钵,萧十三知道,自己却不可能在尚父那边得到多大的支持。

南下的大辽铁骑已经占据了兴庆府、灵州、顺州、定州、怀州,等所有兴灵之地的州府。原本生活在此处的党项部族有好几家选择了西撤。他们如此识趣的确让人欣喜,但真实的情况却是这些蕃部投靠了宋人,帮宋人卡住了一条交通要道,让灵州随时都受到威胁。但剩下的党项部族不肯迁走,为了他们地盘,眼下已经有了剧烈的冲突。

将会被迁移到兴灵的,是五院、六院、国舅和奚族中的一部分部族成员,他们都会在其中被分配到土地。这是尚父耶律乙辛带给他们的好处,至于能不能稳定占领下来,那就得看他们的本事了。就算站不住阵脚,也不是耶律乙辛的错,而是他们无能而已。就算稳定下来,也是要付出巨大的代价,对耶律乙辛的好处也是一样的。

党项人不会轻易的臣服,就跟所有大辽境内的部族一样,都要强力打压,说简单点,就是屠杀。清理掉兴灵之地剩余的党项人。

而现在黑山下的河间地,现在也正在清理残存的党项部族。那里是耶律乙辛预定的斡鲁朵辖区,要迁移一大批契丹部族过来生活,不需要多余的丁口。相对于兴灵之地的党项人,黑山河间地的党项部族,在预定的计划中,一个也不会留下来。

当然,杀光还是驱除,就得看实际领有此事的萧十三的心情。之前萧十三怕将这些部族仅是简单的赶走,日后他们会卷土重来——只要有这种可能,就会被尚父视为办事不利,毕竟那将会成为尚父斡鲁朵的属地,有一点风险都会影响斡鲁朵的安全——所以采用的是斩尽杀绝为主的做法。

不过萧十三打算改一改方法了,“传令给耶律成吉,让他把那群黑山党项往南面赶,赶到宋人的的地界去。让他们先跟宋人斗一场,不指望他们能给宋人带去多大损伤,消耗粮草、马力,耽搁一下修筑的时间也就够了。诸位回去整顿兵马,做好准备,随时待我号令。”

诸将面面相觑,但在萧十三的威势下却不敢反对。若是萧十三不肯出战,诸将肯定要争上一争,但现在既然已经答应要作战,只是用兵方略上的问题,那就没什么好争的了。就是不能如愿,也是北院枢密使萧十三本人的问题。

众将离开,萧十三的幕僚走到他身边,低声道:“枢密,驱虎吞狼……驱狼攻虎诚然是妙计。要是两家不斗怎么办……那群黑山党项他们肯定是会投降宋人。”

“投降宋人最好,可以乘势追杀过去嘛,看看宋人保不保他们。宋人守着营寨,往墙上撞只会头破血流,将他们拉出来打,才有机会……逃跑时牲畜是带不走的,最多只能是一些马匹而已。宋人要养他们,不知要消耗多少钱粮。”

萧十三当然不愿意跟韩冈死拼。一旦他失败,他在耶律乙辛心目中的地位肯定会一落千丈,被人取代,甚至踩上一脚都不是不可能。

当然,萧十三相信韩冈也不会愿意冒着毁掉前途的风险来挑起战争。不过韩冈神仙弟子的名头就是在辽国也是如雷贯耳,犯了再大的错,也有挽回的机会,但换做自家可就没那么容易了。

能顺便借用一下黑山党项的力量,也算是多一分把握。韩冈若是不保护归附的党项部族,有的他苦头吃。若是保护他们,为了抵挡追击下去的大辽骑兵,不是将之收拢到城寨周边,就是将大军从寨防中拉出来,有党项人拖后腿,赢下来并不难。若是那群党项人在逃亡的过程中伤亡殆尽,也是一桩美事。

不管是什么结果,对萧十三来说,只有好和更好的区别。只要保证与宋人的交锋不要是惨败的结果,就没有什么好担心的。

……………………

半只耳引发的冲突,让韩冈也觉得啼笑皆非。

但不管理由再无稽,也必须坚持下去。既然已经选择了认同折克仁的报复行动,那么就需要坚持到底。

韩冈很清楚有些事不可能是心想事成,人心本来就难以判断。但真正一怒拔剑的情况,是很少出现在身居高位者身上。外人看来冲动的行为,实际上依然是取决于自身的利益。

蔺相如在渑池之会上要挟秦王要血溅五步,也不见秦王怒而杀人。就是以辽人的好斗,做到枢密使一级之后,也当是有着足够的沉稳,乃至不愿意冒一星半点失去权力的风险。

千金之子,坐不垂堂。

这是世间的通则,身家越丰、权位越高,尤其是那种权位、身家都是自己一点一滴积累起来的人物,就越是不愿意冒险。或许有例外,但萧十三绝对不是。

与其隔着雁门关对峙了半年,韩冈很确定萧十三不是冲动的性格,否则绝不会在一次小规模的冲突之后,就偃旗息鼓。而这一次冲突,韩冈一直尽量将之局限于简单的报复上,以防规模扩大成战争。

不过要做的准备,还是得以战争为标准,要是这场冲突当真被辽人扩大化,韩冈也能针锋相对,将冲突随着升级,就是变成战争,也得咬牙撑下去。

“粮草的情况怎么样了。”韩冈问道。

“麟州这边有粮五万三千余石,草两万束。太原府和晋宁军加起来也有三十三万余石,二十万束草。府州的存粮超过十万石。足够两场大战的使用。”黄裳对几个关键性的数据如数家珍。

“重要的还是战马和士卒。”折可适紧跟着说道,“经过了半年多的战事。人心厌战,战马也支撑不住。”

韩冈点头,这些他都是知道的。不过粮草等后勤物资才是关键,精神方面总有办法激励起士气。战马问题虽然难以解决,但这个世上,本来就没有万全的说法。

正想着事,亲兵引着一名信使进来,“龙图,李都知还有半日就到了!”

