\section{第31章 九重自是进退地(九)}

“三千人……”

赵顼点着头,微微眯起来的眼睛似乎对韩冈的回答还算满意。

赵顼还记得百年前开凿襄汉漕渠时,是调动了京西数州,总计十万民夫,且到最后没成事。而韩冈就只要三千人,说是要民夫,其实调动厢军就可以了。比起开挖河道的成本来,铺设轨道的确省了不知多少。

“以臣之愚见,兴修工役,能不扰民就尽量不扰民。”

反对兴修工役的大臣,最强有力的理由,就是工程扰民,韩冈这一次将主意打到襄汉漕渠上,并不是没有人反对,也是有人指责韩冈贪功妄兴。不过出来的都是些小鱼小虾,几位宰执都对此保持沉默,这让韩冈都觉得有几分不习惯了。

“只要翻山轨道修成,穿过方城山的漕渠,就可以用个几年十几年的时间慢慢开凿,将需要耗用的人力和财力,分摊到十年之中,平均一年所要消耗的数目,也就不算太多了。”

“韩卿的想法的确是正理,的确是能不扰民就尽量不扰民。”赵顼不掩对韩冈行事宗旨的欣赏,“记得两年前,朝廷准备调集大军膺惩交贼的时候,枢密院曾经说,为了给安南大军输送粮秣,至少要调动二十万的民夫,这样才能保证足够的钱粮供给。本来朕都准备动用封桩钱了,没想到到最后,就是广西出了点人而已,耗用的钱粮还不比鄜延或是环庆一次防秋的花费。”

韩冈的行事一向如此,总是能花小钱办大事,这便是赵顼欣赏韩冈的主要原因之一。在成事的前提下,为国家尽量节省开支。什么叫做能臣,这就叫做能臣。

最近的例子便是安南之役,花费比起赵顼和枢密院一开始的预计省了九成还多——如果只算钱粮开支,光是广东、广西两路的出产就撑了下来,连荆湖南路诸州的储备都没有怎么动用。

得了天子的夸赞,韩冈倒没有忘乎所以:“料敌从宽,枢密院当年的计算也不能算错。安南经略招讨司也是用了一年的时间,才看透了交贼的虚实,方才敢于以万人出征。”

赵顼微微一笑,韩冈不想得罪人的心思倒也不难看出来:“那如今韩卿意欲先修轨道,也是看透了轨道的虚实喽?”

“轨道一物自发明后,已经在天下各大矿山、港口试行了两年有余,其形制比臣旧时所创,已经改进了许多,各处皆有效验,如今不过是在方城垭口越过一道十丈髙的缓坡,总长不过六十里,比起矿坑、码头上的道路,并不算多难,甚至可以说是简单。”

现在汴河两岸的港口中,全都在用轨道来沟通码头和仓库,而徐州乃至天下多少矿山也都开始使用轨道来运送矿石,韩冈这个轨道的发明人,会放着这么好的手段不用,想来也不可能。

陆运也好、水运也好,只要用成本较低的运输方式,将南方的货物——尤其是粮纲——运抵京城,只要能看到粮食,赵顼他并不在乎韩冈用的是什么办法。

越过方城山的渠道开挖起来不容易,百年前的两次修河失败就是现成的例子。如果韩冈只说有办法能让船只通航,就算以赵顼对韩冈才能的信任,也免不了要担上一份心。而现在韩冈将要采用的是行之有效成熟可靠的手段,动用的人力又少,不用担心惊扰到百姓,这样一来,赵顼当然是放心不少。

只听当今的天子笑道:“此事朕也听说了,不过两年而已,利国监中的轨道加起来据说都有数百里长了,穿越方城山的五六十里的确不算什么。”

“臣也听说利国监中的石炭、赭石【赤铁矿石古称】,如今都是从矿坑里用轨道运出来,人工只用旧时的十一,而运出来的矿石数量却翻了好几倍。还有几百万石的生铁,也是从高炉边铸锭后,用有轨马车运上船去。光是利国监中的运输量,一年近千万石总是有了。从荆湖运来的纲粮,使用轨道转运,一两百万石当不在话下。”

经过与韩冈的一番对话,赵顼对打通襄汉通道的工程已经充满信心,注意力又转回到京西路的沙盘上,低头看着方城垭口的那一段,“韩卿打算怎么修造轨道,是调用军器监的匠人?”

听到天子相询,韩冈立刻答道:“不仅是军器监的工匠,臣还打算从利国监调来一部修筑轨道的匠人。他们的经验,京中的工匠不一定能比得上了,利国监的匠人也当有所心得,两家互相参考,相互切磋,当是能精益求精。”

韩冈这两年远在广西,得到的消息远比不上身为天子的赵顼详尽。不过军器监的匠人们依然遵循着他的嘱咐,继续对轮轴技术加以改进和研究,这一件事,他是知道的,两年间也是有些成果。

不过要跟后世的轮轴比起来,那还差得太远,甚至还没到实用的时候。倒是轨道技术在利国监有了个小突破,利国监中原本使用的是硬木轨道,在车轮不停碾过的过程中损耗极大,甚至有些地方,每隔几天就要换上一次,为此消耗的成本也不少。不过在一年前,利国监有了个新发明,人们懂得了在硬木轨道上面钉一层锻打出来的铁皮,用以保护木料。维护轨道的成本,一下就降低了一半还多。

所以韩冈希望两边能通力合作,共同钻研,加快轨道和车辆技术的发展。就在前两天,他还私下里让人传话,要军器监里的匠人们不要气馁,顺着既定的方向继续钻研改进。只要量产化的轴承处理来,就算不是上等的钢材,只是使用普通的钢铁,也比如今的木质轮轴要强出百倍。

韩冈没指望一口吃成胖子,技术的发展总是一步步来的,不是说砸钱进去,就能看到想要的结果,许多都是打水漂了,甚至连个泡都不会冒——不过有一点则更加肯定,那就是不去研究,就永远也不会有成果。

与韩冈一番问对,赵顼也算是放心了下来,“襄汉漕渠有军国之重,此事就多劳了韩卿了。”待韩冈谦虚了几句之后,他则又笑道,“说起来也是韩卿的功劳。只是疟疾一事,就让多少旧日的多少医家束手无策。当初狄青领军南下,竟有一多半得了疟疾,病亡近半数。但韩卿到了广西之后,一下就找到了疟疾的成因,就不到一成得病。”

“关于疟疾与蚊虫的关系,臣一开始只是推测,不敢妄下定论。根究起来,可以说是碰运气给试出来的。”韩冈谦虚着。其实除了预防以外,他还派人去找治疗疟疾的特效药,他知道青蒿素,但他命人找来的诸多青蒿,却是没有给试出来哪一种合用。

“不是格物所得?”赵顼笑着问道。

“正是格物所得。臣是观禽兽而又所得。禽兽之属亦是生于天地之间,许多时候,甚至比人更懂得如何自保。比如大象,喜欢在身上抹上泥浆,比如水牛,总喜欢泡在水里,牛尾也是用来驱赶蚊蝇。虽是庞然巨.物,却是对蚊蚋畏如蛇蝎。当时臣便猜测,这应该不是怕痒,而是畏疾。”

赵顼愣了一愣,偏头想了想之后,慢慢的点头,“原来是这么回事。”抬头又看着韩冈笑道,“不过世间都说韩卿是药王弟子。经过广西一行,这一下子,可都是给认定了。”

韩冈叹了一声,依然是绝口不认。外面的传言他也知道,因为出兵广西的西军士卒少有得病,的确更让人认定了他药王弟子的身份。

世间有种说法,他韩冈不会施针开药,是因为他只学到了孙真人的一半医术,是万人医,而不是一人医。学到的医术只能用来医治万人,而不是一人,所以才有疗养院,所以才能保着西军不受疾疫之苦。

这种说法,从韩冈还在关西时就有了,到现在越发的被人所认定。但韩冈依然是不承认的,身为根正苗红的儒门弟子,神鬼之事只能远避,决不能近身。

从武英殿中出来,已经是黄昏了。赵顼还留在幽深的大殿中,专注的望着他治下幅员万里的土地,似乎怎么也看不厌倦。

深灰色的天空下,一座座殿宇沉浸在暮色中,显得阴气森森,在阴暗处仿佛潜藏着无数妖魔鬼怪。

一阵夜风从殿阁之间刮了过来,寒意透骨,让韩冈不禁打了个寒颤,宫禁之中的确不是住人的地方。

举步向宫外走去,韩冈回忆着今天在武英殿中的一番对答。从头到尾想了一遍之后,算是放心了,应该没有问题,天子也终于同意了他铺设轨道的方案。从今以后,轨道不再局限于矿山和码头,而推广到天下所有需要运输的地方。

水运的成本的确不高,所以就算到了后世,也是一条十分重要的运输手段。不过水运的局限性实在太大,而有了轨道之后,大部分的平原、乃至河谷都能派得上用场。这才是他的初衷,开凿襄汉漕渠只是其中一个目的而已,韩冈做事,不是一石数鸟,可是懒得动手。

只要这一次成功,轨道就能在国中推广开来,有了更为便捷且运力更大的交通工具,对于商业发展的好处不言而喻。也能大大降低物流成本,相应的,工业的发展也将得到一个更为有力的推动。

