\section{第31章 战鼓将擂缘败至(11)}

一片死一般的寂静中,梁乙埋踏上了血色的道路。跟在身边的将领、近卫皆是默不作声,视线随着他的身形而动。

梁乙埋曾说,他不在意前军受挫,只要能缠住宋军就行。可前军受挫到全军溃散的地步,伤亡上千,还折了一名大将,他却不能不在意。

在嵬名济的无头身躯便停下脚步,梁乙埋眼神沉沉。将旗、头颅都不在了,甚至连盔甲也给剥了去,要不是他胯下的战马,还有丝绸质地的内衣,谁也认不出这具只剩内裳的无头尸,会是宗室中颇受期待,被寄望于未来的几十年里,能统率国中大军的年轻人。

视线在嵬名济的尸身上驻留良久,梁乙埋心里中纷乱如麻,一败再败,还接连丢了都罗马尾和嵬名济这两位与他关系紧密的大将,这让他回去怎么向人交代!?他梁家在国中的地位还如何再维持下去!?

而就在梁乙埋身后,景询皱着眉头,在长长的一片凝结的暗色血迹中,不知该如何落脚。

他昨日曾说,高永能光明正大的撤离必有诡计,没想到就真的给他说中了。但景询收起了一言成谶的得意,低眉顺眼的跟在西夏国相身后。梁乙埋个性外宽内忌,尤其是受挫的时候,更是如同一个点着了引线的爆竹般危险,稍有不顺,便会送掉一条小命。

但景询还是想叹气,昨夜一战,被斩首的铁鹞子就超过六百,而在黑暗中逃跑的过程中,因为落马、冲撞,又有上千人受了筋骨伤,其中当有很大一部分,再难恢复。而且究竟有多少人在黑夜中慌不择路,掉进了冰冷湍急的无定河,眼下也是计点不清。唯一可以确定的,就是前军的四个千人队彻底失去了战斗力;以及三个部族,要从身居朝中高位的豪族名单上掉下去了。

现今跟着梁乙埋南下追击的中军,就只有七千铁鹞子,即便他们都是从各部军中挑选出来的精锐,可眼下的战局,使得景询完全失去了取得胜利的信心。

跟在梁乙埋身边,原本昨日抢着要追击的一群人,现在眼里只剩下庆幸。

景询不屑的瞥了他们一眼,还没抢到财物就想分赃,这世上有这么可笑的事情吗?这群蠢货做出来了,而且还败了!要在黑夜中拖延敌军的行动,怎么能不堤防他们的反击?!

“结明爱和旺莽额现在该到哪里了?”梁乙埋突然开口,打断了景询的思绪。

“午后时分,就该到抚宁堡了。”

景询恭声回答,可他不认为今次绕道前方的计划还能成功。吃一堑,长一智,在丢了抚宁堡之后,宋人不会再无防备。而且当日偷袭抚宁堡的那一支偏师,还在细浮图城的守军手上吃了不小的亏。损兵虽不多,但来回一趟什么都没赚到,连老底都亏光了。今次受命堵截高永能前路的结明家和旺家,这两家洪州宥州的豪族只要运气差点,怕是也要从朝堂高位上除名了。

而且现在最大的问题是没粮了。

银州的存粮连积年的老底都被翻了出来,横山周边能找到的蕃部,所有能下肚的存货也都被洗干净了。可再过两日,除了出来追击的铁鹞子还能靠多余的战马和骆驼支撑几天外,后面的步兵就要彻底断粮。如果不能现在就下令,让他们去银州西面的石州、夏州去就食,并继续往西去盐州以保证粮食的供应,保不准饿着肚子的他们会做出什么事来。

在景询看来,与其在在这里追击壳子硬得能把牙齿都崩掉的对手,还不如集中兵力去攻打混乱中的环庆路。可景询眼下不敢劝,只能先等梁乙埋在继续碰钉子后,自己冷静下来。

可梁乙埋现在看起来却没有丝毫冷静下来的迹象。西夏国相重新跳上马,对着众将怒声吼道:“还等什么?!宋人鏖战一夜,已是神衰力疲,不趁此机会追上去,。还有去抄截高永能后路的结明爱和旺莽额,你们想把战功和斩获都让给他们两人吗?!”

虽然心中惶惑不安,但各部将领还是躬身领命。之前各家都已经投入那么多了兵力和钱粮,如果就此放弃,前面的损失就算打了水漂。想来想去,他们觉得还是得追加投入。

七千党项骑兵强打起精神,在梁乙埋的督促下,开始继续向南进发。

……………………

趁着大捷的余荫,罗兀守军一口气向南撤出了近十里,在河谷中稍显宽阔的地方,扎下了营盘。

由于有了足够的时间和空间,营地不是昨夜的长蛇阵,而是武经总要所载的李靖立营法,以六营环绕中军,宛如六出之花。道侧高坡上更立一小营,驻有一个指挥的弓手,居高令下压制攻至营前的敌军侧翼。

昨夜将四千铁鹞子追杀了五六里的数千环庆锐卒,此时就在高永能所部护翼的中军处酣然入睡。而王中正由于数日奔波劳累,也支持不住,放心下来的他也去睡了。

种朴却还在沉浸在计策成功的兴奋中,怎么都睡不着。而韩冈则是精力过人,也是半点睡意都没有。所以他们两人在听到了敌军追至的报警声时,都是第一时间来到了高永能的将旗下。

战鼓声中,等候已久的宋军将士飞速的列阵而出,在无定河边与渐次抵达的七千名铁鹞子遥遥相对。在他们所选择的战场上,选择与再次追至的敌军正面抗衡。

号角声响起,刚刚抵达的党项骑兵,毫不停歇的向着尚未集结完成的宋军阵列冲锋而去。

不过在宋军尚未完成的箭阵面前,仿佛是当日上当受骗、预备追击宋军离城车马时的翻版,依然碰得头破血流。而当箭阵最终成型,一波波的铁鹞子轮番上阵,也只不过时增添了己方的伤亡数而已。

一名党项将领终于失去了战意,在轮到他带兵出击的时候,他冲到梁乙埋的面前,摇起了头。

“再冲!”梁乙埋命令毫不容情,他沉沉问着,“宋人还能有多少箭矢?!”

“冲不了了!”那名将领在梁乙埋面前抬着头叫着,他的族人承受不起更多的伤亡。

梁乙埋并不与他多话,就像看着虫子一样竖起了一根手指,轻轻一划,“斩了!”

西夏国相的亲卫立刻将人架起,而周围的环卫铁骑也一下子就控制了那名部族将领的护卫。

看着转眼就送到眼前、犹向下滴着血的首级,梁乙埋叫着环卫铁骑的第二部将官的名字,“浪讹迂移!你率本部为督战队,若有人敢于临阵退缩,格杀勿论!”

浪讹迂移领命而去,可是督战队的作用也不过是更加证明了宋军神臂弓的赫赫威名。

当在督战队的促迫下,站到阵前的那一家骑兵,被一丛丛利箭射得全军溃散的时候,梁乙埋终于面无表情的下令道:“可以退了!”

‘不会吧……’景询突然间惊觉。

没有被逼着上阵前的各家精锐骑兵,全都是亲附梁家的豪族。而与梁家关系疏远的几家,他们的族中精锐已经损失殆尽。方才以不从军令、违反节制而被杀了领队将领的那一家,更是与梁氏兄妹不合已久。以方才之事为借口,回去后,梁乙埋当可轻易将之灭族。至于迂回向抚宁废堡的结明爱和旺莽额两人,他们也都是不太听梁氏兄妹的话。

‘他们什么时候达成默契的?!’

望着一脸侥幸的诸部族酋,景询目瞪口呆。他并不觉得难以置信,只是对梁乙埋的狠厉和决断自叹不如。换作他坐在梁乙埋的位置上,也会为了保住权势和身家性命不择手段。但能在短时间内就下定决心,如梁乙埋这般不动声色的就改变了目的,利用宋军解决了后患,景询自问他肯定做不到。

‘但这是饮鸩止渴啊!’

低下头,不再看准备离开战场的梁乙埋,景询的心中突然觉得堵得慌。

……………………

梁乙埋终于还是退了。

在宋军的欢呼声中,最后一名铁鹞子消失在北方的山间。一队斥候一人三马,吊着党项军的尾巴跟了上去,以防他们偷袭回来。

不过韩冈认为这一可能性不会太大,党项人一败再败,志气已衰,无力再回返。接下来回绥德的几十里路,当是不会太难走了。

今次的横山攻略,从结果上看,的确是大败。消耗良多,却毫无所得。但从战术上,却是一战都没有输过。反而是连番大捷,打得党项人抬不起头来。

三十年的卧薪尝胆,三十年的养精蓄锐,使得陕西缘边驻军的实力,开始在整体上压倒党项军。只是因为主导全局的主帅的失误,使得今次战事功亏一篑。

韩冈事前对战局的判断,虽然从结果上看并没有问题。但宋军的实力却是超乎他所预想,战略和地理上的劣势,竟然为战术上的强势所弥补。这让韩冈也不得不感叹,就跟足球是圆的一样,战场上的事果然是难以预料。

但败了就是败了,无论几多胜利,几多斩首,罗兀城的确是丢了。当韩冈再次回到绥德,一切重新回到了起点。

横山攻略数年内不可能再翻身,等陕西宣抚司解决了咸阳的叛军,就会随着韩绛的去职而烟消云散。

接下来,主持拓边河湟的秦州缘边安抚司,将会取代他们站上舞台……

重新擂响大宋的战鼓!

