\section{第33章 女儿心思可知否(中)}

韩阿李和云娘一边收拾着韩冈和李信带回来的包裹,一边不停的抱怨着:“王官人也真是,年节都不让人过好。”

韩冈打着哈哈:“事前谁想到会下那么大的雪……不然除夕前肯定能回来。”

从两人带回的包袱里,翻出来一堆零零碎碎的杂物。除了几件换洗衣物和书卷外,剩下的都是蕃部送的节礼。王韶到得巧,既然蕃部的礼物有刘昌祚一份,当然也得有王韶的一份,连同韩冈、王厚都沾了光。

礼物贵重倒是不贵重——贵重的王韶和韩冈不会要,蕃部也送不起——并非金银财货,都是西北常见的土产,几张上等兽皮,几块打磨得极粗糙的玉石,还有刀、匕之类的短兵,十几个部族送来的礼物都差不多的类型。

韩冈把收到的礼物送出去大半,都是给了王韶身边的亲兵,最后留下的是四张完整的硝制过的羊皮,其中有两张说是自纳木措边野羊群中捕来的上品,由逻些城【今拉萨】的商队带来河湟。

可这两张羊皮都不是山羊皮,韩冈怎么看怎么都觉得应是藏羚羊。如果真的如他所想,那他可谓是为灭绝藏羚羊的事业又出了一份力。若是哪天有人送给他一张花熊皮,韩冈可是一点都不会意外——如今的秦岭,正有大熊猫满山乱跑。

另外几件礼物就不如藏羚羊皮那般珍惜,一串像石头多过像玉的杂色玉佛珠串,一对份量比工艺更有价值的银镯,三把装饰朴素的尺半短刀,如此而已。

韩冈把玉佛珠串递给韩阿李,最好的一柄短刀给了他老爹,银镯则留给韩云娘。又道:“剩下里面有一半是给表哥的,云娘你记得给表哥缝一套跟我身上一样的衬里内褂,剩下的给爹娘缝个靴筒。”

韩云娘低着头应了,自韩冈回来后,她一直都默不作声,低着头做事。韩冈看着她的样子,微微一笑,小女孩子的心思还真不难猜。

李信这时又出去了,他喝了热汤,烤暖和了身子,便到院中照料他和韩冈骑回来的两匹马。韩家的院落一角,搭了一间牲口棚,原来养着驴牛各一头,后来都卖了给韩冈换药钱。现在里面空着,安顿两匹坐骑正合适。

韩阿李拿起几张皮子,一张张对着灯光比划来比划去,似是在计算着该怎么做才能最省料子。突然又放了下来:“对了,三哥儿。你舅舅过年前托人送了礼来,谢你荐了信哥儿进了经略司衙门……”

“都是自家人,还谢什么?而且也是表哥武艺高强,孩儿只不过是在机宜面前提了一句罢了。”

“信哥儿的事,你要多多上心。你上次不是说王家的小哥比你还小一岁,可再升一级就是官人了。信哥儿哪点比他差了?!性子比他稳重得多,长得还没他那般老态,身手跟你外公年轻时也差不离了,如何做不得个官人?”

“是,是,孩儿明白,孩儿明白。”韩冈头点得小鸡啄米一般,不停的应承着,反正他知道这些事跟老娘是有理说不清的。

听出儿子是在随口应付,韩阿李狠狠地剜了他一眼,又重重地哼了一声,“今次你二姨也一并托人送了信过来,她家还有你的两个表弟。你二姨夫也是个吃兵粮的,教出的两个儿子都不差。听说你现在做了官,信哥儿也有了出身,便想着一起过来。都是自家人,能照顾就照顾一二。你如今是官人了,身边也得跟着些知根知底的。”

“娘说的是。等孩儿从京城回来,肯定会给二姨家的两个表弟找个上进的门路。”

韩冈本身并不太喜欢一人登天、鸡犬飞升。但在家族观念浓郁的古代,不睦亲族都是罪名,亲亲相隐是法律提倡的行为——如果亲人犯法,只要不是十恶不赦的重罪,可以理直气壮的为他们隐瞒,也不会因此而得罪——提携一下亲友,只要他们足够称职,无人能说不是。

当然,前提是称职。如果没有什么本事,那也别怪他不讲人情。内举不避亲,外举不避仇,本质也是以举贤为重。李信武技了得,性格寡言可信,所以得了王韶青眼。如果李信庸庸碌碌,又怎么入秦凤机宜的眼界。

听韩阿李说,他二姨家的两个表弟也是打算在军中混个出身的武夫,韩冈心中不免有些失望。他一直都很希望有个商业头脑出色的亲戚。宋代并不歧视商人,不像唐朝,商人连参加科举的资格都没有——三元及第的金毛鼠冯京,便是商家出身。而且官宦人家做生意的情况也多得是,自来都是官商一家亲。

世风如此,韩冈当然也想有个可信的亲族帮忙打理产业,也省得他手头总是缺钱花。王韶正管着与蕃部有关的营田和市易工作,其中不需要歪门邪道便能够发家的机会多不胜数。但韩冈搜遍身边,还是找不到一个有用且可信的帮手。

‘若是亲戚再多点就好了。’韩冈很自然的就有了这方面的想法。

韩家是从韩冈祖父辈时才从京东密州【今青岛】老家迁来秦州。韩千六是独苗,韩冈如今也成了独苗,两代单传,使得韩家在关西别无亲族。韩冈若想得到亲族支援,眼下也只有靠韩阿李那边的亲戚。要不然,韩冈就得给自己找门好亲事。

这不是为了少奋斗三十年的做法,而是此时的通例。通过血缘和婚姻联系起来的士大夫,他们之间的关系如同一张张网,形成了庞大的官僚士绅阶层,覆盖了大宋的四百军州。

王韶的两任妻子,皆是德安大族的女儿,王厚未过门的聘妻也一样是江州士族之女。韩冈的老师张载,他的两个表侄便是鼎鼎有名的二程。晏殊的女婿是富弼,富弼的女婿是冯京。晏殊、富弼翁婿两任宰相,而冯京已经做到了有计相之称的三司使,离宰相之位也是一步之遥。

韩冈若是能攀门好亲,对他的前途发展,助力匪浅。只是韩冈对此兴趣缺缺,自家已经有了官身,并不着急娶妻。平常人多有想靠着裙带关系升上去的念头,而韩冈并不觉得有此必要。这个时代讲究着父母之命,媒妁之约,韩冈并不会奢望去谈什么自由恋爱,只盼能找个贤淑的浑家。

韩阿李已经将几块皮子都仔仔细细的看了一遍,皮子的质量没有话说,能让人拿出来送礼,也不可能有缺憾,这些都是自己儿子辛辛苦苦挣来的。儿子为她在兄弟姊妹中挣了光,韩阿李其实恨不得将所有亲戚都通知一遍,告诉他们自己的儿子做官了。而提携自家兄妹,韩阿李心里也做得很畅快。

放下手中的皮子,她又一次叮嘱着儿子:“三哥儿,你答应了就千万别忘掉,等过几日娘就托人给你二姨带信去。”

“娘尽管放心,孩儿绝不会忘记。”

“还有你四姨,等他收到你为官的消息,肯定也会来的。她家好像也有个儿子,也别忘了。”

“是……是……”

韩冈连声应诺。韩阿李并没有其他兄弟,韩冈的舅舅只有一个,但还有两名姨妈。两人都在凤翔府,一个嫁了个小军官,另一个据说是攀了一门好亲,嫁给了一个姓冯的豪绅做续弦。但出嫁后便与兄弟姐妹没了往来,最后只听说后来生了个儿子。

韩冈对他的四姨根本没有什么映像,而且因为秦州和凤翔间隔数百里的关系,就是舅舅和二姨旧时也是托人带信寄物往来,十几年来也就见过两三次。

门帘一动,李信把马安顿好后,又走了进来。韩冈问着他道:“表哥,四姨嫁的冯家的表弟,你可曾见过?”

李信愣了一下,摇了摇头,“就十年前外公过世的时候见过一面,后来就没见了,只跟二姨家的两个走得多。”

“是吗?”韩冈想了一下,决定不去想冯家表弟的事,反正他也不一定会来,来了也不一定有用。他站起来,“算了,不说这么多。夜也深了,爹爹,娘娘,你们早点睡。表哥,你也累了几天,早点休息吧。”

李信点了点头,起身回房。韩阿李和韩千六也站了起来,道了一句:“三哥儿你也早点睡。”也回房去了。

房中就只剩两人。小丫头低头拨弄着火盆里的木炭。韩冈看着她,突的咳嗽了一声,道:“我先洗个澡再睡。”

韩冈喜净,在路上奔波了三天,回来后肯定要洗个澡才去睡。韩云娘当然知道韩冈的这个习惯,按说现在就该烧水去了。但她一动不动,仿佛什么都没听到。

韩冈笑了,看起来是没有及时回来惹得祸,虽然有充足的理由,但女孩子要闹起别扭可不管什么理由不理由,不论千年前后,皆是一般。

韩冈做事直接了当,一把将小丫头强拉过来,紧紧抱住,贴着她耳边道:“想我没有?”

可小丫头在怀里用力挣扎,不是过去那种欲拒还迎的推拒,而是真的生气了。

ps:宋人结亲不尚阀阅,但另一方面,却爱好投资,每每挑选能成大器的女婿。而许多高官也喜欢提携亲近的后进。如富弼,他能成为晏殊的女婿,就是范仲淹的推荐。韩冈如今虽然入官,找到一门好亲也有些难度,他缺着一个进士及第。有进士出身和没出身,晋升速度天差地远,打个比方,相当于一个是高铁,一个是普快,两个差距是很大的。而要弥补这种差距,则要靠军功。

今天第三更,继续红票,收藏。

