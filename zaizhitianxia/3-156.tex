\section{第41章 礼天祈民康(三)}

离着腊月越来越近,天气一天冷过一天。

几场寒流下来,黄河上的冰层已经冻得如同钢铁一般。厚厚的有两三尺,想凿出一个洞来,都要大半天的时间。

韩冈面前就有了个冰窟窿,并不算大,只有一尺见方。但从冰面到水面,就有三尺髙。时不时就能看到一条鱼窜上来,在水面上翻腾一下,立刻就钻回水中。

竹制的钓竿拿在手里,一根钓线垂到了冰窟中。

韩冈正在黄河冰面上钓着鱼。

与韩冈差不多,在黄河冰面上钓鱼的人数不少。凿上一个洞,便将鱼钩挂了饵放下去,不用片刻就能钓上一条鱼来。其实甚至可以不用鱼钩钓,只用拿根长枪向冰洞下一搠,就能扎起一条上来透气的大鱼。

不过韩冈是来休闲的,不会这么没有耐心,用鱼叉来破坏情调。他盘膝坐在一辆平板雪橇车上,拿着钓竿,戴着毡帽,除了没有白胡子之外,就是一个姜太公的架势。

但他身旁坐着周南。年轻娇美的花魁披着腥红的连帽斗篷,帽子照在头上,边缘缝了一圈白色的兔毛蓬蓬松松,衬托得绝美的小脸更加娇俏。玲珑丰韵的娇躯裹在皮毛中,软软的抵在韩冈身后。时不时递过来一杯热汤,让他喝了暖和身子。

韩冈今日也是临时起意,看着天晴,就带着妻妾家人出来到黄河河边上来钓鱼,看着悠闲得不能再悠闲了。不过过了半天,远处的渔民不停的大呼小叫的,但韩冈这边动静却很少。

“官人,钓到了没有?”王旖从河边俏生生走过来,问着韩冈。

韩冈举了举钓竿,很无奈的说着:“才有两三条了。”

官宦人家的女眷不便随意外出,更不能随便被外人看到。即便春来踏青,到了郊外坐下来,都要拦上一重步障。但韩冈不在意这些,带来几十名衙中的军士,在黄河边圈出了一块僻静的地方。

今天出来的,就只有韩冈和他的妻妾儿女。他的三位已经得到官身的幕僚中,魏平真和方兴,都去了京城参加铨选。而游醇是准备要考进士的,无意铨叙,依然在县学里督促着学生功课。

至于王旁,因为王旖叔叔王安国最近身体不适,他便去了东京探望——王安国在京中担任着秘阁校理,不像韩冈身上的集贤校理是个空头加衔,以示天子看重,王安国是真正在崇文馆中做着事,整理着馆中的书籍文牍——因为王旁不在,只有韩冈在,王旁的妻子庞氏也不便出来。

看着妻子走近了,韩冈拍了拍,示意王旖在身边做下。他能陪着家人的时候实在太少了,今天也算是一个补偿。

王旖先是看了一下周围,确认了没有闲杂人等,连韩冈的随从都远远躲到一边,方才赧然的在韩冈身边坐下。周南忙跪起来,给主母奉上温补的热汤。

王旖捧着杯子暖着手,靠在丈夫身边,心头也是暖暖的。微微笑着:“能钓到鱼也算是好了。奴家小时候跟二哥去钓鱼的时候,只钓上过虾子,就没见过鱼。”

“想不到你小时候也是爱玩闹的。”韩冈笑了笑:“不过在黄河上,能钓到黄河鲤鱼才叫好,其他鱼都不能算数!看我今天钓个十条八条鲤鱼上来,卖到京城去,也有个三五贯赚头。”

冬天的黄河鲤鱼在京城中很受欢迎,不但肉质肥美,而且比其他季节要少了不少的腥气。是做鱼脍的好材料。不过冬天的鲤鱼活动少,似乎是在冬眠一般,钓到的难度很大,所以在京城中售卖价钱也便很高。想在冬天吃到鱼羹、鱼脍,少说也要费上四五百钱。

王旖偎依在韩冈身边,看着冰窟窿里的钓线一动一不动,过了一阵,她忽然道:“官人,不要紧吗?”

韩冈静静的把着钓竿,满不在意的说道:“还有十天才到冬至,两天后再去京城,能赶上斋沐就没问题。”

韩冈刚刚辞了天子的委任诏令,没有接下中书都检正的差事,正巧郊天大典的工作该忙的也都忙完了,可以歇上一歇。

桥道顿递使毕竟是孙永,而不是他韩冈,没必要整天顾着、看着。京中的流民如今也是一日少过一日,不是回了河北,就是报了名,往熙河路和荆湖路屯田去了。

加之府界提点衙门里的公事,耽搁两三日也没有关系,更不用说他马上就要去京城,随同参加大典,衙门的公事本就可以交给下面的属僚来处理。

他不知道孙永会怎么想,但韩冈要感谢天子的这份诏令。就是因为拒绝了中书五房检正公事这个职位,所以韩冈才可以一起将身上的府界提点一职的公务也放上一放,以向天子表明,他并不是贪恋眼下手上的职位,才不肯接下中书都检正这项工作的。

这等假撇清的做法,是习俗,也是惯例,就像天子即位前要三辞三让,而臣子们接受要职,也要多次拒绝一样。身在宦海,不能免俗。

而韩冈却也乐得清闲一下。

“为夫辛苦一年,歇上几日,天子也不好怪罪的。”韩冈笑说着。一把圈住了妻子已经恢复纤细的腰肢,手也顺势向上探了上去。

“官人!”王旖涨红了脸,连忙站起身,闪到一边去。这等夫妻间的亲昵举动,在家里能做,在外面怎么能行?嗔怪着:“都是要陪天子奉祀天地,哪有这样不知体统的?!”

韩冈哈哈大笑:“敦伦尽分,夫妇大义。仰不愧天,俯不愧地。”

王旖又羞又恼,抿着嘴直跺着脚。眼中泛红,已是泫然欲泣,孩子气的指着韩冈:“你就会欺负人。”

“官人过两日就要去京城,随侍天子奉祀天地。”周南看着闹了起来,慌忙开口,“奴奴过去只是听说过,仁宗皇帝主持明堂大典时,韩相公、富相公,都是头戴进贤冠,罩以貂蝉笔立,身穿朝服,随扈天子。天子拜于堂中,八侑舞于殿下。而出城郊天更是难得,那样阵仗,能见一次都是好的。”

周南说话只为了缓和气氛,但说起来后,却是变得一幅悠然神往的样子。

教坊司的任务可不仅仅是在妓馆酒楼中陪笑挣钱,或是参加宫宴酒会,也有参与朝廷大典的工作。比如祭天时的八侑之舞,就是由六十四名乐班的成员一起跳起——不过都是男性。

而女子也有任务。教坊中的童女,在许多典礼中都要上场。周南的小时候曾经作为教坊司的舞班成员,与一众小姐妹一起参加过皇后亲蚕的典礼。

王旖转到周南这边坐下:“我们也只是看个热闹,其实做了天子,一辈子都出不了开封地界。一年去一次金明池,三年去一次青城宫,官家能出东京城的机会,一只手都能数得完。”

王旖生长在士大夫的家庭中,对于皇帝的看法,自不会如普通百姓一样,听到皇帝二字,就肃然起敬。清楚所谓的皇帝,不过是个被无数规矩拘束起来的普通人而已。

“说得正是。做官的人,天南地北能去得。河北之雪,塞上之尘,江南的风月,蜀地的山水。做臣子的都有机会看个一遍,但天子便不可能。”韩冈心有感慨,黄河千里冰封之景,千万人都能看到,唯独赵顼看不到。他叹着,“所以天子常为奸臣所欺瞒,乃是见识不足之故。”

除非封禅、亲征,否则开封城南五里的青城行宫,就是天子赵顼能离开京城的最远距离。汉家天子可以去上林苑行猎,唐时天子能去华清池洗澡,但宋室的皇帝,自太宗之后,就没有了游猎习惯了。而当今天子封禅泰山、亲征敌国的可能性,也可以说是零。

纵然提封万里,拥有万邦,但天子能活动的空间,也只有东京城那么大。其中绝大多数的时候,更是只能蜷居于深宫之中。抬头望着周围不到十里的天空。

从没有看过大漠孤烟,从没有看过海上日升,更不可能了解得到天下黎民的生活、工作,甚至都不会知道,他所继承的土地到底有多宽广。

这样的人却掌握着国家,控制着亿万人的命运,让从亿万人中奋斗出来的佼佼者都不得不跪于其下。

韩冈其实不甘心的,尤其他身体里有一个来自于千年后的魂魄。前段时间又有割地之事,让韩冈对如今的皇帝更有了看法。

说句实在话,韩冈觉得天子还是在后宫中多亲近嫔妃比较好,平时主持一下祭祀、典礼,如此就够了。军政之事,还是交由更为合适的人来处理,天子最好不要乱掺合。老老实实的当个装饰品多好!向东出了海三四千里,就有一个现成的好例子。

韩冈说得肆无忌惮,王旖、周南甚至不敢搭腔。半晌之后,王旖才勉强开口劝道:“官人,这话只能在家里说”

韩冈笑了起来:“这是自然,在外面可不会说的。”

