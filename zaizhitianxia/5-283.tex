\section{第30章 随阳雁飞各西东(14)}

韩冈是辛苦做事,积功走上来的。所以一直以来,看那些走言官路线的大臣,并不是很顺眼。

旧年黄河决口,改为东流,致使水患频频。朝廷准备整修河防,向群臣征求意见。司马光连番上奏,朝廷见其在水利上说的头头是道,便决定让其都大提举河防工役,按他本人的提议去主持修筑河防工事。然后吕公著便说,这非是优待儒臣之法——‘非所以褒崇近职、待遇儒臣’。而司马光却也没有主动自请上阵,倒是之后接下了检视河防利害的差事。

也就是说,所谓儒臣只需要叉着腰说话,不需要做事。监督可以,做事就免了。要是一定要派他们去做实事,那便是‘非所以褒崇近职、待遇儒臣’。

但新党的几位核心皆是做实事出身的。王安石、吕惠卿、章敦都没做过御史,韩冈也是一样。都不是靠嘴皮子骂人出头的。

这就是为什么他对新党多有认同的缘故,好歹是做事的。不做事永远都不会有错,更是可以站在干岸上笑骂由心,但做事的官员,怎么都能被挑出刺来。

韩绛闻言也是怫然不悦:“非是为商人,而是为国事。否则我等又何须劳动圣驾,在崇政殿中为此事议论?!”

李清臣张口便道:“难道这几日齐云、赛马两社里争的不是国事?”

李清臣彻彻底底的不给宰相面子,韩绛脸色发青:“那是商人无知,难道读了圣贤书后还不知道以何为重?!”

“可不只有商人!”李清臣驳得更快。

李清臣与韩绛一争起来,韩冈也就好说话了。他站了出来打圆场,“相公,台丞,且听韩冈一言。”

韩冈开口,李清臣立刻退了一步,不再跟韩绛争执,只是暗骂韩冈狡狯。他也不想顶撞宰相,而且要是蔡京事后说愿为国事牺牲一下,他到时可就是要枉做小人了,但韩冈特意当众点出蔡京的官职后,他就必须要出来维护御史台的权威,否则便无法服众。

韩绛则很直接的皱起眉头,方才韩冈将李清臣逼出来,殿上的宰辅可都看出来了。

他转身对向皇后一揖,“殿下,以臣之见,这一回除非是宰执出使,否则耶律乙辛也只会见上一面。遣出来商谈的,当也不过家奴一流的人物。虽为国事,遣一良臣去应对家奴亦未免过当。终究还是得让商人去谈。商家的事让商人谈最为合适,就算失败了,或是传出去,也不损朝廷体面。”他看了看殿中的宰执,“至于取信辽人,不如遣一老成稳重的贵戚,任职厚生司,让李明德作陪,去一趟辽国就可以了。”

殿上谁都知道李明德是谁,专为贵人家子女种痘的痘医,在京中名气极大。而所谓贵戚,本指宗室,但也可扩展到外戚的行列。宗室肯定是在考虑之外,当然就只能是外戚,其实也就是向家人。在厚生司挂个名,能得个好名声。而且厚生司遣人去辽国,即有前例,也不会惹来议论,而且比任何朝臣更能取信于耶律乙辛——纵使宋人有何诡计,也绝不至于把垂帘的皇后的娘家人丢出来当牺牲品。

可向皇后犹豫了。本来于辽人打交道的商团中,就有家里人参与其中,但那毕竟是向家另一边的姻亲,好歹不伤颜面。但现在明着要向家人出面,被勾引的坏了门风怎么办?而且韩冈是提举厚生司,外戚绝不可能跟他并肩,同提举都不可能,该安排什么职位为好。

她望向宰辅们,希望听一听他们的意见。只是两府诸公却都陷入了沉默。

韩冈已经就此事明确表态,还是坚持他之前的主张。西府的两位自然不会去反对,而东府的三人,在李清臣一棒子打掉了所有近职文臣去辽国的可能后,更不打算去冒惹怒整个文官系统的风险。哪个文臣不认为自己很重要?李清臣的话才是他们爱听的。既然如此,没必要将脏水往身上揽——一个是士林清议传些怪话,一个是惹怒朝中文臣,孰轻孰重?谁会想不明白——让韩冈自己应付去!

作为宰辅,需要考虑的是权衡内外。韩冈的提议真要计较起来也不损国家颜面,正如韩冈所说,与商人们坐下来谈的只会是耶律乙辛的家奴,朝廷没必要去为商人的行为负责。逃亡辽国的百姓、士兵年年都有,朝廷倒是可以去怪罪地方上是否治政过苛,但商人们跑去辽国赚钱,朝廷对此又没有下文主张,士林再议论,罪责也加不到两府的头上,韩冈得先出来为他的提议顶缸。

整件事最关键的就是这默认二字!

也就是说,在局势转变时,朝廷可以一声招呼都不打便将绳索收紧。既然已经将绳圈套在商人们的脖子上,宰辅们自然没有二话。纵然会有些杂音,可哪项政策会没有反对者。两府诸公几乎都是新党,当年旧党几次反扑的海啸都撑过来了,难道还会担心一些小风浪不成?

向皇后等了半天,见等不到宰辅们的争辩,想想,觉得他们应是韩冈的意见是正确的,所以才会默认。既然宰辅们也没了意见,那么也必要再多想了。

“这厚生司中任官,是勾当,还是管勾?”她问着韩冈。比起提举或同提举厚生司,以管勾为前缀就差了一等,勾当则更低一等。跟韩冈平级或相近的提举、同提举既是不可能,那么也就能在勾当和管勾两个衔头中选。

“此事自由殿下和相公们裁断。”

韩冈前面已经插手到人事任免,若是再插手具体官职上,可就是侵犯职权。里子都攥到手了,脸面好歹得给人留下。

向皇后和东府的三位宰执讨论了几句,很快就做出了假借管勾厚生司的差事,去辽国出使的人选。

让自己的堂兄出马,向皇后算是解决了一桩大事。歇了口气,她又开口:“商人与辽人交易,需要大量的丝绢,朝廷是不是和买一部分,提供给他们?”

“不可!”蔡确和章敦同时反对。和买已是恶事,而且好处还给商人们得去了,朝廷中谁是傻瓜!?

章敦出来辩解:“由那些商人自行处理,该收的税要收,犯了法后要罚,除了与辽人的贸易,其他概不干涉。万一官府插手进来,那些商人甚至可能会采取向百姓和买。比如两浙贡绢,定例乃是和买。旧年定价合乎人情,而且是官府预先出钱,故而百姓人人踊跃。这是便民之法。但直到市易法施行前,市面上的绢绸价格比百多年前增加了一倍,而和买的价格却丝毫未变,甚至有不给钱强行取绢的做法,这就失去了便民的本意。”

“既然是商人,就要遵循商人的做法。该从哪里买就从哪里买,不能借着朝廷的声威来鱼肉百姓。”

韩冈也怕好事变成坏事。家中有天下闻名的大商行,但韩冈从不相信商人的操守。尤其是与权力结合的商人,他们往往更为贪婪而不知收敛,行事肆无忌惮,必须要压制住他们。

而且他对这一次的任务看得没外人想象得那么重。想靠经济手段收买耶律乙辛,进而影响到辽人统治根基,不这是不可能,只是太费力气。难道还要把希望寄托在敌人的愚蠢和贪婪上,这跟韩冈的性格完全不合。

真正的依靠依然是工业,纵然只是手工业,也比纯粹的商业要强。韩冈对此很清醒。真正决定两国胜负的是铁和血,不是叮叮当当的金币。

韩冈想做的是扩大利益的覆盖面,将蛋糕做大。耶律乙辛在自己得利的同时,还能分出一份好处给其他人。他必然会全心全意的支持韩冈的方案。而在大宋这边,贸易范围扩大的榷场。其实可以吸引更多的辽人来做生意,有大辽尚父领头,其他部族、豪门肯定也会挤上来分一杯羹。随着两国贸易范畴的扩大,说不定连战马都能买的到。对于辽人来说,抢来的好处比不上做生意,又有谁会去犯傻搏命?

整件事差不多敲定了下来,基本上计划就是按照韩冈的想法来设定。最后的总结,向皇后正听得聚精会神。石得一这时却目不斜视的托着一份奏章在外告了罪,然后走到殿中央,掌心中有一块金漆的令牌,“圣人,环庆遣金牌急脚急报,辽人十日前已经兵围溥乐城。”

“兵围溥乐城?!”好几人同时开口惊问。这是准备要打了?!不,算一下奏报在路上的时间,溥乐城那边应该已经打起来了。

“可是要遣人去援救?”向皇后急急问着,甚至坐立不安起来。

“殿下勿忧,溥乐城城主种朴乃种谔之子,精擅兵法,多有功绩。麾下又有三千余精锐。除非缺少粮秣兵械……”

章敦打断了他的话:“溥乐城、韦州、盐州、鸣沙城,关西的几座与辽人邻接的城池、军寨,粮草兵械无不充裕。溥乐城的粮食更是足够一年支用。”

韩冈搭配得很好:“这样一来,那就更不用仓促发兵,让宣抚司自选合适的时机。”

这样的提案当然找不到反对者,陕西宣抚司便是为此而成立。

看似解决了一桩事,向皇后却仍心结难解,忍不住又问道:“万一辽人真的开始攻打溥乐城,这使辽之事是否要缓一缓?”

韩冈摇摇头:“依臣之见,组建使团的计划不用停。只是何时动身,就要看看关西那边的局势究竟会变得如何。如果溥乐城破,一切休提。从此整军备战,为死难的将士复仇。但若是守住了溥乐城,辽人败退,那就可以照常派人去了去了。”

章敦一向与韩冈配合得好:“不攻打韦州,而选择了溥乐城。可见辽人根本就没有破盟的想法。只消再等上三五日。五天内不能破城,接下来再想破城,少说也要耗上一两个月时间了。以辽国的攻城水平,恐怕就要等上太多时间也不一定能破城。”
