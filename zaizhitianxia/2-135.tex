\section{第27章 京师望远只千里(五)}

【新月新气象,三月因为种种原因,基本上都是一更。不过四月份开始,将回复正常更新。求红票,收藏。】

王文谅正得意。

自言一语可置众人于死地,十几个面目猛恶的蕃军瞪着,谁敢质疑?哪人不两股战战?就连他一向看不顺眼的吴逵,也只能站在一边,在心里咬牙切齿。

两人过去因争买一匹河西好马而结下仇怨,最后王文谅靠着在韩绛耳边的一句小话,就把整个广锐军的战马全都夺了过来,将旧日的怨恨以千倍还回。

‘你是有本事,但上面没人啊!’王文谅气焰万丈,‘怎么样!任你再英雄,也有韩宣抚在俺背后撑着。在关西,谁能比当朝首相、陕西宣抚更大的?!’

可偏偏有人硬要落他的脸面。

“本官倒不觉得你有这能耐!”

平和中透着如屋外风雪一般冰寒冷意的一句话,霎时将厅中冻结。

‘本官?!’

听见韩冈如此自称,除了何四、小九两人早有所料,其他人都大吃一惊。吴逵瞪大了眼睛,前面在韩冈面前耍酒疯的军汉,更是浑身酒意化作了冷汗从八万四千个毛孔中涔涔的冒了出来。

而王文谅则是一点一点的转过身,循声望去,就见着个二十多岁的年轻人,澹然坐在厅中一角。那个角落并不只是他一人,但神色从容、风仪自蕴的气质,却能让人完全忽略掉了他身边的甲乙丙丁,目光只会集中在他一个人身上。

韩冈他为官日久,平日里颐气使指,又是久经磨练、饱读诗书,气势自不同于凡庸之辈。虽然没有穿着公服,但的确是个官人模样。

只不过还是有人不长眼,王文谅的一个手下冲前了一步,指着韩冈:“你是哪里来的措大,敢……”

王文谅抬起手拦住手下,如蛇一般的阴冷眼神盯着韩冈,一个字一个字问着:“你是何人?”

“欺压良善,蒙蔽上官,狂悖妄言,目无王法。”韩冈屈起手指,一下下的敲打着桌子,一句句的报着王文谅的罪名,他抬起眼,盯着得了韩绛青眼的蕃人,“王文谅……你就这么回报韩宣抚对你的看重?”

王文谅仰天哈哈大笑而起:“本官堂堂阁门祇候,在韩丞相面前听候使唤,节制一众蕃军,位高权重,岂是你这小儿污蔑得了?”

只是在他的笑声中,听得这年轻人轻轻说着:“不论在关西,还是东京,我韩冈的话……还是有人信的。”

刚刚报出自己姓名,王文谅笑声一顿,人群中也或高或低的接连传出几声惊呼,“是韩机宜!”

“是药王孙真人的弟子。”

“带兵打了两次大捷的韩冈,”

“破家绝嗣的韩玉昆。”

虽然其中混了让人无法付之一笑的一句话,但不论王文谅还是吴逵,却全都变了颜色。人的名,树的影。韩冈在秦州折腾了一年多,几次边地大捷,几次人事变换,背后都少不了韩冈的身影。他这个名字,至少在关西的官场上,已经是无人不知、无人不晓。

陕西的官员虽多,但能威名远播的屈指可数。要么至少是经略相公一级的显宦,要么是久历战事的老将,又或是最近屡立战功的名臣,眼下能例外的,就只有韩冈一人。据王文谅所知,连韩绛、种谔、赵卨的嘴里都提过这个名字。而吴逵也是听说,在庆州的白虎节堂中看到的新制沙盘,就是由眼前这个年轻人所发明。

何四一开始看韩冈觉得他太年轻,官品不可能高。但现在韩冈的身份暴露,官品的确不高,但地位和名望的却是一等一的。他紧张的开始回想韩冈进来后他有没有失礼的地方,生怕得罪了这个有名的官人。

“……原来是大名鼎鼎的韩玉昆,你好好的缘边安抚司不待,好端端的从秦凤路跑来关中,到底是为什么?”王文谅终究不敢再放狂言,只能把官威收起,拿门户之别来堵韩冈的嘴。虽然说得理直气壮,但面前的这个从任何地方让人看不顺眼的年轻人,他仅仅是静静的坐着,眼神沉甸甸的几近千钧,就已经翻江倒海的把王文谅心中的虚怯全都翻了出来,更无力去怀疑韩冈的身份。

韩冈盯着王文谅,“韩冈虽是在秦凤任官,管不到陕西宣抚司中。但王阁职方才说的那番话,韩冈却不能听之任之。”

“……本官一时口误,当会到韩宣抚那里自请责罚。韩机宜,你看这样如何?”王文谅双眼轻轻眯了起来,微垂下来的眼睑遮不住眼神透出的凶芒,

韩冈向来感应敏锐,见到王文谅的样子,他心中一动,心道这厮该不会想铤而走险吧?也就在这时候,李信有意无意的侧了侧身子,右手也搭到了放着刀的桌上,随时可以抽出刀挡在韩冈身前。

韩冈眼神深沉起来,既然不仅仅是自己有这种感觉,那就绝不会是错觉。他将视线低垂,却见王文谅露在外面的双手正半握着,青筋根根凸起,看起来虽然尚在犹豫间,但怕是转眼就要发作了。

不能再等,他摇头一叹,突然上前几步,把王文谅扯住。趁他惊讶得尚未反应过来,就生拉硬拽着他到了自己的桌边坐下。招呼了吴逵坐过来,韩冈又朝李信使了个眼色,李信与韩冈甚有默契,也扯过一张凳子坐了下来。三人前后三面一堵,把王文谅硬是挤在了里面,紧贴着整整两桌广锐军卒。

被十几条大汉围在中央,王文谅一张黑脸煞时变白了。方才他还想着灭口,现在是人在虎口,反而是他。他现在依稀想起,也是方才有人叫出声的,韩冈好像还有个外号——破家绝嗣。

韩冈却是笑得温和,仿佛老友一般,左右拉着王文谅和吴逵的手,“同僚不合那是常有的事,一时气话也不能当真。知错就该,善莫大焉,既然是王阁职的口误而已,也不必闹到韩相公哪来去,伤了人情。”

“都是同朝为官,有何深仇大怨无法化解,阁职和都虞何必为此耿耿于怀。”韩冈倒了两杯酒,分别放在两人的面前,“且尽此杯,一笑泯去旧日恩仇。”

韩冈逼着两人把酒喝了,一杯酒下肚,又向两人介绍起自己亲友的身份,“这位是在下表兄,今次得荐入京,正要去三班院挂个名字。”

“李信。”李信指了指自己。

两个字就结束了自我介绍,韩冈看着李信的处理方法,不由得苦笑起来:“此事非是怠慢,实在是我这表兄不爱多话。”

韩冈声音委婉平和的就像在跟朋友聊天,说了几句。他回过头,提声唤了一声:“店家。”

叫来了点头哈腰的何四,韩冈也不说话,只把眼睛往王文谅的一众手下们身上一扫,老于世故的何四顿时心领神会。连忙小跑过去,低声下去的向其他客人告罪,给十几个蕃兵安排下了座位。

其实不用何四来撵人起来,几十个商人中,没一个想留在大厅里,纵然现在风雪漫天,但仍至少有三分之一选择了冒雪上路,其他人也被小九带着躲到了里面去了。这一票人在江湖上奔波多年,因为身份的缘故,见识的人物多不胜数,眼力、识见皆过常人。王文谅方才动了杀机,有不少人都感觉到了。

有了这个认识,再看韩冈把王文谅和吴逵两个明显有仇的对手,硬拉着坐在了一张桌上,不知什么时候这里就会化为修罗场。暴风雪纵然可怕,但待在这间小客栈里也是一样危险。许多人心里都想着,大不了再走十里八里,不信找不到一间能让人安心住下的地方。

屋外传来风雪交加之外的声音。没有王文谅亲口下令,他手下的蕃人不会聪明到拦截跑掉的商人。可王文谅现在怎么下令?而且杀人灭口的盘算还没启动,就被韩冈扼杀在萌芽阶段,使得他更是坐不安宁。

被韩冈的右手抓着手腕,笑眯眯的谈天说地,王文谅只觉得仿佛被一条过山风缠上,衣袍背后很快就被冷汗浸透。‘他该不会都看透了吧?’

地狱般的煎熬一直持续了一个多时辰,王文谅和吴逵都是一样觉得方才是在油锅中走了一遭,只有韩冈一人喝得兴高采烈。

商人们全都退了房,到了晚上,将会在大厅里休息,空出来的房间,便安顿了韩冈、吴逵和王文谅三拨人马。韩冈没有再找两人的麻烦,读了一会书,就听见门外传来了有节奏的敲击声。

‘是吴逵还是王文谅?’

韩冈并不喜欢自己读书被人打断,合上书,猜测着。李小六过去开门,吴逵便闪了进来。

次日清晨,雪止天晴。

一早起来,王文谅和他的人已经不见了踪影,听何四说他们往长安的方向去了。惶惶如丧家之犬,忙忙如漏网之鱼,王文谅逃跑一般的急窜,让韩冈觉得有些好笑。而广锐军卒,还有一些留宿在小客栈中的商人,看到气焰嚣张的王文谅夹尾而逃,无不暗笑于心。

韩冈已经从吴逵那里了解到了环庆路内部的情况,也知道了王文谅为人处世的手法,以及靠什么得到了韩绛的信任。

信任是根深蒂固的,尤其是对自信到刚愎的程度的人来说,更是如此。韩绛就是这样的人,韩冈无意在当朝宰相的前面把昨天的话拆穿,韩绛不可能会相信——或者说,相信了也不会自承其错——而且他跟王文谅也没愁没怨,只是争口闲气而已。

不过韩绛所用非人,举荐不当,让军中不得安宁,掌握到这样的第一手资料,使得韩冈在进京之前,对陕西宣抚司军中的内情有了更为直观的认识。

眼望旭日冉冉升起,将鲜亮的红色铺满雪原的东方:‘该去长安了,有韩绛,有司马光在的长安。’

