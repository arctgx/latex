\section{第46章 了无旧客伴清谈(11)}

【昨天因故断更,得向各位书友说声抱歉,今天补齐。】

从清风楼的二楼向外望去,街道上正为满天飞舞的雪片所妆点。

不远处的开封府衙完全淹没在纷乱的白色里,偶尔在暴雪的缝隙中,露出了一只飞挑起的檐角。

楼下的街道上,为数不多的行人都用连帽斗篷将自己裹紧,碾过路上青石的马车上,车帘也都罩得不留一丝缝隙。

寒风从敞开的窗户中窜了进来,呜呜的咆哮。雪片飞进房中,贴着浸矾密纹素锦的雕花窗棱啪嗒啪嗒的在风中响着,房内的温度陡然而降。但贴着房间内的炉火,对坐在桌边的韩冈和苏颂,却是只感到一阵扑面而来的清新清凉。

“瑞雪兆丰年,明年当又是个好年景。”苏颂微笑着举起酒杯,为明年的丰收祝祷,温热的酒气从杯中散逸而出,酒香清洌。

韩冈已经将杯中酒一饮而尽,放下酒杯:“丰收一事,即在天,也在人。瑞雪兆丰年,可也要得人才行。不知可有贤良接掌开封。”

苏颂不以为意的笑着:“已经很长时间没人能权知开封府两年以上了,愚兄岂能例外?”

苏颂其实已经将陈世儒弑母案审得差不多了,但御史台却出手将案子抢了过去。

就在两天前,几个御史上奏,说此案初审时勘官不公及吕家因缘请求,迁延多时。如今又欲仓促结案,似有情弊,恳请移交御史台重鞫。

对于御史台的意见,天子点头首肯。苏颂见到此事无法挽回,只能请辞出外,以示自己的清白。

无可奈何的事,苏颂不想多说。他顺手将杯中酒一口干掉,啧着嘴:“清风楼的烧刀子毕竟是不正宗,远不如玉昆你前些日子送来的那两坛。”

“若子容兄喜欢,明天就让人再送上两坛。”韩冈知道苏颂喜欢烈酒,这在出身南方的士大夫中其实不多见,倒是北方人喜欢得不得了,“……不过烈酒伤身,还是不能多喝。”

“天下哪里还有人不知道烈酒不能多饮的道理?”苏颂笑道,“现在烧刀子的名号,比樊楼的眉寿、和旨还要响亮,曹太皇家瀛玉、高太后家香泉更不用说。听说如今一干练气之士服食寒性的丹药,都拿烧刀子来伴服了,而且出自韩家正宗的方好,玉昆你若是遣人当垆卖酒,少不得日进斗金。”

“本来是不想让人多喝才起这个名字的,没想到到成就了这烈酒的名号了。”韩冈无奈的笑了一笑,现如今世人把烈酒都叫做烧刀子,可是他起名时从来没有想过的,“自家酿的酒自家喝,哪里有向外卖的道理。”

大宋酒水官卖,想要酿酒,得去承包——此时叫买扑——酒坊,并从官府购买酒药,否则就是犯法——朝廷设立监酒税的官职,不是为了安排给贬官重责的罪臣的。

不过这些规矩都是针对普通人和低层官僚的,高官显宦自家酿些酒水自用,顺便馈送亲朋好友,已经不算是罪名。更有甚者,皇亲贵胄,如高太后、向皇后、濮安懿王家里,都是酿酒出来贩卖,根本都没人敢于管束。

只是韩冈没兴趣这么做。留人口实并不好,尽管他也想给自家的酒起上五粮液、剑南春的名号,但在一番考量之后还是放弃了。而且烈酒的用途极广,光是用来浸花露造香水,就要消耗许多。给女子用的香水,可比烈酒值钱百倍。

樊楼中一角最贵的眉寿,也不过百文而已,市售的烧刀子也没有比这个价格更贵的。韩冈就是弄个飞天茅台出来,也不可能卖到一贯往上去,除非他能打上五十年陈的牌子。但一小瓶大约二两重,以脂砚斋为品牌的玫瑰香露,装在白玉瓷质小瓶中,从来都是自三贯起跳的。

不过知道韩冈跟脂砚斋香露之间关系的,世上也没多少人,苏颂自然不知,也不说跟酒水有关的话题了,“今天早上在崇政殿,天子的口气玉昆你也听到了,可能要愚兄去河北,都提举河北轨道事宜。”

韩冈当时就在殿上,自然不会没有听到,举杯对上苏颂:“以子容兄的大才,天子自然是要借重的。”

苏颂神色淡淡:“能否去河北还说不定,是否可以建功更不一定,在河北修建轨道没有那么简单。”

韩冈奇道:“以子容兄之材,难道还担心轨道修不成吗?财力人力物力都不缺,子容居中运筹,两年之内建成轨道当非难事。”

或许天子赵顼对苏颂在陈世儒一案中的表现心怀不满,但苏颂在机工之术上面的名声,赵顼不可能会愿意浪费这个人才。有沈括的旧例在前,安排他去河北提举轨道工役,完全是在情理之中。

韩冈本希望苏颂能留在开封府,这样举办赛马联赛的计划,也能更加顺利一点,就跟吕惠卿希望苏颂留下,以保证手实法没有干扰的在开封府界推行。

现在苏颂不得不离开,如果换上一个反对手实法的开封知府,且主动出手干预,吕惠卿就该吐血了。至于新知府会不会反对赛马,韩冈倒并不在意。手实法得罪所有官员富户,可赛马却是对上层有着充分的诱惑力,有人反对,也不过造成一点小麻烦而已。

相对来说,苏颂立功更是韩冈所乐见。

但苏颂没有韩冈一般的信心:“王禹玉有想法,元厚之同样有想法,就是吕吉甫难道不想这个位置?”

“子容兄说得的确没错。王禹玉的确对河北轨道的都提举一职虎视眈眈,想要安排让自己的人出任。之前也来找过小弟。小弟当时就将李诫和其他几个出了力的门客推荐了过去。都提举的位置,小弟手上没人,无法与王禹玉这名宰相相争。但中层的几个实权职位,凭借方城轨道的成功,小弟有充分的理由给自己的人争一争,给他们找个立功的好机会,有能力有经验,没理由不选他们。”

韩冈在苏颂面前没有半点遮掩,“既然王禹玉这名宰相都想要这个位子,那么元厚之、吕吉甫想要这个位置也不足为奇,但决定这个职位归属于谁的权力,终究还是天子手上。王禹玉会违逆天子?”他反问,继而又笑道,“子容兄何须妄自菲薄,元厚之和吕吉甫,他们手上哪里会有比子容兄更合适的人选?”

苏颂依然无话,只是提起放在热水里的酒壶,给韩冈和自己的倒酒。

“难道子容兄还有什么顾虑不成?”韩冈疑惑的问着,“如果怕掣肘太多。小弟推荐的那几人,子容兄都不要也可以。”

“玉昆,你说得是哪里的话!如果愚兄要去督造轨道,少不得要劳烦玉昆你来推荐帮手。”苏颂苦笑了一阵,终于说了实话:“关键还是土地。玉昆,你可知道征地有多难?能铺设轨道的肯定是一马平川的土地,且交通便利。你想想,那些地会是无主的荒地吗?这么麻烦的事,州县中肯定是一推了之,怎么解决?两年的时间,光是征地还不够用!”

“这件事小弟怎么会没想过。为了打通襄汉漕运,可是征了不少地皮。”

韩冈怎么可能没想到?之前在京西征地的事就不说了,千年之后,征地的纷争更是充斥在各色媒体之上。大事营造时会出现什么问题,韩冈再清楚不过。

“子容兄,前两年开封修城墙,被平掉的坟地还少吗?事关河北防务,下面只要有人敢于推托,直接奏报天子,让他轻松一点。至于能不能顺利征地,”韩冈嗤笑一声,“只是钱多钱少的问题。”

公共利益和私人利益孰重孰轻的问题,是千年后各家学派争论的焦点。在韩冈看来,有些时候牺牲私人利益的强硬是必须的,只要将补偿给足就行了。

苏颂无声的笑了,只有韩冈这样的年轻人行事才会如此锋锐,换作是自己,要顾虑的事就太多了,“慢慢来吧。”

他举杯,与韩冈对饮而尽。

数日后,苏颂经过一番考虑,还是放弃了去河北的差事,被安排去了亳州。河北轨道工役,就暂时只能由河北两路转运司进行先期勘察,确定最为合适的路线。至于都提举的人选,则要到明年才能出来。

《桂窗丛谈》的样刊已经出来了。带着墨香的十卷新书摆在面前,厚厚的一摞。韩冈很有满足感的翻着。苏颂不愿去河北所带来的不快,也渐渐消失了。不愿意也没办法,这件事本就讲究你情我愿的。

《桂窗丛谈》是个引子,属于科普读物。要树立起自己儒门宗师的地位,还要设法关联到经义上。这些年来,张载已经做了很多事,韩冈将格物致知到处宣扬,他那边也不得不设法配合,如今因为张载早逝,未竟全功,但有了基础就容易了许多。而且韩冈说话的份量也足够了,不论如何,摆在眼前的事实,说服力永远是最强的。

《桂窗丛谈》即将刊行于世,《三字经》那边也敲定了最后的版本。为此辛苦了一年的邵清和田腴,被韩冈所举荐,在京西的唐州、襄州担任州学教授。

韩冈的一干门客被他荐了不少作为学官,虽然不入流品,但终究吃着朝廷俸禄,日后也是有机会挤入流内品官的行列。韩冈要不是献上了牛痘,抵消了许多的反对意见,想做到这份推荐其实也不容易。

当夜幕降临,韩冈放下书时,抬头看见的,是书房中笼在纱罩下孤独闪耀着的烛火,以及窗外偶尔响起一阵的鞭炮声。

虽然肯定会有更多的士子来投到他的门下,每天也有许多人上门来求见,但相熟的朋友、门客都离开了京城,让韩冈有了几分感慨。

自家都有些像是驱虫药,怎么回京没多久,相熟的朋友一个两个都没法儿在京城待了?

幸好自家的妻儿也该入京了,也就在这两天。

