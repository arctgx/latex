\section{第一章 一年穷处已残冬(下)}
王安石问得突兀,又有些不清不楚,但他想问的什么,韩冈又哪里会不知道?

“却是不清楚。”韩冈摇头,“小婿这几日待罪于家中,又不出门,哪里能听到些什么。家里的人出去听到什么消息,也不会一五一十的说出来。如果外面传说一切都是小婿的过错,岳父你说,谁敢告诉小婿?”

“怎么会传玉昆你有干系?没人会这么想的!”

王安石比任何人都清楚他女婿的好名声。放在任何时候,都不会被百姓认为他会做出谋害天子的举动,就是现在韩冈确实犯了罪,拿出真凭实据来,都会被认为是陷害。

只有朝廷上的一干大臣,与韩冈日日相见,才会明白他不是药王弟子,不是药师王菩萨的门徒,也不是什么转世投胎,而是心思缜密、对万事观察入微的学者,性格刚毅、善于筹划的大臣。除了出众的才能之外,也照样会做错事。两府之内,没人会将他当做神佛来拜。就像只有真正接近天子的大臣,才知道坐在这个位置上的也不过是个普通的凡人。接近了,也就了解底细了。

但是朝臣中哪个没长脑袋?想想就知道这件事的发生根本与韩冈无关。能从这件事中得到好处的人……一个都没有。韩冈作为师长,学生犯下了大错,他难道能脱身?纵是构陷,韩冈也能轻易自辩。

“架不住有人这么想。也会有人设法让人这么想。迟早的事。”

“玉昆!”

王安石脸色沉下来了,韩冈这是故意将话题引偏。

“好吧。”

韩冈无可奈何。王安石对赵煦实在关心的太过了。只看王安石的态度,就知道他终究还是放不开。正常臣子别看表面上忠心耿耿,可到了现在的局面,绝不会一条道走到黑。

若说忠心,王安石能把其他宰辅都比得不能见人。尤其王安石与赵顼之间的关系,可以说是君臣相得的一段佳话。毕生抱负也是依靠赵顼才得以实现。赵顼发病,将儿子托付于他,现如今天意弄人,无法再去实现赵顼的嘱咐,但对于现如今赵煦在天下士民中的名声,他还是切切在心。

可是韩冈不一样啊,忠于职守这一条上是没话说,但对皇帝的忠心那是半点也没有。保住赵煦是形式使然,可不代表他不清楚这样做的后果。

“官家才六岁,没人能说他什么,最多说一句夙世冤孽。岳父你还是担心一下新法。王襄敏是腹生疽痈而死,外面就传他是在河湟造了太多杀孽的报应。这一回,不知会有多少人说是熙宗皇帝推行新法的报应?”

因果报应此事深入人心。也经常在政治上为人利用,用来攻击政敌。如果哪天韩冈被外放或是贬官,他是凉水都不敢喝,尽量的保养身体,免得生病了被人说是报应,又或者被说成是怨望于心。

王安石心情更恶劣了几分,这是他难以容忍的。但韩冈说得又偏偏合理的紧。洛阳的那些老朋友,还有他们的子弟,明面上会为熙宗皇帝哭几声,暗地里还不知怎么欢呼鼓舞。

“玉昆,想喝点什么汤?!”

王安石心情大坏,直接下了逐客令。

韩冈拿这个倔脾气的老头子没奈何,起身告辞,“过两日小婿再来探视岳父。”

“算了,玉昆你每次来,老夫的心情就坏一次。还是多隔几天再来吧。”

韩冈的脚步差点绊了一下,“岳父说笑了。”

“不是说笑,玉昆你哪次来让人心里痛快的?……还是让钟哥、钲哥他们多来几趟好了,老夫心情还能好一点。他们年纪也不小了,可以自己出门了。”

“……只要岳父少给他们糖吃,弄坏了牙齿,小婿这就让钟哥、钲哥登门聆听岳父教诲,住上十天半个月也行。”

“那就这么说定了。”说到外孙,王安石脸上终于又笑容了,“也到了学诗赋的年纪了。放在玉昆你手里,都给耽误了。”

韩冈咳了一声,欠身一礼,然后掉头离开。

都说骂人不揭短,可看这王安石这短揭的,一点面子都不留。

虽然王安石是说笑,也是有几分真心在。

韩冈不想跟王安石的关系弄到这般田地,只是他心里面,隐隐的总将王安石当成对手。想必王安石也是一样。

虽云是翁婿,但韩冈对王安石的感觉却是尊敬而难以亲近。几年来翁婿内斗,多少人在看笑话。到了如今的地步,说不清是谁对谁错。王安石几次三番的压制气学,韩冈也没少给王安石找麻烦。要不是看在王旖的份上,加上都是公心,政治立场相似,说不定早就割席绝交了。

也幸好王安石还是疼外孙,家里的孩子不论是不是王旖亲生,看到了就高兴得很,这才没生分了。

只是离开王安石的府上,返身回家,回忆起王安石的话,心中却踯躅起来。

‘应该是说笑吧。’韩冈回想着,却是没那么大的把握。

……………………

过了年,就是初春。

一年将尽,按历法算,已经是残冬了。不过也是一年中最冷的时候。

骑在马背上,寒风迎面而来,手套,护膝,斗篷等一应俱全,但凡迎风的部位,皆刻意加了防护,可一路迎风,韩冈照样是手脚冻得跟冰块似的。

都说骑马是运动,尽管这话没错,但该冷还是冷,从王安石府到家中的几里路,身上并不见暖,反而冻得更厉害了。

到了家里,韩冈也没立刻进屋,用力的跺着脚,用力的搓着手,手脚恢复了,又搓了搓鼻子和耳朵,等血脉通畅,这才进了屋去。

书房中,融融暖意,仿佛春日。顿时让韩冈感觉好了许多。稍稍休息了一下,他便遣人去唤何矩来。他跟章敦约好的时间还有一阵子,可以先处理一下当务之急。

何矩是顺丰行在京的大掌柜,耳目一向灵通。京城中多少传闻,都是从他那里转送到韩冈手中。

而韩冈现在最关心的,当然就是王安石方才问他的问题。京城百姓到底是怎么看待福宁殿中的那桩四条性命的公案?

事情过去才几天时间。具体的内情,照理说应该还没完全传到下层的百姓中。不过不论是不是与宫中和重臣关系紧密,大部分东京士民,肯定已经是知道赵顼的死因跟他的儿子有关。

朝廷在将赵顼的死讯公布天下的诏书里面,并没有牵扯赵顼死因,真相通报到重臣已经是很难得了,绝不会再向下通报,更不会落于文字。不过朝廷也没有对传言进行辩解和掩饰的打算。

正常来说,朝廷公布出来的消息,通常都不会被百姓采信。除非之后有明证,才会信上那么几分。耸人听闻的小道消息才是京城百姓的最爱,太上皇突然驾崩的蹊跷原因,已经足够让好事的东京士民暗地里奔走相告,掩饰也掩饰不来的。

而在向辽国告哀的国书中,也不可能说赵顼是被儿子害死的,同样是什么原因都没提。也没有另外伪造一份遗诏。一个是因为早已内禅,没有遗诏也没关系,另一个原因,整件事本来也瞒不了人,伪造遗诏反而贻笑外邦。

在朝廷无意隐瞒,又无意公开的情况下,市井中的流言蜚语理所当然的又一个爆发式的增长。韩冈已经让何矩去详细打探,希望能有一个完整的认识,这样化解起来才能有章法。

“……什么样的猜测都有……”

何矩拿着个小本子,打算详详细细的跟韩冈说上一通。

韩冈很干脆的打断了他,问:“有人说是我做的吗?”

“……的确有。不过很少。绝大多数都是看看再说,不想乱猜测。”

“能看出什么就好了。就不知做个一个实验,省事省力,还能省口水。”

在任何人看来,这桩案子都是一团乱麻。真相匪夷所思如,实是千古未有之事。

怎么翻史书都找不到一个六岁天子弑父的先例,谁也不知该如何处置他。而当日服侍太上皇的宫人,同样很难处置。

将福宁宫内殿中的宫人一体治罪很简单,但不符合现在的形势,也找不到能重惩的罪名。只能以失察之罪,加以责罚,甚至都不会是流刑。

韩冈的一句事故,不仅仅救了福宁殿中数十名宫人,也帮了向太后一个大忙。

否则这桩连太上皇在内总共四人枉死的大案,就是大索宫城,掘地三尺也要将罪人给挖出来。在对死因没有基本认识的情况下,抓出来的只会是替罪羊。

如何骗得了有见识的人,到时候,外界少不得会乱猜测,嫌疑最大的向太后岂能脱得了身?

其他人都要感谢韩冈,只有赵煦是最该恨他的。

但韩冈偏偏不想以这个罪名将赵煦弄下台,究其因,不过是不违本心这四个字。

如果这件事放在千年后,没什么会责怪想为父亲尽孝的赵煦。纵然是做错了,但也只是个不幸的意外。若说有责任,周围的成年人,包括韩冈在内,他们的责任更重。除非愚昧无知之辈,谁也不会将责任推到一个三尺孩童身上。

韩冈的本心中也明白这一点,纵然世情与千年之后截然不同,韩冈也不可能附和世俗,觉得这是赵煦的罪过。只是事故而已。
