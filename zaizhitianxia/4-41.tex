\section{第九章 鼙鼓声喧贯中国(一)}

十月初的汴河,是一年中最为忙碌的时间,任何一座码头上,都停满了船只,不停有船只进来,也不停地有船只离去。再过一个月就要闭汴口,没有了水源,汴河在冬天只能靠着雪橇车运送些稀罕的什物,而大宗的货物,只有现在才能运送。

不过在汴河边,不仅仅有装货卸货的码头,还有一架架水车在随着水流而不停的转动。

水车之后,是一片用高达一丈的围墙圈起来区域。从围墙内部,当当的锤击声密如雨点一般,而且不止一道锤击声,而是随着水车的转动,有着十几道锤击声同时在响着,传进耳中时,都模糊了起来。

王雱用手呵着气,抬头望天,阴阴的快要下雨的样子。

想着又摇了摇头,这天气,再冷些就不是下雨了,而是要下雪,多半会是今年的第一场雪。

“时间过得还真快。”王雱对着身边的韩冈说道,“西贼已经没有余力,鄜延那边也准备好了。以现在的情况,多半在明年之前分出胜负。”

“西北二虏都是攻强于守,西贼如今进取无功,退守当然更不会有用。”

就在七月的时候,西夏国中终于对宋人攻取横山的图谋有了更进一步的反应。梁乙埋点集十二万大军,号称五十万,以仁多零丁为主帅,南下攻打秦凤路,希图牵制宋军在鄜延路的进攻。

西夏选择秦凤路作为突破口,也是因为在几年前的横山战事中,六盘山的蕃部没有受到太大的伤害,可以提供足够的补给,而不是帮着守军一起消耗宝贵的储粮。而且五月麦收、七月马肥,这时候南下,更可以因粮于敌,从宋人那边来补充消耗。

这一次的重点进攻,铁鹞子一路攻到了甘谷城和笼竿堡下,沿途有六座军寨被攻破,上千官兵战死。但党项人的攻势也到此为止。作为秦凤北方防线的核心枢纽,两座城寨不仅被修筑的坚实难摧,而且也集结了足够多的精兵强将。仅仅破掉了六个寨子,存粮加起来也不过十余万石,以十万人马来计算,这点收获一个月都支持不了。党项人的这番进攻是个赔本的生意。

同时在甘谷城的防卫战中,斩马刀和板甲大放光彩。集结成阵的宋军在用神臂弓激射过后,身着铁甲、手持陌刀,只一击,就将党项人引以为自豪的步跋子全数击溃。而来去如风的铁鹞子只要离得稍近,就有可能被密如飞蝗的箭矢射落马下,更是不敢接近宋军的阵列,只能任凭宋人耀武扬威。

这一仗打得党项人寒了胆,没能攻下驻军众多的城池,这是意料中事。但野战中也输得如此之惨,却是让他们难以接受这个事实。铁甲钢刀再加上重弩,这一套行头下来,放在西夏国中没有百十贯绝对做不到。可从甘谷城派出来出战的每一名禁军士兵,身上都是闪着新磨的银光。

大宋一向富庶,在党项人的眼中,就是一头肥羊。只是这些年来,肥羊的牙齿越来越锐利,每次南下,都少不了要吃些亏。而这一次南下,不但牙齿利了,连身上的皮都变成了硬甲,这不是吃亏,而是要要命了。

但党项人攻击宋人的城寨也不是没有倚仗,被攻破的几座寨子,兵力稀少是一方面的问题,但更重要的,是西夏有了出色的攻城战具:

他们竟然拉出了配重式的投石车!

过去党项不善攻城,许多器械都不知道该如何打造。其中有工匠的问题,而更多的也是他们很少见识到宋人的攻城器械,不知道从何仿效——几十年的战争下来,没有几仗是宋人围城。

但河州一战,霹雳炮上场的次数不少,当时党项人少不了有探子出没,也有一批不肯投降大宋的吐蕃蕃人,翻山去了西夏。那等结构简单的投石车,只消多看两眼,就能知道其中运作的原理。用上几年时间来仿制,一国之力还是不难的。

可是当这个消息传回来,朝堂上下却是一片哗然,连赵顼都坐不住了。

不是为了西夏,而是为了契丹。

现如今西夏国力衰退,整体的形势是宋人进攻,党项防守,西夏用得到攻城器械的机会并不多,可是契丹人却是能用得上。

既然契丹公主如今正做着西夏王后,霹雳砲对辽人来说绝对不再是秘密了。以南京道的汉人工匠们的手艺,他们仿制出威力更大、效率更高,甚至各方面都接近于宋人的投石车的可能性,恐怕是接近于百分之百。

另外还有飞船,原理天下人都知道,只要费些手工就够了。大宋这边的酒楼都能拿简化的产品出来打招牌、做广告,辽人多半能做得出来载人的飞船来。

所以韩冈就有了些麻烦。

不只一个人攻击韩冈求名心切,将国之利器泄露于外。只不过这些攻击,对韩冈来说仅是些小麻烦而已。

“契丹、西夏有没有弓?没有弩?没有甲胄?没有刀枪?他们都有,只要不如我大宋精良罢了。”韩冈当日在崇政殿上回答天子的疑问,“有了飞船和霹雳砲又如何,我们能让刀剑更为犀利,能让甲胄更为坚实,也可以做出更好的霹雳砲和飞船来,可以投得更远,在城中就把契丹的霹雳砲砸毁。可以飞得更高,直接在天空中用劲弩将契丹的飞船射落。大宋的精工名匠,只靠仿效是学不走的。”

韩冈根本就不将受到的弹劾放在心上,反正逼到最后,他将火炮拿出来就肯定能过关,根本就不需要有半点担心。而他这番近乎强词夺理的一番辩驳之后,赵顼就打了个圆场,让他将功抵过。

七月、八月、九月三个月里,军器监城内城外两个厂区,总共打造了五万三千套板甲,是过去两年的总产量,十倍于过往,这份功劳,就抵了韩冈所受到的罪名。

但韩冈不干了,他可以辞了这份功劳,但他绝不认罪。宣讲格物致知的道理若是成了罪名,日后还怎么推广他的学术?

韩冈的态度很是恶劣,不过王安石过去其实也做过这等事。

因为对一桩杀人案的判罚有不同的看法,当年正做着开封府推官的王安石与同僚争辩起来。而后经过朝堂公论,判了王安石输。按规矩,王安石应该为自己的错失上表请罪。可王安石就硬是不认罪,拗相公的脾气在那时候就已经崭露无遗,而最终的结果则是‘诏不问’,就这么算了。

韩冈现在为了推广气学,同样是梗着脖子不认罪,赵顼也拿他没办法。最后同样是诏不问,顺便将监察御史们的弹章一起留中,糊弄过去了。翁婿两人一个脾气,闹得世人都说他们不是一家人不进一家门。

对于这一场风波,赵顼心头也是有点不舒服。他都出面打圆场了,可韩冈还是一点面子也不给。

只是处置了韩冈又能怎么样?是能让契丹人不再打造霹雳砲,还是会让契丹人相信飞船上不了天。既然是挽回不了的局面,强要治罪韩冈,又有什么意义?不怕让天下人寒了心,不敢再献上自己发明的军器?

所以赵顼最终还是选择了支持韩冈。

这些事已经算是过去了,韩冈也的确依言辞了功劳,王雱无意再说及此事。韩冈愿意为了推广气学而付出这些代价,从王雱的立场上,也不便劝说。

军器监厂区的空气中,四出飘散着烟灰,这是焦炭大量使用的结果。王雱来得多了,也不以为意。看着一片空地上正在堆着砖石土料的工匠,王雱笑道:“这是第三座炉子了,不会再倒吧?”

韩冈无奈的苦笑着:“想来应该不会了。”

炼铁炉此时多得很,但韩冈要造的炼铁高炉有些贪大求全,最近已经倒了两座,又成了他的一个罪名。不过现在在建第三座比前两座足足矮了一半,只比正常的炼铁炉大了一圈而已,多半就能成功了。

看得出韩冈有些尴尬,王雱又笑道:“不过现在京中官宦人家都开始用焦炭了,也是玉昆你的功劳。”

“这份功劳我可不敢要。”韩冈摇着头,“防都防不住啊!”

高炉尚未成功,不过在炼铁时用焦炭倒是有了些成效。矿石、石灰、焦炭一起炼出来的生铁并不输木炭多少,同时炼焦后的产物煤焦油也成了猛火油的原料之一,让韩冈有了充分的理由来推广炼焦工艺。

不过也是盛名所累,韩冈如今在百工上的名气越来越大,已经超越了他在医疗领域的名声,所以等着偷学他的发明,用来赚钱谋利的人数不胜数。焦炭一出,多少人都开始试用。用了之后就发现,焦炭比起煤炭更为耐烧,且少了烟气。所以如今汴京附近,就一下多了许多做烧制焦炭的炉窑,一干大户人家都开始在生活中使用焦炭。

“就是传得也太快了,就跟轨道一样,才多少日子,就遍地都是了。”

“这样其实也好。一人之智不及众人之智,等焦炭的使用遍及天下,肯定能会有更好的炼焦手段出来。”

王雱就很难理解韩冈的想法,完全不把技术扩散当成一回事,反而是盼着有人学过去,“给西贼北虏学了去终究不是好事。”

“学去了又如何?能跟大宋比产量不成?”韩冈哈哈笑道,“何况只是些看了就能学会的东西,本来就保不了密的。简单的东西好学,工艺复杂的手艺,可没一样流传出去。似是而神非!”

