\section{第15章 焰上云霄思逐寇(19)}

“好一条汉子!”

“好一名猛将!”

“武勇只比李都监稍逊。”

城上城下都在暗暗赞着在乱军中,挥舞着巨斧收割着性命的猛将。而就在这名猛将身后,同样手持大斧的战士们,也同样在战场上恣意砍杀着混乱中的敌军。昆仑关关城的大门敞开着,从中一队队手持钢刀大斧的官军冲了出来。

整整两个指挥的兵力,是从桂州赶来的援军。自章惇将他们派出来后,这一部人马一路疾行南下。同样先是乘船而行,然后换了徒步行军,就在昨天午后时抵达宾州,到了入夜时分,就进入了昆仑关城。

用了一夜的时间在关城中休息。一觉好睡之后,望着关外的无数战功,他们已经难以忍耐。当韩冈下令开城出战,他们爆发出来的冲击,如同虎兕出柙,让交趾军无可抵挡。

攻到城下的交趾军已经彻底溃散,在如狼似虎的宋军面前被杀的落花流水。正要攻上昆仑关头的时候,偏偏从城内冲出来一批生力军。而且都是穿着红衣的宋国官军,这让一心以为关城中只剩少数广源军的交趾士兵,心中都慌乱了起来。

还有人心存侥幸,几名交趾军官大喊着这是广源人假扮的宋国官军,后方也匆匆派上了一队援军过来。可是当两个指挥的荆南军从猛冲猛杀中恢复秩序,在城下组成了阵列,他们的身份无可置疑。

关城前血流成河,方才冲在最前面、想要抢着率先登城功劳的交趾军中勇士,都被堵死了退路,一个都没能逃回去。一地的残肢断臂,还有无数仍在抽动的尸块,短短半刻钟的时间,关城之下就被清理干净。血水在官道上肆意流淌,就像前几日的雨水一般,只是换做了一片鲜红。

严整的军阵随即顺着官道压了下去,大斧一起一落,就是一条人命被带走。沉重的大斧挥砍起来猛恶无比,就算穿着甲胄、举着盾牌都无法承受住自上而下的猛力一击。劈开头颅、肩膀,砍下四肢、腰肋,当百十柄大斧同时挥下来的时候,站在阵前的士卒当即粉碎,交趾军的任何抵抗都显得徒劳。

刀斧如林,缓缓而行的军阵如同一具石碾,将前方的敌人碾平碾碎,沉默却有整齐划一的挥斧前进,并不像方才在敌军中冲杀那般让人热血沸腾。但这样的攻击,却能让每一名敌军心都冷了起来,失去了反击的战意。

鸣金声在交趾军阵后方响了起来,在宋军猛烈的攻势下,李常杰也只能选择暂避锋芒。宋军则是紧追不舍,从后不断砍杀着落队的敌军,追着他们追到了关城一里外的由鹿角、拒马组成的防线处。

李常杰所设立的防线一道一道,一直延伸到后方小石坡的营地去。这是交趾军缺乏安全感的象征,也是他们对官军感到畏惧的证明。不过就是靠着这一条条防线,交趾军坚守在拒马、鹿角之后,抵抗着宋军的猛烈攻击。

一柄柄大斧重重的劈砍着鹿角,手腕粗细的木料层层扎起的障碍,要比砍人更难。而躲在栅栏后,拼了命的拉弓攒射的交趾人,给了官军带了不小的伤亡,让他们清楚障碍的行动又变得更为艰难起来。

这样的情况下,要带着弓弩手射散了敌军才好下手。不需要韩冈在后面下令,后排的宋军上来了,拿着神臂弓的他们开始与敌军对射着,不过交趾弓手有着厚重的木盾作为防御,神臂弓的效果减弱了许多。

韩冈就在城头上观战,眯起眼睛看着自己的兵在交趾军的防线处受到了激烈的抵抗。看见自己的兵中箭倒地,眉头皱了起来。偏过头,叫着身后的一名将领:“黄全!”

“小人在!”黄全中气十足的吼着,用力踏前一步,双手抱拳行礼。前方战事胶着,官军一时打不开局面,这时候就终于轮到他出场了。

“会说交趾话吗?”

“……会。”黄全闻之楞然,疑惑着,“不知运使有何吩咐?”

“再点十几个嗓门大一点、同样会说交趾土话的。”韩冈吩咐着,“本官拿他们有用。”

“啊……是,小人遵命!”黄全立刻吩咐了亲兵,让他们将合适的人选带上来。交趾土话,广源州人大半都会说,大嗓门的也不少。韩冈要的人,很快就上了城头。

高矮胖瘦各不相同的广源兵在韩冈面前排成一排,黄全向韩冈缴令:“小人已将人都带了上来,还请运使差遣。”

“你做通译,让他们喊出去。”韩冈简洁无比的吩咐了一句,不待黄全醒过神来,转身对着鏖战中的战场径自喝道:“李常杰!”

“李常杰!”黄全呆了一下,就立刻用交趾土话喊着敌军主帅的名字紧随着他,就是十几名大嗓门的士兵同声大吼,将韩冈的话传到战场上的每一个角落,甚至压下了战场上的厮杀声:“李常杰!~~~”

不论是交趾士兵,还是大宋官军,听到这片连天接地的吼声,手都缓了下来。被叫到名字的正主,也抬起头来,远远的望着关城上。

“三千官军已至昆仑关。”韩冈说着。

“三千官军已至昆仑关!”依然是黄全翻译,士兵们大吼。

对宋军畏惧已深的交趾兵面色如土,李常杰眼皮一跳,冷喝一声:“真有三千,这时就该杀出来了!”

韩冈继续让人传话:“穿越山林、偷袭宾州的七百兵已经全军覆没!”

李常杰咬着牙:“胡说八道!”

几十面交趾战旗这时倒挂在昆仑关的城头上,就像一记记耳光打在还在强辩的李常杰的脸上。

“前日横州,你又有七千大军被我官军击溃,主将李玢已然授首!”

李常杰心腹爱将的名字,交趾军中人人知晓。而且为了提振士气,李常杰也没有阻止麾下将校,向士兵们透露李玢领军偷袭宋军后方的消息。胜利的希望都在李玢身上,哪一名交趾士兵不讲这个名字一天念上三遍。当宋军将人名报出来的时候,就如同晴天霹雳打在他们的天灵盖上。也就在同时,一枚枚首级在城头上吊了起来,从左到右,粗粗一数,就有几百颗头颅。

李常杰脸色煞白,他身前身后士兵纷纷回头望着他的脸。

“你已日暮途穷,速速归降……”韩冈停了一瞬,运足中气,一声怒喝:“可免一死!”

“速速归降,可免一死!”

“速速归降,可免一死!!”

“速速归降,可免一死!!!”

最后的吼声在山中回荡,交趾士兵们木然呆愣,李常杰在马背上晃了一晃,猛然一口鲜血喷了出来。

他紧紧抓着马鞍,努力不让自己昏倒。对着失魂落魄的围上来的部将,他勉强的开口,“撤军!”

韩冈停了下来,黄全当即上来追问:“运使,爹爹他们当真赢了?!”

“不,我是在说谎。”韩冈回答得干脆利落,让黄全变得一脸呆相。

韩冈的确没有收到捷报,但一面面交趾战旗在关头上倒挂起来,一颗颗首级也吊了起来,要想骗一骗军心不稳的交趾人则已经足够了。

李信和黄金满与交趾李玢所部已经交上了手,不过第一次交锋只是小胜而已,并没有歼灭敌军,反而让他们退到一座村庄中。

不过从俘虏口中,李信三人已经掌握了率领那一支偏师大体情况,尤其是交趾将领的姓名,传到了韩冈的手中。加上此前在几次胜利的战斗中所缴获的一面面战旗和人头,足以进一步动摇敌军军心。

宋军从昆仑关冲出来的时候,交趾军就已经动摇了。当韩冈让人喊破了李常杰的图谋,戳破了最后的希望,就连李常杰本人都压不下军心浮动的军队。

一方士气大涨,一方士气大落,结果就已经注定。韩冈的谎言成了压倒骆驼的最后一根稻草,已经慌乱不堪的交趾兵连最后的一点希望都给抹杀了,哪里还有奋力一战的胆气?!

李常杰已经决定了撤退,留下来断后的一支部队由他的心腹指挥。而留在最前沿的交趾士兵连拉弓射箭都乱做了一团,后方传来的动静也让他们无心再做抵抗。趁此良机,宋军连连出手。连接一段段拒马的绳索被砍断,分散开来的路障立刻就被扯到了路边。宋军一方勇不可挡,交趾人的几道防线一道道的被他们突破。

“黄全!”韩冈叫着部将的名字。

“小人在!”黄金满的儿子这一次很确定,韩冈点他的名,不会是让他找声音大的士兵。

“你带一千兵出城,过了小石坡后,下面就由你打头阵,到了大央岭就停下来,要提防贼军的反击!”

黄全重重的一抱拳:“小人遵命!”如风一般转过身,下了城头,带兵出击。

终于赢了!韩冈想着,多亏了李常杰的贪心,以一千五百精兵破了十万,这个胜利可算是辉煌了,也算是报了几分钦州、廉州和邕州的血仇。

不过真正的战争才开始,富良江畔的升龙府才是最后的决战地。

血海深仇,要交趾人十倍相还!

