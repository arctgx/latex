\section{第12章 共道佳节早(八)}

当天夜里,王韶、王厚各自从宫中回来。就问起今天韩冈赴约的事。

当王厚听到韩冈请他上门回复当初王安石的提亲,顿时拍案而起,厉声问道:“玉昆,是不是王家逼你的?!”

韩冈坐在座位上,气定神闲,反问着:“如果是王相公家的两个衙内真的来逼我,处道你以为我是会被迫答应下来呢?还是一口拒绝掉?”

王厚讪讪的坐了下来,韩冈的脾气他怎么可能不清楚,只是一会之后就改弦更张,韩冈什么时候变得这么好说话了?

王韶也同样有些想不通,遂问道:“玉昆,这究竟是怎么回事?”

“只是谈了一阵话而已。今日清风楼之约,若是王家以势压人,韩冈肯定是不会再理会他们。但好言相商那就没办法了,韩冈也不是不近人情之人。耽误了王家二小娘子近一年,心里也是过意不去。”

韩刚并不是故意要隐瞒,可他也不能说是王家的二女儿找上门来,才改变了主意。如果事成,王家的二小娘子就是自己的妻室,韩冈怎么能看着她的名声坏了?即便王韶、王厚不是口疏之人,但自家的事,还是留在自家心里比较好。

王韶不出意料的误会了,哈哈大笑,笑得极是欢畅:“想不到王元泽他也有低头的一天。”他边笑着,边对韩冈和王厚道:“在经筵上,王元泽可是口舌便给,丝毫不肯饶人的,比王相公都厉害几分。吴冲几次给他逼得下不了台。”

一眼瞥到韩冈欲言又止,王韶笑容微收:“玉昆放心,此事不会对外说的,更不会问那王元泽……王家的大衙内,心胸可没有多广。”

“大人明天就去相府?”王厚问着。

“即是玉昆的嘱托,就当尽快了。”王韶点点头,“想不到我这女方的媒人做过了,男方的媒人还要做上一次。看着眼下的情况,玉昆你的家长,还要我代理一下呢……”

每一科的进士,有许多就是在黄榜下被拉去做女婿的。也没有什么媒妁之言,更没有父母之命,直接看了嫁妆后,就进了洞房。哪家招女婿的敢拖延时间,放着抢到手的女婿回乡去禀明父母?

近的倒也罢了,那些福建、两广进士,隔着千山万水,还不知拖到什么时候。基本上都是找个有身份的高官来代理。

韩冈的情况也是类似,不可能让自家的父母赶来京城。王安石更不可能让自己的儿子送了女儿去陇西成亲。只会是先在京中办了婚礼,然后夫妻一起回乡再见父母。这样的情况下,少不得要劳烦王韶。

此事在家中商定,第二天王韶便上了王安石的家门。以枢密副使的身份访问相府,王韶还是第一次。见到西府的副职一队人马过来,王安石家门前求见的官员都纷纷避让。

过去王韶来王安石府上拜访,都是在正门旁的偏门被领进去,而今天他的身份已经不一样了。名字刚刚报进去不久,王安石家钉着数排铜钉的正门便吱呀呀的打开,王雱和王旁两兄弟联袂迎了出来。

王雱兄弟都是笑意盈盈,知道王韶今日所来为何,老早就在等着他。打躬作揖,将王韶从正门迎进家中。

正厅中,王安石降阶相迎。女儿的婚事,王韶居中奔走。王安石也算是欠了他一份大人情,此前的一些龃龉和不快,在这份人情前,都如纸屑,被风吹了个一干二净。

……………………

虽然并没有对外宣扬,但消息还是很快就传出去了。毕竟王韶上门拜访王安石的事,怎么都不能瞒人的。枢密副使和宰相私下里的交流,必然要引动皇城司的神经。

就在王安石和王韶将韩冈王旖两人的婚事敲定的第二天,崇政殿议事后,赵顼就留下了王韶。

赵顼当然不可能直接询问昨天的事,而是先问着公务:“王卿,熙河路经略使的位置该定下来了,不知你有何想法?”

“此事全凭陛下处断,又或与中书商议。臣乃枢密副使,此事岂可插言?”

不出意料,王韶不肯直接回话。决定边地守臣,是中书的权力,而不是枢密院的权力。但赵顼知道,说起对熙河路的关心,王韶是在朝中的任何人之上。

“不知沈起此人如何?”

赵顼这回问的是王韶对人的评价,这样他回话就不需要避忌了,“沈起为秦帅,治兵严谨,数有功勋。臣观其人有班超、马援之志,非是因循苟且之辈。”

王韶虽然是在夸奖沈起,但赵顼哪能听不明白,暗里分明是在说沈起好大喜功,担任熙河经略使后,必然会掀起风波。

“那蔡延庆又任何?”

“蔡延庆自执掌秦凤转运司后,熙州、河州两战多得其力。臣能后顾无忧,延庆之功也。”

‘后顾无忧吗?’赵顼已经明了王韶的倾向了,而说的也正合他的心意。

“关于梅山之事,王卿有何想法?”

赵顼继续询问,王韶向天子说着自己的意见。一通公事问对之后,赵顼歇了口气,让人送上茶汤来,一口口的啜着。

天子故作漫不经意的问着王韶:“听说昨日卿家去了丞相家中拜访,不是有何公务要商议?”

王韶今日被留对,王安石却没被留下来,就是知道天子必定要问起昨日拜访相府之事。前面絮絮的说了一通废话,最后终于问到了正题上。

王韶的心中藏着火气,想着什么时候找个借口,将在他家门口窥视的皇城司暗探杖责一顿,省得他们太肆无忌惮。

不过发狠归发狠,天子的问话,还要尽快回答:“昨日拜访王安石,倒不是为了公事,而是一桩喜事。”

“喜事?”赵顼记忆力不差,还能记得王安石家有个女儿云英未嫁,“是为卿家的哪个儿子……”

赵顼说到一半,就停了口,自己都摇起了头。世间虽说不是没有老子为儿子登门求亲的事,但正常士大夫间的提亲,不是请身份合当的媒人,就是亲笔写封信请人送过去。

何况王韶这样的身份,也绝不可能跟王安石做亲家。枢密使都已经是宰相的亲家了,再添个枢密副使跟宰相联姻。把他的朝堂当成什么了?

赵顼很快就想通了:“对了,韩冈就住在卿家家中。”

前日从皇城司这边听说了韩冈已经上京,赵顼还问了王韶,想将他召进宫来问对。只是在王韶口中知道韩冈的心意之后,方才作罢。

询问的目光投向王韶。赵顼的枢密副使点头:“臣正是为了韩冈而去,昨日已经与王安石的二小娘子定下了亲事。”

“韩冈算是难得的年轻才俊,同一辈中少有人能及,卿家怎么就没有捷足先登?”

赵顼对如今进士的行情了解得很清楚,而韩冈更是远远胜过一个进士了。心中便有了点疑问,王韶怎么不招韩冈做女婿?

“臣的女儿年幼,定亲又早,不过也曾经将外侄女许配给韩冈。只是臣那外侄女福薄,前岁因一场时疫而病夭了。臣家没有更合适的人选,便一直都没有再与韩冈定亲。而韩冈前两次上京,都曾到丞相府上拜访,深得其看重。前次臣上京,安石就托了臣为他的女儿提亲。现在韩冈上京,就正好给了回复。”

王韶删繁就简的说了一通,在天子面前的奏对,没必要说得太详细。

赵顼听后点了点头,道了声原来如此。他担心的是重臣之间的勾结,将他这个天子架空的危险。像王韶这样仅仅帮着传句话,就没什么好担心的了。

“韩冈屡立功勋,却始终谦益自守,大有古人之风,世所罕有。而王丞相更是为国劳苦。韩冈与其女大婚之时,朕也当随一份厚礼才是。”

赵顼不会放过施恩的机会,而王韶为天子的这份恩德赞颂不已。

等着王韶告退离开,崇政殿中宰辅们走得一干二净。赵顼起身向殿后走,示意修起居注的吕惠卿并不需要再跟过来。

赵顼沉吟着,行走在通往后宫的廊道中。唐代的皇帝即便是在宫廷中,许多时候都要乘着肩舆。但宋代的天子,只要在宫中,不管去那个地方,都使用着自家的双脚走的。

走了很长一段,赵顼突然叹起:“想不到韩冈连宰相女婿都做了。王相公也是心急,怎么不等到发榜后呢?”

今日当值,跟着在赵顼的身后的管勾皇城司石得一,弓身回着天子的询问:“王家的二小娘子也有十九了,腊月一过就是二十。她的婚事拖到了这个岁数,想必王相公心里比谁都急。现在能找到韩冈这个女婿,哪还肯等到明年发榜之后?”

“原来是这样啊……”赵顼的疑问被解释,但新的问题跟着又出现了,“是不是王安石的女儿有何恶疾,或是别的什么,怎么到了快二十还没有出嫁?”

赵顼并不是关心王安石的女儿。如若王安石的次女生有恶疾,而韩冈还愿意娶过去,那他的人品就值得怀疑了。这可不是对聘妻不离不弃的德行,而是趋炎附势的卑下。

“这倒没听说。”石得一摇摇头,虽然关于王家的二女儿许久不嫁,外界的确是有些谣言在传着,但不论哪一个都不靠谱。这种没有根据的传言当然不能跟天子说,因此得罪了宰相,他可就危险了。

想了想,他又补充了一句,“逢年过节,也能看到丞相家的吴夫人带着女儿去大相国寺上香,不见有何病症。”

“这样就好。”赵顼点头,又是一声,“这样就好!”

