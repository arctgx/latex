\section{第43章 竹纸知何物(中)}

靠着韩冈在蕃人中的威望,顺丰行这些年赚的钱不少,冯从义都有了十几万贯身家。加上棉布作坊,韩家逐渐积累的财富,也足以支撑得起韩冈和李信两个在外为官的子弟大手大脚的花销。

尤其是李信,他升官比韩冈还要快,连爵位都有了,正是韩冈推拒的开国县男一爵。文官要由一般得做到正六品的少卿监一级,而武将则是在从七品的宫苑诸司副使开始,便有了封爵。李信现在正好是宫苑诸司副使中最末一位的供备库副使,便有了爵位。

李信在荆南升得如此之快,主要也是靠得军功。他是章惇手下与刘仲武其名的头号得用的大将,每次冲杀在前,立得功劳也是数一数二。一名武将,如果在一场大战中占了首功,直接就是七转三官,一跳数级。其晋升之速,文官怎么都比不了。自从九品的小使臣,到从七品宫苑副使,李信就只用了三年。

韩冈希望李信能在军中继续高歌猛进,所以不想他因为在经济上犯下什么过错。另外王舜臣、赵隆他们这些亲近友人,韩冈都有资助。

他会赚钱,也更会花钱。

“说起信表哥,前些日子舅舅来信说,信表哥在荆南纳了两名小妾,其中一个已经有了身孕,这下舅舅可以放心了。”

韩冈点点头,李信也给他写的信上提到过此事,不过也不是什么大事,到时记得要送礼就是了。他问道,“舅舅的身体可还好?”

“舅舅身子硬朗得很,老封翁做着。现在凤翔城中,哪个不敬他?过凤翔的时候,小弟还特地绕去州城见了一面,将姨父姨母的礼物送了过去。一直在说想着搬去陇西,就是要守着外公的坟茔,不好搬。”

说起坟墓,韩冈想起了一事:“四姨的坟去看过了吧?墓土有没有损坏?”

“没有,没有,”冯从义摇着头,“舅舅一直在盯着,也坟茔和墓碑都重修了一边。”

“你那三位兄长现在怎么样了?”当年离开凤翔府之后,韩冈就没再问过被他送进大狱里的冯家三子,想来不被敲骨伐髓是不会被人从大狱里放出来的。

“娘亲的遗骸仵作查验过了,没有毒斑和外伤。所以前两年,小弟就买了百来亩地,让他们守着爹的坟。”冯从义吞吞吐吐的说着,生怕引起韩冈不快。

“做得对。”韩冈却点头,“再怎么说都是你的兄长。四姨的事既然与他们没有关系,也不必赶尽杀绝,留条后路也是好事。”

得了韩冈认同,冯从义放松下来,感激的说着:“也多亏了表哥,否则小弟也不会有今天。”

“你都给赶出家门了,做哥哥难道能坐视?一家人别说两家话。”

冯从义重重的点着头,感叹了几声,放下了过去的心结。转过话头,像是想起了什么:“对了,小弟前天从洛阳出来的时候,正遇上郭相公,不过没敢上去搭话。看着他急着往东京赶,难不成是要调职了?”

“是郭逵?”见到冯从义点头,韩冈说道:“郭逵是要调去太原府的。虽然已经割了地,让契丹人满意了。但还是要防着他们谋图不轨,再起事端。有了郭逵坐镇太原,开封这边才能安心下来。不仅如此,种谔也要回鄜延路了,盯着党项人。”

“难道这一次当真割了七百里地?!”冯从义随即凑近了一点,低声的问着。

“从代州往南七百里,差不多都快到黄河边上了。你说有没有七百里?”韩冈笑着反问。

“果然。”冯从义一拍手,“俺就说不可能吧。还跟林家的四哥打了赌,赌了一坛五十斤的烧刀子。”

“恐怕你要输。”韩冈笑着,笑容冰冷:“其实要看这七百里是怎么算的了。虽然国界只是向南后退了数里,退到了分水岭上。但宋辽两国边界绵长,如果计算土地面积,也的确有七百里了。”

冯从义点点头:“如果只是这个七百里,倒还算好,输了就输了吧。”

“还好?!”韩冈脸上怒容顿显:“国土不可让人,此事连匈奴人都知道。契丹一句讹诈就得了七百里土地,此乃我等朝臣之辱。”

冯从义被吓了一跳,看着韩冈,小心翼翼的问道:“表哥弃了中书,反而去军器监,是否有这个心思在?”

韩冈叹了口气:“也有此一因。”他笑了笑,“明天就要去军器监上任,就不知军器监中的大小官吏给我准备了什么接风宴。”

……………………

军器监衙门设在旧城右军第一厢的兴国坊,

从前朝后周时开始,位于皇城左近的兴国坊,就是为禁军打造军器的所在。坊区如今分为东西二作坊,下设五十一作。如火药作、青窑作、猛火油作、金作、火作、大小木作、大小炉作、皮作、麻作、窑子作等等。用后世的话说,就是集团公司下面分成两个分公司,下面再设五十一个工厂,各自负责不同军器装备和零部件的制造。

“舍人的霹雳砲、雪橇车,主体的架子就是分别出自大小木作,铁钉出自金作,绳索出自打绳作,上漆有漆作,装饰有画作。”军器监丞白彰,领着韩冈在兴国坊的巷道中走过,周围的一座座院落中,斧锯刨磨之声不绝于耳,必须得大着嗓门才能听见彼此的说话。

听了白彰的介绍,韩冈觉得这是应该算是分工合作了,一个个车间生产不同的零部件,然后再加以组装起来。

“这么多作坊参与其中,制作的军器不会有什么差错?”

“就为了能让天下兵甲犀利精良,所以才有了如今的军器监,如何会有差错?”白彰自豪的说着,“过去还没有设立军器监,东西二作坊还属于三司胄案的时候,刀剑锋芒极脆,弓弩一张便折。但自从吕大参和曾学士开始掌管军器监,只用了一年,便皆以完备。”

白彰忽然停步,指了指左手边的一座大院,叮叮当当的捶打声从里面不断的传出来,门前一圈禁军守卫,看守森严,“这是斩马刀局,专一制造斩马大刀。如今关西边军,用得大刀正是此中所造。”

韩冈随着白彰走进去,看着他从匠人手上刚刚打造好的大刀。沉甸甸的刀身,有着三尺许的刀锋,一尺长的刀柄,柄下镶有铁环。双手握着轻轻一挥,便呼啸作声。白彰将刀拿给韩冈看,“当真能将马也斩下来。”

韩冈对章惇说过他要萧规曹随,但并不代表他会将监中之事一概置之不理,总要看一看,瞧一瞧,若真的有不对的地方,心中也得有个数。

不过走了一圈之后,韩冈当真有些佩服起吕惠卿了,能将军器监上上下下安排得井井有条,难怪能在短时间内就打造出质地优良的军器来,让他想挑刺都难。而且看着白彰说起话来,对自己现在的工作充满自豪的态度,即便自己想对军器监的制度有所改进,恐怕也不是那么容易的事,必然会受到极大的阻力。

回到衙门中,曾孝宽正慢吞吞的喝着茶。

他虽然也是判军器监,但主要工作还是在枢密院。曾孝宽正担任着枢密院都承旨一职,很快就要升为枢密院直学士了——这也是因为他主管新法中的保甲法一事。不比韩冈是专任军器监。虽然从排序上他要压过韩冈,但实际主持监中工作,还是得韩冈来。

见到韩冈近来,他笑问道:“玉昆,如何?”

“参政和都承于监中所立种种,让韩冈无所更易,当可坐享其成了。”

曾孝宽呵呵笑道:“吕吉甫尚在军器监的时候,就已经开始编修《军器法式》,作为军器制造的标准。如今已经修订出一百一十卷,《辨材》一卷、《军器》七十四卷、《什物》二十一卷、《杂物》四卷、《添修》及《制造弓弩式》十卷。玉昆若有闲暇,可以拿来一观,只是决不能外传。”

“这是自然。”韩冈点点头,转身对着罗列在堂下的一众衙中属僚道:“监中制度一切如旧,望尔等勤勤谨谨,循之如初。”

白彰领着下拜。曾孝宽微微而笑,而韩冈也在笑。

接下来一段时间,韩冈的确什么都没有干涉,每天上朝之后,就按时去军器监上班,到傍晚在按时下班,平平静静的行动,让许多想看好戏的人大感失望。

只是吕惠卿素知韩冈的为人心性,知道他此时的沉寂,必然代表着他准备一鸣惊人。所以军器监那里越是没有动静,吕惠卿心中就越是没有底。他现在正想着该如何对付冯京,绝不会希望此时身后起火。

“鸷鸟将击,卑飞敛翼;猛兽将搏,弭耳俯伏。韩玉昆所谋非小。”

“韩冈如今仅仅是逐日督作,吉甫何必心忧如此。若真有动静,再做理会不迟。”

章惇在吕惠卿面前虽是这么说,但心中却为着韩冈担心。韩冈不与吕惠卿过不去,一点也没有动静,这对吕惠卿是好事,但韩冈本人就不好办了,天子正等着他的回报。

