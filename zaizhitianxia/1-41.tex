\section{第19章 城门相送辙痕远(下)}

根本没有停下来等王厚消息的意思,韩冈很快的收拾完毕。拉车的骡子早已喂饱了草料,按照与王厚的约定,韩冈留下三名伤员,以及一辆装着缴获武器和首级的骡车。他并不担心有人会趁他不在侵夺这些战利品,有王韶的儿子关照,没人敢吞没他的功劳。再说,伏羌城中除了王厚以外,也没几人会知道他在营地内留下了这些战利品。

几声响鞭过后,辎重车队随即离开了营地。韩冈的启程没有惊动到其他人,一行车队离营后,就沿着城中大道向北行去。今天是最后一程,总计六十里路。沿着甘谷水【散渡河】向北,三十里到安远寨,再三十里就抵达了甘谷城。

虽然甘谷如今局势不稳,但到安远寨的前半程不会有问题。可以先赶到安远寨,再确定行止。若甘谷城破,那就不怨他的事,若是没破,就设法送进去。无论如何,伏羌城都是留不得的。昨日韩冈他已经把话说出来了,今天再改口,不去甘谷城,等于是给向宝一把刀,让他来捅自己。向宝也不须亲自动手,只要呶呶嘴,包管有一票小人冲上来,让他韩冈生不如死,或干脆就丢掉性命。

王厚倒底是把他父亲王韶找来了。当车队抵达伏羌城北门处的时候,父子两人加上几个护卫就在那里守着了。

“是王机宜!”赵隆压低声音兴奋的对韩冈说道,他守着城门,王韶的模样再熟悉不过。

“真的?!”王舜臣的心情也高昂起来。想不到王厚真的将他老子拖了过来,看来韩三秀才真的能得到抬举了。

“嗯,我看到了。”韩冈的声音平稳如常,见着王厚跟在其人身后,他在赵隆说话前就已经确认王韶的身份。

第一眼看到王韶,韩冈就知道秦凤路机宜绝不像他儿子那般好蒙骗。黑瘦的面颊上,有风刀霜剑留下的痕迹。平直的双眉下,是一对看透人心世情的眼睛。他的眼神没有多少侵略性和压迫感,却凝定如坚石。以韩冈前世的经验,拥有如此眼神的人,是极难被言语所动摇,不必在这样的人身上浪费口水和时间。

“学生韩冈拜见机宜。”

来到王韶身前,韩冈恭声行礼,神色如一,就像见到了一个普通的上官,弯下腰不过是尽到礼节。韩冈很清楚,遇上王韶这样的老江湖,最好的策略就是本本分分行事,把该做的做好。

王韶身材并不高大,当韩冈直起腰的时候,王韶还得抬头看他。但就算不计入经略司机宜的身份,王韶散发出来的存在感也绝不在韩冈之下。

王韶负手而立,看不出任何情绪,但他摆出的这个姿态,本身已经说明了很多事情。韩冈目光闪动,心知今日是不可能听到王韶招揽他的一言半句,让他所精心准备的义辞高官、坚往甘谷的剧本,大义凛然、以国事为重的表演,完全失去了登场的机会……

……既然如此,那就退而求其次,让王韶帮自己解决一些头疼的问题——充分将资源利用也是韩冈一贯的坚持。

韩冈斯文挺拔的外形很能给人以好感,可王韶从来都不是以貌取人的性格。他无意多做浪费时间的寒暄,直接令韩冈说出他最关心的事情:“昨日裴峡中一战的前后,你原原本本的说来给我听。”

韩冈的表情几乎是王韶的翻版,面上平静无波,眼中的锋芒深深敛起。他将昨日一战用平实朴素的语言描述了一遍,不像普通文人那样喜欢加入夸张的修饰。也没有增添进去自己的感想和推测,更没有半句自吹自擂,完全忠实于实际。若是说有什么歪曲的地方,就是韩冈将自己的功劳推给了王舜臣和民伕们许多。不过,有些地方他故意漏过了一些关键,但韩冈深信王韶能看得出来。

不出韩冈意料,王韶显然对军事了解很深。一眼就发现了韩冈故意漏话而出现的破绽:“裴峡谷中多有草木,支谷众多。来袭的贼子只有百多人,很容易就能隐藏起来。不是韩秀才你是怎么看出来的?何以刚进裴峡就加以防备?”

王韶正正问到关键点上,伏羌城以下的渭河谷地一直都在大宋军队的控制中,谁也不会想到会有蕃贼出没。但为什么韩冈在通过裴峡谷时,能提前提防?如果在行军中突然受到敌军突击,就算是能征惯战的老将也难以将手下的兵将及时整合起来反击,可随时保持警惕对行军速度影响也很大,一个三十多人的辎重队伍,在快速行进的同时,怎么可能有余闲盯着裴峡谷地中的各处能够隐藏的地方?

王韶在秦凤已经一年了,很清楚从秦州往北方各寨堡的辎重队的行进路程安排。昨日韩冈的车队能在未时前后进入裴峡,肯定是以全速前进,这样的情况下,百名蕃贼突然从山上杀出,不是事先有所准备,又或者韩冈的车队中有个有如字面意义的以一当百的勇将,全军覆没是必然的结局。

王韶的眼神在问话的同时一下锐利起来,盯着韩冈脸上的表情变化。

韩冈的演出没有半点破绽。他苦笑,有股子发自内心的无奈:“因为学生早在出秦州之前,就知道这一路并不好走。”

黄德用一案是被定性为西贼奸细妄图焚毁军器库。黄大瘤是陈举的亲信,此事秦州尽人皆知,可陈举用了几万贯钱钞就将黄大瘤跟自己的牵连斩断。不过有心人若想罗织罪名,要将陈举陷于万劫不复的境地,却并非难事。

韩冈很简洁的将陈举与自己的恩怨向王韶说了一通,然后将叙述的重点放在了陈举的势力和财力,“陈举父祖三代在成纪县衙之中,县中吏员皆为其爪牙,纵是朝廷任命的一县之主也难动其分毫。被陈举陷害而得罪的知县、主簿不在少数。他今次能轻轻松松就拿出数万贯来为自己脱罪,可见其人通过与蕃部回易,积攒了多少不义之财!”

一番话还没说完,王韶看似神色依旧,但他眼廓和嘴角的轻微变化已经映入韩冈的眼中。如何对症下药的编织语言、控制语调,让自己的话更为可信,是韩冈最为擅长的能力。而看人下菜牒,直接触动听众的内心,也是韩冈早已惯熟的手段。

王韶是经略司机宜,按说管不到秦州的内部事务,但裴峡谷一战后,通往前线的要道出了问题,王韶就有了充分的理由插手。权力无人嫌多,如果王韶能将陈举拍倒,主持瓜分那数十万贯家产,他在秦州官员中的影响力和威慑力必然会大大增强。王韶如何不心动?

将心中的得意藏在郑重严肃的表情下,韩冈总结道:“……黄德用不过一走狗,如何有胆去焚烧军器库。二十年间,成纪县三遭祝融,又岂是黄德用一人能做下。在成纪一手遮天的是陈举,有能力纵火的也只有陈举,跟蕃部交往紧密的更是唯有陈举一人。无意间坏了陈举的大事,学生才虽庸浅,也不至于看不到他对学生的杀心。以陈举的数十万贯身家,要想驱动一蕃部,又有何难?今次如不是学生有点运气,又提前从吴节判那里请了王军将随行,跟随学生的三十多人肯定一个也逃不出来。”

韩冈说完,便静静的等待王韶的发落。他知道王韶绝不会听信一家之言,回到秦州城后,必然还要调查一番。但陈举的命运已经确定了,是不是西贼奸细那是小事,他的几十万贯身家才是大事。如今韩冈递了把好刀给王韶,不信他对肥羊一般的陈举不动心。

王韶陷入沉思。他在秦州已有一载,陈举之名当然听说过。韩冈小小的一个衙前与陈举交恶后,还能快快活活的活到现在,当真是不简单,而韩冈与节判吴衍的关系也让王韶有了几分看重。如果他说的有一半是真的,就足以让陈举万劫不复。但韩冈的心机从他的那番话中已经看得很清楚,有了足够的利益,王韶并不介意给韩冈借刀杀人,但让他吃点苦头的心思,却也越发的重了起来。

并没有思考太多时间,王韶先对王厚说道:“二哥儿,你去韩秀才昨日的宿营里,把车里的首级和兵器都送到城衙去,验证确实后,为韩秀才请功。”

“孩儿遵命。”王厚茫然不知这是老子支开自己的手段,直以为王韶要最后验证一下韩冈一番言论的真实性。很兴奋的点头应下,冲韩冈使个眼色,领着两名护卫急急向城中去了。

王厚走远,王韶的目光从车队上一扫而过,道:“甘谷城急待支援,这批辎重不容有失。”

韩冈叉起手,正正经经的回覆道:“此学生职分所在,自会尽心完成。”

“你能这么想,没有白读圣贤书,”王韶赞了一句,抬头看了看旭日渐渐高起的天空,低下头来,有些漫不经意的催促韩冈:“天色不早了,再迟入夜前恐怕就赶不到甘谷城了。”

韩冈毫不犹豫地再一拱手应诺:“既如此,不劳机宜相送,学生告辞!”

自始至终,王韶都没有表现出半点要招揽韩冈的意思,反而催着上路,替韩冈高兴了半天的王舜臣和赵隆甚至愣愣的没有反应过来。只有韩冈的心情始终如一,回答得十分爽快。

没有投入希望,就不会有失望。既然王韶现在无意招揽他,那就继续做该做的事好了。能表现自己的机会有的是,能体现自己能力的地方也不难找,总有出头的时候。何必靠王韶?无论如何,韩冈都不会把希望寄托在别人身上,能让王韶对陈举起了心思就已经足够了。

没有怨愤,没有期待,韩冈按照正常的礼节向秦凤路经略安抚司管勾机宜文字行礼如仪,再与还发着愣的赵隆殷殷道别,便带着队伍洒然北去,并不回头。

太过洒脱的辞行,反让王韶看得皱眉不已。目送韩冈的车队沐浴着晨光缓缓远去,心中暗道自己是不是误会了韩秀才:‘是我看错了吗?’

ps:好事多磨,韩冈做官的道路并非坦途,事事如意从来未有,但最初的目标已经达成。一言毁人家,陈举族灭可期。

今天第二更,求红票,收藏。

