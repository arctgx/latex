\section{第三章 岂得圣手扶炎宋(中)}

“太后何在?天子何在?可是被尔等逆贼害了?”

韩冈在殿上旁若无人的怒吼着。

“太后与延安郡王自安然无恙,韩冈你何以胡言乱语?”

行了。

不论太后和皇帝两人到底是死是活,韩冈要的就是这一句。

蔡确参与了对赵煦的拥立,而且是主导者之一,他绝不可能否定赵煦的天子身份。

而赵煦既然是天子,那么赵颢想要他儿子接位,要么直接弄死赵煦,要么则是废立。

在事前的密谋中,蔡确绝不会同意弄死赵煦,然后让赵孝骞顺理成章的即位,宋用臣、石得一也不可能答应。已经有了拥立之功的内臣、外臣都绝不会参与其中。

废昏主犹是忠臣之为,而弑君就没有任何解释的余地——弑父如此,弑君亦如此。换上来的皇帝,日后也不会容忍。政敌更是会拿来做武器。杀了魏帝曹髦的成济,究竟是什么结果?

不论赵煦和向太后是几天后因伤心和悔恨而病死,还是被锁在深宫几十年。宫中日后的一切,都是由高滔滔和赵颢来负责。而在这之前,太后和小皇帝都必须还活着。

从情理中推测,很容易得到结论,但只有当事人亲口承认,才能让周围的人安心。

“蔡确。谁给你那么大的胆子?!”

王安石的手指颤抖着,几乎指到了蔡确的鼻尖上。

蔡确神色不动。

这完全是败犬之吠,没见其他宰辅都没有出来?过去他要敬王安石的地位,但现在却不一定要了。

终于可以放下心来了。蔡确想着。

这不是顺理成章的继承。而是彻头彻尾的政变。若是韩冈等人都在宫外,闻说宫中有变,立刻就能火炮袭城,那时候宫内又有谁能挡得住?

幸好大祥后一日的朝会,在京的朝官都要入宫上殿。抓住了这个机会,让韩冈和王安石糊里糊涂的走进大庆殿中,还不是任凭搓扁捏圆?

太宗皇帝接位,便是这样做的。太祖暴毙,他先一步入宫即位,等到群臣来拜,君臣之分直接就定下来了。

蔡确便是如此准备。今天的大朝会,是最好的机会,也是仅有的机会。等到群臣入宫,他领头带着同僚们一拜。君臣之份既定,事情也就结束了。

王安石、韩冈纵是满心不甘,三五力士就能让他们无能为力。

李信、王厚等爪牙,也不敌过石得一手下的几千皇城司亲从。

皇城司控制着城门,大门不开,禁卫军中,就是有人想通风报信,除非变成苍蝇,还得能在冬天里飞。

只是还没到宣布胜利的时候。蔡确不敢冒险。在韩冈的背后,还有看似沉默,但绝不可能认输的王安石。有两人在,无论怎么讨论,结果都不会改变。

见蔡确没有反应,王安石就将目标转到了曾布的身上。两名宿直的重臣若不是已经参与进去,又怎么可能留在宫中,还安然上朝。

这时御史班中,一人闪了出来:“王安石!韩冈!尔等岂得渎乱朝仪,喧哗殿上?!”

韩冈看过去,却是刑恕。

也有他一个?

韩冈想着,又怒斥道:“谋朝篡位不喧哗,朝廷养我辈何用?倒是刑恕你,在程伯淳那里学到了什么?”

“恕惟知忠孝而已。”刑恕冷声道,“忠臣孝子,德配天地。弑父之君,便是汉废帝与商太宗也瞠乎其后。”

王安石怒声呵斥:“先帝崩阻,乃天子孝心之误。岂能与太甲、刘贺相提并论!”

韩冈此时暴怒如狂,心中却寒如冰雪。

不意一时的疏忽,就被人抓住了机会。

已是性命交关的时刻,现在半步也不能走错。

韩冈扫视着周围,殿中有上百名班直禁卫,还有钧容直的乐班。不过乐曲已经停了。

敢于上殿面见群臣,最差也已经能够指挥这些班直。而更重要的是,太后与天子还在他们的手中。正是手中有了足够的底牌,他们才敢大喇喇的坐上来。

如果自己坚持反对,高滔滔会不会直接让殿上的班直来扑杀自己?

不。韩冈立刻在心中否定。只要自己还没有表现住颠覆一切的势头,他们还不敢放手杀人。

上面有高滔滔,居中有蔡确……以及曾布和薛向。外面还有握有兵权的石得一,甚至有可能还有王中正——倒是张守约,他还在殿中,就在对面,他现在安是一脸的疑惑,以及愤怒——上下内外都齐了,所以才能成功。

“刑恕自束发受教,便习忠孝之道,不能奉弑父之主!”

听着刑恕抓住忠孝二字,与王安石辩驳,蔡确十分安心。

韩冈虽有天纵之才,王安石的威信更是重于泰山,却也无能为力了。大势所向,谁能逆水而行?

韩绛的身周正散发着阴冷的气息。虽然看不见,但蔡确也能猜到他现在是什么样的心情。

蔡确知道,这一位同中书门下平章事兼昭文馆大学士,甚至可能比韩冈还要愤怒——对他的暗中策划,对他的独断独行,必然是恨之入骨。

但有当年韩绛独自让慈圣光献曹后撤帘一事在前做例子,蔡确完全没有考虑过将韩绛一并拉过来。

大不了就像韩琦和富弼一样从此割席断交,左右他与韩绛根本没那么好的交情。

而且韩琦与富弼之间的恩怨,是富弼单方面咬牙切齿一辈子,而韩琦好端端的做他的两朝顾命、定策元勋。甚至还能悠悠然的摆出高姿态,每年给富弼送寿礼,激得富弼丢人现眼,被世人认为是有失风度。

不过是韩绛跳脚,这份功劳,有什么必要分出去的?

皇城中五重禁卫,皇城司亲从官第一重,宽衣天武官第二重,御龙弓箭直、弩直为第三重,御龙骨朵子直第四重,御龙直为第五重。由外而内,一重重将天子保护在中央。

石得一控制了皇城司,宋用臣掌印玺,又设计将御龙四直掌握住。张守约在殿上,王中正被囚禁,宽衣天武和诸班群龙无首,看似惊险,却没有多少风险。

韩冈枉为大儒,却根本不知道,他一力要维持住赵煦帝位的行径,正是让宫内人心惶惶不安的元凶。没有他,就不可能会有太皇太后和二大王卷土重来的一天。

要不然,已经几乎到了内侍能拥有的最高位的石得一和宋用臣,此二人如何会反叛?尤其是宋用臣,他对先帝是真正的忠心耿耿,不是失望到极点,又怎么会转投高太皇?

“臣蔡确,请太皇太后颁下大诏,并晓谕国中……”

蔡确对着上面行礼,打断了王安石和刑恕。

他不满的看了刑恕一眼。这个时候,最忌讳的就是乱。而王安石和韩冈,最喜欢的就是乱。越乱,他们就有机会浑水摸鱼。

刑恕终究是年纪轻,不知道虚中内守,以不变应万变的道理。却差点给王安石带进水里。

蔡确对赵煦有拥立之功,现在又让赵煦退位,另立新君,他的作为,几乎可比拟霍光。但也正是这样,蔡确才分外的警醒,许多事情他都交给了外人,而不是自己去做,或是从自己的人中挑选。

就如这一篇诏书,明赵煦之罪,让废立之事变得顺天应人。并非一定要苏轼的手笔,蔡确自己也能做得来。但苏轼有声望,现在的朝廷需要他的名声。

所以苏轼被连夜招入宫中写诏书。明明是外制的中书舍人,做的事却是内制的翰林学士。而事实上,等今日事毕,他就要进入玉堂,成为真正的翰林学士。

宋用臣已经抑扬顿挫的开始念着诏书。

那位准翰林学士的大作,韩冈没有去听。

也许写得很好……或者说,肯定能写得很好。

以苏轼的水平,甚至可以媲美扬雄为王莽写的《剧秦美新》,不会在《为袁绍檄豫州文》与《讨武檄》之下。

但韩冈没那份余暇去听废话。双手藏在长袖中,正一根根的屈起手指。

一根、两根、三根、四根。

蔡确。

必然的主谋,没有他在外配合主持,太皇太后还只能被软禁在宫中,而赵颢,更是得继续疯下去。

曾布。

薛向。

虽然不知道他们什么时候参与进去的,但正好在他们当值的时候出事,自然是早早的就决定下来的。

苏轼。

应该是拉人头的。以苏轼在京城士林中的声望,包括民间,都算得上很不错。不过禅位大诏写得的确不错,还真把宫闱政变变成了顺天应人的禅让。

石得一。

宋用臣。

赵颢家做监视的内侍,都是宋用臣安排的。而皇城司那边是石得一在管,手握重兵。

他们都会反叛,从利益上,很难说得通。

韩冈心中自省,是自己慢了一步,也低估了赵煦失德,对宫中人心的影响。

废立天子的诏书才念到一半,不想再听废话,韩冈提声打断,“太后临朝,权同听政,此一事出自先帝。尔等欲废天子,那太后呢?”

还不死心?赵颢放声道:“先帝这一诏令就是错的,以母改子,有何不可?”

“我只闻在家从夫、出嫁从夫、夫死从子。不闻以母改子。”

“失德之君,不可王天下。”

“篡逆之辈,难道可以做天子?!”韩冈声色俱厉,上前两步,与赵颢对峙着。

他这一段,是将太祖皇帝都骂进去了,但没人觉得好笑。

这是困兽之斗,已经没有了反败为胜的可能。

垂死挣扎的韩冈,不免让观者腾起一股兔死狐悲的伤感。

看见韩冈又愤怒的上前了几步,两名站在台陛下的御龙骨朵子直禁卫,立刻跨了出去,一左一右夹住韩冈,拦着他继续往前。

两名禁卫,皆是一身金甲,外套红袍,手中一支涂金铁骨朵。这是大宋军中,最为精锐、也最为亲信的班直侍卫,守护在天子左右。现在,则是保护着屏风后的高滔滔和坐在御榻上的赵孝骞。

在声名显赫的韩冈面前,两人虽然带着为难和畏缩的神色,但依然是毫不动摇的拦住了他。

韩冈没再上前,他抬头向上,盯着屏风,以及屏风背后的高滔滔。

屏风后沉默着,不是无言以对,而是嫌有失身份。她在看着韩冈的挣扎,这是猫戏老鼠的余裕。

没有得到回应,韩冈垂下头去,然后又抬起来,“韩冈虽愚鲁,却不敢逆圣人之教,奉篡逆之辈为主!”

他音声冷澈,神色愤然。

双手摘下了头上戴着的长脚幞头,递给了左手边御龙骨朵子直禁卫。

那禁卫手忙脚乱接了下来,却是一脸的茫然。他不知道韩冈这是何意。

紧张了半日,蔡确在旁却松了一口气。

韩冈是认输了!

这不是鸭子死了嘴还硬,而是以辞官归隐为条件,祈求宽恕。

可到了这步田地,又岂是辞官就能了事的?!

就在殿上,数百道目光注视之下,韩冈解下了腰带,扯开了官袍,露出了内里的一身劲装。冬天公服的宽袍大袖容易招风,官员们都在里面穿着贴身的短袍,袖口、襟口都扎得很紧。

韩冈亦是如此,一身劲装的他,身形笔挺,矫矫犹如劲松。

可是让人无话可说的殿上失仪,只怕在大庆殿修起来后,还从来没有人当朝在殿上宽袍脱衣。

但御史们并没有出声痛打落水狗。

就是刑恕也没有出来指责,他等着韩冈表演完毕。

韩冈这一举动,怨望昭著,罪证分明。

不过反对最力的韩冈一旦离开殿中,便是大事抵定,只凭王安石一人,绝无回天之力。

他们正盼着韩冈掉头离开,让新君登基的第一场朝会顺利的进行下去。

就在殿外,还有石得一领人等着,韩冈一出去,就会被捉起来。等此事一了,自有处置,到最后当是一杯毒酒赐死了事。绝不会给他出皇城调动兵马的机会。好不容易才将朝臣们都弄进殿来控制住,怎么可能让他轻易出宫去?

结束了。

章敦闭上了眼,他终究不能拿着全家老小的性命与韩冈一起死拼到底。

而这样的韩冈,现在也认输了。

他亲眼看着韩冈将属于公服的配饰一件件的摘下,又一件件的交给两名禁卫。幞头、鱼袋、腰带、方心曲领,最后只剩下浅紫色的官袍,团成一团,然后塞进了禁卫的怀里。

十八岁出仕,十二载为官,从卑微的从九品选人,做到了宰执的位置上。传奇一般的生涯,现在,终于走到了尽头。包括他的官职,也包括他的性命。

章敦不想再看下去了。

“拿好了。”韩冈正轻声的对那禁卫说道。

他将最后一件官袍递出去后,双手顺势下拖,搭在了禁卫手中的骨朵上,微一用力,便轻轻巧巧、自自然然的将那支涂金铁骨朵,从抱着衣物和饰品的手中给抽了出来。

生铁为质,外饰金粉。虽是骨朵,却如同蒜头。

沉甸甸的铁骨朵五六斤重,握在掌中,趁手得很。

韩冈抬头向上。

双瞳中的眼神,没有一丝绝望,惟有毅然决然的坚定。

明黄涂金的御榻映在深黑色的眸子里。

正在十步之内,只隔台陛数阶。

