\section{第33章 物外自闲人自忙(八)}

文及甫如坐针毡。

韩冈就坐在他的侧前方。四平八稳的坐在交椅上,正带着谦和的微笑与自己的父亲谈天说地。

尽管他依然十分注重礼仪的没有与身为前任宰相的父亲分庭抗礼,但这位年轻得让人嫉恨的京西都转运使,无论是他的神态,还是他的坐姿,甚至是说话的语速、腔调,在文及甫看来都是一幅胜利者的姿态。

如果事情仅仅如此,文及甫最多也只是拿着憎恨的视线配上应酬式的笑容,闭起嘴巴坐在厅中,做好一个称职的摆设就够了,不至于觉得自己屁股底下的交椅让人难受得如同针插一般。可韩冈作为一个不受欢迎的客人,表现得实在太过自在了一点。

为了不让作为陪客的文及甫太过清闲——在外人看来,这是韩冈礼貌的表现,不让地位不够插话的文及甫被冷落——韩冈时不时的就将话题移到他的身上。

“文翰旧日曾入崇文院直史馆,韩冈几年前亦觍颜得入崇文院,也曾一睹文翰的手稿。书法精妙正如文翰二字,三馆楷书是不用说了,一手飞白足证潞公的家学渊源,韩冈是钦羡不已啊!”

“愧不敢当。”文及甫憋着一口气,谦虚着向韩冈低头。韩冈呵呵两声笑,转过去趁势与文彦博说起荆湖几块有名的金石古碑。

过了一阵,韩冈又转过头来,“文翰如今在西京粮料院当值,再过几日,韩冈南下主持开漕之役,许多地方可是要靠着文翰相助。”

文及甫又低下头,咬牙切齿的应承道:“不敢,不敢,龙图若有指挥,及甫敢不尽力。”

韩冈又是笑着谢了一句,转过再与文彦博聊起行军打仗时如何安排粮秣运输的经验。

每一次与文及甫说上两句,韩冈便又转回去,跟文彦博又交流了起来施政、用兵之类的心得,以及一些来自南方、尤其是岭南的奇闻异事和神怪传说。

看到韩冈坐在那里言笑自若,文及甫就难过得浑身发痒。偏偏在这个场合连动都不敢乱动,弄得他仿佛就像是在锅里被熬着油,心里一个劲的叫着苦,这份陪客的差事到什么时候才是头!

自己的父亲应该是在竭力压抑着心头的怒火。两任宰相、两任枢使,三十余年的公侯,竟然不小心落到了一个黄口孺子的陷阱里——自家父亲做宰相的时候,韩冈连毛都不是——最后还要让这灌园小儿再次登门来化解,多少年没感受到这样的耻辱了?

别的文及甫不知道,但他可是知道他父亲正在喝的茶里面是放了祛风活血的消风散的。

只是此事在表面上一点都看不出来,韩冈和自家的老父言谈正欢,如同一对忘年之交,小声说、大声笑,毫无纤毫芥蒂。

韩冈赞一句文相公功业骄人,仰之弥高,钻之弥坚,后生晚辈追之难及。文彦博就回一句后生可畏,老夫须得让出一头地。

一团和气,你来我往互相吹捧的样子,表面上根本看不出来两人之间仇怨已深。

文及甫在费尽心力的忍着自家不露出惊讶的表情,维持住现在的虚浮在脸皮上的微笑。

难怪世人都说韩冈日后当能做宰相,要是做宰相的都必须有这份言不由衷、表里不一、转眼就能‘化干戈为玉帛’的心性,自己是不用指望一窥东西二府的院墙了。

外面都说韩冈才学不足,一个进士第九,是天子因为他的功劳而特意提上来的,本来该是排在第五等的榜末才对。但现在文及甫看着韩冈他与自家老父聊天时,经义、史料都能信手拈来,显是浸淫极深,甚至朝廷中的故事,也是一点不见生疏。

恐怕韩冈差就差在诗赋上,但这个话题别说文及甫,就是文彦博都不好提,若是拿出来当话题,韩冈会怎么反应谁都不敢保证,眼下这和谐的气氛尽管是装饰出来的,但要将之保持下去,一直到韩冈聊够了自行告辞,也是文及甫现在唯一的心愿。

所以他也只能忍着,等着韩冈话说腻味了,自己起身告辞。但若是他现在就告辞,却是必须强留着。文及甫摸了摸茶盏,从通过天青色的薄胎瓷盏的热度上看,过去的时间还并不长,至少还要留着韩冈半个时辰的时间。

文彦博的儿子心中叫苦不迭,但他也只能堆满僵硬的笑容,等着韩冈隔上片刻便来上一次的垂顾。

……………………

喝了一口消风散的清茶,藿香叶和厚朴的姜味顿时在口腔中散开,陈皮和人参的淡淡甘香也缓缓释放,文彦博感觉稍微好了那么一丁点,心头上的憋闷也随之散开了一些。

但文彦博也知道,只要面前的灾星不离开他家的客厅,依然坐在这里高谈阔论,这兑了消风散的清茶,就要一直喝下去。

将贵重的瓷盏放下来,文彦博道:“玉昆旧年在陕西宣抚司,轻易平定了庆州广锐卒之乱,那时候老夫还在枢密府任上,听说玉昆不辞性命之危,毅然入城说降,一席话说动了叛军开成而出,老夫也不得不为之击节叫好。”

“远不及潞公当年平定贝州之乱。”韩冈对文彦博的恭维礼尚往来,“庆州广锐军叛乱只是因为赏罚不公而已,并非有心叛离,加之叛军又被困于城中,人心惶惶,说降不难。而王则是蓄谋已久,自称神圣,为了造反筹划多年。他的信徒心意坚定,要不靠了有潞公一手主持平叛,贝州如何能如此讯快的收复?”

文彦博和韩冈哈哈哈的笑着,赞美的都是对方值得一提的功业,言辞恳切,像是发自于肺腑,完全是真心实意。但文彦博就是知道韩冈是根本没把自己的成果放在心上。

自家的确是剿灭了叛军,并因此升任宰相。但韩冈不仅仅平定了叛乱,更开拓了国土,还灭掉了一个国家,这份差距可不是韩冈的一两句恭维就能当做不存在的。他的奉承话说在耳边,而实际上又有几分诚意?

文彦博心中的冷淡,反映到脸上,却是温和厚重的笑容:“玉昆说降的这些叛贼,他们在河湟之事上,立下了不少功勋,这也是玉昆的功劳。”

“韩冈那个时候不过是个新入流品未久的小官而已,河湟之事,上有天子护翼,下有王副枢主持,韩冈也仅仅是赞画而已。”

“有玉昆你于其中赞画辅佐,下面的士卒才敢奋勇作战……毕竟是药王弟子啊。疗养院不知救了多少带伤的士卒。”文彦博笑赞着。

“药王弟子即是市井谣言,纯属无稽之谈,潞公就别拿韩冈取笑了。韩冈在熙河经略司设立的疗养院,也是得到了多方协助方才成功,并非是一人之力。”

交谈还在继续,话题也是天南地北,韩冈年纪虽轻,但历事甚多,说起南北趣闻,在见多识广的文彦博面前,半点也不见怯场。

没有说的,一番深入的交流之后,文彦博明白自己之前的确是太小瞧这位灌园子了。可以说是几十年难得一出的策士,贸然将把柄留在他双手上,落到如今的田地也不足为奇。没被害的家破人亡,声名尽丧,已经是难得的运气了。

不过他的七十余年的人生也没有虚度,只要韩冈露出一点破绽,文彦博就能立刻把握住。

‘只要等着就是了。’文彦博想着,又狠狠的灌下了一大口掺了消风散的茶水。

……………………

韩冈从文彦博每说上两句话就抿一口茶水的动作来看,至少这位潞国公心中依然带着浓浓的不甘心,甚至是想着日后加以报复——此事也不足为奇。

文彦博隐藏得很深的恨意,韩冈却并不放在心上。早就知道的事,也不足为奇,堂堂宰相恨他一个都转运使,也算是光荣了。

就是不知道文彦博能不能压得下现在的恨意,再过几个月,襄汉漕渠破土动工,民夫们所要消耗的钱粮有很大一部分要经过洛阳,只要判河南府的文彦博不致仕,钱粮转运的单子都要从他这里走上一遭。

“开凿襄汉漕渠一连失败了两次,在太宗皇帝之后,就没有人再敢与此事上做文章了,也只有玉昆才高于世,能有所成就。”

“此事全是韩冈在去广西的路上看到旧时的遗迹,故而才动了心思。汴河一年一疏浚,耗费的钱粮一年几近百万,运送上京的纲粮也不过六百万而已。若能打通襄汉漕运,京城也不用全然依赖汴河。狡兔亦有三窟,东京百万军民,宗室官宦几近万家,怎么能只依靠一条汴河?”

“玉昆此言说得正是。东京汴梁为天下之中,怎么能只依靠一条汴河,若能打通。事关国运,玉昆宜当勉之。”

“有潞公垂顾,坐镇于后,韩冈何愁工役不成?”

“有玉昆统辖,必能水到渠成。”

虽然是令人作呕的互相吹捧,亦是言不由衷,但文彦博的态度算是明朗了,眼下他面临的局面,也不能在此事上扯韩冈的后腿。

今天的这一次拜访,算是有所收获,并不仅仅是上门来帮文彦博解围的。就算是再心不甘情不愿,短时间内文彦博也必须得支持自己。

对韩冈来说,已经足够了。

看着文彦博再一次端起茶盏,微微颤抖的手将瓷盏凑到嘴边,韩冈笑得更为和煦,犹如春风一般。

