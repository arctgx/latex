\section{第46章 了无旧客伴清谈(六)}

韩冈对蜀国公主家的私密事没什么兴趣,随口两句就带过去了。

从蜀国公主的驸马都尉王诜身上引申开来,吴衍还想多聊两句他的好友苏轼,

在熙宁四年被逐出朝堂后,大苏的文名越来越盛,一首首佳作翩然而出,与四方文士相唱和,已经渐渐有了一代文宗的架势。如果什么时候能像欧阳修一般,做一两任礼部试的主考,妥妥的又是一位文坛座主。

“最近,自《眉山集》、《钱塘集》后,苏子瞻又有一部《元丰续添钱塘集》付梓,听闻正是托付给王诜。”

“哦,不知选的是哪一家印书坊?”韩冈似乎是很有兴致的问道,“能得驸马都尉看重,想必印版刻工都是上上之选,日后韩冈若有文字想要刊行于世,也可做个参考。”

吴衍不意被问到这个问题,愣了一下后,终于想起来韩冈本人不擅诗词,加之苏轼当年被赶出京城的缘由,据说跟韩冈和他房内人有牵扯不清的关系,提他的名字的确有几分不合时宜。

“这就不清楚了,等过几日打听到之后,必会转告玉昆。”敷衍了两句,跳过不合时宜的话题,吴衍将谈论的重心回归他的本职工作:“雍王家的长子第一,蜀国公主的儿子第二,这几天下来,宗室公卿家的子女,已经有上千人种了牛痘,尚无任何不幸的消息回报。利用善堂和慈幼局内的孤儿,痘苗的数量也够了,医生也培养出来了不少。现在东京城中的每个厢都已经设立了专一负责种痘的保赤局。等过了年后,开封府界的二十余县,也都会派出得力人手去县中设立保赤局。”

韩冈既然被天子钦点了不管事的差事,对厚生司中的事明面也不便多加干涉,吴衍说什么,只管听着了,偶尔才插一句嘴。

“京城是种痘术的重中之重,不过天下士民皆是天子治下的百姓,厚此薄彼做得太过明显也不好。”

“厚生司已经开始准备在天下各地推行种痘法。会先给太医局中的医生练练手,等他们有了经验之后,就能独当一面了。”

“最好能先从边缘地区安排人,”韩冈提醒道,“否则人人拈轻怕重,根本就没办法将天子的恩德,告之每一位大宋子民。”

“前两日李德新来,就说了此事。还说熙河路和广西路,缺医少药,全靠朱中和雷简两人支撑着。能早一步推广种痘法,对他们的差事也有帮助。”

在熙河路的朱中和广西的雷简,早就积功被韩冈推荐为官,是有名的翰林医官。

“安焘是怎么做的?”韩冈问道。

“早安排了急脚递,送去了第一批痘苗。”吴衍请韩冈放心,“雷简和朱中都是玉昆你提拔任用的医官,有他们主持两路种痘,想必是不用让人担心了。”

“该派人监察还是要派人监察。否则御史台那里就别想过关。”韩冈说道,“种痘也是要收钱的,得防着不轨之辈,趁机捞取不义之财,坏了朝廷拯济百姓的本意。”

“那是自然。玉昆你大可放心。”

又与吴衍聊了一些闲话,送走了吴衍,韩冈回到了他的书房。摆在他案头上的,是从群牧司拿回来的一份誊本,是沙苑监刚刚呈递上来的报告,今年监中开支的详细列项,以及军马的繁殖、病殁和出栏的具体数据。

别的韩冈倒没在意,他只看到了一个四十万贯、一个六千匹、一个三百匹。

整整四十万贯经费,牧马六千匹,可一年军马就出栏了三百匹。而且作为长于军事的朝臣,沙苑监调教出来的军马究竟是什么水平,韩冈很清楚,别说上阵作战,根本是‘无以任骑乘’!

幸好如今群牧司中,河南河北的主要牧监年年裁撤,最后就只剩这么一座沙苑监了——群牧监的粪钱也是越来越少——要是还保留至十二监的规模,那就是吞吃钱粮的无底洞了。

所以几年前曾经有个在熙河路任职的官员,建议王安石在熙河路设立牧监,但给王韶和韩冈联手阻止了。监中的官吏和只是群蠹虫而已。

当然,军马出栏数量如此之少,并不完全是监中官吏牧兵牧养不力的缘故,也有土地被侵占的因素在。

韩冈当年和王韶一起谋划茶马互市的时候就已经了解过了,沙苑监在籍簿上的九千顷牧地,最多只有三分之一还保留着,剩下的都给占去做田地了,眼下又是七八年过去了,想来数量只会更少。

开国之时,正值晚唐和五代百年乱世,人少地多,所以在京畿之地都能圈出来左右天驷监四,左右天厩坊二,总共六个牧监,而且三衙辖下的各部马军,也都有自己的专用牧场。在真宗大中祥符年间,京畿及河南河北牧监总数一度达到了二十二座。

可惜好景不长,随着人口繁衍,以及官绅世家的胆量越来越大,牧监不断撤并的同时,监中土地也被春蚕食桑叶一般的不断侵占。不仅仅各大牧监和禁军中各部马军放养本军战马的牧地,就是作为孽生监的七座牧马监——孽生监用后世的话说,就是种马场——也是大片大片的土地给人占去种田。

想想吧,连培育种马的马监连地皮都给人占了,国家的马政还能有什么样子。

侵占牧地的并不是普通的人家,不是官户,就是形势户——所谓形势户,就是地方上有势力的豪富之家,主要是州县衙门的高阶吏员﹑乡里的上户,有时候会将官户也包括进形势户的范围,但更多的时候,官宦人家是不屑与吏户并称的——每一家都有几分背景,肉进了他们的肚子,哪里还能讨得回来?

据韩冈所知,在王安石上台时,左右骐骥院管辖下的河南河北十二监马,基本上都没有剩多少牧地供给放养马匹。所以做为保马法的推行一句统计出来的熙宁三年的马政数字才那么凄惨——河南河北十二监,岁出栏一千六百余匹,而可以成为战马的,只有两百多而已。相比起辽国动辄几十万、十几万的战马,大宋的战马数量实在是可怜之极。而花费,则是一年一百万贯。

这样的情况下,要马的话,得从虎口夺食;不想开罪太多官绅,那就干脆放弃从牧监得到战马的念头。

王安石可没蠢到跟那么多官宦豪门相对抗,青苗、市易争得是浮财,好歹还有些说道,可土地才是人家的命.根子,哪里能抢得回来?就算是他为了大宋着想,闹得民怨纷纷时,天子还不一定领情。不但成功不了,还会将自己给搭进去,还不如想办法去开拓马源,并承认各地马监被侵占土地的现实。

王安石裁撤牧监,实行保马法,让民间养马,就是这个不得已的缘故。而在王安石之前,仁宗年间就已经开始在河北施行了官卖马匹,由民间来饲养,官府在需要的时候加以收买的制度。都是看到了中原各大牧监的最后结果。

河南河北十二监,从保马法开始施行时起,逐年废除,仅仅留下了同州沙苑监。而牧马监废处之后,清理出来的土地被占去的不论,没被占去的也都租佃出去,收取租税。一出一入,一年财政上能多出百万贯来。

这一百万贯,除了一部分供给市易,剩下的就是拨给熙河路,充作茶马互市的本钱,现在,则是又多了广西来分账。一南一北,市易而来的马匹基本上能达到五万匹,其中合格的马留在军中,不合格的马匹则由群牧司负责转卖给民间,做个合格的二道贩子。

有了茶马互市来的军民,加上保马法寄养在民间的马匹,军队的使用算是足够了。而韩冈敢于提议在河北修建轨道,也是因为国中马匹数量大幅增加到缘故。

从眼下的情况看来,保马法的确是有效果的,只要能忽视掉其他问题。

只可惜,如果要上战场的话,有些问题是没办法忽视的。

单从数量上来说,民间的马匹不算太差,至少要比过去利用牧马监养马的情况要好——当然,其中有很大一部分原因是作为对比的对象,牧马监的情况实在是惨不忍睹了一点,些许进步,只要对比一开始的惨淡基数,都是一个让人兴奋的进步——可是从质量上来讲,就未免让人难以满意了。

将马匹下放到民户手中,让他们代养,由此而来的结果,就是培育出来的马匹几乎没有一匹能上战场。战马不仅仅是肩高、毛色、体重、体格等方面的问题,性格也很重要,要胆子大、不怕人,面对箭雨和号角能毫不动摇,关键时候能与骑手一起拼命的战马。可从民间培养出来的军民,就跟小家小户出来的人一样,上不得席面,拉犁耕地倒有一手。韩冈给它们找到了一份拉车的工作,正是选对了行当。

