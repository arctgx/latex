\section{第十章 却惭横刀问戎昭(四)}

韩冈分心,还是不免让李宪窥探出来几分。

“经略!经略!有急报!”

一名身着青袍的小官神色慌乱的跑到了厅外,打断了韩冈和李宪的对话。也不知到底是出了什么急事,韩冈认得他的那张脸,是衙中的勾当公事。

“慌慌张张,成何体统?!”韩冈一声呵斥,心道孙永是怎么用人的,毛毛躁躁、大惊小怪的性子如何能在衙中公厅里做事?

见自家下属老实站好,收敛了惊慌。他这才将人招进来问道:“到底出了何事?让你如此惊慌。”

勾当公事却迟疑起来,一对眼珠朝李宪的方向瞄了一下。

韩冈是河东路兵马都总管和河东路经略安抚使,而李宪则是经制河东兵马,从职分上,两人的管辖范围相互重叠,而且集中在能起到关键作用的军队上。正常情况下,两人连照面都会尽量避免——当初孙永在的时候,跟李宪都没见过两面,上书反对阉人领军的也有他一个——只是眼下的形势逼得两人必须坐在一起。

但要说韩冈会喜欢一个分自己权柄的阉人说话谈天,六月飞霜的可能性还大一点。

要是李宪是有出身的文臣,那他跟韩冈还有些说道,可一个没下面的宦官,纵然一时权势熏天,但说不定哪一天就会被天子所疏远。远远比不上韩冈这样的有功名、有师友的新任经略。

李宪惯会察言观色,见到韩冈下属的作态,知道自己在这时十二分的碍眼。立刻手压着袍服,欠身便要起身告辞。

韩冈伸出右手朝着李宪向下虚虚压了一下,比了个稍安勿躁的手势,转头对勾当公事道:“李都知奉旨经制河东兵马,若为军机,但说无妨。”

“经略,都知。”勾当公事抱拳道:“代州急报,辽人于雁门寨新铺犯界,杀伤我军民十余人,恳请经略速做处断!”

“问本帅如何处断?”韩冈眼眉剔起,眼中似乎有怒意在燃烧:“他刘舜卿是怎么处理的,辽人难道就在新铺处等待本帅的处断?!”

第一任代州知州是杨业,杨家将中的杨老令公,与第一任太原知府相印成趣。而现任的代州知州是刘舜卿。

勾当公事心惊胆战,但韩冈的问题他也不知该怎么回答,犹豫了半天,除了嗯和啊之外,没有别的话

“刘舜卿他没有在公文中说他怎么应对的?”韩冈的眼神越来越危险。

“刘希元乃是当世名将,纵然只是小股人马,也不至于让他忙得忘了该如何去处置。”

韩冈哼了一声:“希望如此。”

李宪说的不错,刘舜卿的确是‘名将’——名气很大的将领。跟他同名的那一位窦姓名将差不多。

窦舜卿是捕盗三百海贼,然后在南方平定蛮夷立功,也就是破了一个山寨,又将一个杀了十三位羁縻州州将、并吞其土地人口的叛贼招降——那个叛贼降伏后既没有受到朝廷的惩处,也没有吐出他夺走的人口土地。

而刘舜卿则是招降八百泸州蛮,然后坐镇边地。至于能拿得上台面的战绩,韩冈倒是没有听说。他既然领了河东经略之职,之前在京城时就着意打听过河东路排在前几位的将领们各自的事迹。已经升入横班,成为军中高层的刘舜卿,算是战功最少的一个——只可惜架不住他得圣眷。

刘舜卿曾经在秦凤路任职,不过韩冈与其没有打过交道,任职的时间正好岔开了。韩冈对他了解很是泛泛。

韩冈转脸过来,问李宪道:“都知在河东已近一载,不知刘希元为将如何,治政如何?”

前面韩冈已经表现出了对刘舜卿隐约的反感,但李宪不觉得自己有落井下石的必要,“刘希元长于练兵。当年曾经在京东用一年的时间,汰弱留强,最后留下的一支千人队,在天子面前表演阵列队形。”

韩冈对此根本不屑一顾,能拉到天子面前演武的,也就千八百人,从京东两路军中挑选精锐,然后用一年时间加以操练。练出一支看上去像那么一回事、队形操练的精兵来。

李宪心中暗叹了口气,看起来韩冈对刘舜卿颇有几分微词。李宪又观察了一阵,最后道:“龙图成竹在胸,想必已经有所应对了。”

韩冈反问道:“知道为什么过去辽人南下乐此不疲吗?”

“为何?”

“用买卖的手法来比喻。南下打草谷那是赚钱,只要让两虏的劫掠生意变成亏本买卖,他们就不会再继续做了。所以澶渊之盟后,辽人只有讹诈,不再强抢,因为他们知道,抢来的不如赚来的。”韩冈一声长叹:“党项人年年劫掠,那是因为成本太低,抢到一点都是净赚。”

……………………

折可适坐在夏州城的城门里,嘬着种师中不知从哪里弄来的上好狗肉,与种师中两人一起喝着掺了七八成水的淡酒。

外面炽烈的太阳依然散发着热毒,而城门门洞中,却有着难得阴凉。卸了甲,连衣袍都扯了半边下来,将右侧的肩膊和胸口都暴露在门洞里凉爽的清风中。

从嘴里拔出一根骨头,折可适看了看,甩手就就到了地上。转头又从锅里捞出一块带肉连骨的狗肉,塞嘴里嘎嘣嘎嘣的嚼了起来:“想不到这件事太尉当真不管了……”

“不敢管啊。”种师中守了多少天的城门,终于有个人能伴着闲聊天了,折可适与他坐在一起,就感觉身边如同打开了一个话匣子:“徐禧身后有人,他家的亲家可是正当红,指不定现在就能宣麻拜相了。”

折可适可不会在口才方面示弱:“徐禧那厮心狠手辣,其寡母与一莫姓秀才私通,徐禧和其弟便设计将莫秀才灌醉了淹死在长江中。前些日子这些事被蔡承禧揭了出来,但江南东路上报查无实据,就不了了之了。要是真跟他硬顶,他动不了五叔那尊大佛,俺们这等小鱼小虾可是会被拿出来杀鸡儆猴的?没人想做焦用吧?”

“等他做了参知政事再说吧,想学韩老相公的本事,至少也得一个经略使。就一个体量军事、边事,吓得倒谁?”

种师中说得肆无忌惮,折可适也没有半点畏惧,听得摇头晃脑,嚼得有滋有味。

种师中还想再多说两句,孰料身后一身冷到了冰点的呵斥:“二十三!”

声音入耳,种师中就立刻条件反射的跳了起来,毕恭毕敬的站好。

下一刻,板着脸的种建中走了进来。

他狠狠的瞪了折可适一眼。有关徐禧的这个传言,折家人可以肆无忌惮的乱说,种家人就不行。

别看种谔是三衙管军,军中最高位的十几人之一,而折克行仅是个知府州,本官也只是宫苑诸使中的礼宾使,但折家近似于诸侯,蓄私兵,养死士,拥有一府之地;而种家不过是个普通的官宦人家,四叔种詠为人所害,瘐死狱中,最后连仇都报不了,换作折家看看?有哪个敢这么对付折家人?

种建中这些日子心情正糟,自家堂兄弟在灵州之败中折损了好几个,全都是他这一代的叔伯兄弟中能上阵领军的英才。本来是想趁机占个便宜,挣个前程回来,孰料前程没挣回来,人也同样没回来。

现如今,种家同班辈还能在军中拼一拼的,也就自家两兄弟,和排行第十七的种朴了。将门种氏的门庭,还不只能维持多久。

种建中大步走到种师中的身边,用力一拍肩膊,“二哥、八哥和十一哥都没有回来,过些日子人到齐了,就要做一场罗天大蘸,连五叔现在都在吃素,你倒好,在这里狗肉吃得痛快。”

折可适大马金刀跨.坐在小小的交椅上,听着就不顺耳:“种十九。不是俺跟你过不去。这一战难道我折家就没死人?光是运送粮草的事务,折损了多少折家子弟?没见俺摆个晚娘脸吧?”

种建中脸色更难看了,怒瞪回去,“兄弟死、不尽哀,可为人哉?”

“算了,这事争不出个是非对错来,俺读书不多,也没拜在横渠门下。”折可适意兴阑珊的站了起来,“俺这就要回弥陀洞,前面已经跟太尉辞行过了,也不方便再耽搁时间。等李经制从太原回来,俺还没回去应卯。他能给俺爹面子,俺家老爹可不会给俺面子,半个月就只能趴着睡,那滋味可不好受!”

种建中神色缓和了些,“赠与令尊和令叔伯的礼物皆在包裹中。一点土产以表心意,还望不要嫌礼轻。一会儿还有事要忙,恕建中不能远送。”

亲兵牵来坐骑,折可适一跃上马,居高临下的俯视:“你们的确忙。三万人送去当鱼饵,种太尉等着收鱼线呢!能不忙吗?”

种建中倏然变色,转又冷笑起来:“徐禧身后有政事堂中人撑腰,谁能挡得住他?再说,令尊之前可是从头看到尾,一句话都没说过!”

“不管俺折家的事,府州上下都会做个瞎子、聋子,有什么好说的。但新来的经略可不是瞎子、聋子。十九哥啊,你说他会不会看在你们种家和他的情分上装聋作哑?!”

