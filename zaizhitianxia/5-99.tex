\section{第11章 城下马鸣谁与守(11)}

晦暗的天空如同李舜举的心情。

包围在盐州城外的西贼如山如海,四个方向全都扎下了营盘,封死了每一条道路。

攻城的喧嚣,日以继夜,从无一刻休止停息。

当当的脆响,是板甲挡住了箭矢。铮铮的弦鸣,是神臂弓在射击。轰轰的巨声,是霹雳砲投掷出的石弹命中了目标。

板甲、神臂弓、霹雳砲,这些都是让宋人足以自豪的利器。但这些声音,来自于城墙之上的,只有一枚枚撞击墙体的石弹。

一切的装备都来自于大宋官军,无数战利品让党项人也同样武装到了牙齿。穿起板甲、配上钢刀、拿起神臂弓,将三寸的木羽短矢射上城头。

而伴随着弩矢的,是一枚枚或大或小的石子。

俘虏了多名专门打造攻城器具的宋国工匠,党项人用了六天的时间,造出了八具霹雳砲,长长的稍竿将一包包石子或是人头大小的石弹送上城头。虽然因为木料、时间都不如过往,仓促之间打造出的霹雳砲只有六七十步的射程。

从城头上神臂弓射出的箭矢,可以轻而易举的命中城下的砲手们。但在付出了几条性命之后,党项人很快就在霹雳砲的前方,竖起了用厚重的木板拼起的挡箭板,让砲手们可以毫无顾忌发射石弹。

对于一寸多厚的木板,神臂弓也无用武之地,用战弓抛射长箭、甚至火箭,倒是能给砲手们带来一点干扰,但也只是一点而已。

换作是几天之前,对于城下的攻城器械,完全可以点起兵马直接从城中杀出去。纵然这几具霹雳砲,都是远远避开城门,但要冲上去毁掉也并非难事。可前日的一场惨败,使得城中守军,已经失去了出城反击的力量。

李舜举决然没有想到,不过是几个工匠投效了党项人,就变得如此棘手。

片刻之前,就在李舜举的面前,一名士兵走避不及。被从城下飞来的石弹砸中了头颅。精铁的头盔顿时成了纸片,从铁片的缝隙中,暗红色的血水缓缓的流淌而出。

阵亡的士兵被匆匆拖走,但城头上尤留着残迹。望着眼前的一滩血迹,李舜举心中一片混乱。

为了对抗城下霹雳砲,城上甚至搬来了床子弩与其对射,但在狭窄的城头上给床弩上弦,远比城下的霹雳砲更难,也更慢。八牛弩号称合八牛之力,上弦的难度可想而知。九具床子弩反而比不上六架霹雳砲,被一顿散碎的石子,劈头盖脸的砸了一通之后,上弦就更慢了。双方的一番交换下来,还是城上更为吃亏。

这一仗的希望到底在哪里?李舜举已经对胜利绝望了,他抬眼望着不远处的敌楼,高永能正在上面住持城墙西壁的防务。不知他和曲珍可有扭转战局的能力?

高永能默然立于城楼上,城防的攻守布置下去后,自有部将去主持,并不用他事事过问。在李舜举望着他的时候,他却在想着正在城楼下睡觉的徐禧。

徐禧这几日都在城头上奋战。每天就带着两个烧饼巡城,累了就拿亲兵的大腿当枕头假寐。西贼来攻时,亲发矢石射击贼军。

在士兵们的眼中,这位主帅的确是蠢货,但看着徐禧与将士同甘共苦的样子,士兵们却又觉得他这个人还是很不错。原本高高在上的将领在他的带动下,也与士兵们一个锅里吃饭,一间屋里睡觉。

要是徐禧懂一点兵法就好了,这样的想法也只是在歇下来的时候才会在脑中转上一转。

六天前,当杜靖、朱沛帅京营禁军出城迎战。开始时的表现相当不错,与铁鹞子的鏖战始终未退一步,当缠战到午后,甚至将战线压到了两里之外,逼近了西贼的前军大营,看起来十分的顺利。

可到了这时,西贼大营中的动静,让高永能和曲珍终于看出了西贼的计划,步步退让的表现分明就是诱敌之计。

当一支多达三千人的铁鹞子从营中杀出,插到杜靖背后,战局就此逆转,只用了半刻钟,战阵便宣告崩溃。在南北两边压阵的王含、符明举立刻举兵营救,却又被困住,继而崩溃。几名将领只能率军向盐州城突围。

在溃败之前,两边的损失还是差不多的。从战场到城边不过两里而已,却成了一面倒的屠杀。伤亡、逃散的士兵多达八千。剩下的还是高永能和曲珍联袂领军出城救援,才被救回来。

当士兵们逃回来时,铁鹞子就追在他们身后,想要趁势攻入城中,而徐禧却硬是要等所有能进城的士兵进城才肯关上城门。

当铁鹞子趁机冲进城中时,还在战场的另一边奋战的高永能当真以为盐州城保不住了,幸好城中留守的部将奋战,好不容易才保住了城门,将两百多精锐的铁鹞子围杀在城中。

徐禧用兵失措,致使大败。但他最后开放城门救援众军的做法却让他得了军心,至少在兵败之后,没有变得无人听从号令。

如果徐禧有中人之上的军事才能,高永能倒能承认他是一名合格的主帅。可惜的是,徐禧的军事才能,是在平均水准之下。只不过是个读兵书读呆了的措大,嘴皮子俐落,却不是运筹帷幄、临机决断的人选。

眼下只能等着种谔或是高遵裕领军来援,不过兵败已经六天了,鄜延军和环庆军都还是没有出现。

城外的喊杀声不绝于耳。每隔片刻之后,就有一群党项士兵两人一组的抬着架长梯,直奔城墙而来。城上的箭矢立刻就密集起来,但依靠着板甲的帮助,还是有七八条云梯搭上了城墙。

党项士兵们一声欢呼,蜂拥在云梯下,爬得最快的转眼就上了城头,可是他立刻就受到了围攻,转眼就被七八条长枪洞穿,最后就是。而云梯之上,一勺勺滚烫的热油泼洒下来,让攀在梯子上的党项勇士捂着头脸滚到下地,紧接着一支燃烧着的火炬将沾了油的长梯点燃。

每一次的攻击,都在消耗着城头上的人力和物资。守城六天,阵亡和伤重不治的士兵多达两千。

死得人多也不是全然都是坏处,至少城中的粮食能多吃几天了。但这话高永能也只敢想一想,却决然不敢说出来。

轰的又是一声响,城墙在声起时,也颤抖一下。这是霹雳砲投掷出的石弹又撞上了城墙。紧接着,稀里哗啦的细碎声响,让高永能的眼神里又充满了焦躁。

西贼将霹雳砲集中在城墙的西南侧,半日下来,发射的次数以百计。夯土的城墙刚刚完工,还没有完全风干,这时候的墙体分外脆弱。本身损坏和城上床子弩的攻击,西贼今天拖上来的六具投石车,眼下只剩下一半,但城头上守具的情况也一样糟糕。

楼梯上传来沉重的脚步声,曲珍从楼下上来,与高永能并肩望着城外的浩荡大军,片刻之后才说道:“南门的八牛弩又坏了一具。”

“……这下子就只剩十二具。”高永能低声回道。

“十一。”曲珍更正道,“半个时辰前,南门就已经坏了一具,还伤了一个绞弦的小卒。”

高永能低低骂了一句,却没人听得见。

在开战前运来盐州的各色型号的床子弩总共二十一具,其中力道和射程皆是最强的八牛弩有六架。六天下来,床子弩不停的发射,在给党项人带来重大伤亡的同时,自身损坏的情况也越来越频繁。

党项人的霹雳砲虽也损毁了大半,但身在城外,他们总能找到合适的木料,就算离得再远,也只是费些手脚而已。但床子弩城中却造不了。再过两天,霹雳砲就能直接压着城头的守军,而不用担心床子弩的反击。

要是城上也有霹雳砲就好了,可就因为是守城,都没想到要留一个会造霹雳砲的工匠。

飞船在天上监视着远近敌情,但上面传下来的消息,只有在说外面的西贼越来越多。

楼梯间又响起了脚步声,李舜举也上来了。

见到曲珍和高永能,李舜举低声道:“盐州危在旦夕,夏州、环州的援军急切之间也难以赶来救援,如不早作准备,恐有不虞之祸。”

“这时候撤不出去。”曲珍当即说道。李舜举打着什么主意,他一眼就看得明白。

“撤不出去的……”高永能也是给出了同样否定的答案,‘在这个时候。’他在心里补充强调着。

盐州不是千山万壑的横山,周围都是平陆,最利骑兵追击和围堵。要想跑出去,只有内外皆乱的时候才有机会。但那样的机会,想必李舜举是不想看到的。

李舜举的眼神黯淡下去。如果眼前的局面真的如曲珍和高永能所说的一般,那么就是他把密旨拿出来,也改编不了大局,而徐禧这几日在城上的奋战,可以说他的努力将低落的士气给维持住了。如果夺下他的军权,恐怕不肯服从乃至抗命的将士会比预料的要多得多。

“盐州城还能支持几日?”他颤声问着。

“能支撑几日就是几日,等到援军来了就能赢。”高永能没有一个确定的答复。

