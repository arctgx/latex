\section{第五章 平蛮克戎指掌上(三)}

【第三更,求红票,收藏】

昨天韩冈卖了个关子,并没有说出他的计划。只是留下了一句话,让王韶王厚等上一天。王韶能耐得下性子,而王厚却做不到。虽然他学着他父亲的模样,硬是等了一夜。可到了第二天,便再也忍不住,就想过去找韩冈,打算问个明白。

谁知道,韩冈没等王厚去找,便主动上门。在韩冈手上,王厚并没看到什么锦囊妙计,而是见到了一个四十多岁的汉子。那汉子脸上的皱纹如条条深沟,沟壑间还带着尘土,名副其实的灰头土脸。

“玉昆,他是谁?”王厚低声的问着。

韩冈反问道:“不知处道兄听没听过邠州田家?”

“邠州田家?没听说邠州有田姓大族啊。”王厚低头想了半天,终于想起来邠州田家的田,是哪里的‘田’:“就是那个卖泥人的田家?!”他奇怪的问着,韩冈的计策,跟做泥人的田家有什么关系?

田家的泥人倒的确卖得高价,一对往往价值数贯,而一套七只,那就是十几贯才能拿下,相当于几亩地的价格。王厚曾经想给自家留在老家德安的弟妹捎几个过去,但一问价格后,当即打消了念头。

但泥人价格再高,也不可能跟韩冈说的扯上关联。王厚立刻怀疑起自己的猜测,摇头道:“不可能是泥人田家。”

“不,小弟说的正是邠州的泥人田。”韩冈伸手向王韶和王厚介绍:“这位田员外,就是邠州田家出来的远支子弟。”

“田计拜见王官人,王小官人。”田计上前向王韶和王厚行礼。

王韶脸上看不出什么异样,他知道韩冈不会在正事上乱开玩笑。韩冈带田计过来,必然是有大用的。欠了欠身,示意田计坐下来说话。

王厚则是又深深的看了田计几眼。还是四十多岁诚惶诚恐的乡农模样,横看竖看都没有什么特别的地方,也就身上的衣服应该是贵价货色。听韩冈称呼他田员外,显然他颇有些身家。

但这于王韶所面临的问题有又何干?

“机宜和处道兄还记得春牛吧?这十年来,每年祭春用的春牛都是田员外所亲制。”韩冈坐下来,继续介绍着田计这个人。他相信王韶、王厚能记得起来立春祭典上的春牛。

王厚回忆起几个月前在城南看到的祭春春牛,被百姓哄抢之后,就剩下几块土而已。但王厚还是不明白韩冈带来田计,提起此事究竟是为何?

“玉昆,别卖关子了,快点说啊。”王厚催促着,他是心急难耐。而王韶虽然没说出口,但他略略前倾的姿态,也暴露出了他心中的急不可耐。

韩冈笑了一笑,揭开谜底:“昨天韩冈已经说过了,要想让天子相信机宜的话,就必须让天子更加了解秦州地理。不过机宜也说了,用舆图是不行的,天子不一定能看得懂,而且地图上也分不清山岭和谷地。所以给天子看得东西,必须直观清楚,易于理解,而且一目了然。”

王厚猛然惊起,指着擅长雕塑的田计,张口结舌问着韩冈:“玉昆的意思是?”

“玉昆是打算用泥塑一个有山川城池的舆图出来?”王韶慢慢的问着。

韩冈点点头,他要做的就是沙盘。虽然韩冈并不知道如今实用化的沙盘究竟出现没有,而且沙盘的原型在史书中都能找到,但他能确定,至少秦凤路上是没有的。

“将秦州山脉河流城池关隘重现于桌案之上,呈于天子御前,想必天子也不会再惑于窦舜卿之辈的污蔑之词。”

韩冈将自己的想法解释过后,又向王韶父子推荐田计,“不过若想做到这一点,非田员外的手笔不可。田员外家学渊源,立春之日,一头泥牛塑得与真物一般无二。如此塑工,是制作沙盘的不二人选。”

想把沙盘做得能吸引住天子,技术上光靠韩冈这样的外行是不成的,须得要找专家来做。当昨日韩冈起了制作沙盘的心思,第一个想起来的就是把春牛雕得活灵活现的工匠。

虽然只是邠州泥人田的远支,但田计技术不在本家之下,靠着手艺,他也是饶有身家。寻常也被人称一句田员外。但田员外如何比得上田官人?韩冈昨夜直接找上门去,与田计一番分说,并许诺道,“蕃人李定献偏架弩,官家亲自提名为神臂弓,李定也因此而得官。若田员外能将此事办好,其功不在神臂弓之下,少不得一个官人身份。”

田计就这么给韩冈钓上了钩,而王韶听到韩冈在他面前一说,也点头道,“此事之功绝不在神臂弓之下,若田计你用心将此事办好,本官必保你一个官身。”

一个是不费吹灰之力便将陈举族灭的韩冈,一个是使计将都钤辖向宝气中风的王韶,两人都是秦州城中口耳相传的奢遮人物。他们都做了保证,田计哪有不信的道理。

当天晚上,得到韩冈的指点,还有王韶私下收藏的秦州舆图,田计便留在王韶家中,使人回家拿了工具和惯用的软泥来,秉烛赶工。第二天清早,就给他拿出了个原型出来。

三尺见方的木板上,用软泥塑成了秦州山川的模样,无论是渭水藉水,还是秦岭六盘,又或是秦州州城,缘边百寨,都在沙盘之上得到了标识——王韶、王厚这两年走遍了秦州内外,有他们做监工,这块沙盘的正确性却是比任何舆图都要更高。

王韶站在沙盘前,俯身下望,一览山川。对韩冈笑道:“祖龙‘以水银为百川大海,相饥灌翰,上具天文、下具地理’,如今不必去问祖龙,只看这眼前三尺,便是河山一隅。”

韩冈回道:“马伏波‘聚米为山谷,指画形势,开示众军所从道径往来,分析曲折,昭然可晓’,故而光武曰‘虏在吾目中矣’。”

王韶捻须长笑:“若将此呈到天子驾前,是非利害,便亦在天子目中矣。”他又对站在一边的田计道,“也是多亏了田计你,要不然,不会如此顺利。”

田计辛苦了一夜,已是精疲力竭,但听到王韶夸赞,当即精神一振,拱手谢道:“多些官人夸赞。”接着却又叹了口气,“不过泥塑不易精雕,有些细处难以塑出。最好还是用着蜜蜡混着木屑来做。”

王韶闻言,扭头看了一眼韩冈。韩冈会意点头,“今天我就去把这两样都弄来。”

“最好多找一点来。”王韶提醒了一句。

“韩冈明白。”他点着头。这三尺沙盘,本就只是个初步的模板,看看效果而已。要想打动天子,必须要制作更为精细的沙盘。

韩冈相信,只要把制作精美的沙盘送到赵顼面前,窦舜卿说什么赵顼都不会相信了。任何言语和文字,都不如实物更有说服力。

为什么韩冈在另一个时代做的工作报告,都由文档改成了幻灯片?还不是因为图表比文字更要直观的缘故。打口水仗难以取胜,但换成更直观的沙盘模型,相信会给赵顼耳目一新的感觉,而大大增强王韶这边说话的可信度。

“今次给天子做个沙盘是为了跟窦舜卿争口气,不过沙盘更大的用处却是给将帅们使用。不管从哪个方面,沙盘都比地图管用。”即使只看到了试作品,王韶就已经能确定,给将帅们运筹帷幄带来什么样的帮助。

“机宜说的是。不过为了给天子御览,有些地方还得再强调一下。比如古渭这边的山谷,应该更大一点……”

“玉昆!”王厚听着一惊,“古渭所在的山谷没这么大!”

“二哥,你要知道,这是给天子看的。得让天子知道自渭源至秦州,河道究竟有多长,河岸两边的土地有几何……”

王韶没有继续说下去,但王厚听明白了。给皇帝看的东西和给将帅看的东西是不一样的。给天子看,是为了得到他的支持,内容上当然得有所取舍,而给将帅看,则是为了打胜仗,必须准确无误。

其实要把沙盘做得标准,与实际相符,必须要把等高线地图画出来。可韩冈对此只是很粗浅的了解,虽然那是沙盘模型的基础,但不知道就是不知道,只能日后慢慢琢磨了。只是就像奏章一样,反正给皇帝看的,真假都无所谓,关键要有说服力。

这些话各自心里明白,却是不能说出来。王厚会意的笑笑,就看着田计照着韩冈的意思去修改。

“这沙盘还是小了点,只有再大一点才能让人看得清楚。”韩冈提着要求,沙盘不能小,太小了就不能体现出王韶的万顷荒田的存在。

“但太大了又不好运输,一路颠簸,送到东京城时早就坏了。”王厚则摇头说着。

“那好办,分割成片。送到地头后再一块一块的拼起来。”田计卖力的出着主意,“大不了多做两套,到了东京捡没坏的拼在一起。”

“最好是田员外随着沙盘一起上京。”韩冈对王厚道,“机宜和在下都不便擅离职守,不过处道兄却能走得开。不如让处道兄押送托硕部一众首酋去东京献俘,顺便与田员外一起把这个沙盘送去。”

“玉昆,这可是你的功劳,真的要让给二哥?”

“处道兄和在下何分彼此,也没什么舍不得的。”

“玉昆!”王厚感动至极。

王韶则长笑道:“玉昆的功劳不能夺,在沙盘模板后,刻上玉昆和田计的名字,这样谁也夺不走。不过,让二哥儿也附个名,沾沾光也好。”

