\section{第12章 平生心曲谁为伸(二)}

【第三更。不好意思,年前事多,现在才赶出来。】

“陛下不理国政,沉湎于游戏之间,通宵达旦,不知昼夜,长此以往,将如天下何?!将如百姓何?!”

文彦博说得痛心疾首,在他看来赵顼在军棋上浪费的时间实在太多,武英殿里的那种玩意儿,实在应该放把火烧掉。

赵顼沉默的听着,越听越不是滋味,肚子里咕哝着的满是腹诽:

‘朕一文百姓膏脂也没乱花,也没有纵情恣意的游宴享乐,只不过摆弄一下沙盘而已,就算通宵达旦也只不过就一天而已。你文相公没少游宴,看到大雪就坐在亭子里连喝三四夜赏雪酒,喝到守卫的士卒气起来烧亭子。一场兵变侥幸被你压下去,就称为是名臣手段,但你不喝酒不就没这回事了?’

文彦博说完,又骂起韩冈:“韩冈不过是灌园之后,素无才学,又性刚好杀。王韶爱其奸狡,荐他为官。天子不以其卑鄙,为他亲下特旨,擢其于布衣。可韩冈不思殚精竭力以报君恩,却心怀诡谲,示人以诈术。都钤辖向宝为王韶所欺,以中风疾。王韶事后奏功,便道韩冈为之赞画。今韩冈又献游戏之物以诱天子疏离朝政,如此奸佞之辈,如何可用之为官!?”

文彦博把韩冈说成是混入官员之中的奸佞小人,要逐之而后快,赵顼根本不去理会。韩冈的才能、人品明明白白的摆着,他对此清楚得很。救人之后,不留姓名便洒然而去,如此任侠之辈,岂是小人?

韩冈帮王韶出谋划策,为得是国事,又不是私利。而他献上的沙盘军棋,一开始就说是给将帅所用,并不是给天子的玩具。

赵顼知道,他的这位枢密使只是莫名其妙的讨厌韩冈。

托硕之捷,王韶在奏报中称韩冈有赞画之功,但枢密院却弃之不录,反而要定他欺瞒主帅的罪名,而韩冈也的确没有参与战斗,而是跟在向宝身边,最后他的功劳便不了了之。

赵顼对此心中有些不满,但枢密院已经定下功赏,中书那边也没有反对,他也不好为一个从九品出头——那样太骇人听闻——所以他把韩冈的名字写在屏风上,想等着有机会把封赏补给他。

等到王厚入京,献上了韩冈首创的沙盘和军棋。赵顼一览之下,便为之大喜。他知道两者都是军国之器,韩冈编订的军棋规则虽然简陋到可笑,但修改后,却也是培养将帅武臣的好道具。

赵顼要为此提拔韩冈,甚至想把他调进京来。因为这幅秦州山川的沙盘,同时也让他明了了,在荒田之事上究竟是谁在骗他——支持窦舜卿的,到现在都没能拿出一个可信的证据来。而三百里河道,怎么看都有一万顷田——让天子不受臣子所欺,这是韩冈的功劳。

但文彦博又是横加反对。赵顼在刚拿到沙盘和军棋的那两天,通宵进行军棋推演的事,便被他当作证据来攻击韩冈的发明实是一桩祸害。

赵顼都有些奇怪,为什么王韶的儿子王厚同样是因攻灭托硕和沙盘军棋之功授官,文彦博却只提了几句,却对韩冈穷追猛打,硬是压着他,不给他出头。文彦博可是连张守约升任秦凤路钤辖的事也没这般激烈的反对过。

堂堂枢密使跟一个从九品过不去,赵顼都觉得有些丢人。而跟着文彦博一样,对赵顼玩通宵看不惯的几个御史,也一起上奏。不过他们的谏章中,却是骂赵顼的居多,而对韩冈只是提了寥寥两句——骂一个从九品,他们也觉得丢人。

真不知道究竟是怎么回事,赵顼纳闷得紧。只是有文彦博反对,韩冈的功劳就始终没有被确认下来。到了今天,王厚得了官都要出京回秦州了,但韩冈仍是做着他的从九品。

文彦博还在骂着,目标已经从韩冈又转到了王韶身上,又骂起王韶对同僚使计,故意害了向宝。

赵顼听了几句,心中越发的不痛快。河湟之事可是他亲自批准的,王韶也是他当先提拔的。他看了看王安石,但他的这位参政到现在还是保持着沉默。赵顼不耐烦了,亲自下场,道:“向宝与王韶素不相能,对河湟之事多有阻碍。王韶能以蕃部平蕃部,他身为管勾蕃部,却要统领官军去进剿……”

文彦博眉毛一挑,他等得就是赵顼的这一句,音量陡然拔高:“就是王韶以蕃部平蕃部才闹出今日的事来!”

“王韶身为秦州西路蕃部提举,不能安定蕃部,却好大喜功,致使木征、董裕攻打古渭。亲附朝廷的各家熟蕃前日为王韶所诱,齐攻托硕,而今日便遭木征、董裕报复,各部无不残破。试想日后,看到七部的结局,秦州蕃部又有哪家再会来投效朝廷?!”

文彦博得意的攻击着王韶,前两日收到的紧急军报成了他手上最好的武器。朝臣都在沉默着,殿中除了王安石,吕惠卿和章惇三人,其他人都无心为王韶辩解半句。

章惇看着文彦博唇枪舌剑的骂着王韶,连带着敲打王安石和天子。又看着王安石的眉头越皱越深,心道王相公应该快忍不住了,就跟自己一样。

吕惠卿则是心平气和的听着,文枢密最近的调门很高,抓着一件事,就扯起来大骂,他是不得不如此。要不再闹出一点事来,把人心聚起,枢密院的权力可就要在他手上被割走一大块。

王安石最近做了个釜底抽薪的事。他上奏请求设立审官西院,将原属枢密院的高阶武臣的任免权和管辖权,转给审官西院负责。而原来负责文臣京朝官的审官院,则改名为审官东院。

按照王安石的说法是‘枢辅不当亲有司之事’,言下之意,就是既然政事堂并不直接管理京朝官,而是要审官院从中过一道手,凭什么枢密院可以直接任免七八品高阶武臣?——六品以上官员,无论文武都必须由天子过目点头,这是哪一边都插不上手的。

一旦天子同意王安石的提议。自此之后,官员的铨选之职将分为四个机构:主管京朝官的审官东院和主管选人的流内铨,负责高低两级文臣;主管内殿崇班至诸司使的审官西院以及主管大小使臣的三班院,负责高低两阶武臣。

枢密院对武臣的人事管辖之权,现在是文彦博压制在边境军州任官的武臣,不让他们跟着天子一起闹着开边拓土的重要武器。而一旦设立审官西院,他就再无法让那些武夫听他的话,上书反对一动刀兵。同时,枢密院一直控制了上百年的权柄在文彦博手上被划走,对他的声望也是一个重大的打击。

所以文彦博现在要拼命,行事说话毫无顾忌。

王安石这是为了回敬文彦博他们对三司制置条例司的攻击。三司制置条例司这个新生机构,从一开始就主管着变法大局,被反变法派着力攻击,言其无故事无先例,应当将其撤销。

在御史们的攻击下,王安石也不得不同意撤销三司制置条例司,将其人员归入中书。但他们却乘势改以六部九寺中的司农寺来主持变法政令,实质上却更加名正言顺。

但反击是少不了的,枢密院就此成了目标。

朝堂上的事务没有一件不是牵一发而动全身,吕惠卿看得很清楚,河湟之事不是光凭殿中两方扯一通就可以处理的,纠葛实在太多了。除非王韶那里出大篓子,不然,文彦博怎么攻击都没有用。所以他很平静,根本就懒得插话。

但赵顼难以平静,而王安石也难以平静,当文彦博的调门越来越高,王安石背一挺,就要站出来。

但这时,一名内侍双手托着一份奏报,跨进外殿的大门,高声道,“陛下,秦州急报!”

各地的奏章、文字一律是发往通进银台司,然后由通进银台司按不同类别分发到政事堂、枢密院或是直接呈于天子。不过一般来说,只有动用了急脚递或是马递的紧急信报,才会直接放到天子案头上。普通的文字,都是由两府自行处理,该转发到转发,该批奏的批奏,等到处理完毕,再把其中重要的分拣出来,奏于天子。

而秦州、绥德等缘边四路的军情,是赵顼钦点,一旦发进银台司就直接送入宫中。如果是西贼主力入寇的消息,就算他已就寝,也必须把他叫醒。

赵顼正被文彦博劈头盖脸的训着,虽然唾沫星子没溅上脸来,也不像仁宗皇帝那样‘差点被臭汉熏杀’,但也是够让他憋闷的。一听到秦州急报,他便连声说道:“还不快呈上来!”

天子要看急报,臣子也不能耽搁。赵顼低头看着军情,方才几乎要把崇政殿的琉璃瓦都要震下来的声音也静了。

文彦博躬身退回班中,四平八稳的站定。以他的身份可不怕赵顼能把他怎么样。再怎么说,他所经历过的几个天子,都是怕在青史上留下拒谏的坏名声,而不会对臣子言语上的冒犯而当庭动怒。

就是不知这封秦州来的新奏报究竟说得什么,是不是古渭出了事情。文彦博暗自冷笑了一下,若真的如此,他这个枢密使可是要说话的。

