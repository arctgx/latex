\section{第41章 千嶂重隔音信微(下)}

“还是没有消息。”

面对沈括的询问,韩冈摇了摇头。已经快十天了,自从王韶领军进了露骨山后,只在第三天有一人带回来王韶的密信,说是正在顺利前进。但从那天之后,到现在就再也没有一个准确的消息传回来。

“会不会出什么……”沈括欲言又止,下面的话不能乱说。

“露骨山崇山峻岭,林深草密。进去之后,当然不容易将话传回来。在下已经派人去岷州了,从那里得到洮州的情报,还是要快上一点……存中兄不必太过忧心!”

沈括看着只有自己一般年纪的韩冈,沉稳得根本不像一名年轻人。而且同在狄道城中的这十来天,他更是亲眼看到了韩冈处置事务事的表现,衙门中积年老吏都很难比得上他。好几次见到韩冈一边跟人说话,一边批奏公文的场面。分心二用的情况下,两边却一点也不乱。这份治才,让沈括也不免要赞叹一二。

而眼下表现出来的心性,则越发的难得。早慧的所在多有,沈括自己就是。但心性老成,做事举重若轻的少年人,世间却是少有得见。就如他曾见过的王家大衙内,才学不差,名气更大,但行事可就要逊色韩冈许多了。

不过看着这样的韩冈,沈括的心里更是不喜欢。这样下去,他根本找不到插手经略司军务的机会。难道他沈存中巴巴的赶到河湟来,就是为了摆着算筹,来计算钱粮的吗?

但韩冈现在把事情做得滴水不漏,沈括一时间也找不到机会,干笑了两声:“既然玉昆你这么说,那就再等等,希望王经略吉人天相……能马到功成。”

正在说话的时候,忽听外面的卫兵来报,说是王中正王都知来了。

韩冈和沈括起身走到厅门外,迎着王中正进来。

王中正找韩冈有事。行过礼,他便板着脸问道:“韩机宜,临洮堡临洮堡那边传话过来,说是出城樵采的士兵被蕃人杀了十好几个。景都监说是要出兵,为何机宜你移文去阻止?”

韩冈一听,心头顿时大怒,继而又是一阵疑惑,什么时候王中正的手伸得有那么长,耳目有这么灵敏了?他才把批复的文字让人移送临洮堡,这监军就杀上门来了?

心中虽是不快,但王中正眼下毕竟是名正言顺的压在韩冈头上。他不得不按耐下性子,向王中正解释道:“禹臧花麻其人狡诈无比,不会闲得无事,便杀樵采之人来解闷。多半是有什么阴谋诡计要施展,一个不小心,说不定就会落入他的陷阱。。”

“禹臧花麻不是退兵了吗?!”王中正质问着。

“但禹臧家的老巢就在兰州,才百多里的路程,夜里回兰州喝酒吃饭,第二天就能又赶回来。”

韩冈说得有趣,王中正笑了两声,继续问道:“那韩机宜你说禹臧花麻会有什么阴谋诡计?”

“不论禹臧花麻转着什么主意,只要以不变应万变,守着临洮堡就够了。”韩冈可不会随便乱猜测,万一说错了,话语权便会有所损失——王中正……还有沈括,都在这边虎视眈眈呢——只有一些颠扑不破的道理,才是眼下该说的话。

“但樵采多被杀,临洮堡该怎么办?总不能不开伙吃饭吧?”王中正反问道。

“樵采被杀,那就不要向北去砍柴,改去南边砍柴好了。这几天吃的亏,终有报复回来的日子,眼下不是置气的时候。”韩冈坚持着要维持河湟的稳定局面,王韶消息不明,河州城哪边正在清理周围木征的亲信蕃部,熙河路再也动荡不起,“不知都知能不能让景都监安稳一点,一切等经略回来再说?”

“这可不好办。”王中正很是为难的模样,“中正虽然奉旨前来监军,但终究还是一个外人啊!”

见着王中正边说话,边瞥眼看自己。韩冈心神一凛,知道前面自己说错话了。王中正现在是趁着话头,要让自己承认他的指挥权!——不,不是自己说错话。而是王中正过来时,就打着这个主意,只是自己不觉察间被他引了过去。

想要帮着压制景思立很容易,承认他王中正拥有指挥全局的身份就可以。

这么可能!

承认一个阉人指挥众军的权力,他韩冈还要在文官的队伍中混迹吗?沈括在旁边都变了脸。

‘嗯?’

韩冈突然很奇怪的看了沈括一眼,他怎么不说话?

一般的文官不是应该在这时候将话题引开,或是直接叱骂吗?——两种做法就看各人对阉宦的厌憎程度了——但沈括却不开口,只是脸色稍稍变了一下,难道是要看自己的笑话?!

韩冈心头多了一阵猜疑,更多了一点怒意,王韶这么一走,牛鬼.蛇神全都蹦出来了!

只是王中正的进攻还是要应对的。却不是同意或是反对,而是叹了口气,低声说了句‘这就不好办了’。又猛然抬起头,“即是如此,那韩冈不敢让都知为难,还是再给景都监写封信去,述说利害吧。希望景都监能听得进去。”

韩冈顺着王中正的话,将他本人的逼宫轻轻卸到一边去。韩冈宁可让景思立出兵,也不会让王中正能够指挥全军。两者的性质和危害完全不同,他可不敢在自己手上开这个口子。天子下令倒也罢了,自己把宦官请来主持军事,要被天下的士大夫戳脊梁骨的。

事办砸了,日后还有改正的余地。但名声臭了,可就再难以挽回。

王中正不意韩冈如此说话,盯着韩冈一阵,见到他始终没有半点改口的迹象,黑着脸站了起来:“那就照玉昆你说的去做好了。希望景思立能听得进去!”

“也只盼望如此了!”韩冈虽对此不报希望,也只能顺口这般说下去。总不能说,景思立必然会把劝告放一边,去出兵挣功劳。

他起身送了王中正出去,回来后对沈括叹道,“真真是让人闲不得啊!”

沈括也叹道:“幸好玉昆你没有搭他的话,不然可就要出大乱子。传到京中,御史台都不会放过。”

话声一入耳,韩冈登时又是疑惑起来。这马后炮不该说的啊……现在说出来,反倒让人以为他是因为软弱,而不敢当面指斥,只敢在背后说话。这还不如一直装傻.比较好!

韩冈想不明白沈括为什么这么做,只觉得他的做法还真是有些让人摸不着头脑!

沈括又说了一阵话,也起身告辞,他本是来问军情的,既然没有消息,当然就得回去做他自己的事。粮秣转运虽然没有之前那般辛苦,但同样还是一桩繁重的工作,不论是韩冈、还是沈括,都不能离开岗位太久。

沈括走了,官厅中重又清静下来。随侍的亲兵端了待客的茶下去,又给换了一份滚热的茶汤。

喝着煎煮后的热茶,韩冈闭起眼睛盘算着。

这些天,二姚十分卖力。在灭了两家之后,河州城那里已经有六家蕃部宣布臣服。跟随官军出战的蕃人,就像滚雪球一般会越来越多——河州那边可以安心的不用去管。

至于岷州,哪边肯定是要派军去了。两千人的粮草他已经备好了,看看王舜臣能不能领本部走一趟。如果木征跟王韶在洮州打起来,这一支队伍就能起到决定性的作用。

至于临洮堡的景思立,韩冈并不报希望。吐蕃人设下的诱敌陷阱是很明显的事,景思立多半是知道的。但姚兕姚麟两兄弟正在河州那里建功,为了与他们一较高下,景思立很可能会将计就计,硬是踩进陷阱去。就不知道到时,是吃还是被吃?——韩冈摇了摇头,还是提醒一下吧,也算是尽到一份责任。

接下来的几天,韩冈……不,应该说整个狄道城,甚至整个熙河路,关中,直至东京城,都在等着露骨山那边传来他们所期盼的消息。四月已经到了,但王韶那边还是没有消息。

这一日的午后,一名信使慌乱的冲进韩冈措置公务的官厅。韩冈为之停笔,当他听过信使上气不接下气的报告之后,闭了闭眼睛,然后命令下面的亲兵道:“去将王都知、沈中允还有王都巡一起请来。”

当王中正、沈括、王舜臣闻讯过来的时候,韩冈就站在庭前的院中,仰头看着北方的天空。

看着韩冈的动作神情,两人便知事情不妙。沈括立刻问道:“玉昆,出了何事?!”

韩冈叹了口气,回头道:“景思立妄自出战,在河外遭遇伏击,眼下已经兵败身死,出战的三千将士也几乎全军覆没!”

王中正和沈括乍听噩耗,脸色突的都白了。王中正甚至摇摇晃晃的,差点站不稳身子。王舜臣先一步恢复过来,追问着:“临洮堡怎么样了?!”

韩冈转头望着北面的天空。临洮堡和结河川堡都是新近修筑起的堡垒,而两座寨堡周围的防御措施都没有时间继续修筑下去,就连最基本的烽火台也同样欠奉。如果应该就是烽火连天,满目黑烟,直上九重云霄了。

“消息传回来的时候,临洮堡尚在坚守之中,但现在已经不知道了。”韩冈对着王中正和沈括,“临洮堡事关大局,不得不救。狄道的事,就得拜托两位了。”

他转向王舜臣:“你跟我走!”

