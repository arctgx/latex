\section{第六章 见说崇山放四凶(13)}

夜阑人静。

吕嘉问尤未入眠。

已经是三更天,他仍坐在桌前,在煤油灯下读着书。

他连着两个晚上都熬了夜,眼白上密布血丝,红得如兔子一般,但精神却反常的旺健。

每隔片刻,吕嘉问便会将手上的书册翻上一页,但这本早就倒背如流的《三经新义》,他却半点也没看进去。

若不是出自王安石、吕惠卿等人之手,又是新党的核心理论,这等枯燥无味的书又有什么好看的?

自从入朝为官以来,除了《三经新义》出版,以及道统之争最为激烈的时候,他连九经也没有再翻过。

但为了能与王安石、吕惠卿关系更紧密,吕嘉问当初在三经新义出版后,只用了三天便将十万字的著作,硬是从头到尾给背了下来。

在吕嘉问眼中,这世上的东西截然两分,于己有用,以及于己无用。

而人,也一样如此。

在还没等到一个有用之人的回复前,吕嘉问就算躺到床上,也是一样睡不着。

油灯中灯油一点点的减少,但吕嘉问等待的消息却始终不见回音。最后他烦躁的将手上的书丢了下来,呆然的望着窗外。

不知过去了多久,昏沉的纸灯笼照亮的走道中,终于有了一点明亮的光芒。

透过玻璃窗,一盏灯笼飞快的接近吕嘉问的书房,而灯笼后的光影中,两条人影疾步前行。

很快,门外传来唤门声:“学士,何二回来了。”

吕嘉问停了一下才出声回应:“进来吧。”

“学士。”何二进来后行了礼,便递上一封书信:“这是黄侍制的回信。”

“嗯。”

吕嘉问的神情出奇的平静,完全不见之前的烦躁。只是伸手从何二手上接过回信,却仿佛强抢一般。

只是展信一看,吕嘉问便难以自抑叫了一声,“好!”

‘欲将何物助强秦’,仅仅是王安石的一句诗。但已经说明了黄履的态度。

何二垂首待问,聪明的不去关注主人的失态。

吕嘉问兴奋了须臾片刻,便放下了信,和声问道:“黄安中还说了什么?”

“黄谏院看了学士抄的王平章诗,就一直在说王平章诗词好。不过之后还拿了苏轼的文集,说了这一回苏轼死不足惜,可惜了他的诗文要受牵累了。”

“什么《文集》?”

“《钱塘集》。”

吕嘉问嘴角微微扯动,在灯光下露出了一个嘲讽的微笑。黄履既然如此有自知之明,那就当真可以安心了。

“先下去吧,明儿去账房领两贯钱。”

“谢学士赏。”

家丁千恩万谢的退了下去。吕嘉问从桌上抽出一本账簿,打开来端端正正的将这笔赏赐先记了下来。

当吕嘉问可以坐下来的时候,一股安心感涌上心头。

龙图阁侍制、知谏院黄履,这是第七人。

现在离廷推之日还有一段时间,到了那时候,吕嘉问有信心保证有十人支持自己。

要挟,请求,交换,吕嘉问相信自己能使用的手段,比起李定更强一些。至于沈括等人,那就更不是一个等级,完全不能拿来做比较。

只要这两日的情况持续下去,吕嘉问不愁成为不了排位最靠前的候选人——只要黄履这样的人更多一点就行了。

黄履一向与蔡确交好,而且是非同一般的好,据说黄履已经和蔡确之弟蔡硕为子女定下了婚事。

如果不是韩冈那一骨朵,黄履事后少不了会水涨船高。当然,也是韩冈那一骨朵挥得太早了,迟个半日,黄履就是蔡确逆党的一员干将。

但现在蔡确家烧了个干净,书信等可以作为罪证的凭据都化成了灰烬,黄履只要将自家的书信给烧了,再将婚贴给烧了,也就彻底的没了罪证。

蔡确作为宰相,每日写信,车载斗量。但凡只要能拉上一点关系……好吧,就算拉不上关系,也照样不知有多少人写信给他,以求能得到宰相的看重。如果这批书信给翻找出来,多少官员都要,就算可以自辩清白,但到了晋升的时候,与他人竞争,只要有人说一句他曾经给蔡逆写过信求过官,那这件事就算是完了。

所以不管王厚日后怎么犯下大错,只是他坐视蔡确家人纵火,又拖延不救这一条,在朝堂上不知要受多少人感恩戴德。

只不过黄履一贯借用蔡确的地位,这世所共知的。黄履在谏院和朝堂上,已经没有了立足之地,早早的就在转着请郡外放的想法,只是光是请郡外放,背后没有实权人物遮风挡雨,外放的位置很有可能逐渐南移,直至岭南等荒芜瘴疠之地。比起常为冤家对头的李定,吕嘉问当然更受黄履的欢迎一点。

鼻子里哼着不成调的曲子,吕嘉问翻开一本笔记,在中页上写上了黄履的姓名。

‘快没有人了。’

吕嘉问心情松快的想着。剩下的那群人中,韩冈不可能找得到多少支持者。

十三天后就是廷推之日,能够参与到其中的名单将会比现在更长一点。

因为这份名单并不局限于在朝堂内任职的重臣,就算是回京诣阙,但只要是侍制以上官就能够上殿进行推举。

吕嘉问确认过这半个月内即将回京的侍制名单,在那三人中找不到一个能够确定支持韩冈的人选。

论身份,论地位,还有威望,韩冈别说进入廷推的前三人,就是排在第一。

只要他能够登门造访,或是仅仅是写几封书信,都能将一些中立甚至明确属于新党的重臣拉到身边,至不济也能起到威逼的作用。除非王安石能够明确的站出来表示反对,否则其他人在韩冈的威势下,都得向他低头。

但这需要韩冈为此付出一定的努力。这世上,没有一点辛苦不费,便能达成所愿的好事,有人先天上就超人一等,可世上超越常人者为数众多,他们之中,并不是所有人都能够有所成就。

在吕嘉问看来,可喜可贺的一件事。

或许其心有顾忌,或许其根本就没有做宰相的打算,韩冈对自己提议的选举廷推,没有表现出半点兴趣。

到现在为止,吕嘉问还没有发现韩冈有任何寻找同盟者和支持者的表现,所有在京的侍制重臣,都没有表态要支持韩冈。

吕嘉问并非一厢情愿,他对此还是经过了一番调查。尤其是为了联手阻击韩冈,作为御史中丞的李定,将他的权限发挥到淋漓尽致。

据李定调查,韩冈与外界的联络,这几日并没有大幅增加,甚至减少了不少——多半是为了避忌人言,免得为人嘲讽讥笑。

此外,在两府之中,除了苏颂之外,就找不到其他支持韩冈的宰辅了。王安石就不用说了,章敦也完全不表态。

章敦跟韩冈的关系是不错,但从宫变之后,章敦与韩冈的交情就日渐疏远,虽然表面上完全看不出来。但宫变当日朝会后的反应,吕嘉问能看得十分清楚。而且章敦在这件事上不表态,就已经将态度表现得极为明确了。

上至宰辅,下至重臣,能够给韩冈助力的人选越来越少,到了最后,连翻盘的机会都不会有了。

但吕嘉问还是要确定一点,必须要让韩冈进不了前三。

大体上,这一次的廷推,有一个难点必须跨过去。

韩冈的提议,并不是选出来便能够就任,而是必须要太后从三名被选中者里面再挑选一人出来。

极端点说,如果二十六票中有十三票选吕嘉问,十二票选李定,只有一票选韩冈,但只要韩冈是在前三之列,那太后也必然会选择给韩冈一张清凉伞。

一旦韩冈在三人之内,那么结果就必然注定,其他人就都可以去睡了。谁能争得过他?

不过一旦韩冈名讳出现在三人之外,情况就会陡然不同。那时候,就是韩冈本人,也别想改变这个结果。

吕嘉问并不担心太后会否决这样的一次没有韩冈名讳的选举。

这个廷推提案是韩冈提出来的,如果太后直接否决,一个不选,那么丢脸最大的还是韩冈——多一番波折完全是画蛇添足,到最后,一切还是要秉承太后的心思。

幸好韩冈太过托大,他的自负,让他没有去联络一众重臣,仿佛他天然就应该成为宰辅。可是其他人都不这么想。如此一来,莫说是第三,就是第四也不是不可能。

“学士。小人有事禀报。”刚刚离开的何二突然又书房外面叫门。

“什么事?”吕嘉问让他进来。

“小人今天在外面听到一些谣言,方才忘了说。”

“什么谣言?不算重要的就明天再说。”

吕嘉问没什么精神的摆了摆手,黄履一确定,通宵了两天的疲惫便彻底的爆发了出来。

“嗯……学士,小人不知重要不重要,只是之前去奔丧,却是听见有人在议论学士。”

“说,快点,”吕嘉问催促道。

“就是有人先骂学士,然后另一人又抱怨,又是三个南人。”

吕嘉问闻言一下跳了起来,然后稀里哗啦一阵响,,桌子椅子都给他带翻了。

他脸色铁青,“什么南人北人!”

