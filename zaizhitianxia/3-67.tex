\section{第24章 携眷西返家(下)}

【有事耽搁了一阵,没能赶在中午时发,真是抱歉。这是昨日的第二更。今天两更照常】

四月中旬,从历法上已经算是初夏。

窗外已经能听见蝉鸣,正午的阳光从南窗投射进来,使得屋中也带了一点暑气。

王旖从外面端了一杯凉汤进屋来。

她穿着一身鹅黄的襦裙,外套一件无袖褙子。内外都是棉布缝制,比起单薄的丝绢来,布纹经纬要粗上许多,但穿到身上也更为保暖一点。不似丝绸衣服要一层层裹得紧,以如今的气温,内外两件就够了

韩冈手上正拿着颗珠子,对着阳光,一闪一闪的,耀着王旖的眼睛。

王旖走过来,放下为韩冈准备的凉汤,好奇的问着:“是琉璃?”

韩冈没回答,张开手将她搂在怀里。

一开始对此王旖还害羞得紧,但几天下来也习惯了这样的亲近。头枕在宽厚的胸膛上,换了个舒服的姿势,听着身后传来的声音:“还是唤作玻璃更确切一点。”

同样一件东西,天南地北的名字都不一样。这个时代,很少有人会想着将名词专一化,精确化。叫琉璃的,有的是琉璃瓦,有的就是玻璃。而叫玻璃的,有的指得是后世一样的东西,但另外尚有一种水玉也被称为玻璃。

韩冈手上的玻璃珠子,却是真正的玻璃。微微还有点发绿,但可以算是晶莹剔透,里面也见不到一个气孔,这也是将作监中名匠的产品,让韩冈为之惊讶不已——其实到了南宋,透明澄澈得能做鱼缸、花瓶的玻璃盏都已经普及开来,为此作诗写词的不胜枚举。透明的程度要超过波斯的舶来货,只是不耐热水,不能做杯子,只能盛鱼盛花

韩冈依稀还记得玻璃镜的制法。不是银镜——银镜反应的条件太苛刻——而是水银镜。用水银融了锡后镀在玻璃上,外面涂层保护漆就够了。以宋人工匠的手艺,给了他们制作的基本原理,三五年内应该就能又成果了。不过这个的先决条件,是弄出透明的玻璃再说。

所以他设法弄来个一颗玻璃珠,如今的市面上,杂色的玻璃或琉璃饰品很常见,透明的也有,但透明到能做镜子的看来只有宫匠。不过,要从宫匠手中拿到配方,

给献给天子,那是最蠢的做法。自己一人赚也是很蠢。最好的办法是组织人手起来入股。如果能早日将关西的豪族、商行组织起来,变成一个利益集团,对自己日后的发展有着不可估量的作用。关西豪族对棉布的渴求,已经可以从中见到雏形。不过熙河土地不足,棉田发展潜力有限,日后到了一定程度,便会停滞下来。

但玻璃、镜子不一样,相比起农业对土地的要求,工业就少了许多,到时候,能用工业带来的利益将他们捆到自己身边。韩冈前两天已经带了冯从义去过了种谔府上,事先多多联系,日后也好做事。一个稳固的根基是日后身居高位的先决条件,若是能成为一个利益集团的代言人,朝堂上永远都会有一个位置的。

看起来回去后,就要与那些土豪们多多走动了,现在以自己的身份地位,应该可以轻易的拿到主动权了。他们都有心在京师扩展,韩冈作为王安石的女婿,当然是个最好的选择。

……只是要打开京城里的局面可不容易。

已经到了夏天,地方州县都开始要忙碌起来,夏税的收取工作是每年的重头戏,而夏天又是雨季,雨多了有洪水,雨少了就是旱灾,只要是合格的地方官员,都知道这时候就要开始做好预防措施来。

而京城之中,自汴河,物价的确稍稍低了一点下去,不过另一方面,物价降低的幅度,远远不及旧时春来汴河水运重启后,南货一下打了五六折的情况。都四月往五月去了,情况比起韩冈估计得要差得多。

也许是自己小瞧了京城商人们的财力,要不然,就是市易务内部有问题,吕嘉问没管好下面人。但不论是哪一种情况,对于棉布在京中的推广完全没有好处。

“市易务……市易务……”韩冈将玻璃珠子放在桌上,指尖来回拨弄着。

昨天王雱来访,与韩冈说起此事,王旖在旁也听到的。见着韩冈心不在焉的念叨着,转头问道:“还在想着市易务的事?”

韩冈一笑,屈指将玻璃珠子弹开去:“不在其位,不谋其政。管不来的,也是白操心!”

“只是大哥还说要举荐官人……”

“我可不趟那浑水。现如今,吕吉甫和曾子宣明争暗斗,岳父怕是头疼得厉害。我要插足进去,你爹爹的头会疼得更厉害。”

曾布曾经一肩挑着十几个职司,不过因为吕惠卿的到来——更是因为不符合组织原则——他的权力被转移了一部分出去。现在,已经是翰林学士的曾布,官位虽仍在吕惠卿之上,可他在新党中却是很难再有以前那般一人之下的地位。看赵顼和王安石对吕惠卿的安排,甚至有将他越过曾布,提拔成新党第二号人物的意思。

而且经义局已经在紧锣密鼓,王安石兼任经义局提举已经是确定了的身份。不过王安石作为宰相,不会有太多时间,判国子监的吕惠卿和王雱拥有着实际的领导权。在韩冈看来,经义局加国子监类似于后世的中央党校,对新党的意义不言而喻。从未来来看,王安石一旦从宰相的位置退下来,吕惠卿很有可能继承他的位置。

这样的情况下,曾吕二人怎么可能和睦相处?不斗起来那就有鬼了。

韩冈没兴趣插上一杠子。除了经义局以外,他对于新党的各项事务暂时都没有涉足的想法。可惜经义局已经成立在即,而他此前的举荐去全然无用。韩冈和王安石翁婿之间看似和睦,但原则问题那是一点也不相让。

王旖在韩冈怀里抬起头,看着他坚毅冷冽的眉眼,觉得他和自己的父亲脾气其实很像。公事归公事,私谊归私谊,都不会因私废公,不能讲人情的时候,那就根本不去理会。

“等从陇西回来,就请一个州郡,做一任不管事的通判。”韩冈搂着王旖,对她也不隐瞒自己的想法:“前面已经做了一次通判,再任一任通判后,担任什么职位都方便了。”

虽然此前韩冈已经做过巩州通判,但那个职位只是附带而已,他当年主要工作,还是属于军事方面的机宜文字。真正地方治政的资历还是不够。没有地方州县的经历,入朝时,很难被安排上一个好职位。就算被安排上了,也少不了被御史和士林一顿口水乱喷。另一方面,也要考虑到王安石为避嫌疑,故意安排自己低一点职位。

与其这般麻烦,还不如先去熬资历,以掌握主动权。凭着韩冈的功绩,资历并不需要熬多久,一任即可,用一年半到两年时间走过场就行了,并不用熬满三年。他现在是第二任通判资序,再做一任通判后,就是有了知州的资格。以第一任知州资序,入朝之后,就能统管一个部门,而不是给人打下手。

王旖不知道韩冈想得有这么深,但她也希望韩冈能不要掺和进新党内部的纷争中。以自己夫君的性格,跟人起冲突时免不了的。

又过了几日,到了韩冈离京回乡的日子。

前一日韩冈夫妇先去王安石那边辞了行,又是大包小包的得了一堆礼物。三辆大车,主要是王旖的嫁妆,还有不少贺礼。

冯从义还要在京中稍留两日,汴河边这座院子韩冈订了一年的契约,正好让他住着。早上还没出门,王厚和种建中都到了。转头过来,吕惠卿和曾布也来相送,虽然朝中人人知道两人不合,但现在看起来还是一团和气。

吕惠卿一下马,就拱手对韩冈笑道:“玉昆回乡省亲之后,还是早日回京,天子可是正要大用你。”

韩冈连声谦逊,却也不以为意。

前两天,被赵顼以陛辞的名义召进宫中。说起来,真正要陛辞的,是朝官出外任官,要在离开前聆听天子圣训,所以才需要陛辞。如果是重臣,可以在崇政殿中说上一些自己对朝政的看法。若是普通的朝官,则是照常例,在朝会上说两句场面话就可以滚蛋了。而不论是进士或是朝官返乡,并没有陛辞的说法——从此事中可以看出韩冈得到的看重。

但天子的看重,也比不上家中的温暖。离乡半年,回去的时候,身份已然不同,而身边随行之人也已经大变模样。

随着在京中日久,韩冈越来越惦记父母,周南、素心、云娘,还有自己的一对儿女,不知他们现在可还安好。归心似箭,韩冈只恨不得能立刻回到陇西。

在城外,饮过饯行酒,与送行的亲友们告辞,韩冈翻身上马,当头领着车队,向西疾驰而去。

