\section{第18章 青云为履难知足(14)}

当年被逐出家门,不得不前往秦州讨生计的田计。在受到韩冈的启用后,命运就发生了改变。现如今是朝中管着沙盘制作的匠作官,官职虽没有多少变化,不过时常能面见天子,受到的赏赐也为数不少,已经让他在邠州的家族不得不向他低头。

经过了多年的锻炼,田计制作沙盘的技术很高,他手下的匠人们也都是行家里手,也就半个时辰的功夫,从广州至交趾再南下到占城,环南海周边地区的地形沙盘就一点点出现在武英殿的偏殿中。

“广南海边有红树。生长于滩涂之上,不畏咸水,蔚然成林。其数通体皆绿,唯有砍伐之后,故名红树。红树最奇特之处,是其树上怀胎,种子萌发于树上,发芽生根后方掉落于滩涂。”

“廉州合浦,以南珠闻名天下。廉州疍民率以采珠为生,只是采珠之苦,世间少有,每每取蚌三五只,才得上珠一枚。且采珠者极难长寿,年纪稍长瘫痪于床者为数众多。”

“交趾国都升龙府,旧名罗城,其后在富良江中得见黄龙,故而改名。其国官职仿自中国,都中禁军皆于额上刺字,号天子兵,此乃交趾国中最精锐者。”

“占城国中有大江,据说发源于大理,穿真腊,入占城。至占城后一分为九,汇入海中。据闻旧年交趾攻占城,在江边俘占城王。占城王献女方得脱大难。”

韩冈在向田计描述地形地貌的同时,也将广东、广西、交趾乃至占城的风土人情穿插在其中,向着赵顼和宰辅们娓娓道来。武英殿上倒成了他的独角戏。

赵顼不住的点着头,而吴充则是越听越是皱眉。不仅是他,其他五名宰执也都能看出韩冈如此说话,究竟是有何用意。

韩冈如此信手拈来的将一桩桩南方的奇闻异事当成闲谈说出来,越发的证明了他对南方的了解是扎扎实实,毫无虚假。这是在不断巩固和加强他在岭南事务上的发言权。日后朝堂上论起岭南之事,他的意见就会有着举足轻重的份量——就像他和王韶对熙河路的发言权一样——眼下更是让他即将说明的安南方略,还没有开始,就已经将天子说服了一半。

但韩冈在邕州不过数月,若说他能对两广、交趾的地理、民情了如指掌,吴充怎么都不可能相信。可殿上君臣,基本上都是对五岭以南两眼迷雾。任何人从岭南回来,只要能打听到几件奇闻异事,将其当成他深悉地理的证据,谁都没有没办法立时戳穿。

吴充心中忽的一动,抬眼望向冯京。

冯京不是广西人吗?明明出身是宜州的,只要随口半真半假的问两句,挑个错出来,就能破了韩冈的金身。

可冯京一直都在低头看着逐渐成型的沙盘。从韩冈嘴里一个个熟悉的地名报出来,十岁出头就随父离开家乡,再也没回去的冯京,也没办法从韩冈的话里找出毛病来。如果自己随便说话,说不定就会给韩冈抓住错处,他可不想丢人现眼。

冯京不敢轻易发言,吴充也没有办法。不过他有心挑错,倒也是渐渐听出有哪里不对。

韩冈对广东、广西地理的了解倒也罢了。竟然连交趾、占城都了如指掌,就让人很有些疑问。韩冈抓到的俘虏,当真会有这个能耐,能将交趾、占城的山川地理详尽的描述出来?就算是在大宋军中,有这本事的都不多。

“臣从俘虏的口中,打听到的消息杂乱无章,交趾国中的政事民事史事都有,只是在地理上十分粗略。”韩冈抢先一步堵上漏洞,“不过交趾的大致地形,则是不会错的。富良江的江口位置,升龙府的周边地理,甚至通往占城只有山海之间的窄窄一条通路,都是经过了多番确认。”

这算是滴水不漏了,吴充的心里给堵得慌。

当沙盘最终成型,城市、军寨一个个标定,韩冈给以了肯定的确认之后,田计退了下去。

韩冈站在沙盘下首,拿起作为小棍,解说的同时在上面比划着:“此前交趾来犯,是水陆并进。陆路过永平寨后,就言一路北上,直抵邕州。而李常杰在永安州上船渡海,攻下钦州廉州之后,也同样转往邕州。所以官军攻打交趾,也当是同样的手段。以陆路为正,以水路为奇,水陆两路相辅相成。广西、交趾在十月至二月时,雨水最少,瘴疠、疫病也同样稀少,如要用兵交趾,当选在冬月出阵,约期百日而还。”

“陆路好说,韩卿你之前已经以三十六峒蛮部打前站了。但水路是从廉州出兵,还是从广州出兵?”

“广州出兵?”韩冈怔了一下,然后点头道,“的确是要从广州招募船只和水手,用来运送兵员。”

在场的君臣知道韩冈误会了。王韶出来为他解释道:“不是仅从广州招募船只、兵将,而是直接由广州出兵。广南东路驻泊都监杨从先日前上本,如果是水陆并进,陆不过自邕州至左右江、横山寨等路,由甲峒、广源进兵,水不过自钦、廉等州发船,诸州邻近交趾,若有动作,其国中必然设备。当出其不意掩其不备,方可指日克捷。”

“广州并无水师,需要临时招募。水手从未经过训练,猝然上阵,必然难以获胜,只能用来运兵。”

吴充摇了摇头,他终于等到了韩冈的错处:“陛下,韩冈此言大误。海上多贼,但凡海上营生,没有不擅长厮杀的。臣在乡里,时常得见水手跨刀持弓而过,其中骄悍者,往往杀贼过数以十计。”转眼一瞪韩冈,斥责道:“韩冈,臆测须知当不得准,军国重事,不可妄言之!”

他是福建人,海上之事,殿中除了同样出身福建的吕惠卿,没人比他更清楚,生长在关西的韩冈更不可能——他见过海吗?

赵顼的视线投向韩冈,吴充的话提醒了他,韩冈生长在内陆,甚至都没有见过海。那他之前所说的……

韩冈这时抿了抿嘴,吴充是不是已经自暴自弃了。过去得罪狠了,如今也不在乎了?一边想着,一边很快的接上去:“若是真能招来远洋商船的水手,的确正如吴枢密所说。可泛海一载的所得,远比兵饷为多。吴枢密既然出身福建,应该知道水军和水手的差别何在——水手可是能在船上带货的!”

“海贸风险之大,岂是水师可比?”

“远涉鲸波是拼命,但上阵临敌不一样是拼命?难道这一次招募来的水兵,是为了在广州港中养着他们吗?同样都是要拼了性命,收益高下却有别,试问如何能招到堪战的水兵?……如果当真招募的话,被招来的只会是吃够了捕鱼、采珠苦的疍民。”

疍民在福建、广东、广西为多,常年生活在船上,所用的船只如同蛋浮水面,故名疍民。

韩冈对赵顼道:“疍民水性虽好,可只会捕鱼、采珠,根本无法上阵。一生皆在小船上过活,也不会操作高达数千石的海船。就是如此,所以臣才会说,广州所募水军只能用于运兵运粮,不能让他们在水上厮杀。不过就算只是在富良江口设立一座营寨,做出让船只进入富良江的姿态,也能让交趾人不得不分兵防守。如此,足矣。”

当日韩冈与苏子元商议时,苏子元还想要让水师深入富良江。可不知富良江中的水文地理,一个运气不好,船只说不定就会搁浅在沙洲上。与其冒那个风险,水路的用处还是用来分割交趾水师的兵力,让主力打到富良江边后,可以依靠木筏、小舟顺利渡江。

“昨日广东转运副使陈倩上本,广州去真腊、占城的商船谊舶,都要避过九月至十二月的飓风,需要正月初的北风乃可过洋。韩卿你意欲在冬月兴兵,水陆两路可能配合得上?”

赵顼这话问得就没水平了,宰执们都不约而同的双眉微皱。陈倩的那篇奏章吴充也看过了,当时就丢到了一边去。真当人没记性,去年李常杰是何时登陆攻打钦州、廉州?不过吴充现在倒是想看一看,韩冈会如何说,才能不伤到天子的自尊心。

“南海夏秋飓风多,往往至十月方止,偶尔也有一直延续到十一月。”韩冈并没有提醒赵顼他的记性有多差,“只是到了冬月、腊月,北风早起,又无飓风之患,完全可以漂洋过海。不过时近年关,商旅都会过了年后再出发。所以趋往真腊、占城的船舶,是正月而不是腊月。可若是兴兵,又何须在意节庆?其实如果去查一下广州过往有没有腊月上报风灾的记录,应该会比臣说得更明白。”

“原来如此。”身在九重之中,对广南风土只能从臣子的奏报听来,但韩冈一加点破,赵顼倒也能判断哪边更合情理,“韩卿所言甚是。”

水陆并进的方略差不多可以确定了,具体的行军安排,要到兵将抵达后再行筹划。另一方面,赵顼也确认了韩冈对讨伐交趾所做的功课,不论文事、武事都是准备充分,让赵顼对剿平交趾增添了许多信心。

所以现在最后的一个问题,“不知韩卿打算选用那一路的兵马?”

