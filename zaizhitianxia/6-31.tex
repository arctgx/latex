\section{第五章 冥冥冬云幸开霁(十)}

黄裳终究还是放下了笔。

心情已经乱得让他写不下去了。

推开窗户,望着灰蒙蒙的天空。

颜色变得暧昧起来的云层,正仿佛此刻的局势,让人捉摸不透下一步将会如何变化,是天朗气清,还是风雪降临。

火炮的轰鸣声,方才便传到了黄裳的耳中。

一声紧接着一声。

尽管隔得很远,都没有惊动到了黄裳的家人,以及家中的仆婢。但黄裳对类似的声音极为敏感,隐约的轰鸣,在他人耳中是会被忽略的杂音,而在黄裳这里,却是如同耳边炸响的惊雷,霎时间便警觉起来。

第一声警觉,第二声便是确定,之后又有了让黄裳担心起来的第三声。

不是礼炮发射的时间,火器局更不会选在朝堂大典时进行试验。

是意外,还是事件?

对此甚为挂心的黄裳,坐卧不定了一阵之后,只能派家人出门打听消息,自己则耐下性子想继续复习。

但是他终究还是无法静下心来。

看不进书,也写不了字。

黄裳很清楚火炮在韩冈眼中有着什么样的的意义。而情理之外的射击,其中蕴含的可能,以及会导致的结果,让黄裳无法不去深思。

尽管此时考试已经迫在眉睫,黄裳还是做不到心无杂念。

为了参加制举的资格,他这段时间以来,除了无法推托的极点,基本上杜门不出,几乎与世隔绝。

昨日参加了大祥,今日只是朝会,就告了病,不想再耽搁时间。

这些天来,他除了写文章,就是读书、查找资料。

书房中到处是摘抄下来的片段,以及灵光一闪的心得。

从决意参加制举开始,黄鼠狼尾尖的制作成的毛笔,黄裳已经写秃了几十支。要都拿去屋外埋了,也能堆起一座小小的笔冢。

直到现在,黄裳对通过制举也还是没有太大的把握。

连续败退于南省,黄裳少年时的狂狷已经点滴不剩。在韩冈帐下多时,剩下的只是逐年沉淀下来的稳重。

进士的资格也是通过取巧的办法才得到。对黄裳而言,这样的进士身份,无法给他以荣耀和自信。现在只有不断的苦读,才能维系住他的信心。

时间紧迫,黄裳不敢有丝毫浪费,吃饭和睡觉的时间也是一省再省。

如此时在房中踱着步子,不是考虑文章,而是胡思乱想,这样的状态,已经很长时间没有过了。

黄裳在来回踱步中,越发的心浮气躁起来。

砰的一声响,刚刚派出去不久的亲信家仆极为无礼的撞开了黄裳的书房门,跌跌撞撞的进了门来。

那名仆人在数九寒天里亦是满头大汗,神色慌乱地让黄裳将到了嘴边的呵斥又吞了下去。

“怎么了?!出了什么事?”

知道情况不对,黄裳急忙询问。

“二……二大王,和……和太皇太后叛乱!”

家仆喘着气,丢出了一个石破天惊的消息。

“啊!?什么!”

乍闻凶信,黄裳的心顿时便冷了半截。他的恩主韩冈如今在朝臣和太后心中的地位,有四成是依靠当年压制太皇太后和二大王的野心才建立的。

高太皇和赵颢若是卷土重来,韩冈还能有什么好结果?

不过慌乱只是一瞬间,黄裳立刻便恢复正常。他想通了,如果是太皇太后与赵颢成功,就不可能被说成是叛乱。只不过以太后对宫中的控制,就是太皇太后不甘寂寞,也最多是个几名演员的闹剧,旋起旋灭。究竟是发生了什么事,才会他们将闹事变成叛乱的?

“然后呢?”黄裳问道。

那仆人大大的喘了两口气,“好象是两府诸公救出了太后和官家,逼退了叛党。”

不是这么简单。黄裳脑筋转得飞快。太皇太后和二大王叛乱,朝臣之中,韩冈必是首当其中,若要平叛,不是韩冈领头,就是韩冈首倡。

“还有呢?”黄裳心急的追问着。

“……这件事小人不知真假。”家仆脸上的表情有着心中挣扎的痕迹,“只是小人听到有人在说,蔡相公也死了,是韩宣徽亲手拿着铁骨朵给砸死的。”

笑话!

黄裳第一个念头就是想要呵斥。可是他心中一转,竟不由的呆住了。

完全说得通。

或者说,没有蔡确倒向太皇太后和二大王,就根本不可能会有叛乱。

既然蔡确都能倒向太皇太后和二大王,那么皇后身边的石得一、甚至宋用臣,也不是没有投向太皇太后的可能。

有宰相和内侍总管的相助,太皇太后甚至能够兵不血刃的坐到大庆殿上。

而在那样的局势下,以黄裳对韩冈的了解,必然是采用最决绝的手段,将局面扭转过来。

一骨朵砸死蔡确,听起来可笑至极,可越想越是可能,也越符合黄裳对韩冈的了解。

“这个消息实在太可笑,只是事情仓促,小人没来得及再去查探。”那名仆人唠唠叨叨的补救着,心中还在后悔自己说了多余的话。

黄裳则一言不发,直接起身便往门外去。

黄裳的浑家已经被接到了京城,就是因为家眷来了,黄裳才会离开韩府另找宅院。她听到黄裳这边突然间就要出门,忙从内院追了出来。

“官人。”黄裳的浑家脚步急促,“现在是去哪儿?”

“去韩府。”黄裳说道。

身为韩冈的门人,这个时候不能在韩冈身边参赞机宜,也必须去其府上走一趟,以尽人事。

“……那也要换了衣服再去。”

黄裳低头看了看,一身家居的宽袍,里面夹着棉袄,看起来有几分臃肿,完全没有形象可言。

“这样就好。”黄裳脚步不停,不打算耽搁。

到了门前,他回头吩咐浑家:“关好门,别的不用多担心。”

骑上马,黄裳匆匆出门,向韩府赶去。

黄裳心中一团热火,这一回若是他料想的不差,韩冈肯定能够回到两府宰执的行列之中了。

行至半路,就看见一队队兵丁开始进驻街口。

仔细分辨了一下这些兵丁身上的服饰,都是开封府辖下。

沈括派人出来了?

叛乱初定,而人心难定,派人封锁街道,镇压城中,这是应有之理。

黄裳正想着,就听见背后一声叫:“那不是勉仲兄?”

黄裳回头,却是熟人:“章府判。”

在路上见到这位熟人,黄裳不以为异。

沈括作为开封知府,必须留镇衙署,不可能出来直接指挥军士。

能奔走在外的,是他衙中的幕职。

比如黄裳他面前的开封府判官章辟光。

当年熙宗皇帝即位后,第一个上书请求将还留在宫中的两位亲王迁出宫去,以避嫌疑的便是章辟光。但为高太皇所阻,被赶出了京城。

从此之后,章辟光都在酒税、盐税之类的职位上打转,直到去年,先帝发病、皇后——现在已是太后——垂帘听政,才又得到了启用。

因为开罪了高太皇而被贬居出外,也因为同样的理由而得到了向太后的看重。才一年多的时间,章辟光就已经做到了开封府判官的任上.

开封府没有设立通判,两位判官便是开封知府处理京城中日常庶务的副手。

相对于另设衙门于京外、管理权遍及京畿,只除了京师城墙之内的提点开封府界诸县镇公事,能逐日上朝面君的开封府判官,其实更受朝臣的看重,其地位甚至能比拟台官,一旦外放,甚至有可能直接授予大州知州,甚至是一路监司。

但如此品阶,又如此深得圣眷,章辟光却对黄裳不敢有任何怠慢失礼之处,不管有半点规矩。

“勉仲兄今天没有上朝?”

对于在路上看见黄裳,章辟光还是挺惊讶,毕竟也是升朝官,就是没有差遣在身,遇上朝会也是该上朝的。

“昨日偶感风寒,故而告病在家。”

章辟光看了看黄裳的气色,完全不是病人的模样。不过他自不会点出来,而是问道,“这是要往韩东莱府上去?”

“正是。”

“今日殿上之事,勉仲兄可是知道了?”

黄裳双眼一亮:“只听说了一点,含糊不清。府判今日当是入朝了,不知能否解黄裳之惑?”

“多亏了东莱郡公。”章辟光拍了拍自家的脖子,“辟光首领方得保全。”

章辟光今天也上了朝。当他看到上首宰执班处一片大乱,得知是太皇太后临朝,脑中登时嗡的一声响,连心脏都停止了跳动。

以得罪了太皇太后受到了太后的重用,当太皇太后卷土重来,掀翻了太后之后,留给章辟光的,也就只剩一条死路。

是回家后就拔剑自尽,还是回去后将妻儿安排妥当了再自杀?

当韩冈在陛前大声喧哗的时候,章辟光的心中只转着尽早自尽,以免之后活受罪的想法。

拔剑自刎有些难,跳河则也下不了那个狠心,用正流行的炭毒也可以。只要不透风,据说没有任何痛苦。

但之后的变化,却让章辟光看呆了眼。

章辟光亲眼看见韩冈是如何捶杀了宰相蔡确,而李信和王厚更是从他眼前疾冲而上,粉碎了叛贼一党在殿上最后的反扑。

从大悲到大喜,区区一刻钟,章辟光像走过了一个轮回。

等到正主驾临的朝会结束,他便随着沈括一起从宫中出来,受命平靖京城局势。

有此一事,章辟光对韩冈的感激自是极深,对韩冈手下最受看重的亲信,当然同样不敢失了礼数。

黄裳日后也会大用,此时示好,总比日后混同在众人之中,更能留下印象。
