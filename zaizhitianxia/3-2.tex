\section{第一章 坐忘渭水岸(中)}

欧阳修去世消息,惊动了京城,却还没有传到关西。

离着锁厅试还有一个月的时间,韩冈此时并没有着急着赶去秦州,而是安稳坐在渭水之滨的家中。心头一点也不急,仿佛忘了即将到来的考试。

他常年不在家中,有机会还想是多陪陪父母妻妾。为官者,别妻子,弃坟墓,远行数千里,说不准那日就会出现意外。所以孝心要经常表现出来。

王韶走后,高遵裕如今独揽大权,但也没有糊涂到改动王韶定下的规矩,萧规曹随的手段并不丢人。若是跟着自己出的意见做了改动,万一出了意外,高遵裕也担当不起。

另外高遵裕的小妾也同样身怀六甲,算时日,也就在这几天了。陇西缺名医,同样也缺高水平的接生婆,高遵裕遣人已经去秦州请了最好的稳婆过来。

韩冈已经吩咐过家里的管家钱明亮:“如果人来了,等她服侍完高家,就把她请到家中来,千万不要误了事。”

坐在家中,韩冈还是很有些不放心,毕竟他也是第一次经历这样的情况。周南和严素心是在去年腊月中确诊,那时候已经是一个多月的身子。推算过来,她们的预产期基本上就是在八月前半,也就最多还有半个月的时间,说不定还可能会提前上一点。

韩冈正推算着日后的变数是,却有人出来打断他的思绪了:“官人,承恩村的刘保正来了。”

司阍的老兵知道刘源在自家官人心中的地位,不会拦着外人一般的将刘源给拦下来,而是将他请进门房坐着,让打下手的小子,进去通报。

果然不出意料,里面很快就传话出来,‘官人请刘保正入内面会。’

跟随着韩冈,刘源他们在刚刚结束的河州大战中立下了不小的功劳,攻城拔寨有他们,守卫营垒也有他们,救援危城是他们,追袭残敌还是他们。比起各路禁军,只能算是乡兵的广锐军的功绩,不在任何一支精锐之下,不论是哪一路的选锋,都只能勉强跟刘源一众平起平坐。但到最后,广锐军的封赏还是以金银财帛为主。而且同样的功劳,比起普通的参战士兵们来说,都要低上很多。

不过韩冈看到赏格之后,便当即上书建言。提议道,为了日后能继续驱用广锐叛卒为朝廷上阵杀敌,最好是能以地充赏,用熙州、巩州的荒地,来补充赏赐中不足的部分。

韩冈的提议,朝廷很快的就批复下来。如果能让过去的叛军老老实实的开荒种田,不论是新党、旧党的哪一边,都不会反对这个方案。而且只要他们将赐予的荒地开垦出三分之一,两年之后,熙河路一年的税入中,又将多上过万石的粮赋。这是惠而不费之举。

同时因韩冈之言,本来封赏刻薄的广锐军卒得到了土地作为补偿。虽然还是荒僻之土,但用心料理个几年,就是一份上好的基业。所以韩冈的建言之德,更加上他过去的那一桩桩恩德,广锐军上下对韩冈都是有着效死之心。

被小厮一路引到书房,刘源就看见韩冈已经在房内站起来等候。

连忙行过礼。韩冈就示意刘源坐下,抖了抖拿在手上的礼单,半是感叹,半是质疑道:“这又是何必?”

薄薄的礼单上,写着一行行的金银绸缎,贵重器皿,还有一些土产,比如皮子,药材之类的,都是来自左近的山中。韩冈虽然没有他的表弟那般识货,但他一眼扫过礼单上罗列下来的礼品,还是知道这些礼物的价值,林林总总加起来快有一千贯了。这份礼,未免太重了一点。

只听刘源道:“听说官人最近要纳妾,而且马上就要有小官人或是小娘子了,我等也想是表一表心意。现在来还算方便,等过些日子热闹起来,小人也不便来走动了。”

韩冈听了就有几分欣喜,刘源也算是有心了,知道等到自己纳妾或是庆祝得子的时候,不方便出现,就赶在现在来送礼。广锐军的这份心意韩冈领了,但礼物他却不能照单全收。

“里面的土产我收下了,至于金银财帛等物,你还是带回去吧。下次来也不要带这么重的礼,你们的身家我也清楚,这些都是博命来的东西,还是留着自用,也要为日后儿孙留下些本钱。”

刘源连忙道;“贵重不贵重倒是其次,只是聊表寸心,官人对我等叛逆的恩德难以计算,要不是怕反而连累了官人,下面都要有人摆官人的长生牌位了。这点身外之物有算得了什么,官人还是收下了吧……”

“心意我收下来,金银之物还是不能收。”

韩冈坚持不要,刘源却强要留下。最后,韩冈有些不耐烦了,抬眼看了刘源一眼,“怎么,现在我锁厅了,说话就不管用了?”

说话的人虽然是在笑,但刘源已经不寒而栗。对文臣的畏惧,几十年来已经根深蒂固的刻在他心里,一次放纵,现在却更加敬畏。尤其是韩冈,刘源很清楚他的手段和性格,并不因为年龄的差距,敢小看他一星半点。

“就留下一半如何?”他陪着小心的问道,却还是不忘要把礼送出去。

“也罢!也罢!就收一半。”韩冈叹了口气。送上门的贵重礼物,不能全收,也不能不收。这送礼收礼的学问,千年前后都是差不多的。

放下礼单,韩冈问着刘源:“今天就刘源你一个人来城里?”

“还有一些小字辈,知道官人正在读书,不敢来打扰,都跑去看球赛了。”

“都已经开始了啊……过得还真快。”

随着河州大战的结束,陇西城中的足球联赛也重新开始。七八月份虽然天热,但球场上同样热火朝天,为了争夺一年中最为丰厚的回报,每一支球队都拼尽了气力。另外,还有私下里的赌球行为,让比赛的气氛更加热烈。大受欢迎的做法,当然难以禁绝。当然,州衙对此也是睁一只眼,闭一只眼。因为背后站着的,可是王、高、韩三家的商行,还有包顺、包约、张香儿他们。

不过前些日子因为开战的缘故,陇西城这边的足球联赛也不得不中止。十几支球队中的成员,个个都是身体健壮,孔武有力之辈。不但球技出色,上阵的本事也同样出色。不论是蕃部还是民间的球队,一旦征发令下达,都少不得被征调起来。现在的比赛,不知没有有联赛中断前表现出来的水平。

“这一战下来,也不知少了多少熟面孔。”

“没有!没有!”刘源摇着头,“一个都没死!连蕃部那边都一样。”

韩冈愣了一下,“这是什么缘故?”

“回官人的话。这些球员,军中爱他们球技的不知道有多少。今次大战,全都被安排到了后面,一点损伤都没有。”

韩冈哭笑不得,球迷、球星都有了,怎么就变成跟后世差不多的样子?

又说了一阵闲话,刘源带着韩冈没有收下的礼物告辞走了。韩冈要读书应考的事,他当然知道。并不敢久留,只因害怕耽搁韩冈读书。

让人送了刘源出府,韩冈让人找来了家里的官家钱明亮。

“钱明亮,你把刘源留下的礼物捡贵重的送到县里去,说是下面的百姓捐给县学的。说我韩冈做主,让他给收下。”

韩冈的吩咐很让人莫名其妙,但钱明亮并没有多问,应了一声就离开了。

韩冈并不缺钱,他缺的是人脉和根基。虽然他的影响力并不尽仅仅局限于巩州、熙州,但他很快就要离开熙河,总得留下些点东西以备将来。

他现在以刘源的名义给正在修建的县学捐上几百贯财货,这样日后就可以顺理成章的安排广锐军的子弟进入县学旁听。虽然不可能得到朝廷的给俸,日后更不可能有机会做官,但学上几年后,进县衙中担任吏员却不会有问题。

韩冈自知他无法控制来如同走马灯一般来熙河上任的官员,但他有办法控制衙门中的胥吏,不论是秦州还是巩州的衙门,他在其中都有人。如果广锐军的子弟能进入陇西县衙中,这座城市的底层,也就被韩家控制在手中。

即将离开这座城市,即将离开他起步的地方。但并不代表韩冈要放弃在这里打下的基业。韩千六将会继续留在熙河,负责屯田之事。与世无争,只管种地的老父,韩冈不担心后来者会跟他过不去。如果有人想从韩千六这边下手,来打击他韩冈,韩冈不介意让人知道他的破家绝嗣的匪号是从何而来。

善男信女四个字,从来都是跟韩冈无缘。想反,穷凶极恶还差不多。虽然看着他脸上的笑容还无影响,但熟悉他的人们,都会立刻给自己准备一个跑路的机会。

名声已经传扬出去,韩冈剩下的就是要稳定现在的大好局面。

ps:看到有书友说上一章关于谥号的问题说的不对,说谥号‘文’很了不得,比文忠要强。单谥更是要比双谥要好。俺在这里解释一下。赠大臣谥号的是朝廷,看的是政治地位,跟后世名望没半点关系。韩愈、朱熹、王安石的谥号都是文。富弼、欧阳修则都是文忠。

韩愈不过一个户部侍郎,朱熹更是卑官。富弼可是三朝宰相,谁能跟他比?而王安石得谥号的时候,都已经是旧党上台。旧党给司马光的谥号是文正,给王安石的文难道会好过文正?还有,朱熹本来是准备谥号文忠,但因为他的经历不足以支撑一个忠字,所以才被谥为文。

在北宋,不存在文比文忠强的情况。出现在宋人笔记中的常秩将文改文忠的故事,只是笔记作者的造谣罢了。北宋后期、南宋前期的笔记小说,这样的政治谣言有很多。

