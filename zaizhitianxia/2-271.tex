\section{第41章 千嶂重隔音信微(上)}

已是三月末,天气一日暖过一日,离一年一度的金明池争标的日子也越来越近,屈指计算,也只剩两根手指的时间。

天子即将驾临池上龙舟,观看军中健儿争标。为了准备天子出巡之事,来来往往的车马也是一队接着一队,这东京城西边的两座靠近金明池的城门——新郑门、万胜门处,也便是越发的拥挤繁忙起来。

坐在万胜门边的班楼酒店的二楼上,权户部判官吕嘉问独据一桌。桌上的酒菜上来后,他只稍稍动了动筷子,就一直把玩着手中的银杯。楼下的喧闹被一层竹帘给遮挡,倒也让他耳根清净了不少。

权户部判官中的户部,并不是六部中的户部,而是大宋财计的三司衙门下面的盐铁、度支、户部这三司中的户部。

身为户部司判官,吕嘉问算是十分称职了。他所主管的天下人丁税赋,至少可算是账目清明。兼管的诸司库务,也同样让人挑不出错来。而京中官造酒水,也是他所分管——前日他在官酿的酒场中推行连灶法,能为国库每年省下十六万贯柴草钱。

所以前日天子问起三司事务,王安石才会说,三司判官中唯有他吕嘉问最为称职。

只是光靠称职还是不够的。吕嘉问他的心很大,仁宗朝的权相吕夷简的曾孙,怎么可能因为一句‘称职’就心满意足?

所以他提出了一项新的法案

——市易法。

来自于京城中一名小商人魏继宗的提议,让官府插手进商品的贩卖与出售之中。尽管不是他吕嘉问所首倡,不过若是没有他的一力主张,根本就得不到王相公的首肯。

这并不是与民争利,依然是之前新党所秉持的与兼并之家争夺利益。

东京城是大宋的中心,人口百万,天下货品输入京城的数目多得难以计算。但这些货物运抵京城后,并不是直接在贩售,而都是必须转卖给各个行会的行首,再由行会的行首分给行会中的商人们零售。

行首们只是在中间过上一道手,就将利润的大头赚到了手中,而且一点风险都不用冒。这等坐地分赃的手段黑得让人发指,也让官府留着馋涎,但不遵守这等规矩的商人们,根本在京中待不住,行首们的势力可是能一直通到后宫之中!

不过自从王安石秉政之后,均输法推行于世,已经从行首们的手中抢到不少的份额。现在市易法的主要目的,就是将行首们的转售之权彻底夺过来。

当然,市易法在具体施行的时候,所用的措施和手段不会这么简单,甚至可以由官府出面收购滞纳商品,以收买行商。但从行首们手上抢钱的实质,却不会有任何改变。

吕嘉问对此心安理得,在地方上,但凡多余下来的便民贷款,都会强制本不需要借钱的上户们借贷——也就是所谓的抑配,以赚取利息。既然能明着从乡绅手中抢钱,那他的市易法推行起来自然也是名正言顺。

现在吕嘉问正在等着崇政殿中的那坐着、站着的十几位,对这项法案作出最后的决定。

用力握着祥云连枝的银杯,吕嘉问的脸上表情让一名准备坐在他对桌的客人,立刻起身,远远的躲到远处的角落里——他现在已经没有任何退路了。

自从前两年他把叔祖吕公弼抨击新法的奏章草稿偷了出来,给王安石过目之后,他在家中就没有了立足之地。因为这份投名状,王安石对吕公弼的攻击提前有了对策,让吕嘉问的叔祖在崇政殿中栽了一个大跟头。回来后,吕公弼就在家中上下彻查,查明了来龙去脉,便大骂吕嘉问是‘家贼’。

‘家贼!?’

吕嘉问冷笑一声,不过是成王败寇而已!

“望之,你好自在!”

突如其来的一声唤,将吕嘉问从个人的小天地中惊醒过来。

吕嘉问抬头一见来人,便立刻起身,“原来是圣美啊,这可真是巧了……怎么没看到王衙内?”

来人闻言,脸色微微一变,却又展颜笑道,“王衙内现在宫中讲筵之上,望之难道不知?”

吕嘉问暗地冷哼一声,浮起了同样应酬似的笑容,邀请这位王子韶王圣美坐下来说话。

王子韶前日进京诣阙,就紧紧地跟在王家大衙内的身后。才一个月的功夫,就在京城人嘴里落下了个衙内钻的名号,自然并非什么正人

——熙宁二年、三年的时候,王子韶还做一任过监察御史里行。能进御史台,自然是飞黄腾达的基础。可惜他在王安石炙手可热的时候跟着攻击旧党,而后在旧党反扑,王安石称病的时候,又动摇起来,倒向吕公著。最后便是被赶出京城,落了个知上元县。过了两年,又转到了荆南转运判官的任上。

荆南不是什么好地方,王子韶吃过了亏,自然知道该怎么做了,奉承巴结的事,做起来还真是不辱一第进士的头衔。不过这王子韶其实还是有些本事的,能重新攀上王安石和王雱,也是靠着他年未弱冠就考上进士的才学。

一句‘即云不见诸侯,因何又见梁惠王’,就算孟轲复生也只能勉强自辩的指责,让他在王安石和王雱面前重新得到了一个展示自己的机会。

——‘迎之致之以有礼,则就之’,吕嘉问自问没有王子韶的这番急智,能用孟子的话,让宰相依礼相待。

一张嘴能说会道,引经据典也绝不输人,也难怪王衙内会喜欢他。也就是人品方面,有待商榷了。

让人上来撤掉桌上的酒菜,换一桌新的上来,吕嘉问又是暗暗自嘲,自己好像也没脸说他人不正。

不过只要能让市易法推行于世,在新党之中稳住自己的位子,日后总有一天能在政事堂中得到个座位。到那时,看现在跟自己划清界限的那些族人,还能继续跟自家割席断交下去?!

自家的曾祖文靖公【吕夷简】身前身后,还不是被人骂成奸佞、奸相。数次为相,把持朝中大权,范仲淹、韩琦、欧阳修、富弼没少在他手上吃过亏。被天下清议给骂惨了,但最后怎么样——陪祀真宗!这是臣子少有的荣誉。

这笑贫不笑娼的世道,官位才是第一。别看现在吕家没人敢跟自己亲近,逢年过节都没人通知自己去祭祖,但过两年再看看!等那两个老鬼死了之后再看看!

“怎么圣美今日有暇,会往这座酒楼上来?”给王子韶满上一杯酒,吕嘉问貌不经意的问着。班楼酒店在京城七十二家正店中,也是排在很后面的,来的人并不多见。

王子韶在炙鹿肉的上夹了一筷子,轻描淡写的说道:“学士院锁院了。”

“什么!”吕嘉问差点惊叫起来。天子驾临内东门小殿,学士院锁院,书诏的翰林学士不得出,这是宰执拜除或是宰相出外的先兆,“是政事堂还是枢密院!?”他紧张的问道。

“说笑而已。”王子韶露出了一个恶作剧的笑容,然后看着皱起眉头的吕嘉问,“不过等渭州的那一位回来,当是要锁院了。”

“可是‘谁念玉关人老’?”

“正是!”王子韶哈哈笑了两声,“如今京城中遍传此曲,早传到了天子的耳中。这不,蔡子政【蔡挺】就要回来了。”

“蔡子政为渭帅多年,把泾原一路打造得如铁桶一般。枢密副使一职,他也当得起!”

王子韶之前没有说蔡挺回京将会担任何职,但吕嘉问也能猜测得出天子会给他什么职位。

王子韶放下筷子,微眯起双眼,神情变得深沉起来,“其实谁念玉关人老。其实也有另外一种解法!”

吕嘉问立刻摇头,“这绝不是蔡子政本意!”这是构陷啊,他纵然胆大,也不敢插上一句嘴。

“蔡子政这首小词做得虽好,但能忽然间传遍京城,肯定是有人在暗中推波助澜。”王子韶脸上的笑容,让吕嘉问感觉他是仿佛被周兴、来俊臣附身一般,“谁念玉关人老啊,自今上登基以来,陕西用兵可有一年停过?”

“其实也无所谓了,河州城已经攻下,王韶也就要进京。凭着开疆之功,也许在枢密院中同样能得张交椅坐坐。陕西自然也会清净下来。”吕嘉问可不想这个节骨眼上出什么事,不论王子韶想做什么,他都无意掺和。

王子韶笑得更为意味深长:“照理说河州城都攻下来了,怎么说也该庆祝一番,为何至今还悄无声息。”

虽说来自河湟的消息都是军情机密,但这东京城中从来就没有秘密两个字,就算王子韶这名上京诣阙,等候天子召见的外臣,也同样很容易就能打听到消息。

吕嘉问知道,肯定是今天有什么新的消息传来了,“可是出了什么大事?”

王子韶将脸一板,凑近了,压低声音,“王韶领军翻越雪山,据说已经断了消息。”

