\section{第二章 天危欲倾何敬恭(中)}

“母念亡儿,儿念父母,伤心过度以至神智昏乱,这都是常见的事,不足为奇。”韩冈笑了一下,“不见英宗皇帝,也曾在仁宗梓宫前伤心过甚,以至重病不起吗?我等做臣子的,得体谅才是。”

章敦眼睛顿时瞪了起来,深呼吸了两下,又摇了摇头。跟韩冈说话,有时候的确得有些耐心和涵养。

过继给仁宗皇帝的濮王府十三郎,到底是怎样的孝子贤孙,朝中没人不知道。不过韩冈拿英宗皇帝做例子,也是明说了,赵曙在仁宗梓宫前的那滩烂事能压下去,难道太皇太后的疯话还压不下?那时候还有不少大臣为仁宗叫屈,富弼都能当面说伊尹之事、臣能为之,可现在,有哪个重臣会站在太皇太后一边?

“玉昆,那十年之后呢?”章敦不绕弯子了,“你有没有想过天子亲政后会是什么样的情况!”

怎么可能没想过?!

纵然之前小皇帝似乎一直对韩冈心有芥蒂,但那样的态度保持到亲政,最多也只是将韩冈请出京龘城。但现在发生了这样的事,一旦他能够把持大政,正常的皇帝都会杀人泄愤,而且还会钳制天下言论,根究敢于谈论熙宗皇帝死因的人。

韩冈想了想,正准备说话。

“玉昆!”章敦出声打断。

他冷着脸,冷着眼,激动起来的声音也是冷的,“别跟我说什么官家聪明睿智,必是明君,不至于如此;也别说什么让天子的儿子即位,让天子为太上皇,多少小人等着那机会呢,会让你顺顺当当的行事。更别说什么让太后听政下去,寿数天定,你有几成把握?”

韩冈摇了摇头:“子厚兄。若说预测人的寿数,小弟是半点也没有的。不过,真的需要担心十年后吗?”

韩冈的态度更加诚恳,能在自己面说出这番悖逆不道的话,可见章敦已经是推心置腹了,但也足见他心中的不安。

韩冈当然知道包括章敦在内的宰辅们现在会有什么想法。

他当时力主招侍制重臣入宫,可以说是毁掉了废去天子的唯一机会。

所谓谋不可决于众人。只要人一多,那些极端的意见就不可能得到认同,最后总会是最平庸和安于现状的决定占上风。当向太后派出了内侍去招侍制以上的重臣入宫,宰辅们就失去了他们控制朝局的机会了。

如果没有韩冈的这一龘手,只有宰辅们在宫中,谁也说不准之后会不会有什么变数。不受干扰的冷静思考,恐怕每个人都能想到怎么做才是最有利的,尽管那时候大都想着日后有祸大家一起分担,但直接废掉皇帝其实更安龘全。

失去的机会不会再来。仓促间,被引上一条看不见未来的道路,宰辅中至少有大半是在担心日后祸及子孙。不过章敦现在不是来秋后算帐的,而是想打探一下韩冈的心思。

“玉昆你说。愚兄洗耳恭听。”章敦说着。

想要将天子架空,只有群臣同心,否则上面的皇帝就能拉一派打一派。废立天子时也是一样,韩冈既然第一个表明立场,支持赵煦,那么宰辅们就已经没有别的选择。有韩冈在,谁也没把握说服皇太后,何况背后还肯定有一个王安石。对未来,章敦心中自然担忧,但他已经承认了现实,无意追究。只是他希望韩冈能有一个让人满意的解释。

“在说之前,小弟想问问子厚兄你,什么是皇帝?”

章敦眉头微微一皱,还是耐着性子跟韩冈扯开话题:“皇帝,天子也。德兼三皇,功高五帝。始皇为之。”

“天子?天没有儿子。想必子厚兄你也明白了,所谓天人感应,不过是董仲舒用来钳制天子妄为的手段。拿着望远镜观天,星辰之数,千百倍于星图。三垣二十八宿的周天星官之外无数星辰,又是什么?”

“玉昆。这有关系吗?”

“有!”韩冈点头,“华夏拥九州,三代之时其土只在黄河南北。西至陇右,东至海,北不过燕山,南不及岭外。禹贡之中,九州也就这么大。世所谓天下尽属王土,但九州所具有土地,不及大地的百分之一。西域之西,更有国无数。”

“愚兄是明白你的意思了。”章敦才智高绝,韩冈说到这个地步,还能不明白,“文彦博是老糊涂了,变法以来没做过一件好事,不过他有一句话说的好,为与士大夫共治天下。是这个意思吧?”

“差不多。”韩冈点了点头,“我等士人,应以具有常识的态度看待陛下。”

所谓‘普天之下,莫非王土,率土之滨,莫非王臣’,只是大吹法螺。天子非天子,只是凡人,天下的土地也不是天然属于他,是要靠人帮他征服下来。

章敦的眉头皱得很紧:“玉昆,我怎么感觉你是在找借口?”

的确是借口。

如果那一日,当真要重立新君。宰辅们为了他们自己的利益,推举上来的决不可能是一幼童,必然会选择长君。否则让向太后垂帘十年再归政,面对亲政的皇帝,他们岂能自安?十年时间,什么样的恩德都会消磨了。纵然皇帝要念着拥立之功,也不会让他们留在朝堂上。

向太后不能继续听政,这伤害了向太后的利益,也连带着伤害了韩冈的利益,更对他推广气学不利。既然如此,还不如留着赵煦在位置上。现在有好处,不利的未来也可以扭转。

“子厚兄还记得这句话吗?天子者,兵强马壮者为之。”

章敦之前的话已经够悖逆的了,韩冈却比他更甚一筹。

章敦霍然而起,指着韩冈,厉声道:“玉昆,你到底在想什么?!”

章敦想要的是什么?不过是辅君王,相天下,一展长才。拥立之功不过是因势利导,形势使然。章敦自束发受教,从没想过要与皇权对立起来。

“子厚兄,别想太多了。所谓天心,不过是人心。那是五季之事,如今大宋开国百三十年,亿万子民都认定了赵官家,国势正盛,谁能反?智者不为。”

延续了百多年的王朝,坐在皇位上的皇帝,天然的就能得到臣子们的臣服。天下士民都会认为这是理所当然的一件事。重臣们想架空皇帝,等于是走在独木桥上,一不小心就会连人摔下去。危险性太高,而好处又太少,还不如扶起一位皇帝,得享三代荣恩来得安心省事。

所处的位置不同,看事情的角度也不一样。如果韩冈现在是坐在大庆殿中最高的那个位置上,谁敢跟他说分权,他会毫不犹豫的让谁去跟阎罗王讨价还价。但现在既然他只是一名大臣,又不可能再进一步,则就又是另一种说法了。一个还在鼎盛期的王朝,权臣也好,叛逆也好,想要成功上龘位的可能性远比王朝末年小上千万倍。

章敦喘息了几下,坐了下来:“玉昆,那你到底是什么意思?”

“子厚兄,你可知道天竺。以氏族相高,国主大臣,各有种姓,苟非贵种,国人莫肯归之;庶性虽有劳能,亦自甘居大姓之下。”

“是沈括的《笔谈》?”

韩冈点点头。“沈存中的笔记中,记录了天竺的氏族种姓,这一点很有意思。天竺国中,将士庶分为四等。其中婆罗门掌祭祀,刹利主政事,毗舍为农、工、商,至于最低一等的首陀,那是做佣工或是其他低等的杂工\footnote{有关印度种姓制度的记载,出自《梦溪笔谈》:唯四夷则全以氏族为贵贱。如天竺以刹利、婆罗门二姓为贵种:自余皆为庶姓,如毗舍、首陀是也。其下又有贫四姓,如工、巧、纯、陀是也。其他诸国亦如是。}。四民之外,还有贱民,不得与士族接触。”

章敦紧锁着眉头,思考着韩冈为什么要提起天竺的种姓。感觉已经抓到了一点头绪,却还是差了一层。

“沈存中说的好,士人以氏族相高,虽从古有人,然未尝著盛自。但释教传入中原,却把四夷之风也一并带来。所以魏晋铨总人物,相交先论氏族高下。三世公者曰‘膏梁’,有令仆者曰‘华腴’。尚书、领、护而上者为‘甲姓’,九卿、方伯者为‘乙姓’,散骑常侍、太中大夫者为‘丙姓’,吏部正员郎为‘丁姓’。得入者谓之‘四姓’……”

“玉昆,你觉得氏族种姓很好?”

“子厚兄,你觉得我会喜欢这样的制度吗?小弟可是灌园子啊!若是在天竺,一辈子都难以出头。”

“为什么史迁书陈胜吴广,不入列传,而入世家?子厚兄想过没有。”

理由很多,历代学者都有解释,但从韩冈的话中来推断,却是一句话,“王侯将相,宁有种乎?”

“没错,在韩冈看来,就是因为这一句,所以太史公不以臣庶待之。王侯将相,宁有种乎?换个说法,就是物尽天择、适者生存。不仅人如此,文法制度亦如此。”

