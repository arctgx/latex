\section{第十章 却惭横刀问戎昭(12)}

“看来我还是将你看得太高了。”萧十三居高临下的俯视着领军回返的耶律盈隐,“契丹人的脸面,今天可都是给丢尽了。”

耶律盈隐一脸不服气:“末将今日虽没有攻下西陉寨,但西陉寨北的各处寨堡、烽燧,都已经被末将清除。如果枢密想要攻打西陉寨,不必担心宋人从侧方偷袭。”

“本帅不知道怎么攻打西陉寨,所以想问问你。究竟是打算怎么将西陉寨攻下来?躲到两里之外,你是能射到秦怀信呢,还是能砍到秦怀信。”萧十三笑着,“本帅可是没有这个本事,不知喜孙你能不能传授几招?”

萧十三的话引起了下面的一片窃笑,耶律盈隐心头羞恼,宋人的八牛弩摆在城头上,现在笑自己的,可有一个敢走近到一里之内?

萧十三脸上的笑意忽而一收,换上了一个阴沉沉的表情:“你带出去的兵马加起来六千,甚至没能向西陉寨上射出一箭,最后只打下了两个村子。伤亡却有六百余。军令状是你自己立的,我问你,你说这是算胜还是败?”

“末将斩首一百二十七级,宋人伤亡更是十倍于此!”耶律盈隐提升叫道,“末将可是缴获了弓六百余张,刀剑四百,长枪、长矛近千。就是宋人最宝贝的神臂弓,没有来得及毁坏的也有十三张。”

耶律盈隐深有自信,无论如何,他都是第一个出兵的将领。契丹人最重勇士,萧十三不敢出战,自己则出战了,还有丰厚的战果,人人都要承认自己的功劳。

“宋人乡兵多用弓箭,故名弓箭手。禁军则多用神臂弓,其佩刀都是夹钢锻打而成,能截金断玉,斩铁如木。不知你缴获了多少柄禁军佩刀?”萧十三可不会承认耶律盈隐的功劳,主帅的权威不容任何人挑衅,“军令状就在这里,违逆帅令,强自出兵,最后无功而返,你愿领的军法想必不需我提醒你。姑且念在你好歹有几个斩首,死罪可免,活罪南逃。拖出去,四十鞭!”

强令亲兵将人拖了出去,萧十三冷哼一声,闹剧算是结束了,正戏也该上台了。

……………………

“辽人这是玩的哪一出?西陉寨就在面前,却正眼看都不看,只打几个村子回去,就心满意足了。这还是契丹人吗?”

“欺软怕硬,这不就是契丹人的本色吗?辽军旧年攻入河北,什么时候敢攻打坚城了?杨六郎守广信军,梁门、遂城,哪一座城池他们攻打过。”

“辽人本就不擅攻城。洗劫村寨倒是一把好手。”

“没那么简单。都做好准备了,还是给攻破了两条村子,辽军还是很有些实力的。”

几名代州的将领在下面窃窃私语,韩冈也在和刘舜卿议论着这一次发生在西陉寨外的战斗。

西陉寨外的一场没有什么意义的交锋,其结果用了一天从雁门寨传回到代州。已经确定了的伤亡情况,与其说是上万契丹铁骑和边境坚寨的战斗,还不如说是打草谷的强盗和缘边弓箭手之间的交手。

尽管报上来的数字水分很大,但凭借多年的经验,以及斩首的数目,韩冈和刘舜卿都能从中推断出大体正确的战损和战果。一边是两座村寨被攻破,一边则是两路兵马被伏击,从结果上看,双方的伤亡应该差不了太多,都只有两三百而已。

谈到战果,刘舜卿很有几分得色:“不过我军伤亡的多是缘边弓箭手,去助阵的禁军没有多大的损失。辽人那边可都是精锐的骑兵!”

韩冈摇了摇头,“也不能说辽人吃了大亏。大宋的缘边弓箭手和辽人的头下军,说起来身份其实都差不多的。”

尽管兵制上有很大的差异,但总体上说,在辽国能归入禁军行列的,也只有皮室军和宫分军,而其他部族军、头下军,以及属国军,从等级上看,也就跟大宋的厢军、乡兵差不了多少。

“头下军中的精锐,都是辽国贵胄的私兵,并不比宫分、皮室稍逊。属国军、部族军其实也是如此!”刘舜卿还想再多说几句,但当他看到韩冈嘴角的笑意时,就立刻醒悟过来,面前的这一位自做了官后,就时常领军上阵,经历过的万人以上的大战远比自家为多,心明眼亮,军中情弊了如指掌,不是可以欺瞒的主。干笑了两声,话锋一转:“不过话说回来,雁门一带的缘边弓箭手,守土之时,也都是勇猛难当,而且其中能开石五硬弓的豪勇之士比比皆是。”

“事先我们预计到辽人会先拿周边的村寨下手,也命缘边各寨小心提防。秦怀信更可以算得上是宿将,在西陉寨周边又不会有他指挥不动的情况,而且还是在山中应付骑兵。地利、人和皆在,天时也不能说在辽人一方,可这一次偏偏还是被攻破了两个村子。由此可见,辽人也不是那么好对付的。”

“经略说得是。”刘舜卿附和了韩冈一句,道,“辽人的确是不好对付,最后能赢下来,还让辽人损兵折将的无功而返,算是难得的功绩了。”

韩冈沉吟了一下:“从大局上说,的确算是官军赢了一着,但从战果上看,只能说是平手。”

韩冈一向将战略和战术分得很开,在战略上,让辽人没占到什么便宜,损失远大于收获,的确是小胜一筹。可从战术上,说平手都是勉强。再怎么说,都是两个村子被攻破,守御的一方和攻击一方的伤亡竟然差不多。

刘舜卿见韩冈对这一战评价不高,便有些头疼。他知道韩冈过去领军上阵,总是几百几千的斩首,或破军,或灭国,眼界都给撑大了。可这世上并不是什么人都能有韩冈的功绩,否则二三十岁的经略使就能满地跑了。能从辽人骑兵那里抢下四十多个斩首已经很了不起,西陉寨总共才多少人?一个村寨又才有几人?

“的确如经略之言。”刘舜卿说道,“只是退敌逐寇,算不上大捷……也多亏了下面的将士拼了命,否则也难有这一次的功劳。”

韩冈深深的盯了刘舜卿一眼,道:“这一仗算是开门红,有斩首、有缴获,挫了辽人的锋锐,肯定是要向朝廷报功的。”

“但这毕竟是与辽人交手……”刘舜卿试探着韩冈的态度。

“不用担心,这是杀贼,朝廷的功赏不会少!天子和朝廷当不会吝啬。”

韩冈当然不会拦着不去上报这一次的功劳。就算是打得辽人,他也找照样报上去。何苦为朝廷省钱,而将怨恨归咎于自己?

刘舜卿放下了心头事,心情放松的与韩冈谈笑起来。下面的将校有人耳朵尖,听到了韩冈和刘舜卿的对话,窃窃私语的声音也大了起来。

韩冈的一名亲兵不知何时出现在厅外,扯着门口的卫兵,让他们送了一封信进来。

韩冈中断了与刘舜卿的交流,接过信,先看了下信封上面的印章,竟是是用马递从太原送来的急报。

厅中似乎是在瞬息间就静了下来,几个还在专神说话的,发现莫名其妙的就安静了,一个个心惊胆战的停了嘴,不知道究竟出了什么事。而其他的人则是望着韩冈,能在军议时递进来的急报,当然不会太简单。

韩冈没多话,打开用火漆封住开口处的信封,抽出里面的信笺展开看了一看,然后不动声色的将信折好。

“希元。”他亲切的叫起了刘舜卿的表字,“看来我要先回太原去了。”

刘舜卿脸色一变:“经略,可是太原出了事?”

刘舜卿这一开口,厅中众将精神顿时更加集中,竖起耳朵静待韩冈的回答。

韩冈微微一笑:“没什么。本来是因为辽军压境,担心雁门有失,所以才来的代州。不过这几日亲眼看到了希元,以及诸位都用心国事,我也就能放心得下了。今日一战便是明证!上万辽师甚至连西陉寨都破不了……呵,甚至是不敢去攻,只能去打劫山里的村子,这样的贼寇,已经不足为虑。”

他将手上的信扬了一扬,“现在太原府那边也在催我回去。出来时,让通判权摄州事,本以为得几日轻闲,没想到才几天功夫,就写了信来催,大概是不忿我这边偷懒呢。”

韩冈轻松的语调,引起厅中的一阵轻笑,让人不再怀疑他收到的信中有何紧急军情。当然,绝大多数还是故作轻松,并不是当真相信了韩冈的话。但既然韩冈这么说了,便姑且当做是这样,没有眼色的人,坐不到这间厅室中。

结束了军议,从厅中出来,众官众将纷纷散去。只剩刘舜卿跟在韩冈身侧半步之后。韩冈脸上温和淡然的微笑渐渐收敛:“辽人在云中屯兵几近十万,或许并不是针对代州。”

刘舜卿闻言,了然于心,问道:“经略,是方才那封信……”

“信的确是从太原来的,只是信中的内容则是说的西边的事,有人心不死,却要数万人跟着他冒险。不能不回去了。不过不要以为这里的局势影响不了大局,关键的时候,还要河东……乃至代州出来支撑局面。”

韩冈说得太过含糊,刘舜卿的眉头也就越皱越紧,眼中的疑色也越来越浓。

韩冈回头望了刘舜卿一眼,也不瞒他:“这封信只是确定了一件事……徐禧当真是疯了。”

