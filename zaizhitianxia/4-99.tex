\section{第18章 青云为履难知足(五)}

邕州城西,就在江水之滨。是一片片阡陌纵横的稻田。田中水稻已经长得老高,一片片浓浓的绿色,犹如一幅幅绿色的地毯。

韩冈带着人、骑着马,自田边巡视而过。夏日炎炎,正是一天中最热的时候,但满目的苍翠让他掩不住嘴角的笑意:“长势不错啊……”

跟在韩冈身后,是新近从桂州调来的通判,已经有五十岁,在两广诸州来回做了二十年官,却一次都没离开过五岭以南的这一片地去他处任职过。在只有自己一半年纪的韩冈面前,他保持着足够的恭敬:“下个月就能收获了,当又是一个丰收。”

韩冈点着头,心情如他胯下坐骑的脚步一般轻快:“如果能与上一季同样的亩产,那差不多能收到十五万石了。”

就在一个月前,邕州城外,在大战结束后匆匆忙忙下种的占城稻,已经经过了收割、晾晒,最后入库了。

今年第一季的收获,总共有六万石之多,平均亩产一石六七。在生长的过程中没有多少照料,连肥料都没有施过几次,只是望天收获的情况下,能有接近两石的亩产。邕州土地之肥沃,实在让生长在关西的韩冈嫉妒不已。要是关西有邕州的水土阳光,再养活两三倍的人都没问题。

只是最终回到邕州的百姓,新近的计点,总共有三万八千之多。而第一季收获的六万石粮食,在碾磨去壳过之后,也仅是三个月食用的份量。

幸好正在城西的这一片地,开垦时由于城中废墟已经收拾完毕,腾出了大量人力,面积扩大了许多。另外,韩冈让随军的铁匠指点邕州铁匠们打造耕犁也起了大用——岭南的耕犁形制与北方差别很大,甚至许多都不用耕犁。耕田下种时,都是将土刨开就行了,从来不深翻。其农作技术远远比不上中原——靠着先进的耕犁,加上广西数目多得惊人的水牛,开垦田地的效率高了许多。

如果真能按照韩冈的预计,收获在十五万石上下的话,那就足以支撑全城百姓七八个月。加上接下来的第三季可以耕种得更多。种得还是占城稻,不过是晚占城——占城稻分为早占城和晚占城,早稻、晚稻分得很清楚,不过生长快、收获早却是统一的,播种下去,最多两个月就能收获——而且就算是晚稻,收获也不绝不会少。

这样一算,支撑全城百姓到明年早稻收获没有问题,而且还能多余一些粮食出来,供给一两万人的大军几个月的食用。

不过韩冈这么做,还是得罪了不少人。在田边走了一阵,韩冈再次开口问道:“这些天,还有没有人再来闹着要将田还给他们?”

通判笑道:“之前龙图已经敲打过他们了,又给了那么多的好处,早就不闹了。”

韩冈为了让邕州百姓熬过这一年的灾荒,将人力集中起来,对邕州附近的田地进行统种统收,而不管田主究竟是何人。前期从宾州运来的赈济,加上新近收获下来的粮食,则是分配仅够果腹的一部分,剩下的则作为工钱,用来招募人工来开辟水渠、修筑道路、整修城墙。除此之外,也有给老弱妇孺的活计,比如洗衣、除草之类的,保证愿意干活的都能填饱肚子。

另外还有两千交趾俘虏,都被砍了脚趾,根本都逃不了,也被使唤来料理田地作为赎罪。十人一队,一人犯错,全队株连,互相监督下,都是老老实实的在田间地头被监督着做事。不过韩冈行事一向讲究着有赏有罚,只要做得好的小队,则有从江里面捞上的鱼来做奖励。

韩冈这样的赈济制度,邕州城中的贫民当然愿意,他们中的绝大多数都能吃饱。但逃脱大劫的豪门富户不干了,靠城的好田都是他们的,种出来的粮食理所当然也该归自己。

就在第一季收获的那几天,三五百人串联起来闹到了韩冈面前。不过韩冈可不管那么多,想说田是你的,得拿凭证出来,而且还要跟官府里的存簿对照,这样才能确认

所有的籍簿都给烧了,要想要找回旧档,得去广西转运司的架阁库里找——新近编制的田籍离完工还早着呢——除此之外,就算京城的三司衙门里,都只有税籍略本,哪还可能找得出田籍地契来?

可韩冈是什么身份?新晋的广西转运使!就算掏出地契来,他只要推说一下等从桂州拿了存簿来对照,拖个一年半载都没问题。

不过韩冈也不全然是浑赖,就算是统种统收,也不过是一年而已,并不是将私人的田产收归官府。对于私田,他保护得也很严格。敢在这时候私自移动田中界碑的,韩冈可是用大板子杖了几十人,用二十斤重的团头枷枷在临时的州衙大门外,排一溜站着。

加上他还在田间开辟了沟渠,引水来灌溉,将原本的旱田改造成了水浇地——位于江畔,田间竟然连一条沟渠都没有,这件事是让韩冈觉得最不可思议的地方,种地全靠天上的雨水来滋润,几里之外可是水比黄河还多的珠江——亲眼看着自己的田地被改造成上等田,想偷田的贼人又被严刑整肃,加上支持韩冈的贫民们群起而攻的诟骂,富户们就不怎么闹腾了,也不敢闹腾了。不过一年而已,就当是被交趾人多烧了几间房子罢了。当然,若是韩冈一年后不将田地还回来,他们可都会往京城里去告状了。

在广西的中南部和东南部,宾、邕、钦、廉这一片,有好几个面积不小的盆地和冲积平原,土地肥沃、气候宜人,如果农业水平有中原的一半,都能成为数得着的大粮仓。就算以此时的耕作水准,宾州、邕州都是从不缺粮、而且富余很多。

韩冈想要做的,就是尽可能的在岭南推广中原的农业技术,只要广西、广东的田地能尽可能多的开发出来,出产再丰富一些,大宋对岭外之地的控制力就能上一个台阶,以此为跳板,整个南海地区都在辐射范围之内。

《尚书·禹贡》之中,将天下九州的田地分为了几个等级,其中扬州是最差的,所谓‘厥土惟涂泥,厥田唯下下’,那时的扬州基本上就是淮河中下游以南的广大土地,也就是如今每年六百万石运往京城的地方。而千年之后的广东广西,只要是平原地带的农田,粮食产量也都不低。

心中盘算着要如何开发岭南,韩冈一行的方向,又转过去了邕州北面,那里有几座提供给城中使用的石灰窑,就在从山上流下来,汇入郁水的河流边,隔着老远就是浓烟滚滚。

石灰窑中的燃料都是从山上砍下来的木头,劈开晒过作为柴禾。另外还有秸秆,也一起利用了起来。可惜火力不旺,烧制起来很费时间。

缺少柴薪是困扰韩冈的大问题,他时常在想着,要是邕州附近有煤矿就好了,也派人去找了,只是到现在也没有回报——他对后世邕州的记忆很模糊,也不知道在哪里能找到煤矿,不像他对徐州的了解,知道那一带有煤。刚刚收到京城的消息,就在徐州城外发现了一座煤矿,自此有煤有铁,这是煤钢联合体的基础。

视察过石灰窑,韩冈上马返回邕州城,他下午还有课要上。

几个月过来,邕州的官员已经差不多配齐了,吏员也招募了不少,州中的庶务韩冈都卸了下来。管得也就是军事、刑名、钱粮和医疗卫生方面的大事,其他时候,他直接做了甩手掌柜。

作为官员,韩冈现在肩上的责任并不重,转运使的本职,暂时交给副使来处置。不过作为儒者,他还有许多事要做。儒门重教化,闲下来时,教书育人是本职。韩冈早前颁布条令,邕州七岁至十岁的儿童,不论男女,只要来上学,中午都会供给一餐伙食。许多父母贪着这个便宜,都让儿女过来上学。另外州学也恢复了,是邕州城第一批重建的建筑,比州衙还要早一个月。州学中的士子一边聆听韩冈的授业,另一边也要充作蒙学教授,为儿童上课。

刚刚抵达城门口,韩廉就带了一队骑兵迎了上来。

原本是韩冈家的家丁,但现在已经积功为官——一条瘸腿在功劳面前算不上什么——还做了古万寨寨主。不过古万寨已经被焚毁,在韩冈腾出手来,调拨人力物力重建古万寨之前,韩廉就只能先闲着,在邕州城中做个巡检。

“城里可是有事?”韩廉迎过来后,韩冈就问着。

“泗城州、思恩州还有忽恶峒三家的洞主方才一起进城了,现在就两家洞主还没到。”韩廉在韩冈马前禀报着,“钤辖要下官来禀报龙图,是不是再等一等。”

“还等什么?”韩冈的脸阴沉了下来,“泗城州隔了几重山,近千里地都赶来了,左州、忠州离得这么近却还没到,难道要等到明年过年他们来州中赶集吗?!”

韩冈一个月前,使人传令本州,召集左右江三十六峒洞主来邕州议事——邕州下辖几十个羁縻州,那些洞主管着的一个溪峒,在宋人这边往往就是一个羁縻州——已经等到了现在,连远在左江上游山中的泗城州洞主都到了,还没有到的那就是态度问题。

“进城!”韩冈一夹马腹,气势汹汹的往州衙杀了过去。

