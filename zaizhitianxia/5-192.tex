\section{第22章 早趁东风掠马蹄(下)}

这是比赛有趣的地方,不论事前怎么推算,总会有意外发生。就算是事前掌握了大部分信息资源的如何矩这一班人,也一样不可能将比赛的结果猜得半点不差。

第一场比赛结束,下面骑手牵着马站成一排,几个赛马总会的会首开始给他们颁奖。

头名是运气好到爆的狼居胥,一匹成了黑马的灰马——如今的赛马禁止重名,以防赌马时扯皮,故而各种稀奇古怪的名字层出不穷,到跟后世的网名差不多,这个名字算是比较正常了。

第二名是刘家车马行的西风紧,一匹契丹马。这匹马全程一直都是在第五第六的位置,不是最好,也不是最坏,但运气来了,挡都挡不住。跑在前面的卷毛青和黑风追云擦撞,使得速度骤减,倒让西风紧的骑手看到了机会,就此一跃而。以素质论,契丹马的确赶不河西马,尤其是肩高,站在一起一比就更是显眼,比同一赛场的河西良驹差了近两寸,尤其是拥有大宛马血统的黑风追云这样的赛马,差得实在很远。但依靠时运,很惊险的拿到了第二名的位置。

第三名是一开始领头的飞里黄。至于天子小叔家的那匹后来居的卷毛青,以及天子舅公高遵裕想拿来打名气的黑风追云,则是很令人遗憾的落到了第四、第五的位置。

赌马的马券落了一地,看台骂声一片,谁也没想到会是这个结果。

尤其是这一场的前两名都是默默无闻看不出优势的赛马和骑手,却因为高遵裕的保送战术和一点运气,笑到了最后。

韩冈大笑着对妻妾道:“要是让你们买马券,估计也是输光的份。头三名想说猜中其中两名的名次了,就是猜中一名都难说。”

王旖不看韩冈得意的模样,拉着严素心要往前面坐。

“就是多年的老手也猜不到今天这一场的意外。”何矩打圆场的说着,他心里挺惊讶韩冈对家人的态度,据他所知,有许多高官显宦待妻子如严君,就是在家里都是一本正经板着脸的,韩冈这般普通人家的感觉若不是亲眼见到实在很难以想像,“事先看好卷毛青的居多,飞里黄,黑风追云同样算在内,狼居胥也不是没人买。但西风紧是冷门,真正的冷门。”

赛马的马券有两种,一种是猜名次,头名、前三,乃至所有参赛赛马的名次,赔率一个比一个高——当然,最后一项尽管少,却也有人买,可从来没见人中过。另一种就简单了,只猜前三名是那三匹马,由于不计较名次,一场比赛中有资格争头名的赛马也就那么几匹,事先预测出来的几率就很高,故而赔率便低了下来。相对的,买的人则远比前一种要多得多,自然中奖的也多。但今天的情况,估计是没人中了。

虽然这开场戏让数以千计的观众和赌徒失望和愤怒,但这一场比赛也只是垫场而已,接下来还有更为激烈的赛事。

不过一个比赛日中,不会全是一场场的比赛,中间也有些小插曲。

比如现在正在赛场出现的马术杂技。四匹用绢花和彩帛装饰出来的骏马在跑道奔驰,马背的骑手做着各式各样危险的动作。

踩着马鞍站起算是很普通了。从倒骑,转到倒立。再从倒立的姿态一个跟头正正的坐回马鞍。看着就是惊险万分。自马背钻到马腹下,又从马腹下再转回来,动作更是如同行云流水,马术惊人可见一斑。当速度提到最高的时候,甚至四名骑手一跃而起,在空中交换了自己的坐骑。

家里的三个小家伙抓着栏杆为骑手们的动作惊叫着。前面的比赛他们还能记得要守规矩,但看到这精彩马戏的时候,终于将规矩跑到了脑后,叽叽喳喳的吵闹了起来。

当四名骑手驾驭着坐骑到了包厢前的时候,一声唿哨后,他们齐齐扯起缰绳,四匹马几乎在同时人立而起,用两只后蹄轻巧的转了一个三百六十度的圈,接着前蹄轰然落下,四人四马组成的队列又继续向前飞奔。

“好俊的马术。”韩冈看到他们的表演,就算就在军中的他也不由得为之惊叹。

“这几个是从河东胜州招募来的,全都是归化的阻卜人。”何矩叹着说道:“都说南人擅舟、北人擅马,但马术到了这个境界,真的是不一样了。难怪能成中原大患。”

“那又如何?现在还不是到了京城中耍百戏给我们看。如今与国初时不一样了。”韩冈的微笑中,却有着让何矩不寒而栗的冷意,“已经不一样了。”

何矩闻言悚然,眼底却不由自主的带出了几分崇敬。十多年来,无数异域外族的蛮夷在身前之人的手中折戟沉沙,数以万计的尸骸支撑着他的这一句论断。韩冈既然这么说,那就是事实,有资格驳斥这番话的,世也没有几人。

他的尊敬发自内心,“所以说还是端明的功劳。若不是端明,这些阻卜人进中国来,只会是跟着契丹人抢.劫,如何会老老实实的来赛马场跑马卖解?”

韩冈朗声而笑,“再过个十几二十年,来这里耍马戏的不会只有阻卜人。”

何矩跟着笑起来:“小人也盼着手底下有契丹人使唤的一天。”

马戏表演过后,紧接着就是新的一场比赛。

依然是新人的垫场赛,不过却是长程赛马,长达十五里的赛程。如何分配赛马的体力,以夺得最后的胜利,成了比赛的关键。在过去的比赛中,不是没有出现过赛马死伤的场面。

赛马的项目有长程、短程,最长的十五里,最短的三里,除此之外,还有挽马拉动重物的障碍赛——比赛场地是被跑道环绕的赛场中央——这么多的比赛项目,使得报名登记参赛的赛马已经在两百匹以。没能通过基础测试,而被拒之门外的,更是十倍不止。

每一匹新报名的赛马都是这么从最低一级的新人赛一级级的比去,到了午后接近黄昏的时候,在京城中声名广布的甲级赛马一匹匹登场,那时便是一个比赛日的最高潮。

只是韩冈对赛马的兴趣不大,包括蹴鞠在内,他更喜欢看或是旅游。锻炼身体,打熬筋骨,也不过是想健健康康的活得长一点罢了。观看比赛,他很难融入进场内激烈交锋的气氛中去。尽管两项赛事都是他心血的结晶,可即便坐在场边,韩冈的心中仍全都是对现实和未来利益计算。

有时韩冈也在想,这样的性格还真是无趣,可几十年的性格养成,他也没有改变的意思。让妻妾儿女在前面继续看比赛,自己坐到包厢最后跟何矩说闲话。

“曲礼说了些什么?”韩冈问着。

何矩低声对韩冈道,“只聊了两句。他想打探端明的身份。其实也就看端明气度不凡,想结识一番。”

“是因为有你这个顺丰行京城大掌事在身边,所以才高看一眼?若非如此,想来他也不会自己送门来。”韩冈笑了一声,又问,“曲礼是做什么营生的?”

“曲礼在东城外的河港附近很有些名气,密州人氏,在京城中做了有十几年买卖,在五丈河那一条线有一支船队。熙宁八年天下灾荒的时候,他在京东捐了一千八百石稻谷,换到了一个从九品的县尉。这两年他在京城中经常在蹴鞠球场和赛马场与人结交,认识了不少宗室和官宦人家的子弟,生意越做越大。”

韩冈点点头,示意自己知道了。熙宁八年饥荒的时候,能捐出一千八百石的粮食,家底当不是一般的厚实。

不过留心这个密州的豪商,也只是一面之缘后的心血来潮而已,对海贸易的希望,韩冈不可能放在外人的身。最后也只是吩咐一句,让何矩平日里多查一下京东商人的底细,尤其是做海贸的。虽说现在无用,但迟早能派用场。

何矩应下了,问道,“端明还有什么要吩咐的?”

“赛马是没有了,就这么比下去好了。”韩冈看看正关注着场中比赛的妻妾儿女,今天这一天用赛马打发时间看起来并没有做错,只是他又想起了在不远处的另一座球场中正在举行的比赛,“今年行里的球队在厢中联赛第一是没问题了,季后赛能走多远?能不能拿个头名回来?”

“恐怕有些难。前面大半个赛季也没能将积分拉开。如今还剩下五轮,只要败两场,季后赛可能就没有机会了。”何矩叹了一声,“今年城西厢这边的球队,进步速度太快了,”

“这样才好看。”韩冈笑了笑,并不为自家的球队担心,“可惜在赛马场这边看不到今天的球赛,要是能在一个赛场中比赛就好了。赛马场这么大,放一支蹴鞠队进来也没什么。。”

蹴鞠联赛的季后赛还没开场,但常规赛已经到了尾声。今天韩冈得闲,看一场蹴鞠比赛其实也不错,正是赛况白热化的时候,只是韩冈更想看一看赛马,故而才带了全家到赛马场这边来。若是能同时看到不同的赛事会聚一堂,感觉会更好。

“两边的总会天天打嘴仗,谁为主谁为次?哪边都难让步啊。”何矩则叹道。

韩冈摇头,就他所知,蹴鞠和赛马两个总会的关系的确是很恶劣,虽然比赛类型完全不同,但面向的人群相似,很有些瑜亮之争的意思。

又是一轮比赛结束,欢呼声猛然间从观众席爆响起来,隔壁包厢里也不像前一场比赛后那么安静了,看起来这一回不是冷门。

听着隔壁欢呼雀跃的跺脚声,还有从窗口传进来的声浪,观赛的万人似乎都陷入了狂热之中。韩冈也不禁再想,到了午后的更高级别的场次,这样的气氛不知还会如何热烈。

东京百万军民,来此观赛的有万人之多,一百人就有一名。虽然比不蹴鞠联赛比赛时,一个坊中的男女老幼全体出动,为本坊的球队加油助威;也比不两年前开始,金明池畔天子驾前争标大赛的盛况,但在绝对数目,也是足够惊人了。

东京城庞大的市民阶层中,至少有三分之一沉迷于两项赛事之中,这项产业所吸引的财富,也是一个让人惊骇,也让人趋之若鹜的数字。

单纯的农业社会,支撑不了这样的比赛。大部分农村,只有一年一度的社赛和年节时,才有百戏、杂剧或是赛事之类的活动。只有大型的城市,大量的人口和财富,才会有组织化的体育比赛。

或许这就是在工业**的进程之中附带的成果了。在棉布的生产,纺纱机和织布机的运用,大批雇工的出现,使得整个棉纺织业已经开始半工业化,与此同时纺织技术也开始向丝织业扩散。随着技术的进步,思想会转变,社会会变革,文化风俗也会相应的发生变化。文化和娱乐,越是能普及到民众,就代表着社会的文明程度就越高。

新式的生活方式,会逐渐改变了男耕女织的传统,当然不会受到普通士大夫的喜欢,贱视工商的思想仍是文人中的主流。但变革的潮流是无法抵挡的,随着工商业逐渐发达,行会的实力也在逐渐加强,市民阶层更是在不断扩大,他们需要一个与他们相配合的社会文化。

这是韩冈所期待的变化。

一场场比赛让时间过得很快。

韩冈留着何矩说了一阵话之后,就打发他出去做正事了。顺丰行的京城大掌事还是很忙的。何矩中间只是在午饭时亲自领人送了一个丰盛的席面来。

但到了午后时分,何矩脸色难看的匆匆来见韩冈。

“端明,出事了。”何矩脸色铁青,“今天行里的比赛出事了,两边球迷打起来了……死了人!”