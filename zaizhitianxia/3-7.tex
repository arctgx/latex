\section{第三章 参商稻粱计(中)}

凤翔府天兴县主簿拉着韩冈的手,亲热无比:“旧岁一别,两载倏忽而过。不意这一别之后,玉昆都已经是名满天下了。”

“思文兄太夸赞了,小弟只是薄有微名,侥幸而已。”韩冈谦虚着,“前日收到信函,知悉思文兄也将参加锁厅试,小弟可是惊喜不已,今次上京可是有伴了。”

慕容武哈哈笑道:“多蒙玉昆吉言。”

韩冈陪着笑了两声,又问道:“只是思文兄已经一榜明经,不知怎么有心再来考上一次进士?”

慕容武已经有了明经的身份,也算是个出身,只比进士差上一等而已。韩冈对此是有点惊讶,基本上没听过考了明经之后,再去考进士的。

慕容武叹了口气:“今次的机会难得啊。诗赋改经义,南方人措手不及。如果这一科赶不上,日后根本就不会再有如此好的机会。愚兄当年就是因为诗赋不成,才选了明经。如今进士科改考经义,当然得搏上一搏。”

他冲着从转运衙门中出来的几位官员呶呶嘴,轻声道:“若在平日里,哪一科都不会有这么多现任官锁厅。可今科不同。不包括玉昆你和愚兄,另外的十三人中,还有四个是辞了正当任差遣的。”

“原来如此!”韩冈点点头。

这世上还是聪明人多,科举科目改换的这点关节,他能看得出,其他人当然也能看得出来。不论是参加锁厅试的人数,还是其中的现任官的比例,应当都是远胜过往的一次。

慕容武犹在叹着:“现在都是在赌了。一个差遣得来不易,今日辞了,下一次再轮上,就不只是猴年马月了。但如若得中进士,那情况可就不同了。”

韩冈默然点头。天下文官两万,但京朝官只占十一,而官场上的进士,也仅有两千人。这两个十分之一,基本上就是重合的。只有很少的一部分京朝官不是进士,也只有很少的一部分进士不是京朝官。

一般来说,只要中了进士,升了京官,差遣就不会缺。

故而慕容武就艳羡的对韩冈道:“选人毕竟不比京朝官。玉昆你到政事堂走一趟,当场就是个差遣,挑三拣四都没问题。我等选人,就只能在流内铨外守阙了。”

“以思文兄之才,日后一榜进士,京官朝官也是等闲。”

慕容武陪着韩冈往里走。迎面而来的官员中,有不少人认识韩冈,就算没见过,听着身边同伴提醒,也都知道名闻关西的韩玉昆来了。

韩冈的晋升速度让人匪夷所思。向他投来的羡慕、嫉妒的眼神,也是韩冈所见惯的。但该有的礼数,这些眼神的主人却没有一个敢缺。韩冈的官品,眼下在转运副使刚刚离任的秦凤转运司中,也只有蔡延庆高过他一头,就连转运判官蔡曚都已在韩冈之下。这样的身份,没几个敢于在礼节上有所疏失不敬。

与这群官员行礼问候,一番纷扰之后,韩冈方才脱身出来。

刚刚转上一条长廊,无巧不巧的,蔡曚正好带了数人,面对面的走了过来。

与韩冈对上眼,蔡曚便停下了脚步。

从品级论,蔡曚低于韩冈,不至于要避道,但先行行礼却是应当的。但蔡曚站着没动,而韩冈停了一下,便主动上前拱手:“韩冈见过运判。”

见韩冈先有动作,蔡曚这才板着脸,回了一礼,就径自扬长而去。

看着蔡曚走远,慕容武便道,“不是听说玉昆你跟他不和吗?怎么还对他先行礼!”

“既是锁厅试同知,礼法上已是吾等师长,自是要让上一让……前面思文兄推着小弟,难道不是在提醒吗?”

慕容武呵呵两声,笑而不语。

参加科举,主考官与考生之间,理所当然有着师生之谊——也就是所谓的座师、门生的关系。在唐代,甚至有着传衣钵的说法。虽然本朝的太祖皇帝因为不喜官员结党,在礼部试之上,又设立了殿试,所有的进士,便都成了天子门生。不过在下面的贡举中,却并没有严令禁止这样的师生关系——因为并不需要。贡生中不了进士,第二次就要重考,无法稳定下来的师生关系,朝廷也不需要顾忌。

只是这个名分依然存在,韩冈尽管根本就看不起蔡曚,还在熙河的时候,他还将蔡曚压得束手无策,一点面子也不给。但换成是眼下的情况,他却不会做些坏名声的事。何况遵守一下世间通行的习惯,也不会掉块肉。

而就在长廊外侧的庭园里,被几株郁郁葱葱的桂树挡在后面的凉亭中,一人收回视线,“官位在他这个年纪,已经算是绝高,却难得还如此守礼,你那个运判就差多了……张横渠还真会教徒弟。”

蔡延庆笑了,意味深长的道:“若非如此,韩冈哪能得到如此多人的看重?”

与蔡延庆对坐在凉亭中的那人有些苍老,年岁五六十的样子,但一对眼神却是犀利深刻,仿佛能穿透人心。如果这等眼神用到审案上,一扫之下,被审的贼人怕是要汗出如浆了。

他盯着手中的酒杯:“不管是真心守礼,还是虚饰而为,能做出来就是好的,不必求全责备。”

“……公佐还是这般宽厚!凤翔府上下有福了。”蔡延庆笑着举杯致意,不以为忤。

老者捏着酒杯:“韩玉昆应是来报到的,仲速你不去见一见他?”

“不必。虽说锁厅试没有什么可避讳的,但是见面还是事后再说吧。”

“说的也是。”老者一笑,遂举杯与其相和,将杯中水酒一饮而尽。

……………………

韩冈去了转运司衙门,只跟蔡曚打了个照面,并没见到转运使蔡延庆。他从衙门里小吏口中,打听到蔡延庆是跟来访的新任凤翔知府苏寀喝酒去了。真不知道,苏寀不去接任,跑到秦州来作甚?而没能见到主考官,当然是个遗憾,但登记下名字也已经足够。

要知道,如果是礼部试或是地方军州中的贡举,考生根本都不能事先面见出题的主考官,以防考题事前泄露。只有锁厅试,在事前才会管得松一些——因为在这项考试上作弊根本没有意义。

就算能在锁厅试上蒙混了过去,到了京中的礼部试上,照样折戟沉沙。而且连续几次应考不中后的特奏名得官,对参加锁厅试的官员们并没有任何用处,他们求得就是一个进士出身,而不是普通书生要的进士背后的做官资格。如此一来,对锁厅试的约束之宽松自是理所当然。

不过去转运司走了一趟,好歹见到了一个同学,同时也了解到了一点参加锁厅试的考生们的情报。今次参加锁厅试的总共有十五人,按照十中选三的比例,入贡的名额有四个——这里面没有四舍五入的说法。

“能放下到手的差遣而参加举试,那四人当是有些自信的。至于其余几个,就不用太在意了,大半是参加过前科的老面孔,只是来凑趣的。”

从衙门中离开,慕容武依然跟韩冈同行。

秦州的贡举也就在这两日,街巷上的士子多了许多,他们多半是借住在寺庙中,而有钱的,则是进了好一点的客栈。不过韩冈就没必要去挤寺庙或是客栈了,而是直接邀请慕容武,到他过去住了很短的一段时间的旧宅中。

慕容武当然不会拒绝,在街口分了手,准备回去收拾行李,韩冈则是派了剩下的一个伴当跟着去,也好带路。

韩冈当先回到旧宅,正厅中就是摆放一堆礼物。听着事先带着行礼回来的伴当的汇报,这是秦州的几大商号知道韩冈要来参加贡举,天天派人守在门外。一等伴当回来,听到准信,就立刻送上了各色礼品。

韩冈低头看着礼单,琳琅满目的倒也有着不少贵重的东西。商人重利,送得东西多了,要得回报当然更多。随手就将礼单递给伴当,“造册后就收起来,不要弄混了。”

在考试之前,韩冈无心于这些商人打交道。而商人们也很识趣,也不在这时候打扰韩冈。

与慕容武在家中一起读书。几日功夫,就一晃而过。转眼之间,就到了考试的日子了。

坐在转运司的偏厅中,十几张桌案整齐的摆着。韩冈等十五名参加考试的官员,都已经就位。

只有十五名考生的锁厅试,并不需要什么参详官、封弥官,如同礼部试那般林林总总几十名考官,蔡延庆和蔡曚直接担任了正考官和覆考官,题目也是两人所出。

考卷都是印发好的,书写姓名、籍贯都有规定的地方。因为官场上用的空白公文,有许多也都是事先印好的,需要用时,就直接填空。习惯了的官员们,也不需要人提醒,自己就将姓名、籍贯填好。

题目张榜而出。其中有经义十道,加上策问一道。

不需纠缠日久,一天之内就这场考试解决。

跟上京后的礼部试一样,地方上的举试同样要糊名誊卷。也就是将考生的姓名掩盖,再让吏员将之誊抄。考官是看着誊抄后的副本。这是以防考官与考生私下串通。锁厅试管束虽不严格,但必要的程序却不会少。

但蔡曚有十足的自信,就算所有考场上的吏员都是蔡延庆亲自挑选,但他依然有把握将韩冈的卷子给认出来!

确认了考生身份,不论是要悄悄的将之黜落,还是要将蔡延庆一起陷进去,都不会很难。

阴阴笑了起来,他就等着考生们最后交卷。

