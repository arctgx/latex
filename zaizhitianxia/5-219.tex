\section{第24章 缭垣斜压紫云低(11)}

赵颢自然是得意的。
原本都已经绝望的心思,却在最后一刻峰回路转,要说这不是上天对自己的垂青,赵颢是怎么也不会相信的。

眼下赵颢自问已经把准了几名重臣的心思,全然不为自己的突兀而担心。

也怪躺在床榻上有出气没进气的兄长,有气节,有胆略的臣子不是请出朝堂,就是担任闲职,选择的宰辅几乎都是谨小慎微的性格,哪里可能会拿举族安危冒风险,只为了逞一时之快。

当年以包拯之清介,在他上书请求立嗣而遭到仁宗的质问后,也得自陈年迈无子,没有私心,才让仁宗皇帝释疑。

事关举族安危,谁敢不多加思量?

其实这也是个试探,赵颢倒向看看两府宰执中最后有哪位或是哪几位会忍不住站出来驳斥自己。此外,他更关心目下在寝宫中的一干人中,到底有几人对自己有着提防之心。

赵颢注视着每个人神色或是动作的变化。但也有人在确认了赵顼的病情后,便反过来关注着他这位离御座越来越近的雍王殿下。

殿中的明眼人甚多,赵颢用来试探的小伎俩,除了一门心思放在皇帝身上、无暇他顾的几人外,不论是站立在数万官僚最顶端的宰辅,还是在深宫这个污水缸里挣扎出头的内侍、女官,大多都能看出个大概来。

赵颢直接针对的就是两府宰执,他们的感触是最深的。但就跟王珪这位宰相一样,其他执政也都基于同样的理由,而对赵颢的挑衅视而不见、听而不闻。

谁能保证雍王无法继承大统?

谁能肯定赵颢登基后不会为今日之事而报复?

赵家的皇帝,除了太祖和仁宗之外,倒是小鸡肚肠的占多数。太宗皇帝是其中的典范,兄长就不说了,嫂子、侄儿、弟弟,都是死的不明不白;真宗对寇准的处理也是一条明证;至于英宗,他上台后便将传言中反对他即位的蔡襄远远地发遣到福建,仁宗皇帝尸骨未寒就闹起濮议之争,同样是他赵曙心性最好的证明。

与天子日日接触的重臣之中,没人会对赵家皇帝的品性寄托太大的希望。

让看不顺眼的臣子家门败落从来不是难事,甚至简单到并不需要明面上的报复,只要让其家族后人无法进入官场,那个家族自然而然的就会破落下去。要知道太宗、真宗时的名相王旦,其子是熙宁六年去世,以工部尚书致仕的王素,他家传到重孙辈后,就已经败落得要靠天子恩典才挣得一份俸禄了。

轻而易举,举手之劳。动动嘴皮子就够了。

为了一个已经让权力从手中掉落的天子,而付出家门毁灭的代价,年轻人或许还有着这样的棱角和热血,但早就在朝堂上打磨得光滑圆润的金紫重臣,如何会这般意气用事?

并不是所有人都有王安石或是韩琦的胆略,这也是名臣和庸碌之辈的差别。可又有几人会愿意为一个虚名而毁家纾难?几位宰辅,当然是一个个装聋作哑,对赵颢的话不做任何反应。

不,确切的说,是除了一人之外的所有人。

拧着眉头的章惇咬了咬牙,终于还是决定自己站出来。

在他的立场,是绝对不能让赵颢登基。当年出手帮助韩冈,讨好赵顼、打压赵颢,三方面做得漂漂亮亮的可就是他章子厚。纵然明面章惇并没有参与太多,但他私底下的出手,也别指望赵颢登基后会查不出来。

既然其他人都退缩了,就连最应该维护宰辅权威的王珪都不发话,那么能出来打下赵颢气焰,让赵顼的后妃们明了赵颢的迫不及待,也就只有他章惇了。

至于韩冈,章惇暗叹了一声,两府的权威,区区一个端明殿学士、判太常寺是没资格来维护的。

章惇挺直了腰,狠厉的眼神锁住赵颢,便要踏前一步放声说话。但眼前突然之间插入一个背影,让章惇的动作一滞,不得不停了下来。

定睛一看,却是韩冈不动声sè的抢先一步拦住了他。仅仅是悄然移动了小半步,却正好挡在了章惇的前面。

章惇又皱起眉头,却没有说话。他不知道韩冈打的是什么主意,但他相信韩冈的判断力,不至于在这么要命的事情犯浑。强压下一口气,章惇站定不动,等着韩冈的解释。

“稍等。”

韩冈的声音很轻,只让章惇一人听见,又简洁得让人郁闷。可多年知交,又在南疆同历生死,互相间深厚的信任,让章惇勉强压住心头的急怒,决定暂且稍等片刻,等待韩冈的行动。至少他能相信韩冈的头脑,不会糊涂到将全家人的xìng命放在雍王殿下的人品。

赵颢重又将注意力放回到病榻的兄长那里,他已经得到了自己想要了解的信息,让他心情十分愉快。他甚至看见了韩冈和章惇私底下的小动作。韩冈以胆大著称于朝,可事到临头,连他都退缩了。

所谓的忠心,所谓的大胆,到了生死关头,也不过是个笑话。

当真以为逃得过去?!赵颢暗自狞笑。

当年的旧事,这些年来成为杂剧和唱本在都下传唱。贵为亲王的颜面,在做皇帝的兄长的放纵下,早就被践踏得如同死水沟里的臭泥一般。多年积累下来的羞辱,一rìrì的在胸中yīn燃。这一回,当真能如愿以偿,搓扁捏圆也是一句话的事。当年的旧怨,可不会那么简单就放过!

有大半的心力都放在赵颢那边,雍王千岁的一举一动,一点不漏的映入韩冈的眼底。

也难怪赵颢会如此得意。

原本作为天子,尤其是掌控天下十余载,有军功有政绩的权势天子,赵顼能轻而易举的控制宫廷内外。他要立谁为储君,高太后完全干预不了,只有附和的资格。可现在的情况,却让形势倒了个个儿。

病榻的赵顼,如果不能在短时间内清醒,那么皇权必然会转移到高太后的手中。到时候宫门一关,谁知道宫城会变成什么样?没有赵顼做后盾,少了主心骨的皇后能争得过太后吗?

纵然控制了内宫的太后不会对孙子下手,但她最疼爱的次子,想要收买几位宫女内侍却不是难事。届时赵佣又能活几天?就算之后事发,高太后还会为了孙子,将儿子丢出来让世人唾骂不成?甚至赵颢不用对侄儿下手,只消求到高太后面前,说不定高太后心一软,直接就让赵顼退位,让赵颢来了。

所以眼下的关键是必须能让赵顼与人沟通才行。得在有众多宰执佐证的情况下,让赵顼做出有意识的反应。可以赵顼现在的状况看,恢复正常的可能xìng几乎为零,甚至很有可能会在下一刻便就此睡过去,不会再醒来。

将一切的希望放在一名重度中风的患者身,希望他能够醒转并清晰的表达自己的心意,难度可想而知。

为什么章惇会这么心急如焚,就是他一切都看得清楚,却束手无策。

赵颢也看得到这一点,同时也明白他兄长的臣子们对此完全没有办法。一股充满恶意的快感传遍全身,雍王殿下胸中的得意几乎掩藏不住。

韩冈能清晰的感应得到从赵颢身透出来的得意。雍王千岁很隐晦瞥过来的眼神,让韩冈心中明白,当年的仇怨没有半点解开来的迹象——这一件事,在桑家瓦子、朱家瓦子等京城中的娱乐场所里演的杂剧剧目中,早就有了认识,今天只是更加确认。

如果当初的争端仅仅是一介女子,或许雍王殿下还能一笑了之,以求在青史中留个好名声,但个人的名声在十年间一直被人践踏,那已经不是恨意,而是接近于杀意了。

要远离充满危险和杀机的未来,就需要病榻的天子能够清醒过来。可是韩冈可以将期待投注在别人身,但自己的命运他只会想方设法的控制在自己的手中,绝不会交托给他人。

所以必须立刻有所行动。韩冈深吸一口气,已经没有时间给自己犹豫了。

幸而并不是没有办法。韩冈自问还是有能力将赵颢看着就让人心情不快的那张脸,一巴掌给打回去的。

倒也不是说韩冈比章惇——还有其余宰执——聪明多少,但比起见识,他们至少差了有一千年,论起装神弄鬼来,以韩冈给自己编织的光环,一干宰辅更是远远比不他。要不然方才也不会宰执们全都只能守在外殿,而韩冈一到就被迎进寝宫之中。

韩冈离着赵顼的病榻有一丈之遥,在他面前的,是赵顼的皇后、嫔妃、儿女,以及母亲,以及似乎都要哭出来的王珪。宫女、内侍和御医,跟韩冈一样进不了榻边的那个圈子。

亲疏有别嘛,不会有人自讨没趣。

其实赵颢也站在外围。作为一母同胞的兄弟,的确是十分亲近。但作为帝位的威胁,说疏远,却也是一点没错。不过韩冈猜测他他应是想更加仔细的观察所有人,才特意站在外面。

要走近天子的病榻,韩冈只是轻咳了一声。

声音不大,却仿佛夏rì午后的一声惊雷,划破被yīn云遮蔽的天空,震散了空气中的压抑。

赵颢浑身一个哆嗦,凶戾的目光一下便移到韩冈的脸。眼神惊疑不定,心也一下抽紧了。难道这厮还有什么手段扭转乾坤?纵然恨其入骨,但赵颢绝不敢小觑韩冈半点。

只是转眼之间,高太后回头,王珪等宰辅也回头。当向皇后、朱贤妃回望过来,发现出声的是韩冈时,她们的眼眸中便重新焕发起神采。

章惇盯着韩冈的背影。

‘他要做什么?’不止一人这么想着。
