\section{第15章 焰上云霄思逐寇(13)}

从帐中出来,宗亶抬头看了眼看不到星月的夜空,淅沥沥的细雨打在脸上。心中无悲无喜,只有一声长叹。

他没有说服李常杰,却反而被李常杰给说服了。

李常杰说得没有错,不能让宋国小瞧大越。

大越国偏处天南,从中原到国中,有万里之遥。余途又多有瘴疠,北人水土难服。要是宋人当真派了三五十万大军南下,最开心的就该是国中一众君臣了。

就算是富庶如大宋,要想支撑三五十万兵马的日常食用,也是极为吃力,而且还是往边疆运送,难度只会更大,这样的攻势根本不能支撑太久。

而更危险的是疾疫。人聚集的越多,疾疫就越容易发生。他们是更南方的交趾人,这一次北征都只敢选在冬天,而北面的宋人往交趾去,就是冬天也一样容易染上疾疫,到时候就是几千几万的不断病死,不用开仗就必须要退了。

而且人马一多,调集起来的难度就越大,无论前进撤退都是要大费周章,这样的大军,如同猪一般的榔槺夯货,根本就不需要担心他们能起到多少作用。

即便宋人只派十几万兵马,辎重的转运,疾疫的防治,难度也不会降低多少。只要设法拖延时日,就能让宋人不战自退。

可要是宋国派兵派得少了,大越真正用以对抗宋国这个庞然大物的武器,就要失去了作用。

如今宋国的一个新任的转运副使,加上苏缄的儿子、荆南的都监,领着八百兵就闹得近十万大军天翻地覆。只要这份战报传回汴京城,宋国的君臣多半就会认为只要五六万人就足够踏平交趾了。

相对于十几万、几十万的浩浩大军,少数的精锐,对大越的威胁反而更高。

虽然心中不服气,但从这几战的表现上来看,只要排除掉广西的一群久不老弱,真正能上阵的宋军,其战斗力都是要高过大越国最精锐的天子军,尤其是他们所用的强弓硬弩,更是难以应对。

如果派来征讨大越的宋军不输这八百兵多少的实力,只消五六万人来周旋,几场大战后,就能将国中的主力给扫平。无论是疾病还是辎重,都不会对几万宋军有太大的影响。

正如李常杰所说:“如果只是三五万兵,宋人肯定是用得起,也耗得起。但我们耗得起吗?”

——大越国不怕宋国派的兵马多,只忧其人少。

李常杰所以才会要重夺昆仑关,所以才要消灭那八百兵马。

秦国灭楚。始皇一开始不想多发兵,先派了二十万去,结果全军覆没,后来没有办法,同意了老将王翦的要求,点集六十万兵马,才将楚国一举灭亡。

李常杰向宗亶提起这个典故,就是要让他明白,越是表现出强盛的国力军力,宋国对大越的就会越重视。要让宋人对大越国实力的判断,如同秦将王翦对楚国的判断。要让宋国多派兵马,到时候,只需要用天时、地利、人和三项,就能让宋军自灭。

其实相对的还有一个办法,就是尽量向宋人示弱。让宋人小觑大越到了登峰造极的地步,以为只需要用上一两万兵马就能成功,那其实也是件好事,但可能性寥寥。大越国再怎样也是万乘之国,从十五到六十的男丁全数征发起来,至少能组织起三十万大军,宋人再小觑也不至于会到如此的地步。

“现在宋人有了黄金满,只要他回到州中,依仗宋人威势振臂一呼,原本依附在刘纪等人帐下的小部族全都会投靠他门下。不过相应的,刘纪三人为了自己的地位,则会全心全意投效大越。这样一来,我们又平添了几万助力。也不用担心他们会望风而倒。如果示弱过甚,刘纪三人就算不甘愿屈居黄金满之下,也必须投靠宋国。到时候,我们还要多对付广源州的几万敌军。”

李常杰的解释掩盖了他的私心,宗亶则是心知肚明。为了他在国中的声望地位,也是为了自己身家性命,李常杰就算死也不可能去选择这一项。

……………………

轻易的解决了李常杰的前军,主力又顺利的返回,昆仑关中一片喜气洋洋。

虽然没能打到邕州,但让贼军撤离了邕州城,保住了城中百姓,同时又通过几次战斗,立下了诸多功勋。最后还安然返回昆仑关,这样的战斗虽然累上一点,用来交换即将到来的封赏,八百荆南军将士只会盼着多来几次。

而对广源军来说,跟着大宋官军,最需要拼命的战斗有人打前阵,而摊到自己头上的则是更为轻松的追击和迎击。轻轻松松的捡功劳,几场大战下来,连人都没有损失多少,比起跟着交趾人要好得太多。

韩冈也满足了,他这一路上立下的功劳足够多,而且每一步的行事,除了稍显急进以外,没人能挑出错来。此前撤退,也是手中的实力不够,非战之罪。而且除了苏缄以外,自己已经尽可能多的救下了满城百姓,他也没有什么不满意的。

只不过他的对手好像很不满意。在交锋中已经经过多次失败,李常杰依然不肯撤离,反而在大央岭驿扎下了营盘。这个消息让韩冈的脸上多了一丝讥讽的冷笑,就算在军议时也没有褪去。

“李常杰贼心不死!”李信嘲笑着李常杰的愚蠢,“这是自寻死路!”

“羞刀难入鞘,他是不愿意丢人现眼的回去。不过他应该还是有所谋划,”韩冈提醒着表兄不要太过小瞧了敌人:“有了夺下昆仑关的希望,否则也不会有这般愚行。”

李信对昆仑关的几次易手有所了解:“当是打算前后夹击。”

“说得正是。”韩冈点头道:“李常杰至少有两万兵马,必要时还能调出更多的兵力来。想必以李常杰的打算,是从山间小道绕行至昆仑关背后,试图前后夹击。”

“小人已经派了得力之人去监视,一万多人想在近处绕过去,绝逃不过他们的耳目。”屡立功勋,黄金满现在在韩冈面前有足够的分量参与军议,“如果从哨探不及的地方绕道,则至少要七八天的时间,这还不算这几天的雨水。”

韩冈低头看着地图:“多半还是从近处走。两边事先确定好时间,一边攻打昆仑关,一边则强行通过小道。”

“那以运使来看,我等该怎么应对?”黄金满问着。

“在关中好生休整就是了,等李常杰出兵来攻,直接出关反击。他既然分兵,我们正好可以各个击破。再怎么配合严密,两边消息不同,也会有一天半天的差距……”韩冈呵呵笑了一笑,“已经足够了!”

八百兵都是精锐,加上这些天来的战事,只消耗了体力,并没有损失人马,军心士气正是高昂。只要休整三两天,就能彻底恢复过来。

韩冈打算采取的战法依然与第一次归仁铺之战相类似,以荆南军为先导,给交趾军猛力一击,等交趾军被击溃之后,就交由广源军为。在狭窄的山谷中,兵力多寡的问题,远不像平原上那么严重。直接出兵击溃李常杰,韩冈有充分的把握。

军议很快就结束了,当务之急还是休整。李信和黄金满告辞离开,韩冈则留了苏子元下来。

这两天苏子元沉默了许多,许多时候,只做事,不说话,方才的军议上也是如此。韩冈觉得有些不对劲,要与他聊一聊。

被韩冈单独留下来,苏子元也知道这是为什么,但他还是提不起精神来。前面有着邕州城作为诱饵,就算邕州城被攻破,他心中还有一丝希望,拼命的为韩冈献计献策,但现在他们所处的位置离开邕州越来越远,在邕州归仁铺绕了一圈子后,就又回到了起点。

虽然苏子元很清楚这不是韩冈的问题,巧妇难为无米之炊,八百人就算再怎么折腾,都不可能变成八千人。说起来只有李常杰手下总兵力的百分之一。以相差这般悬殊的兵力,韩冈能取得如今的战果,说起来也算是难能可贵了。

但心中的郁结不是用道理能开解得了,父母兄弟妻儿子侄,很可能已经都不在人世,一家近四十口人,到现在就他只剩下一个,这几天他满脑子的都是家人的音容笑貌。

“伯绪,你是知道邕州存粮数目的,”不论从桂州军事判官还是从苏缄的儿子,苏子元对于李常杰抢掠到手的军粮数目,应该是眼下最清楚的,“以你看来,李常杰他们还能支撑多久?”

苏子元怔了一下,想了一阵后道:“至少再有一个月,邕州是边城,永平、太平等几个寨子中都有大量存粮,这就能支撑他们到现在。而且还有当地的百姓,交趾军烧杀抢掠,百姓的囤粮也都被抢光,再多一个月很容易。”

“果然还是用拖还是不行,只能全力一战。”

“李常杰贪功好杀,不知进退,这是自取灭亡之道。”

韩冈决定还是不说安慰的话,许多时候,男人不需要安慰,而是需要用工作来分心。他不会说什么吉人天相。知父莫若子,苏子元既然都认为苏缄已经不在,韩冈也不会觉得他想错了。以韩冈对苏缄粗浅的了解,也很清楚他必然会死战到底。而且要是他落在交趾人手中,必然会被拿来劝降,苏家人甚至连一个都没有出现,很有可能是满门死节。韩冈能想明白的事,情官至亲的苏子元如何会想不到。

一番讨论之后,韩冈送了苏子元出来。一出帐,下面亲兵就送上了油布雨衣,苏子元停了步,望着头顶上漆黑一片天空看去。

“怎么了?”韩冈问道。

“雨好像大了一点。”

不是大了一点,到了午夜之后,类似于清明时节的纷纷细雨,已经噼噼啪啪砸着,虽然不到暴雨如注的地步,但一刻也不停歇的大雨,在山中已经汇聚成河流。

用兵三要,天时、地利、人和。

“老天爷这是不想让人打仗啊!”韩冈在关城上低头望着城下的水洼,皱着眉喃喃自语。

