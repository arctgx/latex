\section{第36章 可能与世作津梁(三)}

【春节将至,先祝各位书友新春愉快。另外再说一句抱歉,节日的几天,更新可能会有些问题,不过会保证不断更,至于第二更,也会尽力。】

说是接风宴,但韩冈和沈括在席上都没怎么喝酒。

酒宴过后,韩冈一行便被安排到了寅宾馆住下。沈括则是前面领路,将韩冈请到了后堂中。

一进大厅,就看见了一幅六尺见方的沙盘模型。

韩冈不以为异,沈括主持制作的熙河路沙盘,比起自己当年所制作的旧型,要精确了许多,在测绘测量上,沈括的水平远比韩冈要强,沈括到了唐州后,当然也不会把自己的特长给丢掉。只是走进了才发现,沈括制作的这幅沙盘,比他预计得还要好。

这是一幅是以襄汉漕渠为中心,将方城山附近的地理地形塑造出来的沙盘模型。比例尺、经纬线、图示、方位标识,后世地图该有的标识,一切都不缺——其中有韩冈的功劳,但更多的还是沈括的天才——不过最重要的是,这副沙盘,跟韩冈在京城的时候所看到的唐州地图并不一样,有很大的区别。除去精细度的问题,主要就是山区等难以通行的崎岖地带的面积收窄,而平原面积放大。

“这是以飞鸟图为本吧?”

沈括很是自得的点着头:“正是飞鸟图。”

飞鸟图是沈括所提倡的制图法。所谓飞鸟图,顾名思义就是仿鸟飞直线所绘制的地图,排出地形地貌的干扰。

过去的地图是受到地形地貌的影响,许多时候会参照实际行程而来,越是难行的地方,就会有越大的误差。这一点当然有问题,在山区,一天三五十里,到了平原上,百里也很寻常,所以绘制出来的地图与实际地形有着很大的差异。但日常使用起来反而效果不差。毕竟此时的地图,多用来指挥行军,参考地图能确定实际的行程日期。

如今沈括绘制制作地图和沙盘所使用的飞鸟图,图上的距离如空中鸟飞直达,排除地貌所引起的距离误差,用后世的话说是实际地形垂直投影到平面上的距离。这一制图法究竟是谁发明的现在是说不清了,但出自于制图六体——晋代裴秀所确定的‘分率、准望、道里、高下、方邪、迂直’这制图六原则——是毫无疑问的。韩冈是沈括自制作熙河路的地图和沙盘开始,才第一次看到。而能制作出飞鸟图,并以此来制作沙盘,其实是测量仪器和手法上了一个台阶后的结果。

不过眼前的这幅唐州地理,比起前两年的水平更进了一步,韩冈指着沙盘问着:“比旧时精确多了。存中,你是不是又有什么发明?”

“的确。”沈括自负的点点头,“前些日子为了测量唐州的山势、水程的高下,以及道路远近,愚兄将经纬仪和测距仪又改进了一番……只是时间太短,只能匆匆而就。若是再给愚兄两年,唐州的山水便能尽在图上。”

“可惜存中兄不一定能有两年时间。”韩冈笑道,“一旦襄汉漕运打通,区区唐州又岂是待贤之所?”

韩冈说得好听,其实就是要让沈括卖命,但沈括听得也高兴,要不是为了将功赎罪,去江南不比唐州更好?

韩冈与沈括共事的时间并不长,两人的交情就是依靠对自然科学的共同兴趣而逐渐发展起来的。相对而言,韩冈是刻意而为,沈括就是出自本心了,一说起机械构造时,立刻就变得滔滔不绝起来。

沙盘模型还算是正常,但沈括接下来展示给韩冈的却是让人瞠目结舌。

——竟然是多级船闸的模型。不是放在后堂中,而是砌在州衙的后花园中,架在从外面引来的一道活水上。整座船闸模型有一人髙,从高到低分了五级,船闸的闸门开合启闭都可以通过绞盘来实现。而船闸中的流水是靠着一架小水车来提供。

沈括一声令下,用人力推动的小水车就哗哗的将水提到船闸的最上端,让清澈的水流从自上流淌,船闸上的闸门在绞盘的控制下依次打开,让一艘小小的木船模型,沿着船闸通道,从最低处的水面一直上溯到近一人高的最上层。

沈括指着船闸模型,对韩冈道,“玉昆你的这个发明一下就解决了船只翻越堰坝的难题,看着简单,可若是不点破,任谁都想不到。”

韩冈摇摇头:“我那只是图上文章,真正实现的还是要靠存中你的手段。”

韩冈和沈括互相吹捧了一阵,最后哈哈一笑。

沈括又问道:“玉昆你是不是打算将漕司治所安在襄州?”

韩冈点头:“襄阳是漕运源头,襄州比洛阳更合适。”

“那你最好能留一两个心腹之人在唐州,这样你我也好联络。”

“也好,我这里正好有两个合适的人选。”韩冈点点头,忽然又想到了什么:“前两日还在洛阳的时候,李楚老【李南公】说他的次子在工程营造上小有才干,想荐他入我幕中。等李二衙内来了之后,我会安排他也留在唐州,在存中你身边听候差遣。李楚老是聪明人,当能知道我的心意。”

“李南公的次子?好像是是考了好几次进士都没有得中的那一个,听说都已经弃了功名,出去云游了。还当真敢推荐了过来。”沈括无所顾忌的嘲讽着:“李楚老要是能少些私心,也不至于四十年宦海,才到了如今的位置上。”

韩冈笑了一笑。沈括说得其实没错,李南公自然是看到了打通襄汉漕运的好处,才想方设法的把他考不上进士的次子塞进韩冈的幕府。不过李南公是转运副使,漕司中的许多事少不了他帮手,韩刚也不介意来个闲人分功劳,“李楚老家学渊源,想必他的次子在治事上,也能有所助益。”

韩冈与沈括一直聊到了深夜,回到了寅宾馆,短短的睡了两个时辰,韩冈又起来和沈括去了城外,查看即将成为襄汉漕运主要通道的泌水和堵水。

唐州的州治在泌阳县,襄州的州治自然是在襄阳,两地之间相距很近,而且可以通过泌水相连,交通十分方便。

这条泌水是襄汉漕渠最为重要的组成部分之一。襄汉漕渠的南段是汉水,当漕运自汉水抵达襄阳后,便转由泌水北上,抵达泌阳。过了泌阳之后,转入泌水上游支流堵水,进抵方城山下。

而需要开凿的渠道,提升水势而设立的堰坝,都是从堵水的上游设置和出发,越过方城山,汇入北麓的沙河。当水流越过方城垭口之后,接下来就是一路平川,直抵开封,真正需要开凿的渠道其实很短,绝大多数的地段,都能借用现有的水路。如果不是因为方城垭口的让人叹息的地势,如此难得的漕运通道,绝不可能被放弃。

初春的桃花汛已过,而夏天尚未到来,泌水也显得份外和缓,但发源自方城山深处的河水,有着足够的深度,可以通行额定七百料的纲船。

“不过汉水、泌水都有足够的水深,只要将上游几处险阻的河槽清淤,千料甚至千五百料的纲船都可以抵达方城山下。”沈括和韩冈驻马河畔,望着潺潺的流水。

“过了泌阳县往上,堵水的河道可是变浅。”韩冈经过方城山南下时,已经查看过沿途水道的情况,“汴水为了能通行七百料纲船,水深要保证在六尺上下。千料船、千五百料船,又该要多少?怎么说能让千五百料艘通行至方城山?”

“在河中筑低堰来抬高水位。连续筑堰,能将沿途的水位逐级抬高。”沈括答道。

“就像是灵渠?”

沈括点头:“为了打通漓水和湘水,秦人就在湘水中修了拦河的大小天平,抬高了上游水位,又分出一部分湘江水流注入灵渠。但这么一来,湘水上游下游的沟通就被大小天平给堵死,为了能让船只顺利的往来湘水上游下游,就又开辟了一条北渠出来。若设了船闸,就不用开凿北渠,直接用船闸翻越拦住湘水的天平石堤。而灵渠中,也不用三十六道斗门,一座船闸足矣。”指着面前的河水,“在堵水上逐级设立堰坝抬高水位,纲船通过一个两级船闸就翻过一座堰坝,几个船闸前后一过,很容易就能抵达上游。”

韩冈深有感触的点头表示同意,他对灵渠很熟悉,沈括说的话很有道理。

“而且筑堰之后,”沈括继续说道,“引水浇地也变得方便起来,还有如水车这样的各色利用水力的器具,都能派上用场了。”

有了沈括,果然省了不少的事。开启襄汉漕渠,最大的难题就是如何让渠水越过看似平缓但却比南北两侧都要高出六丈的方城垭口。现在由沈括在唐州主持工役,一方面抬升堵水的水位,另一方面则向下深挖河道,北面的汝州再加以配合,要打通这一道关口,却是便得轻而易举,仅仅需要时间而已。

韩冈现在要做的,就是将方城垭口的轨道修起来。打通漕运后,将整条运河完工的工作,完全可以交给沈括来做,自己只要在襄阳享受成果就行了。

韩冈甚至可以将最后的功劳都让给沈括,占个首倡和掌控的功劳对他来说已然足矣,他现在的目标是学术上的地位,而不是实务上的功绩。

