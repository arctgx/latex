\section{第12章 兵蹙何能祓鬼傩(下)}

苏子正一向不喜欢下雨天,尤其是又湿又寒的冬雨。一遇到冬天下雨的时候,他的胳膊就隐隐作痛。当年侬智高作乱的时候,他随父上阵,却不小心从马背上摔下来,伤了一条胳膊。尽管早已经将养好了,也看不出来曾有过旧伤,但二十年来,苏子正一年比一年更加讨厌不依时节的降雨。

但今天的雨,让苏子正觉得很快活。尽管又麻又痒又酸又痛,五味杂陈的感觉,难受得让他想将胳臂给砍掉。但交趾贼军难以攻城的现实,一下就压过了胳膊上的旧创。

他手下的士兵也都兴奋的看着天上阴沉沉的雨云,雨下的越大,交趾人就越不可能出来攻城。

“要是下个十天半个月就好了。”也不知是谁人说得,却立刻得到了所有人的赞同。

“好雨知时节。”苏子正哈哈笑着,不应时节的雨同样是好事,阻止了贼军,也帮了城中五六万军民一个大忙。

回头看看城内,多少人将家里的锅碗瓢盆桶缸坛罐,只要能盛水的东西,全都拿了出来。摆在了屋檐下,接着从屋顶上淌下来的雨水。

终于有水了。

苏子正张着嘴,不顾仪态的接了几口雨水。他的嘴唇与下面的士兵一样干裂着,冰冷的雨水对于他们就跟甘露一般。

交趾人一个月前就断了城壕连同左江的源头,城壕中的水都流光了,苏缄让人堵上了水门,省得交趾贼军钻这个孔子。可是没了城壕输水进城,邕州城就断了水。

邕州城中缺乏水井,也就是几个大户人家和衙门里掘了井。普通百姓日常生活,都是靠引入城中的左江江水,只比桂林城边的漓水略浑一点,直接就能喝下肚的。

现在水源既然断了,就只能依靠不多的几眼井水,刚刚挖出来的几眼水井都不堪用,大部分人一天只能分到一两碗水。

‘要是子容【苏颂】表兄在就好了。’苏子正在喉咙火烧火燎的时候一直在这么想着。或者是他父亲曾经大家夸赞的韩冈也行,都是精通机关巧器,应该都懂怎么掘井。听说及时将神臂弓送到邕州的韩冈,旧年曾在京畿开凿深井,井水旱涝不绝,但韩冈还要向子容表兄请教。

天色渐渐的暗了下去,虽然有云遮挡,已经快要到黄昏了,再过一阵就要全黑了下去。“去将灯油准备好,天再暗一点就点上。”就算是雨夜,也要将城上的灯火都点起来,能照亮城头,苏子正有点担心着交趾贼军会来偷城。

“衙内,该吃饭了。”

不用亲兵的提醒,苏子正已经闻到了晚饭的香气。几十名健妇抬了热腾腾的米粥从城下上来。米粥一锅熬得浓浓的,里面掺了盐。除此之外,也就没有别的了。城中的粮食将尽,现在只能省着点吃。现在就要跟城外的贼人比比,究竟是哪边更能熬了。

一个多月下来,旧时高高在上的衙内与下面的士兵厮混久了,苏子正已经什么仪态都不顾了,坐下来,跟士兵们吃着一样的食物。从常平仓中运出来谷子,一石能磨出八九斗。这样的米用来填饱肚子,如果是直接煮熟的话,会粗粝得难以下口,只有熬成粥才好食用。只要是口能填肚子的热饭菜,城上的守军心满意足了。

几口将碗里的米粥喝光,肚子还是有些饿,也不知能不能撑到明天。但他不好要第二碗,将碗丢在一边,回头看着仍冒着热气的铁锅,“不知又拆了几间屋子。”

邕州城中市民生火做饭,一向靠着城外的柴薪,又不像北方的城市会在冬天囤积炭火。城池一被围起来,没几天就都开始拆屋拆房,用城中的屋舍拆下来的木料来生火做饭。在战前鳞次栉比的街市民居,如今已经是东缺一块、西缺一块。而占地光大的寺院和园林,都是最开始就被拆下来的地方。

虽然苏子正很喜欢其中的一座园子,据说是请得北方的名匠打造,园中满是竹林,用竹子搭起了回廊、楼阁和小亭。夏日于园中休憩,听着风过竹林的沙沙声,苏子正和他来往的友人,作了不少首歪诗。但在战争面前,吟风赏月的诗情画意,全都被一双双翻山越岭的大脚踩进了烂泥里。

苏子正现在都觉得与市井小民聊天,也是很有意思的事。他的麾下大多数是刚刚被征发起来的平民,说着的也是市井中的事。

膀大腰圆的汉子姓柳,本是个市井中一个泼皮,欺行霸市的事情没少做。苏子正去城南宣化县衙的时候,倒是见过几次他被知县欧阳延让人拖下去用板子狠狠打着,只是他皮糙肉厚,与衙役们也有些交情,根本不在乎。本来这样的人,苏子正一直认为早早的发配出去才是好事,谁能想到他应募投了军,在城墙上杀贼最多的就是他。

正在帮忙收拾的十几岁的少年,两只眼睛亮亮的,叫李三四,本来是绸缎铺里的学徒,曾经打算用二十年的时间升到掌柜。看他待人处事的麻利劲儿,苏子正觉得他有个一半的时间就够了。

阴沉着脸靠在墙角的胡子花白的老汉,姓乔。他是永平寨人,有儿有女,连孙子都三个了。会来邕州是为了采办年货,哪里想到会碰上交趾贼军来袭。听说永平寨被攻破之后交趾人屠了城,就主动投军,要讨回个公道。

还有许多人,都各有各的经历,不过他们现在的心思都只有一个,要跟交趾贼军拼到底。

几个士兵将一束束火炬浸了灯油后点起,城头上亮起了一排星辰。天色晦暗,这个时候交趾人除了偷袭,正常已经不会再来攻城了。

“今天可是难得清闲。”在城下休息的副手宣化县尉周颜人未至,声先到。然后一个身着甲胄的文秀士子就踏着台阶上城来了,要与苏子正做着交接。

“这场雨下得好。”苏子正知道周颜的文秀只是外表而已,他在宣化县的两年里,可是几次三番的带队冲进蛮部的寨子,将作奸犯科的几个盗贼捉回来受审。“不过今天晚上说不定会有些麻烦。”

“衙内放心,周颜理会得。”周颜拱着手,“我们西壁这里靠着江水近,事少。倒是高钤辖管的南壁、薛都监管的东壁更要提防些。”

“他们都是老将了,这些日子也习惯了,不会不做提防。”

苏子正现在管着邕州城西壁。为了守城,苏缄将城中兵员分做了五部,东南西北各面城墙都放了一部兵马,而剩下一部,则是通判唐子正领着。如果哪一面城墙受到攻击,就可以及时过去救援。

邕州西面的这一段城墙都由苏子正分管,城上城下两千多人都听他的支配,不过这个辛苦活,在交趾贼军攻城的时候,连退都不能退,他父亲亲掌的督战队就在身后。不过也没有人会退却半步,都打到了这个份上,如何还有退却的余地。上上下下一条心,都要跟交趾耗到底,耗到他们支持不住要撤军。

刚下城头没几步,脑后就忽的传来了战鼓声。苏子正猛地跳了起来,转身就往城上冲去。冲到城墙边,望着战鼓隆隆的城外张望。

攻城的交趾军分作四处,每一处都相隔很远,投入的兵力为数不少。东西南三面四处同时发难,贼人这一次看来是势在必得。但苏子正越看越是惊讶,冲上来没有拿着长梯、攻城车那样的器械,甚至有人连刀枪都没有扛着。只是他们人人都带着个不小的包裹,有的抱在怀里,有的则是顶在头上。苏子正看着纳闷,一时没想通这是为了什么。

“不好,他们是要堆土上城。”周颜就在一边惊叫了起来。

一声霹雳在耳边炸响,苏子正脑中一晕。

“好贼子!”苏子正在南方长大,只随着父亲去过一次京城,但他还记得北方的冬天,就是如他现在感觉到的这么冷。

在城中领着预备队的唐子正收到了三面城墙传来的急报。他手上掌握着所有的神臂弓,每一处贼军来袭的方向,都催着求着他赶过去援救。

神臂弓如果不能集中使用,也只是劲道稍强、射速稍快的重弩而已。仅存六百张的神臂弓,该对哪一处集中使用?但唐子正根本没有余地多想,将手下的神臂弓手立刻分作四队,任何一处都不能留下空隙。

带队冲上西侧的城头,并无余暇与苏子正和周颜打个招呼,邕州通判就命手下的弩手们在城头列队,给神臂弓上弦。他要先解决应该是最容易处理的地方,然后去救援其他方向。

“射!”唐子正指挥着,两百多名神臂弓手同时扣下来牙发,可劲矢离弦的声音,完全不像前几天一般充满力道,而是软得像块炊饼。落在城下贼军的身上,也没有像过去那样箭到人倒。攻城的交趾贼军将土包顶在头上,猫着腰冲刺。

苏子正知道,弓弩的威力在雨天要大打折扣,而神臂弓用的筋角胶等畏水之物虽少,比普通的重弩更能承受湿气,但淋了雨之后,威力一样会大减。

“怎么办?!”苏子正急了。

“收弓、换刀,熄了灯火出城冲一番。”唐子正则大叫着,“不过一死而已!”

