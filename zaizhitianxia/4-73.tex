\section{第15章 焰上云霄思逐寇(四)}

“黄金满竟然投了宋人!”

申景贵的叫声中充满了惊讶和不可思议。黄金满行事一向沉稳,甚至在许多人眼里都到了胆怯的地步,怎么敢就这么毅然决然的跳到了宋人的那里去,照常理,好歹也要犹豫一阵子,再跟他们三人联系一下才对。

因为这条紧急军情,来自广源州的三位蛮帅皆聚集在刘纪的大帐中,讨论着接下来的应对,但起头的并不是对未来计划的商议,而是抱怨。

韦首安愤恨:“难怪李常杰那么大方,将邕州城让了这么多出来。”

“现在大半散在城中,真的给李常杰算计了。”刘纪脸上看不出来怒色,可心中同样是怒火冲天。

因为在邕州城中争夺劫掠的目标,广源州的蛮兵还跟交趾人起了好几起冲突。刘纪本来还准备着与李常杰和宗亶为此事扯皮的,没想到整件事根本就是一个陷阱。

“得快点将人从城里撤出来。”申景贵说了一句废话。

其实在听说归仁铺大败的第一时间,他们三人都已下令将散在城中的部众即刻召回,但邕州城这么大,下面的人都抢红了眼,能有多快可想而知。

“全撤出来的差不多还要有两天的时间。”韦首安板着脸,咬着牙,“宋人能给我们多少时间,李常杰又会给我们多少时间?”

“绝大部分一天之内就能回来了,剩下的就由他们自己去好了。”刘纪冷漠的说着,“汉人有句话:当断不断,反受其乱。这时候,一切都耽搁不得。”

“收回来后怎么做。”韦首安望着另外两人,“要撤军的话,李常杰和宗亶会不会点头?”

“要不然干脆……”申景贵没将话说下去。但刘纪和韦首安都明白他的意思。

韦首安正要说话,帐外的守卫进来通报,“洞主,大营那边派人来了。”

“李常杰他们要做什么?”申景贵立刻问道。

“绝不会有好事。”韦首安立刻道。

刘纪恩的一声点点头:“若是让我们去大营,必然有诈,谁都不要去。”

申景贵和韦首安立刻点头:“知道了。”已经吃了一个大亏,交趾人在他们的心目中再没有信誉可言。

从中军大营而来的信使,走进了帐中。

刘纪脸上堆起虚假的笑容:“不知李太尉、宗太尉有什么吩咐?”

“黄金满投敌,宋人援军已抵归仁铺。奉李、宗二太尉之命,着三位洞主撤军归国。”

刘纪、韦首安、申景贵三人都没想到李常杰会如此干脆,直接将摊了出来。

“李太尉要我们怎么撤?”

韦首安冷淡的问着。若是李常杰敢拿他当殿军,他转头就去投靠宋人,追着交趾兵打。

“李太尉与宗太尉商议已定,由他本人领军亲自殿后。而宗太尉则指挥向南撤军。宗太尉的军令,要三位洞主尽快从城中撤出来,渡过左江。”

既然并不是要招他们去大营,也就不需要砌词反对或是拖延。刘纪低头,领着韦、申二人一同接下军令。

信使走了。两对眼睛一起望向刘纪。

方才申景贵的提议还没有得出结论,而李常杰和宗亶又释放出了足够的善意。接下来该怎么做,到底该站在哪一边,现在韦首安和申景贵两人都疑惑,拿不定主意。

“……先回去再说。”刘纪没有犹豫太久,“回到广源州看宋人会怎么做。”

“可刘永……”申景贵欲言又止。

刘纪是广源州四位大首领中领头的一人。其他三位蛮帅虽不能说是对他马首是瞻,但刘纪说话的份量最重却是没有任何疑问的。而且刘纪的亲弟弟也在昆仑关中,黄金满既然投靠了宋人,那刘永当然没有可能例外。

要是这样的话,是不是该往宋人那里靠上一靠?申景贵和韦首安都在这么想着。

只是刘纪心存怀疑。

弟弟刘永到现在都没有消息。黄金满投宋,如果刘永也一起跟着投宋的话,少不得会立刻派人来联络自己。这消息,绝对会比宋人进兵的速度要快。但到现在也没有一点动静,尽管可以用各种意外来解释,但刘永的直觉上,已经觉得自己同胞兄弟已是凶多吉少。

而且这不仅仅是直觉的问题!

刘纪记得他前天最近一次收到刘永传回来的消息,上面说他已经进兵宾州,如今收获颇丰。他的那个弟弟一向贪得无厌,以刘纪对他的了解,不抢个盆满钵满,就绝不会打道回府。不可能只抢了三两天,便转回昆仑关。

从时间上算,刘永在劫掠的时候,正好会撞上来援救邕州的宋军……不论从什么角度来看,都可以说他的弟弟必无幸理。

另外黄金满的为人,刘纪也很清楚,绝不会因为被许了一点好处就轻易反叛。尽管之前的几年,因为宋人严禁缘边市易,他的部族损失极大——只比自己少上一点——但交趾派人来劝说一同出兵的时候,黄金满是最后一个才点头,而且听说是因为下面部众强烈要求才不得不同意,同时出兵的数量也是最少的。

黄金满既然有着这样的性子,绝不可能宋军一到,就乱了阵脚。必然是见到宋军的威势,并加上丰厚的回报作为补充,才会毅然决然的投到宋人门下,充当起走狗来。

至于宋军是拿什么来表现自己的实力,刘纪几乎已经可以确定了。

“就按照李常杰和宗亶说的去做,一起退回去!”刘纪已经拿定主意,作为统帅广源州诸多部族的大首领,他不会因为一个弟弟而将全族的性命放到悬崖上。但要说他会在形势还没有出现一个清晰明白的走向,就投向仇人,那也绝不可能,

而且即便他猜错了,弟弟刘永与黄金满一起投了宋人,那也没关系。有着这一层关系在,即便是退回了广源州,照样能与宋人联系上。何必在眼下的这个节骨眼上,与身边的交趾人起冲突,“这里可不是昆仑关。如果我们不从军令,有所异动,恐怕李常杰和宗亶第一个就是先对付我们。”

“先回去,一切等回去了再说。只要我们手上还有兵,不论是宋国还是交趾,都不能拿我们怎么样。”

……………………

收到了刘纪三人都听命行事的消息,李常杰和宗亶都稍稍放心了一点。但也只是一点而已。谁也不说不清他们现在的顺服到底是真是假。即便眼下是真,也会在形势转变的时候,跟着一起变动。

“关键还是要挡住宋人的兵锋。”宗亶对着李常杰说着,“只要能挡住宋人,一切都还好说。若是挡不住,刘纪他们可不会跟我们同生共死。”

“如果是眼下在归仁铺的那三千兵马,我还不至于会输,要赢也只是费点气力而已。就算是昆仑关的敌军都来了,我要退走也容易。”李常杰依然对自己有着足够的信心,“关键还是你这边要快,尽快聚拢城中兵马,渡过左江南归。等回到国中,宋人也奈何不了我们了。”

“这件事你放心,最多三天,我就能撤过江去。”

左江冬日水缓,当初数万人渡江,也没有花费宗亶太多的时间。这一次反过来,也同样不会有太多麻烦。

就算是面对来势汹汹的宋军精锐,李常杰和宗亶的心中,依然没有太多的惊惧。只要能维持好军中的稳定,拥有十倍以上的兵力,想要顺利撤退绝不是一件难度多大的一件事。

但两人都没有松懈下来,接下来的几天,就是最关键的一段时间。

……………………

一声声惨叫透过狭窄的缝隙传进小小的地窖中。

一名妇人紧紧抱着怀里的小女孩儿,缩在地窖中,一动也不动。一句句听不懂的交趾土话正从头上传下来,尖利的狂笑让人心惊胆战。交趾贼军就在外面杀人放火,着火后的烟雾从通气孔中透了进来。

烟气呛人,但妇人仍竭力忍着咳嗽的欲望,用湿润的布匹,捂住自己和怀里小主人的口鼻。

就在最后的时刻,服侍的主人和小主人也没有一个肯离开州衙,只是夫人把家里最小的七娘让她带了出来。

怀里的女孩儿动弹了一下,不知是不是因为姿势保持了太久,手脚麻痹了。

“七姐儿,不要动!”妇人连忙压低声音的责备着。

小女孩乖乖的在怀里缩了缩,静静的不再有任何动作。

前日从州衙中出来,她们就躲在深深的地窖里,匆匆带出来的食物还够吃上几天,但不知道头上的贼人,究竟还有多久才会离开。

时间一点点的流逝,从缝隙中投下来的光,明了又暗,外面已经很久没有动静了。

没有了用交趾土话发出的吼叫,也没有了死亡前的呻吟,静静的一点声音都没有。

到底是怎么回事?

是交趾贼军退了吗?还是城内的人都给他们杀光了。

妇人什么都不知道,但她决定还是多等上一点时间,等到真正撤退了再从地窖里出来。

