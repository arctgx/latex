\section{第十章 千秋邈矣变新腔(11)}

这一日的天气不是很好。

都快到午时了,天光依然十分黯淡。

一年来,宫中已经将诸多殿阁旧时糊纱的门窗,都换成了玻璃窗,但文德殿内依然得燃起一支支巨烛,才能保证殿内的光亮。

这些手臂粗细的巨烛,在制造时都掺入了少量的龙涎香和沉香屑,燃烧后,香气便在文德殿中缭绕。

韩冈抽了抽鼻子,他对这种通过燃烧产生的香气,总是感觉有些不舒服。尤其通风情况又不算太好,龙涎香的香气就在殿内缭绕不散。

过去龙涎香还没有这么多,宫中也很少大方到在文德殿这样的大殿中,使用龙涎香蜡烛。

不过随着交州的大开发,还有国内海上运输的发展,泉州和广州市易司不仅多了白糖、棉布这类新特产,旧日的特产,如丝绢、瓷器、茶叶的价格也有所降低。来自大食的海商数量,在短短几年内就翻了一倍还多。市易司上缴的利润相应增加,来自海外的商品自然也多了许多,其中就包括龙涎香。

此外,龙涎香等海外商品增加,也有玻璃制品、香水开始自产的缘故。在过去,玻璃和香水这两类商品,在大食商人能够卖给中国的奢侈品名录中,排在很靠前的位置,但是现在,原本一箱箱运来的货物,变成了一箱箱的运走。为了能够达成贸易,就必须增加其他商品的份额,龙涎香、象牙、宝石、香料,都被用来填补空位。

不过韩冈在他出版的笔记中,早揭开了龙涎香的真面目,不过是海中的巨鲸排泄出来的东西,勉强点说,可以类同于麝香。如果剖开大块的龙涎香,在里面能看见巨鲸吃下去的食物残骸。而在略早一期的《自然》中,也有人投稿,内容是研究来自西方的香料,在这篇论文中,也赞同了韩冈揭露的龙涎香真垩相。

随着龙涎香的真垩相被揭开,宫中内部对龙涎香的消耗立刻就变得少了许多。倒是臣子们,便能时常得到来自宫廷中的赏赐。崇政殿、内东门小殿等向太后常去的外殿,用龙涎沉香蜡烛的时候不断减少,而文德殿这样的大殿中,却不再吝惜使用贵重的龙涎香蜡烛。

女人总是有些洁癖。这点不足为奇。

但龙涎香的价格却没有因为被透露的底细有丝毫降低,依然价比黄金。这就像是麝香一样,麝香的出处知道的人不少——韩冈在书中提及龙涎香时,也顺便将麝香的来源也提了一下——但也没见麝香的价格降低过多少。

韩冈对龙涎香没有什么爱好,不过什么时候龙涎香的利益,高到让人们敢于出海捕捉鲸鱼,剖腹取香,那远洋海军的水手来源就不再成问题了。

在龙涎香气的笼罩中,一名内侍,又在陛前的白屏风添上了一笔。

这是大宋开国以来的第二次廷推。

在三年一次的进士科礼部试,和考官全数被逐的制科阁试之后,再一次成为朝堂关注的焦点。

经过了半个月,之前参与廷推的选举人中,有几位已经离开了京垩城,另有几位因为不同原因而不能参与,不过也有抵达京垩城的新人,今次参加廷推的总共二十五人。

选举人的面孔有了变化,报名参选的被选举人,也有了些许改变。

除去已经成为两府中人的韩冈,上一次落选的吕嘉问、曾孝宽和李定,无一例外皆参加了这一次推举。而另外一个,让很多人都没有想到,王居卿推荐了李肃之参加了推举。

事前除了极少数的知情者之外,没人能想到在最有可能参加选举的沈括选择了放弃之后,李肃之还会被推举上来。

李肃之曾经担任过三司使,有着等同于两制官的资格,尽管他去担任了知贡举,不能参加廷推,可作为被选举人,如果不打算投自己票的话,完全可以参加进来。

依照尚未成文的推举条例,在被推荐出来的候选人中,推举出三人,供太后做出最后的选择。如果仅有三人参加廷推,廷推根本就毫无意义,不免受到非议。

韩冈自然不能容忍这样的事发生,也许日后会出现这样的情况,但至少在现在,这一件新生事物必须得到保护,一直到成为可以延续下去的惯例。所以韩冈在李肃之进入贡院之前,就已经与其联络过了,而王居卿那边,就更是简单,几句话的事。

不过李肃之成为候选人,并没有改变韩党选票不足的现状。远远落后于吕嘉问、曾孝宽和李定三人。即便吕嘉问、曾孝宽和李定三人都选择了弃权,而不像前一次选举,约定好后互相投票,也没有让李肃之的选票追近多少。

这第二次推举,也便由于报名者的缘故,显得波澜不惊。

这一回,曾孝宽的票数最高,吕嘉问、李定两人并列,与曾孝宽的差距也仅止于一票,而李肃之居于末尾。

李肃之因此被淘汰,票数最前的三人进入最后的阶段。

对于这样的结果,在廷推开始之前,很多人就已经预料到了。曾、吕、李这三人之中,究竟谁会成为太后挑选出来的幸运儿,才是人们猜测的目标。

向太后没有考虑太多,很简单的便选择了票数最多的曾孝宽担任枢密副使。

听到了这一结果,吕嘉问的脸色立刻变得精彩起来。因为不想太过显眼,而没有去争取选票,否则依照前一次的情况,他必然是第一。如果这一次,太后是以票数的高低来决定,那自己输得真是太冤了。

不过没人在意吕嘉问到底在想什么,再一次,知制诰就在文德殿上开始起草拜除曾孝宽为枢密副使的诏书。

章敦若有所思。

御内东门小殿、学士院锁院的旧日惯例,似乎已经成为了过去。

在连续两次推举之后,两府之中的新人,估计都会通过这样的途径成为宰辅中的一员。

没有惯例的辞让,曾孝宽在受命之后,立刻拜领了诏书——真要不想就任,在选举之前就可以推辞掉了,这时候再推辞,除了恶心人和自己成为世人的笑柄,没有任何意义。

这又是被改变了的旧日惯例。

不过新的惯例,不仅仅这一点。还有惊喜等待着殿中的所有朝臣。

在曾孝宽成为枢密副使之后,宰执班中又迎来了新的变动。

太后的声音在屏风后响起:“苏颂可知枢密院事。”

比不上前一次,将韩冈从西府调去东府那样的突兀,不过苏颂的晋升,也的确没有半点先兆。

没有枢密使的时候,知枢密院事便是枢密院的主官。但有枢密使的情况下,知枢密院事要低上半级。

旧时,枢密院一把手、二把手的头衔,要么是枢密使、枢密副使,要么是知枢密院事、同知枢密院事,基本上不会混搭,在过去也只有过一次枢密使和知枢密院事同时在院的情况。不过这两年,随着吕惠卿任枢密使,章敦知枢密院事,反倒像是成了惯例。

虽说在去年的时候,吕惠卿因故改任宣徽使,章敦便从知枢密院事晋升为高半级的枢密使,加上担任枢密副使的苏颂和薛向,使得枢密院中的头衔,变得纯粹了一点。不过现在,苏颂就任知枢密院事,枢密院中,又重新回到了两套头衔并行的时代了。

看起来,太后是打算让苏颂与新晋枢密副使的曾孝宽拉开差距。不过连续两次廷推之后,先是韩冈,接着又是韩冈一系的苏颂,都得到了擢升,一个可能是韩冈得太后信重,另一个,也就是太后对新党有看法,希望借助韩冈来压制势力遍及朝堂的新党。

这两件事,其实也是一而二,二而一。

新党势力虽盛,但死心塌地的成员并不算多,许多成员不过是趋炎附势,看见新党势大而来投,如果太后表明态度,要支持韩冈,驱逐新党,那新党的势力,立刻就能缩减一半。

韩冈、苏颂两人分据两府,同时将黄裳黜落的四位阁试考官都被请出了京垩城,太后打垩压新党的苗头越发的明显起来。

不过现在王安石还在朝堂上,以平章军国重事而镇垩压文武两班。以他在之前平叛一事中的功劳,太后就算再不念旧情,也不可能将他给逼出京去。

只要王安石还在,新党就不会倒。四位考官出京仅仅是小挫,从情理上说,三馆秘阁的官员参与修书,是他们的分内之事。调任西京,助司马光修《资治通鉴》,相对于原来在皇城中的工作地点,的确是贬斥,但其本职并未更动,这样看来,更多的还是双方妥协了的结果。

此外落榜的黄裳,也有说法是他将会去边地就任,可能是去河东或是河北,不能通过制科,就打算用实际的功劳来体现自己的能力。

当事双方都离京出外,怎么看都不是韩冈大获全胜的结果。否则就应该是黄裳被特旨提入制科御试,而蹇周辅等人就出京监税去了。

不论是太后还想维持平衡,还是韩冈无意跟自家的岳父彻底拉开阵仗,新党的势力依然稳固。

只要这样的局面还能维持,愿意冒险的人,永远都不会是大多数。

韩冈也不打算在近期内改变现状,他还没有足够的胃口吞下新党倒下之后留下来的蛋糕,要是被外面那群虎视眈眈的老家伙抢去,就太让人糟心了。

结束了廷议,从文德殿中出来之后,先恭喜了苏颂,顺便又连带恭喜了曾孝宽,在往崇政殿的路上,韩绛凑了过来,

“玉昆,你拟定的那几条策问,老夫之前看了,怎么觉得不对啊。”
