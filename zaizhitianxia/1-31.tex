\section{第15章 三箭出奇绝后患(中)}

韩冈并不知道这个时候秦州和甘谷都有人意图杀他而后快,即便知道也无力去顾及,因为他眼前,就有一群人手拿刀剑想要他的性命去。

“数……数目好多!”一名年轻的民伕被吓得结结巴巴。而他能说出话来,已经算是好的,其他的民伕都是瞠目结舌,面如土色,直如雷惊的蛤蟆,连句话也说不出。他们都跟韩冈一样,随身带着弓箭,但此时贼寇来袭,却都忘了将长弓举起。

“‘树木’多了又如何?树多了就砍!树少了就栽!”王舜臣悠悠然开着玩笑。长弓提于手中,下马独自上前。

前行二十步,王舜臣双脚一前一后站定,以弓挂臂,大喝道:“只是爷爷不会栽树砍树,只会插花!”

韩冈终于知道了,王舜臣的自信从何而来,也知道了王舜臣为什么没有要他人一起上前。韩冈从来没想过,一个人、一张弓,竟然能射出一瀑箭雨!

在山林间冲出来的蕃贼接近五十人,冲在最前面七人看起来最为精悍。王舜臣的目标正是他们。

开弓搭箭,箭矢离弦。

第一支箭,射入第一个贼人的左眼,第二支箭,在第二名贼人的脸上开出一朵血花,第三支箭穿喉而过,第四支箭,则将第四人的心口洞穿,而此时第一个贼人才刚刚栽倒在地。其后三人见状,反身就逃。王舜臣又是连珠三箭,直贯其背,将他们一一射倒。

套在拇指上的铜扳指前后闪动,小指粗细的丝麻弓弦幻成一抹虚影。长箭破空的尖啸连绵不绝。弦声鸣动,演奏出阵阵杀伐之音。万人敌那是虚言夸大,但一人敌百,王舜臣却做得如吃饭喝水般轻松自在。

王舜臣所用的长弓并非强弓,力道也许只有一石二三,尽管禁军中的上四军招收士兵的最低标准是开九斗弓、两石七斗的弩,但武将用弓不到一石五斗力,射不穿敌军的铠甲,出门都没脸对人说。可王舜臣掌中的那张一石出头的战弓,也许射不穿党项人身上的精铁瘊子甲,但精准异常的落点,让长箭的箭头完全不需要与坚实的甲叶对抗。

哀鸣声遍地响起,箭落处非死即伤。一支支白羽箭在蕃贼身上轻轻摇晃,正如被插上了一朵朵随风起伏的白色鸢尾花。

好一个插花!

王舜臣一人一弓就将蕃贼射得不能前进一步,可他毕竟只有一人,贼人的反击随之而来。只听得后方一名蕃贼大喝了几声,十几名蕃贼同时立住阵脚,向王舜臣射出利箭。十余支长箭齐齐攒射而来,逼着王舜臣横着退到了路边一颗树后,肩膀上还中了一箭。

躲在树后,听着身前的树木被射得噗噗作响,看着在肩膀上晃动的箭矢,王舜臣痛得龇牙咧嘴,暗悔没有穿着盔甲出来。若是有盔甲在身,他就可以硬抗一下贼人的弓箭,多射死几个,定能让贼人彻底丧失战意,可现在却是他被蕃贼压制得探不出头来。

“日他鸟的!”王舜臣恨得直磨牙,“这么多战功啊……”

……………………

王舜臣战局不利,民伕们开始慌乱起来。见势不妙,韩冈挥手指前,对着薛廿八和董超道:“独木难支,你二人速去相助军将!否则我等今日皆是难逃一死!”

不出意料的,韩冈在薛廿八和董超脸上看到了浓浓的嘲笑。董超摸着脸上被王舜臣鞭出的伤痕,狞笑道:“韩秀才,贼人势大,趁王军将堵着贼人,我们还是先逃罢!”

他的声音透着得意,而韩冈的回答更是干脆。双眉一轩,双手一抬,便嗖的一箭射出。射自五步外的出其不意的一箭,董超根本连反应的时间也没有,腹部刹那间便被长箭贯穿。

“乱我军心者死!”韩冈一声大喝,伴着董超的惨叫同时响起。

民伕们目瞪口呆,薛廿八也是目瞪口呆,“你……”

韩冈再无二话,又拉开了手中长弓。内部火并总是先下手为强,他只占了个‘奇’字,本身并不是薛廿八和董超中任何一人的对手。第二箭闪电般射出,穿透了薛廿八并不粗壮的颈项,带血的箭头出现在他的脖颈后,薛廿八顿时捂着喉间翻倒在地。

他这时方才知道,为什么刘三三个人去杀这位痨病秀才,却一个也没能活:

‘这措大下手好快!’这是薛廿八在这世上的最后一个念头。

“乱我军心者死!!”

韩冈再次厉声大喝,有薛廿八的性命为韩冈的命令做证,民伕们不敢再有妄动。可董超却在这时候忍着腹内的剧痛爬起,面容扭曲着拔出腰刀,死命向韩冈一刀劈来。

韩冈慌忙侧身,有些狼狈的让过呼啸而来的刀锋,但他的右手顺利的抽出又一支箭搭在弓弦上,第三次拉开战弓。弓弦震荡,长箭电闪,直奔董超而去。可这一箭没能让韩冈如愿以偿,董超适时的挥动弯刀,将箭矢用力格开。

临死前的反扑最为恐怖,董超怒吼一声,如风一般猛冲了过来,韩冈再没时间从身后抽箭,丢下战弓,反冲上去,一手架住董超持刀的右腕,另一只手攥住插在他肚皮上的箭杆,不顾董超的左手已经扼住了自己的脖子,用尽力气狠命的一搅。

与董超面对着面,只隔着半尺不到,彼此呼吸可闻。韩冈清楚看见陈举的这名手下瞳孔放大,眼神渐渐涣散,而紧扣在脖子上的手掌也渐次松开。浑身的气力都随着体内传来的剧痛消失,董超最终软软的瘫倒在地上。

一场火并如兔起鹘落,转眼间便是分出了结果。韩冈从地上捡起董超的腰刀,又戳了两人要害几刀,确认了他们的死信,才一脚踩住尸体,血淋淋的刀尖下指,寒声道:“谁再敢不听号令,他们就是榜样!”

三十七名民伕无人敢直视韩冈,低下头去,老实听命。

韩冈松了一口气。这是个机会,他很清楚两人的身份,以及他们跟着一起向甘谷城运辎重的用意。以陈举的老道,不会只有一套计划,半路劫杀是一个方案,恐怕到了甘谷城还有人来对付他韩冈。

但已经死了黄大瘤和刘三,现在薛廿八和董超又被自己所杀。如果再加上鼓动蕃人部族劫道的行动又告失败,陈举他的那个小集团,还能保持多少向心力,那实在是个问题。就算甘谷城还有点麻烦——费了一番气力去搜集情报的韩冈也清楚究竟是谁会来找麻烦——但兵来将挡,水来土掩,自己有的是手段去应对。

内部一安,韩冈便把注意力放回到前方。王舜臣还在与蕃贼对峙,韩冈这里发生的一切,他根本没有发现。蕃贼畏惧王舜臣的神箭,不敢冲得过快。但还是有十几个人在射箭压制王舜臣,剩下的七八人在箭雨的掩护下开始向王舜臣靠近。

局势不妙!

“把车横过来!快点横过来!”韩冈急促下令道。“快把来路堵上!再把靠山的这边堵上!”

民伕们都有些茫然不解,也不愿自断退路,但韩冈刚刚杀了两人,威势正盛,谁也不敢出头反对。听着韩冈的话,慌慌张张地将一辆辆骡车并排着堵死了后方的道路,同时又把靠山的一面堵上,不敢有丝毫拖沓。

韩冈不停的催促着,指挥民伕将他们所在的这段道路围成一座车阵。

蕃人虽然不比汉人聪慧,但奸猾狡诈并不或缺。劫杀军需辎重,这样的罪名,秦州的任何一个蕃落都承担不起。再怎么想,韩冈他们一行人都是必须被灭口的,只要逃出一个,便有可能给整个部族带来灭顶之灾。

但如果能顺利将韩冈他们全数歼灭,在得到足以让部族过个肥年的物资的同时,还可以顺便布置布置,陷害一下敌对的部族——秦州的蕃部绝不团结,尤其是比邻而居的部族,往往由于水源、田地、牧场的归属而争斗不已——如果真的如自己所料,那身后必然还有贼人埋伏在退路上,等待他们逃跑时动手,因为这样才能保证全歼而不让一个活口逃出。

就像赶着验证韩冈的猜测,刚刚有了雏型的车阵尚在调整中,韩冈等人的身后来路处,还有身侧的山坡上,同时响起了喊杀声。

埋伏在韩冈后方的蕃人,本是想着趁辎重队与拦路的分队厮杀正酣时,再攻出来前后夹击。联络他们的汉人说过,辎重队中早早就安排了两名内应。能让他们不费吹灰之力便夺财灭口,所以他们一直在等着内应发出信号。

可远远的看着辎重队中只乱了眨眼的功夫,就恢复了平静,而且还有开始准备组成车阵的迹象,没有其他的选择,他们便不得不提前杀奔出来。

“不用惊慌!”韩冈胸有成竹的对民伕们喊道,“贼人只是虚张声势,人数绝对不会多!否则他们就应该与前面的贼人一起冲出来,而不是躲在后面等我们的破绽!我们就在车阵里,他们一时半会儿攻不进来!”

韩冈仅仅是在信口胡诌,对于蕃人的计划,他并没有多少认识。不过他带的民伕都是关西汉子,许多都是被征发起来上过战场的,手背和脸上刺了字占了三分之一还多,射术没一个会输人。只要他们能冷静下来,击败只有自己一两倍数目的蕃贼,简直是轻而易举。而他们现在需要的也不是事实,而是领导者毫不动摇的信心,以及准确有效的命令。

这一切,韩冈都能给他们:“拿起你们的弓,把箭给我搭上!听着我的口令!……射!”

ps:先下手为强,后下手遭殃。关键是要快啊。

今天第一更,征集红票,收藏。

