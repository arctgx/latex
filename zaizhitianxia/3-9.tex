\section{第四章 秋来暮色寒(上)}

秦州本州的解试方才开始,锁厅试的成绩就已经张榜而出。

策问和经义皆是第一,韩冈的名次就在蔡曚的自作聪明之下,首冠鳌山。转运司八字墙的贴榜处,韩冈的大名高挂于上,而慕容武的名字陪敬末座。

“恭喜玉昆。”

“同喜同喜。”

韩冈和慕容武互致一礼。韩冈一直充满自信,但慕容武就一直有些心神不安,尤其是听说韩冈如何做答之后,更是如此。看到他的样子,韩冈也难对他抱有信心。但现在,同门的两人同时上榜,都是喜出望外。

在旁看榜的众人中,有人黯然而去,也有人喜笑颜开。锁厅试参考的人数虽少,但发榜后考生们的喜怒哀乐,却也是如寻常的贡举一般。

“不知思文兄接下来的行止如何?”

“愚兄要先回乡里,后日就出发……再读上一个月的书,等到了秋后就上京。”

“……先生的书院可就在思文兄的家乡附近!”

慕容武点头:“自当要拜访一下先生。……玉昆你呢,”他问道,“先回陇西吗?”

“肯定是要先回去一趟,也是后天就走”韩冈道:“小弟表兄就在秦州为将,人还在笼竿城中。本想着考完后见他一见,没想到已经上京去了。”

慕容武知道韩冈的表兄是谁。当年他帮着处理过了韩冈表弟冯从义的家产一案是,曾与李信父子打过照面。当时慕容武并不觉得李信有何特别,只是身手很好而已。但在德顺军笼竿城一役之后,他可不会这么说了。

“在笼竿城下七矛杀七将的李巡检,玉昆你这个表兄可是不简单!”

“当然!小弟母舅家几代嫡传的掷矛之术,本就是军中一绝。”韩冈笑着拉起慕容武,“家表兄既然已经被调入京中,这事就不说了。思文兄,今日我俩还是先去晚晴楼庆贺一下,等明日一起去衙中拜见蔡转运和蔡运判。”

慕容武与韩冈并肩走了。就在他们的背后,一位老迈的行商盯着他们逐渐远去的背影,阴冷的眼神绝非商人所有。

“东家……”一声似是在提醒的低喝,让行商一惊。

他回过头来,收回了凶戾的目光,又变成了一个敦厚老实的行脚商人模样。对着身后神色木讷的伙计,行商道:“生意都做完了,我们还是先回去吧!”

十日后。

兴庆府紫宸殿中,梁氏兄妹还有几位西夏国的重臣在列,梁乙埋的儿子梁乙逋汇报着来自于外派密探的情报。

“……河州一战消耗甚大。据细作回报,秦州的常平仓如今只剩正常年景的三成。从此看来,宋人一两年内不会在秦凤路动手……”

“这个谁不知道?!”宗室大将嵬名阿吴花白的双眉挑起了一个不耐烦的角度,打断梁乙逋的废话,“说些新东西!”

嵬名阿吴是元昊的侄儿,曾跟着他的父亲浪遇,与元昊一起打天下。梁氏兄妹上台后,撤掉了浪遇都统军的位置,连带着阿吴一起受到压制。但不久之前,也就是仁多零丁统军南侵时,他被拖出来坐镇朝中,担任统军一职,甚至有了郡王之封。

但阿吴的身份究竟不如梁乙逋身为一国宰相的父亲,不但打断说话,而且一点面子都没留下,梁乙逋心中顿时大怒。但他的父亲立刻咳嗽了一下,让他藏起怒火,顺服的换了个最新的情报汇报。

“景思立日前在熙河路战死,德顺军诸寨堡又在仁多统制的攻击下残破不堪。为了重振德顺军军势,刘昌祚可能要调去坐镇笼竿城……不知这一事,大王知不知道?”

“甘谷城呢,谁来接手?”

梁乙逋摇头:“这个还没查到。”

“好本事!”嵬名阿吴冷哼一声,不说话了,梁乙逋的脸色也就此全都黑掉。

“还有没有其他的消息?”仁多零丁似是缓和殿中气氛。

“……倒是有件闲事。就是韩冈,丢下了熙河路的差事,去秦州考中一个贡生。为了明年的进士考试,他年底之前就要进京。如果韩冈中了进士,那么熙河路最高位的几人,除了苗授以外,多半是要全都换人。”

“高遵裕不是还留在熙河?他怎么会走?”梁太后开口问道。

“高遵裕肯定要走。”汉臣景询在下回答,“他是与王韶一起建立了熙河路的功臣,只是由于武将和外戚的身份不便担任熙河经略。但只要他留在熙河,如果有新的经略使去任职,必然会给他架空掉。要是来的是个强硬一点的贵官,那他与高遵裕肯定拼斗起来。为了保证熙河路的安定,高遵裕很快就会被调走,而苗授会接手他的位置。”

“只要王韶不回来,那就可以高枕无忧。”梁乙逋笑说着。

仁多零丁声音却冷了下来:“若是只看着王韶,日后不知要吃多少苦头。能收复河湟蕃部,不仅仅靠着王韶一人。那里有高遵裕,有苗授,还有刚刚说的要去东京开封考进士的韩冈!”

景询附和着点头:“韩冈的确不简单,才二十出头,就已经是朝官了。如若他今次中了进士,肯定是飞黄腾达。到了十年后,说不定就另一个韩琦!”

“韩琦?……”仁多零丁瞥了一下嘴,“若是韩琦倒还好了。”

西夏君臣从来都没看得起曾经宣抚陕西,靠着在此地积累的军功,年纪轻轻就成为东朝宰执的韩稚圭。他挑选的任福,给刚刚称帝的景宗【李元昊】送了一份大礼;他主持的进攻战略,让铁鹞子得以横行关西。连个修补匠都做不好,还得范仲淹为他擦屁股,这就是韩琦。‘夏竦何曾耸,韩琦未足奇。’太师张元题在边界庙中的这首诗,可不仅仅是发泄殿试被黜落的怨气。

“王韶、高遵裕南下追击木征,是韩冈一人撑起了熙河大局。而且他年纪轻轻,用兵却稳当得很。熙州、河州,几次大战,王韶都是留了他来镇守后路,自己领兵在前冲杀。韩冈从来没有过一点疏忽,功劳立得比谁都多。临洮堡一战,他率援军而至,不入城而在城外结寨,这一手,正是没能攻下临洮堡的原因。”仁多零丁不论是从自家的侄儿那里,还是通过别的途径收听到的情报,都能确定韩冈的危险性,“他若是再有几年历练,国中想找出一个能压着他的,可就难了……”

“那就杀了他!趁他人还在陕西,找几个心细胆大的细作去刺杀他好了。”

“有用吗?……成功与否且不论,宋人那里岂止一个韩冈?!”

仁多零丁几乎要为梁乙逋的糊涂大骂出口,“王舜臣这个名字有没有听说?箭术堪比刘昌祚,领军时更是勇猛无匹,是王韶帐下第一得用的陷阵猛将。种朴这个名字有没有听说?在罗兀城设伏杀了嵬名济的便是。李信有没有听说过?笼竿城下,他七支飞矛连杀七个铁鹞子的正副指挥使,直冲进笼竿城,我都没能拦住他!这些宋将,可都不到三十岁啊!”

而且王韶才四十出头,乃是正当年的岁数。同样四十上下的出色将领,在东朝的关西军中,一抓一把。

一想及此事,仁多零丁就如赤身卧在冰雪中,寒气直逼骨髓。他在紫宸殿上摇着头,怒声说着:“这不是刺杀一两人,就能扭转过来的局势!大势已变,不再是二十年前的局面。再不设法扭转,那就是大白高国的灭顶之灾!”

东朝年轻一辈英才频出,无论文官武将,都不是几十年前东朝仁宗时,满朝文武、无一堪用时的惨状可比。韩冈、王舜臣、种朴、李信

而当仁多零丁反观大夏,却没有几个趁得上手。

自家的侄子就算了,没能攻下临洮堡不怪他,可不在已经确认无法击破宋人援军的情况下抽身撤退,那就是他的问题了。

禹臧花麻算是不错。但他的不错,也仅仅是在宋人手上没吃大亏而已。禹臧花麻几次出手,还没在宋人那里占到半点便宜,只除了一个景思立。

至于其余,梁乙逋都可以算是不错的了。要跟东朝那边相比起来,实在是让人痛心疾首。

宋夏两国国力天差地远,原来能立国,就是因为宋人的战斗力不堪一击的缘故。但现在,在人才上,差距也越来越大。宋人的皇帝虽然因为年轻,毛躁了一点。可若是他做了十年天子后,掌控朝局,稳定内事的本事肯定会大涨。到那时,可就是大夏的祸事了。

“兀卒年岁已长,也到该婚配的年纪。”仁多零丁提起并不在场的西夏国君。秉常今年十四岁,虽说是早了点,但这个年纪娶亲的绝不算少。

“不知枢密家的孙女中,有哪个有心侍奉天子?”

老将摇头,对梁乙逋的试探回以一丝嘲笑:“是大辽!”

他一扫被他的提议镇住的殿中众人,森然说道:“必须要与大辽联姻。”

