\section{第42章 壮心全向笔端含(上)}

新修的方成轨道已经正式运行超过十天了。

这十天来,短短的六十里道路,已经成了京西——不,也许是全国——最繁忙的陆上通道。每隔一刻钟到两刻钟,就有满载着数万斤纲粮的有轨马车从山阳港出发,沿着轨道向北行去。

不论白天和黑夜,都能看到一列列马车从港口出发,与从北面回来的马车交错而过。

为了能在夜中也保证车辆的安全,车前车后都点起了油灯,而押车的士兵也是每隔一段路程,就吹向联络前后的号角。经过训练和实习的挽马习惯了暗夜中的旅行,没有因为看不清脚下的道路,而迟缓了脚步。

方城轨道上的运输工作,就这样日以继夜的运转着,全天一百刻钟,发出去的粮车的总数,竟多达五十余列。

六十里长的轨道,是维系大宋京畿粮食安全的大动脉,不过这条新造的动脉在进入港镇前,在南北两端都分出了一条支线。

北面山阴港的支线,通向一个不大的维修厂。而南面的支线,重点则是一片面积有十余顷的工场——这是一开始打造有轨马车的工坊。

如今打造马车的任务犹在,不过工坊中的匠师们更多的精力是放在对马车的维修和整备上。就是到了夜中,这一家工坊也跟发车的管事房一样灯火通明。被确定有所损伤、不能继续上路的车厢,都会拉倒这家工坊中进行彻底的维修。

夜间的工坊,人声鼎沸,有工匠喊着的号子声,有打理木料的锯刨声,也有捶打铁件的敲击声,这么热闹的场所,很容易就让人忘记了外面还是更深漏尽的子夜。

“为了这一次的纲运,总共打造了六百零七节车厢。但十天下来,已经确定毁损、无法修补的有十七节,”李诫指着厂房角落中的一堆零碎,“拆下来的零件,能用的都放进了仓库,作为以后替换的备件。不能用的都在这里。”

方兴点点头,他深夜造访这间车辆工坊,就是为了查看一下在李诫的主持下,工坊夜间运作的情况。

瞥了那堆垃圾一眼之后了,方兴又问道:“那修复的有多少?”

“换个零件就能修好的,除了现在正在修的这两节,都已经停到库中去了,排队等着轮换。山阴那里有十一节,山阳这边则是正好三十节。”李诫如数家珍。

“也就是才坏了十七节?看起来情况还不错嘛。”方兴轻松的笑道。看着工匠们为着两辆已经拆得只剩架子、还被翻了过来的两节车厢,一幅兴致勃勃的模样。

“主要是车夫人选得的确不错。”李诫说道,“他们说哪里有问题,拖到工场中一看,当真就是哪里有问题。没有等着坏到不可收拾的地步。”

在方城轨道上领差事的近百名车夫,都是方兴奉韩冈的命找来的。李诫说他们称职,方兴当然觉得有自己的一份功劳:“这一次找来的车夫都是军中的老把式,在驿馆中驾车赶马多少年了,随便挑出一个,都能自个儿给车子换轮子、轮轴。虽说如今不会有时间让他们自己动手,不过在上货卸货的时候,查看一下车子是否有伤,我那边没听说这些天他们出过纰漏。”

拉着李诫从嘈杂的厂房里出来,走在月色笼罩下的工坊中。

“十天了……这十天,运出去的纲粮,已经有十九万石——方才愚兄过来时还差一点,现在应该到了——以这个速度,一个月再多上两天,六十万石纲粮就能全数通过方城轨道,”方兴长叹了一口气,把疲劳都吐了出来,只留下了自信的微笑。“等到这些粮食从山阴港运出去,愚兄这边的差事也可以算是交代了。”

这些天来,方兴他至少轻了十斤以上,腰带和衣服都变得宽松了,脸颊也变得比一个月前更加瘦长。但在工作顺利、全功在即的时候,之前的付出也算是有了回报。

不过李诫没有感染上方兴的信心,韩冈将他破格提拔,先让他作为副手参与道路和渠道修筑,等到他上手之后,就把轨道修筑的监理权交给他负责,同时还包括了马车工坊以及港口码头的监察权。

得到了韩冈的重用,李诫感念知遇之恩,在差事上下足了功夫。不仅将手上的大小事务都捉摸了个透,甚至为了盯着工程的进度,两天里面就有一天吃住在工地上——另一天则是在港镇上或工坊中度过。

他眉头紧锁:“这些天来,发出去的车一例都是重载。对车辆和路轨的损耗,都会在后半段体现出来。”

“车厢不是排队轮换吗。比实际需要多打造了两倍三倍的车厢,不就是为了能保证后半段不出问题。”

“路轨呢?”李诫反问了一句。“听徐州过来的匠人说,方城轨道上的路轨,比起矿山里面,损耗的还要快。”

“坏了就换。”方兴毫不在意的说着,“替换的备件都是齐的。”

“路轨只会在马车压上去时才会坏,一旦坏了,就会连累到上面的车子。”李诫咬了咬下嘴唇,“光是损耗在路上的纲粮就为数不少。完全损毁的十七节,上面的纲粮都落地了,而已经修好的四十一节车厢,也有一半是倾覆,还死了两个人啊!这还只是方城轨道,六十里而已。两头的漕渠,还有一千里!”

“汴河上的纲运损耗是多少?”方兴停住了脚,眯起的眼神如刀,似是要将李诫的真心剖开来看一看,“在薛直学任职六路发运司之前,风浪、鼠雀、浸渍之类的损耗,基本上都是在一成左右,六十万石——正好是我们这一次运送的纲粮数目。等到了薛直学上任之后,将民船官船杂合编组,就降到了百分之二三。看着虽少,其实也有十多万石了。我们这里可能比得上?!”

李诫皱着眉,嘴唇动了动,欲言又止。看到两人针锋相对的样子,两人的随从都立刻躲得远远的。

方兴看着李诫的样子就缓和了下来,“当然,襄汉漕运的路程只有汴河的一半,若有个百分之二、百分之三的损耗,也是多了。从襄阳运来方城的这一路上,我千叮咛万嘱咐,派了多少人盯着,还是翻了一艘船。北面还不知会怎么样。现在计较起来,路上损耗的比例不会比汴水少。”他冲着李诫笑了一笑,“倒是落在这轨道上的,却比落到水里的好多。坏了那么多车厢,里面的粮食也有几千石了。不过绝大多数都收回了,包括粮食和车子。要不然你这里哪有这么多车子好修?”

轨道边上就是旧时的官道,坏掉的车子,以及洒落的粮食,全都堆在轨道边,都派有专人从官道上拖回去。由于道路很短,派出去维护轨道的人手又足,沿途的乡民都还没能做到靠山吃山靠水吃水的地步。

方兴抬头看着深秋的星空。正是月初的时候,上弦月只有弯弯一钩,越发的显得天空高远,星光璀璨。

如今昼暖夜寒,呼吸时已经有了白白的雾气,方兴长吁了一口气,一团白雾在空气中飘散,“今年还算好,从漕司到州县,上上下下都盯着,哪一个皮不是绷紧的?可等到了明年,没有今年的这般严厉的约束,什么鬼鬼祟祟的东西都会冒出来了。”他幸灾乐祸的轻笑了一声,“不过那就轮到襄汉发运司的头疼,不干我们的事了。”

李诫皱眉:“不是说如果发运司当真成立,龙图已经事先定下发运判官一职吗?”

“等发运司确定成立了再说吧。”方兴冷笑,瞥了李诫一眼,“你还没发现吗?龙图如今对襄汉漕运已经看得很淡了,并不是很放在心上。”

李诫身子一震,视线就投了过来,瞪大的双眼在追询方兴说出这句话的理由和证据。

方兴却又抬眼看起了天上的繁星,过了半晌方才说道:“换做是你,方城轨道正式通车,会不会缺席?”

“不是说到了成功后再……究竟是怎么回事?!”李诫的声音惊急,“龙图难道要放弃襄汉漕运!?”

“胡说什么?都这地步了,京西整整一年的税赋都砸在了里面,怎么可能说放弃就放弃?我只是说龙图看得淡了。”

“为什么?”李诫像是恢复了冷静,沉声问道。

方兴摇摇头,似是无奈摊开手:“龙图的心思不是我们能猜测的,也许他有更重要的事。”

他转过头。李诫身量不高,方兴平视过来,正好可以看到他头上的软脚幞头。略垂下视线,是李诫严肃沉思而板起的一张脸。

方兴呵呵笑了一声,“不要想太多了,以龙图地位,眼中是朝堂、天下。襄汉漕运对我等来说,是身登青云的捷径,也是日后倚之为本的依仗,升官发财全靠它了。但对龙图来说,不过是个造势的工具而已,既然几乎可以确定能够成功,当然就不会太过挂心。”

