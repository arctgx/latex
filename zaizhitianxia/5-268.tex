\section{第29章 浮生迫岁期行旅(八)}

赠了谥号,又加了张载一个从三品的文散官,又有赏赐,可谓是备极哀荣。

安焘领旨,韩冈当即拜倒下来,向皇后恭声致谢。

李清臣有些得意,王安石虽是面色不愉,可最后也只能叹一口气。这样的封赠,张载也的确当得起。道统不能让,但人心是欺瞒不得的。

张载的谥号就这么定了下来,日后世人也就能尊称他一声文诚先生,或是简单一点的张文诚了。

之前千里镜的禁令,改成了总长度一尺以下,同时镜片径圆最大处在寸半以下的千里镜才属于禁令之内,也就说便携式的千里镜依然是军器,但更大的能用于天文观测的则不再禁止。

至于《自然》期刊,年后就会刊行。朝廷还特地拨了款,而且还可以使用国子监的印书坊——国子监版的书籍精美冠绝天下,说起来只有一些私人刻印的图书可以在质量上与其相比,比起杭州版、福建版要强得多。

韩冈的三封奏章中最后一条也如愿以偿,当他回到太常寺中对苏颂一说张载得谥,苏颂便立刻向他拱手贺喜。

“对了。”苏颂道过喜后,坐下来问韩冈,“今天愚兄听说河南的嵩阳书院那边出事了,殿上最后怎么议的?”

“那件事啊……政事堂好像没有报上去。”韩冈摇摇头,他估计苏颂多半是听到消息后就急了半日,毕竟他立场偏近旧党,嵩阳书院里面有不少人与他关系匪浅,“也是那一帮子学生年轻气盛,又没见识,所以糊里糊涂就上了当。现在一部分人准备上书,另一部分人准备叩阙,却还没离开河南府呢,不知道会磨蹭多久。等他们入了开封地界自然会报予皇后。”

他笑了一笑,恐怕嵩阳书院里面的学生都不会知道朝廷的耳目有这般厉害,“这也是政事堂想息事宁人,毕竟嵩阳书院里面有不少世家子弟。而那些流言蜚语,传到皇后耳朵里,也不是美事。”

“蔡相公有这么好心?”苏颂狐疑的看了韩冈一眼,外地的流言报上去后,可是只会让皇后更恨旧党,忽然他有了些明悟,“二程就在嵩阳书院吧?”

韩冈摇摇头,虽然他猜不到具体原因,但以蔡确的为人,肯定不是这个理由。随口道:“两位先生都不是会逞于口舌,惑于众论之人。而且伯淳先生已经接了诏,不日将会抵京,应是与两位先生无关。”

区区一个嵩阳书院,又是在洛阳,根本就影响不了大局。京城才是天下至中,要想控制士林清议,京城的国子监才是关键。西京虽然有个国子监,但规模和声势上可就差得远了。

如今国子监中,尽为新学弟子。纵然不一定认同新法,但他们中的绝大多数倒是对新党占据朝堂持支持的态度,这代表他们的前途将会依然稳定,不会受到朝局的干扰。万一旧党上台,又改回以诗赋取士,那就是哭都哭不出来了。

尽管还是有极少数人听信了谣言,想要起来闹事。但十个八个的异论者,在两千人的国子监中,根本连个泡都冒不出来。

“至于”

“这是蔡相公该考虑的吧?”韩冈笑道,“昨日子容兄去韩玉汝府上,他是还坚持请辞?”

“没有兄为宰相,弟为参政的道理。又是北人,又是支持新法,同时还有资格做宰相,只有一个韩子华。他回来,自然韩玉汝留不得。”苏颂对韩冈扯开话题有些不满,又扯了回去,“不说这个了,玉昆,你倒是一点都不挂心啊,是不是又是因为事不关己?”

“怎么会?”韩冈笑道,“没听到流言中小弟被骂成什么样了?”

“玉昆你会在乎这点小事?”

“那是因为现在只是流言,但要是被世人认定是事实可就吃不消了。”韩冈哈哈笑道,“幸好没接下那两个差事。”

苏颂知道韩冈说得是参知政事和枢密副使二职,他现在也能明白韩冈的决心为何如此坚定了,跟着笑了起来:“说得也是!幸亏没有接下。”

当初他就没有给参知政事晃花了眼,如今又怎么会给枢密副使迷惑?

韩冈拒绝枢密副使的理由跟拒绝参知政事一样,之前是怕被新党当成出卖利益的死敌,而这一次是怕被关中士人视为出卖旧党。毕竟关中士人只是因为西事的关系,才对新法持赞赏态度,对南方人居多的新党,则有着不小的成见,反而更看旧党更为顺眼。

不过他倒是没想到会有这么疯狂的流言,更没有想到会有人准备叩阙上书,现在看来自己倒真是做对了。

邓绾能说‘笑骂由汝,好官我自为之’,更有刘筠为清凉伞而‘生病’,韩冈却是说不得、病不得。邓绾、刘筠,心在朝堂,而韩冈则心在学术。

如果想要在学术上走得更远,让气学更加光大,自己的名声比起什么都重要。至于枢密副使,没看到自己现在天天进崇政殿吗?与宰执班共议军事,这跟宰辅有什么区别?

现在能攻击韩冈的指责,其实归结起来,只有沽名钓誉一条而已。以韩冈过去的声望,让世人相信的可能性也很小。如果韩冈当真接下了枢密副使一职,那就绝不只这么简单了,不但对手们自此有了把柄,就是气学门人,也会有不少人会感到失望。

而且韩冈的名声对眼前之事也极为重要,只为皇帝、皇后和太子,他的名声也坏不得。只要名声还在,他对天子病情的确认就会为天下人承认。一旦他的名声坏了,那么这段时间他所参与的一切事务,都会陷入世人的怀疑中。

……………………

吕公著就要去大名府了。

从枢密使的位置上落下来,而且还是引罪被责,使得他带着全家老小离城时,身边孤伶伶的只有五六人相送。

只是经过了这么一段时间,他倒是看得开了,觉得至少应该比王珪要好上那么一点。王珪他按照御史们弹劾他的罪名是罪恶昭彰,尽管天子,可朝堂上还是避他如避蛇蝎。吕公著估计送王珪下扬州的官员,绝对会比自己要少。

“晦叔先生。”刑恕骑着马,跟在吕公著的身后。

“不要送了……都已经送了十五里了。”吕公著感慨万千,前些日子还是宾客盈门,但如今还跟在身后的门客,只剩下寥寥数人,

刑恕闻言便笑道:“天色还早,再走一走。”

吕公著还想说什么,但看到刑恕脸上的坚定后,便又不准备开口了。能坚守此心,已经是极之难得。所谓疾风知劲草,板荡知良臣,也只有到了这样的绝境,才能知道谁为忠,谁为奸。

一路将吕公著送了三十里,刑恕这才会返回东京城。

回程时能稍稍走得快了一点,用了一个时辰,遍传过了城门。进了城后,刑恕便径直往西,当眼前皇城城墙已经快要仰头来望的时候,他便轻车熟路的向右一转,立刻转进了一条大街中。再向前走了几步,又是一个巷口出现在前方。

刑恕骑着马在正巷口上向里面一张望,三丈宽的巷子——叫街其实更合适——完全给车马堵上。巷内除了车马外,只能看到连绵不绝的院墙和一道大门。一眼望过去,黑压压一片,只在中间留个一条仅可容一辆马车的小道,比这段时间门可罗雀的枢密使府强了不知多少。

不过刑恕并没有挤进去,而是摇了摇脑袋,叹息了一声吼便拨转马头,换了一个方向,沿着这间府邸高达丈许的院墙,绕了大约半里路,终于在前面出现了一道一丈多宽的大门。只看门宽,在普通的官员府邸肯定是正门的形制,但门扉仅有两扇,也没有涂上朱色,更没有门钉,却是不折不扣的偏门。

能使用偏门的,不是家中亲友,就干脆是仆役家丁,正常的访客都是得在正门外候着。但刑恕是个例外。

当他到了门前,守门的司阍只张望了一下,就立刻陪着笑脸迎了上来,“刑官人,你可是好久没登门了。”

“近日事忙啊,奔走来去。”刑恕笑吟吟的,并不以说话的是个地位低微的司阍而小觑,“你家的三哥最近的身体可还好了一点。”

“谢刑官人挂念。”司阍打躬作揖,连声道:“多亏了刑官人啊。前些日子说的那个方子的确管用,家里的小儿两幅药下去,还真的就缓了过来,如今也能下床了。家里就剩这根独苗,还是靠了刑官人给保住了。”

“能救人是积德,说起来我还要谢你让我极了德。”刑恕笑笑,“虽然不比韩学士的医术神授,但洛阳的邵先生也是阴阳五行、医卜星相无不通晓。富、文几个相公平日里有个头疼脑热都要求到他门上。这副方子,就是从他那里得来的,自然有神效。”

“说得是,说得是。”司阍连连点头,笑得脸上的皱纹都堆了起来。

刑恕整了整衣冠,正色对那司阍道:“请报与持正相公,刑恕来了。”

