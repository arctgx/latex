\section{第29章 百虑救灾伤(四)}

【有事回来迟了。现在才赶出一章,下一章也要迟一点了。】

听着枕边人下床的声音,严素心被惊醒了过来。身边还有熟悉的味道,但床铺的一半已经空了下来。迷迷糊糊的睁开双眼,屈肘支起身子,望着正站在窗前爱郎的雄壮背影。

“起来了?”韩冈听到身后的动静,回过头来,顿时眼前一亮。

素心一夜承欢,半眯着的眼睛虽显着疲惫,却有一种难以描画的媚态。她拖着被褥掩着胸口,如云的秀发垂在枕边。但露在外面的一弯玉臂白皙娇嫩,虽是纤细却瘦不露骨。而锦被下,正侧过来的娇躯跌宕起伏,映出一条让人口干舌燥的曲线。

韩冈走过来,坐在床榻边,将素心的身子扳过来,靠在自己的胸口上。动作中,遮着胸前的被褥拖了下去,一对皓洁如玉的丰盈亭亭挺立在空气中

县衙中的厢房,韩冈都让人改成了热炕。撤掉了不方便使用、而且在冬天经常会闷死人的火盆,房间的温度却比旧时还要高出不少。

严素心还是不太习惯白天时的亲昵。虽然房中只有自己和韩冈,但阳光已经从微敞的窗户处透了进来,连同着清寒的空气,刺激着暴露在外的细腻肌肤。

“官人!”素心扭着身子,微嗔道,“天亮了,还要做正事呢。”

“正事早就安排好了。都快过年了,不会有什么大事的。”韩冈轻笑,轻轻重重的啮咬着素心敏感的耳垂。

几个月来的枕边空虚,这十几天来使得韩冈夜夜笙歌,妻妾都是雨露均沾。不过他早上起来却依然还是精神奕奕。自从妻儿到了身边之后,韩冈对于政务上的公事操办得没有之前那么紧迫了,给自己减压之余,也让衙门中的官吏们稍稍松了一口气。一方面是韩冈想多陪陪家人,另一方面,也是主要的原因,还是各项筹备工作基本上做得差不多了,只等着好戏开锣。

已经到了快过年的时候。虽然今年的年景看着不对,明年的情况很可能更糟,绝大部分的百姓都开始俭省起来。原本会买三五匹绢给全家做身新衣服的,现在只给家里的孩儿买;准备买羊买鱼过个肥年的,现在改成买更为便宜的猪肉狗肉。都是如此去想,市井间免不了就有些萧条,只有粮价依然维持在高位上。

“不是还有其他的事吗?”素心知道,现在丈夫的大部分精力都放在防灾救灾之上。要不然区区百里之地,以韩冈的才干何至于忙成这般模样?

“现在说这些做什么?”韩冈看透了了怀中佳人要转移目标的用意,把着盈盈一握的酥软胸房微微一用力,便将她还想说的话,全都堵在了喉咙里。

白皙的娇躯,修长的双腿,自己看着都觉得害羞,更别说被人光天化日之下一分一寸的摸索着。但她对此也不敢反对,更不愿反对,只能闭起眼睛任由韩冈摆布。

一只略嫌粗暴的手掌在胸口用力揉捏着,痛楚中混杂着快感。随即一阵饱涨感充满了全身,素心鼻间一声低吟,双手用力搂住了情热如火的爱郎。一番酣战之后,韩冈这才搂着爱妾起身梳洗。

到了吃早饭的时候,一家人坐在一起。韩冈、王旖并排坐着,家中也没有长辈在,就算周南、素心、云娘做妾室的,也都坐下来陪着一起吃饭。

喝着稀粥,韩冈夹了一块作为小菜的酒糟鹌鹑,味道鲜甜可口,带着淡淡的酒香,比起此时常见的腌菜可是好得太多。他多吃了两块,赞着严素心:“素心的手艺当真是越来越好了。”

严素心因为今早的事还有些不好意思,低着,听着韩冈夸自己,这才抬头道:“不是我,是南娘做得。”

“哦?手艺大涨啊!”韩冈略带讶色的望过去,周南琴棋书画都不差,歌舞更是一绝,但她却不擅烹饪,教坊司中也不会教她这些事。过去下厨房的时候,糟蹋食材的本事让人惊叹,后来就不让她下厨了。

“士别三日当刮目相看啊!”王旖开着周南的玩笑。

曾经的花魁红了脸,低声道:“是素心姐手把手教奴家的。”

素心笑道:“是南娘聪明,一教就会!”

“素心姐姐也教了我做,下次换我的。”云娘献宝式的也说道。

吃饭时谈谈笑笑,几个妻妾之间没有什么龃龉,关系都还不错,这是韩冈所想看到的。一对儿女都已经会爬会走,在府中被当成最金贵的宝贝照顾着。有女人,有儿女,这样才是一个家。

也夹了几块酒糟鹌鹑吃了,王旖问着韩冈:“官人,今天还要不要出城去?”

韩冈点点头:“今天要校阅各乡保甲,城外的校场都已经准备三天,晚上要赏赐参加校阅的保丁酒食,可能要迟一点回来。”

白马百姓冬天的生活,并不是休息。在保甲法推行之后,各地的保丁每月都要进行操演,而到了冬天更是要连续多日进行军训,习练弓法、枪棒,还有小规模的战阵。这些事,主要由县尉负责。不过知县本人也有必要参与其中进行监督,而且还要参加检阅。

“保甲的校阅还要办,最近不是要节省钱粮吗?”王旖奇怪的问道。

“这一份钱粮省不得。就算占用了其他方面的开销,开封府也能给补上。”韩冈又叹道,“更别说要防着贼人乘势作乱,只要灾情不减退,白马县的各乡各里,就一直要时刻准备好出人出力。”

从内院出来,就是韩冈的工作场所。主要的公事,还是在三堂的官厅中解决。如果要审案,则试情节轻重

经过了两个月的磨合,县中的政务已经上了正轨。官吏们都熟悉了韩冈的行事作风,而对于韩冈来说,谁堪用谁不堪用心中也都有了数。

诸立算是个得用的,不过韩冈平时处理公务,却多指派了胡二出来做。虽然在县衙的胥吏中,胡二的势力远不及诸立,平日里也对诸立也是恭恭敬敬。但他跟诸立明显不是一条路,所以得到了韩冈或明或暗的支持。不过这一偏袒,是建立在处事决断大体公平的基础上的,韩冈不会为了维持平衡,而坏了更为重要的公平。

韩冈抵达官厅的时候,负责凿井的井十六就已经守在门外。

坐下来后,韩冈命人招了他进来道:“你那边的情况怎么样了?”

井十六恭声回着:“回县尊的话,现在已经凿到了有十五丈。不过这两天正在破石,要慢上一些,但过去后就能见水了。”

韩冈听着点了点头,这个进度还算能让他满意。再问道:“那你今天来县衙又有何事?”

“禀县尊。”井十六一拱手,“眼下水井越来越深,原来县中所批的五十根楠竹已经不够用了,还请县尊再拨下五十根,以护井壁。”

楠竹,也称毛竹。并非白马县所产,在河南也少见,主要生在长江以南。蜀地的日常生活中,用上楠竹的地方有很多。如炼铁,南方用的木炭,北方多用石炭,而蜀地用得则是竹炭。富顺监开凿盐井,毛竹或者叫楠竹,也是必不可少的原材料。

幸好白马县靠着黄河,这一段的河堤甚至号称金堤。为修堤岸,各项物资当然不能少。根部如海碗般粗细的巨竹就是防洪用的储备物资,所以白马县的仓库中也能找到。

储备物资无故不可动用,不论今生后世,都是一条铁律。不过为了开凿深井,韩冈也不管这些规矩了,反正以他的资格不需要担心这方面的攻击,借口也是十分充分的。只是他批下去的投资不小——虽然五十根巨竹数量并不算多,但已经是库存的四分之一——没想到还要追加。

“也罢,我这里还有一百五十根楠竹,就都给你。”韩冈也不管用光了储备后面怎么交代,总能有办法弥补起来的,关键还是在水井上,“但你要记住,这竹子如果能用其他木料替代的尽量替代,实在不行才可用上。决不许有多余的浪费。”

井十六连忙磕头答诺:“县尊放心,小人明白。”

开凿深水井所用的工具,从原理上类似于冲击钻。实际上就是将一个竖起来一人高,几十斤重的铁质冲锤吊起来,让其自由下落,将挡在前面的石板一下下击碎。

据井十六所言,这种重锤叫做圜刃,是蜀地盐井特有的工具。为了将井十六所说的圜刃给打造出来,花了城中铁匠六天的时间。圜刃冲钻出来的洞只比碗口略大,需要用楠竹来做套筒以护住井壁不至于坍塌——不过这么狭窄的水井,如果不能自流的话,要想提水就会很麻烦。

韩冈对这种开井法很是有兴趣,既然盐井、水井都可以如此开凿,那么油井当然也应该可以。韩冈记得后世在白马县,也就是滑县附近,有座规模不小的油田。说不定,就在韩冈的脚底下,便有黑色的黄金在流淌。只要能向下开上三五千米的井深,那么多半就能看到黑色的石油喷上天际。

韩冈自嘲的笑了笑,开玩笑的想法到此为止。在兴趣之前,他更为重要的工作是救灾。

真想要挖油田,还是去延州【延安】更合适一些。延州石液那是有名的猛火油的原材料,鄜延路,乃至关中百姓所用的灯油,多有用着这些渗出来的石油。已经露了头的矿产,理所当然要比潜藏在地下的矿藏更容易开采。

