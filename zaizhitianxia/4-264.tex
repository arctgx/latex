\section{第43章 庙堂垂衣天宇泰(11)}

沈括派来的人当然是来学习种痘法的医生,总共有八人,领头的还是沈括的幕宾,在沈家兼着家庭医生的差事。

韩冈前几天才寄了信给沈括,同时把《桂窗丛谈》一并寄了过去。才几天的功夫,唐州就派人来了,沈括那里的反应算是很快了。

沈括给韩冈的回信也一并带来了。

在回信中,沈括从头到尾都是花团锦簇的一整篇好话。韩冈边看边笑,有文采的马屁他也喜欢听、喜欢看,且受之无愧——任谁能拿出让天下人都都到恩惠的,什么样的称赞都当得起。

不过在信笺的末尾,沈括也隐晦的提出了自己心中的疑问,他想知道韩冈将种痘免疫法推出来,到底是怎么打算的。

韩冈盯着信纸的最后一段,带着微笑抿着嘴。沈括也当有此一问,韩冈在给他去信的时候,已经有所预料。

来到京西之后,韩冈做的事还算是有章法,但突然间,却毫无征兆的将种痘法拿了出来,包括沈括在内,给他吓了一跳的绝对不在少数。

在方城山轨道只是在纸面上的时候,世人都以为韩刚是打算为开封再造一条汴河,打通襄汉漕运,并以此取功。在世人看来,已经可以算是不世之功了。

当方城轨道落成,并展示出强大的运输能力之后,世人的看法又变了一个样,联系起正要风起云涌的陕西,明眼人都恍然大悟,韩冈是项庄舞剑意在沛公。襄汉漕运只是目的中的一小部分而已。他更打算做的,是推广轨道运输,并通过修造轨道加强河北的作战能力,同时延缓西北对西夏作战的进程。用一个轨道来影响军国战略,韩冈一石数鸟的谋算,不知有多少人暗中兴叹。

完成了襄汉漕渠,修成了河北轨道,韩冈晋身宰辅功劳可以说是捞足了,明明白白的当世第一能吏,只要在地方上稍待时日,到了不惑之年,必然能身登两府。

拥有如此光明的前途,几千京朝官,数万选人,哪一个不要羡慕得流口水。但现在韩冈做的事,就让人闹不明白了。于国于民,的确是善莫大焉,但对于韩冈本人来讲,却是显得多余,甚至于前途有碍。

沈括在信中没有细说,但看得出来他对韩冈的做法有几分不以为然。尤其是得到种痘法之后,没有在第一时间献与天子,眼睁睁的看着六皇子成为第一顺位的继承人,如果换作是沈括,想来绝不会拖延……世上应该不会有第二人学着韩冈这么做了。

而且前事还没有结束,就匆匆拿出种痘法,在时机上也有问题。种痘法的推广、河北的轨道工程,还包括襄汉漕渠的方城山水道,不论哪件事,都够人忙上好些年,韩冈现在急急忙忙的一下子拿出来,到底是想要做什么?

沈括用辞隐晦,看起来很是平和的语句中,韩冈似乎能看见一根手指指着自己的鼻子在质问。幸好现在没有玉米,否则沈括那边多半会暗骂他是狗熊掰棒子,掰一个,丢一个了。

沈括不理解也没办法。

韩冈如果说区区宰执根本不在他的眼里,自己的目标,是改变这个国家乃是世界,沈括不是说韩冈疯了,就是掏着耳朵以为听错了。但对韩冈来说,宰执天下只是达到目的的台阶,并不是最终目标,他可不是将一顶清凉伞当成毕生所求的庸碌之辈。

“知我者谓我心忧,不知者谓我何求。”

韩冈低着声,开玩笑的感叹了一声。收起了信,让衙中小吏将沈括派来的人都带过来了。

总共八人,高矮胖瘦都不缺。行过礼后,站在韩冈的面前都有些紧张,就是领头的沈括幕僚,过去曾经见过两次面,今天也变成了木头人一般,关节和舌头都是僵硬的。

“种痘法学起来很简单。本官治伤风的药方都开不了,大黄、石膏该放一两还是一钱都拿不准,就这样还能找到种痘免疫的方子,以诸位的以医术,也就几天的功夫就能学透了。”

韩冈开自己的玩笑,下面却没人敢笑上一笑。

年轻的龙图阁学士虽然在医道上是权威中的权威,可他在下针施药上却偏偏半点天分都没有,这也不是什么秘闻,从韩冈出名的那一天开始,连着他的名气一起传开了。

越是离奇古怪的故事,越是有传奇性的轶事,越是传播得快、传播得广。就是江南乡间的百姓,也有不少人知道,孙真人的嫡传弟子,连张像样的药方都开不了。一个伤风感冒的病人放在他眼前,说不定还能开出大黄、石膏的方子来。但偏偏就是开不了药方的韩冈,却镇着南下的大军,没有被瘴疠所击败,反而灭掉了交趾。

能考中进士,当然就文曲星,韩冈还是进士第九,要说他看不懂医书,谁也不会相信。所以世人在传播谣言的时候,就为此想出了各种各样的解释。其中最流行的说法,就是韩冈在遇到孙真人之后,在一人医和万人医之间选择了后者,从此医书就与他无缘了——虽然听起来神神怪怪的,但依韩冈本人的情况,也只有这样的解释才合理。

眼下韩冈拿出了种痘法,保护天下幼子不再受痘疮之苦。八人都是行医的,当然知道痘疮有多么可怕,每年因此病而死的幼子,少说也有几十万。一年几十万,几十年下去,就是几百上千万。这份功德做得太大了,试问这是普通人能做得出来的?行医时间越长,感触就越深,这几位看韩冈的眼神都是庙里拜菩萨的感觉。

韩冈微微苦笑,要破除迷信,穷尽千年之功都不可能做得到。亲近的人,在惊讶过后就恢复正常了,沈括、黄庸、黄裳这等士大夫,给他们一个合理的解释也就释然了。但衙门中的小吏还有外面的百姓,却跟这几名医生一样,将韩冈神话了。韩冈昨天还从严素心那里听说,他丢下的废字纸,已经有人在高价搜集,准备拿出去当做护身符卖了。

挥挥手,让唐州来的几名医生下去了。从见客的外厅回到内院书房,坐在桌案前,韩冈闭着眼睛,用手揉着额头。

“三哥哥。”韩云娘端着温补的药汤进来。在私下里,她与韩冈还保持旧时的称呼。

看到韩冈深深皱眉的样子,韩云娘立刻就放下了药汤,站到韩冈的身后,熟练的帮着揉起额头。

韩冈靠在椅背上,头向后枕着云娘的酥胸。做了母亲之后,云娘的身子还是有些单薄,不如周南,甚至王旖,不过韩冈头枕的地方,绵软而又充满弹性的触感不输给任何人。她忽轻忽重的帮韩冈揉着额角,动作娴熟,很快就让韩冈放松了下来。

局势已经到了关键的时候。

在他正式将种痘法提到台面上之前,还能有收手的余地。但到了现在,卫生防疫局都成立了,已经是是盅落骰定,要开盅看大小了。

远在种痘法出现之前,甚至是在韩冈推动重启襄汉漕运之前,他的声望和地位已经极具压迫感,天子都不想让他留在京城,害怕没办法安置。但韩冈现在的目的,就只是想回到京城,再兴气学。

可随着韩冈地位渐高,声望日隆,天子对他的忌惮也在一日日的加深。做得越多,惹起忌惮就越重。谁敢让一个二十多岁就攒下了煌煌功绩的臣子留在朝堂上,宰辅们哪一个能压得住他,万一他在两府中做个三十年四十年,皇帝还有落脚的地方吗?而且韩冈传说还是孙思邈的弟子,看孙真人的寿数,韩冈活到八九十岁都有可能。

所以韩冈出外是必然。

在他人看来,想要得到,就必须先学会放弃。想要回京,就必须有所取舍。官位也要,名望也要,御榻上的那一位怎么可能会答应。韩冈想要身登两府,就不得不先在地方蹉跎十年再说。要不然,做些事,将名望拉低一点也行。

只是韩冈从来没想过要靠自污来化解天子的忌惮。世间评价一个人,总是先论德,后论才。名声坏了,做的事再好,也会被人指斥。因人废事的例子,从来不罕见。旧党攻击新法,往往先从新党的人品着手。

世间的风气如此,韩冈也无力改变。他想要推广气学,使世人对自然科学产生兴趣,必须要有一个让人仰望的德行,来配合他的声望。如今韩冈可以自豪的说,在德行上,尊师重道四个字上,世人没几个人能比他做得更好。

种痘法一出,德惠天下亿万兆民,随之而来的声望,就绝不会局限于士大夫之中——王安石负天下重望三十年,那是士大夫给的,新法触动士绅利益,士大夫也能将之收回。但韩冈声望无人能收得了。

韩冈不打算讨好皇帝,王珪那等三旨相公最得天子所喜,却不是韩冈学得来的,也不是他想要做的。他可以肯定,凭借种痘法,就是天子再不情愿,也得给他一个应有的回报。不一定是官位,但必然要请他回京城。

累积了千年的知识,纵然只是吉光片羽,但也是韩冈手上最大的武器。

拥有如此利器,何须曲中求?

韩冈轻笑,大丈夫行事,当做直中取!

