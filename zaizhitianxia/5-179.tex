\section{第20章 土中骨石千载迷(八)}

“王介甫招的好女婿,王介甫招的好女婿啊。”

暖阁中,富弼半躺半靠在炕头上,带着嘶哑的残喘,低声的笑着。

今年春天的一场病后,富弼明显的苍老了。有半年时间没有见外客,都在是在家里休养。厚厚的青海毛毡盖在膝头上,刚刚改造过的暖阁中,早早的烧起了火炕——这是最近开始在洛阳城中流行冬季取暖手段——房内如同暖春。

皱纹爬满了脸上,脸上的老人斑也越发的显眼。露在外面的一双手,青筋突兀,瘦骨嶙峋,似乎只被一层皮包着。原本很是富态的韩国公,已是瘦不胜衣,浑黄的双目半睁半闭,完全不见旧日的神采,只是嘶哑干涩的笑声,依然能撼人心魄。

“殷墟……殷墟……”富弼的笑声存在喉间,暗哑浑浊,“韩冈的手段永远都是这么别出心裁啊。真想看看在金陵的王介甫是什么样的表情。殷墟的事。文宽夫【文彦博】他可不会闲着。范景仁【范镇】也坐不住,王君贶【王拱辰】的宣徽南院使刚卸任,回洛阳来休养,他一向是喜欢随大流的,更别提司马君实了。洛阳城中,想看王介甫笑话的不是一个两个。”

富绍庭将滑下来的毛毡向上拉了一点:“也是前些日子新党的那一帮人做得过头了,竟然禁了千里镜。以韩冈的脾气,哪里可能会忍得住。”

富弼支起眼皮,看着儿子:“还在念着你的那具三寸半的千里镜?”

富绍庭头低了一点,没敢搭腔。他的那具千里镜,光是镜筒前那面三寸多径圆的镜片,连人工带物料就花了整整两百贯,磨制时间近三个月,失败了二十余次才成功。造出来的千里镜,沉得拿不住手,只能安装在架子上,但用来观远望月,比能拿在手上的那种货色,要强了不知多少。

在洛阳城中,沉湎于自然之学的富家子弟有着自己的小团体,每隔数日集会一次,谈天说地,也互相比较着各自手上的显微镜和千里镜。在秦楼楚馆中一掷千金是斗富,较量谁家的千里镜和显微镜质地更优良也是斗富,而富绍庭那具千里镜给他挣了不少面子。可也因为名气大了,朝廷的禁令下来,就不方便藏在家里,只能交到官府里去。

富弼瞥了儿子一眼,重又垂下眼帘:“在千里镜的禁令出来前,韩冈咄咄逼人的样子,你也不是没看到。论《诗经》,攻《礼记》,韩冈可是一点没手软,逼得新党只能从千里镜上着手。”

“可终究还是王安石要‘一道德、同风俗’,才会闹得如今翁婿相争的局面。”富绍庭说道。

富弼点点头。当年富弼还在朝中的时候,争的只是权柄而已,儒门道统上的纷争,则仅仅是在儒林中,像如今道统之争闹得朝野上下动荡不安,全然是王安石‘一道德’的结果。这样的争斗,在未来会给大宋带来些什么,还真是让人担心。富弼可是明白的,秦人焚书坑儒,其实也是‘一道德’的行动。只是在富弼看来,韩冈能闹出眼下这么大的乱子,终究是新学朝中无人的结果。

“韩冈会抓时间,他选的这个时间真正好。”富弼闭着眼,慢慢的说着:“王珪和蔡确两人站干岸;章敦则与韩冈交好,新学诸书他也没有参与编写过。朝中的新学中人,权位连一个比得上韩冈的都没有。若是王介甫和福建子在朝中,至于如此狼狈?”

“当年王、吕二人皆在朝中,但张载最后还是进京讲学了。”

富弼摇摇头:“也不看看那是韩冈用什么换回来的。”他笑了一声,当初还有人拿他出使辽国和韩冈的功劳相比,来打压他富弼的名声,不过现在早就对这种事不在意了。富弼看看儿子,“王介甫就不说了,论手段,福建子其实也不差。他前些日子一大家子从洛阳过,一点声息也没有,让多少人失望了?”

富绍庭点头,这件事还是他跟富弼说的。

吕惠卿前段时间出外,去陕西任职,正带着全家从洛阳过境,还在洛阳城中的驿馆里住了一夜。正常执政出外,就算引罪,一路上照样是饮宴不断。但在洛阳的这一个晚上,吕惠卿是清清静静的过了一夜,并不是没人请,而是他全数都谢绝了。一早出城,走的也是无声无息,家里的上百仆婢,在路上走时,连点声音都没有,治家更甚治军。

 “程家就在靠着城西正门,吕惠卿从西门出城,几十辆车马竟然无声无息的就过去了,大程说他根本就没听到一点动静。”

“福建子多聪明的人的,否则王介甫为何要用他?”富弼冷笑:“在洛阳,他是半点破绽也不敢露给人看的。”

洛阳的显贵们全是吕惠卿的仇人,就算在洛阳境内,犯了丁点大小的错,也能给闹到天子面前——司马光还管着西京御史台。吕家上下几百口,过境的时候,多少只眼睛盯着,可硬是没挑出一个毛病来,连扰民的罪,都安不到他头上。这就是吕惠卿小心的地方,一点也不给人打落水狗的机会。

吕惠卿、章敦,甚至还可以包括韩冈,这些年轻一辈的心术、手段和能力,并不输他们庆历皇佑年间当政的这群老家伙们。

富弼看着盖在自己膝头上的毛毡,要不是自家没几年好活了,真想跟那些小辈周旋一番。

说了一阵话,富弼也觉得累了。富绍庭感觉出来了,轻声问道:“大人,要不先喝点茶歇一歇?”

富弼先点点头,立刻又嘱咐道:“熟水就行了。”他这几个月喝药喝伤了,占点药味的茶汤、饮子都不想碰,也就没滋味的熟水喝得下口。

富绍庭应了,招呼外间的人端熟水上来。之前父子说私话,贴身的仆婢都在外面候着。

富弼喝了两口水,外间这时有了点动静,一人进来禀报,“去独乐园传话的人已经回来了,还带了司马家的人来,说是来给相公问安。”

一接到韩冈借殷墟与王安石辩《字说》消息,富弼就派了人去通知司马光,司马光回复得倒是快得很。

富绍庭问了一下富弼,“大人要不要见他。”

富弼摇摇头,“人就不见了,你去回个话,说劳他挂心。为父又老又病,没心思管这些,这件事让司马十二出面是最好的。”

富绍庭应了就要出去,却又被富弼叫住,“顺便将去独乐园的人叫进来。”

待人进来后,背后垫了两个靠垫,富弼略坐直了身子:“你去独乐园,司马君实怎么说?”

那仆人低头道:“回老相公的话,司马学士只说知道了,并没多问。只问相公的身子好了没有?又遣了家中的亲随来向老相公问安。”

富弼手指动了一下,示意那仆人出去,静静的坐了一阵,忽的一声嗤笑:“也是个不甘心的。”

被人服侍着躺了下来,富弼合上眼帘,静静的休息起来。

富绍庭出去亲自打发了司马光的家人,刚要回去看看父亲是不是歇下来了,一名家丁就拿了张帖子进来:“潞公使人送帖子来了。”

富绍庭接过帖子,却是文彦博意欲约期拜访,问富弼午后有没有空。文彦博身份不同,不是小了一辈的司马光,他的帖子是不能耽搁的。富绍庭拿着帖子进去后,将刚刚准备入睡的富弼请了起来。

富弼皱着眉,翻来覆去看着帖子,叹息着:“文宽夫怎么就这么沉不住气。”

但等他在富绍庭写得回帖上签了名,送来的帖子又多了两张,都是城中致仕老臣的问候帖子,幸而没有说今天就上门拜访。富弼摇着头:“还真是一个接着一个……天上响了雷,地里的蚯蚓就呆不住了。”

到了午后,文彦博果然到了。看到被富绍庭搀扶着的富弼,文彦博立刻快步上前。两人年齿相近,但现在站在一起,富弼明显比文彦博要苍老许多。

“彦国,你可是清减了。不过看着还是精神,倒让我放心了……秋风带寒,先进去再说话。”

一个夏天没相见,文彦博上门来便是嘘寒问暖。待到在见客的小厅中坐定,奉上了茶汤之后,文彦博就捋着胡须笑了起来:“千挑万选的女婿都离心离德,王安石的眼光终归是不到家啊。为争千里镜,可真是敢下手。”

“韩冈不是因为千里镜的禁令。在上请编修《本草纲目》之前,他就已经就有将殷墟发掘出来的念头。编药典,恐怕就是为了将殷墟甲骨给带出来。”富弼感叹起来:“也亏他想得出来!”顿了一顿,又道:“心性也难得。”

富弼可不管当年文彦博和韩冈的旧怨,照样对韩冈赞许有加。

文彦博脸上没有任何异样,低头喝了口茶。

文彦博也知道富弼当年同样是跟他的岳父晏殊过不去,看到现在的韩冈,多半是想起了他自己。只是还是有些不痛快。

富弼岔开话题:“二程当是也收到消息了吧,他们那里怎么说?”

“程伯淳去拜访了司马十二。程颐则是到了我这里坐了坐。我便顺手送了两本金石拓本给他。这件事也没他们说话的份。但与王介甫争道统,他们也不比气学稍差。”

要不是有着开宗立派的地位,以程颢程颐的年纪和地位,如何够资格在富、文这样的豪门家里被视为上宾?

新学成为官学之后,把持了科举,使得门中失了许多弟子。二程一直都是隐忍不发,苦苦挨着时间。但王安石、吕惠卿几年间接连出外,韩冈近日又不断与新学交手,甚至将王安石准备一锤定音的《字说》,给闹得站不住脚,这么好的机会,二程无论如何也不会放过。

富弼道:“进士一科以诗赋取士,从唐时延续至熙宁三年,经过了近四百年时间,才被王介甫给推倒。自熙宁六年开始,科举纯以三经新义取士,至今也仅仅三科。根基尚且不稳,犹有动摇的机会。不过一旦给新学扎下根来,说不准又会是个几百年。”

“说的正是。”文彦博略提了声:“只为圣教正道,也得让人明白新学的错缪之处。岂能让韩冈一人居前?

