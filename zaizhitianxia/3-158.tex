\section{第41章 礼天祈民康(五)}

“我倒想看看韩冈能拒绝诏令多少回?!”冯京色如严霜,罗列于桌上的珍味一口未动,只见他浮在脸上的笑容内,饱含着怒意:“王安石一顶十几次,看他敢不敢学!”

坐在冯京对面,是他的亲家蔡确。

御史台官经常拜候宰相执政,其实有乖议论。但两人连亲家都做了,平时见个面,喝个酒,也是符合人情的。

以蔡确之智,当然知道冯京真正的怒意出自于哪里。

不只是因为韩冈——此等官员,论人数,朝中车载斗量。即便天子再看重,但年岁未免太少,要想侧身二府,至少也要十几年后了——而是因为天子没将冯京这位宰相当做一回事。

他也是宰相,他也是朝堂之中一言九鼎的人物,朝臣在道上见了他,都得立刻避让到一边去。可天子任用他,却似乎只是因为他是跟新党唱反调的。

开国以来,曾经连中三元的,只有寥寥数人而已,而他冯京可是其中之一!

但异论相搅——天子需要的是异论,而不是冯京冯当世。

若说冯京心中没有一点火气,当然是不可能的,是人都会生气。

偏偏韩绛举荐了韩冈,吕惠卿在沉默了一日之后,也同样上书举荐,天子甚至没有征求冯京、王珪的意见,就为此下诏,征召韩冈为中书都检正。正好成了点燃冯京心中火气的诱因。

蔡确看得分明,却故作不知,反而笑道:“相公,难道这不是好事嘛……”

“韩冈推拒了中书检正,却只求军器监。为的什么?就是为了张载的关学和格物之说。这尊师重道的名声都出来了,让天子都破例要召见他来劝说。今日不做中书检正,明日只会升得更快。待到日后,怕是要比韩稚圭都要快一步入二府。”

孙永尽管只在天子面前说了韩冈的真实心意,但这番奏对当天就传出来了,冯京是为宰相,自然是最先听到的一人。

御史台中的蔡确,与所有的御史一样,耳朵长得如兔子一般,当然也听说了。不过他没有冯京的怒气:“全则必缺,极则必反。韩冈进用如此,难得其终啊……”

蔡确其实是在推脱。

宰相在御史面前怒斥一名官员,目的到底是为了什么,难道蔡确会不明白?

只是他不想迎合冯京的心思罢了。

看着亲家不肯点头,冯京心中又多了一层隐怒。

他始终看韩冈不顺眼。原因有很多。王安石的女婿是一条;太过年轻,二十出头就成为朝官也是一条;还有韩冈在流民图一案中的一番话,挡了他半年的时间才得入相当然更是最为重要的一条。

自然,冯京是绝对不肯承认自己是在嫉妒或是愤恨。甚至在他内心里的想法中,也只是觉得韩冈一步一步的向上攀登,待到而立之年,便能公辅在望,其日后必然难制,对后世的天子是个巨大的隐患——他是为了大宋着想,才不喜欢韩冈。

“韩冈虽薄有微功,但其进用过速。甫及弱冠,便已为右正言、集贤校理。不日将及直阁、侍制、学士,以至于宰辅。陛下千秋万岁之后,可有能制之者?!”

蔡确暗暗叹了一口气。

冯京的这番话,肯定是很有道理的。以韩冈眼下就拥有的官品和地位,再有个十年二十年,他升任宰执至少有七八成的可能。而等赵顼死后,到了下一任皇帝登基时,能压得住他的可就不多了。

——皇帝长命的不多,能活过花甲之龄的,十个之中也不一定有一个。大宋开国以来,更是一个都没有。太祖五十,太宗五十九,真宗五十五,仁宗五十四,而英宗更是只有三十八。六十岁仿佛一个魔咒,连续五任天子都没有跨过去。

而臣子长寿的则很多,六七十岁依然身体硬朗的,朝中比比皆是。冯京都五十多岁了,照样康健如旧日。更别说有名的张三影【张先】,已经七十多岁了,可前两天随着新的词作传到京城,又听说他新纳了一房小妾。

韩冈——蔡确见过多次,想必冯京也见过。

身强体健,不让武夫,甚至据说他能开石五硬弓。又是传说中的药王弟子,不说他医术有多高,但如何保养肯定是有一手的。而赵顼则是一幅病弱态,身体一直都不算好,几乎每年都要病上一回。要比起寿数,韩冈压倒赵顼的机会,远远过之。

但这话冯京能在天子面前说吗?能当着面说赵顼活不过韩冈?

这个话,如果有人敢对天子说,而不是私下里抱怨。那只会是包拯,不会是冯京。

蔡确很头疼,他可以跟宰相为敌,因为上面还有一个皇帝。要违逆天子的心意当然没问题,这是表现他作为御史的气节的好机会,蔡确不是没有做过,也因此得到了丰厚的回报。

但高回报的同时,必然有着高风险。顶撞天子那也是要看时间地点的,万一有一点差错,那可就是鸡飞蛋打。在蔡确看来,眼下绝不是个恰当的时机。在韩冈圣眷未消的情况下,蔡确决不愿意明着跟他为敌。

“少年得志,极易骄狂。如杨亿、胡旦之辈,少年成名,后事难终。”蔡确勉力顶着冯京的不快,“以蔡确愚见,还不如多说他的好话,极力举荐,以重任委之,便可坐观其自败。”

这算是什么主意!冯京阴沉着脸,指出了蔡确话中的破绽:“……别忘了,少年成名的还有晏元献在!”

十四岁被赐进士的晏殊,最后官至宰相。仁宗朝时有名的富贵相公,太平宰相。‘梨花院落溶溶月,柳絮池塘淡淡风’,这等从平淡中隐透着富贵的词作,即便至宝丹王珪的堆金砌玉,也难以与之相比。他任官的闲适,即便是现在,也是让绝大多数官员深深羡慕的。

谁能保证韩冈不是第二个晏殊?

蔡确笑道:“晏同叔乃至诚君子,无事敢隐于天子。韩冈可是这等人?”

蔡确这一回并不是在敷衍,在他眼中,晏元献的确是有着大智慧的人物,而不是寻常人的小聪明,韩冈聪明外显,很难比得上晏殊。

晏殊之所以被真宗看重,就是因为他的诚实。以童子科被荐入朝面圣,看到真宗亲自出的诗赋题目之后,晏殊却说他前两日刚刚做过类似的题目,恳请真宗另行出题。

到了在馆阁中任官之后,其他官员都喜欢出外参加宴会,日复一日。只有晏殊却留在家中读书。当真宗为太子寻找东宫官时,第一个想到的就是晏殊——只因为他不喜饮宴,堪为太子之师——可晏殊到了朝堂上时,却很老实的说他之所以不参加宴会,是因为没有钱,若有钱,肯定也要去的。

这样的诚实,反而让真宗更为看重。而且晏殊的这番言辞,又避免了得罪同僚——这叫做智慧,而不是聪明。

晏殊的行为举止,深为蔡确所敬佩。若有可能,也想学上一学。

而那边的冯京,他既然不喜韩冈,自是不会认为韩冈的人品有多少。心中有对人有了成见,不论什么地方都能看出奸猾狡诈来。蔡确说韩冈不如晏殊,冯京也不会有反对的意见。

“韩冈当然比不上晏同叔,可其人善作伪,等他身败,国事当已被其人所乱。”

无论如何冯京都不能遂了韩绛、吕惠卿的心思,也不能让韩冈得意,否则他这位宰相就当真成了摆设,所以冯京要用到蔡确。

“那也是日后的事了,现在说出来,谁又会相信?”蔡确知今日之事难善了,若不出个主意,可就是要开罪冯京了,“既然相公不愿意一同推举韩冈,那就先看着他会怎么答复天子——天子最近不是要见他吗?以韩冈的性子,在天子面前肯定还会坚持到底。到时候,设法让他恶了天子便是。”

“怎么让他恶了天子?”冯京立刻追问,“韩冈可正得圣眷!要不然,天子也不会特意召见他。”

“韩冈东施效颦,仿效其岳父以博高名,以天子之聪明睿智,岂有看不出来的道理。只要风声传出去,韩冈百口莫辩。试问天子难道会喜欢这样心思诡诈的臣子?当圣眷一去,韩冈还能升得多快!?”

蔡确帮着冯京出着主意。但他心中却是另有一番盘算。

他借冯京为臂助,但有冯京在一日,他就没有在朝堂的可能。御史中丞和宰相是亲家,天子怎么可能能坐得住?吴充之所以能与王安石一掌政事堂,一掌枢密院,那是因为他们关系险恶,换作是他蔡确和冯京可就不一样了。

蔡确现如今真正在想着的,是到底要怎么才能赶着顶头上司邓绾,顺便不露马脚的请走冯京这位亲家,而不让自己纠缠其中,那就更好了。

冯京点着头,似乎已经被蔡确所说服,但他的心中却是暗暗冷笑着,蔡确仍是在敷衍他罢了。

大宋的状元不少,但最后能做到宰相的,可就为数不多。真当他冯京是糊涂人吗?蔡确为了能博取高官重名,与王安石反脸。如今,真正挡在蔡确面前的就只有御史中丞邓绾和他冯京了。

不过只要有用,冯京就会用着。蔡确的身份和眼光,对冯京来说,目前还是很有用的。

举起酒杯,冯京与蔡确对饮而尽,各自心怀鬼胎的笑了起来。

