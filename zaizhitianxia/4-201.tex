\section{第31章 九重自是进退地(15)}

事情变得很奇怪,让许多人都觉得不可思议。

韩冈本来就是要外任,弹劾他根本就是多此一举,就算成功了,难道还能让他罢职回家?还不是到外地做知州等待卷土重来,而且在他即将去主掌襄汉漕渠的当儿,他也几乎不可能受到责罚。

弹劾韩冈,让许多人都想不明白。不过天子为此的雷霆震怒,则让更多的人想不通透。

京城之中,省寺诸衙,皆是朝南而开,唯有御史台北向。这是从隋唐传下来的故事,已经没有人知道是为什么了,就这么一直延续下来。就算是三月阳春,日头正好的时候,房中都是黑黢黢的,只从南面开的小窗中透进一线光来。

彭汝砺坐在阴暗的房间中,也觉得很委屈,作为领衔上书弹劾韩冈的御史,他只是揣摩圣意而已,谁能想得到韩冈抵京后只召见了他一次的皇帝,会对他的弹劾会有这么激烈的反应?

虽说天子的旨意给知制诰孙洙驳了回去,但天子的反应已经证明自己马屁拍在了马脚上了。而且这还是在韩冈还没有任何反应的情况下,天子就下诏了,也就是说这不是被韩冈所逼,而是天子主动要惩治弹劾之人,由此可见天子的怒意。

御史的作用是天子用来制衡宰相,监察百官,所以御史是位卑而权重,希望他们能不顾惜自己的官位,而主动与权臣为敌。因而在天子的刻意纵容下,即便弹劾失败了,也是虽败犹荣,还能大涨声望,最多到外地绕一圈,就能加官晋爵的卷土重来。可一旦御史失去了天子的信任,那结果就是两样。

彭汝砺实在是想不透,明明是天子对韩冈的年轻有所忌惮,不想他晋升太快,也不想他留在机会较多的京城。在彭汝砺想来,自己若是在其中帮着敲打一下,说不定能攀上天子。

而且就算天子不想治韩冈的罪,对于弹劾的奏章,能做的也不过个留中。而韩冈为此闹起来,彭汝砺也不惧,正好可以掀起士林的反感,同时让御史台同仇敌忾,哪里想到天子一动手就是雷霆万钧,让人无可抵挡。

在御史台特产的乌鸦的伴奏下,彭汝砺苦思着脱身的办法,是从此沉默下去,还是变本加厉的反击。

同为监察御史的黄履走了进来.彭汝砺抬头,想露出一个宠辱不惊的笑容,但最后还是失败了。保持着难看的笑容,彭汝砺苍白着脸问道:“出什么事了?”

“有个新消息。”黄履平静地说着,“韩冈引罪避位了。”

彭汝砺的脸色顿时更苍白了,他哪里不明白,韩冈这并不是服罪,而是不依不饶,定要天子分个谁是谁非出来,否则襄汉漕渠就另请高明好了。

可要说韩冈错,那也不对。受到御史弹劾,就连宰相也该避位,韩冈区区一介转运使,哪里能例外。他待罪听参,这态度摆得很端正,任谁也挑不出刺来。

彭汝砺心头堵得慌,黄履带着些许同情的看了他一眼,摇摇头走了出去。弹劾错了人,失去了天子支持,无论哪一位御史都别想在乌台中做得长久。

……………………

“韩冈成不了事!”知谏院的蔡确很肯定的对黄履说着。

“难道他打通不了襄汉漕渠?过去已经修好,如今只是原地疏浚一番就够了。不费什么事啊。”黄履疑惑着。

“并不是襄汉漕运能不能打通,也不是方城垭口的轨道能不能建成。而是建成了之后,到底能不能派上用场!”蔡确对韩冈打算做的事有过深入的了解,“水运的好处是什么?是便宜。不要搬运、不要骡马,只要顺着水走就够了。但韩冈要修轨道,却是省不了多少人工。”

“不是说轨道只是暂时的吗?”黄履反问道,“等渠道挖好,就能由襄阳直入东京城了。”

“所以说韩冈聪明,这是一点没有错的。先修轨道,人工要高一点,手尾要麻烦一点,但只是临时的步骤,下面还会挖渠。可谁知道,他到底会不会将渠道给掘出来?”

黄履忧虑起来,“不过这有违他先前的奏疏,可天子到底还是帮了他。”

“现在帮,不代表以后帮。要是按照韩冈的说法,水渠要向下挖掘六七丈,不会少碰上石头。在东京城,只要向下凿井五六丈,肯定会碰上石头。山地里的石块难道还会比城里的要稀罕?修渠过山,自然是难得的功臣,但失败的情况居多。”

黄履想着蔡确的话,缓缓地点着头。

“渠道开凿肯定是难以成功,韩冈自己都在殿上说要十年八年,说起来,这就跟他造板甲时,先将铁船拿出来做幌子。这么些年了,五十六万禁军,全都有了铁甲傍身,但军器监说是要用钢铁铸龙骨,到现在连个影子都没有。这开渠一事,必然是韩冈拿出来的幌子,真正要大用的还是他苦心积虑要建的轨道。”

黄履听蔡确继续道:“轨道一修,就意味着轨道两端就要设立两个港,来回转运费时费力,到了京城之后,不论是什么货物,价格都要涨个几成,远远比不上水运来的廉价。到时候,轨道太贵,水道又未成事,看韩冈怎么办。”

黄履对蔡确的判断心悦臣服,没有任何异议,“那今次的事怎么说,毕竟那也是御史,总不能不闻不问。”

“该怎么做就怎么做,尽点人事好了。”蔡确满不在乎的说道,“不过不要陷下去,否则就难脱身了。”

……………………

比预定的计划推迟了三天,韩冈离开了京城。

天子和政事堂难得的表现出了高效率,以彭汝砺为首,一应弹劾韩冈的官员,以劾论不实、诬讼大臣的罪名,或出外,或追官,或罚铜,没有一个逃离处罚。御史台和谏院都为此抱不平,但天子不加理会,本来就是装装样子的邓润甫和蔡确,也就各自偃旗息鼓。

只是韩冈在士林中的名声却因此事而坏了不少。御史本来就是该风闻奏事,不必为自己的话负责,但现在只是弹劾了韩冈一下,却让两名御史一同出外,十几人一同受罚。都觉韩冈还没做宰执都能这般跋扈,等他做了宰相还了得?!骂韩冈奸邪的可不止一个两个。为彭汝砺作诗相送的,也有十好几人。

但韩冈并不在意,哪个要往宰执路上走的人,身上没背过跟自己等身高的弹章?能收到这样的待遇,可见自己也算是重臣了。

在朝堂上任职,总得踩几个不开眼的。跟文彦博、冯京、吴充这些宰执们比起来,这两天他遇到的小麻烦,在天子的袒护下,连饭后的水果都算不上。

不过韩冈也不会感激赵顼,要不是当今天子,本来也没这些麻烦。纯粹是赵顼玩脱了,给了外界错误的印象,让一干嫉妒自己的小人,自以为找到了让天子满意,又能踩一下自己的机会。

韩冈出了京城之后,领着全家往西而行。

春天官道,因为道路解冻,十分容易翻浆。沉重的马车车轮压过,就是深深的两道车辙,转瞬间,新碾开的车辙,就会滋滋的冒出水来。

又一次车子陷入了泥泞中,家中的仆人去设法将车子脱出来,韩冈则在一边来羡慕起沈括来。沈括是往唐州去,大半程的道路都能通水路。而韩冈得先去洛阳,只有过了汜水县才能有船可坐。

一路在泥泞中艰难跋涉,韩冈一行很快过了汜水县,道路两边,不再是望不到边的平原,而是连绵起伏的山丘。这就是护卫洛阳的汜水关所在。

“山河拱戴,形势甲于天下。”方兴赞着洛阳,“说起来还是洛阳的地势好,比起无险可守的开封,强出不啻百倍,也不要几十万大军守在京城中。”

韩冈不以为然:“虚外守中是因为晚唐五代藩镇割据,就是定都洛阳,也是照样要有一二十万禁军镇守京中。”

“但相比起洛阳来,开封府还是不好守,要不然契丹当年一入侵,东京城可就一夕三惊了。”

“隋唐长安,自古雄阔无如此城,可隋唐三百年,长安又被攻破了多少次。被敌军打到国都之下,基本上就是日暮途穷,想守也守不住了。”韩冈摇头,“这种想法根本就大错特错。御敌于国门之外已经是错了,更何况御敌于都门之外?”

方兴诧异:“为何说御敌于国门之外都是错?”

“贼众,则以策分之;贼强,则驱夷攻之。弭祸于将生,削敌于无形,此乃不战而胜之法。等到蛮夷兴起时再来布重兵守着边陲,便已经是亡羊补牢了。”

韩冈的一番话不过是寻常的道理,但从他这位南征北战多年、靠着军功上来的官员嘴里,却有莫大的说服力。

方兴沉吟着,缓缓的点头。

但韩冈却话锋一转:“不过话说回来,强军才是根本,谋算仅是枝叶,若无根本,枝叶也不能独存。蛮夷畏威而不怀德,必先使其畏,方能制其用。若是手中并无精兵以供驱用,即便说得天花乱坠,哪家蛮夷会听命?就如小孩子使大锤,吓不了人的。”

