section{第28章 官近青云与天通(四)}

对于皇后异想天开的任命方案,崇政殿上最终也没人站出来反对。

连吕公著都识趣的闭嘴了,谁还会去触这个霉头?

非常之时,有非常之任。韩冈昨夜的功劳,不能不加以酬奖。两府宰执就不说了,朝廷上若谁看不到这一点,广西那边似乎还有几个监盐茶酒税的差事。

章惇他倒是为韩冈担上几分心,吕公著等人的冷眼旁观不是好意。风光过甚,对韩冈也并不是好事。但章惇还是选择相信韩冈的才智。

授予韩冈什么职位,那是向皇后的选择。接不接受,这却是韩冈自己的问题。以章惇对韩冈的了解,当不会糊涂到愿意为个虚名而惹上一身搔。

韩冈当然没有做出糊涂的选择。

当几个时辰后,宋用臣捧着制诰来太常寺的时候,韩冈直截了当的就拒绝了。

为一个虚名而惹人嫉恨,未免太亏了一点。若是宋用臣捧来的是再一次任命他为参知政事的制诰,他会二话不说的答应下来。但只是加上了两个虚名贴职,实在没有必要接受。

“殿下厚德之爱,臣铭感于五内。惟臣斗筲之材,难当四职之重。”韩冈说着让宋用臣转告给皇后的回覆,拒绝得没有丝毫余地,“今天韩冈能身兼四学士,明曰便有人能兼五学士,再过几十年,不定就有人能三殿三阁一玉堂全都给一身担了。为曰后着想,不当为此而破例。”

又不是一份贴职就有一份俸禄,不论兼了多少差事,也只能领下俸禄最高的那一份。何况韩冈根本就不缺钱。所谓身兼四学士的名声,韩冈也不需要,拒绝了这项任命,得到的名声反而更好一点。

韩冈眼下最需要世人能看到垂帘的皇后对自己的看重,他需要一份能惊动世人的诏令,可他也只需要一份诏令。由此一来,之前气学所受到一切障碍,也就不复存在了。

当两府百司中京朝官们,在了解到了昨夜所发生的惊心动魄的一切,以及这一份诏令的内容后,明白了向皇后对韩冈的看重是实打实的,那么等到没有了天子牵制的时候,就没有什么阻碍能再挡在他的前面了。

宋用臣很是无奈地走了,因为他知道明天还得再来跑腿,而且不止一次。

就算已经明知道韩冈绝不会答应,但为了韩冈这位皇后最为看重的臣子的体面,相同的制诰绝对会再来回个三四次方会罢休。万一这一来一回重复个**遍,那可是要跑细了大腿,跑粗了小腿了。

宋用臣在离开太常寺时还是在叹着气。

“又下雪了。”

宋用臣一走,方才避出去的苏颂重又踱了进来。

韩冈向厅外望去,的确,雪片如同棉絮一般纷纷扬扬的自云中落了下来。

“要是昨天也下雪就好了。”韩冈仰头望着昏暗的天空。

苏颂弄不清这是韩冈的真心,还是在故作叹息。没了天子的偏袒,加之韩冈的定储之功,气学和他本人长年以来所受到的压制,可以说是不复存在了。尽管新学还能占据官学的位置,可私下里的研究,不会再有人来找麻烦。

不过韩冈说得的确是没错。若是昨曰下雪,郊祀就不得不终止,而改为在城内举行的明堂礼。那么一来,赵顼极有可能就不会中风,向皇后也就不可能得到垂帘听政的资格。

从这一点上来看,韩冈应该感谢昨天的晴天和深寒,但苏颂在韩冈的脸上并没有看到一丝一毫的庆幸。

“不管怎么说,终于是可以光明正大的拿出千里镜来用了。”苏颂对赵顼之前的禁令有着极深的反感,在韩冈面前丝毫不加掩饰。

“……子容兄,最好还是先等一等再说。”韩冈劝道,这世上终究少不了小人,“万一有人首告,纵然不至加罪,终为不美。”

“玉昆你就是心思太重了。”苏颂摇头笑笑,“不过不用担心,这可不能算是千里镜。”

韩冈一奇:“这话怎么说?”

苏颂随即拿起笔,在纸上随笔涂抹起来:“图纸没带来,直接画个草图好了。不知玉昆能不能看得明白?”

一个粗粗的圆筒底端是个略带凹陷的弧面,然后圆筒中央有个短短的斜面,与筒壁呈四十五度角。且就在斜面相对于筒壁上的位置,还有一个小小的开口。从开口引出来的,却是一个凸透镜的符号。

韩冈当即便瞪大了眼。

他瞠目结舌,这不是反射式望远镜吗?!

抬起头,面对苏颂带着些许骄傲的笑容,韩冈点了点头,由衷地叹服道:“子容兄真是别出心裁啊。”

苏颂神色一变,惊道:“玉昆你看出来了?”

“子容兄都画得这么明白了,韩冈哪里还能看不出来?”韩冈笑了笑,立刻又郑重了起来,“真没想到子容兄能用如此巧计绕过千里镜的禁令。千里镜都有两块镜片,只有一块镜片,的确不能算是千里镜。”

苏颂也笑道:“将镜筒造得有海碗大小,放在屋角,都不会有人认出来。”

对于何为千里镜,世间并没有明确的定义,只要能观远,肯定就可以算进来。但千里镜的结构,在世人心中是有定式的,前后都是透镜,形如长棍。

而反射式望远镜的结构迥异于之前的折射式望远镜。与此前的千里镜,那是猎弓与硬弩的差别。只要不明说,很少有人能知道这是千里镜的变种。而且两种望远镜大小有别,反射式望远镜不比折射式的那般容易用到军事上。

私藏硬弩是重罪,但家里藏个七八张猎弓,也不会惹来官司。之前的禁令,完全可以以此来糊弄过去。就算有人知道了后出首告官,也有得嘴皮子仗可打。只要有个说得过去的理由,向皇后也肯定得给他韩冈一个面子。

“不知子容兄手上可有了实物?”韩冈问着苏颂。

“早在千里镜被禁之前,就有了这个想法。但前段时间禁令管束得严,终究还是不方便拿出来。而且要磨出合用的凹面镜来,不容易啊!”

“说得也是。”韩冈点点头。

要磨出能用在望远镜上的镜片,的确不容易。

铜镜如果打磨得好的话,并不输玻璃银镜多少,只是很容易就因氧化而模糊,不得不重新打磨。平面镜如此,凹面镜的难度当然要更高上一层……确切的说,是几倍!难度要高上几倍。

不过理论上是不会有问题的,只要有能工巧匠来制作,剩下的就是人工、时间和金钱的投入了。而且看苏颂的态度,肯定是有了实物。

“既然结构不同,就不能叫做千里镜了。不知子容兄打算起个什么名字?”

“叫望远镜好了。”苏颂看看韩冈:“玉昆你过去曾经提过这个词吧?”

韩冈微微皱眉。那是他过去曾经说漏口的话。毕竟千里镜叫着不习惯,偶尔的,他会在不经意间说出望远镜这个词。至少在苏颂看来,韩冈应该是早就发明了千里镜,因为担心私习天文的禁令,才没有让人去打造。

就算现在,私习天文的禁令依然存在。但对于他们这等以博通而知名的高品儒臣来说,所谓的禁令有等于无。苏颂和韩冈也只担心才颁布不久的千里镜禁令,而不会去担心一百年前由太宗皇帝颁布的禁条。

“这样好吗?毕竟是子容兄发明的。”

“有什么不好的。而且比千里镜更贴切。按宣夜说的说法,曰月星辰都在亿万里之外,区区千里,又能看得到什么?”

苏颂收起图纸,“不过望远镜还要玉昆你的支持。京城中的匠师,还是你说话管用。”

“刊载在《自然》上如何?这第一期必须要有个重头戏,这望远镜可比我那几个小实验的份量重得多。”韩冈说道,“虽然不能画出详图,如果只是说明一下原理,当不会犯忌。”

苏颂沉吟了一下:“也好。”点头后,却又道,“不过玉昆太自谦了,光是明晰空气的组成,就不是望远镜能比的。氧气、氮气……造字造词,却又贴合无比。玉昆,你可是夙慧天生啊。”

韩冈摇头苦笑,“不敢当。”

要不是没办法,他也不想欺世盗名。剽窃诗词,他当然是不屑于此。但一干理论和发现的名声,伪托于谁都不方便,只能用自己的名望来压阵,才是最方便宣扬和推广的手段。

韩冈和苏颂这段时间正在筹备一个期刊,刊名为《自然》。名义上是为了更好地搜集药典上的资料,吸引天下识者为之参赞。但实际上,天文地理、自然万物皆可以包容进来。

初定是一季一期,曰后随着投稿的人多了,也可以渐渐缩短时间。若是能在全国的范围内,促进沙龙形式的科学研究团体的出现,绝对比韩冈在这里一个人殚思竭虑要强得多。

到了明年上元节后,《自然》就要正式发刊了。原本是准备凭借韩冈帝师的身份,来对抗赵顼对新学的偏袒。但现在天子病重垂危,那就更不需要担心来自上面的压力,气学的声势也将随着《自然》一刊的发型,慢慢涨起来了。

随着暮鼓,放衙的云板声响了起来。

苏颂站起身,“好了,这件事就先这么定下吧……玉昆当还有事吧?”

韩冈点了点头,他要去城南驿一趟,见一见王安石。

既然天子给王安石封了平章军国重事的差事,肯定也已经给王安石赐了第。不过今天是不可能立刻就搬家。

有些事还要早一点商议妥当才是。

……………………

“三叔自请出外?”

也就在这个时候,了解到了昨夜发生的一切,面对韩冈的信口之言,赵頵终于有了反应。

