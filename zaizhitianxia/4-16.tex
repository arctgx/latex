\section{第二章 凡物偏能动世情(三)}

【越是过节,码字的时间就越少,今天就又只有一更了,明天白天又有事,只能保证两更。看看后天或是大后天能不能给补上了。】

傍晚的时候,韩冈坐在熙熙楼后园的包厢中,凭栏下望。

正下方是一池莲叶,而一条条锦鲤就在青青的莲叶之间欢快的游动着。临池观鱼,夕阳在西边的院墙上只露出半张脸,将最后的余晖洒向池中,金鳞点点。鲤鱼不时的跃出水面,溅起的水珠闪着夕阳,如碎金,如玉屑。

韩冈低头看着水面上一道道波纹生灭,听到背后的房门打开的声音,也不回头,却开口道:“一波未平,一波又起,却不知何日能平息?”

章惇大步走了进来,“风浪再大,也有玉昆你的一份功劳!”回头对着脚步钉在门口没有踏进来的掌柜,吩咐了一句,“一切照旧。”

章惇和韩冈是老主顾,他们的口味,熙熙楼中的主厨都已经熟悉了。掌柜沉着稳重的告退,带上了房门。

“学士的话,韩冈可不敢当。”韩冈也早站起了身,与章惇见礼,笑道:“是吕吉甫要下手,却把我给拖下水了。”

章惇就在十天前刚刚升了翰林学士,腰上系了条御仙花带,而鱼袋则照规矩不再佩戴,正是意气风发的时候:“但我说的可是王子纯今天的奏事。”

关于韩冈提议在军中设立教导队,一直争论未休。赵顼本有问政军中将帅的想法,不过给文臣们齐声给否决了,也只有到了这种时候,文官们才会齐心合力起来。但文官们将天子的想法顶回去后,接下来依然还是争论不休,得不出一个结果。

而就在争论不下的时候,王韶站出来提议,教导队中的成员并不限于伤残士卒,而是立有军功的老卒都可加入进去——这项提案出自韩冈,他不好出言更改,故而请了王韶来帮忙。但这个提案还是没能得到通过,无法确定下来。怎么看都很有可能再闹上几个月,最后不了了之。

“对于如今的朝堂,此一事,又何足挂齿?”韩冈冷笑着。

这一项一案明显已经陷入了党争之中,能争出个结果才有鬼,但从另一方面来说,议题已经成功被他给偏转,不会有人再来追究他家家丁实力问题了。但除此之外,还有一件事……应该说是两件事,在朝堂上闹得更为厉害。

韩冈与章惇相邀着坐下来,伸手倒了杯凉汤:“我不过是池中兴波,那两件事可是海中巨浪。”

“沈括、范百禄审了那么久,不就是想将王相公一起拉进谋反案中吗?能绕得过天子去?根本是痴心妄想!”

“沈存中性子软弱你也不是不知道?他哪里能压得过范百禄!想来他也不敢有那个心思。”

韩冈越是了解沈括,就越是想叹息。沈括的确是个博学的通才,甚至还在苏颂之上,去辽国出使一趟,回来后将一路上的山川地理全都制成了沙盘献给了赵顼。韩冈看过之后,以他对从古北口出燕山,直到后世的承德的那一段山川地理的记忆,找不出什么错来。沈括能在后世留下那么大的名声,绝非幸至。但他的性格上却是有些欠缺,实在是太软弱了一点。

章惇冷笑一声,他知道韩冈跟沈括有些交情,不过应该也不深才是。沈括的才学,章惇有所了解,但他可不会太看重畏妻如虎的人物。

“此外吕吉甫为了在政事堂中争一口气,把小弟弄到风尖浪口之上,也是一桩啊。”韩冈笑道,“学士可不能漏掉。”

李逢谋反案将宗室赵世居扯了出来,而赵世居谋逆一案又将道人李士宁牵扯出来,现在世人都在拭目以待,主审此案的几位官员,是否会将前任宰相王安石也一并牵扯进来。这一点,当然让新党无法容忍。

而另外一件案子——也就是汴河水磨坊的厢军攻击韩家一案——吕惠卿揪住了此事,在那边喊打喊杀,一门心思要做成大案。也有许多人,打算看着吕惠卿到底打算将责任最后追到谁头上。在猜测中,多半是两府之中的某一位。

两桩案子已经在不知不觉中,变成了新旧两党之争的延续,支持者和反对者渐次变得泾渭分明起来。而两件案子从刑事案变成了政治案,又从政治案变成了党争的借口,到现在,连是非都无法分清,更不用说判处结果来了。

说到底,如今的局面还是赵顼造成的。章惇和韩冈早已就此交换过意见,两个胆大包天的人物私底下说话时,也没有什么顾忌:“要不是天子打算钧衡朝堂,如何会闹到如今的地步?”

“有着韩子华、冯当世、王禹玉掣肘,又没有当初家岳的名望,天子的支持更不会有当年的全心全意,吕吉甫能顺顺当当的压下政事堂中的其他人才叫有鬼了。再这般闹腾下去,恐怕天子也吃不消。”

门外的廊道上传来故意放重的脚步声,两人不约而同的端起了茶杯,饮了一口凉汤。掌柜亲自带人送来的是正和韩冈和章惇口味的葱泼兔和熏肉脯,另外还有热菜冷盘五六碟,加上熙熙楼特产的两壶美酒,供二人小酌是绰绰有余。

雕花的银器摆满了桌上,门一关,包厢中又只剩韩冈、章惇两人。

“你还是太小瞧了吕吉甫,许多事他都已经提前,就算没有这一次的事,他也能找到几桩事来。”章惇拿起酒杯,“你以为冯京、王珪都是正人君子,身上找不出一点错来?他狠起来,可是会不管不顾孤注一掷的赌一把。只要天子还要推行新法,最后冯京肯定是赢不了。”

“不还有韩子华吗?”

“要说到稳定新法,他如何比得了吕吉甫。”章惇摇摇头,“不说这件事了。倒是玉昆你,这段时间许多事都做岔了。尤其对付打上门来的那帮厢军,忍一时之气,才是最好的应对。”

韩冈叹了口气,半真半假的说着:“谁能想到那百人会这么不堪一击?”

“那也不该急着去抢人家的地。”章惇没怀疑韩冈的话。要说韩冈是事先算好用六七家丁打翻百人,他怎么也不可能会相信,“应该先让监中的铁匠们给闹起来,再来提案那就好了。”

“我是不想冒一点风险,谁知道最后会闹成什么样?”韩冈这一回是真心话,“若是出点意外,毁了监中的工坊,我成了笑柄倒也罢了,板甲的事怎么办?”

韩冈宁可被天子忌惮,也不愿将自己打扮成一个受害者的模样——因为他打扮不了。在用着虚虚实实的手段,以板甲、飞船让世人目瞪口呆之后,他的形象已经确定下来。足智多谋,谋定后动。这样的才智之士,如何会看到闹出乱子才去忙着解决?若是依照章惇的话做了,反而添人口实,还不如一硬到底。

“将作坊迁往城外本身的确没有什么,可若是民间的作坊都开始水力锻锤,到时候玉昆你怎么办?行事不谨、泄露机密的罪名都会落到你的头上。”章惇说着都有些痛心疾首起来,韩冈做出事来之前跟他商量一下,“玉昆你这是授人以柄啊!”

“藏着掖着就能防得住吗?我使人打造的器物,说是军国之器,可仿造起来一点也不难,只要看两眼差不多就能明白。”韩冈很没礼貌的拿筷子敲了敲酒杯,“学士住在城东边,每天应该都要路过观音院。应当看到那一段汴河的码头上,有什么不对的地方吧?”

章惇每天离家早、回家晚,都是匆匆而过,真没有怎么留心。但当他皱起眉来,仔细搜索记忆,却赫然发现,这几天韩冈说得那处地方,的确感觉有些与之前不同。少了一些个力工,却多了两条铺在地上的怪东西。

章惇将他回想起来的发现说给韩冈。韩冈就笑道:“那就是轨道!军器监里面还在试验中,外面就已经拿出来用了。”一想到前两天突然听说城里绸缎商竟然开始在自用的码头上铺设了轨道——虽然轮子仅仅是将旧时的包铁车轮稍作改进——他脸上的表情就变得似怒非怒、似笑非笑,很有些诡异,“给我查出来究竟是谁泄露出去的,定要给他点教训!”

“轨道?”章惇早就听说了此物,知道是最近韩冈在督促军器监中努力研发的东西,惊问道:“怎么就传出去了?!”

“从码头运到库房,原本是靠着人力,但现在车放在轨道上,只要双手来推就行了,轻松得跟冰橇一样,一人能抵二十人的工,用骡马则更方便。码头上搬运的人手至少可以削减三分之一。可以想想能节省下多少人工?有人要卖,当然有人会买。”

“这些奸商!”一听很有可能是奸商收买了军器监中的工匠,章惇立刻发狠骂着。

听韩冈说轨道能省大量人工,他也不是惊喜,而是脸色骤变。京府乃一国之中,天下四方商货都齐聚东京城中,码头和水道边的搬运力工,少说也有数万,如果一下失业了三分之一,对东京城来说,很有可能就会变成一场动乱。

“如果轨道在京城内传播开来,恐怕我家当真要被烧了。”

不比争夺水力磨坊的地盘,韩冈没有一点心理负担——厢军还有他们的俸禄可以养家糊口——也有足够的借口。在军器监中用上轨道,节省下来的人力他也能安排妥当。但如果轨道在京城中推广开来,夺去了力工们的衣食,在世人眼中,可就是他无可推脱的责任。

“比起烧玉昆你家宅院,更有可能是直接烧了轨道。”在章惇想来,丢了饭碗的力工哪有那么多曲曲绕绕的想法,什么东西夺了他们的口食,他们的火气就会朝哪里发。

“这件事可是说不准,”韩冈半眯起眼睛,声音轻得仿佛在说给自己听:“有心人总是有的。”

