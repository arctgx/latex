\section{第16章 三载愿终了(上)}

已是二月下旬。

下了两场雨后,不但京畿一带的旱情稍见缓解,连同比起往年要高出不少的气温,也连带着回复到正常的水平上。

在等待南省发榜的这段时间里,韩冈的生活变得轻松了许多。书还是要读,至少殿试那道关还没有结束,但已经没有礼部试之前,那种火烧火燎的急迫感。

每日里,韩冈都是读书、品茗,偶尔还出去逛一逛街,约上慕容武,和同样结束了考试的种建中,坐在一起喝酒。

闲来无事,韩冈还跟王韶、王雱讨论过殿试时,天子可能会出的题目。看起来根本不去考虑自己会落榜的情况,显得自信心十足。

“肯定是策问!”

韩冈昨日与王雱会面时,王安石的长子是这般说的。在礼部试上,已经出了论,那么到了殿试上,天子会出的必然是策问无疑,这点事不用多想。

具体到策问何事,由于通过礼部试的进士们来自于天南海北,肯定是不会针对任何一个地区的具体情况来发问。

依照王韶的猜测,以及韩冈自己的推断,多半与三年前的殿试题目相类似。

三年前的殿试题目,天子问的是如何是如今的朝政臻至三代之治——‘生民以來,所谓至治,必曰唐虞成周之時,《诗》《书》称其跡可见……要其所以成就,亦必有可言者。其详著之,朕将亲览焉’——也即是如何变更旧时法度,一扫朝中积弊,让赵顼可以做一做一个明君。

今年题目不会偏离这个大方向太多。当然,大方向并不是指变法,而应是针对过去几年施政上的问题,让新科进士们畅所直言。考核进士们的治政水平,征集改进朝廷施政的手段,并向来自四面八方的士子们,询问各地新法施行的真实情况。

尤其是最后一条目的,了解如今天子性格的王韶和王雱,都给了韩冈一个肯定的回答。几乎可以确定,天子不会放过这个了解地方政事的机会。

猜题猜得八九不离十,韩冈自然知道该怎么去做。针对性的去模拟几篇策问,王韶看了之后,还不忘帮着韩冈改上一改其中的词句。

不得不承认,通过诗赋出来的进士,文学水准就是远远高过只明了经义的韩冈。即便十几年来,再没有考中进士前那般用心苦读,但王韶的一番修改之后,韩冈模拟的几篇策问,顿时吟诵起来琅琅上口,而内蕴的含义也因此让人感觉着一下深刻了许多。

韩冈只读了一遍,当即便对王韶拱手一揖:“枢密之才,韩冈自愧不如!”

“玉昆,你以后还是在经义上多下下功夫,至于诗赋……”王韶摇起了头。他倒不是在嘲笑韩冈,但没有天赋就是没有天赋,韩冈在诗赋上的水平,其实比自家不成气候的二儿子强不了多少。

“当年嘉佑二年的进士中,张子厚和程伯淳,都不是以诗赋名世,名次其实排得也很靠后。但他们如今都是天下有名的宗师,玉昆还是学着你的两位师长,扬长避短为好。”王韶安慰似的说着。

“其实若有闲空,玉昆可以向王相公学一学作文写诗的本事。都做了岳父了,总不会敝帚自珍的。”王厚拿着韩冈开玩笑,浑不想他自己的水平,还不如韩冈。

“学不来的!”王厚的话让王韶登时摇起了头,放下了手中的茶盏,极严肃的向韩冈、王厚说道,“当朝才士,有一个半人的文章,是学不来的。”

“一个半?哪一个半?”韩冈立刻追问道。

“半个是苏子瞻,一个就是王介甫。”

王厚咦了一下,眯起眼,眼神漫无焦点的追忆着旧年的记忆:“记得大人以前曾经说过,让儿子不要去学王相公的文章,说是天下文章皆可学,就他一个不能学。怎么现在又多了半个?”

“那是因为苏子瞻当初还没有吃过什么苦头呢……”王韶笑着瞥了韩冈一眼,让苏轼吃了大亏的元凶祸首可就坐在这里,“苏子瞻旧年文章,虽是出众,但也只是十数年、数十年一出而已。但他如今因故通判杭州,传出来的诗作,已经渐渐有脱出窠臼的样子。只是还没有完全得脱旧型,所以他只得算是半个……至于令岳!”

王韶对着韩冈一声长叹:“文章到了他这个地步,已经算是登峰造极了。看似平实古绌,但细细想来,却是一字难易。王介甫任知制诰和翰林时,两制才士中,以他的行文最为简洁,但文字却是最好的。一字褒贬,近于春秋之法。王珪之辈,即便用满了好词,都一样望尘莫及……白首想见江南;欲寻陈迹都迷。这笔力,无人学得来的。”

韩冈点头受教,对王韶看人看事的眼光又更加深了一层认识。

唐宋八大家,宋六家中以王安石和苏轼后世的名气最大。虽然有着各种各样的因素在,但也可以说他们的两人的文章,要高出侪辈一等。

而以韩冈的了解,苏轼如今的文名虽高,但还是没有到后世的水平。几首千古流传的名篇,现在也没有出炉。文章憎命达,在他离开京城去杭州之前,苏轼一路得到贵人提携,来往的朋友,也皆是天下间的第一流人物。人生一片坦途,要想能作出触动人心的作品,当然是很难——直到他被迫离开京城,才有了向更高一层攀登的机缘。

就不知道还没东坡之号的苏东坡,日后会不会谢自己。韩冈想着。

至于王安石的水平,那是几十年的积累出来的结果,当然不是眼下的苏轼可比。厚积而薄发,不经意间写出的诗作,并没有太过追求文字的华美,而是将心中感触随笔而发。他诗赋文章的水平,来自于心胸、见识和经历,文采反而只占到很少的一部分。这样的文字,的确不好学,也不便学。

“先不说这个了,都是以后的事。”王韶将方才说得都丢到一边,“再过两天就要发榜,玉昆你倒是养气到了家,竟然一点也不见你担心。”

正如王韶所说,再有两天就要发榜,能在发榜前还能如此轻松谈笑的士子,当真并不多见。韩冈就算文学上的才华不到家,但他这份养气功夫,也当得起他如今的名气了。怎么说他才二十出头,平常人在他这个年纪,心思浮动得厉害,很少有宠辱不惊、安如泰山的沉稳。

“其实也不需要两天后,明天夜中应该就能知道消息了,昨天见到王元泽,他便是这么说来着。”

殿试上不会黜落考生,仅仅是决定名次高下。只要能登上礼部试的录取名单,那便是一榜进士。榜下捉婿,有哪个会等到殿试之后才挂出的进士榜来捉?直接看到礼部试的结果就该出手了。

大宋皇宫,那是四处漏风,宫内的一点消息,转眼都能传得满城风雨。贡举合格的名单送进宫中,当天夜里就能给抄出来,而排在前几位的,更是天子刚刚看过,转头外面就得到消息了。

会守在在黄榜下捉女婿的,那都是些没有门路的商人而已。若是手眼通天,礼部试合格名单送到宫中的当天夜里,就能派人去守到心仪人选的落脚地,第二天人一出门,就能给捉将回来。

如果今科得中,以韩冈的名望,不同于没有背景的贡生,关注得人绝对不少,基本上明天入夜前后就会有消息传出来。而韩冈的身份,足以让他在第一时间了解到今次考试的成绩,最多也只会比天子迟上一两个时辰而已。

王韶也是点头,“那就等明天了。”

第二天,又开始下起了雨。

一个月来,国子监的大门,还是第一次不是在考试时间开放。一名内侍在一队班直的护卫下,倏进倏出,匆匆离开国子监。

……………………

朝堂上最近并没有什么大事,赵顼的耳根子也难得的清净了一些。

科举是三年才得一次的大典,牵动着天下士子的心。在这件事面前,什么都要放一边,聪明的人都会选择换个时间闹事。

赵顼正等着,今年的考生有名气的不多,比不上去年,更别说与群英荟萃的庆历二年和嘉佑二年相比。王安石、王珪、韩绛,皆是庆历二年及第。吕惠卿、曾布,都是在嘉佑二年博了个进士出身。

想来想去,能让赵顼看高一眼的也只有韩冈一人而已。

韩冈的才智,赵顼当然相信,若是他能在家苦读三年,一榜进士不足虑。只是他这三年来,为国事兢兢业业、出生入死。学问都耽搁了,现在来考进士,就不知会有几分成算。

要是韩冈能有写出些像样的文章,编成几卷文集,直接赐个进士出身,甚至进士及第也不是不可以。就如王安石的弟弟王安国那样,五十卷的文集一献,赵顼直接就给了他一个进士出身。

可惜韩冈现在不论是文名还是著作,都还是欠奉。以他的年纪,当然也不可能会有。唯一可以让韩冈拿到,就是在插手如今。但科举是朝廷安稳的基石,赵顼就算再看重韩冈,也不会在礼部试上动手脚。如果不能通过礼部试的考核,赵顼想给韩冈赐进士头衔,也得再等上几年。

‘唉,这就要看他的运数了。’

殿外的阁门使进来禀报,说派去贡院的人已经回来了。

“来了?”在崇政殿中,终于等到了消息,赵顼精神一振,“快点让他进来!”

