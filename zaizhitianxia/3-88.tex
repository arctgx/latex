\section{第28章 临乱心难齐(三)}

大名府乃是北地重镇。

当操控冀州之地数百年的邺城,在杨坚手中化为废墟之后,大名府就一步一步的成了河北的核心。

庆历二年【西元1042年】,契丹集结重兵,作出南侵的态势。当时朝中迁都洛阳的提议甚嚣尘上。时任宰相的吕夷简则说‘使契丹得渡过河,虽高城深池,何可恃耶?我闻契丹畏强侮怯,遽城洛阳,亡以示威……宜建都大名,示将亲征,以伐其谋。’

虽然吕夷简在他身后,时常被庆历新政的失败者们,在私人笔记中描绘成蒙蔽圣聪的权相或是奸相,但他的见识却是绝对与宰相这个身份相匹配的。

仁宗皇帝,接受了吕夷简的建议,将大名府定为北京,做出了迁都抵抗的姿态,同时派出富弼等一干使臣,在澶渊之盟上所订立的三十万匹两银绢的岁币基础上,又加了二十万。

战争的阴影消散了,岁币增加了六成,契丹人满足了,天子和朝臣也算安心了。而大名府的大宋陪都地位,也就此给定下。

作为大宋北京,大名府向来是河北流民的首要目的地。随着今冬的灾情愈演愈烈,涌进大名府的各地流民也越来越多。

以眼下的形势,就算是文彦博,他现在也不便再继续邀客饮宴。进入十一月以来,他都安坐在家中读书习字,隔上一日,才出外视事一次。因为汪辅之的下场,大名府的官员再也不敢用繁芜的公事来打扰文彦博,这日子,也算是过的清净。

不过文彦博的僚属不敢打扰他休养,但他的儿子敢。

文及甫踏着轻快的步子,走进父亲的书房。脸上的红晕不知是冻出来的,还是兴奋的:“大人,城外又有流民来了!”

文彦博低头看着书,手上拿着个放大镜,在纸面上移动着:“流民来了,值得你这么高兴?”

文及甫嘴角带着笑意,“这么多流民,只要大名府这边稍稍收紧常平仓的放粮,他们肯定要往南边去。”

“这有什么用?”文彦博放下用银框卡住外缘的水晶凸透镜,很平静的抬起头,千沟万壑的苍老面容中,一双浑浊眼睛藏着万千心绪,看不见一丝表情。

文及甫则是阴阴笑着,“只要流民进了京城……”话声这时突然又定住,以他父亲的才智根本不需要他提醒。

文彦博脸色一点点的阴沉下去,如同夏日午后的雷暴就在眉眼间酝酿。这个儿子当真把他给气到。话虽说到一半就停了,但用意已经说了出来。他怎么会有这么蠢的儿子!

抬起手,手指都戳到文及甫的脸上,“小奸小恶,不成大器!到底是谁教你的……”

只是训话训到一半,文彦博突然就给口水呛到了,猛的就咳了起来。年纪大的人,一咳嗽起来,声音就是撕心裂肺。文及甫见着不好,连忙上去拍背舒胸口,一边喊着外面的人进来。

儿子连同侍婢,七八人围着好半天,文彦博这才缓过气来。这时文彦博他心里的火气也消了些,抬手示意下人们出去,这才叹着气道:“你这是授人之柄,自取其辱。真以为大名府这边没人盯着?”

“那……”文及甫发了急,做梦都想回东京那个花花世界去,这么好的机会怎么甘愿就此放弃。

文彦博冷哼着:“流民要来,就尽管让他们来,来个三万五万也没关系。我这边开仓放粮,都会救下,支撑到明年元月一点问题都没有。”

“元月过后呢?”文及甫狐疑的问着。

“今年冬天下雪倒也就罢了,若是不下雪,明年有的王介甫好看!”文彦博抬眼看了一眼儿子,“流民的事,你就不用多想了。多盯着对面的韩冈,学学他怎么做事的。”

“韩冈?!”文及甫一想起自己当时在何双垣墓前,被千万人的呼声给惊得失魂落魄,便是恼羞成怒,“韩冈有什么本事,扇摇暴民,于乱中定案!没治他的罪就够便宜他了!”

“暴民?天子都说了是忠孝之民,你还敢说是暴民?!你以为韩冈那般审很简单吗?仅仅是哭一场就做分辨而已?!”文彦博看着儿子的眼神完全是恨铁不成钢,恨不得一巴掌把儿子打得有韩冈一半聪明,“那是春秋决狱啊!‘哀至则哭’,出自于《三礼》。抓着这四个字,韩冈就是立于不败之地,《刑统》《疏议》都要靠边站。除了你,没人敢不服气!”

文彦博过去在韩冈手上吃了不少亏,而韩冈的行事作风,文彦博也向来看不惯。只是成见归成见,但要说他会看不起韩冈的才智,那也是太小觑他文宽夫观人的眼光了。

远的不说,就是今次断案,根本没证据的三十年积案,换作他文宽夫自己来审,也只能从‘孝’字入手,作出来的决断,也就跟韩冈差不多——毕竟用春秋决狱,才可以将刑统定不下来的案子给断了。自董子以经典要义来断案之后,这样的案子,就算刑律在上,都别想驳得了。

只是文及甫被父亲教训了,心里也对韩冈多了几分忌惮,不敢再小觑那个灌园子,可他嘴巴上还不服气,“韩冈再有本事,总不至于跟韩琦一样,三十四五就升到宰执之列!”

“韩琦?”文彦博冷笑连连,胡子都在抖着,眼神冷冽:“韩稚圭也就是在朝堂上有本事,出了外就没成过一件事!要不是因缘际会,他能有枢密副使做?!”

作为元老重臣,韩、富、文等人之间,在表面上都会保持着基本的交情。可私下里,文彦博对两有定策之功的韩琦是又羡又妒。在他看来,韩琦几次出外,从来都没立过什么功劳,不过就是个庸官罢了,他所举荐的任福甚至还全军覆没,让西夏得以顺利立国。能有如今的地位,也就是在朝堂上站对了位置,适时说话罢了。换作是自己,一样能做到。可恨自家几次任相,时候都不对。要不然,也没有韩琦得意洋洋成为定策元勋的机会。

听出来父亲对韩冈的评价竟然要超过韩琦,文及甫惊得瞠目结舌。虽说父亲一向看不起相州的那一位,但拿韩冈比韩琦,未免太看得起那个灌园小儿了吧?!

文彦博皱眉瞥了儿子一眼,对文及甫目瞪口呆的样子越发的看不顺眼。

灌园家的儿子政事、军事、刑名样样拿手,在经义上还有发明,格物格出来的这个水晶阳燧——现在都叫放大镜了——在士大夫家中已经流传开来。年纪大一点的,都会想办法从宫里讨上一块。当年欧阳永叔,就是眼睛不好,平常读书,都要别人念给他听,若是当时就有这放大镜,也会方便点。

再看看宰相家的儿子,各个都不成材。自家八个儿子,出外任官的,在身边守家的,竟没有一个能算上出色。也幸好不止他一家如此,富弼的儿子也一般。而韩琦家的儿子,也不如乃父多矣!

当真是一任宰相,将几代人积攒下来的福德都耗尽了吗?文彦博无奈的想着。

“眼下都冬月了,天气也冷。今年你就不要出门了,就在家好好读书。”文彦博对儿子彻底失望,现在这个时候,决不能给人抓了把柄去,“明年有的要忙!”

……………………

天气一天天冷了。

宋代的冬天,在韩冈的感觉中,要远远冷过千年之后。位于白马县这一段黄河上的冰层,在农历十一月竟然已经有一寸厚了。韩冈站在又萎缩了一半的河道边,眉间的忧虑怎么都掩饰不住。

脚下的土地全都冻得硬梆梆的,因为近着河水,在干裂的河床缝隙中,还能看到冰。但在城中,就算是清晨的时候,在瓦上、檐下,甚至都见不到白霜。

他身后的方兴正捂着鼻子,仰着头。这空气干燥的,一不小心就会流鼻血。而鼻血还是小事,城里的屋舍就如干柴一般,哪家不小心走了水,火势转眼就能烧起来。

“回去最好要将潜火铺给多设几个,人数也要增加一些。”方兴抽了抽鼻子,感觉终于好一些了,“以眼下的人手,一片火烧起来,根本就救不了。”

“嗯,的确。”韩冈点了点头,想想又道:“白马渡也要安排人,待会我们就去看看。”

白马渡作为黄河上的大渡口,来往行人既多,在渡口周围,便形成了一个六百多户人家的镇子,户口还在白马县城之上。白马县的商税,大半来自于渡口的镇子,说到加强防火,渡口镇要比城里更重要。

韩冈说这就转身往堤上走,边走边说,“还要小心城外的流民营。现在人还少,不会有火患。可过一阵子,要是人多起来,就会越来越危险。”

方兴道:“听说大名府的文相公已经下令将常平仓敞开放粮,这些日子,渡河南来的流民比起预计可要少多了。”

“这是好事啊!”

韩冈原本还担心文彦博会为了政治上的斗争,而将流民往南边来驱赶。现在想想,自己也许是将对方想得太龌龊了一点。做人也是该有下限的,这么多百姓,都是活生生的人,正常人怎么都不可能将他们当成工具。

