\section{第24章 兵戈虽收战未宁(三)}

苗授有那么一瞬间以为自己要败了。

苗授以一千对四千,虽然抵抗得有些吃力,但他的兵胜在阵型严整。稳固如大河长堤一般的展现,将冲杀过来的吐蕃骑兵,用强弩堵在阵前。他的一番出色的指挥,将手下千人的实力发挥得淋漓尽致。

随着一通通鼓响,离弦而出的箭矢,密如飞蝗。禹臧家的吐蕃精骑,根本无法突破箭雨划出的防线,甚至不能接近到宋军阵前三十步的距离。吐蕃人不是没有想过利用着兵力上的优势。自开战以来,禹臧军已经有两次派出分队绕过正面的战线,试图侧击宋军的阵列。

但数十年领军,苗授对于战场地势的把握,早已炉火纯青。他所选择的布阵地点靠近着山麓,黄土的地表,被夏日的暴雨冲刷出道道沟壑。虽然此时沟中早已干涸,但这些细小的沟壑,足以让骑兵举步维艰。而缓下步子、无法冲锋的骑兵,是弓箭手们最好的收割对象。

付出了数百伤亡,从对面的白色大纛下传出来的号角,一声比一声急促。但无论大纛下的吐蕃主帅怎么催逼,但在宋军的阵列前沿,依然有着一条不可逾越的空白地带。

气急败坏的号角,让苗授眯起眼睛享受着。在战场上时时刻刻都不停回荡着的吐蕃人的惨嚎,在他听来,却是比京城教坊中花魁们的歌声还要动听。

“哈哈哈!射得好!!!”

看着一名仗着身上的盔甲、硬顶着箭雨往前冲的吐蕃战士,连人带马被四五石的强弩射成了刺猬,苗授放声狂笑。上了战场之后,温文尔雅的外皮早被被他丢到了九霄云外。如果古渭寨中的官吏们能来到战场上,来到苗授的面前,绝不会相信这名正咬牙瞠目、为战争而兴奋得脸皮涨红的的中年男子,竟会是比进士出身的王韶还像名士大夫、一贯雍容闲雅的苗都巡。

苗授自到古渭之后,心情从没有这般畅快过。他今次受命领军出战,放弃了渭源堡中的纳芝临占部的蕃人,也没有征调乡军弓箭手,只带着一千上过阵的禁军。虽然王韶对此不无忧虑,但从现在的情况看来,他的选择带来了最好的结果。兵力并不一定代表实力,精锐且久经历练的关西禁军,并不是蕃人和乡军可比。

苗授相信,他只凭手上的这一千人,就足以击败禹臧花麻拼凑的六千大军。就算速度跟蜗牛比高下的韩冈最后能赶来,也只能吃些残羹剩饭了。

想到自己可以一人独占领军得胜之功,苗授便忍不住心中的狂喜。而西路都巡检的这份兴奋之情,一直保持到从星罗结城的方向突然杀出来一彪吐蕃骑兵的那一刻。

苗授正因雷霆般的战鼓而沸腾起的血液,在看到了对方一瞬间,一下冻结了起来。闯入战场的军队,打着的将旗是西夏的样式,博来了禹臧大旗下的一阵疯狂欢呼。差不多有着接近两千人的兵力,让苗授和他的儿郎们要对付的敌人一下增加了一半。而且这些骑兵手中还摇着许多属于大宋的军旗,更是把宋军的士气打倒了最低点。

为了让麾下的将士保持足够的信心,苗授一路赶来时,没有少向他们灌输韩冈将会把援军带来。可眼下出现在他们面前的,并不是名震秦凤的韩机宜,而是属于敌军一方的吐蕃骑兵。

援军的出现,使得战局开始向禹臧军一方偏移。鼓点透出了慌乱,箭阵在一瞬间出现了破绽。觑准这个机会,一声尖利的号角之后,一队披甲骑兵突然启动,顶着稀疏下来的箭雨,开战以来的第一次,冲击到了宋军的阵前。

尽管用着自己亲领的神臂弓队,将这一支骑兵逼退,但苗授已经在考虑该如何才能安全的撤退了。只是片刻之后,又一支吐蕃骑兵冲进了战场。看到骑兵们的装束,自苗授以下,许多人一阵手脚冰凉。韩冈让人高高挑起的将旗,在他们眼里,已经变成了星罗结城的宋军失败的象征。不过,禹臧花麻接下来的反应让他们终于明白过来,今次来的是自己人。

“是韩玉昆!是韩玉昆带回来的青唐部蕃骑!”

眼神如鹰隼一般锐利的苗授,他在那支队伍中,发现了一队汉家装束的骑兵。虽然他们看起来有些狼狈,但苗授观其军容,却绝非败阵之军。

尽管未能弄清韩冈为何会跟两千贼军前后脚赶来,也不清楚这些贼人为何会拿着大宋的战旗,但宋军这一方将士,已经开始为援军的到来而欢欣鼓舞,降到底限的士气,也开始徐徐回复。

禹臧花麻看着对手的援军再次出现在自己的面前,木然的脸色下面,是满肚子的恨意。对于一群废物,他并没有抱着多少希望,但看着他们拿着宋人军旗、盔甲,还以为出人意料地获得了胜利。没想到,却是这么一回事!

过人的才智让禹臧花麻很快就想透了一切。他派出去的一群废物,吞下了宋人奉送上来的饵料,却成功的把钩子吐了出来。没有中了埋伏的确是桩好事,可他们也没有完成他交代的任务——竟然让瞎药带着他的兵重新回到了战场。

已经失去了胜利的机会,禹臧花麻心中有了数。四对一都没能做到的事,当六对二的时候,更不可能成功。趁着星罗结城中的宋军步卒还没赶到,他得早点走才行。当然,他需要有人帮忙为他拦一下追兵,做个殿后——望着与本阵会合的那支由附庸部族为主体组成的偏师,禹臧家族长的眼神越发的幽深了起来。

百十只号角同时吹响,号声从天际回荡下来,多了几分沉稳。在禹臧花麻的命令下,先一步从通往星罗结城的道路上回返的吐蕃骑兵,慢吞吞的转回头,去攻击已经在战场边缘立足的青唐蕃军。

韩冈没有再退,他可以在渭源堡援军还未赶到战场的情况下展开游击战,但在苗授和禹臧花麻已经开战的情况下,他的一点退让,都会造成友军的崩溃。同时他这次的出场已经很丢人了,他不想再丢脸。何况他的步军很快就要到了,千名来自关西禁军中的精锐,单是出现在战场上,就足以改变战局。若是能与苗授会合,胜利就近在眼前。所以韩冈必须先为他们守住战场上的一角。

战场上终于出现了骑兵们的厮杀场面。三千多骑兵的对冲,从高处往下,正如两道黄色尘土卷起的巨浪,瞬间猛.撞在一起。人声马嘶从烟云中传了出来,比起方才戏耍般的追逐,这样的战斗要惨烈上十倍。

韩冈与瞎药一起站在了阵后的大纛下,指挥着自己的队伍,以半数的兵力对抗敌人,却至今胜负未定。韩冈抖擞精神,就准备在这里立下一番功绩。可就在这时候,大来谷口处传来了一片喊声,韩冈惊讶的望过去,位于那里的禹臧花麻竟然撤退了,帅旗一拔,走得干脆利落,选得时机也是巧妙非常。

在禹臧花麻的抛弃了他们情况下,正与瞎药决一死战的敌人,一瞬间便丧失了所有的战意。瞎药的冲击仿佛一柄热刀切开黄油,让禹臧家的附庸军纷纷逃散,窜入山中。而禹臧花麻的主力则趁此良机撤退得得更为迅速。

“他们应是禹臧花麻的弃子吧……”苗履揣测着禹臧花麻的用意。

苗授回头瞪了儿子一眼,‘这种事知道就好,何必说出来……反正都是斩首,管他是哪家的?’

留了人下来顶缸,禹臧花麻退得越来越快。在他的大纛方才插着的地方,几具被斩下头颅的尸体横七竖八倒伏着,他们生前曾经是禹臧家的长老,但在今天战场上,也不过是些无名尸罢了。

在渭源堡和星罗结城下吃了两次亏后,禹臧花麻的确是迫切需要一个胜利。但他目的是维护自己的权威和地位,胜利只是达成目的手段而已。如果能用其他手段达到同样的目的,他也不会拒绝使用。

在蕃部中,要想维持自己的地位有很多手段。可以用金银财帛去买通,也可以通过连续不断的胜利来加强自己声望,还可以借助外力来巩固,当然,更方便的做法则是杀鸡儆猴,杀几个底心存不满、怀有异心的反叛者,剩下的自然会老实起来。

禹臧花麻就是因为今次出战不顺,临战之前被人逼宫。站出来的,不是附庸部族的族长,就是本族的长老,皆是手绾兵权的实力派。他虽然熬过了一关,只是禹臧花麻为人心狠手辣,做得便很干脆,把不听话的附庸部族直接就当弃子给丢下了,同时还直接以不从军令的名义动手,将本家中最为不顺服的几人,一起斩了脑袋。

该丢的丢了,该杀得也杀了。现在部族中是禹臧花麻一人独大,一旦没有了会为他人之事出头的对手,就算是今次失败,也不会影响到他的地位多少。

