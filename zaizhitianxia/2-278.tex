\section{第44章 一言镇关月燎辉(中)}

王中正虽然不是什么贤才智士,在经略司中连打酱油的事都不会让他做。但他毕竟在步步险关的宫廷中混迹多年,又在熙河经略司中,与韩冈等人朝夕相处。韩冈隐藏在方才一番话中的用意,他甚至比沈括还要早一步听了出来。

这是在为应对京中的消息提前做准备?

难道真的打算顶回圣旨不成?

王中正睁大眼睛,看着眼前过于年轻的面庞,‘你可不是郭逵啊!’

在王中正的眼中看到了疑惑与震惊,韩冈微微直了一下腰,‘但我是文官!’

韩冈知道王中正想要什么,也知道王中正惧怕什么。在目前的形势下,韩冈可以确定,直到王韶那边最后的结果出来之前,就算自己要顶回圣旨,也不会触碰到王中正的底线——只要不是要让王中正本人出头,他肯定会乐意站在一边看着,顺便祈祷王韶能安然归来——只要还能维持眼下的局面,韩冈只要给王中正一个希望,他就会坚持下去。

至于沈括,韩冈不知道他有什么打算。但在河湟根基不稳的沈括,韩冈一点也不惧怕。就连蔡延庆都拿区区一个王厚没有办法,自己要让这位名震千古的大科学家无所用事,也一样不费吹灰之力。

苗授那边韩冈不担心,别看他与自己关系不睦,前些天还因为香子城下的战事,暗地里有了纷争。但同在熙河经略司中,是一荣俱荣,一损俱损。在保护现有战果的前提下,他们的利益关系是相通的。

前两日韩冈不回来,那是因为还不能确定西贼到底有没有断粮。但现在他已经有了底气,更是与王存联络上了,进一步确认了王存和堡中守军坚守临洮堡的意志。

既然韩冈确认了河州和临洮堡都不会有问题,他自然可以安心的坐在狄道城中,准备着与朝中使节周旋。

七八日的时间一晃而过。

陇西那边送来的家书上,都说他父亲韩千六已经开始主持巩州麦田的收割工作。只要接下来的半个月不下大雨,今天的丰厚就可以确定了。而怀孕的周南和严素心都安好,都没有什么意外,让他放心,照顾好自己。另外还有几套夏天的衣服。棉布缝制的衣衫针脚细密,缝得十分的贴身。

在家书中,还有李信的消息。熙河路与秦凤路分家后,不可能再及时收到秦凤路的情报。但通过私人信件,却一样可以得到。西夏军的前锋十天前已经抵达了好水川。张守约此时正在后方的水洛城坐镇,李信则是受命去了德顺军治所笼竿城。

看到将军中布置泄露无遗的家信,韩冈苦笑之余,也希望李信能安然无恙,并能在此役中立功受赏。

今天韩冈的心情,不免有些紧张。当然不是为了李信,而是李宪。

比家信还要早一天送到手上,王厚传来的消息也抵达了狄道城。在东京城来的宣诏使臣在陇西休息一晚的时候,王厚派出了快马,连夜将这条情报送到了韩冈手里。

“李宪……”

韩冈当然知道这一位大貂珰,也曾经见过他。李宪可是王中正的老对头了,为了争夺监军熙河的职位,据说两边使了不少阴招。但最后,还是靠着运气混了个宫中知兵第一的王中正给赢了。

来的是王中正的对手,韩冈的应对却是该怎么办就怎么办!

一切如常。

兵来将挡,水来土掩。

四更天就上路,只用了一天的时间,在暮色将将笼罩大地的时候,李宪一行抵达了狄道城。

从明面上说,李宪此行没有事先通知,韩冈应该是不知道的。但到了衙门时,出迎的韩冈却是很自然的模样,将李宪迎进了官厅中。

在大厅中站定,闲杂人等都在韩冈事前的命令下避让了出去,只有韩冈、沈括和王中正焚起香案,叩拜接旨。

因为一口气赶了几千里路的缘故,李宪比韩冈上次见面时要瘦了不少。而他身后,背着敕令的小黄门皮肤黝黑,看起来不像个宦官,倒像个武夫。见到李宪伸手过来,他连忙把包裹打开,恭恭敬敬的将包裹中的一卷诏书递到了李宪手中。

“不是在庭中……”

身后低低的传来沈括狐疑的声音。韩冈心头一松,果然,不仅仅是自己在这么想。

‘宣诏’中的一个宣,有着公开、公布的意思。诏书中的内容,丝毫瞒不得人。但韩冈在官厅中接旨,甚至提前将闲杂人等都赶出去的做法,李宪却竟然默认了。以他身为内侍的身份,没有秉持上命,或是明了天子的真实心意,一般来说是不敢如此妄为的。

而且退军的命令,直接让急脚递送来其实会更快。选择了让李宪带人来,肯定是带着体量军事的责任。既然如此,当然就是有得商量,或者说,扯皮了!

精神一震,希望李宪自重一点,不要插手军务。不过有王中正应当会设法牵制他,

李宪念着诏书。

韩冈越听越是轻松,里面的话语虽是命他从河州撤军,却不无余地。有罗兀城为前车之鉴,赵顼肯定会犹豫三分,诏书中并不将话说死,也是情理中事。

而且这份诏书指名道姓的发给他韩冈,没有让其他官员来压制自己,而是相信了他的能力。不然就是让蔡延庆来暂代熙河经略一职,都是个大麻烦。

听着李宪抑扬顿挫,用着唱歌一般的腔调将诏书念出,韩冈能想象得到背后沈括脸上的狐疑。

明着下令让韩冈退军,但实际上却是进一步确认韩冈的指挥之权。他完全可以凭借被天子承认的权力,而把退军的命令顶回去——只要韩冈能承受失败后的结果。

真是个好皇帝啊……赵顼首鼠两端的态度,让韩冈冷笑不已。

毕竟不是开国之君,换做是赵匡胤等明君,肯定是有个明确而不容拒绝的说法。不论是退军,还是坚持下去,都不会把选择之权交道臣子的手中。

天子诏令的权威性才是要他们维持的关键,而不会像赵顼这般犹豫不定,让臣子为他来做决定。

算了,他本来就没有对京中的命令报太大的信心。

双手接下诏令,请沈括代为接待李宪,韩冈托着诏书转身出了官厅。被驱赶在院外的将校和官吏们涌了上来,有人出头紧张的问着:“机宜,天子可是要退兵?”

“退兵,谁说的?”韩冈朗声说着,“天子心忧河湟之事,下诏体问而已,怎么会让我们退军?锲而不舍,金石可镂。最后的胜利就在眼前,如何能够放弃?!”

韩冈的声音其实能够传进厅中,而李宪竟然没有跟出来,任凭韩冈大放阙词。

‘真是聪明!真够识趣!’

但李宪的识趣也到此为止,等到韩冈安抚过军心,他传达着天子的口谕,开始质问着韩冈为什么顿兵不前,至今未能将临洮堡解围。

因为是口谕,韩冈也不得不站在李宪的面前,“请都知上覆天子,西贼狡诈,在外多有埋伏,都监景思立亦是因为妄自出战而全军覆亡。韩冈承蒙天子不弃,授以重任。自是以前车为鉴,不会妄自跳入贼人陷阱,而是将计就计,反其道而行之。还请都知放心,眼下贼人在临洮堡下进退两难,粮草快要断绝,到时候,就是官军机会了。”

“为何不征发乡兵?”

“围困临洮的西贼只是癣癞之疾,若是贸然征发乡兵,惹得路中人心惶惶,才是大患。”

“王韶可有消息。”

“尚无噩耗。”

李宪与韩冈一问一答的对话。他代替天子的询问,韩冈都是尽量圆滑的回覆了过去。到最后,李宪都不得不佩服起韩冈,滑不留手的答复,让人挑不出刺来。心头一阵不舒服,眯起眼,突然问着:“听韩机宜的口气,看来是不想奉召退兵了?”

“全胜在即,眼下绝不可退军。天子几年的顾盼,为臣者岂能辜负。千万人多年的心血,也不能付诸于流水。妄改天子诏令之罪,韩冈愿以身家性命相赎,虽死无憾!”

韩冈语气平静,仿佛根本不把关系到身家性命的事放在心上。

“……希望韩冈你能担得待起。”李宪冷言冷语了一句,起身离开,回韩冈安排给他的住处。

李宪走了,王中正走了上来,低声对着韩冈道:“很有可能有第二道诏令,天子更改心意,是常有之事。”

“唉……希望王经略能快一点回来。”

在王中正看来,韩冈的做法是赌在了王韶的身上。一切都要看王韶那里的结果,如果王韶败了,河州之事就无法再挽回。而韩冈本人,也将落得悲惨的境地。

但韩冈不是这么看。

‘只要河州平定,只要守着露骨山口,只要临洮堡的西贼撤离,就算王经略不能回来,熙河照样是一片乐土。’

但他没有说出来,这未免太过没有人情味了,也不符合他的形象。

他信心十足的微笑着,“先将临洮堡外的西贼解决,下面就安心的等着王经略的捷报传回来!”

