\section{第20章 冥冥鬼神有也无(四)}

一座营寨正在山下拔地而起。

一千多随军而行的民夫,正在山坡下为大军设置营寨。他们的号子声从城下直传入折可适的耳中。

民夫们堆土掘壕,将周长数里的预设营地围了起来。蜿蜒的寨墙沿着河道向山坡上延伸,有水可用,有险可凭,道路易行,这是立寨最基本的条件。

折可适睁着满是血丝的双眼,几天来的不眠不休,让他看着有几分憔悴

十二里之外的丰州城,是处在视线勉强能达到的极远处。在那个距离上,绿色的山峰、土黄色的城垣,都变得模糊了形状和颜色。

倒不是折可适不想在更近一点的地方扎营,但丰州城附近的草木都给党项人清理光了,如果要在接近丰州城的地方设立营盘,就需要到花费更多的人力和时间来砍伐、运送木料,筑营的时间也会拖得很长,自然危险性就会成倍的上升,也只能从权了。

不过现在扎下的是主营,等到攻城时,还要在丰州城下一两里的地方,设置更近一步的攻城营地。到时候,事情就很麻烦了。折可适眯起眼睛,望了望左右的山林,看来只能等控制了丰州城周边的山地,借助水势,从更上游的山中放了木头下来,飘流到扎营的地点。

只是事情可没那么简单。

“前几天已经有两次了,今天不知会不会有第三次。”折可适是在自言自语。

不过他的话听在李铁脚的耳边,倒像是在对自己说话,应声道:“狼崽子只要吃了人,下面就会盯着人来咬了。肯定还会有。”

“应该还是那一队契丹骑兵的手段。”李铁脚咬着牙继续说着,他族中的一个侄儿就死在第一次失败的押运过程中,“如果是西贼,用神臂弓就能射走了。”

“多半就是他们。”折可适对党项人有着深入地了解,能一下抓住突袭的机会,又毫不犹豫的烧光所有的粮秣,只会是不愁吃喝的队伍,想必党项人不敢让他们来自北方的援兵饿着。

这是针对官军粮道的攻击。由于官军的主力还在向丰州城下进发,道路上留给敌军突袭的时间并不多,尽管官军还没有完全控制整条道路,两天来受到多次袭击,但只让西贼成功了两次,暂时还没有问题。

“四郎,不能再让辽狗在后面猖狂了。”李铁脚脸上满是忧虑,“等到郭太尉领着中军到了丰州城下,只凭沿途寨堡的守备,后路多半会更加危险。”

“一开始的方略就是以一半兵力——也就是三万兵马——攻打丰州,而剩下的军力则是分散开来,护卫粮道,分镇沿途寨堡,并监视道路左近的各处险地。如果真的一切顺利,之后也不用太过担心。”折可适动身往山坡下走,“只是也不能光想着粮道,说不定今天就会轮到我们。”

他回头看着李铁脚,“中军赶到这里,至少还要两天的时间。营垒必须要加快,今天入夜前就得建好。”

“只要营垒修好,有这里的五千兵马,就算丰州的贼军都来了,也照样能守住。”、郭逵所率领的本阵刚刚进入丰州地界,不过陆续进抵丰州城下的宋军兵马已经有五千之多,李铁脚很是自信,“若是契丹人当真敢过来,正好杀他们个落花流水。”

折可适笑了笑,再看看修建得热火朝天的营寨工地,心中又感到一阵遗憾,“可惜留在丰州的百姓都给杀绝了,要不然就地调来一批民夫,修建得能更快一地见。”

“哪里还有人了,连个牲畜都看不到了。”李铁脚摇摇头,他可是来过丰州不知多少次,眼下的丰州与他过去的记忆差了不知有多远。丰州陷落之后,一年的时间里,此地百姓死的死逃的逃。本来就是准备用丰州来交换罗兀城,西夏人根本没有治理安抚的打算,一番劫掠杀戮之下,丰州的地界之中已经看不到多少普通百姓。

李铁脚的牙齿咬得咯噔咯噔的直响,“日他鸟的西贼,就是夺回来,丰州也是废掉了。”

折可适紧抿着薄薄的双唇,眼神冷冽。这原本可是折家的地盘,如果不是旧丰州的沦陷,也不会分割出去。这里的百姓有许多都与府州沾亲带故,比如李铁脚的亲戚就有许多生活在丰州。在党项人占据之后,逃出来三分之一,剩下的全都没了消息。

回到正在兴建的营地中,属于折可适的帐幕已经在营地一角搭建了起来。走进帐中,不一会儿,下属的偏将裨将和指挥使们,一个个都聚集到他的面前。

折可适正要吩咐麾下将校今夜小心防备,就听见有人骑着马直奔自己的营帐。蹄声在帐门前一停,然后就有一人直冲近来。

“种……种谔……”冲进来的人是折家的子弟,急切之中忘了礼法,就在帐中大叫着,“前日种谔领鄜延军在葭芦川边,大破西贼两万,斩首不计其数。”

帐中骚然,没有人能不惊讶,怎么鄜延军会往河东这里过来?

‘葭芦川……’折可适将这条消息放在脑袋里一转就明白了,‘想不到给种子正捡了个大便宜去!’

帐中的其他人还在想着,折可适就抬眼瞪着自己的族中兄弟,吊起眉梢,喝问道:“会不会说话,难道没学过规矩!”

被折可适一瞪,信使就知道规矩了,恭声道:“启禀四将军。前日,种太尉遣罗兀城主王都巡领军往河东来,诱使西贼误以为他们意欲支援麟府,故而从神勇军司急追过来,最后就在葭芦川边一场大战,王都巡鏖战于前,种太尉追摄于后,最后大破西贼两万。”

折可适又瞪了他一眼,‘鏖战于前’‘追摄于后’,这小子根本是在背种谔得意洋洋散布给周围军州的捷报:“‘大破西贼两万’这话当是吹嘘。神勇军司才多少兵?不可能倾巢出动!”叹了一声,“不过神勇军司兵溃,此事不可能有假。他们落入种太尉的算计中,伤亡也不会小。”

折可适扫了一眼帐中渐次醒悟过来的众将校,“神勇军司遭逢惨败,西贼在银州夏州的驻军一个办法就是直攻罗兀城,设法歼灭从葭芦川赶回的官军,这是反败为胜之策。不过以种谔……种太尉之智,当不会留下这个破绽,很有可能再趁机咬上一口。如果不打算采用攻打罗兀城这一个策略,银夏的西贼就只能设法分兵去支援神勇军司的防务……接下来鄜延军会怎么做,你们应该能想得到。”

“种太尉是想打银夏?!”

“那不是当然的?以种太尉的脾气,会甘心看着郭太尉在丰州建功立业?”折可适问着,帐下众将一齐摇了摇头。种谔是什么性格,大家都知道。他与郭逵的恶劣关系,在军中也不是秘密。

“而且银夏的驻军与神勇军司的几个大族都没有什么交情,”折可适继续说着自己的推断,“非亲非故,为他们拼命的可能性很小。故而不大可能会去冒风险攻打罗兀城,反倒是分出一部分兵马去神勇军司守着,几面都能交代得过去。”

“那我们该怎么办?”没人甘愿自己拼命,却让种谔在旁边捡便宜。

“当然是趁此良机大加宣扬,动摇党项人的军心,以求尽速破城克敌。”

“说得没错!”帐外传来熟悉的声音,折可适抬眼一看,更加熟悉的身影正掀帘走近帐来。

“父亲,你怎么来了!?”

看见父亲折克行掀帘进帐,折可适吃了一惊。就算折克行命外面的守卫不要通报,自己也该听到兵马入营的声音啊。

折克行似乎猜到了儿子在吃惊什么,走进帐来,很自然的坐到了主位上。

“路上遇到了契丹军,拼了一下,给他们跑了。伤了一些人,走得慢拖在了后面。”折克行浑没把丢下大队、带着十几个亲兵赶到前线的危险之举当做一回事,“不过也捉到一个活口,削了几根手指之后,倒是承认自己出自西京道的皮室军了,后面就给了他一个痛快。”

此前虽然猜测着有如此威势的骑兵,必然是出自辽国最精锐的队伍,但当听到折克行亲口确认之后,帐中众将还是变了颜色。

“辽狗是偏帮着党项人了。”李铁脚发狠道,“抓几个活口押到北面去,看看辽狗的皇帝宰相们认不认。”

‘没用的。’折可适心中摇头。即便他们的身份被证实,开封那里也是不敢戳破的,否则官场、士林、民间都会起大乱子。唯一能做的就是全都当成党项人一起斩首算功劳,反正北面既然让他们打着西夏的旗号,当然也不敢承认他们的身份,“这一队契丹骑兵必须尽早解决他们了。要是让西京道那一部皮室军误认为我们好欺负,又多派人来,事情就不好办了。”

“办法也是有,不过这就要有人冒一点险。”折克行的一对鹰隼般锐利的眼睛盯着自己的儿子。

“孩……”折可适声音一顿,改口抱拳:“末将愿往!”骁勇之意,气冲斗牛。

“小心一点。”折克行如上司看待下属的眼神中,这时又多了分父子连心的温情:“要小心一点……”

