\section{第30章 肘腋萧墙暮色凉(13) }

大清早,天上就是灰蒙蒙的一片。到了快中午的时候,天上的云彩更是一片灰黄。沙尘落了满地,积雪的山头也给染成了黄色。营地中人人名副其实的灰头土脸,连关在营中的马匹,不论是黑毛的、栗毛的,还是白毛的,现在全成了黄毛。

韩冈呼吸时,都能感到一股浓浓的灰土味道,口中鼻中都发干发涩。在外面站上一阵,头上身上便满是落下来的沙土。为了解决这个问题,他让下面的人帮忙用细麻布缝了几个口罩,准备上路时试着用一用。

左近的山头上本都被未化的积雪所覆盖,也就罗兀城这片工地上,积雪都被清理掉,加之挖地取土、垒墙夯筑,弄得到处是尘土飞扬,风一卷就是漫天灰。但今天的情况特别恶劣,平日里,风再大也不会有这么多灰土。韩冈估摸着,多半是从横山对面的瀚海中刮来的沙尘。

浑浊的天空下,韩冈与种建中在凝固的无定河边并辔而行,从他们的身侧,一彪上千人的军队沉默的在风沙中迤逦南行,中间还护送着四五十辆马车,车篷之中躺满了伤病。

种建中望着被染做昏黄的天空,侧过头对韩冈道:“这些风沙都是从北面来的,翻过了横山灰土落得还是这么厉害,多半瀚海那里起了狂风。运气好的话,能让西贼耽搁上三五天的时间。”

“的确是有些运气。”韩冈点着头,“从时间上算,西贼此时的确当是在瀚海中。”

不知天文、不知地理,不可为将。种建中出身将门世家,天文地理方面的水平都很高的水准。古代的天文其实有一半是气象学的成分。种建中说得并不差,韩冈也是这么想的。今天的这场沙尘暴也许还不及后世韩冈见识过的威力,但一想到在无遮无挡的七百里瀚海中行军的西夏人,也算是有点运气了。

不过,前几天韩冈还在想风向要变了,可老天爷兵不是很给他面子。但两三天的耽搁,不至于能把不利于大宋的局势扭转过来,西夏人哪年没经历过风沙洗礼,除了耽搁一点时间,却不会影响到他们的战斗力。

而种建中也不会去奢望西夏的铁鹞子、步跋子能因为一场沙尘而有何损伤,单是能拖延一下党项人的队伍,就已经让他喜出望外了,“多了两三天的时间,罗兀城也会更加稳固,其他几座城寨也当能及时完工,就算是抚宁堡,也当是能把外围城墙给修得差不多。”

“一军分作两地,绥德、罗兀远隔数十里,位于中段的抚宁堡当是重中之重。若有疏失,罗兀城必然难保。”

种建中摇头轻笑两声:“玉昆还是这么爱操心,放心好了,这点如何会不提防。”

一边说着话,一边驱马前行。不知走了多久,身侧传来的脚步声突然稀落起来,一千多南行的队列已经从韩冈和种建中两人身边全部超越了过去,出城时韩、种二人尚在队头,现在却已经落到了队尾。

韩冈就此勒停了坐骑,对着种建中道:“此间到绥德不过是几十里的路程,彝叔兄用不着送得太远。”

“玉昆一路小心。”

种建中也是爽快人,哈哈一笑就跟韩冈拱手告别。

正月廿五,离开攻下罗兀城的日子,已经过去了二十多天。从进城到离城,韩冈也在罗兀待了快半个月。今次种谔意欲南返,他便得许当先离开罗兀。韩冈是作为管勾伤病事来到罗兀城,当罗兀城中的伤病员都要转移回绥德的时候,他也就顺理成章的随队回绥德去。

第一批的七十人前几天已经走了,韩冈今天所在的这一批,也就是最后的一批。而以护送伤病回绥德的名义,种谔一口气派出了三个指挥。这就有点像是蚂蚁搬家,在不惊动到其他士卒的基础上,一点点的把五千人调回去。而等到罗兀城的城防大体完工的时候,种谔也将以护送完成任务的民伕的借口,率部回返绥德。

同意种谔率部回返绥德的公文,是昨天刚刚送来的。从前日听到河东败阵后,种谔就即刻上书延州,通过四天的公文往来,与延州取得了联系,并最终得到了韩绛的认可。

韩冈有些恶意的揣测着韩绛在点头同意前,究竟经过了多少复杂的思想斗争。至少可以确定,长安城里的司马光,必然有几分幸灾乐祸的态度。

司马光前段时间的三本奏章,一本批评河湟开边是生事;一本拒绝在长安增修城防,同时反对增加环庆路的南部重镇邠州的兵力;最后一本便是对韩绛、种谔的横山战略横加指责。即便司马光的德行高致,人品出众,也少不得会向人展示一下他的先见之明。

河东军的败阵丢人现眼,而直接导致这次惨败的韩绛当然也是脱不了干系,而韩绛允许种谔在大战前回镇绥德,更是证明了韩绛和他的宣抚司刚刚经历了一次大挫。许多事先反对今次战事的官员,心中的得意也是显而易见。

但不管怎么说,韩绛终究没有因为面子问题,而硬逼种谔留在罗兀,这点是值得赞赏的。虽然这其中,必然有着担心绥德失陷的因素存在——罗兀代表对横山进取的态度,而绥德却是整个横山战略的根基,在战略中的地位,还是有着很大区别——可是能够把面子放在一边,闻过即改,在身居高位的文臣之中,也是不多见的素质。

而在这等待延州回书的四天里,以罗兀为主的城寨修筑工程陡然加速。韩冈能看见的罗兀城和永乐川两处,城墙都是一天一个样,在收到回信的正月廿四的那一天,永乐川寨周长两百多步的城墙已经先一步宣告完工,而罗兀城的墙体也已经升到了平均两丈三四的高度上,总工程量,离完工还剩下四分之一。

但这几天,由于监工们加紧催逼,就算没有明着公布出来,罗兀城内的士兵和民伕都是知道情况有些不对了,不过尚没有人传出河东军失败的消息,仅仅是有流言说,西贼的大军即将抵达罗兀。

在这种情况下,种谔领军回师绥德,对军心士气的负面影响不言而喻。尚幸他只选择了带走五千兵,只占了整个罗兀防线的总兵力四分之一的数量。

韩冈心想,这种程度的兵力减少,让城中士卒们心底的惶惑,还不至于扩大到爆发出来的地步。种谔作为一名宿将,他对军心的拿捏和控制至少还是靠谱的。

在三个指挥的精锐军队的护送下,韩冈离开了罗兀城,两天的行程中,并没有什么太大的波折,很顺利的抵达了绥德。

韩冈在绥德城中的居所,则是被安排在城衙中的一间偏院里。边境军城的城衙一般都是作为要塞来修建,外墙高厚如小城,占地面积更是广大。韩冈身边才几个人,也照样能占一间偏院居住。周南跟着韩冈来到绥德,当韩冈继续北上罗兀的时候,她便被留了下来——罗兀城那里算是临战前的军中,不方便带家眷过去。

韩冈随军回返的动静不小,周南很快就得到了消息,自己在守在小院中坐立不安,虽不便走到门前张望,但还是让钱明亮去前面打探。

到了近晚的时候,韩冈处理完手上的一应琐事,安顿好伤病,终于回到小院中。这十几天的分离,周南的形容有些憔悴,但见到韩冈回来,却登时容光焕发起来。

洗去了满身的风尘,换了一身干净的衣服,神清气爽的韩冈在内间坐下来。摇摇晃晃的灯光下,桌面上摆着几盘周南亲手做的小菜,一支银壶就放在碗碟边。周南和墨文在桌边守着,家庭中的温暖气氛,让韩冈奔波劳碌的心顿时平静了下来。

他搂过周南,抬手捏了捏她变得尖削起来的下巴,怜惜的问道:“瘦了不少,有没有好好吃饭?!”

周南娇软无力的靠在韩冈怀里,很轻声:“有。”

墨文却在旁边道:“姐姐这些天可都是没吃好,一直在念佛。”

“这样可不好!饿坏了身子可不好,以后可别这样了。”

周南像个小女孩一样,很安静的老老实实听话点头。

韩冈笑了,周南越是娇弱,他的心头就越发的火热起来。他一抬手,抓着周南肩头上的衣襟稍稍用力,半边浑圆白皙的丰润登时暴露在灯光下。一轮细小如钱的红晕中,红玛瑙一般的凸起轻轻的颤动着。韩冈张开手一把握上去,白皙的嫩肉在指缝中挤了出来,“还好这里没有瘦下去。”

韩冈的动作,让旁边的墨文惊叫一声,忙捂着眼逃开。

周南却不管那么多,翻过身,玉藕般的双臂,用力搂住了韩冈的脖子,在耳边呵气如兰:“官人,要我……”

刚刚尝过欢愉滋味的少女分外痴缠,韩冈也是忍耐了许久,也不顾着酒菜就在桌上,抱起她就向床边走去。

白天在绥德城中的一处营地设立的疗养院里,处理一下公务,夜中又有体贴可人的周南尽心侍奉,在种谔回来前的这几日,韩冈过得到是惬意自在,丝毫没有被城中越发紧绷起来的局势所影响。

二月初二,所谓龙抬头的日子,留下了高永能驻守已经大体完工的罗兀城,种谔终于率领最后的本部亲兵,护着结束了任务的数千民伕回返绥德。而与此同时,当朝首相,陕西河东宣抚使韩绛的车驾,也一并抵达了绥德城中。

