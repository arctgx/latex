\section{第48章 一揖而别独骑归(上)}

自从由边地军寨改为一州治所之后,陇西城中就开始在道路两旁遍植树木。

行道树是一座城市的重要组成部分,根据各地水土气温而有所不同。

中原和南方的城市多是柳树、榆树,有时还会有桃树、杏树,而关中以京兆府的州县,则多用槐树,或是杨树。陇西位于渭水之滨,可河道并不经过城中,只有几条从渭水引来的水渠穿城而过,当然没有柳树出场的余地,而跟所有关西城市一般,以槐、杨为主。

只是行道树种下不过两年多的时间,长势再好的树木,也不过是小腿粗细,一点树荫,只比手中油纸伞差不多一样大小,对于在夏日中奔忙的人们来说,也是杯水车薪的感觉。

位于州衙左近的韩府门前,地面也是被盛夏的阳光照得散出明晃晃的白光。从地表反射上来的热量,使得

一名儒生打扮的中年人,束手立于太阳底下,而他的随行伴当,则是上前敲响了韩家的大门。

门环啪啪的被拍响了好几下,正门没开,但侧面的一扇小门被打开了。

从司阍人住的门房中,走出来一个三十多岁的汉子。只有一条左腿,右腿上及膝而断,装上了一只木腿。木腿打着地面,哒哒的响着,走起路来步履艰难。但这司阍的行动举止中,却不脱精悍,一看便知是在遍地血腥的战场上,行走过多年的军汉。

陇西城中人人都知道,韩冈管勾一路伤病事,家里的仆役有不少是难以恢复的伤兵——也不仅仅是韩冈,如今的世情,只要领过军的官员,多有将用得顺手的兵丁脱了军籍,收录入自家府中——只是像韩府一般,用的多是残疾的,却是很少见。

这个木腿汉子自然就是韩府司阍。他拐着脚走到中年儒生面前,儒生的伴当便立刻递上一份门贴。

韩府司阍接下门贴,只一拱手,“官人的名帖,小人代为收下了。但我家机宜现今闭门谢客,还请过些日子再来。”

儒生伴当对此是早有预见,高官显宦家的门房刁难地位不高的陌生访客,也是常见的事。他卑笑着上前,下面递出来一锭一两多重的小银锭:“这位大哥……”

还没将惯常的话说完,韩家的司阍就连忙推辞,死活也不敢收下递到手边的银钱:

“这位官人,不是小人有心刁难,实在是我家机宜已经辞了差遣,准备明年的科举,正闭门读书,根本不见外客的。还望官人能体谅小人!”

司阍鞠躬作揖,姿态放得极低。中年儒生看了他一阵,也是没办法,只能叹了一口气,悻悻然的离开。

目送来人远去,司阍的老兵踩着木腿哒哒的击地声,一拐一拐的回到了门房之中。啪的一声小门关起,韩府门前重又恢复了平静。

韩冈现在是炙手可热的红人,若不是挂上了闭门谢客的牌子,家里的门槛,三五天内就会被访客踏平。

现在的韩冈,因为锁厅的缘故,身上的差遣都卸掉了。他参加举试的结果不论是中与不中,韩冈现在丢下的职位,都不会给他留着。本来就是僧多粥少的局面,不可能为了韩冈一人,而将巩州通判、经略司机宜这样的重要职位,空留上近一年的时间。

不过韩冈的本官,已经是从七品的国子监博士。如果他不是没有一个进士出身,本官应该是太常寺博士——在进士远多于非进士的朝官行列中,国子监博士的数目,远比太常博士要少得多。可不论是不是进士,韩冈现在的品级,已经比当年韩冈刚刚投入王韶门下的时候,还要高出数级。

跟韩冈一样,韩冈的父亲韩千六,官名韩谦益的熙河屯田管勾,现在也已经是熙河路中排得上号的官员。有着身后浑家的指点,韩千六在巩州民间的声望并不低,在官场上,有着韩冈这个儿子,也没人敢给他脸色看。而他所主导的棉田推广种植计划,更是被来自秦州的一众豪族日夜记挂在心里。

熙河一路的各家蕃部,韩冈靠着疗养院救治了不少蕃部中的重要人物,多多少少都有些香火之情。一同征战的广锐军,自刘源以下,都是韩冈的亲近从属。他的一句话,比起熙河经略、巩州知州,都管用得多。

而韩冈表弟冯从义执掌的顺丰行,由韩冈决定的细水长流的策略,商行出让了一部分利益给来往的蕃部,使得顺丰行成了熙河蕃部对外交易的代理人的首选。不再仅仅是熙河一路最大的商行之一,而是已经成长为在秦凤地区有着很大影响力的商行。

现在论起势力,韩家已经在巩州稳稳扎下根来。如果再有一代人的时间,使得韩家人丁再充足一点,就是一个稳当当的地方豪族。日后凭着与蕃部的关系,以及在地方上的势力,不需要什么辛苦,轻而易举就能让子弟进入官场之中,控制这一州之地。

不过现在,韩冈还得为着一个进士而刻苦用心。只是他今天预定的学习计划,却还是被一个不能拒之门外的客人所打扰。

“天子在紫宸殿接受百官朝贺。”王厚在韩冈面前,重复着前两日刚刚说过的故事,“王相公佩御赐玉带而上,亲为天子捧觞。”

为了庆祝河湟功成,京中的朝贺大典,韩冈早就听说了。实质上不过是奉承天子的把戏而已,跟自己无关,跟王韶也无关。虽然站在紫宸殿上,从头看到尾的王厚说的口沫横飞:“只是家严和玉昆你都没有能参加,实在是可惜了。”

但韩冈还是没什么兴趣,岔开了话题:“大典不过是个仪式而已,学士入朝之后,必然能得大用。”

揽稀世之功,王韶入朝已成定局。六月时他馆职尚为端明殿学士,七月朝贺大典之后,就换成了更高一级的资政殿学士,而十天前,他又更进一步,晋为了观文殿学士。

通常来说,观文殿学士只会授予离任的执政,是诸殿学士中的最高一级,而宰相去职后,就是会改授观文殿大学士。现在王韶得受观文殿学士,是大宋立国以来的第一遭,也代表了王韶进京后,便会成为宰执中的一员。枢密院中,继新近入朝的泾原经略蔡挺之后,又将迎来另一位枢密副使。

——“定然不会逊于蔡子政!”

王厚哈哈笑着,故作谦虚:“还不知道呢!”

从尚未入流的选人到一国执政,只用了不到五年的时间。而从担任缘边安抚使时的著作佐郎,到现在的谏议大夫,更是只有两年。王韶的这个晋升速度,甚至不比当年宣抚陕西的韩琦稍逊!

而且凭着今次的军功,还有在西军中的威望,以及边事的发言权,日后枢密使,甚至宰相,王韶都是有机会问鼎的。

王厚正是知道此事,从京中回来后的这些日子,心情才分外得好。如果有着一个宰相的父亲,日后从武将转为文资,就不会受到什么刁难了。以他现在的官品,转为文资后,日后坐镇边陲也一样都是有机会的。

他看了看韩冈摆满案头上书卷:“如果今次玉昆你能与家严一起上京,觐见天子之后,一个进士出身有何难?”

“可能吗?学士是这般说的?”韩冈摇着头,“一个贡生资格还差不多。”

王厚笑了笑,他也知道得赐进士不是那么容易,并不是天子想赐就能赐的。开疆拓土比不上一个状元及第;边功虽多,也赶不上一个进士出身。世风如此,不是人力能扭转。

“赐个贡生,那也省了一次考试了。”

“锁厅试而已,省不省都是一样。”

要是能得赐一个进士,韩冈他保管就去京城了。就算惹人议论,他也不会在乎,他要的本就是一个资格,而不是跨马游街、金明赐宴的荣耀!但若只给一个贡生,他何苦去丢这个脸,在秦凤路这边他轻轻松松就能考到手。

王厚感叹道:“也只有玉昆你能这般放言。要是挑女婿,也是先找玉昆你这样的。”

韩冈不说话了,开始盯着王厚。关于秦凤锁厅的好处,他和王厚两人都是心知肚明,早就一起分析过的。只要稍有才学,从秦凤路脱颖而出实在是容易得紧。就是王厚,只要努力两三个月,也照样能过关。跑来说这些车轱辘话,难道是今天闲得慌?

盯得王厚神色变得越来越不自在的时候,韩冈才又开口“……处道兄,你今天来找小弟,不会是来跟小弟说这些话的吧?”

王厚怔了一阵,苦笑的摇摇头,“就知道瞒不过玉昆你……其实小弟今次押送木征上京,受王相公所托,给家严带了一封信回来,不过信中的内容,却是关于玉昆你的。”

韩冈心头有了一点不安的预感,问道:“是什么?”

王厚坐得凑近了一点,低声问着韩冈:“只是想问问玉昆你,想不想做宰相家的女婿。”

