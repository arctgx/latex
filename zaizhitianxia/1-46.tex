\section{第22章 声入云霄息烽烟(下)}

摇头挥去满心杂念,韩冈将自己从失落和混乱中拔了出来。长时间默不作声的行军,让队伍里的空气变得充满了压抑,连自己这样意志坚定的性格都受了影响,其他人的情况恐怕更是不妙。

如果在行军中说说话,唱唱歌,这种沉郁的气氛应该很容易就能打破。但行进在危机四伏的谷地中,两侧的山谷中不知隐藏了多少杀机,韩冈和王舜臣的神经都绷到了极点。带队首领的紧张理所当然的感染到了全队身上,让所有人都提心吊胆。

脚下的官道转过了一个角度,原本挡在视线前的山壁退了开去。一条星河在前方的地平线上浮现,突兀的映入众人眼帘。星河黯淡,摇晃着似有似无,唯有一点最为炫目。韩冈不禁眯起眼睛,定睛再看,才发现那不是星辰,而是一座城寨上亮起的火光。

深深的吸气,将接近冰点的空气吸入肺中。从体内泛出的冰寒让韩冈精神振奋,悲观刹那间让位于现实。

那是甘谷城!

数百支火炬将城墙的上缘从黑暗中勾勒出来,星星点点的光明无法照亮夜空,却照入了韩冈一众的心中。就算甘谷城告急的烽火是燃于城头上的星光中最为灿烂的一颗,他们也没放在心上,那至少还代表着甘谷城依然在宋人的手中。

“是甘谷城!”队列中响起了一阵低低的欢呼声。“终于到了!终于到了!”

虽然至少还有近十里的距离,但目标就在视线范围内的感觉,让人人兴奋不已。不待韩冈催促,个个挥鞭驾骡,将车子赶得更快了三分。

“不对!”王舜臣忽然靠了过来,声音里透着紧张:“三哥,情形不对啊。”

“怎么了?”在韩冈的记忆里,一向大胆的王舜臣很少有声音发颤的时候,一股不祥的预感出现在心中,“出了什么……见鬼!”

韩冈话到一半突然就停住了,改而爆出一声咒骂。就在官道左侧的山坡上,隐隐约约的能看到一团团黑影如同幽魂一般不知从何处冒了出来,无数碎乱的脚步声,在几个呼吸间就连成了一片。

山坡上影影绰绰,细细碎碎的声音不断从上面传来。不知聚集了多少蕃人,多少弓刀枪剑。坡上的黄土被千百只脚反复踩过,崩塌的土石哗啦哗啦的落了官道满地。

“是心波三族的蕃狗!”王舜臣厉声喝叫,充满了怒意。

对,只会是心波三族的蕃人!如果能跟着党项人一起杀入富庶的秦州,他们也能过上个肥年。心波三族不是小部族,不需要担心会被拿去当鸡杀给猴儿看。他们汇合起来的总兵力超过四千,足以让秦凤经略司投鼠忌器。他们的行事,也便一贯的肆无忌惮,只有在甘谷筑城后,方才消停下来。对心波三族来说,甘谷城就是套在脖子上的枷锁,如果能打破,必定是乐见其成。

甘谷城头的烽火依旧熊熊燃烧,但在韩冈一行的心目中,那已不再是即将抵达目的地的信号。烽火所传达的真意,他们已经用切身体会明白了过来。

“三哥,快点把火炬都熄掉!”王舜臣急急叫道。既然能直接看到甘谷城,前面的路就不会太曲折。就算没有亮光,小心点也是能走的。下方忽然一团黑暗,山坡上的贼人应该不敢下来。

韩冈没有听从王舜臣的劝告,反而反道而行,他喝令全队:“大张火炬!每人都给我拿上两支,车子上也给我插上去!越多越好!”

“三哥,人太少,吓不住的!”王舜臣的声音更为焦急,总共才三十多人啊。青蛙再怎么鼓气,也鼓不到牛那样的大小。

“谁耐烦吓他们?”韩冈厉声喝道:“我是要让甘谷城看见!”

心波三族没有反叛,否则他们现在就应该攻打甘谷城去了!他们仍然是在观望!韩冈很确信这一点。只要甘谷城还没丢,这些蕃贼就得顾忌着日后。他让所有人多多点起火炬,就是要让甘谷城的守军知道有人从伏羌城那边过来了。

甘谷城会不会援军出来接应?能不能在援军接应前解决这只胆大包天的车队?心波三族的主事者想得越多,就越不敢下来搏上一搏。而他们越是犹豫,车队离就越近;等到他们下定决心,说不定自己的一行车队已经走到甘谷城门下了。

官道上,原本才三十多支稀稀落落的火炬,转眼间就变成了上百具。拉成长条的队列,看起来很有一番声势。正如韩冈所料,山坡上的蕃贼果然没有下来,他们在观望着,盘算着。而辎重车队却在他们的犹豫中不断向前。

一步步的走着,韩冈荒谬的想起了过去看过的电影。在许多无聊的电影中,都能看到主角从交叉的刀枪组成的通道中走过的情节。他现在就是感觉自己仿佛成了无聊电影中的主角,顶着头上的雪亮刀光往前走去。不过在那些电影中,主角都是顺顺利利的通过了刀枪阵,只不知自家今次能不能如此顺利。

“秀才公……”朱中凑了过来,为斩首的死囚缝脑袋的裁缝学徒也承受不了眼下虎狼环绕的压力,声音发着颤。他也不知要问些什么,说些什么。就只想听到韩冈说句话,好给自己和同伴带来一点勇气。

“走!看着前面!继续往前走!他们不敢下来!”

韩冈的意志毫不动摇,声音坚定如钢。此时只能进不能退,狼群在外窥伺,只要稍稍露怯,它们就会扑将上来,将自己撕成碎片。

瞄着远处甘谷城的灯火,刻意不去理会身边的贼人,韩冈领着他的队伍深一步浅一步的向前移动。甘谷城的烽火火焰冲霄,告急的黄色火光却成了辎重车队在猛兽环伺的黑夜中最为温暖的救赎。

可谁也没有想到,就在下一刻,那团最为浓烈的火焰在几下短促的闪动之后,刹那间消失得无影无踪,若不是在人们的视网膜上还留下了一点印迹,甘谷城报急的烽火就仿佛从来没有出现过一样。

烽火熄灭只有两个原因,一个是胜利,一个是沦陷。究竟是哪一个?韩冈给不出答案,但山坡上的蕃贼自己已得出了结论。

一瞬间,山坡上的暗影中一齐鼓噪了起来。无数身影一阵摇晃,一个两个接二连三的向下方移动。

哗啦啦的落石让车队中一片慌乱,数只拉车的骡子仰脖嘶鸣。

“不要慌!”韩冈一声怒吼,没有时间再考虑甘谷城中的命运,“所有人都围过来!张开弓,听我的号令!”

韩冈令行禁止,聚在一处后,民伕们都半开着弓,竖起耳朵静待他的号令。但下一刻,传入他们耳中的不是开战的命令,而一阵雄壮豪放,远远的仿佛是从天际飘来的歌声:

丈夫气力全,

一个拟当千。

猛气冲心出,

视死亦如眠。

如同在和应,数里外的城寨中,一阵欢呼声同时响起。千百人的欢声,惊动了天地。而欢呼声中,让人熟悉的旋律交织缠绕。

“是得胜歌!”

“是张都监回来了!”

这是关西男儿得胜归来的歌声。多少年来,匈奴、西羌、突厥、吐蕃,一代代的关西男儿为了抵御层出不穷的鞑虏蛮夷的侵袭,高唱着军歌走上战场。而后又提着敌人的首级,踏着月色,高唱凯歌得胜归来。

“丈夫气力全,一个拟当千。猛气冲心出,视死亦如眠。”

得胜歌声出自于千百人之口,越过数里的距离,飘扬自天际,其中的兴奋,韩冈一众听得分明。

“率率不离手,恒日在阵前。”

数千人的合唱声震天地,直入云霄。

“譬如鹘打雁。左右悉皆穿!”

不知何时,王舜臣也加入了合唱的行列。他高声唱着,吼着。抬起手,张开弓,一支响箭直蹿山壁之上。黑暗中传来一声短促的惨叫,转眼便被歌声淹没。

面对小小的一支辎重队的挑衅,心怀悖逆的蕃人也许并不甘心,但在得胜归来的大军眼前,他们终究还是没有那个胆子,终于选择了退却。僵持了一阵后,淅淅索索的声音再次响起,只是越来越小,重重黑影复又隐入黑暗之中,很快便一点不剩。

一切恢复了一刻钟前的状态,只多了反复唱响的嘹亮歌声环绕着空中,充斥在谷地:

丈夫气力全,

一个拟当千。

猛气冲心出,

视死亦如眠。

率率不离手,

恒日在阵前。

譬如鹘打雁,

左右悉皆穿!【注1】

歌声中,韩冈放声大笑,多时的紧张、满腔的心绪化作一声长啸倾泻而出,他大吼:“走!去甘谷!”

用词一如早前,心情已然不同。

注1:按照沈括在《梦溪笔谈》中记载,西军得胜后都会高唱凯歌而还,所以写了这一段。但因为找不到合适的军歌,沈括此时也还没到关西来任官,只能用敦煌曲子词来凑数,建议大家可以去找来听一听。

ps:敦煌曲子词,盛唐时流传于西北,被藏于敦煌莫高窟中。现在已经按照原谱复原的曲子,网上便能找到。很不错的歌曲。

今天第一更,征集红票和收藏。

