\section{第十章 却惭横刀问戎昭(18)}

【不能再一天一更了,要发奋努力。】

韩冈觉得五千到一万应该差不多是两边的极限了,正如折可适的推测,多和少的可能性都不大。

“当如经略所言。”

折可适双眼低垂,隐藏起眼中一闪而过的惊异。韩冈的判断虽然出自于自己,但就在听了分析之后,转瞬间便得到了答案,可见他对军事了解之深,正所谓盛名之下、固无虚士。

“会……会不会是阻卜人私下里潜来助阵,只是得了西贼的收买,并没有得到耶律乙辛的许可?”黄裳的质疑一开始有点缺乏自信,但他看了折可适一眼之后,气势却莫名其妙的涨了起来:“由此一来,那一伙贼寇的兵力人数也能解释得清,藏头缩尾的原因也就找到了。”

“这个可能不能排除……要是真是这个原因,那就太好了。”韩冈笑了一笑,黄裳似乎是在针对折可适,这文人对武人的鄙视看起来几乎都是根深蒂固了,“……不过,事情还是得往坏处准备。至少这样不管怎么变化,情况都不会变得更糟,若是反过来可就不妙了。”

黄裳一时沉默了下去。

“那些阻卜部族没有那么大的胆子!”折可适则是三分不满、七分冷淡的瞥了黄裳一眼,若非他是韩冈的幕僚,有什么资格在这件事上插嘴?黄裳是韩冈的幕僚,的确得给他留三分颜面,但折可适却是忍不下来,小事倒也罢了,这等军国重事岂能由得书呆子说嘴:“如果仅仅是几个阻卜的小部族私自出兵相助,他们是不会过来攻打紧邻西京道的丰州。否则一旦事发,他们只有死路一条。只可能是耶律乙辛,那窃国老贼为人奸狡,阻卜人的出现,不过是他用来讹诈朝廷的手段。”

韩冈一见折可适针锋相对,心中就叹了一声,这折可适还真是年轻气盛。正想打个圆场,却见黄裳拱了拱手,向折可适低头道:“黄裳受教了。”

黄裳诚恳受教的态度,很有风度,但从折可适的角度来看,却等于是给他添堵。不过世家出身的折七郎还是很擅长应对这样的场面,立刻回了一礼,“还要多谢秀才的指点,指明了在下的疏漏之处。”

两人一人一句,三两句话的功夫,就仿佛化解了两人之间的芥蒂,言笑甚欢起来。

韩冈一切都看在眼里,很有几分欣赏他们这样的作派,要是他们针尖对麦芒的斗起来,那就让人失望了,幸好不是这样。不过也该打住了,韩冈没有时间陪两位闲人谈天,见折可适,只是要听一听丰州屠村之案的详情的,不是征询折可适的意见。

关于目前西北将三个国家都牵扯进来的战局,议论得太多就过头了,这不是折可适、黄裳有资格掺和的话题。即便韩冈想对此做出些应对,应该是召集河东路经略司的主要官员和将领们来议事,然后向朝廷建议。集众人之智才是正途,可不是随便找两个人议论几句,拍拍脑袋就下决断的。

招了属吏进来,点了汤。折可适喝过饮子之后,识趣的起身告辞。此是为点汤送客。

折可适离开,韩冈啜着温热的香薷饮,皱着眉想着眼下的局面。黄裳不敢打扰,静静的守在一边。

目前还不知道朝廷那里对于阻卜人的出现是什么样的反应,这是肯定要写奏章上报的,甚至还得在奏章中请罪——丰州、麟州都是河东治下,被屠了村,韩冈难辞其咎。

抬起眼,吩咐黄裳道:“勉仲,你帮我拟一份请罪表,丰、麟两州的事,我总得给个交代。”

此乃应有之理,黄裳没有多话,站起身,道了一声是,却是去内厅找笔墨写表章去了。

只剩韩冈一个,一直保持在脸上的沉稳微笑,终于维系不下去了。咬着牙,从牙缝里迸着声音:“吕惠卿!徐禧!”

之前确认了徐禧要镇守盐州之后,韩冈不顾自己仅仅是河东经略而不是任官陕西,写了劝谏的奏章上去,希望还能来得及挽回局面。但眼下阻卜人既然出现,韩冈明白,局势如同破堤的奔流洪水,已经不是区区几个沙包就能堵上了。

要不是吕惠卿和徐禧贪功,根本就不用为区区几千阻卜人而担惊受怕。韩冈甚至不担心辽人出兵帮助西夏攻打夏州、银州——党项人的后勤体系根本支撑不住太多的兵力。而且补给线越长,中间受到攻击的可能就越大。

当双方战力相差不大时,后勤决定一切。将战争的关键点放在盐州,等于是自曝其短。不过这时候后悔也罢,抱怨也罢,一点意义都没有。

韩冈和折可适对阻卜人的推断,等种谔发现他们之后,必然能做出同样的判断。但种谔敢不敢冒险?他又能不能说服下面的将校冒险?韩冈对种谔没有一点把握。

之前契丹人试图劫掠西陉寨外围村寨的时候,守在雁门寨的宋军如果有胆量,有实力,完全可以出寨迎战,堂堂正正的将契丹人的野心给砸烂,谅远在朔州城的萧十三也救援不及。可惜就是韩冈都不能下这个命令,他对河东的兵马没有信心,下面的人恐怕也不敢依从。

……对了,想到这里。韩冈突然惊觉,阻卜人南下的消息,种谔到底知道还是不知道?万一还没有撞上,就还来得及让他们做好准备。得赶快遣人去通知鄜延路,就不知道还能不能来得及。另外,发给京城的军情急报,也要尽快写好发出去,不能再耽搁了。

但回到眼下的战局上,却只能暂且先看看后续。‘这一战的关键或许还得回到银州、夏州上。’韩冈想着。

给京城的请罪表和军情急报同时发出去了,提醒种谔的急件,也通过马递发往鄜延。此外昭告河东西侧缘边各军州做好防范,韩冈也同时安排了下去。

几件事一办,一时间,韩冈似乎就清闲了起来,连着两天,政务军情上都没有什么大事发生。

尽管手上的事情依然很多,但经过了之前几个月的忙碌,韩冈的工作算是上了正轨,只不过他之前的精力有九成偏向了军事方面。太原府的政务,却还有许多地方亟需他关注。

秋税就不用说了,今年的冬播也要开始准备——关键的是要将人力合理分配和调遣。一两个月之后,也就是小麦种植开始的时候,战争很有可能进入最为激烈的环节,那时候,河东要调动大量的民夫。不再眼前做好准备,到时候,可就麻烦了。万一明年太原的税赋大减收,第一个忍不住秋后算帐的甚至有可能是天子。

但民夫的使用,是免不了的。转运粮草,在南方有船的情况下的确是不难,但四方的道路总在山中打转的太原,却只能依靠人力。每一次河东临战,总会有民夫逃亡、或是阖家远走的情况。而在转运的道路上,更多的被征集而来的民夫,每天都要挣扎在死亡线上。

韩冈不想这样驱用民夫,效率实在是太低了。他也在考虑着怎么让尽可能少的征发民夫作为辅助。眼下就有现成的办法——轨道。

河东是山区加盆地的地形,轨道想要在这里铺设起来,达到贯通南北的目的,韩冈不指望能在十年内成功。但如果是在盆地中铺设轨道,然后在山区则是利用旧有的山路,这样一来,为了后勤转运而征集来的民夫,就可以集中在几段山区,能节省大量的人力畜力。

不过韩冈想想就放弃了,这样的轨道,只能军用,在民用上成本就太高了,无法用商业收入来回补。而且想要修造长距离的轨道,至少要一年以上的时间进行先期勘察,确定路线,将预算方案做好。河北轨道到现在还没有什么动静,一方面是受到战争的影响,另一方面,也是先期的勘察还没有完成的缘故。却没办法在这一场战争上派上用场,等到战后再修造轨道,那还不如费点时间,连同山路一起设法铺设起来。

在河北轨道还没有成功的情况下,缺乏足够的经验和人才,河东轨道的事,只能暂且先放到一边。韩冈现在在政务上,除了一名知府应尽的义务,另外还有心关注一下河东的煤和铁。

粮食产量是要受土地数量约束的,一时无法改变。但原始的工业,情况却要好很多。

钢铁是工业化的关键,韩冈希望大宋的十几个路,都能有一个煤钢联合体的出现,至少在几个大区域上,有足够多的钢铁产出——这还是很有希望的,后世年产万吨的钢铁厂,是关停并转的目标,但在眼下,就是一个国家一年的产量。区区一个万吨级的煤钢联合体,矿石和煤炭的需求量都不高,大宋的东南西北,基本上都能找到合适的地方。

山西是浮在煤田上的。后世韩冈不止一次的听过这句话。而在听到朔州这个地名,韩冈就想起了后世的平朔露天煤矿。可惜朔州眼下在辽人那里。大同的火山火坑,韩冈在太原这里听说过几次,许多人当成是奇闻异事。煤层自燃的现象,证明了大同附近也有露天的煤炭矿藏,可惜那也是被辽国占据的地方。但在河东这里,还是有煤有铁的,也早有了生产,尽管规模不大,不过拓展起来也并不难。

韩冈希望在他离开河东的时候,能留下一个足够大的钢铁工场。钱多了那是肥羊,而钢铁多了,却是震慑周边国家的武器。

就在韩冈命人搜集河东煤矿铁矿的资料的时候,李宪重又回到了太原。

