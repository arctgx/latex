\section{第14章 霜蹄追风尝随骠(12)}

【昨天有事耽搁了,不好意思。一点之前还有一更。】

韩冈抵达旧丰州并没有引起太多的波澜。

无论是从整合河东、麟府两军的角度,还是从尽早解决丰州之事的角度,他将帅府行辕移驻旧丰州都是必然的选择。

不过现在也不能叫旧丰州了,天子降诏,朝廷移文,旧丰州改名胜州。这也是应有之义,要是此地复旧名,当初从府州划分出来的今丰州就不好安排了。而胜州亦为旧有,恢复此地古称,放在哪里都能说得过去。

午后时分,韩冈就在城头上望风景。远山近水全都被一层新雪所覆盖,白茫茫的一片,其实也没什么景色好看。

但这是韩冈的习惯。每到一处新地,总要先上到城头上看一看,了解一下城池周围的地理,顺便做些笔记,以待日后有用。

旧丰州,也即是胜州的城垣并不高峻,两丈出头而已,而且跟低矮的城头相配合的,墙垣也单薄的很。站在城头上,韩冈估算着需要下多少工夫便能攻破这座城池,结果很是让人忧心——就是以西夏的攻城水平也不要太费神,不用说契丹人了。

重修胜州城,是迫在眉睫的一件事。不过韩冈现在手上钱粮有限,人力有限,时间更有限,而要应对的寨防工程量却大得惊人。三人份量的食材,却要做出十人份的饭菜,这就是摆在韩冈面前的难题。但不管怎么说,旧丰州都是排在第一位的。

“依照龙图之前的要求,胜州城的重修,折府州给出了上中下三策……”黄裳在韩冈身后说道。

“我方才已经看过了。”韩冈道。

折克行的高中低三套方案。高方案是以顶级的边防重镇标准来修建,韩冈是不打算用的,“所谓的上策,钱粮耗用太多,民夫需要也多,如此修丰……胜州,其他城寨就不要修了,只能是算作未来的目标。折遵道【折克行字】也只是拿来凑数而已。能动用的钱粮和人力的预算也给出了,具体修造的方略,只能在中下两策之中择优取舍。”

“龙图觉得何策为佳?”黄裳问道。

韩冈转回身,对黄裳道:“这事交给你,可以召集众将来同议,定下来后上报给我。”

黄裳愣了一下后,心下顿时大喜,韩冈将这个任务交给他,等于是将他从经略使身边的门客,转为拥有实际职司的幕职官的前兆,哪里还有不愿意的道理,“黄裳明白!”下定决心,要将此事办妥当了。

听见黄裳答应时的声调都变了,折可适在旁一笑,并没有嫉妒之心。看看旧丰州重修乃至整个屈野川寨防体系的修造,到底是谁在建策,就知道这个差事本就轮不到他来做。

韩冈远眺北方覆盖在皑皑白雪之下的峰峦叠嶂,半晌后方道:“黑山党项为契丹驱逐南下,此事已经确定。最后南下的人口,预计不会少于两万。其中人心各异,甚至可能会有辽人混迹其中。之前我说过文攻武卫,以攻心为上,教训这群黑山党项转投大宋,不过眼下看来是不够了。你们觉得现在的局面该如何应对?”

韩冈问策,黄裳新得了差事,便不去抢风头。折可适回复道:“胜州一地,大宋得之故有,契丹失之本无。对辽人来说,其实无所得也无所失。眼下兵事不解,乃是萧十三的私心作祟。而可供利用的当也是他的私心。”

韩冈点点头,道理说得没错。

就像耶律乙辛本人的利益与辽国的利益不一致一样,辽国的利益与萧十三私人的利益也并不是一致的。这是很正常的事,个人利益和国家利益之间,总是有差距的。

再如韩冈,他也不可能无私到一切以大宋的利益为依归。只不过韩冈目前最大的利益是在他幕僚那边,尽量完美的解决目前的问题,才能让他幕僚们——也就是气学弟子——陆续得到进入官场的入场券。因此,便与大宋的利益共通。

黄裳接口道:“要利用萧十三的私心,就要了解他到底需要什么,然后做出应对。官军抢先一步夺下了丰州,萧十三颜面大失。最大的希望是挽回颜面,但这其中也担心争斗下去,会损失更多的脸面。”

韩冈呵呵笑了起来:“要做个贴心人可不容易,算计耶律乙辛和萧十三的私心倒也罢了,体贴他们的私心,这不是太难为人吗?”

“龙图误会了!”黄裳连忙道,“跟在龙图身边,又怎会有对辽人委曲求全的想法?!”

“管他是党项人还是辽人,没必要去细细分辨。只要行动间有可疑之处,一概杀光就是了。”折可适更是爽快。

黄裳道:“换做是平日尚不须如此。但放在眼下,还是这样最省事。”

“原来如此。”韩冈点点头。粗暴归粗暴,但却是简洁有效。

不管萧十三有什么想法,只要他还有顾忌,不敢撕破脸皮,派出来的人被杀光了也只能让他有苦说不出。

韩冈对辽人的威胁不是很放在心上,大不了打一仗。耶律乙辛刚刚稳定局面,不可能贸然撕毁澶渊之盟,即便打起来,战事也只会局限于河东一地。依靠山河之险,让萧十三吃个苦头回去,韩冈不觉得有多难。

但辽人并不是最大的问题,最大的问题是在于如何长久的维持住对胜州之地的统治。

这时候,下方的城门口又热闹了起来。一支马队满载着粮草缓缓抵达胜州城。从装束上,只有两成是押运的宋军,剩下的全都是党项人。从他们过来的方向来看,是从府州运粮过来。

麟州、府州的粮道并不算好走。为了维系这两条粮道,共有一千一百多名黑山党项役夫和三千匹其所属的马匹,组成了二十多支运输队在其中奔走。而这些党项人的部族,则是被安排在胜州之南、麟州之西的旷野中,等待家人将糊口的粮食带回来。

终于结束了漫长的行程,队伍中的士兵和蕃人都显得很是放松,心情也很好。从上往下看,两边似乎并没有太大的隔阂,能看到他们之间也有几句话可说。

折可适越过雉堞,也向下望着这群运输者,“看他们心情都不错,这一趟的行程应该是很顺利的样子。”

“这一下能多带一点粮草回去了,越顺利带回去的可就越多。”黄裳转头看韩冈,“也是龙图对入中纳粟化用得好。”

“只是快而已,到库的粮草还是那么多。”韩冈摇摇头。

给付党项人的报酬的比例并不是固定的,而是以正常条件下运送粮草的耗费为标准,确定了应到的数量,然后与出发时装载数量之间的差额,就是给付这群黑山党项运粮的酬劳。能省下多少,就看他们的本事了。走得越快,节省得越多,到手的粮食就越多。这种手段,后世常常用来节省油料。但在此时人眼中,则是类似于入中纳粟的变种。

“利用这个激励的方法,转运的速度快了许多。没有哪家部族不想给自己的家人、族人多带一点过冬的粮食回去,运粮的时候都是尽可能的快。”韩冈低头看着那支运输队通过了城门口的查验,穿过城门,进入城中,“不过如果查看账簿,用在转运上的花费依然不减,中间的损耗大得惊人。”

折可适不确定韩冈的意思,试探的说着:“……快和省,本来就是难以两全吧?”

黄裳则道:“最好的办法还是尽快增加胜州的人口和耕地。河东路地处黄河之西的几个军州,人烟稀少,土地没有开垦,资源也没有开发,自然不会有财税收入。朝廷驻扎大军于此,军饷粮草都要从河东运送,中途的耗费巨大,迟早会惹来议论。只有成为税赋之地,而不是朝廷的负担,才能阻止小人的议论。”

韩冈点点头,“勉仲说得正是。”黄裳正说到他心中去了,不过这也是他一直灌输给气学门下的理念。

尽管胜州地处险要,一城事关河东、银夏两路安危。可总会有鼠目寸光或别有用心之辈,抨击占据胜州是劳民伤财。就像他们当年说夺取河湟,乃是空耗人力的无益之举,全然不管占据战略要地对国家安全有多大的意义。

不过这群人身份不低,明确说,就是一干旧党重臣。这群人话语权和分量都很重,门徒又多,要跟他们辩论,还要提防上面头脑发昏。与其跟他们争论,还不如把丰州和河西诸州建设好,有足够的财税收入来抵偿朝廷军费的支出,用事实回话。

当年拓边河湟,在渭州一石一贯的粮价,到了河湟前线,要付出十倍的运费。不过之后开发了巩州、熙州的粮食生产,以及棉花、棉布、油料、咸鱼等特产。加上雍秦商人形成了庞大的利益集团。使得现如今无人再说熙河路得之无益,是好大喜功之举。

若胜州、包括胜州以南的这一片土地被开发出来,也便不会听到有人说放弃了。

