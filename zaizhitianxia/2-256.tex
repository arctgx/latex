\section{第36章 万众袭远似火焚(12)}

官军高歌猛进是一件好事,可韩冈也发现,一旦战线拉长,对于随军转运使来说,的确是让人很是头疼,同时也会腿疼的一件事。

从马上下来,韩冈双腿都直发颤,大腿内侧火辣辣的,不用看,肯定是皮都磨破了——他这两年没少骑马,大腿内侧都快长起茧子了,可再厚的老茧,也经不住长时间的摩擦,

两天时间,韩冈围着狄道城和珂诺堡,绕了个长达四百里的圈子。河谷走过,山路也走过,连接狄道、珂诺堡的两条路他都走了一遍,连同沿途的寨堡也都巡视了一回。

韩冈不知道其他的随军转运使会不会像他一般勤力,可在眼下出征河州的三万大军中,他走的路应该算是最多的一个。

恍若无事站在马边,跟着上来牵马的马夫随口聊了几句。在马夫诚惶诚恐的回话中,韩冈了解到了这几天来珂诺堡中,骑兵们的出战情况以及战马的出动率——虽说骑营和中军放置马匹的地点根本不在一起,但好歹草料都是一处领的,多少都能听到一点准确的消息。

不过马夫知道的东西还是很少,鸡零狗碎的。唯一可以确认的一点,就是春天果然不是出动骑兵的好时节,即使是都配发了一定量的豆粕来加强营养,但还是陆续有近一成的战马失去了战斗力。

但是往好处想,吐蕃人那边的情况只会更差。至少韩冈前面在景思立那里,看到的来袭蕃军骑兵,他们所骑乘的战马情况并不算好。景思立手下的几个深悉马性的将校观察了一阵后,都说禹臧家的这些战马,如果不能及时休息和补充食料,回去后肯定要毙命一批——如果双方的战马都出了问题,明显的对更为依赖步兵战力的宋军更为有利。

跟马夫说了一阵后,韩冈的腿脚也终于安稳了下来。向着堡中王韶的行辕走去,外人虽是看不出来,但韩冈自己感觉着,走路时双腿还是在打着飘。

“景思立那里的情况怎么样?”

见到韩冈回来,王韶劈头就问。虽说前面已经收到了情报,但韩冈毕竟是亲眼见到禹臧家的骑兵,他嘴里的话更为直观。

“下官回来的时候,禹臧家的骑兵仍在骚扰临洮城的工地。”

“骚扰?”苗授追问了一句。

“只是骚扰。”韩冈点头。

两千骑兵来攻打万人前后驻防的大营,吐蕃人又没有发疯,怎么可能会硬拼。但他们的骚扰也给临洮堡的修筑带来很大的麻烦,预计的工期肯定要拖延,至少夜中不敢让民伕们继续干活。否则派出百八十名骑兵绕过山间来夜袭,疲累中的民伕很容易会炸营。

高遵裕闻言皱起眉:“那还要几天时间?”

“还要七天到八天,比预定的工期要拖长三日。”韩冈顿了一下,补充道,“不过不会影响景思立率秦凤军来报到,有五千人堵着禹臧家的骑兵已经绰绰有余。就算禹臧花麻大举南下,已经抵达狄道的姚兕姚麟,要去支援也很容易。”

高遵裕和苗授满意的点头,这是他们想听到的。

“这边的情况呢?”韩冈问着。

“香子城已经攻下来了。”

这件事韩冈已经在马夫那里听说了,听到这个最新的战果,他也不觉得有什么意外的,理当如此。“那斩首呢?”

“二十四个,吐蕃人是主动放弃的香子城……木征肯定是要在河州决战了。”苗授肯定地说着。

“放弃珂诺堡越是轻易,放弃香子城越是轻易,就越是证明木征不会放弃河州。”

王韶的判断,韩冈心有同感:“如果有了一次惨败,木征可以压倒所有的反对声,主动放弃河州,然后设法在山岭间拖垮我们。但一次激烈点的战斗都没有,他就放弃河州,必定会落到树倒猢狲散的下场。”

“那样谁都会认为他怕了。”这是连同高遵裕在内的共同的判断,“一个胆小的首领,没人会跟着他的。”

“关键还是在河州城!”

王韶一句总结陈词,熙河路的四名主官相视一笑,几年来养成的默契尽在不言之中。木征既然在河州城摆下了宴席,他们也就却之不恭了。

召集来珂诺堡中诸将佐,还有一应幕僚,十几人济济一堂。

韩冈出面,将几位主官的判断向众将说了一遍,又说起在河州城下可能面对的敌军数量。

“以河州诸蕃部的帐数,如果木征将他手下的蕃部全数动员起来,当能组织起十万人以上军队,这还不包括各部留守的兵力。”

韩冈的话并没有引起众将的骚动,这是他们预先都知道的。

他继续道:“当年包顺都号称他青唐部及其辖下诸部,总计有十二万口之多,而木征下辖的蕃部比他只多不少,十万并不出奇。但以各部的粮秣和战马的情况,木征最多也只能能维持三万人到五万人一个月左右的战斗——这个数字,包括他的援军。”

“董毡还是禹臧?”赵隆问道。

“董毡已经出兵相助……领军之人都打探清楚了,是青谊结鬼章。”韩冈说着,眼睛转向王韶背后的智缘。

智缘会意,出来介绍道:“他是鬼章部的新任族长,贫僧曾见过他一次,不是个简单的人物。”

“一起提防起来就是了。”高遵裕不在意的说着。

“粮秣可能供给得上?珂诺堡的贮备是不是够用?”

珂诺堡将是即陇西、狄道之后的第三个中转站,比起沿途的兵站,其地位更为重要。所以宋军在攻下了珂诺堡后,又花了近半个月的时间,来向堡中运送粮秣军资,以用来准备河州前线的需用。这是所有熙河将校都知道的。王舜臣为人外粗内细,军事之中,他最关心的就是后勤。

“眼下珂诺堡中粮秣军需都已逐渐齐备,大可放心。”

“安全呢?”

这下是苗授的儿子苗履代韩冈回答,他是负责珂诺堡的守卫工作:“这几天还找出了三处暗道,都填埋了起来,堡中的安全不用担心。论起土木之事,蕃人拍马也赶不上我们汉人。”

“秦凤、泾原两军的情况呢?”王舜臣继续问着。

“五千秦凤军后天就能抵达珂诺堡,而泾原军昨日已经抵达了狄道,正待经略的命令。”

一个个疑问都得到了满意的回答,众将的心中渐渐的涌起了必胜的信念。

“各路兵马都已经准备完毕,而我们离河州还有五十里!”王韶闭起双眼,转瞬又猛然睁开,喝道:“就只剩五十里!”

……………………

听着殿下准备外出任官的朝臣说着千篇一律的废话,赵顼强忍着打哈欠的欲望。

这些日子以来,赵顼都没有睡好觉。自从下诏同意了熙河路攻打河州之后,他就夜不能寐,食不甘味,到了早间,也自然没了精神。

赵顼很为熙河担心。从狄道到河州,超过两百里的这一步跨得太远。加之还要翻山越岭,与早前的历次战事都截然不同。

狄道城离渭源并不算很远,去年攻打狄道,翻越鸟鼠山,穿过大来谷,几乎没有费什么气力,也就蕃人偷袭渭源让人吓了一跳,但也带来了让赵顼喜不自禁的捷报。

但眼下,随着熙河、秦凤、泾原三路的三万精锐一步步的深入蕃人杂居的不毛之地,身后的粮道拖得越来越长,赵顼也不能不为他们的安危担心。一旦失败,不知其中有多少人能安然而回,

尤其是前日,听说王韶急匆匆的提前出兵,更是让赵顼的心头蒙上了一层阴影。究竟是为了争夺功勋,还是如王韶连同捷报一起送到的奏疏中所言,是为了打木征个措手不及。赵顼都不知道哪个更为切合实际。

幸好昨日一并听到了攻下珂诺堡的消息。熙河路的沙盘模型赵顼不知看过了多少次,重要的寨堡名称早就在心中滚瓜烂熟。珂诺堡是河州门户官军占据了此处,攻打河州的战事,至少是顺利完成了前半段。

另外的一点阴影,就是西夏方面始终没有反应,横山的另一侧始终静悄悄的。赵顼甚至觉得就像是夜中行于孤巷之中,总感觉着背后有人,只是回头看时,却是空空荡荡。少年时,赵顼很有过几次这样的经历,现在想起来,失笑之余,依然还有些心头发毛。

‘党项人真的会坐视吗?’他忧心不已。

……………………

兴庆府的宫室森严。

朝臣皆已退去,秉常也被带进了后殿,偌大的宫室中只有梁氏兄妹一坐一立。

“听说河州打起来了?”梁氏端坐着,并不赐兄长坐下。

“是。”梁乙埋低头。由于罗兀之战的损兵折将,这年来兄妹两人的注意力一直放在国内,但并不代表他不关心周围的变故,“昨日兰州禹臧来报,宋军出兵号称十万,实际亦至少超过两万,由王韶亲领,目标正是河州。”

“木征能不能抵挡得了?”

“恐怕很难。”

“董毡会不会助他?”

“……不会全力。”

“东朝咄咄逼人,今天打下了河州,明天就该是兰州了……”梁氏黯然一叹,“大哥,你说该怎么办?”

“禹臧花麻已经准备出兵援助木征,暂时交由他处置……而且这个时候,我也不方便离开兴庆府。”

梁乙埋暂时不能离开京城,他必须再坐镇一段时间。经过了一年的大清洗,在兴灵一带,梁氏兄妹的控制力的确是增强了许多,但眼下兴庆府中的稳定,还是少不了他来维持。

梁氏沉默良久,“那就速将仁多零丁招进京来吧,我想听听他的看法。”

