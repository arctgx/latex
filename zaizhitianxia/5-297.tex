\section{第31章 停云静听曲中意(四)}

宰执们一路出了皇城,默契的相互致礼,而后便四散而去。

就算皇帝能动一动手指了,也不可能坐在大庆殿上,自然不会有正旦大朝会。既然不用早起,当然是各自回去补觉。至于天子留下王安石说些什么话,过两天就会见分晓。

此时已是下半夜,熙宁四年的正月初一,天穹上只有星光。在御街上放鞭炮、放烟火的人已经少了许多,不过在街头巷口等位置,却多了一些鬼鬼祟祟的身影。

对于京城这一特产,韩冈早就是见怪不怪,骑着马昂然而过,瞥都不瞥一眼。

“玉昆你倒是不怕冷。”章敦他与韩冈正好同路,羡慕的看着韩冈迎风而行、毫不畏寒的坐姿,自个儿却只能直往手中呵着气,他今天带的皮手套一点也不保暖。

“好歹小弟也是北方人啊。”韩冈回头笑道,挺直的腰背也放松了一点:“秦州在山口上,巩州也在山谷间,到了冬天,寒风吹得那才叫冷,京城已经好很多了。不过子厚兄你虽说是福建人,可在京城时间也不短了,早该习惯了吧?”

章敦将披风裹紧了,摇着头:“今年冬天比往年冷得多,前两年可没这么厉害。”

“说得也是。”韩冈仰头看天,今夜天朗气清,澄澈的夜幕上,银河清晰可辨,能发现许多寻常模糊得几乎都看不见的星辰。

在冬至前的一场暴雪后,近两个月时间,就只下了两场雪。但阴天不少,一旦放晴,就是北方的寒流南下了。北风一吹,不算很低的气温也能让人冷得够呛。其实以今夜的寒冷,如果能有温度计来测量的话,估计也就摄氏零下十度上下的样子。

零下十度左右的天气在河南一带的冬天一年也没几天,但也不算稀罕,只是现在迎面来风,当然吹得冷。韩冈也不是当真全然不怕冷,只是比较耐寒。但他穿在公服内的冬衣是特制的,双层羊皮对缝起来,十分保暖,另外还套了一件雁绒的夹袄。膝盖处有皮制护膝,而且还是花熊皮;手套同样是精制的。章敦尽管有不输韩冈这般稳妥的保护,可在耐寒一项上,福建子终究是比不上关西人。

想起温度计,韩冈就有点想叹气。巩州的玻璃工坊倒是能开始为温室提供小规格的平板玻璃,玻璃灯罩更是开始批量化生产,但温度计连影子都没有。别说温度计,就是能耐火烤的烧杯、试管都没有造出来。现在玻璃工坊正在努力攻关更大尺寸的平板玻璃和玻璃镜,韩冈想要的实验仪器,还不知要等到猴年马月。比起军器监的成果,真是差了许多。

章敦自不知韩冈心中所想,举起马鞭冲前方黑黢黢的州桥指了一指,“可惜是年节,夜市摆不出来,要不然就在那里喝杯热酒再回去了。”

“子厚兄你这么一说,小弟肚中的酒虫都要给逗起来了。”韩冈笑了起来,“还有那一道旋炙猪皮肉可是难得的美味,家里做不出那等味道。”

“那家做猪皮肉的店家,玉昆你和薛子正上门给他家打过招牌后,这两个月听说赚钱赚得来不及数。已经在南城买了大屋了。”

“钱醇老是不是该谢我?”

“啊?”章敦没听明白。

“开封府不是又能多收税了?就是买房的契税也是一笔啊。”

章敦嗤的一笑:“……玉昆你若能从州桥夜市到鬼市子都去吃一圈,钱醇老会不会谢你那是两说,但在京的小店家肯定愿为玉昆你立长生牌位。”

韩冈正色道:“京中正店利厚,脚店、食肆则要清苦得多。可在脚店、食肆中讨生活的百姓却反过来远比正店中雇工要多得多。若脚店、食肆生意好了,京城市井倒是能更安稳了。”

“玉昆你是操着宰相的心啊,再操心一下北方如何?”章敦看韩冈一眼,摇了摇头,又缩着肩膀抽起气,“现在京畿都冷得这么厉害,河北那边应该更冷上许多,辽国自是更甚。只是比起耐风寒,南人的确不如北人,但北人终究还是比不上北狄啊。”

韩冈笑说道:“幸好战场决胜,不是比的谁更不怕冷。就是辽人更耐寒,也耐不住刀箭。”

“河北军事有郭逵节制,又有李信镇守边关,当可高枕无忧。只不过……”

之前在寝殿中晾了赵顼一回,章敦心中没底,其他宰辅其实同样没底,天子毕竟是天子,不过有韩冈做了保证,倒是一时都能安心。

在章敦看来,韩冈如今在朝堂上的地位十分特殊。在太子成人之前,他的地位几乎不可能动摇,比任何一位宰臣都要稳固。同时在医学上,他的眼光可以信任。没有他的一句天佑,宰辅们很难真正下定决心。而他身为王安石的女婿,对如今的平章军国重事有着一定的影响力。

“只不过什么?”

“只不过愚兄最担心的是内部人心不齐。”

韩冈自知章敦说的不是北方之事,只是有些话不可能明说。他轻声道:“欲要上下齐心,先得内外同欲。如今两府可谓是同欲齐心对辽,子厚兄又何须担心。”他声音顿了一下,“别的小弟都不担心,只是怕曾参政心不一。”

韩冈的这一句说得直白了,只是他声音更小,小到只有章敦能听得到。

章敦本来想说的可不是曾布,但听韩冈提起,眉头就皱了起来,“曾子宣初来乍到,何况平章对其依然存有旧恨。”

“不过在京百司中,可有不少人是他旧日提拔起来的。要坐稳东府之位,对曾参政来说,当真不是难事。”

在吕惠卿丁忧回乡,曾布作为王安石的副手主持变法的三年间,是新法从初兴到稳定的三年。曾布最多时曾经身兼十数职,变法之事,事无巨细,皆总于其手。多少新党中坚,都是他提拔任用上来的。所以当初他的背叛,才会让王安石衔之入骨——对新党的打击实在太大了。

章敦苦笑了一下,他可不敢为曾布作保,“曾子宣应该会顾全大局吧。”

“谁知道呢?”韩冈冷笑。不同人眼中的大局可是不一样的。要不然吕惠卿也不至于发足狂奔去追种谔。

“至少在西北局势,并无他置喙之处,他当也不会有何异论。”章敦说道。

因为吕惠卿吗?韩冈默然自语。河北那边,他的表兄都坐镇在对辽的第一线,就是唯一的河北人韩绛也不能说什么。现在韩冈推动两府保种谔,实则抛弃了吕惠卿,曾布那边多会先看一阵笑话。否则几个宰执联手将吕惠卿救回来,曾布也别想落个好。

“可那也要家岳不帮吕吉甫说话才行。”韩冈说道。

在王安石第二次拜相期间,吕惠卿虽然有所疏离,但比起背后捅刀的曾布强了不是多少倍。而且吕惠卿在任上一心一意推行新法,维护新学,在王安石的心目中,自己这个女婿可远远比不上能维护新法、新学的政治继承人。

“之前在殿上,平章也没帮吕吉甫说话。”章敦正说着话,突的咦了一声,在马背上坐直了身子,仰头看着东北面:“那边是不是走水了?”

韩冈顺势望过去,远处红光一片,随着风,还有敲锣打鼓的声音隐隐传来,当真是起火了,“还真是走水了。钱醇老今夜别想安生了。”

“哪年年节时,开封知府能安生的?最苦不过冬日!”

韩冈和章敦说得轻松。越冷的冬天,失火的几率就越大。入冬后的这几个月,隔三差五就是一场火,都是见怪不怪了。而且京城的火灾

“将作监就在那个方向上吧?”章敦的脸色又是一变。

韩冈摇摇头,“哪里那么容易烧到将作监……”

可虽是这么说,但两人的心情也不再那么轻松,各自点起家丁,派去起火的地方打探消息。待骑手飞奔而去,两人交换了一个眼色,同时叹道:“幸好不是军器监。”

停了一下,章敦又道:“曾子宣的参政府就在那边吧?”

……………………

曾布刚进家门,妻子魏玩迎了上来,“怎么这么快就回来了?”

“天子有没有话吩咐,当然就回来了。”

魏玩跟在丈夫身后,“不说是官家病好了吗?”

“不过是能动动手指而已。”进了屋,曾布在火盆边舒展几乎冻僵的手脚,“又不是能坐能说,还能怎么样?”

“就为了这件事,将两府都招进宫中?”

“不止两府,还有一个韩冈。”提到韩冈的名字,曾布的眼神就冷了下来,“韩冈现在可不简单。他要保种谔,章敦、薛向都跟他站一边。蔡确与其一个鼻孔出气。甚至韩绛也给他稳住了。”曾布大事小事从来不瞒着妻子,方才在宫中耳闻目睹的一切都倒了出来,“张璪有他没他都一样,为夫都只能附和。”

魏玩能听出曾布话中之意,失声惊道,“难道官家的病……”

曾布沉声:“韩冈说是天佑。”

魏玩脸色一变:“也就是非药石所能挽回?!”

曾布摇摇头,韩冈的话可以这么理解,但他若不承认也找不出毛病:“别乱说。”提醒了妻子一句,他又笑道,“反正吕惠卿这一回有难了。”

夫妻俩正说着话,突然外院一阵嘈杂喧哗,治家严谨的曾布不快的望着外面,一名家丁跌跌撞撞的冲进来,急声叫道:“参政,对街的宅子起火了!”

