\section{第20章 敌如潮来意尤坚(上)}

张守约回头顾望,身后旌旗招展,将士密集如蚁,人与旗帜似乎已将整片谷地给填满。但若是认真数来,人马数目其实也只有两千——这便是他秦凤路兵马都监手上仅有的一点兵力。

年近六旬的张守约须发已然斑白,浓重的双眉长长的压着眼皮。老将半眯起眼,眼角的鱼尾纹一如条条深邃的沟壑,黝黑的脸上尽是皱纹,仿佛是干涸很久的田地。平静如常地脸色看不出一点异样,只是紧抿的双唇已透露出他心中的紧张。

昏黄的双眼,盯着东面的敌人,足足有上万的党项西贼,有纵马持槊的铁鹞子,也有披甲挺刀的步跋子,人海绵延,大白高国【注1】的马步禁军从谷地的一头连到另一头,将张守约回甘谷城的去路完全堵死。

张守约暗恨自己今次巡边时太过贪功,中了如此简单的计策。甘谷城建立在大甘谷口处,南面就是六十里长的甘谷谷地,也因为有温泉汇入,而被称为汤谷。而甘谷城北,出了谷口,是甘谷水上游谷地,因为处于马岭之南,名为南谷,是如今宋夏两国势力的分界线。

张守约带队巡边,本意是找机会驱逐侵入南谷中的千余名西贼,但没想到那些贼人只是个诱饵,真正的敌人早埋伏起来,正等着他自投罗网。当他带着两千兵马追追停停,弯弯绕绕,花了两日的时间跟着西贼来到南谷的一条支谷时,万名贼军便从埋伏的地方杀出来,拦住了两千宋军的归路。

现在张守约和他的军队所在的位置,离甘谷城大约有三十余里。这个距离看似并不算远,也就急行半日的路程。可一旦开战,却是咫尺天涯一般。当年三川口一战,大帅刘平带着麾下人马离延州最近的时候就只剩五里,眼巴巴的望着延州城墙的影子,鏖战竟日却硬是没能突入城中去,最后万多人在延州城外全军覆没。

相距三十里地;退路上还有五倍的敌军;自己又是追着贼军连续跑了两天,打了一仗;最后被贼军埋伏,士气大损。摆在张守约眼前的形势,也许跟当年刘平所面对的局势一样危急,秦凤路的张老都监也因此捻着胡须,沉默不语。

“都……都监,怎么办?!”

“慌什么?不就是一万多西贼吗?看你们吓得这德性?!”

张守约不耐烦的冲着心惊胆颤的部将骂道。部将们的怯弱,反而让老而弥坚的张守约摆脱了陷入贼人陷阱后的不安,意志重新坚定起来。如果除去贼人的陷阱造成的士气大落不谈,其实困扰张守约的也只不过是五倍于己的敌军罢了。

没错!就是‘只不过’!

张守约是关西宿将,二十多年前,宋军在几次会战中连续惨败于西贼。虽然他都无缘参战,可事后的驰援和补救都参加过。对刘平在三川口、任福于好水川以及葛怀敏在定川寨的三次惨败的内情了解甚深。

由于地理条件的关系,关西沿边被分割成秦凤、泾原、环庆、鄜延四路,理所当然的,边防西军也被分割成四个部分。从大宋布置在关西的总兵力上看,的确是远远超过西夏,但如果从单独一路来说,却是在西贼之下。

而且一路军队由于要分兵防守各处要隘,从不可能聚齐。可西贼却能随心所欲的调集举国兵力,猛攻其中任何一路。故而三次大败,都是兵力居于劣势的宋军,在陷入狡猾多诈的李元昊的陷阱之后,被以逸待劳的西贼以绝对优势的兵力击败。

如三川口之战,就是刘平的一万多因党项人的计策而来回奔波了数日的疲兵,对上李元昊亲领的十万养精蓄锐的党项大军。虽然上了敌人的当,只能怨自己蠢,怪不得敌人狡猾。但以两军决战的兵力之悬殊,尚且在三川口厮杀了近两日方才结束,其中刘平还能立寨防守。党项战力如此,也怨不得许多西军将领对当年的失败耿耿于怀。

如果在公平的情况下,以同样的兵力正面相抗,不论是野战还是城池攻防,宋军失败的战斗其实并不多。以少敌多,将西贼赶跑的情况,也绝不少见。而现在,不过是两千对一万罢了。而且作为诱饵的一千西贼,已经给张守约他稳当当的吃到了肚子里,没能遂了党项人前后夹击的美梦。

“还有得打!”张老都监很肯定的想着。如果能再拖一拖,伏羌城和山对面鸡川寨的援军应该就到了,那时便是宋军前后夹击西贼了。

只是援军现在并没有到,西贼已经开始准备攻击,而初升的旭日正从党项人的背后照来。位于西侧的宋军,便必须同时应付敌人和阳光的挑战。天时地利人和,三样丢了两样。张守约想来想去,他也只能与西贼比拼一下人和了。

心中诸多的盘算,一个接一个腾起,继而便一个接一个被否去,到最后,留在心中只剩下了一个名字:“王君万!”

“末将在!”

就在张守约身侧十几步外,一名高大英俊的军官应声从马上跳下,灵活的动作并没有受到一身重铠的影响。他在张守约马前单膝跪倒:“请都监吩咐!”

张守约抬起有些沉重的右臂,指着前方浩荡如渊海的敌阵,“你带本部兵马,去冲上一冲。”语气平淡得就像让王君万去街上打壶酒,买个菜。

“冲?”王君万疑惑的抬头。

昏花的老眼,在一瞬间变得锐利如刺,张守约的眼神恢复了年轻时代的精悍,他厉声问道:“你敢……还是不敢?!”

王君万长着一对略显秀气的凤眼,相貌端正,白皙的皮肤让他完全不像一名整日里风吹日晒的军汉。但正是这位俊秀得过了头、不到三十岁的青年,身上铠甲和袍服还透着斑斑血渍,这是他之前带队歼灭西贼诱饵而染上的印迹。

王君万听到张守约的反问,霍然站立。凤眼剔起,面皮泛红,扶着腰间刀柄,怒声吼着回道:“有何不敢!”

他一阵风的回身上马,拔起插在地上的丈许长枪,在头顶用力一晃。枪刃破风的啸叫一下吸引了麾下将士的目光,他吼声如雷:“儿郎们!跟俺杀过去!”

王君万作为一名骑军指挥使,指挥着四百骑兵,官阶仅是为无品级的殿侍,距离从九品的三班借职,还有一段不短的距离。可看他带兵冲阵的模样,却是百战名将才有的气势。

四百骑兵旋风般冲出支谷,惊雷般的蹄声在谷中回荡。在王君万的率领下,一头撞入聚集在南谷中的西夏阵列。王君万手持长枪,亮银枪尖闪动,直似梨花飞舞。人马过处,带起一条血浪。四百名骑兵紧随王君万之后冲杀过去,如同轻舟破浪,逼得当面的敌军不住向后退开。

白色的西贼将旗就在眼前,王君万吼声更烈,长枪吞吐,接连挑翻数名党项勇士,率队冲散了数支西夏铁骑的阻挡,直冲大旗之下,誓要斩下领军敌将的首级。

眼见着王君万即将直捣西夏的中军本阵,党项阵中号角急促的响了几声,一阵呐喊,一支少有披甲、服色不一的步军横刺里杀出,硬是用血肉之躯堵在了宋军骑兵之前。

张守约呼吸一促,猛地攥紧马缰:“不好!”

堵在宋军骑军之前的队伍,唤作撞令郎,是西夏将国中的汉人组织起来,编练而成的军团,每到遭逢强敌的时候,就会强要他们冲上去。赢了,后队跟着掩杀,败了,死得不过是汉人。正是这支汉奸军团,在关西四路造成的血腥,绝不下于党项西贼。

被撞令郎死死缠住,王君万的四百骑军冲势渐缓。一队铁鹞子觑得时机,拦腰向他们撞来。王君万指挥得当,一扯缰绳,带着全队斜刺里避了过去。但他们的攻势,却也随之土崩瓦解。一支支党项军队伍呼喊着冲杀上前,如同群狼围攻饿虎,将王君万他们团团围起。猛虎虽然凶恶,但每次交击,都会被狼群撕下一块皮肉来。

杀入敌阵的宋军骑兵以肉眼可见的速度急速减少,每一刻都有人受伤坠马。王君万回头看顾,顿时目眦欲裂。随着一声惊动整个战场的暴喝,王君万的长枪于风中再次带起呼啸,滚滚枪影接连掠过十几名西夏勇士的喉间和胸膛,枪尖上闪耀着血光。一瞬间,挡在前路的滔滔敌军,竟被势若疯虎的王君万一人逼退。

“跟俺走!”

王君万又是一声大喝,双腿一夹坐骑,抢在党项人再次合围之前,率领麾下残存众军冲了出去。一行骑兵在西夏阵中左冲右突,费尽全力才寻到了个空隙,终于退回了自家阵地。在敌阵一出一入,虽然杀敌数百,但王君万麾下的铁骑也只剩下在马上摇摇晃晃、人人带伤的三百余。

注1:西夏的自称,党项人尚白,许多时候都自称大白高国,大白上国。

ps:都说强汉盛唐弱宋,但北宋自立国后,大败的次数并不多,胜率上看,至少比唐朝要好一些,尤其是正面防御性质会战,往往都能击退敌军。当然,这也跟北宋的国势有关,失去了燕山屏障,失去了养马地,北宋败不起,只要几次连续惨败,便会落到灭国的地步。

今天第三更,求红票,收藏。

