\section{第23章 内外终身事(中)}

“终于可以喘口气了。”

韩千六捶着腰,在一地礼物中,坐了下来。

韩冈得中进士的消息前些天就已经传到了陇西城中。

沸腾起来的不仅仅是韩家,还有巩州上下。进士第九,弱冠之龄的朝官,再加上宰相的女婿。这三条几乎决定了韩冈日后的未来必然是一片光明。

多少官员忙不迭的上来奉承以下韩千六。就算是新近上任的熙河经略使蔡延庆,也让人送上了一份厚礼。

韩云娘正此时正在堂屋中收拾礼物。她的形容有些憔悴,有时候会不知不觉的停下手脚,双眼也是毫无焦点的四处乱瞄,似是饱受了相思之苦。

“素心和南娘都有孩儿,心里有个寄托,所以她们还好一点。就是云娘多了点问题。”

“如果去了东京……”韩千六还在为不能去京师的参加儿子的婚礼而感到不快,“参加三哥的婚礼,云娘应该就会好了。”

韩冈的信,早在去年除夕前就收到了。他不便借用朝廷的驿传系统,派人回来送信,走得算是快了,也用了二十天的时间。信上主要说了一件事,就是请父母上京。

一般的进士榜下被招婿,基本上就直接送进洞房。但韩冈早在去年腊月就跟王安石家的女儿讲亲事定下,所以还是有写信让父母来京城。王韶再是亲近,也不能替代父母,有时间当然要让自家父母主持。

可是韩千六做着官,三四月份正是棉田下种,麦田也到了收割前的重要时节。韩千六官位虽卑,但事务极重,须臾离不了人。而且韩阿李也觉得进士不禀父母而自行成亲,这是常见的事。但作为男方的父母,儿子成婚时却跑到女方家去见礼,世上就没有这个规矩。她宁可不去东京跟宰相打擂台,等到回来后,再来看看宰相家女儿教养如何?

现在的韩阿李心气极高,笃笃定定的认为儿子日后肯定能做宰相,也没有多少畏惧王安石的念头。堂堂宰相,求上门来提亲,还不是因为儿子有本事吗?

韩阿李道:“等成了亲,三哥肯定要回来一趟。那时正好让她们一起去上任。家里有义哥儿在,也不要他们在面前尽孝心。”

韩千六思忖着:“也不知三哥会被安排在哪里。”

韩阿李冷笑一声:“你操哪门子心?!自有亲家公记挂着。”

……………………

韩冈这边的确是快到成亲的时候了,离预定的四月初六还有五天的时间。

认识的和不认识的都送了礼来,堆满了他在汴河边刚刚租下的一间小院。送礼的人,不仅仅局限于东京城,甚至还有张载和二程的贺礼。

韩冈过去曾经在信中跟张载说过他与王韶内侄女结亲的事,后来就没有提过此事了。自己跟王旖定亲后,又写信向张载这位老师解释,同时也没忘记跟洛阳的二程提一下。不管怎么说,他都不会愿意因为婚姻的问题,而跟自己的老师而翻脸。

——因为他的媒人是王韶,这一点就可以让韩冈的婚姻有着很好的解释。不是韩冈阿附王安石,而是因为要顾及王韶的面子。

另外,韩冈举荐张载,并在政事堂上推荐诸多名儒入京,共参经义局事。尽管此事看起来是没指望了,但已经从宫中传了出去,并被人当成王安石找错女婿的笑话来传播。可只要这消息传到洛阳和横渠,至少能让两位老师知道他并没有忘本。

而王安石这一边,虽然有那么几天,王雱没有来找韩冈。但重新坐到一起后,韩冈和王雱跟没事人一样,照样喝酒聊天。韩冈没有因为他的所作所为而向王雱赔不是,而王雱也没向韩冈问罪的意思。

王旁吃惊的看着兄长和韩冈,“这是怎么回事?”

“吾与元泽,乃是争于国事,非是私事。公私岂可不分?如小弟对新法的支持,是为理也,非因亲也。”

韩冈说得义正辞严,王旁倒是没话说了。

“玉昆说得好,”王雱给韩冈倒了酒来,再给自己和弟弟满上,端起酒杯,“不过市易法的好,可从来没见玉昆你提过。”

韩冈曾与王雱多次谈论新法,均输法、农田水利法、便民贷、将兵法、保甲法,都得到了韩冈的赞许。可这些法令之中,只有市易法,韩冈从来都不提,一句话都没有。他的态度,只要稍稍留意,就能知道端的。

“市易法不是不好,但推行此法得不偿失。”韩冈的回答,正符合他一向以来的倾向。

王雱和王旁两兄弟都不说话了。

市易法的造成的后果,眼下都见到,这条法令所引起的反扑,现在已经变得十分激烈。就在韩冈因科举前后之事而忙碌的时候,京城中的物价飞一般的涨上去,只要是市易务在卖的商品,都是在涨价。正如有人上书弹劾市易务,说如今京城中,是市易务‘卖梳朴,则梳朴贵;卖芝麻,则芝麻贵。’

这并不是市易务为了赚钱而胡乱抬价的结果,从政治利益上讲,直接负责此事的吕嘉问也不会允许这样的情况发生。赚了再多的钱,也抵不了物价飞涨对他政治前途的危害。

究竟是谁在背后做手脚,不问可知。

韩冈是知道后世共和国开国后,上海的投机商是如何来对抗新的统治者的。不过那些商人们的反抗,在组织力无可匹敌的国家机器面前,就如螳臂当车一般可笑,很快就耗尽了家财,。

但王安石此时的新党,却不可能拥有后世那个党派的组织力和控制力,也无力保证足够的物资供应。更别提豪商们和宗室、和外戚,都有千丝万缕的联系。从他们虎口夺食——正如韩冈所说,此法得不偿失。

得到的财税利润,远远抵不过被消耗掉的政治资源。而且本已渐次稳定的朝堂局势,就是因为市易法而再起波澜。要说王雱不后悔,那是假的。再多的国库收入,也比不过新党的根基再一次被动摇。

新党内部,已经有人说要废除市易法。但王安石和王雱却是一步也不肯退让。一旦退让,就是大堤决口的时候到了。到时候,就是新法被尽废的结果。但也有人提议道,明面上不废除市易务,但慢慢的松弛禁令,让市易法不废而废。

两个方案都是要废除市易务,不过一个急进,一个缓进罢了。

“不知玉昆有什么办法?”王雱问着。

虽然说是对韩冈此前的意图插足经义局的行为没有芥蒂,但王雱的心中还是给他对二妹婿记上了一笔。他要看看韩冈对市易法能出什么意见?同时也盼望他能提个意见,改变现在不利的状况。

“坚持到底!”韩冈的回答出乎王雱意料之外,“六路发运司加速运货,放开来发售,将京城的物价打下去,看看那些人有多少钱来收购。”

市易务并不是由官府完全掌控,除了账本,估价和贩售的环节,其实都是让商人们来参与。而市易务的收入,其中有很大一部分利润,是酬奖给这些与市易务合作的商人们的。如果能在短时间内,培养出新的一批豪商,取代如今的豪商阶层,便是一切可以放心。

“坚持到底可不是那么容易。”王旁摇着头,他可是对此深有体会。

“其实也简单。现在的正在闹腾的那些豪商,其实都已经是无源之水,无根之木。完全在耗家底了,只要能撑过去,他们不是负荆出降,就是坐以待毙。”韩冈冷笑了一下,“而且张、田、王、李能娶宗室,难道市易务中的那些就不能娶吗?一个县主不过是一万贯而已,宗女更是只有两千三千。何况娶了宗室的豪商中,总有不跟他们一条路数的。”

王雱叹道:“其实这些都有想过,只是缓不济急,需要别寻良策。”

王雱兄弟期待的眼神看着韩冈,韩冈摊开手,摇摇头:“到了战场上,若是没了粮草,诸葛武侯都要掉头往回走。”没有物质,也只能靠精神了,“小弟也变不出东西来。除了咬牙坚持,我也没办法了。”

韩冈的确是没有办法,但凡遇到有人哄抬物价,最靠谱的办法就是杀鸡儆猴,但闹得大的基本上就是曹、高、赵家的亲戚,而且是近亲,王安石也不能那他们开刀。另一个办法都是用洪水一般货物,耗光对手的钱财,将他们的气焰给打下去的,这就要靠掌控汴河水运的六路发运司的本事了。。

“对了,始终没有说过市易法之事的,记得还应该有一人吧?”韩冈看了一眼王旁,这还是王家的二衙内上次来见面时,不经意间说出来的。

肯定是曾布。

曾布从一开始就对市易法持有一些看法,前面吕惠卿回来执掌中书五房检正公事,大力推行市易法后,曾布对此法的态度就变得更加暧昧。

内部不靖,就是新党现在要面临的最大的问题。

