\section{第43章 庙堂垂衣天宇泰(七)}

【第一更】

“牛痘?”

“牛也生痘疮?”

黄氏兄弟一前一后的追问。

对于农事,他们都不在行,背下《齐民要术》是为了应付科举中的策问。有水的是稻,没水的是麦,至于田地里种植的其他作物是什么,如果没人告知,他们根本都分不清麦苗和韭菜的区别。

其实也不能怪他们。福建八山一水一分田,当地的世家也不可能靠着田地的出产来支撑家门,开作坊的,从事贸易的,都比田地出产更丰厚。如果韩冈提的是工商业方面的事,黄庸和黄裳倒是都能说出个一二三来。

两人的无知,韩冈不以为意,详加解释:“牲畜和人有时能得同样的病,小到感冒、腹泻,大到痨病、瘟热,人能得的病,牛马猪羊等牲畜也一样会得。所以人和牛同染痘疮一点都不奇怪。出在人身上的叫痘疮,出在牛身上的自然就是牛痘。”

“原来如此。”黄庸连着点头,“原来这就是牛痘。”

黄裳也拱了拱手:“不得龙图解释,学生还真的不知道牛还会生痘疮。”

韩冈笑了一下,什么都不知道才好忽悠:“牛生的痘疮,与人截然不同。只有小小的几个痘而已,不注意根本发现不了。但只要人接触了,就会染上,而后发病,不过发起的病并不重,也只是出几个痘,完全不像人痘那般,满身满脸都都给长上。”

韩冈说得口干了,停下来喝了口茶,却见两位听众却动也不动的等着他继续说故事,心中得意,这是彻底上钩了。

放下茶盏,他继续述说:“同样是痘疮,为何由人传来的重,有牛传来的却轻?……韩冈在发现此事后,也考虑了很久,有很多种解释。不过联系起灭毒种痘法,却是最合理的。”再看看两人,对上黄裳投过来的视线,笑问:“不知勉仲想通了没有?”

韩冈明里暗里的提示了很多,黄裳抿了一下嘴,抬眼反问:“可是牛灭毒比人要强?”

“正是这个道理。”韩冈点头,“牛比人健壮十倍,人会死于痘疮,而牛却不会。人要轮回七次方能将痘疮中的毒性拔出,那么以牛的健壮,只要一次就够了。”他又笑着,“这一番推理演绎,就不是孙师的传授了。”

黄裳双目灼灼有神,身子更前倾了一点:“格物致知!”

韩冈开怀笑着,他给了黄裳表现的机会,对方也没让人失望,顺势的抓到了:“没错,正是格物致知。世间万物,只要悉心观察,用心去格,总能格出其中的道理。牛痘免疫法,正是最好的证明!”

“格物致知之说,让飞船上天,马车入轨,如今又出来能救治万民的牛痘。都说龙图学究天人,往日尚有三分疑虑,今日一见,方知盛名之下固无虚士。”黄庸没口子的赞叹着。

黄裳则是呼吸一促,比起他的兄长,黄裳对韩冈所说这一段话,体会得要深得多。

‘形而上者谓之道’,现在所有学派,都是在‘形而上’中做文章,争得是对大道的诠释。而韩冈宣讲的‘格物致知’则别开蹊径,是以实证道。‘形而下者谓之器’,从‘器’上推入‘道’,而后再从‘道’返回到‘器’上,也即是‘明体达用’。

在此之前,韩冈通过《浮力追源》和飞船互相印证,已经向天下展示了他走上的这条道路。张载能在京城宣讲,多得其力。不过那只是雏形,还没有对儒门各派产生颠覆性的影响,只是让气学走入了京城。

但现在牛痘一出,则是大战拉开了序幕。这是韩冈对所有学派的挑战,是将学派争论的战场,从经书的释义,拉到实际运用中来。韩冈为气学、为格物致知拿出了飞船、拿出了牛痘,其他学派又能拿出什么相抗衡?

……来到韩冈所擅长的战场上,试问要怎么样才能击败他?

黄裳心中感叹,比自己年纪要小上七八岁的韩冈,不仅仅是在官场上让人只能仰望,就是在学术上,同样是远远将人甩得不见踪影

韩冈这时笑着在书桌上拿过来一卷书,翻了几页,就手递给了黄庸,“这一部《肘后备要》虽然是借用了葛稚川的书名,却是韩冈为政多年的一点心得,主要是如何应对旱涝蝗瘟之类的灾异,其中也包括种痘之术,只希望能对世人有所裨益。”

葛稚川就是葛洪,东晋有名的方士、医者,在《抱朴子》之外,还有一部《肘后备急方》,主要是常见病的药方,故而名为‘肘后’。韩冈既然以‘肘后’为书名,其中的内容当然就是针对着常见灾异或是疾疫。

探出双手接过来,只看了一下纸页和装帧,黄庸就看出了这本书的底细,是手稿,而不是印刷的制品。端端正正的小楷就不知是手抄本,还是韩冈本人撰写的原稿。但再仔细一瞧,却发现书中的正文,却是一句句分得很清楚,句与句之间用奇怪的符号来分割。

哪有这样的书?!正常的文章都是连绵而出,断句全要靠自己,就是《论语》《尚书》都是一样不分句读。断句是老师的工作。‘师者,传道、授业、解惑’,句读也是授徒时少不了要教给弟子的。就是寻常的手抄本,也是在句子末尾的那个字边上的行列空白处点上一点,不会将正文用符号分割开来。

黄庸很惊讶,抬头指着书正想发问,就见韩冈说道:“事关人命,就不让人费神去句读了。让小吏多少也能看懂一点,省得断句错了误会。说我文字浅薄是事小,害了百姓就不好了。”

黄庸讨好的笑着:“龙图一片仁心,那还会有人有脸说龙图文字浅薄。”

韩冈皱着眉,感慨着:“就韩冈看来,医书都该如此。要准确无误,不能让人误会。不仅是断句上,就是勘误上,也得下功夫。上次看一本医书,竟将‘餳’印成了‘錫’,丸‘餳’服用,变成丸‘錫’服用。本来是药弄软了做成药丸,现在是加个锡丸一起吃下去,这是要死人的!”

黄裳义愤填膺:“这样的医书该烧掉才对!”

黄庸翻翻书,“龙图在序中也写了这件事。”

“是啊,就是因为有了此事,才动了念头。”

黄庸翻着书,韩冈一开始翻给他看的那一篇,正是方才所说的牛痘。只是扫了一下,一切就都得到了解释。大概是为了避免一些不必要的麻烦,韩冈在上面将有关牛痘的整件事的来龙去脉说得很明白。而其他篇章,则就是韩冈所说,如何应对旱涝蝗瘟之类的灾异的手段,说得很详细。以黄庸多年为官的经验,如果当真遇到灾异,依法施为,倒还真是能免去不少问题。一时间竟舍不得放手,十分专注的翻看着。

黄裳有点不顾仪态的斜着身子,张望着黄庸手上的书。远远的看了了两眼,心道‘果然如此。’

从牛痘和《肘后备要》上,可以看出韩冈是双管齐下,从道理上和实用上同时下功夫。

黄裳在江南士林中也算是小有名气,对如今儒门学派之争有所了解。他曾听说张载病逝之后,气学无人出来担当大任。而传承气学衣钵的韩冈,当年亦曾立雪程门。所以在江南的士林中,多数都认为在韩冈的率领下,气学弟子多半会投奔二程门下,要不然就是归于新学。

恐怕没人能想到,韩冈竟然还在坚持,之前的沉寂,只是为了以种痘法为刀枪为战鼓的惊天一击。

两名客人的注意力都集中到了书上,一时间冷了场。韩冈等了半天,终于忍不住咳嗽了一下。

黄庸和黄裳惊醒过来,抬起头,不好意思的笑了笑,“龙图此事一出,我等为官也有了依循的凭证,天下百姓可都要受到龙图的恩德。”

“过奖了,只是打算让人用来参考而已。”韩冈谦虚了一句,看两人的模样,也到了说正事的时候,清清嗓门:“前两日,伏龙山中试行种痘功成,我便给唐州去了信,沈存中昨天回信,想在唐州推广种痘免疫法。本来我是打算让他先做好准备,等我上书朝廷后,待天子和政事堂批下来,再行推广于唐州。等唐州那里有了功效,转运司也就有了向全路推广的底气。”

“那要耽搁多少时间?!”黄庸知道韩冈是在卖关子,便是一副焦急的模样,“龙图,朝廷、唐州两个地方一耽搁,至少要一年半载,试问这段时间中,又会有多少幼子因此而夭?既然伏龙山中已见功效,应当早日推广才是。”

“龙图……”黄裳也道,“转运司可就在襄州。襄州满城的百姓也都知道了龙图有种痘之法。不能拖延了。”

“是啊,没想到外面就这么给围上了。”韩冈苦笑了一下,“所以我打算提前在襄州推广种痘法,今天请常伯来,就是为了此事,想要劳烦一下常伯。”

“利国利民,岂可曰‘劳’?”黄庸陡然起身,“固所愿也,不敢请耳。”

韩冈似是满意的点着头,“如此,就商量一下章程,也给之后推广立个范本。”

