\section{第20章 冥冥鬼神有也无(16)}

【抱歉。昨晚写得有些慢了,补更的一章只能欠着,看看今天能不能补上。】

十则围之,五则攻之。

上兵伐谋,其次伐交,其次伐兵,其下攻城。

尽管李洪真对孙子兵法并不了解,但也知道用兵是攻心为上,攻城为下。要想攻下一座决心坚守的城池,究竟有多难,之前李常杰在邕州城下的表现已经做了很好的说明。

想要攻下门州城,至少要有三五倍人马才足够!以门州城中的四五千守军,再差也能抵挡个三五天。

李洪真一开始是这么想的,当他看到了宋人派出的军队只有两千多的时候,便没有在第一时间开城投降。李洪真并没有为李常杰殉葬的打算,但他想将自己卖个好价。越能多抵挡一阵,自己能卖出的价钱就越高。

可他的幻想在短短的一刻钟之后,就登时化为了泡影。

在宋军的阵列之后,几堆火被点了起来,伴随着烟雾,两只球形的怪物缓缓飞上了天际,下面垂着个似乎是尾巴的东西。隔着半里地,与下面的人做个对比,能看得出两只怪物至少跟房子一半大小。怪物在空中缓缓的旋转着,圆形的头颅上,有三张恶鬼一般的脸相,一张脸在哭、一张脸在笑、还有一张脸在发怒,喜怒哀三种不同的表情,走马灯一般出现在御守城池的交趾军眼前。

看着两只怪物在宋军的阵列后方撑腰,李洪真喉咙嘶哑,“那是什么东西?”

不仅仅是李洪真,城头上每一位守城的官兵,心理上都受到了极大的冲击。惊慌失措下,有人跪倒,有人念佛,有人闭起眼睛又睁开,有人拉开弓箭对着远在射程以外的飞船射过去。

人人心中都在疯狂发问,‘难道宋人能驱使鬼神不成?!’

但随着对两只怪物的仔细观察之后,李洪真却发现,那并非是什么妖魔鬼怪。就像军中通常使用的盾牌,往往都会在表面上绘制虎豹熊罴的图像,举起盾牌来,就有一股冲击力迎面而来。飞起的异物上的三张怪物脸谱,肯定是有人绘制上去的,而整件异物应该也是高手匠人所打造的器物。

李洪真模模糊糊的又想起来,他似乎曾经听人说过,宋国有能让人飞在天上的船。虽然眼前的两个怪物,怎么看都不是船的船的模样,但的确是飞在了天上。

不过看破了真面目之后,李洪真却更加慌乱,甚至感到了一丝绝望。

李洪真听多了峒中巫师神婆们的能驱使各色妖魔鬼怪、蛊虫毒药的故事,但能让人飞上天的传说却少得可怜。这个跟御魔驱鬼有什么区别?这可是平地飞升啊!若不是鬼神相助,如何能莫名其妙的让房子一般大的飞船直上云霄?

作为全军先锋的李信仰着头,看着两只高高飘扬在天空上的飞船。从下往上,看不清飞船上的气球的全貌,工匠们花了太多的时间在飞船的装饰上,让他的表弟直皱眉头,但李信却觉得将画上一张鬼面,还是很涨士气,看着也觉得威风,“看到飞船,城中交贼当会被吓得苦胆都破了。”

黄元眼神呆滞的张着嘴,根本就没听清楚李信再说些什么。虽然他已经不是第一次看见飞船腾空而起,但每次看到将人送上天空的奇迹,都忍不住一阵心悸。一次又一次的庆幸幸好投效得早,想想大宋连飞天的神物都有了,交趾人哪里还能抵抗天兵到来?

已经成为行营参军的李复,奉命随行,以记录具体的战况。听到李信如此说着,遗憾的摇着头:“也瞒不了多久,看得多了也就变得寻常。飞船应该用在升龙府上的,只打一个门州浪费了点。”

“做将领的也许有见识,可以见怪不怪。但下面的士卒全都是愚夫愚妇,磕头都来不及,有几个还敢正眼看?”

要不是麾下的军队都来自北方,早见惯了悬停在城市中一家家大酒楼前面的热气球,第一次见识到飞船上天,也照样会在混乱中胆战心惊。李信第一次见到飞船,就算是早就听说了来龙去脉,一样是惊得合不起嘴。

“而且飞船的用处也不在吓人上!废话就不多说了,要尽快攻打门州城,今天晚上让全军将士在门州城中休息。”李信抬头又看了一下高高飘在空中的飞船,“飞船能在天空中大约停留一个时辰的时间,也不用第三艘准备了,在这两艘飞船落下来之前,给我攻下门州!”

号角声在飞船的吊篮中响起,从天上传下来的奏鸣,配合着在李信的将旗下擂动的战鼓,顿时又引发了守军的一阵混乱。

宋军开始向城墙稳步前进,城头上的李洪真恐惧的发现,在不知不觉之间,城头上的军力已经少了一些。

城墙的优势在于其高度,居高临下,视野、射程都大幅度的加强,从而能够顺利的压制住前来攻城的敌军。

但眼下在门州两丈高的城墙前,是悬停在二十丈高处的飞船,城中的一举一动,都瞒不过飞船上的瞭望哨,而宋军使用的神臂弓,更是在六七十步开外,就能瞄准城头将一支支木羽短矢射上去。

飞上城头的羽箭密如飞蝗,而从城上反射回来的箭矢不仅够不到城下的官军,数量也在急剧减少。在官军弓弩手们的强力压制下,城头上有百步长的一段,活着的趴着,死了的躺着,已经是没有一个人还能站立在城墙上。

“怎么办?”李洪真惶惶不知所措。

还留在城上的守军一个个双股战战,手上的弓刀也拿不稳。要他们对抗能够驱使妖魔鬼怪的宋人,根本就不可能。

但他必须要挡住宋军的这第一波攻击,要是挡不住,他根本就没有讨价还价的资格。必须要先挡住宋军的攻势,只要能拖到晚上,就能遣亲信出城去献上降表。

“宋人就只有两千!那两个是飞船,不是怪物!”李洪真疯狂的大喊大叫,“好好的将宋军的攻势挡住,要是挡不住,让宋军攻进城来,你们的性命难道还想还能保住吗?”

就在李洪真竭力激励士气的时候,数百名宋军战士已经扛着长梯,向着清理出来的那一段城墙冲过来。射向那段城墙的木羽短矢开始向城墙两侧延伸过去,挡住了赶来填补空缺的援军。

“看来交趾人并不打算从城里出来了。”李信遗憾地说着。正常的守城,城中守军只要兵力足够,都应该派出均对倚城而战,而不是仅仅依靠城墙的高度。

‘可惜了。’从飞船上回过神来的黄元想着。要是城中守军杀出来,他手下的一个指挥的马军,就能立刻出动,将交趾军彻底踩平,埋进深山中的谷底。

但交趾军既然没有出城来反击的胆量,就只能被动挨打一条路。作战的计划早已拟定,根本就不需要多费思量,有着远远超出交趾军的精锐,考虑如何破城都是十分容易的是一件事。

以神臂弓压制城墙上的守军,天上的眼睛则通报敌军的移动,剩下的士卒一步步逼近城墙。

砰、砰、砰的闷响,李洪真眼睁睁的看着宋军将几十架沉重的长梯撞上了城头,而他麾下的士兵,却一个个都是失去了战斗的勇气。

“挡住!一定要给我挡住!”李洪真大吼着。

“太子,挡不住了!还是先撤吧。”亲信在身旁苦劝着。宋军的攻势就像是狂风暴雨一般,猛烈地一如秋日的台风,根本就抵挡不了。

“撤?!”李洪真满是血丝的狞恶眼神瞪着亲信,一旦撤离门州,李常杰就能趁机砍下他的脑袋。

“滚!”李洪真放弃了斩杀亲信的打算,将人一脚踹开,指挥还听着他的命令的几名将校,“从速女墙下潜过去!”

‘只要挡得住这一次,就派人出降。’李洪真已经不指望能守到入夜。只是这时候逃跑,只会让自己的军队都被砍杀,必须要等宋军攻势稍缓才行。

但宋军的攻势却越发的猛烈。

女墙,也就是雉堞之后,是神臂弓的死角,只要冲过那一段正在被神臂弓扫射的城墙,到了宋人上城的地方,神臂弓肯定也不敢再射,防着误伤友军。

“援军从女墙后面绕过来了!”飞船上的瞭哨突然叫了起来,但他的声音传不到下面,匆匆写了一张小纸条,装在铜瓶丢了下去,接着有拉出了一条长长的旗带。

冲到城下的宋军,也看到了从飞船吊篮中飞出的鲜艳彩绸。不同情况,有不同的旗号,眼下的标志就是让他们在友军上城之前,先用弓箭来射击城头。他们立刻高高举起背负的长弓,依靠飞船吊篮中拉出来的旗号的指挥,向城头上抛射着长箭。

立刻就有了惨叫声作为回应,刚刚穿过神臂弓交织成的生死线的交趾兵,在从天而落的箭矢中,又是进一步的损兵折将。

李洪真呆然站在城头上,看着宋军轻而易举的攀上北门城楼东侧的城墙,一点阻碍都没有。

第一个上城的壮汉,身材高大厚实,体重应当能抵得上两个寻常的交趾士兵,他一踏上城头,就是重重砰然一声响。左右一看,便立刻向着李洪真所在的城楼冲过来。

跟随着壮汉上城的宋军士兵越来越多,而参与的守军根本就没有抵抗片刻的实力。领头的宋军开始清扫城头,将人变成尸体,将尸体变成可以拿回去点算的功劳。

‘只能投降了。’李洪真哪里想到门州竟然这般轻易的被攻破,两丈多高的城墙就跟纸糊的一般。

摘下了自己头盔,解下了腰中长剑,李洪真的精气神连同幻想一起破碎,垂着头、垮着肩,‘只能看运气了,运气好,还能在章惇、韩冈面前混个好处。’

从城楼中出来,李洪真刚想喊话,却见那名壮汉从腰间拔出一支手斧,虎吼一声用力投掷过来。手斧在空中急速飞旋,越过二十步的距离,磨得锋快的斧刃一下在砍在了交趾国四太子的天灵盖上,深深的劈了进去。

围着李洪真的近卫们都愣住了,看着从他们主君的脑门上嵌进去的利斧,脑中一片空白。而那名壮汉却是毫不废话,提着大斧,冲过来一阵切瓜砍菜的乱杀,跟随在他身后宋军士兵一齐掩杀过来。

半日之后。

韩冈和燕达在李信的陪伴下跨着马进入了门州城中,看着从城墙排水沟中流淌下来的血水,摇头感叹道,“想不到守将竟然死战不退,这洪真太子还当真是个忠臣!”

