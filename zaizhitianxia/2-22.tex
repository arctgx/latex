\section{第八章 太平调声传烽烟(一)}

【第一更,求红票,收藏】

喝过茶,王舜臣又拉着韩冈喝起闷酒。就坐在韩家的偏厅中,王舜臣一杯接一杯的把酒灌下去。严素心新端了酒菜过来,却不见他动上一筷子,就只见他喝着酒,三斤上下的一坛白云露,几乎给他一个人喝光了。

一直喝到院外巷子里传来二更的梆子响,酒坛空空的歪倒,王舜臣才沉沉的睡去,嘴里却还不住骂着李复圭。

对着烂醉如泥的王舜臣,韩冈摇头叹气,他这个样子也不好送回家去,若是在路上撒起酒疯,骂将起来,给外人听到就不好了。将他安置在客房中睡下,韩冈又让李小六去王家送了口信,省得王舜臣的老娘惦记。

回到书房,韩云娘年幼易困,熬不得夜,这时候坐在外间就沉沉的睡着了过去。韩冈推了她一下,想把小丫头叫醒。她却在睡梦中含含糊糊的不知说着什么,把韩冈的手一下打开。

韩冈笑了笑,轻手轻脚的将她抱了起来。小丫头身子一向偏瘦削了一点,韩冈抱起她来没费什么气力,轻飘飘的仿佛没有重量。不过她是属于骨架比较小的那一型,外面看着瘦,其实还是挺有料的。韩冈抱着她,隔着衣服的手感都很不错。

悄悄把韩云娘送到床上,帮她盖好被子。出了房,严素心就迎了上来。她的眼神中带着点羡慕,“官人对云娘真是用心。”

韩冈微微笑了,坦陈道:“因为她对我也用心。”

举起袖子,韩冈嗅了嗅,一股酒气扑鼻而来。虽然今天的酒都给王舜臣一人喝了大半,韩冈并没有多喝,但他还是沾了一身的酒味,闻起来有些薰人。

见着韩冈这个动作,严素心便会意的去帮他烧热水。虽然天气已经有些炎热,但韩冈宁可热着,也不想在这个时代冻出病来。而且泡过热水澡后浑身舒坦的感觉,也不是用着冷水能比的。

躺在浴桶中,温热的水冲刷着全身上下的疲累。韩冈半眯着眼,似睡非睡。忙碌了一天,这时候终于可以放松下来。而严素心就站在浴桶外,她将两条袖子卷高,又用一根带子把袖子扎起。露出两截玉藕般的皓腕,用力帮着韩冈擦背。

韩冈很舒服的享受着。只是他的身体虽然放松了,脑中的神经却还在飞速的转着。每天他泡澡的时候,都喜欢把当天发生和经历的事情,在脑中回想一遍。想想他在其中有没有疏失,再考虑一下接下来可能的发展,以及局势的演变。韩冈能跨过道道坎坷,并非他才智有多高,而是他凡事能多想一步,多考虑几分。若是只凭着一点小聪明,他也不可能走到这一步。

今天收到的关于庆州李复圭的这条急报,对王韶和他的事业来说,并非好事。韩冈也不禁要叹着,李复圭这厮当真害人不浅。

据韩冈所知,在朝堂上,枢密使文彦博是一直在反对任何对外战争和扩张的行为。其中最大的理由,就是赵顼对开拓横山、拓边河湟两件事的支持,将会引发边疆守臣对军功的贪欲。若是每一个到了边地任官的守臣都想做出一番事业,届时大宋边陲将永无宁日。

在过去,无论赵顼和王安石都对文彦博的担心不以为然,将帅们的行动,总得通过朝廷的认可,否则就无法调动大军,只能小打小闹,不可能将事情闹大。

但今次李复圭的行为却印证了文彦博的话。虽然用着干扰西贼筑城的名义,派出的军队也是他身为一路安抚使,在无朝命的情况下所能动用的极限——也就是三千人。但失败就是失败,李复圭事后以违令致败为名,斩了一路钤辖、都巡检,瘐死监押的行动,也证明了这是一场惨痛的失败——否则一点损失,不至于要把一路中的几个重要将领都给杀了。

因而这场失败也就正好成了文彦博攻击朝廷关于横山、河湟两项拓边战略的最新武器。

王安石不会任由文彦博攻击横山、河湟,天子也不会。理所当然,他们就必须保护李复圭,保护他不受反变法派的攻击,也就必须无视掉他推诿责任、枉杀将佐的罪行。所以说政治这玩意儿就是个污水坑,不论私德有多完美,一旦关联到政治上,都会脏得一塌糊涂,即便是王安石都不能例外。

而且李复圭会不会领情还要两说,因为李复圭本身好像并不是支持变法,韩冈上京时,正好听说过庆州等缘边诸军州的青苗贷——也就是如今利民低息贷——被拖延施行。这其中正是李复圭和前任陕西转运副使陈绎的谋划。

“真是乱啊。”韩冈突然叹出声来,抬手用力捶了一下水面。严素心吓了一跳,登时被溅起的水花泼了全身。

天气热了,又在更热的浴桶边上,严素心便穿得很单薄,这下被水溅到身上,湿透的衣服一下贴住身子,把她婀娜多姿的身材展露无遗。

韩冈的眼神顿时幽深了起来,盯着眼前峰峦起伏的胜景一时移不开目光。严素心脸色绯红,紧咬着唇,双手环抱着身子,把关键部位给遮住。

韩冈湿漉漉的站了身,精壮的身材也不遮挡,伸出手就一把将少女拉近了过来。被擒住手腕,严素心惊叫一声。脸上的绯红一直透到了耳朵上,她用力推拒着。只是她的力气哪里比得上韩冈,越是挣扎越是无力。很快就娇`喘吁吁的停了手,眼神也迷离起来。韩冈的手抚上她的肩头。

“六姐姐!”一声从门外传来清脆的呼唤,惊动了快要沉迷下去的两人。

严素心被吓了一跳,立刻推开韩冈,回头一看,却是本应睡着的招儿。她忙跑过去,蹲下去问着:“招儿你怎么醒了。”

“六姐姐你怎么在这儿,是不是不要招儿了?”小女孩软软的带着哭音,扁着嘴就真的哭了出来。

“招儿莫哭,姐姐就在这里。”严素心忙安慰着,把韩冈丢下,就抱着小女孩走了。

韩冈有些郁闷的从浴桶里出来,拿起干布给自己擦着身子。他平日在家里也不是多威严,严素心把他说丢下就丢下,弄得他心头的火不上不下的。

算了!韩冈摇了摇头,反正以后还有机会。

不过接下来的几天,韩冈却忙得抽不出半点时间去享受他的‘机会’。先是陪着王韶和高遵裕去了古渭寨。就是王韶前日说过的,甘谷城告急,刘昌祚带他手下的两千人马赶去甘谷助守,而王韶便得去镇守古渭。趁此机会,正好顺便让高遵裕看一看,接下来他们要展开工作的地点。

等到韩冈跟着王韶他们从古渭回来,奉旨复查秦州宜垦荒地数目的陕西都转运使沈起,这时候也到了秦州。

“毕竟不是宣抚使,韩绛一来,他这个都转运已经变成跑腿的了。”王韶在韩冈身边尖刻的说着,从古渭回来就要出城迎人,王韶也是有点脾气的。

韩冈笑道:“宣抚使的权威谁能比得上?不是现任执政,都不可能当上,岂是转运使可比?”

宣抚使名字中带了个‘宣’字,体现了其担负着代天传诏的任务,抚绥边境、宣布威灵,统兵征伐,安内攘外皆为其责。陕西宣抚使管辖的不仅仅是兵事,而是实质上的执掌陕西军政的最高长官。比起安抚使、转运使的管辖范围来,确是要宽泛得多。

当然,就是因为宣抚使的职权如此之重,故而就仅仅是临时性的差遣,事毕便罢使还阙,而且必须是如韩绛这样的执政官才有资格。

而在陕西有了宣抚使之后,陕西转运使的名字虽不变,但实质上的地位却一落千丈。沈起现在几乎就成了陕西随军转运使,跟在宣抚使之后,做着后勤方面的工作。

不过沈起到了秦州,却还是个大人物,李师中都要出城相迎。

“不知这沈起是个什么样的人物?”韩冈问着。

王韶摇了摇头:“不清楚,没打过交道。只听说过治才不差。”

沈起才能不差是肯定的,能做到陕西都转运使,就证明了他的能力。一般来说,能主持转运司的官员水平都不会差。转运司又称漕司,主持天下各路钱粮财计和运输,关系到国家命脉,基本上都是会选用处理政务手腕出众的官员,而不是名气高声望隆的君子清流。

比如如今主持均输法的六路发运使薛向,他是荫补官,而不是进士出身,两年来没少被反变法派骂过,司马光、吕公著都指名道姓的弹劾过他。但薛向照样稳稳坐在管理汴河运输的要职之上,谁也动不了他。究其因还是因为薛向是如今朝中首屈一指的理财名臣,在财计、物流方面的能力无人可比,难以替代。

就如薛向,沈起能做到陕西都转运使,他的才能值得肯定,但这不代表他的人品,能力和品德是两码事。

还是等着看吧,韩冈想着,希望能比环庆的事有趣一点。

今天刚刚收到消息,环州和原州同时出兵,共击环州蕃部折平部,大获全胜,斩首近千。韩冈可以想见,李复圭的脸应该绿掉了。

环州知州是种诊,而原州知州是种诂,种家大郎和二郎一起动手,合力共击一个蕃部,虽然韩冈没听说过折平部这个名字,但他还是很同情这家倒运的部落,竟然犯到了种家将的枪口上。种家为了清洗李复圭栽给种詠的污名,这段时间已经要拼了老命。而折平部不知犯了什么事,变成了送上门来的猪羊,给种家将好生料理了一番。

虽然环原二州紧邻着,但毕竟不是同一路,一个是环庆路,一个是泾原路,种诂、种谊绕过两路的安抚使——其中一个就是李复圭——而相互联络,其实还是犯了忌讳。但胜利者不受指责,就算是在武将最忌讳主动行事的北宋也是一样。这一战后,至少不会再有人拿种詠来说事了。

