\section{第23章 弭患销祸知何补(二)}

初更的时候,还在城西医院的韩信让人送信回来了,带了最新的伤亡数据:死者十七人,轻重伤两百一十四人——这是城西医院收治的人数。

在报信的家丁描述中,城西医院中哭声阵阵、哀嚎不绝的惨象,跟西北战争之后,疗养院中的情况也差不多。

“你再去跟医院里面说,尽全力救人,不要吝啬医药。”韩冈吩咐带信回来的家丁,“人命关天的事,能多救一个是一个。”

那家丁回话道:“禀端明,金簇正骨两科的医师和医生都已经到了医院中,三十九人全都到齐了。”

在金簇、正骨两科,也就是外科上,来自于军中的医官们的技术,远远要比寻常给人看病的医官强得多。由他们来救治伤员,结果过会更好一点。

韩冈现在的身份不方便去城西医院,否则未免会有干扰开封府的嫌疑。尤其是蹴鞠这项赛事本就出自于韩冈,其中的一方球队又是跟他有千丝万缕的联系,为了避嫌,只能在家里下命令。这也是为了方便日后使力,现在若是牵扯进去,之后有些话就不好说了。

“至于账单,这是齐云总社的责任。不要向病人收,收账的单子送到兴化坊去。”韩冈转回来对何矩道,“你也别在这里待了,也去兴化坊。想来这时候不会没有人在。”

何矩立刻答诺应承。正如韩冈所说,这时候的兴化坊中的齐云总社会所,聚集了绝大多数的会首和球队东主,正等着他带着韩冈的吩咐回去呢。

这个时候,能在此事上说得上话的重臣,也就那么几人。韩冈虽然从来不干预齐云总社的事务,但到了危机关头,还是得求到他的头上。料想韩冈也不会眼睁睁的看着他的一番心血付之流水,甚至被人拿来当做攻击自己的。

何矩之前被遣去确认消息的时候,就已经与几位能联系上的会首通过气了,要尽量在韩冈这边得到一个应对的章程出来。

得了韩冈的吩咐,何矩和来报信的家丁正要走,外面却通报韩信回来了。

心道莫不是医院中又出了什么大事,招了韩信进来,韩冈直接就问道:“韩信,你怎么回来了?”

“端明,不好了。”韩信可能是赶得很急,有些气喘,脸色还泛着青,“这一回出事的里面有一个贵人!”

韩冈脸色一变:“谁?死了还是伤了?”

“南顺侯……”韩信干咽了一口唾沫吗,“肋骨被踩断了好几根,在医院里伤重不治。”

听到所谓贵人的身份,韩冈神色立刻就放松下来,“南顺侯?只有他吗?”

韩信愣了一下,十七个死者里面就有一个开国侯,难道还不够?

“没关系,没关系。”韩冈笑了起来,向外赶着人:“这一位死了反而好,去做正事,没关系的。”

韩信和何矩带着满头的雾水离开了。

韩冈将桌上的稿收起来,神色间也放松了一点。韩信没有去过南疆,所以在这件事上有些糊涂。换作是跟着他一起去岭南的几人,就不会有这样的疑问了。

这几年,交州一直都很安定,交趾人也被分封在交州的左右江三十六洞诸蛮死死压制着。经过几年的垦殖,白糖、水稻,每年的产量都在稳步提升。不过在诸多种植园中的交趾奴工,已经死了有两成还多。

如今有不少在海中做过的贼人,受到巨利的驱动,已经开始从环南海的诸多国家手中搜集奴工,为交州数以百计的种植园提供劳动力,洗白了自己的身份。其中最大的受害者,也就是离得最近的占城和真腊,已经几次派人来京中哭诉,尽管有些朝臣认为要为藩国做主才对,只是天子对此不予理会,两府之中也没有读读傻了的呆子。

稳定并顺利发展的交州,使得朝廷并不需要一个活着的南顺侯。若是死于疾病,或许还有违命侯和邓忠懿王的前例在,会让世人疑其死另有他因,与朝廷名声有碍,不过若是死于意外倒是正合适了。

……………………

“南顺侯死了?”赵顼比韩冈还要早一点收到消息。京城中出了这么大的事,也不可能瞒着他这个天子。

赶来禀报的石得一还有些气喘:“回官家,南顺侯是在乱中被人挤倒,后被踩踏受了重伤,被送到西门医院后伤重不治。”

赵顼眼神闪动:“……确定是意外?”

“应该是意外。”石得一道:“都这么多年了,也没听说过有人对其谋图不轨。且乱民有数万之多,就算有心谋害,事到临头也没办法。”

“一场比赛观者就有数万之多?”赵顼神色一变。

石得一道:“官家明察,人数只多不少。尤其到了季后赛的时候,听说每场都有几万人挤不进球场。”

赵顼当然知道蹴鞠有多受欢迎,但他想不到比赛规模已经有那么大了。

每年春时,天子都会驾临城西的金明池,观看水军演武,以及各项争标的赛事。而在这两年,除了寻常的水中争标外,还多了一了蹴鞠争标。但赵顼怎么也想不到,比起在他面前的比赛,民间的比赛规模竟然更大,而且是大得多。普普通通的地区联赛,竟然能有数万观众。

陪侍的宋用臣也在身边对赵顼说道:“官家,东京蹴鞠联赛的参赛球队,包括开封、祥符两赤县在内,总共有两百七十四队。这是在齐云总社报了名的,那些没挂名的就更多,如今的街巷中都能看见小儿踢着球。”

随着参赛的队伍的数量越来越多,旧时的规则已经不能附和现实的变化。但一时还没有定下来。如今还是将蹴鞠联赛在京城中按照厢坊分成了多个分赛区,然后让头名出来参加季后赛。

而除了祥符、开封两县以外,其余二十县的联赛也归于东京城中的齐云总社管辖,但比赛还是独立的。毕竟是隔得太远了。

“都没想过将府中所有的球队聚起来比赛?”赵顼看起来并不是很在意今天发生的惨案,也没有什么

“回官家,要从开封府治下各县几百里的路上跋涉,实在有些难。”宋用臣说道。他仗着是正得宠,有些话可以放胆直言。“不过若是什么时候轨道能将开封府全都连起来,从南到北、从东到西,横贯开封府只要一天,什么时候就能有统一的开封府的联赛了。”

就算区域范围仅仅局限于一府,但开封府有二十余县,比寻常的州府要大得多。旧日曾名为京畿路,乃是一路之地。数百里方圆的开封府,不论是赛马,还是蹴鞠,都没办法一统江湖。只能各县分开来各自玩自己的。

其实大部分州府多半如此,总不能为了一场比赛,在路上奔波三五日。基本上都是一个县内部的球队比赛。只有少部分地域狭小的州府,会在春播结束后,组织各县的头名去州城里用几天时间来踢季后赛。

赵顼听了汇报,不置可否。尽管向他报告的石得一明里暗里都在说局面因为是两支球队维持不力才变得那么乱,但他私心中并没有深入调查此事的打算。

“这件事就让开封府处置。”赵顼没有什么心情的挥了挥手,

罚不责众,尤其是像这样的群殴,最后造成的骚乱,根本就抓不到真凶。到时候,除了拉人顶罪,并没有解决问题的手段。赵顼无意看人欺君,根本记不加理会。

他现在所关注的,一是新学,一是资善堂。至于其余,都可以放一放。

“朕倒想看看钱藻是怎么处置这件事的。这件事若是办得不好,他也没有必要在开封府的位置上多留了。”

……………………

但东城一角的小院中,正有几人围坐在幽暗的灯火下,脸上都有着难掩的兴奋。

他们都可以算是消息灵通人士,平日里互相之间又有往来。一听说在西城外的球场上,发生了大规模的伤亡事件。他们便立刻互相遣人联络,想要在其中为自己或是友人,找到一个利益最大化的可能。

“天欲灭韩冈。否则如何会有今天的这一桩事?”

“听说是,今天的事,多半跟韩冈一样牵扯不清。”

“球队胜负、进球多寡,世间多有为此赌赛。诱人赌博,大坏风俗,韩冈此人当真适合侍讲资善堂吗?”

桌边众人眼神中全都变得深沉起来。

若是想要阻止韩冈,只要声势足够大,出面的官员足够多,就算是天子也不可能强行安排他去资善堂任教。士林舆论若是一面倒,中舍人、翰林学士,哪一个愿意坏了自己的名声为天子草诏?

“还是先将蹴鞠联赛给停下来,等待朝廷的处置……赛马也该一样。每天都是几万人聚集,什么时候出事都不奇怪。”

“看看钱藻会怎么处置了?他若是胆敢在此事上徇私枉法,一纸弹章可是少不了他的!”
