\section{第46章 龙泉新硎试锋芒(一)}

【第一更,求红票,收藏】

从崇政殿出来,王安石疑惑丛生。

虽然赵顼在崇政殿议事后照例将他留下来单独奏对,并说了不少好话加以安抚,但王安石很明显的感觉着年轻的皇帝有些心神不宁,这在过去,并不多见。真不知吕公著昨日究竟说了些什么,让天子变成了这副模样。

回到政事堂后,曾布就赶了过来。就在王安石留在崇政殿中的时候,他打听到了吕公著昨日奏章的内容,一等王安石回来,就大惊失色的赶过来通报。

困扰天子的原因找到了,而王安石也惊到了。他当真没先到,他的老朋友为了反对变法,竟然连这等两败俱伤的策略都用上了。

要知道,也就在两年前,吕公著曾经为了王安石,在新近即位的天子面前说过不少好话,为他的进京秉政助了一臂之力。但如今,几十年的交情,却成了天边消散中的浮云,只能追忆,无法重来。

“吕晦叔这是何苦?”王安石叹着气。这根本是损人不利己的做法,吕公著既然这么做了这么说了,他本人肯定不能再留在京城,一个月之内必然要出外。至于变法派,也免不了要吃苦头,天子心中的犹豫就是对变法最大的伤害。

但最可怕的问题,还是他在天子的心中埋下了一条毒蛇,不但会让赵顼怀疑起群臣的忠诚,甚至天子还会因此而疏离至亲骨肉。皇权之争,毫无亲情可言,而吕公著一番言辞的最后结果,就是让天子无法再去相信自己的亲人。

“韩稚圭不知会怎么做?会不会上章自辩?”曾布问着。

吕惠卿走了进来,他也是听到消息匆匆赶来的,他接口道:“韩琦怎么做都错,最聪明的做法就是当什么都没听到,什么都没看到,也好给天子台阶下,否则闹起来后,韩琦左右都是罪名。即便吕公著本心不是针对他的也是一样。”

王安石不关心韩琦会怎么做,他在担心赵顼。变更法度需要天子坚定不移的支持,但吕公著的奏章,却是要让天子怀疑起变法会不会动摇他的皇位。

“不打消天子的心头之疑,做什么都没用。”曾布叹着气。

“官家又没有明说出来,现在跟过去也没什么不同,继续将事做下去,用不着想太多,等有了成果,吕公著的谎言不攻自破。”

“吉甫说得甚是。”王安石最后还是放弃了去考虑这个让他头疼的问题,至少赵顼现在还没有表现出要废弃新法的苗头来,他指了指桌上的一份奏折:“看过窦舜卿的奏章没有?”

“是一顷四十七亩的事吧?”吕惠卿点了点头,王韶的一万顷到了窦舜卿嘴里就变成了一顷,这事朝堂上都传遍了,御史们闻风而起,今天就递上去了五六封弹章。但吕惠卿对窦舜卿的说法半点不信,他家是福建大族,田产为数不少,一顷四十七亩究竟才多大,他一清二楚。

“这窦舜卿还真敢说!”

“说谎不碍事,圆不了谎才会是问题。”曾布冷笑着,窦舜卿敢这么信口胡言,是因为他有底气,“窦舜卿父子两代皆在军中得意,父为横班,子任贵官。论人脉,可比王韶深厚百倍。他自从军以来,就靠着一点微末之功,便一步步的跳上了正任观察使的位置。这样的升官速度,不是世家子弟,谁能做得到?”

曾布虽然也是世家出身,几个兄弟和内弟都陆续做了官,但他们无一例外都是辛辛苦苦考进士出头的。自他祖父辈起,南丰曾家七十年来出了近二十个进士。故而他分外看不起窦舜卿这等靠着父荫,而身居高位的无能之辈。

可曾布也很清楚,窦家两代人几十年编织起来的关系网,足让窦舜卿的荒谬谎言变成天子心目中板上钉钉的事实:

“不论派谁去重新丈量土地,窦舜卿怕是都能跟他们拉上关系。如果他们跟窦舜卿一个声音又该怎么办?所有人众口一词的话,天子还能不信?

还有陕西转运司那边,转运副使陈绎至今不肯在鄜延环庆推行青苗贷,而且还以供给绥德的军资粮饷难以支撑的名义,大肆在关中各州设卡抽税。如今刚过正月,道上难行,他这么做的影响还不大。等到春暖花开的时候,路上商旅渐多,不知会有多少人会怪罪到横山开拓之事上去。”

曾布忧心冲冲,就跟京师里一样,关西局势最近越发的严峻,反变法派仿佛联络好的一般,就赶在年节前后一齐发难,让人措手不及。

现在想想,秦州那边的窦舜卿是韩琦的乡里,自然跟韩琦同声相应、同气相求。没有韩琦,没有他父亲留下的余荫,凭窦舜卿的那点芝麻粒大的军功,根本做不到现在的官职上——他在京东防备海盗,招募了三百人,斩首也不过四十余,而昨天提到的韩冈,连同王韶在私信中提到的西贼内奸余党,他的斩首数都已超过五十了!韩冈才一个从九品,可窦舜卿又是什么地位?

而陈绎是开封人,别的不说,惯看朝堂风色可是京师本地人特有的本事,外地人不历练个几十年却学不来。即便不论他与京师豪商、宗室之间,可能有的千丝万缕的联系,只看如今的朝堂动向,他也必然会主动投靠韩、文、司马一派。

曾布能看到的,王安石自然不会看不到,但他倒能放得下,“王韶那边就先看一看再说,天子已经遣了王【和谐补丁】克臣、李若愚两人去秦州重新体量。等他们回来再做计较。”

“李若愚?”吕惠卿眉头一皱,心道怎么选了这人,“下官记得他曾经在广西任过走马承受,而当时的广西提点刑狱兼摄帅事的……确是李师中。”

“如果李若愚胆敢偏袒窦舜卿,一同欺君,那就再换一人去。朝堂上那么多人,总能找到与李师中、窦舜卿没关系的。”李若愚和王`克臣已经走了,不可能再追回来。王安石知道他现在能做的,就是在他们把消息传回来之前,先给赵顼做个预防,以便让赵顼同意再派一队更为公正的使臣去秦州。

“绥德那边呢?陈绎怎么办?”曾布又问道。

“陈绎其人好功名,无甚德行。他敢这么做,是看着朝廷风向现在是往韩、文那边吹,等到天子决意一下,他必然会倒过来。”

“那怎么办?放着他不管?”曾布不以为然的反诘道。

吕惠卿摇头:“还是将其调回京中,省得给绥德添乱。陈绎品行虽陋,但按狱还是有一手的。”

……………………

又是一桩出乎韩冈意料之外的……意外。

当韩冈与路明一起回到驿馆时,走出来迎接他们的第一个是堆着谦卑笑容的驿丞,第二个便是看起来一脸心浮气躁模样的王旁。

“衙内怎么来了?”韩冈心中起疑,跳下马来。

王旁上前道:“是家严让小弟来请韩兄!”

“相公今日可有余暇?”

王旁拱了拱手,算是道歉:“家严翘首以待。”

韩冈哈哈笑了两声:“相公既然有招,又是衙内亲至,韩冈哪能不识抬举。”

王旁的模样更显恭敬:“……如蒙韩兄不弃,还请直呼小弟表字便可。”

韩冈微微一愣,这实在太不正常。但王旁既然这么说了,却不能不给他面子,韩冈郑重行礼道:“仲元兄。”

王旁一还礼:“玉昆兄。”

路明在后面看傻了眼,而驿丞也惊得张大了嘴,显然他们是因为看见参政家的衙内对一个选人低声下气的去结交,而震惊的难以名状。

“时候已经不早,家严也该从政事堂回来了,玉昆兄还是与小弟早点走吧。”

韩冈想了一下,抬了抬袖子,上面还有些方才在樊楼喝酒时留下的污渍,他笑道:“还请仲元兄少待,且容在下更衣。”

说罢,便丢下王旁走进驿馆中,路明也慌里慌张的跟着走了进来,他紧追在韩冈身后问道:“韩官人,你真的只是跟王衙内下了两盘棋?”

‘下了两盘棋就有这等用?’韩冈冷笑,没有回答。

‘这怎么可能?!’

王旁当是代表他的父亲来的。昨日明明是王安石找自家去的,但最后却让自己白坐了许久,今天让王旁亲自来,大概是有赔罪的意思在。

这样的做法说是前倨后恭就有些酷毒了,一国参政能对从九品的选人尽到礼节,韩冈的自尊心还是被满足了不少——‘未能免俗啊。’韩冈自嘲的笑着。

来了这么一手,韩冈对王安石顿时生起不少好感,如此地位,如此名气,王安石却没有摆出一副高傲的架子来,确实让人尊敬。

当然,这样的想法只是一闪而过。‘礼下于人,必有所求’这句俗语,韩冈记得更为清楚,并没有因为受宠若惊而昏了头去。

韩冈不知王安石到底是为了什么事,才这般殷勤。他一边换着衣服,一边心里也在来回盘算着。不管怎么说,见着王安石后就能知道缘由了。

换好衣服,李小六正好也回来了,省了自己让路明转口,韩冈直接吩咐他去张戬家报个信,最近天天都去张戬府上,今天去不了,按理得打个招呼。

将琐事一一交代完毕,韩冈终于从驿馆中出来,对着王旁歉然一笑:“累仲元兄久候了!”

