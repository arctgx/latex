\section{第14章 霜蹄追风尝随骠(五)}

南北相距近两百里,消息的传递有一天以上的延迟。就在韩冈收到麟府军驻屯子河汊的消息时,折克行已经领军进入了旧日的丰州城。

就算在西夏的手中,丰州依然是边境上的一座重要城寨。只是防御的对象,从党项人变成了宋人。不过当年西夏攻打丰州的时候,王家上下抵抗激烈,使得丰州城损毁严重,虽经修复,但几十年下来,还是没有恢复旧观,使得城中守军完全没有信心。

西夏灭亡、国土尽丧的消息,也已经通过各种途径传到旧丰州城中。城内仅存的八百守军,根本无心守御,一等宋军的前锋斥候抵达城下,便立刻开城投降。

当折克行帅折家军出阵以来,这是第六座不攻而克的城寨,而到此时为之,被宋军占据的城寨,正好也就这六座。

出兵一来,一仗都没有打,便将丰州旧地收入囊中。可尽管一桩桩功劳在面前闪着金光,但折克行的神经却越来越紧绷。

以辽人的性格,他们会愿意看着宋军攻城掠地,而自己却只能在旁坐视?

这当然不可能,所以便有了耶律乙辛让人出乎意料的神来之笔。现如今宋人形如虎口夺食,驻守在西京道的辽军肯定是不甘愿的。

‘得快一点才行。’折克行站在荒草丛生的城头上,下意识敲打着雉堞的手,难掩心头忧急。

“大人,唐龙镇那边来人了。”折可大走上城头,他是折克行的长子,未来有望继承折家家主之位的人选之一,“是来佛奴的次子,带了十五匹好马和牛角、弓箭做礼物。”

“来佛奴终于肯低头了?”折克行微微笑了起来,却是带着讽刺的味道。

唐龙镇是南北蕃部互市良马之所。掌握了唐龙镇的来家依靠这一财源,在屈野川一带的蕃部中,拥有数一数二的雄厚兵力,当初就连丰州王家也不得不让他们三分。

来家如今的家主,在云中道上的酋帅中是有名的油滑,几十年来在辽夏两国游走自如。不过还比不上他的先祖,当年旧丰州还在大宋手中时,来家是三面下注,同时受了宋、辽、夏三国封爵。元昊攻丰州,甚至没动来家一下。

“恐怕还是两边倒。辽人不得罪,大宋这边也不得罪。”

“我想也是,来佛奴这个人,不见黄河心不死,绝不是那种会铁了心投靠哪一家的人。”折克行嗤笑一声,“不过由不得他,也不看看现在的局势和他家唐龙镇的位置。屈野川的上游是柳发川,就在唐龙镇西北。那一片地向北过了一重山口,就是辽国东胜州的河清军【今鄂尔多斯东胜区】。地势如此,岂能再让他来家再做个墙头草,继续两边卖好?要么降顺,要么就随嵬名家一起去黄泉!”

折可大精神一振:“是要灭了来家?!”

折克行摇摇头,“先看看来佛奴是什么说辞。只要他愿意让官军入驻唐龙镇,饶了他也没什么关系。”

折可大张了张口,却忍了下去。唐龙镇是来家的命.根子,让官军驻扎唐龙镇,对来家来说,这不就是引狼入室?肯定是不会答应的。这是明摆着在欺负来家。

不过折可大没有为其叫屈的意思,这样的强势,倒是对了他的胃口。

……………………

“府州十一堡,麟州十二堡,等到收复丰州之后,都可以减少驻军的数量,投注到丰州去。要不然,多出来的寨堡就没办法填满了。”

韩冈白天时经过了神木寨,先对其下方深藏的以亿万计的煤炭资源感慨了一阵,然后快马加鞭赶到麟州的新秦县。从新秦县沿着屈野川北上,还有一个连谷县,处在宋夏两国的边界上,驻扎在期间的兵力不在少数。

不过眼下的当务之急还是屈野川上游,要有足够的兵力驻守其间,而后方则是放置足够数量骑兵或是龙骑兵,能够在军情紧急的时候随时支援前方。

“关键的位置还是浊轮川全线和屈野川上游的柳发川,两条河都是有路直通辽国,不得不全力防备的地方。”

“东胜州现在的主要兵力都去了黑山,短时间内,暂时不用担心他们。但日后就必须小心盯防,东胜州的驻军中,可是有着”

旧丰州的对面是辽人的东胜州。

之所以有这个名字,是因为从麟府丰云中三州,到西夏的黑山威福军司的河南区域,再到辽人的西京道的西北部,整个方圆近千里的土地,也就是黄河几字的右上角,在唐时都是属于胜州这个辖区。辽国现在占据的就是唐时胜州的东部。

东胜州是辽国西京道的重镇,常年驻有上万大军。不过因为要平定黑山的党项人的反抗,大部分军队都给调走,只有边界上的武清军守军没有调离。

韩冈并不知道辽人的想法,但他还是不觉得耶律乙辛在捡了大便宜之后,还会冒着风险,挑起一场很难确定胜负的战争。当然,大宋这边也不会主动挑起战事,甚至要竭力避免发生这样的情况。

两边都不想要一场战争,但最后到底会不会有战争,却怎么也说不准,谁也不敢保证会不会出意外。真要到了骑虎难下的局面,上面想不打,恐怕也压不住阵脚。

“先做好背水一战的准备,不开战是最好,但开了战,我们也不需要怕。”

韩冈已经做好了与辽人开战的准备。秋冬本来就是骑兵活跃的时节,养精蓄锐的辽军铁骑,在头脑发热的将领指挥下,当真能撕破旧日的约定。

不要指望所有人都能权衡利弊,越到高层的确越是偏向保守和稳重——因为他们首先想到的是防止意外动摇自己的权位——但底下人的想法,却不一定跟上面是一条心。想往上爬的人很多,但位置就那么多,不立下让人印象深刻的功绩,怎么能压倒一干竞争对手?

在大宋是这个道理,在辽国,道理当然也是相通的。对功劳的渴求,贯穿了每一名士兵的心中,他们只信服能将他们带上战场、最后又给他们带来胜利的统帅。就是不知道萧十三是不是这样的类型。之前在雁门关与其打过几次交道,虽没有正式会面,但对耶律乙辛的这位亲信,韩冈也算有了几分了解。

“相信折府州应该做好准备了。”韩冈相信折克行的头脑,不会不加以防范。

“肯定的。龙图可以放心。”折可适打着包票。

李宪那里还没有消息,但想必也已经做好了准备,只等朝廷的令旨和封赏一到,便立刻领军北上。

折克行那边的进展太过于顺利,让人都怀疑其消息的真实性。但一战不打,就这么松懈下去,等到真正的敌人抵达时,可就连守城都成了痴心妄想。折克行再怎么筹备完全,也很难拦住全力南下的辽军。

‘得让他们将皮绷紧一点才行。’韩冈想着。

……………………

结束了与来家使节的会面,折克行在黑暗中沉思着。

出乎折克行的意料,来佛奴虽然是棵墙头草,但却是不折不扣的聪明人。他派出来的信使,对折克行开出的条件,几乎是全盘接受。谈判顺利的难以想像,甚至到了让人怀疑的地步。

来佛奴的儿子转达了他父亲开出来的价码,竟然愿意以安排来家南迁安置为条件,让出唐龙镇及附近方圆数十里属于来家的土地。

官军进驻唐龙镇,同时还要分兵去把守柳发川。浊轮砦、暖泉峰,这都是需要把守的据点。同时还要进一步整顿周边蕃部,以减少他们倒戈的可能。摆在折克行面前的差事,想要解决得好,都不是一件容易的事。

“幸好韩龙图之前调拨了一批铁甲过来,否则手上的这两万人马,怎么也不敢分兵的。”折克行笑叹道。

让折可适在韩冈身边做幕僚的确是做对了,虽然朝廷对折家这个事实上的藩镇一向提防约束,就连江南那点不成气候的禁军,都配发了全套的板甲和钢刀,但折家依然只有四个指挥的具装步兵。

但韩冈上个月大笔一挥,便从府库中调拨了四千步卒的全副装具,板甲、头盔、斩马刀、腰刀、神臂弓,一应俱全,而且还包括了指挥使以下各级军官的装备。折家所掌控的核心私兵,全都顺利换装。

精良的武器带来的不仅仅是战斗力上的提升,士气也随之大涨。折家之所以上下一致同意收复丰州失土,也有很大一部分原因是因为装备提升的缘故。

折可大道:“其实四千也不算多,灵州城下一次就丢了近十万,也不见有多心疼。”

“同样的一贯钱,在穷人和富人眼中,分量是不一样的。”折克行摇头笑了笑,又道:“以我折家的六千子弟兵为本,辅以麟府军剩下的两万人马,北面的契丹人只要敢杀过来的,就不要再想回去!”

