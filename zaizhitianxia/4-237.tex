\section{第39章 遥观方城青霞举(二)}

【第二更。】

结束了对方城轨道的检视,天色已然不早,韩冈、沈括一行便在方城县歇了下来。

在方城知县诚惶诚恐的招待下,吃过拖长了时间的晚饭,回到寅宾馆的房中,沈括喝着消食的清茶,问着儿子沈博毅:“你跟着一起走也有几天了,对韩玉昆,你怎么看?”

沈博毅有些紧张,沉吟好一阵才试探的说道:“韩玉昆的确是奇才。日后入两府不在话下。就是年纪太少,对他眼下的前程恐有阻碍。”

沈括眉头微皱,心下不愉。都是给人说滥了的评语,还有自己说过的话。也就是说,这两天与韩冈的相处,他什么都没看出来。自家儿子见识平庸他是很清楚的,但再一次被确认,沈括心里还是觉得不舒服。

“就这些?”声调有些尖锐。

沈博毅身子颤了一下,连忙道:“只是把年齿放一边,韩冈的胸中的确有一篇治国的大文章……韩忠献【韩琦】在他的年纪,差得不知多远。之后忠献公能方过而立便晋身两府,不过是因人成事,撞了大运而已,之后才显出本事。”

沈括摇摇头,失望道:“我不是要听这些。”

沈博毅神色更加紧张,“孩儿是想说,以他的聪明,难道当真不能做诗词吗?”

“哦,为何这么说?”沈括闻言一喜,对错不论,以自己儿子的性格,能有想法就是最好。

被父亲追问,沈博毅心中发慌。但看见沈括鼓励的眼神,他大着胆子说起自己的想法:“都说韩冈不通诗赋,但西太一宫中的那一首枯藤老树,到现在都没人去认。传说是韩冈,也有人怀疑。但往深里去想,这样的一首小令,纵使如王介甫和欧阳永叔,一辈子又能做出几首?不是对此道不屑一顾的韩冈,谁会放着不认?”

“可他的文章你也不是没有看过,的确是平平无奇,不见华彩。”沈括故意反驳道,“文章讲究韵味悠长,言不到而意到。韩冈的文章却是少有典故,文字也失之于繁芜。按刻薄点的说法,直如胖水牛,臃肿榔槺而不见妩媚。”说着又摇头哧笑了一声,啧啧嘴,“苏子瞻好利的舌头。”

沈博毅争辩道:“初看的确如此,可再想想,读他的文章,可会产生半点歧义?他文章中说的事,又是哪一件不深刻入骨?直是刻意如此写来。而且诗词歌赋写得不好是一回事,能不能写则是另一回事。韩冈几年间,文字有十数万言之多,难道连一首诗一阕词都写不出来?只要想写,乡儒拿着韵书也能拼凑个四句、八句出来,何况进士第九的韩冈!”

“那韩冈为何如此?”沈括转着茶盏,慢悠悠的问着。

“一则应是心不在此,第二当是不想让诗赋拖了后腿。韩冈于诗赋肯定是能写,但多半写得不好,枯藤老树也只是特例,难有可以比肩的第二首。若是滥竽充数,少不了会被一干刻薄之人指着鼻子嘲笑。现在干脆不写,就算有人想嘲笑,又能嘲笑多久?说多了也就厌了。且更能反衬他在其他方面的才华。”沈博毅沉吟了一下,更低的声音说道:“以孩儿看来,韩玉昆外似谦和,实则高傲,根本看不起那一干饮酒作乐多过做正事的词臣。诗赋于他,小道而已,他想做的,是穷究天人大道。区区文名,对他来说,有等于无。”

沈博毅说完,就紧张的看着父亲,等待他的评价。沈括默默等了一阵,见没有下文,视线从茶盏中的浮沫上收回,抬起眼:“没了?”

沈博毅一愣,心虚的小声道:“……没有了。”

沈括笑了一声:“前面倒也罢了,不过能看到最后这一点,也算是不错了。”跟着却又摇摇头,“但还是没有说到正题上。”

看着疑惑中的儿子,沈括道:“韩冈是奇才,学问博通,为人沉毅。不出意外,日后定然少不了一个宰相。但他想做的,绝不是韩琦那般相三帝立二主的元勋,他的心思更大。”

“襄汉漕渠自太宗时两次修筑不成,尤其是第二次,全线掘通后才发现水浅难以行舟,世人皆视方城垭口为天堑,自此搁置百年,直到韩冈出现,才重新将襄汉漕渠提上桌面。你可知他靠了什么天子和朝堂会相信他能将漕渠修起?”

“多级船闸……”沈博毅想了想,补充道:“还有过去立下的声望。”

“对。”沈括点头,“光有船闸是没用的,但天子不知道。霹雳砲、雪橇车、板甲,任何一样拿出来,都是让人叹为观止的发明,能让人吃一辈子功劳,而这些都是韩冈一人的。等飞船上天之后,加上《浮力溯源》营造声势,韩冈在工器、营造上说话的份量,就变得比谁都重,已是由技巧之术进抵于大道。为父,还有苏颂,都远远不如。”唐州知州眼神中闪动着羡慕,“他说船闸可行,没人能驳斥得了。天子只会相信他,不会相信别人。世人也只会相信他,不会相信别人。”

“韩冈是先拿多级船闸出来,等天子和朝廷意动之后,又将轨道拿出来,告诉天子,可以先拿轨道替代漕渠在方城垭口的那一段难关。既避免了开辟漕渠在长期的工程中受到干扰,更让轨道不再局限于矿山和港口,从此有了更为广阔的用武之地。”沈括叹了一声,“这一步步都是按着他的计划来的。”

“大人是在说韩玉昆从一开始就在想着推广轨道?”沈博毅问着。

“确切点说,应该是一石数鸟,开辟漕运,自是有功——与中原更加畅通的联系,还能稳定他主持夺占交州——而推广他所发明的轨道,也同样有功。更重要的,轨道推广后,还能给他带了更大功劳,实现他的目标。”

“……什么目标?”

“你可知道有轨马车真正的用武之地不是在京西……而是在一片坦途的河北。一名兵卒,连同战具在内,总重也就在两百斤。一个指挥按五百人算,不过十万斤。不过人不是货物,不可能两三趟车就运走。但像方才的车子,六匹马、五节车厢,挤一挤,载一百人没问题吧?一个指挥,也只要五趟车,三十匹马。一万人也就一百趟车,六百匹马,几个时辰就能装完上车了。”

沈括喝了口水,见儿子听得专注,就继续说道:“再算算速度。有轨马车按只要能做到按时换马,一天不停歇都可以。一个时辰三十里来算,十二个时辰就是三百六十里……想想河北才多大?如果用轨道将河北各州府连接起来,两天,最多三天,就能将一万全副武装的大军,从黄河边的澶州送到最北端的定州。”他声音猛然拔高,“契丹人的骑兵全速前进时,也就这个速度啊!”

“更别说,运粮有多方便了。”沈括叹了口气,叹气声中满是长江后浪推前浪的感慨,以及深深的敬服,“明白吗?只要方城山这里见了成效,韩冈转头就能让天子点头同意在河北铺设轨道。一旦开始建设轨道,进而投入使用,几千上万匹挽马从哪里来?——只有熙河。韩冈的老家。扩大茶酒易马的交易,能进一步稳定了熙河。”沈括斜睨着一脸震惊的沈博毅,“怎么样……又是一石数鸟。”

“拖着为父来检验轨道,韩冈其实已经将自己的想法和盘托出。”沈括看破了韩冈的盘算,也是有了决断,“依眼下的情况,为父肯定要为他奔走鼓吹。不仅是铺设轨道以便用武河北,甚至是在气学上,为父也得站到他的一边。大哥儿你跟着他,好好学着点。里外都留个人情,日后也有好处。”挥了挥手,“你先回房去想想,日后在韩冈身边该怎么做。”

沈博毅不敢多话,躬身告退,走出去时还是沉浸在震惊之中。哪里能想到韩冈光是要打通襄汉漕渠,私底下能有这么多想法。

儿子离开,沈括从怀中掏出一个小布套。打开来,里面的东西银亮亮的反射着灯光,自己的相貌,也在其中被照得纤毫毕露。

方才教训了儿子好一阵,看似是觑透了韩冈的一切,但实际上,对沈括来说,韩冈身上的疑团更多。

学问不说了,张载肯定教不出来,要么归于天授,要么就是像韩冈自己说的,是格物致知的成果。而韩冈所拥有的势力,更为让人疑惑。

圆圆的水银镜只有巴掌大,套在软布套中,用的时候拿出来,不用的时候收在套中,不用担心划伤镜面。

格物致知并不算什么,韩冈在古书中找到汞锡齐的制法,并用来造水银镜,这一点也不足为奇。唯一让人惊讶的,是韩冈从哪里找的人为他制造镜子。

一个十年前还是穷困垂死的灌园子,哪里来的人手?两条腿会吹拉弹唱的清客幕宾到处都是,但双手上有把子好手艺的工匠,能够炼制水银的匠人,这样的人才,可不是想找就能找的。沈括出身官宦世家,但他养家里的几个清客,可没有一个有这等本事。

而且拥有了这项发明,不去设法保守机密,反而毫不在意的说给外人听。要知道,这可是能养活一个家族数代人几十年的宝贝,可比在家里挖个坑将黄金白银埋下去有用得多。

放弃聚敛钱财的好手段,却又能收拢有用的人才,这完全是相对立的两桩事。对于沈括来说,韩冈手上掌控的资源才是让他觉得不可思议,不过这不便与儿子多说,不小心传出去,很可能就会恶了韩冈。

将镜子收起,沈括双眼定定的看着灯火。韩冈帮了自己这么多,眼下的情况,自己也只有站在他的一边。只望韩冈能达成他自己的目标,日后自家也能藉此摆脱现在的困境。

