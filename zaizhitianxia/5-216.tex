\section{第24章 缭垣斜压紫云低(八)}

“我看到了什么?那是牛吗?”

大庆殿前的广场上,上千名的官员汇集于此。此外还有护卫宫掖的班直护卫、上四军的兵马,总数上万。但这么多人马,放在大庆殿广场上,却一点也不嫌拥挤。华阴侯赵世将是其中之一,他望着大庆殿前台陛下的玉辂,深深皱起了眉头。

他视线的落处不是金宝缀体的玉辂,而是车前搭着车辕的几匹马。确切的说,是中央靠前的那一匹有着如同绸缎一般的淡金色皮毛的高头大马。赛马总社的会首,东京城中号称最知马性的宗室。在他望着应该是东京城中最为高峻的马匹时,眼神和言辞一样,全都是不以为然。

“三一,你不能小声一点啊。”身边的同伴,同时也是同族的亲戚,听到赵世将的声音,一下就心惊肉跳起来,“那明明是浮光啊?”

“浮光是大宛种,轻捷善奔,神骏无匹。将浮光养得肥水牛一般,这是为了养大了吃肉吗?”赵世将低下去的声音充满了痛心疾首。浮光依然是丝绸一般能反光的光滑皮毛,但完全看不到肌肉的轮廓,充满了油脂的身体看起来的确跟牛差不多了。

“秋天马上膘啊。”

“是上膘不是养膘。一天不溜个十几里,哪匹马能养得好?动得少,病就多。人和马可都一样。”

“别说嘴了。”另一名金吾卫上将军在旁低喝,“想接弹章也别选在这时候!”

听人这么一呵斥,赵世将也闭上了嘴。只是眼睛依然在瞟着大庆殿前的浮光,难舍难分。

不过是祭天而已。这句话赵世将没说出来,但撇下去的嘴角已经说得很直白了。那份赏赐,担任赛马总社会首的赵世将如今可不放在眼里。要不是不想引人注意,他就直接称病了事。

赵世将身为太祖一脉的近支宗室,除了华阴侯的爵位外,还有一个金吾卫上将军的官职。原本应是护翼天子的环卫官,到了如今已经是安排给宗室们白领俸禄的闲差。但到了今天这等朝堂大典的时候,这等只拿钱不干活的工作,却都成了辛苦站岗的差事。

赵世将手持大钺,身上穿着鱼鳞金甲,头戴金盔,鲜红的披风系在身后。打扮得很光鲜,但架不住寒风直往甲胄的缝隙里灌,冻得他只想跺脚。

赵世将长得身宽体胖,而且很可能是因为出面主持赛马的关系,日日游走于各家的宴会中,一年之内倒是长了二十多斤肉,穿着稍厚一点的丝绵袍就整个人就塞不进甲胄中。不得不换了一身单薄的衣物,可即便这样,原本合身的甲胄也依然被满是油水的肚子撑了起来,连系带都不得不给松开。

用力抽了抽鼻子,赵世将暗忖,这一回祭天回去,说不定就要大病一场,真还不如请假了事。早早的称病,说不起就避免了眼下的寒冻之苦。

此时天色未明,黑沉沉的天空下,广场上只有跳动的火光。天幕中繁星点点,银河在今日也清晰无比。

从天地皆白的暴风月,一转变得朗朗晴空,只用了半日而已。看到这样剧烈的变化,谁能说这不是天人感应的结果?

自然赵顼就是这么想的。就是坐在四面漏风的玉辂中,大宋天子也是一幅好心情。不过随着伴驾的队伍逐步南行,高昂的情绪也渐渐低沉了下去。越来越冷的感觉,让赵顼升起一股几乎连五脏六腑都要被冻结的感觉。

天子出行祭天的玉辂,从唐高宗用到现在,几百年的老古董,保养得虽然好,但坐上去远不如普通的马车舒服。赵顼旧年曾经想将这玉辂换一辆新车,可惜刚刚造好的新玉辂在第一次展示时就因意外而毁损,天意难违,换车的心思就此便淡了下来。

玉辂轻轻摇晃,赵顼想着今天之后的变局。祭天本没有什么,由于是三年一次,也算不上大齤事。等回去后就是宫宴,届时让六哥出来奉酒,在正式场合公开露面,压在心头上好些年的大石也能放一放了。

韩冈低垂着眼,混迹在人群中,沿着御街一路南行。

这一回的暴雪来得太急,偏偏又赶在祭天之前,开封府组曱织人手用了半日的时间,也只将御街正中曱央给清扫出来。天子的车驾行驶在用黄土垫高的中曱央车道上,而行走在御街两侧的马步军,则很是辛苦的踏雪而行。以韩冈看到的情况,应该不止一个人在肚子里面骂娘。虽然状况情有可原,但加上青城军营的事,钱藻的开封知府,或许是做到头了。

正午时分,天子已经站在了上下三层的圜丘顶端。臣子们环绕在圜丘下,更外围,则是千军万马静声肃立,人马衔枚。除了乐班的曲乐声外,就只有一面面旗帜在风中呼啦啦的响着。

冬日稀薄的阳光似乎没有任何暖意,反而更让人觉得寒冷。高旷的晴空下,寒风无所阻挡的席卷而来,带走了身上的每一分暖意。可能今天是这元丰三年一年中最冷的一天,估计汴河河底都要冻透了。平日里养尊处优的官曱员们,享受着寒流的侵袭看,全都是脸色发青发白,有许多人都变得摇摇晃晃,似乎下一刻就能跌倒在地,从此不起。

韩冈的情况,在文武百官中算是不错的。自幼冬日酷寒的西北生长,他倒也不畏寒冷,虽然也冻得很厉害,但风刀霜剑的袭曱击,韩冈早已在过去的军事生涯中变得十分习惯了。他站在自己的位置上,等待仪式的完结。

本官官阶是从四品的右谏议大夫,韩冈所在的幕次,其实是属于言官的行列。尚记得六年曱前的祭天大典,那时候好像也是属于言官一波,乃是正七品的右正言。

幸而一干正牌子的台谏官,他们的本官官阶基本上都是七八品的博士、寺丞、中允,,倒是不用听乌鸦聒噪,但他现在站立的位置,十分靠近乐班。编钟、玉磬、笙、竽等乐器在耳畔齐鸣,不消片刻,便震得人头昏眼花。半个多时辰过去,韩冈只觉得右边的耳朵似乎都要聋掉了。

曲乐一首接着一首,配合着高台上天子的行动。降神时的《高安》,高曱亢嘹亮;天子登坛时的《隆安》,庄严肃穆。《嘉安》为进献玉币伴奏,《丰安》、《禧安》,奉俎、献酒。亚献、终献,《正安》的曲调反复奏起。

习惯了之后,乐曲声渐渐远去,已是充耳不闻。韩冈在宽袍大袖中活动着冻僵的手指,算起还有多久才能结束这套见鬼的郊天大典。

韩冈在太常寺的时间虽然不长,但礼乃儒门的核心,乐是礼不可或缺的重要组成部分。郊祀上的曲乐歌词,他早已熟悉。只听唱到了那一部曲子,就知道仪式进行到了哪一个环节。这一才能,在今天派上了很大用场。被乐曲在耳边轰炸了一个时辰,韩冈已经听不到担任赞礼的翰林学士张璪的号令声,他暗自庆幸自己没有省去了解祭天典礼的时间,否则说不定会不能及时应对。

片刻的停歇之后,编钟轻灵而幽远的声音重又在韩冈耳边响起,引领起乐班重新奏响《高安》,伴唱着‘倏兮而来,忽兮而回,云驭杳邈,天门洞曱开’的歌词,送神而归。

熟悉礼乐的官曱员都精神一震,《高安》送神之后,接下来便是天子降坛。当皇帝从圜丘上下来,这冗长难熬的仪式,自然是到了尾声。

沿着台陛,赵顼缓步而下。在圜丘顶上合祀天地,再祭拜过陪祀的太祖太宗,祭天大典上的核心仪式已经宣告结束。官曱员们垂头望着脚下,用眼角的余光目送天子回到名为大次的帐幕中。

天子回帐,臣子们也回到行宫中各自的房间,换下只有祭天时方才穿上身的玄衣纁裳,穿回正常的朝服。

更曱衣也算是休息,从滴水成冰的室外,回到温暖的室内。虽说乍寒乍暖对身曱体不好,但冻着对身曱体更不好,在火盆边歇了好一阵,韩冈才算是缓过来。只是这也仅仅是中场休息而已,很快就有内侍找过来,一间房一间房的将人都通知到,让朝臣们回到圜丘下的广曱场上。还有回程的几里路,以及剩下的三分之一的环节。

重新排好队伍,站回到方才的岗位上,上万名官曱员和士兵等待着赵顼的出现。时间一分一秒的过去,却不见赵顼的身影出现。

韩冈隐隐的觉得有哪里情况不对。郊祀大典上的每一道环节都是有着严格的时间规定,到时间天子该出来却不出来,那么必然是有什么意外发生了。环目四顾,所有人都在自己的位置上站着。最前面的王珪也是一动不动。

又过了片刻,天子依然没有出现,场中的气氛越来越凝重。正想着王珪会不会离开自己的位置前去探问,大次的帐帘终于动了,赵顼很平静的从帐篷中走了出来。

看到天子出现,不止一个人松了一口气,韩冈也放松的轻叹了一声,看来是自己多心了。虽是这么想,但不知为什么,他的心头已经悄然蒙上了一层阴云,似乎有什么事要发生的样子。

天子的车驾从南郊的青城行宫一路返回皇城。半日前聚曱集在大庆殿广曱场上的万余人,又重新站到了大庆殿前。

皇帝已经回到大庆殿,但郊祀的仪式并没有结束。王珪作为百官之首,动身进殿,而其他官曱员则依然是站在原地,等待仪式的下一步。

片刻之后,当王珪出来后,双手上便捧着一份赦诏。在金吾卫的护卫下登上宣德门城楼,向天下亿兆元元颁布天子德音。

王珪下城缴旨,接下来大庆殿中门大开,朝官们鱼贯而入。宫宴的宴席已经在大庆殿中布置好。从高到低,依照品阶、班次,在自己的位置上坐定。

作为今日大典终结的宫宴终于开始,王珪起身,手持金杯,率领群臣,向天子敬酒。

赵顼也举起斟满酒的金杯,正要抿上一口,忽然间脸色陡然一变。以五爪蟠龙为外饰的金杯,竟脱手而出。

当啷一声脆响,惊动了整间殿堂。

