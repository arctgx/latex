\section{第六章 见说崇山放四凶(十)}

“推举?”

让群臣推举宰辅?

向太后咀嚼着韩冈的提议,一时反应不过来。

王中正更是大皱眉头。

举荐之事,世所常见,也是朝廷除磨勘铨叙之外,任用官员的重要渠道。但举荐从来都是高级官员举荐低级官员,没有说由一群低品官员推举出一名高官来。

就像宰辅和各路监司长官推荐僚属,御史台的正副和翰林学士举荐谏官,无一例外都是高官举荐,所以才会有宰辅门前的车水马龙,所以才会有罗织党羽之说。

方才向太后让韩冈举荐,也是因为韩冈是确定要入两府。而且若将站在西府班中的宣徽使也算作是宰辅序列,韩冈等于是两为辅弼,已是宰辅中的老资格了。

可是让低级官员推举高级官员,就不一样了。

这往大里说是乱上下之序,平常而论,也是有悖常例,属于非常之举。

如果韩冈直接推举朝廷中的哪位重臣,那还好说。可如今让朝臣公推宰辅,日后成了惯例,州县亲民官是不是也可以由当地衙门里的幕职选出,各路监司是不是由本路州县官选出,各军将领,是不是也能有样学样?真要变成那个样子,这朝廷还怎么统治天下?

一帮人推举一人上位,五代时候倒是多见,本朝天下明面上也是着么得来的。现如今,一群盗匪共推一个头领出来,又或是乱党想要个能顶罪的傀儡,推举一个倒霉鬼上来,这样的例子也不鲜见。

王中正久在军中,对军队里面的积弊了解不少。

朝廷一向是厚遇武将,而苛待士卒——相对于普通士兵,对中高阶也就是有品级的武官的待遇,绝对算是宽厚了,所以有意叛乱的武将几乎没有。

本朝绝大多数兵变,都是由底下的士卒因受欺压而开始。在他们起事后,是要挟长官一起叛乱。如果,就从普通军官中推举一人出来,最后朝廷招安,领头的死罪,下面的士兵运气好的话就能逃得一条性命。

就像当年太祖皇帝,在陈桥驿黄袍加身,有说法是早有准备,当时只是故作姿态,但也有说法,赵匡胤是为下属裹挟,在那种情况下,退让不得——即便太祖皇帝事先全不知晓,当亲弟弟和下属拿来了黄袍,他难道还有推脱的可能吗?

这就是五代的惯例。而五代的习俗,有很多延续到今。兵卒裹挟上官的例子,王中正随手就能举出十几个。

当年曾经经历过的广锐军兵变,吴逵虽被称为元凶罪魁,可实际上兵变开始时,吴逵还在监狱中,是他手下的将校们见他无辜下狱兔死狐悲,更重要的是一直受到不公平的对待,所以才会起兵反叛,从狱中劫出了吴逵,裹挟他一起叛乱。

仁宗时保州兵变,也就是郭逵扬名立万的一次叛乱,也是士卒先行叛乱,将几名将领架上去做头目。不肯从贼的几名军官,没一个能活下来。

而类似的情况,五代时更是多如牛毛。当底下的兵将黄袍拿出来后,双方就都没有退路了。太祖皇帝要么穿上去,要么就要眼睁睁的看着众叛亲离。

不过就在王中正满腹猜疑,揣摩着韩冈的用心的时候,就听见韩冈的补充说明。

“臣之所谓推举,只是提供候选者以供陛下参考。方才臣也说了,不论太后是准备用在东府,还是西府,只要确定何处有阙额,便让公推出三人,由陛下在其中挑选一人。”

也就是说,最终决定权依然还在天子或代掌天子权柄的皇后、太后手中。

一口吃不成胖子,一步也走不了千里。现在只是顺便脱身,利用一下机会。潜移默化,才是正道。

何况他的最终目标从来就不在这里。

所以韩冈并不心急。非要弄出什么通行数百年的制度来,那样的人或许有,但绝不是行事极端现实的韩冈。

王中正感觉这样听起来就好些了。不过另外还有种感觉,就是觉得韩冈这是不愿意接受宰辅的举荐之权,然后临时想到的变通办法。

在一转念间,王中正已经想到韩冈到底是从那件事上得到的灵感。

如今蹴鞠和赛马两大联赛,其中的会首选举,即是一人一票。如今用在宰辅的人选上,也不算是别出心裁。如果韩冈的意见传出去,世人只认为韩冈这是在用民间之故智。

荐举之权,看着是好事。但要荐举的对象贵为宰辅,臣子就不可能将这份权力拿到手,一旦传出去,必为众矢之的。宰相那边能答应的可能性也极小。就算通过了宰辅,选了合乎自己心意的人选。但他举荐上来的人,说不定会反过来落井下石,以避嫌疑。

这样就好,这样就好。王中正想着。总比韩冈自己来选,能少去大半非议和争论。

“哪些人来推举?”

向皇后没王中正想得那么多,但大体上还能了解一些。

“既然是为了两府,备选者至少得有两制官的资格,而推举者则不能局限在两制官中,至少得侍制以上官来参与,否则也称不上是众望所归,只是少数人的私相授受。”

“两制……侍制以上官……”向皇后慢慢琢磨着。

内制翰林学士、外制中书舍人。即是官职,也表示等级。就跟侍制一样,过了侍制这一条线,就是重臣。

通常两府晋用新人。若不计外路,只看朝中,三司使、开封知府、御史中丞,以及翰林学士,都是在备选的行列中,尤其以翰林学士居多。

两制以上官,就包括这些人。

“依臣愚见。两制官以上可被推举,在京侍制以上官则皆有推举之权,不过一次只能推举一人。届时在陛下面前,侍制以上官于殿上公推。得举最多的两人或三人中,由陛下选择一人就任。”

约束权力,不如扩散权力。想要压制皇权困难重重,但顺手将太后送来的礼物来个见者有份,那就容易多了。

只要最后的决定权还是在太后手中,韩冈自问他的建议要通过并不难。

“这样啊。”

听完韩冈的叙述,向太后便轻声应答。缓缓点着头。表示自己听懂了。

韩冈的提议,乍听起来是没有什么问题。最后还是由她本人来选择,其实就跟常见的举荐是一样的。

“也差不多。”她把心里的话说出了口。

一开始肯定没什么区别,的确应该差不多。韩冈心中说道。

不过时间长了就不一定了。

一旦选举成了惯例,当哪位重臣有资格晋身两府,其门生故旧都会主动为其奔走。

一旦有机会掌控朝政,家中的子弟、门人,投效的僚属,都会对他们产生期待。而政敌,也不会忘记秋后算帐怎么写。到了那个时候,就绝无退步的余地。身后就是悬崖,前进方能得保无恙。

受人之托忠人之事,若是这位宰辅是被朝臣推举上去的,那么他就免不了受到这些朝臣的牵制甚至裹挟。

而且选举能选出合格人选的可能性很小。

尤其是这种人数不多,地位又相差不多的选举。如果没有人四处勾结许愿,最后选出来的,多半是最为平庸、最为无害的一个,太过突出的往往都会被视为另类,难以在选举中出头。

但那是以后了。这第一次,表面上还乱不了。

“王中正,去请楚国公。”

向太后其实已经很累了,今天一天,是她面临过的最大危机,情绪上也激烈波动,早耗尽了她的精力。但她还是在咬牙坚持着,想要尽早将所有事都处理好。

王中正没有迟疑,立刻跪下领旨,然后便出去了。

小半个时辰之后,王安石出现在内东门小殿中,向太后没有多绕圈子,很直接的说道,“家门不幸,出了这等忤逆之辈,亦是天下不幸,竟有贼人敢。幸有卿家能拨乱反正,如今京城人心不安,须得卿家维持,不知卿家可否屈就平章一职?”

王安石过来时,虽已做好的心理准备,但还是没想到太后会如此直接。要任命臣子就任要职,首先得写诏书吧,哪里能当面询问,却没有一个纸面上的文字记录。

看到王安石没有即时作答,太后又道:“还请楚国公勿要推辞,就算不看吾和官家,也要看在先帝的份上。”

向太后的水平见长。虽说言辞不算出色,但正好抓住了关键,将王安石架了起来。而且王安石也不可能不顾念旧情。

“先帝之恩,臣粉身难报。臣如今虽昏老无用,若能稍补于朝廷,又何敢惜身?臣不才,愿领命。”

“这就好,这就好。”向太后喜动颜色:“有平章在,吾就安心了,这就让内翰过来写诏书。”

王安石又在望着女婿。难道韩冈还没有就任两府之位?要不然,翰林学士就该在这边站着。

“这是令婿的举荐,正好跟吾想的一样。”

王安石皱了下眉头,不过想想也就罢了,韩冈的推荐影响不了他的行事。但下一刻,听到太后的话后,他的脸色就变了。

推举?让太后自己做最后的选择?

王安石只觉得匪夷所思,茫然不解的看着自家的女婿,全然闹不明白他到底是在想什么。

韩冈当有自知之明。若是由京中军民来推举,朝中无人能与他相提并论。可换成是朝臣,有多少人会推荐?

木秀于林,风必摧之。韩冈在朝臣心目中的形象,可不如他在民间的声望。

至于种痘法的感激?那些朝臣就别指望了。

仗义每多屠狗辈,那些读书做官的,尤其是做到当朝重臣,就别指望会有所谓感恩、节义。

这是不打算入两府了?



