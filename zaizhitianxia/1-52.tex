\section{第25章 欲收士心捕寇仇(上)}

秦州城。

成纪县户曹书办刘显脚步匆匆走进陈举的书房。平日里刘显总是竭力学着士大夫们的闲雅从容,总是一副气定神闲的样子,行走时方规矩步,少有如今这般仓促,甚至可以说是惶急。

书房中,秦州道上赫赫有名的陈押司,正貌似悠闲坐在桌边喝着茶汤。一名秀丽脱俗的侍婢手持茶杵,研磨着产自福建的入贡团茶——虽然只可能是最为普通的一品团,而不是只供御用、有金箔包装的一斤二十饼的龙团和凤团。但能弄到一块,也是难能可贵。

拈着茶杵的纤手嫩如葱管,白皙如玉。手腕轻转,便将雪白的团茶研磨成末。注入滚水后,水脉翻腾,似有无数花鸟虫兽浮现于水中,继而又悄然隐去,如此绝妙手段,如是与人斗茶,甘拜下风者不知凡几。

陈举侍婢严素心的茶艺,在秦州城中也是颇有点名气。青茶盏,白茶汤,被一对柔若无骨的玉手端到陈举眼前,茶香扑鼻,看她素手烹茶的韵律,似与旧日并无两样。

可再看原本保养得甚好的陈举,虽然端坐在茶桌边,举杯而饮。但浓浓的忧色缠绕在眉间,显得心神不宁,全不知味。才几天功夫,他须发间都已经有了星星点点的斑白。一见刘显进来,陈举便对侍婢一摆手,“素心,你先收拾了出去。”

严素心轻声应了,低头收拾起茶具。而陈举连茶盏都忘了放下,上前急问道:“怎么样了?韩三回来了没有?!”

刘显颓然摇头:“没有回来。”

严素心悄步出门,只听得陈举在身后房中怒叫:“没回来?他怎么还不回来!延期不归,他想作死不成?!”

“爹、娘,终于等到了吗?”严素心低声喃喃,脸上看不出喜怒哀乐。她脚步不停,泪水却难以抑制的从眼眶中溢出,‘老贼,你也有今日!’

书房内,刘显从袖子里摸出了一份盖着朱红色大印的公文递给陈举。他叹气声很无奈:“韩三被张守约留下了。这是五天前甘谷城发到州衙的公文,说是要留韩冈在甘谷听候指挥,但到今天才转发来县衙中。这件事就算有过,也被张守约担下来了。韩冈攀上了张守约,现在是有恃无恐。”

韩冈是在成纪县有差事的衙前,按法度,张守约无权将其留用。但谁敢为了一个衙前而跟一路都监过不去?

就连李师中都不会做的事,成纪知县怎么可能有这个胆子?

即便陈举能瞒着知县私发一份公文去甘谷要人,如果张守约不加理会,丢到一边,甚至拿去擦屁股,还能把官司打到李师中面前去?

韩冈算是逃出生天了——靠着张守约的帮助。陈举一阵怒起,但转眼他便平静下来,无奈苦笑。

韩冈其实早就脱离了他的掌握……

裴峡谷蕃人惨败的消息早在战后的第三天就已经传到陈举的耳中,单是因为这事,曾经与陈家来往了几十年的末星部就跟他翻了脸,直接杀了陈举派到部中联络的亲信。在末星部看来,他们是上了陈举的恶当。能在被伏击的情况下击败两倍的族中精锐,护送着辎重车队的又怎么可能会是普通的民伕?

但陈举也一样暴怒,是近百人去埋伏人数不过四十的车队啊!整整两倍的兵力——

怎么还会败?!

怎么还能败?!!

怎么还敢败?!!!

难怪末星部一年不如一年,被隆中部压着打。

还有董超、薛廿八两人,是死是活,是投了韩冈,还是继续听命于他陈举。这些陈举都不知道。再加上黎清那混帐东西,到了甘谷后连句话也没传回来,让他完全是两眼一抹黑。

倒是韩秀才在伏羌城射了向宝家奴一箭,才几天整个秦州就传得沸沸扬扬,但都钤辖家连个屁都没放。而向家商队回到秦州的第二天,从向府后门就抬出去个席子包裹,直接抬到了化人场,说是急病而死,恐有疫症,要尽快烧掉。

都近腊月了,有个哪门子的疫症?

堂堂都钤辖拿韩三都没辙,他区区一个押司还能将韩三如何?

曾将仗着威势,陈举将成纪县视作自家的后院,直以为凭借三代人近百年的积累,自己的地位如同铁打的一般。但现在看来,却不过是一层窗户纸,不见韩冈费什么手段,就给戳得到处是洞。

刘显原本就是脸色苍白,现下更是如纸一般,“押司,现在该怎么办?”

陈举紧紧捏着茶盏,啪地一声轻响,薄胎青瓷在他的掌心碎裂。滚烫的茶水泼了出来,他却恍若不觉。这几日陈举都睡不安稳,多少次在噩梦中惊醒,浑身都是冷汗。每次醒来,梦里的一切都已模糊不清,犹能记得的,是在鼻尖心头缭绕不去的浓浓血腥,还有每次都会出现在梦境中的那对太过锋利的眉眼。

“放出消息去,我给一百贯的赏格。有关韩冈的事,有一条,我付一条的钱,有十条,我付十条的钱!先把韩冈的底摸清楚。”

陈举咬着牙,韩冈不死,他如何能安心!

刘显点头应了。

“还有,他的父母不是逃到凤翔府去了吗。找人把他们弄回来……不!”陈举改口,神情更为狠厉:“让他们得个急症,看韩冈会不会赶去凤翔尽孝!”

“是在半路上……?”

陈举瞥了刘显一眼,眼神森寒,户曹书办慌忙应是。

“你再去找王舜臣。什么都不必说,直接给他一百两金子,如果他不收,再加一百两。”

韩冈没回来,王舜臣却回来了,可见两人的交情还未拉得太近。两百两金子足以兑上五千足贯铜钱,陈举不信一个赤佬能有多清高。因为韩冈,他已经将家里明面上的财产用去了三分之一,而暗地里的家财也大半暴露在外,现在再用上五千贯,其实也算不得什么了。

“什么都不说?”

“王舜臣是聪明人,该知道怎么做。”

刘显点头记下。又故作轻松的勉力笑道:“有押司你这几招,我便不信,小小的村措大还能翻了天去。如果他死了,都钤辖肯定高兴。”

陈举没理刘显在说什么,他右手捏着额头,血淋淋的左手一下下的在桌面上敲着。嗒嗒的响声持续了许久,突然停下了,陈举脸色泛着铁青:“经略司王机宜是前日回来的吧?”

刘显茫然点头,不知陈举为何如此发问。

“王机宜前段时间可是在伏羌城?!”陈举的声音问得更急。

“王机宜主管蕃部事务,所以这几个月,都是在边境的各处城寨来回走动。达隆、者达、安远、通渭,还有甘谷、伏羌,他……”刘显的声音又顿住了,一个让他全身冷透的念头从心底浮起:“押司,难道……”

“……你说他有没有碰到韩冈?”陈举幽幽发问。

“不会!不可能!绝不可能!”刘显拼命摇着头,但他的否认连自己都难以说服。计算时日,裴峡谷一战以及韩冈抵达伏羌城的那一日,正是王韶从北面赶回来的前两天。从甘谷到秦州,快马一日可至,而王韶是跟甘谷城的报捷信使一起回来,他和他的护卫的十几匹坐骑,据说有两匹倒毙于马槽中。

甘谷当时已然平安,还有何要事须王韶不惜马力,也要全速赶回?除了裴峡谷之事,陈举和刘显想不出其他理由。而韩冈正是当事人,王韶不可能不向其问明来龙去脉。

陈举又恨起末星部来,如果能在裴峡中将韩冈一众一举铲除,就算有后患,也能栽到别的部落身上。但现在有这么多活口在,谁能保证陈举他和末星部不会暴露出来?

“只是一个机宜文字,又有甚么可怕!”刘显叫起来,只是他声音越响,越是显得心虚。

“时间呐!”陈举的双手都在抖着,面色惨白,“从王韶回来,我们到底耽误了多少天?!”

经略司机宜虽然权重,但品秩不高,毕竟不是经略安抚使。如果陈举能倾其所有身家,发动他的一切关系,还是能拼上一拼。可耽误的时间却追不回来,王韶从北面返回,自己却没能在第一时间反应,现在王韶还会再给他们时间吗?

“老爷!老爷!不……不好了!”陈家的老管家这时跌跌撞撞地奔进内院,冲到书房,已是上气不接下气。

“什么不好?!”陈举瞪眼怒道:“待会儿去领二十棍家法!”

“老……老爷!老爷恕罪,”管家心中一慌,喘得更加厉害,“门外……门外……”

他‘门外’了半天,也没说出个所以然来。但陈举和刘显已经不需要听他再说了。只闻得前院轰然一声巨响,陈家宅院的大门被人猛然撞开。两扇厚重达数百斤的门板向后倒去,扑起满地的灰尘,将几个家丁压在了下面。

一个粗豪雄壮的声音随即在前院响起:“洒家奉经略相公之命,捉拿西贼奸细陈举、刘显,及二人亲族、党羽。凡有妄动者,一例格杀勿论!各自细细搜检,莫走了陈、刘二贼”

管家面色如土,舌头忽然间也不打结了:“门外是王舜臣带着兵给围上了!”

半刻钟后,陈家的宅院中,各处仍有着搜捕的喧嚣,但王舜臣已经站在书房中,俯视着脚下。在他身前,两名被指名要缉捕的罪魁陈举和刘显捆得如粽子一般,被强按在地上,等待王舜臣发落。

陈举和刘显一贯是衣冠楚楚的士绅模样,但如今,两人衣服被扯破,头发披散着,脸上更是有着擒拿时留下的青紫伤痕。

刘显面色狰狞,过往刻意表现出来的雍容气度全不见踪影,他在地上用力挣扎着抬起头:“王舜臣,你别得意!等我们出来,有你哭的时候!”

“出来?是再投胎吗?”王舜臣自眼底瞥着他,冷笑着:“爷爷就等你十八年!”

一脚踢开刘显,他又在陈举身边蹲下,低头狞笑道:“你不是要杀三哥吗?怎么样?现在是谁杀谁?”

陈举脸色苍白,三代人建立的基业被一个身份卑微的穷措大一脚踢垮,而陈举的自信,也随之东流,唯一记得的是要给陈家留个香火,“王将军……”他向王舜臣脚边挪了挪,仰起的脸上挤出一个谄媚的笑容:“只要王将军你肯放人带个口信去凤翔给小人的儿子,给我陈家留条生路,小人愿把家里旧日藏的窑金都献给将军,足足一万贯!”

“呸!”王舜臣一口浓痰吐在他脸上,“这时候倒肯服软了?!过去害人的时候,怎么不见你饶人一条生路!想想你家三代害了多少人?积了多少阴德?!实话告诉你,去追捕你家两个儿子的人早走了,追不回来了!走,带他们回去!!”

王舜臣押着陈刘二人回到外院中,陈举的一众家眷哭哭啼啼的被赶了过来,都用绳子绑成了一串,谁也逃脱不了。另一边,陈家的数十名仆役和婢女被圈在一边,也都是哭丧着脸,小声抽泣着。

唯有一名身着白衣的秀色侍女,怀里搂着个小女孩,宁宁定定的站在角落里。王舜臣多看了她一眼,却见她的一双眼睛只死死的盯着陈举,头发上,一朵白花在寒风中晃着。

ps:陈举终于被捕,韩三的后宫也要招募新人了。

今天第一更,求红票,收藏

