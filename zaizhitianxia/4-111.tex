\section{第18章 青云为履难知足(17)}

【会用挖掘机的司机太有才了,害得俺值了一天的班,回来后就只赶出一章。今天补休,一起补上。】

王韶回到家中时,也是二更天了。

与一行元随骑马进了崇仁坊的时候,坊中的更夫都敲着更鼓绕了达官贵人聚居的厢坊一圈,差点就跟王韶的一行人撞上。

他是在衙门中为了韩冈的提议和天子的任命,在公文案牍上决定西军南下的部队。尽可能的挑选着各路的精锐,充作进攻交趾的军队。一忙起来就忘了时间,等一切差不多都敲定的时候,早就过了散值的时辰了。

进了家门,走入正厅,一个五六岁的男孩子一马当先的从内侧小门中跑了出来。穿着锦衣华裳,脚下一对虎头鞋。圆头圆脑,乌溜溜的一双大眼睛晶亮有神,看着就讨人喜欢,是王韶最小的儿子王寀。跑到王韶面前,就换作一幅大人模样,一本正经的向着王韶行礼,“孩儿拜见爹爹。”

只有在面对最疼爱的小儿子的时候,王韶的表情才会放松下来,弯腰抱起王寀:“十三,今天有没有淘气?”

王寀摇着小脑袋,“没有!孩儿跟着娘娘和四姐姐习字来着。”

“谁说没有?把冰桶打翻了,闹得书房都是水的是谁?”王韶的长子王廓笑着跨步入厅。

跟在王廓身后,王韶的儿女们也都一起出来了。除了一个还在熙河路为官的王厚,还有两个已经出嫁的女儿,王韶其他儿女现在都跟在他的身边。十几个子女,聚在一厅之中,站满了王韶面前的地面。子嗣之多,足以让当今的天子羡煞。

王韶听了王廓的报告,捏着小儿子的脸:“怎么闹得书房都是水?”

王寀抬着头,理直气壮:“水曰润下,自然之常性也。”

王韶闻言先是一笑,然后心中的惊讶就难以遏制的涌了上来。不过小孩子的强辩,竟然连《尚书·洪范》里的词都迸出来了。看看几个儿女,都是一脸的讶色,并不是有人事先教着说的。

王韶的惊讶立时变成欣喜,知道儿子早慧,却不意聪明到这个地步。‘只是太聪明了!’王韶随即又患得患失起来,一时都忘了要称赞儿子的聪明。

长子王廓带着弟妹出来迎接王韶,王寀也从王韶怀里挣扎着下地,重新跟着兄姐们一起行礼。一众行过礼后,王韶的其他子女,就纷纷回去了自己的房间,王寀也被几个姐姐带了回去。只有王廓跟着王韶,随口问道:“大人今天怎么回来的这么迟?”

“还不是韩玉昆给闹的。”王韶用力的哼了一声,抬脚就往房中走去。

不过王韶虽说是在抱怨,但王廓听得出,他的父亲并没有半点责怪的意思在,倒是开心得很。

王廓已经好些天没看到父亲心情放松下来,心知必然是朝堂上发生了什么事。他随在身后,试探的问着,“是交趾的事吗?白天儿子去王相公府上,听说韩玉昆被天子招入禁中,难道就是为了此事?”

“还能有别家的事?!”王韶反问了一句,就笑了起来:“冯当世想将攻打交趾的事拖着,吴冲卿就想着让河北军去邕州。这些天他们一直都各自在谋划着,可今天韩冈一上殿,一句重行更戍法,让河北、京营两部禁军可堪一战,就一下变成了西军南下,从河北、京营的禁军中挑选精锐去关西填空!”

“河北、京营去关西?”王廓闻言便是呆愣住了。只愣了短短的须臾片刻,他砰的一声重重拍了下桌案,“原来就这么简单。”

“田忌赛马赢得不也是那么简单,为何除了孙膑,一直没人想到?”王韶其实也是感叹,自己身在局中,被蒙了双眼,如果从中跳出来,以自己的才智不当想不到,“这就是韩冈一句话的功劳。”

“一言兴国、一言丧邦,一言便让朝政改弦更张……”王廓喃喃的说着,神色很是复杂。

王韶瞥了长子一眼,“韩冈是个异数,你们羡慕不来的。”

天子的信重与否,与官位高低无关。韩冈已经是一路转运使,话语权远在普通的知州之上,加上凭着过往的功劳在军事上得到的权威,让韩冈能轻而易举的说服天子。不过这也是他的建议一直都有道理的缘故,所以天子才会相信他。

进屋换下了身上的公服,王韶转过来在放了冰块的书房中坐下。

书房中的水已经干了,用来降温的冰桶则重新装满了冰块。如王韶这样的宰执官,每年冬夏时节,朝廷都会大批的赐下冰炭。夏日解暑、冬日取暖。同时宰执们所居住的府邸都是周围数百步的大宅院,在京城中,这样大的宅邸没有一座冰窑,

“韩玉昆我等的确是比不上。”王廓同样在书房中坐下,对父亲笑道:“日后就得看十三的了。”

“十三也不指望,能安安稳稳的读书做官就够了。”王韶叹了一声,重复着之前的评价,“韩冈是个异数!不要去比,不要去学。”

对于韩冈,王廓的感觉很复杂。自己的二弟与他是生死之交,而自己的父亲又是韩冈的恩主,关系之密切,可以说日后几十年王韩家都是连在一起,要在朝中互相扶持的。

可作为王韶的长子,王廓也有一番雄心壮志。看到年纪比自己还小的韩冈如今功成名就,连父亲也如此推重,而自家连嫉妒都不够资格,心中免不了百味杂陈:“二十五岁就已经判一路漕司,开国以来的确是无人能及。”

王韶喝了一口冰镇的酸梅汤,心脾间一阵沁凉。心情一舒畅,就难得的对儿子多了口,“年纪是韩冈的优势,也是劣势。如果他已经是不惑之年,凭借他立下的累累功绩,进入政事堂也是理所当然,甚至可以说,只要有一半甚至三分之一就够了。”

对上王廓惊异的眼神,王韶叹道:“为父是凭着河湟之事入了西府,而韩冈在河湟之前、之后,又立下多少功劳?”

他摇着头,“想想韩琦,他进入政事堂的时候,他所凭借的功绩又是什么?在陕西的经历!可他在陕西又有何功劳?任福的好水川,还有张元的一句‘韩琦未足奇,夏竦何曾耸’。要说韩琦曾在蜀中安抚灾民百万,韩冈也有在京城安置河北流民数十万的成绩,绝不比韩琦稍逊。可韩冈能比韩琦更早入两府吗?”

“难道不能?”王廓疑惑问道。只要韩冈与章惇顺利的平定交趾,凭此功绩再外任一任经略,资序攒足,再不能晋身两府可就说不过去了。

“很难!”王韶很肯定的摇头,“三十出头的执政,日后可就有三十年的时间进出于两府之中,到时候,说不定就是天下官员出于一人门下。天子怎么可能会答应?”

“韩忠献不也是几十年出入二府吗?还是三朝宰相!”

“那是因为有富弼、曾公亮、文彦博之辈在。想想韩冈,同一辈中可有人能与他相提并论?”王韶感慨着,“而且就算是韩琦,也照样受着忌惮……最后的这几年,韩琦都是在哪里?!看看天子给韩琦墓碑上题的字:‘两朝顾命定策元勋’!韩稚圭的确有大功于国,若无他镇压住朝中、宫中,则嘉佑、治平的那十多年,宋室绝难安稳。但宫中没人会想出再出一个韩琦!”

“但韩冈的功绩、才干俱全,只要资序够了,日后朝廷要选人入二府,就很难绕得过他。”

“只要一直让他在外任官就可以了。赢了交趾之战后,韩冈在十年之内,恐怕很难再回京为官…………不过话说回来,”王韶的话锋一转,“韩玉昆行事不依常规,思路不落窠臼,往往出人意表,说不定他还真的有能耐让自己很快就回到京城!”

……………………

不仅仅是王韶,吴充回到家中的时候同样也是很迟。

不过没有他没有王韶的好心情,回到家中的时候也是板着脸的。

出来迎接他的,也是长子,在京中任官的吴安诗。

吴充进门后不见次子吴安持,问道:“二哥儿还没有回来?”

吴安诗道:“方才已经遣了人回家来,说今晚不回来了。”偷眼看了看吴充的脸色,并不敢多问。

吴充不喜欢跟家人说起朝堂上的事,满肚子的国家大事也极少会泄露给家人。只是偶尔要点拨儿子的时候,才会多说上两句。吴充的两个儿子都知道他的脾气,也很少主动发问。

吴充回房更衣,出来后问着儿子:“韩冈现在是不是也在王介甫那里?”

吴安诗愣了一下,怎么跳到了韩冈身上?但转眼就想到了,自家的老子肯定是在殿上受了韩冈的气,“此事儿子就不得而知了,论理应当在的……是不是要让人知会二哥儿一声,离着韩冈远上一点?”

“为什么?”吴充脸色一冷,“为父向来只为国事与人有隙,岂会记恨私仇?既然是亲戚,自当尽人情。每逢年节,二哥儿去见王介甫,为父何曾拦过?”

如果嫁给吴安持的王家大女儿此时在这里,她肯定连连摇头。若当真如此,她也不会在吴家一直都心情郁郁。

但吴安诗哪敢反驳,连声道:“爹爹说的是!”

