\section{第30章 众论何曾一(三)}

【真是要了老命,三点半才写完。】

韩冈并不知道京中他岳父和大舅哥现在的困扰,他现在正在接待他的二舅哥。

大过年的,。就算要见面,也是韩冈这个女婿去京城拜见岳父岳母。但王旖终于有了身孕的消息,被韩冈命人急报东京城的岳父家,王安石夫妇听了之后,也不管是不是过年,就立刻让王旁带着一堆滋补的药材来探望。

韩冈亲迎了王旁进衙,问过岳父母安好,又设宴款待。到了晚间,韩冈安排了王旁在偏院中睡下,回到房中,王旖却还点着蜡烛,坐在桌边没有睡。

“怎么还不睡?”韩冈进来后就问着,孕妇可是要多休息的。

王旖转过身,递上来一封信。

韩冈拿着信纸,有些糊涂:“这是……”

“是娘写给奴家的私信。”

“……是说了什么不能给仲元知道的事?”韩冈一下就明白过来。

如今托人寄送的信函,有的封口,有的不封。不过托自家人带的信件,就不可能涂了浆糊或是火漆上去。王雱写给韩冈的信,王旁也许会看。但吴氏写给女儿的信,王旁怎么也不会有心去看的。

“还是二哥和二嫂的事。”王旖话声中带着忧郁。

韩冈瞥了一眼手上的信,吴氏写得倒是一笔好字,一手的快雪时晴让只擅楷书的韩冈自愧不如。只是信中的内容,韩冈没有去看,直接放到了桌上。想来除了要王旖安心养胎的话,就是家里的事,且多半是在说王旁。

王旁与妻子庞氏不合,因为儿子长得不像自己,日夜吵闹不休。这一事,韩冈在与王旖成亲之前就已经知道,现在快一年了,王旁夫妻的关系还是没有改善,看起来反而更恶化了。

韩冈明白,王旖将岳母写给她的私信交给自己看,是想自己能帮着解决这个问题,可他在这方面却一点经验也没有。

“仲元夫妻俩的事,我这个做妹夫怎么开口?”韩冈摇摇头,没有兴趣掺和。

自己的那一个才两岁的内侄,的确不像王旁,但也不像王雱,或者说并不像王家的人。可庞氏本就是大户人家的女儿,大门不出二门不迈,且又不是早产的惹人疑窦,还能有什么猜疑?相貌不似父母的世上多有,怎么也不能作为证据。可王旁却认定了那不是自己的儿子,谁来说都没用。

“二哥只是认死理,官人你跟二哥一向合得来,能不能开解一下。”王旖拉着韩冈的衣袖,像个小女孩儿一样轻轻摇着,轻声问着:“好不好?”

认死理就是偏执。而偏执是一种病,韩冈知道这一点,但要说救治,他可没辙。精神病医生或者说心理医师不是光靠说话就能解决问题,许多时候还要用药。而且以自己的行事作风,从来都是简单明快,做事都是快刀斩乱麻一般。纠结的家务事真的不是他所擅长的,而且掺和亲友的家中事,他也没有这个习惯。

韩冈有心拒绝,但看见王旖抬着头,波光盈盈的眼中尽是祈求,泫然欲泣的样儿,心中也不由得一软:“开解不好说。这方面的事,你越提他就会越火,我这边就陪着仲元多散散心好了。”

王旖破涕为笑,瞬间绽放的笑容如春花一般灿烂。

韩冈搂着她过来,“照我说,要真的不行,还是让岳父安排个差遣,让仲元出去做点事。天天见着,当然容易看着生厌。隔着远了,日子一久说不定就会挂念起来。”

王旖听着转过脸来:“官人是不是天天看着奴家也生厌?”

“胡说什么呐!”韩冈反手弹了下王旖的额头,“我可是一日不见,如隔三秋。”

王旖捂着头:“骗人。”

“是真的!”

韩冈赌咒发誓,嬉闹了一阵,王旖才又理着披散下来的头发,将话题说回去:“二哥要到明年才满二十五。爹爹怎么会为他请官家特旨?”

韩冈拍了拍额头,竟然忘了这一茬。进士等有出身官员不到二十岁,荫补官不到二十五岁,都不可任实职,只有天子特旨可以例外。韩冈是个例外,但他不觉得王旁有资格例外。

“要不,让仲元出去寻师访友也可以……”韩冈说到这里,突然愣了楞,顿时恍然大悟:“原来是打的这个主意!”

王旖乖乖的缩在韩冈怀中,“大哥也是怕爹爹日夜烦心,所以跟娘说了,让二哥到家里来住上一段时间。”

“我这边就不烦心了?你大哥还真是会使唤人!”韩冈知道自己又有的头疼了,“要拖住可不容易,我也没有多少时间陪他。我看还是找点事请你家二哥帮忙吧。”

怎么都是自家事,能帮一把就帮一把。而且自己的夫人有是冰雪聪明,自家要是随便敷衍的话,她一下就能看破。因为王旁的事,弄得自家吵起来,可就是太蠢了。

当然也是因为王旖是自己的枕边人,无意用心机待她。换作是外人,他多年磨练出来的脸皮和口才,能很好的发挥作用。

……………………

第二天,韩冈就拉着王旁去城外。此时还没有到上元节,县中虽然年假已过,可过年的气氛还很浓。衙门里也没什么事要处理,韩冈上午就可以出城去。

由于粮价降了下来,物价也都跟着降了,白马这边的百姓,至少在过年时,还是有着轻松的笑容。只是到了城外,渐渐靠近了流民营地,就能看到一片紧张的劳动场面。

在此时,救灾最常用的策略就是以工代赈,让流民中的精壮能填饱肚子,却又累得没有造反的力气。流民身无余财,有没有储备,一家老小都靠着衙门里安排的活计来挣佣钱。一天一个壮劳力能挣上百十文,买米买炭,再买些日用品,一天的工钱将将够用。

至于韩冈,他付给流民的只有一小部分是钱,而大部分是库中的稻谷和小麦——平常粮店里卖的米面,都是十成的谷子,出七成的粉或是米。但流民自己来磨,甚至能出到九成。连麦麸和米糠都不放过——现在在白马城外,已经安顿下来的七八百流民,都有着事情来做。

“他们在做什么?”王旁就指着围着个轱辘的一群人,不时的还能从那群人中听到咚的一声闷响。

“是在打井!为了抗旱,现在县中四处打井,而且要深过二十丈的深井才保证出水。”韩冈说着,将他提拔井十六开凿自流井的事也说了一通。

王旁听了有了点兴趣:“愚兄素来只见过泉眼,但开凿出来能自动吐水的深井,还真没有听说过。开成了没有?”

“没有!”韩冈摇头,“井十六的深井倒是凿成了,但却不是自流井,井水的水面的确上涌,但到了两丈深的地方就不再上升了。不过这个深度足以使用手压式唧筒,用浸了油的丝麻作为活塞填缝,以竹筒为本体,上下提动摇把,就能将井水给提出来。”

“又是唧筒取水。”王旁笑着,他对韩冈的发明没有多少一探究竟的兴趣,道:“玉昆你真是什么都能变得出来。”

“这也是没办法,要是有自流井,小弟还要费那等气力作甚?”韩冈无奈的说着,“其实自流井,在蜀中多一点,关西那里也有。这次没能一次头给打出来,多半还是运气不够的缘故,没有找准水脉。不能算是井十六水平不够,我这边也是犯了点迷糊,只打一眼就正好撞上自流井,也不可能能有这等好运。”

说着韩冈又叹一口气,望着这一片黄河大堤下的平原。从近到远,都是一色的只见泥土的土黄色,完全没有半点正常年景的冬日,积雪覆盖原野的景色。“这件事其实就跟之前岳父要开汴口、凿河冰的情况一样,我这边也算是急得没办法了。从去岁来此上任,三个月来一滴雨一片雪都未见。地里出苗只有一半。明年开春若是没有水,想补种都没办法。要是真有一口自动冒水的深井,不知能浇灌出多少田地。”

从这口深井中提出来井水清澈甘甜,没有普通井水的涩味。可没能打出自流井,井十六还是失望不已。与近在咫尺的官身错失,使得这位井师一下变得颓丧起来。韩冈倒是安慰了几句,又赏了不少银绢作为奖励。无论如何,旱涝保收的一口好井,就算不能自流,也是人人争抢的宝贝。

韩冈还是想要能自流的井水,自然的办法不行,那用机械的办法也可以。他打算将其改造成自动提水的装置,“小弟的悬赏已经贴出去了,用风车驱动或是畜力驱动都可以,只要能汲出水来。就看哪一个聪明人能拿到五十贯的赏钱了。”

“希望能早一点有人揭榜。”王旁看过干裂后的土地,心中也为之黯然,今年的灾荒只会更重:“如果真有人能发明此等机械,那可是善莫大焉。不知会有多少百姓为此而感恩戴德。”

