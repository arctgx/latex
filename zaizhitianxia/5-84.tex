\section{第十章 却惭横刀问戎昭(16)}

对面的骑兵大约三百骑上下,比自己这边人数稍少一点,但随行的马匹数量却几乎多了一倍。种朴手下的这个指挥,已经是难得的一人双马——一战马、一骑乘——而对面的马匹数目,感觉就像出来放牧的牧民。

尽管从与骑手的身材对比上,看得出那些马匹都不是太出色,个头不高,体格较小,但架不住数量多得惊人。能在坐骑上这般豪奢,显而易见绝不是大宋这一边的军队,而且种朴见识过的多少支铁鹞子,都没这般阔气的。

虽然对方不知来历,但有一点可以确定,他们绝对是敌军!

瞬息可至的距离,没有观察等待的时间,种朴当机立断,没有丝毫犹豫的下令:“吹号!举旗!”

随着种朴的暴喝,苍凉的军号吹向云端。掌旗官双臂使足力气,用力一抬,将收起的军旗高高举起,殷红的旗帜呼啦啦在风中舒展开来。

这是战斗的信号。

宋军的骑兵们立刻向着旗下汇聚,绵长的行军队列开始飞速的收缩起来。

而对面军队的反应同样迅急,三声号角响过,一名名骑手就在马背上飞身换马,而后就向着宋军这里猛冲过来。

一道道黄尘在马蹄下卷起,在奔马的洪流之后汇聚成一条黄龙。这一群骑兵,起步时前后不一,但散乱绵长的行军队列在奔驰中自发的转换为衔接紧密的冲击阵线,完全省去了聚兵列队的步骤,骑兵战术运用之精妙,竟远在宋军之上。

来不及换马了!种朴心中叫着不妙。

聚兵,换马,然后冲锋,这是宋军骑兵进入战斗状态时的一贯顺序。但突然间出现在近前的对手,根本没有给他们留下准备的时间。

没有换马的余暇,就算换了马,缺乏足够时间起步提速的骑兵,还不如步兵管用。种朴也并不觉得,自己麾下的这群骑手,有资格与对面的敌军在骑术上一较高下。

转身逃跑——或者叫转进——的念头在种朴脑中一晃而过,但立刻就放弃了。胯下的坐骑驮着他在风沙中走了半日,又是专门用来代步的乘用马,跑不了半里,就会给敌军追上。

“刘庄,回去报信!”种朴急急的遣了一名亲兵,随即翻身下马,喝令道:“取弓,下马!”

跟随种朴出来的这一支骑兵,都可算是精锐。种朴又是种谔的嫡亲儿子,说话管用得很。听到命令之后,纷纷下马,抓起刀枪弓箭就列队布阵。摆开的阵型,比起之前在马上的时候又聚拢了一些。

只是一转眼的功夫,越来越近的敌方骑兵,已经让人能分辨得出他们身上具体的细节。

种朴在调整阵型,安排人手看管马匹的时候,也不忘紧锁着越来越近的敌骑。

完全陌生的旗帜,完全陌生的装束,甚至连进兵时的号角声都与种建中所熟悉的铁鹞子截然不同,这是一支他从来都没见过的骑兵。

一股寒流从脊柱传上来,差点让心脏停跳。种朴立刻就想起了一直都在心头中缠绕不去的忧虑,难道是契丹人已经决定发兵帮助西夏了?但狂奔而来的敌军,与种朴曾经听说过的契丹服饰仍有着一点差别,一丝侥幸在心底的黑暗中冒起。

身边的向导突地打起了摆子,黄了面皮,颤声道:“娘呐,那……那是北面的兵!”

“是黑山威福军司【河套】?”种朴犹然抱着最后一份希望,那是西夏最北面的一个统军司。

“是阻卜【即鞑靼,蒙古前身】……”向导挤出一个哭一般的笑脸,“是北方草原上的阻卜人!”

“阻卜……是什么都无所谓了。”

并不是说听到非是契丹人,就让种朴安心下来,而是他已经没有多余心力去在意了。一里的距离,只剩最后的三分之一。

蹄声已经充斥在耳中,种朴拔出腰刀,斜斜指向长空。

下面的士兵在种朴做出这个动作之后,便被都头、队正们催促着举起了掌中战弓。

骑兵不会携带神臂弓,也不会携带斩马刀,骑兵装备的胸甲更是只有冲阵时才会装备,寻常出巡,连营门都带不出来。

一般骑兵在马上的武器只配备有加装了倒钩的环子枪,军官多个铁鞭,弓箭则都是有的。近年来钢铁产量大增,四百人都有柄腰刀。此外种朴所领的这一部兵马,还加配了杀伤力很弱的诸葛连弩,算是给骑兵的优待。

借助不了战马的冲力,锋锐的铁枪毫无意义,骑枪要比步卒所用长枪短了不少。而弓力甚绵的诸葛连弩,发射速度很快,却都是用来惊吓敌军战马,作为不擅马战的骑兵们的秘密武器。而现在在全军下马之后,真正能派得上用场的,只有弓一张、刀一柄,然后还有短枪一支。

不过宋军缺马,故而对骑兵最为珍视。能成为骑兵,弓马娴熟乃是第一条。下了马后,一干骑兵都可算是步卒之中的佼佼者。

脚踏实地,骑手们张弓搭箭。随身携带的一壶二十支长箭也许只能支持片刻工夫,但在箭矢用光之前,没有哪家的骑兵,能在正面相抗中压制他们分毫。

但他们身上的的压力也绝对不轻。区区三百骑兵的奔驰,迎面过来时,却有千军万马的气势。

许多人的脸色越发的难看,握住弓箭的双手不由自主的颤抖着。而种朴喉咙有点发干,心跳得很快,掌心莫名其妙的冒出汗水,后背也是黏黏.湿湿的好不难受。

“百步!”当敌军立刻就要进入弓箭的射程,多年来接受的教育,让种朴收摄起心神,“都准备好了,举弓!”

四百余人听命,哗的一声,几乎就在同时举起掌中的战弓,而一支长箭也搭到了弓身上。

也就在这短暂的几秒钟,敌军狰狞的面孔已经看得很清楚了,甚至似乎感觉到了对面的人马喷出来的热气。

种朴的双眼中只剩坚定,掌中长刀向下用力一挥,用着最大的气力怒吼着:“射!”

四百骑兵同时松开弓弦,爆裂般的弦鸣,响彻无定河北的荒野。

……………………

“余古赧该到。”燕京析津府的宫城中,耶律乙辛一圈圈慢悠悠的踱着步子。

“这两天就会有消息了。”萧得里特恭声回道。

耶律乙辛走了两步:“他的八千骑虽不算多,赢也是不一定能赢,但吓一吓南朝还是没问题的……宋人就算分清阻卜和大辽本部的区别,也肯定会乱作一团。”

一直都是辽国西北藩属的阻卜部族出现在西夏,其政治意义远在军事意义之上,之前的恫吓重复千遍,也比不上八千阻卜骑兵的出现。

萧得里特谄笑着:“南朝的君臣听说,肯定是会吓得魂飞魄散,尚父说什么就是什么。岁币翻倍,割让土地,都在情理之中,到时候,就看尚父想要哪一条来实现了。”

“能多个十万岁币就够了,我也不贪心。”耶律乙辛笑着说道。距离自己的目标越来越近了,心情也越是放松。

他要的是稳定自己权位的声望,无论是胜利、土地还是岁币,只要能从宋人那里得到一点实质性的好处,就能压制国中的反对派。如果这一次能成功,不但能斩断宋人对西夏伸出的贪婪之手,还能轻而易举的将所有反对派斩草除根。

“余古赧领兵去了西夏,下官就有些担心磨古斯会不会趁这个时候做出些亲痛仇快的事来。”

阻卜是个大的范畴,其下分作三部,东阻卜、西阻卜和北阻卜。

余古赧是西阻卜的一员,靠西夏最近,关系也亲近,甚至与党项部族经常联姻。而磨古斯则是北阻卜的大首领,麾下上万帐,控弦三万,乃是阻卜诸部中实力最强的一部。

阻卜诸部都穷,故而秉性凶悍。起家于草原之上的契丹,对其极为提防,严禁铁器输入。偶尔为了压制某个大部族,甚至还禁盐禁茶。只是这些年,北阻卜以磨古斯为核心,乌鲁古河和薛灵哥河附近的部族渐渐有联合起来的趋势,放在西北路招讨司和阻卜大王府的几万人,已经越来越难以压制住磨古斯的野心。

相对而言,东阻卜和西阻卜就比较听话了。这一次促成西阻卜南下,不仅仅是西夏拿出来的好处,耶律乙辛的默许也是一条。耶律乙辛不打算让契丹本部赤膊上阵,那么让阻卜却帮个忙,也就是正和他心意。

这几年耶律乙辛要专心于国中,外围的藩国部族暂时很难分心去压制,能放出去祸害宋人,可是难得的机会。对于百多年来,断断续续的不断举起叛旗的阻卜诸部,眼下就是听话的东、西两阻卜,多死些人也不是坏事。

“余古赧的能力如何不好说,磨古斯的野心也不好说,但他们中间有阻卜大王府盘踞,这时候无论是什么样的的消息,都是好消息。”耶律乙辛最大的希望就是他的职位稳如泰山,而眼下则是给了他一个再好不过的机会,“只要能削减西夏和阻卜的人丁,什么买卖都是好事!”

