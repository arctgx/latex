\section{第14章 飞度关山望云箔(八)}

李信默然,最后也化为长声一叹,这时候,也不能指望交趾人会手下留情。

李常杰攻打的钦州、廉州,就算是主动开城,也都被他纵兵大掠,在城中的杀戮与屠城也差不离了。而邕州城的抵抗的时间如此之长,给交趾军造成的伤亡又极为惨重,开城后必定会有报复性的屠杀。就算李常杰也挡不住下面士卒要屠城的压力。而且,以李常杰在钦州、廉州的劣迹来看,他只会主动推动,而绝不会阻止。

“对了。捉到的那三个奸细,明天都送去宾州斩了,也算是给宾州百姓看一个证据。至于阮平忠,”韩冈想起了从长山驿中被营救出来的一干女子在他面前的哭诉,有什么样的部下,就有有什么样的主帅,黄金满在昆仑关可没有长山驿的交趾人做得那么绝,“也带去宾州一起剐了。”

“可这是交趾的一个将军!”李信提醒着韩冈,俘虏和斩首的价值可不一样,“还是章经略的意思。”

“我会写信给章子厚的。”韩冈不能容忍外贼侵害中国,可这些蛮夷只要愿意低头求饶,朝廷多半就会给放过去,以示中国的泱泱大度,“朝廷对这些外人太过宽大。黄金满与何缮他们将功抵罪倒也罢了,可有些人只要肯磕头,就有官俸拿,我可不想与他们同朝为官。”

韩冈心意已定,李信也就不多劝了。只要是明正典刑,加上一干苦主的供词,也不算是过错。

商量过这些事,李信要值夜,而韩冈就回房中准备休息去了。本以为这一个晚上不会有什么事,可到了下半夜,快天亮的时候,之前派出去的游骑斥候回来了两人。

韩冈得到李信的通报,匆匆穿了衣服起身。

走到正堂,李信正寒着脸。韩冈心中就咯噔一下,知道事情不好了。

见到韩冈这位主帅走出来,两名斥候连忙又跪了下来行礼。

“都这时候还行什么礼。”韩冈心急的催促着,“快说,到底出了何事?”

“禀运使,邕州起火了。快走到归仁铺的时候就看见邕州方向尽是火光,天都照亮了半天,三十里外就能见到。殿侍看着情况不对,就派小人两人回来禀报,他则继续向前去查探明白,说是到了明天就回来禀报。”

韩冈心一点点的沉下去,又是一阵无以名状的颓然。能烧红半边天,火势绝不会小,邕州肯定是失守了。尽管他对此早有心理准备,可他从京中一路赶过来,几千里地都没有歇过一次脚,就是为了救援邕州。眼看着离邕州只有六十里,自己又多方设计去营救,尽管已经登城,但只要城中再挡上两三天的时间,交趾人应该就会退了……

功亏一篑啊……韩冈心中一阵发闷,难受得想吐血。

“纵火焚城?”韩冈、李信闻声回头,就见到苏子元扶着门框,脸色一丝血色都没有。

李信连忙道:“军判,这事还没确定。”

苏子元摇着头走过来,刚走两步,两条腿就撑不住身子,晃了一晃,就一头栽倒在地上。苏缄的为人他这个做儿子的最清楚,如果不能保住邕州城,必然是一死殉国。

韩冈、李信连忙上前扶起他,苏子元并没有昏迷。他用力抓着韩冈的手腕,瘦削右手中传来惊人的力量,“运使,莫忘了邕州城内还有十万百姓!”

‘为百姓吗?’

“李信,整顿兵马,听我号令。”韩冈霍然起立,“火势再大也烧不光满城老小,交趾贼子就算要屠城也快不到哪里,无论城中百姓还剩多少,救出一个就是一个!”

……………………

“黄金满反叛?!”

邕州城破带来的喜悦荡然无存。邕州城中火焰熊熊,李常杰的心中同样火焰熊熊。他怎么也没想到,黄金满竟然倒戈了,投向了宋人,献了昆仑关不说,竟然还将驻守在长山驿的一千大越官军给害了。

宗亶的心也沉入深渊之中,黄金满的这份投名状献得可真够狠的。

逃回的黎生,还有四五百陆续收拢起来的败兵,要不是守在归仁铺上的那一支人马中途拦了一下,让他们都逃回来,这个仗就不用打了。

如果时间能重来的话,李常杰绝不会再派黄金满去驻守昆仑关。可当初让黄金满守昆仑关,就是必要的时候让他做殿后;另外还有将宾州送个他做补偿的意思,所以才没有将自己的兵一起放在昆仑关中。可没想到黄金满没有为他殿后,也没有感激送他宾州做补偿,而是送了一份大礼。从背后捅来的这一刀子,李常杰恨得刻骨铭心。

“将骑兵都放出去。”虽然李常杰手中就只有可怜的三四百骑而已,平时被他视如珍宝,根本不会随随便便派出去,只不过眼下可顾惜不了那么多了,“一定要封锁住昆仑关到邕州的所有联络。宋人收了黄金满,肯定会对刘纪他们动心思。”

“不能让刘纪他们知道。拖上一天就是一天。”对李常杰的命令,宗亶表示同意。看着辅国太尉签发出军令,宗亶又叹道,“宋人来得好快!”

李常杰紧紧咬着牙关,说不出半句话来。宋军的速度让他也感到惊惧不已。

从桂州一路赶回来,将宋人援军抵达广西的消息送到自己手中的密探,竟然只比昆仑关失陷的消息早了两天。这是什么样的行军速度?!李常杰也是带兵打仗的人,很清楚这意味着什么。

“黄金满是个精细人,不会无缘无故的投靠宋人。他将昆仑关献与宋人可以说是得到了好处的话,再卖力的攻打了长山驿,肯定是有另外的原因!”

宗亶点点头,至少宋军应该是表现出来了足够的实力,才会让黄金满死心塌地的投靠过去,而不用顾忌之后可能会受到的报复。在这之前,除了邕州有所表现,交过手的宋军,何曾能让交趾、广源两家联军的将帅们高看一眼。

“现在我们该怎么办?撤军吗?”

方才派出骑兵,只是为了阻断消息传播,防着内部生变。但宋军近在眼前的事实却是没法儿改变的。

“不能立刻撤军!”李常杰坚定的摇着头,“这一次在邕州城下的伤亡太大,至少也要让下面的士卒在城内过一次手。不然军中怨气难解,士气也提不起来。到时候,在宋军面前怎么撤军?”

李常杰说得没错,宗亶也明白这个道理。如果不能纵兵大掠的话,士气根本不能恢复,“那宋军怎么办?以他们进军的速度,明天后天就会杀到我们眼前了。”

“绝不可能!宋军急速南下,不及十日,就进兵千里,就算中间有一段可以利用水路,也不是多轻松的行军。兵疲师老,他们肯定不会选择贸然开战。”作为一名身经百战常胜不败的将领,李常杰坚信自己的判断,就算丢了长山驿,那也是宋人狡猾和黄金满背叛的缘故,“压倒黄金满不难,想要与我数万大军对垒,他们绝对不会有这个胆子!他们肯定也会歇下一段时间,用来恢复军力,同时还要窥探我军的情况。”

宗亶沉默不语,并不搭腔。李常杰这番话,未免说得有些太过自信了,宋人要当真如他所料,眼下也不会占了昆仑关。

李常杰脸色又难看了几分,但这个时候他必须得到宗亶的支持,“而且邕州已经破了,城中的火焰数十里开外就能看见。宋人想必此时也该知道。失去了救援邕州城的理由,他们还有什么必要再冒着风险拼命赶来?换作是你我,当也是会在昆仑关好整以暇的休整兵马,等待更好的时机。”

宗亶给李常杰说服了,点起了头:“多半是如此。不过抵达昆仑关的宋军究竟有多少,他们到底什么时候准备进攻,这些事都要尽快打探清楚,我们才好做决断。”

“黄金满反叛,之前派出去的那些探子不知给他卖了多少。”李常杰恨得直咬牙,好好的计划都给一个叛徒坏了,“就让黎生留在归仁铺,戴罪立功,拼死也要给我打探出来昆仑关中的情况。”

“最好再将邕州交给刘纪他们一半。”宗亶提议道,“在城中烧杀抢掠过,就算宋人来招揽,他们也得好好想一想后果。”

李常杰想了想,摇头:“……那样太大方了,刘纪他们必然心生疑忌。只给他们四分之一,不过散开来后就不管了,任他们来。”

李常杰起身走出大帐,望着邕州城上的一片艳红,咬牙切齿,要不是苏缄,他有哪里会有今天的狼狈,“传我将令!掘地三尺也要将苏缄全家给我找出来,寻到一人,重赏百贯!找到苏缄,有千贯之赏。”

宗亶摇了摇头,任李常杰发狠去了。对于苏缄,他倒是有几分敬佩,若不是他领军把守着,邕州也不会这么难破。虽为死敌,但也是个英雄人物。如今邕州城陷,苏缄这位知州,恐怕也只会做出一个选择。

宗亶望了眼面目狰狞的李常杰,冷笑一声,他肯定是不能如愿了!

