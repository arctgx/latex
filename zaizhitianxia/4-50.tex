\section{第十章 进退难知走金锣(下)}

自永安州登船出海,在海上一帆风顺。登陆的两日后便攻破钦州,后三日,又破廉州。到了第十一天,李常杰留下了还在钦州、廉州洗劫子女金帛的一部分兵力,率领两万精锐站在邕州城下,与领军取陆路北上的宗亶胜利会师。

抵达邕州的当日,交趾的辅国太尉便是一身戎装,在众将的陪同下,遥遥眺望着两百多步外,高达四丈的城垣。

李常杰的身材有别于周围只有五尺上下的交趾男丁,竟高达六尺有余,长相也算得上英挺,就是鼻梁略钩,显得有几分阴鸷。

“只可惜不能再走近一点了。”李常杰眯起细长的双眼,细细看了一阵摆上城头的防具,回头问道,“神臂弓当真这么厉害?”

宗亶闻言,脸色就变了一下。就是因为神臂弓的存在,李常杰和他都不能再往前走了,下面的士卒还可抵近到城下半里的地方,但他们都是主帅,不能拿自己的性命去冒险。

而且神臂弓的威力和射程,也是宗亶以血的代价,用自己麾下将士的性命换来的。

李常杰自海上攻钦州,而宗亶领军走得陆路。从升龙府渡富良江北上后,一路攻下永平寨、太平寨,连同刚刚被劫掠的古万寨也一并攻了下来。连克多寨,收获颇丰,交趾和广源州联军一时气焰正盛。到了邕州城外,也不做休整,直接就往城门底下杀过去。

并不是宗亶他们看不到邕州城高墙厚,而是之前的几个寨子都没有怎么反抗就自己开了城,弄得他们都以为只要大军开到城下,邕州城中的守军就会杀掉城中主帅,乖乖的开了城门出城投降。

为了能第一个进城,在邕州城这个花花世界里好好发上一笔,几个蛮帅还为此争夺起来,争着攻城的次序。

可谁能想到邕州城上迎接他们的是一蓬密如飞蝗的箭雨。八百具神臂弓齐发,嗡嗡的一阵弓弦响过之后。仅仅数轮射击,就让四百多在城下耀武扬威的蛮兵变成了刺猬。

而领军冲在最前面的蛮帅申景福,戴着头盔、穿着甲胄,照样被射了个通透。箭簇甚至深深的扎进头骨里,费了好半天气力,才从尸身上拔了出来。

这一败,差点就让面和心不合的联军散了架子,最后宗亶没奈何,一口气退了七八里才敢扎下营盘,两天来都没敢去攻城。直到李常杰领军而来,方才声势复振,重又进抵邕州城下。

“神臂弓乃是宋人用来对阵党项、契丹的神兵利器,猝不及防之下,就算是契丹铁骑,也照样提防不住。此番小挫非宗太尉之过。”

声音从身后传来,宗亶立刻转过身。是一个穿着士人服饰的年轻人,仰起的头有着装腔作势的作派。从长相上,一看就不似越人,而是汉人。

那名汉人士子是跟着李常杰一起来的,宗亶也没细问。现在开口插话,士子便走上前来,向着宗亶一礼:“徐百祥拜见宗太尉。”

“你就是徐百祥啊。”宗亶眯起了眼睛。

这个名字他听说过,因为在宋国屡试不第而投书国中,在信中说宋国欲大举以灭交趾,兵法有云:‘先人有夺人之心’,不若先举兵,并请为内应。

虽然一个不第秀才的信,影响不了交趾朝廷的战略规划,所谓内应更是笑话。但他在北进的定策上,也起了推波助澜的作用。

宗亶盯着徐百祥上上下下看了一阵,板起的黑脸逐渐解冻,最后化作一笑:“听说宋国过去曾有个秀才,投了西夏元昊,最后坐到了太师的位置上。不知可有此人?”

“此人名叫张元。”徐百祥宗知道亶想说什么,心情高涨起来:“其人因屡试不中,便愤而投效西夏。元昊能纵横西域,多得其力。若论用兵,韩琦之流远非其敌手。”

徐百祥对张元的遭遇感同身受,他自负才学,腹有韬略,可始终得不到一个官职。既然朝中上下都不长眼睛,遗珠于外,也别怪他投靠交趾。

宗亶哈哈大笑:“张元能做到西夏太师,你投了大越,也未必不能如张元一般。”

徐百祥略略低头,“多谢太尉抬爱。”

只说了几句闲话,让人带了徐百祥下去休息,宗亶脸上收敛起了笑容。徐百祥摆出来的一副卧龙凤雏的态度,让他看了很不舒服。背主的狗竟然还是敢这般倨傲,给根骨头吃就该跪下来山呼万岁感激涕零了。

宗亶哼哼了两声,冲着徐百祥的背影呶呶嘴:“听这措大的口气,似乎是对攻下邕州城有些把握。有说过什么吗?”

“什么都没有,我也没有去问。”李常杰微微一笑,“所谓待价而沽,大概是想等着我们去求他。若是我们在邕州城下碰了头破血流之后,求到他的面前,他恐怕会更高兴一点。”

宗亶眼露凶光:“干脆拿刀跺了他几根手指,看他说不说!”

“何须如此!?邕州城内,连禁军,带厢军,加上溪洞枪杖手,打探得总共有十几个指挥。但宋人的军力你也是知道的,空饷不知吃了几成,实际上最多也只有两三千兵。侬智高当年攻下邕州城时才多少人,我们可是加起来整整有七万兵!难道还会攻不下区区一座邕州城来?!”

就在滔滔左江之滨,李常杰与宗亶指着邕州城,议论起该如何打破这座南疆有数的坚城。邕州城高壕深,的确不是那么容易攻下来。可人数是关键,李常杰和宗亶两人,而且从钦州、廉州、加上太平寨、永平寨,所得到了粮食,足以维持数万大军两个月的战斗。

“不过桂州【今桂林】那边肯定会派援军来,刘彝也不敢坐视。”宗亶沉声说道,“得去堵上昆仑关。”

李常杰冷笑着:“当年侬智高就是太不小心,让狄青连夜冲过昆仑关,弄得只能在邕州城边的归仁铺决战。否则绝不至于败亡得那么快。”

“还有出战的檄文也得早点宣扬出去。”

“那还用说,名正方能言顺,”李常杰哈哈大笑,“‘今闻宋主昏庸,不循圣范;听安石贪邪之计,作青苗助役之科,使百姓膏脂凃地,而资其肥己之谋……’”

这一段李常杰可是每次念起,都觉得妙不可言。

“……本职奉国王命,指道北行,欲清妖孽之波涛,有分土,无分民之意。要扫腥秽之污浊,歌尧天享舜日之佳期,我今出兵,固将拯济……”

这檄文不是让开封城中的皇帝、宰相看的,而是让宋人明白,这一战究竟是谁的错。

“我们可是王师!”

一声尖厉的号角打断了两人的讨论。抬起头来,只见两艘如梭快舟沿着河道飞快的驶近。报警的号角声从前方一直传过来,驻扎在前沿的士卒正拼命的往回赶。

“是宋军!”

“他们竟然敢出城?!”

没等李常杰、宗亶再多惊讶几句,两艘船上的宋军看见这边人多,就直冲了过来。隔着只有三十步的距离举起了神臂弓。

围城的交趾上下,对宋人的反击哪里有防备,船一过来匆匆忙忙的就向后跑。回头一见船上举弩,跟着李常杰和宗亶的亲卫、将佐就连忙将李常杰和宗亶扑倒在地。

“太尉,小心!”

李常杰头被闷在地上,江岸边阴湿的泥土气息充斥了满鼻满口。头上箭矢嗖嗖,听在耳边还有入肉后的闷声喝惨叫。两艘快舟上的弩手射了一轮之后,就立刻放舟顺流而下,直奔邕州城而去。回过神来的交趾军纷纷冲到岸边,向他们张弓怒射,只是船轻水急,转眼就入了护城河中,从水门进了邕州城。

李常杰在亲卫的搀扶下站起身,抹了脸上两把,看看宗亶,也是满脸的污泥。李常杰心头怒火熊熊,突然间周围人看着自己的眼神不对,全都瞄着他的腰背。

侧头下视,却见一支弩矢扎在腰侧。李常杰心头先是一凉,再定睛看时,则松了一口气。抬手拔出了箭矢,箭簇已经穿透了甲叶,要不是身着价值千金的山文甲,换作是皮甲,正中肾门的这一箭就能要了他的性命。

“元福!丁满!”

心中的惊悸和侥幸还未平复,身边又响起了带着哭腔的呼声。李常杰循声看过去,他带着身边的两名裨将此时眼睛睁得老大,如同死鱼一般毫无光泽,身上中的短矢都是扎在了要害处,已经是断气了。

李常杰额头上的青筋一下下的跳着,瞪着邕州的城墙,面目狰狞起来。

“太尉,攻城吧!”

“杀光城里的汉狗!”

涌上来请战的全都是李常杰带过来的精锐。李常杰环目一扫,只见广源州蛮帅没一个出来吭声,宗亶虽是寒着脸,却也没搭腔。

“这是当然的。”李常杰的脸色平复下来,堆出了个如同寒冬的微笑,“不过要按部就班,先将护城河水引走,填平壕沟,这样才好攻城!”

