\section{第28章 官近青云与天通(二)}

韩冈和苏颂领旨后,便匆匆往崇政殿过去。

只是韩冈有些意外。

今天是皇后垂帘的第一天,也是参与崇政殿议事的第一天。正常的情况,最好不要立刻处理实务,先熟悉一下流程再说。

朝廷哪里来的那么多大事?崇政殿议事,也不是天天在说着军国重事。地方人事政事,军中粮秣器械,还是繁琐的居多。

如果皇帝想管,天下四百军州发来的奏章可以让他一天忙上十二个时辰,若是不想管,每天花半刻钟,用朱笔写上三五十个‘可’就够了。要知道,天子诏令的题头从来都不是后世的奉天承运,而是门下——政事堂中书门下的门下。

有些事拖一拖也根本没什么。就算是辽国的使臣来了也一样。现在是霸州的传信,算上消息在路上的时间,萧禧也不过才离开国境南下三天,估摸着还在真定境内,等人到了京城再说也不迟。

立刻让皇后处理实务,这是宰执们想给皇后一个下马威,拿着繁琐的公务将其吓倒呢?还是皇后不想做个纯粹的盖章画押的印把子?

“那个是二大王的车驾吧?”正走着的时候,苏颂的脚步缓了一缓,望着不远处会通门的另一侧。

隔着一道长墙,会通门是沟通崇政殿和禁中两片建筑物唯一的通道。一队车马此时正要通过会通门从禁中出来,多达百余人的队伍,只为了护卫其中唯一的一辆马车。

韩冈眯起眼睛。

人群之中的那一辆四轮马车,形制十分让人眼熟。秋天以来,他不止一次看见过,是将作监精心打造,由天子赵顼特赐给他的三个弟妹。

不过,更重要的是马车周围的士卒。

“是福宁殿那边的金枪东班……”韩冈冷笑起来。

蜀国公主是不可能需要整整一班的天子近卫来‘护卫’,而小心谨慎的赵頵,在听说了昨夜的一切后,不是告病,就是收拾一下行装准备出京,怎么也不可能入宫给自己找不痛快。

“保慈宫那边不知是御龙直,还是御龙骨朵子直。”韩冈低声道。

苏颂咳嗽了一下,韩冈会意一笑,不说了,继续往崇政殿去。

韩冈的眼力好,依稀能看到车厢中那对阴狠怨毒的眼神。也许用不共戴天四个字都不能形容赵颢对自己的恨意,但韩冈现在根本不在意。

雍王已经完了。就算现在太子赵佣出了意外,天子也龙驭宾天,向皇后也照样可以从同族近支那里过继子嗣。三大王赵頵那边还有两个儿子,濮王府那边的选择更多,绝不会轮到赵颢这一支出头。

而且赵顼的身子骨可能支持不了太久了,想必也不会留着他的这个弟弟太久。

没多久,韩冈和苏颂便到了崇政殿外。通了名,便被传入殿中。

殿内的宰执们一个不漏,还有张璪——估计是之前被叫来写诏令的,比起学士院中的其余几位内翰,看来更得皇后信任。几人都被赐了座,更赐了茶。而太子则不在——崇政殿不是礼仪姓质更重一点的朝会,还需要监国太子来妆点门面,怎么不可能让一五岁小儿枯坐在殿里一两个时辰——只有一道屏风拦在御案前。自然,皇后就在屏风后。

韩冈和苏颂向着御座的方向行过礼,起身后便同被赐座赐茶。

“韩学士、苏学士。”皇后的声音从屏后传来,“霸州急报,辽国今岁遣了萧禧为正旦使。方才吾与各位相公商议过,如今圣躬不安,”

苏颂想了一想,先开口道:“天子虽一时抱恙,但也不是辽人可以欺上门的。当镇之以静。”

“不过辽人贪婪,耶律乙辛尤甚。”韩冈继续道,“当初就是因为他,伐夏之役才不能圆满。若听说天子的病情,当是不会坐失良机。”

“臣以为韩冈所言甚是。”韩冈话声刚落,王珪便立刻接口,“耶律乙辛绝不会放过这么好的机会。虏寇畏威而不怀德。退让一步,寇虏便会逼近一步,绝不会见好就收。以臣之见,不论其有何索求,当严词拒绝为上。只要边境上励兵秣马,严阵以待,即便耶律乙辛谋略不输契丹太祖、太宗,也绝难讨好!”

王珪义正辞严。韩冈眨了眨眼睛,苏颂也在发愣。

三旨相公仿佛变了一个人,脱胎换骨一般,在崇政殿上叫嚣着对辽人要强硬到底。

这是谁啊?

在王珪指斥太后、雍王之后,韩冈也清楚当今的宰相是要给自己换个角色形象,至宝丹是做不得了。但转变得太快,还是让人始料未及。

是寇忠愍复生了吗?

在韩冈和苏颂疑惑的时候,宰执们各自表明自己的态度。

吕公著、蔡确、韩缜、薛向主张一切如常,等着辽使上京再做应对,至于边境,则不要做出刺激辽人的反应。而王珪、章惇则主张河北、河东和银夏边境先做好准备,以防万一。

苏颂也是觉得镇之以静比较好。辽使还没进京,何须自己吓自己?自真宗后,已经换了好几个皇帝了,也没见辽人占了什么便宜去。

韩冈当然是觉得边境上要做好准备才是,宁可被辽人小瞧了,也得防着辽人撕毁澶渊之盟的可能。纵然被辽人嘲笑两句,也是不痛不痒,但万一耶律乙辛发了疯,那可就是伤筋动骨了。

“北方年年防秋,至春乃止。有此足矣,何须弄得人心不安?”

“不然。防秋只是依循故事,河北七十余年不经战火,人心早已懈怠。不督促河北四路加紧防备,若事有万一,可是悔之晚矣。”

两边一时间有了些争执。此事说大不大,辽人纵然要南下,也得有一个月的时间来调集兵马。在大宋而言,纵使侦测到辽人的异动后再防备,也是来得及的。

但向皇后却没理会这些争执,反而问,“那萧禧来了该如何应对?”

“一切如常就是了。”皇后两次开口,都提到了萧禧。怎么让人感觉向皇后更担心这位辽国使者,而不是北方数以十万计的契丹铁骑?韩冈心中犯着疑惑,继续说道:“殿下,正旦使年年皆有,萧禧也不过是一介使臣,纵入京,又能为何患?”

“辽使是要上殿陛见的吧?”向皇后却又问道。

“这是自然。”韩冈更是迷惑,不知皇后为何如此发问。

这时宋用臣突然从内侧小门出来,在屏风后低语几句,就见皇后起身离开,继而又把张璪给招了进去,

不同于方才的滔滔不绝,皇后一离开,王珪立刻就变得沉默了。也不似平曰离开崇政殿时那般,还会与同僚聊上几句,就如木偶石雕般坐在一边。

“这是怎么了?到底是出了什么事?”韩冈疑惑的问着。

“令岳方才入宫了,当是为此事。”章惇反问,“玉昆你不知道?”

韩冈摇摇头,这还真是不知道。召了张璪进去,难道是要封王安石为宰相?虽然不是御内东门小殿,又没有锁院,但以现在情况,一切从权也没什么不妥。难怪王珪一下就变得如此沉默。

不过赵顼就算在病榻上还这般勤政,他的身体不知能拖多久?都是一夜未眠,在赵顼这个中风患者身上的影响肯定是更大。

不过他想问不是这一件,韩冈道:“韩冈是想知道,为什么皇后好像不想让萧禧上殿的样子。”

“当然是因为太子!”崇政殿中之人全都惊讶的望着韩冈,“太子才五岁啊,若是被辽人惊吓到怎么办?!”

韩冈还真没有想过这件事,愣了一下后才点了点头,道了一句“原来如此。”

“玉昆。”蔡确有几分迟疑的开口,“皇后如此问,是想让你担任馆伴使。”

“依例当是翰林学士吧?”苏颂立刻诘问道,“怎么能让玉昆来做?”

“就是韩冈接下了馆伴使,也挡住不辽使上殿啊?”韩冈微皱眉,“如果阻止萧禧上殿,岂不是给了辽人以借口?更是示弱之举!”

蔡确解释道:“皇后的意思是有玉昆你陪着几曰,辽使再上殿,也就不容易冲撞到太子了。”

韩冈脸色沉了下来,这是要他来消煞气?!

“玉昆切莫介怀。”蔡确连忙劝着韩冈,“要知道小儿魂识不全,若是太子给辽人冲撞到了,我等做臣子的可是万死莫辞了。”

苏颂不好开口了,其他几名宰执也都有些担心看着韩冈。

宋辽之间的外交采取的是对等的原则,馆伴使在大宋是翰林学士,在辽国则多为林牙——也是翰林学士。论地位,韩冈已经在翰林学士之上,殿阁双学士兼太子师去陪辽国正旦使,这不是对韩冈个人的侮辱,也是国家的耻辱。

只是朝廷的面子的确重要,太子的安危则更重要。谁也不敢说一切照旧,要是太子当真被外表有异于华夏之人的辽国使臣惊吓到了,这个罪责谁来承担?

不过韩冈并没生气,他是啼笑皆非啊,作为拿药王祠当借口的反作用来了,这也是药王弟子的光环带来的麻烦。

他并没有什么消除煞气的能耐,去给萧禧作陪又能怎么样?可是他也不便拒绝。想了一阵,也只能无可奈何的叹上一声。所谓小儿魂识不全的说法,韩冈是不信的。大不了在大庆殿里隔得远一点拜见,让赵佣看不清楚就行了。

这么想着,韩冈就点了点头,“与萧禧周旋一番也无妨。”让众宰执同时松了一口气。

“不过还是得有个翰林学士的差遣。”吕公著道,“否则就是名不正言不顺,也会让辽人小觑了。”

难道带了翰林学士,就不会被萧禧嘲笑?韩冈立刻摇头:“韩冈殊乏文采,不擅四六,当不起玉堂之选。”

“玉昆莫自谦。”韩缜笑道,“你可是天子钦点的进士第九,主编本草。著作都等身了!”

蔡确也十分果决的说道,“若不想书诏,不带知制诰就行了。”

韩冈仍是推拒。没过多久,皇后和张璪出来了,跟在后面的宋用臣手上捧着一封诏书。

皇后在屏风后坐下来,让宋用臣将诏书递给王珪,“官家担心朝堂不稳,北虏窥伺。方才见了王相公后,就任了王相公为平章军国重事。”

