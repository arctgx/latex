\section{第九章 拄剑握槊意未销(六)}

一行骑手从横街的青石板路,走上东十字大街的黄土路面。

蹄铁不再击打石板,清脆的马蹄声消没不见,而大街上行人车马的喧闹则立刻充斥在耳中。

夏天天亮得早,还不到卯时,东面的天空就已经泛白了。清晨鬼市比冬日要早上两刻钟闭市,蒲宗孟的元随也不用打起灯笼来照亮前路。

沿着大街越是向前,大街上的官员就越多,前进的速度也慢了下来。不过他们看到蒲宗孟一行的声势,绝大多数都自觉的将中间的道路空出来。

有资格参加早朝的皆是朝官,在大宋数以万计的官僚中,他们是处在树梢上的那一批人。不过担任翰林学士的蒲宗孟所立足的位置,则更是树梢上最高挑的那几根树枝。除了两府中的宰执,他可算是站在最前面了。尽管还不到宰相那等群臣避道的地位,但也让人不敢跟他争道。

蒲宗孟春风得意,马蹄声急。接连越过几位地位不及他的朝臣,就看见很是醒目的一队人出现在前方。

那一队不论是骑手,还是坐骑都很惹眼。马匹皆是膘肥体壮的河西良驹,而骑手的骑术也都是一流的,在马背上的坐姿,与蒲宗孟自家的元随截然不同。

“可是韩龙图?”蒲宗孟示意身边的元随向前喊话。

只见前面的那一队骑手中央身穿紫袍的官员回头,然后整支队伍就跟其他官员一样,向路边让过去,将中道让了出来。

蒲宗孟收敛起脸上的笑容,却是得意打马上前,龙图阁学士终究不如翰林学士。

到了近前,蒲宗孟轻提马缰,缓了下来,拱手与韩冈行礼致意,而后并辔而行。

“又是一天,”蒲宗孟仰头看了看幽蓝的天空,自嘲的笑道,“昨天听了玉昆的话,夜里都没能睡好觉。一直都梦到灵州有变,官军功亏一篑。”

他瞥了眼韩冈,见其默不做声,叹了一声,“昨天灵州的消息,说是军械、地道皆已准备完毕,次日开始就要全力攻城。以官军之力,今天、明天,消息当是就能传回来了……虽然玉昆反对此战,但想必与宗孟一般,都想听到官军胜绩的捷报吧?”

蒲宗孟说得诚挚无比,让人根本感觉不到其中的恶意。

韩冈转头深深的看了蒲宗孟一眼,叹声问道:“传正,可知夜中天子召宰辅入宫?”

蒲宗孟先是一愣,继而脸色大变:“竟有此事?!”

纵然韩冈没有说明内情,但究竟是为了什么要在半夜召集两府重臣,理由不问可知。不是兵败,就是受困,不会再有其他的原因。

“也有韩冈一份,故而知之。”韩冈丝毫不瞒人,“传正你也知道韩冈在兵事上薄有声名,所以一并被传召。”

“玉昆你夜里奉召入宫了?!”

蒲宗孟话声刚落就知道自己问了蠢话,果然韩冈就笑道:“韩冈这不是跟内翰同行吗?要是半夜奉召入觐,才两个时辰,哪里可能出宫再入宫的?”

蒲宗孟神色数变,最后沉声问道:“究竟是为何故?”

“昨夜没有细问,直接就推了。真要为了聆听详情,奉召夜入宫禁,京城今天还不知怎么传呢?想必几位相公、执政,也都能稳得住。”韩冈又叹了一声,“不过传正昨夜之梦确是梦兆,西北的确是兵败了。”

蒲宗孟脸色由青转红,深呼吸了一下,压下心中火,待要细问,但韩冈却自称不知详情,没办法回答,让蒲宗孟一路心神不宁。

等到了宣德门前,韩冈上前与相熟的官员见礼,找到机会的蒲宗孟忙找来一个平常走得近的文官,向他追询此事。

“的确是有此事。”那名文官比蒲宗孟早到一步,已经听说了。京城之中没有秘密可言,才两个时辰之前发生的事,已经在宣德门前传得尽人皆知,“天子的确是夜中召两府和韩玉昆入宫。”

“可是因为灵州兵败?”蒲宗孟心急的追问。

“内翰方才与韩玉昆同至,难道没听说此事?”那名文官惊讶的反问了一句之后,继续道:“似乎是高遵裕和苗授在灵州城下败了,不过还不确定就是了……但夜中就王相公一人奉召入宫,其他人可都没动。”

“……元厚之也没去?”

文官摇摇头,很肯定的回答:“没有!”

蒲宗孟沉默了下去,右手紧紧握住了拳头。

……………………

韩冈完全没空去考虑蒲宗孟的心理健康问题。

文德殿的常朝,天子例不与会,只由宰相押班。不过王珪并没有到,执政们也在朝会前便被召去了崇政殿。

而作为如今朝中最为知兵、同时也是唯一一个拥有统帅大军经验的文臣,韩冈也同时与吕公著、吕惠卿、元绛三人一起被传召。

跨步进殿,殿中弥漫着一股浓烈的烟气。

添加了龙涎香的御用巨烛向来以烟火气绝少著称,不过从半夜到现在,这几个时辰殿中几十根蜡烛点着,

殿中只有天子赵顼和宰相王珪,两人双眼烟熏火燎,都是红通通。看样子是王珪昨夜奉召入宫,与天子商议了半夜下来的结果。

宰执们终于到场,赵顼犹豫了好一阵,才出声让王珪向其他几名重臣通报了灵州的战情。

听到了具体战败的细节,殿中一时间静默了下来。

等了半天,不见有人就此事发言,赵顼忍不住了,点起元绛:“元卿,你对此事有何看法?”

元绛想了一想,道:“夜半召宰辅入宫掖,虽说因为军情紧急,可当年三川口、好水川和定川寨王师接连败绩,仁宗皇帝也没有半夜大开宫门。西北只是边患,京城民心动摇才是腹心之疾。臣恳请陛下三思。”

“朕知道。”赵顼很是冷淡应了一声,板起的脸有着缺乏血色的苍白。

韩冈在最下首,赵顼和王珪的脸色尽收眼底。元绛昨夜都拒绝入宫,还指望他继续支持王珪吗?

见两人听到元绛的发言后,表情别无二致,韩冈心中有了点,难道之前天子和王珪独处的那段时间里,已经达成了某种默契,希望有人来支持?

“吕卿。”殿中有两位吕姓执政,赵顼叫的是吕惠卿,“不知吕卿有何高见。”

“泾原、环庆的伤亡不明,西贼的动向不明,臣不敢往下定论。”吕惠卿推搪了一下,道:“不过西贼大胜之后士气正盛,此时要抵挡他们的攻势,不论是王中正,还是种谔、李宪,都很难做到,而且少了高苗二帅,两路有被各个击破的危险。还是暂且退兵,日后也好卷土重来。”

这段时间,新党被王珪压制的很惨,太学一案,看声势就是要将新党的根基和未来一网打尽,眼下这么好的机会,吕惠卿不会甘心放过。

赵顼的脸上看不到任何表情,他放过吕惠卿,问吕公著,“吕卿家,你是枢密使,以你之见,究竟该如何方是上策?”

“臣亦是与吕参政同样看法。环庆、泾原两路在灵州城下受到重挫,兵败如山倒,西北战局已经难以挽回。”

吕公著难得的支持吕惠卿,他终于找到翻身的机会。之前因为陈世儒弑母案,吕家在其中牵涉太多,甚至利用大理寺来干扰开封府的断案,吕公著尽管没有被赶出两府,但他说话的份量已经跟他的职位完全配合不上了。如今西北惨败,他的机会终于来了。

“而且还有辽人虎视眈眈。以耶律乙辛之狡诈,听闻官军败绩,岂有不乘火打劫的道理。”吕惠卿附和道。

赵顼脸色难看,吕公著却毫不在意的跟着又道:“陛下此番兴兵伐夏,乃是见及旧日王师连连胜绩之故,以为官军兵锋之锐,世间无物可阻。但西夏之强,非交趾远可比。臣问兵法有云,百里争利则厥上将军。千里突袭灵州,焉有不败之理?此番出兵及民夫几近百万,远趋千里之地,不但军中怨声载道,而且民间也同样困苦不堪。”

韩冈看得都想笑,当真难得……新党和旧党,十几年了,难得一次站在同一条战壕中。

王珪见势不妙,连忙出声道:“王师虽然受挫,但主力尚存,依然坐拥二十余万人马。西贼兵力亦不能过于此,岂有不战自退的做法。”

“自陛下登基以来,用兵兴役,年年不断,国力空耗,而胜果寥寥。今日之败,乃是情理中事,纵然一时夺占兴灵,也难以保全长久——须知李继迁之前,兴灵却也是中国之地。十年之内,臣请陛下不再言兵。”

吕公著毕竟是旧党,终于图穷匕见,吕惠卿这一下就不能再与他统一战线了,“陛下施行新法多年,国用丰足,甲坚兵利,将校堪用,故而有河湟、荆南、横山、西南和交州诸多胜绩。灵州一败,乃是西贼奸猾,致使王师小挫。眼下虽不宜再战,但休养个一年两年,再挑选名将、举兵伐夏也并非难事。”

“四路精兵犹存,如何可退?!”王珪厉声喝问。

元绛则是依然滑不留手,“王师不幸败绩,与国事虽有小损,却幸无大碍。惟国中情势堪忧,臣望陛下对此稍作留意,以防流言,以及奸人作乱。”

