\section{第30章 肘腋萧墙暮色凉(八)}

【从现在开始改变更新时间,改为中午,和晚上。而不是半夜。这样大家看起来也方便一点。】

韩冈和种建中进帐的时候,种谔和种朴父子都在帐中。种朴低头站在一边,种谔脸上则是余怒未消的模样,看起来种谔前面正在训斥种朴,只是听到韩冈和种建中来了,才没有再继续教训儿子。

种谔的相貌与种朴很像,与种建中也有七八分肖似,父子叔侄三人站在一起,没人会怀疑他们的血缘关系。

种谔前面不知因何而生气,不过见到韩冈后,脸色就缓和了许多。韩冈跟他儿子、侄儿都有交情,如果算上王舜臣,更是不一般的关系。虽然也听说了,韩冈在延州跟韩绛顶着来,但看韩相公没有处置韩冈的意思,种谔也不觉得有必要跟关系不错,而且天子都看重的韩冈生分了。

“玉昆,疗养院现在的情况怎么样?”种谔丢下儿子,问韩冈。

“情况很不好。”韩冈摇了摇头,毫不避忌的给现状定了性,“士卒民伕病倒的本就不少,而自残的又是一日多过一日,再这么下去,疗养院快来不及处理新的伤病了。”

“比之前要好就行,左右也没多少天了。”种谔对韩冈忧心很不以为意,死人多点如何,按时完工才是正事,就算民伕闹将起来,这里还有两万大军呢!种谔可是半点不惧。他笑着道:“玉昆你来罗兀后,病死的士卒民伕当即就少了一多半,果然是盛名之下固无虚士。”

韩冈来到罗兀之前,雷简虽然是草创了军中疗养院,但里面的工作一团乱,偌大的病房中,取暖的炉子只有三个,房内跟冰窟一样,护工又像是没头苍蝇,高烧的病人连口水都喝不上,不死人才怪。

就算现在,送进疗养院躺着的病人还是为数不少,但至少有热水喝,有毯子盖,有人照料。护工也有了指派,知道自己该做什么,一切井井有条。

所以此前只听说韩冈名头的种家老少三人,这下才真正佩服起他的手段。至少韩冈这理事之才,是没话说的。

“大帅太夸赞了。这还是多亏了天候的缘故。”韩冈对种谔的夸奖保持着谦逊的态度,不至于一被人夸就得意忘形,“要不是现今是冬天,三万人、数千牲畜齐聚谷中,疾疫当是在所难免。”

“所以说五叔这出兵的时候选得好!”种建中终于找到说话的机会,“冬天疾疫少是一条,而兵出贵奇,党项人也想不到我们会在年节的时候出兵攻打罗兀。”

种谔微微扬起的唇髭,显是他很是为自己挑选的出兵时间而得意。

韩冈也是点头,无论在哪个时段出兵,其实都是有不利的因素存在,当然也存在有利的方面。如何选择出兵时机,就要通过权衡有利和不利的条件来确定。种谔很明显的选择了出其不意,而放弃了能够顺利筑城的季节。

他的这个选择,韩冈无法做出评价。但从种谔一击便攻破罗兀城,并顺利的击败了银州的守军,从而得到了至少一个月缓冲时间的这一点来看,至少这个出战时机,可以算是不错。至于如今筑城时的困难,那就是为了顺利进兵,而需要付出的必要代价了。

不论现在士卒、民伕怎么苦于劳役,但在战术上,种谔的选择没有问题!

“不知玉昆还有什么要求,只要我这边能做到的,只管提。”种谔很大方的说着,对于他欣赏的人,他一向如此。

“大帅能给的都给了,药、粮、人都不缺,韩冈哪还会有别的要求。”韩冈停了一下,又道:“不过,恳请大帅今日能对民伕也能一视同仁。”

“一视同仁?这话怎么说?”

“今次罗兀之捷,虽然卒伍用命。但民伕们也是出工出力,连年节都过不了,说起来与卒伍一般的辛苦……”

种朴打断韩冈的话:“对民伕,在口粮上可没有克扣半点。玉昆你要的热水,也是都给他们安排了下去。你可知道,这两天多耗的柴草,足够日后驻兵时用上一个月的。”

“如果不能让民伕身体康健的把罗兀城筑好,日后也不会有驻兵的机会。”韩冈毫不客气的反驳着,虽然只是管勾伤病,但他在说着民伕也不算越线。要想把伤病之事管好,最好的办法就是从预防疾病开始做起。不仅仅韩冈有这个认识,种谔、种建中他们也都有同样的认识。所以韩冈为民伕要热水热食,还有必要的取暖物资,种谔都尽量满足了他的要求。虽然才两天,但民伕们陆续病倒的势头已经开始渐渐得到遏制。

“民伕急需的不仅仅是粮食和热水,还要有足够的……”韩冈斟酌了一下措辞,吐出了一个字:“爱!”

“爱?”种谔有些嫌恶的拧起眉,“怎么跟燕达一个说辞?”

现在秦凤路兵马副总管燕达,当初就是在鄜延路与种谔公事。他的口头禅就是治兵要以爱为先,在天子面前也是这么说的,就差在脸上刺个爱字出来了。种谔与燕达不对付,早前郭逵守延州,便是弃种谔而用燕达。听着韩冈跟燕达一个调门,心中就是有些不舒服。

“非是与燕逢辰一个说辞,只是人情而已。今天是上元夜,大帅赐了民伕酒肉,只听到方才的呼声,就知道他们的士气当是振作了不少。”韩冈看了种朴一眼,“前几日民伕们士气低落,只在棍棒下拼命。逃亡的民伕的数目可是多的让人吃惊。”

种朴就是负责防备逃卒的,方才种朴被种谔训斥,其原因,多半就是因为捕捉逃人的效率太低了。昨日跑掉七十四人,抓回来斩首有六个,前天大概六十人,追回十一人斩首。这半个月来,总计已经有超过四百人逃亡内地,而被抓住行军法的,则超过六十名,逃亡民伕和士兵的的首级已经在栅栏上挂满了。而有一点可以确定,要不是种谔下令给民伕们赐酒赐肉,今天逃亡的人数还会更多——谁让今天是上元夜。

“可上元节只有一天,如果照着之前的状况继续下去,也许会耽搁最后完工的时间。”

种朴道:“但酒水不多了。”

“伤马还有一些,”韩冈说道,“疗养院中也用不到许多肉。将之赐予民伕,也是大帅的恩德,想必会更为用命。”

韩冈的提议,种谔他低头考虑起来。他并不是不体恤帐下的士卒和民伕,他跟着他父亲种世衡用兵多年,也知道善待部属。不过,种谔善待士卒的目的是胜利,而不是反过来。如果善待士卒和胜利相冲突,他只会选择后者。

种谔他想了一阵,只接受一部分,道:“昨天玉昆你不是跟十七和十九说要做什么分段包干嘛?——先完成的享受就好一点,有肉吃,延误的就照原样来。我看这样就好,要是不论好歹,一律散赏,反倒让人失了上进之心。”

韩冈谢过了种谔选择了他的方案,诱导:“……另外,若有可能,最好能每日公布工程进度。让民伕心里存个希望。”

“有这个必要?!”

这个时代的大部分官员,好像凡事都采用保密主义,不谋于众人,认为愚民就该老老实实听指派,不必动用头脑,韩冈也不以为怪了。殊不知,了解自己的工作内容和进度,对人的工作热情有着极大的促进。

要把人当成人!

即便是善于用兵的种谔,也不知道唤起人们主观能动性的好处,现在只会采用粗暴的强迫手段。比起当年的老种太尉种世衡来,在操纵人心的手法上,着实差了不止一筹。

种世衡守清涧城,以相扑比赛,引得观众主动抗寺庙的大梁上山。以悬银为靶,去引诱帐下子弟去习练箭术。尤其是运用计策,让李元昊杀了起兵时倚之为臂助的野利旺荣和野利遇乞两兄弟,更是种世衡透彻人心的绝佳表现。

野利家是党项大族,从李元昊的祖父李继迁开始,就已经是党项集团的中坚力量。当年李元昊继承父位后,起兵反叛,也得到了野利家鼎力相助,所以李元昊的第一任皇后,就姓野利。

但到了后来,野利家势力日增,李元昊渐渐感到他们尾大不掉。种世衡看到了这一点,就派人带着给野利兄弟的私信潜入西夏,让其故意被捉。这等粗浅的离间计,当然骗不了李元昊这等精明狡诈之辈,但李元昊选择了相信,因为可以以此为由将野利兄弟处决,铲除野利家的势力。

种世衡算计着人心,助李元昊消灭了心腹之患,自己也顺便得到了陷西贼大将致死的功劳。双方虽然不见面,却有着难以明言的默契。而血债累累的野利家的消失,对大宋军民也是个好消息。

说实话,种世衡这等看透人心的眼力,还有将之利用的手腕,就连韩冈都心惊。从种建中的只言片语中听说此事,更是叹息种世衡在官途上的坎坷,如此才智,如何入不得枢府——同时,这也是为什么当年种家的大郎种诂,会进京告宰相庞籍的御状的原因。明明是其父种世衡的功劳,庞籍却不认账,硬说李元昊不至于上这种当。其实,若没有种世衡把刀子递到李元昊手中,想铲除势力庞大的野利家,没有借口的李元昊也不好下手。

种世衡的心计为一时之选,只可惜种谔只学到了皮毛。

韩冈不得不向其解释:“这是让他们知道还有多久就能脱离苦海。越做到后面,就会越拼命。否则,只会越来越疲沓。”

种谔想了一阵,决定还是先看看实效:“也罢,只要能快一点完工,都依玉昆你。”

