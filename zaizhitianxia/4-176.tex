\section{第27章 鸾鹄飞残桐竹冷(上)}

【昨天的断更是众所周知的原因,希望各位书友能够谅解。欠下的两更会在今明两天补齐,】

王安石头脑昏昏沉沉的,尽管戴着水晶眼镜,但手上的一封信笺却仿佛有一层雾在中间挡着,是怎么都看不清楚。

镜片后的两只眼睛死死盯着似乎在摇晃的信笺,好不容易才一个字一个字的读着女婿韩冈寄回来的信。只是没看上两行,就是一只手伸过来,劈手将信纸夺过去。

吴氏气哼哼在床边坐下,板着脸将亲自端来的药汤塞进王安石手中:“都病成这幅模样了,怎么还不肯歇下来?!”

王安石也有些无奈,的确是该歇息的。但躺着睡不着,便又坐了起来,找出韩冈的信来看。

他的这位女婿在交州的一番布置,尽管距离交州收复只过了半年多一点的时间,但大宋在当地的统治已是彻底稳固下来。

这可以说是韩冈在治政上的才华又一次的体现,虽然其中有些手段值得商榷,但都为了国事着想,天子那边也很是赞赏。

而且韩冈的一番行事,值得借鉴的地方很多,他寄回来的每封信,王安石都看过多遍。

只是最新的一封被浑家吴氏生气的攥在手中,王安石也只能无奈的笑道:“这是玉昆的信啊,说着交州的事。”

“辞表都上了,你还操哪门子的心?!”吴氏指着药碗催促着,“还不趁热喝了,冷了可就走了药性了。”

“才上了第二封,来来回回还要两个月的功夫。”

王安石将苦涩的药汤分作几口喝下去,将空碗递给吴氏。吴氏转手又递给站在一边的侍女,将擦嘴的手巾递给丈夫,带着讶异的问道:“难道还是想留在京城?”

王安石摇摇头,叹了一声:“玉昆年底就该回京入觐,有两个月时间,正好可以在出京前,将他的事给安排好。交州事已了,也该调玉昆他回来了。立了这么大的功劳,还让他在岭南待着,也说不过去。”

自从入冬以来,王安石便开始告病求退,辞相的奏表已经上到了第二封。尽管天子都驳回来了,可第三封辞表也已经写好草稿了。

不过折子中的老病本是借口,但今日天气突变,倒是当真让他言出成谶。

开封城的初冬本不是太冷,可唯独今年的天气有些诡异。

前两日还是暖和得如同小阳春一般,往常年份理应已经上身的丝绵夹袄还在太阳底下晒着,府后园中甚至有几株花木乱了时节,在初冬时节的开放。但转眼之间,就是寒风呼啸,北风带着冰雪劈头盖脸的砸向猝不及防的东京城。

这气温降得太快,转眼就是隆冬,让人措手不及。乱了时节的花木在一夜之间尽数凋谢还是小事,东京城中一天就送了七十多无名尸去城西的化人场,加上有主的两百多路上倒毙之人,这才是让开封知府都头疼的麻烦。

同时,急速的变温也带来了大规模的伤风感冒,以及在气温变化中被引发的宿疾和新病,有不少体质衰弱的老人和幼儿没熬过去,开封府中的医生和和尚,都开始了痛苦又幸福的赶场子的生活。

王旖刚刚和素心、周南、云娘三人,商量过要怎么从衣食住行上照顾好儿子女儿,不要生了病。家里面六个小孩子,大的也才五岁,小的还不满周岁,这个时节最是让人担心。

住在相府中,每日的晨昏定省少不了,而王安石生病后,王旖更是要去照看着已尽孝道。当她往父母的房间来问安时,正好看见王旁从父母的房中出来。

见到王旖,王旁的脚步一停,“是二姐儿啊。”

“二哥。”王旖向着房中问道。“爹爹怎么样了?”

“还好,”王旁点着头,“药也吃了,刚刚才睡下。”

“那就好!”王旖放下心来,这个天气对年纪大的人很有些威胁,很容易就出个中风、肺病之类的意外,王安石只是小小的感冒发烧,算是好运气了。

王旁可不觉得‘那就好’,眼下城中到处都有人生病,医生忙得不可开交,连带得他都没有一个清闲。

“今年的天气不对劲。这两天市易务里面十个倒有三个告病。”王旁还记得今天衙门里有多少空位,偏偏赶巧是最忙碌的月底,堆了一堆差事在手上,辛苦了一天,才解决了一部分。

“那还真是要小心了。二哥你也别一起躺下来要人求医问药、”

“也不会有大病,没有什么可怕,倒还能歇一歇了。”王旁满不在乎,忽然又想起了什么,“对了,张横渠真的快不行了。本来前些日子送药过去的时候,他的病情还有了点起色。可这天气一冷下来,他的情况就一天比一天差。刘医正昨日来府里给爹爹问诊时,还顺口说起玉昆的这位恩师,说如果到了春天就不会有大碍了。”

到了春天就不会有大碍了……王旖容色变得微微发白,她如何不清楚这是医家讳言,其实本意是在说张载基本上冬天熬不过去了。

“二哥。”她连忙叫道。

“知道,我知道。”王旁心领神会的忙不迭的点着头,“我明天就上门去探病。”

韩冈不在京城,王旖肯定是不便代夫上门问候,只能转托给王旁。

王旁两天前已经去看望过张载一次。回来后将张载的病情一说,王旖便写了信通知远在广西的丈夫。

如今名震天下的横渠张载,他的肺病已经磨了有十年之久。韩冈本来建议他去找个山清水秀的地方养老,这样运气好时,还能多拖上十几年,可他偏偏要留着京城宣讲经义,最后短了寿数。

王旁也不知是该叹气,还是该感慨和佩服。张载为了宣讲关学,连命都不要了,王旁自问可是做不到这一点。当今世上也少见能像张载这样能毅然决然的不顾性命安危,而将剩余的时间全都投入到对事业的追求上。

王旁摇摇头,虽然自家是做不到,但并不影响到他对张载的这项行为的尊敬和佩服。

第二天,王旁带着一些精选出来的上好药材来到了张载的府邸。宽敞的院落,精美的房屋,这是韩冈和几个学生一同出钱,为张载租用的屋宅。位置不差,环境又好,能从开封府中租到这套宅院,韩冈的面子加上张载的盛名在其中占了大半。

州桥外张载的家中,进进出出的都是士林中人。外院全是人,基本上都曾经聆听过张载的多次宣讲,是他在京城收到的学生。而内院,更有十几名登堂入室的弟子守候着。而除此之外,还有许多敬慕他声望和学问的官员,这些天来陆陆续续的也有几百上千上门来探望的。

这两年国子监中无名儒主持,而张载却是声望正隆,在京中弟子甚多。虽然他在崇文馆和太常礼院都有职位,但他日常精力和时间投注的地方,还是讲学。宣讲关学为主体的经义,同时也包括韩冈在格物致知上所总结出来的学问。

从熙宁八年到熙宁十年,两年多的时间,张载在京中教授的弟子以千百计,而关学一脉对于经义大道的阐述,也逐渐深入人心。

为天地立心,为生民立命,为往圣继绝学,为万世开太平。

这四句作为关学一脉的根本经义,甚至京城中普通百姓,都能说个一二来。而天子据说对这四句话也很赞赏,甚至亲笔在集英殿的素色屏风上写了下来。

王旁将带来探病的礼物让张家的人收下,进去探视了一下。张载差不多已经是进入了弥留之际,妻儿皆在身边,一干得意门生在旁守着。

张载传道授业,用了几十年的时间,教授过的弟子数以千计,但真正得他看重的并不多,也就寥寥十来人而已。人数虽少,却也足以传衣钵。吕大忠、苏昞、吕大临这样的得意门生就守在他的身边。

病痛的折磨下,张载已经瘦脱了形,脸上一片由疾病引起的潮红,呼吸时喉间带着嘶声,甚至许多时候都感觉着他好像连话都不能说,根本都喘不上气来,仿佛溺了水一般。

吕大临看着呼吸艰难的老师,难过的转过身,不忍再看下去。

到了最后的时刻,张载的意识反而愈加的清醒,平生的经历在眼前一一闪过。

幼年时,随母扶亡父灵柩出蜀,因无钱回返开封乡里,最后停在了半道上的横渠镇。自此以后,他便与横渠和关中紧紧联系在一起。读书习文,娶妻生子,被范仲淹所勉励,自此钻研经义大道,考上了进士,又回到关中讲学,直到如今,门生遍及天下。

回想此生,未有虚度,也可去见范文正了。

睁开眼睛,望着房中。

“进伯、季明、与叔……”张载是一个个叫着房中他最亲近的弟子们。吕大忠等人都立刻凑前了上来。

“差不多到时候了。”张载低声说着,

众人闻言都是一震,有几个都不禁落了泪下来。

“‘存,吾顺事;没,吾宁也。’当记着这句话,生死有常,切勿做小儿女态。”张载挣扎的要坐起来,连忙就有人扶上去。

“我要沐浴更衣。”张载临终不乱,依然谨守着儒门的礼仪。

房中一下就忙碌起来。张载望着朝南开的窗户,没能在最后一刻,与得意弟子见上一面,张载有些遗憾。若说日后能光大关学门楣的弟子,韩冈必然是其中一人。

“只可惜玉昆不在。”他低声说着。

