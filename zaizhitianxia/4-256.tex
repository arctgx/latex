\section{第43章 庙堂垂衣天宇泰(三)}

文三在睡梦中惊醒,从床上猛地坐起来。

哭嚎声隐隐从窗外传来,那是让他睡不好觉的元凶。

叹了一口气,文三披上外衣下了空荡荡的床,浑家带着儿女睡在了另外一间房里。

外面的天还是黑的,不过听着外面的更鼓,快到家里的铺子开门的时候了。

冬天的凌晨阴湿寒冷,炉膛中的火也快灭了。文三瑟缩着又回到了床上,提高嗓门喊了一声,不一会儿,睡在外间的小养娘,就打着哈欠上来服侍。

文三当街有个铺面,后面有个两进的院子,能养得起两名小养娘,一个小子,连同浑家和亲生儿女,一家七口人,吃穿都不差。算得上是殷实的人家,不过离富贵二字还远得很。

文三在养娘的服侍下洗漱过后,带着睡在铺面里的小子将铺子的大门开了,给祖宗牌位上了香。转身穿过后门,又到了后巷。

隔邻的李绣线家今日有丧事,周围的邻居听了一夜的号丧。一大早又请了七八名和尚道士。前街是铺面,丧事犯人忌讳。李家就开了后门,来吊祭的亲朋好友,以及做法事的僧道,全都挤在后巷吵吵嚷嚷。

文三倒也没有抱怨什么,婚丧嫁娶哪家都免不了,也不是经常能有的事,说不定哪天就轮到自己,为此开罪邻居也不好。

看了一阵热闹,文三正准备回家去。就看见一个三十多岁五大三粗的壮汉,带着一个老苍头,还有两个挑着担子的伴当,大阔步从巷口走过来。

文三一见来人,便是满面惊喜,冲着家里喊了一声:“三娘,大舅来了!”

浑家的娘家在是伏龙山下的清源村,与襄州城隔了几十里路,进城一趟不容易。平日若是无事,也就是快到了节庆,进城采办时,才会顺便来走一遭。

那汉子昂首阔步的来到文三面前,与文三抱拳行礼,“姑爷,许久不见了,向来可好。”

“一直都好。就是你妹妹和外甥一直惦记着大舅你,都说快仲冬了,怎么还不来。……怎么比往年迟了这么久。”

“乡里有些事耽搁了。”

文三一边将自家的大舅子往屋里迎,一边问着:“岳母身体怎么样?”

“硬朗得很,昨天还带着你嫂子舂年糕。”

“今年年糕舂得还真够早的。”文三笑说了一句,又问:“嫂子、侄儿都还好吧?”

“都好,都好。”

两人叙着寒温,文三的浑家就带了文家的一对儿女出来了。领着儿女行过礼,又将方才文三的问候重复了一遍,从老母一直问到两个外甥,转身就让儿女回了房去。

大汉看得眉头皱了一下,没多说什么,对文三夫妇笑道:“两对熏鸡、两对熏兔、一对熏腿、两斤柿饼,三娘你最喜欢的后院树上生的枣子,娘也特地让俺带了十斤过来,还有今年的新米,都摆在外面的院子里了。”

“大舅太客气了,每次来都带这么多东西。”文三客气着。家里的养娘端了热乎乎的茶汤上来,连着几盘上好的时新果子,一起拿出来招待着大舅子。

“都是田里长的、地里跑的,在乡下也不值什么。”大汉坐下来,喝了口茶,问道:“街口的哪一家谁死了?挂了个白帐子出来,转过街角,一蓬纸钱差点泼到头上,没的撞得晦气。俺连文功近来也是脾气好了,换做是往日,早把他家的门给砸了。”

“大哥有所不知,街口开绣线铺的李家,他家里的大哥好不容易养到十三岁,亲事都说了,偏偏前儿发了痘疮,突然间就病倒了,没拖过七天,昨天人没的。”

“痘疮?”连文功眉毛一挑,嘴角都带了一丝笑意。

文三没在意大舅子的表情,点着头,“就是痘疮,闹得也厉害。现在李家隔壁刘家的两个儿子也跟着前后发病了,你妹妹就怕你外甥和外甥女儿出事,圈在家里不让出门,也不让见客,一起住在西厢里。也就是大舅你今天来,才让他们出来的拜见一下。”文三看了眼脸色苍白的浑家,叹着气,“从李家大哥发病开始,她整天担惊受怕的,一宿一宿的睡不好觉。”

“呦,还真是巧了!”连文功手一拍,说出的话让妹妹、妹夫都想不到。

“巧?!”文连氏脸色一下就刷白了,“村里面痘疮也传开了?!”

“不是,不是!想到哪儿去了。”连文功大笑着摆手。笑声响了好一阵,这才俯身凑前,很是神秘的将声音压得低低的,“你们可知道,伏龙山周围六个村,现在没一家担心什么痘疮了,俺们家也是。今天过来,想给姑爷和三姐提上一句。”

文三眨了眨眼睛,试探的问道:“莫不是来了什么名医?”

文连氏拍案而起,急问着连文功:“那名医诊金多少,我们砸锅卖铁也出得起。”

“用不着砸锅卖铁,李神医每治一人就只收十文钱,几个村子加起来的诊金还没凑足二十贯。”连文功感慨了一声,“的确是个神医啊,都不把钱放在眼里的。只可惜人家来了又走了。”

“走了?”文连氏脸都白了,这不是逗人在玩吗?

“别急。”连文功立刻道,“听俺细细说。若整件事没有个眉目,俺这个做哥哥的怎么有脸来见三姐你?”

文连氏耐着性子坐了下来,文三说道:“大舅,你就别卖关子了,俺和你妹妹心里都急。刘家跟俺家就只隔了两户人家,说不准今天、明天你的外甥就染了病。”

文三,文连氏忍不住眼睛红了,从袖口里掏出手帕,抽抽嗒嗒擦着眼睛。

“不说是不要急吗?我这个做舅舅的还能看着外甥和外甥女出事!?”连文功摇摇头,“鑫哥、青姐现在都还没生病,这就不要紧。若是发了病,就难治了,药都没大用,得靠身子骨去熬,熬过去就算活了,熬不过去那就没办法,不得过痘疮,这孩儿就只能算生了一半。”

听到这里,文连氏一下用手捂着脸,呜呜哭得更厉害,

“我说,我说,三姐你怎么变得这么个急性子,不听完你哭什么。”连文功不高兴的皱起眉头。

“别理你妹妹,大舅你继续说。”文三冷静点,催促着。

“但三姐和姑爷你们也知道,痘疮发过后,不会再得第二次。”连文功顿了一下,看着文家夫妻两个点头,正聚精会神的听着,“所以就有了防痘疮的手段……先得个不伤身子的轻症,出了痘后,重症就染不上身了,一辈子不会有事。这个啊……就叫做种痘。”

“种痘……”文家夫妻念着这个让人陌生的词汇。

连文功点着头道:“就是叫种痘,就跟种花种草一样的种。前些天,家里的才哥儿、二哥儿都跟着村里人一起种了痘。也简单,俺是亲眼在旁边看的,就是肩膀上划一个小口子,种了痘进去,发了一天热,生了两个小痘疮,就没事了。不仅是村里的小子,就连年纪大的,只要没生过痘疮,也一并种了一遍,就十文钱,谁不愿买个安心?俺十一岁的时候得过痘疮了,没去种,但你们的嫂子却种了。”

听了连文功一通话说完,文连氏惊喜不已,一个劲的扯着文三的袖子。但文三做买卖了几十年,骗子见得多了,很冷静:“大哥你也别恼,不是俺不信你,但种痘俺是从来没听过,可是确实能防痘疮?”

文连氏叫了起来:“就是假的,也该试一试!”

连文功没生气,笑道:“姑爷不信不奇怪,俺一开始也不信,村里不信这回事有一大半。但人家李神医献给槐树下刘更家的三小子种了痘,隔了几天,跟着就从隔邻的柳坞庄上找了个生痘疮的小子,从他身上去了痘浆抹到身上,一点事也没有。后来俺大着胆子,让才哥儿也试过了一下,真的是一点没有事。一开始种痘的上百户人家,人人都试过了,没一个发病的。想想吧,几千人里面能没一个聪明的,还能都给骗了?而且人家李神医骗什么,六个庄子,近两千口人,收的诊金还不到二十贯,事后送的礼一点不要,你说这是骗子吗?”

都说到这样了,文连氏信了十足十,而文三也不能不信了,追问道:“这李神医现在究竟在哪里?”

连文功笑了,又是神神秘秘的说着:“你们可知道,李神医不是单独一人到了庄子上的。他身边跟着十几名关西大汉,眼神瘆人,种痘时在旁边盯着,小孩子都不敢哭。听柳坞庄的刘保正说,他们肯定是上过战场的,杀过人。这样的人跟在李神医身边,你们说这李神医究竟是什么身份?”

文连氏急得心里难受,坐立不安,“哥,你就别卖关子了!”

文三见识多,看着大舅子脸上的笑容,脑中灵光一闪:“管着京西转运的小韩学士就是关西人,听说他还是药王孙真人的嫡传弟子!”

连文功手一拍:“还是姑爷聪明。李神医私下里也说了,这种痘的方子是先在广西试过,到了京西再试一次,成了,就能推到天下去了。听听,这么大的口气,又是广西、京西的,不是征过交趾的小韩学士又是哪一个?”

