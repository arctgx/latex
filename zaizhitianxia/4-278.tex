\section{第45章 仁声已逐春风至(下)}

天子和宰执们兴奋得一头热,韩冈平静的问了一句,“那方城山渠道该如何处置?”

场面一下就冷了下来。

被韩冈提醒,赵顼和众宰辅都反应了过来。到底还要不要继续开凿方城渠道?现在开始这就是个问题了。

如今一个月就给朝廷带来两万贯净收入的方城轨道,从一开始,就被韩冈说成是方城渠道修成前临时性的替代方案。一旦方城渠道修成,整条襄汉漕渠贯通,那么方城轨道的作用也就随之消失。

从水到陆,再从陆到水的复杂过程,哪有一艘货船直放东京的顺畅?白痴都不会选择给朝廷在方城山上剥一层皮,就算依靠预定中的船闸也可以收费,但哪里可能比得上方城轨道的收入。

赵顼和一众宰辅很是为难。统治和治理亿万子民的几个人,双眉都向内蹙起,在眉心处挤出几条沟壑来。

运河渠道只要及时疏浚,就能保证长期使用。比如汴河,都是冬天动用民夫疏浚,然后春暖花开之后,就可以用上一整年。而轨道,则是少不了日常维修,人、马、车辆都得备齐。万一哪一天轨道出问题,整条运输线都要瘫痪。

可要是为了这个万一,继续开凿方城渠道,一年几十万贯的净收入就没了。养一名上位军额的禁军士兵,一年差不多要五十贯。少了方城山的五十万贯,也就是少养了一万精兵!以朝廷在襄汉漕运上投入的巨量资金——不仅仅是方城轨道,还包括港口的建设,河流整治,车辆、船只和牲畜的费用——三年就能回本,之后全是净赚。善财难舍,到嘴的肥肉要丢掉,谁能舍得?

吕惠卿皱眉了半天,问韩冈道:“轨道维修上,可会有什么难处?”

“日常维护和整修,之前都有考虑到,安排了人手常年巡视。但到底这份安排能不能让轨道保持稳定,就得看日后常年运行的结果了。光是六十万石纲粮的加急运输,还不足以为凭。”

韩冈说得似乎很保守,但谁都能听得出来,韩冈这是在帮轨道说话。六十万石纲粮的加急运输,其表现出来的运输能力和安全性并不输给汴河,只是没有时间来验证而已。可要是依从了韩冈,日后出了事,他这番话也让人挑不出毛病,追究都没办法。

因为之前的累累功绩,韩冈在营造工程上是朝中数一数二的权威。如果他拍胸脯保证,在场的人都能放心去使用轨道,但他话说得圆滑,顿时便让人少了两分信心,没人愿意就此事拍板。

韩冈不是不想下定论,他一心一意的就是想要推动轨道的发展,以日后的火车和铁路为最终目标。但他不敢保证之后轨道的收入能比得上现在。技术很重要,但管理更为重要,要是人人伸手,轨道走的人少了,也没是钱赚的。

韩冈是不太相信地方官吏的人品,眼下是刚开张,管理严格,加上对怎么从轨道中榨取油水还没有经验,一时不敢伸手,也不知该如何伸手,但时间长了,哪一个都不会放过捞钱的机会。

不过话说回来,阳光照不到的灰色地带,也是有规则的,这个规则在轨道的运行过程中会逐渐成型,然后稳定下来。划定了朝廷和个人的利益分配,到那个时候,朝廷的收入才是正常的收入。可能比现在多,也可能比现在少,韩冈无法确定。干脆丢出去,让天子和宰辅自己去想。

吕惠卿不说话,元绛不说话,王珪当然更不会表明自己的意见,而枢密院唯一到场的郭逵就是块石像,作为由武职担任执政的将领,在政事上的发言权,还不如下面的监察御史,他没资格说话。

赵顼眼睛扫来扫去,见几个宰辅都贯彻着沉默是金的格言,只能道:“此事等薛向来了再说。”

韩冈没想到赵顼还召见了薛向。不过薛向是当朝数一数二的财计大家,长期担任六路发运使,维护朝廷命脉,是纲运上的权威,他的意见自然份量极重,也是必须要听取的。

在薛向到来之前,方城渠道和方城轨道如何取舍的问题,只能先放在一边。但变得没有问题的就是河北轨道。

六十里的方城轨道都能有这么多收入,那七百里河北轨道只会更多。赵顼说出来时,双眼又开始发亮。尽管河北轨道的初衷是军用,可用在民事上,也不会影响到对契丹人的威慑力。

“连接河北各大州府的轨道第一期工程,总共七百里,从白马县对岸,一直延伸到真定,途径河北最为富庶的几个州府。”韩冈停了一下,环目一扫,天子和宰辅都是聚精会神,“从方城轨道上看,过税和运费是相应的,从这几个州府收到的过税反推回去,应当能对轨道货运的收入有个大概的预计。同时长途客运比起短途客运更为艰难,选择轨道的人会更多。官员出行用轨道,各州县节省下来的驿站开支,也不是一个小数目。”

赵顼双眼不只是发亮了,而是在闪光:“王卿,回去后即刻让人去估算一下,建成河北轨道后能增收多少、节省多少,然后尽快报上来!”

“臣恭领圣旨。”王珪躬身领命。

“不过千里转运和百里运输,难度截然不同。就是修路,长达千里的道路和区区六十里,难易与否也有着天壤之别,需要贤能之辈统掌其事。”元绛不是白白做着参知政事,长途轨道的问题看得很清楚。

“诚如元卿所言,当择以贤才。”赵顼瞥了眼韩冈,这位贤才眼下不能用,到底让谁去主持轨道工役,之后又让谁去掌管轨道发运事,都是让人头疼的选择。

宰辅们继续讨论着河北轨道的好处,以及建设和运行时可能会遇到的问题,薛向这时奉召进了殿来。

专业人士到来,赵顼立刻就向他咨询起正在讨论中河北轨道。

听了吕惠卿和韩冈的叙述,薛向沉吟了一下,便开口道:“河北的轨道,一开始的第一条,也就是韩群牧所说第一期工程,是南北向,从京畿渡过黄河开始,一路北上,直抵真定,甚至可能再往北,抵达三关。以其转运之速,震慑北蛮。以臣观之,当可抵得上十万大军。”

赵顼连连点头。薛向曾经在陕西任职过,负责过战备物资的转运,也参与过战略战术的谋划,在军事上也是有一定的水准。说起这方面的话题,也有几分份量。

薛向赞了几句之后,话头一转:“但这条轨道尽管贯通河北,可河北商货往来的数量,远远比不上襄汉漕运。最后朝廷能得到的收入,不能以方城轨道为凭据。”

韩冈不置可否,只是天子视线投过来的时候,头微不可察的上下动了动。不是同意,只是在朝堂上做个优秀的壁花。

顺畅的交通,会促进经济的发展。要致富,先修路,这句话千年之后小孩子都知道。不过这一点,就不是现在的朝臣所能理解。不是他们才智不够,而是他们本身的局限性。

天下财富自有定数,不在此,便在彼。朝廷开源,就是与民争利——这是司马光的观点。

如果是在以农业为主体的社会,而且是和平了一百多年的社会,司马光的观点是没有错的。值得商榷的,也就是对‘民’的定义而已。

“而且轨道几近千里,一车商货自真定运来京城,在河北为止,途中所历州县有真定的栾城、元氏,赵州的高邑、柏乡、临城,邢州的内丘、龙冈、沙河,洺州的永年,磁州的邯郸、滏阳,相州的安阳、汤阴,安利军的黎阳。”薛向扳着手指一个县一个县的数着,显示了他对河北地理了如指掌,“沿途十四县,过税两成八,轨道上车马不能久停,试问过税如何收取?”

赵顼立刻将视线投向韩冈,韩冈还没有动作,就听到薛向自问自答:“以臣愚见,可视沿途所历州县数量,在装车前收取,定好契约关防,至卸载处查验便是。免去了税卡中小吏的刁难,朝廷税入不减,而商人更得其便,如此必有更多商人选择轨道……唯一可虑的,就只是州县中必会有所怨言。”

赵顼摇头,朝廷能直接掌控收入越多,对地方的控制力也就越大,这个目标是开国以来,里代天子就一直在努力的:“此事无妨。”

“另一方面,河北商事不如江南,一是由于民风,北人敦厚朴实,难习商事,另一个,则是缺乏南方水运的便利,运费太过高昂,使得商人无利可图。于河北兴修轨道,商货往来转运便利,必然是商事大兴,与国家财计自有一份补益。”

从薛向开始数着轨道沿途州县开始,韩冈就惊讶得盯住了他。能这般熟悉河北地理,要么就是他对河北官道如掌上观纹,要么就是对今天的入对有了预判,事先下了功夫。

这已经让韩冈有几分惊讶和佩服了,直到听到最后一段,韩冈才发现自己还是小瞧了此人。

