\section{第十章 却惭横刀问戎昭(八)}

【第二更,还有一章,可能要迟些。】

“那个南朝的录事就说了这些?”

“回枢密的话,的确就说了这些。宋人实在是狂妄之极,我大辽铁骑二十万,只要尚父和枢密一声令下,不日就能踏平雁门,竟然还敢如此无礼!”

从雁门县负气而回的使者,在北院枢密副使萧十三的面前,战战兢兢的将自己的经历和对话一五一十的说了一遍的。说到最后,却忍不下心头的火气,差点在萧十三面前破口大骂。

虽然派去的仅仅是一个录事参军,但与宋人的交流事关全局,萧十三还是将他招来详询。从对话到接待,每一个细节都不厌其烦的问了一通。

宋人的愤怒,萧十三也是事先预料到了。已经割了地,又划定了疆界,才几年功夫就又来打草谷了,是可忍、孰不可忍。愤怒是正常的,只是没想到宋人的愤怒会贯彻始终。从出面接待的人选,以及接待的地点,都能看出宋人在刻意将愤怒表现出来,甚至让人感觉有些装模作样。

不过就像大辽将之前对雁门寨新铺的攻击,说成是盗匪所为一样。宋人将两支商队拦了下来,并把商队的主人关进牢狱,也找了个掩人耳目的借口,而不是赤裸裸的说这是报复——其实也算很清楚明白的报复了,只差捅开窗户纸,只是从宋人所找的理由来看,其实事情还有挽回的余地,只要随便砍几个首级,让宋人能够有台阶可下,这件事多半就能不了了之。

但萧十三还不准备这样做,如此一来,他就不太容易下台阶了。肯定会有人说他对南朝太过软弱,这对他的声望并无好处。而且暂时他也不觉得有必要,眼下的这个局势,是没有必要讨好宋人,甚至要尽量让他们分心——这是来自大辽尚父的命令。要不然他也不会下令,让朔州的边军随便去找个宋人的军铺打一下。

只是话说回来,小小的一次对边境军铺的攻击,也并不是单纯的挑衅,那样毫无意义,也未免太小瞧了他萧十三。萧十三是想要趁机看一看宋人的反应。尤其是新任河东路经略使的反应,他受了什么样的诏令,手上有多大权力,对于下属的控制又是如何,都能从宋人的应对中查探出个大概。

从这些天来,宋人对于此事反应,已经能看得出河东军,尤其是代州和太原府之间,很是有些问题。

韩冈的名气虽大,但好像在河东并不是很管用,并没有出现传说中名震军旅、一言九鼎的情况。

代州对于雁门寨新铺之事的报复,快到让人始料不及。而从细作报上来的韩冈北巡的行程看,刘舜卿并没有征得他的同意,完全是自把自为。

可等韩冈到了代州之后,却并没有由此兴师问罪——至少细作没有打听到。

但萧十三并不相信,韩冈能真心为刘舜卿的行为做背书。一个自入官之后,完全没有收到挫折的年轻官员,能忍受下属的自作主张。年轻气盛四个字,简直就是贴在韩冈身上的标签,要不然他又怎么会才抵达河东旬月,便北上代州,未免太积极了一点。

韩冈会默认刘舜卿的所作所为,他的顾虑,萧十三也能体会一二,想来韩冈也不想落得一个对外软弱的评价,只能跟着刘舜卿的脚印继续走下去。

但韩冈也不是全无反击。对大辽使节太过于刻意的慢待,将招待使节的权力,从代州州中转到了下面的雁门县,萧十三都能从中看得出韩冈对刘舜卿的压制。

要想完美的击败对手,就必须彻底了解对手。相应的,能彻底了解一个人,也就能完美的击败他。

当年耶律乙辛和废太子相争,萧十三会选择站在耶律乙辛一边,便是因为他太了解废太子耶律浚的为人。

挥手让犹在废话的使节退了下去,萧十三觉得,至少这一番的出使,让他对韩冈和刘舜卿都多了解了一点。

不过还远远不够,这么想着,萧十三又命人招来了驻扎在朔州的西南面巡检,向他询问边境的那一边,宋人近日有什么动作需要多加关注。

“按照昨日的回报,守卫似乎比往日更为森严……大概是加强了防备。如果想要再命人拔掉宋人一座军铺,就要多费不少手脚。”

“……也只是防备而已,依然是不敢越境反击,比起当年的杨六郎要差不少。宋人可谓是一代不如一代。”

当年承天太后带着圣宗皇帝南下攻宋,因为不擅攻城,从杨延昭镇守的广信军开始,便不断绕过有铜铁之名的梁门、遂城等一众坚城不打,一路南下,直抵黄河岸边的澶州。当两军对峙的时候,一直坚守在广信军的杨延昭,便立刻领军攻入辽国境内,辽人怎么在河北做的,他就在南京道做个同样的。虽然辽人对杨六郎恨之入骨,但佩服他的照样很多。燕山要隘古北口上的杨无敌庙,供的就是杨业,陪祀的也有杨延昭一份。

“纵使英雄如杨业,也不免败亡一途,韩冈就算想要立功,在他立功的过程中,可是有很多人想要拦住他。想要做事可没那么容易。”

……………………

位于群山之中、勾注山颠的雁门寨,的确是一座险关。

雁门寨东西山岩峭拔,中有一路如线,盘旋崎岖,关城便位于绝顶。不过两边的山石,比起关城则又高了许多。

仰首向天,就是早早南行的秋雁,似乎也比寻常所见要低飞了很多,只在隔着关城的两峰山腰上飞过。不愧是雁门,不愧是天下九塞之首。

相对于同为天下九塞之一的方城隘口,地形之险峻,不啻百倍。乃是真真切切的一夫当关、万夫莫开。自战国李牧在此驻军抗击匈奴时起,雁门关就是赫赫有名的天下雄关。

而此时的雁门其实是由雁门、西陉二寨组成,在同一条险道上,一南一北设置了前后呼应的两重防线。二寨倚群峰之险,能将任何由此入关的幻想击碎。

韩冈正在雁门寨中。身为经略使,也只能走到这里,想再往前去更靠近边境西陉寨,正在一边作陪的刘舜卿和一众幕僚,都不会答应。就在几个时辰前,韩冈已经有过一次被下属联手抗命的经历,不打算再来第二次。

之前的几天,由于听说朔州一下进驻了整整两千名骑兵。代州城中的气氛顿时紧张了起来。

韩冈本来预计这两天就回太原府——西面的岢岚、火山虽然也是边境,但那边的道路不适合大队人马出行,没必要巡视——但偏偏遇上了辽人骑兵开始南下边境,要是在此时离开,很容易就给人栽一个临战而逃的罪名可就太冤枉了。

所以韩冈干脆选择再到边境军寨来巡视一趟,宣示自己的胆略。

“想不到雁门也有杨无敌庙。”韩冈在参观过李牧的靖边庙,上香献礼之后,又被引至位于城池西南的地方。他惊讶的望着庙中的金身,没想到这是供奉着杨业的庙宇。

听到韩冈的话,刘舜卿也惊讶起来,“难道秦凤也有供奉杨太尉的庙宇?”

刘舜卿对韩冈的经历也听说过一点,过去没有来过河东,而杨业一生征战,则只在河东一地,两人之间根本没有交集。

“不是,是辽人的南京道那里有一座。”韩冈对刘舜卿道,“我也是听出使辽国的友人说的。在辽国南京道通往中京道的隘口处,有一座杨无敌庙,是辽人感其忠勇,为其修造的庙宇。听说出使辽国的宋臣,过古北口时基本上都要在庙中上一次香。”

韩冈曾经与苏颂聊起过杨业——那是熙宁八年,正巧时任定州路兵马副都总管的杨文广病逝。韩冈和苏颂坐在一起,自然而然的就扯到了杨家祖孙三代身上。

杨业和杨延昭都没有在,但杨文广却在陕西任职过,甘谷城就是他主持修造,韩冈的老爹韩千六曾经为甘谷城运送过粮草,而韩冈两位兄长的战殁,也跟修建甘谷城脱不开干系。说起来的确是有些渊源。

“苏子容【苏颂】当年奉使出京,到了辽境之后,最惊讶的就是古北口上的杨无敌庙。记得他还写了一首诗——做的事跟大部分使臣都差不多。”韩冈轻叹,“可惜的是,都是刻在杨无敌庙山石上的名作,当初还听了一遍,想不到已经记不得了。”

“汉家飞将领熊罴,死战燕山护我师。威信仇方名不灭,至今奚虏奉遗祠。”

将苏颂的描写杨无敌庙的一首诗全都背了出来,这并不是韩冈,而是黄裳。韩冈对诗词一向不是很在意,加上苏颂的这一首水平并不高明,虽说是听苏颂念过,但转眼就忘了一干二净。想不到黄裳竟然听说过,而且记得。

黄裳欠了欠身:“在下曾经拜访过苏学士,也听苏学士说起过使辽时的一些经历。”

“原来如此。”韩冈点点头,这件事倒也不足为奇。

