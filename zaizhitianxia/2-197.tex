\section{第32章 吴钩终用笑冯唐(四)}

韩绛愣了一下,以他的政治智慧,还是很快就反应了过来,投向韩冈的视线中,甚至多了一点感激。

春风得意的时候,他人的礼敬直若常事,而一点不恭就会放在心上;但到了窘迫之事,一点雪中送炭的作为,便能记得很清楚了。韩冈眼下,正是雪中送炭。

可在列的将领却都有些失望,很有几个同时咂了一下嘴,这好戏看不到了。这是他们都知道,却不肯说出来的办法,但给韩冈戳破了。不过韩冈在军中人缘毕竟好,倒没人心生不满,而且韩冈要帮韩绛,也是冒着风险的,谁也不能说什么。对这些军头们来说,只要能弄死王文谅,也就不差了!

——用不擅攻城的蕃将,领着同样不擅攻城的蕃兵,去攻打一座城防森严的雄城,这是让他们去死!

韩冈就是要让王文谅去死。

不过行军法杀人,和让王文谅战死在疆场上,性质是完全不一样的。

一个是伏法的罪囚,一个则是牺牲的烈士。

韩冈倒不在乎王文谅是怎么死的,罪囚也好,烈士也好,人死了就行。可对韩绛来说,就完全不同了。

一旦王文谅舍身成仁,所有对他的指责和攻讦都将嘎然而止。没有人能攻击一位为国捐躯的将领,用生命表现出来的忠诚比言语更有说服力,即便他之前犯过多少错,都不会再被计较。

这就是为什么三川口之败的主帅刘平,好水川之败的主帅任福,以及定川寨之败的主帅葛怀敏,在他们葬送了数万大军并同时葬送了自己之后,还能得到赠官、并且得以封妻荫子的缘故。

当王文谅因殁于王事而不再被追究责任,反而受到封赠的时候,那么他的举主韩绛,也一样不可能再受到指责——一切到此为止!

韩冈的提议,绝对是一个两全其美的计策。韩绛虽然一直对韩冈有些看法,但今天这一下,便彻底改观过来。

“王文谅,韩冈所言确有道理。吴贼虽是污蔑之词,但你也得自证清白才是。这咸阳城,你得用心去攻打。”韩绛也不待王文谅回话,又叫起一人,“白玉,你率本部陪同王文谅去一趟阵前,不要让他有后顾之忧。”

白玉是鄜延路钤辖,韩绛用他去监视王文谅,省得这蕃将狗急跳墙,闹将起来。

白玉领命出列,磕了头后,接过了令箭。

可王文谅却还是在发着愣,他没想到韩冈竟然还有这一手。方才他为了保住自己一条小命,发了疯一般的把人都拖下水,反正早就得罪光了,也没什么好怕的,韩绛也的确是要保着他。

可韩冈这一招实在太过阴毒,一句话就让他必须自蹈死地。王文谅很清楚,他是肯定要去咸阳城下了,他若是不干,今天就别想走出这座白虎节堂。

他恨恨地盯着韩冈,都说措大阴毒,却是一点不差。王文谅现在很后悔,并不是后悔当初得罪了韩冈,而是后悔初次见面时,没能下定决心一斧头生劏了这措大。

韩冈心平气和的劝说着:“王阁职,贼人困于城中已近月余,早已疲惫不堪。以王阁职之武勇,当是能马到功成!”

风凉话说得王文谅好悬没一口血给喷出来,上面的韩绛又开口了:“王文谅,明日本相希望能在咸阳城中为你庆功。”

王文谅出去了,他知道他现在只有一条生路,就是真的把咸阳城打下来。杀了吴逵,逼反广锐军的罪名自然也烟消云散。只是他心中充满了恨意,不仅仅是韩冈,还有韩绛,竟然像丢掉一摊臭狗屎一样,把自己丢了出去。

王文谅脸上的恨意尽数落入韩冈的眼底,他清楚,这其中肯定有针对自己的成分,当然,更多的怨恨必然是指向韩绛。

“韩冈。”韩绛一下变得和颜悦色,“听闻你在罗兀城中尽心尽力,不但份内之事无可挑剔,甚至几次大败西贼,还有你的赞画之功。本相当报之天子,为你请功。”

韩冈低头自谦了几句。他让人看透了韩绛的本来面目,可韩绛却还要承自己的人情,他倒是觉得这事真是越来越有趣了。

为何韩绛在军中人缘这么差?看看他现在如何对待王文谅就知道了。

不过赵瞻,韩冈用眼角余光看了看,好像也正盯着自己。看来帮了韩绛解围,就被他记恨上了。

现在韩冈当真是羡慕起了王中正,这阉货在罗兀城把功劳赚足了,到了延州就很巧的病倒了。根本就不来咸阳,即便平叛之事出了乱子,也与他无关。而天子还要夸他忠勤为国、带病上阵。

不愧是在宫里长大的能人……

“玉昆,你何必多嘴。”散场之后,在堂外听到了内部消息的种建中,陪着韩冈往外走,“王文谅一介小人而已,成不了事,也坏不了事,若非韩相公,何止于此。”

“行了,行了。”韩冈笑着打断,种建中这是掏心窝的跟他说话,他也不会生气,“彝叔你说的我都知道。但韩相公岂是我们动得了的,自有天子去评判。而王文谅那厮实在天怒人怨,早前送他轮回也是一件功德。就不要再说了……”

种建中见韩冈不想提此事,也就不说了,却又叹起:“现在回想起来,玉昆你还真是有先见之明,说今次不能成事,就当真功亏一篑了。”

“再是先见之明,也不可能知道是因为兵变而坏事的。”知道历史的韩冈能确定罗兀城攻防战的最终结果,却猜不到导致结果的原因,拿出来的理由都是凑数的臆测,所以与实际大相径庭,“能料到西贼围城,能料到契丹插足,能料到抚宁堡失陷,却料不到环庆会兵变……世事每每出人意表!”

……………………

泾阳紧邻咸阳,两座城池相距也只有十几二十里,王文谅和白玉奉命出战,几千匹战马转眼渡过泾水。不过一个时辰,就全军抵达了前线。稍作休整,王文谅便领着他的本部,穿过咸阳外围高墙上留下的通道,冲向咸阳城下。

咸阳城中守军虽然以三千叛军为主,但被征发起来的百姓也是在刀枪下,被逼着上城。被重重围起的城市,只能靠着库存来解决日常消耗。幸好咸阳是大城,不缺粮秣军资,就算被围困,也足以支撑一年。

收到消息的吴逵,连忙上了城头。如鹰隼一般锐利的双眼盯上了来敌的旗帜。

“王文谅?……王文谅!”吴逵的声音从疑问到肯定,继而变成了咬牙切齿,“王……文……谅!”

真的是仇人找上门来了!

看着王文谅的将旗在城下飞驰,吴逵突又自言自语起来。“这是诱我出城吗?”

但接下来,王文谅却是带人直奔城下,甚至还能看到一些空着的战马背上,还绑着长梯,竟然是摆出了要攻城的样子。

王文谅能得韩绛看重,不是光靠了溜须拍马,真本事还是有那么一点。先是派人绕城试探了一圈,探出了城防上的薄弱之处,便立刻集中了麾下战力,利用骑兵的高速冲到那里,用弓箭扫射城头,清理出一块空地后。趁守军主力还没来得及赶到,把一同携来的十几具长梯斜斜的往城上一架,王文谅便一手举着盾牌,一手扶着云梯,背着惯用的大斧,领着挑选出来的精锐,一马当先的往城头爬了上去。

在投靠大宋以前,王文谅拼命的时候从来没少过。自幼生长在除了盐和沙子外,什么都缺的西夏,他杀人放火博命的时候,与他同龄的宋人,还不知有没有断奶。被逼到了绝境,王文谅胸口中久违的狠戾,终于又冒出头来。他咬着牙,顶着不断砸到盾牌上的石块箭矢,拼命的向上爬,竟然给他冲上了咸阳城头。

用盾牌挥开刺下来的长枪,王文谅跳上城头,反手取下背上的重斧,用力一挥,便将城上守军斜斜砍成了两截。顺手将重斧横拖竖砍,砍出了一片空地,正要返身把后面的人接上来,一支铁枪嗖然一声直戳了过来。

闪身避过,看清来人,王文谅先是一惊,转瞬又是狰狞起来,“吴逵!”

吴逵却是咧开嘴在开怀笑着,但亲切的笑容中却是满载着杀机:“王阁职……”

两人再无一丝废话,只要杀了对方,自己就算赢了。王文谅将掌中重斧一举,箭步冲前就向吴逵挥了下去。而吴逵也是挺起铁枪,用力向前一戳,毫无畏惧的正面交锋。

环庆路上赫赫有名的一杆铁枪,在吴逵掌中舞动起来,幻化出万千虚影,犹如鬼神一般激荡着嘶嘶尖啸。一圈圈枪影将王文谅笼罩,他纵然亦是武艺精强,但在陕西军中排得上号的枪术宗师面前,却还是差了老远。

不过数合,只听得铛的一声脆响,王文谅的重斧被蕴含千钧之力的铁枪荡开。他踉踉跄跄的连退了两步,一道黑光却是不给片刻喘息的追上了后退中的身形。沉暗的枪尖在王文谅的胸口一搠即收,血水随着铁枪的回收,从创口处迸射出来。

一声凄厉的惨叫震惊四野,王文谅捂着致命的伤口,身子渐渐软倒,可脸上的表情依然狠厉:“吴逵……我在下面等你下来!”

他最终仰倒在地,渐渐失去光彩的双眼望着澄清的天际,最后的一点残存意识让他喃喃出声,“韩绛、韩冈,我在下面等你们下来。”

把王文谅的首级狠狠地跺在了枪尖上,反手拄着铁枪,吴逵在咸阳城的城头上放声狂笑,

“王文谅,只要比你活得长一点就够了!”

