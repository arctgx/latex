\section{第14章 霜蹄追风尝随骠(13)}

【第二更。】

随着韩冈的到来,李宪和折克行两军业已会合在胜州,也即是旧丰州。屈野川一线的防御体系,也变得越发得牢固起来。若是常年有四万以上的兵马坐镇此地,有没有坚固的城垒其实都无关紧要了,可惜人马粮草的消耗,使得四万兵马待不了多久。

同样困于粮草的,还有被迫离乡背井的黑山党项。穿越荒漠的数百里道路上,因为缺乏足够的口粮,黑山党项各部无论人马都是死伤枕籍。在雪下残余的草茎,都被先前经过的部族马匹吃光。从探马传到胜州的消息中可以知道,死在大漠之中的蕃人以成百上千来计算,而且在这些亡者中,有很大一部分是来自于自相残杀、争夺粮食的结果。

“谁让他们引狼入室又不抵抗的。自作孽,不可活啊……”

“当年石晋不正是如此!?耶律德光在开封城中猖狂一时,可一遇河北雄杰,便是暴毙杀胡林!”

“按照龙图的训示,对于来犯的黑山党项,先以攻心为主,现已有十七人愿意代表朝廷去说服他们投诚。一般来说,说服他们不成问题。不过事有万一,也要做好动手的准备。”

“那样的蠢货死光了对任何人都是好事。……再往好处想想,若是辽军南下,想从大漠里绕道,也是以一个结果。”

“没有打草谷的地方,想穿过几百里的大漠南下,能有两三千就了不得了。”

“那时候杀起来可就叫痛快了!”

主持军议的韩冈还没有到,只有李宪和折克行两人带着麾下将领候在正厅中。

两名主将有一句没一句的说着不相干的闲话。即便以折家地位之特殊,折克行也不愿意得罪李宪这个手握军权的大貂珰。而李宪也不会故意跟折克行过不去。韩冈既然前来坐镇胜州,他的态度就摆在那里,闹得军中不合,得罪了折家这个藩镇倒没什么,一并开罪了韩冈,那可就亏大了。

上面两位还能留一份情面,没有撕破脸皮,下面的的将校互相之间却是一句话也没有,甚至连视线都不相交。厅内的气氛像是一根绷紧的弓弦,而且越缴越紧,顿时便险恶起来。

折家色彩太过浓重的麟府军,在河东军中本来就属于异类。麟府兵马司刚开始设立的时候,本是为了监视府州折家和丰州王家之用,有别于折家之前的私军。可西夏立国之后,在抵抗元昊入侵的过程中,不仅仅是丰州陷落,府州被分割,麟府兵马司也受挫极重,不得已开始吸收当地兵源来补充,使得折家将手伸入了这一支本该是监视他们的军队。底层的军官中,大半与折家脱不开关联。

时至今日,朝廷几乎已经默认了折家对麟府军的影响力,尽管主要将领的任免依然由朝廷过问,但实际调用时,却时常将其配属在折家的麾下。

如今麟府军独自斩获大功,惹来河东军的嫉妒也不足为奇。葭芦川大捷功赏虽众,但李宪带人北上,却是特意挑选了在葭芦川大捷中没有立下什么功劳的几部兵马。有功大家分,这是主帅的工作。太过偏袒麾下某人,日后带兵的时候可就麻烦了。

西夏已亡,眼下大战已经进入尾声,胜州一战。现在跟随李宪的河东军兵马,可都卯足劲要在最后关头挣一笔功劳回去。只剩辽宋两家,万一以澶渊之盟为本,再来个七十年不生战火,那到了儿孙,岂不是要吃糠咽菜来过活了?

老资格的将领们都知道,在澶渊之盟后,元昊起兵之前,几十年间,军中将校士卒得到的究竟是个什么样的待遇?管着俸禄发放的文官在面对武夫时,又狂傲到什么样的程度?!在军中时间长一点的老人,现在还记得清清楚楚。

难耐的躁动在每一个河东军将领心中,封妻荫子不是光荣,而是当务之急!

“经略到!”门外传来响亮的通报声。守门的护卫吊着嗓子提醒着厅中,河东路经略使韩冈的到来。

李宪和折克行的对话立刻就停了,帐中的气氛陡然一变。在两名主将的带领下,齐刷刷的站起身,恭候韩冈升帐。

踏进厅中,李宪、折克行领众恭声致礼。韩冈目不斜视,径直走向厅中的主位,转身坐了下来。

折可适、黄裳领着几个幕僚跟在韩冈身后进门,但两人却没有再往里走一步,而是站在最下首两张空着的交椅前。其余还没有得到官身的幕僚,则是站在两侧厅壁处,只有一人走到角落中,那里有一张几案,供他记录军议上的对话和决议。

帐中众将帅行礼后肃立如松,鸦雀无声。

并非是点卯会操时的升帐,不需要太过繁琐的礼仪。韩冈环目一扫,见人人恭谨,遂道了一个字,“坐。”

哗啦啦一阵交椅响后,厅中又安静了下来。众人屏气凝神,静静的等着韩冈发话。

韩冈没有吊人胃口,开门见山。看了一眼折克行:“旧丰州算是占下来了,朝廷也赐了胜州一名。胜州得复,实乃折克行你领军有方。”

折克行立刻起身,拱手道:“末将愧不敢当,此乃末将帐下诸将之功。”

韩冈点点头,“遵道过谦了,朝廷不日当有功赏颁下。”

麟府军的将领们顿时脸上泛起喜色,功赏两个字,从来都是将校们最爱听的话。

韩冈又瞅了瞅一脸不服气的河东诸将,“不过诸位都是老于用兵,应该知道向来是夺城易,据地难。重夺旧丰州殊为不易。不过相比起将北虏的野心打回去,保住胜州之地,还是要容易些许。所以才有李经制领军北上,同心协力,共守丰州。”

就在座位上,李宪欠了欠身。河东众将的脸色也好了了不少。

一番开场白之后,韩冈对众将道:“想必目前的局面不用本帅多说了吧?”

“末将明白。”“李宪明白。”

李宪、折克行带着诸将一阵点头。

韩冈向下首比了个收拾,黄裳立刻会意的站出来:“最近南下的黑山党项增加的数量并不正常,比起预计中数目多了两倍还不止。很明显是辽人背信弃义,连当初投向他们,放他们直取兴灵的那一批部族,也都被驱赶南下。很有可能,辽人主持西京的萧十三,正打算利用这批党项人来骚然胜州。”

这算是对最近的局面做个总结,也是对辽人行动的推测,基本上跟在座将领们的判断相同,并没有人站出来的提出异议。

见没有人反对,事先已经定好口径,折可适便接下去说道:“如果事实如此,便可以看得出来,辽人没有撕破脸皮的打算。所以只能借用黑山党项的手,最多也不过伪装成党项人,给官军添些麻烦。”

韩冈轻叹了一声:“鬼鬼祟祟,不成气候。契丹人真应了一句俗话,黄鼠狼下老鼠,一代不如一代。”

韩冈说得刻薄,顿时引得厅中一阵狂笑。

不过韩冈没有笑:“耶律阿保机看到如今辽国,多半在坟墓里都睡不安稳。窃国之贼非立国之主,耶律乙辛终究还是差一点。记得他当年派去救助西贼的那一队皮室军,也是藏头露尾。正要上阵便不敢,只敢捡捡便宜。”

韩冈直接指称辽太祖的姓名,分明还没有将了辽国当做与大宋平起平坐的国家。不过这个厅中,不可能有人站出来指责。反倒对韩冈的说话,感觉煞是痛快。

都是为了征夏之役出了一份力的,被耶律乙辛做了渔翁,哪个心里都不会舒服。一听韩冈贬低辽人,却感到说到心里去了。

“龙图说得是。”

“龙图说得没错。”

“正是!正是!”

一片声的附和,让几名老成持重的将校,都不好开口。李宪和折克行都由几分不解看着韩冈,不知这样贬低辽人有什么用意。

李宪微微皱眉,正想冷一下热过头的气氛,韩冈却抢前一步开口:“辽人如此自甘堕落,其战力可想而知……诸位,没人会嫌斩首多吧?!”

将校们更是兴奋:“不会!”“当然不会!”“成千上万最好!”

厅中将校们的心情被炒了上来,折克行站出来,试图压一压过火的气氛:“能南逃的黑山党项皆以精壮为主,但近日安置下来的大小六十一族,没有一个完全由精壮男子组成,或多或少的都有些老弱妇孺。若是纯粹由丁壮组成,必然是辽人或是听命于其的党项部族冒充。”

“不,只要不肯听命,或是有所推诿,不论是否是辽人或是受其指使的黑山党项,一律杀之无赦。眼下没有多余的精力去分辨他们。”韩冈杀气腾腾,习惯了背叛的黑山党项,人数少的话,收留下来没问题,但要是太多了,可就麻烦了。胜州之地,可是关键之地,如何能放在他们手中?

眯起眼再一次扫视帐中,“没人嫌斩首功多吧?!”

众将的眼神闪闪发光,如狼似虎一阵狂嚎,“当然不会!”

