\section{第31章 停云静听曲中意(一)}

换过新衣,拜过祖先,吃过年夜饭,给过压岁钱,院子里鞭炮和烟花都准备好了,剩下的就是等待新年的到来。

韩冈一家围坐在正屋中,等着子夜的钟声响起。孩子们都很兴奋,往日这时候早就被逼着去睡觉了,哪里可能熬夜守着天明?

小孩子们自有乳母和婢女服侍,不用太操心。周南和王旖下着棋,云娘在旁边看着。严素心又有了身孕,倒是有些经不住累,眼皮发沉,像是要睡的样子。

韩冈小声的问着,“要不要先去歇着?等钟响了再唤你起来。”

“还好。”严素心抬起头,丈夫眼中的关切之色让她心头暖融融的,“一直闹腾了五六个月,最近已经安稳下来了,熬上一夜也没什么关系。”

“生了四哥之后,官人就想再要姐儿,现在七哥都有了,却没如愿。”周南笑吟吟的说着,手上却不慢,啪的飞了一着,“这一回就看素心你了,可要好好养着。”

严素心的身孕正好六个月,抚着小腹,微笑中满载着幸福:“是男是女都定了下来,怎么养也都迟了。”

“男女都无所谓。”韩冈笑道,“不过家里的玉璋太多,再来一片金瓦才好。”

韩冈不想要太多的子女。时人以五子二女为至福,花瓶、屏风和年画上常常能见到内容相似的图案,韩冈现如今七子一女,觉得也差不多了。儿子多了真不一定是好事,也闹得慌。

韩家的几个儿子精力一个个都旺盛得很,到了现在还都精神十足。小五啃着一块椰子干,啃得满手满脸都是口水,乳母忙忙的在后面拿着手巾在擦。刚刚从交州运来的椰子干香甜的椰奶味道很合小孩子的口味,几个孩子都是舍不得放手,一口接着一口。

王旖看了子女一眼后就有点皱眉头,落了一子,吩咐着下人:“晚上不要让哥儿姐儿吃那么多甜的。”

几个大孩子听了之后,就立刻将手上的椰干丢下。但年纪小的几个却舍不得放手。小五抬头看看王旖,又低头看看手中的椰干,不敢不听话,但就是舍不得,睁着圆溜溜的眼睛,可怜兮兮的望着韩冈。

韩冈对儿女一向宽松,笑了起来,对小五的乳母吩咐道:“吃完后记得让五哥漱口刷牙!”

乳母低声应了,小五欢喜的叫了一声之后,就又开始不管不顾的啃起了椰干来。

王旖狠狠的剜了韩冈一眼,却拿丈夫没办法。

士大夫中一向都很重视口腔的保洁和保养,牙刷、牙粉不必说,柳枝、苦参平常人家都有用。士大夫家饭后还会有专门的漱口水——京城中大一点的酒店也会为客人预备——平时也不忘嘴里含一片鸡舌香。正所谓‘新恩共理犬牙地,昨日同含鸡舌香。’饭后吃甜食,伤牙伤脾胃,更是世间的常识。平日里王旖管得很严,可药王弟子今天却在这里唱反调。说不得,她也只能抱怨:“再这样下去,把孩子都惯坏了。”

“就今天一天。毕竟是除夕嘛。”韩冈摸着五儿子的头,和声说道:“平时就要听娘的话。”

小五乖乖的点头,其他几个孩子也都应了声,又抓起了蜜饯,不过还是收敛了一点。

韩冈也随手拿了一块糖渍的木瓜干尝了尝,甜得厉害,又尝了尝椰子干,同样甜得厉害,“现如今这蜜饯在市面上倒是多了起来。荔枝木瓜芭蕉不说,椰子干过去可少见。”

“还不是官人的功劳!”王旖说着,“全都是岭南的水果。”

“算不上吧。”韩冈并不喜欢甜食,都是咬了一口就丢了下来,跟王旖道:“我可没吩咐过。”

“看到有钱赚,又何须吩咐?”周南笑说着,“争先恐后还来不及。”

各色来自岭南的果脯蜜饯在市面上越来越多,并不是韩冈的吩咐,而是工商业主自然而然的选择。经过二次加工的商品利润,必然是要远大于初步加工的农产品。今年交州五分之一的白糖,都做了蜜饯。

岭南的水果难以储藏,比如荔枝,也得红盐法、白晒法和蜜煎法来炮制,吃不到新鲜的。而白糖虽然金贵,可利用岭南多到只能埋进地里的水果制作成蜜饯之后,价格还能翻了两番——当然将水果做成蜜饯方便。此外还有用糖蜜酿的酒,因酒色金黄,被称为琥珀酒,在内地也十分受欢迎。

以大宋腹地的繁华,白糖也好、蜜饯也好、琥珀酒也好,都是有多少就能消化多少。各家商行当然不会放了钱不赚。比起一开始的时候,单纯贩运白糖、稻米和木料,交州的各家商会赚的可要多得多。

也不仅仅是蜜饯,来自交州的特产甚至还有烤鱼片,倒是挺合韩冈的胃口。喝着低度的琥珀酒,吃着烤鱼片,韩冈与家人等着新年的到来。

来自城中数十寺院的钟声终于响了,悠悠扬扬在空中合奏,左邻右舍都传来噼里啪啦的响声,小儿女们顿时就精神了,立刻欢呼着冲到了院子中。

韩家的鞭炮和烟花早就准备好了,几个家丁拿着燃起的线香点燃了引线。院子里硝烟弥漫,一朵朵烟花飞窜入夜空,鞭炮声也一下变得喧嚣起来。

孩子们被乳母抱着、拉着,捂起耳朵兴奋的看着天空中五颜六色的花朵。只有老大和老二得了韩冈的准许,让云娘带着他们拿着线香去给几个小烟火点火。金娘也想去,却被王旖拉在怀里抱住,不让她乱动。

“爆竹声中一岁除,春风送暖入屠苏。千门万户瞳瞳日,总把新桃换旧符。”韩冈轻轻的念着王安石的旧作。这一首千古名篇,现在想起来却有几分讽刺的味道。

王安石在刚刚开始变法时,意气风发写下了这一首诗。可现在他绝不会有那时的心境了。在朝堂上的王安石,沉默得像是一尊雕像。几乎很难听到他的发言。

鞭炮声震耳欲聋,王旖没听到韩冈的低语,捂着嘴打了个哈欠:“明儿还要去宫里拜年,放过烟火可就要睡了。”

“哪里是明天,已经是今天了。”韩冈笑道,“幸好为夫天亮后不用上朝了,”

因为赵顼的病情,今年的正旦大朝会给免了。在曹太皇重病的时候,也曾罢朝过。要是天家年年都有些三灾六病,倒也不是坏事。韩冈带着几分恶意的想着。他最怕的就是这等繁文缛节——其实也不独是韩冈,绝大多数的朝臣都不喜欢繁冗的仪式,能甘心冒着天寒地冻来参与大朝会,只是为了之后的赏赐——王旖当然也不喜欢。穿着沉重的朝服,绕着宫廷走上半日,能活生生把人累死。

她气哼哼的瞪了韩冈一眼,转又叹起:“不知会不会拜见太后。”

“应该不会。”韩冈摇摇头。这个节骨眼上,不可能让太后参加任何政治活动。

韩冈正跟妻妾说话,前院却突然跑来一人,是守门的司阍,慌慌张张,“学士,外面来了中使,说是宫中传召。”

韩冈与王旖面面相觑,还在欢闹的孩子们也安静了下来。半夜里中使上门,终归不是什么好事。难道是西北的局势有变?韩冈想着,却也不便耽搁,立刻命人大开中门,请中使入府。

“今夜谁人宿卫?”王旖蹭前了两步,小声的问道。

韩冈顿时心中一凛,大过年的,宫中并没有安排任何一名宰辅宿直。这个时间点突然来了人,可说不准时什么事!

不过看到派来宣诏的是向皇后放在太子赵佣身边的刘惟简,韩冈便稍稍安心了一点。再看刘惟简带在身边的几位小黄门和班直的神色,就更放心了几分。只是刘惟简带来的口谕,并没有说明到底是什么缘故。

韩冈一领旨,韩信转身就去安排马匹和随从,。

韩冈看看左右,家人立刻全都避得远了。他低声问刘惟简:“究竟出了何事?”

“官家手能动了。”刘惟简不敢隐瞒,“所以圣人命小人来招学士。”

韩冈一听,不再犹豫,带了人上了马就出门。

中风也是能恢复的,赵顼的病拖了快两个月,其实不论是好转和恶化都不足为奇,只是赶在年节时病情有变,倒跟他在冬至发病一样,让韩冈觉得有点巧合。

可能是回光返照也说不定,要不然口谕中也不会含含糊糊。

上了御街,空气中的硫磺味立刻重了起来,燃放鞭炮的市民三五成群,在如广场一般的御街上随处可见——只有正中央被两条水渠夹隔而出的真正的御道没人敢走上去——韩冈前后左右望了一圈,都没看到宰辅们出行的队伍。

心中平添了一层疑虑,难道事先已经召入宫中了,还是根本就没招?若是同时派人出来,应该能碰上的。

犹疑不定的心情一直持续道韩冈走进福宁殿。

宰辅们没有安排宿卫,不过领兵的武将还是有的。今天宫中值夜的带御器械是王中正。韩冈进了福宁殿,看到王中正在外殿坐镇,最后一分心也放下来了。

不,是一事刚刚放下,一事又上心头。

当韩冈走进寝殿时,躺在病榻上近两个月的皇帝,倏然张开的双眼闪烁的是对权力的渴望。

哈……韩冈低头行礼,事情果然是有趣起来了。

