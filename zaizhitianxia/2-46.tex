\section{第11章 五月鸣蜩闻羌曲(九)}

【第一更。终于达到五十九万字了,离俺承诺过的两个月六十万字之差一步。俺说话算话,各位兄弟也该给个表示吧?】

“果然是针对于本官的。”

韩冈也没想到窦舜卿把仇一闻的徒弟关进大狱,真的是个针对于他的阴谋。对于早前阴谋论式的猜测,虽然在王韶、高遵裕面前说得煞有介事,但他实际上只是抱着有备无患的态度。在韩冈的判断中,除非自己亲自从大狱中捞人,才会点醒李师中和窦舜卿,把他和西贼奸细联系起来。

也幸好韩冈有着有备无患的想法,他让王九、周宁去外面打听关于此事的消息,才听说了窦解和王启年一起去过大狱的事情。窦解死了儿子,他本人去大狱发泄一下愤怒很正常,但王启年跟着去就不对劲了,他是在勾当公事厅里听差,跟监狱毫无瓜葛。

韩冈当时已经有了不好的预感,又让王九、周宁找人去盯着王启年。而就在一个时辰前,他收到急报,王启年被招进了窦舜卿府。如此一来,王启年在其中的作用几乎就是坐实了。

韩冈行事向来直接,从王韶身边借了杨英做个见证,等王启年走出窦府,就将他强行请来一审——小酒店的掌柜和小二也都是王五的熟人——一切便是真相大白。

“窦副总管的关照,还真是让韩冈受宠若惊啊……”韩冈低下头去,冰冷的眼神扫过王启年惊慌失措的脸。

王启年缩着头,眼中尽是畏惧。他被韩冈的手段吓到了,诊脉辨谎,韩冈露出的这一手,他是闻所未闻,但确实把真相给辨了出来。他现在最担心的就是自己为窦家出主意陷害韩冈的这件事被他本人察觉,如果给眼前的这位心狠手辣的韩三官人发现了,自家的小命丢了不算,说不定会把家里人全都连累进去。

王启年的恐惧,在韩冈的意料之中,任谁被人看透了心底,都是会害怕的。但动用了孙思邈弟子这个虚假的身份,却让韩冈有些担心着日后的麻烦。

靠着测量脉搏,来判断言辞真伪,或是事实真相,韩冈只在后世的小说和电视中看过。即便真的存在,那也是传说中的神技,他自己是不可能有这等本事的。

但通过言语、行动来制造压力,突破对手的心理防线,韩冈却是行家里手。何况他又是传说中的药王弟子,更是为表演加分不少。他的这一番精彩演出,由不得人不信。当王启年闭口不言,以为可以让韩冈无所施为的时候。韩冈却奇兵突出,揪出了真相,连旁观的杨英以及王九等四人都惊得发怔,佩服得五体投地。

“你还有什么想说的?”韩冈问着,并示意王五他们把王启年放开。

王启年一被放开,便向后连退数步,只想离韩冈远上一点。但他被押着久了,手足酸软,被周宁伸出脚尖在后一绊,却跌了个四脚朝天。

在哄笑声中王启年爬了起来,心中的羞恼一时间让他忘记了害怕,强咬着牙坚持道,“小人……小人什么都不知道。”

“你还嘴硬!”杨英在旁边狠狠的拍着桌子,只是他的相貌没有王舜臣和赵隆那样的威慑力,不然王启年又得摔上一跤。

韩冈也是不耐烦了,直言道:“不要以为本官会顾忌什么。以我今次在古渭立下的功劳,抵消非刑而杀的罪名,已经绰绰有余了。王启年,你是不是要赌一赌我敢不敢把你乱棍打死在官厅上?”韩冈身子倾前,“就像黄德用,就像陈举,当然还有向钤辖,还得包括那些蕃人。王启年……你想学着他们一样,赌我的手段吗?”

王启年在韩冈身上第一次真切感受到随身而来的杀机。韩冈虽然笑得更为平和恬淡,但眼底的杀机,让他不寒而栗。

王启年张了张嘴,还是什么都没能说出来。韩冈和窦舜卿两个他都不敢得罪,一只蚂蚁夹在两只大象之间,就算韩冈这头大象比窦舜卿要小上许多,但对王启年来讲,都是可以轻而易举就毁了他的大人物。

是君子不吃眼前亏,还是为窦舜卿尽忠到底,王启年犹豫着。

韩冈此时却在心底喊着丢人。为了逼出王启年藏在心中的秘密,他方才的一番话,就像是市井泼皮老大在威胁对手,一点士大夫的风度都没了,实在是有伤脸面。

‘算了,换个手段好了。’他想着,便送了口:“也罢,你既然不想说,那我也不逼问你了。”

韩冈此话一出,王启年便是心惊胆战,周宁、王九摩拳擦掌,又要上去把他夹起来。但韩冈这时又说了,“王启年,你可以走了。”

王启年和杨英他们五人一样都愣住了,韩冈的话让他差点怀疑起耳朵来。但转眼他就反应过来,如蒙大赦,忙不迭的点头。他惊吓了许久,差点胆都要被骇破掉,听到韩冈的话,他转身就往外走,也忘了礼数。

王启年走得急,几步就跨到门口,正要跨出门去。就又听到韩冈在后问道:“这个主意是你出的吧?”

韩冈的话从身后传入耳中,正准备庆祝逃出生天的王启年,顿时如五雷轰顶,浑身就是一抖,腿脚一下都软了,连忙扶住了门框方才站稳。

“原来真的是你啊……”韩冈拖长了声调,这当真是意外之喜,他也不过是心血来潮,顺口问了一下罢了,“你这是何苦来由?”

“你这狗贼真是吃了熊心豹子胆,当真是不要命了。”杨英也被惊到了,“韩官人也是你能算计的?”

被拆穿了藏在心底里的秘密,王启年这下不敢走了,陈举一党的下场,刹那间就在他脑海中走马灯一样的如风旋转,被凌迟的,被斩首的,被绞死的,被流放的,还有被韩冈亲手杀掉的,哪一个有好结果?还有都钤辖向宝,还有吃了亏却始终报不了的窦七衙内。

王启年狠狠骂着自己,他早前真是糊涂了,身后站着的可是西北江湖中传说的破家灭门韩玉昆!

他一下转过身,扑过来抱住韩冈的腿,哭喊着,“韩官人,韩官人,这真的不关小人的事,小人也都是被逼的啊……”

……………………

韩冈站在秦州大狱之外。这座监狱其实就设在州衙之中,全部是用青石所垒就,里面关着的都是些待审的囚犯。而审判过后,有的受刑,有的被流放,还有的被送进各地牢城,充作工役。都不会留在大狱中。

他已经从王启年那里听说了事实真相,却在想着自己的庙算之才,还是比不上传说中的那些名帅。前面自己只算到了大方向,而细节方面却多有错误,尚幸没有影响到大局。

‘幸好还能来得及就下那个老头子。’

韩冈不是心慈手软之辈,如果仇一闻是个陌生人,他绝对是不吝牺牲。但仇一闻是帮助过他的,以德报德也是韩冈的坚持。他做事再直接,再狠厉,行事却也是有原则的,并不是恣意妄为。

高遵裕已经想着牺牲仇一闻这个在秦凤路上广有名声,又深得军中礼敬的老军医,将窦舜卿给拉下马。但韩冈不能坐视,此事他已经跟王韶说过了,今次又让杨英带话给王韶。

韩冈的底限在不让自己陷入危局的情况下,保住仇一闻。这是他的第一目标,除此之外,他得到的都是添头。

站在大狱外,韩冈无意进去一次,看一看仇一闻的弟子,只是为防窦、李二人,他就不能走进大狱半步。但韩冈的耳中却却到一阵笛声,声调有些高亢悲凉,“这是羌笛之声吧?”他问道。

“是那个得罪了窦副总管的党项郎中在吹。”身边跟着个狱中孔目为他解释。

“还挺有兴致的。”韩冈笑了一笑,又望了一眼青苔处处的青石高墙,“就让他多吹一阵子好了。”

韩冈转身便走,根本不进大狱中去见人。

不管窦舜卿在桌面下面做些什么手脚,韩冈都无意奉陪,他所想做的只有一件,就是把桌子给掀掉。

……………………

当天夜里,韩冈在王韶和高遵裕面前,将窦舜卿阴谋的来龙去脉述说了一通。当听到整个阴谋计划竟然是一个小吏要搪塞窦家的那个废物长孙而临时想出来的,无论王韶还是高遵裕,都是摇着头表示难以置信。

而最后,韩冈对整件事的处理,则让高遵裕感到不快。

“你让人送信给窦舜卿了?”高遵裕寒声问道,他还想用此事将窦舜卿或是李师中从秦州赶走。

就因为知道高遵裕是这种想法,韩冈才自作主张,不去征求他的意见。

“到底写了什么?”王韶问道,他很好奇韩冈会写一点什么。此事王韶已经从杨英那里知道,并不是很生气,韩冈知恩图报的表现,让他心中放松不少。王韶半开玩笑的对高遵裕道:“不知窦副总管今晚是吐血,还是会中风?”

“什么都没写。”韩冈却是没有回应王韶玩笑的义务,“信封里就装了空白的一张纸而已。”

“这是什么意思!?”王韶奇怪的问着。曹操送了个空食盒给荀彧,将其逼得仰药自尽,但韩冈送个装着空白一张纸的信封给窦舜卿,又是何意?

高遵裕对韩冈乱了他计划的自作主张,本是很不痛快,但现在听出兴趣了,“是不是嘲笑他白费心机。”

韩冈笑着摇头:“提举可是猜错了,根本就没有任何意思,就让窦副总管拿着张白纸费神去猜好了。”其实除了纸张以外,韩冈还塞了点石粉进去,算是对后世的一个纪念。但实际上,韩冈真正要对窦舜卿说的话,却不在信上,“这封信下官是逼着王启年送进去的。看到王启年,窦舜卿当是明白此事已经被看穿了,他短时间之内也不可能再有什么动作……”

‘而下面就该换我来了。’这一句,韩冈并没有说出来,睚眦必报,向来都是他的优点之一

