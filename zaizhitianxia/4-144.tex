\section{第20章 冥冥鬼神有也无(20)}

【不敢再说什么了……】

旱季中的富良江,远比长江要细要窄,比起水丰时的珠江支流——左江看着也不如。但暴露出来的宽阔河滩,则告诉北方来的异国之人,到了雨季,流淌在这条横贯中南半岛的江河中的水流将会多宽多广。

这一点,韩冈感觉着倒是与黄河有些相似,都是水枯仅在河床中心有水,而到了洪水来临时,便是一下宽阔了数倍,一望十几里,根本都看不到对岸。

江面上没有船只,空荡荡的,只有偶尔跃出水面的不知道什么品种的江鱼。滚滚浊流中,偶尔也泛起从上游飘下来的树干、枝叶、以及人和兽类的尸体。

虽然已经打到了富良江边,但要过江就要让人破费思量了。在升龙府北岸扎下营盘的这几天,都能看到江面上一艘艘交趾人的战船在耀武扬威,欺负着宋人无法过江。不论是用木筏,还是临时打造的渡船,想要将大军运送过去,都要先冲破水面上的这一道防线。

“战船在哪里?”韩冈眯着眼睛。

“就在对岸的港中。”

李信的视力比起韩冈要强些,上百艘大大小小的船只,就停泊在江流对面的港口中,其中比较大的十几艘,与普通商船、渡船在外形上有些区别。

“上游也来了!”

李信的一名亲卫同时喊了起来。

三艘交趾战船从上游直放而下,每一艘都不算大,只有七八丈长,窄长的船型在江水中箭一般的乘风破浪,气势汹汹的顺水冲了下来。

“三哥,快回堤上去。”李信立刻一声断喝。

站在河滩上的一群人太过显眼了,这三条船很明显就是冲着韩冈、李信他们过来的,江水就在十几步外,万一给他们贴着河滩一阵箭射过来,灰头土脸倒是小事,伤到韩冈这位副帅就麻烦了,李信可不打算去堵他们会不会搁浅在河滩上。

韩冈听了,并不逞英雄,转身就往岸上走,几名亲卫簇拥着他,遮挡着箭矢可能飞过来的路线。

李信倒退着也向堤岸上退过去,他惯用的掷矛放在岸上的坐骑那里,腰中只有一柄佩刀。看见三条船上全都张弓搭箭,反手从亲卫手中抢过一张战弓,一箭射了过去。

韩冈走出了射程范围,远远地喊着:“好好招呼,别让他们太得意!”

敌船势如奔马,神臂弓都来不及张开,只有弓箭派上了用场。李信连着射出三箭,他亲卫将长弓保养得还不错,力道要远胜交趾人的战弓。船上射出来的箭矢大部分的都没有飞到李信的脚边,隔着十来步,密密麻麻的插在河滩上。而李信射出的三箭,一箭中了桅杆,两箭射中船帮上的挡箭板,都没落空,就是没有射中人。

三艘轻型战船贴着河滩一晃而过,并没有期待中的搁浅,转眼就去得远了,只留下了一串狂笑和叫骂。韩冈身边的亲卫们都气白了脸,自攻入交趾境内之后,还是第一次见到交趾人如此嚣张狂妄。除了道路难行以外,在交趾国境之内,官军根本就没有受到一点阻碍,现在看到船上的交贼水兵骄狂无比的样子,一众士卒怎么也咽不下这口气,有几个就当场破口大骂。

“你们气个什么?”李信将手上弓丢还给原主,若是换了掷矛,他有足够的把握将挡箭板后的贼人前后开出两个洞,走过来说着,“眼前这条江,打不过去,自然得让他们笑。打过去了,就有得他们哭!”

李信御下甚严,本身又是韩冈的表兄,连韩冈的亲卫,都老实的缩起头来听训。

韩冈笑道:“这群交趾贼寇的狂妄多半是装出来的。他们这几日连番骚扰,当也是想着让我们尽快渡江。不然我们等我们做好准备,一口气杀过去,他们可是撑不住!”

有交趾人的战船在,要么就是暗中潜行到上游或下游、没有交趾战船的水面去,从那里潜渡过江。又或者,在黑夜用木筏小舟渡过江面。两种方案不是不可以,但危险性都不小。谁也不敢保证交趾人一定发现不了。手上的兵力太少了,韩冈和章惇都不愿冒这个风险。

另外富良江口的永安州——在唐时,被称为海门镇的地方——是去年李常杰越海入侵的出海港口,那里决不会缺少船只,只是李常杰也绝不会忘记此事。尽管还是派了一队人马,赶去永安州,韩冈并不指望。但他和章惇都断定,在北岸肯定还能搜集到船只。

“反正我是不信,李常杰能将北岸所有的船只都搜走。”韩冈与李信并肩走上堤岸,“交趾的官府没有这个能耐,只看他们怎么坚壁清野的就知道了。别说交趾,就是在国中,每逢夏秋两季,胥吏下乡催税的时候,哪州哪县的村子不是转眼就少了小半人丁?耕牛猪羊能计入丁产簿的家当,全都不见踪影!富良江边渔民不会少,一艘船就是他们命根子,再怎么都会想方设法的藏起来。”

“现在官军都到了这里,他们肯定都躲到对岸去了。”

“不用担心。”韩冈摇头道,“财帛动人心,往上游去,在江岸边还有上万流民,藏在芦荡和树林中。章子厚已经派人去周围的几家部族,让他们不要去惊扰。这一个个都是有钱的,日夜盼着能过江,你说躲到对岸的渔家,能放过这么好的赚钱机会?”

“已经派了人去找了?!”李信问道。

“表哥你才从西面回来,所以不知道,早就派了人去盯住了。”

李信的确是才回来,自门州南下后的半路上,他就奉命带着两个指挥,去援助攻势受挫的广源州军队。

交趾人并不是完全没有血性,也不是任人宰割的鱼腩,就在西北侧的清州,也就是广源州南下富良江平原的正面,当地的州官组织起了当地的百姓,共同抗击侵略者。左近州县的兵马和百姓全都往清州集结。一时声势浩大,集结了有数万人之多,还包括多达四百头的战象。黄金满和韦首安、申景贵三人,一时大意,都吃了不小的亏。

不论是三十六峒蛮部,还是广源州的蛮军,都缺乏足够的攻坚能力。安南经略招讨司早就跟他们约定好,如果敌军,就会派出援军。不过若是谎报军情,那也不会轻饶。而且章惇、韩冈事先都申明过,如果官军出手帮忙,当先就要瓜分六成的好处。

无利不早起,诸多蛮部都是出来赚钱的,哪一个愿意如此分账,能克服的困难尽量克服,都不愿到官军这里挨上一刀。加之一路上州县中的官员都事先逃跑了,缺乏组织的交趾百姓无人率领反抗,他们的抢劫都是顺利无比,故而这还是第一次有人求救。

章惇点了李信的将,让他领军去帮黄金满一把。尽管清州敌军的声势浩大,看起来也像那么一回事,但李信过去之后,也不用什么计策,直接拿着神臂弓和斩马刀就直冲敌军大营。

交趾军驱动战象反击,只是在李信的指挥下,官军先神臂弓射击,继而又是用掷矛,大象虽然皮厚肉糙,但神臂弓和铁矛都是能扎透铁甲的利器,那是血肉之躯抵挡得了,而且当象军靠得近了之后,官军将士又挺着斩马刀去砍大象的鼻子,那是大象最大的弱点。甫一交锋,象军就节节败退,李信所部并没有付出太大伤亡,就将象军给反着赶了回去。最后踏破大营的,竟然是交趾人自己的军队。

最为依仗的队伍被击败,为首的几人也被斩杀,失去了核心的交趾军溃不成军,最后被蜂拥上来的群狼给分食。而与此同时,一直不肯决定投靠谁人的广源最后一位大首领刘纪,也终于选择了放弃仇恨,投靠大宋。其实他若是再迟一步,李信就会依照韩冈、章惇事先的命令,配合黄、申、韦三家,一起瓜分掉刘纪的地盘和人口。

韩冈与李信一起沿着江堤走着。江堤有三丈多的厚度,一丈多高,夹着江水上下延伸。这一段堤岸虽说比不上黄河金堤,但耗用的人工当不在少数,以交趾的国力,要不是因为富良江两岸是交趾国的腹心精华地带,绝不会耗用上这么大的人力物力财力去整修堤防。

“只要能守住那群流民,少说也能弄个五六条船回来。”韩冈继续着原来的话题。

李信摇摇头:“五六条渔船恐怕不够用。”

“足够将交趾人的战船,一口气给清理掉了。”

李信疑惑的看着韩冈,摸不清他的表弟葫芦里卖的什么药。

“就像李常杰想让我们仓促过江一样,我们也是盼着他的水军主动攻过来。”韩冈指着前方的远处,汇入富良江的支流河口处,一片面积广大的芦荡,“待会儿就去看看造船的地方,两边一起发力,好歹能将交趾水军给逼过来”

李信皱眉想了想,算是大体上有了数,笑问道:“又是三哥儿你的计策?”

韩冈摇了摇头:“不是。是行营参军们集思广议的结果,小弟可没有插上一句嘴。”他转头对李信笑着,“可比一两个人绞尽脑汁容易多了。”

