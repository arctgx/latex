\section{第三章 素意兰心得君怜(上)}

【第二更,求红票,收藏】

跟随着大队,韩冈回到秦州。

当向宝被王韶气得中风的消息在秦州城中传开,往常都对李王之争高谈阔论的秦州官场一时都为之失语。

王韶的手段实在是够狠,抢在向宝前面把托硕部给消灭,让他在几千人面前把脸丢尽。若不是在点将台上听到这个消息,心情急转直下,向宝也不至于被气得中了风。

而且一开始,向宝领军出征的计划,王韶本人也是同意的。但有谁能想到,军议过后,他便直奔古渭寨,抢在向宝之前把功劳攥在自己手中的同时,还顺势将向宝害得万劫不复。这样的心计手段,让人心中不免有些畏惧。一时之间,王韶在秦州官场上的名声,可就往着奸猾狡诈方向去了。

对于此,韩冈则一点也不为王韶担心。的确没有什么好担心的,对于王韶,人们是畏惧,而不是鄙视,是敬而远之,而不是嫌弃。王韶的手段让人有了畏惧之心,但也可以让他们变得安静一点。李师中现在再想设计王韶,要费得手脚可就不是那么简单。

聚七部之力,一举拔掉了木征安排在青渭地区的一颗钉子。王韶在自己能力范围内,可以拍着胸脯说他做到了最好。

当然,他这个最好仅仅是指团聚众羌,共破托硕部这一件事。至于他违反了多少官场规则,得罪了多少官员,这都是王韶现在所无力去考虑的。

王韶的这一带着一丝疯狂的举动,究竟是为了什么,李师中其实隐隐约约的有着认识。作为王韶的老对手,别人没看出王韶今次行事的异样,只以为他是一鸣惊人,但李师中却是看出了王韶,表现了一个与过去两年完全不同的行事风格。

这个风格,并不是属于他,而是属于那个老老实实跟着向宝一起西行的韩冈。韩冈行事,向来是单刀直入,从无一丝退避,军器库、裴峡谷,还是伏羌城,莫不是如此。今次王韶夺向宝之功,也是没有犹豫半分,直接去古渭寨调集蕃部,让向宝的进取成了笑话。

李师中有理由怀疑王韶的做法是得自韩冈的建议,不然他的行事风格不会如此剧烈变化。习惯成自然,要改变行事习惯总是会有外力的因素。

‘这灌园小儿着实惹人厌。’李师中想着。在东门迎接向宝的时候,他的眼神便不时地扫过韩冈。

这个身材高大的灌园子,他为王韶出谋划策也许是为了自保,但他的自保不是寻常人的趋利避害。普通人看见路上跳出一头豺狼虎豹都是绕着走,而韩冈却是会不辞辛劳的直接把山里兽窝一股脑儿给掏了,扒了皮下来给自己做罩衣。

行事从无半点顾忌,无视一切成法。韩冈这样的性子,让李师中都觉得十分的棘手。

他俯下身子,瞧着躺在车上的都钤辖。原本生龙活虎的一条汉子,现在却是动弹一下手脚都觉得吃力。脸色蜡黄,双颊也陷了下去,一副气息奄奄的模样。

李师中的心突的一阵发寒,心道自己跟王韶为敌是不是做错了。王韶本人倒没什么,但韩冈这厮实在是一身晦气,跟他过不去的无不是家破人亡,现在向宝都变成了这副模样。

秦凤经略行事虽然一向不避忌,对鬼神之事也只是泛泛而听。可他看韩冈,想起韩冈的经历,却不得不变得迷信起鬼神之说来,总觉得韩冈是个不折不扣的——灾星!

李师中心中有些混乱,一时忘了该说些什么,城门口,突然间变得静了下来。突如其来的寂静,让李师中惊觉。很快便反应过来的他,低声劝慰了向宝几句,便转身回衙。

韩冈冷眼看着李师中转身而去。隔得远远的那身紫袍渐渐被人群所遮挡。秦州地位最高的官员,现在对自己怕也是无可奈何,要不然也不会看了自家几眼后,就把目光闪躲了开去。

他很清楚秦凤经略对自己有杀心,要不然也不会硬是把他派发给向宝,想着让向宝废了自己。不过现在这样的情况,不知李师中短时间内,还有没有机会对自己动手?还有没有胆量对自己动手?

弄到你死我活的情况,韩冈知道李师中是不怕的,但要是事情激化成你死我也死,两败俱伤的情况呢?若是运气更差一点,李师中难道不会担心,最后事情变成向宝这种情况?

兔死狐悲,是因为狐狸会担心下一个就是自己。而李师中会不会担心自家落到向宝一般的境地?秦州城中,与王韶为敌的官员会不会也有同样的担心?

任何争斗都是要看成本和收获的。一旦与王韶相争,付出的成本让人难以承受,而得到的收获又太过渺茫,这样的情况下,人们又怎么会做?

原则问题有人会坚持到底,但大部分人还是趋利避害的居多。看到向宝的模样,谁还会再为朝堂上的那些大佬、以及一点可能的功劳而跟王韶过不去?

所以事情也就这样了。

韩冈一声冷笑,事情也就是这样了。

在衙门里缴了令,韩冈今次的任务也就告一段落。就是出外走了一圈,什么都没做,只看了一场好戏,倒像是旅游。当然,这种在路上提心吊胆的旅行,韩冈不想来第二次,但向宝在最得意的时候被打落地狱,这样的痛快场面却是看几次都无妨。

勾当公事厅里四个同僚都到齐了,这还是第一次。即便是韩冈刚刚上任的最初的那几天,官厅中也都是有人休沐,有人请假,而人数始终凑不齐全。韩冈进去打了个招呼,就转了出来。那半个月,他一人忙得团团转,现在暂时还不想坐在官厅中,而他的几个同僚,也没脸让韩冈再留下来做事。

出了衙门,韩冈径直回家。今天这一程是从陇城县过来,走了也有半日,时已过午,韩冈肚中也饿了。

听着肚子咕咕在叫,韩冈想起来当日他娘要找的厨娘,现在应该选定了才是。

只是见到家中新添的那名厨娘,韩冈却一下愣住了。他真是没想到,牙婆找来的厨娘他竟然认识……说认识有点太过想当然,只是在路边有过一面之雅,顺便帮了点小忙,但这未免也太巧了一点。

却见她亭亭走到韩冈面前,敛衽为礼,道了声万福:“严素心拜见官人。”

“这位严小娘子,长得一副好相貌,做得一手好菜,女红也是一般的出色,三哥儿看看,她绣得这个鞋样有多精致。”

介绍严素心来的牙婆韩冈没见着,但韩阿李却仿佛变成了媒婆的模样,在韩冈面前尽夸着严素心的好。

韩冈笑了笑,问道:“严小娘子,令嫒可否痊愈?”

自从前两天进了韩家门,严素心一直都在想着韩冈见到自己时会说什么。但她还是没想到韩冈会问到这件事。先呆了一下,知道韩冈的误会,忙回道::“招儿非小女子之女,只是她娘亲过世,举目无亲,所以跟在小女子身边。素心多谢当日官人解囊相助,救了招儿的性命。”

“所以说这事巧得很,当真是缘分。”韩阿李笑得很开怀,她很满意严素心,她本意找得也不是厨娘。而且自家儿子当日还帮过她,在严素心进门时她就已经说过了。早早的就结了善缘,难道还有比这更理想的人选?

韩冈心如明镜一般,自家娘亲转着什么念头,自己这个做儿子的怎么会不知?不过他看严素心的感觉也很好,而且谈吐文雅,举止从容,倒有些像是大户人家出身。

多半是在书香门第里做过事。韩冈猜测着。世间大户让仆人读书的不多,但红袖添香,素手磨墨却是每个士子的梦想,婢女读写诗书却是很常见。

“不知严小娘子早前在哪家做事?”

“是在陈举家。”严素心毫不隐瞒。

韩冈心神猛然一凛:“是那个陈举?!”

严素心低下头:“小女子不敢欺瞒官人。”

‘陈举啊……’韩冈对严素心的身份有些顾忌。虽然他看严素心,不像是会为陈举报仇雪恨的模样。但自己让陈举家破人亡,举族尽灭,对陈家出来的人,自然会有些心结。

但韩冈又看着韩阿李,却是一副毫不在意的样子,难道自己有什么误会不成?

严素心这时在韩冈面前跪倒:“家严本是成纪主簿,曾欲举发陈举不法之事,却为陈举所害,连家慈亦是被陈举凌迫而死。”

说起家仇,严素心泪水不住的从眼中流出,划过白皙的脸颊,一滴滴的落在地上。只听着她哭诉着:“小女子在陈家苟且偷生,本意是想着为父母报仇雪恨,让陈家举族覆亡。但这些年来,始终没有等到机会。本以为这辈子无法再如愿,不意有官人出手,让小女子的血海深仇终于得雪。官人大恩大德,小女子粉身难报,愿从此做牛做马,服侍官人。”

“三哥儿,素心她说的都是真的。前两日周家小哥和王五过来,也是这么这么说的。”

韩冈点了点头,自陈举倒台后,成纪县衙有了不少空缺,韩冈趁机在其中安插了不少人手,比如周宁周凤、王五王九,有他们在,严素心有没有撒谎,的确是一查便知。

只是韩冈没想到,韩阿李能想到利用他们,自己的这位老娘,还当真不简单。

