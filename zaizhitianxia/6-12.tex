\section{第二章 天危欲倾何敬恭(九)}

蔡确的宰相府,韩冈没怎么来过。

其规模要略逊于王安石的平章府,却比韩冈的府邸要大上许多。

门口的系马桩上,已经被一道道绳索捆扎得结结实实,也幸亏有这样的大门面,才能站得下每天都会涌过来的官员,以及请好问安的信使。

入暮时分,蔡确府上依然宾客盈门,能容十马并行的街巷,被数百车马堵得水泄不通,想往里面走,都踩在人头上过去。

不过无论韩冈到了哪家宰辅府上拜访,都会让守在宰辅门前,等待召见的官员们一片混乱。当韩冈出现在蔡府巷口,等候接见的官员和他们的随从,立刻就给他让出了一条道来。

与沈括议论过开封府的资金问题,次日韩冈便致书蔡确,约好上门拜访。

被蔡确的弟弟蔡硕接入府中,蔡确就在中门处迎接韩冈。

韩冈走上前,当朝宰相迎面就大笑:“玉昆可是稀客,难得,难得。”

韩冈拱手行礼:“当初身冇份尴尬不便登门,现在倒是方便了。”

“玉昆不在朝堂,答疑解惑可就少了人了。”

“相公远见卓识,何须韩冈在朝堂上多言?”

韩冈与蔡确相互谦让着,寒暄了几句,蔡确就抬手迎韩冈往内院走。

蔡府比起韩冈常去的王安石、章敦两家,要奢华许多。两侧廊下挂着一排的玻璃灯盏,映得院中一片透亮。奔走的仆役数量也多,百来步的距离,倒有五六十个

“玉昆今日登门,可是有所指教?”为韩冈引路,蔡确徐徐问道。

“只是过来讨杯茶喝。”

“茶?”蔡确笑了起来,“玉昆你家占着十几株百年老茶树,秦州出产好茶叶全都进了你家,不说分润一点,却来我家蹭茶喝,你这可算是盗劫贫家,当罪加一等啊。”

“就是不值钱的野山茶。过去有谁喝?看都没人看。也就如今才金贵起来了,却也不过是一阵风而已。相公若想要,韩冈明儿就让人送些过来。”

“那我可就却之不恭了。”

蔡确大笑着,拉着韩冈的手,一起进了见客的小厅。谦让着落座,蔡家仆人端上来的两杯碧绿的热茶汤,正是韩冈惯常所用的炒青散茶。

依据路陆羽的茶经,世间过去喝茶,流行的是蒸青。将采来的茶叶,上屉蒸过后冷水清洗,小榨去水,大榨去茶汁,去汁后置瓦盆内兑水研细,再入龙凤模压饼、烘干,是为团茶。

喝团茶时,是要先磨成粉,再调和成膏,而后将热水冲入杯中,一边冲一边再搅合,搅出厚厚的沫子来。世人斗茶,就是看这一套泡茶手法的水平,以及最后沫子上凝出的花样。这斗茶的风气,上至王公,下至走卒,都有这一爱好。

可是韩冈嫌麻烦,口味上也不习惯,所以只喝炒青。茶叶摘回来后在铁锅里炒一冇炒就好了,要喝开水一泡就行。早年他这样做,还被人嘲笑是小门小户出身,寒酸惯了。

不过随着他精于医道的名声渐广,尤其是种痘法问世之后,身冇份顿时特别起来,一举一动惹人注目,专喝炒青的习惯,便被世人认定是养生的法门,连带着秦岭深山中的那些野茶树,都一下子价值千金。

秦州天水县,就是韩冈平常所饮山茶的出产地,位于秦岭之南,如今多少人家都开始在山中采摘野茶,成了贴补家用的又一门买卖,虽刚刚开了头,但眼见着就兴盛了起来。

蔡确呷了一口茶汤:“喝多了炒青散茶,团茶倒是难喝惯了。”

“炒青能见真味,苦而后甘,余韵绵长。而如今的团茶,掺入香料太多,就感觉味道太杂,失了真趣。”

“杂?玉昆这话说得好。的确是太杂了,没了纯粹,不见本来面目。正如行文当求本真,浮艳雕饰就失去了原味了。”

蔡确正说到点子上了。如今文章讲究自然复古,作画也是重气象、意境“师诸物者,未若师诸心’。像龙团那样,外饰金银,内掺香料,看着贵重,却背离了近年来士林中渐渐流行起来的自然求真的风气。反倒是炒青散茶,却十分贴合这一流行。

“相公这本真一词用得好。求本求真,方能明心见性。”

“玉昆你倒是三句话不离本行啊。”蔡确哈哈大笑。

韩冈抚着茶盏。他不辨瓷器,不知道这茶盏是哪里的出产,不过宰相家里拿出来待客的,自不会是凡品。

“清茶本真,纯而不杂。不过散茶有一点不好,就是不宜输送,压紧了便碎了,茶饼、茶团就要好很多,吐蕃人、辽人都喜欢。”

蒸青后,要经过压榨,压制成的茶饼,自然比散茶更方便运输,也更受蕃人、夷人喜爱。就是千年之后,蒸青发展成砖茶,还是北方和西北民族日常饮食的不二选择。

蔡确举杯笑道:“好东西还是留给自己吧。”

“相公说得正是。不过辽人那边,但凡中冇国有什么好东西,都会想方设法的弄过去。每年的岁币几乎都是在他们手中转上一圈就回来了,瓷器、茶叶、各色器皿。如今京冇城中喝散茶的渐多,怕辽人也不会吝啬。”

“加上现在又从日冇本赚了一笔……?”蔡确问。他等韩冈绕来绕去,终究是绕到了想说的话上了。

韩冈点头,他现在刚刚离任而已,还不至于人情冷淡。不过时间一长,还想要对朝政保持原来的影响力,那就难说了。至少要维持自己在擅长领域上的发言权,让朝廷必须借重自己。

“日冇本多金银,辽国这一番攻打日冇本。若韩冈所料不差的话,每年从日冇本得到的收获,恐怕不会比岁币少。”

蔡确点头:“玉昆你的话,我们都是相信的。”

韩冈叹了一口气:“这一回主张入寇日冇本的辽帅,是耶律乙辛的嫡长子耶律保宁。若日冇本的金银产出被他抓到手中,他的地位立刻就稳固了起来。”

“自然。”蔡确又点头。

有关耶律乙辛和他儿子的事,已经在朝堂上讨论过了。当初朝中议论辽国内事,都觉得耶律乙辛年纪已老,寿数不永,其子耶律保宁又声名不显,素无威望。就算给耶律乙辛篡了位,等他死后,耶律保宁也守不住,辽国必然要乱。

可现在辽军一下就占了高丽、夺了日冇本,高丽的土地、人口,日冇本的金银、特产,都成了辽国的财富。这让耶律乙辛、耶律保宁两父子在辽国国中的地位比过去稳固了十倍。而且敢于主张过海攻打日冇本,耶律保宁想来也不是一个简单的人物。

“人心所向,手上又有了钱,辽国国内,当已是无人能阻止耶律乙辛了。”

蔡确叹了一声,听起来似乎有些羡慕:“竟当真给这个乱臣贼子赢了。”

“不过辽国有钱,也不会存在库中发霉。”韩冈又轻松的笑了起来,“终究还是要用出来。用来买中冇国的特产。不管他们从地里挖出来多少金银,最后都会送到国内来。”

“难道辽人就没拿东西走?”蔡确哼了一声,又不是岁币,那是买卖。

“矿总有挖空的时候,但茶叶、丝绸、布匹、瓷器,这些商货却是源源不绝,永远都不会冇断的。百姓得了生计,国家得了金银,辽人有了钱,也就没了南下犯境的想法。这不是好事吗?”

蔡确稍作沉吟,怡然点头。叹着:“若是耶律乙辛早一年打下日冇本,说不定就没去年的那一场大战了。”

“也说不准。北虏如虎狼,想要让他们不吃人,得将他们打痛了再说。几十年不吃教训,都忘了痛了。就算拿到日冇本的金矿银矿,可大宋这边是金山银海,岂是日冇本能比?”

“说得也是。”蔡确端起茶盏,喝了口茶水,慢慢的问道,“玉昆你今天来提日冇本的金银,你可打算让铸币局铸金钱、银钱?”

“大额的钱币,总是有用处的。朝廷用来付账,用铜钱总是不方便。”

蔡确想了想:“封桩库中,储存金、银钱,比铜钱也更合适。”

“还是用出去的好。铜钱放在库房里,库吏偷钱也就一百、两百,换成金钱、银钱,可就是十贯、二十贯。真要放库中,铸成数百斤重的金块银块,容易清点,又能让贼人搬不动。”

蔡确失声笑道:“这话说的有道理。朝廷花钱的地方很多,可不包括养老鼠。”

“此辈硕鼠,杀之不尽。”

“也只能尽量防着了。”

“花钱的地方虽多,不过能节省下来的地方很多啊。光是军费就多少了?”

“去年没能省下,不过今年可就没问题了。朝廷的手头上也能宽裕些了。”

西夏灭亡,关中腹地再无外患。原本至少占去天下军费一半的西军,开支有了大幅度的回落。单纯的维持费用,远比战时要少上许多。如果去年不是因为辽国入侵,花在百万大军头上的军费,至少要减去一千万贯。

“没辽国捣乱,光是战时军费,当然能节省得下来。还有西军裁撤,又能节省一笔。”

