\section{第23章 弭患销祸知何补(一)}

已经是更漏夜深的时候,韩冈还在等待城西医院的统计报告。

房中的灯火一直亮着,听到消息之后,韩冈并没有立刻从城外的赛马场回来,而是让何矩去现场做确认。在得到了何矩的更为详细的汇报之后,他才派了韩信去通知城西医院,让他们做好接收伤员的准备,并要求韩信留在那里,居中传递消息。

在韩冈定下的章程中,一旦地方上出现任何灾害或意外造成的大规模伤亡,加派医护人员加以救治,并对伤亡人数进行统计,是厚生司及其外派机构的分内之责。

不过这本是对之前厚生司在自然灾害上的责任,加以明文确认,顺便添了一条意外伤亡而已,没想到这么快就派上用场了。

何矩站在韩冈的对面,韩冈有让他坐下来,不过他还是坚持站着。低垂着头,一张愁眉苦脸摆在韩冈的面前。

死伤人数虽然没有确认,但超过一百是确定的,在何矩来禀报的时候,已经能确定有十人死亡,以及十倍于此的伤员。

一边一个用木头榫接起来的门框,后面还有兜着球的网。球场中央一条线将球场一分为二,开球的地方就在这条线的正中央。专门用来计时的信香点着,以确定比赛时间。蹴鞠用的球场就是这么简单。没有守门员,没有越位,也没有禁区什么的,只有禁止手臂和手触碰气毬的规则。另外红黄牌也有,这是用来惩罚恶意伤人的球员。而问题就发生在一张红牌上。

这是棉行喜乐丰队和京北第二厢第一坊福庆坊的福庆队的比赛,两队都有争夺季后赛入场券的希望,所以这场比赛的结果,将在很大程度上决定祥符县这个分赛区的最终排名。

灾难发生在比赛结束之后。这场比赛,喜乐丰队是以大比分取胜,但在比赛的中段,裁判将一名福庆队的主力球员罚下了场。故而最后的结果,惹起了福庆队支持者们的怒火。从争吵,到投掷杂物和石块,再到球场上的斗殴,最后变成波及整个球场的骚乱,只用了半刻钟的时间。

韩冈低头看着,晕黄的烛光照在他手中的稿上。

在韩冈的桌上散发光芒的不是旧有的用纱罩笼着的烛台,而是一个透明的玻璃罩子,无遮挡的光线更加明亮,上的字迹也便更加清晰。不过韩冈没准备让陇西的作坊大量制作用玻璃做灯罩的烛台,他希望看到的是煤油灯,而且是后世的那种发光比较稳定,不会因为摇晃而漏油的煤油灯。家乡的玻璃工坊,现在正召集了一干一流的工匠,依照韩冈的要求进行开发。或许一两年之内,就能看到成果。

他一页页的翻着,却没有看进去多少,对于今日的惨剧,着实让他有些后悔,当初要是多坚持一下就好了。

“要是当初将球赛的赛场安排在赛马场中就好了。”韩冈突然放下,长声叹道。

何矩脸上挤出来的笑比哭还难看,“两家从一开始就合不来,哪里能想到会有今天的事。当初冯东主也曾经在两社中提过这一事,但两边有门户之见,都不肯答应。”

“这不是门户之见……”韩冈摇头,“是怕短了自己的那一份钱。”

归根到底都是利益。

韩冈曾有意将两项赛事放在一个赛场上,打造一个综合性的体育场。赛马场中的空地,也能用作蹴鞠比赛的场地——不过反过来就不成了,赛马场远比蹴鞠比赛的赛场需要更多的土地,城中还有几处能充作球场的空地,但赛马场就只能安排在城外——偌大的场地当然不能浪费,赛马场中央的空地可以改作蹴鞠的球场。反正如今的比赛对场地的要求没有后世那般严格,一块平地不论是给马跑还是给人跑都一样没问题。

东京城内寸土寸金,城外也好不到哪里去,价格只是稍稍便宜一点。多买一块地皮,就要多花上万贯的资金。借用赛马场,付些租金就够了。赛马也好,蹴鞠也好,其实都是赚钱的买卖,在韩冈看来,能节省一点就是一点,没必要浪费。在赛马的间隙,用蹴鞠比赛作为垫场,是合则两利的好事。

只是韩冈没想到门户之见如此根深蒂固,赛马总社硬是拒绝蹴鞠比赛借用赛马场,而齐云总社也警告所有人不得与赛马总社有瓜葛。同出一源的两个协会,竟然变成了打擂台的冤家。冯从义在旁边使尽了气力,也只能眼睁睁的看着两边成仇敌,都没办法拧过来。

不过当时在韩冈看来也只是小插曲而已,争也罢,和也罢,两个协会的关系和睦与否,并不在韩冈的考量之中。赛马和蹴鞠这样的体育运动能够组织化和正规化的遍及天下,这就是韩冈的胜利。哪里能想到,赛场的问题最后会造成这么大的伤亡。

赛马场的出口很多,而且由于中央包厢存在的关系,看台被分成两个部分,完全可以将两队的球迷给分割开来。但其他的蹴鞠球场,由于大多数是演兵的校场,无法对场地进行改动,一旦数以千计的观众发生骚动,造成的伤害也就无法阻止。

这是一场让人难以置信的灾难,如果仅仅是斗殴,那还不至于如此大的伤害,骚动发生后,由于人群中的慌乱,踩踏致死致伤的人数占了绝大多数。

对于此事,开封府应该是在第一时间收到消息,不过到现在为止,韩冈还没有得到开封府对此事作出反应的报告。

“不知开封府那边会怎么处置?”何矩小声问着韩冈。

“事情发生在东京城外,归属于祥符县。毕竟是隔了一层。不比城中,是直接由开封府管理。开封府现在最多也只会是派了人去祥符县,责成县中用将整件事整理明白后,再报上去。钱藻也要时间去了解齐云总社的背景。”

何矩听得出韩冈话声中的隐隐怒意,直接叫着现任开封知府的名讳,不敢多话,低头等待韩冈的训示。

韩冈的确有些隐隐生怒,开封府每年从齐云总社手中收取的各项税费超过万贯,而开封府上下得到的好处十倍不止,眼下出了事,不管从哪个角度都不该是坐视的,及早将整件事的处置权收归开封府中,对联赛和受害者都是一件好事,“就算这件事发生在祥符县治下,但也是在开封府中,钱藻接手过来,并没有太多的问题。甚至可以说,祥符县巴不得将这个烫手的栗子交给开封府。但钱藻的样子,现在肯定不想多掺和。”

“端明觉得该怎么处置?”

“杀人者论法,闹事者重罚,这是不用说的。但发生在球场中,又是喜乐丰队和福庆队的比赛,齐云总社也脱不开关系。”韩冈眯起眼睛,“原本社中就有定例,哪一家球队的球迷犯了错,干扰到比赛,那球队就要受罚。这一回,别指望能脱身,做好降级的准备。还有在人情上,要做圆满了,不要忘了,出事的可都是球队的球迷。”

尽管没人知道球迷这个词是从哪里来,且在蹴鞠联赛中,也有不少没来历的新词汇,但传了几年后,大家也就习惯了,说得也顺口。

何矩是京城的大掌事,也是顺丰行在棉行中的代理人,而喜乐丰队则是棉行在外的形象代言人,他对球队的情况最是放在心上。听到韩冈的话,点头称是,“小人明白,该有的抚恤不会少。”

“不是钱的问题,人命是钱买不到的,是人心的问题。带着全队去祭拜出了事的球迷,给受伤者补偿,在下一场比赛开始的时候,请人做个法事……”

韩冈说,何矩点头一一记下。

“不要我说什么,你们才做什么,如何抚慰球迷的人心,你们也要多想一想……”韩冈摇摇头,“不过这一回,下一场比赛要到什么时候,还真说不准。”他叹了一声,靠在椅背上。倒不是为了比赛惋惜,而是今天的事,暴露了蹴鞠联赛安全上的隐患。

“小人会将端明的吩咐去转告总社的。”何矩点头,表示自己听明白了。

他明白,韩冈是想让他转告齐云总社中的一干会首和他们的后台,不要为了钱太心急,否则结果只会更坏。

不解决安全上的隐患,联赛是不能继续向下进行的。天子和朝堂都不可能答应,不论齐云总社的背景有多深。

由于东京城内外球场有限,而球队众多,基本上球场是由多支球队共同使用。一个分赛区,也就是一个厢中的球队,都会集中在一座或是两座球场中比赛。场地的问题不解决,同样的情况日后还有可能会发生。

几年下来,球迷们对球队的感情越来越深,投注在上面的金钱也越来越多,变得分外的接受不了失败的局面,戾气也是越来越重。出现今天的场面,韩冈不会感到惊讶,迟早的事,更是意料中事。

从前几年开始,便有对阵的两队的支持者们在比赛前后、乃至进行中大打出手的情况。为了区分不同球队的球迷,在座位上就要将两边安排得泾渭分明,齐云总社为此制定了不少规则。如今的球迷,连身上的衣服都跟他们所支持的球队一个颜色,也都开始有了标志,一方面让球迷们对球队更加深归属感,另一方面,也更加容易分辨他们的身份。只是因为场地上的问题,还是没有避免惨剧的发生。

