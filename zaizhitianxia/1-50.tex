\section{第24章 自有良策救万千(中)}

两名郎中闻声回头,一见来人,仇老郎中眉头就皱了起来,“齐独眼?……你哪来的那么好心?”

雷简也瞥着眼,就像看到了什么脏东西:“管勾是要雷某去给你送到伤病营的衙前治病?”

仇一闻资历老,人面广,承过他人情的军汉秦凤遍地都是、成百上千,齐隽即便有个官身,他也不会放在眼里。雷简自京中来,也不惧一个进纳官,对经常给伤病营增添死亡数字的齐独眼同样没什么好感。

齐隽笑了笑,貌似没把两人的蔑视放在心上,“这不是合了仇老的心意?你哪次来甘谷,不是伤病营走一遭的?”

“……那也罢,俺就去一趟看看。”对于齐隽的提议,仇一闻想了一想后,还是答应了下来,又对雷简道:“小子,要不要比试比试?”

仇一闻也是好心,他不论到哪个城寨,看到伤兵都会收治下来,不过他是在秦凤路的五州一军到处跑,运气好碰上他的,还是不多。而能跟仇一闻分个胜负,雷简也不会怯场,唤了随侍的药童,背起药囊就走。

伤病营离着也近,也就几步路的功夫,三人就已经站在了营地的门口。

仇一闻惊讶的停住脚,‘才四个月不见,怎么变成了这般干净?’

而在同时停步的雷简的心中,也是一样的想法,只不过将四个月换成了三个月。

不同于来甘谷镀金的雷简,仇一闻可是货真价实的老军医。他走过的桥多过雷简走过的路,吃过的盐多过雷简吃过的米,而治过的人,也比雷简多出数倍。没别的,多活了一倍时间而已。在仇一闻四十多年的行医生涯中,他治疗过的伤兵数以万计,见识过的伤病营也不知多少处,但他还是第一次,见到如此干净清爽的地方。

偌大的伤病营中,遍地的污秽垃圾不见了,露出了被石灰界过的黄土地面;充斥在营房内腐臭味也淡了许多,应该不绝于耳的哀声听不到了,还有欢声笑语传来。

“这是伤病营吗?!”两个医生都是怔住了,不敢相信自己眼睛耳朵,“走错了罢!”

“没走错!”齐隽在两人背后冷笑着,“雷大夫,你在甘谷已有不少时日;仇老,自甘谷立城后你也来过多次。可是看起来,还比不上人家一天的手脚啊……”

………………

“这是怎么回事?!”

随手从身边拉过一个要出门的军汉,雷简怒声质问着。他是甘谷城的医官,虽然他几个月也不会踏足一次伤病营,但营中事务还是属于他的管辖范围。可现在却没人跟他说起,这让雷简火冒三丈。究竟是谁篡夺了他的权力?!

军汉急着要出去,用力挣了一挣,随手指了指房内,“是韩秀才来着。”

“韩秀才?!”

雷简丢下军汉,一步跨入营房。视线只一扫,便一眼盯住了韩冈。营房中有着上百号人,但韩冈的文翰之气让他如鹤立鸡群,决然不会认错。

雷简几步冲到韩冈面前,不顾礼节,厉声问道:“你就是韩秀才?!”

“在下正是韩冈!”韩冈退了半步,拱了拱手,“不知兄台何人?”

只看雷简身后背着药囊的小僮,他的身份便呼之欲出,韩冈却是故意相问。

雷简没回答韩冈的问题,反而更进一步逼问:“你来伤病营是奉了谁的命?!”

“救人何须上命?”韩冈干脆利落的回道:“韩某行事只求心安,与他人何干?”

雷简心中莫名火起,狠声道:“军中自有规条,不是你想作什么就做什么?”

韩冈还未作答,一旁的伤兵和他们的亲友不干了,他们都认识雷简,对这位明明闲得很,却从来不为他们治病的医官没有半点好感。

“雷官人,你不救俺们,也别拦着不让别人救啊!”

“昨夜秀才公为俺们忙了一宿未睡,也不见官人你来看一眼。自俺们躺到这里,就没见过你一面。现在来了,不是来治病,却是跟秀才公过不去。”

“救人你不干,人救你不让,你是不是要逼死俺们才甘心?!”

为十几名赤佬围着周围,雷简脸色发白。军汉中脾气好的不多,被他们围起,不是吃点皮肉之苦就能了事。

“闹什么?!”韩冈突然发火,为雷简解围,“雷官人不是来给你们诊治了吗……”

韩冈一怒,围上来的军汉纷纷退了下去。雷简惊魂不定,气焰顿时息了许多。

齐隽在后面看着情形不对,他没料到才一夜带半日的工夫,韩冈就已经在伤病营中竖立这么高的威望。不得不亲自上阵:“韩冈,虽然你妄称秀才,可医术不是读过几本书就能学来的。庸医杀人,你乱出手医治,想要害死多少甘谷城的军卒?”

仇一闻一直站在后面看热闹,雷简吃些亏,他倒是看着开心。齐隽虽然是在找韩冈麻烦,但他说的也没错,人命岂可儿戏,如果韩冈肚中有货自会反驳,若是只会将营房打扫得干净点,仇一闻乐得让这个高个子的年轻后生受点教训。

老家伙站在后面,左看看,右看看。干干净净的营房,他看得很是喜欢。想着是不是等韩冈吃点苦头后,跟张守约说一声把他捞出来,不经意间却瞥到了一名伤兵身上。

老郎中顿时瞪大了眼,他一步冲上去,抓着那名伤兵的胳膊,惊问道:“这是谁做的?!”

伤病营中认识仇一闻的不少,他一露面,伤兵们几乎要欢呼起来。而他现在一惊一乍,众人便一起向那名伤兵看过去。伤兵其实也没什么特别,全身上下有四处伤,其中最重的是胸前一刀,差点将他开膛破肚,除此之外,还有右大腿被一支长箭洞穿。现在两处伤口都被处理过,包扎得妥妥贴贴。至于他右胳膊骨折,就根本算不上什么,韩冈让人将他的断骨对上,再用夹板固定。一切按照后世的规程,只是找不到石膏,也没法将所有手续全部做完。

仇一闻将上了夹板的胳膊看了又看。在秦凤路,用夹板固定骨折伤处,这算是他的独门技法,除了他的几个徒弟外,少有人知道这一手。不过当仇一闻再看看充作夹板的木头,就摇起了头,‘只学到皮毛,没学到实在!’

韩冈自是对正骨之术一窍不通,朱中只会做点针线活,但周宁不但开过蒙读过书,还学过一点跌打技术。他帮着把骨折的伤员骨头正位,再按照韩冈的意思,用木夹板两面固定绑好。

雷简也把视线投到了伤兵的胳膊上,当下也叫了起来:“怎么用木头?骨折伤该用杉木皮裹上!”视线又投向韩冈,摆明了是要找不痛快。

但为韩冈解围的是仇一闻,他从鼻子里嗤笑出声来,“杉木皮顶个屁用!骨折就得用柳木夹缚住。柳木易生发,插在地上就能活,木性正适合催发愈骨。”

吃脑补脑,吃心补心。古代医学都是有许多想当然的成分在。仇一闻的想法正是依照这个道理,因为柳树能扦插成活,只需将一段柳枝插入泥地中,不用多久,就能长出一棵小树来。看到柳树的这种特性,便认定其有再生催愈的功效【注1】。

韩冈将之用心记下,而雷简则不屑一顾。在他看来仇一闻用的只是江湖小术,靠着运气才治好的人,论起医道,当以医书为本:“骨折而未破皮,当敷以药物,用杉木皮夹缚。”

韩冈皱起眉,一副吃惊的样子:“骨折怎么能用杉木皮来固定?!”

“不用杉木皮用什么?”雷简反问道,“用杉木皮夹缚可是《理伤续断方》【注2】上白纸黑字写着的。”

“尽信书不如无书!”韩冈声音激昂:“杉木皮绵软无力,如何能用?谁的骨头软得跟树皮一样?柳木愈骨才是正理,想骨伤好得快,必须用坚实如骨的柳木板夹着!”他又叹了口气,“只是这次第,哪里去找柳树去,只能随便找些木板来先夹着。”

其实骨折固定用什么板子都可以,但韩冈深悉借力打力,顺水推舟的道理。那名江湖老郎中比起雷医官看起来要靠谱得多,也不似雷简那般仇视自己,当然要顺着老郎中的话说下去。天知道,韩冈还是第一次听说柳木愈骨这回事。

不过光附和别人还不够,还得表现出自己的才能来。而该怎么说韩冈很清楚,老郎中经验丰富,但理论上则差一点,只要往中医学里的五行相和上凑,就足以把他镇住。这也多亏了韩冈前生曾经做过的一份与医药有关的工作:“只是光用柳木夹板还是不够的。上了柳木夹板后,还得再用土敷起、扎紧,以作固定之用。人秉五行之气而生,治疗骨伤,必须要木性、土性相和,才能见功效。”

韩冈向周围一圈聚精会神的听众问道:“谁见过柳枝插在水里就能生根长叶?须得插进土里才是罢?”

众人大点其头,纷纷称是。草木不得土石如何得生?雷简无法反驳,仇一闻捻着花白的胡须沉思不语,韩冈说得浅显,人人能懂。但道理自在其中,让人无从驳起。

注1:柳木愈骨被系统的描述是出现在清代,据传言,当时的某个医生用绞碎的柳木碎片做成骨头的形状,给人安到身体里。当然,这应是无稽之谈。但用柳木做小夹板倒是事实。

注2:《理伤续断方》又作《仙授理伤续断秘方》,为唐时蔺道人所著,是古代重要的骨科专著。

ps:今天第二更,求红票,收藏。

