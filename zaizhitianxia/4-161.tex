\section{第24章 南国万里亦诛除(上)}

米彧弯腰穿过低矮的舱门,从船舱里走出来。

清新的海风吹散了身周来自于舱底的浑浊空气,来自于海天之间耀眼炫目的光线,让习惯了舱中黑暗的米彧,在一瞬间闭上了眼睛。

不过他很快又将眼皮张开,并不大的一对眼睛眯缝着。

自从满载的进入了珠母海【今北部湾】之后,连着数日都是雨天,今天却是难得的晴日。海面上反射着阳光,天和海都是澄蓝澄蓝的,透明一般的宝石光泽,是最上等的吉贝布都染不上的颜色。

几名水手就在甲板上,连同船老大,都好像很闲的样子,不是在做事,而是一齐仰着头,看着桅杆顶部。

米彧随之抬头看过去,就在张起的船帆横桁上,一名瘦小的瞭手两条腿正踏着横桁,一只手抓着杆顶,眺望着船头所对的方向。

过了半刻,那名瞭手低下头来,拖长了音调有着别扭的口音,悠悠的向下喊着:“看~到~啦!是~海~门~镇!”

“黄猴儿,到底看清楚了没有!”船老大不放心的高喊着。

“看~清~楚~啦!就~是~海~门~镇!”然后他就想真的猴子一般,三两下就从五六丈髙的桅杆顶端翻了下来,如同鸿毛一般轻飘飘的落在了甲板上。

“到海门了?”米彧欣喜的问道。

浑身黝黑的船老大回头过来:“米东主,前面就是海门镇。”

“可是确实无错?”米彧不放心的追问着,

黄猴儿一下窜过来,高高的颧骨,陷下去的双颊,凸起的扁嘴,看着的确是个猴儿。不满的说着,“东翁,小的就是靠这对招子吃饭的,哪里可能会看错?!早已经看得分明,旗号就在港口上挂着,哪里还会有错!”

米彧长吁了一口气,说了声对不住,便又双手合十,“阿弥陀佛、阿弥陀佛”的念了好几声。他从广州出发,在船上奔波了十数日,眼下终于到了海门镇。

几步冲到船首,瞪圆了双眼望着依然是海天一线的前路,能否一举翻身就看今次的运气了——要么发财回家还了欠债,要么干脆就死在这里,再不用考虑其他。

福建出身的米彧,过去是在做着棉布转运的买卖。

福建是八分山林、一分水、一分土,养活不了多少人口。古时少人居住,秦汉时,两广都已设立多少郡县,而福建却只有海边的几座城。而如今,从乡里出来.经商做买卖的也是数不胜数。

米彧自家乡出来,就从琼崖的黎人那里贩来棉布,然后万里迢迢的转运到京城中去,藉此养家糊口。江湖上奔波十数载,虽然不能算是大富,可也算得上是小有身家。

不过那已经是五年前的事了,自从熙河路开始种植木棉,米彧的棉布生意就是每况愈下,一日不如一日。

而自熙河路之后,出产棉布的州县也越来越多,就是关中、京畿诸路,都有人开始种植木棉,进而纺纱织布。

陇右棉商做事很正道,没有借着黎人打招牌的意思,打出来的名号就是陇右棉布,靠着优良的品质,几年下来名声也遍传天下。

棉行大行首之一的冯从义,米彧都见过,很直率爽快的一个人,听说娶了太后家的女儿——这其实没什么,比起娶县主、宗女为妻的京城豪商还有不小的距离——但他是韩冈的姨表兄弟,能与当今宰相拉上关系,二三十年后,多半又能跟着新的宰相。

其他的棉贩则是奸猾狡诈的居多,不是伪称是陇右棉布,就是冒充琼崖吉贝。

但不管怎么说,无论是奸商的仿冒品,还是熙河路的竞争者,都是米彧生意日蹙的元凶。物以稀为贵,旧年吉贝布能卖上天价,那是因为数目稀少的缘故。

可如今棉布充斥市场,价格卖得越来越便宜,原本是堪与上等蜀锦相媲美的吉贝布,如今已经快要落到江南苏锦的价格上去了,整整跌了一半还多。

在去年之前,棉布的价格还没有低落得太多的时候,米彧的买卖还能保证不亏本,只是赚得少了。而到了去岁,陇右棉商为了将仿冒者挤出市场去,仗着熙河路风调雨顺、棉花丰收的机会,一口气将棉布的价格降了三成。

米彧好不容易的到了京城之后,一看价格便傻了眼。他本来是准备做上最后一次,然后就收手换门营生。但这最后一次,就让他几乎要倾家荡产。他手上的真品吉贝布要想卖出去,价格也只能随行就市的一降再降,能收回一点就是一点。

将旧时天价的吉贝布三文不值两文的卖出去之后,把运费、人工、商税、库房租赁还有间中产生的其余花销一刨,米彧发现平均一匹布他都要赔上五贯还多。

整整六千余匹吉贝布,米彧折光了自家的一点本钱不说,还将从亲戚朋友那里的借款全都赔了个一干二净。

这样的情况下,米彧当然回不了家乡。失魂落魄的回到了广州后,本来是希望能找个门路东山再起,却是在打探消息时,顺道听说了官军已经灭了交趾,还有安南经略招讨司准备将迁移至富良江口的海门镇的消息。

一旦海门开港,只要能在这里站住脚,就能分到足以发家致富的一块大饼。手上还有价值几千贯的银钱,这是他卖掉之后,虽然远远抵不上欠债,但作为起家的本钱却是够了。米彧直接雇了一条海船,从广州直放海门。

米彧没有来过海门,但海门在过去也算是一个有名的港口,与交趾人有着生意往来的商人为数不少,在酒宴之上,往往能听说道许多关于交趾的风土人情,其中也包括海门港。

不过米彧所听说的海门港,与他眼前所看的完全不一样。如同一个大工地一般,到处都是雨后的泥泞,满眼尽是正在兴建的建筑。

到得早,不如到得巧。米彧到得巧,而且也算早了。加上米彧,眼下在海门港的商人也不过几十人,还要刨掉其中五六名夷商。

这个时候,韩冈正在设法打响海门港的名声,扩大海门港的影响,千金市骨的手段,从来都是少不了的。

虽然眼下他去了升龙府,但韩冈留下来处置庶务、监督工程进度的几个幕僚,却是秉持着他的指令,对于这一干有眼光、敢赌敢拼的商人们好生对待。并派人传信升龙府,同时韩冈他等的人已经到了。

此时的升龙府,则是又聚集了当初围攻此城时汉夷两家的将校和头领们。

他们齐聚在章惇麾下,攻下了升龙府,灭亡了交趾,而眼下,他们又来到章惇的麾下,共同见证代表中国镇压天南的铜柱的落成。

巨大的铜柱矗立在高耸的台基上,周围已经没有更高的建筑。

数千人围在台基周围,静寂无声。在他们的注视下,一头黑色的公牛被牵到了铜柱前。四名力士将公牛牢牢绑定按住,李信赤着上身,在响起的鼓声中,亲手拿着犀利的短刀向着心口的要害直搠了进去。

浓浓的血浆从创口中喷涌而出,继而流淌到了摆在地面上、满载着上百斤酒液的铜缸中。

章惇穿着最为正式的朝服,走上了台基。拿着一支三足的青铜酒爵,在缸中舀起一杯酒,面向北方,跪下来,举在头上,“此一杯,献与天子。”

数千人一起跪下,齐声喝道:“恭祝皇宋天子千万岁寿。”

领着所有人,三跪九叩,章惇起身再舀出第二杯,洒在地上,“此一杯,以祭英魂。”

“这第三杯,以此铜柱为誓。”章惇再一次高高举起酒爵,返身面向所有人,“若有不顺朝廷,意图谋乱者,各部举兵共击之。”

每一家部族的洞主们都随着章惇一起举起了手中青铜爵,他们手上的酒爵,都是与铜柱一同铸造而成,混合了牛血的烈酒在爵中摇晃,齐声应承:“我等以铜柱为誓,若有不顺朝廷,意图谋乱者,各部举兵共击之!”

歃血为盟之后,一场盛大的酒宴就在台基下举行。

一坛坛美酒在席上传递,用来歃血为盟的壮牛,在烤架上变成了香喷喷的烤肉。数百人在席上喝酒吃肉,还有人跳起来唱着哪一位幕僚写得赞诗:

“天之所覆皆王土,南海之滨亦王臣。昔年伏波定交趾,今日王师复守巡。赵氏开国号南越,立柱标铜后安民……”

韩冈听了想打哈欠,他虽然不会作诗,但眼光还是有的。这首长诗真的不怎么样,还不如李常杰那首绝句有气魄。

“怎么选的韵脚,什么不好押,偏偏押了上平十一真……胶柱鼓瑟。”

“普天之下,莫非王土,率土之滨,莫非王臣。不就为了和这段才押的真字韵。”章惇在旁大笑着,虽然喝了不少酒,但还是没有醉,只有回头望着身后直指云空的铜柱的时候,他的脸上才带起了一抹仿佛醉酒的殷红,“虽不能封狼居胥,但也是标铜立柱。有此一功,不枉此生!”

