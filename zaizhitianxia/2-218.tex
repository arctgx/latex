\section{第33章 旌旗西指聚虎贲(五)}

【对不住各位,回来晚了】

清晨,韩冈被传遍县中的晨钟之声从睡梦中唤起。

窗外的鸦雀声声。朝东的窗口,那新糊的窗纸上,也透着明亮的红光。

睁开沉重的眼皮,韩冈脑袋里还隐隐作痛,酒醉的后遗症,让他只想在睡上一阵。对于一向精力过人的韩冈来说,这样的情况实在很少见。

昨天的接风宴上,他喝得多了一点,没想到就这么醉了。韩冈努力回忆着昨天的宴会,希望自己在席上没有失仪。

高遵裕和王中正一行,是在前天到的。秦凤、泾原两路的援军,也是或前或后,陆续抵达通远。泾原军最终还是以姚兕为首,秦凤军则是由转了钤辖的刘昌祚率领。让韩冈惊喜的是,李信也带了一个骑兵指挥过来。虽无选锋之名,但这也是秦凤兵马副总管张守约身边最为精锐的一支队伍。

到了昨天,随着最后一支援军抵达通远,秦凤路转运使蔡延庆竟然一起也到了。这就让人很惊讶了。虽说按照朝廷颁下的条贯,各路转运使一年之中,必须有半年时间在辖下各军州巡视,但蔡延庆赶在战前跑来通远还是出乎意料之外。王韶和高遵裕也都没想到漕司的大头目会来,一时之间只感觉被打了个猝不及防。

有了蔡延庆,准备好的接风宴上,王韶、高遵裕都成了陪客。不过蔡延庆并不是崖岸自高的那等人,没有自持身份,韩冈依稀记得,在宴席上蔡延庆还与自己对饮了几杯,又说与韩冈是乡里——韩冈的祖籍是密州胶西,而蔡延庆则是莱州人,的确离得很近。一路转运使,能不在意地位尊卑与人交谈,反倒比他身边随兴而来的转运判官蔡曚要亲切不少。

今次大战,韩冈并不需要上阵,而是负责后勤转运。可真要计较起来,后勤方面的工作比起上阵要麻烦许多。但在眼下的通远军和缘边安抚司的官员序列中,也只有韩冈的地位和身份,能安稳的坐上这个位子。还有他的能力,更是让所有人放心。王韶对韩冈的提名,没有二话的被所有人都认同了。蔡延庆重视韩冈,一是因为韩冈名声响亮,可也有因为他的职位的缘故。

只是在宴席上,蔡延庆的官威还是太重,使得酒宴的气氛稳重甚至僵硬。直到蔡延庆、王韶、高遵裕他们识趣的提前退席,并吩咐苗授来主持宴会,场中的气氛才热闹了起来。

接下来的记忆,已变得模糊了,韩冈已经回想不起最后到底发生了什么事。以前自己酒量还不差,但他一向喝酒不多,又是很长一段时间不锻炼,想不到这酒量渐渐就退步了。

看着窗外的红光渐渐淡去,心知时候不早,韩冈想要坐起身,却一时没挣扎起来。左右看看,周南和严素心两张如花俏脸正一左一右的与自家同床并枕。两具香软的娇躯紧紧贴着他的身子,耳畔的呼吸声浅浅细细,犹然是在海棠春睡之中。

一大清早,正是阳气充沛的时候。周南丰腴挺拔的双峰押着手臂,而严素心修长笔直的双腿又缠着腿上。韩冈头疼中,便多了分口干舌燥。只是看到她们的脸上都带着几道泪痕,却心知不好。

一点点的掀开被褥,暴露在空气中的两女晶莹白皙的肌肤上,竟然有着许多处手指揉捏和唇齿啮咬过的痕迹。尤其是双峰和颈项,一抹抹鲜艳夺目的瘀痕,让韩冈都觉得自己昨夜借酒逞凶,做的事有些过头了。

韩冈怜惜着用手抚过两女的伤处,只见她们在睡梦中都蹙起了双眉,不堪痛楚的模样,让韩冈心中不忍,不想打扰她们。从肢体交缠中轻轻抽出身来,轻手轻脚的下了床。

“三哥哥,起来了?”听到屋内终于有了动静,韩云娘敲了敲门,在外面问着。

“起来了。”韩冈轻声答话。回头看了看,身后床上,两女还在沉沉睡着。

门被推开,韩云娘端着脸盆,进了房来。一转眼就看到床上的两女风情,小脸顿时红了,她不敢多看,忙上来帮韩冈梳洗更衣。

韩冈昨夜醉醺醺回来,便把周南和严素心强拉着进屋。韩云娘都看到了,只是年纪渐长,已经少了许多旧时的孩子气的嫉妒,而是变得更加稳重。加上韩冈一直对她都很亲昵,从来没有让她感觉因为有了周南和严素心而被冷落,心思也安定了很多。现在韩云娘跟年纪差不多的墨文,还有招儿,三个小丫头的关系都很好。

韩冈透过敞开的大门,看看天色,已经是没有了锻炼身体的时间了。在韩云娘的服侍下,匆匆梳洗过后,赶紧换了身干净的衣服。今天作为前锋的苗授和王舜臣,今天晚些时候就会率领两千人马当先出发——以王韶和高遵裕的想法,攻打武胜军还是以自家人为主——而接下来,明天后天,中军、后军也将分批开拔。

留给韩冈的,是为九千四百余名将士,两千六百余匹战马,运送粮秣军资的任务。同时,还有超过三千人、两千牲口的辎重队伍要他管理。等到夺下武胜军之后,增筑城池,修建寨堡,还得组织起大批的民伕人力。钱粮等物,已经事先筹划好,并不缺乏。但中间不能出一点乱子,没有浪费和损失的余地。

在经历了熙宁三年的连番大战后,平静了近一年的河湟之地,从今天开始,重新又沸腾起来。韩冈也必须尽快赶去衙门。

帮着扎好了腰带,韩云娘又掂起脚把韩冈的襟口整理好。神色间是聚精会神的专注。这一年来,小丫头的个子没有再长,身形也是有些纤弱,但相貌越发的明艳起来。略深的眼窝,和挺直的鼻梁,带着一丝异国风情的容色,并不比周南和严素心稍逊。

既然人在家中,对父母的晨昏定省那是少不了的。换好衣服,韩冈便带着云娘去正院拜见父母,而唤了墨文在房中照看周南和严素心。

他到的时候,韩千六和韩阿李都也已经起来了。韩千六正穿着青色的官服,腰背挺直,端坐着,很有几分官威,已经没有一开始怎么看都不搭调的感觉了。而韩阿李虽然没有太多装饰,但穿戴也是有了官宦人家的气派。

就在七月,通远屯田喜获丰收。新辟的千多顷良田,总计有了接近二十万石的收获,远远超出了一开始的预计。王韶、高遵裕都得了嘉奖,而指点农事的韩千六,也便得了天子恩旨,改了个韩谦益的大号,正儿八经的当上了官人。

现如今,身为军事判官的韩冈在通远军官衙中地位排在第四,仅次于王韶、高遵裕和苗授。可一旦排起座次,韩千六却要抢在韩冈的前面——没有老子坐在儿子下首的道理,这不符合孝道。故而上个月中秋开宴时,王韶、高遵裕并坐在上首,右边第一位坐着苗授,而左边首席便不是韩冈,而是捋着胡须一直在笑的韩千六。

韩冈向父母请过安。看着儿子只带了云娘过来,韩阿李便问着:“怎么不见南娘和素心?”

世间的规矩,做新妇的早上若不起来服侍舅姑,那就是不知礼法,要受罚的。放到妾侍身上,情况也是一般。若是在平常,周南、严素心都是循规蹈矩,服侍长辈都是小心谨慎,唯恐哪边疏失。

只是今天情况特殊,韩冈陪着笑脸:“有孩儿和云娘服侍爹娘还不够吗?”

韩冈为她们遮掩,韩阿李也不多问。她巴不得儿子在周严二女身上多用点心,早点给她带个孙子来。

家里也多了好些仆从,厨房中,也不需要严素心亲历亲为。这些人中,有些是原来的乡邻,生活不下去过来投奔。有的则是韩冈救助过的士兵,或是年长或是体弱,不适合再留在军中,故而投到韩冈门下。还有几个,是领了与韩冈交好的官人家的荐书,被荐到门下——就如章惇荐来的钱明亮夫妇。

现在韩家大约有二十多个下人,都是忠勤听话的。并不是没有有作奸犯、心怀诡谲之辈,来投奔韩家。这本是暴发之家都难以免除的情况。韩冈在官衙中,经常能听到黠仆欺主,坏了一家名节的故事。外界看到韩家扩张门楣,有许多人都想着看看笑话。但有韩阿李主持家务,赶走的赶走,治罪的治罪,一点也不心慈手软,很快就把急速扩张的门户治理得井井有条。这事传到外面,便有人笑说,韩冈治事的本事多半是跟他娘学得。

匆匆吃过饭,韩冈和韩千六一起赶去衙门。

今天的天色看起来并不好,大清早还有着太阳,才半个时辰过去,天就阴了下来。铅色的天空,让人有些不安。就在快到衙门的时候,一点冰凉就轻柔的落在了脸上。韩冈抬起头,细小的雪珠从灰色的阴云中洋洋洒落,这是今年冬天的第一场雪。

“幸好不是下雨。”韩冈低声自言自语。

