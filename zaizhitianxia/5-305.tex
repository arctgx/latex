\section{第31章 停云静听曲中意(12)}

一步一个台阶,赵隆步履沉稳的走上兴庆府的南门城头。

城中各处尚冒着缕缕青烟,而种谔的大纛就已在城头上猎猎飞扬。

在阶梯上越走越高,兴庆府的全貌也看得越来越清楚。周围近二十里的城市中已经看不到多少还完好无损的建筑了。被宽阔的街道所分割出来的几十个厢坊中,到处是一道道或白或黑的烟柱腾空而起。听不到什么人声,也看不到几多人影,只有凄厉的风声时不时的将烟柱给吹散。

这座城已经死了。

赵隆心中不由得闪过这一句话。

他没有赶上攻下兴庆府的战斗,更没有赶上之前种谔联合党项人大败辽军的会战。但这并不代表他不知道种谔是怎么攻下的兴庆府。

兴庆府是西部难得的大城,当年嵬名元昊定都于此时,为了大白髙国的脸面就往大里扩建,足以容纳三十万军民的城池里,只有不到二十万人口生活,有许多地方甚至为菜地、鱼塘填充,几座兵营的校场,占地能跟紫宸殿前的广场差不多大小。大公鼎率族人迁移来此,整个渤海部族也只占了城市的一小半。

到了今天,包括所有逃入兴庆府的各族族人,总计也不过七八万的样子,只占了城市的一半还不到。城内的党项人虽是死的死逃的逃,可还有不少藏身其间。他们与辽人的血海深仇自不必说,当种谔、仁多零丁和叶孛麻开始与辽人决战,他们就设法在城中放起火来。先是无人的街坊,继而是楼宇重重的寺院,然后是住了人的深宅大院,最后就连囤放粮草的仓屯也一起都烧了起来。

能上阵的士兵当时大半给带走了,城中留下的兵力仅仅能守住六座城门和王宫。只靠老弱妇孺,如何阻挡得了矢志复仇、又深悉地理的党项人?最后就连兴庆府的王宫也给烧成了断壁残垣。

在放火之前,整座王城并没有被毁损。并不是辽人刻意保留,只是没那个时间,大公鼎领部众进入之前,占据此处的辽军仅仅是将犯忌的东西给处理掉了。在大公鼎到来后,更是直接将整座王城封锁起来,自己则住进了前西夏国相梁乙埋家的宅子。不过残存的党项人的一把火,使得王城内外全都化为了灰烬,大白髙国的最后一点象征也不复存在了。

当种谔领兵抵达兴庆府城下的时候,城中已经是烈焰熊熊,城中辽人早就打开了北门四散而逃。进城的时候,甚至一点力气也没有花费。

已经站在城墙顶上的赵隆,除了脚下的城墙和街道桥梁,看不到任何完整的建筑。只是对一名出身关西的宋人来说,又怎么会为这一座浸透了宋人耻辱的城市而感到惋惜?除了兴奋,赵隆遗憾的仅仅是自己没有能参与到毁灭这座城市的战斗中来。

接到来自于帅府行辕的军令时,青铜峡中的党项人早就走了好几天,赵隆没有半分犹豫就立刻整军北上,只是还是没有能来得及赶上这一场会战。

站在敌楼的门口,赵隆清了清喉咙,然后恭恭敬敬的朗声:“赵隆拜见太尉。”

……………………

吕惠卿已经将自己的帅府行辕放在了溥乐城。

帐下大将曲珍还在做着北进兴灵的准备,从永兴军和环庆两路调集而来的兵马才到了不到五分之一,就已经将这一座小小的军城给填满——至于鄜延路中的精兵,则是去支援了空虚的银夏路,以免为辽军所乘。

现如今在横山以北,即便连一个月拿着六百文口俸的小兵都知道,新任的枢密使兼宣抚使就是种五太尉的大后台,种总管敢于北上攻辽是得到了吕相公的准许。

早些日子吕相公就派了人去通知泾原路的赵隆,命他领兵北上,以便能支援种五太尉。这正好是在得知了青铜峡中党项人北出峡口的消息当天下达的,一点也没有耽搁时间——从灵州川边出发,只要向西横越百里山岭小道,便能抵达青铜峡谷地南端的鸣沙城。那条小路大军难以通行,不过几名信使要通过就很简单了。

到了这两天,更是调集了宣抚司一时间能动用的所有兵力,摆出了要全取兴灵的架势。

这是对辽人破盟的报复,竟然敢于撕破刚刚签订的协议,来攻打皇宋的边城,堂堂中国难道还能忍受这样的挑衅?

刚刚灭掉了生死大敌,却因为辽人的狡诈而功亏一篑的西军将士,对吕惠卿的果敢敬佩有加,话里话外都在赞着吕相公。

但种朴和种师中却是忧心忡忡,他们可比下面的士兵多了解许多,自然也不会被吕惠卿做出来的姿态瞒过去。

“我已经给东京城送信过去了,就不知道韩学士能不能体谅。”种朴对身边的堂弟说道。被叫去帅府行辕的路上,脸色和脚步一样沉重。

他的父亲犯下的是几乎所有士大夫都不能忍受的大忌,即便亲近如韩冈,会出手保他种家的可能性也并不高。

种师中的心情就放松了一点:“五叔不也说过吗?如今朝堂上虽然是变法一派,旧党虎视眈眈,这一回只要胜了,朝堂上的诸公想要严办也得投鼠忌器。等到与辽人开战,还不是得让五叔出山?”

种朴默然摇头。这就是要赌一把,赌士大夫对武人的忌惮和新旧两党之间的嫌隙哪一个更深。

可他的父亲一向好赌,赌运却一向不佳。输了一次又一次,但每到看见机会的时候,都会忍不住铤而走险,豪赌上一把。

只是有这样父亲,做儿子的又能怎么办?

“十七哥,不用担心啦!”种师中为堂兄打着气,“出兵兴灵的事,吕相公不都已经认下了,有罪名也是他先挡着。”

种朴扯了扯嘴角,笑了一下。种师中说得的确是事实。

调了赵隆北上,等于是派去支持种谔的。甚至种谔的行动,有这一件事在,都等于是在帮着背书。

吕惠卿可以说是要将整件事揽过去,不论功劳还是责任。

种朴对此自然是乐见其成,他本来已然做好了被父亲连累,从此不能再领军的准备,甚至连他的叔伯兄弟都可能被调去闲职,直到多年后才能被起复。

可现在吕惠卿为他父亲的行动背书,朝堂上要怪罪下来,板子就会打到吕惠卿的身上,至于独断独行、妄起兵戈的罪名,虽说瞒不过朝堂上的宰辅,在明面上也能敷衍得过去了。

但终究也只是明面上。暗地里的刀枪剑戟,未来不知会有多少。

到了行辕中,两兄弟立刻就被传了进去。

“方才入城的露布飞捷可看到了?!”吕惠卿开怀笑着,“王师已经攻下兴庆府!”

种朴和种师中之前就听说了。这个捷报在两天前,决战中击败的辽人消息传来后,更是就已经可以预计到了。

他们齐齐躬身一礼:“恭喜宣抚立此殊勋!”

“非吾之功,这可是令尊的功劳。”吕惠卿摇头对种朴道,嘴角间的笑意写满了讽刺。

种朴道:“全托宣抚运筹帷幄,方有家严之功。”

吕惠卿笑容中的讥讽更重了。又说了几句,便示意两人退下,而脸色也随即阴沉了下来。

对于吕惠卿这样的人,无能二字比任何罪名都让他不能接受。事情到了这一步,与其认下一个御下不严的责任,成为士大夫中的笑柄,还不如行险,将所有的责任都担下来。理所当然的,击败辽人的功劳或者说罪名也会落到他头上,

来到溥乐城之后,吕惠卿毫不掩饰对种朴、种师中的看重,就是基本上被撇到一边的李清也大大的夸奖了一番。

纵然心中恨不得将种放、种世衡在终南山的坟都给刨掉,可吕惠卿仍然带着宽和的笑意,进城第一天就将小小的溥乐城巡视了一遍,第二天还去了不远处的耀德城走一趟。

现在这一番布置总算有了一点成果,不论承不承认,大宋开国以来,第一例从辽人手中夺回土地的功劳,就是他吕惠卿立下的。

只是心中的窝囊气,却怎么也挥之不去。

‘大不了降罪去南方,以我的身份翻身也就是几年间的事。日后一旦举兵伐辽,又有谁能阻我吕惠卿回朝……功劳就是功劳!’

吕惠卿安慰着自己,心中却仍是咬牙切齿。

“枢密。”一名幕僚突然递上一份文稿。

只看了一眼吕惠卿就看出来了,这是前几日整理出来的记录守城经过的奏报。吕惠卿为了表面上做的圆满,这些天来都是露着一张笑脸做事,心中够窝火了,没心思听种家的子侄自吹自擂。前几日就收到了这份记录,可他只是看了一下伤亡数字和斩获的功劳,剩下的文字连瞥都没有瞥一眼。

可是被翻开来的这一页上,由幕僚画出来的几个字顿时吸引了吕惠卿的目光:“火箭?飞火枪?”

作为军器监的创立者,吕惠卿对任何一种新式兵器都有着足够的兴趣和好奇心。

“东西在哪里?”他问着幕僚。
