\section{第25章 晚来萧萧雨兼风(上)}

跟在王珪、薛向身后走进天子寝殿,韩冈登时就感觉到殿内的气氛跟方才离开时已经大不一样了。虽说还没有亲眼见证,不过韩冈已经有八九成把握可以确定赵顼的神智当真是清醒了。

对于赵顼依然保留着清醒的意识,韩冈其实挺惊讶。在他的预计中,赵顼无法恢复。之前的一番做作,也只是拖延一点应变的时间。要真有把握,当时就直接拿韵书来表演了。

但韩冈对向皇后找到与清醒的赵顼交流的办法,倒是一点不觉得意外。这也不是要多少想象力的事。自己能想到,其他人也一样能,不过迟早的问题。如果世间有字典的话,即便是局限于一两千常用字的简明字典,恐怕当时就有人想到了。

“陛下,臣王珪来了。”

王珪还是一副忠心耿耿的老臣模样,一进寝殿便趋步上前,小碎步的跑到御榻边。带着激动和欣慰的打量着赵顼。

赵顼眨着眼睛,也不知是不是错觉的关系,眼瞳中看上去的确是比之前有了几分神采。

韩冈、薛向跟在王珪身后,看到赵顼的反应,心中顿时也轻松了几分。

向皇后回过头来,对韩冈道:“多亏了韩学士。”

“不敢,是陛下福德。”韩冈当然不能居功自傲,随口谦虚了一句,问道:“陛下可有何训示?”

王珪也将腰又弯了一弯,问赵顼道,“说得也是,不知陛下招臣等来,有何训示?”

既然能有办法与赵顼沟通,他们三人肯定要作为朝臣们的代表进行确认一下,免得有人假传圣旨。

向皇后心领神会将手上的韵书递给王珪。在薛向和韩冈的见证下,让王珪在韵书上翻出了‘等’和‘太’。

“等……太……?”王珪问道:“是太后要来吗?”

“太后快要到了。”向皇后点头道,“方才官家也要请太后过来。等太后到了,官家当有话要说。”

三名朝臣脸色都是微微一沉,招了宰执,请了太后,自然不会是小事,有七八成可能是跟帝统传承有关。

“陛下龙体初愈,应该多休息才是。”韩冈皱着眉头说道,“其实也不用急在今夜。”

王珪强忍着回头瞪韩冈的想法。这位太子蒙师明着说皇帝,实则是在说太后,竟然是旗帜鲜明的站在皇子的一边,丝毫也不犹豫。能创立蹴鞠和赛马联赛的人就是不一样,关键时候还真是敢下赌注。但话说回来,以韩冈和雍王之间的恩怨,用尽一切手段打掉赵颢登基的可能,尤其是针对其唯一希望的高太后,也不是多让人意外的事。

薛向却暗暗纳闷。韩冈的表态,怎么感觉就像是将太后和延安郡王给对立起来一般,难道他能确定,太后一定会支持雍王上位不成?

赵顼又眨起眼睛,王珪见状连忙翻起韵书。

“无……妨……”

赵顼既然表态,那就没有什么好说的了,韩冈也只能保持沉默。旁敲侧击虽然没用,他却也不能将话挑明了说。

‘看上去是真的要准备退位内禅了。’韩冈心道,只是在赵顼僵硬的脸上看不出他心中的想法。

难道说赵顼是准备趁自己还有几分清醒,想赶紧将皇位让给儿子,自己改做太上皇不成?早点将天下转给赵佣,免得赵颢总是惦记着这块肥肉。若赵顼真的准备这么做,这份决断力,韩冈倒是不得不佩服了。

能在第一时间摆脱成为废人的失落感,这份定力已经是超乎常人想象了。韩冈都没把握自己处在赵顼现在的位置上,能不能保持如此稳定的心境。

说起来韩冈曾听说当年英宗临终病危,遗诏都向重臣们颁布了,赵顼也做好了即位的准备。但待到英宗回光返照,看起来似乎有恢复的迹象的时候,韩琦很强硬的说即便英宗皇帝康复了,也只能为太上皇。后来赵顼表现得对韩琦很有成见,将他请出京城的时候没有一点犹豫,传言便是因为此事。

或许就是因为亲身经历的这件事,天子才会有现在这般冷静的表现吧?至少控制权还在他的手上,也免得不省人事后,被奸人伪传遗诏的事情发生。

只是这么做是不是急了一点?而且还是等太后来了才说。

韩冈难以理解赵顼的想法。难道这位皇帝不知道他的亲生母亲是他儿子最大的威胁?

自然,这个威胁并不是说高太后会拿她的孙子怎么样,而是说在有可能造成赵佣无法登基,或无法活到成年的人中,高太后能做到的几率是最大的。在帝位的争夺中,并不是做了什么才有罪,而是有那份能力便是罪名了。太祖太宗三兄弟中的那位秦悼王,究竟是怎么从杜太后的嫡子变成宣祖小妾生的庶子,其中种种黑幕,让后人看了都想笑。

在这样的情况下,纵然不可能针对太后下手,赵顼也不应该全盘信任才是。好歹先跟王珪这样忠心的宰相,或是自己这等绝不可能站在赵颢一边的臣子商量一下才是。至少韩冈现在看不透赵顼,还说在自己再次被请到寝殿之前,有什么事发生了?

韩冈往御榻处看去,只见王珪捧着韵书等着天子吩咐,但赵顼闭着眼睛,貌似没有半点与其交流的打算。在太后到来之前,不打算与任何人说话。

该不会是赵官家的脑袋在中风的过程中弄坏了吧?韩冈猜想着。这或许是必然的答案。

中风,又叫卒中,不过韩冈知道,加个‘脑’字更确切一点——脑卒中。缺血也好、失血也好,导致瘫痪、面瘫、失语这些症状的直接原因都是大脑损伤。赵顼的智商在这一次的中风中出了毛病也当在情理之中。

但现在看起来也不像是变成白痴的样子,能利用韵书说话,蠢人可做不到。最怕的还是性格出问题,

“官家、圣人。太后到了。”站在门口的小黄门,在外高声通报。

片刻之后,随着派去保慈宫的宋用臣,高太后又驾临寝宫。高太后来得很急,之前应该已经就寝。脚步匆匆的扶着陈衍的手跨进门时,脸上并没有化妆,能看到有不少皱纹,头发也只是很随便的挽着。韩冈看了一眼后就低下了头去,王珪和薛向也是一样,这般模样的太后不能随便乱看的。

但随同而来的不仅仅是高太后和她的一般近侍,还有雍王赵颢。当二大王的身形出现在门前,殿内的气氛顿时为之一冷。

王珪、薛向面面相觑,皆是心头凛然。雍王竟然没有出宫!看样子,是住在了保慈宫中。难道太后已经打定了主意不成?

可即便赵颢在保慈宫住了下来,现在也不该随着太后一起过来。宋用臣可是带着口谕出去的。天子既然没有邀请,雍王就没资格走进福宁殿。天子寝宫又不是菜市口,想来就来想走就走。又不是之前昏迷的情况,天子可是已经清醒了。

当然,相比起赵颢今晚住在宫城中的事,其实也算不上什么了。三名朝臣偷眼去看赵顼和向皇后,观察着他们的反应,皇宫的主人终究还是赵顼,雍王留宿的事,鬼才相信皇帝皇后心里会不恼火。

高太后并不管那么多,径直在床榻边坐了下来。听说找到了与儿子交流的办法,她亦是欣喜不已。毕竟是母子天性,再怎么偏爱次子,终究还是关心赵顼这个长子的。

韩冈在一旁看着高太后和赵顼通过韵书来交流,问了几句之后,也确认赵顼恢复了神智。

应该差不多了吧。不止韩冈一个人这么想着,赵顼似乎也是这么想的。

当高太后用韵书翻出了上平八齐中的珪字,高太后便转手将韵书交给了王珪。

王珪接过韵书上前半步,“陛下有何吩咐?”

所有人也都立刻关注起赵顼眼皮的变化。

“下平。”

“二萧。”

王珪的声音圆融醇和,在过去还担任翰林学士的时候,是宫宴白席的不二人选,也是在郊祀或是明堂等大典上担任赞礼的第一人。

——“招。”

是要将王安石招入宫来吗?还是说奉旨书诏的翰林。韩冈想着。早点招两个翰林进来,正好就可以宣麻拜相了。但当着高太后的面,却做着近乎于托孤王安石的事,似乎有些不太对劲。

盯着赵顼眼皮的一众视线也更加凝聚,屏气凝神。内侍和宫女更是大气也不敢出,只有王珪一人的声音在回响。

“是上平?”王珪问着。

赵顼的眼皮眨了两下。

不是王安石,王是下平。翰林的翰倒是上平——上平十四寒。不过王安石的安好像也是上平十四寒。只是韩冈不写诗,对韵目的了解得不是那么深。

但赵顼并没有等到上平十四寒,而是到了第四韵部,便眨了两下眼皮。

上平四支。

“司。”

随着王珪的声音在韵书中一个字一个字的数过,最后停在‘司’上,韩冈的心一点点的沉了下去,事情不对了。司开头的名词并不多见,人名也好,官职也好,也就那么几个。

不仅是韩冈,所有人都知道,朝堂上能对得上号的,也最合适的,只有一人而已。

王珪的手颤了几下,声音也没有之前那么稳定,但韵书还在翻着,赵顼的眼皮也在继续眨着。

上声。

韵部二十一。

马。

韩冈呼吸一滞。不会有别的可能了,赵顼找的总不可能是别称大司马的兵部尚书。

这到底是怎么回事?!

韩冈无可奈何的闭上了眼睛。眼前是一片黑暗,韵书翻动的声音却依然不停,王珪的嗓音则沙哑艰难了许多。

下平。

七阳。

传入耳中的王珪那本是圆融醇和,却变得沙哑的语声,最后发出了一记变调的破音:

“光!”

招司马光。

不是王安石,而是司马光。

旧党赤帜——司马光。

