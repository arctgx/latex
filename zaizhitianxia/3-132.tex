\section{第35章 甘霖润万事(上)}

甘霖终降,开封全城都振动起来。

上至天子,下至小民,无不为此而欣喜欲狂。

淅淅沥沥的雨水浇灌着干涸的大地,无数人冲进雨中欢呼雀跃。

时隔近八个月,开封城终于开始有了雨水,这怎么能叫渴盼已久的百万军民按耐得住心中的喜悦。

高阳正店二楼雅座中,刚刚卸下了提点开封府界诸县镇公事一职的屯田员外郎吴审礼,望着骤雨如瀑,还有在雨水中手舞足蹈的民众,轻声叹道:“明日可就是同天节了。今日幸降甘霖,贺天寿,慰黎民,王相公也随之得脱大难啊。”

“寻常女婿都是靠着岳父帮忙,那韩冈倒好,却是让他的岳父靠着他。”坐在对面的大理寺丞张景温笑道:“王相公今次逃过一劫,这相位至少还能再坐个一年半载。”

“谁说不是呢?”吴审礼悠悠然的微笑着。

明日就是四月初十的同天节,也就是天子赵顼的诞辰之日,赶在生日前一天下雨,等于是老天爷帮着赵顼一个大忙,证明他是确确实实的真命天子。而民怨因这一场雨暂时散去,赵顼也就不需要赶在现下让王安石出来为大旱负责。

张景温举杯相邀:“此一杯还要恭喜仲由兄得受监司,用事于河北,当可一展长才。”

“不过一个河北西路转运副使而已,吃苦受累的活。”除了权发遣河北西路转运副使,算是升了一级。吴审礼当然高兴,只是故作矜持:“只是在开封任亲民官,整日价提心吊胆。生怕不小心冲撞了那家贵戚,就算下面的小吏都是手眼通天,做起事来也是束手束脚。”

“但仲由兄还不是将开封府界中事,安排得无可挑剔,连天子也是赞许有加。迁调河北,也是因为仲由兄的名讳早在天子心中留着了。”

“太夸赞了,愚兄可不敢当。”

吴审礼抱怨归抱怨,但他也算是难得的能吏。不论是在京府诸县推广保甲法,还是撤除只会浪费朝廷公帑、豢养闲人的京畿马监,都是卓有成效。

话说回来,能在开封府任职的官员,施政能力绝大多数都不会差。不论是知府、还是提点,又或是下面的判官、推官和知县们,没有点水平,都不会被安排到京畿之地来任官。京畿一带,遍地勋贵豪门,皇亲国戚。要在其中辗转腾挪,同时将政事处理妥当,都少不得要有足够的手腕。

“河北如今大灾,盗贼宵小为数不少,真要清剿起来,并非易事。”吴审礼叹道。

张景温笑道:“总比在开封府界中捕人要容易。”

“说得极是,京畿的这一摊子事就丢给韩玉昆操心好了,能者多劳嘛!我等才德浅薄,还是挑着清闲的差事做!”吴审礼也随之哈哈大笑,举起就酒杯,与好友一齐痛饮起来。

……………………

雨点不断敲打着园中小亭顶上的琉璃瓦,久违的哗哗雨声,听在亭中的韩冈和王韶耳中,就是一曲动听的歌谣。

从亭中向外望去,如同瀑布般的一道水帘挂于檐前,模糊了视线。看着雨势,仿佛要将七八个月来,积存起来的雨雪在一天之内全都还回来。

满园的竹林,原本在吹了一个春天的风沙中沾满了灰黄色的尘土,此时在雨水冲刷下,终于变得青翠欲滴起来。

从林中收回视线,王韶举起酒杯:“玉昆,这场雨下得可喜可贺啊!”

“何来之喜?”韩冈举杯相和,却叹了口气,“雨下迟了一个月,河北的田地已经来不及补种,流民还是少不了啊……”

这是韩冈此次进京后第二次拜访王韶,前一次只是匆匆一会,没有来得及多说。不过现在韩冈接手府界提点一职的大体事了,明日拜贺天子生辰之后,就要离京返回治所,今天就趁着余暇再来拜访。

“不是说这个。”王韶摇摇头,“久旱逢甘霖,这场旱灾总算是过去了。怎么能说‘何来之喜’?”

韩冈一笑:“是韩冈失言了,能见到雨水,的确是可喜可贺。”

两人对饮而尽。

放下酒杯,王韶又道:“上书的那名监门官,怕是难逃重责。擅发和妄言二罪不论,单是诳言欺君就能让他编管远恶军州。”说着,王韶微微眯起了眼睛,“十日不雨,乞斩于宣德门外……好大的赌注!”

韩冈在延和殿上的奏对,此时已经在高层中传开,王韶当然也听到了一些。郑侠以性命相赌的言辞被韩冈轻巧的破去,乍听到时,基本上人人都认为是韩冈纵横之术了得,王韶也是一样这般想着。可现在雨水一下,情形一下反了过来。就连王韶也认为郑侠是事先算到会下雨,才敢如此说来。而天子则更是早就认定郑侠欺隐,现在甘霖一至,他就再无翻身的机会。

“如此一来,令岳也算是渡过了这一关了。”王韶将酒杯放过来,让韩冈为他斟酒。

这几个月来,朝堂上虽然波涛汹涌,两党相争激烈。但王韶不趟浑水,他安然的做着他的枢密副使,只盯着军事方面的事。说起王安石来,口气如同一个看客。

韩冈知道王韶一直以来不怎么支持新法,对他现在的态度并不以为怪,笑道:“家岳身为宰相,要操心的事太多。原本还以为能清闲起来,现在看来还是要继续烦心下去。”

王韶摇头笑道:“旱灾缓解;与北虏相度边界一事,又派了韩缜去了;市易务眼看着曾布要败;流民又有玉昆你来照管,令岳现在哪还有要烦心的事?”

“还有蝗灾。”韩冈补充道。

“今年地里又没有吃的,蝗虫再多也不用担心。”

韩冈摇着头:“其他州县不知道,不过白马县,最近补种了春麦,已经出苗了,经不起蝗虫。”

“玉昆。”王韶忽然神色变得郑重起来,“说实在的,如今你已经是府界提点,就算白马县的春麦都被蝗虫啃光了,也不会影响到你。你的心思最好要尽数放在流民身上,千万不要分心。”

韩冈明白王韶这是为自己着想,低头谢道:“韩冈明白。”

“以玉昆你之才智,当知道如何取舍,我也只是多话罢了。”王韶笑了笑,又问道:“不知玉昆你准备怎么处置流民,数目以十万计,恐怕不会容易。”

“推广深井开凿,还有风车取水,同时兴建沟渠。”韩冈扳着手指,一桩桩数过来:“正是这等时候,推广才最是容易。还有堤防、水道,甚至修葺开封城墙,都需要人力。以工代赈,劳力也绝不会缺。至于无劳力的老弱之家,而则是让各保保正上报人头,逐日派给口粮。有水源,有沟渠,日后遇上旱涝,京畿百姓也能好过上几分。”

王韶听着韩冈说着,点了点头。摸着酒杯,又道:“玉昆,有没有想过招募流民实边?”

韩冈不知道王韶是不是在开玩笑,但他说的并不可行,“京畿离着熙河几千里地,募流民过去不容易。倒是陕西今年也旱,熙河路正好可以就近收人。”

王韶也是随口一提,笑了一声,“蔡延庆也是这般上奏的。”

“是吗?……王舜臣前日寄信来说,蔡仲远【蔡延庆字】在熙河路做的不错,今年在河州又开辟了六百多顷田,以茶易马的生意做得也越来越大,”韩冈回忆了一下,“听说今年怕是能有三万。”

“所以说今年熙河全路如果没有灾情,钱粮二事,就能够自给自足了。”王韶很自得的说着,熙河路由他所创,如今不过两年,就已经可以在不开战的情况下自给自足,这是他最为自豪的地方。

“此皆是枢密之力。”

“也多亏了玉昆你辅佐之功啊。”

互相吹捧的喝了一杯,王韶像是想起了什么,神色又沉了下来:“玉昆,你可知道,畿内监马场一年有多少出息?”

“京畿的监马场不是已经撤了?还是前任府界提点吴审礼下的手。”韩冈奇怪的反问道,京畿一代的牧马监就是因为没有出产,朝廷不断要往里面贴钱才会被撤的,王韶怎么这么问?但立刻就反应过来,惊问着:“朝廷要在熙河路置监马场?!”

见韩冈反应过来,王韶用力一拍亭中石桌:“玉昆你说说,群牧司什么时候办好过一件事的?!”

熙河路茶马互易,不仅仅是换到合用的战马,同时也是将吐蕃诸族捆上大宋战车的必要手段。如果在熙河路设立马监,以群牧司的水平,一年能出个三五百匹战马就已经谢天谢地——熙宁二年到熙宁五年,河北河南十二监,平均一岁出马一千六百四十匹,可给骑者两百六十四匹,就这水平,一年还要吞掉朝廷近百万贯的投入。

韩冈也绝不会相信群牧监的那群只知吃粪的废物能在熙河路做出什么好事来,当即说道:“此事韩冈肯定要跟家岳分说个明白,熙河路绝对不能设置监马场!”

