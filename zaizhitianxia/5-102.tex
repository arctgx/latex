\section{第11章 城下马鸣谁与守(14)}

马蹄声在晋宁城衙前一阵阵的响过,怀揣墨迹淋漓的军令,一名名河东军的将领,跨上他们的坐骑,纷纷赶去他们的战场。

挥舞在骑手掌中的马鞭已经看不分明,只留下一抹淡淡的身影,黄裳目送着最后的一名将领离城远去。战争之前的那一份紧张,如同沉甸甸的青石,莫名的压在心头。

在雁门关的时候,黄裳还是旁观者的感觉。但这一次围剿阻卜强盗,他却是全程参与。虽然敌人仅仅数千,可深入接触到战争的每一个环节之后,黄裳才知道,组织一场会战究竟有多么繁琐。

天候、地理、人员、粮秣、军器、敌情,方方面面都要考虑到。而且在战前这些的谋划运筹之中,还要考虑到将校间的关系和势力平衡,要激发出他们最大的作战意志,同时还不能让他们变得为争功而轻敌冒进。人心的把握,比起物资的调动,还要难上十倍。

一场战争,绝不是存在于史书中的那些个大捷、大溃、胜绩、败绩之类的冷冰冰的文字,参与其中的全都是活生生的人。

亲眼看着一名名将校抱着必胜的信念启程离去,其中又有多少能平安回返,没有人能预料得到。作为参与谋划的幕僚,他们的性命很可能就在自己错误的一句话下终结。

黄裳心口憋闷着,回到内厅时,便看到折可适正在誊写军令。

充任韩冈幕僚的气学同门,个个都是以饱读诗书而自诩,为韩冈起草军令时都免不了带上一股文酸气,一开始的时候甚至出现过好几次骈四俪六的文章来。被韩冈教训过几次之后,四六文体不见了,但就是黄裳来写也一样,还是显得过于文绉绉的,有些词汇很容易让本来就不识字的将领们、以及他们的水平不高的幕僚一头雾水。

在起草公文时,韩冈就这么要求。而在他过去所发布的几篇有关医术的条令和书籍时,也是宁可失之于繁琐,也绝不追求辞章的文华,绝不以辞害意。所以在他将自己的要求放在军事条令上的时候,也是不足为奇。

韩冈曾经说过,军中的公文、条令,用词必须精确而无歧义,同时还必须浅显易懂,免得接受命令的人产生误会。这是军中的通则,并不是韩冈所订立的规矩,不过执掌军事的文臣,很少有人愿意损伤自己的颜面,被人嘲笑文采,只为了让下面的将校们,不至于误读上命。

折可适现在正在做的差事,就是将一些写得过于晦涩,以至于产生了诸多歧义的军令草稿,以将校们能理解的文字重新翻译一遍,再呈递韩冈过目确认后,遣人送出去。

黄裳回来,看见折可适忙得连话都没空说,便没有打扰他。但折可适听到了黄裳的动静,却放下笔,“勉仲,人都送走了?”

黄裳点点头:“全都出了城,”

折可适又抬眼看了看黄裳,“勉仲兄,你出战前朝廷还向你说了些什么?怎么见你现在心情似乎有些不对。”

黄裳脚步停了一下,想了想,还是将自己的感触向新交的好友简洁的说了一遍。

“……习惯了就好。”折可适听了之后很不在意的说了一句,就当做是劝诫。

只是黄裳见到他的态度,却变得十分的震惊。没想到折可适这个平时都减少了与人针锋相对的的和气之人,对战争的态度竟然是这番模样。

折可适没空,但他现在正在忙着,头也不抬的说着,“既然吃了这一口饭,死在战场看多了也就习以为常。在下的祖父辈,十六个兄弟,现在只有一人独存。剩下的十五位伯祖、叔祖,六个死在各个战场上,三个死于旧伤复发,剩下的也是各式各样的疾病而或迟或早,寿终正寝的只有先祖父一人。这一点,可以问龙图。他可是从跟随王相公一起从西北边陲起家,刚开始的时候,手上的人比你我更少一点,与上阵的将校也更加亲近。”

“不是人人都比得上龙图。”黄裳叹了一声,却往韩冈的客厅走去。并不是要问一下韩冈的心路历程——他也不打算去问——而是回去缴令。

黄裳通名后进厅,韩冈正在看着一封信,在他的桌上放着根黄铜圆筒,是之前黄裳都没有看到过的东西。

“人都走了?”韩冈放下手中的信函,他的问话跟方才的折可适竟然差不多。

“都走了。”黄裳点点头,“离开得都很痛快,没人犹豫耽搁。”

“……都是想早日立功受赏。”

人为财死鸟为食亡,这是免不了会出现的结果。黄裳忽然之间,那份沉重的感触忽然去了些许。随即笑了一声,对韩冈道:“有龙图做出来这些布置,阻卜贼寇必然插翅难逃,如此一来,过上半年,北阻卜吞并草原诸部的消息当能传到太原府中来了。”

“事情没说的那么简单。”韩冈摇摇头,“你以为我们能想到的,耶律乙辛会想不到?作为执掌辽国的权奸,对于辽国国中形势的了解,他远比我们要强出百倍、千倍,甚至万倍。西阻卜既然南下匡助西夏,那么阻止北阻卜趁火打劫,以耶律乙辛的才智,会不做这方面的准备?”他笑了一声,“就是过几天听说大辽尚父将计就计,将南下准备吞并西阻卜各部的北阻卜给打回去,甚至全歼,我都不会太惊讶的。”

“……那龙图为何要去做,”

“什么都不做,永远都不会有成果。只有去做了,才会有机会博取一个成功。”

“成功?……龙图的成功可是要让阻卜贼寇血债血偿?”

“是的,血债血偿。”韩冈抿起了嘴,双瞳变得幽深起来,“自从见识过邕州的惨剧,对于四方蛮夷在我汉境留下的血债,就只有用血来偿还。”

黄裳很能理解韩冈的心情变化。

由于韩冈的主导,至今交趾男丁尽数受了刖刑,成了广西洞蛮的奴隶,为瓜分了交州土地的洞蛮种植甘蔗和水稻。他还记得曾经有友人指着雪白如霜的交州糖说过,别看这些交州糖白得跟雪一样,但里面实际上全是血。

但换作是现在,在黄裳去查看过被阻卜人攻破的一个村子之后的现在,当听到有人为屠戮了邕州的交趾人叫屈,他肯定会当面骂出声来。

韩冈抬起眼,问黄裳道:“勉仲可还记得汉书列传第四十?”

黄裳扬了扬双眉:“明犯强汉者,虽远必诛?!”

“啊。没错。”韩冈笑了笑,“虽然如今给人说得滥了,招人骂的时候也多。但百卷汉书,我最喜欢的还是这一句。‘宜悬头槁于蛮夷邸间,以示万里,明犯强汉者,虽远必诛!’班孟坚【班固】虽然在卷后的赞中没有说陈汤的好话,但这一卷中几篇列传,陈汤传是最长的一篇,甚至比起其他几篇加起来还要长。班仲升【班超】的这位长兄,想必在撰写陈汤传的时候,难以遏制自己的笔锋。”

黄裳点着头。陈汤的这一句,寻常时说来只不过让人一时激动,但眼下战火正炽,应时应景,却不免触动人心。

“邹衍旧有大九州、小九州的说法。观我中国之地,也不过一赤县神州。神州之外,不知有多少土地和人口。普天之下,莫非王土,率土之滨,莫非王臣。汉唐倒是为此努力了,但接下来的确实不成器。三国、五季的中原内斗,螺丝壳里做道场,太小家子气了。天下之大,可不仅仅局限于九州之地。所以读起陈汤等人的列传,比五代史可要痛快得多。汉书能下酒,新旧五代史只会想让人摔茶杯。”

黄裳不便随着韩冈一起说史书的不是,他还不够资格,遂岔开话题:“大地之广,记得学士过去也曾说过。《桂窗丛谈》中便提起过大地乃是球形,因其内径万里,所以外面的周长几近十万里。也因如此,人居其上便发觉不了实乃球形。”

“如何确定大地乃是球形,方法早就说透了,但缺乏准确的数字,反倒像是臆测了。待此间事了,当设法精确的测算一下子午线的长度,唐时僧人一行曾测算过,但谬误太甚。气学当以求实为上,求实切理。格物致知,求得就是一个真实无误的‘理’字。”

组织人手测量子午线,韩冈不是一时心血来潮。在关西,程颐刚刚结束了为期一年多的讲学,返回洛阳。他在关中的一年多,已经将程门洛学灌输给了许多关中士人。苏昞现在还在横渠书院独撑大局,却无力对抗程颐。韩冈不可能光是将同门师兄弟塞入自己幕府,在学术上必须要有新的成就,或是证明他独有的观点。虽不是迫在眉睫,但留给他的时间也不多了。

韩冈想着,顺手将桌上的那个黄铜圆筒拿起来递给黄裳:“这是天子遣人送来的新什物,以佐军用,最近才由将作院中一名眼镜匠献与天子。”

黄裳接过来,随手摆弄了一下。发现这个黄铜圆筒是单纯的两节套筒,前后皆有一个水晶镜片。

“是显微镜?”他一边问着,一边的轻车熟路的拉开圆筒。一头对着自己,一头向着桌面照过去。

“调过来,看窗外。”韩冈指点着。

黄裳依言施为。对着窗外一照,院中的一株老梅在镜中竟然一下跳到了眼前,他的身子竟不由得向后一仰。黄铜圆筒的镜头移动,院中的一草一木,一砖一瓦,皆被这个形制与显微镜的主体相差不大的东西拉倒了近处。

将黄铜圆筒从眼睛上放下,黄裳瞠目结舌的问道,“龙图,这是……”

“此物洞烛千里,天子起名做千里镜。”韩冈说着摇摇头,赵顼起名字还是没有创意,“这个名字夸张了些,叫望远镜其实更确切一点。不过天子既然起了这个名字,就这么叫好了。显微镜能让人明察秋毫,千里镜能让远处之物犹在眼前。勉仲可知道其中的道理?”

黄裳颠倒着看了几眼:“是凸透镜和凹透镜的重叠。”

“可不是随随便便拿两种透镜叠起来就能成望远镜的,要不然也不会到了今天才有人发明。形而上谓之道,道便是理。明白了道理,就返归于形而下的器。了解到了千里镜的原理,就能造出望得更远,且更加清晰。”韩冈从黄裳手上接过千里镜,“这东西与飞船搭配起来最为有用。不过这一次是用不上了。看看这一战谁的功劳最大,当个彩头好了。”

韩冈的大方,让黄裳吃惊非小,“这可是御赐之物……”

“宝剑赠烈士、红粉赠佳人,千里镜还是给领军上阵的将校好了。我倒希望贼人被绑到我的面前,而不靠千里镜。”韩冈不以为意,“只要能格出千里镜内蕴的道理,便可回报天子。到时候,千里镜成为寻常之物,每一艘飞船上都能配上。”

