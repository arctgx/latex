\section{第27章 舒心放意行所愿(上)}

吕公著须发皆张,显是怒不可遏。他昂首立于寝殿中,厉声怒斥道:“皇宋以孝治天下。陛下今日以皇后权同听政,不知孝在何处?”

吕公著的斥责,让向皇后脸色骤变。这个罪名太大了。忠孝是国家的根本大节。在家思孝,入朝思忠,忠孝二字是一体两面,是儒家社会稳定的根基。就算是天子也不敢明着违反孝道,否则如何劝导臣子忠心?

“吕枢密何有此言?”王珪站了起来,挡在前面,“此事陛下自有因由。”

“纵有因由,也不当陷太后于不义。”吕公著冲前一步,声音更大了三分,“陛下不以太后、而以皇后同听政,敢问世人当如何视太后,太后又当如何自处?!到了英宗皇帝忌日,不知陛下在神主前能无愧否?!”

吕公著如此激烈的反应有些出人意料,甚至连外殿的王中正和张守约都听到了声音,咬着牙跑过来看风色。

韩冈也同样觉得意外。他不信吕公著没看出现在的风向,硬顶着来也没有任何意义。即便不想落一个反覆无常的名号,也不应该这般义愤填膺。

吕公著从来都不是王安石那种倔强得认死理的臣子。吕夷简阴狠狡诈从来不缺,家学渊源,他的儿子怎么可能是刚直严正的清介之臣?

两年前的陈世儒弑母案中,吕家的人为了自保,几乎将大理寺都给收买了。没吕公著点头,能这般肆无忌惮?知情识趣,那是必然的。可现在吕公著一脸正气凛然,却好似包孝肃附身的模样。

韩冈冷眼看着吕公著到底要玩什么花样,也不站出去跟太子太保打擂台。反正他今天做得够多了。过尤不及,现在该发扬一下风格,让其他人有机会做个表态。

韩缜站起身,打着圆场道:“吕枢密,这不过是依章献明肃皇后旧例,依循故事而已。”

章献明肃皇后,也就是真宗的刘娥刘皇后,她在真宗晚年病重的时候,曾经以皇后的身份代为处理政事。

但吕公著立刻驳了回去,“天禧年间的皇宫里,可没有皇太后在!”

吕公著的气势高涨,但王珪今天也是第一次做得像一名宰相,他沉下脸:“王珪有闻,宫保曾治《春秋》。不知吕宫保怎么看郑伯克段于鄢这一条。郑庄公待共叔段,做得是对是错?”

殿中众人闻言,齐齐悚然一惊。王珪的这个比喻好狠!韩冈都被吓到了,惊讶莫名的看着王大丞相,心道他还真是敢说。

郑伯克段于鄢,是《春秋》开篇第一年最有名的一桩公案,是有关郑庄公和他的母亲武姜及弟弟共叔段的故事。

武姜生郑庄公时难产,所以讨厌这名长子,而喜欢幼子共叔段。当共叔段成年后,觊觎国君之位,小动作不断,而郑庄公却一直优容,甚至给了他最好的封地。直到共叔段在武姜的支持下,举起叛旗,郑庄公这才整军讨逆,杀了共叔段,并将武姜囚禁。

在历代儒生们的眼中,这一件事,武姜和共叔段纵然有过,但郑庄公的过错也不轻。有弟不教,纵容太甚,也是共叔段敢于谋叛的原因。所以夫子微言大义,用一个‘克’字,来表达了对郑庄公的不满。

王珪这个比喻,等于是在说,赵顼就是为了避免这个结局,才特意让皇后而不是太后来垂帘。但用武姜和共叔段来形容高太后和赵颢,如果没有相应的行为,那就是极为恶毒的污蔑了。

蔡确回头看了看,发现赵颢已经连站都站不稳了,手扶着高太后方才坐的交椅的椅背,整个人都在发抖。

蔡确只觉得自己的思路变成了一团乱麻。在自己入宫之前,福宁殿里肯定发生了什么,只有王珪、薛向、韩冈和张璪这几位宿直宫中的人才知道的事。

只是蔡确想不通,要是在他们几位回家的执政重新回来前,对天子现在的这个安排已经有了决定。为什么当天子要皇后垂帘,王珪、薛向会那么惊讶?而太后也早该拂袖走了。而且吕公著的宫保又是怎么回事?

想不通啊。蔡确恨不得用锤子敲自己的脑袋,将灵感敲出来。

章敦也狐疑将视线左转右转,想在王珪和向皇后的脸上发现点什么。方才他还准备站出来表态呢,但王珪的一句话把他都惊得缩了脚。王珪的话等于是在给高太后和雍王定罪,并不仅仅是为了驳斥吕公著。到底发生了什么事,才让王珪这枚滑不留手的至宝丹如此迫不及待的表忠心?

吕公著也看到了,狠狠地瞪了已经失魂落魄的二大王一眼,“太后纵有过,可以私下规劝,哪里能弄得满城皆知。这世上岂有曝父母之过的道理?!”

原来如此。韩冈算是听明白了。

前面吕公著请皇太后垂帘,现在情况有变,也不方便立刻改口。将错就错的强硬到底,还能博取一个直名。但吕公著口口声声不离孝道和太后的脸面,调门的方向明显的转向了赵顼所用的手段,而不是他这个诏令的内容上。

韩冈暗自啧了一下嘴,比起这等成了精的老滑头,自家还有得磨练。

坐在床榻边的向皇后这时候起身,端端正正的面对着朝堂上地位最高的一众臣子:“方才韩学士有言,陕西耀州,河北祁州,有两座药王祠灵验非常,若有至亲去祈福,或有奇效。敢问吕宫保,不知这两位至亲是该去还是不该去?”

寝殿内顿时静了。

“好手段!”章敦喃喃低语。

蔡确和韩缜也立刻抬眼望向韩冈,眼里只有震惊。

三人都是人精,一下便想得通透。

吕公著也是气焰一收,一下就怔住了。看看赵颢,又看看韩冈,难以置信的再转回来:“难道太后……”

“长辈的过错,做晚辈的怎么敢说?”向皇后态度强硬。

在内有丈夫的支持,在外又有几名宰执和韩冈等重臣辅佐,而且还抓着太后和雍王的把柄,一下就变得底气十足。

吕公著低下了头:“臣无话可说。”

他前面纵然已经服软,只是要维持一下体面,但他决然没想到,事情的性质会这么严重。

若太后所言为实。这件事如果传出去,没人能说天子半句不是,而都会指责太后不识大体,雍王有不轨之心。以太后和雍王的今夜表现,王珪用武姜和共叔段来比喻,并没有太多不合适的地方。

眼角的余光只能看到韩冈的脚尖,吕公著心头憋得发慌。眼下的一切全都是这个灌园小儿带来的结果。

以吕公著的才智,就算只有向皇后的几句话,也能想明白是怎么回事。

同样的话由不同的人来说,得到的结果是不一样的。如果是自己私底下劝说,纵然艰难一点,但使太后点头同意,让两位亲王出外为天子祈福,还是可以做到的。可这话换成是韩冈开口,那么听在高太后的耳朵里,就只有四个字——包藏祸心。

吕公著自问,换作是自己心里面也要打鼓。难道韩冈的打算就只是让人出京吗?一路上就不会做手脚?就算天子不做,也会有人想为天子分忧!

但这番心思如何能公诸于众,如何能取信于世人?人们只会说高太后太偏心,想趁长子重病,让最喜欢的次子占据皇位。当士林清议和民心全都在天子和皇后一边,那么太后、雍王无论如何都翻不了身了。

这个机会是韩冈带来的,是韩冈让天子可以理直气壮的将权同听政的资格交给皇后,而不用担心朝堂上的反弹,更不用担心皇宫内的暗流——人心向背,今夜一过,皇后可以轻而易举的控制住皇宫内外。

吕公著已是哑口无言。

韩冈自吕公著身上收回了视线。从他的反应上看,朝野上应该不会有反弹了,最多也只会有点杂音。

今夜虽是百转千折,终究还是有了一个完美的结局。

为什么要起用旧党,因为太后将会垂帘。

为什么要无视多年心血,因为太后将会垂帘。

为什么要忍辱负重,因为太后将会垂帘。

赵顼之所以拖着残躯,百般谋算,根子就在太后身上。

只要太后无法垂帘,进而控制朝堂,那么旧党无法上台,新法不会被废,而雍王也只有回家闭门思过的份。

当权力落入皇后手中,太后在宫中的地位将会随之缩减,皇子的安全更能得到保障。换作是太后垂帘听政,那么后宫中,向皇后连站都没地方站了,至于赵佣,只能将性命托付在太后的心意上。

所以韩冈必须要赌一把。

提议二王出京,与其说是赶人,还不如说是逼赵顼和高太后撕破脸皮。刻意引发高太后的怒火,让赵顼明白妥协退让也不会有好结果。

以妥协求团结,而团结不可存。以斗争求团结……现在也不需要团结了。

要引发太后的怒火并不难。韩冈一直都清楚,太后恨自己。这并不出奇,若是自家最疼爱的儿子的名声被人毁了,而且一日一日的被世人嘲笑,韩冈也绝不会轻饶。所以自家说得任何话,落到太后的耳朵里,都会被扭曲成别有用心的图谋。

而天子这边,并不需要赵顼对太后怎么样。一边是韩冈定然被重责,以至独子性命多半难保,另一边,不过是顶撞一下母亲,又不会伤其性命。孰轻孰重,自不用多说。只要赵顼能想得到,只要敢去想,要做出韩冈想要的决定,那是必然的。

只是高太后的反应如此激烈,逼得天子痛下决断,还是超出了韩冈的预计。甚至让他暗暗心惊,高太后藏在心中的恨意不知积累了多少,恐怕已经将自己视若仇雠,一旦由她垂帘,结果当真堪忧。

幸好赌了这一把,也幸好对手是个更年期的老太太。

结局近乎完美,韩冈的思绪已经飞到了明天……应该是今天的早朝上。

宰执齐齐入宫的消息肯定是传开了,吕公著被封太子太保的消息也定然保密不了,但具体细节却不会有人知道。

届时,朝堂上的乐子不会少。

韩冈带着些许恶意的想着。

