\section{第30章 随阳雁飞各西东(21)}

天色沉黯。

一众宰辅才从城中鱼贯而出,身后的右掖门就迫不及待的被合上了大门。

由于西北边事的缘故,宋辽两国已经处在了战争的边缘,京城中的百万军民已经都了解到了这一点,甚至由于谣言更能深入人心的缘故,年节前的气氛也变得紧张甚至诡异起来。

与辽人交手和与党项人交锋,完全是两回事。虽然平日里,也有许多人高声赞着一众名将的武勋,以及大宋官军的威猛,但事到临头,却还能保持着自信的已经不多了。

两府——确切的说是西府,在万马齐喑的现在,却还拥有着最为强烈的自信。

掌握着最充分的情报,也拥有着足够的战略判断,更对军事有着充分的了解,这让章敦、薛向,以及参赞军事的韩冈,对战局保持着强烈的信心。

——除了一件事,这场边境冲突到底会不会扩大成战争,这是他们都无法给予保证的。尤其是今日午后,来自于银夏路的奏报,让他们更多了一层忧虑。

章敦和韩冈并辔走在御街之上。一路沉默,快要到了州桥,章敦方才开口:“吕吉甫看起来压不住种五。”

“吕吉甫不是说有宣抚司总理西北边事,不日当可安定,请天子、皇后勿须忧虑。”

“他是要保着他的脸面。”章敦顿了一下,声音低低的给了一句评语:“顾头不顾腚。”

从边地发来的情报上看,种谔已经在调集银夏路的精兵强将,要跟辽人打上一仗了。要不然也不会奏报说要鄜延路出兵帮忙镇守夏州。

不过更为诡异的是宣抚使吕惠卿那边,他一直都在说一切尽在他的掌握中,完全没有提种谔自把自为的行事。

吕惠卿的私心,京城这里不是看不明白。作为枢密使兼宣抚使都无法掌握住麾下将领,那么他想再进一步往宰相班中走,那可就是笑话了。

“也不知现在溥乐城那边怎么样了。”韩冈仰头望着夜空,阴云密布,看不见一颗星子。

“围城弥月,溥乐城下的辽军差不多快到极限了。”章敦说道:“种谔老于兵事,不会看不到这个时机。”

“这也要溥乐城不破。”

“哦?”章敦饶有兴致的回头,“玉昆你会担心溥乐城?”

韩冈沉默了一下,而后摇头,“……不担心。”

吕惠卿并不蠢,他能争权夺利,就代表形势并不糟糕、

与溥乐城前线有着十天的军情延误,与京兆府之间也有五天的间隔。现在说不定就要出结果了。但朝廷所收到的最近一个消息,除了种谔的奏报,就是吕惠卿打算去延州坐镇。

去庆州远比去延州要更易于指挥,可吕惠卿偏偏选择了延州。

这不是指挥,而是压制。吕惠卿没脸说出来,但他不得不去弥补。

因为延州离夏州更近,因为鄜延路是种家的根本所在。

就是因为有了这份奏章,所以韩冈和章敦才会确定种谔肯定是将宣抚司丢在一边自行其事了。可换个角度,吕惠卿也是有自信最后能压住种谔,才没有上书指责——权衡利弊后,他更相信自己的能力和手段。

既然他都如此表态,两府也就只能暂时观望,而不会去插手宣抚司中事。

……………………

离灵州川边的大道大约两里的一处荒坡之后,种建中和他的麾下一众骑兵正耐心的等待着目标的到来。

地平线上的火光映红了半幅天空,耀德城中的熊熊烈焰卷起的滚滚热浪,远隔十里似乎还能感受得到。

呼吸中还有浓浓的血腥气,这是他们攻下耀德城的证明。虽然杀人放火的行为只过去了半日,很多人还沉醉在半日前的兴奋中,不过更多的人都已经半闭着眼,抓紧一切时间休息,以便能更快的恢复精力。

脚下的大地微微的颤动了起来,沙砾在地面上跳起了舞。

原本半眯的眼睛一下瞪圆,懒懒散散如同睡猫的种建中也豹子一般恢复了精神。

倚着战马,抱着弓刀在假寐的骑兵们也一个个跳了起来,他们守候的目标看来已经出现了。

之前就有了动静,但直到现在才让所有人都感受得到。

“人好多!”

一名精瘦干练的军官俯下身子,刚将耳朵贴上地面,就立刻叫了起来。

紧接着他的第二句话就是:“来的好快!”

“有多少人马?!”种建中紧张的问道。

“乱得很,听不太清楚,但至少在五千人以上!”那名军官抬起头,“十里开外,再有半个时辰就该到了。”

“五千……”种建中知道伏地听声的极限,一旦兵力多过一定数目,就无法细细分辨数目了。不过五千应该不会错,他相信自己手下这名军官的能力。而以辽人在溥乐城下的兵力数目,回师不应该超过三千——再多,剩下的兵力就不足以继续围城了!

可既然确定了现在回师的数目在五千人以上,那么就只会有一个可能。

“要准备动手了?”种建中的副将上来问道。

“找死吗?!”种建中骂了一句。现在不用伏地听声,也不用推断,只听这逆着夜风中都能传入耳中的声势就知道,溥乐城下的辽军肯定全回来了。上万大军行动,领头的肯定是精锐中的精锐,只要被缠上一时半刻,就别想走了。

“都上马,走!”

种建中放弃得很干脆,手上的兵力不到九百,想要伏击这般规模的对手,可是会崩坏了牙。

辽人的主力就在十里开外,因为人多的缘故,前锋的动静被掩盖了去,但推算起来,也就喝杯茶的时间就能到眼前了。再不走可就迟了。

要是再多两千就好了。种建中扯过缰绳的时候满心遗憾。若是手上有三千精兵,就算不全是骑兵,他也敢去赌上一把,给赶回来的辽军一个好看。甚至打出一个斩首五百以上的大捷出来。

跟之前河东路与藏头遮尾的契丹人打得几仗不一样,这可是与旗帜鲜明的辽师明明白白的较量!

这是能留名青史的功劳!国史上,自己绝对能留下一篇独立成篇的列传!

可惜啊!!

种建中只想叹气。但又立刻收起心思,跳上马,领头就往东行去。

种建中的命令立刻得到八百多大宋骑兵的执行,远方传来的动静,其实已经让这群五天内绕行了近千里的勇士们心惊胆跳。

与辽军厮杀一场也没什么,反正之前连城池都攻下来了,士气正盛,再厮杀一场正合人意。要不然种建中又怎么会在路边设伏,准备再捞上一把?人心所向啊!

可冲到辽军面前送死那就是另一回事了。

齐刷刷的跳上马,打个呼哨就跟着种建中向东面飞驰而去。

夜风料峭,凌冽的寒意穿透了外罩的衣袍,种建中半日来在峰谷间急剧变化的心绪也逐渐沉淀下来。

直到现在,他才终于可以冷静下来好好算一算今天的战果。

烧掉了屯满军粮的耀德城,连带着还灭掉了两支辎重队,斩首两百多,最重要的,是解救了溥乐城,怎么看也是一场大功劳了。

回头而望,淡淡月光下,奔驰在荒原上的八百多骑兵,深色的剪影正随着地形而起伏。一人双骑,队列又分散,一眼望过去,竟然充斥于视野中,仿佛有千军万马在同路奔行。

驻泊在银夏路的七个将五万禁军,是从鄜延和永兴军两个经略司辖下的兵马中挑选出来的的精锐。种谔手上所掌握的马驼等牲畜的数量,比去年多了许多。能分给种建中的骑兵也比过去要多得多,而且还是一人双马,这在连骑兵都没有马匹的过去,根本不敢想象会有如今的场面。

不过凭借不到千人的骑兵,能攻下耀德城还是出乎意料,包括种建中,也肯定包括他的叔父种谔。

在出战前的计划中,种建中从他叔父那里领到的将令,也只是动摇辽人军心,骚扰进而破坏他们的补给线。为之后主力的决战,做好铺垫。

如果一切都依照计划,辽军因为补给线和后路受到干扰,必然要设法解决他这一支与苍蝇差不多的队伍。而以辽人的贪婪,又绝不会放弃在野战中击败种谔的想法,所以最终辽人将不得不将麾下主力一分为二。让种谔在决战开始的时候占据一定程度上的优势,进而将优势转化为胜势,甚至全歼兴灵的辽师主力。

可现在自己竟然攻下了耀德城,反而破坏了一开始的计划。溥乐城下的辽军全师而回,全歼他们已经不可能了。

这样也好,种建中很是轻松的想着,至少不用担心叔父去攻打兴灵了——若真能在溥乐城下,将兴灵的辽军给全灭,他的五叔可是已经做好了一战收复兴灵,彻底挽回旧日遗憾的打算。

至于打下来的后果是什么,种谔是准备让京兆府的吕宣抚,京城的两府诸公,以及天子、皇后去头疼。一名边臣、战将、武夫,是不需要考虑那么多问题的!

尽管接下来很可能是种家被打压,甚至占据下来的土地都会被还回去,如今遍及西军要职的叔伯兄弟更是有可会被分散到全国各地,乃至贬官、降罪。

但只要辽国还与大宋为邻,不论如何被打压,种家的子弟终有被重新启用的一天。

击败辽军,这个胜利就是未来起复的本钱!就是长保种家家门不堕的希望!

虽然应该不会有这样的结局了,不过能安安稳稳的保住门户,其实也不错。

种建中心情愉快的想着,挥鞭打马,将身后那千军万马、犹如九天惊雷的震动给抛诸脑后。

直到他会合了种朴、种师中,在两天后见到了种谔。

“走!”种谔在马上一扬马鞭,并没有在溥乐城久候的打算。

“去哪儿?”

种家的十七,十九和廿三三兄弟同时问道。

“兴灵!”种谔望着北面,眼中闪烁着灼灼精光,语气却尽其可能的平淡:“青铜峡的仁多零丁和叶孛麻攻入兴灵了。”

