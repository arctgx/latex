\section{第19章 此际风生翻离坎(下)}

在经学上,苏颂并不是完全认同张载和韩冈的观点,不过在自然之道上,他却是站在韩冈的一边。看到如今天子将气学视为敌寇,苏颂不能不为依然坚持在《本草纲目》中与新学争战的韩冈担心。 

“子容兄多虑了,韩冈本也没打算逆风行船,过往行事,也多是借势顺水推舟。不过子容兄,如今风势当真是一封诏令就能扭转过来的吗?” 

“玉昆何出此言?” 

“听说南京【商丘】那边已经有人开始设私窑造玻璃了,能磨制镜片的透明玻璃日后将会越来越廉价。”韩冈笑了一下,会挖墙脚的不独他一家,毕竟这个市场并不小,“市面上也有了专卖眼镜的店铺,会磨镜片的匠人也将越来越多。天子总不能不让人戴眼镜吧?” 

这分明是硬挺着不肯服输,苏颂叹了一口气,却听韩冈继续说下去。 

“何况还有放大镜,显微镜,都不可能一并禁绝,这些镜片,只要形状一样,与千里镜的镜片有几人能区分得开的?试问朝廷又能用什么手段将千里镜给禁掉?制作千里镜的成本不会超过五贯——这是从军器监中传出来的——而等到禁千里镜的诏令正式推行,外面的市价少说也会涨到五十贯,至少十倍的利润,由不得人不心动。” 

“玉昆,商人好利,但钱再多也比不上性命珍贵。”苏颂警告道:“千里镜虽被归入禁兵器,但私藏千里镜,多会一并犯下私习天文的禁令。如今朝廷喜酷吏,到时候拥有千里镜可就是两条罪名一起算进来了。” 

“听子容兄这么一说,倒是让人想起了汉先主和简雍论私酿的事了。”韩冈忽而笑了起来。 

刘备据蜀后,有一年蜀中发生旱灾。刘备恐粮食不足,便下诏禁私酿。当诏令下达后,下面的执行也极为严格,甚至打算将拥有酿酒器具的人家也一并处罚。简雍和刘备一起出游,看到一男一女一起走路,便对刘备道:‘这两个人意欲行,淫,为什么不速擒之,依律法办?’刘备疑惑的问道:‘卿何以知之?’简雍道:‘他们身上都有奸.淫的工具!与有酿具者相同。’ 

被搜集进《太平广记》中的这个故事,不必韩冈多解说,大多数士大夫纵使记不得《简雍传》中的细节,也一样耳熟能详。 

韩冈的笑声中有浓重的讽刺味道:“只要有眼睛,抬起头来便能观星,禁得了千里镜,难道还能连眼睛也一并禁了不成?” 

听出韩冈的言辞中似有怨怼,苏颂的脸色都变了,急声道:“玉昆,你只记得先主和简雍,怎么却不记得先主与张裕?!” 

刘备入蜀后,以旧恨欲杀张裕。诸葛亮问刘备张裕究竟犯了何罪,并称张裕人才难得。刘备的回答很妙,让诸葛亮无言以对。 

“‘芝兰当门,不得不锄。’想想张裕是怎么死的?玉昆你想做张裕不成?!”苏颂一时间声色俱厉。只是看着韩冈的神色,口气又软了下来,“人亦是,物亦是,道亦如此。天子若无意主张气学,玉昆你暂且放一放又有何妨?!” 

能说出这番话,苏颂算是掏心挖肺了。韩冈起身,端端正正的向苏颂行了一礼:“多谢子容兄之言,韩冈理会得。”他苦笑一声,“为千里镜叫屈的话也只在这里说,日后自能见分晓的事,韩冈也没打算上书诤谏。不过在气学,是绝对不能让的。” 

对于韩冈这种宁折不弯的脾气,苏颂有三分无奈,但也有五六分欣赏。说起来,当年他做中书舍人的时候,也是不当让时,绝对不让。硬是不肯给天子草诏,与其他两位中书舍人,号为‘三舍人’,最后贬官出外。 

苏颂也不再劝了,转开话题:“洛阳的大程小程,听说最近有新书出来了。” 

“新书以《易传》为名。”韩冈一直都在关注洛阳,收到消息,自是比苏颂要早,“很早就开始写了,只是最近才出来……也是赶着近来的风气,要争一争道统。” 

 “《易经》源出三圣,如果不论后人伪作的可能——其实也就欧阳永叔说《周易》中有几篇为后人伪作——算得上是诸经中最早问世的几本之一。只比《尚书》迟上了那么一点。圣人之学,其根本便在这一部书中。”苏颂顾视韩冈,摇头轻笑,“二程作《易传》,这也是一般的要从根源做文章了。” 

王安石作《字说》,这是从一字一词的训诂释义上下功夫,由此来抢占儒门经典的注疏权,加强之前《三经新义》的根基。就像后世一级级升上去的教育制度,小学是中学、大学的前提和基础;此时的小学,也同样是一切经学的基础。而程颢程颐如今以《易传》传世,也是有着同样的心思。 

二程的《易传》,韩冈的自然之论,王安石的《字说》,都是从基础中来,将根本攥住。一旦事成,道统便在手中。而三家所选择的着眼点不同,便是体现了三家学派根本性的差异所在。 

不仅仅是这三家学派,其他学派治学,无不是用上提纲挈领的做法。 

前代诸儒,孙复著《易说》、石介授徒以《尚书》、胡瑗有《洪范口义》、《论语说》;欧阳修则是通观诸经,乃至考订其真伪。但他们都有一个共同点,就是皆崇法《春秋》,在春秋三传上下的功夫极深。 

此乃当时的天下局势,西北二虏猖獗而中国衰弱,必须在尊王攘夷四个字上下功夫,以振奋人心,除了《春秋》,别无他选。同时在唐时所定下的《周易》,《尚书》、《诗经》、《礼记》、《春秋》这五经中,以《春秋》一经最易着手,毕竟孔颖达所编纂的《五经正义》,只有《春秋左传正义》,《谷梁》、《公羊》两传说得就少了,这就给了宋儒上下其手的空间,宋儒颠覆汉唐经学亦自此而始。当时的纲领,便在《春秋》之上。 

《春秋》是鲁史,到了如今,三苏父子重史论,司马光重史学,其实可算是《春秋》一脉。而邵雍、周敦颐,一个钻研象数之《易》,另一个讲得是太极图,皆属易学一脉。都是有源流的。 

韩冈要争道统,当然对开国以来的儒门变迁有过一番深入的了解。那些有所成就的大儒,他们治学的根基在何处,韩冈清楚,他们自己更清楚的。如今新学、程门一争道统,没有人会知道该在哪里做文章。 

天子压制气学,说实话,韩冈的心里可是憋着一口气。虽说在情面上对岳父王安石有些过不去,不过这一场道统之战,韩冈可是半点都不准备退让,也退让不得。 

检视着桌上做样品的药材,韩冈的眼神愈发的冷冽,“家岳《字说》成书之时,曾经将初稿寄送给我。不过看了之后,便回了一句‘刻舟求剑’。如今洛阳的《易传》,在在下看来,也是一般的‘刻舟求剑’。” 

“此话何解?”苏颂问道。 

“要治《易经》,首先得明其源流。”韩冈解释道:“虽说《周易》源出三圣,但在《周易》之前,有《连山》,有《归藏》,可不光是拿着《周易》解说一下就算了事的。” 

“《连山》、《归藏》不早就失传了吗?唐时虽有此书,但那已经考订过是伪作了。” 

“那两部的确是失传了,但三部《易经》,时间有先后,《周易》出世时便有所依照,要解释《周易》,先将与《连山》和《归藏》的关系解说明白。” 

赵顼有什么样的算计,那是他的事。但韩冈并不打算做个惟上是从、亦步亦趋的臣子。不跟皇帝拧着来,那还配叫士大夫吗? 

韩冈费尽心思,给自己加上一层药王弟子的光环,求名只是一部分——不打算学王安石养望三十年的韩冈只能设法去走捷径——但另一部分,便是让赵顼投鼠忌器。 

因为皇嗣不振的关系,赵顼是决然不敢与自己撕破脸皮的。比起能不能有儿子继承帝位,‘国是’也好,‘一道德’的目标也好,在赵顼眼中都是要放在一边。自家辛辛苦苦为国为民,最后让弟弟或是侄儿坐在皇位上享受,赵顼可能会甘心吗?就是在后世,也没多少人愿意牺牲自家的子女,将遗产留给兄弟,何况是在这个极端重视血脉和香火传承的时代? 

只要他韩冈不去犯十恶不赦的重罪,有什么问题赵顼都得忍下来。何况韩冈准备做的事,直接针对的目标,还是在新学和程门上,赵顼最多也只是附带而已。 

将药材和草稿整整齐齐的放好收起,桌上就只剩下二程的《易传》和王安石的《字说》,这就是他如今要对阵的目标。 

虽然赵顼一直拉偏架,但韩冈也不觉得他还能有多少办法。韩冈手上的底牌,不是赵顼等人能算计得到的,准备也做好了,一切比预计得还要顺利。 

该下棋的时候,他会安安分分的下棋,但必要的时候,掀棋盘也是一个选择。

