\section{第18章 青云为履难知足(11)}

【这一章写得有些慢,迟了一些。】

因为要与父母随行,韩冈没有走得太快。当一行人抵达开封城的时候,已经是夜幕将临。

在此之前,韩冈已事先遣人提前一步去通知了王安石府上,到了开封西南的戴楼门时,王旁带着两名身穿红衣、腰扎金带的相府元随就在那里等着了。

“仲元,怎么劳动你出来相迎?”

韩冈笑着下马,心中略感惊讶,王旁应该还是在开封府界提点司中,没听说他调任,平日都是该留在提点司如今的治所白马县,没事不该回京城的。

“玉昆你携胜而归,哪能不出城相迎?”王旁虽是在笑着,但笑容很是勉强。

见到王旁强颜欢笑的表情,韩冈心中一惊,忙问道:“元泽情况怎么样了?!”

王旁默然摇了摇头,韩冈脸色一黯,叹了一口气。让过身子,将王旁介绍给父母。

王旁连忙上前向韩千六和韩阿李行礼,恭恭敬敬,不敢有丝毫失礼数的地方。两边见礼之后,便是验了关文进城。

外任的官员入京,照规矩还是得先去城南驿报到,另外韩冈更是入京陛见,还得去一趟宣德门登记姓名。

韩冈领着父母先顺道去了驿战,留下了姓名之后,一行车马直接回到了他在京城的住所。

尽管之前王旖她们已经搬去了相府,但这间院落还是留了五六个人看守,日常洒扫内外,整理得干干净净。听到韩冈遣人传回来的消息之后,王旖四女也都带着儿女,匆匆从相府中赶回家来。

新妇拜见舅姑,加上孙子孙女拜见祖父母,光是行礼问安,就是忙活了一通。韩千六夫妇见到了活泼可爱的孙子孙女,喜得合不拢嘴。韩冈的四名妻妾,有三人大着肚子,韩家这一脉人丁兴旺可期,更是让韩千六韩阿李心花怒放。而在院子外面,还有冯从义指挥下人,安置车马货物。寂静了许久的韩家宅院,一下就热闹了起来。

王旁在一边陪着笑脸,只是微蹙的眉头,不停挪动的脚尖,显得他是心急如焚。

韩阿李见惯人情,催着韩冈道:“三哥儿,既然你已经到了京城,哪有不去拜望岳父岳母道理?今天你先去一趟,代你爹和为娘问候一二。等这边安顿下来,亲家得空,我们夫妻两个就去登门拜会。”

韩冈点了点头,匆匆梳洗了一下,换了身衣服,就带了伴当离家外出。先去了宣德门登了名,便匆匆与王旁一起去了相府。

进门的时候,已经是深夜。寻常寂静的相府却依然喧闹,尤其是位于相府一角的王雱夫妇所居宅院,更是一片灯火通明。

韩冈和王雱脸色一变,都知道事情不好了,也不去正厅,直接快步往王雱的小院走过去。

进了院子,却见到了方才还在家中的王旖,眼睛红红的站在院子里,身边还有王安国家的女儿陪着,她的夫婿就是当初与韩冈分列第九第十的叶涛。

看见丈夫脸上带着些讶异,王旖解释道:“官人走后,是姑姑催了奴家过来,说家里没什么事,而这边事急,要奴家安心的在这里多留几日。”

韩冈轻叹一声,点点头,这个时候做妹妹应该来的。王旖是直接坐车过来,自己去宣德门饶了一趟,则是耽搁了不少时间,慢上一步也不奇怪。

也不与王旖多说,韩冈直接进屋。王安石夫妇都在外屋坐着,王安国、王安上等王家的亲戚都在。王安石腰背佝偻,显得老态龙钟,而吴氏拿着手绢擦着眼睛,身旁还有了两名妇人在低声劝慰着。

韩冈和王旁的到来,让厅中瞬间静了下来。韩冈两步跨上前,拜倒行礼:“小婿拜见岳父、岳母。”

看见了女婿,王安石凄苦的脸上,勉强挤出两份笑容,“玉昆,这半年你在广西可是辛苦了。将千五之军,败十万之敌,俘斩万余的大功,立国以来,更是从未一见。”

“不敢,此功得来侥幸。”韩冈转头看了一下通往里间的小门,问道:“不知元泽现在如何……”

韩冈只是这一问,吴氏就又立刻用手绢捂着眼睛,哭了出来。旁边不知是哪一家的女眷,连忙将她搀扶了起来。

王安石看着老妻被扶着进了偏厢,不生悲怆的叹了口气,对韩冈道:“玉昆你进去探视一下吧,大哥儿一向与你交好,最后也要见上一面才是。”

掀开帐帘,韩冈往里屋走了进去。就在房内的一众女眷忙避让到一边,只有萧氏抱着儿子在旁抹着眼泪。

“玉昆你来了!”见到韩冈进来,首先出声的竟是躺在床上的王雱,这时候的他精神却好了不少,声音也是响亮的很,“愚兄这幅模样,不能下来与你见礼了,还望勿怪!”

王雱的脸上此时泛着红润的光泽,只是早就瘦脱了形,高高.凸起的颧骨在陷下去的双颊上留下深深的阴影,眼睛都是。韩冈没想到才半年的时间憔悴成了这副样子。哪有半分当年韩冈与其初见时意气风发的模样,只是气度依然不减当年,言辞也依旧洒脱。

韩冈心中黯然,王雱现在明显就是回光返照的样子,已经只有最后的短暂时光了。他走到床边,就在一张方凳上做下,勉强笑道:“你我兄弟,何须在意这等俗礼。”

“说得也是。”王雱呵呵笑着:“玉昆你若是回京再迟一点,我们兄弟可就见不到了。”

“这话怎么说的。”韩冈摇头道,“元泽今日气色不差,安心调养,想必很快就能康复了。”

“玉昆你这话说得就不实诚了。你我皆非凡俗之辈,何必说这些虚言。”王雱神情中有着看破一切的平静,“愚兄这身子是不成了,也就是这一两天的事。”

听见王雱这么一说,萧氏在旁就抱着儿子,低声呜咽了起来。

韩冈一听之下,鼻中也免不了有些酸涩。

王雱哈哈一笑:“人事有终始之序,有死生之变,此物理之常也。存没皆是常事,何必做小儿女态。”

韩冈知道王安石父子皆习《老子》,王安石的《老子注》韩冈拜读过,王雱本人在《道德经》上同样是钻研精深。旧时与韩冈辩经,王雱曾拿着《道德经》上的文字来做论据。以儒家思想来诠释道家章句,韩冈没少摇头。只是眼下到了生死之际,王雱依然故往,而韩冈已经没了争辩的心思。。

“不失其所者久,死而不亡者寿。”王雱仰靠着背后的靠垫,偏着头,眍下去的双眼幽暗,紧盯着韩冈,“经传新义一事,乃是愚兄必生所学。愚兄虽然寿数止于今日,若三经新义得以长行于世,虽死如生,不为夭也。”

韩冈沉默下去。他很清楚王雱在说什么。这个时候,就算是骗也是可以的。只是说些好听的话很容易,但韩冈说不出口。就是因为在垂死的王雱面前,他才不能出言欺骗。

房中静了下来,只有萧氏时有时无、压得低低的抽泣。

盯着韩冈不知多久,王雱终于移开视线。“大道难易。也怪不得玉昆你,只是现在怎么不说两句,宽慰一下愚兄?”

韩冈依旧沉默。王雱摇头苦笑了几声:“要是玉昆你在根本大义上会虚言伪饰,却也不会有今日的成就了。”歇了好一阵,才又开口,“不过新法诸条,玉昆于其中出力良多……”

“新法推行有年,功效已显。就算其中有错处,也可以在施行的过程中逐渐改正。虽说是摸着石头过河,但只要一步步走稳一点,富国强兵的好处只会一年更胜一年。”

听到韩冈的回答,王雱微微颔首,轻轻阖上了眼皮。说了这么些话,他也有些累了,萧氏过来帮着他整理好了盖在身上被褥。韩冈起身静静的离开了房间。

半夜的时候,宫中来了使臣。蓝元震这一次来,不是为了给王雱送汤药,而是带着一份圣旨。其中备赞王雱参赞三经新义的编纂,将他刚刚晋升为天章阁侍制不久的文学职名,进一步晋升为天章阁直学士。

在女婿成为直学士之后,连儿子也成了直学士。与王安石一家来说这是难得的荣耀,是天子的恩赐。只是这一项任命,没有带来多少欢喜。虽算是冲喜的手段,以王雱眼下的情况,甚至连起床谢恩都不可能了。

到了四更天,韩冈和王旖被安排在休息下来。王安石和吴氏如今心力交瘁,家中的事务都交托给了弟弟王安国夫妇帮忙打理。

王安国夫妇指挥着家人忙里忙外,韩冈扶着挺着肚子的王旖在床上躺下来。

王旖的一对剪水双瞳没有了往日的神采,抓着韩冈的手臂,轻声问道:“大哥当真好不了了?”

韩冈摇了摇头,嘴角扯动了一下,温声道:“好好歇息吧,这些天应当是累着了吧?”

王旖闭上了双眼,莹润的脸颊贴着韩冈的手,低声说着,:“官人回来就好了。”

