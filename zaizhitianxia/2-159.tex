\section{第29章 顿尘回首望天阙(11)}

韩冈拍着桌子大骂了两句声音就低了下去,他本是到教坊司这里等好消息的,却没想到收到苏东坡这样操蛋的回复。

韩冈实在难以相信写下这份判词的会是名传千古的苏东坡,但这份文采却是谁也学不来的。一个女子的命运,在苏轼眼中,竟然是他展露文学才华的工具。还有蔡确,竟然出尔反尔,这一桩,韩冈也是记下了。

不过韩冈也清楚,苏轼也许并不知道周南跟自己的关系,否则应该不会干出这等自损名声,而亲附宗室的蠢事。如果他能知道周南突然申请脱籍的原因,他的判决当是会有不同的结果。

但现在说什么都迟了。

即便周南再次申请脱籍,除非开封府接手此事的官员明着要跟苏轼过不去,否则都会转发给大苏,让他自己去擦屁股。而这判状,苏轼他自己都不便改动,不然处事不谨、行事反复的罪名就要落到了他的头上——他的政敌不会放过他。

坐在周南闺房外间的韩冈,无奈的叹了口气。本来是想有两条路可走,没想到蔡确言而无信,让苏轼从中横插了一杠子,变成了现在这副田地。若是只靠自己,事情不会落到这个地步。

幸好这一事,已经传遍了东京城,让韩冈因此多了许多手段。他仰起头,放肆的笑着,还是那句老话,“我只怕事情闹不大!”

此事还是有挽回的余地,他也无心再拖下去韩冈,就要起身告辞。

“姐姐!”内屋中突然传来墨文的惊叫,“官人,你快进来!姐姐要划自己的脸!”

韩冈闻声脸色顿变,连忙冲进内屋。就看到周南拿着一把剪刀要往自己脸上划去,而墨文正拼命拉着她的胳膊,不让她毁了自己的绝世容色。

韩冈箭步上前,一把夺过周南手上的剪刀。白皙如玉的脸颊上,已经有了一点米粒大小的血珠。夺下剪刀,韩冈惊魂未定,怒道:“南娘,你这是做什么!?”

周南坐在床沿,方才的一番挣扎,让她的满头青丝全都披散了下来。肩膀瑟瑟缩起,脆弱得一碰就坏。空洞的双瞳中毫无神采,仿佛失去了灵魂。声音也是毫无起伏,有种不祥的平静:“苏推官不肯放人,全都是因为我这张脸。若是毁了这相貌,他怎么还会再强留着我?!”

“这倒是好办法……”韩冈微冷的话声,让周南身子一颤。墨文也惊得跳起来,惊叫道:“官人!”

韩冈却是安安定定的继续说下去:“但这事你得先与我商量才是。你我虽无媒妁之言,但已有三生之约。你人都是我的,想自伤,也得先问过我,让我这做官人的点头吧?”

韩冈说得霸道,周南勉力笑了一笑,笑容中掩不住酸楚和绝望。无暇如玉的俏脸上写满悲伤,却反添了她一分脱离尘世的美态。

“不用担心。”韩冈亲昵的捏了捏周南细白如凝乳的脸颊,充满自信的笑着,“相信你家官人好了。男主外、女主内。外面的事,还是交给我来处理。”

韩冈话声中的坚定,给周南惶恐的心中平添了几分安全感,她仰头望着韩冈坚毅的双眼,泪眼汪汪的呢喃问道:“官人?”

“放心吧!”韩冈回了周南一个更加自信的笑容,站起身,“这两天就让你风风光光的离开这个鬼地方。”

韩冈转身而去,宽厚而坚定的背影,让周南眼神迷离起来,一时忘记了悲伤。

………………

周南脱籍的这一桩公案,事关皇家,又跟一位薄有微名的士子脱不了关系,加之还有让人痛心的结果,整个一个说书人口中的传奇,是个绝好的八卦话题。才一天的功夫,就传遍东京内外城中,大大小小的酒楼茶社、衙门官邸,都能有人在说这桩新闻。自然,其中不值苏轼所为的为数众多,正好跟因为矫矫不群而得到士林赞许的蔡确成了鲜明对比。

连曾布也不能免俗,在王安石这里说起了此事。

“苏子瞻也是糊涂了,看这事闹得……”

曾布惋惜的声调中排满了幸灾乐祸。主管新法施行的司农寺,在年前的时候变得比较轻松,只有到了明天二月,将兵法开始施行,而免役法在全国范围内推广,到那时,才会重新忙碌起来。所以,这一天的午后,才有在汇报工作之余,与王安石聊起天来的闲空。

“不过他现在当是后悔了,没问明内情便乱下判词。苏子瞻的名声,从此以后怕是在风月场中就是有些不好听了。”

王安石沉稳得很,没有曾布那等露骨的幸灾乐祸。只是时不时的点点头,算是对曾布的回应。

没办法,谁让曾布前些时候在跟苏轼廷辩的时候,吃了一个闷亏。要不是天子拉偏架,王安石又拿出宰相的身份压人,说不得就会灰头土脸的败下阵来。。

论口才,能跟苏轼一较高下的,巡遍朝中也没几人。吕惠卿能算一个,他曾经在朝堂上把司马光驳得说不出话来,也曾拿着韩琦的奏章一条条批驳回去,正所谓‘面折马光于讲筵,廷辩韩琦之奏疏’,但吕惠卿已经回乡守制,两年之内都不可能出现在东京城中。

章惇勉强也能算一个,堵得文彦博气急难耐的情况也有过。但他和苏轼两人交情深厚,即便政见不同,可在公事上的分歧,倒也不会闹到面红耳赤的地步。

而曾布的口才就差得远了,他本就不是以舌辨著称,遇上了苏轼,就只有被其肆意欺凌的份。心里一口气,堵了几个月了,一直堵到了现在。

“苏子瞻这判词一下,其实是把雍王推到了风尖浪口。人人都道是他得了雍王的授意。现在都有人说他附会亲王,德行堪忧。”曾布眼中闪烁着喜色。原本对韩冈很有些看法的他,现在倒是想请韩冈好好喝上一顿。

王安石终于叹了口气,曾布的心情他也能理解,是给苏轼欺负惨了,但总说这些话,也有失大臣体面。

“‘士大夫捐亲戚,弃坟墓,以从宦于四方者,宣力之余,亦欲取乐,此人之至情也。若凋弊太甚,厨传萧然,则似危邦之陋风,恐非太平之盛观。【注1】’还记得这一段吗?”王安石忽然问起曾布。

曾布皱眉想了想,反问道:“是苏子瞻前日反对免役法的奏疏中的一段?!”

王安石点了点头。那段话就是苏轼的本心。

士大夫离乡出来做官,虽是为了天子出力,但也是为了能因此而取乐,否则何必告别亲戚,远离乡土,出来走遍四方?

如今朝廷废掉差役法,改收免役钱来雇佣百姓来做事。原本在衙门中卖力之余,还要在官员家中做牛做马的免费劳力,现在变成了必须花钱来雇的佣夫。驱用衙前在自家门下做点事没问题,但用公家的钱来雇佣仆役,却是会被弹劾的。

所以当免役法推行后,官员家中的人力就显得捉襟见肘起来,苏轼才会在奏章中抱怨说,官员家中‘凋敝太甚,厨传萧然’,就像危亡小国的情形,不是如今太平盛世该有的景象。

王安石把苏轼的为人看得很透,如今大部分士大夫想法也都是如此。他们所谓的仁,是得由他们高高在上的赐予百姓,并不是视民如伤的感同身受,以己推人。

“不知苏子瞻他现在,是因让一洁身自好的女子无法脱离教坊司而自责,还是因为毁了自己名声而后悔?”

王安石的话犀利透骨,曾布觉得有些尴尬,其实他也是为苏轼的名声大损而幸灾乐祸,却没有去想周南那里的事。

曾布跟随王安石日久,知道他的性格。王安石虽然很欣赏苏轼的文采,但对其放达而不顾于下的言行却是颇有微词。从学术上说,王安石推崇孟子,对‘民’是很看重的,而苏轼以及其父其弟的学术,在王安石等人看来,却是近于纵横苏张一流。

干咳了一声,曾布提议道:“不管怎么说,苏子瞻挡回了周南的脱籍申状。韩玉昆肯定是失望不小。他那里是不是要安抚一下。”

“周南就让她脱籍好了,教坊司不缺她一个。不过现在此事闹得太大,不宜有所动作。过几个月风声小一点再说。”王安石笑了笑,“天子其实也知道这一桩公案,当是有成人之美的想法,届时让韩玉昆自己上表请了天子恩典就是。至于安抚,章子厚会做的,子宣你就别管了。”

“是!”曾布点头应承下来。“对了,”他又向王安石问道,“元泽应该快到了吧?”

说起最得意的长子,王安石的脸上就添了点笑意:“应该就在这几日!”

注1:这一段出自苏轼熙宁四年二月的奏章。因为本书中,免役法已经提前实施,所以这份奏章也便提前出台。

ps:看了下书评区,说俺在前一章抹黑苏东坡。但苏轼阻人赎身脱籍,并非杜撰,而是史实。在他任杭州通判的任上,先有一个老官妓请求脱籍,他的判词是‘九尾野狐,从良任便’,而当杭州教坊中花魁也想趁机赎身的时候,他的判词就是前一章中出现了那段,‘慕周南之化,此意虽可嘉;空冀北之群,所请宜不允’。论人品苏轼并不算差,至少比他的弟弟好,但他是个标准的士大夫,不要指望他能从底层民众的角度去考虑问题。

