\section{第七章 都中久居何日去(三)}

西夏国中,除了兴庆府护卫国主和宫掖的御围内六班和环卫铁骑,加上数万分镇要地的铁鹞子,其余的士兵都是平日为民,战时为兵。号称六十万的大军,其实是由西夏全国十五到六十岁的丁壮组成。一旦兴庆府点集大军枕戈待战,消耗的可都是西夏的国力。

以塞上江南般的兴灵,还有横山北麓的银夏这个两个核心地区的地理条件,养活两三百万人口不成问题。就像熙河路,能适宜居住的地方也就那么几条河谷,但吐蕃蕃部人口总数,林林总总加起来也有近百万。

不过西夏作为一个国家,则没有自给自足的能力。一部分已经职业化的军队,还有官僚、国君,这些人的存在全都是纯粹消耗,没有任何产出。这一点与族长的子嗣几乎都要下地放牧的部族截然不同。而这些多余的消耗和需要,西夏国中无法供给,就只能从宋人那里吸血。

所以自从元昊起兵立国之后,党项人年年挑起大战。就算宋人给了所谓的岁赐,也不足以将逐渐扩张的国家财政支撑起来。随着西夏国家建立日久,逐渐完备起来的官僚体系对钱粮的消耗也越来越大,加上从部族长老转职而来的贵族们的难填欲壑,便只能通过战争和劫掠来博取——因为西夏的经济支柱是大宋。

至于同样在大宋身上吸血的辽国,其本国的国力足以支撑整个国家的运转,每年大宋送上的五十万银绢的岁币,仅仅是用来买通辽国统治阶层。只要能满足契丹贵族们的要求,就可以让他们不动南侵的心思。这就是西夏和辽国的差别所在,也是为什么澶渊之盟得以保全至今,而庆历和议只用了二十年就成了废纸。

不过自从横山、河湟两役之后,加上梁氏要腾出手来整顿国内,两国在三年之中都没有大战,仅有边界的一点小冲突。而且随着陕西、河北、河东推广将兵法用来整顿军队、汰弱留强。加之有了军器监后,接收的装备也日渐精良,三路禁军的实力飞速上升。辽国、西夏受到的压力,越来也大。

辽国强行索要边界土地,就是在示威。而辽国公主下嫁秉常,也是一种应对。

现如今辽国随着他们的要求被满足,一时间已经平静了下去。但西夏这边,主动挑起战事的可能性越来越大。与其等到党项人在契丹暗中的支持下举兵来攻,还不如先行动手,抢占战略优势,将横山控制,自此便可以高枕无忧,等到合适的时机,就可以举兵北向,将兴灵给收服,一举平灭西夏。

种谔就以这个理由说服了赵顼,得以出外担任鄜延路兵马副总管。但让赵顼最终下定决心开战的理由,却是在最近。说起来,还是韩冈给他的原动力,没有韩冈能让北方禁军在三年之内全数铁甲化的保证,赵顼也不敢贸贸然的决定重新挑起战火。

但这个决定,有很多人反对,之前韩冈就反对过,他觉得首要目标该在兰州,王韶与他也是一个心思,应该出战,只是目标不该是横山。而蔡挺,则对此表示支持,看他样子,也是有意争一下领军出战的主帅之位。

至于政事堂,王安石一个人说话就压倒了其他四人的声音——如今的政事堂中有三相两参,难得的满员情况。其中说话有力的就只有王安石一人。韩绛不在意,当初他说话没人理,现在还是没人理,但只要冯京和吕惠卿得意不了那就够了。而

韩冈来拜访王安石时,就听他的岳父又提及此事:“吴冲卿亦曾有言,秉常年岁渐长,归政只在眼前,可以稍待时日,坐看西夏内乱。”

前两天在朝堂上的争论,早传入了韩冈的耳朵里,吴充如何被驳倒的,他也知道一二。

王雱就在旁边,从鼻子里发出一声冷哼:“秉常娶了辽国的公主,日后若是要清除梁氏,必然是要借助辽人之力。等下去不是坐看西夏落到辽人手中吗?”

党项人对契丹提防甚重,一直以来都是游走于宋辽之间,同时向两国称臣。可是眼下西夏越来越贴近辽国,说不定等到几年后,不肯归政的梁氏与秉常起了冲突——只要看看史书,甚至回顾一下被前任西夏国主谅祚清除的外戚讹庞家,就可以知道这个结果是必然的——辽人肯定会借机插手进来。

当年元昊叛乱,宋军即便接连惨败,仁宗皇帝都没有为了以防万一,派军去镇守潼关。可若是换了契丹铁骑出入横山,如今的天子赵顼别想再睡好觉了。但如果宋军夺取了横山,就算换了契丹人过来,也要在群山之中撞得头破血流。

所以赵顼的心意才如此坚定,韩冈、王韶都没办法动摇得了。

“陕西六路前日奉旨点算,尚需步人铁甲总计九万六千四百余领,不知玉昆你那边何时能打造完成?”王安石问着韩冈。

韩冈道:“现在每天出产的步卒板甲在三百领上下,专供军校使用的新式明光铠则是十领左右——前两天也给岳父看过了,全身铠有四十八片大小甲叶,比起八片甲叶的板甲打造起来工序更繁,工时更多,不过防护性更好,用上了铆钉,也更容易活动。”

明光铠只是借个名字而已。本质还是板甲,只是用了更多的零件,做到了更好的防护性,也打造得更为精致、闪亮——所以沿用了明光铠的名字——但重量也随之上了一个台阶。

王安石满意的点了点头,这个速度已经足以让人瞠目结舌了。换在过去,这是将整个军器监十天的产量。十倍的速度,五分之一的造价,略胜过往的防御力,能在数年内给近六十万禁军全数换装。不仅让赵顼心中多了勇气,也让王安石平添了对抗西北二虏的信心。

有时他都在想,若是韩冈提早一点将板甲拿出来,说不定去年辽人索要河东土地,天子也不会无奈的说什么‘姑从其欲’。

只听韩冈继续道:“当汴河边的水力锻造工坊落成之后,京城内外两座工坊加起来,每天产出的步卒板甲能达到五百领,如果改进一些工序的话,八百领也是可能的。”说到这里,他眉心皱了起来,“唯一担心的就是生铁供给不足。徐州利国监的生铁产量,很难赶得上。”

“是不是要利国监加大产量?”王雱道。

韩冈点头:“这是必须的,日后也能给朝廷增加一部分收入。”

在全军换装之后,对甲胄的需求就会降低大半,但韩冈为了保证工匠们的工作,打算让军器监打造民用的铁器。这样一来,对生铁需求照样不会减少。这一件事,王安石和王雱都知道。

王安石双眉同样微微皱起:“我知道玉昆你发明轨道,就是为了能让矿山中运送矿石更为方便。用马车代替了人力运矿石出坑,的确能开采出更多的矿石。但炼铁的木炭就不行了,哪里来的那么多的木炭。”

王雱叹了口气:“若不是石炭炼出的铁质不佳,就可以直接用河北的生铁了。”

“最近军器监也有在实验,”韩冈说道,“如何用用石炭来炼出好铁。”

“不是炼铁五行缺木,必须用木炭竹炭吗?”王雱奇怪问道。

“但木炭、竹炭总比石炭要贵。尤其是徐州,附近的树木差不多快砍光了,要从登州运木炭过来。扩大产量之后,成本只会越来越高。而一旦改用石炭成功,就可以用上河北的便宜生铁了。另外日后若是能在徐州附近寻找石炭,铁价只会更为便宜。”

“玉昆,你那个实验还要多久才能成功?”王安石听得怦然心动。

王安石好言利。当年推行农田水利法的时候,外面还有笑话流传说,有人建议甚至向他填平梁山泊,即可得八百里良田。虽然这是无稽之谈——梁山泊是东京连通京东东路的的转运通道,五丈河、济水、汶水的水运都要从这里经过,每年几十万石的粮食要从此经过,怎么也不会有人打梁山泊的主意——但王安石对为朝廷省钱、挣钱的迫切,却是没有半点虚假。

韩冈沉吟了一下,摇头苦笑。焦炭的事还不一定能成功,就算有了焦炭,炼铁高炉也需要时间,韩冈无法确定是否成功,不会说出来给自己找不痛快:“恐怕要不少时间,不可能即刻建功……只能慢慢来好了。”

“只怕利用石炭降铁价,也是为了日后的铁船?”王雱笑问道。

“算是吧,只是没有十几年也见不到成品。”韩冈笑了笑,“不过往铁船去的每一步都能见到功效,倒也不至于会半途而退。”

“也就是说,玉昆你暂时无意出外?”

王安石的问话听着有些奇怪,韩冈皱眉一想,心中就有了数,笑道:“是哪一位点了小婿的将?是不是种谔?要小婿做什么?随军转运吗?”

王安石脸上浮出的浅笑蕴意颇深:“玉昆你可知有多少西军将校,听说要开战后,就想要你去管着他们的粮秣?”

韩冈怔了半刻,最后化作一声笑叹:“幸好不是帅位。”顿了一下,又道:“幸好只是粮秣。”

