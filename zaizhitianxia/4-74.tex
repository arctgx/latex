\section{第15章 焰上云霄思逐寇(五)}

地面在颤动,骏马在奔驰。

虽然只是寥寥数十骑的交锋,依然有着血染沙场的壮烈。

掌中的铁鞭挟着奔势从空中斜斜一挥而下,抢先一步击中了对手持刀的肩膀,顺势就将他砸下马来。

生死只在一瞬间,冲锋时就屏住了气,当击败对手后,重新开始的剧烈呼吸里就带着淡淡铁锈的味道。身后同伴沉重的马蹄在落马骑手的胸口踏过,清晰的骨裂声随着惨嘶传入韩廉的耳中。

“第三个!”韩廉随即一声狂吼,让围过来的敌骑为之胆寒。

双手紧紧握着铁鞭,韩廉鹰隼般的双眼重新盯上了一名敌军。一夹胯下马腹,立刻如箭般直冲而去。下马时韩廉只是一个腿骨被摔断后没能长合好的瘸子,但当他跨上马背之后,就成为一名军中第一流的骑兵。

刚刚围拢起来的交趾骑兵,在韩廉猛如恶鬼的冲锋中,如同赶鸭子一样被赶散。韩廉和他的同伴死死咬住一开始盯住的那一人,如同荒原上追逐野兔的群狼,前后交替着追击,互相配合着将速度同样不慢的猎物给捕捉到手。

依靠胯下河西马身高腿长的优势,韩廉从身后渐渐追近猎物。在逃敌回头时惊骇的眼神中,他又是一鞭挥下,连着头盔带着头骨一起砸得粉碎。敌人最后的慌乱,凝固在眼球上,被一阵猛力从眼眶中挤了出来。

“第四个!”韩廉回头大吼,“刘三,赢你们两个了。”

可就在同时,稍远一点的地方,也传来了另外一道吼声:“第三个!”接着吼声转为一阵畅快的大笑,“殿侍,你还是只多一个。”

韩廉大骂了一声,调转马头,又要往回攻过去。

“殿侍,刘三哥,你们高抬贵手,留几个给俺们啊。”更远处响起一声叫喊,“你们可都是稳当当的能进三班院了,俺们也想弄个军将、大将的俸禄养家。”

“没出息的东西!上山打猎哪还有让手的道理,再卖点气力,来抢就是了!”

韩廉回头又是一声吼叫。他在回到军中成为斥候的同时,已经被韩冈提拔为不入流品的殿侍。以他这些日子前前后后出的力气,只要将交趾军逼退之后,稳稳的就能进入品官的行列。但他统领一队骑兵,要是杀敌比下面的人少了,岂不丢人现眼,半分也不肯相让。

本来说着也是在开玩笑,但一想起战后封赏,则是人人都用心起来,争先恐后的杀过去。

在韩廉看来,他今天所面对的交趾骑兵比起昨日要聪明了许多,至少不会在傻乎乎的冲击箭阵,而是开始做他们应该做的事。

不过散布在归仁铺周边一片旷野上的交趾骑兵,韩廉一路数过来,就只有五六十骑上下。可见昨天的大败加上来回奔波,还是对交趾骑兵有着很大的影响,让幸存下来的大部分敌骑一时无法再上阵。

尽管经历了昨日的战败,可卷土重来的交趾骑兵的战意,依然保持着一定的水准。只是战意并不能直接转化为战力,他们的马术也就比笑话强上那么一点,基本上还是个笑话。在关西阵上与党项骑兵厮杀过的一众骑手,挥舞着沉重的铁鞭,毫不客气的收割着战果。

奋力拼杀仍不见有所收获,在宋军骑兵远远强出许多的武勇、战术和战马面前,再拼命也依然只是在给宋人增光添彩。这一队交趾骑兵,终于坚持不住,放弃了对归仁铺的监视,向后撤了回去。

看着剩下的交趾骑兵逃远,韩廉从身下的河西良驹背上跳下来,骑上了一匹体格要小上一圈的矮马。节省马力时时刻刻都要注意。虽然河西马只是接敌时骑乘,但从荆南移动至广西,有三分之一的战马因为水土不服而生病,倒是人还好些。

驱逐对方斥候游骑告一段落,骑兵之间的交锋以宋军的胜利而告终。留了一队继续扫荡归仁铺周边的原野,韩廉带着方才的战果返身回营。

只用了一天的时间,宋军已经以归仁铺为核心,修起了一处形制简陋的营地。不过虽说简陋,也是与西军行军作战时设立的营寨相比,其内外布置一切还是按照标准的立营法而来,保护营中驻军的安全,防御力并不差,而且还在不断的加筑中。

韩廉正要进营门的时候,正看见有一队人赶了六七十匹背着辎重的矮马入营。

粮草当是从昆仑关转运过来,“马是哪里来的?”韩廉问道。

押送粮草的小官连忙回答,“都是赵知州连夜搜罗起来送到昆仑管的。”

韩廉看着这些驮马,心中很是欢喜,宾州城中能搜罗到这么多马匹,也算是运气了。尽管肩高最高的也不过四尺出头,可韩廉也不指望能骑这些马上阵,只要能用来作为巡逻时的脚力就足够了。真正厮杀的时候,再换马就行了。

韩冈这是也收到了宾州城搜罗一批马匹,作为运送粮草的驮马的消息。他听说之后,就连忙出了营帐。这些马一匹匹都是矮小结实,外型上与韩冈见惯的河西马和青唐马有很大区别。

“想不到广西还有产马?”不对,韩冈立刻反应过来:“都是滇马吧?”他问着苏子元。

“没错。广西的马多半是‘滇池驹’,如今世称大理马。”

滇马,后世因为茶马古道而闻名,善走山路,饶有耐力,而且耐粗饲。韩冈看这些马匹矮小的体格,作为战马肯定是不合格的。但用来在山地中驮运货物却是一等一的优良马种。

“放一半回去继续转运粮草,剩下来的留给骑兵做替换,这里也打不了几天。”韩冈围着这些滇马绕了几圈,这些马被一群陌生人围着,一点也不见受惊吓,很温顺的站着,让识马的韩冈、李信都满意的直点头,“日后打进交趾,若多攒下些滇马来运粮,倒是方便了。”

“运使说得正是。”

韩冈冲着南方指了一指:“交趾人的战马似乎也是滇马?”

这件事,苏子元倒是不太清楚了,“马不耐湿热,交趾的气候当也不能养马,多半就是从大理来的。”

何缮在旁边小心的插话道,“不管是交趾还是广源,军中所用马匹,皆是从大理贩来。只是道路险阻,加上价格腾贵,所以两家的战马数量都很少。”

“原来如此。”李信点点头。

骑乘上阵的战马,从体格、到耐力、再到脾性,每一条都要进行考核。十匹马中间,差不多也有一两匹能充作战马。所以战马的价格往往是普通马匹的十倍。昨日的战斗中,交趾骑兵的坐骑,连死带伤损失了差不多百匹左右。这一下子可就是近万贯大钱不翼而飞,想来李常杰得知后,恐怕都要哭出来了。

“广南西路这边贩马的是走哪条路?”韩冈转头又问着苏子元。

“邕州、宜州都有路通大理,不过要分别经过自杞和罗殿两部转运。没有直接道大理国的道路。”苏子元道,“川中可通大理,不过川中自产马匹,所以不多见。”

“自杞?罗殿?”韩冈对这两个地名很陌生。

“是两家西邻大理国的蛮部。罗殿在夔州路南,广西西北。其国传世久长,据说是诸葛武侯征伐南蛮的时候,就封了罗殿王,至今已经传了几十代。自杞则在罗殿南面,溯右江而上至横山寨,再往西北行十三程便至其境;自象州顺着都泥江上溯,也能抵达自杞。”

听苏子元这么一解释,韩冈差不多有点数了。大理就是云南,西邻大理,便是在云南的东侧。自杞不用说了,应该是滇东山区中的部族,所谓的都泥江,既然在象州汇入珠江,基本上能确定就是后世的南盘江,其上游也就在云南。至于罗殿,前面苏子元说其地在自杞——也就是滇东——以北,那当是贵州西南,大略在安顺市那一片。

不过将地理做古今对照,也只证明韩冈的记忆力还算不错。唯一可以确定的是,南方的军马这下有着落了。

“广西现在没有马市?”

“没有!”苏子元摇头,神色一下又变得伤感起来,“家严旧日曾有意开辟马市,遣人去横山寨查探。本准备一切就绪后,便上书朝廷,可惜交趾贼军来袭……”

韩冈这下明白了,难怪苏子元对马事这般清楚,原来苏缄早就有所准备。

韩冈叹了一口气,“这些事就放一边吧。”知道李常杰不可能弄到多少马匹,他就放下心来,“日后有的是时间去找滇马来补充军中需用。”

夸奖过韩廉,将斩获的首级挂在营中显眼之处,到了午后,一名名游骑带着最新的军情从营外赶回,韩冈终于要面对侵略者的头目。

从邕州到归仁铺,走了半天还多的交趾军,前军后军连接紧密,侧翼都有防止伏击偷袭的千人队,一时之间根本找不到可供利用的破绽。

“李常杰好歹也算是名将。”韩冈立足于营门之外的一处略高的小丘上,眺望着远方,“虽然是交趾的。”

“只需两千兵马,便能直取本阵。”李信将黄金满的兵丢到一边,不服气的说着。

所谓找不到破绽只是因为兵力不足,若是手上足够的军力,完全可以直接出手碾压,或是在试探性的攻击中逼出破绽来,可韩冈手上只有八百。

交趾陆续抵达归仁铺外,离着韩冈的大营五里的距离扎下营盘。

眼见着暮色渐深,交趾军的营垒终于初见雏形。一片片营帐整齐有序,只是外围看起来还并不算坚固,李信问着韩冈,“要不要夜袭?”

