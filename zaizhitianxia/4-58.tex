\section{第12章 兵蹙何能祓鬼傩(中)}

难得一见的冬雨润湿了邕州城外的土地。

宗亶只用靴子的脚后跟在地上踩了两下,就刨出了一个坑。

‘今天攻不了城了。’从脚底传来软烂的感觉,就像踩着刚刚死掉的尸体。烂泥还黏着鞋跟,抬起脚都有些吃力。望着远处的邕州城墙,从前营到城下的半里路,只会比自己脚下的情况更糟。

在中军大帐外,冰冷的冬雨落在头上脸上,冷冰冰的直往脖子里淌。就算是早已经习惯了潮湿,也不可能顶着冰冷的雨水、踩着满脚的烂泥去攻城。而且油火水泼不灭,用云梯车和攻濠洞子照样还会被烧掉。雨水对攻城只见坏处,不见好处。

风向变了,一股子恶臭随风传来,冲得头脑一阵发晕。宗亶揉了揉鼻子,腐烂的味道本来都已经习以为常,感觉像是不存在了。可今天雨水落下之后,却不知怎么的,鼻子突然又恢复了正常,能闻到臭味了。

他去看过处理尸首的地方,没有足够的柴草,烧都来不及烧,全都堆在一处,堆积如山。过去视察的时候,不过停留了片刻,砰砰的闷响声却是一声接着一声。宗亶不知见识过多少死尸,知道那是腐烂的尸体肚子爆开来的声音。

‘营门外挂着的十几个逃兵,肚子也该爆了。’宗亶记得他今天早上在进中军大营时,肚子高高的胀起,就像怀了孕的样子,肚皮仿佛透明,布满青紫色的纹路。全身也都涨了起来,泛着扭曲的青绿色。记得昨天尸身的变化还看不到,只是一夜之间他们身上的衣服不见了,‘应该不会是有人要的。’宗亶想着,都已经给军法的鞭子抽成了碎布条。

围城超过四十天了,军中伤亡惨重。逃兵渐渐多,杀了几个挂在营寨寨墙上,但当天夜里,又出了几十个逃兵,大部分都捉了来,当众用重锤敲断了脊椎骨,但还是跑了几个。

邕州城下的战事惨烈超乎所有人的想象。宗亶回想起自己几十年的征战,大越从没有过在一座城池下损失如此之多。当年跟着太宗【李佛玛】攻下占城王都佛誓城,俘获占城王乍斗,伤亡都远远不及这一次。如果在一个半月之前,能想到此时进退不得,他肯定会尽全力劝谏李常杰撤军回国。

“妄言撤围,动摇军心者……斩!”主帅李常杰用力挥去了佩刀上沾染的血迹,用刀尖指着伏在地上的裨将,从喉间伤口中喷出来的血,转眼就给雨水冲淡了。

几天来李常杰已经杖责了好几位建言退军的将校,这一次终于杀了人。

几名蛮帅都紧抿着嘴,这是杀给他们看的。宗亶的脸上则看不出任何表情,‘杀人再多也无用,还是多想想怎么破城再说。’李常杰明摆着快要疯了,没必要这时候跟他为敌。

在国中一力主战的就是李常杰;坚持要攻下邕州的也是李常杰。如果不能将邕州城夺下来,损了他在军中的根基。他凌逼太后殉先帝,将顾命太师发遣出外的事,原本视而不见的人们,眼睛和嘴巴都会恢复正常。

国中还有十几个太子,都是圣宗【李日尊】的弟弟。而现在当政的是毫无根基、也无外戚匡助的孤儿寡母,若是李常杰犯了大错,哪一个都不会放过这么好的机会。

李常杰按着佩刀,瞪着麾下将校,看看还有谁敢来再来试一试他手中的军法。

如果没有围攻邕州,或是打下就撤离,也同样是一场辉煌的胜利。只可惜现在骑虎难下,损失如此惨重,不攻下邕州,军中的怨气就难以消除。他以武功建立起来的威信,就不能维持。

李常杰少年时就因为武勇和相貌从上上代国主李佛玛的,后来在先王李日尊,御弟。几十年来的战功,成就了如今权倾当朝的辅国太尉,如果能攻破邕州城,用战功加强自己的地位,用其中的财物堵上贵胄们的嘴,他们就会对太后的死从此绝口不提。

尸体抬下去了,李常杰下了‘今天暂歇一日’的命令,众将匆匆散去。宗亶也没留多久,说了几句,也就走了。李常杰回到帐中,在交椅上坐了下来,没有考虑多久,就下令道:“请徐秀才来。”

城池攻防是宋人的特长。云梯车、攻濠洞子都是宋人献上,当时在李常杰看来,已经可以轻松攻下邕州。哪里想到只用了几桶油就轻轻松松的烧了个干净。

羞刀难入鞘,李常杰不能选择退兵。但利用权威压制反对的声音,不可能压制太久,如果再攻不下邕州,不是他坚持不下去,就是下面的人自己闹起来。宗亶离开时的眼神,李常杰看得清清楚楚。权衡两边利弊,他只能选择向徐百祥求教。

前几天看到云梯车在邕州城下变成了火炬,徐百祥他知道李常杰肯定要来找自己。

交趾人从来没有攻打坚固城垒的经验,南方的大城也就升龙府一座。没有足够的经验,怎么可能知道该怎么攻城守城?世间流传的兵书中,具体到交兵细节的,可是一本都难找。

前来传唤他的士兵,脑门上刺了‘天子兵’三个字。徐百祥对交趾兵制稍有了解,这是交趾国中以御龙、武胜、神电、捧圣为军额的上殿班直。

保护宫廷的班直出来做大将的护卫,这不是犯忌讳的问题,而是李常杰怎么敢于使唤他们?如果联系起一些让交趾先王头上发绿的一些传言,李常杰在交趾国中的势力广布,看来并非虚传。

徐百祥被养在大营后方的一顶小帐中,几十天来甚至不能走出十步之外。再一次看见李常杰,劳心劳力的憔悴样儿,让徐百祥看得心情大为舒畅。

‘早一点来求自己,就不至于现在这副模样。’徐百祥在李常杰面前拜倒,“百祥拜见太尉。”

李常杰忙扶起徐百祥,“月来常杰困于军务,不敢打扰先生的清净。不过今日天降甘霖,不得攻城,难得得空,故而来请先生一叙。”

前倨后恭,徐百祥感叹不已,而李常杰乱咬文嚼字,更是让人笑。顺势站起身,在下首落座。

说了几句不咸不淡的闲话,李常杰终于等到徐百祥开口:“太尉围攻邕州月余,想必不久就能破城了吧?”

“王师吊民伐罪,但邕州愚顽拮抗不已。如今王师顿兵城下,不知先生可有以教我?”李常杰忍住要杀人的冲动,低声下气的请教着。

“如何破城,百祥的确有个主意。只是不是什么良策,所以之前不敢献于太尉。”

徐百祥就是想要看着李常杰在邕州城下碰得头破血流,反过来求自己。富贵险中求,就算要冒点风险他也愿意。如果交趾人没有吃什么亏,就轻松的攻入邕州城,谁会把他的功劳放在心上?之前钦州、廉州也一样破得很轻松。所以需要一个对比。徐百祥要做交趾的张元吴昊,可不是随随便便几十贯就打发的士兵。

李常杰向前凑近了:“先生究竟有何良策?若当真能一举破城,我堂堂大越,千里之国,又岂吝封侯之赏。”

“很简单,就是囊土攻城。”徐百祥不在卖关子,“只要太尉下令,让军中士卒,都用衣服包上一包土,趁夜送到城下。太尉麾下有十万大军,一人一包土,堆上城头乃是轻而易举。堆在城下的土,烧不掉、推不倒。只是冲城时,要顶着城头上的弓弩,损伤当不在少数,所以之前不敢妄加建言。如今说出来,就是看太尉愿不愿意用了。”

‘什么不敢妄加建言?是为了奇货可居吧!’李常杰心中大恨。却拍着大腿高声叫绝:“先生果然是妙策!。这两日正好下雨,城头上弓弩难以施用。如果趁夜垒土成山,那就更容易了。”

虽然恨着徐百祥囤积居奇的行为,但李常杰也是知道这是个绝妙的策略。‘为什么没有想到这么简单的一个主意。’李常杰暗恨自己的疏忽。若是之前早早的想到,哪里会损失这么多将士。

徐百祥的策略完全是仗着交趾军兵力人数上的优势,要硬吃城头上箭矢渐渐不足的守军。就算是苏缄,也只能望而兴叹。一点点的堆土成山,看着虽是愚蠢,但在优势的军力上,却是再合用不过的策略。

“击鼓!聚将!”已得良策,李常杰当然就要实行。他已经在邕州城下待得够久了,一天也不想多耽搁。接到命令的亲兵立刻奉命飞奔了出去。

鼓声响了起来,一通鼓、两通鼓,三通鼓,聚将的鼓点连响了三遍。

帐外的脚步声、马蹄声,一阵阵的由远至近纷至沓来。帐帘被掀开,亲卫在门外高声报着应招而来的将领的姓名,一名名将佐走了进来。

徐百祥这时站在李常杰的身侧,入帐后的交趾将领们惊讶的眼神,让他很是得意。

待到最后一名将领赶到,李常杰站起身来,“本帅新得方略,只要尔等皆听我号令。三日之内,必破邕州城!”

