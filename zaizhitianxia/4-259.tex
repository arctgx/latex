\section{第43章 庙堂垂衣天宇泰(六)}

【第二更。】

襄州知州黄庸,韩冈时常会面,就是最近见得少了。身边跟着一个姓名很熟悉,但人很陌生的中年书生。

普普通通,让人有些失望,眉宇端正平和,不像是身负血海深仇的模样,举手投足,也不似身怀绝技的样子。如果让他拉弓,估计能有五六斗就差不多了,飞檐走壁更不用想。

韩冈将注意力从黄裳身上收回,与黄庸行礼如仪。寒暄了两句之后,韩冈的视线又转回到了黄庸身边人身上,“这位就是令弟?”

黄庸让了半步出来,抬手介绍着自己的堂弟:“舍弟黄裳,表字勉仲。今科福建南剑州的贡举第一。”很有几分自豪。

黄裳上前一步,躬身行礼:“学生黄裳,拜见龙图。”

“原来是勉仲。”韩冈还了一揖,点头微笑,赞道:“能在福建拔贡,已经不比中进士容易。而且还是贡举第一,勉仲想必才学是极好的,今科定能在金榜上高居上游。”

“龙图谬赞,学生愧不敢当。”黄裳黯然一叹,“年过而立,尚不得名登黄榜,蹉跎科场多年,远不如龙图初次入贡便高中进士第九。”

“只是侥幸而已。”韩冈说着都要脸红了。

南剑州军额是延平军,第一次见面,黄庸自我介绍就是出身福建延平。不过想来黄庸也做不到南剑州贡举第一。福建路贡举竞争之激烈,倍于江南,五倍于开封,十倍于陕西,至于韩冈当年参加的秦凤路锁厅试,百倍都不止了。

“记得浦城章状元子平【章衡】,金榜题名时是年过而立;癸巳科郑状元毅夫【郑獬】也是年过而立方高中,大器晚成之辈所在多有,勉仲何须自失。”

世间有所谓‘三十老明经,五十少进士’的说法,但实际上仁宗之后的历科状元及第,多是在二十多岁。科举已经算是很公平的选拔考试,当真才剧器博,有那份能耐,在反应、精力、记忆力、创作力和学习能力都处在最出色阶段的二十多岁,基本上就能中了。最小的王拱辰,年十七,进士第一,状元、探花一起拿了。韩冈举的章衡和郑獬都是三十二岁高中,已经算是岁数很大了。

黄裳当然知道这一点,韩冈的话也只是安慰而已,不过释放出来的善意,黄庸和黄裳都感受到了。

黄裳又躬身一揖:“多谢龙图开解,学生明白。”

介绍过自己的兄弟,黄庸就看向韩冈的身后。

一个三十近四十的中年人恭恭敬敬的站着,很不起眼的相貌,眼尾和眉梢下垂,一幅愁眉苦脸的长相。韩冈迎上来寒暄的时候,他一句话也没说。但能跟着韩冈,自然不会是普通人。自己带了堂兄弟,就不知韩冈带的又是谁?

不待黄庸示意,黄裳先行开口:“龙图身侧当无凡士,不知兄台高姓大名?”

黄裳垂询,李德新便上前拱手:“回秀才,在下姓李,双名德新,延州一布衣。”

‘李德新?’黄裳、黄庸同时眉头一耸,这不就是从伏龙山中传出的那位名医的名号?

在与自己会面的场合中,他出现在韩冈的身边,那么韩冈的打算也就可以确定了。心中一喜,黄庸原本还是很严肃的表情也松弛了下来。

“可是伏龙山中的李神医?!”黄裳貌似惊喜的追问。

李德新连忙摆手,谦虚道:“神医二字在下决不敢当,只是奉命行事。种痘秘术也是龙图所授,德新遵循而已。”

一句话就将底全漏了,黄庸和黄裳惊讶的望向韩冈,韩冈形容不动,抬手相邀:“先进厅中说话。”

客随主便,黄庸、黄裳哪里能有意见。一起进了厅中,分宾主坐下,等府中服役的老兵上来奉了茶汤,耐着性子喝了几口之后,才听到韩冈慢吞吞的开口:“漕司衙门外面的情形,想必常伯兄都看到了。”

黄庸低头:“是黄庸治民不力,致使百姓聚众于漕司衙门之外。”

“哪里能怪到常伯头上。”韩冈笑了一笑,“是韩冈行事不谨之故。”

黄庸放下茶盏,挺腰端坐,正容道:“黄庸有一事,不知当不当问。”

韩冈却双手拢着茶盏,感受着掌心传来的热力:“常伯想要问的,韩冈多半能猜到。是不是想问种痘之术到底有没有效?”

当然不是,黄庸是想问一下种痘之术从何而来,还想问一问韩冈打算怎么解决眼下襄州百姓的问题——有了民意为凭,黄庸自问要从韩冈手上分出一份功劳,那就一点不难了——不过既然韩冈肯说及与种痘之术有关的话题,他也没有意见,拱了拱手,“正是如此,龙图可否为黄庸解惑?”

“有了两千人为证,基本上可以确定是有效了,想必常伯也已经打听明白了。”韩冈微微一笑,倒有几分讥讽的味道。

黄庸神色不动,黄裳开口道:“李神医以一人之力,能在数月之间,便给六村两千人种上痘,想必种痘之术应该不算很难吧。”

韩冈很坦率:“种痘之术其实做起来也简单,只要有痘苗,种痘一点也不难。”

“不知种痘之法从何而来?”黄裳立刻追问道,“是龙图自创……还是来自曾经救助过龙图的那位孙道长?”

“一半一半。种痘法的本源,的确出自于孙师。”韩冈人前人后,对那位虚构的道士都尊之为师,“与筋骨疗伤之术一样,是韩冈卧病在道左废庙时,与孙师闲聊中得来。”

听到韩冈开始讲古,黄裳和黄庸身子都下意识的前倾了少许,专心致志的聆听着。

韩冈双目迷蒙,语调深沉,沉浸在往事之中,“种痘法被孙师称为灭毒种痘法。所谓灭毒就是灭去痘疮中的毒性,使痘疮不至害人性命。要先从得痘疮而病愈的患者身上取下痘浆,这称为生苗。将生苗种到另一个身体健康的人身上,等那一人发病生痘,如果不死,再从他身上取下痘浆,种到第三人的身上,如此循环施为,至少要传过七代,得到的痘苗方为减除毒性的熟苗。痘疮得过一次就不会得第二次,只要依靠熟苗,让人先染上症状轻微的痘疮,就不用再提心吊胆了。”

这样的灭毒种痘法闻所未闻,但听起来很有几分道理。生病而不死,想必身上的痘疮之毒要少于病死者,从他身上取下痘苗当然毒性要小。不断的循环灭毒,最后得到的熟苗,肯定是不会伤人的痘苗。

黄庸连连点头,看着韩冈的眼神都多了几分敬意,能想出如此绝妙的种痘法,想必就是孙思邈孙真人无疑。

但黄裳没有点头,面沉如水:“……若这七次中,有人死了怎么办?”

黄庸闻言一凛,连忙望向韩冈。

“前功尽弃,从头再来。”韩冈冷冽的声音让人不寒而栗。

黄庸手都颤了起来,声音也颤了:“难……难道……”

“可是在交趾?!”黄裳则沉声追问。

韩冈摇摇头,笑了:“常伯、勉仲误会了。如此不德之事,韩冈岂敢用?”他伸出手指比划着,“将成人包括在内,痘疮的死亡率是两成,也就是十人得病,有两人救不回来。但如果只计算幼儿,则高达四成。就按两成算,第一次的生还率是八成,第二次就是六成四,第三次只有五成一,到了第七次,就只剩两成了。如果一开始参加制作痘苗是一百人,到最后就只能剩下二十人,这是杀人还是救人?!”

这番计算如果深思下来肯定是有问题的,但糊弄人是足够了。黄庸和黄裳都紧抿着嘴,无法回答。

韩冈自嘲的苦笑着,“如果说出来,不知会害了多少人。所以纵然怀有种痘之术,韩冈也是一直藏在心中不敢明言。为救人,先杀人,这件事,韩冈也做不出来。”

厅中沉寂了半天,黄裳有些迟疑的开口:“如果能造出痘苗,能造福亿万生民,只是八十人的话……”

韩冈沉下脸:“人命关天,岂在人数多寡!”

黄裳低头道:“是黄裳失言了。”但他又立刻抬头,问道:“但龙图现在用的痘苗又是哪里来的?!”

“从广西。”韩冈又接下去讲他的故事:“身怀灭毒种痘法,韩冈苦思了多年,却始终没有一个合适的解决方法。直到在广西发现一桩奇事之后,才茅塞顿开。”

“什么奇事?”黄家兄弟异口同声。

韩冈不卖关子,“韩冈抵达广西的时候,正是邕州城破,交贼肆虐之时。当时邕州民生凋敝,在交贼退兵之后,为了能尽快将因战乱而撂荒的田地耕种起来,韩冈派人搜集了大量的耕牛。广西牛多,有许多人家一家就养了几十上百头牛,专门用来贩卖,而这些人家家中,却少有人得痘疮——当时因为广西多瘴疠,加上邕州兵灾,百姓流离,韩冈对各种疫症十分在意,却意料之外的发现了这一点。”

“这是为何?”黄庸惊讶的问道。

“因为他们都是接触过牛痘!”韩冈揭开谜底。

