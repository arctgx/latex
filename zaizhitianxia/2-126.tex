\section{第25章 阡陌纵横期膏粱(五)}

“这支铁枪,是当年梁太祖【朱温】帐下大将王彦章王贤明所亲用。王彦章号为王铁枪,有万夫不当之勇。持此枪,他领军力拒后唐庄宗【李存瑁】,若非其败于庙堂奸臣之手,朱梁不至败落如此之速。王彦章惯携双枪上阵,一执在手,一横在鞍,如今一柄枪供奉在其庙中,号为铁枪庙,另一柄便在此处。世间传言,王彦章所用铁枪重达百斤,不过实际上是二十二斤重——已经是很难得了。”

“这把弓,是六十年前曹宝臣【曹玮】在三都谷,大败吐蕃时所亲佩。有其父必有其子,曹宝臣不辱韩王【曹彬】声名,威震关西数十载,党项、吐蕃皆在此弓下俯首帖耳。追想名将声威,确是远在我辈之上。”

“这柄古铁刀,名为大夏龙雀。别看此刀锈迹斑斑,可是十六国的夏国国主赫连勃勃所铸。玉昆你看此刀柄以缠龙为大环,其首类鸟,龙雀之名便因此而来。乃是种仲平【种世衡】当年筑清涧城时掘地所得,当地正是夏国旧疆。不过这柄铁刀出土时无人识得来历,还是靠了刘原甫的博识。刘原甫以博学著称于世,也只有他能一眼看透古董的真伪和时代。”

“至于这支铁杵,乃是家兄旧物。家兄惯使双简,两只铁简加起来超过二十斤,不过当年三川口之役中,家兄却只带了铁杵、枪、马槊三物上阵。用此三支长兵,家兄在敌阵中三进三出,最后西贼还是靠着绊马索才把家兄击败。后家兄遗蜕连同兵器甲胄一起,被西贼送还。甲胄、马槊和铁枪随葬,不过这支铁杵,本帅却留了下来。这支铁杵当年在三川口杀人太多,平日里就是阴气森森,魑魅缠绕。有机会会找个高僧来超度亡魂。”

郭逵现在给韩冈的感觉,就像一个父亲在向邻居炫耀自己聪明的儿子。他近乎自傲的将家中收藏的兵器向韩冈娓娓道来。每一件藏品的背后,都有一段令人热血沸腾的故事,

韩冈今次来秦州,是因为他的的工作中还包括秦凤路的伤病营事务,并不是为了对抗郭逵。郭逵对韩冈的看重,已经世人皆知,韩冈自己一开始对此都有些纳闷。

拥有收藏癖的人韩冈见了不少,前生今世都有。不过由于这个时代有此雅兴的都是有钱有闲的人物,所以他们一般多是集中于古董方面的收集,都跟后世的收藏家同样有着保值的想法。如果仅仅是单纯的兴趣爱好,文人则会去收集字帖、碑拓和金石器物,而武夫则收集上好的兵器甲胄。

在韩冈所知的武将中,刘昌祚对弓弩的喜好最有名气,据说刘家有着数百张各式弓弩,皆是出自名匠之手。王舜臣用着艳羡的语气对韩冈提过不知多少次。而郭逵今天展示出来的收集品,比起刘昌祚的珍藏更强上一筹,让韩冈都为之赞叹,一时之间,甚至忘记了去揣测郭逵此举究竟有何深意。

只不过虽然他没有多想,但韩冈也还是猜个八九不离十。郭逵这是明显的在示好,再联想起莫名其妙在秦州城中散布开的自己要去延州的传言,韩冈怎么都觉得有股子阴谋的味道。

他在秦州待得快活得很,家室、产业、乃至人际关系也都在秦州。要他丢下已经有了规模的关系网,改去人生地不熟的延州,韩冈没有那个兴趣。何况韩绛虽然是座能遮风避雨的大靠山,但这座靠山并不算牢靠。

韩冈一直以来都不看好韩绛的冒险行动,虽然这只是他在军事上力求稳妥的性格得出的结论,但他怎么看,怎么觉得韩绛作为主帅实在不靠谱。文官领军关西,几十年来,冒险的计划全都失败了,而老成持重的策略,却一直延续至今,有着很好的的结果。

一将无能,累死三军。

郭逵很少向人炫耀自己的收藏,在郭忠孝的记忆中恐怕一年也不定有一次。而今天郭逵不但向韩冈展示了自己多年的收藏,还备下水饭再三邀请他留下,直到时近三更,韩冈方才告辞离开。

“韩玉昆文采武略皆有所长,治事之才更是过人一等,日后前途不可限量。”郭忠孝不会妄自菲薄,他虽然对韩冈免不了有些竞争之心,但韩冈的出色表现并没有换来成功的收获,所以郭忠孝不会对韩冈的名声嫉妒如狂,也因此能够正确的看待韩冈的优点和长处。

而郭逵喝着醒酒汤,对韩冈评价越发的高涨起来,“韩冈日后前途也许还不好说,但他在军中的人缘却不用怀疑了,问遍军中,谁人不想自家的营中有个杏林圣手?哪位将帅不盼着有人能把麾下伤病全数救治?”

郭忠孝迟疑了一阵,最后小心翼翼地把这事写上:“……所以大人你肯定韩宣抚会把他调去延州?”

“韩子华现在把关西的钱粮、军器、兵员都往鄜延调集,韩冈之事就算为父不提,种谔那边难道会不说?等过几日,将韩冈调任的文书肯定会来。”

“钱粮皆汇聚一城,辖下战士都是号为精兵,又有韩玉昆在后方安定军心,鄜延路今次当是能大胜而归了。”

郭逵闻言便冷笑,“就像韩稚圭提拔任福任主帅,都以为大军一出,便能马到功成。”郭逵难得的在儿子面前表现出自己对韩琦、韩绛之流的文官的不屑,“你知道他们这种想法叫做什么吗?”

“……什么?”

“一厢情愿!”

……………………

离着冬至已经不到半个月的时间。如今的节庆甚多,春夏秋冬无论哪一个季节都有三五个节日等着。不过除了年节以外,就得数冬至和上元两节最为世人所看重。

冬至一阳生,冬至的到来,代表了世间阴气渐收,阳气转盛,又是一年循环的开始。也因此明堂大典、南郊祭天,这些朝廷中排在头等的礼仪,便都是安排在冬至这一天。

每年冬至之时,纵然穷困潦倒.也会花去一年来积累,又或是向人借贷,在这一天更易新衣,备办饮食,去享祀先祖。亲友之间庆贺往来,一如年节。

这一天,仇一闻正考虑着该怎么让疗养院里的医工、病员们快快活活的过好这个节日,韩冈便出现在他的眼前。

“仇老,久违了。不知近日安好否?”

仇一闻惊得跳了起来:“韩机宜,你什么时候到得秦州?!”

韩冈拉开椅子,自坐了下来:“昨天午后到的,先去见了郭太尉,今天便来疗养院中看一看……上舍病房的事,总要看一眼才能放得下心。”

在秦州,有关疗养院的传言,对事实的扭曲和神话已经很严重了。其实论起照顾病人,疗养院中的水平比起旧时伤病营的确强出百倍,但跟家中疗养的安适相比,却并没有好到哪里去。但偏偏有人就是相信传言,认为住在疗养院就是比在自家调养要好。

很早以前,就已经有许多官员向韩冈要求,专门为他们和他们的家人开办一间疗养院。韩冈不想得罪人,又不愿浪费手下不多的人才,所以他便决定在疗养院中划出一栋必要的病房,用来安排来住院的官宦人家。也幸亏这些人基本上都在秦州城中,让韩冈不必在其他两处疗养院费心思。

尚未彻底完工的上舍病房已经得到了所有参观过的官员们的一致赞美。不再是通铺隔出的空间,而是一间间精致的单人房。这里的一切的形制都按照后世的病房来设计。每间病房的墙壁都用石灰粉刷过,地面也是抹了水泥,窗户都朝着南面,虽然没有玻璃,但质地良好的窗纸也可以挡风透光。

榆木打造的单人床上铺着洗得很干净的麻黄色床单,显得干净整洁。床边还有着摆放杂物的床头柜,上面还可以放着油灯,一根绳子从床头垂下,那是连着门外呼唤医护人员的铃铛。病房中的每一间房间,都是与其他房间一模一样,大小,装饰都没有区别。

疗养院是前线医院的别名,而眼下的上舍病房则是民间医院的雏形,如果能够发展起来,让医院制度传遍天下,韩冈光靠这一事,就足以名留青史。

陪着韩冈将一间间病房查验过,仇一闻问道:“机宜,听说你要去延州了,不知是不是真的?”

“要调我去延州,传言倒是比事实传播得要快。”韩冈摇头,笑叹一口气,“谣言而已……倒是雷简要走。”

雷简要走了,不过一直留在甘谷城的那位京中派到秦州的医官,并不是调回京中,而是要转去庆州。而他这一去,甘谷疗养院就少了得力之人去掌管。

仇一闻手底下的确有人,当年铁面相公的威名比如今的种谔还要强出不少,而铁面相公李士彬的儿子,仇一闻的徒弟,曾经被韩冈拯救出狱的李德新,这的确是个上上大吉的人选——只要忽视掉他的党项身份。

幸好在关西,党项身份算不得什么。折家就是党项,不过跟西夏打了几代人的仗,如今也没人真的把他们当作蕃人来看待。。

‘究竟该如何是好?’韩冈考虑着这个问题。

