\section{第25章 闲来居乡里(六)}

这一天,刘源起得很早,比起惯例的五更初刻起床,要早了半个多时辰。

就算是夏天,四更天的时候,天色也还是黑的。可不仅仅是刘源,胡千里等一干旧时将校,都早早的起床,派了自家的小子,去庄子外打探消息。

在河州会战结束之后,刘源等一干广锐军将士,已经在渭水河畔,安稳的度过了一年多和平时光。这一年多来,武艺虽然没有放下,但做的更多的是土里刨食的活计。田里的农事乃是韩千六亲手教的,麦子、棉花、菜蔬,都是手把手的传授。

靠着前年、去年的战后封赏,这些广锐军将校家中的吃穿用度都不差。可坐吃山空可不行,光靠五月收获的小麦,留下家里一年的用度后,剩下的麦子根本卖不出多少钱来。不论是哪一家,都需要一个更好的财源,棉花就是其中最重要的一项。

前两天,他们已经得到了韩家私下里的通知,说是顺丰行和其他秦州城里的大商号,今次要来承恩村商议今年收购棉花的价格。虽然不知道为什么还没到收获的时节他们就来,但棉花的收购商不好得罪,何况还有韩家的顺丰行在,怎么都得给韩冈父子一个面子。

这么一等,就从四更天,一直等到巳时初。各自等的不耐烦的时候。刘源的大儿子骑着马,跑进了庄子来。跟在他的后面,还有好几个同时被派出去的各家的小子,一叠声的喊着:

“来了!来了!”

不移时,冯从义领着商人们和他们的随从,一行三十多人,到了承恩村前。

见面之后,一番客套。一众商人被领着去了棉田处转一圈,然后坐下来讨论今年的收购价格。

刘源本有着讨价还价的打算,但广锐军将校们没有想到的是,去庄外的棉田绕了一圈后,商人们当场就拿出了一份合同来——一份让他们没法儿拒绝的合同。

按照商人们拿出来的方案,只要在棉田出苗的时候签下协议,当场就能拿到两成定金,可以用来度过青黄不接的春天。等到了秋收后,将收获的棉花依照合同交付,便可以拿到剩下的八成余款。

这份合同定下的供货数目,以之前两年棉田的平均亩产为准——等到明年之后,则就要就改为三年——至于收成后的丰歉,只要在七成以上,那么就给付全款,不足七成,付款的数目则以协议金额的相应比例来定。若是收获比起预计数量还要多出三成以上,那么多出了来的数目,同样是按照约定价格的相应比例来付账。

“即便是绝收,也只退回定金的一半,也就是说情况再差,还有一成的钱可以拿。”冯从义从头到尾细细的向刘源他们解释了一遍。

这样的合同,刘源等人从来没有看过听过。在他们的想法中,卖棉花不过是跟卖粮一般,卖的价格要看当时的市价,还有商人们的良心了,何曾听说不见实物就提前半年多下定金的情况。不过冯从义不经意间的几句话,透露出韩冈对此帮着说了不少话。让一众广锐将校,更加确定韩官人的确是自家人。

成轩并不奇怪刘源等人的惊讶,毕竟此等协议一般只出现在南方的果园中。隔了几千里,西北的军汉如何能知道?韩冈了解得如此之深,直接指示让他们依照来定,已经让成轩等人惊讶不已,后来想想,应该是冯从义向他解释的缘故——尽管冯从义本人不承认。

这份合同,刘源再满意不过,再讨价还价,就显得他们没有诚意了。战场上厮杀的汉子,没有多废话,直接拍了板。各家各户验过了田亩面积,在合同上画押按了手印。

与广锐军将校聚居的承恩村签订下协议,下面还有几十个村寨,不过都可以让自家的伙计去处理。有承恩村作为榜样,不必他们这些掌柜、东家再跑腿了。接下来,应该是签了约后的宴会,刘源也的确让人去杀羊沽酒做准备来请客。

只是冯从义看看天色,回头道:“此时天光尚好,先去看看家里的庄上看看纺纱作坊,回来再来赴刘保正的宴也不迟。”

“如此甚好。”

成轩等人忙不迭的点头,他们早就盼着能去韩家的纺纱作坊一看究竟了。

离着陇西城二十里,在渭水南岸两里处的一处高地上,有着一座高墙环绕的庄子。这座庄子全属于韩家所有,里面的都是投靠了韩冈的庄客,多是在阵中伤残的士卒,离开了军队后,被韩冈收留。不过真要厮杀起来,四肢健全的普通人也很少能胜过他们。而棉纱作坊,就在韩家庄的内部。

去年的棉桃早已处理完毕,今年的还没有收获,韩家的纺纱作坊已经结束了工作,关着大门。由于事先已经得到通知,作为庄头的一名老兵见着冯从义带人来,不待吩咐,便让人将工坊给打开。

工坊中,到处都能看到‘严禁烟火’四个大字,四个字上面都附着一个图案,红色的火苗上画了一个黑色的叉。不管识字还是不识字,都能知道织造工坊中有何禁令了。

西北地多,这座工坊占地也广,两间厂房,两间库房,还有一件管事居住的小院,各自离得甚远。工坊内的水井有三眼,盛水的大缸摆得到处都是,对于防火,做到了极处。

不过没人在意这里的布置,厂房内的东西,才是成轩他们今次所在意的。

瞅着黄土垒起的厂房,刘广汉问着:“十六锭的纺纱机可就在里面?”

“当然。”冯从义点头笑道,让庄头去开门,“几位兄长既然已经同意共襄盛举,自然不会有半点隐瞒和藏匿。”

厂房大门打开,冯从义手一伸,“请!”

一拥而入。

冯从义微笑着,跟在后面进了厂房中。

今次能将这些商人们团聚到陇西,韩冈同意向他们公布新式纺纱机的承诺起了关键性的作用。

若是其他地方,纺纱的工作其实也是棉田的田主家来完成。也就是说,从种植,到采摘,再到取棉、纺纱,全都是一路顺下来,织布作坊只要收购纱锭就可以,那就根本不需要来此通过实现下定来划分棉田。

但陇西这里不一样,纺纱工坊的建立,是跟棉田的推广种植几乎是同一时期来完成的。单是顺丰行下面的作坊,就有三十台十六锭纺机。

之前在冯从义看来,在棉田没有扩大种植面积之前,使用人力就已经绰绰有余,盲目用上这些机械,纯属是浪费而已。农桑二事从来都是一家的,现在换成棉花,本质还是一样,男耕女织又有什么不对?只不过,韩冈的坚持让他不敢不遵从。

而从一开始,韩冈就没有打算让纺纱这一道工序,变成单门独户的营生。即便是最简单的珍妮纺纱机,其效率上的进步,相比于旧时的单人纺车也是天翻地覆的。要是让陇西的棉农形成了男耕女织的小农经济利益体之后,再想进行这方面的改进,必然会引起他们的强烈反弹。利益上的损失,可以让人对任何效率上的改进恨之入骨。韩冈无意给日后留下后患,未雨绸缪才是他一贯的行事习惯。

韩冈知道机械化的纺纱和人工纺纱最大的区别是将纱锭换个角度立起来,但究竟是如何去‘立’,韩冈也只能摇头摊手。他的旧时记忆,完全排不上用场。可当韩冈找来几个将作营的工匠后,他只提点了几句,工匠们仅仅用了两天的时间,便将单锭的纺纱机改造成了八支纱锭的纺机。

韩冈在欣喜之余,也为改造的简单而吃惊。看了新式纺机与旧式的对比,差别根本就是一层窗户纸而已,只要点透了,改造起来完全没有任何难度——难就难在那层窗户纸上。同时新式纺机的改进也是很容易,到了一年后的现在,在工坊中使用的纺纱机已经变成了十六支纱锭。更多的纱锭也是可行的,但动力的来源,就不能依靠人力了,下一步的改进措施,要往水力或畜力方向考虑。

只是成功的仅仅是纺纱机,织布机的改进并没有突破性的进展。飞梭这个名词想必所有学过历史的学生都还记得,可怎么一个‘飞’,韩冈不知道,也无法通过这简单的一个词来向工匠进行明白的解释。只能告诉了他们这个词汇后,让他们自己去琢磨。

不过让他惊喜的是,用来处理棉桃的轧花机,却不用他吩咐,却已经有人造了出来。并不是工匠,却是第二年就开始学着种棉的一家农民。两根人力驱动的锯齿状的木杆,棉桃从木杆中碾过去,棉花外面的皮和里面的棉籽就给轧来出来,而棉絮则沾在木杆上。这可是难得的发明。

单是靠着陇西城的承恩村中,两百多户人家各自都种了二三十亩棉田,总计就有四五十顷之多。沿着渭河再往下的村寨中,棉田种得有多有少,但合起来差不多有五六万亩。今年一下就能收获上万担的籽棉,没有一个快速处理的手段,可就要干瞪眼了。

