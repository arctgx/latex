\section{第40章 帝乡尘云迷(二)}

离着冬至越来越近,开封府的上上下下都为郊天大典而忙碌起来。

韩冈虽然在外,依然也要听着东京城中的命令,为大典准备钱物、人力。而且还传令京府各县,加派弓手、巡检,并牢牢盯紧一干曾经有过旧案的不法之徒,如果有什么可疑之举,可以先行扣押,等到大赦令下达之后,再将他们给放出来。

不论是政事堂、还是开封府,都是三令五申,在这一次国家大典的时候,绝对不能出任何乱子。

韩冈签发命令的时候,都忍不住有些觉得好笑。千年前后的官僚政治,差不多都是一个模子刻出来的,连做的事情都是一般。人虽变,可世情不变,古今中外,概莫能外。

祭天的地点,位于开封南薰门外,被称为青城的地方。离着城池虽不算远,但也属于郊外,所以那里修起来的宫室,就是正儿八经的行宫。

祭天用的圜丘,并不要韩冈来多手。那一座用黄土垒积而成的八十一尺高的土台,已经用了几十年,就算有些损坏,也自有大工匠来处理。但为了整修青城行宫,韩冈还是被命令调来一批流民,听候府中的指派。

东京城分为开封、祥符两县,就跟唐时的长安城分为万年、长安两县一样。不过东京城五十里城墙括起来的这一片地,是由开封府直接管着。只有廓外乡镇,才是由两县管辖。从地位上,开封、祥符并称为赤县,比起白马、陈留这样的畿县要高上一级。

在名义上,韩冈可以管得到开封县和祥符县。但历任府界提点,从来没有去管过两赤县的事,都是让开封知府去处置。韩冈上任半年多了,巡视诸县也从来没有去过赤县的辖区,有故事惯例在没有必要自找麻烦。

从开封府最南端的扶沟县回来,经过青城行宫的时候,韩冈也仅仅是向里面瞥了一眼,就打马而过。多站一会儿,说不定祥符县的知县就要上面报告他韩冈侵犯职权了。

快到南薰门的时候,正好午后,平日这段时间猪走得比人多。韩冈绕了个圈子,从新郑门进了东京城,城门官不再是‘直言敢谏’的郑侠郑介夫,换上来的一个监门官,有五十多岁,见到韩冈来,就立刻小心翼翼的亲自将他迎进城来。

离开东京城不过十数日,城中已经是物是人非。

崇仁坊的王相公府此时已回归开封府管辖,门前街巷变得冷冷清清,不复往日的喧闹。门可罗雀这个成语并不是形容词,韩冈骑马经过,当真就在门前惊起了一群在地上啄食的麻雀。

王安石的旧邸原本就是官宅,由天子所赐,归于宰相居住——基本上两府宰执,在东京城中都没有私宅,住着的宅邸统统都是官产,由天子赐予或是收回。想及京城的地价,韩冈对这个现象也不足以以为怪。

即便是一任宰相,想在京中买个符合身份的宅子,不靠贪污受贿,除非能在相位上盘踞二三十年。而且当真有哪位宰相买下来一片豪宅,御史们的眼睛都会如同遇上磁铁的缝衣针,一起被吸过来。

现在热闹起来的,是隔邻景明坊的冯相公府。冯京还未有赐第,所谓的冯相公府就是过去的冯参政府。韩冈没有从冯府门前的街巷经过,只是从路口向里面看了一眼,便发现那条路,已经是人山人海,车马辐辏。

韩冈摇摇头,一起一落,本是世间常理,用不着太多感叹。

他此次回京,公事上是要去开封府见孙永。天子离城出行,不论是奉天子灵柩归葬山陵,还是出城郊祀,开封知府都照例要担任桥道顿递使,负责道路安全。韩冈是开封府下属,必然少不了要参与进来。

另外在私事上,还要见一下吕惠卿和章惇。王安石刚走,吕惠卿和章惇都来了信,请他上京时顺道一叙。

吕惠卿自不必说,自升任参知政事后,已经是新党在朝堂中的核心人物。韩绛虽然是宰相,可他的作用仅仅是扶持而已。就如同庆历新政时的宰相杜衍,王安石初变法时的宰相曾公亮,都仅仅是来保驾护航的,并不会是真正的核心。

而章惇回朝后,凭借着在荆湖的功绩,已经升任知制诰、直学士院,现在正是炙手可热的时候,说不准什么时候就升了翰林学士——如今因为曾布出外、吕惠卿晋升,正好学士院又多了两个空缺——才半个月的时间,就已经稳坐了新党第二号人物的位置。

至于朝堂上,新党的第三号究竟是谁,就有些争议了。

论理应该是判军器监兼中书五房检正公事的前任宰相曾公亮之子——曾孝宽。但京城中人有很多都认为,王安石的女婿,如今名声响彻朝堂内外的韩冈韩玉昆,只要他卸下府界提点的职位,进入朝堂任职,压倒曾孝宽不会有任何问题。

但韩冈一直以来,对新法虽是支持,在关键的时候又帮了新党渡过了多次难关。无论是雪橇车运粮也好,还是流民图一案也好,新党上上下下,都要承他的人情。

但韩冈究竟对新党的支持能到哪一步,现在也没人心中有底。因为从本质上,韩冈的学术和理念,与以王学为治国圭臬的新党,并不一致,甚至有许多地方截然相反。

过去有着王安石来压着他,不让韩冈始终坚持的气学和格物之说在京中传播,并在经义局中严防死守,不让韩冈有涉足其间的机会。

但现在王安石离开了,经义局的主要成员都随王安石去了江宁,只有吕惠卿升任经义局同提举,留在京城。远隔千里,又有长江浩浩,还能不能压制得住韩冈,不让天子收起蛊惑,这就是个能让新党头疼,而让外界颇为期待的问题。

儒门重师传,学术上难以苟合的纷争,到了朝堂上就是不可磨灭的矛盾。韩冈会不会趁机兴风作浪,如同他在琼林宴上所作的一样,也是新党在王安石离开后,能否紧密团结的起来的一个极重要的关键——无论如何,韩冈从他的身份地位,还有多年来表现出来的才干才智,再加上在天子面前的话语权,都让他成为如今的政局中一个无法忽视的人物。

韩冈并不知道有多少双眼睛在盯着自己的一举一动,由此来评判新党是否能如天子所愿,团结起来将朝政给稳定下来。

但韩冈明白王安石的卸任去职,虽然说这把遮天大伞不再覆盖在新党身上,自此之后,从吕惠卿开始,都要独立承受京中的风风雨雨。但从另一个角度来说,王安石之前所背负的那些矛盾,也随着他一起去了江宁,在某种程度上,新党也可谓是轻装上阵。

朝局已经是进入了一个崭新的阶段,或者用后世常用的说法——后王安石的时代。

谒见孙永,并没有耽搁韩冈太多的时间。关于天子出城后的桥道顿递一事,韩冈和孙永已经坐下来商讨了好几次,今天也不过是将过去说过的事再重复一遍,当然也不是完全的重复,因为一些突发的新情况,也要将过去准备执行的方案稍加修订。

从开封府出来,韩冈便望着吕惠卿府上过去。就在开封府门前,吕惠卿派来的两名家丁,就已经混在韩冈的随从之中,等着他从衙门中出来。

不能叫求贤若渴,也不能叫做迫不及待,而应该说担惊受怕。

韩冈只要不清清楚楚的表明态度,吕惠卿都不会安心下来。即便章惇肯定会在新任的参知政事面前为韩冈拍着胸脯,打着包票,吕惠卿都不会全然相信。

王安石辞相,就像是在水池中,一下丢进了一块巨石。水势翻腾汹涌,使得朝局尚未稳定下来。吕惠卿和章惇都不希望这个时间段,有人会在后面捅上新党一刀,在曾布离开之后,有这个实力的,曾孝宽还差了那么一点——只有韩冈。

在吕参政府上的仆人的带领下,韩冈一路往西。就跟冯京一样,吕惠卿也没有得到他的赐第。韩冈估计,应该要等到韩绛出现,到那时候,天子才会从高到低,一个个赏赐过来。

向着城西的吕惠卿府上行去,从吕家仆役略显焦躁的神色上,韩冈能想得到吕惠卿正在家中焦急不安的等着自己的到来,这不知道是不是因为吕惠卿第一次进入政事堂的缘故。

宠辱不惊的涵养,不是这么容易养成的。韩冈也不认为吕惠卿在一两年间便飞升参知政事,能做到几十年身居高位的重臣才能表现出来的气度。

不知这等心态会不会带来不好的影响,天子需要一个能稳定朝局的政事堂,新党需要一个能安定党内的领袖,吕惠卿若是不能该换心态,新党的未来会怎么样,就有些难说了。

轻轻摇头,韩冈将这个想法压到了心底,自己的猜测并不一定是真实,究竟如何,还要亲眼看了再说。

拉起缰绳,勒马止步,吕惠卿的府邸已经就在眼前。

