\section{第41章 礼天祈民康(四)}

王旖欲言又止,而周南仍是花容失色的样子,一个字也不敢多说。

看着自己不小心将妻妾给吓住,韩冈无奈的叹了口气,宽慰的笑道:“放心好了。只是为了爹娘,你们几个,还有奎官、金娘和二哥儿,为夫到了外面后,肯定会谨言慎行,怎么也不会乱说话的。想想过去,为夫什么时候做错过。”

王旖小心的又劝过了韩冈几句,和周南一起,起身走回到岸边上的帐篷里去看着儿女了。

韩冈静静的坐着,手上的鱼竿动也不动。半天过去,也不见动弹,如同一座雕像一般。

这还算不上是悖逆之言,只是将事情说破而已。就算到了天子面前,韩冈其实也敢说出口的,也不会因此而得罪。真要说其来,韩冈依稀记得包拯对仁宗皇帝说过更为刻薄的话。而直言天子孤寒的臣子也是有过的。

真正悖逆的是韩冈的心思。

他不可能如这个时代的人们,对天子都要保持着一份敬畏。

但即便只为了妻儿着想,韩冈都无意走上九死一生的险路。可就算是走在安全的道路上,韩冈也会向着目标去努力。

韩冈自信他有足够时间,走到能让他实现目标的地方。

并不仅仅是权力。

权力并不足以为凭,此时宰相的权力再大,也是建在沙滩上的。名声更为重要——并不是王安石的那等名声,毁誉皆出于士大夫之口,一日反目,三十年重名顿时化为飞灰。而是要更高一层。

得学学周公,得学学王莽。

虽然结果一好一坏,可两位先贤都有值得韩冈学习的地方。

首先就是要在军器监做出点功业来。

“三哥哥,有没有钓上鲤鱼?”韩云娘欢快地跑了过来,打断了韩冈变得阴郁起来的思绪。

冻得红扑扑的脸,笑得如鲜花一般。俏巧的鼻尖,也是红红的,让韩冈忍不住想捏上一下。常年待在家中不能随意外出,也的确闷坏了她。今年韩云娘才不过十七岁,虽然已为人妇,但还是处在最为活泼的年纪上。

韩冈回头望望河滩上的帐篷边,王旖和周南都在向这里看着。若想韩冈恢复好心情,自幼相伴的韩云娘是最为合适的人选。

转回头,对着如花俏脸:“还没有呢。”

韩云娘一手敛着裙裾,在冰窟前蹲下来,好奇的向里面张望:“什么时候能钓上来?”

韩冈哈哈笑道:“我怎么可能知道?你的三哥哥也不是能掐会算的。”

他正这么说着,忽然面前的钓鱼竿一沉,一下弯了起来。

钓竿弯得如同月牙一般,云娘一下急道:“咬钩了!咬钩了!三哥哥,咬钩了。”

小手一下下的扯着韩冈的袖子,很是为韩冈急着。

韩冈苦笑了一下:“我可没咬钩,咬钩的是鱼。”

虽然在开玩笑,但他抓着鱼竿的双手一点也没有松劲。咬钩的鱼挣扎得很厉害,扯着鱼竿的力量甚至让韩冈从雪橇车上站了起来。

韩冈一下变得兴奋起来:“看来是条大鱼”

韩云娘在旁边也急着催促着:“快点。三哥哥,快点。”

韩冈双臂用力,使劲向上提着。他所用的鱼竿,可没有后世那么多零碎装备,就是竹竿上拴上根结实的麻线。但这样的鱼竿还是老渔民手上买来的,钓起鱼来一点也不耽搁事情,反而顺手得很。

韩冈这里的动静很大,周南和王旖都跑了过来,看这韩冈到底能不能钓上一条大鱼来。

钓钩上鱼儿挣扎了半天,终于松了劲,被韩冈瞅准了机会,双手用力,一下就扯了上来。

哗的一声响,在冰窟中来回窜动的鱼儿终于被提出了水面。在钩子上上下蹦跶着,扯得钓竿一阵阵的抖动。

这一番动静甚大,韩冈都出了一身汗。但上钩的猎物却是出乎意料的小,仅仅是一条只有巴掌大的小杂鱼。在空中来回挣动,溅了韩冈一脸的水。

韩冈悻悻然的摇摇头,从钩子上将鱼给取下来,丢到了冰窟旁的地上。旁边的王旖和周南都笑弯了腰,方才心中的抑郁,一下就散去了许多。

韩云娘拿着鱼篓,看着韩冈将鱼丢到了冰上,也一起将篓子丢了下去。她白白期待了半天,有些不高兴的嘟着嘴,很是孩子气。

韩冈此时放弃了,觉得再钓下去也没有什么意思,与妻妾一起回到了河滩上的帐篷处。他钓了半日,钓上来的两三条都不是鲤鱼,看着也不认识。全都丢在了冰面上,片刻工夫冻得硬梆梆的了。

幸好韩冈带来的随从们,有几个懂渔情的,他们远远地在外围守着,顺便也在冰面上打洞,给韩冈弄上来了七八条黄河鲤鱼。

都是一尺多长,已经在寒风中给冻僵了。

严素心掌着厨刀,指挥着随行而来的两个厨娘,在河滩边处理起鲤鱼来。

一边的小锅里开始咕嘟咕嘟的煮着鱼羹,而严素心又开始在砧板上料理起去腮去内脏的其他几条鱼来。做得不是别的,而是京中如今最为流行的鱼脍,也就是生鱼片。

鱼脍,一个是要看着鱼的新鲜程度,还有种类。黄河鲤鱼算是河鱼中最好的一种了,又是刚刚钓上来的,再新鲜不过。

而同样重要的则是刀工。严素心于此事上最为擅长。她片出来的鱼脍,纤薄如蝉翼,白得近乎于透明,吹口气仿佛就能飘起来的样子。

韩冈夹起一片,占了点调料放进嘴里,冰鲜嫩滑的口感顿时在口中扩散开来。

放下筷子,韩冈对着素心笑道:“若是欧阳文忠和刘原甫犹在,若能尝到素心的手艺,必不会时时提鱼造访梅圣俞家【梅尧臣】。”

梅尧臣家侍女善做鱼脍,欧阳修、刘敞,‘每思食脍,必提鱼过往’。虽然没有尝过梅尧臣家侍女的手艺,但韩冈确信,严素心的手段绝对不在其人之下。

“梅圣俞?就是那个鲶鱼上竹竿?”王旖问道。

“对!”周南笑着点头,她对京中故事比韩冈、王旖都要熟悉,“就是那个鲇鱼上竹竿,猢狲入布袋的梅尧臣梅圣俞。”

梅尧臣以诗知名三十年,与欧阳修等重臣交往甚密,可惜始终不得一馆职。晚年参与修《唐书》,对其妻刁氏道:“吾之修书,可谓是猢狲入布袋。”刁氏则回道:“君之仕宦,何异于鲇鱼上竹竿。”

梅尧臣说他修史书,如同猢狲钻布袋般容易,而刁氏则笑他做官却比鲇鱼爬竹竿还要难。梅尧臣夫妻的这番对话,正是一句佳对,被人听了后,很快就流传开来。

无论是韩冈,还是王旖、周南和云娘,对素心的手艺都是赞不绝口。今天的鱼脍,更是验证了她的厨艺。

韩冈吃了小半条,停了筷子。鱼脍虽好,却不能多吃,尤其是在冬天,吃多了会伤脾胃的。而其他几位,也都没有多吃,

韩冈的一对儿女,这时闹着要下地来。两个孩儿到了河边上,始终都是由乳母给抱着,一刻也不让他们下地。毕竟是在冰面上,被凿开的洞,大人掉不下去,小孩子可说不准。尤其三岁上下的小孩子还喜欢乱跑,很容易出事。

韩冈将儿女抱到膝前,对着妻妾笑道:“偷得浮生半日闲,今天可是难得的清闲。”

严素心笑得有些悲伤:“可是等过两日,官人就又要忙起来了。”

“那也只是一时而已。”韩冈安慰的冲她笑了笑:“我不想多掺和现在朝廷上的事。韩子华、吕吉甫都有私心,为夫何必趟那汪浑水。做些自己喜欢的事,心上能轻松一些。接下来的日子,也可以多陪陪你们。”

……………………

“韩冈还是不肯奉诏?”

“回官家的话,府界提点韩冈的确不肯奉召。”

奉旨前往白马县的童贯连头也不敢抬,他前日第二次去白马县,诏令韩冈接手,但韩冈又给拒绝了,一点也不松口。

赵顼暗叹了一声,终究都是不省心的。

他此前也从孙永那里听说了一点消息,韩冈只想要一个军器监,却不愿接受中书检正。虽然去了中书容易升官,但会掺和进如今纷乱的朝局中,从韩冈的角度来说,这的确不是好事。

可韩冈的盘算赵顼也能看得清楚。

这算什么?!

看到王安石走了,正好可以在京中兴风作浪了?

将关学送入京城,让张载在开封城中宣讲格物致知的道理。如果给了他一个机会,说不定转头就要再一次建言,让张载进入经义局了。

做臣子的都有私心,赵顼也能体量,韩冈的私心算是好了,是为了他的老师,为了他的学术而努力。总比为了钱财、子孙要光明正大上一点。

但私心就是私心,对于朝堂来说,对于天子来说,其实都是一样的。

赵顼不是不能容忍臣子的私心,但要想有私心,最好还是不要表露的那么明显比较好。

“童贯!”

“奴婢在!”

“你去白马县,传朕的口谕,宣韩冈即刻入觐。朕要亲自问问他!”

