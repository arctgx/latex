\section{第23章 天南铜柱今复立(下)}

【迟了一点,下一更在七八点的时候】

韩冈负责重整海门港,要将这座并不算太大的港口,变成交州对外输出的通道。而章惇的工作,除了主持和审核身为行营和经略招讨司的主官无法推卸的任务,剩下的心思就全部放在重立铜柱上。

他希望能在入京之前,亲眼看到铜柱的竖起。将自己的功绩用永不磨灭的青铜传递到千秋万代,也让朝廷的声威,继续震慑这一片位处于天涯海角的南疆。

这一具有象征意义的工作,首先就是要找工匠来。章惇从广西调集了一批铸钟匠——普通的铜匠没有铸造大件器物的本事——接着又派人去广东借调。以他如今的威望,加上日后的前途,广东的几位路中监司官,都不会也不敢从中作梗。

很快,来自于岭外两路的高手匠人陆陆续续的都抵达了海门,没有太多耽搁的就开始了设计和铸造的工作。

至于工料的成本倒是不用在乎太多。一贯小平钱十斤上下,就算融化成数万斤甚至十万斤重的铜柱,也就几千贯、上万贯而已,数目并不算多,而且也不一定会铸得那么重。

从交趾国库中,官军并没有缴获多少财物。为了激励守军的士气,在官军过江之后,李常杰几乎将国库里面的财物全都散尽了。但领钱的人毕竟还是在城中,等官军攻入升龙府后,这些钱基本上都又收回来了,而且还翻了几番——多出来的部分,自然是民间原有的财物。

按照事先约定的条款,士兵、将校、官中,以四三三的比例,将战利品进行分配。掌握在安南经略招讨司中的现钱就有三十余万贯,其中基本上都是铜钱——大宋铸造的铁钱,在境外并不通用,与铜钱并不相同——章惇已经为此上书,从中拿出一万贯来在交州重设铜柱,料无不允之理。

匠人们已经在升龙府开工了,章惇则是拿了自己推敲了好久的《平南记事》来找韩冈,这是准备同时铸在铜柱上的铭文,准备让韩冈过目一下。

不过到了韩冈临时的衙署中,却看见在他的桌上摆着一条色做深紫的杆棒,再仔细一看,这杆棒却是一头有叶,一头有须根,“这不是甘蔗吗?”

“是甘蔗。”韩冈拿起来给章惇看,“是榨糖用的。”他掀开与甘蔗放在一起的一个素色的小瓷盅,里面不是茶水,而是褐色的糖。

“黄糖。”

“红糖。”

韩冈和章惇同时说出口的却是不同的名词。不过黄糖也好、红糖也好,只是对粗糖不同的称谓而已,区别并不大。不论何种称谓,都代表此时市面上流通的蔗糖并不纯净。

“玉昆是打算在交州制糖?”章惇问道,惊讶之余却带了点欣喜。

糖业在此时是暴利,如果交州开始种植甘蔗,章惇倒不介意让自家兄弟来分一杯羹。韩冈在熙河路的一番布局,如今得到的成果,章惇也是艳羡了好久。

“交趾本来就产糖,只是数目不多而已。”

将章惇拉下水,那是顺水推舟的事,一点力气都不用。韩冈则是想着,能不能将李宪和燕达都拖下来,不过燕达出身开封,而李宪的阉人身份也同样让他感到忌讳。

“如果甘蔗种得多了,出产的粮食可就会少上不少……”章惇坐了下来,把自己要找韩冈的事丢在了一边。

“如果让分派在交州的蛮部只从事粮产,将命脉送到他们手里,时间长了就受到蛮部的很大牵制。若是出点意外,天灾人祸什么的,国中或许就会出大乱子。而将糖、油、棉之类的粮食以外的作物交给他人之手,却是没有太大的关系。没有棉花还有丝缎,没有油料那就用些清淡的菜肴,没有糖更不会活不下去。”

韩冈的盘算,章惇略略一想也就明白了。顿时抚掌大笑,拍案叫绝,“如果分出一半来种植甘蔗,蛮部的命脉可就控制在海门港手中了。”

韩冈点头。

这就是殖民,让殖民地从事单一的经济生产,将其变成母国经济体系中的一个环节,藉此来牢牢控制住殖民地。

在千年之后,有一段时间,殖民地纷纷独立建国,但他们建国之后,就立刻陷入了困境之中,从旧有的经济体系中分离,却不能建立起新的体系,有许多到了几十年后都没有恢复过来。

“而且光是粮食和木料,对一个港口来说还不够,再加上糖就差不多了。”

拥有吸引力的特产,是保证一个港口能持续繁荣下去的主要条件。另外就是稳定合理的制度,安全的周边环境,以及完善的交通体系。

除此之外,还有南方特产的各色水果,经过处理之后,就可以运往京城贩卖——用红盐法处理过的荔枝,往往能保存长久,不像唐时,那样要让人用快马一程程的运往长安,只有天子、贵妃才能吃得上,市面上都在卖的。

不过章惇现在最关心的还是铜柱的问题,那是他日后青史留名的关键,至于怎么让海门港变得繁华起来的方略,由韩冈这位专家来考虑就行了。自己在旁沾光,不用费心也能有所收益。

当韩冈问着章惇意下如何,章惇便道:“能者多劳,玉昆你在此一事上天下无人可及,愚兄也不敢班门弄斧了。这些天,愚兄都看着升龙府的铜柱。”

“过两天就去升龙府看一看,不是说最多再有半个月就能成事吗?小弟也想亲眼看一看镇压天南的铜柱铸好竖起。”韩冈笑了一笑,“还有燕逢辰那里已经将升龙府拆得差不多了,听说他还从城中的达官富户家中的宅院中,挖了十几处窖金。数目可不少,光是黄金就有三四千两之多!”

“黄金必须没入官中,不过都计入账内,到分账时一并算进来,该如何分一切都照规矩来。”章惇不在乎二三十万贯的金银,他可不想为这点财物坏了军心。

“对了,”韩冈突然又想起了一件事,“溪洞各部已经将招讨司吩咐的女子都送来了。据周毖回来说,大概是土地还没有分账的缘故,都是挑着好的来。”

“玉昆可是动心了?”

韩冈一笑:“与子厚兄一般无二。”

“你也不想要啊。”章惇笑了笑。他和韩冈都是目光长远,所图甚大,对于这些会损害名声的行为,并不沾手。仅仅是给军中士卒分配女子,可以说是一片公心。但若是从中为己牟利,那就是私德有亏了。

“不过这一事小弟无暇分身去处置,子厚兄可能勉为其难?”韩冈打算将烫手山芋丢出去。

但章惇也不想要,人不是金银财帛,有美丑妍媸之分,有长少强弱之别,要分得人人信服,可不是那么容易,不知要耗多少心神。

“君子不夺人之美,这既然是李宪提议,让他去做牙婆好了,玉昆你我还是别插手为是。”章惇心情很好的拿着李宪开玩笑,转又严肃起来,“过两天你我就去升龙府,亲眼看着铜柱为中国镇住天南之地。”

十天之后,当章惇和韩冈重又回到了升龙府的时候,偌大的升龙府城已经被拆去了一半。而交趾李氏用了六十余年方才逐步修建起来的宫室,更是都成了废墟。不过殿上的梁柱,全都没有浪费,已经扎制成木排顺着富良江直放海门。

交趾王庭所选用的木料,自是上品中的上品,尤其是作为主殿的紫宸殿,二十四根庭柱都是两人合抱粗细的金丝楠木,叩之渊渊有金石声。金丝楠木为主料的棺材,在东京城价值千金,而两人合抱、高有数丈的木料更是见都见不到,有价无市。

韩冈和章惇商议过后,就开始寻找海船,准备将其运回京去。尽管拿来打造宫室有些不吉利,可用来修建庙宇倒是合用,只要运进京城,就是他们开港海门的行动,最有强而有力的证据。

至于铜柱,其位置就选定在旧时的紫宸殿。富丽堂皇的殿宇已经被拆得七零八落,但高达四五丈的台基依然存在。章惇和韩冈就是打算在台基之上,将铜柱给树立起来

铜柱树在紫宸殿的旧址之上,而章惇得到他幕僚的建议,同时准备铸造一批铁柱,分镇各地,以镇压交州气运。之前招讨司收缴了交趾国中所有的兵器箭矢,总共几十万斤的铁料,正好派在这个用处上。

铜柱铸造得很快,只是铸范倒模而已,一根实心的铜柱,比起铜钟、铁鼎之类的空心器物,工序要简单得多,最麻烦的也只是要在模子上阴刻上铭文。章惇亲笔写了安南记事,两千多字的文章要同时铸造在铜柱上——不过依然不是难事。

细雨绵绵,熙宁十年的三月初,交趾紫宸殿的台基上,红亮的铜液倾倒入模范之中。

热浪滚滚而来,天上细雨落到了铜液上,便化作了漫天的迷雾。站在三四十步之外,章惇和韩冈也能感受到从赤红的铜水上传递来的那股澎湃的热力。

铜柱用了三天的时间进行降温,当外面的模子打开的时候,黝黑深沉的青铜上,有着让人心神一凛的金属光泽。

模子被敲碎,一块块的撬下来,片刻之后,完完整整高达三丈的铜柱,出现在每一位的面前。

章惇的双眼中有着无法掩饰的激动,声音都在颤抖着:“标铜立柱,永镇天南!”

