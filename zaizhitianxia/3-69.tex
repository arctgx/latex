\section{第25章 闲来居乡里(二)}

【昨天的第二更,现在才赶出来。看看今天能不能按时完成今天的份。】

见着王旖和韩冈要长谈的样子,云娘又一次起身,“姐姐你三哥哥先聊着,小妹先去看看娘娘那里有什么吩咐。”说着便要避让开去。

“妹妹莫急着走,还有事要问一下妹妹。”王旖拉着云娘不让她走,一起坐了下来说话。

见着王旖越来越有大妇的气度,王家家庭教育出来的结果,让韩冈心中越发的安心起来。要是如沈括的贤妻那般含酸夹忿,这家里就没法儿呆了。

韩冈不想闹得家中不宁,就必须将平衡踩得好,其实很是伤神。也难怪王安石不肯纳妾,说起来,在家中踩平衡,并不比在官场站队要简单。

不过这也是韩冈自找的。不是说他纳得妾室太多——区区三人就算多,让那些随随便便身边就十个八个妾室侍婢的官员笑掉大牙——而是他将云娘她们太放在心上。

如今的世情,少有将侍妾当常人看待。都是如同货物一般,想送人就送人,缺了后就再买。换得勤的,三四年身边人的脸就换光了。有时候生了儿子,或是怀着身孕,照样能遣离家中。

比如名震千古的包拯包孝肃,他的儿子包绶,就是妾室所生。而且这名妾室是在怀孕的时候,就被发遣回家。要不是包拯守寡的长媳崔氏,派人送钱送物,等包绶出生后又抱了回来,包家真要绝嗣了。

包拯去世时,包绶才五岁,因包拯的遗表而被荫封为太常寺太祝——这个从九品京官的本官职位,是专门为宰执官的儿子所准备,用来荫补的官职,王旁身上的官职,便是太常寺太祝——前几个月韩冈尚在京中时,聊天中说起最年轻的京官,正好包绶因覃恩而升为正九品的大理评事,被王雱拿出来当现成例子。

包拯为人正直,世所公认,包青天的名字到千年后依然如雷贯耳。他将妾室遣送出门,不能说他有错,从如今社会的风气和道德上,他也完全没有不对的地方。

只是韩冈不可能有这样的想法,对于周南、素心和云娘,他都是发自内心的去关爱。而正妻王旖,温婉坚强的性格也极是让韩冈喜欢。

就是因为挂在心上,自然要为之烦心。只有心无萦怀,才能冷静处理事务,那是无欲则刚的境界了——对自家人韩冈倒是做不到。

幸好四女都很平和的性子,也知道韩冈不会喜欢她们争风吃醋,心中也许各有想法,但为了良人还能做到谦让体谅。使得家中的事情,并没有占去韩冈太多的精力。

现在他关心的是如今熙河路核心的巩州,该如何处理屯田方面的事务。夏粮的收获,比起开疆拓土还要重要。而要想获得长远的发展,棉花为主的经济作物则更为重要。

只是韩千六能看到的档案,韩冈就看不到。他已经不是熙河路的官员,有些数据必须得靠韩千六给找来。

从韩千六口中,韩冈听到了今年的夏粮产量。比他预计得情况要好得多。自家父亲的确在农事方面有一手——这已经得到熙河路上上下下的认同,甚至还得到了天子和宰相信任——说起来,善于种植的老农所在多有,但有运气得到这个职位的也只有韩千六一人。

巩州今年的粮食收成,已经超过三十万石,往四十万石走。对于正式开发不过三年的边疆州郡来说,这个数字绝对不少了。

只不过这些收入并不是能归入常平仓的数量。其中虽不包括移民们开垦荒地的出产,但即便是官田,也只能拿一半入库,剩下的还要给租佃和屯垦者留着,从没有一口全吞的道理。而且田地开垦虽说越来越多,却因为人手跟不上需要,无法悉心打理,粮食平均亩产量只有一石出头,比起刚开始的时候,还要低了一些。

真正能放入常平仓的口粮,只有十五万石上下。上阵厮杀的军汉,消耗的粮食一年最少也要四百斤。马匹对粮食的胃口是普通士兵的三四倍,此外还需要更多一倍的草料补充。要供给熙河路两万三千名常驻军、三千八百余匹军马的日常消耗,十五万石也就能满足七成左右。

另外不能忘记,这些士兵有四分之一是把家人迁到了熙河来,他们也要吃饭,虽然是用军饷购粮,不是官府免费提供,但吃掉的粮食还是实打实的,都是来自于本路。

这样一算,常平仓每年的收入至少要二十万石才够保底。而要想对灾荒、兵事做准备,必须要达到三十万石。幸好开垦下来的田地,几年后就会变成所谓的熟田,只要管理得宜,日后也许不比关中的白渠要差。

晚上一家人吃过饭,韩冈坐在父母的院子中。喝着冰镇的蜜酒,一边享受着夜中的山风,一边与韩千六一起说着路中农业生产上的事。韩阿李则带着韩冈的四名妻妾都在一旁飞针走线,为两个孩儿准备着秋冬时的衣服——就算是富贵人家,女红也是不能丢的。

“……以孩儿的想法,最好能施行田地轮作,隔上两三年便休耕一次,以免地力不足,最后收成越来越少。”韩冈却不是要继续扩大屯垦的面积,磨刀不误砍柴工的道理,他也是早就明白的,“休耕的土地也不是任凭其荒着,种上些苜蓿,那是能肥田的草料。”

听着韩冈如此说,韩千六很是惊讶:“不见三哥你下田,什么时候知道田地要轮作的?!”

“从古书上看来的。”接着对眼中有着疑问的王旖补充了一句,“是《齐民要术》。”

轮作制是古法,从上古时起就一直都有施行。将田地分成四块,三年一歇;或分作三块,两年一歇。同时在休耕的地上,种些豆科植物,用来肥田。此乃世间的常识,南北朝时,北齐人贾思勰所编写的农学巨著《齐民要术》之上,便有详细的记载。

豆科植物能肥田的道理,韩冈前世就听说过,而他在京中买来的《齐民要术》也找到了证据。不过贾思勰说‘美田之法’,是以绿豆为上,胡麻、小豆次之,韩冈并不知道在巩州这片地上适不适合种植苜蓿。

如果苜蓿参与到轮种中,不但军马的喂养就可以减少粮食的消耗,而且对于土地肥力的加强和维持,也有足够的好处。再说,必要的时候,苜蓿还可以充作口粮。虽然味道不会好,但营养不会差太远,还能填饱肚子。

过去由于巩州的田地不足,所以韩冈没有提及此事。但现在情况已经变了,田地超过了目前人员数量照管的能力,这就给轮作制带来了足够的发展空间。依照《齐民要术》这等权威性的农书来种田,就算看起来田地没能都用在粮食上,但照样能堵上所有人的嘴。

“爹爹也能知道书上的耕作法,孩儿当真是没有想到。”韩冈笑着拍自己父亲父亲的马屁,他的做得的确是好。

韩千六摇头笑而不语。

“你爹不是说你不下田了吗?”韩阿李停了手上的针线,对着儿子道,“当年家中百多亩地,你以为你爹和你大哥两人能料理得过来?就是分作三片来耕作的。”

韩冈张口结舌,家里的田原来是轮作的?

一起在缝着衣服的素心和周南背过脸捂着嘴去笑,很少能看见韩冈犯糊涂,被人挑出错来的时候。

“云娘……”韩冈转头向在家中待了十年的童养媳问着。

韩云娘也抿着嘴,忍住笑的点点头。她从小就在家中,而且不像韩冈前身,只需要闷头读书、家事一概不管,农忙的时候一样要下地拾麦子的。旧时的家中农事,比韩冈都要清楚得多。

韩冈叹了口气,他的前身,还真是一门心思放在书本上,家里的事什么都不知道。不过这样也好,韩千六这样的老农既然知道轮作制的好处,也曾经施行过,那推广起来就很容易了。

“今年试一试苜蓿,可以以提供军马草料的名义来报备。”韩冈说道,“新开垦的土地用来种粮,最早一片田都换成苜蓿来种。”

“有些太急了。”韩千六对儿子意见摇头,“大豆倒也罢了,苜蓿过去都没有种过,不知道习性。还是跟棉田一样,先试种一年两年,等熟悉脾性后,再多种起来也不迟。”

“爹爹说的是,是孩儿太急了。”韩千六在农事上是专家,韩冈虚心接受,“就按爹爹说的来。”

能让儿子心悦臣服,韩千六很是有些得意,“明天义哥儿就回来,他在秦州耽搁了几日,跟秦州的几家应该都商议过了。对付那些奸商可是要费口舌,织造作坊的事,也该好好的合计一下了。”

韩冈笑了,“不用担心,棉田都控制在手中,到了采摘的时节,更是要靠大量人力,优势全在这一边,谁能争得过去?不过也不能独占,各家都有赚头,要做到共赢,才是长远之计。”

