\section{第12章 共道佳节早(五)}

这些天,韩冈一直在用心读书,不过间中还是跟着王家的子弟、门客来往交流。

王韶的儿女多,多到让韩冈叹为观止。发妻杨氏结缡十五年,育有七个子女,加上妾室生的两个,九人中活到现在的有六个。杨氏过世之后,治平二年继娶的续弦徐氏在王韶前往秦州之前,两年内连生了两个儿子。加上王韶在秦州纳的两名小妾,也生有一子两女。光是儿子的排行,都已排到第十了。

而且生活在王韶府中的这么一大家子,并不仅仅是王韶妻妾儿女的这十几口人,还有王韶的父母、兄弟,从德江乡里前来投奔王韶的亲戚朋友,加上七八个清客,一班家妓,十几名在熙河路用得顺手的亲兵转成的家丁,几十个仆役婢女,差不多有一百三四十号人。这还不包括,朝廷派到执政门下听候使唤的两队厢兵。

除了清客和厢兵之外,在户籍上,这就是一户人家。如此多张嘴,王韶每个月的拿到手上的俸禄,根本经不起流水一般的花销。要不是有着熙河那边的王家商行源源不断的送钱了过来,加上王韶在老家还有一些产业,家计之上早就要捉襟见肘了。

基本上,大宋的官宦人家都是如此。一人得道、鸡犬登天的事,在这个时代十分的正常。一旦升到高位,前来投奔的亲友会是络绎不绝。不止一个重臣感叹过,他们在做州县官时,往往还能天天喝酒吃肉,但升到了侍制之后,却变成了三五天才能吃上一次肉。

韩冈也算来王韶家白吃白喝白住,王韶为了安顿好韩冈,甚至一口气掉了四个男仆,四个婢女来伺候着。侍候他的仆婢,比王韶的长子王廓身边的都要多。

韩冈倒是安居如常,仅仅多了句谢而已。王家的人不会因此而觉得他失礼,韩冈的身份和关系,足以当得起这样的款待。

在所有的打扰韩冈读书习文的访客中,还是王厚来得频繁一些。不过不同于其他客人,想要跟韩冈拉近关系的盘算不同,王厚倒是多为韩冈着想的比较多。

“玉昆,你已经到了京城的事。王相公家有没有去知会一声?”这一天,王厚来见韩冈,便问起了此事。他有些担心韩冈会不会做得太过了一点,“虽然不便去拜见,但最好还是说一下缘由,这样也能在王相公那里说得过去。”

王厚的提点,让韩冈感到几分暖意,点头笑道:“多谢处道兄提醒,不过今天小弟已经遣了人去送信了。王相公和王家的二衙内,都写了信给他们。该说的都说了,希望他们能谅解。”

王厚呵呵笑了两声:“玉昆你还是这般周全,愚兄倒是多说了。”

“是人总有想不到的时候。若没有处道兄帮手,不说别的,当初支撑河州前线的转运之事,怎么成功不了的。”

王厚与韩冈又现聊了几句,便告辞离开。到了晚间,王安石那边有了回音。相府的仆人送了一封请帖来,指明邀请韩冈。但请帖的主人,不是王家的老二,而是王大衙内——王雱。至于地点,则是在离着王安石府很近的清风楼。七十二家正店中,只有唐家老店比清风楼更近相府,不过相对而言,清风楼就更安静了一点。

这份邀请,韩冈去也不好,不去也不好,犹豫再三,还是决定走上一遭。

人至清则无朋,水至清则无鱼。拒绝的太甚,反而显得着相了。韩冈自入京后,不见天子,不见宰相,都已经做到这个份上,见一见王安石的大儿子,也没人能说闲话。

……………………

王雱约在了午后未正,已经过了午餐的时间,大约是品茗而已——进了酒店,并不一定要吃饭喝酒。可以品茗,可以聊天,可以下棋,甚至还可以做一些特别的娱乐活动。这一点,古今都是一样。

韩冈比起约定时间提前了两刻钟,先行一步抵达清风楼。虽然他是客,但还是表现出一点的诚意比较好,他并不想跟王安石家太过疏远,尽管还没有确定,但他有七八成的可能会娶王家的女儿。

果然也不出韩冈预料,作为宰相家的公子,就算是请客也不会到得太早,只是遣人在清风楼中定下了位置。韩冈进门后,只报了王家大衙内的名讳,就立刻被迎进了三楼的一间厢房中。

能看得出来,王雱所预定的这件厢房,装饰陈设并不是清风楼中最好的一间,但亲自带着他上来的清风楼掌柜,却对韩冈道,“官人有所不知,王衙内遣人来定房时,直说着要最清净的一间。小店背街这一间房,虽然风景不是很好,却是清净无比。”

掌柜的话声未落,就听着隔壁一阵哄堂大笑,笑声恣意狂放,丝毫不顾及周围包间里的客人。清风楼掌柜奉承式的笑容一下凝固。很尴尬的道,“官人,隔壁正好是今次上京来赶考的贡生,就在王衙内订房之后才来的,也是要的清净包厢……”

韩冈倒是明白了,最清净已经给王雱定了,又来要清净包厢的客人,就被安排到隔壁的房间中,正常情况下也是清净的。

见着韩冈似有不满,掌柜提议着:“不如官人换一个位置……”

韩冈摇摇头:“请客的主人定下的位置,我这个急匆匆的客人先到了,却没有越俎代庖的权力。”他挥了挥手,示意掌柜离开,“我就在这里等着好了。”

掌柜诚惶诚恐的退下去了,隔壁包间传来的声音便越发的清晰了起来。

“……今科的考官应该快决定了,不知主考得是吕吉甫还是曾子宣?”

一个稳重点的声音说着:“不论是谁主考,该说什么,不该说什么,看一看前科状元的叶祖洽,也就该知道了。”

“以正道兄之才,争得当是第一人的位置,至于要担心主考官的问题,还是留给小弟几人。”

“过奖了,余中实不敢当。”

另外一个沙哑嗓门开口说道:“其实不需要担心主考官的还有一个。”

“谁啊?”几人同声问道。

“韩冈!”

一众恍然:“原来是那个灌园小儿,他又有何才学,不闻其人有何诗文传世。”

“他可都是朝官了,还来考进士……不就是知道武功不足为凭,学问才是第一。”

“说起了灌园小儿,小弟就想起了一件事。”最先说话的声音传了过来,“国朝开国初年,曾有一显贵,少年时乃是屠户出身。后请人书写行状,便是感到棘手无比。最后胡大监胡旦,他帮忙写了一句——‘少年时即有宰天下之志’,当这是贴切无比!现在那灌园小儿今次来考进士,你们觉得该怎么说?”

“怎么说?”

“澄清天下之志!”

一句拿韩冈开涮的俏皮话蹦了出来,七八张嘴哈哈哈的一阵哄堂大笑。一人笑得上气不接下气:“好个有澄清天下之志。不知灌园儿用起五谷轮回之物,究竟怎么一个澄清天下法?”

“此话不可妄言!”应该是自称余中的那名士子在阻止:“韩冈如何,与我等无关。且不要胡乱开口。”

韩冈呵呵冷笑起来:“澄清天下之志吗……说得倒也不错啊。”

也许隔壁的士子当真比自己才高,韩冈也不觉得自己在经术上的学问,当真能独树一帜,一览众山之小。自家在文笔上的差距,韩冈看得很清楚。能写好诗赋,文学水平就不是韩冈可比,能一较高下的,也就是自己对经义,还有对于策问试题的思考和判读的深度广度。

曾布最近升了翰林学士,而吕惠卿为知制诰、兼判国子监,说起来礼部试的主考官究竟是谁,用脚指头想都能猜得到。如果能让他们找出哪一张是自己的卷子,想来他们应该不会吝啬在卷头上圈上一圈。

不过礼部试的阅卷工作,并没有这么简单。比起韩冈在秦州参加的锁厅试还要繁复上百倍。光是人数就是天差地远,锁厅试就有十来人,而天下四百军州解来的贡生则总计五千一百余人。自己的卷子也许能让曾布和吕惠卿两人看到,但他们要能发现是韩玉昆的,,可能性几乎为零。

不仅仅是科举,韩冈还参加过其他事关命运的重要考试。虽然说如果让两边的考生去考对方的卷子,基本上可以确定都会是全军覆没。可是,这应试时的道理却是相通的。

文章一定要特别,文字也好,论点也好,至少其中一项要让人眼前一亮。这样才能让批改试卷而变得昏头涨脑的考官们,留意起这份卷子来。五千一百多份试卷,要从中取中三百人,除了最前面的二三十人外,排在后面的两百多人,跟被黜落的四千多人中的大部分,差距不可能很大——毕竟是都从千军万马中杀出来的成功者。

选中者之所以会被选中,黜落者之所以会被黜落,也许只是一句两句,一个词两个词的差别。但这点差别,就决定了谁能站在城池之内,谁又被排斥在护城河之外。

也许每一个参加过决定十二年读书生涯的最终结果的学生,他们的语文老师都这么提醒过学生。作文时最忌陈词滥调,千篇一律的文章,也许在考试时能得个不过不失的分数,但在礼部试时,只有一个结果,那就是黜落。

韩冈的优势也就在这里,第一次参加科举,就总结归纳出应考原则的,贡生中能有多少人?他无意去挑战前几名的资格,他只求能在黄榜一列大名,就算是一个与如夫人相对的同进士也无所谓。因为在告身上,最上等的进士及第,与最末等的同进士出身,都只是会被登记为简简单单的两个字——进士。

进士就是进士。

韩冈正在想着,房门先被几声敲响,然后被推开,清风楼的掌柜引着三人走了进来。

当先进来的年轻人眉目疏朗,身材颀长,就算没有韩冈熟悉的王旁跟在身后,也是能辨认得出他的身份。

“韩玉昆?”

“正是韩冈!”

韩冈微笑点头。而视线从跟在王雱之后的王旁,钉在了最后一人身上,笑容转瞬收敛。

‘开什么玩笑?!’

