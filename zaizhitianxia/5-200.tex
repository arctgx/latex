\section{第23章 弭患销祸知何补(八)}

“鲁国和蜀国两位,应该是不会看这个热闹。”

仁宗的十一女——鲁国大长公主,以及当今天子的妹妹——蜀国长公主,都是以淑德贤良著称,自不会没事停在路边看热闹,而且跟在车边的护卫中,并没看到侍女,车内自然不可能是两位公主。韩冈没明说,但苏颂还是听得明白。

“雍王、曹王,一半一半。玉昆你能确定是哪一家?”苏颂问着。

不知是出了什么事,南顺侯府的方向这时候突然间轰然一片声起,顿时喧闹了起来,街头的人群鼓噪,叫着喊着,一派义愤填膺的模样。

街面上人声如鼎沸,便有不少马匹受到了惊扰,纷纷扬蹄嘶鸣。韩冈和苏颂的坐骑也受了惊吓,连带着队形也乱了起来。

韩冈回头看了一眼,冷然一笑,却没有关心到底出了什么事。随手拍了一下坐骑的脑袋,便让这匹躁动不安的河西良驹立刻安定了下来。剩下的就是用双腿控制,夹着马身,让坐骑稳定的在街上徐步缓行。

但苏颂可学不来韩冈这手控马的技术。手上紧拽着坐骑的缰绳,控制胯下马匹不被周围的喧闹给惊吓住,最终还是要靠两名随从在前面一左一右的把住辔头。

韩冈身边的随从,绝大多数也都是骑术高明,不费吹灰之力的就将马匹安抚住了,而苏颂这边,大部分则是立刻翻身下马,才将坐骑给控制住。

好不容易在马背上坐稳了,苏颂看看韩冈在马背上肩张腰挺的稳定坐姿,不由得赞道:“玉昆好骑术啊。都说南人擅舟、北人擅马,看玉昆你就一目了然了。”

“是马被调教的好。有个好马夫,家中的马都被教训得不错。”韩冈谦虚了两句,又道:“最近甘凉路那边打通了往伊州【今哈密】的路,好马也多了,正好家里送了两匹过来,刚刚训好不久,性情都挺温顺的。要是子容兄不介意换匹新马,明日就送一匹到府上。”

苏颂的马估计有十二三岁往上了,看起来老态毕露。从后臀和侧腹上的烙印看,曾经是做过驿马。体格应该是够了战马的标准,肩高比韩冈的河西良驹只矮了一寸上下,也看不出有什么缺陷和残疾。这样的军马却没能通过战马的选拔,最后只做了驿马,一般来说性情不会很好,不是胆小就是暴躁——确切点说,应该是性情很坏才对,以大宋军中对战马的渴求,性格上的标准一向是放得很低的。

韩冈打量着这匹马一阵,最后道:“子容兄的马,也的确该换了。”

“那就多谢玉昆了。”苏颂也不谦让,他性格豁达,和韩冈又是忘年知交,而且还是有通家之好的姻亲,人情往来上完全不需要推却。

“对了,方才那马车上到底是谁?”苏颂又提起了方才的话题。

“是曹王。”

“何以见得?”苏颂饶有兴致的与韩冈扯着没什么意义的闲话。

“快天黑了,曹王府的人已经将灯笼拿出来挂在车前。是玻璃灯笼,跟寻常灯笼差别很大,离得远也一样能分辨得清。”韩冈指了指前面的元随,挂在马鞍前的玻璃灯笼很是显眼:“这是在顺丰行中贩卖的新玩意儿。雍王心思重,一惯简朴。曹王就没那么多顾忌了,专门向顺丰行定了十二盏玻璃灯笼。”

韩冈说完笑了笑,事先看到底牌,与作弊没两样。

苏颂怔了一下,摇摇头,“难怪玉昆你辨得出!”

陇西有了玻璃工坊,也是最近才传出来的,不是用来造透镜或是器皿,而是做灯笼,在店铺中普通的式样五贯一盏。说贵不贵,京城中等以上的人家都用得起,但也不便宜,相对于纸灯笼,同样易损坏,但两者的价格差别可就大了,所以也只有富户才会去买。苏颂这边,前几天韩冈就送了两盏当礼物,却没舍得挂出来,放在房里当灯用了。

韩冈打了个哈哈,算是就此揭过。当然,他对雍王、曹王的评价,也就不提了。

韩冈跟曹王都没见过几次面,相对于雍王赵颢,天子的这个三弟,也的确没有什么存在感。就像太祖太宗和秦悼王三兄弟,有资格登位的就前两人,老三一般没什么指望。在太后那里又不比他二哥更受宠,很容易让人将他忽略,也就前两天,韩冈才刚刚从何矩那里听说他入宫为齐云总社说话。

转头过来,韩冈倒是叹起了李乾德:“可怜的李乾德,死后也要被拖出来当替罪羊。”

“这样最好。”苏颂并没有多少对异族一视同仁的博爱之心,尤其还有在邕州殉国的苏缄的缘故,对交趾余孽从来都没好感,“说起来不是玉昆你给出的主意?”

《蹴鞠快报》可是京城之中发行量第二大的刊物,仅次于一年一换的黄历。先将罪名推到李乾德的身上,再将邕州的旧事提上台面,引发同仇敌忾之心。京城中满城风雨,十几名死者的家人,抬着棺材堵到了南顺侯府的大门前,人多得都挤到大街上了。在苏颂眼中,如此犀利的手段,极似韩冈过去的作为——熙宁七年八年的那次大灾,王安石利用民心,一举将京城中势力极大的粮行给断了根。苏颂知道,韩冈在其中可是没少出力。

韩冈却摇摇头:“这件事用不着我操心。身处嫌疑之地,这些天来,我可是一句话都没敢多说。”

“那就是齐云总社的那帮会首和他们背后的人了……真亏他们想得出来。”

“这世上本就聪明人居多,尤其是在推卸责任的时候。”韩冈笑道。

韩冈一口否认了齐云总社的行动跟自己的瓜葛,说起来,这个主意也的确不是他出的。他倒也是很佩服齐云总社和赛马总社两个组织的会首们,能这么快就找到了突破口。

在推卸和转嫁责任的事上,他们的努力的确是让人佩服,转得飞快的脑筋也是让人赞赏。

齐云总社的那一群人的为人品性,在这一件事上表现得淋漓尽致。死人是不会说话的,挑起事端的责任安插在十七名死者身上是再顺理成章的事。而在这其中,李乾德就是最好的靶子。

当整件事的起因不再是大宋土生土长的子民,而是李乾德这位降臣,那么事件的性质也就不一样了。不再是聚众致乱,而是降臣心怀鬼胎所导致的结果。

若是定性为前一种,那么为了避免日后相同的事故再次上演,御史台可以理直气壮的建言天子挥泪砍掉两项赛事,顺便将韩冈也牵扯进来——韩冈说自己身处嫌疑之地,就是这个原因。

但若是后一种,南顺侯一死百了。为了朝廷体面,也不可能将大越国的太后拉出来惩治一番,最多将丧葬、抚恤、医疗的费用算到南顺侯府的头上,至于齐云总社,以及两家球队的东主和主事,也就训斥一顿了事。

御史台又能怎么样?

为李乾德叫屈?脸还要不要了?!

如果一切只在朝堂上,还有的嘴仗可打,但昨天的《蹴鞠快报》上就已经将开封府断案的结果给曝光了,让受害人的家属杀到南顺侯府门前哭灵,加上对引发平南之役的交趾入侵事件的回顾,整个民间的舆论全都给《蹴鞠快报》给煽动起来了。

天子脚下的百姓可不是好欺负的,闹将起来,天子和朝廷都得反过来安抚民心。市民阶层比起农民阶层来,更容易受到煽动,也更加敢于维护自己的利益。尤其是现在,有宗室、贵戚和显宦在背后做推手,更是如此。而韩冈本人也就能置身事外,只需要看热闹就够了。

“也不知是推卸责任的事。我是知过开封府的,”苏颂瞥了韩冈一眼,“府中的官吏还是有所了解。下面的那群胥吏,欺上瞒下的事根本管不过来。唆使证人改一下口供,更是多见。若是说到出主意,多半是他们,做了几十年,什么招数想不出?就像李乾德的元随,他们的供词都与其他人证如出一辙,估计就是被府中胥吏唆使撺掇的。”

“胥吏们要唆使,也得能说服人才行。供词上将责任往李乾德身上推,对李乾德的元随也是有好处的。”韩冈说道。

“证人中只有朝廷派去的元随,李乾德身边从交趾带出来的亲信呢?”苏颂冷笑道,“这便是府中胥吏的手段。”

“也是有人给他们撑腰的缘故啊。终究只是出主意,而不是掌大纛的。”

“嗯。”苏颂点了点头,“都混在一起了……因为蹴鞠联赛。”

韩冈微微一笑,都是明白人啊。

李乾德身边是有元随的,而且是朝廷派出来的人,估计在皇城司中还能领一份俸禄。李乾德出外看球,他们必须贴身跟在左右。李乾德死于骚乱,几名朝廷派来的元随保护不力,这是逃不掉的罪名。更何况,天子为了自清,或者说下面主审的官吏为了不让天子‘蒙冤’,定然会加重处罚,乃至祸及家人,只为了给南顺侯府一个交代。

但李乾德之死,如果是他自己挑衅,最后点火烧到自家身上,那么元随身上摊到的罪名就截然不同了,罪责怎么说也能轻上三五成,。

纵然李乾德出门看球的时候,身边除了两名皇城司派来的元随以外,还有其他几名从交趾带来的随从,但开封府却根本就没有将他们给传上公堂。也不怕有人会以此发难,民众已经给煽动起来了,士林更是一边倒,即便御史台也不敢去拿交趾人的口供来驳斥开封府的结论。

换作是韩冈,决然没有这个一手遮天的能耐——换作是在陇右或许没问题,但在京城就不可能了。只有上有皇亲国戚,下有开封府中一应底层官吏,加上市井中一应好汉、豪杰,通吃了黑白两道的齐云总社,才能将整张网撑起来,顺顺利利的将浑水泼到李乾德身上。

一个希望维持现状的利益团体,完全被金钱所收买,为了自己的利益,欺君的事也不在乎多做几件。这叫有志一同。

苏颂感叹起来:“京中的俗谚有‘忤逆开封府,孝顺御史台’之说,开封府的吏员,对卸任的知府向来是不放在眼里的。”

“那是开封知府常是引罪去官。而且想要管好开封府,对那些胥吏也只能多下几分功夫去约束。若讨得了他们的好,满城百姓可就没好日子过了。”

“钱藻卸任肯定是不一样了。”

对于开封府来说,太平时节的京城突然间爆发了造成十七人丢掉了性命的惨案,除此之外,还有更多的伤者。对满城百姓,和朝廷,开封府必须有个交待。而今开封府在最短的时间里找出了真相,给天子、朝廷、百万军民一个合情合理的回复,开封知府钱藻的功不可没——虽然他不一定愿意居功。

“算是他运气,说不定还能在开封府衙中多待上一两年。至于李乾德,”韩冈笑了一笑,“人都死了,又不能翻出来鞭尸,反正就只能含糊过去。”

韩冈已经不关心之后的发展了。在庞大的京城利益集团面前,民间舆论又被其掌握,御史台和其他反对者,并没有足够的实力来对抗,结果已经注定。

开封府既然已经审结,两支球队也就能无事脱身,就是御史台也只敢说这是由于聚众过多以至于生乱,不可能说两支球队就是罪魁祸首。整个案件从刑律上找不出相应的条款,甚至不用交由审刑院和大理寺复核,开封府的责任是查,而不是断——没有被告,没有原告,甚至不能算是案件。

在韩冈的眼中,倒是西城医院在这次的球赛惨案上表现得可圈可点,名声更加响亮。这样的事故,若要是多来几次,在外科治疗上的成就,或许就能再上一个新台阶了。

