\section{第36章 可能与世作津梁(五)}

王韶出知楚州。

这个消息在韩冈抵达襄阳没多久,就传到了他的耳中——世间总少不了有人为他传递信息,甚至是不待吩咐,便将最近的一年多的邸报,包装得整整齐齐的的送来给他。

“果然还是如此。”

韩冈叹了口气,王韶终究还没内能逃过这一劫。他看得是最新一期的邸报,上面并不只是说着王韶,还有朝堂上其他方面的人事安排,对判读朝堂上的变动,有着不可抹杀的巨大帮助。韩冈将新送来的邸报折了一折,他就换了下一页来看。

稳定了西府五年的三位枢使,如今先后离开京城,而东府政事堂中,也是如同走马灯一般。两府人事上的动荡,就像屋外的狂风骤雨,只见是越来越猛烈,却不知道什么时候能有个结果。

旧为京西南路转运司的衙署,现在在韩冈带着全家老小住进来之后,就只是将‘南’字去掉而已——京西路转运司衙署。

这一座由四十多座大小楼阁所组成的建筑群,规模完全符合一名转运使的身份,不过这一片建筑,比起韩冈的年龄都要大,已经是老态龙钟,大约有三十多年没有进行一次全方位的整修。从前面的大堂直到最后一进的后花园,几乎每一处楼阁都是漏风漏雨得厉害。

韩冈书房的窗户,在初夏的暴风雨哗哗的直响,雨水和风暴从门窗处的缝隙中挤了进来,用丝绵麻絮堵着缝隙后,感觉就好一点了,但天花板上的渗水就只能干瞪眼。从房顶上落下来的雨水不是滴滴答答,而是几乎流成了一条水线,几个装酱菜的小坛子放在滴水的地方,转眼就积了小半坛出来。

房间里到处都是水,桌上地上皆是湿漉漉的,雨水沿着书架向下淌着。但韩冈却不心疼自己的藏书,他一向只对书中的内容感兴趣,倒是不怎么在意去收集所谓的珍本、孤本,就算全都泡烂了,大不了再花钱买就是了,反正里面也没有多金贵的版本。他就安安然然的坐着在潮湿的书房中,继续翻看他的邸报。

王旖在外面喊了一声,等了片刻之后,推门进来,看着韩冈依然故我的不动如山,王旖好气亦复好笑,“方才几个小厮过去的时候,都说官人你是宰相气度,雨水都临头了,还一动也不动。。”

韩冈笑了,将邸报丢到一旁的书桌上:“若为夫能耐雨淋便是宰相气度,外面街巷上的小贩,就都是宰相候补了。”

书桌上刚刚擦过,也全是水迹。王旖乘着邸报还没有完全被湿透,赶忙揭了起来,双手的食中两指的指尖捏着,随意瞥了两眼,对上面王韶的出外并不感到如何惊讶,之前的征兆太多,而韩冈也跟她说过王韶可能要出外了,只是平平静静的问了一句:“王副枢终于出外了?”

韩冈点点头,叹气道:“王子纯离了西府,元厚之进了东府,一出一入,人数倒不见少,可这事乱的……真不知道一年之后,两府之中还有几人能安然无恙的?”

“官人真是替古人担忧。”王旖笑说着韩冈,“官人何须操心朝堂上之事?难道这是京西转运使的差事不成?”

“说得也是,”韩冈自嘲的一笑,自己关心过度了,其实不论两府中的姓名怎么变动,他都是无关人等,根本不应该去多想,“不该管的,也管不了,只能是看看热闹好了。”

在王韶出外之后,紧接着就是元绛入政事堂。前后就差了一天,所以登在同一张邸报上。

元绛算是新党的同情者,但也没有旗帜鲜明的站在新党一边。论资历,元绛可以傲视同侪,官场上的辈份可以比拟文彦博、富弼,比王安石还要年长十几岁,只是升得慢了些。做过三司使、翰林学士、知开封府,眼下也终于做到了参知政事的任上。

韩冈与元绛不熟,但几乎是在同时升任翰林学士兼权知开封府的另一人,韩冈却是极为熟悉——是苏颂。权知开封府和翰林学士都是通往两府的最后一级阶梯,眼下的苏颂,无论是地位还是资历上,都很可能是参知政事的候补。如果政事堂不能取得平衡,说不定苏颂也会被招进去。

在后王安石的时代,朝堂上要想重新找到一个平衡点,恐怕还要不少时间进行调整。只是这一切正如王旖所说,当真跟韩冈无关。

丢下了去为‘古人’担忧,韩冈同时也丢下了湿透的书房,随着王旖一起离开。书房中的阴湿,自有家中的仆婢来打扫,但唯一的问题,就是如果没有一个晴天,再怎么打扫,都会转眼就变回原样,一点也看不到改变。

此时雨水如注,倒悬而下,晶莹透亮的水瀑就挂在围廊前的屋檐上。

韩冈摊开手伸进雨帘中,让滑落的雨水砸在掌心处,感受着清凉的湿润,以及暴雨的猛烈,“雨要是能早些停就好了,四月初夏,雨水太勤对地里的庄稼可不是件好事,对城里的百姓更不好,”

王旖在后面摇摇头,他的丈夫总是挂心着政事,无论是身在朝堂和地方,很少见他能轻松一点。除了处置眼前的公务,就是在书房中写些什么,或者是接见一下宾客,却不见他出去与同伴找官妓喝酒,顺便嘲风弄月。

不通诗词歌舞、琴棋书画的坏处就在这里,会占据大量时间的不良嗜好一概没有,韩冈当然只能是将多余的心思放在正经事上。

当然,这样的问题在韩冈看来并不是问题,而且这也让他能抽出更多闲暇时光,去陪着他的妻妾和儿女。

过了两天,笼罩了京西路南段的阴云,终于雨收云散。被暴雨拘束在衙门中的襄州知府和襄阳知县,亦终于能出来忙里忙外,韩冈也派了人去监察和清点这一次暴雨所带来的损失损失,同时以防有奸人想趁着如今的暴雨,将他们之前对府库的亏空,全都给名正言顺的报上来。韩冈心思细密,又深悉官员们的无耻,可不会给人利用这一场暴雨的机会。

清点灾害伤亡人数和粮秣库中损失的情况,大约用了六天的时间,而从报上来的数字上看,有很多值得韩冈皱眉的地方。

韩冈正拿着报上来的清单,一项项的仔细查看。清单上的许多地方,都被他用笔描了出来,那是值得商榷甚至审核的项目。但不管韩冈怎么对上面的数字质疑,但报上来的死亡人数,都已经远远超过他的心理预计。

死亡者竟然有一百六十余人,同时还有六百多间民房垮塌,至于失踪人数,到现在还没有一个准确的数据。

韩冈皱着眉头,正想着该如何解决这一个难题,却听见外面在通报,说是李家的二衙内到了。

“李家的二衙内?”韩冈几乎都要忘掉了这个人,甚至连姓名都快忘了。借着拜帖和残存的记忆,终于想到京西转运副使李南公次子的姓名。

虽然韩冈已经跟沈括说好,将他安排在唐州,但李诫既然是韩冈的幕僚,自然也得先来拜访一下他的东主。

李诫是李南公亲自推荐到韩冈幕中,是为了让他没有功名在身的这位次子,能搭上韩冈飞黄腾达的路子。

说实话,韩冈对这位走后门的李家二衙内根本都没放在心上。不是说考不上进士就没本事,而是李诫据说是在天下各地游历,有着这样的经验,很容易就能投入任何一名州县官或是监司官的幕府之中,但李诫却是从来也没有这个经验,依然是一名布衣。

这跟李南公地位不高,功劳不显,无法荫补子孙有关,但更多的应该还是李诫本人缺乏足够的能力。

——韩冈本来是这么想的。尤其是见到李诫之后,发现他长得还算是周正,口才也不算差,在官宦人家做个帮闲一般的幕僚根本不成问题,这就更让韩冈怀疑起他的能力来。

可说了几句之后,却发现李诫对营造匠作之事的了解,可算是真正的专家,并非世人只能看见成物一般的肤浅。

“营造一事,首要乃是度、量、衡。尺规衡器若有差异,前后制作出来的两样器物,就是天差地远。”李诫拿着茶盏和盖子,比划给韩冈看,“如果制作杯盖、茶盏的瓷胚时,定下的标识有所不同,这杯盖就别想稳稳的盖在茶盏上。”

韩冈很欣赏李诫的见识。工欲善其事,必先利其器。精准的测量仪器,是工业和建造业大发展必不可缺的先决条件。他点头附和着:“漕渠的开凿当也是如此。”

“龙图说得正是!”李诫一拍桌,“开凿渠道,自然是要依靠水流。两地之间高差是决定水流方向,哪能缺少精良的器具加以测量?绘制地图、打造沙盘,一切的根基都取决自一开始的对山川地理的测量。”

看着李诫在面前侃侃而谈,‘当真难得的人物。’韩冈心中想着。

