\section{第五章 心念亲恩思全孝(下)}

这

这炊饼便是武大郎卖的那种,原来唤作蒸饼,几十年前为了避仁宗赵祯的讳,改为炊饼。其实呢,也就是后世的馒头。至于此时的馒头,其中夹有肉馅,乃是后世的肉包子;菜包则唤作素馒头。

作为下饭的配菜,是几碟各色腌菜——韩家自家种出来的新鲜蔬菜自己都舍不得吃,皆是卖到城里的大户中去换钱。

做汤饼和炊饼的面粉都是一斗麦子磨出九升半的粗面,连壳子都磨在里面,而不是那种把麦子磨得只剩一半的白细面。这样的一餐能填饱肚子,却也没什么滋味可言,何况还是一日两餐,每日总有半天时间肚子咕咕在叫。

此时的普通人家,也都是跟韩家一般无二。原本韩家还算殷实,至少每隔十天半月,入城卖了菜之后,都能买些酒肉犒劳下自己。但如今家里骤穷,肉就算买来也是给韩冈补身子的,韩千六想打个一角酒来过过干瘾,也是舍不得费那份钱。

而是在惯熟的酒坊那里讨了些不要钱的酒糟回来,用开水灌进只老酒壶中,咂吧咂吧味道,解解酒馋。不过自己吃得虽都是粗食,可看着韩冈很有精神的大口大口的吃饭,夫妻两个却都是眉花眼笑。

韩千六、韩阿李也许有些不清楚,但拥有在外游学两年记忆的韩冈却是知道,他的两个哥哥战死,肯定是有抚恤的,钱和绢都该有个五六贯、七八匹。可这抚恤在衙门里就像流水过沙漠,转了几道手,也就无影无踪了。如果这些抚恤都能足数发下,韩家的家用肯定能再宽裕一些,赎回一亩半亩的菜田也是没有任何问题。

韩阿李吃得很快,韩千六却是举着碗,一小口一小口的慢慢抿着兑过水的酒糟。韩冈的眼睛没有因为常年苦读而变得近视,能看清刺在韩千六左手手背上的两行小字。小字因皱纹多了给模糊掉了许多,韩冈勉强能分辨出‘弓……手……四’这几个零零碎碎的几个字。

韩冈对此有所了解。这是韩千六所属的秦州乡兵组织的番号,弓箭手第四指挥。由于身属军额最下等、在陕西是三丁抽一的沿边弓箭手,所以只刺了手背。如若是禁军厢军那肯定是要刺面的——韩冈那位战死的二哥便是在脸上刺了字——而乡兵中的保毅、强人弓手等上位军额,也是要在面颊上刺字。

一日两餐,勉强饱肚,时时还得从军上阵,死后连个抚恤都到不了手,这便是宋代陕西的普通人家。

韩千六啜着酒糟水,不知想到了什么,放下碗唉声叹气起来:“唉,人若是贪起来,连脸皮都不要了。三哥儿病都好了,正打算把田赎回来呢。李癞子倒好,竟然还想着要把典卖改成断卖!”

“呸!想疯了他的心!”韩阿李啪的一下把筷子拍在桌上,虎着脸,“要钱救命时他还价,还尽介绍些庸医,害得家里钱用得像流水一样。现在俺们又不缺钱。让他做梦去!等三哥儿病大好了就上门去,把典给李癞子的地都给赎回来。有一亩的钱就赎一亩,有两亩钱就赎两亩!”

“俺今天不也是跟李癞子这么说了吗?河湾菜田俺是肯定要赎回来的。”

“屁!今天李癞子还是老娘骂走的,你就会在旁边干看着!他就是看着你是个锯嘴葫芦,才敢欺上门来!换作是老娘,早一扁担打息了他的心!他亲家黄大瘤也是一路货色,前次在渡口见到云娘,口水差点都流出来了。老娘当时擀面杖不在手,不然就在他脑门上再敲个更大的瘤子出来!”

韩冈这时才知道,在碰见自己之前,李癞子已经跟父母打过照面,谈过菜田的事了。难怪他见到自己提起就立刻翻脸。想来因是午后父母在南面山中采到了足够的山货,准备北去州城的时候,在渡口跟李癞子碰上的。

韩冈停了筷子,低下头:“都是孩儿不好……害爹娘要受李癞子的欺。”

“胡说什么!”韩阿李回头又是一声断喝,“治病救命,再多钱都该花的!”

“说得是啊,救命用再多钱也得花。断了香火,下去了也没脸见韩家的祖宗。”韩千六举碗一饮而尽,用手背抹了一下挂在胡须上的残酒,“三哥你也别多想。当年你爷爷从京东密州老家到关西贩货,折了本钱,那是分文没有,连随身的衣物当得也只剩一件,家都回不了,只能在秦州定了居。可你爷爷从给人租佃,到他走的时候,就已经给你爹俺置办下了那块三亩二角一十五步【注1】的菜田。俺花了二十年,又置办下了一百一十亩地。

现在就算都没了,不过是回到你爷爷刚来关西的时候。再过二十年,你爹照样能把田攒回来,也照样能喝酒吃肉。这世上的人啊,不怕穷,只怕懒。只要勤快,做什么都能成事。三哥儿你是读书人,圣贤书装了满肚皮,爹也没什么可以教你的,也只有送你勤快二字,读书要勤,做事要勤,日后做了官也是一样要勤。”

“爹爹说得是。”韩冈低头受教,韩千六虽大字识不得一箩筐,可见识却不差。他抬头又笑道:“圣人亦曾言‘敏于事而慎于言’,即是多做少说。爹爹的话已经有圣人的一半道理了。”

“不愧是圣人!”韩千六被儿子拍得开心得很,一仰脖子,一碗浑浊的酒糟水便灌了下去。咂了咂嘴,拿起酒壶摇了摇,又叹道:“跟官坊里的酒也没个两样嘛。官坊里的酒啊,一年淡似一年。卖得是酒价格,出的是水味道。一斗粮下去,出的几升酒那是三倍五倍的兑水。”

“那你过去还喝得那么欢?!”韩阿李又是一声断喝,韩千六自感没趣,自顾自的去咂那壶酒糟水。自家的婆娘泼辣厉害,韩菜园那是能让则让。

韩冈笑道:“要能自家酿就好了,给自己喝怎么也不会兑水的。”

韩千六摇摇头,叹了口气:“谁说不是呢。可这秦州哪个敢私酿?!从秦州再往外三千里就不知刺配到哪里去了!”

韩冈一愣,一段未被触动的记忆一下跳了出来——对了,大宋的酒水可是官府专卖的。

自从大宋开国以来,为补国用不足,便沿袭了五代时的旧规,各路酒坊泰半是官营,要么直接是官酿,要么是承包出去,而且还是公开招标——这一招此时唤作‘买扑’。不仅仅是酒,盐和铁也皆是官营。而茶、矾、香药,官府都要过一手。

若有人想从官府手中抢食,如若是官户,看情况也许会轻轻放过;但若是民户,最轻的也是刺配,重的直接就是掉脑袋了。尤其是秦州,有多少人栽在了这上面。秦州是边境,大小寨堡百十,临着蕃部的寨子都有开官造酒坊,专门做蕃人的生意,那些寨子还一一派了监酒税的小官,只为了让官府独吞酒利。

‘看来开个蒸馏酒坊来赚钱是不成了!可是要掉脑袋的。’韩冈暗自摇了摇头,私开酒坊,铁定的斩首或流放,就算能承包到一个官酒坊,只要进行一点改进,生意好起来后,不是被官府收回就是给眼红的家伙给夺了去,这样的路不用想都知道肯定走不通。

韩千六不知韩冈心中所想,他始终盼着儿子能有个出息。他一边喝酒,一边叹着:“三哥儿你能做官就好了。有了官身,自家酿酒也没人管。今天去给城里惠徳楼送菜,正见着安抚相公家里奔走的老兵从楼后酒坊拿了酒药回去,说是府中要自酿……”

“喝你的酒糟去,扯那么多作甚?!”韩阿李又冲了韩千六两句,回过头来对韩冈道,“当日三哥儿你病重的时候,俺和你爹到李将军庙里许了愿,捐了二十斤香油。自那天之后,你便一日好过一日。这是李将军的福佑。俺和你爹商量过,再过二十天是个吉日子。到时候,村里各家的麦都种了下去,左右也没什么事了。正好到李将军庙里办个几席,一是酬神,二是给你洗洗晦气……”

韩冈笑着点头。韩千六、韩阿李都是好父母,自家舍不得吃的给儿子吃,自家舍不得用的给儿子用。能遇到这样体贴的双亲,在韩冈的心中,莫名的将他们与留在另一个时空的父母的形象重叠起来。

树欲静而风不止,子欲养而亲不待。

韩冈为自己感到庆幸,重生后还能有为双亲尽孝的机会,弥补心中遗憾之万一。不过种菜却不是什么好营生,他并不愿像韩千六那样每天一股粪水味的从田头回来。

韩冈现在想得并不多,要让父母脱离劳作之苦,要让自己活的轻松自在,这些都必须自己去拼搏。不过钱财不足为凭,只有权力才是保证。不论从什么角度,韩冈都有理由为自己寻个官身。

注1:亩、角、步,中国旧式土地面积计算单位。一亩合四角,一角合六十步。

PS:北宋酒水官卖,如果没有个好后台,就别想把酒坊做大了。如果真的穿越,这点一定要小心。顺便说一下,盐、茶、酒、矾,在宋代都是专卖的。有名的酒楼、酒坊也多是国营。只要赚钱,北宋朝廷都会插上一脚,可没有什么不与民争利的说法。

今天第二更,求红票,收藏

