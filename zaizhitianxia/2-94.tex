\section{第21章 山外望山待时至(三)}

【差点没赶上。求红票,收藏】

“权发遣秦凤路兵马副总管……看枢密院为郭太尉想得多周全?这个位置都敢随便给人。枢密院真是越来越不择手段了!”

“这事就不必再说了……燕逢辰【燕达字】都已经到了秦州城里,再提这事根本是多余。”

“郭太尉手下又多了一员大将,还能叫做多余?……也不知天子和王相公怎么会答应下来的,前任副总管可是观察使!”

“都总管若是不同兵事的文臣,那副总管必然要是能镇得住场面的名将、宿将或是老将。就像李经略和窦观察那样。但如今的都总管可是郭太尉,凭他他的身份,镇住陕西都够了,何况区区一个秦凤?有他在,副总管对秦凤来说,其实是可有可无。所以燕逢辰能升副总管……哎,处道你的那只靴子好像是没法儿穿了。”

“见鬼的靴子,泥水都浸进去了,看起来真是穿不得了……喂,你们还不快回去找双新的来,想让我光着脚回去吗……这些浑人就木头一样,不说出来就不会自己动的。”

“过段时间就好了。”

“希望如此。”

天阴着,空气中湿漉漉的。下了两天的雨,终于停了下来。渭水涨了许多,也变得越发的浑浊了起来,汹涌的流水如同闷雷,在河岸上响彻。

韩冈一边闲极无聊的跟王韶说着话,一边砰砰的用力跺了跺脚,就像要把脚下这条狭窄的田间小道跺坏一般。随着他的跺脚,黏在靴子上的黑泥,就从靴面和靴底上一块块的掉了下来。

位于渭水之滨的河滩上,有着一片面积广大、被火烧过的土地。原本长在这里的郁郁葱葱的荒草灌木被一把火烧了个干净。而连这两天的密雨,将原本风一吹就漫天黑灰的河滩荒地,浇成了烂泥塘。

韩冈就是刚刚从这块烂泥塘上走上来,高帮的牛皮官靴上,满是半干不干的草灰、黄泥和雨水混成的灰黑色的泥浆。

而在他身边,王厚则是坐在一张皮索遍成的小马扎上,左脚的靴子上跟韩冈一样都是泥浆,而右脚却是光着的。他方才从泥塘中拔出脚时,可能是靴子没穿好,一用力,脚倒出来了,鞋子却还在泥地里。

王厚翘着脚坐着,他的一个跟班帮他把靴子从泥地里拔出来,正在清理着上面的泥水。不过泥浆已经浸到了靴子里,一翻过来就有黄浊的泥水一条线般淌了出来,根本就不能穿了,而那跟班却傻乎乎的还在清理着。王厚看着不耐烦了,喝了一句,让他去找个干净的新鞋来。

跟班骑着马往古渭寨方向去了,王厚转过来继续跟韩冈说着:“倒是玉昆你这样分析也听多了,但再怎么合乎情理,还是让人不舒服……过两天,燕副总管就要到古渭来巡边了,玉昆你倒坐得安稳。”

“我当然安稳,燕逢辰跟郭太尉一样,都是被韩宣抚从鄜延踢出来的。天子看重他,是因为他有绥德大捷,有功于进筑横山。当然,估计天子也有着安抚郭太尉的想法——韩宣抚事情实在是做得太果决了一点。但若是他敢在河湟之事上有所干扰,看天子还会不会看重他?”

“文枢密待燕逢辰如此优厚,连跳两级的越次拔擢,不信他没有知遇之感。何况以燕达的官阶,竟然能坐上副总管之位,谁看了心里都不会痛快。”

“燕逢辰来做副总管,心中会不痛快的该是张钤辖和高钤辖,处道你生着哪门子的气?”

“……呵呵,这两天高公绰的脸色的确是难看。堂堂阁门通事舍人只为一个钤辖,而一个连遥郡都没有的东染院使却是做了副总管……还有张老钤辖,听说他也是跑到了水洛城去,看起来一两个月内不会回秦州了。”

如郭逵、窦舜卿那般拥有节度留后、观察使这等官阶的将领,被称为正任官,是军中最高位的统帅。但也有的武将,他们同样有着节度使、观察使或是刺史这样的官名,不过他们另外还有一个官阶,那么节度使、刺史的名头就只是虚衔,称之为遥郡官。

就像高遵裕,他是阁门通事舍人、绛州防御使。张守约,他是文思使、永州刺史。两人的本官分别是通事舍人和文思使,而防御使和刺史则是遥郡,与郭逵的节度留后、窦舜卿的观察使并非一类。

正任官虽然稀少,但遥郡也同样难得,多是入了横班才有资格,俗称美官,中层将领中能得到的寥寥无几,高遵裕因为他的身份,张守约因为他的资历,燕达便没有。而燕达的本官东染院使,无论跟张守约还是高遵裕比起来,也都是差得甚远。

所以看到燕达升任了秦凤路兵马副总管,高遵裕连日都跟有人借了他几万贯后就失踪似的阴沉着一张脸,而张守约也是找了个借口跑到水洛城,不想回秦州见着燕达生闷气。

“高钤辖若真的不喜欢看到燕达在他头上指手画脚也简单,早点想办法说服天子,把缘边安抚司改为古渭军或是古渭州就行了。”

“哪有玉昆你说的这么轻松。榷场刚起,屯田也才开始烧荒,要想改安抚司为军、为州,好歹要到明年有了出产之后,方才能让天子点头同意……这是不是玉昆你自己都说过的!”

“是吗,大概吧。”

雨势刚停就下地,王厚有着满肚子的话要抱怨。但离他和韩冈不远处,就是韩冈的父亲韩千六。在长辈面前,王厚也不好意思把怒气发泄出来,只能没话找话的迁怒到枢密院和文彦博头上。

韩千六也是刚从泥地中上来,他的脚踝处还有着泥浆的印子,但他现在穿着的一双多耳麻鞋上,却没有留下什么痕迹。韩冈和王厚从没有下田的经验,而韩千六可是老于农事,当然知道下田时先把鞋子脱了,光着脚下去。

他望着眼前,整整三百五十亩刚刚经过烧荒后的河滩田,手上捏着一块黄黑交织的泥土,笑得心花怒放,全然没有韩冈和王厚的心浮气躁。这些都是分给韩家的田地,只要细心耕作,多施好肥,绝不会比韩家过去的三亩菜园差到哪里。

“三哥,厚哥。这可是真正的好田啊,”韩千六把手上的一捧烂泥展示给儿子和王厚,“一看就知道,从没损过地力,把种子撒下去,连肥都不用施的”

前段时间,韩千六对王厚还是道一声王衙内,但等韩冈和王厚的表妹定了亲事后,称呼便很自然改了过来。

“爹爹说的是。”“韩丈说的是。”

韩冈和王厚有气无力的回打着,没有沾染到韩千六的半点兴奋。

这片田是韩千六早早就选定的,离着古渭寨只有三里多一点。在附近,沿着河滩还有上百顷荒地,韩千六都查看过了,只要开垦出来,就都是出产丰厚的上田,足以养起数百户的人家。听到韩千六的估算,王韶就准备在附近找块高地,开辟一处护田的军堡,以便让来屯田的弓箭手住进来——在蕃区屯垦,汉人们都是聚居在一处,住在专门设立的护田堡中。

自从选定了田地之后,这些天来,韩千六是天天都要出来看一看自家的产业。就算是下雨,也是要举着伞穿着蓑衣,确认一下河水不会淹到地里。

今天韩冈和王厚是为了来确定护田堡的位置,跟着韩千六一起出行。韩千六一到地头,一看到田便就忍不住下了地,而韩冈跟王厚确定了建堡的地址后,反身一看见老子下地了,这个做儿子的也便没有站在田垄上看热闹的道理,也不得不跟着下田。既然韩冈都往泥地走,王厚也同样不好意思站在田头上。最后两人都沾了一身的泥点,靴子也是给烂泥糊上了。

等到王厚的伴当不知从哪来找了双干净的木屐回来,韩冈便对韩千六道:“爹,还是回去吧。这地也飞不了,用不着天天来。”

韩冈并没有继承了韩千六对田宅的重视,在他眼里产业都是一样的,只分赚钱和亏本两种。自家的田地看了看就没多少兴趣了,这片田要想有收入,可是要到明年夏天!哪像冯从义,他在榷场中已经混得风生水起,做成了好几笔生意。只是他还不满意,说这只是在试水,最近正有想法去青唐部一趟,联络上俞龙珂和瞎药,好把生意做大了。

韩千六点了点头,再看了几眼,便也骑上了马。

骑在马上,还不时回头。这一片黑色的土地,到秋后播种前,都会保持现在的模样,但到了明年初夏,遍地金黄色的麦浪就会出现在土黄色的激流边。

韩冈回到了古渭寨中,和王厚一起,想把筑之事禀报给王韶和高遵裕。但他们一进正厅,先说话的反而是王韶:

“渭源那里有消息传回来了。跟木征勾连上的不是康遵星罗结,而是别羌星罗结……康遵一个月前病死了,他的弟弟别羌接了族长的位子,星罗结部现在是彻底的投靠了木征。”

