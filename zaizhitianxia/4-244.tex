\section{第39章 遥观方城青霞举(九)}

【第一更】

收拾了帐本,一家人围坐在一起,吃了有些迟的晚饭。

今夜的轮班是王旖这位主妇,韩冈进了她的房间。王旖房中的婢女就上来给他端茶递水。

“官人,纲粮应该不会有问题吧?”王旖在梳妆台前坐下来,对着台子上的镜子,卸下白天的装束,一边就在问着,贴身的小婢将一条条温热的手巾递给她,让她擦去残留在脸上的化妆。

“早就安排好了。汝州那里有一个指挥使是曾经跟着去我广西的,现在山阴港的防务就是他在总掌。”韩冈向着妻子解释,作为夫妻,韩冈如今越来越多的让王旖参与到他的生活和工作之中,“至于山阳港,沈括是个近视,只看得眼前,他做事我一般是不放心的。但事关前程,他也肯定知道轻重缓急,会把山阳守得跟铜墙铁壁一般。能下手的地方,也就是襄州这里。”

襄州从来都不是纲粮的集散地,守备明显比不上扬州、泗州和东京。但最近一批批纲粮从襄州北上。纲的数量是有限的,大部分的纲粮都存放在港口边的仓库区中。只要凑近去放把火,就能让运抵京城的粮食少上几成,将韩冈的心血给毁掉——保不准有人敢做出这等龌龊事。

“那官人你……”王旖欲言又止。

“不用担心,早安排妥当了。大不了杀一儆百嘛!”韩冈摸着润洁剔透的汝窑瓷杯,微微的笑道。

杯中只有青青的茶水。这样的炒青冲的茶汤,并不合时人的口味,但韩冈却很喜欢。入秋后的襄州,还是有些闷热,为了却除湿气,饭菜中花椒放得甚多,吃完后不喝口清茶,嘴巴里总是有些发酸。

喝了口微涩的茶汤,他对王旖道:“襄汉漕运事了,大概要到年底了。到时候我准备上书天子,在河北修建轨道。第一步是从黄河边到定州。接下来,从北京大名府,到沧州还有真定府,河北的大州府,全都可以用轨道联系起来。”

拔去了发钗,坐在梳妆台前,正拿着一柄黄杨木梳子,梳理一头青丝的王旖,疑惑的转过身,“官人要上书河北修轨道……”她外头想了想,“这是要提醒天子,河北比陕西重要。”

王旖答非所问,韩冈笑了一笑,“如果你去考进士,一条路是要用十年的时间寒窗苦读,然后才有八九分把握金榜题名。另一条路是只要打通了关节,当年就能得中,”

王旖多聪慧的女子?早听明白了韩冈的意思,不过还是配合着:“后一条风险很大吧?”

“当然,成功率大约四五成吧。如果找到了合适的门路——比如像为夫这样的高官显宦——还能加个一成半成。但若是坏了事,就是天大的罪名。”韩冈笑容收敛,声音沉了下来,“你说,改选哪条路?”

“当然选前一条路,总是稳当的,八九分的把握,基本上就是稳拿稳的。后一条路,查出来就是死路。”王旖神情郑重,韩冈的话分明就是在说想要一举攻下西夏,难度实在是太大了,“难道攻打西夏有这么大的风险?!”

“如果从用兵的角度来说,这把握已经是高得不得了了。非必取不出阵,非全胜不交兵,这话根本就是在做梦。寻常用兵于外,脑袋都是拴在裤腰上,开战前甭管多大的优势,只要在战场上的一个小疏忽,全军覆没都有可能。”瞅着一脸震惊的妻子,韩冈说道,“当真以为为夫在河湟、在广西,是靠着名将强兵,轻轻松松的捡功劳吗?全都是提心吊胆,担心能不能看到明天的太阳,把握也就是那么丁点大,换个人来做,亵裤都能输脱掉。”韩冈冷笑了一声,“要不是事情棘手,为夫这么根基浅薄的灌园子,能抢得过那些高第良将?!”

王旖站起身,让随身使女脱了外袍,只穿了一身月白的小衣。她在房中紧皱着眉头,“朝廷用人也是看人才的。能比得上官人的可不多。”

“多谢贤妻夸赞。”韩冈轻笑着,探身将王旖一把拉过来,搂着坐在腿上。压在两条大腿上的弹性,差点让他忘了自己的词。回想了一下,道,“种谔本是准备请我去鄜延路的。他能看到这两年伐夏已经有了五六成的把握,再加上种谔这等名将,西军这等强军,还有为夫在后勤辎重上的一点名声,就算不能打到兴庆府,也不会大败,所以想赌上一把。”

“难道肯定赌不赢?”王旖问道。

“先不提能不能攻下兴庆府。如果西面打得热火朝天,皮室军、宫分军突然南下攻打河北,这仗还怎么打?天子也得担心契丹人一直打到黄河边上。难道有谁愿意看到这边官军攻进兴庆府,那边黄河上的那几条浮桥都得烧掉防契丹?”

王旖摇摇头,这当然不可能。河北是家国之重,没了河北,开封就是被敲去壳子的核桃,任人鱼肉了,丢了两广都不能丢河北的。

“所以说为夫的计划应该不难说服天子,就像为夫前面打的比方,一个有八九分的把握,只是要耽搁一点时间,另一个则是最多五六分,胜了还好说的,败了就十几年难以恢复元气。”

王旖慢慢的点着头,换做她来决定,也肯定是选择丈夫的方案,而不是种谔的。

“以朝廷能动用的财力人力物力,只要能有个两年的时间,就能轻而易举的从黄河边将轨道铺到保州去,那时候,河北也就安稳了……”韩冈搂着妻子犹如少女般纤细的腰肢,贴着她耳边说道:“你想想,契丹人刚在鸳鸯泺点集兵马,我这里就能一万、两万的往前线塞禁军去。等到河北几个重要的州府都铺设上轨道,那时倒是要轮到契丹人担心官军什么时候打过去。”

被温热的气息喷得耳朵阵阵发痒,王旖很是不自在的扭着身子,但力气小,没几下就气喘吁吁的了,狠狠的掐了一下韩冈按在小腹上的大手,问道:“那西夏这里呢?官人打算怎么做。”

韩冈收紧了双臂,得意的看着王旖在自己怀里挣扎,“种子正【种谔】不是要出兵吗,就让他出兵好了,不过不是兴灵,而是横山北麓的银夏。只要不越过瀚海,区区横山,粮秣输送起来还是没有太大的难度的。有了功劳,想必种家也能消停一些了。”

“但那是灭国之功啊,而且还是西夏,种家能放得下?”王旖还是有些担心,不挣扎了。只是一个交趾就让章惇坐到了枢密副使的位置上,韩冈也是成了龙图阁学士。

要知道,西夏的地位远在交趾之上。在北方,没听过说过西夏的宋人没几个,而在南方,知道交趾的才几个。一旦大功告成,那就是几代人的富贵,和传唱千古的名望。

“不是说打下银夏后,攻打兴庆府的功劳就没他们的份了。种家上下都是聪明人,饭要一口口吃,事要一件件做,想必这个道理,他们上下都明白。”韩冈顿了一下,“而且话还是为夫说的。种家在两府中吃得亏太多了,首先种世衡就是一例,接下来种诂又是一个,想必他们肯定会盼望有人能在两府中为他们的撑腰,同时有机会的话,还能得到提携。”

话说到这里韩冈就没有再说多下去了,想必王旖也能明白,与种建中系出同门,与种朴交情匪浅,甚至与种谔关系都不差的韩冈,与两府距离已经不剩几步了,一旦韩冈在宰执之位上坐稳,不犯大错的话,种家能安泰上三十年。

“鄜延、环庆两路联手难,党项人在银夏的驻军抵抗不了这样的进攻,夺占银夏之地后,只要官军不贪功,维持住银夏一地不在话下,且当能继续消耗西夏的国力。”韩冈说道。

“契丹人怎么办?”王旖转头望着这韩冈。

“契丹人要掺和进来就让他们进来好了。在河北、河东,让他们找不到机会,在西夏……”韩冈翘起的嘴角,笑得有些奸诈,“靠瀚海挡着就够了。党项人可忍不了他们在兴灵肆虐。到时候,为官军引路、与我们并肩作战的,说不定还是党项的铁鹞子。”

丈夫一旦指点江山起来,便是神采飞扬,不是窝在家里的痴呆书生在疯人呓语,而是当世名臣在议论国家的对敌战略。王旖像是被魅惑了一般,抬起手,抚着韩冈嘴角上的笑纹。

韩冈一把抓住捣乱的小手,张开嘴一口咬了过去。指尖夹在唇齿间,王旖一颤。

王旖不抽不动,任凭韩冈轻轻啮咬着指尖,只是身子一直在颤着。她很清楚韩冈的心愿,忍着体内的道道热流,勉励开口问道:“听官人一下说了这么多,难道当真打算去河北?”

“怎么会?!”韩冈哈哈大笑,“这点小事难道朝廷还找不出人来执掌?荐了李明仲去打下手,上面再派个掌总舵的,朝廷的财力人力堆起来,两年都是往多里说。”

