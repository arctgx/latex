\section{第46章 龙泉新硎试锋芒(二)}

【第二更,求红票,收藏】

王安石府,韩冈已经来得多了。在门房中,就坐过不少次,而在昨夜,他又在偏厅中与王旁下了两盘棋,但韩冈还是第一次见到王安石,连同他的三位核心助手一起。

王旁与韩冈一起回到府邸,问了门子一下,父亲是否已经回来。得到肯定的答案后,就直接领着韩冈往后院的书房走去。王安石事先就已经说过,只要韩冈到了,不要在偏厅中等,直接把他带到书房外厅去。

韩冈站在厅门外,王旁进去通报,王安石,以及与他正在厅中说话的三人便一起看过来。

与传言一样,高壮如牛的王安石的确长得很黑,比面如锅底的昆仑奴好一些,但也是在琼州海滩上晒了二十年太阳的模样。他身上穿的青布常服有些发皱,又褪色发白,看来这身衣服自做好后就没有浆过,只是洗得多了。都说王安石不拘小节,倒真的是一点没错。

而同坐在厅中的另外三名中年人,当是吕惠卿、曾布和章惇。他们都穿着公服,显然是放衙后,直接从衙门里到王安石这里来的。

章惇是韩冈第一认出来的,他与章俞眉眼间有七八分相似,神态间风流自蕴,不会认错。

剩下的两人中,身着朱袍,相貌俊雅的一个,应该是吕惠卿。吕惠卿才学出色,相貌气度也同样过人,曾深得欧阳修等人赏识,不过等他参与了新法,就摇身一变,成了反变法派咬牙切齿的福建子了。而他最近被天子特授五品服,以正八品太子中允的身份,穿上了只有四五品才能穿的朱红色公服朝服。章惇和曾布,还都没有这个福气。

剩下的一个自然是曾布,相貌普通,身材瘦削,除了眼神锐利点,看不出有什么特别,可一想到他有一个叫曾巩的兄长,本人又深得王安石信重,当然也不可能是普通角色。

“可是韩玉昆?”王安石的视线投了过来,开门见山的问道。

韩冈跨步进门,在王安石面前行礼道:“韩冈拜见大参。”

王安石看着行礼后站起来的韩冈,浅笑点头,不掩心中的欣赏。韩冈的外形本自不差,匪夷所思的遭遇和两段人生的经历所磨砺出来的气质,更不是等闲士子可比。

王安石看韩冈的气质,有着读书人的温文尔雅,宠辱不惊的恬淡,看体格,又是不输武将的雄壮。文武双全四个字,看来并不是王韶帮他吹嘘。

吕惠卿和曾布交换了一个眼色,同时微不可察的点了点头,这位秦州来的年轻人的确比他们想象中的还要出色一点。

章惇则走了过来。在韩冈方才进门的时候,王安石、曾布和吕惠卿都是坐着的,只有章惇站了起来。论地位,论年龄,王安石几人坐着是应该的,而章惇会站起来,却是因为韩冈对他父亲的救命之恩。

“大恩不言谢,我观玉昆也非俗子,无谓的客套话就不说了。玉昆对家严的救命之恩,章惇铭记在心,日后必有回报。”章惇说话豪爽,有点像是市井好汉拍着胸脯说自己一言九鼎的感觉。

“见义不为,无勇也。同为羁旅,岂有不守望相助的道理。”韩冈说得谦退,并不引以为功。

章惇很爽利的哈哈笑了两声,返身坐回座位上。

王安石将吕惠卿和曾布向韩冈介绍过,各自行了礼后,韩冈便在王安石的示意下,在下首的空位上坐好。而引韩冈进来的王旁则从厅后小门退了出去。

坐在最外面的韩冈,却被上首的四个人一起盯着,有点像是在参加考试,气氛比昨日结束的铨试还要严肃一点。

王安石首先发话:“吾日前观王韶荐章,言及玉昆出身寒家,世代务农。以玉昆之见,这青苗贷对百姓利害如何?施行起来又有何弊病?”

韩冈没想到,王安石的第一个问题不是问得河湟开边之事,而是自己对新法的看法。

也对,这不是理所当然的吗?!河湟开边的重要性甚至还不如鄜延路的横山拓土,又怎么可能与青苗贷相比?

不过韩冈对此也有准备,只是顺序变动而已。别看他每天到处晃着,但拜见王安石时,可能被会问到的问题,他都有预备。凡事有备无患,韩冈过往的经验多少次提醒过他这个道理。

“青苗贷至今未在秦州推行,韩冈不敢妄言弊病利害。”看着王安石眉头微皱,韩冈笑了一笑,又道,“但韩冈知道一事,秦州民间借贷,年利往往在一倍左右,是倍称之利。因借贷了三五贯钱,使得子孙都背上巨债的例子,数不胜数。去岁韩冈重病卧床,家无余财可以延医问药。双亲怕累及子孙,就不敢借贷分文,只把家中田地尽数卖去。如果世间借贷的利钱真能降到四成,不论这钱是官府的,还是私人的,对百姓都是好事。”

“就是这个道理!”章惇立刻接话,却是在作哏一般的帮着韩冈,“可恨韩琦之辈,却道青苗贷祸害百姓。”

吕惠卿也道:“还有御史李常,他前日紧跟在韩琦之后,上书说地方上有官员推行青苗贷时,不贷本金而要百姓直接缴纳利息,但问他究竟是哪里的官吏这么做,他却说不出来。继而又说,天子一造宫室耗钱数百万,一宴之费耗钱数十万,为此才要推行青苗法来与民争利。”

“这就是胡说八道了。”王安石说着,微带怒意,赵顼于他有知遇之恩,而他又的确把兼济天下的希望和期许放在了赵顼身上,分外看不过眼御史往他身上泼脏水,“官家虽是统御亿万生民的天子,但自登基后,只有为太后和太皇太后修过宫室,从来没有为自己享乐而耗费公帑。”

“何止是李常,司马十二不也是与韩稚圭之辈一般声口?都说地方州县中有抑配青苗贷之事,还说以县官督责之威,蚕食下户。”吕惠卿狠狠说着,儒雅的脸上带着极深的愤怒。

曾布亦是愤愤不平难以自抑:“青苗法中本有规条,愿借则借,不愿借的也不强迫。若真有犯禁,有一桩查处一桩,天下各路都派人出去督察了。司马君实却还拿此事攻击青苗法。”

说起新法被攻击之事,在座的几人都有一肚子苦水,就像一个被接起引线的火药桶,蹭着点边就爆了,吕惠卿、曾布都是一般。

听得几名变法派的核心人物,像普通人叹着东家刻薄,工钱不高一样的一通抱怨,韩冈能体会到,最近这段时间,反变法派给他们造成的压力有多大。他笑道:“《刑统》禁人为奸盗,可世间奸盗之事从来不绝。按着司马内翰的想法,这是《刑统》的问题,还是把《刑统》废掉了事。”

厅中先是一静,然后一阵哄堂大笑便爆发出来。章惇性格豪爽,毫不介意的肆意大笑,曾布和吕惠卿比章惇稍稍收敛一点,但也只是一点点,就连王安石也是低头抿了口茶水,免得自己失态露出来。

“都道自石参政【注1】故去之后,如今朝中好谑的只有刘贡父和苏子瞻,想不到玉昆刻薄起来也如此锋锐。”章惇放纵的笑过之后,很快就正经起来,对心情收放自如,也是身居高位的必要条件之一,“只是司马十二可是会说话,拿玉昆之言驳他都没用。前日吉甫不就是为此跟他争起来了吗。”

“不知司马内翰是如何说的?”韩冈很好奇司马光的理由,《资治通鉴》可是帝王学的教材,能编纂出如此巨著,司马光的辩论能力绝对不差。

吕惠卿冷笑着:“司马十二是这么说的,‘愚民知取债之利,不知还债之害,非独县官不强,富民亦不强。’”

——愚民只知借债的好处,不知还债的坏处,县官不强迫他们借贷,但过去富民也没强迫他们借啊。

韩冈听着愣了一下,然后直摇头。看司马光这话说的,因为是愚民嘛,所以只看到眼前借贷的好处,却不顾后果。对于这些乡愚,就让他们跟富民去借钱好了,官府不该掺和。

这个结论是怎么从论据推出来的?完全不成逻辑啊!

韩冈低声叹息,司马光也许才智高绝,人或许也不坏,但屁股歪了那就没办法了。屁股决定立场,司马光的立场当然与变法派站不到一起去。

他说道:“家师曾言,庶民虽愚,关乎自己利益之时,却会变得聪明起来。此是人之常情,司马内翰说的实在没道理。”

“司马十二是揣着明白装糊涂。”吕惠卿说得毫不客气。

注1:即石中立。有名的性格诙谐。当员外郎时,与同僚去御苑参观狮子,同僚听说狮子一日要吃五斤羊肉,便抱怨说做官的连狮子都不如,石中立道:我等员外郎,安敢比园内狮。任参知政事时,有人劝他已居两府,莫要再诙谐戏人,他拿出敇书,道,敇命‘可本官参知政事,余如故’。是天子命我什么都不要变。

