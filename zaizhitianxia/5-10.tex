\section{第一章 庙堂纷纷策平戎(十)}

【昨天的第三更,今天还是三更】

赵顼看着李德新面色蜡黄的样子,知道他前面受了不少的累,耗了太多元气。还让人给李德新端了参汤来的,回神补气,还赐了座,让他休息一下。

李德新感激涕零,忙不迭的磕头谢恩。休息了片刻,李德新定了神,恢复了力气,站起身,向赵顼示意自己可以继续,唯一的皇子便被抱到了他的面前。

很让人意外,完全有别于淑寿公主,种痘的时候,虚岁才三岁的赵佣并没有哭,整个过程都是安安静静的。黑白分明的双瞳是儿童特有的清澈,张着眼睛看李德新拿着一根新的银针,消毒后划破了手臂,敷上了痘苗,完全没有闹腾。

王珪和韩冈交换了一个略感惊异的眼神,生长于深宫妇人之侧,从小就被当成宝贝养着,一点苦头都没有吃过,像淑寿公主一样哭得昏天黑地才是正常,沉静到赵佣这样的可算是异数了。

当初种痘的时候,韩冈的几个儿子,也就年纪大的两个好些,而三个小的也快比淑寿公主稍好一点罢了。不过小孩子的脾气也说不准,也许今天例外也说不定。

但王珪会凑趣,知道怎么让赵顼心情好起来:“均国公年方冲龄,即沉静如许,当为天授之德,真乃异数。”

赵顼立刻眉花眼笑,连连点头,眼睛又转到韩冈身上。

韩冈知道赵顼是等着人夸奖他儿子,无奈的说道:“幼子种痘,几乎没有不哭的。能如均国公一般沉静,确实是难得。”

赵顼依然猛点头,听着王珪、韩冈说儿子的好话就是高兴。

这边在对父亲赞着儿子,那边已经将种痘的最后一段流程给结束了,赵佣和淑寿公主的手臂上都扎了一圈红带,有辟邪的用意,同时也算是做个记认。

李德新絮絮叨叨的向皇子公主的身边人说着种痘后该如何保养,以及会有什么样的反应。尽管说的这些医嘱,都已经印了小册子,免费送给每一个种过痘的小儿。但一群人都在专心致志的听着,唯恐漏掉一个字。

纷纷扰扰了半日,种痘的主角们终于离开了后殿,安焘和李德新一众人也都出去了,殿中只剩王珪和韩冈。

赵顼回到他的座位上,说了两句方才种痘的闲话,恢复到天子该有的表情。望着韩冈:“韩卿……”

韩冈低头:“臣在。”

“可还认识徐禧?”赵旭问道。

“……仅在朝堂上会面过,并无来往。不过其《治策》二十四篇曾经有幸通读过,文章写得甚好。”

赵顼神色稍动。徐禧是最近朝堂上风头正劲的官员,尤其是以他对攻打西夏的态度最为知名。韩冈身为朝臣,当然不会不知徐禧鼓吹攻夏时灭此朝食的急迫,可他仅仅赞许徐禧的文章,丝毫不评价他看过的二十四篇《治策》。可见韩冈对于攻打兴灵的的态度依然不变。

“徐禧有心于西北,又长于谋划,用心之处,不逊于王韶。”

赵顼拿徐禧比王韶,韩冈如何听不明白其中的用意,心下冷笑。东西还没抢到,就开始分赃了。谋定而后动,好了不起,的确是‘不逊于王韶’。

“陛下。”他提声反问道:“辽国内乱在即,此事尽人皆知,亦是板上钉钉。可辽国能内乱多少时间?却是无法确定。万一其中一方击败对手,一统国内,而官军精锐正陷于西夏的兴灵两府之中,到时候,可就是河北的灭顶之灾。”

如果郭忠孝在殿中,听到韩冈的发言,肯定少不了惊讶一番,怎么说的话跟郭逵在家里面说的一样。但如果是郭逵在场,却绝不会有半点惊讶。换做他站在韩冈的立场上,也只有这个理由最为合适。韩冈要设法将郭逵送到河北去,能选择的借口当然只能是着眼在辽国对河北的威胁上。

韩冈的话似乎是在证明速攻的方略绝不可行,但赵顼从中听出了破绽,如果河北有威望素著的帅臣坐镇,那么又何惧刚刚经过内乱的辽国?

这算是妥协了吧。

韩冈和郭逵反对速攻西夏的态度对赵顼来说,其实有着很大的压力。如今朝中最为精通兵事的文武重臣全都抱着以稳为主的立场,纵然究其原因,或许皆是两人拥有私心的缘故,但赵顼心中还是有几分没有把握。现在终于给出了替代条件,心头上的一块巨石好歹是落地了。

“韩卿,你觉得当由谁来出镇河北为佳?”赵顼问道。

“此事当由陛下圣裁。”韩冈绝不会点名,那样反而会惹起赵顼的怀疑,平添一份阻力,“以臣观之,当以威信素著、通晓兵事者为佳。”

郭逵的要求,韩冈其实觉得有点难办。不能主动推荐,就必须让天子自己上钩,有着很大的难度。最后他只能设法划出一条线,为郭逵量身定做,让赵顼自己来配合。

韩冈眼下将话题移到河北,其实已经隐隐有放弃在陕西纠缠的想法了。如果稳定河北,西北自然可以安心攻打西夏。而稳定河北的人选,只有寥寥数人,想必韩冈绝不是自荐。

威望还是第一条指标,合乎这个条件的,其实人数已经不是很多了,再加上能力来限制,伸出一只手,用上面的手指来数,还是嫌多。能力和威望并重的合格人选,搜遍朝中,就这么几个。

那么究竟是章惇还是王韶?

章惇的资历还是浅薄了一点,去河北的主要目的是坐镇,而不是领军出战,威望远比能力更重要。而王韶在河北也没有任何人脉,虽有军功,但想要压得住阵脚,难度还是嫌大了点。

转了一圈之后,赵顼心中有了数,但他对韩冈突然提起河北,心中还有几分怀疑。该不会是与某人事前商量好的,

“郭逵如何?”他问着,盯住韩冈。而一直沉默的王珪,也将注意力集中到了韩冈身上。

韩冈却是愣了一下,赵顼好歹应该考虑过章惇和王韶之后,才会提到郭逵,怎么先一步就提出了郭逵的名号?大战在即,朝堂上必须要有一个深悉军事的辅臣,除了郭逵之外,还有谁能在关键的时候担当大任。除非章惇或是王韶回归。

“这样一来,两府中不就没有经过战阵的辅臣了?”他试探的说着。就算不能帮助郭逵,完成交换,也总比自己落水要好。

赵顼放下心来,自问知道了韩冈的盘算,笑道:“无妨。有韩卿提点也就足够了。”

韩冈也算是了解到了赵顼的想法。十年君臣,赵顼会怎么想,从结果上反过来推测其实不难。大概会以为自己一开始是准备推荐王韶去河北,之后又有推王韶重返两府的想法。

这样的想法不足为奇,不过最后一句,就实在是太自以为是了。韩冈哪里听不出来,赵顼并不是在说有自己提点就足够了,分明是在说两府中并不需要人指手画脚。

这是个会给前线的将领送阵图的皇帝,喜欢依照沙盘,给前线的将领制定战术。在熙河的那几年,在广西的那段时间,收到阵图和战术规划也有多次,章惇和韩冈没少在奏章中赞美天子的指示卓然有效,然后顺便将其塞进不见天日的架阁库中。

对自己的军事才华十分自信,喜欢掌控一切。眼下正是志得意满的时候,大概并不认为少个郭逵会有什么麻烦,而且还能免得再听他反对速攻,耳根可以清静一点。

郭逵算是如愿以偿了。虽然对于郭逵想去河北的理由还是猜不透,但韩冈也不准备多费神,自己之前给出的借口,其实也是合情合理的。这个老家伙,其实就差了一个进士及第,论才智论能力不比任何人差。有他去河北坐镇,基本上可以安心。

用了一个时辰的时间,韩冈和王珪同时从崇政殿中出来的。虽然整件事没有定下,但只要郭逵本人点头,以枢密副使的身份出掌河北军事,基本上不会有意外了。

王珪在前面走着,出了崇政殿的范围,他脚步缓了一下,偏偏头:“玉昆。”

韩冈随即上前半步。

虽然两人在速攻和缓攻上有纷争,但论起关系来,却不算差。王珪帮过韩冈几个大忙,欠下了人情债,原则问题上虽不能让,但平日往来,韩冈都对王珪很是尊重。

“郭仲通要是知道玉昆你无缘无故举荐他去河北,不知道是愿意,还是不愿意。”

“此乃天子之意。”韩冈叹道。

王珪回头看了一眼,被韩冈的态度误导了,点头道:“王子纯去河北其实也不差,如今留在南方的确是浪费了。”

韩冈是真的在叹气。王韶出外时间并不算长,才一年多,回京的可能性不大。而且从来信上看,他身体还不好,召回来也没用。被御史知道后,说不定会被逼着告病。

王珪并不知道王韶的病情,见韩冈不想提王韶入京,笑了一笑,又道,“玉昆,你可知道鄜延路都巡检王舜臣被人首告其谎报军功?”

