\section{第14章 霜蹄追风尝随骠(14)}

折克仁从厅中出来,在庭院中轻呼一口,浓白的雾气在空中消磨不散。

将领们一个个从折克仁身边经过,相熟的还不忘打声招呼,约个时间一起去喝酒。

在他们的眼中,折克仁看到的是兴奋。为韩冈方才的鼓动而兴致高昂。呵气成冰的冰寒,压不住他们胸中熊熊燃烧的火焰。只有通过热闹的一番酒宴,才能稍稍宣泄出一点他们功劳近在咫尺的兴奋之情。

可如果有人能注意折克仁的神情,便会惊讶的发现,能为自己半只耳朵而冲入辽境的折家十六郎,却是全然的冷静。

前一天还说要以攻心为上,可一旦发现南逃的黑山党项多于预期,便立刻起了杀心。韩冈的态度转变间,潜藏的杀性可是一点也不遮掩。

“很少见这样杀气腾腾的经略。是吧,三哥?”折克仁忽然开口。

“啊,真的很少见。”吱吱的踏雪声响起,折克行从他身后走过来,帐中军议,不经意间已经是一场薄雪下过。在折克仁身边停步,抬头望着比入厅前更为阴沉的天空,“不过十年见过有几分相似的,三十年也见过。”

“是谁?!”折克仁立刻问道。

折克行慢悠悠的道:“十年前的是李复圭。”

“李复圭?”折克仁记不起有这个人,能跟韩冈相比较,或者说与他有几分相似。

“十年前在环庆任职的。”折克行道,“就是瘐死种詠,杀了李信——不是韩经略的表兄——的那一位经略使。连杀两将,其下更有十余军校被治罪论死,人心由此大坏。之后广锐军叛乱的肇因,其实有他一份功劳。环庆军的战力,也因此从十年前开始,就没有缓过气来,灵州之败不是那么简单。”

“这位哪里跟韩龙图相似?”折克仁疑惑道。

“杀性重的地方。”

“李复圭的杀性可没用对地方。”折克仁从鼻子里哼了一声,又问道,“……三十年前的是谁?”

折克行呵呵两声笑,“未足奇的那一位!”

“韩……”折克仁张开口,又闭上了嘴,笑了起来。

这两位的确是杀气腾腾的经略使,不过他们是喜欢杀自己人来立威,或是推卸责任。比如韩琦,比如李复圭。可这两位,对上西贼或是北虏就不成了。

李复圭那个废物就不说了。韩琦偌大的名头,手上够分量的战果似乎就一个算不得好男儿的焦用,把狄青吓得面无人色勉强也能算他的战绩,由此赢得了西夏太师张元的衷心夸赞——韩琦未足奇。

“幸好韩龙图与他们都不一样。他对贼寇满心杀机,对我们这些武夫,倒是优遇有加。”折克仁说着,与折克行一起踏着积雪向外走。

折克行慨叹道:“有人适合在朝堂勾心斗角,有人适合在边疆建功立业。可惜总是被放错地方。”

“韩龙图做经略使不是正合适?这一次朝廷可没放错人。”折克仁道。

折克行摇摇头,“韩冈更适合在朝堂上。他自己也知道。”

折克仁沉吟了一下,笑了起来:“……的确,快刀斩乱麻,韩龙图这一次不想在丰州拖得太久了。”

“为了安定黑山河间地,辽人选择了最简单的做法。把党项人都赶走就行了。韩冈的做法跟他们一样,不管萧十三有什么盘算,用最简单的办法解决。”

“想必十六你也听说了吧,从黑山党项那里,耶律乙辛打算将它的斡鲁朵安置在黑山河间地。萧十三现在所做的,必然是领了耶律乙辛的吩咐。”

“也算是好事,只要耶律乙辛的斡鲁朵在黑山河间地一日,丰州的防守就一日不能松懈。”

“说的可不是这事。胜州需要运输的粮草是有限度的。囤积在胜州粮草也是有限的。这点差事,肯定不足以安置下所有的南下部族。”

折克行回头看着折克仁。

“小弟明白了。”折克仁心领神会的笑道,“修筑可是个苦活,黑山党项肯定没这个能耐,也不会愿意老老实实的去做。下手重一点、狠一点,这才好让他们听话。我想这就是韩龙图的本意吧。”

世人对折克仁的印象就是冲动易怒四个字,可是折克仁他若当真是这么冲动的人,怎么会让他去负责修筑关键的寨防,又怎会让他陪着下一任的家主走到最前线?

就是在身体受残的暴怒之下,折克仁也只是烧了两间巡铺,而不是杀人泄愤,纵使没有韩冈担待,也不会有太重的罪责。而如此一来,看到他的半只耳朵,没人会嘲笑半句,反而都会因为他有着杀入辽境放火的胆识拱手致礼。

“无论韩冈的本意为何,但南下诸部可不能全变成斩首功,至少要把修城修宅的人手留下来。”

“小弟会协助大哥儿做好此事的。”折克仁点头,“能多用上黑山诸部的一份人手,就少动用府州和麟州的一份人力。”

折克行叹了一口气,“这一次的功劳大半要给河东军占去,我们争抢不得。也只有靠十六你和大哥儿。”

…………………………

“这个机会,可不要浪费了。胜州事一了,几年内可能都不会有大仗了。”

李宪出来后,就吩咐着下面的将领,要把握好机会。且不说机会的难得,即将结束的战争,使得立功的机会已经寥寥无几,就是韩冈今天所说的话,换作任何一位边臣,几乎都不会去说。

对于投奔朝廷的降人,大宋一直以来都十分厚待,官职、钱粮从不吝啬。可韩冈这一回一动手,便是拿上万降人的性命来换功劳。纵然找到了借口,可也不是轻易就能搪塞得过去的。

朝廷中,要挑他毛病的可不是一个两个。现在授人以柄,必然惹来一身麻烦。想要为自己辩解,少不了也要大费口舌。

隶属于河东军的将领们,没有太多的想法,只知道尽可能快的完成韩冈的任务,争取更多的斩首,这关系到他们和妻儿的未来。

不肯听命,或是有辽人嫌疑的南下部族。他们是这一次的目标。而河东军的成员,都会将精力锁定在目标上,绝不会分心些许。

…………………………

东胜州的雪越来越大,听着大帐上沙沙的落雪声,萧十三心绪变得十分平静。一切都已经布置下去,就等着最后的收获了。

不须动用大军,只要在合适的时候动手,就能让南面的宋人无可施展,只能硬咽下自己制造的苦果。

“枢密。黑山的消息,那龙部也南迁了。”一名亲兵走近萧十三的身边,低声向他说着最新的军情。

“那龙阿日丁终于也撑不住了?”萧十三惊喜的扬起眉头,那龙部可是黑山大部族中,一贯坚持到底的死硬,不过终究还是胳膊拧不过大腿。他冷笑道,“要不是看在那龙阿日丁能给宋人添些麻烦的份上,早就派人将那龙部给灭了。”

亲兵低着头,不敢多插嘴。萧十三停了一阵后又问道,“这一次回来报信的是谁?”

“是萧孝先太尉的亲信。”

“重赏。赏他十匹绢,二十两银。”

萧十三手上很是宽裕,难得的如此大方赏赐。甚至让亲兵一下就呆愣住了,直到萧十三厉声催促,才反应过来,跌跌撞撞的冲出大帐。

这一次,黑山党项几乎是被连根拔起。从数百大小部族手中,萧十三得到的牲畜、财产不在少数。只是这一切不可能全属于他,参与其中的各军、各部、各族,都有资格从中分上一杯羹,至于耶律乙辛,黑山河间地就是最大的收获,劫掠所得的浮财,他不会与人争夺。

眼下那龙部也走了,黑山河间地再无阻碍,剩下就该等尚父的斡鲁朵迁移至此。土地所有权转变的结果,不仅仅耶律乙辛可以培植本身的实力,还有其他同属尚父斡鲁朵的成员,也将会同样有足够的施展空间。

不过并不包括萧十三,朝廷的高官,不可能去成为斡鲁朵的一员。

萧十三本人知道,没有足够的实力,自己能派得上用场的手段很少。幸好对面的韩冈能用的手段更少。冲突到现在,宋军越境的情况就只有折克仁一次,可见他绝不愿意将战事扩大化。比起大辽来,宋人更怕一场战争。

经过了征夏之役接近一年的消耗,又举兵收复旧地,还要接济南逃的黑山党项,宋人的粮草还能支持多久?一旦不能给付他们足够的粮草,党项人可不是老实等死的类型。

萧十三的双眼眯了起来,宋人最是优柔寡断,又好个中央之国的名声,对降人一向礼遇非常。韩冈肯定会给他们粮、给他们地,将一群饿狼当成家犬养在家中。可是一旦粮草供给不上,这群饿狼可就会从近似于家犬的身份上回到凶狠得要撕碎对手的状态。

那个时候,便是下手的好时机,若有机会,他是绝对不会轻轻放过。

