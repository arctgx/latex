\section{第22章 瞒天过海暗遣兵(七)}

【第二更,求红票,收藏】

木征懒洋洋的一番话,让禹臧家的使者为之气结。

武胜军是河州屏障,若是丢给了宋人,河州定然难保。而对于禹臧家来说,武胜军紧邻着兰州,在唐时,其地便是属于兰州辖下。宋人据有武胜军,向西是河州,而向北穿山而过,可就是禹臧家的兰州城了。

现在两家共同的大敌就是名说着要拓边河湟的宋人,这是明摆着的事情,使者想不通为什么木征还是这副完全不放在心上的态度,仿佛都要睡着了一般。

‘如今国中尽起三十万大军南征,等到梁相公挟胜而归,听说观察竟然不肯受命,一怒之下,河州城必然无存!取舍与否,还请观察速决!’

使者很想这么说,但他不敢。他清楚,在木征面前最好还是保持的谦逊一点的态度。总是半睡半醒、凡事都不在意的木征,并不是好脾气的人。真的惹火了他,直接斩了使者的先例也是有过的。

而木征却是从眼皮缝中,玩味着禹臧使者气急败坏的神色。当年的结吴叱腊,还有董裕,都曾在他面前露出这副恨铁不成钢的表情。

木征并不是只能看到眼前一亩三分地的愚人,唇亡齿寒的道理他也懂。单看他能在众敌环伺的河湟中心安坐至今,这鼠目寸光这个词就用不到他身上。

谨守河州,不是木征没有胆略,而是他有着自知之明。木征知道自己地手有多大,能抓住多少东西。贪求得太多,反而原有的都会丢掉。生存在夹缝中,不小心谨慎所人,下一个倒下的可就是自己。木征的信念始终如一,仅仅保住河州而已,至于其余,他都不会去贪求。

而且宋人纵然咄咄逼人,但西夏何尝不是?李元昊从他的祖父辈起就没少跟吐蕃拼杀过。河西凉州的六谷联盟,就是被党项人所灭。而为了稳定河西,李元昊又提兵南下,不过被木征的祖父、也就是前任的赞普唃厮罗打得溃不成军。这一战,是李元昊起兵之后,败得最惨的一次——虽然日后李元昊还败给过契丹人,但他后来又讨了回来,不比对吐蕃,到最后也没能报仇雪恨。

而眼前的这位禹臧家的使者,也让木征无意跟他深谈。背叛了吐蕃,投靠了党项,禹臧家在木征心目中的地位,可是狗都不如。投靠汉人倒也罢了,毕竟跟汉人们都打了几百年的交道了,但跟着党项人,却是丢尽了吐蕃人的脸面。木征自负是吐蕃王家嫡传,可没兴趣跟党项人养的狗打交道。

用着懒洋洋的态度,打发走了怒气冲天的禹臧家使者。木征想了想后,便叫来了自己另外一个同母弟弟结吴延征,“你带本部去武胜军帮一下瞎吴叱。若是汉人不光是在渭源筑城,还转着攻打狄道的主意,就一起把他们打回去,不能让他们占了大来谷。”

结吴延征愣了一下,他没想到会被交托这个任务:“若是没打过来呢?”

“那就该做什么做什么,你跟瞎吴叱要块地住下来就是。”木征慢吞吞的说着,“瞎吴叱在岷州有块地,现在他到武胜军了,那块地你向他要过来,也好安顿下你的部众。”

结吴延征原来是满心的不情愿,但听说终于能拥有一块土地,他立刻兴奋得跪下来磕头。

“还有,”木征一直眯着的眼睛倏然睁开,单眼皮下的一双小眼锐利如电,提醒着叩头不已的弟弟,“也要小心北面!”

……………………

晨光尚未泛起在东方,天地之间,仍是一片黑沉。九月朔日的天空,没有月亮的痕迹,镶在天穹上的密密麻麻的星光,加起来也比不上明月时的一星半点,只是,已经可以让人看清前方的背影,紧紧追随而不会落队。

黑暗之中,一支多达一千五百人的队伍,正静悄悄的行走在山谷之中。人衔枚,马裹蹄,笼头和嚼子紧紧锁住了战马的嘶鸣。伴随着潺潺的溪水,只有密集而又低弱的脚步声连续不断。

苗授与他手下将士们一起牵着马穿梭在黎明前的黑幕下。脚下的路面并不似官道那么平整,但也是商人们经常使用的要道,至少不会让人举步维艰。

低着头走了不知多久,苗授抬头看了看天色,还是黑沉沉的,看样子至少还要半个时辰,才能见到东方天际处的一抹红光。

在黑夜中行军,是一件很冒风险的行动。不过苗授并不怕夜袭,老于兵事的他,早在三天前就陆续派出了足够多的哨探,去检查沿途每一处可能藏兵的地点,并驱赶来刺探的蕃人。现在这些哨探,有一部分带着消息回来了,还有一部分则听着他的命令,在各处要点守候着。

最关键的,王舜臣和苗履已经领着一个都的骑兵,在通往星罗结部的要道处守了四天的时间。他们并没有掩饰行踪,更没有躲藏,几天下来与星罗结部的蕃骑几次对峙。苗授这是用最强硬的态度在赌别羌星罗结不敢破釜沉舟——只是找借口

而王舜臣和苗履的手下只有一个都的数目,也让别羌星罗结不会太过紧张。当看到王舜臣所部连续几天都没有动静的情况下,即便别羌再狡猾,也只会误会这只是用来防止星罗结部偷袭渭源的措施而已,一开始的紧张便会松弛下来。

谁能想到这是,这是为了渭源出兵的掩饰?放弃筑堡而突袭蕃部,这完全不符合宋军过往的惯例。突如其来的奇兵,这是苗授自信能成功的底气。多管齐下,以有心算无心,苗授对自己今次的作战有着百分之百的把握。

一名哨探急匆匆地自前方赶来。他从苗授身边高高举起的大纛留在夜色中的剪影,以及苗授的亲卫所骑乘的、比寻常骑兵战马都高出两寸三寸的河西良驹身上,辨认出了苗授所在。他在外围通报过姓名,被亲卫领到苗授身前,“都巡,前面就是大来谷。”

终于到了!

苗授松了口气下来,他于四更天不到,便自特意设在渭源西侧三里的营地领军出发,走了一个多时辰后,终于抵达了十里外的第一站。

大来谷是沟通渭源和狄道之间的要道。从渭源堡到狄道,要翻过鸟鼠山这座分水岭——东面是渭水,西面则是洮水——而鸟鼠山中,有一条谷地直通东西,这就是大来谷。

尽管大来谷的南面,还有一条名为南谷的谷地,也能沟通渭源和狄道。而在鸟鼠山中,还有好几条可供行走的山道。但从地势上,以及路程上,还是以大来谷更为优胜。大来谷作为洮州的东侧门户,一向是兵家必争之地。唐时开元年间,唐军曾在大来谷一战击败屯兵在谷中的十万吐蕃大军,逼得来犯的吐蕃军逃回洮水。

而今次的任务,并不是要穿过大来谷——这条谷地不是那么容易就能通行,在对面的谷口,有吐蕃人的一处军寨。小股人马会被堵住,若是有大军穿谷而过,则必然会引得木征警觉起来——苗授的目标是星罗结部的聚居地,位于大来谷之北,白石山下。如果急行军的话,最多再有一个半时辰,就能抵达星罗结部主帐所在的谷地。

但苗授并不打算夜袭,要是他想捕捉的对象趁黑跑掉就麻烦了。选在下半夜出发,以他行军的速度,抵达星罗结部时正是白天,可以有更多时间作战。要利用夜色,反而应该在黄昏时出兵。

“就地休息一刻钟。”苗授将自己的命令传到队列中。辛苦了小半夜的士兵们也不多话,纷纷坐下休息,吃点干粮。

而苗授仍站着,只是转着脚,活络一下有些酸胀的脚踝。心中又一次将今次作战计划从头到尾理了一遍,这个从沙盘上制定出来的计划,除去开头时的瞒天过海的伎俩,剩下的的就只有以快打慢一条。

星罗结部是典型的吐蕃部族,分据在几条山谷中。虽然跟普通的蕃部一样,只要是成年男子都能上阵拉弓,让星罗结部可以拼凑出五六千兵力,但这样的军队并不可能枕戈待旦,平时都是分散开来,各自放牧做活。以蕃部的组织松散,就算现在听说宋军已经抵达大来谷,给别羌留下的半日时间,最多也只能让他召回一千多一点的部众。

这是个很简单的策略,在作战开始后,就没有了任何计策存在,但在苗授看来,却已经足够了。

因为简单,所以易行。

休息片刻,苗授便起身急行向北,直扑最终的目标而去。当苗授所领大军出现在星罗结部谷口外的时候,谷地中惊惶一片,号角连声。顺着初起的北风传来号角声中,满载着惶急和不安。

九月初一,苗授大破星罗结部,斩杀别羌星罗结,斩首四百余。王韶得到捷报,随即从渭源堡赶到星罗结部的主城所在。当王韶褒奖过参战的将士,领军回到渭源堡的两天后,出现在堡外的,不是来自古渭的贺功信使,而是多达五六千人的大军。而他们所举着的旗帜,也不是吐蕃人的风格,却是明明白白的西夏战旗!

禹臧!

