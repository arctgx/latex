\section{第33章 旌旗西指聚虎贲(二)}

【迟了一点,很不好意思】

离着酒场渐近,一股酒糟味便扑鼻而来,近于腐败的臭味直透囟门。王厚喜欢喝酒,但他绝不会喜欢到酒场闲逛。但韩冈偏偏挑了这件事来做,自从回到通远军的这几个月来,没事就跑酒场里去。还弄出了什么蒸馏锅,用来蒸酒。

走到了酒场门口,王厚翻身下马,空气中传来的不再仅仅是浓烈刺鼻的酒糟味,还有韩冈饱含怒意的训斥,“这酒精是用来外用消毒的,不是给你们喝的。好不容易才出了几十斤,转过眼来就没了?我说你们啊……一个个都是官人了,怎么还做这等投机摸狗的事?!”

王厚连忙进门,只看到傅勍为首,王舜臣、苗履,还有几个将校,都站在韩冈面前,低头挨着训。

韩冈不论是在河湟还是横山,都是屡立功勋。虽然官位还差一点,但在军中已是积威深重,现在的缘边安抚司,越来越多的人对他又敬又怕。一发起火来,就算最亲近的王舜臣,或是年纪最大的傅勍,都不敢稍膺其锋。

“怎么了?……发这么大脾气?!”王厚的印象中,韩冈很少会这般发火。

“还能什么?给疗养院准备的酒精,好不容易酿出来的,全都给他们偷了去!”韩冈回头,怒意不减。但看到是王厚,却惊喜的站起来:“处道兄你都回来了。”

有了王厚打岔,王舜臣等人缓过气来,他上前涎着脸笑着,“三哥你弄出来的蒸酒喝过,别的酒就是跟水一样,怎么都喝不过瘾?本只是解个馋,谁想到一不注意就喝了这么许多……”

“你们喝得太多了!”韩冈回头又训斥着。

王厚在离开前,也曾尝过了一点蒸酿过的烈酒,给他的感觉并不好,“玉昆弄出来的酒精,烧得慌,喝一口就像着了火,你们怎么还喝?”

“是啊,我给这酒精起个了名字叫烧刀子,喝下去就是烧过的刀子在戳肚肠。”韩冈冷冷的笑了一笑,脸色突的一变,声色俱厉,“万物生长都要阴阳调和,孤阳不长,孤阴不生,人也不例外,无论阴气阳气,哪边重了都要伤身体的。伤口感染溃烂,便是阴气染疮所致。酒是至阳之物,所以用来祀神驱邪,喝起来也暖身。不过原本的酒因为水多,阳气不算充裕,所以我才会让人蒸酿酒水,蒸出酒精来清理伤口。可酒精阳气过重,也只能外敷,用来清洗伤口没问题,但喝下肚子,会烧肝烧胃,坏了身子。”

韩冈冒充医道高手已经冒充了很长时间,别看他一直不肯承认药王弟子的身份,但编起话来却是一套一套,而且一点也让人戳不出破绽。活灵活现,宛如真的一般。

他再一瞪眼,扫过面色如土的几人,狠狠的说着:“以后喝出病来别来找我!”

王舜臣、傅勍他们担惊受怕的被韩冈撵走了。而王厚也被吓住了,扯定韩冈:“玉昆,你说的都是真的?!”

他惊问着,看到韩冈方才一脸认真,心中已是打定主意,以后还是少喝酒为妙。

“半真半假,只要不多喝,其实也没大碍。但不这么吓他们,迟早就给偷光掉。”韩冈摇摇头,他可不喜欢喝烈酒,想方设法让下面的工匠弄出蒸馏酒来,也是为了清洁伤口,保证疗养院中的医疗,不是让人喝得。但没想到,还是被几个酒鬼盯上了。若只是偷喝一点倒罢了,但傅勍和王舜臣却是一次几乎给偷光掉,韩冈哪能不暴跳如雷。

“不过这酒精……还是叫烧刀子好一点。喜欢的人不少,如果真的暴饮后才会有大碍,那拿出点散酒来卖也没关系。而且,玉昆你看……”王厚指了指脚下的酒坛,“这一坛酒大约十六斤,装酒精一坛,装普通的酒水还是一坛。但运送起来就不一样了。一坛烧刀子运到地头,只要兑上水就是三五坛出来了,相对于那些淡酒,省了多少运力出来?三五倍啊!”

韩冈发楞,他没想过还有这等说法,他清楚在苦寒之地,烈酒比过去的淡酒肯定会更受欢迎,不过再受欢迎,也不一定能弥补蒸酿过后、酒液浓缩的损失,直接卖淡酒反而更赚一些。

不过他没想到王厚能从物流费用上打主意。物流的确是困扰现在这个时代的难题之一,运输通道不畅,也是困扰大宋政府攘外安内的重要因素。

可是王厚的提议,对他韩冈、对缘边安抚司,又有什么好处?

通远军因为要保证粮草供给的缘故,酿酒是很少的,韩冈辛辛苦苦,弄出来的蒸馏酒不过是几十斤上下,勉强能装满三四只十六斤重的坛子罢了。也只有其他位于蕃区的寨堡,才会向蕃人贩卖酿出的酒水,这是边地军州最为重要的收入之一。

如果要私酿赚钱,更是不可能——酒水专卖的制度,在内地也许管得很松,但在陕西缘边,却是禁令森严,容不得有人违背。

“难道不能是由外地向通远军运酒?”王厚笑着韩冈的疏忽,这是很难得的情况,“原本要三车的酒,现在只要一车就够了。那样难道不方便?”

“那还要先把这个蒸酒的方子传到外面去。再让人把蒸酒的作坊搭起来。我们还有能有多少时间?”韩冈反问着。

看着王厚张口结舌,韩冈不为已甚,笑了笑,“还不如想想能不能赶在开战前,让缘边安抚司正式升格为经略安抚司。这可比运酒重要得多。”

“难说……”听到关心的话题,王厚把前面的话顿时丢到了一边去,“今年是不可能了,就不是到明年夏天总攻前,能不能让家严如愿。”

河湟之地转为经略安抚司,从秦凤经略司独立出来,这是自王韶一下,每一个缘边安抚司成员的梦想。如果能成为关西的第六个经略使路,以王韶的身份,他将能顺理成章的晋升为经略使,而他之下的官员,也将随之水涨船高。

“不过这个前提是夺下武胜军。现在只有通远军一地,安顿一个缘边安抚司只是勉强,如果有几个州一级的区划,这样才好组成一个经略安抚使路。”王厚又对着韩冈问着,“玉昆,你说是不是?”

韩冈这时正在叮嘱酒场的管事,让他重头开始蒸馏酒精,并让他小心提防,不要再被人偷了去。

拉着王厚出门,他才继续说起方才的话题,接着王厚的话头,“而且通远军最好也要由军升州。从编制上,没有一个经略使路的治所会放在一个军的位置上,至少得是州。而当下的通远军人口还不足,不到万户,升为正式的州还是很勉强。就算天子和政事堂特别批准,阻力也很大。我们这边必须要先配合起来,不然事情会很难办。”

“说的也是!”王厚点了点头,走出门外,跟韩冈一起翻身上马。却是一眼瞥到路边走过的一名应该是厢军的小卒。愣了一下神,却又兴奋得叫了起来,“厢军!”

他返身过来对韩冈叫着,双眼亮得像是捡到了宝一般:“将兵法不是已经在关西全面推行了吗,朝廷可是要开始汰撤厢军了!”他愈加的兴奋,“光是陕西要汰撤的厢军听说都有三四万之多,要是其中能有十分之一转到通远军来,户口数转眼到了!”

靠着韩冈的争取,流放来的两千四百多户叛军,让通远军一下多了一半的户口。虽然暂时没有把他们编组成军,但光是组成保甲,就已经让渭河沿岸的屯田点防御力大大增强。前些日子就有了厢军要汰撤的消息,而且多达三四万。当时没放在心上,但现在想起来,却让王厚兴奋得无以名状。

“看看粮食吧,打一仗后还有多少存粮?”韩冈摇着头,当头一盆冷水,“厢军实边,那是之后的事了。现在别说弄个万儿八千,就是三五千户,再勒紧裤腰带都赶不上粮食的消耗。”

关于平定河湟一系列的规划,韩冈全程参与。攻下武胜军和彻底解决河州木征两个阶段的用兵,之所以要跨年度,就是因为粮食不敷使用。

攻打木征,要等到明年五月。是准备先用存粮开战,然后等新粮上来补足,时间掐得很紧。如果有足够的粮食,那直接就能平推过去,到明年开春就可以总攻了。

可惜行军打仗,一切取决于粮食补给。再高明的将领,都没办法变出粮食来。无论是王韶还是韩冈,虽然都算是在军事上有所才华,但身处偏僻荒凉的边疆,出产难抵消耗,都必须精打细算的来过日子。韩冈有时都在想,以他现在善于节约的水平,回到家中,能把家计开支省去个四五成都没问题。

被冷水浇过,王厚冷静了下来。的确,粮食是困扰着河湟开边的最让人头疼的问题。如果没有这条束缚人的绳索,说不定现在王韶的帅府行辕已经摆到了河州城中。

韩冈看着王厚变得愁眉不解,突然说到:“王中正要来了。”

王厚刚刚回来,听得这个消息,当即吃了一惊,“他来做什么?!”

“监军!”

靠着在罗兀城的功绩,轻松的击败了最大的竞争对手李宪,从诸多竞争者中脱颖而出。御药院都知王中正,他现在来河湟做监军,就是为了分上一杯羹。

