\section{第15章 前路多坎无须虑(五)}

【昨天夜里家里断网了,这第三更只能拖到现在上班后来发,今天不知家里的网络能不能修好,如果少更的话,还请见谅,俺会在后面补偿的,除了补回少更的章节,另外再加更一章。】

虽然韩冈说得好听,但韩阿李却听出了问题:“三哥,你是个官人,在古渭那个偏僻地方弄块地下来是不难,让俺和你爹两个搬过住也不难。但地谁帮着种?总不能要你爹再下田吧?那里可找不到佃户。”

要种田,罪犯,厢兵都可以。本来要屯田,他们这些人力就都得要用上。在开垦官田的时候,顺便让他们带一手,也不是什么难事。不过韩冈觉得此事还是不要明说的好,这世上许多事都是能做不能说,传扬出去就麻烦了。

他笑着对韩阿李道:“这件事孩儿自有办法,地也能种得,也不会让爹爹再下地吃苦,娘娘你就放心好了!”

韩阿李看了看儿子脸上自信的笑容,却哼了一声,又拿起鞋样对比起来,不冷不热的说着:“是啊,三哥你算好的事,娘是从来都不用担心。娘现在只想着一件事,三哥你什么时候给娘添个孙子?”

“看娘你说的,孩儿还没娶妻,怎么给你老人家添孙子?”韩冈笑得发干,看看门口,就想抽个空逃出去。韩阿李想抱孙子快想疯了,只要在这事上提上一句,韩冈接下来不被念上一个时辰,就别想她能停嘴。

韩阿李一瞪眼:“那个现在关在大牢里的窦七衙内,不也是没娶妻吗,还不照样有了儿子!?虽说是被人治死了,但有了就是有了!”

“娘说的是,娘说的是!”韩冈猛点着头,忙不迭的附和着。他在外面,就算见着王安石时,都没这般低声下气过。

但韩阿李还是不肯饶了儿子:“三哥!你说没娶妻,生不了儿子。可现在家里媒人来了一个接一个,只要想娶,你点点头就行,人家嫁妆全都准备好了。可你倒好,是推了一个又一个。你还在磨蹭个什么?王机宜不是说帮你说门好亲吗,怎么到现在还没个消息?!”

韩冈被暴风骤雨的一顿好骂,几乎不敢抬头,只是听到最后一句,才精神一振,“好叫娘娘放心,王机宜那边已经有消息了,孩儿过来,就是说这事的!”

韩阿李一听,脸上顿时多云转晴,但很快又怀疑起来,“真的假的,三哥别为了糊弄过去骗娘。”

“孩儿怎敢?”韩冈陪着笑脸,忙把王韶介绍的女方家事竹筒倒豆子一般的全都说了出来,生怕韩阿李心急起来,再训上他一通。

说起来,这事本是应该王韶这个媒人来跟韩千六夫妇提才对,韩冈根本就不该插手的。父母之命,媒妁之言,中间根本没新人的事。但王韶那边忙得把事情耽搁了,韩冈为了耳根清静,也不介意自己来说。

王韶的内侄女,又是德安大族家的闺秀,家世配上韩冈绰绰有余,还能与靠山王韶联系得更加紧密起来,不论人品相貌,只看身份,的确是门好亲——而人品相貌,韩冈也不担心,王厚拍过胸脯,王韶也不会找个不像样的过来,惹得自己的得力助手离心。

只是韩阿李听了后,却皱起眉头,“怎么才十三岁?就算明年嫁过来,要生小子,说不定也要等到两三年后。”

韩冈到没想到,自家老娘对儿媳妇的好坏判断,全都放在能不能生孙子上了。虽然两个哥哥都不在了,韩家在关西的这一支只剩他一个独苗,但也不至于急成这样吧?韩冈觉得这样的想法他能够体谅,却难以理解

韩冈其实真不急。如今的世情虽然都是早婚得多,正常就是十四五,过了十八就算迟了,但士子却是特例。读书人晚婚是很常见一件事,范仲淹成亲时据说已经三十多岁了。王韶成婚也是在冠礼之后。王厚现在二十了,不见王韶逼着他成亲。

而一般的寒门士子,在婚姻上高不成低不就,更是容易拖时间。娶名门闺秀他们不够资格,让他们放下身段,去找普通百姓,他们也不甘心,就这么一年年的蹉跎下去。如果他们不能考上进士,或是通过其他途径得个官身,往往要拖到三四十岁,婚姻大事都决定不下来。韩冈都听说过,五六十岁的光棍进士那一科都没少出过。

韩冈觉得自己才十九岁,迟个一年也没关系。可韩阿李却心急抱孙子,传香火,“三哥,婚事就任你拖去,娘也不再催你。但今天娘要做个主,你把素心和云娘都纳了做小,到明年就得给韩家添个后。”

韩冈听了当即叫起苦:“娘!哪有还没娶妻,就先纳妾的道理!”

“谁说没有!在河西大街上开质库的李大户家的两个儿子,前街刘药铺家的大哥,不都是十五六就纳妾,过了两年才成亲的?”韩阿李重重的一拍床沿,怒道:“这也不成,那也不成,素心和云娘哪里不好了,你还推三推四,拖来拖去,是不是想气死娘不成?!”

“娘,你先消消气。”韩冈心中喊冤,他哪里拖了,只是前段时间忙得脚不沾地,现在虽然清闲了,又为了考个进士,把精力放在书堆里,好肉一时忘了吃。不过收房没问题,纳妾却是有些不好办,“他们能做,孩儿不能做。这样不合礼法。”

“不孝有三,无后为大。说什么狗屁礼法,孝你讲不讲了?!”韩阿李只当儿子还在拖延,指着韩冈的鼻子,“家里的两个,哪个不是美人,哪个心思不是放在你身上。就你个瞎眼的,天天在书房里之乎者也的拽酸文,你的聪明都用到了哪儿处去了?!读书都读傻了!”

她啪的一声再一拍床,“这事娘做主了,你不好娶妾,那也就先停一停。但收房三哥你还有什么说的?云娘年纪小,等明年满十四了再说。素心那里,你就快一点,不要耽搁了!若是到了七夕,素心还梳着丫髻,娘可不管你是什么官人不官人,照样打断你的腿!”

‘哪有这么仓促的?!’韩冈心中叫苦,却不敢再回嘴。外面的对手再强,韩冈也有自信与他们周旋一番,但对上自家不讲理的老娘,他却是什么手段都没法儿使。这件事上,他虽然本是有心,可被人像种马一样催着,反而弄得都没心思了。

在韩阿李面前,陪了一箩筐的好话,韩冈觑了个空,终于逃了出来。只是刚走出门,他的脚却停了。严素心端着个托盘,上面放着两杯凉茶,脸红红的就站在外面,低着头不敢看韩冈。而在她旁边,韩云娘则抬头看着他,一对如潭水般清澈的秀眼中,有着希冀和恋慕,也多了这个年纪不该有的幽怨和不安。

韩冈不知道两女究竟在外面站了多久,但看她们的模样,该听的应该都听到了。气氛变得很尴尬,没有人开口说话,韩冈咳嗽了一下,想缓和一下气氛,但却是一点用也没有。

这下该怎么办?

让人窒息的沉默中,韩冈摇头叹了口气,眼神变得锐利起来。这事何须纠结,依着本心,放开手去做好了。犹豫不决这个词,不该属于自己。

上前一步,韩冈抬手抚过云娘细嫩的脸颊,柔滑的触感从手上传来。十三岁的少女光洁细腻的皮肤犹如最为上品的瓷器,而柔软而又富有弹性,却又是瓷器所不能媲美。韩冈对这种感觉爱不释手。他弯下腰贴在小丫头的耳边,柔声问道:“在想什么呢?”

韩云娘摇了摇头,没说话,小巧挺翘的鼻梁下,略凹的双眼更显得如春水汇成的深潭。一双清澈的眼睛还是不离韩冈。

“我都说过不用担心了吧?”韩冈笑了,他知道她在害怕什么。小丫头从小就被卖到家中,历经坎坷,心思本就是早熟。如今她一颗心都放在了自己的身上。而随着自己的地位越来越高,她也就越来越不安起来——一开始她还有着童养媳的身份,现在却连个妾室都还不是,这能不让她担心?

“用不着担心,耐心等着就是了。我做的保证难道还不能信吗?”紧紧贴在耳边说出的话语,有种奇特的说服力,韩冈柔和却坚定的声音传入耳朵里,韩云娘眼中的幽怨和不安就一分分的逐渐消退了。

官宦人家的婢女、歌妓甚至侍妾,被出售、被转赠的情况有很多,如今的世情,让韩云娘心中始终缺乏安全感。如果她没有喜欢上韩冈,也不至于总是处于惶惶不安的情况,但现在一颗心早已失陷,却免不了有着患得患失的心情。

不过小丫头的心思还是单纯,韩冈的一句承诺,就能让她很长一段时间里都不用担惊受怕。她很郑重的点头,“云娘相信三哥哥!”

当韩冈放开抚摸着云娘小脸的右手,转向严素心的时候。她的身子就是一颤,手中托盘上的杯盏一下都翻了,撞在一起叮当脆响,酸梅汤全都淌了出来。韩冈饶有兴味的看着她心慌意乱的模样,带着调笑的口吻:“今天夜里的夜宵是什么?”

