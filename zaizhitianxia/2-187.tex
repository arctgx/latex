\section{第31章 战鼓将擂缘败至(五)}

八牛弩在历史上的最大战果,就是真宗朝在澶州城下,一箭射杀了入侵大宋的契丹前军统帅萧达凛,直接摧毁了得领军的承天太后萧燕燕将战事继续下去的意志,从此便诞生了维持宋辽两国七十年和平时光的澶渊之盟。

一个改变了历史的神兵利器,的确让人赞叹不已。放在这件神兵利器上的箭矢,是一种特制的五尺铁箭,除了铁质的翎尾,其形制和大小与一柄长枪一般无二。在床弩弩身上,有着三条刻槽,也即是说可以一次并射三支铁枪,故而也被称为一枪三剑箭。

被一支支放入刻槽的铁枪很有些年头了,上面还带着斑斑锈迹,但钝重的枪头,看得就让人不寒而立。根本不需要打造出锋锐的矢尖,只凭其被射出的威力,就足以将挡在箭锋去路的敌人串成肉串。

韩冈围着在城头上被组装起来的八牛弩转了一圈,在眼下的这个时代,在威力上的确挑不出毛病。但就是需要的人手好像多了点,几十人围着一张床弩。如果是换作射出同样威力炮弹的火炮,并不需要这么多人。如果能把火炮造出来,上了战场的大宋军队,当是要轻松许多。

用热兵器来解决冷兵器时代的对手,是韩冈梦寐已久的一桩美事。若能装备上足够的火炮和火枪,也许今后的战争,就会像西班牙人毁灭印加帝国那般轻松。不过这要等自己有了足够的地位,能掌握兵械制造这个职司后,韩冈才会把这项发明拿出来。如此巨大的功劳,他完全没有分给别人的意思。

所以现在,就只有让八牛弩来充当战场上的最终兵器。

在飞骑掠城过后,党项人的步兵已经穿出了混乱的烟尘,密集如蚁的浩荡.声势,一眼望去,就知道近乎有万人之多。而在蜂拥而来,队形比宋军要混乱得多的步跋子后阵中,一面大纛高高的挑起。

“竟然是都罗马尾!”韩冈身侧的种朴惊声叫道,声调中带着狂喜的颤音。那面旗帜上的字号,有过前段时间的交往,罗兀城上下都很熟悉。

党项人与大宋交战多年,当然知道宋人床弩的威力。在今次的围城中,只是对罗兀城稍作试探,就安坐下来静静的围城。而领军的将帅也完全没有进入床弩的有效范围之中。同样的,罗兀城这边也是因为西贼没有,也便把八牛弩这样的重型兵器当作杀手锏而收藏起来,并没有使用。

想不到今天为了能攻下罗兀城,西贼的都枢密都罗马尾,竟然把他的将旗移到了八牛弩的最佳射程之中。

狠狠地盯了一眼不知死活的都罗马尾,韩冈又立刻向南面望去。

眼下的两处分战场。一处是高永能在外率领主力在堵截敌军——实质上是要趁机要从梁乙埋身上要下一块肉来。另一处则是攻来罗兀城的都罗马尾。梁乙埋派他出来攻城的用意,无非是要动摇高永能的军心,但在事先都已经有所预案的情况下,这也只是痴人做梦而已。

在阻截西贼骑兵追击己方车马的同时,也同样被阻截在城外的高永能,已经在环卫铁骑和铁鹞子的轮番冲击中,顺利的把阵型调整完毕。

宽达百多步的坚实阵列,将河谷的最狭处彻底堵上。密集的箭雨让党项骑兵难以越雷池一步,要突破宋军箭阵,就算是强如契丹铁骑也只有两个办法,一个是绕路,第二就是用轮番进攻来冲击敌阵,不是为了冲散,而是为了拖垮。

党项人明显的在采用第二种办法,但这就是要靠人命来消耗。高永能得意的摸着胡须,契丹人也只是把宋军战阵四面围困起来后,才敢玩这一手,远比不上契丹人的西贼,竟然敢东施效颦。见着一名名精锐的党项骑兵在箭雨过后落马坠地,高永能看到的都是叮叮当当掉到手上的赏赐。

刚不可久,都罗马尾收回投向南面战场的目光。他很明白,那样不计伤亡的冲阵不可能持续太久。如果他不能在短时间内改变这里的战局,南面的精锐骑兵必然会失去继续进攻的锐气。相对的,一旦大夏的白色战旗能飘扬在罗兀城头上,那南面的出城宋军,则会当即崩溃。

‘罗兀城当在我手里夺回来!’都罗马尾用力盯着飘扬在罗兀城头上的宋军大旗,恨恨地想着。“把城里的汉狗给我屠光!”他疯狂的叫着。

在号角声中,步跋子们纷纷嚎叫着,向城墙冲来。云梯、壕桥车被纷纷推着上前。

长长的木板架在四个轮子上的壕桥车,拉到濠河边后再用力向前一推,一辆辆四轮车,顿时就成了架在三丈多宽濠河上的座座桥梁,而四只轮子就正好是卡住濠河两岸的桥墩。

不费吹灰之力就解决了濠河的阻碍,看起来,至少在围困罗兀城的这段时间里,党项人并不是干坐着。

城头上,张玉半眼也不看冲到城下的敌军,只是指着几张八牛弩,转头问着韩冈,“玉昆。你觉得射哪边比较好?”

韩冈知道张玉的心意,他轻笑着回答:“挽弓当挽强,用箭当用长。”

张玉哈哈大笑,紧接着把下两句念了出来:“射人先射马,擒贼先擒王!”他双目一下圆瞪,大喝一声,“把箭给我冲着那面大旗下的人射去。”

服侍着六张八牛弩的士兵们领命调整了射击的角度,举着木槌,用力的狠狠砸下。

咚咚的几声响,六张床弩的弓弦于瞬间绷直,甚至没有一丝颤抖的尾音,就这么一眨眼的时间里,从弯曲到极致的形状变成了一条直线,而架在弓槽中的铁枪也在这一瞬间,离开了原位。

十八支铁枪自城头上破风而下,此时的都罗马尾却正在为他的兵顺利冲到城下而欣喜如狂。数线飞速掠动的黑影在眼角余光中留下了深深的阴影,他心中一惊,猛抬头,只见着一点乌光直扑双眼而来。

十八支铁枪各自有着各自的去处。有半数直接撞进地里,有几支将骑手和战马牢牢的连在了一起,而其中有一支,也只有一支,则准确的命中了目标,直接撞上了都罗马尾面门。

坚固的头骨、沉重的头盔,在飞速而来的铁枪之前,像鸡蛋壳一般脆弱。五尺多长的铁枪扎进都罗马尾的头部,并不是简单的穿透,而是像一柄冲击着城门的攻城锤,将蕴含在其中的猛恶力道传递进了前方的阻挡物中,让西夏国的都枢密使脖子上的部分,如同落到地上的西瓜一样爆碎开来。精铁头盔四分五裂的被弹开,红色和白色的瓤子溅了一地。

失去头颅的身躯犹安坐在马上,从海碗大的创口处泵出的血液如同喷泉,击碎头颅的铁枪仍固执的继续飞下去,擦着战马的后臀,深深的扎进地里。被铁枪带去了一大块臀后皮肉的战马嘶叫着,载着都罗马尾的尸身,在大旗下疯狂的奔跑、跳跃,最后一头撞倒了无人扶持的大纛。

大纛缓缓落地,在都罗马尾的战马蹄下,金白色的将旗被踩进了泥地中。无头的身躯,依然在马背上僵直着,代替了大纛,成了最为醒目的一件物体。

战场了有了那么一刻的静默,紧接着,万胜的欢呼声轰然响起,震得天地间一阵颤动。

种朴右手握拳,用力一锤掌心,疯狂的叫了一声‘好!’。而城头上的一众将校,也在纷纷把自己心中的兴奋狂叫出来。

一击绝杀敌军大将,这份战果比起预计的结果还要好上十倍。都罗马尾的身份,罗兀城中无人不知。一位都枢密的性命,足以抵得过一千名西贼的首级,就算是东京城中的天子也不能奢求他们取得再高的战果。

但韩冈在张玉的脸上,却能看到很明显的遗憾。

“实在是太可惜了。”张玉喃喃自语的声音,随着风,飘到了韩冈的耳朵里。

韩冈也是深有同感,的确是太可惜了。

眼下的局势跟当年澶渊之盟前的契丹入侵有些相像,同样是床弩击杀敌军大将,如果不是庆州兵变,罗兀城的战局恐怕便能就此而定了。但现在,却还是改不了弃守罗兀的最终结果——除非死的是梁乙埋。

而张玉的遗憾不止这一点,他先一步派出去的骑兵,其实在预定的计划中,还会抄小道绕回来作为奇兵,但现在却是毫无必要了。

因为本在猛攻高永能的党项骑兵已经溃退了,而已经冲到城下的上千名步跋子,则还处在混乱之中。城上等候已久的守军齐齐在城墙上探出头来,开水热油,石灰檑木,再加上一支支利箭,疯狂的向城下撒去。

惨叫声冲天而起,油炸后的肉香在城下飘荡,原本让党项步兵快速过河的壕桥,现在被数百张神臂弓锁定,无一人能从桥上逃走。而跳进深壑一般的濠河中的士兵,更加容易成为利箭的目标。

残存的环卫铁骑和铁鹞子已经回到了出发点,没有穿越濠河的步跋子们也终于退回到了安全的地方,城下的惨叫声渐次消失。

“把伤员送出去吧!”张玉这时下令。

南面的城门再次打开,城中的最后几十辆马车,满载着伤员,在一个骑兵指挥的护送下,光明正大的离开了罗兀城。

他们走得毫无顾忌,就算梁乙埋看破了其中的问题,但在士气尽丧的情况下,他也不可能再派兵出来追击了。

‘接下来,’韩冈想着,‘就是自己该怎么回绥德了。’

