\section{第37章 相叹投残笔(下)}

韩冈在院子中来回踱着步。

他个头髙、步子大,寻常人要走十步的院子,他五六步就走到墙边,一下转回来,又是五六步跨到对面。

在院子中这么来回转着,眉头紧锁的样子,就是七月连遭京府知县、朝中御史的弹劾时,都没有出现过在他的脸上。

韩冈为人深沉,喜怒皆少形于色。心比山川,胸如城府。若是在平日,根本就别想看到他坐立不安的模样。可一旦事关至亲,这心头的烦躁焦急怎么都按捺不下去。

王旖的身子在比预产期拖了十天后,终于有了动静。上午正在说话的时候,突然就是有了阵痛。

听着房中一阵一阵穿出来的嘶喊,韩冈知道王旖在里面已经痛得死去活来。

从京中请来的稳婆,就在产房中忙碌着。她来的时候,特地向韩冈拜谢——韩冈当年使人打造的产钳,已经在京城中传播开,虽然有说法用产钳会致子痴愚,但性命攸关,救命的时候谁还会在乎?而且也不仅仅是产钳,如疗养院中所用的烈酒消毒等事,也在产房中传开

因为是头胎,王旖一直都没有大补,韩冈想着她生产不会太难。而且还有严素心和周南在前面做例证,应该很快就能结束。只是没想到拖了这么长时间,还没有个准信。

“玉昆,你还是歇一歇吧。”

几名幕僚不便进内院,也就王旁陪在妹婿身边。看着韩冈心神不宁的样子,一开始还为妹妹感到高兴,但几个时辰下来,都已经觉得好笑了。

韩冈应着声,点点头,但他根本就没有听到王旁在说什么

忽然王旖已经变得嘶哑的喊声停了,韩冈心头一跳紧张的望着房中。幸而一阵低微的啼哭传了出来,他这才浑身放松了下来。

产房的门打开了半扇,一名头发斑白的老妇从房中走了出来,向着韩冈福了一福:“恭喜提点,乃是弄璋之喜。”

专门在京中官宦人家服侍的稳婆果然不一样,单是说话就不同一般。生了儿子,就文绉绉的说一句弄璋之喜,换作是普通的稳婆,多半就会直接说一句生的是衙内、公子或是小倌人了。

韩冈闻言便是大喜,王旖给他生了个儿子。

而王旁就在旁边大笑着拱手祝贺:“恭喜玉昆,贺喜玉昆。”

当家主母生下了嫡子,家中的仆人婢女立刻同来道贺,韩冈开怀笑着,很大方的遍赏府中一众老小。

等到人众稍散,这时心中冷静下来,突然就感觉着身子发凉,竟然满身是汗,衣裳都湿透了。抬头看看时间,已经是红霞满天,王旖用了四个时辰才将儿子生下来。

产房收拾完毕,心急着要见妻儿的韩冈终于被稳婆放行。

王旖已经换过了衣服,又擦了去汗水,但头发上还是湿漉漉,脸色也极是苍白。用了整整四个时辰,才将儿子生了下来,原本精力就不算太好的她正沉沉睡着,丈夫进来的动静也没有惊醒他。而韩冈的第二个儿子就在包在襁褓中,放在枕边,小脸皱巴巴,紧闭着眼睛。

轻轻的理了理王旖乱掉的头发,韩冈转身又向稳婆和她的助手连声道谢,让下人奉上了厚礼。

韩冈终于有了嫡长子。上门道喜或是送来贺礼的人便络绎不绝,场面比起周南、素心生产时要大得多。从八月初开始,外面就有人打探消息,等到到了预定的产期,更是多少人在竖着耳朵等消息。韩冈为官算是清廉,都没人见过他收受重礼贿赂。许多人想结好韩冈,都无门而入,而眼下的机会是很难得的。

不过,在京城不比在边地,盯着自己的太多,而前面又得罪了御史。即便是人情往来,会招致人言的厚重礼物,韩冈还是尽量的给推掉,只收下了一些价值不高的礼品,其中县中百姓和流民们送来的长命锁、护身符倒是最多,韩冈都是亲自道谢后收了下来。

而到了第三天,收到消息的王雱也到了白马县。

看到大舅子,韩冈很是惊讶,“元泽,你怎么来了?”

“当然是看我那外甥的!”

看到被抱出来的外甥,王雱欣喜不已。妹妹既然生了儿子,韩冈和王家的关系就再也斩不断了。

韩冈摇摇头,刚出生的婴儿不宜多见外人,让王雱看了一阵后,就让人抱了回去。

请了王雱在书房坐下来延礼奉茶,韩冈问道:“朝堂上正乱着,元泽你还真能放心离开?”

“玉昆你呢,你就当真放得下国事?”

韩冈摇头苦笑,“此非我等可挽。”

这件事上,与其将责任归咎于那几位元老重臣,还不如说是皇帝本身的问题。

天子畏敌如虎,做臣子的也没办法。在软红十丈的东京城泡大的皇帝,想要找个硬气的当真是难。当初寇准将真宗皇帝请过黄河,不知费了多少气力。

如今的皇帝一口一个唐太宗,对天可汗三个字羡慕不已。可李世民在洛阳城外,亲着玄甲,带着麾下的千余玄甲重骑为前锋,一举击败王世充、窦建德两路诸侯的主力,决定了天下谁属。李世民的胆识武勇,赵顼连根脚趾头都比不上,不要他亲自上阵,只是要他硬气一点,将契丹人的无理要求直接回绝,又有什么好怕的?

韩冈都懒得在这方面多说了。他岳父王安石说得好,焉有拥万里而畏人者?坐拥亿万子民,国中带甲百万,经历过战火的精兵强将亦为数众多,还怕个什么?这两年在河北整顿兵备,又是为了什么?

要不是因为这一次的大旱,韩冈本有心上书,奏请朝廷对西夏重新开战,夺取横山和天都山,藉此消耗西夏国力,争取在十年之内,分步解决西北边患。可看着赵顼的样子,他的提议恐怕根本得不到回音。

河湟开边是熙宁五年结束的,如果连续作战,兵将肯定难以支持。但若是长久不战,战斗力也会逐渐减退。所以休生养息两三年,便是最好的开战间歇。

只是大旱还有一年才能收尾,为了解决河北流民,开封府的常平仓耗用了大半。要不是夏天的时候从汴河大批运粮进京,东京城七成的粮库都要空了。不管怎么说,攻打西夏今明两年是没指望的。

而且韩冈依稀记得,魏平真曾经说过,大宋建国以来的气候,都是涝上一二十年,跟着就旱上一二十年。从熙宁二年开始,天下旱情增多,到如今也不过五六年,若是明年再旱起来,韩冈也不会惊讶,但他平灭西夏的计划,肯定都要打水漂,只能在心中幻想了

王雱叹了半天的气,突然问道:“……玉昆,是否有心入朝?”

韩冈摇摇头,笑道:“有元泽在内辅佐,何必小弟。”

王雱的职位远不如自己,王安石太过要求自清,所以到现在为止,王雱也只有一个侍讲、加上经义局中的职位,除了在经筵上给天子讲课,然后编纂经义外,根本没有给王雱安排任何重要的差遣。

看到王雱,韩冈不会认为自己入朝后,王安石又能给他什么重要的职位。且即便会给,御史们也会将闹起来的,最后很有可能鸡飞蛋打,还不如再等上一等。

王雱叹了口气,韩冈推三阻四,心意已经很明白了,但他还是想多劝一句,“天子对玉昆你信重非常,若是换了玉昆你来说,多半能说服天子。”

韩冈正得圣眷,尤其是妥善的安置好了流民,让他在天子眼中更加受到看重。在王雱看来,也许王安石做不到的事,韩冈能做到。就像郑侠上流民图时的那一次。

但韩冈知道自家事,他不过是个做了四五年官的小臣,有些事可以说动天子,因为他在这些事上表现出了足够的才干,加上他所处的位置有资格发言。

可遇上事关国运的咨询,天子却是决不会相信一个小臣的。赵顼为何弃王安石的忠言于不顾,而亲颁手诏问政于韩、富、文等人。不就是因为这等元老重臣为官日久,威望素著,能压得住阵脚,可以给他以信心。

“元泽,你当真以为在此事上,小弟说话能比得上韩、富、文等一众元老不成?”

“难道就坐看他们败坏国事不成?!”王雱厉声反问。

他心急如焚,如果天子当真接受了契丹人的要求,罪名就都会加在王安石身上。以王安石的性格,肯定要称病不朝,逼着天子改弦更张。但经过一场大旱和一场蝗灾之后,还要加上曾布的叛离,王安石和新党的政治根基已经彻底动摇。再想如熙宁初年的旧例,已经不现实了。

而韩冈明白王安石是绝对不会顾及这一点的。即便根基不稳,他照样会强硬的逼着皇帝。天子若不能答应他的要求,他脾气起来,多半真的会辞官。

韩冈眉峰一挑,单刀直入,“岳父应该没有让元泽你来说这些吧!”

王雱声音一滞,的确,王安石并没有让他来找韩冈说这一件事。如果是流民图这一桩公案,要主持流民安置的韩冈上殿分说,那是顺理成章;而现在的边境划界,与府界提点根本毫无瓜葛,以王安石的脾气,怎么会找到韩冈头上?

韩冈叹了口气,“元泽,说句实在话。有的时候,退一步海阔天空,岳父今年也才五十三啊!”

