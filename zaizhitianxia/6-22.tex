\section{第五章 冥冥冬云幸开霁(一)}

为什么没收到消息?

理由不必多想。

韩冈现在的身份是一重。

从朝中退出来,韩冈身上只剩一个图书馆馆长这个说来可笑的差遣。不可能还能像之前一般,能够及时得到朝廷内外消息。而在宫廷中,很多时候,半个时辰的差距,就是生死之别。

而宋用臣、石得一,这两人亦是关键。

若有人劝说太后废立天子,那么向太后事后不可能不通知自己。朝中宰辅,能确定支持赵煦的只有王安石和韩冈。

但如果中间有人设置障碍,使消息到不了自己的手中,石得一、宋用臣两人合力,肯定能够做得到。

更有可能他们直接劝说太后,找个理由拖上一天的时间——大祥祭典就是最好的理由。

但不管是什么原因,造成的后果就是韩冈直到章敦说破之前,对此事都是懵然不知。

这一回,若是能……韩冈暗自摇头,其实一样的,他就算事先听到这个消息,也不可能想得到蔡确会敢于选择直接叛乱。

就如章敦,他在入宫前,就知道了蔡确、曾布劝说太后失败了,可他一样没想到蔡确、曾布会直接联合宫中的太皇太后,直接将太后给赶下台。

不过,整件事依然疑云重重,不是那么简单。

章敦的态度才是重点。

“子厚兄,这件事是什么时候听说的?”

韩冈用漫不经意的声音问着,双眼则望着大庆殿前的广场。

郭逵在那边正在约束班直禁卫,命他们回护大庆殿。以防他们追杀性起,反而让皇城司残余的叛逆来个狗急跳墙。二大王被押回殿中去了,王厚跟着一起回去,不过有一名将领被郭逵招了过去,大概是要吩咐他做什么事。基本上是清扫和收尾,等到派人诏谕皇城司剩余的叛军,这一场变乱,便算是再无反复了。

但韩冈的注意力还是集中在章敦身上,等着他的回答。

“昨日夜中。”章敦声音略沉,挥手让已经站得很远的了禁卫躲得更远,“如果不是大祥和宵禁,应该能更早一点。”

前一日晚间的消息,一般第二天就该知道了。但天子的丧礼使得宫禁森严,消息传出不易。而大祥祭礼持续了一整天,更是耽搁了时间,而宵禁也阻碍了消息扩散。

章敦说得的确没什么问题。

但章敦夜中收到消息,没有直接通知自己倒也好说,时间上来不及,但今天入宫前也没有多提一句,就是问题了。

现在说出来,是因为明白瞒不过去,才早一步说破?

“玉昆。”章敦双目平视前方,“蔡确的为人想必你也清楚,你觉得为什么蔡确会做这等大逆之事?”

韩冈敛容不语。

打了十来年的交道,韩冈当然了解蔡确这个人。

蔡确的赌性的确很重,这是世所共知,但他一贯赌得极精,从来都是以小博大,都没见他输过。

十年间,从给韩绛溜须拍马谄媚献诗的芝麻官,摇身一变,做到了与韩绛平起平坐的位置上。论功劳,两府之中,有哪个比他稍差?但他偏偏官运亨通,让谁都比不上。跟蔡确比起来,三十为公辅的韩冈,也都只有撞墙的份。

以蔡确的性格,如果没有不得已的理由,以及足够高的胜率,蔡确根本不可能会选择走上谋反这条路。

若说胜率,这没话说。也许蔡确在劝说太后废立天子之前,就已经在做准备,但时间也不会超过二十四天,甚至不会超过半个月,乃至十天。

在这么短的时间里,参与到叛乱中的人数和身份已经多得让人胆寒,若不是韩冈出其不意的捶杀蔡确,蔡确他是赢定了。至少超过九成的胜率,正常的赌徒都会去赌。

但光有胜率,没有迫在眉睫的危机,蔡确也肯定不会去做这等杀头买卖。

就算他之前劝说太后废幼主、立新君已经失败了,但他还有时间去联络其他宰辅,将声势更为壮大,逼迫太后同意废去赵煦。

除非太后的态度实在有异,让他嗅到了危险,又或是有什么事让他失去了自保的信心。

“玉昆,你当日去蔡确府上到底说了什么?”章敦又进一步问着。

韩冈的脸色更为严肃。章敦在问他跟蔡确的对话,更是在质问自己,还记不记得当初跟他说的那番话。

他转头直视章敦。

子厚兄!你是说这都是韩冈的缘故?

章敦毫不动摇的对视着。

还能有别的原因吗?

“啊!太尉!”

一声尖叫打碎了章敦和韩冈之间几乎凝固的气氛。

韩冈立刻循声望去,只见方才还站得笔直的张守约,突然间就倒了下去,旁边看护他的班直抱着他大叫。

韩冈忙丢下章敦,几步下了台阶,心中却为不必跟章敦对峙下去而松了一口气。

章敦只差明说是韩冈造成了今日的结果,而韩冈都找不出话来给自己辩解。说这一次宫闱政变全是韩冈的错,或许过分了,但要是说韩冈对废立之事的态度是主要因素,那还真没错。

见到韩冈过来,那名班直叫道更大声了,“宣徽!宣徽!太尉他……”

“别慌!”韩冈一声轻喝,让他住了嘴。

走到近前,韩冈直接在张守约面前蹲了下来。测了呼吸和心跳,还好都能感觉得出来,只是昏了过去。

韩冈低头仔细查看张守约的伤口,从正面只能看见短短的一截翎尾。

长箭是射到了张守约胸前的位置上,箭杆连着衣服,韩冈不敢扯开直接看伤口。这样的伤,创口内夹进了衣料,得用剪刀剪开衣服才行。

韩冈又小心的摸索了一下背后相应的位置,能感觉到衣服里面有个尖锐的凸起,这是贯通伤。

十来步的近距离,力道就算不太大,两尺箭杆也足以人射个对穿,这不是难以想象的事。

“将老太尉侧着身子。”韩冈吩咐道,“小心一点。慢一点。慢。再慢一点。好……扶好了,别动。”

班直听着韩冈的吩咐,将张守约的身子侧过来时,已是满身大汗。

韩冈又仔细观察了一下张守约的背部情况。射中他的长箭,并没有穿透背后的衣服。但韩冈是老带兵的,又管过军器监,只通过触摸,就能感觉得出箭矢的类型。

从手感上看,这是常见的破甲箭簇,呈略尖锐的三棱锥。对于普通的板甲有着不错的杀伤力,不过班直禁卫的全身甲,是外层铁板而内衬牛皮,相当于铁甲加皮甲的双重甲胄。一般的破甲箭也无济于事,只能用破甲弩和更为专业的箭矢才能射穿。

韩冈不知道这样的箭矢,石得一是从哪边弄来的,但不得不庆幸是这样的箭射中了张守约。

幸好是这等讲求穿透力的破甲箭簇,造成的伤口不大,换成是普通的扁箭头,洞穿身体的伤口就没那么简单了,穿过体内时,不知要伤过多少内脏器官。

“玉昆。张太尉怎么样了。”章敦早走了过来,见韩冈检查得差不多了,便轻声询问。

“口中无血,没伤到肺。脉搏也安好。可能没伤到脏器。”韩冈又重新上下检视了一番,点点头,略大声的对章敦说着。

只是他站起来后,在章敦耳边的声音则低一点,“不过也可能是内部的伤口给箭杆挡住了,拔去箭杆的时候就立刻大出血的伤兵,从来没少过。且即便没这么重,以张老太尉的年纪,能否吉人天相,真说不准。”

“怎么办?”章敦眉头顿时就皱了起来。

张守约在宫中威望极隆,如果他安好,由他宣布宰辅们对参与叛乱之人的决定,宫中的乱象转眼可定。但他现在受了重伤,光靠宰辅,即便是郭逵去,也没那么顺利。

韩冈心情其实更差。

他跟张守约的关系,比章敦之间更近了一筹。只是因为文武殊途的关系,不能太过接近。

当年就是张守约与王韶一同推荐韩冈为官。而李信更是在张守约麾下多时,最后也顺利得官。要不是双方的地位都已经太高,分据文武两班的顶端,韩冈早与张守约直接定下姻亲了。而且即使有这个因素在,李信的儿子也已经在与张守约的孙女议婚,只待写婚书了。

“得找外科的翰林医官来,而且这里更不可能动手术。”

“宫中现在谁能排得上用场?”章敦又问。

“精于外科的医工多在边军中。现在太医局里面,只有一个曹景圣算得上出色。”

河东有一群外科学水平超越这个时代上百年的军医,可他们还在解剖尸体,都还没调回来。

“得快点将这边的事给收拾了。那曹景圣今日是在宫中轮值,还是宫外医院问诊?”

“要知道就好了。”韩冈叹道。

按照韩冈定下的规矩,太医局的医官、医工和医生们,都会轮班去城中的医院给士民看病,一方面练手,免得医术退化,另一方面,专卖成药、隶属于太医局下的和剂局,也能赚上一点。如果今日曹景圣在宫中当值,那么张守约能保住性命的可能又高了一成。

这时,喧嚣声大起。

一彪人马从西面的文德殿方向赶来,但章敦和韩冈在看了几眼后,就放下了紧张。

“宽衣天武的兵到了。”

诸班直加上宽衣天武的人马,足以压下皇城司的残兵。

