\section{第20章 土中骨石千载迷(三)}

赵顼闭了闭眼睛,旋又睁开,“现在编修局中,到底有多少殷人占卜的龙骨?”

“之前从药铺中送来的样品中发现了可疑之处,派去安阳搜集的一队人,轻易的就从当地的村人处买到了一千余片,包括两件礼器在内,只用了五天的时间。”苏颂忍不住叹了一声,“安阳的百姓,拿地中的龙骨当成是疗伤的名药,煅烧成灰后敷在伤口上。多少年来被毁损的龙骨可以说是不计其数。”

苏颂说的这些话,完全是复述韩冈的言辞。究竟是真是假,苏颂不能确定。但若是能敦促天子保住殷墟,从中找到儒学一脉的源流,那么这些无伤大雅的谎言,他也不介意说上两句……纵然是欺君。

赵顼沉默了起来。苏颂静声等着天子的回复。

韩冈并没有明说,但苏颂确信,暗中影响药铺,使得送来的龙骨是殷人的遗物,必然是韩冈无疑。整套的戏码当是全都在韩冈掌心里攥着。

要么韩冈事先定下策略,自家适逢其会;要么就是韩冈的运气好到天怒人怨。与韩冈结识久了,苏颂只会相信是前者。而且不论是哪一种可能,对韩冈的目的,以及他所想要的结果,都不会有什么影响。

过了好一阵,赵顼终于开口,缓缓的又重新确认了一遍:“有一千多片?”

“埋在地下的只会更多。”苏颂正色回道,“陛下明察,那可是殷都,卜辞只是一部分,祭器礼器当不在少数。”

……………………

“日前韩冈派人去安阳确认,昨日第二批甲骨已经送到了京城,眼下共有一千余片之多,还有两件殷人的礼器。而埋在地下的只会更多。韩冈正准备写札子,禀明天子。将殷墟中的甲骨和礼器都发掘出来,整理造册,以明上古文字,卜辞,并殷商礼仪。也就这两天了。”

听了韩冈的话,王珪闭上眼,腰背无力的靠上交椅。蔡确则是摇头,喃喃的不知在说什么。其他几名官吏,脸上的表情变来变去,想不信,却又不知怎么去驳韩冈的话。

一千多片甲骨,还有礼器,韩冈即使要作假也没这个能力,这世上有这个能力不会做此事,会做此事的,没有这个能力。

自古而今,伪造先代典籍,全都是以献书的形式。一卷书、两卷书,多也不过五卷、十卷,哪里可能会有人能拥有这么大的手笔?一下一两千片,而且地里面还要埋上更多。那些器物,想要做旧了,都不是那么容易的事,更何况还要瞒过当地的百姓。纵然是宰执或是巨室豪门都做不到。

若是地下还有更多的证据,那么完全可以确认是殷商的遗物。《字说》若不能迈过这一关,那么就是有天子主张,也无济于事。韩冈这一棒子砸得实在是太狠了,不是一部书两部书,是拿着殷商一朝来砸人。寻常天上落陨石陨铁,不过是拳头大小,但这一次落下来的,是一座山啊!

‘新学算是完了。’不止一个人这么在想。

不过韩冈辛辛苦苦的将殷人占卜的甲骨从地里掘出来,难道仅仅准备局限于《字说》和《周易》吗?王珪和蔡确可不会这么小瞧人。

韩冈也的确如他们所想:“孔子删述《诗》《书》。《诗经》风雅颂,先圣只留下了三百篇,而商周时流传的诗篇又何啻千万?《书》百篇,虞夏二十篇,商周各四十,但典、谟、训、诰、誓、命,各色体例的夏商周三代文书,在先圣手中被删去的又有多少?”

蔡确不自在的扭着身子,看着韩冈的眼神甚至有了几分惊惧。《六经》之中,除了《春秋》,其他五部经籍,看起来韩冈他一部都不准备放过。

“但殷墟之中,不一定正好有《尚书》中的篇章。”虽是在驳韩冈,但王珪声音干涩,仿佛是被驳斥的一方。

“诚然如此。”韩冈身子微微前倾,看起来是谦逊的回话,却带来了更多的压迫感,“但可以用来对比《尚书》残篇中的文字,从遣词造句上也能得到许多。”

王珪的喉头咕噜一下,干咽了一口唾沫,他已经不知道该说什么好了。

《尚书》的古今之争到了此事依然没有一个确定的说法,汉人从西汉闹到东汉,打得头破血流,到了隋唐,《古文尚书》成了正溯,但今文也没有衰落得太厉害,对于两家的分歧,唐儒和稀泥的为多,到了北宋,也不再是儒林争论的焦点。可韩冈这么一说,等于是要重新今文古文之争的战火。

……………………

“此外还有《三坟》、《五典》、《八索》、《九丘》,三代典籍几千年来散佚殆尽,若是能在殷墟中得到一二残篇,也是儒林的千载盛事。”

伏羲、神农、黄帝之书,谓之以三坟;少昊,颛顼,高辛,唐,虞之书,谓之五典;八卦之书,谓之八索;九州之志,谓之九丘。

这些是传说中三皇五帝时的典籍,如今只能在《尚书》中看到一星半点。左丘明作《左传》,说是看到了这些传世之篇,但《左传》中毕竟没有详细说明。

三代之治究竟是什么样?官制、兵制、田制、刑名,后人只能从先秦的书籍,或是《史记》等史书中得知一二,而且还不能确定真伪。

赵顼脸上看不到表情有何异样,但他按在御案上的左手,却是在微不可察的颤抖着,被苏颂尽收眼底。

“三代之治,千年来争议甚多,甚至有一干人等随心杜撰,甚至让人难辨真伪。苏轼‘杀之三宥之三’,以欧阳修才学,亦不知其伪。而殷人不同,殷商敬天事鬼,占卜也是呈于天,非是欺于人。卜辞上,当不会有假。”

……………………

说了长长的一通话,韩冈口干舌燥的端杯喝茶。心下冷笑,想靠《字说》来抢占训诂释义,那就先将甲骨文解释明白。想靠《易传》来争道统?商人的卜辞就在骨头和龟壳上,还是先将《周易》中的爻辞对照清楚再说。

只要能将殷墟中的甲骨文解读出来,《尚书》今古文之争,说不定可以得到一番证明。而已经散佚无传的《乐经》,说不定也能从商人音律上倒推出来一部分。

这是韩冈画出来的大饼,丢出来的骨头。儒门道统,如此一来,将会争夺得更加激烈。而气学的格物之说,却是能独树一帜,让任何人都无法动摇。只要是拿着实物考古,那就是格物,当韩冈将甲骨文抛出来后,格物致知四个字,已经印在了甲骨之学上。

甲骨文一出,儒学免不了要有一番大地震,纵然孔子说郁郁乎文哉吾从周。但周的制度和文字又是从哪里来的?

是用‘圣人生而知之’这句话来搪塞;还是自三皇五帝,由夏而商,由商而周,这一条脉络传下来?

哪边的说法更能让人信服,这是不用多想的。

……………………

苏颂亲眼看着赵顼脸色骤转,心中不免暗叹,实在是接了个苦差事。但看到天子瞠目结舌的样,私心中又有几分快意。

石渠阁辩利义,白虎观议五经,两次开辟儒门大义的会议,虽是千古之盛世,也没有说天子亲自下场选边站的。皇帝应该执中道秉公心,怎么能拉偏架?

王安石初变法,曾经要赵顼法效三代,不要去学李世民。那么当安阳殷墟成千上万的甲骨出土,殷人祭祀用的鼎器一只只被掘出来,可以丢到一边去不加理会吗?

就是天子也逆转不了人心和大势。

这也是以实证之的用处所在。能拿出实物来说话,永远比单纯的书本更有说服力。从甲骨文到大小篆,到汉隶,再到如今的楷书,这一条演化的脉络下来,只要看了字形,就能确认。不是拿楷书来解字的《字说》可比。

……………………

韩冈抿茶润喉,让王珪、蔡确等人消化这个惊人的新闻。

金石之学,乃是如今儒林中的显学,儒者多有研究。就算赵顼想要不加理会,那么多的数量,也别想瞒住世人。

今人崇古,如今可算是太平盛世,在金石上下功夫的士人不胜枚举。几千几万片的商人占卜的甲骨,可比区区十座石墩上的几百来字石鼓文要强得太多,转眼就能兴起一门研究殷商文化的热潮来。

殷商距离尧舜禹上古三代圣王,可远比两千年后的宋要近得多,当然更近于三代之法。世人尚古崇古,那就比比哪个更老资格了。

韩冈对甲骨文也不甚明了,但其他儒者同样不明白,仅仅是争论甲骨文的字义,就够多家学派争上几十年了,至于《字说》,还有谁会在意?

这便是韩冈底牌所在,新学既然用君权来压人,那就一拍两散,掀了桌子,大家一起从头开始玩!

闹个几十年,到时候,能安然存活了下来,必然是气学。韩冈有这个自信。

他所缺少的,仅仅就是时间!
