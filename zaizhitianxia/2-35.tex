\section{第10章 弹铗鸣鞘破中宸(上)}

【第二更,求红票,收藏】

张香儿现在哭都哭不出来了,软瘫瘫的被人扶着站在城头上,看着一队董裕的骑兵在城外举着面小旗来回奔驰,耀武扬威。

纳芝临占部的族长完全没想到,董裕那厮来得竟然这般迅快。昨天他的前锋还说是在百里外,而现在就已经杀到了城下。

既然眼前已经有了贼人在游荡,那他的吹莽城往古渭寨过来的通道,现在当然已经被眼前的这支董裕军的前锋所封锁。而且张香儿还听昨日逃回来的斥候说,率部作为前锋的将领,竟然是董裕手下最为狠毒的赞及。

那个可是真的能活扒人皮抄经书的疯子!

想到自己尚留在吹莽城的妻妾,还有才七八岁的独孙,想起他们即将要面临的遭遇,张香儿的身子都抖了起来。

古渭寨可以不怕董裕,他们只凭着千余人就能稳稳守住城池。但小小的吹莽城可挡不住董裕的上万大军。

吹莽城虽然号称是城,但就是个小寨子,比起宋人的军堡都远远不如,更别提与那些周长九百步、一千步、千二百步的军城相提并论。纳芝临占部的核心吹莽城,其实就是个破烂地方,要不然他张香儿也不会隔三差五的就跑到古渭寨来住。

张香儿此时心中只剩下后悔两个字,早知道会有今日他当初就不趟那场浑水了。秦州城里的两个大官人争功,他掺和进去算什么。

听着王韶的话,七个部族一起去把托硕部给灭了,还把董裕打了一顿。最后是一家的产业几家分,自家也没占个多大的便宜。而现在,纳芝临占部却要为着那点微不足道的进账,把整个家底都赔进去。

张香儿哭丧着脸,恨得直跺脚,他家都快被灭门了啊!这青唐部的援军究竟什么时候才会到!

此时高遵裕的心中也是火烧火燎,嘴唇被心火烧开了几个口子,唇角也尽是水泡。贼人都杀到眼前了,韩冈怎么还不传个信回来?青唐部的主帐离古渭寨又不远,三十里地,两重山而已,现在再怎么也该有个消息回来了。

高遵裕回头看看身后不远处已经面无人色的张香儿,更是心急如焚。若是今次董裕成功复仇,使得七部尽灭,王韶在河湟一带好不容易打下的基础必然烟消云散。而朝堂那边,也许刚刚为托硕部的功劳定下了封赏,天子也许才看过俘虏在他面前三跪九叩,乞求宽恕的表演。可现在转眼间,就是一场惨败,这让天子的颜面往哪里放?!让朝廷的脸面往哪里放?!

高遵裕并不是为王韶担心,他是为着河湟开边的事业着想。天子还年轻,心思容易浮动,一场出乎意料的大败,而且还是赶在天子兴头上给他的当头一棒,足以让皇帝不想再听起任何关于河湟的消息。那他高遵裕来秦州自讨苦吃,不就成了个笑话?

高遵裕紧紧咬着牙,咬牙切齿的狠劲,让腮帮子上的肉都鼓了起来。

“公绰,要不要下去歇一歇?”王韶看着高遵裕,

“子纯,你好像一点也不心急!?”高遵裕声音沉沉的,显然已是气急败坏,正欲找人迁怒。

“我也心急,但这事急也急不来。”王韶依然平静,这两年他受到的打击够多了,也不差这一桩。

“是啊,你倒是对韩玉昆放心得很……”高遵裕心头的怒火已经往韩冈身上烧过去。韩冈久去不回,到现在连个音信都没有,让高遵裕恨透了他的拖延。

王韶这时突然间精神一震,眼望着北方,“来了!”他的声音难以掩饰内心的兴奋,“回来了!”

“回来了?!”高遵裕忙顺着王韶的视线望过去,只见古渭寨北面两里多的地方,正有两名骑兵直奔城寨而来。他们冲得极快,随行的还有六匹空马,两人八马在寨北刻意留出的荒地上极速突进,甚至在身后拉出来一道黄龙般的烟尘。

城中的守军突然爆发一阵欢呼声,他们也看出了来的是自己这边的人。听到欢呼声,张香儿猛然抬头,突然间有了气力,忙抢前一步,踮着脚向北望去。

韩冈派回来的两名骑兵,仗着马多,远远的绕过了守在古渭寨西门处的蕃骑,董裕派来的这十几二十人也堵不上古渭寨的几个大门。

片刻之后两名骑兵已经单膝跪在王韶和高遵裕的面前,一五一十的将韩冈让他们带回来的话,向王、高二人做了通报。

张香儿一听之下,当即跳了起来,须发怒张,目眦欲裂,狂叫着:“俞龙珂那狗贼该千刀万剐!该千刀万剐!”

“什么?!”高遵裕也是脸色大变:“韩冈只说得俞龙珂去捣董裕后路?他不管古渭寨了?!他不管七家蕃部了?!”

王韶听了本也是心头噌噌火起,但很快他就冷静下来,他为韩冈解释道:“不能怪韩玉昆。他能说动俞龙珂就不错了,哪里还能驱动俞龙珂那条老狐狸为我们赴汤蹈火。”

王韶的心情比方才还是轻松了不少。对他来说,他不能放任七部被灭,这对他在蕃人中的威信是个极沉重的打击。不过就算七部被灭,若是事后他能把董裕所部全歼,情况也不会很差。至少在天子和枢密院面前,自己也有个为自己辩解的理由。就是在古渭一带,能听话受教的部族又少了几个——俞龙珂这厮,怎么也不可能比七部更听话。

王韶忍不住去想,若是昨夜去见俞龙珂的是自己,那情况也许会好上许多,说不定还能把青唐部的那条老狐狸给劝来直接与董裕硬拼。

只可惜一念之差啊……

………………

站在路边一处高丘上,董裕已经可以看见古渭寨的影子。

在古渭寨南面,还有纳芝临占这最后一家部族。听着赞及传回来的消息,他已经把古渭寨门给堵上了,而寨里的守军则没一个敢出来。

快了!

董裕心中想着,今天入夜就可以灭掉纳芝临占部,到了明天,就可以回师了。今次曾经冒犯过他的七家部族一起被灭,在古渭以西,河州以东的这一片土地上,应该不会再有敢于反对自己的部族了。

几个月前丢掉的脸面和人望,也终于回到了自己的手中。

董裕得意的拈着胡子。他的兄长木征实在太没有进取心,只会守着自家在河州的一亩三分地。声望是打出来的,而不是守出来的。别看木征是长孙,但今次一战后,他董裕可就能在人望上压倒自家的哥哥了。

董裕之所以能联络上瞎药,也是因为他们都有个安于守成、不思进取的兄长,让他们的野心得以膨胀。说两人同病相怜也罢,有志一同也罢,反正都是一模一样的心思——彼可取而代之。

董裕不想被称为掣逋。悉编掣逋这个官职,在吐蕃控制着整个西域、河西的时代,是统领大军的都护,声威赫赫,即便是唐皇也要畏惧。但在现在,却只不过是董毡用来安抚人的工具而已。

董裕的眼神深了下去,掣逋哪及赞普好听。吐蕃赞普这个位子,自家的叔叔现在坐着,自家的哥哥则是想坐而不敢坐,但他董裕很快就能坐上去了。

他狞笑了起来,今次一举灭了七部,方圆几百里的土地上,对于自己的命令,还有谁敢不服?!

………………

一条长龙逶迤于黄土高原被水流切割出来的千丘万壑之间。看似不停的前进,但真正算起走得距离,却是连十里都不到——山路实在太难走,而俞龙珂好像又在刻意拖着速度。

“三哥,这走得也太慢了。”王舜臣心中甚急,驱马靠上前去,跟韩冈说着。

韩冈摇头:“不慢。”

“俺是担心机宜那里,还有高提举,他们等着青唐部的援军。现在不管古渭,而是绕去抄董裕的后路,他们会高兴?”王舜臣是替韩冈担心,怕他因此事得罪了王韶和高遵裕。

“俞族长不是留了三千兵吗?”

“那可吓不住董裕。”王舜臣觉得韩冈是在敷衍他,“俞龙珂怎么下的命令俺也是听到的——不得走进古渭寨二十里内,就是出了谷就停下。可古渭寨离着纳芝临占最近的一处寨子也有二十里。这一南一北隔了就有四十里,董裕能安安心心的把纳芝临占部都抢个底朝天。”

“不用担心,这世上总是聪明人比较多。”见王舜臣不肯放过自己,韩冈想了想,还是透露了一点自己的想法,“你以为瞎药在哪里?”

“瞎药?……他在哪里?!”

“我也不知道。”韩冈笑着摇头,“虽然猜了几个可能,但都做不得准。瞎药有瞎药的想法,就像俞龙珂有俞龙珂的想法。我提的建议俞龙珂会采纳,是因为跟他的想法一致。若我让他火中取栗,你以为他会理我?”

韩冈的话像是在打哑谜,王舜臣听不明白,但韩冈心里却很明白。

董裕为自己的利益行动,俞龙珂为自己的利益行动,瞎药当然也是会为自己的利益行动,只要他的野心如传说中的那样大,那他必然会有一番动作,去为自己博取利益。

同样道理,王韶有他的利益,而他韩冈也不可能例外。至于最后谁能如愿以偿,那就看个人的本事和运气了。

“还是走慢一点比较好。赶上董裕并不需要太急。”韩冈笑道。

