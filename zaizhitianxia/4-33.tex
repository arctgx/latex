\section{第七章 都中久居何日去(一)}

张载携弟子进京,在京城中引起了一阵轰动。

自从王珪、吕大防和韩冈一起上表推荐张载,不仅仅是士林之中对这位名震天下的儒门宗师翘首以待。连京城中的百姓也都心生好奇——能教出军器监韩舍人这样的弟子,又让王相公几次拒其于京城之外的大儒,当然不会是简简单单的人物。

由于如今各家名儒的宣讲,埋首于汉唐注疏,孜孜于章句之中的行为,已经不受如今士子们的喜好。人人都想从各家学派里,找到符合自己心意的解释。

在这其中,一直宣称要‘大其心’的气学,尤其是有着‘为天地立心、为生民立命,为往圣继绝学、为万世开太平’这四句,最为符合这些立志要改变一切的年轻士子们的脾胃——张载在士林中的名望,并不全然是靠着韩冈的发明而来。

张载奉命抵京,先照规矩去了城南驿落脚。但韩冈已经帮他在京城中租了一间合适的宅院。而就在宅院附近,还有一间清静的寺庙,虽然关学严斥佛老,但并不妨碍张载的弟子们在寺庙里寓居——韩冈还为此散了一笔香火钱,让里面的和尚对关学弟子的到来关心备至。

张载守着朝廷规矩,在城南驿暂时落脚。京城的儒生们则连日造访,比起宰执重臣入住,还要热闹得多,让城南驿的驿卒不胜其劳,盼着张载早日搬出去。

不过赵顼并没有在第一时间召见张载,虽然他也想早一点见见这位名满天下的大儒,但宰相的面子要顾及,而且张载的官位又不髙,所以他觉得稍等数日再说。

在驿馆中歇了几日脚,尽管期间也见了不少宾客,张载的精神还是好了不少。也能走出来拜访亲友。一日闲空,甚至还来军器监看了一看韩冈的发明。

如板甲局、弓弩院,就算张载也不便入内。但打造风车、水车的工坊,却并不介意有人参观,世间都有的器物,就不需要太过严谨的保密。而且张载是韩冈的老师,在世人眼中,根本就没有隐瞒的必要。

就在兴国坊一角的一座院落中,几架风车正呼啦啦的转着。

这是军器监打造风车、水车的工坊。风车有大有小,但形制都差不多。只有一架例外,四片长长的扇叶十字形的舒展开来,中轴平行于地面,与其他风车截然不同,但照样在迎风转动。

“这一架风车倒是特别得很。”张载很有兴致的抬头看着。

“这一架是学生让工匠打造的新式样,要试一试与寻常所见的风车哪个更好一点。”

后世说起风车,就是四片扇叶十字形的伸展开来,但此时韩冈所见的风车,中轴是竖着的,七八片扇叶挂在轴上拉出来的长杆一圈,就像是拉起船帆的桅杆,只是挂在桅杆上的帆多了一点而已。

哪种风车的效率更高,韩冈心里也有数。至少千年之后世间通用三片叶的风力机器,是以哪一种风车为蓝本,他还是记得的。不过放到如今,材料不同,结构有别,就不能遽下断言。如今要做的,就是要让人将两种风车都打造出来,进行一番对比再说。

“先生应该听说了学生与水磨坊的一点龃龉——前些日子信上也写了。”韩冈向张载解释着理由,“既然学生在手上抢走了他们的位置,照情理也该还回去一个能抵数的。虽然比不上水力驱动的方便,但冬天汴口不开,水磨坊其实也是无用,而风磨到了冬天可是能派上大用场的。有风时用风力,无风时用畜力,四季都能使用。”

“事情补救了就好。”张载点头微笑,又叮嘱道:“玉昆你一干发明虽好,但也是夺人口中之食,行事不可不慎。”

“学生明白。”韩冈低头受教,知道轨道使许多力工失业的事,还是让张载知道了。

……………………

进抵京城的不仅仅是张载,过了两天,种建中就给韩冈带来了一个消息,说王舜臣就要调任鄜延路,任延州东路都巡检。

王舜臣要调任鄜延路,这是种谔的提议。新任鄜延路兵马副总管的种谔究竟在打什么主意,那是司马昭之心,路人皆知。等再过几日,说不定他就又要上表求取横山了。

种谔既然存了这个心思,自然就要在身边聚居精兵强将,而且是听他号令的精兵强将——他离开鄜延路这几年,人事变动频频。为了能求取出兵,种谔需要一个与他同样求战的鄜延路军官团。

王舜臣虽然是在熙河路出头,威名赫赫,如今再熬两年,甚至就能往都监一级去了。不过他毕竟出自种家,种谔也能信用于他。有了这位在军中得享盛名的年轻将领,北取横山的把握又多了一分。

但韩冈并不喜欢这个调令,在他看来,以熙河路的现状,攻打兰州的时机已经成熟。如果要对西夏动手,还不如先从熙河路发力。这时候调走熙河路的核心将领,其实得不偿失。

熙河路当地驻军的俸禄和粮食都已经能做到大部自给自足,就是兵器、甲胄,如果在岷州的滔山监设立军器坊,照样能够自产自销——尽管这只是指得和平时期,到了开战后,军费粮秣肯定还要外来补充,但消耗绝不会。

另外,兰州城中,禹臧花麻已经动了背离西夏的心思,与熙河路暗通款曲。只要他叛投过来,甚至不用大动干戈,兰州城就能拿下来。

兰州一下,不仅可以将吐蕃诸部与西夏分隔起来,兵锋也能直指被党项人占据的河西走廊,只要再溯咯罗川【庄浪河】北上,攻下洪池岭【乌鞘岭】,便能恢复旧时的丝绸之路。而更重要的,兰州越山向北,就是西夏兴灵腹地了。这比起在鄜延路攻打横山,对西夏的直接威胁要更大。

这几点,其实不用韩冈说,只要熟悉西事的人,都能看得出来。自然,王安石父子也能看得出来。

“但由谁领兵下兰州?”

“鄜延路的种谔岂会愿意为熙河路打下手?”

韩冈连夜来拜访王安石,并不是来阻止此事。这一件事多半是天子的主张,枢密院中吴充、王韶都没能阻止,王安石也不好为一个都巡检反对天子的意见。他只是打算来推动攻取兰州——兰州一下,熙河路在北地有了山河之险,也便安全了。但韩冈没想到,王安石父子已经在为攻取横山做准备了。

“熙河、鄜延两边同时动手都可以,大宋有足够的实力支撑起两地同时开战。党项人就不一定了。”

“只是横山,不会惊动辽人。如果两线动手,辽人岂会坐视?”王安石摇头道,“玉昆,别忘了现在秉常已经是辽国的驸马了。”

秉常是辽国驸马,娶了挂着公主名头的契丹宗女。虽然这个亲戚关系在大部分时间都不过是个幌子而已,但大宋如果是以灭国为目的的大举进攻,辽国就有充分的理由来出兵干涉,这一点不能不顾及。

王雱也笑道:“先攻下横山,禹臧花麻岂有胆量再抗天军?必然举城来投。这可是一举两得。”

“玉昆,知道你出自熙河路,但事关全局,横山必须先拿下来。”

韩冈当然知道横山的重要性,一旦据有此地,关中腹地便能就此高枕无忧。而兰州只关乎熙河路,与关中隔得太远。所以王安石、种谔都看重横山不是没有道理,可怎么就不想想天子在辽人的威胁下会做出什么样的应对?吃过一次亏难道还记不得教训吗?

“辽人不会为一个兰州城而出言威胁。可一旦夺下了横山,甚至是举兵攻打横山,辽人就有可能立刻干涉。”两地的战略意义还是有很大差距的,在名气上也有很大的区别,这点不用韩冈多说,“一旦辽人干涉,是继续打下去,还是撤军,谁也说不准。如果官军畏于辽人而退缩,禹臧花麻还会心向中国吗?”

赵顼此时必然信誓旦旦不惧契丹,但事到临头会怎么样,韩冈可半点也不看好。

王安石叹了一口气,说道:“朝廷准备让熊本去熙河路,任熙州知州。”

这几年大宋朝廷不仅仅在熙河、荆南兴兵,收服羌、蛮。在西南,也有熊本领兵为大宋开疆拓土。不过他功业不及王韶,名气不及章惇,被压制得黯淡无光,但他的能力无可挑剔,也是第一流的人才。

“熊本在西南的确有所成就,但对于陇西事务,他可是毫无经验。”韩冈反对这项任命。以熊本这个人选去熙河,的确有抢准时机攻占兰州的用意,但熊本对当地的情势不熟,恐怕会贻误战机,还不如调沈括去。

“……玉昆,是不是你打算回熙河路?”因为王安石担任了宰相,韩冈最近就上表自请出外,王雱自然会有这方面的联想。

韩冈心头微怒,他是这样的人吗?……怪不得一开始不提熊本去熙州的事,原来是怕自己听说后要抢着去。

他眯起眼睛,笑了起来:“小弟在熙河路有事挂心,但军器监也有放心不下的事,水力锻造作坊还没有完工,板甲局也还待磨合。还想在矿山中推广轨道,另外还有飞船的改进。事情太多,每一个有些难以放下。”

“玉昆。”王安石诚恳的说着,“既然你也有如此想法,那就在京中多留一阵子。京城需要你的地方很多,不要急着出外。”

