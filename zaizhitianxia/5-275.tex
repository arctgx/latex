\section{第30章 随阳雁飞各西东(六)}

韩冈再看看王安石,昨天夜里,章敦派人传话,说王安石已经点头了。 

“不知其他几位相公如何说?”他问道。 

“王平章也同意了。”向皇后回道,“此乃军国重事,所以便请教了王平章。而韩相公和蔡相公则觉得当以慎重为是,枢密院不能没有正堂官。” 

想不到韩绛倒是愿意吕惠卿早点回京。看来他也是不想看到吕惠卿在西北立功,然后直接晋升宰相班列。不意旧日的恩恩怨怨,竟还是留到了现在。 

“辽人南犯,而枢密使却在关西,诚可谓是天意。三路力分则弱,力合则强。以枢密使宣抚陕西,统和缘边诸路之力,抵挡住辽人的侵袭当不在话下。” 

向皇后还是比较相信有军事经验的章敦和韩冈,而王安石的威信也比韩绛加上蔡确都高。可是这么一来,就是要大战的姿态了:“但如今国中不稳,贸然与辽国开战……” 

“何谈不稳?!”韩冈立刻说道,他给向皇后打气,“前有陛下打下的根基,又有殿下秉政,如何不稳?若当真有人不思报国,却为辽人助长声威,真当煌煌天律乃是虚设?!” 

韩冈毫不退让。即便是要召回吕惠卿的韩绛、蔡确,他们也同样是不惜一战,绝不可能妥协。 

也许向皇后回去后还要问一问天子,也许天子会认为退让一步也不是不可以。但也要看宰辅们答不答应。就算赵顼还会有什么想法,也抵不过宰辅们拒不奉诏。 

一旦对辽人妥协退让,丢脸的是两府宰执,毁掉的是他们的名声。熙宁时河东弃土,当时的朝中宰辅,无论新党、旧党,哪一个点头答应割地?而在外的元老重臣,却毫不在乎的吓唬着六神无主的皇帝——身处的位置不一样,要负起的责任就不一样! 

如今旧党崩溃,近年内暂无力再卷土重来,皇后能倚重的只是新党。在两党相争的情况下,宰辅们更不会答应任何有损声名的决定,即便天子能使动皇后,但所有的诏旨,到了政事堂中就全都会给挡回去,没人会副署! 

韩绛不会!蔡确也不会!张璪同样不会!至于西府,就更不用说了。 

难道躺在病榻上的赵顼还能下密诏给前线的将官不成?就算他做出来了,看看前线有几人敢拿脖子试刀! 

但向皇后却没有宰辅们的决心,面上仍是有着犹豫之色。 

韩冈便道:“辽人欲壑难填,尤以耶律乙辛为甚。萧禧便在都亭驿中,旧事仍历历在目。如今溥乐城之围,自是耶律乙辛之命。悍然发兵南下,又挑动青铜峡中党项余孽,当其时尚不知陛下病情。倘若得知陛下玉体违和,不知又会有何索求?那时他索要银夏、甘凉,可要给他?再伸手要太原、延安,难道也给他吗?且耶律乙辛新近弑君,所立新君,名为宣宗遗腹子,其实来路成疑。辽国国中并不安稳,耶律乙辛想要亲自领军南下,也不会那么容易。即便他能举兵南下,以大宋的国力、军力,要退敌逐寇,也远比真宗年间更为容易。” 

韩冈长篇大论,向皇后听着头有些晕。但之前殿上所有臣子都不主张退让,就是两名宰相也只是要稳妥一点。韩冈现在也是一般的想法。重臣们众口一辞,倒是没什么好犹豫了。 

“那都亭驿处该如何处置?”向皇后担心的问着。西北边事起,按韩冈的说法,两边又不想往大里打,到最后多半是各退一步收场。而以过去的惯例,便是大宋这边出点银绢来息事宁人。萧禧肯定是要狮子大开口了。 

“辽人违约背盟,臣这就去问他一问!”韩冈大义凛然,“其曲在彼,看看萧禧怎么辩解!” 

…………………… 

“果然。” 

萧禧收起了刚刚送到的信函,唇角翘了起来。终于是确认了发生了什么事才会让韩冈匆匆而去。也确定了一切正在按照事先安排的步骤顺利进行。 

冷笑了两声,他对翘首以待的折干说明道:“尚父决定在兴灵动手了,来信时已经下了令。” 

折干闻言,便是大喜。这样一来等到了明天,就能看到宋人沮丧恐慌的脸色了。 

“用不着再跟小韩学士打哈哈了。是药师王佛座下弟子又如何?”萧禧的笑容阴狠,“这可正是到得早,不如到得巧!皇帝中风,皇后秉政,太后被拘,雍王发狂,太子更是只有五岁。有本事让皇帝复原,没这个本事就老老实实认赌服输。” 

折干也不禁点头,“实在是巧!运气实在是太好了!” 

他们这一支正旦使团,原本是准备跟宋人讨价还价、求个安稳就回去的。幼主病夭,耶律乙辛另立新君。如今的第一要务,是维持国中的稳定。只是大辽行事,从来都是以进为退,以攻代守。为了维持国内稳定,而对外敌妥协退让,这样的想法从来不存在真正的契丹人的想法中。一旦耶律乙辛这么做了,结果只会更坏,无论内外都会镇压不住。 

为了震慑住宋国天子的野心,耶律乙辛才会不顾兴灵尚未安定,就开始准备挑起边乱。本意上还是为了防止宋人有所异动,让国中不安。但若是能顺便拿着青铜峡的党项人跟宋人换些银绢、特产回去,那就更好了。所以才会启用他萧禧。有曾经逼宋主割让河东边地的萧禧为使,这本就是为了向宋人宣告,大辽国中稳定,有底气向宋国索要更多的好处。 

可无论是耶律乙辛,还是萧禧,事前都不会想到,宋国的那位雄心勃勃的年轻天子,竟然会在祭天时中风不起。如今更是苟延残喘,没有多少时日了。原本甚至可能是委曲求全的交涉,反倒变成了能大大的割下一块肉来。 

萧禧敲着桌子,“青铜峡的党项人意欲归附我大辽,河东胜州的黑山党项想重回北方,兴灵的党项人也想重回韦州旧土,这些地方都可以好好谈一谈了。” 

天子垂危,皇后仓促接掌国政,正是人心惶惶的时候。以萧禧对宋人的了解,他们哪里会有强硬到底的决心。攘外必先安内,最后肯定是要给钱免灾。问题到底是给多少。这就要靠自己的口才了。 

两人都在等待明天吗,盘算着面对韩冈时该如何说话。今天给韩冈掐着脖子,实在让他们不痛快。但韩冈却没过多久就回来了。 

进门时的韩冈脸上完全没有笑容,眼神如外面的寒风一般凌冽。这让萧禧心中为之一凛,向折干丢了个眼色,‘看来事情有变。’ 

韩冈一进厅中,连坐都不坐,“敢问林牙,贵国对澶渊之盟如何看?” 

“我大辽素重然诺,盟约既定,自然会信守承诺。贵国河北七十余年不闻战火,岂非盟约之力?”萧禧自不会上当,脸板了起来,“倒是贵国,在边境上修城筑堡不遗余力,却不知可还记得那一纸盟约!” 

“林牙说的修城筑堡,可是在分水岭的土垄上?!” 

当年萧禧叫嚣着要以河东北疆当以分水岭上土垄为界,但这其实只是他信口开河,并没有经过实地勘察。大宋这边派了官员去当地一看,分水岭上根本就没有什么土垄。 

韩冈讽刺了一句,更不等萧禧回应,厉声道:“鄙国恪守盟约,岁币七十余载未曾拖延一次。但贵国呢?!若贵国不顾旧日盟好,当澶渊之盟不存在,那就一切休提,韩冈这就请命将林牙礼送出境,日后两国是战是和,就跟林牙无关。若林牙还念着澶渊之盟,韩冈就要问一下了。贵国兴兵围我边城到底是何缘由?!” 

韩冈猛然间揭出了他们的底牌,并不在意被利用。萧禧面不改色,显示了极其出色的心理素质,但呼吸却也不免稍稍变得急促。而折干就差了很多,甚至瞪大眼睛,怀疑韩冈是不是气疯了,哪有自己把自己往陷坑里送的? 

“内翰之言,在下全然不知。”萧禧一脸无辜,“但鄙国尚父曾耳提面命,让边境守将勿要挑起边衅。若是真如内翰之言,其缘由或许是在贵国守将身上。” 

“林牙好口舌,万余兵马围我边城,倒是能一推了之。韩冈便告知林牙,今日朝议上,已经决定设立陕西宣抚使司,由枢密使吕惠卿为宣抚。以两府之意,当宁为玉碎不为瓦全。惟中宫念两国宿日盟好,亦不忍两国生灵涂炭,故而遣韩冈来问一句,贵国是否打算毁约?人而无信,不知其可,国而无信,可久存乎?” 

朝堂上的人事变动,相信萧禧也知道了。人不同,应对危局的手段也会不同,萧禧不会不明白。尤其是在他强调之后。 

“倘若贵国打算弃约背盟,鄙国也将会一战到底。不论对手是谁,鄙国绝不会畏惧!贵国要和平,自然会有和平。但贵国若是选择了战争,那大宋便会回报以战争!”韩冈的声音一句比一句更高,到了最后,便是吼了出来:“林牙可别忘了去岁的河东!” 
