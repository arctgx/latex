\section{第11章 城下马鸣谁与守(18)}

秸秆在火盆中噼啪作响,呛人的烟雾从火焰上腾起,在屋中弥散开来。

余古赧眨着被熏红的眼睛,透过烟雾,看着围坐成一圈的首领们:“说说吧,到底该怎么办?”

房中很安静,没有一人接口。人人都是低着头,专注的看着火盆中火苗的窜动。在宋人开出的条件传来之后,这样的安静已经维持了很久。

但屋外并不安静,密如雨丝的弩矢,每时每刻都从村外射进村中,由此而受伤发狂的战马不断增加,一声声的嘶鸣,让视爱马为生命的阻卜人不忍卒听。可并不算大的村落里,房屋只能勉强安置下所有的战士,他们的坐骑就只能留在外面,承受箭雨的洗礼。

“干脆杀出去好了!”终于有一人耐不下性子,用力在地上一锤,怒吼着:“再拖下去,连马都没得骑了!”

“怎么杀?”余古赧闭着眼,颓然的说着,“村外可还有一条好路?冲出去全都得陷在沟里。到时候神臂弓一阵乱射,没一个能活下来。”

就在将大王庄围困的时候,宋人除了射箭之外,还为了防止村中的阻卜人逃脱,在道路上下足了功夫。村外的几条道路上,全都给挖出了一道道类似于陷马坑的宽沟。

余古赧方才趁着最后一缕阳光,远远地向那几条宽沟望过去。发现宋人掘开的道路上,都是平行排列三条一丈宽、间隔也有一丈的沟壑。想凭借战马的跳跃力跳过去,根本不可能实现。就算那些沟只有两三尺深,也足以让冲到沟边的战马成为神臂弓的靶子。

在道路之外,除了几处实在陡峭崎岖的地形,都能看到宋军点起的火堆。火堆边,还有趁着火光,继续挥锹开挖陷马坑的宋军士卒。

“除非插上翅膀,否则根本就逃不出去了。”

余古赧不想接受宋人的要求,那等于是让他们像一头羊一样,自己走到屠刀前,是死是活,得看宋人的心情。若是按照他的想法,让宋人将他们收编,那还是一支能上阵的军队,若是不合意,还能设法跑掉。

没有人愿意将自己性命完完全全的交托给别人,阻卜人的首领们还在犹豫着,但村外的大宋官军却没有那么好的耐心。

外面坐骑的惨嘶,突然响亮了起来。院中噼噼啪啪的声音,仿佛冰雹倾泻一般,似乎有什么重物落到了地面上。还没有等余古赧等人反应过来,就听见上方喀喇一声响,一道黑影在众人眼中闪过,面前的火盆突然间就翻了个底,火星溅得老高,盆中的柴草更是飞了起来。溅起的星火,燎着了两个首领的胡须。他们立刻就在大厅中打起滚来,而其他人也都脱了外袍,帮他们扑灭身上的火焰。

火盆被捡了起来,底已经被砸出了一个窟窿。从翻倒的灰烬中,韩冈发现了一块鹅卵石,就是这个东西将位于村子子正中央的火盆打穿了底。

一天的时间,足以让大宋的工匠造出简易的配重式投石车。大大小小的鹅卵石飞舞在空中,洒向小小的村落。这是连一间瓦屋都没有的村子,铺在屋顶的是一束束茅草,拳头大小的石块,轻而易举的就穿了茅草铺就的屋顶。

飞石不仅仅穿透屋顶茅草,对于战马则更为有效。村中战马的哀鸣,一声比一声更为凄厉。而火箭也开始用上了,燃烧着的长箭划破夜空,在村中点燃了一栋栋房屋。

外面已经是红光满地,余古赧再没有时间耽搁了:“先让宋人得意一阵吧。”

……………………

官军前方大捷。

数日间一直都处在惊恐之中的晋宁城百姓,在一名接着一名传递捷报的信使们从西门奔向城衙的过程中,终于安心下来。

持续了七八天的宵禁,也随着知军的一纸令文,而宣告终止。压抑了多日,城中的大小酒肆一时间爆满,达官富户、贩夫走卒共贺官军告捷,几家大酒楼和妓寨,甚至为此喧闹了一夜。

城衙之中,也是喜气洋洋。在灵州之败和盐州眼下的险峻局势衬托下,河东这边全歼阻卜贼寇的战绩便显得分外惹人注目。寻常的

一夜安寝之后,在接近中午的时候,韩冈和李宪等到了阻卜人投降的消息。

被困在一座连围墙都破败不堪的小村中,阻卜人在死亡和听话之间,终于还是选择放下手中的弓刀。

“还是识时务的嘛!”

韩冈如此说着,但李宪分明在他的脸上看到一丝难以掩饰的遗憾。

“龙图觉得这是好事还是坏事?”李宪心生好奇。

一般的官僚,都是宁可少一事,不愿多一事。前面阻卜人已经决定降伏,但韩冈却偏偏强加了一个放下武器的要求,让整件事平生波折。

“好事吧。”韩冈说道,“这样也就能将这群阻卜人明正典刑了……之前已经俘获了不少贼子,却还少一个够分量的来杀鸡儆猴。”

“明正典刑?!”李宪差点要跳起来,“将余古赧明正典刑?”

“没错。杀人、放火、劫掠,能做的恶事都做了一边。依律可是要受到重惩。”打从一开始,韩冈就没有想过放过这群强盗,“他们老老实实投降,正好能让刑场上多几人一齐上路。”

“龙图,他们可是已经降伏了!”

“所以我清算的是他们之前的过恶。强盗就擒,难道不是依律处断?!”韩冈眼神冷着,“还是说,我曾说过招安二字?”

李宪一时沉默了。

韩冈更进一步的说道,“他们是强盗,是在官军重围下不得脱身,方才放下武器,可以当自首论吗?”

韩冈从来就没有把贼寇们当成是可以利用和挽救的对象。战争时的杀伤,甚至劫掠,最后都可以睁一只眼闭一只眼的过去。韩冈不想看到这一幕,只能动用刑律。但这么一来,牵扯上刑律,麻烦事也就多了起来。

“劫盗民家,依律当斩,累犯更是决不待时。”李宪很是头疼,揉了揉发胀发痛的太阳穴,“自首减等也轮不到她们。不过这边有几千人,肯定要留下一批。到时候可以将他们的约束放松一点。”

“此辈岂可轻信!?”韩冈对此深有了解:“一旦给他们松了绑,最后会发生生么事,根本无法想象。能让他们”

李宪叹了一声,放弃了对韩冈的劝告:“杀了也干净,给天子、给朝廷、给百姓都能有个交待。就是不要出乱子。”

“能有什么乱子?只诛首恶,胁从不问。首领和部众分开安置。这样一来,想怎么处置就能怎么处置。。”

到了入夜前,具体的数字也传来了。将几个数据在纸面上与其他方向上的回报加起来,这便是此战得来的战果。

官军对阻卜贼寇的袭击,光是斩首就有千五,投降的则更多。阻卜南下的队伍总数五千人,全部是骑兵。这一战,韩冈能将其中的八成都留了下来,七千多战马完好无伤,而财货更是数不胜数。阻卜人本来是准备抢到最后一笔就收手,谁想到最后一步却遇上了韩冈。

“缴获的财物怎么办?缴获的战马是否直接归公就够了”

李宪记得韩冈一直都是采用四三三分账的。南征交趾时的战利品,都是士兵四成、军官三成、归公三成,然后抚恤就从这三成中取得。虽然跟过去军中的习惯不同,但比起毫无组织性的烧杀劫掠,严整有序的洗劫和分配,却能让人得到更多。不过李宪知道,这件麻烦事韩冈肯定会推到自己身上,与其之后夹缠不清,还不如这时候就问明白。

“将士们都是辛苦作战,没必要占他们的便宜。战马完好的二十贯一匹,受伤无法恢复的五贯。我这里边还会从府库中拿出一部分前朝来安民,至于损失的财货这是没办法计算的。”韩冈想了想道,“将会在丰州、麟州等所有受到阻卜贼寇劫掠的村寨行刑,血债血偿嘛……”

看见李宪欲言又止,韩冈叹道:“此事如何处置,其实无关紧要,还是想想盐州的情况如何了?”

盐州,从城下战败那一天开始算起,应当有十天了。

韩冈总觉得辽人那边是不是太过于沉默了一点。这段时间所展现出来的耐性,完全不符合耶律乙辛或是萧十三之前的表现。难道安排一下阻卜人帮助西夏之后,就坐下来等着局势的发展?

正常的情况下应当是出手引导局势向对自己有利的方向发展,这是普通人都会有的想法。而韩冈则是心中警惕感大起:“得再给北方诸州去信,要刘舜卿他们加强前线防备。”

通过小规模的冲突,向对方施加压力,这是尚不愿撕破脸皮,却又想在对手那里获取利益的国家必然的选择。

之前发生在雁门关外的边境冲突,就是契丹人在施加压力。韩冈知道,天子和朝堂无论如何都不会答应将低烈度的边境冲突扩大成一场战争。

河东能够动用的军粮也差不多快用尽了,各军州的常平仓中虽然还有粮食,但那是备急备荒的存粮,非到危急时刻,只能以新粮替陈粮,绝不能随意动用。

眼下的局面究竟会如何发展,还是得看盐州之战的结局。

