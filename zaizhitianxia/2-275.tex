\section{第43章 长风绕城遥相对(上)}

夕阳西下,漫天风沙中,一队骑兵缓缓踏上回营的路途。

百多名骑士的身上都是血迹斑斑,带伤的也为数不少。小规模的交锋对阵,也不输决战般的同样惨烈。刘源押在队尾,身上的甲胄上能看到好几支被截去后半段箭杆的长箭。都是被利箭射穿了硬铠,不好拔出,准备回去找工匠修理。

自从前日抢在西夏人反应过来之前,韩冈命麾下将士强行设立营寨以来。党项人来进攻过营地两次,但都被寨中守军给打了回去。而党项人不再骚扰营地后,韩冈就开始命令王舜臣、刘源等将领轮番出击——在敌军身边,不能一点动静都没有,这样会损伤士气,也不利于让城中守军坚守下去。

“回来了!?”

一声极有精神的问候,穿过黄色的沙幕,传到了众骑兵的耳中。

几名身着甲胄的战士就站在营地的大门前,最前面的一人矮而壮,宽阔的肩膀将一副山文甲紧紧的绷起,厚实的身躯看上去就像一块放在地上的磨盘。不是王舜臣又是谁人?

听到声音,又模模糊糊的看到了王舜臣的身影,刘源当先跳下马,抢过去拜见,“小人见过都巡。”

“尔等辛苦了,就不必多礼了。”

王舜臣看着刘源一众的马上,首级倒没看到几枚,但有着好几头骆驼。

刘源见着王舜臣的视线留驻的地方,便苦笑道:“今天与西贼狠斗了一场,斩首没几个,就是抢了些牲口回来,也算没白跑了。”

今日刘源出战其实是吃了个小亏,损失虽不大,虽说抢回来一些战利品,却也还是无法弥补损失。

王舜臣则是不以为意的哈哈一笑:“这等鬼天气,能有些收获已是万幸,其他就不必再多说。”

自昨夜开始,从六盘山对面吹来的沙尘便是遮天蔽日,睡在在帐篷中的宋军士兵听了一夜的风声,还有不停的落在帐篷上的沙沙的声响。清早起来时,天地都是土黄色的,回头看看帐篷,也都染成了黄色。迎着风张口说话,转眼就是满嘴灰土。一不小心,就会被风迷了眼睛。

不过只要天上不下刀子,恶劣的天气反而是出兵作战的良机。

从昨天晚上风沙起时,韩冈就让王舜臣加强了营外戒备,又立刻派了得力人手,顺着洮水河道潜入临洮堡。一个时辰后,派出去的几路斥候,就分别从不同地点听到了城头上传来的事先约定好的信号。虽然他们往敌营放火的行动没有成功,但能与临洮堡沟通上,也算是完成了任务。

到了早间,天壤之中更是变成了伸手不见五指的情况。韩冈筹划着进一步的行动,而对面的敌人,已经开始调兵遣将,加急攻打临洮堡了。

刘源当即奉命领军出阵,战鼓一遍遍地敲着,而山中的青唐部蕃军也同时在骚扰着敌营。两边同时动手,硬是要逼得西夏人将他们攻打临洮堡的兵力调回。

西夏人在韩冈和包约两部的威胁下,坚持了不短的时间。不过最后还是撤了下来,这也让韩冈松了一口气。如果西夏人再不回营戒备,他就不得不领军出击了。

王舜臣陪着刘源一起往营中走,韩冈听到消息,也迎了出来。走到近前,正听见王舜臣说着明天要上阵练练手,不能让箭术荒疏了。

可当王舜臣也看到了韩冈的时候,不待韩冈瞪眼,他就立刻就停了嘴,不多话了。

王舜臣昨日领军出阵,到了阵上便一马当先,名震关西的连珠箭术依然让人叹为观止,但等他回来,迎接他的就是韩冈的训斥,“你是主帅了,不要随便上阵。”

韩冈的命令,王舜臣不敢不从,而且说得有理,现在就只能羡慕的看着刘源和其他几个将领在阵前拼杀。

“三哥你看,这些骆驼看起来还不错!”王舜臣掩饰一般的走过去,想想拉着一只骆驼给韩冈看。不成想差点被咬了一口。骂了一句‘好畜生’,他一把扯着缰绳,赌气般的用力踹了骆驼一脚才走回来。

韩冈为之失笑,转过来正色问着刘源:“可是撞上了泼喜军?”

刘源摇摇头,“不是泼喜军,只是骑骆驼的铁鹞子罢了。”

泼喜军是西夏军中的汉人部队,但与被征发起来的炮灰‘撞令郎’不同,他们是西夏军中为数不多的技术兵种。使用的是架在骆驼背上的旋风砲,也就是一种小型的投石机。战斗时往往抢占高地,然后在高地上‘纵石如拳’,一片飞石砸下,比起弩弓威力更大。不过人数倒不多,据韩冈所知,才两百人的样子。

王舜臣一攥拳头:“要是在沙场碰上了泼喜军,定是要杀光这群忘了祖宗的西贼走狗!”

“等遇上了再说。”韩冈看了看一脸郁闷的刘源,前面他的侦查行动可是明着说泼喜军到了,现在才知道是个误会。党项族中,有许多部族并不算富裕,出兵时往往都是一匹马一头骆驼,平时骑骆驼,战时骑马。但到了风沙飞舞之日,骆驼比战马要可信得多。“既然遇上的不是泼喜军,只是群骑着骆驼行军的党项人,那么应该就不是什么主力。”

刘源点点头:“这两天小人跟西贼斗了几场,也的确没发现他们有多精锐。比起禹臧花麻和木征手下的骑兵要强些,但与真正的精锐感觉还是有些距离……感觉景都监败得有些冤。”

王舜臣也又跟刘源同样的感觉:“恐怕他们能吃掉景思立和他的两千兵是个意外之喜。”

韩冈沉吟起来:“泼喜军不在,那西贼领军的将帅也就不可能是仁多零丁了。”

虽说泼喜军并不归仁多零丁管,但两边都是兴庆府中的王牌。如果泼喜军出动,主帅的地位必然不会低。同样的道理,如果仁多零丁出阵,最精锐的环卫铁骑虽然不能动,但其他几支精锐必然要出动其中的一支或几支,不可能是擒生、撞令郎这样的队伍来敷衍塞责。

三人一路回到营帐中,韩冈让人拿了水盆和茶水来,让刘源洗脸漱口。

解决了个人的卫生问题,将满是灰土的甲胄卸下了下去,刘源整个人都感觉轻松了不少。坐下跟韩冈和王舜臣继续方才的话题。

韩冈说着,“统军使出战,本来就不可能只带着万多人。不过仁多家的旗号既然在临洮堡城下,那必然是有仁多家的将领出来领军。不知除了仁多零丁以外,刘源你知道几个仁多家的将领?”

刘源皱起眉来,在记忆中仔仔细细的搜索了一阵,最后颓然的向韩冈摇了摇头。他的消息并没有这么灵通,以韩冈的身份都不知道的事,他更不可能知道。就算他多了几十年在缘边地区的征战经验,也还是不可能了解到西夏国中这般详细的内情。

“管他是谁人领军……等机会来了,就将他们埋到地里去。正好这两天刮沙,转头就能把他们的坟头堆起来!”王舜臣叫了起来。

韩冈瞪了他一眼,这小子是在装粗呢。王舜臣外表粗豪,内心却一贯的细密深沉,在武将中算是考虑问题比较周全的难得人才——至于喜欢上阵厮杀,只是他太年轻的缘故,年纪大点就会好的。

不过,王舜臣捧场般的说话,自己也好顺便将话题转移到他要说的方向上去。

“话不能这么说的,没有身份足够的将帅压阵,就证明熙河这边并不是西贼的主攻方向。党项人援助木征虽是今次目的,但也不一定要攻打熙河路。秦凤、泾原都可以!”

“但如果景思立败阵的消息传回兴庆府后,他们会怎么做?”刘源问着韩冈,“主攻方向难道不会改变?”

“什么都不会!他们来了吃什么?”

韩冈不信梁氏兄妹手上有能变出粮食的口袋。熙州北部的蕃部早给禹臧部和青唐部联手给洗个了干干净净。西夏人能在临洮堡下撑到现在,韩冈已经是很惊讶了。

“如果梁氏兄妹打算想增兵熙河,先让他们手下的党项人学会餐风饮露的本事再说。反倒是秦凤、泾原两路,这两年缘边蕃部都算是丰收,随便开个堡子,就是几万人一个月的口粮了。”

西夏军来得终究是迟了一步,要是再早一点,赶在大军还陷在河州城下的时候,那样的情况就危险了。这也是一开始就在经略司考虑的范围之内,消息传播的速度是有限的,大宋攻下河州城的行动也没有耽搁时间,等兴庆府反应过来,当然就已经来不及了。

现在的一番骚扰,要不是王韶率军南下,根本掀不起波浪来。

眼下的局势,韩冈自知是走在平衡木上,一点差池,都会造成难以挽回的后果,但独控大局的感觉却也让他难以舍弃。现在至少局面当真给他稳定下来了。

“拖下去,党项人快要撑不住了!”韩冈肯定的说着。

