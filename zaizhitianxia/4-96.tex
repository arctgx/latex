\section{第18章 青云为履难知足(二)}

【今天大封推,晚上七点加更一章。顺便向各位兄弟求一下红票和点击,想看一看能冲到第几名。】

用了一个多月的时间,韩冈指挥着邕州城的居民,总算将城中的废墟给清理了一遍。

当韩冈和两万多邕州百姓终于从忙碌中抬起头来的时候,城外田地里面的占城稻已经长得郁郁葱葱,都开始抽穗灌浆了。

尽管不可能用焕然一新来形容,但至少不再是尸臭漫城,而位于主干道附近被焚毁的房屋,也都处理干净了,站在高处放眼望过去,日前还瓦砾遍地的废墟,已经变成了一片片空白的场地。大部分场地的一角,还整齐的堆放着尚能继续利用的瓦石和木料。在这次的劫难中,被回到的屋舍也多在主要街道附近,城中偏僻一点的位置,损坏并不算多,也不需要刻意去清理。

章惇早已回桂州去了,他在担任广西经略使的同时,也还是桂州的知州。不论从什么方面来说,他都不能将手上的职事交给通判太久。

韩冈现在还代管着邕州的大小事务,不过他已经将具体的施行工作交给了苏子元来负责。他和章惇联名推荐苏子元为邕州知州,不论是从苏子元立下的功劳上,还是他本人的身份上,这项提名,朝廷驳回的可能性很小。既然苏子元做定了邕州知州,自己就没必要管得太宽太多。

一开始,苏子元其实是准备报丁忧,为父母守制三年的——居父母之丧,庐墓三年乃是常理。

“父母之仇,不共戴天。不思复仇,可为孝乎?”这是前些日子韩冈劝说苏子元留下来任官时的质问。

在章惇韩冈一起承诺了要奏请朝廷,继续攻打交趾之后,苏子元同意了担任邕州知州。不过为了合乎世间的规矩,他照例上书请丁忧走个过场,省得有御史闲得没事找茬;而韩冈和章惇则上书为其请求朝廷夺情。

这并不能算是特例,镇守边疆的重臣——虽以武将为多,但文臣亦偶有为之——朝廷往往都会下旨夺情,不会让个人的丧事问题,影响到边疆的防务安全。只要朝廷同意章惇韩冈的申请,批复下来后,苏子元就不会拒绝。

韩冈现在的主要精力,还是放在医疗卫生上。攻打交趾,最大的敌人不是交趾兵,而是南方恶劣的自然条件。没有完全开发出来的土地,加上让北方人无法适应的气候,这是狄青当年所领平叛军最大的伤亡原因。

在韩冈看来,抵御疾病主要的手段还是要靠严谨合理的卫生制度,防病永远都是要放在治病之前。如果想要保证南下大军拥有足够的战斗力,至少在医疗制度上就必须更加严格。

在邕州一个多月的时间里,韩冈领头,加上雷简等一众随军医官的襄助,一部比起韩冈当年编写、几年来又屡经修订的通用版本、要更加严格的南方卫生管理条例,逐渐有了雏形。

当韩冈将定下的制度一一颁布之后,由于性命攸关,绝大多数的士兵还是按照制度来行事。甚至连邕州的百姓,也依从了其中的大半规矩。

韩冈足可以感到自豪,在他收拾残破的邕州的时候,没有发生一场流传广泛的瘟疫,只要一有苗头,就立刻加以隔离。虽然不是没有病死者,但只要不是由疫病转为疫灾,为数不多的个例并不会影响韩冈在邕州的威望。

只不过要保证军中的医疗卫生安全,首要的就是成本问题。单是军营中每天都要喝开水这一项,消耗的木柴就是大数目,为此要付出不少的代价。幸好还是初战,加上兵力不多,朝廷给钱给得爽快,且韩冈又是专管南征事务的转运副使,为此调拨钱粮的事,只需要自己的签名就够了。只是到了数万大军南下的时候,又该怎么说服朝廷花这笔钱,同时还要保证燃料的供应,就要伤脑筋了。

就在韩冈苦思着该怎么节省卫生成本的时候,来自京城的中使一行抵达了邕州,来的是老熟人李舜臣。

‘考功郎中、龙图阁直学士、广南西路转运使’,这是韩冈新任的本官、职名和差遣。

说实话,韩冈虽然有心理准备,但真正听到李舜举念出这几个官衔后,还是免不了有些激动。考功郎中属吏部,在六部中排首位,尽管依旧不领实职,但排序还在,属于前行郎中的行列,离正六品也只剩一步——当年的秦凤经略李师中,就是正六品的右司郎中。而龙图阁直学士更是一步越过了侍制这条区分重臣和寻常朝官的分界线。所以以他的本官和职名,做到一路转运使已经连权字都不需要——尽管是天下诸路中排位靠后的广西。

只是兴奋了一下,也就过去了。本官官阶韩冈早就不放在心上了,进了朝官序列后,只有决定俸禄的用处。职名跨过侍制之后,也没什么大用,不过勉强能被人称为韩龙图了,就是感觉有些怪。

章惇是端明殿学士,还赐了爵,有了实封。李信则是广西钤辖、文思副使,从都监至钤辖,又在四十阶宫苑诸使和使副中,一下向上跳了十阶——正常的一等功赏,只是七阶而已。苏子元那边没有出意外,进了朝官行列,夺情后并权发遣邕州。

黄金满也有封赠,开国子的封爵没人稀罕,但将名义上属于大宋的广源州,从刺史州升格为团练州,并让他担任团练使,则是难得的奖赏。因为他还有一个蕃部都巡的差遣,是能得到俸禄的实职,而不是送给四方蛮夷的那些好听官衔。除此之外,所有有功的将士,连同何缮那个降人,都有了丰厚的功赏。

就在城门前,韩冈领着一众文武和麾下的将士们,向着李舜举手上的诏书拜倒行礼,感谢皇恩浩荡。

结束了对功臣们的封赏,李舜举带来的另外一封诏书,则是追封以苏缄为首的邕州殉国文武众臣的。

就在城外搭建了到一半的苏缄的祠庙中,面对着以苏缄为首的邕州官员的灵位,李舜举张开了手上的诏书。

苏缄得赠谥号忠勇,追赠奉国军节度使,而通判唐子正以下,皆是厚加追赠。且只要还有子孙在,一并得以荫官。

“官家说了,‘邕州死事之臣,非可与钦、廉州比也。自为贼围,坚壁弥月,竭力捍御,而外援不至。守死一节,忠义不衰,录其子孙,宜加死事者一等,士卒倍赙其家。’”念完诏书,李舜举向韩冈和满面泪痕的苏子元述说着天子的心意。

李舜举此番南下,不仅是为了宣读诏令,同时也将韩冈急需的药材送来了一批。都是从在京的药库中挑选出来的上品,韩冈让雷简查验了之后,微笑着连连点头。

看着韩冈心情好,李舜举装作不经意的说道,“听闻龙学邕州大捷、俘斩万余的消息,官家欣喜不已,都说此一战以少胜多,历来少有得见,想亲眼看一看是怎么赢的这一仗……”

韩冈瞥了李舜举一眼,脸上的微笑不改,李舜举话中之意他哪里不明白:“首级全都存在库房中,还请天使少待,韩冈命人将首级都搬出来。”

韩冈早有准备,带着李舜举到了专门存放战利品的库房。一声令下,上百名库兵就将一副副口罩带在脸上,进了库房之中。李舜举隔着老远就嗅到一股浓香,里面当是放了此地特产的香料。

“首级存放的时间太久,如果没有这口罩,不知要薰昏多少人。”韩冈向李舜举解释着。

用了一上午的时间,堆满了仓库的首级终于都搬了出来。五十来个堆做一个金字塔,一堆堆排在仓库外的空场上。中人欲呕的恶臭味弥漫在库区中。

李舜举站在一堆堆首级前,成千上万张大着嘴、眼眶凹陷的骷髅挤满了视野,而口鼻之中,更是一股股恶臭气息直钻着进来。不仅眼睛受到刺激,连口鼻也同样受到激烈的刺激,忙向韩冈讨了十几个口罩给他的从人们戴起来,

斩获的首级,用盐腌了近两个月之后,已经变得干瘪瘪的,而且都发了黑,几乎就是一层厚厚的黑膜蒙在头骨上面。不过这些首级身份确认倒是不难。先不说其中全都是成年的男子,与宋人装束相近的交趾兵,脸上都刺了字;而从交趾国中部族里招来的士兵,相貌和装束又迥异于宋人。

李舜举和他的从人一个个的仔细查验着。

“想不到还是要查验……”李信在韩冈耳边低声说着。

从这件事上,就看得出赵顼的为人秉性了,说起来是有些失了天子的身份。

既然在遣人点验之前,就已经将功劳认定,并将封赏都发了下来,何必再多此一举。万一给确定了是他韩冈在谎报功劳,到时候是罚还是不罚?罚了,朝廷之前对捷报的宣扬都成了笑话。不罚,这欺君之罪难道就一笑了之?这是自陷窘境的愚行。最好的做法,就是大方到底,用个好听一点的名义——比如筑京观什么的——让韩冈将所有的首级直接埋了了事。

如果是王安石、吕惠卿,甚至是冯京、吴充,由他们来处理这件事,绝对不会犯这等错。可惜的是,李舜举明显奉的是天子的密旨。也幸好韩冈不是吹嘘,否则可就要尴尬了。

