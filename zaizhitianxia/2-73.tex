\section{第17章 家事可断百事轻(上)}

【新的一章终于写出来了。求红票,收藏】

“几个表弟?”韩冈诧异的问着,“四姨不是就生了一个?”

“你四姨是续弦,你姨父原配还生了几个。”韩千六为儿子解惑,他今天没去普修寺,正好留在家中。

“那些个哪算!?”韩冈嗤笑了一声,连二姨家的两个儿子他都不想认他们当亲戚,何况这种八杆子打不着的?

韩阿李不耐烦道:“不管算不算,你舅舅被人打了,你这个做外甥的就在旁边干看着?”

“娘说哪儿的话,此事孩儿怎会放手不理?不过这是什么时候的事?为的什么缘故?舅舅的伤势究竟如何?要不要紧?带信的人呢?他在哪里?”被老娘催逼着,韩冈不敢敷衍,一个问题接一个问题的问出口。凡事都要先了解才好说话,不明不白的被打了,也不知对错在哪边,怎么都扯不清的。

韩阿李则一摊手:“带信回来的说了半天也没说清楚,送了信就走了,也没留个地址,不知现在人在哪里。”

韩冈眉头蹙起,这叫什么事?!自家老娘是精明人,该问的不会不问,但她都说不出个所以然来,肯定是传话人的问题。真不知舅舅那边怎么挑的带信的人。

不过事情的起因、过程,对如今通行于世的律法来说,并不重要。虽然韩冈的四姨只是续弦,但这亲戚就是亲戚,从法理上说,韩冈四姨父原配的儿子,的确是韩冈的表弟,也即是韩冈舅舅的外甥。晚辈殴打长辈,在后世会被人指责,但在此时,却是个天大的罪名。

“皇宋是以孝义治天下,最重孝道,外甥殴伤舅父,这罪名可不轻。”

韩冈回想着疏律上的文字,十恶不赦的大罪中,排第四的恶逆一条,就说得是晚辈殴伤长辈,当是斩首,而且不必等待秋决,也不要指望有大赦。但里面的长辈主要是本家的父母、祖父母、叔伯兄长之类的,而提到外家,只有殴伤外祖父母算在里面。殴打舅舅应该无法归入恶逆,但从这一条推算下来,罪名应该不会轻。

“那好!三哥你去凤翔走一趟,不让冯家分说个明白,这事就不算完!”韩阿李一拍桌子,比起上阵前的大将还要有气魄,“让他们也知道,我老李家也不是好招惹的!”

“但传话的只说是舅舅被打,没说被打伤,程度上就差了许多。若只是一拳、一掌,却不好定案。”韩冈一听说要自己跑腿,便又改口推托着,他对这等家务小事都没什么兴趣。一直以来他对上的都是能让他家破人亡的主,从一开始的黄德用、陈举,到了后来的李师中、窦舜卿、向宝,很快又将迎来鼎鼎大名的郭太尉。区区一个冯家,值得他去跑腿?

何况还有李信这个做儿子的在,“先让表哥去。哪有儿子不出头,外甥先出头的?去天兴县衙也好,或是凤翔府衙也好,直接去告官,把那几个混帐东西都置之于法也就是了。”说了两句,韩冈又奇怪起来,“怎么不去找表哥,反倒找到咱们家了?”

韩千六道:“报信的说找信哥儿不方便,只能来咱们家。”

“表哥现在在张老钤辖帐下,天天在衙门里面。传信的也许不知道。孩儿现在就让小六去找表哥,这事肯定得先跟他说。”韩冈借着找李小六的名义,丢下一句,就往外走。

走在院中,韩冈心中还在想着这件事。自家舅舅是个都头,虽然不是官,但从韩冈他外公时起,李家就在凤翔军中任职,人脉广得很。而冯家,韩冈只听说是个豪富,至于其他就什么也不清楚了。两家斗起来,韩冈说不清谁高谁低,但从自己舅家请人来送信,而不能在凤翔府自行解决,应该是落了下风。

说起来自己做官半年多了,自家老娘托人带去凤翔的信也有五六次,但始终没个回话,现在有了消息,却说是舅舅给冯家的儿子打伤了。如果舅舅是跟李信一个性子的话,不是大事不可能跟人起冲突。也许是四姨或是冯从义的嫡亲表弟,在冯家受了什么委屈,所以舅舅出头会打抱不平,接着就被人打了。

李信从韩冈这里得到消息,当天就跟告了假,连夜往凤翔府赶去。李信现在虽无官身,但他是秦凤钤辖张守约身边的得力之人,又是他韩冈的表哥——时至今日,韩冈这个名字至少是名震秦凤,而凤翔府就在秦凤路左近,怎么想自己都该有点名气,凤翔府衙应当给点面子。

而且不管舅舅究竟是因为什么理由跟冯家起了冲突,既然冯家的几个小子动了手,那就是违反了孝道,都是自家占理。

李信走后,虽然自家娘亲还在耿耿于怀,但韩冈就把这件事抛到了脑后。一方面是没兴趣,另一方面,就在第二天,一件盼望已久的消息终于降临。

“信都白写了,白忙活了那么久!”王厚拿着刚刚到手的有着天子签押、中书副署的诏令,听他说的话的确是在抱怨着,但看他脸上的笑意,却是口是心非。

王韶和高遵裕的心情也是明显的好转,虽然写的一堆书信都要成了废纸,但他们仍然心情愉快。

就在王厚手上的这份诏令,是给予古渭大捷的功臣们的最好的赏赐——朝中终于下令,设立秦凤缘边安抚司衙门,以古渭寨为治所,管理秦凤路缘边地区的一应事务。

王韶为管勾秦凤路缘边安抚司,兼营田市易;而高遵裕是同管勾安抚司,兼营田市易;至于韩冈,则是管勾缘边安抚司机宜等事,王厚与韩冈差遣相同,不过跟高遵裕一样,前面也加了个同——同管勾缘边安抚司机宜等事——这代表了两人之间的排名关系。

虽然这一个秦凤路缘边安抚司,仅仅是附属于秦凤路下的分支机构,可这个衙门却是给了王韶半独立的财权、军权和人事权。而且治所放在古渭寨,明显的就是给日后古渭建军做铺垫。

大宋四百军州,两千县治,其中的编制、区划经常变动,有的地方县升军、军降县,来来回回都七八次了,什么事都没有,就是公文上改来改去,让人觉得麻烦。

但古渭却是个特例,位置也好,历史也好,人情也好,都已经不同于汉晋隋唐。简单的区划改变,牵扯到的变数太多。刚刚修筑好寨子的时候,朝中曾经有过复古渭州的动议。但为了不让附近的蕃部疑惧,朝廷最终还是决定只立寨堡,不设军州。

而现在朝廷终于有了在古渭寨建军的意向。第一个要感谢有个好大喜功、喜欢开疆辟土的天子,第二个,就是连续两次大捷的功劳,让朝廷的重臣们看到,至少大宋的权威在古渭一带能通行无碍,有着良好的根基。

拿到这份诏令,王韶自此就有了缘边安抚使的头衔,高遵裕职位与他相同,只是略低半级。而机宜的头衔,现在落到了韩冈的头上,虽然远远比不上秦凤路机宜文字,但‘管勾缘边安抚司机宜等事’,至少可以简称为机宜,而不是抚勾这个名字。

同时随之而来的是参赞军务的权利,让韩冈终于可以跟勾当公事厅里的繁琐公务——虽然很清闲——说再见了。不过韩冈的另一个差遣——兼理秦凤伤病事,却没有被削去,依然如故。

另外,赵隆、王舜臣和杨英三人得任缘边安抚司准备差事,虽然王舜臣和杨英现在应该才到京中,还没有正式在三班院挂名,但他们的差遣还是照样颁下来。也不知中书和三班院之间的交流上是不是出了岔子,不过不同部门之间由于交流不畅,搞出了扯淡的笑话,也是常有的事就是了。

“这些都是差遣上的调动,不知古渭大捷的封赏什么时候能到?”高遵裕有些迫不及待,前次受赏,是因为跟他八杆子打不着的托硕大捷,沾了点光,将食邑增加了一点,而且还是虚的,并没有实封。但今次古渭大捷可不一样,他可是全程参与的,又在战时,站在了古渭寨这个前线上,功劳、苦劳都不缺,以天子会军功的慷慨,肯定不会差到哪里。

王韶想了想:“大概中使还在路上,大队人马走的总不会有铺递快,不出意外的话,十天半个月之内就该来了。”

遣使赐诏是特例,正常情况,就是直接通过驿传把诏书送过来。但古渭大捷也算是特例,比起托硕大捷还要辉煌,托硕大捷能遣使,这一次,多半也会派个天使来传诏。

高遵裕突然叹起来,“如果来的不是郭逵就好了。”

如果秦州知州还是李师中,王韶担任缘边安抚使后,完全可以跟他在西面的军务上对着干,毫不理会秦州的命令,他已经有了这个权力,而李师中却没有压倒权限的实力。但郭逵完全不同,他在军中的地位、威望和功绩,窦舜卿、李师中之辈都望尘莫及,李师中在秦凤路上说句话,凤州、陇州的知州可以当他是放屁,但郭逵说一句,他们却不敢不重视,

“如果来得不是郭逵就好了。”王厚也跟着叹着,说了同一句话。

韩冈却为郭逵说话:“这话等郭逵到了再说,先要听其言,观其行。至于是不是阻碍,现在没必要想太多”

“即便郭逵与我为敌,我们这边也有天子在……还有王介甫!”王韶沉声说道,充满了自信。

