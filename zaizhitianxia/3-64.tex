\section{第23章 内外终身事(下)}

终于到了四月初六。

一大清早,天还是黑的,韩冈在汴河边租的独门院落,已经是灯火通明,人声鼎沸。

这座院子,前后两进,十来间房。虽然不算很大,但地势绝佳。靠着汴河,后门出去,就能叫来河中小舟出行的。租金一个月就要八贯,从九品判司簿尉的俸禄,只有这个租金的一半。韩冈要不是多了个集贤校理,多了一份俸禄,光靠太常博士的俸钱都要被吃掉一半去。而且因为是官产,不但盘不下来,就是以韩冈的身份,租金也打不了半点折。

有了独门的院子,原本韩冈带上京的两个伴当,便不敷使用。幸好王韶送来的两名婢女,韩冈再找人牙子雇了一个厨娘,家里的情况也就差不多像个样子了。而前两天,王韶又一气借了二十多个家丁,把韩家的门面给撑了起来。

此时韩府门外扎了彩棚,一溜沿出十几丈,虽然碍了几户人家出行。可一来,韩冈地位高、名气大,他事先遣人上门递帖子赔礼,周围邻居也都不会不讲人情。二来,韩冈要娶得是王安石家的女儿,大名鼎鼎的王副枢来韩家坐镇,谁敢找不痛快?

为了准备这场婚事,韩冈找了专门主持官宦人家婚礼的司仪,时称‘白席’。带了一帮手下过来,都是做了十几年、几十年事的,将婚事安排的井井有条。

又从附近的正店唐家楼定了宴席,等婚宴开始后,来款待客人。在韩冈看来,在东京,服务业发达得不比千年后稍差。一应事务都有专业人士照应,并不需要自己忙里忙外。

现在韩冈穿了一身玄纁朝服,黑色深衣,赤黄色的下裳,头戴三梁进贤冠,踩着皂色的厚底官靴,犀带系在腰间。从进贤冠两侧有珠玉垂于耳边,又称‘充耳’,随着韩冈行动,而轻轻晃动——这是有官身的士大夫娶亲时的装束。

冯从义作为监督,视察着韩府内外情况。他在韩冈婚期前赶到京师,是韩阿李的吩咐。自家儿子要成亲,总不能身边一个亲眷都没有。而且冯从义也正好借这个机会,来为熙河的特产开拓京城的商路,并去探望一下他的岳家——韩家跟太后家有着亲戚关系,这件事,也只有寥寥数人知晓。

从前几日开始,上门送礼的就络绎不绝。到了今天,受了邀请,上门来参加婚宴的亲朋好友陆陆续续的都来了。

基本上都是以同年进士为主,这是基本的人情往来,一榜同年,在官场上算的是紧密的关系。

四月前后,是今科进士结婚的高峰期。只要是未婚且没定亲的,基本上都是这个时候进了洞房。而已经定亲的,也赶着回乡去完事。韩冈前几日还受邀参加了另外两名新科进士的婚礼,今天就轮到了他。

而韩冈交好的高官之中。王韶父子当然不可能不到,他们今天各有职司,还不能算是客人。

吕惠卿等新党人众,则是在女方家等着。另外章惇如今去了荆湖,不在京师,但他父亲章俞却在,与路明一起过来了。老头子风仪还是那般出众,笑呵呵的恭喜着韩冈。

不过种谔没有到,他是三衙管军的太尉,龙神卫四厢都指挥使。尽管与韩冈关系不错,可也不便参加宰相女儿出嫁的婚礼。文武高官交相勾连,那是天子最为忌讳的。而种建中就没有问题了,有张载那一层关系在,没人能够从鸡蛋里挑骨头,所以他早早的来了,还带着种师中。至于种朴,他并没有到场——他二月时外放原州,在他的伯父种诂的手下听候使唤去了。

种建中一到,连着拱手:“恭喜玉昆,贺喜玉昆。金榜题名,洞房花烛,这下可都全了。”

种师中也上来作揖行礼:“恭喜韩三哥。”

“彝叔你也就不要笑话小弟了。”韩冈与来贺的种建中说笑了几句,拉着他问道:“审官东院可定下了去处?”

种家的十九哥,已经通过了明法科的考试,有了一个出身。靠着这个出身,种建中从武资转为了文资。他旧有的官阶属于大使臣一级,转成了文资后,登时就成了从八品的京官。一旦外放,职位不会很低。

种建中摇头苦笑,“还没最后定下来。多半还是在陕西,不是下县知县,就是在经略司中打个下手。总也跳不出去。”

“将门世家嘛……”韩冈安慰的拍拍种建中的肩膀:“太尉多半也是希望你能在陕西多立功劳。”

种建中也是无奈的叹了口气。世家子弟的仕途,先天上就比寒门中人要平坦,但他们却控制不了自己的前路。

——如今种家第三代以种谔为首,下面种诂、种谊皆是一流的将领,其余兄弟也无不统领大军。鄜延种家,现在正是蒸蒸日上的时候。可是到了第四代,有点前途的,也就是种朴和种建中两人。如种师中这般,年纪尚轻,还看不出有多出色的地方。

别看种十九现在转成了文官,可若是日后种家后继乏力,第四代靠着种朴一人独力难支,说不定种建中还有投笔从戎的日子。比起缺乏根基、没有保障的文官传承,保持武将将门的传统,才是维系种家代代富贵的唯一途径。

“不说这些了。”种建中忽而洒然一笑,“今天可是玉昆你大喜的日子,怎么陪着聊这些事。”

拱了拱手,自去与其他认识的朋友打招呼。

韩冈是新郎,就算是迎客,也只需要见几个重要的主宾,至于闲散客人,由代为知客的王厚和慕容武来负责。

客人渐渐到齐,看看已经日影西斜,亲迎的时间将至。王厚就过来催促,“玉昆,时候已经差不多了。”

韩冈点了点头,所谓婚礼,就是该在黄昏时举行,现在日头已经西落,便是到了迎亲的时候。

虽然穿着宽袍大袖的礼服,韩冈依然是很利落的跨上马,带着一部鼓吹,还有随行一众亲友,浩浩荡荡,去王安石府上迎亲。

……………………

王旖坐在在梳妆台前,对着磨得发亮的铜镜,里面是一张如花似玉的俏脸,而身后则是自家的母亲。

今天王旖被精心装扮过,原来便是有着水乡女儿的清秀,如今更是显得仪态万方。但她被修过的双眉轻蹙,还是为了已经到了眼前的婚事而忧心不已。

本来这桩婚事已经没有多少波折,可是前段时间,因为经义局的事,韩冈是跟父兄争执了起来,王旖为此担心得夜中难以安寝。害怕这桩婚事最后落到她当初所担心的地步。

只是事情过后,大哥、二哥出门见了韩冈回来,说起韩玉昆,依然还是一团和气。王旖记得,韩冈当初曾经对自己亲口许诺,不会因为公事上的纷争,而坏了私谊。至少在现在,他还是信守了诺言。

但日后呢……王旖不敢去想,却又不能不去想。

“来了,来了!”王安国的夫人,慌急慌忙的走了进来。

“娘……”王旖转过身来,珠泪颗颗不由自主的从脸颊上滑下,抓着母亲的衣襟,“孩儿不要出嫁!”

吴氏一直都盼着女儿早点嫁出去,但现在看着二女儿,眼中也不禁下留泪来。捧起女儿的脸,用手巾擦着泪水:“痴儿,哪有这般说的。今日之后,就是韩家的人了。到了夫家后,要好生遵从妇德,悉心侍奉舅姑……”

吴氏絮絮叨叨再一次嘱咐着女儿。滴滴答的鼓乐声中,韩冈骑着高头大马已经到了近前。久在军中,骑在马上的气势非是等闲。背挺肩张,再加上庄严的礼服,让人看着就三分敬意。

王安石作为女方家长,在大门前相迎。亦是如韩冈一般,穿着最为庄重的朝服,上黑下黄的玄纁,与陪着天子祭天时,还有正旦大朝会这样的大典礼一样的装束

若是按照如今的风俗。新郎上门迎接新娘,岳家要用两只椅背靠着,上面放上马鞍,让女婿骑上去饮了酒或是做了诗才给下来。不过这等俗礼,也不会在王安石嫁女儿时出现。

依照官中礼节,韩冈和王安石,一向东、一向西,互相对拜过后。王安石正要引着韩冈入内,这时,李舜举带着天子的诏书,还有捧着礼物的一众小黄门到了相府的门前。

韩冈与王安石对视一眼,都是感到惊讶无比,而周围观礼的宾客中更是低低的响起一片喧哗。

天子直接具礼馈赠新人,情况其实很少有。除了宗室娶亲,地位够得上的官员,基本上是续弦,让天子不便为此赐物。就像范仲淹,他为族人设立义庄,寡妇再嫁,义庄出钱资助,而鳏夫续娶,就什么也没有。韩冈的官位有些勉强,但赵顼却还是下了诏,这一方面是给宰相面子,另一方面也是韩冈正得圣眷的缘故。

天子下诏赐物,乃是聊表寸心。诏书上的一番话说得四六骈俪,但总体上的意思还是祝两位新人百年好合,白头偕老。

李舜举念完诏书,韩冈上前一步独自拜谢。

一相,一参,为了韩冈的婚事而奔走。加上天子的参与,新科进士中从无这般荣耀。

【白天有事出去了,还望各位书友见谅。另外多谢三水的指正。前面要赶着发文,都没有好好校对过。字句上的错漏有不少,还请各位书友谅解。顺便推荐一下三水的作品九霄天帝,如果能不断更就好了。】

