\section{第八章 朔吹号寒欲争锋(11)}

听了韩冈的解释,苏颂不再就此事多问了。

既然洛阳元老有求于韩冈,那么占据优势的韩冈,也就有了选择的余地。

并非旧党支持韩冈,韩冈就要反过来回报他们的。韩冈做了参知政事后,一切的人事调整,他都会依从气学的需要来安排。他会感谢旧党的支持,却不会为他们与新党争夺位置。

当韩冈通过推举成为参知政事,那么这些在政见上没有共同语言的党羽——其实这两个字也值得商榷——在外比在内,对韩冈更为有利。

之前孙觉特意推荐了能够做事的傅尧俞给自己,其实也是那些元老政治智慧的体现。若是他们恃恩求报,韩冈连一句话都懒得与他们多说,一拍两散可不是多难做出的决定。

“那玉昆下一步打算先做什么?”走了几步,苏颂问韩冈,“玉昆在任上,定是想要有所作为吧?”

“邮政。”

对苏颂,韩冈绝不隐瞒。

苏颂微微一惊:“这是枢密院的事吧?!”

“但邮政既然为民而用,那就是政事堂的事了。”

“是因为玉昆你现在在东府中吧?要是还在西府,玉昆你会这么说?”

“不会。”

韩冈更是坦诚,苏颂哈哈的一阵笑,韩冈算是将他的心思给透露出来了。

下放邮政驿传于民间,是韩冈当初提出来政见之一。他重归两府,当然有心以此为核心,展开自己的规划。不过以如今韩冈从新党,邮政一事很可能会受到地方上的新党掣肘。为免于为人干扰,韩冈很想亲自督促一下邮政体系在全国范围内的铺开。

若是操作得好,邮政很快就能收费了,这笔收入,韩冈肯定是要拿到政事堂辖下。

“这可不好办了。”苏颂作难道:“玉昆别忘了愚兄坐在东面还是西面?”

“难道子容兄不想转任东府?”

“哪有那么容易。”苏颂摇摇头,“还得将各地门牌号码记录、补上。”

“还有各地的区划问题。”韩冈说道。

邮政所要面临的第一个问题就是行政区划的问题。

邮政体系与旧日的军情驿传不同。旧日,以的一条条线,而邮政则更近于一张网,通过不同等级的节点进行传输。

在过去,行政区划的改变影响不了军情驿传的稳定。但换成是民间的信件,地名改了,目的地的归属改了,这信就要大费周折才能送到,甚至有可能送不到。

“要及早将需要改变的政区给调整好,等各地地名正规化,城中户名和门牌号确定,这区划调整就尽量减少。”

“玉昆你是在说关西吗?”

韩冈点头笑道,“还要多谢子容兄相助。”

“关西各路的调整、裁撤本就是朝廷的需要,是苏颂的分内事。有助于邮政,算是一个意外。”

“许多人连分内事都做不好,子容兄在做的,更不能算是分内事。”

韩冈说着,更是想着接下来他要怎么安排。

关西是韩冈的基本盘、根据地。只有在关西的邮政体系有了出色的表现,他才能。而因为军事的原因,关西的驿传体系更加密集,辖下的人力畜力也更多,更近于民间邮递的需要,要早日成型,远比南方更容易。

为了让邮政能安然在关西推行,地方上的助力少不了,朝中的安排同样少不了。

王厚在兰州多年,他转调开封,兰州知州便安排了韩冈同门的师兄范育接任——加直龙图阁守兰州。气学的其他成员,虽然还没有多少能执掌州郡,但州中幕职官已经多见气学门人,而关西诸军州的州学县学内的教授,更是绝大多数为气学门人所占据。

另一位身居高位的张载弟子——游师雄,如今依然在凉州,执掌一州政事,并掌控一路军机。这依然算是边镇,地位远在内陆的安抚使之上。

但旧有的缘边五路,因为已经成为了内地,加之西军缩编,就没有了存续的必要。西夏灭亡后的短短时间里,泾原、鄜延、环庆、秦凤、熙河五路随着旧日敌人的消亡,而被陆续撤销。

于此同时,关西转运使路的区划也发生了变化。

熙宁之前,潼关以西只有一个漕司——陕西转运使司。但随着先帝赵顼接受了王韶提出的平戎策,将开拓河湟定位为独立于关中的战略方向,秦凤转运司便划分了出来,而陕西转运司也改名为永兴军路转运司。

不过随着吐蕃、党项这西北两大异族所建立的国家、部族相继相府,甘凉、熙河、以及宁夏三路都陆续归入了秦凤转运司辖下,旧有的秦凤路则显得过于庞大臃肿。

所以朝廷便决定,将包括凤翔府在内的秦州以东诸军州划归了永兴军路。而失去了凤翔府的秦凤路,便由此改名为陇右路。如果按照唐时区划来算,这是陇右道加上关内道西北的一小部分——唐代的陇右道,便是秦州向西,将北庭、安西两大都护府都包括在内。

现如今关西的区划,若以转运使路来划分,便是陇西和永兴军两路。

若是以经略安抚使路来划分,则是甘凉、宁夏,陇西、以及永兴军路,另外,还要加上新近要设立的安西都护路。

邮政区划遵从转运司区划,如果转运司路的区域划定后不再改动,这样一来,信件递送也容易许多。

这就是韩冈的打算,一步步的影响并控制政事堂,就要先从第一步开始。

在正门前与苏颂道别,韩冈回到政事堂中,继续熟悉新的岗位。

公务处理,自不必说。随着批阅的公文越来越多,处理起来也的确越来越顺手。

除了京内京外的政事安排,剩下最重要的便是人事。

军器监是韩冈肯定要拿下来的位置。

韩冈已经拟定将黄廉调离,但他不会急着将其请走,一时间韩冈还不打算将此事放在议事日程上,一两个月之后再动手也不算晚。先放出些风声去,然后看黄廉愿不愿意成为两党相争的焦点。不过现在正在给他确定一个好去处,如果黄廉知情识趣,韩冈也懂得如何酬劳他人。

在为黄廉确认下一任位置的同时,韩冈还没忘了将傅尧俞安排为唐州知州。尽管他对范纯仁、李常和孙觉这三位支持者显得格外苛刻,不过傅尧俞是元老们所推荐,当然值得看重。

但韩冈也不能阻止其他人视他为无信无义的卑鄙小人。为此,韩冈已经有所准备。

“推荐李公择任职河北,亏那灌园子有脸!”

“没有李公择,有他的参知政事能做?!”

“不仅仅是李公择,范尧夫和孙莘老都要外任,没一个留在京城中。”

“早知有今日,当初看着他落选就好了。”

一群人聚集在吕希哲家中,低声咒骂着韩冈。

“不会啊。”吕希哲对客人们很是无奈,两边的眼界差太多了。

吕希哲曾经在张载门下听讲,其时间还远在韩冈之前,但他受到虔信佛教的吕公著的影响,所学多偏近浮屠,求学于张载不久便又离开,如今与气学主流更是差了十万八千里。虽然说吕希哲试图糅合众家之长,所学所论也有方今气学的成分,可谁也没将他当成气学中人来看待。

不过他在京师,即是吕公著的耳目,也不是没有自己的看法。

“韩三向来是人敬我一尺,我敬人一丈。如今的确不能补偿,但依他的性格,不久定有回报。”吕希哲劝说着门徒们。

有人半信半疑,而全然不信。

但这个不久,的确‘不久’得可以。

次日开封府急报,刑恕自尽身亡。

刑恕。

苏轼当年在乌台诗案之后,虽没有受到重惩,但与他书信往来的许多朋友,包括司马光等人在内,都被朝廷课以罚铜。这样的处置,让旧党再一次明白了何为国是?也让苏轼的朋友一下少了许多。

这一次苏轼被卷进大逆案中,许多人都大喊侥幸,若不是之前的乌台诗案,使人不敢与苏轼结交,这一回大逆案,不知会有多少人被卷进去。

而曾经游走在诸多旧党元老门下的刑恕,他在洛阳,远比经历了乌台诗案的苏轼的人面要宽广得多。

只要他还活着,洛阳元老就不能安寝。天知道,刑恕的口供会被用来做些什么?以他们在政坛上多年的经验,也不难想象他们的政敌到底会怎么利用这千载难逢的机会。

而现在沈括、章辟光两人把持了审判和羁押之权,生死都在韩冈手中。

刑恕的猝死,让人怀疑其其中是否有黑幕。但身在政事堂中,没人敢将这份嫌疑宣之于口,就连张璪也只能改骂程颢:“程颢教出的好徒弟。幸好没有让他继续教授天子。”

张璪的发言稍稍冷场。赵煦做了什么,天下间已经无人不知了,‘幸好没有让他继续教授天子’这一句,恐怕是说晚了。

急忙补救,张璪立刻便道:“程颢所学不正,故而才会教出刑恕这样的弟子。”

韩冈脸色有些难看。

虽然与道学分道扬镳,但韩冈对程颢的尊敬依然未改。现如今程颢为刑恕所连累,让程颢本人安然无恙简单,可免不了在各种场合为人讥嘲。韩冈不在乎道学,但若是程颢被人讥嘲,韩冈坐视不言,未免有忘恩负义之讥,而且这样憋着话,他心里也不痛快。

“君子学道则爱人,小人学道则缢死。圣人早有先见,夫复何言?”
