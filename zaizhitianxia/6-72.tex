\section{第八章 朔吹号寒欲争锋(12)}

“君子学道则爱人,小人学道则缢死。韩大参倒是与当年的石参政一般的爱说笑话。”

“平常见参政,都是望而生敬,没想到还有如此诙谐的一面。”

“这下大程便能安然脱身了。谁还敢说他是教坏了弟子?”

“圣人门徒三千,能称贤者不过七十二。总是圣人门下,也免不了有不肖之辈。何况韩参政都曾在大程门下求学,抵得过十个刑恕了。”

孔老夫子曾经说过的是‘君子学道则爱人,小人学道则易使’,讲的是教化的重要性。韩冈一句谐音的缢死,倒把圣人之言,与刑恕之死给挂上了钩。

许多时候,一个笑话往往比义正言辞的驳斥更有用。因刑恕而来、围绕在程颢周围的议论,在韩冈的一句谑语下烟消云散。

站在开封府狱前,大理寺少卿李达倒是很佩服韩冈。不是为了韩冈尊师重道的一面,而是为了他的心狠手辣。

刑恕若是想要自杀,早就自杀了。时至今日方才在开封府狱中自尽,要说没有黑幕,也要人相信。而眼下控制着开封府内外的,正是韩冈一党,幕后黑手也就呼之欲出了。

可知道谁是幕后黑手,并不代表需要说出来。

任官在大理寺,平冤狱、断积案,这是李达的本职工作。但李达不觉自己有必要为枉死的刑恕喊冤,也不觉得自己有必要出头与韩冈为敌。

应付过去就是了。

李达想着,与开封府判官章辟光,继续谈笑风生。

李达与章辟光说笑了一阵,紧闭的开封府狱大门终于从内部被打开来。

木制包铁的大门厚达三寸,高近丈许。不知是上足了油,还是为了这些天进出频繁的人众,重新整修了一下,开启时没有一点声音,静静的将门后的世界展示了出来。

大门在李达等人面前敞开,一股腐臭阴湿的风便扑面而来,几声惨叫若有若无,从监狱深处传入人们的耳中。

站在门前,向内望去,入口后深深的长廊黑洞洞的,仿佛聚集了无数冤魂的巢穴,让人望而却步。

大理断刑少卿李达,毫不犹豫的抬起脚,走了进去。

大理少卿分为左右两人,左断刑,右治狱。断刑少卿决断诸路狱案,治狱少卿则推治刑狱。

这一次的大逆案,太后交由开封府审理。在开封府审结上报之前。刑恕好歹是重要的犯官,他的口供关系到整件案子的内幕。没有任何先兆的突然自缢,大理寺不能视而不见,李达便是被派来查验其尸身,到底是自尽,还是被人灭口。

开封府的仵作早写好了验尸的单据,李达也看过了。在发现刑恕自缢后,仅仅是将他解下来试图救治,发现没救之后,并没有搬动尸体,而是立刻上报。

这是开封府上报的内容。一层层的传递,一直抵达了御前。

但这些文字,他是一点不信,他只信自己的眼睛。

狱中廊道两侧牢房,塞满了男女老幼各色人等。

整个开封府狱,已经为大逆案的相关人犯及其亲属所填满。因其他罪名而被拘入开封府狱的囚犯,则全都转移到另外的地方。

牢房明显经过了清理,但多年积累下来的腐败气息,却残留难去。

看得出来,里面的犯官家眷至少没有受到通常犯妇在狱中受到的侮辱,饮食上也尽可能的做到了洁净卫生——若是无罪开释,便能留下一份人情。就算最后被判抄家灭族,官宦人家的妻女也都会没入官中,若是在开封府狱中留下无法治愈的伤害,教坊司那边少不了会闹上一闹。

但监狱毕竟是监狱,对比起过去的生活,这些官宦家属如今在狱中所感受到的落差感,比普通百姓被关进旧时监狱所感受到的落差,要远远超出许多。

李达往深处走着,对两侧牢房中交织着畏惧和期待的眼神视而不见。一名犯人看到李达、章辟光这几位官员进了狱中,扑过来大声喊冤,但无论是他凄厉的叫声,还是喊出来的几个让人耳熟能详的名字,都没能让李达的脚步慢上一点。

这些人与他的任务没有关系,该看到什么,听到什么,李达比谁都清楚。

不过紧随在后的狱吏却不会当做没看见。随即便有两人出来,熟练的往那名叫冤的犯官身上各泼了一盆冷水。在不便用棍棒教育一番的情况下,用冷水让人冷静一下,就是最好的选择。在初春的寒夜中,湿漉漉的身子会让人更加明白冲动的坏处——在这段时间里面,很有几个发了高烧,然后就被人从监狱中抬出去了。

嘴角含笑的李达,与随行的章辟光继续聊着。

“今日怎么不见知府升堂?”

李达今天过开封府来,虽没有往正堂去,但从那个方向上也没听道什么动静。

“大府告病在家了。”章辟光回道。

李达的脚步总算是慢了一慢,惊讶道:“昨天还好好儿的啊!发了什么急症?!”

章辟光叹了一口气,“是急症,病的夫人。”

“仁和县君病了?”

“病!的!夫!人!”

章辟光一字一顿,让李达终于恍然,不是夫人病了,而是夫人病。少了一个‘是’字,意义就大不一样了。

沈括当然会病。河北的李承之,进了贡院。李常又接了河北漕司的任。再过几天,南京的孙觉、齐州的范纯仁,全都要离京。沈括想要进枢密院,从哪里找票来?

他丢下新党帮了韩冈,以为能得到韩冈的帮助进入西府。可韩冈做了参知政事后,转头就将他丢到一边。不仅仅是沈括,韩冈可是将所有支持者都丢到了一边去,属于旧党的支持者一个都没留下——当然,以刑恕之死作为回报,对那些旧党已经足够了。

或许这就是韩冈的行事作风,肯定会给予回报,但不一定会是最想要的。

李达一边想着,一边笑着说道:“圣人有言,君子有三畏,畏天命、畏大人、畏圣人之言。这沈大府也有三畏,畏光、畏风、畏见人!”

“……其实还有第四畏?”章辟光故作小声的说着。

“什么?”

“兼畏夫人!”

两人对视一眼,一起哈哈大笑,丝毫不介意身后一众官吏的存在。

沈括在府中没有什么权威,在朝堂上也被视为反复小人,而章辟光却是因为早年要求二王出宫而开罪了太皇太后,在太后面前留下名字的,该奉承谁,在开封府中熬了多年的吏员们比谁都门清。

“不过沈知府进西府,想也不可能。”李达又说道。

“何也?”

“其他相公只要听太后的吩咐就够了,沈知府可还要再请示了仁和县君才敢去做。”

“说的也是。”章辟光连连点头,“要是大府做了枢密副使。太后说要向东,县君说要向西,那可如何是好?”

“那只能降黄巢了!”

唐中书令王铎惧内,曾受命领军抵御黄巢。其出兵后,只带姬妾随军。其妻闻之大怒,紧追而来。听到这个消息,王铎慌忙召集幕僚,‘黄巢自南来,夫人从北至,旦夕情味,何以安处?’幕僚回答,‘不如降黄巢。’

这是个流传很广的笑话。而当今的权知开封府沈括沈大府,若比起惧内来,却是半点不让先贤。

因此阴森恐怖的黑牢中,便又再一次响起一阵快活的哄笑声。

终于走到了牢狱的最深处,章辟光在一扇门前停下了脚步。

刑恕的牢房就在这里。

守在牢房前的狱卒上来行了礼,将门打开后便退到了一边。

“少卿,请。”章辟光伸手指向门中。

李达点了点头,并不辞让,举步跨进了牢门。

一走进牢房,李达举止神情立刻就变得沉稳起来。

一个笑眯眯的爱开玩笑的官员,变成了淮南路上让贼子夜不能寐的李二郎。

跟随入内的章辟光,也收敛了笑意,打量着这位新上任的大理寺少卿。

惟有眉心聚拢起来的皱纹,微微泛着暗红色,仿佛第三只眼睛,难怪会被称为李二郎。

在淮南东路提点刑狱衙门中的三年,李达接连清理了一百一十七桩积案,平反了十七桩冤狱,由此名震淮东,这是他能在四十岁的时候做到大理寺少卿的主因之一——另一个,就是在大理寺盘踞了三十年的正卿崔台符、少卿韩晋卿这对老冤家,他们两人的恩恩怨怨终于宣告终结,在一个月之内先后致仕,据称是领会上意,不得不退,这样才空出了两个重要的位置。

李达围着地上的刑恕尸身慢慢的转了一圈,又上前从头到脚细细的查验了一遍。

手指甲很干净,整个人也没有多少死前挣扎的痕迹,喉间的绳索痕迹十分清晰,在脑后分八字,痕迹并不相交,看起来的确像是自缢的样子,但也只是看起来像。

李达直起腰,抬头看了看房梁,又看了看刑恕的身高,张开手掌在绳索上比划了一下。

刑恕的身高加上绳索的长度与房梁的高度比起来,至少差了两尺,普通的牢房应该就没办法了,但这座牢房里,却突兀的放了一张凳子,正常的牢房中可没这种东西。

而且凳子只是一桩,还有几处无法掩盖的漏洞,让李达觉得极为刺眼。

这是谁做的?

李达直起腰,不满的向后面看了一眼。

这活儿做得太糙了,开封府狱吏就这水平?
