\section{第31章 停云静听曲中意(22)}

冲在最前的合里只只看到一丛飞矛直扑面而来,自己就像主动迎上去一般。他立刻低头俯身,紧紧贴着马背。但下一刻,胯下的坐骑惨嘶人立,合里只只觉得天旋地转,重重的栽下马来。就在他坠马之前,张大的眼睛中,映出了一支支飞矛扎进了儿郎们身体的画面。

第一排掷出的飞矛扫过阵前。最后一排的选锋大步前冲,越过了前面的同伴,冲到了阵前振臂一挥,紧接着又是一排快步而上。五排选锋轮次上前,滚筒一般飞掷出手中的铁矛。长枪入肉的闷声接连响起,人和马的惨叫声伴随左右。

笼竿城外,七矛杀七将。

梅山之中,单骑制敌寨。

李信掌中的飞矛是他克敌制胜的法宝,不过到了南征之时,他培养出来的选锋便替代了掌中飞矛,为他博来胜利。

在沉重犀利的飞矛之前,坚固的铁甲如同纸张一般脆弱。点钢枪尖轻易的撕开了甲胄,一下扎进了脆弱的肉体中。人也好、战马也好,都经不住来自三十步外的飞矛一击。

李信提矛在手,身侧的亲卫还抱着十几支,尚是一矛未发,可冲在最前的那一批辽军战士,竟是在眨眼间给选锋一扫而空。

滚倒在地的人马,挡住了将赶上来的后排骑兵。慢下来的骑兵,更成了一支支投枪最好的目标。就在几个呼吸的时间中,选锋一排排的上前,随身携带的飞矛一支接一支的投出。

哀哀人声,喑喑马鸣,血色大纛前三十步的距离如同天堑。大宋选锋以辽军人马的尸体和血水,在阵前画下了一条生死线。

合里只恍恍惚惚的从尸堆中爬了起来,他的头盔掉了,却幸运的没有受伤,那一支飞来的铁矛从他的背甲上划过,把坐骑的臀部给扎穿,却只让合里只掉下马来。

背后一片声大叫着,合里只头脑昏昏,什么都没听清。他犹记得要盯着,但不远处那名宋将手中那支显眼的长矛,是什么时候不见的?

还没等他想明白过来,胸前就传来一下猛烈的冲击。合里只向后一仰,眼前就不再是近在眼前却接近不了的敌阵,而变成了阴云密布的天空。他脑中浑浑噩噩,只觉得身上气力突然不见了,再低头时,却见胸口上不知何时一支长枪,正如旗杆般骄傲的竖起。

辽国年轻的战将眼中的光芒消失不见,双手垂了下来。李信脱手而出的飞矛,竟是将他前后连铁甲一并贯穿,长长的矛尖从身后探出,撑着尸身而不倒。

城上城下一片彩声暴起,李信面无得色,做了一件微不足道的小事一般从亲卫手中接过另一支飞矛。

看前方,就在被钉在地上而不倒的尸身之后,辽军骑兵已顿步停足,怯弱的不敢再向前。再瞥眼向两翼看过去,那里的情况如何?

前后交错的齐声怒喝,适时的从左右两侧传来。怒喝声的伴奏下,只见如浪尖碎影的刀光直劈而下,在惨呼声中,染得鲜红。

弓弩都是越近威力越大,抵近至二三十步,从神臂弓射出的木羽短矢,比不上标枪的威力,但瞄准了战马射过去的箭矢,也给辽军带来了不小的损失,但最可怕的,却是后方的骑兵在避让中失去了冲击力。

射击之后,步军没有再给重弩上弦。直接丢下了神臂弓,换上了斩马刀。宋军的步军阵列,紧密的阵型肩膀几乎挨着肩膀,高高举起的斩马刀只能看到雪亮的刀锋连成一线。李信连拒马枪都不准备便将八百步卒带出来,就是因为依仗着斩马刀的存在。

来自身后的鼓声,节奏又变了,沉重而缓慢,宋军步卒便用力的踏着鼓点,一步步缓缓向前,逆着如同洪流一般辽军骑兵,挥着刀毫无怯意的反冲而上。

刀锋倒卷,前方的刀光方落,侧方和后方的刀刃就紧跟着劈斩下来。步兵的阵列远比骑兵紧密,阵前的一名辽军骑兵,就要面对三五柄长刀。进退不得,抵挡不住,被反冲上来的斩马刀劈砍得血肉横飞。

一步一喝,一喝一刀,陌刀阵前,人马皆碎。

来自交锋处的惨叫声传遍了战场,见前面的同袍被砍杀得毫无还手之力,后面跟上来的契丹铁骑立刻转向,无论是包抄,还是回撤,都强于拥堵在阵前。

只是迟了。

同样射空了弩箭,又拔出了锁住缰绳的短戟,用力向冲到步军阵前的辽骑投出,宋军骑兵便纷纷翻身上马,配合着步卒的攻势,斜斜的向阵前辽军的侧翼撞过去。正欲转向离开的辽军铁骑被当头堵上了去路。宋军步骑左右交击,辽军军势一乱,瞬间溃败,毫无还手之力。

李信在遂城练兵经年,无干没、无空饷,抚行伍如亲子,训士卒如严父,精力和金钱不断的投入进来,也只练出这不到两千的强兵。但就是靠这出战的千五强兵,他便让七十余年未与宋军交手的辽人,终于用性命明白了为何先人会传下‘阵列不战’这句四字箴言。

耶律菩萨保在后看得目眦欲裂,挥旗驱马,直接冲出了中军。萧敌古烈一下没拉住他,就看着耶律菩萨保带着他的本队向着前方交战的区域冲了过去。

只是刚刚冲前了百十步,耶律菩萨保的眼前几条黑影闪过,只见三支黝黑修长的铁枪远远地从前方飞来,噗噗噗的一串闷响,就在耶律菩萨保的面前深深的钻进了地里。最近的一支就贴着耶律菩萨保坐骑的鼻子,刚刚起步的战马被惊得人立而起。

对八牛弩的畏惧,深藏在辽军将领们的骨子里。耶律菩萨保紧紧攥住缰绳,发白的脸上冷汗涔涔。他终于记了起来,遂城靠北的这面城墙上还有二十架八牛弩,是他方才拿着千里镜一架架数过来了。

“可惜!”城头上多少人捶墙大叫,只差一步就能斩获一个辽军先锋大将了。只差一点,就又是一个萧达凛!万一一举格毙敌酋,谁说不能再有一个让北虏无望而返的澶州?

但这三支从八牛弩中射出来的铁枪,也震慑住了辽军主将,让耶律菩萨保不敢再上前。主将退缩,出援的兵马就变得犹犹豫豫,行动之慢,却让人想象不到这是威震万邦的契丹铁骑!

阵前死伤狼藉,后方援兵不至,冲击军阵的辽军终于不支而退。宋军的骑兵挥舞着铁鞭、钢枪,追杀得辽军骑兵狼奔豕突,四散奔逃,直到两军之间的战场中央,方才得意的回转。

骑兵举着枪回到阵列中,更为响亮的欢呼声在城墙上响起,一排亲兵毫不客气的提着大斧上前去斩首取功。

李信拄枪而立,他出战本为一壮声色,又欲打压辽军气焰,让其不敢贸然深入国境。即便无法阻止,也要逼辽人在广信军多留兵马。但现在的这个结果,却比他想象的还要好一点。

远远地看了辽军一眼,李信面无表情的抬手向后一摆,鼓声停了,城头上的欢呼声停了,阵中的官兵们也静了下来,战场中一片寂静。

对面的辽军在看着,身后的将士在等着。

“回城。”李信说道。

清脆的钲声随即响起。

鸣金收兵。鼓车先退。两翼步卒后转,徐徐退了三十步。马军紧随在后,三十步后,重新列阵回头。最后只剩李信拄枪阵前,亲兵列队左右,可辽军气为之夺,竟不敢稍动。

再多看了对面浩荡敌军一眼,李信转身退后,随即没入阵中。

万余辽军就这么眼睁睁看着宋军一队一队交错着倒卷而回,由步至骑,又由骑至步,直至那面招展的赤红大纛消失于城门之内。

当遂城北门缓缓合上,城中的欢呼声再一次如雷霆一般暴然响起,万胜、万岁的呼声撼天动地。声浪滚滚,传之四野,竟把城外的数千战马给吓得乱嘶乱蹦起来。

在麾下骑士们的手忙脚乱中,耶律菩萨奴和萧敌古烈面色如土,相顾无言。

一军先锋才越境,便惨败在遂城城下。三两百阵亡虽不多,但一点战果都没捞回来,丢尽了自家脸事小,堕了三军士气乃是无可挽回的大错。到了尚父那里,保不准就会被摘了人头来提振军心。

两人心中皆是惶恐,却是半点也笑不出来了。

……………………

郭逵正在看着前方传回来的战报。

辽军入侵的急报,这两天堆满了他的案头。但情况远比他预计的要好,三关所在的雄州、霸州,依靠地势,顺利的挡住了辽军骑兵。虽说那里并不是主攻的方向,但这个结果,已经不能再好了。

作为主攻方向保州至广信军一线,遂城那里,李信战绩显赫,兵马虽少,却硬是给了辽军一个刻骨铭心的教训。

而保州的张利一也稳稳的守住了城池,击退了接近保州、北平的敌军,没有任何动摇。

放下了战报,郭逵轻轻的一拍桌案,外表纵然平静,心中也是沸腾不已。河北之西与河东只隔了一座太行山,中有八陉相同。河北鏖兵,河东自会出兵相助。只要河东的兵马到了,这一战就会纠缠在边界上。

北虏不能深入河北,对大宋来说,这一结果就意味着胜利!

就不知河东的兵马什么时候能到了!

