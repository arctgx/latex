\section{第11章 安得良策援南土(三)}

哪件事更紧急?

广西还是江南?又或是陕西、河东?

两边争执不下,这一日的廷议,终究也没能议论出个结果。唯一敲定的,就是刘彝要倒霉了。另外还有之前的沈起,他的责任也别想跑。再往前还要追溯再前一任的萧注,他也曾经提议要伐交趾,所以一样少不了受罚。

但这有意义吗?

从崇政殿中出来,韩冈也是为这见鬼的结果而心中苦笑不已。不过这也在他和章惇的意料之中,朝堂、官场就是这么一回事,有功时大家抢着领,出了事了则是最先想着决定责任归属,而不是首先解决事端。

廷议之后,宰执们被留在了殿中,而资格不够的两制以下的官员,则一起从殿中离开。

韩冈与章惇并肩而行,冷笑着,“看来陛下终于明白了,大事不可谋于众人。七嘴八舌,人各异心,坏事的几率比成事的可能要大得多。”

“现在还不是一样。”章惇回头看了一眼高耸的殿阁,口气中带着不屑,可眼底也藏着希冀。什么时候他也能成为留在殿中的一员。

章惇是个不甘寂寞的,他尝到了领军出战的甜头,这一回当然不愿意放过。但他领军的经验只有平定荆南山蛮,资历太过浅薄。远远比不上预测中可能会领军的蔡挺、王韶还有郭逵等统领过一路大军的将帅。所以他需要韩冈的帮手,调动他的老部下去打头阵,这样他才有一丝机会领军——虽然几率不大,但章惇一向是敢赌敢拼的性子。

而韩冈则是因为要尽速救援邕州,需要得到章惇的支持。他与苏缄有着几分交情,就不能坐视朝堂上拖延时间。虽然他心里也明白,自从邕州被围开始,到消息传到京城,已经二十多日过去了。以钦州、廉州被攻破的速度,邕州的确很有可能已经被攻克,说不定此时的大宋官军,已经退守昆仑关,甚至宾州、象州、柳州、桂州了。不过再想想苏缄几次三番的上书要朝廷提防交趾,韩冈又觉得还有一线希望,对于苏缄守住邕州城还是有几分信心。

韩冈走了几步,又道:“北方的丰州和罗兀,天子哪个都不愿放手,江南诸路,要保证不至于民乱。而广西,敢于犯界的交趾更是不能轻饶。但人力有时而穷,国力也是一般。”

“谁说不是。”章惇冷笑着,一说起大宋的军队,了解内情的都是冷笑的表情居多,“中国虽然号称拥兵百万,但河北、京营的禁军几乎都烂掉了,以教训士卒为主要目的的将兵法,也不能在短时间内将几十年不经战事的士兵,变成能战敢战的强军。南方更不必说,当年我去荆南,当地驻军的空饷吃到了五成,而整个南方诸路也才三万禁军。河东连麟府军都不行了,现在实际上堪战的精锐,也就是陕西的那么二十来万——禁军,加上一部分蕃军。”

“还有荆湖南路的一部分,至少还没来得及烂掉。”韩冈补充道,冲着章惇笑了一声。

章惇也苦笑着无奈的摇摇头。这才是他对说动天子将平定交趾一事交给他的信心。在河东、陕西两地禁军难以腾得出手的时候,只能当先调动荆湖南路的军队。

仰头看着渐渐接近的高耸宫墙,章惇对韩冈道:“今天廷议上的争执不是坏事。想必天子也不愿意再看到军情再被耽搁。如果天子明日绕过二府直接下旨,那我们就赢了。”

韩冈点了点头。他看如今的局面,的确是内外交困,天灾人祸,留给赵顼的选择并不多,总得要冒些风险。相对于交趾肆虐的广西,一直以来还算平静的江南其实不是那么容易出事。

在陕西——种谔北进银夏无功而返,只是占据了山口。虽说有了日后攻取银夏之地的根基,但逼迫占据丰州的西夏军队撤离的目的却没有达到。在韩冈看来,这其中也有种谔不肯冒险的缘故。只要卡住了山口,罗兀城就可以说是保住了。但再继续往北,直扑银、夏二州,就要冒着全军覆没的风险。

种谔的判断不能说有错。此次从绥德跃进罗兀城近百里,是因为这一路上的横山蕃部早就因为四年前的大战而残破不堪,才显得波澜不惊,官军也不用担心粮道的安危。但直驱横山北麓的银州夏州,不仅准备不足,而且谁也不能保证身后补给线的安全,只要西夏派出千余人在山中骚扰,山道上就别想走人,种谔小心行事也是理所当然。

在河东——丰州没有夺回来,一万五千的麟府军,加上太原紧急调援的一万兵马,却因为大雪封山,而不能越过古长城所在的山岭。因为这不是罗兀城。罗兀相对于西夏,是阻隔在横山之南的孤城,若事有缓急,山北驻军难以救援。而丰州则正好反过来,相对于麟府路隔着一重重被冲刷出来的沟壑和山岭,也是孤城。

为了攻取罗兀城,鄜延路准备了半年,而仓储积蓄更是有着四五年的积累。可命麟府军收复丰州,是仓促行事,就算临时打造雪橇车,也运不了多少兵。面对严阵以待的党项人,想攻上去都是件难事,夺回丰州根本不用想了。

在河北——有了辽主的警告之后,不仅是河东,连同河北,也要面对蠢蠢欲动的契丹铁骑。不论契丹人会不会进攻——可能性应该很小——但河北四路都必须进入战备状态。这样就会跟陕西的情况一样,在短时间内,很难调兵出来。

在江南——旱灾接着蝗灾,灾情严重,致使流民在道。虽然说南方这个时候,不至于会有农民起义;但韩冈记得就在几十年后会有圣公方腊,他依仗的明教这时候也该在江南传播开了。虽然被宗教勾引起的起义发生在此时的可能性不高,但要说一点也不用做防备,连韩冈也不敢下此断言。

荆湖南路的潭州是南方的战略要地,驻留军队的实力要远过江南的几个大郡——杭州、江宁的那些地方,在官员家跑腿、在酒店里跑堂的士兵,说不定比接受训练的士兵还要多。凭着江南的驻军水平,若有万一,也只能靠京中或是荆湖派兵了。

最后就是广南——交趾军现在可以继续围攻邕州,但也有可能放弃邕州,往攻广州。

“但此时未免太迟了一点。前日钦州陷落的消息传来,不就是已经下旨,让广南各州军各自谨守城防,不得妄自出战。广州有当年侬智高的教训在,更是不敢有所疏忽。听说了交趾破了钦州之后,必定会提防起来。”章惇道:“说起来,邕州虽然在广西路中算得上是一个还算富庶的州府,但还是远远比不上拥有市舶司的广州,攻下了钦州、攻下了廉州,只要交趾人肯多走一点路,猝不及防的广州很有可能瞬间被攻克。”

“谁让邕州更近?!”韩冈冷笑着,“而且交趾人也不一定是为了金银财帛,他们的野心一向不小,关起门来称帝,不事朝贡。说不定还会说只要木棉花开的地方,就是他们的地盘。”

“木棉?”章惇疑惑的问着。

韩冈笑了一笑,“是南方的特产,与西北种的棉花有别。”就把话岔开了。

在翰林学士院的玉堂中,韩冈和章惇重新将当今各地的局面推敲了一遍,不论怎么说,他们两人都是主张要尽快出兵援救广西。

最重要是在北方,这点毋庸置疑,但迫在眉睫的则是南方。虽说以交趾的国力,即便破了邕州,也攻不破桂州。可北方辽人入侵只是可能,而南方已经是现实了。

天子应该是要急着拯救邕州,而王安石也是有着同样的想法。只要噩耗还没传来,少不了要

议论终了,章惇放松一般的长舒了一口气,对着韩冈笑道:“如果愚兄当真能成行,少不了要劳动到玉昆。”

韩冈叹了一声:“盛名所累啊……”

韩冈他没有独立领军的经验,攻伐交趾的领军之任绝不会交给他。所以章惇要搏一把,韩冈就很干脆帮着他。最后如果天子不点他的将,想必章惇也没办法有怨言。

不过韩冈他也知道,自己肯定少不了要被点将。正如他所说,是盛名所累。如今提到军中医疗,就肯定避不开韩冈这个名字。章惇以西军为核心在荆南奋战,却没有多少因瘴疠而死,是最好的例证。而当年狄青领军南征,因病折损将近三一之数,而带去的蕃落骑兵,更是病亡大半。

两相一对比,谁都清楚,如果要领军南征交趾,韩冈肯定是其中的一员。章惇要靠韩冈出言相助,也是因为他在用兵南方上的发言权,不下于老于兵事的将帅。

韩冈与章惇又聊了几句,告辞出来。在外看来,他依然是平静如常,但在韩冈心中,此是已是烦躁的要命。照他的估计,邕州可是等不了多久了。

到底能不能调动荆湖南路的军队?邕州现在的情况又是如何?这些事,都不是他能确定的。可是只要有一线可能,他都会为此而尽力!

