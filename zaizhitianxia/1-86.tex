\section{第36章 不意吴越竟同舟(中)}

【第二更,求红票和收藏】

吱呀的推门声轻轻响起,“三官人,该起来了。”李小六的声音紧接着传入耳中。

韩冈从睡梦中醒来,朝东的窗户纸上泛着的旭日红光顿时映入眼中。成群结队的鸦雀,在楼下马厩中吱吱喳喳的叫着。

“什么时候了?”他有些困顿的问着。

“过五更了。”

“都这时候了!”

一惊之下,韩冈彻底清醒,掀开被子从床上跳下。一夜睡过,满脑子的酒意已经不翼而飞,只觉得神清气爽。随意的活动了一下筋骨,对空挥了两拳,呼呼有声。才几个月的修养,之前近半年卧病在床的生涯所留下来的遗患,便一点也感觉不到了。

毕竟还是年轻啊!韩冈庆幸的想着,幸亏投了好胎,十九岁的身体恢复力毕竟不一样。

简陋却还算清净的厢房内,铺在地上的地铺已经被收起,由于是二楼的缘故,李小六即便贴着地板睡了一夜,也不用担心地气侵体。而外间的刘仲武连同他的行李也是不见踪影。

“刘仲武呢?”韩冈指了指外间,问着李小六。

“刘官人刚过了四更天便启程出发了。”

“……跑得真快!有老虎追着他吗?”

韩冈只觉得好笑,惶惶如丧家之犬,急急如漏网之鱼,刘仲武的反应让他觉得很有趣。跑得这么快,好像身后被老虎追着一样。冬天日出得晚,他刚到四更就跑了,不知要在黑地里走多久,运气差点的说不定脖子都能摔折掉。

“三官人在刘官人眼里就跟大虫一样。”李小六也陪着笑。刘仲武昨夜被韩冈灌了一肚子的酒,今天一早又狼狈而逃,他看着也觉得有趣。

韩冈倒是没想到自己给刘仲武带来这么大的压力。看起来向宝的风评在刘仲武心中也是有数的。向宝自入军中以来,便一帆风顺,升到一路都钤辖也不过费了二十年出头的时间,晋升之速足以让张守约这样在边疆踯躅多年的老将欲哭无泪。

一生没受过什么挫折,故而向宝心气极高,权欲旺盛,全容不得下面的人有半点异心。而分了他权柄的王韶,还有落了他面子的韩冈,在他眼中便是死敌。刘仲武肯定就是对这一点心知肚明,才会跑得跟兔子一样迅快。

只不过现在刘仲武跟自己都是一条路上走着,又都是骑着马,一程程的速度又不可能差不了太多,就算想躲着他韩冈,也是躲不掉的。

虽然韩冈现在的地位远不比上一路都钤辖,但寻事恶心一下向宝也没什么困难。刘仲武是秦州本地人,在军中颇有令名,王舜臣和赵隆都听说过他,若能将他从向宝那里挖来,也是一桩美事。

其实韩冈自己并没有发觉,自他离开秦州后,心情比过去的几个月要放松了许多,否则也不会腾起什么恶作剧的心思。自他重生之后,一直被沉重的现实给压迫着,每每死里求活,虽然以强硬的手段将所有阻碍一剑斩开,但心思始终沉重。直到今次离开秦州那个环境,心头才豁然开朗,也有了开玩笑的心情。

“请官人早点洗漱上路,今天还有百多里路要赶呢……”李小六方才进来,早端了一盆热水放在桌上,连洗脸的手巾和漱口的青盐、牙刷也都为韩冈准备妥当。

韩冈应了一声,在李小六的服侍下更衣洗漱。平常人家刷牙用的是咬去皮的柳树枝,而富贵人家则买来牙刷使用,马鬃穿在木柄上,一根也不过六十文,沾了青盐刷牙,感觉比柳树枝要好。听说京中还有用茯苓等药材制作的牙粉,刷牙效果更强。

韩冈过来洗漱,李小六为他卷起袖子,递衣服,递手巾,小小年纪便干练非常,服侍得妥帖周全。韩冈一边刷着牙,一边看着李小六手脚麻利的打理行装,注视着十四岁少年后背的眼神微冷。

李家的家境旧时远比韩家要好,即便李癞子儿孙众多,李小六这个庶出儿子并不起眼,也不受他喜爱,但好歹也是个小舍人,但转过来服侍起韩冈,却能一板一眼,一点儿也不出差错。但这世上可没有天生下贱的仆役!

在外人看来,韩冈饶了李癞子这个罪魁祸首,是世间少有的宽宏大量,李癞子也是千恩万谢,一副要重新做人的样子。但韩冈深透世情,眼力如刀,怎么看得出来李癞子藏在心底的恨意,是如海一般渊深。人都是这样,往往看不到自己身上的错误,而总是归罪于他人。李小六能低声下气的小翼做人,若不是心有所图,如何会这般卖力?

宰相门前七品官,在高官显宦家中奔走的仆役,实际上的确能荐为官身。宰相、执政都有推荐家仆为官的权利。而即便不做官,官员家的仆役也能有许多狐假虎威的地方。韩冈前途无量,李癞子纵然恨韩冈毁了他家几十年的积累,但只要他想着重振家业,便只能把宝压在韩冈身上。

不过韩冈并不会计较这么多,李癞子恨自己毁了他的家业,若是对自己感恩戴德反而不合常理,就由他去吧,反正他也做不出什么。而李小六是个聪明伶俐又肯吃苦的小子,看得出来并不是跟其父一条心,倒是可以栽培一下。

洗漱打理了一番,韩干带着李小六下了楼去。李小六早早的就已经在厨房吃过了,端到韩冈面前的早餐,是西北有名的羊肉泡馍——虽然如今不是叫这个名字,而是称为羊羹,但实质上千年前后却都是一样的东西,也就加进去的调味料的种类要少上了点。

摆在韩冈面前的大海碗可以做脸盆用,装得满满的羊羹全吃下去足以把人撑死。这样多的份量是因为如今普通人家都是一日两餐,吃完这顿,要抵上一天的饿。而韩冈习惯于一日三餐,即便人在旅途,也要在中午时分,吃点东西垫垫肚子。也因如此,一海碗的羊羹韩冈勉强吃了大半便放下了筷子。

驿丞这时小心殷勤的走了上来。他手上捧来的簿册与后世旅馆登记没有区别。韩冈凭着秦凤经略司开出来的驿券,在七里坪驿站吃喝了一夜,这些吃的用的,都需要他签名画押来确认,以作为驿站年终审计时的凭证。

其实从制度上来看,宋代的官僚体系已经十分完备,文官治国代表着卷帙浩繁的公文地狱,任何牵连到官方的事务,都要留下字据凭证。

韩冈提笔在簿子上签名画押,随手向前翻了两页,除了刘仲武,没有见到什么熟人的名讳。毕竟还没有过完年,等过两日正月十五的上元节后,走上这条路的秦州官员便会络绎不绝起来。

韩冈吃完便继续上路,昨日骑来的马已经给换了两匹新的,都是在驿馆中修养了三五日脚力的良马,能支撑着韩冈主仆二人继续奔行。

穿梭于山峦之间,一日之后,跨下的坐骑已经汗流浃背,土黄色的皮毛被汗水浸透成了深黄。抬眼前路,陈仓山已遥遥在望。千多年前,刘邦自汉中出兵,明烧栈道、暗渡陈仓,重新开始争夺天下的地方,便是位于陈仓山下。而韩冈第二程的目的地——宝鸡【今宝鸡市】,也是位于此处。

此地已是凤翔。

韩冈进京须路过凤翔,他的舅舅李简便在凤翔府军中担任都头。只是凤翔府的府治天兴县【今凤翔】,位于渭水支流的雍水上游,离渭水有百里之遥,而他舅舅位于凤翔府北界的驻地隔得更远。韩冈虽是途径凤翔,也便没有必要特地绕过去打招呼。

早上走得迟了,当韩冈抵达宝鸡的时候,天色已晚。夕阳早早便没入西方群山之后。抬头上望,金星正在天边闪烁。狠狠又给了坐骑一鞭,再迟上片刻,城门一关,主仆二人就要在城外找地方住了。

骏马奔驰,远远的望着宝鸡西门处,一条入城的队伍正排在门前,韩冈心中松了一口气,好歹是赶上了。走得近了,又看见在队伍中一个高大汉子正牵着匹枣红色的骏马,排着队等着入城。

韩冈在马上哈哈大笑,那不是刘仲武,又会是谁?!

“子文兄,当真是巧啊!”韩冈远远的叫着,他直接道着刘仲武的表字,对刘仲武的称呼,越发的显得亲热。。

韩冈带着一点恶作剧的心理,看着回过头来的刘仲武挂下了一张脸。韩冈不理他的脸色有多难看,上前拉着他,也不去排队,凭着手上的公文直接进了宝鸡县城。

在城中的驿馆里住下,韩冈又扯定刘仲武到外厅喝酒。他有驿券在身,照规矩在沿途驿站都有一天三百文的饮食标准,昨日和今日他拖着刘仲武喝酒,计算着数目,也都正卡在标准上。

殷勤的给刘仲武倒上一杯凤翔府的名酒橐泉,清冽的酒浆在杯中摇晃,韩冈问着:“子文兄即是要同去京城,今早为何先走了,不与韩某一路?”

“小人见官人睡得正好,不敢打扰。”

韩冈脸色突的冷下来,微微眯起的双眼盯住刘仲武,盯得他视线左晃右晃,不敢与自己对上,方才轻声说道:“旧日的一点小事,韩某早已忘却。而向钤辖为人宽厚,也不会计较什么。难道子文兄还要放在心上不成?”

韩冈说话直截了当,反让刘仲武不知该如何回话。

几次接触下来,刘仲武的性格韩冈心中也有了点底。沉着稳重的性子,让他受到了向宝的青睐,带兵出征也不用担心他轻敌冒进。但这样的性格,遇到不按理出牌的对手,便会束手束脚起来。

刘仲武无话可说,只能低头喝酒。韩冈忽的又哈哈笑了两声,打破了尴尬的沉默,“说笑罢了。韩某知刘兄是心急着上京做官,才走得匆忙。不提此事,来,喝酒,喝酒!”

