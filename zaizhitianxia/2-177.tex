\section{第30章 肘腋萧墙暮色凉(12)}

【第二更。】

河东的败阵,究竟是什么原因,种谔显然无心多说什么。只是要麾下众官回去各自用心做事。并要求加快筑城的速度,并保持缄默,不得泄露这个消息。

散场之后,韩冈转身就走,也不跟其他人私下里讨论。他对河东兵败的原因还是有兴趣的,但种谔看样子不想说,多半是有什么内情,韩冈还是决定不去探听究竟。不过,当韩冈回到疗养院,恍若无事的照常处理公事,转过头来,种建中却来找他。

种建中来找韩冈,是来要送回绥德的伤病员的名单。前日陕西转运判官李南公押守城军械来,今天午后就要回绥德去。在预定的计划中,他顺路也要把罗兀这里的伤病员都送回后方——即将开战的罗兀城,当然不是养病的好地方。

韩冈早已经把名单都列出来了,人也安排妥当,就等着送上马车。没费什么手尾,就把事情与种建中一起敲定了。种建中拿到名单,该回去跟种谔回报。但他却愣愣的在门口站了半天,最后转过身来,问韩冈:“玉昆,你当真不想知道河东军因何而败阵?”

韩冈不问,种建中却自己送上门来。他来这里,本就是有心理准备韩冈会追问河东惨败一事,谁料到韩冈根本就不提,老老实实的遵照种谔的将令,只专注自己的一份工作,其他根本都不打听。作为一名下属,韩冈的表现可以说是模范,但种建中很不适应,河东败阵的事,让他有话堵在心里,不说不痛快。

韩冈看了看年轻的种家十九哥,意味深长的笑了一笑。从房间中的小火炉上拎了冒着热气的水壶下来,亲手给种建中和自己煎了两杯茶。把两杯茶在小几上对面放好,他这才坐下来慢悠悠的问道:“究竟是什么原因?”

看见韩冈不紧不慢的摆出了畅谈的姿态,种建中紧锁的浓眉稍稍舒展开来一点,摇头笑了笑:“玉昆你还真是临到大事有静气,这养气的功夫着实让人佩服。”

他把手上的名单收进怀里,回过身来也跟着坐下。却也不喝茶,而是长吁短叹一阵,才说道:“因为韩相公给河东军的限期是五天!……所以在神堂道上中了埋伏。”

“十五天?!”种建中没说清,让韩冈给听岔了,当即皱眉道:“这还走神堂道做什么?绕道走南面永和关旧路不好?在西贼眼皮底下走路,这不是找死。有十五天的……”

“不是十五天,是五……是一二三四五的五天!”种建中无奈的打断韩冈的话,“韩相公下令要河东援军必须在五天内赶到,所以他们没有绕道永和关,而是走得北线的神堂道。不过在路上被西贼居中伏击,因此大败。就太原出来的那一队仗着有守太原的吕公弼撑腰,照走永和关,并没有中伏,不过现在也退回去了。”

听了种建中的更正,韩冈发了怔。原本气定神闲的姿态,荡然无存。有些发傻的张开手,把五根手指张开来:“就五天?!”

种建中叹了一口气,扭过头去,摸着粗瓷茶盏,不说话了。

韩冈却急起来了:“韩相公怎么这么糊涂?!发这道令文发出的时候,没人劝过他?!……赵公才【赵禼】难道眼睛花了不成?!就让这文书从自己手上过去?!”

韩冈责难的诘问一句接着一句,让种建中无比难堪。去信让韩绛催促河东出兵的,可是他的五叔种谔。虽然其中具体条文,种谔事先不知,但韩绛的所作所为,也是为了能尽快让罗兀城安稳下来。

可是,要河东的援军在五天内赶到罗兀……

这要多低的智商,或者说多疯狂的头脑才会下达这样的命令?!

从河东往鄜延来,就算今次援军的集结地离着罗兀城稍远,其实也不过是一百多里地的距离。这点路程,如果走得是内地普通的官道,莫说五天,三天的时间也绰绰有余——也就是因为离得近,要不然,也不可能让河东出手修筑罗兀城的外围寨堡。

但那里几乎能算是敌境了!

神堂道所经过的地方,并不是大宋稳定的控制区,仅仅是近两年才因为宋夏两国的军势逆转,而被西夏放弃驻守的。但党项人的骑兵依然经常在其中飞驰而来,继而又飞驰而去。

西夏人驻守在左厢神勇军司的两万大军,能在河东和鄜延的夹缝中安然存在至今,其战力可想而知。今次河东出援,虽说北面的麟州府州那里,能牵制一部分神勇军司的兵力,但再怎么说,援军都是要在西夏人的眼皮底下行军的。

敌军随时可能出现,步步为营都嫌不够谨慎,韩绛竟然勒令他们要在五天内兼程赶到罗兀,在路上遭到了伏击还能怨西贼太狡猾吗?!

已经有不止一人说过,韩绛和种谔所制定的横山战略太过冒险。不论出兵罗兀,还是河东派援,都是走在钢丝绳上,一个不小心,就会摔下悬崖。第一次冒险,靠着种谔的能力,的确是成功了,但这不代表第二次也能成功。

韩冈也是从一开始就不看好这一次的战事,前面罗兀城成功得手,不过是出其不意罢了。而眼下河东败退,只是在夺取罗兀城后,兴奋的火焰上的第一瓢冷水。而后……当是陆续有来。

河东兵败,出去接应的高永能率军回返。而原定于由河东修筑的等四座寨堡,自然也是不了了之。西夏的左厢神勇军司经此一战后,士气军心大振,而河东方面,大败之后,短时间内基本上不可能再次出兵。也因此,罗兀防线的右翼有了一个阔达百里的缺口,如果西夏人够大胆,甚至可以出兵抄小道直插绥德城下!

——这还不如河东军一开始就不出援军!只要把今次败阵的几万兵堆在边境,都可以让西夏军不敢深入,而不至于沦落到现在这样的境地。

以河东军的情况,当支存在舰队都比出来丢人现眼有用。

韩冈拿起茶杯,毫无所觉的喝了口滚烫的茶水,立刻给烫得差点跳了起来。甩手把茶盏丢在地上,他也不管碎瓷片溅了满地:

“大帅什么时候回兵绥德?”韩冈单刀直入的问道。

种建中对韩冈的问题没有一点惊讶。眼下的局面,的确让种谔无法再继续留在罗兀城了。随着河东军的失败,罗兀防线的破局,使得即将到来的罗兀城守卫战,其关键点已经回转到绥德城处。

其实这也是明摆着的事,黄土高原千沟万壑,大小道路众多,派出一军深入百里偷袭,都不是多难的一件事。这也是宋夏两国交战中很常见的一幕,宋军之所以很长一段时间被西夏人压着打,就是这个原因。而为了解决这个让人棘手的问题,宋人才开始不惜人力物力,用了几十年的时间,构筑起了一道连绵千里、纵深百里的筑垒地域,来堵住每一处可能供党项骑兵入侵腹地的道路——但神堂道所经过的地区,却是缺乏这样的防御体系。

如今在河东兵无法来援的时候,罗兀城要想保持无恙,后方的安全,尤其是绥德的安全,必须得到保证。

“至少要带五千人回去!”种建中也不向韩冈隐瞒机密军情,虽然是私下里种谔对他和种朴说的话,但在韩冈已经看透了的情况下,再行隐瞒,就未免太蠢了一点,

“鄜延精锐尽在罗兀,就算韩相公能从他处调兵过来,也是不堪战斗的居多。长安那边又有司马光在看笑话,韩相公要是从他手上调兵,反而会造成关中局势动荡。不过绥德城本就留了三千兵,再加上带回去的五千人,以家叔的手段,足以稳守。西贼想要偷袭,却要防着反过来被吃掉。”

种建中看看时候不早,他还要回去把名单回报给种谔。起身告辞,韩冈送他出门的时候,他却又在门口停步:“玉昆,过几天你还是和我们一起回绥德。”

“也好!我就跟你们一起回去,到时再去哪里,就视情况而定好了。”韩冈也不故作姿态,他始终不看好横山攻略的态度,让他就此离开罗兀城,丝毫不用担心被人小看。

而种建中见韩冈答得爽快,突然又展颜笑道,“玉昆还是放心好了。自来用兵,顺风顺水的事情,我们从来都没奢望过。敌强我弱的情况见得太多了,还不是一直打过来了?上阵时只要不怕死,总能挣出一条路来的。就算西贼大军皆至又如何,去年梁乙埋统领三十万军南侵,中军全力攻打大顺城,可曾打下来?只要尽早把罗兀城修起来,光靠这座城,就足以让西贼无功而返!”

韩冈微微颔首,种建中这番话其实是不错的。战场上,本就没有必胜必败之说,一点意外就能使得战局完全逆转。就算韩冈自己,也不能说罗兀城必然失守。

可是……眼下的风向已经变了啊!

战术上的胜利,真的能改变战略上的劣势吗?

韩冈拭目以待。

