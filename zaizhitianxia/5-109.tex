\section{第12章 恶客临门不待邀(上)}

骑着雄壮的战马,梁乙埋昂首挺胸的进入了盐州城。

经过了一夜和半日的巷战,盐州城终于被西夏大军彻底收复。

徐禧、高永能,还有个叫李舜举的阉人都死在了城中,盐州城中的主要的将领和官员,只跑了一个曲珍。而宋人在盐州城中的军队,则可以说是全军覆没。在十多日的守城中,守军损伤太大,甚至连像样的突围都无法组织起来。

一将无能累死三军,这是徐禧的功劳。但这不妨碍梁乙埋为此而自豪。

不过兴奋的心情只有片刻,来自东面军情急报传到了盐州城中——种谔已经击破了设置在左村泽、柳泊岭和铁门关的防线,向着盐州直扑而来。

种谔来了。

其麾下的三万鄜延路马步军精锐,沉甸甸的压在西夏太后和朝臣们的心头。

比起高遵裕,种谔的用兵要更加圆熟老辣,难以抵挡。

而比起已经在灵州城下精锐尽丧的环庆军,鄜延军甚至大一点的损伤都没有受到,几个月来都在养精蓄锐。

要想保住银夏之地,肯定要挡住、而且还要击败种諤和他的麾下大军,这样才能去收复银州和夏州。

已经无力去责难,派去阻截种谔的将领办事不力,现在的关键是谁先去抵挡种谔?消磨他的锐气?

盐州城衙的大堂中,没有人回答梁太后的问题。

这个议题之前在攻击盐州时就做过议论,当时的决定是再议,等种谔的反应再做应对。

种谔对盐州的态度,细作早就打探得明白,不少人都认为种谔绝不会帮助徐禧,对于援救西夏,肯定是能推则推,只要派兵阻截鄜延军,种谔当会顺水推舟。而种谔之后的表现,也证明了这个观点。

可现在种谔在盐州陷落之后疾奔而来,却必须要给出一个答案了。

没人愿意去阻挡种谔的锋锐,尤其还是在经历了盐州之战以后。不经过充分的休整,就立刻上阵应对强敌,胜利的希望虚无缥缈,巨大的损失也绝对避免不了。

“先守城如何?然后断他的粮道。”叶孛麻提议道:“种谔从宥州出来,带出来的存粮肯定不多。”

“这座城能不能守得住?”梁太后进城时,也是亲眼见识了盐州城墙的惨状。要有谁说肯定能守住盐州城,梁氏她第一个不信。

“两三天当是没问题。”仁多零丁说道,“除了一个缺口之外,其他地段的城墙尚能撑上几天。只要及时补上缺口,再放上重兵把守,完全可以多撑上两三天。种谔远道而来,粮草又不济。等到铁鹞子恢复气力,到时候击败他也不在话下。”

仁多瀚跟着道:“附近数十里内,能派得上用场的木料都在之前被用上了。没有攻城的器械,就是宋人也别想轻易的攻下一座城池。”

“而且还有大营在。十万大军不可能全数进入盐州城驻扎,肯定要有一部分放在外面的大营中。”梁乙逋想要证明自己一般的补充道,“盐州城和西面的大营成犄角之势,可以互相支援,即便是种谔也不能随心所欲的攻城或是攻打大营。”

梁太后点着头,反正是不可能逼他们这几支老狐狸带着自家的儿郎去堵种谔的刀口,能有信心守城已经是不错了。而且自家的侄儿说得不错,十万大军想要坚守,种谔的兵力是远远不足以击破盐州城的守卫。

“李清。”梁太后点起了始终默不作声的汉军主将,“你看盐州城可守与否?”

站在队尾,几乎要化作石像的李清向着梁氏欠了欠身。他之前都在沉默的听着梁太后和重臣们对,他没有在朝廷议事上插话的权力,但当说到守城的时候,却绕不过他。汉人善守,这个观念,在当世的每一个人脑中根深蒂固。

“回太后的话,方才微臣已经看过了城中的武库,弩箭多不胜数,神臂弓也有许多。拿着神臂弓上城防守,纵使种谔亦难有施展之地。泼喜军的旋风砲最好也搬上城墙,居高临下,不比神臂弓差多少。”

梁氏对李清的回答还算满意,“如果让你为主将,需要多少兵力来守城?”

李清的心猛地跳了起来,他强自镇定,“至少五万,得轮换着来守。”

梁太后没有立刻作出决定,而是沉吟着,一名内侍出现在大堂门外,“太后,黑山威福军司急报。”

“那里会有什么事?”来自西夏最北面的一个统军司的紧急军情,突然间让梁氏有了不太好的预感,“呈上来!”

将奏折接过来展开一看,梁氏便是头脑一晕,整个人摇摇欲坠。

“太后!”梁乙埋、仁多零丁和叶孛麻一齐惊叫。

“老身没事。”梁氏强自坐定下来,手上紧紧攥着急报:“盐州城不需要守了。去派人跟种谔说,盐州城,可以让给他!”

“什么?!”

……………………

鄜延军离开了无定河河谷,向着盐州城快速的挺进。

在宋军步卒紧密的队形之前,党项骑兵只能是骚扰。可在宋军的骑兵全力牵制下,许多时候,他们在骚扰之后,都没能来得及及时脱离战场,便被步兵追上,然后被消灭。步骑之间的出色配合,使得铁鹞子失去了用武之地。

种建中、种师中都在这个过程中立下了不少的功劳,但种建中兄弟都没为此而沾沾自喜。

牵着马,与大军在荒凉的土地上疾行,种师中神色黯然:“竟然还是迟了一步。”

“之前耽搁的时间太多了。”种建中叹了口气,又振奋起来,“盐州必须收复!否则在河东面前,就没有我们的鄜延路的立足之地了。”

种师中很不服气:“河东能胜,那是欺负阻卜人是实心眼,见识少,换成契丹或是党项,看看他们会不会上当!”往步兵的军阵上冲,种师中还真没听说过这样愚蠢的骑兵。

“因地制宜,相人施计。本来就是在欺负阻卜人没见识过官军的实力,换作是党项铁鹞子,想来韩玉昆也不会用那样的计策。”

“可惜了那么好的战马。”对于阻卜的愚蠢,种师中都为他们的战马而感到可惜,“都使唤了这么长时间,还能用来奔袭。比起河西马,耐力要胜出不少。”

“说什么废话?!”在前面的种谔听到了侄子们的窃窃私语,回头怒喝。

种家的两兄弟顿时噤若寒蝉。

种谔手上的是鄜延路所能带出来的全部兵力,除了留守的两万人之外。整整三万大军,八千骑兵,两万两千步卒,其中有一半,是来自于青涧城、绥德城和罗兀城这三个种谔威信最高的城寨。当种谔发出号令,如臂使指,也不难做到。

想要阻止进入盐州城不容易,就必须挡在他们的道路上,也就是与宋人正面作战。不论是城池的攻防战,还是在野外的围追堵截,都要有着对抗到底的觉悟。光是靠骚扰,绝不可能拖延下种諤这等名将的脚步。

十天的攻城战,西夏军的体力和精神已经消耗了许多,之前一直保持着自己的节奏,直到盐州城破时,才猝然发力,由此爆发出来的冲击力,并不是简简单单就能抵挡得住。

种谔对自己的指挥和麾下的将士有着充分的信心,卡准了时机,复夺盐州,并不是多难的一件事。

“怎么了?”前方突然发生了一点骚乱,让种谔随即变色,“出了何事?”

一名小校很快就回来了:“盐州那边派人过来了。”

种谔哈哈笑道:“派人来作什么,投降吗?”

“西贼的太后说了,愿意让出盐州城。”小校转述着信使的条件。

周围的嘈杂声都静了,人人都在怀疑自己的耳朵。种谔愣了半日,突然冷笑一声:“……别管他。继续前进。”

“五叔!”种建中在后面叫道。

“什么事?!”种谔不耐烦的回头。

种建中小声的说道,“没必要拒之门外吧,可以听一听他的具体条件。”

种谔理都不理:“如此大事,岂是我区区一个武夫能决定的?送他去东京城,让天子和朝廷来决定。”

“大事……啊!”种建中突然间叫了一声,“五叔你这是……”

种谔头也不回,“这时候不彻底占了银夏,还等什么?没有斩首,可就没有功劳!”

“这到底是怎么回事?”种师中一头雾水,他的五叔和兄长仿佛是在打哑谜,他还没有想个通透。

种建中摇摇头,以他的才智,捅开窗户纸并不需要费太多时间,“没有别的可能。肯定是兴庆府那边出事了。”

种师中随即也一下明白过来,一拍脑门,惊问道:“叛乱还是辽人?!”

“没有辽人支持,决不会有叛乱。”种建中说道,“而从辽人那边看过来,直接占据兴灵,比起煽动叛乱收获更多!”

种师中勃然变色:“好个耶律乙辛!我们辛辛苦苦的一场下来,全给他捡了便宜去!”

种谔一怒回头,“少说废话,今天入夜前,要进抵盐州城下!”

