\section{第15章 焰上云霄思逐寇(11)}

追逐着撤退中的敌军,一队士兵踩着被踏成烂泥的道路,深一脚浅一脚的在泥浆中跋涉。在他们的前面,是延伸进山中的烂泥路,而在他们身后,还有更多的士兵紧随着他们的足迹。

尽管两天来的雨都是绵绵细细、时断时续,可这条通往昆仑关的山道,依然被雨水所浸润。只要脚步踏过去,就是一个陷下去的印坑。而今天不知多少双脚从这条路上走过,道路的破损也越发的严重。

走在全军最前面的丁安从泥浆中把连着草鞋将脚拔出来,黏糊糊的烂泥带着吸力,套在脚上的草鞋好几次都差点被黏住,走上一步都要平常多花上三五倍的气力。喘着粗气:“到底什么时候才是个头?”

身边的李仓则已经光着脚,赤足在泥地中走着:“都头不是说了吗?宋国派来的援军就来八百人,只要能攻下昆仑关,就可以一直攻到桂州去。”不过安慰同伴的话语,连自我安慰都做不到,李仓脸上只有疲惫,没有半点对打到桂州的期待。

丁安低下头去,奋力的在泥水中向前面挪过去,低声嘟囔着:‘回头还来得及。’。

作为全军的先导,率先追击敌军的前锋,获得战功的几率很大,而冒的风险则更大。只看他们这个都前出整个指挥足足有两里之遥,早就知道他们所起的作用,就是一个提防伏兵的警哨。

年纪稍长的李仓,比身边的同伴更要忠于职守,或者说更清楚作为全军先导的这个位置到底有多危险。没有将精力放在更多的抱怨上,用着小姑挑剔新嫁的嫂子的目光,看着左右的山丘。

昆仑关所在的这一片山,都不算很髙,而且岔道众多,值得疑心的地方实在太多。不过撤退中的敌军,要想不着痕迹的在周围山谷中藏身起来、守候伏击的机会,也不是那么容易。

随着道路,转过一道弯,李仓望着侧面的山坡。一抹红色跳入他的双眼,李仓难以掩饰心中的震惊,一下停住了脚步。丁安被同伴突然的停步惊到,也停了下来。

“怎么回事?!”后面的队正大声吼着,问着前面领头的丁安、李仓为何停下脚步。

李仓抬手指了指山林中,那一抹完全没有遮掩的出现在数百交趾士兵视线中的红色:“是宋军的哨探!”

发现敌踪的消息顿时引起一阵骚动,有人提议上去驱逐,可那名哨探藏身在山林中,所站立的位置绝佳,就算派人追上去,也很容易就能逃掉。

领着这支百人队的都头狠狠地盯了那名不遮不掩的哨探一阵,又环视周边,没有发现更多的可疑之处,用力的哼了一声:“不要管他,他这是故意要耽搁我们追敌!”

自从进山后,小小的骚扰就没有断过,时不时就是一支冷箭射了过来。虽说在细雨中弓弩的威力大减,但总有运气不好的士兵,挨上一箭两箭,将他们前进的速度耽搁上片刻,不过主动出现在他们面前的宋军却还是第一个。

不过这样做,反而体现出了宋人的心虚。若是他们什么都不做,那样平静的追击过程,倒会让人毛骨悚然,一路胆战心惊。

丢下山坡上的宋军哨探不再理会,在都头的催促下重新起步,“快一点,到了长山寨,就能歇下来了!”

事先收到的军令,如果没有追到宋军,到了长山驿就止步,就地安营扎寨,等候大军上来。

连都头都不指望追上宋军,只盼着平平安安的抵达长山驿。李仓摇摇头,这仗何苦再打。

……………………

从归仁铺到昆仑关,一路有金城驿、大央岭驿、长山驿三个驿站。

韩冈等人带着殿后的六百步卒这时候刚刚望见了长山驿的山头,而身后尾随而来的敌军,一路疾行,很快就通过金城驿,待到午后时分,其前锋已经追至大央岭驿。

“后面追得越来越紧了。”黄金满忧形于色,“该不会一路追到昆仑关下吧?”

离着就在身后十里的地方,不过韩冈倒不为此担心。“李常杰不会那么蠢,一路跑到昆仑关下,我们难道还会给他喘气的机会?”

李信回望着身后的山道,“再不走快点。不到昆仑关,就会被追上了。”

“因为都累了嘛。”韩冈风清云淡的笑着,似乎对此毫不放在心上。

从归仁铺到入山的金城驿,这几十里地,都是一片坦途,中间连个设伏的地方都没有。而入山后,有了设伏的位置,哪还有气力对上跟在身后的交趾军——就像长跑,领跑者永远都比身后的追逐者更容易疲累;撤退时,当然跑在前面的更累一些。

“就在长山驿会会他们好了,洞主你的兵应该已经快到了。”

并不需要他们这些走了半夜再带上一个白天,中间只休息了小半个时辰的人,气喘吁吁的藏进山中埋伏起来。撤离时打头的两个百人都,这时候已经在苏子元的率领下,提前抵达昆仑关中,将黄金满守在昆仑关中的部众顶替了出来。所有的布置都是放在长山驿附近,只要交趾军当真追到长山驿,以逸待劳之下,连伏击都不需要。

黄金满对于韩冈的布置当然皆已知悉,只是他依然满是不放心的神色,“会不会有什么意外?若是将交趾人引到昆仑关下,应该更好一点。”

黄金满的犹豫,落在韩冈的眼中。而他的私心,韩冈看得更清楚。

已经不是前日要递投名状的时候了,那时一是占着交趾军还不知他叛离的便宜,另一个,他也需要想韩冈证明自己的价值,所以敢于拼命。但现在荆南军不在后面为他撑腰,交趾军又是拿他当仇人看,打起了损失的可都是自家的部众。

“李常杰如果真能犯这样的糊涂就好了,只是不能指望。”

“但他已经够糊涂了,这一次就不该追来。”

“翻看史书战例,聪明人少见,糊涂的倒是多了去了。所谓名将,也是要分成色的。不过他就算再差,还是有一定的才智,要赢他不容易。”韩冈道:“不过我们也累了,打垮他的前锋,让他知难而退便足矣。”

作为核心的荆南军兵疲师老,而广源军则难以让人的放心,韩冈并没有全歼李常杰这一支的打算……以及能力,只想给交趾人一个深刻的教训。

不过李常杰当真追到昆仑关下,或是打着宾州的主意,这份送上门的大礼,他韩冈也就却之不恭了。

……………………

已经是傍晚。

没有彩霞、没有夕阳,只有越发变得晦暗起来的天空,只有随风飘下的细雨,另外还有从前线撤退下来的队伍。

黄全就在长山驿等待着。黄金满的这位长子带着昆仑关城中的守军,与苏子元交换了差事之后,就立刻领军南下。他对韩冈的吩咐,不敢有任何耽搁。用了最短的时间,赶到了这里。看着自家的族人一队队的饭回昆仑关,等到最后,终于等到了断后的韩冈和他的父亲。

看着韩冈和黄金满停在了旧营地的门口下了马,黄全连忙过来行礼。

“准备的怎么样了?”下马后,韩冈就劈头问着,一切筹划妥当,他现在只担心下面的人不能按照计划行事。

黄全将手指向驿馆周围的营地里,安安静静等着战事开始的士兵:“回运使的话,都已经准备好了。”

韩冈环视了四周,很满意的点了点头。

黄全补充道:“就是战马太少,不能即时查探交趾贼军的动向。”

“战马我这里也没有,韩廉他们不及时回来,就算我要派出斥候出去,也只能凭着双脚为主。”

昨夜韩冈在归仁铺营寨中只留了五六十人而已,当时的营地中的马匹,连同运货的驮马一起算进来也就这个数目。虽然驮马不怎么适合骑乘,也没有鞍鞯,但只要有缰绳,有马镫,经验丰富的骑手只需在马背上再铺上一层软垫,照样能骑上去。

靠着马匹为助力,迅速的穿过了入山前的那一段平原。只是进山之后,断后的骑手们不可能再沿着官道直奔,驮着一百多斤跑了几十里,骑乘的马匹肯定都是疲惫不堪,只能改为步行。四条腿变成了六条腿,韩冈没哪里还敢让他们走在交趾军的正前方,按计划是直接转进小道,从小路回昆仑关。

不知是谁吹响了号角,尖锐的声音在山中回荡,一直传到了韩冈等人的耳朵里。

“想不到我们才到不久,交趾军已经到了,这速度可不慢。”韩冈丝毫没有危机感的说着。

的确到的够快,基本上就是前后脚的差别。为了赶路,韩冈这边一路都没怎么休息,只不过昨日为了准备北归,特意让所有人在营中多睡一会儿,而交趾人哪里有这个条件,他们的累可是实打实的。

在韩冈的指挥下,黄全他麾下的所有战力都守在了驿站外围的营地中,等候着他们敌人的到来。

自号角声响起来后,音调一声急促过一声,等到最后一声响起,交趾军的先头部队已经出现在他们的眼前。

‘终于到了。’韩冈心里想着,接下来,就是广源军大展神威的时候了。

