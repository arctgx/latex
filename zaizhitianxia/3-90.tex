\section{第28章 临乱心难齐(五)}

郑铎是从小妾的床上被叫下来,衣服都没换好,跑得浑身是汗,也没听清韩冈说了什么。但他知道该怎么回答。郑铎在韩冈面前连坐下的资格都没有,束着手站着,陪着笑脸,“正言说的是,正言说得正是。回去后,下官就好生的教训这群不长眼的”

文官找借口立威的故事太多了,新官上任三把火的时期,聪明的都知道要老老实实做人。现在被下面的人害得自己撞到了韩冈——这个在军中传说的能杀人能救人的狠角色——面前,生撕了他们的心都有了。

韩冈却好说话,“今天将帐给结了就行了,打坏的东西也要照数描赔,赔礼道歉想来不用本官提醒。将这些做完,今天这事就算过去了。但日后……就要劳烦郑都监你多加整治!”

郑铎闻之一愣,就这么放过了。但立刻醒悟过来,一下训着几个犯事的小卒,“还不快谢正言的宽仁大量!”

在一片谢声中,看着如释重负的郑铎,韩冈摇了摇头。

不过吃白食而已,这个罪名能将人怎么样?就算要立威,也不至于用这等小事。方才他说了一通话,也没见有个人趁势上来喊冤,看起来,这些禁军士兵平日里也就是如此罢了,未有大恶。由此来治罪,未免有大炮打蚊子的嫌疑。

他在军中本身的威望就足够高了,就算是京营禁军而不是西军,愿意得罪传说中的药王弟子的将校,打着灯笼也难找。一手完善了军中医护制度的韩冈,在军中总能得到足够的尊敬,没必要特意挑刺找毛病。而且过一阵子,说不定还有用得到他们这些军汉的地方。

另外自己做人行事在表面上也该缓和一缓和,太过锋锐对他日后的进步不利。老成持重,同时能宽严相济,才是重臣的模样。

处理过禁军的白食案,军士们连忙离开。而郑铎留了下来,与王阳名一起陪着韩冈,去了前面镇上最是干净清爽的酒楼进用茶饭。

只是刚到酒楼楼下,又听见一阵骂声,却不是吃白食了,而是在骂着王安石,“就是见臣当道,上天才有如此警示。废新法,逐奸相,这旱情肯定就能化解!”

王阳名脸色尴尬,‘奸相’的女婿就在这里呢。连忙道:“下官这就派人将他们拿下来治罪。”

韩冈摇了摇头,岂能以言罪人。而且以眼下的灾情,这些传言是免不了的。

天人感应之说早就深入人心,智者虽不取,乡愚却人人皆信之。遇到大灾,百姓们总得有个抱怨的对象,王安石自然是首当其冲。

天子和宰相要为当今的灾情负责,皇帝不能卸任,走人的当然是宰相。这样的言论根本弹压不住,也解释不清。就算是教育普及的千年之后,也还有将自然灾害归咎于天谴的‘人才’,眼前的民间舆论,韩冈听了也只能苦笑而已。

不过只要今年冬天能下雪,这个坎,根基深厚的王安石还是能够渡过。但要是不下、或是下得少的话,百姓们的怨言将无可阻挡,而河北的流民恐怕也会吃光常平仓的救济粮后蜂拥南下。

那时候,就是他这个白马知县首当其冲,要设法将流民尽量拦在东京城外。

……好吧,韩冈其实从没有想过,自己目前最重要的工作竟然是维稳。以他的个性来说,朝堂上还是乱一点才更有趣,也更有自己施展的余地。

但眼下的情况不太一样。

京城安稳,朝堂的政局才能安稳。稳定的朝堂,这样才能保证救灾工作的顺利。

谁能保证换上来的新人,首要工作是救治百姓,而不是清算之前的政敌?怨有所归,有了足够的借口,该做的正事完全可以拖延一阵子,将敌人斩草除根才是最先要做的。

韩冈从来都不会高看官僚们的道德水准,包括他自己。

话说回来,只要对政治稍有了解的,都不会有着太过天真的想法。临阵换将乃是大忌,这个道理人人都知道。除非天子身上承担的压力实在太大,否则自家岳父的相位当是能拖到大灾之后,处理完一切手尾,然后让王安石他自己主动辞官,以保护他的颜面。

只是……韩冈回头看着楼外的青天白日,这一点还要看老天爷帮不帮忙了。

………………

在厅门处目送都水丞侯叔献离开,王安石回到座位上,双手按着额头,脑中隐隐的作痛。

前日他与儿子所商议的,要在汴河破冰,以便在冬日运输粮食进京。侯叔献这位朝中首屈一指的水利专家,给出了他的意见。与黄河接口处的河口可以开,一旦汴河中有了流水,冰层就会变薄。再用小脚船数十艘,船头安装巨碓,用来敲砸冰层,开出一条水道来。但也要做好纲船损毁的准备,流冰伤船是肯定的。

王安石一时难以决断,用巨碓在河上碎冰,这个发明过去从来没有用过,究竟有没有成效确难以知晓。要是出了差错,被人耻笑倒也罢了,误了大事才是让他头疼的关键。

“就算是春夏纲运,纲船也没有少毁损过,损失大一点,也能承受得起。”王雱则是全力支持侯叔献的方案,他送了侯叔献出门后回来,就对王安石道:“只要有粮食在冬时进京,就能让囤积居奇的奸商们血本无归。不要太多,十几二十万石就绰绰有余。三月到十月,单单是纲运就能运送六百万石,加上民间的运输,更是不止这个数目。难道眼下区区二十万石还做不到?”

如今京中粮价飞涨,其实有许多是因为恐慌情绪在,但是京城内外几个大粮仓中的存粮,就超过百万石,而诸多粮商手中的粮食、富户囤积的数目,加起来足够东京城半年食用。只要能安定下民心,粮价能应声而落。

关键就是在民心上。

想当初,陕西传言废铁钱。市面上铁钱顿时无人肯收,而铜钱币值飞涨。时任陕西安抚使的文彦博,从家中拿出百匹绸缎让人出去贩卖,声明只收铁钱,不要铜钱。见到文彦博支持铁钱的举动,民心立刻就安定下来,铁钱在陕西也重新恢复了流通。

王安石和王雱明白,只要汴河畅通,能运来江南的粮食,京城粮价随即便可安定下来。

而且并不需要从江南运粮。明年开春后就要北运的粮食,现在主要囤积在泗州。大约五十万石上下。更近一点的宿州,控制在六路发运司手上的也有二十万石的存粮。而且泗州、宿州之间冬季虽然会结冰,但冰层往往不厚,加上又有淮河来水的补充,水位稳定,不至于伤到纲船。

只是宿州再往上,情况就不一样了。尤其是过了南京应天府【今商丘】后,接下来的三百里,通往黄河的河口关闭,渠中水量不足,同时因为水流静止,比自然河流要容易结冰得多。不但要开河口来放水进汴河,同时还要凿去河中厚达尺许的冰层,这样才能保证通航。

这就是王安石所要面对的问题。宿州到东京总计六百里,其中后半段的三百里的河冰要靠侯叔献的发明来处理,不知道到底能不能行。

但该做的还是要做。王安石知道,只要几场大雪下来,旱情缓解,什么事都不会再有。可做事不能靠老天,如果旱情继续下去,就必须保住京城的稳定,汴河水道必须打通!

王安石是坐言起行之人,如此急务,当夜便写了奏章,第二天就递到了赵顼的案头上。在崇政殿中,经过了一番争辩,王安石得到了赵顼的首肯,冬日开启汴口,同时破冰通航。

议事结束后,因为争论耽搁了时间,王安石没有留下奏对,随着其他辅臣们一齐而出。往着政府过去,同时出来的王韶走近前来,说道:“相公提议那是极好的,但为何不用雪橇车,反而要费力破冰呢?”

“雪橇车?”王安石脚步一停,复述这个陌生的名字,记忆中什么印象都没有。

“相公怎会不知?”王韶似是奇怪的问着,“前年与蕃人交锋,在下与高公绰冬日屯兵于新近攻下的狄道城。狄道与渭源虽然只有一山之隔,可由于大雪封山,消息和补给都断绝了。不过当时洮河冰结,通过雪橇车将粮饷酒水从河道上运了过来,士气由此而振!”

王安石一听,连忙追问:“不知雪橇车是何形制,是否是熙河特产?!”

“所以问相公为何不知。这本就是相公家女婿的发明,为何问我这外人?”王韶慢条斯理的回答,然后就不出意外的见着王安石神思不属的拱手道谢,急匆匆地离开。

看着远去的高大背影,王韶摇了摇头,要不是看在韩冈面上,还有过去的一点情分,他可懒得多说这些。他所在的枢密院,可是被政事堂压得死死的,憋屈得很呐!

王安石是个典型的急性子,回到政事堂就让人找来王雱,问道:“近日没有给玉昆写信?”

“出了何事?”见到父亲的样子,王雱就知道发生了什么。

王安石匆匆的将从王韶那里听到的消息一说,王雱就失声跌脚。“竟有此事!”

后悔不迭,既然有此前成功的例子在,又何必去冒险去开河捣冰凌,“我怎么就没想到问一下玉昆!孩儿这就写信让人送去白马县!”

“如此大事,翰墨往来肯定说不明白,要让玉昆进京一趟,或是你去一趟白马县。”王安石连忙阻止儿子。开河之事已经奏闻天子,两三天内就要动手开始做了。这点时间只够书信走一个来回,哪能将事情给说清楚。

“可是……”王雱现在日日上殿面君,请假不太方便。以他的身份突然跑出京去,也会惹得人们的猜疑。

这时候,一名小吏在外面通报,“相公,府上有报,说相公家的二小娘子回来了!”

王雱眼睛一亮,一拍桌子:“二姐回来了!”

