\section{第九章 长戈如林起纷纷(二)}

【第二更,求红票,收藏】

韩冈并不知道他的名字已经写在了崇政殿的屏风上,即便知道,也会自嘲的想着自己终于能与宋江、方腊平起平坐了。

他现在倒是挺想自己有着宋江方腊一般的本钱,当然不是用来造反,而是手上若能有个上万人驻守在古渭寨中,也不会任由董裕嚣张。

王韶和高遵裕正在商量着如何联络俞龙珂:是先让人带封信,然后再去找他说说话;还是直接去。如果是要先写信,还要考虑着如何措辞才能不失朝廷体面,又能打动俞龙珂,这是件费思量的活计,不过他们最迟也要在明天天亮前商议出个眉目来。

至于韩冈,也有他的事情要做,他可不是王韶、高遵裕面前的小跑腿,没事出出主意的清客。他是官,当然有差遣要做事。勾当公事是一件,而管勾路中伤病事也是他的工作。

前次向宝领军去解决托硕部,韩冈就奉命带着他手下管理甘谷疗养院的朱中等人随军而行。而后向宝被王韶抢在头里去,气得中风,进军不了了之,朱中等一众人等便被韩冈派去了古渭,在古渭寨打造新的疗养院。

而经过了近两个月的打理,古渭寨的医院已经有了初步规模,古渭疗养院的门额就挂在寨中南部的一处阳光好的军营大门上。寨主刘昌祚很会带兵,善抚士卒是不用说的,自然对疗养院十分上心。而以朱中为首,被韩冈带出来的一批人,对韩冈是顶礼膜拜,把他的话当作圣旨一般依从。韩冈所编订的管理暂行条例,更是一丝不苟的去执行。

疗养院的内部布置一切学着甘谷城的样式,打理得干干净净,布置得井井有条,一看就是个宜住人的好地方。虽然在里面治病救人的都是如朱中这般只有不到一年的医术生涯的赤脚医生,但有治疗总比没治疗要好;有专人照料,再加上洁净的饮食、干净的住所,更是比旧时在肮脏的床铺上等死要强出百倍,一个多月下来,有不少生病的士兵康复出院,因而韩冈在古渭寨中便是备受尊敬。

“韩官人!”“拜见韩官人!”“小人拜见韩官人!”见到朱中陪同的韩冈,疗养院中的士兵们纷纷退到路边俯身行礼——韩冈上次来过古渭,认识他的人并不少。

韩冈一一点头还礼,对着朱中笑道:“看起来你很做得很用心啊,要不然我也沾不了光。”

朱中比起半年多前跟着韩冈押运军需的落魄样子,已经截然不同,仿佛两个人一般。有些富态,满面红光,须发都梳得整整齐齐,像是个有身份的乡绅。身上穿的衣服的料子虽不华贵,也是能算上不错的货色——虽然士兵们都不富裕,疗养院都不会向他们收钱,但他们病好之后,总是会送些礼来作为感谢——而且洗得干干净净,却是连浑家都有了,而不再是四十岁的光棍。

朱中知道眼前的这一切都是韩冈给他的,士卒们的尊敬,丰厚的俸禄,还有一个完整的家庭,都是跟了韩冈之后才得到的。他恭恭敬敬的对韩冈道:“都是官人的功劳,小人只是费些辛苦罢了。”

“你们的确辛苦了……”看着被打理十分干净整洁的古渭疗养院,韩冈感慨油然而生。

他算是个甩手掌柜,带出了甘谷疗养院、编写出了管理制度之后,便没在医院上面花多少心思。他在甘谷打造疗养院,本也是因为功利之心。当了官后,也只是稍加关注,心思和精力还是放在经略司衙门里面。但朱中不同,他和他的几十个同僚都是把疗养院当作改变命运的唯一事业,投入的心血和功夫不是韩冈能比。

陪着韩冈在疗养院中视察了一圈,安慰了一些重病的士卒。在疗养院特有的长条交椅上坐下,朱中小心翼翼地问着韩冈:“官人,今次木征带了五万大军来攻打古渭,这寨子能不能守得住?”

朱中身份低微,不知其中内情。他只听说过传言,并不知道木征仅是个幌子,那五万大军更是空谈。

韩冈当然要辟谣,不然单是传言就能让古渭寨里的守军不战自溃,他大笑道:“传言多是无稽,不能妄信。来的不是木征,兵力也决没有五万,而他们更不敢攻打古渭寨。皇宋天威,也不是小小的蕃部能招惹的。只不过是蕃部间的自斗罢了。”

韩冈的声音很大,他的话本就是说给疗养院中的士兵们听的。王韶和高遵裕忘了下令辟谣,只是寨中人心惶惶,韩冈既然碰上,也不能看看就算了。

而因为疗养院的事,韩冈在秦凤路军中的名声很好,他说的话自然不缺人信。周围的士卒、护工们听到他的话,神色便为之一松。

“那就不会打仗了?”朱中惊喜的问着。

韩冈不能就此下断言,也不想诓骗周围的士卒和护工——他一向很看重个人信用:“今次被贼人攻打的蕃部,是听命于朝廷的熟蕃。在情在理不能任凭他们受欺。谨守门户,是你们的事。至于解救蕃部,平息纷争,自有人去做,尔等不必操这份心。即便真的有贼人敢犯古渭,到时你们听命行事就行了,古渭寨高墙厚,也不是只会骑马射箭的蕃人能攻下,等个几天,都巡检自会率大军来援。”

韩冈把和战两面都说到,没有欺瞒半点。‘民可使,由之,不可使,知之’,虽然韩冈对这一段的句读与此时流行的说法并不相同,许多士大夫都觉得乡愚不足以论事,但韩冈一直都认为,什么事都向下隐瞒,用些谎言来欺诈部下,绝不会有好结果,只会降低个人在人们心中的信用,狼来了的故事韩冈并不想模仿。子曰:自古皆有死,民无信不立。足兵、足食,都很重要,但民众们的信任却是最重要的。

听到韩冈的解释,周围的人们虽然心中隐忧没有被化解,但他们至少能安下心去等着结果。韩冈知道,古渭寨不算大,而且这时候人人都在打听着消息,他的这番话很快就能传遍寨中。自家既然还有些信用,这番话自然不缺人信。寨里人心安定下来,那今次古渭寨也就不会再有什么乱子。

散去了众人,走遍了疗养院中,朱中陪着韩冈向外走。韩冈边走边说:“过几日朱兄弟你们可能要辛苦一点,被攻打的蕃部也许会退到古渭来求庇护。到时也许会有些蕃人的伤病过来求医,他们都是同听王命的熟蕃,要好生照料,日后还要用得上他们。”

朱中头点得跟小鸡啄米:“官人放心,小人不会慢待。”

出了疗养院,辞别了朱中,天色已经全黑了。夜风热燥燥的,就算迎着风,可呼吸都让人感到烦闷不堪。韩冈额头细细密密出了一层汗,胸背更是都汗湿了,但他只希望天能再热一点,吐蕃蕃人可吃不住这样的天气。

回到城衙中,韩冈去找王韶和高遵裕,他们应该商量出个眉目,但当他到了衙门的正厅中,却见一个胡须花白的老蕃人正跪在厅内的地板上哭诉着:

“王机宜、高提举,两位要为小人做主啊!董裕那厮已经绕过了渭源堡,一口气灭了苽黎五族里的两家。现在他的前锋已经离着青渭只剩百里了,指着名要小人的脑袋。小人为朝廷不惜性命,但小人家里还有几千孩儿,看在小人为朝廷卖命的份上,总得给小人的孩儿一条活路吧。”

这个已经被吓得语无伦次的蕃人,韩冈认识他,是青渭一带最为亲附大宋的纳芝临占部的族长,唤作张香儿。今年在古渭寨过年时,韩冈就见过他。也是前次攻打托硕部,第一个响应王韶号召的部族。只是别看他哭得这么伤心,其实在上次齐攻托硕的七部中,纳芝临占部是位置最安全的一家。

纳芝临占部所据有的三条谷地紧挨着古渭寨,在古渭南面不到二十里处,便是族帐所在的吹莽城。其族酋皆是张姓,本就是吐蕃化的汉人,早在真宗时就投了大宋,世代被封作蕃部巡检。本代族长张香儿甚至还在古渭寨里有一套宅邸,时常过来居住——缘边的各处城寨并不是完全由士兵充斥其间,而是类似于城池,有商人,有平民,当然还有些靠着保护的富户、大族。像张香儿这样亲宋的蕃部,在自家附近的主城中,买间宅子都是很常见的事。

所以说这厮其实安全得很,他现在在王韶和高遵裕面前哭诉,不过是为了把族人都弄进古渭寨中来。

王韶和高遵裕安慰了张香儿两句,把他打发了出去。但张香儿已经表露出来的要求,他们却要大费思量。虽然救援这几家亲宋蕃部是必然的,但万一放进寨来的蕃人中有人心怀不轨,那古渭寨可就完了。

“玉昆,你觉得该怎么做?”王韶问着韩冈的意见。

“老弱妇孺可以进寨,同时不许携带兵器。至于精壮,则只能临寨结帐。”韩冈说得很干脆,这些事过去都是有先例的,照着来就是。紧接着他又说道:

“不过董裕进兵的速度却是令人意外,想不到竟然已经灭掉了苽黎五族里的两家,兵锋距古渭又只剩百里。联络青唐部之事刻不容缓,还是今夜就走,由下官去打个前站。

韩冈从来都很珍惜自己第二条生命,若无必要,绝不冒险。可如果不能趁董裕杀到古渭寨之前赶去青唐部,等他抵达寨外,那时再想出城可就要冒着绝大的风险了。去青唐部之事,现在看来是躲不掉的,既然如此,还是早点去比较好。

其实方才王韶和高遵裕商议的结果也是直接先由韩冈带着口信过去,然后有了回音,他们再去见俞龙珂不迟。王韶当即点头道,“如此那就拜托玉昆你了……王舜臣!”

王舜臣站了出来:“小人在!”

“你带一队人随玉昆去,务必要保护好玉昆的安全。”

王舜臣躬身答诺:“机宜放心,小人必不负所托!”

