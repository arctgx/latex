\section{第16章 夜凉如水无人酌(下)}

半个月的时间,不能让身怀丧亲之痛的人们走出悲伤,但邕州城中的创伤,则是日渐一日的再修复。

从一开始,韩冈就集中城中人力,一个片区一个片区的清理着废墟。在他的领导下,百姓们的行动也很迅速。正在春天和暖的空气中败坏下去的遗骸,已经运出去了大半。有主的择地掩埋,而无主的,则是掘个大坑一起——韩冈一开始说要全数焚化,后来看到需要消耗的木材数量,就将自己的决定收了回去,眼下实在消耗不起。

而就在北山脚下,一日日烟火不绝。从山上采伐下来的木料,直接被当作了石灰窑的燃料。原本在邕州城边就有两处,那是苏缄从京城回来后,从韩冈那边得到了一些医疗方面的知识,特意让人修建起来的。

只不过以现在的情况,无奈是埋葬遗骸,还是在城中喷洒消毒,都需要用到石灰。区区两个石灰窑的产量实在不敷使用,更多的石灰窑就这么搭建了起来。邕州城附近,木料和石灰石都不少。有技术、有人手、又有原材料,石灰的产量自是一天比一天高。

现在的邕州百姓,都是知道了如何最快的手段清理废墟。先是掘出互相连通的壕沟,引来流水——在此之前,重新掘开的城壕已经同左江又联系上了——将废墟中的尸骸都收拾起来,用木船或木排运出城去。清理出一处完整的地块,就洒上一片石灰水,两三轮过后,就算是清理干净了。

“本以为还以为有些日子,没想到进度那么快。”半个月下来,苏子元的心情也平复了许多,好歹身边还有着一个女儿做伴。加上他全身心的投入了工作之中,也没时间去想些其他的事,“百姓虽然辛苦,却并无怨言,全都是运使以工代赈的功劳。”

韩冈揉着额头,“几万灾民嗷嗷待哺,要运来足够的粮草,我这些天头疼得厉害。”

“听说运使前两年在京城,安置了数十万的河北流民……”

“巧妇难为无米炊。没有足够粮食,手段再好也没用。”从宾州运粮过来,要翻山越岭,而且运得还不是几千人的几天口粮,而是要维持几万百姓连续两三个月、甚至可能更长的生活,韩冈是当真头疼,“不知道刚刚种下去的这片占城稻,两个月后能不能安安稳稳的收割下来。”

“还有一个半月。刚才下官已经出去看过了,田里的秧苗长势好得很。只要没有什么意外,到了四月中就能收获了。”

占城稻早熟、高产、且不挑地,除了口感不好,没有别的缺点。自从宋初传入国中,这些年早就成了南方水稻的主要品种,灾后补种也是都靠着这种稻子。韩冈在处理城中废墟遗骸的同时,就是下令在城外荒废下来的田地里及时补种,事关几万百姓的肚皮,这农时半点都误不得。

“不过秧苗长得好,也多亏了运使的严责。邕州种稻,哪里会育秧、移秧,既种之后,旱不求水,涝不疏决,既无粪壤,又不耔耘,一任于天。”苏子元摇头叹气,“先君在邕州的这些年,几次三番要督促邕州百姓深耕施肥,但民情习惰,始终扭不过来。连衙中的官吏都劝,只要下面交足了赋税,何苦强逼。”

说起此时广西的农事,韩冈看了只想摇头,他也没见过这么种稻子的。就在田里直接下种,没有说要插秧。不开沟渠,不是肥料,连地里的杂草都不除,一切全靠老天。要开种时,只要烧个荒,甚至都见不到多少用耕犁的——要知道,广西水牛多得逢年过节,百姓就杀牛庆祝,官府都禁止不了,江西商人年年来广西贩牛回去。

“北方风土恶,不辛苦一些,就是一年就得饿肚子。而南方水土肥沃,即便不事稼樯,望天而收照样能维持一年。水土不同,人情便是相异。”韩冈也跟着叹了一声,“要不是眼下有难,我下的命令也会跟令尊一样,没人搭理。”

为广西的民情叹了一阵,韩冈又振奋起来,要做的事还多得很:“邕州的田籍簿册全都烧毁了,人也死了三四万,许多田地都成了无主之地。眼下为了救急,我是统种统收,也不管田地属于何人,但这一次收获之后,就得重新整备起来。这件事,还得伯绪你来做……”

因为州衙被烧毁,存于衙内架阁库中的田契、房契、税簿、产簿等文簿档案皆毁于一旦。这些籍簿,是国家统治的基础。要想重新建立起政务体系,只依靠存放在转运使司中的复本是不够的,必须加以重修。

苏子元一拱手:“分内之事,不敢推辞。”虽然他是桂州军事判官,但邕州籍簿被烧,是苏缄放得火,不论清又如何,苏子元认为自己有义务重新为此恢复。

起身送了苏子元出去,韩冈看着他的背影心想,邕州城下一任知州应该有着落了。

政务上的事情处理了一遍,而医务上同样还有许多事要韩冈来处置。

韩冈之前下达的一干条令,运尸出城,清理城池,保证水源,加上使用石灰消毒,这么多的措施下来,士兵和百姓们依然免不了生病。邕州的气候实在是太过特殊,让从北方来的士兵们难以长时间忍受这样的天气,而邕州城中居民们,在一场劫难之后,担惊受怕忍饥挨饿之余,体质都有大幅度的下降。

病患的人数越来越多。多是腹泻,也有往更严重的痢疾方向上去发展的个例。韩冈只能保证提供他们以足够的淡盐水,药材则极为欠缺。

韩冈对此很是头疼,他希望章惇能早一步将药材给运来。否则没有了药物,韩冈只能让卧床不起的伤病们,靠着自己的体力和抵抗力去强撑着熬过去。

“不知还会有多久?”

“什么?”韩冈低头看着手上的公文,抬眼问道,不知何时,李信走进了房来。

“药材!”李信有些急了,他说了半天话,韩冈好像没有听到一般。

李信麾下的几个指挥经历了多场战斗,每一次战斗,伤亡的人数都算得上极少,但几次累积下来,一算比例,数目就很吓了人了。尤其在疗养院中的,没有药物只能灌着盐水洗肠胃,这算是什么救治,“章学士就快到了,不是说他正带着药材和雷简一队医官过来?”

韩冈低头下去看着公文上一个个细小如蝇头的小字,排得密密麻麻的数字,是如今军中每日的收支情况,——多半只有支出,没有收获——邕州的库房中没有粮食、没有财帛,苏缄散财散得足够干净,没让交趾人捞到一分半点。不过到了韩冈入城后,对着坐吃山空的情况,只能猛揉着额头。

过了好半天,他才反应过来,抬头问道:“药材?昨日的消息是已经到了象州,如果走得快的话,也就在这两三天。”

………………

章惇从收到邕州大捷的消息后,就立刻动身南下。一路上,都在为韩冈的成功而惊叹不已。

如果换作是自己领军,绝不会走得有韩冈那么快。说不定就救不到宾州的百姓。没有全歼广源军兵将千人的战绩,要想说服黄金满也不会那么容易。不说服黄金满投效,不但昆仑关得不到,甚至只能坐视李常杰屠了邕州。

章惇敢于冒险,也敢拼命,可他不认为自己能做到韩冈这一次立下的功劳。尽管他在桂州为韩冈提心吊胆,如果换他站在韩冈的位置上,许多事都会做下同样的决断,甚至可能会比韩冈更大胆。但他与苏缄没有交情,小小的桂州军判也影响不了他的指挥,一开始就不会兼程南下,而要体恤着帐下儿郎。

一路抵达邕州城,韩冈率领城中众将官出迎。

章惇就在马上拱手道贺,“以八百当数万,以千五破十万,玉昆用兵,可谓是鬼神莫测。”

“非韩冈一人之力,有学士为后盾,下有众将死力,再得义士相投,如此方得一胜。”韩冈拉着章惇往城中走:“此次大战,交趾损兵折将,其军力大减。不知学士有何主张?”

“玉昆你怎么看?”章惇反问着。

“如今交趾国中,幼主当朝,妇人秉政,若无李常杰支持,如何能逼杀国母?如今李常杰大败,其麾下亲信多有折损。过往畏其权势兵威者,想必都蠢蠢欲动。如果王师不至,他们还有整顿国中的时间,不过若中国大军压境,其国中必然生乱。”

“玉昆所言,正合愚兄所想。听闻交趾国中用政酷虐,残民害物。普天之下,莫非王土,率土之滨,莫非王臣。李氏窃据交州行有余年,百姓苦之。孟轲有云‘南面而征北夷怨,东面征而西夷怨,奚为后我?’岂能让南交百姓再受李氏之苦。王师当吊民伐罪,救其于水火。”章惇身子略略前倾,盯住了韩冈,“不知玉昆你意下如何?”

韩冈冷笑了起来,慢慢念诵着诗经中的句子:“投我以木瓜,报之以琼琚。投我以木桃,报之以琼瑶。投我以木李,报之以琼玖。匪报也,永以为好也……”

