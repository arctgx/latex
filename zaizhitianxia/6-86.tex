\section{第十章 千秋邈矣变新腔(八)}

“所以他们都没有来考制科。陛下,臣意一如前言,如果黄裳不考制科,那么让他去边陲立功便可。”

韩冈不介意,“不是进士,却能给制科出题,这件事,平章是如何看的?”

“馆职非是贴职。能得馆职,皆有考试,能够通过方可留于馆中。赵彦若虽非进士,却能得授馆职,足可见其才学。”

“空有才学,却连科目不同,考题也不该相同的道理也不懂。”

“从无先例,何谈应该?”

韩冈与王安石你一句我一句,不肯有半点让步。两方辩论,目的并不是驳倒对方——这样的难度很高——而是要争取到旁观者的支持。现在所谓的旁观者,自然就是屏风之后的太后。

“错试考题,有失朝廷引用贤人之意,蹇周辅事后必须加以惩处!”韩冈语气强硬。

王安石摇头:“臣不曾闻阁试上曾有因出题不当而惩治考官的旧例。”

韩冈冷笑了起来。

王安石说得的确没错,的确没有这样的旧例。但制科开国以来才几回?阁试当然也没几回。其中出不了一个因出题不当而被处置的考官,又有什么不对的地方?在其他考试上,因为出题不当而受惩处的官员,决不是没有,韩冈随手就能举出来。

“韩冈记得熙宁初年,国子监学官颜复曾以王莽后周为题,论王莽后周改法事,以隐射变法。其中生员苏嘉因极论二者之非,被列为优等。而后监中学官便尽数被逐。当事者颜复于今不知何处,不过被评为优等的考生苏嘉正是苏颂之子,陛下可招苏颂前来询问。”

韩冈故意抛开了王安石话中的定语,将十余年前国子监中的一幕给揭了开来。当年太学还为旧党所盘踞,王安石正好利用了那桩太学公案,将所有支持旧党的学官一扫而空,将许多新党的支持者塞进去占据了他们留下来的空缺。

王安石今日会如此卖力的为蹇周辅等人说话,肯定也正是看见了历史有着再一次重复的可能,不愿意让韩冈和气学得逞。

王安石的反应正好印证了韩冈当时的猜测。王安石闻言,便立刻勃然作色:“此辈唱和,诽毁时政!”

“诚然。”韩冈心平气和,王安石的心乱了,自己想要获胜,却不能乱,“当初国子监中学官以此为题,的确是意在杯葛变法。但今日黄裳以气学解题,却被判错,蹇周辅等又意在何处?”

韩冈字字句句都是将太后的心思往党争的方向去引,王安石心中犹如一丛火焰在燃烧,“臣之前已再三说明,如今朝廷取士是以三经新义为法,此法乃先帝所制,如今也没有被更易。黄裳的作答,不合三经新义之义。”

“既然如此,等御试时,三位已经通过的考生,是否也要以三经新义为圭臬,不论太后到底是问了什么问题?”

“此事早有定论,何须再多说。”

“不论对错?”

“除方才螟蛉义子一条,还有别的错处?”

韩冈与王安石反反复复,都是在说车轱辘的话。但这么一番争辩听下来,向太后也明白了,黄裳的落选是彻头彻尾的党争。

所以王安石明知蹇周辅等人犯了大错,却还要包庇四人。

黄裳考的虽然是军谋宏远材任边寄科,但他的才学不算差,即使是不属于边郡统帅应该了解到的学识,他都掌握得很好。尽管没有通过,但这已经难能可贵了。朝廷又不是要一个能写诏书的知制诰,这样的人根本不缺,在北方契丹人的压力下,需要一个最好几个能带兵打仗的帅臣,分别镇守一方。

至于韩冈说自己做不来蹇周辅的题目,这个应该是谦虚。向太后很信任韩冈的才学,否则也不能有那么多的发明,更不可能跟随张载的脚步,

“平章和参政议论,吾也听了。两位的想法,吾也明白。黄裳落榜,这件事就不用讨论了。正如方才参政之言,为了朝廷的威信,阁试的结果不能改。”

“太后圣明。”韩冈点头,等着太后下面的话。

“不过蹇周辅等四人出错了考题,此事同样无可辩说。上阵的将帅,能明白经义要旨就不错了,让他们对注疏都倒背如流,也未免太过分了。”

王安石的脸色越发的难看起来。

换作是十年前,遇到这样的争议,王安石肯定就是直接告病,甩手让天子决定要留谁。但现在若是使脾气告病,保不准太后就顺水推舟了。从此韩冈便能一人压倒整个政事堂。

凭借他的底蕴,在新学乏人保护的情况下,韩冈他不用多少时间,就能彻底颠覆整个新学系统。至少能把一堆气学的门人,塞进国子监,或是其他重要的位置,并开始在抡才大典中,塞入有关气学的内容。

“陛下,蹇周辅等人是尽其职守,不当罚!”

王安石打算坚持到底,在他看来,韩冈的就是为了黄裳试卷中那一题的对错而喋喋不休。新学、气学对同一条句子的解释不同,今日还能坚持,但再过些日子,说不定朝廷就会在各项考试中改变标准了。为了不看到这一幕,王安石绝不会退让。

“何来尽其职守?题目出错,俱当罚!”韩冈抗声道,“只有确认是错误,日后才能避免再犯。难道下一次有人报名参加军谋宏远材任边寄科,还要去考六论不成?”

“怎么不能考?依惯例当如此。”

“军谋宏远才任边寄科从未开科,哪里有惯例、先例?蹇周辅四人皆是迂腐颟顸,故而只知依循,不敢为后世立标。”

“蹇周辅年虽已老,犹才识敏捷,并非迂腐颟顸。”

“既非如此,如此出题那就是别有用心了。”

韩冈扣死了这一次题目,如果不是愚蠢,就是别有用心。王安石想要为蹇周辅辩驳,就必须同时解释两件事,而这样的解释,却偏偏被韩冈引向党同伐异这四个字上。

太后相信谁。王安石不敢奢望太多。但现在不是退让的时候,“什么叫做别有用心?朝廷开制科,是为了引用朝野内外才识卓异的贤人。到底什么才能叫做贤人,被区区六道考试刷落可算不上贤。”

“即便是韩冈,那六道试题也一样过不了。即便是过了,也称不上贤人。只知经义,即便算是贤人,也绝非能够镇守边地要郡的边臣。”

见韩冈和王安石又要绕回去你一句我一句的说轱辘话,向太后连忙出声调解,“平章、参政,还是说回蹇周辅等四人该如何处置吧。”

“好吧,臣不敢让太后困扰。蹇周辅等人可以稍减惩处,但至少得罚铜。”韩冈很干脆的退让了一步,先定义性质,下面才好展开。

王安石却依然坚持:“无过如何罚铜?”

异论相搅四个字,向太后随着在朝政上浸淫日久,渐渐也明白了是什么意思。

党争无可避免,甚至得鼓励,不过争执得有节制。要是争得一方不能立足朝堂,那就是天子的控制力不够。

放在如今,韩冈为气学一脉,与王安石的新学争议始终不绝。光是之前蹇周辅对黄裳考试的判决,就可以看见新学、气学几乎是势不两立。

如何处置蹇周辅等人,韩冈已经几次退让了,而且并没有因为黄裳是其所荐,又曾为门客,而忘掉了保持一颗公心,坚持考试的结果不能改变,可见其公忠体国之心。

反观王安石,这位平章军国重事却寸步不让,这未免也太过分了一点。要是他能跟韩冈一样,也让上一步,这件事不早就解决了吗?

“平章、参政,有关此事,吾已有定见,还是不用再多说了!”向太后极为决绝的,瞅了瞅韩冈,“还是说回代州的事吧。”

……………………

蹇周辅和他的三位同僚,已经回到了崇文院中。

各自的心中都是惴惴不安,韩冈既然已经去求见太后,而王安石也追了过去,他们能做的,就仅仅是等待着太后的发落。

见气氛如同守灵,蹇周辅笑道:“不要担心什么,就算是太后偏袒韩冈,王介甫也会顶住的。”

赵彦若摇头,光是王安石当真能够抵住韩冈和太后吗?对黄裳判定,到底会不会被改易?蹇周辅再怎么宽慰人,也都是空的,重要的是朝廷的判断。

太后没有让他们提醒吊胆的等待太久,处罚的结果很快就出来了。

蹇周辅知阁试因出题不当,罚铜十五斤。其余三位考官,也都是罚铜,八斤到十斤不等。

这样的处罚看似不重,却是明确了他们出题的错误。而这种处分,日后也会给各人的前程带来难以预测的变化。

不过蹇周辅不算担心,他是为坚持新学而受到处分,王安石和章敦那边肯定会做出补偿。

尽管韩冈必然会杯葛自己的进用,但王安石既然与韩冈争与崇政殿上,这党争的态势越来越明显,党同伐异四个字之下,必定会有好处回来。

但紧随而来的第二封诏书,让蹇周辅如坠深渊,

“《资治通鉴》编修局?!”
