\section{第九章 长戈如林起纷纷(三)}

【第三更,求红票,收藏】

韩冈领队行走在夜空下。星月之光虽然黯淡,不过胜在没有云翳的阻隔,依然很清晰的照着他们的前路。

星汉灿烂,璀璨的银河横跨于天际。星宿二与火星并于与南方的天空,两颗火红色的亮星在天空中交相辉映。星宿二以大火为名,到了七月时,大火向西而行,就是诗经中所谓的‘七月流火’,乃是入秋的标志。

而此时,却是‘五月鸣蜩’,蝉虫在路边的树上欢叫着,树下草丛深处,还有蟋蟀一起合唱,萤火忽隐忽现,山风徐来,带着草木的清香,正是初夏时节的风物。

为了保证韩冈一行的安全,在遇上贼人时能及时逃掉,王韶特意给每人加配了一匹战马。在狭窄曲折的山道上,十二人,二十余匹战马拉出了长长的队列。为防敌军斥候,马颈下的铃铛被摘下了,只有细碎的马蹄声在山壁上回响。

夜中急急而出,王舜臣也知道事情有些不妙,他提缰上前,问着韩冈:“三哥,你今次领下去青唐部的差事,到底有几分成算?”

韩冈头也没回,两眼盯着眼前的路,以防马蹄失足。口里则说道:“没有把握我也不会去的。”

王舜臣可不像韩冈这般充满信心,说起来他才不会管蕃部的死活。一直都跟在王韶韩冈身边,对话听得一清二楚,王舜臣清楚董裕不会来攻打古渭,所以他对韩冈冒着风险去青唐部做说客,只为了拯救蕃部这一事,却是看不过眼:

“按理说,这蕃部的事跟三哥你根本就没关系,你待在疗养院里不就得了。王机宜也真是的,何苦把这么危险的活计都推到你身上,不是有个高提举吗,他也管着蕃部的事,应该他去才是。”

“你也是这么想啊……”韩冈轻声说道。王舜臣的话虽然只看着眼前,可韩冈的心思却被触动了。

让自己独自先去青唐部,说服俞龙珂,这件事王韶不该答应下来的。韩冈方才在王韶、高遵裕面前的自荐,其实只是个挑起话头的技巧。在他想来,王韶不可能点头,而是应该考虑再三后,带着自己一起去青唐。

其实如果在过去,韩冈肯定会把他的想法直接说出来,他相信王韶不会在意这些小节。刻意忽视一些小处的礼节,也是拉拢关系、表示亲近的手段,不过韩冈却采用了迂回的方法——因为他心有顾忌,想测试一下王韶的想法。

就在前几天,因为李复圭冤杀将佐之事,王韶曾经对韩冈说他性格与李复圭相似,要记着日后不要学着李复圭的样子。虽然王韶是半开玩笑的口气,可韩冈却总觉得有些不对劲,因为这话实在不像是王韶该说出来的。

事有反常必为妖,韩冈对王韶很了解,他们都是一类人,心思极重,城府甚深,说话基本上都会在心中绕个几个弯子,才会说出来。王韶对他儿子可以毫无顾忌说着些犯忌讳的话,却也从不开玩笑,既然如此,他又怎么会跟自己说这等不着调的话。

所以韩冈自那天之后就有了心结,想确认一下王韶到底是不是对自己有了忌惮之心。如果有了,韩冈就肯定要防备起来,王韶再亲近,都不如自己可靠。

就拿今次去青唐部说服俞龙珂的事来说,最好的人选决不是韩冈,而是身为太后亲叔的高遵裕。他出面做说客,俞龙珂投过来的可能性要比韩冈出面至少要大十倍。王霸之气一放,小弟纳头便拜都不是不可能。身份越贵重,说话的分量就越重,此事理所当然。

不过韩冈甚至王韶,都不能提议让高遵裕去找俞龙珂。请太后叔叔亲犯险地,即使能成功,都会被记恨——君不见寇准力劝真宗亲征,在澶州定下盟约后,真宗皇帝高兴了几天,可王钦若一番话就让他翻了脸,还为词典添了条成语‘孤注一掷’——高遵裕就是去做说客,也必须是出自他本人的意思。

而次优的选择便是王韶。经过了托硕部之事,王韶在秦州缘边地区,尤其是青渭,已经有了不低的声威。本人又是提举蕃部,他去找俞龙珂,名正言顺。不像韩冈,他的两个差遣都跟蕃部毫无瓜葛。

韩冈相信王韶和高遵裕都能看出这一点,所以他才会自我推荐。正常情况下,王韶肯定会反对。可事实证明了韩冈的猜测,王韶果然对他产生了忌惮之心,让自家先去找俞龙珂,而不是选择机会更大的方法。而且看王韶点头的速度,他跟高遵裕应该早就商量好了。

王韶又是什么时候对自己忌惮起来的?韩冈想不出来,不是找不到原因,而是可能的原因太多了。

王韶和王安石之间有书信联系的事,韩冈知道,他现在都怀疑王安石是不是把他的几条不能曝光的意见都跟王韶说了,但怎么想都觉得不可能。

要不然就是对付向宝的事,王韶虽然接受了他的计划,但向宝的结局,也许让王韶心中有了兔死狐悲的想法。韩冈并不后悔当时自己的手段,因为他要自保,但在王韶面前,当时的确是应该再装一下的。

韩冈的头有些痛,总是揣测人心,其实是很累的一件事,但不去想那么多,心中的不安全感,却会让韩冈感到更累。

这也许是聪明人都免不了的烦恼。韩冈苦笑着,将疑心藏在心底,与王舜臣带着一队骑兵,踏着月色向北行进。

韩冈最终还是放宽了心。因为王韶心里的想法对他没有任何意义。就算王韶把他当作洪水猛兽看待,只要小心谨慎,做好自己的事,也不会有什么问题。还是那句话,能走到眼下这一步,韩冈靠的是自己,而不是王韶。现在也是王韶需要韩冈的帮助,而不是相反。

抬头看着挂在五月初的夜空中的如钩弯月,韩冈突然想了起来,今天可是端午,应该挂菖蒲、艾叶,薰苍术、白芷,喝几杯雄黄酒,镇一镇恶日的邪气。

端午在后世是节令,但在此时却是疫症开始传播、毒虫开始肆虐的恶日。在五毒并出的日子,却碰上蕃人侵攻,而自己又要去找另外一家蕃部借力。说起来蕃人的打扮在普通的汉人们眼里也跟妖魔鬼怪差不多了,这日子还当真不吉。

在月色星光下,翻过两重山峦,前方黑沉沉的几条山谷,就是青唐部所居住的地方。其实青唐部主帐的位置,离古渭寨很近,满打满算也不过三十多里的距离。韩冈漏夜出发,过了子夜,就到了青唐部与古渭寨的交界处。

俞龙珂所居城寨继承其部族之名,而被称为青唐。不过此青唐非彼青唐,俞龙珂的青唐城跟如今的吐蕃赞普董毡所居住的青唐王城【今西宁】虽然同名,但规模上却差了很多。韩冈听说过,青唐王城城墙周长八里许,为秦州以西有数的大城,城中商旅来往不绝,以回鹘商队居多。而俞龙珂的青唐城就只有盐井,听说跟古渭寨差不多大小。

当然,董毡的青唐王城是羌中道中段的枢纽要地,这是俞龙珂的青唐城所不能比的。羌中道则是几条中国通往西域的丝绸之路之一。虽然羌中道地势远不及河西的甘凉道,也不比经过西夏境内的灵州道,但自五代开始各部战乱毁了河西走廊的交通,继而党项人又在灵州道上抽取重税之后,许多回鹘商人都不得不改从羌中道往来。董毡的富庶,就是靠着回鹘商人的税金。而俞龙珂的钱,却是来自于他的盐井,一年三万贯左右的收入,除了董毡和木征,河湟蕃部中,也没哪家能比得上他。

点起火炬,向青唐部的蕃人昭告自己的到来。沿着山道从山坡上向下,韩冈一行已经走进了属于青唐部的山谷。在黯淡星空下的行进,只能看到长条形的天空,身边只有寥寥可数的同伴,又被稀稀拉拉的火炬所驱散不走的黑暗包围,这一切都有点像是半年多前,行走于甘谷之中的那一夜。

当时甘谷城安危未定,两侧山上杀机四伏。而如今,韩冈也已经听到前方谷地以及周边山坡上的骚动。

青唐部的蕃人已经发现了自己,报警号角声接二连三地响起。韩冈闻声便勒住缰绳,命全队止步,在山谷间的道路上等待主人的出迎。

虽然除了王舜臣以外,其他随从都是有些慌乱,从他们手上晃动着的火炬就能看出他们心中的恐慌。但韩冈依然冷静自若,他出来时自信满满,现在也是一样。任何信心都必须建筑在现实之上,如果是毫无根据的信心,那是自大,不是自信。

韩冈却是自信,他能完成任务。他能肯定俞龙珂不想看到自己的出现,从俞龙珂的角度来看,最好情况是宋人从古渭寨滚蛋,木征、董毡还有夏人都安安分分,让青唐部独霸古渭州。

不过无论如何,俞龙珂都不会选择跟董裕合作,这对他完全没有利益可言……

可为什么董裕会看不到这一点?还是说他已经有了应对的把握?

疑问突然而起,在前方突然亮起的无数星火照耀下,韩冈一下皱起了眉头。

