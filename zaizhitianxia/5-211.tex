\section{第24章 缭垣斜压紫云低(三)}

铁轨吗?

韩冈向烤架的方向扫了一眼,发现店主和小二早把韩冈、薛向和两家的元随所点的各色酒菜都做好送上,识趣的躲到了隔邻的铺子中。

“如今只在码头上用了铁轨,若是能将方城山轨道改造成铁轨,再经过半年的验证和对比,就能正式确认铁轨的好处了。到时候,推行天下也能有几分底气。”

韩冈配合着薛向的说话。薛向主动提起铁轨,这自然是示好的表现,理所当然得有一个善意的回应。

“铁轨的好处其实不用验证就能看得出来,总比木轨要方便。”薛向正色道,“现在铁轨的成本比硬木轨道还要便宜,将木轨换成铁轨,道路的造价也能降下来。而且还有日常维护的费用,也能省下许多。”

韩冈点头表示同意,他就是轨道的倡导者,铁轨这个名词还是从他口里流传出去的,一应数据他比任何人都要熟悉:“节省个一半应该不在话下。”

旧有的以硬木制成的轨道的价格其实并不便宜,而且更换频繁,就算用铜皮为垫,也很容易损坏。但因为有着让人叹为观止的运费收入,木轨高昂的维护费用,也不过是让利润摊薄了几分而已。不过任谁都会愿意看到更高的利润,有谁会嫌钱烧手?

相对于木轨,铁轨要强得多,让人担忧的锈蚀问题,相对于木料的损耗,根本微不足道。据韩冈所知,京城码头上的轨道,在更换了铁轨之后,维修费用一下就下降到只有之前三成。若是是方城山轨道也换成铁轨,随着维护成本的下降,那么节省下来的成本自然便意味着利润的上涨。如果换个思路,将运费稍稍下降,由此将能够吸引更多的商家利用这条通道,相对的也能得到更多的收入。

对方城山轨道换装铁轨后的利润预测,韩冈稍嫌保守一点,薛向则是更为乐观,不过两人交换了各自的观点后都能确定,绝对是能让天子也欣喜不已的数字。

君子不言利的‘贤良’或许会对韩冈和薛向的对话嗤之以鼻,贵为宰辅、学士,却还在计较锱铢之利,但韩、薛二人说话的时候,虽然的并不是公廨中的正经严肃,但郑重的语气,也完全不似酒桌边的闲聊。

“……只不过要防备着有贼人贪图小利,从轨道上窃取铁料。”

“这世上哪有完美无缺的事?就算出点意外,有点波折,也不会影响大局。何况京城汴水上的码头已经开始使用铁轨,却没听说哪家被偷盗,不需要顾虑太多。”韩冈不以为然,“而且若是知道窃取铁轨会害死到多少人,还敢丧心病狂下手的,当也是极少数了。。”

“说得也是。”薛向点头,少了轨道,马车一旦出轨,很有可能会造成人车内员死伤,在考虑到严重的后果之后,的确不会有太多人了,“一旦知道轨道上翻车会造成多大的伤亡,世上应当不会有几人敢下手了。”他停了一下,“其实薛向还有一个想法。”

“什么?”韩冈立刻问道。

薛向露出了一丝笑意:“御道是没人敢动的。”

韩冈正拿着酒盏的右手震了一下,但他立刻就仰头饮酒,看不住有什么异样,但对薛向的打算却是看明白了。

御街中央由两条御沟护起的御道,以黄土垫成,没人敢随便踏上去。如果将轨道视同御道,敢于破坏之人以大不敬之罪论之于法,想来也没几人敢于犯禁。

但只是没几人,并不是完全没有,钢铁和黄土不一样的,而丧心病狂的贼子,韩冈在任职地方的时候也判过几个。不过铁轨毕竟不是能卖高价的东西,一点小钱换了全家的脑袋,很少有人会那么蠢。

韩冈不会一厢情愿的认为只要钢铁生产得更多,价格就能越低,会打铁轨主意的贼人也就会越来越少。这么想,就实在太天真了。无论钢铁的价格再降,也不会比无本买卖成本更低。不过后世的铁路既然能够顺利推广,韩冈相信,这个时代也一样能够做到。

薛向很高兴韩冈能够这么配合,今天与韩冈到这家店里喝酒虽然是一时兴起,但与韩冈好好谈一谈却是长久以来的想法。

如果对辽开战的话,一条运力几乎能于水运相媲美的运输线,是战胜辽国的关键所在。只要稍通兵事就知道稳定畅通而且运输量巨大的补给线,对战争的结局能起到什么样的作用。而且薛向很早就想将手伸到没有运河的地方了,只是这件事,必须经过韩冈。

韩冈和薛向两人自然不会交浅言深,但利益交换则是很正常的。韩冈就算眼下不受待见,但天子照样要让他做太子师,日后执掌朝政也不是幻想。

薛向就算不为自己考虑,也要为家中子弟着想。能在财计上见功劳,又怎么可能是那等只能靠清白寒素来妆点门面?为子女考量,为家族筹谋,与韩冈打好关系,也不是什么丢脸的事。

韩冈在年龄上优势太大了,再加上未来帝师的身份,至少在东西两府之中,不会有人愿意与其为敌,交好是主流,最坏也只是不来往而已。文武百官,除了要踩人上位的台谏官,绝大多数朝臣都不愿无故开罪韩冈。

“不过下一条轨道的位置,子正兄觉得放在何处为好?河北吗?”韩冈做着最后的确认。

薛向似乎有些犹豫:“……辽国的那位尚父,说不定正等着借口用兵南方。”

“三月不磨,宝刀也会生锈。十年不战,西军大概就会落到跟河北禁军差不多的等级了。”韩冈郑重其事地说着,“光是甲坚兵利是不够的。”

换而言之,韩冈的言下之意就是耶律乙辛等得起。

薛向脸上有着几分苦涩,宋辽之间有和约在,除非当今天子敢于将岁币免除,否则朝臣们都不会支持辽国,也就是说,没有人会为天子的独断所带来的后果负起责任。既是如此,赵顼还能怎么做?他可没有赵匡胤和赵光义的控制力,能强压着两府为他的决断扫平道路。

枢密院同知和端明殿学士在州桥夜市上对坐饮酒,京城里到了明天,这个消息恐怕早已经传得沸反盈天。

不过薛向不在乎,今天在坐到这里之前,与韩冈对坐饮酒会在天子那里得出什么样的结论,他早做了预测。这个损失,他承担得起。

……………………

回到家中,已经是二更天,连雪都停了,但家里的妻妾却还都醒着。

“官人怎么回来迟了?”亲手接过韩冈身上的披风,交给身后的婢女,王旖貌似随意的问着。

“路上给薛子正耽搁了一点时间。”

“是因为辽国小皇帝的事?”

想想也是,赵顼在文德殿上亲口说出来的事,文武百官与闻,一个白天过去,说不定消息都传到南京应天府去了,京城里面耳朵长一点的当然就听说了。

“嗯……沾了点边吧。”韩冈满不在意,“其实说大也不大,不过死了个人而已。说他是皇帝,其实也勉强。”

“不会有什么事吧?”在旁随侍的韩云娘小心翼翼的问着。

“能有什么事?”韩冈微微一笑。不过是当天子清醒一点,算不了什么大事,与薛向的一番恳谈,才是今天的重中之重。

说是这么说,韩冈自知自己今天在殿上肯定是又让赵顼不痛快了。不过这也没什么,韩冈不是很在乎。与绝大多数朝臣站在同一条战线上,将天子的一厢情愿挡回去要更重要的一点。

判断耶律乙辛在辽国国内的地位稳固与否,赵顼和臣子们有着很大的差异。

赵顼这个皇帝总是一厢情愿的认为奸臣肯定不得人心,坐到皇位上,天生就该得到所有人的忠心。就算是契丹那等蛮夷,也是该有无数忠臣等待时机将耶律乙辛这名窃国奸贼给赶下来。

这个想法是没错。对于辽国的朝臣、宗室和豪强们来说,一个黄口孺子做皇帝那没什么,毕竟是从太祖太宗圣宗传下来的嫡脉,世间的规矩不是如此吗?而耶律乙辛在头顶上发号施令,就让人不忿气了,同是臣子,凭什么他有资格?肯定有许多人想要将耶律乙辛给踢下来。

只是,愿意为此付出生命代价的又有几人?

大臣们看得很清楚,至少是时常能见到天子的重臣,或多或少都明白皇帝这种生物不过是个坐到了一个好位置上的普通人,根本就不会相信有无缘无故的忠心,以及无所顾虑的付出,只是不敢明说出来罢了。

天子,兵强马壮者为之。这是五代武夫们共同的看法。难道当今文臣的见识会还不如五代的武夫,还有人会认为皇帝是天授?所谓受命于天,到这个时代了,读书人中,除了些个老冬烘以外,已经没有几人会全心全意相信了。史书中的反例可是多如牛毛。

这便是天子和臣子决定性的不同。

当然,也不是韩冈这般全然不信,绝大多数还是半信半疑。就跟求神拜佛一般,有几个士大夫会相信去上一炷香,就能一切平安的?但有空没空拜一拜,求个心安而已。

只要耶律乙辛能治国,辽国国中安泰,做一个隋文帝又有何难?怎么得人忠心,听话的富贵荣华,不听话的那就是***,等到在这样的胁迫下习惯了,那么忠心自然而然也就有了。实力才是第一位,王莽要不是自寻死路,玩什么复古,新朝延续个两百年也不是不可能。

以耶律乙辛的手段,要做到这一点,自然不会有什么问题。没用太大的代价就从大宋这边抢下了西夏的半壁江山,想来也是极得人心。除非他年老糊涂,或是病重无法理事,否则想要撼动他的地位,那是千难万难。

做臣子的,有几个看不出来?

当年皇太叔耶律重元起兵造反,耶律乙辛为辽宣宗耶律洪基平定乱事,之后数十年一直致力于打压近支宗室。耶律乙辛如今能如此猖狂,也跟辽国近支宗室无力有关。

只要辽人还没有主动挑起战事,大宋北界依然得继续保持着和平。对韩冈而言,今晚与薛向的会面才更为重要。这是在对天子施加压力,更代表韩冈在朝堂上影响力越来越大,对实现自己的目标,韩冈更添了许多信心。

