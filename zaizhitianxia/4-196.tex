\section{第31章 九重自是进退地(十)}

朝觐过天子,该走的流程也就走完了,再下来就是陛辞。也就是接下来再歇上两日,等到常参之日,向天子辞行。

韩冈也不指望天子还会再召自己入对,除非皇帝的心意有所变化,但看今天入对时的样子,赵顼可没有半点要留自己在朝中做事的想法,而是对打通襄汉漕运十分期待。

反正他也没兴趣像某些官员一样,拖着不去就任,反而再上书求见天子,奢望能撞个大运。就任京西本就是他自己的选择,否则就不会对外提起。比起在京城,到京西,对自己的计划有不少好处,但这并不代表韩冈愿意为了让皇帝将他的那点小心思放稳了,而在京城之外任官十数年。

韩冈看得出来,他的雇主对自己的年纪有很深的顾忌,不想看到自己三十上下入居两府,从京西开始,自家很有可能十几年内都要在外地不停的转任。

如果能回熙河路镇守,韩冈倒是乐于就任,发展当地经济的同时,甚至可以想办法夺占兰州,若有机会,可以顺便将河西走廊也收回来。但韩冈是不指望了,除非立有殊勋,否则想回乡里去任亲民官或是监司官,基本上是没有什么机会了。只可能在需要他才能的地方,被调过去做事。

正常的人肯定是不可能心甘情愿的。

而韩冈很正常。

既然做了这么多事,韩冈当然也希望能得到合理的回报,若是被人当做马牛一般使唤,他自然是不甘心,不愿意,也不打算默然承受。如果在后世,有这样的遭遇,该怎么做自是不必说,可是在这个时代,通常也只能叹气,再怎么说都是垄断,留给人们的并没有其他选择。

不过这是后话了,京西的差事韩冈还是很乐意的,至于以后的事,那就放在以后再说,先顾着眼前。

韩冈伏案疾书,他还有些想法要传递出去,为了能够顺利将这一桩差事给完成,韩冈还需要一些助力。

周南进来的时候,就看见自己的丈夫一刻不停地动着笔。她也不去看韩冈在写什么,只是送了一件狐皮夹袄过来,前两天趁着日头好给晒过,而方才也已经就着火烤过了,穿在身上,却是暖和得很,还带着周南的一片心意。细细嗅了嗅,还沾染了一丝周南身上特有的清甜淡雅的香味。

周南带来的狐皮夹袄算不上多珍贵,但暖和的很。比起前些日子,从熙河路的家中送来的一张花熊皮要强不少。那花熊皮黑白纹,很是惹眼。

不过现在京中贵妇们正流行的用狐狸腋下的那一小块皮毛,几千块缝制成的皮裘——也就是所谓的集腋成裘——韩冈家里则没有,那样实在是太夸张了。还有用还在胎里的羊羔皮缝制的皮袄,有人送来,韩冈听说了之后就给推了。就是狩猎,也不会射杀怀孕或是带着幼子的母兽。

披上夹袄后,牵过周南的手,让她在腿上坐下。周南的小手细腻滑.润,只是她体质偏寒,到了冬天手脚就很容易变得冰冷。韩冈紧紧握着,掌心的热力传到了周南的小手上,渐渐的就暖和了起来。

“官人明天是不是在家里歇着?”周南依偎在韩冈怀里,贪恋着坚实胸膛带来的安全感,过了半天,才低声问着。

韩冈摇摇头,指了指桌上的一封名帖:“明天要去见见沈存中,想必现在上门,用不着挑日子了。”

韩冈也不知道沈括愿不愿意屈就,毕竟是从翰林学士下来的,地位并不低。从合班之制上,翰林学士比龙图阁学士高上一级,正常情况下就是离任,朝廷也会给予相应的封赠,以维护朝廷重臣的权威。

不过眼下沈括当是要因罪贬官,该有的封赠应当都不会有,如果能将他调去汝州或是唐州,自己也能轻松一点。

……………………

吴充自担任宰相后,便是门庭若市。上门来拜谒的大小官员数不胜数,其中有的可以拒绝,但有的就不能拒绝,从宫中回到府邸,每天还要接见几十名官员,比起西府时,忙碌了一倍还多。

好不容易终于得空下来,在书房中坐下来休息,长子吴安诗便亲自端了茶过来。

“听说韩冈今天已经上殿了?”吴安诗笑着道,“这一次韩冈入觐,拖了快一个月才得以面圣,看来在天子面前失宠了。”

“你从哪里听来的?”吴充抬头看看自己的长子。当初韩冈炙手可热的时候,他曾劝说自己不要太针对韩冈,但现在韩冈看起来在君前不再受到重视,便又变得幸灾乐祸,这让吴充为他的前途还有他吴家的未来担心起来。

“外面都这么说。韩冈一任都转运使,若是天子看重,哪里可能要在阁门处依序轮对。”

吴充一向不喜欢跟家里面提及公事,尤其是晋身两府之后,崇政殿中计议的国事基本上都藏在肚子里。

不过若是儿子在官场上犯了事,做老子的也逃不过罪名,所以该提点的时候,也会提点一二。

“襄汉漕运若能成事,对国中不无补益。要跟韩冈过不去,等他真的弄出了事再说。”

“前些天不是有人说韩冈是好大喜功,要上本……”

“别与他们多来往,不是什么好人。”吴充瞪了儿子一眼,“都是些钻营之辈,见风使舵,就如那沈括一般。”冷哼了一声,“也不看看西京御史台由谁主掌,判河南的又是谁,韩冈去京西,多少只眼睛盯着,没必要越俎代庖。”

政坛上的斗争,没有说将哪人置于死地。尽管上表弹劾时,总少不了对目标喊打喊杀,要以谢天下、以正纲纪、以儆效尤。但实际上,就算成功解决对手,基本上也只是贬官而已。

甚至还不会太苛刻,去江西或是荆湖就已经是很严厉的处罚了。而自丁谓之后,就再也没有因为政争而将对手踢到岭南去的例子。

韩冈既然在外任官,吴充也没必要再多此一举。

何况吴充在两府中多少年了,哪里能不清楚汴河对开封的意义,从天子到小民,人人都知道,一旦没了汴河,开封这座城市无法独存。

所以当韩冈被确定主持襄汉漕渠,吴充根本就没想过再下手。谁敢在这时候与韩冈过不去,天子就会跟他过不去。

反正韩冈几年之内也进不了京城,天子打算如何对待韩冈,还有今日为何没有让韩冈越次入对,明眼人都看透了。既然如此,贸然出手反而会让韩冈得利。天子可以将韩冈晾一晾,但若有人攻击他,天子反而要提拔他了,否则日后谁还敢做事。

韩冈要在京西做事就让他做好了,不必下手干涉。还是先担心一下自己,为了一个相位,自家已成了众矢之的。吕惠卿就不说了,就连王珪也是如同乌眼鸡一般。吴充手按着桌子,叹息声不由自主:“高处不胜寒……”

“是苏子瞻中秋咏月的小词?”吴安诗反应很快,却一时没有领会到吴充为何如此感慨。

吴充神色平平淡淡,儿子木讷也不是一天两天了。

尽管在反对新法上有志一同,但吴充并不怎么喜欢苏轼的行事,不过这不代表他会不喜欢苏轼的文字。

自从苏轼到了密州之后,文风大变,超脱了旧时窠臼。有些人不喜欢苏轼风格大变后的文字,认为不合词旨,却也有人十分看好。不管怎么说,苏轼都是自出机杼,开创出一片新的天地。

吴充也对此感觉还不错,前两年新出的《眉山集》正摆在他的案头上,时常翻阅。而之前流传出来的《密州行猎》,吴充感觉还搔不到痒处,用江城子的旧调唱起来也很是怪异。可如今这一篇写于丙辰中秋的咏月词,还有那一首悼亡词,则隐隐有了卓然大家的风范。

此前苏轼在密州任满,本来说是要调往徐州。不过徐州自发现了煤矿后,京城打造兵甲的几百万石生铁全都得靠利国监提供,为防万一王安石还是让吕嘉问去了徐州,而苏轼则是留任密州。依王安石的意思,当是让他且去填词,至今已经又是一任将满了。

“其实徐德占的文字,也是不错的,就是诗词不如苏子瞻。”吴安诗忽然又说道。

“差得远了。”吴充摇头,“文章憎命达,苏轼出外几年,笔力越见圆熟,徐禧已经远远不及。”

吴充放下心头事,评论起如今的文坛来,“当今文坛,自欧阳文忠去后,王介甫已独占鳌头多年,现在终于多了个苏子瞻。”

在苏轼之前,徐禧的文章享誉一时,世人争相传颂,不过眼下已经给苏轼掩了过去。吴充不喜徐禧,最近在朝中宣扬攻打西夏,收复兴灵,徐禧是其中声音最大几人之一,听说他还与吕惠卿联姻,为才两岁大的幼子,与吕惠卿的女儿定了亲事。

吴安诗不敢与父亲争辩,只道是吴充憎恶吕惠卿,因而连带着将徐禧一并看不顺眼,一时便沉默了下去,就看着吴充从书架上拿出了眉山集,随手翻阅了起来。

