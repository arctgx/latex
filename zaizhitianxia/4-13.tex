\section{第一章 纵谈犹说旧升平(13)}

【国庆有事,今天只有一更了——这是算在十月一号的份上。十月二号开始,还是照常两更】

吴充的诛心之言刚出口,赵顼听了脸色便是一沉。

不论做皇帝的再怎么宽宏大量,朝中的臣子家中藏着一队百人敌,总是难以忍受的。以数人大败百人,怎么想都绝不会是运气的结果。韩冈坐拥此等死士,就算他没有反逆之心,也是个威胁。

韩冈用眼角余光瞥了吴充一眼,就见他的神色恬淡平和,好像他方才说的不是要致人于死地的谗言,而只是一句家常话而已。

‘好聪明啊……’韩冈心中冷笑着,迎头对上赵顼的目光:“臣家中的家丁是上过战阵的军中健勇,纵然因残病而退,各有内疾,再上不得阵,但眼光还在,历练犹存,岂是磨坊中的厢兵可以欺辱?对上从没有见识过战事的厢兵,若是还能输掉。曾经败给他们的吐蕃、党项两族的贼寇,在坟墓中也不会甘心。”

“不论是否残病,其所对阵厢军,纵未上阵临敌,终究也是百名身体完好,体格壮健的军汉。以数人胜百人,其武勇岂是等闲?”

吴充像一头团鱼,咬住了韩冈就不肯放口。这么难得的机会,他怎么可以错过?韩冈过去露出来的破绽,从来都是陷阱,吴充也吃过了好几次亏。但今日之事,就算还是陷阱,他也要一脚踩下去。‘蓄养死士’这四个字只要揪住了,韩冈就是挖了多少坑,照样别想脱身。

韩冈立刻加以驳斥:“臣家家丁能胜,非是胜在武勇,双拳难敌四手,就是万人敌,四面被围攻,又怎能立敌?而是靠着多年行伍的经验和眼光。”

吴充呵呵冷笑,对着赵顼道:“以臣观之,更多的当是胆略。岂不闻一人奋死可以对十,十可以对百……”

‘……百可以对千,千可以对万,万可以尅天下矣。’韩冈在心上将下面一段帮吴充念出来了。出自《韩非子》的这一段,用到现在,对他来说可不是好的比喻。

“吴枢密有所不知。”韩冈心平气和,“臣家门前街巷狭窄,仅可容一车或是两马,两侧又是高墙深院。如果放在战场上,就是一夫当关万夫莫开的地形,用三五人就可以守住了。对手人数虽众,可一旦封堵巷道,要面对的也只是眼前寥寥数人。不信陛下可以命开封府详加询问,看看臣家家丁究竟是如何做的?”他说着,又微微一笑,“皆是百战余生,如何不明临敌陷阵?遇上身陷谷道的敌方大军,要从何处下手,根本不需要多想,熟读兵书如赵括、马谡者岂能及之?”

韩冈语带讥讽,又是盯着吴充说话,等于是指着鼻子在骂如今的这位枢密使,不过是只懂纸上谈兵的赵括、马谡而已。

两名臣子之间雷霆风暴一般交锋,赵顼如何听不出来。吴充要陷韩冈于死地,赵顼也不可能看不出来。但他的心中有着深深的疑问:“韩卿,这些军中精悍为何会投奔到你家?”

“臣家家丁多为阵上伤残,难以恢复,不得不离开军中。正好臣主管疗养院事,故而多来投奔。臣家本是寒门素户,而陇西又非乡里,户牗乏人,也只能来者不拒。”

“韩冈!军中因战伤而残,什么时候会将人汰撤出去?只是降入下等军额而已,照样能领着一份俸禄。”吴充一声断喝,“你这是欺君!”

“嗟来之食,不知枢密可愿食之?!”韩冈冷声质问,问得吴充神色一变,又继续说下去:“但凡战事,只要不是大败,会在战阵上受伤的,无不是立于阵前、直膺敌锋的勇夫。此辈向以勇力傲视同侪,率为心高气傲之人。一日以病残而落于下等,纵然能忍得下旧时的骄悍之心,也免不了会受到一干庸人的嘲笑。如此情状,试问又有何人愿意留于军中,为人耻笑?”

“不为五斗米折腰,想不到军中有那么多士大夫!”

对于武夫的鄙视,在士大夫们的心中根深蒂固,吴充对韩冈的话嗤之以鼻。要怎么对待武人?从太祖皇帝开始,就秉持一个宗旨:薄其官称,厚其爵禄。投军只要有战功,就能得到丰厚的赏赐,但到了文官面前,就要老实做人,别把自己看得太髙。当兵的在此时只有一个字——贱。脸上刺字的赤佬,就算显贵如狄青又如何?妓女亦可辱之。

“燕赵多慷慨悲歌之士,秦雍岂无之?”韩冈冷笑着,“若无为国效死的忠心,如何会陷阵冲营?!只凭区区财物,能招来的不过是啸聚之辈,利来则至,利尽则去。难道在枢密心中,国朝百万大军,尽是此辈不成?……而且还有一事,枢密应该很明了。将兵法推行于军中,各路整军设将,于军力上确为上上良策。但各军汰撤剩员,却也不免有些错漏。尤其是下等军额之中的老废,裁撤的则是最多的,臣家的家丁,倒有一半来自于此。韩冈敢问枢密,汰撤剩员的军令到底是不是盖了枢密院的大印!?”

吴充声音一滞,倒不是因为韩冈突如其来的一击,而是突然发现话题已经给韩冈带偏掉了。天子的视线投过来,吴充匆忙说道:“无论如何,此乃是收买人心之举!”

“若依吴枢密之言,日后至于修桥铺路、扶危济困,设粥厂、散汤药的事,就不要让人做了,因为人心会被收买。若是遇上灾年,百姓流离,就算官府不及救治,他人也不能来救,因为人心会被收买。让他们饿死好了,吴枢密是不是就是这个意思?!”韩冈几句话下来,已是声色俱厉。转身对着赵顼,一指吴充:“陛下,吴充此人奸邪,岂可留于朝堂!为政者当劝人为善,而非让人不敢为善!造悚言,危天子,试问日后谁人还敢行善事?!若陛下以为收留残病之人有罪,臣甘当其罪!”

赵顼能定韩冈的罪吗?当然不能。他不满的盯了吴充一眼,这个话不能乱说的。

吴充也不能定韩冈的罪,但他能让赵顼对韩冈心生疑忌就已经满足了——现在也许并不在意,但等到私底下想起来,必然会升起一丝隐忧。现在即便当面被韩冈骂,吴充也不怒,反而很平静的说道:“韩冈所为或许是善心,但日后若有奸人仿效,可能免其乱?”

“若日后伤残军卒皆能得到妥善安置,后人如何能仿效?”韩冈冲着赵顼一躬身:“陛下,尽管此辈不能再上阵杀敌、为国效死,但皆是老卒,经验丰富。若于一营中设立教导队,将经历过战阵,已有残病的老卒调入其中,加以勇号,饰以美名,让其教训士卒,其人必当尽心尽力以报陛下恩德。”

这是能示好军中卒伍的举措,不管最后能不能成功,只要外面的士卒知道创立了疗养院的韩舍人帮他们说过话就行了。当然,能成功自是最好!

赵顼沉吟起来,韩冈的话的确引起了他的兴趣,而韩冈家的家丁也表现得足够出色。如果依照韩冈所言,以曾经立过功勋的残病士卒为教导,厚给封赐,让他们在军中言传身教,或许当真能让禁军的战力上一个台阶。

看见赵顼的反应,韩冈趁热打铁:“京营、河北两地的禁军久不交战,其战力堪忧。可若是从外调来将领日加督训,又难免惹人议论,启人疑窦。但如果仅仅是设立教导队,以老卒带新卒,则不必担心会有任何后患。”

“吴卿……”赵顼转过头来问着,“韩卿此议可行否?”

吴充没想到韩冈轻又是这般轻而易举的就转移了话题,惹起了赵顼的兴趣。现在再对韩家家丁的武勇紧咬不放,可就是会引起赵顼的不满。

“更易军制非同小可。臣请陛下将此议下中书、枢密院,并两制以上官共议,以定可否。”

吴充拖延着时间。虽然韩冈跟自己的儿子是连襟,但他越看韩冈越是碍眼。有这个女婿在,对王安石的帮助实在是太大了。过去他能撺掇着天子整修黄河金堤,现在又撺掇着天子考虑起改变军制,说不定再过一阵子,就能撺掇着让王安石复相!

只是想要找个由头将他赶出去,总是难以如愿。韩冈身份虽卑,与枢密使天差地远,但想要动他,必须要有天子的同意,绝不是件容易的事。

是不是往熙河路派几个人去?虽然麻烦点,但总能抓到把柄。当不会像面对韩冈,看着纵有错处可以攻击,谁想到全是陷阱?想法、行事总是出人意表,让人全然捉摸不透。

韩刚亦是冷冷的用眼角余光撇着吴充。

跳的太欢不是好事,方才吴充一个劲的乱喷口水,当已经给天子留下了深刻的印象,日后吴充再攻击自己,就很难让天子相信他的言辞。但话说回来,如果一名宰执级的官员盯着一名小臣,有很大几率,天子会为了安抚重臣,而将那名小官给踢出朝堂。这样的先例有很多,吴充说不定就在打着这个主意。

不过这样就要赌一赌在天子的心目中,谁的份量更重了。想必吴充自己都不敢确定,他的份量能胜过自家。

只是韩冈心中对此没有一点欣喜,他想要的是任何人都动摇不了的地位,而不是将自己交由他人来衡量——即便那人是皇帝。

