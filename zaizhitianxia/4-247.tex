\section{第40章 雁度长空迹不彰(下)}

【写着写着就睡着了,下一更稍微迟些。】

就像耶律乙辛和萧十三所议论的,刚刚亲政的西夏国主秉常,拿着亲舅舅的亲信景询开刀的事,也通过安排在兴庆府中的诸多细作,传到了大宋一方。

景询是有名的汉奸,跟在张元吴昊之后带着党项军攻入中国烧杀抢掠,用汉人的尸骨垫起了他在西夏国中的地位,是西夏国中数得着的汉臣代表。

陕西这里曾经把他的老娘给捉了起来准备治罪,不过英宗皇帝觉得这么做实在太丢份,有失朝廷体面,就下诏将人给放了。

但对景询本人,无论是英宗,还是当今的天子,都是念念不忘,一抓到西夏逃人,就是转起交换的念头。甚至当种谔说降嵬名山,夺了绥德之后,西夏来人讨要,大宋这边开出的条件,就是你把景询送过来,我们就把绥德和嵬名山还给你。

西夏并没有答应这个条件。他们不相信宋人的诚意是一条,但景询在西夏高层的眼中,也的确比起绥德城更重要。

这样的人死了,对大宋当然是好消息。而他死因的背后所代表的意义,更是让人欣喜欲狂。

什么消息能比得上敌人内部分裂,自相残杀更让人感到兴奋?

是以种建中和种朴弹冠相庆,种谔捻着胡须,连着几天的好心情。听说环庆路的种诂、种谊更有随便找了个不着调的理由,然后暗地里为此事摆酒庆祝。

“如此一来大局定矣。”种建中脸颊上是参杂了酒醉和兴奋的酡红,“西贼内乱,正是官军征讨的时机。乱得越厉害,离心离德的部族也就会越多,这样一来,说不定就会变成官军攻打交趾时的情况,不必官军动手,跟随官军的蕃部就会出头平定,大事小事都是由蕃部出手,官军只要坐镇后方,做好后盾就够了。”

“韩玉昆如果听说了景询被杀的消息,不知还会要官军打到银夏便好?”种朴端起酒盏,笑道,“官做得越来越大,旧日的锐气倒是见得少了。”

“韩玉昆也是担心瀚海难渡,便一心求稳,所以之前才在信上说,最好只收复银夏。”种建中道,“不过眼下西夏国内的局势既然变了,想必他也会改弦更张。”

这份情报从兴庆府传到陕西,数日后又从陕西传到了京城,传到了天子的案头上。

最近收到的都是好消息,让赵顼因为七皇子的病情而忧心忡忡的心情,得到了许多安慰。

今年秋粮继夏粮之后,又是个大丰收,这当然是头一桩值得庆贺的大事。因为熙宁后半段的连年灾异而日渐空虚的各地常平仓,也终于可以回复到安全的界限上。

襄汉漕运终于开始启动,则是第二桩喜事。虽然水运通道还没有全然完工,但中间的转运轨道据称有足够的运力将荆湖的秋粮赶在京畿河道封冻前运来京城。赵顼深悉韩冈的为人,知其言不轻发、发必有据,又知其才干,对襄汉漕运抱着深深的期待。

多上一条生命线,京师就能多安稳一分,只要能补充上汴河的几分运力,日后汴河即使有什么问题,也不用担心得食不下咽,至少还有襄汉漕运能为京城运来足够的粮食。

手上有粮,心中不慌。有了粮食打底,赵顼也就能抱着期待的心情看着西夏国中变局。

叛臣景询之死,让赵顼欣喜不已。这个叛逆,对于大宋两代君臣来说,几乎是一个心结了。明大宋虚实,又深悉兵法,虽然这几年西夏被压着打,他也无从表现,但他在治平年间,以及熙宁初年,几次三番的引来了党项骑兵,指点他们在大宋境内烧杀抢掠。

此人的存在,让赵顼难以忍受,故而一直都想将他弄回来,或干脆点,将他送下去重新投胎。不过即便是贵为天子,也不是想做什么就能做成什么,景询一直好端端的活着,让诅咒他去死的人暗地里咬牙切齿。不过如今他终于是死了,而且成了新一轮政治。斗争开始的标志。

内部分裂趋于明面化,浮上了水面的纷争,必然能给灭夏带来积极的因素。说不定官军还没有抵达兴庆府城下,城中就已经杀个你死我活了。

“死得好啊!死得好啊!”赵顼大笑着,完全失去了贵为天子的矜持和稳重。

……………………

“辽、宋两边肯定已经收到消息了。”兴庆府中不止一个如此议论着,“两家都不是好惹的,给他们看到了机会,恐怕不会坐视。”

“把破绽给了敌手,他们能不笑纳?只恐日后国中便再无宁日。”

“最好能尽快解决这个问题。等到兵临城下,再来议论此事,那可就是来不及了。”

“不知李将军准备站到谁一边?”

同样的问题在短短的时间内,于兴庆府中出现过多次。眼下的国中局势,容不得人首鼠两端,已经到了必须分边站了的时刻

最大的麻烦是秉常的长儿为耶律氏所生,也就是所谓的嫡长子。只要不出意外,那个让人怎么都不能看顺眼的小子,就是未来的西夏国主。

都已经到了这个田地,母子亲情也没什么好讲究的了。在权力之争前,亲情什么的,都是可有可无的。

“有梁太后在,肯定是会幽禁,但总能留陛下一命,而陛下如果得势,那就说不准了,辽人性格凶残,可不会愿意留后患。李将军,你说能让陛下成为弑母之人吗?……还有泼喜军,宋人之中,能工巧匠颇多,旋风砲他们最多看上两眼就能仿制出来,比起契丹,肯定是要强出百倍。如果契丹人想仿制的话,他们肯定会将泼喜军的老底搬空,一点也不留。”

泼喜军是西夏军中难得的一支纯技术性的队伍,通过加载骆驼背上的一具具小巧灵动的旋风砲,将一块块拳头大小的石块送到敌人头上、身上。几十年前第一次出现,就是在三川口的时候,只有区区两百人的泼喜军,就打出了西夏汉军的名头,不再是撞令郎一般的敢死队。

就是瞧不起西夏工程技术水平的宋人们,看到了泼喜军的威力后,也都纷纷惊讶起来。投掷石块的器具,从来都是往大里造,哪里有放在骆驼背上的设计?不但新奇,而且实用。就是工艺太过繁琐,迄今为止,才三百多架成品,让泼喜军想扩张都难。

就是由于人数的关系,两三百人的数量实在是太少了点,根本参与不到正经的大战上。否则对面一个冲锋,泼喜军就会像是碰上了滚水的新雪,转眼就化作泡影。

不过统领泼喜军,让李清在西夏军中的地位变得很高,同时又加强了他在朝中的地位,在兴庆府中,他仿佛是汉人从军者的代表。出战时,不仅是泼喜军跟在他的身边,有时候纯属敢死队的撞令郎,也归入他的管辖范围。

也就是因为如此,他也成了双方争取的对象。该站在那一边,这让李清颇费思量。

是站在太后一边,还是站在国主一边?一旦站错队了,可没有后悔药吃,能保住小命都是万幸,更不用说眼下的富贵。

昨天夜里国主一方有人来访,但态度和口才,均不及今天的客人。而且态度有些倨傲,言辞间少了几分礼貌,让李清很是不喜。

从情理上来说,李清他必须带着他手下的兵将,全力支持他的君主。但从实际的情况来看,支持秉常,就是支持契丹。

原本在西夏国中,汉人就在党项人之下,等到契丹人来了,肯定还要再降上一等。

李清低头沉思。他是汉人,如果要他在大宋和辽国之间做出选择,他当然愿意偏向大宋。

若是西夏支撑不下去,他投靠宋人,不仅能有个富贵终老的结局,想去去苏州杭州去好好逛上一逛,也不是太难的事。而继续忠于君主,就要做好到契丹国养老的准备。以眼下宋辽两国的国势来看,还是宋国一方的机会更大一点。

点了点头,李清知道自己该怎么选择了。从个人的前途,还是泼喜军的前途上,李清觉得只有梁家才是最合适的选择。

尽管眼下梁家还没有投降大宋的打算,但随着时局的一步步发展,李清觉得,孤伶伶的梁家如果没有靠山的支持,只会有一个灰飞烟灭的结局。

梁氏兄妹并非蠢人,该如何投效,在什么时候投效,从而实现自己最大的利益,甚至在宋人的手上保持自己的独立性,这些都是需要多费思量。

这就是小国的无奈和悲哀,一旦持之以立国的根基不在了,就只能等着大国之间在剑拔弩张之后分出个胜负来。

也许在西夏与无定河畔第一次惨败之后,今天的结局就已经注定了。小国没有失败的权力,不比宋辽一般大国,连着败上数次,都不会影响到国家的地位。

他冲说客拱了拱手:“请回复太后和相公,李清唯命是从。”

也就在西夏国中势力正一分为二,天下各国正关注着此事发展的当儿,一名党项医师带着十数名护卫,还有几十名强要护送他一程的乡民,从盘桓已久的襄州南方的伏龙山中走了出来,向着襄州城进发。

