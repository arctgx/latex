\section{第29章 顿尘回首望天阙(15)}

【第二更。求红票。】

‘文王拘而衍周易,仲尼厄而著春秋;屈原放逐,乃赋离骚;左丘失明,厥有国语’

司马迁的《报任安书》中的这一段,是安慰人的话。文章憎命达,无论李青莲还是杜工部,哪个不是一生坎坷,才有了流传千古的名篇。

但韩冈决定还是不让章惇这番话传给苏轼了——在他口中说出来,那就变成讽刺。传到已经上书请求出外的苏轼耳中,也显得自己太过咄咄逼人。而前来恭贺他得赐佳人的章惇,恐怕也会听着不舒服。

——虽然韩冈是真心想安慰苏轼。苏轼的一封判状,其实是帮了他大忙。在如今已经佳人在抱的情况下,韩冈也不会对名传千古的诗人再留着怨气,转着报复的心思。

而宫里现在还没有消息,赵颢还好端端的安住着,不过苏轼已经不想在东京待了。他这个开封推官本作得就不痛快,不幸天降灾祸——推官主管的其实是刑名,要不是蔡确撂挑子,周南的申状也不会压倒他案头上——再留京城暂时找不到什么人喝酒聊天了,不如远放江湖之外,散散心,等今次的事消停了,再回来也不迟。

所以赶在今日苏轼就上了一本,又老调重弹,把新法骂了一通——这是范镇传下来的绝活,许多官员现在都用上了,让天子不好挽留,直接放人。

章惇对苏轼的做法显然很不以为然,但今夜来道贺的时候,也没有对韩冈说太多。

今天有不少人恭喜过韩冈得赐佳人,但知道他赶在今晚就要纳妾的,就只有亲自送来中书调令的章惇,还有李信、李小六这样韩冈身边的人。

听说了韩冈今晚就要纳妾,章惇就主动留了下来,帮着主持了小小的仪式,也算做个见证。等先送了周南入洞房,韩冈就坐下来陪着章惇喝酒。

将杯中酒一饮而尽,举杯底向韩冈示意了一下,章惇摇头叹着,“苏子瞻这也算是无妄之灾,糊里糊涂的就坏了名声。等明日愚兄便打算在堂除的差事里,寻个风土宜人的去处,让他过去修养两年。”

低品京朝官的任免和差遣注授,依律都要通过审官东院,但政事堂也直接掌握着许多职位的任免权,可以跳过审官东院而直接任命官员——这种政事堂直接除授官职的做法就称为堂除。章惇是中书五房检正公事,要帮苏轼寻个外任的好地方,却也是不费吹灰之力。

“江左水乡,苏杭之地。若论风土,再无胜过这两处的了。”看着章惇对苏轼的维护,韩冈有着几分感慨,与章惇做朋友还真是让人安心。

韩冈只是随口说说,但章惇倒是当真去考虑了这两个地方,“……杭州通判到了年后,磨勘就满两年了,考绩也是中上,当可迁官……若要换人,还正好趁现在!”

杭州?……看来苏堤应该不会湮没在历史的长河中了。

韩冈为自己的提议而感到庆幸,又对章惇道:“其实韩冈慕苏子瞻大名久矣,本还想着寻机借着检正的光,去拜会一番。只可惜出了今次的这一桩事……”

韩冈倒是想见见苏子瞻,不过今次是没机会了。不提苏轼糊里糊涂犯下的疏失,让两人不便相见,就算想见面,韩冈也没有那个时间。

上午接到天子口谕,直接把周南赐了他,午后,中书省调令终于下来。从调令的字里行间,看出了催他上路的意思。韩冈知情识趣,甚至准备不过夜,直接就收拾行装离开。倒让亲自来送调令的章惇措手不及,好说歹说,才把韩冈劝下来,再留上一夜,也好把纳妾的事办完再说。

——好吧,其实这是韩冈做做样子,他可不想浪费了洞房花烛的良辰美景。不过尽速离开京城的打算,却是真心的。

风头火势烧得皇城漫天红光,他这个煽风点火的罪魁祸首当是早点离京为宜。今次一桩公案,不像前次请二王出宫,只有章辟光一人冲杀在前。王安石的赞同又不幸引来了反变法派赌气式的针锋相对,而宫内赵顼虽然千肯万肯,也不便违逆高太后的心意。朝中分裂,宫中也反对,天子也只能干瞪眼。

可如今风势已经闹得很大,士林清议又一面倒,苏轼一时之误,便不得不自请出外。朝堂已是奇迹般的用一个声音说话,高太后就算再反对也无济于事。后续情节的酝酿和发展,可能要一个月到几个月的时间。但已经开了头,声势造了起来。天子的两个弟弟就不可能再安居于宫中。

“愚兄的第二份奏章已经写好了,明天就呈上去。不仅是愚兄的,有好几个御史都有打算。而且再过几天,相公也要上书,请为二王于宫外近处造邸,以便二王能时常进宫。”

章惇城府甚深,但成功的在天子面前表现了一下,也免不了有些兴奋,不过他也知道这是谁的功劳,“还是多亏了玉昆你的计策!”

章惇说得毫无顾忌,在座的只有他、韩冈,还有李信三人。李信的身份和性格,决定了他不会泄露任何关于韩冈的秘密。

韩冈谦虚了两句,转对李信道:“表哥,小弟明天就要离京,你一人在京中可要万事当心。”

李信重重地点头,吐出两个字:“放心!”

“玉昆你放心好了,前次是愚兄的不是,找错了人。不过今次就算用强,也要逼着蔡持正把事情办妥当。你就尽管静候佳音。”

章惇赌咒发誓的要好好帮韩冈盯着要将功赎罪的蔡确。这时,就听到院外的敲门声,李小六过去开门,放进来的驿卒传来的消息,竟然是蔡确过来贺喜。韩冈和章惇面面相觑,如今用得着蔡确的地方很多,也不便拒之门外。

让李小六出去迎接,蔡确很快就微笑着走了进来。能毫无愧色的前来恭贺,韩冈都为他的脸皮厚度而感到惊叹。不过蔡确很快就要上任三班主簿,李信任官的事还要托他照顾。韩冈也便毫无芥蒂的上前迎接,看样子好似完全忘了蔡确前日的背信弃义。

对着韩冈,蔡确表达自己的歉意:“应承玉昆你的事没有如约,愚兄也是很过意不去。不过在那时,非是愚兄要故意毁诺,实在是不能向刘庠跪下去。也幸好玉昆吉人天相,有天子垂青,不须我等多事,轻易逢凶化吉。”

“蔡兄所坚持的乃是正事,公而忘私,韩冈怎有脸皮去怪责蔡兄。何况幸得天子看顾,蔡兄也无需耿耿于怀。”

听得韩冈如此说道,蔡确脸上的笑容就多了起来。他指着李信对韩冈:“你这表兄性子沉静,这是极好的。但不擅与热火打交道,就有些让人头疼了。不过有愚兄在,必然帮你处置到最好!”

蔡确又一次拍着胸脯向韩冈表示今次将会重信守诺,李信站起身向蔡确表达了谢意。四人接着又痛饮起来,韩冈与蔡确言笑不拘,看似已是毫无芥蒂。

喝了半夜的酒后,章惇、蔡确告辞离开,“洞房花烛,不能轻负。就不打扰玉昆了。”

他们都是通晓人情的人精,不会打扰韩冈洞房花烛夜的快乐。而且两人明天都要上朝,也不能耽搁太久。而李信和李小六也回了自己的厢房去了。

韩冈醉醺醺的进了内间,墨文便上来搀扶。韩冈笑了,他看着酒气重,可是没喝多少。

进了房,就见着换了一身桃红色喜服的周南就坐在床边上,头上的盖头仍在。两支儿臂粗细的红烛在桌上静静烧着。堆在床后的还有好几个箱笼,这是教坊司今天送来的,有周南的私人财物,也有姐妹们凑的贺礼。

韩冈径直向床边走过去,脚步声让窈窕柔美的娇躯紧张得绷了起来。

伸手掀开盖头,一张宜嗔宜喜的俏脸轻轻扬起。双眸中的深情,如同一汪秋水,让韩冈整个人都陷了下去。

“官人……”周南轻轻叫着,不再是应酬时称呼客人,而是叫着三生所寄的良人。

周南动情的呼唤,让韩冈坚如铁石的心都变得酥软。他侧着身子坐了下来,“让娘子久等了。”

墨文用着银杯,端了两杯酒跟着过来。交杯酒前日其实也喝过了,但那是起哄,今次才是正经的仪式。

交替着喝光了两杯酒,周南却变得更加紧张。接下来就是今天的正戏了。在这方面仍是一张白纸的花魁,说出去也许还有人不信,但只从听过一点理论知识的周南,被韩冈一下搂着腰肢,顿时手足无措。

“娘子,还是早点歇息吧。”

让人梦寐以求的绝色佳丽就在怀抱之中,韩冈有些迫不及待。红烛高燃,烛花噼啪响了两声,见着韩冈毫不客气的搂起周南,墨文红着小脸,低头慌慌张张的退到了外间去。

周南要起身帮韩冈脱衣。韩冈却阻止了她,又亲了她小嘴一口,在她耳边轻笑着:“今夜就让夫君来服侍你。”

少女越发的紧张,重新坐下来的身子绷得更紧。却没有反对,闭上眼睛,浓睫微微颤动,就任凭韩冈为自己宽衣解带。

闭紧双眼的黑暗中,其他几种感觉却分外明晰起来。酒气带着浓烈的男性气息就在身前传来,一对的大手在腰间摩挲着。耳畔越发沉重的呼吸声,让周南浑身都热了起来,脸颊、胸口还有那最私密的地方,都热得发烫,仿佛要融化一般。

“官人……”韩冈粗重的动作下,娇躯不住的轻颤,周南细声呢喃着,一遍遍叫着韩冈。

韩冈一声声的答应着,一点点的将绝色佳人毫无瑕疵的完美身躯,展现在自己的眼前。

