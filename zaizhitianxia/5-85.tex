\section{第十章 却惭横刀问戎昭(17)}

战争已经开始了。

当丰州外围的蕃部受到不明身份的敌军攻击的时候,韩冈知道,战火已经重新燃起。

三天内三个村寨被毁,其中两个村寨被毁,来犯者将两个村子烧杀一空,能在杀戮中及时逃脱最后残存下来的,十中只有一二。据前去查探的官员回报,两座村庄几乎是鸡犬不留的情况。能吃的、能拿的,都不见了,男子被杀,女子被奸.淫,村中处处可见尸骸。

连续两个村子被毁,还有一次是攻击不克,主动放弃。而且全都是选择在夜间进攻。这样的敌人摆明了就是来骚扰地方,散布恐慌。

“不敢正面示人,足见其虚弱的,连妇孺也不放过,足可见其暴虐。但更为重要的,似乎他们不想让人发现他们的底细。”

韩冈看了黄裳一眼。他还是经验少,有些东西都已经看出来,说话却没有说到点子上,不过能发现两处惨案的犯人在掩饰自己的身份,也算是有几分见识。

“但做得如此干净利落,积年惯匪也很难做到,这样的情况在边地很是少见。”折可适在旁接话,为了此事,折可适成了派来报信的铺兵,到了太原,便被领到韩冈的面前,“……末将的叔祖亲自去看了之后回来,就说绝不是西贼做的,逃出来的人有好几个都说,那群贼子攻入村子后,互相之间说的不像党项话。”

正是如此。韩冈赞许的轻轻点头。

居住在丰州外围山间的蕃部,全都是党项部族。几十年来,西夏攻过来的次数没有十次也有八次,可从来没有说斩尽杀绝的。上溯几代便能联上宗,同为党项部族,纵使敌对,互相之间总会留着一点香火情,不会把事情做绝。就像折家将统麟府军出征西夏,也几乎不会劫掠当地的党项部族,而对于横山蕃,则下手极重。

战阵上厮杀倒也罢了,赢了之后,抢钱抢粮甚至强抢女人回去,同样也是正常,但毫不犹豫的屠村,鸡犬不留,这样的恶性.事件,却几乎没有出现过。不过其中的关节,折可适还是没敢明说出来,只能含含糊糊的说一句,否则真相给传扬开去,对折家只会带来麻烦。

“那依令尊的看法,究竟是谁下得手?”黄裳问折可适,“是契丹吗?”

韩冈暗自摇了摇头。虽然是明摆着的事,会攻击大宋的,除了西夏,就只剩契丹。但韩冈总觉得这个答案太过理所当然,总有哪里不对。

之前在代州的时候,辽人也同样攻破了两个村寨,下手也十分狠辣,但同样的斩尽杀绝之间,风格还是有差别的。就像两名连环杀人狂,除非是刻意模仿,否则都会有自己的风格。

仿佛听出韩冈心声中的否定,折可适回黄裳道,“也不像是契丹人,在村中发现的箭矢全都是西夏的样式。而且据可适所知,自从官军开始大规模给战马镶马掌之后,不论西夏还是契丹,这两年给战马镶马掌的也越来越多,但从这一次的贼寇留下的蹄印上,却没有发现一匹镶了马掌的。”

折可适说着,偷眼关注着韩冈的神色。如今世上有说法,给战马镶马掌也是他面前的这位经略相公的发明,好让本来马匹数量就不够的官军,不用担心战马因为马蹄磨损而无法派上用场。但将近来的一系列发明都附会到韩冈身上,最近越来越多见,甚至连神臂弓不知什么时候起,都成了他的功劳,反倒让折可适对此心中存疑。

韩冈对这个情报细细思量。很有用的一个情报,解释了缠绕在他心头上的一个疑问,大体上是确定了之前袭击丰州、麟州的那一支神秘军队的身份,同时也算是对耶律乙辛打算使用的手段有了点眉目。

抬起眼,却听到黄裳叹了一声:“可惜因为是夜袭,逃出来的人都没看清楚服饰。府州的骑兵,又没能追上那伙贼寇。否则现在早就能查明了他们的身份。”

折可适的脸板了起来。不过这一次的事,的确是让折家丢了大脸。

云中丰、麟、府三州之地,乃是折家的根本地盘,突然受到身份不明的对手的攻击,就像被一脚踩了尾巴的老虎,当即便上下动员了起来,派出了精兵去追击,也派了见识广博的官员去当地探视,但对于来犯敌军身份的探查,直到在折可适动身前来太原之前,都没有什么进展。

“本来为了此事,末将家的九叔领军去追查了,但那群贼寇却往沙漠中逃走,突然间便不见了踪影,甚至没有留下什么痕迹。”

折可适感觉那群贼人就像一抹晨雾,消散在日出之后。否则那么多人的追踪,怎么会一点踪迹都追查到?

“并不是突然间不见,而是先退回沙漠,然后绕了个圈南下。麟州的连谷县外,就在两天前有一个村子被夜袭攻破,村里的情况与丰州相同,当是同一伙贼寇所为。”韩冈笑了笑,笑容中却没有一丝温度,“麟州的马递走得快,比遵正你早一步到太原。”

“麟州?!”折可适失声。之前还往沙漠里逃去,现在就又到了麟州,追在他们身后的自家精锐肯定是被他们在半路上甩掉了,“跑得还真是快!”

“以遵正【折可适字】你的看法,这伙贼寇的规模有多大?下一个目标会是哪里?”韩冈问道。

“规模很好推算,战马千匹,但人数只在三百到四百人之间,跟契丹骑兵一人三马的情况很相像,而如今的铁鹞子,因为连年向辽国进贡,已经只能勉强配起一人双马了。只是他们下一个目标……”折可适苦恼的摇摇头,“这还真是猜不到,是牵制河东兵马,还是想引开经略的注意力,都是有可能的,也都能解释得通。不过……”

“不过什么?”韩冈追问。

折可适迟疑着,吞吞吐吐的说道:“只是末将觉得,关于这伙贼寇的来历,其实还有一种可能,能够说得通。”

“遵正也觉得他们可能是阻卜人?”韩冈漫不经意的问道。

“经略你……”折可适这一次真的惊得跳了起来,瞠目结舌,“怎么……”

韩冈微微一笑,示意折可适做下,“这还要多谢遵正你,要不是你说贼人用了西夏的箭矢,却没有为马蹄镶上蹄铁,我也想不到这一伙贼寇,竟然可能是草原上的阻卜人。”

太原府经略安抚使司衙门里的白虎节堂中,有着河东路最为精细的沙盘和舆图。河东周边势力在上面都有标注,其中也没有漏下西北方的阻卜。更重要的是京城中关于怎么对付辽人,由天子主持的军棋推演已经进行过不知多少次,而动摇辽国的根基,从辽国的外藩身上入手就是最简单有效的手段,高丽、女直、阻卜是最常出现的字眼。

折家直面辽国和西夏,心神全被这两个庞然大物所侵占,很难让他们的思维发散出去,折可适能够大胆猜测,已经算是难能可贵了。

而韩冈这般从京城出来的显宦,多次在武英殿上参观天子的军棋推演。当他得到了折可适带来的详细情报,就像是被拼上了最后一块碎片的拼图,却不用太多猜测,就推断出贼人最有可能的身份来。

“拥有西夏的箭矢,却没有西夏、契丹已经流传开来的蹄铁。杀戮劫掠的风格与契丹人迥异。还喜欢用夜袭。这是我们所知道的这群贼寇的几条特征。”韩冈向折可适和黄裳解释自己的思路,“由此来推断一下。他们所拥有的西夏箭矢,肯定来自于西贼的武库。没有蹄铁,那就不会是铁鹞子、皮室军,而他们在村寨中犯下的罪行,也确认了这一点。”

“至于夜袭,可以有各种各样的解释,隐瞒身份是一桩,为了减少伤亡同样也是一桩,没办法确定。将之丢到一边。只考虑前面几点,那么答案就很简单了。临近西夏的部族,又有足够的实力帮助西夏,就只有阻卜。遵正,我说得可有错?”

“经略说得正是。”折可适不知道韩冈一言即中的缘由,投向他的视线里平添了几分敬畏:“阻卜人受契丹所困,铁器绝少,箭矢甚至多用骨箭。出兵协助西贼,肯定从西贼武库中得到了许多兵器,故而箭矢皆是出自西贼。此外,他们援助西贼,必是先得到了契丹的命令,至少是首肯。没有西贼提供的的好处,阻卜不会赤膊上阵,没有契丹……耶律乙辛的允许,他们也不敢南下援助西夏。”

“嗯,没错。”韩冈点点头,“正是这个道理。”

“以末将之见,阻卜南下的兵力当不会少,太少了,未免就太丢大辽尚父的脸。”折可适语带讽刺,“不过太多也不至于,一来西贼养不起,二来阻卜人也没那么大的实力。”

韩冈完全认同折可适的判断,将种的绰号并不是白叫的。他沉吟着:“由此看来,阻卜至少五千,应当不会过万。”

