\section{第38章 逆旅徐行雪未休(三)}

【今天第二更,今年倒数第二更,红票还是不够多啊】

章俞一愣,看着韩冈扯着刘仲武要上马离开的样子不似作伪,连忙叫道:“两位恩公且慢一步,还请留下姓名。小儿亦在京中为官,两位恩公若至京师,老朽也可让小儿一酬救命之德!”

“施恩望报岂是君子所为,老员外有心了,却是不必!韩某告辞!”韩冈拱了拱手,十分洒脱的一跃上马。哈哈笑着,带着犹有些发懵的刘仲武三人,转眼便去得远了。

章俞望着韩冈渐渐小去的背影,悠然神往,为韩冈的洒脱和豪爽深深的感叹着:“事了拂衣去,深藏身与名。此子大有古人之风啊。”回头一看百无一用的仆人们,气便不打一处来,大骂道:“还愣着作甚?追上去啊!人家是要入京的,正好一路去!快!快啊!”

“为什么?”刘仲武很奇怪韩冈的举动,骑在马上,靠过来问着韩冈,“我们救了他的命啊,难道当不起他的谢?”

寒风刮着脸,直往衣服里灌,天色越发的阴沉起来,星星点点的雪屑如飞絮在空中飘荡,真的要下雪了。

将速度放低,韩冈侧着头,对着刘仲武喊道:“时间不早了,还是早点进城去,何必再耽搁?谢礼什么都是假的,早点上京,挣到官身才是真的。”

刘仲武皱着眉头,心中有些不快。章俞看起来便是个有身份的,听他最后还说有个儿子在京师做官,虽不至大小,好歹也是个官。能结好章俞,也不枉自己一番辛苦。但韩冈强拉着自己骑马离开,现在也不好回去了。可惜啊,可惜了一个好机会。刘仲武的神色变得冷峻起来:‘莫不是怕自己结交了有用的助力,真的得到官身不成?’

路明腆着脸靠过来:“刘兄,其实韩官人做得不差。这章俞并不是什么好路数。离着远点也是好的。”

路明说完便闭起了嘴,卖起了关子,等着刘仲武追问。可刘仲武从来都看不起路明,又亲眼看着他一个劲的巴结韩冈,哪会信他的话,根本问都不问。而另一边的韩冈,更一副毫无兴趣的样子。天色已经不早,他可不想因为听着八卦,而在京兆府城外过夜。城中有驿馆,有饭菜,还有上元夜的灯会。只要路明还在,八卦随时都能听到,没必要在这里浪费时间。

不过韩冈看透了刘仲武心中的不痛快,他这么做,也有很大一部分原因就是要引起刘仲武的不满。他突然没头没脑的说道:“子文兄,到了明天你就会谢我的。”

在刘仲武的一头雾水中,韩冈抖了一下缰绳,当先冲出。如果他没料错,刘仲武明天肯定会感激自己。即便自己猜错了,方才没头没脑的一句,还有其他的解释可以敷衍过去。为了拉拢这位向宝也看好的人才,韩冈把突发事件都利用了起来,虽然成功几率不低,但脑中不断转着算计人的主意,着实有些累人。

……………………

入夜时分,小雪细如棉,从天空中洋洋而落,京兆府的城墙,也终于地平线下升起。

京兆府不愧是关中的中心,尽管远远比不上隋唐时代的‘百千家似图棋局,十二街如种菜畦’的长安,可已经远远超过秦州城的繁荣。距着城池还有四五里的样子,官道两边,便是一间间的店铺。离着道路稍远点的地方,民居鳞次栉比。

隋唐时的长安,是当时世界排名第一的巨城,规划、人口、商业,与城市有关的各个方面,无不是独占鳌头。只是经过了数百年的沧桑巨变,长安历经战火硝烟,吐蕃人在其中三进三出,终于在朱温的一场大火中,化为瓦砾。而北宋的京兆府,便是建筑在这样的一座城池上。

时值上元,城墙上的灯火,如灿烂的银河,比之韩冈当日在甘谷城下看到的那一条尤要绚烂上千百倍。一朵朵烟花不时的自城头升上天空,在夜空中绽放。无数灯火汇聚,将低沉的云层映成了红色,自韩冈来到这个时代,还是第一次看见这样的景色。

毕竟是上元之夜。

人如潮涌,为了观灯,往往都是一大家子同时出游,小孩子手上提着各色的小灯笼,兴高采烈地走在前面,父母兄姊则跟在身后。韩冈一行入城之后,便在人潮中艰难跋涉。周围人头涌涌,幸亏有了路明这匹识途老马,才没有在人海中迷失方向。

上元节是一年中的大日子,甚至可以说是北宋的狂欢之夜。元旦正日,人们都是在家中与家人团员。立春则是与农事息息相关的祭典。而上元节,便是以居住于城池内外的市民——此时称之为坊廓户——为主力的节庆。东京城要放灯五日,而寻常军州,也要放灯三天。

一座座由彩灯组成的灯山、灯棚矗立在街市中,金碧相射,锦绣交辉。这些都是城中各家行会、富户豪商所制,互相之间还要较量个高下。

雪停了,可风未停。积在屋顶和树枝上的雪粉,随风而起。稀疏而又轻柔的雪意,并不会打扰到人们的兴致。灯光在雪雾中散射,空气中都闪着柔柔的黄光,宛如梦幻一般。

走在流光溢彩的街巷中,韩冈突然想起一事,都是急着进城,他倒忘了一件事。长安不是秦州,平日里并没有宵禁,而在上元之夜,更是夜间也不闭城门,他本不用赶得这么辛苦。不过这样也好,不用等到明天,今天晚上,现在板着脸的刘仲武心情就能变好。

刘仲武这时候却好像忘记了心中的不快,饶有兴致的看着周围的花灯街市,原本板着的脸上浮起了一丝笑意。秦州地处边境,平时便便不如京兆府繁华,节庆时更是远不如京兆府热闹,他也不禁看得入迷。

不同于刘仲武,还有已经看花了眼的李小六和路明。韩冈眼望四周,却有一股茕茕孓立的淡漠涌上心头。

喧闹的街市,欢腾的人群,孩子们天真的笑容,无不在述说着此地的和平幸福。虽然有苦役,虽然有交不完的税,但毕竟是听不到战火硝烟的和平之地。

大宋立国百年,尽管时有动荡,边境更是没少过战乱,但国家内部还是保持着大体的和平。对生活在熙宁年间的内地百姓们来说,也许很平凡,可在晚唐、五代的数百年间,却是难得一见的幸福时光。

只不过,在五六十年后……也许是四五十年后,眼前的太平年景,就会因为两个蠢皇帝和几个奸臣,而在来自北方的铁蹄下,被踩得粉碎。

第一次……穿越以来的第一次,韩冈思考着他来到这个世界的意义。

记得前些日子闲暇时读得《李太白文集》,诗句读过便罢,但其中的一段序文却让韩冈铭记甚深:夫天地者,万物之逆旅也;光阴者,百代之过客也。

逆旅……韩冈觉得这个词实在很好,用来形容他再合适不过,在时光中逆流而上的旅客。只是他不再是过客,而是已经定居下来。

他能为这个时代做些什么?

是更为富足,更为安定的生活?还是——对了,他的老师有一句话——为万世开太平呢!?

应该能做到罢!否则到这里走一趟,又是何苦?

不知什么时候,又开始下雪了。但这场雪并不算大,风则变得更弱,雪片就如柳絮杨花,飘飘荡荡的从铅色的天空中落下。

韩冈抬眼远望,举目茫茫,视野只及十数丈之远。可今早在驿站里看得黄历,却是明明白白的写着宜出行。

宜出行吗?韩冈哈哈大笑,真是好黄历。

笑声里,他用力一抖缰绳。马身一动,在漫天的雪花中,向着驿站行去。

……………………

京兆府的驿馆,远远胜过韩冈这几天来【和谐和谐】经过的诸多驿站。不但编制上有一名官员直接主管,在建筑更是楼台园囿皆备,单是门厅就仿佛一座酒楼,或者说就是一座三层高的酒楼,只不过接待的是来往陕西的官员罢了。

正是节庆之时,厅中的桌子已经被占了大半。韩冈这样的还没拿到告身的从九品,在厅中诸多官人中,一点也不起眼。验过驿券,韩冈在偏院弄到了三间厢房,放下行李,留下李小六看守,同着刘仲武、路明又回到大厅中。

照着低品官员的待遇标准,在驿馆中充当小二的驿卒为韩冈三人端来了一桌子的酒菜。韩冈尝了一下,酒菜皆是上品,不愧是京兆府。就是他们坐得位置不算好,三楼他还不够资格,而二楼的靠窗,能看到灯火的座位,一个个都早早的被人占了,只能找了个近着楼梯口的角落坐下。

韩冈的邻桌贴着窗子,坐了三人。身侧靠着窗的两人,一个四十多岁的中年,一个才二十出头,都是武人模样,身材健壮。单是坐着,便像是两山对峙。剩下的一个打横相陪,显示地位最低。他面朝外,背对着韩冈他们,只看他的背影,也是一个体格雄壮的汉子,却穿了儒生的装束。

韩冈只瞥了他们一眼,便收回了目光,与着刘仲武和路明一起拿起筷子、填着肚子。

