\section{第39章 遥观方城青霞举(六)}

【昨天有事没能更新,今夜三更,明天再三更。第一更。】

政事堂。

每天结束了崇政殿议事,回到政事堂后,就是宰执们在正堂定例举行合议的时间。

在合议上,宰相和参知政事,都会就今天要处理的几桩大事商量一下,就算党派、政见都不相同,也会互相通个气,省得闹得太过难看。

除非有什么无法调和的矛盾,必须分出个你死我活,否则都会尽量在合议上解决,就是当年王安石和冯京、王珪都在政事堂中的时候,也没有说天天争得面红耳赤。

入秋之后,在合议上没了什么大事要讨论——真正有关天下大局的几桩事,都要跟西面枢密府讨论过之后,才能做出决定——也就是一年一度的秋税能惹起宰执们的注意。

可能当真是改了年号的缘故,靠着一个好口才,元丰元年的天下诸路,竟然绝大多数都取得了丰收。夏粮早早的完税,秋税的情况也是十分的喜人。青苗贷、免行钱,还有市易司的出息,都让几名宰执松下了一口气。

不过除了秋税以外,还有一桩事让东府中的宰相、参政牵肠挂肚。

吕惠卿拿着天子批下来的一份奏折,摇摇头,向着王珪、元绛扬了一扬:“韩冈倒是自信,要在冬月京畿水道封冻之前,将六十万石纲粮运到京城。天子都给他撺掇的一头劲。昨天奏折直送御览,都不在崇政殿中问上一句,就直接批了下来。”

王珪笑道:“韩冈为人稳重,说得出来,多半是能做到的。天子也是因此才信他。厚之,你说是不是?”

元绛在几位宰执中年纪最长,但他在政事堂中的时间却是最短的,对韩冈不算了解,也不往深里说,“天子既然批了,我等副署就是了。做成了,自有封赏,做不成,少不了一个欺君的罪名。想那么多在做什么?”

“也不是不信韩冈,我也知道他素来是言出必践的。只是想不到有什么办法能将六十万石纲粮运进京来。只是想猜猜韩玉昆这一次想用什么手段。”吕惠卿很是有些好奇的模样,“板甲、飞船,可都是让人怎么都追不上的奇思妙想,不动声色的就给他做出来了。这一次,不知他又打算给人带来什么惊喜。”

“才六十万石,应该不难吧?”王珪看神色是有几分疑惑,“从汴水运上京城的,可是六百万石。”

元绛捋着保养得极好的胡须,慢悠悠的说道:“延行百年的六百万石,和初来乍到的六十万石,肯定是后者更难上一筹。无人手、无规程、无故事,一切从头做起,全都要韩冈来创立。换作是在下来做,就绝不敢在成事之前,先在天子面前下军令状的。”

“厚之说得正是。”吕惠卿对元绛笑着点点头,转头就对王珪道:“相公有所不知,六十万石纲粮哪有这么容易运抵京城?就像东南六路的纲粮必须在扬州换用纲船一样。沿着汉水将秋粮运抵襄州的船只,大小形制各不相同,要在襄州换了一色七百石的纲船才方便北上至方城山下。”

他一声嗤笑:“当真以为有了轨道、水道就能见功了?搬运纲粮,需要大量的人力。一名寻常的苦力一次最多也就能扛上两百斤,扬州单是力工就有三千多人,这样才能在九个月中,将六百万石的纲粮送抵京城。算算六十万石要多少人次的搬运工,就知道绝不是修好了轨道、打通了水道就能成事。”

王珪皱着眉,不是为韩冈,而是为吕惠卿的态度。论起做事,他的确不能算是行家里手,但说起如何取得天子的信任,自己是不会输给任何人。要不然也不会自己坐上了宰相的位置。

见王珪无法回答,吕惠卿微微一笑:“一个月之内,要把六十万石在三个不同的港口搬上船、运下船,而且还是刚刚扩张和新建的港口。就是以财计和转运之术,闻名国中的薛向来了,都不能拍着胸脯说自己一点问题都没有。”

他问着两位同僚,“缺乏足够的人手,韩冈总不能调来厢军、或是征发民夫做力工——码头上的搬运工作也是有技巧的,不是说有把子蛮力就能安安稳稳的将米袋送到船上,保不准就有个几万斤连人一起掉到水里去——韩冈他到底想怎么做?”

“看起来吉甫你还是觉得韩冈做不到……”王珪笑着,

“只是觉得自己的做不到。而韩玉昆他多半……”吕惠卿斟酌一下言辞:“以其之才,当是肯定能做到。我只是想知道他是打算怎么做的罢了。”

王珪和元绛沉吟着,吕惠卿的一番话,也让他们升起了好奇,韩冈到底打算怎么赶在时限前,亲手向天子证明他说出来的话的正确。

……………………

韩冈自然不会去满足宰执们的好奇,他也没那个义务,但只要人到了襄州城外的港口上,就自然能明白韩冈转运纲粮的手段。

襄阳城还没有后世的护守一国数十年的地位,更不用说让一支能远征数万里的军队,多次无功而返的坚实城防。

不过眼下的襄州,还照样有着以宽阔闻名天下的护城河,只看阔达百步的河面,就知道想要攻下这座城市,究竟有多么困难。

来到襄州的商人为数众多,都是听说了襄汉漕渠的计划,赶过来打算亲眼瞧上一瞧,看看这条通道到底能在多大程度上代替汴河水道,让他们也能有个应对的计划。

方兴的视线,又回到了前方高大宽阔的背影上。他的恩主现在正抬头看着这一次为了渡过难关而打造的杰作。

纲粮转运比起筑路、开渠并不算难,但两边若都是还没有上手,多少聪明人都宁可去屈居下沉,也不肯去接手转运一事。

港口、码头只是一个方面,只是为了实现目标做出的准备而已,合用的工人才是主力。襄州的旧港中的人力不敷使用,而北面的两座新港,同样是新近落成,就只有几个主事者,下面合用的人手一个都没有,更不用说港口中永远都不能缺少的搬运工。

一开始时,沈括、方兴和李诫都在为即将完工的轨道和港口感到高兴,但当他们从幻想中脱身而出,一想到即将面临的严峻考验,脸上都变了颜色。

这时候,就显出韩冈的能力来了。

困难如拦路虎,横亘在眼前,绕过去和跨过去都是得小心翼翼,更不用说直面相对了。但韩冈对此有足够的自信,所以就有了现在让出现襄州城外的港口中,让来往于港中的人们都要仰头观看的新奇事物。

高达三丈的门式吊车,横跨在码头上。从横梁上垂下来的钩索能轻而易举的将粮食吊起,通过绞盘来牵动吊索,送到另外一艘船上。

“李明仲【李诫】的确是有一手!”韩冈仰着头,望着高高耸立在港口中的庞然巨.物,由衷的赞叹着。

“龙图当是首功。”方兴跟在他的身后,“若非龙图给了图样,谅李诫也造不出龙门吊来。”

“不。”韩冈并不居功,摇摇头,“一支笔、一张纸而已,算不上功劳。没头没脑的图样,也亏李明仲能看得懂。这些都是他的心血,我这里费点墨水、一点口水,可没脸去占他的便宜。”

韩冈和方兴两人都盯着那一艘显得很是破败的船只,拿着这艘船来运送粮食,甚至可以说是冒险。不过眼下的情况,让他们还算是很满意。

绞盘转动,刚刚放下货物的龙门吊就又转了回去。先是三百斤,接着又是三百斤,船上的水手将粮袋整理好,放在吊索上。龙门吊总是很顺利的将粮食运抵不远处的纲船之上,每一次吊运都是顺利的让参观者连连点头。

木结构的龙门吊,可以直接从船上吊起三五百斤的货物,移到岸边的有轨马车上,也能移到另外一条船上。这般巨大的龙门吊耗费的木料,用来盖座同样高度的酒楼都够了。

“最髙能运多少货物。”韩冈问着方兴,“有没有让人试过?”

“回龙图的话,下官这两日已经让人试验过了,最大试过吊运六七百斤,再重绳子就不一定能吃得消了。不过还是三百到四百斤最稳当。”

韩冈满意的点头,方兴的老成持重,他素来是知道了。能先通过多次测试,了解到一些必须了解的关键性的数据,这就为了将来主持发运之事,事先能有所准备。

方兴向韩冈细细解释:“只从船上运到船上,只要两条船并在一起,将舱中的货物运过去并不算难。但利用龙门吊,难度则是更低了一层。一个时辰的时间,一船粮食就能发出去了。六十万石看着多,其实根本不费什么事。有一个多月的时间,怎么也能全都运过去。”

“说的好。”韩冈的头一直点着,将此事交给方兴的确没有错,“等李明仲到了,你就跟他说说怎么打算。”

“还得要龙图提点。”方兴恭声道。

正说话间,一艘小船顺水直下,很快就抵达了港口。船头上的旗帜让人一看便知这是一艘官船。不用通报,韩冈和方兴就知道李诫已经到了。

