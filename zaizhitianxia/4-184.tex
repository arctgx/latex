\section{第30章 狂潮渐起何可施(上)}

【年终聚餐,弄得太晚了,也只有一更了。】

踏着官道上薄薄的冰雪,就在熙宁十年还有几天就要结束的时候,韩冈一行人马正向着东面的汴梁城前进。

舒亶当然不在队伍中。他在第二天,便跟韩冈告辞,继续去前进,去查他的案子。不知道他最后究竟会怎么打算,但韩冈则是想了一想之后,便抛诸脑后。

通往东京城的官道,就在蔡河这条京西主要水道左近修建。韩冈一行人一路走过来,离着蔡河大堤最远也没超过半里路。而且蔡河对韩冈来说很是重要,如果襄汉漕渠修通,也是连通到蔡河,然后方才驶入京中。

也因如此,当韩冈走在蔡河边的时候,他的注意力多次被身边的水道所吸引。蔡河上最让人吃惊的状况,是完全不见雪橇车的存在。

蔡河之中,流淌着活水,冰层想要变得厚实可以通行车辆难度便大了许多。韩冈的发明虽然对冰层的要求不算很高,但眼下的蔡河依然是远远不够。

今年中原冬天暖,韩冈和他的随行伴当们刚刚从一年到头只见春夏的广西回来,分辨不出这一点,但方兴任官数载,基本上都是在中原打转,却是很敏锐的感到今年与往常年份的差异。

“今年天气暖啊。”方兴满载着遗憾,“都不见雪橇车了。”

“不过汴河应该是上冻了。”韩冈笑说着。每年冬月开始,连通黄河的汴口一封,自宿州符离以北,这条运河之中,就见不到活水了,一点底水很容易就冻结起来。在冬天,继续作为沟通江南和汴京的重要通道而发挥作用。

观察蔡河只是顺便,正经事还是早点抵达京城。一行二十多人,在宽阔的官道上疾驰,过年的前几天,道上的行人锐减,正好利于他们快速前进。

就这么小步快跑的有了小半个时辰,方兴突然讶异的叫了一声,前进的速度一下就慢了下来。

就在方兴望过去的方向,浑浊的黄色烟柱直上云天。身处平原,周围没有任何山峦或是建筑物的遮挡,即便烟柱腾起的位置几乎就在地平线上,韩冈一行也是清晰地看到这个奇异的景色。

“是哪边走水了?”方兴惊讶的望着浓烟腾起的地方,从烟柱的浓度和高度来看,火势不可能会小,所不定那里连村镇都一并给烧掉了。

“不,那是炼铁、炼焦产生出来的烟雾。”房屋楼宇烧起来,不会是这个颜色的烟。更重要的是哪个方位上是什么地方,曾经担任过判军器监的韩冈再清楚不过,“那里是军器监辖下的铁冶,有炼铁炉和炼焦炉的。”

蔡河在接近东京城的那一段,北面不远就是汴水,那里不仅仅是有着前出城外的锻造作坊,也是有一座实验性的铁冶作坊,当初实验修建高炉,就是在那个方向。

“原来如此。”方兴点着头。

韩冈执掌军器监虽然是在他任官外放之后,但韩冈在军器监中的历历功勋,他这位身上已经打了韩字印记的选人,不可能不去了解。以高炉焦炭炼铁,尽管在飞船和板甲的炫目光芒掩盖下,显得并不如何惹人注目,不过只要对政事稍有了解,就会知道究竟是孰轻孰重。

“想不到区区一座铁冶,竟有这么大的阵仗。乍看上去,当真是以为哪座镇子全都烧了个精光。”

“一座炼铁炉,一年产铁量至少有百万斤,声势当然不会小。”韩冈笑着解释。

方兴倒抽一口凉气,“百万斤?!”

“如果送来的矿石和和煤炭没有限制的话,其实数量会更多。”韩冈态度宁宁定定、说话气定神闲,可心中对方兴的惊讶很是满意。

尽管滚滚浓烟会带来各式各样的问题,水源、土地,甚至当地人民的健康,都会在不同程度上受到损害。但这是工业化的标志,是让人兴奋的技术进步的结果。相对于造成的损害,其得到的好处是远远过之。至于污染不污染的事,只能先放在一边了。

方兴在马上眺望着远方,凝固在脸上的惊讶半天也不见变化,“一年百万斤,想不到京城中的铁冶都能有这个数目。”

韩冈笑了笑,心中更是有了几分自豪,“这只是军器监消耗掉的铁料的一小部分而已。光是徐州每年通过五丈河运往京城的生铁就超过五百万斤。”

“这么多?!”方兴心中的惊讶更甚,几乎都要叫了起来。

这些年朝廷一年的铁课数量也不过是五百多万斤,这是全国各地的数字——自从熙宁以来,矿区的生铁冶炼已经逐步转给冶户私人生产,朝廷从中以二八抽分,以两成的税率作为铁课——现在竟然是徐州利国一监,就达到了过去全国的数目。

“多?”韩冈略带不屑的摇头,数目听着的确挺大,可换个单位,也就是两三千吨而已。如果是千年之后,连个村办铁厂的年产量,都是这个数字的几倍甚至几十倍。

韩冈在军器监中不过一年多,但他留给军器监的财富却是丰厚无比,在他卸任之后,他之前所安排下的各项研究和实验,有许多得到了回报,尤其是钢铁,其产量有个爆发性的增长。

韩冈还在广西的时候,时不时的从他留在军器监的几个家人中,收到关于钢铁工业的发展情况。

短短两三年间,朝廷收入的生铁数量就翻了一倍,而且增长趋势并未减缓,压下了所有的质疑,还有对技术外流的攻击。与此同时,也就是韩冈以国力上的优势来压制西北二虏的言论,在朝堂上渐渐响亮起来

——话说回来,用财富击败对手的方略,并不是韩冈的发明,当初太祖赵匡胤曾对臣子们说,契丹堪战之兵不过二十万,一人悬赏十匹绢的话,只要两百万匹就足够,但这样的想法在太宗赵光义的高粱河之败后便宣告破灭,而到了澶渊之盟时,更是变成了花钱买平安。

“当初,使北之人回来后,说到辽主也造了一艘飞船用以游猎,当时还有人弹劾于我,说我求名心切,使军国之器沦于敌手。”韩冈将自己了解到的军器监的现状说了几句后,就笑道,“但现在看到钢铁产量,就没人再说了。对付西北二虏就是得靠足够的钢铁。一千万斤不够,那就两千万斤;两千万斤不够,那就五千万斤;五千万斤不够,那就一万万斤,五万万斤,十万万斤,终能压倒二虏。”

“十万万斤……”方兴忽而笑叹,“给天下人一人配发一件甲胄都够了。”

方兴对于生铁年产量达到十万万斤难以置信,看他的神色也只是以为是韩冈一种夸张的修辞手法。

但韩冈却不是这般想的,他一点也没有夸张。以大宋的人口规模,一旦进入工业化的进程,五六十万吨的钢铁年产量,其实算不上多——自然,这是要经过几十年的顺利发展才有可能实现的梦想。而眼下,如果仅仅是三五千万斤,则是很容易就达到的数目。

对旧型炼铁炉的改进工作,在韩冈离开的两年里并没有停顿。由旧时只有两丈不到的炉体,变成三五丈髙的庞然大物,一次产铁万斤都是等闲。

这样的高炉,在徐州有四座,东京一座——这是韩冈半年前收到的信上所写,如今的情况如何却是不得而知——而河北,因为太过靠近前线,所以封锁了技术,并没有修建高炉。

修造高炉,技术含量不低,加上甚为显眼,还要吞吃大量的矿石、焦炭,所以也只有官办,民间就算想造,没有技术,没有财力,而且一下就能给查出来。

其实这也是为什么徐州利国监的生铁产量如此突飞猛进的原因——尽管其中的确有矿冶几方面技术进步的结果——但更多的还是官办的炼炉不会与冶户二八抽分,而是由官府全数吞吃。

韩冈不喜欢私人的小作坊,他们太脆弱,而且对于钢铁产业技术进步的用处不大,除非他们能联合起来,否则任何一项技术发展,都会将他们推下破产的深渊。

徐州利国监要不是靠着遍布矿区的轨道,让矿石的出产数量已经远远超过了当地冶户的接纳极限,使得官办的高炉不需要争夺矿石,恐怕几万户冶户起来捣毁高炉都不是不可能。

不过出现这个情况也为时不远了,随着全国禁军铁甲装备即将告一段落,军队对钢铁的需要就会下降一个数量级,多余出来的钢铁只会流向民间,韩冈当年安排锻造作坊转产民用铁器的计划,也会大规模的推行。生铁和铁器的价格会立马下降一个台阶,到时候,不知多少冶户会破产。

一行人继续前进,方兴仍在为百万斤、千万斤、万万斤的钢铁产量而惊叹不已,而韩冈则想着当军器监转产民用铁器之后,自己会受到多少弹劾,到时候,一个与民争利的罪名肯定是少不了的。

罪名如何韩冈不在意,他有办法洗脱,但要是有人以此为借口,去破坏钢铁技术的发展,那他就绝对无法容忍。

要想毁灭一个千万人口、带甲百万的大国,不是单纯的比拼军力,而是在较量国力。钢铁业的发展,是工业化的里程碑。一个半工业化,甚至只是初步工业化的大帝国,那就是一个怪物,有钢铁和火药作为车轮的战车,能轻易碾碎任何野蛮的游牧部族。

到时候,所谓的西北二虏,也就在指掌之间。

