\section{第42章 皇祚思无疆(上)}

从天子所居的端诚殿中出来,吕惠卿回到了青城行宫安排下来的住所内。

参知政事在这场大典之中,能做的事不多,重要的工作都是正任宰相来担任。所谓的副相,只有靠边站的份。

如今的大典,许多地方都是参照了《开元礼》,也就是唐明皇时编订的礼仪制度。那个时代,参知政事这个职位就是宰相,地位犹在同中书门下平章事之上。只是到了宋时,才变成平章事的副手。唐时礼仪中当然也不可能留给他一个管事的差遣。

吕惠卿的房间并不大,更没有多少装饰,连房中所用器物的形制,都是以简洁为主。不过青城行宫本来就容纳不了多少人,如今一下涌进了几千官员,能有一个单间已经宰执官的特权了。再到下面的小臣,都是四五人、十几人挤一间房间。而数万士卒,更是只能在行宫外住帐篷。

吕惠卿在圆墩上坐下,从袖口中掏出一册薄薄的书卷来,翻开来细细的看着。桌上摆开了笔墨砚台,吕惠卿时不时的提起笔在纸面上点点画画。

不知过了多久,房门被轻轻敲了两下,吕升卿随即推门走了进来,口中却连道着:“晦气。”

“怎么了?”吕惠卿视线从手上的书卷中离开,看着自己的弟弟。

“方才见到了韩冈。”吕升卿坐下来说着。诚心相邀,而韩冈却一点面子也不给,使得吕惠卿的弟弟对韩冈很有些看法。

“他是右正言,住处自然就在附近。”吕惠卿却是没有什么反应,反而叮嘱着弟弟,“你待会儿回去后也别乱走动,入了夜后,行宫中管束就会严起来。有点差错,少不得会被御史盯上。”

“小弟明白。”吕升卿回了一句,依然愤愤不平,“大哥一片好心,却给他当成了驴肝肺,去了军器监自找苦吃。”

吕惠卿心情则是很平静:“人各有志,出处异趣。韩冈既无意,那也就罢了,岂能强求。”

“他不来也好,省得给手实法添麻烦。”吕升卿坐下来的位置,吕惠卿手上正拿着的一卷手稿,他正好看得清清楚楚。

吕惠卿将他手中的卷册放到了桌上。这一份卷册,就是手实法中各项条例的手稿。大字小字写得密密麻麻,几乎都见不到多少空白的地方。

手实法不同于此前新党推出的其他法案,从筹划到拟定,再到实施,都将由吕惠卿一手主持和操控,与王安石全然无关,是属于他的新法。

要想成就功业,就不能沿袭前人之功。如果他吕惠卿仅仅是‘萧规曹随’——就像韩冈前日说给章惇听的——那么日后人们提起新法来,也只会想到王安石。

提到吕惠卿,则最多一句‘啊,他是有些功劳。’——吕惠卿岂能甘心?!

所以吕惠卿从唐时的旧制上吸取经验,准备将手实法提上台面,令百姓自报田亩及田地等级,据此以征税赋。

“手实法若能成事,乡中隐田必然无处藏身,朝廷财计又可宽上几分。”吕惠卿笑叹了一声,手指点着桌上的条例手稿:“韩冈并非等闲之辈,安置流民数十万而不见其乱,可见他一番治才。如果有他相助,推行手实法起来也能容易上一点。”

吕升卿不服气:“韩冈要置身事外就由他去好了,过去新法推行,他也只是动动嘴皮子,何曾出过力?如今嘴皮子也不指望他动。只要不添乱就行了。”

“不要小看韩冈。”吕惠卿摇了摇头,他不会轻视韩冈。他弟弟与王安石的女婿没怎么接触过,而且嘴巴又硬,不肯承认韩冈的才能。但吕惠卿可是很清楚韩冈的才干不会比自己差到哪里去:“韩冈去军器监,说着萧规曹随,但实际上必定会有所动作。要不然他何必苦求这个职位?其人不可小觑,你可想落到杨绘那般下场?”

“他不是去造船吗?”吕升卿讶异的反问道,“章子厚回来后不是这么说的吗?说韩冈的盘算与船有关……除非韩三骗了他。”

“韩冈不会!”吕惠卿又摇了摇头。他不认为韩冈会骗章惇。尽管韩冈将他的打算说出来,就是为了让章惇转述给自己听,但吕惠卿可以肯定,韩冈不会糊涂到欺骗章惇。

“韩冈可以卖个关子,遮掩一部分事实,但绝不会说谎。章子厚的为人其实甚为偏执,要不然他也不会弃了进士,又去重考一个进士。关系好时的时候能推心置腹——对苏轼便是如此——但若是成了敌人,那也是翻脸不认人的。韩冈若真是骗了章惇,再好的交情都会灰飞烟灭……他当不至于这么蠢。”

“如果韩冈当真准备造船,那就是个天大的笑话了。”吕升卿嘿嘿笑起来,“若韩冈是南方人倒也罢了,他一个关西人,见到的水也就洮河渭水,再加一条黄河。金明池在他眼里,怕就跟海一样。他能造出什么船来?等他下辈子投胎去福建差不多,那时他说不定才会有本事造一条去福建的船。”

吕家是福建大族,亲友之中,做海贸生意的也有不少。福建人往高丽去得多,高丽朝廷中多有林姓者为官。为什么这几年朝廷忽然间跟着高丽关系密切起来,还不是因为朝堂上福建人渐多,朝廷对那个远隔重洋的国家了解日深的缘故。

“高丽……”吕惠卿忽然想起了什么,“为兄也有想过命明州船场打造一条万料巨舟,载使渡海,以震慑高丽王氏。想必他们那时必得西来。只是刚刚任职政事堂,时间仓促,还没有动作。不知道韩冈是不是打着这个……”

吕惠卿话说到一半,却渐渐慢了下来,语气也是越来越疑惑。

“怎么了?有什么不对?”吕升卿连忙问道。

“韩冈曾在天子面前自言传习格物之说,那他在军器监做的事,少不了也是为了推广格物致知的道理。光是造一艘船可是算不得什么大事……”吕惠卿这些天来其实一直都在推测着韩冈的想法和准备使用的手段,但始终没有一个头绪,又皱眉想了一阵,终于放弃了,“算了,只要张载不入京师,他又有何能为?”

吕升卿皱起眉头来:“……张载之学与韩冈所倡导的格物可是有些分别。”他为了给《诗序》作注,翻看了当今不少学派的理论。而且吕惠卿忙于政事,他在经义局中参与的部分,有许多都是吕升卿代为撰写初稿。论起经义理论,他并不弱于吕惠卿多少,“张载在关西多说义理,天人之说也都是本于孟氏,虚空即气也与格物无涉。怎么到了韩冈这边,就完全变了样了。”

其实这个疑惑也在吕惠卿的心中。虽然与张载没怎么打过交道,与张载的弟弟张戬的关系更是恶劣。但程颢还是认识的,在当年程颢尚在三司制置条例司的时候,也有过不少次交谈,儒理也多有提及,格物二字也曾听闻。只是韩冈所说得格物致知,却与程颢的截然不同。

韩冈从学于张载,第一次上京时又求学于程颢。但他所倡导格物致知之说,却既不同于张载,又不同于程颢,这到底是哪里来的?

圣人生而知之?这是胡扯!韩冈没这个本事。

若论聪明,韩冈的确过人一等,却也算不上远胜。

吕惠卿可不会认为二十出头的年轻人,能在大道义理上有何独创的高见,必然有所传承。

难道还能是孙思邈不成?那更是一个笑话了。韩冈死活不肯承认的身份,是不能明着拿出来的。而且孙思邈留下来的医书,吕惠卿也看过,也完全没有谈及格物致知的成分在。尽管隋文帝曾经征召他为国子博士,但孙思邈并没有在儒学上有何成就。将韩冈的道理往孙思邈上靠,也照样不通。

“只能先看着了。”吕惠卿唉的一声,长叹了一口气,他实在是猜不透韩冈的葫芦里到底卖的什么药。也不能因为这点疑惑,而出手干预。

韩绛同时举荐了韩冈担任中书都检正和判军器监两个职位,如果韩冈先行接下中书都检正一职,吕惠卿肯定会全力阻止他接手剩下的一个——韩绛的举荐针对性太强,任何人看了就知道是针对他吕惠卿的行动,自己出手阻止,就算王安石都不能意见。

但眼下韩冈换成了仅仅担任判军器监,而放弃了中书五房检正公事,吕惠卿便不能再向他出手。否则就是在明着与王安石过不去。而章惇也不会坐视。

只是他立刻又微微笑了起来,很是有些自信,他在军器监两年,早已扎稳了根基:“不过不论韩冈想做什么,我肯定是第一个知道的。”

吕升卿点点头,又笑道:“说不定韩冈还是自作聪明,一番盘算,都不能成事,反而是个笑话。”

吕惠卿也为之一笑:“那就不得而知了。”

