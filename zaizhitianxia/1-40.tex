\section{第19章 城门相送辙痕远(上)}

初冬的清晨,微风中都带着冻透血脉的冰寒。屋外的地面上,早早便镀上了一层的薄霜。西面的天空尤是点缀着群星的深蓝,但东方的已经褪去了瑰丽动人的绛紫,而渐渐晕起了漫天的红光。

鸟鸣声声。冬天仍能留在西北的鸟类,多是褐羽白肚的麻雀,在屯有大量粮秣的伏羌城中飞来跳去,叽叽喳喳仿佛在和应城中军营点卯的号角。

待到鸡鸣,两间营房中的民伕们早已起身。他们已不再需要韩冈督促,都自觉的收拾起行装。经由昨日一战,韩冈在民伕心目中威信已著,没人敢在秀才公面前稍显怠慢。因为处理过伤患,有了一点威望的朱中,不知何时已经成了民伕们的头领,当先收拾好行李,走到军官厢房门口。

朱中看着薄薄一扇对开木门,心中有些怯弱。听着里面传出来的声音,好像酒宴还未结束的样子。被自己打扰到,不知会不会惹怒秀才公。朱中害怕受到责难,手举着犹豫不定。但一想到耽误了启程时间,最后还会累及韩冈,方才一咬牙,轻轻敲响了房门。

厢房中的酒水本不多,一开始买的两坛很快就给喝光。后来赵隆又出去找了三坛回来,四人边喝边聊了一夜。此时王厚已经醉得昏头涨脑;王舜臣和赵隆也是半醉半醒;只有韩冈会躲酒,心事又重,看着频频举碗,其实并没有多喝,他熬了一夜,眼瞳倒是越发的幽深起来。

不知屋外已是旭日东升,四人仍是有一句没一句的聊着。听见敲门声,他们一起向门口看去。王舜臣跳起来拉开门,门一开,却见是朱中。

“什么事啊?!”王舜臣不耐烦的问道,血丝密布的双眼不用瞪起已是仿佛透着杀意。

王舜臣在民伕们心目中可是个杀人不眨眼的狠角色,朱中被他横了一眼,身子就是一颤,腿软软的不禁向后倒退了一步。但他一眼瞥到后面的韩冈,还是壮起胆,小心翼翼的提醒着,“秀才公,上路的时候快到了。如果迟了,今天怕是不能在天黑前赶到甘谷城了。”

“说得也是。”韩冈没犹豫半点,站起身向王厚道别。一夜深谈,两人的交情已经好得可以称兄道弟、互称表字了:“处道兄,我们一见如故,本再想与你痛饮数日。只可惜小弟还有军令在身,不能耽搁,只能就此别过。等过几日小弟从甘谷回来,在伏羌,又或是州城,我俩再好好喝上一顿酒。”

王厚愣了一下,酒意顿时不翼而飞。说得好好的,怎么韩冈这么急着走。他急问道:“玉昆,你不去见家严了?!”

韩冈摇摇头,整了整衣裳,抬脚跨出门去:“小弟所受押运之命,定有时限,哪能耽搁片刻。甘谷离伏羌又不算远,往返不过两日,一切等我从甘谷城回来再说!”

见韩冈仍坚持要走,王厚追在他身后,拼命想着理由:“玉昆,你一夜未睡,怎么能现在就上路?”

韩冈大笑:“出门在外,也没那么多讲究,少睡个一两宿也无甚大碍。大不了在车上躺一会儿。”

“玉昆你不是有军情要上报吗?先去了城衙再说!”王厚继续为留下韩冈找着理由。

“不是已经说给处道你听了吗?小弟这里还有一名重伤的民伕,再多加两个比他稍微轻一点的,让他们留下来做个人证,缴获的军械和首级则是物证。请处道兄代小弟出面,哪还有什么问题?难道处道你会贪墨了小弟的功劳不成?”

“当然不会!”王厚猛摇头。

“这不就得了!有处道你帮忙,相信机宜和副城都不会再忽视裴峡安危。既如此,小弟还有什么好担心的?”韩冈淡淡定定的说着。

太轻易到手的东西,没人会去珍惜。如果是经过千辛万苦才得到的物件,即便是一枚贝壳,几片残简,都会有人精心装饰起来慎重收藏。这个道理,对人才来说也是一样。没有三顾茅庐的辛苦,诸葛武侯如何能一入刘备帐下,就能得到破格重用?如果只是喝了一夜的酒,便给招揽过去奔走,如何能把自己卖个好价钱?韩冈并不急着去见王韶,却希望王韶能来见他。

朱中这时拎来装满井水的木桶和手巾,为韩冈准备好了洗漱用具。韩冈道了声谢。拿起手巾沾了寒冰刺骨的井水,用力擦了擦脸,又就着木桶漱了下口。被冰水内外一激,韩冈整个人顿时精神起来。晨曦的微光照在他脸上,只见其人气度温雅,神采内蕴,不见半点疲色。

王厚眉头紧紧皱着,凑到韩冈身边,压低声音道:“甘谷城如今岌岌可危,玉昆你贸然而去,恐有不测啊。”

“人人趋吉避凶,那国事还有人做了吗?”韩冈反问道,一抬头,天边竟然已有几缕狼烟腾起,正应了昨日赵隆之言。他将手巾丢给民伕收拾,神色却丝毫不为所动。

王厚见劝不住韩冈,求助的看着王舜臣和赵隆。两人都摇摇头,他们皆以韩冈马首是瞻,且相信韩冈如此行事必有道理,不会有多余的意见。他们这一摇头,只急得王厚直跺脚,好不容易遇到一个贤才,哪能就这么放跑掉。

“玉昆你先慢点收拾着,愚兄找家严去。”说完,便风一般的跑着走了。

看着王厚消失在营门外的背影,韩冈的脸上露出了一点若有若无的笑意。

……………………

城衙寅宾馆中,早起的王韶穿了一身青布直裰,正在院中转着圈子缓步徐行。次子一夜未归,他也并不担心,派给儿子的两名护卫都有传回消息,说是儿子跟韩秀才饮酒尽欢,秉烛夜谈。

王韶心知,那位韩秀才既然能借势而为,压得都钤辖向家的人赔礼道歉,要将自家自负聪明、但对人心险恶仍了解不深的儿子留住,并不会很难。费点口舌,将儿子骗得来要钱要官,也不是不可能。而正如王韶所预料,他还没在院中转上两圈,王厚就突然跑了进来,直嚷嚷着要荐韩冈为经略司幕僚官。

王韶顺着围墙下踱着步子,头也不回的问着跟在身后、亦步亦趋的儿子:“荐韩秀才为经略司勾当公事?”

“正是!”王厚兴奋地点头说着,“玉昆实是有大才,天文地理,兵事水利,无所不知,无所不晓。尤其对西贼和青唐吐蕃的看法,与大人极其相似。玉昆是张子厚的弟子,大人又曾经为河湟之事与横渠先生议论过,难怪他能将河湟之事说得通通透透。”

“是吗?”王韶面现冷笑,脚步仍然不停。

他的《平戎策》受张载启发的地方的确不少,但开拓河湟的策略并非张载或自己独创,关西有识之士谁人说不出个一二三来?别说受张载教诲甚多的学生,就是向宝、张守约等武将,都是清楚河湟吐蕃对大宋的意义何在。

王厚看不见走在前面的父亲脸上的神色,尤滔滔不绝的向王韶举荐着韩冈:“玉昆为人有气节,有才智,有勇略,昨日在裴峡中以三十余名民伕大破贼寇,斩首三十一,缴获军械近百。如此人才,如何不荐之为官?!以他的功劳,也足够了……”

“等等……”王韶突然停步回头,抬手打断儿子的话,皱着眉:“你说裴峡中有贼寇?!”

王厚点头:“正是!玉昆……”

王韶再一次打断儿子的话头,很着急的追问道:“是西贼还是蕃贼?人数呢?”

“听命于西贼的蕃贼!人数百人以上!”

“斩首和器械都有?”

“孩儿亲眼验过了!玉昆这边也有伤员。”王厚其实都没有看过,但他对韩冈毫无半点怀疑之心,韩冈怎么说,他就怎么信。

“此事当立刻通报给李经略,伏羌城和夕阳镇都得出兵!”王韶说着便要回屋写信,让人紧急送往秦州城。此事非同小可,能出动百名蕃兵,后面至少有一个部族,如果这只是前兆,那就更加危险。秦州通往渭水附近各寨的要道绝不容有失!

王厚在后面忙忙叫道:“爹爹,那玉昆的事?”

王韶回过头来,问道:“还记得为父昨日说的话吗?韩冈心机极深,二哥儿你远远不是他的对手。”

王厚立刻正色回应:“大人误会了,玉昆是正人君子。孩儿想请他来寅宾馆与大人一叙,他却辞以公事。此举岂是小人可为?若是一般人,不待孩儿提,自己就投过来了。”

“是吗?”

听王厚说了这么多,王韶倒是真的打算收韩冈为门下,做自己的臂助了。大宋从来不缺吟诗作对的才子,但有才能,有胆略的人物,却总是少得可怜。只用了一个晚上,就把一贯心高气傲的儿子给慑服了。更加令人惊讶的,是他还能不贪一时之利,而是表现出自己的气节,等待更多的收获。大约才二十出头的韩秀才,绝不是个简单人物,说不定真得有用。

“我会荐举他的,但不是现在。必须压他一压,等他在我门下有了足够的表现再荐举不迟。”王韶笑了一笑,对上太聪明的人就不能顺着他们的意,不然就会被他们牵着鼻子走,“现在说这些也太多了,等他从甘谷城回来再说。”

“韩玉昆现在可是在服衙前役啊!”王厚急叫道。

王韶不在意的说道,“少年人吃点苦是应该的,不会有坏处,二哥儿你就是太顺了。”

“甘谷城如今如此危局,大人你还能眼看着他往死路上走?!”

“不用担心,韩三秀才比你知进退。”

“大人!”王厚猛然提高了嗓门,冲着王韶怒吼起来。

护卫们见王机宜父子相争,都避得远远的,不敢靠近。王韶皱眉看着一向孝顺听话的二儿子,王厚则不甘示弱的与他对视着。能让儿子如此维护,王韶对韩冈的评价高了些许,但感观却又差了许多。挑拨着儿子跟老子争吵,这样的朋友,没有哪个父亲想在儿子身边看到。

王韶沉吟着,儿子对韩冈的偏袒,让他不禁怀疑起裴峡谷之战的真实性和可靠性。一直以来,王韶在几个儿子中最为信任次子王厚的才能和眼光,所以才将他一人带出来,放在身边学着做事,但现在王韶已经无法再向过去那般信任儿子。若是将裴峡谷之事不加确认就急报李师中,最后成了秦州城中的笑料倒也罢了,要是影响到东京城中对他的看法,那样的损失,怎么也难以挽回。

‘到底还是要确认一下。’王韶最终点头道:“好吧,就去见他一见!”

王厚并不清楚王韶这一转念间,对自己的眼光和能力不复往日的信任,只知道父亲终于同意了自己的要求。他转怒为喜,忙着唤护卫过来准备出行,却没发现身后王韶已变得淡漠的神情。

PS:今天第一更,求红票,收藏。

