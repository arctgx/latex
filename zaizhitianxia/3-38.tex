\section{第14章 贡院明月皎(上)}

锁院十余日,终于等到了引试的这一天。

来自贡院东南面的谯楼上的钟鼓声,传进了简陋的房间中。吕惠卿有些艰难的睁开眼睛,头脑依然是昏沉沉的。短短两个时辰的睡眠,完全不足以抵消他这些日子以来所消耗的精力。

这十几天,吕惠卿为了今科的考题,与曾布、邓绾和邓润甫三人争论了许久,直到昨日才将进士科的题目给定下来。三年才得一次的抡才大典,天下都在盯着,谁也不敢轻忽视之。题目的设定,更是关系到方方面面,不但是新党挑选合用人才的关键手段,更是向天下人宣告新党依然稳如泰山的声明。

理由很简单,吕惠卿在被定为同知贡举之前,天子已经向他透露,准备同意此前王安石申请,设立经义局。

原本是因故暂时被搁置的申请,天子现在主动提了出来。虽说可能是为了安抚王安石,但经义局一出,改易旧时注疏,以王学取而代之,从此以后天下的士子皆要以王学为宗。这将会更加牢固的扎稳新党的根基,不至于落到人亡政息的地步。

纯以经义论,吕惠卿的水平要在曾布之上,只要王安石不出头,他吕吉甫兼领经义局是板上钉钉的事情,最多加上王雱。控制着经义局,就是用朝廷的力量来推行自己的学术理论。

天子对王安石的恩信远不如以往,却并不代表他对新党和新法已经感到了厌倦。在吕惠卿看来,情况可能恰恰相反,就是因为赵顼要继续推进新法,才需要排除王安石对新党的影响力。要不然,也不会准备设立经义局——要安抚王安石的手段有很多,没必要用上这一项。

从床上起来,被派来服侍他的老兵送来了梳洗的水盆手巾和青盐。水盆里的水终于是热的了,但还是那般的浑浊,手巾也没有清洗干净。而用手指沾着青盐刷起牙来,吕惠卿就分外怀念起在家中,用着的牙刷、牙粉。

如果是主考官倒也罢了。为曾布做着副手,被锁在临时贡院中超过半个月,做什么都不方便的生活,吕惠卿已经很是腻烦。虽然今天就是进士科引试之日,但要等到解脱,却还有同样长度的一段时间。

进士科礼部试最早,三天后是明法科等诸科考试,再过两天,则是最后的特奏名考试。虽然进士才重头戏,但后面的两场也算是正经出身,吕惠卿监考的任务要持续到六天后。而阅卷的工作,更是要持续到二月下旬。

“还是早点了事吧……”

……………………

韩冈抵达考场的时候,才四更天刚过,天色尚是黑沉,空气更是清寒。不过宋代的礼部试都是一天内结束,所以开场也就会很早,不似明清那般要连着考上三天。

这一方面是考试科目的不同,另一方面,也是因为东京城中尚没有建造正规的贡院。这百年来的多少次考试,不是借用武成王庙,就是占了国子监的地盘。韩冈前世在南京夫子庙参观过的一排排比鸽子笼还要小上一圈的号房,在东京城中是见不到的。

在狗舍猪圈一般的小房间里考试,的确是个悲剧。而且一考三日,吃喝拉撒皆在其中,更是悲剧中的悲剧。韩冈在临时贡院的大门前暗自庆幸。

隔着百来名士兵,望着从国子监的院墙中探出来的一支红杏。被绕着院墙一周的灯火映照着,半开半放的杏花,分外惹人眼。自然而然的,两句七言便脱口而出,“春色满园关不住,一支红杏出墙来。”

慕容武就在韩冈身边,听到韩冈低吟诗句,笑了起来:“国子监中可没有那满园春色,肃杀之气却是重得很。”转又问着:“玉昆,这是你做的诗?”

‘难道这首诗现在还没出现?’韩冈心中一惊,弄不清楚的情况下也不敢冒认,反问道:“思文兄你倒是很安心,一点也不见要考试的样子。”

慕容武抬头远望长空,一副看开了的表情:“成也罢,败也罢。到了这个时候,再想着也是无用。命里有时终须有命里无时也是强求不来。”

韩冈摇头,看起来慕容武大概是已经放弃了。而周围的考生,偶尔也有几个是跟他一样的想法,看开了一切。但大多数都是紧张万分,神色绷得很紧。

当然,充满了自信或是自负的考生,也同样是有的。一个二十出头的年轻人这时从旁边擦身而过,瞥了韩冈一眼就向前走去。举步徐缓,气定神闲的模样给了韩冈很深的印象。

前面一群人看起来正等着他,隔着老远便扬起手叫了一声:“致远贤弟,你可来迟了。”

年轻人拱了拱手,笑着致歉:“叶涛来迟,诸位兄长勿怪!”

看众人围上来的模样,虽然他年纪最幼,却是这几人中的核心。

跟几位朋友见礼过后,叶涛回头望着自己方才走过来的方向,“那一位就是韩冈吧?”

“就是那个灌园子!”几人一齐点着头。

虽然韩冈并没有像另几个锁厅的官员一样,穿着一身的官服。但认得他还是有着不少,当他来到国子监门前之后,认识他的人暗暗指指点点,窃窃私语,他的身份便立刻传了开去。

“果然是贵人气派,一点也不见担心呢……”叶涛看了韩冈两眼,便收回视线,哈哈笑着,“小弟这两夜可都是没有睡好觉,若能有韩玉昆一半的气度,那就能安枕了。”

“宰相之婿,当然不会睡不好。”一人冷笑着,眼中满是嫉恨,“看看主考的那几位,哪一个跟王相公没有关系?!”

另一人愤愤不平的附和着:“谁说不是!吕惠卿、曾布、邓绾、邓润甫都在王安石门下奔走,现在韩冈来应考,当然少不了他的一个进士!”

“何必如此。”叶涛吊着眼斜睨着韩冈。“若是曾、吕之辈真敢徇私,登闻鼓院就在不远处。击鼓叩阙,徐士廉能做的,到时候我们一样也能做!”

大宋朝的文人胆子不大,上阵时,吓得腿软脚软绝不鲜见。但要是争名夺利,却没有一个肯输人。叶涛说得狂妄,他周围的人仍纷纷点头应是。

叩阙又如何?

欧阳修旧年主持嘉佑二年科举,排斥当时所流行的险怪奇涩的太学体,以平实畅达取士。以他的文名和权威,照样被落第的士子围着责骂。

何况叶涛所说的徐士廉,他可就是靠着敲着那登闻鼓,硬挣来了一个进士的身份。

太祖皇帝之时,进士科举试并没有殿试,礼部试便是最后一道关卡。到了开宝六年,李昉知贡举,所选进士不孚众望,而徐士廉击登闻鼓,控诉其‘用情取舍’。最后宋太祖赵匡胤下令由他自己来考核举人,从此以后便有了殿试。

“韩冈本无才学,能遽得进用不过是因缘际会而已,听说他连诗都不会做,看着今科改诗赋为经义,才赶过来应考。”

“也不能这么说,方才小弟正好听到了他吟了两句。”叶涛说着,就将方才路过韩冈身边时,听到两句诗给念了出来。

众人各自默念了两遍,皆尽摇头,“只有两句而已,不见全篇,也看不出好坏。”

其中一人又道:“念着倒是平平,画出来就有些味道了。”

叶涛笑道:“公长既然这么说,那就没错了。若以丹青取士,这五千人中,公长你当能拿个头名。”

“难比上一科的李公麟。”公长自谦一句,又仰头笑了起来,“不过若以浇菜种地为科目,状元不用考就能定下了。”

几人登时哈哈大笑,惹得周围考生皆尽侧目,连韩冈、慕容武都望了过去,暗暗摇头。

随着几声锣响,国子监大门终于被打开。两名监门官——虞部郎中胡淮,职方员外郎穆珣威严肃重的带着一群兵丁走了出来。拥挤的人群渐渐的安静了下来,叶涛诸人也都收敛了狂态,听着胡淮和穆珣的指挥,敛容正色的排起了队。

几千人在国子监门前慢慢的向前挪动,渐渐汇入考场之中。

太阳终于出来了,蓝紫色的天幕被漫天的红光所取代,依然是个大晴天。

自从韩冈上京,这段时间以来还都是好天气,今天也没有例外。天气好,应考的心情也便好了起来。

门后的照壁上,贴着布告,注明不同地域、不同来路的贡生,在什么地方考试,又安排着吏人来引导。考生人数虽众,却一点也不见混乱。同乡的贡生之间要互相作保,考试的地方也在一起。而韩冈这样的锁厅举人,则是与他其他参加考试的官员一起,被分在一间偏殿。

不过进门后,贡生们并不是立刻分流去各自的考试地点,而是被引到文庙大殿之前的广场上。

知贡举曾布,同知贡举吕惠卿、邓绾、邓润甫,领着其下一众考官,立于大殿之前。祭拜大殿中所供奉的至圣先师,是开考之前,必须走过的流程。

听着赞礼官的口令,与数千人一起拜倒,屏声静息的向着‘天不生仲尼,万古如长夜’的孔圣人叩拜。

一拜,再拜。

紧张、期待,各种各样的杂乱思绪,在一拜一起之间,为之一扫而空。

当韩冈重新站起来的时候,已是心如止水,再没有一丝杂念。

