\section{第24章 兵戈虽收战未宁(二)}

伏击圈就在眼前,王舜臣喜上眉梢,恍恍惚惚的瞧见前方正有一份泼天的功劳在向他招手。只是当他回头一看,便立刻叫了一声苦。不知何时,身后的追兵已经停了下来,然后直截了当的掉头离去,没有一点拖泥带水。

奔驰中的队伍也逐渐的慢了下来,最后在失落中停住了脚步。加速远去的蕃骑卷起的尘烟遮挡住了韩冈的视线,他望着灰黄色的幕布掩盖起的来路,暗道这世上果然没有蠢货。而竟然连不读书不知史的蕃人都骗不过,看来自家的演技也实在有待磨练。韩冈再看了看身边丢盔弃甲的一众骑兵,狼狈不堪的模样就跟打了一场败仗没有两样——他苦笑,今次诱敌,却是折了大本钱。

王舜臣紧皱着眉,来到韩冈面前:“三哥,这下该怎么办……”

韩冈故作轻松的微笑道:“往好处想吧,这等于是又拖了禹臧花麻近一个时辰的时间。”

他抬头看了看天色日影已经西斜,再有两个时辰就要天黑了。那时候,禹臧花麻就失去了撤兵的最佳时机。

韩冈不认为黑夜能遮盖一切,趁夜撤走可不是像字面上说起来那么简单。夜间行动,关键在于一个‘奇’字,而不是‘黑’。黑暗能掩盖一切,但不论是哪一方都同样能公平的利用黑暗带来的便利。相对而言,在黑夜中,大军行动可比小股行进的难度要高上许多。

如果禹臧花麻想在夜间撤离。他点起火炬,就会成为最为显眼的目标,若是不点火炬,黑暗中将不知军,军不知将,那样的情况下,只需要出动一两百人,就能造成让禹臧花麻全军崩溃的混乱来。

韩冈要把禹臧军拖到渭源堡的援军赶来,为了能最后击败禹臧花麻,他必须为王韶和苗授争取时间。而韩冈之所以会咬着牙死死拖住禹臧花麻,是因为他相信渭源堡的战斗力。就算这座寨堡刚刚被禹臧花麻重重围困过,但韩冈他还是相信,只要能让他们来得及布下阵势,禹臧花麻就绝对没有获胜的机会。

阵列不战,这是所有与大宋步军交手过的异族的共识。除非能设计不让宋军摆开阵势,否则阵势一起,箭矢如雨而落,就算强如契丹也要退避三舍。曾经仔细查阅过几十年来在关西发生过的大小战例,韩冈对自己的军队有着充分的信心。

“王舜臣!”韩冈突然冷声叫着他最为信任的名字。

严肃的神色让王舜臣愣了一下,不过他立刻醒觉,上前躬身:“……末将在!”

韩冈指了指山道两侧,“把你的兵带上。”

在山坡上,是从星罗结城受命而来的伏兵。只是他们白白被蚊子咬了,并没有能得到他们想要看到的结果。但他们的战力,依然还有发挥的余地。

王舜臣大声应诺,“末将遵命……那三哥你呢?”他又问道。

韩冈向南望去,锐利的视线仿佛穿透了迷雾和距离,落到了大甘谷口:“追回去!敌进我退,敌退我追,总不能让禹臧花麻轻松下来!”

……………………………………

“花麻,撒解他们怎么还没回来?”一个年迈苍苍的蕃人一边问着禹臧花麻,一边翘首北望。他视线投去的方向,便是星罗结城所处的位置。老蕃人身上穿的衣服闪着丝绸的光泽,而他对禹臧花麻的口气,更表明他的身份不同一般。

“不必为他们担心。近两倍的兵力,怎么可能还会输?”禹臧花麻随口敷衍着,但他冷漠的口吻,昭示了他们的死活其实并不放在禹臧家族长的心上。而神经质一般不停敲打着马鞍的手指,也透示出他心底的不耐。

“万一输了怎么办?!”老蕃人一下急叫了起来,絮絮叨叨的说着,“我可就只有这么一个孙子……”

通过常年的蚊虫洗礼,禹臧花麻已经可以对这些废话做到充耳不闻。

年纪轻轻就登上族长之位,为了维护自己的地位,他不得不在一定程度上听从老家伙们的摆布。禹臧花麻本打算按部就班的在十年间将他在部族中的敌人全数解决,那时就没有人再敢跟他过不去了。禹臧花麻的计划正在一步步的实现中,可一场战争便光临到他的头上。但危机就是机遇,禹臧花麻本想着通过胜利让自己权势更加巩固,谁能料到他竟然会输,这也就给了对手最好的攻击口实。

昨天向他逼宫的应该也有着这个老东西在。禹臧花麻瞥眼看着纵横交错的重重皱纹下,一张一合的缺牙瘪嘴,心中发狠,迟早要把这些老骨头丢进火堆里当柴禾烧了。

在禹臧家的年轻族长眼中,这些老东西都是一样的惹人厌烦,甚至不想多看一眼。对于老东西的孙子究竟会怎么样,禹臧花麻也同样不关心。胜也好,败也好,只要能把瞎药家的近千骑兵拖上一两个时辰就好,等到他与渭源堡的王韶决战之后再回来也可以。

在韩冈看来,他逼得禹臧花麻分兵来追击自己,虽然没能把他们引入伏击圈加以歼灭,但实质上却等于是把禹臧花麻拖了一个时辰下来。

可是从禹臧花麻的角度来看,他何尝不是用着一千多名出自于附庸部族,在战场上肯定会出工不出力的废物,换来了一个与渭源堡的出战守军单独决战的机会。而且如果那群蠢货还有一点头脑的话,说不定还有夹击这些宋军的可能。

当然,禹臧花麻的盘算有一个必不可少的前提,就是他统领的大军,能单独击败渭源堡的军队。

对此,禹臧花麻有着绝对的自信。

一名骑兵自远处狂奔了过来,一到阵前,他便从马背上摊到了地上。他是禹臧花麻前面派出去的哨探。那一支不知由谁统领的骑兵的离开,对禹臧花麻最大的好处,就是他终于可以派出斥候,对渭源堡方向进行侦查。

哨探身上的袍服破破烂烂,还有几处伤口正在向外渗着血。被人扶起来后,已是气息奄奄,命悬一线。这不是露在外面的伤口所能造成的,在衣服底下,应该还有其他伤痕存在,那才是致命伤。不过没等禹臧花麻让人在哨探身上找寻伤处,进行救治,哨探已经拼尽最后的力气,匆匆向他通报了最新的军情,

“渭源堡出兵了!已经到了八里外!”

望向渭源堡的双眼被山壁阻挡了视线,但禹臧花麻期待已久的敌人很快就会从那一处弯道拐过来。

克敌制胜,就在片刻之后!

……………………

“韩冈和瞎药在哪里?”随着离大来谷越来越近,苗授的双眉也就锁得越来越紧,皱起的眉头在眉心处拧成一个川字。

从渭源到大来谷,几十里地的行军对一支历经多次战事的军队来说算不了什么。在派出斥候确认了敌军的位置,苗授便留下了随行的民伕,让他们在后方扎营,而他自己则领着主力赶来大来谷。

只是他手上掌握的兵力并不多,迫切需要汇合韩冈手上的蕃骑,还有王舜臣那里的一千多人。没有韩冈、瞎药率领的青唐军,也没有留在星罗结城的士兵,让他就此对抗实力数倍于己的敌人,实在是一桩令人吃不消的苦事。

可是由于交通中断,苗授现在还不知星罗结城究竟怎么样,也不知道韩冈到底有没有联系到王舜臣。更不清楚,他们有没有按照韩冈自己请人带回的建议,聚歼禹臧花麻。

什么都不清楚,这让一向行事稳重的苗振,也有些想骂人,本不该这么仓促的。但王韶对韩冈深具信心,一接到韩冈传回的口信,便当即命苗授出兵接应。

苗履在旁劝慰着自己的父亲,“大人勿需担忧,即便星罗结城不保,还有韩机宜在。瞎药的蕃军是新锐之师,而韩机宜又是才智闻名关西,必然不至于会轻易的输给禹臧花麻!”

‘如果真的这么简单就好了!’尽管心中不以为然,但苗授并没有指出儿子话中的错误。在战前,顺耳吉利的好话,总比一些锋利刺骨的实话要让人安心。

已经远远的看见了吐蕃人的身影,数以千计的聚集在大来谷口。当苗授一声号令,鼓点响起,这一群蕃人便被雄壮的号角声吓了一跳。

如果韩冈在场,能亲眼看到苗授指挥布阵的手腕,他肯定不会吝啬一声称赞。在西路都巡检的指挥下,他带来的千多名士兵,自下马后,从行军队列转换成临战阵型时,走势如行云流水一般顺畅。从细长绵延的队列,一边向前,一边逐渐向两侧拉伸,当他们在敌前站定,已经是整整齐齐的变成了一个中军突前、两翼后弯的倒偃月阵。

不论双方在战前有过多少谋划,都希望揪住了对方的破绽,而得到胜利。但到了最后,决定今次一战胜负的,却还是面对面的战斗。

