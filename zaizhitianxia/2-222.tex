\section{第34章 山云迢递若有闻(三)}

暮色苍苍。

寒风中,庆平堡的最高处正有一面旗帜在猎猎飘扬。

这座封锁了大来谷西口的寨子,吐蕃人给其起的名字官军中没人懂其含义,王韶在踏足此处之时,便直接将之改名为庆平——庆贺平定。

攻下庆平堡的功臣苗授父子和王舜臣,都带着他们的麾下将士在堡中休息。两千兵马将这座面积并不算太小的寨堡,挤得满满当当。使得随之而至的中军,便不得不驻扎在于堡外。

庆平堡在吐蕃人手里,是个略大一点的土围子。尽管守军因为听闻宋军将至,而增加了不少。但在在苗授、王舜臣这等猛将率领的精锐官军面前,也不过是从鸡蛋壳变成了鸭蛋壳而已。

但庆平堡的位置极为重要,是大来谷的出口,连接武胜和通远的要道。宋军攻下此处,代表着王师终于踏上了武胜军的地界,临洮已近在眼前。

夜将至,高遵裕和王韶聚于主帐中。

拿着从后方传来密信,高遵裕哈哈大笑数声,“文枢密手上真真没人了,派来的蔡曚都成了笑话。韩玉昆都没怎么费气力,就让他连站的地方都没了!”

王韶略显冷淡的说着,“有赵大观【赵瞻】殷鉴在前,现在文宽夫【文彦博】还能使唤动几人?有平叛之功的尚且被晾在一边,没有功劳的还能有什么机会?到我这河湟来,不想方设法地挣军功,反而听命于枢府居中阻挠,聪明人又岂会做这等吃力不讨好的蠢事?”

自从熙宁二年和三年年初,旧党闹过一阵后。其首脑除了一个文彦博,其他都陆续被赶出了朝廷,这两年其实已经消停了不少。中层官僚中,许多人也便渐渐的转向了新党一方。

王安石的变法成果,世人都看在眼里。不论旧党如何抨击,被损害了利益的豪商、宗室们如何抱怨,至少眼下国库充盈了,在对外战事上,大宋也是由弱转强,捷报频传。横山攻略虽然无功而返,可也是非战之罪,运数不到而已。

眼下在军政两方面,都是新党正得意的时候。除了几个愣头青以外,谁还会在正得天子关注的河湟之事上,

“只是蔡曚未免太蠢了点啊,”

“他并不蠢,只是遇上了韩玉昆罢了。玉昆在通远恩信深重,人望亦高。城乡内外奔走听命,亦不足为怪。岂是他官可比?”王韶说,“这世上有胆子顶撞朝官的选人有几人?有能耐让一城上下令行禁止的军判又有几人?蔡曚输得不冤。换作不是玉昆,而是别人,他早就得逞了。”

换作是一般的官员争权,衙中胥吏都是站到一边看热闹,谁会搅和进那趟浑水里去?嫌命长了不是?给风尾扫到,就是有家破人亡之虞。哪像韩冈,一句话就让胥吏们与蔡曚划清界限。

“也幸亏有韩玉昆坐镇陇西,不然我们怎么能走得这么放心。”高遵裕又哈哈大笑了两声,在他看来,王韶两年前的举荐,实在是捡了大便宜,“蔡曚就是不知道这一点,才做了如此蠢事。”

苗授就在旁边听着,韩冈是怎么踩着一路转运判官的蔡曚,他都听在耳中。听着听着,便有些心惊胆颤,“韩玉昆是不是做得过了点?”

“这个时候,就是有点嫌疑都不能放过,何况蔡曚这样自己跳出来的?身为随军转运,却不思尽力报国。只奉权奸之命,直欲陷数万大军于死地。韩玉昆做得一点都没错!”

王韶身上传来的阵阵杀气,甚至比前天他亲自压阵攻打大来谷时,还要重上许多。苗授浑身一阵发寒,不敢再说了。

王韶半点不敢忽视蔡曚的危险,碰上运气不好的时候,猪都能坏事。

横山攻略,韩绛怎么败的?用错了一个王文谅而已。庆历新政,范仲淹因为什么给赶出朝中的?欧阳修写出《朋党论》,明着跟天子说我们要结党——欧阳修的确才高,但从政治上,他只会拖累自己人:不论是庆历新政,还是后来的濮议之争。

这时忽闻帐外通报,王都知来了。王韶收起了满身的杀意,换上了一幅笑脸,“快请都知进来。”

王中正从掀开帐帘进来,高遵裕也把蔡曚的事权且放在一边。王韶都动了杀心,以他的身份下起狠手来,可比韩冈更为暴烈。当韩冈离开陇西后,蔡曚若是敢趁此机会在后方搅风搅雨,王韶纵然因为进士身份杀不得他,好歹也能从他身上剥下一层皮来。

“安抚、钤辖。”进来后,王中正寒暄了两句,便开门见山,“官军已经攻下了庆平堡,不知之后行止如何?”

王韶微微一笑,反问道:“临洮就在眼前,都知如何还问行止?”

“……啊!”王中正微楞了一下,自嘲的笑了起来,王韶的答案让心急的他很满意。但他又道,“不过蕃人狡诈,安抚还是要小心后路为是。”

高遵裕暗道,王中正这纯属废话,都是老用兵的,后路怎么可能不提防。

“担心是肯定担心的。”王韶对天子近臣保持着礼貌,他指着铺在桌上的简易沙盘,为王中正解说起来,“瞎吴叱最大的可能就是坚守临洮,然后等待木征的援军。而且少不得会抄截官军的后路。不过临洮离临洮只有一百三十余里,除去鸟鼠山,更是只剩百里。这么一点距离。没有大军辗转腾挪的余地。就算吐蕃人来抄截后路粮道,也只能派出小队人马。人数稍众,必为我耳目所侦缉。而且还有青唐、纳芝两部的蕃骑,他们也会为官军打探消息。”

“尤其是包约【瞎药】。按照事先的约定,官军一旦夺下武胜军,这里的蕃部,都会交由他来统领。由不得他不卖力……包顺【俞龙珂】则是会接手包约留下的地盘,而张香儿那里也会有回报。所以今次攻城将是由官军打头阵,但阻援的先锋,便是青唐、纳芝两部三家。木征在南面的岷州还有一个弟弟,一旦武胜军被攻占,其与河州的联系便会被阻断,他想必也会出兵援救瞎吴叱。”

王韶和高遵裕的回答,让王中正放下心来。他笑道:“那下一步就该去临洮了。”

“不!”王韶摇了摇头,指着沙盘:“临洮城前面二十里,还有一道野人关。不过野人关虽说是关,但也仅是在略显狭促的一处谷地处修的小寨子,只有一道栅栏而已,并不难攻克。”

王中正低头看着沙盘,又问:“那出兵攻打野人关是在明日还是后日?”

“何须等明日?!”高遵裕口气豪勇无比“如今军中士气正旺,主力又已修整了一天。只待一声令下,即刻便能出发。”

王中正猛抬头,惊问道,“今夜就出兵?!”

“斥候已经都派出去了,青唐和纳芝临占两部的蕃骑都在监视着沿途的要点。只需急行半夜,日出时便可赶到野人关,正好打吐蕃人一个措手不及!”高遵裕隔着帐幕,遥指着天顶满月,“今日月色正明,正是行军的好日子!”

……………………

“怎么禹臧家的援军还没来?!”

瞎吴叱愤怒的把手上的酒杯砸到了亲信的脸上。跪在地上的亲信脸上被砸出了一个血口子,涌出的鲜血污了脸庞。可还是能看得出,他正是前日去兰州求援的使节。他把好消息带回了武胜军,可到了现在,这个好消息依然没有被验证的迹象。

帐外突然传来沉重的脚步声,瞎吴叱和他的亲信立刻用期盼的眼光望着帐门。

一名士兵摇摇晃晃的冲进来,混忘记了礼节。喘着气,说出了与瞎吴叱的期待完全相反的噩耗:“……野……野……野人关被攻破了!”

“什么?!”瞎吴叱一声惨叫。

揪着从野人关赶回来的信使的脖梗子,瞎吴叱咬着牙从他嘴巴里逼问出宋人的情报。在失去了最前沿的寨堡后,他依然还认为会有两三天的时间让他等待援军,谁想到宋人竟然会这么快,而且竟是夜袭。

怎么办……怎么办?

是退还是守?

瞎吴叱团团转着,只又过一个多时辰,他再一次惊声叫起:

“宋人的斥候已经到了城外了?!”他摇摇晃晃,差点都要昏倒。

被亲信扶着,瞎吴叱在城头上看着十几骑宋军在城下耀武扬威,从城下射来的一箭甚至差点扎中了他的耳朵。

瞎吴叱如同一只兔子一样跳起,“退……快退过洮水去!”

……………………

高遵裕亲率千骑夜袭野人关。至关口时,关中蕃军犹在睡梦中,猝不及防,关隘一鼓而破。紧随而来的主力并没有在野人关多加停留,越过关隘,直奔临洮城而去。在数千大军的威逼下,瞎吴叱狼狈而逃,匆匆退过了洮河西岸,而将临洮城拱手让出。今次出征,竟然不费吹灰之力,便已经把最终的目标夺占。

当韩冈抵达渭源堡的时候,正听着欢呼声冲霄而起,声浪汹汹,几乎要把锁住渭水源头的这处寨堡给掀翻掉:

“王师攻下了临洮城!”

“王师攻下了临洮城!”

众军的兴奋之中,韩冈却在低声细语,用着只有自己能听到的声音说着,‘不要又是一个罗兀城。’

