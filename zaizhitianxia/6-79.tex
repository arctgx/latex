\section{第十章 千秋邈矣变新腔(一)}

在考题公布之后,宗泽便松了一口气。

进了贡院中之后,宗泽便一直感到有些压抑。贡院里面的空气,都仿佛比外界重上几分。更何况由两位知贡举带领考官、考生一起向先圣参拜的仪式,庄严肃穆,更是给一众士子平添了一份压力。

宗泽曾经听前辈说过,贡院中多有冤魂,全是屡考不中、郁愤而亡的士子。应考的贡生们只要心思一乱,立刻就会被缠上。

再有才学的士子,一旦乱了心境,也会连普通人都不如。

当然,为什么有圣人坐镇贡院里面还会有冤魂?何况这座贡院还是新修,开门迎客也就几次,能死几个?

这一点,那位专爱说鬼故事的前辈就不能自圆其说了。

今科的考题,在经义上没有什么特别,只不过出自《诗》、《书》、《周官》中的内容比预计中少了很多,很可能是《三经新义》给人琢磨透了,所以干脆减少一部分,以加强难度。

而之后策论的题目,让宗泽在安心之余,又忍不住摇头苦笑,为那几位爱猜题的同窗担心起来。

熙宁六年礼部试的策论是史论:以秦与商鞅之事为题;九年则是策问:天子因天下灾异频频,而问策于考生;元丰二年也同样是策问,因为当时的形势,加上主考是去过辽国的许将,策问的内容有关西、北二虏。

连续两科都是策问,所以这元佑元年的礼部试,大部分士子都觉得应当不该是策问了。

但宗泽没有管过去是什么情况,策与论,他都下了功夫去用功,

事实证明,铜板连丢两次叉,第三次还是有可能继续是叉,而不会变成快。

宗泽也赌博,掷铜板有字的那面叫叉,没字的那边叫快。他平常常玩三星,三枚铜板要掷出一色的浑纯,难度甚大。但一枚铜钱除非是要掷出侧面朝上,否则叉和快都是很容易出现。

不过有一点宗泽是清楚的,这一次不论是出现那一面,都跟上一次的结果没有任何关系,只看老天和运气。

虽说考题的内容与人有关,不过猜测人心所向,大概也就跟掷铜板的差不多。

所以这一回以为策论的体裁会是论而不是策的考生,全都赌输了。

宗泽虽是赌赢了,不过也没敢太沾沾自喜。

不论是策,还是论,一般都会切合当今的形势,但同样一件事,在不同立场的人眼中,必然是有着不同的意义。

故而还要看主考官,他在朝堂上是站在什么立场,过去又有什么经历,本身又是什么样的文风,又有什么样的忌讳。这都是需要事前去了解的。

若是不去注意,一头撞上墙去,喊冤都没人理。

君不见当初欧阳修为一洗文风,在他主持的礼部试上,刷落了多少名震士林的考生,以至于在路上被人围攻,可终究是一点用都没有。被取中的去宫中参加殿试,被刷落的扎欧阳修的草人也没能让欧阳修少吃一碗饭。

宗泽仔细的审视着题目。

去除无谓的辞藻,今次策问的论点只在于绍述二字。

这道题乍看起来难度并不大,也符合考前的猜测。就算猜错了体裁的考生,看到内容后,就会安心许多。

绍述就是继承,先帝新丧,若要说针对何事,不问可知。题眼当然是论语中的‘三年无改于父道,可谓孝矣’这一句。但要如何联合实际进行阐发,并给敷衍出一篇让考官满意的文章,就很让人头疼了。

宗泽越是思量,越是觉得这道题里满满的皆是恶意。

父在,观其志;父没,观其行;三年无改于父道,可谓孝矣。

但新法便是号称效三代之法,变祖宗之制。

这当如何说?

说起来,也不是没有办法。

操两可之说,设无穷之辞。这不是名家独有的特技,正常的士人都能做到这一点。

而在不同人面前,将一件事正说反说都说通,也非是纵横家的特权。

只不过今科可有两名知贡举。一个是蒲宗孟,一个是李承之,这两位,大家都不熟。被任命为知贡举又太晚。他们的立场还好判断,但喜好、风格,一时间能了解到的内容并不多。而且有一点很明确,两位知贡举绝不可能和睦相处,一个不好,就有可能卷入两位主考的争斗中,然后死得莫名其妙。

宗泽想了一下,就将这道策问暂时放到了一边,先从经义的题目做起。

有关经义的部分,在国子监中,常年系统性的练习过,宗泽写起来得心应手。

出处在《三经》之中的题目,只要遵从三经新义就够了。三经新义没有解释到的地方,一部分遵循孔颖达的注疏,一部分则是出自国子监的新义。

这些年以国子监为主的新学团体,对新学的钻研日渐精深,对三经新义所没有涉及的其他经书,又有了许多新的阐发。

在经义研究的前沿领域,国子监出来的贡生,对此有着先天上的优势,外路的贡生远远没有这么好的条件。

这其中大部分的观点都只是在京中流传,甚至仅仅在监中传播,但在之前不久,却经过了经义局的审核,成为国子监的教材之一,也是考试的标准答案。

在考试中用上新义,并不需要太在乎知贡举的身份。知贡举一般只会看后面的策论,前面是经义通过初考官和覆考官的评阅就够了。而知贡举下面的一干考官,无一例外都是新党中人,其中还有研习新法最为精深的几位国子监博士、教授,监中出身的贡生们可以放心大胆的写那些新释义。

宗泽解决前面的问题没有耗费太多的时间,但当他的注意力再一次回到策问考题中时,便陷入了一阵长考中。

可是长时间的思考,除了让他心烦意乱之外,没有别的结果。

一旦立论错了,就又要多费三年,可两名考官又该迎合谁人?两全之说,又必失之平庸,更不可能通过。

这一道题,难处不在题上,却在题外。

一时难以拿定主意,宗泽最后放下了笔,用力的搓了搓脸。深呼吸了几下,放下手时,他的神色终于安定了下来。

宗泽性格谦退,常常曲己从人,但若是事涉正道、本心,那便不同了。

开头若是扭曲了本心,日后做了官,也会是个逢迎上司的庸官。

与其曲己以媚主考,还不如将自己的心志和见解,痛痛快快的表达出来。就算考不中,至少不会感到憋屈。

提起笔,蘸上墨。

下笔时尚有些忐忑,但笔落纸上,宗泽的笔锋便不再停滞。

一名下来巡察的考官走过宗泽面前,看到他运笔如飞,不由得轻轻点了点头。

差不多到了后半段,考生们都完成了。面对这一回的策问,还能笔走龙蛇,的确不简单。

方才将这一片一圈走下来,也就这一位考生落笔最是畅快。

他看了一眼贴在一边的姓名……

宗泽。

……………………

放衙的时候,韩冈正在回家的路上。

不用当值,该处理的事情也处理得差不多了,韩冈自不会在皇城久留。

但回去后,前来求见的官员能够塞满家门前巷道,今天晚上至少再接待十几人,点十几次汤水。

当初韩冈在枢密副使任上时,由于时间太短,期间朝中又颇多风浪,还没来得及享受到多少宰辅级的待遇,而如今就大不一样了。

想到回去还要看一群官员游移在矜持和谄媚之间的笑容,韩冈就想能不能偃旗息鼓,换身装束从后门回家算了。

不过再想到这是扩张声势的机会,韩冈还是耐下性子。核心与根基要好生培养,而外围摇旗鼓舞的人也不可或缺。

而且,这也算是公务的一部分。

政事堂最重要的工作就是人事,不设法多加了解各方官员,难道要抽签决定堂除的人选?

一群士子从前面走过,听到喝道,避让道路边,然后又冲着韩冈指指点点,低声说些什么。

这些士子看神态很放松,但又有着几分紧张,一看就是刚刚获得解放的贡生。只因还有一道殿试等着他们,不能完全放松。

到底能通过礼部试的考生有多少,韩冈根本都不会去在意。

考题已经拿到了手上,看似浅显的题目,但却因为各种试卷外的因素,会让贡生们大感头疼。

等到最后的结果出来,了解到评判标准,事后怕是有不少会撞墙。

穿过拥堵在门前的官员车马,韩冈终于回到家中。

等待他的,不仅仅是外面官员、士人送来的拜帖,还有一堆的书信等待韩冈拆阅。

将拜帖先放在一天,韩冈拿起那一摞书信,翻了几下,突然发现一封信的发信人姓名很是眼熟。不是认识已久的眼熟,而是刚刚听闻、突然又见到的那种熟悉。

尤其是在收到那份密信后,崇文院成员的姓名,就分外让韩冈敏感。

将信打开来一看,韩冈便摇了摇头——果然如此!

跟他之前毁去的那条密信是同样的内容,只是稍稍有些差别。

韩冈轻轻弹了下信纸,是不是可以从这里面得出新党江河之下的判断?至少愿意投机的人多了起来。

不过韩冈的态度依然故我,却连信封也一并装好,打开灯盏的外罩,拿着信封的一角放进去点着了。

火光闪动,一缕青烟之后,不该存在世上的这封信,连同写信人的私心,彻底化为乌有。

但韩冈还是将两人记下来了。

天生万物,自有其理。当物尽其用,不能浪费。

张嘉问……李嘉问……

‘啊,记错了。’

韩冈拍拍脑袋,不是偷了叔祖私信的那一位,要更恶劣,恶劣得多。
