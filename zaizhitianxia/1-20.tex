\section{第11章 诛心惑神幻真伪(下)}

韩冈的句句质问如一道道滔天巨浪,不断的冲击两名库兵心中的堤防。就算在微弱的月光和灯光下,仍能很清楚看见王五和王九的脸色一点点的苍白下去。

王五和王九张了张嘴,想说什么,却又不知道该说什么。

‘成了!’两人的表情,韩冈都看在眼里。趁着两人被吓得面如土色,也不等他们回过神来想明白,他的话兜兜一转,又道:“不过呢,若刘三他们是翻墙而入,你二人也不过担个失察的罪名。而且三人现在又已授首,火也没点起来,又有什么好担心的?”

“翻墙而入?”两名库兵被韩冈的话所吸引,在无穷无尽的黑暗中,仿佛有一扇光明的大门被打开。

不远处的大街上一阵嘈嚷,韩冈向那个方向看了一眼,“哈,援兵已经来了!”转过头来,对两人催促道,“喂,快点想想,这三个贼子到底是怎么进来的?!”

“啊……?”两人心中仍旧有些畏惧陈举的势力,想开口说,却还顾忌着。

“到底怎么进来的!?”韩冈却不等他们,声色俱厉,步步紧逼,而外面的嘈嚷声也越来越近,就像催魂的丧钟,一声声让两名门兵胆战心惊。

王九还犹豫着,难以决断,王五年纪轻,顾忌反而少,忙忙叫道:“是翻墙进来的……”

只有一个人说话,韩冈并不满意,眼睛盯着王九,提高声调,重复再问:“是怎么进来的?!”

这一次王九看了看王五,一咬牙跟着两人一起喊,“……是翻墙进来的!”

“怎么进来的?!”

“翻墙进来的!”

“怎么进来的?!”

“翻墙!翻墙!”

韩冈一步紧一步的重复逼问,就像后世的传销或是邪教,通过不断重复的问话和回答,进行条件反射式的洗脑。时间虽短,可是在紧急情况下,反而更容易让人陷进去,而难以挣脱。韩冈对这等手段熟极而流,借助形势,几句话的功夫,就让王五、王九彻底站到他这一边来。

军器库外的横巷中已经传来了急促的脚步声,韩冈最后再一指三具尸身:“这几个贼子到底是怎么进来的?!”

王五和王九异口同声:“俺们两个只是看着门,绝没放一人进来。想来刘三他们定是翻墙而入,谋图不轨!该死!该死!实在是该死!”

“说得没错!此事跟两位毫无瓜葛,纵有罪名也赖不到两位头上。”韩冈双手一拍,击节赞道。可是他转而又是一叹,“只可惜没有功劳啊……”

韩冈这么一说,王九眼睛便是一亮。他行事老辣,闻弦歌而知雅意,自知当下该如何去做。呛啷一声,抬手拔出腰刀。一脚踩在刘三的尸身上,刀光连闪,刷刷刷的便在刘三的要害上剁了三五下。

王五看着先是一愣,但转眼也明白过来。便学着王九的样,一刀搠进了躺在另一边的衙役肚腹,又横里一拖,划出了个大口子。

两人的这几刀,有个名目,唤作投名状。刀子都沾了血,跟韩冈便算是一伙了,下面再想反口可就迟了。

一切刚刚抵定,几乎就在同时,大门处轰然作响,传来一声震耳欲聋的撞门声。听到警号赶来援救的队伍,终于抵达了德贤坊军械库的门外。

王五、王九忙提着带血的腰刀小跑着过去,移开堵门石,打算开门放外面的人进来。韩冈追在后面,急着叫道:“且等一等!”

两名库兵现在以韩冈马首是瞻,立即停下了手。韩冈几步走到大门后,冲着外面大声喊道:“是谁人撞门?!”

一个粗豪沙哑的声音在外回应道:“是巡城!快开门!”

“可有凭证?!”

“……要个鸟凭证!快给洒家开门!”门外一怔之后,紧跟着一声虎吼,顺带着大门又不知是什么被什么东西一下重击,震得门头上的石灰扑簌簌的直往下落。

王五和王九有些迟疑回头看着韩冈。韩冈摇了摇头,不到开的时候,他隔着门继续喊话道:“军库重地,非许勿入。无有凭证,如何能开?!”

“给爷爷撞开!”门外的吼声更怒,当真是在命令手下开始撞门。

王五、王九有些慌了,而韩冈仍不为所动,“不能开!”

“等等!”另一人清亮斯文的声音适时自门外传来:“本官可不可以做个凭证?”

王九听声连忙凑到门缝处,向外一张望,紧张的回过头来对韩冈道,“是州中的吴节判!”

“州里的节判?”听着来人并不隶属成纪县,韩冈这下方才点头,“开门罢!”

吱呀一声,德贤坊军器库的大门刚刚移开门闩,打开一条缝,便被人从外猛然一下用脚踹开。躲避不及的王五被撞成了滚地葫芦,一队士兵随即一拥而入,各持刀枪,将三人团团围住。

“是谁夜吹警号?”一名身穿公服的中年文官跨过门槛,问着韩冈三人,听声音,正是刚刚说过话的吴节判。

宋代的重要州府,大抵都有三个名号——州名、郡名以及节度军额。比如秦州,州名为秦,郡名为天水,节度军额则是雄武军。州名是属于地方行政区划用名,最为常用。郡名则是古名,大率是爵封之用,比如天水郡公、天水郡君等。而节度军额,则是承继自晚唐五代,节度使自太祖杯酒释兵权后已无实际意义,只是高品武臣的官名,但节度使司的幕僚官们,依然是节度州中执掌政务重要的官员。

吴衍便是隶属于秦州的雄武军节度判官,与成纪县两不相干,不过占了个近字,故而当先赶了过来。作为节度判官,有执掌州中兵事的资格。

如今西夏人主力正攻打秦州隔邻、属于泾原路的原州,而偏师则在攻击甘谷城,虽然只是按照惯例一年一度的打秋风。但今年年初的时候,秦州刚刚被十万西夏军全力攻打,几个寨堡被攻破,厮杀得极为惨烈,原任秦州知州因此罢职——韩冈的两位兄长也是死于此役——故而今次也无人敢疏忽。秦州知州、秦凤路经略李师中已遣一军前往扼守秦凤、泾原之间要道的笼竿城【今隆德县】,以便能够直接支援泾原路,而自己又去了秦州转运枢纽的陇城县【今天水市麦积区】,去检查当地的城防和粮道安全。

李师中不在城内,本是知州副手的通判又刚刚调任,所以吴衍便代掌其职,主管兵事。吴衍做事兢兢业业,也知道如今知州不在,权力三分,实是一点差错都出不得的。每日晚间他跟节度推官和录事参军三人,再加上司户、司理两参军一起,轮流在州衙中值守。

今夜正好是吴衍值夜,当听到警号响起,便立刻出了州衙带着一队巡城甲骑急急赶来。半路上,他心中一直都是忐忑不安,胡思乱想着,只担心军器库出了大事。可当他进了军器库大门,却见也没有什么反常,心中却是微有怒意,只想找出吹响警号之人好好敲打一番。

韩冈不知吴衍所想,正要上前禀报。这时,已经冲到院子深处进行搜查的士兵,突然在后面大叫道,“节判!这里有人死了!”

吴衍循声望去,借助火炬之光,他终于看到了在三十步外的庭院地上,正躺着三具尸身。急急改口问道,“究竟出了何事?”

这次甚至不用韩冈出头,王九丢下手中的带血的长刀,上前将串通好的谎言极有条理禀报给吴衍,“启禀节判,今夜有三名贼子,谋图不轨,翻墙偷入军库。幸亏韩三秀才警觉,他们才没得逞!……”

韩冈低下头,将表情隐在灯火不及的阴暗处,暗自窃笑。千年的时光,进步的不仅仅是自然科学,同时还有社会科学……就不知恶性洗脑算是自然科学呢,还是社会科学?

王九提到了韩冈的名字,吴衍从他那里了解到事情的大概经过后,当即开口问道:“韩三秀才何在!?”问是如此在问,但他的视线已经落到了韩冈的身上。身材虽是高大得像个武人,但身着士子才穿的襕衫,眉宇间又有着浓浓书卷气,读书人的相貌和气度,跟普通士兵截然不同,没什么人会错认。

韩冈上前,作揖行礼:“启禀节判。韩冈在此!”

韩冈走到近前,借着火光,吴衍更仔细的上下看了两眼。眼前的年轻人,看起来骨架很大,却有些病弱态的瘦削,眉眼稍嫌锐利,可说起话来斯斯文文,的确是秀才作派,让他心生好感:“你是何人?现任何职?”

“启禀节判,学生韩冈,今忝为成纪监库。”

“你是个读书人?”吴衍明知故问。

韩冈恭声回道:“学生的确读过几年书。”

吴衍皱眉:“既是读书人,怎么接了如此贱职,岂不是有辱斯文?!”

韩冈叹道:“县中有招,乃是衙前之役。家严已近半百,为人子者怎能让老父操此苦事。”

吴衍点了点头,看着韩冈的目光也柔和了一点,百善孝为先,孝子通常都是与忠臣并立。韩冈出头应役,让老父得闲,的确是孝顺:“倒是有孝心的!方才吹警号者可是你?”

“正是学生。”

“你再将今夜之事原原本本的说给本官听……”

ps:王八之气一露,两个小弟纳头便拜,这才叫主角。这几章看得爽的话就捧个场,红票收藏都要。

