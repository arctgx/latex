\section{第37章 青山声碎觑后影(四)}

天微阴。

清晨的时候,交错行进在喧闹和寂静中的夜色已经退去了,曦光渐渐爬上了山头。

河州城附近的山谷内,炊烟一注注的腾起在空中。互为死敌的两支军队,试探了几天,深夜时也不忘相互遣人偷袭,现在都无甚心思在进食的时间中干扰对方。

王韶吃完还算丰盛的早餐,从后侧小门走进大帐。众将已经在大帐中按官位高下罗列,正等着王韶前来下令。

诸将人人神色严肃,皆知今日并不同于前日。而王韶也是一派气貌严重的模样,跨步走到主帅交椅前,稳稳坐了下来。

当禹臧军开始攻打临洮堡的消息传来的时候,王韶就知道决战的时候到了。尽管他还想给木征麾下的军队以更大一点的压力,但身后燃起的烽火,让他不能再继续拖延。

通过数日对支流河谷谷口的争夺,宋军兵锋已经直面离水谷地。木征军又退了一步,被压迫得更加靠近河州城。

压缩吐蕃骑兵的活动空间,借助战场地形上的优势,将兵力上的差别一点点的抹除,这就是宋军这些天一直在做的。如果再多两天的时间,不但胜负的天枰将会倒向宋军,王韶认为木征的军心就很难继续保持下去——毕竟会为木征死战到底的愚顽之辈,在五万蕃军中最多也只有三四成。只要击败其中的这一两万人,剩下的便都是些只能打顺风仗的乌合之众。

‘只可惜木征也不是蠢人。’

王韶暗自感慨了一句,便将自己之前的如意算盘丢到一边。即便被迫提前出战,眼下新的局面也不过让人觉得稍稍棘手了一些。

熙河经略使锐利的视线从众将的脸上一一扫过,静默了片刻,他终于开口:“我们今日是背水一战!”

王韶语出惊人,一下就在众将校中惹起了一阵轻微的骚动。但主帅沉沉的眼神立刻压过来,众将情绪上的波澜随即被强摁了下去。

他继续向众将说着官军眼前的形势:“禹臧花麻已经在攻打临洮堡,如果临洮堡沦陷,结河川堡和北关堡都无法抵挡禹臧家的兵锋,那时候,我军就不得不退!”

“敌前撤军,不是每一次都有张玉、高永能在罗兀城的运气。”

“而我们的退路,更是曲折难行远过罗兀。”

“不要抱着任何幻想,此处距离陇西超过三百里,沿途山路迢迢,群蕃环伺,一旦退兵之后,想回到陇西,这里的两万大军将会十不存一。”

仿佛是威胁,一句句不吉的言辞,向众将宣告若是败阵就没有归路。

决战之前,主帅不当如此说话,但这是王韶的判断。被迫提前决战,与其将后方的敌情用言辞伪饰掩盖,还不如更加危言耸听一点。置诸死地而后生,关键看的是是否能让将士们了解到失败的危险。

“相对于敌前退兵后的九死一生,击败眼前的蕃军,可谓是轻而易举。”煽动起众将心中的危机感,王韶的口气稍稍轻松了一点,“三年来,河湟与吐蕃人历经多次交锋,却没有败过一次。”

“而三年来,朝廷的封赏,更是从没有辜负我等边臣的一番辛劳。由布衣而入朝官者有之,由小校而升崇班者有之;由敢勇而得享朝廷重禄者有之;”王韶看了看赵隆,又微微笑了笑,猛然提高声调:“由选人而为封疆边臣者亦有之!”

“诸位皆是西军中的翘楚,武艺兵法皆为一时之选。今率大军,临危城,不奋力杀敌,博一个封妻荫子,又待何日?!天子就在大庆殿中设席以待,就看诸位能不能把功劳铺到陛前!”

王韶说道最后,提气高声,霍然站起。而众将发出了一阵低吼,战意如火。

眼见自己战前动员的恰到好处,王韶说着今天的上个,“今日一战,第一,是要攻下河州,先入河州者,为首功,官阶七资三转。第二,就是木征。木征其人事关河湟大局,生擒、击杀皆可。若有谁能将之擒杀,为殊勋,即便是一介布衣,本帅亦会保举其为一任团练!”

经略使的许诺,更是让众将兴发如狂,恨不得立刻攻破河州城、生擒那木征。

在诸将的兴奋中,王韶抽出腰中剑,斜指帐外河州城的方向:“今日就是决战!……记住,我们是背水一战!”

……………………

战鼓隆隆。

先是骑兵出阵,在两军营地之间来回奔驰。

接着步兵鱼贯出营,在骑兵的护翼下排兵布阵。

虽然吐蕃人尚没有动作,但三千宋军骑兵,依然紧张的注视着敌军营中的一举一动。但在宋军全师出阵的情况下,吐蕃人并没有给他们以迎头痛击,反而是分别向河谷的上下游退了开去,一直退了约有两里地才停了下来,将河州城暴露在宋军的眼前。木征的大旗,随着吐蕃中军也在同时退回了城中,转眼已经在城头上高高飘扬。

待到烟尘甫定,阵列俨然的宋军终于看到了吐蕃人摆出的姿态。

“这是放着让我们攻城?”

“不,这是想让我们无法攻城。”

很快就有人看得明白。一旦宋军进抵城下,两翼就立刻会遭到退离的吐蕃骑兵攻击。虽然数以万计的蕃骑离开了河州城下,但他们并不是避让宋军的兵锋,而是将双拳收回到肋下,等待出手机会。

这就是最易互相支援的犄角之势,让敌军无法下嘴。几天来,吐蕃军齐集河州城下,反倒成全了宋军,可以全力攻击。而眼下,吐蕃军一分为三,其中任何一处的兵力都与宋军相差不远。攻击其中任何一处,都会被其余两处袭击侧翼或是后方,而以骑兵的速度,宋军绝无可能在其他两处蕃军赶来前,全歼其中的任何一处。

只是这样的举措,未免太过保守,一点也不像兵力远过敌军的主帅该下的命令。

几万人的大阵仗中,少数人的武勇毫无用处。浩浩如海的军阵中,赵隆带着只剩半数的熙河选锋,留在了王韶的身边。他很纳闷:“怎么木征还是在避战的样子?”

“是要等禹臧花麻那里的消息?”

王韶身边的幕僚们一时间有些闹不明白。

“别管那么多,有霹雳砲在,攻城也不需要太多的手脚!”王韶厉声喝道,“传令景思立,让他领本部去提防北方的贼军。再传语姚兕,让他去防着南面。把霹雳砲推上来……攻城!”

数十辆霹雳砲车被推向了阵前。又改进了一次的配重式投石机,比旧型号变得更加高大,接近四丈的高度甚至超越了河州城墙。

这五十辆刚刚被打造好的砲车,如同一排巨人矗立在军阵中,给人以巨大的压迫感。原本鼓噪蠢动的吐蕃蕃骑一时间也变得安静下来,无不为之震撼。

若在过去,一辆行砲车至少需要五七十人服侍,而眼下的霹雳砲,却是需用的人数少到极致。出战的大军能使用多少霹雳砲,只取决于工匠们的制造能力,而不是兵力。

木征站在城头上,望着渐渐推前的霹雳车。虽然不知道那是什么兵器,但用来攻城的道具却是不会猜错的。青筋毕露的右手紧紧按着刀柄。宋人这是要猛攻城关,逼其招兵回援。

‘王韶竟然这么自大?……还是自信?’

木征的眼神凶戾:“宋人未免太小瞧人了。”

但不止一名将领被霹雳砲车惊到,回来向木征请求,急招城外两路大军来堵截宋人继续向河州城逼近。

“不!”木征摇着头,语气依然坚定,“让他们再近一点……一天时间,怎么也能坚持得下!就算是失了河州城,也要把宋人给缠住!”

……………………

天光渐渐黯淡了下来。

高耸的山壁上草木森森,枝叶的遮挡下,狭窄的山道变得阴暗模糊。青谊结鬼章也只能用头顶上,被群山压缩得只剩小半幅的天空来判断时间。

前后众军沉默的随着青谊结鬼章而前进,在渐渐变得阴暗的山道中,宛如幽魂组成的队伍。

“为什么要去珂诺堡?”青谊结鬼章的身边,有一名与他同样年轻的吐蕃贵族在追问着,“攻下香子城不是更简单?”

鬼章家的族长沉默着攥着马缰继续前行。

而年轻人问了两句,也不能回答,忽又自说自话的恍然大悟起来:“是不是担心宋人回来后不好撤离?的确,河州离得太近,若是宋军遣精锐回来,根本躲不过。还是让木征家的人去送死好了!”

年轻人嘿嘿笑着,仿佛捡了便宜一般。青谊结鬼章却转过头,用三九腊月的眼神盯了他一眼,“是木征希望我们去牵制珂诺堡的宋军。”

“啊……?”年轻人惊讶着,“真的是听木征的话?!”

不比普通吐蕃人那般都对吐蕃王家血脉多多少少的有着一点敬畏,青谊结鬼章对木征一点尊敬之心都没有,但眼下宋军压境,为了维持鬼章部东侧的屏障,他甘愿吃上点亏,“都这时候,还想窝里斗吗?!”

年轻人急了起来,像是要为自己争辩,青谊结鬼章冷着声音说着:“禹臧花麻都出兵了,也别让他看笑话。”

一声号角突然响起,并不是吐蕃人惯用的牦牛号角的音色。

是宋人的暗哨在传递警讯。大概是发现了走在最前面开路的两百骑兵,但只有号角之声传讯,看来这条路上并没有埋伏。

青谊结鬼章当机立断,大呼一声:“快!”便立刻纵马前冲。

随着年轻的鬼章家族长,董毡帐下的精锐骑兵不再保持悄无声息的行动,而是立刻奔驰起来,直如流入峡谷的河水,变得汹涌激荡。

