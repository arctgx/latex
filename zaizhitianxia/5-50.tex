\section{第七章 苍原军锋薄战垒(六)}

【算是昨天的第三更,补前面的。】

种谔、李宪刚刚打下了宥州,盐州则是给高遵裕派出的偏师捡了个便宜去。银夏之地基本上是收复了,消息传到京城

王中正离着灵州尚远,却已经在报告在进兵的过程中受到了西贼顽强的阻击。不过他们沿着黄河河谷走,接连打穿了几处峡口。叫苦虽然厉害,但成果却是最丰厚的。而且王中正还有一支偏师,是往凉州府去的。

王舜臣攻下了济桑寨,翻越了洪池岭,正向着凉州进发。京城和王舜臣之间的消息往来有二十多天的延误。想必此行如果顺利的话,凉州应该已经攻下来了。

至于环庆的高遵裕、泾原的苗授,两人都顺利抵达了灵州城。只要休整两日,就能出动攻打灵州。以官军在城池攻防战上的水平,以及霹雳砲等战具的使用,拿下灵州城不成问题。如果有胆量的话,更可以顺便将兴庆府也一并拿下来。

“眼下局势,全都靠了相公。”清风楼中,现任知制诰的蒲宗孟举着酒杯,“精兵悍将齐集灵州城下。灵州转眼可得,灭夏也就指日可待了”

王珪轻轻笑了一笑,抿了一口酒,“哪里。这是陛下的功劳,我也只是辅佐罢了。”

“天子岂能少了相公的辅佐?伐夏之策一出,顿遭群小非议。若无相公居中一力主张,如何能有如今观兵兴灵的这一天?百年之患终得解脱,此乃相公之力也。”

王珪叹了口气:“只要日后的诽谤能少一点就好了。”

他难道会不知道三旨相公的称号?王珪既不是聋子也不是瞎子。讥讽往听得到,嘲笑他看得到。紧紧跟着天子,所有的行事全都取决于天子。王珪将自己的官场哲学执行得很完美,但他终究还是不甘心的。

身为一国宰相,辅佐天子治理亿万子民,王珪既然占到了这个位置上,终究还是想为后世留下点什么。让自己的名字能刻画进青史之中,能走上更高一点的巅峰。

王珪很少有机会表现自己,他的任务就是统管大局,既不是上阵杀敌、也不是领军灭国,这些相对于宰相来说,并不在职权范围内的事务,决定了王珪根本掺和不进去,只能坐视一个个机会被人拿去。

幸好王珪有的是耐性,只要还在宰相的任上,就还有希望。等了半年,终究还是给他等到了一个机会。

郭逵、王韶、章惇甚至韩冈,他们有能力,有功绩,也为大宋的国势流汗出力,但他们都没有这个运气,将果实收入怀里的运气。

但他王禹玉有。

所以他一力主张攻夏,只要能顺利的攻下灵州和兴庆府,自己的地位和声望必然能够跟韩琦、富弼相媲美,而远远超过那些庸庸碌碌的朝臣。

如今天下安定,可动荡的时局随时可能会出现的。一旦时局动荡,到了关键的时候,天子决不会信任一个只会说请圣旨、领圣旨、已得圣旨的三旨相公,但必然会信重一个恭顺、有能力而又不乏实绩的宰相。

有了灭国的功劳,即便因故出外,也很快就能回京。坐镇朝堂的总不会是一干元老,更不可能是倾向性太强的新党、旧党,而是像他王珪这般,有能力,有声望,还经得起摔打,对天子的忠心也始终不变。不用这样的人,还有什么人可用。

这就是王珪的想法,对于一名已经走到了官位巅峰的宋人来说,人望、地位和可以卷土重来的机会,这些才是他一心想要到手的关键。

而且有件事十分值得庆幸,因为他就快要成功了。

正如韩冈所说,只要官军打下了灵州,这一仗就赢定了,怎么也不可能再输。

“高遵裕、苗授先后抵达了灵州;王中正很快也要到了;种谔、李宪那边或许有些问题,但以他们手上的军力,度过瀚海也是迟早之事。”

“还有灵州。”王珪还要维护一下身为宰相的矜持,不会在外人面前乱放豪言,“灵州城防坚固,想打下来也不是很容易。就连韩冈也都说过——灵州难下。”

“韩冈说灵州难下,难道他不知道官军攻城的实力?霹雳砲都是他的发明,其他战具也有同样的威力,只要环庆路和泾原路将他们带在身边的工匠们都拿出来。让他们日夜打造,三五天的时间,足够造出将灵州城围成一圈的战具。”

王珪点点头:“韩玉昆行事,如今的确有点过于稳重。”

“韩冈已经不仅仅是稳重的问题了。西军将校皆曰利于速攻,可他偏偏要缓进。总不能说西军将校的见识加起来也比不上他一个。”

王珪呵呵一笑:“焉知韩冈不是自污?他不是被人说他跟种家来往密切吗?这时候反对激进,倒是能乘机洗脱。”

韩冈看起来是要洗脱过去强加给他的不实之词。而对于王珪来说,一直压在自己头上的污名也总算能洗清了。三旨相公和至宝丹,两个绰号无论哪一个都是让人心中不快。

“所以说他小器速成,难堪大用。世人碌碌,有几人可知相公辛苦。多有如韩冈者,横加阻挠。”蒲宗孟眼神闪动,“在下在城中,多曾听人说相公是固宠,保住现在的权势。又有谁知道相公一心是为了给陕西百姓一个长治久安。”

王珪长声一叹:“知我者,其惟《春秋》!罪我者,其惟《春秋》!知我罪我,在所不计。”

蒲宗孟起身,向王珪一揖到底,感动直至泣下:“后人当知相公为国事的一片赤诚。”

……………………

“自来有起错的名字,没有起错的绰号。王禹玉能一直坚持用兵,还不是希合上意。天子想要用兵,所有他支持用兵。若是天子厌武,他要是能为用兵说上一个字,天都能塌下来。”

“这时候抱怨就没意思了。”韩冈骑着马,侧脸对身边作陪方兴道,“还是等着看结果。”

今年前五个月,襄汉漕渠货运、客运的净收入加起来超过十二万贯,同时还有六十万石纲粮抵京,方兴上京述职时因而趾高气昂,底气十足。他在中书门下,就连户房检正都对他和声细气。

不过在韩冈面前,方兴绝不敢拿大,抵京的当天就特地在清风楼订了一个雅间宴请韩冈。在站位和观点上,也都紧随韩冈:“结果还不是那个样子?想赢除非老天帮忙。这一仗就不该打。”

“出战是没错的,但不该浪战。夺占银夏、河西,将党项人压制在兴灵之地。以官军的实力轻而易举,粮草不济的情况也会好很多。”

“龙图说得是。”方兴点点头。

韩冈是反对激进,并不是反战,不过在外面以讹传讹,说是韩冈反战。

对此在京城之外的民间产生了不小的波澜,不少人认为反战也有其道理,药王弟子都这么说,多半是掐指一算给算到了。眼下进展再顺利,最后结果不会好,药王弟子说的总不会有错。

但在士林和官场乃至在京城的百姓中,由于他们见识较广,对韩冈身上的光环所受到的影响较小,便是各有各的说法。一开始倒是有不少人因为韩冈在军事上的经历支持他,但随着战局的发展,官军的高歌猛进让越来越多的人转投阵营。

对于这样的谣言,韩冈也只能摊摊手,想辩解都难。不过他也不需要辩解,只要朝堂上清楚他的态度就行了。

“还得小心辽人。”方兴又将话题跳到了北方,“二十万辽军在鸳鸯泺不是来踏青的。”

“二十万或许没有,十万是肯定有的。耶律乙辛带着他们到鸳鸯泺也的确不是为了吓唬人。如果官军有什么不测,他肯定会动手。”韩冈对耶律乙辛的决断力看得很高,能把耶律洪基一家四口两代夫妇都做翻,心不狠手不辣是做不到的,“不论是土地,还是岁币,只要能从大宋手上要咬一块肉来,都能让耶律乙辛增加他在辽国国中的威望。”

方兴叹道:“耶律乙辛能从一介穷苦宗室,做到如今只差篡位的大辽之主,可算是世所罕见的枭雄了。有他在身后盯着,也亏王禹玉敢让这场大战继续打下去。。”

“官军抵达灵州城下的消息是前天传来的,但苗授抵达灵州实际上是在十五天前。而高遵裕的环庆军则是在十四天前,昨天传到京城……这么多天过去了,如果现在已经攻下灵州倒也罢了,若是攻不下,粮食也该吃干净了。”

“粮草难道当真运不上去?”方兴问道。

“你以为西贼诱敌深入,刻意拖长官军的补给线是为了什么?他们早已做好准备,也肯定会全力去完成计划,怎么可能让粮草顺顺当当的送到苗、高二人手中?在灵州城下,官军胜则大胜,败则惨败,留给他们攻城的时间最多也只有一个月……”

清风楼之前,韩冈勒住马。神色淡漠:“差不多该有个结果了。”

