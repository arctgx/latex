\section{第七章 惊闻东邻风声厉(上)}

【第一更,求红票,收藏】

四月下旬,天气越发的燥热起来。天空中寻不到半丝云翳,靠着地面的空气都是无风自摇,扭曲着远处的景物。

今天不知是怎么回事,常年不断的山风突然停了,转眼间就闷湿起来的空气,使得秦州城变成了一个巨大的蒸笼。韩冈终于明白,河谷这个地理构造,真要热起来,跟盆地也没有什么区别。

也不知是受到了地气的影响,还是天气暑热的缘故,路边的树上已经趴着不少夏蝉,不停的吵着。单调刺耳,如同拉锯的蝉鸣声,在人们原本就热得心烦意乱的心火上,又连着倒了几瓢油。

马也好,狗也好,往日在秦州的街巷上经常能见到的畜生,现在都是藏身在树荫下,躲避太阳的直射。而就在这不按节令来的暑热中,韩冈正穿着一身严严实实,结束整齐的公服,坐在道左的凉亭中——为了迎接高遵裕。

高遵裕是外戚,只要在京城,便经常能见天子。不过他虽然后台大,但身份相对于李师中和窦舜卿却不算高。他从西京左藏库使的位置上调来秦州,本官也不过一个阁门通事舍人。

一位从七品的通事舍人来秦州任职,李师中自持身份不会出来迎接,有着观察使本官的窦舜卿也不会去接他。倒霉的韩冈被抓了差,而王韶为了与高遵裕打好关系,也不辞辛劳的主动接下了任务。

这事说起来没有任何问题,合乎常理,但秦州官场如今是壁垒分明,其核心处便是河湟开边一事。本就是剑拔弩张的情况,突然间天子却派了一个外戚过来直接插手核心事务,李师中、窦舜卿对此无动于衷,反而显得事情不正常。

但韩冈现在被热得头脑发晕,即便李窦二人没有插手高遵裕的接待任务,让他感到十分惊讶,却没心思去细想为什么李师中对高遵裕这般冷淡,反而心烦地在抱怨着:“高提举可谓是先声夺人……人未至,声先至。通报他行程的急脚递从六天前开始,一天一骑,一日也不断。”

“玉昆,你是不是不喜欢看到高遵裕来秦州?”

“什么时候家国大事轮到外戚插手了!天子喜欢宦官、外戚这样的近臣,是乱政之始。”韩冈随口应着,前面王韶说的其实是他自己的心情,问话也是他真实想法的反映,不过韩冈的想法跟王韶一样,都不喜欢看到一个外戚来秦州。

士大夫们对于宦官和外戚,一个是生理的反感,一个心理上的厌恶,基本上都不会有好感,在这方面,不论是哪一派,士大夫们都是有志一同。

就如王韶,如果高遵裕不能在河湟之事上助他一臂之力的话,他是很希望世上没有这个人。而韩冈的想法就更直接,如果高遵裕是来帮忙的也就罢了,分功给他也是无可奈何下的唯一选择,但如果是来添乱的,那就最好有多远死多远。

“话虽是这么说,但历朝历代宦官、外戚干政的情况何曾少过?以仁宗之明睿,也有张尧佐惑乱国政,以章献之果决,犹有雷允恭动摇朝堂。”

“以冈之愚见,也只有察其言,观其行。先入为主固为不好,以观后效却是没错的。”

身为外戚,高遵裕的位置就是单纯的提举西路蕃部,除此之外,秦州的一应事务都不干涉。赵顼交给他的任务明明白白的是来分功,王韶和韩冈当然能看得出来。但经历过李师中、向宝和窦舜卿之后,他们要是还会以为天子派来的人,就是来帮着拓边河湟的,那他们的智商也就跟虫子一个等级了。

王韶和韩冈说着闲话,身上却是汗流浃背,心里都在后悔着没有带把扇子过来。就在他们越来越不耐烦的时候,一骑当先奔驰而来,带了王韶和韩冈期盼已久的消息,他们所等待的高遵裕终于到了。

远远的望见了一支车队,王韶和韩冈就走到了亭子外,在路边垂手等候。

高遵裕骑在马上,顾盼自豪。他虽说是外戚,其实也是世家子弟。他是开国功臣高琼的亲孙,真要论起家世,不要说韩冈,就是王韶也是差之甚远。自幼接受家中教导,高遵裕不论外形和气质,看上去都不差,跟普通的士大夫没有什么区别。

王韶拍马上前相迎,韩冈紧随在他身后。当高遵裕看到王韶后,便立刻勒缰止步,返身跳下马。而几十人的车马队列,跟着高遵裕停了下来,也不照规矩按顺序停在道路一边,而是就在官道当中停步,将整条官道全都占满。韩冈看着心中不快,高家的奴仆当真是霸道。

高遵裕和王韶显然有过一面之缘。老远就听得到他喊着,“子纯兄,自京城一别已是八年。多年不见,向来可好?”

“在下已经老了,也只有公绰风采不减当年。”王韶大笑着上前见礼,心中芥蒂也不露分毫。

“官家命遵裕提举秦州西路蕃部,初来乍到,事务不熟,还望子纯兄多多提点。”高遵裕说得谦逊,但只看他的家奴们的作为,怕是到了关西,就已是横行无忌。

“哪里!哪里!在下却是对公绰翘首以待。”

王韶和高遵裕正在交换着一些毫无意义的客套话,一阵急促蹄声由远及近的传来。

循声望去,一名骑兵急匆匆的从东赶来。只见他风尘仆仆满面倦容的样子,肯定是赶了不短的路。到了近前,看到王韶等人的车马,他也不避让,将马鞭挥了两下,就打算在车队中一冲而过。

“这是高舍人的车子,你敢动一动?”高家的管家立刻跳出来拦着他,并毫不客气的训斥着骑兵,他自入关西之后,作威作福的事没少做,也容不得有人敢轻视他的主子,“来人,把这个不开眼的家伙拖下来!”

“住手!”韩冈连忙叫道,“此人必有军情在身,事关重大,不是故意冲撞车队。”

“出了何事?”王韶举起了他腰间的银鱼袋,证明自己的身份,他本是为了迎接高遵裕,才把公服以及所有的饰物都穿戴上,没想到就这么派上了用场。“本官是秦凤经略司机宜文字,这位是阁门通事舍人。与秦凤有关的军情我们都有资格察看。”

有银鱼袋作证,那名骑手也不敢不信,只看王韶、高遵裕的样子也不像作伪,便直言相告:“小人不敢欺瞒官人。小人今次赶得路急,不是因为他事,而是两天前环庆李经略遣将攻打闹讹堡,但被西贼埋伏于道左,以至于全军覆没。惨败之后,西贼号称十万,随即兵犯环庆!小人就是奉知州之命来请援的。”

“什么?!环庆大败?!”王韶顿时大惊,当即怒道:“李复圭这是看着绥德和古渭眼热,想着为自己争取边功!这下自己败了不说,还要拖累他人。”

李复圭这下却是偷鸡不成蚀把米,连高遵裕都变了脸色骂着:“没有金刚钻,就别揽瓷器活。李复圭办得蠢事,整个关西都要给他乱了!”

韩冈尚且保持着冷静,问着王韶:“不知李复圭的为人如何?”

“眼高手低之徒,虚言夸饰之辈……而且没有担待!”看得出来王韶对李复圭的评价很低,但最后一句是最致命的——这是对李复圭的下属而言。

“在李复圭的手底下做事,可就要提心吊胆了。”韩冈摇着头,为李复圭的部下担心起来。突然间又想起一事。

韩冈记起来了,种詠不就是在庆州吗?那位种家四郎,也就是种谔的兄长,种建中的四叔,好像就是做着庆州东路监押。今次环庆军惨败,不知会不会连累到他。

种家最近的确流年不利。

种谔在绥德被压制,郭逵宁可用燕达这位相对于种谔而言,太过新嫩的年轻将领,也不用已经证明过自己能力的种谔。

而环庆是一路,庆州军惨败,知环州的种诊也难逃干系。虽然罪名到不了他身上,但短期内要晋升也是没希望了。

剩下的种家老大,小隐君种诂,他在原州已经有两年还是三年,韩冈只听说他是苦劳多,功劳少,没有什么光彩的事迹。而且种诂曾经为了帮父亲种世衡辩功,得罪了当朝宰执,他争功的名声在外,没有哪个士大夫会喜欢种诂这等武夫。在世间所传的三种中,种诂晋级横班的机会是最低的。

韩冈有心跟种建中多结交,只是前些日子,王舜臣去延安走得太急,韩冈没来得及托他送封信过去联络感情。反倒是今次王厚、赵隆入京,韩冈就让赵隆带了好几封信走。

种家的事可以放一放,韩冈关心不了那么多。而李复圭如何也并不重要,现在的问题是环庆路的失败会对河湟开边带来什么样的影响——情况应该不会好。正如王韶前日所叹,要做好一件事可真难。

不过韩冈的特长是从黑暗中找寻光明的一面,凡事都有两面性,祸福相倚是韩冈贯彻始终的看法,而他的老师张载也秉持同样的观点,只是将事物的两面性说成是气之阴阳并存。

“李复圭兵败,看似会让天子忧心日后贪功之辈日多,使得边塞不宁。但他这一败,却也让天子和中枢为之警醒,不会再奢求能各线齐进,而会将支持集中在几个已经证明过能力的地方……塞翁失马,焉知非福,这说不定也是件好事!”

高遵裕与王韶见面后,还是第一次注意到他身后的韩冈,听着韩冈一番言辞,他动容问着韩冈:“不知君乃何人?”

