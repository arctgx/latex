\section{第19章 虎狼终至风声起(上)}

【祝各位书友上元节快乐。上元一过,年节就算过去了。接下来的更新终于可以恢复正常了。】

慷慨的最高境界是慷他人之慨,韩冈两句话就把冯家的家产全都送了出去。前面韩冈的确在诉状上署了官名,此时又穿着公服站在堂上,摆出一副强龙过境的样子,让凤翔府的官员都想给他点颜色看看。但那不过赌口气而已,现在韩冈一块大饼送上来,又有哪个还会把气堵在胸口?皆在心中暗赞韩冈识作。连原本收了冯家兄弟贿赂,而跟韩冈过不去的刘节推,也是迟疑了起来。不再抬杠,跟着就趁李译上堂,就转身返回自己的座位上去。

当天的会审很快就结束了。知府李译本就是身体不适,勉强支撑着出来了,虽然看着韩冈身上的青色官袍觉得扎眼睛,却也只说了两句就匆匆退了堂回去将息。而陪审的陈、刘则对案情皆是不置可否,也跟着起身。‘三堂会审’的大阵仗,才开个头,就偃旗息鼓,暂待后续。

冯家三兄弟见状,便是冷笑一声。在他们看来,韩冈靠亲笔写的诉状辛辛苦苦拉起的阵仗就这么没了,根本就是大败亏输。下次开审,他难道还能再穿官袍上阵?真的如此,几次下来,他就要成官场上的笑话了。而且开审一次,就要上下打点一番,比起身家来,他们三人可比老四强得多。

冯从礼、冯从孝嘿嘿冷笑着举步就走,而冯从仁却面朝着冯从义,眼睛则斜睨着韩冈,嘲笑着:“如何?!有本事再来下一次。”

李忠和冯从义的脸色顿时就阴沉了下去,李信拳头一攥,将视线转向韩冈,却发现自己的表弟正淡然而笑,眼神却仿佛是从高处投下,看着脚底下的一场闹剧。

冯从仁见韩冈几人都没有反应,心中大畅。像是打赢一场战斗,大笑着转身跟着两个哥哥出门,好转回去找刘节推道谢。但几个衙役却在大堂门口处横着拦了过来,领头一位班头谦卑的笑道:“大府尚未定案,三位员外怎么能走呢?”

“什么?!”冯家三子登时又惊又怒。

“三位还问‘什么’?”班头假笑着,脸唰的一下板起,森然说道:“三位可是弑母之罪啊!不待确认无罪,谁敢放你们离开?!”

班头说着便使了个眼色,便立刻有六名公人从身侧左右各自架住了冯从礼三人。他们脸色开始泛青,惊望向韩冈,那唇角边地浅浅笑意,落入冯家三子眼中时已是狰狞无比。直到此时他们方才恍然大悟,领会了韩冈的险恶用心。

大声高喊着冤枉,冯从孝用力挣脱了押着他的两名衙役,连滚带爬的向快要走出门的刘节推那里跑过去。不过砰砰两声响,两名衙役手上的水火棍呼啸着挥下。被包了铁皮的棍头敲到了小腿,冯家老二惨叫声起,滚倒在地上。接着就跟他兄弟一样,被横拖竖拽的硬扯了出去。而他们所仰仗的刘节推,却眼皮也不抬的小声的跟陈通判说些什么,一起从堂后小门离开,好像什么也没看到。

见到了闹剧的主角们终于退场,韩冈这才收起脸上的笑意,领着自家犹在云里雾里的舅舅和表兄弟回身欲出。堂中剩下的公人都是向他欠了欠身,表示自己恭敬。

财帛动人心,冯家的家产已经让凤翔府城中的大小官吏垂涎了许久,前日冯家老员外病死后,三兄弟没有争夺家产,让他们失望至极。而韩冈此时却带着失踪已久的冯家老四出现,先给三人栽了个弑母的罪名不提,还明着说要把官司磨个二三十年,等于是把冯家的家产双手奉上。虽然在这其中他们这些衙役拿不到大头,可各自少说也能分润个十几二十贯。

韩冈四人步出大堂,冯从礼三人的喊冤声尤远远的传入耳中。今天的事峰回路转,李忠只道是韩冈的诉状起了作用,心中解气得很,大赞着韩冈:“还是三哥儿有能耐,一封诉状就把那三个畜生送进了大狱。”

‘哪有这么简单!’韩冈微笑着转过头看向冯从义。他的表弟正望着冯家三子被拉走的方向。

“担心他们在狱中会吃苦头?”韩冈问着。

“不担心。”冯从义收回视线,摇头道:“不把三位哥哥的身家全数榨出来,他们都会被好吃好睡的养在大狱里的。”

韩冈笑容变得更明显了一些,他这个表弟也算聪明了,至少看出了后续……就是不知看没看出自己到底是用什么手段才打动了这些贪官污吏。不过堂外却是有人看得清楚明白。

慕容武就迎在门外,他的长兴县主簿的身份,让他进不了审案时的府衙大堂。一直等到韩冈出来,他才忙上前,笑道:“一直都听说玉昆你在秦州,是翻手为云、覆手为雨。可只是口耳相传,心中犹有犹疑。只是今日一见,果真是名不虚传。”

“思文兄谬赞了。些许小事,举手之功。”韩冈显得很平淡,他去京中的时候,连国家大事、朝廷新政都参合了一脚,现在用上手段对付起三个土财主,哪有不手到擒来的?他又向慕容武道歉,“昨日从舍舅和表弟处惊闻先姨母之事的来龙去脉,便当即写了诉状。本是想过向思文兄求助的,后来小弟转念一想,冯从礼三人不过是些个土豪劣绅,手到擒来之辈,何须兴师动众?便不敢惊扰到思文兄和陈通判。”

慕容武凑过来,压低声音笑道:“也就是玉昆你才能举重若轻,换作是他人如此行事,怕是要吃个大亏。冯家可是送了刘节推整整两箱好处,少说也有千贯。”

韩冈但笑而已,却不接话。

“好了,”慕容武见韩冈不打算再提这个话题,便转过话头,问道:“不知玉昆接下来行止如何?”

“该回秦州了。这里有舅舅在盯着,下次再审此案,也不需小弟再赶来凤翔。”韩冈说着,回头看了看冯从义,这位小表弟识趣,离得远远的。韩冈会心一笑,也压低声音对慕容武说道:“先姨母的坟茔还请思文兄多多看顾,开棺验尸时,望能保证骨殖不被毁损。”

“玉昆放心,愚兄理会得!”慕容武猛点着头。

百善孝为首,开棺不是一件小事,做得岔了,做儿女的就要被指脊梁骨。有时父母的死明明有怨情,但子女为了不惊扰到父母遗骸的安宁和完整,往往会拒绝官府开棺验尸。虽然这种做法在韩冈看来很可笑,但却是儒家社会的现实。

不过今次为了证明韩冈诉状上的言辞,韩冈四姨的棺椁肯定是要被打开的——韩冈并没有主动撤诉的打算——这时若无人关照,一点陪葬品怕是都要被掳走,连尸体说不得都要受辱。

慕容武停了一下,却又笑道:“大府如今身体有恙,甚少理事。无论今后知府之位是换人还是延任,今次一案,少不得先拖个半年下去。”

听到慕容武这么说,韩冈算是放心了,能有点时间缓冲是最好。等他把冯从义弄到秦州去帮自己把摊子做起来,再有这个消息传来,不然说不定会因为此时,心里会有些芥蒂。而他娘韩阿李那里,也要先打些预防针。

当天韩冈做东,在凤翔府的一家有名的酒楼上置办了酒席,请了陈通判和慕容武入宴,表示一下感激之情。韩冈行事的老练让陈通判感到惊叹,昨天夜中还生着韩冈的气,今天收到邀请,便应承了下来。

几人喝了一夜,到了第二天,韩冈带着李信和冯从义一起返回了秦州——慕容武已经说过,此案半年内开审的机率又不大,冯从义当然要投奔韩冈,以便大树底下好乘凉。李忠虽然也想去见一见自家的三妹,但原告的几人不能都一股脑跑到外地去,他必须盯着案子,也只好作罢。

回到秦州,韩冈带着冯从义,到了自家拜见爹娘。听说了四妹的冤死,韩阿李跟冯从义抱头痛哭了一场。哭完后,韩阿李对儿子道:“三哥,你四姨就剩这一个独苗了,你自己看该怎么做吧!”

“表弟不是读书做官的料。”韩冈说得坚定。他在路上跟冯从义谈了许多话,算是了解了他究竟是有着哪一方面的擅长,而结果,让他喜出望外,“不过在货殖之术上,表弟倒是家学渊源。”

次日,韩冈回去见了王韶、高遵裕。私下里又跟王韶父子把自家的事说了一通,他们一同唏嘘了一阵,又为韩冈的手段拍案叫绝。接下来,韩冈就为了这段时间丢下的工作忙碌着。

而过了几日,王厚却面色古怪的找了过来:“玉昆,凤翔府出事了!”

韩冈心中一跳,急问道:“出了什么事?!”

“凤翔的李大府前几日病死了。”王厚成功的诈了韩冈一下,觉得很有趣,便哈哈笑了起来,捧腹道,“玉昆你刚到凤翔走了一圈,李大府就死了。下回你再往外州去,那里的知州知府,都得要先念上一卷金刚经再说了。”

韩冈嗤之以鼻:“胡说!天天有人死,难道都跟我有关,阎罗王还有地藏王菩萨都没这本事。”

王厚又道:“不过李大府死时,据说有群蝶起舞,却是个祥瑞。”【注1】

“你真是闲得慌。”韩冈摇头叹了口气,又埋首于公案。

“等郭太尉来了就闲不了了。”

韩冈被王厚的话带起来心思,眼望东方,‘郭逵怎么还不来?’

注1:张舜民《画墁录》:李译谏议知凤翔卒,有蝴蝶之翔。

