\section{第五章 九州聚铁误错铸(五)}

“怎么会没说好?别小瞧你三哥,这么大的事,难道会不跟王中正先敲定。有人跟他一起进京,该说的早就都说了,王中正也托他回来向我传话。”韩冈笑着解释了两句,提声叫了一个家丁进厅,“去唤胡义来。”

“胡义……就是韩义吧?他进京了?”王舜臣问道。

“跟着王中正一起上京的,不过昨天城门落锁前早一步进了京城。”韩冈瞥了一眼王舜臣,“昨天夜里我倒是想找你说一说这件事……”

王舜臣摸摸头,干笑道:“三哥你让俺看的那几本兵书太难了,根本都不是让俺们这等武将看的,看着看着就困了。本就是给文臣看的文章,根本不是上阵能用。照俺说,还不如李家哥哥写的交趾作战心得。就是不识字,让人念出来,也不需要多解释,下面的小卒都能听得懂。”

韩冈摇摇头,苦笑了一下。王舜臣的话不为错。

如今世间流传的兵法,说战略的地方太多,对于战术细节上的问题讲得太少。就是军事百科全书式的《武经总要》,也同样显得过于简略。

这方面的问题,是一直为韩冈所诟病的。他所参与的与军事有关的遍敇和文案,则是一贯的不厌其烦。比如他所编订的《军中卫生条例》,如今通行,于世的修订版多达四万字,而当初在广西增修的南方版本,因为地理气候和疫病种类的不同,则又多出了两万字,数十条款。

但韩冈也不能放纵王舜臣:“话是这么说没错,但这些兵书里面都是蕴含了真知灼见,多读一点没有坏处。若是嫌太过简略,你可以补充嘛。春秋和左传之间的关系,记得我当初跟你说过——关羽和郭逵都喜欢读左传的——还有,既然你看过了南征纪行,难道我那位表哥被调任河北之后闲下来后在做什么你不知道?当初你也不是写了一篇横山作战的心得吗?”

王舜臣抓了抓脖子,他的这位韩三哥当老师当上瘾了,让李信写南征作战的心得,让赵隆写平茂州的心得,让自己写在横山中作战的心得。虽然知道其中的用心很深,但在油灯下咬着笔杆子的日子,实在不是人过的。

李信是兴致盎然的在写、在学,上次来信,还说找了个先生教授春秋,也不知是不是打算日后做个教书先生。而赵隆则是口述经历,让自家的幕僚做记录,然后自己亲自整理。他们都不敢把韩冈的吩咐不当一回事,王舜臣同样也不敢,一样是费尽了心思,将作业给完成了,但他实在不想做第二次,上阵杀敌,与贼人勾心斗角都比这个简单千百倍。

韩冈看得出来王舜臣心中的抵触,叹了一口气,道:“回去后好好想想。不求你苦读不辍,只求能有会于心,与现实做个印证。之前让你写的那些心得,也不是要你写出多好的文章,只是让你有条理的记录而已。在天子面前、在同僚面前、在下属面前,都是有好处的,总不能只凭箭术做依仗吧?”

韩冈几乎是苦口婆心的规劝,王舜臣也不再装傻充愣的推搪,很诚恳的点头:“三哥,小弟明白,回去后会认真读书的。”

韩冈也不多说什么了,这件事还要靠王舜臣自己自觉。胡义马上就要到了,对王舜臣说的话,也不好当着他面说。

胡义他本来有另一个名字,只是在投身韩家为庄客后改名做韩义。之后因功授官,也仅是恢复原姓,名字却没有改回去。

前一科犯了事,改个名字重新参加科举的事也是有的。前科状元刘几在欧阳修第一次知贡举的时候,因为文风被欧阳修锁厌弃,故而被黜落,甚至还张榜贴出,给了个大纰缪的评语。

等到下一次欧阳修再次知贡举,刘几改了个名字再来考,特意改成了欧阳修喜欢的文风,以其文采便被擢为第一。在揭糊唱名的时候,登记名字叫做刘煇。欧阳修拿着这篇文章向朋友大加推崇,因为又录用了一个出色的弟子。之后方才知道,这一位其实就是他一直拿出来当反面教材的刘几。

而有些官员,因为得罪了高官,怕影响前程,改名的情况也为数不少。刘义乃是广锐军出身,旧名可是留了底,改回去只会自讨苦吃。

除了胡义之外,还有两个跟着韩刚立了功得了官的亲随,他们跟胡义一样,都是广锐军出身。尽管在投入韩家门下时,没有改名换姓,但在得官之后也都聪明的改了名。

胡义得了韩冈相招,很快就到了。也是很年轻的一个人,看模样就是精明干练,等他行过礼,韩冈吩咐道:“你把王都知说的话再说上一遍。”

胡义拱了拱手道:“王都知只是在过潼关的时候,跟小人说过一次话,问了小人的出身来历,还有投到龙图门下后做了什么才得官。之后直到进了城,才让小人来向龙图道谢,说上次送来的茂州生药甚好,他很喜欢,天子更喜欢。近日听说凉州的马鞍好,不知龙图是否能带上一具,以便能献与天子。”

听过胡义的转述,韩冈问,“明白了没有?”

王舜臣竭力抑止心中兴奋,点了点头:“哪里还能不明白?!”

王中正的话一点都不委婉,没有弯弯绕的说辞,王舜臣又不蠢,怎么可能听不明白。

前一次,靠着韩冈的推荐,王中正以赵隆、苗履为部将,一举平定了茂州叛乱。这一次也是一样要借助韩冈的力。而韩冈推荐的,正是王舜臣。当然,这个胡义也多半一样能沾了光,要不然王中正也不会细问他的身份来历。

“王中正会愿意分兵凉州,多半也是知道这一仗不是那么好打,灵州的功劳也不好挣。六路约期齐集灵州城下,说着简单,但实际上只要带过兵,就知道这样的计划根本是一张废纸。前后差个几天,就能有各个击破的机会。”

王舜臣皱眉道:“六路伐夏,其中两路合兵,都能与党项人一较高下。所以王都知才会兼领秦凤和熙河两路,而泾原路也要受环庆路的高总管节制,河东路的兵马同样是得配合鄜延路的进兵。说是六路,等到杀到灵州城下,其实等于是三路。”

韩冈摇摇头:“不能这么算。河东路地理上相隔太远,从一开始就只能跟着鄜延路。而熙河、秦凤的主帅是王中正一人,也没有可争的,到了黄河肯定会合兵。唯独泾原路和环庆路,仅仅是节制而已。从没有说,泾原路要等环庆路,或是环庆路要等泾原路。到时候,说不定就会被各个击破。”

“苗总管是高总管的人。”

“都是一路兵马副总管……谁是谁的人?眼下他们可都是平级的大宋臣子。”韩冈笑问道,其实王舜臣自己都是说得犹犹豫豫,原本是老实听话的下属,但地位高了之后就平起平坐的例子实在太多了,哪个没见过。

王舜臣振奋起精神:“有板甲、斩马刀和神臂弓护身,又有飞船监视远近,西夏的铁鹞子拼不过官军。”

“党项人爱用诈术,从继迁开始便是如此,也就是前些年气焰嚣张时才会蠢到冲击军阵,眼下可不会再犯傻了。”

在冷兵器时代,一副上好的铁甲,对士兵的战斗力能起到倍增的作用,再加上斩马刀、神臂弓的普遍配发,让列阵之后宋军步卒能轻易击败大辽和西夏两国的精锐部队——但这要加个前提,得让辽夏两国的骑兵自己犯傻往军阵上撞,而不是凭借优越的战场机动能力直接绕开宋军军阵。

同样的道理。此时的西军,拥有绝不逊于契丹的宫分、皮室那样的精锐部队。任何两路合力,都有跟西夏一较高下的实力。纵使鄜延路和河东路因为瀚海阻隔不能及时赶到灵州,只凭剩下的四路,也足以堂堂正正的击败西夏——只可惜,要想让党项人打一场堂堂正正的战争,除非兴庆府上上下下都变了白痴。

王中正来了又匆匆走了,离开的时候顺便将王舜臣一并带走。说服了天子让王舜臣戴罪立功,是王中正给韩冈的人情,而韩冈的回报,就是让王舜臣帮他夺下凉州。

不过韩冈和王中正私下里的密约并不仅仅是对王舜臣说的那些。

打通了丝绸之路的主线,之后还能有开拓西域,恢复汉唐旧疆的好处。到时候王中正若是想过一过班超、张骞的瘾,韩冈也是要出来支持他的。

话说回来,韩冈本人也乐于见到西域被收复,丝绸之路重新掌控在中原王朝手中。他并不在乎究竟是谁夺取的,就是阉人也一样。

在党项人控制河西走廊的时候,由于盘剥太甚,许多回鹘商人都改走丝绸之路的南线,从青海湖畔绕行,董毡继承父业之后,很长一段时间,他的财政都是靠回鹘商人的过路费来支撑——不过如今的董毡,已经利用棉花、油料乃至可以替代食盐的咸鱼发家致富,对于过路费的依靠小了许多。

有了河西走廊,丝绸之路的收益还是小事,对熙河路的帮助却是实实在在的。

随着天气一天比一天更热,种谔领军回到出发地,战争的筹备也还在继续,而到了快入夏的时候,来自辽国的使臣带来了一封强硬的照会,与之同来的还有二十万铁骑抵达鸳鸯泺的消息。

“这可不是夏捺钵该来的地方。”韩冈在崇政殿上说道。

