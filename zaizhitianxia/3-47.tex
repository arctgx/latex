\section{第18章 诸士孰为佳(上)}

熙宁六年三月初六,乃是礼部试举人参加殿试的日子。

位于宫城东南的集英殿,这时早已经打扫干净。四百零八张桌案在大殿的东西两端排得整整齐齐,只留下殿中央空着,以供考生们进来之后叩拜天子之用。

在每一张桌案的左上角,与礼部试一样,都贴了著有姓名籍贯的纸条,以防考生混作一团,失了朝廷体面。

按照多少年来的惯例,殿试贡生们的座位排列顺序,都是照着他们在礼部试上的名次来的。离着天子越近,这名次就越高,离得越远,自然名次就越低。

李舜举拿着名单,一个个对照着桌上的姓名籍贯。从东头最近陛前的礼部试头名邵刚,一直查验到位于大殿东南、西南两个角落里的慕容武和孙中。

虽然昨日已经有小黄门对照过两遍,李舜举自己都照过了一遍,但李舜举一向知道宫廷中什么事都可能发生。

既然刚生下来的健健康康的皇子,第二天就能暴毙,从仁宗皇帝开始,宫中多少年来只见公主,生下来的皇子却一个都养不活;那么昨天布置好的一切,今天起来全变了个样子,也没什么好奇怪的。不再一次亲眼对上一边,他怎么可能放心得下来?

用了小半个时辰,提着灯笼,将每一个桌案都对照过,李舜举最后站在大殿门口,松了口气,点了点头。一切就绪,全都已经准备好,就等着天子和考生们来了。

………………

这时候,才不过是卯时初。

天色还是黑沉沉的,尚能看见天上的成千上万的繁星。但就是这个时间,韩冈与所有的士子都已经来到了皇城外的左掖门处。

今天是最后一道关口,只有顺利通过了,才能够得到进士的资格。但左掖门前的气氛,却是比当日国子监前要轻松许多。每个人都知道,今天只要不犯蠢事,进士已经稳拿稳了。

贡生们小声谈笑着,等着宫门打开。但也有人凝神静气,不与他人多言语。

“这些人多半是争状元的。”慕容武低声对韩冈说着。

韩冈点了点头。礼部试中高高在上的余中、邵刚,都在这些神情严肃的士子之中。

不过韩冈认识的另外一个准备争夺状元的贡生,却没有学着邵刚和余中,而是挤了过来,“玉昆兄,原来你已经到了!”

韩冈脸上浮起了应酬式的微笑:“不意致远兄也到了!”

韩冈认识叶涛。第一次见到他时,是在国子监门前,就是放声大笑的那群人中一个,看起来各个自信非凡。可当时叶涛身边聚集的那些个士子足足有十五六个,今天却是只有叶涛他单独一人到场。

第二次见到叶涛,则是在两天前,应邀赴王雱邀请,在状元楼上,由王雱亲自向韩冈介绍的,不为别的,只是为了让亲戚之间互相见个面。

这才是最让人惊讶的。

韩冈跟叶涛现在算是姻亲——尽管中间隔了一层——韩冈与王安石的女儿结了亲,而叶涛则是韩冈岳父的亲兄弟,也就是王安国前两天刚刚招来的女婿。

叶涛的文章写得很好,但韩冈并不喜欢他。不是因为别的,而是叶涛的说话,一直带着居高临下的味道,为自己的文采而骄傲。傲王侯,慢公卿,这是真正的士人所为。但傲慢到自己头上,韩冈的气量虽不差,不至于因此而动怒,但要让他贴上脸去迎合,却也是休想。只是在表面上,他的应酬还是到家的。

王雱并不是钝感的人,前日宴会后,便问着韩冈对叶涛的看法。

韩冈摇摇头,回了一句:“无他,只是今日乃有子迟之问。”

子迟,是孔子的弟子,七十二贤人中的樊须。在论语中他曾向孔子问何为‘知’,而孔子的回答是‘敬鬼神而远之’。

自家亲戚,打脸不好,我干脆离你远一点好了。

王雱先是扑哧一笑,而后便是摇头一叹,“其实叶致远也不是故意如此,只是性格使然而已……既然如此,下次就不带他来见玉昆了。”

韩冈不欲于叶涛打交道,到了的时候见到他在前面,却留在后面不上去打招呼。但叶涛眼尖,不知怎么看到了韩冈,就挤了过来,打过招呼,便道:“小弟在前面隔得甚远,现在才看见玉昆兄,还望勿怪!”

韩冈脸上浮起了真挚的笑容,“不敢。韩冈也是眼拙,没有看到致远兄。”

两人哈哈哈的互道着没关系,然后交换着天气之类的寒暄话语。

韩冈压着心头的不耐,沉下心来应付着叶涛,这时候,几声钟响从宫中传出,宫门终于开了。

当值的阁门使走了出来。

不用他多话,考生们按着名次先后,立刻排起队来。前日太常礼院的礼官,已经向这四百零八位贡生们教导了进宫面圣时改有的礼节,没有哪人敢于错上半点。

叶涛连忙挤回去,他礼部试的名次很靠前,比起一百五十七名的韩冈要强得多。韩冈冷淡的看着他的背影一眼,在自己的位置站定,不再去想这只烦人的苍蝇。而慕容武,此时早就回到了最后面。

从左掖门进了宫中,考生们被阁门使领着直趋集英殿。周围有上四军的士兵护卫监视。旁边还有监察御史盯着,没有人敢于做出任何失礼的行为,也不敢抬头张望。各自看着脚下的路,盯着前面人的脚后跟,向前疾步走着。

一路了殿上,宫廷韶乐从集英殿中回响。天子还未到,但今科的考官已经都在殿中等候。

韩冈作为四百零八人中唯一的朝官,却还是第一次来到宫殿之中。虽然低垂着眼,装出一副谨守礼仪的模样,却也不忘用眼角余光打量着周围的布置。

寻常而已。

殿中陈设的器物,装饰的布幔,梁柱的油漆和彩绘,都已是老旧。根本比不上有着无数善男信女捐献的大相国寺、开宝寺等大丛林的主殿。

也只有规模,算是勉强让人不会小觑了皇家的体面。尽管集英殿只是诸殿之中,规模排在后面的殿阁,但二十多丈的宽度,十余丈的进深,还是让每一个贡生心中震撼不已。也就是韩冈眼界高,见得多,没放在心上。

两排有数人合抱粗细的梁柱,在大殿内,隔出了东西两厢。两厢之中,排满了桌案。桌案都是旧的,跟国子监差不多。而且桌案都不高,只有一尺多,不到两尺的样子。给考生们准备的是蒲团,而不是马扎或是杌子。

‘这是要跪坐啊!’韩冈先是暗骂了一句,又庆幸自己幸好已经习惯了,不然可是要出丑。

在考官们的监督,四百零八名贡生们在集英殿中央排好了方阵,打头的三人是在礼部试排名最前的三个。

几声净鞭响过,乐声止歇。在礼官的叱令下,所有的考官和考生,无一例外的都跪拜了下去,静静的等着天子的到来。

寂静的大殿中,韩冈低着头,研究着大殿地面上作为铺垫的砖石。虽然是烧制出来的砖石,却是泛着幽暗的金属光泽,也难怪外界传言说,宫中使用金砖铺地。

如果是汉代,殿上都是铺着地板,进殿要拖鞋。但到了南北朝之后,周时的礼节就已经开始变了。到了现在,已经可以穿着靴子走在大殿上。

连串的脚步声终于从前方传来。

并不吵闹,很整齐,静悄悄的响起,又静悄悄的结束。

然后礼官的又吊着嗓子半吟半唱的发号施令。

三跪九叩。

向着当今的大宋天子,统御亿万兆民的皇帝,叩拜下去。

一拜一起之间,都能看着殿上的人物。但隔着有些远了,光线又很昏暗,看不请坐在御榻上的赵顼是个什么模样,只是站在陛前,一开始并没有出现的一个高大声影让韩冈很熟悉。

是他的岳父王安石。

看起来王安石是跟着天子在殿后等待,然后与天子一起出来。不仅仅是王安石,还有两人也在列。不出意外,应该是参知政事的王珪和冯京。

一连串事先已经被礼官传授的礼仪之后,考生们终于可以落座。在内侍们的引导下,一盏茶的功夫,就已经各自就位。

然后今次的考题题目便出来了。

‘古之明王,求贤而听之,择善而使之。法不足以有行也,改之而已;人不足与有明也,作之而已……以守位则安,以理财则富,以禁过则听,以讨罪则服,以交鬼神则飨,以来蛮夷则格,以上治则日月星辰得其序,以下治则鸟兽草木得其性……朕夙兴夜寐,心庶几焉,而未知所以为此之方。子大夫其各以所闻,为朕言之……朕即位於兹七年,行义政事之失,加於天下多矣。往者或不可救,来者尚可图也。以所见言之毋隐。’

果不其然,别看今次皇帝亲自出的考题洋洋洒洒数百字,本质上就是一句话:地方上的行政阙失,可以放胆直言。

这就是策问。

