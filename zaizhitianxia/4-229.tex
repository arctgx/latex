\section{第38章 岂与群蚁争毫芒(二)}

午后的小睡之后,严素心打着哈欠坐了起来。钗横发乱,斜靠在方枕上动人身姿中满是慵懒。

房内的空气中带着湿润的水意,严素心挽了挽散乱下来的青丝,问道:“才下过雨?”

“才下过,很大的雨,转眼就过去了。”

贴身小婢说着,就将挂在窗前的竹帘拉起了一半,然后严素心就看到了窗外的一丛纤细的湘妃竹。

夏日的阳光透过修长如剑的叶片,洒在竹枝上,枝叶上的滴滴水珠如同宝石一般闪耀。被遮掩的光线下,娥皇女英的斑斑血泪所染成的紫黑,与斑驳的树影交织在一起,如同落在澄心堂纸上的一滴浓墨,全然都晕了开去。

方才刚刚结束的一阵骤雨,让从门窗处刮进来的风清凉了许多,让人难耐的暑热也散去了。

襄州属于京西路,可偏偏却是南方的气候。一直以来都在西北的严素心住得并不习惯。

她曾听韩冈说起过,在过去,襄州属于荆州的范畴,甚至是荆州的核心,不过如今却是属于京西,是荆襄、乃至岭南通往中原的门户。自家丈夫这些天来,就是为了让门户之后的通道能更为畅通而奔波劳累。

严素心坐到梳妆台前,让贴身小婢帮她梳理着头发。拿在手上的新磨铜镜中,照出是一张正处在最艳丽的时刻的如花玉容。纤细的手指抚过镜面,从深如潭水的眼眸,到挺直的鼻梁,再到,就是还有些模糊。

前些日子,她的丈夫还让人辛辛苦苦做出了一幅水银镜来。银亮亮的,的确是照得纤毫毕露,但没几天镜面就给磨花了,再排不上用场。王旖笑他说水银镜就是个样子货,又贵又没用。但她们的丈夫却说,若是有了透明的平板玻璃护住,就能把铜镜给砸了。

听都没听说过的平板玻璃,还不知是什么时候才能造得出来。要像他要造的铁船,要是十几二十年的功夫,那就又是笑话了。世人都说韩龙图一言九鼎,但在严素心心中,却是常常说话不算话。

本来出去的时候,说好最多十天半个月就能回来,但现在三个十天过去了,却还不见人影。

“冤家,怎么还不回来!”严素心喃喃着。

正在为严素心编着发髻的小丫鬟停了手,“娘,说什么?”

“……没什么。”严素心脸红了一下,放下了铜镜。

夏日午后的转运使公廨的后院是安静的,从窗外传来了读书声。那是设在西院的蒙学,韩家排行在前三的儿女,已经就学读书。上午一个半时辰,下午一个时辰,三刻钟一节课后,就能休息一刻钟,课程和进度都是韩冈安排的。

严素心知道,对于小儿开蒙,韩冈一向放在心上。并不是只看到自家的儿女,而是放眼天下,说是要教化万民,为世作则。

要想教化万民很简单,就连严素心都知道,让更多的人读书识字,明了儒门大义。但具体要怎么做,很多人无法回答,而韩冈给出了答案,一个类似于商人的答案,就是降低学习的成本,让更多的人能读得起书。

不过降低成本,不代表粗制滥造。准确无误的教科书,以及有足够水准的蒙学教师。至少要保证这两样,才不会将‘郁郁乎文哉’变成‘都都平丈我’。

外面卖的粗制滥造的书本,加上连句读都读不通的庸师,自然是误人子弟。而且误人子弟之后,连改正都难,要不然也不会有‘都都平丈我,学生满堂坐。郁郁乎文哉,学生都不来’的笑话。

在保证质量上要降低印书的成本,难度不低,韩冈一时间也没有办法。但他还是找到了如何降低书写的成本,还有加强讲学效率的办法——这是严素心亲口听韩冈说的,当时说这话时,韩冈充满了自信。

木匠打造器物时,总少不了要在木料上标线作记,从古至今都用麻线墨斗来弹墨线。不过韩冈却弄出了炭笔,用粘土和石墨做成的炭条,可以在木料上划出清晰的印记。而用小小如枣核一般的炭芯,插在细竹管上,更可当做笔来使用。工匠使用方便,就是普通士人出外,也不用再带着笔墨纸砚全套,只需要笔和纸。

还有给先生用的代替纸张的黑色木板,用白垩烧结的粉笔,在挂在墙上的黑板上写字,可以更清楚明白的给每一位学生讲学,而不是简单的口述,让弟子去记录。黑板粉笔,严素心很早以前就看到成品了,只是现如今才被当成加强蒙学教育的一个部分给整合起来。

严素心倒没有韩冈放眼天下的想法,只是若能把自家的儿女都教出来,日后韩家也算是安稳了,她也能安心了。

对着镜子,仔细整理着妆容,外面突然一阵喧闹,然后严素心就听着一片声在喊,“龙图回来了,龙图回来了。”

听到家人赶过来的通报,严素心三两下将妆草草画好,急急忙忙的就出来了。王旖、周南和云娘很快也都前后脚的到了堂屋中,蒙学中的读书声也停了。

全家人都在等着一家之主,但韩冈过了半个多时辰,才从前面的衙署回到了家宅中。

一别经月,韩冈瘦了些许,也黑了些许。

或许是真的是劳碌命的缘故,在严素心的记忆里,她的丈夫脸上的肤色,总是脱不了晒出来的黝黑。韩冈并不是王相公天生的黝黑,如果能养尊处优,也能如普通书生一般白净,这一点,严素心作为枕边人是能确定的。可惜偏偏天南地北的在外跑,连将养的时候都没有。

就是回来了,心还在公事上。严素心看着韩冈手上拿着一张密布着小字的纸,说着话时还时不时瞥两眼。

“官人,那是什么?”先不高兴起来的是王旖。

“是才编好的《三字经》。”韩冈手扬了一下,“是邵彦明【邵清】、田诚伯【田腴】的一番心血。”

家里面的妻妾都知道他们的丈夫现在想做什么,在做什么,听到《三字经》,都惊讶道:“都已经编出来了?!”

韩冈摇摇头,看着纸上的文字:“还是要改,得再浅显一点才好。这是蒙书,不是注、传,没必要解释,提到几个字,知道有这回事就够了。说得深了,小孩子反而难以领会……已经改了三次了,这毛病还是没改好。”

“慢慢来,不用急的。”周南劝道,“就是念着《千字文》《兔园册》,官人你不也是中了进士了吗?”

“此事不能不急啊。”韩冈笑了笑,却是让人将文稿收了起来。

平常私塾中为子弟开蒙,也不都是拿起《千字文》就来读,拿起《论语》就来背,也是有更为浅显宜用的书册,比如《兔园册》之类的书,只是太过粗浅,被士大夫看不起。

但汉晋留下来的蒙书失传的失传,无人使用的无人使用,也就《千字文》用得多,与之并称、从西汉用到唐代的《急就章》如今都少见了。

《百家姓》和《开蒙要训》是时人所编,流传的并不广,但韩冈却也搜集了过来。

他搜集蒙学课本当然不是为了给儿女们,而是要编订一本在这个时代还没有出现的蒙学书籍——《三字经》。

算学的课本,韩冈已经编写好了初稿。基本上就是小学数学的水平,又将加减乘除等数学符号、还有民间用来标记货物的草码进行标准化和正规化,放进了课本中。

自然地理的课本,韩冈是亲自在编写,文字尽量浅显,但要在其中融入格物之理,写起来要花些时间,不过也已经写好了大半章节。

至于社会学科的课本,其实就是被士人瞧不起的《兔园册》——是唐太宗之子蒋王李悍命僚佐杜嗣先,仿效应试科目的策问,引经史编纂而成的一条条解答。比起经义,更有实际作用。五代时声名显赫的宰相——不倒翁冯道,甚至都时常读来作为治国的参考。

但除此之外,还需要一本提纲挈领,将气学要旨灌输给蒙学中的儿童的教材。所以韩冈还需要一本《三字经》。

《三字经》一文看似浅显,却是抢占儒门道统的先锋。比如开宗明义的第一段,人之初,性本善,性相近,习相远。便是孟子的要义,一旦流传开来,荀卿一脉自然就会变成非主流。

至于之后的篇幅,韩冈也记不得了——想来出自后世的这一篇蒙学文章,有许多都是不能拿来用的——但将自然、历史、纲纪、人物,一些世间的常识集合在文中,掺杂进气学要义,却是不用记得三字经,也是知道该向什么方向去编纂。

韩冈将这份工作,交给了他的幕僚们。他们基本上都是气学门徒,领头的邵清、田腴更是追随张载多年。听到韩冈的解释后,要将太虚即气、天人合一、民胞物与的观点,以及格物致知、学必为圣、经世致用、笃行践履的行事原则,都糅合进蒙学书本中,让气学传之天下,自然是人人尽力。才一个月的功夫,就有了初稿。但问题就是太深奥了,改了又改,还不能让韩冈满意。

蒙学课本是用来灌输,而不是让人理解的。不明白这一点,《三字经》可不是那么容易能编纂好的。

