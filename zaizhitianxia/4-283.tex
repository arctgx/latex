\section{第46章 了无旧客伴清谈(五)}

送走了章惇,韩冈在京城中,就又少了一个能多聊几句的朋友。

先是王韶,继而是章惇,韩冈都觉得赵顼对王韶和章惇两名拥有大功的枢密副使,有着刻意打压的味道。

难道当真是为了在对夏战争中起用两人,现在先贬一下?

这种手段未免太过儿戏了。不能参与进战争的筹备工作之中,临战时怎么可能顺利接手?只是靠身份地位,可是不管用的。就是王韶回熙河,想要一下掌握全路的情况和人事然后领军出战,照样是不可能,少说也要几个月。

且不管天子怎么想,那还是要等上一阵才能知道究竟,反正韩冈眼下是没什么机会摊到领军的机会。

接下来的几天,韩冈因为茶马互市之事,提前被韩缜请去了群牧司中。

熙河路茶马互市,以及在广西,以茶叶和丝绢交易大理国的滇马,都有韩冈的一份功劳。

韩缜待韩冈比较冷淡,这是正常现象,韩冈不以为意。反正韩缜要处理于茶马互易的公务,想绕过他韩冈也不容易。

如今两边的生意越做越大,每年接近五万匹。听起来很多,但这么多马匹,其中勉强达标的战马也就只有不到十分之一的样子。

而且青唐马生长在高原,并不适合平原作战。在西北高原奔驰无阻,可入了中原之后,很难适应过来。至于滇马,个头矮小,不善奔、只善走,适合做战马的百中无一。

真正合适的养马地,应该在蓟北或是河套,可惜都被人给占去了。

韩冈去了群牧司,而开封府这里,苏颂将陈世儒一案审理得差不多了,在大理寺、审刑院和御史台的共同关怀下,已经向天子将最后的判决结果报了上去。

原本韩冈还认为这个案子牵扯太多,就算是苏颂决定秉公直断,为了做成铁案,也得用上好一阵时间来将口供、人证、物证等一系列证据做得完满了,才能下定论。哪里想到苏颂早就准备好了,一但下定决心,立刻就能在棺材上敲上钉子。

但苏颂继续担任权知开封府的可能性已经微乎其微了。

他开封府在任上的时间已经有一年了。

开国之初的三位开封府尹,太宗赵光义、魏王赵廷美和真宗皇帝,能在这个位置上盘踞了很长时间。但朝臣们的权知开封府,基本上没有能做满一任的。最短的根本没上任就给换了,上任后,短的几个月的,长的也不过两年。苏颂想要跳出延续百年的规律,自是可能性不大。

御史台的舒亶最近正咬着苏颂。倒不是因为陈世儒这桩案子,苏颂已经将这件案子砸成了铁案,御史台就算想要在这件案子中找麻烦,也只能去咬唆使大理寺下文保陈世儒和其妻陈李氏的吕家人。但苏颂身上不是没有可供下嘴的地方,以御史台风闻奏事的风格,就是没错都能给你编出错来,何况有把柄在外的苏颂。他可不是韩冈,能让天子不得不保着他。

对于舒亶的行为,吕惠卿肯定是心情糟透了。

韩冈这两天在常朝时,见到吕惠卿时,虽然对方神色上看不出什么异样,但他的与人寒暄交流的次数,却大幅下降。而韩冈昨日还听说,前天江南有一名知县,上书议论手实法扰民且有碍教化的问题,被吕惠卿请动天子,下诏严斥,并贬去荆南监酒税去了。正常情况,惩罚是不该这么重的。

吕惠卿是准备以开封府当做突破口,将手实法推行下去的。有了天子脚下的样板,下面的州县很难的抵挡得了朝中的压力。这一点,只要眼睛不瞎,就都能看得出来,要不然吕惠卿也不会将他的弟弟吕升卿安排做提点开封府界诸县镇公事了。

开封府这里,苏颂虽然没有全力支持手实法,但他也没有给吕升卿设置障碍。可要是换上一位新的权知开封府,那情况会怎么样就说不准了。开封府地位之重,仅比执政稍逊,贵为参知政事的吕惠卿绝对不会有插手权知开封府这个位置的人事安排的资格,只有天子能对此拍板。

只能说舒亶选了个好时机,利用这个机会,充分表现了自己的正直,并与吕惠卿划了一条界线出来。

一名御史,要是什么事都听从宰辅的话,坏了风评,这辈子就再难有进步的机会——监察御史的后台,不是哪家宰辅,而只能是天子。御史的责任也只有一个,就是监察百官。汉唐时,言官大部分的精力应该是针对天子的,拾遗、司谏这些官名,都是最好的证据。可到了此时,言官却成了天子制衡臣子的工具。

监察御史可以有倾向,但不能成为宰执豢养的家畜,也就是说,必要的时候,回头咬上一口也是可以的,就像蔡确当年咬王安石,成就了他的直名,在赵顼面前留下了一个好印象。

天子想保陈世儒,苏颂上报却将夫妻两人都定了死罪,让皇帝都没办法保他们,接着舒亶就拿着苏颂之前对某个犯法的和尚事涉开封辖下某知县的案子的宽纵行事说事,怎么看都有些问题。

但不论苏颂的职位最终能不能保住,京城内外还是洋溢着过年前的欢乐气氛。加上新成立的厚生司和开封府,赶在年节前联合在京城中设立保赤局,专一负责小儿种痘之事——所谓保赤,就是保护赤子的意思——更是喜上添喜。

种痘之术的原理,已经在京城中流传得很广了——为了自家的儿孙,甚至许多还为了自己,世人都是着意去打听其中的奥秘。眼下种痘的原理基本上人人皆知,种痘只是预防而已,并不是治病。所以能早一天种上,就能早一日安心。

从京西报上来的成功率来看,种过痘的小儿,至今都没有染上痘疮的迹象,不过种痘之后的半个月内,因为各种各样的原因——其中不一定是牛痘的缘故——而病死的个例,却也是有的,不过几率并不大,从现在上报的数字看来,暂时只有万分之一而已。

只是仅仅是万分之一的失败率,还是没人敢拿着六皇子的性命来做赌注,但东京城中的公卿宗室,基本上都是在保赤局报了名,争抢一个排在前面的位置。

“排在第一的是雍王长子,接着是蜀国公主家的独子,下面基本上都是宗室,王相公家的孙子,都排在五十过后了。”

当年推荐韩冈为官的三人之一,如今反过来被韩冈推荐到厚生司中担任判官的吴衍,这几天也终于放下了清高,上门来拜访韩冈,并为韩冈的举荐来道谢。厚生司眼下最重要的工作,不说其中的功劳有多少,光是接下的善缘和积攒的功德都能让人遗爱子孙三代。

吴衍于韩冈有大恩,到了韩家,并没有按照官职来行礼,只分了宾主,平头坐下。

韩冈听吴衍说着厚生司中的大事小事,他现在不便干涉,只能私下里聊一聊而已:“蜀国公主的驸马姓王吧,……那个书画很好的。”

“王诜,据说与苏子瞻交情甚深,据说山水是一绝。说起烟江远壑,柳溪渔浦,晴岚绝涧,寒林幽谷,桃溪苇村,李公麟都要让他三分。不过前些日子刚刚以奉主无礼而被贬官。”

“奉主无礼?”韩冈听得就有三分不快。即便是公主之尊,嫁人后也不过是人家家里的媳妇,家里的事,家里解决,闹到朝中降罪算什么。

在过去,驸马成亲后,立刻就会提上一个辈分,使公主不需要向驸马的父母——也就是舅姑——行礼。但时至如今,早就没有了这个规矩,该行礼就得行礼,根本没有妄自尊大的道理。

吴衍心中凛然,他再一次感觉到了自己和韩冈的差距。

韩冈只是略略皱眉而已,但流露出来的威势已经有几分迫人,换作是普通的官员,恐怕舌头就要打结了。

不到十年前,在秦州第一次见到韩冈,那时候,现在的龙图阁还仅仅是一个有几分傲骨且头脑聪颖、胆识过人的年轻人,一个穷措大而已,可如今已经泽被天下、名满中外的名臣了。

人与人的际遇相差竟然如此之远,若说嫉妒,吴衍心中的确有,但念头一起,就给压下去了。

能如韩冈这般不及而立便为学士,必然是有大气运在身,即便自己没有帮他一把,肯定能化险为夷,过丘壑如履平地。他只是后悔自己没能坚持附和王韶,否则现在决不至于才一个京官。

吴衍心念千转,与韩冈的对话并没有耽搁,“蜀国公主之贤,在宗室中也是有名的。其姑卢氏病重,侍奉床第边,亲和汤药,数日不解衣。只是王诜为人不谨细行,甚至狎妓而夜不归宿,故而受此责罚。”

“哦。原来如此。”

那就是王诜的不是了。仔细想想,韩冈似乎也曾在与人闲聊时听说过此事,只是没放在心上,吴衍提到时也没放在心上。

韩冈记性不差,但并不代表他连阿猫阿狗也都记得。大宋的公主不是唐代的公主。唐时公主有墨敇斜封,干涉朝政者不知凡几,大宋的公主只有老实做人的份,与朝堂很少有瓜葛。

韩冈听说过蜀国公主的性格很好,侍奉舅姑、晨昏定省与普通的儿媳一样,在士大夫中很受称赞,但也仅此而已。韩冈也没兴趣去关心。

