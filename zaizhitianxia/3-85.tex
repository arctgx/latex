\section{第27章 片言断积案(下)}

此案就此而定,就算是文及甫,在民心凝成的气势前也不敢再质疑韩冈的判决,毕竟不如乃父多矣。战战兢兢的样子,韩冈都为文彦博感到丢人。

当场写下判词,将坟茔和田地交还给何允文。又拎过瘫软成一滩烂泥的何阗来教训一番,说了句‘念在你是读书人,此事就不追究了’直接将之遣放,宽宏大量的姿态也做了出来。

最后在百姓们的欢呼声中,韩冈邀着文及甫一起上马回县,回到县中,县吏们见着韩冈的态度,都多了一份敬意。

晚间,韩冈设宴招待文及甫。但文家的六衙内食不甘味,喝了几杯后,就推说不胜酒力,告辞离席。

一番酒宴匆匆而散,韩冈领着幕僚回到偏厅,坐下来喝着茶再说起此案时,游醇便道:“今日一案,总觉得正言未免有些行险了。”

“一点也不冒险。”韩冈则笑道:“其实在事前,我就已经知道了何允文乃是何双垣真孙,而何阗必为伪称。”

“为何?”游醇惊问。

“何允文素号富户,能在京畿一带称富,家中少说也有几万贯甚至十几万贯。他不像一贫如洗,只有一群士人支持的何阗。有钱的何允文,必定会是胥吏们捞钱的金主。这些年来,他为了三千贯的祭田,砸进去的钱怕也有三千贯。若不是何双垣亲孙,如何会舍得做这等得不偿失的举动?”

游醇深思着其中的道理,慢慢的点着头:“原来如此。”

韩冈嘴角微微翘起,肚子里却在暗笑,这个说法当然是假的,他信口胡诌而已。

何允文虽然家产远远超过三千贯,但试问有多少股民因为心疼之前的投入,舍不得割肉,然后不断的追加投资,最后损失越来越多的情况。此事古今如一。对于富裕的何允文来说,说不定这三十年的投入已经超过了地价,亏得太多,已经越来越难以放手。要不然,他说一句只要坟头不要田产,这个案子早就结束了。

游醇全盘接受了韩冈的说法,只是疑问随之而来:“那为什么正言还要斋戒三日?直接断案不成吗?”

韩冈放声大笑,“偶尔兴致来了,吃个几天素很奇怪吗?‘每因斋戒断荤腥,渐觉尘劳染爱轻。’白乐天的心境,我偶尔亦有之。”

韩冈明显的是在开玩笑,魏平真在旁叹了口气,对游醇道:“这番道理说出来有理,但做不得数。也只有让何阗自曝其短,才能让人信服。为了墓前一哭,正言从开始时就在造势。斋戒沐浴是造势,拖了三天也是在造势,引得全县近万人都来围观,那就是正言造出来的势啊!如果节夫你被这么多对眼睛盯着,能安安稳稳地站住脚吗?”

游醇说不出话来。在白天的清水沟边,他也被万众共一呼的场面给惊到了。游醇从来没有想过,千万人齐声呼应会如此让人惊心动魄。虽然不忿气魏平真的诘问和小觑,但仔细想过后,感觉着心悸的摇了摇头,很诚实的回答:“不能。”

“如今方知要在千万人厮杀的战场上站住脚有多难。”方兴想想那个场面,也是觉得心悸不已:“除非正言这等见惯了战阵的,有谁能稳得住脚?心无底气,当然做不出孝子贤孙的样儿来。”

“‘虽千万人吾往矣。’‘千夫所指,不病而死。’”游醇回想着断案前的一番话,心中对韩冈的敬意油然而生,起身一揖:“如今方才明白,什么才叫读透了圣贤书。”

“节夫太夸赞了,我可是万万当不起。”韩冈连忙扶起游醇,笑道:“其实我没想到何允文竟然能哭得如此动情,让本案一下就定了下来。本来依照我的估计,两人都哭不出来才是最有可能的情况。”

三人闻言一呆,的确,这个情况才是最可能出现的。何双垣死了有五十年,何允文这个真孙都没有见过他祖父的面,哭不出来可能性很大。游醇连忙追问:“正言你那样会怎么判!?”

韩冈一声冷笑:“哭坟无哀,那即是不孝。如此不孝子孙,有不如无,如何能将祭田断给他?我本准备着趁势质问,将两人的面目彻底拆穿,那样县学的学田也就有着落了。到时候,将坟茔也归入县学中,吃着人家田里的出产,县学的学生四时八节带着祭拜,那是少不了的。总比只惦记着田地的孙子强。且若是日后有些灵异之处,还可以请封其庙,那就再也没有争议了。”

韩冈一番解说,三人皆恍然大悟。韩冈最初的计划,其实根本就是不承认何阗、何允文的继承权。反正他们也没有证据证明自己的身份,如果哭坟不哀,这个判决只要用民心一压,外人无可置疑。再将田地归入学田,支持何阗的士子们全都要转向,何允文的钱更派不上用场!

而且什么叫‘若是日后有些灵异之处’,分明早就有计划的,三日斋戒,还有坟前的那段话,全是在做铺垫。要是照着韩冈的计划一路下来,何双垣被朝廷封神,有了香火,还要不孝子孙作甚?

韩冈若是如此判决,不但不触犯律条,甚至还正合朝廷以孝义治天下的本意。就算何允文当真是嫡亲子孙,传扬出去后,也会被他人当成是一桩韩冈聪明决断的轶事,谁会当真为不孝子孙叫屈?

三人拍案叫绝,韩冈的计划其实当真是绝了。

韩冈则笑着自谦了几句,毕竟他的计划还是失败了。

何双垣死了五十年,在韩冈想来,他们能哭出来才有鬼。就算他们中间有真货,韩冈也能以哭之不哀的理由将两人指为假货。几千上万人看着,只要将他们当众挤兑住,逼着他们同意捐出土地作为学田以证自己清白,乃是轻而易举。

到时候,没有土地的坟茔,两家还会争吗?不争最好!若是还争,韩冈也可以说他们已经证明自己的纯孝,不如冤家宜解不宜结,干脆结为兄弟,自此四时八节一起来上香奉安。如果不愿意,一切就可以按着他的计划来了。

将周围观众的情绪调动起来,以势压人,此事又有多难?

至于他们日后要反悔,韩冈手上有千万人作证,谁还会帮着他们?站在道德制高点上鄙视他人,那是最容易的。韩冈一番煽动,就是让白马县的百姓自认品德高致。

方正之县,忠孝之民?!笑话,一万人中不忠不孝难道会少?!可但有几个愿意承认呢。就算是平日里不孝于父母的逆子,在这样的情况下,也会用着鄙视的眼光看着此案的原告和被告。一旦此案定下,两人必然要受到舆论的指责和嘲笑。就算转眼就死了甚至自尽,也可说他们是羞愧而死,根本不用担心有任何后患。

至于是不是冤枉了谁,韩冈根本不在乎。只要其他人相信就行了。以韩冈的想法,这片田与其留给两个只盯着田地的贪婪之辈,还不如用来奉养县中的读书人。

只是没想到,何允文竟然可以哭出来,像一个真正的孝子贤孙一般哭出来!韩冈对此也只能无可奈何的叹一口气,的确是有些小瞧这个时代的人们对祖先的孝心了——对田地的贪心是真的,对祖先的孝心也是真的。让人意想不到啊!

这时游醇又有了一个问题:“难道不会两人都哭得悲天呛地?万一变成这个情况,那该怎么办?”

“可能吗!?”韩冈嗤笑一声,抬眼反问。

“绝不可能!”方兴帮着韩冈回答,“作假的一方的心中又有鬼,心虚胆战,根本无心祭奠。就算明知道要悲恸欲绝,哭天抢地,可近万对眼睛看着,也演不出那股真情实感来。更何况,就算是无良之辈,在大庭广众之下,也断断不会有甘心厚颜而真认他人之祖为祖。天良未尽梏亡,人之所以异于禽兽者只在此刻。天日昭昭,众目睽睽,正言说得那是一点也不错!”

韩冈笑着点头,正是这个道理。

他此前装神弄鬼,一番行动、言语做下来,就是要坐实他已经知道了真相,而哭坟只不过是走个过场而已,关键的审判断案就在后面。弄虚作假的一方,心里本来就是虚的,心思必然不会放在哭坟上。并非专心致志地表演,能抵挡得住上万人围观的压力吗?

嘴皮子说得厉害,真做起来就拉稀的人物,韩冈见得太多。说句实在话,他现在的本事,也是一点点的历练出来,初出茅庐的时候,上了阵照样舌头打结。没有经过历练,突然面对大阵仗,有几个腿不软的?影帝级的人物有那么容易出的吗?何阗真要有这本事,这桩案子早就定下来了。而且即便是影帝,上场的时候也要酝酿感情,猝不及防的情况下,真的就能在镜头前一次就过?

韩冈早计算清楚了一切,根本就不会担心。即便有一点差错,也可以利用民气人心反过来压着。上万人中除了最前面的一干人,有几个能看清墓前的情形?只要把他们煽动起来,就算看明白的,也会在一片吼声中变得糊涂起来。在前世中,这样的例子不要太多!

韩冈与此前所有审案官员最关键的一个不同点,就是他需要的不是真相,而是影响力和控制力。只要能控制住场面,操纵着围观者的思路想法,颠倒黑白,指鹿为马,此等小事何足道哉?!许多时候,真相不重要,只要声音大了就能赢。

自然科学的发展水平还不到。何双垣死了五十年,坟墓被争了三十年,骨头都能用来敲鼓,没有后世的一系列科学手段,除了让他们自己暴露出来,根本没有别的办法验明真相。

幸好社会科学也算是科学的一种。论起如何煽动人心,韩冈还是有不少经验的。

今日可谓是一举数得。这个自我介绍,比起一个乡一个乡的跑断腿,可要管用得多。白马县的百姓,这下都该知道有个韩青天来了。

说了一番话,见了天色晚了,三人告辞出来。走在衙门中的青石小道上,三人犹在回味着今日这桩必然会传扬开来的案子。

方兴低头数着脚下踩过的一块块石板,叹道:“只凭哭一场就下定论,原本觉得这样的判断太过简单,但真正听了正言解说了一番之后,才知道这后面有这么多计算在。”

“看着很简单,真的做起来,又有谁敢这般行险?不将人心算计到底,如何敢用此策?”回忆着这三天韩冈的表演,魏平真也不禁要感慨着后生可畏,“正言心计手段都是第一流的,能有今日的地位,绝非幸至!”

游醇也是被韩冈的表现所慑服,点头附和:“那是正言通晓了先贤之言,行事才如此举重若轻。”

方兴笑着,停步对两名同僚道:“以正言之才,白马县的百姓可以有几年的好日子过了。”

“经此一案,白马县的百姓对正言当是心悦臣服,日后驱用起来,也当容易了许多。”魏平真叹了一口气,仰头望着天上清晰无比的无数繁星,“要想安然度过这一次的灾情,也只有上下一心!”

ps:顺便说一句,这个案子是真实存在的,断案手段也是如文中所述,文中乃是借用。如有兴趣的朋友,可以去看一下《兰苕馆外史·张静山观察折狱》。

