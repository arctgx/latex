\section{第18章 向来问道渺多岐(四)}

自从韩冈奉诏开始修纂《本草纲目》之后,太常寺一角的院落,便聚集了之前的几十倍上百倍的关注。

毕竟种痘之术源自于韩冈,谁都想着他还能有什么惊人的创见。而《本草纲目》编修局中,也没有下缄口令。将腐草化萤证伪,这样的新奇之说,自然是最容易散布出去的。

“腐草化萤竟然是错的……”赵顼脸上的表情中有着几分迷惑,“消息确实吗?”

宋用臣连忙道:“回禀官家,这是奴婢自太常寺亲耳听到的,不敢妄改一字。只是是对是错,奴婢就不知道了。”

“你下去吧。”赵顼点了点头,示意宋用臣下去,但立刻又将宋用臣给叫住:“等等……”

宋用臣停住了,弓着身子等赵顼发话。

但赵顼迟疑了半天,最后仍是一挥手:“还是下去吧。”

待到宋用臣这名内宦离开崇政殿,赵顼就神态疲惫的揉起了额头。韩冈要编《本草纲目》的原由,他也知道有几分是为了气学,只是没有想到会是从这个角度,用这样手法。

赵顼不觉得有必要让人去重新验证这条传言的真伪,韩冈这名大臣的品性为人,赵顼很了解。他既然将话说出口了,那么就肯定是拥有着十足的把握。韩冈一贯的标榜实证,自然不会在这方面出篓子。

也就是说,《礼记》之中的腐草化萤这一条,便是延续千年的误解。

好手段啊。

纵然心头憋了口一气,赵顼还是觉得要为韩冈的行事赞叹两声,不愧是朝廷中数一数二的帅臣,声东击西的用兵手段,已经用到了道统之争中。

赵顼一直都在关注着《本草纲目》编修局中的消息,韩冈张挂在局中正厅内的两株生命树,都在第一时间复制到了福宁殿中。

对于韩冈所创立的这种看起来繁复异常的分类法,赵顼只觉得有趣而已,但当方才宋用臣带着最新的消息回到崇政殿,赵顼却无奈的发现自己似乎又弄错了。

从父亲英宗登基,到赵顼本人继位,再到元丰三年的现在,这十几年间,一直都在接受第一流的学者的传授和教诲,即便仅仅是中人之智,也早就拥有了足够的才识,赵顼自然能明白韩冈对腐草化萤的证伪,绝对不是表面上的那么简单。

目的也好,影响也好,这一回气学一脉在争夺儒门道统的道路上,又前进了一大步……不对!赵顼摇摇头,是将对手向后给扯了回来。

无论如何,韩冈对诗传礼记下手,都是毫不容情的在掘对手的根基。《诗经》被攻,过去所有有关螟蛉之子的注释都有问题,《礼记》被斥,那么这部书的《月令》一篇,乃至对这一篇加以注疏的历代传注,都成了笑料。

韩冈起意编修药典真正的目的终于浮上水面。之前的臆测,在韩冈的真实目的面前,显得太过肤浅了。

韩冈一直以来的作为,都是为了宣讲气学。让他去管理太医局和厚生司,编修《本草纲目》,的确是压制了他晋身两府的可能,但另一方面,也给了他光大气学的机会,等于是将猫丢进了鱼堆里,正合他的心意。

赵顼阴沉着脸,与殿外艳阳高照的截然相反。身为天子,赵顼绝不喜欢看到事情脱离他掌控的方向。这一件事,赵顼虽不认为自己是被愚弄,但他还是很不喜欢这样的结果。

因为他要一道德。

作为天子,赵顼希望朝廷所主张的一切,从儒学,到法度,不受到额外的挑战,这事关朝廷的威信,也关系着朝廷中人心的稳定。

若是普通的经义论辩,完全可以不加以理会,但韩冈从来不跟人争辩,从他在琼林宴上丢石块和秤砣开始,就一直用可以眼见的事实来为自己张目。

明显的错误是无法去掩盖的。就如螟蛉义子的谬误,当韩冈指正之后,驳斥者成百上千。就是在经筵上,给赵顼讲学的几个经筵官,也都严斥韩冈的荒诞不经。但等到越来越多的人通过实证,证明了韩冈的正确,之前在赵顼面前义正辞严的几个人,都没脸再讲《诗经·小雅》中几篇诗章。

以实为证。

当韩冈举起这一面大旗,便让人再无法与之辩驳。讲学就是一个说服世人的过程,而要想说服人,永远都不能脱离事实。经书、释义,无论哪一家的言论,都必须向这面旗帜低头。而这便是以格物致知而扬名的韩冈,最为得意的战场。

赵顼也曾了解过气学的观点——形而上的道,必须返归到形而下的器,所谓的器,就是可以眼见的现实。

韩冈一心一意的在刨新学的根基,《诗义》已经给他戳了一个洞,眼下又盯上《礼记》。

虽说《三经新义》所注疏的经书是《尚书》、《周礼》和《诗经》,而王安石当初给赵顼讲学时,也曾经批评过《礼记》中多有伪篇。但当韩冈在挥斧看法《礼记》的根基时,难道会放过涉及草木虫鸟百余条的《诗经》,难道会放过天下所有不免涉及于此诸多经书及其注疏释义?三经新义中,有关这一方面的条目,数量可是为数众多。

赵顼并不知道什么叫做意识形态,但他作为皇帝,天然的就明白新法的顺利推行和延续,取决于一个稳定的理论基础。只为国事,新学这面大旗便是绝对不能倒的。

但韩冈摆明了要以实证来宣讲气学的正确,不仅仅是新学,可以说,儒门诸多学派,他一个不漏的都有踩在脚底的打算。道统之争的残酷,比起争霸天下,也不遑多让。

怎么办?怎么办?

赵顼在心中喃喃念着。

难道要撤掉《本草纲目》的编修局?还是跟韩冈说,让他只要将药典编好就行了,不要再给朝廷捣乱。

但要是当真这么做了,韩冈多半会直接辞官回去讲学。赵顼很清楚,这么点小事,脾气硬一点的士大夫都做得出来,甚至可能会更兴奋,就像受到了挑逗的斗鸡,不啄人两口是不可能放手的,到时候丢脸的可是他这个皇帝了。

而且要是自己错了,气学在多少年后压倒了新学,那么后人加以更正时,他赵顼留在史书中的形象,必然是跟主张异理的梁武帝,唐宪宗相差不远了。

还有皇嗣的事,有韩冈这名药王弟子在京城坐镇,皇嗣的安全也能多一番保障。赵顼可以在职位上打压韩冈,雷霆雨露皆是天恩,这是皇帝的权力,但他却不愿逼得韩冈请辞出外。

反反复复考虑再三,赵顼招来了

“将《字说》刊发于世,并发送国子监……还有,从明天开始,经筵上开讲《字说》。”

虽不便明着来阻碍韩冈对气学的宣扬,但只要朝廷的进退之路还在手中,气学就只能在外讲学,而进不了朝堂。赵顼倒想看看,当《三经新义》的根基《字说》一书上了给天子讲学的经筵,韩冈还有什么招数来动摇新学的地位。

……………………

“官家这是拉偏架啊!”

韩冈是在家里听到了赵顼的这个决定,对于此,也只能笑叹一声。

“官人,不要紧吗?”严素心小心的问着韩冈。

“有什么关系?”韩冈不在意,从严素心手中的盘子上拈了一枚葡萄吃了,“皇帝能做的也就这些了,难道还能将为夫又踢出去不成?也不看你姐姐现在在宫中有多受欢迎。”

王旖刚刚从宫里面回来,这是她半个月来第二次受到皇后邀请入宫。不管怎么说,韩冈诸多子女一个个健康活泼,还有种痘法让天下幼童免去了痘疮夭折之苦,这就是王旖在宫中受人欢迎的最大资本。

在宫廷中,皇后和朱妃,都希望韩冈这位药王弟子的身份,能庇佑六皇子健康成长。就是天子赵顼本人,让韩冈来掌管厚生司和太医局,从本心上,自然也有保护皇嗣的一份意愿。

韩冈在医道上的表现,也间接推动了气学的发展,加固了气学的根基。只要天子还有一分为子嗣考虑的心,就不敢直接出手打压气学。只是在经筵上让人讲学《字说》,也正是证明了这一点。

气学如今展示在世人眼中的学说和文章,只是争夺儒门的道统,并无动摇朝廷统治的悖逆之意,还不到需要孔子诛少正卯那个等级。

而且以实为证就是最强的武器。韩冈所主张的格物致知,最大的长处就是实证,身边随处可见,随手便可实证之,而其他学派,无不是以己意解天心,新学也好,程学也好,道德性命之说,哪里比得上格物致知更直观?

韩冈看着自己摊开来的右手手掌,得意的又攥了起来。儒林中的局面并没有脱离预计的轨道,随着《本草纲目》的编修,所有与自然有关的经学篇章,都要在实证这柄大刀下走上一遭。

如今为了编纂《本草纲目》药材堆满屋,生药熟药将编修局小院的一半房间都占了去,在局中待上一天,身上就免不了沾满药味。

韩冈嗅了嗅衣襟,就是洗过澡之后也没能散去,不过这样的一点代价所换来的成果,倒是让人乐意付出呢。

