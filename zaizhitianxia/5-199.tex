\section{第23章 弭患销祸知何补(七)}

徐缓而又平稳的行驶在大街上的马车突然停了。车中闭目假寐的赵頵身子由着惯性向前一冲,立刻惊醒了过来。

“怎么了?”身为天子三弟的曹王赵頵很是不快的敲了敲车厢壁板,问着外面:“出了什么事?”

跟在车旁一路步行的元随随即出现在车窗边,弯着腰,“回大王的话,是前面的路被人堵起来了。”

“堵?”赵頵一愣神,透过车窗向外一张望,丑婆婆药铺的招牌便在眼前挂着,“这不才到踊路街吗?离了皇城没几步,谁吃了熊心豹子胆,敢拦在这街上?!”

“已经遣人去打探究竟了。”

过了片刻,一名骑手从前面驭马返回,充作护卫的骑兵队正听了他的禀报,下马后来到车前:“大王,是前面的楼子巷出事了,聚集的人太多,才连踊路街也一并堵上了。”

“楼子巷?”赵頵对这个路名没什么印象。

“南顺侯府就在这里。”元随提醒道,“就是过去的张宣徽府,也就是七大王府。”

“哦!”得了下人的提醒,赵頵终于想了起来,“说什么楼子巷,直说是楚王府不就得了。”

南顺侯虽然是个新爵位,但南顺侯府却一点也不新。占地也不小,整条巷子的北侧只能看到一扇大门。曾经是仁宗皇帝赐予温成皇后伯父张尧佐的宅子,再往前,便是太宗第七子赵元偁的楚王府。待到交趾国灭,又为当今天子赐予李乾德母子安身。

从楼子巷中出来,向南是汴水,沿着踊路街东行,就是西角楼大街,抬头就能看见皇城的西角楼,再往前一点,便是贯通京城南北的御街。

“这是为了前日开封府中审下来的案子?”

“也不会有其他的事了。”元随说道,“十几条人命都是南顺侯害死的,还有上百人受伤,蹴鞠联赛也不得不停了,这些都要南顺侯府给个交待。哪能让他一死百了!”

昨天的《蹴鞠快报》上就有关于此事的报道,当时赵頵还在想齐云总社的那一群会首下一步会是什么,原来就是欺负孤儿寡母来着。“……这么快就闹起来了?真是一点都不耽搁。”

赵頵笑着,心中有点烦。两天前出面入宫帮人说话,似乎是画蛇添足了。

“大王,是绕路还是过去将人给赶开?”同在车厢边的骑兵队正问话时表情木讷,不过赵頵听得出来,他是在建议绕路,不要去招惹麻烦。

踊路街算是内城中很繁华的去处了,每天都是人来人往,楼子巷中闹起来,也难怪看热闹的能堵上巷口。但这么一闹,沿街的商家不知有多少要跳脚。眼下前面堵满了人,硬挤过去,是把浑水往自家身上泼。

哀乐声从前方随风传来,赵頵坐在车中,透过掀开的车窗,还是听到了一句半句。

竟然是堵在南顺侯府门口哭灵!要是没有人在背后指使,那些丧家当也用不出这样的招数。开封府也在不远处,但开封知府多半还没有收到消息。齐云总社在开封府衙中的影响力,将开封知府给架空估计还做不到,但将一些事情欺上瞒下,拖个几个时辰,一天半天,倒是不在话下。当然,串供什么的更是一点不难。

昨天的《蹴鞠快报》并不在手边,但大体的内容,赵頵还能记得一点,而且之后他还让人去打探了详情。

在开封府的提审中,过堂的证人总计三十余人。即有棉行喜乐丰队的球迷,也有福庆坊福庆队的球迷,还有事发当地的商家、住户,甚至连路边小店喝酒的酒客都被一股脑弄进了开封府的大堂。

这么多的证人,都看到了李乾德向敌队球迷挑衅的一幕。这一点无可厚非,否则也当不了证人。但奇就奇在他们的证词如出一辙,没有一点差异。

从道理上说,这根本就是不可能的。赵頵身边的一个精通刑名的清客直接就说了,绝对是事先串通好的。正常的案子里面,就是证人在案发时并肩站在一起从头到尾都看的分明,但他们过堂时陈述的口供,怎么都会有一些差异在,没可能如此清晰明白。
只不过,赵頵可没有帮南顺侯府说话的打算。先不说降臣的身份,孤儿寡母离乡背井,让赵顼很是照顾他们。眼下更是因为老实做人,被赐予了城中的清静花园。只是李乾德身死,他的宅子估计也要便宜他人了:“南顺侯这一回看起来要绝后了?”

“大王有所不知。南顺侯还留了一个刚出生的儿子,应该能承宗祧。”

“哦?是吗?”赵頵叹了一声,“想不到还留了一个。”

“其实实在不行,京城中还有好些个交趾的王孙,当初也是一并降顺的。只要官家还想保着南顺侯府的名号,就是李乾德的儿孙不能接位,他的兄弟也有资格。”

听着前方的喧嚣,赵頵沉默了一阵后,又开口问道,“南顺侯今年才十五岁吧?”

“……不是十三,就是十五,肯定是没过十八——年纪并不大。”元随说话饶舌得很,但他是赵頵的亲信,口齿伶俐的特点倒是更讨赵頵的喜欢。

“十三、十五就有了子嗣……”赵頵笑了一下,“南顺侯就是没有死于意外,恐怕也活不长久。哪能这么早就沾了女色?根本未固,却时常摇动,就是一棵树都活不了太久,何况是人?”

“说起医理,大王当也不输太常寺中的那几位。”元随凑趣般的说着。

赵頵倒是喜欢医术,家里搜集了不少药方,也养了不少名医。前几年,他所任用的一名医官被卷入赵世居、李逢谋反一案,为此还不得不上表请罪。

想到这件事,赵頵顿时就对眼前事没了兴致,敲了敲前面的车厢内壁,“掉头,从西角楼大街绕过去。这条路等到明天怕也走不通。”

放下车帘,赵顼一声吩咐。前面的车夫随即便将马鞭一挥,四轮的轻型马车重新启动,转了一个很小的圈子,很快便消失在街角。

隔了一条街,韩冈也几乎在同时收回视线。抖了下缰绳,胯下的坐骑乖乖的掉头转身——南顺侯府巷外的踊路街都被堵起来了,看来只能绕路回去了。

苏颂比韩冈还要早一步掉头离开。虽说以他的身份,让旗牌官上前驱散人群,打开一条通道不为难事,可前面堵在南顺侯府巷口的人群有许多事是丧家、苦主。看这声势,明天必然是传得满城风雨,没事掺和进去作什么?这是苦主和肇事者之间的事,官员们本就不该在其中表态。

“前面是谁家的马车?”苏颂他扭头对着跟上来的韩冈问道,“那式样怎么没见过?”

“将作监新献上的新制马车。东京城中见过的的确不多。”韩冈笑道,“前轮后轮各在不同的底盘上,中间是活动的,能自由转向,比起旧式四轮马车,要灵活不少,只不过只能用来载人,载货就不行了,底盘不够结实。”

苏颂瞧了韩冈一眼。韩冈虽然是只管过军器监,但在他的领导下,军器监连年立功,使得如今的将作监中,有不少人是从军器监升调过去的,官员、工匠都有。韩冈不能对将作监了如指掌那才叫奇怪。

“是谁家的车子?”苏颂重又问道。

“若是两个月后子容兄再来问,那还真猜不出来。不过现在倒是不难猜。虽然是有听说京中的车马行也闻风而动,招揽不少匠人,但眼下除了将作监的车船院,暂时还没有其他作坊能仿造得出同样形制的马车来。”

与如今在京城中替代了旧有的太平车,变得越来越普及的四轮载货马车不同,那一辆消失在对街街角的精致的四轮马车,在底盘上拥有转向结构,在外观上有着截然不同的差别。苏颂方才第一眼看到时,注意力就被吸引了过去,觉得这辆马车的样式很是特别。

“将作监又见功了。自玉昆你执掌军器监后,这几年军器、将作二事上,倒是时常给人惊喜。”

“在军器监也不过一年多……元丰以来的功绩,我可没脸去冒领。”

“到底是哪一家的车?”

“一位大长公主,一位长公主,还有两位亲王。天子赐物,就是前几天的事。我都听说了,子容兄不会没有听说吧?”

苏颂眉头微皱。他哪里会去关心天子赐了什么东西给亲王、公主?也就韩冈,估计是一直盯着将作监的新发明,才会知道天子赐了马车。

除了韩冈之外,又有几个士大夫会在意这等器物上的发明?就是以苏颂对自然、机械等方面的爱好,也不会去刻意去了解将作监或是军器监中,又有什么新花样。

‘该不会是为了要弄个赛车联赛出来吧?’韩冈有两次前科,苏颂不免会有这方面的猜测。不过他并没有将心底的疑惑说出来,“别卖关子了,到底是哪一家?”
