\section{第五章 冥冥冬云幸开霁(八)}

回声从天际传来。

青天白日下,仿佛无云的天空中,打了一声旱雷。

那是火炮的声音。

统率天下第一支炮兵部队的李信,对此十分肯定。

只是不知是什么因素,火炮的轰鸣却似乎是从高空中传到了李信的耳朵里。与火炮应该所在的位置完全不一样。

不过从望远镜里,还是能清楚的看见北面接近皇城的地方,有着浓烟腾起。

烟火升起的地方,与军器监的一侧外墙似乎很接近。但从质地精良的千里镜中,依然能分辨得出火场与军器监有着一定距离。

将千里镜拿开了一点,李信偏偏头,瞟了一眼朱雀门的城门官。

“那是两位大王的府邸。”狄贤心领神会,小声的在李信身边确认道。

“叛乱的只是赵颢。与三大王无关。”

时至今日,再不用对赵家的二大王保持敬意,已经可以直呼其名。

狄贤不敢乱言乱动。

随着朝会结束,朝臣们纷纷离宫,赵颢与蔡确叛乱失败的消息也传到了京城之中。

而狄贤这位守着内城正南门朱雀门的城门官,却更是早一步得到了消息。

看到李信带着一部兵马赶过来,还拖着传说中神乎其神的火炮,误以为是叛乱的狄贤都已经做好了死战……好吧,是战死的准备。

幸好在过来的兵马前面,有一名内侍先行一步,将诏书宣读,让他不用从战死和降贼两条路中再纠结了。

‘看起来很顺利啊。’

李信想着。

李信与王厚一同出皇城。王厚去军器监拿弓弩,而李信也去军器监走了一趟,不仅仅是带出了手下的兵,更将轻便的虎蹲炮都带了出来。

至于更重一点的野战炮,安装了炮车的仅有两门,他分了一门给王厚,留给了自己一门。还送了弹药去宣德门给郭逵,皇城中的火炮只是礼炮,平日只是放空炮而已,但装上弹药,立刻就能杀人。

将二大王的府邸都点着了火,是不是王厚一炮轰到了厨房或暖阁,将柴堆、石炭堆给点着了?

王厚倒是干得好,二大王府烧起来后,不少人就能安心了。待蔡确、曾布和薛向家里都烧起来,日后不知会有多少人感激王厚和背后的韩冈。

将千里镜的镜头稍稍开了一点,李信顺着内城的城墙望过去。一点细小的艳红色,就映入了眼底。

从近而远,每一座城门的敌楼处,都挂起了一面红旗。

东面的保康门、汴河角门子、旧宋门、旧曹门,西面的新门、旧郑门、汴河水门,都在一片素白中,有着微小却显眼的艳红。

当镜头移到正西的梁门处,正正看见一面红旗在缓缓升起。

‘手脚倒是麻利。’李信微不可察的点了点头。

他方才出了军器监,便带着人马和火炮,径直来到了朱雀门上。

就像皇城的宣德门和外城的南薰门一样,位于正南方向上的朱雀门,就是内城的正门。在正门处,驻屯的兵马最多,地位也最为关键。

在拿下朱雀门前,李信没有分兵。

包括三水门在内,内城总共有十二门,归属李信的占了其中的四分之三。他手上兵力太少,分散开来,一旦生变根本无法镇压。

而在拿下朱雀门以及东面近处的保康门后,李信手中一下多了四个指挥,运用的余地宽裕了许多。将炮兵和城门兵配合起来,分遣去内城诸门,控制住城门自是十拿九稳。

红旗便是成功的标志。等到各门再遣人当面回报,就能彻底确认。

眼下南东西三面都已经控制在手,剩下的就只是北门。

北面的三座城门是王厚的任务之一,李信出发时便与他议定了各自的任务范围。王厚的位置离北门更近,如果已经拿下,也应该有着红旗挂起。

不过当李信越过二大王府,向更北面的地方望过去后,却一片模糊。

有烟的因素,也有距离的缘故。

纵然都是内城,但从南面的朱雀门这边望过去,北面的旧封丘门和旧酸枣门也几乎已经看不清了,更别说约定好的暗号。

李信皱了皱眉,放下千里镜,转头问狄贤,“这里有望远镜吧?”

望远镜和千里镜,因为一个有禁令,属于军器,一个没有,可以民间使用,在世间分得很清楚。不过这并不意味着反射式望远镜就不会用在军中。

尤其是周围五十里的京师,拿在手中的单筒千里镜,只能照顾到周围的一两里的地方,再远就难了。

为了能够更好的掌握京城中的点滴动静,朝廷从来都不会拒绝更先进的工具。

“是。”狄贤回手指着背后的敌楼,“就在敌楼顶上,寻常夜里都在看着城里城外哪里有警。”

一架大型的望远镜,不仅仅可以控制京城,也能起到潜火铺的作用。

李信不多话,直接登楼。

千里镜小而望远镜大,里面的原理有区别,但对李信来说,就是一个易于携带却只能看清周围一两里,另一个难以移动,但能够看到更远的地方。

楼中的望远镜,大小比起李信在韩冈家里看到的新制望远镜也不差多少了,就是保养差了,楼中的地面也脏得很,都是斑斑痰迹,甚至还有尿味。

李信从韩冈哪里听说过,苏枢密如何看重他家里的那具望远镜。只要不用,就会拿细绸缎缝制的布套给罩好,看得比儿子都重。若是今天换作苏颂上楼来,包管将管理不严的狄贤拖下去一阵乱棒。

李信不是苏颂,并不在乎。转动镜筒,对准北方,低头看过去。

来自镜中的景象,远比千里镜要清楚得多。

首先映入李信眼中的是开宝寺的铁塔。

十三层砖砌宝塔如宝剑般直插云霄,色泽深黯如铁。铁塔行云号为京中胜景。在望远镜中,每一层的门洞和琉璃瓦都能看得分明。

看到了铁塔,就给李信指明了方向。微微调整了一下镜筒角度,就看见了内城城墙。

开封府的外城城墙前几年才经过整修,但又被称为旧城的内城,却是年久失修,只有城门的周围方才完好。

望远镜中的内城城墙,好些地方都有大片的墙体剥落,显得破败不堪。只看新旧程度,就能分得清内城与外城。

沿着城墙横移过去,一座城门出现在镜中。

旧封丘门。

城门上的赤旗鲜艳夺目。

再向西去。旧酸枣门上,一面红旗招展。

而内城西北角,俗称金水门的天波门尚无变化,不过北面的两座主要城门已经拿下,剩下的最后一座也不会再拖多久。

“看好了。”李信点了一名班直,“城中何处有乱,立刻来报。”

安排了人手监视城中,李信随即下楼。

城门控制在手,并不是为了防止叛逆的家属逃窜。逃出去几个也无妨,跑也跑不远。关键是要能够控制得住京城。守住了内城城门,不论外城内城,一旦有变立刻就能出动,更能阻隔内外交通,让叛逆的残党不至为乱。

站在城头,脚下就是朱雀门。

朱雀门的门额,嵌在青砖砌起的墙面中。

朱雀之门四个大字,在城头上看不见,不过进出城门时,李信早看得多了。

当年太祖皇帝经过朱雀门,看见门额上写得却是朱雀之门,便问赵普,为什么不直接写朱雀门,却要加一个‘之’字。

赵普回答说:‘语助耳。’

太祖皇帝嗤之以鼻,‘之乎者也,助得甚事?’

这个典故,李信从韩冈这边听过,也从张守约那里听到过。

对文酸措大的嘲笑,张守约是暗里说,韩冈却是讲的明白得很,在李信这位做武将的表兄面前,丝毫没有为同类遮掩的意思。

到了今天,就是彻底的见了真章。见言语不通,直接就挥锤敲碎脑袋了事。

要从骨子里来看,李信觉得自家的表弟尽管把文职都要做到了顶,可终究还是武夫的脾性,有李家人的血。

好痛快!

不敢宣之于口,可李信还是这么想。

好痛快!

……………………

“惩治叛逆,不能只求一个痛快。”

韩冈用火炮炮声,给了众宰执一个再充分不过的理由,让他们可以去维护誓言。

“如今军心不稳,人心不定,要安抚人心,就不能只图刀下痛快。”

“如曾布、薛向之辈,诚然死不足惜。但万一因为忧惧王法,叛逆余党铤而走险,蛊惑军心,发动兵变又如何?”

“此刻贼众必心怀忐忑。更要提防狗急跳墙才是。”

“现如今,误从逆贼的禁卫和禁军,皆在看朝廷如何处置曾布、薛向等叛逆。如果朝廷饶了他们的性命,所有人都会安心下来。如若不然,忧惧之下,必会有人要做搏命一击。”

宰辅们轮班上阵,将兵变这块警示牌高高竖起。

没有一位重臣,现在敢拍着胸脯说不会有兵变。万一说了之后兵变当真发生了,他们就立刻会被推出来做替罪羊,成为安抚乱军军心的牺牲品,而在兵变中受到伤害的京城百姓,更不会原谅他们。

如此危险,包括李定在内,一个个都沉默了下去。

