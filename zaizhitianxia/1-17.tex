\section{第10章 霹雳弦动夙夜惊(上)}

“看管军器库!?”

韩冈没想到他的第一个任务竟然这么快就到了。早上才跟黄大瘤斗过,到了午后便被派了差役,若说其中没有关联,也只有三岁小儿才会相信。

秦州是边境重地,城中分属不同衙门的军器库有十余处之多。其中以秦凤路经略司和秦州州府拥有的库房存储兵械最多,诸多城防用具也尽属两库。至于成纪县辖下的两个小军器库,一座位于县衙中,主要用来存放隶属于县中的弓手、衙役所使用的刀剑弓弩,而韩冈要去的则是放置备用武器的仓库,位置不在县衙中,反倒在城内偏僻角落处的德贤坊。

领着韩冈往德贤坊军器库走的差人大约有三十多岁,方才被户曹的刘书办唤作李留哥。见李留哥身上穿的并不是皂色的公服,韩冈猜测着应该跟他一样也是服衙前差役的乡户,而不是长名衙前——即衙役。

差事来得莫名其妙,用脚趾想也知道军器库中肯定暗藏着陷阱。韩冈正组织着话语,想从李留哥嘴里掏出点什么。没想到李留哥反倒先开口说话:“监军器库可是县中衙前能得到的最快活的几个差事。不知韩三秀才你花了多少钱钞?”

“钱钞!?”韩冈微微一愣,随即摇摇头,“韩某刚刚生了场重病,家中骤贫,哪有钱弄个好差事!”

李留哥皱了皱眉,道:“不想说就算了。”

“韩某向来不喜说谎。”韩冈道。李留哥的语气不像是作伪,但衙门中一向消息最为灵通,要说他没听说黄大瘤当街与自己起冲突的消息,韩冈是决计不信的。

“等到了军器库,你去问问现在守库的周凤费了多少钱钞才买到这个差事。”李留哥看起来半点不信韩冈的辩解,边走边道:“为了能留在户曹下面奔走,俺整整用了六十四贯!”

“这么多?!”韩冈当真吃了一惊。

衙前差役都是由乡里的一等户充当,而一等户的标准虽然因为全国各地贫富不一,而各不相同,但最少最少也要百贯以上。韩冈重病前,韩家尚拥有一顷多地,一头牛和一间院落,当时给算了一百五十余贯,比一般一等户多上一点。但李留哥如今只从县衙中买一个跑腿的差事,竟然就用了六十四贯!相当于秦州一等户平均家产的二分之一!再听他的口气,买一个监军器库的差事,费得钱要更多!

一年衙前破全家,当真不是虚言。

李留哥回头瞥了韩冈一眼,“等秀才你摊到押送粮饷和犒军的银绢茶酒的差事,就知道这钱花得有多值了。”

李留哥领着韩冈转过一道街角,出现在眼前的巷子正通向两人要去的军器库。军器库的库墙有近一丈高,也是用黄土夯筑而成。夯土的建筑听起来不怎么样,但实际上却极是坚固耐用。秦汉的长城到了两千多年后仍能屹立荒野中,大宋北方的建筑基本上也都是用黄土夯筑。韩冈走过去时,用指甲试了一下,只划出了一道白印,指尖还磨得生疼。

守着军器库大门的是两名士兵,他们带帽檐的范阳毡帽上的红缨掉了只剩一半,穿着的花锦袍也是皱皱巴巴,只腰间挎着的黑鞘弯刀还算入眼。韩冈和李留哥过来时,两人正坐在门口的台阶上,就像两只疲沓的老狗,在深秋的阳光下打着哈欠。看着韩、李两人走近,两名库兵站了起来。一大一小,一高一矮,一黑一白,一有须一无须,对比强烈的两人并肩而立,只显得错落搭配得煞是有致。

“王九哥,王五哥。”李留哥冲着两人行了一礼,韩冈也随之拱了拱手。

两个士兵同姓王,却不是一族的,年长排行第九,年幼的排行第五,所以名字唤起来,反倒是年纪小、个头矮、肤色白、没胡须的王五的排行在前面。

“是李大啊……”年长的黑胡子王九跟李留哥搭着话,“你一来从没好事!带着的这人是谁?”

就在王九和李留哥说话的同时,王五站在韩冈面前,上下打量了几眼,眼前这位身穿青布襕衫,貌似病弱的秀才传言多多,让他很是好奇。问道:“你就是韩三……”可只问了半句,却突然断了音。

韩冈眼角余光一瞥,却见是王五腰上给王九的手指暗地里戳了一记。

被领着进了军器库,两个库兵甚至都没再多看韩冈半眼,方才李留哥还问了韩冈花了多少钱买个差事,但两个兵却问都不问。很明显黄大瘤打过了招呼,知会过两名守卫。

‘君子报仇,三年不晚;小人报仇,从早到晚。’韩冈暗自叹着,‘老话果然永远都是有道理的。’

黄大瘤刚刚在街市上受辱,转眼便报复回来。县衙里动手太危险,普修寺中和尚嘴杂也不好下手,但这座军器库多半连守库的兵士都跟黄大瘤亲近。韩冈进了库来,只要把门一锁,那便是关门打狗,他的小命已经有一半攥在黄大瘤手中,只要军器库中出了些乱子,很容易的便能栽在韩冈的头上……再说了,陆虞侯为陷害林冲敢烧草料场,黄大瘤纵然没有高俅那等奢遮的后台,怕是也敢在军器库里烧点不算重要的东西。

李留哥领着韩冈进了军器库院子,身后的大门随之关闭,王五留在外面,王九跟着一起进来。

‘真是个好地方。关门打狗的……好地方!’韩冈环视周围,下意识的握紧了藏在袖中的匕首。不过他很快又放松了手指,他很清楚,黄大瘤费了这么些工夫,绝不是遣人埋伏在军械库中教训他一顿那么简单。韩冈尚记得,黄大瘤临走时的那个眼神,可着实不善,那是起了杀心的神情。

李留哥领在身前,王九走在身边。身处绝地,韩冈心中反而愈加沉静。每临大事有静气,他偏有这等能耐。在过去,不论考试和面试,他总是能有超水平的发挥。再回想起让他来到这个时代的空难,他在飞机失事前,也是冷静到淡漠的地步。

成纪县的备用军器库,大约只有两三亩地那么大,其中修了五间东西并排的长条状库房。每间库房的两侧屋檐下,都排了六个近五尺高、盛满水的大水缸。这种水缸装满水后大得能淹死人,说不定跟司马光小时候砸坏的那件是同一号。看水缸中的挤满浮萍的臭水,显而易见,这个军器库的安全系数并不算低。不像县衙,二十多年来已经被火烧过了三次。

就在东头库房的一角,有一间靠着库房墙壁修起的小屋。李留哥领着韩冈走到小屋外,冲着屋内喊了一声:“周凤!你出来!”

一个中等个头的朴实青年从屋中走了出来,他大约只有二十三四,看见李留哥和韩冈一脸严肃的站在门口,神情便有些瑟缩。再看到两人身后的的王九,更是浑身一颤,“是李家哥哥啊,怎么?有什么事要吩咐小弟?”

李留哥指了指身边的韩冈,道:“你的差事从今天起就由韩三秀才顶了,你快点收拾收拾,俺还要回去复命。”

周凤愣住了,眼睛一下瞪得老大,“这……这……这怎么可能!俺不换,俺可是花了八十贯!八十贯呐!能在京兆府买间好宅院啊!”

周凤卖力的用双手在韩冈三人眼前比划着,很努力的想表示出八十贯究竟是多么大的一个数字。王九不耐烦,上前踹了周凤一脚:“叫你走,你就走,哪那么多废话!”

周凤被一脚踹倒,二十多岁的汉子也不爬起来,就这么瘫在地上大声哭喊:“俺家的家当都花了一半去啊,俺家家当已经花了一多半去啊……”

“嚎什么丧!?”王九怒道。他再一步上前,抬脚用更大的气力再给了周凤一下。周凤的哭喊声被王九一脚踹进了肚子里,随即被连拖带拽拖出了门外去。

韩冈看着周凤脸皮蹭着地被拖走,心里免不得有些发寒,当真是不把人当人看。

李留哥视若无睹,转过头对韩冈道:“韩秀才,你真真好运气。刘书办看你是个读书人,才抬举你。莫要辜负了刘书办的一片心意。”

韩冈略略定神,拱手谢道:“刘书办的恩德韩某自不会忘,定当用心酬谢!”再回头看了看库房,“不知监库该如何交接?库房里的军器也该在交接时点算一下罢?”

李留哥满不在意的一挥手:“这些等明天再说!”

“万一库中有个什么短少,又该如何?”韩冈单刀直入的追问。

“就算这只是县中的军器库,也没人敢从中偷盗。盗取军器,轻的也要三千里流,重的便是黄泉路上走。谁有这胆子?!”李留哥也许是怕韩冈在追问下去,转身便要走,“今夜现在这里歇一夜。等明日办交接时再清点。”

“是!是!韩某知道了!”韩冈冲着李留哥的背影连连点头。心中的仇敌名单上又添上了刘书办和李留哥的名字。少说也要八十贯的位置,竟然随随便便就让给了没有送钱的穷措大,而这位穷措大还刚刚往死里得罪了一个有实力的同僚……可能会是好心?!也只能骗骗呆子罢了。

ps:今天第二更,高潮就在下一章。

