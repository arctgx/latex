\section{第16章 三载愿终了(下)}

黄怀信很少见到天子如此急切的模样。

刚刚将用火漆封缄好的礼部试录取名单呈递上去,赵顼就立刻让身后随侍的蓝元震将之拆开。根本不问黄怀信他方才去贡院,有什么见闻,考官之中是不是有下情要禀告。接过拆开的名单卷轴,就立刻展开翻看了起来。

在最前面的十几人中,赵顼并没有找到自己想要看到的名字。赵顼微感失望,一路向左边看过去。

看到一半,赵顼移动中的视线定了下来。盯着纸面上那一列的姓名、籍贯、年甲,以及入贡的所在,看了好一阵。便抬起了头来。

“黄怀信。”他叫着下面内侍的名字。

“奴婢在。”黄怀信连忙将脸压得更低。

“金明池里的龙舟是你监修的?”

黄怀信一愣,“的确是奴婢奉旨监修的。”

“这事做得好。”

赞了一句好,示意跪在下面摸不着头、但仍叩头谢恩的黄怀信退下。赵顼将卷轴一收,问着身后的蓝元震,“听说王安石昨已招了韩冈为婿?”

蓝元震是同提举皇城司,京中的传言消息当然知道得很多,韩冈的婚事也是他前些天向天子禀报过的。听着天子明知故问,他仍连忙弯下腰:“回官家的话,正是如此,亲事是在腊月的时候决定下来的。”

赵顼点了点头:“代朕去中书恭喜王相公吧……有了个进士女婿。”

蓝元震方才在赵顼身后,就已经看见了写在名单上的‘韩冈’两个字。暗惊于天子对韩冈的的重视,竟然不等正式发榜,就要先派人去给跟王安石说。

他凑趣的向赵顼拜贺:“恭喜官家又得一良臣。”

赵顼呵呵笑了起来,很是开怀:“本就是朝中大臣了……”

……………………

春雨绵绵。

比起数日前两场让王韶府中后院里的水塘都漫起来的暴雨,今天这细细的雨丝才像是春天该有样子。

雨丝落于刚刚生发的树叶之上,都没有一丝声响。只有从屋檐上滑下来的水流,才在墙角处的青石板溅起绵绵不绝的水声。

两株韩冈叫不出名字的小树,刚刚生发的枝条,嫩绿中掺着嫩红,掌心一半大小的新叶,在雨水的冲刷下,清新可爱。

韩冈发着呆,望着窗外沐浴在春雨中的庭院。写了一半的文章摊在面前,手上的笔却已经不知停了多久,笔尖软毛上的墨迹都发干了。

韩冈的性格和为人,让他不习惯对他人暴露自己软弱的一面。只有独自一人的时候,隐藏在心中的情绪才会泛起。

今天就要出成绩了。究竟是中,还是不中,都将在几个时辰后有一个准信。

对于这等事关官场生涯的要事,再深的养气功夫,也免去不了他心中的紧张。韩冈从来都不是淡泊名利的人,既然有心在这个时代一展,就不能因为一个仅是资格,而被绊了手脚。

在水声中发了一阵呆,韩冈涣散的视线又重新凝聚起来。自嘲的笑了一笑,能做的都做了,心慌意乱的是等,心平气和的也是等。结果都不会因为自己现在的心情而改变,根本没必要去多想。

重新给毛笔沾了墨水,韩冈提笔挥毫。

王韶、王雱还有他自己三人猜测出来的殿试题目,韩冈已经模拟了五六份卷子,从不同的角度来评价新法推行数年来的优点和缺憾。最后到底取用哪一篇,就要看天子所出题目的偏向了。

不过这些文章基本上还是熙河、秦凤两路说得多一点,一方面提醒天子他韩冈的功劳;另一方面,这也是附和天子的意愿,让赵顼了解到他所想了解的情报。

如果没能通过礼部试,现在写得这些文字自然便是个笑话。只是一旦他被取中,就是他韩冈未雨绸缪的过人识见。

埋头于笔墨之上,韩冈振笔疾书。自从去年年中开始锁厅,这半年多的时间,他连续不断的挥笔作文,平均下来,基本上就是两日一篇的速度。时间长了,文笔进步是不用说了,而他写作的速度则进步得更快上一分。

不用半个时辰,韩冈已经完成了一份二千余字的习作。就算在快速的书写中,纸上的文字也没有一丝一毫的歪曲变形,依然工整无比。旧时的近于三馆楷书的笔力,几年来,也更上一层楼,个人风格重了几分。

慢慢的细读着文章中的词句。手上的笔在文稿上点点画画,干干净净的一份手稿,很快就被一团团墨迹的给充满。

当屋外水落石面的声音终于小了起来,韩冈也觉得他这篇文章已经改得差不多了。前后看了两遍,他重新拿过一张纸,开始动笔誊抄。

一行行文字出现在纸面上,修改、删减到只剩一千五六百字的文章,很快抄写完了大半。

天色暗了下来,雨也快要停了。门外的走廊上,一串急促的脚步声由远及近。

没有敲门,王厚就一下冲进了韩冈的房间,大声的喊着:“玉昆,恭喜了!”

韩冈的笔一顿,,但立刻又继续的写了下去。

‘……愚憧仓促,言不及究,敢具所闻以献,伏惟圣心加察。幸甚。’

横平竖直,一丝不苟,就算听到了这个期待已久的喜讯,韩冈依然没有一点动摇的将一篇文章的最后几行字抄了出来。

写毕,放下手中笔,收起身前纸,才起身对王厚拱手谢道:

“多谢处道通报。”

王厚见着韩冈舒缓自如的举动,先是为之一楞,继而摇头笑叹:“玉昆,你这是要做谢安吗?”

韩冈微微一笑,“小弟可没穿木屐,不会跌着绊着。”

两人对视一眼,顿时又爆发一阵大笑。

东晋谢安听闻淝水之战谢玄大获全胜,九十七万前秦军全师溃散,也不过平平淡淡说了句‘小儿辈胜了’,照样下他的棋。但当他起身外走的时候,却在门槛处绊掉了脚上的木屐。

看似平静,其实已经激动不已。

韩冈纵声大小。

三年了,盼着这个资格由三年了。

寤寐思服,辗转反侧。

夏练三伏,冬练三九,辛苦如许,终于是一个进士了。

拿到了进士资格,挡在他走向宰执道路上的的制度阻碍,已经不复存在。

王厚仍有些惋惜:“只可惜名次不甚佳,在百名开外。”

“能得中已是万幸,就算是最末一名也没有什么好遗憾的。”

殿试定高下,省试定去留。极端点来说,省试的最后一名跟第一名的地位是同等的。要分出高下,还是在殿试上决定出来。说是这般说,不过韩冈也无意去争一个好名次,有一个进士他已经心满意足。

“说的也是。”王厚又道,“久旱逢甘霖,他乡遇故知,洞房花烛夜,金榜题名时。玉昆,四喜之中,这下可是有三喜了。”

……………………

韩冈已经是进士了。

王安石带着这个消息回到家中,对此最开心的不是王旖,而是她的母亲吴氏。

三月初殿试,接着是琼林苑赐宴。赐宴之后,已经二十岁的二女儿就终于可以嫁出去了。

韩冈考试后,吴氏阿弥陀佛不知念了多久,深怕性子倔强的韩冈,脾气上来硬是要考中进士再娶女儿。

现在终于可以放下心来,也不用整日念佛了。

吴氏喜不自胜的,拉着王旖的手,一个劲的说着,“过两日就去大相国寺还愿,当初娘为了二姐你的婚事,不知许了多少香火,今次终于要去还上去了。”

王旖却是沉默着。

韩冈通过了礼部试。她有几分欣喜,也有几分烦忧,甚至有些心慌意乱。

那一位要共度一生的良人,她还是觉得他真的是难以琢磨,心思、个性都是。

韩冈的人当然不差。

二哥对他赞不绝口就不提了,心高气傲的大哥见过他几次后,也点头赞许了几句。王旖也知道能让大哥认同的同辈中人,究竟有多难得。王旖更清楚,一向疼爱自己的父母,也不会随随便便为她选一个不成体统的夫婿。

而王旖当日去见韩冈,也觉得他,并不比她族中那些文采飞扬的叔伯兄弟稍差。甚至在英武之气上犹有过之。

可她去见韩冈是为了拒婚的,却不知怎么就变成了议婚。

只因一番话,韩冈就改变了心意,不但大哥、二哥都惊讶莫名,父亲母亲也是一样。

但王旖真的不知道是自己怎么说服他的。

每每回想起当初与韩冈的对话,王旖不由得苍白了脸。

难道是可怜自己吗?

真的是认为耽误了自己的婚期,而为了补偿才娶自己的吗?

王旖捏着手上绣的一幅鸳鸯荷花图,指节都发白了。比起她过去的作品,这幅刺绣已经进步了很多,都是这些日子来,母亲和大嫂催着她日夜练习出来的。

可韩冈是真心诚意愿意与自己白头偕老的吗?

也许这个想法是太奢求了一点,但王旖真的不希望自己的丈夫,是因为同情或是可怜的心思来娶自己。

婚期在即,王旖仍是心乱如麻。

