\section{第20章 冥冥鬼神有也无(22)}

今天是熙宁九年的除夕,而明天就是熙宁十年的正旦。

驻扎在富良江北岸的征南大军的军营中,上上下下里里外外都是喜气洋洋。就是天气暖得如同初夏,没有冰雪点缀,让人感觉有些异样——刚刚从北面运来的一人两匹绢、二两绵的冬料,看着都像是对这天气的讽刺。

只是不发也不行,这是该有的军饷的一部分,短了少了,兵变都有可能的。除了应该有的俸料,另外也少不了与敌军作战的赏赐。每逢作战,依例要开双饷,行军、上阵都要有银钱发下来。光是出兵后的一个月,就是价值二十多万贯财帛发了下去。

此外在韩冈私下里的提议下,章惇还亲自签发了军令,给参战的一万多将士,又增发了一份厚赏。靠着从交趾人手里得到的几十万贯财帛,官军从上到下都算是发了笔小财。

将士们欣喜之余也暗恨交趾人太穷,又不事生产,发下来银钱绢帛,都是宋钱宋绢,明显有许多是之前从邕、钦、廉三州抢回来的赃物。不过看到这些赃物,倒是人人盼着去升龙府,把交趾人他们抢去的都拿回来。

另外还有件事算是给了将士们一个惊喜,正抢得在兴头上的诸多蛮部,知道人情世故,全都派人送了厚礼过来。财帛什么的倒是不多,他们也舍不得多拿出来,只给章惇、韩冈几位主帅大将送了重礼,但几乎能塞满营地的一头头肥牛,却是让下面的士兵们对他们孝心赞不绝口。

广西从不缺牛,牛肉比猪肉都贵不到哪里去,各峒也不在乎多送,半个月下来,竟然陆陆续续送了上千头过来,另外还有三十几头大象,说是象鼻子要吃新鲜的才好。

“这要吃到哪一年啊!?”章惇摇头叹气,在北方就是天子也尝不到几口牛肉,可他到了广西之后,隔三差五就能吃上一顿;一入交趾,更是日日不缺;而且还因为近着广东,一些来自海外香料都不贵,这口腹之欲可是好生的享受了一番,“就算营中一人一天一斤肉,一万人也要吃上两三个月才能解决,真正要这么放开来吃,肯定会吃伤掉。这还是七八百斤重的牛,几千上万斤重的大象还没算进来。李信去西面帮了一次忙,说是一千多人用了三天,才把五头大象身上的精肉给吃光。”

“这不要买个好吗?打下升龙府后,可就要分地盘。”韩冈笑着,自从来到这个时代,他也就是在陇右的蕃部里吃过牛肉,平常有禁令在,牛不是死了或是受伤,根本就不给杀来吃,广西这里倒真是不错,“现今诸部一个个用足了气力在各个地方拆屋烧房,大一点的庄子连围墙都捣掉,不就是怕分地之后,给人捡了便宜去?”

章惇点着头,眼下交趾的形势就是等着打下升龙府后分赃了,一起攻进来的诸多蛮部,没有人还会认为交趾有能力抵抗官军。自官军率领溪峒联军攻入富良江平原之后,近一个月的时间,近十万兵马如同犁地一般,将平原上的反抗基本上都给肃清了,出头的抵抗者即便抵挡住了蛮部的攻势,也会在装备精良的宋军攻击下灰飞烟灭。

只是他又叹了起来:“不过区区一个交趾,用了一个月才攻下了一半土地。不打下升龙府,前面赢得再多,也不能算是成功。希望黄金满他们不要犯糊涂。”

“想来还不至于。”韩冈倒是相信诸部洞主们足够聪明,“没有攻下升龙府,谁也不能安心的瓜分富良江北岸的土地,光凭诸部旧地,可养不活这么多新添的生口。”

“他们能这么想就最好,这最后一战也需要将他们派上用场。”章惇抬眼看看帐外的阳光,“今天是除夕,如果交趾要来,多半就是这两天了。”

“来也好不来也好,都会做好准备的。”

“虽然不指望太多,但要是交趾水师当真来了,这一战也可以早一点结束。”章惇顿了一顿,深有感慨的笑了起来,“都没想到这一次南下,竟然会用上两年的时间。”

“……说得也是。”韩冈抿了抿嘴,由衷的表示赞同。时间过得也真快,不知不觉之中,他和章惇已经在五岭以南过了两个新年了。而且两个新年,都是在战鼓声中度过,都没有一个安生的日子。

时如逝水,说得一点都没错。如果能早一点结束,韩冈也是十分乐意的。

章惇接下里要去巡视营中,韩冈也有事要做,两人遂一同从中军帐走出来。经过了行营参军们的小帐,就听见里面传出声来,章惇停了脚步,韩冈也跟着停下。

“……官军的口粮足够,现在的份量吃个三年都不成问题。溪洞诸部那里也没有传话来说断粮,能送这么多牛和象来,根本就不用担心。”

“《孙子》里面,满篇都是因粮于敌。难怪契丹、党项这么喜欢开战,只要放手去抢,粮秣什么的都不用担心。若是大宋攻打西夏和辽国时,也能劫掠到足够食用的粮草,早就分出个胜负了。”

“只可惜天下万邦,能当得起富庶二字的唯有中国。南方如交趾这等小国至少还种田,如果是攻打西夏、辽国,想因粮于敌也没处去找。”

陈震和李复的对话,韩冈在帐外听见了,两人对视一眼,都是微微一笑,便举步离开。经过了一个多月的战争,原本还带着书呆气的几位幕僚,都飞快的成熟起来,已经能毫无顾忌的高谈阔论劫掠的好处。

因粮于地的确是好事。

韩冈之前为了保证军需调集了上千匹驮马来运送粮食,紧急在广西开办的两处马市,全都为了给安南行营服务。但当官军攻入平原之后,只是在粮食就不再需要任何补充了。全都是靠了沿途州县的存粮来补充。

“如果是在国中,临阵脱逃,又将粮秣留给敌军,这样的州县官都是该论死的。”章惇扭头对韩冈说着。

“他们的失职,全都便宜我们了。”韩冈笑着回道。

“也多亏了玉昆你定下的方略。要是依照在熙河、荆南的行事,即便能将他们迫降,也不会有如今这般省心,日后反而会添多少麻烦。”

“交趾也有科举,这些官员有许多都是科举出身。不过能读书全都是富户大族,若是要想并吞交趾,设立州县,全都得靠着他们来支持。”韩冈摇摇头,嗤笑一声,“这哪里能让人放心?还不如清光了爽快。”

章惇嘴角扯动了一下,同样是冷笑,“也是多亏了十万蛮军,才这般省时省力,换作官军来动手,不可能这么收拾得这么快。”

交趾一国的总人口,除去受到羁縻的部落以外,估算是一百到一百二十万上下。北岸全部的人口不会超过五十万,韩冈让行营参军们粗略的统计过几个州县的丁产簿,总计七万户的样子,以一户五口人来算,也就是三十五万上下,依照大宋隐户逃人的比例,大约还能再加上十万人。

这四十多万不到五十万的数量,是男女老少的综合,其中的其中可以派得上用场的男丁最多也只能占三成,而攻入这一片平原的蛮部就接近十万。

论起实力,交趾和广西南方的溪洞蛮部,其实差不了多少。就算只是三十六峒蛮部集合起来,也能抵挡得住交趾的正规军。广源州诸部联合,也同样能与交趾较量一番。可偏偏这一干部族矛盾丛生,根本无法统和,只能任由交趾鱼肉。但如今他们以官军为核心,成为拥有了共同目标的整体,实力迥然一变,远远压倒了交趾军。

“现在这江北一片,算是差不多了。广源、左右江几十家洞主都给了回复,总计三万精锐,都已经聚集到了江边上。只要经略司一声号令,他们就能立刻乘上木筏渡过富良江。”韩冈陪着章惇在营中走着,左右前后的将校士卒跪倒了一片,挥了挥手让他们做各自的事情去,他对章惇笑道,“如果船场那边能多毁掉几艘交趾战船,他们过去的也能方便些。”

“那是自然。不过若是交趾人没上钩,也容不得任何人推脱。我就不信,这区区几十艘战船还能将几百里长的江面都封起来。”

韩冈轻轻点头,慢慢走着。就算交趾军没咬钩,还得照样杀过富良江去,只是多了些阻碍而已。

用兵使计,怎么可能把希望寄托于敌军是否会上当?哪里可能这般用兵!韩冈一直都是抱着能钓上来最好,没钓到也无所谓的心态。即便交趾水师不来,也能打过富良江。

诸部中的精锐全都已经奉命集结在江岸附近。这是安南经略招讨司第一次向下达正式的指挥,之前的命令只算是放羊吃草,但想要攻过富良江,就得集结他们的力量,在两百多里长的江面上同时渡江。万舟齐发,看看交趾人怎么防得过来。

“一旦万舟齐发,区区几十条战船根本算不上什么。”

