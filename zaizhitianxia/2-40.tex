\section{第11章 五月鸣蜩闻羌曲(三)}

【第一更,求红票,收藏。】

古渭事毕,王韶和高遵裕启程回秦州去了。他们在古渭寨也不过待了六七天的样子,却是在地狱和天堂里走了一圈,如今终于要返回和平安定的人间了。

跟他们同行的,有俞龙珂,有瞎药,有全族死了近一半,又给抢成了穷光蛋的张香儿,还有七部中的其他几部幸存下来的几个族长,他们都带着从人,青唐部的两位族酋还各自领着百十位功劳甚大的将佐,浩浩荡荡的队伍一齐往秦州进发。

不过,这群人中间却没有韩冈的身影。

站在城门处,望着行在路上都互不相让的俞龙珂和瞎药的部众,韩冈不得不承认,竞争心理有时候很管用。

俞龙珂和瞎药都想要封赏,却不都想受到宋廷的束缚,对献上田籍丁簿之事毫无兴趣。王韶和高遵裕便分别找了两人说话,先对着俞龙珂大赞瞎药精明能干,又在瞎药面前赞赏俞龙珂忠勤为国。看准了两兄弟之间不会互相通气,王韶和高遵裕肆无忌惮的欺着两人,挑拨得两人的关系愈发的紧张。到最后,利诱威逼之下,俞龙珂和瞎药都答应先向朝廷做个恭顺的样子出来。

而俞龙珂本也是不想去秦州,只想派着两个得力亲信过去,疑心重的老狐狸向来都在意着自己的安全,但见到瞎药答应随行,却也跟着点头。而他所不知道的,在前一天夜里,韩冈曾找过瞎药谈了心,隐隐透露着高遵裕有心支持俞龙珂,一席话就让觊觎青唐部族长之位的瞎药,主动要求去秦州。

这等一家吃两头的招数,并不是出自韩冈的建议。虽然他有想着出个主意,但王韶和高遵裕却已经先做了出来,他夜中去找瞎药谈未来谈理想,也是奉得王韶的命令。真的论起心机,能在官场上混得风生水起都不是蠢货。看出俞龙珂和瞎药兄弟之间的微妙关系,眼光锐利的王韶和高遵裕都能做到。而趁机在其中混水摸鱼,他们也是一般的行家里手。

其实俞龙珂和瞎药也不差,就是被个‘利’字弄昏了头脑,任由两名官场老手从中牟利。但两人依然保持了底线,尚没有为了压倒自家兄弟,把自己的老底都丢出去,也占了不少便宜。谁让王韶和高遵裕有求于他们呢,这一点,青唐部的蕃人也同样看得出来。

三方四人勾心斗角,到最后的结果,却算得上是皆大欢喜。看着这样的结果,韩冈不由的叹着,这世上果然还是聪明人居多。

目送着返回秦州的队伍渐次走远,韩冈返身回寨中。刘昌祚不在,王韶、高遵裕又走了,现在的古渭寨,他可是官品排在前三的官人——现在寨中的文武官员,其实也只有四人。

韩冈之所以还留在古渭,没有一起回秦州,还是因为蕃部的事情。俞龙珂和瞎药出战,虽然打了个董裕措手不及,以加起来都不到一半的兵力将董裕本部彻底击溃,是个辉煌的胜利。但这一战。终究不可能毫无损伤,两边都有近百人的战死,总计又有两百多的轻重伤。

如果这些伤兵送回家去将养,在缺医少药的蕃部中,却很难得到有效的医治。而正好韩冈事前就答应过俞龙珂会救治此战受伤的伤员,便让古渭疗养院将他们都收留了下来。将四百多张床位的医院,占去了一多半。

王、高两位提举都下了指示,要尽一切可能将他们救治,而韩冈也很高兴,这代表又可以为伤病营伸手要钱要物,同时朱中他们又可以练练手了——前段时间古渭寨谨守寨门,一点风险都不冒,刘昌祚又带了两千兵走了,只剩下三分之一兵力的寨子,病人自然也少了许多,搞得医生护工比来求治的伤病还多一点。

不过青唐部送来的伤兵中,有一多半轻伤员住个几天就能出院了。他们都只是受了一点皮肉伤,若在往日,在河里沟里找点水洗一洗,止住血、包起来,也就算是治过了。之后有的安然痊愈,但也有许多化脓感染很快就死掉了。

尤其是如今的这等炎炎夏日,小小的只有一寸不到的伤口感染流脓,甚至发黑发臭,变成坏疽,最后要了人性命的情况,多不胜数。

就是因为有这种事,俞龙珂才会特意在出战前跟韩冈提了要求。一场大战下来,死掉的不说,重伤员始终是少数,更多的是轻伤员。缺胳膊断腿等死的重伤员死了倒好,省得浪费族中的粮食,但轻伤员因为一点小伤口,就病死了的结果,任谁都难以接受。

而这一切在疗养院中,却极少出现。整洁的卫生条件,干净的饮食,充足的药物,还有周到的护理,这样死亡率如何会降不下来?

对于韩冈给予的无微不至的关照,入院治疗的蕃人们都看在眼里。就算是吐蕃蕃人,也许不如传言中淳朴,也许有些狡猾,但忘恩负义的人始终是少数。其中的绝大多数,对主持救治了他们的韩冈,都是感激颇深。

当韩冈走进疗养院时,庭院中,已经不少轻伤的蕃人在走动。他们一见到韩冈,便纷纷合十行礼,口宣佛号。

孙思邈的名声不知是谁传到了吐蕃人的耳中。孙真人药王的头衔,到了蕃人口里就变成了药王菩萨。而传说中身为药王弟子的韩冈,也变成了药王菩萨座前的行者,好像还带着护法金刚的身份——因为韩冈让人一刀斩了结吴叱腊。

斩了声名远播的名僧,却反倒成全了韩冈的名声。韩冈既然在蕃人们的心目中坐实了药王菩萨座下弟子的身份,他所斩杀的,自然是佛敌。可怜的结吴叱腊,便成了混入佛门,谋图不轨的妖魔。据说此事连俞龙珂和瞎药都信了几分,要不然韩冈后来的一番话,也不会那么容易就说动精明能干的瞎药。

韩冈很和气的与向他行礼的蕃人们打着招呼,有些多见了几面认识的,甚至走过去嘘寒问暖一番。这等亲切待人的做法,自然使得他们感激涕零。

在重伤员的病房中巡视了一圈,查看了食水和药物是否完备,韩冈最后回到自己的房间。小屋简陋得很,除了桌子、床榻和几个木墩,便没有其他的家具。也不是没有人劝他住进城衙,里面的寅宾馆,就是给暂住的官员准备的。不过韩冈给拒绝了,留名示好的机会他怎么能放过?他就住在病房旁边,日夜守候,籍此收买人心。

拿起一卷随身带来的《孟子》,韩冈细细研读。虽然后世并称孔孟,但在此时,孟子的名声还未达到亚圣的高度。在汉唐,孟轲也不过是跟子思、荀况,后世的扬雄等人并称的儒家先贤之一。直到韩愈横空出世,推崇孟子,并创立道统论,说明了儒家道统是尧传舜,舜传禹,禹传汤,汤传文、武、周公,文、武、周公传孔子,最后由孔子传给孟子。而‘轲之死,不得其传矣’——轲是孟子的名字。

不过韩愈并没能一下扭转儒林对孟子的看法,就算到了现在,儒家学者中仍有许多反对者。如司马光就不喜欢孟子,反而推崇扬雄和荀况,曾经说过‘唯独荀子、扬雄二人,排攘众流,张大先王正术,使后世学者借以明了王道所在。’

韩冈在程颢那里,没少听他批过司马光的学术观,说司马十二空谈至君尧舜上,鉴史知得失,却不知儒门大道之所在。

但在韩冈想来,司马光毕竟是写出《资治通鉴》这本帝王学教材的人物,当然不会喜欢孟轲民贵君轻的观点,甚至著《疑孟》,说孟子是‘为礼貌而仕’,‘为饮食而仕’,是‘鬻先王之道以售其身’,跟此前一位有名的学术大家李觏一样,都视孟子是‘五霸之罪人’,以仁义乱天下。儒家道统也不是如韩愈所说的自孔子传孟轲。

但王安石尊崇孟子,程颢程颐尊崇孟子,而韩冈的老师张载也一样尊崇孟子。不论从师传角度,还是日后参加科举的角度,韩冈都有理由去研读孟子的文章,去研究从孔子传曾参,曾参传子思,再从子思传给孟轲的这一儒学支脉的理论——孔子述《论语》,曾参著《大学》,子思著《中庸》,而《孟子》自然是孟轲的著作。朱熹总结出来的四书,其实就是这一支脉的流传。

只是不过韩冈没能读多久,一个让他想不到的客人上门来拜访。韩冈只听了通名,连忙放下书,快步出门去迎客——秦凤道上有名的老军医仇一闻竟然来古渭寨找他。

站在门口,仇一闻鹤发童颜,雪白的尺半须髯,飘飘有仙人之态,身后一个小药童,背着他的药囊。

一见仇一闻,韩冈赶忙行礼,仇一闻的年纪和人望摆着,德行又高,容不得他摆着官人的谱。直起腰后,他便责怪道:“仇老,如今天气暑热,你怎么还在道上奔波?!等天气凉下来再走不行吗?”

“唉……”仇一闻叹了口气:“老夫是向韩官人你求援来的。”

