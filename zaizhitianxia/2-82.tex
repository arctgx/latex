\section{第20章 心念不改意难平(一)}

韩冈第一眼看到郭逵的时候,就被惊到了。

这倒不是郭逵长得骇人听闻,惨绝人寰。前任的延州知州,新任的秦凤经略有着一副堂堂相貌,眉正鼻直,须髯盈尺,威严自生。身材又是雄壮硬朗,比韩冈还要高大一点。再加上可能是因为他在枢密院镀过的金身,虽然与李师中等出迎官吏言笑不拘,但仿佛随身就带着一种若有若无的威压,让他身边的人不由自主的感到拘束。

不过韩冈连王安石都见过,也不是没见过世面的村夫,不至于被郭逵的气场惊到。之所以会吃惊,却是因为郭逵实在年轻。

韩冈一直都听人在说郭逵是宿将,久在军中,老于兵事。听得多了,耳朵里都要生茧。渐渐的,在他心里形成的郭逵,是一个须发花白,虽然显得老态但眼神锐利如电,精神矍铄不甘服老的老将形象。

但今天一看,郭逵却是才五十不到的模样!比他旁边年近六旬的李师中看起来要小上许多。而正与郭逵说话的张守约,他这个老军头常年熬打筋骨,看上去比实际年龄要年轻。而当了钤辖之后,心怀舒畅,更是显得容光焕发,六十岁的人说他五十岁都有人信。可他在郭逵面前,也同样显老。

韩冈站在人群中,看着郭逵微笑着跟来迎接他的官员一一问好寒暄,毫无不耐之色。他笑起来亲切温和的样子,根本不像传说中的那样难以相处。

“玉昆,怎么一直盯着郭太尉。”王厚的声音自身侧传来。

韩冈将头微微偏过,神色依然庄重,用着只有王厚能听到的声音说:“小弟是在想,郭太尉是在年轻,比起李经略来,就像是两代人。”

“李经略比郭太尉大了十岁还是九岁,当然显得老气。”王厚同样保持着严肃端正的姿态,嘴皮微微动着,“不过这些日子,李经略也的确显老了……心中不痛快嘛!”

韩冈没再听下去,王厚的第一句话就让他又惊讶了一下。李师中今年虚岁五十八,几个月前,他做寿的时候,韩冈还跟着王韶去他府上送了寿礼。如果王厚说得没错的话,郭逵比李师中还要上九岁十岁,这么一算,几年他虚岁才不过四十九!

韩冈在心中又算了算,既然郭逵现在才四十九,那他英宗治平二年进入枢密院的时候,就仅仅四十五岁。这个年纪就已经升到了本朝武将所能达到的巅峰,再看看张守约,或是被踢出秦州的窦舜卿,怕是每一个都会在心里叫着,这人和人的际遇当真不能比——就像韩琦三十多岁进位宰执,而以王安石之才,则是到了快五十岁才在崇政殿中有了一席之地一般。

而所谓宿将的说法,也很容易就能解释了。领军多年的将领就是宿将。如果二十多岁就开始领军,到了五十,领军二十余年,一般就可得到这个称号了。郭逵是靠着父荫入官,而他的兄长郭遵三十年前战死在三川口后,他就靠着郭遵阵亡得来的荫补升了两级,这时已经可以算得上是将领了。三十年领军,得称一声宿将,也是理所应当。

韩冈在打量着郭逵,同时,也有人在打量着他。

郭忠孝沉默的跟随着他父亲向前走着,不过他眼角余光都在人群中梭巡。没费他多少功夫,很容易的就从一群人中找到了韩冈的身影。

秦州年轻的官员并不多,二十上下的就那么几个。而在这几人中,有一高大俊朗,年岁介乎青年与少年之间的年轻人。气质纯粹、风仪出众,立于一众卑官之间,就如鹤立鸡群一般,显得分外显眼。

而且站在他旁边,有一个与他年岁相当的青袍官员,跟方才通过名的王韶长得极为神似,当是王韶带在身边的次子王厚。会与王厚并肩而立的,不是敢于孤身夜入古渭,于军事上亦多有发明的韩冈韩玉昆,还会是谁?

郭忠孝自己也不过二十三四,以家世论,足以自傲,右殿班直的荫补就在身上。以学问论,他弱冠之前,就已经在二程门下就学过两年,深得程颐赞许。只是看到了风姿秀挺的韩冈,他原本因为郭逵对韩冈的赞许,而升起一点嫉妒心没了,却多了一些不服输的念头。

——韩冈能做到的,自己一样能做到,二程的门下,不会输给横渠弟子!

韩冈总觉得有人在瞥着自己,就是那个跟在郭逵身后的青年,相貌与郭逵有几分相似,多半是儿子。而郭逵本人,也是不时地扫过来一眼,有几次他和韩冈的视线差点就给对上。

韩冈不知他们父子两人到底为什么总是看着自己,但他们的视线,让韩冈觉得很不舒服。有窦舜卿、李师中在前,郭逵父子对他的关注,登时就让韩冈心中警铃大响。

不再看着郭逵,韩冈的注意力落到了差着郭逵半步的李宪身上。勾当御药院的大貂珰脸上的笑容有点发僵,眉心微微皱着,感觉上他对眼前的郊迎之礼有些不耐烦了。

韩冈此时心里,也在想着快点结束这个见鬼的郊迎仪式。早些回到州衙,也好看看李宪到底带来了什么好东西。

郭逵好像是听到了韩冈的心声,在跟十几位州中文武高官一一见礼之后,他不再跟穿着青袍的底层官员用着些废话寒暄了,而是跟着李师中,和李宪一起从来自秦州的成群的文武官中走了出来。

‘终于完了。’韩冈正这么想着,却不提防郭逵在他面前停了步。

与韩冈面对面的郭逵,眼神幽深难测,看不出什么情绪波动。只是上下打量了韩冈几眼,便问道:“可是韩玉昆?”

‘甫见面就找上门来了,还真是心急。’韩冈暗叹了一声,向着郭逵拜倒:“韩冈拜见太尉。”

“不须多礼!”郭逵伸出双手,一把将韩冈牢牢托住。韩冈腰腹用力,想要硬是拜下去,把礼数做足。但他却偏偏弯不下腰,郭逵的双手如同铁铸,从被抓着的两条手臂上传来的力道中看,他的阻拦决不是在做样子。

韩冈又试了两下,发觉郭逵没有松手的意思,终究还是顺势直起身,“韩冈失礼了。”

郭逵却微笑着,“果然是英雄出少年!”

说完,也不待韩冈出言逊谢。径直走到坐骑身边,跳上马,与李师中、李宪一起先一步向秦州去了。

周围的官员都看了过来,而韩冈神色平和,看不出激动、也没有惊讶。只是他的心中却在翻腾。从郭逵的言行中看出了他对自己的看重,但这情况,比郭逵一门心思跟自己过不去,有着同样的麻烦。

他瞥了眼脸色骤变的王厚:‘这墙角挖得可不地道!’

夜中,州衙灯火通明,数十支巨烛将大堂照得透亮。接风的酒宴正是最热闹的时候,尤其是以参与过古渭大捷的几人,都被轮番敬过。

就在酒宴开始之前,李宪已经宣读过了诏书。

王韶因功加官。不过官品到了他这个等级,又是刚刚晋升过,不可能让他一飞冲天。仅仅是晋了一阶,多了个检校水部员外郎的官职,同时又有了一个开国县男、食邑三百户的爵位。

而高遵裕,他还在古渭,没有来得及赶回来。不过李宪肯定是要去古渭寨的,不然给青唐部的封赏,以及安抚纳芝临占等部的赏赐,都不好派发了。

至于韩冈,以他在古渭大捷中光彩夺目的表现,使得他入官不过半年,便得到了第二次晋升。只是他从试衔知莱州录事参军事,升到威胜军判官一职,算起来仅仅是晋升了两阶。依条贯,文臣在选人和京官阶段,有出身、有军功者,可越级晋升。韩冈有功于战事,便一次晋升两阶,这点并没有错。但他的功劳真要计较起来,决不止只跳一阶。

韩冈奉王韶、高遵裕之名,夜入青唐城,说服俞龙珂出战,他执行的任务是古渭大捷中最为关键的一环,。而他得到的,则比起郭逵当年孤身说服保州叛卒时要微薄了不少。当叛乱军队因郭逵的劝说而出城投降时,他可是得以直升环庆兵马都监、和从七品的阁门祗侯。

唯一值得安慰的是,韩冈的晋升速度却又比进士出身的官员快得多。今年的新进士,除了状元叶祖洽和二三名的榜眼外,其他人都在判司簿尉的这一文官中的最低层熬着资历。自然,进士一步步提升是循例,而韩冈的晋升却是靠着军功来的特例。如果日后再无功劳补充,韩冈还是只能看着进士们一步步的超过他。

不过可能是为了弥补韩冈在官阶上亏欠,他在其他方面便得到了补偿。由于父母俱在,以韩冈选人的身份不便封赠,因而他的两名殁于王事的兄长,便各自得到了追赠。这对朝廷来说是惠而不费,而对韩冈来说,他两位兄长的灵位和墓碑都可以换个大一号的了,老子老娘那边看了肯定欣慰。而且还有三百两银,两百匹绢,作为赏赐。

‘算了!’韩冈想着,这也是早在预料之中。才二十岁就由选人转为京官,而且还是入官才半年的新近,不知会遭到多少人的嫉恨。无论是从保护自己的角度看,还是从饿鹰易于驱用的角度看,天子和王安石都会选则把他的官位压上一压——这种做法,正常无比,连王韶都是被刻意压制了。

不过如果自己若是再立新功呢?不知到时天子和王相公又会怎么做?很难再压制了吧?

——尤其是又有了一个对自己赏识的新上司的时候。

韩冈举起酒杯,回应着郭逵的善意。

