\section{第21章 克敌破虏展神臂}

远隔数十里之外,张守约还在用力敲着战鼓。战斗打响到现在,年近六旬的老将呼吸已变得很急促,汗水在褐色的肌肤上流成小河。刺骨的寒风中,赤裸的肩膊上热腾腾的白气冉冉而起。可双臂灌注在鼓槌上的力量依然能撼动山岳,敲击出来的鼓声仍旧惊天动地。

“给我杀!”

鼓声下,张守约兴发如狂。四十载从军,无数次上阵,张守约不知多少次的在鼓声中稳步上前。一名名西贼倒在他的枪下,一面面战旗落在他的脚边,震荡的军鼓就是张守约的另一颗心脏,在战场上,鼓声一响,便能让他的血脉沸腾如烟。

谷地中,两军激战正酣。一阵阵的箭雨犹未停歇,时时刻刻都有战士们中箭后的闷叫。一队队铁鹞子不断轮换着从两翼冲杀上前,向宋军阵地抛射出一阵箭雨之后,又转身退回出发点。而带甲步兵的步跋子则在正面整列上前,与宋军的弩弓对射着,以保护骑兵在回转的途中不受攻击。

弩箭从弦上劲射而出,一连串的惨叫随即在目标处响起。党项人的战术,在宋军箭阵之前,却并无太大意义,步跋子和铁鹞子的队列中,被箭矢凿出了一个个缺口。宋人恃之为金城汤池的箭阵,只要阵列成型,便能让任何敌军饮恨。论起射术,关西男儿不在党项之下,论起兵械,宋军的硬弩全无敌手。

不过交战至今,弩箭的发射速度已经渐渐慢了下来。纵然张守约率领的两千兵皆是秦凤路上有数的精锐,也吃不住连续不断的射击所消耗的大量体力。

宋军所用硬弩,力道往往有三石之多,而战弓也是在一石上下。给弓弩上弦,消耗的体力极大,普通的士兵往往张满弓射出十几二十箭后,便手足酸软,无力再起,这也是为什么一壶箭矢只有二十支上下的原因。如果战弓只拉开一半幅度,的确能多射几箭,但这样射出的长箭都是绵软无力,除非拥有极其精准的射术,能直接贯穿敌人的要害,否则就只能在敌军的盔甲上听个响。至于硬弩,却只有拉满一个选择,每次用上三百斤的力道上弦,即便是用的腰腿全身之力,也没有几人的体力经得起这样的消耗。

张守约很清楚,参战的每一位宋军将校都很清楚,这样的相持持续下去,输得肯定是兵力匮乏的一方。两千对一万,意味着党项人可以轮换上阵,而宋军只能咬牙坚持下去。

张守约苦恼的考虑着,在他面前的选择很多,可却没有一个稳妥可靠、能让他将手下的儿郎们顺顺利利带回甘谷城选择。

退无可退,进无可进,如何破局?!

………………

胜利仿佛唾手可得,禹臧荣利强忍住心中的激荡。

身为镇守西夏西南边陲,依附党项的头号吐蕃大族——禹臧家下一任族长的有力竞争者,禹臧荣利一直暗中对自少年时起便光芒四射的兄长禹臧花麻,有着很强的竞争心理。同为新一代中的佼佼者,禹臧花麻却始终牢牢地压在禹臧荣利之上,更得族中长老和族人们的喜爱。也因此禹臧荣利对军功的渴求,对压倒兄长的期望根深蒂固,愿为之付出任何代价。

今次是禹臧荣利第一次统领大军,本想着从甘谷城中骗出几个指挥为自己添些军功,却出乎意料的钓出了张守约这尾大鱼。

两百多步外地红色大旗上,黑字金边的‘張’字,炫花了禹臧荣利的双眼。老将张守约在秦凤路上威名显赫,即是秦凤路都监,又是甘谷城的中流砥柱,若能将其一战击杀,提着他的首级趋往甘谷,那座雄城亦当不攻自破。泼天地军功近在咫尺,让禹臧荣利兴奋莫名。

一切都近在咫尺。

张守约近在咫尺,胜利也近在咫尺,而禹臧家的家主之位,也同样的近在咫尺。

只是宋军的抵抗还在继续,上前冲击宋军箭阵的马步两军,都在不停的承受着巨大的伤亡。

“让撞令郎再上去冲一下。”禹臧荣利清楚,没有一个将领会反对这个命令,汉人不是讲究着以夷制夷吗,撞令郎就是以汉制汉的产物,“只要能冲破了宋人的箭阵,入了甘谷之后,任其快活三日。”

撞令郎听命冲了上去,这些汉人中败类,没有气节,没有尊严,在党项人手下连性命都不能自主,但让他们劫掠同胞,却是个个都争先恐后。

望着前方重新激烈起来的战线,禹臧荣利轻提缰绳,驭马前行。

“少将军!”亲卫不知道禹臧荣利的想法,直以为他打算亲自去冲击敌阵。

“击鼓!”禹臧荣利的命令随即下达,他在战鼓声中放声大喝:“拔旗!中军前进五十步!全军给我听好了!斩下张守约的首级,入甘谷之后,十日不封刀!”

………………

张守约还在苦思一个出路,但党项人并没有等他想出个眉目。对面鼓声已经响起,击鼓进兵同样也是党项人的习惯。原本位于一百五十多步之外的西夏将旗,这时开始缓缓推进,在西贼的欢呼声中,前行了五十步后,又定了下来。

老将军死死的盯着百多步外的那幅白色将旗,旗帜之下的身着全副甲胄的将领,必是西贼主将无疑。将旗的前移,意味着中军本阵的移动,代表下一次攻击即将展开,同时也证明接下来的攻击将更加猛烈。

一万党项精兵随着鼓声开始怒吼,他们的吼声在河谷中回荡,攻势一如张守约所料,突然猛烈起来。前面的撞令郎已经让守在战线上的将士手忙脚乱,而现在,一队队铁鹞子又开始不顾伤亡,不断上前冲击着宋军弩手们的阵地。体力消耗大半的弩手已经跟不上铁鹞子突击的节奏,兵力上的劣势逐渐的暴露出来。防线正在崩解,如同抵御着洪水的长堤,在千军万马掀起的狂涛中一段段的崩塌瓦解。

“都监!”王君万大步上前请命,“让末将去取那贼将的首级!”

张守约低头看看王君万,年轻英俊的骑兵指挥使的眼神坚毅中透着悲壮。张守约又抬头看看一百一十步外的敌军将旗,他慢慢摇头,在鼓声中突的哈哈狂笑,大笑声中透着解脱般的轻松自在:“用不着你啦!……”

张守约甩手将鼓槌丢给就站在一边的鼓手,让他保持节奏,继续击鼓。自己在得力部下的满头雾水中横里走了几步,左手向后一伸,甘谷城的张老将军沉声道:“拿神臂弓来!”

一张形制有些奇异的硬弩,随即被亲兵用双手递到张守约掌中。

‘以檿为身,檀为弰,铁为登子枪头,铜为马面牙发,麻绳扎丝为弦’,虽形为弩,却名为弓——神臂弓!

比起过去的弩弓,神臂弓的前端多了个圆形铁环做成的脚蹬。有着这脚蹬,就用不着踩着弩臂上弦,自不用再担心踩坏弩弓,所以弩弓的力道可以造得更大、更强,普遍达到了四石到五石。这是去年,由蕃人李定献入朝廷。天子赵顼试射过后,亲自取名做神臂弓,并下令军器监加急督造,以期能尽速下发部队。现在张守约手中的这柄神臂弓,正是新近下发到关西诸路的第一批。

一百一十步,这个距离对于长箭来说,除非是顺风,而且是台风,才可能飞到那个距离。对旧式的弩弓来说,也是处在失去了杀伤力的极限射程上。可如果用的是神臂弓,一百一十步却是已经进入了有效的杀伤半径——神臂弓的最大射程,可是达到了三百步!【注1】

神臂弓被递到手中时,已经提前被上好了弦。搭上了木羽箭,张守约举起了硬弩,跟着张守约一起,一个都的神臂弓手齐齐上前,也同时将目标对准了敌将。超过一百具的神臂弓,这是张守约现在最大的依仗。

对准敌将瞄了又瞄,张守约一声令下,自己也随之扣下了牙发扳机。

百十弦响和为一声,百余短矢同时射出,一片飞蝗直扑敌军将旗之下。

胜利就在眼前,但禹臧荣利的眼中只剩下一片血红。与他同站在大旗下的亲兵,和禹臧荣利一起,被百十支利矢,扎成了一只只刺猬。已经仰天躺倒,脸上插着七八根短矢的禹臧家新生代的右手,仍不甘心的高高举着,可转眼就落了下来,连同他的野心,一起砸到了地上。

神臂弓在秦凤战场上的第一战,便是以斩将破敌拉开了序幕。

隔着一百一十步,根本看不分明对面的情况。但转眼间敌军大旗下已是一片慌乱,那名身穿一身硬甲的敌将不见了踪影,张守约眼定定盯着看了半刻,终于确信自己或是其他神臂弓手的确射中了目标。

“当真是神兵利器!”张老将军抚摸着还有些毛刺没有磨去的弩身,对这张神臂弓爱到了极点。

敌阵中传来的号角声呜呜咽咽,如泣如诉。万余西贼,便随之向北潮水般的退去。张守约整个人都放松下来,终于是赢了。但当他看到骑兵指挥的伤亡数目,心情就又变得很糟。

四百骑兵战死有八十多,剩下的几乎是人人带伤,其中重伤的超过一百。张守约很清楚军营中医官的治疗水平,今次受了重伤的一百多名精锐骑兵中,能有一半活下来就不错了。

张守约咬着下唇,最后叹道:“都是些好汉子啊!”

注1:宋代的一步长为五尺,相当于现在的一米五。在《武经总要》的记载中,神臂弓的射程能达到三百步,也就是四百五十米,这点值得商榷,很可能是特例。不过在《宋史•张若水传》中,有七十步连续洞穿铁甲的记载。从这个数据来推算,在一百一十步的距离上,神臂弓应该还能保持一定的杀伤力。

ps:神臂弓是北宋最为有名的神兵利器,传说中能射出三百步。虽然其中必有夸大,但实际威力,从神宗朝之后的历次战事中,便可见一斑。

今天第二更,红票,收藏。

