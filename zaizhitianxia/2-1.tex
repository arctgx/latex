\section{第一章 一入宦海难得闲(一)}

韩冈回到秦州已经有半个月了。不同于上京时的天寒地冻,也不同于出京时的乍暖还寒。三月末的西北早不是冬季时黄色和白色的混和,春风已吹至玉门关头,举目秦州,皆是郁郁葱葱的绿色。

春天的阳光再舒服不过,气温也是一样舒适。清早起来,韩冈穿着一身单薄的短打,照着往常锻炼身体。即便是在东京城的时候,韩冈依然保持有规律的健身活动。在院子中打上两套拳,出了身薄汗后,汗湿的衣衫透出的健壮身材,完全看不到一点半年前重病垂死的病态。

练下拳法,是早上的热身运动。俯卧撑,仰卧起坐等后世最普遍的健身项目,才是主菜。说起来,韩冈学不来赵隆的天生神力,能把石锁玩得跟手上转的麻皮核桃。若是自家玩石锁,中间的那根木杆不够结实,不小心断了,或是干脆是自己失了手,伤筋断骨的毛病不是那么好治的,也少不了要留下后遗症。所以韩冈只敢选一些安全性比较高的运动来做。

韩冈的这几个锻炼的动作算得上是有些新意,王厚、李信、王舜臣他们都看过,不过也没人学着练,各人都有各人的锻炼方法,多半是军中流传多年的一些操演技巧。虽然韩冈有时也想过把自己的这一套传入军中,日后要整人的时候,让他去做一千个俯卧撑也蛮有趣的,可他没资格插手军务,不可能有机会把这些锻炼的招式在军营里传递。至于他所能管理的病号,多是需要调养,真的能开始活动筋骨了,第二天就会被拉回去。

“一二三四,五六七八……”

韩云娘甜糯的嗓音帮韩冈轻声数着数。

小丫头就站在庭院中的一株梅树旁。比起冬天韩冈离开的时候,她又长高了一点,但人却清减了许多。就像一株梅花,虽然清丽不减,大大的眼睛更为幽深,但还是显得过于苗条了。韩云娘小小年纪就受尽了相思之苦,见到韩冈后,白天人多还能忍住,到了夜里,是哭着让韩冈哄了半夜才睡着。

而且自韩冈回来后,她就变得更加粘人了,每天送着韩冈出门,虽然什么都不说,但眼神都是可怜兮兮的。韩冈知道这是小丫头心中缺乏安全感的表现,而现在自己所能做的也只能是尽力安慰。

一天俯卧撑和仰卧起坐各两百个,习惯下来也不算累了。也不需要多少时间,就完成了今天的份量。韩云娘忙服侍着韩冈去换洗。虽然这时候已经不像冬天的时候,锻炼过后就立刻要去洗浴更衣,不然就会感冒。但一身汗臭的去衙门里,也不会招人待见。

等韩冈换好衣服重新出来,二老已经起来了。韩冈赶忙过去请安问好。虽然前些时候儿子不在身边,但过了几个月的舒心日子,韩千六和韩阿李两人的气色好了不少,也富态了些去,身上的穿着打扮同样有了点富贵气象,看上去就是一个普通的家庭逐渐走向上层的模样。

看着韩冈头发上还带着点水意,韩阿李脸上不高兴,“又在熬炼筋骨了?照娘说的,三哥儿你还是早点成亲,我和你爹也好了笔心事,也省得你天天跟个军汉似的,没个官人样。”

韩冈为着自己叫屈:“娘这话怎么说的,两件事不是一桩吧?”

“你若不是有力气没处使,干嘛天天坐起来躺下去的,又趴在地上撑着?”韩阿李理直气壮,“还是早点娶了妻,等明年云娘满十四了,你再纳了她。日后多生几个,也可以帮你的两个哥哥留点香火下来。”

不知道这段时间以来,上门提亲的又来了多少,让韩阿李这般催促。不过范仲淹到了三十六岁才娶亲,世间士子成婚的平均年龄也比普通人明显要迟上一些。韩冈倒不是很着急,笑着推脱道:“还是先找些人来服侍爹娘,现在家里这间屋子也不算小,就是空空荡荡的不像样子。”

如今韩家入住的这套两进宅院,是韩冈回来后刚刚买下来的,位于秦州城内以官宦商人为多的厚泽坊中。今天才是乔迁后的第六天,为庆贺乔迁之喜所燃放的鞭炮碎屑,还没有打扫干净,在院墙外角落处还能看到不少。

与周围的房子比起来,韩家新宅的庭院房舍算是比较新了。只有七八年的历史,庭院中的两株梅树才一人多高,青苔也是才薄薄一层。但整体建筑修造得十分精致精致,从进正堂的台阶处都雕刻着的富贵连枝花纹,扣之如玉磬声的青黑色瓦片和折枝莲瓦当,以及涂了不知多少层大漆的房梁屋椽和柱子,可以看得出这宅子是花了大本钱去打造的。

而实际上这间韩家新买的宅院,也的确是名匠手笔。原本就是陈举为自己建的外宅——那位被剐成碎肉的陈押司,除了在家中多蓄姬妾,在外面也养了几个——而在陈举的家产给一众官员私分了之后,这宅院就成了留给韩冈的酬劳。虽然韩冈实际上也付了钱,但价格却是标准的‘内部价’。

同样的价格虽说能在城中的几个偏僻角落买下同样大小的宅子,但想在州衙附近买到第二处修建得如此出色的宅院,价钱再翻个三五倍都不可能。

有了房子,韩冈自然要把父母接到了城中住下。下龙湾村的老宅放着不动,也没人敢占他的便宜。现在再要做的,就是找些仆婢来服侍家人。虽然韩冈已经有资格动用杂使的厢军来为自家看守门户,但他觉得还是先找些老实勤快的下人来比较好。

正如韩冈所言,新家里人气实在不足。当一家四口在一起吃饭的时候,空空的内厅就显得太大了一点。原本寄住在韩家的李信,因为职位的调动而离开了秦州城;韩冈二姨家的两个表弟,则是来了又走了。

就在二月中的时候,李信在经略司的一次比试中,被来秦州述职的张守约看中,跟王韶讨了个人情,调去了甘谷城任步军副指使。有张守约罩着,李信日后的前途是不用愁了,就是现在,他的官职已经超过了韩冈的外公和舅舅一辈子的辛苦。

而韩冈的两个表弟,是在韩冈刚刚入京的时候就到了秦州。虽然韩冈从没指望他们能跟李信媲美,但他想着,既然都是一个外公,总有同样优秀的基因传下来。岂料,在传承中,变异也占了很大的比例。这两人,实在不成样子,太不是东西。

他们到了秦州后,就住在韩冈家里。却整日游手好闲,挑吃捡穿。李信帮他们找了两个巡城的活计,想让他们先历练一下。但他们却不肯干,说要等着韩家三表哥回来安排个好差事。李信当时就冷了脸,偏偏两人还没有自觉,照旧好吃懒做,其中的老大甚至还想籍酒调戏韩云娘,被忍到极限的李信狠揍了一顿,然后又给韩阿李让李信将他押了回去。

这不是韩阿李不顾姐妹的情分,但自家的侄儿做事连个分寸都没有,还指望他能帮上什么忙?日后肯定会拖累自家儿子。韩阿李读书不多,但见识不少,又有决断,便丝毫不留情面。

而小一点的,在他大哥被赶走后老实了不少。他也曾说过,想要回凤翔,却给韩阿李瞪了一眼,吓得不敢再说话。等到李信再去甘谷城时,韩阿李便让李信一起把他带了去,说是要好好锤打一番,省得日后也做出不知分寸的混事来。

“真想不到二姐的两个畜生都是这般德性,也不知怎么养出来的。跟信哥儿真是一个天,一个地。早知道他们不成器,就让他们呆在凤翔府老家,省得来了尽给人淘气!”

一想起来两个没家教的混蛋小子,韩阿李就是一肚子的火,就算凤翔那边已经托人赔了不是,她吃着饭时也不忘开口骂。而韩云娘站在韩阿李身后,也是鼓起腮帮子,很生气的模样。她那一日,也真是被吓到了,幸好李信就在旁边,直接了当把借酒装疯的色狼一脚踹开。

“那四姨家的表弟呢?他怎么样了?”韩冈问的是嫁进冯家做续弦的姨娘的儿子,他回来后都忘了这一茬,现在才想起来。他的那位冯表弟生长在富贵人家,也不知是不是养出了一身纨绔脾气。

听着儿子问起冯家,韩阿李也有了些疑惑:“说来这事也怪,已经让人捎了三次信去了,怎么都没个回音?来与不来,总得回复一句,报个平安也是好的。”

“他们真的把信送到了?”韩冈猜测着没消息的原因。如今世上可没有邮局,驿传系统更不是给跑平民用的,要寄信,都是托亲友或是同乡来送。这其中,有受人之托,忠人之事的正人君子,也有一转脸就把信丢到河里去的。

注1:诗赋重韵,在写诗时,一般都要翻查韵书,以防用错韵脚。而在科举时,也是要分发韵书,以防考生出错。

