\section{第18章 诸士孰为佳(下)}

只是该给韩冈什么名次比较好?

状元是不可能的!

依故事,有官身者不得为状元。

赵顼侧头看了一下王安石。他的这位宰相当年所在的庆历二年壬午科这一榜上,一开始被排在第四位的杨寘,之所以能当状元。第一,是因为头名的王安石犯了仁宗的忌讳;第二,就是排在第二位和第三位的王珪和韩绛,当时都有官身。所以杨寘一路上攀,占了状元郎的位置。

榜眼……

赵顼犹豫了半天,最后还是摇了摇头。韩冈在诗赋上的欠缺,一直关注他的赵顼哪能不清楚。若是当真把他提到第二第三名的位置上,必然成为众矢之的。那时候,不但有失赵顼奖誉韩冈的初衷,让韩冈蒙受不必要的攻击和嘲讽;更重要的,也会让世人小觑了天子提拔人的眼光。

集英殿中,静如子夜,贡生们无人敢于窃窃私语,而考官们更在耐心的等着天子的评判。思量再三,赵顼终于提起朱笔,在韩冈的卷子上写了下去。先是一横,然后是一竖。

十。

第十名。

赵顼给了韩冈这个名次,不会惹得太多嫉妒,也足以体现了他对韩冈的重视。原本被排在第五等的卷子,现在被提到了第二等中来,想必韩冈本人也不至于会得寸进尺,心生怨怼。

而且发榜之后,一甲中的二十人的卷子都会被公开,示以评判的公正。以韩冈卷子的水平,给他一个言之有物的评价,放在第十名上,世人也无话可说,绝对当得起。

可他看了看叶涛的文章,又对比了一下韩冈的文章,再一次犹豫了起来。

一个文字好,一个内容佳,但都是因为瘸了一条腿,所以比不上前面的八人。不过在各自的强项上却皆是出类拔萃,第九第十也绝对当得起。就是两篇文章之间,孰优孰劣,则让人还要计较一番。

前前后后比较了一遍,叶涛的文章毕竟只是文字好而已,而韩冈更加切题。更何况选的是能治国理民的进士,又不是在挑选词臣。最后一刻,赵顼坚定想法,提起朱笔,勾去了叶涛的九,改成了十。又勾去韩冈的十,改成了九。

最后一次看了看交换了名次的两张卷子,韩冈并没有问题,就是对叶涛未免就有点亏欠了。

‘记着他好了。’赵顼想着,过去亏欠韩冈的更多。

将韩冈、叶涛两人提了上来,赵顼就没有心思再看其他人的卷子。已经入夜了,下面还要唱名,耽搁到三更天也不好。

不用再去征求考官们的意见,也不觉得有必要现在让人再重新誊抄一遍,赵顼直接将修改后的名单让李舜举递给下面的王安石。

为一甲中人唱名的工作,依例要由宰相来完成。

头三名,为第一等。

第四名到第二十名,为第二等。

以上二等同属一甲,为进士及第。

第三等为二甲,进士出身。

至于第四、第五等,则是三甲,同进士出身。

王安石接过名单,只一瞥,就看到了被朱笔修改的地方,手不由自主的就是一颤。

第九,韩冈;第十,叶涛。

韩冈和叶涛,一个是他未来的女婿,一个是给他未来的侄女婿。

这个名次一旦公布,可就要掀起轩然大波来了。

对于这两人,王安石自问了解得很清楚。

一个是军政两面皆有长才,性命道理也有自己的一番认识,却是不擅文辞,与诗赋无缘;另一个则是文多质少,诗词文章可算得上是出色,可对朝政尚未有太多的了解。

优点显而易见,可缺点则更为明显,他们两人怎么能排到这么高的名次上去?

王安石皱着眉头,狐疑的抬头望向赵顼。

赵顼知道王安石为什么犹豫,但他并不在意。

这样的修改并不算什么。既然是殿试,最高的评审官就是天子一人。别说第九、第十,就是状元、榜眼,也是赵顼他说什么就是什么。他说谁是状元,谁就是状元!

上一科,也就是熙宁三年的殿试,状元叶祖洽便是由赵顼钦定。叶祖洽的卷子初考在第三等,覆考在第五等,但到了赵顼面前,直接让宰相陈升之当庭宣读,就这么给提拔成了状元郎。

‘祖宗多因循苟简之政,陛下即革而新之’,叶祖洽在卷子上写下的这一句,在考试官、副考官眼里,根本是让人恶心的阿谀奉承,可赵顼就是喜欢。皇帝要让人知道他对新法的支持有多深,便刻意将说的好话最为中听的叶祖洽提拔了上来。

天子是这样的性子,王安石很明白,韩冈、叶涛没有被提到前三名已经是赵顼慎重考虑过了的结果。

但他还是不得不说话,上前半步,“陛下……”

赵顼抬手拦住了王安石的谏言,“为国抡才,与他事无关。又是朕自己挑选的,相公就不必多说了!”

天子拒绝得干脆无比,不仅让王安石明明白白的听清楚了其中不容违抗的味道,也传到了屏声静气的等着王安石唱名的每一个人耳朵里。

是名单上出了什么事?每一个人都在猜测着。不知道为什么,远远见着王安石犹豫的转身回头,韩冈突然有了些不好的预感。

王安石已经开始唱名。

等了不知多久,终于等到了名次公布的时候,叶涛精神一振。回想起自己的文章,那是做得花团锦簇,状元难说,但在第一等列名当不在话下。

只是第一名状元,从王安石嘴里报出了余中的名字。

看着惊喜难耐的宜兴贡生,上前叩拜谢恩,叶涛安慰着自己,

‘还有第二、第三名。’

第二名、朱服;

第三名、邵刚。

王安石先后念出了成为榜眼的两人的姓名籍贯,叶涛的眼神已经变得失落不已。

而韩冈却是在想着榜眼这个名次与后世的差别。

后世科举,榜眼是第二名,但如今的榜眼,却是第二、第三名。

不得不说,第二第三名为榜眼,才是合乎情理的说法。

天榜之中,状元郎高居正中最上,是为魁首。其下二三名,左右并列,就像是位于两只眼睛的位置上,所以叫做榜眼——正常人怎么可能只长一只眼睛?

而后世作为第三名的探花,此时却是跟名次无关。探花郎的渊源来自于唐时。进士高中后,在曲江宴上,一榜进士中最为年轻的一人便会受命去园中摘花,回来后,分给所有进士插上,所以名为探花。理论上,状元都有可能成为探花郎。

韩冈很是闲适的神飞九霄,他有足够的自知之明,一张卷子就算有着王韶的修改,也不会有太高的名次——王韶此前曾说过,当初嘉佑二年科举,韩冈的两个老师排名都靠后。但王韶本人,他当年中进士的的时候,排位也是一样不高!

‘当是到后面才会叫到自己。’

一个名字,一个名字,由王安石念了出来。被叫到姓名的贡生,便上来叩谢皇恩。

念完鼎甲三人的姓名后。王安石稍稍停了一下,

再往下,就是一个个让贡生们听着都有些耳熟的名字报了出来。基本上,能考进前十名的进士,他的文名多半早就已经在东京城中传开,韩冈也是听过他们的名讳。

第八名,留光宇,一个三十上下,胖乎乎的仿佛商人的士子,上前拜见天子。

第九名,韩冈。

韩冈一愣,是重名吗?但籍贯随之而出,那就不可能有问题了。

稳步上前,在殿中的数百道羡慕、嫉妒还有惊讶的目光中,韩冈走到大殿中央。

赵顼很满意的看着这名给他带来太多惊喜的新科进士。

而下面的叶涛,则是用着难以置信的眼神,在望着韩冈于殿中央叩拜行礼。

连一首诗都做不好的人,他怎么可能能超过自己?

直到韩冈返回远处,叶涛这两个字被王安石念到,叶涛他本人都没有恢复正常。只是当王安石提声又叫着他时,才恍然大悟连走几步,到了殿中拜倒。

从大殿中央谢恩回来,叶涛的惊喜之情已变得很淡。不仅仅是因为自己排在第十,而韩冈排在的第九。更是因为他们这两个王安石的晚辈,同时跻身前十,在外界的士子中已经成了众矢之的。

现在韩冈也方才明白,为什么前面王安石要回头问着天子,就是因为这个名次上的问题。

回忆天子方才的两句话,韩冈终于知道是谁将他提到了第九位。可他没有半点欣喜,他本也不需要多高的名次,只求一个出身。现在糊里糊涂的被提到了第九位,反而麻烦就要多起来了。

罢了!

韩冈一扫周围投过来的眼神,变得冷静下来。

也不是什么大事,不过是些闲言碎语而已,根本就没必要放在心上。既然天子要卖人情,自己就承他的情好了……

不过如此!

报完一甲的十人,王安石将名单交回给李舜举。接下来的二甲、三甲,就不能劳动宰相的大驾,改由同时监考的翰林学士杨绘继续念着下面的名单。

四百零八名进士,自酉时开始,一个接一个出来叩首谢恩,一直拖到了戌时之后。

等到冗长的进士唱名仪式结束,新科进士们都谢恩离开宫廷,有着他们姓名的金榜也挂了出去。回到寝殿,赵顼提起了笔,在御桌旁的素色屏风上写下了四个字:

文章叶涛。

这个文章做得很好的进士,赵顼打算将他记住。至于韩冈,已经不需要屏风来提醒,这个名字自三年前起,就一直简在帝心。

