\section{第29章 顿尘回首望天阙(六)}

【第一更,求红票,收藏。】

‘这章子厚到底有什么盘算?’

一边喝着周南奉上的酒,韩冈一边揣测着章惇的用心。

蔡确却好像并没有发现章惇的正在刻意导引话题,顺着章惇的话道:“说起薛师正,其理财之能的确是难得一见。每年的六百万石粮纲,若不是换作他来主持,还是照样要损耗两成在路上……当初曾听薛师正说起过,押运纲船的军汉许多都会私底下把船上的新粮新绢,跟沿途的奸商偷换成浸过水的损坏品,然后就报称路上遇风雨毁损,籍此牟利……”

蔡确话声稍稍一顿,章惇就立刻附和上去:“我也听说过此事。以次换好还算是小心的,更大胆的直接报了倾覆的都有。那些奸猾小人上下打通了关系,就算追赔都赔不到他们身上!”

“现在薛师正做了六路发运使,把民船和纲船集合后一起发来。路上是否有风雨,参看民船便知。有民船上的货物做对照,那些奸猾之徒可就再玩不了什么滑头。有他主持均输法,这‘徙贵就贱,用近易远’八个字,当是不难做到。”

薛向对蔡确有知遇之恩,蔡确说话时自然都向着薛向。不过如今均输法的顺利推行的确都是靠着薛向的功劳。

在均输法之前,漕运实行的是转般法。也就是将东南六路——江南东路,江南西路、荆湖南路,荆湖北路、淮南路、两浙路——上供朝廷的物资,先在真州、扬州、楚州、泗州设转般仓储,然后再由纲船通过运河分批运往京师。

从运输效率上说,转般法的确不差,但纲船侵盗现象严重,因此而飘没的物资,最后有很大一部分要通过提前加征而得到补偿,地方上当然会有所怨言。加上转般法年年征收的入京物资数量几乎固定,丰收时六百万石,灾荒时还是六百万石,对地方州县来说,荒年时就是个很大的负担,所以才有了更能适应现状、视州县丰歉与否,而改变征购数量的均输法。

章惇和蔡确都是那种能看清现实、而不宥于义利之辩的官员,也很清楚均输法的意义所在。

“江湖有米则可籴于真州【今仪征】,两浙有米则可籴于扬州,淮上有米则可籴于泗州,不但无岁额不足之忧,亦可以此而宽民力。”蔡确说的,就是均输法的本意。

“东南纲运不绝,则京师安定。京师安定,则天下太平。”章惇说着,“江南、荆湖、浙、淮这六路,实是关系到天下的命脉。若是其中有哪一路有贼子作乱,即便是只占了纲运两成的荆湖之地,天下也就安稳不了了。”

“可是大宋开国以来,西北乱过,河北乱过,蜀中也乱过,但东南诸路可从没乱过。”路明难得的反驳章惇,韩冈却觉得有哪里不对。

“不,荆湖两路可从来都没少夷人作乱!”章惇丢出一句后,便开始喝酒吃菜。

韩冈眨了眨眼,隐隐的抓到了一点头绪,章惇好像说的并不是均输法和纲运的问题,而是意在荆湖,路明的插话也是证明了这一点。他开口,缓缓说道:

“荆湖虽多有蛮夷作乱,可地理绝佳,上接蜀地,下通江南。水土皆是上上。虽然水患频频,但如果治理得宜,一二十年后当是又一座粮仓。东南六路每年六百万石的纲运,其中八成以上,是来自两江、两浙和淮南四路。以东京的仓囤粮储,只要连续两年这四路中有两路同时灾荒,京中也便要慌了……如果说是如今开拓河湟是为了免除外患,那么开发荆湖却能缓解来日内忧。”

韩冈指点江山,章惇、蔡确和路明都放下杯盏,停筷下来静听。

韩冈对关西的确了若指掌,但说起荆湖两路却只有后世的一点印象,对东京仓储则更是半点不知。他这一番话本就是信口开河,仅仅是试探而已。

不过章惇明显的上了钩,立刻顺着杆子爬了上来,“只可惜荆蛮众多,不顺朝廷,时常下山骚扰,让汉民不得安宁!如何能安心屯垦。”

荆蛮的反抗当然多,历朝历代,都没少派兵去镇压过。要不然后世的荆湖地区,尤其是湖南,也不会有那么多带着征服意味的地名——保靖、永顺、靖州、宁远,这些名字中,从里到外都写满了中原王朝对南方少数民族的征服与统治。

“荆蛮虽多,不过是乌合之众,以天兵相临,必然俯首帖耳,手到擒来。”

路明此话一出,韩冈就撇了撇嘴,连带着蔡确也露出了一个看透了一切的笑容——路明的这句话,还是说多了。

学着韩绛、王韶的样儿,领军进剿荆湖两路不肯归顺朝廷的蛮夷,从中博取军功,以期飞黄腾达,这就是章惇的打算。

但他在席上说这些做什么?

并非是韩冈自大,从方才所了解的蔡确的经历上看,其对兵事并不精深。章惇的话只会是说给他韩冈听的。

‘这是要借助我的力量吗?’

韩冈微微一笑,终于全都明白了。

比起北方如蝗灾一般恐怖的游牧民来,南方的少数民族其实要容易对付得多。当年侬智高叛乱,南方诸路束手无策,而当狄青带着西军精锐赶到昆仑关,旬日之间,便大败侬智高。可真正让前去进剿的官军头疼的,是当地的气候条件。狄青带去的西军,回来的连七成都不到,其中战殁的尚不及病死的半数。

如今军中精锐依然皆是北人,南方的军队只有吃空饷的本事是在北军之上。章惇想要在荆湖两路立下功劳,还是得从北方调兵,因而也就必须克服水土不服对军队战力造成的影响。

而如今军中医疗的权威,则正是韩冈!

这是交换吗?

当然!

怪不得章惇会把蔡确请来,蔡确的管干右厢公事也能管到教坊司歌妓脱籍之事。教坊司的歌妓要赎身脱籍,不仅仅是缴纳赎身金的问题。对于官妓来说,她们脱籍必须要由所在州府主官的批准。只有拿到准许脱离乐籍的文书,官妓方可解脱贱役的身份。也就是说,周南想要脱离教坊司,就必须得到开封府的批准。

不过如今知开封府的韩维不可能管这些闲事,他是天天能去崇政殿面见天子的重臣,国事都有份参与。基本上东京城中每日要处理的琐事,都是由通判、推官等一众属官处置,而以蔡确的身份,的确可以干预其中。

看见韩冈唇边的笑意,章惇心有灵犀的点头微笑。

与聪明人说话就是方便,不用多费口舌,就把心意都传递了过去。而且韩冈还给了他一个出兵荆湖的更好的理由——屯田荆湖,让国之重心不再偏重于江东。其实大宋立国以来,荆湖两路一直都在开发中。两路的进士数量一直都在上涨,由此可以看得出,两路的民生都是处在稳步的发展之中——开发荆湖,难度虽有,却绝不会比河湟更高。

看似毫无瓜葛的闲谈,韩冈和章惇已经默契的达成了协议。

韩冈虽然对如何把周南拉出火坑自有想法,但章惇愿意帮忙出招,韩冈也不会拒绝。好歹是一条路,也得走走看,说不定就走通了,即便不成,还有自己的手段做底。韩冈虽然常说‘我只怕事情闹不大’,却也不是什么时候都希望把事情闹大。

‘如此甚好!’

韩冈举杯,与章惇对饮而尽。互相一亮杯底,便同声哈哈大笑起来。

蔡确在旁冷眼看着,脸上也带着淡淡的笑意。章惇拿自己跟韩冈达成了协议,这一点,蔡确已经看透了。

昨日的朝会上,韩维受到了两个御史的弹劾,现在已经避位在家,不出意外,他的知开封府一职近日多半就要卸任。韩维即将出外,而韩绛远在关西。即便他能得胜回来,也会照惯例被投闲置散几年,然后方会重用。韩绛能得起三五年,但蔡确等不起,为了自己的前途,他急需为自己找个新的后台。

官场上的交情本质上就是互相利用,有利用的价值是件好事,即便是为了一个官妓脱籍,只要能交好章惇,还有韩冈这个听说是王安石面前的红人,蔡确也绝无怨言。至少在眼下,没有一个靠山比得上王安石更为牢靠!

烛泪已尽,残沥犹存。一番酒一直喝到深夜。路明领着人送周南回去,蔡确也告辞离开。一同走在依然车水马龙的夜市中,章惇说起了周南的脱籍之事。

“关于周小娘子脱籍的事,有蔡持正在,当是不会有多少阻碍,这几日静等佳音便是。千万不要送到推官厅去,”章惇对韩冈提醒道,“苏子瞻那性子,若是年老色薄,多半就放手了,但换作周小娘子,他是肯定不会放人的。”

做着开封府推官的苏轼,日后名传千古的东坡先生,想不他的朋友这么不看好他的人品。

韩冈无意去怀疑章惇的判断,毕竟他跟苏轼不熟,他点点头,“多谢检正,韩冈记下了。”

