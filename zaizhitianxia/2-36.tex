\section{第10章 弹铗鸣鞘破中宸(中)}

【第三更,求红票,收藏。不好意思,字数较多,迟了一点】

天色已然深沉。

夏日的晚霞,绚丽灿烂,留下了多少诗篇让人传唱。但在今天,占据了半幅天空的彤云,却是让古渭寨中的每个人都感到厌恶。

一个时辰前,就在映天红霞的衬托下,自古渭寨西南方的山峦之后,突然腾起了数道浓烟。深黑的烟气随风卷动,直入天际,散入血红色的霞光之中。就算现在已然入夜,无法再看到烽烟,但艳红的火光仍反照着天空,仿佛晚霞已经自西面转移到了南方。

那是吹莽城的方向。

只看着血色光芒笼罩下的山峦,便可想见在群峰的另一侧,究竟发生了什么样的惨剧。古渭寨中,不知有多少人不时的望着南方山后升起的红光,他们都在担心着今次入侵青渭的蕃贼会不会杀得兴起,趁着寨中人少,转头来攻古渭。

而从吹莽城腾起烟火的那一刻,张香儿已经惨叫一声,吐血昏死过去,被王韶唤了人抬下城头去照料。没有人嘲笑他,若是换作自己全家遭难,肯定都是一般儿会昏倒。

入夜后,董裕军就开始在城外点起篝火。古渭寨外也有村落,村中都是来此屯垦的宋人,仰仗着古渭寨的保护,垦荒开辟。虽然村中百姓此时都已逃入寨内,但加起来以万斤计的柴草秸秆还是堆在村内。

董裕蕃军便拿着这些柴草在城外摆了无数堆篝火,绕着城寨整整一圈。几座城门外都是一团团火焰在闪动。尽管只是虚张声势,但看上去却是铺天盖地,比天上的星星还多。黑夜中,只有篝火在闪,也不知有多少蕃人围在城外,让城中守军因此望而生畏。

高遵裕已经没心思去理会董裕在城外做得把戏,天黑后他就下了城去,回到城衙去休息。反正董裕不敢攻打古渭,而古渭寨中的守军又无力出城作战,站在城头上与城下的蕃贼大眼瞪小眼,只会让自己生上一肚子的闷气,新来的蕃部提举可没这等好兴致。

对于如今青渭的局势,高遵裕心中憋屈的要命,但他却找不到出气口。要怪也只能怪攻打甘谷的西贼,若不是他们引走了刘昌祚,古渭寨也不会任由前些日子的丧家犬欺负——其实若是刘昌祚不从古渭带走那两千兵,以董裕的胆子,也绝然不敢来犯。

虽然一开始高遵裕骂了韩冈几句,但他也明白,俞龙珂这等老狐狸,都是不见兔子不撒鹰的主。韩冈什么权力都没有,空口白牙,能把他请出来抄董裕后路,已经是难能可贵了。想让俞龙珂跟董裕硬桥硬马的对碰上,俞龙珂可没长个猪脑袋,不大出血怎么可能请得动他?

高遵裕只可惜自己昨夜没去,他能许诺给俞龙珂的条件,可比只能向俞龙珂摊手苦笑的韩冈要强得多。

而王韶还站在城头上。虽然知道城外的蕃贼绝然不敢攻打古渭寨。但谁也不能拍着胸口打包票,说不会有个万一。古渭寨内地位最高的两人中,总得留一个在城上看着。

王韶看着城外的多如天上繁星的篝火,心中隐隐作怒。董裕这是明欺着城中守军不敢出战,才嘲讽一般的做出这么大的一番声势。

可恨王韶对古渭寨的守军没有名正言顺的指挥之权。就算强行命令他们出寨攻击,也不知道领着这一千人的几个将佐中,有哪一个可堪一用。

若是赵隆、李信、王舜臣他们还在身边就好了,王韶不禁这般想着。现在就可以命他们带兵出去冲一下。可如今李信跟了张守约,赵隆跟着王厚,都一起去了京城。王舜臣则保护韩冈去找俞龙珂。王韶现在身边的几个亲信就剩个杨英,而杨英仅仅是他自乡里带来的听候使唤的,会做事,会做人,却不像王舜臣、赵隆那般武艺高强。

今天这一天,唯一值得庆幸的,是纳芝临占部至少还是在董裕的刀下逃出了一批人。他们皆是早早的得到消息,做好了准备。一见看到谷口的通道被封堵,便立刻四散而逃。

虽然最好走的一条道被封锁,但其他道路也照样存在,他们这些土生土长的地头蛇远比董裕麾下的军队要熟悉周围地势,很快便逃出生天。

现在他们中的大部分还躲在山岭间,带领着他们这群幸存者的几个首酋,派了两名得力人手把消息传到了寨中。相信张香儿醒来后,至少还能感到安慰一点。

一群蕃人坐得离城门只有百步不到,围着一丛篝火喝酒欢唱,喝多了还冲着古渭寨撒尿取笑。望着狂妄的他们,王韶现在也只能咬紧牙关,苦苦忍耐。他现在就希望着俞龙珂当真能断了董裕的后路,把他打得全军覆没,好出一口憋在心中的这番鸟气。

……………………

韩冈此时已经休息了下来。

今天在山间的小道上了走了大约五六十里路后,俞龙珂和他的八百甲骑就在一处隐蔽的山坳中扎下营盘。山坳近着一条小溪,夏日雨水丰足,队伍也无缺水之虞。

青唐部的骑兵们抓紧一切时间休息,他们中的大部分都上过阵,有过杀人经验的,韩冈估计着应该不在少数。

俞龙珂的大帐在扎营的时候就被竖了起来,韩冈和王舜臣被邀请进帐休息。作为贵宾,韩冈与俞龙珂吃了顿双方都食不知味的晚餐。韩冈看着青唐部族长的模样,应该是有隐忧藏在心中。

刚刚吃完由烤羊肉为主调的晚餐,韩冈正喝着茶水,消解饭食中的油腻。这时一名满面风尘的蕃人大步走进帐来,应是个在外打探军情的斥候。俞龙珂一见他便是霍然站起,向韩冈告了罪,性急的与他说了起来——只是他们说得是吐蕃话。

韩冈不懂吐蕃话,跟俞龙珂他们交流顺利,也是因为蕃部上层没有人不通汉语。因而尽管向俞龙珂禀报军情的斥候正在说的话,不断的随风传来,但韩冈却是半句都听不懂。只不过类似于‘瞎药’这个发音的只言片语,一下触动了韩冈的神经。

难道是俞龙珂派出去监视他弟弟行踪的斥候?韩冈心中揣测着。

瞎药与俞龙珂早就分了家,带了一部分青唐部的部众在与青唐城隔着几条山的地方过活。今次董裕敢深入青渭,韩冈估计着是因为董裕和瞎药有勾连的缘故,相信俞龙珂也会想到这一点,派人去监视他的弟弟也是情理中事。

不过今天俞龙珂说出兵就出兵,半点没有被阻碍的样子,瞎药没尽力帮董裕也是显而易见的。

两人说了好一阵,斥候躬身出了帐。俞龙珂回过头来坐下,脸上带着喜色,韩冈便出声问他:“看着俞族长满面春风,不知是什么好消息?”

俞龙珂怔了一下,张口结舌的啊了几声,方才回答道:“啊……啊……是、是留在家的孩儿们把董裕给吓住了,让他没敢攻打古渭寨。”

扯淡!韩冈暗骂着。

这么明显的谎言怎么骗得过韩冈。方才来的斥候可不是从古渭寨和青唐城所在的东面方向上来的,而是自西面过来——据韩冈所知,瞎药现如今的领地,可就在那个方向上。

俞龙珂大概是想隐藏自家内部兄弟阋墙的纷争,所以不肯说实话。不过韩冈见他方才笑得挺开心的样子,大概是瞎药那里没有什么动静。

韩冈并不追问,却与重新坐下来的俞龙珂说着闲话,心中却在疑惑着:难道瞎药真的有这般不济?

韩冈总觉得不对劲,半年前在古渭过年时,他遇见代表青唐部来拜年的瞎药。那一对桀骜不驯的眼睛可是给韩冈留下了很深的印象,里面作为燃料而燃烧着的完全是野心。而俞龙珂对瞎药的忌惮和监视,也证明了韩冈没有看错。

而如今这么好的机会,像瞎药这样有野心的人怎么可能会放过去?

韩冈抱着疑惑,与俞龙珂说了阵无聊的废话,继而又辗转反侧的睡去。

一夜过去。

青唐部的骑兵收拾了昨夜宿营的场地,翻身上马,继续上路。已经与古渭隔了有六七十里,现在向东面望去,根本看不见董裕的军队。不过计算过董裕行军的速度,他们现在不是在古渭寨外,就是已经占据了纳芝临占部的吹莽城。

这一天依然是在山间行军中度过。韩冈在随行的过程中考虑着董裕的行程,如果他够聪明,现在就该回师。

而在午后时分,俞龙珂派出去的斥候就证明了韩冈的猜测——董裕今日已经率军回返。

俞龙珂和他的部族没有不熟悉青渭一带地理的,计算过路程和渭水边适宜扎营的位置,他们便在一处山谷中埋伏下来,谷中溪流正是渭水支流,谷口自然正对着渭水。

青唐部走得是山道,而董裕行的是山谷,行进速度自然要快过青唐部的队伍。到了傍晚时分,就看着一支七八百人的队伍,满载着战利品,得意洋洋的从谷外横过。

藏在隐蔽处看着他们,俞龙珂没有动,让手下的将士们继续埋伏,那是董裕的前军,说不定也是董裕防着埋伏而放出来的诱饵。

“等董裕的本队。”他对手下们说着。

这一等就是一个多时辰,韩冈觉得不对劲了,王舜臣也被草丛中的蚊虫咬得抓耳挠腮,俞龙珂的神色也急躁了起来。

一名哨探此时匆匆赶了回来,脸色惶急,急叫着:“出事了!董裕的中军在后面遭袭了!”

俞龙珂一听不妙,追问了两句,便连忙吹响了号角。埋伏的八百甲骑立刻呐喊着杀出谷中。分出一队去追前军,而剩下的五六百骑则跟着俞龙珂沿着渭水向东杀去。

但他们到得已经迟了,就离着他们埋伏的山谷不到五里的地方,就见着一彪人马正在董裕的队伍中横冲直撞。被山谷阻挡,俞龙珂和韩冈他们竟然没有听到这么大的喧嚣。

猝不及防的强盗被不知从何而来的敌人杀得溃不成军,如切菜砍瓜般被砍倒,一声声惨叫回荡在渭水边,而董裕的帅旗在人群中晃了几晃,就在韩冈和俞龙珂的眼前落在了地上。

不知多少人一齐喊起:“杀了董裕了!杀了董裕了!”

“他们是谁?”王舜臣放下手中的弓,疑惑的问着韩冈。只是他却见着韩冈嘴角微微翘起,一抹笑意一闪即逝。

“是瞎药!”韩冈缓缓地答道。

