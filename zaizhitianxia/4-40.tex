\section{第八章 欲谋旧地重兴兵(下)}

秉常百无聊赖,事关国政的对话他根本插不上嘴,也没人会问他的意见,只是他越听心头火气就越大。

十五岁的西夏国主有着少年人都有的自信,总想着要证明自己已经成人,而不用再听从长辈们的教训。满腔的雄心壮志,恨不得今天就掌控国政,然后指挥国中数十万大军,再来几个好水川之役,将咄咄逼人的宋军打得三十年不敢北望。

可眼下在他面前的母后、舅父,乃至一干重臣,面对咄咄逼人的宋人,近在眼前的战争,竟然都是畏畏缩缩,全然没有太祖【李继迁】、景宗【李元昊】当年横扫六合的豪气。

执掌西夏军国大政的太后、宰相、枢密使,以及一众重臣们聚在一起商议了半日,也没有讨论出个结果。唯一确定下来的,就是增加罗兀城中的守军,以及加强横山北麓银、夏二州的防备。同时还要盯着兰州,防着禹臧花麻突然叛离,让人措手不及。

只是这个决定,与前一日、再前一日,乃至一个多月下来的多次商议的结果,根本没有什么两样,做了等于没做。

脚步重重走过后宫的回廊,靴底踏着地板,咚咚的响着,如同战鼓,在诉说着秉常心中的愤怒。

天子一怒,伏尸百万。

换作是太祖、景宗,遇上宋人敢打横山的主意,不论真假与否,都会立刻跳上马,带着身边的御围六班和环卫铁骑向南冲去,同时吹起号角,召集国中健儿。等到了横山,就是十数万兵马如洪水破堤一般的冲进宋境,杀个血流成河!

在秉常自幼听闻的故事中,他先祖就是这般的英雄豪杰。

可惜自己连在朝堂上说话的份量都没有,秉常心情郁闷的往寝宫中来。西夏宫廷一切都是学着汉人的制度,主殿名为紫宸,连护卫都叫做班直,至于寝宫的名字,也同样是仿效而来,叫做福宁殿。

“臣妾拜见官家。”到了殿前,一名衣着华贵的少女,就领着宫人出殿相迎。

秉常喜欢汉人的物件,丝绸、瓷器,还有诗词文章,当然,也包括了称呼。他不喜欢满是胡风的‘兀卒’,而喜欢让宫人称呼他官家,就跟几千里外的开封皇城中,那一个个身穿绫罗绸缎的内侍宫女,称呼汉家天子时所用的称谓一样。

迎出殿来的少女,只有十六七岁的样子,比起秉常要年长一点。双眼因为过于细长,看起来显得有点阴险,还算耐看的相貌,也因此而变得有几分慑人。

秉常上前搀扶起她,一起往殿中走去:“怎么到福宁殿来了?”

“听说南人有意攻打大夏,就过来问一问。”

虽然穿着汉家女子的罗裙,出来迎接秉常的王后耶律氏,但还是不脱契丹儿女的直爽。身为王后,耶律氏有着自己的寝宫,但她既然嫁过来了,并没有在自己的寝殿做个摆设的想法,而是着意亲近小了一岁的夫婿。

虽然耶律氏只不过是个宗女,没有魏王耶律乙辛的推荐,她与契丹皇室的关系也就仅仅沾上一点边而已,但她现在的身份却是实打实的帝女,得到大辽皇帝耶律洪基认可和册封的仁寿公主。

转进秉常日常起居的偏殿,耶律氏道:“你们都下去吧,我和官家有话说。”

只吩咐了一声,殿中的宫女内侍立刻都顺服的退了出去。

原本在福宁殿中服侍的宫人,都是梁氏所安排,逐日向梁氏通报秉常的日常起居。不过当耶律氏嫁过来之后,没几日就找了个借口杖杀了数名太过‘忠勤’的宫女,自此就再没有人敢于违背她的吩咐。就连梁太后本人,也不愿因细故而与她这位背景深厚的儿媳妇为敌,只能让自己安排的人手更加小心的行事。

……………………

隔着千里横山,宋夏两边都开始了整军备战的进程。

只是宋人这一边,还要提防着北方的契丹——西贼只是边患,北虏若是来了,大宋则有灭国之危。

所以赵顼每隔数日,就要问着韩冈板甲如何如何,尽管军器监原本就是逐日上报监中的生产情况,但他还是要问。这也是心急之故。

战争迫在眉睫,可战场虽然在西北,但胜负的关键却有很大一部分放在河北。为了加快河北禁军尽数换装铁甲的计划,赵顼已经不止一次催促韩冈,要尽速准备好至少四万幅铁甲,以便给河北北方的高阳关、定州两个经略安抚司路的禁军,补齐不足铁甲数额——在板甲出来之前,两路禁军的铁甲率也只有五成而已,但这个比例,已经是证明了朝廷有多看重河北防线。无论党项、契丹,有甲的最多四分之一而已,而铁甲更是不到一成。

不过韩冈不仅仅是甲胄方面的工作要操心。各种兵械、城防用具,都要由军器监来生产。最近因为在城外设窑炼焦,焦炭有了,而煤焦油也有不少。如今正在监中的作坊里看看能不能制成猛火油。当然,更为重要的还是炼铁,他现在正在打报告,要河北或是京东,运送一批铁矿石过来,而不是过去的生铁锭。

过去宋廷一直在鼓励天下军民发明堪用的军器,因此而得到的神臂弓,如今已经是宋军倚仗来压倒西北二虏的利器。而在韩冈做了判军器监之后,更加鼓励监中的发明创造。而且也不再急功近利的偏重军器,而是更注重对于工具的改进和发明。尽管如今监中工匠们的精力,尚都放在锻锤和水车、风车上,但韩冈相信,也许再过一段时间之后,就能看到机床的出现。

“听说罗兀城又多了一千五的铁鹞子。”

这段时间,王雱与韩冈来往得越来越多,隔三差五就来找他说话。王安石的长子一贯的心高气傲、目无余子,眼中从无那等庸碌之辈——他的名声,其实有三成是他恶劣人缘给败坏的。但对上才智、功业都不输于他的韩冈,王雱倒是有着惺惺相惜的感觉。只是今天关于横山的话题,有一半是代替王安石来咨询。

“加上之前的驻军,西贼放在罗兀城里的就足五千马步军了。”韩冈咂咂嘴,呵呵笑了起来:“亏他们养得起。”

“养不起也得养。”王雱冷哼一声,反问道,“难道还敢就此放弃不成?”

“说不定西贼会一路退到兴庆府。坚壁清野、诱敌深入,最后来个关门打……”韩冈抿了抿嘴,没把最后一个字说出来。

“看来玉昆你也不是什么都敢说嘛……”王雱顿时哈哈大笑,“可惜西贼绝不会这么大方。他们若是放弃银夏,官军正好占下来。至于兴灵,上上下下都没做好准备,粮秣不及,兵力一时难以调集,怎么敢过瀚海去。换作是准备克复兴灵、剿平西虏的灭国之战,那时倒是要担心西贼会这么做了。”

种谔当年一见罗兀城要弃守,就立刻在横山中大开杀戒,无定河一带的大小蕃部少说也给灭了几十家,再加上围攻罗兀城时,梁乙埋也同样为了获得了足够的粮草,而大肆压榨横山蕃部。

大战才不过过去了四年而已,横山蕃部的元气远远还没有到恢复的时候,罗兀城中的守军,口粮从哪里来?

种谔敢以鄜延一路为主力去强取几年前,可不仅仅是因为甲坚兵利,远胜以往。更是因为党项人在横山中得到的支援已经远远不如过去。

这几年,罗兀城中的西夏守军,本身就得依靠山北的银州、夏州来支持。如今为了抵御宋军,西夏这一个多月来在横山南北,少说又添了上万兵马。只要他们驻扎上半年,就足以将银夏地区这两年攒下了一点存粮吃空掉,调来的援军越多,吃空的就越快。到时候在横山南麓开战,党项人甚至得隔着瀚海从兴庆府运粮过来,粮食充裕的反而是北攻的宋人。

“只要防着西贼主动来攻,抢了粮食回去。”王雱补充道,“光是粮草不足,都能将西贼逼得退回横山北麓。”

“缘边四路无论哪一个城寨,现在肯定都是在小心戒备中。党项人的脾性,西军上下比谁都清楚。”韩冈笑了笑,“西贼肯定会主动出来的,他们已经习惯做强盗了。给他们当头棒喝,再乘势进攻。这样就算辽人来问,我们也是理直气壮。”

韩冈如此笑说着,口气却是带着讥讽。王雱听着摇摇头,打着西贼,却还要防着北虏,任谁都觉得气闷。

“玉昆你现在还反对攻取横山吗?”王雱问道。

韩冈笑而不答,他反对攻打横山的理由从来都不是前线上的问题,辽人何尝会讲道理?去年辽人来争代北之地,大宋岂是没理。只是再说下去,又是要为天子讳了。

“最近的都放在北方,南边也要小心一点。”

王雱奇怪的问道:“南边?南边能有什么事?这段时间,哪有动静?”

韩冈仰靠在椅背上,微皱起的双眉又舒展开。说得也是,已经几个月过去了,交趾那边也没有什么动静。多半还是自己听了苏缄的话后想得太多,现在的问题还是在西北。

“横山……”韩冈轻声念着。

这一次至少有七八成的把握,将其收复。也许攻灭西夏,也就在三四年中了。

