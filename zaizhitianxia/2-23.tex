\section{第八章 太平调声传烽烟(二)}

【第二更,求红票,收藏】

当王韶终于说服了高遵裕直接向天子请款,以加快开边河湟的实施进度,时间已是四月末。麦子早已抽穗,沉甸甸的直欲垂下去,叶面也逐渐泛黄,再过几日,到了端午,基本上就可以收割了。

来秦州体量荒田的都转运使沈起,也到了有数日,只是他现在也没有表现出要沿着渭水上溯,去点验宜垦荒田数量的态度,而是日复一日的赴宴会客,喝酒聊天。

又是一日的忙碌过后,王韶闲下来,随口问着韩冈:“沈转运今天又是赴哪家的宴席去了?”

“好象是窦舜卿和向宝一起请客。也没去细打听,是不是也不清楚。”

这位陕西都转运使来秦州后,倒是长袖善舞。李师中给他接风洗尘,他毫不推辞。窦舜卿设宴请他,他高高兴兴的赴宴。前日高遵裕和王韶一起在新开张的晚晴楼摆酒,他也照样去喝个痛快。

韩冈听说上次李若愚来秦州,可是一家酒宴都没有去,板着脸做足了阎罗包老的模样——自然,这只是明面上的事,暗地里他和王`克臣从李师中、窦舜卿那里拿了多少好处,外人就不可能知道了。

沈起这副作派,让人感到疑惑难解,不论他做出偏向哪一方的判断,对立的一方都可以拿着他频繁赴宴的举动,让他的证词失去说服力。

所以韩冈现在已经没兴趣去猜测沈起到底是站在哪一边。反正王厚一行端午前后应该就要到京城了。只要他们把沙盘献上去,无论沈起帮着哪一边都无所谓了。

就因为韩冈抱着这样的想法,所以第二天,当他听说都转运使终于不再赴宴,而是出了城往西北去做正事,也没有多在意。

但几天后,也就是端午节的前两天,当韩冈听到沈起这次出行检查荒田,最后抵达的地点时,却是大吃了一惊。

“沈兴宗到了甘谷城了。”

高遵裕进门后便劈头说道。自从前日向京城发了请款的文书,高遵裕每天都等着朝堂的回音,心里挺不耐烦。但他还是有做事,为了立功他也是极热心。天天到勾当公事的官厅来,让韩冈打开架阁,把库里翻了个底朝天,将里面有关蕃部的文档都翻了出来细看。

不过今天,韩冈是在王韶的官厅里碰到他,也正好听到了关于沈起的最新消息。

“到了甘谷城?”王韶站起来迎接高遵裕,有些疑惑的问着,“他去甘谷城作甚?该去古渭才是!”

“是不是哪里弄错了,”韩冈也怀疑着高遵裕这条消息的可靠性。“去古渭寨也是同一条路,在伏羌城看到他,并不一定是往甘谷去。”

自秦州往甘谷城和古渭寨去,前半程都是一样的,一直要到伏羌城,才一条往北,一条往西的分道扬镳。不能看到有人准备绕过陇城县往西去,或是进了伏羌城,就说他去甘谷。

“不会弄错,我直接从李师中那边听来的。”

高遵裕身份特殊,虽然他现在是站在王韶这边,但李师中和窦舜卿的官厅,他还是能照进不误。

“沈兴宗究竟是在想什么?”王韶的脑门上几乎就写着问号,他和韩冈这等喜欢步步算计的性格,最烦的就是不按理出牌的家伙:“他到甘谷检验个什么荒地,那里的四千顷田都是明明白白的,早就丈量过了!”

高遵裕摇着头:“谁知道他是怎么想的,不耽搁我们的正事就行。”

韩冈揉着太阳穴,也是有些头疼:“现在去甘谷可不是好时候。过了端午之后,麦子就该熟了。西贼去年的存粮支撑不起大规模的作战,所以前些日子在甘谷只是虚晃一招。即便是在庆州号称十万的打了一仗,可实际上最多不过出动了万余人,要不然李信、刘甫和种詠带的三千兵早就全军覆没了,他们也不会轮到李复圭来杀。但今次肯定完全不同,不会是风声大雨点小,为了抢收边地新粮,西贼可是真的要拼命——不论哪一年都是如此,今年也不会例外。”

如果把党项人的战略目标和战斗目的做个简单的归纳,那就是七个字——抢粮抢钱抢女人。至于更宏大更长远的规划,他们是没有的。李元昊倒是喊过打到长安,割据关中的口号,但跟宋军打过几仗后,虽然都是赢了,但西夏国力损耗更大,根本支撑不下去继续进攻。最后终其一生连陕北的山区都没能突破,距离长安更是有几百里。

在宋夏两国巨大的国力差距下,西夏不论取得多少战术上的胜利,也无法变成战略上的胜势,但他们还是不停的进攻。不仅仅是为了以攻代守,籍此自保,而是西夏本国贫瘠的出产根本满足不了党项贵族的难填欲壑,为了维持凝聚力,必须不停的抢掠。

现如今统治西夏的是梁氏兄妹——梁太后和他的兄弟梁乙埋,作为党项化的汉人,他们的根基并不深厚。为了维护梁家并不算稳固的统治地位,光靠对内高压并不管用,必须在对外战争中——也就是对宋国不断取得胜利,抢来足够多的战利品分给各大部族以收买人心。

高遵裕和王韶也一起沉默了下去。每年麦熟之后,便是西贼开始活动的时候,秦州上下,哪一个不知道,此事根本不出奇,缘边诸寨都会在这时候做好警备,只是今次,沈起却是在甘谷。

沉默中,王韶突的哈哈笑道:“前几日宴会上还唱着清平乐,若是今天……”

一阵急促的脚步打断了王韶的话。脚步声从前院沉沉的奔过来,绕过机宜文字所在的院落,一直往后院的安抚使官厅去了。王韶往韩冈使了个眼色,韩冈会意的出去,转眼他就急走回来,脸色也有了些变化,“甘谷告急!”

王韶又是猛的站了起来,脸色这回是当真变得苍白,一脸惊容:“真的打起来了?!”

韩冈摇着头:“我没来得及细打听。不过传信回来的是个急脚递的铺兵,看他的神色也不是小事。甘谷那边怕是西贼再进一点就要点烽火了。”

“沈兴宗会不会出事?”高遵裕立刻问着,前面他对沈起可能会遭遇到西贼的事也只是泛泛的想了一下,并没有当真。但他怎么也不会想到,党项人当真说来就来,一点也不耽搁。

“还理会他作甚?死活由他去,轮不到我们操心。”王韶猛的站起身,把他收藏在厅中的一份缘边四路的舆图找了出来,指着上面向高遵裕解释,“如果是平常时候,秦州这边肯定是偏师。有环庆的马岭水不走,却过来走甘谷道,西夏人不会自找麻烦。

但现在是麦熟之时,西贼的目的却是粮食。马岭水两岸的田地并不比甘谷大,打下的麦子也不可能比甘谷多。西贼两条路都不会放过,就算抢不到新粮,也会把麦田烧掉,让缘边寨堡今年就只能靠着后方把粮食运上去。这对他们入秋后的进攻好处多多。”

“刘昌祚已经在甘谷城了。子纯你不是赞过他多次吗?有他在,应该不用担心甘谷城吧?”高遵裕问着王韶。

“甘谷我才不担心。我担心的是古渭和渭源。对于西贼的习惯,蕃部那边也是了若指掌。前次木征为了硕托部吃了那么大的亏,今次肯定会趁着西贼调走了刘昌祚,古渭、渭源的兵力空虚,而起兵报复。”

他转过头来,对着韩冈道:“玉昆,我去找李经略报备,你现在去准备好,午后就跟我去古渭寨。”

高遵裕听了,当即叫道:“子纯,即是要去古渭寨,我也一起去。”

王韶抬头,看着高遵裕。前日王韶因为心里清楚党项人的攻击只是个做做样子,刘昌祚带去甘谷城的两千兵马随时可以来援,所以他才安心的把高遵裕带去古渭寨。但今次情况不同,无论西贼还是蕃贼,都是要玩真的了。若是高遵裕出了一点事,他这边可就麻烦了。

王韶犹豫再三,但见着高遵裕他是一脸坚持的模样,最终还是点了点头,“那就请公绰与我同行。”

所谓坐言起行,王韶也是往古渭走得多了,上午把琐事处理完毕,匆匆的与又准备去陇城县坐镇的李师中打了个招呼,午后就带着一众护卫,与高遵裕、韩冈一起出城,往古渭寨疾行而去。

队伍中高遵裕带来的随从各个紧张万分,脸色紧绷得如同家中一下死了一半人口。而道路上的气氛比他们半个月前走过时也要紧张得多。

虽然西贼意欲大肆入侵的消息还没传扬开,但秦州人毕竟是久历战阵,知道西贼什么时候的进攻只是骚扰,而什么时候的进攻却是要拼命。在秦州,这样绷得紧紧地气氛每年都要重复多次,

真不知道这种紧张什么时候是个了局,韩冈骑在马上,心中忍不住想着。

在他想来,其实要对付党项人很简单。就是让他们每次进攻得不偿失,对他们连续放血,一边高墙深垒的严防死守,一边偷空杀入西夏境内进行扫荡,一二十年后,西夏必然崩溃。但在政令一年三变的北宋,想维持这样的策略,却比聚齐大军直接攻入西夏境内还要不现实。

夏天天黑的晚,虽然王韶他们走得迟,但赶得路却不少。当天入夜时分,一行人就赶到了一百多里外的三阳寨。而在三阳寨寨中,他们却见到了一队熟悉的队伍:

“这不是沈转运的车马吗?”

