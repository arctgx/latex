\section{第15章 焰上云霄思逐寇(三)}

“是谁给你胆子违背军令的?!是谁让你冲击宋人军阵的?!让你带出去的四百骑兵,回来不足三百,你还敢来见我?!”

厚重的幕帘,遮挡不住从中军大帐中出来的咆哮,守着帐门的两个李常杰的亲兵对视一眼,又都低下了头去。

派去归仁铺的援军败得太惨了,点选的也算是军中的精锐了,在宋军的攻势下,竟然连一时半刻都没能支撑下来。而且还折损了太尉心头上最看重的骑兵,直接下令砍下几颗脑袋作为惩戒都是不足为奇的。

率领骑兵出战的将校跪伏在李常杰的面前,脸贴着地,额头上的冷汗簌簌直落,连自辩都不敢多说一句。

“好了,他也是正好撞上宋军,步军又败了,不得不出阵。”宗亶打着圆场,统领骑兵可是李常杰一向看重的亲信,过去又多有功绩,要说因损兵折将而处以军法倒也还不至于,“就让他将功补过好了。”

李常杰余怒未消,胼指指着跪伏在底下的亲信,“还以为是占城军吗?都是带着神臂弓的精兵,冲上去不是找死?!”

李常杰原本是让这四百骑兵作为斥候游骑,探查宋军的底细,同时封锁道路,让刘纪等人无法及时得知宋国援军抵达昆仑关的消息。哪里想到刚随军抵达归仁铺,还没来得及散开,就在战阵上失去了四分之一的军力。

他越看地下的亲信,心头火气就越是旺盛。“滚下去!”

一声怒喝远远不足以表达李常杰燃烧在心头的怒火,但他更清楚,眼下不是杀人的时候。归仁铺一战的失败,若说有责任,他自己的责任最重——太低估了宋军将领的猖狂和胆量。

得蒙大赦,那名骑将磕头谢了不杀之恩,连忙退了出去。

李常杰坐了下来,眉心皱成一个川字。

骑兵新败,且一日中在邕州和归仁铺之间来回狂奔,马力消耗又大。一两天之内,剩下的三百骑,能有一半派得上用场就是万幸了。这下怎么查探宋军的详细军情?两次大败,都是败在敌情不明上,误算了宋军的行动。

所以有件事让他疑惑不已,“怎么来的这么快!?”

同样的疑问也盘踞在宗亶的心中。

从归仁铺前线传到手中的紧急军情,完全出乎李常杰意料之外。宋军来得也太快了,占领了昆仑关之后,根本都不多做休整,就挥兵直扑归仁铺,即便长山驿一战是黄金满缴的投名状,可宋军将领难道不知道如此激进的连日进兵,究竟要冒多大的风险?

但宋人就是来了,而且轻而易举的就赢了。

为了防止走漏消息,给刘纪等广源蛮帅得知黄金满已经投敌。李常杰费尽心里也只调出了三千兵马去支援归仁铺。在他想来,攻下了昆仑关,又看到邕州陷落,不需要再兼程救援的宋军肯定要在关城中休息个一两天。而他派出去的兵马,有一天时间就能将营垒初步建好。一旦有了固守用的营垒,怎么都能拖延上一段时间,让自己可以从容整军并顺利撤回国中。

还有前两天收到宋国荆南军进抵桂州,当时他怎么想都觉得宋军不可能立刻南下,留给他至少会有十天以上的时间,可以顺顺利利的解决了邕州城防,让毁掉的城池来迎接宋军。

可一切都计算错了,想不到宋人这般心急……不是,应该是气焰正盛。骄兵悍将都有这个毛病。他当年领军攻打占城,也照样是高歌猛进。根本不惧占城军有什么地方能够对他产生威胁。

而自己这边,则是“兵疲师老。”宗亶将心中的想法喃喃念了出来。

李常杰的神色郁郁。其实行军打仗引发的疲劳,对双方来说,情况都差不多,但士气上的差别就差得太远了。宋军破关克敌,接连大捷,正是兵锋最盛的时候。而己方则是猝不及防,在城下鏖战两月方才破城,正要洗城来提振士气,就当头一盆冷水,这士气就根本就挡不住的要往下落。

“攻打归仁铺的只有三千军,其中宋军不过一千之数。”李常杰狠狠咬着牙关,从败兵那里他也得知了,归仁铺之战,完全是宋军为主,而黄金满的两千蛮军,只不过是在后面捡漏而已。

宗亶道:“能成为一军的前锋,必然是精锐中的精锐,荆南军中的翘楚。否则区区千人就敢直逼邕州城下,任凭谁也不会有这个胆子。”

“抵达归仁铺的也只是前锋,昆仑关中必然还有主力没动。”李常杰怎么都不会去设想,眼下直奔邕州而来的大敌,就只有出现在归仁铺的不足一千的宋军。

“可宋军到底有多少?”宗亶问着。

三千,还是五千?或者更多。李常杰也没有答案。

眼下困扰他们的关键还是敌情不明,一切纯凭猜测。要是知道来袭的宋军到底有多少,至少能有办法做出适当的应对。

“怎么办?是派人去打探?”宗亶问着李常杰的意见。

“要撤了!”李常杰站了起来。做出了决定之后,缠绕在心头的迷雾一扫而空,对眼下的局面看得也更为清晰明白。仰天长舒一口气,“宋人的底虽说现在仍没弄清,但也没事就再冒险和等待结果了。我亲自领军镇守后路,你带着人先撤。”

纵然纵兵掠城,任何一名将帅都不会将手上所有兵力如同撒豆子一般的都撒进城中,总会在手上保持一支可靠的机动力量。这样的军队并不需要入城洗劫,在府库等大宗收获中,有很大一部分都是留给他们的。

李常杰留在大营里的有一万三四的兵力。留在身边的这些一万多人,都是他最可以信赖的队伍。就算让他们为全军断后,李常杰自信,凭借自己的威望并不会引起他们的反弹。

前面犹疑轻敌,连败是他自找。但眼下既然有了决断,就绝不会再错下去。莫要小瞧他李常杰。

“那刘纪等人怎么办,继续瞒着他们?”

“瞒不过的。”李常杰摇头叹息,兵败归仁铺的动静实在太大了,刘纪等广源蛮帅又不是瞎子聋子,“这时候他们肯定已经都知道了。不信现在派人去请他们,没一个会再赶过来。”

形势急转直下,宗亶也没有别的选择,“那就只能将宋军来援的事跟他们说了。”

李常杰点点头,“虽然他们都知道了,但说与不说是两回事。将宋人来援的事跟他们说明白,然后一起撤军。在邕州城下留得越久,他们就越可能投向宋人。不过南返之后,离着宋人越远,叛投的胆量就会越小。”辅国太尉眯起的双眼变得危险起来,“不能给刘纪他们多余的时间,要逼他们速下决断。”

宗亶心领神会,快刀斩乱麻是唯一解决办法,“那我就派人先去找刘纪他们三人了。”说着他又冷笑一声,“不知道他们现在还在不在邕州城中。”

“若是听到了消息,肯定就回营了。”李常杰叹道,“他们入城本来是好事,可现在就不同了。”

“广源军如今大半都散在城中,早杀红了眼,要将他们带出来,的确不好做。”

李常杰点了点头。不仅是广源军,还有他的大越官军,要想将他们从邕州城中拉出来整顿好,宗亶身上的任务可不轻。不过自己也一样。

散在邕州城中的队伍要大部收拢起来至少要两天的时间,而广源蛮军也许费时更多。为了给他们争取时间,自己这边至少要抵挡五天以上,然后再设法从阵前撤退。

敌前撤退要做到也许很难,但并不是不可能。这两个月来,他已经将左江附近渡口的船只全数控制,只要能顺利渡过左江,宋军一时间也只能望江兴叹。

“击鼓,聚将。”

中军鼓声响了起来,李常杰和宗亶打算将眼下的形势与摊牌。归仁铺的大败在败兵讨回来后,已经传遍军中,想必下面的将校都在等着他们的解释和决断。

在鼓声中,赶来的第一人不是将领,而是从邕州城中而来,奉上士卒的军卒,“禀太尉,邕州州衙起火,多少间屋子库房全都烧了,苏缄也没有抓到。”

李常杰低声咒骂了一句,也不知怎么回事,钦州、廉州都开城投降,两州官员都顺服得很,可邕州城中竟然无一人降顺,不是战死就是自尽。要不是他们,他如何会陷入如今的窘境,恨不能将这些人挫骨扬灰。

很快,除了仍沉湎于城中烧杀劫掠的十几个将领,李常杰麾下的将佐都到齐了。没有丝毫隐瞒,李常杰将眼下的敌情通报给他的部下。

在主帅口中确认了能以一千破四千的精锐宋军就在几十里外的归仁铺,每一个人的脸色都很难看。

李常杰没有多说什么鼓励军心的话,眼下需要的是去做,而不是说。

“宋帝不仁,任用奸臣,我大越王师吊民伐罪,自出战以来,连克多州,直至邕州城破,宋人闻风丧胆。如今虽有小挫,但与大局无碍。不过出战时日已久,也到了该回国中的时候。撤退之事现在皆由宗太尉总掌。至于本帅……则为殿后。”李常杰挺腰起立,扶着腰中长剑,豪气干云,“就让我去会会领军来援的宋将!”

