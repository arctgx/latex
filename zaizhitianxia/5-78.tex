\section{第十章 却惭横刀问戎昭(十)}

【放假果然还是不得闲,还有一章请各位书友明天上午再来看,不会少的。】

当帐中终于只剩萧十三一人之后,一直都是堆着微笑的一张圆脸终于拉了下来。

大辽的北院枢密副使现在的脸色很是有些难看,从信使回来的韩冈口信中,萧十三听到的满满的都是威胁。

卡准了耶律乙辛一派现阶段的弱点,韩冈狂妄也便是肆无忌惮。毫不在意的折辱着他派去的使者。甚至连话都不让说出来,就将人赶回来了。

萧十三虽没有出使的经验,但好歹见识过不少宋国派来的使臣,从没有见过这样的人物。连一句话都不让说啊,这怎么能不让人愤怒。

但萧十三还是没有决定就此动手。

受到如此的羞辱,没有攻下西陉寨等缘边军寨的把握,贸然攻击,只会得到更大的羞辱。而仅仅是骚扰的话,则就完全是笑话了。

一团火在他的心中烧。

低头看看手上的纸条,熊熊燃烧在萧十三心头的怒火顿时消退了许多,毕竟秋天已经到了,该出动的,能出动的,全都可以动手了。

大辽、西夏,为此准备了有一年的时间,眼下最多再有一个月就该收网了。

虽然说这一番两国合谋的计划,在施行的过程中并不是一帆风顺,甚至在宋军攻到灵州城下之后,几乎都到了不可收拾的地步——一旦灵州失守,什么样的计划都没有意义——但党项人终究还是撑过来了,而眼下宋人的愚蠢,也给了大辽、西夏绝好的机会。

只要计划能成功的话,尚父的位置将会稳如泰山,任何人都无法动摇。

到时候,即便韩冈再强硬又能如何,他所能影响的地方只在河东而已。而且壳子再硬,内芯却是软的,东京汴梁,有跟韩冈一样不听任何条件,就直接驱逐使者的天子吗?

“枢密,蔚州团练求见。”门外的禀报,打断了萧十三的思路。

‘喜孙,他来做什么?’萧十三疑惑着,但转又恍然。

表字喜孙的耶律盈隐出身五院部,与耶律乙辛同帐,而且本身还拥有两千披甲骑兵,都是精锐,与他走得近的,也皆是实权贵胄。在萧十三的麾下,一向是横着走。甚至对萧十三也不是很看得起。

“什么事?”当耶律盈隐带着七八个同伴来到帐中之后,萧十三直截了当的问道。

耶律盈隐昂着头:“宋猪羞辱我大辽使节,末将是来请求出战的。”

“军国重事,岂是儿戏。不行!”萧十三一口拒绝。

“难道副枢是怕了不成?”耶律盈隐咧嘴笑道,“南朝的那些猪猡竟然如此狂妄,奇首可汗的子孙,可忍不下这样羞辱。”他回头,对着一起来的同伴,喝问道:“你们说,是也不是!”

一片声的回答,为耶律盈隐壮着声势。

萧十三连眉毛也没动弹一下,扫了几人一眼:“想要出兵,当然可以,但给我先立军令状!不敢立军令状的,就老老实实在营中待着。谁敢私自离营一步,军法从事!”

“不就是军令状吗?如何不敢立!”耶律盈隐大声道:“若不能拿回三五百个宋猪的首级,我耶律盈隐甘当军法!”

耶律盈隐不愿耽搁时间,当即就让文书写了军令状,按了指模,发了毒誓。拿起军令状,递给萧十三,纵声大笑,“还请副枢收好了。稍待片刻,待我砍回几百个宋猪的头颅,便来缴令!”

萧十三望着耶律盈隐等人转身离开的背影,眼中只有淡淡的讥讽。

不过是想拿宋人百姓的首级充数而已,难道以为他萧十三会看不出其中的门道。未免太小瞧人了,不论是对他萧十三,还是对对面的韩冈,哪有那么容易的事。

但有些人,死了倒是好事……

……………………

挥手让去雁门寨送信而回的次子退下去休息,西陉寨主秦怀信问着侍立身侧的长子,“大哥儿,你怎么看?”

自家的嫡亲弟弟刚刚用了兴奋的语调,详详细细的描述了正在雁门寨的新任经略是怎么折辱辽人的使节。秦琬正在沉思中,便听到父亲的讯问。他抬眼道:“韩经略刚勇无畏,不惧北虏的威胁,也难怪二哥儿会在一见之下,便心服口服。”

秦怀信抿了抿嘴:“为父是问你怎么看你二弟说的那番话。”

秦琬笑道:“孩儿跟二哥儿一样,有这么一个经略使,乃是河东之福。”

不过见一次韩冈,就让次子那般兴奋,让长子如此推崇,这让秦怀信始料未及。但仔细想想,如果换做自己十几二十岁的时候,多半也会对这样性格强硬、毫不畏惧辽人的主帅顶礼膜拜。

其实韩冈的态度在比次子早一步返回的辽国使节脸上就能看出端倪,挂着寒霜匆匆离开,怎么也不会是占到便宜的表情。所以,在次子述说了来龙去脉之后,说惊讶,也只有一点点而已。

秦怀信在河东路军中打了一辈子的滚,祖上上溯三代,甚至还跟着杨业杨无敌一起杀进朔州过。在他的记忆里,近几十年,可没有一个对外如此强硬的经略使了。

不过新来的韩经略会对辽人这样针锋相对,毫不退让,当还是太过年轻的缘故。还不到三十!过去,哪一个不是四十五十往上去的?但这位新任的河东经略使识见和能力,秦怀信不会去怀疑,他的成就已经让多少人都暗叹自家的一把年纪都活到狗身上了。

秦怀信叹了一口气:“要是当时来主持划界谈判的是这位小韩经略就好了。”

秦琬撇了一下嘴:“割让代北地,吕直阁【吕大防】和韩玉汝【韩缜】龙图都是反对的。即便是后来的沈学士【沈括】,也是在政事堂的架阁库里查到了多少辽国国书,证明是大黄平、萨尔台、天池子都是属于大宋,主张言辞拒绝。可惜京中……”

秦怀信脸色一变,当即厉声喝道:“这话不许在外面说!”

秦琬低头回话:“孩儿明白。”

这话当然不能在外面说,逼着韩缜、吕大防割地的可是当今天子,写信威胁一直在谈判中设置障碍的韩缜的也是当今天子。如果皇帝咬紧牙关,对辽人的讹诈不加理会,大宋的疆界如何会向南收缩十几里,一直推到西陉寨外?

一切的责任,应该由天子来负。不过秦怀信不敢这么想,只敢愤怒于当时朝中大臣不能阻止天子的胡作非为。天子是没有错的,有错的一定是奸臣,是那些恐吓天子,甚至说宋辽大小八十一战,其间只有一胜的奸臣。

看了一眼似乎还有些不以为然的儿子,秦怀信心中暗叹一口气。

自己的长子,虽然没有以一当百的武勇,但眼光见识都可以用出色来评价,领军上阵也不输人。放在河东军中,秦怀信确信他能轻易侧身挤进年轻一辈第一流人物之列,也就比将种折可适差了一筹。

就是有些傲气,这些棱角是年轻人所特有的,却也是必须打磨掉的。就像新任的河东路经略使一样,还没有来得及在官场上被冲刷得如河底的石子一般圆滑,可那身棱角迟早会逐渐消失。

但儿子的看法并没有错。责任不该由吕大防、韩缜等一众参与谈判的官员承担,他们只是听命行事而已。

当初朝廷划界割地,对于天子和朝堂诸公来说,不过是争一争嘴皮子,丢不丢脸面的问题。但被划出去的土地上,可是生活着成千上万的百姓。

主户一千五百户,客户倍之,男女老少不啻虑数万,全都被迫放弃了家园和土地,迁移回内地。光是为了安置他们,代州知州以下,各县、各寨,都是伤透了脑筋。失地的百姓到如今都没有完全安定下来,时不时的还有一场械斗,发生在他们和安置村庄的土著之间。

秦怀信在西陉寨任寨主前前后后已经有十年了。中间只在熙宁八年因为反对割地,又故意拖延在谈判地点设置帐幕的任务,而被转了差遣。但一年后就又被调了回来,因为需要他安抚被撤回的百姓。秦怀信在代北诸寨中,名望甚高,也只有他才能安抚得下流离失所的代北百姓。

相对来说,韩冈这样的经略使,还真对了他的口味。

但这样意气用事,也很难说会有什么样的结果。

从情理上说,辽人的确不会贸然攻打地势险要的西陉寨,就算韩冈的言辞近乎于挑衅,对面的辽军主帅萧十三也不可能命令麾下的将士往据山而守的坚寨上硬碰。

可世间之事哪有全然依着情理来的?谁能拍着胸脯说辽人绝不会来攻?万一发了疯,硬撞上来,还能指着萧十三的鼻子说这不合情理吗?万一他们分散开来,沿着各条小道去洗劫附近的村寨,除了大骂他们违反盟约,还能怎么做?上面能答应他出兵援救吗?

也只不知道答应下来的两个指挥援军,什么时候能到。

秦怀信正烦心,一名军官慌慌张张的冲到了门外,大声叫道:“寨主!西陉东谷那里的辽狗有动静了!看样子是要来攻城了。”

‘我就说吧。’秦怀信一声暗叹。

