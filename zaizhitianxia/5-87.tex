\section{第十章 却惭横刀问戎昭(19)}

【十分钟后第二更。】

时隔多日,李宪重又回到了太原。

李宪抵达州衙前的时候,就看见两名穿着丧服的中年人被人用门板抬了出去,脸色蜡黄,有气无力的呻吟着,臀股处的白麻鲜红一片,向外渗着血,似乎是刚刚被板子好生教训了一通。而后又是一名荆钗麻衣的少妇牵着一个五六岁的小儿从衙门里出来,小儿亦是披麻戴孝,看样子像是母子。

母子二人一出来,拥在两名中年人身边的几个男女立刻对这一对母子怒目而视,有人甚至诟骂出口。只是回头见到李宪一行的派头,又紧张的停了口。

李宪看了他们几眼,心道这模样,多半是刚刚审结了一桩家产析断的案子。李宪没什么兴趣理会他们,便往衙门中走进去,被小吏一路引到韩冈的公厅中。

“都知一路辛苦,未能远迎,韩冈失礼了。”韩冈起身跟李宪见了礼之后,歉然道:“还请都知少待,这个月百来桩案子的判状,今天就要发去审刑院。”他低头看了看桌上的卷宗,叹了一声,“眼看着就要天黑了……”

“请龙图自便,李宪就在这里等候。”韩冈熟不拘礼,李宪也不以为意。瞅瞅桌上高高堆起的卷宗,笑了一笑,就在一边坐下来等待。

韩冈本来是想让人领着李宪去见客的花厅暂歇,但见李宪就在公厅中坐下,却也没说什么,改让黄裳去陪着他说话。

韩冈埋首公务,李宪看了一阵后,低声问黄裳道:“今天审的是争产的案子?”

“嗯,是兄弟争产。”黄裳点点头,“龙图以其母尚在,不得妄言分家,将提议分家的两人各杖二十,打了出去。”

李宪听了点了点头,但又立刻觉得有点不对,回想了一下方才看到的场景,问道:“……可是继母?”

“都知可是在进来时看到刘家人出去?”黄裳笑道:“正是继母。寿阳人刘玉德续弦之后又生了一个幼子。刘玉德月前病卒,其子刘大、刘四,为了多分家产,先是指称其继母刘王氏不是续弦而是妾室,又说其弟刘六不是刘家的亲生子,而是刘王氏携来。为此还买通了稳婆、邻里、族人,乃至县中和府中的胥吏……”

“何至于如此!?”李宪惊讶了,多少人一起收买,不可能是为了区区几千几万,“刘家的家产值得他们这么费尽心力?!”

“刘玉德在寿阳号称刘半城,光是在太原就有三个庄子,一百多顷田地,至于在寿阳乡里,就更多了。而且刘家在河东各州县,有数百处处上好市口的商铺。如果都知曾有留意的话,在晋宁城中都应该见过丰和号的牌匾。”

“难怪了。”李宪点点头,光是黄裳所说的这些田宅,说不定都能有上百万贯了,就是放在东京城中,兄弟之间也少不得反目成仇,“难怪龙图会出面审理此案。”

李宪这么说,可心中还是疑惑难解,以韩冈的身份,一般来说只要不涉及大案要案,根本就没必要亲自上阵。就是有人敲了衙门外的冤鼓,交给府中的推官来处置也足够了。

推官在名义上是知府的僚属,负责审理案件,一桩争产案,不该轮到韩冈出马。尤其是这件案子关系到几十万上百万贯的家产,韩冈违反惯例贸然涉足,难道不用避忌瓜田李下的嫌疑?

黄裳似乎看出了李宪心中的疑惑,解释道:“这个案子涉及推官何必中的姻亲,依例避嫌。所以龙图就接手过来。不过也不只是避嫌了,”他不屑的撇了撇嘴,“都知你是没看到,府中这两天因为这个案子被龙图惩治的胥吏,有十一人之多。寿阳县中,还没有查,查出来更不知道有多少。”

韩冈审阅着即将发去京城的一份份判状,黄裳和李宪的低语也传入他的耳中。

每天呈送到太原府中的案子,并不是以刑事案为多,绝大多数的是普通的民事案。在韩冈经手的案子中,民事案和刑事案的比例,大约是十比一。而刑事案中,杀人案等重罪,更是只有百分之一二。

如陈世儒弑母那等逆人伦的大案,更是几十年都不会碰上一次。一旦事发,甚至能震动一州、一路,直接送到天子的案头上。一个不好,就会给审刑院、大理寺乃至御史台穷追猛打。运气好点,也少不了教化不力的罪过,最轻也要罚铜二三十斤。

幸好韩冈没有碰上人伦大案的坏运气,今天的这一桩仅仅是争夺家产而已,韩冈是在复核时发现了其中的问题,然后移牒寿阳县,将这件案子的涉案人都提来太原。

作为知府,对下属诸县报上来复核的案子,经过检查之后,有问题的发回去重审,没问题的加以确认——一般需要着重检查的,是流刑以上的刑事案,以及涉及金钱和土地数目比较大的民事案。

亲民官之所以地位特殊,就是因为他们什么都要管。军事、政事、司法和仓储运输,全都在亲民官的管辖范围。但事必亲躬是不可能的。作为州府一级的官员,韩冈已经很少直接断案。大部分州官,除非是上门敲冤鼓,才不得不升堂,而且升了堂之后,转给推官的也极多。

其实亲自审这一桩案子,韩冈倒是有五成是想起了他的表弟冯从义,刘家争产几乎是冯家争产案的翻版。而且刘德玉本人,算是晋商中的头面人物,韩冈前几天收到冯从义的信,上面正好还提到了刘德玉和他的丰和号。

丰和号刘家是晋商的中坚。这时候的晋商,还没有后世的气派,甚至还不如雍商在京城名气大,但什么都敢卖的胆子却半点不输后人。就如刘德玉,在北面的辽国也是有着许多门路。这也是冯从义的来信中所提到的。

不过冯从义写信时,只是到韩冈任职河东,所以顺便提起了晋商中的几个有名的人物,却还不知道刘德玉已经病死。但他在信中,无巧不巧的提到了曾经在刘德玉续弦时,遣人补送了贺礼的事,使得韩冈在开审前,就对真相有了底,不至于被篡改过的文件和一干被收买的证人所蒙骗。

刘家现在的情况,与当年冯家极其相似。当年韩冈的处理办法是直接让冯从义放弃了家产,来个损人不利己,一拍两散。不过当自己成了审判者之后,就不能这么玩了。

通过对比邻居、稳婆和族人的口供,并检查过各项文书,韩冈将其中被篡改的证据一一罗列出来——事前确认了证据是伪造,从中寻找破绽很是容易,比起不知真伪的情况时要简单得多。

文书被确认是伪造,所有经办此事的胥吏自然逃脱不了国法的惩处。等到胥吏都被处置,剩下的证人也就好解决了,总共也就三天的时间,韩冈很轻易的就还了刘王氏的公道。当刘王氏续弦的身份被承认,那么接下来他的判决也就顺理成章。

——父母在,依律是不能分家的,这是孝道的根本。世间虽多有违反律法的情况,父母老病后,主动为子女分财,省得死后子女争产为世人所笑。但韩冈拿着律条为证,没人能说他判得不对,这可是遵守三纲五常、敦化风俗的典型判例。

黄裳明显得对韩冈在这桩案子上表现出来的手段赞赏不已,甚至是崇敬,不知道韩冈是拿着答案找证据,还以为他是明察秋毫、洞悉情弊,“刘王氏入门的时候,是带了陪嫁的,虽然很微薄,但陪嫁就是陪嫁,也列了单据。给人做妾,纵使能带着私房,也不可能大红单子列出来。这个证据一查出来,就没得说了。而且伪造的几份契书上,签押和时间都有问题。龙图由此着手,将一干涉案的胥吏捉了,杖责、除名一条条下来,等传了人证过堂,还没等龙图细问,被收买的证人全都改了口……都是怕的!”

“当是怕的。”李宪频频点头。

这一次韩冈断案,多半是杀鸡儆猴。李宪如何看不出来?

韩冈本来名气就大,到了河东,对地方军事上的影响力,就跟一干出外的宰执重臣一般,甚至犹有过之。再特意挑选几个错判的案子来审,顺便惩治了衙中的吏员,震慑了僚属,配合上他在民间的声望,在太原府里的局面也就打开了。

出手毫不迟疑,以韩冈的手段,镇住衙中的一众宵小毫不费力。而接下来却没有深究,也省得追究下去,乱了人心。鬼才相信刘家的两个成年儿子,只收买了证人和胥吏。所谓推官何必中因姻亲避嫌,其中还不知有多少名堂。

一贯都说韩冈才智高绝,但这一桩案子所表现出来的,不是才智的问题,而是他处事手段的圆熟老辣。一点也没有年轻人的毛躁。要是身边有老于世故的幕僚那还不算出奇,可李宪看过来,韩冈身边的门客幕宾基本上都是年轻人,皆是气学门下弟子,黄裳都算是老成了。

