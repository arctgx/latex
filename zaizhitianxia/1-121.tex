\section{第45章 樊楼春色难留意(一)}

【第三更,求红票,收藏】

寒夜之中,开封内城远比不上外城的热闹。踏步在宽阔的御街之上,只听得马蹄笃笃的敲着地面。御街宽达两百步,在无光的夜里,完全看不清街对面的情形。只有挂在马前的一盏灯笼,驱散了前路的黑暗。而前方朱雀门的灯火,也指明了去路。

韩冈告辞离开王安石府,骑的马还是借自王家。王旁送韩冈出门,知道和安坊不同于闹市区,难以雇到马匹,便遣了自家里充作马夫的一名小校送韩冈回去。

此时王公官府,通常都有厢军走卒充作仆役,王安石家也不能免俗,不过就只留了几个老兵守外院,再加一个照料坐骑的马夫。而平常护卫着王安石上朝的七八十人的随从队伍,却都是住在外面,天天早上赶来,算不得王家仆役。

在王家坐了半晌,就喝了两杯清茶,韩冈肚子都有些饿了。回头看看在王家做马夫的小校,正拉着一张脸。深夜中睡得正香,却被人唤起去送客,换作是自己,免不了要大骂一通,即便不能骂出声,腹诽是肯定的。韩冈心知小校必然在肚子里暗骂自己,只是这个仇结得有些冤枉。

韩冈两人从内城南面的朱雀门侧门出来,守门的士兵并不仔细检查,看到小校亮出的牌子便放了行。韩冈看了直摇头,他方才进来时都已经入夜,甚至连检查都没遇上。

开封的内城真可以说是有名无实,单是韩冈这几天从朱雀门进出,就发现有好几段城墙的墙头都崩落了,放在那里没去修,更别提还有更多的城墙韩冈还没有看到。这与设施完备、墙体坚固的外城和皇城完全不能比。不过内城城墙本来就是无用,不过是旧年还未升为京城时的汴州城墙,以如今朝廷的财政状况,即便挤出钱来,也只会拿去修外城城墙。

出了朱雀门,过了门前宽阔的龙津石桥,当面横着的就是朱雀门街。虽比不上御街的两百步,但朱雀门街也有五十步宽。是外城的几条主街之一,亦是店铺林立,排满了街道两侧。不过朱雀门街不比小甜水巷,做得是白天生意,到了夜间街两侧的店铺基本上都关了,街中黑黢黢一片。

唯有几个在街边支起的摊子,就近着御街和朱雀门街的交叉口,生着热腾腾炉火,挂着几盏防风灯笼,有着些许微光。他们有点像是后世夜市上的小吃摊,晚上摆出来,到了凌晨再收回去。

即便是临近子夜,街市中依然有人行走,韩冈还看到一队巡城十几人围着一家摊子的火炉旁,喝着热汤。有这些人来来去去,小吃摊也不用担心没有生意可做。

还有不少醉汉在路上歪歪倒倒,有的干脆就躺在路边,不过通常他们都被更夫和巡城一脚踢起来,让他们赶快回家,省得被冻死。

一群醉汉就横在路前,唱着不着调的歌,东歪西倒的迎面过来。韩冈提着缰绳,操纵着坐骑躲避着他们。参知政事家用的马匹被训练得不差,虽然韩冈骑的这匹是身材不高的驽马,却很聪明的从人群中间穿过,连衣角都没蹭到。

“那不是韩官人吗?!”这时一声大喊,惊到街上不多的行人。

声音一传入耳中,韩冈就撇了撇嘴,这是刘仲武的声音,就是有些大舌头,多半是酒喝多了。他在马上回头,就见着大街对面,李小六扶着脚步蹒跚的刘仲武,醉醺醺的和路明一起走过来。

看到是他们,韩冈便跳下马,拱了拱手,道谢说:“夜中出行,劳烦小哥不少。下面我跟他们一起回驿馆,小哥还请自便。”说着他又从怀里掏出一串钱递了过去,“天寒地冻,小哥拿去买点热酒暖暖身子。”

小校板着的脸缓了下来,推让了几下,便笑眯眯的把钱收了。向着韩冈道谢作揖,然后才上马往来路上去。他们一人两马回头时,又穿过了那群醉汉,现在韩冈看清了,小校双手完全笼在袖中,根本不碰马缰,只凭两匹马自己就从醉汉中顺利的穿了过去,

韩冈看着小校的背影,若有所思。方才他骑的马能规避行人,看来不是因为自己提着缰绳,而是被训练出来的。刘仲武的赤骝韩冈见识过,那匹河西良驹都没这般灵巧,不知是不是这位马夫的功劳。

应该是吧?韩冈想着,能被派到参知政事家里照料坐骑,水平不会差的。只是这样的人才却不在前线立功,也不在牧监做事,反而成了高官家门下的走卒,难怪大宋的十几个牧监,每年砸进去百万贯,也不见有几匹好马出来!

对面的三人这时已经走了过来。尤其是刘仲武,也不知喝了多少酒,走的踉踉跄跄,瘦小的李小六要撑着人高马大的他,几乎都给压垮了。刚刚得到官身的刘仲武还带着酒意大声喊着:“韩官人,怎么你在这里?”

他们走到近前,一股子和酒味混在一起的香粉味道顿时扑面而来。刺鼻的气味让韩冈往后退了小半步,皱着眉头看着醉醺醺的两人。不是倚红偎翠,身上哪会弄得这么些怪味道。看起来他们在状元楼也是风流快活了一阵。

不过状元楼是官办,里面来自于教坊司的官妓按着律条是不陪夜的,也就是卖艺不卖身。虽然例外的情况不少,但刘仲武和路明可不够资格,好歹也要有些才学和文名,才能让那些心气颇高的歌妓放下身段。想来两人应该是只是闻到了腥味,没吃到鱼才是。韩冈为两人遗憾,若是章俞在小甜水巷请客,不至于这么早就回来。

路明的酒意比刘仲武少上一点,还保持着一定的清醒,他小心翼翼地问着:“听说官人去了王相公府上?”

韩冈点点头,遗憾道:“要不是王参政使人招我去私邸,就能与子文兄和路兄一起去状元楼喝酒了。”

确认了韩冈的确是被王安石请去,路明顿时肃然起敬,又问道:“章老员外还说他的儿子也去了王相公的府上,不知官人见到了没有?”

“这却没见到,只去跟王家的二衙内下了两盘棋。”

韩冈说得平淡,路明却更是一脸惊羡,“寻常人去宰执家,也就能跟门子说两句。官人能得王衙内一起下棋,在王参政那里必然受看重,日后飞黄腾达自是不必说的。”

韩冈闻言冷笑。与王旁下棋,跟他老子又有什么关系?!自家当初跟王厚一夜深谈下来,都是称兄道弟的交情了,但王韶会拿出经略司勾当公事这个位子,还不是看在自己的才智和能力上,跟他的儿子全然无关。王安石一国宰执,又是留名青史的人物,说他会因为跟王旁下棋下得好而另眼相看,韩冈只会大笑,可不会相信。

王安石让他空跑了一趟,韩冈心中本不无微词。只是反过来想,这还是自己地位不够的缘故,若是如章惇一般成了变法派的核心人物,王安石怎么也不可能让自己白跑。如此一想,韩冈心中释然,放宽了心思。他向来看得开,一向认为抱怨别人很容易,但没意义,不如求诸于己。等有实力了,可以去报复,而不是像女人一般抱怨。

不想提自己在王安石府受到的冷淡,韩冈转过身子,当先往城南驿方向走去。韩冈走得不快,悠然自得的像是在花园中散步。深夜月下,漫步在千年之前的都城御街边,眼前一条拱桥如虹,飞跨在五丈河头,看着周围一重重飞檐坡顶的楼阁屋舍,有着一种超越现实的魔幻感觉。但刘仲武和路明却一点也不魔幻,他们带着酒臭气跟了上来,拖沓的脚步声踩碎了韩冈一时的恍惚。

韩冈轻叹一声,侧过身子问着路明和刘仲武:“不知两位在状元楼有什么遭遇?”

“不外乎美酒佳人。”路明故作平淡的说着,学着韩冈方才的语气。

“都好,人也好,酒也好,菜都是好的。到了京城,才知道秦州的几家酒楼,都是狗屎!那时还仰着脖子看,掰着手指看什么时候才能领了俸禄去逛上一逛,现在请俺去都不去!”刘仲武则是醉得厉害,口无遮拦,“就是章老员外带着的伴当太娘气了,不像个汉子,说个话都翘着小指头。”

“是刘官人你不懂,有人就好这一口。”路明不愧是八卦党,眼光甚毒,笑得淫【和谐】荡:“章老员外这叫水旱同行,男女通吃!”

“走水路有奶吃,走旱道能吃什么?吃屎吗?”刘仲武哈哈大笑着,自以为说了个有趣的笑话。试射殿廷上的得意和状元楼的美酒佳人,把他的沉稳囫囵个儿的冲进了下水道,说话也没个顾忌。

想到下水道,韩冈左右一看,眼前的五丈河对岸正巧有条下水道通过来。黑沉沉的外口像个藏兵洞一般,至少有一丈多高,两丈宽,看起来甚至可以行船。

