\section{第16章 千里拒人亦扬名(上)}

冬天天黑得早,不过申时【三点到五点】中,天色便已经黯淡了下去。

“怎么还不换班!?”

赵隆守在伏羌城东门城楼上,百无聊赖的打着哈欠。城门下面,嘈杂声不绝于耳。位于群山间一个小盆地中央的伏羌城,守着官道水路,一天倒有千百人进出往返。而城门上头,赵隆却困得只想睡觉。

又一队骡车渐渐从远处的官道上走来,赵隆懒洋洋的趴在城墙上,看着他们越来越近。如今时近岁末,一队队载着军资往西北各寨堡的骡车、驴车、独轮车还有挑夫的队伍络绎不绝。现在过来的,已是今天的第四队了。

赵隆没精打采的看着来人,这一队看起来也没什么特别,就是人懒了点,怎么有几个闲人坐在车子上。赵隆奇怪的看了抵达城门下的车队,突然瞪大了眼睛。扶着雉堞,他探出头去,惊异的向下唤道:“王舜臣?!这不是延州的王四吗?”

在坐在骡车上,靠着一堆软绵绵的绸缎,半眯着眼休息的王舜臣闻言抬头。也是一下坐直身子,奇道:“赵大,怎么是你?!”

“怎么不能是俺!?”

王舜臣隔着两丈高的城墙,对赵隆喊道:“赵大你不是应了敢勇吗?怎么到伏羌城来守城门了!?”

赵隆的脸色有些难看,反诘道:“俺一个敢勇守城门也没什么,倒是堂堂正名军将,怎么做了押运的长行?!”

王舜臣连嘴仗也不肯输,“押运好啊!至少能顺路混点军功,总比天天坐在城门口,磨得屁股生茧要强!”

赵隆被堵得没话说,撇了撇嘴,把头缩了回去。

韩冈正等着监门官查验过路关防,听见王舜臣跟城楼上的守兵斗嘴,微微一笑。听着两人的对话,彼此间也是有点交情的。能与伏羌城的人搭上关系,在城里将军功和敌情报上时,至少能得到一些指点,不会两眼一抹黑,找错了人。

监门官看起来也是累了,只看了看关防,并没下去查验车辆,对躺在车上、看起来受了伤的几个民伕,也只是看了两眼,并没有细问,直接挥手将车队放行。

赵隆这时已从城墙上下来,正在城门内等着。他的身量跟韩冈差不多高,相貌则与王舜臣差不多丑,年岁大约二十上下,浑身上下的肌肉将外袍高高撑起,壮实得像头牛。论起武艺,赵隆能被招入敢勇,至少不会太差,但他的运气,却是相当的糟糕。

韩冈知道什么是敢勇。对于官位、军功,地方上的豪杰没有一个不喜欢的。但一旦从军便要在脸上手上刺字,这对好汉们来说,算是个极大的侮辱。所以宋廷特意设立了不须刺字的敢勇制度,让那些顾惜身体发肤的好汉们,能有机会参军求功。以敢勇的堪战,一般只要稍稍立些功劳,便能入官带兵。敢勇都是善战的精锐,往往为将帅所倚重,如赵隆这般落到城门守兵地步的,却也难得出一个。

骡车一辆辆的驶入城中,赵隆跟监门官打了个招呼,便施施然走了过来。

趁着这片刻,韩冈从王舜臣这里打听到了一点关于赵隆的情报。赵隆是成纪县人,自幼横行乡里,与来秦州避祸的刺头王舜臣不打不相识,时常酒肉往来,不过几个月的时间就混出了不浅的交情。他是在今年八月,党项兵犯秦州后应募敢勇的。但不知犯了什么事,才两个月的工夫,竟被发配来守城门。不过看赵隆找个由头就能走,监门官也不敢拦的样子,他在城门队里混得倒也不差。

“伏羌城内不能乱走,俺来给你们带路!”

走到车队边,赵隆也不理其他人,更是看都不看站在车边的韩冈。只自来熟的说了一句,自己就跳上车,给辎重队指了指方向,便学着王舜臣的样,舒舒服服地躺下来。转过头,一眼瞟见了王舜臣肩膀上包扎过的伤处,笑问道:“是不是在惠民桥私窠子里嫖了没付帐,给婊子咬的?”

“没错!”王舜臣一口承认,大言夸口,“爷爷大发神威,夜战十五,日战十八,干得几十个蕃族的婊子唉唉直叫。那些个婊子被干得痛快不过,才咬得爷爷一口。”

赵隆突然半抬起身子,望向后面装着蕃贼首级的车子。尽管首级都被盖住了,但此时风一起,血腥味还是透了出来。掩不去脸上的讶色,他惊问道::“装了半车子,怕是快三十了罢?”

被赵隆骚到痒处,王舜臣得意的扬起下巴,自傲道:“来了小一百,留下三十一!”

“……长能耐了啊!”王舜臣能痛痛快快的杀敌立功,自己只能苦守着城门,赵隆的神色分不清是羡慕还是嫉妒。

王舜臣哈哈大笑了几声,坐起来正想再吹嘘一下,但刚张开口就看到走在前面的韩冈,话便被堵在了肚子里。干咳了两下,自家也觉得不好意思,便改口道:“这都是韩秀才的功劳!洒家只是……俺只是占了一点光。”

韩冈笑着回头:“军将太自谦了,一张弓便射死十一个,如此勇武,放哪里都是件值得夸耀的!哪是韩某的功劳。”

“韩秀才?!”赵隆吃惊的扭头看着韩冈,一个走在前面的民伕,突然间就变成了秀才。

“韩秀才才是今次带队的,俺是……顺路,顺路!”王舜臣有些尴尬的为韩冈解释。

方才的一战后,韩冈让受伤的民伕和王舜臣坐在了骡车上,自己则下车走路,几天没更衣、洗澡,一身上下都被尘土笼罩,哪有半分读书人的模样。

“见过赵敢勇!”韩冈冲赵隆拱了拱手,赵隆也急忙跳下车来,向韩冈回礼。

大宋开国日久,右文左武已深入人心,对于有些能耐的读书人,武夫们都是有几分敬畏的。如果没有王舜臣提醒,赵隆也许还不会注意,但现在仔细一看,韩冈的确与其他民伕差别甚远。不但神情举止不类凡庸,就是身材、相貌皆是过人一等。尤其那对如长刀刀刃一般的双眉微微挑起,幽暗难测的双瞳看过来的时候,甚至让赵隆心中莫名生寒。

在赵隆的带领下,韩冈一行横穿伏羌城中,向今夜歇息的地方走去。

如果拿秦州城相比,伏羌城并不算大,但在军事城寨中,算是个大号城池。按照国中筑城立寨的惯例。城寨周长达到九百步的,称为城;九百到五百步的,称为寨;而五百步以下,就仅仅是堡;至于不到两百步的,勉强算个烽火台。

城、寨、堡各有定规形制,里面的建筑、仓储、衙门以及兵力布置,都不尽相同。作为军城,普通的是九百步城,千步城,最大也只有一千两百步,换算成里,也就三里出头,四里不到的样子。

位于甘谷水和渭水的汇合处,以两河交夹护翼的伏羌城,正是最大的千两百步军城,驻有四千官兵和他们的家人。城中也有坊市,酒店,除了军营多些,仓库多些,甲马多些,与普通的县城并无什么区别。

已是黄昏,按理说都是该回营、回家吃饭的时候,可城中现在却都是人来人往,总有点兵荒马乱的感觉。韩冈看着有些不对劲,王舜臣也觉得奇怪,问赵隆道:“城里有些乱啊,究竟出了什么事?”

赵隆神色郑重起来。他压低了声音,只让韩冈、王舜臣两人听见:

“今天午时才传来的消息,甘谷对面的西贼突然多了一万,其实这本也没什么,凭甘谷城足以抵挡。但偏偏前天守甘谷的张老都监却正好带了两千人出去巡边,据说是迎头撞上了,到现在还无半点音信回来。

甘谷里都在传张都监已经全军覆没了。甘谷城内如今只剩不到两千老弱,若是西贼攻来,根本抵挡不住,恐怕连谷内的心波三族都有些不安稳了。你们看着吧,如果张老都监再没个消息,到夜里烽火就要点起来了。”

“那秦州岂不是要大乱?”韩冈知道点燃烽火的意义,非是十万火急的紧急军情,不会有狼烟升起。反过来说,一旦烽火被点燃,狼烟腾起于天际,秦凤路的兵备都要全数动员起来,甚至还要发急脚递,速报京城。

“少了张老都监镇守,甘谷城多半会破,能不乱吗?”

秦凤路驻泊都监、甘谷知城张守约是关西一位赫赫有名的宿将,曾是杨文广的副手,参与修筑了硖石堡、甘谷城两座要塞。这两座城寨都是在党项人的眼皮底下修起,期间还遭到了几次攻击,却是安安稳稳地修筑成功。也因此,带兵防卫的张守约得了主帅杨文广之下的第一功。

他可以说是甘谷城中的定海神针,有他在,西夏的马步禁军——铁鹞子、步跋子来个三五万,都是不在话下,连援军都不用。但若是他不在,那就是眼前的这般情况,从北面的甘谷城,到中段的安远寨,再到韩冈现在身处的伏羌城,绵延六十多里长的甘谷全都乱了套。

ps:前路多蹇,这时就要看韩冈的表现了。

今天第三更,求红票,收藏。

