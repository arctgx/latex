\section{第48章 浮云蔽日光(中)}

上元节已经过去了半个月,二月就在眼前,可天气还是一般儿的冷。

守着东京城正西门——新郑门的狄贤,并不是多勤力的监门官,若依着他过去的脾气,多半在城上城下绕上一圈,就回温暖的房中去烤火。

只是狄贤现在站在城头上,迎着风,手持长弓。而他身边站了一队守门士卒,手上都拿着弓弩,严阵以待。

在风中站得久了,身子都冻得僵硬,但狄贤的两只眼睛犹如鹰隼一样盯着前方漂在半空中的那个上顶圆球的异物。

也就在一刻钟之前,那个异物随着一阵疾风从西面飘来,摇摇荡荡越过了新郑门的城垣。狄贤听了手下人急报上城时,异物已经深入了城中,让他不得不立刻遣人去通报开封府。

不过此事若是仅止于此倒也罢了,怪罪怪不到他头上。偏偏后面跟了不知多少看热闹的闲人,都闹着要进城去追着看个究竟。狄贤费尽了气力才将城门外的秩序给整顿好,没想到那东西又回来了,这下城外骚动又起,且连城中都涌来了一群人流,甚至比起上元节时都不差多少。

但这一次返回,异物上面的球已经微微瘪了下来,没有一开始那么圆,而高度也降低了不少,已能看清吊在圆球下面的是个盛物的大篮筐。

不论到底是什么东西,狄贤都不能任其来去。张弓搭箭,就准备对着圆球射过去。可突然迎面的来风一下猛烈了起来,异物飞速接近,眼见着就要正面撞上,顿时吓得城上一片混乱。

狄贤也给一个忠心的部下给扑倒,然后就眼睁睁的看着异物低低的擦着城墙顶上的雉堞滑出了城去,高度不断的下降,竟然就在横跨护城河的石桥上落了下来。

原本拥挤在桥上的围观者在恐惧中,纷纷退让,拼命散了开来,有几人来不及躲开的甚至直接跳下了护城河,幸好如今河水浅薄,而冰层也不厚,落水后扑腾了两下,就湿淋淋的站了起来,仅仅淹到胸口。

上千人围在桥头两端,一时不敢都上前一步,几千只眼睛都望着石桥中央,已然瘪了下来、盖住了整幅桥面的异物。

狄贤疾步下城,很欣慰的看到他的部下不及自觉的挡住了城门内外拥挤的人群,还拼命挤上了石桥两端,然后守在了那里。

狄贤穿过人群走近了,终于发现那个圆球不过是个下端开口的气囊,跟着如今蹴鞠中踢得气毬差不多。但下面吊着的篮子里到底是什么谁也说不清。尤其是在气囊的覆盖下,浮凸出来的篮子正不住的晃动,还有一阵阵怪声从中传出来,就更让人心生畏惧。

新郑门的监门官也是心头发毛,眼睛一转,就看到方才将自己扑倒在地的忠心部下,“张九四,你上去看看。”

“啊……?”

一片忠心换来了打前锋的资格,张九四满腔不愿的在瞪起眼的狄贤的逼迫下,小心翼翼的凑近了桥上的气囊。他几次想回头,却又在狄贤恶狠狠的瞪视中不得不又哭丧着脸往前挪着。

城内城外一时静了下来,人人屏气息声,几千双眼睛皆在看着张九四的行动。新郑门的守兵也都拿起了弓弩,只待蹦出个怪物来,就立刻动手。

漂在天上时看着是个篮子,张九四走到近前,掀开盖在上面的绸缎,也的确是个篮子。掌着腰刀,趴在篮子边上,低头向着里面偷眼望进去,张九四原本为了妖魔鬼怪而做的心理准备,却一下都落了空。如坠梦里的转回身,深一脚浅一脚的走了回来。

“到底是什么?!”狄贤立刻问道。

“猪……猪……”张九四恍恍惚惚的舌头打了半天的结,最后蹦出一句话来:“猪该走南薰门呐……”

啪的一声脆响,狄贤反手就是一个嘴巴,将说胡话的手下打醒。他大步走到篮子边,低头一看,的确就是一口浑身长毛的黑猪。被绳子捆得结结实实的生猪哼哼唧唧的,一个劲的挣扎,撞得篮筐不停地在抖。

当狄贤抬起头来,周围已经是里三层外三层,多少人一起挤上了桥。放眼望过去,黑压压的全是人,都张着嘴、踮着脚、勾着脖子向里张望。而且不知什么时候,连城头上都挤满了人。被堵在外面的闲人大声叫骂,拼了命的向往里面挤。守在里面顶住人流的守门将士,也快要吃不住劲了,各个脸都涨得通红。

“到底是什么?”仗着身份,郭忠孝几人爬上了城墙,扶着雉堞向下望着。

“猪……”何六耳尖,听到了一些声音,惊诧莫名:“这是用来运猪的?!”

“是韩冈……肯定是韩冈做的,难怪说是买船。”郭忠孝没头没脑的发言,让几名同伴都转头看向他。

“给洒家闪开!”一声虎吼,如同一记惊雷震慑当场,又将望着郭忠孝的几道视线扯了回来。

一名身高六尺有余的壮汉带着四名伴当,在城下的人群中左推右攘的排众而入。毛茸茸的一张胡子脸,面如锅底,双眉如帚,鼻子扁而宽,相貌猛恶无比。最特别的是他在不用瞪起眼睛已经让人心底发寒。

“尔乃谁人!?”狄贤一声断喝,几个守门小卒也随即持刀挡在狄贤的身前。

“洒家是军器监的!”壮汉操着浓浓的关西口音,左手探入怀中,掏出个做身份证明的腰牌来,甩手丢给狄贤。

“军器监?”听到这三个字,狄贤就是一怔,转而就有些不快。

不是因为军器监,而是因为判军器监的韩冈,将郑侠踢出京去的韩冈。虽然狄贤是武职,而郑侠是文职,但同样做着一桩差事,也算是点头之交。虽然整件事是郑侠本人不长眼,但他全家被发配去恩州,狄贤也免不了有些兔死狐悲物伤其类的感触。

至于周围,则是一片哗然。‘军器监、韩舍人’这几个字在人群中飞速传递。

狄贤低头验过腰牌,来人的姓名、身份都在上面,的确不是伪造,但这说明不了什么:“周全,尔来何事?”

“还有什么?”周全抬起右臂,没有手,只有钩。右腕上装了一只铁钩,钩尖寒光闪闪,遥遥指着石桥中央,“这飞船是军器监的东西,要马上回收!”

“飞船?这是军器监的?”狄贤傻愣愣的问着。

“还能是谁家的?”周全大大咧咧的说着:“洒家受了我家舍人的吩咐,正管着造飞船的差事。今天绳子没拴好,给风刮飞了。要不是这样,洒家吃撑了才出来追,还累得跟狗一样。”

说了两句,一下仿佛醒悟了过来不该说这么多。一瞪眼,冲着狄贤狠狠一声大喝,“还不快点赶紧让人散了!没看到这么多人都在城堵门口?”又回过头,冲着围观的人群很不耐烦的吼着:“散了!散了!”

狄贤看着周全没把自己放在眼里的指手画脚,心头火气大起:“你有什么证据说这是你的?!”

周全冷笑一声:“你倒说说除了军器监的韩舍人,还有谁能造出这飞船?”

“韩舍人不是说要造铁船吗?”人群中有人亮着嗓门喊着,惹起了几千人一起点头。

“铁船、飞船,都是一个道理的东西,有什么大惊小怪的?!”周全回头一扫城上城下、数千近万的围观群众,下巴扬得老高,只拿两个黑洞洞的鼻孔冲着人,“一点见识都没有!”

虽然他只是个关西人,但他投向周围的鄙视眼神,却分明跟皇城脚下的居民看着外地乡巴佬时的眼神一模一样。

狄贤被这个在军器监挂着吏职的汉子气得脑袋充血,他可是官啊。但一想到身后被称作飞船的异物在天上飞的样子,对韩冈的畏惧顿时又冒了出来,那一位还有什么做不到?

“你再不快点,等我家舍人来,发了火,那就不关洒家的事。”周全现在却是一点不急了,“这飞船不值什么,可要是被辽人的奸细给偷了去,洒家一人可担待不起!”

听到‘辽人’二字,狄贤便心底一惊。要是当真被辽人偷学了去,眼前的这毛胡子脸要被治罪,他狄贤绝对也少不了一个罪名。而韩冈肯定要偏帮他的人,到时候难道要去恩州跟郑侠做邻居不成?

可也不能就这么放人啊……都已经通报了开封府,很快就该有人来了。而且不经城门逾墙而入肯定是个罪名,只是飞过去的是猪,不是人!再看看周围,已是人山人海,一个不好就要出乱子,这到底要他怎么处置啊?

狄贤脑中一团浆糊。

郭忠孝几人这时在城头上愣愣的望着下面,飞船的名号已经传到了他们的耳朵里。方才还在嗤笑着韩冈人品低劣,不顾师门大义。转眼就是飞船到了天上。虽然这一回是装了一头猪进去,但下一回说不定就是人了。

能让人飞天!

只在传说中出现的事迹,如今就在眼前。

只要韩冈说一句这是格物致知的功劳,不知会有多少士子赶往关中横渠,求着一个门生的资格。

笑韩冈?可笑得都是自家!

