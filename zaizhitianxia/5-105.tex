\section{第11章 城下马鸣谁与守(17)}

战事已经到了尾声。

折可适陪着主将李瑛,漫步在战场中。主力围定了敌军盘踞的村寨,剩下的人正在打扫战场。大部分的敌军之前已经蹿进了前方的一座村寨中,但没有来得及逃离的百十阻卜骑兵,已经在绝望中拼死作战。

周围还有着尚未完结的厮杀,但历经战火的两人浑不在意。

就在侧前方的不远处,一名高壮如熊罴的阻卜骑手,与另一名宋军战士扭打着下了马。仗着身高体壮,阻卜骑手几刀下去,便将对面的宋兵逼入了绝境。

折可适瞥眼一见,一副弓箭已经持在双手中。张弓搭箭仅仅是在瞬间,一支轻巧的箭矢从弦上飞出,掠过五六丈的距离,精准的扎进了阻卜骑兵的眼窝中。

正想拽着眼前的人一起上路,这名阻卜战士便感到眼眶中一阵剧痛,半张脸都麻木了,面前的视野黑下去一半。还没有反应过来究竟是出了何事,就看见前面正被他穷追猛打的敌人,挺直了腰杆,挥舞着手上的利刃,一步冲到近前。身子一下变得轻飘飘的,再也感受不到任何重量。

雄壮的身躯轰然倒地,劫后余生的士兵又是一刀上去确认敌人的生死,而后才转身过来,向着李瑛和折可适跪下来行礼道谢。

“好箭法。”李瑛转头大赞。战马上射箭,能在五六丈外命中眼眶,这等骑射的精准,足以让人震惊。

“献丑了。”折可适则是谦逊的一笑,将战弓收起。

折可适用的是短弓轻箭,向来只适合用来射击鸟雀。不需要太大的力道,却弓手却必须要拥有国人一等的箭术。

指挥使一级的军官,还需要冲锋陷阵的本事,但指挥使以上,正常情况下都只要在站旗下指挥全军。但弓马之技乃是武将的立身之本,没点水准,想在军中站稳脚跟就得大费周章。折可适的武艺从小就被家里逼着练出来。二十不到的时候,就是靠了一身的好武艺才得以让麾下的士卒信服,然后才能够在阵上立功,最后得到将种的美誉。

两人在战场上并辔而行,李瑛望着前方的村寨,“葭芦堡那边拦下了一伙贼寇,斩了一个叫乌八的贼酋。说是仅次于领军南下的西阻卜部族长余古赧。”

“幸好大鱼还在我们这里。”折可适笑道。

李瑛点点头,“幸亏如此。”

捉到的贼人多了,一些有关人员、人数的情报也就审问出来了。最大的一条鱼就在自己这一边,李瑛拼了性命也不会让这条大鱼脱钩逃了出去。

一支支阻卜骑兵,逃窜入在群山峻岭之间,但他们却无法脱离笼罩在头顶上方的巨网。

分散逃离的贼寇,就像是泼到沙地里的一盆水,不停的消耗在干涸的沙砾中,仅仅一天的时间,还算完整就只残留下区区数百人的残部。

剩下的,就是漫山遍野的散兵游勇。分散在山林中的保甲乡兵,正在为一个首级五匹绢的报酬,对他们紧追不舍。

……………………

已经是日暮途穷。

八百残兵困守在一座村庄中,四面的道路全都被堵上了,就是想突围,也没那么容易。

胜败只在转瞬之间。

前日此时,人人意气风发。而今天此时,则是各个垂头丧气。

而且许多熟悉的面孔都不见了。余古赧看着自己帐下的一众部将凋零殆尽,心中就痛苦难言。而且剩下的人们,还不能同心协力,反而是开始互相攻击。

“图木同,都是你要往葭芦川这边来,要不然哪里会变成现在的这副模样?”一人首先向余古赧的智囊发难。

“谁能想到乌八他们会犯蠢,全都往葭芦川赶过来?”图木同是个瘦小的阻卜男子,看着就是武技不行,只能在头脑上下功夫。他为自己辩解,“要是在葭芦川这边,只有我们这一千多人,又听了我的话抢了一把之后就早早离开,宋人怎么会摆下这么大的阵仗?”

“现在不是说这些话的时候!”余古赧一声大喝,转头又道,“图木同,你说怎么办?”

图木同立刻说道:“还是投降吧。宋人已经将村子围得水泄不通。突围的话,不知要死多少人。”

“丢下刀枪,出村投降?把自家的性命交给宋人?”一名余古赧帐下的首领摇着头,“我可不干!大不了拼上一拼,冲到山里面去,谁都别想把我找出来。绕上几天,迟早能绕出去!”

其他一些首领也附和他的意见,投降的结果就是性命堪忧,在场之人,谁愿去相信宋军会宽宏大量,不念旧恶?将自己的性命交到别人手中,在场的阻卜人都不愿意,他们只相信自己。

“没人愿意手上的刀丢下来,但不丢下武器,宋人愿不愿意相信我们?”图木同反问道。“要是他们不信怎么办?”

余古赧想了一下,道“……宋人现在肯定是想保着盐州,只要我们答应到时候能反过来给党项人一刀,他们还能不愿意?绝不可能强求我们丢下手中的武器。”

“对!”屯秃古斯叫了起来,“我们还可以帮宋人!”

另一名首领也跟着叫道,“嵬名家从来都没出过什么好人,这一次用了那么大的酬劳请我们南下,还不是要买我们的命?到现在也只付个定金了事,能不能活着拿到还没付的那一部分酬劳,还说不定。保不准最后儿郎们死了大半,他们就赖了账,到时候,家里面还能派人出来讨账吗!?”

几名部将纷纷点头:“投靠宋人的好,还是投靠宋人的号。宋人有金山银山,仓库里面丝绸绢帛堆积如山,只要宋人能从手指缝里漏出来一点,就够我们吃上三五年了。”

“前面我都听着宋人再喊,一个脑袋五匹绢。”老古青懂一点汉人说的话,“一个婆娘才多少钱?要是宋人肯拿这份来买我这条老命,我早带着家里面的儿郎投过去了。”

“投降宋人,从他们手中赚钱!多带点财货回去给家里面的女人孩子。”

余古赧说道:“就是有人心想党项、契丹,不愿为宋人卖命。投了大宋之后,也方便找个机会就逃走。”

图木同无奈的看着这一切,就听见余古赧道:“谁愿意去村外联络宋人,跟他们说我们愿意降顺?”

……………………

棋盘上已经进入收官的阶段,黑子和白子看起来占的实地相差不大,不过白子稍嫌零散。连成一片的黑子,将白子分成三块。这在还棋头时,是要吃亏的。

这是两天来两人下的第九盘棋,韩冈以他拙劣的棋艺,却每一次都能将棋局拖入官子的阶段。但他的官子水平,却让李宪得以准确的将胜负差距维持在三四个子之间。

似乎就是因为仅仅相差三四个子的缘故,受到鼓舞的韩冈,成了为回本而不断砸钱上赌桌的赌徒。听着接连不断的捷报,然后一盘盘的将彩头输给李宪。

李宪对此无可奈何。一名文臣——而且是重臣——肯跟他这个阉人下棋,其实是给足了面子。如韩冈这样连皇帝的面子都可以毫无顾忌的驳回的重臣,若是请自己一起下棋,而自家还推三阻四,那就是给脸不要脸了。李宪可不愿意这样把本来能交好的对象给得罪了。

何况韩冈下棋又不像他岳父那般有名的浑赖,认输认得痛快,彩头给得也爽快。尽管只是笔墨纸砚等不算值钱的文房四宝,但其来源自韩冈,也算是弥足珍贵了。

不过李宪还是想早点解脱,这样实在太累人。

“相公、太尉。”一名满面风尘的小校被人领进了厅中,单膝跪倒,向韩冈和李宪禀报:“小人乃第九将郭军将麾下行走,奉军将之命,向相公、太尉禀报。昨日得令严防贼军主力沿河西窜,故而郭军将领军严守小红崖。今日辰时,贼军窜至此地,遭我官军迎头痛击。如今郭军将已将其围在了小红崖南三里处的大王庄中。”

“是前几日被他们洗劫过的大王庄?”李宪立刻问道。

“回太尉的话,正是那座大王庄。”小小口齿伶俐,朗声说道:“村中现在有大约八百贼寇。贼首余古赧已经派人出村,愿就此降伏官军。”

“就这么投降了?!”李宪的声音忽然变得尖利。

小校猛点着头:“回太尉,的确就这么投降了。而且阻卜人传出来的话还说,若能既往不咎,甚至还愿意听从号令,对西贼反戈一击。”

放松下来的微笑出现在李宪的脸上,他扭头对韩冈道:“这群阻卜贼寇当真是能屈能伸。”

“总算了了一桩事。”韩冈将手中的棋子往棋盒中一丢,站起身笑道:“这两天辛苦都知了。”

李宪摇摇头:“只是下棋而已。”

“就是下棋才辛苦啊!”韩冈哈哈笑道,“自知之明,韩冈还是有的,这两天可是让都知受累了。”

转过身,韩冈脸上的笑容转瞬收敛,对小校吩咐道:“你回去跟李瑛和折可适说,让他们转告村中的贼寇:放下武器,出村听候发落。这是我韩冈的要求。如若村中阻卜贼寇不肯从命,就给他们一个痛快。”

小校怔了一下,李宪在旁也疑惑的说道:“龙图,这事可以慢慢谈啊。有他们在背后倒戈一击,大败西贼也不是不可能。既然他们都有心投诚,何须如此强硬?”

韩冈声音冰冷:“釜底游鱼,没资格跟我谈条件!想活命,只有两个字——听话!”

