\section{第24章 自有良策救万千(上)}

王舜臣疑疑惑惑的走了。

送了他出去,韩冈回来就着水盆中的清水洗了洗手,将为伤兵换绷带时沾在手掌上的脓血洗去。一名民伕过来,将脏水端出去倒掉,又换了一盆净水过来。不仅是使用的清水不断更换,连原本肮脏污秽的地面也都给打扫了个干净。

“这一条绷带,要用滚水煮过才能再用。”韩冈捡起丢在地上、沾满脓血的麻布带,交给另一名民伕,又大声提醒营房内地所有人,“每一件的被褥衣物,还有换下来的绷带,都要用滚水煮过,放在阳光下晒干,才能再次使用,这是为防疫病留存在衣物上。还有营房中,也要每日清理一番,否则必生疾疫。”

才一夜功夫,韩冈还没在伤病营中建立一言九鼎的威信,大部分伤兵们对突然跑来照顾他们的韩冈,还有些莫名其妙。不过能得到苦盼不来的救治,他们的确发自内心的感激。同时,韩冈所说的话,也得到了所有民伕们的响应。人人喊着‘秀才公’,无不点头应是。

以朱中、周宁为首的来自成纪县的民伕们,现在都在伤病营中忙碌着。他们跟韩冈不同,韩冈服得是差役,有差事在身。而民伕们服得夫役,到哪里都是卖力气的。张守约有权留住民伕,却无权留住韩冈。

为了整修这段时间被损坏的甘谷城防,张守约回来后便立刻颁下禁令,禁止所有进入城中的民伕们再离开甘谷城一步,并将整修城防的决定上报给经略司,等李师中批准后,就立刻动工。

民伕走不得,韩冈不想走,两方一拍即合。民伕们早得韩冈指点,皆知这是难得的机会,整修城防是个苦活,饿肚挨鞭是家常便饭,而在伤病营中服侍人,虽是腌臜了一点,但总比吃皮肉之苦强。趁着动工令还没正式下达,韩冈把民伕们拉到伤病营,希图造成既成事实。不管怎么说,成纪县来的这些民伕服侍的都是受了伤的袍泽兄弟,张守约再无人情,也不会将他们调走,拉去工地卖气力。

韩冈忙得脚不沾地,心中却有一种一切尽在掌握中的痛快,‘王韶你不是不想举荐我吗?那我就找张守约!反正都是做官,文官、武官也没什么好在意。即便张守约不荐举我为官,爷爷在军中结下了这么大的一个善缘,看谁还能找我麻烦?’

能利用他人的时候就要利用到底,但依赖他人却绝对不行。自己决定方向,前途要靠自己。这便是韩冈一直以来身体力行的原则!

……………………

“韩冈一夜都在伤病营?”

听着亲信的回报,齐隽心中直犯嘀咕。照理说韩冈拿到回执后就该尽快回去覆命,张守约刚刚颁下的命令,只针对民伕,而不是衙前,韩冈要想走,只要把回执在城门一亮,便能出城了。怎么跑去伤病营去磨蹭着?

给韩冈平白捡了个大便宜,让齐隽心中不忿。他既然收了陈举的厚礼,就没打算再还出去。受人钱财,自要与人消灾。韩冈虽然已经拿到了回执,但只要他还没离城,自己就还有出手的余地。

齐隽非是只会在衙前身上盘剥的蠢人,他拥有寻找后台的眼光,还有对库中物资不动分毫的自制力,但要让他从韩冈身上分清楚运气和坚持,齐独眼却还没有那么出色的判断力。

所有能坚持走到甘谷城的队伍,本都可以捡到这个便宜,可最后就只有韩冈把握住了。机会随处都有,却没有不冒风险、不付出努力就能落到手上的。

“雷简在哪里?”齐隽不打算放过韩冈,自己本是找不到出手机会,可韩冈在伤病营的愚蠢举动让齐独眼看到了机会,“伤病营是他的事。”

齐隽的亲信犹疑不决:“雷大夫几个月都没往伤病营去了,有人帮他处置,他应该高兴都来不及……”

齐隽嘴角动了一下,似笑非笑。纵然是看不上眼的臭骨头,可是自家碗里的就是自家碗里的,给不知从哪里跑出来的野狗叼了去,哪条狗不会追上去、抢回来?天下事悉同此理,雷简何能例外?齐隽不信雷简能忍得下去。还有韩冈在伤病营中的所作所为,也是明摆着在指责京里来的这位雷大夫玩忽职守。

是可忍孰不可忍?雷简如何能忍?

通过雷简这个大夫栽韩冈一个暗害受伤将士的罪名,只要下了狱,不愁弄不死他!

……………………

当秦凤路军中有名的专治跌打损伤的游方郎中仇一闻,从安远寨被加急请到甘谷城,为几名军官治疗的时候,韩冈和他的民伕们在伤病营中忙碌着。快一天了,伤病营里堆积多年的垃圾都已运出去焚烧,该清理的秽【和谐万岁】物都打扫得一干二净。可就是这么长的一段时间,竟然没有一名有品级的武臣来探视伤兵,倒是普通的士卒和小军官们有人情得多,纷纷过来探望自己受伤的袍泽兄弟,看着韩冈他们忙碌,还会主动过来帮忙。

“朱中,你去甲十五床,照规矩把他的伤口给缝上!”

“喏!”朱中不习惯拒绝,韩冈说什么他就做什么。

不到一天的时间,韩冈已经将伤病营中的几条通铺,以及上面的铺位都编上了号,按着甲乙丙丁,一二三四排好,就算民伕们不识字,也都能数得分明。

朱中急急的跑到甲十五床,躺在上面的士兵是大腿上被刀砍伤,虽然受伤之后就做过急就章的包扎,但效果并不好。朱中几下拆开绷带,鲜血一下从伤口处涌了出来。经过十几二十人的磨练,又受过韩冈的指点,朱中至少学会了一点最基本的急救法。学着韩冈教给他的做法,用止血带扎紧,拿盐水清洗伤口,趁伤员被盐渍得麻木的时候,趁机用麻线缝合起来。

“多谢朱郎中,多谢朱郎中!”看护伤兵的一人连声谢着,不停的弯腰鞠躬。

活到四十多年,朱中还是第一次得到他人真心实意地感激,还被尊称为郎中,成就感油然而生,更加卖力的为受了伤的士兵们缝合伤口。

虽然只是医官中最低一级的翰林袛侯,尚没有品级,雷简在甘谷城的地位依然比较超然。他既不属于文官,也不属于武官,而是个不掌实权的伎术官,平日为城主等城内大小官吏和他们的家眷治病,打算混点军功和资历,再等两年时间就可以回到东京,游走于宫廷宦门。三十出头的医官,背下了满肚子的医术典籍,但其中没有一条是让他和跌打郎中比拼谁的医术更有效。

对于一名在战事中受了伤的副指挥使,雷简和仇一闻有着不同的治疗方案。军官不同于下面的士卒,自家在城内有宅,都是回到家里养伤,谁也不会去伤病营等死。王君万正好也到自己的副手宅里来探视,却看着雷简和仇一闻在那里争吵。

“用金针放出淤血,再敷上老夫特制的散玉膏。三五天就能还你个能走能跳的大活人。”

“不要看皮上的一片青,被铁简砸到背上,伤势已经深入内腑。放血有什么用?”

“又没有咳血,呼吸也不过促了一点,脉象稳得很,伤得哪门子内腑?”

“江湖村医也知道什么叫治病?!”

“嘴上没毛的黄口孺子也别出来让人笑了。”

一个是在秦州成名已久的老大夫,一个是来自东京开封的医官。他们的话,普通人也分不出谁对谁错。王君万的副手脸色蜡黄的,躺在床上看着只有一口气,副指使的妻儿则只知在一旁哭,王君万不耐烦了,一拳捶在墙上,怒道:“人都快死了,还争个什么?!”

“胡说什么!?”仇一闻在秦凤路上资格极老,许多老军头都承他的情。倚老卖老,也不怕王君万这后生,“别看着现在这般模样,不过是重一点的皮外伤,折了的两根骨头都已经对好了,拖半个月都没事!”

“你才是胡扯!”雷简再次跳出来反驳,“伤及内腑,不急加调理,最多四五天!”

王君万给烦得不行,暴怒道:“那就两样都治!仇老你放血,雷大夫你用药。一个内服,一个外用,也不会干扰。人治好那就一切无话,人治不好……你们给洒家等着!”

王君万丢下狠话走了,仇一闻和雷简便是一通忙活,一个开药方,一个施针敷药,虽然争了半天,都指责对方是庸医,但他们的治疗却颇有效验。扎了针,喝了药,骑兵指挥的副指挥使脸色便好了许多,呼吸也平稳了下来。

“看,老夫说得没错吧?放了血就好了。”

“那是喝了本官药的缘故!”

仇一闻和雷简在副指使妻儿千恩万谢中出了屋,犹自争论不休。一人突然在他们身后出声,“两位要争个高下也容易,城南就是伤病营,你们将伤兵各治一半,看谁的救下的人多,高下不就分出来了?”

ps:在北宋,有医术可以做官,但光有医术却很难做官,何况韩冈也不懂医术,只有一张能说会道的嘴。猜猜看他是怎么力压两位名医,在甘谷城里混出头的?

今天第一更,照例求红票,收藏。

