\section{第四章 惊云纷纷掠短篷(五)}

将近黄昏的时候,审官西院衙门终于变得清静起来。

来往的人流稀稀落落,只有提前一步回家的官吏脚步匆匆。隔壁御史台的乌鸦在叫着,给暮色下的宫院,平添了一分萧瑟。

审官西院负责大使臣的考课选任。横行以下、小使臣以上的中阶武官——大略是正从七品的诸司使、诸司副使——他们的铨选和考核,都是由审官西院统管。

虽说比起管理低阶武官的三班院,在审官西院候阙的武官人数要少上许多,诸司使、副使们轮不到一个好差遣的几率也小得多。但毕竟是主管人事的衙门,寻常时便是人来人往,仅仅是不会争先恐后而已。

“快打申时三刻的鼓了吧?”叶涛有些不耐烦了。他和沈铢已经约好了去喝酒,就等着鸣鼓放衙。

“今天是晚了一步,让陈三、李九先走了。我若是再一走,李判院面皮须不好看。只能等暮鼓了。”

沈铢是审官西院主簿,不过他还兼着国子监直讲一职,与他对坐约同喝酒的叶涛份属同僚。而且两人还是亲戚。沈铢之父沈季常是王安石的妹婿,叶涛更是王安国的女婿。但他们两个跟另一位王家的女婿却没有什么来往。

叶涛毫不避讳的翻着沈铢桌案上的公文,随性问道:“伐夏的将帅已经定下来了?”

对于叶涛乱翻写满了军国机密的文件,沈铢视而不见,完全没当回事,“到今天才定下来。河北和京营的将帅多少人都争着要去陕西,要不是王相公坚持必须由经过战事的将校统领,还不知道要拖到哪一天。”

“那些个武夫,眼里就只有杀人放火博功赏。”

“谁说不是?但争到最后,还是从东京调了七个将三万九千步骑去陕西助阵。”沈铢道,“王相公也不敢将京营开罪得太狠。”

“三旨相公能有多大胆?”叶涛冷笑了一声,随手就拿起了一份公文来看,“还是王中正领熙河兵马、高遵裕领泾原、种谔领鄜延?”

“这三人自然不会变。”沈铢将手上的公文一边翻一边签名画押,“王中正统帅熙河秦凤两路兵马;高遵裕是环庆兵马副总管,领一路兵马,而苗授权摄泾原、听命于高遵裕;种谔在鄜延;李宪不及王中正,战绩差了一点,但在征伐交趾的时候也捞足了好处,领着高永能和折克行出兵河东。六路齐出,合攻西虏。”

叶涛丢下了手上的公文:“三十万大军,可号称百万了。”

“秦凤、熙河共计五万步骑加三万蕃军;泾原五万;环庆路是高遵裕统领,他把南面永兴军路【长安】的兵都要到了手底下,总计八万七千步骑;鄜延本属有五万五、京营的七个将也一并归入种谔帐下,几近十万;至于河东,加上折家的一万,则是出兵六万。”沈铢如数家珍一般,将各路出兵的兵力向叶涛报上:“你说总数多少?”

叶涛屈指心算了半天:“这不快四十万了。”

“嗯。”沈铢点头,“总计三十五万正兵。后面还有差不多同样数目的民夫,十万余牲畜,两万余大小车辆,为大军提供粮草。”

叶涛随手又拿起另一份公文,漫不经意的问道:“差不多一百万张嘴,谁管得过来?!”

“秦凤和永兴军两路转运司统辖。鄜延、泾原、秦凤、环庆四路权置随军转运司。加起来看着是多,可各路归各路,总不至于会饿死。”沈铢左手一握拳,道:“六路并进,当能一举灭贼。”

叶涛都没听到沈铢再说什么,他看着手上的公文,惊讶得张着嘴:“这个赵隆是前两年跟着王中正那个阉宦去蜀中的赵隆吧?怎么都升到了东染院使,领熙州州务了!我看他这家状上,年纪还不到三十!”

“王中正好福气,是福将,跟着他,当然有前程。”沈铢抬头看了看叶涛拿在手上的公文,就冷笑,“记得种谔之父种世衡,当时号为名将,在关西与狄青并称,终其官,也不过一个东染院使。”

叶涛从眼睛里透着羡慕,但撇下的嘴角好像是在不屑,“名将打了一辈子的仗,都不入横班。小小一个敢勇跟对了人,偏能鸡犬升天。”

“也是命数。”沈铢道,“种世衡的命数不及狄青,也不及他的儿子。”

“说到有福,王中正还真是福将,好像就没败过。”叶涛又道。

“败过一次,是当年进筑罗兀一役。”

“那不关他的事吧?”叶涛反问道,“不是说本来就要撤军了,只是被梁乙埋领着十万党项军咬住,没办法脱身。可王中正去了之后,就平平安安的回来了,还得了一个斩首数千的大捷。”

“所以说是命数啊。”沈铢摇头叹着,“韩子华攻略横山,他奉旨去罗兀城,正好给他撞上了,天子说他是为国不惜己身。到了河湟开边,王韶、高遵裕失去音信,韩冈硬挡着圣旨,王中正帮了韩冈一把,最后王、高回师,又得了一个勇于任事的评价。而后平了茂州之乱,便被称为内侍中知兵第一,跟着去了交趾的李宪都不如他。”

“谁说不是呢?”叶涛不知想起了什么,深有感触的叹着,“王中正真的是运气好。去年福建剧盗廖恩作乱,官军几次围剿不得。小弟乡贯龙泉,家中正好受廖恩之扰,福建的几十个巡检司的巡检、都巡检,全都引罪去职。最后天子没办法,钦点了王中正去领兵平乱。谁想到刚刚抵任,廖恩就归降了。”

福建近年出了个剧盗廖恩。说是剧盗,也就百来名喽啰而已。若在陕西,一个巡检带着土兵就能给灭了。可换作是兵力不振的南方,福建一路都给闹得地覆天翻。最后路中实在奈何不了他,只能奏请朝廷发兵。天子遣了王中正去。当时还有人反对,谁想到王中正领军方至,廖恩就立刻跑来投降了。

没打上一仗就赢了,当然不能说是王中正的能力出色,叶涛也不觉得是王中正的名声有多大,将廖恩给吓得跑来归降,分明是老天帮忙,让王中正捡了个大便宜。

“对了。”沈铢放下笔,“说到廖恩,这两天从三班院传来一个笑话。”

“什么笑话?”叶涛将赵隆铨叙公函丢到了一边,很有兴致的问着。

“廖恩不是降顺了吗?所以他便被授了官职。今日来京中三班院缴家状,好得个差遣回去。”

叶涛嗤笑一声,“得了官身,也是个贼。”

“致远你是知道的,家状的文字立有定式。廖恩的家状是这么写的,‘自出身历任以来,并无公私过犯’。”

叶涛顿时放声大笑起来,声震屋瓦,连声道:“好个‘并无公私过犯’,好个‘并无公私过犯’!”

沈铢没笑,他摇头,“这还不算好笑。跟廖恩同时在三班院缴家状候阙的官员还有不少,其中就有一个出身福建的。你可知他递到三班院的家状是如何写的?”

叶涛笑声收止,擦了擦笑出泪水的眼角,“是怎么写的?”

沈铢双手抓起桌上公文,装着在读:“‘前任信州巡检,为廖恩事勒停。’”说着便忍不住笑,“两人一前一后,同在一天都来三班院等差事,致远,你说此事可笑不可笑?”

叶涛这一次却没笑了,摇头叹道,“官亦官,贼亦官。官即是贼,贼亦是官。”

沈铢收起笑容,将纸笔一丢,叹道,“如今两府诸公,可都不在乎这点小事。”

正说着,就听见外面的暮鼓声响起,终于到了下班放衙的时候了。

沈铢和叶涛随即起身。沈铢先去了正厅,与审官西院众僚属一起向两位判院行过礼,便和不耐烦的叶涛一同向外去。

沈叶二人急着离开,脚步匆匆。走在两人身前,还有一个个头不高,却健壮如磐石的身影。

那个矮子身上的衣服并非官袍,在皇城中,就是亲王也得好端端的穿上公服,只要有官职在身,没人能微服而行。一看就知道是个没有官职的布衣。但几名武官一见到他,不是立刻让到一边,就是上前问好。

趁着那人和几名武官停下来说话,叶涛和沈铢超了过去。

在擦身而过时,叶涛用眼角瞥了一下,是个满面虬髯、相貌有几分狰狞的汉子。但围在那汉子身边的几名将校,却无一例外的有着一副带着几分谄媚的笑容。

向前走了十几步,叶涛方低声问道:“那是谁啊?”

“致远应当听说过他的名号。”沈铢顿了一顿,“是大名鼎鼎的王舜臣啊!”

“就是那个杀良冒功,被夺了官职的王舜臣?”叶涛忍着没回头:“想不到还有这么多人奉承!”

“听说当年韩冈微贱之事,遭逢厄难,是他救了韩冈一命。而且眼下他还是种家的女婿。与王中正和高遵裕都有几分交情,在王韶、章惇面前也能说得上话。要不是有这些靠山,以他谎报、杀良、欺君的罪名,十个脑袋也该砍了。”

叶涛顿时愤然:“这等庸鄙武夫,不依律处断、以儆效尤,已经是朝廷的宽贷了;竟然还敢呼朋唤友的出没于审官东院中,真当三尺剑斩不得他!?”

沈洙报之一笑,“武夫不就是如此,贪功好利,还能指望他们清正廉洁不成?”他笑了一声,“这边一个犯事被夺官的已经进了京,过几日还会有另一个犯事被夺官的也要进京城了。”

“苏子瞻?”叶涛胆战心惊的转头望了望不远处的乌台,门前的槐树上,一群乌鸦正在盘旋,“算了,不提此事了。不要让龚深父【龚原】久等。”

“恐怕深父兄当是急了,耽搁到了现在。”沈铢加快了脚步,“国子监里的事,今天得商议个对策出来,总不能任人摆布。”

