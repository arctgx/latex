\section{第32章 吴钩终用笑冯唐(12)}

“说话算数?……吴逵是这么说的?”韩绛问着。

“吴逵正是这么说的。”陆渊连忙点头。

他虽然被吴逵小瞧了,却也不敢将吴逵让他传的话有丝毫隐瞒和扭曲。城里有几千张嘴,吴逵和他的对话根本瞒不住,若是他敢扭曲半点,事后一旦暴露出来,等着他的就是枭首一刀。

可是这个营帐中,担得起‘说话算数’这四个字评语的也就两人——韩绛、赵瞻。

另外种谔、燕达两个副总管勉强也能搭点边——好歹可以被称为太尉了——至于其他人,那都是听候使唤的宣抚司僚属。他们说出的话,只要几个大佬不点头,那都不算数。

只是韩绛自是不可能纡尊降贵;种谔和燕达乃是一军主帅,当然也不能去;所以最后就只剩下一人,二十多道视线便齐刷刷的往赵瞻看过去。

赵瞻脸色微变,他从没想到自己也会有去劝降叛军的一天。不过他也不是胆怯之辈,在这里退缩了,他脸上也挂不住。一扬脖子,就要站出来自荐。

“此事万万不可!”先一步跳出来的却是种谔,他急声道:“赵郎中乃是天子使臣,代天巡狩,岂有屈从叛贼之理!”

种谔这话说的是没错,叛将吴逵一句话,就要让赵瞻这位天子使臣跑去咸阳城里,这朝廷的脸面丢不起。

可种谔并不是要为赵瞻解围,而是他和韩绛还想把今次横山之败的罪名让赵瞻分担一点。要是让赵瞻出面劝降成功,这些盘算就只能留在梦里了。无论如何,都要把赵瞻撇到一边去。

“吴逵故意刁难,分明是无意降伏。”

“相公,不如直接打吧。末将可立军令状。”

“末将也敢立军令状。城墙今天都已经砸塌了一块,明天就能破城。”

种谔起了头,下面的将校也纷纷表达自己意见。自己得不到的功劳,也没必要让其他人得了,干脆拉倒。反正今天都看到了新型投石车的威力,比起旧式样,强出百十倍。用几天时间,造出个百八十具,一口气把咸阳城的一圈城墙都砸烂掉,看吴逵怎么办?!

可韩绛不去理会他们。他沉声对陆渊道:“陆渊,你把你跟吴逵的对话,从头到尾的说一边来听。”

陆渊听了吩咐,不敢有丝毫隐瞒,一五一十的将他进城后,跟吴逵的对话全都说了出来。

听完之后,众人的目光重又聚集在一处,只是这一次,他们看的不是赵瞻,而是站在班次最后的韩冈。

‘说话算数’有两种解释,本来众人都是以为指的是为高权重、说话有分量,但现在看来,吴逵却是想找一个说话算话的至诚君子。

结合起吴逵前面与陆渊的一番对话,最后说话算数的这四个字,当是着落在关西军中名声最好的韩冈身上。

众人的目光灼灼,韩冈被刺很不舒服。他暗叹了口气,想不到这招降的任务,终究还是着落到他的头上。

韩冈无意跟在列的众将争夺功劳,但吴逵既然指了名,他也不好不出头。要不然那就真的要开战了。若是这一战中城中百姓伤亡过大,他韩冈可是脱不了的罪名。加之为了那三千叛军,为了能充实河湟地区薄弱的汉人势力,他都是得去咸阳城里走上一遭。

韩冈迈步出列,向着韩绛行过礼,道:“说话算数,韩冈绝然当不起。但息兵销灾,使咸阳百姓不受兵燹之苦,韩冈何敢推却?当把朝廷的恩典和相公的宽大,传与城中叛军,让他们束手而降!”

……………………

入城劝降的人选定下,军议便宣告结束。不过韩绛把韩冈留了下来,接下来韩冈要去劝降吴逵,依理也该吩咐一番。

韩冈垂手而立,等着韩绛发话。

韩绛看着他过于年轻,却沉静稳重的面容,沉默了很长的时间。

韩冈并不是王文谅那种会溜须拍马、招上司喜欢的性格;只看那对锋锐的眉眼,就知道他绝不是甘居人下的脾气;不论是对自己,还在京城对雍王,又或是这两天对上了赵瞻;都可以看出韩冈宁折不弯的性子——一个标准得过了头的士大夫。

刚硬起来,不给任何人脸面的脾性,韩绛说不上多欣赏,如果真的碰上,最多也是为了维护自己的形象,才会赞上两句。但韩冈不同于一般的士大夫,他有过人的才能,如果能善加使用,总能带来最丰厚的回报。

而对于韩绛来说,或者是对每一个上位者来说,溜须拍马的手下当然也要有一两个,但能给自己带来足够利益的僚属,才是他们最为倚重的。

韩冈才智胆略皆过人一等,早前累累功绩就不说了,在罗兀城的事也不需多提,光是他到了平叛的第一线,才几天工夫,就轻轻巧巧的帮着自己解决了大问题,让自己不再焦头烂额;又在兵械上有所开创——新式投石车对军中的意义绝不下于神臂弓。

如此人才,世所罕有。

而且最关键的,是韩冈懂得投桃报李,并不是忘恩负义之辈。他得到王韶的荐举,便用心于河湟之事。为了让空虚的通远军,多上三千户汉人,他可是不顾身份低微,而出头建言要保下这三千叛军。

“王韶真是有福啊……”韩绛忽然叹了口气。

韩冈没料到他等了半天,却等来这一句话。抬眼看看韩绛,明白了他的心意。

但韩冈并无意改换门庭,并不是他对王韶有什么忠诚,而是他对自己的事业忠诚,对自己选定的道路忠诚。

他也不怕韩绛会因此恼羞成怒,他知道韩绛看重自己,是因为他能给韩绛带来足够利益。

为什么韩冈自转生后的一年多来,每每都能得到看重,并非是他才高八斗,也并非他有积淀千年的知识,而是他在关键的时候,都能给人以助力。无论王韶、王安石,还是现在的韩绛,韩冈没少为他们献计献策,出力流汗,这样的人才如何会不被重用?

至于他一心于河湟,那可是加分,这个时代士林的风气,也在鼓励这样的行为。

所以对于韩绛委婉的招揽,韩冈也便保持沉默,仅仅是弯了弯腰,表示自谦而已。

韩绛叹了一声后,韩冈的态度并不出他意料。韩冈对王韶忠心耿耿,当不会为了一句话而改换门庭。但眼下能给自己带来惊喜,这也就足够了。

“玉昆,以你的才智胆略,多嘱咐你也没有什么必要了。只望你能多加小心,安然回返便是。”

“多谢相公关心。韩冈必不负所托。”韩冈拱了拱手,说着。

韩绛微一沉吟,却又不厌其烦的叮嘱道:“吴逵是个聪明人……他不会做蠢事。”

韩绛的多话,让韩冈更加确认他对自己的示好之意。而韩绛的话中隐义,韩冈也点头表示同意

——吴逵当不是甘心就死的人。

吴逵对陆渊的一番话,摆出自我牺牲的姿态,让下面叛军对他感恩戴德,如果接下来的使臣说错一句话,三千名被围在咸阳城中,本已经开始动摇的叛军,很有可能就跟吴逵一条道走到黑。

不过吴逵能用言语做到的,他韩冈也是有一些自信。以名声论,他韩冈也不算差,论口才,他韩冈更为出色,而说起透析人心,吴逵可是要瞠乎其后。

……………………

月色微明,咸阳城的城头上点起火炬,一条光带绕城一周,照着城墙顶端一片晕黄。

吴逵静静的盘膝坐在咸阳南门的城头上,远眺渭水,听着若有若无的水声。七尺长的铁枪横放在腿上,右手紧紧攥着枪身,从冰冷的铁块中,传来夜色的清凉。

夜风习习,从他背后吹来,带着清淡的桃花香,让人忘了眼前烦忧。咸阳城中多有桃花,在二月中旬的春风中渐次开放,到了三月初便为极盛,直至三月中旬,方才凋零殆尽。

每年的这一个月的时间里,城中总是花香浮动,片片花瓣随风而舞。几处名园之中,更是灿烂如锦,游人如织。

吴逵曾经在咸阳住过不短的时间。他年少风流时,也曾呼朋唤友,携妓而游。虽没有文人吟风弄月的风雅,但也纵酒高歌的癫狂,醉后论兵的豪放,也不输于那些措大。

只是一切都随时间远去,就像城外的渭河水,再也追不回来。唯有掌中这杆纹理沉黝的铁枪,才是几十年不变跟随着自己,给他带来一丝微不足道的安全感。

“都虞……”来自身后的轻声呼唤,打破了吴逵身边的宁静。

吴逵回过身来,见着是自己的亲卫。“是外面的官军又遣人过来了?”他问道。

亲兵躬身回话,“回都虞,是秦凤路的韩机宜。”

吴逵呵呵的笑了起来:“果然还是韩玉昆。”

他一转枪身,当得一声响,用力杵在了地上。扶着枪杆,霍然长身而起,“走,就去见一见韩玉昆。看他带来了什么好消息!”

