\section{第五章 冥冥冬云幸开霁(九)}

韩冈到底在想什么?

李定完全无法理解韩冈的想法。

从情理上说,韩冈放弃了与残存的几位宰辅共商国是,而选择将诸多重臣一并拉入了崇政殿,这应该有借重他们的地方。

李定一开始便觉得,韩冈肯定是有用得到自己的地方,才会这么做。

李定不觉得韩冈是因为没有自信与宰辅们分庭抗礼,才会选择拉重臣入伙。

韩冈需要借助一应重臣之处,理应是想进一步强压下宰辅们一头才对。

但自开始议论如何处置叛逆,韩冈都是站在了宰辅们的一边。甚至是在引导话题,带动两府宰执来压制所有反对者。

难道除了这件事外,韩冈还有别的地方需要自己帮忙?可现在不协调一下,待会儿能联手起来?

而且依今天的情况,若韩冈在某件事上坚持己见,宰辅们多半会选择退让。就是要保天子之位,也是一样的结果。

亲眼看过他一锤击毙蔡确之后,就是王安石跟这位好女婿说话,恐怕心中也得带几分颤。而与他曾经交恶的一干朝臣,更是得多谢不杀之恩。

李定有自知之明,别看现在能顶着韩冈和众宰辅,只不过是仗着人多,能互相壮壮胆子。加上韩冈本身只是引出话头,主要还是交给了王安石、韩绛、章敦他们。

韩冈根本没必要多此一举。

看不透。

拥有多年为官的经验,李定依然看不透韩冈的打算。

无欲则刚。

没有任何欲求的人,是最难对付的。

而有所求却让人完全猜不到目的,这样的情况,一样让人觉得棘手。

李定忽的哑然失笑。

韩冈的目的迟早要暴露出来的,保持耐心,等到他图穷匕见。

至于现在的情况,没必要再去与宰辅们顶撞。

韩冈给出了最好的理由,为京中人心军心计,权且饶了他们一命。

从李定的角度来说,留下苏轼一命才是好事。

从逆之辈,就是能逃过一死,也必然是毁废终身。

在御史台中,李定看多了一心求死,恨不得一死以求解脱的罪囚。许多时候,活着反而才是最痛苦的一件事。

看着死对头一辈子都不能再出头,终身都要被人监视,日夜不得安寝。子孙沦为贫贱之辈,有宋一代,也不会有重回士人行列的机会。这比直接活剐了那位老对头,更要让李定痛快一百倍。

连御史中丞李定都沉默了下去,其余重臣更难有立场说话了。

殿中静默了下来,向太后看了看韩冈,又看了看几名宰辅,问道:“依各位卿家意。那些叛逆究竟该如何处置。”

王安石道:“四名主犯之中,蔡确、宋用臣、石得一已死,暂不论。赵颢立刻赐死。曾布、薛向追毁出身以来文字,籍没家财,流放远恶军州,阖门皆如此。不过为定人心,不追支族、姻亲。”

王安石的处理意见听起来很宽厚,朝廷将不降责蔡确等叛党的亲族,但他们在官场上的前途,基本上已经宣告终结,而姻亲,都得以离异告终。可谓是终生不得翻身。

“那苏轼、刑恕,还有那些叛党呢。”

“交由法司审问即可,依律定罪后,太后再行赦免。”韩绛跟着道。

“怎么?不直接判了?”向太后问道。从声音中,听不出这是质问还是疑惑。

王安石低眉垂眼,完全不去猜测向太后想法:“太后既贷曾布、薛向死罪,朝堂内外当知太后仁恕之心。那些叛逆余党纵有人还心存叛意,也不可能再蛊惑不了人心。不必要越过法司。”

停了一下,让太后消化这段话,王安石才接着又说道,“事有经权。经者为常,权者为变。曾布、薛向不经法司定案,便蒙太后之赦,已是权变之举。而其党羽、走狗,就没必要在破坏朝廷的法度,当依正常的流程来。”

“孝骞怎么处置?”向太后沉默了片刻后,又问道。

赵煦的儿子都坐上了御座,这可不是可以一带而过的小事。

“孝骞年幼无知,无罪。”王安石却如此回覆。

赵煦都无罪,孝骞也必须无罪。都一样是不懂事的小孩子,没有道理弑父之罪能当做没有,而篡逆之罪就得论死。

“不过因其父之罪,当宗籍上除名,废为庶人。”韩绛跟上去补充,“此乃赵世居旧例。之后送至南京看管,或流放亦可。”

“……韩卿可有意见?”

越过了韩绛之后的章敦等宰执,向皇后向韩冈征求看法。

“王、韩两相公如此处分,臣无有异议。”韩冈回复道,“事不宜迟,臣请太后速速下诏,公诸于世。”

……………………

内城诸门都已在控制之中。

各门先后派回来的信使都向李信作了汇报。

城头上,一只只警惕的眼睛正监视京城各处的军营。

而从军器监中携带出来的火炮,也随时能推上路口。

“若有贼人敢于上街作乱,杀之勿论。”李信杀气腾腾的命令,从朱雀门传到了每一座内城城门中。

除了朱雀门外,其他城门都只有半个都炮兵。连副都头、十将、将虞侯等军校在内,共计五十余人。按照预定的编制,当有八门虎蹲炮,不过现在基本上都只能分到三到四门。

唯有朱雀门的火炮最多。

十二门轻便的虎蹲炮和一门带炮车的野战炮,就安置在大部分时间都紧闭着的主城门门洞内,守住了朱雀门的正门。

州桥上人头涌涌,纵然是丧期之中,亦是开封府中最为繁华的区域。而门后的御街,也同样是最为热闹繁华的路段。

但只待李信一声令下,不论是城内御街,还是城外州桥,只要有人敢于冲击城门,立刻就会被蓄势已久的铅弹打成肉酱。

李信并不担心蔡确的余党。

宋用臣安排在赵颢家中的班直是不是全体被策反,现在根本无法确定。但其中的首脑必然是参与叛乱的从犯中的一人。所以必须要王厚亲自领军去围困。

可蔡确、曾布、薛向这样的文臣就没什么好担心了。

他们能发号施令,甚至让将帅们闻风丧胆,是他们所拥有的官职在起作用。给与他们权力的是体制,是规矩。失去了体制的保护之后,他们家里的仆佣,都没几个会跟着他们一起走。

大宋的历代皇帝之所以不担心文臣的原因就在这里。

文臣们不论多么权势煊赫,一旦失去了官职给他们的地位和权力,就只是个连鸡都杀不了的措大。而那些领军的将领,多多少少也有十几几十个能为其出生入死的亲信。

蔡确已死,其主要党羽皆已就擒。剩下的还有一些杂碎,根本不足为虑。

李信最担心的还是皇城司在京城中的余党。

既然能够走街串户的打探消息,当然也能够在京城中掀起动乱。

纵火烧屋,散布谣言,甚至当街砍杀,都能让京城中一片混乱。

随着暮色将临,京城各方已经得到了政变的消息,而他们的反应也会即将浮上台面。

这叛乱后的第一个夜晚,是最为关键、也是最为难熬的一个关口。

李信只希望叛逆的余孽们在群氓无首的情况下,再犹豫一阵。等到明天天明,朝廷宣谕四方,侦骑四出,贼党一份机会也不会有了。

开封府,左军巡院,右军巡院,旧城左厢公事所,右厢公事所,新城左厢公事所……

李信默默数着京城中掌握着人力和兵马的关键衙门,其中有没有人被收买,又有多少被收买,也许只有已经死了的石得一最清楚。

李信对此鞭长莫及,守住城门已是他手中兵力的极限,剩下的就只能依靠开封府的知府了。

李信不知道表弟韩冈现在有没有出宫,但权知开封府沈括则已经出宫来。

希望沈括能够尽快腾出手来,控制住京城内外。

……………………

曾布等人的处理意见,韩冈没有多话。

向太后几次向他询问,但韩冈只是就事论事的应答,其他都任凭王安石和韩绛来处置。

章敦那边一直在闪着狐疑的目光,李定的神态也跟章敦几乎一模一样。

韩冈明白,怀疑他用心的不在少数,每个人都在猜测他的想法。

但韩冈的确没得有太多的想法,只是觉得过去宰辅们议论,然后交由太后、天子决定的做法不太好。拖些人进来,情况最坏也只会是维持现状,情况好点,可就会向韩冈所希望的方向转变。

要想对抗皇权,宰辅们必须要齐心,树立起一个共同的敌人就是关键。

韩冈将重臣拖进来,至少有一半是抱着这样的想法。

尽管这会让重臣们看起来地位大涨,但实际处理政务时,宰辅们手中的权柄可以轻而易举的压制住,就是御史台也一样。经过这一场变乱之后,宰辅们的地位稳如泰山,御史台纵然不听话,也不会有实际上的影响。

而且一旦重臣共议成了惯例,不论是谁,就会去设法调换上听话的党羽。

韩冈希望宰辅们从此之后能够主动去揽权。

韩冈没指望能心想事成,在他而言,能成功最好,不能成功那就再想办法。

还有十年的时间,自家又不打算再‘高风亮节’下去,他有足够的信心将所有宰辅都领上想要他们走上的道路。

也许还要很久才能让这些同僚们明白,但韩冈依然有信心让他们明白

失去的只是枷锁。

得到的将会是整个世界。

