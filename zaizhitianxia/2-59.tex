\section{第13章 不由愚公山亦去(四)}

【夜里还有一更。继续求红票,收藏。】

已是五月末,真正的盛夏已经降临这片大地。热浪铺天盖地,稍远一点的景物都在晃动的空气中变得扭曲起来。树上的蝉鸣也听不到了,这般热的天气,就算蝉虫都受不了。连黄土夯筑而成的路面也变得白得发亮,反射着热辣辣的阳光。路边干燥的草木,大概只要一点火星,就会燃烧起来

秦州已经多日没有下雨,藉水河面比他们上京的时候,低了有两尺还多。王厚侧头看着河水,旁边的赵隆凑过来,一起望着再低一点就能看到河底的水面,就听王厚叹道:“若是江南的河水如藉水一般,那今年的收成就全完了。”

“王官人说的是。幸好关西这一片种得都是冬麦,现在地里只有草,没有粮,也不怕不下雨。”

“王官人?”王厚转回头笑道,“那我是不是要唤赵子渐你作赵官人?”

“不敢,不敢。”赵隆连声自谦,但看他一脸满足的表情,却是明显的在说着‘多叫俺几声’。

王厚、赵隆,现在都已得了官身,理所当然的是王官人和赵官人。而且在回程的时候,又听说了古渭大捷的消息,两人现在的心情,比任何时候都要轻松。

王厚、赵隆今天都换上了青色的官服,虽然已经被汗水湿透,但他们都是毫无觉察到样子。早点回到秦州,好好炫耀一番的想法,充斥在他们的脑中,全然忽略了外界的炎热。

“会不会有人来接?张钤辖和王都知都一起回来了,李经略也该出城相迎吧?”离着秦州越来越近,赵隆又憧憬起空城相迎的场景。

王厚当即泼了盆冷水:“不可能的,王都知和张老钤辖都没派人通知秦州。怎么会有人出迎?”

赵隆回头望了望跟在他们身后的车队,一辆马车被护在队伍中央,李信和一众护卫围在马车周围。安坐在车内的,就是两人所说的张老钤辖和王都知——新任的秦凤路钤辖张守约,以及奉旨往秦州宣召的入内副都知王中正。

张守约却是老了,一趟长程的旅行消耗了他不少的精力,没有在夏天烤火的心情。躲在马车里,跟着细眉小眼的王中正对坐,有一句没一句的闲聊着。

张守约自京中走得比王厚要早,但他经过京兆府时,被陕西宣抚使韩绛强留了两天,向他询问秦凤军情。这一耽搁,便被王厚和赵隆从后面赶了上来。

而王中正奉旨出京,走得比王厚还要迟上两天,但他一路快马加鞭,也是在过了京兆府一日路程后,与张守约、王厚碰上了面。

追上了张守约和王厚,王中正便不再紧赶慢赶。他的心中也有计较,刚出京,人还在京畿的时候,走快点代表自己忠于王事。但入了关中后,急着往秦州赶,却会给人一种他迫不及待要把人逐出秦州的感觉,这样太得罪人,当然要走慢一点。

各自有着各自的心思,三拨人马便合作一路,一起向秦州进发。

昨日一行人在陇城县歇息,王中正并没有让人先一步通知秦州。还是那句话,这么做太得罪人。如果宣召使臣手上拿的是擢升的诏书,当然会早早的遣人通知过去,但如果是降罪、免官的诏书,却不会事先通知当事人,有怕罪臣畏罪潜逃的用意,也有怕强迫遭贬官员出迎会留下怨恨的想法,这也是多少年来不成文的惯例。

王中正今次来秦中,手上的几份诏书并不是发给一个人的,有人会喜,有人会悲,所以干脆都不知会。而张守约老于世故,对朝中惯例也是极熟悉,当然不会让王中正为难。

就这么平平静静的一路进了秦州城,一行队伍往秦州州衙行去。可是到了城中心的州衙前面,却见着数百名百姓不顾暑热的围在州衙大门口。

王中正听到通报,掀开车帘一看,便大吃一惊,“出了何事?!”他急问道。

张守约下了车,花白的双眉蹙着,也是百思不得其解,他见那群百姓安安分分,不像是来闹事的样子。

李信受命去打探消息,转眼就回来了,“回禀钤辖、都知,是窦副总管的孙子窦解犯了事,李大府正在衙中审问。外面的都是苦主,来听消息的。”

“窦解……”王中正的声音一下小了起来。

李师中和窦舜卿的关系,王中正是知道的。李、窦二人在秦州是联起手来跟王韶为敌,一顷和万顷之争也在朝堂上掀起了轩然大波,两人可以算是盟友。可今次窦解都押上公堂,被李师中亲审了。

如果不是李师中跟窦舜卿翻脸,那么窦解的罪名绝对小不了,罪证也肯定是明明白白,使得以秦州知州的权力都压不下去。

“都知,你看如何是好?”张守约随口问着。

王中正宣旨之事与他无关,职位已定,赏赐已收,用不着旁听、旁观。他现在最应该做的是回他在秦州城中的私宅休息,顺便等人上门拜访恭贺。等向宝要走了,他再出来做个交接。张守约也准备这么做,只是他与王中正一路同行而来,在告辞前,还要先问上一句比较有礼。

“钤辖请自便。”王中正知情识趣的回了一句,又抬眼看着衙门前的拥挤的人群。

他代表天子而来,自是要在州衙大堂上宣诏。就算李师中在大堂中审案,也要给他腾出地方来,何况是在二堂。

王中正命人托着用明黄绸缎盖起的圣旨,随即便举步前行。他手下的从人连忙上前驱赶人群,为他开路,直奔州衙而去。

……………………

杨英快步走进王韶的官厅中。厅中王韶和高遵裕对坐着,在他们中间摆了一张棋盘,黑子白子占满了棋盘,已经终局的模样。而韩冈同样也在厅中,就坐在棋盘横头,正在为他们数子。

听到杨英进门的动静,高遵裕低头看着棋盘,口中则问道:“二堂那边的情况如何?”

由于窦解是官身,又牵涉到窦舜卿这位高官,故而此案并没有大堂上公审,而是改在在二堂审讯。

王韶和高遵裕他们都不是秦州的官员,而是秦凤路经略司的属官。李师中审案,是以秦州知州的身份去审,而不是以经略安抚使的身份去审。王、高二位,以及韩冈都没有插话的余地,连旁听的资格都没有。只能派着手下人去二堂打听。

杨英站定打躬,而后说道:“窦七衙内倒是把所有的事都推到他手下的钱五和李铁臂等人身上,但被传上堂的钱五等人都说一切皆是窦七衙内亲手做得,包括奸杀案,都是窦解一人所为。”

高遵裕听着奇怪,跟着窦解的那些地痞无赖怎么有这等胆量指控窦解,窦舜卿还好好的做着他的兵马副总管呢。他疑惑的问韩冈:“玉昆,你昨夜是不是去大狱里跟他们说了什么?”

韩冈摇摇头:“没有,下官如何瞒着李经略和窦观察的耳目进大狱里去?!”

但高遵裕还有几分不信的样子,韩冈看得苦笑不已。心道日后阴谋诡计还是少用为妙,自己辛苦建立起来的形象要好好保持才行。

王韶在旁帮韩冈说了两句,“这世上还是聪明人居多,谁都能能看得出,眼下的情况帮窦解说话,就是在自己脖子上套绳结。无论钱五还是李铁臂,他们只是一群狐朋狗友,不会为窦解两肋插刀。”他说着又对杨英道,“你再去二堂打探,有什么新的进展,就回来报告。”

“诺。”杨英唱了喏,便转身出去了。

“玉昆……”王韶将棋子一个个收回棋盒,同时问道:“王启年的遗孀现在如何了?”

“机宜放心。王阿柳看似甚重,其实只是皮肉伤,有仇老关照,当不日即可痊愈,王家的一对儿女也没有大碍。”

韩冈说得欣慰,他的这番计划并没有伤害到人命,让他心中感到很轻松。韩冈不介意杀人,他杀得人也多了,但用无辜者的性命却陷害敌人,他却是不愿去做的。

虽然王阿柳未死,她的儿女也安然无恙,但窦解夜入人家的罪名洗不脱的。而他逼问王阿柳,等于是对流言不打自招,将他过去罪行全都带出来了。当窦解被拘押到衙门消息在秦州城中传播开,第二天一早,就拥了几百人来州衙递冤状,现在州衙外面围着数百百姓,都是他的苦主。

“不知窦舜卿会怎么做?”高遵裕跟着王韶一起收拾起棋子,同样随口问着,“他总不会眼睁睁看着孙子去死,自家还要被牵连进去。”

“今早城门刚开,就有人看见有两个窦舜卿的门客带着三四匹马赶出城去了,大概是想找韩琦帮忙。”王韶说道。

“恐怕是远水救不了近渴。”韩冈笑得讥讽,“王启年被杖死的这一桩公案肯定会把窦舜卿拖下水,天子那一关他不好过。”

王韶和高遵裕正要重开棋局,杨英这时又急匆匆的走了回来,向着韩冈三人禀报道:“机宜、提举、抚勾,天使来了,要三位去接旨。”

