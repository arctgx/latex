\section{第29章 君意开疆雪旧耻(中)}

吕公著阴着脸走进文德殿中,文德殿又称外朝,比起主殿大庆殿形制略小,可面积也足以容纳千人以上。殿门之后,略偏东南点的地方摆着一张交椅,那是御史中丞的位子。依本朝礼制,参加朝参的文武众臣中,唯有其一人可坐,取得是独坐之义。汉代朝臣有三独坐——尚书令、司隶校尉、御史中丞——如今千年流传下来,也只剩御史中丞一人。

吕公著站在交椅前,两名殿中侍御史则分立在殿中的两处角落里。三人站定,净鞭鸣响,就在殿堂边缘,乐工们开始吹笙敲钟,奏着赞美圣君贤臣的韶乐,阁门吏则合着乐声高声唱着班次。两名宰相曾公亮、陈升之手持笏板,领着众臣依唱名、按班次陆续进入殿中,在台陛下站定。

净鞭再次响过,殿后有了动静。先是两名起居舍人走出来,他们是记录天子言行的侍从官,一东一西站到了殿内两角。继而是一班手持扇、剑等礼器的黄门宦官。等黄门站好位置,圣乐曲调突然猛然高起,迎接天子出场。

二十出头的赵顼从殿后徐步走出,身穿赭黄袍,头戴平脚幞头,为天子常朝之服。青年皇帝脸色显得苍白了些,相貌以宋人的审美观念,算得上是俊秀,唇角留了髭须,多了些稳重,就是身形太过单薄,不是福寿之相。

天子就坐,群臣跪拜。

一切都是前一次的重复,下一次也不会有任何区别。赵顼坐在御座上,无聊的等着月月都要重复的朝会仪式早点结束。

国计是他关心的,战事也是他关心的,唯独这套繁琐的仪式是他所不关心的。

均输法到底会不会影响到百姓的生计?青苗贷推行准备的情况如何?农田利害条约刚刚实施,其中会不会有什么差错?

西北绥德城的战局稳定下来了没有?聚集泾原路的西贼退还是没退?攻打秦凤路甘谷城的西贼有没有卷土重来?

还有王韶,说是要开边河湟,可他这一年什么动作都没有,现在到了年底了,突然上了份荐书过来,又是什么意思?

一心想做中兴之君的赵顼日日忧心着政事。家国多蹇,大宋自立国以来,便远不如汉唐强势。北方契丹虎视中原,屡屡南侵,太宗皇帝两次北伐皆告惨败,最后还死于高梁河边留下的箭疮。

到了仁宗时,契丹被每年五十万银绢的岁币喂饱,看似天下太平,但西贼元昊又举起了叛旗。三次大战皆惨败,最后让西贼在灵武立国。仁宗朝的名臣们给出的办法是什么?用二十万银绢买回西贼一个口头上的臣服!

君辱臣死,可他堂堂华夏天子却要跟北方的蛮夷称兄道弟,把民脂民膏送给永不满足的西贼,他的臣子对此不以为耻,反以为荣。说是用区区财货,以使生民免于涂炭之苦,乃是圣君所为。

赵顼冷笑起来。不愧都是进士出身,总有是话说!如果他们手上跟嘴上一样有才,早早将二贼剿灭,生民又怎会涂炭?!

仁宗能忍,英宗能忍,但他赵顼忍不得。韩琦老了,富弼老了,文彦博也老了,仁宗朝留下的名臣都已经毫无锐气,只知道要他二十年不谈兵事,却让他独自忍受噬心的耻辱。

还好有个王安石。

现在朝中弹劾王安石的朝臣很多,甚至有许多早前还是称赞并举荐过王安石的,比如富弼,比如吕公著。能有一人能像王安石那样给出一个富国强兵的方略的吗?

没有!司马光没有!文彦博也没有!

赵顼低头望着文德殿中,如神道石像那般站得齐齐整整的文武两班。要实现他的理想,满朝文武,却只有一个王安石。

朝会仪式依旧按部就班的进行着。几个被调入京中的朝官出来谢恩,几个须告老的官员出来陛辞。没有任何意外和惊喜,朝会就这么结束。百官自高至低卷班而出,到了文德门外,各自返回公厅,只有两府宰执,主管财计的三司使,以及内制翰林学士和外制中书舍人中,带了知制诰头衔的两制官留了下来,向皇城后部的崇政殿走去。

朔望大朝会,仅是礼仪性质的朝会,四五百人聚于外朝文德殿中,又能讨论起什么政事?真正处理国家政务的地方,是平日里只有宰执和一些重要朝臣参加,举行常起居的内朝垂拱殿,以及朝会结束后,天子‘阅事之所’的崇政殿。

今日是朔望大朝参的日子,故而没有常起居,结束了朝会,赵顼直接到崇政殿处理政务。有两府与会,将需要天子批准的朝事一一上报。而其中,最为赵顼关心的便是西北的战局。以绥德为核心的横山攻势,以秦凤为后盾的河湟辟土,关系到日后伐夏的得失成败,绝不容有失。

位于鄜延路的绥德城战事已经平息,党项人曾经想利用几座废弃的旧寨换回绥德的计谋也宣告失败,横山地区的战局如今正向大宋一方倾斜,只要绥德城能稳守,日后便可步步为营,并吞整个横山地区。横山一失,西夏东南屏障顿毁,连重要的募兵地也将失去,自此瀚海天险便会为西夏和大宋所共有,就像失去了淮河流域、长江天险便不足为凭的南方偏安政权一样岌岌可危。

在西夏秉政的梁太后及其担任宰相的兄长梁乙埋,对此看得也很清楚。便学着大宋的做法,在绥德城北开始修筑寨堡,而且一修便是八座!妄图用一个寨堡群,来抵消宋军在绥德地区逐渐把握在手的战略优势。

赵顼对此很是忧心,不但加紧向鄜延路运兵运粮,甚至将如今国中仅有的几名能征惯战的宿将中的一人——郭逵,调到了鄜延路,任延州【今延安】知州兼鄜延路经略安抚使,全面主持绥德城事务。郭逵曾任同签书枢密院事,近几十年来,除了狄青曾任了一次枢密使外,这已是武将能达到的最高位置,也算是有过担任执政的资历。将郭逵调职鄜延,赵顼对绥德城的重视由此可见。

赵顼关注着陕西局势,他不问枢密使文彦博和吕公弼;不问宰相曾公亮和陈升之,而是直接向王安石询问:“王卿,鄜延路和绥德城处可有新的奏报?”

王安石出班回道:“郭逵宿将,其人在一日,鄜延安一日,陛下并不必太过忧心。”

赵顼岂能不忧心,鄜延路走马承受传回来的密报让他忧思难解。走马承受是天子外派的耳目,大多数都是由宦官出任,也有的是从天子身边的班直挑选,他们密报的可信度,在赵顼看来要高于地方官们的奏折:“但郭逵与种谔不和。种谔如今刚刚自随州起复,郭逵便对人说其是狂生,徒以家世用之,必误大事。将帅不和,如何用兵?”

“郭逵年已老,行事求稳。种谔正当年,锋锐正盛。两人行事参差,自难相和,郭逵不喜种谔,乃人之常情。陛下不须忧虑。”

鄜延路将帅之争,王安石毫不犹豫地站在种谔一边。郭逵并不差,但打开绥德局面的人是种谔,其人有勇有谋,其父种世衡又在鄜延路威信远布。王安石他深信,假以时日,为大宋开疆辟土、讨灭西贼的,不是郭逵这班锐气褪尽的老将,而是如种谔一样的新锐。

“陛下,郭逵向以知人著称。当初葛怀敏虚名远传,无人不赞,唯郭逵言其‘喜功徼幸,徒勇无谋’,后果有定川寨之败。其论人成败,自有其理,不当视之以武夫挟怨。”王安石既然支持了种谔,枢密使文彦博自然要支持郭逵。尽管郭逵反对他退还绥德的提议,还戳穿了西夏意图用塞门等几个废弃的旧寨交换绥德的阴谋,让文枢密大丢脸面,但为了打击支持种谔的王安石,也顾不了那么多。

文彦博说得似乎有理,赵顼又转头看向王安石。

王安石反驳道:“郭逵当年在延州时,因忠义社与内附羌人争斗致死之事,与种世衡有过龃龉。岂可谓之无旧怨?”

“竟有此事……”赵顼还是第一次听说,沉吟了一下,向王安石征求意见:“王卿,以你之见,是否当把种谔调去他路?”

王安石摇头:“郭逵老成持重,虽有旧怨,亦当止于言辞,不至因私害公。郭逵前次洞悉西贼奸谋,谏阻以绥德换回塞门、安远二废寨,枢密院至今尚未定下封赏。以臣愚见,不若陛下亲下手诏褒奖,再遣一内臣以封赏之名前往延州,暗中加以训诫,自当无事。”

王安石一番话连打带敲,将枢密院的两次失误拽了出来,堵得文彦博无话可说,反对不是,同意更不是。而赵顼尚年轻,登基不过三年,也看不破两名重臣之间的暗流汹涌,只觉得王安石的处理办法顾及了老将郭逵的颜面,又能让其警醒,的确可行,颔首道:“便依王卿之言。”

ps:横山开拓和拓边河湟,同是熙宁初年宋人在陕西的战略规划,聚集在同一个区域的不同战略,互相之间影响很深,也是必要的背景描写。而且郭逵和种谔也是后文中的两个重要的人物,需要先提一下。所以韩三仍在休息中。

今天第一更,求红票,收藏。

