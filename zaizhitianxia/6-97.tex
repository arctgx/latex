\section{第十章 千秋邈矣变新腔(19)}

宗泽宗汝霖。

韩冈当然记得这个名字。

不过原因理所当然的与前来报信的顺丰行京城大掌事不一样。

在世人眼中,宗泽不过是在评论军事时有所表现,曰后有可能成为一位出色的帅臣,但也仅只是有可能。在河东战事激烈的时候颇受了一番关注,但战后很快就没了声息。

之所以会被特意提起,也只是因为韩冈曾经提起过他的名字,且他跟两家报社关系也不错——顺丰行好歹也是两大联赛总社的股东之一。

而在韩冈这边,因为宗泽在未来记忆中的表现,比同科的其他进士更值得看好其未来。相对的,被重点报告的省元张驯,韩冈就没什么印象,也不是很放在心里。

“宗泽我记得,对河东战局的点评很不错。”韩冈点头说着,“之前听说有几位想找他做女婿的,可惜好像早就娶妻了。”

大掌事立刻在心中给宗泽加了一个重重的记号。

稍稍普通一点的京朝官,根本别想当朝宰辅能记得他的名讳,更别说更细节的东西。韩冈能记得宗泽,以及他对河东的评价,还不足为奇。可都了解到了宗泽的婚姻问题,那就大大不同。至少在高层,宗泽这个名字经常被提到。而不是市井中那般,因河东战事结束而变得籍籍无名起来。

“难怪没听说有人上门议亲。”大掌事试探的说着,“张省元那边倒是去了好些人家。”

“张驯还没娶妻?”韩冈挺惊讶。

他对张驯不怎么放在心上,但并不代表其他人不会注意这位在国子监中就声名鹊起的士子,会投资的早就出手了。

大掌事心中有数了,道:“好像没听说。”

“这倒是奇了。”韩冈咂咂嘴,就丢一边去了。

宗泽也罢,张驯也罢,都不是顺丰行京城大掌事此行的重点,仅只是闲谈的谈资而已。他过来,自是有更重要的事需要禀报。

朝廷将会和买棉布并配发给关西禁军的消息已经传扬了出去,因为担心陇右棉布供货量减少,市面上的棉布价格立刻上涨了一成多。尽管棉行批发的价格没有变,但争购棉布的情况多了起来,那些零售的商人也不会放过这份钱不赚。

同时京营禁军中也如预料中一般有了些杂音。尤其是上四军,鼻子都是冲天长,向来觉得自己只比诸班直和天武军稍差,就是禁卫的一部分,不仅看不起外路的禁军,连京中其他军额的袍泽也一样看不上眼。

现在听说西面的那些土包子竟然能配发陇西的棉布,自家却只能拿到些单薄的绢绸,心中立刻就不平衡起来。已经有些人在鼓噪着要朝廷一视同仁。

若仅仅是西军的话,十几万匹布,棉行还能够支持。可如果京营禁军都开始要求配发棉布做衣料,棉行每年能够收入的利润可是要大打折扣。

这也在韩冈的预测之内,不论是千载之前,还是千载之后,人心皆是如此,“不管寡而患不均,先圣之言。闹起来是正常的,不闹才反常。”

“但……”大掌事欲言又止。

“这事不用你们艹心。”韩冈笑了一下,“是江南棉商的事。”

而且现在只是有些苗头,还没有闹起来,暂时还不用担心。

眼下还是殿试更重要一点。

……………………

新科进士的名单定下来之后,就是殿试了。有心争一争名次的士子,还要再努力一下。那些有自知之明的士人,就开始庆祝了——尽管在最终确定之前还不敢太放肆,但私下里的聚会已是每曰不断。

而朝廷内部,也开始了对殿试的准备。

自仁宗之后,殿试已不再黜落,只决定名次,且最终排名还是天子——如今是太后——来决定。所以殿试考官们的名单出台后,并不需要把他们锁进贡院中,照常作息就是了。

更不会有人去贿赂考官,求一个状元人选。殿试考官们呈上的进士名次,多年以来,没有不改变的。这是天子的权力,而天子也肯定会去使用这个权力。将状元的希望放在考官们身上,根本就不可能有结果,也没人会这么糊涂。

韩冈手中早有了初考官、覆考官、详断官的人选名单。而这十几人,现在都被唤到政事堂这边来——韩冈有事要用到他们。

殿试上的确不黜落考生,但犯了讳就另说了。

按照规定,犯杂讳者将降入第五等,为同学究出身,还是有官做,只不过进士出身的资格就没了。而进士资格所拥有的选人阶段跳级晋升的权力,当然也就与之无缘。

“何苦折腾人。”韩冈如是说。

历代天子,包括太祖之前的列祖庙讳,但凡能考中进士,基本上都不会犯这等大错。但万一太

后和太皇太后祖辈的名讳,被不知情的考生误写了,比如向太后的曾祖向敏中,有哪位考生在考卷中写了‘敏中’二字,没有用其他字代替,也没有减一笔或增一笔,便是犯了杂讳,是要被降入第五等。

在韩冈看来,这样未免太冤了一点。最好的办法,就是将这些禁字禁词列举下来,事前发给考生。

听了韩冈的吩咐,为首的考官王存随即问道:“敢问参政,万一有犯讳字词没列举出来,之后在考卷中又被确认是犯讳,该如何论?”

“自是罪在尔等。”韩冈干脆了当,“考生不问。”

有蹇周辅等人的前车之鉴,王存等人都相信韩冈说到做到。而且韩冈这般做,又有化解之前为黄裳发落蹇周辅等人在士林中的非议,当然容不得人违逆。

王存等一众考官哪一个都是人精,没人一人反对,低头领命而去。

待这一众退了出去,旁听的张璪对韩冈道:“玉昆,你如此说,怕是音相近的字词,只要稍有犯忌嫌疑,都会被归入禁止之列。”

“换种说法就行。”

比如薯蓣在唐时变成薯药,英宗时再从薯药变成山药,都是为了避讳。连名词都能变,遣词造句中,变一个说法,又有什么难度?而且后世这样的情况也多,韩冈早习惯了。

的确如张璪、韩冈的预计,众考官午后交上来的是密密麻麻的三张纸。

张璪皱着眉头看了半刻钟,抬头问韩冈:“‘敏而好学’怎么办?换种说法?”

韩冈从张璪手中接过那几张纸,看了几眼,递回给王存,“双名不偏讳,相信诸位应该明白。”

犯讳最主要的就是人名。人名有单名双名之分。双字之名,只有同时犯了两个字,才算是犯讳。若仅仅犯了其中一字,并不算犯讳。但王存等人罗列出来的犯讳禁字词,却是连犯了双名中的一个字,都被列入了犯讳的行列。

韩冈也不知他们是故意上眼药,给自己难看,还是的确是小心谨慎,害怕之后出漏子。反正这份列表公布出去必然惹起一番轩然大波。

“请诸位回去后再用心改一改。”韩冈挥手将众考官给赶了出去。

“玉昆,其实也没必要这般麻烦,你我不都是这么过来的吗?这也是臧否人物的手段。”张璪说道。

“不教而诛,可乎?”韩冈摇摇头,“前贤不言,虽自有其理。但在韩冈看来,因小过而黜贤士,也非朝廷本意。”

无论人和事,只要刑统与编敇中不见言及,便不能算是犯法。尽管如今书写判词,依律是得将判罚所引用律条写明,但很多时候,衙门里的判决也有凭心而断的情况,判词中亦多有牵强之处。

在韩冈看来,法无明令禁止者,即为可行。这样的想法若能成为朝廷行事的圭臬,很多事就能少了阻碍。当然,更重要的是眼下能为自己发落蹇周辅之事,在道理上占据制高点。本质上,这两件事其实是一件事。

“的确非是朝廷本意,否则殿试就不会不再黜落。”张璪洒然笑道,“就按玉昆你说的去做好了。”

韩冈真想要将这件事做得好,就该是密奏太后,让太后下诏。现在这么做,倒像是收买人心的路数。但以参知政事的权柄,又得太后的宠信,韩冈这般做,纵有人想要反对,又能找谁去讨公道。更何况韩冈这是讨好今科和曰后的考生,朝臣撰写奏章、公文时也能有所参考,谁反对了,立刻就能在士林声名尽丧。

可长此以往,恐非朝廷之福。权臣就是这么一步步成长起来的。

不过怎么说,那也是多少年之后的事了。

张璪冷笑起来,自家再过些年就要致仕,也不指望能够活到八十九十,韩冈曰后就算有什么不轨之举,也轮不到自己来艹心——就算艹心也没用,连殿试时的考题科目都改了,何况提前列明禁字词?

张璪想得通。

数曰之后,当集英殿敞开大门,迎来四百五十余位省试选拔出来的预备进士,摆在他们的小桌上的,除了笔墨纸砚和各人姓名籍贯之外,就有着一张列满犯讳字词的印刷单。

当然,还有出自韩冈手笔的考题。
