\section{第17章 往来城府志不移(八)}

赵顼听着韩冈对编纂药典的陈述,默默的不停点头。

韩冈对医术一窍不通——这是世间公认的看法。但人贵能学,韩冈这十年来,不好声色,不事游乐,闲暇时只以读为消遣。就是寻常的凡庸之辈,能潜下心来专心十年向学,也能有所成就,何论韩冈?

十年之功,韩冈医也读了许多,要说给人问诊治病,那依然是不成的,可至少他对这个时代的医药,已经有了比较深入的了解。一时间倒也说得头头是道。

赵顼对韩冈怎么编订药典医典没有兴趣,可编纂出来绝对是一桩美事。韩冈素不轻言妄语,过往的经验让赵顼很清楚这一点。韩冈既然对《神农本草经》能胸有成竹的批评,自然是有所依仗。以他的才能来看,应当是一部尤胜前人的大典。能有这样的一部本草药典问世,便是他赵顼文治武功的一个证明。

编纂类典籍,是彰显一朝文萃的盛事,太宗皇帝在高粱河兵败后,便着令宰辅李昉等人主持编修《太平广记》、《太平御览》、《文苑英华》,由此来挽回失去的声望。其中《太平广记》,仅是对前代的小说和传奇加以收集编订,可领衔的依然是宰相之尊,所受到的重视可见一斑。

“此事乃是一时盛举,还得韩卿上条陈之,朕当细览。”虽然同样是要韩冈进札子,但这一回赵顼的语气要郑重十倍。

主编典籍的功劳,足以将一名重臣推送入两府之中。韩冈当是想以此为功,赵顼自问看透了韩冈的心思。但一部大典的编纂,穷十年之功亦是等闲,不成、不论功,若是能像《资治通鉴》于司马光一般,耗费去韩冈多余的精力,对赵顼来说倒也是好事。

韩冈躬身领命:“臣遵旨。”

终于如愿以偿,韩冈也是放下一桩心事来。向着目标稳步前进,总是能让人心情舒畅。

拥有的来自后世的学问并不多,韩冈知道自己能做的很有限。毕竟他没能力推导出物理和数学上的一干公式,也不知道,只能用仅有的一点常识,来拼凑出一个大概出来。

物理中的力学、光学,化学中的元素论,生物中的分类学,在数学中则是近似于后世代数的天元术,在自然哲学上,则是一力主张着实证。虽然都是十分粗浅的理论知识,但韩冈相信,只要假以时日,必然能顺利的生根发芽,最后得到丰厚的成果。望远镜和显微镜的出现,便是最佳的证明。

赵顼又微笑着说道:“药典若成,定为本朝一大盛举。令岳近日又进呈了《字说》,考订先王之文,欲以一道德。卿家翁婿,无论文武,皆是有大功于国。”

韩冈想不到王安石的那部训诂都已经定稿成了,还赶在自己入京之前送到了赵顼的手中。这个速度还真是令人吃惊。王安石这是在煽风添柴,新学这一下子声势又上去了。

“家岳的新作,曾与臣共议过。的确是难得的佳作,只是也有一些地方,是臣难以苟同的。”韩冈并不遮掩自己对王安石新作的看法。

“是吗……”赵顼低低回了一声,却不置可否,也没有细问。

韩冈没等到赵顼的回音,向上瞥了一眼,赵顼略皱着眉,向后靠着,看似是有些疲累,又是在想些什么。

见状韩冈并不多言,转而低头告退。现在还不是时候。

赵顼也没有留他,而是叮嘱了韩冈尽快将有关厚生司的工作以及编修药典的条陈札子递上去,还安排了一名内侍领着他去太常寺——可惜不是童贯,韩冈回今后就听说他去南方担任走马承受了。

这个未来的奸佞运气还真的是不怎样。若是有机会,韩冈还是愿意帮一帮他,至少有童贯这个人在宫中,也不是什么坏事。

从崇政殿出来,接下来便是去太常寺上任。等到接了太常寺的印,还要往厚生司和太医局去,这两天都得尽快接手。

自殿阁间刮来的风带着宫城中特有的阴冷,仿佛是身处洞穴之中。但绕过回廊,出了文德门,头顶上的阳光立刻又炽烈起来。一想到接下来几天还要在这样的天气下走家串户,韩冈的脚步也变得沉重了。

如今的重臣之中,身兼几任的为数不少,但很少有人是一下接受几项差遣,都是隔一段时间,才会被派上一门差事。韩冈却是一下子接了三门,加上他又打算有所作为,自然是有的忙了。但这样的忙碌,却是他心甘情愿的。

只是今日的廷对有些问题让人警醒。在廷对上,赵顼并没有向韩冈询问河东的境况,以及之后在西北边地应对辽人的方略。对于一名刚刚从河东离任,又积累下了大量战功的经略使来说,这样的情况并不正常。

韩冈很确定,这肯定不是赵顼忘了问,而是不想让自己有机会对河东、陕西继续保持着影响力,甚至有警告的成分在。反正与辽人已经定下了国界,需要知道什么消息,都能从其他官员和走马承受那里得到回应。

幸好他已经提前做好了转换角色的准备,几个差事上该做什么,能做什么,都有了计划,这样才没有在崇政殿上丢人现眼。

有天子亲遣的内侍领着,就不必先去政事堂走一遭。绕过政事堂和枢密院,太常寺就在眼前。

位处皇城西南角的太常寺,是一个十分冷清的一个衙门,比起不远处人进人出的司农寺和都水监来,太常寺的门前只有两个守门的兵丁,百无聊赖在檐下的阴凉处坐着。在这个酷暑难耐的日子里,门可罗雀对太常寺来说,看起来并不是个形容词。

担任判太常寺的敕就在身上,在前面替韩冈引路的内侍也是对身后的新任判太常寺恭恭敬敬,还没近了大门就已经开始高声喝道。

两名守门兵丁见了韩冈几人过来,只是懒洋洋的站起身。可一当他们听到了内侍的吆喝声,立时吓得面如土色,直挺挺的立在门前。

韩冈也没理会他们,就在大门外停了脚,仰头看着太常寺的牌匾。竟然还发现了一个燕子窝,真是离谱到了极点。

见韩冈抬头只顾着牌匾,两名兵丁手足无措,一时不知如何是好。

“还傻站着作甚?”内侍尖着嗓子呵斥道,“还不通知寺内开正门迎韩端明入衙?!”回过头来,他又对韩冈叹道:“清闲的衙门,都懒散惯了。”

两名兵丁先是慌慌张张都想进去通知,但一看到同伴也在往里走,又同时停下脚。反复几次,才一人进去通知,一人走过来向韩冈请罪。

韩冈摇摇头,轻声道:“这是掌管礼法的太常寺啊。”

若是先去了政事堂,肯定不会遇到现在这样礼数不周的情况,政事堂肯定会先行知会太常寺。但崇政殿的内侍,就不会管那么多了,只管将韩冈带到。

在门前停了片刻,只听到里面一片脚步声,然后正门吱呀呀的打开了,迎出来三十多名官吏。

太常寺本有卿,少卿,丞,博士,主簿,协律郎,奉礼郎,太祝等众多官员,管理着一应朝廷与礼仪祭祀有关的工作。但现在这些官职,全都变成了本官官阶,而不再是实职的差遣。

真正从属于太常寺的实职官员,其实只有七八位。韩冈看到三十多人中,最前面的几个都身穿官服,倒是知道他手下的官员,差不多当是到齐了。

只是这一群从太常寺中迎出来的官吏,衣着寒酸得紧,看起来就是一群破落户的模样。

一般来说,朝廷不发成衣,只发布帛,官服必须要自己去找裁缝量身定做。所以有钱的官员,身上的官服总是簇新的,而身家匮乏的,衣着则是黯淡褪色——这个时代的染色技术算不上,只有新衣才能色泽鲜亮,一旦洗过,登时就会褪色,洗得次数越多,褪色的就会更厉害——从衣着上看,太常寺无论官吏,都是穷得可以。

只有一人还不错,衣着光鲜,迥异他人。站在官员班列的最后,看起来当已是年过不惑,相貌却是英俊,只是没有留须这一点却让韩冈很纳闷,到了三十之后,就看不到不留须的官员了,就是他韩冈,为了形象更稳重一点,也没有免俗。

不过当韩冈的僚属们一个个上前通名见礼后,韩冈便释然了。

乃是教坊使丁仙现。身为教坊使,自然能得不少供奉。管了十几年的教坊,若是没些身家那就好笑了。

丁仙现名气不小,韩冈都有所耳闻。他的名声也跟他曾经公开讽刺新法有关,世言曾有‘台官不如伶官’的说法,便是指当时的台谏官们还不如丁仙现敢于抨击新法。王安石甚至被气得火冒三丈,想要将他治罪,不过给赵颢保护起来了。

韩冈上下一打量:“丁仙现?那就是传闻中的丁使了。”

丁仙现此时似乎没有了变法之初的活跃,沉稳的向韩冈行了一礼,“贱名有辱端明清听。”

一个伶官,当然与殿阁学士一级的重臣没得比,但伶人自古就有讽谏天子的惯例,丁仙现这么老成倒还真是让韩冈意外。