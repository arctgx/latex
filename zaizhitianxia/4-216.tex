\section{第34章 云庭降鹤宴华堂(上)}

“朝散大夫、右司郎中、上轻车都尉、临淄县开国伯、京西路都转运使、龙图阁韩学士到!”

韩冈一边佩服着富家门前唱名的迎宾,自己长长的一串官衔,竟然不待喘气的就唱了出来,一边在富弼的长子富绍庭的引领下,向着富府的宅邸深处走去。

身穿全套公服、紫袍金带的韩冈,一路上,引来了不少视线。韩冈来洛阳不过半月,但已经靠着文彦博出了名了。过去他的名声固然响亮,但毕竟不是发生在洛阳,不过是远方的奇人异事罢了。可经过之前的府漕之争,韩冈已经在洛阳城中深深刻下自己的印记。

富府的正堂就在眼前,不过富绍庭并没有将韩冈往正堂中带,而是绕过去后面走,“家严正在还政堂中,前面吩咐下来,龙图若是到了,可往后面来。”

见客的地方越是私密,就代表着关系越是亲近。还政堂是富弼致仕后日常起居的居所,就像昼锦堂、独乐园、安乐窝一样,名气甚大。这应该是富家极亲近的亲戚,又或是关系极好的友人才能走进的地方。而韩冈他甚至还没有拜见过富弼。

“……那就劳烦德先兄了。”

聪明人有聪明人的做法,韩冈对于富弼给他的特别待遇,没多说什么谦辞。只是与富绍庭聊着些无关紧要的闲话,连旁敲侧击都没有。反正富弼到底有什么打算,只要见了面就会一清二楚。

韩冈还是第一次见到富弼。相貌上与富绍庭很有几分相似,但给人的感觉截然不同。略瘦,身量中等,虽老而筋骨强健,就是走路有些跛。

韩冈被领到堂前院中时,就看到富弼在一名老仆的搀扶下,一拐一拐的降阶相迎。

“末学韩冈拜见郑国公。”韩冈连忙前行两步,依着拜见宰相的礼数,向富弼行礼。

富弼甩开老仆的搀扶,向着韩冈回了一礼,“玉昆累有功勋,世所难匹。老夫久欲与玉昆一晤,不想延及今日。”

韩冈侧过身,避让过富弼的回礼,“富公年高德劭,韩冈后生晚辈,岂能当得起如此赞许。”

韩冈素知,宰相礼绝百僚,不与他官分庭抗礼,唯有富弼做宰相时,就是接见的客人官位再卑,也照样保持着谦恭的作派。他对富弼的态度并不以为异,只是看见一个快八十的老人与自己平头行礼,怎么也是不敢当。就是在律法上,七十以上,上堂都不须跪的。

富弼却是一板一眼的将礼数尽到,直起身后又笑道:“玉昆莫要自谦,老夫当年可是远不及你。”

“韩冈岂能与富公相比?韩冈一点微功,不过是安定边州而已,而富公却是安定天下四百军州,譬如以星辰比之皓月,可望而不可及。”

人敬我一尺,我敬人一丈,富弼对韩冈表现得足够尊重,韩冈也不可能崖岸自高。官场之上,面子都是互相给的。

富弼笑了笑,转身领着韩冈往厅中去,一步步的慢慢迈着阶梯,富绍庭也上前扶着富弼。

等分宾主坐下来,富弼拍拍自己的腿脚,笑道:“这两条腿,一年比一年差了,要是换做过去,应在照壁前相候。”

“韩冈岂能当得起老相公如此重礼?如今已是折福,再来恐怕也要折寿了。”

韩冈一个劲的谦虚,反正高帽子不要钱,丢给富弼多少都无所谓。

富弼似乎不想再听韩冈说着日常听得太多的奉承话:“文宽夫其实过去腿脚也有问题,只是不知怎么回事,自己就好了。”他叹了一口气,“老夫可是没这个运气,文宽夫倒是荐了几次他日常所吃的药,但不论老夫怎么吃,都不见好。”

富弼一番话似有深意,韩冈也能想得明白,只是不便搭腔,就只从字面意思上去理解。

他前两天刚刚见过文彦博,并没有发现他的腿脚哪里有不便,但他的确听说过文彦博过去有足疾。

当今天子在即位之初,曾经问人道:‘文彦博跛履,韩琦嘶声,如何皆贵?’——韩琦声音尖细,而文彦博则是腿脚不便。得到的回答则是,‘若不跛履嘶声,陛下不得而臣。’——文、韩两位若不是有这等缺点,陛下你怎么有资格让他们当臣子?

看起来似乎是平日里吃得太好了,最后得了脚气病的缘故,不然文彦博的脚病不会突然不药自愈——很可能是改了饮食的缘故。

富弼的话中似乎也有求医问药的打算,但韩冈想想还是决定不去出这个风头。

他对医术其实这些年来也不是没有研习,但诊脉一关是要靠经验累积,韩冈由于种种顾虑,又不敢放手去练习,多年来根本就没有任何进步。望闻问切都不会,还指望什么?

在富弼面前对病症说三道四,无论能不能治好,日后的麻烦肯定少不了。可能是脚气病,也可能是风湿,又或是其他种类的疾病,说不准的事。还是按照市井中的传言,韩冈他只知如何医万人,却不知如何医一人。

不过随口提点两句也没什么:“韩冈有闻,世间有句俗话叫药补不如食补,也许是文潞公日常饮食的关系。”

富弼也没当真认为韩冈能开出药方,他的脚病时好时坏,也拖了多少年了,多少名医都看过,就是不见大好。韩冈随口之言,他也就随便听听,“倒是有几分道理的。过两日就去问问文宽夫他日常吃些什么。”

喝着茶汤,富弼和韩冈谈天说地,以两人的心性城府,自然不会交浅言深,只是说些朝堂上的趣闻轶事,又或是韩冈在陇西和广西的见闻,半句也不提变法之事。不过富弼对韩冈的赞赏,溢于言表。

韩冈暗自猜度着,富弼是不是后悔了。毕竟新法实行后的成果越来越明显,而且熙宁十年间对外开边的成功,也让赵顼登基之初,富弼所说的‘愿陛下二十年不言兵’成了笑话。

如果以英宗时的情况,富弼说得也不算大错。当时的确也打不起仗,以当时的军队状况,硬是上阵,也不会有如今的成功。但现在靠着新法带来的政务和财政上的发展,使得军事上也有脱胎换骨的变化,眼下继续硬抗着,只会被天子丢到一边,再也不会理会。

“……记得皇佑三年,汴水于六月断流。当时沿河诸州,动用了十六万军民,连日疏浚,耗工三百余万。当时朝堂上都乱了,东京百万军民全都靠了汴水运来的六百万石纲粮,一旦有失,后果不堪设想。”富弼说了半天,终于说到了正题上,“当时曾有人提议要重新开凿襄汉漕渠,但翻看旧档之后,又都放弃了这项提议。如今幸有玉昆在,襄汉漕运如能顺利开通,汴水即便又有变故,也不至于再让京城一夕三惊。”

“此事非韩冈一人能为之。行事有唐州的沈存中,钱粮还得靠德先兄。”

就跟王旁担任应天府诸司库务,文及甫管辖西京粮料院一样,富绍庭这位宰相之子,如今管着的是西京诸司库务。韩冈是不清楚,为什么这些老臣之子,都被安排到油水丰厚的差事,但富绍庭手上的差事,对韩冈的工作有不小的影响。

“老夫传家无他,惟有忠孝二字。若逢王事,富家子弟无人敢不尽力。”

韩冈点了富绍庭的名,富弼的目的也就达到了。派了胸脯保证后,就没再多说自己儿子的事,态度要靠做事表现出来,能否结好韩冈,就要看之后富绍庭的表现了。

喝了口茶,富弼又道:“不知玉昆听说了没有。吴冲卿五天前因为其子吴安持牵连进了相州一案,被拘入台狱审讯,上表恳辞相位。”

韩冈乍闻此事,也不由得感到几分惊讶。可既然天子没有干涉御史台对吴安持的拘禁,那么肯定是有抛弃吴充的打算。从熙宁十年年中,到现在才几个月的时间,看样子就要有第三位宰相下台了,朝堂上要重新稳定下来,看起来还要费上不少功夫。

“天子应该不会就此答应吧?”韩冈问道,“如今朝中可只有吴冲卿一位宰相。”

宰相身荷一国之重,朝中无相的闹剧,只有在开国之初出现过一次,此后百多年,没有说哪一日朝堂上没有宰相压阵。只要天子还没有任命第二位宰相,吴充就不可能就此下台。

富弼一笑:“两天前天子就已御内东门小殿,锁院宣麻,擢王禹玉为集贤相。”

“王禹玉终于如愿以偿了。”韩冈叹了一口气,心中却不无惊叹。此老耳目还当真灵通,韩冈发现他这位都转运使还不如已经致仕的富弼耳聪目明。

虽说宰相就任的消息,是用马递加急送往各地,但前天拜相,应该是昨天才发出消息,于今天抵达洛阳。而这个新闻,竟然一点也没耽搁的就传到了富弼的耳中。

“眼下政事堂中只有一相一参,东府中肯定要进人了。”富弼闲闲的提了一句,又道,“还有文家的六哥,他不合为陈安民说项,当是有些麻烦了。”

文及甫是吴安持的姐夫,而韩冈是吴安持的连襟,说起来也是亲戚。不过这份亲戚关系,韩冈并不是很在意。反倒是富弼的这番话,让韩冈破费思量。

