\section{第30章 随阳雁飞各西东(17)}

仁多零丁的呼喊被一圈圈的传递开去,震撼着每一个人的心灵。

贺兰池的泉水,五台山寺的钟声,那是多少人魂牵梦萦的思念。自从被赶出了家乡之后,只有在午夜梦回时才能见上一面。

一年多来,辽人每每耀武扬威,他们就只能忍气吞声。实力不如人,而且背后的靠山根本不是靠山,而是百多年来的死敌,更是畏辽人如虎。即便是在辽人那里吃了亏,也决不会帮上一把。

要不然辽人怎么敢明着将手插进来,唆使他们去攻打鸣沙城?那是实实在在的有恃无恐啊!

“我贺浪家愿随太尉杀回贺兰山!”依附仁多家的小族族长贺浪罗第一个站出来回应,“杀回去,杀回家乡去!”

“我讹庞家愿随太尉杀回贺兰山!”曾经权倾国中的大族,如今残留下来的余孽,也同样回应着仁多零丁的呼喊。

“我移聿家愿随太尉杀回贺兰山!”

“我妹勒家愿随太尉杀回贺兰山!”

一家家部族的族长站了出来,他们受够了,也不想再忍受了。士兵们开始振臂高呼,越来越多的人参与了进来。

“杀回贺兰山!”

“杀回贺兰山!”

“杀回贺兰山!”

到了最后,就只有贺兰山一遍又一遍的被重复着,连仁多零丁也在挥臂高喊。

那是党项人数百年来生活憩息的土地,那是他们自小痛饮的水源。在山下,有雪水和河水共同灌溉的田地,有饲养着牛羊驼马的牧场。

那是他们的家乡。

万众同呼,声势一圈圈的扩散开来,如雷霆回响在山间,直冲云霄而去。察哥同样心情激荡,但他还记得方才的对话,他震惊的看着仁多洗忠:“你事前都知道了?!”

“叶孛麻那边也会一起跟着走,他也是受不了了。”仁多洗忠没有直接回答,他正沉醉在眼前这万众同心的场面中,他回头大声冲着察哥喊道:“察哥你难道还没有受够在这里的日子?!在青铜峡中,我们是脖子上栓绳的狗!回到兴灵,那才是能奔行千里的狼。我宁可死在贺兰山下,也不活在这山沟里!”

“真要想占据兴灵,辽人也不会,宋人更不会坐视。”察哥恢复了一点冷静,“宋辽都不会想看到再出一个大白髙国。再出一个景宗皇帝!”

“那么将兴灵送给宋人就是了。”仁多洗忠笑容中有着仁多家特有的忠厚,“辽贼攻打韦州,我等大宋臣子怎么能不为君分忧?”

说罢他哈哈大笑,“到时就由得宋人辽人去争吧!只要能回到贺兰山下,我们的前路是海阔天高!”

……………………

种朴冲着城外的敌军打了个哈欠。

已经是……已经是……到底多少天了?!

到底被辽人围困多少天了,种朴手指曲曲伸伸了半天,也没数明白。他只知道现在脸颊上的伤口总是痒得想让人用力挠上两把!

日子过得昏头昏脑,胡须都有好长时间没打理了。

种朴现在都没弄清楚,城外的辽人到底是想讹诈朝廷而出兵,还是为了消耗一下兴灵的党项余孽才出兵的。

几天下来,党项人在城外死了无数。城下的土坡堆到一半就堆不下去了,垒土攻城的战术固然有效,但党项人在这一过程中死伤太重,已经支撑不下去这样的作战方式。

要不是城外总有两三个辽军骑兵的千人队守着,种朴估计斩获的首级数目能上两千了,其中还能有三成是改了契丹人发式的假契丹——首级跟服饰不一样,换了发式衣服不换,照样能看得出是党项人,但脑袋一砍下来,可就是真契丹了。

种朴以下,三千官兵抓心挠肝,对辽人将尸首拖回去的行为愤愤不已。六七百契丹军的斩首,怎么都能换上三级功了。

城外的霹雳炮更是偃旗息鼓。这些天来,辽人总共造出了五十多门霹雳炮。但都被城上的八牛弩和霹雳炮给摧毁了大半。在这过程中,城墙上塌了几处,但并不严重,仅仅是外墙墙皮,本身的墙体依然结实坚固。

不过辽人也适应了城上的反击手法,在悬停在高处的飞船的观测下,城中守军的一举一动都在他们的观测范围内。能做到一发现宋人将两件利器给运过来,就立刻转换攻击位置。就这样一躲一追,最后让城外残余的霹雳炮全数逃出生天。

不过也仅此而已,现在的情况,是城外的敌军攻不进来,但城中的守军也攻不过去。两边大眼瞪小眼的对峙着,中间点缀一些攻城守城的戏码。

种朴这几天都在怀疑,辽人的营地中多半已经没有多少人影了,除了两三千骑兵以外,大多数辽人应该都改去了埋伏韦州援军。

只要拿着援军的首级回来,给城内守军的打击不啻于一口气从营地中推出百多具霹雳炮。

想到霹雳炮,就看到远远地漂浮在高空上的两艘飞船。有了一双锐利的眼睛在头顶上悬着,使得种朴几次三番都放弃了出城反击的念头。想要出击,就只能选择黑夜,但辽人如何还会再吃亏?

要是有个办法能将那飞船给射下来,种朴接下来的选择余地就会多上很多了。

种朴在每日必开的集思广益的会议上,才提了一句,就已经有人想到了办法。

“用烟花如何?”一名比种朴还要年轻几岁的幕僚冯真问着。

“烟花?”

“冬天来了,转眼就要过年,从京城里正好乖乖送来了一批烟花火药。”冯真似乎对京城了如指掌,“刘家铺子的烟花火流星,可是能冲到天上去的,比京城中的那座铁塔都高。传言说最好的一种飞火流星,能飞上五六十丈高再爆开来。而用火药带动的箭矢,也同样能飞得更远。”

种朴闻言便沉吟起来。

隔着一里地,而且还是在三十丈的高处,以八牛弩的射程,不是够不到,而是根本射不中。飞船在空中飘来荡去,要想稳稳射中,跟实力完全无关,真的要靠运气。

澶州城头一箭射杀萧达凛的运气,种朴这几天不是试过,但事实已经证明,他可以不用去买今年甚至明年的马券了,肯定中不了。

“要是够不到怎么办?”种朴怎么都觉得看,辽人的飞船离得有些远。

“那就多填些火药进去,装的火药越多,自然就能飞得更高、更快。”

“竹筒可不一定能压得住?”种朴摇头,“而且这等规模的拆烟花,也不是外行人能负担得起的。”

说是这么说,但种朴现在觉得**了众人之智的会议当真有用,很多事合计一下结果就出来了。至于今天的这项提议,种朴并不在意,反正都是烟花爆竹罢了,

再过两天就该祭灶神了,就当提前两日送灶王上天好了。

“就这般去办好了。先试一试成色。”

种朴想着,又打了个哈欠。挥手让冯真去负责他的提议了。

冯真说的那种火药箭,种朴其实有些印象,似乎是《武经总要》中看到过一次。眨眨眼睛,种朴又觉得好像是另一本兵书。

可能是熬夜的缘故,脑中实在是一团浆糊。种朴想了想之后,便完全放弃了继续去思考问题。

真是闲得无聊啊。

城外的辽军攻也不攻,退也不退,硬是坐下来耗时间。虽然过去没有跟辽人有过深入的往来,但辽人的行事风格早就深深的印在了每一位北方宋人的心里。这根本不像是辽人的作风。

种朴也不是没考虑过再出击,可是吃过一次亏的缘故,辽人的戒备森严,完全没有机会。现在韦州城那边还是没消息,估计是已经在防着辽人的围点打援。就不知盐州城那里怎么样了,种朴对自己的父亲很了解,这时候应该已经准备出击了。

“巡检,巡检。”

种朴抬头一看,却是冯真又回来了,在他的身后,还有一小队十几人抬着挑着一堆东西。

“回来得到快!”种朴咕哝了一句,“早就在做了吧?”说着又瞪了冯真一眼。

冯真却是笑而不答,回身命人将他射击的兵器拿出来。

其实就是两种,一种是单根的竹筒,比较修长,前面安了带钩的枪尖,后端插着两片长木片,像是长箭的翎尾。而另一种则是横五竖四的将二十根竹筒绑扎在一起,每根竹筒里面都装了足够数量的长箭,不过长箭上都套了一圈,一根引线从中引出来,与其他引线会合成一条线。

“先试个一次看看。”种朴懒得听人解说原理和步骤,直接上实验。

冯真拿了一根仿佛长箭的竹筒,用一个木架子架上了城头。高高翘起的头部直接瞄着辽人飞船所在的方向。

八牛弩又名一枪三剑箭,特制的铁枪可不便宜,而且极难打造。就算现在铁价大跌,民间的铁器也越来越便宜,专供八牛弩的铁枪也不是廉价货,重量要前后均匀,若是偏了一点,发射出去也飞不了预定中的距离。这样打造的结果,成本据说跟一幅步人甲差不多,而这个竹筒加火药的飞火流星,一看就知道要便宜得多。

拿起火炬,点燃了引线,滋滋的火花一下就没入了竹筒内。

然后……然后就没动静了。

没有火光、没有声响,安安静静的就是一根单纯的竹筒。

种朴、冯真等人屏气凝神的等了片刻,见还是没有动静,冯真便上前,拿起了这具哑了火的流星。种朴也上前了两步,想看个究竟。但就在这时候,呲的一声响,一蓬火焰从竹筒尾端猝然喷出,长长的焰尾将种朴笼罩在内,一下就燎着了种朴蓬乱的胡须。

慌得冯真连忙丢掉了竹筒,上前连扑带打,将种朴的头上身上的火星都扑灭。但种朴此时已经是灰头土脸,连胡须都焦黑了大半。

空气中有着一股刺激的焦臭味,种朴的脸黑着,连生气带烟熏,黑得跟锅底一般。正想开口训斥,却听得砰的一声巨响,所有人都身子一震,却是方才丢下城去的竹筒爆炸了。

种朴的火气收了,沉吟了起来,半晌后方抬头:“这什物,似乎有用。”

