\section{第26章 仕宦岂为稻粱谋(下)}

“末星部如此心腹大患,当是灭得越早越好。”韩冈义正辞严。

王韶摇摇头:“末星部只是小患,不过有八九百帐【注1】,官军一到,举手可灭。真正的大患,远的是西贼党项,近的是诸部吐蕃,都是难以剿灭的隐患。不知贤侄对此有何高见?”

韩冈心知这该算是考试了,如果通过了,一切好说,如果通不过,王韶大概就会掉头走路。幸好他这些天做了点功课,王韶去年上书天子的《平戎策》的内容并不是秘密,而在担任过渭州军事判官的张载门下,他过去也曾记下了许多资料和数据,不会在王韶面前露怯:“具体的措施,机宜的《平戎策》中都已说尽,不外乎以夷制夷,收吐蕃,攻党项。”

王韶轻轻点头,没有说什么。韩冈很清楚王韶要听的并不是这些,大手一挥,开始谈古论今:“吐蕃与大唐同时兴起,其为祸中原,三破长安,烈度远在西夏之上。幸好其覆灭也几乎与唐同时,如今已不足为惧。不过吐蕃国虽亡,部族仍在。如今关西四路,大小部族数以千计,而以秦凤为最。秦凤路沿边十三寨,大部百廿三,小部五百九,户口倍于汉人,其中吐蕃诸部占了九成以上。”

“是啊,秦凤路的吐蕃人太多了。再往西则更多。”王厚在后面插了句嘴,算是帮韩冈做个哏,好引出下文。

韩冈扭头对王厚会意的笑了笑,回过头来继续道:“不过吐蕃有一桩好处,就是畏服贵种。从松赞干布传下来的血脉,最为吐蕃人所敬服。否则李立遵也不必远赴西域去把唃厮罗请回来,再立为赞普【吐蕃国王】,以占一个大义的名分。”

李立遵是几十年前河湟吐蕃的大首领之一,但他没有吐蕃王家血统,无法就任赞普,所以去了西域高昌将传承松赞干布血脉的唃厮罗弄回来做个傀儡,还把自己的两个女儿嫁给了才十二岁的唃厮罗,做足了挟天子以令诸侯的模样。他这一招也算管用,河湟吐蕃中的另一位大首领温逋奇都不得不在名义上低头听从他的号令。

“可叹李立遵妄自尊大,竟然想废唃厮罗而自立为赞普,不想唃厮罗先行一步,转投了温逋奇。”

韩冈说到这里,王韶冷笑一声:“魏武不是那么好做的。”

“机宜说的是,自与唃厮罗反目,李立遵势力大衰,不复旧日之观。唃厮罗投温逋奇后,抛弃了李立遵的女儿,但他以李立遵为殷鉴,不娶温逋奇家女子,而改娶吐蕃大族乔家族之女为后,其势力扩张又为温逋奇所不容,到最后一场火并,温逋奇被杀,唃厮罗成了真正统治河湟的赞普,甚至还大败过李元昊那反贼,让他退回六盘山后。”

王韶似有感触,道:“幸好他家中不靖,不然又是一个李元昊。”

“的确。唃厮罗家中不睦,他弃李立遵之女,便与其所生长子瞎征和次子磨毡角反目。最后却是幼子董毡继承其位,其余两子皆自立。瞎征和磨毡角甚至曾阴助党项,逼得唃厮罗离开青唐王城而远避历精城。如今唃厮罗已死,董毡手段远不如乃父,河湟一带又趋分裂。西贼对河湟虎视眈眈,如果朝廷不加重视,让西贼趁虚而入,关中危矣!”

对于韩冈的一番话,王韶很满意,从中完全可以看出韩冈对河湟局势深有了解。知己知彼,百战不殆,如果连要针对的目标是谁都不知道,这样的人如何能用?

“那依贤侄的意思,对青唐吐蕃又该如何处置?”

第二道考题出来了,韩冈照旧胸有成竹:“汉设伏羌校尉,以羌人攻羌人,唐设安西都护,以西域定西域。以学生愚见,当以汉家兵屯为根本,亲附者用之,不顺者攻之,威服董毡,团聚众部,十万大军举手可集。此一事,可谓之断西贼右臂。待王师北上兴灵,河湟吐蕃便可自西而攻。如此西贼可灭,兴灵可复!国耻得雪,青史上亦可留下名号……”

王韶轻轻击掌,神色却是淡淡。韩冈的话几乎是他上书天子的《平戎策》的翻版,与他心意相合。但其中的空话很多,任何一个对西事有一定了解的士人都能说出这么一番话。王韶他需要的是能处理实际事务的人才,如此大局性的言论,应该是由自己说给天子和宰相们听。

“不过在河湟屯田可不容易!”王韶像是在挑刺,“那里可不是种地的好地方。”

“河湟两千里,为汉陇西、南安、金城三郡之地。汉宣帝时,赵充国留屯金城尽平诸羌。东汉建武年间,马援也说河湟田土肥壤,灌溉流通。如此沃土,只要有人,如何屯不起田?反倒是收服诸部要麻烦一点。”

“如何麻烦?”

“有党项在,吐蕃诸部就多了一个选择。如果逼得太紧,让他们投了党项,反而会弄巧成拙。必须攻心为上,利诱为辅。而征讨最好只用在其中一家身上,用以慑服众蕃。”

“如何攻心利诱?”

“如今吐蕃诸部多虔信浮屠,唃厮罗之名便是吐蕃语中佛子之义,可为明证。当请朝中遣派胆识、才学、医术皆是过人的高僧大德入河湟弘法,他多收一名弟子,我大宋便多一个忠心的蕃部。忠心的蕃部多了,河湟自然再无法脱离中国控制。至于利诱,无外乎册封、赏赐,还有市易。”

“那攻打的又该以谁家为宜?”

“河州为河湟北部重心所在,处于水陆要隘之上。其地之主木征是瞎征之子,唃厮罗的长孙。其人素来狂悖不逊,不服其叔董毡号令,又交通西贼,有取董毡而代之的野心。剿灭木征,夺下河州,可以示好董毡,亦可威服之。河州地处青唐北部,王师领有此地,董毡便无法与西贼联络,也只能投靠于我……”

韩冈侃侃而谈,一切都已烂熟于胸。王韶的问题都在他的准备之中,更确切的说,他回答王韶的考题时,都是刻意将话题带往自己准备充分的领域,从而影响王韶的出题偏向。这种与人辩论上的进阶技巧,韩冈前世是刻意练过,连声音、手势、眼神都在计算之内,可不是王韶一时间所能看破。

一问一答到了最后,王韶也不得不点头称赞:“张子厚真是会教徒弟。”

走得累了,王韶在路边一张长椅上舒舒服服的坐下,韩冈和王厚没资格坐,只能在两边侍立。王韶抬手轻抚还没有打磨过的椅身,对韩冈笑道:“这长条交椅倒不错,坐和躺都可以,亏你想得出来。”

韩冈微笑的一欠身,前面他已经通过考核,如今就该说正题了。看得出这只是王韶的开场白,他便没有搭话。

王韶果然也不等韩冈回话,又道:“只观疗养院中布置,便能看出贤侄你腹中自有锦绣,不枉了子厚的一番教导。张守约荐你为官,不是没有道理。只是弃文从武,怎么说都是辱没斯文的一桩事。贤侄在子厚门下游学多年,不知是甘心还是不甘心?”

“儒门弟子以仁为本,伤病垂死待救,学生不忍弃之。至于文武殊途之事,也顾不得那么多了。”韩冈回得滴水不漏。

‘小狐狸!’王韶暗骂了一句,不得不自揭底牌:“贤侄倒是一番仁心。不过管勾伤病营一事是归于经略司名下管辖,却不一定要武官才能提举。即便是文资也是一般可做。”

“机宜的意思是?……”

“从九品的判司簿尉。秦凤经略安抚司勾当公事,兼理路中伤病事务。经略司中事务繁芜,勾当公事一职也是千头万绪,再加上还要兼理路中伤兵事,旁人怕是难做得周全,不过以贤侄之材,当是举手之劳。”王韶很干脆的开出价码,静静等着韩冈回复。

韩冈沉吟不语,心中比较着王韶和张守约的出价。

对于向宝和张守约之间的牌局来说,韩冈他可算是鬼牌了。现在张守约既然把他这张牌丢了出来,只要向宝反对,张守约就可以名正言顺的使人向枢密院甚至天子上书,把向宝家奴在甘谷城危的时候,拦截辎重车队的事给抖出来。

以韩冈于伏羌城射出的那一箭在秦凤道上流传的广度,凭向宝的权势根本遮瞒不住。一旦此事被朝堂得知,向宝少不得灰头土脸,多半还会被降职。就算向宝不反对,让他赞成,肚子里保不准要积蓄多少怨气,日后向韩冈报复,到时张守约再找人爆料也是一样。

给人当刀使,韩冈并没那般大方。如果王韶没有给他荐书,为了一个官身,韩冈绝对会去拼命,被当刀子也认了。但现在,王韶推荐韩冈任的同样是最低一级的从九品,不过本官却是属于文官系统的判司簿尉——顾名思义,也就是主簿、县尉和监司官的统称——并不是武官。对于王韶的这份推荐,身为武臣的向宝插不了口,相对的,韩冈也便不会再深入一步得罪向宝,何况还有文臣和武臣的地位差距在……

该如何取舍,韩冈自不会弄错。

注1:蕃人多居帐幕之中,一家便是一间帐篷。所以计点蕃落户口,都是按帐篷计算。

ps:一番纷扰,韩冈的官位终于确定,他下一步的晋升路线,也就确定了下来。

第二更,求红票,收藏。

