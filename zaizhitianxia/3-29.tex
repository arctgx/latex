\section{第12章 共道佳节早(三)}

已是腊月初一,离着祭天宗祀的大典也没有几日了。为着今次的大典之仪,朝中上上下下,从今年的四月时,便开始忙碌了起来。不仅仅是各项事务的准备工作,其中的典礼仪式也要做好预定安排。最后,最关键的要祭祀的对象,还未有作出决定。

君子之泽,五世而斩——这说得是诸侯、大夫,除了始祖之外,只需要上溯四代祖先去祭祀。

七世之庙,亲尽而祧——这是天家的礼制。除了始祖以外,每一任天子只从他开始上溯六代去祭祀,更早的祖先神主,就从宗庙迁到祧庙里去。

现在朝廷上下,正围绕着禧祖文献皇帝赵朓该不该毁庙,而争论不休。

赵顼其实对这些繁文缛节也挺烦的。可这是朝廷大典,弄错一点,不仅仅是不敬先祖的问题,传扬出去,民间都要议论纷纷,而辽夏等外国,也都是会嘲笑的。事关重大,也只能让赵顼继续烦闷。

禧祖究竟该不该将神主迁去祧庙?

现在是众说纷纭,争论的关键,是禧祖赵朓到底算不算是大宋的始祖。

大宋天家传承,按如今通行的说法,第一代是圣祖赵玄朗,然后不知传了多少代,到了赵朓。禧祖生顺祖惠元皇帝赵珽,顺祖生翼祖简恭皇帝赵敬,翼祖生宣祖武昭皇帝赵弘殷,最后宣祖生的,便是太祖皇帝赵匡胤。

所谓的圣祖赵玄朗,是真宗皇帝所创,只为了压上李唐攀上的老祖宗李耳【老子】一头。最早一代被追封的皇帝,是开国时太祖所定的禧祖,是赵匡胤的高祖父,这是照规矩上溯五代追封。

只是现在,从禧祖开始往下算,赵顼已经是第九代了,上面有着八世祖先。一代代的排下来,祭祖时,这么多神主,在宗庙中也不好摆。照礼制,现在就得迁移一世先祖出宗庙,留下七庙——也就是禧祖,该从宗庙中迁走,迁到祧庙待着。

照赵顼想来,这件事只要太常礼院给出个合情合理的回复,两府、两制再讨论一下,差不多就够了。偏偏有人夹缠不清,说禧祖是大宋始祖,不能迁庙,该走的是顺祖皇帝。围着这件事,讨论范围扩大到了侍制、台谏、礼官。

为了此事,朝堂上下,断断续续吵了有半年之久。

赞成禧祖迁庙的那一方,拿出汉朝的例子,说汉高祖之父虽为太上皇,但并未以其为始祖。而反对一方,则上溯到更早的时候,商周之时,并不是以汤和文王为始祖,而是以封国之始的契、稷二人为始祖。

为了此事,朝中重臣把新旧两党的区别丢到一边,另分作两派,上书争辩。最后还是王安石做了结论,无功者不可为始祖,本朝始祖为太祖。禧祖当迁庙。

不管怎么说,这是天家的大事。赵顼现在有了结果,也要跟太皇太后、太后汇报一下。

赵顼先去了高太后居住的保慈宫,不出意料的看到二弟赵颢也在。没有多说什么话,问候母后、兄弟之后,三人便一起前往慈寿宫。

这几日天气倒是好,虽然冷了一些,但天上澄蓝澄蓝的,看不见一丝云翳。阳光落于宫廷中,晒得人暖洋洋的。

曹太皇半躺在一张软榻上,阳光从窗外照了进来。已近六旬,太皇太后越发的见老了,她从十六岁开始侍奉仁宗,几十年都在宫中度过,到如今对外面的世界已经很陌生了,但她所顾念,还是这个仁宗皇帝留下的这个国家。

只是眼下,让她担心的事,有很多很多。

看了赵颢又进了宫来,曹太皇眼中闪过一丝让人难以觉察的不悦。有哪个出外的亲王能天天进宫的,老四从来都是老老实实的待在王府中,就是这个二哥,天天去保慈宫报到。

心头的不快被遮掩得很好,曹太皇听着赵顼慢慢的将着朝臣们商议好的宗祀新制,以及如何处置禧祖宗庙的结论,都一五一十、不厌其烦的跟她说了一通。

听完之后,曹太皇没有说好,也没有说不好,而是颤巍巍的站了起来。赵顼一见,连忙上前扶着她。走到窗边,看着外面的天气,太皇太后回头对赵顼道:“天朗气清,若是大礼日也是如此,乃是大庆也。”

赵顼点点头,深有同感:“娘娘说得是。”

“老身过去侍奉仁宗的时候,听闻民间疾苦,必会诉于仁宗,每每德音因此而降,今次也当如此。”

赵顼神色变得冷了点:“今无他事。”

曹太皇转过身,在赵顼的搀扶下,回到坐榻上。抬头看着身前侍立的皇帝,“老身听闻民间甚苦市易钱、免行钱,官家还是趁今次宗祀后的大赦,将之尽数罢去。”

话题不出意料的转到了新法上,赵顼心情顿时又变得糟糕起来。耐下性子,对他的祖母道:“此诸法,多有利民,贫民岂有苦之。”

曹太皇叹了口气,这个孙儿就是个固执到底的性子,为了大宋基业,什么都可以不管不顾。可他不想想,国库充盈的确是好事,但国家的安稳不单单是在国库上。即便国库库房盖了一间又一间,但若是上上下下都一片反声,他这个位置怎么能安坐得下去。

她老婆子虽然坐在宫中,但眼睛还是能看到东西的。下面已经是暗流汹涌,已经让她不得不提点一下了:“王安石诚有才学,为相经年亦是劳苦,然其怨之者甚众。官家欲爱惜保全,不若暂时出之于外,待一两年之后复召用之亦可。”

曹太皇的老生常谈,赵顼越发的不耐烦起来,“群臣中,唯有安石能横身为国家当事。新法非其不行,熙河非其不得。如今国事日盛,正是安石之功!”

赵颢见兄长和祖母之间的气氛变得僵硬了起来,便上前一步,对着赵顼道:“太皇太后之言,至言也,陛下不可不思。”

“是我在败坏天下吗?!”赵顼见着弟弟当着面卖好太皇太后,心头火起,口气一下变得杀气腾腾,眼神也危险起来,“待汝自为之!”

这话一出,高太后脸色全然都变了,这话哪是能随便说的。“大哥!”她又急又怒的叫着。

曹太皇先横了赵颢一眼,又叹了口气,对赵顼道:“官家,此话不当说。”

……………………

“最后怎么样了?”

韩冈从王韶那里得知了昨日慈寿宫中发生的这一出,听到天子赵顼竟然说出了‘汝自为之’这句话,立刻就追问起下文。

“什么怎么样了?”王韶反问。

“当然是问雍王啊……”韩冈瞪大眼睛,“天子可是说了‘汝自为之’啊!”

“雍王说了句‘何至是’,然后哭了一场。”

韩冈楞了一阵,“这就没下文了?”

“还要有什么下文?!”

韩冈咂了咂嘴,摇摇头:“……燕懿王那还真冤。”

王韶咳嗽了一声:“玉昆……”

王韶提了警告,韩冈也便不说了。

不过赵顼说的这一句,百年前曾有另外一人说过——太宗赵光义。时间是攻打幽燕而不果的高梁河大败之后,地点是东京宫城中,人物呢,则是太祖皇帝的次子赵德昭。

早在高梁河兵败之时,军中不见赵光义的踪影,当时就有人准备拥立赵德昭。等到赵光义安然回到京城,一直没有给前面攻克太原、灭掉北汉的将士赏赐。赵德昭去劝说,赵光义便回了一句‘待汝自为之,赏之未晚。’。听到这话后,赵德昭回去后就拿着刀自尽了。然后放心下来的赵光义大哭一场,便追封了他做燕懿王。

哪知道,同样的一句话,反是雍王一句‘何至是’——何必说到这种地步,哭上一场就没事了。

‘汝自为之’,天子能说出这等话,可见心中已经猜忌到了极点。雍王倒是胆子大,哭哭就当什么事都没发生。

韩冈真的是很遗憾。这位二大王也真是不干不脆,要是跟着燕懿王一样拿刀子自裁那就有趣了。

王韶也能猜得到韩冈在想什么,叹了口气,“若是雍王真如燕懿王一般,恐有伤天子仁德。”

韩冈嗤嗤一笑:“唐太宗可是仁君……”

一年只有十三个死囚,斗米三钱的贞观之治,唐太宗当然是仁君。赵顼可是一直都是想要学着唐太宗,若能他学到三五分,韩冈就有乐子看了。就算学不成李世民,学学今朝的太宗皇帝也行啊……

“玉昆你啊……”王韶无奈的摇了摇头,这还真是唯恐天下不乱。不过韩冈能在自己面前畅所直言,也可见对自己的信任。这一点,王韶倒是乐意见到。

韩冈也不再对这事再说什么了,反正也不犯什么忌讳,当着天子面都能说的。

虽然韩冈从他记忆中的那点历史知识里,可以确定赵颢不会有登上皇位的一天,但说不准那天历史就变了样。要是上朝时看到坐在御榻上的是二大王,韩冈恐怕就要准备流亡海外了。

‘算了!’

这话赵顼能脱口而出,可见已经对赵颢深深的提防了起来。兄弟情分还有多少,基本上谁都能看得出来了。只要赵顼能活久一点,儿子也早点生出来并养大。赵颢就没有做上九五至尊的机会。

只有韩冈,也就不需要在这里杞人忧天,或是唯恐天下不乱,读书才是正经。

