\section{第八章 朔吹号寒欲争锋(14)}

“这个王朝云,虽是女流,又是乐籍出身,不过倒是难得忠心。其他侍妾都忙着逃出生天,就她不肯出去。”

张璪轻轻拍着手中的奏章,对在座的两名同僚说着。

按照法令,官员犯法之后,其蓄养的侍妾侍婢,皆尽发遣,只有名登族谱的妻室才会受到惩处。

苏轼既然成了大逆案的从犯,苏轼的妻子也就全都给收进狱中。但他的一众侍妾,在查明与案件无关之后,便一个个都放了出去。王朝云并非苏轼妻室,只是侍妾,而且还没有为苏轼剩下子嗣,现在却是死心塌地的要留在狱中,照顾主母。被强行架出去后,硬是留在开封府门前痛哭。

事关重案,当事人又极为出众,才两天的时间,就闹得城中尽人皆知,连报纸上都开始报道此事。甚至有传闻将王朝云此举,与沈括连着几日告病联系在一起。

沈括本就因为家里的葡萄架子而焦头烂额,现在又是遇上了这么一场无妄之灾。赶又不好赶,关又不好关,在整件事传遍了京龘城之后,他对此便不敢擅专,上书请求朝廷决定。

“国家自有法度,犯法之人不能脱狱,无关人等也不能随意关在狱中。王朝云非是苏轼妻室,她本人想留也不该留!”韩绛说道。

其实这件事根本没必要让大宋的宰相、参政浪费上半刻时间,可开封府的奏章上既然已经提了此事,宰辅们就得将自己的处置意见写上去,以供太后参考。

“将她安排在临近的尼庵中,容她去探视苏轼,并照看苏轼妻室。”

“玉昆,你对苏轼倒是宽待得很。”韩绛对韩冈说道。

“忠孝之举,本就值得奖誉。正好也能反衬出苏轼的所作所为……何况韩冈不做,章子厚也会做的。”

韩冈如此安排,却非为了苏轼。

既然王朝云愿意为苏轼付出,就让她实现自己的愿望。对韩冈来说,只是顺水推舟而已,也让记忆中的故事,在时间、事龘件都变得完全走样的情况下,依然能有着原本的模样。

“不过沈括也是,这么点小事就办不好?”张璪抱怨着,却拿起笔,在一张之后写下方才韩冈的意见。

虽然这种事不值得让日理万机的太后浪费时间,但一些奇闻轶事,让太后看着散散心也是好的,免得将精神放太多在与政事堂争夺权柄上。

“沈存中现如今快结案了,无法分心。他也是太爱惜羽毛了。”韩冈说道。

韩绛立刻呵的一声嗤笑,完全不在意正当着韩冈的面。沈括要真的在乎自己名声,当初就不会反复不定,哪边势大就往哪边倒了。

韩冈脸也不红,继续为沈括辩解:“知错能改,善莫大焉。”

两句话的功夫,张璪已经收起笔,将沈括的奏章放到宰辅们已经写好意见的一堆奏章中,听到韩冈说,不屑道:“沈括到底能不能改,还是问他家的张氏吧。”

韩冈都不好为沈括辩解了。

朝廷中这段时间本就忙得很,而开封府更是负担着赵颢、蔡确大逆案的审判,沈括却突然间请两天的病假。如果是真的生病了倒也罢了,但他哪里是生病,分明是被葡萄架子砸伤了,没法儿见人。

昨天就有御史上本说沈括闺门不肃——并不是说沈括头顶上的帽子换了颜色,而是他对妻室管束不严,有违礼教——如果定案的话,沈括最轻也得罚铜。

韩绛也是不屑的一笑:“家里的浑家都治不住,还指望他能制得住三军和外敌?”

“还可做房玄龄。”韩冈笑道。

房玄龄怕老婆是出了名,顺着韩绛的话,沈括就可做宰相了。

韩绛摇摇头,对韩冈坚持回护沈括大感无奈,“沈括的儿子还在玉昆你的门下吧?”

“沈存中的长子博毅去年就上舍及第了,次子清直如今正在横渠书院读书。”

韩绛摇头,也难怪韩冈会回护沈括,而沈括又会坚持投韩冈的票。

张璪这时拿起一本奏章,来自于开封府,是关于一众大逆案人犯的财产问题。

由于此时还没有定案,当然还不可能抄没家产,所以犯人们的家产仅仅是封存起来,给贴上封条。

派出去封存财产的官吏只是走了那么一圈,各家少说也有半数浮财落入了参与者的手中,也就是一干犯官的家产加起来也不算太多,远远比不上三位已经抄家的宰辅,更不可能到‘和珅跌倒,嘉靖吃饱’的等级,也没人会去计较这点损失。占大头的地产、田产不损失,就没有问题。

只不过那些犯官的家中,如今都空无一人,尽管有封条封门,可京龘城百万人口,少不了一些不肖之徒。好几家都被偷儿摸得一干二净。开封府上奏,表示府中人手严重不足,需要朝廷加派士卒来看守门户。

这是请求增兵的,顺道推卸责任,而且后者更重要一点。

“该如何处置?”

“兵给他就是了,但贼要抓到。”韩绛一声冷笑。

张璪又问过韩冈的意见,见韩冈不反对,便随手写了几个字,准许了沈括的请求,但要求开封府要尽快抓住贼人。

“曾布家还有两个在室女……”

韩冈指着另一份奏章,依然是来自于开封府,说得是曾布、薛向两人妻女的处置。

曾布、薛向,早早的就被确定发配交州,所以开封府那边还没有定案,两人的家眷都已经开始发落了。

张璪道:“在室女若已定了人家,可先询问对方是否愿意履行婚约,若不愿娶回,也只能依照法令了。”

“玉昆你看呢?”

韩绛和张璪都知道,曾布家的女儿是王安国妻子的亲侄女,韩冈又是王安石的女婿,也算是亲戚。特意提起此事,必定是想解救的。

“理应如此。不过若男方不愿践约,也不必送入教坊,一起跟着南下便是。曾布妻魏氏,薛向妻柳氏都可以如此安排。”

曾布家的女儿,多半已经聘人,若是男方愿意娶回去,韩冈也不觉得有必要硬是让良家女子沦入贱籍。即使不愿意,也没必要送人进火坑。

“留在京师尚能活命,去了南方可不一定能熬到明年。”张璪道,“男丁须远流,女子能安居,这本就是律法宽容之处。不见曾巩、曾肇流放岭南了吗?”

韩冈苦笑了起来,与王韶当初说法真没多少区别,生命和名节之间,的确不好做决定,“让她们自己选吧,留京在教坊,或是南下随夫、随父。”

“也好。”韩绛没当回事。又不是什么大龘事,而且韩冈的提议,在太后那边一句话就能通过。

“说到曾布……”韩绛又说道,“曾巩、曾肇这两人,朝廷处断得重了。”

“的确。”韩冈点头。

曾巩、曾肇这两位曾布的异母兄弟和他们的儿子,因为是男丁,故而被发配岭南,只是没有交州那么远,而是雷州、新州——‘春、循、梅、新,与死为邻;高、窦、雷、化,说着也怕’里面的雷州、新州。

曾布算是主犯之一,只比蔡确低一级,他能逃过一死,的确是朝廷的宽大,不过曾巩、曾肇两人的判决的确是重了。

当时朝廷议论的是如何处置曾布、薛向,由于之前耽搁了太多的时间,最后做决定时太过匆忙,对判罚不及细究。另一方面也是曾布、薛向的判决实在是太轻了,十恶之罪都能逃了一命;所以在他们的兄弟子侄身上做了补偿。

“但不好改了。”张璪说道。

如果以对曾布的判决为标准,曾巩、曾肇最多也只是追夺出身以来文字,削职为民。但是用御玺盖上了红印的诏书,是不可能简单的收回,更不会为了几名叛贼的亲属而收回。

“可以等以后大赦时让他们能回来。曾布、薛向遇赦不得归,但曾巩,曾肇并没有。”

“嗯。”韩绛轻轻颔首,也不知他是为谁出头。

包括曾巩、曾肇在内的曾布、蔡确两名叛逆的近亲,全都是发配了岭南。

在京内的,早已上路。在京外的,就算距离最远一位,现在也应该已经被派出去的使者收捕归案,押解南下。

反倒是对蔡确亲族,以及其他党羽的审判,一直拖到现在。

在元佑元年的礼部试即将开始,而第二次廷推也近在眼前的时候,对一众叛贼党羽,以及叛逆亲族们的审判终于告一段落。

尽管沈括因‘病’耽搁了几天的审理,但朝廷对他的要求并非是穷究,而是尽快结案,而且在王朝云一事后,他也怕再闹出什么幺蛾子的事,所以当权知开封府带着依然显眼的指爪印,在大堂中坐了六天之后,赵颢、蔡确大逆案,便有了一个结果。

来自开封府的卷宗,在政事堂中厚厚堆了一摞。

从犯人的自供,到证人的证言;从审判时的记录,到沈括亲笔写的判词;还有数以千计的证物的详细单据,与大逆案有关的一切都在这里。
