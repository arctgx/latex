\section{第四章 岂料虎啸返山陵(五)}

闰四月的京城彻底的进入了夏季。

赤日炎炎,阳光直射下,仿佛能将地面给晒裂开来。水面上荷叶亭亭,但看着一朵朵或粉或白的荷花,却一点也解不了心头的暑气。

倒是如今朝局却冷了下来。在王安石即将回归的现实面前,两边都失去了继续争斗下去的原动力。

韩冈今天上殿,将军器监几个锻造作坊迁移出外的进度奏禀天子的时候,冯京和吴充都没有再多说什么,只是如平时一般冷淡而已。韩冈这倒不在意了,要是他们笑脸相迎,反而要疑神疑鬼了。

退朝后出来,吕惠卿倒是说了几句笑话,从外人的角度看起来,两人的关系很是亲密。只是韩冈很明白,他和吕惠卿之间是彻底决裂了。不知王安石回来后,会怎么处理与吕惠卿的关系。

只是回头再想一想,吕惠卿毕竟没有明着阻止王安石复相,而且前日在殿上,还自称要以身家性命担保王旁与谋反案无涉。不论他私心如何,但表面上的文章做得还是很到位的。王安石大概还是会继续重用于他,就不知道吕惠卿自己能不能甘心了。

煮后又放在井水里凉下来的绿豆百合汤,加了蜂蜜调味,掺了一点点碎冰,当韩冈回到家里的时候,。喝上一口,就沁人心脾。夏天赐冰,冬天赐炭,在京城为官,就有这个好处。

“天气暑热,不知爹娘和大哥他们这个时候上京,能不能吃得消。”自己在家中享受着清凉,王旖也不由得担心起要冒着炎炎烈日上京的父母。

“从江宁到京城是二十二程,就算蓝元震去的时候是骑马兼程赶路,但岳父上京的时候,也只能坐船,差不多要有一个月才能到。的确不是好时节。不过呢……”韩冈却又笑道:“岳父毕竟是上京为相,沿途的州县还是会好生服侍的。过了扬州后,哪一个州县没有冰窑?拿几块冰放在船舱里,一路也不会有什么大碍。”

王旖道:“记得几年前从瓜步过江,也是在夏天,不过那是在四月初。那时候,扬州就有卖冰镇乌梅汤了。”

“瓜步……京口……”韩冈沉吟一下,看看外面院子里被阳光晒得发白的地面,“现在是夏天吧?”

“嗯。怎么了?”

“可惜了。”韩冈咕哝了一句。

王旖奇怪的看了韩冈一眼,自己的丈夫是不是热糊涂了。

韩冈哈哈一笑,从对千古名诗的遗憾中摆脱出来,“仲元明天也该从白马回来了,这些日子不知道他有没有担惊受怕,要好生得准备一下,给他压一压惊。”

“官人说的事,回头我就让素心妹妹去准备着。”王旖点着头,笑得很开心。

赵世居、李逢谋反案已经定下来了。既然赵顼已经要让王安石重回相位,就不能落下宰相的脸面。沈括虽然性子软弱,但有了天子的支持后,要排挤掉范百禄的发言权并非难事。别说被牵连的王旁置身事外,就是跟赵世居有着直接联系的李士宁也只是被杖脊后发遣荆南。

不过其他人就没有这么好的结果了。赵世居被勒令自缢,子孙从宗室除名,李逢则是凌迟。其余被牵连进此案的从犯,医官刘育凌迟,将作监丞张靖腰斩,父母妻儿皆流放广南。所有与赵世居有过书信往来的官员,或罚俸、或降阶,无一例外的受到了惩罚。这也是要给天子一个交代。另外,一开始判李逢无罪的提点刑狱王庭筠上吊自杀,而首告李逢谋反的朱唐,则是得到了丰厚的奖赏。

这一桩荒谬的案子,以荒谬开局,以荒谬结尾。韩冈冷眼看着这一桩案子的开局和结束,心也越发的冷了起来。

再说另外的一桩与韩冈息息相关的厢军聚众为乱案,由于王安石太过于强势,赵顼还是需要一个反对派。所以针对冯京的这件案子,也给赵顼断了下来。最后领头之人判了斩首,

从这两件案子最后的结果来看,看起来赵顼是准备将朝局调整回到熙宁五年、六年的时候,在新法继续推行的同时,维持着朝堂上的平衡和稳定。

虽然不知道赵顼能不能如愿以偿,韩冈这边还是乐见其成,这是正常的官场生态。原本朝堂一分为二的状态,才是不正常的情况。

“就是大哥的身体了,不知他现在怎么样了……”

“京中的名医多,有他们照看,元泽能比在金陵时得到更好的调养。”

“还要多吃点蜂蜜,还有那个蜂王浆……是叫这个名字吧?”王旖问着韩冈。

“嗯,是叫蜂王浆。”韩冈的回答有点无奈。他日前只不说顺口一说罢了,没想到王旖就给记了下来。虽然王旖知道丈夫不是药王弟子,但他说得关于医学养生方面的话,却是信了十足十,张罗着就要找蜂王浆来。

韩冈哭笑不得,这时候,那里能常年提供不放在冰箱里面,就无法长期保存的养生补品来?

相对于蜂王浆,蜂蜜倒是好办了。上等的蜂蜜,保质期能有很长时间,又是做菜做汤炖饮子的好材料,韩冈家的厨房里总会备上一两罐。不过与韩冈记忆中的蜂蜜有个不同的地方,这个时代的取蜜是直接割了蜂房下来压榨,有时过滤不干净的蜂蜜里面,还有蜜蜂残骸——从卵到成虫一应俱全。

但蜜是好东西,在种粮之余,韩家现如今在陇西的田地也种些当地常见的芸薹。芸薹可以拿来直接吃,也可以等着开花后收籽来榨油。黄色的花,加上用来榨油的籽,韩冈基本上就可以确定那应当就是后世的油菜,有了油菜花,当然也就有了蜂蜜。

后世蜂箱的结构,韩冈还能记得一点,只是不知道具体的养殖方法——此时也有蜂箱,就当真就是个空箱子,让蜜蜂在里面筑巢,等出蜜的时候,直接将蜂巢挖出来榨蜜——所以没有多提,就按现有的方法养,照样能出蜜。过个两三年,陇西就能出产蜂蜜了。但蜂王浆应该是没戏的,最多也就是他正在陇西的父母,一年有那么几次机会吃上一点。

想想,由着王旖去折腾好了,反正也不是什么大事。

夫妻两个又说了些闲话,韩冈回到书房,从架子上抽出一封信笺,这是张载写来的书信。

最近关学好生兴旺,关中各地的士子齐集横渠那是不必说了,便是关东的读书人,也有许多不远千里的往横渠镇上去。但张载的信中却没有多提这方面的事,而是与韩冈商讨,如何处理韩冈对格物致知的解释,与天人之道之间的分歧问题。

韩冈一直自称在学术上只得一偏,更偏重于推究自然之理。真正贯通天人大道的,还是要数他的老师横渠张子厚。虽然张载至今未能再至京师,但早有无数士子心向往之。

可是科学与天人合一的理论毕竟是相背离的。张载在作为气学理论大纲的《钉顽》【即西铭】一篇中说:‘乾称父,坤称母……大君者,吾父母宗子,其大臣,宗子之家相也。’也就是将三纲五常与天地至道合二为一。

韩冈越是将后世科学理论一桩桩的用实验证明,就有越来越多的事实在清楚的表明,所谓的君臣父子之道与天地自然毫无瓜葛,扯不上半点关系。

不过韩冈是一步步来的,已经得到证明的一干理论在传播时,都是打着气学的旗号,两者早已紧密难分。也就是说,在不知不觉间已是鸠占鹊巢,将气学给绑架了。现在已经不是韩冈要将科学理论装扮成儒学的一个分支,而是张载要反过来拿气学理论,去配合韩冈已经验证的一整套科学理论。

不仅仅是关学,即便是二程的洛学,王安石的新学,都必须面对这个问题。谁也不能直接否定已经得到证明的几个科学理论。儒学是个十分现实的学说,不但要解释社会,也要解释自然。韩冈已是先入为主,格物之说的定义现在就在他的手上,不论是谁家的学说,都不能轻易的绕过去。

但韩冈明白自己也不能走得太远,超出时代半步是天才,超过一步,可能就要送命了。所以给张载的回信他犹豫再三,还是没能提笔落字。

叹了一口气,韩冈将张载的信重新收了起来,回信还要再想一想。

吕大防最近上京来了,这两天去抽空见他一面,说不定还能就此讨论一下。虽然吕大防并不是张载的弟子,但他的三位兄弟——大钧、大忠、大临三人——都拜在张载门下。而从学术上,吕大防也是贴近张载。在韩冈没有横空出世前,在朝野内外一力支持张载的就是蓝田吕氏这几兄弟。

另外还有张载入京的事,当面讨论也许会更合适一点。

但要快一点了,韩冈想着。

以王安石的脾性,绝不会将国子监交给张载来主持,韩冈也不会去奢望。但他还是想要张载上京,为气学张大声势。实在不行,以个人的名义请其上京,看谁还能拦着。

