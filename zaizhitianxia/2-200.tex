\section{第32章 吴钩终用笑冯唐(七)}

宋人重兵器。

在个人战力无法与北面、西面的敌人对抗的情况下,宋人自建国时起,就分外注重各种武器、战具的发明和使用。八牛弩就不用说了,化学武器性质的毒烟球、用来挖掘坑道的头车,还有正在大规模装备军队的神臂弓,攻城、守城、水战、陆战,在宋人军队中,林林总总装备着总计数十近百的各色兵械战具。

向朝廷进献与军事有关的发明,都能得到不小的回报。神臂弓的发明人李宏,虽然还是蕃人,却已经在京城了做了官。还有韩冈,他本人能被天子记在心里,有很大一部分就是因为他的沙盘和军棋。连那位在韩冈的指点下,打造了第一具沙盘的田计,原本只是一个泥塑匠,如今也混了一个官身出来。

韩冈前日在京中的时候,曾经在章惇那里听说因为神臂弓效用明显,李宏刚刚又被升了一级官。章惇当时都说,以李宏现在升官的速度来看,以及神臂弓的威力和在军中的欢迎程度,日后升做防团——也就是贵官中的防御使、团练使——都是有可能的。而以木征的势力,都已经成了河湟开边中的眼中钉,到现今在宋人这里也不过是一个河州刺史,而在西夏那边也仅是个河州防御使。

韩绛这个承诺的确是有诚意,不过对于韩冈来说,就算不得什么了。他更希望韩绛能推重他以数达道的想法,而不仅仅是把他对投石车的改进奏于天子。不管新型投石车的威力有多么大,在士大夫们的眼中,终究也不过是个高明匠人的手段,但别出心裁的学术见解,以理论透析器物,却是能在士林中掀起波澜。

对于此时士人轻视工匠之术的潮流,韩冈是希望能用理论将他们潜移默化,而不是与其直接对抗。只是宰相的善意是不便拒绝的,韩冈也不是不识好歹之辈,遂躬身向韩绛表示谢意。

韩绛他现在上前线来,是以视察攻城准备的名义,因而会来工匠营中走一遭。在门外听了韩冈的一番言辞,又看到了新型投石车的前景,兴致就变得很高,不顾污秽的在工匠营中转悠了一圈,种谔、燕达陪在他后面,韩冈本想退上几步,与游师雄,和跟着他叔叔的种建中走在一起。可韩绛却说要去下面要去看一看疗养院,让韩冈走在自己的身侧。

韩冈有些无奈,韩绛这是不遗余力地拉拢自己了,要是当初他能有今天一半的热情,和自己的关系也不会闹得那么僵。不过终究也是好事,韩冈想着,便跟游师雄、种建中打了个手势,又向种谔、燕达表示了一下歉意,越过他们走到韩绛身后半步的地方。

游师雄和种建中都是在看着他们的同门师弟。不卑不亢的走在韩绛身边,沉静如初,并没有因为宰相的看重而受宠若惊,士大夫的自信和自重在他身上表现得很明显。

在两人的印象里,韩冈才智过人、能力出众,无论是兵事、政事都有所擅长,而在军中医疗一事上的贡献,更是让他在军中的人缘没哪个文官能比得上。以韩冈此前的功劳,前途不可限量这几个字就是为他而量身订造的。

但韩冈从没有在两人面前表现出经义大道上的才华,直到今天。他自出机杼,别开蹊径,喊出了‘以数达道’的口号,自称要以旁艺近大道,其在学术上的见识和野心,却是游师雄和种建中想都不敢想的。

才二十出头,就放此狂言,往往会惹人嗤笑,偏偏韩冈还能说出个门道来。游师雄是从头到尾听了韩冈的解说,而种建中是跟着他的叔叔和韩绛,只听到后半截。不过不管听到多少,单是‘格物致知’,‘以旁艺近大道’这两句,韩冈的气魄和眼界已经崭露无疑。

格物致知的新解,是从张载、程颢而来。自从韩愈开始宣扬道统论,宋儒对于汉唐时通行的儒家经典的注疏,已经越来越看不上眼。如今学派林立,出来的理论都是把汉唐注疏丢在一边。

张载宣传天人合一,二程则说天人本无二,道有小异,本源却都是承袭思孟学派的源流,研究着万物自然之理,以人心体大道,试图将世间纲常与天道合而为一。

韩冈‘以数达理’的理论,游师雄在听过了他解说之后,已然有所领会。这套理论眼下虽然浅显,可只要能深入的阐发下去,当真用数和算式将万物之理给出一个明确且易于推演的解释,必将能成为天人合一理论上的一个关键的基础。而韩冈可能继承不了张载的衣钵,但将之发扬光大当是板上钉钉的一桩事。

此前游师雄和种建中都自持才华,绝不会认为自己会比韩冈差多少。可现在,他们心中隐隐的已经开始对韩冈多了几分敬意。

跟着韩绛视察过了疗养院,韩冈又得到不少赞许——虽然他是今天才开始接手这座疗养院的。当韩绛连几处兵营也一并视察过,回到主帐时,赵瞻已经在帐中等候。韩冈听种建中解释,赵瞻是跟韩绛一起来咸阳的,只是没下去陪韩绛走路罢了。

如果排除偏见的去看赵瞻,这位来自京城的使臣,也算得上是深具仁爱之心,并不是只顾争功的恶人。

虽然由于军事方面的才能缺陷,做得几乎都是蠢事,但他的目的就是把对百姓的损伤压到最低。无论是命令秦凤、泾原两路援军,在不毁损城下民居的情况下攻城;还是用围墙把咸阳城给包起来,防止叛军流窜关中;都是他仁心的体现。

可是结果虽不能说与其初衷是截然相反,但也算得上是大相径庭。就是因为赵瞻这样的人,都有同样的一个缺点——那便是自以为是!

只要认为自己的做法是对的,便会强硬到底,看不到别人反对意见中有用的地方,而是把所有的反对声,当作耳边风,甚至当成死敌。

郭逵给他气回长安;燕达给他逼得贸然攻城,损伤了上千精锐;种谔也被逼得放弃罗兀城。赵瞻的存在,对于平叛来说,是个最大的妨碍。

韩冈对赵瞻的跋扈还没有切身体会,但转眼一看燕达见到赵瞻后的脸色,几乎是眨眼之间,就从温文的笑容,变成了跟道边小庙里粗制滥造的神像一样,一点表情都没有,木然冰冷,种谔那边的神色几乎也是一个模子出来。

赵瞻在到了陕西后的一番作为,已经彻底的把这些西军中的高级将领,得罪得干干净净。这吸引仇恨的速度,这开罪同僚的能耐,韩冈也不得不想对赵瞻说一声佩服、佩服。

韩绛当先坐在了主帅之位上,聚将的鼓声随着他的命令当即在帐外响起。鼓声在瞬间传遍了环绕咸阳的各个营地,很快,统领各营的将领便一个个骑着马飞奔而至。

亲兵在帐外同名,将领们则一个个进帐来,行了礼,然后站到了自己的班次上。等营中众将官在帐中排定,赵瞻便当先出来,对着众人道:“相公今日亲来营中,尔等当好生戒备,勿要让贼人惊扰到相公!”

“怕什么贼人惊扰?反过来才是。”韩绛很明显的有了重新夺回了指挥权的心意,毫不客气的打断赵瞻的话,对着他道:“本相方才已经巡视过了营中,战具皆备,军心可用,当是可以出战了。”

虽然还有几天围墙和壕沟才能彻底完工,而要开辟地道,还要加上一个月的时间,但韩绛已经等不下去了。他方才在营中看了一圈,觉得以眼下的实力,已经足以在十天之内攻下咸阳城。

“现在攻城,恐怕还是仓促了一点。”赵瞻现在的行事分外保守,也不喜韩绛从他手上回权力,“当是再等几日,等地道挖出来再说,用兵当谨慎从事。”

“也不是说立刻就动手……”韩绛也算稳重,他现在也怕再出事,也不愿头脑一热就让人去冲城,“当是先礼后兵,先选一得力之人,去城中说服叛军出降。”

昨日已经送了一个大礼给吴逵,想必已经杀了王文谅的情况下,一众叛军心里的愤怒也该能散去了一点。这个时候招降虽不指望能一举成功,但肯定能让叛军坐下来。历来招降都很少一次成功,边打边谈才是正常的情况。等过几日,围墙修好,再把打造好的战具摆出来,练上一练,足以逼得城中叛军失去战意。

只是现在关键的问题是让谁去招降?

韩冈看看左右,却都是热切已极的神情。

在寻常人眼里,去叛军老巢招降是要冒风险的,但实际上,没有哪个叛贼敢于随便杀害代表朝廷的招降使臣。这是自绝后路,就算领头的贼酋想干,下面的喽罗都不会答应。只要招降使臣不犯浑,绝不会有大碍,反而是立功的大好良机。想想郭逵,他能一造青云,还不是因为他在保州去劝降了城中的叛军。

“玉昆……不知你对招降有何看法?”韩绛点了韩冈的名。

