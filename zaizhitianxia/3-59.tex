\section{第22章 明道华觜崖(二)}

定下赌约,杨绘虽然心急,却也不便立刻前往华觜冈。

宫宴还没有正式结束,至少要等进士们和上天子的御制诗后,才能前去。不过韩冈人就坐在这里,杨绘也不怕他能变出什么花样来。

琼林苑的管勾官这时听了召唤过来,杨绘吩咐着:“去准备一个十斤以上的石锁,还有一个一斤上下秤砣。”

管勾林深河已经四五十岁,官场上摔打了几十年,心眼活络,更会做官。方才就从手下的吏员那里听说了杨绘和韩冈的赌赛,当然不会就傻傻的等着命令。

林深河没出身、没后台、没才学,只是靠了家族中唯一做了州官的伯父的临终遗表,才被荫补了一个没品级的流外小官。熬了几十年,靠磨勘磨到了从九品,却没能攀上一个像样的贵人。虽然他活动的能力是有,但也只不过弄来了一个管勾琼林苑的差事,还是升不上去。而且头上还压了两个宗室出身的琼林苑提举、同提举,平日里事都是他做,却还要受闲气,几年来都是憋闷不已。

但现在终于有了个机会,自知正是他表现的时候到了。韩冈自不量力,已成了众矢之的,林深河当然不会站到那艘破船上。肯定是要帮着杨学士,为他好生出一口气。只要这一次拍好杨学士的马屁,做了身前的亲近,做了他门下的走马狗,日后说不定还有转官的一天。

林深河垂着手,半弯着腰,声音谦卑无比:“下官前面已经让下面的人去准备了,学士尽管放心。”

杨绘点了点头,道了句好。不过想了一想之后,又招了招手,示意琼林苑管勾走近一点。

林深河忙凑上前来,压着心头的兴奋,陪着笑脸:“敢问学士有什么吩咐?”

杨绘侧过脸,低声问道:“苑内可有黑狗?”

“黑狗没有,但有公鸡,为数不少。”林深河心领神会的神秘的说着,“公鸡.鸡冠血也能破邪术,下官已让人先行准备去了。

杨绘惊讶的回头看着这位知心可意的琼林苑管勾,就见林深河继续低声道:“下官想着,韩进士是孙真人的弟子,保不准会变什么术法,这么做也是有备无患。如果当真是如韩进士所说的自然大道,那一点公鸡血也不会有影响。”他望望左右,更凑近了一点,“下官这里还让人去准备了妇人天葵,到时与公鸡血一起抹上去,包管什么样的邪术都用不了。”

杨绘深深看了这位近五十岁的卑官一眼,口气不无赞赏:“办事倒是得力。”

“下官最恨赌中出术之人,只为了赌赛公平而已。”林深河说得义正辞严,一脸正气。

杨绘一笑,说到底,能帮翰林学士出力,哪有不屁颠颠的凑上来的,倒也不算什么了。“你叫林深河吧?我记下了!”

对于在琼林宴上闹出这一桩赌赛,殿中的每一个进士都是兴致盎然,各自低声讨论着,韩冈和杨绘之间究竟谁赢谁输。基本上都是站在杨绘的一边。用脚趾头想都能知道,越重的东西越沉,越沉的东西当然落得越快,怎么可能一同落地。不过还是有人觉得韩冈有那么一两份胜算,但其中并不包括慕容武。

慕容武作为张载的弟子,还有韩冈的好友,在众同年的讨论中,当然是第一个要受到咨询的。他完全不能认同韩冈的说法,这也因为他比韩冈早一个月上京,并没有在韩冈去横渠镇时,在旁聆听韩冈对于力学三律的一番解说。

所以当韩冈和杨绘打起赌来的时候,他想阻止,却没能来得及。现在众同年过来相问,他明明心中直在摇头,还偏偏得站在韩冈这一边。回答的时候就免不了很是勉强,让众人都看在了眼底。尽管他的回答,全是帮着韩冈,但每一个看到他表情的进士,都摇着头。

“已经没得赌了。”邵刚对余中摊开了手,摇头叹道。

余中也叹了口气,好好的琼林宴变成了赌场,身为状元的他,当然不会乐于看见。而韩冈所面临的境地,余中都是要敬而远之。他望了一眼,独坐原位、无人敢近的韩冈。这一科名声最响的一人,今天可就要折戟沉沙了。

“可惜了。”余中的低声呢喃,说不出喜悲。

吕惠卿看了一圈殿内的情况,转身对曾布道:“看来就我俩在赌韩玉昆赢了。”

“那不是正好,可以通杀啊!”曾布笑着,瞥着正与管勾琼林苑的小官窃窃私语的杨绘,眼神中尽是鄙视。

曾布应该殿中最相信韩冈的一人。虽然在新党中,最为反感韩冈行事作风的就是他。但韩冈的才智,曾布却是最能认同。能在第一次上京时,就出了一个撬动天下大局策略的谋士,绝不可能在这件事上犯浑。而且在跟杨绘争辩时,话题都是由韩冈领着,怎么可能会出现自己造陷阱,然后自己跳进去的情况?!

吕惠卿也笑了一笑,他看了看食欲丝毫没有受到影响的韩冈,却又皱起眉来。虽然他赌着韩冈赢,但吕惠卿的心中,却怎么也想不通,为什么韩冈敢说十斤重的铁球会跟一斤重的铁球落地一样快。

“当真会是两个铁球或是秤砣、石锁的同时落地?还是韩玉昆会变什么术法?”他问着曾布。

曾布摇着头:“不知道,还是眼见为实吧。”

“眼见的可不一定为实。”吕惠卿道,“子渊攫灰而食,子见而疑之。先圣都犯错的事,我等凡夫俗子,如何能做到?”

子渊就是颜回。孔子率弟子周游列国,在陈、蔡之地被困,粮食已尽。颜回出外找到一些米回来,烹煮时房梁上有灰尘落尽锅中,颜回将沾了灰的一点米捞出来吃了,却被孔子看见,便被误认为是先师长而偷吃,非礼也。一直到颜回解释清楚后,孔子为此而叹道:‘所信者目也,而目犹不可信。’——原以为眼见为实,谁知实际上眼见的未必可信。

曾布则念着孔子紧随在后的一句话,“‘所恃者心也,而心犹不足恃。’这一句正合今日之事。韩玉昆说杨绘,就是说他是凭心臆测,到头来也不一定可靠。”

“‘知人固不易矣。’”吕惠卿背着孔子那段话的最后一句,冷笑道:“先圣不知子渊。恐怕王相公也没想到他这个女婿会有这一手吧?”

“但韩玉昆应该都算计好了。”曾布声音突然透着阴冷,“……想一想,今天这个鱼钩如果不是杨元素咬上来,你说韩玉昆是准备钓谁呢?”

吕惠卿闻言一怔,但深思起来,脸色也变了。以韩冈步步算计的性格,既然在天子面前推荐张载,必然有所依仗。只看他今天说话作态,就知道必然早有准备。杨绘只是运气不佳,想在韩冈身上表现,却反过来被韩冈利用上了。但杨绘仅是个送上门来的意外,以韩冈的为人,必定在之前就找好了牺牲品。

只是筹划阶段的经义局,如今确定了职位的只有两人。

“真是要多谢谢杨元素了。”吕惠卿幽幽说着。

“嗯。”曾布说得更为直白,“杨元素的确是帮你挡了灾。至于王元泽,韩冈这个妹夫是不会跟他过不去的。”

三巡酒后,众进士为天子的御制诗写了和诗。四百多篇七律,并没有什么出彩的,而韩冈的一首也还能凑活。但以杨绘的眼光,肯定是看不上,如果没赌赛的事,他当是要摆出文坛前辈的姿态,好好落一下韩冈的面子,这样也算跟北面的两位有个交代。不过现在,就不需要为此多费唇舌。

杨绘起身,说了几句场面话,就一马当先,雄赳赳、气昂昂的往华觜冈走过去,韩冈紧随其后。已经等着这个节目等了许久的几百号人,也都一涌而出,一起跟着往琼林苑东南角的高丘而去。

华觜冈高约十多丈,是琼林苑中挖了金明池后,用土石垒起来的几座高坡中的一座。在华觜冈陡峭的北侧悬崖下,有着一汪清池。湖面不大,比左近的金明池要小上许多。但正好就在华觜冈上,那座高楼延伸出来的外廊的正下方。站在外廊上,韩冈手扶栏杆向下望去。波光粼粼的池水,离着他估计有着五十米的距离。

上得高台的并不多,大部分进士都在池边等着。二楼、底楼也用着一群人。而能站上三楼外廊的,基本上都是参加宴会的朝官,还有今科的状元和榜眼——官场上等级森严,任何时候都体现得很明白。

除了几名小吏,楼台上唯一的一名卑官,就是琼林苑管勾林深河。他为这场赌赛准备好了实验物品:“……石锁倒没有。这一块,是抵门石,约莫有三十斤重。而这块秤砣,则是正好一斤,乃是厨中所用。”

“玉昆,可以吗?”杨绘问着韩冈,眯起的双眼、翘起的嘴角,上面写满了得意。

韩冈看了看放在地上的两件试验品,用脚推了一下,感受了一下重量,觉得没有问题,便点了点头。只是心中有些奇怪,为什么他踢两件东西的时候,琼林苑管勾会有一下提心吊胆的神色掠过。

不过这些都是末节了。在林深河的指派下,两名小吏一个抱起抵门石,一个拿起秤砣。楼上楼下一下变得安静了,所有人的目光都集中到他们的手上。但就在这时,一声高喝远远的传来。

“且等一等!”

一匹奔马,从琼林苑大门处直奔华觜冈而来,看骑手服色,竟然是个宦官。到了楼台下,那内侍下马,沿着楼梯跑上来。气喘吁吁。韩冈看过去,竟然是童贯。

童贯喘了两口气,对着惊讶不已的官员们高声道:“御驾转眼就到,天子有诏,此事稍停片刻。”

闻言便是一片喧哗,竟然天子要来!

