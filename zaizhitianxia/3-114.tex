\section{第30章 众论何曾一(七)}

熙宁七年的上元节也算是平平静静的过去了。

比往年要平淡一些的上元灯会之后,东京城中,如今议论得最多的,还对那三十七名奸商的审判。

且不说构陷二字有多好写,就是只算实实在在的罪名,真的要追究起来,粮商们各个都是一屁股的烂帐。作为御史台的第二号人物,蔡确奉旨领头审了近一个月。弄出来了一长串罪名,罪状多到要申请分开来另案处理的地步。

看到有份旁听的吕嘉问拿来的厚厚一叠供状,王雱看着惊奇:“想不到罪状这般多,蔡确是怎么拷问出来的?”

“三木之下什么口供得不到?不过蔡确可不是这般糊涂的人。”吕惠卿当先接过供状,当先翻看了看起来。

“嗯,说得也是。”王雱点了点头,想起了去年的这个时候,自家老子被蔡确捅的那一刀子,当得起‘稳准狠’三个字,“不知蔡确给粮商们定得什么罪?”

吕惠卿看着第一页:“占盗侵夺他人田产,三十七名粮商中人人都不缺。”

王雱一听就觉得不对劲:“这算什么罪名?!在官侵夺公私田者,最高也就徒两年半!”

吕惠卿没理会,翻过一页,“校斗秤不平,人人皆有之。”

吕嘉问道:“一干粮商改动店中秤斗售粮,从中牟利。依律校秤斗不平得利赃重者,当以盗论。粮商们差不多都是贪了几十年的,赃款也是几千几万贯。”

王雱摇着头:“窃盗之罪,流刑也就到顶了。修桥铺路的善人少见,为富不仁者则举目皆是。若以斗、秤之物论罪,当真根究起来,东京城中大半商贩都能给捉入大狱。”

“可不止这一些。三十七人中,居丧生子十一人,父母在别籍异财四人,居丧为婚者一人。”吕惠卿停了一下,“这里还有诈乘驿马……”

“一辈子的罪全都给拷问出来了!”王雱猛然哈哈大笑起来:“有没有不惜字纸,礼佛不敬?蔡确还真是本事,全是鸡零狗碎的罪名!”

这一串罪名看着多,其实也就是杖责而已。而判罚不到刺配一级,都是可以用钱来赎,的确正如王雱所言,就是鸡零狗碎。

“倒也不能这么说。”吕惠卿道:“有谋杀之罪者,二人。唆使部曲殴人至死者,三人。”

王雱的笑声嘎然而止。这一下罪名就重了,谋杀之罪基本上就是论死,唆使致死也是一般。

吕惠卿一页页翻着供状,平直的声调继续念道:“犯奸者六人,其中奸父妾者二人,奸兄妇者一人。”

奸父妾是重罪,违反伦理纲常。属于十恶不赦之罪中的内乱,通奸者绞,强奸更加一等,都只有死路一条。

“内乱者绞。至于私通兄妇……”王雱回忆着刑统中的律条,“是三千里流刑吧?”

“和奸两千里,强者加一等。”吕惠卿更正着,接着念道:“私有禁兵器者五人,其中三人藏弩过五张,一人甲胄二领。”

私藏兵器同样是重罪,有谋反的嫌疑。弓、箭、刀、盾、短矛,这些寻常的兵器民间可以持有,北方人家基本上都能找出一两张弓来。但长兵不可收藏,劲弩不可收藏,而甲胄更是严禁。依刑统,私藏甲三领或弩五张,就可判绞刑了。

“不过犯了这几项罪名中有重复的,依律当论死者五人而已。”吕嘉问在旁解释道。

王雱听着不住摇头:“正经的罪名不去根究,却在这些零碎之事上做文章……”

“也有正经罪名,把持行市啊!”吕惠卿虽是如此说,嘴角却是不由自主的向下撇着,“蔡持正定得好罪名吧!”

王雱立刻冷笑起来:“把持行市得利多者以盗窃论,但其罪是免刺……不会有流配!这个罪名还真是重!”

吕嘉问叹道:“谁让在刑统上,囤积居奇的罪名找不到呢……”

吕惠卿道:“张乖崖以一文钱杀库吏,‘一日一文,千日一千,水滴石穿,绳锯木断’,这判词没人说他错。律法不外人情,真要致其于死,即便律法上所无,也完全可以加以处置。更何况当初京中粮秣供应充足,而物价飞涨,那是因为有谣言传世。由此入手,一个死罪也能定下来。”

“没错!这一干奸商囤积居奇,致民惶恐。勾奸生利,动摇国本。加上妖言惑众这一条,挂上谋逆都可以的。”王雱狠狠的说着。

一般来说,朝廷对付豪商们囤积居奇的正常做法,都是利用经济手段,而不是暴力。如战国时李悝的平籴法,西汉时桑弘羊之均输法,王莽的五均六筦,几乎都是利用手中的权力,通过行政力量来打击豪商囤积居奇的行为。

而韩冈和王雱的计策,则是改从民心入手,裹挟民意以制奸商。这也是时势所迫,否则要想用经济手段解决问题,除了开常平仓,别无他法。就算是和籴——也就是官府强行征购民粮——也动不到与宗室有亲的豪商们头上,到时候,反倒是中小粮商吃苦。

但蔡确在罪名中根本没提这一茬,可以看得出来他就是在帮着粮商们开脱。但他做得很聪明就是了,所列出来的一系列罪名,往重里说,也能将粮商们尽数远窜四荒,但宽纵起来也很方便,毕竟没有栽上十恶不赦的罪名——只除了几个被审出犯了死罪的。而三十七名粮商中,有了五名干犯重罪的,完全可以拿他们来开刀,在民意上就能有所缓和。

“蔡确当真是聪明。”吕惠卿感叹道。

在这一案中,蔡确表现出了自己的刚直不阿和严守律法,且又给了天子宽纵赦免的余地。只看他这一手段,的确不是普通人物。而且蔡确之前因庭参礼一事而得到王安石看重,又因宣德门之变而得到天子青睐,每一步都算计得恰到好处。揣摩上意的心思,用单纯的见风使舵来评价,就显得太屈才了。

王雱抬头从窗户中望了一眼政事堂主厅的楼阁,他的父亲正在厅中与其他宰辅们讨论着军国大事。如果王安石看到这份供状,必然不肯干休。

若说处置,依眼下的罪名,的确可以将粮商们置之于法。以罚赃的名义,将之前抄没一百三十万石存粮的行为合法化。但对于王安石和新党来说,如此论罪等同于混淆是非。不能将囤积居奇的行为处以重罚,而是别以他罪来惩治,那么日后……或者说就在几个月后,又有什么条律能阻止商人们的贪婪?!

在主审蔡确的放纵下,粮商一案的审判很快就得到结果。

三十七名粮商中,除了几人重罪难赦,被处以绞刑外,其他都是判了流刑或是徒刑,为首的九位行首甚至连刺字都没有,从律法上可以缴了罚金就此开释,只有那一百多万石的粮食被当作不当之利而被罚没。

但王安石登时将之驳回,并说粮商们犯了妖言惑众一条,当置于绞刑。几乎所有的粮商,都曾说过如今大旱乃是朝廷德政不施,所谓‘妄说吉凶’之罪,用以惑众而取利,绝不可以饶恕。

这几天朝堂上正在争执着,御史台、开封府还有审刑院都维持原判,而王安石则坚持己见,要将为首者重惩。民心士论多偏向王安石,而诸法司则维护着他们的权威,天子没有开口,局面一时争持不下。

对于这一件案子,京中官吏众说纷纭。曾布则是觉得,天子的心意已经很明白了,王安石要将之顶回去,几乎不可能。

坐在三司的公厅之中,曾布听着派去市易务小吏的回报:“禀学士,吕提举说此事早前奏禀中书,已得王相公和吕检正的批复了。”

对于小吏的回答,曾布不动声色,从面色上看不出喜怒,“也罢,你先下去好了。”

厅中只剩曾布一人,积蓄在胸中的愤怒从颤抖的手上曝露了出来。吕嘉问的确越来越跋扈了,他可是市易务的顶头上司,竟然所有事都跳过他,直接呈递给中书。

不知过了多久,曾布抬头对外唤了一声,将门外听候指派的小吏叫了一名进来:“去唤魏继宗来见。”

魏继宗乃是市易法的提议者,由布衣而得官。之后吕嘉问提举市易务,从一开始的建议到后来的各项条令的增损措置,都有魏继宗的参与。但如今魏继宗却不知为何,被吕嘉问排斥在外,自此不得参与市易务中事。如今他就在三司之中无所事事,干拿着一笔俸禄。

过了片刻,魏继宗过来报到,向曾布行过礼,起身问道:“不知学士着下官来可有何吩咐?”

“河北自去岁旱灾,至今未有雨雪,天子忧心不已。本官已受命去河北相度市易之事,并察访当地民生灾情。只是市易中事,本官多有不知,需要一个熟悉个中情弊的人为助力……”曾布话说到这里,便停了下来。

魏继宗愣了一下,抬头看着同判三司平静的看不出任何一样的神情,顿时全明白了,立刻躬身行礼:“下官明白,愿为学士效犬马之劳。”

“不是为我,而是为官家!”

