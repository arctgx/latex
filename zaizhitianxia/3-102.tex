\section{第29章 百虑救灾伤(七)}

高扬和金平骑在马上急匆匆地往粮行的会馆赶回去。跟着两人的伴当也骑在马上,一行七八人,脸色一个赛似一个难看。

高扬刚刚喝了一坛子的锦夜白。因为是平日最是悭吝无比的吴楼掌柜燕四白送的好酒,他喝得极是开心。只是现在骑在马上,急急的往回赶,整个人上颠下晃,肚子里的酒水就一个劲的往喉咙上涌。

直到前面人多了起来,不得不放慢马速,高扬一直在翻腾的胃部这才感觉好一些,不过心里面泛着的堵,却是一点也不见减少。

来报信的亲信紧紧跟在身后,马蹄声一点就追在耳边响。方才他从楼下跑上来,高扬和金平正是喝着开心的时候。听到也只是抬抬眼,漫不经意的问着有什么事。

“马车,发运司用马车在河上运粮!昨日已经到了南京!”

当慌慌张张的这句话传入耳中,高扬就想一个巴掌将说胡话的家生子打醒。可旁边的金平听着听着就脸色变了,“莫不是雪橇车!?”

高扬的醉意由此也一下全都醒了,紧接着,一阵寒意传遍全身。

当侯水部的四条碓冰船在黄河中挤成了木片的时候,哪一个粮商不是想看着王安石第二条手段的笑话?只是为了有备无患,行会才派了人手去南京应天府【商丘】打探——坐在汴河边守着,总能先一步得到消息。本来高扬只当是白出了一份人力而已,但他怎么都没有想到,这雪橇车竟然还真的给薛向办成了。

高扬心中发慌,即便在熙熙攘攘的人群,也是觉得惶惶不安。他转头瞅着旁边的金平,行会的大行首此时阴沉着一张老脸,当年他死了亲娘老子,高扬也没见他这副模样。

钱比爹娘重要——至少在高扬和金平眼中如此。他们以及整个行会,为了囤积居奇好在明年大赚一笔,这两个月不但刻意减少了粮食出售的数量,甚至还动用了大半家产来高价收购京畿一带大户手中的存粮。

今冬的物价大涨,只是他们在利用民心,逼迫朝廷开常平仓平抑粮价。等到断了朝廷所能动用的最后的手段,到了明年的春夏时分,便是粮商们大发横财,为子孙攒下一辈子都赚不到钱财的时候了。只要将赚到的钱分给亲家们一部分,还怕朝廷能查抄到自己的家里去?那时候,王安石肯定要倒台,有什么罪过都可以推到他身上!

但当雪橇车载粮入京,这个如意盘算登时就要化为泡影。

“怎么办?!”高扬颓然的问着,坐在交椅上都是有气无力。

米行有着自己的会所。包括高扬、金平在内,九大行首会聚一堂。此前他们都已经得到了消息,现在仍是面面相觑。谁也没想到,王安石、韩冈、薛向,这三人加起来竟然当真在冬天将粮食运到京城中。

不过大行首金平此时已经恢复了平静,心中的隐忧只是放在粮食入京给百姓增加的信心上,“慌什么!还没有入京呢。就算当真入了京,能运来的粮食也不会多!我就不信,雪橇车还能跟纲船比?!真要有这等运力,早就在天下传开了!……一个冬天最多也不过二三十万石!”

得金平这个主心骨一说,行首们的脸色便顿时好了许多,如果只是几十万石的数目,他们还真不会放在心上。

其中一人便道:“就算翻一倍好了,也不过五十万石。朝廷要是想籍此发卖,到时候出来多少我们买多少。”

高平恶狠狠地狞笑道:“朝廷平抑粮价,必然是六七十文,想办法买下来,日后可是有赚的。”

一阵附和的笑声中,金平保持着平静:“尽量不要太冒风险,区区几十万石,对京城百万军民那是杯水车薪,转眼就能卖光。到时候,朝廷还是要开仓放粮!”

……………………

此时王安石正在中书中,与冯京争辩着是否要开常平仓放粮。

“六路发运司北运的粮纲已经到了南京,还有什么必要开常平仓?!”

粮商都能收到消息,政事堂中的王安石当然早就收到了。王安石一直都跟薛向有着联系,对于六路发运司的进度了若指掌。只是最近他在最近碓冰船失败后,刻意收敛了自己的强硬态度,使得开常平仓的意见在朝中甚嚣尘上。只是眼下宿州的粮食终于到了南京应天府,而泗州的存粮也顺利的向宿州转移。此事再无法遮掩,王安石的态度才重新变得决绝起来。

“薛向在奏章中都说,雪橇运粮乃是初行,不知其可否。即便侥幸功成,也绝不会多过纲船的运送,如何能压得下粮价。如今市面百物皆贵,没有一个售价不翻番的。再过半月就是年节,市面上却不见多少置办年货的。只要粮价跌,百货都会下跌,介甫相公,这常平仓是不能不开了,好歹让百姓过个安稳年吧!”

冯京作为参知政事,当然知道薛向在六路发运司做着什么,而且进度如何。但写给王安石的私信,和六路发运司呈递上来的公文,说的虽然是一件事,但只要词句和语气上稍作更易,给人的理解便截然相反,同时还不能说其中有错。使得冯京绝不看好王安石的坚持能带来什么成果。

“不能开!现在粮价上涨,根本不是缺粮的缘故,乃是奸商所为。常平仓的储备是为了防备灾荒,不是要给奸商补漏!”

王安石绝不可能答应,只要他在这里一点头,报请天子后,转眼消息就能传出去。诏令一下,粮价的确会跌。但跌多少却不可能说得清楚,那要看粮商们的态度。

朝廷不放粮,粮商们有充分的理由将粮价保持在高位上。若是常平仓放得少,同样打不下粮价。王安石都不用多想,也能猜得到,常平仓主持粮食平价发卖的官吏,有多少已经与粮商们勾结起来的。从常平仓发卖的粮食,恐怕会有三分之一给运到粮商们的库房中去。只有一口气将常平仓中的储粮卖出大半,那些粮商才有可能顺势将价格降下来,不过他们会拿出多少来卖,就不问可知了。

“难道就要看着京城百姓在年节时吃着一百三十文一斗的米不成?”

王安石的倔强,让冯京怒气难遏。不但恨起眼前这位拗相公,同时还把韩冈也一并恨上了,要不是他弄出什么雪橇车,王安石如今哪里还敢孤注一掷?!

冯京作为参知政事,绝不想看到粮价飞涨的局面,这事关朝堂是否稳定。另外他也要为日后着想。这时候不一舒己见,等到秋后算账,‘不作为’三个字就是自己的罪名。王安石下台早成定局,冯京可不想将自己也赔进去。

“南边的粮食很快就会到京城中,粮价不会再涨,只会下跌。”王安石的坚持依然毫不动摇,“而且明年更加重要,常平仓绝不能轻动!”

常平仓是除了举起屠刀之外,朝廷手中的最后一个武器。只要常平仓的存粮还在,粮商们就不能肆无忌惮的囤积居奇。如果明年灾情不减,没有了常平仓的制约,这些一干粮商就能肆意妄为。眼下的不过一百三十文粮价,能飞升到两百文去。

到时候,只剩一干强硬手段的朝廷,再无其他办法对付奸商。可天子还当真能下手对付自己的族人不成?恐怕也只有任凭朝臣将所有的脏水泼到自己身上!只要灾民的怨气有所依归,不动摇到朝廷的稳定,天子当不会介意牺牲一个宰相。

“既然介甫你坚持己见,冯京也没有什么好说的。开仓放粮的奏章明天我就会呈上去,到时候,还是劳烦相公你跟天子说吧!”冯京说罢,便一甩袖袍怒气冲冲的离开。

现在的政事堂中,只有王安石还在继续坚持,王珪虽然没有过来跟王安石顶牛,但他也是支持开常平仓。只是因为王安石一人的坚持,以及不断有好消息从六路发运司传来——多少还是靠了皇帝对韩冈发明的信心——使得天子尚无立刻动用常平仓的想法。

但王安石并不知道,赵顼的意志还能坚持多久。昨日就已经听说曹太皇和高太后找了天子过去询问如今的灾情和外面的物价,其中会说些什么,王安石都能猜想得到。说不定,今天晚上就能局面大变。

幸好运粮的车队已经到了南京,以车队在河道中的速度,两天后就能抵达京城。这个消息传到天子的耳中,应当能让他按奈下两天的性子来。

只是粮食还没有到京师,王安石还不能就此安下心来。他坚持不开常平仓,却也不会坐视京城百姓忍耐如今的物价过上一个年节。如今他就在盼着已经到了应天府的粮纲能顺利抵京。

只要有十几二十万石粮食进入京城中,如今浮动的民心肯定能由此安稳下来,而自己也能顺利的去应对明年的灾情。

