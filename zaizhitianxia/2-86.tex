\section{第20章 心念不改意难平(五)}

【第三更,求红票,收藏】

韩冈近距离的跟李宪打了照面,也没觉得他有什么特别。他身材比起王中正要健硕一点,相貌却朴实得很。除了没胡子外,李宪跟普通的官员几乎没有区别。

据说李宪在宫中有着数得着的箭术,很有些名气。而他能得同管勾御药院,在天子面前也定然是极亲近的内侍。但看他迫不及待要跟着王韶往古渭寨去,又跟毫无架子的跟韩冈拉着关系、大声谈笑,完全不见宣诏使臣应有的高傲。

王韶与韩冈对视了一眼,心中通透,这又是一个王中正。说实话,王韶和韩冈都不喜欢这些阉人,但只要能派上用场,却没有放过的道理。

王韶今次去古渭,已经不同往日。地位高了,名望涨了,一力反对他的几人也被他逼着离开了。眼下的王韶正得圣眷,红得发紫,出城送行的官员也便为数众多。

而郭逵亲自来送,也没有出乎王韶和韩冈的预料。郭逵在寒暄了一阵之后,对王韶道:“过些日子,等秦州诸事安定,本帅亦要往古渭走走,看看子纯的功劳。不知是否有打扰之嫌?”

王韶拱手笑道:“古渭本是秦州治下,太尉拨冗前来,如何能说打扰?古渭上下必洒扫内外,静待玉趾。”

就算没有这一问一答,依例郭逵也是要巡视秦凤各处紧要边寨,他是秦凤经略使,朝廷也不会允许他一直坐在秦州城中。两人这只是在互相表明自己的态度——郭逵表现了自己对王韶足够的尊重,而王韶则也做了相应的回复。

至少在此时,两人之间看不到任何裂痕,显得很是融洽。

王韶仅是去近处的古渭,洒泪赋诗的场面也就没有出现,秦州的官员还是很要脸面。喝过两杯水酒,王韶、李宪便带队走了。

送行的官员目送着一行远去,都回头看着郭逵,只有他先回去,其他人才能走。

可郭逵却不立刻上马动身,反而叫着韩冈:“玉昆。”

在几十道尖锐的目光中,韩冈不徐不急的走上前,拱手行礼:“下官在。”

“陪本帅说说话。”郭逵丢下一句,转身就走,韩冈拖后半步也跟了上去。

走在城门前宽阔的官道正中央,道路两边的空地上尽是避让他的行人和车马。一个人占据了四丈宽的要道,郭逵却全无堵塞交通的自觉。

他沉默着向前走着,韩冈则亦步亦趋的追在后面。郭逵不说话,他也不开口。跟在四五丈后,是一群身着青绿的官员,也是不出一声的跟着走,宛如一场沉默的行军。

张守约今天也出来送王韶,他看着郭逵在前面踱着步子,也不知他什么时候能走到城门下,便没兴趣跟着做傻瓜——他的身份也不惧郭逵能把他怎么样——便在路边找了间小酒店坐下来。李信就跟在他旁边,张守约让店家送了点酒菜,李信便帮着斟酒,侍候他吃喝起来。

张守约蘸着醋,吃了两块白切羊肉。用筷尖指了指已经走了老远的队伍,问着李信:“你那表弟是怎么回事,怎么跟郭仲通搭上了?”

李信茫然无知,摇着头:“小人不知。”

张守约不满的瞟了李信一眼。他这个亲信从来都是都是话不多,凡事绝不多说多问,守口如瓶,张守约也是看上了他这个性子,才把他从王韶处要来。就是因为李信可靠稳重,要不然张守约也不会才几个月功夫,就这么信任他,把他留在身边做亲卫。

但现在连表兄弟的事都推说不知,不管是不曾问过,还是明知却不说,都让张守约有些不高兴,也有点怀疑李信是不是因为到现在还没有官身,而在闹脾气。

他便又指着远处的人群,很直率的试探道:“以李信你的武艺才干,还有跟韩玉昆的关系,王舜臣的位置本应该是你的。”

“命数而已,各自凭缘。”李信信佛,对自己的失意并没有半点怨言。

张守约在李信脸上没有看到半点虚伪,看起来倒是真的不在意。这让他感到有些愧疚来,道:“再等一阵,到了八九月,西贼肯定坐不住的。到时放你出去挣个功劳,省得外人说跟着我还不如跟着王韶。”

“谢钤辖提拔。”李信跪下谢过,却依然不多说一字。

“你呀,就是这点太过了。”张守约摇了摇头,又自顾自的吃喝起来。

韩冈则是跟着郭逵走了一阵,送别的地方不过是东门外一里多地,走了几步,城门就在眼前。

郭逵这时停住脚,抬头眼定定的城门上的门额。过了一阵,他突然开口相问:“玉昆,你在秦州多久了?”

“下官自出生就在秦州,就跟下官的年纪一样,已有二十年了。”

“二十岁就已经靠天子特旨得了差遣,又立下了这么多功劳,”郭逵淡淡笑了笑,侧头看了韩冈一眼,“玉昆你日后前途不可限量啊!”

韩冈躬身逊谢:“太尉过誉了,下官愧不敢当。”

郭逵仿佛没听见韩冈的谦辞,像是在对韩冈说话,又像是自言自语:“二十岁就成了军事判官,而且是半年时间就从判司簿尉升到了初等职官,这速度的确是很快了。想本帅二十岁时,才不过个三班奉职,而且还是靠着父兄的余荫,不比玉昆你双手挣来的光彩。”

“太尉四十五岁身登枢辅,就是如今的王大参,也难跟太尉比进速。”

“但还是有人更快。”郭逵又开始向前走,“玉昆你应该知道,主持建造这座城门的,可是三十多岁就入政府了。”

韩冈道:“韩相公【韩琦】的际遇是个异数,并非常例。”

郭逵听了之后,突然嘿嘿的冷笑了起来,而笑了几声后,忽而又停了:“当年韩稚圭守关西。任福奉其命出战,范相公劝谨慎从事,要未虑胜,先虑败。但韩稚圭却道,‘兵须胜负置之度外’”说到这里,他又冷哼了起来。

接下来的事,关西人人耳熟能详,不必郭逵来说。

韩琦命令任福出战,虽然事前他说要将胜负置之度外。但任福惨败于好水川后,韩琦在撤军的半路中,阵亡将士的家属数以千计,手持故衣纸钱招魂而哭:‘汝昔从招讨出征,今招讨归而汝死矣,汝之魂识亦能从招讨以归乎?’当时恸哭之声惊天动地,逼得韩琦掩泣驻马不能前行。范仲淹听说此事后,便叹道,当此际,如何置之度外?

当时范仲淹和韩琦同守关西,一主守策,一主战策。虽然韩琦的进攻策略看起来很解气,可关西的军队却是已经因为多年来少有战事,堕落了许多,难以与李元昊相抗衡。范仲淹的策略却是符合实际。

“文正公当时筑堡戍守的策略是极好的,当年的西军多年未逢大战,无论兵将,都难以对抗元昊帐下的党项精骑。不似今日,即便是面对面的迎战也不会露怯。前些时候,燕都监奉太尉之命,于绥德连破西贼八寨堡,逼其狼狈而逃,正是西军战力在蒸蒸日上的明证。”

韩冈明着在拍郭逵马屁,实际上也是在说,西军憋气太久了,也该到了敲响战鼓的时候了。

“范相公在关西遗泽甚广,本帅当年也多承其教。”郭逵说着,“说起来,本帅当年还见过玉昆你的老师。那时候的张子厚年轻气盛,好武厌文,投书于范公,说是要领乡中健儿收复河湟之地,以攻西贼软肋。而范公则是看过子厚的策,对文字赞赏不已,说他是读书种子,不当沉湎于兵事,勉励他回去努力攻读。那日本帅正在范公帐下,还是本帅送张子厚出了衙门。”

郭逵将旧事娓娓道来,韩冈听得入神,说道:“想不到太尉与家师竟有如此渊源。”

“不仅如此,”郭逵回头看了看远远的吊在后面的一众官员,郭忠孝正走在人群中,“我那不成器的儿子弃武习文,弓马不见长进,就是读起书来还算过得去。是程伯醇和程正叔的弟子,跟着他们两年有余。张子厚是二程的表叔,从这边算来,你跟我那儿子也算是很亲近了。”

“衙内岂是韩冈能比?”韩冈心中暗自摇头。以郭逵的身份,他这样直白的拉近关系,这种拉拢方法,实在有失官场上的含蓄,而显得过于粗暴直接了。

郭逵不理韩冈的自谦,继续道:“虽然当年范公劝阻了张子厚,让他好生去读书。从此关西少了个英雄豪杰,却多了个淳淳君子。但子厚直到去年还在渭州做着军判,帮着蔡子政【蔡挺】整顿行伍,重划编制,号为将兵法,可见他对兵学上,是一日也不曾放下。现在又教出了如玉昆你一群出色的弟子来。”

“家师学究天人,不让先贤,非韩冈能望其项背。”

郭逵笑了一笑:“玉昆总是这般谦虚。”他举步走进城门,守门的官兵如爻倒的麦子,一个接着一个跪下。转眼就跪了一片。进门后,却不往城中去,而是叫着韩冈从门后的阶梯上,走上了城头。

