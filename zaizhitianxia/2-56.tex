\section{第13章 不由愚公山亦去(一)}

烈火熊熊。

刘希奭跟着傅勍急急赶到火灾现场,迎面就是一阵灼热的气浪。就在他们眼前,净慧庵两丈多高的主殿在火海中轰然崩塌,卷起了一片连着火星一起飞出的烟尘,淹没了小小尼庵所在的崇福坊。

烟与火冲散了救火的人群,沿着狭窄的巷道滚滚涌出。二十多匹马一起嘶叫起来,被吓得狂奔乱跳。傅勍和他的手下的甲骑不费什么气力将坐骑安抚下来,但刘希奭对马性不熟,控制不了胯下的马匹,不得不俯下身子,紧紧扯住缰绳,可在颠簸的马背上他依然摇摇欲坠。

刘希奭吓白了脸,手上的气力越来越小,缰绳渐渐的就在手中打滑,眼看着就要落马的时候。只见傅勍在旁一手伸过来,将笼头一扯,硬生生的将这匹马给扯定了。刘希奭的坐骑摇头晃脑,四只蹄子蹬着地,可不知傅勍用了什么手法,硬是将其按住动弹不得。

傅勍得意的哈哈笑着,对惊魂甫定的刘希奭喷着酒气:“走马,你骑的这畜生只是看上去膘肥体壮而已,胆子这么小,又没有好好训过,上了阵就会拉稀,明天还是换一匹胆子大的。若是走马不嫌弃,俺帮你挑!”

这边马匹受了惊,而净慧庵旁的救火人群却还要惊慌失措许多。方才净慧庵主殿被烧得坍塌下来,围着火场的不少人猝不及防,被滚烫的热灰伤了眼睛,大声的哭叫着,任由火势越烧越大。

傅勍纵马上前,一声大喝:“乱个什么!?全都站好了听本官发落!”他的口齿依然因为醉酒而吐词不清,但音量足够大,顿时便镇住了全场。

傅勍环目一扫刹那间就安静下来的人群,更加得意非凡,抬手一指众人,便点派起人手来。

虽然仍在醉中,但傅勍指挥起来却是条理分明,丝毫不乱。他把带来的二十多名骑兵分作数队,在火场外维持秩序,防着地痞无赖趁火打劫。潜火铺的铺兵救火经验丰富,被他派去防止火势蔓延,而剩下的百姓,傅勍则是让他们形成几条人龙,传递着灭火用的井水。

一番得力的举措,让火场周围本来混乱不堪的救火场面顿时井井有条起来。刘希奭在旁看着,啧啧称奇,暗叹傅勍这只醉猫能混个官身确非幸致,如果他不是老酗酒,说不定已经跟刘昌祚一样出头了。

傅勍指挥着扑救,刘希奭下马走到人群边,趁着他们传递水桶的间隙,问道:“火起后,在庵中修行的比丘尼可有伤亡,有没有没出来的?”

一个老头子回话道:“回官人,火头起的地方是净慧庵厨房边的柴草篷子,离着庵堂远,庵里的八个师太该是都跑出来了。”

“何止八人?”另一个年轻人在旁边怪笑着,“俺先到的场,亲眼看到从庵里跑出来十几个!”

即便火势仍然汹汹,但周围众人还是忍不住哄堂大笑。净慧庵的女尼,除了一个做庵主的老尼姑,个个都是带发修行,做着惠民桥后的营生,各自的身价还都不低。

笑声中,夜风乍起,连带着一阵热浪和风卷来,火星四溅,烟灰扑面。而随着风起,几条火舌也乘势冲出了净慧庵,舔上隔邻的房屋,虽然立刻就被傅勍指挥人手给扑灭,但已经再没人能笑得出来。

刘希奭呸呸呸的把灌进嘴里的烟灰吐掉,当即尖起嗓子喊道,“拆屋子!快把离火近的房子拆出一条道来!”

刘希奭想造出一条防火带来,以防火势蔓延,这是个正确的做法。可在场众人都是你看我我看你,没人先肯动手。现在在火场中救火的,巡城甲骑和潜火铺铺兵加起来才三四十个,而附近百姓赶来参与救火的却多达数百。虽然明知火势蔓延下来,会把周围的房子都给烧个精光,但不先看着房子被火点起,谁肯出手拆屋,得罪这几户邻居——都是抬头不见低头见的街坊,现在动了手,日后可就不好相见了。

刘希奭见没人搭理他的话,脸色顿时就难看下来。

人群中这时有人喊了一声,“先给周围房子浇水!水浇湿了就烧不起来了。”

这个主意立刻得到了所有人都赞同,刘希奭向人群中张望了两眼,却没看到究竟是谁的提议。

“水不够用!”另一边又有人接着喊道:“现在就三口井出水!”

“除了现在用的这三口井,还有哪里有水?!”刘希奭急问着,从三口井提起的一桶桶水,光是压制眼前的火势以是勉强,再想给周围房屋都泼上水,那是名副其实的杯水车薪。“里正呢,里正在哪里!?”他大喊着,“崇福坊还有哪处有水井?”

崇福坊的里正连忙排众而出,他在傅勍刘希奭他们赶来之前,就领头救火,脸上被烟熏的黑一道白一道,胡须也被烧了半拉。他在刘希奭面前躬身回话:“回官人的话,整个崇福坊就六口井。三口是路边公井,现在都用上了。剩下的三口都是私井,一口就在净慧庵中,一口是坊东角刘老赫家的,最后一口则是在刚刚死了的王启年家。”

“就六口?!”刘希奭惊问道。

“回官人的话,的确就六口。秦州大户人家的不是住在城东,就是住在州衙附近,城北这一片都是小门小户的人家。整个崇福坊有两百一十四户,可连一间前后三进的大宅子都没有。”

傅勍刚把前面的事重新分派好,转过来就听见刘希奭跟里正在扯着。他很不耐烦的说道:“别说这么多废话了,有几口井就用几口井。让那三家快把门打开!让人进去提水!”

净慧庵烧得跟炉膛似的,怎么进去提水。刘希奭看得出傅勍脑袋还有些醉意。只不过净慧庵的水井现在是用不上了,但刘家、王家的两口井却是能派上用场的。

傅勍一声令下,从人群中当即点出了三十多号人,跟在几名巡城甲骑之后,分头赶去有水井的刘家和王家。

……………………

王启年的未亡人已经被鞭打的奄奄一息,她的一对儿女也被吊在水桶中,降到了井底。听着井中传来的凄厉哭喊,相信只要再逼问一下,王家寡妇就会松口吐实。

不过窦解他们已经没时间等下去了。

听着外面砰砰砰的拍门声,喊着‘王家大嫂,借水井一用。’钱五欲哭无泪,他刚刚把王启年的儿女丢进水井中,但现在他却都有跳井的心了。

被人堵在王启年家,这等于是不打自招,就算窦解能靠着他祖父脱罪,但他们这些从人肯定没有好下场。

要逃!要立刻逃!

可王家就是一个小院子,四间房,连个后门都没有,就是有口水井!

钱五的视线转到了院墙上,李铁臂这时已经当机立断,指着院墙连声道:“翻墙!翻墙!”

窦解犹豫了一下。王家与邻居的围墙也就六七尺高的样子,只要身手还算灵活,跳起来手一撑就过去了。窦解带来的五六个伴当,哪一个都能轻轻松松翻过去,但他本人肯定例外,翻墙入户偷鸡摸狗的营生他半点经验都没有,

李铁臂急得跺脚,一把拉起窦解:“七衙内,耽搁不得,俺们会托你翻过去!”

窦解被扯着走到墙边,突然想起了一件事,回头指了指王启年的遗孀,“她们呢?”

李铁臂会意点头,命令道:“把她们都杀了!”

“杀不得!”钱五连忙拦住,“王家真要被灭了门,七衙内肯定脱不了干系。”

但李铁臂却坚持道:“还是杀了干净,外人怀疑就怀疑。只要没证据,谁能硬指着说是我们干的?”

‘找死啊你!’钱五又急又怒,已是惊得面无人色,‘事后想被灭口吗?!’

“只不过是绑着一阵,又没伤了她家的性命。吓唬她一下,谅她也不敢乱说。就是说出去,这点小事不用惊动副都总管,就会有人帮七衙内压下去。”钱五已经急得满口胡言,现在这种情况,秦州已经待不得了。若是杀了人,海捕文书肯定要落到头上,如果不杀,至少不用担心被缉捕。

李铁臂还待要辩。这时砰砰的拍门声更加急促,重得像是在撞门,外面的喊声也大了,不论钱五还是李铁臂都没心情争论了,一齐回头怒声道:“还不快把七衙内推上去!”

几个伴当也慌了,一齐动手,七手八脚把窦解吃力的推上去,却忘了先翻一个人过去,查探一下。

窦解搭着墙顶,被人推着扶着,终于在围墙上撑起身子。他正要翻身过墙,这时院墙对面,却突然冒出一个脑袋来。与窦解面对着面,脸贴着脸,鼻尖几乎撞在一起,两对眼睛就隔了几寸的距离相互对视着。

“啊!~~”窦七衙内被惊得尖叫起来,双手不由一松,身子往后一仰。整个人就失去了平衡,砰的一声,重重地落在地上。

李铁臂和钱五忙奔过去扶起窦解。

而那个探头出来的人,向院中一张望,当即就把头缩了回去。很快就一连声的喊了起来,“王家有贼!王家有贼!”

