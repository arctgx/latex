\section{第30章 肘腋萧墙暮色凉(16) }

夜深了,罗兀城疗养院的病房内还是有着灯光。

两间大型营房改造的病房,总计上百张床位上,躺满了伤兵。而且都是重伤员——轻伤包扎一下就归队,只有重伤才会留医。浓烈的药味弥漫在房内的空气中,还有断断续续的呻吟,让人不忍卒听。

现在罗兀城的护工基本上都是才挑出来的,原本的一批人,却在上次回绥德时,一起跟着伤病走了。缺乏得力的人手,韩冈也不能再在旁边看着,也不得不出手帮忙。

韩冈蹲在一张病床边,帮着一名腹部中箭的士兵更换伤药。

伤兵很年轻,上唇处才刚长出融融的髭须,当比韩冈还小上两岁。在换药时,他一直忍着痛,额头上冒出豆大的汗珠,都没有吭上一声。只是当韩冈为他缠好绷带,正要离开,他才抬手扯着韩冈的袖口,惶然的问着:“韩官人,俺会不会死?”

韩冈一直略显锋锐的眉眼柔和了起来,“不用担心,好生休养一阵就会好了。”

在罗兀城士兵的心目中中,韩冈的威望甚高。极少有哪个官人能像韩冈一样,为士卒尽心尽力到这个地步。也就在这几天中,韩冈得到了整个罗兀城的尊敬。他的一句话,就让年轻的士兵平静了下来,松开了抓着韩冈衣角的手。安慰了惶惶不然的士兵,韩冈从病床边站起身。到了这个时候还没有入睡的一些伤兵,皆感激涕零的目送着韩冈从身边走过去。

出了病房,韩冈仰头看了看天上的一轮明月,虽然清辉依然堪比昨日的满月,但已经可以看见有了一点缺口,正往下弦月变化去了。

算起来今天已经是党项围城的第十天了,经过了十天算不上激烈的攻防战,城中守军伤亡虽然不大,也有五百多人了。但韩冈现在对罗兀城的守御能力极具信心,罗兀城依然完好,甚至永乐川寨也至今犹在。

西夏国相梁乙埋的旗号,城头上的守军看到了不止一次,但城高濠深的罗兀城始终没有被打下来。而有罗兀城在背后牵制,小小的永乐川寨,党项人也一样没有打下来。

一旦西夏人准备进攻永乐川寨,高永能都是毫不犹豫的派军出战,让党项人无法顺利的进兵。永乐川寨离罗兀只有两里,这样狭窄的战场,缺乏两面作战的活动空间。而如此积极的防御姿态,也是至今保住永乐川城的关键。而且就在两天前,在出战的步军阵列的掩护下,一个指挥的骑兵还冲进了永乐川寨,加强了永乐川寨的守卫。

高永能毫不犹豫的坚守着罗兀城。被派进来劝降的使者,如果是党项人的,那就割了耳朵和鼻子和双手赶出去,如果是汉人的,则直接在城头上剁翻。梁乙埋派了两次使者后,就再也不派人进来送死和找虐了,转而变成了加强攻势。

经过这些天的战斗,城中上下都明白了,以罗兀城的城防水准,还有党项人拙劣的攻城器具,梁乙埋想要在短时间内攻破罗兀城,那等于就是在做梦——为了保证不会有内应开城,种谔事先连一个蕃人都没有留在城中。

罗兀城如今城高近三丈,外面的壕沟,以及城下的羊马墙,其防御力虽比不上绥德、古渭这些已经有些年头的军城,日后也需要加以增筑,但眼下的防御,抵抗西夏人的进攻还是没有问题的。加上城中还有一万多人的守军,城下的战场又过于狭窄,甚至连供大队骑兵纵马驰突的空间都没有了,使得梁乙埋纵然拥有七八万大军,也无法将手上的兵力全数派上去攻城。

而且城内口粮也不虞匮乏,守城的物资也十分充足,虽然水井只有十几眼,以城中的人马来算,的确是少了一点,但在西贼无法彻底围城的情况下,连接无定河的水道还是通畅的,也不至于会渴着。

罗兀城本身很安全,上上下下都有坚守到底的自信,但是……抚宁堡却已经陷落了。

前几天来自告急的狼烟,就算隔了整整三十里,已经浅淡得几乎成了天幕中的一缕阴翳,仍深深的烙在韩冈等人的眼底。抚宁堡的失陷,使得突出在前的罗兀城成为了孤军。不过城中的局势从一开始的混乱,到后来则逐渐的安稳下来。

得到抚宁堡失陷的消息,高永能并没有任何慌乱。从他身上,韩冈能看到胸有成竹的自信。因为接下来,就是绥德和细浮图城一起出兵,击败了攻夺抚宁堡的西贼,虽然在路途中始终要受到干扰,但来自绥德的信使始终没有断过。

每天都有来自绥德的信使进城。为了保住连接罗兀和绥德的交通线,韩绛孤注一掷的调兵遣将。不仅仅是鄜延路的兵将,最近处的环庆路调兵更多。从最新传来的军情上看,韩绛已经把有名的老将、人称张铁简的张玉也调来了,还有与西军中,与种诂、种谔、种谊三种并称,二姚中的姚兕也奉命领军来维系罗兀后路。

不过韩冈还是抱着悲观的态度,一座大型军城的日常消耗难以计数,眼下也许还能支撑,但时间长了,是不可能在后勤要道受到干扰的情况下坚守下去的,如果不能将抚宁堡重修并稳守下来,罗兀城必然要放弃。

但西夏一方,韩冈也估计他们的粮食不会太多了,现在的情况,就得看哪边先支持不住!

………………

顿兵在罗兀城下已经超过了十天,梁乙埋始终处在进退维谷之间,

现在驻扎了一万多精锐的罗兀城在前面顶着,梁乙埋也不可能孤注一掷的将全军都绕过罗兀城去,只能分出一部分兵力,绕道南方,主力还是放在罗兀城。

而且罗兀城所在的地方,也摆不下跟随梁乙埋而来的全班人马。三四万兵就已经撑满了谷地。同时党项人多马,需要的营地远大于宋人。在野地里驻扎的营地,又没有罗兀城这样的墙体让人能安心睡觉。布置出来的各部营帐,就不能挤得太紧。而是要分割出一段距离。不然一旦有个风吹草动,就会是波及全营的骚动,甚至动乱。而最差的情况,便会炸营。

而此前种谔扫荡罗兀周围附夏蕃部的行动,也成了勒住梁乙埋脖子的一根绳索。

随着时间一天天的过去,梁乙埋越来越多的精力都要放在后勤上。八万大军消耗的粮食是个天文数字,过去党项人南侵,要么是从横山蕃部处得到补给,要么就是靠打下宋人的寨堡从而得到存粮。因粮于人四个字就是西夏军的后勤法则。

但现在,两条路都走不通,亲附大夏的部族被清剿,而其他蕃部都采取了观望的态度,惹不起宋夏两家,但往山沟里一钻,谁也拿他们没办法。

梁乙埋现在也只能企盼他事先埋下的手段,能及早起到他所希望的作用。

……………………

其实头疼的不只是梁乙埋,大宋天子最近也是寝食难安。

虽然种谔不费吹灰之力就攻取了罗兀城,让赵顼度过了一个快乐的上元节。但紧接着河东军的失败,却是相当于当头浇下的一盆夹冰冷水,把他从讨平灵夏的美梦中惊醒过来。

对照着现在摆放在武英殿偏殿正中央,横山和无定河的地形沙盘,河东方向的失败对整个战局的影响,赵顼有着极为直观和明确的了解,并不为韩绛轻描淡写的言辞蒙混过去。

而且雪上加霜的是,接下来就有人开始质疑罗兀城后路的安全性。

郭逵当先上书,说抚宁堡必须着重防守,否则罗兀必失。因为郭逵远在秦州,他的话赵顼半信半疑。但前些天,韩绛的副手——宣抚判官赵禼也上本密奏,说抚宁堡由于筑城不利,形制小于预定,使得无法驻守足够的兵力,很难抵挡西贼的进攻。

郭逵管着秦凤,离抚宁堡有千里之遥,而且又跟韩绛不合,他说的话赵顼可以不当一回事。可赵禼就是宣抚司中人,是直接的当事人,赵顼一见他的奏章之后,便大惊失色,忙遣人去罗兀、抚宁视察真相。可是人刚走没几天,这抚宁堡陷落的消息就传来了。

这个消息犹如晴天霹雳一般,让赵顼失去了言语的力气,加之就在当天,刚刚出生才两日的皇子又夭折了。两桩噩耗顿时将体质并不算好,加之又劳累过度的年轻皇帝一下击倒。

等到赵顼终于能起床理事,已经是三天后了。值得他庆幸的是,经过了这几天,罗兀城的情况又渐渐开始好转。西夏军虽然占了抚宁堡,但却在自绥德出兵的种谔,以及驻屯在细浮图城的折继世的打击下,吃了一个败仗。现在罗兀城和绥德之间的要道,正在被环庆、鄜延两路的兵马稳守,以防备西贼的骚扰。

听到陕西战况平稳的消息,赵顼心情好了不少,今日中午时补身子的药粥还多喝一碗。但到了午后,王安石匆匆求见,并呈递上来了一份辽人国书。

赵顼只展开一看,脸色顿时发白,一阵头晕目眩,怎么契丹人也掺和进来了!

