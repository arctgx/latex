\section{第18章 弃财从义何需名(中)}

【新的一更,求红票,收藏。】

“就是那个韩冈手段太狠,秦州有名的陈押司就是惹了他,才全家死得连个承香火的都不剩。就怕他今次来凤翔,不光是为了把保李家小子保出来。”冯从礼想起这两天打听到的传言,心中有些发毛。而他的两个兄弟听到这话,脸色也变得发白起来。

前几个月他们虽然连续收到秦州的几封来信,说是那女人的姨侄受荐为官,但当时冯家三子都没放在心上。又不是本州的官,而且也不是有出身的进士,以冯家的豪富,根本不须放在眼里。

当前段时间他们为老子办丧事的时候,那女人的哥哥打上门来,不知底细的三人毫不犹豫的就命人动了手,把他强丢了出去;前两天,那女人的侄儿又打上门来,吃了大亏后,三人又厚礼请动了州里的刘节推下狠手。但事后为了保险起见,他们又稍稍打听了一下两人满口说着的韩冈的事迹。这一打听,三人顿时心都凉了。

横渠先生的嫡传弟子,把赫赫有名的陈押司家灭了满门,还没当官时就跟一路都钤辖放对,等得了天子亲下特旨赠官,就帮着他的举主把那位都钤辖气得中风,并一股脑的连同经略相公和兵马副总管两位重臣都赶走了。而且他还说服了桀骜不驯的蕃部,帮着打赢了一场战果辉煌的胜仗,韩冈的一桩桩事迹,还有他的手段,成功的让冯家三兄弟一起都陷入了冰窟里去。

冯从礼唉声叹气半天,终于觉得在这样叹气下去实在于事无补,站起来对两个弟弟道,“在这里叹气也没办法,先去见一下刘节推,再请他帮个忙吧。”

“刘节推的价码太高了,上一次只是对付一个赤佬就要去了八十贯的财帛。现在要跟韩冈对上,没个上千贯下不来。”冯从孝抱怨着。

冯从仁也心疼着钱,提议道:“不如去跟韩冈说些好话如何,冤家宜解不宜结……”

冯从孝立刻摇头道:“那女人夜里突然病死了,老四要不是怀疑她被下了毒,如何会离家……”

冯从仁叫了起来:“明明是她守着爹的时候突然就倒下去了,怎么给她下毒?”

“你以为韩冈会信哪一边?!”

冯从礼开口道:“就算韩冈不怀疑此事,单是我们将她划出族谱,就已经把李家得罪狠了。这事怎么也不可能挽回。”

三人互相看了看,最后一起叹道:“还是去找刘节推。”

一个时辰后,凤翔军节度推官刘德在自己的官厅中,训着只用半边屁股沾着交椅,斜签着坐下的冯从礼:“你们担心什么?!那李信本官打也打了,关也关了,还想要本官判他个流放不成?他是自首,不论何罪,就当先减二等论处。你那些随从又没个轻重伤,不过是皮肉吃痛而已。怎么判他重罪?要怪就怪你们挨打时不受点重伤!”

刘崃对冯从礼擦了伤药的脸视而不见,说得又是跟他现在的请托毫不相关的事,但冯从礼并不敢反驳。

“小人哪里敢怨节推,只是害怕李忠得了他家外甥的助力,再来小人家里纠缠。还请节推能看在小人一向恭谨的份上,稍稍看顾一二。”他恭恭敬敬的递上了张礼单,担惊受怕的模样,唯恐刘崃不肯收下。

刘崃看都没看就把礼单收进了袖中,现在冯家有求于他,谅他们也不敢少给。收了好处,他的脸上就多了一点笑模样,提点了冯从礼一句:“你们可以放心,韩冈是秦州的官,跟凤翔府毫无瓜葛,他若是在府中肆意妄为,李大府不会饶了他。”

说罢,他也不多说什么废话,直接点了汤,冯从礼见了,连忙识趣的告辞出来。走出衙门,面对迎上来的两个弟弟,冯从礼狠狠狞笑了两声,为自己壮着胆,“不用担心,刘推官说了,有李大府镇着,韩家小儿不敢闹大。”

………………

当韩冈跟着李信,在慕容武的陪同下,走进李家小院的时候,他已经换上了一身青色的官服。

他和慕容武骑着马过来,马蹄声敲打着小巷中的石板路,让不少邻居冲着李家张望。而两人身上的官袍,则让这些看客变得老实起来,不敢跟着上门来打探八卦消息。

一进里屋,韩冈就看到一个五十多岁的老者正躺在床上,他长得跟李信很像,就是被单下的身躯显得有些瘦削,在他脸上看不到伤痕,只是蜡黄蜡黄的,透着浓重的病容。而在他床边,站着个十七八岁的少年,让韩冈为之一惊,正是他当日在三阳寨看到的那一个冯从义。

李信见到老子,先抢上去在床边跪下,难得的开口多说了几个字:“爹,你看谁来了!”

李忠看着被关入大狱的儿子,现在站在自己的面前,已是惊喜万分。听了儿子的话,将视线后移,两件青色的官袍顿时映入他的眼中。李忠心中一惊,便要起身拜见。只是他看着站在前面的那个年轻得有些过分的官人,动作却停了。虽然他不认识,却莫名的感到亲切。

“可是三哥儿?”李忠抬起昏黄的老眼,颤声问着。

韩冈应声跟着跪下行礼:“韩冈拜见舅舅。”

李忠见着韩冈在床边下跪,连忙坐了起来。先让儿子将韩冈扶起,又看着韩冈身上厚重的青色。不禁热泪盈眶,花白的胡子直抖着:“三姐生了个好儿子啊!”

“表兄在张老钤辖帐下也不差,很快就能得官了。”韩冈为李信说了句好话,侧过身子,将慕容武让出来,“这是县中的慕容主簿,也是甥男同在横渠门下的师兄,最是亲近不过。今次表兄能得脱牢狱,还是多亏了慕容主簿相助,将甥男引见给府里的陈通判。”

李忠当即在李信的搀扶下,起身向慕容武道谢,“小老儿多谢主簿看顾。”

“李老丈哪里得话,我与玉昆是极亲近的同门兄弟,玉昆既然有事相求,我怎么也不能袖手旁观。”

看到儿子、外甥都在眼前,李忠精神顿时好了不少,他也是在冯家被欺负狠了,回来后才病倒的。现在情势扭转,靠着外甥又搭上了县里的主簿、府里的通判,他父子两人在冯家受得气,也能报上一报了。

韩冈这时将视线转到冯从义身上:“这位可是从义表弟。”

冯从义这时也认出了在三阳寨中帮了他一把的官人,见韩冈问过来,也忙跪下问好:“从义拜见三表哥。”

韩冈将他扶起,感慨道:“当日在三阳寨,阴差阳错没能相认,今天终于见到了。”

慕容武说了几句就告辞了。人家亲戚相见,肯定有些话要私下里说,自己还站在屋中,那就是没眼色了。韩冈将他送出门外,却是约好今夜找间酒楼摆酒,并要把陈通判一起请来,洗洗李信身上的晦气,也要顺便谢两人相助之德。

韩冈回到屋中,不再多说废话,向冯从义问起事情的来龙去脉。尤其是四姨的身份不确认清楚,他也不好决定手段。

韩冈相问,冯从义和李忠便把事情一桩桩的说给他听。

韩冈的四姨少时是个远近闻名的美人,这跟容貌普通的韩阿李的完全相反,故而引了不少人家来求亲,其中便包括丧妻不久的冯德坤。而当年韩冈的外公手头拮据,看上了冯家的聘礼,所以将她嫁给了年纪大了二十多岁的冯德坤——的确是出嫁,而不是送女作妾。

但可能是因为对婚事不满,韩冈的四姨跟家中便有了点隔阂,也只是在十年前韩冈的外公过世的时候,才跟家里人见了一面——这一点是韩冈猜得。

“娘是明媒正娶嫁进了冯家,又生了小弟。但三个哥哥因为家财少分了一份,一直都跟娘过不去,几个嫂子也是。娘去年突然病死,也说不清究竟是怎么回事,是不是有人做了手脚。

没了娘护持,爹又是躺在床上,不能自理,小弟知道在家里站不住脚,便出来跟人做个买卖。谁想到小弟一走,他们就买通了族里的人,骗过了爹爹,将娘的名字从族谱里划去了,灵位也不给放进祠堂,还暗里传言,说小弟不是冯家的人。

甚至办娘丧事的时候,他们也不通知舅舅,二姨、三姨,却骗小弟说已经都通知到了,但都不肯过来。”冯从义说着,恨得咬牙切齿。

他跟李忠相认,还是前些日子,听到其父病死,赶回来奔丧时,看到了李忠跟三个兄长起了冲突,才知道他被骗了。

“四姐在家中年纪最小,没想到却第一个走,连个终都没能给她送上。”李忠叹着气,眼角处有着泪光。

陪着舅舅叹息了一阵,韩冈问着冯从义:“冯家的家产,你是不是要争上一争。”

冯从义小心的看了几眼韩冈的脸色,最后摇头道:“小弟不想跟几个哥哥相争。只想为娘亲昭雪冤情,恢复娘亲在冯家的身份。”

“孝悌二字你能记在心上是好事。若你只想着家产,而罔顾四姨的冤情,我倒是要失望了。”韩冈很满意冯从义的回答。

子不言父过,依儒家纲常,就算长辈有错,可以劝谏,但不能跟他们明着吵闹,尤其是闹上衙门,更是不该。要是做儿女的控告父母,依律可以直接斩了。跟兄长闹着家产,虽然如今也是常见的事,但遇上爱较真的官员,也少不得一顿好打。而现在冯家有钱收买官员,尤其是那个刘节推,真闹起来时,他可就是有借口了。

而韩冈本人是儒门弟子,当以敦厚风俗为己任,撺掇他人挑战纲常日后却是要被人骂的。大事上,把挡在道前的规矩一脚踢开,那是勇于任事,不拘泥于小节。而这些家常小事上,却是不能不注意一下自己的形象。

不过冯从义的几个哥哥他也不可能放过,“殴伤舅舅的事不能放过,还有表哥的事,都要跟他们算清楚。另外,四姨的死,则更是要他们给个交代!”

