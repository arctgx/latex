\section{第31章 战鼓将擂缘败至(四)}

党项军的环卫铁骑,用了很短的时间便击退了宋军骑兵的骚扰和阻截。领着身后的铁鹞子,正面直奔而来高永能的将旗而来。毕竟宋军也是匆匆堵到这个位置上,阵型尚有些散乱,并不像前几次出城邀战时那般整齐。而且为了追上潜离罗兀的宋军车马,他们也必须击垮在河谷最狭处列阵的宋军。

大地的震颤,让胯下爱马紧张得转动着耳朵,可直面着铺天盖地一般的党项骑兵,高永能却还是冷静如常。虽然他是罗兀主帅,但张玉的地位远高于他。若要出城作战,都是高永能领军外出,而由张玉坐镇城中。

红底黑边的战旗在山谷中的烈风下激烈的舞动着,旗尾时不时的拂过高永能的面颊,但没有让他专注于发号施令的集中力有过哪怕一点的波动。

手下拥有着上万精锐,这些日子高永能便日日带兵出城去邀战。可党项人那里却始终没有决战的想法,让他好生憋闷。不过今天终于能一决胜负,这让高永能在冷静中还带着一丝期待。

阵列而战,党项人如何会是对手。在高永能的心中有着满满的自信。

战鼓声在高永能的将旗下响起,尚有些混乱的阵型也在快速的调整之中。排在阵前的弩弓手已经当先将队列整备完成,各自张开随身携带的神臂弓。将重弩平平举起,把数寸长的木羽短矢放入箭槽,锋锐的三棱箭头便对准了奔驰而来的敌骑。

当领头的环卫铁骑最终冲到了百步之外,在各级军官们的号令下,一片弦声在前沿阵列中响过,从神臂弓中迸出的利矢,向着来敌劲射而去。

最前面的十几名骑兵,顿时人仰马翻,浑身上下被射得如同刺猬一般。而跟在后面的骑兵,也或多或少的受了几箭。

神臂弓射力之强劲,乃是如今天下重弩之中的佼佼者。在御前演射时,当着天子的面,能在七十步外洞穿铁甲。百步的距离,虽然比七十步远了许多,但骑兵和战马身上的披挂,都没有铁甲的坚固。

五六寸短矢深深的扎入皮肉之中,如果不是命中要害,人多还能够咬牙支撑。只是战马却做不到,它们在惨嘶声中乱蹦乱跳着,颠翻了背上的骑手,搅乱了冲锋的队列。

不过紧随在后的环卫铁骑们丝毫没有停步的意思,他们展现了作为天子近卫的完美马术,轻提马缰,轻易的绕过了混乱的前阵之后,继续加速前冲,想要赶在下一轮发射前,冲进宋军的阵列之中。

但迎接他们的,是又一丛犹如被惊起的飞蝗一般爆开的箭雨。

“高永能如此博命,看起来离开得那队车马中,必然有着重要人物。”

罗兀城外的一处高地上,梁乙埋远远望着战线处被宋军箭阵横扫的己方骑兵,神色并不为之所动。他现在并不是很在乎兵力的损失,只要能给宋人造成更大的伤亡,这一点的交换还是值得的。

——因为他手上的兵力比宋人更多。

河谷中的战场实在太小了一点,不擅于攻城的党项人,让梁乙埋手上的几万兵只能远远望着高耸的罗兀城头,分兵上去攻打,只是给宋人送点心。

而他前日派出去抄小道的偏师,尽管攻下抚宁堡的过程虽然顺利,但堡中的粮食也给烧的一干二净。就是因为粮草不济,他们在跟宋军交战之后,不得不退了回来——事先谁也不会想到,南下沿途的村寨都已经被宋人当先劫了一遍。想是就地征发,都找不到多少口粮。梁乙埋从下属的口中,听说了一个个被烧光的村子,种谔下手之狠绝,让他这位西夏国相都觉得惊讶。

到现在为止,梁乙埋手中的粮草已经不足以支撑全军五日,而种谔的心狠手辣,使得他不得不去搜刮位于横山北麓的蕃部——但由于地理位置的原因,山北蕃部的身家和存粮都要远远小于山南——要不是因为还对契丹人的干涉抱着一个希望,他手下的这些豪族族长们早就闹起来了。

而宋人今次派了大队车马悄悄离城,虽然尚不知是什么原因,可梁乙埋的直觉还是让他嗅到了对自己有利的味道。

“是不是派兵绕过去追击,硬冲箭阵实在是伤亡太大啊!”

一名跟随梁乙埋领军而来的党项豪族族长如此提议着。在前面冲击宋军阵列的几千骑兵中,有他的族军,看到自家的子弟兵像被割下的麦子一样一群群的从马背上翻下去,他心疼得几乎要叫起来。

“宋人是骑马走的!”梁乙埋很不快的冷喝了一声。

虽然马车的速度会比单纯骑马要慢上一点。但从小路翻出无定河谷地,再绕道向南去抄截前路。从时间上看,根本不可能。反而会引起屯兵在细浮图城的折继世的注意,一个不小心,就会被堵着回来退路。然后被绥德的种谔给咬上来

而且西夏国相现在已经并不再坚持着要追上那队车马了,虽然派出去追击的铁鹞子和环卫铁骑被宋人的阵列所阻挡,前进不得。但换个角度来看,城外的宋军何尝不是已经被他的几千精锐给分割在外,已经而无法顺利退入城中。

“都罗马尾!”梁乙埋忽然叫着丢掉了罗兀城的都枢密的名字。

都枢密这个官职在以党项豪族为主体的西夏国中,其实并没有宋国朝廷中枢密使那般的威势,但都罗马尾原本是梁氏兄妹的亲信,加之都罗家也是党项豪族,因而此前他在西夏国中的地位并不算低。

但在罗兀陷落之后,都罗马尾为了收回丢失的城池,连番大战,不但葬送了大批银州守军,和诸多附夏蕃部中的丁壮,连带在身边的本族兵力也损失许多。在梁乙埋领军到来之后,就被晾在了一边,一直没人搭理他。现在终于听到传唤,便立刻上前听命。

“你去攻击罗兀城,攻得猛一点,让高永能不能安心,望你能将功赎罪。”

梁乙埋对他也没有多余话说,随手点起几名将领,让他们跟着都罗马尾。西夏国相下命令的口气冷硬,微眯的双眼也危险的瞪着都罗马尾和几个被点起的将领,不容他们拒绝。

北方忽然的号角声吸引了城头上韩冈的注意力。把视线从南面的激战中转移过来,却见一群党项骑兵开始向罗兀城扑来,卷起了一片尘浪。而在骑兵之后,还有黑压压一群被掩盖在尘土中的队伍,数千上万,看他们的方向,也是向罗兀城而来。

“来了!”韩冈一声压抑着兴奋的低喝,让身边的种朴得意的笑起,连带着引起了一群将官的笑意。

这个局面,正是他们想看到的。

党项兵多,不过狭小的战场局限了他们投放兵力的数量。如今宋人刻意将战场两分,梁乙埋当然会乘势投入更多的兵力——但也正顺了韩冈他们的心意。

那队多达千匹的敌骑当先奔驰而来,快速的冲到城下,向城头驰射出一片箭雨之后,转而就又飞驰而去。在过程中,来自旗帜林立的城头上,去只有零零星星的箭矢反击。

守城宋军这种虚弱的反应,与城头上多如牛毛的旗帜截然相反,落在都罗马尾眼里,便是让他精神一震。

他这段时间以来,与宋军已经交战多次,知道一开始在种谔手上有两万多兵力,但后来种谔却是率军离开罗兀。究竟走了多少,都罗马尾不知道,只能靠猜测。但现在看来,种谔当是带走了大部分的兵力,眼下城中的守军不会超过两千,加上出战的六千人,当只有八九千的样子。

他带出来的一万余兵是以步跋子为主力,抬着云梯,推着过濠河的桥车。若是城中守军只有两千左右,都罗马尾却是有自信能击破这样空虚的城池,而不仅仅是扰乱高永能的军心。

随着西夏的步军接近,张玉开始发号施令,罗兀城的城头上,一件件的摆出了守城的用具。

檑木、滚石、油锅、狼牙拍,应有尽有,六张巨型的三弓床弩也一起被摆上了迎面的城头。并排着的三条六七尺长的巨型弓臂,前面两条弓臂正装,而最后的一条则是反装,反曲弓式样的弓臂相对放置,看起来就像个葫芦。

这是俗称八牛弩的重型兵器,也是罗兀城中威力最为强悍的一件武器。攻城时,能把长枪一般的专用箭矢,深深的射到城墙墙体中,作为士兵攀城而上的落脚点。而在守城时,又能一击射穿敌军阵列,像串糖葫芦般,连着串上七八人方才会力道用尽。

其弓力之强,号称需用八头牛才能将之上弦。虽然这是过于夸大,但也的确是需要二三十名身强力壮的大汉一起转动着绞盘,才能把用马尾、丝线和细麻混合绞成的拇指粗细的弩弦搭在牙发上。发射时,也不是像腰开弩、厥张弩还有神臂弓那等单人弩一样用手指扣动扳机,却是得用一柄木锤,把扣住弩弦的牙发用力敲下去。

