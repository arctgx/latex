\section{第23章 天南铜柱今复立(中)}

【昨天白天有事,欠了一更,今天补上。】

大宋的太皇太后曹氏,可以说是当今世上最为尊贵的存在。

即便是贵为天子,到了她的慈寿宫中,也是得跪下来行礼。而对于远在北方的另一位皇帝来说,依照旧年的盟约所定,她同样是必须尊称为叔母的长辈。

曹氏也不仅仅是靠着辈份,她也曾经在英宗皇帝重病的时候,作为天子的代理,统治过这座拥有亿万人口的帝国。垂帘听政的权力,古往今来,也只有少数女子曾经拥有过,而这些女子,往往沉醉于这份权柄,如同尝到了蜜糖的蚂蚁,得到之后再舍不得放弃。

只有曹氏,当英宗皇帝病愈归来之后,便将手中的权力毫不恋栈的放弃了,尽管中间有些小波折,但这份德行至今仍备受世人敬仰。

但不论她的身份有多么尊贵,她的能力有多么出众,她的名声得到多少赞美,在不断流逝的时光面前,她并不比站在她的寝宫门前的内侍们更有优势。

唯有时间带来的老迈和死亡是平等的。

尽管曹氏也不过是花甲之年——官场之上,到了七八十岁依然坚持着为皇宋奉献着忠心的臣子人数并不少,宫廷之中,真宗皇帝的贵妃沈氏也是新近以耄耋之年辞世——但她的健康状况,这些年一年比一年衰弱,生命正一点点的走向最终的结局,也许还有五六年、甚至八九年,但也有可能就在下一刻。

宫中的初春依然很冷,宫城外已带着春日暖意的和风,吹到了宽广幽深的殿宇之中的时候,却莫名的变得阴寒起来。

炭火时时燃烧着的暖阁中,倒是暖意盎然,嗅不到半点烟熏火燎的味道,若有若无缓缓弥散开来的浅色烟雾,那是沉香在香炉中燃烧。

从午后的浅睡中醒来,曹氏听见了暖阁外间的动静,有些困顿的睁开眼:“是谁来了?”

“太皇,是官家来了。”给太皇太后捶着腿的贴身女官回着话,手上动作并不见停。

“官家来了,怎么都不喊老身起来。”曹氏责怪着。

“娘娘难得睡得安稳,孙儿不敢打扰。”赵顼走进内间,笑着说道。

当今天子意气风发的样子,这两年来是难得一见。嘴角的笑意,恍惚十年前刚刚登基时的模样。

“官家今日殿上受贺。平灭一国的大胜,自太宗皇帝之后,可是再没有过。”

“只是为了交趾而已,若是为了西夏那就好了。”赵顼遗憾的口吻似是不满意,但曹氏哪能不知道孙子的想法,心中早已经是乐开了花。

曹氏还记得,赵顼初登基的时候,便身穿金甲来拜见自己,还询问穿戴得到底怎么样。那时候的皇帝,不过是个什么都不懂的黄口孺子。如今十年过去了,当时还显得甚为稚嫩的天子,也在三千多个日月交替中,变得深沉起来,往一个合格的皇帝靠拢。

曹氏从榻上起来,赵顼连忙上来搀扶着。祖孙两人从暖阁中走出来,曹氏问道:“官军什么时候班师凯旋?”

“大约要两个月。”赵顼扶着祖母,散步似的慢慢走着,“交趾境内的道路因为雨水坏了不少,只能借道海上返回邕州。正好要在海门镇开港置州,也是顺便走上一趟。”

“交趾要设州了?”曹氏问道。

“正是。只是差点就看不到。”赵顼感叹着,“今年交趾的雨来的也比往年早,雨水还大,要不是章惇韩冈当机立断,放弃等待援军,径直攻进了交趾境内。这时候也只能望雨兴叹。那样下来,可就又要多耗一年钱粮。”

曹氏望着殿外的草木,已经有着融融嫩绿,快要到踏青的时节。一年年的过得当真很快,仁宗朝的事还在眼前,但睁开眼后,新帝已经登基十年了:“当年为乱天南的侬智高,只是一个被交趾欺压的叛逆而已,却一举引得天下骚然。但这一次,平掉的却是交趾。论起战功,狄青也不能与章惇等人想比。”

“章惇、韩冈、燕达等人的确是有大功于国。”赵顼点头说道,“等他们回来后,孙儿也不会吝于封赏。”

当今的天子正在最得意的时候,由于新法的成功——不论民间有多少怨声,至少是富国强兵的初衷已经达到了。这就证明了当初皇帝一意孤行的正确,当一个人习惯于自己的正确,那么他就很难再听从别人的意见,

“章惇回来后,当能入西府了?”

赵顼点头道:“一个枢密副使而已,肯定是要给他的。”

“那韩冈呢?”曹氏问道:“是要进学士院了吧?”

赵顼默然,韩冈如果回朝,想挑个合适的职位将他安排下,很是有些难度。翰林学士的地位太髙了,但以韩冈的功绩,却是绰绰有余。

曹氏叹了一声,“韩冈今年也就二十五六吧?放在他这个年龄,考上一个进士都是难得的很。可看他这些年立下的功绩,就是韩琦也要比他差许多。”

“韩冈是治世之材。”

“韩冈有才,德行也自不差,最难得的是敢于任事,就算偏远之地也不退避。日后当是能入两府,做宰相,”曹氏瞥了眼孙子,“不要让他没了好结果!”

赵顼抿起嘴,点着头,“孙儿明白。”

驾驭臣子,要有节、有度,不能超过应有的限度。自古宠臣,有好结果的不多。太过于受到重用的能臣,也往往难以做到富贵终老。而且世上也多有少年显贵,易于早夭的说法,甘罗十二岁拜相,但他连弱冠之年都没有活到。

治世之材,必须要多多历练,韩冈需要的是在地方上的历练,而不是未及而立,便侧身都堂之中。

“孙儿会好生安排下韩冈的。”

……………………

海门镇地处富良江的入海口,出产并不算丰富,加之两百多年前,还是行交州治所的时候所修建的海堤,这些年来毁损严重,使得自海岸,往内陆去的十来里,都是一片无法种植粮食作物的盐碱地。

不过这座港镇,至少还能看得出旧年的规模。城墙周长五里许,虽然无法跟好大喜功的李公蕴建起来的升龙府,但比门州还要大上一圈。只要稍作加固和修补,就能变成一座镇守天南要塞。

新的港口就在紧邻海门镇的地方修建,旧日的港口不敷使用,因为所处位置不佳的缘故,就连扩建都有些麻烦。

带着工匠,章惇和韩冈派了亲信,一路在海门镇境内绕着圈子,寻找着更适合安排港口的位置。从河口到海边,用了两三天的时间,工匠终于找到了一处更为合适的位置,就在海岸线上。

章惇和韩冈在忙碌中抽出空来,跟着去见识一下最合适的地点。

海边的空气带着几分咸腥,但海天一线的辽阔,让第一次看到大海的人们,从心底里叹为观止。

就是韩冈有些例外。自从来到这个时代后,他还是第一次看见大海,不过并没有什么感触,也没有分心去看风景,一门心思的就放在了修筑海门港的上面。

章惇对于韩冈这等对海上美景视若无睹的态度,感觉很是奇怪,“玉昆,你可是出身关西,怎么看到大海一点没有反应。”

再怎么说,在看到一轮明月从海中冉冉腾起的时候,但凡士人至少该感慨一二。但韩冈却是什么话都没有,很是让章惇觉得匪夷所思。就算是李宪,可也是在海滩边望着大海愣了半天才回过神来。

“为何?”韩冈正专注的看着工匠们画出来的图纸,闻言讶异的抬起头,“正事要紧吧?”

“难怪玉昆你做不得诗赋,只是心境上就差了一层。”章惇摇着头,感慨不已。他估摸着这就是韩冈为什么不擅诗赋的原因了,“诗词歌赋,言情言志,皆是发自肺腑。玉昆你对这天地造化的景致视若无睹,哪里可能做得了诗赋。”

韩冈啧了一下嘴,凯旋在即,章惇倒是有心情拿自己开玩笑了。也不想想,海门港规划才开了头,不在上京前将千头万绪的事务给敲定下来,走了之后,可就是要乱作一团,不知会拖到哪一年去。

韩冈并不清楚朝中对自己的安排基本上已经达成了难得的默契,但他知道,章惇作为主帅,过段时间肯定是要领军凯旋回京,在京城中宣扬此战的辉煌战果。

当章惇离开了之后,为了保证广西局势的稳定,韩冈这位转运使就不可能同时离任,至少要有半年以上的间隔。

记得当年河湟开边胜利之后,王韶凯旋归京,而自己则是留下来处置后续。五六年过后,自家还是少不了这样的差事。

最好还是早点将最后一点工作给完成,然后试试调回内地,凭着自己的功绩和手腕,到了任何一路,都能轻松胜任。

不过在这之前,还是要将海门港建起来,要控制住南洋,一座合格的港口必不可少。

