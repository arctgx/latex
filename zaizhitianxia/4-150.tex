\section{第20章 冥冥鬼神有也无(26)}

【忘了设定时发布了,抱歉,抱歉。另外白天有事外出,中午的一更会迟一点。】

盖着经略招讨司大印的军令,被信使们分头带了出去,去召集已经分散到富良江南岸各处州县的溪洞蛮军。

接下来的数日,雨势忽大忽小,就是不见停歇。富良江水越发的汹涌澎湃,滚滚浊流奔腾之声竟然如同雷霆重鼓,时时刻刻冲击着营地。

而平陆上,更是水坑处处,雨水集合起来后甚至都形成了道道小溪,向低洼处汇去。

幸好官军扎营的地点,地势要高出平地数尺,其实主体就是离着升龙府城不远的一座村庄——交趾年年雨水都不少,时常又洪水泛滥,村庄多半都是尽量建在高处——水流也只从营外绕过。

不过按照向导们的说法,这样的一开始下就不见停歇的情况,在过去也并不多见。正好撞上雨水多且早的年份,这运气可以说是背透了。

就在这几日中,受到召唤的部族几乎都到齐了,城外远远近近的村庄一个个被他们所占据。虽说是要参与攻城,必然会有所损伤,但一方面是关系到最后的分配,任谁都不敢也不愿缺席,另一方面则是因为对这场提前了太多的暴雨感到有些胆寒,希望能离着主心骨更近上一点。

黄金满并不是第一次来到升龙府,不过他之前来这座天南最大的城市的时候,是屈辱的作为降伏交趾的臣子来献上贡品的。好不容易积攒下来的一些财货,都被交趾人搜刮过去,广源州产的金块,只要是稍大一点,都得双手献上。

而做了大宋的臣子之后,自己有俸禄不说,他所控制的广源州,要上交朝廷的贡赋也只是象征性的收取。一年四两黄金加上一点土产,只是代表朝廷对广源州的统治,听说是用来供奉太庙,根本没有交趾那般穷凶极恶的态度。

对比过交趾和大宋的区别,有些洞主或许会想着保守实力,但黄金满却打算要为大宋尽全力攻城。只要讨好了如今领军的两位大帅,还有燕达、李宪这些在天子面前说得上话的重臣,有皇宋官军作为倚仗,南方的这一片土地上,还有什么能阻止他黄家成为交州境内举足轻重的一个大族?

他只是遗憾如今是下着雨,换作是晴天,那就容易得太多。城中的屋舍全都是竹木所制,只要用神臂弓将火箭射进去,或是用将自己的儿子惊得咋舌不已的霹雳砲,将点着了的油罐投进城去,风向合适的话,就能将全城给烧得一干二净。

很快,黄金满被引进了帐中,在所有的洞主会聚一堂的时候,他让人羡慕的站在第一位。

有别于军议时的忧虑,章惇在一众蛮部洞主们的面前,显得自信心十足,“城中交贼只是苟延残喘而已。天理循环,报应不爽。当日李常杰是如何攻下的邕州,如今本帅便要如何攻下升龙府。”

“雨中的确攻城不易,但守城也同样困难,箭矢难以派上用场。”

“当初邕州有援军,今天升龙府可不会有援军!”

“如果交趾敢于出战,自有官军来抵挡。”

章惇的一番话说得人人都安心了许多,有雨水的掩护,不用担心城中射出来的箭矢,冲到城下将土包丢下来,这个倒是一点也不难,只要官军能挡得住交趾军出城逆袭就行!有些部族甚至都不用官军掩护,因为他们顺道带来了他们刚刚在富良江南岸劫掠到的生口。

外无必救之军,内无必守之城。可话是如此说,但交趾人的援军就是天上不断落下的雨水。

升龙府不比东京。

出了周围五十里的东京城,城外依然是鳞次栉比、屋舍连绵的繁华地界。但出了升龙府之后,基本上就是田地和乡村,并不是繁华富力的市井。

这对于攻城的大军来说并不是好事。水稻田中最易积水,大部分雨水汇集成的水塘,原本都是种植着水稻。除了几条修建时就刻意加高加固过的官道,想从其他途径靠近升龙府的城墙,甚至就得划船过去。

章惇召集一众洞主训话之后,便立刻下令出兵。天上的雨水再大,其实也算不上有多危险,但军营在雨水中泡得时间长了,就会出大事。

安南行营是靠着干净卫生的饮食,来保证南下西军的低发病率,但现在想要保证干净的饮食,难度是越来越高。只要是水灾,往往就会引发霍乱等疾疫。食水不净,加上柴薪因雨水而难以生起,这样的条件不能拖过十天,再拖下去,士气低落不说,疾病就要营中流行了。

章惇和韩冈虽然自负,却也不觉得自己有司马懿攻打辽东时的水平,以他们对军队的控制能力,不能在冒险等待更为合适的机会,必须要尽快出战,攻下升龙府。

从官军大营出来,洞主们纷纷赶回驻地,一时号角连绵,响彻升龙府城外。

‘宋军要攻城了!’

听到了城外传来的号角,正在黄龙庙中,与李常杰一起,陪着倚兰太后和大越天子李乾德,向护国黄龙祷告的宗亶暗中一叹:该来的终于还是来了。

当初与李常杰一起领军攻打大宋,最后因为邕州大败,又与李常杰同时受到贬斥。不过李常杰的贬官,仅仅是做个样子,而宗亶的降责,则是实打实的被投闲置散。只是眼下到了危急关头,城中人心惶惶,还是宗亶这样有能力的将领能派得上用场。直接官复原职不说,还加封了爵位,并将家中子嗣尽数荫补,甚至连同兄弟、侄儿一同受了朝廷的恩惠。

只是他心中一点也没有底,军队并无作战之力,只靠着天上的雨水,又能起得到多大的作用,这一点,宗亶的心中很是怀疑。宋军绝不会轻易言退,既然领军来攻的两位帅臣之一的韩冈,正是当初从桂州一路疾行南下,打了他们一个措手不及的韩冈,那么指望雨季能将他们击退,几乎是幻想。

只是眼下正是在祭拜黄龙,宗亶自知不便多言。

“宋军要攻城了。”宗亶不便说出来的话,李常杰却说得毫无顾忌。

“太尉、宗卿。”倚兰太后对两位臣子的称呼,十分明显的体现了李常杰、宗亶二人身份上的差别。艳冠后宫的大越太后,双眉轻蹙:“不知京城的城防能否挡得住宋军?”

“章惇召集诸部联军汇聚升龙府外,本来就是为了攻城。眼下雨水未停,就强行进攻,其实是自取其败。只要两三次失利,就不会再有多少士气来攻城了。”李常杰对此深有体会,这是他充满血泪的亲身经历。

倚兰太后眼睛一亮:“也就是说,只要能将这一次的进攻打退,宋军便会撤兵了?”

“再下个两天,宋军再不退兵,就得做江里的鱼虾了。”李常杰的笑容中充满自信。天无绝人之路,就在濒临灭亡的时候,突然天降豪雨,提前了一个月出现的雨季,不是上天的安排,又会是什么原因?无论宋人攻城是用什么手段,即便是累积土山,李常杰都不在乎。有上天相助,心中有了底气,哪里还会担心什么,“宗太尉,你说可是如此?”

宗亶低头又抬起,似是在点头,对着李常杰、倚兰太后,还一直都静静的守在母亲身边的天子李乾德,回复了一个肯定的笑容:“就算宋人有什么阴谋伎俩,我大越自有黄龙庇佑,可保京城无恙!”

李常杰哈哈大笑着:“宗太尉所言正是!”

“拿笔墨来!”李常杰一摊手,仿佛主人一般,使唤着殿中随行的内侍。

饱蘸了浓墨的毛笔拿在手中,李常杰站在黄龙庙的正殿一侧墙壁下。素白的粉壁是前日得到李常杰的吩咐,刚刚粉刷过的,簇新簇新,甚至在角落处,因为雨水一直未停的缘故,还有着一些潮湿的痕迹。

在太后、天子还有同僚、侍从的注视下,李常杰在雪白一片的墙壁上挥毫疾书:

“南国山河南帝居,截然定分在天书;如何送虏来侵犯,汝等行看其败虚!”

一首诗写完,李常杰放下笔,他只是稍通文墨,书法也不出众,但这一首诗却是发自胸臆,不用多少虚词装饰,道尽了他的心情。

从头到尾又念过几遍,提笔书了姓名和年月,转过头来,李常杰复又纵声大笑,“太后、陛下,大越有神明庇佑,宋人贸然来攻,不知进退,此是自取其败。且稍等时日,臣必领军将宋人逐出国门,还我大越朗朗乾坤!高奏凯歌回师京城,以报太后、陛下。”

年幼的李乾德读着权臣写下的诗句,一张脸涨得通红,有些结巴的激动地说着:“等太尉得胜归来,朕……朕当亲为太尉置酒,共贺大捷!”

李乾德和李常杰一同陷入狂热的兴奋之中,宗亶附和着赔上笑脸,视线稍动,却发现年轻的太后眼中,却有着浓浓的忧色。

宗亶暗暗摇头,李常杰将希望全都寄托在上天,但看来并不是所有人都能如此放心得下。

