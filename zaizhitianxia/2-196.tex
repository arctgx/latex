\section{第32章 吴钩终用笑冯唐(三)}

韩冈随军从绥德到延州,又继续从延州南下,花了近十天的时间,一路抵达咸阳东北面不远处的泾阳县。陕西宣抚司的帅府,现在就暂设在泾阳县中。

山南为阳,山北为阴。水南为阴,水北为阳。

泾阳理所当然就在泾水的北面,但咸阳却是在泾水之南——咸阳之得名,是因其在渭水北岸,九嵕山南麓,兼有山水之阳,故而得了个‘咸’字——为了能让兵马顺利通过泾水,与前线相联系,河面上在原有的一条浮桥的基础上,又设立了两座浮桥。通过三条浮桥,种谔带来的五千骑兵,韩冈估计大约半个时辰就能过去了。

泾水虽是浑浊,但河边的柳树倒是不错。泾水两岸遍植垂柳,绵延上百里。如今正是春时,堤岸上芳草茵茵,百花繁盛,嫩绿的柳枝长长的垂在水面上,河面上一阵风吹来,飞扬起的柳丝如同一幅幅绿色的绸缎,是关西难得一见的胜景。

若是在往年,当已是城中百姓出城踏青的时候了,但现在的能看到的就只有来来往往的军汉。兵荒马乱的样子,让人感慨万千。

韩冈突然觉得有哪里不对,泾河灌溉着关中的主要粮区,取代了郑国渠,成为关中最为重要的渠道的白渠,也是自泾河取水灌溉。泾水两岸都是田地,青青的麦苗一眼望不到边,这是关中农业最为发达的区域。

可是现在,韩冈放眼望去,田间地头却看不到多少农民忙碌的身影。

这就究竟是怎么一回事?!

“都被调去南面,绕咸阳挖濠筑墙了。”

出来迎接种谔一行的是宣抚判官赵禼,而韩冈的师兄游师雄也混进了迎接的队伍中,现在与韩冈并辔而行。见韩冈纳闷,便出言为他解惑。

韩冈当场被吓了一跳,脸色大变的惊道:“泾阳、高陵、栎阳可都是关中粮仓啊!”

始建于西汉、经过泾阳三县的白渠,如今灌溉着大约四五千顷的最上等的田地,平均亩产接近三石。这在江南也许算不上什么,但在关中却是一等一的好地。四五千顷,换算成亩,那就是四五十万亩,也就是说,每年的粮食产量超过百万石以上的,韩冈说其是粮仓,那是一点都不夸张。

挖沟筑墙,用的当然都是征调来的民伕,但眼下,这可是要误农时的,经过了一个冬天,麦地正是需要施肥上水的时候,开春后不及时料理田地,白渠灌区的泾阳三县今年夏天还能有多少收获?这一百四五十万石的收获若是因此有个什么意外,整个关中都要出大问题了。

游师雄叹着:“赵郎中急着要把叛军都围起来,其他的事他哪想得那么多?”

“韩相公他就不管管?!”韩冈更为惊讶,韩绛好歹还是宰相啊,“年后关中灾荒,弹章可都要砸到他头上。”

“……玉昆你待会儿见到韩相公就知道为何他不管了。”

韩冈跟随着种谔进了泾阳城。与城外荒芜中的平静不同,城中是一片肃杀之气。城头上旗帜林立,而街道上来来往往的又多是巡视内外的骑兵。行人稀少,商铺大门紧闭,好端端的一座泾阳城,变成了边境的要塞一般。

一队种谔在赵禼的陪同下往帅府行辕行去,韩冈跟在后面,而走在种谔之后、韩冈之前的一名将领,则是同行南下的王文谅。

这个蕃将在罗兀攻防战打得正激烈的时候,奉命在延州北面的招安寨驻守,防备党项人偷袭延州。与种谔一样收到了领军南下的通知,在种谔、韩冈抵达延州的时候,与他和他的一千多蕃军会合,一起南下泾阳。不过种谔和韩冈都不待见他,一路上也没有搭过一句话。

在行辕外向里面通报过姓名,韩冈跟着种谔、赵禼,还有王文谅一起走进白虎节堂。

韩冈是宣抚司中属官,虽然位卑,但职分在此,走进白虎节堂的资格还是有的。不像游师雄,到现在也还不够资格,只能在门口候着——不过他也快了,大挫叛军、保住邠州不失的功劳,报上去后,以他的进士身份还有资历,多半就要由选人转京官了。

韩绛老了,这是韩冈见到这位仍是当朝首相的宰臣后的第一印象。

须发斑白,脸上突然多出来的皱纹,就像刚刚被犁过的田地。腰背也弯着,看起来这一次的失败,对他的打击不小。战场上的胜利无法掩盖他的失误,罗兀城的得而复失,让他也成了天下人口中的笑柄。

赵瞻倒是精神甚好,虽然他办的蠢事,让秦凤、泾原两路派来平叛的大军中的精锐损失了大半,但好歹已经把叛军围在咸阳城中了,天子和朝堂诸公都要承认他的这个功劳。

虽然韩绛仍是高踞于上,赵瞻站在下首,但两人的精气神明显有着鲜明的对比,难怪游师雄说看到韩绛,就知道他为什么压不住赵瞻的盲动了。

种谔、王文谅和韩冈三人行过礼,韩绛好言抚慰了种谔几句,但种谔脸色和回应都冷淡,看起来因为强逼罗兀撤军之事,两人之间的和睦关系已经破裂了。

韩绛看样子也无意与种谔弥合关系,摆摆手,示意三人站进班中。但赵瞻却在这时厉声叫了起来,“王文谅!你可知罪!?”

赵瞻的大喝声震内外,韩冈站进队尾,便回头看着热闹。而王文谅却仿佛胸有成竹,跪倒答话:“末将不知!”

“不知?!”赵瞻嗤笑一声,“吴逵口口声声说你逼他做反,你还不知?!”

“郎中明鉴!”王文谅摆出很委屈的姿态,“吴逵早有不顺之心,所以才与忠心耿耿的末将不合。现在赶着要杀末将,还不是因为末将曾经戳破他的心思。”

“种总管、白钤辖、程监押,哪一个没跟吴逵喝过酒?!”王文谅跪在地上质问着,手指一个个从堂上众将官身上划过,最后又一指韩冈,“还有韩管勾,前日他可是跟着吴逵同行了数日,一见如故。现在吴逵做反,不穷究他们不能明察吴逵反心,却来听着叛贼的话来处置末将,末将不知是何道理?!”

王文谅振振有词,也不怕得罪人,因为他知道,韩绛必然要保他。

听着王文谅把自己都扯进来,韩冈眼皮一跳,心中大骂,都这时候了还要攀诬。继而又很奇怪的看着堂上众将,以他们这群武夫的脾气,怎么不跳出来反驳?

“倒是伶牙俐齿,难怪能惑乱上官。”赵瞻冷笑一声,完全不理会王文谅的自辩,他转过来对韩绛道:“相公,这厮败坏国事,又惹得吴逵做反。当处以军法,让叛军无由再举叛旗!”

“不行!”韩绛果然如王文谅所料,拒绝得毫无余地,“不是本相要留着王文谅的一条性命,但这是朝廷的脸面问题,容不得向叛贼低头。”

不是韩绛不想处置王文谅,换作是任何人,灌注了自家多少心血的成果,因为亲手提拔起来的某个蠢货而功亏一篑,就算千刀万剐都解不了心头的怨恨。

韩绛也想杀王文谅,只是王文谅是他提拔起来的,两边的命运联系在一起,如果不能保住王文谅,那接下来,他不但颜面难保,还将直面政敌的攻击。

而且,若是真的按照叛军的要求这么做了,朝廷的体面该往哪里摆?王文谅再如何不是,都是朝廷命官,因为叛贼的口号,而杀掉朝廷命官。当年在贝州都无人敢作的事,现在倒还敢提出来?!只要韩绛点了头,御史台就要兴奋得跳起来,反倒是提意见的赵瞻不会有什么事——斗郎中哪如斗宰相!

韩绛的顾虑,其实在场的每一个人都能理解,这是很简单的官场常识,所以王文谅才有恃无恐。韩冈也知道,但他却完全没有保住王文谅的心思,这厮实在是太让人厌了。不过要解决王文谅明明有着变通的办法,只要多带过几年兵,又混多了官场,当是没有不会用的。

韩冈左右看了看,从种谔开始,下面的诸将都是木雕土塑般的一张脸,却是隐隐带着幸灾乐祸、看好戏的神色,他顿时明白了。

呃……原来如此!

看起来韩绛在这里的人缘真是坏透了,竟然没人出头帮他解决眼前的问题。当然,大概其中也有不想掺和进新旧两党的战争漩涡之中的因素在。

对于韩绛这个人,韩冈没有什么好感。但韩绛是王安石的重要盟友,而韩冈也算是新党的一份子——至少是被旧党看不顺眼——不管怎么说,都得顾念着一点香火情。最重要的是,王文谅这厮实在惹人厌,还是早早去死比较好。

韩冈想定,当即站了出来,向韩绛行过礼:“相公,下官有一言当说!”

韩绛深深的盯了韩冈一样,不知道这个在罗兀新立大功,深得军心的年轻人会说出什么话来:“你说!”

“以叛贼而杀命官,不但无济于事,徒留笑柄与人,此事必不可为!”韩冈先是一口否定了赵瞻的意见,在韩绛和王文谅惊讶的目光中,话锋一转,“但因为叛贼的谣言,使得王阁职蒙受不白之冤。还请相公下令,命王阁职领本部全力攻打咸阳,一则自雪冤屈,二则围城日久而不攻,已是兵老将疲,亦得振奋一下人心!”

韩冈朗声说着自己的建议,眼角的余光瞥着身边蕃将瞬息间煞白起来的一张脸,暗自冷笑:

‘王文谅,请你去死!’

