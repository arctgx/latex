\section{第41章 顺风解缆破晴岚(中)}

就在商人们窃窃私语的时候,前来观礼的官员基本上都抵达了。

今天来到方城县的有官员、有商人,还有被请来参加表演的官妓,在这么多双眼睛的注视下,第一列满载着纲粮的有轨马车,即将要运过方城山。

方城轨道投入使用已经是一个月前的事了。在山阴山阳两港的码头修好前的一个月的试运营阶段,以经过核算过的成本价,将轨道运力开放给普通的行人商旅。让他们可以体现享受一下轨道运输的好处。

经过了一个月的磨合,山阳港和山阴港中应付差事的官吏、士兵,都已经熟悉了手上的工作,在官员、商人云集的时刻,也不见他们有什么慌乱,一切井井有条。

沈括前两天就到了方城县,今天一大早就往山阳港这边来了。前任翰林、现任知州,沈括的到来,立刻让港中沸腾了。在他的身后则更是多达二十余人,有一半是来自于州中的教坊司。

“方管勾,韩龙图还没来吗?”方城知县向方兴询问着。

方兴摇摇头:“龙图不会来的。前两天,龙图还派人来信说,等功成之后将摆酒庆贺。”

其言下之意,自然就是现在的开通典礼,他是不准备参加了。

“是不是出了什么事情?”方城知县进一步追问。

方兴摇头,他虽然是幕僚,但并不代表他知道韩冈的个人私事。比起在白马县拯救灾伤的时候,他和韩冈更多的已经是上下级的关系。

他瞥了一眼方城知县,这名已经有中年富态的官员,眼中满是由热切转化下来的失望。

‘运气还当真不好。’方兴想着,要是韩冈来了,好歹能混个脸熟。

“今天来的人还真多。”方兴打岔道,“真不知道其中有多少是来看热闹;有多少是打算来探探底细,打算花点钱尝个新鲜,看看轨道是否可以当真将货物运去北面;又有多少是等着日后水运的?”

方城知县依然没精打采,随口应付着:“三一三十一吧。”

“在下倒觉得还是准备坐车的为多。”方兴说道,“水运应该还不多,毕竟还没有了解到运费的多寡问题。都是稳重行事的。”

单纯的水运,成本十分低廉,但襄汉漕运中间要经过方城轨道,加大了运输成本。不过趋之若鹜还是很多,一部分人想着坐一坐有轨马车,就当是尝个鲜。

“不过眼下已经没有多余的运力了,只能让他们失望而归。”方兴对着方城知县笑叹道,“今年是特殊情况,要在四十天内将六十万石纲粮运进京城,没有多余的空闲供给客运。等到明年,情况就能好上不少。”

“好?!……”方城知县勉强提起精神,冷笑的双唇上多了两分乖戾,“不知乘坐这方城轨道,收的税归谁,缴纳的费用又归谁?”

方兴脸上的微笑没有了,眼神也变得凌厉起来,这是在撒气呢,还是当真打算争一下税费的归属?

对于地方州县和代表朝廷的监司来说,权益的分配才是第一位的。

方城轨道是漕运交通的一部分。船行水上,没有说收通行费的规矩。在轨道的客运和货运交通开张之前,没有人意识到这是一门能赚钱的买卖——‘也许除了制定了规程的韩冈’,方兴想着——但眼下,只要看到商人们对轨道的兴趣,就都能听到叮当的铜板声在响。

轨道设在方城山中,横跨汝州、唐州,试问通过轨道赚取的收入该算是谁家的?

以襄汉漕运的重要性,客运和货运的收入,一年下来,十数万贯总是有的。如果汝州和唐州二一添作五,两州的知州,以及方城县和叶县的知县,都能得到一个优良的考绩,而且还能从其中混水摸鱼一番。但这笔钱,如果算进转运司的账本中,对于转运使也是一样的有好处。

盯着方城知县好一阵,方兴忽而展颜笑道:“过税的话,漕司当然不会跟州县争。”

方城知县没有被糊弄过去,“除了山阳港,唐州还有哪里能收过税?”

在普通官道上,商人带着货物穿州过县,少不了有税卡伺候。商货过一道关就是百分之二的税——这叫做过税。

为什么说‘千里不贩籴,百里不贩樵’,就是米和柴的利润太小,运费和税金一扣,运得越远亏得越多。就算卖的不是米柴等家常物,只卖些丝绢、棉布、药材、香料等贵重货物,也照样要付出成本的大半乃是数倍的税费。在原产地或是上货的港口卖得还算便宜的商货,到了京城,就是几倍几倍猛涨。

所以商人们恨透了税卡,却又不得不缴税。又不是贩私盐,能挑着担子在山里走小路,他们这等商人能不的能带的动自家的,就是能带着进了山中,多半就给人抢了。说是太平盛世,但这天下,也只有出了黄巢这样的狠角色的盐枭,才有胆量在荒山野岭中穿行。

但在轨道上奔行的车辆怎么收税?

轨道很明显有个特点,就是不能随意停摆。一旦马车在轨道上停下来那就不知要耽搁多少时间,就这么一条路,后面的车子超不到前面去,前面停了,后面就必须停。这么一来,整条路都不得不中断。

所以在韩冈制定规程中,所有的有轨马车,在路上都不能随意停下来。如果挽马出事了,丢下马继续上路,如果车子坏了,就要把车子拉下轨道,绝不能空占着道路。

所有的车夫和押车在韩冈的要求下,都经过了上岗前培训,一干应急预案应是让他们背了下来,一旦没有依照预案处理紧急事件,那就是绝无二话的重罚。

既然在两条路轨上奔驰的马车不可能停下来,更不能在道路上拉上一道鹿角坐地分财,想要在对坐在有轨马车上的乘客和货主收税,就只有在上车和下车的时候。

税卡一个县少的设一个,多的能设两三个,这是处于交通要道上的各州县极重要的财政来源,同时更是当地官员胥吏们的聚宝盆,他们从道路上弄到的财物远比官府要多得多。可商旅行人一旦在山阴、山阳港上了船,船只穿州过县,税卡还有什么用?

其实方城轨道还好,仅是六十里而已,只与唐州方城、汝州叶县有瓜葛,就是绵长的两段漕渠,也有汴河或是京东五丈河的前例可以依循。

但如果日后修建更长的轨道,两百里、三百里,穿越多个州府,那问题就严重了,会造成大量税收流失。

“韩龙图那里据说已经有了腹案。”

“什么腹案?”

“即是腹案,在下如何得知?”方兴笑道,“不过有一点可以肯定,龙图是不会让朝廷亏钱的。”

……………………

低头批阅着永远也不见减少的公文,吕惠卿突然抬起头来,“京西的漕粮大概就要开始运送了。”

“襄汉漕运应该是今天还是明天才开通吧?”吕升卿惊讶道。“不是说要等山阴、山阳两港的码头修好后,才会开始运粮吗?”

“码头没修好,运不了纲粮。不过普通的行人已经可以走这条线了,都已经开通一个月了。乘坐过方城轨道上的有轨马车的,已经有不少人抵达京城了。”

“赞好的多,还是骂的多?”吕升卿问道。

“倒是没听到几人在骂,多是叫好的。花费少、速度快,”吕惠卿漫不经意的将手上的公事放到一边去,“等京西事了,我倒想提议让他去河北主持修造轨道。不知他是想着独占全功?还是愿意大家分一分的好?”

“这不是分不分功的事。”吕升卿摇摇头,“有人想着一步登天,去乌寺呵佛骂祖一巡,就能直冲云霄。而韩冈是一心做事,比起走御史之路,更得圣心。就算强去挂个名字,最后也是得罪人。反正他又回不了京城,他如果想去河北,让他自己上书好了,何苦去招惹他?”

其实吕惠卿很清楚,韩冈在外做事,得到的成果有助于国,对稳定朝堂的政局有莫大的好处。既然如此当然就不能让韩冈回京,得让他一个路一个州的跑过去。

朝堂上的政争,最常见的结果就是一方得势,一方出外。失败的一方,并不会受到太大的责罚。不能在朝中任职,就是政治失败的象征。

韩冈眼下既然要在外任官一段不短的时间,朝堂上的重臣,没一个愿意去跟他过不去。天子不敢重用他,不代表天子不期待他的成果,若是有人干扰到韩冈做的正经事,天子也不会轻饶。只是吕惠卿另有想法。

“找个看起来没关系的,让他提议在河北铺设轨道。运兵、阻敌的好处让他好生的说上一说。”吕惠卿抬手阻止想劝谏的弟弟,“王禹玉想着稳住相位,竟然支持种谔直攻兴灵。正好眼下有这个机会,挡他一下也是好的。”

“韩冈支持种谔?”

吕惠卿摇头:“这就不知道了,不过也没必要知道。我这不是让他可以继续立功吗?”

