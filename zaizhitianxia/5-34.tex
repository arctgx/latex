\section{第四章 惊云纷纷掠短篷(八)}

【第二更】

前些日,环庆路发文来唱莲花落,伸手向群牧司要乘用马。韩冈一口气就给了一千两百匹。

接着,泾原和鄜延都看着眼热,有志一同的一起伸手。至于秦凤和熙河,要么是还没有来得及收到消息,要么就是伸手的公文已经在路上了。

当时群牧司上下都觉得韩冈给得太痛快了,将他们的贪心给惯出来了。缘边五路数万骑兵的胃口,仅仅是暂养在沙苑监里的四千一百匹马根本填不上。群牧使韩缜都颇有微词,只是因为是韩冈的决断,所以暂时放着,以观后效。

而韩冈不仅仅将暂养的四千匹军马送了出去,甚至把养在沙苑监内在册的超过十二岁的公母种马也全都给了前线的骑兵部队,在籍两千,实数也一千多匹——以沙苑监的育马水平,这一批年纪大了的种马留在监中也是浪费牧草。

可这依然不够,近万匹乘用马的缺口也只填了一半,但韩冈有办法:“等到这群牲畜上去,就能剩下的缺口给堵上了。只要朝廷愿意给钱收买,牲畜的主人不愿意的不会太多,反正驴子、骡子用来代步也是一样,不上阵就没问题。”

官军缺马。

没马的骑兵有之,只有一匹马的骑兵有之,一马一驴或是一马一骡的亦有之,反倒是能做到一人双马的马军指挥寥寥无几,契丹人的一人三马则更是只存在于传说中。大宋的骑兵部队,能有驴子代步,谁都没法抱怨。

这个道理也简单,点破了就不足为奇了。

韩冈的话说完,一个勾当公事率先拱手赞道,“龙图才智过人,我等实不能及。”

另一名官员也叹道,“朝廷昨天已经下了堂札,让京西和开封府调集牲畜,一个月之内,当能如数送抵关中。谁能想到龙图早就看到了这一点。”

韩冈端起茶啜了一口,将自嘲的笑容藏在了茶盏之后。谁让他一力反对这一次的军事行动?这时候自是得要钱给钱,要粮给粮,这样一旦出了事,谁也不能将责任推到自家的头上。

不过话说回来,就是他支持这一次的战事,也照样会这么去做。就是让几路兵马明着占便宜也无所谓,谁还会嫌事前准备太充分?

这个道理,在官场中混老的官员没理由想不到,现在一个个啧啧称叹,还不如说在叹息失去了看笑话的机会。

用已经变冷的茶水稍稍润了润喉咙,韩冈笑道:“只要朝廷肯花钱,哪里买不到马?为了打仗,几千万贯都拿出来了,拿个三五十万贯出来买民间的牲口,难道天子还舍不得。只要能赶得上时间就行。”

韩冈一开始就是将主意打到了京畿和京西头上,京营的禁军都去了关西,粮草也有许多是从京中调去,没理由说牲畜就不行。

用关中的牲畜,补马军的缺口,再用京西和京畿的牲畜,去补运粮队的缺口。这就跟运粮一样,只要节奏不乱,军马和粮秣都不会出问题。

韩冈一个上午就坐在公厅中拆东墙补西墙,做着泥瓦匠的活计。忙碌的时候,时间过得飞快。

很快就到了中午,就听见外面传话,“龙图,内翰回来了。”

韩冈站起身,带着厅中的几名属官走下中庭去迎接群牧司的主官,就看着韩缜面色不愉的摇着头进院来。

见了礼,回到厅中,韩冈让闲杂人等都退下,问道:“内翰,出了什么事?”

韩缜沉着脸:“种谔提前出兵了!”

“……我就说嘛。”韩冈先是一怔,继而轻笑起来,“种五怎么可能是老老实实、按部就班做事的人?!怪不得他最近这么老实,原来是打着这个主意。”

“玉昆!”韩缜提声喝道,“种谔这是为争功而枉顾君命,坏了大事,如何还能笑?!”

“这不是打了党项人个措手不及?”韩冈反问道,“党项人的细作肯定是将缘边各路的情况都送去了兴庆府,恐怕梁乙埋都比我们还清楚官军何时会出阵。种谔这一下,却是出奇制胜的一招,夺占银夏不在话下。”

韩缜立刻反驳道:“官军的目标不是银夏,而是兴灵。”

“银夏一丢,只剩兴灵一地的党项人,便是做困愁城。除了地势,别无他处可以借力。而且种谔出其不意掩其不备,也是将党项人的士气给打掉了。虽说是有违命之嫌,但结果并不差。”

种谔突然出兵,让朝廷和党项人同样措手不及,不过也顺便解释了韩冈之前的疑问。虽然韩缜很是愤怒,但在韩冈看来,只要结果好就行了。

‘看起来这一战还真能给种谔赢了。’韩冈心里想着。

韩缜却是抿着嘴,看了韩冈半天,最后才说道:“天子已经下诏,命种谔回兵!”

韩冈听得呆住了,楞了半晌,猛然站了起来,扬眉瞪目的厉声问道:“这是谁的提议?!坏了军心,他可担当得起?!”

“玉昆!”韩缜惊了一下,韩冈如此失态的情况都没有见过。他沉声提醒道:“这是种谔做错了事!”

“士气可鼓不可泄!出兵了都还能叫回来,合围兴灵,就可当没鄜延路这一路了。”韩冈张开双手手掌,十根手指比在韩缜面前,“几近十万人马,出阵官军的三分之一啊!”他连连摇头:“我要入宫!”

韩冈望着韩缜,正容说道:“虽说这场战争不是我韩冈支持的。但食君之禄,忠君之事。我也不能眼睁睁的看着天子乱命。”

“玉昆,等一等。”韩缜揽住了韩冈,“种谔为争先仓促出兵,粮草还没有集齐,就是京营的七个将也一样还没有到延州。种谔是孤军深入,粮草又不济,这是必败之局。”

“种谔若没有几分把握,他不会出兵。他敢出兵,自然是能因粮于敌。党项人囤粮的地点,必是已经遣细作都打探到了。”

“因粮于敌。”韩缜哼了一声,“这个风险玉昆你知不知道有多大?”

“党项人常做的,官军一样能做。”别人倒也罢了,韩冈却知道种家摆在横山南北的耳目有多厉害,“当年在罗兀城,之后在横山,种谔都曾因粮于敌,从来都没有在粮秣上出过差错。他是老用兵的,又是想立功劳,哪里会犯蠢?”

“玉昆,你与种谔相熟,所以信他,但天子能放心吗?”韩缜看了看韩冈,“即便给种谔做到了,但他提前出兵,其他几路人马会怎么想?如果朝廷不将鄜延路的兵马召回来,其他几路将帅会怎么做?是干看着,还是跟着一起出兵?其他人能有种谔的本事吗?”

“……承蒙内翰提点,但这件事韩冈还是得说。”韩冈沉默了片刻之后,向着韩缜拱手一揖,大义凛然,“今日一事,天子听与不听,自有其判断。但韩冈为人臣,却必须得说,否则便是不忠。”

韩冈现在已经冷静下来。

韩缜说得没有错,种谔的提前出兵,放在其他几路兵马的眼里,是彻头彻尾的争功——虽然他们这么想并没有错。但要是他们为了与种谔争功,一起提前出兵,那战局只会变得不可收拾。

各路的粮草都还没到齐,民夫、骡马也一样,甚至连兵马都还未完全就位。贸然出兵,最后的结果必然是大家都不想看到的。

方才的一番话如果没在韩缜面前说,那怎么样都没关系。但既然已经说出口了,就必须在天子面前留个底档。否则日后被捅出来,天子的面前可就难看了。韩冈可不会将自家的把柄放在不能相信的人手上。

匆匆辞了韩缜,韩冈便起身去求见天子。

脸上已经恢复平静,看不出异样,心中却是在叹息,

天子召回种谔,也是在情理之中,但为此付出的代价,却是六路中战斗力最强的鄜延路,就此只能做看客了。

种谔这个赌徒又是在赌,赌天子不会将已经出阵的大军给召回。

只可惜这一次他又赌输了。

等他回去后再协同其他几路一起出兵,鄜延路的兵马绝不会还有现在的士气。六路中兵力最多的一路,被当做主力的一路,仅仅是经过了一场垫场戏,就给废掉了大半。这一仗,想赢是越发的难了。

若说到赌性重,当今的将帅中,种谔算是排在第一。

当年他夺占绥德城,种谔就在赌,赌天子会留下绥德城,不会理会枢密院的反对。结果他赢了,同时也输了。绥德城靠了郭逵的谏言保住了,但种谔本人则是被投闲置散三年。

罗兀城一战,他又在赌,可惜摊上了一个不会用人的韩绛,使得广锐军叛乱,最后功败垂成。

横山一役,只是按部就班,不算赌博。但这一次,可就是把天子都耍了。置朝堂已经敲定的作战方案于不顾,先行出兵。很遗憾,他又失败了。

韩冈叹了一口气,往宫门走去。种谔是个一流的将领,但他仅仅是将才,而不是帅才。缺乏足够大局观,以及不会看人。

看错了韩绛,看错了天子,出现今天的局面,是意料之外,却也是在情理之中。

