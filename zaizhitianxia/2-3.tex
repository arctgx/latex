\section{第一章 一入宦海难得闲(三)}

【第三更,求红票,收藏。】

公厅中并没有椅子,一尺多高的桌案,本就是平放在地板上。做起事来,要么跪坐,要么盘腿箕坐,找张小几来坐,都会嫌高。韩冈就是盘腿坐在一张蒲团上,处理着递到他面前的公文。

韩冈抬手从桌面上已经分门别类送到自己面前的文件中,取下最上面的一本,展开一看,却是者达堡发来增修两座望楼,并配属两具八牛弩的申请。

“想不到他都已开始做事了。”韩冈轻声笑道。

刘仲武就是新任的者达堡主,前几日刚刚去上任。而他在上京的这段时间跟韩冈处得不错的事情,好像并没有被向宝知道,也许知道了当作不知道。当见到刘仲武在试射殿廷上大发神威,博来一个三班奉职时,向宝还在秦州月前新开的酒店绿柳居上,给刘仲武好生操办了一场宴席,又是赠钱赠物赠宅子,收买人心的手段做到了让外人看了觉得恶心的程度。

不过向宝这么做的效果却很好,至少他千金市骨的目的达到了。向宝在军中的人望也因此事而提高了不少,韩冈最近在衙门前的老兵那里,经常听到他们向钤辖长,向钤辖短的。

但王舜臣的心情就很不好了,真说起来,他积攒下来的功劳远在刘仲武之上,箭术也在刘仲武之上,刘仲武的机会本该是他的,但现在遇到刘仲武,他还要唤他声刘大官人。当天,王舜臣大骂了几句娘,然后跑到野地里一天,到了晚上射了一堆野味回来。只是一只兔子都给他用箭扎了十七八个洞,其他的猎物身上也都是一个洞一个洞的全是箭孔。用连珠箭射来的野味,皮是没法用,肉也是不能吃了,拔了箭出来,全丢了喂狗。

想起那几只可怜的兔子,韩冈就是想笑,转手把这份公文放到脚边。李师中要求所有与钱粮有关的公文都要通过他的手笔,刘仲武要修望楼少不得要用钱,而且八牛弩是国之重器,这种有三根弓臂组成的床弩据说在澶州城射杀了辽军大将萧达凛,直接导致了澶渊之盟的出现,刘仲武要这玩意儿,估计很难要到,就算向宝出面都没用。

韩冈就像处理刘仲武的申请一样,将桌上公文一件件的翻看,随手在自己准备的一个小本子上写上几个字做个简断的摘录,又一件件将之分类。他看得很快,判断也很准确。至少到现在为止,韩冈做的一直不错,如果在邮局,会是个出色的分发工。

桌案上的公文厚度维持稳定,而韩冈身边的公文堆则不断增高,这期间陆续又有秦凤各地的公文呈递进来,让韩冈完全停不了手。而且不仅仅是文件,来要定例的笔墨纸张的,要进架阁找旧档的,窗户坏了要找工匠修补的,都找了过来。

王启年他们十几人有三个是检查来往文书的文吏,有两个是管理架阁库——也就是管理档案——,剩下的还有的是撰写公文的书办,又有跑腿倒水的,还有做些力气活的。其中大半是长名衙前,常年留在衙门中奔走,剩下的几个则是来服差役的普通衙前。但与其他曹司打交道,他们却都躲了开去,让韩冈处理。

韩冈低头翻阅着公文,耳中听着传话和要求,一边在纸上写着划着,一边下令道:“王启年,你去找佥厅的笔墨杂用账来,慕容鹉,你去把佥厅要的笔墨纸张备齐;参议厅窗户坏了的事本官记下了,今天明天就会有工匠去修的。”

“抚勾,窦相公可是等着要三阳寨十年前的兵籍……”来自窦舜卿的副总管厅的小吏催促着韩冈。

“请窦副总管写个文字过来,本官才好开启架阁。没有文字,光凭你一个小吏空口说白话,怎么能妄自开锁?要快的是你,拿了窦副总管的文字就快去快回,莫让副总管等的心急。”

如果除去恩怨不理,王启年等人还是挺佩服韩冈做事爽快麻利。当然,这样的长官,没有一个胥吏会喜欢,好糊弄的哪种类型,才是他们的最爱。

大概花了一个多时辰,桌面上的公文方才消失一空,而陆续来勾当公事厅办事的吏员也被韩冈两句一个的打法了个干净。几个小吏走过来,把韩冈身边的几堆公文,一堆堆的抬出去,按着分类送到不同的衙门中。韩冈上午的工作也总算告一段落,而上午的时间也告一段落——就在韩冈的忙碌间,已经是中午了。

“玉昆,歇下来没有。”王厚在门外喊了一嗓子。

“不耽误事。”韩冈回了一句,却又拿起笔,在自己的那本小本子上记着些什么。

王厚笑着走了进来。三个月的时间里,变化比较大的,也有他一个。大概是这段时间王韶让他独立处理了不少事,使得王厚的性格比过去变了不少,人也精干了。

“玉昆,新来的朝报你看到没有?”

韩冈自早上过来,就忙得不可开交,哪还有时间看朝报?何况以他的资格也不可能那么早看到,什么时候朝报给存到架阁库,他什么时候才有机会看,不然,就只能在王韶那里蹭着报纸来。“却是出了何事?”韩冈问着,手中笔却不停。

“猜不到?”王厚半开玩笑的问着,他也不惊讶韩冈一边说话一边写字的本事,本朝还有一边写诗,一边判案的高手在,韩冈仍差上一点。

韩冈摇了摇头,半真半假的抱怨了一句:“你真当我是瞎儿先生了?要不要我找几根草来,给你算个吉凶?”

王厚笑了两声,方才说道:“是关于今次殿试的事。”

省试的结果,韩冈回到秦州的那一天就知道了,省元是陆佃,据说是王安石的弟子。不过省元能做状元的却不多,殿试第一的状元不大可能是他。殿试是三月初,到了三月底的今日,载着今科的进士名录的朝报也该到了。

“殿试上能出了什么事?”韩冈问道,“该不会秦州今年终于出个进士吧?”

“怎么可能?特奏名倒是有几个!四个还是五个。”王厚嘲笑了一句,也不卖关子,“照故事,殿试的内容是诗赋论各一篇,本来今科预定的也没有不同。但编排官准备分发《礼部韵》【注1】的时候,天子却突然下令,韵书不必再发,今次殿试考题改成策问。”

“策问?!”韩冈笔终于停了,双眉纠结起来。

他没想到赵顼是这么的沉不住气,也不与朝臣再行商议,便做出了决定。虽然常言道殿试定高下,省试定去留。殿试的结果只关系到名次的高下,是否是进士,早在省试结束后就决定了。但他这么做在所带来的政治影响,却远大于殿试的范围。而且既然今科殿试用得是策问,下一科的考试科目为何,等于已经向天下公布了。

“玉昆,听到这个消息难道你不高兴?!”

高兴什么?本来是仅属于少数人的消息,现在成了全国皆知的秘密,本来可以比天下士子多一年复习经义的时间,现在只能站在同一条起跑线上面。韩冈如何会高兴:

“下一科要改诗赋为经义,也不是没这么猜过。现在不过是证实了而已。”虽然这个‘证实’其实是早在一个多月前就已经证实了,但那件事必须得保密才是,“当日说起科举的经义诗赋之争,也是有猜过那一次只是试探,实际上改革的时机应是放在下一科。苏子瞻当日也许还以为自己赢了,谁能想到天子根本就没听他的,一直揣在心里。”

王厚回想了一下,好象是说过,也好像没说过,几个月前的随口闲聊,谁能记得那么清楚。他问:“不知玉昆你准不准备考?”

韩冈又拿起笔,忝了忝墨:“即使是解试,也要在两年后才开始,而机宜的拓边河湟,可是眼前的事。”

“眼前?!……眼前个鸟!”王厚也许是跟王舜臣一起玩得多了,口气也越来越像军汉,“‘阉’人不去,怎么个‘前’?!”

“还是因为王、李两位?”

“还能是谁?”一提起两个可恶的阉人,王厚心中烧得就不是火,而是火药。王【和谐】克臣、李若愚两位内臣奉命体量秦州宜垦荒地,等他们到了秦州后,在秦州城中走了一圈,就上书说窦舜卿错了,他所说的一顷四十七亩其实是有主的,已经给人认领了回去。秦州的宜垦荒地,其实一亩都没有!王韶和窦舜卿,都犯了欺君之罪。“那两个没卵蛋的阉狗,到了秦州就搅风搅雨……”

韩冈忙扯了王厚一下,“小声一点,要骂也不能在这骂!”

王厚顿时惊觉,韩冈的公厅的确不是发泄怒火的好地方。被韩冈这么一打断,他也没心情说话了:“算了,不提他们。”

站起来,王厚就要走。走了两步又转回来,苦笑着摇头,“都给那两个阉货气糊涂了,本是想做个东道,找玉昆你去衙门外喝点酒的,扯了一堆闲话都给忘了。”

“处道兄即是要请客,小弟哪有不愿的道理。”韩冈将笔一放,小本子收进怀里,丢了两句话,就跟着王厚走出官厅。

“玉昆,这样下去不行啊。”离开官厅几步,王厚便向后一指,“我知道你另有心思,但五个人的事压在你一人身上,铁打的也吃不消。”

“这几天虽然忙了些,但了解到了不少事,衙中的公文不亲眼看一看,不亲手做一下,就不可能明白。”韩冈看了不以为然的王厚一眼,又笑道,“不过处道你说得也没错,的确不能像这样下去了。拿着一份俸禄,凭什么让我做五个人的事?”

注1:中国自古方言众多,为了让考生不至于弄错韵脚,诗赋考试时,都会分发韵书,作为参考。

