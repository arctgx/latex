\section{第28章 临乱心难齐(九)}

韩冈说自己只是问问而已,但诸立怎么会相信。

不怕贼偷,就怕贼惦记。给官人惦记上,比被贼惦记还要让人害怕。俗话说匪过如梳,兵过如篦,兵比盗贼都厉害。而官员却更上一层楼,那可是一口就能全吞下去,一点汤水都不会漏下来。

诸立对面前的这位眉眼如刀一般犀利的年轻知县,有着一股说不出来的畏惧。一开始要算计韩冈的心思虽然还在,但大半已经是要用于设法自保,而不是当初预想的攻击。

保护家业的决心让诸立大起胆子,试探着韩冈的心意:“正言,如果只是小人这边降下粮价,对如今的情况乃是杯水车薪。小人家中也就那么几百石粮食,卖光之后,东京城的其他粮商过来还是要卖高价。就算正言强压着白马县的粮价,他们大不了不来白马县卖粮,到时候吃亏的反而是白马县中的近千坊廓户。”

“……那你有什么办法?”韩冈问着,平静的面容不透露任何信息。

诸立在韩冈的脸上没有发现答案,只能继续道:“如果开封府肯调出仓中存粮来发卖,只要数量有仓中两三百万石的三成、四成,这一百多文一斗的米价,转眼就能落下去。回落到六七十文一斗,也就三四天的功夫。”

“这事就不是你该说的了。”韩冈冷淡的瞥了诸立一眼,“此事天子和朝堂自会有决断。”

“正言说的极是!”诸立唯唯诺诺,一副谨小慎微的态度。但他跟着却又陪笑着道:“不过正言乃是官家钦点的进士及第,又是王相公家的娇客,身份地位乃是高高的在云霄上。过几年,侍制、学士的一路做上去,转眼就是宰执了。为官家和相公分忧也没人能说不是……”

诸立就是开封粮行行会的一份子,又是宗室的亲戚,跟东京大粮商们当然不会没有联系,当然知道如今粮行的靠山们究竟是在打什么主意。韩冈是王安石的女婿,如果能从他这边探听到消息,对行会的图谋起到作用,自家在行会中的地位当然水涨船高。

“若是朝廷当真开仓卖粮,你这等粮商可不就要少赚不少?”韩冈单刀直入的问着,“不心疼吗?”

“只要正言一句话,小人这就将家中的存粮全都拿出来开粥场,一文钱都不要。”诸立挺着胸口,言辞动情,感慨着:“小人家中虽算不上富裕,可吃饱穿暖还是能做到的。钱财本也是身外之物,若是能为子孙积攒些阴德来,怎么样都是合算的。”

诸立会说话,言辞恳切,一幅真心诚意要做善事的模样。不知他根底的恐怕一看他正气凛然的样子,就会全盘相信了诸立所说的一切。

“你有这份心就行了。”韩冈也神色缓和了一点,只是心中却全然不信眼前的这名押司,会为了什么阴德而舍了家财。

好人在衙门中可做不长久,诸立在白马县衙做吏员做了三十年之久,心肠早就黑透,泡在水里,都能拿来写字画画了,哪里还会有这副好心肠?!骗鬼去吧!就算当真给平白拿出来,也是要用东西来换的。

心中的想法,韩冈只是不说,到时候看着就知道了。不置可否,却另挑话头,问道:“城中的药房是不是也是你家开的?”

诸立暗恨韩冈,话题说转就转。却也得老实回答:“只是间生药铺子,小人仅仅占了两成股而已,不能算是小人的。”

韩冈闻言一笑:“是哪一家要在县中开药铺,硬被你坐地起价,吞了两成干股?”

“小人哪里敢如此!”诸立连忙叫起了撞天屈,“生药铺的东家肖白郎,可是娶了位县主,正儿八经的环卫官,小人哪敢得罪他?他将生药铺子分了两成股份,那是看着小人在白马县中做了几十年的事,微有薄名而已。但那两成股,小人可是真金白银的掏了出来买的,一点价也不让。”

诸立的话,韩冈还是不信,只是他的注意力被其他事给带了去:“肖白郎?”

“正是。”诸立点着头,“肖白郎人称肖生药。是东京城药行的行首之一,药铺开遍了开封府各县。”

韩冈记得好像在哪里听过这个名字,就是一时间想不起来。不过他想了一想之后也就罢了,这种似曾相识的既视感,过去也有过,反正不会是什么重要人物,不过是个药行行首而已。对比起粮行、粪行、车马行这等事关民生、人力物力充裕的大行会,药行在东京城三百六十行中,地位排名要靠后不少。

诸立偷眼看了看韩冈,问道:“不知正言问及药铺,可是有什么要吩咐小人的?”

“想必你也知道,本官要在白马县开设疗养院,以收治百姓。”韩冈在白马县的主要精力还是放在灾情上,但该做的事也不会忘掉,“等疗养院开起来后,有医生坐馆的同时,对外也会向发售汤药。到时候,不免要影响到县中其他药铺的生意。”

创立疗养院,药材乃是第一位。不过韩冈没打算购买私人的药材,直接向开封府要就可以了。东京城中本就有施药局,为百姓免费诊断,并平价散出汤药,所以药材是不缺的。

诸立脸色微变:“难道要免费施舍汤药?”

“那还不至于。”韩冈说道,“免费施药那要看情况。给得起当然要给钱,实在给不起,也不至于将人丢出去。还是以人命为重。整体上要保证不折本。”

韩冈并没有廉价卖出药物、并免费医治百姓的想法。要想让一件事长久的维持下去,稳定的利益收入才是关键。不惜工本的好心施舍百姓,迟早会被嫌浪费钱的官员奏上一本,不是直接将之废除,就是另外加捐向百姓摊派,绝不会从官员的俸禄中挤出钱来。

舍弃了利益的善行,从来就不可能长久,迟早会停止或是变质。

《孔子家语》中,曾有孔子批评弟子子贡的一番话。当时鲁人多被卖到外国为奴,鲁国由此定下法令,如果有人将在外为奴的鲁人带回,将会给予相当数目的奖励。但子贡带回一名奴隶后,却推辞了赏金。孔子听说后,就批评他这件事做错了。

“赐失之矣.夫圣人之举事也,可以移风易俗,而教导可以施之于百姓,非独适身之行也,今鲁国富者寡而贫者众,赎人受金则为不廉,则何以相赎乎?”——圣人所做的事,都是用来移风易俗,通过教导而让百姓能够仿效,并非特立独行只有自己能做到。如今鲁国富者少而贫者众,若是赎人后领取奖励被认为是不廉,日后又还会有几人损害自己的利益而去赎人?

而结果也正如孔子所料,‘自今以后,鲁人不复赎人于诸侯。’

一心专注于利益,当然不是件好事。但视利益于粪土,而将道德标准抬得过高,又会有几人能遵守下去?如今多少人高喊着君子不言利,可事实却是伪君子一堆,真君子难觅踪迹。

堂堂宰相,为十万贯争夺寡妇。榜下捉婿,也是明码标价。说的和做的早就不是一路了!

韩冈始终秉持着有利才会有义的想法,疗养院的制度要面向民间,而不仅仅局限于军中,就必须成为一项可以赚钱的生意——医者父母心,但医生问诊都是要收钱的,此亦是常理。

可就不知道他以此来推行疗养院制度时,会不会惹来一片反对声。

毕竟《孔子家语》在考据中是被人指称为伪作,经史子集四部分类中,原属于经部,到了此时则降入子部,不再视为记录孔子言行的经典。

……想拿来做证据,也许还是徒劳!

……………………

诸立从韩冈那里告辞出来,疗养院的事他并没有挂心多久。就算韩冈是免费施舍汤药,亏的还是肖白郎。自家的本钱在地皮上,在粮行上,还有乡中的田地上,生药铺的收益对自家来说只是略有小补而已。

转头他就得到了消息。昨天快入夜的时候,从东京城相府来人,进了县衙中。说是王家的二娘子,也就是如今的知县夫人,已经从关西到了东京,特来通知,过几日就能到白马县了。

“这情况就不对了。”连诸霖都听出了其中的问题,“传递消息而已,在县中歇上一日又没什么关系,有必要赶得这么急?昨夜到,今天早上就要赶回去。竟然还要准备马车?!”

“而且来的人也太多了,这等口信,一个人来传还不够吗?”诸家老三也说着。

诸立点头道:“不出意外,不是相公家的两个衙内,就是其他能参与公事的幕僚或是戚里,必然是有大事要与韩正言商议。”

诸霖一听,便俯身凑前:“要不要去知会东京里的那几位?!”

“不打探明白说得究竟是何事,说了也不会有人放在心上。”诸立摇头。诸家虽然在白马县势力广大,但到了京城中,却是一条小鱼而已,“只有打听明白,呈报上去才会有好结果。”

“怎么打听?”诸霖皱着眉头。

诸立胸有成竹的笑着:“既然是来商议一桩大事,今日东京城内必然有什么变动,竖起耳朵仔细听着好了!”

