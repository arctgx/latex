\section{第21章 飞逐驰马人所共(中)}

韩家一行人的马车,在一里长的赛马街上用了近两刻钟,才抵达了目的地。

车子刚停稳,韩冈的两个大一点的儿子当先跳下来。然后金娘也想跟着跳,却被严肃心紧紧的抓着手,被婢女扶着下了车。看她一脸不情愿的样子,若不是严素心抓着,金娘就跟着她的兄弟一样从车上蹦下来了。

韩冈今天带出来的就三个年长的儿女,剩下的几个年纪都太小,只能留在家里,由周南照看着——韩冈本也想将周南一并带出来,但周南自从进了韩家门之后,尽可能地不抛头露面。尤其是在京城,甚至都不愿出门,韩冈也不便勉强,只能由着她。

韩冈早一步从车中跨出来,回手又搀扶着王旖下车。三个孩子正兴奋的左右张望。钟哥、钲哥还闹着要护卫把他们抱起来,好看得远一点。可被王旖板着脸一招呼,立刻就老实了。回到王旖的身边站着,一起望着赛马场。

东京的赛马场,其实只是类似于碟子形状,中央凹陷四周凸起的土围子,四周用夯土垒起看台的地基。不过上面还用炼钢后的废渣,三合土以及水泥,一层二层三层的铺上去,不惧被水泡坏,可以直接当成座位。

只是大部分时候,挤一挤能容纳近三万人的大型赛场,里面的观众都是站在看台上冲着场中狂呼乱叫。至于水泥台阶座椅,在看台上的人们心中自然是不存在的。只在位置最好的两排包厢里有正经的座位,坐在里面的人,比较注意自己的形象。

尽管赛马场的形制如此简陋,但此时的赛马场外面已经是人山人海,韩冈个子高,向周围望过去,黑压压的一片攒动的人头。个子矮的如韩云娘,视线更是被挡得严严实实。

韩云娘透过面纱,望着周围人山人海,惊讶的微张着嘴,扯了扯韩冈的衣袖:“好多人啊。三哥哥,怎么这么多人?!”

寻常蹴鞠比赛,能有个几千人来看比赛已经很了不得了,只有到了季后赛,人数才会上万。而到那时候,就得借用京城内外的几处大校场来作为比赛的场地。韩云娘方才才听说了,这个赛马场,自从第三个月开始,每一次比赛日,人数都从没下过两万。

“到这里打发时间省钱啊。”韩冈笑着对云娘说道,当年的小养娘如今都已经做了母亲了,但很多时候她举动还透着天真。

韩冈指了指周围。

就在赛场的大门外的广场周围,有着一圈店铺和楼阁。有的是酒馆,有的则是,几条小巷深处,还有一些私窠子,让中了马票的赢家能将他们赢来的钱都花出来,至于输家,可以去酒楼里借酒浇愁。

“没有城门税,尤其是酒水的税比城中少一半,在这里吃喝的花费比城里面低了整整两成。更别说这里也有瓦子,看百戏,看杂剧,都有地方去。到这里来打发时间,省的钱不是一成两成了。”

“四表叔不得了呢。”王旖在旁边轻叹着。

虽然出身耕读世家的王旖不喜家里满是铜臭,但冯从义将生意做到如今这个地步,带领着雍商闯遍天南海北,已经是陶朱公一般的能耐了,谁还能小瞧他?

韩冈不由得也点点头:“义哥就是没我给他撑腰,他照样能打下一片天地……二表兄也是如此,他在河北一番成绩让人赞不绝口,前几天,天子就批复了枢密院的札子,给他减了两年磨勘——不打仗,武将想减磨勘,只比登天简单一点。”

因为韩冈的缘故,李信可不是军中重点提拔的对象。能在和平时期立下减少磨勘的功绩,他的能力可想而知。

“或许还是外公家的传承好,给点机会就能冒出头来。”韩冈笑得有些让人捉摸不透。

从一开始,冯从义就打定主意,将赛马场打造成一个类似于京西瓦子的综合性娱乐场所。将同样属于娱乐的项目聚集在一起,让每个人都能找到自己的乐趣,便能像漩涡一样吸引人气——这里可是京城,天下财富汇聚的地方。京城里的人气,便代表着无穷无尽的金钱。

怎么才能更好的将钱从客人的腰包里掏出来,无论哪个时代,商人们都是舍得动脑筋的。韩冈在这方面,远远不如他的表弟,还有其他精明过人的商人。

“不论包厢,一张入场的赛马门票都是十文钱,与蹴鞠联赛相当。从京城西门和南门外过来,坐马车也不用太多的花费。乘坐专门走城门到赛马场这条线的四轮马车,十来里路,一人只要五文钱就够了。在这里,并不只有单纯的比赛和赌博,城中瓦子有百戏,有说书,有杂耍,还有男女皆赤膊上阵的相扑,这里一样也有。就是不想看蹴鞠和赛马,一样能在这里找到乐子。比起州西瓦子、桑家瓦子、朱家桥瓦子这样京城中有名的去处,花费还要便宜不少。”

韩冈板着手指头跟王旖说着,“对于京城中普通的士民来说,也就是一顿午饭钱,早上买水洗脸,还要两文钱呢,这点花销又算是什么?普通家庭,一个月来个两三趟不成问题。许多人每隔三天的比赛日,都会准时来报到,更有的闲人,一天到晚都泡在这里。”

只不过论起吸引力,终究还是蹴鞠和赛马更胜一筹。除了春播秋播、夏收秋收,以及一些重要的节日,两项联赛不得不停摆。一年中的大部分时间,蹴鞠也罢,赛马也罢,都是最为吸引观众的大众娱乐活动。

“官人知道的还真多。”听着韩冈将赛马场吸引人的地方娓娓道来,王旖笑着称赞,声音中似乎还是带了点揶揄。

“或许在一些人眼里,治国只在耕战二策,一手持剑,一手扶犁也就够了。除五蠹,抑工商,国家才能安稳。但治政得认清现实啊,已经不是秦人争天下的时候了。现在的大宋,若是没了工商二事,国政完全无法维持。如今的天下,乃是士农工商,四民是缺一不可。”韩冈笑道,“在《淮南杂志》中,复井田、循周礼,这六个字,岳父可是长篇累牍的在说。但岳父执政后,以变法清扫天下积弊,但这田制可是动都没有动,复井田的念头再没有提过。”

“每次说两句,就立刻一通大道理,官人你跟爹爹辩去好了。”王旖扭过头去,使了小性子不理韩冈。转过来盯着三个小儿女,不许他们太闹腾。

韩冈无奈的笑了一声,在士大夫家里长大,有些观念在王旖头脑中根深蒂固,纵然能明白韩冈的正确,也无法全盘接受。当然,韩冈也清楚,有事没事的对家里人说这些大道理,本就是自家的错。

转过头来,韩冈看着这一片赛马场外的广场。

这里其中房子和地皮的产权都属于赛马总社,就是更外面名号已经约定俗成的赛马街,两边的店铺也有一半属于赛马总社。可以这么说说,这一块的地皮,在赛马联赛启动之后,就从连种地都要折本的荒土台,变成了一座金矿。

创办还不到一年,赛马总社在财力上就已经直追齐云总社,当世的两大运动在受人欢迎的程度上无分高下,不过在场地规划和布置上,任何一座球场都要逊色于赛马场。

已经买了票的观众进了场中,但广场上还有许多人。七八个小摊贩穿梭在人群中,卖些菓子、水果之类的零食,生意倒是很火爆。愿意拿钱买点零嘴看比赛的人多得很,让这些小摊贩忙得脚不沾地。

韩冈也不管妻子教训儿女,站在人群外,饶有兴致的扫视着。

很快发现拥挤的人群中,大部分人都一只手拿着各色小吃,另一只手则压着藏着钱囊的衣襟。走动时小心翼翼的,看起来是防着小偷。不过其中有好几人,从他们衣服上透出来的痕迹看,藏在怀里的可不只是钱囊,短棍一样的形状,分明就是千里镜。

什么时候望远镜已经这么普及了,韩冈惊讶莫名。而且此时禁令犹在,光明正大的将千里镜拿出来,难道就没有人担心后患?

在韩冈一家下车后,何矩陪着韩家的管家吩咐了车夫,将马车和骑乘的坐骑一并赶到附近专门的车马场,又跟前面安排在此处等候的手下。一通忙活之后,终于又小碎步跑了上来:“学士,小人已经在场内安排好了包厢,请学士和夫人跟小人来。”

不过当他注意到韩冈的视线方向,只瞥了几眼,便对韩冈心中的疑惑了然于心,低声笑道:“学士,在这赛马场上,没有千里镜可看不清比赛。”

“不怕开封府来查?”

“这里可是城外,由祥符县管,开封可是隔了一层。”何矩声音更低。

韩冈从鼻子里哼了一声,祥符县中的情况那是不用再多问了。却也不多想,一边跟着何矩向赛马场中走去,一边又道:“千里镜可不便宜,也亏他们也买得起。”

“学士有所不知,千里镜这个月已经回到了原价上了。”何矩陪着笑脸,韩冈脾气温和,倒让他的胆子大了起来,揭开了一些瞒上不瞒下的秘闻,“从将作监和军器监两座玻璃窑中流到民间的镜片,一个月就有几千片之多。换成千里镜,一千架总是有的。因为玻璃的缘故,白水晶这两个月降到了之前的六成,用得起水晶镜片的人也多了起来。而且人工也便宜了,会磨镜片的匠人,京城里面差不多有百十个了。”

“只要能赚钱,砍头的买卖都有人做。”韩冈笑着摇了摇头。

“是赚大钱。”何矩强调道。

“是啊,有三倍的利就够让人拼命了。”韩冈感慨着,当真是古今如一啊。

“小人听说陇西的玻璃窑已经开炉了,日后可是一桩一本万利的好买卖,三五倍的利肯定是跑不了。这千里镜,过些日子肯定又要落下几成了。”

“暂时别指望,镜片一时还出不来。”韩冈收敛起脸上的笑容,有些头痛的说着。

可能是原材料的问题,陇西的玻璃作坊已经烧了几十炉出来。白玻璃的确有了,杯碗盘盏、花瓶灯具,也都一一试制,弄成平板形状在技术上也成功了。但用平板玻璃磨制透镜,却始终没办法成功。不是碎了,就是花了。

冯从义写信来向韩冈讨主意,可韩冈也没办法,只能回信让冯从义先去拿着平板玻璃做镜子。赚到钱后,吸引其他商人一并投入进来,到时候,也能让所有人进行技术攻关。韩冈并不在乎技术流失,通过竞争,促进生产技术的进步才是他的目标。

但话说回来,当行会规模到了一定程度,就算外人想挤进来,也必须向先行者低头,如今棉行就是一个最好的例子。

江南也开始种植棉花,尤其是长江口一带的通州、泰州、苏州新近淤积出来的荒地不在少数,越来越多的人在那里开辟荒地种植棉花,只是江南出产的棉布想要在京城中贩卖,却被棉行以行规给约束住,从运输到贩售不得不接受棉行的控制。要不然,棉行祭起降价的杀手锏,还没有形成规模的江南棉布,很快就会支持不住。

