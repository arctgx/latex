\section{第十章 千秋邈矣变新腔(23)}

哐的一声响,房门被重重的关上。

送了客人回来,疲惫不堪的宗泽已经没有了太多的力气,坐下来后,就不想再动弹一下。

成为状元已经过去了数日,宗泽门前依然宾客不绝,却也让他疲于交接。如果是正经言谈,纵是抵足夜谈也。可是过来的客人,都是些凑趣的、讨好的、打探的,甚至还有来讽刺的,这一干宾客,让宗泽实在提不起精神来与之交往。

敲门声响了起来,随即,主持和尚的声音也在门外响起。

宗泽轻轻叹了一声,站了起来。先整了整衣服——即便再累,礼节上的细小之事,他依然会注意——然后才过去开门。

一前一后,两个光头便出现在眼前。

主持和尚脸上完全没有作为房东的倨傲,笑脸上只有小心翼翼的谦卑,“状元公这两日辛苦过甚,清减了不少。贫僧寻了个方子,让人熬了点饮子,配上茯苓糕,正好可以滋补一下。”

老和尚轻声细语,与他白天训斥小沙弥时的声音截然不同。知道宗泽疲惫,特地送了滋补的饮子来给宗泽,还附带了几块茯苓糕作为夜里的甜点。

“师傅有心了。”

自从宗泽住进来之后,主持和尚的态度接连变了几次。一开始宗泽只是一个普通的国子监生,只是普通应对。作为一名在京师住了几十年,又在僧录司挂名的僧官,见过的官员、进士和贡生太多太多,普通的国子监生实在不值得他恭谨对待。

但得知宗泽曾经给快报写过文章,而且受到了很多重臣的赏识,立刻就变了一个人。等到宗泽得中贡生,继而通过了礼部试,再被太后钦点为状元,老和尚在面对宗泽时的态度一变再变,腰也弯得越来越低。

不过宗泽的回礼始终不变。以他的年纪,尚做不到宠辱不惊,但待人前后如一,不因成了状元而目无余子,宗泽还是做得到。

老和尚送来的夜宵,宗泽推让了一番,见无法推辞,方才收了。然后谢过,又寒暄了几句,再送了主持和尚出去。

重新回到房中坐下,看着桌上热气腾腾的银碗,宗泽只有苦笑。

他也知道,寓居的寺院,从主持和尚,到看门的火工道人,这两日都是兴奋不已。不仅仅是因为寓居寺中的考生里面出了一名状元,而感到与有荣焉,还有利益上的好处。

每日登门造访的多少宾客,在礼节上都会顺手给点香火钱。而更多地是一干为了沾点状元郎的光的客人,出手更是大方。

据宗泽从住在隔邻院中的一名国子监同学那边听来的小道消息,短短数日,在东京城中并不起眼的小小寺院,每天得到的香火钱,比他中状元前多了怕不有百倍。而且不说宗泽对寺院名气的提升,光是居住过状元郎的房间,想到未来会有多少贡生愿意以天价来租住,就足以让主持和尚抱着他的账本整夜整夜睡不着觉。就是寺中跑腿的小沙弥,也能多吃几顿狗肉了。

旧日同学与朋友一如既往的谈笑,让宗泽感到很欣慰,幸好有些事还是没有变的。

作为状元,宗泽除了迎来送往之外,也有许多工作需要负责。

比如《同年录》之类的主编工作,还有与其他同年的交往,再比如近在眼前的琼林宴。

可是到了夜阑人静,送走了最后一批客人,宗泽在灯火下回忆起前日殿上唱名,依然犹如梦中。

当日殿上唱名时,听见自己的姓名第一个被报出,宗泽几乎不能相信自己的耳朵。

自家的事,自家最清楚。宗泽很清楚自己考得怎么样,完全没有想到自己会被提到第一的位置上。

而且据事后传出来的消息,殿试考官们因为文辞犯忌,将自己排在了最后。但太后说好,宰辅们都不反对,自家便成了状元。

可回头再看一遍自己的文章。因为仓促之间临时改文,其实有很多值得商榷的地方,从结构到用词都要大改。若以这次考试的答案来算,完全当不起状元郎的称呼。

一个进士,已经足以让家中的父母与妻子感到欣慰,实在没有必要再加上一个状元的头衔。

名不副实,岂不是要受人耻笑?而且如今已经不是‘岂不是’,而是业已受人嗤笑。文章好坏,多少也有一个标准,宗泽的答案若是拿那个标准来衡量,不能算是合格。

宗泽尚年轻,对外界的攻讦,还无法做到一笑了之,也没有安之若素的厚脸皮,始终都在想着要如何得到世人的承认。

盯着银碗上的花纹,他的眼睛渐渐亮了起来。

既然无法推辞,那就干脆做到名副其实。

前两天宗泽听到一则消息,结合之前种种传闻,也算是可以确认了。

尽管一榜状元完全没有必要去,但宗泽觉得,

或许……自己应该试一试。

……………………

“这是勉仲你刚刚写的吗?”

韩冈放下了一张写得密密麻麻的字纸,轻轻拍着。

“不知参政以为如何?”黄裳虽也是在笑,但紧绷的肩膀看得出他的紧张。

韩冈看了他一眼,笑得意味深长:“比状元郎的要好。”

黄裳立刻一脸认真的追问:“可能入前十?”

能否中状元要凭运气,但天子不可能改变所有排名前列的考生的名次,所以真正出色的还是排名前十的考生。只是黄裳这么问,当真是想要与今科的进士们分个高下。

苏轼昔年为了反对新法,熙宁三年殿试策问,他也曾经跟黄裳一样凑过趣,然后呈了上去。理所当然的被赶出了朝廷。

黄裳这么做,虽不会像苏轼一个结果,却也不是什么好事。破坏抡才大典的权威性,这是朝廷所不能容忍的,不论什么理由都不可以。苏轼当初被逐出朝堂,也不只是开罪了王安石的问题。

“那就不好说了。”韩冈缓缓地说道,“或许可以,或许就又要受到牵连了。”

黄裳不让韩冈避开问题:“如果考官没有偏私,不知参政以为如何?”

韩冈认真的想了一下,“……这申论一题,勉仲你太占便宜了。”

尽管今科考官的水平不高,对申论一题的评判可谓是一塌糊涂。宰辅们能将名不副实的第一打回去,却也没精力去查阅所有考生的评卷,但毕竟第二题申论,几乎都没有得分,或是只得了七分半,对名次的影响不算大,策问一题写得好坏,基本上就决定了谁排在前面,谁排在后面。

可黄裳对申论一题的回答,却肯定能得高分,至少第三等。若不是按照制科一二等不授人的评分,第二等也是可能的。这样一来,就算策问不如人,在申论上就能将分数拉回来,甚至反超。当然是占便宜。

“参政说的是。”黄裳低头道,“黄裳素乏捷才,文字上也不擅雕琢。在殿试上,乍逢新题,的确难以应付,不如现在的深思熟虑。”

“勉仲你误会了。”黄裳的语气有些无礼,韩冈不以为忤,摇了摇头,“还记得申论考得是什么?”

“……实务。”

“正是。以处理实务的经验来说,勉仲你太占便宜了。”韩冈轻叹了一声,“这本就是为了御试所出的新题,可惜为群小所坏,只能先用在殿试上了。”

“是黄裳准备得太轻率了。即使以那六题为论,也应该通过的。”

“实绩比什么都重要。”韩冈道,“去一趟边镇,立下让人无话可说的功劳,回来后谁还能说勉仲你落榜之误?也可以让判你落榜的那几位一辈子不能得到重用。”

“用于不用,那是朝廷的事。而会被黜落,更多的还是黄裳准备不足。但黄裳若是去了西南边镇,不会遽然开始用兵,也许任内三年都会招募流民、开垦荒地、修建城池和寨堡。”

黄裳如此沉得住气,让韩冈很欣慰:“王襄敏昔年献《平戎策》,为先帝所重用,任官秦凤路。但他在大举用兵之前,整整用了三年时间在秦凤路上了解汉番内情,查探地理,以及搜罗人才。正是准备充分,所以当他开始用兵西向,遂一举功成。勉仲你若能如王襄敏一般三年不鸣,政事堂不会不成全”

“黄裳明白。”黄裳点头,他是当真明白了。

韩冈的态度很明确了,不支持黄裳将自己的文章递上去跟考生们争一个高下,那是完全没有意义的,韩冈更看重实际的才干。

像是心头放下了一件事,黄裳脸上的笑容顿时轻松了许多,他笑问道:“方才参政说黄裳能胜过状元郎,可是因为状元郎的策问不尽人意?”

“只是以论事为说,不为不佳,只是他是运气,遇上了太后能够体谅。”

不是能够体谅,是根本看不懂。黄裳腹诽道。群臣皆知,向太后的文化水平还不足以让她读懂一篇文章。

“状元郎的文章,黄裳也拜读过了,的确多有恶犯之词,幸好太后有心求言,故而将他提到第一。”

“是啊,不然这一次殿试,前百都绝对没有他的份:以仁宗的恢廓,也受不了一句‘天监不远,民心可知’。”
