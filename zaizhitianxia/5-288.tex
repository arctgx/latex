\section{第30章 随阳雁飞各西东(19)}

被大公鼎严词训斥,大昌嗣也不敢再多说。

只是知子莫如父,大公鼎只看儿子低下头时的神色,就知道他根本不服气。而旁边的长子,也是一般不以为然。

两个儿子根本就没看得起种谔,让大公鼎心头堵得慌,知道他们大概是受了都管耶律余里和左详稳奚乌也的影响。

迁移至兴灵的各部并不惧怕战争。统领军政的耶律余里更是一贯好战,整日宣扬内平党项,外惩南朝。使得有许多年轻人都跃跃欲试。大公鼎的三个儿子,成年的昌龄和昌嗣都与其他几族中的年轻人一样,成日里叫着要去打下韦州,不过都被大公鼎给强压下来了。

大公鼎绝不会小瞧种谔!

一名南朝将军的名声能传到大辽国中,就绝不是会那么简单的一个人。从种谔过往的经历来看,甚至可以说是南朝数一数二的名将。所以他才能坐镇在银夏路上。

以溥乐城为核心,来围点打援是既定的方略。

从属于环庆路的韦州和银夏路的盐州,是最靠近溥乐城的两座军州,环庆军和银夏军就是这一次出兵的第一目标。

溥乐城只是韦州外围的军堡,之前溥乐城又曾残杀大辽将士,兴灵兴兵围攻溥乐城也能说得过去,比起直接攻打韦州要名正言顺一些。只消灭宋人援军同样也是基于这个道理。

耶律余里纵然叫嚣着要惩治宋人,但他还是贯彻了尚父不欲于宋人撕破脸皮的底限。

只是环庆路的韦州到现在也没出兵,盐州方向更是没有丁点消息。

环庆路倒也罢了,领军镇守的是个文官。但银夏路可是种谔,溥乐城城主是他的亲生儿子,不可能不救的。种谔竟然还能耐下性子来,这已经是名将的作为了。

这段时间以来,盐州城方向上的辽军斥候损失极大。从斥候们的回报来看,盐州城的宋军已经将他们骑兵的搜索范围放出了一百里。

这基本上是大辽军中远探拦子马才能达到的距离,是进攻时为了防备敌军攻击侧翼,同时也是搜索一切可以劫掠的对象。

不管种谔有什么理由,有一点是十分确定的,单纯的防守,绝对不需要那么大的索敌范围!

大公鼎又瞪了儿子们两眼。

叫嚣着攻打韦州,也不想想光是打一座溥乐城就要投进去多少条人命?现在死伤惨重的是党项人,换作是官军呢?同样会是损失惨重——大辽精骑从来都没有说善于攻城过!

只是在兴灵周边,不去攻打城池就得不到任何好处。

这里跟河北不一样。在南京道,一旦过了界河,就是富庶的河北地界。大公鼎曾听先人说过,那里的一个乡镇都比国中的一座军州要富裕。绕过一座座重兵防守的城寨,去劫掠乡里镇上,照样是丰收。

而西平六州这里,面对的是宋人新得的土地,几百里内都只有一座座坚固的城寨。翻过南面的横山,还是绵延几百里的寨堡。再过去,才是人烟稠密的关中腹地。想要打到长安城下,就要打破这总计一千多里的山峦城寨的屏障。

有多少人马都不够往里面填的。家里的孩儿们纵然勇猛,赶起女真来跟撵兔子一样,但也不能这么浪费他们的性命。

要做什么?又能做什么?大公鼎心里有分寸。

……………………

城头上,种师中拿着根棒了几圈绳索的长竹筒,左看右看。

虽然这只竹筒方才被他拿着,让一名党项人脸上开花,可惜在他的眼中的,基本上还是一件玩具。

“这东西也就是守城时有些用。”种师中很是遗憾的将竹筒丢下,乓的一声响。空洞洞的。

这样的一根装满火药和铁砂、石子的竹筒只能用上一次,论威力还比不上一根由神臂弓发射出来的六寸长的木羽短矢,或是一瓢烧热的滚油。只是占了新奇而已。

“挺好玩的。”他对种朴和冯真说道,“玩过就算了。”

“谁说的?献上新兵器可不是小功劳。没看到神臂弓的好处吗?”种朴却并不认同种师中的看法,“这支飞火枪的确只是寻常,但飞火箭可是能射下飞船的。有实战的成绩,”

种朴同样拿着根竹筒在手中摆弄着,竹筒上也帮了几圈绳子,不过跟种师中手中的竹筒还是有些区别。这一支竹筒中装的是火药飞箭,只是之前已经射出去了,同样是空的。

“辽人也有飞船。守城时头顶上多了双眼睛,有多碍事,廿三你这几天也看到了。”他双手一前一后扶定竹筒,将尾端搭在肩头,瞄准了头顶的夜空中属于宋军的飞船,“现在官军有了飞火箭,以后可就方便多了。”

冯真自从飞火箭射下了辽军的飞船后,就一直保持着好心情——因为这份功劳,种朴和种师中都不会跟他抢。

“辽军还有一艘飞船,如果也能射下来了的话,就是再确凿不过的战绩了。”冯真带着很深的遗憾,

“烟花就那么多,毒烟火球剩下的也都拆了,那点火药用用就光了。”种师中还是不看好这些火器,消耗太大,火药运送可比箭矢要危险得多,“而且竹筒容易裂开,用绳索并不方便。”

“用铁箍箍上两圈好了。”冯真十分果决的说道。

“是做桶吗?”种师中笑了起来,“那么是不是找两个箍桶匠来比较好?”

“如果真的有用的话,两个恐怕还不够。”种朴想了想,“我记得为枢密院和武英殿造沙盘的泥人匠可是有二十多个。”

正说着话,一片蹄声暴然而起,由远及近,眨眨眼的功夫,就来到了城下,随即十几支长箭从下方的黑暗中飞了上来。

夜色中,根本看不清有多少骑兵在城下奔驰,他们倏忽而来,疏忽而去,冲着城头一番驰射,又立刻远飙而去。

种朴的亲卫早已举起盾牌守住了种朴、种师中和冯真,而士兵们也都避了开去。

“烦死了!”

种师中从手边抄起一张弓,随手又抽出一支箭,拉开了便射了出去。方才只有他的头盔上挨了一下,那一下没有造成任何伤害的冲击,却引燃了种师中的怒火。。

一声拉得很长的尖啸从城上窜入夜色之中,种朴这才发现自己随手抽出的竟是一支带着骨哨的响箭。

这一支响箭也不知给射倒了哪里去了,反正人是肯定没射到,蹄声依然稳定。但鸣镝的尖啸声,在夜色中远远的传了开去,倒也让城外的声音离得稍远了一点。

“怎么都不摔下来呢?”

种师中恨得磨牙。辽人骑兵每天夜里都会来绕上一趟两趟,往城头上射上几箭。虽然没有让他们得到什么战果,可也让人恶心透了。

城上的射击由于城头的火光的缘故,完全没有准头,零零散散靠运气射下几个,还都被救回去了,也不知死活。

而宋军的骑兵也不好出城追击,他们不敢在深夜中飞马奔驰的,绊上一下小命就送了。可辽军的骑兵仿佛有恃无恐,尽情狂飙,几天下来却也不见有人摔下马来。骑术相差太远,想追都追不上。种朴也试图伏击过,可惜同样没能成功——辽人在吃过亏后,就没再上当过。

种师中气哼哼的丢下弓,问种朴道:“十七哥,援军什么时候来?”

“这可要问廿三你吧?你不就是援军吗?”种朴摇摇头,然后道,“赵经略估计要等到辽人放弃他们伏击援军的想法。至于盐州城那边……”他迟疑了一下,最后一叹,“我真不知道爹他是怎么想的。不过……”

“不过什么?”种师中立刻追问。

“不过……”种朴很是无奈,“不过眼前的机会,我爹他绝不会放过。”

……………………

大公鼎望着溥乐城头,如今围在城外的大军,根本就拿这座城池毫无办法。除了骚扰,还是骚扰。

很是无奈的摇了摇头,他就准备结束今夜的巡视,返回自己的营帐。

一名骑兵这时候从中军的方向奔来,远远的看见大公鼎一行,便翻身下马,小跑着过来,大呼小叫的带着喘:“原来节度是在这里,倒让小人好找。”

他在大公鼎面前单膝跪倒,行了个礼:“节度,都管有事相商,命小人来请节度。”

“我也正要找都管说一说事。”大公鼎点点头,立刻便要上马往中军大帐过去。他并不知道是什么事让那个耶律余里请自己过去,但都快三更天了,应当不是小事。

就在此时,军营中突然起了一阵骚动,然后声音猛然间扩大,多少士卒都从营帐中钻了出来。

营啸……?!

传说中的炸营难道就要在眼前出现,大公鼎心中一紧,甚至有些纳闷。这些天在溥乐城下损失的基本上都是党项人啊,军中也镇压了不少临阵脱逃的党项人,本军的兵力伤亡加起来还不到五百,怎么先是自家的军营先闹了起来?

随即他就知道原因了,但他宁可不知道。

早已入夜,可西北方的地平线上却不知何时却有了一片刺目的红光。

大公鼎如坠冰窟,被突然而来的寒意冻得僵硬,双眼试试盯着如血如霞的地平线,没有一点动静。

大昌嗣从喉间挤出一声呻吟:“那……那是耀德城。”

没错,就在那个方向上,正是囤积了全军粮秣的耀德城!
