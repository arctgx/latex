\section{第五章 月满完旧诺(上)}

月色如晦,夜浓如墨。

只有中庭小门处挂着的两盏灯笼,散着微弱的光芒。

夜风刮了起来,院中的两株梅树摇摆,婆娑树影倒映在韩家内厅的窗棱上,如同鬼影憧憧。但疾风穿过门缝,发出的鬼啸一般的声响,却丝毫没有影响到房中三人喝酒的兴致。

为了恭贺韩冈今次纳妾,王舜臣和赵隆今天一齐到了陇西。

他们如今都不在巩州任职,为了来凑个热闹,便各自找了个借口。正好如今熙河局势平静,他们也能抽出几天空来——当然,王舜臣和赵隆来陇西,不仅仅是为了恭喜韩冈的娶妾,主要目的还是为了提前来给韩冈饯行。

两人都清楚,韩冈这一去京城,不论中与不中,数年之内不可能再回熙河,甚至关中。几年前结下的贫贱之交,随着各自的官位增长,并没有变得生疏,而是日渐深厚起来。韩冈这一去,都有些舍不得,自然要来送一送。

“可惜李二哥不在,王衙内又跟着去了京中。还记得当年在三哥老家,五人坐下来一起喝酒,等着陈举的儿子来送死。那样的日子还真是痛快!”王舜臣回忆着旧日,一杯酒就灌了下去,“那一夜,杀的也是一个痛快。”

“王衙内跟着学士走了,肯定有个好去处。在笼竿城下七矛杀七将,李二哥那边,也终于是时来运转了。”

“以李二哥的武艺,只要碰到一个机会,当然能转运!”

在座两名将军,都有资格为李信的运数叹息。

王舜臣和赵隆一直跟着王韶和韩冈,河湟开边的历次战事,基本上都轮上了。官位跟着功劳,飞速的往上涨。只有李信,早早被张守约挑了去,就一直留在秦凤,根本没有多少立功的机会。他得官还是去年年初的事,跟着韩冈一起上京,那时候,王舜臣都已经是一方镇将了。

王舜臣半眯起的眼睛,好像在看着笼竿城下的那一场大战:“李二哥这次实在是太出彩了。拦在路上的五千铁鹞子,被他连杀七将后,竟然在一冲之下便溃散了,援军就这么进了笼竿城……唉,仁多零丁肯定是吐了血。”

“李二哥的掷矛之术,如今已经名满关西。跟王大的连珠箭术一样,已经没人不知道了。”

“子渐你也不差,不要妄自菲薄。”韩冈亲自给王舜臣和赵隆倒上酒,“张铁简之后,当时轮到你赵铜简了。”

统领着熙河路选锋的赵隆,他所惯用的两支熟铜简,如今同样名震军中。虽然在外面的名气还比不上王舜臣的神射和李信的掷矛,但以他的勇武,迟早能杀出头来。

王舜臣又喝了两杯,韩冈今天拿出来的酒,正对了他的胃口。他又问着:“三哥,听说疗养院的朱中也要走了?”

“朱中今次功劳不小,也终于得了官。”韩冈对王舜臣和赵隆道,“你们也知道,王相公身边的章子厚与我素来亲厚。”

“俺知道。”王舜臣立刻道,“当初还帮三哥你送过信呢。”

他说着又看看赵隆,赵隆点头,“也送过,还得了一份礼。”

韩冈微微一笑,当初让他们做信使,本就是要让两人顺便跟章惇结个缘。多认识一人,就多一条路。要不是韩冈如此事事为身边之人着想,王舜臣、赵隆如何会跟他这般亲近,凡事都以他马首是瞻?

“那你们知不知道,朝廷如今已经开始准备解决荆湖两路的山蛮?……这领头就是章子厚。”

王舜臣奇道:“上次不是听说是先要收拾西南夷吗?”

“夔州【今重庆、贵州一带】鬼夷之事,那是另外一桩,今年先动的荆湖。”韩冈细细解释,“论身份,章子厚是察访使,要比当年来秦州时的王学士还要高上几级。他愁着荆湖山中瘴疠太重,所以向求我个人。本来是推荐给他的是雷简,但朝廷前日的诏令已经把雷简调回太医局了,说要在京城禁军之中,设立疗养院。现在就只能让朱中去了。”

“……那陇西疗养院怎么办?”王舜臣惊问,这可是攸关帐下儿郎性命的大事,由不得他不关心。

“如今陇西疗养院不缺人替他。看到朱中能得官,各自又更加用心,现在的情况反而好了许多。”韩冈摆了下手,让王舜臣放心。继续道:“还有我那表哥。他今次上京其实也不仅仅是诣阙面圣,转头就要跟着章子厚去荆湖。还有刘仲武,当年被向宝推荐,与我一起去京城的。他同样救了章子厚之父,今次就被点上了。”

刘仲武这个名字,两人都已经没有印象了。而李信被章惇调去领军,倒是让王、赵二人感到羡慕。

赵隆举起酒杯,“李二哥既然去了荆湖,少不了要立功受赏!当为李二哥干一杯!”

王舜臣也举杯相和:“李二哥的时运当真转了,倒不像熙河这边,都要歇个几年了!”

“机会总是有的!”韩冈与他们干了一杯,“你们不想想,如今国中百万大军,真正能派得上用场的也就是只剩我们西军了。河北、京营两处的禁军多少年不打仗,早就烂透了底。日后四边用兵,都是要从关西调兵遣将。不要光想着关西,要放眼天下。日后做事用心勤谨一点,平日里的功课也不要耽搁。闻鸡起舞的故事你们都该听过,多学着祖士雅【祖逖】、刘越石【刘琨】。”

“三哥放心,我们一定用心!”王舜臣用力点着头。

“放心什么?说得就是给你听的!”韩冈瞪了王舜臣一眼,“子渐【赵隆字】一向用功,兵书都在读着,白虎节堂偏厢里收藏的那十二卷《武经总要》的节选,谁借谁没借,我一清二楚。子渐都借阅了一遍,你借过几卷?!”

自河州大战之后,这几个月,韩冈听说王舜臣时常放下军务、出外游猎,着实让人担心。他提起朱中、李信要去荆湖的事,也是想刺激一下王舜臣——日后立功的机会多得很,想要把握住,就不能耽于眼前的逸乐。

“我知道你上过蒙学,跟着种十七读过几年书。《武经总要》要看,史书也要多翻一翻。闲暇的时候,不要尽想着游猎作乐!”

就算是武臣,读书也要勤。范仲淹当年守陕西,曾经嘱咐过狄青多多读书。狄青日后出入枢府,为一时名将,也有着听从范仲淹而多读书的功劳。并不是说在春秋、汉书,对用兵之道能有什么启发。但多读书的将领,在文臣那里,往往都能留个好印象。日后升迁时,也能因此而加分——读书知礼,能明忠义之道,世人往往都有这样的想法。

王舜臣虽然被训得有些难堪,但他也知道韩冈这是当他是自家人,才如此苦口婆心。从座位上跳起来,重重的向韩冈磕了一个头,大声道::“多谢三哥教诲,俺回去后就用心读书习武,绝不会再荒疏了功课!”

韩冈连忙将王舜臣扶起坐好,责怪道:“听了就好,行这等大礼作甚?!”

“王大,日后就不能再出去玩了!”赵隆凑过去取笑了王舜臣一句,聪明的调转话题:“不知官人有没有看到董毡的便宜儿子带来的那匹西域马,都有五尺挂零了!今天看到的时候,俺的眼睛都挪不开,这辈子没见过这么好的马!”

说起战马,王舜臣的兴致又上来了,“俺也是看到了,真真是好马。董毡对他便宜儿子,也真是大方……就不知道能不能用酒换来,俺手上还有二十斤烧刀子呢!”

蕃人喜欢汉人的酒,只看位处蕃区之中的各大寨堡,有多少监酒税的官吏,就知道这门生意做得有多大了。在熙河路,现在名气最大,就是韩冈所创的烈酒烧刀子。

士大夫中喜欢烈酒的几乎没有,就连高遵裕尝过之后都摇头。但一干在外厮杀的武将,却一个个喜欢得不得了。王舜臣领头当日去酒坊偷酒,还被韩冈训过。但转过头来,几个将领还是缠着韩冈要这烧刀子来喝。烈酒的名气也因此而打了出去,蕃部贵人们尝过一次后,都立刻出重金搜求。

韩冈曾经用来吓唬他们的那番话,各个还都记在心上,也传了出去,但世上拼死吃河豚的都有,肚中的酒虫闹将起来,谁还管什么阴阳不调的问题了。大不了一口酒后,再喝上一口水就是了。

而就在王韶还没离开熙河的时候,用烈酒换马的想法已经被提上了台面。但韩冈觉得烈酒消耗的粮食很多,供给药用压力压力很大了。正常情况下,要先保证路中的粮食能自给自足,才能放开手脚酿造。

但在几家蕃部提出用烈酒交换马匹的提议后,在战马和粮食安全之间,朝廷和经略司不约而同的选择了战马。

再过几日,等监盐茶酒税的官员到陇西来报道,以茶酒换马便将开始运作。

王舜臣和赵隆继续和韩冈聊着,酒喝得多,不到半夜就醉倒了。韩冈今天拿出来款待两人的酒水,虽然已经是将新酿的烧刀子加淡酒勾兑后的产品,只是如果按度数算,韩冈估计着,差不多也有四十度了。加起来喝了快有五斤,醉了也是正常。

命府中下人将两人送到客房安顿下,韩冈往后院走去。

能早早结交上这样的两个猛将为臂助,也是自家的运气。自己虽然是帮了他们一把,但以两人的才能武艺,不管放到哪里,都能脱颖而出。

自住的小院中,大半的房间都黑着。只有偏厢中,有一盏孤灯幽暗,韩冈停了一下,便向那房间走了过去。

