\chapter{第三卷 浪拍岸乱云穿空}

\section{第一百五十四章 初到法国}

当楚云飞他们在机场候机的时候,人头攒动的大厅里,一个不起眼的角落处,站着一女两男三个黄皮肤的人,那个女人不动声色地努努嘴,说的是汉语,“喏,就是那三个人,肯定是中国人,就是最前面那个瘦高的说他们手里有一百多条人命的。”

一个年纪略微大点的,鼻孔有些上翻的中国人说话了,“好了,小严,过去拍下来他们,注意安全。如果被发现,你记得说英语,千万,千万注意,不能冒充是日本人,别以为黑社会的就没有爱国者了。老话说得好,‘仗义每多屠狗辈’啊。”

年轻点的男人点点头,他长了一张很讨女人喜欢的脸,“我知道了,窦处也说了,他们说的是亲手杀的就那么多,意思是手底下还不定有多少人命呢。”说完应声消失在人群中。

另一个角落处,两个白种人也在指指戳戳,“就是他们吧?三个黄种人,一个白种人,应该是一伙的,笑话,就这几个人也想挑战我们工人党?”

两人正言语着,却发现那三个黄种人中一个瘦高的扭头看了过来,目光中,寒气逼人,那是楚云飞隐隐觉得这个角度似乎有点什么不太妥当的事。

当天晚上,一份普通的电报由中国驻英国大使馆发回了中国,是用那种不太重要的密码发的,有心人能很容易破译的,内容也很普通,“有三人乘坐11:30的飞机飞往巴黎,疑似在刚卡叛国的三名中国维和士兵,他们的名字是……”

于是,楚云飞他们的消息再度进入了国家关注范围,不过说实话,真没什么人对他们有太大的兴趣。

这时候,楚云飞他们已经在巴黎找到了住处。

为了保密,一下飞机,多尼就拿楚云飞的手机打了个电话,几个人就打车走了。

在巴黎这么繁华的城市,反跟踪实在是件很简单的事,他们四人在几个大商场和游乐园一转,很轻松地就甩掉了可能跟踪的人。

三个中国士兵跟着多尼钻进了一家酒吧,一个非常小的酒吧,这时是下午四点多,酒吧里只有两三个人,坐在阴暗的角落里不知道聊着什么。

多尼像是很熟悉这里,一进门直奔西南角。

走近才能发现,这里有一道很隐秘的小门,门口一张桌子,一个头发花白的人趴在桌子上,似乎正在睡觉。多尼走过去,手指在桌上敲出一串节奏奇特的叩击。

那人身子懒洋洋地动了动,头都没抬,就听到那门轻微地“嗒”地响了一声,多尼二话不说就去推门。

门里还是那么阴暗,四人走进去,那门在后面缓缓关闭,又发出一声轻微而沉闷的响声,感情这门居然是铁的!

等到大家的目光都比较熟悉了里面的阴暗光线,才发现这屋子其实不算小,有将近二十平米,而且四周环绕的是一扇扇的小门,屋子正中间一圈沙发,还有张圆的茶几或者说矮桌子,一张沙发上坐着一个穿着深色衣服的人,年龄和相貌看不清楚。

多尼走上去和那人嘀咕了一阵,又给了他个小东西,那人指了指一扇小门,多尼就带着三人进去了。

门里是个小小的屋子,跟国内饭店的包厢类似,多尼让三人坐下才开始解释,“这里,是巴黎很有名的黑货交易店,咱们先得给自己买点东西。”

刘宁和成树国都不懂法语,楚云飞倒是学过几天,不过也就会说说“谢谢”“你好”之类的,再加上一些数字而已。

楚云飞刚才确实听到了几个数字,不过他也不确定多尼说了点什么,“你要了点什么东西?武器么?”

多尼点点头,“对,武器,还有汽车、假护照,护照是我的,我把照片给他了。”

几个人要的武器早就计划好了,起码两支自动步枪,四支手枪,手雷和子弹若干,如果可能再弄几颗反器材或者人员杀伤的枪榴弹。刘宁想要一枝狙击步枪,但多尼说一时未必能搞得到,因为法国的枪支管理还是比较严的。

听到这话,三个中国人也眼红起来,刘宁骂了句,“操,不早说,早说我们哥几个也去照相,弄几本假护照。”

多尼笑笑,“哈,我弄了好几本假护照呢,不过,我想他们给我提供的照片应该不合适咱们任何人用。”

成树国还想说点什么,多尼摇摇头,做了个“噤声”的手势,大家马上就知道怕是有人偷听,于是都闭了嘴。

等了有将近一个小时,有人进来了,抬着一个大大的帆布包,里面是武器,刘宁随手拉开包看了看,“靠,FAMAS步枪,制式武器,牛逼。”

所有东西对方开价六万美圆,多尼从他随身的包里拿出钱,几个人扬长而去。

汽车是一辆八零款的厢式标致车,虽然老了点,车况还不错,还有几个仿真度极高的假车牌,多尼做了司机。

多尼的朋友是个红头发的矮小女人,长得不算好看,脸上雀斑很多,不过身材不错,一个人住在巴黎市郊的一幢大房子里,外面还有院子。

女人早早就等着他们了,多尼一按门铃,那女人就出现了,本来是怒气冲天的模样,不过看到还有三个年轻的“外国人”在场,立刻换了副优雅的面容,不过话里似乎还是有一丝火气的。

“¥\#(\#\#……*(……。!¥%”

说的是法语,语速还极快,楚云飞勉强听懂了一个单词——“宝贝”,后来他们才知道那女人说的是:“宝贝,你还活着啊,我以为你的丽迪雅已经改嫁了呢。”

多尼讪笑一下,用英语给对方介绍了楚云飞他们三人,大致意思是这三人是他的好朋友,弄丢了护照,要在这里住几天。

那女人还信以为真起来,很热情地同三人打了招呼,居然还问起了事情的详细经过。不过,她的英语实在是够糟糕的。

随便聊了几句,楚云飞他们就知道了,这女人名叫玛兰娜,是多尼在巴黎一个比较固定的相好。她家里有钱,父母死得早,哥哥在美国,就剩下她一个人在巴黎。

这女人重感情又爱浪漫,情人很多,而且居然都相处得比较融洽,离婚后,她也没再同谁结婚的意思,就这么过着逍遥的日子。

当天晚上,多尼离开了楚云飞的视线,楚云飞觉得有点好笑,回了法国,你倒是不怕被人偷袭了?

\section{第一百五十五章 恶魔希伯伦}

第二天,楚云飞伴着多尼去了法兰西银行一次,不但拿了点预付款,还为多尼重新办理了新的户头。然后多尼就窝进了玛兰娜家不再出来。

楚云飞他们倒还能偶尔出来一下,但也是非常小心的,而且从不三人一起外出,还好在英国买了那么多衣服,常换换并不太怕人会认出他们。

接下来的一个月,差点把四个人憋出毛病来,但楚云飞坚持这么做,原因无他,维伦斯家的情报应该是对称的,那个工人党肯定也知道有人要找他们麻烦去了,所以大家需要多呆呆,等事情冷却些的时候再动手。

当然,这一个月里大家也没闲着,忙着准备装备,适应装备,抽空还要练练气。成树国的钢针得到了补充,大家还开着车出去找了片野地试枪。

楚云飞居然在这个月里学了不少法语单词,这家伙的语言天赋实在是有点惊人。

至于多尼,则是忙着用电话打探“克鲁梭工人党”的消息,为保险起见,他的朋友大多是不方便联系了,不过总还是有那么一两个可以帮忙的人。

终于在十一月的上旬,四个人出动了,不但玛兰娜松了口气,大家也有了“重见天日”的感觉。

破旧的标致车载着四人到达了马赛,天色已晚,多尼直接把车开进了一座院子,从房间的门楣上拿上钥匙,三人走进了房间。

房间里空荡荡的,基本上没什么摆设,看样子是好久没住人了,不过还好,冰箱冰柜都有,微波炉也有,里面摆满了食品。

多尼打了个电话,没多久他的朋友就到了,不过楚云飞他们没有露面。

那人留下了一辆更破旧的标致车,还有一条情报,“工人党”的三号头目希伯伦今天晚上在“第三个晚上”夜总会过夜,那里有他的一个二十岁的俄罗斯情妇。

怎么办?干还是不干?

多尼对自己的朋友非常信任,“我帮过克鲁很多忙,他是不可能骗我的,不过他不是混黑社会的,所以也只能打听到这种程度了。”

大家的目光都转向楚云飞,他却在那里不知道想着什么默不作声。沉静半天,成树国提了问题出来,“多尼,这个希伯伦是怎么样一个人,你清楚么?”

多尼点点头,“这家伙是工人党里名声最大的,做事招摇,心狠手辣,曾经把个偷袭他的小孩活生生肢解,然后又亲手绞成肉沫。算是工人党里第一号的厉害人物了,不过脑子不太够用,要不他就该是老大了。”

刘宁想的是别的问题,他皱皱眉头,“这家伙的身手好么?保镖多不多?”

多尼又点点头,“我想,他比达克厉害,而且枪法也不错,身边倒没几个保镖,不过里面有个俄罗斯保镖很厉害,听说能跟他打个平手,叫什么茨基,托洛……茨基?”

楚云飞终于发话了,“他为什么要那样对待那个孩子,有那么大的仇恨么?”

多尼斜瞟了楚云飞一眼,这个人对朋友是绝对仗义的,不过,他实在不能算是个正直的人啊,为什么会这么问?

“那个孩子一家都被他杀了,小孩是找他报仇的,他自然要杀给别人看,也算是警告别人别打他主意吧?”

这个答案显然很令楚云飞满意,他微微笑了一下,“呵呵,看来这个希伯伦,应该是仇家遍地了吧?”

果然够邪恶,听到这种事,居然笑得出来!

多尼一边猜测楚云飞说这话的用意,一边点头,“是的,他的仇家绝对不少,而且,很有可能……丽迪雅的事也是他做的。”

嗯,这就好办多了,楚云飞又问,“多尼,那你记得不记得有什么有名的人或者势力痛恨他?就是绝对会报仇的那种。”

听到这里,刘宁和成树国就明白了,感情这家伙要把对方的思路引偏,以便减小下一步行动的危险。

多尼撅着嘴,想了半天,才期期艾艾地说,“楚,你知道的,其实……呃,其实我本来对黑社会不是特别熟的,虽说认识不少这样的朋友,但家族有规定,尽量少牵扯这种事,所以我是不可能知道太多的。”

三个中国人对视一眼,不再言语。就这点本事,亏他平时也好意思说如何如何神通广大。

多尼等待半天,看没人说话,开始着急了,“楚,咱们好不容易来了,情况也打听清楚了,克鲁这情报来得不容易,下次这种情报,都不知道会是什么时候能弄到了。”

刘宁皱皱眉头,“你哪里那么多废话?该怎么办是我们考虑的,你只管搜集情报好了。”

楚云飞无疑是在考虑怎么利用今天的情报来做文章,但刘宁实在是清楚,对朋友,云飞并不是表面上表现的那么冷漠。所以,他必须制止多尼的那些废话,以免影响了云飞的思维。

大家毕竟是拎着脑袋在玩,计划自然是越完善越好。

成树国比刘宁还了解楚云飞,知道这家伙考虑起来问题的时候是非常冷静的,小小一点响动根本造不成任何影响,也笑嘻嘻地回敬多尼,“好了,多尼,这么长时间都忍了,还差这么几天?”

听成树国这么说,楚云飞马上想到了个问题,“多尼,照你这么说,希伯伦这个性格,怕是在工人党里也不受欢迎吧?戴维斯也忍他很久了吧?”

多尼点点头,“这个倒是,戴维斯和希伯伦的老爸都是工人党元老,戴维斯的人气比他高多了,所以才不怕他胡搞。不过,戴维斯真的也很头疼他,希伯伦很不赞成把工人党局限于马赛北部,他总想吃下整个马赛。”

“看来希伯伦的脑子确实不怎么够用”,楚云飞一时想不出什么好的主意,也加入了评论,“工人党一旦在马赛坐大,再加上在克鲁梭的地位,想不引起政府注意都很难了。”

刘宁突发奇想,“那照多尼这么说,工人党同南马赛的黑势力关系肯定紧张吧?恨希伯伦的肯定不少,要不……我们?”

“没用的”,楚云飞摇摇头,“现在双方处于平衡的状态,要打破平衡,一定要有诱因,而咱们显然找不到什么好的借口。”

\section{第一百五十六章 策划假绑架}

多尼也点点头,“是的,南马赛的科托和维杜利都不是好糊弄的,前些年工人党灭掉北马赛的布隆迪尼家的时候。手段太狠了,所以这两人现在是小心翼翼地防备着呢,而且大家都说他俩联起手来了呢。”

“他们对工人党的怀恨不是一天两天了,不过,以他们的实力,实在是不敢主动掀起什么风浪的。”

楚云飞根本没理他后面的话,“布隆迪尼?那是怎么回事?这不就是有名有姓的仇家么?”

多尼笑笑,“布隆迪尼,那已经是历史的名称了,最后死的就是那个被绞为肉沫的小男孩。”

成树国瞟一眼刘宁,诡异地笑了一下,“不错,有门。”

刘宁反应慢点,不过一眨眼也明白了,点点头,“对啊,谁说布隆迪尼家的人死完了?咱们几个不就是么?”

对几个中国人来说,多尼实在是缺乏了点想象力,不过他终究还是明白了过来,“原来,原来你们只要个名义,根本就不考虑合理不合理。”

成树国撇撇嘴,“我呸,只要我活得可能性大点,我不用考虑它什么合理不合理,只要符合逻辑就行了。”这句话是楚云飞常说的。

刘宁也很高兴,“好吧,就这么定了,咱们现在动身么?”

楚云飞点点头,“好的,动身,不过,咱们不能杀了他,要活的,带回来。”

多尼吓了一跳,“带回来?带回这里么?”

看到楚云飞点了点头,多尼马上抗议,“不行,不能带回这里,这是我朋友借给我住的地方,我不能给他带来什么麻烦,我看还是直接干掉希伯伦好了。”

刘宁还是那副酷酷的样子,很不耐烦地说,“云飞要这么做,自然有他这么做的道理,多尼你就不用操心了。”

楚云飞皱皱眉头,刮刮鼻子,“好了,你俩不用说了,咱们现在是一根绳上的四只蚂蚱,一拎就是一串,大家也不用吵了,我给你们讲讲我的想法。”

楚云飞最擅长的,就是把事情搞乱,把一切都弄得似是而非的时候,再从里面获得最大的利益,这次也不例外。

“多尼,你先给我写句法文,意思就是‘这是来自布隆迪尼家族的报复’,我会把它写到墙上去。然后再写封信,记住用左手写,要他们在第五天晚上22:00拿五百万法郎赎人,至于地点,你看着写吧,要那种比较偏僻、地势复杂而又介于南北马赛之间的地方。”

多尼是个很聪明的人,但他实在是猜不透楚云飞的想法,目光中满是迷惑,“绑架么?楚,你不是答应我说要干掉他们么?”

成树国和刘宁已经明白几分了,不过他们还在等待楚云飞的进一步说明,这家伙的大脑……真的跟别人的结构不太一样。

楚云飞舔舔嘴唇,“绑架他做什么?我只是想把事情弄得乱点,谁要他们平时结怨太多的?他们见了墙上的字,自然会被引开点思路,见了信,又会觉得两个信息有点冲突。”

“这种情况下,他们需要考虑的因素就会很多的,既要考虑布隆迪尼家的报复,还要想想是不是南马赛人搞的鬼,就算他们不相信这是绑架,他们也必须要花心思在筹钱和赎人上面,他们总不会不做任何准备吧?”

“当然会有人怀疑是不是咱们干的,不过,没准还有人怀疑是戴维斯或者法国政府干的呢。所以整个事情弄得迷雾重重,不但能保证咱们的安全,还能最大限度地分散对方的力量,我们就好采取下一步的行动了。”

这家伙的脑袋果然跟别人的不太一样,听到这番话,三个人交换了一下惊讶的眼光,最后还是多尼问了出来,“那把希伯伦的尸体带走就好了,何必一定要活口?”

楚云飞有点发毛了,你是真傻还是假傻呀?“这个也需要问么?带活人带死人不是一样的带么?你不打算从他嘴里问问你的事到底是谁做的?你别是真的想我们把所有工人党的成员都干掉吧?”

“再说了,那家伙弄回来,没准什么时候还用得着呢,当然,多尼你真想报仇的话,问完话杀了他我也不反对。”

多尼一时语塞,在内心深处,他的本意还真的是想让这三个人把工人党连锅端了,因为他不能向“波兰复兴运动”报仇了,自然难免是会有迁怒的心思。不过现在仔细想想,剿灭工人党实在是不太现实的事情。

楚云飞答应的是帮他报仇和取钱,可没说要帮他泄愤而滥杀无辜。

大家虽然在玛兰娜家休养了一个多月,但为了保密,也为了女主人的安全着想,并没有太多地谈论这些话题。

想明白了这一点,多尼很痛快地承认了自己的愚蠢,他点点头,“对,还是你说得对,楚,那我们还是尽量把他抓回来好了,当然,如果危险太大的话,我建议你们还是杀了他好了。”

后面这句明显地是在讨好楚云飞他们,所以成树国老实不客气地顶了回去,“多尼,我们不是傻瓜,你也不用说这些废话,还是说说怎么去那个夜总会吧。”

成树国的脾气多尼早领教过了,所以他也没在意,“那个地方不太好找,刺激点的夜总会总是在不太好找的地方,还是我开车带你们去吧。”

楚云飞他们实在没想到,“第三个晚上”夜总会居然是个虐恋者聚集的地方,夜总会并不大量提供这种服务,主要是靠消费者之间的互动,夜总会也提供简单的包厢来满足客人们的临时需求。

需要说明的是,能来这里的性趋向异常者,大多还有不错的条件和比较坚实的经济基础,而且不少人也有一定的社会地位。

所以,“第三个晚上”夜总会不是那种单一的虐恋者的服务场所或者说俱乐部,门里门外也没有诸如此类的暗示,使得各行业精英可以肆无忌惮地踏足这里。

当然,像多尼这么老到的人,都知道这个夜总会实际上是就是一个虐恋者聚集的酒吧。可里面也有不少误入的客人,这酒吧的名声多少还是传出去了一些。

于是就有一些记者或者说好奇宝宝来这里挖掘内幕,可工人党是干什么吃的?对付这种人简直是一拿一个准。所以说,“第三个晚上”夜总会的名声是有了,可保障也提供得足足的,同好者还是可以大摇大摆进来的,毕竟,理论上讲,这里还是个普通夜总会不是?虽然价格高了点。

\section{第一百五十七章 擒获希伯伦}

有个合理的幌子是很重要的,虽然大家都知道那只是一个幌子。

这就是“第三个晚上”夜总会之所以红火的地方,“误入”这个词并不是仅仅在绅士进入女厕所时才会用到。

这些话,都是多尼在开车的时候讲的,名叫“楚云飞的脑子”的那种机器又开始“轰隆轰隆”地转动。

“多尼,照你这么说,那里包间很多,很不容易查出希伯伦住宿的地方吧?”

多尼认真地想了想,然后摇摇头,“不一定吧,夜总会下面三层对外,最上面的半层是用来办公的,我想希伯伦应该是在那里休息。”

“哦,这样啊,那就算了。”楚云飞本来又盘算了点计划,打算制造些混乱,趁着混乱的时候弄清楚希伯伦住的地方。

时钟慢慢地指向了十二点,夜总会里的人也都三三两两地开始往外走,欧洲人的作息还是比较规律的,象巴黎什么之类的地方,虽说是被称做“不夜城”,但那其实是说城市的照明和灯光装饰,接近午夜的时候,街上基本就没什么正经人了。

为避免人注意,楚云飞他们的车停在距离夜总会足足有五百米远的一个停车位上,纵是如此,还有那比较落魄的夜女郎兢兢业业地在众多车辆中寻找可能的客源,害得几人连话都不敢大声说。

等到凌晨两点的时候,楚云飞和成树国悄然地从车里溜了出来,两条人影鬼鬼祟祟地向夜总会那里摸去。

夜总会已经关门了,不过两人也从没打算从大门进去,楚云飞一托成树国,成树国悄然无声地凌空上了二楼的平台。

楚云飞助跑两步,就是一个纵身,成树国本来还有心思伸手拉战友一把,不过一看楚云飞向上跳的身形速度,伸出的手又缩回了背后,低声嘟哝了一句,“我操。”

楚云飞也听到了这一声,不过,他已经稳稳地站到了二楼上,耳机中传来刘宁的声音,“我日,声音还是调太大了,你俩注意。”

楚云飞和成树国相对笑了一下,成树国就把准备好的胶带拿了出来,横七竖八地贴在了面前的窗户上,然后微微用力一推。

“喀啦”,一声低沉得几乎难以听到的声音响起,面前的玻璃上就布满了蛛纹,不过由于有胶带的牵制,并没有任何的碎屑掉到地上产生什么响动。

两人没有任何的犹豫,扯动胶带,直接穿窗而入,进去才发现是个小房间,不过,房间里并没有人。

门外就是二楼的走廊,走廊上没人,昏暗的灯光一闪一闪的,成树国才要蹿出去,被楚云飞一把拉住,“扯床单蒙脸,小心他们安装有摄像机。”

成树国恨恨地拿起床单撕扯起来,满腔的不高兴,我操,大家都是人,脑子的差距怎么就这么大呢?这么想着,撕扯的时候难免就用力了点。

楚云飞像是看穿了他的心思,对着他的屁股就是轻轻一脚,“操,小声点,有什么想法回头再说。”

回头还说个屁!说我嫉妒你?成树国不再说话,蒙住了脸,指指门外,“云飞,带路。”

楚云飞才说要往外走,却发现面部一处冰凉,伸手一摸,凑到鼻子上闻闻,“我操,这他妈的什么东西,沾我一脸骚味,等等……我换一块布。什么鸡巴操蛋服务员,人都走了也不知道收拾房间……”

两个无良战友幸灾乐祸地低声笑了起来,急得多尼推着刘宁低声问,“新、新,你笑什么呢?”

玩笑归玩笑,正经事还是要用心去办的,两人顺利地摸上了四楼,三楼门口的房间里,隔着玻璃,明显地能看到有两个保安性质的男人在那里坐着。不过,说不出来算谁幸运,反正那俩现在已经坐着就睡着了。

到得四楼,楚云飞明显地听到了一个房间里有人声,男人沉重的喘息声和女人低微的呻吟声,拜托,大半夜了都,至于那么大的兴趣么?

又仔细地听了听,楚云飞确定那房间四周并没有什么明显的人声,他向成树国示意:那个房间,有两人以上!

给那房间门锁滴了点润滑油,楚云飞掏出万能钥匙,小心翼翼地挑动着门锁,悄然无息中,锁被打开了。

小心地推开房门,房间里的灯光比外面亮多了,两条白花花的人形智能生命正在床上激烈地搏斗着。

这时候,房间的门轴出了点意外,发出了“吱扭”的响声,两只白花花似乎听到了这轻微的响声,动作略微停顿了一下。

不过,一切都太晚了,两个床单蒙面的不速之客快逾闪电地扑了上来,手起掌落,无声无息的攻击,床上的二位当场就晕了过去。

两人把那男人的脑袋翻过来一看,额头上没痣,脖子上没疤,显然不是希伯伦,成树国小声骂了句,“操,不是这个,给他们来针‘嘎都因’?”他觉得那东西做麻醉剂挺管用,专门从索度带了一小瓶出来。

楚云飞摇摇头,“没必要,这种人,不值得。”说完,两只手压住两人颈侧的迷走神经,一分钟后再松手,床上的二人已经停止了呼吸。

在这个房间隔壁的隔壁,两人遇到了正主,不过希伯伦已经睡得相当的酣甜了,听到异声,迷糊的他下意识地去掏枕头底下的手枪,但一切都太晚了,他也被打晕了。

希伯伦身侧酣睡的是那个俄罗斯美少女,白生生细腻的皮肤,长长的睫毛和前凸后翘的惹火身材,在昏暗的光线下显得非常地诱人。

成树国突发奇想,“要不我们把她也带走吧,正好你背一个我背一个。”

楚云飞非常怀疑成树国的真实动机,不过,这个当口,肯定不是较真的时候。

瞟他一眼,楚云飞懒得说话,去卫生间找了支口红出来,在微微泛黄的墙壁上写下了一行法文,“这是来自布隆迪尼家族的报复”。写完字,顺手又把那封信丢下。

鲜艳的红色字体,在昏暗的光线下,显得说不出地诡异和冷酷。

“你要带她走我没意见,不过,你要背那个重的。”低沉的声音中,充满了恶意的戏谑。

\section{第一百五十八章 情况有变化}

这次行动实在是顺利得有点过分了,只是在把两个俘虏从二楼弄到一楼时出了点小麻烦,希伯伦的身子实在是重了点,将近两百斤,成树国往下递的时候楚云飞差点没接住。

等到四个人带着俘虏回到暂住点的时候,还不到凌晨四点,楚云飞他们找到了主人房间的地下室,将二人绑好后,指点多尼,“弄醒他们。”

多尼接了一大盆凉水,“哗”地浇到了希伯伦的头上,工人党三号人物一个激灵就清醒了。

希伯伦半躺在地上,迷糊中还没搞清楚状况,刚要开口大骂,就被多尼一脚踹到了脸上,“闭嘴,畜生!”

希伯伦被这脚踹得清醒了过来,使劲晃晃脑袋,四下看看,用疑问的语气嘟哝了一句什么。

多尼自然是听得懂的,冷冷地用英语回答,“没错,你现在是俘虏了,不想受苦就配合一些吧。”

希伯伦大怒,有人敢这么跟他说话?他努力地想支起身子,嘴里爆发出一串恶毒的字眼。

成树国一步就跨到了他的面前,寒光一闪,手起刀落,就把他的左耳朵削了下来,顺手卸掉他的下巴,把那只血淋淋的耳朵塞进了他的嘴里,“你似乎还没有睡醒?”

希伯伦左耳掉处,鲜血呼呼地向外淌着,随后他也明白了过来,眼前这几个人,手段的凶残绝对跟自己有得一比。

不过他也是作威作福惯了的人,最初的惊讶过后,从鼻中发出一声不屑的哼声,表明这套根本吓不住自己。

楚云飞最烦这种搞清了状态还要胡乱惺惺作态的人,他也拔出刀子顺手在对方身上扎了三个眼,一时间,希伯伦身上血流如注。

“再哼一声,我扎你九刀,不信你可以试试,用英语回答问题。”

说毕,楚云飞合上了希伯伦的下巴,“现在你可以说话了。”

楚云飞那三刀扎得还是比较讲究的,捡了那些肉厚血管少的地方。

希伯伦嘴里含着自己的耳朵,吐也不是,咽也不是,不过他终究算个狠人,还是恶狠狠地把自己的耳朵吐到了地上,“你们,会后悔的,知道我是谁么?”

多尼上前就是一记耳光,“希伯伦,你知道我是谁么?”

自己的名字入耳,希伯伦知道现在发怒也没用了,由于失血过快,他的脑袋有点晕,定下心来仔细看对方四人,看了半天,他终于反应了过来,“你是多尼?那个波兰佬?”

这带着侮辱性的称呼入耳,多尼又走上前想收拾对方,不过看看希伯伦那惨样,多尼还是心软了,“是不是后悔了?你这杂碎!”

希伯伦斜眼瞟他一眼,脸上是说不出的蔑视之色,“哼,我希伯伦从来不后悔,不过,你要找我,好象是找错人了。”

成树国上前又是一刀扎进肉里,“好好说话,你这是什么态度?”

恶人总是要有恶人来磨的,世界上哪里来的那么多不怕死的人?希伯伦被对方这么不分青红皂白的几刀扎下来,再不敢嚣张了。

“你们就是那三个中国人?我和你们没什么仇恨吧?……咦,不对呀,你们怎么还会有三个人?”

“哦”?刘宁来兴趣了,“我们不该有三个人,那该有几个人?”

希伯伦嘴角抽搐两下,脸上露出了一种很奇怪的笑容,有无奈,也有一丝狡黠,“我为什么要告诉你们?”

成树国又被激怒了,这家伙这么记吃不记打么?就想上前再次殴打希伯伦。

楚云飞一把拉住了成树国,“等等,希伯伦先生说他跟咱们没什么仇恨,我想,也许是有了什么误会,我认为有必要听听他的建议。而且,在没弄清楚事实以前,我们还是尽量不要怠慢希伯伦先生。”

“希伯伦先生”这两单词个发音很正常,但成树国和刘宁马上从那细小的声调中接到了暗示:云飞这家伙又要玩花样了!

随后二人也反应了过来,既然对方已经不再大声叫喊,那么,想把话从他嘴里掏出来,最好还是使用一些技巧的好。如果对方确定自己活不了,以他那暴躁的性格,什么也不说就不好了。

想清楚了这些,成树国很有默契地做出一副不服气的样子,怒视着楚云飞,下一刻却被刘宁“好心”地拉开了。

多尼的心瞬间凉了一多半,这几个中国人怎么变卦变得这么快?不过,他终究是个聪明人,几分钟后,终于明白了过来:这几位怕是在玩什么花样吧?

希伯伦并没有注意到这些细节问题,说实话他的神经确实大条得离谱,“想让我说话,总得把我的伤口包扎一下吧?……咦,冬尼娅?她怎么也被你们弄来了?居然还没穿衣服?你们,你们这帮杂碎,我饶不了你们!”

成树国马上反驳了回来,“你要再这么张牙舞爪,信不信我当着你面强奸了她?”这话的真假连俩战友都听不出来,不过,楚云飞总觉得,丫不至于这么下作吧?

显然,希伯伦对冬尼娅不是一般的喜欢,听到这话,他的情绪再次低落了下来,嘴里嘟哝了一句法语,不过大家都没听清楚。

多尼这时候表现出了他的聪明,跑出去拿了一床毯子回来,从上到下把那个昏迷的赤裸美女包了起来,连着凉都不会有了。

刘宁也拿来了绷带什么的,乱七八糟地给希伯伦裹上了伤口。不得不承认,他的包扎水平比成树国差太多了,可是没办法,谁叫只有他刚才没出刀呢?

希伯伦看到对方的“善意”,也不再罗嗦,开始交代他所知道的事情。他确实是杀人不眨眼的主,但这并不代表他不怕死。别说是死,只要他的威风吓唬不住对手,通常他都是会很合作的,没事就吃吃苦头那是傻瓜才愿意干的。

原来,工人党对付脱特斯基家族的时候,希伯伦并没有参与,事实上,这事是工人党二号人物普皮一手策划的。

普皮和老大戴维斯的关系很好,他俩再加上四号人物贝维尔,是工人党领导层的核心小团伙,希伯伦和五号人物多普度都是游离在这个中心之外的。

普皮是属于那种智囊型的人物,虽然他的智力水平经常引起工人党内部个别人的质疑,但不可否认的是,同他在其他方面的能力相比,似乎这是他能力最强的方面了。

处于这个地位,同“波兰复兴运动”合作的事,自然是由普皮来操作的。

\section{第一百五十九章 云飞被绑架?}

“波兰复兴运动”找上工人党的时候,希伯伦就听说了这个消息。但波兰人并没有让工人党动手的意思,而且为首的波兰人言语间对工人党的动手能力非常怀疑,这点让希伯伦非常的恼火。

听到这里,几个绑架者交换了一下眼神,大家都觉得,这点实在不能怪那些波兰人,包括黑手党的维伦斯家族都不看好工人党的动手能力的,毕竟他们不是靠这个吃饭的,而相对那些有实力的组织,法国的黑社会也实在是太不成气候。

希伯伦听说波兰人只需要工人党提供情报,站起身就骂骂咧咧地走人了。不过普皮并没有把这个当回事,他清楚地知道外国的黑势力是怎么看待自己这个团伙。更别说难保波兰人还怕工人党会和脱特斯基家族有什么联系呢。

既然波兰人愿意干脏活,普皮根本不介意面子的问题,他的全部心思都用在了敲诈波兰人身上了,你不是怕我们同你的目标有勾结么?那不狠狠诈你们一下实在对不起这个机会了。于是,这件事后,普皮超乎寻常的谈判技巧在工人党内被广为传诵。

波兰人何尝不知道法国人在狮子大张嘴?不过,他们身后有人买单,并不操心需要花多少钱,他们要操心的是,这种额度的费用,是不是足以阻止工人党伸手帮脱特斯基家的忙。毕竟是那么大的家族,不跟黑势力有某种程度的默契那是不可能的。

于是,后面一切的一切,都是波兰人干的。

说到这里,多尼斜眼瞟下多尼,“波兰小子,保斯鲁是你的朋友吧?你能安全跑掉,是他的功劳吧?不过,他是先向我报告的,不是我点头,你以为现在还活得了么?”

听到这话,多尼一脸的震惊,这表情落在其他人眼里,无疑肯定了希伯伦的说法。

多尼真的没想到,自己能活得了还全靠眼前这个缠满绷带的家伙,保斯鲁确实是工人党的外围成员,这点他是知道。不过,他没想到的是,花了那么多法郎交来的朋友,关键时刻还是出卖了他。

幸亏没有杀了希伯伦,这是多尼的想法,因为从某种角度上讲,这位可以算是他的救命恩人。

楚云飞根本没理会那碴,对于他而言,他信奉的一句话是,“朋友的朋友,未必是我的朋友。”再说,他也不认为希伯伦会毫无理由地帮多尼的忙。

所以,楚云飞很欣喜地发现,双方有了共同的话题和相互信任的基础,那么,想问的问题总算可以提出了,“希伯伦先生,没想到您是这么有同情心的一个绅士,我为刚才的失礼道歉,另外,我很想知道,你刚才说的那个……为什么我们不该是三个人呢?”

楚云飞的马屁拍得恰倒好处,希伯伦大嘴一咧,“哼,那帮垃圾,还说已经抓到一个中国人了呢,亏他们还是拿照片对过的。他们早放出风去了,要你们去拿钱换人,咦,你们不知道么?”

成树国在一旁听得入了神,不由得来了一句,“怪不得今天这么容易得手,感情工人党……他们的力量全埋伏到那里了吧?”

希伯伦对他的印象最差,狠狠瞪了他一眼,“废话,你们的情报实在是太糟糕了,找我纯粹就是找错人了,你以为我的防备从来都是这么松懈么?呃……你们是不是该把我身上的绳子解开了?”

解开他的绳子?在场的谁也没有疯掉,楚云飞很堂皇地回答了这个疑问,“这个,希伯伦先生,虽然我们基本可以肯定确实是找错人了,但是,事情没彻底弄明白前,我想,还是不能解开你的绳子。再说了,谁也不能保证你获得自由以后不会把我们的消息说出去,毕竟工人党里有你的不少兄弟,你说不是么?”

“当然,我可以肯定一点,在未来的几天里,你会得到该有的待遇的,中国人,并不是野蛮人,只要你肯合作。”

希伯伦恨恨地盯着地上自己的耳朵,“我现在只想知道,我这只耳朵是不是还能缝回来?”

一时间,大家都不知道该怎么安抚这个心灵受伤的家伙,最终还是成树国继续做恶人,他冷冷地哼了一声,“哼,如果你肯配合的话,我想,你身上是不会有别的器官被割下来的。你最好别以为我们全是绅士。”

恶人还是要有恶人来磨的,成树国这番话说出口,希伯伦居然不再言语了,也许他真的明白了,不可理喻这种情绪,并不是他自己的专利。

楚云飞很隐秘地瞟了成树国一眼,那意思很明显,适可而止些,不要激起对方太多的反感!又掉头问希伯伦,“对了,你们抓的那个中国人藏在什么地方?埋伏的人多么?”

刘宁有点忍不住了,“你还问这个做什么?难道你还想去救人么?云飞,不是我说你,咱谁也不是菩萨,没有解救众生的能力。现在咱还自顾不暇呢,哪里有那么多正义感?”

希伯伦虽然听不懂刘宁的中国话,不过把前后说的话一对照,再看看刘宁说话的语气,是人就能猜个八九不离十,“普皮和贝维尔都在那里,还有不少的波兰人也在那里,就等剩下的两个中国人去救人呢。”

楚云飞使劲摇摇头,努力把救人的欲望从脑子中甩开,那个不知道名字的哥们,对不住了,兄弟我这里也是一团糨糊呢,虽然你被误抓是我们的原因,但是,谁叫你的模样长得那么凑巧呢?多多包涵吧。

“好了,多尼,给他也拿床毛毯,让他们休息吧,咱们也该休息了。”楚云飞站起身来。

收尾的威胁话,自然是由成树国来完成,“你可以尝试偷跑,其实我们的防备不是很严。”

这话要是由多尼在私下里告诉希伯伦,希伯伦还会相信那么几分,不管是出于什么目的,自己毕竟是曾经放过他一马的。

不过,由成树国这个典型的恶人嘴里说出来,那怎么听都是有太多恶意在里面的。

刘宁也站起身来,冷不丁地问了一句,“那个中国人长得象我么?”

希伯伦下意识地摇头,“不……”说话间,眼睛瞟了楚云飞一眼。

\section{第一百六十章 困惑的普皮}

回到客厅,成树国忍不住哈哈地笑了起来,“云飞,想不到人家已经把你抓住了,哈,要不要我和刘宁去救你?”

楚云飞没理他,他一直在思考着下步的行动计划,“别玩了,合计合计下步怎么走吧。”

“没想到希伯伦这个家伙真是无辜的,而且和另几个头目关系还不是很好。”楚云飞边想边说,“早知道是这样的话,那封信和墙上的字留下一个就够了,现在反而把事情弄复杂了。”

刘宁哼一声,很不以为然,“没人告诉咱们他是无辜的,像他名声这么臭的,杀也就杀了,有什么了不起的?”

多尼从地下室上来,正听到这句话,一时间有点着急,“刘,你不能这么说,他毕竟是救了我一命的。”

成树国冷笑一声,“哼,好了,多尼,你真以为他是想救你么?他根本就不认识你,他这么做,不过是想从你身上或者从这件事上多得到些什么东西就是了。”

这话是着实在理的,希伯伦离魔鬼的距离绝对比天使要近得多,他救多尼就算没任何功利性,也肯定是有目的的,起码他能破坏工人党核心集团的小算盘吧。

多尼不知道什么时候变成了死脑筋,他一边点头一边发表意见,“成,你的话是不错,不过,他毕竟是救了我,这点,不可否认。”

切!在场其他三人不约而同地发出了这个音。

“多尼,如果你还想继续报仇,就不要提这事了。”

楚云飞的话,一锤定音,算是这件事情的最终解决方案。

多尼还想反驳些什么,不过要张嘴的时候才反应过来,他的立场实在是有问题,还是不用辩解了。

楚云飞又提起刚才的话题,“工人党肯定是认为咱们早就潜伏在这里了,所以抓了一个人以后,觉得稍微宣传下就肯定可以吸引咱们的注意。”

“这么来说,他们的防范重心肯定是关人的地方,而且希伯伦又跟这事没什么关系,咱们才能顺利得手。”

“事情现在肯定是算复杂化了,你们想想,工人党见了墙上的字和信会怎么认为?”

刘宁想了想,“说不定他们会以为咱们在玩调虎离山的计策,故意引开他们的注意。”

成树国也琢磨了半天,“有字不需要有信,有信那字绝对是多余,两者都有,那摆明了两者都是假的。工人党应该不会以为是咱们做的,要是咱们做的,那留一样就够了。”

楚云飞点点头,“我也是这么想的,没准他们会以为是其他敌对势力搞的鬼,因为咱们要引开他们,只能把事情表面做得更合理才对,像这么似是而非反而不合逻辑。”

多尼也点点头,“我同意你们的分析,但愿他们的智慧……能想到这一点吧。”

废话,这又不是在索度,工人党那么大的帮派,还能没人分析出来这么点事?

事实上,工人党二号人物普皮就是这么认为的,希伯伦被绑架这事显得过于蹊跷了点,打死他也不会相信这是波兰人和中国人干的。

这个二号人物正呆呆地坐在那里考虑,这事到底会是谁干的呢?

嫌疑最大的,肯定是南马赛的那俩竞争对手,普皮知道,科托和维杜利都不是什么吃斋念佛的主,“布隆迪尼家族”虽然是他们曾经的竞争对手,但这个家族的悲惨下场给他俩敲足了警钟。

如果时光可以倒流,相信科托和维杜利不会在工人党恶斗“布隆迪尼家族”的时候,再给工人党提供什么支持了。打死了只狼,引来的却是一群鳄鱼。

可是,理由呢?理由在哪里?他们这么做,究竟想达到什么目的呢?

想到这里,普皮又多了层顾虑,他倒不怕科托和维杜利敢真的翻脸,因为那无疑是为工人党吞并他们提供了借口,普皮担心的是:这会不会是希伯伦自己玩的什么花招?

普皮这么担心是有他的理由的:希伯伦和工人党几个首脑关系并不算融洽,当然,这不代表他就会有什么怨怼的心思。

可现场的情况实在是太让人生疑了,戴维斯的侄子,帮内的骨干亨利被赤裸裸地杀死在床上,他的生殖器甚至还没来得及从另一个死者体内拔出。

而且,失踪的不只是希伯伦,那个美貌动人的俄罗斯女孩也同时不见了,试想,要是真的有人绑架,那女人会有什么价值值得也被弄走?

亨利被杀,而一个骚女人却是被绑架,完全颠覆了价值逻辑,这实在不像是外人干的。当然,绑架者怎么也想不到一时的随意之举会给对方带来这么大的困惑。

所以,失踪是希伯伦自编自导的可能性也很大,当然,以他的简单头脑肯定想不到这么做的,但是,难道不能是有人挑唆么?

普皮想来想去,却想不清楚内中的玄机出来,长叹一声,算了,还是各方面加强戒备吧。

贝维尔是该从埋伏的地方撤出来了,那个中国人,就让那些波兰人去看着吧。

普皮最后决定,贝维尔负责去盯着南马赛的那俩家伙,自己却得为希伯伦张罗赎金,不管这事内幕如何,大家都知道希伯伦被绑架了,于情于理都要把该尽的心思尽到的。

这事做得仁义点,哪怕真是希伯伦弄的鬼,最后被动的也是他自己。

至于怎么向老大交代,这更是普皮的工作重点了,撒出人马去打探消息吧。

虽然混黑道的都要有横死的心理准备,戴维斯也不能就此说普皮什么,但是普皮要是把这事调查不清楚,后果会是非常严重的。

在普皮焦头烂额的时候,楚云飞正在指点多尼,“我就不明白了,多大点事,你不会说自己是旅游公司的职员,给游客订房间么?你的嘴那么厉害。”

于是,一小时后,多尼用一个黑人的假护照在马赛市中心的“云丝顿宾馆”登记了一个房间。

房间面对的是繁华的红海湾大街,宾馆背后是街巷纵横的商业区。

这是个暗杀的好地方,计划中,多尼和刘宁将在这里伏击工人党的车队,前五号人物里,任何一个都是极好的目标。

楚云飞和成树国始终呆在车里,准备接应得手的同伴。

万事具备,只等工人党的重要人物经过这里了。

\section{第一百六十一章 轰动马赛}

选这个地方,是很费了大家一番心思的。

这里离工人党的马赛总部很近,只隔着“自由路”和“东方大街”这纵横两条街,属于工人党的心腹之地。

同时这里又毗邻马赛北部最大的商业街区,那里也是工人党的地盘,赶不上靠近港口的南马赛商业区,但也是相差仿佛的。

南北商业区内,都是小偷横行,蟊贼众多的,黑帮在这里的势力很强,那些玩单帮的小混混更比比皆是。

但同时,由于两大商业区涉及到了马赛的经济命脉,又没有黑帮敢在这里肆无忌惮地横行,你再横,横得过政府么?

这“云丝顿宾馆”虽然同其他宾馆一样要给工人党上供,但多尼在没离开马赛以前就知道,这宾馆其实是有深厚背景的。

这里,似乎是在野党的领袖伦布尔妻子的家族在经营,每年要为在野党提供相当的经费援助,多尼甚至怀疑这个宾馆的赢利都捐给了在野党。

不过,虽然“云丝顿宾馆”有这样的背景,但在江湖混,自然要守江湖规矩,幕后老板也实在不便随便出头,索性交点费用,就算请了保安公司了,倒也省去不少闲事。

工人党是不可能不知道这个宾馆的后台的,不过,既然有人愿意花钱消灾,他们也乐得伪做不知,很多事情,说穿了就没意思了。

所以,选这个地方来袭击,实在是一举数得的事情,而且,工人党应该是怎么也想不到的吧?

工人党确实没想到,在刘宁住进去的第二天,他们的第四号人物贝维尔就带着浩浩荡荡的车队出现了。

看着十辆车组成的车队,带头车和其他两辆车上还有工人党锤头齿轮的标记,刘宁有点接受不了,扭头问道,“多尼,这就是你的说不太嚣张么?”

多尼点点头,“没错,像西班牙那些地方,黑帮老大出来,车上的人手里还有拿枪招摇的呢,你快准备你的反器材枪榴弹呀。”

刘宁一边把榴弹往枪上装,一边反问,“你确定,不用杀伤榴弹么?”

问归问,刘宁已经把枪举起,开始瞄准了,枪榴弹这东西,射击误差总是比较大的。

多尼懒得跟他计较,“第三辆,白色的雪铁龙,那应该,肯定是贝维尔的车……我觉得那是辆防弹车。”

刘宁没有再吱声,三秒种后,扳机扣动,“轰”地一声巨响,那白色的加长雪铁龙化做了一团燃烧的火球。

袭击来得太过突然了,四周的人甚至没有意识到到底发生了什么事,雪铁龙车前面那辆车还好,后面那辆车没来得及踩刹车,司机方向盘一打,无巧不巧地撞上了偏离路线的雪铁龙,跟着爆炸了,又是一声巨响。

“操,你的情报工作实在是太差了,还‘应该’?……看什么看,跑啊!”刘宁边唠叨边收拾起步枪和弹药。

烟雾弥漫中,两人拎着长长的旅行包扬长而去,五分钟后,呼啸的救火车和救护车都到了,工人党的人也包围了这栋宾馆。

这时候,偷袭得手的人早就上了接应的汽车,开车的还是多尼,楚云飞得以有空问刘宁,“刘宁,逮住谁啦?很热闹的样子。”

刘宁轻描淡写地说,“大概是贝维尔吧,那车肯定是他的车,不知道车上坐的是不是他。”

成树国打趣刘宁,“就算坐的是贝维尔,也未必一定会挂吧?你又没亲眼看见他死。”

刘宁撇撇嘴,一副得意洋洋的样子,“扯淡,你不看看是谁出的手,我刘神枪手下有活口么?”

“呃,刘神枪,那是什么武器?我只听说过六神丸,”楚云飞笑笑,神色很轻松,“但愿能杀了贝维尔,咱们就能继续混水摸鱼了。”

说笑归说笑,几个人效率奇高地把这辆破汽车开进了垃圾堆,换上多尼买的车,第一时间溜回了暂住地。

溜回来以后,大家就不敢太轻松了,时不时总要站在窗户旁四处看看,想象中的报复,应该是如水银泻地般地涌来了吧?

楚云飞更是被大家剥夺了聊天打屁的资格,他的侦察能力那么好用,不让他保持警惕实在是浪费资源。

就在天刚黑的时候,多尼那个叫克鲁的朋友又来了,他带来了最新消息:雪铁龙上坐的不是贝维尔!

车上坐的是贝维尔的妻子和女儿,另外还有一个大家都听说过的人:希伯伦的得力助手托洛茨基,这几人当场就死了,只有司机重伤正在救护中,估计断气也就是眨眼间的事。

闹市中整出这么大动静,又死伤了这么多人,整个马赛都乱了。马赛市民们群情激奋,一定要市政府给个合理的解释出来。

这也难怪,虽然马赛的治安一直就不太好,但也不过就是小偷小摸的多些,晚上走路不太安全而已。光天化日之下,居然肆无忌惮地用重火力袭击闹市中的汽车,这事也未免太大点,谁还坐得住?

当然,袭击事件中无辜被波及到的一老一小两个行人,更是赚足了大家的同情。

马赛市民哗然!

马赛媒体哗然!

马赛各帮派哗然!

科托和维杜利马上派出了说客,去工人党做了解释,表明这事同南马赛的势力无关。他们实在是不能不解释,这和示弱无关。

帮派间打打杀杀很正常,这次袭击也是冲着贝维尔去的,这是白痴都明白的事。但毕竟是误杀了贝维尔的妻子和女儿,如果工人党铁了心以血还血,以牙还牙,把各人的妻儿老小算到报复名单上,那可就天下大乱了。

马赛市政府也出面了,工人党在里面有利益代言人,好歹算是把局面控制住了,不过,那几个代言人也受到了巨大的压力,因为工人党这次遇袭的动静太大了点。

工人党也嚣张不起来了,被人在门口扇了一记响亮的耳光还是小事,又重新落入政府的关注下才是大事!

仇是要报的,不过近期要收敛那是一定的,已经有些不怕死的小报记者在挖掘内幕了呢。

楚云飞他们实在是高估了法国人对血腥场面的承受能力了,他们这种战场风格明显地不适用于现代都市中。当然,仇深似海的多尼肯定是不会提醒他们注意的。

不过,歪打居然能有正招,这么一来,麻烦缠身的工人党实在是分身乏术了,楚云飞他们又可以直接继续下一步行动了。

显然,这次袭击,除了没击毙目标外,其他的都很完美。

有效果的行动,就是好行动!

\section{第一百六十二章 高尚不得}

楚云飞早就计划好了,这次偷袭得手的话,几人就奔赴工人党总部克鲁梭,目标是工人党最大的BOSS——戴维斯家。

马赛肯定会乱得像一锅粥,在这种风大浪急的情况下,再贸然在马赛执行什么行动的话,危险性是有点高了。

所以说,克鲁梭是个不错的目标,这个貌似危险的地方,冒些险没准能得到极大的回报。

当然,谁也不能保证戴维斯什么时候在家,但是,就算杀不了戴维斯,把他家人干掉的话,同样也能让对方进退失措。

楚云飞已经发现了,他其实没有使自己高尚的能力。

这几个屠夫看惯了生生死死,贝维尔家人的不幸,不会让他们有一点点的内疚,在保障自身生存的这个目标面前,任何的指责都是苍白可笑的。

于是,第二天黄昏,四个人离开了暂住的地方,开着车出发了,车里还有两位被绑着的“客人”。

多尼一再地要求众人手下留情,不过,他也知道,自己的意见是不可能被采纳的,车上那两位至今还活着,仅仅是三个杀人犯要找个合适的地方抛尸。

希伯伦还有没有可以利用的剩余价值?肯定还有,而且说不准还有大用。

但是,他和那个俄罗斯女孩的存在实在个很危险的变数,老套电影里,经常有弱智配角在开枪前喋喋不休反而被制的桥段,楚云飞绝对不会做那种人。

于是,路过一座大桥时,天色已黑,瞅瞅四周没人,两个五花大绑的客人被撵下了车。

不需要再做任何解释了,两人都已经知道此刻就是他们生命的终点,那女孩甚至在车上遗留下了些许尿液,牛仔裤的档部也散发出一股恶臭,实在有损她的美丽形象。

希伯伦还想说点什么,可惜他的嘴被胶带缠了若干圈,与身上的绷带相映成辉。他只能用哀怨的眼光望向多尼。

多尼一副“不关我事”的样子,刘宁可没那么好心了,飞起一脚就把希伯伦踹下了大桥,六十多米高的桥,不信摔不死人!

楚云飞担心的是那俄罗斯女孩,他可不想让校友再节外生出什么枝来,成树国这小子目前有向色狼转变的嫌疑,于是几乎在同时,他把那女孩也踢了下去。

和女孩离得很近,楚云飞在出脚的一瞬间,看到了女孩眼中的无助和绝望。

随着那美妙身姿在视线中远去,楚云飞摇摇头,暗骂自己:傻逼,什么时候了,你居然还有闲情逸致来怜香惜玉?

车行不到两个小时就到了克鲁梭,几人连车都没下,就在车里硬生生地熬到了半夜。

三个中国人还好说,多尼真有点受不了,作为个曾经的花花公子,他倒不是不能熬夜,实在是,内急都不好下车去方便。

终于熬到了十二点,几个人钻出车,伸伸懒腰,走向一辆早就看好的老款雷诺车旁。

楚云飞干这个不算内行,不过该怎么做他还是知道的,既然不介意损坏汽车,没几下就钩开了车门,坐进车内碰着了火。

依旧是多尼开车,没办法,谁让他看得懂地图上的法语呢?

戴维斯住的是一座小庄园,院子倒不算大,也就一亩多地的样子,很有些“韬光养晦”的意思。

院子和房屋看起来规模不大,楚云飞带路,后面跟着成树国,一路潜行了过去。

这次就不是床单蒙脸了,好歹弄了几块手帕,楚云飞和成树国躲在暗处仔细观察这个院落的结构。

门和院墙都是外涂铜色漆的两米多高的铁栅栏,门口是守卫住的小屋子,院子中间有个小喷泉,然后就是三层的主楼,主楼后两侧还有两间不算小的屋子,里面住的应该是打手或者说佣人。

楚云飞用手背轻触了一下栅栏,还好,没电,不再犹豫,手一搭栅栏就翻了进去。成树国紧跟着翻了过去。

楚云飞正在犹豫是不是要拿小锯子锯开几根铁栅栏,在必要时也能通过缺口快速撤离,猛然间觉得有点不太对劲。

他猛地回头,四条黑影快速地向自己和成树国扑来,还发出低沉的“呜呜”声。

那是四条半人高的大狗,脑袋有小牛犊那么大,不但凶猛,而且绝对地训练有素。怕是比跟楚云飞打闹过的“老油”厉害多了。

顾不得多想,也顾不得考虑暴露目标了,楚云飞冲上去就是一拳,同时右脚飞起。

全力一击,绝不留手!

狗身上最硬的就数脑袋,不过跟别的东西比就未必算硬了,那两只狗的脑袋登时被楚云飞的一拳一脚打个稀烂。

成树国手里的钢钉也飞了出去,势大力沉,钉上了另两只狗的脑门,可惜的是,钢钉实在是太小了点。

那两只狗受了伤,却越发地凶猛起来,齐齐直扑成树国。

说时迟,那时快,楚云飞一个转身,又是狠狠的一脚,踢爆了一只狗的脑袋,另一只已经扑上了成树国。

成树国的身手绝对不是白给的,不过,他有点担心惊动别人,身形就稍微慢了点,堪堪躲开这一扑,也是狠狠的一脚飞出。

他可没有踢烂狗脑袋的水平,那狗本已经受了伤,又被重重一击,见势不妙夹着尾巴就想跑,它才一扭头,楚云飞的拳头已经到了。

这一切说起来挺长,其实总共不超过十秒钟左右,就这样,轻微的响动已经引起了门卫的注意。

一个瘦老头睡眼惺忪地走了出来,打开了门外的小灯,披着一件夹克走了出来,嘴里还在嘟囔着什么。

成树国愣了一下,虽然他是绝对的铁血分子,但是向这么一个貌似无害的老人下手,猛然间他还是狠不下那个心肠。

但楚云飞根本懒得去考虑这些无关的事,身形暴起,在昏暗的灯光下如一道黑色的闪电掠过,还没等那老头反应过来,掌缘已经重重地击中老头颈侧。

随手一捂嘴,一扶身子,楚云飞把已经断气的老头慢慢地放到了地下,向成树国指指门卫的房间,那意思很明显:里面还有人。

顺手关掉那盏灯,还是老规矩:成树国守外面,楚云飞进门房里面搜查和杀人。

我没有使自己高尚的权力,楚云飞再次撇了撇嘴。

\section{第一百六十三章 偷袭变强攻}

楚云飞摸进屋去,仔细听听,似乎有三个人的轻微呼吸声,想来都是睡着了。寒光一闪,匕首拔出。

摸到熟睡的人身边,楚云飞一刀一个,干掉两人。等到摸到第三个人身边,他略微犹豫了一下,收起了匕首,伸出双手,“喀啦啦”几声轻响,硬生将对方颈骨拧断。

这倒不是楚云飞残忍变态,实在是他突发奇想,用多种方法杀人的话,对方起码不好判断己方有几人吧?

既然能诱偏对方的思路,又为什么不欺骗它呢?

成树国也没在外面闲着,等到楚云飞出屋的时候,他正在拿着搜出来的钥匙尝试着开门呢。

楚云飞轻轻一笑,丝毫没有因为刚杀了几个人而影响情绪,“呵呵,你行啊,居然找到了钥匙。”

成树国也压低声音,“操,这个大锁真不太好开,费了我半天劲。”

刘宁的声音从耳机中传来,“门打开了?要我们把车停到门口么?”

楚云飞四下看看,“可以,不过记得别开大灯。”

由于拧断了一个人的脖子,楚云飞又想起似乎有点什么纰漏,拉着成树国去把那两枚钢针挖了出来,那可是成树国或者说中国特种部队的招牌,不能留在现场的。

处理完了这一切,两人蹑手蹑脚地向小楼走去,楚云飞掏出了准备好的万能钥匙。

大门是坚硬漂亮的红松木,开关的时候一点声音都没有。

两人才跨进门去,异变突起,警铃大作!

成树国大骇,“我操,怎么回事。”

楚云飞皱皱眉头,这破地方感情防守这么严密,还有红外探测器?

“你向大门退,我进去看看。”

说话间,四周已经是人影晃动,人声鼎沸了,楼里的灯也亮了。

眼角看到成树国已经在向外退了,楚云飞不再犹豫,直接冲上了二楼。

按理说戴维斯家的重要成员就在二楼了!

楚云飞随脚踢开一扇门,不管里面有什么人,直接就是一枚手雷扔了进去。

楚云飞目光一扫,发现二楼有七八间房间,看来手雷要节省点用了。

正在踢另一扇房门,眼角处人影一晃,楚云飞随手就是一枪,那边人影倒地的功夫,楚云飞已经看见房间里空荡荡的,不是卧室!

院子里也已经乱成了一团,成树国手雷扔得又准又远,身上的手雷要比楚云飞的多得多,他一手持枪,一手不停地扔出手雷,“轰轰”声此起彼伏。

刘宁在院子外,用自动步枪远距离压制着火力,打得对方不敢冒头。

多尼也拿了步枪在那里胡乱地射击,算是……威慑力量吧。

说话间的功夫,楚云飞已经扫荡了二楼的七个房间,用手枪杀了四个人,六枚手雷也用掉了五枚。

上三楼还是下一楼,楚云飞火速一盘算,算了,估计一楼都不好下去了,上三楼吧。

楚云飞随手捡起一具尸体旁的突击步枪,风一般地刮上了三楼。

危险!楚云飞直觉地感到有问题,身子前扑,就地就是一个前翻,身后立刻被密密麻麻的子弹覆盖了。

子弹虽然密集,不过也就是两枝枪射出的,楚云飞火速出枪,两个点射,对方就再没了反应。

随便踹开一扇门,里面居然是卧室,楚云飞连发愣的功夫都没有,四下一扫,被窝凌乱,却是看不到人。

没人就算了,楚云飞端起步枪一顿横扫,柜子里,床下都照顾到了,直打得木屑横飞。

“树国向车里退,刘宁火力压制,我要出去了。”

保持着警惕,楚云飞边说着边向窗户走去,枪托一砸,玻璃破碎。

楚云飞刚要蹿出窗户,脚下被一个软绵绵的物体绊了一下,是人!

他顺手探去,却感觉到一丝凉气冲着他的手臂就来了,躲闪已经来不及了,一把匕首划上了他的手臂。

操,楚云飞暗骂一声,也顾不得手臂受伤,一把夺过对方的匕首,日了,是水果刀?

他扔掉水果刀,一手拎起了那人,原说顺手一枪要结果了他,可人拎在手里分外的轻,怕是只有七八十斤。他脑子一转,拎着人穿窗而出。

他刚从楼上跳下,成树国的手雷就又扔了过来,刘宁的枪榴弹也发射了出来,是那种杀伤枪榴弹。

院子里早已经是灯火通明,在十几个枪手的众目睽睽之下,楚云飞手拎着一个人,以奇快的速度冲出了大门。

众枪手面面相觑,这是什么东西,还算是人么?

一个年轻人在那里跳脚骂,“还看个屁呀,他抓走了三小姐!”

众枪手轰然,硬着头皮向门外冲去,三枝自动步枪以奇猛的火力扫了过来,大家不得不又纷纷卧倒。

马达声响起,多尼对着了火,疯一般地驾驶着雷诺向远处冲去。

早有有心人开了车跟了过来,四个枪手动作娴熟地跳上车,“追”。

又过半分钟,又是三辆车跟了出去。

头一辆车咬得很紧,而楚云飞他们雷诺车的车况实在算不得很好,这也怪楚云飞,选车的时候只顾着考虑不显眼了,却没想还有这汽车追逐战。

只一辆车当然不会在刘宁的眼里,他上了一枚杀伤榴弹,在颠簸的汽车上稳稳地托起枪身,开始瞄准。

成树国在用手枪还击着追逐车辆的射击,反正大家是谁也打不着谁,无非是打打车玻璃,相互恐吓而已。

楚云飞正在逼问着带回来的少女,那女孩年约十五、六岁,长得眉清目秀,眼睛出奇地大,睫毛也分外地长。

有点……有点像周琳琳,当然,琳琳什么时候也不会让楚云飞看到披头散发的形象的。

那少女异常地执着,不算太小的嘴巴紧紧地闭着,一副倔强的神情。

这么难调教!楚云飞懒得理她,刚要看看手臂的伤势,却发现刘宁在瞄准那后面的车,“别,刘宁,等到个窄路口再发射。”

刘宁马上就明白了过来,到个合适的地方再打,一来对方不好左右躲闪,二来燃烧的车辆能阻止后面跟来的车。

别人没意识到什么,但开车的多尼脸上浮起了笑意,楚是用中文说的,这么说,这个小姑娘不用想活了,为了灭口,楚绝对不会放过她的。

工人党的垃圾,你们全都去死吧!

\section{第一百六十四章 云飞的检讨}

郁闷了,喝醉酒,狠狠地摔了一跤,两手全破,半个身子血淋糊碴的,还好有存稿,没时间加精的话,大家包涵~~~~~~看在随缘这么执着地更新,大家多扔几票吧~~~======================================================楚云飞抬手看看手臂,还好,只划破了衣袖,也不知道是自己运气在身还是对方惊骇之下力度太小。

再看看那女孩,表情倔强依旧,楚云飞暗暗觉得好笑,就像极度的自尊不能掩饰深藏的自卑一样,平静到僵化的外表反而更说明了她内心的恐惧。

不过,说这些都没什么意思了,楚云飞带人出来的时候早就想好了这人的用法,能在戴维斯家有个专用的卧室,这丫头的身份不简单吧?

三分钟后,疯狂追逐的两辆车终于走到了一个狭窄的路上,而且马路两旁全是停着的汽车,中间空着的路不过也就是能容三辆小车并排开过。

刘宁的手轻轻一扣扳机,几乎在同时,“轰”的一声,追逐的车爆炸了,双方的速度都太快了,榴弹相当于一出膛就碰上了后面的车。

燃烧的车辆失去了控制,太快的车速使得这车连翻几个跟头,终于重重地砸在一辆停靠在路边的汽车,又引起了爆炸。

爆炸过后,那辆追逐的车的残骸又无巧不巧地横在了路当中,熊熊燃烧着的躯壳,继续肩负着阻碍追兵的使命。

看到绑架者威猛的火力,少女的心又是“砰砰砰”一阵乱跳,脸上再也沉稳不住了,张大嘴就想尖叫,楚云飞的手掌及时地击晕了她。

楚云飞轻轻地吁了口气,可算是摆脱了追兵,都怪自己,太冲动了。

当他和成树国被发现的时候,两人要是能果断撤离,那什么事也不会发生的,可他对自己太自信了,也不想白来一趟就这么离开,才弄出这么大的动静。

差点就拉着两个战友在这里陪葬,楚云飞不能原谅自己的好大喜功。

当然,死倒未必一定死,只要人一多,把自己这几个人又逼得像在刚卡一样东躲西藏,那就跟死了没什么两样了。

十五分钟后,雷诺车已经到达了出发的地方,身后是响彻城市夜空的警笛声。

几人抱着装备匆忙下车,换乘上了那辆标致厢车。

这个女孩该怎么办?楚云飞皱了下眉毛。

他知道自己现在没有怜香惜玉的本钱和心情,他的本意,是想以女孩为人质,在被追得匆忙的时候,用不流血的方式杀死此人再推到车下。想来以她的身份,应该是能让追兵观察一下她的死活吧?

至不济,起码也是能阻一阻追兵的,不是么?

所以,楚云飞是一直把这个女孩当个死人来看待的。

不过,楚云飞马上就想到了这女孩的另一种用法,他不再犹豫,把那昏迷的女孩抱下车来,顶着其他几人不解的眼光,把她塞进了标致车里。

“看什么看?开车啊。”楚云飞招呼着多尼。

多尼扬扬下巴,“那车里,那车里有我的指纹。”

刘宁鄙夷地瞟他一眼,好胆小的家伙,二话不说跳下车,拎起车里的塑料油桶走了过去,“哗”地撒了一大片在车前脸上。

标致车开出去有五十米的时候,车速也起来了,刘宁一枪射去,雷诺车轰然起火,“但愿,这车是买了保险的。”

楚云飞已经习惯了俩战友的转变,刘宁的幽默并没有让他发笑,他的心情沉重依旧。

长叹一口气,楚云飞开始做检讨,“今天,今天都怪我,我不该那么好大喜功的,差点连累了你俩。”

刘宁瞟他一眼,“你没事吧?受人之托,当然要忠人之事,你要跑快了,我才鄙视你呢,再说大家也没什么事,你有什么好内疚的?”

成树国更绝,“我是要鄙视他的,本来我想的是,能弄点狗肉来吃呢,唉,你没见啊,那么肥的狗。”

楚云飞苦笑一声,“你们要这么说我就心安多了,不过,还是小看了工人党啊,计划得不够周密。我发誓,以后再不犯这种错误了。”

刘宁和成树国相对无言,楚云飞的态度,他俩实在是很不以为然的。

他们自从离开军队,虽然坎坷众多,困难重重,但从没有什么困难能真真正正地难得住他们。这里面,云飞绝对是功不可没的。

没他,刚卡边境就未必过得去,对塔尔人和武装分子的两场战斗,也绝不会轻易地得手,在索度墓地的事就更不用提了。

云飞的武力固然超强,但大家心里清楚,每次能活着闯过一道道难关,他的筹划起了不可估量的作用,还有那些一次又一次的临机谋变。

要说他的计划都不算周密的话,不知道什么样的人才能弄出周密的计划,情报不够才是决定性的制约因素。

不过,两人自然不可能去拍云飞的马屁,那实在是太没必要太见外的事的。

沉默半晌,刘宁才来了一句,“其实,你做得已经很不错了,咱们都年轻,怎么可能算无遗策?严格说起来,还是情报太少了。呃,对了,树国,你确定那狗是骟过的么?”

没骟过的动物有骚味,这是成树国常挂在嘴边的一句话,听到这个,大家轰然笑了起来,沉重的气氛不翼而飞。

多尼终于可以插得上嘴了,“楚,你弄这么个女孩回来,做什么?为什么不杀了她?”

多尼问这话的时候,那女孩正好悠悠醒转,楚云飞刚才那一掌,击得略微轻了点。

听到多尼的问话,那女孩情不自禁地抖了一下,说明她的英语是比较过关的。

楚云飞想了想,觉得还是照实说的好,“这个女孩,我本来是要拿她做人质阻碍追赶的,不过,现在是用不到了,我想拿她换一个人。”

“换人?”多尼有点不解,不过他马上就醒悟了过来,“是换那个咱们不打算伤害的人么?”

多尼的意思是,换来那个班克斯再三强调不要动的人,工人党的瑟利尼,不过有外人在场,他不好明说。

成树国也知道这事,不过他很轻蔑地笑了下,“切,子弹没长眼,绝对不伤害的可能是不存在的,别人说什么就是什么,那我们还混个屁。”

刘宁听了这话,觉得成树国的思维方式真的在向楚云飞靠拢,利害当头,什么都是假的,看来云飞的影响还真是不小。

他也自然知道楚云飞的想法,“云飞,你真想拿她换那个中国人?”

\section{第一百六十五章 束缚解除}

楚云飞点点头,“嗯,就当废物利用了,明天一大早,趁工人党混乱的时候,我带多尼去取那两个帐户的钱,把钱全部取完,也不怕他们知道我们是谁了,反正他们在明,咱们在暗,好好干他一家伙。”

大家都知道,制约自己几个人肆无忌惮行动的,就是多尼在银行的钱,他的钱一天取不出来,大家就一天不能随心所欲地行动。

理由很简单,当工人党把注意力全押在这件事上的时候,马赛那几家银行,会布满眼线的,只说进去还容易点,但只要工人党一天不解散,他们就永远有机会让多尼走不出银行大门。

钱都转移了的话,就算工人党明明知道楚云飞他们志在与工人党作对,可人海茫茫,他们找这几个人又谈何容易?

那样,工人党就只剩下挨打的份了,除非他们把帮派由明转暗,不过这是绝对不可能的。

想到不用再这么憋屈地不见天日,刘宁和成树国登时跃跃欲试起来,工人党是吧?我们要露头了!

虽然已经是凌晨三点了,但高速路上还是偶尔会有一两辆车跟这破旧的标致车擦肩而过,那女孩被再次打晕,没有任何的呼救机会,尽管,楚云飞非常怀疑她是否有呼救的勇气。

不止有单独的车辆夜行,居然偶尔还有十几辆车组成的车队。

工人党的老大戴维斯就在这样的一个车队中,此刻的他脸色发青,眼中透出慑人的寒光,这次丢人丢大了!

他本来正在巴黎同一个西班牙的帮会老大杜塞特。休伊会面,正是所谓的“联络感情”之际,就传来了马赛一而再再而三的坏消息,老三被绑架,老四遇袭,自己的亲侄子也丧了命,还引起了政府的注意!

二当家的普皮在马赛灭火,戴维斯也托了著名的交际花艾丝美。肯瑟为他在巴黎说项,务必要减小这件事的影响。

安置完这些事,戴维斯认真地想了想,还是不太放心,乘车连夜赶往马赛,好督促普皮办事。

没想到,他在马赛屁股还没坐热,就传来了家中遭袭的噩耗,由于他的一子一女已经成年,不在家中居住,家中只有老妻和三女儿。

戴维斯的妻子不是原配,前妻在二十年前就亡故了,给他留下了一子一女。

他现在的妻子在那时是他的铁杆情妇,说了要做他情妇一辈子,妻子死后,她当然被扶正了,后来为戴维斯生下了第三个孩子,就是楚云飞车中的少女。

听说妻子重伤,女儿被绑架,戴维斯再也坐不住了,虽然他如所有法国男人一般地自命风流,情妇无数,但对这个妻子和三女儿,他还是是格外呵护的。

路途短,车速也很快,戴维斯很快就来到了克鲁梭自己的家中,其实,刚才楚云飞能在高速路上伏击下戴维斯的话,效果绝对会是非常理想的。

很可惜,多尼提供不了太多的情报,楚云飞也不是神仙。

看着家中的残砖剩瓦,戴维斯实在是怒不可遏,“朗克,这就是我交给你的家?”

朗克就是刚才要大家追击袭击者的年轻人,此刻的他羞得无地自容,只有老实低头的份。

“到底会是谁干的?你们得出结论来没有?”戴维斯继续咆哮着。

朗克从另一人手里拿了张纸,哆哆嗦嗦地递了上来,“这是我们的初步分析。”

戴维斯一把夺过那张纸,交给自己的副手,“我没时间看,我要去看我的妻子,从现在起,你们不用睡觉了,我希望在我回来之前,你们能找到真正的凶手,还有我的女儿。”

戴维斯问都不用问,朗克交上一张纸来,那证明没什么有用的情报,如果有确凿的证据,他早口头汇报了!

………………

与愁云惨淡的工人党不同,楚云飞陪着多尼顺利地把钱转了帐,但帐上有多少钱,楚云飞没去打听。

事实上,说顺利,也不算顺利,楚云飞又干起了他的老勾当:假扮女人!

当多尼进入法兰西银行马赛支行的时候,银行门口有一个戴着墨镜嚼着口香糖的年轻人在闲逛,他看到多尼后摘下了墨镜,等多尼进去了,又从怀里拿出了几张照片仔细看了看,一副若有所思的模样。

当他从怀里掏出一个手机正要按键的时候,一个个子较高的黄皮肤女郎“恰好”从他身边经过。

于是,墨镜男青年被女郎热情地扶走了,可以肯定这不是一次艳遇。

因为,那个男青年的朋友从此再也没见到过他。

楚云飞他们正在屋里谈笑风生,他的手机响了。

楚云飞头一个念头就是:最好是琳琳打来的,虽然,这个希望的泡沫已经破灭了无数次,但是,他已经把电话号码留到琳琳家,她总是有空闲的时候吧?

一看号码,楚云飞失望的同时,一点点的烦躁也不翼而飞:不是0086(中国的国际区号)的电话,就算不是琳琳的,起码也不是妈妈的。

带着一点点纳闷,楚云飞按了接听键,清脆悦耳的女声从电话中传了出来,是标准的伦敦英语,“你好,是楚云飞先生么?”

楚云飞斜眼瞟一下虎视眈眈的刘宁和成树国,起身走到了窗户旁边,边扫视外面边回答,“是我,请问你是哪位?”

刘宁和成树国对视一眼,都看到了对方眼中狐疑的神情,这实在怪不得他们,这只手机,音响效果实在不错,大家都听到了,是女人的电话。

成树国嬉皮笑脸地凑了上去,正好听到了对方的回答,禁不住发出声来,“咦?索菲娅?”

楚云飞点点头,同时挥挥手,示意成树国回避,“呵呵,我也很想念你们啊,找我有什么事么?”

索菲娅肯定是喋喋不休了半天,楚云飞一直抿着嘴默不作声,成树国不好再偷听,上下打量楚云飞一番,双手一拱,做出了一个“非常景仰”的姿势。

楚云飞提脚,做出个凌空虚踢的姿势,阻止了成树国的继续调笑,沉默半晌才说,“这个……我不能保证,不过,我想,我应该可以很快地回去吧。”

这下,众人用屁股也能想出索菲娅在问大家什么时候能回英国去,要是说回中国,谁敢这么承诺?要知道,楚先生可是大仇还未报呢。

\section{第一百六十六章 悔恨的戴维斯}

接下来,电话明显换了主人,因为楚云飞说了句,“你好,班克斯先生。”

听到这里,一干八卦党没有了继续听墙根的心思,说起了别的事情,直到楚云飞神情严肃地坐回沙发。

刘宁看了他一眼,“怎么了,云飞,一脸苦大仇深的样子?”

楚云飞绷着脸,沉默半晌才说,“工人党已经知道了,事情是我们做的。”

大家都愣了一下,半天才由成树国打破僵局,“切,知道就知道呗,咱们现在还怕他们不成?”

是的,大家都不怕,但是,这个谜底不是由己方揭开的,谁都难免会有几分不爽,虽然这并不算什么要紧事。

要紧的事是:他们是怎么知道的?

楚云飞很快地为大家解开了疑团。

问题还是出在多尼在“云丝顿”宾馆订房间的事情上。

这事情办得很圆满,纰漏不是出在多尼身上,而是出在那护照上面。

工人党顺着护照一路追查了下去,才发现那是张假的护照。

假护照很好打听的,尤其是能骗过“云丝顿”宾馆的的护照,全法国能制造的是寥寥无几的。

于是,这条线就延伸到了某个假护照制作大师的门口。

工人党很想对那大师采取些什么行动,但却发现那大师只是某个黑货交易行的御用工人。

没错,工人党很强横,但在首都巴黎他们还是嚣张不起来的,那里同他们势力相近的组织有三个,而且都是隐约有内阁背景的。

当然,那三个组织如工人党一般,不是发展不起来,实在都是不敢再继续发展的。否则,哪怕背后站的是总理,也有吃不消的时候。

那个黑货交易行就是被两个这样的组织交叉保护的,背景强大的同时,也没失去太多的公正公平性,当然,这点纯粹是相对地下组织而言。

工人党不能用强,只能花钱去买消息了,那交易行倒也是恪守行规,死活不松口,直到知道了买主是同样强大的工人党,才在收钱之余微微吐口,说出了来交易的人数和特征,同时把那几张护照和身份证明也略做交代,至于他们还买了什么东西,那是绝对不可能再说了。

于是工人党惊讶地发现,原来这事情,居然是那个该死的波兰佬搞的鬼!

多尼的身份证明虽然是伪造的,但他的档案是一直都在的,伪造的内容同真实的证件完全相同。

可以这么说,除了颁发证件的机构不同之外,那证件别无二样,是如假包换的。

于是工人党马上动用关系,去查已知的多尼掌握的两个户头,却意外发现已经在上午被人全额转帐,转到了全世界分行无数的花旗银行,同时还有一名负责盯梢的人失踪。

盯梢的人失踪了,不过他的手机可没失踪,目前正拿在多尼手里,多尼还办了三个不需要身份证明就可以办理的预付费手机卡。

前后事情一对比,一联想,普皮和戴维斯终于确定,只有多尼,那个该死的波兰佬,他带的那三个中国人才有能力和有理由给工人党造成这么大的伤害。

到了这个时候,戴维斯才想起班克斯给他的忠告,“不要小看了那三个中国人,他们是有古老中国神秘力量的人,不能单纯地把他们当作士兵来对待。”

于是戴维斯幡然醒悟,他实在是太需要弄颗后悔药来吃了,不过还好,迄今为止,中国人带给他的伤害还在承受范围内,妻子死不了,而嚣张的希伯伦和他保镖的生死并不是什么大事。

利害关系在前,贝维尔的妻小和自己侄子的生死实在不是什么大事。

于是,戴维斯迫不及待地给华尔街的黑手党们打了电话过去:请中国人放了自己的女儿,他那里也放掉那个他们的同伙,从此之后,止息干戈,两不相扰。

楚云飞心里早有了主意,但他不能不征求自己同伴的意见。

成树国登时跳了起来,“开什么玩笑?他们生事在先的,交换人质可以商量,其他的,不死不休。”

刘宁也知道楚云飞卖弄武力的因由,这实在是件很划算的事情,敲打下黑社会,能保证将来对付“基天”时得到更多更准确的情报,他也点点头。

“没错,我们现在钱也拿出来了,哦,错了,是多尼的钱拿出来了,没什么东西能束缚我们的手脚了,停战?那是做梦!”

多尼也坚决地点点头,“我同意,不死不休!”

楚云飞很高兴大家的看法一致,微微一笑,马上拿起了手机。

“你好,班克斯先生么?……我同意他们交换人质的意见,但是,我可以负责地告诉你,我的同伴,刘先生和成先生现在都在我的身边,要他们和你讲话么?”

“……好的,你能相信我的话,我很高兴,同时请你转告戴维斯先生,钱我们已经拿到手了,人质和他们交换完毕后,我们和工人党——不死不休,包括他们的亲属!”

班克斯显然在遥远的英国愣了一下,才开始为他的法国朋友说起话来。

不过,楚云飞已经没有耐心再听下去了,他逮个空子,打断了班克斯的讲话,“事实上,班克斯先生,正如你所希望的那样,我在显示自己的力量,以便将来我们能更好地合作。”

“您应该记得我说的话,得罪我朋友的人,有必要考虑一下我的愤怒,我说的话从来都是要做到的。”

说完,楚云飞不再等对方的回答,毅然地挂断了电话。

班克斯拿着电话愣了半天,才对着在电话站了半天的索菲娅来了一句,“我觉得,你那个同伴给咱们维伦斯家介绍了一只魔鬼,还好,他是我们的朋友。”

索菲娅脑子里想的可不只是这些,“他真的那么可恶么?”

班克斯盯着女儿看了半天,不知道在想什么,好久才长叹一声,“事实上,如果我是个年轻少女,一定会嫁给他的,不过这种事,你们小女孩子是不懂的。”

只有男人才能真正地了解男人!欣赏同类的永远是同类!

可惜,索菲娅并不懂这个,她只是在心里默默反驳,“并不是只有你了解楚先生!”

与此同时,在地球的另一端,李南鸿,那个略微有点滑头又有点点义气的家伙,在相同时间说出了类似的话。

\section{第一百六十七章 李南鸿的得意}

在楚云飞他们离开了维伦斯家后,李南鸿又尝试了几次,试图同索菲娅建立起比较亲密的关系,以便“给老爸带回个外国媳妇”去。当然,他已经了解了对方的背景,也不敢轻易地造次。

可索菲娅对他的努力视而不见,维伦斯家对他的态度也随着楚云飞的离开而变得平淡了起来。

于是,在露丝离开伦敦的第二天,李南鸿也不得不黯然回国。

见识了维伦斯家的气派,他实在再没有勇气“周游列国”了,人和人的差距,原来真的可以是大到如此地步的。

还好,他还年轻,短暂的创伤对于年轻人来说,时间就是治愈它的良药,无论是心灵上的还是肉体上的。

回国一个多月,他的状态早就调整了过来,今天他正陪着自己的死党张拙,那个给他往索度寄东西的朋友,在饭店里吃饭聊天。

饭店不算大,人也不少,两个包间全满了,两人找了张桌子坐下点了些酒菜吃喝起来,谁要这里味道做得好呢?

没吃了几口,又来了三个年轻人,饭店里实在没地方了,同他们并了一张桌子。

可那三个年轻人实在不太安分,不但长得一副彪悍魁梧的模样,脸上还带着一丝杀气。他们不屑地看了李南鸿两人一眼,自顾自地大声聊起天来。

他们聊得居然是“高歪脖”,果然不是什么好东西。

“高歪脖”是他们这里著名的一霸,横行江城近十年,欺行霸市、欺男霸女、无恶不作,盛名可止小儿夜啼,目前事发正在被通缉中。

他那“高歪脖”的外号并不是说他脖子长得有缺陷,而是说此人脖子扭一下就可以不认帐,典型的有奶就是娘,翻脸不认人。

三人中有一人似乎很是景仰“高歪脖”,另外两人却是稍稍地有些不屑,拼命讲些“高歪脖”的糗事来打击同伴。

说到最后,连李南鸿他俩都听了出来,那俩人甚至不惜伪造些谣言来打击同伴。

那景仰者喝了点酒,就有点着急了,直着嗓子喊了起来,“扯淡了,高哥手下两条人命,十几起致残,数遍中国能有几个这样的好汉?”

两条人命?数遍中国?李南鸿听得差点把一口菜喷出来,“两条人命也敢说好汉?嘿!”

他说的声音不大,可那三条汉子却听到了,那景仰者又斜他们一眼,“小毛孩子知道个屁,你有种杀个人我看看?”

李南鸿被这一眼瞟得火气上来,大声回应,“才两条人命嘛,我飞哥手底下一百多条人命呢,高歪脖算个屁!”

那三人听得一愣,然后同时放声大笑,“哈哈。”

一个驳斥者似乎脾气好些,“朋友,吹些小牛没什么,你这牛吹得实在太大了点,不过我们哥仨今天心情好,懒得跟你计较了。”

李南鸿可吃不得什么激,对方又没什么恶意,一激动就吵了起来,“你们才见过多大的天?我飞哥叫楚云飞,不信的话自己打听去好了。”

有名有姓!这着实让三个汉子吃了一惊,脸上嘲讽的神色也不见了,相互交换了下目光,另一个看似喝得最多的驳斥者从怀里掏出个本本。

“警察,现在,你俩跟我们走!”

那本本赫然就是警官证!

那三人原本是来饭店“钓鱼”的,也就是走访、排查兼打听,“高歪脖”是省厅督办的案子,由不得他们不上心。

三人本想通过有褒有贬地评价“高歪脖”,吸引通缉犯的喽罗出面。也没存了真能得到线索的心思,他们可万万没想到,居然能碰到个一百多条人命的大案子,那可是天大的功劳啊!

就算案子不会在他们手里破,他们也提供了线索,而且还有证人!

可见,有时候话的确是不能乱说的。

李南鸿和张拙在江城市公安局一呆就是四十八个小时,前二十四个小时那是没得说,就是疲劳审问了,也就是强迫二人不睡觉。

李南鸿还好,有些东西可说;张拙可苦恼了,他压根就是被捎带进来的,反而落了个“态度不好”的名声,到最后,他只能讲讲听来的索度和欧洲的风土人情来熬时间,他实在不太清楚那个“飞哥”,只是知道李南鸿说过几回,是个狠人,而且聪明异常。

案情重大,警察们没有对二人进行什么体罚,不过,张拙那“恶劣”态度,还是差点吃了苦头。

剩下的二十四个小时,不签发拘留证的话已经不能再羁押二人了,但警察们有的是办法,不能羁押是不是?那好,换个地方。

警察把李南鸿和张拙安排到了公安局小会议室,不但有沙发睡,还有电视看,两人也能在大楼里自由走动。

而且,年轻人是很好骗的,“这个案子实在是太大了,要真像你们说的那样不关你们的事,还是最好不要出公安局的楼门,否则我们不得不申请拘捕你俩了。那样会在档案里留底的,要知道,这是为了你们俩好。”

还好,在有些时候,国家机关的办事效率还是相当可观的,终于在将近羁押四十八小时的时候,上面来了文。

“绝密++!经查证,楚云飞等三人在国外杀人逾百应有此事,国内尚未听闻三人有任何不轨行为,被害人员所在国无配合缉拿要求,且此三人目前不在国内。……另:所有审讯记录由省厅送至公安部,不得留底存档。……”

公安局长陆达明手持传来的公文倒吸口凉气,靠,原来都是真的,中国什么时候出了这么猛的人?

李南鸿和张拙终于被放了出来,虽然被疲劳审讯折腾得面无人色,但送他们出门的警官眼里那异样神色还是让他们趾高气昂。

我哥哥杀了一百多人,你们有这样的大哥么?

李南鸿和楚云飞他们接触的时间比较长,所以他们的经历他大致上是了解的,又由于他从未进过派出所,而且楚云飞他们什么时候能回来也未可知,再说他们杀的都是外国人也没人报警……

总之,抗不住那恶毒的疲劳审讯,又因为有了种种的理由,李南鸿把所有他知道的东西全说了。

当然,因为自己有知情不报的嫌疑,唯一可能给飞哥带去麻烦的案件——索度首都的灭门惨案,被李南鸿坚决地隐瞒了。

事实上,有了这些精彩故事已经足够了,警察们都听得目瞪口呆,甚至有俩警察换班了都没走,继续留在那里旁听。

尤其是楚云飞三人在索度警察局嚣张地救他的事,听得审讯的警察都狠拍一下桌子,“够仗义,好汉子!”

李南鸿不知道,这些审讯记录最终被送到了另一个地方,有了这个报告,叛国的三个士兵才被最终确定了行踪,同时也了解了他们对待同胞和祖国的态度。

\section{第一百六十八章 奇怪的人质}

楚云飞他们当然不可能知道国内发生的事情,他们正在为交换人质做准备。

戴维斯求和不成,心中怒气又起,不过这交换人质他是不得不同意的,因为那毕竟是自己的女儿,最心爱的女儿。

由维伦斯家族出面做中人,双方很快就谈好了条件,楚云飞他们只需要把女孩放到市区任意角落即可。

对中国人质,楚云飞他们就多了点要求,因为他们谁也没见过这个被抓起来的人,必要的核实是应该有的。

楚云飞提的条件也很简单,要求对方为人质提供手机一部,并提供5000法郎备用,当然,人质的随身物品包括护照等是必须准备齐全的。

如果玩花样,双方有权力猎杀已放出的人质。

戴维斯真的是打算玩花样的,维伦斯家族的面子一定是要给的,人质当然是要完好地交出去,这点是毫无置疑的。实现了这点,就算完成了承诺。

可是,这是逮住对方三人的大好机会,顺藤摸瓜绝对是不错的主意,还好,工人党经过多年积攒,好东西还是有一些的。

不过,为了降低中国人的戒心,戴维斯也提出了同等的条件,以便第一时间知道女儿的下落。

巴黎时间第二天上午十点半,双方同时释放人质。

小女孩看着绑架者的车辆离开,第一时间通知了自己的父亲,“爸爸,他们放了我了。”

戴维斯对这个倒不是很担心,在他眼里,这几个中国人还是惹不起维伦斯家族这样的巨无霸的,有他们做中人,女儿的安全是很有保证的。

他关心的是自己安排的事,这几个隐藏的中国人如跗骨之蛆,是不可不除的。

楚云飞用多尼富裕的卡,大约在十点三十五分左右拨通了对方的号码,“你确定自己被放出来了吗?说中国话。”

对方显然也做好了准备,“我确实被释放了,东西齐全,但有人跟踪。”

“好的,”楚云飞加快了语速,“你马上找家通讯商店,买一部手机和卡,给我这个电话号码打回来,响一声铃就可以,然后扔掉旧手机,等我电话,万事小心,明白的话请挂电话。”

对方的反应非常专业,专业到楚云飞在怀疑这是不是个陷阱,电话居然在一秒钟后就挂断了,整个通话过程四十三秒。

楚云飞对通讯是非常精通的,他在部队的时候,曾经修理过电话站的共电交换机,学习了不少关于通讯的知识。

所以他知道,用手机时间稍长,有种很先进的仪器能在一分钟的时间内锁定基站和扇区,时间再长点的话,可以锁定坐标的。还好似乎对方也明白这点。

至于扔掉旧手机和卡,卡是可以复制的,手机里有可能有微型窃听器或者说定位仪。

戴维斯刚在为自己找了个懂汉语的专家而沾沾自喜,马上又被专家翻译过来的语句激怒了,“狗屎,居然这么狡猾,怪不得他要钱呢。”

这么一来,手机里的窃听器和自己复制的卡全派不上用场了。

不过还好,戴维斯暗暗庆幸,亏得自己还在对方的夹克和鞋底安装了窃听器和定位仪。

只是他的得意没持续多长时间。

被释放的中国人质肖逅东看看,西看看,又把夹克上巨大的虎头拉链头托起来看看,露出了一丝神秘的笑意。

他的脸上还有些伤痕,不过看得出来,拜维伦斯家族的名头庇护,肖逅并没有受到太大的虐待。

一小时后,肖逅打了个电话给楚云飞,铃响一声后挂掉。然后顺手扔掉了前一只手机。

一个黑人少年飞也似地跑来,捡起了手机,又脚不沾地跑开,肖逅摇摇头微笑:这马赛的治安实在是够糟糕的。

一个陌生号码马上回了电话回来,刚办的卡,会是谁打来的呢?

肖逅看着陌生的电话号码,笑意又涌了上来,点点头喃喃自语,“果然是这样,他是谁呢?”

接起了电话,没等楚云飞开口,肖逅先笑嘻嘻地说话了,“你好,是我,肖逅,十二生肖的肖,邂逅的逅,我还买了全套的衣服和鞋袜,是不是需要找个地方换换?”

这话本来是楚云飞要说的,现在他被弄得满头的雾水,“奇怪,你怎么会想到这些的?你到底是做什么的?”

肖逅实在是个很喜欢笑的人,“呵呵,见了我你就明白了,好了,我还要洗个澡,等一会儿我给你打电话,先这样吧?”

戴维斯直接跳了起来,端起咖啡就泼到了翻译的脸上:“滚!你给我滚!一分钟内我要还能看到你,我一定把你切碎了喂狗!”

歇斯底里,这是绝对恐惧下的歇斯底里,看来,戴维斯真的是被楚云飞他们折腾得够呛。不过说句良心话,楚云飞他们再这么大手笔折腾下去,不用他们怎么出手,法国政府就要剿灭“工人党”了。

当楚云飞和多尼琢磨自己到底救了个什么人的时候,肖逅的电话来了,“好了,我确定身上没有任何的麻烦了,跟踪的人也被我甩掉了,我们是不是可以见面了?”

楚云飞想了半天,终于确定这应该不是个陷阱,没有哪个陷阱会做出明显的标识——“我很异常”。

当然,工人党应该还是有能人的,也可能使用逆向思维的方式来诱骗自己,但那个肖逅的语气非常地平静,要真是陷阱的话,那他一定是个超级配音演员。

最后一点,楚云飞实在不愿意相信,一个同胞会如此丧尽天良地害他,大家毕竟都是中国人,不是么?

不过,自己的性命可以不太讲究,战友的生命是绝对不能不重视的,在多尼的指点下,楚云飞终于约了一处隐秘的地方和肖逅见面。

肖逅想见到他,必须要路过暗中观察的多尼、刘宁和成树国。

其实,只要他们三个确定此人长得和楚云飞有三分相像就足够了,那就证明戴维斯没有失言,已经把人放了。

匆忙间,找个像楚云飞的中国人来应付可不是那么容易的。

双方实在没有见面的理由的,不过,现在这个肖逅成功地引起了楚云飞的兴趣。

他,到底是个什么样的人呢?

\section{第一百六十九章 肖逅印象}

见面是在一个堆满垃圾的废弃小巷里,楚云飞一人站在巷子尽头,如有意外,他可以很轻松地跳过高有三米的围墙,那边是一大片低矮的小楼,以他的功夫,脱身不难。

耳机中,先后传来了刘宁和成树国“咦”“咦”地两声,就再没声息了。

肖逅大大咧咧地走了过来,身材同楚云飞相差仿佛,脸庞五官也长得非常接近。

虽然天色已经暗了下来,楚云飞还是忍不住指着肖逅念叨起来,“操,不是吧,你黑得都快赶上包公了,居然他们说咱俩像?”

耳机中是一阵哄堂大笑,他俩,已经憋了很久了吧?

肖逅笑嘻嘻地点点头,“是啊,你白得跟曹操有一比了,看来这黄种人的涵盖面还真不是一般地广。”

“不过,我多少也是替你受过,你用不着那么义愤填膺地吧?”

楚云飞看清楚了对方脸上尚存的伤痕,那已经快痊愈的痕迹,绝对不是匆忙之间能弄得上去的,一块石头终于落地。

“好了,弟兄们,不是假的,收工回家了。”

于是,标致车里又多了一名中国人。

楚云飞对肖逅十分好奇,肖逅对他的好奇也不仅仅只有九分。

他在被关押的期间,才逐渐弄明白,自己只是因为长相酷似某个同胞而被误抓了进来。

这个可能姓楚的同胞,居然惹的是“克鲁梭工人党”这样的法国顶级黑社会组织,这个组织,肖逅可是听说过的。

当然,肖逅肯定是辩解过的,你们认错人了。可惜是个人都明白,这时候打死也不能承认抓对了人,所以他辩解两次,发现是徒劳的,就只能认命了。

肖逅当然不会无聊到相信有人救他出去,他只能干等,等着工人党突然发现自己抓错了人,但愿,那时候自己还活着吧。

没想到,老天终于发了回善心,那个同胞不但出现了,还采用了交换人质的办法换他出去,而且工人党还不敢不听,似乎是还有英国更大规模的黑社会做担保。

等到楚云飞干净利落地教他甩掉跟踪的时候,他的兴趣就更浓了,这个人,我一定要见见。

所以他先请教起了楚云飞的来历。

楚云飞自然是有什么说什么,事实上,他从没认为自己和战友们做的事有悖情理,最多也就是与政策不合而已,说出来没什么可丢人的。

等到楚云飞大致讲完,几个人已经在郊区坐了有一个小时了。

肖逅点点头,这几个家伙实在个个是怪胎,强悍如此又蛮横若斯,凭着一股血性,居然也是搞得风云变换,实在是难得的人才。

下面就是肖逅介绍自己了,原来,他是国家安全局的人,怪不得那么擅长间谍的那套东西。

法国巴黎不久后要办一次特大的拍卖会,会上有失落在国外的国宝若干,其中有一件青铜獬豸樽,堪称举世无双。

最要紧的是,它还是西周祭祀时的“九器”之一,其余“八器”现在都已经收藏进了中国历史博物馆,只剩下这一件在出土时被盗,漂流海外。

于是博物馆相关人等是下了狠心,再贵也要把这件国宝拍回来,让它们团聚。

兹事体大,不光可能会动用大量资金,而且还存在国宝的保护问题,于是历史博物馆上报给了有关部门,几乎快上升成了国家行为。

肖逅就是国家安全局派来配合这事的,他精通法语,办事机警,是打前站的好人选。

刘宁皱皱眉头,“不是吧,你们国家安全局的就那么菜?让几个小流氓就抓住了?”

这话弄得肖逅实在是哭笑不得,无心算有心,他再厉害能有什么用?再说,要不是制服他的时候他显示了极其精湛的格斗技巧,也不至于被人死死咬住他就是楚云飞啊。

大家你看看我,我看看你,谁也懒得继续说话了,道不同,怎么为谋?

最后还是肖逅洒脱,冲几人一拱手,“几位兄弟,我肖某今天在这里谢过大家了,你们的冤屈,我会找个机会向领导反应的。”

“兄弟我在巴黎还有公干,不敢久呆,不过,这手机号码不变,有事的话,大家联系啊。”

楚云飞刚想把自己的手机号码告诉他,再一想,连刘宁的老爹都搞不定的事,何必去麻烦人家,还是山高水长,来日相见吧。

接下来的日子里,工人党小心谨慎地做事,一干老大们也尽量地不出大门,不给楚云飞他们任何制造大麻烦的机会。

楚云飞他们盯了几焐裕沼诜⑾痔延邢率只崃耍敲创蠹宜餍郧逑屑柑旌昧恕?

百无聊赖下,大家都迷上了一样新鲜的东西:计算机。

在欧洲,计算机已经开始普及,网络也不是什么新鲜事物了,当楚云飞从报纸得知:很多“基天”组织成员都是从网络上进行联系的,马上就买了一台电脑学习上网。

三个中国人以前还是接触过电脑的,不过那是古老的“OrangeⅡ”电脑,屏幕都不是黑白的,是淡绿的。

成树国和刘宁不甘落后,一人也搬了台电脑回来,接下来就是大家的学习比赛。

心中有事,楚云飞学习的进度就不是很快了,同两个战友类似,不过最让他失望的是:他花费了将近半个月,也才找到两个有嫌疑的网站,数据通讯费倒是交了不少。

不过,楚云飞意外地在一个网站上找到了马哈苏德的照片,正面和侧面加起来一共四张,这计算机多少算是没有白买。

多尼终于坐不住了,开始催促三人:工人党那边似乎放松警惕了,大家可以继续行动了。

“上学时候还是学得太少啊,现在才知道学习机会的难得。”楚云飞一边嘟囔,一边不情愿地关闭了电脑。

按多尼的想法:逮不到大鱼,那虾米也不能放过,有事没事地去工人党负责的地方砸砸场子,迟早工人党会沉不住气的。

可楚云飞他们三个怎么可能同意这样的意见?不但危险系数太大,治安弄得太糟糕的话,工人党以此来举报,导致政府通缉怎么办?

两边都怕政府,都不想触及底线,所以说,各个圈子有各个圈子的规矩,其实是很重要的。

不过,他们很快就有了一个很好的机会。

\section{第一百七十章 干掉贝维尔}

机会来自贝维尔,那个老婆和女儿被杀的家伙。

贝维尔:男,四十三岁,身高一米八六,体重二百一十磅,灰白头发,略有秃顶,左臂受过枪击不太灵活,特征是酒糟鼻子。

贝维尔算工人党里的另一个狠人,冲动起来都敢跟希伯伦动手,在工人党里,他只听戴维斯的话,其他人都指派不动。

因为他这个性格,戴维斯把他派去跟荷兰人一起看守人质,倒也不虞丢了工人党的人。

老婆和女儿被误杀,他是绝对咽不下这口气的,当下就要在马赛全城搜捕。

当时马赛已经乱成了一锅粥,脆弱的局面实在不能再添乱了,戴维斯果断地命令贝维尔去泰国考察,看能不能打通购买雏妓和人妖的路子。

贝维尔带着万分的不甘心走了,不过还好,戴维斯答应他了,如果有机会,一定让他亲手报仇。

贝维尔在泰国一呆就是二十天,期间几次要求回国,都被戴维斯拒绝了,戴维斯是想把事情冷处理一下,看看情况再做决断。

现在,戴维斯实在无法继续把贝维尔留在泰国了,只好答应他回国,飞机直飞马赛。

机会就在这里。

工人党在马赛是很猖獗的,但是马赛飞机场是他们不多几个伸不进去手的地方,由于现在欧洲的恐怖份子四处煽风点火,制造祸端,机场的管理是非常的严格的。

非常时期下,工人党可以挤满机场大厅,但机场外面的停车场却不可能让他们排起一溜车队来显示威风或者保护某人。

贝维尔明天就要回来了,他从出机场到离开停车场的这段时间,虽然只有一两分钟,却绝对是伏击的理想时机。

伏击好打,逃跑却难,几人经过详细策划,终于制定了一套实用的撤离方案。

多尼这次买了好几副手套以备更换,楚云飞更是连夜偷了五辆车备用。

马赛,普罗旺斯机场,下了飞机的贝维尔在七、八个大汉的簇拥下走出了大厅,等着自己的车来接。

就在此刻,一辆两座的兰色雷诺跑车“吱”地一个急刹停在了众人面前十米处,黑洞洞的枪口一闪即逝。

“砰”地一声闷响,贝维尔脑门正中出现一个血洞,在两秒钟之内,跑车急速启动,撇开众人扬长而去。

车里的人是楚云飞和成树国。

一时间,警笛大作,贝维尔的车队第一时间追了上去。

跑车即将路过一个高架桥洞时,车上滚落几枚手雷,爆炸声中,尾随的几车不由得一窒,等到他们再次启动的时候,却愕然地发现兰色跑车就在前面不远处停着,车里的人已经不见了踪迹。

“他们换乘了那辆白色警车!”

呃,警车?那也要追!

追了没几步,一个拐弯后,警车在那里停着,车上的人又不见了。

换了四辆车,走了一个“之”字型路线后,四人终于甩掉了所有的跟踪者,不再停留,直接转道巴黎。

从马赛到巴黎用不了四个小时,众人到达的时候天还没黑。

巴黎绝对不是工人党的地盘,不过大家还是比较小心。偷来的车是不能再开了,可是三个中国人身上携带着长短枪支和一些弹药,又不能就这么大摇大摆地在街上闲逛。

多尼不愧是曾经的“业余混混”,居然让他找到了一个地下黑车改装厂,虽然改装的车都是要走私到那些亚非等欠发达国家冒充好车的。但是白花花的银子花下去,买两辆车还是不成问题的。

多尼现在的状态已经有点不太对头了,楚云飞他们屡次的成功行动已经把他的胃口吊了起来,居然肯花钱来买辆“罗尔斯。罗伊斯”。

尽管是冒充新车的二手车,价钱也不便宜,不过多尼有自己的理由,“我们不能总用便宜车,偶尔出人意料一下是有必要的,只要能顺利报仇,多花点钱我不在乎。”

成树国这喜欢玩车的自然不介意,楚云飞虽然也喜欢玩车,但他还是敏感地注意到了多尼心态的变化,“多尼,你不能心急,报仇这事一定要慎重,急了会出事的。”

多尼却是一副热血沸腾的样子,“事不在你身上,你自然不急,我恨不得马上冲到工人党总部杀个利索。”

成树国撇撇嘴,虽然他对多尼的态度总是不太友好,但实际上对他并没有什么成见,听他说得凄凉,也懒得理会。

楚云飞可是有点冒火,我还想早点帮你做完报我自己的仇呢,不过想想未来的行动还是要自己做主,也闭了嘴,不想理他。

多尼却是不肯放弃这个话题,“接下来,我们做些什么?”

这个,楚云飞早有计划了,“今天十二月三号,还有二十一天就是圣诞节,那时候,我们去找他们麻烦好了,去克鲁梭。”

多尼有点失望,“不会吧?还要那么久?”

楚云飞看他一眼,懒得理会,刘宁可不干了,“你知道什么?我们还要去沙特办事呢,要不是看你人还可以,我们都懒得帮你,你以为我们很缺钱么?”

多尼翻翻眼皮,一句话想说却没好意思说出来——你们是不缺钱,那是因为你们打劫了维伦斯家族,我求你们办事的时候你们可还穷着呢。

其实,楚云飞已经有点厌倦了在法国的打打杀杀:人做事,总是要有个限度的,工人党是做了不少坏事,不过对你多尼也没做什么太出格的事,现在已经差不多了吧?

工人党现在缩成一团,大家只能等着打黑枪了,一点技术含量都没有。

不过,已经答应了的事,楚云飞是不会半途而废的。

楚云飞有种预感,工人党是不会一直这么坐等自己对他们零打碎敲的。最终,他们是会制造些陷阱,或者用其他什么方式来激烈地解决掉这事的。

事实上,楚云飞还是有点私心在里面的,如果能把法国的事在一月上旬结束,那去沙特的时候应该是北半球一年中最冷的时候,能多穿几件衣服,自然是好携带枪支的。

沙特实在是太靠南了点!

他没想到的是,后来他遭遇的天气要比想象中的冷得多得多。

还好,对训练有素的士兵来说,恶劣天气并不是什么不可逾越的障碍。

\section{第一百七十一章 多普度之死}

这次的目标是多普度,工人党第五号头目。

选择他的理由很简单,因为戴维斯和普皮已经失踪了,有情报说他们去美洲渡圣诞节去了,也有人说他们藏了起来,总之这两人不见了是铁的事实。

至于说袭击他们的家属,虽然楚云飞这样威胁过,但那实在是有太大的风险了。在法国,没有人能搞得到重武器进行远程袭击的,枪榴弹基本上已经是极限了,近身交火对楚云飞他们来说确实不划算。

事实上,法国的重武器走私是相当严重的,连导弹都有得卖,像工人党就是吃这样的饭的。

不过,作为买主的话,你只能在港口或者火车站接到你的货,而且必须要在法国人的眼皮底下运出法国,否则,那你就不要想买了。

所以从理论上说,只有工人党可能用重武器远程袭击楚云飞他们,当然要做到这个须有两个前提:一、工人党所有人都已经疯了;二、他们找得到楚云飞们的藏身之地。

所以,当班克斯应瑟利尼的要求,打电话请楚云飞“放弃对无辜家属的屠杀”时,楚云飞想了想,“原则上同意”了,也算不大不小给了班克斯一个面子。

放弃对家属的骚扰,换来的是冷枪袭击,不知道工人党认为划算不划算?

多普度可没想到对方会在这时候把念头打到他的头上。

事实上,多普度在工人党内是个很低调的头目,靠着父亲的人脉坐上了第五号交椅,平时不惹人也不爱生事,只不过是个酷爱美女的花花公子而已。

多尼的事他自然是知道的,希伯伦没参与此事也被弄得失踪,实在是应该归咎于平时的恶名太响。

在多普度看来,这事怎么也算不到自己头上,虽然他也尽量地注意了自身安全,但内心深处,他并不像戴维斯和普皮一样惶惶不可终日。

当然,该有的戒心他还是有的,可圣诞节也是要过的,不是么?

他忘记了,中国人其实不过圣诞节的。

这天,多普度一反常态地没有去那十几个情妇家的任何一家,而是老老实实地去商店里给儿子买了一套电动卡丁车。

就在他推开家门的一瞬间,一颗子弹从远处飞来,穿透了他的太阳穴。

鲜血,淌到了莹白的雪地上,溅到了那包装精美的盒子上。

众多工人党驾车飞赶凶手,一辆白色的“罗尔斯。罗伊斯”迎面驶来,没人有功夫去理会车里坐的“富豪”们。

驾车引开工人党的是楚云飞,他的驾驶水平不算糟糕,不过,在雪地上开车实在不是一件愉快的事情。

等他把车开到计划中的位置,差点刹不住车,工人党的打手们也已经追了上来。

不过,下一刻,众多工人党党徒只眼睁睁看的份:车内的人闪电般跳出,飞身跳上三米多高的平台,又一个纵身,攀上了相隔两米远的另一座将近七米的小楼,几个起跃后,消失不见了。

开枪?开屁的枪,谁能打中移动那么快的物体,更别说大家都已经看傻了。

已经有消息灵通的,想到了绑架老大戴维斯家人的那个人了,据说也是跑得极快的,不过,这家伙真的算人么?

诡异的是:那车里居然没有查出任何的枪支来,难道他能把枪吃了不成?

距离那么远,肯定是步枪才有那么大的威力和准头的,这时候,才有人想起那辆奇怪的“罗尔斯。罗伊斯”。

所以,多尼买的车注定是用不了多长时间的,哪怕它再豪华!

等到楚云飞和同伴汇合后,多尼又开始了他的唠叨,“楚,你真的好棒,居然这么快就甩掉了他们。”

楚云飞不知道这家伙想说什么,不过,无事献殷勤,那肯定是非奸即盗,瞥他一眼,“再棒的计划,也只能用一次。下次我再引开他们也可以,不过,你确定你敢从时速一百公里的汽车上跳车么?”

多尼想说的还真是这个意思,有这么个超人在,每次都能引开追兵的话,那以后的行动不是会很安全么?

听到楚云飞的回答,多尼点点头,“也是,我这个想法,实在是太一相情愿了。”他想的是,早知道楚云飞一个人就这么厉害,那还不如不喊刘宁和成树国来法国,那两人似乎都有点累赘了。

多尼就忘记当时他还嫌三个人少呢,可见人心有的时候确实没有止境。

事实上,多尼又外行了。

这四人里只有多尼才懂法语,楚云飞一个人根本就承担不了既保护他又杀人的责任。

就算每次行动他藏得都是好好的,只有楚云飞拿着地图出马,可单靠打冷枪,这仇要报到什么年月去?

近身混战那是想都不用想的,楚云飞又不是神仙。

接下来就是弃车了,这时候多尼又露出了小家子气,不过楚云飞可不管他,已经照了面的东西,那绝对是危险的,要考虑大局!

当他们溜回巴黎的时候,已经是晚上八点多了,街上空空荡荡,看不到几个人。

就在这时候,楚云飞的电话又响了,这次的区号是“001”

只要打过几次国际长途的人,基本上都知道001是美国,楚云飞很纳闷,那里会有谁给他打电话呢?

居然是班克斯!楚云飞越发地纳闷,“嘿,班克斯,你怎么不在家过圣诞节呢?”

班克斯没有回答他的问题,而是一个劲抱怨他怎么一直不开机。

“开机?”楚云飞有点想笑,买这几个手机是为了三个战友间联系方便而已,“很抱歉,下午我在工作,开机会影响我的工作效率。”

班克斯当然明白楚云飞的工种,“那可太糟糕了,这次是谁倒霉了?是多普度?”

楚云飞有点纳闷,这家伙消息怎么会这么灵通?

“是的,很遗憾,多普度先生永远没有圣诞节可以过了,你能告诉我你为什么这么猜么?”

班克斯在遥远的美国长叹一声,“算,这事回头再说吧,你能帮我叫多尼接下电话么?”

多尼接过了电话,简单的寒暄过后,班克斯开始了滔滔不断的讲述。

多尼的脸色由白转红,又由红转青,最后都要发紫了。

楚云飞看得很想笑,原来人的脸部真的可以有这么多颜色的。

\section{第一百七十二章 和平建议}

多尼遇到了一道不好处理的选择题。

戴维斯绑架了他在波兰的远房表姐玛莎一家四口,要多尼放弃对工人党的报复。

从这上面可以看出,戴维斯的工人党,情报掌握得非常准确。

玛莎比多尼大五岁,多尼小的时候,家里双亲正忙着拓展法国的业务,忙得顾不上照顾多尼,所以多尼是在波兰长到八岁才来的法国,以前一直是他的表姐一家在照顾他,可以说,他是玛莎一手带大的,两人感情非常好。

等多尼十五岁左右的时候,玛莎还来多尼家里住过一段时间,那时候的多尼恰巧是情芽萌动的年纪,他疯狂地迷恋上了自己的表姐。

这种畸形的恋情是可以理解的,多尼只是孩子而已,但绝对是不可以助长的,在双方家长的努力下,终于把这段不正常的感情扼杀在了摇篮里。

不可否认,在多尼这一生中,玛莎表姐始终会是他的软肋。

多尼愤愤然地挂断了电话,相信他如果拿的不是楚云飞的手机的话,绝对会恶狠狠地摔到地上。

对于这种情况,三个中国人实在是无权置喙的,只能很配合地流露出同情的眼神。

多尼突然间像是想起了什么,眼睛一亮,“楚……”

楚云飞知道他想说什么,直接用说反话的方式拒绝了。

“没问题,我可以帮你把她救出来,前提是,你必须提供她被囚禁的地方,还有看守的数量,火力结构。”

成树国不愧是同楚云飞搭档多年,马上继续接住了话题,“不过,多尼,云飞只有两只手,最多也就能救俩人,我们俩要在外面掩护,可帮不上忙的。”

多尼愣了半天,长长地叹息一声。

他何尝不知道自己的想法有点过于疯狂?就算能打听到消息,凭自己这几个人也不可能救出人来的。

楚云飞宁可用戴维斯的女儿换人质,也不去解救自己的同胞,那已经很说明问题了,他们损失不起。

想到戴维斯的女儿,多尼又开始生气,他早就后悔了,早知道那女孩会被放的话,当初自己就应该在车上给她一枪,也让戴维斯这个工人党老大尝尝失去亲人的滋味!

想到这里,多尼突然打个冷战:自己这是怎么了?还算是个绅士,算是个文明人么?自己承受的痛苦,一定也要在别人身上体现么?

关键时刻,多尼从小受的教育起了作用,没让他再继续偏激下去,事实上,他一直是以绅士来自居的。

其实,多尼没有意识到,在这个转变中,刚死的多普度起了相当大的作用,一个父亲带着刚给孩子买的礼物横死在了自家门口,仅仅因为,他是工人党人。

有时候,很多情绪,比如震撼等,是藏在人的潜意识中的。

考虑了半个小时,多尼终于痛苦地决定了,他没有再同在座的中国人商量,拿起楚云飞的手机直接拨了回去。

“班克斯,我决定了,同意终止对工人党的报复,但是,我要得到杀害我情人和朋友的波兰人。”

玩这个,班克斯比他在行多了,“这个没问题,事实上,他们还留在法国的,你愿意的话,我现在就去替你问地址。”

鬼才知道是不是那些人干的,不过班克斯已经这么说了,多尼也只能这么认同。

“还有托尼和他的弟弟沃尔特”,楚云飞在一旁补充,班克斯既然拿了钱来说项,给他添点烦恼,想来他的表情会很精彩吧?

多尼马上做了补充说明,一边打电话一边冲楚云飞点头致谢。

约定好的事情要是想翻悔,想必维伦斯家会很不高兴吧?幸亏楚帮我想到了,多尼如是想。

班克斯很快就再次打来了电话,这次接电话的是楚云飞。

班克斯先抱怨了一番,在他传达戴维斯善意的这阵工夫,工人党就又死了一个大头目,他脸上多少是有点挂不住的。

不过细想这事也能想通,那时的楚云飞正要伏击多普度,联系不到实在是再正常不过了。

戴维斯同意了多尼的请求,因为,托尼和他的弟弟沃尔特目前确实是在法国。

他只有一个要求,就是不能告诉别人,波兰打手的地址是工人党泄露出去的,只当是多尼的情报网自己弄到的好了。

戴维斯这次是真的吓坏了,不是为了别的,还是因为多普度的死。

恐怕多尼提的条件再苛刻点他也会答应。

出生在克鲁梭的戴维斯对这个城市实在是太熟悉了,当他听到楚云飞逃脱的地点和方式时,脑中马上出现了那个地方的景象。

朗克是戴维斯家的打手头目,上次他对楚云飞速度的形容,戴维斯一直以来都以为是失职者们的借口,这次他终于相信了。

想想今后都会有这么个算不上人类的家伙在暗处虎视眈眈、伺机下手,戴维斯就不寒而栗。而且这样的家伙,居然还有三个之多,就算消灭一两个也不算实质上的胜利。

何况,楚云飞的狡猾和老到在他们交换人质时已经领教过了,戴维斯和普皮想来想去都想不出合适的方法:怎么才能“消灭其中一两个”。

既然惹不起中国人,戴维斯难免就有迁怒于波兰人的心思,要不是原来你们一定要放多尼一马,至于招来恐怖的中国士兵么?

普皮更有别的怨念,在谈合作的时候,波兰人很是看不起工人党,曾经为此激怒了希伯伦,希伯伦纵然有再多的不是,终究也是代表着工人党呢。

你们不是厉害么?波兰人比法国人厉害那么多,那麻烦你们帮忙对付中国人好了。

当然,对于贝维尔的死,戴维斯还是相当痛心的,不过从另一方面讲,他的妻子和女儿已经死在了中国人的手上,他要不死,那绝对会成为制约双方达成和平约定的致命因素。

说起多普度,那纯粹是个意外,反正他除了玩女人,什么也不会,他一死,正好可以提拔亲近戴维斯的新鲜血液,比如说……瑟利尼。

至于希伯伦嘛,中国人其实早点来也不算什么太糟糕的事。

以上就是约定中乙方的所有想法。

\section{第一百七十三章 夜袭波兰人}

呃,下面让我们来看看甲方的想法。

多尼已经同意了,他的想法可以不去考虑,对于三个中国人而言,最大的问题还是在于:这个约定是不是可靠。

工人党会不会是设了个圈套,通过维伦斯家族引诱中国人,好一网打尽呢?

想分析清楚这个,就要看维伦斯家族是不是知情了。

如果维伦斯家族不知情,也就是说,工人党连维伦斯家族都敢一起骗,这种可能性实在是太小了。

楚云飞觉得,像工人党经营的这种买卖,绝对是需要维伦斯家族的人脉来支持的,只要维伦斯家族放出风来,说某某某个小帮派信誉不佳,交易时需要注意,那对工人党绝对是接近于致命的打击。

何况还有黑手党无处不在的报复?

还有就是,他们敢那么肯定一定能把自己这几个人一网打尽么?更别说还可能有部队其他的战友前来复仇了。

打蛇不死,绝对是打蛇者的噩梦!

要说维伦斯家族知情,那更不可能是陷阱了。

对维伦斯家,楚云飞虽然接触的时间不长,但也明白,这个家族带有两个鲜明的特征:贪婪而怯懦!

贪婪那自然是正常的,人的本性而已,而班克斯居然敢动多尼的脑筋,丝毫不顾忌才为父亲治好病的中国医生的面子,那只能用贪得无厌来形容了。

不过,这些都在可原谅的范畴内,三岁小孩持金上街,必然是要挨抢的!

可自己一出面保人,班克斯立刻买面子放人,那就涉及到另一点了:怯懦!

其实敢认为黑手党怯懦的,这世界上除了楚云飞也没几个人,不过,这个论断基本上是准确的,从班克斯对待“基天”的态度上就可见一斑。

严格说,这也不算怯懦,“千金之子,坐不垂堂”是应该的。毕竟家业大了,谁肯随便去拿家业随便去赌个什么东西?

正因为如此,楚云飞也相信他们不可能为了配合工人党这个不怎么入流的帮派,来赌自己三个人能不能全军覆没,或者说还要继续赌自己的国家不会出面找回场子。

所以,等到戴维斯提出,不能供出工人党泄露波兰人行踪的要求的时候,楚云飞基本上可以断定,这不是个陷阱!

事实上,这也确实不是个陷阱,从另一个角度讲,班克斯先生拿钱也不是拿得很心安理得。

班克斯其实并不想介入这两者的冲突中去,但是很无奈:瑟利尼,那个需要楚云飞实实在在关照的工人党徒,是他的私生子!

年轻时的班克斯是很英俊的,年少多金且风流倜傥,在法国这个“男人的天堂”里很是荒唐过几年,于是就同一个才貌双全的可人儿有了爱情的结晶。

但是,班克斯最终还是成了“门第婚姻”的牺牲品,对这个不姓维伦斯的儿子,他是有着太多歉疚的。

对工人党,班克斯还是吃得住几分的,可对楚云飞他们,他可一点自信都没有,关键还是能力上有所不逮,制约他们不住,虽然,楚云飞迄今为止还没有什么不良记录。

还好,他们还需要班克斯先生提供“基天”的行踪,理论上,不至于翻脸不认人,三方中,戴维斯还算最不用担心的一方呢,起码他以为公证人有制约甲方的能力。

就在这三方都提心吊胆中,楚云飞他们开始了最后的疯狂报复。

报复,还是开始在这个血腥的圣诞夜,既然不再害怕暴露目标了,楚云飞他们连夜赶回了马赛。

“波兰复兴运动”的人一共有十七人,也居住在闹市区,和“云丝顿”宾馆离得并不远。

多尼开着车回到了他们的暂住地,取出了所有的重火力,甚至还有他们五小时以前刚从黑市高价买来的土制火焰喷射器。

波兰人住的地方,是一栋独立的小二楼。

楚云飞和成树国换上了白色的披风,一路潜了进去。

小二楼是那种很老式的木制建筑,有点类似中国的“筒子楼”,只有一个入口,开在楼的正中间而不是两侧,据说是一战前修建的,没毁于两次战火,倒也算得异事。

已经凌晨两点多钟,“波兰复兴组织”的人早都睡了,连站岗的人都没有。

这倒不怪波兰人托大,事实上,他们也知道同胞和法国人干起来了,这十七个人就是从波兰赶来救火的,当然,为了尽快进入状态,有一多半是上次曾经来过的。

工人党被整得鼻青脸肿的样子,也全落进了波兰人的眼里,这更使他们相信,同自己这方准军事化训练出来的精英相比,法国的这些个小流氓实在是过于垃圾了。

波兰人是来一绝后患的,所以他们也不可能坐视盟友的狼狈,可这个盟友居然始终搞不到对方的下落,使得波兰人有力无处使。于是,他们对工人党又多了一层鄙视。

波兰人始终在纳闷,为什么没人找他们的麻烦,他们毕竟也是上次事件的参与者,而且还是制造惨案的主犯。

杀人的没事,卖刀的死了,没理由啊。

所以,一直以来,波兰人都没有放松警惕,枕戈待旦,严防偷袭。

不过,这世界只有做贼千日,哪里有防贼千日的?长期的平安使波兰人放松了警惕,今天又是平安夜,一阵狂欢过后,醉醺醺的波兰人回到住地,倒头就睡,没人记得该警戒了。

门上的大锁很容易地被楚云飞打开了,小楼门口,两边的房间都算是传达室或者保卫室,里面各睡着两个人,呼噜打得震天天响。

下面的情节又是简单的重复,楚云飞和成树国挨个点名就是了。不过,为了完美地结束这次偷袭,他们杀了俩人,用“嘎都因”木棍麻醉了俩人。

然后就是挨个房间的点名,不过,才点了三个房间五个人,楚云飞和成树国就受不了啦。

原因无他,醉酒的波兰人房间里的味道太难闻了。

有酒味、劣质雪茄味、汗臭味、狐臭味、脚臭味,房间门口偶尔还有尿骚味,偏偏还有个别人使用味道极其刺鼻的劣质香水,这几种难闻的味道混合在一起,怕是重感冒鼻腔堵塞的人都受不了。

有的房间,空气中还有很怪异的一种味道,有点像烟味,却还有种说不出的头晕感觉,后来楚云飞才知道,那是大麻味。

两人对视一眼,黑暗中虽然看不清楚彼此的表情,但都知道对方的意思:算了,受不了啦,撤吧,外面空气多新鲜。

反正还有刚买的火焰喷射器呢,既然带不出法国,为什么不用?

于是,拖着两个动弹不得的俘虏,两人施施然走出了小楼。

\section{第一百七十四章 血腥平安夜}

楚云飞和成树国走出来的时候,多尼已经把车开到了小楼前。

把两个俘虏塞进车里,刘宁和楚云飞一人一把FAMAS步枪,装上枪榴弹就向小楼发射了出去。

成树国早把装手雷的帆布袋子拽了过来,左右开弓地疯狂扔起来。

别说,这家伙手雷扔得确实准,用左手扔的手雷都能准确地砸到窗户上,砸碎玻璃后在屋里爆炸。

多尼早把两个土制火焰喷射器搬出了车,扣动扳机,熊熊的火焰瞬间就封锁了楼门。

楚云飞和刘宁一人发射了两枚榴弹就不再发射了,刘宁持枪点杀了一个试图从门口闯出的裸体汉子。

楚云飞则是放下步枪,抱着另一个火焰喷射器到处跑,像个喷漆工人一样,把整栋楼从上到下喷了一遍,没液了!

还好多尼已经不用喷射器了,他拿着楚云飞丢下的步枪,和刘宁一起射杀那些试图从楼门或者窗户冲出来的波兰人。

平安夜里,火光冲天!

对这时的波兰人而言,时间过得实在是太慢了,持续了有五分钟左右,几个偷袭者收工走人。

波兰人附近也有工人党的人,他们是最早到达现场的,前几个人居然还看到了偷袭者的身影,于是就有人持手枪向他们射击。

这倒不怪他们不听戴维斯的命令,实在是:出卖盟友的事情根本就不合适声张,所以戴维斯绝对不可能跟他们打招呼,让他们视而不见。

对这种可以牺牲的小毛虫们,楚云飞他们自然也不会手软,他和刘宁的两支步枪组成交叉火力,横扫了过去,掩护着汽车离去。

雪地里,车不可能开得很快,警察们虽然由于节日的原因姗姗来迟,还是有辆警车发现了多尼这辆车有点不对劲,跟了上来,并发出“停靠,检查!”的呼叫。

看着紧跟的警车,楚云飞二话不说,抱起了火焰喷射器,扣动了扳机。

可惜,这武器是土制的,汽车开得不快可也不算慢,这次居然没打着火,只是一串液体喷了出去,甩到了警车的前挡风玻璃上。

液体极粘,警车不由得停下来,看看对方喷了什么东西出来,警察们一下车,就闻到了浓烈的汽油味,几个警察交换一下目光,马上有人拿起了对讲机,“呼叫总部,呼叫总部…………”

大过节的,谁愿意找死谁去,大家固守待援吧。

这次的事情玩得比较大,大家也不回那暂住地了,找了个远离马赛的废弃工厂钻了进去。

等他们找到地方的时候,那俩俘虏也基本上能动了,起码嘴是能动了。

两人一个是参与过对付脱特斯基家族行动的,一个是初次来法国。

参与过行动的家伙交待得倒是很痛快,一副敢作敢当的样子,事实上他一直昏迷着,都不知道刚才发生了什么事,气焰还很嚣张呢。

另一个却是不肯回答问题,一直好象在低声骂人,气得多尼直在那里搓手。

楚云飞听不懂他们的波兰语,只是淡淡地对多尼说,“告诉他,我们不杀他。”

不杀绝对是不可能的,多尼以为楚云飞在骗人,点点头示意知道了,无非是换种问讯技巧而已。

多尼随便又说了几句,又跑回车上拿了钱包下来给那人看,那人就痛快地交代了。

俩俘虏的口供证明,希伯伦说的都是真话,后面那个俘虏甚至交代了事情的最新进展:托尼成为族长后才发现,西欧的钱实在是太好挣了,于是就想收回原来的承诺,继续在法国发展。

“波兰复兴组织”肯定又要郁闷了。

听完这些,多尼什么话也没说,绷着脸从车上取下手枪,抵在那个参与过行动的人头上,面无表情地扣动了扳机。

颅腔中喷射出的鲜血溅了多尼一头一身,他就像看不到一样,走到了另一俘虏面前举起了枪。

楚云飞不得不说话了,“等等,这个不能杀。”

多尼后杀这个,自然是因为这人似乎身份又高点,不过楚云飞的话却是另出机杼。

“多尼,你要是为了泄愤杀他,我是不会阻止你的,但我建议你考虑一下自身的处境。”

“你怕那个组织引发民族情绪来追杀你,他们又何尝不怕你暴露他们残杀同胞的事实呢?那样他们的形象就完了。让这个人,把这话带回去吧,以后,大家彼此不相干就算了。”

楚云飞这话,绝对是为了一劳永逸地解决多尼的安全问题,纯粹是从利益的角度出发的。

多尼考虑半天,默默地点点头。

“楚,你真的是个好人,总是尽可能地为朋友考虑”,听得出来,多尼这马屁,实在是语出至诚。

那俘虏意外地来了一句,说的却是英语,“谢谢,中国人,你的话,我会带回去的,不过我不敢保证他们会答应。”

成树国上去照脸就是一脚,不过力道不算大,“我操,你听得懂啊?”

俘虏倒是振振有辞,“你以为,有那几张钞票我就会相信了么?我是听到你们不打算杀我,才说出来我知道的事实的。”

…………

接下来,就是找托尼和他的弟弟了,情报上说,他俩在马赛。

按着班克斯提供的地方,几人来了次偷袭,很可惜,那两位回国过圣诞去了。

不过,躲得过初一,终究是躲不过十五的,这两人终于在三天后被楚云飞他们擒获。

对于怎么杀死他们,多尼很是想了一阵,恪于环境所限,众多歹毒手段无法使用,最后才决定,用火刑烧死算了。

成树国提出个主意,就是把人埋进土里只露脑袋出来,头上开个口子,向里面灌水银,水银比重很重,顺着皮肤自然下走,最后会把整个皮剥掉,既不血腥,又够残忍。

传说中,剥了皮的血人还能从土里面钻出来跑呢。楚云飞对这点表示谨慎的怀疑。

于是,多尼买了五十磅水银来操作这个方案,不过,传言当真是信不得的,他手忙脚乱地弄了半天,总是不得要领,可托尼整个身子被土埋着,血向上涌,头部失血过多,眼看着已经不行了。

一气之下,多尼提出了火焰喷射器,把两人的头部烧成了黑碳。

\section{第一百七十五章 再见狂龙}

多尼的事情就算处理完了,他要移民到美国了。

楚云飞他们得了二百万美圆,再加上刚贝拉给的十万和维伦斯家族给的六十三万英镑,抛开已经花掉的,三人已经拥有了三百多万美圆。

大家商量了一下,每人拿了六十万美圆,剩下一百二十万作为公共资产。

楚云飞汇了十万美圆给叶美,没敢多汇,“这是国家给的,特殊任务的补贴”,有了这笔钱,想来母亲的生活可以宽松很多吧?

成树国和刘宁可没往家里汇钱,他俩家里条件都不错,汇少点没必要,汇多了,别再惹出什么是非吧?

办完这些事,大家也没有在法国逗留的心思,毕竟才在这里大干了一票,为保险起见,还是老实点去英国吧。

回到英国后,楚云飞他们找了个地方住下,班克斯还在美国,大家好好放松了两天。

楚云飞终于联系上了周琳琳,原来,周琳琳的家人根本就没把他的电话号码告诉她。

好久没有联系了,周琳琳在听到楚云飞声音的时候,立刻哽咽了起来。

一个电话花掉了楚云飞将近一百英镑,手机热得能烤熟鸡蛋了,最后还是天太晚了,周琳琳实在困得撑不住的时候才挂断的。

楚云飞自然要问问她为什么这么长时间不联系自己,周琳琳说那时她忙着学生会的活动和去医院实习的事,等忙完之后,已经联系不上他了。

楚云飞正在这里回味电话,刘宁推门进来了,“云飞,狂龙来电话了,说是有事找我们。”

狂龙?这个家伙会有什么事,是惹人了么?

楚云飞本想问问刘宁,是否告诉狂龙大家已经回来了,不过再一想,算了,还是去一趟吧,毕竟那也是条汉子呢。

几人见面,是在一家蜀山风味的饭店,在唐人街的中心位置。

饭店二楼全是包间,狂龙是带着一个女人来的,那女人三十多岁,微微有些发福,不过,看得出来,年轻时肯定也算得上是个美人。

一个多月没见,狂龙似乎憔悴了一点点,起码神色不是很轻松。

“这是我老婆,费楠,上次没机会向诸位引见,这次带她来,认识下诸位兄弟,呵呵。”

刘宁做事比较讲究,又是狂龙的老乡,伸手从包里拿出来五百英镑,递给了费楠,“嫂子,这是我一点心意,带给我那些没见过面的侄子。”

女人才待推脱不要,成树国和楚云飞也被挤兑得拿出了钱,操,刘宁这家伙真黑,一给就是这么多。

狂龙是出来混的,知道这种场合不能太计较,随意推脱了一下,就让他老婆收了起来。

客套过后,点了菜,大家开始说正事。

“这次找哥几个来,有两件事,一件就是,中国大使馆找到我了,要我提供你们哥几个的情报,不知道是谁点的炮,妈的,我觉得十有八九是上次那个贱女人。”

中国大使馆?三个人交换一下目光,还是刘宁发问了,“他们说找我们什么事了没有?”

狂龙摇摇头,“没有,他们只想知道你们现在的情况,我操,拽得很呢,还好咱是英国公民,不吃他那套他也没辙。”

“然后呢?”这次是成树国发问,大使馆要出面,那肯定是有事了,不过基本上可以肯定不是什么好事,他隔一阵就给家里打个电话,没听说有什么好消息。

“然后?”狂龙似笑非笑,“然后他就要我转告你们,别以为你们杀了那么多人,没人知道,识相的话,还是规矩点好,听从祖国的召唤吧,我呸!”

“我也呸,”说话的是成树国,“什么玩意儿,那些被杀了人的亲属都没什么意见,你们跑来多什么的事?”

其实,事实和狂龙说的稍有出入,那来的小伙子说话是不怎么客气,但还不至于那么不友好,最多就是公事公办的样子。

但搁在做惯老大的狂龙眼里,那就跟挑衅没什么区别,所以狂龙最后的话就是,“我告诉那家伙了,我联系不上你们,还有,我是英国公民,没有配合他们调查的义务,希望他们以后不要来骚扰我。”

“他们要跟纯正的英国人打听消息,绝对不是那种态度,我呸,什么东西!”狂龙这话有些偏激,但楚云飞他们认为,这绝对是事实。

有些国人,对着同胞的时候,会习惯性的摆摆威风,尤其对着像狂龙这种他们认为上不得台面的人。

“算了,”楚云飞说话了,他不想再提那些事,“不说这个了,反正你最后也没联系到我们,说说另一件事吧。”

另一件事就很扎手了,而且这事还是跟楚云飞他们有关。

原来,“画框事件”中的那块金表,被一个华人买到了,此人早已淡出这个圈子,属于成功人士了。

无巧不巧,这金表最先是被“飞龙帮”一个小混混买来的,后来转卖出去的。

像这种无主赃物的街头交易,飞龙帮每年不知道要做多少。

于是,那表被人发现后,纵横伦敦的格瑞尔家族就找上门来,要飞龙帮交出木棍。

这就是身在底层的悲哀了,那混混并不知道这表背后签名的意思,依照惯例,一口咬定是自己收藏了多年的东西,而没解释自己其实不记得那卖表人的相貌了。

于是这事自然就弄大了,格瑞尔家族正好与唐人街的另一个帮派有点小小的交情,这次放出风来,一定要扫平飞龙帮。

对于这事的来龙去脉,楚云飞知道得一清二楚,而狂龙在这件事上又没做错什么,无非就是收购了赃物而已。

大家都要讨生活的,这种无关大雅的小错误,实在不能成为家破人亡的理由。

刘宁也听楚云飞讲过这事,率先表态了,“没什么,格瑞尔家族是吧?这事交给我们了,还是那话,你们别犯什么原则性错误,那都好说。”

楚云飞也同意刘宁的观点,这事牵扯到了维伦斯家族,到时候让班克斯出面好了,就算格瑞尔家族不买帐,在伦敦有维伦斯家族的支持,还怕他们不成?

狂龙有点不相信自己的耳朵,“喂,老乡,你听清楚了没有,我说的是格瑞尔家族,和你想的是不是一家呀?”

楚云飞看他一眼,“我知道,在金融街混的那家,是吧?”

不过,“我们帮你摆平格瑞尔家族,可对另一个华人帮派,我们不建议你们火拼。”

狂龙苦笑一声,“我不火拼人家,人家还要收拾我呢,我倒是得有选择的权力呢。”

大家对视一下,成树国发话了,“我操,老龙,怎么每次见面我都让你搞得这么郁闷呢?”

\section{第一百七十六章 人不在沙特}

等到新年的时候,班克斯从美国回来了。

原来,班克斯的弟弟布兰克常年在美国,这次圣诞,班克斯过去替弟弟看着摊子,让布兰克回来休息几天。

见到楚云飞,班克斯先伸出双手紧紧地拥抱了他一下,“哈哈,楚,你在法国干得太漂亮了,把那里弄了个天翻地覆啊。”

楚云飞笑笑,“呵呵,那不算什么,和那些小流氓打交道,我们怎么可能吃亏?对了,班克斯,关于‘基天’的事,你考虑得怎么样了?”

班克斯笑了,那笑容像风中的蜡烛一般捉摸不定,“没问题,家族已经同意了,不过,我的弟弟布兰克提了一个要求。”

又是一个要求,成树国有点受不了啦,“这个,说来听听吧,怎么最近我总碰不到实在人呢?”

班克斯倒没计较,自己家确实有点趁火打劫的味道,他的本意是,同楚云飞他们搞好关系的话,将来提什么要求还不是很简单的?

“布兰克说了,他很为你们在法国做的事叫好,希望,在将来,维伦斯家遇到什么麻烦的时候,你们也能出手帮忙,我们只是需要你们一个承诺。”

布兰克确实为这个动心了,没别的原因,只是因为这几个人超强的战斗力。

楚云飞懒洋洋地点点头,“没问题,只要价钱合适。”

班克斯有点恨自己的弟弟了,看你出这主意出的,我谈钱的时候人家谈感情,你这么一鼓动,我要谈感情的时候,人家谈钱啦。

要楚云飞他们帮忙,班克斯肯定是要出钱的,只是话说到这里,双方都显示了太强的功利性,实在不利于友情的成长,不过,这也不算什么坏事吧?

看到班克斯点头,刘宁为老乡提出了请求,“班克斯,你还记得做画框的戴维么?”

接着,刘宁就把飞龙帮的事同班克斯讲了一遍。

班克斯听罢点点头,“哦,这个没问题,小事情,木棍的事,大家以后就不要提了,谁知道幕后站的是什么人?多一事不如少一事。”

“这点面子格瑞尔家族还是要给我们的,事实上,他们的生意能进入纽约,我们帮了不少的忙。你们让那个买了赃物的家伙去跟格瑞尔家说清楚就行了。”

楚云飞点点头,这件事情,这样解决才是最合情理的,不过他没忘记正经事,“班克斯,现在,你能不能把我要的资料给我呢?”

班克斯从书桌里抽出一张小纸,上面只有一个人名和电话,“这个你收好,他和我们有太多的共同利益,而且他在穆斯林里很有威望,去沙特找他吧,那是个值得信赖的人。”

楚云飞接过纸条,看也不看,装进了裤子口袋,“谢谢,班克斯先生,非常感谢。”

看到楚云飞他们要走,班克斯沉吟半晌,终于没说什么挽留的话,“楚,你一定要小心,我希望很快能看到你们三个回来。”

三人齐齐停下脚步,回头看看班克斯,动作整齐划一地点点头。

伦敦上空,再次阴云密布,像是在为三个勇者送行。

风萧萧兮泰水寒。

在三人走之前,狂龙来拜会了三人,除了感谢的话,还表达了别的意思。

他居然想跟着三人一起去闯荡,因为,他再次意识到了老乡和他两个朋友的能量。

原来,班克斯打了个电话给格瑞尔家族的老大,那老大正在为如何偿还班克斯家族的人情而苦恼呢,听到原来是这事,没口子地答应了。

虽然托格瑞尔家族调查木棍的也是一个很有权势的主,但能量和维伦斯家族相比,也仅仅是旗鼓相当,只是大家覆盖的领域不同就是了。

维伦斯家族在国外的势力和名声都比格瑞尔家强出不少,而在英国国内又是出了名的低调,从不惹人,当然也没人不开眼到去招惹他们。

进入纽约,格瑞尔家族是得了维伦斯家族的大力支持的,虽然维伦斯家族绝对不会帮他们走到自己家的那个高度,凭空给自己树立竞争对手。

这个人情可真的是不小,而维伦斯家又没有求他们的地方,虽然在纽约组成了松散的同盟,但有机会的话,格瑞尔家也是要表示一下自己的善意的。

另一方面,格瑞尔家的人虽然受人所托,在调查失物,但他们死活不相信,世界上居然会有神迹存在:一根木棒,能延长人的寿命,难道是靠打的么?

所以,当格瑞尔家的人出现在狂龙面前时,一改以前高高在上的样子,热情得让狂龙吃惊。

狂龙立刻把那小混混喊了来,要他实在交代那金表到底是怎么来的,狂龙知道:面子是别人给的,可绝对是自己丢的。

于是那小混混又一次地在老大面前强调了实情,这次,格瑞尔家族的人马上就相信了。

临走,格瑞尔家族的人还“善意”地透露:作为在伦敦首屈一指的帮派,他们不希望有任何势力试图改动伦敦地下世界的结构。

这就相当于为“飞龙帮”提供了保护,当然,如果飞龙帮想单方面改变现状,那没准也是有得商量的。

总之,狂龙忽然间发现,楚云飞他们实在没有算计他的可能,因为,大家根本不是在同一个档次上玩。

既然对方只是看上了自己这身功夫,那为什么不趁着还打得动的时候博点什么东西呢?

刘宁实在觉得好笑,挣钱的时候,喊你你死活不答应,现在要去卖命了,你倒要贴上来了?

最后,他还是实话实说劝走了余化龙,“老乡,不瞒你说,这趟买卖实在太危险,又没啥利润,下次吧,只要我们哥几个在,啥买卖不好说?”

三天后,三人登上了飞往利雅得的飞机。

到了利雅得,找到了人,三人才发现,居然来错了地方!

原来,那马哈苏德做了那档子事后,又回到了他的老巢:巴基斯坦!

原因很简单,丫本来就不是沙特人,估计是“基天”组织内部调动,才把他弄来,没想到他来了没几天,就做出了这种天怨人怒的事。

这事要是沙特人做的,那都说得上过分了点,何况是这么个外人。

“基天”内部还好说一点,有些连“基天”都不能容忍的穆斯林却实在是受不了啦:这明显是在祸害沙特嘛。

又由于这次行动引发了政府对“基天”组织的打击,马哈苏德更是台风的风眼。

重重压力之下,马哈苏德不得不乖乖地回了老家。

\section{第一百七十七章 司机是侃爷}

听到这迟来了好多年的消息,楚云飞当时就是一懵,这么来说,得转战巴基斯坦了?

一直以来,楚云飞都是在为去沙特做准备,各种信息收集得足足的,资料也准备得非常齐全,他的包里放着几本沙特的地图和风土人情说明,在最危险的时候他都没有舍得丢弃。

一切……都要从头开始了。

巴基斯坦,无论从民族构成、宗教信仰、利益集团上,还是从地方势力、部落武装、风土人情上讲,都比沙特复杂得太多太多了。

那“异常可靠”的古维帕帕看到几个中国人的失望之情,微微一笑,“你们也不用这么担心,事实上,巴基斯坦痛恨‘基天’的人要比沙特的多得多,我那里也有可靠的朋友,有的人能量可是比我大多了。”

既然如此,那大家就没什么可说的了,三个人立刻改道飞往巴基斯坦。

那些名牌衣服还都存在伦敦,事实上,那里算得上几人比较放心的地方了,当然,如果没有露丝就更完美了。

从赤道附近去那里,御寒的衣服还是要买些的,妙的是,沙特那里卖的多是穆斯林的衣服,三人也算是提前化装了。

“伊斯兰堡,我不喜欢这个名字,”三人刚踏足巴基斯坦,刘宁就皱着眉头来了这么一句,“刚才飞机上人太多,不好意思这么说,嘿嘿。”

“我倒觉得不错,”说话的是楚云飞,“起码能带这么多美圆过境,我是比较满意的。”

听说伊斯兰堡的花旗银行分行被炸,楚云飞提了十万美圆的私房钱出来带上了飞机。怕到时候没钱买武器。

“我知道刘宁为什么这么想”,成树国笑嘻嘻地说,“伊斯兰堡,那就是说不欢迎不信仰伊斯兰教的人,刘宁信仰的,可是共产主义。”

三个人就这么插科打诨地走出了机场,伸手拦了一辆出租车。

司机是个很健谈的小伙子,先热情地同三人打招呼,用的是半生不熟的英语,“你们好,是中国人还是日本人?”

正经事重要,而且这句猜测里带上了中国人,还排在日本人前面,所以成树国没有发飙,只是微微地皱了皱鼻子。

楚云飞从来就没有正规地说过乌尔都语,不过,他的语言天赋可真不是一般地强,依着印象努力地挖掘记忆,“我们是中国人,对了,你这车……跑长途么?”

听到是中国人,司机变得更加地热情,“中国人,中国好啊,是我们巴基斯坦人的真正朋友,比那些日本人强多了。”

这是成树国和刘宁听到的最后一句英语,接下来,那司机也反应了过来,人家是懂乌尔都语的,于是用母语滔滔不绝地发起言来。

他谈是很友善的内容,诸如对中国的喜爱,中国长期对巴基斯坦的支持等等,最后小伙子还来了一句“有钱了一定要去中国看看”的感慨。

看来,每个国家首都的的士司机,都有做侃爷的天赋。

听得楚云飞非常怀疑出租司机对伊斯兰教的忠诚度,有钱你怎么也得先去麦加吧?不过,话显然不能这么说。

“我和我的同伴也非常喜欢巴基斯坦人,”当然,马哈苏德除外,“我衷心地希望,两国人民能世代友好下去。”

那司机嘴一咧,才待要说什么,楚云飞赶紧制止了他的发言,“对了,你这车跑不跑长途?”

司机扭捏一下,不过也不见脸红,“虽然我不跑长途,但你们总要进了市里才能找到长途车啊,你们要去哪里?”

楚云飞直接说出了自己的目的地,“俾鲁弯省!”

那司机听得就是一个哆嗦,他手一抖,车身都为之一晃,“天啊,你们要去那里?”

楚云飞点点头,“对,有什么问题么?”

那司机明显地紧张起来,“我建议,如果没有必须去的理由,你们还是不去的好。”

然后,司机又开始了他的演说,不过,这次楚云飞小心地听着,没有漏掉一个字。

原来,那俾鲁弯省的地方势力非常地强大,巴基斯坦政府在那里政令基本上是行不通的,算是个天高皇帝远的地段。

出于历史原因和地域结构,那里的民族和宗教信仰非常杂乱,各种地方势力恩怨交织、形式错综复杂,哪怕是同一个部族,都会分为不同的利益集团。

不过,大致还是由十几个大的部族和上百个小部族来分片控制的,当然,哪里都有产生意外的可能,每个部族不同阶段也是由不同利益集团的势力来控制的。

总之,没事去那里的话,安全一点也得不到保障。

楚云飞点点头,事情果然如同他了解的一般,这里是比沙特复杂得多了,但这动摇不了他的决心。

“那么,我们怎么才能到达那里?”

这司机真的算是热心人,也许同中国人在这里的口碑有关吧,“这个,让我想想……哦,对了,我认识一个朋友,他是做黑市生意的,也许,他能把你们送过去。”

楚云飞后来才知道,从某种角度上来说巴基斯坦政府是相当腐败的,也许是跟执政多年的军政府有关,只要能在军队里找到相当地位的靠山,做黑市生意是没人管的,这是公开的秘密。

所以司机根本就不怕他们知道这些,大明大方地把他们带到了自己朋友家,当然,他也得到了重谢,楚云飞他们的出手还是算阔绰的。

黑市商人叫阿克塞姆,三十多岁,个子很高,将近一米九,身材也非常魁梧,很有种彪悍的味道。

他上下打量楚云飞三人几眼,“去俾鲁弯,价钱很贵的,三个人一起,便宜算的话两千,贵的四千,要美圆。”

这时只有楚云飞能答话了,“两千和四千,区别在哪里?”

阿克塞姆看到对方不讲价,就有了点兴趣,“两千是组个商队走,不但要等机会,到时候可能还会有人检查;四千是专门送一次,路上没人检查,当然,两种价格都保证你们的生命安全。”

楚云飞点点头,“如果我们再带两辆车去,需要多加多少钱?”

“带车不行,”阿克塞姆一口就拒绝了,“你们是记者么?”

\section{第一百七十八章 黑市商人}

俾鲁弯是不欢迎记者的,尤其是外国记者,因为,里面很多事情都太复杂了,如果一个记者想公正地发些新闻,没有几年的功夫想都不用想。

一件鲜血淋漓的惨案背后,可能是更残忍的部族恩怨,这种部族恩怨,又可能涉及一些历史因素,当然,也不乏大国博弈的痕迹。

就是两个字:复杂!

所以,俾鲁弯,是不欢迎记者的,在那片土地生存,就要按照那片土地上的规矩做事,既然记者未必能做到客观公正,那就只会给他们带去麻烦,而且,那里见不得光的事情也实在多了点。

虽然楚云飞不知道内情,但他一直在注意马哈苏德的家乡,这么多年,得到的信息却是少得可怜,所以,他当然不会去冒充记者这个貌似牛逼的职业。

“我们不是记者。”楚云飞郑重声明,“我只想买些东西带过去。”

阿克塞姆显然意识到了楚云飞的油水,不过他也知道其中的风险,所以,该规避的风险一定是要规避的。

“如果,你能证明你不是记者,我想,我可以帮得上你的,只要你有足够的钱。”

证明?楚云飞越发清楚了俾鲁弯省不被外界了解的原因,不过该怎么证明呢?

他向室内扫视一番,却怎么也找不出合适的东西来,还好,最后终于被他发现了一个空的酒瓶子。

就是它吧,楚云飞走上前,一把攥住了那瓶子颈部,就在那一瞬间,楚云飞觉得自己像极了那些走江湖耍把势的。

“啵”的一声,那瓶子自颈部断裂了开来,等到楚云飞张开手掌,手里的瓶颈部分已经变成了一堆白色的粉末。

“这样,可以证明么?”

阿克塞姆看着不由哆嗦了一下,中国功夫果然是名不虚传,幸亏自己江湖跑得够多,没有怠慢了客人。

不过,中国,那是功夫的故乡,会功夫的人该是很多的吧?

要是别的国家的人,阿克塞姆就凭这一点绝对就可以断定对方不是记者,原因无它,有这手功夫的人去当记者实在太可惜了。

不过,要是中国人的话,那可真难说了。

看着阿克塞姆期期艾艾的样子,楚云飞冲着成树国使了个眼色。

不愧是配合这么多年的兄弟,看到楚云飞示威,成树国虽然听不懂他俩在说什么,可他的钢钉立刻出手!

于是,阿克塞姆珍爱的菲律宾鸡冠鹦鹉从架子上一头载了下来,一条链子拴住了它的腿,鹦鹉的身子在架子上一晃一晃的,还伴随着濒临死亡的无奈的抽搐。

阿克塞姆的脸色登时就是一变,刚要说什么,楚云飞已经拉开了自己的背包。

灯光下,一叠叠整齐码放的绿色钞票马上眩晕了主人的双眼。

“对不起,我兄弟脾气不太好,似乎误会了你的意思,我愿意为他的卤莽道歉,并做出赔偿,请你说个数字出来好了。”

阿克塞姆并不算笨蛋,他马上就明白了:这帮人绝对不会是记者的,他们……确实做出了证明。

不过,横财在眼前,不捞一笔也说不过去,“这鸡冠鹦鹉……是我花五千美金买的,非常纯正的品种。”

楚云飞一把拉上拉链,一挥手,扭头就走。

他没办法多说,说英语对方都未必听得懂,更别说说汉语了,只希望自己的手势能让对方明白吧。

阿克塞姆马上就明白对方这是嫌自己狮子大张嘴了,但他已经见识了对方的武力,四周没人的情况下,他可不想用强。

“喂,中国兄弟,听我说……虽然这鹦鹉比较值钱,但是,恰好我的妻子不太喜欢,你们中国人不是讲究‘女士优先’的么?”

拜托,那是欧洲人比较讲究这个!

“所以,我认为,你给我一千美圆就可以了,当然,大家都是兄弟,有什么话都可以商量的。”

楚云飞懒得跟他计较,虽然这数字依旧比较离谱,但他们现在并不具备“货比三家”的能力,而且,刚才这家伙说的套路还很像那么回事的。

点点楚云飞递来的一千美圆,阿克塞姆笑嘻嘻地金钱落袋,“这位兄弟,不知道你们还想买点什么东西捎过去?”

楚云飞摇摇头,“我怕你弄不到,你只帮我们找两辆吉普车,二手的就行。”

两辆二手车,那才多少钱?有一万美圆就够了,不超过一万五的,就算能卖到三万,才能挣多少钱?

阿克塞姆已经被那一包钞票晃晕了眼,态度是空前地热情,“那不一定,这伊斯兰堡里,我弄不到的东西还真不多,哪怕你们要毒刺导弹,黑鹰直升飞机我也有办法。”

这话自然是在吹牛,不过,阿克塞姆作为个专业的黑市商人,虽然是小打小闹的那种,可真还知道不少这种武器贩子,大家都是靠着军队玩的嘛。

楚云飞盯着阿克塞姆看了半天,直看得对方心里发毛,才没头没脑地来了一句,“其实你知道,你那只鸟不值一千。”

不值一千,那里为什么给了我一千?阿克塞姆被这题外话直接岔到了云端里头,思考了半天才反应过来。

不值一千给了一千,那就是说对方对钱是不怎么在乎的,只要值得花,他们是会出钱的。

另一方面就是,对方是知道行情的,起码知道大概行情,也就是说,自己只能挣该挣的钱,多一点也许没关系,只要自己能拿出对方想要的东西。但是要价过于离谱的话,那就是自己把财神爷向外推。

想明白了这点,阿克塞姆又变得热情了起来,人家既然没走,不是还想跟自己做生意么?

“这个,反正不值一千也差不了多少,”阿克塞姆并不掩饰自己的贪婪,但也表示了自己的诚意,“不知道你们想买什么东西,我有很多朋友,做什么生意的都有。”

“事实上,我们只想同你做买卖,”楚云飞的脑瓜可不是白给的,他做出一副难以决断的样子,“毕竟,我们的身份不太合适出去乱撞,还是只认你比较好点。”

这话鬼才会信,你们不过就是喊了个的士而已,刚才都要走人了呢,阿克塞姆再笨也明白这话是客套话,事实上,他觉得自己比大多数的同胞要聪明些。

\section{第一百七十九章 抵达俾鲁弯}

这么一来,中国人的意思就很明白了,钱让自己挣,不过份即可,但同时还要自己相对保密些,“这个,你们放心好了,我会尽量帮你们讲价钱的。呃,对了,我只是生意人,从不参与任何政治的、民族的或者宗教的纠纷。”

“那样就好,”楚云飞点点头,这世界果然哪里都一样,做黑道生意的,没有笨蛋,“你能搞到什么样的武器?单兵武器,威力越大越好,对了,还要有防弹衣。”

阿克塞姆听得就是眉头一皱,从楚云飞捏碎酒瓶子开始,他就隐约觉得对方是去惹事的,否则,不需要这么强悍的武力出场的。

不过,阿克塞姆还真的不怕这个,俾鲁弯已经够乱的了,多三个中国人去,也不过就是多了一撮炮灰而已,打打杀杀的情景在那里是家常便饭。只要不是记者,他就不会有任何的责任的。

但武器生意阿克塞姆自己是不做的,只有出去找那些相熟的军火贩子去了,希望能找到什么威力和利润都大的东西吧。

阿克塞姆很想问问对方到底有多少钱,以便自己订货时好做选择,不过,手里没货时,探底总不是表现善意的方式,所以只能淡淡地提醒一下强悍的客人,“我希望,你们能买得起我介绍的武器。”

楚云飞却显得比对方还有经验,他一向认为,黑道就是钱和货的赤裸裸交易,亮底是必须的,反正他也不怕对方能怎么样自己,巴基斯坦的治安怎么也该比索度强吧?

“四十万美圆,这是上限,我不想买太多没用的东西。”

阿克塞姆听得哆嗦了一下,终究吧嗒吧嗒嘴巴,掀起门帘,顶着刺骨的寒风出去了,只丢下一句话,“你们在这里等我。”

听了楚云飞的翻译,刘宁马上抱怨了起来,“云飞,你怎么能这样呢?把底子都兜了出来,很容易惹人起歹心的。”

他只有五十万美圆,拿出四十万,那算得上是倾囊而出了。

楚云飞心里长叹一声,可还没办法解释,他站起身来,走出门外,淡淡地说一句,“弟兄们,大家上点心,别叫人咬了。”

他总不能说:你们都是我的好兄弟,为了你们的安全,我再多花点也认了。

兄弟,那是要用心来交的,说得太多,反而见外了。

阿克塞姆真还有点强夺的想法,不过冷静下来想想:自己这名头来得也不容易,还是珍惜点的好。

再说,那包里绝对不会有四十万美金,这点他是敢肯定的,可要制服那三个一脸杀气的人,还要强迫人家去取钱,那……还是省省吧。

楚云飞他们等了大概一个多小时,阿克塞姆回来,他只带回来了三枝美制M-16和四枝俄制AK-74和一枝AK-47,还有一千发子弹。

“现在只能搞到这些,”阿克塞姆神情自若地解释,“防弹衣要明天才能弄到手,这里还有张武器的单子,你们看要点什么,人家见钱才肯发货的。”

言下之意,自然是这种好找的武器可以为他们随便垫垫资,价钱贵的可就没兴趣继续垫了。

楚云飞拿过来一看,第一行就把他弄晕了,“毒刺导弹,200,000$”,晕,真还有毒刺导弹卖的?

再向下看,里面的品种还真的不少,反坦克导弹、火箭、高射机枪、迫击炮等等应有尽有。

这些东西真的有人卖?楚云飞觉得有必要搞清楚货物的来源。

“东西是不少,不过,我觉得,这些东西巴基斯坦的军队也未必能装备得起来吧?”

阿克塞姆早料到了楚云飞有这么一问,伸出手指向西北方指指,“东西都是从那里过来的。”

哦,原来都是从战火纷飞的阿富汗走私来的,难怪这么驳杂。

楚云飞又想了想,提出一个让阿克塞姆郁闷不已的问题,“那照你这么说,俾鲁弯省那里也有这种东西卖吧?那里正经是毗邻着阿富汗的。”

阿克塞姆一犹豫,楚云飞已经明白他的意思了,直接从包里抽了两千美圆出来甩给他,“我不会让你白忙的,再说,从这里把武器运过去,路上要担不少风险吧?”

阿克塞姆马上眉开眼笑起来,“是的,其实,我觉得,从那里买武器似乎会更便宜些,那里局势现在又不算太紧张。”

阿克塞姆想想办法绕绕圈子的话,跟那边的军火贩子也能联系上的,前提是:只要主顾够大方。

“好了,那就没什么问题了,”楚云飞扬扬手中的纸片,“这个我看看,后天我联系你,希望那时候大家能出发,对了,吉普车的车况一定要好。”

多出来的这两天,楚云飞是要提钱的,但这话是不能跟阿克塞姆解释的。

…………

巴基斯坦的路况真的是太糟糕了,楚云飞他们从伊斯兰堡到俾鲁弯足足走了三天。

两辆吉普车,一辆是北京吉普,一辆是三菱吉普,连交通费用,楚云飞一共给了阿克塞姆三万四千美圆。

两辆吉普车都挂的是首都的牌子,各有一个向导带路,还带了不少各式各样的小旗子。

每到关卡,都是向导出面招呼,其中不乏偷塞红包的举动,不过,这些都跟楚云飞他们无关。

这些都不是重点,最诡异的是,越向西走,旗子就越频繁地被拿出来插到车上,很多时候,根本就是在四下无人的山谷或者旷野,向导也要下车,把一些钱物放到石头或者树木上,同时换插小旗。

到了正规关卡,那些旗子又要被收起来,讲究实在是太多了。

楚云飞后来才知道,其实那些小旗子很多时候是亮给自己看的,巴基斯坦并没有那么混乱,最可能带来麻烦的还是那些正规的关卡。

当然,该规避的风险也还是有一些的,尤其是进了俾鲁弯省,各种零散的势力和小股武装还是很多的。

楚云飞要找的人,是俾鲁弯西部克努蒂部落的一个长老,贾德坦,此人年近六十,在部落里也很有威望,是长老会里为数不多去过麦加朝圣的长老之一。

看到古维帕帕的书信,贾德坦非常热情地接待了楚云飞三人,同时也把部落里的一些情况做了简单介绍。

\section{第一百八十章 向导也作战}

克努蒂部落约有二十万人,是三百年前赛普斯部落分裂时独立出来的一支,在俾鲁弯算个不大不小的部族。

可是,由于部落里的武装分为三股,所以,克努蒂部落在这片地区并没有形成强大的战斗力。

最大的一股武装就是部落自卫武装,本意是为部族在这里生存提供必要的保障的;另两股武装分别在政治上倾向“俾鲁弯解放军”和“基天”或者“塔鲁班”。

倾向“基天”或者“塔鲁班”那些人,都是一些比较极端的民族主义份子,至于倾向“俾鲁弯解放军”的,那则是巴基斯坦的反政府武装,说是巴基斯坦的分裂势力也不为过。

还好,部落自卫武装在族里的战斗力量和长老会里的支持力量都是占绝对优势的,总算还可以维持大局的稳定。

据贾德坦的了解,马哈苏德所在的克普塞部族,似乎也分为了好几股力量,马哈苏德那一股,无疑是“基天”的铁杆支持者。

这些都了解了以后,楚云飞向对方提出了采购武器的要求。

不问不知道,一问吓一跳,这里武器的普遍行情,比伊斯兰堡普遍要低一半以上,只有那些类似“毒刺导弹”的高级武器,才是跟首都黑市的价格类似,不过这种货源是不太稳定的,而且价格跳水的水平远超中国的跳水运动员。

所以,当有人打着阿克塞姆的旗号来推销武器的时候,楚云飞婉言拒绝了,那家伙实在是太黑了点。

贾德坦个人对“基天”是深恶痛绝的,因为他觉得,任由他们胡搞下去的话,迟早会为伊斯兰世界带来灭顶的灾难。而且,国际上不会有任何代表正义的势力会接纳他们。

虽然克努蒂部落并没有什么多少武器储备,但通过贾德坦的私人关系,楚云飞他们还是获得了大量的武器和弹药,花掉了将近十二万美圆。

接下来,楚云飞又送给贾德坦个人一万美圆,请他代为打听马哈苏德的详细落脚点。“如果能成功干掉那个家伙,我还有自己对长者的谢意。”

同部落内大部分人相比,贾德坦的日子过得还是相当滋润的,他毕竟也算部落内一小支的家长呢。他更介意的是自己的威严和别人对自己的尊重。

“钱并不是主要问题,你既然是古维帕帕介绍来的,我一定会帮你这个忙的。不过,‘基天’那些家伙是相当狡猾的,想从里面找到指定的人,实在是不太容易做到。”

听到这个,楚云飞想起了多尼那个波兰人,既然报复,索性就闹大点也算,“那么,拜托了,把‘基地’组织的据点也都提供给我好了。”

贾德坦对楚云飞这个要求也没表示意外,不过,他还是提出了善意的建议,“楚,事实上,我认为你的力量太薄弱了,你确定一定要这么做吗?”

楚云飞垂下眼皮,微微一笑,又睁大眼睛看着贾德坦,只笑不语。

贾德坦很赞赏眼前这个小伙子的勇气,他无奈地点点头——在巴基斯坦,点头也是表示否定的。

“好吧,那这几天你和你的同伴呆在房子里不要出来,被别人知道的话,对你对我,都不是什么好事。”

按理说,穆斯林是不可能帮着外人对付自己的兄弟的,但楚云飞此来为的是私仇,这样无畏的男人,总是会得到别人的尊重的。

还有就是,在很多时候,“基天”残忍、无情的行为和作风,哪怕是同族人都无法接受的,因为他们不仅仅对敌人是这样。

那是害群之马,很多穆斯林都这么认为。

一连二十天,楚云飞他们三个人都在屋子呆着,水火不侵的肠胃也被那种叫“囊”、雷打不动的大饼弄得食欲全无。

终于在一天下午,鹅毛般的大雪纷纷扬扬落了下来,傍晚时分,贾德坦长老冒雪上门了,随行的,还有一个三十多岁的汉子。

贾德坦这次来,带来了基天组织在俾鲁弯省西北部的势力分布图,足足有接近两百个藏身点,还有两个隐秘的训练营,马哈苏德应该就在这片范围中。

那汉子叫阿卜杜拉。辛汗,是他们这次行动的全程向导。

辛汗本来是部落内倾向支持“塔鲁班”的人,也是伊斯兰原教旨主义者,民族荣誉感极强,当年曾经带着部落内几个好友自带武器,跨过边境跑到阿富汗作战,支持“塔鲁班”对抗美国人。

他和“基天”的仇恨也是那时结下的,“基天”作为“塔鲁班”的盟友,对这些万里迢迢来支援作战的穆斯林兄弟非常冷漠,经常连饮食都保证不了,而且还抢夺他们自带的武器,让不远万里前来助战的辛汗分外地“辛汗(心寒)”。

在一次战斗中,“基天”组织的几个人同辛汗他们被美国军队穷追不舍,“基天”的头目命令辛汗他们掩护,按惯例,“基天”的人要优先撤走。

当时的形势太危急了,就有人不服气抱怨起来,辛汗的弟弟上去劝解,却被“基天”的人当作儆猴的小鸡,当场击毙。

这是族内大家都知道的事,更使得极爱面子的辛汗在大家面前抬不起头,心态由此发生了巨大的变化。

所以,当他被长老告知有人计划对付“基天”时,异常主动地要求担任向导,并且参加战斗。

这里的爱恨情仇,真的不是一般地复杂。

从地图上可以看出来,“基天”组织的藏身之处大致可以分为两种情况:一是那种人迹罕至的山脉中,二就是群众基础非常好的小村落或者是部落。

由此可见,获得当地部落支持是多么的必要,没有他们代为打听,起码山洞里的那些藏身处,楚云飞他们怕是永远都找不到的。

当然,也有几个藏身点是位于城市中的,不过,就算是俾鲁弯省这么乱的地方,那里也没有什么“基天”的头目在其中,实在是太不安全了。

看着那些大部分位于村落的藏身点,楚云飞实在是有点头疼,他倒不是说心硬不起来,不过,那里实在是有太多的平民了,除了丧心病狂的人,没人下得了那个狠心。

“辛汗,你觉得,我们先从哪里下手的好?”

辛汗一点也不客气,满是汗毛的手指伸出,指向一处,“先是这里吧,贝西哈兰地方不大,但那是‘基天’的一个重要地方。”

从这反应就可以看出,他早就计划好了行动步骤。

楚云飞顺着他的手看了过去,很是纳闷,“这里是村落啊,里面该是有很多平民的吧?”

\section{第一百八十一章 突袭贝西哈兰}

大家都没想到的是,辛汗这个货真价实的穆斯林,对待起同胞来,比他们这些外人还狠。

“事实上,这里没有平民,从怀里吃奶的孩子,到走不动路的老头,他们都是‘基天’组织的成员。”

走不动路的老头是恐怖分子,这个,大家多少还能将就着认可,但这吃奶的孩子,是不是夸张了点?

看着大家用异样的眼光看着自己,辛汗越发地急燥起来,“相信我吧,我说得没错。”

“他们现在还是吃奶的孩子,但是,这整个村子绝对全是‘基天’的支持者和同情者,他们长大以后,注定也会变成穆斯林里面的败类!”

“真主要我们同情友爱,但是,这些败类绝对是例外。”

楚云飞沉默半晌,终于沉重地叹了口气,“好吧,我相信你。”

辛汗长出一口气,“谢谢你的信任,朋友,实际上,这村子里的人很少,只有三百多人,而且因为他们的坚决支持,那里才能成为‘基天’重要的补给站。”

三百多人?楚云飞的头又有点晕,“可是,我们只有三个人啊,再加上你,不过也才四个人。”

辛汗早想过这点了,“我还有两个朋友,比我还痛恨‘基天’呢,再说,我们在暗他们在明,他们三百人又不是个个能打仗,最多也就五十个人能打,你们又有那么多威力强大的武器。”

“天气这么冷,他们不可能一直在野外跟咱们战斗的,我相信,真主一定会保佑我们的!”

楚云飞真的无话可说了,算了,真报得此仇的话,下地狱也认了!

既然老天爷没有雷劈了马哈苏德那个混蛋,凭什么我就要被这虚无的良心和道德所制约?

看着眼前义愤填膺的辛汗,楚云飞忽然间羡慕起他来:人有信仰真好,做什么事都可以理直气壮!

既然决定了行动目标,那剩下的就是行动计划了,这点,辛汗也有自己的打算,“二月天,云从南方来,这场雪,起码要下一个星期的,我们明天就行动吧。”

楚云飞不忍心告诉自己战友计划里最残忍的部分,晚知道会儿,起码少受点良心的谴责吧。

他只是淡淡地跟战友说了大致的计划,刘宁和成树国已经习惯了楚云飞台前幕后的操心,也没太斟酌这个方案,正应了中国一句老话:天下事,自有尔等操心。

第二天中午,辛汗果然喊了两个帮手来,按他的说法,他自己的幼弟也是极其想来参与的,但他考虑此行风险,终于是没有答应。

大家在一起,甚至都没有寒暄几句,就开着两辆吉普踏雪而去。

雪地中,四条漫漫的车辙延绵而去,直至在天际尽头消失不见。

世间男儿,总要如这般过雪留痕才好的。

纵然,须臾间就有可能被那茫茫风雪所淹没,但留下了属于自己的痕迹,也算不枉来人间一场吧?

雪地行车,确实艰难,一百公里路,六个人足足走了将近五个小时,没办法,都是山路。

到了离贝西哈兰村将近十公里的地方,车不能再走了,大家找了个地方下车。

车上装满了武器,顶着寒冷刺骨的风雪,大家把事先准备好的粗制雪橇取了出来,把武器放了上去。

三个穆斯林一人拿了一支步枪和若干弹药,辛汗还多拽了挺班用机枪到雪橇上。

三个中国人的花样就多了,刘宁除了步枪,还抬了挺去了支架的车载重机枪,成树国和楚云飞搬的可全是82毫米无后座力炮。

这些都是中国部队能装备到排一级的武器,大家使用起来没有任何的问题。

看着三个穆斯林的雪橇,楚云飞皱皱眉头,“辛汗,帮我们拉点炮弹和子弹,你们拿那点东西太少了,要不再拿个火箭筒吧?”

辛汗点点头,又向雪橇上多搬了两箱机枪子弹和两箱炮弹,火箭筒也拿了一个。

天气实在太冷了,六个人披上了准备好的白色床单,拽着雪橇,艰难地向目标跋涉而去。

大家走了两个小时,总算到达了目的地,一个中国人一个穆斯林这样搭配,组成了三个小组,简单地划分了一下各自的目标。

等到大家都做好伪装和转移的阵地,又过去了一个多小时,已经接近当地时间晚上十点了。

村子不大,占地大概十亩地的样子,房屋集中在中间,但村子外围还有十几间稀疏的房屋,想来是做了望用的。村中部分地区还有低矮的石墙,看来以前就是用来打仗做掩体的。

楚云飞对着麦克风一声令下,自己率先扛着82毫米的炮发射了,目标就是村子中间的一大片房屋。

一串长长的火舌自楚云飞的肩头向后喷了出去,几乎在同时,村子里响了震耳欲聋的爆炸声。

成树国那里也开炮了。

几发炮弹落地,村子里马上慌乱了起来,男人的咒骂声、女人的尖叫声、还有孩子们的啼哭。

外围那些屋子,楚云飞他们并没有把他们放在眼里,就算那里有战斗力,但对方要不知死活地向外冲,三个训练有素的中国人所带的战斗小组,绝对可以消灭他们的。

那些零散屋子里果然都都冲出了人,但他们根本没有向外冲,而是一股脑地向村中跑去。

村子里的人早习惯了政府军的突击检查和小股其他武装的零星骚扰,眼下这种情况,绝对不是政府军的行为,他们是不会在黑夜突袭的。

在他们看来,开始的这几炮,绝对是一股力量强悍的武装来抢劫了,大家只能退到村中抵抗,等到天色大亮的时候,派人出去求救才是正道。

立刻,就有些武装分子冲到了石墙后,依托着掩体向外放枪,大部分人还在村中闹哄哄地乱跑。

冷枪一响起,楚云飞和成树国立刻警惕了起来,打一炮换一个地方,炮弹,每每在人群扎堆或者那些大点的房子上爆炸,引得村中一片哀嚎。

等到村中的武装分子发现只有两个炮击的火力点后,似乎商量了一下,从石墙后面一下冲出了二十多个人,冲向了楚云飞这边。

没别的原因,楚云飞离他们是近了点,比成树国近了五十多米。

\section{第一百八十二章 贝西哈兰的屈服}

楚云飞没理会他们,向村子里发射出了炮膛中的炮弹,一弯腰,拿起了自动步枪。

同楚云飞一组的克努蒂人早就子弹上膛了,不过,没楚云飞的话,他没开枪。

就在这时,一直不出声的刘宁那里机枪响了,他等这个时间已经很久了,无情的火舌疯狂地吞吐着,沉闷的枪声持续不停。

那二十几人中,登时就有七八人中弹,其他人马上纷纷卧倒。

他们才刚刚卧倒,成树国的炮弹就到了,虽然距离远了点,弹着点不是很理想,但爆炸掀起的雪块和泥土四溅,声势还是大得很。

看到那十几人依旧在雪地上趴着,匍匐着向自己爬来,楚云飞眉头一皱,又装了一发炮弹,这次,大家离得可就非常近了。

“轰”地一声,炮弹在人群正中爆炸,起码又有三、四个人见了阎王,楚云飞拿起自动步枪,“射击!”

这个小组的两枝步枪同时开火,目标是那雪地匍匐着的武装人员,成树国的炮弹又开始向村里转移。

刘宁的机枪还在那里不停地喷吐着火舌,把雪白的地面犁出一道道黑色的深沟。

楚云飞意外地发现,自己的搭档,枪法也非常地不错,三两个点射总能击中一人,水平快赶上成树国了。

几分钟过后,看到实在冲不过来,剩下七八个人连滚带爬地向村里跑去,与此同时,村中石墙后,一枚火箭弹向刘宁所在处发射了过来。

“轰”地一声,刘宁的机枪停止了吼叫,寂静几秒种后又再次响起。

还好,楚云飞暗自庆幸,幸亏花了一阵工夫来选择和修整阵地。

那七八个人终于跑回了村子,不过那发射火箭弹的家伙就没那么幸运了,三个火力点全转移到了那里,与刘宁一组的火箭筒也对着那里发射了过去。

这样的火力之下,那家伙不死才怪。

正在这时,两栋最高的房子上,两挺机枪疯狂地吼叫了起来,目标还是刘宁的机枪,这时候就要看谁压制得住谁了。

对这种家伙,实在是不能客气,两门“82无”对着这两个火力点就是一通乱炸,机枪终于停止了吼叫。

这时候,狙击中国人的,就是那石墙后的自动步枪了,还有若干个房屋上也偶尔红光一闪,射出几颗子弹。

中国士兵的战斗素养这时候就完全展现了出来,比眼前这些乌合之众强了不止一点半点。

步枪对步枪,谁怕谁?

刘宁放弃了机枪,拿起了步枪,在夜里雪地作战,能见度还是有一些的,虽然风雪不止,但起码能看见不少东西和参照物,视线要比索度那次夜战好多了。

随着刘宁那里一声声的枪响,对面时不时地响起一声惨叫,算不上“弹弹咬肉”,可也八九不离十了。

炮弹终究是有限的,楚云飞也不再发炮,拿起步枪跟刘宁一起开始了狙杀,他的同伴意外地发现,自己这部落里的“神枪”,似乎同对方还有不小的差距。

成树国见状,又打了几炮以后,也停止了发炮,像爆豆子一样,那里的班用机枪开始发威。

这时的贝西哈兰村,死伤的人数已经接近五十人了,当成树国那里也出现了机枪声后,他们马上明白了:这仗,没办法再打了。

终于有人大声喊了起来:“你们到底是什么人,跟我们贝西哈兰村有什么仇?”

随着狙击的不断进行,越来越多的人嚷嚷了起来。

楚云飞终于停止了射击,他的声音本身就不算浑厚,在呼啸的风雪中,他的回答显得越发地尖锐刺耳。

“你们把阿卜拉欣。马哈苏德交出来,他杀死了我的亲人。”

听到有人回答,村子里顿时安静了下来。

他们听得出来,答话的人用的是乌尔都语,可又带着不知道什么地方的一些口音。

过了半晌,有个苍老的声音通过喇叭发问,“你说的,是塔西特部落的阿卜拉欣。马哈苏德么?”

“不,我说的是克普塞部落的阿卜拉欣。马哈苏德。”

村子里的人“哄”地乱了起来,还有凌乱的争吵声,等了半天,还是那个苍老的声音发话了,“你最好弄清楚,我们这里是西买特部落,那个克普塞部落的阿卜拉欣。马哈苏德是不会在这里的。”

两边声音都很大,辛汗终于忍不住了,他大声喊了起来,不过,虽然他嗓门大得惊人,但一没气功,二没喇叭,声音实在是小了点,“别骗人了,那家伙也是基天的人,我们肯来这里,自然知道……”

辛汗的话没有说完,不过,村子里的人马上明白对方的真实意图了:人家知道这里是基天的一个藏身处,不用再多解释什么。

恐怖分子之所以会成为国际公敌,并不是因为他们的军事力量有多么强大,而是在于他们平时隐藏在普通人中间,没人分得出来,而一旦时机成熟,他们又能残忍地对社会做出巨大的破坏。

硬拼,不是恐怖分子的强项,何况这里绝大多数只是基天的支持者,严格点讲,连外围成员都算不上。

沉默半天,那苍老的声音又发话了,“我们这里有老人,还有妇女和孩子,如果你们不相信的话,可以进来搜查一下,我们保证配合。”

村子外面的六个人商量了一下,楚云飞他们还在犹豫,辛汗和他的两个同伴早就忍不住了,“还等什么?我们进去吧?”

只有三个对讲机,沟通实在困难了点,大家商量半天才做出决定。

以楚云飞的谨慎,自然要怀疑里面会不会有圈套,不过,面对这种情况,辛汗和他的同伴更有发言权。

事实上,在俾鲁弯省,由于长年的混乱局势,这种个人恩怨实属常见,这种恩怨甚至引发过二百人以上的家族战斗。

如果是单纯的两个穆斯林家族的恩怨,没外力干涉的话,他们之间的战斗通常扯不到外人。

同政府军来搜查时候的情形类似,每当对上强大势力的时候,只要不关自己的事,没有人愿意被牵连进去。

至于死了的人,死者已矣,多计较无用。

这也算一种麻木吧?楚云飞突然有了种身处刚卡的感觉。

不同的只是,那里从来是炽热炎炎,而眼前却是冰雪皑皑而已。

\section{第一百八十三章 雪夜屠村}

商量的结果,就是辛汗带着两个同伴进去搜查,楚云飞尾随但不露面。

三个人是少了点,不过,就算在战场上缴械,上前的头一批士兵也不会很多,只要不是太少就好说。

另一方面,如果楚云飞他们露头的话,“私人恩怨”很可能演变成“民族战争”,那就不好了,毕竟,“基天”是很擅长这方面的宣传的。

与三个战友的惴惴不安的心情不同,辛汗他们三个人是带着“征服者”的心情踏进这个小村子的。

二话不说,辛汗先要他们所有的人员出来,再有就是,把武器都交出来。

村子中央靠东有一块空地,所有人都被赶到了那里,在那里集合。

武器也有人交出了一些,不过,大部分精良的武器已经被藏匿了起来,集合地的火堆旁,只有十几支破旧的步枪和一个扭曲的火箭筒,一看就知道是刚才发威的那具。

辛汗正要说什么,从村庄外围楚云飞的位置传来了一声清脆的枪声。

村庄里,还是派出了人手向附近的村落求救,以免造成更大的损失,虽然最近的村子都在二十公里外。

那个家伙显然够倒霉的,一身素白衣服,结果还是在风雪中被楚云飞发现了他的挪动,楚云飞当然不会客气,一枪撂了过去。

听到这声枪响,辛汗越发地恼怒起来,而在场的村民却愈加地紧张:内有这虎视眈眈的三条汉子,外面又被包围了起来,现在只能,只能希望他们把怒火控制住,大家都是真主的子民啊!

接下来,没有任何的意外,辛汗手里拿着楚云飞提供的照片,把在场的人一一对过。

解除嫌疑的人,都被赶进了一座村中央最大的房子中,这房子虽然在刚才的炮击中被打坍了两个角,但也算非常的结实了。

到了最后,有将近二百七十人走进了这座“超大型牢笼”,屋外,只有两个垂死的伤员和一个躺在床上待产的孕妇。

事实上,村民们都知道,真正的抵抗力量,已经在对方前十分钟的突袭和反击中损失得差不多了。

楚云飞三人也从外围慢慢向村子靠拢,不过,最终没有走得太近,太近的话,火力施展不开。

人,总是这样的,众志成城一心抵抗的时候,再大的伤亡都有可能承受,但一旦放弃了拼命的念头,总会温顺得像头绵羊,惟恐触怒对手。

现在,两百多人的屋子,只有一个人把守着门口,门里是瑟瑟发抖的人群。门外几声枪响传来,屋里的人自然明白明白对方在枪杀伤员。

没有人义愤填膺,人们只是把哆嗦着衣服裹得更紧些,这个冬天,实在是有点冷。

辛汗和另一个同伴熟知这里的各种猫腻,他们自己的部落里也一样,有着各种的地窖和暗洞来躲避可能的灾难。

搜索一搜就是一个多小时,这还是房屋不多。

没办法,只有两人干活,想快也快不起来。

楚云飞他们在村外眼巴巴地看着村中的火堆,风雪中,火焰不停地跳动,偶尔来传来几声枪响说明自己人没有意外,期间还响起过三次手雷的爆炸声。

终于,两人搜索完了,如想象中的结局般一无所获,辛汗端着机枪就来到了关人的屋门口。

辛汗三人是蒙着脸的,这倒不是有意为之,实在是风雪有点大,仅仅是为了御寒而已,待到进村之际,把脖子上的毛毡套随便向上拉拉即可起到掩人耳目的作用。

不过,脸一蒙,在别人看起来,多少是有点心虚的味道的,自然会有人想他们是不想被人认出或者记住。

只有活人才能泄露秘密,这,就意味着大家都有生存的机会的。

但是,屋内所有的人都想错了。

辛汗手一挥,没有再多的解释,爆豆般的枪声响起,屋内的人如同被撞击过的保龄球一般纷纷倒下,枪口所指,没有遗漏。

这是一场一边倒的屠杀,虽然有几个彪悍点的汉子想冲出来反抗,但一来没有武器,二来门口又过于狭窄,冲得快,倒下得更快。

听到机枪声,楚云飞第一时间冲进了村中,身形过处,带起大片的雪花。

等到他发现是辛汗三人在做大屠杀,微微摇摇头,把成刘二人也喊进了村来。

还好,总算不用自己出手了,想到这个,三个中国人面对着如此残忍的场景,居然……庆幸了起来。

屠杀持续了将近十分钟,终于,屋子里再没有人能站着了。

辛汗拔出了腰间的弯刀,就待进去仔细搜索尚存活的人。

够了,楚云飞在远处喊住了辛汗。

他实在是有点看不下去了,当然,里面还有更重要的原因,“辛汗,算了,不要进去,会有危险的。”

此时的辛汗已经杀心上头了,不过还好,理智尚存。

接下来,众人把木柴纷纷扔进了屋里,再丢个火把进去,一时间,火光冲天!

屋里顿时传出了此起彼伏的咳嗽声。

这声音如同催化剂一样,大家扔木头扔得更起劲了。

熊熊火焰燃烧了有一个小时,势头才开始转弱,这时候,除了“毕毕剥剥”的木头爆裂声,再没有任何的人声从屋里传出来。

在这段时间里,楚云飞也没有闲着,深藏在内心的兽性被这几个穆斯林汉子诱发了出来,他手持火把,疯狂地烧起屋子来。

这就是报应!

我不是神,我在用自己的力量报复!

这村里的人并不算富裕,大部分的房子都是砍伐自山中的木材所建,只有三间房子是石头砌成的,可房顶还是木结构。

看着近似变态的楚云飞,刘宁和成树国目瞪口呆,他俩无论如何也想不到,平时以冷静著称的战友,会陷入这么疯狂的情绪中。

成树国沉吟半晌,从地下捡起了一根燃烧的木柴,刘宁愣了一下,也是有样学样拿起根木柴。

不就是放火么?谁不会?

这是一个疯狂的夜晚!

在撤离的前夕,楚云飞没有忘记在石墙上用鲜血划上一句话,“阿卜拉辛。马哈苏德,我会找到你的!”

事实上,屋内被屠杀的人身子底下,很多人都刻下了这个名字,更有人指出是克普赛部落的马哈苏德。

楚云飞他们拖着雪橇渐行渐远,身后,是那熊熊燃烧着的村庄!

\section{第一百八十四章 目标维多城}

贝西哈兰村被疯狂屠村。

包括一名待产妇女在内,整个村子,只有一个冻得半死的十一岁的少年侥幸逃生。

但是,那个孩子还不如死了的好,由于受到过度惊吓,他疯了!

这桩惊天惨案并没有及时传出去,原因很简单,雪下得太大了,大雪封山,没有人会在这个时间出门的。

如果不是那么大的风雪,哪怕是在山中,贝西哈兰村隔壁的两个村庄,也该隐约能听到那晚的枪炮声的,更别说那冲天的火光了。

大雪掩盖了楚云飞他们的痕迹,同时也为他们提出了新的难题,这场雪结束后,山路上起码一个月不能走车。

现在,雪没下多长时间,还没有上冻,山路将就着还能走,但等再下一两天,或者多几辆车走过,把路面碾瓷实了,基本上也就不能行走了。

下一步该怎么走,雪夜出山,还是再隐忍上一两个月?

楚云飞和成树国都是北方人,自然知道这雪继续下下去的后果,可他们对巴基斯坦实在算不得很熟,这一切还是要靠辛汗拿主意。

辛汗沉吟半天,有些犹豫了,“天气实在是不太好,要不,我们等上一阵再动手吧,大家也好养精蓄锐,方便下次的行动。”

楚云飞愿意听他的意见,但这不代表任何借口都可以被容忍,这样的理由他是不能接受的。“我们难,他们更难,现在就是比毅力比勇猛的时候,过了这阵子,天气一旦变好,他们也有机会做准备了。”

“当然,如果说弹药和给养肯定会接济不上;说交通上难度太大;或者说雪后的地面不合适我们脱离现场,那暂时停停也是正常的。”

可辛汗已经被他前面一句话激怒了,穆斯林的汉子,毅力和勇气都不会比你们少的!

辛汗终于做出了一个日后被很多人诟病的决定,“咱们有车,弹药和给养都不是问题,一次多拉点就行了,交通也不是问题,大家都有腿的。”

“至于雪后不好脱离,那我们先在城市里动手好了,雪被冻实以后,咱们再找村庄下手。”

既然决定动手,那打就打呗,谁还怕谁不成?

大家连夜返回部落,拉了满满两车武器和弹药,满到除了司机,所有人都是坐在弹药箱子上和武器上。

他们甚至没有休息一下,直接就下了山,出得山去的时候,已经是第二天中午了。

尽管是中午,但由于依旧有雪花在飘落,道路两边并没有什么人,趁这个机会,大家找了片茂密的树林,把车开进去休息起来。

等睡起来的时候,已经是傍晚了,大家把多余的弹药藏到了树林深处,还留了一个辛汗的同伴在这里看守,另五个人开着两辆车呼啸而出。

目标已经选好了,就是山口的维多古城。

维多是座很古老的城市,古老到它的历史已经无从考究了,反正,辛汗知道就是,三百年前,当他们部落从塞普斯分裂出来的时候,维多已经被称做古城了。

现在的维多城,算俾鲁弯省的一个大城市,原因无他,住在整个古帕尔山区中的人实在是太多了,其中有一半的人是靠这个城市来提供生活必需品的。

由于维多城算个比较关键的商业枢纽,所以,城中有几个隐秘的“基天”联络点,楚云飞他们的情报上就标了三个。

一个联络点是位于城南的清真寺内,里面有两个德高望重的阿訇是“基天”的同情者,他们的观点,通过讲经传教在潜移默化下,造就了一批“基天”的支持者。

这个地方和那两个人,楚云飞他们考虑了半天,终于放弃了对它的袭击计划,原因无他,连辛汗都反对这么做,这毕竟是伊斯兰教传教和举行宗教仪式的地方,没办法放肆的。

不过,辛汗也答应了,他一定会去里面打听马哈苏德的下落的,因为那里面的的环境对“基天”有利,穆斯林们谈论起“基天”也是不怎么避讳人的。

后来楚云飞才明白,伊斯兰教的清真寺,只是“基天”那些人避难的场所,对着政府军的搜捕,他们能躲进去。政府军一走,他们会马上离开那里。

因为,实在是,“基天”的很多极端行为,也是被伊斯兰教限制着的,平时不好说你,但你要在清真寺里长期胡来,既不现实,也容易为清真寺带去祸端。

另一个目标,是维多市警备部一个副部长的儿子,年轻人容易冲动,家庭条件也不错,居然被他活生升地弄出个“维多文化交流中心”出来。

那里面四、五个人成天无所事事,打着学术交流的幌子,私下却为“基天”组织做些提供情报、转移资金、通风报信的勾当。

这一切都瞒不过有心人,当副部长的父亲曾经劝过儿子放弃,但被儿子很鄙夷地拒绝了,不过,“基天”在维多城从来没做出过什么天怨人怒的案子,所以大家看在那父亲面子上,暂时没有去难为他。

还有一家,是个闹市中的伊斯兰饭店,说是饭店有点夸张,不过同中国的地摊类似,规模稍微大点,老板和两个伙计都是“基天”的人,那里是个很好的情报采集点。

当然,偶尔也会有被追捕的“基天”成员去那里避难,事实上,这个联络点就是这么被克努蒂族发现的:有次他们藏了三个“基天”成员二十天。

仔细分析了情报,几个人商量后,一致决定:先从伊斯兰饭店老板那里下手。

原因很简单,饭店老板身份比较隐秘,他有那么一半天不见的话,很有可能有其他原因,而部长公子如果莫名其妙失踪,那十有八九,是有人要对付“基天”了。

为了两个点一网打尽,必须选择好先后的顺序楚云飞倒不怕暴露目标,要不他也不至于在贝西哈兰村留下字迹了。

他的计划,本来就是要通过血淋淋的屠杀,逼得马哈苏德自己现身。

楚云飞和辛汗都认为:作为一个原教旨主义者,马哈苏德不可能无视同伴不断地被他自己牵连,尤其,又仅仅是因为个人恩怨。

说实在话,如果楚云飞是代表美国政府或者其他什么势力来对付“基天”的话,辛汗都未必会这样支持他。

\section{第一百八十五章 基天不足恃}

他们到得维多市的时候,是下午四点左右,那个清真寺里人已经不多了,不过伊斯兰教的重大节日“古尔邦节”快要到了,还是有些人在那里张罗节日庆典的准备工作。

辛汗和另一个同伴很轻易地就混了进去,边帮忙边随口打听“基天”的事。

很正常,他们什么也没有打听到,却是确定了一点:这里,不认可“基天”的人,比支持他们的人要多得多。

兵贵神速是大家都知道的,在贝西哈兰村的惨案还没有传到维多城之前,能快点下手还是快点下手的好。

当天夜里,楚云飞他们就登门造访了饭店老板,把饭店老板和两个伙计一同“请”了出来,当然,被“请”的人没有选择的余地。

店老板是“基天”组织的正式成员,维多城是他的工作场所,所以除了两个伙计,他没有亲属同住,楚云飞他们也避免了再次大开杀戒。

在城外十公里处一个土坑内,仓促的审讯展开了,所有人在袭击时就蒙上了脸。

问话的是楚云飞,他离“客人”的距离比较远,中间站着辛汗和他的同伴。

先被审问的是一个饭店伙计。

“你知道克普塞部落的阿卜拉欣。马哈苏德么?”

那伙计愣了一下,然后马上点头表示不知道。

成树国正在按着另一个伙计,看到楚云飞在那里遗憾地摇头,扬手就是一枚钢钉,钉到了被审讯者的太阳穴附近。

那人发出惊天动地的一声惨叫,身子软绵绵地瘫了下去,手脚抽搐两下,不再动弹。

成树国刚要把手里这个放出去,楚云飞一指那老板,“你,过来。”

这样的审问顺序才算比较合理,刘宁点点头,放开了那个老板。

老板虽然身上绑着绳子,偏胖的身子却没失去平衡,他慢慢走到辛汗面前。

“我们都是‘基天’的人,你们最好放明白点。”

眼前这帮人心狠手辣,老板自然能看出不是政府的人,当然,不是说政府军就是多么善良,不过,如果不是出于私愤,他们起码是没有这样的魄力的。

对于政府以外的势力来说,“基天”绝对是个响当当的牌子,而且生死存亡就在眼前,他自然要拿出最大的一张牌。

辛汗轻蔑地哼了一声,连嘴都没张,而且,他也没有再继续说什么,他明白,今天的主角是自己身后的那个人。

楚云飞轻轻地笑了一声,在这深冬的寒夜里,显得格外地冷酷,“你觉得,除了你的身份,我们还会有别的理由么?把你在这么冷的天气里叫出来。”

店老板马上就明白了自己的处境,毕竟是受过专业训练的,他强自镇定一下,开始拖时间,“你说的那个人我知道,绑架中国人质的那个,是吧?”

废话,这还用你说?

事实上,马哈苏德成名就是因为这件事,他之前并没有做过什么惊天动地的事情,这事之后更是亡命天涯,四处躲藏。

之所以大家能记得那中国人质,是因为法国人质在事件中没有丧生,而中国人,还是巴基斯坦的传统盟友。

“我只想知道他的下落”,楚云飞明白对方在想什么,“当然,如果你想从头开始聊,我可以把你的衣服剥了,大家想到什么就说什么好了。”

那老板想的是争取时间,天寒地冻并不能阻碍他的坚实信仰,在被剥去最后一件内衣时,他还有心情打着哆嗦发问,“得得得……我~我能知道~~~~能知道~~~~你为什么~~~~找~~~~他么?”

得得得,那是牙齿相互叩击的声音。

楚云飞随口发誓,“安拉在上,他杀了我的父亲,我要报仇。”

理由是真实的,但他确实不信安拉的。

那老板想了半天,寒冷的天气并不能帮他清醒头脑,事实上,他的脑袋反而有点晕了,“那你们找我绝对找错了,我想,辛巴会更清楚一些的,他的级别比我高。”

马哈苏德手下人命其实不少的,老板丝毫也没怀疑眼前这家伙是那中国人质的儿子。

“辛巴?”楚云飞开始有点怀疑老板在套他的话,不过,他既然怕冷大家就多聊一会儿好了,“似乎,我不认识这个人。”

老板冻得实在受不了啦,大声地咒骂起来,“该死的,快给我穿上衣服,那是你们的事,我管不着,辛巴就是阿布巴克的顾问。”

阿布巴克就是那部长的公子,这个楚云飞还算明白。

楚云飞递过了两件内衣,“说实话,阿布巴克我知道,不过辛巴只是他的雇员吧?难道他是克普塞部落的人么?”

这绝对是揣着明白装糊涂,楚云飞想问的绝对不是这个。

老板马上接过衣服,哆哆嗦嗦穿了起来。

天气太冷,内衣穿得都不太利索,不过,这不妨碍他的思维,“你要想找马哈苏德,最好是去问辛巴,阿布巴克,哼哼,他才多大年纪?”

楚云飞微笑着点点头,“谢谢你,既然你什么都不知道,那你对我们来说就是没什么用了,你说是不是?”

说毕,楚云飞点点头,刘宁从远处慢慢踱了过来。

老板被这无情的话惊呆了,一时竟找不出什么借口来缓解。

最终,刘宁拧断了他的脖子。

另一个伙计,早已经吓得尿湿了裤子,安拉在上,冬天尿裤子,还是在野外……真的很难受的。

等到成树国推他上前,他已经脚软得走不动了,成树国不得不拎着他走了过去,老天,摊上这么个俘虏确实有点累人。

还没等楚云飞发问,那俘虏早如同竹筒倒豆般交代了起来。

“马哈苏德我知道,不过他已经有一年没在维多露过面了。”

楚云飞哼了一声,“这么来说,你也不知道什么啦?”

那俘虏更加慌乱了起来,“我知道,我当然知道,事实上……”

成树国想起来在工人党老大戴维斯院里的袭击,似乎把钢钉留下是个很不好的习惯,他拔出刀来,走向那已经死去,被冻得梆梆硬的伙计,一刀削去了对方天灵盖。

他正在这里感叹弯刀的锋利,那个活着的伙计却是再也忍受不住,“哇”地一声翻江倒海般地吐了起来。

不过,那呕吐的朋友显然马上就明白了目前的处境,一边呕吐一边强行交代,“事实上,呕,马哈苏德跟我们联系得很少了,呕,他似乎,呕,似乎都快脱离基天了。

“他脱离不脱离基天,跟我们毫无关系。”楚云飞这话虽然冷漠,但绝对是发自内心的。

可惜,那伙计并不知道,不过,他还是在绞尽脑汁想着有用的情报,“所以,只有辛巴这种老人,这种高级别的成员,才能知道他的下落的。”

楚云飞点点头,“你还有什么有价值的情报么?我发誓,你再说出一条来,就放你离开。”

\section{第一百八十六章 维多文化中心}

饭店老板和伙计三人的尸体隐藏得很好,楚云飞他们甚至专门在雪地上挖了个深达半米的坑来掩埋他们,呃,其中雪厚四十厘米。

反正,在雪化之前,怕是没人能发现了。

那最后一个死的伙计交代了一件很重要的事,起码楚云飞认为很重要,那就是:副部长之子阿布巴克的三个雇员,包括辛巴在内。因为副部长的极力反对,并不能受到客人应有的待遇,他们是在“中心”里过夜的。

楚云飞当然不会因为这个消息就免人一死,事实上,那也算不上什么重要情报。

其实,楚云飞早决定了,如果真有人能通报马哈苏德的藏身之地,他绝对会放人的——在按照情报杀了马哈苏德之后。

所以,他也不算是骗人,不是么?

楚云飞原本计划,干完这件活后就休息,毕竟,大家已经连着两个晚上没好好休息了,,不过,听到这个消息,他又有点动心了。

看看手表,凌晨一点,是不是再接再励,把“维多文化交流中心”也端了它?

刘宁知道楚云飞在想什么,插了句嘴,“还想什么呢?干就干呗,咱不差这几个小时的觉。”

成树国也点点头,“不错,现在消息没传出去,那里应该防备比较松懈,等明天他们知道这饭店的人失踪,多少是要警惕些的。”

辛汗俩人没吭气,事实上,他们确实有点累了,不过,现在在打仗,必要时候是要咬牙坚持的,再说,也不能让外族人看了笑话去。

看到大家都没什么意见,楚云飞掉头上了吉普车。

“维多文化交流中心”,牌子很响,但却坐落在一个非常小的窄巷里,巷子里还有一条蜿蜒而来,逶迤而去的臭水沟,有将近半米宽。

那污水的内容显然非常地丰富,这么大的雪,居然没有覆盖了整个沟渠,也没有上冻,还好是冬天,味道不甚浓重。

不过从另一个角度讲,这里确实合适做些隐秘的工作。

没有把车开进去,楚云飞、成树国和辛汗步行摸了过去。

在辛汗的指点下,三个人轻松地找到了那三个雇员休息的地方,手起掌落,三个人被直接打昏,带了出来。

审讯,依旧是在那片藏尸之地进行的,最先被询问的,就是辛巴。

辛巴的地位确实高,从他谈吐间的气度和隐隐流露出来的不羁,说明此人不但受过相当水平的教育,而且可以肯定也是接受过某些训练的。

对于楚云飞的一切问题,辛巴总是那副半死不活的样子,“我不清楚”、“我不知道”……

尤其是楚云飞问到马哈苏德的时候,辛巴的眼中甚至露出了一丝嘲讽。

没错,就是嘲讽,纵然是在夜间,皑皑白雪的反射中,他的表情暴露了内心的想法。

楚云飞被这眼神弄得有点心烦,不说是吧?那好,顺手从腰间拔出了弯刀,“事实上,我最喜欢你这样有骨气的汉子,反正你的话对我来说,没什么太大的价值。”

说罢,楚云飞一刀斩下了辛巴的右臂,瞬间,鲜血激射而出。

辛巴惨呼一声,当时就疼得满地打滚了,不过,他还真的算个男人,居然硬声声地扛住了接下来的痛苦,没再出声。

过了好半天,辛巴才稳定住了身子,不再打滚,不过,透过他脸上浓密的胡须都可以看出,他的脸色白了不止一点。

楚云飞笑嘻嘻地走过来,顺手一挥,刀又斩到了辛巴的左脚上,不过,这次力道不大,没把脚砍下来。

辛巴又是一声凄惨的呼号,楚云飞不满意地摇摇头,不好,看来以后还是要多熟悉熟悉怎么用刀,居然没砍准关节。

等到辛巴再次平静下来的时候,他已经是满头的大汗,堪堪地要晕过去了。

楚云飞摇摇头,神色虽然平常,却偏偏又像要择人而啮的野兽,“求我吧,求我杀了你,我也许会同意的。”

辛巴再次抬起头,眼中满是恶毒的神色,那种悠然和不羁早就不知道跑哪里去了,他咬牙切齿地咒骂着,“你是个魔鬼,你会后悔的。”

这时,楚云飞脸上居然露出了一丝笑意,“我喜欢你的眼神,呵呵,能让你仇视我,我很开心。”

成树国正在为楚云飞的精神状态担心,却见他弯刀挥出,直接砍掉了辛巴的头颅。

事实上,楚云飞真的很想从辛巴嘴里得到些消息的,毕竟是“基天”里象样的一个人物呢。不过,那家伙实在太拽了点,这让他感觉非常地不爽。

心中泛起股失落的情绪,楚云飞走过去,又拽起了另一个雇员。

“你也不知道马哈苏德的消息,是吧?”楚云飞阴森森地问道。

那个雇员嗫嚅半晌,终于像是下了什么决心一样,张嘴待说。

就在这时,楚云飞忽然一个激灵,有危险,他一脚把那人踢出去十米开外,身形暴退!

“轰”地一声巨响,那个家伙浑身上下被炸得七凌八落,左半个身子更是血肉模糊,而那爆炸物,似乎正藏在那人身上。

这,才是真正的死士吧!

这次爆炸的声音有点大了,虽然四下无人,谁知道会不会被远处的人听到?

楚云飞仔细想了下,才想起那家伙的左臂似乎是假肢,爆炸物在那里藏着么?还好威力不算大。

不过,这次可真叫楚云飞见识了“基天”成员的狠辣,居然能长期把爆炸物放在身上,看来,以后能不弄活口还是不要弄活口了。

事不宜迟,楚云飞拎着弯刀走向最后一个雇员,赶紧结果了他走路吧。

那雇员却早已经被瘫软在地上,见楚云飞杀气腾腾地过来,没命地叫了起来,“我知道,我知道他在哪里!”

哦?能出现这么个回答,楚云飞着实地吃了一惊,他走上前来发问,声音居然有一丝丝地颤抖,“你说吧。”

那雇员哆哆嗦嗦地回答,“马哈苏德,他,他现在就在克普塞部落附近,他和‘基天’好象发生了点不愉快,回他的老窝了。”

楚云飞没点头,他怕对方误会他的意思,下意识地刮刮鼻子,继续发问,“你说具体点,他在什么位置。”

\section{第一百八十七章 基天的叛徒?}

听到这问话,那雇员的脸当时就苦了下来,“我实在是不知道啊,我也不是‘基天’的成员,只是偶尔记得有人说过,马哈苏德越来越嚣张了,很多人看他不顺眼呢。”

楚云飞沉吟一下,挥手打晕了眼前这个人。

这里已经不能再呆了,天色尚早,暂时不会有人来,但再拖下去问题可就大了,谨慎点,撤离好了。

埋好那两具尸体,几个人驾车赶回了那片树林,天上,又飘起了纷纷扬扬的雪花。

一觉醒来,已经是第二天的傍晚了,大家点了堆火,开始烤囊吃,顺便也喝点开水。

天上依旧是阴云密布,不过,雪已经不怎么下了,天气已经变得越发得寒冷了。“下雪不冷化雪冷”。

那一直被绑着的俘虏双手已经冻得青紫,忍不住哀声恳求起来,“给我喝几口热水吧,我已经快冻死了。”

楚云飞停止了咀嚼,站起身来走了过去,解开了对方的绳子。

看到他的脸,俘虏惊讶得都忘记了自己的寒冷,用力抬起手臂指着他,“你、你、你……你是中国人?”

楚云飞瞟他一眼,没再理他,“这里有囊,还有热水,一起吃点吧。”

这人提供了相当价值的情报,对楚云飞来说,就算不能留活口,但也不能过于虐待。

那雇员终于弄明白了状况,不再说话,从地上捧起一捧雪,开始搓揉自己的双手。

这是常识,天太冷了,被冻得过度的肢体不能马上接触温度太高的东西,否则轻者脱皮,重者直接会被这大冷大热破坏组织结构、损伤肢体。

他搓了足足有半个小时,才停了下来,手的皮肤已经开始发红,看来不会再碍事了。

等到他吃饱喝足,不待人催,又开始交代他自己了解的情况。

原来,这家伙在中心也是没什么地位的,只是用来掩人耳目的普通员工,但他内心比较同情“基天”,做事也算谨慎,所以,中心里其他人大多时候对他也不怎么避讳,他才得以了解了一些关于马哈苏德的消息。

马哈苏德本身在“基天”也算得上是个问题人物,在“绑架人质事件”以前,他只是一个负责对外联络的小头目,曾经在一次行动中被美国人擒获。

然后,他就被关到了那个非常著名的地方,专门用来关押恐怖分子的古巴“关塔那摩监狱”,据说也是遭到了惨无人道地折磨。

后来,因为他实在是个太微不足道的小人物,又据说在关押期间非常合作,于是他被放了出来。

在“基天”的内部说法中,就是两年的监狱生活没有影响马哈苏德所追求的信仰,又成功地回到了“基天”组织中,并被戴上了“在监狱中坚持真知灼见”的光环,以向世界展示基天“勇士们”不屈的意念。

然而,同所有激进的民族主义者一样,马哈苏德的形象虽然树起来了,但进过监狱并不是什么值得庆幸的事情,毕竟美国人洗脑的水平在世界上也是很有名的。

于是,他的地位并没有什么太明显的提高,他还是需要再经受些考验才能再往上爬的。

这时,蹊跷事就出来了,不知道是运气不好,还是真有别的什么因素,马哈苏德所在片区总是会受到反恐势力的袭击,换到哪里,哪里就是这样。

马哈苏德要营救的阿卜拉欣。巴,就是在这种情况下被沙特政府擒获的。

这样,就可以理解马哈苏德为什么会做出轰动“基天”和世界的绑架人质案件了,他必须坚决地洗刷自己的嫌疑。

这次绑架,让马哈苏德闻名世界的原因,是无辜的中国人质被害。

而让他闻名“基天”的原因,则是这件绑架实在是匪夷所思,虽然他表现了自己的勇敢和坚决,但绑架中国人实在是不合情理的,尤其,在沙特政府军的进攻中,又有不少“基天”的成员被杀。

那些都是“基天”组织里宝贵的精华啊,为了这么屁大的事搞得极其被动而且丧失性命,实在是得不偿失!

所以,马哈苏德的行为理所应当地遭到了“基天”内一些人的质疑。

马哈苏德自然是大为光火,事实上,不管他是否无辜,他都必须生气,否则的话,“基天”的人会成为他一辈子的梦魇。

他生气了,“基天”的人也不可能脾气有多好,只是姑且看在没掌握他“通敌”的证据上,大家两不相干而已。

于是,马哈苏德目前的境地,很是尴尬。还好,由于那次绑架,他“反美斗士”的名声终究是传了出来,在他的部落里还有一些人愿意跟随他的。

楚云飞总算是明白了,那个死硬的辛巴为什么会那样看着他,人家是在笑话他呢:找马哈苏德居然找到“基天”的头上,实在是惹人耻笑的愣头青!

看来马哈苏德在“基天”也不得人心呢,那么,这样继续逼迫下去,那家伙是该会很快浮出水面的吧?

那俘虏交代完问题,看到这几人不再那么凶悍,好奇心居然被勾了起来,怯怯地问道,“你,你真的是中国人?为了那个死了的中国人质来的?”

楚云飞没有说话,狠狠地瞪了他一眼,让他明白自己的处境:你老实点吧。

这次确实得了不少消息,楚云飞正在这里犹豫该不该马上杀这个人,那家伙马上就自己找起死来。

很明显,那家伙受“基天”的毒害比较深,他居然指责起了辛汗他们,“亏得你们也是穆斯林,居然帮着外族人残害自己的同胞,他要报仇,那是他的事,你们帮什么的忙?”

辛汗的脸色当时就拉了下来,无他,这民族的帽子确实是很大的,他还没来得及反驳,那雇员又吵吵起来。

“‘基天’是什么?那是所有穆斯林的骄傲,是我们重建大阿拉伯帝国的希望,你们怎么能帮着他们残害这样的精英?”

“以前的事,我不想再说了,只想求你们,以后还是不要不招惹‘基天’的人,要知道,这样热血的青年,已经不多了!”

楚云飞轻蔑地“哼”了一声,说到底,这家伙还是担心自己的小命不保啊,不过这分化的手段,太下作了点吧?

辛汗被他说得脸上青一阵白一阵的,终于再也按捺不住,抽出了弯刀,“你们‘基天’,杀了我的弟弟,还好意思说是热血青年?我呸,你们就是一堆人渣!”

那雇员可真没想到会是这种结果,他还指望着三个同胞帮他说情,留下他的小命呢。

皑皑白雪中,刀光起处,鲜血狂喷,一颗人头滚落了下来。

\section{第一百八十八章 血雨腥风}

知道了事情的原委,楚云飞又一次面临着抉择:直接杀奔克普塞部落,还是像原来计划的那样,挨个拔除“基天”的据点呢?

楚云飞自觉双手的血腥太多了点,倒是有心直接去克普塞部落寻人,但辛汗不同意。

“才下过雪,天马上会大冷,没人会在这个时候出门的,你不多杀点‘基天’的人,别人根本就不会知道你要找马哈苏德。”

“还有,这时候杀人,消息传得不会很快的,既然会很安全,我们为什么不多杀点?我可是为了杀‘基天’的人才帮你的,再说,你父亲的事,‘基天’还是要负主要责任的。”

楚云飞皱着眉头看看辛汗,又沉思起来。

不知道为什么,最近一想到杀人,楚云飞的头总是在隐隐作痛,他基本上已经达到那种“杀人杀到心软”的境界了。

不过,辛汗是不会满足这几场屠杀的,这家伙一开始的计划就是向贝西哈兰村下手,从这点就可以看得出来怨念之重。

但是,辛汗的分析也很有道理,既然会很安全,那多杀几个也不算什么大事吧?

楚云飞揉揉太阳穴,来减轻那种若有若无的头痛,下了决心。

接下来的二十多天,对于“基天”组织绝对是近似于噩梦般的日子。

克普塞部落还在维多城的西北,是在另一片山区:断背山脉!

这二十多天,楚云飞他们以秋风扫落叶一般地横扫着“基天”的各种据点,如果有地图,所有人都能看得明白:从维多城开始,血腥在向着西北方向蔓延。

看得出来,凶手每次都是人挡杀人,佛挡杀佛,剩下的生物,只有零星的猫、狗、家畜之类!

其中还有两个路上的补充食水可以歇脚的,类似客栈加小卖部的地方,虽然这天气没什么行人在里面休憩,但里面的买卖人也有十几个人。

这两个地方,连房子都被人烧了,奇怪的是,相邻的类似歇脚处却依然活得非常滋润。

从第七、八天开始,终于有传言,这些凶手的目标是“基天”和克普塞的马哈苏德!

又过了几天,各个被害的地方传来的消息证明了这点:凶手每次行凶完毕,都要向马哈苏德发出疯狂的叫嚣。

而且,大家一分析,终于明白,杀气漫天,血腥撼地,那恐怖的兵锋所指,就是克普塞部落!

于是,当贝西哈兰村的惨案终于为世人所知的时候,人们在震惊之余,不由得又发出一丝疑问:这个阿卜拉辛。马哈苏德究竟做了什么天怨人怒的事,居然会招惹到如此恐怖的仇家,造成了这一系列的惨剧呢?

死在这些凶手下的人数,已经激增到了四百人!

有些人会对个别人的遇害表示谨慎地怀疑,不过,这些怀疑终于被确定是不公平的:确实,那些人起码是心向“基天”的,世界这么大,总有人知道内幕的!

这时,那杀气腾腾的战车,已经推进到了距离克普塞部落不足六十公里的地方。

外人评说是外人的事,事实上,长期的昼伏夜出,使得六个人疲劳无比,再说,还有一个穆斯林在夜袭时被弯刀划破了左肩。

还有就是,这些天里,六个人食宿都是在野外,这么寒冷的天气,实在是太折磨人了。

一天早上,成树国醒来,居然意外地发现,他夜里呵出的气,居然把枪和盖的毯子冻到了一起,太夸张了。

眼看着温度有些回升了,几个人商量着,解决完眼前这几间小店,回去一趟,既然又抢了辆吉普车,那还能再多弄点弹药。

大战在即,无论如何,要休整一下了。

何况,大家一度俘获的人中,没人能说得清楚自己这六个人给对方带去了什么样的震撼,倒是有人能证明马哈苏德确实同“基天”分道扬镳了。

不知道对方的反应,实在不是什么好的兆头,即不利于作战方案的部署,也不利于制定相应的策略。

当楚云飞谨慎地进入那片路边店时,却意外地发现了墙上有偌大的一行红字:“阿卜拉辛。马哈苏德已经脱离‘基天’,他就在前面。”

几个店中,居然空无一人,倒是有些零散的商品在地上胡乱地抛着,显然对方撤离得比较慌乱。

不过,楚云飞还是谨慎地发现了一处陷阱,在一间通向里屋的门上,一根极其隐蔽的贴地细线紧紧连着足足有五十公斤的TNT炸药,炸药藏在门后,触发式的引信。

老天,要是这家伙真的爆炸了,怕是方圆一百米内不会有任何活口吧?

从这句话,和那一大铁皮桶炸药,楚云飞可以品味出不少的意思。

首先,这二十几天的连续袭击,确实把“基天”打痛了,为了避免不必要的损失,他们居然把可能已经暴露的人手都撤走了。

没错,仅仅是“可能暴露”,楚云飞相信,自己那些情报绝对不可能是对方的全部实力,但同时,对方也绝对不可能知道自己到底掌握了他们多少内幕。

那么,既然这里有人撤离,同样,肯定还有不少的地方撤人了。

但愿,那些地方的恶毒陷阱,不要让太多无辜的人倒霉吧。

其次,这次的打击之痛,让“基天”确实难以承受了,不得不把内部的创口露了出来,那么私密的事情都公布了出来:没错,马哈苏德曾经是我们“基天”的人,但是,拜托,他现在已经不是啦!

当然,这里面不乏祸水东引的意思,但能让“基天”这种以强硬著称的极端民族势力叫苦,楚云飞他们足能引以为傲了。

至于那个陷阱,却是最终表示出了“基天”的怨念之深:你终究是杀了我们那么多人,如果有机会的话,我们绝对不介意在你们身上踩上几脚的。

当然,还有更深的一层意思,那就是,看到没有?你虽然杀错了人,但,我们依旧是记仇的,这小小东西,不成敬意,下步行动,你还是仔细掂量掂量吧。

总之,那就是双管齐下,恩威并施。

摇头笑笑,楚云飞悄然地溜了出去,手里还拎着那一桶炸药。

正如来的时候一样,脚下生风,却不带走一片的雪花。

\section{第一百八十九章 意外的电话}

这场雪没有像辛汗算的那样下了一个星期,但也零零落落地下了有五天,等路上的雪冻实,也差不多十天了。

虽说冬天的俾鲁弯省行人稀少,不过路还是被压得光可鉴人,异常难行,回撤到维多城,一百五十公里路让六个人走了三天。

途经维多城的时候,楚云飞的手机又响了,是短信。

巴基斯坦的通信实在是很落后,但维多这样的城市,还是有移动通讯基站的,楚云飞在不忙的时候总是开机,万一琳琳打电话呢?

短信是多尼发来的,搞得楚云飞异常地纳闷,他又出什么事了,着急找我?

回电话过去,居然是对方关机,楚云飞又打了另一个号,多尼接起了电话。

天太冷,空气似乎都凝成了固体,楚云飞的声音传得很远,“喂,多尼,什么事啊?不知道我很忙么?”

多尼的声音有点恍惚,似乎还没有睡醒,“你还问我呢,我的一张卡快给你的朋友打爆了……哦,宝贝,别这样,这是我很重要的朋友。”

听筒内传出女人嬉闹的声音。

楚云飞摇摇头,这家伙,事一了,就又露出那好色风流的死相了,“我说,我会有朋友给你打电话?班克斯知道你的电话号码么?”

“废话,当然是你朋友,”多尼抱怨了起来,“你还记得不记得,在法国你救过一个中国人质?”

原来,是那个国家安全局的肖逅呀,“他有什么事么?”

这说起来就有点复杂了,想必肖逅把那张多尼的卡当做楚云飞的卡了,想通过那个号码来联系楚云飞。

多尼手里好几张卡,有天为避个缠人的小姑娘,无意中把这张卡放到手机里,却意外地收到了肖逅的好几条短信,短信中说有重要事,却没说是什么事。

多尼把电话打回去,对方已经关机了,这很正常,因为短信可以在系统保留一段时间的,时间长短在运营商的设定上。

然后,居然是中国驻法国大使馆打来了电话,问多尼是否知道楚云飞的去向。

多尼倒是没感到什么意外,因为,他压根就不知道楚云飞他们和中国政府之间还有什么纠葛,这点上,总释放含混信息的楚云飞难辞其咎。

所以多尼很友善地接听了这个电话,并承诺帮他们联系到楚云飞,由于对方似乎怀疑楚云飞就在他身边而不接电话,奸猾的多尼就没告诉对方楚云飞的手机号。

讲义气的代价就是:多尼的手机就像上了闹钟一般,每隔一小时定期响起,对方似乎真的很着急。

这时的楚云飞早混到俾鲁弯这里了,多尼用尽办法,死活联系不上他们,实在不堪骚扰,只能换卡了事。

确定了多尼确实不知道大使馆找自己什么事,楚云飞又寒暄了几句,挂掉了电话。

刘宁和成树国早就听到了“大使馆”这敏感的字眼,凑了过来,大家相互看着发起呆来。

最后还是刘宁发话了,“要不,云飞,回个电话吧?”

楚云飞对肖逅的印象确实不错,抿抿嘴点点头,“好吧。”

成树国可以被直接忽视了,不过他还是表达了自己的观点,“我也觉得该回个电话,英国那里就在找咱们,法国这里也这么着急,看来真的有事。”

楚云飞按着多尼留的电话,拨通了中国驻法国大使馆的电话。

接电话的是个甜美的女声,说得是法语,楚云飞没管那么多,直接用汉语,“我是楚云飞,听说你们在找我?”

电话那边沉默一阵,有个男人接过了话筒,“你好,楚云飞同志,联系上你们,很不容易。”

同志?楚云飞摇摇头,一种说不出的滋味涌上心头,沉吟一下,才苦笑着说,“我们都叛国了,还说什么的同志?”

那男子似乎早对这个答案有准备,在电话那头轻轻地笑了一声,“呵呵,你们的情况,我们都是了解的,事实上,在你们脱离开部队的日子里,没有做什么伤害祖国的事情,我们很高兴。”

啧啧,这话说得很有水平啊——“脱离开部队的日子。”

楚云飞又是一声苦笑,声音越发地干涩,“多谢你这么说,事实上,我们已经是索度国的公民了。”

那男子又是轻快地一笑,“你们的情况,有些特殊,不过,祖国并没有忘记你们。哦,对了,忘记自我介绍了,我姓彭,是肖逅的同事,你可以叫我老彭。”

好话一句接一句,不问可之,对方确实是有事情来求这三个人的。

楚云飞和战友们相互交换了一下眼神,说话不再客气,我们这里还有事呢,“肖逅呢?怎么他电话关机?”

男子沉吟一下,不再笑了,“这个事我们等等再说,对了,楚云飞同志,你是要找马哈苏德报父仇么?

报仇这话,楚云飞甚至没跟肖逅提过,不过,以一个国家的力量,查出这点事似乎也没什么意外,楚云飞点点头,尽管对方看不到他。

“是的,否则的话,你们早就联系上我了。”男子的声音传了过来,“既然你有这样的心思,而且以你现在的身份,我们不会阻拦,我还可以向你提供个重要情报,马哈苏德不在沙特,他回巴基斯坦了。”

这算哪年的重要情报?楚云飞实在是有点受不了,所以不打算领情,“谢谢你的情报,事实上,我们已经在巴基斯坦俾鲁弯省呆了一个多月了。”

“哦?”听起来,那男子似乎有点意外,不过声音马上就恢复常态,“看来你已经找到凶手了,怎么样,事情还算顺利么?”

“不算顺利,”楚云飞觉得事情没什么好隐瞒的,“我们还没有找到人,而且,我们还有人受伤了。”

有个克努蒂的穆斯林受伤了,楚云飞这话不算假话,正好也可以试探一下对方反应。

男子的声音急切起来,听得出来,他是比较愿意表达自己的关心的,“哦?是谁?成树国还是刘宁,伤得厉害么?”

楚云飞很平淡地做了回答,“谢谢你的关心,还好,不要紧。”

“哦,那就好,”男子显然不计较楚云飞的粗略回答,“其实,我还见过刘宁的父亲刘群。”

楚云飞还没想出来要说什么,对方又开始了关心,“克普塞部落也有跟我们国家关系密切的人,你需要他们的帮助么?”

楚云飞是个吃软不吃硬的主,这样的语气,很难引起他的反感,尽管,看来对方真的有重要事所求。

当然,他也不想让对方因此而小看自己的智商,“这事等等再说吧,你先说说你找我们有什么事吧。”

\section{第一百九十章 肖逅出事}

那男子的声音终于变了,变得惆怅了起来,“肖逅,他已经牺牲了,他临死的时候,想得到你们的帮助,可惜,没联系上你们。”

原来,肖逅负责的那件国宝回归的事,经过他的不懈努力和其他人的共同配合,历史博物馆的代表们最终拍到了那件青铜獬豸樽。

但好事总是多磨,就在他们办完所有手续打算回国的时候,法国也下起了大雪,那时还是刚过元旦没几天。

大雪一下就是三天,法国的所有航班都暂时停飞了,直到又过两天才恢复通航。

空中交通的暂时停顿,使得大批乘客滞留在法国的各个机场,一时间各个机场人满为患。

巴黎的小偷,那是世界闻名的,守着这么个宝贝疙瘩,这几天可是把国家安全局和历史博物馆那几位累得够戗。

所以,当发往首京的班机开始持票上人的时候,筋疲力尽的几个职业保镖,走得越发地谨慎,这样速度一慢,就有人强超。

飞往首京的,自然也是中国人,强超的人,是中国北方一个大城市的一个考察团。

考察团以该市一个副市长为首,他们本来订的是前一天的机票,按“先来后到”的说法,那是不能登机的,需要听从法国机场的安排,让那些等得更久的旅客优先上机。

但用该团长的话,“我管的人,也有巴黎市这么多了。”后面还有不便于翻译的国骂出口,于是终于强行登机成功。

众多苦熬数日的中国同胞只能相视无语,他们没什么权势,那等下一航班自然是理所应当的事。

肖逅的同伴们不得已,又想尽力低调,不生什么事,于是终于被活生生分成两拨走路。

不过,说实话,“戴高乐机场”的小黑屋,是出了名的,那小黑屋不单代表对中国人不友好,更是伴随着客运行李不安全容易丢失等一系列丑闻。

有这强势市长,肖逅他们也占便宜不少,那市长把欧洲购物所得的十几个大包硬生生带进客舱内两个,通过紧急交涉,那个青铜獬豸樽也得以借此先例进入了客舱,而不用担心在行李运输途中忽然消失不见。

肖逅可真算得上早来迟走了,他又被划为推迟走的那拨。

没赶上这趟航班,肖逅趁着算放假的机会,计划索性在法国游玩几天,“难得浮生半日闲”嘛。

不过,这世界上有的人,是注定不能清闲的,肖逅在巴黎逛了才两天,就接到了新的任务通知。

这次是偷拍一家法国工厂的生产流水线。这种类型的生产线全世界也不过三条。

这个叫“罗蒙特”的工厂,生产的是几种精密的数控机床,中国某大型国企正要进口某个型号的机床。

法国人的要价很高,四千六百万美圆,远远超过中方两千八百万美圆的心理承受底限。

中国的企业不想白出这份冤枉钱,可另两个能生产这种机床的企业一在美国,一在日本。

美国那企业不用说了,他们公司的所有产品都是严禁向红色中国出口的,没有任何的商量余地。

日本企业倒是愿意向中国企业便宜出售这种机床,但是很遗憾,那个中国企业不想买,因为,在设备出售这种事情上,中国人吃日本人的亏太多了。

随便举个例子,当电话程控交换机兴起的时候,为了替换老式的纵横制交换机,中国由政府牵头,买了日本“富士通”公司两条程控交换机生产线,来生产交换机,并且为日本企业在中国的电信设备投标中大开绿灯。

甚至,那时电信局的通用通信协议都是日本的“T1”标准而不是欧洲的“E1”标准,数字通讯的协议是24时隙而不是32时隙的。

那时的中国电信局,进去一看,全是日本的设备,因为日本人的价格如此便宜,让所有人都相信了“中日人民世世代代友好下去”不只是口号。

但是,当“富士通”和“日通工”的产品遍布大江南北,长城内外的时候,问题出现了。

所有设备,也包括生产线,它们的维护,价格贵得惊人,那时候,曾经有人笑谈,“买几颗螺丝钉的钱,就够再买条生产线了。”

事实上,笑谈离现实非常接近。

意识到问题的中国政府,没有再忍气吞声下去,直接推倒了大部分的日本通讯设备,重新高价采购欧洲和美国的产品。

但是,那学费交得实在太高了,将近三百亿的设备在使用期内就被直接送入了熔炉。

不过,从那以后,日本的核心通信产品再难进入中国,也是事实。“岛国思维”实在是只能拿住那些短视的弱者。

所以,法国就是那个中国企业的第一选择,但价格差距如此之大,只能拜托国家安全局出马了。

这种事情,国家安全局实在是做得不少,肖逅甚至可以想象得到,拿着号称“收购”来的自己拍的照片,中方企业代表可以洋洋得意地把设备价格压到两千万甚至一千万美圆。

中国人的仿制能力世界有名,几张照片在手,给人的打击绝对是致命的,而且,曾经有过那些不信邪的,中国人还真的就仿制出了类似的东西。

但这计划也有不完美的地方,那就是:藏踪匿迹,实在不是肖逅的强项。

肖逅自然是明白自己的水平的,而那些专业的同事们都已经回国了,事情催得如此紧急,他就想到了楚云飞他们。

以那种强悍的实力,做这种小事,该是轻而易举的吧?

那天短短的一个小时的交谈,肖逅基本上已经掌握了对方的可信任程度,判研人的心理,这本来也是他的长处。

楚云飞他们既然肯无故地救他,就没必要再在他面前撒谎,事实上,他们那种草莽龙蛇的气息,在肖逅面前是无论如何也掩藏不了的。

而且,他们似乎在法国也是很有能量的。

在向上级部门申告了自己的要求并附加上自己的判断后,他“招收外援”的要求很快得到了批准。

但是,这时候的楚云飞已经飞往了巴基斯坦,肖逅死活打不通对方的电话,只能无聊地发发短信,指望对方什么时候能开下手机。

他不知道,那手机卡是多尼的。

后来肖逅只能孤身进去拍照了,虽然他受过这样的训练,怎奈这实在不是他的强项,进去没半小时,对讲机里只传出一句话,“完了,被发现了”,接着就是警铃大做。

听到工厂里彻耳的警铃声,在外面接应的肖逅同伴紧急呼叫他,却没有丝毫反应。

\section{第一百九十一章 快乐建立在……}

三天后,肖逅的尸体出现在距离那工厂三十公里处的高速公路旁,整个人被碾得稀烂,要不是他体内有“外勤人员”的暗记,没人认得出来是他。

整个人都模糊不可认了,自然看不出他死前受到过什么折磨。

可怜,为国家而捐躯的烈士,因为行业特殊,收殓他都不能由国家出面。

消息传回首京,国家安全局顿时对这个案例高度重视起来,因为里面的味道实在是太不寻常了。

问题的关键是:肖逅为什么被杀?他到底看到了什么不该看的东西?

这怕是比少花那几千万美圆还要重要的事。

要说仅仅是有人发现了他商业间谍的身份,就下这样的毒手,那实在是太拙劣的借口,用来骗骗火星人还差不多。

大国之间,每年不知道要有多少相互渗透的情报人员,要是这样的小事都如此斤斤计较,大家都不用活了。

所以大家以为,肖逅被抓,会回报来的,不过就是他被“不受欢迎”地驱逐出境;或者被暗地扣留,等着法国有相同级别的人落入中国人手,相互交换而已。

最多,也不过就是成为政府间什么事的谈判筹码,在天平的一头加点可有可无的分量而已。

甚至,都有可能碰上那种不太精通此事的公司,把他直接扭送当地警察局,作为民事案件来处理,这种情况还不在少数呢。

至于说因为反抗而被击毙,那更是笑话了,安全局,那可是全部受过专业训练的,逃不脱的情况下,只有束手就缚等待救赎和饮毒自尽两条路。

而肖逅执行的这事,他只需要乖乖“束手就缚等待救赎”就可以了。当然,他不能吐露任何东西,除非,他也打算叛国。

收到上级部门的通知后,中国企业马上积极联系日方厂商,一面是为了撇清自己,另一方面,也在为打压法方气焰讨价还价做准备。

不得不承认,中国人,是世界上最擅长“拖”字诀的民族,一面跟日本眉来眼去,一面又暗示法方:条件都已经成熟了,只是因为资金,需要上级核实。

当然,纯粹就设备买卖的事来说的话,法方如果愿意让步,合同会在眨眼间签定,这不仅仅是承诺也是事实。

一条人命换近两千万美圆,自然划得来,毕竟,人已经死了,不是么?

就在这马拉松般的谈判中,国家安全局的精英再次汇集巴黎,目标就是那个诡异的工厂。

当然,智慧如斯的团体,不会再踩上同一团狗屎,查探是必须的,而手段却要谨慎很多,不是付不起人命,实在是……没有机会再供浪费了。

查探的渠道,是多方面的,甚至有个身高腿长、貌美如花的女秘书临时被调进了驻法国大使馆,一时间频繁出入各种社交场合。

在各种先进仪器的配合探察下,国家安全局的精英们不得不承认,这次怕是碰上大家伙了,那工厂里的防范,实在是太严密了。

法国很有几个工厂是以制造精密仪表在世界上著称的,工厂里各个防备和警报系统,实在是细密得连针都插不进去,但从外表却丝毫看不出来。

国家安全局的精英们甚至怀疑,以肖逅训练纪录上的那点水平,怎么能潜入将近半个小时才被人发现,难道他进去了以后再没有挪动过?

事实上,正如大家所猜测的那样,因为发生了“肖逅事件”,那工厂的警戒级别直接提升了好几个级别,现在就是最高级。

而从其他渠道得来的消息,却证明这个工厂,实在是没有什么值得大书特书的地方,他们最著名的,也就是那条世界上只有三条的生产线生产出来的东西而已。

老鼠拉龟,无处下口。

面对如此诡异的情况,中国的国家安全局实在是一筹莫展,于是他们得到了楚云飞三人的资料,据死去的烈士分析,三人的可信度极高,而且实力和能量惊人。

三人国内的相关资料传了过来,李南鸿的口供传了过来,肖逅被绑架后的遭遇拿了过来,维和部队日志也发了过来。

再加上英国和法国大使馆提供的资料,三人的精彩经历让国家安全局的精英们大跌眼镜。

不是说高人们没见过世面,严格说起来,他们这点事还真算不上有什么太高的技术含量,说惊心动魄也谈不上。

只是,那种血腥和残忍,却是足足地让大家吃了一惊,更关键的是,三人不凡的身手和非同寻常的智慧,使得他们屡屡地履险而无夷。

如此人物,实在是国家安全局执行外勤首选的人才,不吸纳进来,实在是可惜。

不过很遗憾,楚云飞三人是绝对没有被吸纳的可能的。

想成为国家安全局这个特殊的部门里的一员,首要的条件就是忠诚。楚云飞他们遭遇的事情,虽然是其情可悯,但绝不能掩饰他们深藏在内心的张扬的个性。

实力不行,可以通过其他方式来逐渐培养和提高,资质实在有限的话,也能做些力所能及的事。但行行精通、事事明了却又不够忠诚的人,还不如不要。

换句话说,中国这么大,找些既有实力又对国家忠诚的人,实在是很容易的。

但是,话又说回来了,不能进国家安全局,并不代表就不能为祖国效力,摆着这么强悍的势力,可能不用么?那些高人们又不是傻子。

当然,刘群和成解放孜孜不倦的活动,也多少起到了一些作用,虽然外面的世界风平浪静,但实际这几个士兵的遭遇早引发了更深层次的思索。

可以肯定的是,政策是不会为此动摇的,实在是太小的事了,但它带来的反思,足可媲美《低谷里的花环》一文所带来的冲击,纵然不被认可,但起码在道德层面上是受不到太多的谴责了。

而且,“慈不掌兵”也不是空话,反思也仅仅是单纯的反思而已,没有任何真正的实际意义。

通过一系列的汇报和请示,国家安全局最终确定,应该增加一条获得情报的途径:邀请楚云飞他们参与调查此事,他们能获得的回报就是:国家不再追究他们的责任,允许他们低调回国。

之所以能有这种情况发生,肖逅和李南鸿功不可没,他们从侧面证实了这三个人对祖国的忠诚。

这桩事件虽然对肖逅有些残忍,却成为了几个士兵回国的契机。

说得更加残忍一点的话,如果肖逅没死还联系上了楚云飞,楚云飞他们也慨然帮忙完成了任务。那这个简单的商业任务还未必能让楚云飞三人轻松地实现回国梦呢。

级别不同,能承诺的东西自然也不同。

当然,这与刚卡国不再一力追索三人也有一定的关系,事实上,那些非洲人,怎么说呢,他们也明白自己国家的斤两。大家叫起真来对他们并不好,毕竟,中国已经给了他们面子了。

总之,对楚云飞他们而言,回国的路,是实实在在地展现在了三人面前,不再是虚幻飘渺的空中楼阁了。

摆在国家安全局面前的问题是,这三个野惯了的年轻人,会不会为了那对故土的思恋,不计较母亲对儿子的抛弃,而放弃在国外的大好前景,去踩法国那变幻莫测的地雷呢?

所以,“老彭”的话,说得非常地暖人肺腑。

\section{第一百九十二章 一定要回去}

负责与楚云飞他们沟通的,是安全局特勤外三室接待科的彭辉煌。

这“接待科”的名称听起来寓意平常,很有人情味,其实不然。这里可不是那种迎来送往的摆设机构,它的性质是上呈下达命令,是特勤真正的核心机构,掌控生杀大权的地方,至于那些出生入死打打杀杀的国安精英,虽然在影视媒体或者文学作品里很是威风八面,但事实上只是那种纯粹的“人型机器人”,很单纯的执行机构而已。

彭辉煌很恰当地表示了对三个年轻人的同情,也把需要帮忙事情交代了出来,事实已经证明,电话那头,是很聪明的三个年轻人,就算玩花样,也没必要在这个上面玩。

楚云飞当时就被巨大的幸福感淹没了,能回国了?能回去见自己的母亲和琳琳了么?

所以,当彭辉煌再次婉转地问起需要不需要国家在克普塞部落的的关系时,楚云飞毅然决然地告诉他上个问题的答案:我需要同我的同伴商量。

三人都勉强算得上富翁了,如果成刘二人不想回国了,那就不要安全局的情报支持了,经历了太多的事情,楚云飞已经不想再跟这种性质的部门打交道了。

其实,楚云飞没想到,他这孤身的报复,实在是大快人心的事情。

虽然,这世界上最污浊的群体就是政治家们,但不代表政治家就没有自己的个人情感。

譬如彭辉煌就是如此,虽然也是个副处级,但他甚至连政治家都算不上,不但这样,他还是是在为级别高于他的政治家们提供一些见不得光的支持。

马克思曾经说过:资本来到人间,从头到脚都流着血和肮脏的东西。彭辉煌他们做的事情,基本上也和人性无关。

不过,脏活抑或干净活,都得有人去干,社会分工不同而已,不是么?

放下电话的彭辉煌,脸上漾起了罕见的真诚的笑容,他走回同事中,笑着摇头,“这帮家伙,真厉害,果然是杀到巴基斯坦去了,唉~~~真是好汉子!”

看到彭处心情不错,难得地聊聊天,就有人长出一声气大胆发表意见,“呼~~~真不简单,这样的事情,实在是大快人心,咱们不方便做的,居然苦主自己去做了。”

都是国家安全局的人,过分的话,没人愿意说,能有人说到这步,已经不错了。

不过,彭辉煌的脸在下一刻苦涩了起来,“唉,我忽然有点担心,这三个家伙,别在巴基斯坦整出太大动静吧?那样咱们到时候可就不好收场了。”

立刻就有那些睿智的参谋们消除了他的担忧,“他们一直渴望回国的话,就不可能弄出太大动静;如果不想回国,那他们已经是索度人了,弄出再大的事情,也跟咱们无关吧?以这三个人的脑瓜,绝对会想到这些的。”

楚云飞想的没错,他一挂断电话,成树国和刘宁立刻就明确表示:一定要回去!

他俩理由基本相同,最大缘故居然都是:不能让父母在别人面前抬不起头!哪怕丢掉国外的一切财产,也要回去堂堂正正地做人。

中国人,毕竟是中国人,“君君臣臣父父子子”的观念,实在是深入骨髓乃至灵魂的。

“那这样吧,你们俩回去接活,”楚云飞沉默半天,撅撅嘴,“这里我一个人就行了。”

既然大家都愿意回去,不用让他俩再在这个危险地方逗留了吧?已经能回国了,弄个伤残什么的就不好了。

“扯淡”,粗口出自刘宁,“干完这票再回去接活,要是把你丢在这里单干,大家不回去也罢。”

楚云飞扭头看看成树国,成树国脏话直接出口,“看鸡巴的看,你想死我俩直接弄死你就算球了。”

楚云飞笑着摇摇头,再次拿起了电话。。。。。。。。。。。

这次收工,再下山来,就是一个月以后的事了,雪后的山路实在太难走了。

三月底的巴基斯坦,大地开始回春,厚厚的冻土已经开始有解冻的迹象,地表的积雪开始消融。

山路依然是光滑异常,不过,多少是能走走人了,楚云飞他们找上了克普塞部落,找到了那个安全局介绍的长老。

这次出来,只有辛汗和另一个穆斯林跟随,有一个穆斯林家里要办喜事,又要准备开春的耕种事宜,没有跟来。

出面联络的,自然是辛汗,楚云飞他们三个中国人实在是不方便露面的。

听说楚云飞他们是为了马哈苏德的事来的,那个叫辛亚拉的长老匆匆出来相见,他已经被此人折磨得受不了啦。

严格地说,辛亚拉长老是个非常传统的穆斯林,所以,他对马哈苏德的意见格外地大,“他不知道尊重长辈的意见,现在又在部落里搞得乌烟瘴气,带坏了不少纯洁的孩子。”

辛亚拉边说边点头。

“由于部落里反对他的人要远远多于支持他的人,他拉了几百号人,在新纳山谷搞了个训练营,完全是‘基天’的那套东西。政府军拿他也没办法,那里谷深林密,每次围剿他的时候他总是能事先得到消息逃进深山。”

楚云飞微微一笑,那笑容格外地冷酷,“山高林密我们不怕,他逃到哪里,我们就追他到哪里。”

辛亚拉再次点头表示否定,“不是的,他要躲避的,是政府军,你们要找他的麻烦,整个俾鲁弯的有心人都知道了,面对私仇,他是不能躲避的,否则他的名声就全完了。”

是的,马哈苏德没有躲避的余地,他虽然杀的人不多,但涉及的命案总是有二十几条的,这时的他,正在新纳山谷大发雷霆呢。

“你怎么就那么笨呢?到底查出是谁干的没有?”

他的三弟桑丘是一个异常壮实的年轻人,他的得力助手和忠实追随者,桑丘的眉头紧皱着,“没有头绪,他们非常残忍,从来不留一个活口,你想,连‘基天’的人都不知道他们是谁,只知道,他们的火力非常猛烈,从贝西哈兰村留下的痕迹就看得出。”

马哈苏德听到“基天”,又开始骂骂咧咧,“亏他们也敢称自己是真正的勇士,连自己人的仇都不敢报。”

“基天”为避免过分的报复,四处忙不迭地宣布和马哈苏德划清界限,这事当然瞒不过马哈苏德,尤其为了达到目的,克普塞部落附近这种留言是最多的。

实在是有点“是可忍孰不可忍”。

\section{第一百九十三章 风起新纳山谷}

“基天”作为个极端组织,绝对不会这样饶恕冒犯他们的人的,但是,在这种神出鬼没、遍地开花的袭击下,他们也只能选择暂时的退避,人要是都死了,还发动什么“圣战”?

马哈苏德对这点心知肚明,所以他才格外地愤怒,“基天”这么做,摆明了是要把自己奉送出去了。

或者,“基天”没准也在调兵遣将,打算利用自己这个“诱饵”来探察对方底细甚至报复呢,这谁又说得清楚?

他也不想想,“基天”是有政治目的的组织,怎么可能为他的个人恩怨买单?

“不管怎么说,我是在新纳这里呆定了,只要他们敢踏进这山谷一步,我就把他们彻底埋葬在这里。”

是的,新纳山谷这里,山高林密,而训练营这里更是丘壑纵横,陷阱密布,各种阴毒的埋伏层出不穷,实在是阴人的宝地。

也只有政府军那种浩大的势力,才敢时不时地来这里扫荡一下。

想到政府军,马哈苏德略微有些头疼,“桑丘,你说政府军会不会参与进来,他们要先来一下扫荡,我这里还真有点头疼。”

政府军每次围剿马哈苏德的时候,他都会事先得到线人的通报而遁走,不会有什么人员伤亡。

但那些恶毒的陷阱,通常是会遭到致命的破坏的,围剿毕竟是围剿,不是儿戏。

桑丘倒很不以为然,“那有什么,你不会跟那人打个招呼,让政府军暂时停止对我们的敌对行动么?”

马哈苏德听得一惊,四下看看,抬腿踢了他弟弟一脚,声音也低了下来,“闭嘴,这话是随便说的么?别人听见的话,咱们就完了!”

别说,门外,还真有一个黑影趴在墙角,手持个什么东西放在木屋的墙上,侧着耳朵在偷听。

听到屋里的人警觉了起来,那黑影忙不迭地收起那用来偷听的工具,蹑手蹑脚地消失在黑暗中了。

黑影走得还是早了点。

马哈苏德长叹一声,自己反倒提起了这敏感事,“咱谁也不知道对方是什么来头,所以,这个关系,能少用还是尽量的少用,要来的如果仅仅是一支小武装,可就太不划算了。”

说到这里,马哈苏德的脸部轻微地抽搐起来,他又想到了另外一种可能:这些号称报仇的人,别就是那人用来灭自己口的吧?毕竟,自己现在已经算没什么用了。

不可能,马哈苏德摇摇头,否定了自己的猜测:我手里还有几百武装,那又是个大名鼎鼎的民主国家。

“大不了,我把一切都公布出来,想要我死?哼,没那么容易。”马哈苏德有点轻微地失态,居然把想的话低声念叨了出来。

两天后,楚云飞一行五人找到了新纳山谷。

山谷由于有训练营的存在,从来人迹罕至的地方居然被踏出了一条小路,而且路还不算太窄,车都开得过去。

由于有人踩踏行走,小路上的积雪基本上化得差不多了,楚云飞他们并没有留下什么足迹来供对方调查。

夜间,站在一侧的峰头向山谷里望去,房屋有将近三十间,都在山谷的中央位置。

很快,大家就发现了四个比较突出的位置,一旦开打,可以很方便地压制住山谷中可能的火力。

管中窥豹是不行的,而楚云飞又是出了名的精细。经过两昼夜的观察,大家终于惊讶地发现,这样能占据地利的地方,居然有十二处之多。

出现这样的情况,显然是不合理的,这点纰漏都发现不了的话,他马哈苏德还搞什么的训练营?

那些地方肯定会有人值守,而且,说不定力量还会很强大。

于是,五个人聚在一起,划分了一下任务。

刘宁带着那俩穆斯林回去采购些东西,包括绳索、铃铛、钉子、钢丝,线香什么之类的。

关键是,要弄到大批的反步兵跳发雷,前苏联在阿富汗大量地使用了这种东西,在这里应该很好买到的。

楚云飞和成树国本来就是特种兵专业出身,扫荡那十二个突出部的任务就交给他俩了。

各个突出部果然是有人值守的。

为保险起见,楚云飞他俩一开始并没有直接摸到那个最具威胁的制高点,而是在十二个点中取了一个居中的地方。

那里居然埋伏了六个人,两个固定哨,两个游动哨,两个潜伏哨。

那潜伏哨还是潜伏在雪下。

固定哨一在树下,一在突出部的岩石上,很好发现,两个游动哨,是来回走动的。还好,这是楚云飞他们对付的头一拨新纳山谷的人,对方的警惕性不是很高,其中有一个基本处于梦游状态。

那俩潜伏哨,严格说其实不算潜伏哨,是一个靠近岩石的地堡,那个位置有点靠后,虽然不利于压制山谷内的火力,但绝对能压制临近两个突出部的火力。

一切的准备,都是为了应对外界的袭击的。

还好,楚云飞敏锐的感知能力不是吹出来的,小心翼翼的他发现了所有的不妥,包括地上的一个深达三米的自然裂缝,那里被掩饰起来当个陷阱使用。

第一个突出部就这样被解决了,六具尸体被直接扔进了那个裂缝。

这样说起来,马哈苏德似乎选择藏身的地点不太好,其实,并不是这样的。

原本马哈苏德选择这里做据点的因素就是为了政府军,而不是为了防范什么超强势力的单个个体。

是的,地形是不太理想,但他当初想的就是怎么方便脱身怎么来,而不是以这里为据点强行对抗政府军。

所以,那些突出部的弱点,对他来说,基本是可以忽视的,因为,对方真有心思占据这里的话,有这时间,他都不知道已经跑到哪里去了,甚至他有信心跑到伊朗首都德黑兰。

他只需要知道对方能怎么接近他,这里所有的安置,都是预警第一,对抗第二,或者没准排到第三第四都有可能。

发现了对方在守卫上的并不是很谨慎,楚云飞马上和成树国分道扬镳,各自对付一个不起眼的突出部。

楚云飞对付的那个只有两个人,成树国很不幸,他那里居然有四个人。

实力有差距,目标有差距,楚云飞没事,成树国那里却发生了意外。

\section{第一百九十四章 不要随便拉屎}

成树国仔细观察了将近一个小时,才最终确定这里埋伏着三个人。

一个坐靠着树,全身捂得严严实实的,脑袋偶尔轻微地转动一下,这家伙该是固定哨。

还有一个固定哨跟他背靠着背贴在同一棵松树上,但那人个头矮小,被那个穿得极其臃肿的家伙的身影挡住了大半个身子,很不好发现。

还有一个游动哨不停地在这块五十平米见方的突出部来回地走动,大概十分钟走动一次。

等那游动哨才巡逻完一遍,成树国放下步枪,摸了上去,肥大的袖子遮住了手中寒光闪闪的利刃,先把这游动的家伙干掉才是正理。

要不此人一时兴起,随便打乱巡逻节奏,可就太容易出意外了。

刀光一闪,才巡逻完毕的家伙就倒在了血泊中。

成树国小心地让开对方脖颈处狂喷的鲜血,把人轻轻地放到了地上。

那俩固定哨的位置非常讲究,成树国想从二人视线的盲点切进去的话,势必要经过一片根本没人走过的雪地。

虽说雪化了又冻冻了又化,表面上形成了一层硬硬的冰壳,但是这背阴之处,实在不能判断冰层到底有多硬。

成树国是北方人,自然知道,万一冰壳破裂,会有那种“喀啦”的声音,或者说,这里也有可能根本没形成那地表一层的硬壳,只是变成那种中间孔隙很大的硬质雪面。

那样,不小心也会发出“吱扭”的踏雪的声音的。

成树国琢磨了一下,确定实在没有更好的解决办法了,只好硬着头皮,一手持匕首,一手持钢针踏上了那片雪地。

那雪果真是没有冻实,这样的话,真要小心行进,还是没有太大问题的,踏雪,也不是那么容易发出声音的。

成树国才慢慢挪了两步,就反应过来一件事:操,自己果然真是个猪头,就不会冒充那个游动哨么?

幸亏云飞不在身边,要不丢人就丢大了。

于是,成树脱掉身上的白色披风,大模大样地向固定哨走了过去。

但是,很不幸,那固定哨中靠近成树国的一位,居然开始说话了,不知道他在跟谁说话。

成树国不懂乌尔都语,而且,那家伙说的也不是单纯的乌尔都语,这种情况下,成树国实在是不知道该怎么拖延下去,只能发出“唔”的一声,那是全世界通用的、表示疑问的发音,然后果断地扑了上去。

也许,是自己走路的习惯不像死了的那位?早知道这样,该多观察下那死鬼的举止的,这是成树国心里唯一的想法。

这边这位更纳闷,他只是想问问即将过来的战友,“你带手纸了没有?”却没想到战友就迎面恶狠狠地扑过来,拜托,我不要了还不行么?

他穿得实在是多了点,所以脖颈上中针处插得不是很深。

一声闷哼,想上厕所的这位捂着喉咙在地上打起滚来,没纸屎也出来了。

背对他的战友警惕性也不算低,迅速地掉过头来,手中的步枪举起。

就在这一刻,成树国已经扑了上来,手中匕首直取对方喉咙,逼人的寒气在瞬间掠过。

瘦小的身体,没有发出一声呼喊就栽倒在地上。

成树国不敢怠慢,手中匕首直取那位还在地上闷哼的手纸朋友。

可他的刀刚刚插入对方喉咙,“砰”地一声枪响,在寂静的黎明中传出老远老远。

一块岩石后面,射出了罪恶的子弹!

这里居然还有个人!

此人其实是在这里睡觉的,是那要手纸的声音惊醒了,结果,后面连续的呼噜呼噜的沉闷惨呼,使得他下意识地拿起步枪发射,至于目标,他倒没看得很清,他甚至以为有人在拿弓箭攻击这里,因为他没听到任何的枪声。

不过,一旦遇警,必须先发出警告,这是教官再三强调过的,山谷里,有太多战友的生死掌握在哨兵手里的。

成树国身子一趴,抖手又是一枚钢针,情急之下,准头欠缺,打在了岩石上,激起一溜火花。

无声的袭击,果然是弓箭!岩石后的这个胆小的家伙越发肯定了这一点,但他不敢伸头观察,反正外面还有个同伴呢,他把成树国当成战友了。

但是,他的枪口依然向外,源源不断地发射着子弹,在他的心中,这是在火力声援外面的同伴。

但搁在成树国眼里,这种行动的挑衅味道太浓了,他拔出手枪,双手稳稳托着,一枪就掀飞了对方的天灵盖。

这个时候,听到枪声的楚云飞已经冲了过来,不过,路实在难走了点,事实上,楚云飞那里战斗结束得快,就算没这枪声,他也快到了。

两人碰头,边挖钢针边说话,还没来得及说上两句,下面已经传来了隐约的人声。

看看天色,已经快要亮了,两人对视一眼,成树国掏出两个手雷就扔了下去,楚云飞手中的步枪也响了起来,“哒、哒哒”,清脆的枪声划破了黎明的寂静。

下面上来四、五个人,他们身后不远处还有七、八条身影也在向这里奔跑着,听到枪声,前面几个人马上就地卧倒,后面的人却是跑得越发地快了。

楚云飞皱了下眉头,“树国你先走,去仓库等我,记得别留下脚印。”

他们的临时仓库在山峰阳面的一个小坡上,一个灌木丛生的地方,地势不太好但非常隐秘。

成树国自然知道现在不是客气的时候,把身上的手雷放下四颗,掉头就走。

突出部,自然有突出部的优势,居高临下,视野开阔。

楚云飞摘掉手套,端起冰冷的AK47,缓缓地瞄准。

“哒哒、哒哒”,两个点射,带走了两条人命。

剩下三人听得同伴的惨叫,就有一个机灵点的没命地翻滚起来,终于滚到了一块石头旁,躲了起来。

楚云飞又一个点射射杀一人后,最后一个人也学着战友的样子向另一块石头滚过去,不过情急之下,他滚得太着急了,方向没控制好,脑袋重重地撞上了那块岩石,当场就晕了过去。

太滑稽了!看得楚云飞直想笑,不过他现在可顾不上多想,瞄准那昏迷的家伙,稳稳地扣动了扳机。

卡嗒!撞针的声音在这时显得格外地清亮刺耳。

中奖了,AK47也会卡壳?还是在这么要命的时候。

那岩石后的家伙,已经开始探出枪管还击了!

\section{第一百九十五章 新纳初体验}

初时子弹还不算密集,但不到五秒种,枪声大作起来。

楚云飞就地一滚,换个地方迅速地探头看了一眼,原来,远处那八个人已经逼近了,有三个半跪在地上向这里射击着,还有五个人分散开来,猫着腰在迅速地向这里跑过来。

楚云飞抿抿嘴,拉开枪栓,把那颗哑弹退了出来,再重新合上枪栓,探出身子就是一个点射。

刘宁要在的话,肯定又要感叹几声了,这动作绝对是刚卡边境的翻版,楚云飞只需要探下身子随手一枪,对方总要有人中弹倒地。

几个点射过后,又击毙了四人。

对方显然意识到了楚云飞的厉害,纷纷找了掩体躲藏了起来,偶尔才冒头出来开一两枪。

不得不承认,恐怖分子的心理素质,比刚卡那种吃兵粮的所谓“士兵”要强了很多,接下来,虽然躲藏起来的人在射击时又被楚云飞击毙两个,但剩下的三个人还是找准机会就探头射击。

就在这短短的十数分钟内,对方又派来了援兵,足足有三十人。

距离远了点,那些人身上扛着的武器模糊不可分辨,不过,可以确定,有些不是步枪,没准是机枪或者火箭筒什么的。

撤了,楚云飞冲着远处的人群就是三个“两连发”,拎着步枪,扬长而去。

这些来参与进攻的克普塞人没有想到对方会这么溜走,因为,抵达突出部的山路是被他们占据着的,另一边过于陡峭,再加上还有残留的积雪,不但不好上也不好下。

于是,当死伤惨重的进攻方终于攻上平台的时候,却意外地发现偷袭的人早就溜之大吉了,这里,只有四具同伴的尸体,早冻得梆硬了。

众人正在四下搜寻偷袭者的下落,却听得一声枪响传来,抬头望去,一个人影奇快无比地远去,身形在山石间忽隐忽现,几个跳跃之后,终于消失不见。

只是远处地上似乎又多出了一个横躺着的人,想必是有同伴撞到了这个杀人凶手,被凶手射杀了。

难道说,凶手只有一个人么?这是横亘在所有在场之人心中的疑问。

疑问尚未解开,就有更坏的消息传来,还有两个观察哨也遭到了凶手的袭击,哨兵们全部阵亡。

一时间,新纳山谷内人心惶惶。

马哈苏德眯起了眼睛,脸部肌肉不自然地抽动两下,“嗯,是他们,他们真的来了。”

桑丘年轻气盛,根本不在乎自己这方小小的损失,“来就来呗,反正他们没几个人的,咱们这两百多号人,压也压死他们了。”

马哈苏德狠狠地瞪他一眼,怒火中烧,“你说话动动脑子好不好?没几个人,那贝西哈兰村的事情你怎么解释?”

看到哥哥生气了,桑丘也不敢乱说话了,不过他还是表示了自己的委屈,“那是他们火力强大,大部分人还是被杀死在屋子里呀。”

其实,马哈苏德只想听听桑丘的分析,并不是真的以为楚云飞他们人很多,要不他早就跑了,躲政府军之类的强大势力并不丢人。毛泽东的军事思想在巴基斯坦还是很有市场的,包括那“敌进我退,敌驻我扰,敌疲我打,敌退我追”的游击战十六字纲要。

就算只是私人恩怨,但实力相差太过悬殊的话,暂时躲躲也很正常。虽然这样难免会被人骂成“胆小的懦夫”,可多少也能被大家理解和容忍。

马哈苏德阴着脸,摇摇脑袋,“问题就在这里呀,你看看早上的事,只在九号点上,一个人就杀了咱们十三个人,重伤三人,很可怕的对手呀。”

桑丘自然知道哥哥指的是什么,单兵作战居然能杀死十三个训练有素的伊斯兰勇士,这样实力的确是很恐怖。而且这样的人似乎还不止一个?

不过,他可不想灭了自家的威风,“这些可恶的山老鼠,他们只敢偷偷摸摸地残害咱们的勇士,一定要把这些垃圾挖出来。”

马哈苏德拿起一把雪亮的弯刀,眯着眼睛若有所思地把玩着,沉吟不语。

片刻之后,他终于下了决心,脸上的表情越发地狰狞起来,“好了,二十个战士一组,派出五个搜索队,争取在白天挖出这些可恶的山老鼠,哪怕挖地三尺,我也要找到他们!”

二十人一组地搜索,作为个可能的作战单位,实在是大了一点,太奢侈了。

不过,这实在是没办法的事情,鉴于敌人强大的战斗力,人数再少的话,可能会在遭遇敌人的瞬间全军覆没的,这么大的山谷,想要第一时间支持被袭击的小组,太不可能啦。

哪怕是从山谷中心的营地直接派出接应的队伍,怕是都来不及的。

桑丘也算是久经战阵之人,知道该怎么安排,搜索队派出去了,还安排了将近二十个人在营地外围接应,更安排了一支接近四十人的支援队,随时准备出发。

这么算来,谷中二百三十多人,除去据守几个突出部的四十余人和早上丧命的二十二人,基本上就没什么人空闲了。

白天就在谷中众人提心吊胆的过程中过去了,搜索队一无所获。

这怪不得他们,楚云飞和成树国藏身的地方,在山谷的外侧呢,搜索队尽在山谷里搜索,怎么可能找得到?

现在,又一个问题摆在了马哈苏德的面前,晚上,那些突出部,还要不要人守呢?

守的话,那些零散的力量,很容易被对方轻易地吞噬掉。

可要是不守,整个山谷,那就门户大开,象是一个身无寸缕的美女睡在色狼身畔,防守起来实在太被动了。

关键是,这色狼力量的大小,现在也没人说得清楚。

要是各点增加防守人数,倒也能多些保障,起码坚持到营地里大部队增援是没有什么问题的,可每个点加多少人合适呢?

增加得多的话,支援部队的人数未免就会有点捉襟见肘,增加得少的话,又能抵上什么事?倒是犯了兵家“分兵”的大忌。

再说,增加的人数该是平均分配还是保障重点?昨天的偷袭已经证明了,似乎……所有的突出部都是对方的重点照顾对象。

做贼容易防贼难,主动权不在自己这方啊。

所有的问题都集中在了一起,一时间,弄得马哈苏德头大无比。

\section{第一百九十六章 毒蛇吐信}

思来想去,马哈苏德决定在最主要的突出部1号、2号、3号增加人手,每个突出部增加十人。

其他的突出部,人员不增反减,每个地方只留两人。

大家都要提高警惕,考验伊斯兰勇士的时候到了!

这么做的原因很简单,就是要通过这些勇士的生死,来判断对方下一步行动的意图和方向,顺便也可以试探一下对方的胃口。

这天晚上,楚云飞和成树国没有再分开,而是以楚云飞执行,成树国接应的方式,无声无息地横扫了四个突出部。

那八个负责看守的战士死了四双,他们没有一个人来得及发出警报。

听到消息的马哈苏德不顾大家的阻拦,执意要上山看看袭击现场。

马哈苏德要出门,防卫自然要跟上,不过,众人也没声张,因为那隐藏在暗处的敌人,实在是太危险了。

所以,马哈苏德戴了厚厚的风帽,身上还多加了件防弹衣以防不测。山谷中派出了四支二十人的搜山队,他就混在其中一支里,为了不让人起疑心,他也扛了一枝步枪做样子。

天气还是比较冷的,这样的装束绝不会引人注目。

到得6号突出部时,马哈苏德发现死去的两个下属全是在岩石后面被杀害的,也就是说,当时他们躲藏得是极好的。

那些凶手,是怎么发现这两个战士的呢?

两人都是一刀致命,划破喉咙的刀很快,鲜血喷溅到了五米外的山崖上。

又是仔细地一番搜索,终于有人在山路的一侧发现了令人起疑的脚印,对方应该是有两个人!

但是,凶手实在是太狡猾了,脚印只有弥足珍贵的四、五双,马上又消失不见了,判断不出凶手是如何发起偷袭的。

这里还在绞尽脑汁地分析凶手的偷袭特征,另一支搜山队已经遇到了突如其来的袭击。

这支搜索队负责搜索南面山口的西坡,这个地方,昨天只有一个突出部遭到了袭击,重点的2号突出部离这里也不远,所以,基本上是可以放心搜索的。

搜山和搜平地是不同的,由于山间沟壑和岩石比较多,搜索队稍微一铺开,队员之间就不是时时能够看见了,山岩和树木间人影时隐时现。

乌业兹施是这搜索队里普通的一员,他加入马哈苏德的队伍将近一年,也算得上是经验丰富的老兵了。

这样的搜索他参加得不多,但他还是支起了风帽的护耳,一边仔细搜索,一边竖起耳朵分辨着可疑的声音。

就在他身后,一团黑影从树下的草棵中缓缓升起,没有发出丝毫的声音。

乌业兹施猛然间觉得凉气透骨,厚厚的棉衣内,寒毛直竖,有情况!

他下意识地攥紧手中的哨子,想要回头看看发生了什么。

但是,已经太晚了,一只冰凉的大手捂住了他的嘴,顺势一拧,他那和头一般粗的脖子就被扭断了。

乌业兹施最后入眼的,是对方另一只白皙的手,在寒风中显得格外地苍白,手背上汗毛不多而且很纤细,他脑中升起一个念头:这个人,应该不属于俾鲁弯的任何一个部落。

偷袭他的人,是楚云飞。

按理说,白天应该是楚成二人休息的时间,晚上才好进行偷袭,但楚云飞眼看着杀父仇人就在眼前,怎么可能睡得着?

还有就是,马哈苏德这帮人,无论从专业性、狡猾性和残忍性上讲,都要远远大于非洲的那些部族武装,楚云飞实在没有掉以轻心的理由。

所以,白天也要加强对山谷的骚扰,不能让他们腾出手来为所欲为。

于是,楚云飞告诉成树国,“不行,我这脑袋最近时不时地就一阵疼,睡不着,你先睡,我出去转悠转悠。”

成树国可就有点委屈了,“我知道我没你厉害,不过,你也不能这么小看我的能力吧?”

楚云飞还真是有点担心战友,在他的感觉里,只要自己足够小心,打不赢也跑得掉的,可要在白天暴露了成树国,那吉凶之数可也只是五五分。

话还不能这么说,太伤人了,楚云飞又开始瞎掰,“你说什么呢?我不过耳朵比你灵点,视力也好点就是了,要不你去侦察我睡觉好了。”

成树国盯着楚云飞看了半天,突然笑了起来,“算,你想去就去吧,忙了一夜,我正好睡觉。”

轻轻把乌业兹施的尸体放在地下,楚云飞又瞄上了身体右侧的一个高个子。

那人高约一米九的模样,也许是因为穿得多,上身很壮实,两条腿却是细得可怜,佝偻着身子,给人一种“下盘不稳”的感觉。

那人似乎也知道自己的弱点,所以一双眼睛一直在地上扫来扫去的,却没考虑到自己的上空会有什么不妥。

当他路过一棵松树时,忽然,松树上悉悉簌簌地掉下了几颗冰晶,正落到他伸长的脖颈里,他脖子下意识地一缩,头顶已经风声响起,脑袋上挨了重重的一击。

下一刻,他永远地回归了大地的怀抱,再也不用担心下盘不稳了。

屠杀,在无声无息中进行着!

带队的队长并没有发现任何的不妥,半小时以后,他大声地发布命令,“好了,找不到人,大家集合,向下一个地区进发。”还吹了三下间隔很短的哨声,那是集合哨。

四周孤零零地响起七八声两长一短的哨声,那是队员们收到了命令的回应,在表示“马上赶到”。

队长很是为这零落的哨声而意外,皱皱眉头,骂了起来,“一帮蠢货,这么简单的哨声也没学会。”

几分钟过后,队员们纷纷赶了回来,队长眼睛一扫,点点人数,很是吃惊,不会吧?怎么二十个人变成十三个了?

“还有七个人呢?他们哪里去了?”队长已经准备发火了。

一帮队员你看看我,我看看你,“是呀,乌业兹施哪里去了?”

“维辛多也不见了,哦,还有木孜买也不见了……”

木子美?楚云飞听得差点从树上掉下去,那不是英国大名鼎鼎的管道疏通公司么?

\section{第一百九十七章 撤兵回防}

队长等了几分钟,再不见有队员回来,马上得出了结论:这几个战士怕是已经遭遇了不测,也就是说,敌人,很有可能就潜伏在自己这队人的周围!

看到队长神色慌张地拿起哨子,楚云飞知道自己已经不可能再继续藏下去了,两枚手雷顺手向人堆里扔了过去。

手雷还没爆炸,队长的哨子已经急促地响了起来,同时有人大喊,“有情况!有敌人!”

回应这报警声的,不仅是山谷里连绵的回声,还伴随着轰轰两声巨响。

地上的人乱做一团,有四处乱跑的,也有到处张望的,不过,还是机灵人多,他们马上趴在地上。

看到这情景,楚云飞的手难免又有些痒痒,拔出手枪点了几个名,才收回枪,双腿用力一蹬树枝,几个连续的跳跃,消失不见。

这场偷袭,一个搜索队里能活蹦乱跳的,只剩下八个人了,八死四伤,其中队长被手枪点名,重伤!

不过,这次大家也都看清楚了,偷袭的人,依旧是只有一个,没什么人接应。

他就强到天,也不过是个人而已,这么多人,压也压死他了!

于是,匆匆集合的队伍向着楚云飞消失的方向追了过去,比熟悉,你能有我们生活在这里的人更熟悉地形么?

但现实,就是那么诡异和不可捉摸,凶手就像融入大海的水滴一般,消失不见了。

等到桑丘想起来营地里还有雪獒的时候,凶手的踪迹已经被人踩得乱七八糟了。

这种情况下,那匆匆调来的几只雪獒,也只有“呼哧呼哧”喘气的份了。

可是,事情还没有算完呢,另一组搜索队在下午搜索北部山崖时,又受到了攻击。

有了上午的教训,这支搜索队搜索得异常小心。对方只有一人,那么队友相互之间一定要注意配合。

起码,要保证每个人都在一个以上的队友的视线之内。

但纵然是这样,差错难免还是会有的。

一个高个子家伙被树上悄然垂下的一根绳索紧紧勒住了脖子,可近在咫尺的队友居然没有发现异常。

高个子临死前拼命地蹬了两腿,蹬掉了几块树皮,终于引起了队友的注意,可一切,已经太晚了。

惊恐的队友在下一刻玩命一样地扣动着自动步枪的扳机,那子弹连绵地呼啸着,听起来都像机枪了。

不过,步枪显然不是这么用的,强大的后坐力使得枪口乱跳,子弹漫无目的地乱飞,差点打到了那受惊者的另一个队友。

楚云飞非常反感近距离内的漫射,这种情况使得他超强的潜伏能力变得无关紧要,那纯粹就成了真正的“子弹没长眼”。

看来,对付聪明人,有时候还是笨办法管用!

而且,7.62毫米子弹强大的穿透能力,使得子弹在穿过松树后还具备相当的杀伤力,这点也很令人讨厌,哪怕楚云飞是穿了厚重的防弹衣也改变不了这个现实。

于是,楚云飞利落地松开绳索,身形晃动,拔出手枪就是一枪,正正击中那接近崩溃的家伙的额头!

就这短短的不到一分钟时间,搜索队的战士们已经纷纷跑了过来,对楚云飞形成了一个接近半圆形的包围圈。

更有那心思敏捷的,或者说被这几次屠杀吓坏的人,已经就地卧倒,开始向这个方向射击了。

那个差点被自己人误伤的家伙冲得最快,几乎在眨眼间就冲到了楚云飞的面前,两支黑洞洞的枪口相向,四目相对!

就在那家伙微微一愣的工夫里,楚云飞扣动了扳机,又是一声撞针的轻响,操,又是哑弹!

“砰”的声音还是响起了,不过,发声的部位不是格洛克17手枪,而是楚云飞的嘴巴,几乎在同时,他就地一个侧滚翻到对方脚下。

声音入耳,那家伙登时吓得毛骨悚然,手中的五六半自动步枪也下意识地扣动了扳机,“哒哒”几声,子弹全打到了眼前那棵松树上。

楚云飞右腿在地下狠命一扫,扫倒了对方,顺势又是个鲤鱼打挺翻身跃起,最后双脚狠踏对方胸口,借着那股冲力,直接跃向后方,而后一个转身,脚不沾地风一般地跑掉了。

身后依旧是密集的子弹相送,但地下躺着的那位肋骨断了七八根,仿佛被车轮碾压一般,是再没能力开枪了。

等搜索队再次集合的时候,这个队的队长才发现,除了那个高个子和被手枪击毙的两人,还有两人也被拧断了脖子,而地上那位被碾压过的也是口吐鲜血,眼见活不成了。

一时间,新纳山谷内阴气森森、愁云惨淡、人心惶惶。

想到那潜伏在黑暗中的恶魔,不知道什么时候就会悄然地出来夺走几条人命,所有人都不寒而栗。

这帮整天给别人制造恐怖气氛的专业人士,终于也品尝到了这有如在地狱中煎熬的滋味。

马哈苏德果断地撤下了突出部的所有部下,仅仅留下了最为重要的制高点,1号地区。

那里存有大量的轻重武器,留守着全副武装的20名真正的精英,都是在伊朗和阿富汗打过仗的老兵。

这个1号地区,实在是没办法不守的,它是山谷两边最有威胁力的制高点,外可以观察山谷外大型的人员调动,内可以压制谷中大部分地方的火力。

而且,它直接扼住了山谷北侧的缓坡。

新纳山谷南方是比较宽的开阔地,政府军或者大点的势力一般是从那里进入山谷,北侧就成了训练营人员从新纳山谷撤退的路径。

而且,马哈苏德隐约还有些不安,那些想象中的重火力并没有出现就已经造成了现在的这个样子,一旦真的出现,那还得了?

当然,马哈苏德并不指望对方能好心或者愚蠢到不用那些重型武器,那些东西的出现似乎只是个时间问题。

所以,最有利的制高点一定不能拱手让人,那是未来压制对方火力的唯一希望。

其他的地方,撤就撤了吧,现在必须要把战斗力集中在一起使用,但愿,对方能够蠢得来这里送死,那样,大量的陷阱和埋伏就用得上了。

现在,要不要去跟别的武装求援呢?马哈苏德揉揉疼得有些发紧的额头,又开始新一轮的思考。

\section{第一百九十八章 休息不得}

楚云飞悄无声息地溜回藏身处,成树国还在那里睡觉,睁眼看了一下钻进来的战友,把衣服领子拽拽,翻个身继续睡了起来。

“嘿,让让,往里点,”楚云飞轻轻踢了成树国一脚,睡得半死的那位闻言又向岩石底下拱了拱,眼睛是说什么也不愿意睁开。

这个地方外有杂草和灌木,还有一块大石头,但妙就妙在石头和灌木之间,有一个很大的缝隙,平时总是被落叶和杂草掩着,从外面根本看不出来。

活动了一天了,楚云飞有些饿了,从“仓库”里翻了一块囊出来,虽然他身上也带得有这种食品,但是不到万不得已,他是不习惯动身上的粮食储备。

囊被冻得梆梆硬,楚云飞轻吁口气,无奈地摇摇头,两指发力,掰了一小块下来,向嘴里送去,冷却后越发浓重的奶腥味,熏得他不住地皱眉。

烤热的囊要好吃得多,楚云飞慢慢噙化口中那团冰冷,一边感叹,一边又掰下了一小块。

下一刻,他的手就停在半空,加热,为什么不试试加热呢?

不要怀疑,楚云飞最近虽然经常性地头疼,但脑子还没有坏掉,他自然知道这里不合适生火,起码在绝望和无可救药之前,不合适生火。

楚云飞想的是,看了老多用内功泡茶的小说了,这内气,是不是真的能这么用,用来生热呢?

团长师傅耿风可是曾经说过的,“心随意动,可化惴惴为陶陶焉。”

想到就做,楚云飞对着手里有如鸡蛋大小的囊开始……呃,发功。

但是说良心话,这功该怎么发,楚云飞还真的不知道,以他的想法,不过……就是把气运到手指上,拼命运的那种。

然后……这囊该自动升温热了吧?

接下来的事实证明,楚云飞的想法显然有点过于一厢情愿,那囊在他手里呆了有十多分钟,却没有什么奶香味散出。

当然,这一小片囊的整体温度,似乎还是提升了一些,因为楚云飞觉得,手感上多少是松软了些。

但是他马上意识到,自己的手指,那是有温度的,没准是自己的体温,消融了那囊中的刺骨凉气。

摇摇头,不甘心,楚云飞又开始新的尝试,能不能,能不能把这一小块囊当作身体的一部分,让内气通过它呢?

如果行的话,这囊的温度应该是可以再上升一点的。

不过,显然这样的构思也是错误的,十分钟后,楚云飞终于推翻了自己的设想,那冰冷的食品并不能传导内气,那么,用生命能量行不行呢?楚云飞又开始突发奇想,他尝试着把生命能量塞了一小团进去。

囊还是没有任何反应。

那就多加点进去,楚云飞牛劲上来,非要弄明白不可。

这倒不是他过于无聊,实在是,杀父仇人就在眼前,他实在是不能安心入睡,可精神一直亢奋也不是回事,只好没事找事地寻些事情来做。

时间没有过了多久,那囊还真的热了起来,腥膻的奶味没有了,但散发出的也不是奶香。

是一种酸臭,那种用来蒸馒头做面引子的臭味!

楚云飞马上明白了,大概,是那睡眠状态的酵母菌,被自己强大的能量唤醒,开始工作了。

什么乱七八糟的事!

成树国也被这新奇的味道弄醒了,事实上,他一直处于“假寐”状态,身在险地,训练有素的军人不可能睡得那么死,这只是恢复体力的一种休息方式而已。

“操,云飞,你搞什么呢?臭烘烘的,你不怕招来人啊?”

楚云飞已经把那囊扔到地上了,抠了点土埋了起来,“我也不知道,怎么会这么个样子,还好发现得早。”

这么冷的天,没了能量支持,那些酵母菌又得老实地去睡觉。

算了,楚云飞扭扭身子,这么危险的地方,别再整事了,还是吃点东西睡觉吧。

他闭着眼睛假寐半天,刚迷迷糊糊地有了点睡意,腰间就被成树国推了一把,“喂,起床了,天黑了,该上班了。”

楚云飞不情愿地睁开眼睛,可一团漆黑下,什么也看不到,又闭上眼睛,“你搞什么啊?我白天忙了一天呢。”

成树国早在这里等他呢,“没关系,你不是厉害么?都超人了,还睡什么的觉?你走不走,不走我自己去上班了啊。”

楚云飞轻叹一声,知道成树国生气了,白天自己一个人出去爽,他肯定是不高兴了。

不过,让成树国一个人出去,那怎么可能?楚云飞晃晃脑袋让自己清醒些,站了起来,“那就走吧,咱们继续杀外围的那几个突出部。”

又是个忙碌的晚上,不过,也是个和平的夜晚。

楚云飞和成树国大半的时间都用在走路上了,挨个搜索过去,他们意外地发现,基本上,所有的突出部都没人值守了。

这样就对了,楚云飞一点都不惊奇对方这样的安排,分兵才真的是找死,不如加强营地中心的警戒。

那个辛亚拉长老曾经说过,根据族里人所说,营地四周,似乎到处都是陷阱,要几个中国人小心呢。

不过,那个最高的突出部,应该还有人吧?那里要再没人,只有两种可能,一、马哈苏德水平实在是太差,二就是他已经准备逃跑了。

两人小心翼翼地向1号突出部摸去,离着那里还有五百多米,楚云飞猛然间又有了种非常不妙的感觉。

有情况!楚云飞做个手势,示意成树国停步。

放松身体,楚云飞深吸口气,想判断一下可能危险会来自哪一方面。

在上1号突出部的必经之路上,那里有非常不妥的事情!这是楚云飞隐约的感觉。

保持着那种空灵状态,楚云飞抬眼向那里望去。

天很黑,虽然有部分残雪的反光影响视线,但楚云飞还是敏锐发现了两个生命能量相对集中的地方,那里,该是埋伏着两个人吧?

下一刻,那两个倒霉蛋就被成树国和楚云飞一人一个解决了,估计他们到死还在纳闷,在雪地上盖了白布,怎么也会被人发现?

成树国用的又是钢针,所以他又得在那里血呼呼地挖自己的武器。

楚云飞皱皱眉头,看到这样的场面,他的头似乎又微微地疼了起来。

不过,现在不是计较这个的时候,楚云飞蹑手蹑脚地开始继续探路。

这个突出部的威胁实在太大了,从整体战局上考虑,必须要拿下它,大家才能比较自由地活动和攻击。

\section{第一百九十九章 成树国受伤}

走了没几步,楚云飞敏锐地感觉到脚下似乎有些不对,缩回脚来一看,地上一根褐色的细线,顺着那线看过去,一块山石旁,一棵小树很隐秘地被拉成弓型,上面是两只黑乎乎的削尖了的树枝,旁边还挂了一个铃铛。

好恶毒的陷阱,不光那树枝可能对人造成伤害,居然还有报警的作用。

前面,还会不会有类似的埋伏?

看看表,时间已经不早了,要是再有几个这样的埋伏,拿下这个突出部,时间未必够了。

等到成树国过来的时候,楚云飞正在那里修改那个机关,线的位置挪动了,树枝也换成了手雷,但愿,能起点效果吧。

计划需要改变了,两人匆匆跑回去,拿了几枚枪榴弹,目标是那营地。

走到枪榴弹的射程之内,两人就不再走了,反正只是为了骚扰一下,加深对方的恐惧感,把自己陷进去就太不划算了。

“轰、轰”,几声巨响,在黎明前最黑暗的时分,枪榴弹的爆炸声划破了寂静的新纳山谷。

发射的距离远了点,精度又不够,这六枚榴弹,并没有给营地带去多大的伤害。

发射完毕,两人刚要掉头撤走,出乎他们的意料,还击马上就来了。

这样的反应速度,实在是太快了点,两人都没有充分的心理准备。

还击的火力点,位于1号突出部,一挺双管高射机枪疯狂地吼叫起来,密集的子弹,打得山石乱溅。

楚云飞轻轻皱了下眉头,他们难道都不睡觉的么?看来,那个一号高地,上面的防备实在是很严密。

这种情况下,自然更要没命地跑了,不过,才跑两步,成树国就发出一声闷哼,他挂彩了。

小队长三人组里唯一没有挂过彩的家伙,终于打破了不败的金身。

伤口在大腿外侧,看上去情况还比较严重。

顾不上许多,楚云飞用了一分钟把伤口包扎一下,扛起他就跑。

路上还得操心是不是有血滴到了地上。

回到藏身处,打开伤口一看,还好,是被子弹划伤了。

虽然那时1号地区离他们有700多米远,但要是被高射机枪击个正中的话,那腿肯定要截肢了。

饶是如此,在这寒冷的早晨,成树国还是痛得冷汗直流,“噢,妈的,这样都会中弹,我该在英国买点彩票的,咝,你能不能轻点?”

血是止住了,但成树国的裤子上的血腥味太浓了,容易被人发现,不能留下。

看来,在刘宁他们回来以前,他只能光着屁股躲在毯子里了。

“要不,白天我再辛苦一下,弄个死人的衣服拿给你?”

成树国赶忙摇头,“还是算了,我感觉他们从来都不洗澡,再说了,他们那衣服都是一缕缕的,我非常怀疑你剥下来的衣服我是不是还能再穿上去,反正算算日子他们也该来了。”

把成树国的衣服扔掉后,楚云飞回来休息了三个小时,等他一觉起来,觉得精神好了不少。

他是被一声爆炸声弄醒的,这说明,起码有一个陷阱是起了作用了。

“不行,我还得出去,有我在外面活动,你这里的安全才能多点保障。”

说着,楚云飞小心揭开顶上的伪装,出去了。

留下来的成树国把所有能爆炸的东西搜罗在一起,要是有人找过来,大家同归于尽好了。

山谷里现在可是很热闹,大家激动地奔走相告:那个可恶的魔鬼,被打中了!

证据,就是成树国中弹的所在处,地上那一滩鲜红的血迹。

这是一个规模达三十人的搜索队发现的,这是新纳山谷早晨唯一出来的一支搜索队,两只带路的雪獒很快就发现那血腥味。

派出三人回去报告,激动的搜索队顺着就追了下去。

不过,大家还没高兴多久,1号突出部附近就传来了一声巨响,恶魔又出现了么?

桑丘刚派出去人打听1号地区那里发生了什么事,没过五分钟,又是一声巨响,方向却是搜索队那边传来的。

他们踩上了楚云飞临时掩埋的反步兵跳发雷。

还好,大家穿的比较厚,没人死亡,只有两个重伤一个轻伤,遗憾的是,雪獒被炸死一只。

另一只受了点轻伤,倒不算什么,但是它被吓坏了,一个劲地向后缩,再不肯继续往前走。

等到新的雪獒送来的时候,空气中强烈的硝烟味破坏了它的嗅觉,没办法继续追踪了。

1号突出部果然被惦记上了,马哈苏德郁闷地想,看来,还是有必要加强那里的防守。

据高点确实有它的优势的,起码,有只恶魔就是被那里发射的子弹打伤或者死的!

不过,山口那两个埋伏哨实在是够垃圾的,居然会被人无声无息地杀死,实在是身为战士的耻辱!

这一切,都没动摇了马哈苏德的决心,现在对方有人受伤,不趁机大肆搜捕一下,实在是对不起死去的战士。

于是,山谷中又派出了五支二十人的搜索队伍,营地里,仅剩马哈苏德等十余人坐镇。

但是很遗憾,几支搜索队撤回来的时候才下午两点,实在是没办法,又有十二条人命丢在了搜索途中。

马哈苏德真的头大了,仅仅就这么三天的功夫,自己这方就死了六十多个人,这么下去,用不了几天,自己这里的人就会被杀个精光的!

现在整个新纳山谷里,能活蹦乱跳的,只有不到一百七十个人了,其中1号突出部那里还用了二十人。

这么点人,搜山是不够用了,别派出去的人太多,让人把老窝端了吧?

还是守紧营地是上策,马哈苏德真的有点无奈了,看来,还真是要找人帮忙了。

他冲桑丘挥挥手,“桑丘,派个人去部落里,把‘神圣伊斯兰’的人喊来帮忙吧。再派个人去林孔部落找噶达斯亚巴求援。”

噶达斯亚巴是马哈苏德的姻亲,是一支两百人左右小武装的首领,平时爱干些打家劫舍的勾当,跟中国的土匪类似。

“神圣伊斯兰”是克普塞部落一股亲伊朗的势力,在原则问题上,和马哈苏德的武装没什么太大的分歧,怎么也是同一部落的,不可能见死不救吧?

况且,该花的钱,那还是要花的。

\section{第二百章 夜袭1 号地区}

两拨求援的人被派了出去,但是,忙中难免就会有点小错出现。

每拨都是两个人,人数太少了。山谷里人手再缺,也不该在这上面节省,被阻击的可能性实在是太大了。

事实也是如此,远处观察的楚云飞马上就发现了异常,他们是要去求救么?

于是,走出山谷不远,派出去找“神圣伊斯兰”求援的两个人被楚云飞截杀了。

杀掉的这两人要不要拿回去示众?楚云飞沉吟了一下,还是决定就地掩埋,现在这里只有自己一个人能自由行动,为了不把对方逼得太狠,还是算了吧。

成树国受伤了,虽然不重可也严重影响行动,为了战友的安全也不能太过分。

这一天剩下的时间里,楚云飞打坐一会儿,观察一阵,熬到了天黑。

楚云飞自认天黑会安全很多,溜回去看了看成树国,发现他睡得极为香甜。

毯子周围,炸药、手雷、炮弹、榴弹被整齐地垛了起来,面对这样的结构,其用心不问可知。

楚云飞闭上了眼睛,实在是说不出地伤感和内疚。鼻子也开始发酸了。

不过,现在显然不是自怨自艾的时候,楚云飞抓紧时间做了两个触发式的机关,绳索直接连进了藏身地,挂在那个偷来的小铃铛上。

当然,铃铛也被楚云飞加工了一下,找了点树叶塞住一头,另一头和铃铛壁也离得很近,这样,就算有人触动陷阱,铃铛也只会发出很轻微的响声。

忙完这一切,楚云飞开始用餐,不过,一个囊还没吃完,他就睡着了,这几天,实在是有点累了。

等他一觉醒来,已经是夜里一点了,总算不错,生物钟还没有失灵。

揉揉针扎般痛的眼皮,楚云飞伸个懒腰,开始收拾东西准备行动。

收拾停当,他刚要出去,又想起了什么,翻出成树国的钢针拿了一把塞进口袋,这东西,成树国用得,我就用不得?

楚云飞居然有心思想起了阿Q那句名言——“和尚摸得,我就摸不得?”

目标在白天就已经选好了,就是那1号突出部,无论是为战友报仇也好,为战局考虑也好,这个钉子,必须拔掉!

路线在白天也计划好了,没费多长时间,楚云飞就溜到了1号突出部下面。

1号地区地势非常地险要,只有东南侧有一个接近四十五度的缓坡可以上去,当然,为了行走方便,那个被开出的小路是“之”字形的。

楚云飞打的主意,是从东北侧的陡坡攀登上去,那里坡度虽然非常陡峭,但是不算太高,从下面一个小平台算起的话,只有三十多米。其他几个可攀爬的方向,可都是起码六七十米以上的高度。

虽然已经是冰雪消融的季节了,但这一侧的山坡由于大多时候是背阴的,坡上还是冰雪遍布,滑溜异常。

确定了四下没人,楚云飞开始了他的攀爬,将近七十五度的陡坡,实在是考验人的勇气和毅力。

楚云飞带了六七根树枝插在背上,一手弯刀一手赤裸地向上爬去,遇到实在不好攀爬的地方,就用弯刀掏个洞,塞进去根树枝,算是个可以轻微着力的落脚点。

这一切,还都不能发出太大的声音。

滑落了两次之后,楚云飞终于在半小时以后登顶,手中的弯刀在他下滑过程中起了稳定的大作用,现在基本已经不能使用了。

就在他刚要翻身上突出部的时候,借着冰雪的反光,他又发现了一根细线,顶端又挂着一个铃铛。

防范还真的很严啊,楚云飞皱皱眉头,镇静,一定要镇静,行百里者半九十,越是快成功的时候,越不能放松警惕。

怀着这个心思,楚云飞小心地跨过这个陷阱,马上就发现了一个哨兵坐在那里打盹,看情况,应该就是防范这片地方被人攀爬的。

那哨兵其实连值了两天夜班,困顿异常,才被派来守这个不重要的地方,他做梦也没想到,真有人能在这个季节爬上来,还绕过了报警的铃铛。

横死之人,自有取死之道,这个哨兵被楚云飞干净利落地割断了脖子。

防守严密、天堑一般的1号突出部,就这样毁在了一个哨兵的手里。

楚云飞微微皱下眉头,这刀纵然是再钝,杀人也不至于如此地费劲吧,他等了一阵,待那血不再喷涌,就上前翻起那死尸来。

果然,那尸体的脖子处,加裹了两层厚厚的牛皮,一直遮掩到脸腮处。可以想象得到,这新奇东西全是拜中国人的袭击方式所赐。

楚云飞啼笑皆非,放下自己的钝刀,拿起那哨兵腰畔的弯刀,抬头向四下望去。

那些有经验的老兵们都隐藏得很好,但他们的隐藏都是面对山下的,所以在楚云飞的眼里,这是一堆可口的点心。

楚云飞又观察了将近十分钟,发现了所有的士兵,也惊讶地发现,这个突出部的火力结构实在是强得离谱。

两挺机枪,其中一挺是放平的高射机枪,旁边还支着一门类似迫击炮的东西,不知道型号和口径。

要是树国在就好了,他肯定知道那是什么东西,想到这里,楚云飞又皱了下眉头,算,别想了,按计划行动吧。

楚云飞早就策划好了如何袭击眼前这十九个人,突出部也有制高点,那是块一人多高的大石头,上面有个人似乎已经睡着了。

就拿他先开刀吧。

楚云飞悄悄走近那石头,轻轻一纵身,就悄无声息地跃了上去,手中寒光一闪,人头落地!

然后,就是由外到内,继续点名了,做这种勾当,楚云飞是越来越炉火纯青了。

其中很有几个人是很清醒的,但是同楚云飞相比,他们的反应速度实在是差太多了。

由于突出部的守备人员已经做好了迎接恶魔的准备,所以,居然很罕见地没有派出游动哨,这却更方便了楚云飞的行动。

还是有两个人的位置非常靠近的,他们的岗位是那挺高射机枪,其中有一个怕是专门压子弹的吧。

这两个人是最后死的,那机枪手反应很快,甚至从嗓子里仓促发出了四分之一个音节。

但是,他能做的也只有这么多了。

\section{第两百零一章 左手一只鸡}

做完了这些,楚云飞抬头四下看看,终于没任何生命存在的迹象了。

不对!前面雪地上,一个人形黑影蹲在那里一动不动,怀里还竖着一支步枪!

楚云飞就地就是一个翻滚,掏出了钢针,蓄势待发。

那黑影依旧纹丝不动。

下一刻,楚云飞明白了那黑影到底有什么不妥了,那黑影身上,根本就没有任何的生命能量!

原来是个假人,应该是那种练格斗的橡皮人吧?

不过,把假人弄到这里有什么用呢?楚云飞的脑子又飞快地转了起来:黑色的假人在白色的雪地上……陷阱,那里绝对又是个陷阱!

果然是陷阱,假人身后有一个掩藏得很好的陷坑,假人胸脯上还绑了带有触发引信的炸药,引信就在脖子处,怪不得这家伙离别人老远呢。

机关设计不可谓不巧,心思不可谓不毒,但是,设计机关的人怕是做梦也没想到,这世界上还有“生命能量夜视镜”这种人体结构。

看着这费尽心机的歹毒陷阱,楚云飞咧了咧嘴,看来,自己真的很招人恨啊。

不过,也正好随了楚云飞的意了,我也正好想玩陷阱呢,大家就比比吧,看谁比谁更玩得好。

接下来,就是楚云飞大展身手的时候了,在上突出部的路上,他一面改造对方留下来的陷阱,一面自己又制造了不少陷阱,多是那种踏发或者绊发的装置,还设计了一个前触后发的陷阱。

不过,这个陷阱做起来真的很累人,他要小心不跟别的陷阱打架,以免影响效果。

做完这个陷阱,天色已经大亮了,楚云飞捡起那把豁口众多的弯刀,又在背上插了几根树枝,顺着来路,背靠山坡,悄然地慢慢滑下。

接下来的一白天,楚云飞除了大小便,一直停在一棵树上观察着谷中的动静,累了就靠在树枝上养养神。

大约早上十点钟左右,营地里的人似乎很怀疑1号地区一直没有什么响动,派了一支十余人的小队向那里走去。

看来山路口上那两个哨兵被摸掉后,那里再没补派哨兵,十几个人就当没看见一样向山上走去。

带路的家伙走得很警惕,但搜索未免有点过于不够精心,他一脚就踢开了一道细线,要知道,那可才是第一道埋伏啊,还是改造的那种,昨天的教训不够深刻么?

路旁雪下的炸药当场就爆炸了,“轰”地一声,带路的那家伙被炸出去有将近三米远,紧跟着他,背靠背做警戒状的俩人也飞了出去。

整个山谷登时变得鸦雀无声,营地所有的人都能估计到:咱们的人,怕是又倒霉了。

寂静过后,就是乱糟糟的人声四起。

一只雪鸡似乎受到了惊吓,扑棱着翅膀飞到了楚云飞栖身的树上,楚云飞第一时间就注意到了它。

那笨头笨脑的雪鸡刚在树上扑扇着翅膀落下,就发现了一直停在那里不动的楚云飞。

雪鸡马上就反应了过来,这是那种很危险的天敌,刚要飞走,怎奈楚云飞已经先行一步扑了上来。

谁说守株待兔是谣传来的?楚云飞得意极了,老天,我居然能空手抓住野鸡,真不错,这下树国有补品了。

下一刻,补品过度惊慌,拉了楚云飞一手的“糟粕”,并且拼命地叫了起来。

楚云飞正待伸手拧断这厮的脖子,却听得又是一声巨响,“轰”!

原来,执意要联系到1号突出部的联络队继续上行,不小心又踩到了一个跳发雷。

这实在不能怪他们愚钝,冻土刚刚开始解冻,走到那里都是那种高一脚低一脚的感觉,偶尔还有冰碴的断裂声。

听到爆炸声,楚云飞灵机一动,居然不再杀了这鸡,而是拎着它的爪子狠狠地晃了两圈,让你再叫!

那雪鸡血往头上涌,被甩得七荤八素地,终于不再叫唤。

楚云飞拎着雪鸡跳下树来,找个地方,埋了两颗手雷,又做了一个机关。

等他再次跳上树的时候,那负责联系的小队已经退了下来,不再继续试图上那1号突出部,已经堪堪走进营地了。

楚云飞把雪鸡夹在胳膊下,端起AK47就是两个点射,两个人打着转就摔了出去。

这下可算是捅了马蜂窝了,呼拉拉,营地里气势汹汹地涌出一百多号人来。

联络队的剩余人员已经大致判明了枪响的方向,各自寻找掩体开始向这个方向盲目还击起来。

新冲出的接应者可还没有搞清楚状况,大家乱哄哄地寻找掩体,观察四周。

楚云飞又是两个点射,射杀两人后,悄然跳下树来。

这次可算真的暴露目标了,密集的子弹铺天盖地般撒了过来,过得一分钟,又有两枚火箭弹在附近爆炸开来。

楚云飞背好步枪,气运喉咙,“啊~~~”,发出惊天动地的一声惨叫,顺手斩断了那只雪鸡的脖子。

妈的,一只鸡的血怎么会只有这么一点点?楚云飞一边恨恨地咒骂着,一边还在拼命地挤着雪鸡的血。

算了,就这么多吧,再呆下去跑都跑不了啦,楚云飞倒提着雪鸡,“行迹明显”地悄然离去。

没过五分钟,就有人已经哆里哆嗦地摸了过来,他们“惊喜”地发现:又有一只恶魔受伤了。

马上有人又牵了雪獒过来,看着地上若有若无的印迹和血滴,大家奋勇地追了过去。

走过几个人之后,终于又有人不幸地踏上了机关,虽然只是两颗手雷,但造成一死两伤的结果也还是正常的。

这次桑丘也在队中,他没有考虑那个陷阱造成的伤害,而是,“追,再死几个人也要追到那个家伙,要不我的子弹可不讲理。”

大家顺着血迹不停地追下去,却是越追越高,开始爬山了!

穿灌木,越石头,血迹终于停止在了一个地方,那里是一处断崖!

难道说,这个恶魔失足掉下去了么?

那些衣衫褴褛的“勇士”们争先恐后地向山下跑去,试图绕过山脚到那断崖底下去寻找可能的尸体。

此时,楚云飞正在不远处的一棵树上冷冷地看着热闹:山这么大,够你们绕一阵的啦。

他自己也不明白,为什么会冒着这么大的风险,制造这么个噱头,只是隐约觉得,这样做的话,起码能让对方花心思在搜捕自己上面,大概可以阻止他们暂时试图恢复控制那个最高的突出部吧。

而且,成树国还需要休养一下,刘宁他们,也该快来了吧?

\section{第两百零二章 刘宁回来了}

然而,就在这天下午,刘宁他们还没到,对方增援的人已经到了,那是噶达斯亚巴的土匪武装。

来的人不算多,只有五十来个人,事实上,马哈苏德为了骗取帮助和节省费用,并没有仔细说自己遭遇到了什么厉害人物。

他只是说有几个仇人在新纳山谷藏身,由于谷中人手短缺,搜索起来极为不便,所以请噶达斯亚巴多派些人来帮他搜出那几个人。

严格地说,他的话也算是实情,不过是少说了一些事情而已,而在噶达斯亚巴看来,派五十个人来帮忙搜山,已经算是“用牛刀杀鸡”了。

这不能怪噶达斯亚巴消息不灵通,不知道有厉害人要找马哈苏德报仇,实在是,他们做土匪的,冬天都是聚集在老巢,很少出门的。

而且,除了贝西哈兰村的惨案比较吓人外,其他零星敲掉的基地各个藏身处都没有多少人,这种小案子在俾鲁弯不能说比比皆是,起码也不会太让人吃惊。

整个战火蔓延的过程和系列惨案的根源,只有“基天”组织和马哈苏德最为关心。

还是在这个下午,马哈苏德的人终于排除了所有通往1号突出部的陷阱,再次登上了这个制高点。

那些陷阱都是被山羊破坏掉的,那些山羊本是营地的储备粮食,用来送死并不打紧,它们来回地踩踏,为人们努力排除可能的隐患。

可惜的是,营地不是生活区,羊不算很多,这次就被炸死了6只,还有两只没死也差不多了。

还死了两个带山羊的人,他们本来一直站在安全的地方,是被那个前触后发的机关炸死的。

登顶的人已经有了心理准备,上去一看,1号突出部的人果然全部死掉了,而看起来那些偷袭者并没有任何的伤亡。

他们是怎么做到的呢?

当下,就有人回去汇报马哈苏德,看是不是要在1号地区再度布置人手。

就在这时候,噶达斯亚巴的人到了,带队的是他的心腹阿孜克。

看着那区区的五十人,马哈苏德实在有点郁闷,这点人还没我死的人多呢!

不过,他也不好马上说就什么,同为穆斯林兄弟,还是先要热情地款待一番来帮忙的朋友。

就在这时候,有人来请教1号地区的防守问题。

马哈苏德看看表,已经四点了,再有两个小时就天黑了,这时候安排人上去,肯定没时间在山路上设置什么陷阱了。

可这几天的经验告诉他,山路上要是不设置陷阱,那上面再安排多少人也是无济于事的。

自己的人是损失不起了,可是噶达斯亚巴的人刚来,也不好马上安排人家上去,索性明天安置好陷阱,再委托噶达斯亚巴的人值守好了。

反正现在这里人这么多,还怕他们偷袭营地不成?

再说,白天不是又伤了一个家伙么,只是不知道摔下悬崖后死了没有。

马哈苏德没想到,他这样的决定,把自己进入地狱的时间又大大地提前了。

一夜之间,一切都不同了。

晚上,刘宁他们三个到了,成树国终于有裤子穿了。

楚云飞一直在监视着1号地区的动静,最后,他欣喜地发现,对方居然没派人值守这个地方,谢天谢地,像昨天夜里那样艰难的攀爬,还是少来几次的好。

楚云飞很高兴地发现,刘宁带来了大量的反步兵地雷,将近五百颗,还采购了大量绳索、铃铛、钉子、钢丝,线香之类的陷阱制造材料。

事不宜迟,趁着晚上突出部没人看守,大家赶紧行动吧。

于是,五个人里除了成树国外,大家各自负责一片,开始布设陷阱和地雷。成树国找了块石头,躲在后面架起了车载机枪。

楚云飞负责谷南侧的入口,刘宁负责谷北侧的退路小坡。

布雷是个技术活,很多年以前,甚至有专门的学科来研究这种消极防御手段,但是随着现代战争理念的不断深入,布雷和排雷大多都已经交给机械化部队来完成了。当然也有特例,这里我们就不做探讨了。

从总体上讲,地雷,大部分场合的用途类似于围墙,起个隔离和阻断的作用,但楚云飞他们现在布的雷却是不太一样。

他们要阻止山谷内的人出入,而他们自己本身却还必须能出入,这对布雷者的水平要求就高了,他们必须能记住自己布的雷在什么位置。

这实在是个艰难而艰巨的活,雷少点还好说,起码有地形地貌可供参考,可上了百的话,给谁记起来也不容易了。

还好,这里不是平原,地势高低形貌各有特色,总算还容易记点。

辛汗和穆斯林只负责东西两侧的布雷,那是山谷的两侧,不需要很多的雷,记这点小事,总不是什么太大的问题吧?

三百个雷布了下来,还要注意掩饰,天色已经大亮了。

大家收工,回去睡觉了,楚云飞却不肯休息,有战友来支援,等于是给他兴奋的神经又打了一针,他要在白天继续盯着这些匪徒!

事实上,他还是能再睡两个小时的,根据他的观察,这帮匪徒总是在9点左右才开始起床出工。

今天的情况却是有点不一样,马哈苏德早早就把人喊了起来。

伊斯兰教教义规定,教徒不能饮酒,马哈苏德又标榜自己是原教旨主义者,所以接待兄弟们的宴会上并没有酒,反正他这里新鲜的死羊那么多,招待大家是足够了。

噶达斯亚巴的土匪们可不这样想,他们那里最畅销的就是酒精饮料,来到马哈苏德这里,接风宴上居然连酒也没有,实在让其中的几个老酒鬼腹诽不已。

当然,这里大麻还是有的,但马哈苏德如此地标榜自己,怎么可能把这软黄金拿出来招待朋友?

所以大家睡得起得还都比较早。

于是,楚云飞很早就被一声爆炸惊醒了,这时的他,正趴在一堆灌木丛中打盹。

触雷的,是一支打算上1号突出部设置陷阱的匪徒,大约有三十人左右。

马哈苏德把人分为五组,这是一组最先出来的,其他是搜索队,每组二十五人,这样,营地里还剩有八十余人。

\section{第两百零三章 见到马哈苏德}

不但楚云飞的注意力被吸引过来了,马哈苏德也在营地里暴跳如雷,“怎么回事?居然让人把雷埋在了家门口,昨天还没有的!”

发火归发火,这个节骨眼上,绝对不可能让阿孜克看了笑话,马哈苏德马上下令,“牵羊出来,让它们探路,先探1号地区。”

说着,他还不忘把脸扭过来,冲着一脸睡意的阿孜克“胸有成竹”地笑笑,“唉,让你看笑话了,我的人还是缺少训练,都是钱闹的啊。”

说毕,还微微点头,一副无奈的样子。

桑丘听得目瞪口呆,却是欲言又止的样子。

马哈苏德不耐烦了,“桑丘,想说什么就直说,都不是外人来的。”

桑丘抿抿嘴唇,“呃,这个……首领,我们的山羊只有三只了。”

阿孜克是土匪出身,但也不是没见过世面的,闻言有点纳闷,“为什么,先探1号地区?别的地方还有雷怎么办?”

他的意思很明显,你的人去那个狗屁1号地区,让我的人拿肉脚趟雷不成?

实力才是最重要的,马哈苏德自然知道对方不想折损人手,所以,昨天见面时他只是轻描淡写地说,“那是几个小毛贼。”

不想趟也得趟,这里你做得了主么?马哈苏德心里鄙视地笑了一声,不过,说话的技巧还是要有的。

“阿孜克,你跟我出来,”马哈苏德带头走出屋去,阿孜克一头雾水地跟在后面。

站在院子里,马哈苏德手指1号突出部的位置,“喏,看到了吗?那里就是1号地区。”

“这是一个可以俯瞰整个山谷的地方,为了保险起见,我们必须把它牢牢地掌握在手里。”马哈苏德没有解释那里还扼着逃跑路径的喉咙,要说出来的话,实在是太丢人了。

可惜,从山下面看上去,那个突出部实在不怎么显眼,阿孜克就看不出来那里有多么重要,不过就是个山头而已,这里到处都是。

但马哈苏德的名头在这里极响,又以原教旨主义者自居,起码表面上是最重教友的感情,在这点上应该没有可能骗他。

不过,想想自己的人手可能因为“战术需要”而遭受到损失,阿孜克还是没办法高兴,可不高兴也没办法,谁要他现在在马哈苏德的地盘上呢?

楚云飞一直在远处观察着这里,透过望远镜,他第一次看到了马哈苏德,但是,很遗憾,距离太远了,他一时没反应过来。

等他突然反应过来时,望远镜回扫,只来得及看到一眼马哈苏德的侧面,下一刻,人已经进屋了。

是他,没错,就是他!楚云飞登时浑身一抖,望远镜几乎从手中掉落,人也差点从树上掉下来。

六年来,那是楚云飞挥之不去的梦魇!

接下来,楚云飞彻底地陷入了一种亚疯狂的状态,一段时间内,他大脑里面一片空白,他想到了很多,又似乎什么都没有想。

懵懂无知的自己,严厉而又爱叫真的父亲,痛哭着的母亲。年少时的一幕幕涌上心头,他的头又疼了起来。

所有的一切,马上就要做个了断了!!!

这时要有人能靠近偷袭的话,楚云飞必死无疑!

不过,他很快就调整过来了,因为他明白,现在不只是他自己一个人在这里:我要为大家的安全负责!

于是,揉揉涨得发疼的太阳穴,他又拿起了望远镜,正好看到了一出很滑稽的场面:一只羊在天空飞!

紧接着,爆炸声就传了过来,那只羊踩到了地雷!

收敛心思,楚云飞略一分析,就明白了对方的计划:他们要优先打开通向1号地区的道路,坚决地控制那个制高点!

算盘打得不错,不过,你问过我了么?我可不同意!

于是,楚云飞手中步枪举起,瞄准那个搜索队就是两个点射,然后跳下树,飞奔而去。

还击的炮火来得很快,大概就是五秒钟左右,最先反应过来的,是营地门口的那两挺机枪,连绵的子弹,打得树皮乱飞。

这时楚云飞早跑到三十米开外了,他脑子里想的是:那个搜索队明显是去那个最高的突出部的,居然有人抗着镐头和铁锨,看来又是要设置什么障碍或者陷阱来长期占据那里。

不能让他们如愿!

紧接着,其他四支搜索队也出动了,阿孜克的人有一队,马哈苏德的人有三队。

一支的目标就是楚云飞这个方向,很明显是来追击他的,另三支各搜各的,开始了行动。

咦,他们走得似乎也不慢啊,难道他们不怕地雷么?楚云飞又小心跳上一棵树,张望过去才发现,原来这几个搜索队前面全是用雪獒来领路的。

马哈苏德本不想用雪獒来带路,这东西看守营地实在是很管用,前两天追击时又接连死了三只,营地里满打满算也就4只了。

可在阿孜克的强烈要求下,他还是同意了盟友的请求,关键时刻,不能太小气。

嗯,正好几个搜索队一队一只。

雪獒这东西带路可比山羊管用多了,它的鼻子一直在贴着地面走,遇到生人的气味,它会停止不前,在它们的面前,别说地雷了,所有的简易陷阱都会无所遁形。

除非……除非再下场雪,才能阻止它们发威。

这样可不行,楚云飞马上就反应过来利害关系了,两害相权取其轻,看来,只能暂时放过前往1号地区的那支部队了。

于是,楚云飞直接瞄准那支离自己最近的队伍的雪獒,端起枪来瞄准。

雪獒在营地附近跑得很快,带它的人必须拽着它脖颈上的皮带,才能控制它的速度,而且,不时要勒上一下,以便等等身后的队友。

就在带犬人再次勒紧雪獒的那一刹那,楚云飞开枪了,正中雪獒身子,一不做二不休,顺手又是一个点射,击毙了带狗的人。

然后,那自然是火速撤退,因为,报复的子弹肯定马上就过来了。

第二只雪獒是正在嗅一片埋了地雷的土壤时被击毙的。

当第三只雪獒也被打死的时候,已经接近中午了,那两支没了雪獒的队伍终于用人脚证明了,营地外面,到处都是地雷!

一个队触了绊发雷,埋在土里的绊发雷,还好他们自从狗死了以后一直很谨慎,地雷只炸伤了一个人。

另一个队更幸运,有个经验丰富的老兵踩到了跳雷,没再起脚。旁边的队友在踏雷者的指点下,用铁丝箍住了地雷压簧,取出地雷后直接扔了出去。

\section{第两百零四章 毒蛇再吐信}

现在,只有一支队伍有雪獒了。

但是,那支队伍是阿孜克的人,他们坚决不参与搜索了!

道理很简单,这只唯一的雪獒在他们这里,他们绝对是下一个被攻击的目标。

当然,如果没有雪獒探路,他们更不可能去搜索了,因为,那样实在太不安全了。

这时候的马哈苏德就不再讲客气了,没错,雪獒他是要收回去的,因为这最后一只一定要留下来看守营地了。但是,队伍还要继续搜索。

你不满意么?马哈苏德没有这么说,但是他看阿孜克的眼神里,明白地表达了这种反问。

阿孜克现在可不敢耍什么威风,但是该有的指责还是要说的,“马哈苏德,你开始的时候,并没有告诉我们敌人这么阴险狡诈。”

是的,我确实没说,马哈苏德微笑着回答了盟友,“我亲爱的朋友,我也不知道他们会这么阴险。”

你去哄鬼吧,阿孜克暗骂一声,不过也怪自己的老大噶达斯亚巴,也不想想,像马哈苏德这样人脉和势力都要超过自己的组织,找上门来求助,会有好事么?

当然,阿孜克也知道,老大主要还是想讨好马哈苏德,挣点钱倒是在其次了。

多想无益,他还是努力地为自己的人谋求保障,“要不这样,尊敬的马哈苏德,我的人抽回来给你守营地,你再派一队出去可以么?”

开什么玩笑,那不是主客易位了么?马哈苏德绝对不会同意的,而且,他有理由一定要压迫对方执行自己的命令,那就是“好的开始是可以允许的,坏的趋势一定要制止”。

“阿孜克,我可以用个人名义送你一箱大麻,但是,你一定要执行你们帮忙的内容,那就是,帮我搜出这几只蚂蚁。”

阿孜克盯着马哈苏德看了半天,忽然笑了起来,“哈哈,那我先谢谢先生了,帮忙,那确实是应该的。”

笑声很轻松,但马哈苏德知道阿孜克一向对噶达斯亚巴忠心耿耿,正如自己送出去的大麻一样,只不过相互给个台阶而已。

但是,马哈苏德蛊惑人心的本事不是一般的厉害,要不他也混不到这一步,这不,他真诚地向阿孜克解释着。

“相信我吧,朋友,现在营地周围已经布满了地雷,你我已经是绑在了一起,否则的话,你可以离开,我向安拉起誓。”

前半句,阿孜克相信,后半句,那可就不好说了,离不开是现实,要真的能离开,你马哈苏德会放我们离开么?这个连姻亲都欺骗的家伙!

这个该死的家伙居然还敢向安拉起誓,难道说,传言是真的,他已经把灵魂出卖给了美国人?

不过,再说什么也是多余的了,阿孜克点点头,心情由衷地沉重,“情况确实很严重,我要出去安排他们搜索,敌人很狡猾,我应该提醒我们的兄弟注意安全。”

至此,双方心里都明白,疙瘩,已经不可避免地结下了。

楚云飞并不知道那支援军和马哈苏德起了纠纷,当他发现最后一只雪獒被牵回了营地,马上掉转头去找那支向一号地区进发的搜索队的麻烦。

三只羊已经死得一只都不剩了,他们正在抱着三米多长粗大的树枝,确确实实地在“扫雷”。当然,不是拿笤帚那种方式,而是用滚木的方式,不过,可惜的是,营地的位置是最低的,而且,地势也太不平坦了点。

昨天和今天,楚云飞连续两天都是在远距离开枪进行射击,对方注意到了这一点,就有那脑瓜灵活的人做出猜测:是不是,那个喜欢近身暗算别人的家伙,已经被击伤或者击毙了?

这种意识并没有占据主流,但是却不可避免地影响了大家的思路,现在这支搜索队就是这样想的,需要防范冷枪。

楚云飞冒险从南侧自己布的雷区潜到营地附近,又顺着对方开出的道路悄然地尾随了上去。

猎杀,再度开始!

最开始猎杀的,是位于队伍后方的工程人员,就是设置陷阱的那些人,大多身上扛着镐头绳索什么之类的。

当然,没人说战士就不能铺设陷阱,或者是工程人员一定就不能是勇猛的战士,大家只是分工不同而已。

但是,能把二者完美结合起来的人也并不多。

再加上,四周都是地雷,前面还有人挡灾,后面又是自家的大本营,谁还会考虑有人能来近身偷袭?

把身子缩得低点,防止那远处冷枪的袭击,这才是大家要重点考虑的。

于是,对着后面这些或坐或站,缩头缩脑的工程技术人员,楚云飞又开始了点名。

最先是从一个偷偷跑到一块石头后面抽大麻的匪徒开始下手的,楚云飞很有点生气,强敌环伺之下,你居然有心思跑出来抽烟,是不是太不给我面子了?

他还不知道那是大麻。

点了八人之后,楚云飞已经离大部队很近了,下手也越发地隐秘了起来。

不过还好,就这么一段路途,他已经发现了三枚地雷,这里,是刘宁负责的片区,彼此的手法都是交换过意见的,了如指掌。

终于有个经验丰富同时也负责工程技术的老兵发现了一些不妥,“后面的人干什么呢,快点跟上,小心被影子恶魔盯上你们!”

影子恶魔,那自然是楚云飞的代号,当然,这老兵也只是瞎诈唬一下,免得大家干到太晚,回去赶不上吃那鲜美的羊肉。

这时的楚云飞正掰着第十个人的脑袋,自从他发现对方对拧脖颈有了一定防范的措施后,改变了偷袭的手段。

掰脑袋,那纯粹是技术活,就是左手捂住对方的嘴,右手用力向后压对方的顶门后侧,如果够快的话,没准那些被实施的人可以从头顶上方划个圆弧,看到自己的脚后跟。

一般,他们都能听到自己脖子发出的那声脆响的,不过很遗憾,别人总是听不到。

楚云飞正在这里郁闷呢,虽然俾鲁弯人的胡须和毛发一般都很浓密,但眼前这个家伙实在太过分了点,胡子不但油旺旺的,很滑手,里面居然似乎还有不少那种“囊”的碎屑。

太恶心人了!楚云飞正在这里感叹,耳中就听到了这样的警告声音。

嗯,软绵绵的猎杀行动怕是要告一段落了,楚云飞有点遗憾。这样杀人,虽然从心理上讲有点不够刺激,不过,还是很过瘾的。

但是,头,真的好疼啊。

\section{第两百零五章 混乱中的偷袭}

那老兵并不只是说说就完了,伴随着这警告声,老兵开始回头查看队友。

老兵不看还好,一看才发现,后面那些不敬业的同伴居然落得看都看不见了。

对着未知的危险,大家抱起团来总是个不错的选择,老兵抱着这样的念头,回头来找队友了。

当然,老兵身经百战,低劣的错误是不会犯的,他不厌其烦地又交代身边的同伴,事情有点不太对头,要注意警戒。

身边的人早被这家伙的唠叨烦得受不了啦,没错,你是老兵,提醒得也都在理,可是出来半天,发了十几次警告了,也没见有任何一次灵验啊。

再说,是个人都知道,这两天,大家要防备的,并不是什么近身杀手,而是那个跑得很快的神枪手。

老兵哼了一声没再说什么,知道别人就算嫌烦,也是会注意的,毕竟活着总比死了强。

保持着戒备心的老兵小心翼翼地往回走,但是很不幸,他还是被楚云飞割断了喉管,因为楚云飞的速度比他快得太多了。

楚云飞刚把刀收起,就见到有人向后扭头,一时顾不得多想,直接就把那老兵拎起来扔到了一颗地雷上,同时身子快速下伏。

身后响起的爆炸声,顿时吸引了所有人的注意,有三个人猫着腰走了回来,想弄明白到底发生了什么事。

楚云飞一见三人交叉掩护的行进方式,就知道不可能再无声地偷袭了,于是捡块石头,向一侧一扔。

轻微的响声立刻引起了三人的注意,脑袋齐齐地转了过去。

楚云飞等的就是这个短暂的机会,起身的时候,膀子微晃,背后的枪就来到了手里,直接就是一通横扫,整个时间不超过一秒钟。

他根本没看对方到底有几人中弹,摸出一个手雷就向搜索队的人数密集的地方扔了过去,同时身形晃动,顺着来路飞快地撤退。

死去老兵的警告,其实还是有相当人在意的,当地雷的爆炸声响起,起码有三个人利落地卧倒,手雷扔过来的时候,其中两人已经开始出枪还击了。

但是,很不幸,其中一人枪开得过于慌张,子弹直接命中了回去查看究竟的同伴身上,被击中的那位,是那三个里面唯一本来没有受伤的人。

混乱还在继续,前面“扫雷”的人听到身后战斗的声响如此复杂多样,慌乱地就地卧倒和翻滚,却有人不小心又滚到了地雷上面,又是一声沉闷的巨响。

这还不算完,营地里北面的那挺机枪听到那里响成一片,一直在待命着、精神高度紧张的机枪手下意识地扣动了扳机,目标就是那个方向。

“哒哒哒……”密集的子弹,有如暴雨一般泼了过去。

在营地机枪手的印象中,这几天里,只要什么地方响起了枪声,那肯定就是隐藏在暗处的魔鬼所为。自己的战友,似乎总是找不到先开枪的机会。

机枪响了足有两分钟,帮忙压子弹的那位才回过来点味道,谨慎地提醒机枪手,“喂,别打了,那里,似乎是去1号地区的搜索队在的地方啊。”

机枪手闻言就是一愣,品味了一下这话,马上停止了射击。

纵然是如此寒冷的天气,冷汗,还是马上就从机枪手的额头流了下来。

不过,搜索队这边还好,他们初遇大敌,全在地上趴着呢,横飞的机枪子弹,并没有机会伤到他们任何一人。

可这么一来,他们也再没有尾随和射击楚云飞的机会了,自家的机枪,火力还真是强得没话说,压制得他们头都不敢抬。

但是,这机枪也给楚云飞带来了点烦恼,扇型面积的扫射,是令人极度不安的。在战场上,死于散射和流弹的,远远要大于死于精准射击的人,成树国的大腿就是个活生生的例子。

他们就不怕误伤到自己人?

楚云飞放低身子,像一只虾米一样,身体一弓一弓地贴地急蹿,走的还是老路。

没办法,这里是刘宁的布雷区,楚云飞虽然能认出大部分地雷的埋设,但是他哪里有那么多时间去一一分辨?只能走原路。

还好,机枪没响多久就停止了射击。

不妥!楚云飞猛然间觉得寒毛直竖,一种莫名的惊觫在瞬间涌上心头。

这种感觉,已经说不清楚救过楚云飞多少次了,他顾不得暴露身份的嫌疑,使劲向前一跃,就地连续几个翻滚。

说时迟,那时快,一秒种后,“嗵”地一声轻响,紧接着又是一声巨响,一颗杀伤枪榴弹在楚云飞刚刚停留的位置爆炸。

连续的翻滚,让楚云飞的脑袋又有点轻微地疼痛,不过还好,这并没有影响他的判断能力。

枪榴弹,是从一棵大松树后发射出来的,楚云飞膀子一动,枪又甩到了手中,然后快速扣动扳机发出个三连射,这一系列动作,是在两秒钟内完成的。

树后,一个身影惨叫一声,张开双手缓缓倒地,手足无力地抽搐几下,终于不再动弹。

这是马哈苏德挑出来的死士,这几天,他一直在考虑一个问题,既然这新纳山谷一直是自己的地盘,没道理对方比自己还熟悉这里。

那么,他们敢在密林中偷袭这里的人,我的人,为什么就不能学学他们,也这样呢?

马哈苏德就这个想法询问了下面的人,但是很遗憾,他的手下比他还明白那隐藏在暗中的敌人的可怕,没有人赞同这个计划。

想想披着白布在雪坑里埋伏的人都能被无声地杀死,1号地区上所有的二十名值守人员也莫名其妙地被全部干掉,跟人家比山地作战?还是省省吧。

可阿孜克的人听了这个计划,却觉得很有可行之处,他们与马哈苏德的人不同,马哈苏德的人主要学习的是城市里如何搞破坏、如何制造恐慌、如何隐藏身份,还有就是,对受众的洗脑。

阿孜克的人就不同了,他们是啸傲山林和平原的匪徒,也许他们的军事素养赶不上山谷里的主人们,但是,讲到如何打闷棍、偷袭人,还有在山林里如何隐藏行迹之类的这种能力,马哈苏德的人拍马也赶不上他们,这才叫经年积匪!

\section{第两百零六章 营地也不安全}

当马哈苏德听说了土匪们的这种反应以后,极力地怂恿阿孜克派人出去执行刺杀,看到阿孜克兴致不高,马哈苏德索性直接开出了价码:每杀一个人,除了可获得对方身上所有物品外,他还奖励五百美圆。

阿孜克细想想,也不愿意挡了弟兄们的外财,再说,暗中的敌人毕竟是双方共同打击的目标,而且现在也对自己构成了极大的威胁。

于是,阿孜克在队伍里征集了一下部属们的意见,反应却是空前的强烈,有十几个自认骁勇的人,愿意拿命去博那区区的五百美圆。

经过筛选,大家排出了名次,决定了出击的顺序,死的这个就是排名第一的。

本来,在雷没扫干净之前,是不能出去猎杀的,但此人求财心切,带了一包食物和武器,直接就跟着搜索队往一号地区去了。

楚云飞在偷袭之前,行动蹑手蹑脚,加上他深谙隐藏之道,并没有被这个杀手发现。但是,他整出那么大动静之后,杀手想不警惕都不行了。

加之楚云飞一路狂奔而回,就被杀手注意到了,眼看此人速度如此惊人,枪法不错的杀手也没信心能用子弹击毙他,于是就换了个枪榴弹到枪上去,指望能一击毙命。

没想到,由于实力相差不啻云泥,杀手杀人不成,反被击毙。

楚云飞登时就被吓出一身冷汗,这汗,不是为自己流的,他是在庆幸:幸亏在自己的坚持下,刘宁他们没有参与这密林搏杀。

他不知道,刘宁和其他三个人,一觉醒来,正分成两组,趴在山头上,严密注视着山谷里的动静,把楚云飞一个人丢在那里,谁也不放心。

经此教训,楚云飞再次放慢速度,一边小心身边动静,一边继续向营地摸去。

不过,再没有出现异常的动静,楚云飞顺利地潜到了营地的南侧,回到了他自己的布雷区。

脱离险境,深出口气的楚云飞忽然间又愤愤起来:妈的,居然玩阴的,得给你们点教训才对。

他可没想想,他自己一直在玩阴的。

于是,楚云飞摘下步枪,找了棵树跳上去,稳稳地瞄准那个刚才乱开枪的机枪手,扣动扳机。

“哒哒哒”,三声沉闷的枪声响起,这么多天,破天荒头一次,有人在营地里被冷枪击毙。

整个营地马上就炸了锅,恶魔,恶魔要进攻营地了!

以前,楚云飞闹得再凶,也没打过营地的主意,所以,包括马哈苏德在内,所有人都觉得,在层层防守、陷阱遍布的营地里,基本上可以算是安全的。

当然,他们也不可能因此而放松警戒,这种低级的错误谁也不可能犯,“营地里是安全的”也只是一种心理暗示,暗示大家,“打不过的话,我们就回营地,起码那里是安全的。”

现在,楚云飞这无情的几枪,把所有人的梦想击得粉碎,至少,现在没有什么地方是安全的啦!

这突如其来的变故,弄得大家都愣在了那里,过了好半天,才有人想起,应该还击的,可这时候的楚云飞早跑得影都没了。

下一刻,营地里的人再也不大摇大摆地走动了,机枪手和迫击炮手都老老实实地躲进了掩体,而不是探着身子四处张望,积极地准备火力支援。

只有那些确实有事的人,才继续在营地里穿行,不过,个个也都如同拉肚者寻找厕所一般地,故做镇定而步履匆匆。

眼见这种情况愈演愈烈,马哈苏德可真的生气了,纯粹是一帮胆小鬼嘛,我的训练营,怎么净培养出这种懦夫?

火气上来,他也顾不了那么许多,喊了桑丘过来,“你,马上把营地所有的人都集合起来,咱们出操。”

“出操?”现在这位的表情,真的如唐。吉诃德的那位跟班一样,脸上全是那种充满了痴呆的愣意,“首领,咱们……咱们可是很久没出操了,自从那些魔鬼来了以后。”

“没错,”马哈苏德冷冷地回答,“几个小丑就把你们吓成这样,我真的非常悲哀,这就是我们伊斯兰的勇士们吗?穆罕默德的在天之灵是不会原谅你们的!”

“以前我没注意,现在我决定了,从明天起,按正常时间,恢复出操,这个话,你一会儿替我宣布一下。”

替你宣布?桑丘马上就明白了首领的意图,原来,首领自己是不打算出去的呀!

不过,这也很好理解,首领身上肩负着太多的重担和使命,他不应该,也是不屑,同这些跳梁小丑们一般计较的。

得了命令的桑丘,马上出门宣布了首领的最新指示,不多时,营地所有的留守人员,都在营地里集合了起来。

桑丘开始训话。

这时的楚云飞,刚穿过自己布的雷区,找了棵树跳了上去。妈的,这是我布的地雷么?怎么到处都是,我一晚上能布这么多雷?

他用望远镜观察到了营地里的动静,但是,他手头又没什么重火力,只能看着那人堆徒呼奈何。

一边感叹着,他一边举起了枪,打算干掉那个正在讲话,似乎是个头目的人。

扳机已经扣到了第二道,他突然想起了自己的计划,沉吟半晌,终于缓缓地松开了手指,微微叹了口气,算了,还是让你多活几天吧。

他想到的是,既然对方有了这种集合的习惯,那一定要在最合适的时间对他们做出打击,不能让这些家伙学习那炸了窝的蚂蚁,有四处逃命的机会。

这样说来,那些先期被清理掉哨兵的高地,该能有最大的发挥余地的。

楚云飞早就计划好了,那些突出部,将来都要作为自己的攻击平台的。

当然,人这么少,不可能每个突出部都有人的,那么,大多数的突出部上,应该制造那种无人值守的机关,通过线香控制,在一定的时间内,利用些抛射装置,把大量的弹药发射到营地。

当然,制定这个计划时,楚云飞就发现了它的可操作性受到一些瓶颈问题的制约。

所以,楚云飞一早就把刘宁他们打发出去,去买可能用得着的东西,也就是说那些零散的生活日用品,楚云飞是把它们当攻击必需品来采办的。

\section{第两百零七章 缩做一团}

马哈苏德真的有点出离愤怒了,藏在暗中的敌人实在是太狡猾了,根本不同自己的人正面交手,而是极其阴险地一点点地蚕食着自己的实力。

现在,他们更猖狂了,居然在大白天就明目张胆地摸到了驻地附近,射杀了营地里的防守人员。

他端着茶杯的手不住地颤抖着,看得桑丘暗自摇头:这群恶魔,他们把首领气坏了。

只有马哈苏德自己明白,手为什么会发抖,他不仅仅是生气,更关键的是,他害怕了,真的害怕了。

他之所以要纠集队伍出操,培养战士们的勇气是一方面,他更想通过这样的展示,表明自己无畏的态度,为营地里低迷的士气注入些信心。

当然,他还有别的考虑,不过,那是不合适说出来的。他想看看,对方的底线到底是在哪里,或者说,他们还有什么手段没有拿出来。

起码,那传说中屠村时所使用的重型火力,马哈苏德至今还没有见识过,这绝对是不正常的,那种强大的火力,是种必然的存在,还是只是一种偶然?

必须探测出对方的底线在哪里,反正,已经死了这么多人了,也不差再多死些人。这样的想法,自然不合适向任何人提起。

如果对方真的没有太强大的火力,那么,躲在营地里不出去,无疑是个不错的决定,不但可以减少人员的伤亡,还可以设置下陷阱引诱对方上钩,从而一举消灭他们。

当然,他们要真的有那么强大的火力,只能集中力量,拼死突围出去,然后再细细盘算复仇的事宜。

想到这里,马哈苏德传令,把那几支基本上在原地停留不动的搜索队喊了回来,大家趁着天还亮着,继续完善营地的防备工事。

楚云飞在树上看得莫名其妙,这些家伙又想玩什么花招?连搜索都不执行了,他们总不是想就这么坐以待毙吧?

等到看到回了营地的人开始热火朝天地扛沙袋,打桩,楚云飞才反应过来,原来,这些家伙想死守营地,再不出来了。

轻蔑地笑了一下,楚云飞摇摇头,既然你们打的是这样的主意,那么,大家歇歇也算,倒要看看你们还能玩出什么花样。

于是,楚云飞悄然折返,找战友们去了。

不过,有福之人不用忙,没福的人累断肠,等他回到藏身处,才发现战友们都不见了,顿时又焦虑了起来。

略微镇定一下,楚云飞摸摸几张毛毯,那里是满手的冰凉,看来他们是早就出去了,估计,是不放心自己单独行动吧。

考虑清楚了这点,楚云飞确定几人弄不出什么太大的动静,也就懒得出去寻人了,拿起张毛毯裹住自己,昏昏然睡去,他实在是太累了。

可在这种极其亢奋和警觉的心情下,楚云飞睡了不到四个小时又醒了,这时战友们都已经回来了,为了不吵醒他,大家都在那里靠着石头打盹。

揉揉自己的越发疼痛的头,楚云飞掀掉毛毯,爬了起来,眨眨眼睛,眼泪还是禁不住流了出来。

用眼过度疲劳,又没有得到很好的休息,他每次刚醒来的时候,眼皮总是如针扎一般地痛,要流不少的眼泪,适应半天才能缓过劲来。

成树国正在那里假寐,听到楚云飞醒了,也睁开了眼睛,却正好看到楚云飞在擦眼泪。

他自然知道云飞为什么会流泪,不过还是打趣着自己的战友,“怎么,想琳琳了?感情挺丰富啊。”

楚云飞笑着啐了他一口,“呸,我在想刘中勤呢,你这家伙,嘴里从来说不出什么好话,不过,说实话,眼睛疼还不算什么,我现在这头时时地疼,疼得叫我好难受。”

刘中勤,是成树国家里的那位,特别擅长写信的。

“嗯?”成树国警觉了起来,“是不是累的?不会是病了吧?”

楚云飞苦笑着摇摇头,“不是病,是那种抽的疼,整个脑袋就象被紧箍咒箍着一样,尤其杀人的时候,才真叫个疼。”

说着,他又微叹口气,显然是很为这事苦恼。

其实,他还是把痛苦的感受淡化了,那种痛苦,实在让他难受得恨不得用头去撞石头,那是种深入脑髓的疼痛。

成树国拍拍他的肩膀,“算了,没准是休息不够的缘故。你还是多歇歇吧,这里有我们在,你不用太紧张,再睡会儿吧。”

楚云飞无意识地刮刮鼻子,眼泪却又止不住地流了下来,吓得他赶紧放下手,尴尬地笑了一声,“哈,我可真睡不着了,对了,你们今天观察到什么新情况没有?”

成树国摇摇头,“没有,不过,我们都很奇怪,怎么马哈苏德把人撤了回去,难道,他真的要放弃抵抗了么?对了,你还是老实躺下闭眼养养神吧。”

楚云飞也不再坚持,又坐了下去,闭目养神,嘴里还在反驳,“那也不能掉以轻心,今天有个埋伏的家伙给我来了一下,亏得我够警觉,要不铁定挂了。”

这时候刘宁插话了,“我知道你的意思,无非就是说我们水平不够,这个不用你说我们也知道,今天你的表现,我们都看到了。”

成树国也跟着一笑,“说实话,不服不行,我还好点,辛汗看得眼珠子都快掉下来了。”

刘宁语气变得严厉了起来,“行了树国,你还嫌他不够骄傲是怎么着?我的意思是,让他以后在行动的时候多注意点,要知道,这些都是些心黑手辣的家伙。”

刘宁年纪最大,成树国和楚云飞也被这样教训习惯了,于是大家都闭嘴不说话了。

不过,成树国马上又想起了一事,“对了,既然他们都没敢出来,我们生堆火吧,正好把云飞前天弄来的那只鸡烤了吃。”

他们在山的另一侧,如果营地没人出来的话,点堆火确实不打紧。

看到大家都不做反对,成树国一瘸一拐地出去寻柴火去了,辛汗看看刘宁,见他摇头示意,马上就跟了出去。

楚云飞就这么放心地打坐起来,心情一放松,很快就进入了状态。

把他从打坐中唤醒的,是一股浓浓的肉香,想来是成树国又在大显身手了。

\section{第两百零八章 关紧笼子}

闻到这久违了的香气,楚云飞惊奇地发现:刚才还疼得要炸开的脑袋,马上就不疼了。

然后,成树国的声音入耳,“操,这只鸡怎么这么小?先声明,我是伤员,又是大师傅,我要吃一半。”

“等等”,楚云飞直接蹦了起来,“好奇怪啊,闻到这个味,我的脑袋‘Biu’的一下就不疼了,树国你加了什么作料在里面?”

成树国愣了一下,“没什么东西呀,就是点盐啊,孜然啊什么的。”

倒真是奇怪了,楚云飞百思不得其解,“难道说,我这个头疼是‘胃亏肉’?”

成树国把鸡一晃,“嘿,拉倒吧你,就算有‘胃亏肉’这种毛病,那也该是胃疼吧,和头会有什么关系?”

不管怎么说,楚云飞的头疼是好了,面对着这好不容易的一顿热乎饭,吃得他是小肚溜圆。

鸡肉是少了点,不过没关系,他们带得有羊肉的,平时不能生火,所以啃不动,现在可是美餐的时候了。

吃饱喝足,大家又围着火堆懒洋洋地聊了会儿天。

看看表,已经是当地时间九点了,楚云飞意气风发地跳了起来,“好了,弟兄们,开工了。”

有啥也别有病,头疼尽去的楚云飞,现在实在是要多精神有多精神了。

听不懂汉语的辛汗和他的同伴目瞪口呆地看着眼前这个家伙,过了半天,辛汗才谨慎地发问,“我们今天做什么?”

楚云飞白天的诡异行动,获得了两个穆斯林的高度敬仰,男人,总是尊重强者的。

虽然这个强者在白天的行动并没有被完全观察到,但是,仅仅从效果上讲,就已经很震撼人了。

今天,楚云飞的计划就是,把剩下的地雷全部埋好,尤其是刘宁那里,白天被开辟的那条通道,必须在晚上重新封死。

再然后,就是尽可能地多做陷阱,楚云飞有种感觉,别看对方眼下这么着意于防守,但谁能保证下一个白天,他们不会拼死突围?

约定了大家的集合地点和时间,四个人又忙乎了起来,成树国依旧是老任务,为大家观察动静。

凌晨五点,大家放下手中活,又聚集到了一起。

地雷已经全部布放完毕,每个人都或多或少地做了一些陷阱。

说起做陷阱,辛汗和他的同伴平时就是以打猎为生的,做这种营生,实在是很简单的事情;楚云飞对这个也不算外行。

刘宁对这个基本上一窍不通,所以,他大部分时间是用来布新地雷和调整雷区的布局。

接下来,楚云飞宣布了下一步的计划,在各个突出部,安放那些捆扎好的炮弹和炸药包,通过钢丝和树枝,制作简单的抛射器,再通过线香控制投掷的时间。

这话说起来容易,做起来却着实有些难度,忙到七点天大亮的时候,在四个突出部,四个人也才做好十六个这样的东西。

十六个就十六个吧,楚云飞通过对讲机通知了同伴,两人一组,回来拿上炮和炮弹,埋伏在两个没装机关的突出部。

忙完这一切,已经是早晨八点钟了。

马哈苏德的人,果然又出来出操了,就在这时,一支线香燃到了根部,机关发动,一个炸药包腾空而起,在众目睽睽之下,大摇大摆落进了营地。

不过,这种简易的抛射器显然不会有什么太好的准头,炸药包在营地东侧空无一人的地方爆炸,炸断了最后一只雪獒的左后腿。

营地里出操的人登时乱做一团,大部分人是原地卧倒,也有那机灵的,直接窜进了屋子,或者找到一些掩体来掩护自己。

营地里报复的子弹和炮火立刻淹没了那个突出部。

可这十分钟的炮火覆盖刚刚结束,又一个突出部那里,一枚迫击炮弹又飞了起来,这次,落在了营地的北头。

那么,又一顿还击的炮火是应该的,但这枪炮声响起还没有两分钟,又是两个炸药包被扔了进来。

这下,乱成一锅的蚂蚁们都不知道该向哪边射击了。

不过,人民群众的智慧是无穷的,他们马上就发现,似乎,各自为战也是个不错的选择,于是,各个火力点向着他们想象中的目标射击了起来。

这样一来,虽然他们能够压制住想象中的、恶魔的火力点,但不可避免地分散了火力规模。

他们在这里压制,可那些机关并不是仅靠子弹或者说弹片就可以被压得“不敢抬头”的,还是有爆炸物不断地腾空而起,重重落下。

终于,不知道是谁安放的机关,抛射的一枚迫击炮弹准准地落进了一架机枪的掩体内,爆炸声响起,机枪手和压弹员直接被炸得飞出了掩体。

营地里越发地慌乱了起来,看到时机成熟了,楚云飞一声令下,两门八二无同时开火,两支火箭筒也瞄准各自的目标发了出去。

四个被打击的目标,有三个当场就哑火,那可都是有固定掩体的重要火力点。

爆炸发生得太频繁了,以至于都没人注意到,有精确瞄准能力的这两个火力点,不是由机关发射的。

过得片刻,那三个哑火的火力点有一个被修复了,配合着其他火力,子弹向暴雨般扑向了这几个突出部。

但是,发射过弹药的人们早就溜之大吉了,只有那无辜的山石在承受着钢铁制品和硝烟的摧残。

马哈苏德现在能做的,也就只有继续在那里暴跳如雷了。

怎么办,这仗明显是不能再打下去了。

分兵驻守各个突出部的话,只能遭到对方无情的屠戮,被一点点地吃掉。

可值守人员全撤回来的话,外围那些突出部就为对方打击营地提供了支撑的平台。

最该死的是,1号地区也不掌握在自己的手里,否则,那里对于大部分突出部而言,绝对是个致命的火力压制点。

可现在,营地四周都被布上了地雷,就算有富裕人手能出去占领1号地区,也没办法执行这个行动的。

双方纠缠到今天,马哈苏德终于明白了一件事,那就是,对方真的没几个人,否则的话,也没必要把如此多的弹药以近乎挥霍的方式抛洒出来。

当然,这也说明:对方火力,确实如传说中的那么强大,起码弹药的储备,是非常惊人的。

\section{第两百零九章 郁闷的阿孜克}

但是,这一切,知道得都太晚了,现在,慌乱的人们已经出不了营地了。

马哈苏德仔细地想了想,他不得不承认,对方的计划,确实是相当不错的,哪怕,从一开始自己就了解了现在已经遭遇到的情况,他还是没有很好的办法来预防。除非他一开始打的就是跑路的心思。

归根结底,还是个实力的问题,以隐藏在暗处那帮人神出鬼没、杀人无形的超强作战水平和火眼金睛的辨识能力,自己能做的,不过是尽量能减少点被动和损失而已。

可话又说回来,如果,能力保1号突出部不失,那就会赢得很大的主动,想到这里,马哈苏德才想起,昨天,似乎向1号地区推进了不少。

那今天,就努努力,哪怕多死几个人,也要打通1号地区这个制高点的路途,重新占据它,最起码,这里和营地可以形成个犄角的钳形结构,相互支援,可以压制对方肆无忌惮的火力。

命令下出去不久,营地里又听到了轰然的爆炸声,又是一次爆炸品的抛射,楚云飞他们白天新做的机关。

这样的爆炸,对营地里的人造不成多大的损失,但对士气的打击可是致命的。已经有人在私下里嘀咕了,大致的意思是,马哈苏德应该勇敢地站出来面对自己的仇家,而不是拿着伊斯兰勇士们的宝贵性命来解决个人恩怨。

更为重要的是,这种爆炸经常发生在营地四周,对人起不了什么作用,但能极大地破坏那些苦心营造的陷阱。

尤其是那些五公斤装的炸药包,爆炸产生的巨大冲击力,可以把周围相当范围内的陷阱彻底地破坏掉。

马哈苏德正在这里恶狠狠地咒骂,又一声爆炸传来,这次声音比较远。

过不多久,有人来汇报,“首领,昨天开出的路,今天又被埋上雷了。”

操,这恶心事还有完没完了?马哈苏德实在是有些无法忍受了。

努力让自己冷静下来,马哈苏德开始考虑这事的解决方案。

营地的人已经指望不上了,对付这种飘忽不定的杀手,现有人手严重不足,再说,大家现在似乎连营地都出不去了。

那么,还是找部落里的人求援吧,该死的“神圣伊斯兰”,他们的人怎么还不到?

为了保险起见,马哈苏德尊崇着“基天”的传统,身边坚决不允许有高科技通讯产品存在,一般的传信都是通过人力来完成的,所以,现在这样的情况,怎么求援也是个苦恼的事情。

还好,马哈苏德这里还有一只信鸽,虽然一向不怎么使用,严重缺少锻炼,但是,飞那么五、六十公里传个信,问题还是不大的吧?

为防鸽子迷路或被猛禽捕杀而起不到通讯的效果,营地原本是有六只鸽子的,可由于他们一向不注意这种通讯方式,后来就不知道怎么回事,越来越少,现在只剩下这么一只了。

信鸽带着信件刚刚起飞,远处的山林中就传来了一声枪响,那只可怜的灰鸽子翻着跟斗栽了下来。

完了,马哈苏德彻底地失去了信心,算了,还是让大家撤回房间,通过房屋的窗户,进行射击吧,这也实在是没有办法的办法。

这样的决定一旦做出,那就只能被动挨打了。

不过马哈苏德也不是个只知道挨打,不知道还手的主,他马上找到阿孜克,要求阿孜克马上把能派出去的杀手全部派出去,“你也该明白,现在,我们已经到了生死存亡的关键时刻,外面,不会有救兵了,我们只能指望我们自己了。”

阿孜克自然知道他说的是实情,所以很痛快地就答应了,“没问题,他们总不能把树上也埋上地雷,我要我的人从树上出去,找两个最擅长爬树的人出去报信,其他人,就埋伏起来,准备杀人好了。”

但是,现在山谷已经被那些潜伏在暗处的恶魔监控起来了,白天派人出去,是非常不妥当。对于这一点,阿孜克肯定是实话实说的。

“但现在不行,他们会被发现的,只能晚上悄悄出去。”

马哈苏德听到阿孜克有办法出去报信,心头不由得一阵狂喜,“应该的,我们的勇士不怕牺牲,但无谓的牺牲我也是不赞同的,就晚上行动吧,对了,我去写信。”

这次,他计划出点血本来喊人,那么,这邀请函的内容就实在不方便阿孜克的人带口信了,省得阿孜克知道自己大幅提高酬金,造成新的怨恨,那样就有点划不来了。

可阿孜克也不是笨蛋,你这信是封了口了,不过,收信的人怎么可能知道马哈苏德写的信原本是有没有信封的?

所以,拿到了马哈苏德的信,阿孜克回到住处,第一时间就撕开了信封。

阿孜克关心的,自然还是钱的问题,到目前为止,同来的五十个人已经死了四个,还有四个被爆炸的弹片所伤,这笔费用,自然是要算在马哈苏德的头上的。

可是,大家也都知道,马哈苏德名气不小,摊子也铺得够大,但他没有得到、也不可能得到克普塞部落的大力支持,主要收入还是靠个人的名声,从一些支持者那里弄钱,经济上实在是困顿得很。

而且,不少人投奔马哈苏德的原因,无非也就是看中了马哈苏德有从支持者那里弄到钱的本事,纯粹是找安心日子来过的,这种人的存在,也让马哈苏德开销大了不少。

看到马哈苏德以高出自己费用五倍的价格来求人帮忙,阿孜克的火就更大了:他以为我们是什么人,随便给点小钱就能打发?我们可是土匪哎,从来都只有我们欺负别人的份,今天倒好,打了个颠倒。

当然,阿孜克能独自带着人马出来,也并不是不顾大局的人,这笔帐,回头跟马哈苏德慢慢算,现在大敌当前,实在不是撕破脸的时候。

不过,隐藏在暗处的那些家伙,只是马哈苏德的大敌吧,跟我们这些土匪,也未必能有什么深仇大恨。

可他能做的,不过也就是这么想想,要他直接跟马哈苏德翻脸,一则老大未必能原谅了他,二则就是,马哈苏德还有他身后“基天”,名气确实太大了,等闲是招惹不得的。

倒像是我们自己凑过来找死似的,阿孜克恨恨地想着,居然有些抱怨起自己的老大来。

\section{第两百一十章 全面袭击}

等到接近十点钟的光景,营地里所有的人都进入了那三十几栋房屋,三个掩体上的三挺机枪也被搬了进去,还有一箱箱的子弹。

只有一个迫击炮的掩体里,还有三个人猫在里面,那东西拉到房子没办法用,他们只能在这里呆着。

看着空无一人的营地,楚云飞真的有点狐疑,他们就这么放弃了外面的工事,要躲进房间里抵抗了么?

房屋都是木制结构的,但大雪刚过,潮气浓重,用火攻那是想也不用想的事情。

对方显然也想到了这点,才大摇大摆地把人手全拉回了屋子里,同外面的工事相比,屋子里不但可以防备从天而降的爆炸物,防卫起来也比外面的工事容易些。

但这样也存在不少弊端,最重要的一点就是,进了屋子,火力支持起来就不方便了,这样只能是个被动挨打的局面。

到底发生了什么事?楚云飞百思不得其解,不过,无非也就是坐等援兵吧?就跟那只鸽子一样,想再拉点垫背不就是了?

想到这里,楚云飞又开始盘算起弹药来,地雷虽然已经全埋进去了,子弹炮弹之类的也用了一些,但如果真来了不少援兵,需要暂时撤离的话,三辆吉普车还真未必能装下那么多东西,毕竟是分两次拉来的呢。

装不下,那就先消耗掉一些好了,反正,现在营地里的火力已经转移到屋内了,自己和同伴们,可以做些精准射击了。

考虑到山地作战,楚云飞是买了一门100MM迫击炮的,那炮还是中国造的80式,标准的中国军队列装武器。实在不知道那个克努蒂的长老是如何搞到这个东西的。

他是想多买几门的,不过,这东西实在是不太好买到,弹药也是个大问题,他现在手头也只有八十发炮弹,全是杀伤榴弹的那种。

后来他也想通了,自己并不是要发动场战争,怕是在战场上,多的还是近距离消灭对方,迫击炮这种东西,买不到就不买好了。

严格说起来,82MM无后坐力炮和火箭筒倒是实用多了,起码能抵近了打。

不过,现在这种情况,还真是用迫击炮比较划算点,它的弹道非常高,标准的大弧度抛物线结构。找块大石头,藏在后面支起炮来,还不是想怎么打怎么打?也不用考虑会有人员伤亡。

想到就做,楚云飞马上跑去背那迫击炮,跟他一组的克努蒂人也抗了两箱炮弹跟着。

将近一百斤重的东西,楚云飞扛着它在山路上轻松地行进着,把跟他相随的穆斯林看得目瞪口呆:这么瘦的身子,居然能有这么大的力气?

刘宁他们也回来,把迫击炮弹搬了一大半去2号突出部。

随后,楚云飞又扛了火箭筒和火箭弹一大堆东西到4号突出部,那里,离营地最近,而且,离外面那个迫击炮掩体也非常近。

等到中午十二点多的时候,一切都准备好了,大家吃点饭休息休息,就到了下午将近两点。

两点整的时候,刘宁的迫击炮开始发威了,目标是营地里那个迫击炮掩体。

先头两炮,只是校对方位,掩体并没有受到实质性的损伤。不过,掩体里的人却着实被吓了一跳,马上支起炮架,开始还击。

木屋里,两挺机枪探出了枪管,向着刘宁那个方向疯狂地吼叫了起来。

掩体里,一个似乎负责校射的家伙站起来,想看看炮击方位,却被远处的成树国一枪撂倒在炮位上。

紧接着,刘宁的迫击炮终于落到了掩体附近,而木屋里的人看到机枪对刘宁构不成威胁,马上掉转枪口,冲着成树国藏身的地方就是一通扫射。

可大腿受伤的成树国早一个翻滚,躲到一块石头后面去了。

掩体里的人并没有受到什么伤害,那俩活着的跳出掩体就想跑,这时,楚云飞的火箭筒响了,直接将二人炸上了天。

跟楚云飞一组的穆斯林也架起了火箭筒,目标是那两挺机枪之一,这人在上次屠村时就表现出了极好的枪法,这次,也不例外,火箭弹轰然炸响,那挺机枪也停止了吼叫。

刘宁还在继续轰炸那个掩体,看他的意思,不把那门迫击炮炸坏绝不罢休。

又一挺机枪伸了出来,这次,目标是楚云飞他们所在的4号突出部,对方也意识到了,这个制高点离营地太近,不能让这里出现什么情况。

对此,楚云飞早有准备,一把就按下了那个穆斯林的头,“躲到到石头后面去!”

那穆斯林没有听从这个指挥,而是趴在地上恶狠狠地说,“你、你、你,你居然用左手摸我的头?”

呃,左手,不能摸头么?楚云飞想了半天,才依稀想起来这个典故,不过,那似乎是印度教的习俗,跟穆斯林没什么关系的吧?

不过,这种场合,这种事情实在没办法计较的,楚云飞一个翻滚,向石头后面藏去,嘴里还在教训同伴,“行了,你希望机枪子弹摸你的头么?还不退过来?”

最初的惊怒过后,那穆斯林也反应过来了这个问题,有样学样地滚到了石头后面。

楚云飞早把身上的步枪摘了下来,趁机枪停顿的工夫,探头就是一枪,下一刻,一声惨叫传了过来。

那个枪法很好的穆斯林还在那里解释着,“事实上,我不喜欢别人用左手摸我的头,这和信仰无关。”

和信仰无关才怪,起码也是受了那些印度教徒的影响的,楚云飞撇撇嘴,没再说什么,因为在巴基斯坦,印度教徒和伊斯兰教徒的关系实在是说不上好,谁知道这人有什么隐私呢?

可同伴还在解释,“我这个习惯,辛汗也是知道,你别误会。”

这不说话还不行了?楚云飞再次探头,又是一枪,才缩回脑袋安慰同伴,“呃,这个,我为我的行为道歉,不过当时那么危险,我确实是来不及换手了。”

这样的回答,终于满足了此人的要求,再说刚才情况的危急也是很明显的,他缩缩脖子,不再说话。

楚云飞又禁不住带着点恶意猜测了起来,难道说,这人是印度教徒,可是辛汗也不像是个能跟印度教徒和睦相处的人呀。

他还在这里胡思乱想,刘宁的迫击炮终于炸坏了那门炮,转移了目标,现在的目标是屋子里的机枪。

几炮过去,屋子里的机枪就安静了下来,不知道是压制住了火力,还是造成了人员的伤亡,总之,那三挺机枪都不再吼叫了。

\section{第两百一十一章 惴惴不安}

现在的马哈苏德,别提有多懊悔了,怎么就没想到对方还会有迫击炮呢?虽然目前发射的只有一门,不过,可是绝对地压制住了自己全部的火力。

早知道是这样,还不如一开始就不做还击,老实躲在一边才是正经,现在好了,能用的机枪,只有两挺了。那可都是钱啊,对了,还有这些战士们的鲜血。

或者说,开始就不该撤回屋子,说得更长远点,再早以前应该死守1号突出部来的,也不至于这么被动。

经过这短短的一战,对方已经初步暴露出了本来面目:有强大的火力,人数虽然不多,兵力分配却是非常合理。

最关键的是,对方的单兵素养,简直是太高了,那个一开始隔着老远打掉校射手的人,还有躲在石头后面开枪的那个,枪法简直不能用单单的“精确”来形容了,那简直是恐怖。

马哈苏德不禁扪心自问,我到底,是在什么时候招惹了这样可怕的仇家?而且还是这么有钱的人。

相对于在营地里惴惴不安的马哈苏德,营地外面的楚云飞他们可是舒服多了,终于能在白天比较方便地四处走走了。

火箭弹的威力有点不够,于是,那两门“82无”就成了接下来战斗的主力,本着“打一炮就走”的原则,楚云飞和刘宁各自扛着一门,满山地飞跑。

成树国则带着一支步枪和一挺机枪,转移到了2号突出部,那门迫击炮已经校对得十分精确了,如无必要就不用调整了,所以,他的任务就是看好迫击炮,顺便火力支援自己的战友。

马哈苏德实实在在地尝到了只挨打不能还手的滋味,每隔那么十来分钟,总能听到“轰”地一声巨响,甚至是两声。

“82无”的威力不算太大,不过,对付这些木头屋子,已经足够了,每当有爆炸声响起,两层碗口粗原木做的屋子,总会露出门板一样大小的洞来,甚至有的屋子能直接塌上一半。

不是没有人想过,再用机枪来压制对方,可是,那实在是太不现实的事了。机枪不能长时间响,否则,必然会惹来迫击炮,可要靠临时的那么几枪,怎么能打得中那两门四处移动的炮?

最好的办法,莫过于用火箭筒还击了,你打我一炮,我马上对着发炮的地方回击一下,管打得住打不住,起码也是种有威慑力的还击。

有人这么尝试了,然后发射火箭弹的屋子,就理所应当地受到了重点照顾,两门“82无”交替轰击,直到把那屋子轰成了一堆废墟,攻击者方才罢手。

那屋子里一共有十一个人,没有人能活着跑进别的屋子,要知道,成树国在2号地区还架着挺机枪呢,见到便宜,那是不可能不捡的。

有这么一个样板,已经足够了,营地里剩余的匪徒们只能老实地龟缩在屋子里,心惊胆战地等着炮弹的光临。

从下午到天黑这段时间里,楚云飞和刘宁一共打出去将近七十发“82无”的炮弹,营地里所有的房子都变得走风漏气,至于人员,估计死不了多少,但受伤的人,绝对会是个惊人的数字。

看到勇敢的战士们个个都变成了受惊的小鸡,马哈苏德恨不得一声令下,要战士们冲出去搏杀,不过,他也只能是想一想。

忍了,我忍了,马哈苏德不停地为自己打气,等天黑吧,一定要跟你们一决高下,本来被黑夜折磨得难以忍受的他,居然意外地盼望起天黑来。

在大家的盼望中,天终于黑了下来。

马哈苏德迫不及待地喊来了阿孜克,要他马上组织猎杀队,出去和敌人硬碰硬。

阿孜克被一块弹片擦过头皮,流了点血,神情看上去有点狼狈,“尊敬的马哈苏德,很遗憾,我本来安排了十二个人,被他们下午这么一闹,只能派出去八个了,其中死了两个,伤了两个。”

“这八个人里,有两个是要去送信的,所以,只能六个人能参与战斗,但愿,他们能给我们带来什么好消息吧。”

听到这个消息,马哈苏德又愣在了那里,什么,只有六个人?那够做什么的?不过算了,有总比没有强吧。

“那好吧,阿孜克,我就在这里等你的好消息了。”

好消息真的能传来么?

正如阿孜克所说的那样,是的,树上没有地雷,但是,树下……偶尔是会有陷阱的。

本来,阿孜克出动的这几个人,都是有相当经验的惯匪,树下的陷阱一般是瞒不住他们的,但是很遗憾,他们是天黑出动的,惯匪也是人类,他们没长夜眼。

一个惯匪不小心踏进了一个锁扣上,机关发动,弹力强劲的松枝直接就把他拽到了离地有五米的空中。

脚踝被套,人被头下脚上地倒悬在那里,就算是再有经验的匪徒,也不由得发出了一声凄惨的嚎叫。

这嚎叫不但让试图暗中下手的队友心惊胆战,更直接引起了楚云飞他们的注意。

怪不得白天装得那么可怜,原来,晚上计划了大行动啊,我说怎么觉得他们白天的行动那么蹊跷呢。正吃饭的楚云飞停止了咀嚼,略微思索一下,就站了起来。

“我去看看发生了什么,你们不要跟来,对了,就在这个突出部坚守吧,毕竟有迫击炮在这里呢,一定——不要大意哦,要小心他们的偷袭。”

大家对视一眼,谁也没有说话,因为他们都知道,以云飞的能力,自保该是没有问题的,可要带上其他人去,那就真不好说了。这家伙,似乎天生就是个单打独斗的材料。

刘宁却还是忍不住叮嘱一声,“你也要小心了,现在形式不错,没有必要的话,不用冒太大险,不值得。”

楚云飞点点头,表示知道了,接着几个起落,消失在茫茫夜色里。

刘宁和成树国也三口两口地解决掉手中的食物,大家各自选择地方,隐蔽了起来。

\section{第两百一十二章 猎杀猎杀队}

由于这猎杀队里有人肩负着求援的使命,所以,他们的突破口依旧选在了南侧。

正好,这里正是楚云飞负责布防的片区,行动起来,没有什么不方便的地方楚云飞悄无声息地向发出响声的地方靠了过去,等到靠近了才发现,出营地不远的地方,有个家伙被吊在半空。

这天大概是中国农历十五左右,月亮很圆,能见度不错,不过,楚云飞不是靠这个发现对方的。

自从上次差点上了那个假人的当之后,楚云飞就强迫自己养成个习惯,在夜里看人,尽量使用观察生命能量的方式来观察,那样,要保险很多。

所以,楚云飞在离那人还有两百米的时候就格外操上了心,再有昨天的那种枪榴弹袭击的话,哪怕这里是自己布的雷,仓促之间,也未必就不会出现意外。

仔细观察一番,楚云飞悄悄地抿抿嘴唇,好家伙,果真不出他的所料,有两个人笨手笨脚地在救人,旁边还有两个人在戒备,以这五人为中心,四周树上,居然还埋伏着三个人。

这八个人的战斗经验都很丰富,并没有马上回去报告遭遇到的情况,而是试图先解救同伴,毕竟,多个人就是多份力量的,关键时刻,不能懈怠。

也许还因为自视很高的缘故,他们并不怎么害怕敌人的偷袭有一个家伙离得楚云飞稍微近点,不过也有将近一百八十米,正在警惕地转着脑袋东看西看。

楚云飞很佩服这几个家伙的胆量,知道自己的夜战能力,他们还敢深夜潜出来,实在是有勇气。要不是有人被陷阱困着,真难说会不会被自己发现呢。

他可不知道,这些人全是后来的援兵里的人,是群惯匪。

楚云飞没按直线走过去,而是顺着起伏的沟壑、石头和树木,利用月光造成的阴影悄悄地摸了过去,他感觉得到,眼前这几个人,从占位和临机决断上都很有一套,不象是初出茅庐的新手。

那人还在树上东张西望,并不知道死神已经向他抛出了媚眼,在他的感觉里,现在最危险的,应该是那个被困的同伴和解救他的人。

忽然,一阵轻风刮过,那人骤然间觉得似乎有什么不妥,猛然一回头,却见一柄弯刀自下而上划过。

莹白的月光映上刀身,配着那奇快的舞动,在这皎皎月夜里显得分外地美丽,一种诡异的美丽。

楚云飞伸手臂挡住刚断气的这位,轻轻落在了同一根枝杈上,找个三杈的树枝把人放下,抬眼看去。

不到三十米处,那几个人终于想出了救助同伴的方法,三个人在树下手腕相搭,组成个简单的人肉网,另一人爬到树上去割那绳子。

这样高难度的动作,亏得他们能想得出来,楚云飞微微摇摇头,心里很是不屑。

原因无他,实在是,那人吊得实在太高了,最近的头部离地都接近四米了,这样掉下来,接得住接不住是一说,还得考虑怎么才能避免颈椎受到损伤的吧?

楚云飞不再耽搁,悄然摸到另一棵树下,树上的家伙本来也在东张西望,不过,眼下同伴被救在即,他的注意力不免就被吸引了过去。

眼看着同伴就要爬到那棵树挂绳索的地方了,一直注意同伴的这位忽然发现自己的眼睛转换了视角,在嘴被捂住的同时,他看到了天上的那轮明月。

今天的月亮,真的好圆,这是此人在这个世界上的最后一个念头。

弯刀很锋利,绳子被一刀割断,而且被吊的那家伙反应很快,努力地弓起身子,让脊背落在了肉网里,再加上缓冲,没有受到太大的伤害。

不过,小伤害还是造成了点,他的头重重地撞在了一个同伴的胸脯上,差点把那同伴撞得背过气去。

就在大家着急检查这两位的伤势的时候,外围最后剩下的那个观望哨也被楚云飞无声地干掉了。

那五人乱做一团的时候,楚云飞已经悄悄地打开了步枪的保险。

看看两人都没事,一个人就轻轻打了两个手势,大概是在招呼那三个放哨的同伴,见此动作,楚云飞不再多想,马上扣动扳机,AK47的枪口上火焰狂喷。

他不能让对方有反应的机会,他们发现有不妥当的时候,自然会散开和卧倒,最起码也要做出战斗戒备的姿态,那样打起来就太麻烦了。

反正可以肯定,这群家伙,不是出去报信的,就是来暗算自己这方战友的,留他们不得。

不过,对方的反应却是极其地迅速,除了两个当场中弹的外,其余三人马上卧倒,向身侧的石头翻滚过去。

其中一个家伙命不好,又压在一颗地雷上面,那地雷直接跳起在空中爆炸了,看来不死也是要残废的。

四散的钢珠和弹片似乎又伤了另一人,他发出一声闷哼,和另一个还活着的同伴滚到了一块石头后面。

这里离营地太近,楚云飞不想多耽搁,掏颗手雷出来,压了一下,暗数三声,扔了过去,眼前这几个家伙很明显战斗经验丰富,不能给他们扔回手雷的机会。

紧接着,石头后面“轰轰”两声爆炸,原来,有个匪徒也掏出了手雷,正要扔出,却被前一颗手雷炸得失去了机会。

仔细观察半天,发现对方已经全部阵亡,楚云飞不再耽搁,通过对讲机通报了战友,让他们小心戒备,自己却独自一人又开始了搜索。

既然这里有人出没,谁也不能保证别的地方是不是还有人出动,为了确保战果,更为了同伴的安危,仔细搜索一下,是楚云飞责无旁贷的任务。

不过还好,这样的搜索,并费不了楚云飞太长的时间,毕竟他有自己的特长,在夜里,他的目光所及之处,基本上是藏不了什么人的。

想到这里,楚云飞刚要迈步走人,又摇摇头停住了,我这是怎么了?大脑缺氧了么?这么一堆死人,可不是现成的线索么?看看他们要做什么吧。

就这样,楚云飞一边保持着警惕,一边蹲下开始搜查尸体。

微微一搜查,对方的目的就昭然若揭了。

\section{第两百一十三章 狂轰滥炸}

这群人不同一般的战士,他们的武器非常齐全,弹药捆绑得都非常紧凑,这样行进间不会发出任何的声响,而且还带了干粮和水囊,一看就是打算在这山谷里长期藏身匿迹,伺机偷袭自己这方的。

而且,那两封求援信也被楚云飞翻了出来。

楚云飞冷哼一声,“神圣伊斯兰”和“俾鲁弯解放军”是吧?这次你们不来还好说,要真来了,给战友造成什么损失的话,拼着不回国我也放不过你们!

不知不觉间,楚云飞的性格冷酷了很多,也暴躁了很多,于是,下一刻,他的头又开始若有若无地疼了起来。

接下来的搜索花费了楚云飞将近一个小时,确定了没有什么危险之后,大家继续开工,虽然已经没有地雷了,但刚才的事实证明,多做点陷阱也会对未来的战斗增加点保障。

刘宁做陷阱不拿手,索性把汽车开了一辆来,车里有不少这次买的炸药,又从藏身处起出了大量的炮弹装到车上,成树国居然也一瘸一拐地帮忙扛炮弹箱子。

白天的战斗已经说明了一个问题,仓库太远,会影响攻击的连续性和效果的,现在既然情况不同了,车也可以往前开开了,当然,藏还是要藏好的。

到得近处,那些炮弹箱子还得靠人力搬运,不过,既然是白天会用到,放哪里不是放,索性东一箱西一箱地乱丢算了,用着还方便呢。

楚云飞他们一直忙到凌晨四点左右,看看时间差不多了,三个人扛枪拿炮地上了1号地区,那里离4号突出部很近的。

刘宁还是和辛汗在一起,埋伏到了2号地区,迫击炮也在那里。

马哈苏德这晚上根本就没有睡着,猎杀队的人出去没多久,外面的爆炸声和枪声就响成了一片,偏偏还是没有人回来汇报情况。

这帮人怎么搞的?马哈苏德不由得暗自咒骂,他们就不知道情报优先么?出什么事也得先回来说一声啊!

马哈苏德没想到的是,他的人注重情报和沟通,总以大局为重,而阿孜克的人精于战斗,总以保命为目的。

这纯粹是两种不同的战斗风格导致的,当然,和战斗素养也有关系。

一直等到天亮,那支猎杀队依旧没有一个人回来,可现在派人出去联系就太晚了,毕竟敌人的火力太强了。

马哈苏德忍不住又把阿孜克喊了来,“你的人到底是怎么回事?昨天晚上打得惊天动地,却不见人回来报信,到现在都没个人回来说话,他们会不会全死了?”

阿孜克也在为这事苦恼,“全死了是不可能的,那可全是经验丰富的老兵,打不过还跑不掉么?不过,咱们现在也只能坐等,像他们出去搞猎杀的,拿人头换钱的话,几天几夜不回来也是常事,你没见,他们走的时候都带了不少干粮的么?”

马哈苏德终于暴跳如雷了,“他们就一点大局观都没有么?难道不知道先回来通知我们一声信送走了没有么?现在,麻烦你告诉我,接下来我们该做点什么!”

阿孜克轻叹一声,他也觉得这种情况不太正常,是的,起码该有人回来通知一声,信到底送出去没有啊。

不过,自己的手下是些什么人,阿孜克也非常清楚,那都是些见钱眼开的贪婪家伙。

“信没送出去的话,该有人回来报告的,要是送出去了,他们可就未必会马上回来了,要知道,你开的悬赏是很高的,多杀一个人,就起码能多三个月的享受。”

马哈苏德无语了,凭你们这群垃圾,也能多杀几个人?开什么玩笑啊。他真有点后悔了:早知道,就应该把对方夜战的厉害通报给盟友的。

不过,一开始就讲清楚对手的厉害,那可能么?那样谁还有勇气去送死?

实在是阴差阳错得厉害,可马哈苏德还不能发火,只好委婉地问一下,“你肯定,他们没有全部战死么?”

阿孜克听到这话就是一个激灵,这次实在是不寻常了点,他也内心总觉得有点不安,千万别都死了,那全是队伍里的精英,可是老大的宝贝啊。

不过,对方有这么强大的夜战能力么?阿孜克深表怀疑,“尊敬的马哈苏德,你为什么会这么认为,还有,你的人怎么不参与猎杀。”

得,话说到这里,马哈苏德只好转移话题了,“我的人,实在是些童子军,作战经验不行啊,对了,我们等他们到今天天黑,要是还没消息,我就派我的人出动,一来接应他们,二来也参加猎杀,你说好么?”

阿孜克想想,还没来得及说话,就被突如其来的爆炸声打断了思路,两人都早已经是惊弓之鸟了,条件反射般地趴到了地上。

楚云飞他们,又开始进攻了。

这次,是三个火力点了,刘宁和楚云飞依旧是“82无”,那个不让摸脑袋的穆斯林手提个火箭筒也是到处乱跑。

大大小小的爆炸,此起彼伏地进行着,密度比昨天要大得多了。

而且,由于炮弹到处都有,他们根本不必为了补充弹药而来回地跑,自然大大提高了效率。

对所有新纳山谷的人来说,呃,还有那些支援他们的盟友,这天绝对是山谷里有史以来最为恐怖的一天。

到了傍晚,整个营地惨不忍睹,大部分的房子都坍塌得只剩下一两个角了,木头虽然未必砸得死人,但他们没有躲避攻击的地方了。

不是没有人还击,事实上,开始还有人试图找出攻击的规律,用机枪还击,但每每不能如愿,倒是剩下的两挺机枪在偶尔吼叫了两声后,被报复的炮火轰得彻底不能用了。

这种情况下,就有人受不了过度的刺激,组织人手冲出来拼命还击,但成树国在1号的那挺高射机枪威力绝对不是吹的,甚至有人被打成了一段一段的。

成树国那个点实在是太固定了,在山谷里人的拼命反击下,那挺高射机枪终于哑火,成树国的脸上也被火箭弹炸起的碎石划出了几道血丝。

不过,成树国那里还有车载机枪,终于再度发威,硬生生地遏止住了山谷中人疯狂的还击。

这么一天下来,山谷里的人早死得只剩下一百出头,而楚云飞他们的炮弹也消耗了一半还多。

天终于黑了下来,双方又开始了新的算计。

\section{第两百一十四章 奢望突围}

楚云飞再次强调,鉴于马哈苏德已经是强弩之末了——这是所有人都能确定的,所以,防止他们夜间的偷袭就成了重中之重,一场战斗打到现在,居然没有人员受重伤,证明战略和战术都是成功的,大家一定要再接再厉,圆满地结束这次行动。

当然,在现在这种情况下,一定要继续做好打恶仗的准备,对方肯定已经无法忍受下去了,他们不可能知道自己这方弹药的储备情况,那么他们只有一种选择:突围!

当生命坐等灭亡的时候,一切威胁都不会成为顾虑了,雷区,那是可以用人来趟的,炮弹爆炸也是有个范围的。

现在的任务就是,大家一起,相互照应着把剩下的弹药运上来,准备明天的恶战,当然,楚云飞不参与搬运,他要继续监视营地。

楚云飞已经连续十几天每天只睡两三个小时了,身体已经疲惫到了极点,但是,有那股深深的仇恨支持着他,他甚至相信自己还能再坚持十天以上。

营地里,马哈苏德愁眉紧锁,他已经不需要再知道猎杀队的行踪了,白天,在一系列的攻击中,对方没有任何的人员损失迹象,证明那支猎杀队已经全军覆没了。

否则的话,那是猎杀队最好的下手时机。

马哈苏德甚至可以判断出来,对方只有四到五个人,因为他们的攻击和掩护的火力点,最多的时候也只有四个,这样算来,再多也超不过八个人。

想到这里,马哈苏德忍不住就想哭,自己一手训练起来的二百多的战士,就被区区这么几个人弄得元气大伤,接近覆灭了,这世界上,还有天理么?

他就没想想,那无辜的中国人质,是否也向老天乞求过公道?

只知有己,不知有人的人,按理来说,是该受这样的报应的!

当然,马哈苏德是不可能这么束手待毙的,他喊来了阿孜克,“通知你的人,准备好,凌晨三点,我们准时突围。”

阿孜克的左手在白天被炮弹炸断了,这时的他,手臂草草包扎后吊在脖子上,眼中居然充满了嘲讽的微笑,“哦,你不派你的人出去猎杀了么?”

马哈苏德脸色一冷,瞟他一眼,又放缓神情,耐心地解释,“那些魔鬼在白天的袭击,让我的人遭受了太大的伤亡,实在是抽不出人手了。再说,你那些老兵组成的猎杀队尚且全军覆没,我这些小毛孩子,怕是指望不上。”

阿孜克冷笑一声,“哼,没关系,没准我的人已经把信送出去了,再等两天,也许援兵就来了呢。”

马哈苏德的火气再度被点燃,“你这是什么意思?别说你的人不可能出得去,就算出去了,等援兵来的时候,我们还会活着么?”

这时的阿孜克可不会再买他帐了,实际上,白天那些奋勇反击遭致惨痛损失的,都是马哈苏德的人,他带的这帮人可是比猴都机灵,纷纷找了隐蔽的角落,打死都不出来。

所以,现在营地里活蹦乱跳、没有受伤的六十多个人里,阿孜克的人就有二十多将近三十号人,占了小一半的比例,他不信马哈苏德还敢在这个时候闹内讧。

“那你自己突围好了,牵扯我们,似乎没什么必要,要知道,我们只是帮忙的,他们不会对我们怎么样的。”

阿孜克太清楚马哈苏德是怎么想的了,没有人比土匪更明白实力的重要性了,马哈苏德通知自己配合突围,那意思实在是再明白不过了。

说突围,周围都是地雷,怎么突?

必然是要有人去趟雷的,马哈苏德要他配合,无非也就是不舍得自己的人去趟雷,驱使他的人去干这活,可阿孜克至于这么傻么?

五十人损失到现在这个程度,老大也许还能接受,但要全军覆没的话,不用等老大发话了,还是自己找棵树吊死算了。

话,阿孜克没办法说得太透彻,但他的表情已经明显地告诉了对方:要拉我垫背,对不起,门儿都没有。

马哈苏德终于明白了,原来这家伙在这里等着看他笑话呢,我操,这就是我姻亲派来帮助我的人?

他也没想想,他自己有实力的时候,做的事情,也未必对得起他的姻亲呢。

马哈苏德沉默半晌,终于明白了他自己现在的处境,当然,他没有愚蠢到再闹一场内讧的地步,“那我就自己走了,噶达斯亚巴也不能说我没照顾他的人了,但愿到最后,他能付得起你们的赎金吧。”

参与别人的私人恩怨,最后失败的话,是要付出相应的赎金来表示歉意的,这也是俾鲁弯地区的规矩。

阿孜克动动嘴皮,还想说点什么,但想到马哈苏德未必就逃不过这一劫,那么言语上还是注意点好了,起码到现在为止,双方还没算正式撕破脸,虽然已经是接近于此了。

马哈苏德说到做到,凌晨三点,他的人整顿完毕,受伤的人都留在了营地,只有四十号人跟在他身边,就这样,还是不免有些挂彩的夹杂其中。

虽然,敌人夜战的能力非常强,而且不排除有高精度夜视眼镜这种装备的可能,但马哈苏德别无选择,因为事实已经证明,白天想出去那实在是天方夜谭,只能赌一赌晚上了。

当然,为了以防万一,马哈苏德在队伍中的位置相当靠后。

突围的位置,选在了北侧,一旦突出去,就可以钻进断背山的深山老林,从此销声匿迹了。

既然是自己的人,马哈苏德就不想派人趟雷了,一方面是为了保存实力,另一方面,趟雷产生的动静,实在是太容易吸引敌人的注意力了。

拖得一刻,就是一刻。

他可不知道,楚云飞一直在监视着整个营地,他们这一群人出动,就算没发出任何响声,那一大团明黄色的生命能量,也足以引起他的注意了。

在这样的夜晚,还真没有什么东西能引起楚云飞的害怕,他只有无尽的兴奋:突围,现在就开始了么?

回头看看,刘宁他们早已经走得不见了踪影,不知道在哪里搬运弹药呢,楚云飞没有任何的迟疑,悄然潜了过去。

虽然,那里是刘宁的雷区,但是昨天晚上猎杀队的举动提醒了楚云飞:树上,是不太可能布雷的。

而刘宁,他不会设置陷阱。

\section{第两百一十五章 此路不通}

进入北侧以后,楚云飞没走几步,就意外地发现,一棵树上,居然有个树衩上很隐秘地挂着一颗绊发雷。绊发线,就若有若无地贴着一根小树枝。

我操,刘宁什么时候也学得这么阴险了?楚云飞摇摇头,还好,他是从侧面过来的,要是从营地的方向出来,那绝对是看不到这颗雷的。

而且,月色依旧不错,才让楚云飞逃过这一劫。

意识到了这些,楚云飞的行动越发地小心,还好,再没发现类似的机关。

终于,楚云飞在半小时后赶到了现场。

这一队,大约有四十个人,而且,有二十个人在前面小心翼翼地探雷,速度看起来似乎还不算慢。

见此情景,楚云飞不再犹豫,举起步枪就是两个点射。

两条人影应声倒地。

楚云飞早瞅准一块安全地方,那是一块扁平的大石头,宽有七、八米,一个翻滚就到了那里。

虽然四周情况不明,楚云飞也不敢随便乱动,可他有了特殊的夜视能力,马哈苏德的这点人怎么还吃得住他的算计。

接下来,楚云飞踢飞两颗手雷,连连出枪,而马哈苏德的人只有应声倒地的份。

大概杀了有十人左右,终于又有个家伙为了躲避子弹,不小心再次踏上个地雷,整个人被炸得离开地面两尺还多。

马哈苏德不得不痛苦地承认,这次突围行动,绝对又成功不了啦,原因无他,他们根本就看不到敌人,而敌人随便来两枪,却总能给队伍造成损失。

意识到这一点,马哈苏德悄然下令:行动取消,队伍返回营地。

这实在是没有办法的办法,因为已经有人开始往回跑了,马哈苏德这么做,也不过就是把这个举动合理化,现在他实在没能力去追究到底是谁的责任。

月色虽然不错,但距离远了点,楚云飞就算再加上生命能量这一项观察,也看不清人群里到底有没有马哈苏德这个人。

不过,他也没有就这么放弃对这些突围者的追杀,重武器他没来得及携带,不过手雷还是有的,摸出两颗手雷就丢了过去:草,老子也有手雷。

然后就是兜着屁股一顿乱枪扫射,对方出来四十个人,但能回去的,还没有一半。

马哈苏德虽然混在人群里,不显山不露水的,但他身边的人总是最多的。人多自然目标大,楚云飞的一颗手雷就正正地扔在这个人堆里,马哈苏德右腿被炸伤,身边的手下拼死把他架了回去。

楚云飞自然不知道他给自己的杀父仇人造成了这么大的伤害,他正在急速地分析对方这次行动的意义。

按楚云飞的估计,这两天的炮击,应该给对方造成了不小的伤害,但仔细想想,这伤害也未必就能大到哪里去。

要是营地里全是刘宁或者成树国这种经验和素质的士兵,能不能造成十分之一的伤亡率,都很值得商榷的,最多是房子没了,晚上不好睡觉就是了。

当然,他们不可能有这种素质,但是身负战友的信任,楚云飞一直在提醒自己,千万不要低估自己的对手,以免造成终生的遗憾。

所以,在楚云飞的估计中,营地里具备战斗力的,还应该有一百三十人左右,大大地高估了他们的素质。

这样考虑起来的话,这次四十人左右的排雷行动,可以视为是突围行动的预演,天亮以后,怕是就要见真章了。

念及这里,楚云飞不再犹豫,马上联系了战友,通知他们来北侧最高突出部(1号)集中,看来,这里将成为白天的主战场。

当然,最主要的出口,山谷南侧也要留人,那里是楚云飞的雷区,他的身手又好,所以他在那里流连策应。

所有的重火力都集中在了北侧,楚云飞那里,只留了一门2号突出部的迫击炮和一个火箭筒。

吩咐完这些,楚云飞隐隐觉得哪里似乎可以做点文章,但头又开始疼了,抓不住那一逝而过的灵感。

马哈苏德狼狈地回到营地,迎接他的,是阿孜克嘲笑的目光:看看,没我的人在你都舍不得用自己的人趟雷,我呸!

马哈苏德有心发火,但他的人又遭如此重创,此刻营地里早已主客易位,就算有那个心也没那个勇气,何况他的伤口也疼得厉害。

他喊了个懂得救护的手下,来包扎自己的伤口,那手下检查过伤口后,却嗫嚅地告诉他:情况很严重,如果不能及时治疗,怕是要截肢了,当然,再拖延下去,都未必是截肢这么简单了。

听到这消息的一瞬间,马哈苏德真的有点万念俱灰了,可悲呀,屋漏偏逢连阴雨,这种情况下,完好的自己都未必冲得出去,何况又拖着一条伤腿。

外有外患,再看看阿孜克那不怀好意的神情,内部还有内忧啊,马哈苏德沉吟半晌,严令那个手下:不得把我的伤势外泄,否则绝不宽恕。

可面对外面这索命的家伙,该怎么应对呢?马哈苏德不顾身体虚弱,坐在那里开始思索,考虑良久,居然还真的被他想出个办法,他马上喊来了桑丘。

等到刘宁他们的人和武器都各就各位,天色开始放亮了。

如今山谷里的人已经是釜底游鱼,大家也都不再掩饰行踪,反正没迫击炮这种弧线攻击的武器的话,1号地区是不怕人攻击的,大不了往后退退。

于是,他们居然在1号突出部上生火做饭了。

大冬天里,呃,是初春,冷东西是没人爱吃的,既然要警告对方北侧已经被控制,此路不通,那么,为什么不顺便吃顿热的?

当然也有“虚者实之,实者虚之,虚虚实实”的意思,反正能给他们添点堵就添点堵。

只有楚云飞最可怜,他要守在谷南策应,是没机会吃这热乎东西了。

可世间事往往是出人意料的,刘宁他们的热乎饭刚刚吃完,营地里就走出一个人,手里拎着个简易的铁皮喇叭。

考虑再三,刘宁还是压制住了开枪的欲望,要沟通么?那看看你想说点什么吧。

那人拿起喇叭,开始向1号突出部喊话,“我就是阿卜拉欣。马哈苏德,山上的朋友,不知道我们有什么化解不开的仇恨,惹得你们如此生气?大家都是真主的子民,为什么不好好谈谈呢?”

\section{第两百一十六章 起了内讧}

营地靠近山谷北侧,楚云飞所在的南侧听不清那个家伙在讲什么,不过,看到情况有变化,楚云飞还是跑了下来,想听听这家伙在白活什么。

他其实想的是,这个场景太诡异了点,是花招的可能性太大了,但是,多了解分析下对方的话总不是坏事。

听得懂的听不见,听得见的听不懂,辛汗那俩又不可能做主,于是,没人回答这个问题。

马哈苏德见没什么反应,又重复了几遍,楚云飞刚刚到达能听见的地方,辛汗实在忍不住了,大喊一声,“别说那么多废话了,你死定了。”

他的声音传不了多远,可山谷的回声不小,对方还是听到了这话。

马哈苏德愣了一下,马上又反应了过来,“那么,决斗吧,像个真正的伊斯兰男人一样,我们决斗吧,不要再杀害我们无辜的兄弟了。”

这话,辛汗可实在不敢回答了,天知道这几个中国人是怎么想的。

看到没人回答,马哈苏德又开始喊话了,“决斗你们都不敢么?你们还是不是男人?只会欺负弱小么?”

楚云飞仔细观察这人半天,发现他还真象马哈苏德,当然,没有凑近看,谁也不能说他就不是。

那男人还在那里唠叨不休,楚云飞已经听不下去了,管你是不是马哈苏德,不就是想找死么?成全你!

当那男人再次举起喇叭的时候,营地南侧两声枪响,那男人双手一撒,缓缓地倒下了。

营地里依旧是寂静一片,看来,马哈苏德的死,没引起任何的不良反应,他的人都被吓破了胆。

楚云飞可是不这么认为,这死的十有八九不会是马哈苏德本人,以他的影响力,找个替身出来应该不是很难吧?

再说了,马哈苏德好歹也是他们的领袖,这么死掉的话,多少也该有点群情激愤的样子吧?

想到这里,楚云飞忽然意识到一个很重要的问题,正如中国人在非洲很容易被人认混一样,马哈苏德对于中国人来说,辨认起来也应该是有难度的,虽然有照片可以做参考。

那么,为了不让元凶逃跑,说不得,只好把这里的人全部留下了。

想到这里,楚云飞猛然想起了天快亮时自己脑中一逝的灵光是什么了。

那就是,以前因为克普塞长老辛亚拉的警告,楚云飞一向对营地四周的陷阱非常警觉,根本就没产生过尝试进入营地的想法,但昨天的排雷部队可是清楚地给他指出了一条进入营地的安全通道。

既然这样,那就索性统统送他们上路好了,楚云飞想起车内还有一百多公斤的炸药,不信这么多炸药炸不死你们,不够的话,把炮弹也全绑上。

事实上,楚云飞买炸药的目的之一,也是想着有机会就给他们来个中心开花,现在好了,连路都有了。

想到这里,楚云飞赶紧通知战友,计划改变,白天保持向山谷里的持续攻击,一来不能让他们有集合队伍向外突围的机会,二来就是,拖到晚上,他们的末日就到了。

“我可以肯定,刚才那家伙不是马哈苏德。”

由于早上出了这么诡异的状况,计划有变动那也是正常的,现在刘宁和成树国对楚云飞的信任,都有些盲目了,所以自然不会反对。

可是,他们还没来得及执行任务,山谷内异变再起。

一支大约三十人左右的队伍,举着白旗,从房子里出来了,那是阿孜克的人。

马哈苏德已经要完蛋了,阿孜克明白地觉察到了这一点,那自己这点人马,实在是没必要跟他陪葬的。

当然,有人会拿贝西哈兰被屠村的事来告诫他们,你们放下武器投降也没用,那些人连妇女小孩都杀,活下去的唯一希望,就是是和我们一起抵抗到底。

可这话听在阿孜克耳朵里,怎么想都觉得他们是在欺骗自己,怎么,见到自己不行了,就一定要拉些垫背的么?所以,他对说这样话的人嗤之以鼻。

他和马哈苏德的恩怨,楚云飞他们并不清楚,所以这行动就显得非常诡异。

楚云飞的直觉就是,又是一个陷阱。

可刚才我已经表现了我的冷酷了啊,楚云飞有点不解,连要求决斗的人都直接枪杀了,他们凭什么还敢这样出来?

不过,楚云飞马上就反应了过来,这山谷里,不止是马哈苏德的人,还有支五十人左右的队伍是后期进来的,想来那就是援兵了。

再想想遭遇的人那种迥然不同的战斗风格,楚云飞马上就判断了出来:这山谷里,其实是有两股势力存在的。

那么,是他们在闹内讧么?

不管怎么说,阿孜克的人,留给楚云飞的印象还是非常深刻的,他曾经非常担心队友会遭到暗算。

那不管是不是陷阱,这支人马也非常彪悍的,目前还是不动为好,大不了,让刘宁他们严加注意,谨防他们使坏就好了。

他不知道,其实阿孜克的人,精华已经损失殆尽了。

不管怎么说,拖到晚上,一切都该结束了。

想到这里,楚云飞火速通知刘宁,那支举白旗的队伍,千万不要去动,但是,一定要严加防范。

至于营地里的其他人,有两个火力点不停地狂轰乱炸就可以了。

当1号突出部开始开火的时候,阿孜克的人还是紧张了一通,大家都听说这些魔鬼可能是屠过村的。不过,当他们发现火力并没有冲着他们来的时候,大部分人甚至把手里的武器都放到了地下,以表示绝对的善意。

整个一白天,山谷里都枪炮声不断,绝望的马哈苏德的人索性无视了死亡的威胁,直接开始攻打1号地区了。

不过,马哈苏德已死,没人指挥了,再加上有密密麻麻的地雷,这点残兵剩勇纷纷地倒在了血泊中。

突围无望,而对手凶残异常,居然有人在痛恨之下,向阿孜克的人展开了报复性的攻击。

阿孜克的人显然不会束手待毙,不等阿孜克吩咐,纷纷拿起武器还击。

一时间,山谷内一片大乱。

\section{第两百一十七章 惊天一炸}

看到他们双方居然打了起来,外面这五个人可是分外地纳闷了,在敌人眼皮底下内讧?太夸张了点吧?

刘宁他们为了考证这冲突的真实性,索性停止了射击,观察起了双方的战况。

当然,防范之心还是要存的,这么离谱的错误,刘宁不会犯,成树国也不会犯。

这一观察,真假立辨,就算做假,也不可能假到身体上血如泉涌吧?

那算,大家索性歇歇,让他们自己先打个痛快吧。

这一打起来,阿孜克才发现,其实,马哈苏德的人,还是很有战斗力的,况且,地形人家要熟悉很多,自己这方吃亏不少。

再说,人数上阿孜克的人也不占优势,不错,他们能活蹦乱跳的人是不少,比马哈苏德的人多点;可是,马哈苏德这里光伤员就有六七十号人,有一多半还拿得起枪呢。

于是,战斗慢慢地就呈胶着状态了,而且,胜利的天平在逐渐向马哈苏德的人倾斜。

阿孜克不愧是投机专家,分析出情况不对,马上大声喊了起来,谋求与对方妥协。

“桑丘,你们的敌人在外面,我们怎么说也是帮过你们的,是你们的朋友,大家还是不要打了。”

桑丘怒火上头,可是管不了那么多了,“我呸,你们也配做我们的朋友?关键时刻丢下我们不管,这样的朋友,不要也罢!”

阿孜克真的有点烦了,“我要不帮你们,还会断一只胳膊?我能帮的忙已经帮了,你们最好还是保留点实力,抵抗你们真正的敌人吧。”

桑丘办事不太爱用脑子,他对阿孜克的背叛是相当愤怒的,虽然这内讧不是他发起的,但他绝对是支持的。

阿孜克的话,说得很可怜,让他的心微微动了一下,不过他显然是属于典型的胆汁质气质类型的,“你去死吧,加上你们都打不过外面的魔鬼,要我们单独抵抗?亏你想得出来。”

双方在不停地喊叫,刘宁和成树国根本听不懂他们在说什么,辛汗他俩倒是能听懂,可他俩又不会说汉语,云飞不在现场,沟通实在是个大问题。

成树国心眼多,生怕对方达成什么协议又携起手来,索性寻了两个铁定是马哈苏德的人,机枪就是一通乱扫,他想表达的意思是,我们只打那些不投降的人。

他这么一打岔,桑丘的心思就又有点动摇了,本来桑丘是想在临死前好好出上口恶气的,惹不起外面的人,也要把阿孜克这个小人干掉。

意识到外面还有大敌等着,再想想早晨马哈苏德的吩咐,桑丘踯躅了半天,还是下令停止了射击。

已经到了山穷水尽的地步,还是给首领留点人气吧,这些,也是我桑丘最后能做的事了,毕竟,首领还是有东山再起的可能,虽然我看不到了,但是也不枉我忠心耿耿跟随他一场。

想到这里,桑丘的眼睛都有点红了,但愿,将来首领能为我们报仇吧。

就在这乱七八糟的形势转换中,天黑了下来。

营地里,一片愁云惨淡,马哈苏德的人在默默地生火做饭,也许,这就是人生最后一顿晚餐了,可是,他们现在还能做些别的事么?

阿孜克也非常不开心,他的人积聚在营地南侧一头,没人进屋子去,在白天的内讧里,他又损失了八个手下,重伤四人,现在这二十三人中,只有三个不带伤的。

他仰天长叹一声,唉,何必这样呢?又何苦这样呢?

楚云飞可没心思看这幅“哀鸿遍野”图,他正悄悄地溜近营地,试探昨天印象中的那条路呢。

路上本来还有些机关能够设置的,楚云飞就发现了一处,那根木头用上面的绳子一拽就能起来,再用那个小木棍一支,延伸个细棍到路面就行,绝对是很有效的陷阱。

可惜,凌晨突围的那一仗,打得马哈苏德的人信心全失,方寸大乱,退回去以后,该恢复的陷阱和机关都没有恢复,白白便宜了楚云飞。

路很快就探明了,楚云飞找了一个爆炸过后的弹坑,搬进去了所有的炸药和几乎全部的炮弹。

炮弹还剩得不少,隔着箱子,实在是太占地方了,效果也未必好,楚云飞小心翼翼地把炮弹挨个拿出来堆放进坑里。然后引了一根长长的导火索出来,足足有十米。

待到一切收拾完毕,已经是凌晨一点了,可楚云飞一点都不瞌睡,兴致勃勃地巡视着营地四周,严防有人玩花样或者逃跑。

这一夜,对山谷内所有人来说,都是非常漫长的。对楚云飞来说,更长,甚至比听到父亲的死讯到现在,所有的日子加起来都长。

刘宁他们早接到楚云飞的通知,知道他要在天刚亮的时候引爆炸药,为了安全起见,他们早就撤到安全的地方了。

天,终于蒙蒙亮了,监视了一夜的楚云飞终于长长地出口气,总算可以点火了。

二十分钟后,山谷内发出了一声惊天动地的响声,“轰~~~~~~~”

响声是如此的大,整个山谷似乎都摇晃了起来,拳大的石块居然能飞溅到七八十米开外,其中更有几块磨盘大小的石头能滚出去百十米远。

山谷中的树木,齐刷刷地下了一阵“冰凌雨”和“松针雨”,还有那枯朽的树枝直接地掉落下来的,半天方止。

刘宁他们离着爆炸点足足有四百多米远,还是被这巨大的冲击震得脸红心跳,耳鸣不止。

楚云飞距离爆炸点还要近些,大概就是三百五十米左右的样子,他藏在一块大石头后面,运起内气极力抵抗。

那卡车般大小的石头,居然似乎也微微晃动了几下,其实,那是地面传来的震动。

不过还好,山谷两侧,并没有滚下来太大的石头,总还算是有惊无险。

爆炸刚刚才止,硝烟甚至还没有散尽,楚云飞就迫不及待地冲进了营地。

他选择在白天引爆,就是要方便爆炸后的搜索,如此强力的爆破,就算震不死那马哈苏德,也足以震晕他们了,哪怕他是藏在地洞里。

马哈苏德,你可千万别这么容易就死了,这么死可太便宜你了!

\section{第两百一十八章 擒获马哈苏德}

楚云飞进入营地,果然不出所料,入眼的,是横七竖八的尸体。

正如解放战争时期发生的那些事一样,汽油桶发射的“飞雷”可以横扫方圆百米的一切生物。这威力比“飞雷”还大了好多倍的巨大爆炸夺去了营地里所有人的性命。

传说是真实的,大部分人,果然是被震死的。

他们身上,还有那种被震死人的特征,浑身不出血,但全身都软绵绵的,似乎所有的骨头都化成了碎屑。

大多数人,还保持着临死前的姿势,一副浑然不知道危险来临的样子,有的人脸上无奈的苦笑都还能隐约分辨出来。

一切,就这么结束了么?太便宜他们了吧?

楚云飞愣愣地呆了一会儿,发疯般地往回边跑边呼叫。

“你们快来,跟着我进营地,帮我找人!”

时机是要抓紧的,万一,仅仅是万一,有人没有被震死只是震晕的话,谁也不能断定他们会在什么时候清醒过来。

那样,难免就会给大家造成损失了。

跟着楚云飞,大家小心翼翼地走进了营地,现在这里,是死一般的寂静。

对于这次爆炸的威力,刘宁和成树国早有心理准备,这可是比“飞雷”还厉害的爆炸,不过他们还是被现场的狼籍吓了一跳:集中爆炸的威力实在太厉害了!

辛汗和他的同伴索性就愣在了那里,半天才回过味来,不让摸头的那位尝试着推了推一个死人,发觉触手软绵绵的,触电般地缩回了手,“怎么会这样?”

楚云飞喊一声,“辛汗,你俩去找有什么地洞没有,放心搜,就算地洞里有人,他们绝对也被震晕了。”

扭头又面向两个战友,“咱们三个搜地面,看看马哈苏德在哪里。”

于是,五个人开始了肆无忌惮的搜索。

营地真的不大,总共也就两千多平米的样子,半个小时就够大家搜两遍了。

但是,其中有相当一部分的死人由于生前受伤的缘故,早已经血肉模糊了,楚云飞三人不得不弄点雪擦去死人脸上的血迹,然后再详细辨认。

这么一来,时间耽搁得就多了,一个小时以后,辛汗他们都搜完了,楚云飞这里还有几个人没辨认完。

又过了几分钟,大家得出了最终的结论:没有发现马哈苏德!

辛汗他们倒是发现了三个地洞,不过,其中两个是用来藏匿大麻和一些药品、部分酒精饮料的,还有一个,似乎是个逃跑用的地道,不过不算长,刚刚二十米左右,从一间屋子通向屋后的一块石头后。

人跑了?楚云飞皱着眉头寻思了起来,他能跑到哪里呢?自己昨天晚上一夜没合眼,绝对可以确信,没人能从自己眼皮底下溜出去的。

成树国歪歪嘴,似乎想到了什么,“这里,会不会有能通向山谷外的地道?”

楚云飞摇摇头,“不可能,这都是石头山,他挖地道?那得多少年才能挖那么一条出来?”

这是实情,马哈苏德离开“基天”后才来的新纳山谷,不可能有这个时间挖地道。

再说,真有地道的话,他也不可能派人出去扫雷突围了,这倒不是说他绝对大公无私,不可能丢下大家跑了。实在是,他本来就是个玩人气的主,就算跑也得带着大家跑,要不以后都不用再混了。

楚云飞寻思半天,抬起头喊辛汗,“走,带我去看看那个地道。”

地道不算深,因为下面都是石头了,没法再挖下去,而且很窄小,堪堪能容一个人弯腰走过去,最窄的地方还需要蹲下身子侧着过,但到那石头以后,就是一面石壁了,于是结束。

看着那条地道,楚云飞又沉吟半晌,忽然想到了一种可能,“辛汗,你俩在这石头附近找找,看看还有什么地道没有,要小心,这里可能有陷阱。”

楚云飞想的是,这个地道既小且短,基本上没有存在的必要,那它的存在很可能还有别的什么用。会不会是,这根本就是个藏人的地方呢?

如果真是的话,那就一定还有别的地道来供躲避的人转换。

你搜这条地道,我跑那条去,你搜那条,我跑第三条去,实在不行,我还可以跑回你已经搜查过的地道。

果不其然,仔细一搜,这样的地道有三条之多,构建成了一个简单的三角形。其中一个五米长的地道最为隐秘,因为,应该没人能想到,就这么点距离,还会有条地道吧。

马哈苏德,就躲在那条只有五米长的地道内,同行的还有一人,两人已经被震得晕了过去,拖出来的时候,尚未清醒。

见到此人,楚云飞禁不住又是一阵的心神大乱,王八蛋,可算让我逮住你了。

他站在那里愣了半天,没人知道他在想什么,只看到他的脸色青一阵、红一阵的,没过多长时间,他额头上竟然汗如雨下。

成树国看了半天,叹口气,拎个脸盆一瘸一拐地出去,不一会儿,端了一大盆雪进来。

刘宁上前拍拍楚云飞,“想什么呢?弄死他走人吧。”

楚云飞终于清醒了过来,他揉揉太阳穴,“没事,我在想该怎么杀他呢,我现在终于理解多尼的心情了,妈的,可惜这里找不到水银。”

成树国不管那么多,直接一盆雪倒在了马哈苏德的头上,就这么等着他醒转。

可众人等了有半小时,马哈苏德依旧没有醒来,说实话,这次爆炸的威力实在太大了。

成树国不耐烦了,拔出了弯刀,刚要扎下去,瞟一眼楚云飞,手腕一转,用刀背重重敲在对方小腿骨上。

一下、两下、三下……,敲了半天,马哈苏德终于醒转,倒也说不清是疼的还是冻的。

刚清醒的他还是有点迷糊,过了半天才明白过来,自己已经是阶下囚了。

抬眼看看,面前,是两张非常陌生的脸,陌生到在俾鲁弯都见不到的那种脸,怕是在巴基斯坦也见不到几张这样的脸。

“你们,是中国人?”

楚云飞冷着脸,没吭声,其他人也没说话。

\section{第两百一十九章 元凶伏诛}

在瞬间,马哈苏德的表情变得异常地慌乱,大声地喊了起来,“为什么,为什么,我怎么听不到我的声音,我听不到声音,我听不到声音!”

凄厉的喊声,回荡在寂静的新纳山谷上空,久久方才散去。

原来,这家伙的耳朵被震聋了,起码是暂时性地听不见了。

其他人依旧没有说话,只是冷冷地看着他。

马哈苏德愣了愣,沉吟一下,终于明白了他遇到了什么样的事情,“你们就是那群人,中国政府派来的?”

楚云飞舔舔嘴唇,眯着眼睛,表情很平静,却又给人一种说不出的狰狞的感觉,“我是那个中国人质的儿子。”

下一刻,楚云飞想起了马哈苏德听不见声音,捡起根树枝在地上写了句话,“我是中国人质的儿子。”

马哈苏德低头看看地上的字,又抬头看看楚云飞,表情异常地丰富。

他先是愣了一下,震惊之后就是面如死灰,脸上的肌肉也不停地抽搐起来,抖动幅度之大,怕是都能用来发电了。

可是到了最后,他不知道想到了什么,脸色居然平静了下来,而且,马上又浮现出一种忍俊不禁的表情。

楚云飞依旧没什么表情似地站在那里,没有任何的反应,不过,仔细看的话,能看出来,他全身都在微微地颤抖。

不过,马哈苏德的表情最终还是刺激了他,那,是一种似曾相识的眼神。

对的,就是那个辛巴,被楚云飞亲手砍了手脚的那个,“基天”组织地位里比较高的那个,也曾经是这样的表情。

这样的思索,终于使楚云飞的情绪变得正常了起来,他在嘲笑我,他为什么要嘲笑我?

不过,这个想法在楚云飞脑中只是一闪而过,眼前这个人,只有死,没有别的选择,那么,死人的想法,活人是没必要去计较的。

他拿着那根树枝,在雪地上又写了行字,“你希望怎么死?”

当然,楚云飞绝对没有按对方的要求行刑的心思,他只想知道,马哈苏德最不希望怎么死。

看到地上这行字,马哈苏德的冷笑停在了嘴边,愣了一下,接着却疯狂地大笑起来。

“哈,哈哈,哈哈哈哈……”

楚云飞开始倒还无所谓,眼见对方越笑越疯狂,越笑越痴颠,忍不住又焦躁起来,拔出弯刀就砍下了对方的左手小指。

他可不舍得多砍,还有很多种滋味等着对方品尝呢。

这阵剧痛,使得马哈苏德清醒了许多,但那笑的尾音还是持续了十几秒才结束。

看看自己血流如注的左手,马哈苏德却没有为自己一点止血的意思,而是对着自己的左手,若有所思地说了一句,“你的父亲,是被沙特政府军……打死的。”

这句话,实在是让楚云飞有点始料不及,他下意识地问了一句,“什么,沙特政府军?”

不过,这话出口,他在反应过来对方听不见的同时,也明白了这事的可能性,沙特政府军解救人质,子弹横飞,父亲被误伤,那也是可能的。

但是纵然如此,始作俑者还是面前这个马哈苏德,这是毫无争议的,讨论是谁开的枪,实在是没有太大的意思。

楚云飞弯下腰,用弯刀在那句话下面又划了两横,意思很明显,依旧是问马哈苏德想怎么死。

到了这一步,马哈苏德也不再心存侥幸,而是很冷静地说了一句话,“我很欣赏你,为了你的执着,而且在你面前,我没有不死的理由。”

楚云飞依旧没有说话,而是冷冷地斜眼看着他。

马哈苏德料定了楚云飞会给他说话的机会,所以悠悠地提出了自己的条件,“我告诉你这件事真正的幕后凶手,你让我完整地死好不好?”

幕后凶手?楚云飞本来已经料定对方要玩点花样,本想借此来好好戏弄对方一下来泄愤的,实在没想到居然等来这么个消息!

难道说,这事里面还有蹊跷?

别是,千万别是二十一冶的内斗吧?楚云飞见多了内讧,对国人有点超级地缺乏信心,实在有点怕听下去了。不过,他怎么可能不听下去?

马哈苏德见对方没有什么反应,不再盯着自己的手看,而是抬头瞟了楚云飞一眼。

楚云飞微微扬了下下颌,示意对方说下去。

马哈苏德也没有计较楚云飞的不做回答,接着缓缓地说道,“是美国人,你父亲的公司,在沙特,同美国人的公司抢活,他们想把中国人赶出那个项目。”说到这里,他血乎乎的脸上竟然露出一丝愧色。

楚云飞这下可就全明白了,为什么美国人会放了马哈苏德;为什么马哈苏德的同伴总是出事;为什么马哈苏德会脱离“基天”。

这一切,都出在面前这个叛徒身上!

马哈苏德没再说话,再说什么都是多余的了,对方很聪明,保持缄默,是他完整死去的唯一希望。

楚云飞沉默半晌,长叹口气,手中弯刀缓缓地举了起来。

他已经不想再追问什么了,也不想知道那个美国公司的名字,仇恨,就到这里为止吧。

资本,注定是要“从头到脚流着血和肮脏的东西”的,这是没人能够挑战的法则。

现在,他对马哈苏德也没有了太大的恨意,对方,其实也是很可怜的,背叛了自己的信仰,成为了屠杀同胞的叛徒。

这些对马哈苏德而言,也是很悲哀的事吧?

有那么一只无情的手,在戏弄着所有的人。

大家都是可怜的,区别只在于:谁会比谁更可怜!

愿赌服输,多少也还算条汉子,那就如你所愿,成全你了你吧,楚云飞打算一刀割断对方的喉咙。

成树国一瘸一拐地走了过来,“云飞,山外面来人了,人倒不多,似乎有三、四个。”

看到成树国的伤腿,楚云飞怒气又起,好吧,完整地死是吧?那我抽干你的生命能量好了。

想到就做,楚云飞站在那里开始强行吸收马哈苏德的生命能量。

开始的时候,很容易,但吸到十分之一左右的时候,生命能量的吸收变得困难起来,一直在那里沉默不语的马哈苏德大声地叫嚷了起来,“你……你在做什么?啊哟~~”

然后,马哈苏德开始疼得满地打滚。

楚云飞从没做过这样的试验,也不知道强行吸收活人的健康能量会这么困难,不过,困难就困难吧,这样吸收下去,也就是十来分钟的事。

既然还能造成马哈苏德的异常痛苦,那实在就是非常令人愉快的事了。

马哈苏德可真的后悔了,他完全可以确定,眼前这个家伙一定不属于人类。

他觉得自己浑身酥麻酸软,而且异常地疼痛,那不是说某个部位如此,而是全身都是这样,身体内外所有的痛觉、触觉等神经全部发作了起来,天下的惨刑,实在是莫过于此了。跟这个味道相比,关塔那摩里的酷刑简直就是天下最大的享受了。

站在那里不动就给自己造成了这么大的痛苦,只有魔鬼,才能做到这一步。

其实他没有发现,身上的味觉、嗅觉等神经也高度地敏锐起来,因为,他实在是太疼了。

一分钟后,马哈苏德就晕了过去,这酷刑已经超过了他精神所能支撑的范围太多了。

马哈苏德晕过去以后,能量就更好吸收了,楚云飞基本上花了十分钟,吸收完了他所有的能量,马哈苏德的身体不再有那种明黄的辉晕。

刚刚停止了吸收,楚云飞的脑袋就疯狂地疼了起来,疼得他不由得蹲在地上直吸冷气。

辛汗发现他似乎出了点意外,走上前来,好意问到,“怎么样,你很难受?”

楚云飞咬牙摆摆手,“没事,脑袋疼。”说着强行挪到马哈苏德的身边,探手出去测试对方鼻息。

操,居然还没死,还有气,楚云飞不过是例行公事地检查了一下,却没想到对方还果真没死,注意力一转移,连脑袋疼都不放在心上了。

闭眼考虑了一下,楚云飞想起了英国那个神秘的木框,似乎也是这样,生命能量不够的话,体表就不能发出那种生命的气息。

脑袋太疼了,不能多想,楚云飞忍着痛苦拔出弯刀,割断了马哈苏德的喉咙。

果不其然,马哈苏德体内又释放出一团大的很粘稠的生命能量,刚要破空直上,却被楚云飞恶狠狠地吸收了进来。

登时,楚云飞的脑袋里就多出了好多东西,很多乱七八糟的感觉,非常破碎而不完整,就像索度墓地里的那样。

剧痛的脑袋实在不能再忍受如此的折磨了,楚云飞实在控制不住了,勉强捡起了那根小指,然后就晕了过去。

山谷外来的,是“基天”的人,刘宁埋伏在山谷上看到了,火速通知了成树国,他是用望远镜看的,所以,“基天”的人可没想到已经被山谷内的人发现了。

他们派了几个人,想悄悄地摸进来查探一下,马哈苏德是否受到了袭击,却正正地闯进了地雷阵,当然,这已经是晚上的事了。

三天后,基天的人才最终进入了山谷,却发现这里死气沉沉,有人发现了马哈苏德的尸体,天气尚冷,还没有发臭。

最后,有人在地上发现了一些文字,有一句话依稀不可辨认了,只能看清楚几个字:“……中国人质……儿子。”

再没有什么明确的信息,能说明这里到底发生了什么。

\section{第两百二十章 离开巴基斯坦}

楚云飞是被一股浓郁的香味唤醒的,这个味道,似乎,成树国又在烤羊肉了?

等他睁开眼睛的时候,才发现成树国和刘宁都坐在他眼前一眨不眨地盯着他。

辛汗正在架子上翻腾着羊肉,另一个穆斯林在烤囊。

看到楚云飞醒转,成树国高兴地叫了起来,“看看,我说什么来的,这家伙肯定又是‘胃亏肉’了,一烧烤他就醒过来了,嘴忒馋了点。”

楚云飞的脑袋还有点痛,听到这话,才依稀记起,自己似乎是头疼得受不了才晕倒的,这个,烤肉这么管用么?

管用就多闻闻,楚云飞欠起身子,大力地吸了几口那烤肉的香味。

果真神奇得很,就简单地嗅了这么几下,脑袋里那种抽筋似的疼痛就不翼而飞了,不过,似乎比上次效果差一点点。

“奇怪,我是很喜欢吃肉,但从来没馋到这份上啊。”

他的话,引起笑声一片,因为他用两种语言各说了一遍。

然后,楚云飞才知道,自己昏迷了足足一天一夜,大家都知道他很累了,这么多天从没睡过个囫囵觉,没人忍心吵醒他。

再扭头一看,楚云飞居然意外地看见个人,那个跟马哈苏德一起被震晕的家伙,全身五花大绑,被扔在一棵树下。

“咦,这家伙怎么还活着,你们没有杀了他?”这话是用汉语问的。

刘宁皱着眉头回答他,“还不是树国弄的?他非要说让你亲手杀,操,反正他也不用背人。”

那是,成树国腿不好,他自己不让别人背就不错了。

原来,虽然那天刘宁发现了几个人,但不知道对方来路,后面还有没有接应。楚云飞又大仇得报,就懒得跟他们计较了。

嗯,没错,搁给别人,总要看看自己的战友是如何处置仇人的,但是,小队长三人组不会犯这样的错误,总是有人自觉去警戒的。

他们的配合,接近于完美了。

刘宁背着楚云飞,怕摸头那位倒背着这个俘虏,几个人挑着小路走了。

吉普车都空了,本来两辆吉普车就够用了,不过辛汗说别人不要他要,坚持开了三辆车往回走,结果走到半路坏了一辆,现在被拖着走呢。

楚云飞想想倒也是,后来又问明白了,为了怕麻烦,刘宁把所有重武器都丢在了那里,只有辛汗不辞劳苦地抗了门82无回来,车里就再没有重武器了。

没了没了吧,反正这里的东西,最终都要丢弃的,想到这里,楚云飞冲辛汗笑笑,“没关系,三辆车都给了你也没问题,不过,你得跟我们去伊斯兰堡自己开回来。”

辛汗的心全在烤肉上呢,听到这话只是点了点头,过了那么一两分钟,他明白了过来楚云飞在说什么,手一松,羊肉直接掉进了火里。

看着那个俘虏,楚云飞有点头疼:你们早解决了就算了,居然还背他到这里?

可这样的话绝对说不出口,大家是为他好,他绝对不能那么不识抬举乱说话。

紧接着,楚云飞就反应了过来,这不是自己琢磨生命能量的最好机会么?用别人试验不合适,用他么,绝对不会有什么负罪感的。

呃,就算有也只有一点点。

现在,该做点什么试验呢?楚云飞又皱着眉头考虑了起来。

这倒不能说楚云飞残忍,事实上,任何人身上有了些常识不能解释的东西的话,想做点探索都是人之常情,毕竟是跟自己切身利益有段的。

然后,此人就成为了楚云飞的玩具,直到快到克努蒂部落的时候,玩具才被销毁。

在这期间,通过孜孜不倦的试验,楚云飞总结出了几点:一就是,生命能量似乎也有分别,有些,是能承载一些记忆的片段的,不过,那是最顽固的生命能量,或者称其为生命烙印也是可以的,就是马哈苏德身体最后飞出的那团东西。

而且,这样的生命能量似乎能控制相当一部分游离的生命能量。

二来呢,楚云飞很痛苦地意识到,每当自己想做恶人,掠夺别人生命能量的时候,尤其是那种生命烙印的时候,脑袋还是会疼的,百试百灵。

还有一点就是,楚云飞发现,自己的烧烤水平并不比成树国差劲。呃,这话,其实应该这么理解:吃起来的话,可能,最终的味道上是有较大、或者说很大差异,但是在治疗头疼的问题上,效果是一样显著的。

此后,一路有事无话,事也是小事,不值得提。

虽然楚云飞他们在路上略有耽搁,但俾鲁弯这地方通讯确实是太落后了一点,贾德坦长老终于吃惊地从楚云飞嘴里知道了马哈苏德的覆灭。

从族人嘴里证实这个消息之后,贾德坦长老长叹一声,“唉,那辛亚拉也活不长了。”

辛亚拉正是国家安全局介绍的那个长老,在楚云飞复仇过程中帮了大忙的,听了这话,楚云飞连忙请教:此言何出?

要知道,中国人,是非常讲究香火情的,纵然,楚云飞已经知道,太多的国人已经不讲这个,不说节操了,但是,他是要讲的。

别人是别人,我是我!

贾德坦并不知道辛亚拉和楚云飞的关系,辛汗虽然知道,但楚云飞严禁他说,冲着他对楚云飞的恐惧,他也只有在楚云飞离去以后才敢说。

贾德坦倒没做任何掩饰,直接解释了原因,其实他也知道辛亚拉和中国人有来往,只是没想到楚云飞能搭上这条线而已。

有这条线的话,楚云飞根本没必要跑到克努蒂部落来找人帮忙!

“因为,辛亚拉阻碍了太多人的路,他是那么地正直无私,只有他,才能或者说才敢正面拒绝马哈苏德,其他人,只能用巴基斯坦政府做借口来抵挡。”

“可他得罪的人太多了,支持他的人,是平民百姓,反对他的人,全是非法武装和极端势力。”

“马哈苏德不存在了,他也没有存在的必要了,皮之不存,毛将焉附?”贾德坦长老用这句中国古话,结束了他的发言。

事实上,果真如此,在楚云飞离开巴基斯坦之后,所有与他接触过的人都在两年内死亡。

不只是辛亚拉,贾德坦和辛汗也死了,没人知道他们死于什么原因,只是,有人隐约记得,某次节日聚会,有人依旧笑话辛汗,说他连弟弟都保不住。

那天,辛汗反驳了很多话,不知道是他突然有钱的原因还是因为别的,说得那人无地自容。

后来,辛汗就死了,没过几天贾德坦长老也死了,俾鲁弯果然是个非常诡异的地方。

这就是后话了。

带着这样的收获,大仇得报的楚云飞及两个战友离开了巴基斯坦。

\section{第两百二十一章 再见索菲娅}

巨大的“空客A330”带着巨大的轰鸣声,慢慢地停在了机场跑道上。

飞机上的旅客秩序井然地顺着舷梯走了下来,其中,就有三个风尘仆仆异常疲惫的黄种人。

离开巴基斯坦后,楚云飞他们并没有直飞法国,而是打算来英国做一番休整,其中,成树国和楚云飞都是需要好好休息一下的。

班克斯又去美国了,考林斯倒是回来了,布兰克也回来了,打听明白这些动向,楚云飞干脆地拒绝了多尼的邀请,自己找了个地方住下。

无视成树国的抗议,楚云飞和刘宁坚持喊了医生来,他们的目标就是:在保证治疗效果的前提下,尽可能快地让成树国康复。

为了让成树国安心养病,刘宁以老大(按年龄说)的名义,勒令楚云飞陪护并监督病人,事实上,楚云飞也是很需要认真休养几天的。

然后,刘宁就出去会自己的老乡去了,当然,他使用的幌子是,“我需要跟狂龙打听一下最近大使馆的动向。”

当然,放在狂龙那里的衣服也要弄几套来穿穿。

然而,他们显然过低地估计了国家机器的能力,就在楚云飞还在向周琳琳讲述复仇经过的时候,刘宁的电话响了。

刘宁正在被狂龙热情款待中,喧嚣的音乐声让他没有注意到自己的手机,于是在下一刻,成树国的手机响了。

楚云飞本来也就讲得有点没劲了,因为他发现,琳琳,似乎对他报仇的过程不是很感兴趣。

毕竟是女孩子,楚云飞这么认为,但他是人又不是神,逮着个能吐露的对象,也实在是不容易,还在自顾自地讲述着。

成树国被医生注射了一种针剂,据说是消炎之余能促进伤口愈合的,不过针剂有点副作用,受者容易瞌睡,成树国现在就睡得很死。

为了不影响伤员的休息,楚云飞迅速接起了电话,同时又仓促地同琳琳道别。

电话,是彭辉煌打来的,他先恭祝了楚云飞的大仇得报。

俾鲁弯省是闭塞了一点,不过,像马哈苏德伏诛这样的重要消息,却是绝对不可能不被传出来的,他实在是太有名了点。

跟他的名气相比,贝西哈兰那样的惨事都是不值得一提的。

当然,安全局得到的消息难免还是慢了一点,毕竟那里的通讯不太发达。

楚云飞自然知道老彭的用意,他是在催自己赶紧接活呢,可是,再急也要等树国伤好了,不是么?

于是,楚云飞很坦白地告诉彭辉煌,成树国受伤了,他需要等战友养好伤再执行行动。

其实老彭还是有别的用意的,他听说了辛亚拉转述来的消息,知道了那山谷中二百多名恐怖分子和五十多名悍匪被楚云飞他们扫荡一空,没有留下一个活口。

老天,这是多么令人恐怖的实力!老彭在感叹之余,自然也想知道楚云飞他们遭受了多大的损失。

杀人一万,自损八千是正常的,所以,老彭听了楚云飞的话不怒反喜,什么,只有成树国受了点轻伤?哦,那好那好,没关系,我们再等等好了。

不过,当彭辉煌听到楚云飞他们飞到了英国,还是表示了不理解,他总觉得,来法国的话,大家相互之间比较好照应,当然,也有比较宽裕的时间来相互了解。

楚云飞心说,正是想避开你们这些人呢,大家无非是简单的等价交换关系,你出赦免令,我们提供相应的服务就是了。

从被宣布“叛国”的那一刻起,楚云飞就对那些政治家充满了根深蒂固的偏见,那些实在是一群翻脸比翻书还快的人,如无必要,远离为佳。

挂了电话,楚云飞却意外地发现了,成树国放手机的口袋里,居然有一节人的小指头,已经被盐腌得严重脱水,缩成了一节类似蜡笔似的小枯枝,还好,天气尚冷,没有什么异味传出。

楚云飞略微一思索,就明白了,这怕是自己从马哈苏德手上砍下的那节小指头吧,成树国一定是想把它拿回国去,当做战利品来向别人炫耀。

嗯,正愁扫墓的时候没东西祭奠老爷子呢,这东西,自然是要没收的,我都答应马哈苏德留人家全尸了,你丫居然敢私藏战利品?

当然,楚云飞知道成树国听不懂乌尔都语,不过,反正这东西他是要定了。

楚云飞刚收起那节手指,马上就明白了一件事情,事实上,英国也不是什么太平地方,能让他头大的,似乎不只是政治家。

索菲娅来了。

几个月不见,索菲娅出落得越发美丽了起来,似乎,又成熟了一些。

笑容变得雍容了一些,身材也变得更加丰满了一点,不再像初见楚云飞时那样还带着股青涩的味道。

也许,是在巴基斯坦见多了裹在袍子里的蒙面女人的缘故吧,楚云飞揉了揉太阳穴,我的脑袋,怎么又开始疼了?

事实上,美女给人的杀伤力是相当大的,比较而言,楚云飞虽然更头疼露丝一些,但索菲娅却经常给他一种束手无策的感觉。

因为索菲娅实在是太漂亮了,如果不得已冒犯一下的话,光是别人谴责的眼神都会让楚云飞分外地受不了。

可现在,他还没有办法对索菲娅置之不理,那样不但非常不礼貌,还因为:她听说中国客人受伤了,做为朋友,她觉得有必要来探视一下。

索菲娅带来了不少的水果,还有一些鲜美的海产品。

弄得楚云飞一时间头大无比,成树国什么时候吸引了索菲娅的芳心呢?

但是,就算关心也不是这么个关心法吧?楚云飞虽然医护能力不如成树国,但他也知道,海产品是发的,容易引起伤口红肿溃疡,延迟伤口的愈合速度,有外伤的人最好少吃。

可是,对于这一切,他还什么都不能说。

其实,索菲娅来这里还是有别的事的,她告诉楚云飞:她的爷爷宾塞斯。维伦斯想见见他们,本来是要派考林斯来邀请他们的,可美国临时有事,她的叔叔和姑父都赶过去了。

宾塞斯的消息很灵通,也许跟维伦斯家族在中东的势力有关,对楚云飞的复仇行动,他居然知道得比中国国家安全局的人还多。

报仇的结果也非常出乎老头的意外,他真没想到这区区三人就能在俾鲁弯掀起那么大的风浪,不但手刃了仇人马哈苏德,而且把整整二百多人的武装分子全部解决。

维伦斯家族的仇敌“基天”也被杀得血流成河,叫苦不迭。

获得这样的结果,这三人居然只付出了一个人轻伤的代价!!!

这是多么可怕的实力?想想对方屠村时那种残忍的手段,自己莽撞的孙子当初居然敢跟对方叫板。

一想到这里,宾塞斯就有种买彩票中奖的幸运感,自己家的命运,实在是……太好了点。

于是,套近乎就成了一种必须有的礼节,何况,人家也是救过自己命的。

只是实在不巧,班克斯回不来,只能由这老人出马接待了。

\section{第两百二十二章 绝代有佳人}

父仇得报,回国有望,楚云飞的心情难得地轻松了起来,很高兴地答应了索菲娅的邀请,而且,言语间还非常地亲切。

“苏菲,你看,变化受伤了,需要我在这里看护,而新这家伙出去找他的同乡了,再说,我们三个的衣服还没去取呢,穿成这样去你家,似乎不太礼貌吧?”

索菲娅本来已经稍微平静的心思,又被这一声“苏菲”喊得方寸大乱,选择先生,你知道不知道,女孩的昵称什么人才能称呼?

那个男人,他不知道,他总以为,昵称总是代表一种友好和亲近的态度。

“哦,那你这里还需要什么帮忙么?要不,我帮你喊几个护士来吧?”

现在的楚云飞防人之心已经降到很低了,他认真地考虑一下这个建议。嗯,树国受伤,要是自己能找两个漂亮的女护士来,估计他会很高兴的吧,毕竟他整天想着发扬“中华雄风”于异域呢。

可是,树国……他能确定自己不会“丢脸丢到国外”么?楚云飞对这点有一定的怀疑,再说,丫受伤部位是在大腿上,不但不合适做剧烈运动,而且,这伤势要是影响到丫的临场发挥,自己的罪过可就大了。

想来想去,楚云飞还是拒绝了索菲娅的建议,用的还是很冠冕堂皇的理由,“算了,他毕竟是为了帮助我才受的伤,要是让别人照顾他,我心里过意不去。”

既然你能用昵称称呼我,那我自然也能稍微地不讲理一点,抱着这么个逻辑,索菲娅开口了。

“那好吧,明天,我和爷爷在家等你,这里就交给新先生照顾吧,你们关系那么好。”

出乎意料的是,这个“性情古怪”的家伙,居然没有介意这样的安排,他略微思索了下,点点头,脸上还露出了罕见的微笑,“也好,苏菲,请代我向你的爷爷道歉,今天实在是过不去了。”

索菲娅还想说点什么,不过,她忽然间发现,两人之间实在是没有什么共同的语言,意识到这一点的她,又有了点沮丧。

“要我帮你打电话把露丝喊来么?既然你们还要在伦敦呆上几天。”

楚云飞脸上的笑容在瞬间停顿了一下,然后就换了口气,有点无奈地请求着,“索菲娅,咱们不提她好不好?”

索菲娅调皮地一皱鼻子,做了个鬼脸,“不提也可以,不过你要老实告诉我,你对她到底做了什么?”

楚云飞眉头微微地皱了一下,不过不知道为什么,他倒也没怎么生气,而是一副若有所思的样子,“事实上,我知道她很喜欢我,不过,我和她注定是不可能的,我俩之间的差距,怕是比马里亚纳海沟还要深一些。我还能做什么?”

索菲娅其实本来就是在没话找话,哪怕楚云飞真求她代为联系露丝,十有八九她也就是把露丝的电话号码告诉楚云飞而已。

可楚云飞要真有这心,怕是电话号码早就弄到手了。

楚云飞的回答还是引起了索菲娅的兴趣,她浓密而修长的眉毛一扬,“哦?为什么你这么说呢?嫌她不够漂亮么?”

“嘿,”楚云飞苦笑一声,“让我想想该怎么说,其实,其实……我已经有女朋友了,不过,我想,就算我没有女朋友,我和她也不可能在一起,因为……这个词我还真不太好解释,我想,那纯粹是一种感觉吧。”

楚云飞想说的是“缘分”,显然,这时候他的英语表达水平就不是很够了。

按着感觉说话?索菲娅认真地想了想,还是不太能理解这种想法,“那么,你是说人种的差异么?白种人和黄种人?”

天色尚早,看到索菲娅一时不愿意离开,楚云飞搬了把椅子过来,“苏菲,坐吧,不是你说的那么回事,我个人,嗯,其实很喜欢《罗马假日》里的AudreyHepburn,所以我认为,人种因素……是不能影响我个人观感的。”

巧了,索菲娅自从十三岁开始就被人叫做“小赫本”,一直叫到现在,听到这样的话,实在是别有一番滋味在心头。

楚云飞也因为这句话,陷入了对自己少年时代回忆,他已经有六七年没有这样的闲暇和勇气了。

“绝代有佳人,遗世而独立。”当时年少轻狂的自己,何尝有没有点对未来的憧憬和幻想。只是,被无情的现实击得粉碎就是了。

摇头笑笑,楚云飞收拾起心情,“哎,苏菲,这么一说,我才发现,你跟AudreyHepburn长得有些像啊。”

这话,说得就有点轻薄的味道了,不过,楚云飞没朝这方面想,倒也理直气壮,毕竟这也算种奉承的,是吧?

索菲娅轻轻一笑,楚云飞这话夸得她很舒服,于是又不自觉地模仿起了赫本的那种典雅姿态,反正小时候她是常模仿的,“呵呵,你这么夸我,我会很不好意思的,对了,这次事完了以后,你们打算去哪里呢?”

“去哪里?”楚云飞抿了抿嘴,“我想,大概是要回国了,过段平静的生活吧。”

回国?索菲娅娥眉轻蹙,“你不是说,要打打杀杀很长时间的么?还说累的时候会来我家呢。”

嗯嗯,此一时彼一时嘛,楚云飞又刮刮自己的鼻子,“呵呵,是这样,这是我的一种猜测,我想国家很快会命令我们回去了,当然,我有选择的权力,不过,我也真的想回去看看了,毕竟平静的生活,是人人都向往的。”

听着楚云飞毫无留恋地讲着这些话,索菲娅的兴致一下子降低了不少,不过,眼前这家伙,居然也是喜欢过平静生活的人么?难道说,那种冷血和张扬,不是天生的么?

“你也喜欢过平静生活?看来,你和我爷爷真的很像啊。”

和你爷爷很像?“那个,我觉得,我看起来没有那么老吧?”

楚云飞想说的是,其实,我怕是没你爷爷那种雄心壮志,踩着别人的尸体,建立自己的冷血帝国,说别人也就算了,要说宾塞斯是个喜欢平静生活的人,怕是谁也不会相信的吧?

不过,几年以后,楚云飞才知道,这样的想法未免武断了,他又不知道当时的宾塞斯面临的是什么样的局面,家大业大者做事,有时候是很无奈的。

这样的人,甚至都未必是在为着自己而活着。

\section{第两百二十三章 人老成精}

一个下午,就在索菲娅和楚云飞的聊天中匆匆过去了,由于某人有意无意地巴结,美女的心情很是不错。

索菲娅甚至发现了一个被忽略的事实:眼前的这个男人,其实也是有感情有童心的,甚至还拥有不少的幽默细胞,再不是以前那种冷冰冰,又臭又硬的样子。

这种状态并没有结束,甚至在第二天见到宾塞斯时,楚云飞依旧给了老人这种感觉。

不过,人老成精这话确实是太对了,宾塞斯很快就发现了些不妥,“楚,似乎报仇之后的你,改变了很多啊。”

楚云飞又笑笑,自己这里是双喜临门呢,当然,他也是存了些别的想法,才着意地释放善意,“说起来,我能报了自己的仇,还多亏了班克斯,要不是你们帮忙,我恐怕现在还在沙特的沙漠里转悠呢。”

宾塞斯很高兴楚云飞这样的回答,面对强者的奉承,开心地笑了起来,“呵呵,也不能这么说,毕竟,我们也出了口恶气,像这种能让大家都高兴的事,我们还是很乐意去做的。”

说到这里,楚云飞就想起了酬谢的问题,“对了,维伦斯先生,法国和巴基斯坦这两件事,我是受了你们不少的帮助,不知道你们想得到什么样的回报?”

宾塞斯又笑了起来,这次就带了丝丝狡诈的味道在里面,“说实话,你要不提的话,我们还真没想要些什么,毕竟,这两件事对我们来说,实在是很容易做的,不过,你既然执意要回报的话,那我们不接受也算不够礼貌,这可是你爱说的话。”

楚云飞翻翻眼皮,刮刮鼻子,一副无奈的样子,“维伦斯先生,听你这么一说,似乎,呃,以往我做人,是不是不太成功?”

楚云飞吃瘪的样子让宾塞斯非常地开心,他又笑了起来,“回报么,我还真想不出什么东西,嗯,要不这样,听索菲娅说,你要回国过平淡日子去了?”

楚云飞点点头。

“那,这样吧,我还是以前的意思,你能不能把过平淡日子的地方改一改,来我们家这里过?哦,对了,如果你放不下家里的情人,把她也接过来好了。”

宾塞斯笑得就像见了拨浪鼓的婴儿一般,异常地灿烂。

楚云飞又刮刮鼻子,撇撇嘴,越发地无奈起来,“这个,这个事情操作起来,难度很大,实在是,不方便答应你,还请原谅。”

老头终于慢慢地止住了笑声,“好了,小家伙,你不要再兜圈子了,这次又要麻烦我帮你做什么事了?”

楚云飞扬起眉毛,盯着宾塞斯看了半天,终于明白了过来:这世界上,聪明人并不是只有自己一个人。

何况,老头年纪大他这么多,论经验和阅历他怕是拍马也赶不上的。

想明白了这点,楚云飞的神情郑重了很多,不过他的手指依旧停留在鼻子上,“这次,怎么说呢,我要办的事,是要绝对保密的,还请维伦斯先生先答应我不要声张。”

宾塞斯想了想,觉得并不是什么困难的事情,“起码你先跟我说说吧,我要觉得不妥当的话,是会拒绝你的,呃,苏菲,你先出去一下。”

以维伦斯家族老大的身份,这话自然不是随便说说的,也就是说,老头哪怕拒绝,也不会泄露这次谈话,他不可能拿整个家族去博那强者的报复的。

事实上,宾塞斯这次纯粹是好奇心理在作怪,眼前这个强悍的家伙,又要去做什么惊天动地的大事了呢?

没错,楚云飞想的就是他即将办的这档子事。

在这几天里,楚云飞认真地把这事捋了一遍,才发现,这个活,真的不是很轻松。

这次去法国,首先可以确定的就是,多尼不在场,他们哥三个没人懂法语,沟通是个大问题。

当然,国家安全局的人懂法语的绝对不会少,但是,楚云飞三人从内心深处非常排斥这种机构,再说,大家并在一起,合作默契不默契是一方面,行动时谁做主才更重要。

国家安全局的人一定要做主的话,他们三个有拒绝的权力么?没有!

可是,他们配领导这三个出生入死而履险无夷的年轻人么?不客气地说,他们还真不配,这三个人组成的战斗小组,怕是不输于这世界上任何的战斗小组。

而且,谁能保证到时候又不会出现那种刚卡版的背叛呢?政府机构,总不能让人太放心,最恶心人的就是,被出卖的还总兼职反派,背那恶名声。

所以,不如八仙过海,各显神通的好。

另一点就是,楚云飞虽然比较嚣张,但他还没有自大到狂妄的地步,拜托,国家安全局大批设备在手,都不敢进那工厂,你凭什么就能相信自己进得去,那不是没事找抽么?

戴维斯家简单的一个红外报警装置,就让楚云飞做了一路的检讨,这次要轻举妄动的话,怕是,做一年检讨都不顶用了。

国家安全局的早就说了,不是没有强行进入的能力,实在是,怕再次失败导致对方毁灭或者转移那……不知道是什么东西的东西。

要是楚云飞他们的行动过于卤莽,真的导致调查失败的话,那后果就实在不堪设想了,纵然国家未必会再发一次国际通缉令,可回国那肯定是永远都不要想的事了。

意识到这几点,而那“罗蒙特”工厂又无巧不巧地坐落在克鲁梭,那就由不得楚云飞不尽力地巴结索菲娅了。

还好那个单纯的女孩不知道这些,她还以为,有人……

再加上前面多尼的事莫名其妙地就被泄露,由不得楚云飞不紧张这事。

当然,这事的前因后果,楚云飞是不能说得太清楚的,哪怕这老头已经答应了不泄露秘密,自然还是要学那孔夫子的《春秋》,做些“削而不述”的事情。

“这事,还是法国克鲁梭那里的事,对了,维伦斯先生,不知道你有没有听说,上次工人党说,他们已经抓住我了?”

这个事情,无论是作为楚云飞的盟友,还是作为个超级笑话,宾塞斯都不可能不知道,他缓缓地点了点头。

“嗯,我知道,索菲娅还着急着联系她那个朋友露丝,想让她找我说情呢,幸亏我发现得早。”老头一说起来这个就想笑。

\section{第两百二十四章 慨然援手}

“对了,就是那个人啊,”楚云飞长叹一声,本来想做点丰富的表情,实在是怕如刚贝拉般弄巧成拙,终于还是用了“不动声色”这个招牌表情来表示。

“那是我的弟弟,在克鲁梭被人杀了。”

“哦?”老头这次可真的是大吃了一惊,这下事情,怕是大条了。

虽然接触不多,可楚云飞的性格,宾塞斯再清楚不过了,当然,只是偏激的那一方面。

能为几句无关紧要的话挑衅大名鼎鼎的黑手党家族——当然,老头认为那是无关紧要的;也能为一个萍水相逢的路人放弃巨额财富;更能为父亲的被害远上巴基斯坦,把整个俾鲁弯搅得血雨腥风,尸横遍野。

这次,他在巴基斯坦到底杀了多少人?怕是没有八百也有七百了。

老天,居然有人敢杀了他的弟弟?完了,宾塞斯衷心地在为那些凶手们惋惜,他们的人生,绝对已经进入倒计时了!

“我很遗憾听到这个消息,”老头没有怀疑楚云飞的话,事实上,两人长得很像,应该确实是兄弟吧,“那么,我能帮你做点什么呢?”

该要求他们做点什么呢?楚云飞也一直在考虑这个问题。

事关国家利益,楚云飞纵然是胆大包天,也不敢再随便不按牌理出牌,因为事情很重要,形势也很微妙,没有人有犯错误的权利。

当然,楚云飞求助的目标,并不是维伦斯家族,在法国玩间谍的话,美国黑手党比其他势力并不具备任何的优势。

他心目中合适的人选,是那个曾经被他无情地蹂躏过的“克鲁梭工人党”,不过,双方既然曾经弄得那么不愉快,那就必须仰仗维伦斯家族在中间大力调解了。

地头蛇自然是会有地头蛇的优势的,在克鲁梭,估计工人党的势力到达不了的地方是非常有限的,没准有都内应也未尝可知。

可是,工厂里的秘密实在是让人头疼得很,因为根本就没人知道那里面有什么东西。

要真的涉及到了绝密的军事情报,工人党人肯不肯出卖国家利益暂且不论,怕是他们都没那个胆子去跟政府作对,那绝对会遭致无情地清洗的,而且以这世界之大,那几个头目估计都无处藏身了。

所以,如果真是这种类型的秘密,楚云飞他们这种主顾反被出卖的可能性实在太大了,如此一来,他们当然要背起“破坏了行动”的责任。

为什么,我总遇到情报不充分这种事情呢?面对这样的局面,楚云飞不抱怨都不行。

退几步说,哪怕是绝密的商业情报,就是说是那工厂自己的事,以工厂严格防范的心思,戴维斯他们怕是也要尽全力才能够奏效的,可是,以楚云飞跟工人党的“交情”,值得他们这样做么?

而且,维伦斯家族也不是什么好鸟,如果这个秘密价值非常大,大到能给在美国的黑手党们带去一定利益的话,会不会弄得大家很被动?

要知道,有很多秘密之所以成为秘密,那只是知道的人少,才使得价值空前。

总之,想办好这件事,一定是要讲究策略的。

面对老头的提问,楚云飞点点头,依旧是一副别人欠了他钱的表情,“首先,我想拜托维伦斯先生问问工人党的人,这事是不是他们干的,这里面还有别的事情,所以还请先生转告他们,尽可能地不要声张这个事情。”

当然,楚云飞知道,这事绝对不是工人党做的,不过,兹事体大,实在不可能合盘托出的,只好先行投石问路。

反正肖逅是死了,无论从哪个角度出发,死者有家人打听死因和凶手是再正常不过的事了,这事搁到哪里都说得过去。

老头明显地觉出了里面的问题,他不相信戴维斯真有那个胆子敢再来找楚云飞的麻烦,不过,没准工人党不知道这层关系,那也是正常的,自己家还不知道这事呢。

“那么,要是他们干的呢?”面临这种场面,宾塞斯绝对不会给楚云飞造成偏向工人党的印象,所以他只能先假设对方是凶手。

“要是他们干的,那我只能要他们交出来凶手和下命令的人了,对了,还有委托人。”楚云飞的表情真的很严肃,不过话倒是很通情达理。

“看在你们维伦斯家的面子上,我提的要求并不过分,当然,工人党要是试图欺骗我,或者掩饰什么,那我会很生气的,非常生气。”

“嗯,”老头点点头,“他们不能欺骗你们,否则我也放不过他们,这个要求并不过分。”

“不过,这事,似乎还有下文,是吧?”宾塞斯谨慎地提出了疑问,“帮你这么个忙是应该的,但是,不知道以后的内容,我们维伦斯家是不是还有抽身退出的可能?”

多年的江湖绝对不是白闯的,老头眼里绝对不容沙子,而且,他非常直接地提了出来,起码在道义上是站得住脚的。

这老东西还真是狡猾,楚云飞暗自嘀咕一声,但站在人家的地位上想想,那就再正常不过了,家大业大,一失足成千古恨呐。

而且,宾塞斯的话也非常地光棍,先表明了他自己愿意帮忙的态度,这样,自己也不能过于小气。

“呵呵,”楚云飞嘴皮扯动一下,算是一笑吧,“我这里先感谢维伦斯先生的好意,面对这样热心的朋友,我郑重承诺,维伦斯家族,随时都有退出的权力,只要能帮我保守秘密就好了。”

宾塞斯本意是指望楚云飞把后续事情也略微交代一下,好满足一下自己的偷窥欲,不过对方已经做出了这样的承诺,那倒也是无关紧要的事了,反正眼前这个人,据他的了解,那绝对是说话算话的。

那么,就该为对方先办办事了,杀弟之仇算不小的仇恨了,总不能让人家催着自己办吧,那样未免就有点不够义气了,老头很自然地问了两句肖逅的情况,然后当着楚云飞的面拿起了电话。

\section{第两百二十五章 图未穷,匕将现}

这一个电话不要紧,克鲁梭工人党那里顿时鸡飞狗跳,乱做了一团。

前文说过,工人党和“基天”是有仇的,班克斯还试图以此为条件,换取楚云飞放弃对多尼的帮助,不过被拒绝了就是了。

经楚云飞在法国的那通折腾,班克斯的儿子瑟利尼因祸得福,地位直线上升,又因有黑手党的背景,看来问鼎“五大”也仅仅是个时间的问题。

可年轻人做事实在有点不着边际,瑟利尼没有感谢楚云飞给他带来的这个机会也就算了,居然还为了班克斯在上次事件中没有偏向工人党而恼怒,连带着恨起了楚云飞他们。

这也难怪,瑟利尼在工人党内有两个好朋友在这次事件中丧命了,而身在法国的他,是没有可能为在英国的父亲考虑的。

班克斯被这个外国的儿子埋汰得实在受不了啦,终于大发雷霆,你知道个屁,我还想抢那波兰凯子的钱呢,可是我敢么?打死你也想不到,那几个人有多恐怖。亏得老头子我跟他们关系一直不错,到最后劝架人家才肯听的,要是一开始就倒向你们,那结果,怕是只有上帝才知道了。

何况,你爷爷好歹也是被人家救活的。

瑟利尼一如所有的逆反青年一般,亏他也近三十岁了,谁的话都信,就是不信老爹的话:我没那么个操蛋爷爷,哼,说得比唱得还好听,要是你们肯出其不意地动手,那几个人还不是小菜一碟?

说到底,你就是要抛弃我了,年轻时你抛弃了我妈,现在轮到我了!

可怜的班克斯被气得浑身哆嗦,绅士风度是一点也看不见了,他跟别人还能讲讲理,跟这混小子,那是没有丝毫的办法。

得,惹不起我总躲得起吧?班克斯挥挥手就到了美国,没有带走一丝云彩。

可是,楚云飞他们最终从巴基斯坦全身而返,给了班克斯足够的理由来训斥这个逆子:你不是厉害么?“基天”你们还不是一样的惹不起?看看这几位,人家可是纵横巴基斯坦,足足杀了七八百人才回来,“基天”一样要挂免战牌的。

去的时候三个人,回来还是三个人,也没人缺胳膊短腿,仔细想想吧,小子,老爹这点智慧,不是你能理解的!

瑟利尼只是逆反心理重了点,这和自身遭遇有关,并不代表说缺少智商,他确实被这个消息惊呆了。这不,他正把听来的事跟戴维斯和其他几个头目闲聊呢:看来,那帮人实在是太狠了点,那个波兰佬命真好。

就在这个时候,宾塞斯的电话打了过去,很遗憾地通知他们:上次被你们误抓的那位,是那三个恶魔里某人的兄弟,他最近横死在了克鲁梭,人家发话了,想知道这事是不是你们做的。

流年不利!戴维斯只能这样评价自己的命运。

楚云飞他们的可怕,戴维斯知道得要比瑟利尼多得多,只是不方便声张就是了,这次再听了人家的“巴基斯坦游记”,那是半点侥幸心理都不敢有了。

再说,人家提的要求也很合理,当然,呃,前提是自己这方不要耍什么花样,否则,美国黑手党也饶不了自己。

当然,也有那对楚云飞等人心怀怨恨,不满意、不服气的主,但是被戴维斯严厉打压了,省省吧你们,人家死了爹拉了七八百人陪葬,这次是弟弟,你们觉得咱工人党有这么多人给人家杀么?

至于上次肖逅打死都不肯承认与楚云飞的关系,而且交换时也是一副陌生人的口气,那是很好理解的,无非是魔鬼们怕陷入被动就是了。

于是,调查的命令被雷厉风行地颁发了下去:两个多月前,帮里的兄弟,有没有人在克鲁梭对一个黄种人下黑手。

当然,还得应人家要求,非常、非常隐秘地调查。

调查结果在一天后汇报给了伦敦,是有人在克鲁梭对黄种人下过手,抢劫二十五次,强奸七次,杀人一次,不过经照片对照,证明死者不是魔鬼的弟弟。

这时的楚云飞,正在维伦斯家里同宾塞斯和索菲娅共进午餐呢。

楚云飞终于可以确定,就如那些西方名著所写的那样,西方人,果然是没有“食不言”的讲究的,他甚至怀疑那餐桌上的嘴巾,本来的用途是用来阻止食物碎屑乱飞的。

当然,主人们表现得还是非常有风度的,索菲娅更是注重仪容,或轻言慢语,或细嚼慢咽。

不过,饭桌上说话,实在是不太符合楚云飞的习惯,他总是习惯风卷残云般地吞咽完毕,然后再慢慢聊的。

再说,索菲娅的某些话经常会影响楚云飞的食欲,甚至因哽咽发生意外的可能性都存在的。

比如说现在,她就在问楚云飞一个很让他郁闷的问题,“既然你觉得我比你的情人还漂亮,那你为什么不愿意留在我家呢?不喜欢看到我么?”

宾塞斯欣然地看着自己的孙女在客人面前撒娇,在他眼里,孙女离长大还有那么点距离,再说,又是帮自己在表示强烈的招揽之意。

这时的管家,在楚云飞的眼中就显得非常地可爱了,他走过来很有风度地来了个三十度角,“老爷,有您的电话。”

于是,楚云飞得以有借口中止了这顿午餐。

听到戴维斯的回复,楚云飞的心情好了不少,这说明,工人党对自己,纵然是敢怒,却绝对是不敢言的,这是一个好的征兆。

那么,下一步的招数就该用出来了,尽量,尽量让它显得合理点吧。

“既然他们那么肯定,我想他们是不介意证明一下自己的清白的,”楚云飞说这话的时候,过江强龙的面目一览无疑,“我听说我弟弟说起过,他有个情人,经常受到一个名叫让。皮篷杜的家伙的骚扰,那家伙,似乎是克鲁梭一个工厂的负责人,我想知道那家伙的所有资料,呃,还有,他们工厂的经营项目和背景。”

这话意思很明显,起码这个叫让。皮篷杜的家伙,是有杀害他“弟弟”的嫌疑的。

\section{第两百二十六章 不许敷衍}

让。皮篷杜,就是那个“罗蒙特”工厂的负责人。

既然楚云飞做出了让步,已经愿意相信工人党的清白,那么他们再做点工作把自己来洗刷得干净点,应该也不是什么太困难的事吧。

事实上,这一步是最危险的,楚云飞也知道,这么做是有点过于自以为是了,前一步还可以说是为自己的弟弟出头,纵然“罗蒙特”对这一切有所觉察,但总也还在情理之中,这世界上,有太多的人不知道自己的亲戚朋友到底在做什么行当。

而且按道理说,如果真的是有心人对“罗蒙特”工厂还打着别的心思,怕也是没可能找到工人党的头上的,这个目标是绝对错误的。

可眼下这一步,就有点过于冒险了,他直接点出了目标,实在有点“图穷匕现”的味道了,这次要真被“罗蒙特”的人知道的话,那后果就严重多了。

会有多严重,楚云飞是不敢去想的,起码会极大地增加探察的难度,再厉害点的话,导致三人永远不能回国怕也是很正常的。

可是,楚云飞真的也没有什么更好的办法了,至于说三个人强行夜探那工厂,倒也不是不可以,可楚云飞从来都不太把自己的武力当回事,倒不是说妄自菲薄,实在是,能通过头脑解决的东西,为什么一定要用武力?

况且,他有把握甚至可以说有信心,进入那工厂并且全身而退,但是他是进去探察秘密而不是杀人的,至关重要的是,完成任务的同时还坚决不能暴露,这难度可就大多了。

事关三个人能否回国,楚云飞没有犯错误的权力。

总之,私下里别出机杼,是有必要的,反正,这事实在是没人可以商量的,一切,都要指望楚云飞自己了。

何况,在这件事情的操作过程中,楚云飞还有自己的小算盘,他想的是,哪怕,在这步行动中出现了什么小小的纰漏,“罗蒙特”工厂因此提高了警觉,甚至于销毁了相关的证据,可引起人注意的,是工人党,并不是他楚云飞。

他甚至可以反驳,不错,国安局是做这种事的专业部门,可是,你们能担保,引起对方注意的不是你们的人么?做事总要讲证据的吧?

事实本来也就如此,谁能保证国安局现在没被对方发现?国家安全局,又不是没有失手的时候。

所以,楚云飞才要尽量保证这事的隐秘,坚决不让过多的人了解其中内情。

这个小算盘,楚云飞甚至都不想让自己两个战友知道,因为,这样的想法实在是操蛋了点,以俩战友“祖国至上”的血性,没准会痛骂他呢。

可楚云飞一点都不认为自己做的有什么不妥,两种方案并不能说哪种一定会比另一种好,或者说比另一种成功率高,只是眼下这个手段,似乎有点卑鄙就是了。

这世界上,以成败论的并不只是英雄。楚云飞这样安慰自己。

工人党的消息很快就传了回来,据说他们做得是很谨慎也是很小心的,终于弄明白了“罗蒙特”工厂的内情,和那个让。皮篷杜的资料。

情报上显示,没有任何迹象表明,这家工厂有什么诡异的地方,要说有,也不过就是这工厂还承接法国军工厂的定单,生产些精密机床来供应这些工厂。

至于让。皮篷杜,是法国男人里少有的“妻管严”,偶尔偷偷嘴是有的,但绝对不可能因为女人而大动干戈。

看着手里翻译来的传真件,楚云飞嘴里露出了一丝冷笑,这些家伙,还真是记吃不记打呀,我可是给过你们机会了啊。

宾塞斯在旁边都被这冷笑弄得心惊胆战,“楚,有什么不对的地方么?”

不对的地方?那自然是有的,要是连判断资料的水平都没有,楚云飞凭什么敢要工人党帮忙?他们以前的梁子也不过是勉强化解的就是了,防人之心不可无的。

楚云飞点点头,“是有些地方不对,看来,法国的这些朋友,并不着急洗脱自己的嫌疑啊。”

他这么说是有根据的,“罗蒙特”工厂的资料,国家安全局是给过他一些的,资料上显示,他们不仅与法国的军工厂有生意,与外国的军工厂也有联系,嗯,主要是欧盟内部的。

“罗蒙特”作为克鲁梭的本土企业,这点东西,工人党是不可能不知道的,他们可也是玩军火出身的呢。而他们在发来的传真中并没有写上去,这实在是在挑衅楚云飞的忍耐力,呃,或者说智商吧。

抑或说,工人党在小看楚云飞的情报网。

当然,也有另一种很善意的可能,那就是工人党觉得提供的这些资料已经足够了,一些枝节就不需要赘述了。

毕竟,恶魔你想查的是人而不是企业。

不过,这个借口,在楚云飞这里是行不通的,同强者打交道,你必须是小心又小心的,这是常识。

既然你们不懂常识,或者有意隐瞒来挑衅强者,那么,楚云飞是不介意再给他们上一课的。

这个理由不怕别人知道,楚云飞很痛快地向老头指出了工人党不负责任的一面:据我的了解,那个工厂经营的并不只这些,显然,那些法国佬想欺骗我,或者说,他们对维伦斯家族实在是不够尊敬。

“维伦斯先生,我想,我有必要去法国一趟了,工人党做事实在欠妥当,我必须让他们明白,做事不负责任的后果,不知道您有什么好的途径,能让我方便地来和去?”

宾塞斯年老成精,已经觉出了里面的味道,对这个工厂,楚云飞了解得实在太多了点,恐怕这里,才是他计划中的重点吧?

“哦,军工厂,我不是很喜欢这个词组,它总是和阴谋、血腥等名词联系起来的,我想知道的是,我现在退出的话,是不是会给你造成很大的困惑?”

楚云飞扬扬眉毛,撇撇嘴,做出一副很遗憾的样子,“哦,那当然会啦,不过,我总是很尊重你的选择的,维伦斯先生,谁要我们是朋友呢?”

\section{第两百二十七章 亮出底牌}

事实上,维伦斯家族的使命基本上可以告一段落了,楚云飞并没有指望能得到太多的东西,他现在最需要的是:保密。

不过,他基本上可以确定,“军工厂”这个词组,并不会是维伦斯家族厌恶的,说是块香喷喷的诱饵还差不多,世界上能比军火更赚钱的买卖并不多。

宾塞斯还是谨慎地下水了,没办法,谁要他好奇心那么强呢?或者说,没准他想再开辟条军火采购的途径也未可知。

“呵呵,我只是随便说说,说实话,在英国,除了我,怕是格瑞尔家族也不能向你提供更加稳妥的偷渡渠道,我怎么能眼睁睁地看着你四处寻找帮助呢?”

楚云飞很开心地笑笑,“呵呵,谢谢你了,我承诺,如果这件事里,能让你们家族获得好处,我绝对会为你们留下充足的份额的。”

当然,这只是一个承诺,如果事情的结果不方便告知老头,楚云飞是绝对不会含糊的,所以,他不怕许愿。

于是,当天晚上,楚云飞就搭上了开往法国的偷渡船,让他震惊的是,这船的豪华,怕是跟那普通头等舱也有得一比了,可它还是用来偷渡的。

第二个晚上,楚云飞又回来了,在这段时间里,他再次绑架了戴维斯的三女儿,这也是没办法的事,谁要他只认识这么一个重要人物呢?

这次戴维斯家的警戒增强了不少,不过,人们的警惕心却没多了多少,楚云飞虽然再次被人发现,可他依旧从容地带走了人质。

女孩又重了一些,扛着这么大的物体又急于离去,被人发现也很正常。

楚云飞还在马赛港的时候,愤怒的戴维斯已经把电话打到了维伦斯家,这次还是那个速度奇快的人干的事,他自然知道到哪里才能讨回公道。

不幸的是,宾塞斯已经睡觉了,半夜被叫起来的老头异常地愤怒,“你还好意思找我?说实话,我们维伦斯家对你们工人党怎么样?”

人比人,气死人,戴维斯就算再愤怒,也不敢跟老头撒野,只能异常委屈地解释,“维伦斯先生,您说这话就见外了,谁不知道您对我们一向是很关照的?这不?瑟利尼现在在我这里已经开始负责对外联络,有了自己的局面呢。”

老头实在是有翻云覆雨的手段的,下床气已经消了,开始教导起对方来,“嗯,这次,中国人给你们带去了什么损失?……对啊,没什么损失,那就是说,人家还没真生气呢,可能,可能你们沟通上有什么问题吧?”

沟通上自然有问题,老头知道得一清二楚,不过,对盟友说话,嗯,传统盟友,还是要稍微讲究点的。

那边的戴维斯不知道是真不知道还是装迷糊,听起来很纳闷地回了一句,“这个,维伦斯先生,您能解释得清楚点么?”

老头比他还会装迷糊,“嗯,我也是猜测,其实,我真不清楚到底是怎么回事,总是和中国人要求你们做的事有关吧?”

说实话,宾塞斯也不知道楚云飞去法国做什么,还要快去快回,不过老头能想到,那家伙怕是教育工人党去了,这种情况,老头巴不得自己什么都不知道,哪还有闲心去追问这事?

戴维斯在那边似乎是想到了什么,沉默半晌,压了电话。

事实证明,工人党确实是耍了心眼,留了一手,这不,楚云飞还在英吉利海峡上漂着呢,维伦斯家又收到了传真。

这次的传真,足足有三十八页纸,上面还写得密密麻麻,详细到让。皮篷杜内衣的尺码、最新女友家小狗的名字,以及前一天抽的雪茄的牌子。

关于工厂,更是拉出了一列清单,那是最近“罗蒙特”工厂的所有订单,有已经交货的,也有未曾交货的。

更有一点指出,恶魔的弟弟似乎曾经在工厂附近出现过,而且,几天后,高速路上出现了一具黄种人的尸体。

情报的末尾,还用大号的字体加重标出——“此情报保证可靠性和隐秘性。”当然,楚云飞是看不懂的,他只能等着宾塞斯的私人秘书给他慢慢翻译。

看来,将来我也有混黑社会的潜质,只是老妈那关绝对过不去,无所事事的楚云飞坐在沙发上胡思乱想。

老头已经从他手里成功地“解救”出了戴维斯的女儿,此刻的小姑娘正在熟睡中。

没等多久,情报就交到了他的手上,楚云飞迫不及待地看了起来。

花了整整三个小时,楚云飞还是没能弄明白手里这份情报的重要性,不过,他倒是可以确定,这次,工人党没敢再耍什么花样。

看来一切正常,并没有出什么纰漏,楚云飞沉思半天,终于决定,走最后一步吧,也是计划里最重要的一步。

那就是,直接向工人党索要肖逅的死因,包括工厂里的秘密,对了,楚云飞还想要那个生产线的照片。

手里有了人质的楚云飞终于不再顾忌,而且,工人党现在给他的这份情报也足以使他们被拉下水了,不过,纵然是如此,楚云飞也没想当然地命令对方做什么事情,而是郑重表态,他愿意花钱来买以上的内容!

不但要有威逼,还要有利诱,这个,才是混社会的王道。

事实上,看到那工厂最近的交易内容,楚云飞大致已经可以断定,这工厂里,实在是没可能有太大的秘密的。

估计,最多就是有些比较极端的商业行为吧。

工人党再次受到了震动,因为,那恶魔终于肯放过他们了,小姐被安置在了盟友家里,而对方居然还肯出钱来买一些情报。

可恶的中国人还说了,如果能帮他实现这个目标,工人党会获得中国人的友谊,老天,我们不需要你的友谊,只是希望大家不要再见面了,行不行?

不过,楚云飞的善意终于被传达过去了。

当然,那仅仅是个善意,并不代表中国人就不会因为下一个情报的不准确而大发雷霆,要知道,上次可是有维伦斯家族做主呢,还不是一样被人绑架了小姐?

于是,在下一个夜里,“罗蒙特”的总工程师、销售部经理、保安队队长同时被不明来路的人绑架!

\section{第两百二十八章 倦云归去兮(大结局)}

其实,戴维斯很想拒绝楚云飞的请求,你的善意我们不需要,大家各走各的路吧。

不过,他显然没有说这个话的能力和勇气,人在矮檐下,怎敢不低头?

昨天的事情已经做出了再明白不过的提示,要么,就是同那些魔鬼对抗到底;要么,就是尽心办好魔鬼托付的事情。

工人党本来就是从流氓无产者中脱颖而出,“罗蒙特”工厂里就有几个外围成员的,当然,这些人未必能很好地保密,戴维斯不打算用到他们。

“罗蒙特”工厂的工会里,有工人党的隐秘力量,那是为了将来万一承接精密仪器走私买卖用的力量,现在居然被用来拍几张照片,实在是有点浪费,可是,戴维斯有别的选择么?

“罗蒙特”工厂几个中层干部被蒙着脸的歹徒一顿敲打,吐露了不少工厂内的秘密,又被隐蔽地放了回去,第二天一早,这些人在工厂里鼻青脸肿地相遇,却是异常统一地没有追问对方的遭遇。

用后来的话说,就是:被绑架的不止一人,谁也不能指认出具体是哪个人说了不该说的东西。

一番辛苦过后,戴维斯总算把情报弄到了手里,他终于明白了中国人到底想知道什么了。

那是一个和中国有关的秘密,不过,对世界上大多数国家而言,却是不值得一提的。

手里拿着这情报,戴维斯又开始琢磨这几个中国人的身份,难道说,这真的是中国政府的力量么?不像呀。

算了,不想了,知道得越多,麻烦越多。虽然对工人党而言,麻烦往往能带来巨额的利润,但交易对象是这几个人的话,戴维斯并不奢求真能敲到什么东西。

情报,被传到了宾塞斯那里,照片也有专人运送,正在路上。

看着手里的情报,老头又有点生气了,不过片刻之后,又笑了起来,这个年轻人啊,实在不是个好东西,居然说这件事里,维伦斯家也能得到利益,我能得到屁的利益!

嗯,是该考虑怎么惩罚一下这个可恶的家伙了,宾塞斯略微想了一下,马上就做出了决定:提高情报的卖价,戴维斯开价五十万美圆,我给他涨到二百万好了。

反正这个情报也只对中国人有用,他是没可能不买的。

说实话,戴维斯的开价已经有点略微高了呢。

不过,天地良心,楚云飞可真的不知道里面的内容的,这次他绝对是被冤枉的,可,这也能理解,谁让他平时净玩小聪明,把名声败坏了呢?

楚云飞接到老头的电话,兴冲冲地赶来过来,心里还在不停地自鸣得意,看看,谁说事情一定要按常情去做?这不是效果也不错么?

迎接他的,是老头诡异的笑容,“楚,你显然又在欺骗我这个可怜的老头了。”

说着,宾塞斯扬扬手里的那一页纸,上面只有寥寥地几行字,“这是从克鲁梭发来的消息,很遗憾,也许是我老眼昏花了,实在是看不出来里面有什么我能得利的地方啊。”

哦,你不能得利,那太遗憾了,楚云飞在心里说,不过,我能得利也不错啊。

“而且,不客气地说,”老头笑得越发诡异,这个表情应该属于白雪公主的后母的,“这个情报,能得利的只有你,我需要补偿。”

好吧,补偿,楚云飞点点头,事实上,老头的表情越做作,那就证明情报的可靠性越高,楚云飞是很乐意见到这个场面的,“没问题,其实,维伦斯先生,就算没有这样的事,需要我的帮助也就是一句话的事,难道不是么?”

“不是,”老头连连摇头,他知道,这样扬眉吐气的时候不多,自然要多嚣张一会儿,“似乎,似乎上次你和我的儿子班克斯说过,你需要我们付钱才肯帮忙的。”

现在的楚云飞真是无话可说了,这个,他平时做人,似乎确实是有点失败,老话说得确实不错——做人要厚道。

不过,宾塞斯也不愿意把弦绷得太紧,经验告诉他:不同的国家和民族之间,是存在文化差异的,对不太熟的人,还是不要开过分的玩笑,尤其是这种比较……比较不讲理的人。

“好了,不跟你开玩笑了,你记着你的话就好,以后不能跟我家收钱,”老头说话,一锤定音,“这个情报你拿去,对了,你需要支付二百万美圆,我个人觉得,这个数字还是比较合理的。”

二百万美圆?楚云飞愣了一下,这个数字,似乎有点庞大吧?我们哥几个,总共加起来也没有三百万美圆。

不过,有了这张纸就能回国了,楚云飞可无暇再计较这些小事了,反正那俩说过,哪怕丢掉在国外的一切,也要回去,那这事就不用再同他们商量了。

拿到这张纸,楚云飞才知道自己为什么样的大便买了单。

原来,这事关系到台湾!

法国同台湾签定了供应六条常规柴油动力潜艇的合同,可在中国政府的强烈抗议下,法国毁约了。

大陆市场,实在比那弹丸之地大多了,而且,实力和影响力也是相差悬殊的。

其实,这只是法国企业的敲门砖,他们进入中国市场比较晚,想引起重视得到照顾,那是要采用些非常规手段的,政府在这点上对自家的企业倒是满支持的,官商勾结,以达到打开市场的目的。

这事本来可以到此为止了,但有家不知道死活的西班牙公司跳了出来,这六条潜艇,他们接了。

西班牙人的素质,实在是太欠缺了点,本身就是一个懒惰而愚昧的国家,却总想着近代史中那唯一辉煌的片刻,“无敌舰队”——有郑和带的船队大么?

活是接了,等他们着手干的时候才知道,制约因素太多了。

中国政府抗议之类的非技术因素就不提了,技术上他们也达不到台湾的要求,其中就有一项:螺旋桨尾叶加工的精密度不能满足设计要求!

这是属于加工能力的问题,“罗蒙特”工厂能生产满足要求的机床。

于是,西班牙人去法国人那里购买机床。

既然已经有了相应的承诺,法国人按理说是不能再对潜艇的生产提供什么支持和帮助了,不过,这毕竟是企业行为,它们有自己的借口:没人能担保那西班牙工厂不会在其他用途上用到这样的机床。

当然,法国政府为了避嫌,他们是不支持这样的买卖的,可是,事关“罗蒙特”工厂的利润,他们只有伪做不知。

“罗蒙特”工厂也是如此,我不但要赚钱,还要稳妥,这事说穿了的话,会给法国政府带去太多的麻烦,没准在下一刻,受到太多在中国有既得利益的大集团的压力,法国政府会对他们做出一些非常离谱的惩罚,那也是难说的。

思来想去,这单买卖“罗蒙特”工厂是接了,不过,在核心部分,他们做了些手脚,让那机床只能做些普通的高精度加工,而绝对不能加工螺旋桨。

这是绝对违反商业道德的,西班牙那里马上就提出了抗议,当然,他们不能说——“我们不能加工为台湾做的螺旋桨”,所以,他们需要证据,“罗蒙特”工厂违反商业道德的证据。

肖逅,就是在这个敏感时候进入工厂的,他显然是被当作台湾间谍处理掉了,反正他没有暴露身份的权力。

事实上,这份情报,中国政府还是有很多种利用途径的,起码,他们能指责法国政府阳奉阴违,没有大国间交往的诚意。

说到底,一切,都是利益在驱使着这些可怜的人,肖逅只是其中的一名受害者。

这个情报,到底值不值二百万美圆,谁也说不清楚,不过,可以肯定的是,使用得好的话,这个情报的价值可以超过它本身的十倍百倍!

于是,楚云飞他们,终于是可以回国了,这时候,成树国的伤口甚至还没有完全愈合!

最让楚云飞遗憾的是,国家安全局只报销了他们百分之五的费用——十万美圆!。。。。。。。。。。。。。。。。。。。。。。。

伦敦希思罗机场,一位头发花白的清洁工斜倚着墙壁,旁边是一根吸尘器的长轴手柄斜搭在身上。老妇人手中是叠厚纸,或者是信件或者是照片,她全神贯注地盯着手中的一叠,布满沧桑的脸上是写满了甜蜜的微笑。

看到这个场景,楚云飞碰碰那俩战友,三人的眼睛同时湿润了……

