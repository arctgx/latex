\chapter{第二卷 天苍苍风卷云舒}

\section{第五十二章 吃肉的苍蝇}

黄沙漫漫,几棵矮小的栎树下,扔着一具小角马的尸体,从脖子上汩汩冒出的鲜血已经不再强劲外涌,接近凝固的血液吸引来众多的非洲绿蝇来争着享受。三只鬣狗远远地绕着角马的尸体转悠着,偶尔向近处凑凑却马上又拼命地跑开。

树下坐着两个黄种人,有一句没一句的有气无力地聊着天,树枝上还坐着一个,脖子上挂个望远镜,时不时地拿起来向四周望望。

中国驻刚卡的维和部队已经撤走七八天了,这留下的三人无疑就是楚云飞、成树国和刘宁了。

他们从军营里偷跑出来有20天了,楚云飞和刘宁的伤口早就好了,其中楚云飞伤口缝的线是成树国给拆的,也不知道这家伙从哪里学的这一手。

在维和部队撤离的过程中,高建军又送来了不少东西,其中以药品居多,这时就没什么人过问了。因为部队走得仓促,很多东西没必要一定带走的,既然驻地四周的本地人都实落了不少好处,为什么不给自己的战友多弄点呢?

高建军走得比较早,但在他走后,约定的补给点上还总是有物资出现,一开始三人很谨慎的接近那些东西,却没发现任何的陷阱,于是老实不客气地收下了。这种现象一直持续到驻地的人走完才结束,最后一次送来的东西居然是英文版的非洲地图一箱,崭新的那种,箱里附有纸条一张,上面只有两个字:保重。却看不出是谁写的。

这期间刚卡军队来过两次搜索三人,三人的厉害他们早有耳闻,所以一来总是四、五十号人扎堆。

但黑种男人的耐心实在没有什么值得夸奖的地方,每次都是草草地筛上一遍就走。上次来的时候,一支搜索队甚至鬼使神差地走到了距离他们藏身的土坑只有5米的地方,当三人都下定决心多拉几个垫背的时候,对方却留下两泡尿走路了。

刘宁懒洋洋地伸手赶赶角马身上越来越多的绿蝇,远处的鬣狗也因为这个动作触电般地散开,“云飞,你觉得他们还会来么?”

树上的楚云飞放下望远镜,“我觉得不会了吧?每次来这么多人,费用不少啊,再说了,他们这样也能和别人交代了,难道非要找出我们不可?这年头杀人的人多了去了,只不过咱刚好杀了几个大点的就是了。”

成树国看他俩说得热闹,也插嘴了,“我也觉得他们不会来了,操,多大点事啊?就咱们巡逻这片,哪天还不死他十来八个的?”

刘宁咬咬牙,“我是怕他们再来,我控制不住自己,要跟他们拼命了。”

三人都不说话了,是啊,这种孤寂的生活,什么时候是个头啊?过了半天,楚云飞才说,“要不咱们再往北走走吧,那里是三族势力交错的地方,看看有什么机会没有,再说咱们总在驻地附近目标有点太大了。”

成树国跳了起来,“总算天黑了,开灶开灶,饿死我了。”

刘宁纹丝不动,“再等等,等天大黑了,再顺便看能不能弄死只鬣狗,角马肉没那个香。”

成树国又坐了下来,“拉倒吧,这几只一看就是老狗,你以为他们是阉过的?肉太骚,肯定不好吃。”

楚云飞苦笑起来,“就算你们不同意,也不用这么明显地转移话题吧?”

成树国笑了,“我不是不同意,问题是咱们这么多东西,怎么拿呀?”

刘宁说话了,“有什么不同意的?我同意,该换个地方了,东西怎么拿,我想办法。”

说话间天就大黑了,三人把小角马拖到一个土坑内,刘宁和成树国剥皮,楚云飞张罗着生火、准备调料、吊架和顺便放哨。

三人白天是不敢起火做饭的,烟一起,十几里以外都能看见,很难说会招来些什么东西。可剥皮也在黑夜那就是上次的教训导致的。那次是三人第一次狩猎,光吃压缩饼干和方便面实在让人反胃,而且调料也齐全,三人就打了只小角马,到不是说打不动大角马,实在是成树国坚持要打小的,虽然肉少,但是好吃,他说大的因为没骟过,肯定是骚的。

因为没经验,他们就在白天剥皮了,这下可坏了,招来了铺天盖地的绿蝇,而且浓烈的血腥味引了一大群鬣狗来,虽然那些鬣狗都聪明得很,一看三人举枪就四散逃命,但那些绿蝇就不同了,基本上在快打到它们的时候才肯飞起,而且这些通体绿色的苍蝇是吃肉的,居然有锋利的牙,把肉啃得坑坑洼洼不说,撵得它们急了还想咬人呢。而它们的天敌,一种小体型的隼却是因为周围有人而不是狮子什么的不敢飞下来用餐。

虽然知道这些绿蝇不是以食用排泄物为生,但是难保它们不吃腐肉什么的,再说它们所属种类实在不能让人有什么好感。所以,上次一只角马大家只吃了一条没来得剥皮的后腿。

夜里剥皮该没什么问题了吧?成树国和刘宁正在嘀咕,却听见楚云飞开始发牢骚,“操,还让不让人吃饭了?这么多的狒狒?”

两人闻言一愣,站起来扭头一看:可不是,足足有20多只,这是非洲唯一不怎么害怕火的动物,而且是群居的,不但智力超群,武力也很强大,单个的成年狒狒就敢在豹子口下夺食,有这么一群,就算狮子都是只有跑的份。

成树国拿起一旁的步枪,“咯啦”就上了膛,“要打么?”这种情况下,配合是很重要的,成树国并不是热血上头的冲动,而是群体中必须有人在这种情况下保持相应的支持火力以防万一,而这“小队长三人组”目前的配合正在越来越默契。

看着楚云飞缓缓地放下扎吊架的铁丝站起来,刘宁摇摇头,拿起剥皮用的军用匕首,“不能打,太费子弹。”

“费子弹?”

“是,费子弹,别以为咱们有六千多的子弹就够,谁知道咱们要在这鬼地方呆多长时间?”刘宁表情严肃起来,“而且,狒狒是很记仇的,惹了它们很不好过,建军和我说过。”刘宁和高建军的关系确实非常好,好到他恨不得当初逃跑的时候拉上高建军。

“我来,”楚云飞眼盯前方,向刘、成二人身后的狒狒慢慢走去,虽然事关重大,但他有种直觉:这群家伙对自己构不成什么威胁。

“等等,”成树国接过刘宁的匕首,三五下就把角马的四条腿卸了下来,“把角马身子给了它们吧,里面有几只小狒狒呢。”

狒狒们明显地被匕首的锋利吓了一跳,不过看到楚云飞手里没拿什么武器,又拖着角马的身子,就慢慢地围了上来,有两只健壮的狒狒还挑衅似向楚云飞蹿来,看到对方没什么反应又急速的向后蹦。

用角马身子把狒狒们引得离开火堆五十米左右,楚云飞扔下角马,掉头就回去了,狒狒们却没一拥而上抢食,而是看着楚云飞的背影蠢蠢欲动,但似乎又在考虑收拾下这么大个家伙会不会得不偿失。

楚云飞慢慢地走近了大坑,成树国窜出来一拍他的肩膀,“牛,还是你牛,居然敢背对它们走回来,哥哥今天是服你了。”

楚云飞刚想说什么,一直观察楚云飞背后情况的刘宁“咯啦”把枪上膛,“操,没完了?”

楚云飞回头一看,四、五只健壮的狒狒正向他们一蹦一蹦的跑来,跑到坑跟前停住了脚。熊熊的篝火照耀映射到它们眼内,竟然反射出一种金色偏红的诡异光芒,可那诡异却丝毫掩饰不住赤裸裸的贪婪之色。

成树国可不管那套,看到有两人警戒,他又拿起匕首,想剥最后一条没剥的马腿。那狒狒们看到他拿起了食物,同时发出了低沉的吼声。

人类和狒狒肯定是没有共同语言的,但狒狒们发出的声音三人可都听懂了:那是赤裸裸的威胁,别动我们的食物!

\section{第五十三章 凶猛的狒狒}

三个年轻人都火了,刘宁发话了,“云飞也拿把匕首,你俩一起上,我掩护。”他知道自己的身手没这俩特种兵专业的好。

楚云飞制止了成树国放下枪的动作,“你别上了,交给我了,不用刀也没问题。”

楚云飞调理着内息,慢慢走上去。自从和周琳琳欢好后,他那种“伪先天境界”似乎上了一个台阶,很容易出现了,虽说想用的时候不是次次都灵,但十次中总能成功个七、八次。

但是知道他有这个功夫的人却不是很多,因为耿风和废人关都跟他说过,由于练气者要比普通人厉害很多,所以以强凌弱是武者大忌,虽然并没有明确的条款规定,但却是大家公认的准则。

而且,以楚云飞的身手,绝大部分时候也不需要非到达那种境界才可以打架,事实上,自从上军校以后他打架就没有吃亏过,自然就没几个人知道他这个秘密。其中有个舍友知道了他这功夫,因为也是特种兵专业,所以就缠着他教授,虽然事实上连楚云飞自己都不知道是怎么练成的。

可楚云飞这话说出来他的舍友自然是不信的,而楚云飞虽然脑瓜够用,实在又不习惯用谎话拒绝战友,后来两人还因此闹了一阵的别扭,他就更不愿意没事就显示这功夫了。这次来刚卡整个部队也就是郭平平知道楚云飞的这套,这还是因为郭某人自认功夫好稍微嚣张了点,才被楚云飞悄悄喊出去“谈了谈心”。

感受着那份空灵和明晰,楚云飞脚步轻快地向五只狒狒走过去,那些狒狒们却是吓了一跳,向后蹦去,而楚云飞再不肯放过它们了,一步一步的逼去。终于,在离开战友10米左右的位置,看到楚云飞已经失去了支援,离得最远、体格最大、貌似首领的狒狒低沉“呜呜”地喊了两声,其余四只就围了上来。

就在一只狒狒又蹿到楚云飞面前挑衅时,楚云飞出手了,迅疾如风的一拳出去,那狒狒虽然极其敏捷地后蹦,但楚云飞的拳头依然是后发先制,“砰”的一声响,那狒狒胸部中拳,直接被击出有五、六米远,其他狒狒吓了一跳,火速蹦开。

因为那只狒狒在中拳时已经后退,而狒狒本身的筋骨也是非常结实的,在地上打了几个滚以后,那家伙站了起来。很显然它已经被打得怒火中烧,虽然头晕眼花,却是发出一声怪叫,又扑了过来。其余三只也是如“黔之驴”里的老虎一般,看到楚云飞“技止此耳”,同时扑了上来。

楚云飞头脑越发的空灵,左手闪电般探出抓住一只狒狒的前臂,蹲身滑步,躲开另一只的下扑,同时借着对方的冲势,把手里的狒狒悠起狠狠地砸在另一只狒狒身上,两只狒狒还没来得及跌倒,楚云飞左脚前出,右脚结结实实地踢在最后扑来的一只狒狒身上,这只体重接近两百斤狒狒飞出去足足有十米远,那是再也爬不起来了。

那只唯一没受伤的狒狒扑过头以后,敏捷的一掉头,才发现三个同伴已经倒地不起了,看到楚云飞盯着它,一时间居然再不敢冲上来了。

就在这时,那只领头的狒狒动了,从楚云飞的侧后方扑了上来,两臂前伸,巨大的嘴巴张开,露出摄人的獠牙,前方这只也不敢再发呆,同样张着嘴扑了上来。

就这两下?楚云飞嘴角露出一丝不屑的笑容,右腿一个后踢,先把狒狒首领踹倒,左拳击出,目标正前方的狒狒,却没想到那狒狒首领前冲劲太大,踹是踹倒了,可左拳这落点就未免有了点偏差,没打到前方狒狒鼻子上,而是一拳击入那血盆大口中。

那狒狒看到首领被击倒,顾不得拳头带的劲风差点打落满口的牙齿,狠狠地就是一口,却意外发现平时连野牛皮都能轻易咬穿的牙齿居然咬不动这细皮嫩肉。它正在加劲努力,却发现手的主人在收回拳头,松口那自然是不可能的,于是这只体重近两百斤就一个踉跄被带到了楚云飞面前。

这狒狒倒也算个狠的,嘴里还叼着对方的手,两臂却又冲着楚云飞的面门抓了过去,楚云飞急切间也懒得挣脱,头一偏,左臂一使劲,就把这二百斤悠了起来,在头上悠了几圈,狠狠地又把它砸到了又扑过来的狒狒首领的身上。

那狒狒刚用双爪抓住楚云飞的左臂,想松开牙根已经松动的大嘴,却没想整个身子重重地冒犯到了首领身上,“砰”的一声响,背脊处传来的剧痛使它不由得松开了紧抓的双爪,然后就是专心地在地上翻来滚去。

那首领两次被击退,正在考虑是不是要放弃争斗,却见楚云飞双腿连环踢出,刚站起的狒狒们又纷纷倒地。

楚云飞狠狠地盯着狼狈的狒狒们,考虑要不要再给它们几下狠的,却感觉嘴唇有点发干,不由得伸出舌头舔舔嘴唇。

没想到这个小小的动作使得狒狒首领惊骇欲绝,发出惊天动地的一声嚎叫,头也不回地风驰电掣而去,听到这遇到狮群也未必会有的警告声,那地上的狒狒们也顾不得身体的不适,踉跄地爬起,没命地四散奔逃。

楚云飞愣了一下,马上反应了过来,感情它们以为舔嘴唇是自己要用餐呢。至于么?好歹你们也是灵长目的近亲,咱哥们象是个没事对你们下手的主么?

楚云飞趾高气扬地走回土坑,迎接他的果然是两道极其钦佩的目光。刘宁走上来重重地一拍他肩膀:“兄弟,你拽呀,这就是武功吧?”

成树国的话里可就夹枪带棒了,“没想到啊没想到,这么多年的同学,还不知道你有这么一手,狒狒都咬不动你,以后遇到狮子也归你对付了啊,我们哥俩给你压阵。”

楚云飞冲着成树国微微一笑,“咱做人厚道,先声明了,功夫用来跑路的话,那是能上奥运会的,你们哥俩到时候可是别怪我没提前打招呼啊。”

接下来的回答肯定是让楚云飞头疼的,刘宁果然说了,“有功夫还藏着掖着,这可不能算是兄弟吧?”

楚云飞早想好答案了,“操,都这地步了,你们要不好好学就等挨揍吧,先叫个‘师傅’来听听。”

\section{第五十四章 人比狮子毒}

楚云飞和狒狒一战之后,黄色人种的厉害给狒狒群留下了难以磨灭的印象,虽然黑种人也很厉害,但是在狒狒们的眼中,那些黑不溜丢的两腿动物也就是敢仗着数量多来欺负自己,这黄颜色的两腿动物那才叫厉害,一个就能把群中的所有壮劳力干掉,对这种领地王者还是离得远点的好。

然而三人实在闲得无聊,每天除去打坐练功,倒有一大半时间在闲逛打盹。成树国看到成年狒狒每每抢夺小狒狒的口粮,总是撺掇楚云飞去教训它们。

楚云飞实在懒得管那些动物界的事:人家自有人家的生存法则,要咱这些等同是外星人的物种插什么的手?可有一天,一只小狒狒偷吃了一只成年狒狒的食物,在追打之下慌不择路地跑向了三人。那追打者知道这几只黄色肉食动物的恐怖,没敢再追,而是幸灾乐祸地想看到偷食者被猎食的场面,但很遗憾,那几只黄色的动物居然没理会那小家伙。

就这么一来二去几次过后,整个狒狒群都知道这几个王者对它们不感兴趣,或许是自己的肉不合对方口味吧?而且王者在食物充裕时,还能救济那些小狒狒们点。

虽然成年狒狒后来因为抢食被那个最瘦的王者痛打过,后来也不敢再在王者面前欺负弱小,但是当一群六只的狮子出现在领地时,狒狒们还是指望王者为它们出头,守护一方平安。

有狮子出没实在是很糟糕的事情,楚云飞三人晚上连睡觉都不敢睡塌实了。而据刘宁说,狮子也是种很记仇的动物,所以在没有完全歼灭狮群的可能时,最好不要去招惹它们,以免引来不必要的麻烦。

其实这点是刘宁记错了,高建军虽然当时是这么说的,但特指的是雄狮,实际上母狮子没那么强的报复心的。而这片土地上新来的狮群里,只有一只雄性狮子,其它是四只母狮子和一只小狮子。只要把公狮子杀死,这个狮群就会立刻解散的。

“小队长三人组”正在为歼灭狮子做准备的时候,当地人也发现了新来的狮群。他们对狮子可就了解得太多了,这种东西绝对不能让它存在,要不吃人那是迟早的事情,而且狮子一旦开始吃人,那就不会再对其他动物感兴趣了。一百年前“食人狮”造成的惨状大家还记忆犹新[注1].再说,刚卡这么落后的地方,是没有“动物保护协会”的,就算有,狮子也绝对不会是被保护的对象。

于是,当地离狮群最近的图西族一个部落派出了将近二十人的狩猎队,对狮群进行了围剿,自然公狮子是他们的首要打击目标。

当狩猎队开始围剿的时候,“小队长三人组”也悄悄出现了,以刘宁的意思,就是配合图西人把狮子干掉。这些图西人里应该有去过驻地的人的,而驻地对当地居民的态度实在是不错的,有情义在先,现在帮忙在后,双方沟通起来应该是不难的吧?

狩猎队乱枪齐发,当场打伤了公狮子,然后经过孜孜不倦的围猎,终于将公狮子杀死。就在同时,中国人的枪声也响了起来,打死了那只小狮子和看护它的母狮子,另三只母狮子看到“国王”、“王后”和“王子”被一锅端掉(姑且这么称呼那三只狮子),立刻逃逸得无影无踪了。

楚云飞三人很纳闷图西人为什么不把那三只狮子也打死,难道他们不怕报复么?却没想过这狩猎队不但了解狮群的真实情况,而且也实在是没能力同时消灭那么多狮子,他们哪里能和万里挑一的中国维和部队相比?

图西人立刻就发现了有人也在打狮子,经过侦察,才知道不是胡图人的大部队,是那三个大名鼎鼎的“中国凶手”。

刘宁笑嘻嘻地向远处走来的图西人招招手,指指地上的狮子尸体,然后指指对方,这动作是个人就能明白:地上的狮子送你们了。在中国人想来,狮子肉虽说不太好吃,但总归是肉,而且狮子皮等东西也能卖点钱的。这应该能表达自己这方的善意了吧?

没想到回答中国人好意的竟然是密集的子弹,还好“小队长三人组”有过类似教训,都在小心戒备着,三人马上利落的滚倒,成树国和刘宁同时开口,“打这帮孙子。”而楚云飞的子弹已经开始发射了。

图西人为什么开枪?三人已经懒得去想了,年轻人本来就容易偏激,再加上所受的不公正待遇,所以愤懑的心里自然想的是:找死,那成全你们!

战斗中三人也很默契,虽然没有商量,但都是以杀伤为主,没有想去消灭对方。然而打仗毕竟不是玩过家家,子弹不长眼的,就算小队长们千小心万小心还是杀死了对方四人,其中有三人是成树国干的。

图西狩猎队比想象中的要顽强得多,二十人打得直到剩下两人才结束了战斗。那两人自然是溜了,可地上躺着十几个伤号呢,而且由于八一式自动步枪也是7.62毫米口径的,杀伤力极大,伤员的伤势应该都很严重。

拨开挡路的杂草,三人一边小心翼翼地前进,一边大喊,“把枪交出来,饶你们不死。”可对方除了伤员痛苦的呻吟,居然没什么别的反应。

走着走着,楚云飞忽然感到一阵的毛骨悚然:危险!来不及多说,一个侧扑,带着身旁的成树国一起扑倒,还没顾得上解释,就看到一枚手雷自天而降,“轰”地炸了开来。

二人掉头看看刘宁,还好他也没受伤,小队长们真的不干了:操,玩阴的?

当下三人分做两组,成树国、刘宁一组点射继续杀伤对手,楚云飞特种作战的水平高,一个人摸上去杀伤对手。

十分钟内,两个图西伤号被成、刘二人的点射击毙,又有两人被飘忽不定的楚云飞点杀,图西人终于忍受不住了,“我们投降,我们投降。”

楚云飞“呸”了一口,“操,真是不到黄河不死心,真是犯贱。”

刘宁命令那些伤号出来到开阔地集合,否则“杀无赦”,图西人站出来两个,等了一阵,看看中国人没有马上杀害他们的迹象,剩下的伤号也相互搀扶着走了出来,枪支和手雷什么的都老老实实地放到了一边。

最前面站的图西人年纪大点,成树国走上去就是一枪托,“操,我们又不是胡图人,为什么开枪?说!”

\section{第五十五章 愚蠢的角马}

那图西人一手捂着肩膀处的伤口,另一只手向上举着一动不动,眼睁睁地看着枪托砸中自己,却不敢露出任何的不愉神色,任另一个膀子剧痛再起,惟有咬牙承受。

已经缴枪了,那就只能任打任杀,这是图西族和胡图族冲突多年认识到的铁的律条:枪不能随便放下,一旦放下,就越老实越好,那样起码理论上还有活下去的可能。虽然眼前这几人不是胡图人,但是图西人想想是自己这方要杀死对方在先,那对方自然就有权力屠戮自己。至于“缴枪不杀”这种骗人的把戏绝对是没人肯去相信的,但总归还是给了图西人一点理论上的希望。

成树国自然不知道图西人已经放弃了对生的留恋,看到对方默不作声,火气更大了,抡起枪托照着对方脑袋又是狠狠地一下,登时图西人的头上就开了个一寸多长的口子,鲜血泉涌而出,“操,说话!”

那图西人想的是既然死都要死了,自然是来得痛快点的好,但对方现在这架势恐怕打的是虐杀的主意。既然这样,那还是配合点,少受点活罪的好。要是回答好问题,没准还真的能捡回条小命,尽管答案惹怒对方的可能性更大。

“大人,政府在缉拿你们,虽然钱不多,但有总比没有强呀,至于你们给我们狮子,杀了大人们我们连你们的枪都能拿到,不是比只拿两只狮子划算得多么?”

成树国虽然心理上有所准备,但还是被对方这话气得七窍生烟,“操,这就叫人心没尽啊!你们知道什么叫知恩图报么?”可气归气,他也没有因此而迁怒对方,而是指指对方,“你、你、你,还有你,一人一个,互相包扎一下。”

那几个图西人面面相觑,互相包扎?那看来真的是放过自己了?马上就有个胆大的来得寸进尺了,“大人,我们……我们没有药啊,你们中国人有药,还经常给我们看病呢,把药给我们点吧?”

这倒是实话,其实别说坎塔卡,就是刚卡首都摩沙都没有一家医院的治疗条件赶得上驻地卫生队的水平,连个疟疾都治不了。成树国刚想答应下来,楚云飞不干了,“活该,你们自己找的,不是你们这些渣滓,卫生队能走么?现在还想要药?做梦去吧。”

其实小队长们的小仓库里药品还是不少的,但在楚云飞看来,对于妄图通过杀人来提高生活水平的人,还是少给他们点仁慈的好,而且想杀的居然还是自己,再给他们药那不是太迂腐了么?再说话又说回来,谁知道自己三人要靠这点药支持多久?

图西人也不敢申辩,相互间惶惶然包扎完毕,其间有俩人因为过于疼痛或者失血过多已经晕过去了。

楚云飞绷着脸,“刚才你们逃走了几个人?”

几个图西人对望一眼,“逃走两个人。”

“哦,”楚云飞沉吟一下,“那援兵什么时候能来?会不会有政府军?”

“援兵?”几个图西人对视一眼,苦笑了起来,回话的依旧是那个被成树国砸了两枪托的人,“援兵是不可能有了,更不可能喊政府军来。政府军要来,会把咱们都杀了的,然后说我们和大人们勾结,那样能多领几个赏钱的。”

楚云飞真的无语了,人命被如此的漠视,奴颜卑乞只是为了简单地苟延残喘,血淋淋的事实无情地展现出了人性的贪婪,当然还有赤裸裸的无奈和麻木,这一切……仅仅是因为落后么?

谁说中国最大的问题是农民问题?相对这些非洲人来说,中国的农民那都是大学本科以上的水准了。

当然,怜悯归怜悯,震撼归震撼,这丝毫影响不了楚云飞冷静地思维。事实上,刚才楚云飞在近距离内一枪打飞一个图西人半个脑袋,而他面对飞溅的脑浆和喷涌的血液,居然没有丝毫的恶心和罪恶感,那时候楚云飞就在想:自己什么时候变得如此冷酷了?

逆境中,人的成长速度会快得不可想象,自然,前提是当局者必须有足够的心理承受能力。

楚云飞不是个爱发呆的人,他没有沉溺在思索中,而是马上提出了下一个问题,“那你们族人就不管你们了么?”

那图西人思维倒也算清晰,“他们可以去别的部落求救,不过那需要太长时间了,还要交礼物,这个不太可能。估计就是晚上族里会来些人,把我们的尸体抬回去,关键是我们手里还有武器,为了武器他们也得来。”

刘宁一直在旁边想着什么,听到这里插话了,“你们族里人来收尸体,会开着车来么?”

那图西人听到这话,不敢随意答腔,他实在搞不懂刘宁问这话的意思。

其实刘宁这么问,只是想看能不能从他们手里弄到车,有车的话转移物资自然就方便多了,那样他们就可以顺利地离开这个是非之地,转移到个不那么引人注目的地方。

可那图西人不这么认为,在他看来,汽车实在是个很奢侈、非常奢侈的物品。以眼前这几个人的能力,抢夺辆汽车是件很容易的事情,他们是想从我们手里抢车么?可部落里就那一辆车,酋长是不会用他的宝贝来搬运死人的。要说族里不会来车,那自己这几个人是不是就再没有存在的价值了?

他在这里盘盘算算,成树国可不乐意了,日了,刚才差点炸死老子的手雷就是你们扔的,现在问个问题你们都不说?这样想着,成树国拎着枪就走过来了。

有个图西人被成树国吓住了,马上大声喊,“不会的,不会的!族里只有一辆车是酋长的,来找我们的人是不会开车来的!”

成树国眼珠一转就知道那个总说话的图西人为什么不回答了,愤懑之心未去,恶作剧的心情又起,“喀啦”一拉枪栓,做出个狰狞的表情,“哈哈,那留你们实在是没什么价值了。”

所有的图西人都吓住了,实在是“从容就义易,慷慨赴死难”,先前本来是存了必死的心的,没想到死不了啦,这欣喜劲还没过去,居然又要马上被枪杀了,这人生的起起落落……未免太快了点吧?

只有那象是头领的图西人,抱着自己的脑袋,不停地喃喃自语,“我就知道是这样的,我就知道是这样的,天哪,他说的和我想的一模一样,连单词都没错一个,你们,你们这帮愚蠢的角马!”

\section{第五十六章 有了新转机}

楚云飞发现成树国有变得发狂的趋向,“不是吧?这么恶劣的玩笑?”走上前去,一脚踹倒那个还在叨叨的图西人,“滚,垃圾,杀你们都浪费子弹。”

成树国还没玩过瘾,舔舔嘴唇,“那好吧,我拿刀子慢慢放完他们的血。”

楚云飞实在懒得看成树国那副故做狰狞的样子,“拉倒吧,这几个垃圾不好玩,想玩咱们可以找些好玩的去,这么下去,你小心心理变态吧。”

成树国脸色一变,楚云飞这话真说到他心里去了,三人在这里孤魂野鬼的日子过了一个多月了,实在是够无聊的,而且头上还戴着一个沉重的大帽子,持续下去别真把人憋出什么毛病吧?想到这里,玩笑也懒得开了,“你们,都给我滚开,你、还有你,你俩伤不重,从地上给我爬走,听好,是爬!要是敢起身,你们就死定了。”

刘宁沉默地看着这一切,看到图西人渐次地离开,犹豫半天还是喊了一声,“等等。”

那些图西人非但没等等,反而加快了脚步,地上爬的也站起来就跑,四散逃开,刘宁摇摇头,叹口气,一枪打在一个伤了胳膊的图西人前方不足半米处,尘土扬起,其他人象中了箭的兔子一样没命地跑开了,只有那位站在那里掉过头来,一动也不敢动。

刘宁招招手,那位哆里哆嗦地走了过来,刘宁冲着缴获的武器一撅嘴,“那里,我会给你们留两支枪的,还有三只狮子没死呢。”

那图西人眼泪当时就出来了,也不知道是感激的眼泪,还是死里逃生的惊喜,反正整个人似乎都陷入了一种亚疯狂的境界,双膝一软就跪下来泣不成声,“谢谢,谢谢大人们,谢谢大人们,谢谢大人们。”

成树国心里更烦了,上去就要打人,楚云飞拦住了他,这个人连惊带吓的,吃了不少的苦了,看现在这个光景,威逼再加上点利诱,应该是不难收服的吧?

等到那人情绪稳定点了,楚云飞走上前,“想不想挣钱?多多的钱?”

那人刚从惊喜中安定下来,听楚云飞这么一问,肺部的氧气似乎又有点不够用了,着急得拼命摇头,等了半天,才说出话来,“想、想、非常想、多多的钱。”

“那好,”楚云飞点点头,顺手从作战服里拿出一小瓶矿泉水,“渴了吧,先喝点水。”

图西人哆嗦着接过晶莹剔透的塑料瓶,感觉到鼻子发酸,这是大人们才能享受到的待遇啊,想把这东西收起来,又不敢违背眼前这人的意愿,颤抖着拧开瓶盖,轻轻地抿了一口,好甜,抬头看了楚云飞一眼,又抿了一小口,却是说什么都不喝了。

楚云飞很满意这个图西汉子的表现,起码他眼里的敬畏之色是个人就能看出来。既然这样,楚云飞也懒得理会这图西人把水瓶放来放去的动作,张嘴直奔主题,“我们现在需要辆汽车,哪里能搞到?”

图西人又被吓了一跳,他们要抢酋长的车么?这主意自己哪里敢乱出?

不过人总归是智慧动物的,聪明和愚蠢之间是没有太大的鸿沟,这不,图西人脑瓜一转,就有了条“祸水东移”的主意,虽然他肯定是不知道这个中国成语的,“我们部落旁边,有个胡图人的部落,他们,他们有汽车。”

“哦,”楚云飞差点被他逗得笑起来,“是么?你们放着自己的汽车不说,让我去打胡图人汽车的主意?你当我们都是白痴?”

那图西哥们当时就急了,指手画脚地说了半天,大家才明白,原来那个胡图部落比他们部落大点,前几年两部落发生过些小冲突,胡图人占了上风,在地域里占了优势后,胡图人就设卡抽税,期间抢劫了好几辆汽车。后来停火协议签定后,他们向政府陆续交还了几辆破损的汽车,现在部落里却还留了三辆,性能都还不错。

三个小队长听了都大感兴趣,三辆?抢他一辆就行了,刘宁就细问了起来,“这三辆车都是敞棚吉普车么?”

图西人开始没听懂是什么意思,明白过来以后马上点头,“不,有一辆是卡车,他们最喜欢卡车,因为能拉很多人,打起架来方便。”

成树国懒得再继续问了,直接说起了重点,“把他们车出来的时间和地点告诉我们,我们抢到车就给你钱。”

那图西人却没想到这些人打的是这个主意,又被吓了一跳,“要我出卖情报?那我不是成了奸细了?奸细可是三族大敌,会死全家的。”停顿了一下,他也发现了不对劲的地方,“哦,忘记了,现在独立战争早过去了,没有奸细这个说法了,不过我真不知道他们的汽车什么时候出来呀。”

楚云飞又发问了,“你们两个部落最近还发生过冲突没有?”

“没有,早停火了谁还愿意打仗?再说,有你们维和部队在,想打也打不起来呀。”

这么一来二去一问,三人才知道那胡图人也很宝贝他们那三辆车,等闲是不拿出来使用的,刚卡人实在太穷,经常使用汽车不但要保养,光汽油费对他们来说也实在是不小的开支。

于是楚云飞就有了挑拨两族再发生冲突,他们潜藏暗处,趁机偷袭运兵卡车的想法。可再细一打听,眼前这族图西人竟是贪图眼前的安逸,确实没有什么再打仗的兴趣。再说眼前这帮刚刚被自己人重创,不但人员伤亡大,而且实在很难说会不会对自己这方有什么芥蒂,还是死了这条心吧,刚卡人不让人放心的地方实在太多了。

前些日子,总是三个小队长相互做伴,为避风头见到外人他们也是远远就躲开了。既然少了和外界的接触,自然也没办法做下步的打算,所以楚云飞从没有仔细地计划过如何转移他们的营地,可现在被眼前这图西人勾起了心思,竟然再也不能拦住向往自由生活的渴望,尽管回带来太多的危险。

“那他们胡图人去坎塔卡也不用开车的么?”

图西人脸有点发白,一副摇摇欲坠地样子,楚云飞一看,却是他胳膊的伤处不断地冒出血来,成树国马上也注意到了,麻利地拿出个战地急救包,把那人包伤口的破衣服扯掉,重新包扎了一下,开玩笑,难得眼前有这么个能打听点东西的活口,可不能让他出什么意外。

\section{第五十七章 关于初夜权}

还好那图西人只是暂时失血过多和过快,等包扎好后,又坐到地上歇了歇就缓过了点劲,不过他的胳膊是穿透伤,疼痛那是肯定难免的。

远处四散逃逸的图西人没有听到想象中的枪声和喊叫,在远处聚集起来张头张脑,却是不敢过来救护那昏倒的两个族人。

看到那图西人被中国人重新包扎,众图西人不但不敢相信自己的眼睛,而自己身上的伤口也变得越发地疼痛起来。其中有位大腿部受伤的汉子,看着自己包扎的扭七扭八的伤口,就有想下去求对方救治的心思,可却被众人死活劝住,“你以为还能象刚才跑那么快么?还是看看噶达会怎么样吧。”

那个图西人口中的“噶达”又嚼了块成树国递给他的压缩饼干,精神逐渐地恢复了过来,“大人刚才问我什么?”

楚云飞又重复了一遍问题,“他们胡图人去坎塔卡也不用开车的么?”

噶达想了半天,“嗯,很少开车,他们只有酋长一家用偶尔用那两辆……那两辆吉普车出门,卡车出来就是打架或者打猎,对了,还有就是保镖。”

“保镖,”成树国一上心,就抓住了重点,“做谁的保镖?”

噶达看到成树国说话,身子不由自主地抖了下,“做尊贵客人的保镖啊,象其他的酋长们,坎塔卡的官员们,还有商人们。”

“商人?”刘宁和楚云飞异口同声地问。

“对,商人,”噶达看到刘宁张嘴了,又开始哀求起来,没办法,谁叫刘宁看起来是最仁慈,最好说话的呢?“卖东西的那种。还有,大人,能让我们族人来救护那两个晕倒的么?”

“嗯,叫他们过来吧,”刘宁又拿出两个急救包来,“看谁需要给谁用,会用不会?”

噶达千恩万谢地接过急救包,“会,多西在部队里学的过救人的。”说完,走到那俩晕倒的图西人面前把急救包放下,“你们快来,把寇卡和笛乐救一下,再看住他们,别让毒蜥蜴来咬死他们。”

楚云飞和成树国戒备地看着躲躲藏藏地走来的图西众人,剩下刘宁继续问那个噶达,“那这么说如果他们知道我们在这里的话,也会开车来和我们打仗么?”

这种情况噶达可是不敢保证,他犹豫地嗫嚅着,“这我可不敢乱说,既然已经和平了,他们也应该不是很喜欢打仗吧?不过大人们的赏金不高,他们不会为那点钱来,除非……除非……”他“除非”了半天也没“除非”出个所以然来,却是忽然想起了一件事,马上又跪倒在地上,“大人们饶命,我们绝对不会把大人们在这里的消息传出去的。”

看着吓作一摊的图西人,刘宁摇摇头,叹了口气,灭口的重要性自己也知道,就别说那俩玩特种专业的战友了。不过,他们三人在附近藏身的事应该有人能猜到的,灭口实在是没有必要的,何况还跑了俩;如果这一队人全死在这里,激起民愤可就不好收拾了;再说,要逃离这里或者生存下去,不和外界打交道也是不行的。

“起来吧,你们说出去我们也不怕,你看我们象是怕事的么?”

图西人拿起急救包,给一个晕倒的包扎起来,看看另一个伤势严重,又由于此人在族里地位不高,就把急救包用在了一个清醒的人身上,成树国看见了,虽然有点不高兴,却是实在懒得去理会。

楚云飞虽然在戒备着,可耳朵并没有闲着,听到刘宁问来问去,总是得不到象样的线索,就顺口问了一句,“什么样的商人值得让胡图人护送?”

噶达想了一下,“有卖药的,还有卖武器的,以前还有很多买俘虏的,现在不打仗,没俘虏了,来得也少多了。”

“哦?”楚云飞他们都知道这里有军火贩子活动,但没想到现在还有这种人,“卖武器的商人多么,是白人么?”

“卖武器的很多,有的便宜,有的贵,有白人也有黑人,不过他们都是从阿拉伯人手里弄到的。”

小队长们一下来了精神,和军火贩子挂上钩的话,是不是意味着能从走私的渠道离开刚卡?可追问下来,那个噶达知道的实在太少,他就知道军火贩子们都在坎塔卡居住,其他的情况就再说不出什么了。

那是不是在坎塔卡附近能抢到汽车呢?楚云飞又开始琢磨了,虽然他在那个城市执行过多次的特殊任务,不过都是跟随车队进去的,在市郊的时候实在不多,虽然坎塔卡城里汽车不多见,但绝对数量还是可观的,就是不知道这出城的汽车多不多。不过这话实在是不方便问图西人的,这家伙的可靠度实在值得怀疑,万一他一个嘴不严,那里可是驻扎着一个政府军的整团呢。楚云飞就算偶尔会嚣张点,但绝对没有疯狂到不把那一团士兵放在眼里。

看到实在问不出什么了,楚云飞“咯啦”一下把枪重新上了下膛,一颗油汪汪的子弹跳了出来,也懒得看远处由此鸡飞狗跳的图西人,“你确定你没有说谎么?”

那噶达又慌乱起来,“我怎么敢欺骗大人们?我不想死啊,我家里还有老婆孩子呢。”

晕,怎么这刚卡人求饶命的理由和中国人差不多呢?成树国在旁边冷笑一声,“是么?你有没有八十岁的母亲?”

“是的,是的,”噶达不住地分辨,“我老婆很漂亮,孩子也很可爱,对了,我老婆的初夜还是酋长大人的呢,我的母亲?我的母亲在二十年前就被食人族抓走了。”

“初夜权?”刘宁皱皱眉头,他倒是听高建军说过这里有的部落还存在这种现象,“你们酋长好可恶。”

噶达却是很不解,“酋长大人很好啊,怎么可恶了?”

刘宁懒得和他多说,“在部落里享有初夜权还不可恶?这里是十美圆,你先拿着,回头有什么汽车的消息来这里找我们,我们会多给你钱的。”

那噶达千恩万谢地接过钱藏起来,嘴上却是有点不服气,“酋长大人不喜欢拿走别人的初夜的。”

成树国接话了,“那为什么拿走你老婆的初夜?”

噶达脸上露出自豪的神情,“那是因为我老婆的确很漂亮,所以我很感激酋长大人拿走了我老婆的初夜,大家都很羡慕我的。”

小队长们面面相觑:这个事情居然可以这么理解?刚卡确实……确实是充满了异域风情。

\section{第五十八章 叛国引发的风波}

图西狩猎队带的枪支不少,有AK47、也有M16,还有3个手雷,小队长们对M16没有多大兴趣,因为不但威力小,故障率高,而且这家伙太能吃子弹了,虽然后坐力小便于连发,但三人都属于那种比较彪悍的军人,还是更喜欢与81式相同口径的AK47.于是三人又拿走了4枝AK47和所有7.62毫米的子弹,子弹不多,有两百来发,但也足以补充今天战斗带来的损耗了,手雷自然要全带走,因为小队长们的库房里只有十来颗手雷,属于紧俏物资。

自然狮子还是给图西人留下了,因为小队长们实在不知道那些剩下的狮子再不会回来了,而且没人说得清楚狮子肉到底好吃不好吃,何必去惹那个麻烦?其实,楚云飞已隐隐想到,以祖国人民什么都敢吃,什么都勇于尝试的性格,再加上海外遍布全球的四千万华人足迹,绝对有不少的人已经吃过狮子肉了,既然大家现在还不能断定狮子肉好吃不好吃,只能说明狮子肉不会是特别的美味。

再说,以成树国的逻辑来说,小狮子的肉也许好吃些,但如果会引来它母亲的报复的话,那代价未免是有点太大了。包括图西在内,没人知道小狮子的父母已经全死了。

这天对楚云飞他们来说,虽然不是个很有收获的日子,却点起了大家对外部世界向往的心火,既然悬赏的吸引力还赶不上一支八一自动步枪,那再在这里猫着躲着是不是有点憋屈自己?

于是三人就正式计划起了搬迁这件重大的事情,时隔多日,篝火旁再次响起了小队长们爽朗的笑声。

楚云飞正式地提出了自己的设想:三人完全可以跑到坎塔卡附近,抢上一辆汽车,当地的通讯并不是特别发达,当下能够跑掉的话,基本上不用考虑当地驻军的追击。

成树国完全支持楚云飞的建议,大家都知道,因为他也是执行“报复”行动中较晚撤出的组员,所以,他对坎塔卡这个城市的认识是远远超过刘宁的。

刘宁的建议比较冷血:突袭胡图人的驻地,抢辆汽车,然后顺手绑架两个重要点的胡图人。这样不但可以阻止胡图人的追击,还有免费的仓库搬运工人,而且到达新地点以后,连挖掘库房的工人也有了,当然,物尽其用以后,知道太多秘密的胡图工人是绝对没有继续生存的必要的。

本着少数服从多数的原则,冷血的建议被否定了,并不是说持否定态度的俩特种兵有多么的仁慈,实在是袭击胡图人部落的话,不确定因素实在太多了点。

于是,大家就聚在一起完善楚云飞的建议,这时就要充分考虑“小队长三人组”各人的特长了。

刘宁在三人中年纪是最大的,22岁,脾气也是最火暴的,但由于已经有过几个月的带兵经验,所以长于决断,谋略略微不足。战斗技巧方面:与其它二人相比枪法好,格斗技巧不是很好。

成树国也是22岁,比刘宁小几个月,脾气本来不是很火暴,但由于近来事不顺心,现在他倒算得上最急燥的,谋略和决断能力都不错。战斗技巧方面:比较平衡,手雷扔得准。

楚云飞21岁,和另二人相比,手雷没有成树国扔得准,决断时爱计较,所以有时没刘宁那么果断。

根据各人的特点,大家做了相对比较严谨的计划就开始准备行动了。

===========================================================于此同时,国内的叶美女士也接到了河东省军分区的通知函,上面很遗憾地告诉她:她的儿子楚云飞在非洲维和过程中,目无组织纪律性,犯了极其严重的错误,导致了非常恶劣的后果,而且犯罪嫌疑人最后叛逃部队,目前因叛国罪受到国家无限期的通缉。

叶美女士当时差一点点就要晕过去了,还好外甥也在家中,帮她稳定了下来。等到叶美女士发现通知上除了罪名,所有内容都语焉不详的时候,自然要拉住送通知的上尉问个明白。

那上尉因为部队里出了这样的叛徒,对罪犯母亲的询问很是不耐烦:这事情事关重大,你没资格知道。

陈小军拿过通知看了看,也有点摸不着头脑,“大姨,给咱们这通知是不是说我哥回来以后咱们得报告,不能包庇?”

叶美的眼泪又下来了,“还用得着么?云飞怕是想回都回不来了,那是非洲啊。”

那上尉也觉得有点蹊跷,接过通知来看了下,纳闷得不由得说了出来,“奇怪,真是没说要家属举报啊,什么乱七八糟的事啊?”

事情的终结是来自一封信,楚云飞托李大龙捎的一封信,信是楚云飞在逃离军营的时候为防万一写的,很简洁,除了对母亲的问候,就是一句没头没尾的话——“相信你的儿子,不管可能发生什么事,我都不会让你丢脸的。”这封信虽说是更让叶美担心儿子的处境了,但做母亲的也明白自己的儿子真是没做什么丧心病狂的事,没准……是个误会吧?

楚云飞自然也写信给周琳琳了,同样是李大龙转的,信上除了对情人的一片思念之情,还略微胡说八道了几句,说自己“因为表现优秀,可能会接到部队的秘密任务,所以有可能在以后的日子里不方便继续写信,也有推迟回国的可能性,但对琳琳的思念不可能因为一时沟通不上而有任何的减少,离别只会加重我的思念。”这信比写给叶美的长了十倍还有余,不知道做母亲的知道了会有什么样的想法?

热恋当中的女人心思是非常细腻的,洋洋长篇的情话怎么也遮不住那几句杜撰的疑云。周琳琳仔细品味半天,不得不承认品不出任何“第三者”存在的味道,于是失望之余,美女的心中又多了层对远方恋人淡淡的担忧。

成树国和刘宁的家里根本用不着通知,部队里的内参上早有这事的前因后果了,普通级别的内参同官方的口径是统一的,但那高级内参上可是写得明明白白的,成树国的父亲虽然级别稍微有点欠缺,但谁又没有几个老战友、老上级之类的关系,事关自己的儿子,成解放最终还是了解到了事情的真相。

成解放很欣喜地发现刘群的儿子也在“叛国”之列,于是两人自然地联系了起来,俩老军人异口同声地表示了对自己儿子的失望,但事情也“绝对不能就这么不明不白地算了”。

这件事虽然不为外人所知道,而且有关人员也被下了“禁口令”,但实际上已经引发了很大的理念冲突。大到对外政策、对外部队形象,小到部队精神文明的建设,不同阵营的成员争论不休。所以,成解放和刘群商量的结果就是:风大浪高,做为当事人的亲属,暂时不宜轻举妄动,还是等等再说的好。至于远方的两人的儿子,做父亲的手实在伸不过去,只求那俩……不,那三个年轻人相互支持,同心戮力地闯过那道难关了。

\section{第五十九章 抢劫汽车}

坎塔卡城西三公里处。

从这里开始,公路已经被土路所代替,路两边也没什么建筑物,事实上,出城的车辆早在前面一公里左右的地方开始乱跑,怎么走方便怎么来。

楚云飞三人在这里埋伏已经是第三天了,前两天根本没落单的车辆经过,仅仅有过个六辆车组成的车队在第一天从这里走过,刘宁为了谨慎起见,没有出来拦截,然后就再没任何车从这里出去过了。

进城的车倒是出现过四辆,但小队长们早就商量好了,回城的车有个油料足不足的问题,所以三人也没对那几辆车下手。

可这么一直等下去也不是个办法,虽然三人隐蔽得比较好,但公路总归是公路,偶尔还是有那么两三个行人出现的,这时间一长,还真难说会不会被人发现,所以今天出来三人就商量好了,如果当天还不能得手,那么把埋伏地点再向前推进五百米左右,也就是说明天凌晨又得起个大早来这里选地形了。

眼看中午已过,已经到了接近下午两点,三人就开始犯嘀咕了:今天怕是够戗了。因为本地人大多是一天两顿饭的饮食习惯,这看着就到了吃下午饭的点钟,怕是不会再有人出城了。等到吃完下午饭,就更没有什么出城了。因为停火协议刚执行没多久,路又不好走,所以就算是往各个部落的回程车也总是在早晨出发。

小队长们正在抱怨,老天就象马上听到了他们的唠叨似的,从坎塔卡方向远远开来了两辆吉普车,楚云飞认得前面一辆是悍马,不是敞篷的那种,后面一辆因为扬起的灰尘太大,看不出来。

两辆车,怎么办,抢是不抢?三人正在盘算,那车就已经开到了跟前,虽说因为走上了土路车速有所减缓,但好车毕竟是好车,速度并没有减低多少,时速还是有个七十公里左右的模样。

在那打头的悍马开到距离三人藏身的地方还有八十米左右的地方时,刘宁直接冲到了路上,AK47冲着地面就是个三连发,“哒、哒、哒”三声沉闷的枪声响起,地上腾起小小的三朵土花。

悍马的司机先是一个点刹,看到拦路的人把枪口指向了自己,马上意识到遭遇到了劫匪。看看离对方已经不到五十米,顺手就甩了把方向,自然不是冲着刘宁跳出来的方向,而是向相反的方向冲去。

这点小小的伎俩早在小队长们的计算之中,楚云飞就埋伏在土路的另一侧,当悍马冲到离他最近的距离时,直起身子,象阵旋风一样刮了过去。

悍马的司机简直不敢相信自己的眼睛,他从来没见过跑得这么快的人,发现对方时人还在十米开外,一眨眼对方已经攀住了悍马的倒车镜,跳上了车外的脚踏板。

司机正要一个急转弯甩落对方,却见一个拳头迎面而来,他下意识地一闪,脚下就是个急刹。

等到司机反应过来自己和拳头间还隔着一扇玻璃时,那钢化过的厚厚的汽车玻璃却在瞬间就布满了蛛网,“砰”的一声传来,整个驾驶室里都是飞溅的圆形玻璃碎片。

副驾驶上坐着的是一个精悍的黑人,司机的急刹车没甩落楚云飞,倒把他的脑袋重重地甩到了汽车前挡风玻璃上,等他平衡好身子,要从腰里拔枪的时候,却听到劫匪标准的英语:“停车,看看这是什么?”

既然对方要让自己看,那车里人自然要看看,只有司机不需要看了,他已经知道了:对方把衔在嘴里的手雷拿到了手上,手伸进了车里,还按住了压发引信。

悍马车乖乖地站住了,几个中国人才看清楚,后面跟的那辆居然……居然是北京吉普切诺基,那切诺基上的人不知道发生了什么事,已经有人从车里探出枪来向刘宁的方向开始扫射了。

这一切都是在短短的几十秒内发生的,当北京吉普内的人发现悍马已经停了下来,不知所措的人们正考虑要不要再继续射击的时候,刘宁一侧的成树国已经把手雷扔到了吉普车前,爆炸声中,刘宁一枪就把司机一侧的倒车镜打了个粉碎,“停车!”

切诺基内的人也乖乖地停下了车,车内的人怕流弹飞来,都趴在座位上不敢随意乱动,最多也就是转转脑袋,琢磨一下到底发生了什么事。

伏击打得非常成功,小队长们在短短的两分钟内就制服了两辆车,又用了两分钟把两辆车集合到一起,并把所有人缴械。然后三人押着两辆车,在众多围观者听到枪声到来之前扬长而去。

两辆车里原本就有十个人,其中悍马内四人,切诺基内六人,现在又加了三个人中国人,车内小小的空间实在是拥挤不堪,可在枪口下谁又敢说什么。

车内有俩西瓦人才叫个倒霉,本来是仗着与车主人熟识,才顺路搭车的,结果旁人羡慕的目光还没回味够,就峰回路转,平白地摊上了这档无妄之灾。他俩在车上哆哆嗦嗦地提出下车的建议,并极力申辩自己只是搭车的,大人们实在没必要和可怜的自己一般见识。

这样的哀求肯定不会被允许的,把这两人一放还怎么保密啊?可俩人不断的唠叨实在叫人心烦,成树国因为抢劫成功,心情很是不错,狠狠地一拳击在了切诺基的车门上。他本身的功夫就非常的厉害,又运起刚学了不久的内气,车门登时变形,凹下去碗大一块。俩西瓦人看得目瞪口呆,很自觉地闭嘴不再唠叨。

这十个人原以为只是抢劫,既然还要带上人走,那恐怕就是绑架了,没准还有被灭口的可能性,所以也不是说没人起这反抗的心思。可一前一后两辆车里的人都见识了中国人骇人的功夫,武器又都没收走,实在是心有余而力不足。

很快地,两辆车就被开到了个隐秘的地方,这地方已经离小队长们的库房不远了,算是楚云飞等人选择的一处躲避的场所,四周都是矮矮的栎树,中间是几个小土丘,最里面是个被三人利用地势加工过的小坑,不过坑虽小,把开进来的两辆车开走的话,埋十来号人也足够了。

带来的人太多了,三个小队长碰了下头,觉得还是筛选几个人算了,多余的直接干掉了事,不过审问一下还是有必要的,谁知道这十个人里面有些什么人物,还是操心点的好。

至于观察哨,刘宁自家知道自家事,自己审讯人的水平实在不如这俩小弟,“我去放哨吧,有云飞在我也懒得操心了,你俩觉得谁没用直接干掉算了。”

\section{第六十章 中国留学生}

刘宁话说完了刚想走,还没等他转过身去就听见一声高亢入云的尖叫,“救命啊,救命啊~~~”小队长们下意识地把枪端了起来。

汉语!居然是汉语?三人死死盯住那出声的黑人,成树国直接用汉语问了句,“你怎么会汉语?”

那黑人约有将近四十岁的模样,个子不高,有个1.68米左右,可长得确是着实的富态,从头到脚没有不胖的地方,怎么也有两百斤种,这种体型在刚卡的黑人里非常少见。

他明显地听懂了成树国的问话,可回答起来却是嗑磕巴巴的,“我,我过到……中国,用……留学。”然后还想说点什么,却是张口结舌表达不出了,一着急用上了英语,才说明白来头。

原来这家伙到中国留过学,不过是人都知道能到中国留学的非洲学生总是非富即贵的,楚云飞也听周琳琳说过类似“白人留学生无所谓,黑人最好别招惹”之类的话。这家伙家里条件不错,去中国也没认真地学习,所以几年下来,汉语也就是听听没问题,说说的话那就差太多了,他从中国回来也有几年了,汉语说起来自然就更不利索了。

自从小队长们把他们拉到这里,刚卡人的心里就七上八下地忐忑不安,不知道等待他们的是什么,这家伙也不例外,所以耳朵一直支棱着注意中国人的谈话。虽然他汉语的听力也有所下降,但既然存了心,还是敏感地注意到了“干掉”这个字眼,这还了得?当下就想喊“饶命”,可他的汉语实在是够糟糕的,一着急就喊成了“救命”。

这人叫刚贝拉,不是刚卡人,是索度国多特族一个部落酋长的儿子。因为刚卡已经停火,胡图人也不再因为与图西族的战争而迁怒于其他种族,所以对于多特人来说,刚卡已经安全了很多。

既然安全了很多,那刚卡战后重建的商机就显得非常诱惑人了,索度国本身的经济发展就强于刚卡,多特人的经济意识自然也不会很差,刚贝拉作为本部落被重点培养的人才,就在这时派到了刚卡来考察商机,抢那头啖汤或者说圈地。

这刚贝拉汉语不怎么样,可用英语说起话来抑扬顿挫,滔滔不绝,很是吸引人,刘宁听得哨都不去放了,直接就站土丘上一边随便看看一边听他说话。

楚云飞和成树国也不言语,任他一个人在那里讲,成树国脸上倒还显示出了三分感兴趣的神色,楚云飞却是一直绷着脸没好脸色,枪口始终也没放下。

半个小时过去了,刚贝拉也说得口干舌燥了,看看楚云飞脸上还是没有任何表情,一生气,公子的脾气就上来了,冲着楚云飞来了一句,“你能不能听懂英语?”

楚云飞不冷不淡地用英语回答了他,“你还有什么要说的没有?”

刚贝拉的火气更大了,也顾不得乱说话吓着自己人了,这次不只是冲着楚云飞了,三个小队长全骂进去了,“我好歹也在中国念过几年书的,我印象中的中国人善良、勤劳、乐于助人,也不歧视我们黑人,你们三个算不算是中国人?不但抢我们的车,还要把我们杀死?我们哪里得罪你们了?”

他这句话不要紧,刚卡人顿时就炸了窝了,楚云飞和刘宁立刻开枪弹压,前后左右几声枪响,人群又老实了。

三个小队长交换了一下眼色,迟疑的目光表明三个人都动摇了,是啊,谁能想到在这里能遇到和中国有香火情的索度人呢?三人虽然被判“叛国”,可内心深处对祖国的眷恋又何曾动摇过?眼前这个黑人又触到了三人的痛处。不就是想弄几个免费劳力么?算了,仓库自己挖吧。

看到战友们默契的样子,楚云飞点点头,“我来说。”

楚云飞掉头面对刚贝拉,用流利的英语表示自己的不满,“你冒犯我了,我很生气,不过看在你去过中国的份上,我仁慈地宽恕你的冒失。”

他说完又面向全体黑人,“你们里面肯定是有刚卡人的,那么你们也知道我们是什么人,没错,我们就是让你们刚卡人害得失去祖国的中国人,更重要的是,我们是冤枉的!”

楚云飞转过头去问刚贝拉,“现在我有杀人的理由了吧?以刚卡的规矩来说。”

刚贝拉悻悻地撅撅嘴,还是嘟囔了一句,“可你们是中国人啊。”

“中国人?”楚云飞轻蔑地一笑,“中国人就不会报复人么?中国人就好欺负?”

他不再理会多特人,头又转向了人群,“虽然刚贝拉去过中国,我们也愿意因为这个放你们一条生路,可是我们不会愚蠢得让你们把政府军引来的,所以只好对不起了。”

唯一没举枪的成树国也把枪举了起来。

那些黑人马上就喧哗起来,反抗那是想也不敢想的,只能声嘶力竭地在那里赌咒发誓不会泄露秘密,众多厚厚的嘴唇上下翻飞,身子却是动都不敢动。

看到中国人迟迟不开枪,有门!黑人们的声音自然是越提越高。

“够了,”刘宁发话了,“这次饶了你们,你们去感谢刚贝拉吧,你、你,你们俩跟我来。”既然不能让他们挖库房,腾空库房总不能放过这样的苦力吧?

“算了,”楚云飞拦住了刘宁,上去小声说了一句,“别让他们知道咱们要走了。”

刘宁想了想,确实是,如果对方有人存心报复,多制造个假象也是好的。就象楚云飞刚才一样,明明要放过对方了,还要吓唬一下,虽然刚卡人办事不能用常情来揣度,但是做个姿态总比不做强。

想到这里,刘宁拉上成树国,一人一辆车,走了,没办法,既然要做好人,那只能自己受累了。

楚云飞目无表情地看着眼前的人群,“等天黑了你们就可以走了,现在就剩下我一个了,你们可以尝试着反抗。”

刚贝拉生怕有人经不住这诱惑,惹得对方凶性大发就惨了,那击碎车玻璃的一拳可是他亲眼所见,他马上发话了,“你们别自找死路,要跑能跑过汽车么?连累了大家我要他的好看。”

可这世上的事真的不能以常情来揣度,楚云飞不但要看着俘虏,还要时不时地观察一下四周。他早想好了,真要附近出现什么人的话,那除了刚贝拉,其余的人都要杀掉,要是出现的是政府军,那对刚贝拉也只好说个对不起了。

就在楚云飞又一次拿起望远镜向四处观察时,一个搭顺车的西瓦人实在受不了对自由的渴望了,站起来撒腿就跑,刚卡人自私的本性在此刻暴露无疑:跑的只是我一个,中国人怎么可能追?再说那个能打坏车门的厉害中国人也不在。

可叹的是,他还没跑出五米,一颗子弹无情地打在他面前不到二十厘米的地上,他不得不站住脚,乖乖地回来。

\section{第六十一章 借刀杀人}

刚贝拉真的生气了,既然已经有言在先,那这无疑是对自己权威的赤裸裸的挑衅,实在顾不得以前这家伙怎么巴结自己了,“打断他的腿,让他再跑。”

除了另一个搭车的,其他七个人一拥而上,噼里啪啦就是一顿痛打,看来刚贝拉还是这群人的首领。

等到这群人打完散开的时候,那西瓦人早躺在地上成了一摊泥,腿断没断不好说,气倒断了一小半。

楚云飞看到完事后的众人眼光都向自己瞟来,面无表情地在地上顿了顿枪托,“他是想害死你们大家。”说完,从腰里摸出一把军用匕首,丢到了刚贝拉面前的地上。

刚贝拉本来对楚云飞就没什么好感,亮闪闪的匕首往自己面前一丢,那印象自然就更恶劣了。

谁也不会认为刚才楚云飞那枪是打不到逃跑者,那一枪的警告意味很浓,大家都看得出来这个黄皮肤的士兵并不想伤害那个西瓦人。刚贝拉也是因为自己的尊严受到了挑衅,所以才命令人痛打那家伙一番,可没想到楚云飞脸变得这么快,一下给他来了顶“想害死大家”的帽子。

想害死大家?这个论断其实理论上是正确的,西瓦人的逃跑确实会导致不可估量的一些后果。但谁都没想到看似没心杀人的楚云飞丢把匕首在刚贝拉面前,意思那自然很明白:要刚贝拉自己解决这个给大家带来麻烦的人。

狡猾的中国人!刚贝拉心里一边咒骂一边捡起了匕首,这下西瓦人想恨都恨不到中国人身上了,哎,算了,人在矮墙下,怎能不低头?话又说回来,要是真因为西瓦人逃跑而导致自己这些人被灭口那才划不来。想到这里,刚贝拉对西瓦人的怨恨不由得又增加几分,你的命是命,我们这么多人的命就不是命?人家都答应天黑就放人了,你没事瞎跑个什么?

可该叫谁杀了这个西瓦人?另一个搭车的要不要干掉?这刚贝拉不愧是在中国呆过几年的,本来就有几分谋算策划的能力,又多学了几分中国式的阴险,沉吟一下,他马上就想出个既能讨好楚云飞,又能把所有人都绑到一起的主意,“他差点害死我们大家,来,我们一人给他一刀,别扎要害,能活下来算他命好。我先来,那个库卡,你扎最后一刀。”——库卡,就是另一个搭车的。

楚云飞有点赏识眼前这个家伙了,看来聪明人哪里都有,刚贝拉这么一做,把所有人都绑到了一起,同是凶手,谁举报谁去?同时所有人都交了投名状那他们生存的几率就更大了。至于同谋杀人的理由也很明了——谁也觉得自己的命比别人值钱。

刚贝拉这么一说,那要死的家伙居然又挣扎着爬了起来,不过他能看到的也就是今天的太阳了,众人一顿乱拳将他再次打倒,这次连库卡都出手了。

接下来的事情就很顺利了,西瓦人身上九个刀口像一个泉水群一样“咕嘟咕嘟”地往外冒着鲜红的血水,闻到血腥味的非洲绿蝇铺天盖地拥了过来,这次楚云飞可真有点想吐的感觉了。

远处隐隐有声音传来,因为要看着现场的众人,楚云飞不能趴到地上去听,又拿起望远镜向传来声音的方向望去,却见一辆吉普车带着滚滚的烟尘向这里疾驰而来。

来的是成树国,他和刘宁正在搬东西,却隐约听到了枪声,虽然两人对楚云飞的信心都很足,但要说一点担心都没有那是假的。于是成树国就开车来看看到底发生什么事了。

从车来的方向,楚云飞基本上可以断定来的是自己的战友:肯定是枪声把他们引来的。等到他能看清楚成树国的相貌时车已经停了下来,楚云飞向他挥挥手示意这里不需要他。成树国拿起枪向天上两个二连射,然后吉普车又带着漫天的烟尘疾驰而去,楚云飞知道,那枪声该是像《平原游击队》里的梆子声一样,在向刘宁表示“平安无事”吧?

看到中国人配合得如此默契,众“肉票”相互看看,彻底地死了侥幸的心思,刚贝拉也慢慢地挪着身子,想退到人群里,他实在不想给楚云飞任何可能误会他的机会。

他们那点心思哪里能瞒过楚云飞?他一反常态地微笑了起来,指指刚贝拉,“你,不用再退了,陪我聊聊。”

听到这话,刚贝拉自然是不敢再退了。

楚云飞笑嘻嘻地问刚贝拉,“你来刚卡做什么生意啊?”

不知道为什么,刚贝拉看到楚云飞微笑的样反而子更觉得害怕,从这开心的笑容里实在想象不到笑脸的主人刚刚做完个“借刀杀人”的游戏,是的,真正的“借刀杀人”。

强忍着心里的恐惧感受,刚贝拉哆哆嗦嗦地开始回答,“我……我是来这里看看铝矿石的,就是铝矾土,刚卡的铝矾土很多,我想登记一下矿的分布和品质,然后回国分析分析,看看哪里合适我们收购。”

楚云飞点点头,“铝矾土”这个单词很冷僻,还好他的词汇量很大,能听得懂,“你是要倒卖铝矾土还是要把它加工成氧化铝呢?”

刚贝拉没想到楚云飞也懂这个,说起专业,话也利索多了,“加工成氧化铝,那个设备很费钱的;不过要是长途运输铝矾土,成本也很贵,要是能占上合适的矿,我还是想先弄点钱建个氧化铝厂。”

听到这里,楚云飞已经没有继续听下去的兴趣了,不过左右是没事,多了解点情况总不是坏事。

刚贝拉还在滔滔不绝地继续说,“等到我的铝厂建起来,也可以收别人的矿石,低买高卖。这样下去,过不了几年,我就可以再弄个厂子来电解生产铝锭了,那时索度的电力供应应该完善很多了,钱还不是随便我赚?”

说着说着刚贝拉突然一个冷颤,不敢继续说了,老天,再说下去,中国人该有绑架自己的心思了吧?

楚云飞关心的可不是这个,“别怕,你好歹也在中国留过学,人不亲土还亲呢,我不会绑架你的。对了,你想在索度建厂?不是在刚卡?”

刚贝拉点点头,“谁知道刚卡这破地方能和平多久?我可没那么愚蠢,将来厂子建好以后倒是可以雇点刚卡人,他们便宜,我还可以……”

楚云飞懒得听他唠叨,直接打断了他的话,“那你铝矾土怎么运过国境?”

刚贝拉早就算过这个帐,“边境那么大,哪里不能过?将来刚卡就算再打起来,只要有人想卖铝矾土,自然还是要找我,有势力的就能过边境,我有什么可担心的?”

楚云飞用眼角瞟瞟其他人,探过身子低声问刚贝拉,“要是把我们三个弄过国境,你想得到什么样的报酬?”

刚贝拉被楚云飞探过来的身子吓了一跳,听清楚楚云飞的问话,又吓了一跳,居然当场就愣住了。

\section{第六十二章 恐怖的惯性}

刚贝拉现在的心情实在是……五味杂陈。

首先可以确定,目前“人为刀俎,我为鱼肉”,话是不能乱说的,要尽量说实话,有技巧地说。

其次,这事实在没办法答应,起码眼下是不行的,可不行的话,又该怎么拒绝呢?

再想想的话,倒是可以把他们骗到关卡处然后再出卖,他们强到头也就三个人,难道还有飞天遁地的本事不成。不过这个……要是万一逃脱个把人?那自己的危险可就大了去啦,所得远大于所失,不划算。再说他们又都是那种有功夫的,看刚才借刀杀人的残忍劲头,也不象自己以前的中国同学那么宽容。

不过真要能帮助他们一下的话,这就意味着这三人拥有的强大武力能被自己所使用,那将来不管是对内还是对外,都可以成为自己手里相当有威力的牌,做明牌有利于提高自己的震慑力,做暗牌那绝对是阴人的好手段。

这么左思右想着,刚贝拉居然在那里发起了呆。

楚云飞知道他是在那里天人交战,权衡利弊,虽然自己能够理解他的心情,可警惕的心不由得又多加了几分。

等刚贝拉稍微回过点神来,看到楚云飞在那里似笑非笑地看着他,猛然泛起一种赤身裸体的感觉,也不敢再继续考虑得失了,战兢兢地实话实说,“这个事情……”回头看看自己这方人,把嘴凑到楚云飞耳朵前小声说,“我这里还有三个刚卡人,他们该都是知道你们的吧?”

楚云飞点点头,“嗯,我想是这样的。”

刚贝拉又解释了,“有一个是外人(库卡)咱们倒不用管他,不过另外两个突然失踪的话,我也不好交待我的合作伙伴,要让他们忘记今天的事好说,可要带你们过国境,我怕路上万一这俩家伙使坏就不好了。”

楚云飞笑了,这次可真是开心地笑了,刚贝拉的拒绝正说明他没有阴人的打算,自己也算没白放他一马,“那错过今天,你以后能不能想办法把我们弄出国境?”

刚贝拉毫不犹豫地摇头,“可以,没问题的,其实国境上没人看管的地段很多,只是你们不熟悉就是了。”

楚云飞知道他说的是实话,哪个国家的国境不是这样呢?“那你能和我们约定个日子么?对了,你还没说你想要什么报酬。”

利益在前,刚贝拉忘记了害怕,开始表现得象个商人了,不过他还是需要先向楚云飞解释一下,“我这次在刚卡还要呆三天,车你们也拿走了,估计又要多呆两天,这次是绝对不方便带你们走了,下次来我也说不定是什么日子,不敢跟你们保证的。”

听到这里,楚云飞淡淡地说了一句,“要是我们能离开刚卡,要你的车有什么用?自然还是要还你的。”

听到这话刚贝拉更高兴了,“那谢谢你们了,我自己的人虽然也有认识路的,可实在不方便留下来,也不敢保证他们会不会使坏,再说还有刚卡人看着呢,下次吧,冲你的这份情谊,我保证把你们带出去。”

楚云飞笑了笑,对方越这么谨慎,证明可信程度越高,虽然不排除这家伙目前只是为了脱身采取的权宜之计,起码战友和自己的安全系数并没有因此而降低,这就足够了,“先谢谢了,不知道你想得到什么?”

刚贝拉也知道自己的回答得到了对方的认可,于是站在那里认真地想了起来,可急切之间脑子有点混乱,判断不出来要求个什么对自己最有利,还是等等再说吧,于是非常漂亮地回答,“中国是我的第二故乡,帮助中国人是我该做的,不过你说了三次了,那我也不客气了,你们将来能帮我做件事就行。”

楚云飞笑了笑,“好吧,难得你对中国人的一片盛情,”,抬头看看天,大声说,“好了,天快黑了,你们走吧。”

众多黑人就要掉头而去,刚贝拉却喊道,“等等,我话没说完呢。”然后又掉头小声对楚云飞说,“那我下次来怎么联系你们呢?”

这就对了,楚云飞心里说,不过他脸上倒没表现出什么来,还是一脸的平静,“哦,你说说你的建议吧。”

刚贝拉不由得对楚云飞略微地轻视了起来,这就是聪明的中国人?不说好下次怎么见面就放人?头脑不太够用吧?他可没想到这是楚云飞的又一次试探,聪明人最爱犯的错误就是偶尔会觉得自己是最聪明的。

“这么着吧,”刚贝拉又沉思了片刻,趴到楚云飞耳朵上说起来,意思大致是还在这个地方,来的人持他写的信来取信于楚云飞,至于刚贝拉自己那是无论如何不愿意再冒这个险了。

楚云飞仔细地想了想,还是没有答应刚贝拉的建议。他选了个离计划中新营地不远的地方,要不新营地离这里实在是远了点,别的不说,只说汽油用完了都没地方去加。而且那新营地还在约定地点和坎塔卡中间,政府军有个风吹草动也便于发现。现在自己这方只有三个人,谨慎点是没有坏处的。

刚贝拉自然要纳闷一下跑那么远做什么,不过自由在即,实在是不愿意再多出点什么差池,于是匆匆答应了。

看看众人要走,楚云飞指指地上的那位,“我看他已经没气了,你们怎么说也算是同伴,他的错误已经受到了惩罚,是不是该帮他收拾收拾,把人埋了?”

那些人你看看我,我看看你,心里想的都是“没工具,我们怎么挖坑埋人?”,不过想归想,却也没人说出来。因为不但楚云飞提的建议很有人情味,而且这个紧要关头没人愿意给自己找不自在。

于是众人用手、皮带头、皮鞋、树枝之类五花八门的工具在地上扒了个浅浅的坑,把人抬了进去,再把挖出来的土盖上,也就仅仅将就能把人掩住而已,只是地上似乎多了个小土丘。

楚云飞自然不会因为这个和他们计较,他也知道,这里的地除了上面的一层浮土,下面的土质还是比较硬的,不太好挖呢。

刚贝拉正要跟楚云飞道别,却见对方走过来拍拍自己肩膀,“我们会为你做三件事,只要你能把我们弄出国境。”

增加了两件?那好啊,刚贝拉下意识地摇摇头表示赞同,不对!他猛然反应了过来,刚才我们要走他怎么没这么说?仔细看看对方那莫测高深的笑容,他越想越觉得不对劲,越想越可怕,一瞬间从头到脚一阵发麻。

想想自己居然还怀疑过对方的智商,刚贝拉恨不得狠狠地给自己一耳光,还好自己是真的想帮助他们,要不很难想象等待自己的会是什么……

其实,刚贝拉早要走也就走了,楚云飞最多也就是失望而已,他再聪明也没想到自己给对方的印象会是如此地恐怖,不过还好事情的结局总还是不错的。

\section{第六十三章 建造新营地}

楚云飞他们的新营地离原来的藏身之地有将近六十公里,中间还夹杂着十几个大大小小的各个种族的部落,再加上各个部落有自己的种植地和默认范围,所以转移仓库并不是那么顺利的事,还好有两辆吉普车,而且悍马的载重量很大,足够一次性把不多的物资拉走了。

这里已经接近一条不知名的小山脉,楚云飞翻了下地图,这山脉意译的意思大致是“守护者”,只是不知道这高不过三百来米的岩石带守护着什么?

选择地点也很费了小队长们一番劲,合适的地方不是少,而是太多了,刘宁最终选择了一块林木相当茂盛的土丘做为三人的新家。他正在同成树国评价这个地方,楚云飞也回来了。

楚云飞也找了个类似的地方,不过那里的林木远没有这里茂盛。可最关键的是距离那个土丘500米处的岩石下有块很古怪的地方,以成树国和楚云飞接受的特种兵常识教育,一眼就可以断定地下不深处就有水源。

照常情讲,生长着香蒲、沙柳、马莲、金针(也称黄花)、木芥的地方,水位比较高,且水质也好;生长着灰菜、蓬篙、沙里旺的地方,也有地下水,但水质不好,有苦味或涩味,或带铁锈。不过非洲有没有类似的植物还真的不好说。

当然水源还有其他寻找方式,有一条就绝对适合形容这里:有羊齿类和苔藓类植物的地方,附近会有水源。

成树国这次可真的是服气了楚云飞了:能分析出下面有水源这很正常,毕竟大家都接受过训练的,不过匆忙之中,在偌大一片地方找寻出来这么个小地方——而且很有可能是唯一的水源,要不早该被当地人发现了。如此毒辣的眼力是怎么培养出来的?

“怎么找到这里的?”

楚云飞用食指捋捋自己的鼻子,这是他无聊时的一个下意识动作,似乎是来非洲才养成的,据他说这样能保持眼睛的湿润,虽然可能难受了点,“没什么,有种直觉,这里让我感到清凉,而且忽然想洗澡了,就认真找了找。”

找到个好地方,大家以后不用再继续用塑料袋收集水了,也不用满世界地去寻找那腥涩的“露露”和仙人掌了,成树国的心情自然会很好,“你小子就宣传迷信吧,不但要练那个什么鸟功,居然还有成为神棍的天赋啊。”

于是新营地就设在了楚云飞选定的地方。虽然林木不多,但够藏人就可以了,植物生长太茂盛的地方也同样会是人们注意的焦点,从这点上讲,林木少点反而安全系数没准会大点。

土丘不高,还不到三米,说是个小坡也不为过,不过那土质可是比较糟糕,虽然松软,但是里面夹杂着大量的石块,很是难挖。而小队长们只有一把不大的工兵锹,三个人轮流上阵,挖了足足有三天,才清理出来个能够睡觉和贮藏东西的地方,而那把铁锹因为小伙子们玩命的劳动磨损已经得相当厉害了。

接下来的日子里,在岩石下方不到一米处发现了渗水的石层,虽说每天出的水不是很多,但足够三人每天洗两回澡了,位置离营地也不远不近刚刚好,于是小队长们加了把劲,顺势在那里挖了个口小肚子大的小水窖,自然出口要严密伪装的,不仅仅是防人,连动物也要防。

不过那出的水有些怪异,无机盐含量高味道有点咸,虽然喝起来口感不是很好,但那属于岩石层内的藏水,所以刘宁和成书国觉得还算正常,起码比那些植物的汁液好喝多了。

但对楚云飞来说,绝对是种非常怪异的感觉:这水居然有助于他自身修炼!自从开始喝这水,楚云飞突破了困绕他很长时间的瓶颈状态,如果愿意,那种“伪先天境界”的境地会很容易地达到,成功率百分之百。而且日复一日下来,他甚至能明显地感觉到那“先天境界”的威力和气势在大幅提升,这样下去,会不会很快地臻达真正的“先天境界”?

对于楚云飞偶尔显露出来的气劲,俩同伴已经见怪不怪了,谁知道这家伙还藏着多少不被人知的秘密?不过,想归想,两人也在辛勤地修炼楚云飞传给他们的无名气功,也许是因为心无旁骛的原因,刘宁的进境要快于成树国。

楚云飞自然要怀疑这石层渗出的水是不是有什么奇特的效果,也许是传说中的“灵石钟乳”或者别的什么的?不过那俩徒弟很明确地告诉师傅:他们没觉得这水有于炼气,就算有也没那么明显!

出于对楚云飞智力的信任,小队长们眼巴巴地等着刚贝拉来人。等待的日子里,闲着也是闲着,在熟悉环境的同时,三人又多挖了两个小仓库和隐藏汽车的地方,安全措施做得足点总是没错的。

新一年的春节就在小队长们进行土木工程建设时到来了,三个游子在对亲人的思念中捱过了整个正月,遗憾的是,仓库里的物资中并没有用来麻醉人的酒类饮料,含乙醇的只有消毒用的酒精。

日子一天天地过去了,而做出承诺的多特人并没有出现,小队长们实在没有耐心再继续等下去了,最关键的是:盐也不多了,而他们是不能出去采购生活必须品的。

“该出去和本地人接触一下了,”成树国懒洋洋地斜靠着草堆对刘宁说,“再这么下去非憋出毛病不可。”

刘宁点点头,“嗯,盐也不多了,早让你腌肉的时候少放点儿盐,你小子就是不听。”

“扯淡,盐放少了肉臭了怎么办?自打水有了保障,每次还不是你吃得最多?”成树国很不以为然,“就是没有盐的问题也该和本地人接触一下了,我看那些索度人根本不会再来了,也怪云飞,好端端的把个人活生生放血放死,估计是吓住人家了吧?”

刘宁瞟他一眼,手里拿着把干草瞎编着,“换了你更狠,还说云飞?你说怎么才能编个篮子出来?”

成树国直起身子一把扯掉刘宁手里的干草,“省省吧你,做手艺人也不能在非洲做吧?还是琢磨一下怎么偷渡出境吧。”

刘宁把手指缝里残留的干草拈出来,狠狠地摔在地上,“偷渡绝对不是问题,边境那么大,问题是咱们就三个人,有个损失担当不起啊,妈的,这鬼日子你以为我愿意过?”

两人正在斗嘴,楚云飞从石头边转了出来,“那就多抓几个本地人,愿意给咱们带路的就留下,不愿意的杀了算了。”

“你怎么跑进来了?”刘宁皱皱眉头,“还没到换班的时候啊,快出去。”

楚云飞摇摇头,“没事,我仔细地看过了,一时半会儿很安全的,不过……我发现了挺奇怪的两个人,要不咱们别躲了,抓住他们?”

“奇怪?”成树国站了起来,“你要说奇怪,那肯定有问题,走,刘宁,咱们一起去看看。”

\section{第六十四章 私奔的男女}

那俩人是一男一女,这就是“奇怪”的地方,因为就小队长们所知,在刚卡,一般是家务和田里的劳动归女人,外出打猎、打仗的事情归男人,像这种男女在一起搭档同行的情况确实很少见,他们要做什么呢?

反正只有两个人,三人也懒得多想,绕了个圈子,埋伏在二人前进的方向上。因为那俩人的组合确实有点奇怪,三人尽量远地离开他们的新居,事有古怪,谁知道对方会不会给他们一个不灭口的理由呢?

那二人似乎在躲避着什么东西,不但走得飞快,还时不时地回头看看。没等多久他们就走到了小队长们埋伏的地段。

楚云飞先轻轻走了出来,两个轻柔的大步从两人身后掩了上去,轻快的脚步没有引起对方任何的警觉。

接着成树国从这两人前方跳了出来,猛然多出个人来!着实把两个黑人吓了一跳。

还没等那男人反应过来,楚云飞就从背后扑了上去,反拧住对方右臂,左臂紧箍对方粗大的脖颈,那男人极力扭动身体想要挣扎,空出的左手居然想去腰间拔刀,楚云飞马上用左手送过去股内气,那男子登时全身酸软,连手中掉下来的步枪地砸在自己的脚上都没注意到。

成树国的目标是那个女的,可那个女人一开始就被吓得瘫在了地上,连出手的机会都没给成树国。

两人制服了对方,刘宁也从隐蔽处出来,顺手关掉手上QSZ92手枪的保险。

只不过是两只菜鸟!小队长们也懒得去摧残他们,只是把那男人身下的步枪拿了过来,腰里那把刀也没收了,然后就地开始了审问。

原来那男人是附近一个胡图族部落的,那女人却是图西人。在前几年两族交战时,这个叫“琳娜”的女人所属的部落被灭族,而琳娜因为相貌出众仅仅得以身免,从此成了该部落一个小家长的玩物,日子久了,众多的胡图男子无聊时也来找她发泄一番,在胡图人看来,反正是俘虏,留她条命已经是很仁慈的事了。

这琳娜本来也有做俘虏该有的觉悟:对于女人而言,那是天生为男人服务的,绝不该有任何的主见。换在图西族部落里,她也未必就能少干了活,少吃了苦,地位也不见得能比现在高出多少。对女人来说,种族的问题是不属于她们考虑范围的。正因为有这样的传统非洲女人的美德,琳娜在异族的部落里也没受多少罪。

这个叫“塔塔”的男人也是没事时就找她来泄火的男人之一,不过塔塔是个很善良的男人,对任何人都很和气,虽然他也很勇敢善战。

塔塔在琳娜身上发泄完后也偶尔地会给她一些吃的或者是别人送他的小饰物,而且此人腰间家伙巨大,而持久力有所不足,正合适琳娜受用,又不用担心时间太长琳娜的耐久度下降被后续的别人弄出事来。

如此一来,琳娜对塔塔就未免多了份体贴的心思,天长日久,两人之间竟然擦出了爱情的火花,不过这种“先做爱后恋爱”的感情成长模式在刚卡实在是常有的事。

近来,图西人在政府和各国维和部队的干涉下,势力急剧膨胀。他们以停战为借口,向胡图人发起了“还我族人”的运动,当然,在前几年的战争中,图西人是弱势一方,手里自然没有多少胡图人的俘虏来表示自己这方的诚意。

塔塔所在的胡图部落虽然在战争中挡住了图西人的攻击,可也有不少妇女儿童在战争中失踪了,其中很有那么几个人据说就生活在不远的某个大的图西部落里。可这个图西部落拒不交人,还说根本没有这么回事,而塔塔的部落又没有足够的证据,只能眼睁睁地看着他们说瞎话。

所以塔塔他们族的人自然也不想把琳娜等图西人交出去,别的不说,单单就为出那口恶气也不能把人交出去。

可胡图人以前一向横行惯了,从不懂掩饰之道,所以所有人都知道这个胡图部落里有图西俘虏存在,现在想抵死不认帐那是不可能了,于是酋长下了道残忍的命令:把部落里所有图西人全都杀了灭口,让他们再要人!

于是塔塔族内现存的八个图西族女人被杀了七个,而琳娜因为平时结得露水姻缘多了些,就有人去跟酋长求情,要求族里放过她。可平时一向顺从的琳娜并不笨,她知道就算酋长能放过她,那个总给酋长出主意的独眼小人也不可能答应。至于原因,琳娜虽然不知道灭口必须要彻底这么高深的道理,却有种女人天生就有的敏感提醒她:危险,再不跑就没命了。

一个单身女人要逃跑那是很不可能的,动乱方止的刚卡大地上猛兽横行、危险重重,再说琳娜也舍不得心爱的人塔塔,就跑去怂恿塔塔和她一并离开,“从此王子和公主就可以过上幸福的生活”,塔塔自然是答应了,结果没想到跑出还没有二十公里就被莫名其妙冒出的中国人抓住了。

听到这刚卡国的私奔故事,楚云飞转过头来又看了看琳娜,实在看不出来这女人有什么迷人的地方,难道是说床上功夫好?不过他还是很敏锐地抓住了问题的核心,“你们打算去哪里?”

“去那里?”塔塔和琳娜对视一眼,“自然是去索度,在刚卡图西族和胡图族永远不可能在一起的。”

刘宁凑了过来,“你们认识路么”

琳娜点点头,塔塔摇摇头。

楚云飞想起来在连队时,有农村来的战友说过,他们村里甚至有相当一部分人一辈子都没有去过六十里外的县城,不由得多问了塔塔一句,“你怎么会认识路?”在楚云飞眼里,远在100公里之外的国境线对于住在这里的刚卡人而言实在是太远了。

那个塔塔是个不太爱说话的人,刚才就是琳娜一直在说话,对于楚云飞的问题,他只说了简单的两个单词,“那是因为战争”——JustFight.楚云飞又问塔塔,“你知道我们是什么人吗?”

塔塔摇摇头,他是个不擅长用心眼的人,这次他说的不少,“知道,中国人,以前你们在这里维和,后来走了。现在你们三个是杀人犯,大家都说你们很厉害,我以前就是哈伦游击队的。”

“那你能不能带我们三个一起过国境?我们能保护你们的安全,我们还有汽车。”楚云飞还有句话没说:告诉你们这么多,你们要不带路,那也只好对不起了。

塔塔不说话了,他掉头看看琳娜,琳娜却真的是很有女人味,“你是我的男人,你决定吧,死活我都要跟你在一起的。”

塔塔坐在那里踌躇起来:人多自然力量大,但是目标也大!怎么办?不过楚云飞紧接着的一句话让他下了决心,“我们会给你们钱,一百美圆。”——关键时候,金钱的力量真的是太大了。

\section{第六十五章 国境线上}

终于找到有共同语言的向导了,楚云飞他们马上忙碌了起来,直接把两个小埋藏点的物资起了出来,又去中心仓库挑拣一番。

虽然是要走了,但谁也不能保证这次的行动绝对地成功,所以楚云飞他们带够了可能用到的物资以后,把剩余的物资全部集中到了主要仓库里,然后把库房和水源彻底地伪装好,当然这些都是背着刚卡人干的。

北京吉普用不着了,那也藏起来,然后大家美美地休息了一天,等天大黑了,成树国驾车,开始了五人的偷渡之旅。

开始的时候不用塔塔指点,毕竟这里也曾经属于中国维和部队的的巡逻区域,虽然不属于楚云飞他们二分队的巡逻范围,但大致怎么走成树国是知道的。

一个半小时以后,悍马开出了中国维和部队曾经的巡逻范围,这时离边界尚有七十公里左右,塔塔靠着自己的印象指点着行进方向。

一路上总的来说还算安全,只是月朗星稀下,有一大片本来没人的地段出现了百十来间小草房,看建筑风格该是有个小的胡图部落搬了过来。没等那些人反应过来,楚云飞喊了一声“冲过去”,等到胡图人听到异声去找寻声音来源的时候,悍马车早走得老远了,给摸不着头脑的人们留下一缕烟尘和一丝疑惑。

靠近边界时就不好走了,不但地面高低不平,而且关卡重重,悍马车虽然机动性能良好,但左躲右闪,还是费了将近四个小时才抵达边境。

塔塔瞪大眼睛,张大了嘴巴,望着五百米外边境线上两米多高的铁丝网,“怎么回事?以前没有这个……这个……铁网的。”

楚云飞瞟他一眼没做声,废话,现在能和那时候比么?索度恨不得和你们刚卡隔上一万里呢。

眼前的国境线上,不但有两米多高的铁丝网,刚卡这边紧挨着铁丝网还有一米多宽的壕沟,还好这壕沟的工程似乎尚未完工,沟沿上还有各种形状的突起,一看就是没来得运走的虚土。

“怎么办?”刘宁也觉得有点意外,“弃车吧,从铁丝网钻过去?”

成树国没有说话,转头望向楚云飞。

三人中虽然楚云飞年纪最小,但现在却是隐隐以他为中心。因为在这逃离军营的几个月里,受恶劣的环境影响,楚云飞展现出了他不为战友所熟知的一面:强健的体魄、超人的身手以及惊人的智慧和毅力。所以现在的三人组倒是以他的话最有权威性。

楚云飞抿了抿嘴,“能不弃车最好,索度那边附近应该没有哨所,因为咱们看不见灯光……再说,边境上这么大的隔离工程不是一年两年能完成的,那这么说只有两种可能,一是这里是该优先完工的地段,二就是这里是必须这么隔离的地段。”

成树国点点头,“不错,不论是一二哪种情况,足以说明这里绝对是重要地段,不是重要关口就是偷渡的重点地区,既然对面没有哨所,那就说明这里是能偷渡的重要地区。”

刘宁有点不解,“那咱们快点过去啊,还磨蹭什么,还不能弃车?”

楚云飞摇摇头,“不急,都到这里了,好好计划一下,要是对面没人,咱们得多长时间才能赶到有人烟的地方?这车要就这么扔了有点可惜。树国开车直接飞过去撞倒铁丝网好象也可以……”

说着说着,楚云飞的脸色严肃了起来,“算了,现在怕是想走也走不了啦。”

成树国和刘宁当时就紧张了起来,云飞从来不是个爱虚张声势的人,他既然这么说,自然会有他的道理。

两人立刻跳下车,成树国在车前,刘宁在车后,左右观望一下,极力压低的声音传来,“云飞,没人啊。”“是呀,我这里也没人。”

楚云飞摇摇头,不可能,这种发自内心的悸动从来没有欺骗过自己的,一定是有什么事要发生了,“别说话,埋伏好。”说完他对着塔塔和琳娜做了个“噤声”的手势,也跳下了车,脚步十分地轻柔。

至于车里那俩人会不会拿起车里的武器攻击自己,楚云飞实在是懒得去考虑了,都是偷渡的,要内讧也得过了边境再说吧?

到了车外,月光的照耀下,楚云飞看到刘宁和成树国都趴在地上用耳朵贴着地在听,听到楚云飞下车,刘宁冲他摇了摇头,成树国却没有改变姿势,继续在地上听着。

既然成树国还在听,楚云飞就懒得也把身子贴到地上了,他静静地站在那里,双眼微闭,平心静气,空明灵台,细细地品味着那种感觉,危险,会是来自哪里呢?

前方右侧,楚云飞渐渐地把握了让自己感到不安的方向,应该是那里。他向两个战友做了个“在这里等待”的手势,猫着身子向那危险的地方慢慢挪了过去。

越往前走,那种不安的感觉越明显,楚云飞只能尽量地保持头脑的空明,脚步也越发地轻柔。半个小时过去了,他才走了一百米。

忽然,楚云飞的脚尖似乎踩到了浮土,有弹性的那种,陷阱!绝对是个陷坑,他越发地小心起来,既然有陷坑,那后面没准还有雷区吧?

小心地绕过那个不大的陷阱,楚云飞又花了半个小时走了一百米,虽然没再遇到陷阱和地雷什么的,不过那种不安的感觉越发地强烈了起来,敌人位置不明,不能再走了!楚云飞趴在了地上听了起来。

前方大约三百米处有轻微的震动,至于是什么,那就不听太清楚了。

楚云飞看看自己手上廉价的电子表:刚卡时间凌晨两点半,时间还够,不过要抓紧了。于是他蹑手蹑脚地继续前行,又花了将近半个小时潜行了一百多米,能够听到人声了。

妈的,这好久不锻炼,一时还有点不适应了,楚云飞暗骂自己一声。然后改变姿势,手肘膝脚并用地匍匐潜行。行不多远就听到身侧不远处有轻微的呼吸声,操,潜伏哨。

绕过潜伏哨,楚云飞又花了将近二十分钟,走到了离那人声大约三十米的地段,悄悄地倾听了起来。

“库卡,精神点,马上到你们换班了。”

“换个屁,你不知道我很困么?和他们说下,谁能替我我给他一瓶罗伦酒,别打扰我。”

“你们两只角马能不能安静点?还有你,坎达,别老讲黄色笑话,笑声很容易暴露目标的。”

“操,你们都住嘴,没听‘钢笔’说刚才可能有汽车么?今天铁丝网再被破坏的话,这个月的薪水大家都不要想拿了。”

“不用那么紧张,都是潜伏哨,谁能跑到咱们跟前来?别太大声就好了。”

楚云飞听了半天才弄明白,原来这里埋伏着一个连的军队,守护着附近一公里的边境,这里就是一个排的潜伏点。倒不是刚卡国部队太多,关键这里实在是偷渡的热点,尤其近两天总有铁丝网被大段地破坏,每次跑过索度去的刚卡人估计能有几百人,这种情况持续下去的话,索度国又要抗议了。

惦记着战友的安全,楚云飞悄然折返,操,今天来偷渡可真不是时候。

\section{第六十六章 战友情重}

五个人坐在那里面面相觑,边境上布满了军队,怎么办?

回去那是想都不用想的,先别说时间够不够,就说汽油,悍马实在是油老虎,剩下的油够不够开回去那都是个问题,就别说下次再来了。

成树国看看时间,已经接近凌晨四点了,很隐秘地向两个战友努了努嘴。

那意思很明显,带着那俩刚卡人实在是不方便,如果把他俩干掉的话,剩下的三个中国军人毕竟有专业水准,想无声地过去就会容易很多。

楚云飞心中也在天人交战:要带上两人过境,风险确实大了很多;可今天一路上的情况说明,没这两人,要顺利来到这里也实在是不太可能,这样地卸磨杀驴过河拆桥,搁给自己还真的是不太能做得出来;何况塔塔和琳娜也是逃亡者,处境和自己人是类似的。

要是杀了这俩人呢,那自己这三人有百分之九十以上的把握不被刚卡军队发现,只要谨慎点,有三个来小时基本上就够用了。三个来小时?三个来小时!楚云飞终于为自己找到个不杀人的理由:再有两个半小时天就该亮了!

再说,塔塔和琳娜一路上表现得很亲昵,给了他俩一块“成家腌肉”,两人也是你一口我一口地相互谦让着,无形中勾起了楚云飞对远方恋人的思念,虽然一时半会儿不可能回国去和自己的琳琳相会,但……天下有情人,终归还是能成为眷属的好吧?

楚云飞先用汉语表示了自己的态度,“这两人算了杀吧,毕竟还有两个来小时天就亮了,五百米,咱们没太长时间悄悄潜伏过去的,多少是会有点响动的,叫这俩人注意点应该影响不大,起码人家光着脚总比咱们穿着鞋响动要小点吧?”

“再说,这种恩将仇报的手段,想想还是不太好做得出来的,毕竟咱们还没到生死关头,你们说呢?”

刘宁点点头,“行,我无所谓,就这么干掉他们我还真觉得有点不地道,妈的,自身难保了还心肠这么软,操!”

两人同意了,成树国同不同意都无所谓了,不过他当时就着急了,“妈的,我还不是为了咱弟兄三个,我倒无所谓,要是伤了你俩我才难受,云飞更是,稀里糊涂就让咱俩连累了……”

“好了,咱没时间说这个,”楚云飞中止了成树国的话,转头用英语问道:“我们要过境了,你们和我们一起走还是分开走?”

这种情况白痴也知道怎么选择,“我们一起走。”

“一起走也行,你俩把衣服脱了包到脚上。”

五人收拾停当,带好东西,楚云飞就要带路前行,成树国转身走向悍马,“我开车冲过去,引开他们,要能撞开铁丝网就更好了。”男人——最不能容忍的就是兄弟的误会。

楚云飞皱皱眉头,一把拉住了成树国,“算了,何必这样?一起走吧,那样太危险。”

成树国却不理他,使劲想甩开楚云飞的手,“别,我的身手也不差,别忘了我也是特种兵。”

刘宁看不过了,上来低声喝道,“树国你再乱来小心我收拾你,我知道你是为我俩好,置什么气?兄弟同心才好,闹鸡巴的别扭!”

楚云飞轻叹一声,他也知道问题的源头在哪里,拍拍成树国的肩膀,“一世人,两兄弟,我给你陪不是了,我现在就去杀了那俩。”说完掉头就走。

这次是成树国一把拉住了楚云飞,眼泪也刷地下来了,“算了,我知道你俩不忍心杀他们,你当我心就那么狠么?我错了,走,咱不开车了。”

“哭个鸡巴,别跟个女人似的,云飞,你带路。”刘宁发话了。

楚云飞带着他们小心地前行,走的还是刚才走过的地段,那自然就快了很多,没用十分钟就绕过了那个陷阱,走了将近三百米,就要到潜伏哨的时候才慢了下来。

又想了想,楚云飞终于放弃了在那个排值守点附近过境的念头,虽然那里人集中,而且潜伏哨肯定会很少,相对会安全些,但一旦被发现,对方的反应会是相当迅速的,留给自己的机会绝对不会很多。

还是离这里稍远点过境吧,不过是一个排,能潜伏的也就最多是一个半班,他们总要倒换休息吧?于是楚云飞向大家做个小心的手势,向着二百米开外的国境线笔直地摸了过去。

时间有限,不能再象上次那样慢慢地摸索了,再说活动了半天,楚云飞已经进入了状态,居然走得快了很多。

细微的呼吸声,又是一个潜伏哨!在离国境线八十余米处,居然又冒出个潜伏哨!楚云飞停下了脚步,做个让大家等待的手势,成刘二人自然是心领神会,塔塔和琳娜居然也相当精明地伏下了身子。

楚云飞悄悄地从侧翼绕了过去,对于这个项目,成树国根本没有跟他竞争的资格。

刀过!血出!

二月份的初春,夜里是非常宁静的,静到能听到被割断气管的“呼噜”声,还有颈动脉向外喷射血液的声音,“嘶~~~~~~~~~~”

楚云飞撩起尸体的衣服就盖住了对方头脸,寂静的夜里,这样的声音也许传不了多远,但人听了绝对不会舒服。

然后五人又花了二十分钟,到达了国境线的壕沟旁,再没有遇到新的麻烦。

从挖出的土看来,壕沟绝对不深,威慑意义远大于实际意义,况且只有两米宽不到,看来远处看和近处看还是有点视觉误差的,不过这点宽度给谁也能跳过去,塔塔倒是回头看了下琳娜,琳娜冲他一笑,小声说“我能跳过去。”

楚云飞的耳朵不是一般的好用,他马上又做了个“噤声”的手势,大家一愣,却听见远处隐约有脚步声响起。

刘宁马上反应了过来,“流动哨?都趴下!”大家赶紧就地卧倒。

楚云飞趴在那里听得脚步声在继续,却隐约觉得有什么地方不妥,到底是哪里出了问题呢?

坏了!楚云飞终于知道是什么东西一直让他惴惴不安了,这家伙,这家伙别是来换岗的吧?

实战经验到底是在实战中才能获得的!

“别是换岗的吧?”楚云飞小声说,“操,该在那里埋伏个人的,你们准备好东西,我去看看到底是做什么的。”

说毕,楚云飞半直起身子,悄无声息地向那岗哨所在地摸去。

但他反应得还是晚了,还没走到一半,那边已经传来了“砰”的一声枪响,然后是响彻夜空的嚎叫,“勒多死了!勒多死了!有人越境!有人越境!”

一时间,人声四起,灯火齐明,老天,居然还有几盏探照灯!

不能再等了!楚云飞飞一般地往回跑,“你们快跳过去,快剪铁丝网!”

\section{第六十七章 混乱的战斗}

灯光猛然亮起,不但偷渡的人一下适应不了明暗的忽然转换,刚卡军队也一样没法接受那强烈的视觉反差,过了足足有半分钟,才有凌乱的枪声响起。

半分钟就足够了,时间充裕得很,留在原地的四人已经反应过来,带着装备和武器跳过了壕沟,至于楚云飞,他跳过去的时间似乎比别人还早点。

枪声越响越密集,虽然五人都放低了身子,但在明亮的灯光的照射下,刚卡人还是轻易地发现了他们,子弹也象长了眼一样不再四处乱飞,五人当时就陷入了困境,只能死死地趴在地上。

楚云飞恼怒得要命,怎么还不赶快剪铁丝网啊?不过,现在显然顾不上说这个事情,“树国火力牵制,刘宁,你打左边的那个探照灯,我打右边的的,树国一开枪咱们等5秒一起动手。”

这种场合下,只要有人做出合理的安排,执行的人绝对是不会犹豫的,何况楚云飞早已隐隐是三人之首。成树国一个侧滚,手中八一自动步枪就开火了。

“哒哒哒”,八一步枪枪口冒出暗红的光焰,在光明和黑暗并存的这个诡异场合里,显得分外地引人注目,至于成树国在向哪里瞄准,那只有天知道了,连他自己都不知道。

当刚卡人的火力都被成树国吸引过去的时候,楚云飞和刘宁动了,他俩的枪法本来就不是开玩笑的,部队里普通三练习就是夜里打灯泡,何况是这么大的探照灯?“哒”、“哒”两个点射,一百米内的两个探照灯登时熄灭,隐约还有人的惨叫,7.62毫米的子弹在这个距离内是相当有杀伤力的。对方的火力马上就分散了开来。

终于能抬起头来对射一下了,虽然刚卡政府军还有若干强力手电筒,隐约中也能看到纷杂的光柱在胡乱晃动,应该是有临近作战单位来支援了。楚云飞和刘宁屏息瞄准,打一枪换一个地方,不多久就杀伤对方五六个人。

楚云飞实在有点忍受不住了,抽空回头看看,却是惊讶得张口结舌,说不出话来。原来那铁丝网并不是部队里常见的那种真正意义上的铁丝网,倒有几分国内高速公路上隔离网的味道,网眼非常地小,网格非常地密集,而且几个网格组成个大点的菱形,又有加粗的大的菱形。这种结构,虽然没有伤人的铁刺,却是不易攀爬,而且剪起来也非常地麻烦。不过这种铁网该是非常昂贵的,刚卡人什么时候这么有钱了?

成树国早退了回来,正借着时有时无的灯光,趴在地上和塔塔拼命地剪着铁丝网,不过,这东西剪起来实在是太麻烦了,而且还要注意躲避子弹,两人现在剪掉的部分加起来最多也就是能钻过去只兔子,还得是那种毛厚不怕刺扎的。

虽然知道二人已经尽力了,楚云飞还是在射击的空余再度强调一下,“尽量快点,没准还有照明弹呢。”

成树国忙得满头是汗,“已经尽力了,不过,我看他们未必能有照明弹,有的话早用了。”

乌鸦嘴的话音还没落下,一颗照明弹从空中缓缓落下,天地间一片光明,纤毫毕现。

楚云飞正在一个稍高的地方准备射击,一时间暴露无疑,当下就有子弹呼啸着飞来,急切中一个侧滚向下翻去,登时觉得腰间一震,却是仓促中顾不得那许多了。

琳娜也在这时发出了“啊”的一声尖叫,可形势极其混乱,没人去注意了。

刘宁打得正兴起,却也被逐渐转移过来的子弹压得抬不起了头,至于成树国和塔塔,却是还在坚持不懈地剪着铁丝网。

刘宁看着实在抬不起头,就回头看看成塔二人的工作进度,不看还好,一看不由得念叨起来,“操,地上还要剪?不是贴地铺的网?”他可不知道这全是前几天有人不断破坏铁丝网,刚卡政府不得已而为之的。

成树国被子弹打得郁闷极了,也没什么好气,“日了,这铁丝网土里还埋了一截,什么鸡巴玩意儿?你也别总看了,火力压制啊,这他妈还让不让人偷渡了?”偷渡历史上,能这么理直气壮地抱怨,成树国也算开了个先河。

楚云飞一如既往地狡猾,探头出去就是一枪,然后迅疾缩头,一溜子弹打得他头上尘土飞扬的同时,听到近处传来一声惨叫,该是趁乱摸过来的刚卡士兵。

刘宁可真的纳闷了,这云飞……未免彪悍得有些离谱了吧?这样也能发现对方,而且还能打中?本来他就很着急了,好强心一起,再也顾不得那么谨慎了,就探头出去观察。

对面的刚卡士兵早就发现了,偷渡的人一共有两个火力点,虽然位置飘忽不定,但基本上就是那一片,早有有心的人盯住了刘宁的活动范围。等到刘宁在明晃晃的照明弹下探出头来,雨点般的子弹瞬间扫了过来。刘宁当场头部中弹!

刘宁下意识地晃晃脑袋,又伸手摸摸头,还好,又是一次擦伤,而且就在才长好的那块附近,“操,这下头发怕是长不起来了。”

虽说伤不严重,但头部是人血管集中的地方,大量的鲜血还是顺着刘宁的脸颊淌了下来,糊住了他的眼睛,“树国你来帮我包下头,血流得我看不见东西了”

楚云飞闻言回头一看,真的有点恼火了,成树国和塔塔的工作效率也太低了点,刚才那口子能钻过只兔子,现在这口子能钻过只……呃,大点的兔子!

虽然知道责任不在二人,但这种危急情况下,脾气再好的人怕也是免不了焦躁,何况楚云飞也不是脾气特别好的主,“塔塔你快点剪,磨蹭什么?”可说话间,刚卡军阵地上又陆续降下来几颗照明弹,楚云飞大怒,“操,还让不让人活了?”

其实楚云飞知道自己在担心什么,但现在这担心怕是已经成了现实:对方其他两个排放弃了其他地段的守卫,调过来增援了么?

\section{第六十八章 逃离刚卡}

这点楚云飞倒是高估了刚卡军部队的觉悟,反正是一片漆黑,就算听到了激烈的枪声,也接到了友军的通知,但相邻的那个排也就是象征性地派了九个士兵过来,不过照明弹是带了几发。

因为交战的这个排位于连指东侧,虽然中间连指所在的排兵力还是够用的,但为保证大局,中间的这个排其实是不宜妄动的,要不万一西侧那个排再有点什么事情就不好协调了,而且又在夜间,部队调动不便。

最重要的还是:交战这个排报告的是对方人数不多,虽然看上去作战经验丰富,单兵素质也能从探照灯的熄灭上分析出来是比较强的,但强煞了也就是那么几个人,这个功劳既然是跑不掉的了,何必邀请别人来分一杯羹?排长倒是小心地提醒兄弟排小心戒备,别落入偷渡者“声东击西”的陷阱。

这样一来,虽说兄弟排看着别人大吃“肥肉”而眼红,却不宜再横插一手,待到后来这排接连有士兵中弹,排长发现这“肥肉”着实难啃时,虽然马上邀请兄弟排来“共患难”,那些伤了自尊的“兄弟”却在慢慢调整队伍,根本不理会连长的催促:我们还怕“偷渡者声东击西”呢!

中间连指所在的排比较倒霉,在连长大人大发的雷霆下,勉强抽出了一个班睡眼惺忪的士兵来支援“邻居”。

楚云飞是不可能知道对面刚卡军队的扯皮的,不过他是真着急了,“树国,你快点包扎,他们援兵来了就麻烦了。”

成树国是什么项目都能拿出手的,包扎伤口也异常地熟练,何况这种紧急情况下,也只能简单地包扎下了。等他把刘宁的伤口包好,侧头一看,却见楚云飞才刚把步枪缩回来,然后是报复的子弹雨点般飞来,打得楚云飞满脸都是子弹飞溅起的半干的泥土。

楚云飞感觉到了成树国的眼光,侧过头来,“完了?你俩牵制他们,我来剪铁丝网。”说话间,楚云飞觉得腰里凉飕飕的,一股液体顺着裤腿在往下流,挂彩了?怎么不是热的?他伸手一摸,才知道一瓶用来装“神奇咸水”的瓶子被打烂了,妈的,还好。

楚云飞刚结束的几枪连发连中,打得对方是胆战心惊,,尤其是一枪打得一个士兵端枪的左手断二伤二,“十指连心”啊,那士兵连滚带爬地往回跑着,嘴里也是哀号连天,实在是影响军心得很。

于是成刘二人两枪齐发的时候,对面士兵也是忙不迭地寻找掩体,等发现这两个火力点威力不大的时候,才又逐渐组织起了压制的火网,那排长严令机枪手:“别怕费子弹,要打得他们抬不起头,天亮再围歼他们。”

楚云飞已滚到了铁丝网旁,塔塔还在努力地剪着,在他的努力下,铁丝网的缺口又大了点,但是这速度实在太慢了点吧?

楚云飞拿起匕首上所带的夹剪,一下一下地剪了起来,才剪了三剪他就受不了了,靠,这要剪到什么时候?抬头观察了一下空中的弹道,“树国,东边机枪手,扔手雷。”

成树国闻言二话不说就是两颗手雷扔了出去,机枪距离他们有一百多米,趴在地上,给谁也扔不了那么远,不过能起到掩护的作用就够了。

手雷爆炸声传来,楚云飞半跪起身子,手持匕首,气运全身,意存右臂,尽力向铁丝网划去,成了!火星四溅中,铁丝网如豆腐般顺手而开,出现个大大的一横。

有门!楚云飞不敢耽搁,刷刷又是两匕首,铁丝网上顿时出现个“∏”形划痕,左手顺势用力一推,铁丝网倒,一个大洞出来了!

看到楚云飞那神勇的一刀,塔塔不由得发出一声低低地欢呼,引得几人纷纷掉头观看,看到楚云飞轻松地手起网断,成树国摇摇头,一声长长的感叹:“操~”

这下用不着楚云飞说了,成树国又摸出两颗手雷,扔了出去。

由于楚云飞没把最下面的铁丝网划开,钻洞的时候,根本不用担心下方会有铁刺,五个人几乎在瞬间就穿越了国界!

非洲女人的忍耐力确实惊人,琳娜的后肩胛被子弹擦伤,而且没有人给她包扎,但她似乎根本就没有注意到自己挂彩来着。进入了索度这个想象中的天堂,逃得生天和获得自由的双重惊喜让她雀跃了起来,“自由了!太好了!”

塔塔一个前扑就压倒了琳娜,他久经战争,自然知道现在实在不是欢呼的好时候,果不其然,他还没来得及解释,长串的机枪子弹就飞了过来。

这时迟钝的塔塔才反应过来,“琳娜,你受伤了?”

琳娜满眼都是欣喜,脸上挂满了微笑,“塔塔,你的琳娜没事,现在已经不怎么疼了。”

三个中国军人自然不会犯非洲女人那样愚蠢的错误,在他们的印象中,刚卡人……也许是非洲人实在是不可忖度的,别说过了国境还要被子弹追击,哪怕刚卡人跟着钻过来都不会让小队长们太吃惊的。

塔塔看到小队长们趴在地上低姿匍匐前进,也有样学样地拉着琳娜跟着向前爬去,嘴里还不停地喊着:“快包扎,琳娜受伤了。”

小队长们没理他,一直快速地爬着,直到离开铁丝网有七八十米的时候,才回头等着他俩,塔塔的体力还真不是吹的,剪了半天铁丝网,又拉着琳娜爬半天,居然还没落下众人几米,虽说匍匐前进的姿势不太标准,却也没受伤。

等到塔塔赶上众人,自然要问问,“你们怎么不理我们了?她受伤了!”小队长们却没理他,急着用枪托在地上划拉出个小坑。

还是刘宁厚道些,“你没看见?我们还要预备他们追击呢,在那里包扎不是找死?”说完拿个急救包出来递给了塔塔。

塔塔回头看看那铁丝网上的窟窿,点点头,“我不会用这个,你们包一下吧,他们不敢过来的。再往前走走我们就安全了。”

小队长们知道塔塔在这里打过仗,紧张的心就松懈了点,成树国猫腰跑过来看看琳娜的伤口,“没事,小擦伤,咱们再走远点再包扎。”

\section{第六十九章 绑住的向导}

被“肥肉”硌掉满嘴牙的刚卡排长恶狠狠地看着黑忽忽的索度国境,虽然天开始有放亮的意思了,但那偷渡者已经都看不到了。事关国家主权之分,他自然也不敢往对面发照明弹了,“我操,别让我知道他们是什么人。”

这时副排长向他努努嘴,“你看看那个铁丝网是怎么被弄开的。”

排长仔细看看,那钻兔子的口子上参差不齐,自然是用夹剪剪开的,可那钻人的窟窿可是整齐地裂开的!其中还有三根小手指粗细的钢筋啊!他吸了口凉气,“用锐器划开的?好锋利的刀!”不错,这证明偷渡者装备太精良了,那么这次失利,差错也不全在自己身上了。

楚云飞自然不知道身后的刚卡军人在杜撰一把以地球上现有工艺都未必能加工出来的武器,他正在淡淡地威胁塔塔,“你们要是不同我们一起走的话,水、食物和武器都不能交给你。”

原来偷渡队伍走到安全的距离,把刘宁和琳娜的伤口好好地包扎了起来,然后双方就开始讨论下一步该去哪里。塔塔和琳娜对中国人实在没有什么太好的印象,而楚云飞那霸道的几刀也给他俩留下了非常恐怖的感觉。所以塔塔接受了一百美圆的酬劳后就提出了分道扬镳的要求。

楚云飞他们对索度的了解非常少,自然不愿意答应他们离开,开什么玩笑,最危险的时候没有杀你们,带着你们这俩包袱拼命地闯了出来,刘宁还为此受了伤,你们居然要过河抽板?

小队长们做如是想是很正常的,毕竟要是没有他俩,三个军人是能够比较容易地潜伏过来的,根本没有杀那个潜伏哨的必要。自然,要是再没有刘宁,光楚云飞和成树国两个特种兵的话,悄无声息地过境的可能性无限接近于百分之百了。

可楚云飞也不想强迫对方留下来,怎么说自己三人也在逃亡中,虽然索度肯定没人来专门追杀自己三人,环境业已经有所改善,可危险还是有点的。过于强迫二人难免又会在逃亡途中增加一些变数,所以只能在对方要求归还武器时做出刁难。

塔塔是很讲道理的,再说这种情况下也由不得他不讲道理,“我们说好的,你怎么能变卦?不给我们武器,让我们怎么自卫?而且没刀的话,我们找水都很难。”

楚云飞依旧是那么冷漠,“给你们武器你们就有能力自卫了?没准你们还会因为这些武器丧命呢,再说,我们要你带我们过境,可也没说要还你们武器的,还想要水和食物?开什么玩笑?”

琳娜其实要比塔塔聪明得多,她已经感觉到三个中国人不想放他们走了,不过女人的直觉告诉她:其实对方没什么恶意。于是她拽拽塔塔,“要不一起走吧,听说索度国的人对刚卡人很凶的。”

一个富翁对经常给他带去点麻烦的穷邻居自然不会很友好,塔塔也知道琳娜说的是实话,“那好吧,你们确实很厉害,一起走吧。”

既然要一起走,大家就计划起了行程,这方面塔塔是最有发言权的,“这里本来有几个胡图和图西部落的,不过后来好象被索度国政府迁走了,只有多特人了。”

“多特”这个单词让楚云飞又想起了那个中国留学生,那人不象是骗子啊,算了,还是暂时不用去想他了,“你们打算去哪里?”

塔塔和琳娜对视一眼,“我们想先去多特人那里,多特人爱用偷渡者干活,虽然给得钱很少,但能在那里干满一年的话,可以申请成为索度居民的。”

小队长们自打从军营里逃出来,就对索度这个国家操上心了,这里是最便利的逃亡处,索度的相关资料三人自然也是努力收集了。

“有这样的好事?”成树国显然有点不太能理解,因为在他的印象中,索度人该是没这么大度的。

“那你知道多特族有个提坦卡布这么个部落么?”还好,楚云飞还记得刚贝拉那个部落的名字。

“提坦卡布?那是多特人古老的叫法,我们都管他们叫‘靠近大海的骑士’、‘隐藏的秃鹫’这种部落名字,我可真不熟悉多特人那套老叫法。”塔塔对多特人还是有点了解的。

既然所有的人都没有更好的去处,那大家只好在塔塔的带领下向西北方向走去,据说那里多特人的部落比较多。

由于有琳娜这个女伤员在,一行人走得并不是很快,等到天色大亮的时候,五个人才走了五、六公里。沿途的景色也起了略微的变化,树木变得更加稀少了,裸露的土层似乎也多了起来,这样一直走下去,估计能走到索度沙漠吧?

塔塔心疼琳娜,不管小队长们的吩咐,执意走一段休息一段,楚云飞想反正刘宁也负伤了,也就没坚持要大家赶路,因为五人行动速度不快,终于在距离国境线不到十公里的地段遇到了索度国的巡逻部队。

双方的遭遇很偶然,塔塔在翻过一个小土坡后愣了一下,然后拼命地往回跑,“索度人的军队!”

跟在塔塔屁股后面飞来的是稀疏的子弹,还有人大喊,“偷渡的,站住,要不就打死你们!”

妈的,这日子还能不能过了?楚云飞压制已久的怒气终于被点燃,“就十来个人就想找麻烦?想死成全他们!”

对方是一支国境巡逻小队,大概就是十一二个人,不过天长日久,他们已经娴熟地掌握了应付偷渡者的方法,偷渡的人数要是多,那就先远远开枪,吓跑大部分人后抓几个腿脚慢的老人或者小孩充数,然后赶回营房呼叫支援,也算是完成了任务,因为毕竟可能遭遇到好几百人的偷渡队伍,凭十几个人实在是惹不起的。

要是偷渡的人数比较少,那巡逻队就先开枪震慑住对方,以防止对方逃跑,然后顺利地将人群缴械,押回驻地处理,毕竟偷渡者在心理上是心虚的一方,面对十几个军人,一般只有三、四十号人的偷渡队伍不是没命地四处逃窜就是乖乖地束手就擒,而对付那些想跑的人,索度军队一般的处理方式就是乱枪打死。

反正不管怎么说,先开枪那是一定的!

这次巡逻队看到的只有塔塔一人,还没确定对方是不是偷渡者呢,塔塔先撒丫子开溜,这做贼心虚的举动自然招来警告的子弹。

\section{第七十章 杀蛇给人看}

等索度国的士兵追上土坡,发现对方只有五人,当时放下心来,不错,又一件功劳是稳获的了,这时眼尖的楚云飞看到军官模样的黑人拿起手持步话机在说着什么,该是枪响惊动了对方附近的部队吧?

不过,楚云飞对自己三人的装束还是很满意的,为了便于夜里偷渡,遵照特种兵的作战风格,他和成树国都主动在皮肤裸露的地方涂上了黑迷彩色,刘宁虽然不擅长这个,但有样学样还是会的。自己这可怜的几个人应该引不起对方的警觉吧?

楚云飞下令了,“塔塔过来,大家先蹲下,把步枪扔出去。”说着顺手把脖子上两支步枪抛到了地上,随后又隐蔽地做出个“准备战斗”的手势提醒两个战友。塔塔本来不愿意扔下自己好不容易要回来的AK-47,不过别人都扔了,他再不识相那肯定是要倒霉的。

看到偷渡者主动缴械,巡逻队果然放下了警惕心,大模大样地走过来了,只有两个士兵在五十米左右的地方停下来警戒,看来索度国士兵的训练还满象那么回事的。

等到那些索度士兵走近,才有人注意到了偷渡者里居然有三个不是黑人,“白人,有白人!”

在这里值守边境的士兵们大多是附近部落的,因为见识有限,大多数人是分不清楚黄种人和白种人的,听到有传说中的白人,士兵们都去掉了警惕的心思,纷纷走上来看希奇,连那俩警戒的士兵也凑了过来。

楚云飞他们倒是见怪不怪了,这种笑话在刚卡他们也遇到过,不过因为日子久了,当地人才知道中国人其实算黄种人。

那军官有点发憷,以他那可怜的见识,也知道白种人轻易是惹不得的,以这个种族的优势和资金的充裕,实在是没有任何偷渡的理由的,要不,向上级汇报一下?

看到军官手伸向了步话机,楚云飞三人暗自握紧了藏在腰后的手枪。可那军官想了想又放下了伸出的手,“白人朋友你们好,我是索度国的肯亚克,少尉排长。”

楚云飞三人的意思本来是把对方全部吸引过来,如果有不妥,凭着三人三支QSZ92半自动手枪,再加上先下手的优势,绝对能迅速地解决战斗,至于为什么要把对方吸引到一起,那自然是因为“灭口必须彻底”。

看到肯亚克友善的笑容,楚云飞难免又动起了心思:听说索度国的士兵很爱钱的,为什么不试试?

于是楚云飞松开握着手枪的手,慢慢站了起来,笑嘻嘻地回答,“很高兴认识你们,我们是美国公民,我是詹姆斯。欧德,受国际野生动物保护协会派遣,来这里调查索度蜥蜴的生存环境,这两位是我的同伴,那两个是我们的向导。”——索度蜥蜴是种接近灭绝的毒蜥蜴,说完顺手塞过去两张十美圆的钞票。

索度人再闭塞,那美圆也是认识的,肯亚克的笑容灿烂了起来,白人就是大方,“哦,我亲爱的朋友们,你们似乎遇到了点小麻烦?”

楚云飞咬咬牙,伸手跟刘宁做个“要钱”的手势,没办法,他实在是没钱了。

刘宁不傻,装模做样掏半天,摸出张十元面值的美圆。“就这么多了,没有了。”

楚云飞把钱递过去,嘴上还在编故事,“很倒霉,我们遇到了刚卡来的偷渡者,被他们袭击了,还好我们跑得快。”

肯亚克接过钱,半信半疑地问,“那……你们需要什么帮助么?”

楚云飞装出一副悻悻的样子,“唉,索度国实在是太不安全了,我要尽快地完成我们的任务,再不来这里了。”——他们总不能派人陪着自己一直考察吧?

这下肯亚克却是为难了,这么丢人的事怎么让自己遇上了?不过他马上想起了什么,“能让我看看你们的护照么?”

楚云飞表情一下严肃了起来,“肯亚克,你要检查我可以,不过能让我先看看你的证件么?”

如楚云飞所料,肯亚克是不愿意让对方看自己的证件的,原因无他,一旦验看证件,证明了自己的身份,将来面前这人把自己受贿的事说出来怎么办?那可是绿油油的美圆啊。“哦,那算了,反正我的职责是防备刚卡人,这俩向导有没有身份证明?如果没有证明的话,他们是有偷渡嫌疑的。”

楚云飞他们自然不肯把塔塔二人交出去,要不他们的身份可没法保密了,再说塔塔他俩也知道了自己有去“提坦卡布”部落的打算。

不过,塔塔和琳娜却是考虑不到这么复杂的事情,琳娜当时就着急了,送回刚卡?等待自己俩人的下场那是想都不用想的,“救命,大人救命啊。”

楚云飞暗骂两人,“笨蛋”,这不明摆这告诉对方你俩是刚卡人么?不过骂归骂,人还是不能让他们带走的,“这个,肯亚克少尉,我们需要什么样的代价才能带走他俩?毕竟人是我们请来的。”

肯亚克真有点头疼了,这三十美圆实在是不好挣,“那不可能的,你们的身份我可以不去管,他俩?没有一百美圆你让我怎么向我的士兵们交代?”

塔塔忙不迭地把刚挣到的酬金拿了出来,心疼那是难免的,不过想到自己偷渡时伤了那么多刚卡士兵,眼下自己的小命是最重要的。

肯亚克看到那一百美圆眼都红了,一把抓了过来,“想不到刚卡人这么有钱。”

“那是我们支付给他们带路的报酬,”楚云飞不介意把事情弄得似是而非一些,事实上,这些半真半假的话确实起到了迷惑对方思路的作用。

可这美圆的诱惑似乎太大了,肯亚克眼珠开始乱转。别是在打什么念头吧?楚云飞有点不耐烦了起来,看来有必要把水搅得更混一些,“肯亚克少尉,我建议你打消你脑袋里不够谨慎的想法,事实上,”他凑近对方,把声音也放低了一些,“我们负有维和部队的秘密任务,希望你能管住自己的嘴巴,不要声张出去,而且,你这几个士兵实在是不堪一击的。”

说完,楚云飞迅速地拔出自己的手枪,向着三十米开外就是一枪,“砰”地一声过后,一条长有一米半,酒盅粗细的土褐色的响尾蛇从土里冒了出来,在地上不停地扭动着,很明显,它被击中了。

几乎在同时,刘宁和成树国也拔出了手枪虎视耽耽,楚云飞淡淡一口气,吹去枪口上的轻烟,“我再次建议你,忘记今天曾经见过我们几个人。这样对你、对你的士兵都有好处,唉,我们美国人就是太仁慈了。”

\section{第七十一章 破财赌帐还}

索度士兵根本没想到楚云飞他们会来这么一手,早在发现有白人的时候,大家就彻底放弃了所剩无几的警惕,因为白人意味着高人一等,意味着遥远的强大和传说中的富足。

所以当楚云飞他们拔出手枪的时候,一众索度士兵还在思索:肯亚克长官拿了这么多钱,会不会请大家喝顿酒呢?根本就没人起心去保持戒备。

楚云飞的枪法看得肯亚克点头不已,好厉害的枪手!不但拔枪快,准头也令人叹服,最要命却不是这个,而是:这个白人居然能在三十米外发现隐藏的响尾蛇,要知道响尾蛇可是伪装的大师,他是怎么发现的?

至于成树国和刘宁也顺手拔出枪来,却让肯亚克轻易地相信了楚云飞低声说的那几句话,那干净利落的动作,有点军事常识的人都知道眼前肯定是训练有素的军人。至于三人同使手枪更让肯亚克确定了自己的判断,要知道,手枪因为威力小,射击距离短,并不适合战场需要,在非洲不怎么受人们的欢迎,通常携带它的人只是为了张显身份,以炫耀的目的居多。

那眼前这三人百分之百是身背特别任务的白人军人了,而且真是像他们说的那样,以他们所表现出来的军事素养,干掉自己这个小队绝对不会是很困难的事。而且自己这方一旦还击,造成对方的什么损失的话,那麻烦就大了,白人……那是能惹得起的么?更别说还可能破坏对方的秘密行动。

况且三人眼里开始显露不善的意味,肯亚克豪不怀疑对方确实有跃跃欲试的意图,而且一旦动手,绝对不会存在“手下留情”这种可能。

反正钱是到手了,这种只能挨打不能还手的事肯亚克是绝对不会去做的,他还下了决心,回驻地以后队里所有士兵一人发两美圆,领钱的前提必须是忘掉这件事。至于这么一大笔财可不能轻易地破去,那等士兵们钱到手后,自己设庄开赌好了。……对了,玩的时候一定要小心,别让连长知道。

想到这里,肯亚克脸上的微笑里就多了那么几分谄媚的味道,“欧德先生,请相信我,我始终认为美国人是我们索度最亲密的盟友,而且,有兄弟那么深的感情,至于您所交代的事情,您就放心好了,还请您有空的时候回来看看忠实的肯亚克兄弟。”

楚云飞三人自然不会因为这么几句话就轻易地放松戒备,但该做的表面工作还是要做足的。于是,两拨人马虽然都发射过子弹,但结局却是“恋恋不舍”的告别。

这等关头,塔塔居然有心去拿刀把那条蛇的头砍掉,还把蛇身子拿了起来,他的理由很简单——有吃的总比没吃的强,多储备点粮食没什么坏处。楚云飞不由得暗自摇头,这人神经大条得离谱啊~偷渡者们一边戒备着,一边缓缓前行,等到巡逻队远得几乎看不清人影的时候,楚云飞才掉头训斥琳娜,“以后没让你说话你就别张嘴,差点让你坏了大事……塔塔你别这样看着我,我不介意自己多杀个人,少个向导并不是什么大事。”

说完,楚云飞理都没理他俩,而是拿起了放在背包中的望远镜,又回头观察一下,看看对方会不会耍什么花样。

还好,索度士兵还那样走着,没有什么玩弄花样的心思,放下望远镜,楚云飞掉头向刘宁说,“我说,你觉得我是不是变得越来越冷血了?居然动不动就想杀人?”

刘宁还没来得及说话,成树国就插嘴了,“你要算冷血,我算什么啊?现在这世界上,也就咱弟兄三个是一条心了,其他根本就再没他妈的可信的人了,不杀人,等人来杀咱啊?”

刘宁的脑袋有点发晕,迷迷糊糊地点了点头,“是啊,我们都在变得冷血,没办法,我们要活下去,必须这么做……咦,我怎么头有点发晕?”

“别是有破伤风或者败血病什么的吧?”成树国一听这话就上心了,“先吃点抗生素吧,别在这紧要关头出点什么差错,等等我找找药,日了……怎么还有四环素?这么古老的药十年前就该扔进垃圾堆了,这个抗菌优倒是还能用用,操,我知道了,这他妈的是给刚卡人用的药,咱的药包呢?”

好容易找到了螺旋霉素,这已经是国内专为维和部队配备的标准抗生素了,成树国拧开瓶矿泉水,递给了刘宁,“前面有几棵树,咱们赶到那里歇歇吧,等刘宁好点再走。”

塔塔自然是忙不迭的附议了,其他人也没意见,几个人走到那里后,简单地伪装了一下,还好周围还有不少的小灌木,再把浮土随便刨刨,从外面看去,这几个人就算是消失了。

等到了中午,刘宁的头晕还没有好转的迹象,反而发起烧来,三人并没有体温表那种奢侈品,那东西在刚卡库房里似乎还有两支,现在只能靠眼皮去感觉了。看到他的病情不太好控制住,成树国只得又加大了药量让刘宁服用,至于继续赶路那是提也不用提的了。

要是没遇到过索度巡逻队,塔塔和琳娜怕是又要闹着走了,可既然经历过了那么一出,余悸犹在,他俩也不吱声了。塔塔看到刘宁情况不太妙,晚上有在这里过夜的可能,倒是辛苦地把这藏身之地又深挖了有半米左右。

还好这里土质疏松,又有琳娜帮忙,塔塔也没费多少力气。深了这半米,如果有必要,晚上也能小小地生那么堆火了吧?

等到了天黑的时候,刘宁的情况越发地恶化,全身滚烫,连呼出的气都是灼热的,人早就失去了清醒,只知道不停地喊“水”了,可是水到嘴边,他根本咽不下去,含到嘴里就又喷了出来。

成树国随后又以为是疟疾,可这里靠近索度沙漠了,附近应该是没有湖或者湿地什么的,蚊子该没有生存的条件的,就算有那么几只远来的、夹缝里生存的,也该不至于能传染疟疾吧?

\section{第七十二章 病危的战友}

不过成树国还是用治疗疟疾的办法试了试,先降温,没水?把湿点的泥土堆到刘宁头上身上,过得几分钟又扒开换新的,这样不停地重复着。

折腾了许久,成树国终于做出判断,大概不是疟疾!因为刘宁迟迟没有到来寒冷的感觉,不过拿土降温这个过程是没错的,起码能抑制刘宁的体温,不会因为体温过高损伤了肌体功能。

成树国现在只剩下一种选择了:给刘宁注射激素,指望靠他自身抵抗力的提高吧,还好注射器倒是带得有。

针打了下去,但是刘宁的情况还是没有好转,汗水越出越多,随后又被他的体温蒸发,恶化的病情急得成树国直搓手:怎么办?怎么办???

等到夜里三点多,刘宁已经是出气多进气少了,整个人也因为失水过多而显得似乎小了一号,成树国看着在小火堆旁靠在一起的黑人情侣,忍不住满心的烦躁,可思绪一转,又忍不住地想哭。

“云飞,你下来吧,宁哥……宁哥怕是不行了。”

楚云飞身子比较轻,而且视听能力特别好,所以一般情况下,他总是扮演坐在树上放哨的角色。虽然这里已经不是刚卡,冒充白人似乎也能给三人提高点安全系数,但还是小心为上的好。

楚云飞从树上下来过几次,知道刘宁病情很严重,可他和成树国的医护水平相比还略微地差劲些,所以他并没有上前添乱,也不提换岗的事。这么做,一来是不忍心让战友再为别的事分心,二来就是暴怒下的成树国其实是很冲动的。

可刘宁的病情确实牵动着他的心,现在一听要“不行了”,他立刻从树上跳了下来,操,这话是随便说的么?这个乌鸦嘴!

楚云飞先走到塔塔旁边,拍拍塔塔的肩膀,“你去外面放哨,小心点,放哨会吧?我不信哈伦游击队的人连这个也不知道。”

塔塔本不舍得离开琳娜,原来在部落里不方便如此亲热——琳娜是公共财产,现在来到了索度,有了这个长相厮守的机会,自然要好好珍惜。不过塔塔也发现成树国的情绪非常糟糕,在这三人里虽说楚云飞给他的感觉是最强悍的,但不知道为什么,成树国却是最让他恐惧的。

楚云飞来到刘宁旁边蹲下身子,发现他的情况确实是不太妙:双眼紧闭,脸色发白,鼻翼在急促地翕动着,却感觉不到什么空气的流动,偶尔喷出股气息却是灼热的,手脚还是偶尔能抽动几下,可抽动间那种有气无力的感觉却给人种实实在在的“濒临死亡”的味道。再把把脉,那动脉的跳动是实在缓慢,每分钟能有四十下么?

楚云飞一屁股坐在地上,皱着眉头仔细考虑着,顺口问声,“打了几针?”

成树国却是不敢看楚云飞的眼睛,小声地嗫嚅着,“一针。”

楚云飞点点头,嗯,一针,是啊,谁能想到刘宁这病来得这么凶猛呢?激素这个东西,一般一针就够了,只有重症才可能大剂量使用,可大剂量使用的后果实在是太可怕了,可能造成的后遗症太多了。

而且,像刘宁这样的病人,本身就非常年轻,而且又是常年训练的军人,按理说是用不着使用大剂量的激素的。何况,等刘宁清醒了,却发现迷迷糊糊地就有了“股骨头坏死”之类的毛病,怕是更愿意永远也醒不过来吧?

要不要再打一针?成树国拿不定主意,楚云飞也下不了决心,这针下去,能不能救回人来不好说,但救回来的人怕是多少会落下点什么毛病的。

成树国看着楚云飞在那里发呆,病急乱建议了,“云飞,要不你用气功试试?”

楚云飞真的愣了一下,他没想到自己在成树国眼里,居然莫测高深到了这样的程度,这种情况什么时候开始的?成树国可是个从不服人的主。

难道刘宁的病情已经严重影响了成树国的行为逻辑?楚云飞摇摇头,操,这时候想这些干嘛?

可楚云飞练的气功到底是怎么回事,连他自己也弄不清楚,这气功治病的事楚云飞也听说了不少,还专门去收集过相关的资料,可考证的结果是:恐怕这东西终究是要归到伪科学那类,就像牛顿试图证明的上帝。

不过,事到临头,已经由不得楚云飞多想了,死马权当活马医吧,他盘起腿就在刘宁身边开始打坐。

有门!楚云飞刚进入那种“伪先天”的境界,就有一种明悟出现在脑中:身边这个人的生命正在不断地流逝中!

怎么会有这种感知呢?楚云飞非常地疑惑,不由得细细琢磨起来,可,这种感觉真的是没什么依据的,但,他确实是感受到了,生命力正像潮水般地一浪接着一浪地远离刘宁!

不行,一定要弄明白是怎么回事!楚云飞发狠了,那……该做点什么呢?自己眼中的刘宁是实实在在的,可那种生命的感觉,怎么可能看到呢?

看不到,那怎么办?要不试试“天眼”?

“天眼”是气功界的一个古老传说,可以说是一种感知的能力,据说开了天眼的人,能看到很多旁人看不到的东西,甚至能看到内气在体外的运行、内丹在体内的养息。很多流派有更离谱的说法:天眼分好几个档次,最高档次可上穷苍穹,下视黄泉。不过楚云飞试过多次,却连最基础的境界都从未做到过。

那就再试试好了,楚云飞正视着刘宁,眼神却开始漂泊不定,慢慢地感知着周围细小的变化,自由地放松眼皮,任眼皮缓缓地收拢……风在吹,树在动,两米外有几只小虫子在爬,远处塔塔的头在时不时地慢慢垂下又猛然抬起,可,这生命的影子在哪里?

年轻自然有年轻的好处,可养气功夫绝不是年轻人的强项,深沉如楚云飞也是这样,战友在身边生命垂危,他坚持了几分钟就有点心神恍惚了,不能再这样拖下去了!他缓缓地睁开了眼睛。

就在心境动摇的这一瞬间,楚云飞看到了!他看到了生命的影子!!!!

\section{第七十三章 生命的能量}

就在那星驰电射的一瞬间,生命的影子,出现了!

其实并不是看到的,它还是楚云飞的一种感悟,但却与刚才的感悟有天壤之别!楚云飞说不出他是凭了什么能做出如此判断,也表达不出究竟这是一种什么样的感知能力,虽然他的文学造诣很高。

虽然无凭无据,但楚云飞真的敢说自己看到了,尽管只是那短短的一瞬,可他甚至可以说出生命的颜色,是黄色的!一种明黄色的,恍惚的,没有固定形状的东西,似乎还有点光泽。

同样是在那短短的瞬间里,楚云飞感受到了那明黄色自刘宁的身体中四散逃逸而出,化做星星点点,如流星般拖曳着短短的尾巴直上夜空而去,而且绝大部分是向着同一个方向,其余的一少部分居然……向自己扑来。

还是在那一瞬间,楚云飞“看”到了刘宁的身体所显示的生命能量已经相当低了。如果可以量化形容的话,旁边成树国的生命能量起码是刘宁的三倍,至于自己的生命能量,似乎非常非常地强大,大概是成树国的……一百倍、五百倍?

一阵迷惘后,楚云飞又开始打坐,这次他没有刻意地考虑或者搜寻什么东西,如果真有那么灵验的话,还是感知一下怎么救治刘宁吧:自己生命力这么强大,能分点出去么?

嗯,不错,又有点意思了,有点类似刚才一开始的感觉,虽然是模糊的认识,却让楚云飞精神大作:似乎救治刘宁,该是件很简单的事。

那么,接下来该怎么做呢?楚云飞还没高兴半分钟,又开始犯愁。

成树国不敢打扰楚云飞,可又实在着急,像热锅上的蚂蚁一样走来走去。楚云飞看在眼里,终于狠了狠心,拉倒吧,大不了就学学武侠小说上的情节,从丹田注入内气吧。

想到就做,楚云飞挪挪身体,凑得更近点,然后就把手放到了刘宁的丹田上,一接触,手的感觉就是松皱的皮肤,明显是过度脱水,楚云飞不由得心一酸。

楚云飞强自镇定精神,开始尝试着把身体里内气通过手心向刘宁的丹田输去,可那内气实在是不太好控制,没有一点驯服的意思,总在手心打转,就是不肯离开楚云飞。

实在是没办法,楚云飞已经把内气隐隐地理解为生命能量了,这东西不肯听话,那只好加逐渐地大意念了。

当意念慢慢增强,最让楚云飞担心的事终于出现了:临界点一过,内气汹涌澎湃地涌了出去,目标就是刘宁的丹田。

完蛋!楚云飞当时就是这么个念头,这不跟自己用内气打了对方一掌一样么?还好,劲道不是很足。

刘宁果然浑身猛地一颤,手脚也随之抽动两下,急促地呼吸了两下,竟然似乎就再没了气息!

楚云飞当时汗就下来了,不停地给自己加油:镇定!一定要镇定!

成树国在旁边脸色登时变得刷白,可他还是不敢说话。

楚云飞仔细感受了一下,那种生命的味道还在刘宁身上停留着,我一定要给刘宁分出去点生命的能量!一定要分出去!

这么想着,楚云飞并没有把手从刘宁身上拿开,因为他有种感觉:只有这样才救得了刘宁!

楚云飞不停地强烈要求自己把生命能量传送过去,手上的内气还在试图输入对方的丹田,却再不敢增加力道了。

就这么持续了有三几分钟的时间,可在楚云飞却有一个世纪那么久远的感觉,终于,一种很玄妙的感受发生了,老天开眼:似乎生命力真的在通过掌心向对方丹田输送了过去。

真的是很玄妙的感觉,玄妙到楚云飞根本无法用语言来表示,而且随着时间的推移,楚云飞的意念也在加强,那种生命力已经不满足于仅仅从手上传过去,逐渐地在向楚云飞的全身蔓延。到了最后,楚云飞居然感觉自己就像一个全身在发光的灯泡,生命力向四周澎湃地散射着,然后拐了不同弧度的小弯,扑向面前的刘宁。

不过同时还有种让楚云飞非常郁闷的感觉,就是那些生命力并不能在刘宁身上呆多长的时间,又大部分流失走了,为了挽救那部分流失的能量,楚云飞不得不源源不断地向刘宁补充着生命力。

成树国在旁边已经看傻了眼,他虽然对楚云飞层出不穷的新花样已经有点习以为常了,但一个人能在半夜里隐隐地变得像个灯泡一样发亮的话,对普通人还是有相当的震慑力的。

其实用灯泡形容现在的楚云飞也有点不太恰当,他其实身上并没有发出什么光来,但成树国感受到了,就像中午的沙漠,被地表烤热的空气人们虽然看不到,但人们能看到物体的外型因空气的流动被扭曲。

楚云飞现在给成树国的就是这个印象,因为在火光的影射下,楚云飞居然变得有些透明了,……也不是透明,是有点发绿,而身体外形的边缘却有些发黄,同样的也有一些不规则的扭曲!

这个……就是能治病的气功么?好神奇!受到楚云飞那种不知名的生命能量的影响,成树国的心态也变得平和了很多。

而楚云飞这时已经有点不太妙了,生命能量在不停地外流,虽然刘宁的生命力也在因此逐渐地增强,那种能量的流失也变得似乎缓慢了一些,可这到什么时候才是个完啊?连身旁的成树国和琳娜都偶尔能接受点能量,可自己只有不停付出的份!

越来越深的无力感涌向了楚云飞,还伴随着丝丝的昏睡的欲望,楚云飞虽然明白自己不是真的没力气了,也不是真的想睡觉,可,这是何其相似的感觉啊!难道说,这是支付生命力的副作用?

楚云飞是真的有点累了,不是身体累,也不是心累,如果说意念是种确实存在的东西的话,那……也不是意念累,可是,他确实是很累了!可这莫名其妙的累在成树国的一声惊乎中又显得不算什么了,“云飞,刘宁……宁哥他好象嘴唇在动啊!”

令人振奋的好消息!拼了!楚云飞也顾不得再多考虑了,生命能量源源不断地涌出,他甚至没有注意到手心的内气已经在缓缓地向对方丹田流去!

楚云飞的神智慢慢地恍惚了起来,他不由得想起了有人似乎对他这么评价过:楚云飞非常厉害,厉害到打人能打得自己昏迷了。那么,这次,会不会救人救得自己昏迷过去?不过想归想,事关战友的生死,那可来不得半点含糊。咬牙坚持吧!

楚云飞最终还是昏了过去,在昏迷前,他脑中还有一个隐隐约约的念头:生命力这么挥发下去,别明天早晨起来自己变成个白头发老头吧?

\section{第七十四章 遇到多特人}

这一切的发生,都是在一种很微妙的环境下,楚云飞其实根本不知道自己到底做了些什么,至于成树国等人,连“其然”都不知道,更别说理解“其所以然”了。

不过生命的奥妙,又怎么是几个年轻人能够轻易地了解的?实际上,放眼世界,遍数古今,又有几个人正常人敢说自己了解了生命的奥妙?当然疯子和白痴不在此列。

其实刘宁的病情并没有他们想象的那么严重,他只是在越境的战斗中头部受伤,被一种非洲的一种不知名的病毒感染了,简单的包扎并没有把病毒杀死。然后他又跟大家一起赶路,没有得到很好的休息和食物补充,身体抵抗力自然有所下降,等到和索度巡逻队剑拔弩张地对峙时,刘宁表面上虽然是波澜不惊,沉稳得很,但实际上加速流动的血液大大加快了病毒发作的速度。

“病来如山倒,病去如抽丝”说的就是这种情况,病毒一旦发作,来势是极其凶猛的,开始成树国如果能果断地大剂量使用药物,把病毒发作的苗头压制住,那刘宁实在是没可能去那阴阳界上玩耍一番的。不过这点实在怪不得成树国,都是二十郎当的小伙子,平时有个小灾小病的连药都不可能吃的,成树国已经算是很谨慎很负责了。

等到后来病毒排山倒海地发作的时候,成树国再加大药量就有些晚了。不过还好,算是及时抵挡住了病毒的攻势,挡是挡住了,但是化解这头波最凶猛的攻势可是个漫长、持久的过程,来不得什么含混的,同时,病毒的威力也会最大能力的体现出来,抗过这次,一切都好说,抗不过去,那就什么也不用说了。

成树国后续的救治手段并没有什么大的失误,严格地说在那种环境下,他每步做得都是再正确不过了,就算是这样,刘宁还是堪堪地抵挡不住病毒的攻势,幸亏有楚云飞这么个怪物在场,事实上,哪怕耿风或者废人关来了,也不可能做到楚云飞这一步,因为……那实实在在的不是用气功救治的。

不管怎么说,三人里唯一是正职的小队长的生命在大家的不懈努力下,终于从阎王的手中侥幸地逃脱了出来。

还好,楚云飞并没有变成个白发苍苍的老人,他悠悠醒转的时候,已经是第二天将近中午了。刘宁在那里运气打坐,成树国在火上烧烤那条没头的响尾蛇,还有不知道从哪里弄来的几只洗剥干净的好象是啮齿类的小动物,塔塔躺在一堆干草上呼呼大睡,而琳娜……在拿着望远镜放哨。

看到楚云飞醒来,成树国先扑了过来,“操,云飞,我早就说,咱们仨里,还是你最牛逼,跟我说说昨天是怎么回事。”

刘宁听到声音也站了起来,“醒了,听说昨天多亏你了,现在我这一运气,内气像坐上飞机一样,刷刷地跑啊,不服不行,师傅就是师傅。”年轻真的很好,昨天那个奄奄一息的家伙不见了,现在的刘宁除了有少许憔悴,居然很有种活蹦乱跳,生猛海鲜的味道。

楚云飞看到刘宁的样子,就知道那一切辛苦……总算捞回本来了,不过,刘宁的气强了很多?那我的呢?

随便地笑了一下,楚云飞站着开始运自己的气,还好,没什么不妥当的地方……不对,怎么进不了“先天境界”了?

楚云飞不甘心,马上盘腿坐下,刘宁和成树国见此情景也不说话了。

不错,努努力还是行的,不过真的有种空荡荡的感觉啊,楚云飞放下心来,慢慢睁开眼睛,却看见成树国和刘宁趴在他面前仔细盯着他看。

这天五人还是没能继续赶路,没办法,大家都太累了,强烈缺乏睡眠。而埋头大睡的塔塔在夜里就成了最不幸的一个,他是放哨时间最长的。

楚云飞是最晚休息的,又是第一个醒来的。倒不是他精神亢奋,恰恰相反,因为救治刘宁时花费了太大的精力,楚云飞的身体状态不是很好,体内的气也隐隐有滞涩的感觉,先天境界也不是很容易能到达了,就算勉强到达先天境界,气势上和感受上都差了很多。他觉得处在目前这么个非常时期,还是抓紧时间,尽快恢复的好。

五人吃过早饭后,又开始向西北方继续前进,走了将近三个小时,前面出现了一大片开垦过的农田,大概附近会有个部落吧,远远望去,地里并没有什么人,不过,地旁边有棵小树,不远处间小屋,茅草盖的,大概是看守作物的人住的。

楚云飞做个手势,让大家藏起来,自己也把随身携带的两支步枪和背包放下,一个人向屋子走去,不过,这早春的时节里,屋子里未必会有人吧?

很幸运,屋子里真的有两个人,还都是男人,正躺在床上呼呼大睡,如果那土坯加上干草的东西能被称做“床”的话。

楚云飞上前喊了一声,就有一个睡得比较轻的人睁开眼睛,抬头看了一下,马上清醒了,“白人?怎么会有白人在这里?”说完使劲推推旁边还在睡觉的同伴,那是他的哥哥。

楚云飞实在受不了屋里那熏人的脚臭和腋窝下大汗腺分泌过剩所产生的恶劣气味,走出了小屋,“我在外面等你们。”

过了大概五分钟的时间,两个黑人出来了,这俩人躺在那里看不出来,站着就很有些彪悍的味道了,个头都很高,四肢虽然不是特别发达的那种,但也充满了爆发力。

楚云飞先开口了,“你们是什么人?胡图人?图西人?还是多特人?”

那后醒来的汉子头发比较短,很好辨认,他没回答而是反问了一句,“你问这个做什么?”

楚云飞懒得费那么多功夫,直接了当地回答他,“我要找个多特人的部落,跟你们打听一下,看谁知道,能给我带路的话,我出二十美圆酬金。”说完,从怀里拿出几张美圆晃了一下又收了回来。

那俩黑人对视了一眼,都看到了对方眼中的贪婪:到嘴边的肥肉不吃,老天都会惩罚的!

还是短头发说话了,“我们就是多特人,多特人的部落我们都知道,你想去哪里?我们带路!”

\section{第七十五章 师出要有名}

楚云飞没感到意外,按照塔塔的说法,胡图人和图西人的部落都迁走了,应该留下的都是多特人了。不过,他也不想弄出什么不愉快的事,所以还是谨慎了一下。

楚云飞向四周看看,没有什么明显的好的目标,那只好就是它了,右手一抬,他的手中已经多出了把手枪,枪声响起,一根手指粗细的树枝应声落下。

楚云飞并没有着急把枪收回来,而是继续在手里把玩着,眼睛紧盯着短头发,“你确定你知道所有的多特族部落吗?”

听到枪声,两人都是一哆嗦,等看到树枝落下,又见到对方拿着手枪在手里把玩,那俩人都是一个念头:搞错了,这不是肥肉,是雄狮!

还是那短头发的机灵,既然形势分析错误,马上愿赌服输,“嘿嘿,白人朋友,我们开个玩笑而已,开个玩笑,我们是给多特人干活的,我们可以帮你把主人喊来,他一定知道的。”他还是没说自己是哪族人。

哦,原来是这样!对这两个明显没安好心的家伙,楚云飞也懒得搭理,事事都要计较日子还过不过了?“行,那你留下,那个,你去把你们主人喊来,如果能不惊动别人,我给你俩五美圆报酬。”

那头发略微长点的犹豫了一下,又去看那短头发的脸色,短头发可不耐烦了,“还不快去?等着人家反悔呀?”

很快长头发就把人领了来,不过不是一个人,是三个人,长头发很尴尬地解释着,“嗯,这个……白人朋友,主人的身份尊贵,所以,所以外出时一般都有人陪着,保护主人,我并没有惊动其他人。”

楚云飞点点头,拿出钞票,找了五美圆出来递给长头发,“好了,这是你们的报酬。”

做主人的果然有做主人的气度,来的多特人看着两人拿钱,并没有表示异议。等看到楚云飞的眼光转了过来,中间年纪稍长的那位说话了,“你好象是亚洲人?”

楚云飞略微迟疑下,还是点了点头,虽然他并不指望所有的黑人能都把自己当作“白人”,不过对于这么直接被人认出来“原产地”的场面,他还是没做多少准备,“是的,我……是出生在亚洲。”

不过那主人显然没有计较那么多,他在为自己的眼光沾沾自喜呢,向左右晃晃脑袋,“看到没有?别一听说白人就以为是欧洲人或者美国人。”那两个跟班忙不迭地摇头表示佩服。

那主人自我陶醉一番才想起正事来,“亚洲朋友,听说你在找个多特部落?”

楚云飞饶有兴趣地看着对方的表现,一种久违的安定的感觉涌上心头,自从逃离军营后,这种感觉很久没有出现了。听到对方问话,他点点头,“是的,我在找个叫做‘提坦卡布’的多特部落。”

那三人尚未回答,那短头发业已发出一声低低地哀号,“天哪,你要找‘提坦卡布’,为什么不早说?我的二十美圆啊!”

主人模样的人瞪了短头发一眼,摇摇头,“‘提坦卡布’我自然知道,离这里不是很远的,不过,你愿意告诉我你去那里要找谁么?”

楚云飞自然不肯告诉他自己要去找谁,事实上,这次大家去找刚贝拉都不知道会产生什么后果。虽然此人答应过帮忙,但却是没兑现过,自己还抢了他的两辆汽车,本来答应了交还,这次也全丢在刚卡了,单从这点讲,自己这方做得未免有点不够上道。不过,以楚云飞的判断,这人应该还是可信度比较高的。

于是楚云飞故做为难地沉吟半天,“这个……我实在不方便说,我去那里有秘密任务,还请你不要声张,对了,我是美国人,亚洲只是我的出生地。”

说完他还装模做样地耸耸肩,很无奈地双手摊开,学足了美国人的样子。从内心深处,楚云飞也喜欢上了伪装美国人的感觉,一个超级国家,又是一个很霸道和护短的国家,冒充它的国民确实是很能享受到些便利的。

“是么?”那主人模样的人笑起来了,不过很有些恐吓的味道,“很抱歉,我的朋友,现在你只有一个人,怕是由不得你了,你必须说清楚。”说完向两边一努嘴。

那俩随从当时就从身后拔出了手枪,一把是意大利的贝雷塔手枪,另一把是不知道哪里生产的左轮,齐齐地指向了楚云飞,得了钱的俩黑人也摩拳擦掌地跃跃欲试。

藏在六十米开外的刘宁有点生气了,本来说得好好的模样,怎么一转脸就变卦了?他当下就把胸前的八一步枪挪到了眼前,拉开了枪拴。成树国看他这个样子,摇摇头,“放心吧,这几个还不够他塞牙缝呢。”

楚云飞本来是有机会在二人拔枪时出手的,可他实在没把面前这俩随从放在眼里,就由着他们去了,其实内心深处,楚云飞还是受到了从小接受的教育的影响,总觉得“师出有名”的话,自己做起事来会更理直气壮点。这点又不太象他正在扮演的“美国人”了。

不过,这主人的嚣张劲楚云飞可是看得不太顺眼。以楚云飞的理解,有事没事地嚣张跋扈,那是头脑简单的草包才会做的,那种人楚云飞甚至没兴趣去看一眼;可眼前这种自鸣得意的伪君子,他是不太喜欢的,拜托,你嚣张也要看个对象,高兴得这么早,那不是小看人么?

于是楚云飞顺手做个套子等对方钻,也能多套点话出来,“我可是美国人,你们这样做要考虑清楚后果!”

那主人如演戏一般,非常配合地接过了台词,言语之间洋溢着强大的自信,“如果你不配合,这里会从来没有出现过美国人,我个人是非常爱好和平的,也希望你能看清楚形势,不过,这里死个把人根本不会有人知道,至于他们……我想你身上应该还有不少的美圆吧?”

看着对方自以为得意的样子,楚云飞越发地觉得此人非常欠扁,不过,都到这一步了,还是能多套就多套几句话吧,“要我说也可以,不过你要先告诉我,你为什么对‘提坦卡布’部落这么维护,这是我的原则。”

那主人越发地找揍,居然开始耻笑楚云飞,“你以为,现在你还有坚持原则的能力么?”

\section{第七十六章 刘宁的枪法}

“不过,”那主人口风一转,“看在你是个即将失踪的人的份上,我可以告诉你原因,因为,我的女儿马上要嫁到‘提坦卡布’部落了,我很愿意为自己的女婿提供点帮助,帮他踩死一只可能蛀坏房子的小小的白蚁……现在,你可以给我个不杀你的理由么?”

哦,原来是这样,楚云飞笑了,自己还真找对人了,眼前这家伙的女儿肯定嫁给了“提坦卡布”部落的重要人物,要不,这家伙根本没有必要随便地对付一个“白人”。

既然算得上是自己人,那一切就都好说了,不过,这口鸟气还是要出出的,要不岂不是会被人小看?反正这非洲人也是只尊崇强者的。

想到这里,楚云飞伸手打了个响指,希望刘宁能理会他的意思吧,帮他解决掉一支枪,那另一支自然就不在话下了。严格地说,其实刘宁打固定靶的水平还略微比楚云飞强点,只不过出枪没他快就是了。

响指一出,楚云飞顺势就扑下身子就势一个前滚,八一步枪的枪声也在这时响起,左轮手枪应声被远远击飞,楚云飞长身而起,手枪已经顶在了主人的太阳穴上,顺势又是一脚踢飞了“贝雷塔”,口中学着那主人的语气,“你以为,我只有一个人么?我们美国人可是不爱撒谎的,我也确定自己从没说过我没有同伴。”

这还真是六月的债,还得实在是快,五个黑人实在是没想到眨眼间强弱已经易势。等众人惊魂初定,成树国已经自藏身处走了过来,两支步枪在他身上吊儿郎当地一晃一晃着。

那主人命在人手,还在那里嘴硬,“朋友,你最好弄明白,这里离我们部落很近的,想想后果吧,你现在放开我,咱们可以友好地谈一谈。这只是个小小的误会。”

成树国走过来,他的目标是那个短头发,“你收了我们的钱?”

那短头发以为“白人”要反悔,这怎么可以?再说钱也不是你给的,所以根本没理会成树国,反正现在枪没顶在自己身上。

成树国笑了笑,舔舔嘴唇,走过来就是一记飞脚,那短头发想做反应已经来不及了,头部被踢中,打着滚就摔了出去,当场晕了过去。

“这个人很没礼貌,我不喜欢他,”成树国笑嘻嘻地说着,露出了一口白森森的牙齿,再配上那话里的寒意,那四个黑人都泛起一种不妙的感觉。

成树国没理会大家的反应,“收了钱就要做事,怎么我看他刚才想对给钱的人动手呢?这主意谁出的?”

那三人眼光不由自主地溜到了那主人身上,做主人的不答话都不行了,人家再来一脚怎么办?

“是这样的,白人朋友,我们,我们之间似乎……似乎有点小小的误会。”

成树国也不知道该怎么接话了,因为他藏在那里只能看清楚事情发生的经过,双方的对话他可是听不到的,做完了恶人,该云飞说话了吧?

楚云飞自然明白战友的配合,“我要他给我找个向导来的,似乎你们愿意挣这二十美圆?”

只要不计较刚才自己杀人的动机,那主人是相当情愿的,“是啊,我们非常,非常愿意为你们带路。”

楚云飞可不愿意就这么放过他,这个当主人的似乎很有钱啊,“可是你刚才说的话让我很生气,你是不是该做出点什么补偿?”

“补偿?”那主人愣了一下,“应该的,应该的,这个带路费我不要了还不行么?”

开玩笑,杀人未遂交二十美圆就够了?楚云飞自然不会答应,“想想你刚才说了些什么,那些话我要传到美国政府的话,不知道会有什么后果呢?”

听到这话,成树国有意无意地瞟了楚云飞一眼,这个“民族主义者”可是很不喜欢楚云飞这么做的,可眼下这情形……他能做的也就是瞟战友一眼而已。

说到这个那主人却是不在乎,“我只是说说而已,又没造成什么后果,再说,美国人我也认识几个,找个帮我说话的也不是很难。”看谁唬得住谁吧。

敲竹杠敲到铁棍上了,楚云飞有点点尴尬,刮了刮自己的鼻子继续找理由,“那么,我受到惊吓了,我要求赔偿损失。”

你受到惊吓了?那主人暗骂一声,我受到惊吓了才是真的,你一直就那副不阴不阳的样子!不过,命悬人手,他自然不能说太过分的话,“这个……合理的赔偿要求,我是可以考虑的。”

“哦,很好,”楚云飞又笑了,他放下了顶在对方头上的手枪,“早这么说不就完了?大家都是文明人,什么事都好商量的,这样吧,我的要求很简单,你这个主人陪我们走一趟‘提坦卡布’部落好了,对了,弄两辆汽车来,走路实在太累了。”

那主人心里已经把那长短头发的兄弟俩恨到了骨头里,没事给自己招惹这些“白人”做什么?自己也是,听到“白人”就出来凑热闹做什么?千不该万不该的就是没事给自己的女婿献什么的殷勤?“提坦卡布”那么大的部落,轮得到自己给人家操心么?这下好了,该自己给自己操心了。

楚云飞开出了价码,那做主人的自然要讨价还价了,再说,尽管那主人确实是有钱人,可有钱人也分着三六九等呢,“部落里还有事,我实在是走不开啊,我派人领你去不就行了?至于这汽车,你们就俩人,何必要两辆车啊?对了,这汽车你不是要拿走吧?”

看看时间不早了,楚云飞也懒得继续涮人玩了,“谁告诉你我们就俩人的?还有你,必须跟我们走,没有商量的余地,至于汽车,你们要表现好得话,我自然是可以还你们的,我们美国人缺那点钱么?对了,顺便给我们弄桶喝的水来。”他还真演戏演上瘾了。不过没办法,为了增强可信度,他必须这么说。

那主人尚未来得及继续讨价还价,又是一声枪响,原来那短头发悠悠醒转了,见大家没注意他,而那把“贝雷塔”就在他面前不远处躺着。正午的阳光下,黑里透射着瓦蓝光芒的枪身给人以极大的诱惑,那短头发就想偷偷去拿那把枪。可他身子刚刚一动,远处一直在观察的刘宁就注意到了,一枪命中“贝雷塔”,将它远远击开。

\section{第七十七章 守护光明的狮子}

大家还没反应过来,成树国又龇牙一笑,“你很让人讨厌,知道么?”上前几步,一脚踢到短头发的头上,那短头发真的是够倒霉的,居然又晕了过去。

这次成树国没再放过他,把昏迷不醒的短头发用左手强行拎了起来,右手当胸一拳把对方打得飞出四米开外,然后又踏步上前打算继续殴打,那短头发嘴角却已经渗出了丝丝血渍。

“好了,”楚云飞知道成树国只是在为他造势,并不是想真正的杀人,于是及时“制止”了成树国,转头向那个主人解释道,“抱歉,我的同伴脾气不好,惊吓到你了,还好,他也不是脾气最坏的一个。”

那主人在听到枪响后就是一惊,待到发现不是在场的任何一人开的枪,马上就知道了对方说的是实话,他们确实不止两个人,而成树国接下来的恶劣表现更是看得他心惊胆颤:这帮人好冷血!

至于楚云飞说的他也不是最坏的一个这种话,虽然属于赤裸裸的威胁,但还真保不准对方还有更心狠手辣之徒,想到这点,那做主人的更是郁闷得想死。

最后还是遵照楚云飞的指示,那主人叫个随从去弄了两辆卡车和一塑料桶水来,成树国如影随形,一直跟着那随从,严防他使坏。

最终的结果就是所有人都上了卡车,一路硝烟滚滚而去,楚云飞跟那主人和两个随从一辆车,其他人又一辆车,那兄弟俩本来是不愿意上车的,短头发还在头晕脑涨,自然不敢乱说话,可长头发居然很有勇气地计较,“大人,没我们什么事啊,我们就不用去了吧?”

没你们什么事?想得倒美,把你们留下好方便通知“提坦卡布”部落,以为我们都是角马不成?楚云飞现在思维的方式也越来越非洲化了。他冲成树国微微点点头,成树国就要走过去动手。

不过这次可用不着成树国出面,那做主人的早嚷嚷上了,“你们两个没用的废物,让你们走就走,多说什么?能跟我在一起,那是你们的运气。”这话他可是说得很有私心,首先,这档事全是由这俩废物引起来的,紧要关头自然不能让他们置身事外;再说了,此一去吉凶祸福实在是不敢断言,多拉俩人壮壮胆子也是好的。

坐在卡车的后马槽里,楚云飞笑嘻嘻地看着咬牙切齿的主人,“罗布,‘提坦卡布’离这里有多远?”

罗布自然就是那主人的名字,他本来想坐在前面的驾驶室的,不过楚云飞怎么可能让他如愿?“大概……六十公里吧。”

“哦,”楚云飞点点头,“那用不了多长时间啊,你不用这么看着我,要不是你要打我主意,我也不会这么对你的。”话是这么说,可当那主人表示出对“提坦卡布”部落很熟悉的时候,楚云飞早就想好了要把这几个人一起架上走。在刚贝拉的态度没有确定以前,楚云飞是不会留给对方太多反应时间的,那自然不可能放过有通风报信可能的任何人。

罗布听了这话,也不得不承认自己做事实在有点欠考虑,太得意忘形了,怎么就没注意观察一下环境?

看到罗布不吭气,楚云飞又“耐心”解释了起来,“其实,你随便派个人出来,他就挣了那二十美圆了,不过,以你当时的态度,我很担心你会影响我的秘密任务,那只好委屈你跟我们走上一趟了。再说,我们美国人是那么好欺负的么?”

罗布一听说没有生命的危险,自然情绪好转不少,虽然无法考证楚云飞的话到底有多少真实度,但对方既然愿意这么不厌其烦地解释,那多少感觉还是有点诚意的。

于是两人“亲切”地拉起了家常,罗布自家人知道自家事,其实他们部落间的关系是众所周知的,根本瞒不过有心人的查探。既然对方已经脱离了他的控制,那事情是很快会被调查清楚的,于是也不再遮掩了。

原来罗布所在的部落叫“提坦脱司”,意为“守护神灵的狮子”,跟那个“守护光明的狮子”的“提坦卡布”实在是渊源久远的,据说是谊属兄弟的部落,都是多特人里出名强大的部落。不过,近百年来,“提坦脱司”部落由于地理位置的缘故,屡屡被胡图族和图西族的战争所侵扰,有不少家族居然离开部落去讨生活,势力已经大不如前。

而“提坦卡布”部落可就不同了,一来有人为其挡灾,并没有受到战争太多的骚扰,二来由于其靠近石油区很得了些好处,三来由于石油产业的开发,挨着他们的索度沙漠也有从侵扰族人的噩梦变成旅游胜地的趋势,日子过得要滋润得多。

罗布本人是“提坦脱司”部落一个大分支的族长,在部落里很有威望,也很有些权势,眼下,全索度的经济情况都在好转,他的身价也自然是水涨船高。

看着罗布越来越放松,楚云飞出其不意地丢出了准备已久的问题,“哦,那你的女儿是要嫁给‘提坦卡布’的哪个人呢?”——别是要嫁给刚贝拉吧?

罗布很自豪,“那自然是‘提坦卡布’的继承人刚斯了,我女儿非常漂亮,出身也高贵,难道不是么?”

钢丝?楚云飞觉得这名字有点好笑,对特种兵来说,那是种不错的武器呢,不过罗布既然这么爽快,楚云飞也不打算再隐藏什么,“刚斯?为什么不是刚贝拉?”

“刚贝拉?”罗布嘴角撇了一下,“他也能算继承人么?咦,你怎么会知道刚贝拉?”

“哦,没什么,我在你们的首都喀津霍见过他,听说他就是‘提坦卡布’酋长的儿子,很可惜,没有留下他的联系方式,要不,你以为我会闲得找你们带路?”楚云飞的话总是半真半假的。

“继承人根本轮不到刚贝拉,就算刚斯不愿意做酋长,还有刚凯瓦呢,单说年龄也轮不到他,‘提坦卡布’的酋长古基斯有十二个儿子呢!”罗布很轻易地就相信了楚云飞的话,没别的原因,刚贝拉的生母地位很高,他本人确实在提坦卡布部落也很有名气,不过,也仅仅如此,那酋长的宝座可轮不到他坐。

除了“钢丝”,还有“杠上开花”?楚云飞摇摇头,这都是些什么名字啊?“你的意思是说,刚斯的年纪比刚贝拉还大很多?我看刚贝拉都有四十岁了,你女儿多大了?”

罗布点点头,“你说得不对,刚斯才四十二岁,刚贝拉?估计才过三十岁吧,我女儿十六岁,跟刚斯很般配呀。”

\section{第七十八章 再见刚贝拉}

楚云飞摇摇头,发誓再也不去问这种难以理解的问题,反正似乎……也没太大的必要,以后只从利害关系考虑好了。

罗布看到楚云飞不吭声,以为他被自己和女婿的身份吓住了,很是得意,继续让楚云飞“弄清状况”,好叫他对自己识趣点,“我女儿即将成为刚斯的第五个妻子,她是最年轻、最漂亮的……”

楚云飞就见不得罗布那种若有若无的,优越感很强的嚣张样,也许是对方给他的第一印象太差了点吧?弱智倒不是你的错,可你不该非要装出副睿智的模样啊。

再说,对方底牌自己都知道了,自己的底牌也翻一下吧,顺便小小地吓唬一下这个讨厌的家伙,楚云飞扬扬他那不太浓的眉毛,做出个遗憾的表情,“罗布先生,很不好意思,我们是去找‘提坦卡布’部落的麻烦去的。”

罗布的脸色登时就变了,说实话,楚云飞很久没有看到变脸变得这么快的人了。等了片刻,罗布又笑了,“开玩笑,去找‘提坦卡布’的麻烦?就你们这三个人?再加上那俩刚卡的穷鬼?”

楚云飞的好奇心又起来了,“咦,奇怪,你怎么知道那俩是刚卡人?”

罗布的心思可不在这上面,他刚才那话是想引对方说下文的,结果对方却是全然不在意的样子,“你们看不出来,我可能够看出来。口音上有小小的不同,关键还是俩人的打扮和动作习惯。他们的气质也差得太远,那穷得发臭的味道,我隔一英里也能闻到。你们这几个人就想找‘提坦卡布’的麻烦?”

楚云飞可懒得跟他说与刚贝拉之间的恩恩怨怨,反正该知道的也知道得差不多了,“找不找他们的麻烦,要看我的心情了,也要看他们头脑清楚不清楚了,”说着似笑非笑地瞟了罗布一眼,“如果他们也很冒失的话,那个结果就不太好预料了,不过,我还是那句话,你以为我们就这么几个人?”

哦,原来如此,罗布总算知道了他想了解的东西,这几个人果然是先头部队,还有大部队在后面。

答案虽然很合理,但这不是罗布想要的,想到这里,他是越发地后悔:都知道美国人不好招惹了,自己还要强出这个头,以后一定要循规蹈矩地做事,坚决不能再头脑发热了。

罗布根本想不到楚云飞是在胡说,更不知道楚云飞这么说并不是随便吹吹牛,而是自有他的打算的。

楚云飞一向的做事风格就是精打细算,趋利避害,这次也不例外,他仔细地考虑了和刚贝拉打交道的过程中可能遇到的种种问题。自己三人现在最大的问题就是没有名义,在刚卡是人人喊打,来到索度虽然情况有所改善,但没有合法身份的外来人,自然不可能有什么稳固的落脚点。

这次来找刚贝拉就存在这么个问题,哪怕刚贝拉全答应了他们的条件,也提供了落脚点,可是……自己三个人敢住么?敢肆无忌惮地住进去?万一人家黑你一把怎么办?要知道,你们这三个流浪者是没有任何后盾的!

还好,楚云飞三人也有值得庆幸的地方,毕竟这里是索度了,关于自己三人的事,索度人就算再关心刚卡也不会有刚卡人知道得多,所以为了自保,楚云飞必须制造出一种假象:祖国并没有抛弃这三个犯了错误的年轻人,在刚卡发生的那一切,也不过是祖国照顾非洲兄弟一个面子而已。而这三个人一旦出了差错,是会有人来追究的!

这样一来,哪怕有人想动些什么歪脑筋,也不得不考虑将来可能面临的无穷的麻烦,还好,相对于索度而言,自己的祖国那算是个庞然大物了,起码吓吓人该是比较管用的。

想骗别人,首先要自己相信那是真的才行,其实楚云飞也希望事实离他猜测不要太远才好,虽然,指望国家机器的力量来保护自己是绝对不现实的。

反正,必要的埋伏工作楚云飞偶尔还是要做做的。

接下来的时间,一行人没有了说话的欲望,一个多小时以后,两辆车到达了“提坦卡布”部落的领地,在远离部落一公里的地方就停车了,楚云飞伸手招过来个随从。

“你进去找刚贝拉,就说刚卡的白人朋友来看他了,说我们还欠他两件事情,要他一个人出来,对了,告诉他,罗布先生在我们这里做客。”

等那人走后,楚云飞把罗布、剩下的那个随从、俩司机和长短头发六个人聚拢在一起,必要的防备那是一定要有的。不过,楚云飞还是找了个不错的借口,“大家都饿了吧,尝尝军用饼干吧,对了,树国,给罗布先生一块腌肉。”

“提坦卡布”部落的人来得很快,看着远处走来的三人,楚云飞伸了伸脖子,用力地咽下最后一口饼干,子弹上膛!

等到再近点,楚云飞看清了来的就是刚贝拉,旁边跟着的一人似乎上次在刚卡也见过。不过,楚云飞也知道自己对黑人的辩识能力非常普通。

刚贝拉看到了楚云飞,很热情地跑了过来。一个多月不见,这家伙似乎又胖了不少,跑起来脸上的肥肉也是一颤一颤的。楚云飞极力回忆了一下,嗯,旁边这家伙确实是在刚卡见过,似乎就坐在司机旁边的。

刚贝拉很热情地伸出手来,“哈,中国朋友,欢迎来到索度。”

楚云飞看他的微笑不像是假的,也微笑着伸出了手,“哈哈,你这句话直接就出卖了我,现在我再也不能冒充美国人了。”

刚贝拉早有准备了,“我说嘛,那家伙还说你们是美国人,想想就知道你是骗他们的啦。”

既然刚贝拉来了,再胡说什么都是多余的了,楚云飞很直接地质问对方:“你不是要接我们过境么?怎么一直不派人去?”

刚贝拉脸上露出了为难的神情,“这个……这个,你叫我怎么说呢?其实,我是想派人去的,不过,我这里最近确实是有事,实在忙得顾不上啊。”

\section{第七十九章 有落脚点了}

其实,刚贝拉这话说得是真假各半,他最近确实是在忙一件事情,这事情也相当让他头疼,可他还是有机会派人去接楚云飞他们过来的。

但是,楚云飞他们上次确实把刚贝拉吓坏了,长这么大,虽然生生死死的场面他也见识过不少,可那都是别人的生死。直到刚贝拉也在死亡线上转过一圈,他才明白,那绝对不是什么愉快的事情。虽然当时他答应了派人过去,不过也只是求脱生的一种手段而已。

等刚贝拉回了家,就开始犹豫到底派不派人去接中国人,急切间却下不了决定,然后就是他负责的事情出了点意外,刚贝拉索性就把这事搁置了,商人的天性暴露无疑:不就是两辆车么?回头再说吧。

可现在凶神到了门口,不接待也是不行的。那随从鼻涕一把泪一把地要求古基斯酋长出人,把罗布先生救回来,把那些该死的“中国骗子”和刚卡的穷鬼处死。古基斯酋长倒不算糊涂,起码他知道对他来说,中国人和美国人没什么区别,都是他惹不起的,而他肯派儿子去中国留学,自然对中国还是有些好感的,上次那些中国人不是也因为这个没杀自己的儿子么?于是就把事情的处置权交给了刚贝拉,“你惹的事情,你自己看着办吧。”

既然对方已经找上门了,刚贝拉也就索性接待了,反正在自己的一亩三分地里,他们总不能再做过份的事情了吧?虽然派人围杀也是种选择,但一来对方好歹是饶过自己的,二来楚云飞他们的强悍刚贝拉也略知一二,不付出相当代价那是不可能的,那么这么做就未免有点不划算了。再说了,在刚卡还无所谓,这几个人在索度被杀的话,谁也不能保证中国政府真的不追究这件事。

整件事情的过程确实和楚云飞意料的相差无几,不过现在显然不是说这事的时候,该解决的问题还没解决完呢,“很遗憾,刚贝拉,由于你没有派人去接我们,我们在越境时把吉普车损失了。”

刚贝拉脸上明显地露出了失望的神色,不过起码有一半是假的,“哦,那……我的父亲要处罚我了,你们难道不能多等一阵么?”

扯淡,楚云飞暗骂一声,如果刚贝拉没有表现得这么失望,他还真会觉得是给对方造成了损失,而正因为刚贝拉的演技纯熟,所以才让楚云飞看出了破绽。

你把我们搁在刚卡不管就不怕车丢失?我们要被刚卡政府抓住或击毙你的车就不会丢失?我们到了眼前,你就算不知道国境上加了铁丝网车过不来,也该看出周围没你的车吧?还装得这么失望?

不过这对楚云飞来说反而是个值得庆幸的事,刚贝拉既然要装,那他自有装的理由,也就是家乡话说的“礼下于人,必有所求”,不过想从楚云飞这里捞点便宜实在不是件容易的事,“多等一会儿?那种日子简直不是人过的,天哪,我没想到你居然会这么想,而且你看,为了保护你的吉普车,我的同伴头部受伤,差点把命丢了。”楚云飞指指闻声站起来的刘宁。

看到刚贝拉脸上悻悻的表情,楚云飞自然也不能再吊人胃口了,毕竟是有求于人的,“不过,我们中国人很讲情谊的,我们找你来,是为了完成承诺你的两件事情,因为,毕竟我们给你造成了损失。”

刚贝拉其实一直等的就是这句话,平白多点臂助总是让人愉快的事情,不过,商人的天性还是再次表现了出来,“哦,很遗憾,现在应该是三件事了,因为我的车不见了。”

楚云飞当场就敲定了,“没问题,三件事,不过,作为补偿,你需要提供给我们身份证明,要不实在是太麻烦了。还有,我们讲信用,来实现我们的承诺,你应该负担在这期间我们的生活费用吧?”不过就是多出了一件事情,不算什么的吧?

就这么你来我往中,楚云飞三人终于有了可以自由活动的身份。

所有事情商量妥当以后,一行人进入了“提坦卡布”部落,这个部落的人着实不少,大约有五六万人的模样,这中间的聚居地居然有四平方公里左右,简直就是个大镇了,后来楚云飞等人才知道,“提坦卡布”部落足有二十万人,这里只是部落中心。

部落里的路也是土路,不过房屋和道路规划得还不错,虽然有些地方通行的道路狭窄了点,但总体感觉却是非常地有规律。可见大部落确实有大部落的气度,看得出是曾经有人专门设计过的,虽然年代已经很久远了,那些房屋也都如中国一般由于住户人丁增加,多出了不少的衍生建筑物,但总体感觉的合理性还依稀可见。

部落在这里不但留下了前人的睿智,还记载了历史的传承和时代的变迁,让人有种怀古思今的回味,真的很不错的感觉,比坎塔卡强多了。

街道两边有一些商铺在叫卖东西,物品也还算得上丰富,还能见到别的族的人在这里现身,楚云飞甚至见到了个阿拉伯人,不过看上去不是纯粹的欧罗巴人血统,该是混有尼格罗人种血统的那种。

说实话,这里要不是“提坦卡布”部落中心聚居点的话,完全是有可能发展成个热闹非凡的集贸重镇的。正因为这里是“提坦卡布”部落的中心,所以部落严格地控制着外人的出入,这自然影响了商业的发展;从另一方面讲,也因为这里基本上全是多特人,种族太单一,又制约了别族商人来这里发展的欲望。

随着时间的推移和索度社会的发展,这些制约因素应该是能慢慢消除的。

现在楚云飞三人要操心的可不是这个,而是从哪里能找到个可以打回国内的电话,还是先给家里打个电话,报个平安吧。顺便还可以了解发生在自己身上的事到底引起了些什么后果。

邮局这里是有的,不过那实在是有点慢,既然有了条件,三人连一分钟都不愿意多等。

刚贝拉先在自己住的地方找了个单独的院子安排楚云飞三人,不过那土夯的院墙只有一米多高,别说是人,连狗都跳得过去。至于琳娜和塔塔,则被刚贝拉派人送到了个类似单身宿舍性质的地方,中国留学生已经答应收容这对可怜的情侣了。

三人连澡都没顾得上洗,虽然身上已经有些发臭了,放下行囊,马上拉着刚贝拉四处寻找电话。

\section{第八十章 两地一线牵}

不过等刚贝拉弄清楚三人的目的时,很明白地告诉三人,这里地方偏僻,电话虽然有几部能打索度国内,但能打国际长途的电话只是酋长古基斯大人那里有一部,大多数的电话还是那种部落内部联系的分机。哦,对了,那部装机容量八十门的小程控电话机就是刚贝拉同学从中国买来的。

那只好去麻烦酋长大人了,三人都是这个心思,可刚贝拉这下不敢再随便答应他们了,“我去问问父亲,不过,我父亲考虑问题很西方化的,所以,呃,这个……你们三个是不是先洗个澡会好一点?”

三人闻言,只得悻悻地回到屋里,还好院子里就有水瓮,虽然那水看起来有些浑浊,不过,在这靠近沙漠的地方,还能强求什么呢?

等到三人洗完澡,刚贝拉还是没有回来,于是无所事事的中国人又开始洗衣服,没有肥皂,衣服上的油渍不是很容易洗,不过衣服上主要是土和泥,洗洗总比不洗强吧?

又等了一阵,刚贝拉才回来,古基斯酋长倒是答应了他们打电话的要求,不过,因为三人身份敏感,酋长大人不便与他们见面,自动回避了。

成树国最先给家里打通了电话,索度这里虽然才下午四点,可国内已经是晚上十一点左右了,不过成解放当了一辈子军人,睡得很轻,电话铃响到第三声他就拎起了话筒,语气自然不会是很好,“谁?”

成树国还没张嘴眼泪就下来了,是啊,这个军人家庭出身的孩子虽然已经是相当自立了,可吃了那么多的苦,又被冤枉,而所有一切还不能对别人说,现在终于听到家人的声音,那份深藏在内心深处的委屈如大堤决口般地泄了出来。

成解放半天没听到回音,更加恼怒了,搁给谁在半夜被人叫醒心情也不会很爽的,“妈的,是哪个熊人?敢打电话不敢说话?”正要狠狠挂掉,却听到电话里有啜泣的声音,思索半天才若有所悟,“是……是……小国么?”

这句话不要紧,听得成树国差点哭出声来,还好,有战友在旁边,他还是控制住了自己,用鼻子发出了“嗯”的一声,然后过了好半天才简单地把自己的遭遇、受到的委屈、以及后来一系列的遭遇汇报给了父亲。

成解放听得火冒三丈,对着电话就嚷嚷起来了,把自己的夫人也惊醒了,他们只是通过官方渠道了解到了事情的发展经过,肯定是没有成树国这个当事人说得清楚的,没想到儿子受了这么大的委屈,不是命好的话,死了都不止一回了,成解放本来脾气就非常的火暴,这次更是,“小国别担心,你老子还活着呢,整人也不是这么个整法,回头我和刘群商量一下,好好出出这口气,你放心,你这事就是放在我们那会儿也不能说全是你的错,徐道士也架过机枪骂过娘呢,只要你别做错事,老子保证把你们弄回来。”

然后就是成母拿着电话的一番唠叨,时间要长得多了,等到成树国放下电话,半个小时过去了。

然后就是楚云飞给家里打电话,他本想让刘宁先打的,可刘宁说了,“我家老头子睡得轻,还是你先给你妈打吧。”这家伙就没想想,这个电话打过去,谁家的父母还睡得着?

楚云飞见到成树国的惨状,很是调整了一番情绪,可电话打通的时候,眼泪还是在眼眶里转来转去,他可不比成树国和刘宁,家里可就这一个相依为命的母亲了。

还好前期楚云飞给家里捎过封信,而楚云飞也没想加重母亲的思想负担,所以同以前无数个“前科”一样,睁着眼睛胡说了半天。叶美自然知道儿子在宽她的心,自己的孩子自己还不了解?他以为那哽咽的声音掩饰得真有那么好么?

不过,对楚云飞的能力叶美也是很放心的,虽然孩子在母亲眼里永远是孩子,但她也知道自己的孩子是很少吃亏的,要比他死去的爹强得多,所以,叶美也就是细细地叮嘱了一番,至于孩子听不听,这孩子早要听话还能成现在这样么?

同样的,叶美也想到了不能加重儿子的心理负担,所以家里有部队人来的事那自然是不能提的,反正他们的通知上也没让她尽什么罪犯母亲的义务。

所以楚云飞的电话是时间用得最短的,可是,再短的时间也改变不了三人中只有他是和母亲“相依为命”的事实的,所以说,很多时候表面现象并不能说明什么问题的。

等到刘宁给家里打电话的时候,家里居然电话占线,三人对视一眼:这肯定是成树国的老爸正“串联”呢。

不过,再打一遍的时候电话就通了,接电话的是刘宁的母亲,原来两个老军人商量了几句忽然想起来刘宁可能还要打来电话,赶紧换成部队的军线继续联系,刘母在电话前守侯可能来的电话。

刘宁虽然没表现得像成树国那样的丢人,不过眼泪也只是堪堪地没落下来而已,说了几句,刘宁居然把话筒递给了楚云飞,“你跟他们编一套吧,这事你熟,我不行。”原来刘母问起了他头上的伤势。

楚云飞暗自唠叨:这话照实说就成了,还考虑什么家里人担心?你不说得惨点,事情能好办么?他就没想想他自己可也不想让叶美担心。

刘母对楚云飞很热情,毕竟他现在是孩子生死相依的战友,不过她和楚云飞还没说两句,刘群就接过了话筒,“叫那混小子来接电话,咱爷俩回头有的是时间聊。”

三个人的电话总共打了一个多小时,期间并没有人进来做任何的骚扰,看来,古基斯酋长肯定是打过招呼了。

等到刘宁放下电话,却发现楚云飞在看手表,他和成树国对视一眼,同时想到了一个问题:电话费,完了,一个多小时的国际长途,这得花多少钱啊?这私人通信刚贝拉未必给报销吧?

不过,刘宁真的很反感楚云飞这小事上斤斤计较的性格,以他的眼光看来,楚云飞算得上是个优秀的军人,可细想起来,似乎更合适去做个奸商。

在三人偷渡过来以后,为了集中使用财产,成树国成为了财务总管,他细细地清点了自己的口袋,还好,还有将近八百美圆,应该够用的吧?

但三人后来还是被当天的电话费吓了一跳,居然用了五百多美圆,索度打到中国,一分钟要八美圆,而中国打到索度的话,一分钟还不到2美圆。

\section{第八十一章 刚贝拉的麻烦}

第二天,刚贝拉开始着手给三个中国士兵办理身份证明。不过,这身份办理很是有些周折的,以“提坦卡布”部落的能力,假的国籍和护照那是弄不来的,也就是帮楚云飞三人落了索度户口而已,可这事的办理也需要个时日的。

然后就是安置塔塔和琳娜了,刚贝拉倒是很尊重楚云飞他们的意思,可楚云飞三人和那俩也不过就是接触了四、五天,现在敌意倒是谈不上了,可也没什么交情,最多,也就是算共过患难吧,实在是拿不出什么合理的建议。于是,三人表示,对于这件事他们一律听从主人的安排。

可按多特人的传统来说,不同等级的客人携带的仆人绝对是应该享受对应的待遇的。虽然塔塔和琳娜其实算不上仆人,但安排过于繁重的的差使,总是对客人不够尊重,所以,俩人居然都被安排了很轻松的活,塔塔甚至能和传统非洲男人一样,基本上不用干活,跟个监工类似。享受待遇,必然要付出代价的,刚贝拉下午就来找楚云飞他们了。

“云飞,我现在有些麻烦,需要你们的帮助。”刚贝拉现在不同了,翻身农奴把歌唱了,言语间也不再那么谨慎,想到什么就说什么,不过他还算清醒,没想到要对中国人失礼。

这不?连名字都是按中国方式叫的,没照英语习惯来个“楚”。

刚贝拉说的就是他最近遇到的麻烦事。

刚贝拉本来是“提坦卡布”部落计划培养的商业人才,只是当时索度与美国关系不是很融洽,才选择了商业并不发达的中国,自然跟“提坦卡布”部落势力不济也有关,等他学成回来后,马上就负责起了部落商业运营的事。

由于最近索度沙漠有成为旅游热点的趋势,刚贝拉就做出了相应的旅游开发计划,由于是把多特人世代头疼的地方变成可以生金蛋的母鸡,自然他的计划很轻松地就被部落酋长和各支族长们批准了。

计划批准了,资金也渐次到位,可执行时遇到了麻烦。

麻烦出在一块绿洲上,绿洲本来是沙漠度假不可缺少的场所,那绿洲又有充足的水源,所以算索度沙漠外围的一块宝地。

既然是宝地,自然有人争夺,其实就算不是为了旅游业,单凭那水源,那块地方也是生活在索度沙漠边缘的部落所要争取的地方。

这地方本来离“提坦卡布”部落就不算近,而它附近又有个塔尔族的大的部落,历史上,两部落为了争这块地,不知道死了多少人,最后才定下个双方把绿洲水源五五分的和平协议。

已经是和平和发展的时期了,所以这次绿洲的开发,刚贝拉的计划是拉上塔尔人一起做的,至于分成,那自然要考虑双方的出资情况来定的,不过,反正是可以协商的。

可塔尔人不愿意,他们听完了刚贝拉关于绿洲的开发计划,很明白的告诉刚贝拉,这块绿洲是属于塔尔人的,要开发也只能是由塔尔人来开发,至于多特人的水源,塔尔人只不过是因为祖上传下来的协议不便更改,也同情缺少水源的“提坦卡布”部落而已。

塔尔人这么做其实也是可以理解的,因为这绿洲离塔尔人的部落是相当近的。“提坦卡布”部落在强盛时可以不讲理,而后来因为那胡图和图西人的大战,“提坦卡布”部落为了支持自己的族人,比如说像“提坦脱司”等部落,也消耗了大量的人力和物力,就逐渐地赶不上这个塔尔部落的发展了。塔尔人早就有一雪前耻的心思,如今既然有机会了,自然不肯放过。

刚贝拉显然想不到塔尔人会拒绝这个能挣大钱的机会,反正“提坦卡布”部落也霸道惯了,再说投资绿洲实在是花不了几个钱的,于是刚贝拉就撇开塔尔人,半软半硬地强行占地施工。可塔尔人已经不是往日的塔尔人了,他们调集了大量的富裕人手来捣乱,双方已经发生过几次大的械斗了,刚贝拉终于认识到了塔尔人如今的实力与决心,工程被迫停工。

楚云飞听到这里,心里已经有了大概的算计,“那塔尔人只是为了出气,抢回这块地方,还是也想开发绿洲呢?”

刚贝拉没想那么深远,“都有吧,不过他们还是承认我们取一半水的权利的,只是不想让我们再动塔提绿洲的其他东西了,他们也在那里施工呢,不过,我们离得塔提绿洲实在是远了点,不能阻止他们施工。”

“那你们不能在分成上做些让步么?”楚云飞建议着,“难道还要大规模地械斗不成?”

刚贝拉非常苦恼,“问题是,塔尔人连讨价还价的机会都不给我呀,至于打架,政府现在也管得很严,招来维和部队就麻烦大了。”

楚云飞很是有些好奇,“部落的小冲突也会招来维和部队?你们索度军队是做什么吃的?”

刚贝拉实在是没心和他说这个事,“军队里什么人都有,他们去劝解这事,那部队里自己就先打起来了。”

哦,楚云飞点点头,思考一下,“这件事情,我有两个处理建议,不过,这算不算我们帮你解决问题呢?”

刚贝拉早知道眼前这个家伙不是盏省油的灯,这次来就是专为请教这个“狡猾人”的,并没有算上成树国和刘宁,虽说从狡猾上讲那俩也绝对赶得上刚贝拉,“如果事实证明你的建议确实达到了要求,那自然要算你们帮了我一件事情。”

楚云飞看刚贝拉有些不太高兴他的斤斤计较,笑笑说道:“那算我没说好了,我帮你出个点子,你给我报酬总可以吧?”

刚贝拉自然不会在乎几个小钱,“有什么想法你说吧。”

楚云飞提出了那两点建议,“一个,咱们可以想办法暗杀掉他们的酋长,他们族里自然要乱上一阵,有这阵工夫,足以让你们把那个绿洲搞好了,没准下任酋长还肯和你们部落继续合作呢。再一个办法就是,工程你索性就不要搞了,派几个精干的人专门骚扰他们的施工队,杀几个人也行,反正就是要让他们干不下去,而且还不能让塔尔人抓住把柄,让他们有苦都说不出来,到最后,他们够聪明的话,还是要和你们合作的。”

刚贝拉听了沉吟半天,因为“提坦卡布”部落里并没有合适采用这种方式的人选,所以他也从来没有考虑过这样解决问题,不过总的来说这两个计划听起来是不错的,都有可行之处,可这执行的人选,怕是还要落在这三个中国人身上。

\section{第八十二章 狡舌如簧}

有错就改是刚贝拉的好习惯,何况眼前三人是中国人,并不是族里的平民,跟他们认错并不是件丢人的事,“呵呵,楚,我觉得这件事还是要依靠你们三人来处理的,怎么能只给你钱就完了呢?这肯定算是你们帮我完成了一件事,嗯,还是很危险的事呢。”反正左右是要靠这三个中国人来完成的,刚贝拉并不介意在嘴上表示一下关心,又不会损失什么。

不过,“提坦卡布”部落人的蛮横在下句里话里就体现了出来,“我觉得,你这两个建议并不矛盾吧?完全可以同时使用的,那样,我们的机会就会更大些的,让该死的塔尔人再不识抬举。”

楚云飞自然不会让他继续这么想下去,开玩笑,两事齐做,那危险也是要加倍的。自己这三个人做打手那是迫不得已的,并不是天生爱杀人;再说了,既然已经和国内联系上了,那下一步三人也就不是孤立无援的了,虽说这落脚之地还要珍惜,但大可不必用性命去相博了。

“刚贝拉,我觉得你这么想不对,”楚云飞先给他的想法确定了性质,“你想想你到底要达到什么目的呢?开发绿洲而已,是为了商业发展,是为了挣钱!而不是为了发动对塔尔人的战争,双管齐下的后果只可能是引发塔尔人的仇视。”

虽然楚云飞对塔尔人、对索度国的的了解远远赶不上刚贝拉,可现在也只能硬着头皮往上冲了,凭的就是楚云飞对人性的分析,分析不了塔尔人就分析刚贝拉,反正是能少点风险就争取少点风险。

“如果骚扰施工队的话,塔尔人肯定能猜出来是你做的,对不对?可你不被抓住把柄的话,他们也没有证据,只能干生气,对不对?而你放弃了工地,他们没处报复,那就不会有什么损失,对不对?你自己都不施工了,却不肯放过对方,那只能表示这事成了意气之争,也就成了中国人说的面子之争了,你去过中国肯定知道面子是什么意思了。”

“塔尔人还肯给你们一半的水源,却不肯和你们共同开发,这只能说明他们也看到了开发绿洲所代表的巨大利益,只是不想和你们分享利润,对不对?当独占利润变得不现实,同时又陷入了损人不利己的面子之争,那么,如果还想得到利益的话,他们只有妥协了,对不对?”

随着楚云飞一口一个“对不对”,刚贝拉这个奸商也只有瞪着大眼愕然摇头的份,这话确实是无懈可击的。

成树国和刘宁在旁边看着有点好笑,看着肥头大耳的刚贝拉在那里被牵着鼻子走,确实是件让人赏心悦目的事情,至于两人话里话外的意思,成树国和刘宁也听得清清楚楚,于是知道有人又懵然不觉地上了楚家的贼船,不过,对于战友的观点,两人却是发自内心的赞同。

看到刚贝拉毫无意见地接受了前半段有破绽的说法,楚云飞自然要抓住机会,穷追猛打的,“至于刺杀塔尔人酋长的事情,如果你们掩饰得当的话,塔尔人根本就不可能知道是谁干的,作为个酋长,怎么也有几个仇家吧?虽然肯定也有人会怀疑‘提坦卡布’,但更多的人会想‘提坦卡布’部落本来就有跟他们一拼的实力,怎么可能做出这种事呢?所以,就算你趁塔尔人混乱的时候去开发绿洲,只会有人说你会把握机会,绝不会有几个人因此而咬定你是凶手的,我说得对不对?”

刚贝拉现在已经是对楚云飞佩服得五体投地了,这么复杂的问题,这么系统的认识,也就是这家伙能说得出来,还是如此的简明和清晰,看来自己这趟刚卡之行不但是没赔,反而赚了,赚大发了,就象中国老话说的:“数天下之英雄,唯楚君与刚尔。”

楚云飞看到说动了对方,自然是要趁热打铁的,“所以说,上面两种方案,随便取一种都是非常实用可行的,唯一不能做的,就是两个方案同时采用!”

“刚贝拉你想,要是你已经表明要和塔尔人抢这个……面子了,那么他们酋长忽然被刺杀,那塔尔人会认为这事是谁干的?或者话反回来说,酋长被人神不知鬼不觉地杀了,而这种神秘现象又在绿洲的工地出现了,他们会不会认为这只是一种巧合?”

“所以说,这是道单选题,你要选多了,会被扣分的!”

刚贝拉坐在那里陷入了沉思,楚云飞的机关设计得太巧妙了,巧妙到大多数人根本反应不过来里面的奥妙,而从楚云飞的话里。自然能品出来他的主张:方案只能任选一种。

“提坦卡布”的商业天才轻易地就陷入了楚云飞的陷阱:这中国人说得实在太浅显,太透彻了!不知不觉中,刚贝拉的考虑重点已经由“双管齐下”改变为这两种方案到底采用哪种的好。

这两种方案选哪个?虽然刺杀塔尔族酋长引起的轰动更大,可利用的渠道更广,而其后得到的东西可能也更多,但这个选择也有其不方便的一面。首先塔尔那个部落的族长身边的保护绝对不会很差,虽然这三个中国士兵很强悍,但能不能成功确实是两可之间的事。如果一旦不成功,继续刺杀显然是不可能的了,而这时再换选第二个方案的话,那就是先前楚所说的,是人都会知道是你们做的。

还有就是,刚贝拉、古基斯酋长、以及对方的酋长虽然属于不同的民族,但毕竟都是处于上层阶级的顶尖人物,如果两族没有处于战争状态,刺杀这种级别的人物绝对是开了坏头,有败坏风气的嫌疑,刚贝拉一想到父亲古基斯也可能被人这样害死,就全身不寒而觫,这样糟糕的先例还是不要从自己这里开始的好。

至于破坏对方的施工,那就相对而言简单得多了,今天不合适下手还有明天,绝对不存在行动失误的可能,再说即使行动失误,也还有下次机会的。而且这事最倒霉的可能无非是自己人被抓,被对方掌握证据而已,可先别说塔尔族那些角马们能不能抓住强悍的中国士兵,即使他们抓住了又怎么样?中国人也不是多特人,这事只要算不到自己头上,那就不算失败。

于是在楚云飞诱导下,刚贝拉做出了选择:就袭击塔尔人的工地好了。

所有人都没有想到,因为楚云飞他们的作战方式和行为目的的不匹配性,导致事情发生了很多的变数。

原因很简单:索度没出现过什么“特种作战”,严格地说,是从来没出现过以商业利益为目的的“特种作战”,这就是楚云飞不理解索度随便出主意产生的后果了。

\section{第八十三章 短暂的休整}

等刚贝拉走后,三人相对大笑,任是谁也知道,袭击工人比刺杀酋长的事容易得多,也安全得多。成树国自然知道楚云飞为什么不同意“双管齐下”的意见,想想刚贝拉被莫名其妙地说服的经过,“云飞,要是刚贝拉说‘双管齐下’难度太大,完成这一件就算咱完成了承诺的三件事,你又打算怎么回答?”

楚云飞眯眯眼,狡黠地笑笑,“那我一定会让他明白,‘双管齐下’是个好得不能再好的选择,其他选择,应该是都有致命的缺陷的。”

三件事,能毕其功于一役的话,有点风险……那实在也是值得一冒的。

两人正在这里嬉皮笑脸,刘宁却皱起了眉头,“云飞,我发现,你实在算不上一个杰出的军人。”

“哦?”楚云飞诧异地看着他,“好端端的发什么的感慨啊?我怎么不算杰出的军人啦?”

刘宁摇摇头,“你和树国都是优秀的军人,不过,树国可能会成为杰出的军人,你嘛,这辈子怕是够戗了。”

成树国听得好奇心起,嘴上却还不饶人,“你才知道啊?我早就说过,云飞,那是没我有发展前途的啦。”

刘宁笑笑,三个人早开惯玩笑了,“云飞的单兵素质那是不用说,所以我说他是优秀的军人,不过,他太爱算计了,凡事斤斤计较,这军人的血性在他身上可是不怎么看得出来。”

“多虑必然导致少断,要是在生活中,不一定是坏事,其实云飞的决断能力也不差,不过,在战场上的话,只靠聪明,不能凭血性振奋军心的话,应该不能算杰出吧?我觉得他也就是一个参谋的材料,不是帅才。”

成树国点点头,“那是,我就有谋有断,此帅才也;你是勇于决断,堪为将才;云飞么,头脑聪明,可为军师焉。”

楚云飞懒得跟他俩斗嘴,这种环境,一不小心就是万劫不复的局面,不操心怎么能行?“那好,我当军师,偷袭塔尔人工地的事,就拜托二位了,要是觉得没兵的话,跟刚贝拉把塔塔借来用两天。”

虽然是玩笑话,可听得成刘二人面面相觑,两人已经习惯了楚云飞拿主意和带头行事了,想想要是行动没楚云飞参加的话,还真是有点不塌实,就算抛开他那盘算的本事不说,只是那强大的武力也是不可或缺的。

“你敢!”两人同时出声,齐齐扑向楚云飞。

楚云飞迅速地闪出门外,放声大笑,“哈哈,还是多练练功夫吧,别到时候弄出笑话,我也要练功了。”

多久了?没有这样畅快地笑过了?楚云飞又有了一点点的感慨,人哪,有了着落,确实是幸福的事啊。不过,我真的有点过于算计了么?

管他呢,一旦把那个马哈苏德干掉,这个鸟兵,当不当吧!有时间的话,还是多陪琳琳过过夫唱妇随的日子好了,人嘛,不过就是这一辈子。

可是,回国的路虽然像是近了许多,但依旧是那么遥远,可望而不可及,跟前些日子相比,不过也就是从绝望变成了略有希望而已吧。

抬头看看天边刚刚升起的圆月,万里之外,那刁蛮又不失温柔的小美人,是不是也在思念着自己呢?

接下来的几天里,三人都是勤恳地练功夫。楚云飞终于发现,那次救治刘宁,确实让他元气大伤,这不,一旦有了充足的时间练功,内气和精力在急剧地恢复着,不过,看来想回到颠峰状态,怕是怎么也得要一两个月吧。

他身边还有两瓶从刚卡带来的奇怪的“灵水”,不过楚云飞已经不打算再喝了,他下决心要搞清楚自己身上到底发生了什么事,等到有朝一日回国,一定要找废人关问问,实在不行的话,废人关不是还认识个神秘高手么?既然这水对自己而言是个有用的东西,那样品是一定要保留一些的。

刘宁倒是因获得福,内气已经修炼得比成树国强很多了,不过直到又过几天,他脑袋上的纱布取掉,才有了那么点点高手的味道。裹着纱布的刘宁只会显得杀气腾腾,那自然和高手风范是无缘的。

成树国倒不怕刘宁超过自己,刘宁手脚上的功夫可是远不如他的,在坚持练气的同时,他不知道抽了什么风,居然想到要练习特种作战的另类武器——钢针。

不过,在这里收集钢针实在不是容易的事,而成树国练的那基本已经不能说是针,看到他手里那六七枚样品,只能说是细点的没帽的钉子而已,至于威力如何,大家没见识过,倒也不能妄加评论。

反正是苦了刚贝拉手下那个跟班,就是上次在刚卡把头撞到车前玻璃上的奇卡巴,到处给成树国收集钢针,却总找不到合格的。

所以,虽然成树国一再强调这个东西“马上要用到”,但等奇卡巴为他弄到满意的钢针的时候,已经是俩月以后了,那还是奇卡巴听了别人的建议,专门去找了个铁匠订做的。

刚贝拉把楚云飞的建议秘密地上报给了他的酋长父亲,古基斯斟酌再三,还是批准了这个比较阴险的计划,不过,他还是再三强调了,“下手的时候尽量不要弄死人,如果实在不行,弄死几个干活的金塔瓦穷鬼也算”。

这也是没有办法的事情,索度经济正值大发展期间,一步落后就要步步落后,实在是不能放弃这个机会。古基斯酋长作为个身负重担的部落决策人,其实算是相当优秀的。

然后就是刚贝拉在秘密地做准备了,先表面上撤出了多特人工地的所有人员,材料什么的也做了相应的处理,该收回的收回,该甩卖的甩卖。然后就是调查对方,包括塔尔人工地的位置,人数,人员组成的结构,听了楚云飞的补充建议后,还调查了工程进度和破坏点的筛选等。

期间,刚贝拉很是担心三个中国人到处乱逛,白人在这里确实算比较少见的,而将来楚云飞三人在行动时万一被人认到,有人想起这三人就在多特人这里,岂不是凭空为对方提供了攻击的口实?

不过楚云飞三人倒是非常自觉,一方面是他们需要休整一下,另一方面也需要提升一下战斗力,而且刚贝拉的这点心思在中国人这里根本就是透明的。在事没办好之前,大家自然不会给他添多少困惑。

等到刘宁的伤口完全长好以后,相应的物资也都落实好了,三个中国人终于开始执行他们承诺中的第一件事。

\section{第八十四章 袭扰队出动}

一辆卡车静静地停在离“提坦卡布”部落大约有一公里模样的灌木丛中,虽然低矮的灌木并不能完全掩饰卡车的存在,但从远处望来,至少不会让它显得那么醒目。

车边零散地坐着四、五个多特汉子,正在低声谈笑着,时不时地拿起手中的烤肉咬两口,有个年纪大点的多特人似乎讲了句什么可笑的话,引起其余几人一阵大笑,可声音并不大。

这就是刚贝拉应楚云飞的要求,给他们配备的人员,车上也有些手雷和炸药,还有几塑料桶的淡水和一些生活用品,虽然不多,但还是比较全的。

那讲笑话的人三两口解决掉手里不多的烤肉,抬头看看天色,“再有一个小时天就该黑了,他们怎么还不来?”话音刚落,旁边就有人低声反驳他,“那些中国人太扎眼了,起码要天色暗下来才能出来的。”

被反驳的人笑笑,抓了抓那说话人的头发,顺手又挠了两把,“迪赞,你小子行啊,再过几年,我这个位子该让给你坐了。”声音又压低几分,“怪不得这次要你陪他们去呢。”

半小时过去了,远处出现了几个人影向卡车走来,可由于天色暗了下来,一时看不清楚,几个坐着的汉子警觉地拿起了身边的步枪,伏下身子。

走到近处,大家才发现是奇卡巴带着三个穿着沙漠迷彩的黑人来了,走到跟前,在场的多特人才发现,那三人不是黑人,是脸上涂了黑色的中国人。

众人不再多说,齐齐上车,奇卡巴把三人送到以后,挥挥手掉头走了。马达声响起,汽车向着那遥远的绿洲驶去。

楚云飞他们和四个多特人坐在卡车的马槽内,没人说话,那四个多特人好奇地盯着三人看来看去,却怎么也看不出刚贝拉嘴里的“厉害”来,难道真的是中国功夫么?

同行的这些多特人都是“提坦卡布”的精英,是刚贝拉的手下,身强力壮训练有素那只是其一,关键他们都是酋长这一支的近亲里选拔出来的,属于可以绝对信赖的骨干。

他们不但要把楚云飞三人运到塔提绿洲附近,还要负责背运物品,把物品背到计划中的隐藏点附近后,还要尽快地帮三人挖掘几个可供藏身和藏匿物品的土坑。

值得一提的是,那个叫迪赞的小伙子还要陪楚云飞他们一起呆在那里,不但是要起到接应的作用,关键是还要给三个人带路。

由于向沙漠方向越走越近,附近也没有几个部落,就算有也是“提坦卡布”部落的外围分支,所以车开得很快,等到离绿洲接近二十公里左右,才有塔尔人的聚居地出现,不过都是比较偏北的。

等到离绿洲有2公里左右的时候,车停了下来,不再走了,毕竟才当地时间9点多,习惯了夜里酗酒作乐的非洲男人肯定还没有休息,再向前走容易被人发现。

连司机在内一共有六个多特人,有人抱着东西有人拿着枪,楚云飞他们也帮多特人抗着铁锨之类的工具,一行人在迪赞的指引下悄悄地摸到了绿洲边缘。

“就这里了”迪赞指着比较平坦的一块硬地,周围是稀疏的半人高的非洲刺蒿,“照你们说的,没有找那些一看就会有问题的地方,草不密,不怕烧;地也硬,不会有脚印;而且,刺蒿附近不会有蛇的。”

楚云飞向四周看了看,再跟刘宁和成树国交换个眼神,嗯,这里确实不错,就这里吧。

来的多特汉子的体力都非常地好,很快就在楚云飞指定的地方挖了三个深一米五、直径半米口小底大的坑,楚云飞他们也没闲着,忙着把挖出来的土运到附近的树林和沟壑中,然后再回来对坑顶做伪装。

坑底下的地方自然是越大越舒服,不过,那也要讲个结构合理性,挖坑没费多少事,修整和伪装可费了不短工夫,凌晨四点多特人离开,楚云飞等四人可足足忙到六点。楚云飞最后还趁天没大亮的时候检查了一下倒土的地方,确认了伪装的迷惑性才施施然回到洞里。

楚云飞三人在刚卡已经积攒了足够的生存经验,这次准备充足,条件可比三个人在刚卡的藏身之处舒服些,还配备有向导可用。不过这次可供藏身的地区不大,危险又大,所以做得稳妥点那是必要的。

休息了一白天,第二天凌晨,三人在迪赞的带领下,悄悄地溜进了塔尔人的工地,半路上,迪赞还不忘记指着一片低矮的建筑物跟中国人低声介绍一下,“瞧,那就是我们的工地,都怪那些可恶的塔尔人……”

塔尔人的工地很大,比多特人的大多了,迪赞找到个相对高一点的地方给三人指点,“这里,要搞购物区,那里,是自然风景区,将来听说要种植物的,那里,是停车场,那里,是游乐区,最重要的是这里,是旅店和民俗区。”

楚云飞三人听得头大无比,难为他是怎么搞清楚这些情况的,不过,开发绿洲应该这样开发么?三个中国人谁也不知道,自然不能妄言。可到最后,听说这里还要搞什么游乐园,楚云飞终于断定:塔尔人显然不如刚贝拉懂经济。

随后迪赞把工人和管理者住宿的地方一一指明,做了简单介绍。

这里施工的人大致是分三个阶层的,地位最低的就是刚卡或者金塔瓦人,这些人大多是偷渡或者被拐卖来的,于是那众多的茅草架子里躺的就是这些人。

第二个阶层的是塔尔人里比较贫穷的,有男人也有女人,来这里是挣工钱的,做些轻松点的活计,厨师、保卫之类也是由这些人来做。不过这次塔尔人下了大功夫在绿洲上投资,尤其是多特人也因此退避三舍,所以工钱给的很高,能来捞钱的也是多少有点办法的。他们住的也是草搭的棚子,不过,起码那顶子下面还是有架子和草帘墙的,人在里面是能站起来的。

最上面,那自然就是管理阶层了,主要是监工,也有几个技术人员,至于整个工地的负责人,那是塔尔人酋长的侄子,不过,他是不怎么来工地的。这些管理者,住的房子虽然也是茅草顶子,四周却是土坯墙,主要用于阻挡风沙。

\section{第八十五章 袭击塔尔人}

迪赞把情况一一介绍完毕,问了声,“你们还有什么要问的么?”他接到的命令就是把中国人带到地方,介绍清楚情况以后就离开,因为行动中他实在不宜在场。

楚云飞皱皱眉头,“他们有枪没有,或者其他的武器,都会放在什么地方?”

迪赞很是纳闷了一下,不过他马上意识到这三人不能以常情来忖度,看来还是自己疏忽了,“保卫都有枪,索度国都是这样,那些人其实都是部落的民兵,白天部落械斗的时候不能用枪,但值班巡逻的人都随身带着枪,可以开枪杀人和野兽的。至于枪放哪里,这个我也不太清楚。”

楚云飞盘算了一下,又问了句让他曾经吃过亏的话,“那些保卫晚上什么时候换班?”

迪赞显然又认为这是一句废话,“他们不用换班,但想起来的时候会出来转转,时间不固定。”

楚云飞挥挥手,示意自己知道了,“好了,没你什么事了,快回去吧,我们要行动了。”迪赞应声离去。

三人碰在一起,简单地商量了一下,开始动手!

最开始的目标锁定了那些土坯房子,一共七间!

成树国在外面观察,楚云飞和刘宁轻手轻脚地从东侧摸了过去。

最东头的土坯房里有俩人,都已经睡着了,楚云飞摸上前去,一掌一个,直接将两人击昏,然后掏出刚贝拉给他们准备的小瓶。

小瓶里是种植物的汁液,据说与卡巴的种子有异曲同工之处,不过是直接麻痹神经的,被麻痹的人就算神智清醒,也动弹不得,药效大概持续不到两个小时,不过有一个小时都足够了。

楚云飞从口袋里抓出两根小木棍,在液体里一蘸,一扬手,就把两支木棍插进了两人的大腿上,放下药瓶,双手齐扬,“咯咯”几声,两人的肩关节被卸掉。

好了,收工,楚云飞和刘宁又摸向下一间土坯房。

在袭击进行到第四间土坯房时,出现点小小的意外,房内出来个人,睡眼惺忪地在屋角“刷刷”地尿了起来,楚云飞难得地用手捂了对方一下,顺手在颈侧一击,那人直接就栽进了自己的尿里。

七间房子,十一个人,只花了两人二十分钟就全部搞定。

刘宁看着楚云飞如表演般地重复他的袭击,很想自己也来试试,不过,再想想,还是算了吧,自己手脚没个轻重的,万一弄死人怎么办?毕竟这里都是些有身份的人;轻了也不行啊,有人喊上一嗓子,那不计划全乱套了?

然后三人又花了十多分钟把随身带的三个炸药包塞进了两栋大点的建筑物里,似乎是塔尔人未来的旅馆,把导火索接到了一起。

接下来楚云飞和刘宁不再掩饰行踪,大摇大摆地向那些草架子走去,站在距离那里五十米处,子弹上膛,对着那些穷鬼们的窝棚就开火了。

“哒哒哒,”沉闷的自动步枪声音在寂静的绿洲响起,塔尔人的工地顿时就像开锅了一样,男人的呼喊、女人的尖叫、还有脾气大的在咆哮自己的睡眠被干扰。

一梭子弹打完,换梭子,又一梭扫完,刘宁和楚云飞对视一眼,撤!

这时已经有反应快的几个保卫向着枪响的地方冲了过来,但同时工地的建筑物里传来三声巨响,“轰、轰、轰”,成树国引爆了炸药。

塔尔人的营地里乱成了一锅粥,没人知道袭击来自何方,对方又有多少的兵力,战争,战争要开始了么?

几个塔尔保卫的枪响了,那是漫无目的地乱射,最多也就是起到点壮胆子的作用。

就在这一片慌乱中,三个中国人汇合到一起,大摇大摆地回到了他们的地盘。

迪赞本来就没有睡着,听到这激动人心的热闹声音,早就窜出了洞外接应三人。

“你们杀了多少人啊?”

刘宁上下打量打量他,“一定要杀人么?不过,我也不知道杀了他们多少人,估计没几个吧。”

迪赞实在是不明白,就这三个人就弄出这么大的动静,而且还没杀了几个人?最后居然就象在自家院子一样散着步回来了。果然刚贝拉是没有说错,这三人就是“厉害”。

实在不能怪迪赞会这么想,事实上索度国里一般人根本不知道什么叫“特种作战”。

不过这事对三个中国人来说,那实在是再正常不过了,有心算无心,军人对平民,这种战绩,算不了什么吧?不过,下次的袭击,恐怕就不会这么顺利了。

直到现在刘宁才有功夫问问楚云飞,“你那木棍是不是火柴梗?用了药怎么还要卸人胳膊?”

楚云飞说得天经地义,“火柴梗吸药多呀,卸他们膀子,我是怕他们半路药效过了,冲出来张牙舞爪被误杀,怎么说也是领导阶层呢,卸腿又太麻烦了,你说呢?”

四人在这里进入梦乡的时候,塔尔人那里还时不时地传出一两声枪响。

等到天色大亮的时候,接到消息的工地负责人气急败坏地赶来了,随同他前来的还有四卡车的部落武装,另外,居然还有一个白人,真正的白种人。

五辆汽车没有停留,直接奔着现场就来了。入目一片狼籍,气得负责人加古勒直跺脚,随手招来了一个保卫,“损失统计出来了没有?”

这个保卫明显属于脑子不太够用的那种,嗫嚅了半天,才说出了一句,“房子死了,人……人也死了……”

加古勒正在火冒三丈呢,听到这话,直接就是一脚踢了过去,该去哪去哪吧,你怎么没“死了”?

还好过来个先期骑骆驼到达的塔尔人的小头目,他是半夜听到这里的枪声和爆炸声被派来打探消息的。

“大人好。”

“行了行了,”加古勒没心思和这人讲什么礼数,一把拉住要行礼的小头目,“你说,怎么回事?”

那小头目神情严肃,脸上的肌肉也不自然地抽动着,“昨天有人袭击工地。”

“我知道!!!”加古勒勃然大怒,要不是看在此人口齿还算清楚的份上,他又要来上一脚了,“说重点,少废话!!!”

那小头目倒还算个灵巧的,要不也轮不到他来做这查探的事,马上简练汇报,“工地被炸,宿舍被袭击,死了一个金塔瓦人,伤的人很多。”

嗯,还好,只死了个穷鬼,事情不算严重,加古勒下意识地摇摇头,“什么人干的?”

小头目调查半天了,自然知道得很清楚,“不知道是什么人,连有几个人都不知道。”

加古勒抿抿嘴,这事,很蹊跷啊,该是多特人干的吧,他们打算两族开仗么?

他掉过头问那个白种人,“多尼,你怎么看这件事?”

那叫多尼的白人倒是一副见怪不怪的样子,“你先说说事发经过。”

\section{第八十六章 欧洲金融家}

加古勒的父亲利都死得早,所以酋长的位子才轮得到他的叔叔昂纳来坐,不过昂纳和利都的关系很好,所以加古勒一家在部落里的地位一直没有被动摇,昂纳对他百般照顾,青睐有加。

这次塔尔人修建绿洲工地,因为资金充裕,算是部落里一等一的肥差了,这种好事也只能落到酋长的近亲身上。不过昂纳自己的儿子们还都小不能担此重任,而他的兄弟里也没有长于规划的,就把这件大事交给了加古勒负责。

加古勒常受叔叔的照顾,这次又把这么重大的事交给了他,他自然要殚精竭智地规划一番,一来报答叔叔,二来也想尽量展示一下自己的才华,以稳固自己的地位。

这个“多尼”就是加古勒专门去喀津霍搜罗人才的时候碰上的。两人相遇时多尼的样子算得上落魄了,正一个人坐在街角狂灌罗伦酒,虽然索度靠近赤道,可十二月的天气还是有点冷的,多尼却只穿了件长袖T恤。

虽然是索度首都,可喀津霍的白人还是很少的——如果不算维和部队的话,像多尼这个样子的估计索度建国以来也只此一人,加古勒自然就上心了。

加古勒上前邀请多尼共进晚餐,反正怎么说也是个白人,管吃管喝实在是小事一桩,多尼也没推辞,于是杯来盏去间两人就聊得很投机了。多尼并不掩饰自己的窘态,他确实是遇到了麻烦。

多尼本来是在欧洲“博睿”投资公司任职的,不过他经手的一笔风险投资被借贷方以“破产”的结果吞并了,他自然是难辞其咎的。

本来多尼也就是个失察的错误,或者最多也就是“渎职”,可他不想坏了名声,也咽不下那口气,就自己出钱请了私家侦探去调查,一调查才发现,是公司出了内鬼,内外勾连,骗取了投资客户的资金。

“博睿”投资公司在欧洲名气非常地大,这次犯的错误又是非常低级、不专业的,影响实在是太坏了。经公司内部调查,真相大白后,那内鬼已经被处理掉了。可那内鬼之所以成功,和公司一个重要领导的纵容很有些关系,于是多尼就被当作替罪羊推了出来。

多尼总算明白了自己隔壁的那个美女为什么失踪了,可他绝不愿意背这么样个黑锅,于是就找到公司总裁,要求还他清白。公司总裁肯定是要舍车保帅的,于是多尼的请求被拒绝,总裁倒是表示愿意私下赔偿他点“名誉损失费”。

多尼还是不答应,他觉得混金融这行业,名声要比那点点“遮羞费”重要得多。最后总裁发火了,多尼为了不步美女同事的后尘,只能亡命天涯,躲到索度这鸟不生蛋的地方。

人倒霉的时候喝凉水都塞牙,多尼本来说隐姓埋名找个地方躲它个一两年,可初来乍到,总得先找个宾馆住下吧?

多尼刚住进宾馆,“宾馆指南”还没来得及看,就有俩妓女上门。多尼感叹索度人热情如火的同时留住了两人,还说泄火之余可以顺便打问下附近有什么房子可以租。没想到一夜消魂以后,发现两人已经不知去向,跟她们同行的还有多尼的行李和皮夹,以及所有的证件。

在街头,加古勒看到多尼喝的是罗伦酒,多尼知道他喝的是自己身上的夹克。

再往后的事情那就顺理成章了,作为个精通金融和投资的人才,多尼自然是被加古勒高薪聘请了。

那小头目也知道加古勒很重视这个白种人的意见,自然不敢质疑对方是否有问话的资格,把自己说知道的一五一十讲了出来。

多尼听后,点点头,对加古勒说,“毫无疑问,是多特人干的。”

加古勒也认同多尼的观点,“你说得很对,不过,多特人既然已经制服了我的人,为什么不杀了他们?反而要费劲地让他们的胳膊脱臼?怕他们出声的话,用匕首也可以啊,多特人什么时候变得这么善良了?”

多尼却是一眼看出了其中的意思,“他们不想激怒塔尔人,所以只拿枪打那些穷鬼。”

加古勒哼了一声,“那这么说他们还有继续合作的想法了?下次他们再派人来,我当场打死他。”

多尼摇摇头,“你这么想是不对的,他们为什么晚上来?就是怕你抓住他们的把柄,你自己工地出事是你自己的事,你要杀了多特人那就是两族间的大事,你会倒霉的。”

加古勒也知道自己说的是气话,不过是显示一下气势而已的,所以并没有介意多尼的反驳,事实上他也知道部落这些年虽然发展得不错,但和“提坦卡布”部落硬抗起来还是不行的,要不那水源他早收回来了。

沉吟一下,多尼又说道,“他们的工地也早停工了,一来不用怕你报复了,二来……他们未必会上门求你合作的。”

加古勒一愣,“你这个,这个第二点是什么意思?”

“他们很可能仅仅是为了出气,不过也可能还有其它目的。”多尼又考虑了一下,“你觉得他们来了几个人?”

加古勒认真地想了想,“我觉得他们来的人应该不多,多了不可能不被觉察的,再说,子弹也不算密集的,大概七、八个人吧。”

“七、八个人都未必有,”多尼若有所思,“我们去看看那些管事的伤的怎么样。”

看了几个人的伤势之后,加古勒大惊失色,一声令下,把所有管事的都叫了过来,其中有一个腿上的木棍还没有拔出来。

多尼看着脸色不断变幻的加古勒,“你现在还会认为对方有七、八个人么?”

加古勒紧盯着木棍还在腿里的那位——当然他看的是腿不是人,终于明白了那小头目说起伤者腿上有木棍的时候,为什么会是那么恐怖的神情了,就这么细个木棍,真是能插到肉里的么?

“我想这么恐怖的人,有一个已经够多了,”加古勒苦笑着说,“我现在希望多特人只来了一个,不过,这不现实。”

多尼掂掂手里的平头木棍,短而轻,“这肯定不是用器械发射的,这么短小的木棍,又没有尾翼之类的东西,要用器械发射很麻烦的,还不如拿刀蘸上毒药来得快,或者用仙人掌的刺,这明明就是把火柴头取了的火柴,还是最劣质的那种。”

“而且,你注意到他们肩膀的脱臼了么?”多尼毕竟是大地方来的,见识是很广的,“塔尔部落里,会让肩膀脱臼的人有几个?”

加古勒想了想,“可能有,也可能没有,不过能治脱臼的医生还是有一些的。”这话不能怪他,因为自古以来,索度男人从没有把“使对方脱臼”做为一种攻击手段,他自然不知道究竟谁会这个。

\section{第八十七章 二袭塔尔人}

“这就对了,”多尼的脸色越发地难看,“首先,我不认为你们说的这种叫‘嘎都因’的液体能在瞬间麻痹神经,事实上它也不能,这些管理者没叫出声来的原因只有一个:他们被人先打昏了,而打昏了十一个人没有在头部和其他地方再出现别的伤痕,证明敌人下手是很熟练,很有分寸的。”

“能熟练地击昏人,能卸开关节,还能把这小木棍扎到肉里,这会是什么人干的呢?”多尼抽了口冷气,眼中是一片茫然,“可能是雇佣军,也可能是特种兵,但绝对不会是多特人。”

加古勒见识还算渊博,特种兵他是听说过的,不过那东西离他似乎很遥远的,至于雇佣军他可真不知道,毕竟他才二十三岁,没机会听说也是正常的,但是毫无疑问,这种东西按道理来说也是很遥远的事。

但理论上遥远的事就在眼前出现了,多特人,他们到底要做什么呢?加古勒这么想着,不由得发出一声长叹,“Why~~~~~”

冷场半天,没人说话,似乎所有人都被吓住了,良久,加古勒才又出声,“多尼,说说你的看法。”

“我的看法……我的看法,”多尼在那里失魂落魄地喃喃着,他正在为自己担心呢,别是冲着自己来的吧?为那么点小事,自己已经躲得够远的啦。

人要心里有事,万事总爱向坏处想,多尼也不例外:虽说找自己麻烦的是杀手的可能性更大,但也不是说只有杀手才能干掉自己啊;虽说追杀自己的人没必要向当地人下手,可也没谁说不能下手啊。

加古勒看多尼呆在那里,脸上青一阵红一阵,魂不守舍的样子,忍耐良久终于憋不住了,伸手去推他,“多尼先生,你怎么啦?”一着急,很久没用的“先生”这词又冒出来了。

外力作用下,多尼终于有了反应,“啊,我……我忽然想起了一些事,失礼了,抱歉。”

加古勒虽然心情非常烦躁,但他从没见过开朗的多尼如此地失态,又是在这么关键的场合,莫非是和这里发生的事有关么?“多尼,你想到什么啦?”

多尼脸一红,左右看看,眼光扫视下,旁边的几人自觉地走到了一边,他凑近加古勒的耳朵,“我在想,也有种可能,会不会是公司派来找我的人?”

加古勒下意识地摇摇头,多尼的事他是很清楚的,是啊,只有这样才能解释为什么那遥远的名词会出现在自己身边,可是……多尼的错误有这么严重么?

多尼话说出口就意识到自己犯了个错误,这不表明了自己是个灾星么?要是加古勒因此又解雇了自己,那可真是天下之大却没有自己立足之地了。

于是当他看到加古勒在那里沉思,马上利用现在还受重视的金融家身份加以解释,“不过,我想我犯的错误不足以引起这样的愤怒的,想来想去,这事多特人怎么也脱不了干系的。概率学上有种说法,‘小概率事件很少发生’……哦,不,不,不,这不是你想象中的废话,这句话的意思是说:对一般人而言,小概率事件发生的可能性无限接近于零。”

多尼叨叨了半天,加古勒终于弄明白了他的意思:他是说即使——仅仅是即使,这次来的确实是追杀他的人,但这些人和多特族没有关系的可能性是不存在的,当然,更大的可能就是这些人是多特人专门找来的,与多尼无关。

加古勒同意他的观点,这事背后要说没有多特人的影子,那是跳羚都不会相信的,除了他们,谁还会有心思把同一个宿营地的人分成三六九等区别对待,只不过是没足够的证据证明而已。

至于多尼本人可能带来的麻烦,加古勒虽然不能说毫无芥蒂,但用人之际,自然不好计较太多。再说,他忽然发现,多尼除了是个经济学家,似乎在见识上也超过了绝大多数的索度人。

所以加古勒还是虚心地请教,“那,依你说我们现在该怎么办?”

“怎么办?我怎么知道怎么办?”多尼两眼发直,若有所思,“我只是个金融学家,又不是特种兵专家,我只希望,这次是多特人心血来潮做的一次行动,或者说是为了泄愤的报复。现在让我们祈祷吧,希望以后不要再有这种事发生了。”

祈祷的事加古勒做了没有,没有人知道,反正所有塔尔人都知道,来的四卡车保卫,将近九十个人住进了工地,戒备加强了好多。

可不幸还是在接下来的一晚上发生了,工地四周增加的十二个潜伏哨,二十四个哨兵被人摸倒了十六个,还是带有“嘎都因”的木棍,还是两只胳膊脱臼。

没人知道这事是什么时候发生的,直到换班的保卫来到哨位,才发现大部分族人已经遭受毒手。

接班的人自然不敢犹豫,哨声四起,人影攒动,整个宿舍区再次被惊扰,这天晚上,只有一个人从夜睡到了天亮,那是个喝多了的刚卡人。

等到大家组成了十个人一组的巡逻队伍四处走动时,才发现,后来这拨换岗的人里面也有6人与木棍有了亲密接触。

实在是太可怕了,在这样漆黑的夜里,未知的危险实在太多了,于是所有能动和不能动的人集合在了一起,共同防范那恐怖的魔鬼。大难当头,此时的绿洲上,没有了阶级和种族的区分。

其实这一切都是楚云飞一个人干的,他们要考证昨天的行动成果,那自是要出来打探一番的,而这种任务,楚云飞是当仁不让的人选。

以楚云飞的谨慎和机敏,很快就发现四周多了不少潜伏哨,不过这也算情理之中的事,于是他就准备多观察一下打道回府。可一阵观察下来,他发现潜伏哨的哨位分布很成问题,就好心地帮对方指点了出来——确实是好心,因为他没有杀人。

然后接班人员的上岗方式也在向楚云飞表示:我们也需要您的指点!

既然人家诚心邀请,那楚云飞自然也是要指点指点的。

至于众人最后在营地众志成城的邀请,楚云飞怕人多,一时招呼不好,手没个轻重,那就有违指点的本意了。再说,民族融合、阶级消失的场面来之不易,能多保持一阵也是好的。所以,楚云飞悄然无声地回去休息了。

\section{第八十八章 各有各的恐惧}

成树国和刘宁很纳闷楚云飞怎么出去了这么长时间,就算再小心谨慎也用不着花三个钟头吧?

楚云飞简单地把事情经过说了一遍,“我估计……弄倒他们二十个出头,谁让他们那么不专业呢?不下手都对不起陪咱们呆在这里的迪赞,不过那些肯定是今天塔尔人加派的保卫,不知道咱这么摆明着对上来,他们会有什么反应,恼羞成怒还是尽量低调处理?迪赞你认为呢?”

他们说的话迪赞倒是听明白了,可是,他真没想到这三个人里看起来最瘦的这个居然能无声无息地弄倒二十多个人,而自己呆在这里只不过是隐约听到了工地那边凌乱的喊声,偶尔还传来几声枪响。

不过,当枪响起的时候,身边的这俩中国人并没有前去支援同伴,还在这里低声地说笑着,似乎一点也不担心同伴的安危,等到同伴回来后说自己弄倒二十来个人,这俩人也没有什么意外的表情。

迪赞实在是个很聪明的小伙子,此刻已经可以确定眼前这三人实在是太厉害了,简直可以用“恐怖”来形容,也不知道大人是用什么办法求得他们办事的。而且今晚的情况,更说明了三人不但很恐怖,而且相互之间的理解和配合也是相当默契的。

于是他再不敢存着什么怀疑或者不敬的念头了,可他实在猜不出来塔尔人会做出什么反应,所以存了敬佩念头的话居然显得格外的不负责任,“这个……我也不知道,这种事情实在是不好猜测的。”

确实是不好猜测,连加古勒自己都不知道这事该怎么处理。

加古勒和多尼第二天一大早就来到了工地,看到那二十多个腿上有洞的精壮小伙,实在是气不打一处来,“狗屎,欺人太甚了!实在是太嚣张了,不行,我要受不了啦,多尼,拜托,快给我一个不去找多特人麻烦的理由。”

多尼的理由顺口就来,“你知道多特人找了多少这种人来帮忙么?没准他们正埋伏在什么地方等你犯错误呢。”

毕竟是做大事的,听到这话,加古勒的火气马上降了下来,是啊,要是他一个人,那倒什么都好说了,可是,重任在肩,一句话可能决定很多人的生死,由不得他不冷静,“那你说说现在该怎么办?这事,绝对不能就这么算了!”

多尼也在那里犯了难,思考再三才开始慢慢说话。“这个事情……最好还是想搞清楚多特人想干什么,自然,他们就这么退出绿洲,是肯定不甘心的,所以给咱们找点麻烦也是理所应当的,不过,他们下这么大本钱……嗯,肯定下了不少本钱,这种身手的人可是不好找的,下这么大本钱自然要达到什么目的的,否则……”

多尼一拍大腿,“我明白了,目的就是不让咱们施工,若是只为出口气,他们弄死几个塔尔人都可以的!为什么不弄死?因为他们怕把事情弄大,这事一大,自然他们那里就有了压力了,也不好继续下手了,也就只能眼看着咱们施工了,所以就这么一直不下杀手,直到咱们放弃施工。”

加古勒摇摇头,“多尼先生,你说得……实在是太有道理了,一定就是这个样子的,哈哈,他们也知道穷鬼死几个无所谓,塔尔人可不是随便什么人都能动的。”

猜出了对手的意图,大家都很高兴,不过,这实在是楚云飞他们表达能力太离谱导致的。

兴奋劲一过,加古勒还是得面对眼前的难题,“多尼,你说,那咱们该怎么办呢?要不咱们也去找些特种兵来?”

找些特种兵?多尼心里暗骂,你找得到么?让我找那我是绝对不干的,好容易才跑出欧洲,你让我再回去送死?“呃,这点我不赞同,先别说找得到找不到的问题,现在,我们根本不知道对方是什么人,虽然‘小概率事件’很少发生,但是……万一发生的话,雇主被杀那也不是不可能的,再说,惹急了对方,没准咱们还要受到什么报复,那可真的……就得提心吊胆一辈子了,你不认为外面的世界其实也很精彩的么?反正对方现在恶意也不是很明显。”

那“提心吊胆一辈子”几个单词多尼说得非常用力,他可是已经有些类似体会了,可听到加古勒耳朵里却是有强烈警告的味道。

加古勒想想也是,要没有多特人的掩护,什么特种兵之类的他根本不会害怕,这里毕竟是非洲,来几个白人那是要多扎眼有多扎眼的,可有了多特人的掩护,一切就都不一样了。

不得不承认,加古勒还是很年轻,他甚至连特种兵里其实有非洲人都不知道,他的恐惧和多尼的恐惧那纯粹是……各有各的恐惧。

“那我们现在总得做些什么吧,多尼?”

“如果我要是你,我就会考虑同多特人讲和,继续合作,有钱大家挣好了,这世界钱那么多,谁挣得完?”多尼的话不但很明智,也显示出了金融家该有的风度和眼光。

“不可能!”加古勒完全不能容忍多尼的建议,“这次狠狠地打击了多特人一下,部落里的人不知道有多高兴,知道么?我是振兴塞脱尔部落的希望,我绝对!绝对不能让大家对我失望!!!”

多尼可不管他的感受,“已经是和平时期了,你觉得你抱着这一套有用么?你确定事情闹大了维和部队不会来干预么?你的部落再强大一百倍能和美国人对抗么?再说了……”多尼四下看看,小声说,“将来的酋长只会是昂纳的儿子,绝对不会是你,这么危险的事,这么做你真的觉得值得么?”

事情的发展又一次应验了楚云飞的话——“我觉得对很多非洲人来说,只存在强权和服从,哪怕是再算上利益交换都行,反正忠诚之类的东西基本上是不存在的,对中国人来说,他们是不可思议的。”

加古勒“痛苦”地思考了一分钟,然后就“痛苦”地接受了多尼的建议,不过随后他又“痛苦”地发现还有一些东西是需要考虑的,“可要有人不同意,怎么办?马库斯就绝对不可能跟多特人妥协的,不过也怪我,非要怂恿大家同多特人作对,现在想收都收不回来了。”

\section{第八十九章 都在往远走}

多尼摇摇头,这个加古勒,念头能转变也就算了,毕竟这是他希望的,不过能做得这么彻底,他实在是也有点接受不了非洲人的逻辑和思维方式,“马库斯不过是个只会说大话的家伙,我看他胆小得很,咱们先这样坚持几天,让多特人继续他们的演出好了。最后你提出合作的建议的时候,如果马库斯不同意的话,让他来这里呆几天,他要能镇得住的话,咱们也不算损失。”

加古勒这次坚决不同意多尼的意见,“不行,我做不好的事让那家伙做好?我宁可告诉多特人他住在哪个房间,反正他也不会死,不是么?”

多尼真没想到加古勒一旦下决心,会走得这么极端,比自己希望的还远了好几条街,可见他在部落里是急需一个稳定的位置的。不过,这种情况下,自己再多劝说几句的话,估计他也不会介意再多走几步的,为了保住小命,还是说说别的吧,“好的,这事我没意见,不过,我们是不是需要在绿洲附近来搜索一下呢?这些人绝对是在附近埋伏着的。”

加古勒摇摇头,“我也这么想,要不他们根本没时间这么来回跑的,不管怎么样咱们也要搜索一下的,能搜到那自然最好,搜不到的话起码我也算做过了,良心上也过得去了。”

良心?多尼也摇摇头,“良心这东西,和民主一样,是为政治服务的,不过是个妓女而已,需要它的时候用用就可以了。不过,你的意见我是赞同的。”

于是,在加古勒的的命令下,绿洲的保卫组成了搜索队,二十人一组,八组人进行搜索,甚至连工地上的刚卡人和金塔瓦人也用上了,组成了四百人的搜索大军在绿洲上进行地毯式的搜索。

不过,除了加古勒今天带来的八十人,其他人全部都是哈欠连天的样子,没办法,睡不好觉的人,都是这么个德行。

大部分人睡眼惺忪是一方面,这两天发生的怪异事件确实也是很吓人的,甚至已经有不少刚卡人和金塔瓦人开始悄悄向保佑他们的神灵祈祷了。所以搜索的人看似很多,搜索力度却是远远不够,再加上非洲男人本身就缺乏耐心,想搜到楚云飞他们,自然是希望渺茫的。

此时的楚云飞等人正在睡觉,不过这么大的动静,警醒点的人早就醒了,更别说还有些参与搜索的人,为了自身安全而胡乱大声叫嚷。

虽然搜索队没搜到人,但却给楚云飞他们带去了极大的困惑:咱们这么嚣张地到处敲人闷棍,对手就这么敷衍了事地瞎搜搜就完了?别是有什么诡计吧?

虽然可能有诡计,但是夜间侦察是不可少的,越有诡计那才越有查探的必要!不过,这种诡异的场合下,更是要楚云飞亲自出马了。

这晚上楚云飞是加倍地小心,却愕然发现对方的潜伏哨等全部撤了回去,等他确定工地四周没人,所有人都聚集到了宿舍的时候,天色已经不早了,没有再炸一次工地的时间了。

这晚上只击昏了一个出门小便的塔尔人,可楚云飞还没来得及掏木棍,就有人出来接应了,只好顺手卸掉对方肩膀开溜了。不过以后几天,绿洲上所有人大小便都是凑上多人才敢出门,又有那不太讲究的人就地解决,于是宿舍里臭气熏天难以忍受,这就是后话了。

虽然只撂倒一个,可这一晚上的小心翼翼,让楚云飞加倍地疲劳,实在是太费精神和体力了。

于是,下一个夜晚成树国就自告奋勇地要求出去侦察,可在楚云飞看来,“事有反常即为妖”,不正常的情况往往预示着不为人知的危险,所以,还是师傅出马可靠些。

依旧是没什么发现,依旧是耗费体力和精神,在天快亮的时候,楚云飞又拖着疲惫的身子回来了,这次,他连敲一次闷棍的机会都没有。

事情不能再这么下去了,太被动了,于是,趁天还没亮,楚云飞拖着疲惫的身子召开了火线会议。

经过大家探讨,得出来很正常的两个结论,一种可能是:塔尔人已经被吓坏了,晚上集合白天停工,这样的日子过不多久,他们就会妥协了,大家也就可以收拾行李回去了,不过,塔尔人会这么容易认输么?

另一种可能就是:塔尔人表面偃旗息鼓,但暗地里在做反扑的准备,尽管大家不能确定他们会用什么方式来反击,可面临的绝对会是场恶战。再想想前天塔尔人漫不经心的搜索,似乎有太多阴险的味道在里面,还是小心点的好。

虽然大家不能断定目前正在发生的事情究竟属于哪种可能,可这么耗下去也绝对不是好事,跟着别人的节拍走路,吃亏的肯定是自己。

于是,不安分的楚云飞又提出了新的建议:既然情况有变,那作战计划也要改变,撤消以前白天不出动的宗旨,就在白天出动打击对方。

大家稍做讨论就同意了,这么做的好处显而易见,如果塔尔人已经打算服软认输,那就再给火上添把柴催化这事;如果塔尔人计划有后招,这么做起码能干扰一下对方做事的节奏,而且白天攻击更表现了自己这方的决心和果敢,没准还能叫对方改弦易辙,重新回到谈判桌前来。

刘宁和成树国根本就是纯粹的军人,面对目标的时候崇尚进攻,鄙薄坐等,根本不是愿意让人牵着鼻子走的主;再说,一连三晚上都是楚云飞出去单独活动,虽说确实是不得已的选择,但作为同壕战友只让一人累死累活的,未免不够仗义吧?

至于迪赞是没有发言的权力的,不过习惯了三人的强悍的他,还是为这三人的胆大妄为而吃惊:没搞错吧?这绿洲工地上起码有五百多的人啊,其中光有战斗力的部族武装就超过了两百的!

于是迪赞怯生生地提出:他要把这事报告给部落,因为,实在是有点冒险的,不过报告是件很容易的事,因为绿洲东南不远有不少的多特人,其中就有刚贝拉暗自安排的接应的人。

楚云飞他们没有介意,想报告由他去吧,反正这事的战场决策权掌握在自己三人手里。至于下次袭击的时间,由于今天大家都是一晚上没休息了,那先睡觉,等中午醒来之后,稍做准备,下午正式开始行动。

\section{第九十章 白天的攻击}

休闲的时间总是过得很快的,等到楚云飞三人醒来的时候,已经是中午了,迪赞也已经走了,估计再回来怎么也是晚上的事了。

三人抓紧时间吃了点东西,凑到一起商量几句,收拾停当,行动开始!

他们从来没有白天在这个绿洲上出现过,这时才发现虽然晚上的绿洲草木茂盛,树影婆娑,显得一派生机,但白天看来,绿色植物实在是稀疏得很,藏身是可以,但转移和攻击时就太容易暴露了。

三人马上决定修改计划,不再以敲人闷棍和实施爆炸为主要手段,拿那些穷鬼开刀好了,机缘巧合的话,顺手随便炸个什么也就是了。

现在,白天就是塔尔人休息的时候,他们都跟着楚云飞阴阳颠倒起来,没办法,晚上还要防备袭击呢,白天不睡好怎么办?

可岗哨在白天还是有的,不但防袭击,还要防小偷,因为工地上建筑材料很多,没人看的话恐怕是要丢完的。

楚云飞他们在去茅草架子的路上就发现了一个保卫,不过因为白天这里从未遭到过袭击,那哨兵岗站得实在不够专心,显然又需要楚云飞的指点了,不过这次成树国实在按捺不住了,坚决要求自己亲自出手教这个保卫“哨兵须知”。

手起,人倒!木棍扬起,胳膊脱臼!其实成树国的功夫也是一等一的。

然后三人一路潜伏到了茅草架子旁,才发现草架中间,穷人们正聚拢在一起吃饭。

等候半晌,终于有个人吃完了饭没有参与聊天,而是出来找个偏僻地方解决生理问题,不过他腰间草绳尚未解开,就有一只大手捂住了他的嘴,正摸不着头脑呢,脖子就被生生拧断,出手的人是成树国。

拖开尸体还没几分钟,又有两人相伴而来,虽然离得远了一点点,但楚云飞和成树国双双出手,根本没给两人任何的反应机会。

“够了,”放哨的刘宁小声说。

于是三人藏好尸体,蹑手蹑脚地离开,不过,既然事情办得顺利,索性再炸点什么东西吧,导火索弄长点就好了。

按刘宁的计划,他很想在远距离射杀一人,显示一下己方神准的枪法以威慑对手,同时也好跑路,不过,显然直接跑回去太容易暴露藏身的方向,而迂回的话草木又有点过于稀疏了,实在不太方便。

爆炸声响起,三人又偷偷地溜了回去。

塔尔人的混乱那是可想而知的了,居然在大白天被敌人摸到了眼皮底下,随后又有人发现了草丛中的保卫,大家还没怎么拿出主意该怎么办,又有负责管理工人的人来汇报:死了两个刚卡人和一个金塔瓦人。

这时加古勒并不在工地,临时负责人“蝎子”受不了这种挑衅,又不知道加古勒暗自打的算盘,于是马上吩咐大家展开搜索,他自然明白:对方肯定没有走远,而白天不抓紧机会搜索,那晚上又要提心吊胆了。

受到毛骨悚然的威胁,这次塔尔人的搜索比上次严谨了好多,还好,楚云飞他们算有先见之明,前期工作到底是没有白做的。

看到大家无功而返,“蝎子”忍不住怀念起了家里的那两只猎狗,于是就想派人去弄来,可夜晚马上要到了,工地所有的人都强烈反对这事,原因无他,多几个人在身边壮壮胆子也是好的。对方现在明显加强了袭击,谁知道晚上等待大家的会是什么?

今夜,塔尔人工地,无人入眠!

不过,楚云飞他们四个睡得很好,因为白天塔尔人严密的搜索,证明他们害怕了,怕得要命,那么,大家休息一天吧,反正塔尔人估计是休息不了的。

第二天,出去带狗的队伍把加古勒和多尼也带了来。

加古勒倒没怎么再生气,因为,对方还是没有对塔尔人下手不是?所以,他匆匆地巡视了下现场就开始向多尼低声抱怨,“你说,这多特人有毛病不是?你杀人也就算了,炸房子做什么?将来合作的时候那还是要花钱盖的,可惜了我和你的一番心血了。”

这个绿洲开发的整体规划就是多尼做的,至于其合理性有多少他自然明白,不过,他绝对不会自己暴露自己的短处的,于是就拿话来搪塞,“我想,多特人也许希望在投资中占大股,你这建筑在的话,那是肯定要被利用的,那样他们就不好多占股份了,所以他们就炸掉了,怎么样,你觉得我的猜测合理么?”

加古勒摇摇头,叹口气,“多特人,实在是……太狡猾了。”这话出口,一股沉重的无力感涌上心头,古老而强大的“提坦卡布”部落,确实不是好对付的。

多尼感觉到了老板的失落,“就这么着吧,今天和多特人表示一下合作的意思吧,现在的情况,你不表态,多特人是不会主动找你的。”

如多尼认为的那样,四只猎狗并没有搜出什么人来。而且就在当天晚上,它们在叫了几声后被四根没眼的针穿透了头骨,那是成树国的钢针。

就在楚云飞抱怨没来得及出手,成树国大呼“过瘾”的同时,“塞脱尔”部落刚演完一场口水大战。

加古勒终于冒天下之大不韪,提出了同多特人合作开发绿洲的建议,不过他没说自己已经先行派人去同刚贝拉谈判去了。

虽然绿洲发生的事情塔尔人都已经知道了,但加古勒的意见还是遭到了疯狂的质疑和谩骂。

还好昂纳是支持自己侄子的改变的,他虽然不明白里面的是是非非,但同多特人和睦相处其实是长期以来酋长们一贯沿用的政策,所以大战的结果就是:“勇敢”的马库斯决定第二天晚上在塔提绿洲守夜。

这个消息在当夜就被传到了刚贝拉的耳朵里,送信来的人正好撞上前一拨表示善意的信使。双方一碰头,既然多特人愿意合作,那马库斯不明智的举动自然是要奉告合作伙伴的。

刚贝拉斟酌再三,还是相信了加古勒的善意,马上就把这个消息送到了塔提绿洲,还加上了加古勒的计划和自己的建议。

楚云飞等人接到消息时,天即将大亮!

\section{第九十一章 再袭塔尔人}

昨天白天进行了偷袭,今天再偷袭的话危险系数实在就高多了。大家正坐在一起商量,要不要冒点危险,给对方火上浇点油的时候,迪赞带回来了刚贝拉的信息。

乍闻这个消息,大家都很高兴,行动即将结束,塔尔人要撑不住了,那回部落的日子就不远了。

一天有大半时间在洞里窝着,给谁也不会喜欢的,尤其是迪赞。不过眼前这胜利来得似乎还是快了些,超出了楚云飞的想象,这塔尔人的意志太薄弱了点吧,就这样的人当初也敢跟刚贝拉张牙舞爪?

所以听到加古勒要求惩罚马库斯的建议后,楚云飞不得不怀疑里面是否会有专门针对自己三人设置的陷阱。

可刚贝拉带来了他自己对整件事情的分析,多特人这方本来没人知道加古勒究竟为什么这么快的妥协,以刚贝拉的理解是塔尔人不胜骚扰之烦了。可加古勒很明白地告诉了多特人:多特人邀请外来势力介入部落纷争,他对此很生气,本来是有采取同样行为,进行报复的心思的。但他实在是悲天悯人,不愿把事情激化,又想到两部落长久以来友好和睦的相处,所以愿意息事宁人,大家共享宝贵的绿洲资源,共同开发之类云云。

言之者凿凿,信之者寥寥,虽然加古勒说得很模糊——他知道的本来就少,刚贝拉听得也很晕乎。但以刚贝拉的智慧,还是不难猜出加古勒对“外来势力”的忌惮之心、恐惧之意。

至于后面突然发生的塔尔人保守势力的反弹,加古勒进行了及时的通知,同时似乎也隐隐有“投命状”的意思,能极大增加双方的信任感,更别说以后分成中多特人还可能会有若干好处。所以刚贝拉认为,这个险,大有一冒的价值的。

操!楚云飞心里又是一阵的乱骂,对你刚贝拉而言,“冒险”只是个单词,对我们来说,那可是实实在在的三条人命,这个险你冒来试试?

再说了,这事就算是加古勒的“投命状”,何尝又不是你刚贝拉的“投命状”?

不过,整件事情在逻辑上还是非常合理的,虽然楚云飞对非洲人的行为逻辑总是有“雾里看花”的感觉,但只要是人,总有个心理脉络所循的吧?

但是,楚云飞的不守规矩也是出了名的,虽然这事似乎是不会出什么问题,但是谁能担保没有个万一?不是个骗局?只有傻瓜才会把别人都当傻瓜看。要他再给马库斯一“木棍”的解决建议他是绝对不会采纳的,太不安全了!

楚云飞和自己的战友交代了自己的战斗意图,不要别人说什么就是什么,这次要强力攻击,给对方个狠狠的打击,至于那个马库斯,杀不死他最好,就算失手杀了他,加古勒还真能咬出大家来不成?他敢么?

迪赞在旁边听得心惊肉跳,可看到楚云飞有意无意地瞟他的那两眼,他还真没有马上出去报信的勇气。不过他确实算个聪明人,当听完楚云飞的分析,他心里居然认为楚云飞的计划更可行,有机会狠狠教育塔尔人一下的话,又何必那么温柔?

马库斯是上午到达工地的,事实上,逞了一时嘴上的快感,他昨天夜里根本郁闷得没有合眼,何必呢?这事就算处理好了,工地的施工还是人家加古勒说了算啊,这个该死的小毛孩,让那个白人带得更坏了!昨天要是能管住自己的嘴巴,让克鲁图冲在前面就好了。

夜晚还是在马库斯的悔恨中如期而至,马库斯把自己带的二十多条狗全部放了出来,天大地大,小命最大,虽然听说这东西不管用,但起码用来警戒还是会有点效果的吧。

可惜这次楚云飞他们已经不打算上门找人了。

马库斯提心吊胆过了将近一夜,天快要亮了,可疲惫的他心里居然有了点隐隐的欣喜:什么强悍的敌人,这不是扯淡么?还好自己够勇敢。等天大亮的时候,一定要派出搜索队好好搜索一下,搜不出来的话,哼哼……

马库斯的意淫还没结束,暴雨般的袭击突然降临!

成树国潜伏到了离那土坯房将近六十米的地方,猛然直身,手里的手雷连绵不断的扔出,一间房子一颗,马库斯那间房子……两颗!

骤然间,群狗乱吠。

当第一颗手雷开始爆炸的时候,成树国已经端起了步枪开始后退,身后二十米处,楚云飞的八一步枪已经开始了扫射!

当成树国的步枪也开始扫射的时候,楚云飞已经准备换梭子了。

虽然是扫射,但两人心知肚明,那子弹还是有一半落在了马库斯那间房子上。

很快就有保卫过来参战,不停的扫射暴露了两人藏身的地方,开始有凌乱的枪声响起,进行还击。

藏在远处的刘宁开始发威,打一枪换一个地方,冲着那远处红光闪现处一个个的点射,还想还击?你们自己找死,那怪得谁来?

射击持续了有六分钟左右,刘宁的点射看来给对方造成了很大的困惑,枪声始终不够激烈,不过时间已到,再不走被人包抄那麻烦就大了,一颗照明弹在塔尔人上方亮起。

就在塔尔人四处寻找掩体的时候,三个中国人站起身走人,留下一地尚在发烫的弹壳。

迪赞迎接他们的时候还是那句老话,“这次杀了几个?”

杀了几个?楚云飞和成树国对视一眼,虽说八一步枪穿透力很强,不过,那土坯房里的人恐怕还是受伤的可能性大点吧?两人不由得扭头看刘宁。

刘宁倒是不客气,算了算,“我一共打了有十二三枪吧?虽说有两百米,也总得死那么两、三个吧。”

迪赞张大了嘴巴,在夜里射击,两百米的距离,有这么恐惧的杀伤力么?吹牛,绝对是吹牛。

事实证明,刘宁确实是在吹牛,那天夜里,塔尔人一个都没死。

不过,伤者众多,有将近30人,大部分是被楚云飞和成树国的扫射击中的,原因嘛,那是因为土坯房相对安全点,里面挤进去不少过夜的塔尔人。

马库斯很幸运,只是往地上趴的时候蹭破点皮,事实上,因为他身份高,那屋子里只有保护他的四个人,其中只有一个被子弹擦伤。

凭心而论,刘宁也不算吹得太厉害,他打死了个刚卡人,那家伙自恃打过仗,自告奋勇地借了杆枪,本来是想借此立功,谋个富贵的。

\section{第九十二章 使用美人计?}

一个月后的一天,古基斯正在屋里同两个侍妾调笑,有仆人进来禀报:“主人,第六少主人找您,说有事。”

“刚贝拉?”古基斯皱了下眉头,“他最近不是在忙那个绿洲的事情么?让他等等,我马上出去。”说完顺手又在一个侍妾档下掏了一把,“等我回来啊。”

刚贝拉斜靠在大木椅上痴痴发愣,见到古基斯进来,赶紧站起来,“父亲大人,你好。”

古基斯挥挥手,示意他坐下,自己也坐到酋长的大椅子上,“上茶。”

古基斯和刚贝拉都很喜欢中国的茶,在他俩的言传身教下,整个“提坦卡布”部落喝茶人数的比例远远高于其他部落。

“刚贝拉,你不是在忙着塔提绿洲的事么?找我什么事?”

刚贝拉皱皱眉头,“塔提绿洲那里有点小问题,需要父亲你的指点,还有,我个人也有点困惑的地方,想请教下父亲大人我该怎么办。”

“哦,看来事情还不少嘛,先说说你个人的问题吧。”酋长大人端起茶杯吹吹上面的浮沫,轻轻抿了一口,古基斯非常喜欢这样的动作,因为在他看来,这象征着一种古老的优雅。

刚贝拉没心思去学父亲,茶还烫着呢,“我的那几个中国朋友似乎……似乎有离开的心思,可我不想让他们走。”

“嗯,我早就告过你最好别那么快给他们办好身份,你就是不听。”酋长看事自然有他的道理的。

“这三个人可都不好糊弄,我不想让他们误会我的意图,再说,他们还欠着我两件事情没办呢。”刚贝拉确实不愿意在中国人面前玩什么花招,尤其是面对楚云飞的时候。

“那你还头疼什么?只要他们遵守承诺,你自然可以想办法拖着不要求他们做事啊,每天什么都不做,养着他们就好了。”酋长大人也知道那三个人的厉害,绿洲之所以现在能和塔尔人合作开发,三人是功不可没的。养着这么厉害的人,花点钱根本不算什么。

刚贝拉明白父亲的意思,“我一直就在这么做啊,可楚说了,总不能一辈子不给他们事做,他们就一直呆在这里吧,我……我也不好说什么啊。”

酋长皱皱眉头,“多大的事情啊,你找两件难的事,要他们去做好了。”

刚贝拉急了,父亲还是没弄明白他的意思,“可我要他们做完,他们就可以走人了!”

酋长更不明白了,“走就走吧,只要让他们做的事划得来就行了,这有什么好担心的,难道你还真留他们一辈子么?”

看着刚贝拉着急的样子,酋长若有所悟,“是不是他们还有别的什么长处?你舍不得放人?”

刚贝拉摇摇头,“对,他们三人,尤其那个楚,很厉害的,哦,我是说很聪明的,上次袭击塔尔人的主意就是他出的。”

哦?古基斯来了点兴趣,“我还说那是你的点子呢,还说你年纪大了,长进了不少呢。”

刚贝拉脸上凝重了起来,“那个楚,他们做事真的不拘一格,就象上次他们打伤那么多塔尔人那次,本来加古勒的建议是让他们把那个顽固份子随便教训一下就行了,没想到他们根本不听,拿着枪就冲了过去,结果效果反而更好,你不知道后来加古勒看我的眼神,那叫个痛快啊。”

“加古勒,小利都的儿子么?”酋长大人也陷入了沉思,不过不是怀旧,“你是说这人不但聪明,随机应变的能力也很强?”

刚贝拉又摇摇头,“是啊,我本来想让他做我的助手的,不过他似乎不太情愿,可我也不好强迫他。”

古基斯笑笑,很慈祥的那种,“呵呵,白人助手,或者说黄种人助手,不要也罢,在索度国这片土地上,谁有资格给我们出主意呢?我们和他们学学倒是可以,不过,多特人的事情,最终还是要靠多特人的,你不看看加古勒的助手,也没出什么好主意吧。”

酋长大人不说还好,一说起来多尼,刚贝拉满肚子的怨气,“我真的怀疑多尼是个骗子,虽然说他金融方面知识还行,规划起来也一套一套的,可我怎么看他那计划也不合适用在咱索度。”

酋长大人向仆人瞟了一眼,仆人赶紧过来给茶杯续上水。

刚贝拉见酋长大人没什么反应,只得继续说下去,“可那加古勒对他是信赖有加,每次我提出建议首先都要先驳倒多尼,我觉得他纯粹是拿着他那点在欧洲的见识瞎卖弄,说起来头头是道,可做起来就差得太多了。只是,加古勒被他那些什么新名词、新概念弄昏了头,遮住了眼。”

酋长大人没有看刚贝拉,只顾低头看着自己的茶杯,一副心不在焉的样子,“你确定是这么回事么?”说完抬头瞟刚贝拉一眼,继续低头看他那茶杯。

明明就是这么回事的,刚卡拉心里这么想,但嘴上可不敢这么说。父亲的威严伴随着他的成长而逐步加深,虽然,现在看起来在很多事情上,父亲的决定未必能好到哪里去,但起码表面上刚贝拉是绝对不敢冒犯父亲的。

刚贝拉已经习惯了父亲这种质疑的口气,轻描淡写的质疑,但是他真的觉得委屈:我已经长大了!

不过,话绝对是不能这么说的,“我认为是这样的,加古勒那个毛头小子,他能知道什么?他这辈子最远也就是去过喀津霍,至于见识那就更差劲了,要不能让那个多尼哄得团团转?”

古基斯点点头,“刚贝拉,你什么时候才能彻底长大,让我省点心呢?加古勒再有不是,但他为了这个工程确实是费了心的,他不费心怎么可能亲自跑到喀津霍去找人?再说,他现在的身份也由不得他不操心。”

又抿口茶,“至于那个多尼,就算有点问题,那也是加古勒急于求成了,年轻人嘛,性子急点很正常。可你不年轻了吧?随便小看人,是要付出代价的!对了,你不是说那个中国人很聪明么?这个事情,你大可以去和他讨论讨论,他的见识未必能比你差多少吧?”

说起楚云飞,刚贝拉又是一脸的无奈,“问题是他不肯帮忙,那家伙,张口闭口就是钱,要不就是在那里练中国功夫,我让他拿个主意,他反而说不了解索度的情况,不能乱出主意。”

古基斯又摇摇头,“嗯,你看人家,也是年轻人,事情做得多稳重?他不了解情况,你可以跟他介绍啊,我怎么生了你这么个笨儿子?中国人特别讲情面,这话还是你同我说的,你就不会用感情拴住他么?他和多……多纳肯定是会有些共同语言的,你想办法让他们见见不就完了?”

刚贝拉很想纠正父亲,“那是多尼,不是多纳。”不过他显然没那个胆子,也不敢分那个心,“感情?您是说美人计?这个……黄种人会喜欢咱们黑种人么?”

\section{第九十三章 把话说开算了}

如果说刚才古基斯是随便挑点刚贝拉的毛病,大致还没什么意见的话,那现在他可是真有点火了,不过他也知道自己十几个儿子大都是这样,一见自己就吓得不能正常发挥思维了,“美人计?你就不能想想别的什么?你以前怎么和你那些中国同学相处的?真是骂你的心都没有了,这样,我给你个建议,不是还有俩事没给你做么?一件就是让他们能留多久留在这里多久,这不算很难的事吧?”

“不难,”刚贝拉的思维还是很快的,重压之下没有走型,算不错了,“我跟他们商量吧,讨价还价嘛。”

古基斯真是有点啼笑皆非的感觉,他决定真的教儿子点东西,总让他们摸索也不是个事,虽然那样得到的最深刻,“嗯,其实你不需要一定使用他们,有时候威慑的力量大于使用这种威慑,拳头没有打出去的时候是最吓人的。”

“还有,你站在他们三个人的角度想过没有?他们现在最需要什么?我给你提个醒吧,谁也不傻,他们自然知道你有心利用他们的武力,可他们好不容易从刚卡逃出来了,会因为丢了辆汽车就来实现诺言么?他们是需要有个暂时停留的地方,但这不意味着你就能用这个承诺逼他们做太危险的事情,你的命是命,人家的命就不是命?他们觉得他们的命比你的还值钱呢。”

两辆车!刚贝拉在肚子里纠正父亲。

“他们着急做完那两件事也是这个意思,怕等你有了更危险的事情再用他们,现在你能给他们找的事,不过就是麻烦点的事而已,他们自然不想再等,要知道,他们毕竟是中国人,不是咱们族里随时方便牺牲的族人!”

“还有,他们不是给家里打电话了?可他们现在还是回不去,在他们被正式赦免前,他们在哪里不是呆?”

“使用好你说的中国人的智力,同时让他们的武力威慑别人,这是我给你的建议。”

……………………

楚云飞听到刚贝拉源源本本转述的酋长大人的话,不由得一阵叹服,看来这高人是哪里都有啊。不过值得高兴的是:看来以后哥仨也不会有太危险的事做了。

刘宁和成树国也听得目瞪口呆,谁说中国人最聪明来着?大部落的首领,果然是不同凡响。

不过,刚贝拉也不能小看,这家伙能把事情全说出来,证明他已经明白怎么用感情拴人了。这父子俩,还真是一个赛一个啊。

不过楚云飞很快就恢复了正常,“那好吧,看来第二件事情也算有了着落了,我们呆在这里的时间就不说了,按一年算好了。”

刚贝拉的攻心策略没能起到效果,心里绝对不会是特别舒坦的,不过已经这样了,再改弦易辙的话,效果恐怕……别说效果了,怕都直接惹人了,“无所谓,时间长短,只是个意思而已,表示下……呃,表示下对我那半个故乡的尊重。”

楚云飞他们都忍俊不禁了,没办法,三人其实对刚贝拉的秉性已经很了解了,也曾经做过深入的探讨,毕竟现在他算三人唯一的老板或者说债主。现在看他捂着腮帮子非要说牙不疼的样子,真的觉得好笑。

不过严格说起来刚贝拉真的没什么太大的毛病,最多也就是爱玩个小聪明,人其实也不算坏,所以楚云飞他们也没过分作弄他。

刘宁是三人里年纪最大的,家境也最好,而且也不是太爱开玩笑,“这样吧,刚贝拉,一年按四百天算,已经过去一百天了,剩下三百天,按日子折你那两辆车,我们要是万一有事,不得不提前走的话,赔你的车钱,超过的话,你也不用出薪水,还象现在管吃管住就行,可以吧?”

成树国也赞同,“对,这期间要有什么小忙,你张嘴就行了,我们也不是斤斤计较的人。”

说到底,刚贝拉还是个商人气息很浓的聪明人,他很快就弄清楚了这事对他来说是十分划算的,很痛快地答应了。

因为这次没出多少力就轻易地获得了刚贝拉想要争取的大部分东西,他不得不承认,有时候听听父亲的话,其实也不是坏事,不过听那刘和成话里的意思,也显示出俩人是有头脑和能做主的,是不是自己以往有点忽视他们了?

既然成树国已经答应了刚贝拉不“斤斤计较”,楚云飞不得不在第三天跟着刚贝拉去见加古勒和多尼。

楚云飞本就有去见见多尼的心思,毕竟比较而言,欧洲人似乎能给自己更多的信息。不过当身份证明到手的时候,楚云飞他们私下商量了一下,觉得还是尽快帮刚贝拉做完三件事的好,省得将来再有什么棘手事来麻烦他们。

至于国内传来的消息,更是和古基斯猜测的差不多:他们毕竟算不遵从命令,虽然不会有人专门来追杀,但赦免那是想都不用想的事。

成解放和刘群孜孜不倦的活动并没有起到什么作用,两个老军人只能劝三个年轻人多等等,“平反总要个时间的,现在你们是先保住自己的小命要紧,说什么都是假的,‘肃反’的时候那么多人不也都是白死了?到最后平反了人也活不转的。”

当楚云飞出现在加古勒和多尼面前时,两人并没有相信他是刚贝拉说的“助手”这一身份,中国人,那该是会中国功夫的吧?

加古勒和多尼对视一眼,心里都有了数,眼前这个人,应该就是弄得工地鸡飞狗跳的人,就算不是,也脱不了干系的。刚贝拉把他带来,会有什么事情?

刚贝拉可没想那么多,反正楚的身份绝对是没有暴露的,就算他们能猜到,那也不过就是猜测而已,“多尼,这是我中国留学时候的朋友,听说有欧洲人在这里,想找你聊聊。”

多尼可没想到楚云飞来的目的是为了他,不过想想,中国人肯定不会是为了追杀他来的,那自然是没什么好担心的,“呵呵,能在索度见到中国朋友,我真是太荣幸了。”

楚云飞淡淡一笑,“我也同样荣幸,多尼先生,您已经知道我来自中国了,我还不知道您是哪个国家的人,能告诉我么?”

\section{第九十四章 助手间的沟通}

没有心理阴影的时候,多尼本身还是个很健谈的人的,他很快就和楚云飞混熟了,于是楚云飞了知道他是出生在波兰,不过很小的时候就随父母亲移民到了法国。

到后面两个人聊得性起,多尼索性把自己的遭遇也告诉了楚云飞,反正中国人是不可能跟欧洲的黑社会势力有什么关系的,事实上,相对即将一体化的欧洲,中国和欧洲在经济方面的联系都算不上是密切呢。

看到两人聊得如此的热火朝天,加古勒和刚贝拉相互看了看,都有点被忽视的感觉,心里也都是一个念头:看来人家的共同语言比咱俩多得多啦。

因为有种被喧宾夺主的感觉,刚贝拉懒得呆在两人旁边,拉起加古勒,“走,咱们上工地看看,让他们在这里聊吧。”

加古勒其实还有心在两人身边旁听一下,两人言语之间有很多话题他都很感兴趣,不过刚贝拉都这么说了,再说被人晾在一边的滋味他确实也不喜欢,就跟着刚贝拉出去了。

看到两人都走了,屋子里没了别人,楚云飞四下看看,“多尼,听说,塔尔人的绿洲开发规划是你搞的?”

多尼心里“咯噔”一下,“不错,是我搞的,有什么问题么?”

有什么问题?问题大了去啦,刚贝拉本来就不是个好糊弄的主,这次有心让楚云飞来帮他考察一下此人的可信程度,两人自然是先碰过头,琢磨过的。

楚云飞是不懂商业的,起码是不够精通的。不过,就他读了那么多书,人又够聪明,又有刚贝拉在一旁帮助分析,他还是马上发现了这人的来路不够正,或者说起码是水平不够专业。

虽说刚贝拉和楚云飞都可以怀疑多尼,但合适戳破多尼谎话的只能是楚云飞,倒不是说刚贝拉嘴皮子不够灵光,实在是因为有加古勒这么个合作伙伴夹在中间的缘故。尽管刚贝拉绝对是出于公心才这么做的,可谁也不能担保加古勒会怎么看待这事。

不过刚贝拉也没有一定要难为多尼的心思,毕竟人家挣的是加古勒的工资,跟他可是一点关系也没有。他只是想让楚云飞帮忙肯定他自己的猜测,顺便让这个聪明人敲打敲打多尼。

敲打也不需要太狠,只要让多尼明白他刚贝拉不是那么好骗的就可以了。同时也能让对方在以后的工程中配合一下,不要无事生非地出那些华而不实的点子,使工程进展顺利点就行了。毕竟是多特人和塔尔人的合作,而这个项目目前看来是经不起太大风浪折腾的。

楚云飞也比较欣赏刚贝拉“利字当头”的想法,本来世界上就不需要有那么多的意气之争的,经历了一番锻炼以后,楚云飞起码在目前是这么认为的。不过,这事具体怎么做,楚云飞并没有完全听从刚贝拉建议的打算。、听到多尼的反问,楚云飞笑了笑,“有没有问题你自己也知道吧?我可是中国人,比索度人知道得总要多点。”

多尼马上就反应了过来,原来刚贝拉带这个中国人来是这么个意思,那刚才自己的话似乎就说多了点,不过,他显然不可能就这么放弃自己的伪装的,“很抱歉,我真的听不懂你的话,不过,似乎你对我有点成见?”

楚云飞根本就没打算从嘴皮上说赢对方,也没打算按刚贝拉的建议从整体规划中指出漏洞迫使对方认输。

因为经济这个东西,没有谁可以确定到底怎么做才是正确的,有争议的地方实在太多了,套用一句评论《哈姆雷特》的话,那就是“一千个经济学家眼中经济有一千种发展方式”。

驳倒对方倒不是最难的,但是要迫使对方因此而认输那几乎是不可能的,再说楚云飞本身也算不得精通经济。

所以楚云飞并没有介意多尼的反应,“我觉得相对索度人来说,咱俩更容易沟通些,不是么?而且你的规划,似乎也是超前了点,似乎,呃,似乎你计划把这里建成个新的迪斯尼乐园?还是说我看到它就会想起欧洲有很多类似的游乐场所呢?”

多尼自然知道对方在说什么,人家是在说自己这点规划,只要是在欧洲呆过一阵的人都能搞得出来,在坐实自己“骗子”的名头呢。

不过,多尼也是个见惯风浪的主了,“呵呵,我想你不是经济学专业毕业的,其实,很多游乐场所都有相似的地方,但是,一定要有不同的卖点和计划的,不过,似乎不是你能够了解的。”

来了,楚云飞心想,果然是这样,先给自己扣个不懂经济的帽子,然后就该是舌战了吧?不过,谁会有兴趣跟你讨论这个东西呢?“呵呵,多尼先生,经济学我自然是不懂的,我觉得咱们是没必要在这个上面抬杠的,我其实很想知道你对自己的将来是怎么计划的。”

我的将来?多尼还真没想到对方居然马上把话题转开,还是这么大的跳跃,他一边在急速地思考着对方的用意,一边打着哈哈,“楚,我觉得你的思维跳跃性很大啊。”

楚云飞笑笑,“你计划在这里呆一辈子不回欧洲么?还是计划将来有机会就来这里投资参股呢?还是说呆几年直接走人呢?”

多尼看不出里面有什么陷阱,自然是有什么说什么,“自然直接走人,说话说这个地方我实在是不愿意呆,实在是没个去的地方,至于投资,这两族已经够热闹的啦,我是不想掺和进来的。”

“那就是了,”楚云飞早想好了怎么回答,不过多尼这个答案还是让他少了不少废话,“你既然不想长呆在这里,这地方的规划还是多听听主人的意见吧,你不觉得,如果他们再这么继续扯皮下去,工程无限期停工的话,你有没有可能再回到喀津霍喝自己的夹克呢?”

“那倒也是,”多尼实在不算个听不进去话的人,问题是刚贝拉一直就没有同他交流的时间和空间,再说,信任度多少也是一种制约因素,“其实他们认真同我讲讲苦衷的话,我还是会充分考虑他们的建议的。”

\section{第九十五章 你们有多少人}

“这是你自己没想到,怪得谁来?”楚云飞说话开始直来直去,不过诡诈也要带上点的,“刚贝拉还跟我说要去‘博睿’查查有没有你这么个投资专家呢,不过我觉得没必要。”

多尼还真没想到自己引起了对方这么大的反感,在他眼里,那些人不过是一帮非洲土著而已,虽说那个刚贝拉是多少懂点经济的,不过他能见识过多少场面?如果自己不坚持自己的观点的话,那自己的高薪水岂不是会有人置疑?

“博睿”投资公司是有多尼这么个人的,不过是彼多尼非此多尼就是了,多尼冒充此人无非是于此人有过一面之交,而名字又恰好相似就是了。

多尼肯定不是楚云飞一句话就能诈住的,不过面对此情此景,他绝对是没有必要再惹楚云飞了,人家肯这么说,起码是有这么个想法了,自己总不能逼着人家去这么做吧?他可不能保证那个多尼也离开了“博睿”公司,再说,对方只要搞到张照片就足够自己喝两壶了。

野蛮人的钱非常好骗,但一旦穿帮,下场的悲惨也会用“非常”来形容吧?

“呵呵,想查你们就去查好了,不过,别给我把公司雇佣的杀手招来,大家怎么也算是朋友一场吧?没必要做得太过份的吧?”

楚云飞自然能品出这色厉内茬的味道,把人逼太急总不是好事,“是啊,我也是这么个意思,不过,作为朋友,我真心地劝告你,如果想这个项目继续下去,你还是多尊重一些刚贝拉的意见吧。”

多尼被楚云飞连捧带逼,噎得够呛,不过这家伙确实算开朗的,“嗯,也算,我专心混我的工资好了,反正是加古勒出钱,这个你总没意见吧?”

楚云飞大有深意地看了多尼一眼,灿烂地一笑,“对啊,加古勒的钱,我实在是没必要费心的,只要你以后记得合作就好了,不过,小地方他们需要指点的话,你这个金融专家也要指点的嘛,毕竟他们懂得太少,不是吗?”

多尼摇摇头,“唉,你这个家伙,我记得以前有人跟我说过,中国人是很善良的。”

楚云飞又笑笑,西方人的性子确实比东方人直爽,这事要搁在中国,当事人不定会怎么打死不认帐呢,多尼可是干脆地认了,虽然是默认,不过总不能指望人家明着承认吧?

“我怎么觉着自己非常的善良,是典型的中国人呢?”

“你善良,那天下就没恶人了,”多尼的话里带着西方人惯有的夸张,话题又轻松了起来,“居然还想逼迫我这个可怜的、受到不公正待遇的流浪者。”

“切,谁知道你那可怜的遭遇是不是自己找的?”不知道为什么,楚云飞居然开始有点喜欢眼前这个家伙了,也许是因为他的直率?

“哦,看来我想的没错,你就是这么恶毒,居然怀疑起我的正直来了,天哪,难道你们中国人都是这么爱怀疑别人么?”多尼看出楚云飞没什么别的用意,居然开始标榜自己。

不过他可没等楚云飞再说出来什么难听话,马上转移了话题,“楚,听说你们中国功夫其实都是吹出来的,其实并不怎么样,是这样的么?”

明明知道多尼不怀好意,不过楚云飞还是根本不介意,毕竟加古勒有投命状在那里呢,眼前这家伙也有小辫在自己手里。对这种话,他当然不会太客气,“哦,你想试试么?”

多尼等的就是这句话,“原来……原来那些事,果然是你们做的?”

楚云飞自然也要学学他的口气,“天啊,你在说什么,我怎么听不懂啊?”

有这句话就够了,聪明人都是一点就透的,他不反驳已经说明所有的问题了,毕竟大家不算熟悉不是?

多尼也学着楚云飞的样子,“大有深意”地看了楚云飞一眼,不过楚云飞根本没当回事,威慑,一点迹象都不表露的话,怎么做到威慑?至于把柄么,大哥就别笑话二哥了。

楚云飞笑笑,多尼笑笑,然后俩人一起大笑起来,越笑声音越大,足足笑了有5分钟。

多尼确实是很直爽的,“一切都是猜测,我保证只是猜测,你也能保证吧?”这种场合,牌少的自然要姿态低点。

楚云飞本来想说句“哪里有什么猜测”之类的话,不过受到多尼的影响,也很爽快地说了句,“那是自然,毕竟对索度人来说,咱们都是外人,不过我跟刚贝拉的交情可是不在钱上的。”

多尼点点头,“我知道了,以后我会尽量地配合你的朋友的,不过,加古勒在场的时候,我就算要做点什么也需要使用点技巧的,你清楚吧?”

楚云飞也点点头,“我知道了,你的话我替你转到,其实,刚贝拉那人是很聪明的,你不用做得太过,毕竟目前你工资是很高的。”

多尼同加古勒还是有点感情的,“是啊,加古勒也很聪明,不过,他们确实是有点闭塞……对了,我可不是说刚贝拉,那家伙去过中国,中国人的狡诈,我看他是学了不少。”

楚云飞白他一眼,“他再狡诈也赶不上你一半的,什么事都敢做。”

听了这句绝对的废话,多尼的表现很奇怪,他居然愣了半天。

等他回过神来,已经是一脸悲伤的模样了,再开口的时候居然充满了感慨,“唉,我要真有那么狡诈就好了,我还至于这样地亡命天涯么?”

楚云飞看到这个开朗的人忽然换了副嘴脸,自然知道他想起了不堪的往事,拍拍他的肩头,“行了,不幸的人各有各的不幸,我还有中国功夫也解决不了的事呢,关键是,我们现在都还好好地活着,不是么?”

多尼点点头,振作了起来,“不错,关键是我们都还活着,这就够了。”

楚云飞向窗外望去,碧蓝的空中,有两朵白云在优游地舒展,感受着这份闲适,活着,真的很好!

耳边多尼的声音又响了起来,“楚,你们……有多少人?”

\section{第九十六章 大嘴巴多尼}

楚云飞从感慨中惊醒,扭头面对多尼,又是那副似笑非笑的样子,“你指的是什么?多尼先生?我听不懂,还请你解释一下。”

看你那样子魔鬼才相信你听不懂,多尼挺挺胸脯,做出一种挑衅的神情,“就是你想的那种意思,你总不会什么都没想吧?”

看来还不算大嘴巴,楚云飞对多尼有了个评价,“嗯,现在有多少人迟早你会知道的,不过,我们到底有多少人你永远都不可能知道。”

“好,有同伴就好,”多尼也知道不能指望眼前这家伙说多少实话出来,“我原以为,你的同伴是索度人呢,看来,是我们的消息不太灵通啊。”

楚云飞觉得有点不安,这家伙是不是有点过于好奇了,于是用一种略带威胁的语气问多尼,“多尼,你能解释一下为什么会有这个问题么?”

“有事相求,”多尼说得也很晦涩,不过下句马上就露馅了,“说实话,我真不习惯你这么说话,咱们能随意地谈谈么?像朋友一样地谈谈。”

楚云飞哑然失笑,“你呀,就不能把气质装得高贵点么?说吧,什么事?”

多尼没笑,不过,他对楚云飞说话的语气很满意,“你知道,我是有点小麻烦的,要不也不可能呆在这个穷得要发霉的地方,我想用你的力量帮我把这个麻烦解决掉。”

楚云飞没想到这家伙居然能在三言两语中把话题延伸到这样的深度,这人不是一般的爽快,“你觉得我有这个能力么?能解决你也解决不掉的麻烦?”

多尼沉思半天,才缓缓开口,“我也不知道,不过,以你们的表现,啊不,以我的猜测,应该问题不大吧,你还有同伴的,不是么?”

楚云飞并不是同情心泛滥的那种人,“你知道么?刚才望着窗外的白云,我在想,活着,真好!”

“而你的麻烦,解决起来,没准是会违背我这种感慨的吧?”

多尼并没有吃惊,眼前这家伙似乎什么都知道,不过,问还是要问问的,“你怎么会这么想?”

楚云飞摇摇头,“我本来只有百分之九十的肯定,现在却是百分之百的肯定了,多尼,我发现,你这人,似乎不能很好的管住你的嘴巴。”

多尼自然知道自己有些交浅言深了,不过,这实在是上帝良心发现,给了他一个千载难逢的机会,再不抓住的话,路瑞、丽迪娜他们的仇要等到什么时候才能报,他要等到什么时候才能重返欧洲那繁华之地?

当多尼发现那种袭击方式绝对不属于索度人的能力范围时,虽然很有点为自己的处境担忧,却又何尝没有丝丝希望涌心头呢?所以,当时他才力主与多特人的合作的。

等到今天他才知道,这一切都出自中国人之手,那所有的事情都好说了,中国人……是绝对不可能和托尼他们有什么关系的,所以,他必须抓住眼前这个机会。虽然这事可以慢慢来办,但在那种突然降临的幸福到来时,他忘形了。

“实在抱歉得很,楚,上帝保佑,居然让我有了解决麻烦的希望,所以我有点失态,我发誓,我不是那种嘴不牢靠的人。”多尼承认自己的错。

虽然楚云飞还是很欣赏多尼的直爽,但他绝不会因为那一点点的好感就丧失原则,“我也非常抱歉,恐怕是要让你失望了,对我来说,没有比我和我的同伴的生命更为重要的东西了。”

多尼显然对这个回答有充足的思想准备,“这世界上,除了少数人的信仰,没有什么不可以交易的,你为什么不开出你的价码来呢?”

这家伙现在才有点商人的味道,楚云飞心里如是评价,不过他实在是不愿意再考虑这件事了,“没有商量的余地,真的,我只能说,我很抱歉。”

没有余地才怪,多尼绝对不会放弃到手的希望,他的嘴皮也是很厉害的,要不能把加古勒骗得不辨东西,找不到南北?

“先不要拒绝我,我的中国朋友,你没有非常想做又做不到的事情么?交易嘛,未必一定要用金钱的,虽然我并不缺金钱。”

楚云飞本来已经决定再不听他劝说了,可多尼的话里一下带出两个他感兴趣的问题,于是他不得不张嘴询问,“抱歉,你现在的身份都似乎不太体面,还说自己不缺金钱,我很想问问你,我看上去是不是很好骗的样子?”

哦,怕钱少啊,那好说,多尼信心满满地继续劝说,心里已经有了九成的把握,“我的钱都在欧洲,卡被偷了,所以取不出来,你开出价码来好了,只要你能保护我在欧洲呆一天,我先给你们五万美金,算定金好么?”

楚云飞操心的重点可不是在钱上,只不过是随口一问,想通过多尼的话判断下真实性而已,不过还是被他的狮子大张嘴吓了一跳,可再想想就明白了,“哦,起码你能弄点钱或者补个卡什么的,是吧?看来索度的苦日子把你憋坏了。”

多尼以为已经打动了楚云飞,自然也是有什么说什么,“操,我根本就没想在索度呆,只想中转一下去加勒比,降低点风险就是了,没想到差点死在这里,真是够倒霉的。好了,开出你的价码吧。”

楚云飞笑笑,摇摇头,“真的很抱歉,我已经说了,没有商量的余地。”

多尼的心情,从平淡到兴奋,又从兴奋到略微失望,这最后一下从兴奋的顶端又掉到绝望的深渊,这种打击,他实在地有点接受不了,眼睛不由得有点发直,大脑也跟着麻木起来了。

看着多尼这副痴呆的神情,楚云飞也不好再继续捉弄他了,毕竟他对另一个话题是非常感兴趣的。

于是,麻木中的多尼又听到了楚云飞的声音,一句话,让他如聆纶音,希望也再次出现,“不过,我还是很有兴趣想知道,你有没有什么途径把我弄到沙特,最好是合法的身份。”

多尼几乎是马上就蹦了起来,“这件事情简单,非常容易做到。”

按理说他完全是可以故做沉吟一番,然后再细细讨价还价,最后很为难地答应。不过现在的多尼已经受打击够多的了,实在是不敢再玩什么心思了,先把事情敲定再说吧。

这就是楚云飞先问资产,然后拒绝,最后再提问所产生的效果了。

\section{第九十七章 多尼很可怜}

不过,可怜的多尼先生马上又被打击到了,还好不是很沉重,但是施压者是逐步加大压力的,很有技巧的那种,原因么,那自然是他先期答应得过于爽快了。

楚云飞最先表示的肯定是小点的事,“多尼,你听我说完好么?这只是我的一个问题而已,是解决你的麻烦的先决条件。”

多尼还沉浸在兴奋里呢,听到这话也没做出什么及时的反应,不过这事对他来说确实是件小事,“好的,这算先决条件好了,开你的价码吧。”

话说出口,多尼才反应过来,似乎,自己答应得是不是太快了点?

“你能在沙特找到非常熟悉当地的向导的话,那咱们的价钱就好说了,我可以优惠考虑的。当然,我说的是很牢靠的那种向导。”多年心愿瞬间变得越来越近,楚云飞也有点失措了,只是在本能讨价还价。

多尼已经冷静下来了,操,这价钱还没说呢,这折扣打不打还不一样?不过没办法,谁叫现在是卖方市场呢?不过,还好也不是什么大事,“好的,沙特……那里不算什么的。”

不过楚云飞接下来的条件就让多尼开始吐血了,“解决你的麻烦是会很危险,是吧?……哦,那这样,只能我一个人跟你去了,我可不想搭上同伴的性命,五十万美圆,不多吧?”

多尼很后悔自己点头告诉对方这事的危险性,待到听完楚云飞的话,马上就蹦了起来,“我出五十万美圆,你再带五个同伴。”

开玩笑,我这里总共不过才三个人,哪里去给你找五个同伴去?“那就十万美金好了,就我一个人去。”

多尼是绝对不干的,“那我出六十万美圆,你还是再带五个同伴,这样可以了吧?”

楚云飞听到多尼这话,真有点心思代战友把事情答应下来,不过是只去三人就是了。鬼才知道在国外还要呆多长时间呢,有人眼巴巴送钱有不要的道理么?不过再想想,自己的战友,自己可以为他们死,但绝对没理由带他们去找死的。

楚云飞缓慢地摇摇头,“只是我一个,我不能代表我的同伴,他们,必须好好地活着。……如果你不接受,咱们就当今天什么都没说好了。”

多尼也摇摇头,颓然地坐在地上,又一次打击降临了,“只你一个?你会被他们撕成碎片的,天哪。”

呆了一下,多尼忽然又想起了什么似的,“楚,能带我,带我见见你的同伴么?”他心里想的自然是再发挥他的嘴皮功夫。

“你能同我们国家主席说上话的话,我想我的同伴是很愿意见你的。”这是楚云飞的答复,一个韵味极长的答复,不但真实,而且让人浮想联翩。

不过楚云飞也知道自己只靠嘴皮子,是不可能让对方相信自己值多少钱的,“我很愿意向你展示我为什么值十万美圆,可惜现在不是很方便,所以,咱们暂时不说这件事了,好么?”

多尼显然对楚云飞的意思理解得不是很透彻,“没什么不方便的吧,这里有张桌子,你可以把它打烂的。”

楚云飞听得啼笑皆非,感情这家伙眼里,自己只代表中国功夫啊?

“打烂那张桌子很容易的,不过,显然你并不明白什么是真正的特种兵,真正的中国特种兵。”这是几个月以来,楚云飞头一次表露自己的身份,四周没人。

听到这话,多尼终于明白了,果然是军人,特种兵,听起来还是很厉害的那种,不过楚云飞下句话更加影响了他的心情。

“多尼,我不得不告诉你,就算我肯帮你,恐怕也是一时动不了身的,怎么也得一年以后吧。”

多尼很想直起身子大吼一声“别以为我一定要请你”,不过想想在索度的痛苦经历和欧洲的悲惨遭遇,咬咬牙,“那好吧,我同意你的观点,暂时不说这事。”

然后多尼一屁股坐到椅子,开始怔怔地发呆。

楚云飞看到多尼的样子,心里居然有了一丝丝的怜悯,虽然以他目前的处境并没有怜悯别人的资格,“别担心,事实上,我比你想象的要厉害得多,你以为你的命是命,我的命就是垃圾么?而且我必须留下我的生命去沙特,即使死在那里。事实上,对我来说,钱多钱少并不重要。”

“我们的人里面,只有我有非去沙特不可的理由,其他人,你是不可能收买得动的,除非你真的认识我们的国家主席。”楚云飞其实很少跟不熟悉的人废话的,不过,大家总也算“同是索度落人”,同病相怜的感觉多少是会有点的。

多尼被他这几句话说得心思又活泛了起来,莫非,这些人是那些中国政要的保镖?那一定会是很厉害的。不过,纵然再厉害,也不可能一个人对那么多人吧?

于是,坐在那里的多尼终于下定了决心:绕过眼前这个中国人,偷偷地去找别人,当你的同伴都愿意去的时候,你总不能拦着他们不让去吧?

不过,这事想起来容易,做起来却是非常不容易的。多尼没有意识到,没有楚云飞的支持的话,他找到其他的中国人需要花多长的时间,而要建立起信任的话又得费多大的心思。

事实上,多尼和楚云飞已经没有继续谈论这事的机会了,因为刚贝拉和加古勒走了进来。

加古勒的兴致看起来不错,“多尼,似乎你们俩聊得很开心?”

而刚贝拉则是一副愤愤不平的样子,“多尼,我强烈要求旅馆目前只建设一层,屋顶要木头的那种,还有,取消那些狗屁自然风景园,那么大的沙漠不够他们看的么?我宁可再建个游泳池让他们洗澡!!!”

楚云飞很有礼貌地冲加古勒点下头,“加古勒先生,我非常惊讶,居然在这里能遇到如此精通经济的学者,我不得不佩服您的眼光。”

听到这话,刚贝拉瞟一眼楚云飞,这家伙的工作效率不错嘛,这么快就搞定了可恶的欧洲人。

可怜的多尼又开始头疼了。

\section{第九十八章 一间小酒店}

接下来的日子里,绿洲工地建设走上了快行线,推崇简约的风尚,不但降低了成本,还在短短的三个月内施工完毕,赶上了当年夏天的旅游旺季。

不过,由于索度人口组成中,白人基本上是不存在的,所以旅游收入是远远比不上南非等旅游大国的。沙漠渡假的主要消费群体还是索度国内的游客,来的白种人主要是各种富有冒险精神的猎奇者,可就是这小小的一部分人支撑着索度沙漠显现的利益。

“提坦卡布”和“塞脱尔”作为有前瞻眼光的部落,开发旅游带来的利益还在其次,关键的是,两个部落已经隐隐成为了索度经济新热点——旅游业的代言人,有形和无形的资产在极大地丰富着。

当然,“打开窗户进来的不可能只是新鲜的空气,还有蚊子和苍蝇”。在一片兴盛景象的背后,各种不是那么美好的衍生物也在滋生中,诸如卖淫、赌博、吸毒、强奸等等。

不过这些毕竟不是社会的主流。

九月份,“提坦卡布”部落的一间小酒店。

这间酒店很有些美国西部的风格,当然不是说现在的西部,而是十九世纪初淘金热时的西部,也是一派简约的风格,但又给人以非常放松的感觉。

简陋的木桌木凳,还有一只高的长凳挡了块木板充做吧台。再夹杂上门外漫天的沙尘,要不是店里坐着的大多是黑人的话,真会让人有种时空错乱的感觉的。

午后的索度非常的热,尤其这里又靠着沙漠,所以酒店里人不是很多,只零散地或坐或站着二十来个黑人,还有一张桌子旁坐了两男两女四个白人和一个黑人,那黑人很明显是白人们的向导。

一个粗壮的黑人正靠在那吧台的木板上跟里面的酒招待嬉皮笑脸地说着什么,酒招待是个又高又胖中年黑人男子,不停地在那里点头。如果凑近点的话,能听出来,那黑人是想跟招待赊点酒。

看到酒招待无情地拒绝了自己,粗壮黑人就有了点恼怒的意思,大声嚷嚷起来,“蒙塞孔,没想到你这么不够朋友,不就是点酒么?我塔多拉是欠人钱不还的么?”

酒招待很是不屑眼前这人,也大声回敬,“塔多拉,有外人在,我本来是想给你留点脸的,没想到你居然醉成这个样子,要赊酒可以,把前天,哦不,昨天的酒钱拿来就好了。”

塔多拉脸上更挂不住了,重重一拍吧台,才待要说什么,那酒招待的话先到了,“你喝得太多了,这里,是你撒野的地方么?”

塔多拉愣了半天,看四周人的眼光都看向了自己,悻悻地闭住了嘴。走近身旁的吧女,狠狠地拧了对方的屁股一下,在吧女的尖叫声中向门外走去,把简陋的地毯跺得尘土飞扬。

吧女屁股被拧,大声地诅咒着塔多拉的长辈,粗言秽语滔滔不绝,终于有人受不住了,拍桌而起,“够了,欧迪丝,有本事你找个相好弄死他,别影响我们喝酒。”

听到这话,已经到了门口的塔多拉回过头来,“找人弄死我?凭这个烂女人?让我在床上弄死她还差不多。”

一个声音自门外响起,“是么,塔多拉?我不介意弄死你,不过,我要的是欧迪丝的妹妹,而且,在弄死你之前你要先把钱还回来。”

听到这个声音,嚣张的塔多拉的脸登时变了颜色,一个激灵过后,他开始慢慢地向后退。

一个同塔多拉身材相仿的年轻黑人走了进来,看着哆嗦成一团的塔多拉点点头,“没用的,欠债还钱是天经地义的事,你能在这里躲一辈子么?不要逼我剁掉你的手。”

那酒招待又说话了,看来他认识的人实在不少,“库提,你要处理这只小羊还是等他出去再说,别打扰了我的客人。”

库提笑笑,“这个我自然是知道的,不过谁叫他欠了迪赞大哥的钱呢?迪赞大哥的钱是谁都可以欠的么?早叫他不要赌了,他非要借钱去赌,愚蠢的家伙。”

酒招待自然不可能相信库提的话,其实,迪赞最近一直就在做放高利贷的生意,还好总还算乡里乡亲一脉,很少做太出格的事情。关于这事,就算酒招待心知肚明,也不可能随便说出来,毕竟,做他们这行是要管紧自己的嘴巴的。

库提看到酒招待不再出头,恶狠狠地想塔多拉走去,狞笑着问,“要我拖你出去么?”

塔多拉早已吓得魂不附体,“库提大哥,你再缓我两天,再缓我两天好不好,我马上出去借钱,我马上去……”

这种垃圾库提见得太多了,事实上,迪赞并不算是很过分的人,被逼到塔多拉这步的总是咎由自取的多些,“我数三声,乖乖跟我出来,一、……”

塔多拉心胆俱裂,没命地喊了起来,“救命啊,救命啊。”

那四个白人却有些看不下去了,其中一个男子属于正义感明显过剩的类型,“那位朋友,他欠你们多少钱?”

库提扭头看看,那是个极魁梧的棕发年轻男子,他皱了皱眉头,因为白人大多是财神爷,部落里有规定不能擅自招惹的,他自然也不愿意多事,“这位朋友,这是我们多特人自己的事,就不劳你们费心了。”

棕发男子还想说什么,却被同行的金发男子阻止了。

可塔多拉自然不会放弃这么个救命的机会,马上扑到了那棕发白人的脚下,“大人救命啊~”

棕发男子本来已经被同伴劝住,不过这么一来实在是不好袖手旁观了,又问了库提一句,“他到底欠你多少钱?没必要弄这么大动静出来吧?”他的想法很正常:索度人之间的经济瓜葛,能有几个钱?

库提也没想到这个白人是如此的执着,一时也有些生气了,语气也不再客气,“白人朋友,我再告诉你一次,这是我们自己的事,与你没什么关系。”

面对着黑人,那白人本身就有很强的优越感,又喝了不少啤酒,见到库提不回答自己的问题,火气就更大了,“野蛮人就是野蛮人,小子,我问你问题呢,少跟我说什么谁的事之类的话。”

库提本来就是横行惯了的,面对这种侮辱的话,他也不再客气,“你,是想打架么?”

\section{第九十九章 凭你,也配?}

“打架?”那棕发男子上下打量一眼,对粗壮的库提很是不屑,“切,就凭你,也配跟我动手?”

库提虽然恨得咬牙,却是不敢主动出手,那种责任他承担不起,别说他,就是迪赞来了也承担不起,只能动动嘴皮了,“我说呢,不过是只胆小狗,也敢到处乱咬人?”

棕发男子果然不能忍受这种侮辱,猛地站了起来,“黑鬼,这是你自己找死,我要打掉你嘴里所有的牙齿。”说完,一记右直拳闪电般地击出。

库提早有准备,身子一侧就让了过去,同时拉开架势,同对方斗了起来。

酒店的地方不大,两人的打斗不可避免地秧及了别人,不过,大家都没什么怨言,纷纷起立让开,无聊的下午,有人愿意表演自然有人愿意看。

棕发白人显然是个拳击好手,步伐灵活,出拳凶狠;而库提却是仗着身体灵活来回地躲避,还击时也用腿多过用拳,很有点泰拳或者说跆拳的味道。

不过,在场的黑人没想到棕发白人那么厉害,仅仅几个回合就把出名勇猛的库提逼得左支右绌,脸上身上接连中招,嘴角都有血渗了出来。

正在这时,却听到有人大叫,“放开我!”大家回头一看,却是塔多拉被一个高个年轻黑人反扭了肩膀,在那里龇牙咧嘴。那白人和库提也同时跳开,停止了打斗。

当下就有一个中年黑人向那高个黑人打招呼,“迪赞,什么时候来的?”

来的正是那个给楚云飞他们带过路的迪赞,不过,他现在的身份比当时可强多了。他派库提来这里找人,自己在街边等候,却没想库提半天不出来,那塔多拉却偷偷地溜了出来,当然要出手拿下了。

棕发白人不再理会库提,上下打量迪赞一下,“你,把他放开。”

迪赞是听了消息进来的,早就看这人不顺眼了,不过,他也没有主动动手的心思,顺手松开了塔多拉,“这样可以了吧?”

塔多拉刚要跟棕发道谢,却觉得身后一股大力传了过来,整个人直接飞了出去,重重地撞到了墙上,木制的墙壁一阵摇晃,天花板上,沙土俱下,塔多拉当场就晕了过去。

收回右腿,迪赞伸手用手掌边缘掸掸身上的浮土,这个动作他是跟成树国学的,他觉得这样做非常潇洒,“这次看这位朋友的面上放过你了,明天我再找你要钱。”他才不管塔多拉听见没有呢。

棕发白人却是受不了这种刺激,迪赞简直视他如无物,怒吼一声,“小子,他的钱我替他出了,不过你要打倒我才拿得到。”他自然看得出眼前这人比库提要厉害一些。

迪赞的脑瓜连楚云飞都称赞,他自然不愿意出头招这个麻烦,不过,他的话却是非常冠冕堂皇的,“这是我们自己的事,不劳远来的白人朋友费心的,你的好意我心领了,不过为了这个垃圾不值得。”

那棕发白人听到对方如此回答,刚刚收回强行出手的心思,马上又被迪赞后面的话激怒了——“而且,我练的是中国功夫,客人万一有个闪失,那是我们不愿意见到的。”

金发男子刚喊了一声,“菲尔,你别……”那棕发的菲尔已经主动开始挑衅了,“中国功夫,骗小孩的玩意,纯粹是狗屎……”

他的话还没说完,酒店里已经是鸦雀无声了,所有人都在用一种奇怪的眼光看着他。

四个白人都感觉出气氛不对了,却不知道问题出在哪里。

半天之后,酒店一角的小门里传出了一个醉熏熏的声音,“迪赞,你这小子又套别人的话,再玩你那小聪明我把你脑袋拽下来当球踢。”

一个浑身酒气的黄种人从小门里晃晃悠悠地走了出来,不是成树国又是谁?

那棕发男子总算明白这气氛怎么会这么诡异了,原来这里有中国人在,“你是中国人?”

成树国斜眼瞟他一眼,没理他,掉头又继续骂迪赞,“你就不能做点正经事,成天歪门邪道不学好,跟这种垃圾有什么好说的?”

迪赞可是一脸的的恭敬,虽然他肚子里恐怕已经笑得前仰后合了,但表面功夫那是做得十足,“不知道大人你在这里,今天,今天酒店里的东西,损坏的算我的。”

成树国懒得多说,挥挥手示意他走人,“那只小虫子也带走,别影响人家蒙塞孔做生意,快滚。”

菲尔再也忍受不住了,“中国人,你是在骂我么?”

成树国知道自己被迪赞利用了,不过他并不是很在意,因为……因为他现在心情很不爽,“你算什么东西,也配我骂?”口气之嚣张,纯粹是不把对方当人看。

那菲尔受此侮辱,二话不说,冲上来就是一拳。

成树国还是那副醉眼惺忪的样子,上身一仰,躲过对方这拳,顺势就是一腿还了回去。

菲尔才躲开那一腿,却见对方的右掌已出现在眼前,“啪”的一声,一击重重的耳光。

这记耳光力道十足,菲尔被打得头晕眼花,知道不妙,马上向后一个纵身,可还没身子落地,肚子又挨了重重的一脚,当时就如翻江倒海似的吐了起来。

成树国皱皱眉头,一脚就把那菲尔踹了出去,“垃圾,要吐滚出去吐,别影响我朋友做生意!”

那金发男子再也坐不住了,示意一个女伴去照顾菲尔,他带着另一个女伴走了过来,“朋友,你有点欺人太甚吧?”

成树国斜眼瞟他一下,也不愿意多说,“呃,就算是吧,我就欺负他了,怎么样?你有意见?……还有,就凭你,也配做我的朋友?”

金发男子男子被他噎得什么话都说不下去了,动手吧,那更是白给,只能悻悻地转移话题,“没想到中国人这么霸道。”

成树国马上又陷于暴走的边缘,不再惜言如金,“中国人霸道?操,你也好意思说,你是哪个国家的?英国?法国?德国?美国?想想八国联军在中国做了多少霸道的事,你还好意思说中国人霸道?”

“再看看你那同伴,人家追人家的债,关他鸟事,以为自己是谁啊?还不是觉得自己是白人就高人一等?纯粹就是一个找死的主,还敢说中国功夫?我揍他这么轻已经是很给你们脸了,要不是不想给多特人惹事,我绝对弄死他。”

说到“弄死”的时候,成树国的语气间杀气腾腾,由不得别人不信。

\section{第一百章 选择新变化}

那金发男子还待再说些什么话,却被身边的女伴一拉。“算了,也该叫菲尔吃点苦头了,省得他到处惹事,咱们去看看他吧。”说罢,那长睫毛、大眼睛的高挑女士还冲成树国轻笑了一下。

成树国还真没想到能有博佳人一笑的机会,他正准备再次出手教训眼前这俩呢,白种人,哼,多特人不敢惹他们,自己可是不怕的。

那个白人的向导跟了过来,笑着问,“是选择先生么?”

成树国摇摇头,“我是变化。”说完又掉头进了小门。

酒吧靠墙的地方有六个黑人共坐一张桌子,一看就是一起的,他们看了半天打斗了,并没有说什么,不过入耳这奇怪的问答,一个瘦小的中年人禁不住说话了:“他们这是在说什么?暗号么?”

正巧六人旁边站着一个多特族的闲汉,很是骄傲地接过了那中年人的自言自语,“我们多特族的中国兄弟嘛,新、变化、选择,三个人非常强大。也代表了老天的意思,让我们多特人能选择新的变化,变得更加强大和富足,难道不是么?”

新——new——刘宁;变化——change——成树国;选择——choose——楚云飞,三人的姓氏,多特人赋予了他们新的含义,而且名头响亮,居然连常来这里的向导都知道。

“哦,”中年人很是惊讶,“你说象他这样厉害的还有两个?”

闲汉更骄傲了,感觉就象对方在夸他一样,“新和变化都很厉害,不过选择才真正的厉害。”

中年人更感兴趣了,“来,这位多特兄弟,坐下慢慢说,”扭头又招呼吧女,“给我这个兄弟来两瓶啤酒,我出钱。”

那闲汉没事来这里本来就是混酒喝的,要他自己买他只能喝得起罗伦酒,还得慢慢咂不敢大口喝。听说对方请喝啤酒,还是瓶装的,那自然是眉开眼笑,不过,他也没忘了加一句,“欧迪丝,我要冰的啤酒,这天气实在是热了点。”

等到啤酒端上来,闲汉就开始了白活,他也明白人家为什么请他喝酒,这事又不是头回做。

“要说这三个兄弟,就不能不说刚贝拉大人,大人去中国留过学,这三个兄弟就是他在那时候认识的。”

“后来,刚贝拉大人说要开发咱这沙漠的旅游资源,族里有不少人反对,不过我可是赞成的。”

“刚贝拉大人的想法虽然好,但说服别人也是件非常不容易的事,怎么说这也是沙漠,也是没人来的话那不是很糟糕么?所以……”

中年人看他说半天说不到点上,明白是什么原因,“你说重点吧……招待,再来一瓶啤酒,要冰的。”

闲汉嘿嘿一笑,不再罗嗦,“我后来听说旅游开发计划就是这三个兄弟设计的,中国人确实很聪明。”

“不只是聪明,刚才的迪赞你看到了吧,他是不想跟白人动手,选择兄弟说过,要挣白人的钱,就要尊重白人。要是迪赞真的动手,那个白人肯定打不过他,迪赞不过也就是和选择兄弟学了几天功夫而已。”

“半年前荷兰‘贝’牌石油公司的事你们都听说过吧?”

六人里一个年轻的薄嘴唇黑人接话了,“全索度都知道,那不是一帮反政府武装干的么?杀了六个荷兰人还有三十多个工人,最后美国维和部队把他们剿灭了。”

“错了不是?”闲汉得意起来,他已经为很多人解释过这个问题了,讲述起来非常地熟练,“美国人去的时候,那里只有六十多个匪徒的尸体,他们根本不知道是谁干的,就把功劳揽到自己头上了。”

“不可能吧?”中年人表示明显的怀疑,而且话问得不算外行,“美国人也会冒领别人的功劳么?那可是个民主国家。”

闲汉显然是反驳过很多人了,美美地啜了口冰凉的啤酒,清清嗓子,“那死的多特人里有一个是刚贝拉大人妻子的弟弟,刚贝拉大人当时就在那个部落,马上喊了‘选择新变化’三个兄弟过去。那里的多特人在外面围着,三个中国兄弟冲进去就杀,根本就没人来得及往外跑。”

这话给谁听了怕是都不会相信,薄嘴唇自然也不例外,“不会吧,三个人杀了六十多个全副武装的匪徒?那还算人么?那是神!”

闲汉见惯了这种表情,“自然也跑出来几个,被当地的部落干掉了,不过,大部分人可都是中国兄弟杀的,迪赞也去了,这话可是他说的。而且,选择兄弟那一仗还受了很重的伤,养了有一个月呢。”

六个人你看看我,我看看你,还是那薄嘴唇的年轻人说话了,“原来……原来中国功夫是可以挡子弹的?那他们拿棍子杀人么?”

那闲汉居然开始耻笑为他买单的人,“你们看起来没那么笨吧?谁不知道那些匪徒都是枪打死的?跟你们说,他们的枪法都很好,新兄弟能打中半英里外的酒瓶盖。”

半英里,那可是八百米呢,子弹能飞那么远就算不错的了,六个人互相看了看,终于确定,眼前这闲汉,也就是似是而非地胡吹而已,不过,似乎有些话听起来也是满有道理的。

几人正在这里闲扯,酒吧忽然又安静了下来,大家回头望去,又一个中国人出现在门口,来的是刘宁。

刘宁直接上前问那酒招待,“他在不在?”

刘宁刚问完,忽然觉得屋子里气氛有点不对,回头一眼就看到那六个人,里面有两人的气息引起了他的警觉,是训练有素的军人!

酒招待满脸的恭敬,“在,变化先生在那里喝酒呢,一个人。”

刘宁点点头,走进了小门,他也懒得想那两个军人什么来头,来旅游的达官贵人他也见识过不少,不过,在这里谁还敢捣乱不成?

酒店里的人现在才敢大声地出气,不过,又有人好奇地问了,“这是选择么?”

问话的还是那个薄嘴唇,不过那闲汉回答的时候就不敢乱攀兄弟了,“这是新先生。”

中年人若有所思地点点头。

旁边那四个白人眼睁睁地看着刘宁走进小门,那大眼睛的高挑女子嘴里念叨着,“新、变化、选择,好,很好。”

\section{第一百零一章 咱们去欧洲吧}

成树国正斜靠在椅子上灌啤酒,看到刘宁进来,指指身边的椅子,示意让他坐下。

刘宁也不客气,一屁股坐了下来,抢过成树国手里的大号酒杯“咕咚咕咚”就是一阵猛灌。

成树国盯着他看了半天,摇摇头,长叹一声,“家里来信了?”

刘宁皱着眉头点点头,抬手举杯又是一气猛灌。

成树国长叹一声,“操,不行的话,咱们还是做雇佣军去算了,好过在这里混吃等死。”

刘宁沉思半晌,神态逐渐恢复了正常,“这个事,我们还是跟云飞商量一下吧,答应刚贝拉的一年可还有一百多天呢。”

楚云飞在做什么呢?他正在房间里练功呢,上次一战,他又受到了些轻微的伤害,不过回来及时运气,发现内气又有了惊人的提高。

那一仗打得没有闲汉说的那么轻松,不过倒也算顺利,只是最后有个歹徒趁三人不注意,引爆了手雷,还好楚云飞及时发现,只有他受了点轻伤。

成树国和刘宁推门进来了,“云飞,我们决定了,不想在这里呆了,咱们去欧洲吧。”

楚云飞收功站了起来,“你俩今天怎么了,又想起这事来了?”

刘宁回答了这个问题,“老爷子今天来信了,说咱们的事情,遥遥无期。”

遥遥无期?楚云飞沉吟一下,“那好吧,一起去欧洲,正好我还要去趟沙特呢。”

“去沙特做什么?”成树国很奇怪,“去找杨华么?他现在早就该毕业回部队了。”

“马哈苏德在那里。”楚云飞终于向战友交待了自己的底牌,以前不说那是因为不想连累战友,现在大家都回不去了,也不差多这么一件事了吧?

马哈苏德?刘宁和成树国对视一眼,这是什么鸟人?

最终还是成树国这“民族主义者”反应了过来,“云飞,你别告诉我你那死去的老子就是那个人质吧?那家伙似乎也姓楚呢。”

“去你妈的,”楚云飞飞起一脚,成树国敏捷地避开,“你老子才是家伙呢,我老子……就是那个倒霉的工程师。”

刘宁也反应了过来是怎么回事,上来冲着楚云飞就是一巴掌,不过没打着,“操,你倒是挺日能,凭你一个鸟人就能干掉人家?”

三个人心里都跟明镜似的,这事不能说楚云飞不够意思,只能说他是“见外”了,成树国兜屁股就是一脚,“操,叫你丫再吃独食。”

楚云飞没有避让,虽然这脚不算轻。

有这样的兄弟,天下虽大,却又怕得谁来?

………………

刚贝拉是很舍不得三人走的,而且按日子来说也远未到楚云飞他们承诺的期限,所以他还是盛情地做了挽留。

几个人的感情已经相处得非常好了,再说日期没到什么的话那纯粹是见外,刚贝拉只是很关心地问了问:“楚,你不再等等你女朋友的信了?”

周琳琳已经一个多月没给楚云飞来信了,看着成树国美不滋滋地隔三差五地收到女朋友的信,楚云飞还真的是有些郁闷难耐。

不过,都说好要走了,楚云飞自然不能为这点小事影响大家的行程,那样做未免太不像个男人。

“谢谢你,刚贝拉,如果她再来信,你帮我收着吧,以后我会联系你的。虽然我们没有家,但是很高兴,我们有朋友!”说完,楚云飞重重地拍拍刚贝拉肥厚的肩膀。

刚贝拉知道再也不能阻止三人的离去了,而楚云飞真诚的话语又打动了他,犹豫半天才说,“你们不需要路费么?”

需要,怎么不需要?虽然三人平时也有能免费去的地方,就像成树国去的那个小酒店,不过,没工资的日子手头始终没有宽松过。他们只是想,多尼薪水那么高,该有点钱的吧?

不过,刚贝拉没跟他们三人要钱已经不错了,哪还能再拿刚贝拉的钱?楚云飞笑笑,“我们自己想办法吧,欠你的钱将来有了就给你。”

刚贝拉点点头,“我不是那个意思,我是说其实……我这里还有点小事,本来想弄明白以后再喊你们帮忙的。不过你们现在都要走了,那就帮我把这事办办吧,这次我出钱,而且,得了什么东西大家分。”

原来,前一阵有几个白人游客在沙漠游玩时同向导走散,失踪了。由于是“提坦卡布”部落的关系介绍的旅游团,刚贝拉就派出人去寻找,两天之后,渴得半死的游客被一个搜索队找回来了,而另一队搜索的人居然找到个被风沙掩埋的陵墓。

那里该是个古索度苏丹的陵墓,这没主东西自然是谁发现谁去发掘,所以刚贝拉去偷偷找专家来鉴定这个陵墓。

陵墓地表部分只露出不到两米,是个类似金字塔的锥体,下面被沙尘填充得非常瓷实,暂时还是进不去的,而盗墓这事实在又不方便张扬,所以多特人只能小心翼翼地派上几个可靠的人去发掘。

专家根据地表已经被发掘出的部分断定,这墓的主人应该是属于苏丹或者王子级别的,时间大约是在公元前五百年左右,而且根据地形地貌分析,那时陵墓四周应该还是绿色昂然,一片生机的沃野。

这个墓最难得的是:看上去还没有被人盗过它。也就是说里面应该很有些陪葬的东西的,虽说那个年代的东西大多数只对考古学家有用,但值钱的古董之类的肯定也会有一些的。

不过,多特人世代生活在这个地方,故老相传,这种古墓里面绝对会有一些很诡异的东西的,比如说“机关”、“诅咒”之类的东西。还有盗墓人遇到过被封印的魔鬼,那放出来的魔鬼杀光了周围七、八个部落的人。

刚贝拉是不相信这些无稽之谈的,否则他怎么敢动这个脑筋?魔鬼可能是瘟疫,诅咒可能是空气流通不畅或者少见的病毒,至于机关,那倒可能是真实存在的。

不过,他自小生活在部落里,周围的环境对他或多或少还是有点影响的,敬畏之心也不能说一点没有。

其他的多特人恐怕敬畏之心要比他多得多,再说这事也不便声张的,于是刚贝拉就想到了三个中国人,中国的古墓应该比这里多得多吧?

让刚贝拉印象深刻的还有楚云飞那种对未知危险的敏锐感觉,这是所有人都公认的。

他本想把那个地方挖得差不多才通知三个中国人,没想到人家说走就要走了,这盗墓的计划只好提前了,还好也不差这几天。

楚云飞三人也是曾经的共产党员,没准现在还是呢,觉悟未必有多高,可这神鬼的东西倒是不怎么害怕。

“那好吧,咱们什么时候开始?”

\section{第一百零二章 尘封的王者之地}

第三天一大早,刚贝拉带着十几个人、两辆卡车和一辆大吉普车,闯进了一望无际的索度沙漠。

索度沙漠流沙很少,汽车在其中是可以放心大胆地开的。楚云飞三人享受的自然是贵宾待遇,坐进了舒适性相对好些的吉普车。

开始出发的时候,天还没大亮,沙漠边缘的气候温差比较大,虽然大家都穿着宽大的长袖衣服,但行进中,坐在卡车马槽里的那些人还是冻得发抖。

不过,太阳一升起来就不一样了,四周的气温急剧地升高,马槽里的人们开始站起身来,好让凉风多吹吹躁热的身体,吉普车的车窗户也摇了下来,反正到处都是飞舞的沙子,关着窗户也起不到多大的作用。

大概走了有三个多小时目的地就到了,大概走了有一百七、八十公里的样子。

陵墓的两侧已经被深挖了将近十米,菱形的底座已经露了出来,其中一侧更是多挖了三米多,挖到了坚硬的石板。刚贝拉一到现场,就是胖手一挥,“给我炸。”

火药和雷管早就埋好了,五分钟后,一声惊天动地的闷响回荡在空旷的索度沙漠上空。

二十公里外,一声欢呼响起,“那里,那里有人,我们有救了!”三个迷路的可怜虫拥抱在一起。

三人一看就知道已经迷路很长时间了,衣衫褴褛,手脸等暴露在衣服外面的部位被风干得皱皱巴巴,细看还有无数个小小的裂口,被晒脱水分的皮肤像绒毛般地翘起,布满裸露的部位,嘴唇上的裂口就更大了,不但长而且深,让人感觉他们只要说说话就会不停滴血。

其中一个个子矮小的该是个混血儿,肤色微黑,穿着一沙漠迷彩服,精神还好点,起码从眼神中还能看出一丝丝的坚忍,另两个人就差得太多了,那中等身材的黄种人还略微好点,起码两条腿还站得起来;状态最差就是最高那个白人,连眼神都是非常恍惚的,要依靠两个同伴扶持才能勉强行走。

三人是英国普雷顿大学的校友,混血儿施雷顿是德国人,黄种人安子豪是中国台湾人,白种人布兰克是英国人。

他们三人是普雷顿大学著名的驴友,酷爱旅游和探险,家里条件也都还不错。这次他们来索度沙漠只是做个初步的历练,为下一步穿越整个撒哈拉沙漠做次预演。

本来布兰克的意思要找几个当地人做向导的,就算旅游组团也罢,可施雷顿和安子豪都是信心满满的:反正又没打算穿越索度沙漠,只是去里面适应几天而已,用不着那么紧张。

既然准备不足,吃点苦头那就难免了,当三人遇到最可怕的地下磁场迷失方向的时候,只能靠坚韧的毅力来支撑了。现在的三人已经把一切该扔掉的东西全部扔掉了,还好,终于在即将支持不住的时候发现了人迹。

希望就在眼前,三人不顾一切地踉跄前行,不过安子豪也即将支持不住了,在攀爬一座沙丘的时候终于双膝一软,跪了下去。

施雷顿措手不及一下没拉住,布兰克可就遭罪了,翻着跟头就滚了回去,跌得头晕眼花。

“狗屎!杰克,你要再这样,我会杀了你的!”布兰克的精神本来就已经到了崩溃的边缘,要不是马上就有获救的希望,现在他应该已经疯了。

杰克是安子豪的英文名字,他本来就已经快累垮了,听到这话再也忍不住了,“布兰克,你个垃圾,要不是一开始就把所有重东西都让我背,我能累成这个样子么?”

黄种人在欧洲地位不高,而安子豪又是台湾这个尴尬地区来的,所以分外地受人歧视,论起来,在很多势利的欧洲人眼里,有黑人血统的施雷德地位都要比他高些。

不过三人关系尚算良好,这次也只是大家火气都有点大而已,施雷德跑下沙丘,拽起布兰克,“好了,你俩都别说了,减少水分挥发。”

陵墓这边,经过了一个小时的清理,炸出的碎石已被打扫干净,露出了黝黑的洞口。

楚云飞跟着一个胆大的多特人跳进了墓道,身后是成树国和刘宁还有迪赞,为安全起见,三人都携带了手枪。刚贝拉身娇肉贵在上面坐镇。

那多特人点燃了随身携带的油灯,有玻璃罩的那种,主要是用来测试空气中氧气的含量,省得大家到时候死都不知道是怎么死的。

其他几人都随身携带了大号手电筒,本来依楚云飞的意思还要搞几个防毒面具来的,不过很遗憾,那东西实在是不知道哪里才有得卖。

墓道里很干净,空气也不算污浊,两千多年以后,终于有人踏上了这片尘封的王者之地。

大家走得都很小心,生怕有什么传说中的机关突然出现,不过将近六十米的墓道一直走完都没发生什么事情。

墓道尽头,是一扇木门,木门上左右对称有两只兽头,不过看不出那是什么动物,也许是古索度人眼中神灵的形象?

如果有什么机关,这里绝对是最佳的布放地点,大家也都停下了脚步。

楚云飞定定神,尝试着感受了一下,嗯,是有点不对劲的感觉,不过倒没有很让人不安的那种悸动,隐约中,他还感受到里面有什么让自己感到兴奋的东西。

“大家小心,那个门似乎有点不对劲。”楚云飞说话了。

听到楚云飞这么说,其他人都紧张了起来,不过,这墓道直来直去的,想找个躲避处都很难,众人只好伏在地上以防万一。

楚云飞气运全身,慢慢地走了过去,行进门口,突然右脚尖一软:不好,陷阱!

说时迟,那时快,楚云飞在一瞬间整个身子凌空后飞!

那速度,在大号电筒的强光下,让众人看得目瞪口呆,感触最深的是那个胆最大的多特人:选择先生……果然不是一般人。

等到楚云飞身子落地,他才发现,自己居然一下跳了将近十米远,这个……真让人意外。

“扑通”一声传来,门前出现个将近两米宽的大坑,地面也为之轻轻一颤。

等了半天,没什么动静,大家都直起身子来,走近陷阱观察,那陷阱深不见底,直抵两边墓壁,宽有两米,再向前一米就是木门了。

虽说不是很宽,能跳得过去,可现在,谁敢随便往对面跳?

大家正犹豫间,忽然迪赞怪叫一声,“门!门!!你们看那门!!!”那刺耳的声音,让所有人在一瞬间毛骨悚然。

\section{第一百零三章 索度兵马俑}

为了避开传说中那种巨型机关,刚贝拉坐在离陵墓起码有两百米开外的地方,汽车之间用布在顶部连接起来,就形成了一个简易的帐篷。太阳已经不在当午,但躁热不但没有减少,四周空气反而还在持续升温。

刚贝拉把手里的矿泉水瓶子一扬,“奇卡巴,轮到你了。”

定期打探,与下去陵墓的人保持联络是十分必要的。

奇卡巴懒扬扬地从地上站起来,四下张望,“手电呢?刚才谁去的?我要手电。”

正在这时,一个多特人匆匆地走了过来,“大人,东面,东面似乎有人来了。”

“哦?”刚贝拉把手里的瓶子向地下一顿,眉头皱了起来,“看清楚了么?有多少人?几辆汽车?”

“能看见的就几个小黑点,没车也没骆驼,不过,后面还有没有就不知道了。”

“嗯,”刚贝拉点点头,“现在这么热,还在用腿赶路的,那就没几个人,都鲁带几个人去,只要不是咱多特人就干掉他们。”

盗墓毕竟是为法律所不允许的,别人看见是很不妥当的,反正这么大的沙漠,失踪几个人是再不能小的事了。

半小时后,高大魁梧的都鲁汗流浃背地回来了,“报告大人,带回来三个人,都快昏迷了。”

刚贝拉有点烦躁了,手一挥,“先救救他们,妈的,别跟他们说咱们是谁,让他们歇歇,问问是哪个部落的,没事跑这么远做什么,操!”最怕的就是遇上多特人,还真是怕什么来什么。

都鲁知道刚贝拉误会自己救回来的是多特人,到嘴边的水杯又赶紧拿下来,“大人,他们……他们不是多特人,是三个白人。”

“白人?”刚贝拉更恼火了,这都添的什么乱?“白人就怎么了?干掉他们,白人最坏了,现在我们又不是旅游公司,我们是盗墓贼!”

都鲁知道刚贝拉的意思,不过,他也有忌惮的事,抹抹头上的汗,“大人,里面有个黄种人啊,中国人的那种黄人。”

刚贝拉一楞,马上明白了,感情是碍着这个人,大家不敢下手,毕竟自己是从中国回来的,而且……更重要的是,这里还有三个很厉害的中国人在,谁知道那几个会不会因为这个事生气?

明白归明白,刚贝拉烦得脸上快能点着火了,“好了好了,我知道你们的意思了,把人看起来,奇卡巴去问问。”

奇卡巴知道主人不想出头露面,刚贝拉那个身材实在是太好辨认了,他顾不上刚从陵墓回来的疲劳,站起身走了出去。

另一辆汽车的一角,三个脱险者正在大口地灌水,旁边负责看护的多特人好心地提醒他们,“吃点盐吧,光喝水会拉肚子的。”

奇卡巴走了过去,没理那俩,直接面对安子豪,“中国人?”

安子豪楞了一下,他一贯倾向于把自己称做“台湾人”,不过对方这么问,他显然是不能否认的,于是点点头,“是,中国人。”

倒是布兰克不知道出于什么心态,在旁边补充了一句,“他是台湾人。”

奇卡巴是不明白二者有什么关联的,瞟了布兰克一眼,皱了皱眉头,转身回去向刚贝拉汇报。

“他说自己是中国人,不过他另一个同伴说他是台湾人,不知道是什么原因。”

这其中分别刚贝拉可明白得很,事实上他的那些大学同学总是把这事挂在嘴边的,他点点头,“哦,台湾人……那也是中国人,这事还是等他们回来再说吧,看好这几个人。”

………………

陵墓里,众人眼睁睁地看着那木门像沙子一样,缓缓地“流淌”了下来。

再结实的木头,也经不起岁月的侵蚀,那木门受到陷阱陷落的震动,再也不能维持原来的样子,化做一堆粉末。

粉末尚未完全落下,原来门前一米处“夺、夺、夺”一阵轻响,自上而下落下了十几根长矛,矛柄一阵晃动后,也化为了飞灰,纷纷落下。

众人看得目瞪口呆,面面相觑,果然……果然还是有机关的,这长矛,该是本来连着门的吧。

接下来就是准备破除剩余机关了,众人退至墓道口,取下木板、沙包、铁锨、飞抓等准备好的工具,再次前行。

世事总是那么无常,也许是古人设计机关的能力有限,也许是年代过于久远机关失效,众人把大大小小一应物件辛苦搬下来以后,居然再没有任何的机关出现。

木门后面是个宽阔的大厅,大厅呈正六边行结构,每边长约有十米,各有一扇门,其中有一个门面积是其他门的两倍还有余。应该是通往墓穴主人方向的。但是门板已经不见了。

其他五扇门上,只有一扇还残留着木板,其他门已经连渣都看不到了。

大厅正中央平行排列着六只棺木,棺盖凌乱地被丢在地上,尸骨散落得到处都是,居然是被盗过的陵墓!!!

迪赞掉头就走,他要赶紧把情况报告给刚贝拉。

大厅里到处是破损的陶器碎片,还有翻倒的陶人陶马之类的各种人兽的造型,凌乱不堪,成树国对这些倒是很有点兴趣,“我还只以为咱们中国才有兵马俑呢,原来……索度也有啊。”

一片的寂静中突然出现人声,那是很吓人的,所有人都被这声音不怎么高的感慨吓了一大跳。

楚云飞也被吓了一下,不过他随之马上皱起了眉头,因为他发现一种非常奇怪的感觉袭来,那种味道,很难说得清楚,似乎……似乎有点刚到索度刘宁病重时的那种感觉。

他看看众人,似乎大家都没什么特别的表情,有的只有那种余悸初定的莫名的轻松,他们都没感觉到么?

蓦然间,一股透骨的寒意从那最大的门中缓缓弥漫了出来,在场的人都感到了若有若无的一阵阵凉意,汗毛莫名其妙地耸立了起来。

只有楚云飞有了一种似曾相识的感觉,同时相伴而来的还有一丝略微的惊喜。

那……那似乎也是一种生命的能量。

\section{第一百零四章 看谁吸干谁}

有了这种感觉,楚云飞马上兴奋起来,把本已经相当高的内气运转提升到了“伪先天”的境界,会有什么样的事情出现呢?

刚贝拉带了四个人下来,他四下看看地面,失望之情溢于言表,“怎么会有人盗过这里?怎么会这样?”

楚云飞早已经不再是当年的他了,这点无关大雅的声响干扰不了他的行功运气。

大家商量了几句,就打算来个地毯式的搜索,既然有人盗过了,该是再没什么机关了吧?

“等等,”楚云飞不得不张嘴阻止大家,“这里……有些古怪,你们再等等。”

楚云飞这几句可吓得大家不轻,谁都知道他是个出言谨慎的人,现在也绝对不是开玩笑的好时候,众人你看看我,我看看你,都从其他人眼中看到了惊骇和恐惧。

就这么等待了几分钟后,那胆子最大的多特人说话了,“咦,我怎么觉得这里越来越冷了呢?”

刘宁也点点头,“是,温度是下降了。”

然后就是死一般的寂静,大家都不再言语了,是诅咒?还是魔鬼要出来了?

众人的眼光齐齐地望向楚云飞,却发现他不知道什么时候盘腿坐下了。

楚云飞又看到了那生命的影子,比上次容易多了,也许是又经历了一次生死考验的缘故?

还是明黄色的光斑,不过流转间似乎没有那么轻松,笨拙了一些,也沉重了一些。或许是大号手电筒影响的缘故,也黯淡了一些。

先是那星星点点的零散光斑从正门中涌出,不过那些光斑很快就越聚越多,在众人身上穿越却并不停留。

不但不做停留,在那光斑穿越过人身之后,光芒又大了许多,似乎……似乎在抢夺众人身上原有的生命能量。

难道……难道这些光斑是有自己的意识的么?还是下意识的一种行为?

一种压抑感油然而生,不好,门里要出大家伙了么?没人告诉楚云飞,但他感觉到了。

楚云飞的手指向后面一个方向,然后用力地摆了一下,他连开口说话都不敢。

大家本来就一直在盯着他看,刘宁马上就领会了,“我们,向那里退退。”

众人移动的脚步十分沉重,时间似乎在这里出现了扭曲,没人能走得快些。

正门中,在众多光斑的簇拥下,一个庞大的光团轰然出现了,足足有一辆卡车那么大,形状像……没什么可象的,就是一团粘稠的液体一样,不断地变幻着自己的形状,但却是悬浮的。

光团明显是有一些意识的,停顿了一下,向楚云飞飘了过来,在它之前,那些小光斑可是没有敢向楚云飞附近飘去的。

光团面对面地直接撞上了楚云飞的身体,一股庞大的信息铺天盖地般地向他脑中涌入。

刘宁他们看到的却是楚云飞的身体猛然地颤抖了一下,然后寒意似乎就开始逐渐地消退了。

那是怎样的意识啊?楚云飞的脑中被冲击得乱做一团,各种各样的情感交织在一起:恐惧、愤怒、不甘、绝望、暴戾、挣扎……

所有的意识,都是负面的!

这究竟是什么东西?楚云飞真的有点迷惑了,但是,他马上不得不面对一个事实:自己体内的生命能量似乎成为了遇到磁铁的铁屑,在极力挣扎地向体外涌去,去同那光团融合!

楚云飞大吃一惊:这样下去还了得?不被这个莫名其妙的光团榨干自己的生命力才怪,这才是真正的魔鬼!

拼了!楚云飞想起救治刘宁的那个晚上,虽然不知道现在自己该怎么办,但他确定首先自己要有强烈的求生欲望才行,你要吸干我么?我也想吸干你呢,你……快些进来吧,进入我的身体吧!!!

那光团明显地领会到了楚云飞的意图,但它却没有因此离开,而是执意地吸取着楚云飞的生命能量。

楚云飞虽然心性比常人要坚定些,但在这种情况下还是不由自主地产生出了惧意,随着生命能量的流失,这种惧意也越来越大。

恐惧归恐惧,但楚云飞还是咬着牙坚持,因为他知道,现在根本没有逃跑的可能。

随着生命能量一点点的流失,楚云飞的怨念也在急剧地增长着:我绝不能,绝不能就这么放弃,我要吸干你!!!

终于在某个时刻,“轰”的一声,楚云飞脑中似乎炸开了什么东西,生命能量停止了流失!

接下来,就是那些生命能量欢快游了回来,不仅仅是他原本就有的那些,那光团里原有的生命能量也在由少到多、由慢到快地向楚云飞身体里涌了过来。

那光团不知道意识到了这个问题没有,反正依旧同楚云飞紧紧接触着,还在徒劳地释放着自己的吸力。

这样不知道过了多久,光团体积小了有十分之一左右,楚云飞也感觉自己的身体似乎已经装不下那么多了能量了,可他稍有犹豫,身体内的能量就又隐隐有向光团回流的趋势。

完蛋,这回楚云飞就像是遇到了美餐的老饕,虽然可以尽情的取用了,但,食物过于丰盛又不得不吃的话,那是会被撑坏的!

肚量不是很大的楚云飞只能咬着牙,继续往身体装这种莫名其妙的东西。

当楚云飞感觉自己已经吸得不能再吸,否则恐怕会活活涨死的时候,异变再生!

楚云飞感觉自己的身体骤然间减轻,内气也轰然间迅疾流转起来,那速度和往日相比,纯粹是乌龟和兔子的差别,而且都不能用强劲来形容了,纯粹就是狂暴!又一个瓶颈被突破了!

同时身体对能量的容量似乎也在瞬间扩大了不少,因为那光团在急剧地萎缩中!

当光团萎缩到原来三分之一大小的时候,一种喜悦的信息从光团那里传递了过来,楚云飞还在纳闷是怎么回事的时候,光团轰然炸开,化作星星点点,同原来的光斑一起,向陵墓外飞去,还有偶尔零星的光斑被身后的众人所吸收。

楚云飞坐在那里没有动,他要体会、品味这种感觉,不过他还是张嘴了,“好了,没事了,你们去看墓里还有什么东西吧。”

众人听到这话,向楚云飞拥来,想问个究竟,刘宁又突然发现了异常,“云飞、你怎么身上全是血啊?”

\section{第一百零五章 台湾是中国的}

大家听到这话,仔细一看,可不是,楚云飞身体裸露的地方顺着毛孔渗出了一层细细的褐色的液体,应该就是血吧?

不过就是在地上坐了坐而已,还没有二十分钟呢,至于这么夸张么?每个人的脑中都是这个念头。

楚云飞坐着没动,龇牙一笑,“没什么,吃多了点。”然后闭上眼睛开始打坐。

吃多了点?大家又相互看了看?吃了什么?他吃了什么?

马上就有多特人联想到了族中古老的传说,这选择先生……在墓里吃空气?别是吃了魔鬼吧?那他不就是上天下来的神灵么?有腿软的一两个马上就要跪下了。

刚贝拉可没管这么多,“好了,大家去看看,有什么好东西,楚说了没事,我想也是没事了。”

尽管有楚云飞的保证,但大家还是存有忌惮之心的,谁知道这里还会有什么神秘的东西?事实上,所有人都不知道刚才发生了什么事情,只是隐约地觉得似乎选择先生为大家解除了个大麻烦,虽然他看上去什么事也没做。

所以盗墓的工作还是在大家的小心谨慎中完成的。

那扇正门的里面就是幕主人停棺的场所,那主人的尸骨现在散得满地都是,棺中物品也不知去向了。

不,去向还是有的,就在离正门不远的墙边,三具尸骨缩做一团,恐怕……这就是那盗墓的吧?

那三具尸骨旁还有三个偌大的皮质口袋,口袋里鼓鼓囊囊,该是有些什么好东西在里面的。

不过这一切都和中国人无关,刘宁和成树国正守在楚云飞的身边,为自己的战友提供可靠的保护。

两小时后,墓里有价值的东西被多特人统统运了出去,不过,他们对那些陶制品的兴趣似乎不是很大,后来刚贝拉才说出了他的想法:他想把这里也开发一下,作为旅游项目的一个保留节目:保留完好的古苏丹墓,多好的卖点!所以有些东西该不该就这么拿走他还需要斟酌,就算想拿走,这次也拉不完。

楚云飞倒没把自己身上渗出的东西当回事,以他的经验就是,每当一个瓶颈被突破的时候,身体内总要有些淘汰下来的糟粕需要被排出,其中,有大约三分之一部分是通过体表排出的,这次也不过就是排得剧烈了点而已。

所以,当刚贝拉决定“今天就到这里了”的时候,楚云飞站了起来,没有任何的不适,反而神清气定,全身充满了喷薄欲出的精力,说不出地受用。

走上地面时已经下午五点多了,刚贝拉忙着处理那些到手的东西,顺手给楚云飞指点一下,“喏,那里来了三个游客,其中有个中国人,你们去看看怎么处理吧。”

楚云飞三人都惊讶了一下,中国游客,这里会有中国游客?

不过三人马上就反应了过来,这游客里面要是没中国人,刚贝拉估计早就解掉决了,现在留给他们,那绝对是很给面子的行为。

楚云飞先揉揉脸,然后又拳打脚踢一番,身上那一层已经干结的嘎巴就褪了下来,好不容易见见同胞,不能太不成样子吧?

刘宁带着成树国已经走了过去,依旧是没找别人,直接找上了安子豪,用英语问道,“你是中国人?”

布兰克在旁边就有点郁闷了,怎么来的人都找的是这个家伙,自己好歹也是大英帝国的子民呢,再看看问话的人,切,不过也是个黄种人而已。

在这里见到黄种人,安子豪自然很高兴,马上点头承认,“是的,我是中国人。”

布兰克又不厌其烦地加了句,“他是台湾人。”

多嘴最终是要付出代价的,成树国从刘宁身后走上前,对着布兰克脸上就是一脚,“我操,我们问你了?”

刘宁也鄙视地看了他一眼,“垃圾,台湾就是中国的。”

接下来就是几个中国人之间的汉语交流了。

安子豪马上就可以断定眼前这几个黄种人是“大陆仔”了,楚云飞他们三个人的“北平普通话”也证实了这点。

大陆和台湾的留学生在海外相遇的话,总是各有各的优越感。台湾留学生一般家境都不错,经济上一般是瞧不起自己的同胞的;而大陆留学生因为背靠强大的祖国,面对任何国家的任何人种都是不卑不亢,所以也瞧不起台湾的同胞那种见人就矮几分的受气样。

安子豪很为能在这里见到自己的同胞而兴奋,与其相比,平时那点点若有若无的芥蒂根本就不值得一提了,总算,总算是有了可以仗恃的依靠了!

三个迷路的探险家虽然年轻,但行得路多,见识也上去了,谁也不笨。当逃离险境的惊喜过后,疑云自然会涌上心头:这些人是什么人?在沙漠深处神神秘秘地做什么?

知道得越多危险就越大,所以当探险家们意识到这个问题的时候,就试探着提出离开的要求,不过被看守以近乎粗暴的方式挽留了下来,还好那两个多特人看在“选择新变化”的面上,没给安子豪什么难堪。

安子毫本来就为自己受到的优待而纳闷,现在终于明白了:这几个大陆人似乎在黑人中很有威信,那自己的小命怕是可以保住了吧?

于是,素来同大陆同胞不怎么往来的安子毫马上就变得万分亲热,乖巧识做是一方面,要没有这三个中国人,自己怕是已经被这些黑人干掉了吧?

楚云飞他们也为这里能来个台湾同胞而奇怪,于是详细地询问了起来。

聊了很久,成树国又想起了那个试图分裂中国的英国人,于是又用英语问道,“你这个英国同伴似乎在地理方面是个白痴?”

刘宁和楚云飞自然是知道某个“民族主义者”还在为一句话耿耿于怀。

布兰克哼哼唧唧地想反驳什么,安子豪素知他的秉性,狠狠地瞪了他一眼。

虽然知道自己这样做是为了两个同伴的安全着想,不过,能如此凶神恶煞地瞪上布兰克一眼,安子豪还是禁不住有了种“扬眉吐气”的感觉。

其实,做个中国人真的很不错,安子豪突然有了一丝丝的感慨,这种睥睨捭阖的感觉,以前只认同自己是台湾人的时候从未有过。

事实上他也明白,在外人眼中,再怎么声明自己是台湾人都没用,全世界人都知道,台湾是中国的一部分。过分强调自己是台湾人,不过是标榜自己富有、不是土包子而已,反倒会经常在身份上不自觉地低人一头。

早就不该这么低三下四了!安子豪终于感受到了势力强大的好处,这个……似乎比名牌手表、汽车什么的更受有男人味一些。

\section{第一百零六章 同胞必须保}

安子豪确实给楚云飞他们带来了烦恼。

在初遇国人的惊喜慢慢平静后,三个中国士兵不约而同地想到了一个问题:眼下这三个游客该怎么处理?

三人配合已经很久了,彼此间递个眼色大家就心领神会了,于是楚云飞在聊了几句后和安子毫招呼了一下,“小安,你们先在这里呆会儿,我们还有点事要办。”

安子豪马上又紧张了起来,“大佬,这个,你们一走,我害怕啊。”

楚云飞被这一声“大佬”叫软了心肠,笑了笑,“放心好了,都是中国人,有我们在,谁还敢害你不成?”

这话说出口,楚云飞已经下定了决心,安子豪他是一定要保的,毕竟……都是炎黄子孙呐。

走到远处,刘宁开始抱怨了,“云飞,你要保他们?”

“嗯,”楚云飞点点头,“一家人,不保说不过去吧?我觉得他挺可怜的。”

刘宁注视楚云飞半天,“我觉得,保不保吧,你没听那个英国人说他是台湾人么?”

成树国一直没吭声,因为他觉得刘宁的思想倾向有点问题,那就该听听理由的,不过这个理由成树国显然不能赞同,“那是英国人说的,又不是他说的。”

刘宁斜他一眼,很是没好气,“操,你怎么变成猪头了?那个小安平时要不这么强调,那傻逼英国人知道个屁的区别。这种数典忘祖的主,保他做什么?”

刘宁这么一说,成树国实在是没办法反驳的,事实就摆在眼前,仔细想想,确实就是那么回事。

楚云飞也知道刘宁说得没错,不过他也不是个喜欢改变主意的主,“还是保吧,同胞嘛,有能力帮就帮一把好了,你没觉得刚才咱们走的时候他的样子很可怜?”

成树国听到楚云飞这话,马上又附和起来,他这“民族主义者”确实是戴有色眼镜看人的,“对呀,咱们现在帮他一把,没准能拉住个支持统一的台胞,起码他以后也不可能跳着脚说自己不是中国人吧?要考虑大局的。”

楚云飞动之以情,成树国晓之以理,刘宁也知道自己的主张被否决了,不过他的脾气真不算好,“那你俩找刚贝拉说去吧,这种人,我想想就生气!”

于是楚云飞和成树国去找刚贝拉求情去了,大不了分成的时候少拿点就是了。

刚贝拉很痛快地答应了楚云飞的要求,“既然你都说了,我还能不答应?那俩也放了算了,我很够兄弟吧?”

楚云飞笑着点点头,“这话是在中国学的吧?”

刚贝拉笑着摇摇头,“是啊,就在那时候学的,对了,你管住他们的嘴巴,不要让他们乱说,这里我有大用处的。”

楚云飞拍拍刚贝拉的肩膀,“这还用你说?我办事你放心好了。”

刚贝拉看看四周没人,“对了,刚才你在下面是怎么回事?”

成树国一听这话也凑了上来,谁不想听这答案?顺便还招招手示意刘宁过来。

楚云飞皱着眉头思考半天,“这个……我也说不太好啊,只是猜测。”

“我估计……这个墓里该是殉葬了不少人。”

刚贝拉摇摇头,“没错,这种墓肯定有殉葬的,应该就在主墓室的下面,回头我还打算挖出来呢,很不错的旅游卖点。”

楚云飞慢慢点点头,“那就是了,这里不知道发生了什么问题,殉葬的那些人的生命死了,能量却没有消失,还融合在了一起,变成了一种很奇怪的东西,那东西能把普通人的生命吸干……”

刚贝拉听得连打几个哆嗦,“好了,好了,我不问了,我只想知道,以后这种情况还会不会发生?”这该死的楚,难道不知道天快黑了吗?

不过事关以后的旅游计划,该问的还是要顺口问问的。

楚云飞笑了笑,很有些得意的味道,“哈,别怕,我把它吃了。”

刚贝拉掉头就走,话都没一句,吃了???老天!!!

成树国挺相信楚云飞的,倒不是盲从,实在是怪异太多已经习惯了,“云飞,你说……我俩把你那个气功练下去,能不能也变得像你一样厉害?”

刘宁也是一脸的羡慕,“是啊,别的我也不要,能跳你那么远就行,去奥运会上拿两块金牌,也算为国争光了,呃,为国……争光”

听着刘宁声音的声音越来越小,显然是想到了痛处,楚云飞赶紧岔开话题,“你们好好练,我想应该有希望的吧,对了,咱们去看看我们小老乡去吧。”

警告三个探险家,这事楚云飞做起来是轻车熟路,他实在是有做奸商的天才的。

楚云飞先是一脸怒气地走了过去,正告那三位:此间主人已经确定你们三个是商业间谍了,要偷偷处决你们,安子豪我保了,其他人……那实在帮不上忙了。

什么?刺探什么商业机密?你倒是够敬业的,死到临头还想打听?

安子豪自然是感激涕零的,自己的生命首先是保住了,可见天下间还是同胞最亲。

不过,同伴的命也是命呀,安子豪马上又为同伴求情,平素间就算有点小小的争执,可也不能就这么眼见着两人就这么挂掉。

楚云飞自然是不会答应的,他很“为难”地告诉安子豪,为你求情,已经是我最大的能量了,保他们?实在看不出有什么必要来保他们。

成树国马上过来“劝解”:这安子豪怎么也算炎黄子孙不是?华夏一脉,同胞的面子怎么能不卖?再说,三人出来只有一人回去,同胞怕是不太好向别人交待,还是让他们写份材料,坐实了个人身份就好了,实在不行再写份认罪书。将来要一旦事发,天下之大,怕是也够他们奔波的了。

楚云飞看了成树国,没再说什么,只是上前拍了拍布兰克的肩膀,掉头就走。

布兰克正为这莫名的“示好”荣幸万分,却猛然间发现,自己身上所有的衣服在下一刻化做了飞灰,跟晚间沙漠的“寒风姐姐”约会去了。

此情此景,就由不得布兰克和施雷顿了,为了保命,他们只好乖乖地写下个人资料和那子虚乌有的“认罪书”。

\section{第一百零七章 有客上门}

楚云飞回到住所,马上拿出了那些准备带回国的“神奇咸水”,今天的事实在对他有了很大的启发,这水里是不是也蕴涵着什么生命能量,才能叫自己提高得那么快呢?

可琢磨了将近一个小时,他也没发现那水里有什么生命能量的迹象,很普通很普通的样子,难道是自己猜错了?

放下那水,他刚要收功,却发现自己所在的空间里,居然若有若无地有些小小的光斑,是游离状态的生命能量!

难道……生命的能量可以长时间单独存在的么?

为了搞清楚这个问题,楚云飞慢慢走出屋子,身后是刘宁和成树国不解的眼光。

果然是这样,感知范围一扩大,游离的生命能量也多了起来,同时还有弱不可辨的能量从熙熙攘攘的人群中不停地被吸收和排出。

楚云飞从没这么清晰地感受到过生命能量,好奇心一起,细细地感受起了院子外面的人。

真如他想的那样,年轻人身上固有的生命能量远远大于年幼或者年老的,那光芒要强烈一些。

而小孩和老人的固有生命能量要弱很多,不过,楚云飞同时有了种感觉,他们身上生命能量的交换频率要强于那些壮年人。

小孩应该是吸收游离的生命能量远大于释放的,老人正好相反,他们对生命能量的吸收能力接近为零。

这就是生命起灭的规则了吧?楚云飞暗自感叹。以前没能力感受的时候,他还没有太多的感慨,今天随着瓶颈的突破,他有能力观察了,却又陷入了更深一层的思考。

——不知道我死的时候,这么强大的生命能量,会不会也变异成那种巨型可吞噬其他人生命的光团?那时的光团会不会也是全部的负面情绪呢?

感慨归感慨,把握现实才是最重要的,楚云飞暂时停止了思考,回屋和战友商量怎么同多尼沟通的问题去了。

多尼早就联系上了成树国和刘宁,连楚云飞总共只有三个人,对多尼而言,未免觉得人手有点不足。但仅仅这三个人的力量就可以把塔尔人的工地搅得血雨腥风,多尼也不得不纳闷这三个人的能量。

多尼在考虑,可笑的是,这三个人根本就不给他考虑的机会直接就拒绝了,至于他到底是做过什么,面临的对手是谁,三人都没有去打探的兴趣。

等到三个中国人携手歼灭了六十多个反政府的武装匪徒后,多尼心底那团火终于被点燃,重新审视三人,并积极怂恿三人陪他回欧洲,可三人依旧没有理睬他。

不过,由于在索度没什么事可做,所以欧洲人和中国人彼此间聊些奇闻逸事也就难免了,楚云飞他们终于知道了多尼面临的是什么样的仇家。

多尼是波兰一个著名家族脱特斯基家族的成员,这个家族是个庞大的家族,虽然历史不够悠久,不过近代还是出了几个杰出的人物,在东欧一带很有些名声,他们的产业甚至延伸到了部分西欧国家。

多尼一家就是家族里为了开拓法国市场移民过去的,虽然不属于决策层,但却是族长近支,手里掌握着财权的,而多尼本人也和这一代的家族继承人路瑞关系密切。

意外总是在不经意间发生的,在某一天族长意外地猝死,路瑞也丧命于莫名其妙的车祸,其时多尼正在和未婚妻丽迪娜畅游罗马。

家族中另一大支推选出了托尼来继任族长,由于事发突然,路瑞这支并没有做任何的准备,于是托尼如愿以偿地掌控了家族的长老会,接管了家族权力。

然后就是丽迪娜被绑匪绑架,托尼要多尼交出手上的帐务和几个私密帐号,多尼忙于路瑞的葬礼没及时反应,一天后,丽迪娜被凌辱至死。

随后就是有杀手追杀多尼,幸亏多尼平时并不是温文尔雅的君子,他马上托庇于几个合得来的黑道朋友,但那些朋友庇护他几天后,通知他火速离开,因为托尼背后的势力太大了,大家抵挡不住。

多尼很明白自己朋友的能量的,既然他们都抵挡不住,那他就只有亡命天涯的份,而且前些时候一系列变故的原因也都就毋庸质疑了:都是托尼、甚至他们那一支搞的鬼。

至于托尼家引为奥援的外来势力究竟是属于什么性质,多尼没问,他的朋友也没说,不过,类似的势力在欧洲还是有不少的。

知道归知道,但三个中国人并没有为多尼伸张正义的欲望,首先这不是自家兄弟的事,而且还很危险,另一个原因就是:三人都有点厌烦打打杀杀了,想过过清闲的日子。

可眼下回家无望,三颗年轻的心未免就随之躁动了起来,既然平淡居家的日子不可得,那么……就让青春在张扬中燃烧吧!!!

楚云飞他们商量好后就去找多尼,多尼显然为他们的选择又惊又喜,于是大家很快地定下了行期。

刚贝拉那里盗墓的收获已经盘点了出来,初步估计那些古董卖到几百万美圆是不成问题的,当然,具体价值是谁也说不清楚的,于是刚贝拉拿出十万美金算是谢礼。

楚云飞他们自然是要推脱一番的,可按刚贝拉的话说,没他们,可能早就把魔鬼放出来了,那几个死去的盗墓贼足以说明问题的严重性了,这点小小礼物实在是不成敬意的。

言语间,大家还是能隐约感到肥胖商人对那种未知力量的恐怖,于是三人不再客气,收下了那份心意。

就在三人一切收拾完毕,打算去找多尼闯荡欧洲的时候,不速之客上门。

来的是成树国那天收拾过的四个白人,还是那个高个大眼睛的女士发话,“我们找选择新变化三位先生,有事相求。”

别人没什么反应,成树国可知道眼前这几个人,“我们没兴趣听你们的事,也不想管你们的事,现在我们要出去,你们让开!”

脾气暴躁的菲尔当下就跳了起来,“露丝小姐找你们是你们的荣幸,中国人,别以为你在我喝多的时候打败我是多么光彩的事,敢和我再打一场么?”他却没想到那天他喝多了,成树国喝得也不少。

这次是刘宁憋不住了,“你算什么东西,滚!”说毕一拳击出!

菲尔挥小臂向外崩格,现在大家都清醒,刘宁的力气大得离谱,整个人被击得踉跄退了好几步。

露丝小姐赶紧上前护住菲尔,不怒反笑,“几位朋友,我们只想跟你们做个交易,为什么不听听我们的条件呢?”

\section{第一百零八章 露丝的风情}

楚云飞笑笑,“好吧,露丝小姐,如果你认为你的条件有可能打动我们,那就说出来听听好了。”

原来,露丝小姐和她的哥哥杰瑞来非洲本为渡假顺便猎奇,同行的是露丝的好友伊琳娜和杰瑞的大学同学菲尔。

本来渡假是他们的初衷,不过,非洲独特的风土人情吸引了几个白人男女,于是猎奇的心思就多了点,多逛几天之后,他们越发地发现非洲有太多神秘和不可想象的地方了,尤其是他们听说了在索度居然还有食人族的存在。

楚云飞他们自然是知道食人族的,刚贝拉说过不只一次的,食人族目前被索度政府局限于索度西南的一小片保护区内,出了保护区,那是人人有权力格杀的。不过普通人进入保护区也不会受到政府保护,因为食人是那小部族的一种习俗,一种文化,虽不提倡,但也没有无故剥夺的理由。

菲尔本想去那里一探究竟,不过同行其他三人都强烈反对。露丝在同另一个手帕交索菲娅在电话里提起这事的时候,索菲娅表示出了强烈的兴趣。

原来索菲娅的爷爷最近莫名其妙地病了,而且请来的医生查不出问题出在哪里。笃信天主的老头天天祈祷,病情却一天重过一天。

当索菲娅听说传说中的食人族真的存在,马上就求露丝帮她打听下声名远扬的非洲“巫师”现在是不是还有传人,既然爷爷的病科学方法治不了,那其他的手段也要试试了。

索菲娅的爷爷是个很了不得的人物,露丝虽然不知道那老头到底是做什么的,却是知道老头子跺跺脚,华尔街是要颤一颤的。于是,虽然是索菲娅异想天开的一个建议,却让露丝加倍地留心起来。

很遗憾,在索度境内并没有什么声名显赫的巫师,不过,露丝却意外地发现了楚云飞他们。

在露丝眼里,中国人的神秘比非洲人不遑多让,尤其是成树国又会中国功夫,轻而易举地击倒了在法国大学生运动会上的拳击八强选手菲尔,不由得让露丝又增加了几分期待。

露丝开出的条件就是,楚云飞等人随她回去看看朋友的爷爷,只当她出钱请三人去欧洲玩耍一番,同时也可以向朋友表示她已经尽到心了。

这件事情实在是有点让人匪夷所思,楚云飞三人相互看看,猛然间居然不知道该如何答复对方。

沉默良久,刘宁和成树国的手臂同时抬起,指向楚云飞,“你问问他吧,他答应我们就好说。”

楚云飞自然知道战友心里是怎么想的,有人给代办签证还出钱,那自然是能省就省了。至于治病,没准成树国和刘宁也认为他能治好呢。

楚云飞本来没心多揽什么闲事,不过,既然对方有此美意,贸然拒绝是不是也不太好?

露丝本来就看着这白皙帅气的小伙子非常地顺眼,待到对方一犹豫,马上又是个笑意盈盈的眼波甩了过去,“选择先生,是不是你还想再增加什么条件呢?”

这动作和眼神,刘宁和成树国全看明白了,不过楚云飞却是一如既往地情商低下,“哦,我需要再考虑一下。”

露丝还以为他在拿桥呢,或者说羞涩的中国小伙不是很放得开,“好吧,只要你答应,其他事情,我们都可以慢慢商量的。”

其他事情?楚云飞想了想,“其他事情再说吧,反正治不好人我也不好多提条件的,对了,你们等等,我们三个先商量一下。”

露丝越发地觉得眼前的小伙子可爱了,这么明显的暗示都听不出来,本来是存了三分调笑的心思,现在可是又加了些欣赏在里面。还有就是眼前的小伙子居然还真有给人治病的打算,露丝不由得暗暗下了几分决心:实在说不动人,那么多少也要在其他方面再做做努力。

门一关,成树国随手就是一个大帽甩向楚云飞,不过他怎么可能打到?

“笨蛋,人家是看你了,任你提条件呢,我怎么会有你这么丢人的朋友。”

刘宁也是笑得前仰后合,“云飞……云飞的样子好傻,哈哈哈。”

楚云飞有点恼了,“别胡说,人家性格开放而已,也未必真有什么心思。”

刘宁好不容易才止住笑,“哈,虽然是这么回事,不过,你总算是被人调戏了一把,还没什么反应。”

成树国也点点头,“就是,我才发现云飞的情商确实不是很高啊,那么多言情小说白看了。”

楚云飞才要张嘴反驳,却见成树国表情严肃了起来,“云飞,我很担心……”

楚云飞和刘宁的注意力被吸引了个足又足,才听到后面的话,“听说白种女人很多人有狐臭的。”…………

成树国终于没逃脱楚云飞如影随形的一脚,为了不再被殴打,他马上打开了房门,“露丝小姐,选择先生很愿意为你效劳,是吧,云飞?”

楚云飞脸上已经是一脸的平静了,他实在不想再被人取笑了,“哦,我想我们需要同塔尔族里的同伴协商一下。看大家怎么走。”

多尼的事是越少人知道越好的,虽然眼前这四个人和多尼有瓜葛的可能性很小,但楚云飞还是很晦涩地暗示多尼那里需要沟通一下的。

露丝还以为还有中国人在塔尔部落,不过人家既然不说,她实在是不好再问了,“好吧,小伙子们,我们等你们的好消息哦。”说完,又是一个媚眼甩了过去,比刚才那个还直接。

楚云飞装做没看见,手向空中一挥,“这里蚊子好多。”

身后,是成树国和刘宁按捺不住的低沉笑声。

露丝就坡上驴,“是啊,蚊子真的好多,咬得我身上到处都是疙瘩,我真的想快点回欧洲了。”说着,细长的手指还轻柔地摸摸自己的肩膀。

欧洲人不着痕迹地勾引人的方式,还真是很有诱惑力的。

楚云飞苦笑着点点头,“好的,我们商量好以后去通知你们。”

露丝点点头,说话的声音快滴出水了,“我们就在云顿宾馆住的,我住218房间,你记好哦,快点来找我们。”

这女人,越玩越上瘾了!

\section{第一百零九章 取钱的风波}

露丝四人回到旅馆,杰瑞开始斥责妹妹。“没事你跟那些黄种人抛什么的媚眼,咱们这里不是还有菲尔么,就算你看不上他也不至于眼力低下到那种程度吧?”

露丝没理自己的哥哥,而是坐在那里不停地轻笑,“好可爱的中国人,嘻嘻,不知道今天他会不会来偷偷地找我啊?”

菲尔也极为不爽这种有差别的待遇,“露丝你疯了吗?去招惹那些不知道好歹的中国人。”

伊琳娜平时不太爱说话,不过她却知道自己好友的性格,“菲尔你不用生气,露丝对付男人很有一套的,别为她担心。”

杰瑞忍不住接话了,“露丝是什么脾气我自然知道,我是怕她玩得太过火,惹怒了那几个神秘的中国人,你不知道选择是里面最厉害的么?”

露丝不停地轻笑,“我知道他最厉害,他还想给索菲娅的爷爷治病呢,不过,他发情的样子一定很有趣的,嘻嘻,你们不觉得选择先生其实很帅气、很有型么?”

菲尔盯着露丝看了半天,然后露出个很邪气的笑容,“听说中国有种密法,可以让男人吸干女人的,你可要小心啊,哈哈。”露丝一路上并没有给菲尔什么机会,虽然菲尔并没心思同好友的妹妹建立什么超友谊关系,可眼睁睁地看着美女投入别人的怀抱,嫉妒之心还是难免的。

露丝对菲尔的恶毒非常地不满意,脸色沉了下来,不过她马上又轻笑起来,诡异地斜眼瞟着对方,“菲尔,似乎跟你好过的女人有些传言,不知道是不是真的?”

菲尔的脸刷地就红了,他自然知道露丝指的是什么,虽然他比较风流,但床上的功夫实在算不上高明,而且耐久力出奇地差劲,没想着这种传言居然能传到露丝耳朵里。

“算,我不跟你说了,你就胡闹吧。”

两个男人离开了露丝的房间,只剩下傻笑的露丝和沉默的伊琳娜。

楚云飞他们和多尼碰了碰头,多尼的意思就是大家一起走好了,虽然先是直飞英国,不过,要从英国到法国实在是很方便的。至于露丝他们四人,多尼基本上可以排除是对头的可能,而且,以前他在欧洲四处闲逛,没准是真有认识的可能,露丝这名字实在是大众化了点。

然后几个人又把价钱定了一下,大家约定,到了法国多尼先支付三人五万美圆,等到三人帮多尼把事情解决掉,再视事情的复杂程度收取四十五到六十五万美圆的酬劳,自然,在这期间,多尼是要负责三人的日常费用的。

楚云飞一行八人在第二天就到了索度首都喀津霍,杰瑞和菲尔还要继续他们的非洲之旅,不过露丝和伊琳娜已经不想再玩下去了,她俩去为四个同伴办理签证。

多尼本来也没想通过法国大使馆补办护照,那样实在太不安全了。他的本意是再来次偷渡,过红海,取道也门、叙利亚进入欧洲,反正在中东多尼是有相当渠道的。不过,现在有人为他们四个的索度身份代办签证,那实在是再好不过的事情了。

于是多尼就热心地陪着两位女士去英国大使馆,留下三个中国人在宾馆里呆着。

楚云飞他们实在闲的无聊,锁上门去喀津霍的街头乱转,由于三人的武器已经全部留在了“提坦卡布”部落,只有成树国带了十来枚钢针,倒也不担心会有人误入生事。

三人先去街头乱转了一阵,意外地发现索度居然出现了大量的中国商品,以日用品为主的轻工产品居多,也有很多诸如打火机、玩具之类的小商品。

于是三人就兴致勃勃地四处寻找中国商人,这东西总是要有人贩卖才会出现在这里的吧?

遗憾的是,转悠了半天,也没找到一个黄种人,白人倒是遇到了两个,三人找到最后,居然意外地发现了中国大使馆。

中国大使馆位于使馆区,远远地,三人就看到了旗杆上高高飘扬的五星红旗,大家互相看看,谁也没心思再逛了。

于是三人就找间酒店狂灌一阵啤酒,喝得有三分醉意了,楚云飞才想起来三人的钱还没倒帐。

刚贝拉给三人的十万美圆是存在索度国家银行的,在索度存取很方便,不过要是去了欧洲或者其他什么地方就不行了。

因为索度的银行系统不是很完善,并没有同国际金融完全接轨,所以不能使用国际通用的“万事达”之类的卡,三人早就商量好,来到喀津霍就把钱转存到美国的花旗银行。

不过十万美圆的支取金额显然大了点,索度国家银行的职员虽然很纳闷眼前这三个黄种人用的是索度身份证,但还是彬彬有礼地告诉楚云飞他们这么大金额需要预约,请他们明天再来。

楚云飞他们已经喝了不少酒了,防备之心就降低了不少,并没有注意到那个女银行职员眼中炽热的贪婪光芒。

如果三人用的是中国护照,或者他们是黑皮肤的索度人,都不会引起太多的事端的,可惜的是,三个人的相貌和身份实在有点不伦不类。

第二天一大早,楚云飞他们三个就去登门取款,这次接待的职员并没有推脱,很痛快地把钱装进一个大的纸袋中递了出来。

楚云飞他们刚跨出国家银行的大门,就发现情况有点不对劲。

四周多了七、八个黑人大汉,有意无意地向三人慢慢拥来,被包围了!

成树国把纸袋交给了楚云飞,冷冷地向四周一扫,“你们是什么人?”

那些汉子看到被三人发现了,二话不说就扑了上来。

不过,上来得快,下去得更快,没有半分钟,八条大汉就“噼里啪啦”倒了一地,开什么玩笑,这点人怎么能招架住楚云飞他们三个?

三人刚要转身离开,楚云飞一眼看到一个汉子在身上掏摸着什么,一个箭步上去就是一脚,“啪”的一声,一把手枪被踢到十多米远的空地上。

那汉子也被这脚踢断了手臂,疼得在地上翻来覆去地打滚。

另一个汉子已经从地上爬了起来,“警察,不许动,你们要是拒捕,我们有权力开枪。”

\section{第一百一十章 好多的罪名}

楚云飞他们根本没理这茬,谁知道他们说的是真的假的?

那汉子见警告没用,大喊一声,“他们拒捕,大家拿枪,击毙罪犯!!!”

这话明白地揭示了这几条汉子的用心,连罪名什么的都没有就要杀人,不是为钱是为了什么?

自己找死,那就怪不得别人了!

楚云飞他们也不再犹豫,手脚并用,冲上去就是一阵痛打,边打边搜对方的武器,八个人居然带了六把手枪,还有手铐什么的东西。

按道理来说,索度警察出警的时候通常是长短枪都带的,不过这帮人显然抱着一些不可告人的目的,不但全部是便衣,而且携带的全是短枪。

八个人开始还有挣扎反抗的心思,不过他们马上就发现三个黄种人下手根本毫无忌惮,招招指向要害部位,不得不乖乖地躺到地下装死狗。

国家银行的保卫听到声音也冲了出来,看看不过是两帮人斗殴,就安心地站在那里看戏,反正是他们职责外的事情。虽说被打的是同胞,不过,在这个城市里似乎只有白人打黑人的份。

把六支手枪拢到一起,放在地上,楚云飞深吸一口气,连着几脚踩下去。六支枪被踩得分崩离析,机簧乱跳,坚固的水泥地面也迸出了几个小小的凹坑。

“嗤,一群垃圾!”三个中国人不屑地看了看躺做一堆的索度人,拎着钱袋,大摇大摆地推开围观的众人扬长而去。

楚云飞百忙之中不忘记向国家银行里面狠狠地瞪了一眼,虽然隔着厚厚的玻璃门,门里那个心怀鬼胎的女职员还是吓得浑身寒毛直竖:天呐,这次可算是撞上铁板了。

三人并没理会他们造成的混乱,直接去半公里外的花旗银行开户,三人要存的钱数目不算少,受到了一定规模的接待。

等到三人开好户,存起了九万多的美圆,出门时才发现外面被索度警察围了个里三层外三层,水泄不通,一副如临大敌的样子。

楚云飞皱皱眉头,“你们这是做什么?”

马上有个貌似领导的黑人走了过来,“你们三个涉嫌走私、袭击警察、危及公众安全等多项罪名,被逮捕了。”

楚云飞他们不太了解索度的法律,不清楚眼前这情况该怎么办,不过刚贝拉在他们临走前说过一句话——“你们可算是我们‘提坦卡布’的人了,在索度咱不惹别人那已经很给他们面子了。”

这话绝对是有些夸张的成分在里面,还是有那么几个部落势力是比“提坦卡布”强大的,不过,如非必要,强者间是很少发生直接碰撞的。

于是楚云飞三人也放弃了抵抗,任由警察给他们戴上手铐。

手铐一戴上,就有一个鼻青脸肿的汉子冲上来挥拳就打,丝毫不顾忌旁边众多的围观群众。

“妈的,给脸不要,”成树国的脏口才出嘴,楚云飞已经一个飞脚把对方踢翻在地,“什么东西?”说罢,楚云飞还恶狠狠地朝对方大腿跺了一脚。

那汉子当时就抱着大腿疼得在地上打滚,嘴里还尖声嚎叫着。

一旁马上有七、八个警察冲了上来,打算用人肉沙包压制住楚云飞,楚云飞身形快得离谱,如鬼魅般闪了几下,双腿连踢,成树国和刘宁也上前来帮忙,眨眼间又放倒一大片。

旁边传来喇叭声,“你们必须放弃抵抗,你们必须放弃抵抗,否则我们要开枪了,否则我们要开枪了。”

原来,早有几个警察已经在旁边架起了枪,跃跃欲试了。

楚云飞旋风般冲了上去,目标是那个貌似领导的黑人,一把拎住了对方脖领子,甩向了成树国和刘宁,“这是人质,你们看好。”

说话间,楚云飞并没有停留,而是冲向了那几个端起冲锋枪的警察,一脚一个,对方尚未来得及做出反应,就被楚云飞踢了个落花流水。

其实那些警察还是有开枪的机会的,不过楚云飞行动实在是太快了,周围围观的人也太多了,贸然开枪误伤民众的话,那责任不是什么人都负担得起的,再说,三个黄种人都已经戴上手铐了,也不方便以“拒捕”的罪名开枪,更何况还有个二级警长被对方拿来做挡箭牌。

那个二级警长肚子里早把那断了大腿的人骂得体无完肤了,没事你瞎冲动个什么劲?想报复,回了警察局那还不是任你为所欲为?实在是害人不浅。

想归这么想,那二级警长还是当机立断,要随行的警察放下枪,“你们这群笨蛋,人家有过抵抗么?还不快放下枪?还有,把地上那个白痴快送去医院。”

轰动的场面当时就被这几句话镇住了,警察们上前把楚云飞三人拥上警车,却是不敢再毛手毛脚了。

不过事情显然不可能就这么结束,当楚云飞三人被拥进警察局,身后的铁门“哐”地重重的关上时,那些警察又重新露出了狰狞的面孔。

开始警察们还做得比较收敛,把三人的手铐都铐到了墙上深埋的铁环中,还有人好心地解释,“这是规矩,没定罪的人都是这么铐的。”

不过,等三人双手都已经被固定在墙上的时候,屋里的六个警察对视一眼,放声大笑,“哈哈,你们三个杂种,也有现在这个时候?”

说笑间,一个矮壮的警察走上前来,手里的警棍伴随着他的大肚腩晃来晃去,“杂种,你们打人呀,你们不是很厉害么?”

伴随着这话,警棍带着风声砸向了楚云飞的脑袋,谁叫他刚才最活跃呢?

楚云飞气运全身,一侧头,用肩膀接了这一棍,身不动,肩不摇,一个“裙里腿”飞出,正正踹中那警察的面部,当下就是,“胶棒与警帽齐飞,眼泪共鲜血同流”。

矮胖警察当时就躺在地上翻来覆去地打滚,别的警察看到这场面更是怒火中烧,“反了你们这几只黄猴子了!”

说着,两个警察从一旁的桌子里抽出两根电棒,狞笑着走向楚云飞,“嘿嘿,别怕,我们不碰你,我们只需要把这个可爱的小东西放到那铁链上就行了,看这十五万伏的电棒能不能让你安静点。”

铁环除了被固定在墙上,还有长约一米的铁链连接着,两个家伙远远地伸出电棒触向铁链。

\section{第一百一十一章 打了多特人}

楚云飞摇摇头,怒极反笑,“我实在是不知道谁该害怕。”

楚云飞他们三个人终日在生死边缘打滚,早就养成了“万事留一手”的习惯,刚才警察能顺利把他们铐到铁环上,那就是留手了,要不他们哪里肯乖乖戴上手铐?

三人里数刘宁的拳脚功夫差劲,不过经历了索度边境上的那次生死洗礼,他的气功却又比成树国高出了不止一筹。这三个彪悍的军人根本不是副小小的手铐铐得住的。

看到那俩警察不知死活地伸出了电棒,楚云飞两手一用力就崩开了手铐,双脚交替踢出,把两根电棒踢飞,接着双手抓住那俩警察的头发重重地撞在了一起。

一个看起来比较瘦弱的警察当时就昏了过去,剩下一个正眼冒金星头晕目眩之际,楚云飞的手又到了。

“啪啪啪啪”连续十几个正反耳光打了过去,那警察登时就口鼻出血,眼见就要不省人事了。

楚云飞恨恨地骂了句,“什么东西,”松开手里快要昏迷的警察,又一脚把那个正要爬起的大肚子警察踢得再次趴下,“给我趴着,我叫你起来了么?”

楚云飞回头看看,刘宁和成树国还在墙上被手铐靠着,做出一副很无助的可怜样子。

摇摇头,懒得理他们,楚云飞把眼光转向了另外三个没动手的警察。

那三个警察没有动手,已经很能说明他们的为人秉性了,楚云飞大大咧咧地走近那三人,不管他们的惊惶失措,伸手拉个凳子坐下,刘宁和成树国既然喜欢站在那里就让他们站着好了。

“你们平时就这么处置待处理的人?我们三个到底犯了什么罪?”

三个警察里一个明显年长的人开口了,“平时也不一定要这么处理人,只是你们打了稽查科的人,还损坏了人家的手枪,他们要求我们给你们吃点苦头的。”

“至于说犯了什么罪,谁说得清楚?稽查科的人一口咬定你们是走私犯,我们这不是正要审讯你们么?”

楚云飞摇摇头,这索度的文明情况实在是有待提高,“问你们就好好的问嘛,居然想先打我们一顿?还不快把我的同伴放下来?”

年长警察只有苦笑的份,他总不能说被他们打的人里有一个是喀津霍第三区警署五分队队长的侄子吧?他还要继续在警察系统混呢。不过说实话,在他看来警察打人实在也不是什么大不了的事。

刘宁和成树国马上被放了下来,地下那三个警察也战兢兢地爬了起来,却不敢出门,因为楚云飞就挡在出门的必经之路上。

接下来就是对三人的询问了。

当楚云飞报出“提坦卡布”部落身份的时候,记录的那位身子颤了一下,那个被扇了十几个耳光的警察怪叫了一声,然后用含混不清的口齿问道,“天呐,你们不会就是选择新变化三位大人吧?”

楚云飞斜眼瞟他一眼,心里着实纳闷,自己三人名声有这么响么,居然能传到首都的警察局?“在部落里大家都这么叫我们,你怎么会知道这个?”

那警察恐惧心立去,马上就扑了过来,红肿的脸庞压抑不住欣喜的笑容,“大人,我就是‘提坦卡布’部落的人啊,真没想到能在这里见到大人,真是太好了。”

楚云飞的眉头皱了皱,不过他从对方的神情上能看出那份发自内心的真诚,所以他也没掩饰自己的疑惑,“你听说过我们三个,刚才还敢对我们动手?还有,把你脸上的血擦擦,什么样子?”

什么样子?还不是你打的?红肿脸马上变得尴尬了起来,“这个,大人,不瞒你说,最近实在是见了不少中国人,我把你们也当成那些中国骗子了,哪里会想到是三位大人来啊?我有一年多没回家了。”

既然连警察局内都有了人证,三个黄种人实在是没必要再审查下去了,别说他们不是走私犯,就算是走私犯,有这么个警察同伙,其他同僚也只能睁只眼闭只眼了。

红肿脸块头不小,地位似乎也满高的,马上招呼其他同僚给三人倒水喝,同时小心翼翼地提出个问题,“大人就是选择先生吧?”

楚云飞自然知道是自己的表现太过扎眼,对方才做如是想的,不过他还是点了点头,“我就是。”

红肿脸马上兴奋了起来,“听说迪赞是大人的徒弟,很听大人话的,不知道这话是不是真的?”

楚云飞当然又点点头,“我教过他些功夫,不过,那小子最近不学好,总是惹是生非的。”

红肿脸的眼中透出了期盼的光芒,做惯警察的人说话是直来直去的,“选择先生,我的哥哥最近欠了迪赞一笔钱,你也知道,一个人在外,很花钱的,我也没有多余的钱帮哥哥还债,这个……你能不能帮忙跟迪赞说一声,让他把利息给我哥哥免了?”

说话自然是很简单的事,但楚云飞内心还是个潜规则的默认者,那些欠迪赞钱的人虽然可怜,但可怜之人总是有可恨之处的,他并不想因为眼前这个不熟悉的警察去贸然干涉什么,“这个,我想你应该明白,各个行业有各个行业的生财之道的,我虽然可以帮你说,但我不愿意帮你这么做。就像你们警察,肯定也有你们来钱的地方,但有人想从你们这里说情,断你们来钱的途径,你总不会很舒服吧?”

红肿脸自然知道对方说的是实情,摇了摇头表示赞同,不过那失望之意是谁也能看出来的。

楚云飞就当没看见一样,话锋一转,“不过,这个事我还是可以帮你问一下的,你哥哥欠了他们多少钱?”

红肿脸又变得愤愤不平起来,“我哥哥不过借了他们三万布索,他们居然要他还八万,还说这个月不还下个月就是十万了,这个该死的迪赞,不要让我在喀津霍看到他,否则,我一定要好好地收拾他。”

十万布索?也不过六百多美圆而已,至于么?楚云飞点点头,“好了,钱不是很多,这个钱我可以帮你出了,我不帮咱多特人帮谁?不过,你必须告诉我一件事。”

红肿脸大喜过望,他原本只想帮哥哥争取到只还本金,没想到选择先生居然这么慷慨,连本带利都要替他还清,“选择先生你放心,能帮你办的我绝不推辞,你说好了。”

楚云飞沉吟半晌,还是决定实话实说,“我的要求不高,我只是有点疑惑,喀津霍那么多中国货,我们怎么一个中国人都没看到?而且你又说中国人都是骗子,你能告诉我这是怎么回事么?”

\section{第一百一十二章 林子大了鸟多}

红肿脸讲述的事实在让楚云飞哭笑不得。

原来,索度在近年里门户开放,由于闭锁得太久了,挖石油又多少挣了点钱,所以吸引了大批外国商品涌入索度。

可索度人别说“刚富起来”,就连“解决温饱”尚未成为普遍的现实,所以,人均购买力是相当有限的。这时候,物美价廉的中国商品就成了抢手货。

中国人是惯爱扎堆的,当有几个勇于吃螃蟹的中国人在这里赚得钵满盆盈,荣归故里后,再来的时候自然会呼朋唤友、携亲带眷。在短短的半年内,喀津霍的中国人呈几何级数地爆炸性增长。

人多了,可市场就是那么大,或者说市场还没完全培养起来,而且索度除首都喀津霍以外,其他地方的安全性还不是很高。这么一来,中国人之间就自然地出现了竞争。

想谋求暴利的人是永远存在的,赚惯了大钱的人不可能再去蝇营狗苟地去挣那点小钱的,所以就有了“以次充好、以假充真”的中国商品。

甚至有些中国商人勾结索度人,从来就是做假货,不做真货的,无他,利益使然。

开始的时候,这还是个别现象,但劣质货低廉的价格对市场的冲击实在太大了,而来的中国人又多是存着“捞一把就走”的心思,于是一夜间假货、劣质货充斥索度市场。只有一些来得较早,有固定销售渠道的中国商人还在奋力抵挡,自然也不乏一些品行较好的国人也是起早贪黑、忙忙碌碌地赚几个不黑心的钱。

当索度某高官的爱妾用了伪劣的中国化妆品弄得脸上红肿蜕皮、双目险些失明的时候,中国商品终于成了过街老鼠,成为了“垃圾”的代名词。索度政府雷厉风行地查封了大多数的中国商品。

大浪淘沙,自然还是有些中国商人以优质的服务,货真价实的商品及良好的口碑幸存了下来,并且在短短的时间内奇迹般地夺回了不少失去的市场。

不过,像这种成功的中国商人在索度那是用手指数得过来的,所以,那些后来的中国商人不论良莠一律被打上了“奸商”的标签,索度不欢迎他们。

楚云飞听这这些讲述,似乎是中俄边贸的翻版一样,不禁摇摇头,这些国人,丑事居然做出了国门,实在是祸及同胞啊。

成树国听到这里,头上青筋直冒,“嗨,你也不能这么说,那些奸商如果不跟索度本地人勾结,哪里能做得这么天怨人怒的?”

红肿脸虽然知道三人的厉害,不过也没有部落里那些人那么敬畏,只是选择先生马上要做财神了,他自然也不便多辩解。心里还是暗自反驳:教唆犯永远比真正的罪犯更可恶,没有那些中国人,索度人想做也做不出这些事。

刘宁喝止了成树国,“好了,这种丢人事就别说了。”

楚云飞随口问了一句,“那就是说现在中国商人只有那么几个,其他人都很少来了?”

红肿脸那红肿的脸上露出了无可奈何的笑容,“哪里像你说的那么简单呀?来是来得少了,可法子变得狡猾多了,一个月总能抓到几个偷运货物进来的,这不我们这里现在还关着一个呢?”

楚云飞挺好奇,“这里还关了一个?什么罪名?你们打人么?最后怎么处理?”

红肿脸不由得暗骂自己多嘴,中国商人可都是财神,自己这帮人还指靠着他们活呢,不过话都说出来了,再收回也难了,“罪名?罪名肯定是走私伪劣商品啦,打人嘛,那要看情况了,不过一般都是要打的,不打他们不好好地出钱,总想少出钱。”

楚云飞正听得心头冒火,成树国早就坐不住了,“我说,你们这么做就不怕中国政府干涉?你怎么知道人家不肯出钱?没准就是没钱呢,你们这手一没轻重,弄个伤残什么的谁负责?”

弄伤残?弄死人也是常有的事,不过红肿脸实在是没勇气这么说,只好委婉地解释,“中国政府对这种人撇清还来不及呢,总声明那些走私奸商是个人行为,中国政府是不支持的。”

这话又有点触及三人心里的痛处了,于是三人又不吭声了。

红肿脸看到对方不接话,只好把想回避的话题兜出来点,“至于说打人,确实是那样,只要他们一挨打,总会想办法多凑点钱出来,我们这里也好放人了。”他实在不敢再多说什么了。

又过了一阵,楚云飞才接了话茬,“你们这里关的这个,能不能放出来让他跟我们聊会儿?”

红肿脸登时尴尬万分,答应不是,不答应也不是,怕就怕这个,没想到选择先生还真提出来了。

楚云飞看他在那里踌躇,也想得到几分原因,轻拍对方肩膀,“好啦,我刚才都说过了,有人想从你们这里说情,那就是断你们来钱的途径,这些我明白的,快把人带来。”

红肿脸愣了一下,拍拍那个脸部被踹的警察,“走,跟我去提人。”

那个警察肚子里的火还没消去,很不满意地瞪了同僚一眼,“要去你自己去,我不想动。”由于嘴唇被踢肿,说的话很有几分含糊。

红肿脸可不管他的感受,那家伙平素就是个嚣张跋扈的主,这次要再不提醒提醒他,没准他又要吃亏,那就有失同僚的情谊了,“我叫你走你就走,废话那么多干什么?”

等到两个警察把人带来的时候,脸上被踹的那位神色恭敬了好多。

楚云飞可是火冒三丈了:怪不得要俩人提人,眼前这中国人被打得遍体鳞伤,眼神涣散,像条死蛇一样浑身软绵绵的,一个人根本搀不起来。

成树国立刻就爆发了,“都是谁打的他?给我站出来!你、你、还是你们?”

一屋子六个警察哑口无言,谁也不愿意出头给自己找麻烦。

还是红肿脸出来解了围,“这种事,大家都有份的,我也想不到你们三个今天会来呀,没想到你们都入了索度籍了,居然对自己的同胞还这么照顾。”

楚云飞无奈地笑笑,不照顾,那自己还算人么?

成树国还要说什么,被楚云飞一眼瞪了回去,用汉语训着他,“你少说两句吧,难道你还要单挑了这个警察局?”

成树国还是愤愤不平,“挑了就挑了,那算多大的事?”

\section{第一百一十三章 救了李南鸿}

大家都没想到的是,两句中国话说出口,地上半死的那位当场就蹦了起来,再看不出一点奄奄一息的样子,连滚带爬地来到楚云飞身边,抱住了他的双脚,声泪俱下,“哥们,拉兄弟一把!拉兄弟一把!!!”

好熟悉的台词,是不是还该有句“看在党国的份上”?楚云飞看看脚下这位,那肮脏褴褛的衣服和污秽的脸庞让他禁不住地皱了下眉头,不过他可以发誓,那只是下意识的反应,而不是对同胞有什么嫌弃。

“哈哈,我说,你刚才还是那半死不活的样子,怎么现在精神头这么好?”

地上那位倒是有什么说什么,“他们本来就把我打得半死了,不过,有中国人来,我肯定爬也得爬过来啊,要不我就该全死了。”

刘宁看到同胞这样狡诈,也是有点挠头的感觉,“嗯,你这装死的法子倒是实用,不过,怕是以后别的同胞再不能使用了。”

地上的中国人振振有辞,不过中气确实有些不足,“操,你以为这些索度警察会相信啊?没前面那些同胞,我也不至于被打成现在这样啊。”

楚云飞不再跟他罗嗦,“你要我们转告谁?我们给你带话,让他筹钱。”

地上的人沉默一阵,放声大哭,“呜呜,我他妈的能转告谁?这破地方我一个人都不认识……呜呜,要等从中国弄钱来,三个我也死定了。”

成树国听不下去了,又犯起躁来,“好了好了,你说保你要多少钱吧。”

那地上的人不知道是狡猾还是真的冤枉,“我怎么知道他们要多少钱啊?呜呜,不过,大哥,你先帮我垫上成不?等我回了国,三倍……啊不,五倍!五倍还你,我现在就打借条,成不?我实在是受不了啦。”

红肿脸等一干警察看到眼前几人“呜里哇啦”说个不停,就算听不懂人家说啥,也知道那犯人正在诉苦。正彼此私下递着眼色,却听到选择先生问他,“这个人到底犯了什么事?”

红肿脸只能硬着头皮解释下去了,“谁知道他犯了什么事?他在街边卖东西,被人举报,我们就抓来了。”

楚云飞实在是不想多事,依着他以前的脾气没准就已经翻脸了呢,“好了,他说了,这里没熟人,看来只有我们保他了,你说要多少钱吧。”

红肿脸又苦笑一下,解释自己的苦衷,“我们收了钱要向上面交一半,还要奖励举报人,要不下次谁向你举报啊?我们这队有二十多个警察呢,大家也得分分,平时没有五百美圆,绝对不可能放人的。”

“这还是好的呢,弄到稽查科,没有一两千美圆,人家宁可把人打死也不放人。”

镇静、镇静,楚云飞不停地告诫自己,虽然眼前同胞被戕害的事实无可质疑,他也不可能对此无动于衷,但是……唉,这事又怎么是他的能力能彻底解决的?只好能做多少算多少了。

沉默一阵,楚云飞刮刮鼻子,开口了,“好了,就五百吧,我也不多说了。对了,你还要还债呢,大概也得五百,你这几个兄弟……呃,还有你,我们也不是故意的,再给你们五百大家分分,该看病的看病,没病的拿去喝酒,算我们一点小小的心意,一共一千五百美圆,你们觉得怎么样?”

怎么样?一千五百美圆,红肿脸他们还能说什么?起码有五百美圆就是这六个人分了,没挨打的怕是还在后悔呢。

成树国一直被刘宁盯着,半天没说话,听到谈好价钱,走上前来,“再多给你五十美圆,你带他去弄身衣服,洗个澡,我们在这里等你们。”

红肿脸带人走了,剩下就有那手快的警察麻利地开出了张收据,大致意思就是:取保候审费,五百美圆。

有钱果然办事快,还不到四十分钟,红肿脸就带着人回来了,头脸也洗干净了,衣服也换了身新的,还带来了他的护照等一应随身物品。

虽然是那种价格低廉的非洲大褂,起码也算有个人样了。

成树国把剩下的一千零五十美圆交到红肿脸手上,“我们能走了么?”

红肿脸千恩万谢地表示了感激,不过他还是表示希望三个中国人能把他们自己的事说清楚。

刚取钱出来就被不亮身份的警察围攻,这事再简单不过了,里面的味道大家也都明明白白,当楚云飞陈述完毕事实后,红肿脸警察把记录纸随手就扯了:这事实在没办法留底子。

等到四个中国人出门扬长而去,六个警察聚在一起分赃完毕,就有人想起来问题,“摩卡,那三个中国人在你们部落住?很厉害的样子嘛!”

红肿脸瞟大家一眼,满脸的得意,“今天幸亏有我在,要不你们就惨了,知道么?那三个人在一个小时里杀了六十多个匪徒,有他们在,我们部落没人敢惹。”

其他警察面面相觑,还好,真的还好,有摩卡在,人家居然还多给了点钱!

楚云飞他们把同胞带回宾馆,少不了又是一顿盘问。

同胞叫李南鸿,今年才二十岁,江南省人,家里是做小买卖的,薄有积蓄。李南鸿两次高考不第,铁下心想子承父业做买卖,却遭到家人的反对。实在郁闷难耐,家里人就安排他出国旅游散心,一口气签了欧洲六个国家。

李南鸿自己认为这是个难得的好机会,出国走走,看看各国的商情,没准里面能有什么商机,回家也好向父亲反映,让父亲明白,你儿子也是很有商业头脑的。

到索度其实很好签的,李南鸿听说这里买卖好做,自己又加签了一个,不过等他欧洲逛完来到这里,才发现索度根本不欢迎他这种后来者。

家学渊源的李南鸿并没有马上放弃,而是托同学寄来点小商品,就地公关,挨门挨户的去推销。

作为陌生人,李南鸿推销的结果那是可想而知的。于是他就亲自上街叫卖,结果立刻被有心人当场举报,货物没收还是小事,总共也没几个钱,人可就受了大罪了。

再加上他随身携带的钞票都被做为“非法所得”没收了,就落到了眼下这步田地。

\section{第一百一十四章 恨屋及乌}

听到这里,成树国忍不住问了一句,“你当时身上带了多少钱?”

李南鸿青紫交错的脸上露出悻悻的神色,“我身上有四千多美圆呢,那货总共才值一百多美圆,我操,这索度人不是一般的黑。”

是可忍,孰不可忍!三个中国士兵当时就火了,大家相互看看,又一起看向楚云飞。

楚云飞皱着眉头,想了半天,这事怎么样处理才能达到最佳效果?

有了!楚云飞想起了国家银行的那个女职员,“这么着吧,走以前咱们去把银行那个女职员干掉,让他们明白,黄种人不是随便好动的。”

刘宁和成树国交换一下眼神,这倒是一举两得的好办法,两人本来就不想放过那个谋害顾客的女职员,既然这样做还能敲山震震虎,那为什么不做呢?

肯定会有人明白这女职员是为什么死的,保不准稽查科的人都会清楚。不过,知道的人是不可能出来做证的,最多也就是阴着来。索度再怎么说也是法制国家,银行职员内外串通,谋财害命那绝对是了不得的罪名。

于是,当楚云飞一行人踏上飞机飞往英国的时候,索度首都就多了一起命案,不,是灭门惨案!

杀人凶器就是死者家的餐刀,一家十口人,死了十一个,多出来的一个是来借钱的,看来借钱有时候并不是一个好的习惯。

刘宁本有心把杀戮的目标只对准一人,不过楚云飞的建议获得了成树国的支持:咱中国人被活活打死的时候,也没人考虑他们到底是不是无辜的。

那女职员到死也没弄明白,她一家的死不是因为泄露商业机密,而是一个她不认识的中国人挨了打。

李南鸿随同楚云飞他们登上了飞机,没别的原因,羞刀难入鞘,出了这么大的纰漏,他也不好意思就这么回国,他还想在欧洲继续一下淘金之旅呢。至于钱,先借几个哥哥的用吧,谁叫他们在首都随便杀人来着,总不能眼看着同胞面临再次的报复吧?再说路费都有人出。

…………

空中客车A330带着巨大的轰鸣声,在英国伦敦的希斯罗机场上方兜着圈子。

楚云飞他们并不知道发生了什么事,可他们不知道并不意味着别人也不知道,议论声慢慢响起,十分钟后,一个穿着洞洞装,看起来有点“雅皮”味道的白人青年大声抱怨了起来,“你们在搞什么?为什么还不降落?”

李南鸿从前面的座位探回头跟楚云飞说话,“飞哥,似乎出了点问题?”

本来是楚云飞和李南鸿同座的,不过这家伙非说自己晕机,鼓捣着露丝同他换了座位,坐到了伊琳娜旁边。

这话谁会信?前调一排座位就不晕机了?

不过露丝笑意盈盈地答应了,伊琳娜也没表示反对,刘宁和成树国更是隔岸观火,两张脸上全是按捺不住的幸灾乐祸。

楚云飞有时候真的会有种“救错了人”的感觉,不过,他心里似乎也有点微微的欣喜,壮慕少艾,总也算是人之常情。

还没等楚云飞回话,露丝悠悠地出声了,脸上的从容也不见了,带了三分的疑惑,“是啊,早该降落了呀。”

四周人们“嗡嗡”的讨论声越来越大,楚云飞很敏锐地注意到了几个关键词——“恐怖分子”、“恐怖袭击”、“炸弹”。

看着大家的情绪有失去控制的危险,机组中的空中小姐马上出面解释,“大家请安静,请安静!”

“我们确定这架飞机处于正常的飞行状态,不过机场似乎出了点事情,跑道被堵塞了,我们正在联系盖特威克机场,看是否能在那里降落。”

听到空中小姐的解释,机舱里躁动的气氛被缓和了下来,只要飞机没事,谁会管下面发生了什么事。

也有那老成些的乘客隐约觉得事情未必是如此,所以还是有人在那里嘀咕,“开什么玩笑,希斯罗机场那么多跑道,能全部堵塞了不成?”

跟楚云飞隔个过道的白发老头说话倒是很幽默,“天哪,我终于有机会考虑公司以外的事了。”

又过了十来分钟,飞机终于在跑道上着陆,缓缓地停了下来。

空中小姐又站了出来,“各位乘客,很抱歉刚才没有向你们解释清楚,现在的情况是,希斯罗机场可能被恐怖份子安置了炸弹,处于橙色警戒状态,还请大家不要拥挤,配合警方的搜索行动。”

都已经落地了,自然没人再斤斤计较空中小姐的谎报军情,至于炸弹,那个东西总有个爆炸范围的,听从警方的安排就是了。

只有李南鸿一个人在那里大惊小怪,声音还很大,“啊啊,小姐,我觉得您侵犯了我的知情权,我有理由投诉您的,不过,您要是愿意把手机号码留给我的话,我可以考虑放弃投诉。”

楚云飞斜瞟他一眼,小声说了句,“就你这鼻青脸肿的样子也想泡妞?还是养两天再说吧。”

那空中小姐显然是见多识广的,轻笑了一声,“这位先生,很不好意思,我没有手机,不过我可以把我男朋友的手机号码告诉你,不知道你想不想知道?”

李南鸿也没把小姐说的话当回事,而是回头反驳楚云飞,“呃,有人倒是没有鼻青脸肿,不过,送到嘴边的美女都不想要,不知道是不是身体某些器官有什么不妥当?”

楚云飞被顶得没了脾气,不过身边又有人说话了,“小姐,我倒是有兴趣知道你男朋友的电话,只要你愿意给我。”

刘宁,说这话的居然是刘宁!

那小姐显然没想到有人居然敢这么冒失,一点绅士风度都没有,惊愕过后,马上找个理由开溜,“呃,不好意思,似乎前面有人需要帮忙,我去看看。”

看着小姐狼狈地逃开,大家都以异样的眼光看着刘宁,刘宁耸耸肩,“你们看我做什么?我只不过想把我自己的角色演好就是了,当兵的时候我要成为最好的军人,做别的职业,自然也要做得最好。”

“还有,那小姐有三分姿色就可以戏弄大家,我自问长得也不差,为什么不能戏弄戏弄她?”

真是精彩,李南鸿率先鼓掌,“好,还是宁哥象个男人。”

\section{第一百一十五章 救了个活宝}

全乱套了!楚云飞心里在哀号,救人救了个活宝,现在连刘宁都发生了翻天覆地的变化。

正想着呢,成树国拍拍他的肩膀,“我想通了,刘宁说得对,既然大家都已经不是军人了,那自然也要做做符合现在身份的事情,抱残守缺其实一点意义也没有。”

恐怖分子在这希斯罗机场安的是炸弹么?怎么感觉像是安了个辐射源呢?要不怎么会所有人都发生了变异?楚云飞无奈地皱皱眉头。

还好多尼没有变异,不过也差不了很多,那个喜欢谈笑风声的开朗朋友不见了,取而代之的是个紧张兮兮,神经过敏的男人,“楚,现在我可全靠你们了啊,保护好你们未来的美圆吧。”

楚云飞脸部肌肉抽动两下,说了句自己也没想到的话,“我在考虑,是不是用英镑结算会更方便点。”

没有经过太大的波折,一行人就出了机场,毕竟大家都有货真价实的身份。

伦敦的街道很是干净,马路两边的建筑物也没有什么太高大的,那些老式建筑占了绝大多数。虽然从结构上看是拥挤了一些,但更让人感受到了这个城市的历史和繁华。

车过泰晤士河的时候,大家终于见识到了这享誉欧洲、闻名世界的名河,河水很清,河面也很宽。楚云飞和成树国不由得又想起了军校旁边那条世界第三大河,什么时候才能再见到那条河呢?

刘宁倒是没那么多感慨,可能这跟他老家在赣通省有关吧,那里多的是山和水。不过李南鸿倒是注意到了二人对泰晤士河充满了留恋的感情,马上开始了即兴演说。

“光是个泰晤士河有什么看头,回头我带你们去看白金汉宫的皇家卫队、华尔街、大笨钟、蜡像馆、海德公园,伦敦好玩的地方不少呢,不过,就是没有巴黎和罗马那么多热情的妹妹……”

话说到这里,李南鸿才意识到同行的还有两位女士,讪讪地笑了两声,不再言语。

露丝自然不会像李南鸿那样大声喧哗,事实上,在伦敦,绅士讲究绅士风度,淑女也有淑女的文静,她小声地反驳李南鸿,“你还想逛街?小心恐怖分子吧。听说恐怖分子里也有美女哦,不过大多数是寡妇。”

楚云飞对这个话题最感兴趣,因为他知道,害死他父亲的恐怖组织“基天”在欧洲似乎也很猖獗的,“你说的恐怖分子是不是爱尔兰共和军?”

露丝非常计较楚云飞的不礼貌,“楚,你在跟谁说话呀,总得有个称呼吧?”

“轰”,一行人哄堂大笑,显然在笑话楚云飞,更烘托出了那份说不清、道不明的暧昧。只有多尼像受了惊的麻雀一样浑身抖动了一下,估计是想起了什么。

李南鸿更是火上浇油,直接来了句,“飞哥,以后说话的时候记得要先说‘露丝妹妹’哦。”

楚云飞恨得牙齿直痒,“小李子,信不信我能在不知不觉中卸了你的下巴?”转头又面向露丝,“露丝,我是在问你啊,不过你要不清楚,伊琳娜回答我也可以。”

伊琳娜微微一笑,没有说话,露丝倒是没再计较,“前几年爱尔兰共和军闹得比较厉害,这几年主要是那个臭名卓著的‘基天’,不过,谁也不能保证两者间有没有什么联系。”

说笑间,机场的接送车把他们送到了市中心,大家下车。

伦敦的黄种人就明显地多了起来,虽说里面大多是日本人和阿拉伯人,但是也有中国人的,走不多远就总能看到三三两两的中国游客在谈笑风声,不过,楚云飞他们也没有同别人打什么招呼。

倒是偶尔有中国人看到这一群人里的四个黄种人,想上前招呼一声,不过人家说的都是英语,不好判断出来是哪个国家的人,自然不方便太过冒失。

李南鸿话总是最多的,看到这情形,深有感触地叹了口气,“看,人家想和咱们打招呼呢,幸亏咱们一直说的是英语。”

“幸亏?”刘宁有点理解不了这种说话方式,“我怎么听你的意思是,很庆幸没有中国人和你打招呼呢?”

李南鸿苦笑一声,“唉,你们没感觉,我可是深受其害了,要不是因为自己的同胞,我怎么可能被索度人抓住?算,这话说起来丢中国人的脸,还是不要说了,还好救我的也是自己的同胞。”

看来李南鸿被抓还有别的因素,不过,既然身边有外人,大家自然也不愿意再继续这个话题。

露丝和伊琳娜的家不在伦敦,不过,她们也没有马上回家的打算,走了不多远,大家就商量找家宾馆住下。

本来楚云飞是想找家哥特式或者罗马式的建筑风格的宾馆住住,还能多欣赏欣赏整个宾馆的造型和结构,不过露丝很明白地告诉他:那种宾馆一般人根本住不起,虽然说老旧了点,时代感差了点。但那种宾馆自身就是文物,住进去的人往往非富即贵,图的是种身份的象征,舒适度实在是差得太多了。

虽然楚云飞他们现在算是多少有点钱了,但这样花法显然不是他们愿意接受的,于是找了家格调普通的宾馆入住。

多尼强烈要求跟楚云飞同住一起,于是,大家登记了三个房间,其中多尼、楚云飞、成树国三人选了一个套间。

第二天,露丝就带着伊琳娜出去找她的朋友索菲娅去了,剩下五个男人在宾馆无所事事,于是李南鸿建议大家出去玩玩。

多尼本不愿意出去乱走,可他更不敢一个人呆在宾馆,只好随大流了。

其实多尼对伦敦的了解要比李南鸿深得多,李南鸿上次不过是随了个旅游团走马观花地转了转,哪里比得上多尼这识途老马?于是一干人玩得很是尽兴,等回到宾馆已经是下午五点了。要不是楚云飞惦记着答应露丝的事,怕是大家要看了泰晤士河夜景才肯回去的。

露丝她们早就回来了,随行的是一位美奂绝伦的少女!

那是震撼人心的美丽,楚云飞在那一刹那终于相信:这世界上真有人当得起“倾国倾城”这个极其夸张的成语!!!

\section{第一百一十六章 初见索菲娅}

那少女身高一米七零左右,体态轻盈,偏又胸丰臀挺,明眉皓齿,肤白胜雪,最要人命的还是那双微微翘起的如水双眸,天蓝色的瞳人纯真而又妖异。

少女上身穿鹅黄无领短衬衣,斜长的衬衣角在肚脐处很随意地打了个结,雪白的肚皮在傍晚的天色里显得异常地耀眼;下身穿条牛仔裤,紧绷着浑圆的双腿,那种笔直和修长,绝对可以媲美世界顶级模特的。

露丝很骄傲地介绍着自己的朋友,“这是我的好朋友,索菲娅。维伦斯,漂亮吧?”

刘宁最先从震撼中清醒了过来,摇了摇手,“呵呵,不错,真的很漂亮,很高兴认识你,索菲娅小姐,我是刘宁。”

成树国和李南鸿也先后反应了过来,同对方打了招呼。

楚云飞陷于对方的美丽,真的是有点难以自拔,他琢磨了半天,这种美丽如果用文字表达,是不是写起来会非常地难以形容?

让他清醒过来的,是少女眼中的那一点点不屑的神情,楚云飞终于反应过来自己这么看人是非常不礼貌的,老脸一红,很狼狈地打了个招呼,“你好,索菲娅,呃,抱歉,我实在是被你的美丽震惊了,请你原谅我的失礼。”

楚云飞还是嫩了点,仓促之下,这种场合居然忘记了做自我介绍。

露丝一直没有跟大家说自己的朋友有多么美丽,纯粹就是想看大家见到索菲娅时那种失态的神情,不过,楚云飞这样的表现还是引起了露丝小姐的一丝不满。

“索菲娅宝贝,这位就是我跟你说的选择先生,非常,呃,非常神奇的一个人,不过,现在显然你带给了他太大的惊喜。”

索菲娅迷人的大眼睛眨了眨,长长的睫毛掩饰不住惊奇的眼神,“天呐,露丝,你说的是这个愣头愣脑的小孩子么?”

小孩子?楚云飞脖子一伸,想说点什么,再想想,苦笑一下,算了,自己认真个什么劲?有必要么?

这时候多嘴的人就显出多嘴的好处了,李南鸿出声了,“索菲娅小姐,我就不知道你有多大,居然说我们飞哥是小孩子?”

露丝虽然不满意楚云飞刚才那呆头鹅的样子,但她也不能让自己的辛苦白费,“索菲娅,你可不能这么说,这选择先生很厉害的,而且,中国人,有时候看起来很年轻,我想他没准已经有五十岁了呢。”

就算瞎子也不会相信楚云飞已经五十岁了,不过,露丝小姐这么说,自然是有她的用意的。

“五十岁?天呐,露丝小姐,”李南鸿又说话了,“你一路上死盯着选择先生不放,我还专门给你让了座位,你会相信他有五十岁?你会对五十岁的男人兴趣那么大?”

露丝小姐的用心当场被人戳穿,自然是有点不好意思的,奇怪,人们不是都说中国人很内向的么?怎么半路跑出来个这么能说的人?“小家伙,大人说话,小孩不许插嘴!”

索菲娅奇怪地看了露丝一眼,似乎想到了什么,“哦,这样啊,就算选择先生五十岁好了,我为我的失礼道歉。”

按说楚云飞这时候应该还之以非常绅士的口气,表明自己的长相也有错的,不过楚云飞显然不是英国人,“哦,算了,那不是什么大事,对了,还是叫我的名字吧,我叫楚云飞。”

索菲娅很奇怪这个看起来有点痴呆的中国人居然这么轻描淡写地应付她,露丝看上他哪里了?怎么也觉不出眼前这人有什么魅力,“听说,听说你会治病?楚云飞先生?”

虽说索菲娅的措辞很有礼貌,楚云飞还是感觉出了对方的那种不信任,不过,这事又不是他有求于人,所以他的回答也不是很客气,“哦,我想你弄错了,索菲娅小姐,我不会治病,我的同伴倒是会点简单的战场包扎技术,呃,还有外伤的紧急处理。”

“战场包扎、外伤处理?”索菲娅显然懵了一下,然后才反应过来,目光转向她的朋友,“露丝,这……这就是你为我找到的医生?”

露丝本来就是计划拿这几个中国人充数的,不过自从楚云飞说过治好病再提条件以后,她就又多了层希望,刚才跟索菲娅在一起的时候就又吹嘘了几句,又讲了讲自己把这几个人请来是多么的不容易。

没想到楚云飞居然说出了这么不负责任的话,露丝真的有点生气了,“楚,你,你,你当初是怎么跟我说的?你说治好病才提条件的,怎么现在成了不会治病了?”

楚云飞上下打量露丝一眼,做出副很奇怪的表情,“我什么时候说过我会治病了?”小丫头,你再猖狂呀!

李南鸿见状,悄悄趴到露丝耳边说了句什么,露丝的脸“刷”地红了一大片。

李南鸿说了句什么?居然能把露丝弄个大红脸?大家正在奇怪,露丝却走上前去,在楚云飞的脸上“啵”地亲了一口,“这下你满意了吧?”

在场的人马上就是转惊为喜,最夸张的是多尼,他居然笑出了声,“哈哈,楚,没想到,你这么会要挟人啊,哈,不行了,要笑死我了,哈哈。”

楚云飞的脸也微微地红了一下,狠狠地瞪了李南鸿一眼,不好意思再卖关子了,“露丝你不用这个样子,我本来就不会治病,说到治病,你们欧洲这里顶尖医生不是到处都是么?”

索菲娅马上明白了楚云飞的意思,点点头,“不错,我们是不需要医生,我们要的是那种能把人治好的结果,巫术、秘术都行,哪怕是灵魂转移术都行。”

大家终于明白了楚云飞的意思,对啊,救人未必要医生的。

不过楚云飞又有点不爽了,“灵魂转移?说得倒轻巧,你找这么个人给我看看?我正要找这种人呢。”

楚云飞的本来意思是想同那种人交流下生死循环,生命起灭的问题,不过索菲娅显然会错了意,淡蓝的眸子里全是迷惘,“哦,选择先生,你是说,你是那种……喜欢消灭邪恶存在的人么?”

楚云飞真的拿这些人没办法,怎么就没人听得懂自己想说什么呢?他转过头去,望向窗外的天空,悠然地说道,“什么啊,我不过是想讨论些东西就是了,好了,我们不谈这个问题了。”

\section{第一百一十七章 中国菜的诱惑}

说到这里,楚云飞莫名其妙地有了种很不舒服的感觉。

在伦敦谁也没什么仇家之类的,多尼也不过就是被吓着的惊弓之鸟,所以楚云飞是相当放松的,不过,说起“灵魂转移”什么的,楚云飞的神经又被拨动得敏感了几分,于是有了这种感觉。

想了想,楚云飞转过头来,“索菲娅,你是不是带了什么人来?”

索菲娅惊奇地望了露丝一眼,发现对方眼里也全是诧异,收回目光点点头,“是啊,天快黑了,爸爸派了两个公司里的保安保护我,不过他们现在在楼梯口呢,离得很远的。”

楚云飞点点头表示知道了,不过……保安?真是保安么?怎么有种人在门口的感觉?

索菲娅当下就想让众人去她家去,因为她爷爷的病实在是不能继续拖了。

美人轻蹙眉,果真是我见犹怜,连刘宁和成树国都有点不忍心。

楚云飞不置可否地说了一句,“我们还没吃饭呢,要不是惦记着露丝小姐,我们怕是要看完泰晤士河的夜景才回来呢。”你当你是谁啊?让我去我马上就要去?

刘宁和成树国自然是知道这家伙的牛劲又犯了,这种场合,他们只能选择支持楚云飞,谁叫三个人是兄弟呢?“是啊,我们还没吃饭呢。”

索菲娅也知道自己做得太过冒失了,这么做显然没为对方考虑,好象把对方当做了家里的佣人似的。

不过,这些中国人也太没有绅士风度了,这么漂亮的小姐的邀请,他们也能拒绝?

李南鸿自然也知道风在向哪里吹,马上表示支持,“那就改天吧,好不好?你们这么一说,我觉得肚子真的有点饿了。”这家伙知道自己做不了主,连个“明天”都不敢说。

索菲娅无助地看看露丝,那意思很明白:你看你都给我找了点什么人?还不赶紧帮忙劝说一下。

露丝实在是没辙,这几个家伙可也未必听自己的,不过,好人自然是不能只做一半,又走上来“啵”了一口,“好了,拜托你们了,咱们去索菲娅家吃饭好不好?她家的厨师还会做中国菜呢。”

楚云飞的眉头不由自主地微微一皱,都是些什么人啊?你这啵一下很值钱么?值得我改主意?

看到楚云飞又有点快发飚的样子,刘宁赶紧出来圆场,“好好,那就去索菲娅家吃好了,我很久没吃中国菜了。”

索菲娅马上就眉开眼笑了,那种天使般的微笑让人目眩神迷,“没问题,就去我家吃吧,我叫他们做中国菜,我也很久没吃了。”

这种情况,楚云飞还能再说什么?

接下来的路上,索菲娅向楚云飞解释了她爷爷的情况。

原来索菲娅的爷爷今年七十三岁了,不过身体一直很硬朗,只是二十几天前忽然变得非常嗜睡,而且一睡就是十几个小时,清醒的时候精神也非常委靡。

索菲娅家的生意本来就是非常繁忙的,这下子老头就有点招呼不过来了,再说因为精神不济,食欲也受到了影响,这样恶性循环下来,没几天老人就支持不住了。

上了年纪的人是非常注意保养的,老人在发现身体不适后,第一时间就通知了自己的私人医生,医生来了以后,做了全面的身体检查,却没发现问题出在哪里。

“莫须有”自然不能算真没有,老头又去几家医院做了检查,甚至还飞到德国一趟,不过,到最后也没检查出来什么毛病,所有的专家给的建议都是:卧床静养,注意生物钟的调整和合理饮食,再加上点适当的锻炼。

到最后,老人甚至怀疑是不是自己的商业对手请人给他下了什么诅咒,这种传言如同“开膛手杰克”和吸血鬼传说一样,在英国还是有一定市场的。所以老人拼命向上帝祈祷,希望来自天堂的圣洁之光能帮他从黑暗中解脱出来。

家里索菲娅的爸爸、姑姑和叔叔也乱做一团,各显神通地给老人遍寻高人,索菲娅的叔叔甚至给老人找了个印第安的除灵师来为老人诊治,却是没什么效果。

半小时后,到达了目的地。

索菲娅的家在伦敦郊区,院落很大,足有两三万平米,不过按索菲娅的说法,这里比她家在德比郡的庄园小得多了。

下车以后,索菲娅把大家让进了她自己的会客室。

其实那会客室不是她专用的,不过是她家里一大四小的会客室里的一间,但一向少有人用,索菲娅索性就据为己有,成了她的专用。

然后索菲娅把管家喊了来,那是个将近六十岁的白人,老人身穿非常正式的燕尾服,脖子上打着领结,彬彬有理地向众人打着招呼。

让管家记录客人们想吃点什么,索菲娅带着楚云飞去了她爷爷的卧室。

倩影轻摇,曼妙旖旎,香风撩人,跟在后面的楚云飞一时间真有点眼花缭乱,再看看腰间那闪现的白皙,脑中莫名其妙地想起了一句,“云鬓花颜金步摇”。

不过一进卧室,楚云飞马上就有了一种很压抑的感觉,说不出来原因,那纯粹是一种直觉。

索菲娅的爷爷半躺在一张床上,那床是非常普通的病床,不过躺在上面的人可以自己调节床的角度,把它改变为躺椅或者沙发什么的。

他俩进来的时候,老人并没有睡着,但精神十分地不好,不过老头还是表示了对索菲娅的喜爱之情,“噢,我说是谁呢,原来是我可爱的宝贝索菲娅来了?你不知道你可怜的爷爷非常需要睡眠么?”

索菲娅上前亲昵地亲了老头一口,又捧着她爷爷的脸用自己的脸摩擦了半天,“爷爷,今天感觉好点没有啊?”

寒暄了几句,老头那昏花的老眼终于注意到了有外人进来,“哦,天呐,索菲娅你带来了一个人啊,似乎是个很不错的小伙子?”

看得出,老头还想做些很暧昧的眼色或者说动作,但是很遗憾,他似乎没那种精力。

索菲娅自然知道自己的爷爷想的是什么,白皙的脸蛋微微一红,不过,她可不敢和爷爷过分计较,谁知道还能再和爷爷这样亲热几天?“不是你想的那样啊,爷爷,这是我为爷爷找的医生,一个很奇妙的东方人。”

听说是医生,老爷子的眼光明显黯淡了下来,“噢,没想到索菲娅小宝贝也知道为爷爷操心了,东方人,是拿根长针在身上扎来扎去的那种医生么?”

\section{第一百一十八章 你饿我也饿}

索菲娅听爷爷这么一说,就知道爷爷对眼前这人没抱什么太大希望,不过也是,自己的爸爸、叔叔和姑姑,什么样的人找不到啊?他们都没办法,自己这个小孩子找的人,怕是更帮爷爷解决不了什么问题吧?

再说,爷爷对东方人那种根深蒂固的排斥,索菲娅也是知道一些的,当年姑姑同事去中国旅游,带回来个中国医生想在伦敦开中医诊所,老爷子知道了只说了一句,“那不是巫术么?夏洛蒂,我们这些上帝的子民,是不能同这种人混在一起的。”

不过,尽管知道这些,索菲娅还是不放弃自己的努力,这点在接到露丝电话的时候她就想好了,别人想笑就要他们笑去吧,为了一贯疼爱自己的爷爷,受点委屈算什么?哪怕是只有万分之一的希望自己也要做。

“爷爷,人家已经长大了,为了我心爱的爷爷,就算去找魔鬼商量,我都不会皱眉头的,何况只是个中国人?”

老头听了大为感动,费力地扭过了头,索菲娅知道爷爷想做什么,主动把脸凑过去,让爷爷亲了她一口。

“说归说,索菲娅宝贝,你要真把魔鬼找来,爷爷这条命不要也不能看着你受欺负啊,呵呵。”

楚云飞没有听他俩在那里念叨,一进门,他就四处寻找那种让他压抑的感觉,那是什么东西发出来的?

那东西肯定就在这间屋子里,不过找起来还真的很费劲,楚云飞站在那里闭上了眼睛,用自己的灵觉去感受。

找到了,就在这个位置,楚云飞睁开眼看去,却是一副油画,不知道是谁画的,看起来很古老的样子。

再进入“先天境界”,楚云飞证实了自己的猜测,那油画的框在不停地吸收着空中若有若无的生命能量,而老头身上的生命能量则只有出的份,一点看不出来有进的状态。

因为有那画框的存在,整个房间根本就没有游离状态的生命能量,难怪老头会蔫成那样。

结论虽然明了,但这个观察费了楚云飞将近一个小时,那种细微的能量交换不是一时半会儿能看出来的。

等他确定了自己的判断停下来的时候,却发现爷孙俩正看着他。

“你们这样看着我做什么?”

索菲娅很诧异地问他,“你不知道你自己已经这样站了一个小时了么?”

楚云飞点点头示意自己知道了,“哦,那真的是很失礼,不过,我想,我已经知道问题出在哪里了。”

楚云飞在“先天境界”下观察东西时,并不是只靠眼睛,所以给别人的感觉,他的目光是游离和迷茫的,索菲娅并不知道楚云飞在观察什么东西。

听到楚云飞已经找到问题的所在,索菲娅大大的眼睛里充满了惊喜,虽然这话未必可靠,“是么?楚,你果然好厉害呀,能告诉我们问题出在哪里么?”

楚云飞笑着摇摇头,“不,我觉得你爷爷对我们中国人的态度似乎是不很好,我没有理由帮他这个忙。”

“你是在报复我么?”索菲娅马上就出离愤怒了,美目含嗔,“我只不过是说了你个愣头愣脑而已,我还没计较你对我的无理呢!”

无理——那自然是说楚云飞在一开始欣赏美色的事。

老头的精神也为之一振,不知道是为了孙女受气,还是自己康复有望,“小伙子,你在别人的家里这样说话,似乎是不够礼貌的。”

老头的精神一振,生命能量显然又活动得剧烈了一点点,那画框毫不含糊把那点游离出来的生命能量据为己有。

老头没注意到身体的不适,还在说话,“而且,你似乎还冒犯了我的孙女,你知道冒犯我孙女要付出什么样的代价么?”

楚云飞斜眼瞟他一下,“哦,说实话我不知道,不过,我知道冒犯我的人会付出什么代价。”

索菲娅站在那里,眼珠转个不停,寻思半天,终于下定了决心,走上前在楚云飞的脸上“啵”地亲了一口,脸也“刷”地红了,低着头小声问道,“这样……可以了吧?”

美人含羞,那景象……真是要多动人有多动人。

虽然索菲娅连脖子都红了,可老头精神实在是不够好,居然没有注意到。

楚云飞心里也是微微一荡,这场面,实在是做梦也没想到。

虽然连遭美人吻,不过,由于主动权不是掌握在某些不解风情的人手中,那人似乎也不打算这么借坡下驴,“索菲娅,现在,呃,似乎是你冒犯我了。”

听到这话,索菲娅的脸红得快滴出血了,不过她还是勇敢地抬起了头,“怎么,楚,你不打算救我爷爷么?”说话间,那股羞意居然爬上了手臂,连小臂都有发红的趋势了。

楚云飞暗叹一声,妈的,我这是招谁惹谁了,嘴上也不由得松了下来,“好了,我先给你爷爷提提神,等他身体好点,我们再继续讨论该不该治疗的问题。”

楚云飞说完就走向了老头,老头由于刚才亢奋了一下,现在的精神更萎靡了,连有人走向他都没注意到。

楚云飞想的自然是把自己庞大的生命能量传输点给老头,他这些日子经常找些不相干的人偷偷练习这个,多少还是有点把握的。

不过,老头年纪确实大了点,能量传输起来,比一般人费劲。楚云飞边传输着能量边想,这没准就是生命存在的法则吧。

那画框也有如大肚汉进入了饭店一般,不停地吸收着楚云飞流失出来的生命能量。

索菲娅原本以为楚云飞起码也要拿根针或者把手放在爷爷手腕上什么的,没想到楚云飞不过是在爷爷面前站了一会儿,爷爷的眼神就慢慢地清亮了起来,然后,爷爷他……居然就自己从床上坐了起来。

老头坐起来,头句话就是,“我怎么全身暖洋洋的?还有,我似乎有点饿了。”

老爷子饿了!

这句话在一分钟内就传遍了整个家庭,老爷子没有吃饭的欲望已经有十多天了。于是,在五分钟内,一家人就挤满了老爷子的卧室。

此时,中国医生已经飘然离去,因为:中国医生也觉得饿了,他要回去找朋友们一起吃饭了,吃中国饭!

\section{第一百一十九章 饭前的争论}

楚云飞还说自己在老头房间里呆了一个多小时,大家没准都吃完了呢,没想到,到了现场才发现,大家都没开动呢,按露丝的说话就是:人齐了才能吃。

李南鸿手里拿个苹果在乱啃,看到楚云飞来了,赶紧抻抻脖子把嘴里那口咽下去,“露丝,飞哥来了,大家能开动了吧?”

其实楚云飞倒不是很饿,似乎三人在刚卡的艰苦岁月里把肚皮练得更加寒暑不侵了,一次吃三斤肉也不觉得撑,两天不吃饭也不会觉得饿。他离开老头房间,纯粹是让对方感受下自己的能力,同时给对方留个考虑的时间。

露丝紧盯着楚云飞的脸,嘴里心不在焉地应付着,“哦,再等等,索菲娅说她也要来吃呢。”

李南鸿实在是有点失望,走到茶几旁又拿起个苹果,“那我接着吃苹果好了。”

成树国开始打趣他,“我就不知道你在索度警察局里饿了多少天,怎么跟个饿死鬼投胎似的。”

楚云飞可没心情搀和他俩的玩笑,露丝看得他直发毛,刚才又被人莫名其妙地亲了一下,没做惯贼,心虚那是在所难免。

女人,在男人最不希望她敏感的时候,往往是最敏感的。现在的露丝就是这样,楚云飞若有若无地躲避着露丝的眼光,更加重了她的疑惑。不过,索菲娅爷爷的病情也是她要担心的,“楚,老人家的身体,你看出什么来没有?”

楚云飞终于用一种比较自然的姿势挡住了索菲娅吻过的部位,他刮刮鼻子,手肘支在桌上就不再往下放了,“哦,那个病啊,我看出点原因来。”

露丝听了心里非常高兴,自己总算是没白为朋友出力,不过,他总挡着自己的脸做什么?难道脸上有什么东西么?“楚,你再刮下去,你的鼻子该破了。”

楚云飞想想也是,露丝又不是周琳琳,凭什么管自己那么多?再说,自己也没打算背叛琳琳,无非,无非……呃,只是被美女非礼了一下,不算什么事吧?

李南鸿早把眼睛瞟了过来,楚云飞的手一放下,这家伙就大惊小怪地叫了起来,“飞哥,脸上,哈哈,你脸上还真是有口红啊。”

楚云飞伸脚就把李南鸿连同椅子一起踹倒,我脸上有口红,关你什么事?还叫得这么大声?

李南鸿在倒地的那一瞬间说完了最后一句,“……不是露丝的那种口红,啊~~~”

李南鸿显然是做了件罪大恶极的事,所有人的眼睛都盯上了楚云飞那有嫌疑的半张脸,那些道目光传来太多的内容,诧异、不可置信,嫉妒、嘲弄等等。

只有露丝看了一眼就不再看了,那眼神也不再活泼,居然有了一丝丝的哀怨。

楚云飞恨不得马上钻到桌子底下,不过再想想,似乎也没什么大不了的吧?于是赶紧岔开话题,“哦,对了,老人的病我基本上有把握治疗,不过他们对我似乎很不信任,我也就帮他恢复了一下,不想再管这事了。”

这话实在是有点震撼了,大家的注意力马上被引开了,其中以多尼最为不解,“楚,为什么?你既然可以救人,为什么不帮他们把人救好?”

“把人救好?我已经救好了啊,你们不信明天看着,老人肯定可以下地走动了,我不过是没把病源去除了而已,这个……似乎已经超额完成露丝小姐的托付了,是吧,露丝?”

露丝的眼神非常地复杂,有点点无奈,有点点失落,又有点点不理解,还有那么一丝的气愤,“楚,你真的这么认为么?治病不去根那还叫治病么?我求你了,把老人治好吧,你提什么条件我都答应你。”

索菲娅正好在这个时候走了进来,露丝的话都听到了耳朵里,她可没想到露丝深一层的用意,只是非常地感动,有这样的朋友,真好,自己以后一定不能亏待了露丝,现在,看看这个黄种人怎么说吧。

楚云飞肯定是可以听到索菲娅的脚步声的,如果他全神戒备,甚至可以感受到索菲娅的气息。不过,在伦敦,实在是没什么危险的,他自然也不会那么操心,再加上露丝的话严重地影响了他的情绪,他居然没注意到索菲娅进来。

“我可没说不去根,我只是觉得他们似乎对我的信任度不够,所以很不高兴,他们应该表示出该有的诚意才对。”

露丝真的有点不高兴了,“我知道了,你还想你要提什么条件,你真的有那么贪心么?还是对索菲娅抱着什么想法?”

李南鸿可是注意到那个绝顶美人来了,话也跟了上来,“呃,露丝,请原谅我的失礼,我觉得,你对飞哥才是抱着什么想法呢。”

果然好一张臭嘴,这句话一出口,就惹恼了三个人。

楚云飞很是气愤这家伙在大庭广众下把事实说出来,你都这样讲了,以后我还再怎么装聋做哑?要不是索菲娅来了,他真有心思再给这家伙一脚。

露丝被人戳穿心思,真正有点恼羞成怒了,不过她自问对楚云飞还没有什么“非君不嫁”的念头,所以,当她发现索菲娅出现的时候没有继续这个话题,也省得欲盖弥彰。

索菲娅也被激怒了,露丝好心把人请来给自己帮忙,却没想到被这个小子做如此评价,中国人的绅士风度还果真是欠缺,不过,那点怒火是不是有别的原因,索菲娅小姐也没去深究。

怒归怒,索菲娅也知道这几个中国人是一起的,自然不方便厚此薄彼。于是强压下那点不愉快,微笑走了过来。

多尼的绅士风度是在法国养成的,虽然严谨是谈不上,但又多了份体贴,他马上站起身来,帮索菲娅把……楚云飞身旁的椅子拉了出来。

多尼的想法自然是没有错的,主人想陪有功之人用餐,自然是要离得近些,也好就近讨论讨论下步该怎么治疗的问题。

道理谁都懂,不过,目前话题下,这种安排,是不是有点惟恐天下不乱的感觉?

索菲娅只能礼貌地道谢,然后坐下,所有人都有点不太自然的感觉,现在,只能把话题引到主线上了。

楚云飞刚要张嘴,李南鸿又抢着说话了,“现在,是不是,可以上菜了?”

\section{第一百二十章 不欢的晚宴}

就这么短短几天里,楚云飞已经记不清楚自己曾经多少次后悔救了李南鸿,不过,他发现这家伙有时候,有时候似乎还算满可爱的,起码现在就是,虽然这个麻烦似乎也是他引起的。

索菲娅点点头,“好啊,上菜吧,我真有点饿了,要你们等那么久,真不好意思。”

那老管家在索菲娅出现的时候也神不知鬼不觉地出现,听到吩咐,马上在墙上的一个按纽上按了一下,然后向大家聚了个躬,悄然离去。

看到管家卓越的表现,楚云飞不由得又想起曹婶,天底下的管家看来似乎都差不多,不过自己什么时候才能再见到那个对自己关怀有加的善良老人呢?

露丝的情绪很快就稳定了下来,开始关怀地问好友,“索菲娅,你的爷爷,听说是好多了?”

索菲娅点点头,“是啊,好了很多,不过他躺在床上时间有点长,肠胃也不太好,精神倒是强了很多,楚的治疗,效果很明显。”

成树国终于知道了露丝的哥哥为什么那么听妹妹的话了,露丝在接下来的话里充分地显示出了她的刻薄,“选择先生,似乎老人并没有像你说的那样痊愈,你刚才的话似乎有点夸张了,我一向以为那只是美国人的习惯呢。”

楚云飞倒是很欢迎这样的语气,起码他知道该怎么躲避和还击,不像那些涉及到情感的东西,根本是没个解决办法,“哦,露丝,我想你没听明白索菲娅的意思,我的理解是,她的爷爷已经完全好了,只不过病了这么长时间,恢复需要个过程的,你说我说得对么,索菲娅小姐?”

说话间,有佣人把一道道的中国菜端了上来。

索菲娅盛情地招待大家动手,同时借这个工夫把头脑里的思路梳理了一番,开始回答楚云飞的问话。

“楚云飞先生,可能你说的是对的,但是我不是很清楚,也许明天早晨起来,看看我的爷爷的情况更容易做出判断,大家觉得我的想法合理么?”

“合理,当然合理,”接话的自然是李南鸿,“我觉得索菲娅小姐的话再正确不过了,不过,这个鱼香肉丝里……怎么会有芥末呢?”

没人愿意去理会这个疯疯癫癫的家伙,索菲娅也不例外,她把话题转向了露丝,“露丝,这次真的谢谢你了,对了,还有伊琳娜,你们居然为我的爷爷请到了这么厉害的中国人,真是辛苦你们了。”

有这句话就足够了!伊琳娜不爱说话,笑了笑算是回答,露丝只能停止咀嚼,回答主人的话,“哦,宝贝,咱俩之间不用这么客气吧?再这么说,我可要生气了哦。”

索菲娅接下来的话就让露丝失去了开玩笑的欲望,“这次麻烦你们在非洲呆这么久帮我找人,一定影响到你们的工作了吧?要不你们先回德比吧,等到圣诞节的时候我一定回去看你们,对了,还有你的色狼杰瑞哥哥。”

露丝登时就有了种过河抽板的感觉,不,简直是过河抢渔妇老公的感觉,不幸的是,那个倒霉的渔妇就是她本人。

虽然露丝和楚云飞根本就没认识几天,也没有什么深的交情,不过,有种人就是这样,他不需要刻意地表现自己,就能自然而然地锋芒毕露,纵然是露丝这种惯爱戏弄男生的主,也因此不由自主地陷了一只脚进去。

女人在什么时候思维最敏捷?就是这种时候!露丝马上就为自己找了个完美的理由,“那怎么可以?索菲娅宝贝,人是我介绍来的,我自然要为自己的行为负责,也省得爷爷好了以后埋怨我,至于工作,那就放放吧。”

索菲娅真的很奇怪自己的朋友为什么这么仗义,不过扭头一看低头大吃的楚云飞,似乎又想到了什么,“好啊,露丝,你要能在这里多呆几天就更好了。”

楚云飞边吃边愤愤地想:为什么在英国这个号称最讲究礼节的国家里,居然没有“食不言,寝不语”的规矩呢?

露丝真的为自己的朋友操起了心,或者说她根本就不是很饿,歇了没两分钟,又开始指责楚云飞,“楚,我对你这么好,你为什么不帮人帮到底,把爷爷的病根去了呢?”

你对我这么好?拜托了,别开玩笑好不好,你哪里对我好啦?楚云飞也不得不停止咀嚼,为自己辩护起来,“这个,我已经说了,他们对我不是很信任,我也觉得没必要再进行下去了,一来是我即将做的工作会有点匪夷所思,或者说你们根本就不会认同,二来,二来……”

“二来就是,你想提些很过分的要求是不是?”露丝早在这里等着他了,“真是白让我对你那么好了。”

楚云飞当场就爆发了,熟归熟,你乱说我一样会告你诽谤的,“露丝,啊不,露丝小姐,你是对我不错,不过是帮我出了路费而已,信不信我马上把路费还你,咱们各走各的?”反正签证早到手了,人也治好了。

不过,这世上的事还真是怪得离谱,多少男人细言软语,根本打动不了露丝小姐的铁石心肠,而楚云飞这般地恼羞成怒,露丝居然不再言语,低头吃起了她的饭。而女士的内心深处,这个男子的形象居然又高大了几分,实在是奇怪得很。

李南鸿虽然言语冒失,其实也是个很聪明的人,出于义气,他很想跟着“飞哥”拍桌子走人,不过,索菲娅家的富贵实在是给他留下了很深的印象,这种势力不攀附,那是要后悔很长时间的。

“飞哥,这个……你的条件还没提,怎么知道人家不会答应,还有,呃,露丝小姐,她对你确实不错,我看到的她就亲了你两次呢。”

楚云飞越发地怒火中烧,不过李南鸿说的也是实情,他这腔怨气还真是没地方发泄,作为个男人,他总不能刻薄地说:谁让她亲我了?

再想想自己被吻也是面前这小子出的鬼主意,楚云飞越发地后悔救了这个人,这么想着,他的怨气居然消了不少,龇牙一笑,“小子,你狠,咱们,呵呵,有算帐的时候。”

李南鸿从没见过楚云飞这么恐怖的表情,一时间全身发凉,原来这世界上有些笑,要比板着脸恐怖多了。

还是索菲娅为大家摆平了这尴尬场面,“楚,我决定了,等明天爷爷的身体好了,你提的条件,只要不是非常离谱,我都会满足的。”

\section{第一百二十一章 雨中的伦敦}

楚云飞可真的有点郁闷了,“我说,你们对我都是什么印象啊?我都不知道你家能满足我什么条件呢,就好象我一定要做恶人一样?没准我只想收十英镑的出诊费呢。”

十英镑,只不过是楚云飞随口说说,他虽然用了很少的一点生命能量,但以其珍稀的程度,怕是按十万英镑那个数量级他也未必想卖的,这么说不过就是为了争口气。

他想争口气,有人可不这么想,索菲娅皱皱眉头,“楚云飞先生,请恕我冒昧,如果仅仅是钱的问题,那再多都不会对我造成困惑,而且,我不认为维伦斯家族的人这么不值钱,十英镑,你的话实在是很伤害人。”

楚云飞满以为出了非洲,有个正常身份,可以随心所欲地做些事情了,没想到事到临头,居然还是有点碍手碍脚的感觉,而面对佳人,实在难以兴起唐突的念头,再说钱多也不是坏事,只得微微笑笑,“抱歉,我实在不是有意冒犯的,好了,现在我似乎有点困了,要回去休息了。”

索菲娅自然不好拦着人家回去,她倒是有心留宿对方,虽然性别有差异,但楚云飞所立的功劳是足以当得这份殊荣了,只需要叫自己的哥哥出面挽留就好了。

但实际上,宾主之间的沟通实在是谈不上愉快的,而且这一大票人实在也多了点,对方既然执意回去,那就由得他们好了。

该叮嘱的还是要叮嘱的,索菲娅自然不可能忘记正事,“那好吧,楚云飞先生,我知道您忙碌了一整天,就不耽误您休息了,明天早晨我会派车去接您的。”话说完,还掉头看看露丝,想知道好朋友是不是要留在这里陪自己。

露丝早就恢复了正常,“好了,索菲娅,时间确实不早了,楚怎么也是我请来的朋友,我得陪着他回去了,咱俩改天再聊吧。”

索菲娅想想也是,正好让自己的朋友劝劝那个毛头小子,省得让他明天再为难自己,到时候那可就不太好看了。于是喊来了管家,安排了司机送他们回去。

李南鸿果然是饿死鬼投胎,没吃过瘾,居然突发奇想,想把没吃完的东西打包,楚云飞瞟他一眼,“哦,似乎你对加了芥末的鱼香肉丝很感兴趣,就打包那个吧,多了就让人笑话咱们了。”

李南鸿自然品到了楚云飞的杀气,“呃,那算了,索性就不要打包了,也免得人家笑话。”

回到宾馆已经是九点多了,大家都累了一天了,各自回房休息,李南鸿本想找“飞哥”问问今天他是怎么救人的,不过看楚云飞脸色不善,终于老实地回房间了。

露丝差点失眠,她躺在床上,翻来覆去地琢磨楚云飞,这个东方人带给她的不仅仅是神秘的感觉,还有一点点的不羁,一点点的霸气,一点点忧郁,连那象征着黑暗的瞳仁都充满了奇异的诱惑力。

留给她的时间显然不多了,两人在索度邂逅,没过几天就来到了英国,等到索菲娅的爷爷治好以后,他要继续他的旅程,她也要回去重复她的平淡生活。就这么分道扬镳么?世界这么大,再见的机会怕是永远都不会有了。更可气的是:他身边永远地有那么几个人,两人根本没有单独相处的机会。

走就走吧,不过是生命中萍水相逢的过客;不行,这么精彩的生命体验,不能就这么白白放弃!这两个念头不停地在露丝的脑中斗争,可这事还不好意思同伊琳娜商量,过了很久,她才不知不觉地进入梦乡。

第二天是阴天,下起了小雨,秋雨中的伦敦,凭添了几分萧瑟的感觉,也显得格外地古朴和厚重。

多久没见过下雨了?有半年了吧?楚云飞他们几个一起床就兴奋了起来,细细的雨丝、低霭的阴云根本压制不住他们心中的那份雀跃,老天,总算可以过过正常人的生活了,在他们眼中,每一滴雨珠都显得那么弥足珍贵。

看着街道上三三两两跑步晨练的人,楚云飞和两个战友也打算出去锻炼一翻,小雨中练练,应该是很惬意的事吧?李南鸿则在那里蒙着头大睡。

多尼也很为这场雨感到兴奋,不过他没有出去锻炼的打算,只是想推开窗户呼吸一下清新空气,但推窗户的手在下一刻停在了空中,“天哪,白色克莱斯勒商务车,那个索菲娅家里的司机不需要睡觉的么?”

几个人对视一眼,还是楚云飞说话了,“管他呢,咱们锻炼咱们的,他们愿意等让他们等着吧。”

其他人都知道楚云飞对索菲娅家有小小的成见,成树国和刘宁更在猜想这没准是不良士兵的要价手腕,于是大家下楼,向着宾馆斜对面一个极其袖珍的小广场走去。

没想到刚出楼门,那车里就下来了一个超级肥胖的人,五官也因为肥胖而挤压得变形,只剩下一个硕大的鼻子还在尖挺着,头上似乎还是戴着假发。

那胖子慢慢地向三人走来,看得出来,他在保持风度的同时已经是尽量加快步伐了,不过,他实在是有点太胖了,大概有将近三百斤了吧?

胖子很和蔼,气喘吁吁地向三人打招呼,“嗨,大家早上好,请问,是楚云飞先生的同伴么?”

楚云飞看看同伴,两人脸上都是那种“我们只管看热闹”的表情,只好笑答对方,“早上好,我就是楚云飞,请问你是?”

胖子也没感到意外,臃肿的脸上费劲地挤出了一丝微笑,不过那善意是表现得十分清晰,“你好,楚云飞先生,很高兴见到你,我是索菲娅的姑父,宾塞斯。维伦斯先生的女婿,考林斯。斯文森。”

“哦,斯文森先生,很高兴认识你,我们正要出去晨练呢,要不要一起去?”楚云飞虽然对眼前这人的印象不错,不过,他也不是个轻易就愿意改变初衷的主。

叫我晨练?我这个样子能晨练么?斯文森明显地愣了一下,不过他本来就是个心胸开阔的人,也没想对方是不是在讽刺自己,笑着点点头,“我锻炼是不行了,不过倒是能陪你们走走。”

\section{第一百二十二章 你们猜猜看}

楚云飞话出口就觉得自己冒失了,还好对方根本没介意,这么一来,楚云飞对这胖子的好感顿时大增。

三人去那小广场草草地练了练军体拳就回了,毕竟是有客人在的,不能太失礼。

斯文森虽说看起来忠厚老实,但却见多识广,眼睛毒得很,从三人下意识的整齐步伐和拳路中,很快就分析出了对方的身份:眼前这三人做过军人。

楚云飞他们一练完,斯文森就鼓起掌来,“不错不错,三位先生以前,做过军人吧?”

“哦?这话怎么说呢?”刘宁的兴趣被勾了起来。

等到斯文森把自己的理由说完,再加上那句“标准的军人”的评价,他明显地感觉到了:自己已经成功地引起了三个中国人的好感!

于是,当三个中国人邀请他共进早餐的时候,他毫不犹豫地答应了,根本提也没提去岳父家里享用早餐的建议,虽然那里准备的早餐比这里的丰盛得多。

造成这种情况的原因很简单:李南鸿不肯起床,任谁叫也不起,“让索菲娅她家人吃了我那份好了,让我再睡会儿。”

楚云飞看看盛装打扮的露丝和伊琳娜,感觉脸上分外地挂不住,“你们先吃,我去叫他。”

没想到那李南鸿有如死狗一般,楚云飞把他踢到地上,还在那里闭着眼睛哼哼,“飞哥,飞哥,我就再睡一会儿,就一会儿,十分钟还不行么?可怜我在索度那破地方就没睡过个囫囵觉啊。”

面对这种情况,楚云飞也实在是懒得理他。还好,他总算是赶上了去索菲娅家的时间。

车里,李南鸿还是那昏昏欲睡样子,还好其他人都是精神十足,没让他再有机会丢中国人的脸。

刘宁看着窗外的风景和建筑,随口问道,“斯文森先生,您的岳父,宾塞斯。维伦斯先生是做什么的?看起来很富有的样子。”

斯文森可是不愿意说得太详细,不过,不回答也是不礼貌的,于是这老实人反问了一句,“你们认为呢?”

成树国先猜了,“肯定是什么企业的老板吧?没准是个大家族的继承人呢,反正你们英国……贵族多得数不过来。”

露丝直接否定了成树国,“不是的,维伦斯这个姓,在英国不是很有名的姓,前100名里面绝对没有。”

成树国很纳闷,掉头面向露丝,“你这话什么意思,难道说,前一百名的姓氏,你都记得住?你有这闲功夫?”

露丝白了成树国一眼,“那是自然,只要有点上进心的未婚女士,相貌又差不多的话,谁会记不住这些?那是通向上流社会的捷径,不过,这种事,你们这些中国人是不会理解的,贵族……那是多么有诱惑力的称呼啊。”

现在的刘宁变了很多,似乎有比成树国更“愤青”的倾向,“你这话,什么意思,是说中国人不够开化么?也许你不知道,在两千年前,中国的贵族就衰败了,那时的称呼是‘士族’,那个时候,怕是你们的祖先还在茹毛饮血吧?”

李南鸿虽然还在迷糊,但好不容易有了插嘴的机会,自然要把自己知道的卖弄一下,“对啊对啊,我还记得书上说,有个士族宁肯活活饿死,也不愿意跟平民里很有钱的人通婚。”

这争论上升到了民族的高度,眼看就要变得不可调解,斯文森马上出来解围,“哦,我想露丝小姐不是那个意思,事实上,就象赛马一样,有些英国人还保持着可笑的血统论。在外人看来很不理解,甚至很多英国人都不认同,但毫无疑问,这个观点确实还是有它的市场的。好了,不说这个了,你们继续猜吧。”

楚云飞张嘴了,“我记得露丝说过,维伦斯先生在华尔街的能量很大,华尔街,那是美国的金融中心吧?作为英国人,在美国的金融中心能发挥巨大威力,这不是很难猜吧?”

这种场合,楚云飞没有很见外地称呼“露丝小姐”,而是很亲昵地叫了声“露丝”,当事人心里不禁暖洋洋地,升起了一份不足为外人道的甜蜜感觉。

斯文森点点头,“你说得不错,而且露丝小姐的话也说得很对,那你能告诉我你的猜测么?”

李南鸿又抢答了,“那还用说?肯定是世界知名企业的控股人了,而且,肯定是世界前五百强的,啊不,前五十强的,维伦斯先生,应该是董事长才对。”

斯文森笑了笑,虽然他的笑容很不容易被人看出来,“哦,这是,这是李先生的猜测,还有补充的么?”

多尼说话了,这家伙既然肯冒充金融专家,在金融方面自然是要比其他人多了解点的,“企业的控股人?那倒也未必,作为实业,想要影响华尔街,那可绝对不是一般的企业,世界前五十强都未必,起码要前三十或者前二十强才行,不过要是轻微的影响,那五十强也足够了。”

“不过,我觉得维伦斯先生更有可能是什么投资基金的负责人,大型的投资基金,那能量比企业还要大,全部的现金流,想想都足够可怕了,这种东西运作好的话,吞掉比自己资产大十倍的企业都没问题,我个人更倾向于这个判断。”

楚云飞也懒得再猜了,因为他隐隐觉得大家的猜测似乎都不是很正确,“好了,斯文森,你该揭谜底了。”

斯文森憨厚地一笑,“很遗憾,你们全部猜错了,不过,事实的真相我是不方便说的,你们还是等等自己问维伦斯先生吧。”

楚云飞不满意地皱皱眉头,“对了,露丝,你当时是和我们怎么介绍的?你知道维伦斯先生是做什么的么?”

露丝看看楚云飞,甜甜一笑,“我真的也不知道索菲娅的爷爷是做什么的,我只知道,她的爷爷跺跺脚,华尔街也要抖一抖的。”

多尼是初次听到这话,沉吟良久,身体猛地一抖,眼睛茫然地望向窗外,嘴里喃喃自语,“原来,原来是这样。”

李南鸿很纳闷,“多尼,你猜到了吗?怎么一副很惊讶的表情?”

多尼下意识地回答,“其实,其实我早该想到了,除了华尔街的黑手党,谁还有这个本事?”

\section{第一百二十三章 老头维伦斯}

华尔街的黑手党?这名字倒是很新鲜,楚云飞不由得纳闷起来,黑手党,那不是意大利的么?怎么和伦敦挂得上钩?居然还是混在美国华尔街?

可斯文森的反应绝对证实了多尼的猜测,他明显地愣了一下,然后扭头看向多尼,“呃,这位先生,似乎,你对这种事情相当了解?”他的心思全在中国人身上了,根本没记多尼的名字。

多尼点点头,还是皱着眉头,“你叫我多尼好了,事实上……事实上我很爱交际,朋友也很多,这种事情,我也只是偶尔听朋友们提起过。”刚才话一说出口,他就后悔了,因为,似乎这种势力不太愿意自己被外界知晓,会不会是犯了个错误呢?

见到多尼局促不安,楚云飞拔刀相助,“嗯,这个我可以证明,多尼的朋友非常多,而且里面有几个身份是相当厉害的。”

这是楚云飞的一种怪癖,对于朋友或者说相处得时间比较长有了感情的同伴,他总是不吝惜赞美之辞的,甚至夸大点也在所不惜。而说到自己,他倒不愿意轻易地夸口,最多也就是不言语,让别人自己去猜测。

楚云飞把这种行为理解为讲义气,事实上,通常也能达到维护朋友的效果。

倒不是斯文森过于敏感这事,实在是,像这种势力,那只是传说中的存在,是相当神秘的。原因也很简单,被外界知道,引起人心震动是难免的。更值得忧虑的是,这其中涉及非常微妙的平衡,窗户纸一旦被捅破,等待他们的也许就是灭顶之灾或者两败俱伤。

于是斯文森有意识地把话题岔开,“呵呵,我个人对那些事情是比较感兴趣的,多尼先生如果有兴趣的话,还请空闲的时候多给我讲讲。哦,快到地方了,见了维伦斯先生,你们可以自己问他的职业的,尤其是楚,维伦斯先生可是很感激你的。”

虽然斯文森本人是厚道的,但他好歹也是四十多岁的人,在这个圈子里混,江湖险恶也是见识过的,所以说起话来也算是比较稳重的。

不过,在座的很有几个聪明人的,自然能看出和听出些味道,既然这个话题不方便继续,那就说点的别的好了。

一点点的小雨不会影响车速,一行人很快地到达了索菲娅家。

这次进的是大客厅,客厅里已经闲坐了三个人在里面,经斯文森先生介绍,大家才知道,戴着眼镜的身材颀长的男士是索菲娅的父亲,班克斯。维伦斯,另一个个头奇高,身材中等的中年女士是索菲娅的婶婶劳瑞,还有一个身材魁梧的年轻人是索菲娅的二哥达克。

看得出来,班克斯对楚云飞非常地感兴趣,“呵呵,楚云飞先生是吧?这次真是辛苦你了,说实话,真没想到中国人里能有医术这么精湛的专家,难得你还这么年轻,我真的很喜欢你。”

长者赐,不敢辞,楚云飞受中国传统文化影响还是满深的,再加上斯文森的豁达做派,心里芥蒂也不是那么很重了,也是微笑着客气,“哪里哪里,班克斯先生的称赞,我担当不起,只是一时侥幸。”

不过,索菲娅的二哥达克对楚云飞的似乎不太友善,用那双明显外凸的大眼盯着楚云飞,表情严肃,心里不知道在想着什么。

知道楚云飞他们来了,等待许久的维伦斯先生也来到了客厅,老头的精神和昨天纯粹不可比较,起码目前是当得起“矍铄”两字了。

老头是自己走来的,一进客厅就直奔楚云飞,看来昨天他的头脑还是很清醒的,认人水平不差。

楚云飞忙不迭地从沙发上站了起来,再怎么有意见,人家也活了七十多岁,该有的礼貌还是要有的,其他人也纷纷起立。

老头一把拉住楚云飞的手就来了个拥抱,“哦,楚,你可算来了,我等你好久了,为你救了我这条老命,呵呵。”

楚云飞虽然想得到对方肯定要态度热情些,却也没想到这老头变脸比翻书还快,在拥抱老头的同时,禁不住想起了“川剧”。

老头似乎猜到了他的心思,用手在他背上狠狠地拍了两下,居然有些疼痛的感觉,老家伙还笑嘻嘻地解释,“看我,现在身体多棒,可多亏你了。”

俩人还没放下手臂,就听得脚步蹀躞,香风扑鼻,楚云飞眼角淡绿一闪,却是索菲娅来了。

索菲娅今天穿得很是正式,也许是要把眼前这事很郑重地办一下吧,身穿淡绿色长裙,不过没有裙绷的那种,上身穿淡紫色紧身小马甲,把曲线玲珑的身姿展示得淋漓尽致,肩头披小小一块天蓝轻纱。整体偏冷的色调,偏偏里面是个娇艳如玉的青春少女。

即使感情麻木如楚云飞者,也不得不承认眼前这少女打扮得实在是太成功了,视觉效果真的不错。

索菲娅化的妆很淡,其实,青春就是最出色的画笔,那种清新和活力一览无遗。

“楚先生,你怎么现在才来呀?我和爷爷等你半天了,先去吃早餐吧。”

楚云飞心里不得不再次迁怒李南鸿,看看你这家伙,给人家造成的是什么印象,一来先说吃饭,他可就偏偏忘记了自己昨天两次把吃饭做为推脱借口的。

“呵呵,谢谢了,索菲娅小姐,我们,我们已经吃过了。”这话说得楚云飞自己都有点不好意思,似乎有自己也很惦记吃的那种感觉,“那个……斯文森先生和我们一起吃的。”

索菲娅冲着斯文森撅撅嘴,“姑父,你就知道自己吃了,也不通知我们一声,我还饿着呢。”

看得出索菲娅平时就很得宠,这次又立了大功,斯文森显然要有麻烦了。

还是老头出面解了围,“好了,索菲娅宝贝,别闹了,刚才考伦来电话说过了,不过那时你在化妆,”说到化妆,老头眼里冒出一丝狡黠,嘴角也露出了笑意,直盯着自己的宝贝孙女,足足停了半分钟,才继续说,“我们一直没告诉你,大家都吃了饭了,就你没吃,呵呵。”

索菲娅的俏脸又是微微一红,她自然知道自己一直呆在房间里,不停地试衣服。至于爷爷的有意停顿,她更明白里面的意思:这个疯丫头什么时候变得讲究起来了,以前不是一直说“年轻就是美”么?

\section{第一百二十四章 这是诅咒}

其实为什么要精心打扮,索菲娅自己也不是很清楚,她曾经问过自己:你准备好了么?要喜欢那个色迷迷的中国人?答案自然是否定的,虽然楚云飞后来的表现证明他其实不是个色鬼,但人的第一印象,真的是很难改变。

再说,好友露丝似乎也很着紧那个家伙,索菲娅虽然不至于为了友谊就要放弃值得追求的幸福,但显然现在幸福还没有出现,眼前这个中国人绝对没有那么大的魅力。

那花这么长时间打扮是为什么?索菲娅想来想去,自己都想不明白,不过,爷爷的调笑又轻轻地触碰了一下她的神经。

老头也只是想调笑孙女一下而已,看到索菲娅有点局促,微微一笑,就把话题岔开了,“宝贝,这次可多亏你了,爷爷生了三个儿女,没想到能救爷爷的只有你,爷爷真是没有百疼你啊,索菲娅,呵呵。”

维伦斯家里人显然见惯了老爷子对索菲娅的宠爱,一时间,所有人脸上都是笑意盈盈,并没有因为老头贬低大家而气恼。

索菲娅被爷爷夸得脸色发红,轻笑一声撒娇,“爷爷,你还有客人啊。”

“哦,对了对了,”老头转头面向楚云飞,“谢谢你了,楚先生,这次索菲娅把你们从非洲请到欧洲,真的是麻烦你们了,现在我宾塞斯的身体也好了,有什么要求你尽管提好了。”

楚云飞最是吃软不吃硬,他能为两句对中国人不够恭敬的话,同老头和索菲娅置气,但混吃混喝的李南鸿捅再大漏子,只要可怜兮兮地叫几声“飞哥”,他还真没脾气。

所以,宾塞斯客气地要楚云飞自己提条件,他还真不会提了,想了半天,才微微一笑,说了一句,“没关系,我也是碰巧会治这种病,实在不是什么大不了的事。”

楚云飞在那里谦虚,老头不干了,调侃了起来,“对对对,我这老家伙的生死对你确实没什么大不了,对我关系可大了,哈哈,你的谦让,我很欣赏,不过对我来说,显然不能容忍。”

楚云飞和同伴们也哑然失笑,这老头还真有几分幽默。既然你不喜欢谦让,那就不用谦让了。

“那我就直说了,维伦斯先生,其实这次我们能来,主要是接受了露丝的邀请,她为了表示诚意,为我们代办了签证,还出了路费,虽然这点小事我们不会放在眼里,但不得不说,你的孙女有个很好的朋友。”

露丝本是坐在离得比较远的位置的,因为,她实在没能力参与下步的事了,不如坐远点,等着索菲娅一会儿来找她聊天。

虽然门外就是连绵的秋雨,这话入耳,露丝浑身都是暖洋洋地,说不出的甜蜜涌上心头,现在她眼中的楚云飞,实在是要多顺眼有多顺眼,这个无赖,怎么原来就不知道他这么会说话呢?

宾塞斯点点头,“这个我知道,索菲娅跟我说了,露丝小姐是吧?等等我自然有我的礼物送给她,维伦斯家,从来不会让真正的朋友吃亏。”

老头的话说到这步,露丝小姐完全实现了她在这件事中想要达到的终极目的,不过,现在的露丝小姐,心思已经不仅仅是这个了,她刚刚做出了点很私人的决定。

楚云飞接着说了下去,“事实上,维伦斯先生,我并不是一个医生,虽然我的同伴勉强可以算个蹩脚医生。我只能说,这件事,纯粹是个巧合。而且我敢夸口,这个世界上,或许还有一些人能治好你的病,但能使你康复得这么快的人,据我所知是没有的。”

老头的话暗示了他的诚意,楚云飞的话,也表明了自己的价值。

老头能管理这么大的家族那么多年,智力是绝对毋庸质疑的,他思考了一下,“你是说,我这次得的……那不是病?”老头已经早有这个猜测,眼下无疑是问询的机会。

楚云飞点点头,“是的,严格地讲,我也不知道那是什么原因,但是我知道为什么,如果不得不说的话,我想,可以把这种情况解释为……一种诅咒吧。”

诅咒?在场的人听到这句话,全部泛起一股毛骨悚然的感觉。

阴谋,这一定是阴谋,针对宾塞斯。维伦斯先生的暗杀,可是,这事会是谁干的呢?维伦斯家族横行华尔街这么多年,可能的仇家实在是太多了,再说,谁又能保证,这事不是家族内部的人做的呢?

只有成树国和刘宁对视了一眼,嘴角露出了一丝不引人注目的微笑,都是一副“事情果然如此”的模样。自家人知道自家事,云飞救人,那绝对是用非常手段的。

李南鸿可不知道楚云飞的神奇之处,震惊一过,见身边两人眉来眼去,说不出地暧昧,忍不住小声地问了句,“你俩做什么呢?很神秘的样子。”

他声音是不大,可现在大厅里静得连掉根针都听得见,几乎所有人都扭头向发出声音的地方望去,却见三个中国人一脸暧昧的表情。

索菲娅的二哥达克早就看楚云飞不顺眼了,至于原因,达克少爷讨厌某人,需要原因么?

不过,对方是爷爷的救命恩人,他也不敢造次。

而且,达克毕竟年轻,听到“诅咒”一词,震惊是难免的,不过震惊一过,自然要怀疑其真伪,待到看见三个中国人鬼鬼祟祟的样子,他终于按捺不住怒火了。

“楚,我知道你们中国人号称很神秘,最多就是无聊的人多点吧,诅咒?可笑,这么荒唐的理由你也编得出来,你以为维伦斯家族里全是白痴么?你要明白,欺骗维伦斯家族是要付出代价的!”

老头有点火了,这个达克,怎么这么说话?不过,达克的话逻辑非常清晰,老头多少是考虑了一下,做出决定的速度就慢了点。

楚云飞哪里肯吃这套,他把手里的茶杯向桌上随手一放,不屑地笑笑,慢条斯理地说,“维伦斯家族?听起来很厉害的样子,不过,说句实话……”

“在我眼里,你们什么都不是,与我为敌,你们,还不配!”说罢,还轻轻地摇了摇头,一脸无奈的样子。

痛快!终于把昨天老头他们涉嫌侮辱中国人的气重重地出了。

\section{第一百二十五章 话不投机}

成树国和刘宁听了这话,虽然表面上还是那副悠哉游哉的样子,不过已经暗自做好了应变的准备,刘宁居然还悄悄地掐了一下李南鸿暗示他做准备。

达克侮辱中国人,两人自然是不爽的,不过他们真没想到,这次居然是楚云飞先发飚了,哪还有不支持的道理?

这时露丝的眼里已经全是小星星了,别人眼里势倾一方的显赫家族,在他嘴里轻描淡写地成了小爬虫。什么叫男人?这才是真正的男人!

多尼听得差点没掉到沙发底下去,他刚在为“诅咒”一词而震惊,继而又因这话琢磨起楚云飞的真实身份,可没想到,三言两语间,楚云飞就傲气地把维伦斯家族踩在了脚下,老天,那可是称霸华尔街的黑手党啊。

维伦斯家族这里几个人已经气炸肺了,达克暴跳起来,刚要开骂,却听得老头一声怒吼,“达克,滚出去!!!”

达克听到爷爷真发怒了,不敢再说什么,冲着楚云飞怒哼一声,走了出去,脚步跺得震天响。

楚云飞眼皮都没抬,又端起了茶杯,慢慢地啜了起来,那么大个人,在他眼里就像根本不存在一样。

老头发怒了,其他几个维伦斯家族的人也不敢胡乱张嘴,过了半天,才听老头说了起来,“抱歉,楚先生,小孙子缺乏管教,让你们见笑了。”

楚云飞笑笑,轻轻地摇摇头,“呵呵,无所谓,年轻嘛,张扬一点,也不是什么大不了的事。不过,我觉得,过头的话,还是少说点的好,免得自己还不知道,就为维伦斯家族惹祸了。”既然撕破了脸,楚云飞说话就不会太客气。

老头越发地生气了,作为维伦斯家族的族长,他是绝对不允许有任何人轻视自己家族的,哪怕自己这条老命不要,也要维护家族的尊严,“哦,楚先生这话,我似乎不太能听得懂,能麻烦你解释一下么?还有,我想顺便问下,不知道楚先生今年多大了?”

既然老头说话阴阳怪气,楚云飞也不会有多大气量,“哦,我觉得,无故招惹我,那就是给维伦斯家族惹祸,至于我的年纪,呵呵,差不多22岁了吧。”

老头真被噎得不轻,“哦?那你也还算年轻嘛,你真有那么厉害么?不过,看在你也年轻的份上,我学学你,不计较你的张扬,不过,年轻人,过头的话你也少说点,维伦斯家族是你招惹不起的。”

楚云飞一直在笑吟吟地听着,等到老头说完,他的神色里又多了丝嘲弄的味道,“是么?维伦斯先生,我实在不这么认为。看看你的病一直找不到人治你就该明白了,这世界上,有太多的东西是不为人知道的,我承认我不太了解维伦斯家族,而你,又了解我多少呢?”

老头这才想起,同自己说话的这个人,实在是有太多神秘的。这种存在,如果能不招惹,那还是别招惹的好,不过,老头也是颐指气使惯了的,有点不甘心就这么被对方压着一头。

于是他斜眼瞟瞟自己的宝贝孙女,又顺着索菲娅的眼光盯住了露丝。

露丝猜不出来老头的用意,不过可以肯定,不是让她解围就是问她是否知道楚云飞的底细,而这两件事,她自问一件也做不到,同时她也不想往这趟水里搀和。

老头看着露丝浑圆的肩膀无奈地耸耸,知道是指望不上了,只好就此打住。

“呵呵,好吧,既然咱们都不了解对方,那这伤感情的话就没有必要再说了,还是谈谈你说的那个‘诅咒’什么的好了,我很有兴趣听听。你不知道吧,年轻时,我可是个冒险家的,呵呵。”

你想转移话题?怎么不问问我乐意不乐意呢?楚云飞显然不习惯对方这么自说自话,那样自己有种被人带着走的感觉,比较被动,“哦,维伦斯先生,不是我执意计较,而是您昨天和您的孙女说话时就对中国人不太礼貌,而今天您的孙子又再次侮辱中国人,我想,在接受道歉之前,我是不会再有兴趣继续谈话了。”

僵住了,局面发展到这里,不可避免地僵住了。

英国人的傲慢,那是世界闻名的,又何况是维伦斯这种非常强势的家族?在老头看来,不计较楚云飞对维伦斯家族的侮辱,算是很给对方面子了,这行为已经迹近于丢人啦,谁叫他是有求于中国人呢?

不过,楚云飞自然是不肯这么善罢甘休的,你既然冒犯我在先,我自然有还击的权力,你想和解也行,你先拿出诚意来,别不疼不痒地说两句就想算了,门儿都没有!人先自辱,而后人辱之!

刘宁还在那里火上添油,“我也是这么认为,谁要你孙子先侮辱人呢?这是遇到了我们,要是遇到别人,还不是任他欺负了?我们不主动招惹别人,不过,想招惹我们,那就要付出代价!”

成树国也站了起来,“我觉得道歉已经很轻了,在这个世界上,强者为尊,侮辱比自己强大的人,我有充分理由解决掉这个人。一百多年前,你们英国人不就是这么对中国人的么?”

老头真的是没办法再说什么了,对方确实是占住了理,可要达克道歉,别说达克,自己都不乐意,刚要恼羞成怒地武力威胁,考林斯走了上来,对老头的耳边说了一句,“岳父,早上据我的观察,他们三个人肯定是做过军人的。”

军人?中国军人?会秘术的中国军人?这都是怎么回事啊?老头的脑袋越发地乱了,也不敢轻易翻脸了,开什么玩笑,谁知道这几个人还有没有同伴?谁也保证不了对方现在就不是军人啊。

此时此刻,索菲娅心里对楚云飞的那点点好感早就不翼而飞,取而代之的是无尽的愤怒。

家族受到侮辱,索菲娅自然是感同身受,达克是自己的亲哥哥,做妹妹的自然要为哥哥出头。

“楚云飞,你实在是太过分了,怎么说达克也是我的哥哥,你这么做,对得起露丝么?”

怎么说也是索菲娅的哥哥,这句话味道怎么怪怪的?

对不起露丝?面对这种异常的逻辑,楚云飞是满头的雾水,“这又关露丝什么事?我自然知道他是你的哥哥,可我更知道,他侮辱了中国人,侮辱了我!”

\section{第一百二十六章 以美人做赌注}

被索菲娅这么一打岔,紧张的气氛登时缓和了不少,不知道为什么,大家都觉得这场面居然变得怪异了起来。

楚云飞反驳的话也没起到该有的严厉的作用,大家都觉得眼前这个年轻人,似乎在徒劳地为自己辩解着什么。为什么会有这种感觉呢?

虽然是索菲娅实实在在地无理取闹,可所有人都没觉得是小姑娘胡闹,总觉得似乎有种说不清道不明的暧昧在里面。

成树国回头看了眼刘宁,发现对方也在强忍着笑意,很没面子地讪讪坐下。

李南鸿长叹一声,语气中充满了遗憾,“唉~~~”

就在这尴尬的时候,达克的声音从门口传了过来,“中国人,你们要为自己的狂妄付出代价的!”原来他出去并没有走远,一直在门外听着呢。

够了够了,实在够乱了,老头一皱眉,“班克斯,你去把你的宝贝儿子弄走,还嫌不够乱是怎么的?”

班克斯闻言站起,向门外走去。

达克着急了,这种事情自己不参与,一会儿要是被爷爷勒令道歉,那可丢人丢大了,马上嚷嚷起来,“不,我要决斗,和那个自夸比我强的决斗!”

号称比他强的人?似乎三个人都表示了强大的,谁来和他决斗?

到现在这个场面,双方都不肯示弱,虽然气氛有所缓和,但对立情绪绝对不可能消除,所以,决斗似乎是种合理的解决方式。

老头很快地就做出了判断,这是个不错的建议,自己的孙子自己知道,击剑、格斗、枪法都绝对是第一流的。而且,决斗输了,并不是说道理上输了。这也是中世纪以来决斗之风屡禁不止的原因,无论输赢,双方都有体面的台阶用来垫脚。

于是老头抬手把儿子喊回来。扭过头来,看看楚云飞,“楚,我觉得达克的说法似乎可以考虑一下,咱们用战斗说话,怎么样?”

怎么样?那好啊,谁怕谁?楚云飞还是那副宠辱不惊的样子,“哦,这个我无所谓,不过,我们为了什么战斗呢?”

老头自然知道对方在商量彩头,无奈地笑笑,“你们赢了,达克自然要道歉,你们输了的话,你们也向我们家族道歉,这样可以吧?当然,决斗过后,大家就不要再说什么伤感情的话了。”

楚云飞才要说什么,达克那里又咋呼上了,“不行,爷爷,我不同意。”

不同意拉倒,谁求你们同意不成?楚云飞就想站起来告辞,反正看老头现在这么活蹦乱跳,自己已经够对得起他们了。

不过达克的话又跟了过来,“他们要输了,光道歉不行,还要帮爷爷彻底把病治好,而且不能再要什么东西。”

老头微微有点感动,还是自己的孙子想着自己多些啊。

治病这事迟早是要提出来的,不过,大家尚未把话谈到这个地步,就已经闹得不可开交了,所以达克借这个机会提出这个要求,维伦斯家族也觉得时机掌握得比较好,就没人再指责达克了。

楚云飞可不干了,你以为你是谁啊?“达克,你不觉得你的要求太高了些么?虽然我不介意答应这个要求,因为你根本没有赢的可能。但是,这种明显不公平的条件,我没兴趣答应,你实在没有把自己摆到合理的位置。”

楚云飞这番话夹枪带棒,连贬带损,把年轻的英国人气得脸色通红,“你、你、你……你只是嘴皮厉害么?”

老头又制止了自己的孙子,他想了想,这么做,确实是对远来的客人不够尊重,达克的错误,就是把自己的地位凌驾在了对方头上。

虽然这是白人歧视有色人种惯有的思维方式,但是老头也不得不承认,眼前这几个人,是有同维伦斯家族平起平坐的权力的。就冲楚云飞在那里一直神色不动,也可以感受得到对方的自信,达克显然是冒失了点。

“好了,楚先生,达克增加了他的条件,你也可以增加你的条件,大家都是有身份的人,有什么事情不能商量呢?”

这话还象那么回事,楚云飞点点头,他知道对方在意的是彻底救好眼前这个老人,生怕因为决斗而导致自己这方不开心,等救治时难免就要做些刁难。所以当附加条件提了出来。

可楚云飞本来就没想好该跟对方提什么条件,面对对方的咄咄逼人,难免一时有些头大。

提些什么条件好呢?要钱自然是个不错的选择,可马上并不能说出比较合理的数字,再说,随便个什么人侮辱了中国人,给钱就能算完么?

楚云飞仓促间是定不下来该提的条件,但他知道不能拖,一犹豫自己这方在气势上就先输了一成,难免有示弱之嫌,好象怕了什么似的,需要马上做出决断!

有了!达克侮辱中国人,那肯定是因为英国人习惯把自己抬得太高,那就以其人之道还治其人之身好了,为了抬高自己的身份,少不得侮辱你们一下了,谁让你们犯错误在先呢?

楚云飞不想再继续想下去了,随手一指,“呵呵,我还真没想到要什么呢,这样吧,就是索菲娅吧。达克你要输了,不但要道歉,索菲娅也要跟我们走,以后就是我的人了。”

楚云飞心想的是侮辱人家,可他这决定做得太仓促了,在别人眼里,这个要求提得没什么连贯性,实在是有点匪夷所思了。

索菲娅最先跳了起来,还好她总算记得自己在扮演淑女,没有破口大骂,但也没好到哪里去,“你想都不要想,我就知道是这样,昨天我就看出来了,你是个真正的色鬼!”

老头也有些生气,你这都是什么跟什么?你和达克的冲突,关索菲娅什么事,难道说,这家伙真的对索菲娅有什么不可告人的野心么?

不过,想可以乱想,话不能乱说,老头强压怒火,很郑重地表示了异议,“楚先生,我觉得索菲娅……她似乎和这件事无关,不知道你怎么会想起提这么一个条件?”

楚云飞已经在后悔了,出言实在是太孟浪了,他赶紧掉头看同伴的反应,嘴里却是还在释放着杀伤力,“哦,维伦斯先生,我觉得这个不是很重要吧,反正达克先生看起来很有信心的。”

达克已经是羞刀难入鞘了,虽然他绝对不想让妹妹掺乎进这件事来,但现在他已经自己把自己架了起来,实在是没有任何退缩的余地了,“是的,我一定要给你一个终生难忘的教训。”

\section{第一百二十七章 老人的道歉}

现在在场的人里,最紧张的自然是非索菲娅莫属了,开什么玩笑,你俩打架关我什么事?同时也不由得暗恨自己的哥哥,你没事硬撑面子,连累得我跟你一起倒霉?要知道,这可是我好不容易找来的高人,是你惹了人家了!

她抬头望向露丝,既然这人是跟露丝来的,怕是露丝要清楚一点这人的能力吧?却见露丝皱着眉头,对着她缓缓地摇了两下头,一脸“不忍目睹”的表情。

这意思再明白不过了,露丝认为这事凶多吉少,索菲娅马上抗议,“不行,我不同意,你俩决斗跟我无关,我不参与。”

楚云飞望着达克似笑非笑,继续挑逗他,“哦,达克,似乎,似乎你的妹妹很不看好你哦,啧啧。”

说完,楚云飞还煞有介事地摇摇头,他根本没有理会索菲娅的意思,那个条件,只是为了侮辱眼前这个人。

达克的脸红得快滴出血了,眼睛里都充满了血丝,低吼一声就要冲上来。

“达克!”说话的又是老头。

以老头的精明,冷眼旁观几分钟,就能弄明白里面的意思,楚云飞就是要侮辱达克!更何况楚云飞心虚地去看同伴的反应,这一幕也落到了老头的眼里。

“好了,楚先生,不需要卖弄你的聪明了,你不就是想侮辱达克么?我承认,开始是我的孙子不对,他冒犯你了,在这里我向你表示歉意,达克提出的附加条件取消,咱们现在只为谁向谁道歉战斗,你觉得怎么样?”

老头的话,一针见血,楚云飞也不好再斤斤计较,人家那么大年纪都道歉了,虽然是为了他的孙子,“呵呵,好的,维伦斯先生,我一向很尊重老人的,事实上,在我的家里,我爷爷说话的时候,我根本不敢插嘴的,就按您说的办好了。”

成树国看了眼刘宁,知道那家伙还在指桑骂槐,别说他爷爷,他爸爸都死了好多年了。

索菲娅的心总算放了下来,可她一琢磨爷爷的话,才反应过来,自己……自己不过是对方侮辱哥哥的工具,虽然这个事实能够证明对方未必是个色鬼,可是,他难道就不知道,女孩子是用来呵护的么?自己怎么也要比露丝漂亮几分吧?

一股莫名其妙的失落感涌上了索菲娅的心头,她忽然发现,这个人比她想象的要可恶得多!

老头假装没听出来里面的意思,事实上他也非常痛恨自己的孙子那么不听话,大人说话,小孩子乱插什么嘴?不过……还是说点别的吧,“哦,那我们商量一下该用什么决斗方式吧,我个人认为,双方没有化解不了的矛盾,就不要用枪了吧,当然,如果楚先生坚持用枪,我们也奉陪得起。决斗是达克要求的,你有权力选择决斗方式。”

楚云飞要是没有如此这般地过了过嘴瘾,真还有三分用枪的心思,不过,对方既然服软了,那他也无所谓了。至于老头撑场面的话,他也没太以为然,人家主场作战,自己总不好把所有风头都抢了吧?

“好的,维伦斯先生,我问问我的同伴,看他们想选择什么样的决斗方式。”

“你的同伴?”索菲娅发出一声疑问,“难道不是你和我哥哥决斗么?”

楚云飞斜眼瞟她一眼,“我和你哥哥决斗?呵呵。”

索菲娅被楚云飞这眼瞟得怒火中烧,这家伙的态度简直越来越恶劣了,等他治好爷爷,一定要想个办法好好地惩罚他一下,至于露丝那里,自己回头再想办法道歉吧。

成树国站了起来,“好吧,我来,我选择空手格斗,不过,我允许达克使用器械。”

成树国的厉害和无情,露丝是见识过的,她痛苦地闭上了眼睛,摇了摇头:完了,达克这次要倒霉了。

达克紧闭着嘴,不敢再说话,他生怕自己忍不住,又说出点什么不合适的话,再被对方抓住把柄穷追猛打,不过,成树国这话他显然无法接受,只好咬牙发出几个单词,“我也不用器械。”

两人都不用器械,那在这里直接比就好了,这个会客厅很大,足足有二百平米,靠着住宅围栏的一侧就有七八十平米的空间,而且地上还有地毯。

露丝说声“开始”,成树国站在那里纹丝不动,达克却开始前后晃动步伐,看样子是拳击的架势。

看到成树国没反应,达克更受不了啦:这是你自己找揍,怪不得我!

一记左刺拳击出!

成树国判断出达克十有八九是虚招,不过他还是动了,眨眼间他的拳头就迎了上去:拳头对拳头,看谁的硬!

两只拳头空中相遇,发出了沉闷的响声,两人都感到钻心的疼痛,成树国就像那拳头不是他的一样,顺势一脚飞出,因为在这时达克已经跳开,只有脚能够得着了。

也就是三分钟的时间,达克身上已经不知道挨了多少拳脚了,终于在成树国一个狠狠的肘击下,达克倒地不起,这招是楚云飞教给成树国的。

看着地下比自己大一号的对手,成树国轻蔑地笑了一声,“我想,这次应该有人道歉了。”

虽然维伦斯家族里的人已经做好了遭遇强手的准备,却没想到,号称家族里“最凶猛”的达克在眨眼间就被人击倒,而且对手似乎还留有余力。

“啪啪啪”,掌声响起,楚云飞根本不用回头就知道这声音出自哪里,除了那个惟恐天下不乱的李南鸿,还会有谁?

李南鸿的声音在下一刻传来,“不错,不错,几个哥哥的功夫果然厉害,能不能教教我呀?”

这种场合,这样的话,对维伦斯家族绝对算得上侮辱,几个年纪大点的还好说,装没听见就好了,索菲娅可是火了,“你个臭小子,给我闭嘴!”边说边搀扶起自己的哥哥。

奇怪的是,楚云飞这边一帮人,对索菲娅的火气没有任何的反应,看来美女的魅力真的是没人能抵挡得住的。

不过,楚云飞心里很清楚,大家不跟她计较,实在是因为索菲娅没有冒犯到大家真正介意的东西,一个小姑娘小小地发发脾气,谁会当真?

美人生嗔,那也算一景呢。

\section{第一百二十八章 能量传递的损耗}

大家在客厅等了还不到十分钟,洗涮干净的达克就进来了,后面跟着怒气未消的索菲娅。

没有达克在,大家沟通起来还是很容易的,这不,现在客厅里笑做一堆,原来,是考林斯刚刚讲了讲楚云飞邀请他“晨练”的事。

技不如人,达克已经打算认栽,赔礼道歉了,不过,看到楚云飞漫不经心而又带点淡漠的笑容,他的怒火再次被勾起。

定下心来的达克,不能说有多聪明,但绝不会再那么意气用事,隔着老远就打上招呼了,“大家好,刚才见识了楚先生同伴的拳脚,果然厉害,我输得没话说。”

老头笑了笑,刚要安排达克道歉,没想到他的宝贝孙子就故态复萌了,还好,他总算还是讲了点策略,“不过,我是真心想请教楚先生本人在中国功夫上的造诣,还请楚先生不要介意。”

楚云飞斜眼看他一下,那脸上的笑意越发地浓重了起来。

达克心里没由来地一颤,下半句话却是没命地倒了出来,“这个,那肯定不能算决斗,只是我想见识见识楚先生的功夫,我知道楚先生可以拒绝我,我愿意出五万英镑,只要楚先生能赢了我这钱就是他的,如果楚先生输了,也不用他出钱,只要他能容忍我的冒犯就行。”

这话的意思就再明显不过了,达克受了教训,心内愤愤不平,就怪楚云飞让同伴出头,想毒打楚云飞一顿出出气,怕人家不答应,还故做大方地抛出五万英镑。

你以为你是谁呀?楚云飞懒得听下去了,你那里一厢情愿故做大方,也得问问我愿意不愿意呢,大话谁不会说?不过,那也得有说话的实力才行。

念及此处,长笑一声,楚云飞骤然站起,一道淡淡的影子划过,等到大家再看的时候他已经又坐了下来。

随后门口传来“咚”的一声闷响,大家抬眼望去,却是达克站到了屋外,脸上是又羞又恼。

达克只觉得眼睛一花,楚云飞就不见了踪影,紧跟着脖领一紧,整个人就飞了出去,还没弄清楚是怎么回事,自己已经稳稳地落在了屋外,头上脚下,站得异常牢固。只是在空中翻了几个跟斗,难免有点头晕眼花的感觉。

楚云飞端起茶杯,轻轻一笑,“呵呵,我也出五万,你自动认输的话,我只收一半钱。”

班克斯和考林斯对视一眼,都能看到对方眼中的震惊,考林斯愣了半晌,喃喃自语,“老天,我现在觉得克里斯蒂[注1]跑得比乌龟还慢。”

达克终于意识到了,自己的妹妹,找了几个远远超出自己想象的人来为爷爷看病。奇怪了,这样的人,她那个朋友是怎么样弄到的?

判断出了形势,楚云飞又给出了台阶,达克要再不懂抓住机会,那他可真枉为维伦斯家族的成员了。

“我突然发现,自己显然犯了太多的错误,在这里,我先向楚先生收回我的挑战,并且会支付两万五千英镑的违约金。呃,还有,我为我刚才对中国人的不尊重表示道歉,希望能获得在座的中国朋友的谅解。”

说罢,达克向门里深鞠一躬。

“啪啪啪”,掌声响起,鼓掌的依旧是李南鸿,他笑容可掬地点点头,“不错不错,早该这个样子的嘛,大家实在没必要搞得那么紧张的嘛,维伦斯家族的人说到做到,果然是正直的男子汉。”

这么肉麻和大度的话,也只有心怀算盘的李南鸿能说出来,其他三个人实在是做不出来的,成树国甚至觉得还不过瘾呢。不过,同胞嘛,自然是一体的,纵然三人有什么遗憾的未尽之事,也只能回头和那小子慢慢计较了。

可在维伦斯家的人眼里,李南鸿这毛头小子突然变得可爱了起来,没别的原因,他为维伦斯家族在表面上争取来了一些尊严。

接下来自然要说说这次的诊金了,老头最先表态,“这次多谢楚先生出手,救了我的老命,不知道我这里该出多少钱,表示自己的谢意?”

老头的话只是个开头,索菲娅的话马上就跟进了,“现在不行,怎么也要楚帮爷爷把病根去掉才行,至于钱,那绝对不是问题。”

楚云飞心情好了不少,虽然知道对方在一唱一和地将自己的军,不过,现在总是可以好好想想该怎么开价了,“这个,对我来说,钱实在不是什么太大的问题,不过呢,我们总也要出个合理的价位,否则的话,对维伦斯先生就太不恭敬了。”

这次是厚道的考林斯受不了啦,这几个人,未免太精明了点吧?“楚先生,你这话是没有错的,不过,你可以按照你的成本来估算的,哪怕再加上些也无所谓的。虽然我岳父不是一般人,但也不能因为他有钱就乱开价。那样做,未免有点趁人之危的嫌疑,实在是不够绅士。”

楚云飞对考林斯的印象不错,现在对方虽然站出来指责自己,但那些话处处都在理上,自己竟然是起不了丝毫抱怨的心思,不过,要讲理么?那好,那就讲理吧。

“考伦,”楚云飞的称呼里透着亲热,“你的话是没有错的,你想知道成本,我也可以和你算算,你岳父年纪大了,给你岳父治病,我花费的能量是给普通人救治的三倍。”

能量?考林斯一时有点接受不了对方的语言习惯,“治病还需要能量?你确定?”

“我确定,”楚云飞点点头,“你也明白,你岳父的病不是通过一般的医疗手段治好的,那自然要用些你不了解的方式。”

“而以我现在的能力,那些能量难免要浪费掉一些,这治三个人的能量,我需要发出九个人的能量才能保证对方能接收到,也就是说,有三分之二的能量损失。”

现在在场的人已经听得目瞪口呆了,包括成树国和刘宁在内,没人明白这个人在说什么。

只有班克斯似乎听懂了点,“你是说,能量在转换中的损耗,就像热能转化为机械能的百分比上限么?”

“对对对,”楚云飞连连点头,“能量转换效率,就像卡诺定理一样,虽然你这比喻不太恰当,但是确实说明了些问题,是的,能量不止在转换形式时会损耗,在传递时也会有损耗。”

班克斯还没弄清对方的用意,不过,楚云飞的下句话震惊了所有在场的人,“既然能量传递时会有损耗,那我吸收这些能量时也会有浪费,九个人的能量,要知道,那需要二十七条人命,我才能吸收到这么点能量。而且是很强壮的那种人。”

\section{第一百二十九章 不要刺激中国人}

大厅再次变得鸦雀无声,那是死一般的寂静,门外是低霭的阴云,连绵的秋雨,门里是阴森的气氛,无尽的寒意。

索菲娅甚至可以发誓:她听到了自己心跳的声音。

所有人都被楚云飞言语间冷酷的事实吓到了,老天,那可是活生生的二十七条生命,怎么听起来,他好象说“吃了二十七个鸡蛋”那样地轻松,而且似乎他只想强调下数字的多少。

连成树国和刘宁都没想到事实会是如此地残酷,何况是别人?

号称凶猛的达克也被吓到了,虽然他是家族里的得力干将,血淋淋的场面见过不少,但那种大规模死伤的战斗听说他父亲也只经历过一回,而且,似乎……当时的现场,双方加起来死了也就十多个人左右,可眼前这人,居然毫不犹豫地弄出二十七条人命的积蓄来为爷爷治病,他到底杀过多少人?五百?还是一万???

“飞哥,”有人说话了,还是李南鸿,他居然用起了汉语!

其实,严格说起来,在场的人里还是数李南鸿对楚云飞他们三个了解得最少,其他人多少还听说过些传闻和介绍,他现在听得可是傻眼了,诈骗!飞哥一定在实施诈骗!“飞哥,你是不是该弄个道冠,拿个拂尘什么的?空口说话,不太好吧?”

被李南鸿这么一掺乎,在场的人终于有了长出口气的心思,犹豫半天,考林斯说话了,“上帝保佑,楚先生,你的话,听起来似乎有点邪恶?”

楚云飞早就在后悔了,他光顾着强调难度了,却忘记考虑他那话会有多么吓人了,看来,过分强调利润,总是要有些差池出现的。而且,事实上他也没花了那么多的能量。

其实楚云飞并不介意对方会认为他有多么邪恶,他后悔的是:似乎不小心把底牌露出个角来,说得太多了!实在是说得太多了!

听到考林斯的质问,楚云飞微微一笑,“考伦,事实上,事情不是你想的那个样子,我只是这么比方,就是个比喻而已。”他居然又叫起了对方的昵称。

多尼从震惊中醒来,神情恍惚地补充了一句,“怪不得你们要在塔提绿洲杀那么多人,‘贝’牌公司那六十多个武装分子也是因为这个能量被你们三个杀光的吧?”

六十多个武装分子被这三个人杀光?在场维伦斯家族的人个个一身冷汗,上帝啊,这样的人,实在是太恐怖了点吧?还有什么绿洲上的人命?这三个人到底杀过多少人?

说到这个,刘宁可不爱听了,“我说多尼,咱们好歹也是一起来的,你不能说点别的么?你不知道我枪法很准么?我们怎么会因为那点小小的能量杀人?”虽然不知道里面的内容,刘宁还是要坚决支持自己人的,用枪杀人总是最正常的吧?

露丝站出来为刘宁做证,“是的,新先生可以打中一英里外的啤酒瓶盖,大家都这么说。”她又给加了半英里,因为她实在不知道普通步枪到底能打多远。

达克听得头都大了,这新先生一听就是狙击手,还好自己没心血来潮地再玩把枪战,这三个中国人,实在是太强悍了点吧?

老头神情严肃,寻思半天才缓缓开口,“这个,楚云飞先生,我是虔诚的基督徒,我只信奉上帝和耶酥,希望你在治疗过程中考虑我的信仰,不要让我太过难堪。”他的措辞异常谨慎,实在是不想再有任何敏感字眼刺激到这几个恐怖的中国人了。

老头的话说得有情有义,楚云飞也想趁早摆脱眼前这个话题,“维伦斯先生,请你放心,这事我是早就考虑过的,咱们只谈科学,呃,就是那种客观存在的东西,不谈信仰。”

班克斯先生的一肚子疑问总算能及时倒出来点了,“楚先生,我实在,我实在看不出来里面有什么科学的东西存在,你似乎……是在谈论些很,很野蛮的东西。”

楚云飞微微一笑,“这个我后面会有解释,不过,我现在想问的是,你们觉得我该收多少的诊金?就算不考虑那二十七条人命,能把这种能量转换的人,你们找得到么?至于那找出病根的问题,你们又觉得出多少钱合适?”

这个问题,实在是难住了在场的维伦斯家的人,二十七条人命的价值,似乎还可以找到些类似的参考,哪怕按保险公司的最低意外伤害赔付,那也是个数字上的游戏而已,不过,人家都不计较这个了,其他两项该赔付多少实在是不好计算了。

老头讪笑簧肮以缇椭雷约赫馓趵厦登耍烧婷幌氲侥芄蟮秸庵殖潭龋蠹揖尤凰悴怀隼础!?

班克斯作为长子,自然该他为父亲争取了,“事实上,我们知道,楚先生,你实在是没把前两项算在内的,要不你就不会提前治好我父亲了,现在,就是彻底去除病根的问题了,不知道我说得对不对?”

楚云飞笑着点点头,“是啊,露丝小姐请我们来一次,我们中国人最不爱欠别人人情,当让要给够露丝小姐面子了。”

就在这一瞬间,露丝知道,自己完了,真的完了,掉进了眼前这个人的陷阱,再也爬不出来了!

即使这人是个邪恶的异教徒,自己……也注定要为他所困了。

在这再三的强调中,索菲娅的婶婶劳瑞忽然意识到,似乎这个露丝,索菲娅的同学,在这个恐怖的东方人眼中有着异常的份量,“如果,我们把接下来的谢意也转交给露丝小姐,不知道楚先生会不会不同意?”

这样也行?楚云飞显然被这突如其来的一棍打得有点晕,对方显然看出了他在待价而沽,索性打起了露丝的主意,小丫头肯定比他好哄些。不过,一时间他似乎还真不太方便做出什么反驳,毕竟……毕竟自己已经弄到了两万五千英镑的赌注了,让别人认为中国人都是贪得无厌就不好了。

还没等其他人反对,李南鸿先不干了,凭什么啊?飞哥好不容易跳大神镇住了对方,怎么能就这么轻易地放弃要求?最不济,也得给自己争取点什么吧?好事都给了那小姑娘?

“我觉得不合适,飞……啊不,楚先生已经表示了对露丝小姐的尊重,要是这事就这么解决,不说对楚先生了,对我们几个朋友,似乎也不够体谅吧?”

听到这话,楚云飞闭上眼,点点头,是啊,光自己做人情了,就忘记眼前这几个兄弟了,这不是见色忘义么?“是的,我也是这么认为,劳瑞女士的话是不错的,不过,不能总是我们表示善意吧?”

\section{第一百三十章 又是三个条件}

老头听了这话,点点头,“没问题,这事好说,我承诺,以后只要是露丝小姐不愿意做或者想做什么事,维伦斯家有义务帮她完成心愿。”他实在不好说“纳入家族保护网下”,那样说的话,黑社会的口气太重了。

而黑社会,实在是不方便在上流社会暴露出本来的面目的。就像一百多年前一样,开拓殖民地也要套个“教化野蛮人”或者“传播主的福音”之类的外套。

而且,这种势力,也实在不便在恐怖的中国人眼皮下卖弄。

“至于你的三个中国朋友,我愿意交给你十万英镑,由你转交给他们好了,数目由你自己定,你看这样可以么?楚先生。”

楚云飞点点头,没表示异议,反正是对方白送的,不要白不要,虽说这十万给谁出都算得上一笔巨额费用了,不过,用来买命的话,出得起的人那就海了去啦。

老头的话得到认可,就继续说了下去,“我再额外出五十万英镑,你帮我把病根去了,然后常住我这里好了,算我的私人顾问,你们四个,楚的年薪三十万英镑,其他人年薪五万英镑,当然,薪水是可以商量的。”

楚云飞笑着摇摇头,“这不可能,我实在不能想象维伦斯先生居然会这么廉价,这样好了,那十万我们不要了,现在就走可以么?”

老头看着楚云飞,皱着眉头思考半天,突然哑然失笑,“呵呵,没什么,我们在谈价钱,有来有往,讨价还价,那才是合理的,像楚先生这样,似乎少了点谈价钱的诚意哦。”

楚云飞淡淡一笑,“有没有诚意,你说了不算,我说了也不算,但你自己心里清楚,你这价钱算不算有诚意。”

老头点点头,“我明白你是怎么想的,不过,我们维伦斯家族虽然生意不算小,但很多不是直属于我的产业,再说,我也不能因为自己,给家族造成太大的损失的,所以,我认为价格已经很合理了。”

楚云飞摇摇头,“抱歉,先不说价钱,我们实在不可能在你这里养老的,也许,有一天我们厌倦了打打杀杀,会来你这里享受平淡的生活,不过,现在是不可能的。”

说到这里,楚云飞又想起了刚贝拉跟自己提的条件,“至于五十万英镑,我觉得比较符合我的价值,但我有个想法,如果,呃,如果病因我说得合理的话,能不能要你们维伦斯家以后帮我完成三件事?”

达克很敏感地注意到了对方的话——“厌倦了打打杀杀”?这句话听起来平常,细细品味下,竟然是有无限的杀气在里面!他不禁看看自家几个人,却发现父亲似乎也被这话所震惊,在那里考虑什么。

“病因解释合理?”老头沉思了一下,“我认为你这个说法不错,事实上,我很想知道是哪些人这么恨我,居然给我下诅咒,不过,三件事,你能概括一下事情涵盖的范围么?”

楚云飞点点头,换了自己也不可能那么容易地就答应,“这个你不需要有太多的顾忌,事实上,我有了你给我的五十万,我甚至可以付出相应的酬劳来,不会是什么严重的事的,我实在是没想好现在有什么要求人的。”

楚云飞倒是考虑了,要对方把他弄到沙特,想来对方能很容易地做到。不过,既然已经答应了多尼了,他也不想做个言而无信的人,赤手空拳想打开片天地,“诚信”还是要讲的。

看到老头还在犹豫,楚云飞放声一笑,“哈哈,我想,到时候你们是有拒绝的权力的,不过,那你们就需要想想我们的能力啦,说实话,只要我们愿意,有的是人愿意同我们交朋友,我这么说,也不过就是想拿你们当朋友结交就是了。”

话已经说到这里,老头实在也没拒绝的心思了,这种性质的朋友相交,那是以实力做后盾的,而眼前这几个中国人,是有资格这么说话的,“好吧,事情就这么说定了,现在,楚,你是不是可以解释病因了?”

楚云飞笑了笑,“好吧,我不求你们一定相信,但不相信的话,后果你们自己考虑吧,我想,我是不会再三地帮助怀疑我的人的。”

大家都知道,下面要说的话肯定会涉及“玄”学了,一个个支楞起了耳朵,露丝不由自主地站起来,慢慢地走到楚云飞的沙发背后,至于她是不是因为害怕才这么做,那就不好说了。

“维伦斯先生,你房间里有副很古老的画,我叫不上来名字,就是画着三个女人在花园的那幅,问题就出在那里。”

“你在开玩笑吧?”达克心里的芥蒂并没有完全消除,他是抱着怀疑的态度来听的,所以他马上就发现了问题的所在,还好,他总还算记得没有失去礼节,“楚先生,那是安吉利科的《黄昏的花园》,那幅画在爷爷卧室里挂了……哦,我听父亲说,他懂事的时候那幅画就在那里了。”

“等等,”说话的是班克斯,达克的父亲,他轻轻皱着眉头,“达克,似乎这幅画最近你整理过?”

达克听着就是一楞,楚云飞没理他,该说的话自己一定要先说出来,省得别人以为自己在蒙人,“是的,那幅画本身没有问题,有问题的是那画的框,虽然我没仔细看,断定不了是什么原因导致的,但是,毫无疑问,那幅画的框,问题很大,希望你们尽量离它远点。”

天阴、下雨、初秋的凉意,除了达克,所有人身上都起了一身的鸡皮疙瘩,露丝更是快要站不住了,轻轻地扶着楚云飞的肩头,楚云飞只道对方被吓到了,下意识地用手回拍,示意对方安静。

那回拍的手被露丝紧紧地攥住,冰凉细长的手指和掌心浸出的冷汗,说明了她的紧张,楚云飞一时倒也不好抽回手来。

达克两腿发软,看样子要滑下沙发了,脸色是要多难看有多难看,他扭头看着自己的妹妹,用带着哭腔的声音喊着,“苏菲,你要给我证明啊,那是你要我帮忙做的框子!”

在场的人脸色都沉了下来,各人打着各人的主意,更有甚者已经把达克先前的态度和现在的表现联系起来考虑了。

不过,索菲娅毅然地拉了哥哥一把,她点点头,“是的,我看原来的框子要散了,就让哥哥出去把画重新装个框,这怎么会跟爷爷的病联系起来?”

达克的哭腔更重了,“没错,就是那个框子,我拿回来还没有三天,爷爷就开始总瞌睡了。”

\section{第一百三十一章 画框不能留}

楚云飞暗出口气,看来,这画框的问题是可以坐实了,自己是不存在“骗子”之虞了。至于维伦斯家族的事,他可没兴趣去管。

其余人闲说几句,确定了那画框就是罪魁祸首。虽然对这种神秘的东西大家感觉不太方便请教,但索菲娅却是不存在这个念头。

“楚先生,你能说说,这画框的问题到底出在哪里么?我看那个框子跟爷爷原来的框子挺像的,为什么以前……哦,天哪,露丝,你们俩在做什么?你不是说认识他没几天么?”

露丝由于越听越发憷,现在基本上整个上身都趴上了楚云飞的肩头,姿势显得亲昵而暧昧。

楚云飞若无其事地从露丝手中抽出手来,“哦,那个画框,我只知道它必须毁掉,否则也许在不远的将来,这里住的人都会有麻烦,它是能够给大家带来麻烦的那种东西。现在,有人能把那个画框取来么?”

露丝又不顾一切地抓住了楚云飞的肩膀,“楚,我真的有点害怕。”

楚云飞眼睛直视大家,就当身后没人一般。

把画框取来?大家你看看我,我看看你,要是没楚云飞的解释还好说,现在大家都知道那东西邪恶了,谁还敢站出来去取那个东西?

达克跳了起来,“我去,我去拿那画,你们等我。”悔恨交加的他,决心要冲锋在前,那画框他也不是没拿过。

班克斯看了他一眼,达克身为嫌疑人,实在是不太合适去,“你,算了,还是楚先生拿去吧,索菲娅陪楚先生走一趟吧。”

索菲娅点点头,没说什么,这个中国人虽然让人讨厌,但陪在他身边,起码不用考虑安全问题。

于是,楚云飞跟在美人后面又享受了次“香风按摩”。

拿到《黄昏的花园》,楚云飞走回客厅,会客室光线不太好,佣人把大灯打开了。

大家都凑了过来,看那画框,不过,外表上实在是看不出有什么不妥当的地方,紫中透红的色泽,简明流利的线条,甚至连镂空雕花的装饰都没有,非常古朴。

早有仆人拿了把裁纸刀过来,索菲娅的婶婶劳瑞干净利落地把画沿着框子裁了下来,同时异常谨慎地没有碰到那个框子,“好了,这个框子现在交给你了,楚,你能告诉我打算怎么处理这个框子么?”

“怎么处理?呃,让我想想,”说完,楚云飞真皱着眉头想了起来。

达克站起来就往外走,宾塞斯喊住了他,“达克,你要去哪里?”

达克见爷爷喊他,咬牙切齿地说,“我去把那个裱画做框子的工匠找来,看是谁唆使他这么做的,好大的狗胆。”

宾塞斯摇摇头,神情有点无奈,自己这个孙子怎么就永远都长不大呢?“你就没想想,工匠那里会不会还有什么诡异的东西?”说罢,眼光瞟了下在沉思的楚云飞。

用手叩叩,那是货真价实的木头,楚云飞终于决定了,暂时不销毁这个东西,先琢磨下这里面究竟隐藏了多少奥秘吧,“我想,呃,我想把这个框子带走,不知道维伦斯先生有什么意见没有?”

宾塞斯摇摇头,脸上的笑容有些神秘,“哦,亲爱的楚,我觉得这么做不好。虽然我十分相信你的智慧和诚实,但是,作为虔诚的基督徒,我十分地不希望世界上有这么邪恶的东西存在,所以,我想还是销毁了它为好。”

楚云飞无奈地摇了摇头,老头这话说得有道理,虽然不排除老头是在讨价还价的可能,但内心深处,楚云飞更相信他们对这种邪恶物质的恐惧。试想,一个自己已经够神秘的了,如果这东西有可能给自己再增加强大的力量,成为坚实的臂助的话,这种势力怕是没几个人会愿意接受的。

“那好吧,”楚云飞同意了老头的意见,说实话他自己本身对这种东西也没太大的掌控信心,要不也不会考虑半天的。他那点可怜的关于生命能量的知识全是出自于自身的经历,而且大部分还是猜测,如果出现些什么不受控制或者说超出能力范围的情况,那可就真的麻烦了。

“不过,在销毁以前,我还是想研究一下,这种邪恶的存在,知道得越多,将来再处理类似的情况就越轻松,这点宾塞斯先生没什么意见吧?”

老头最怕的就是这东西以后成为威胁维伦斯家族的工具,家里再有钱,也不可能五十万五十万那样一直损身伤财地花下去,所以对于别的小要求是一点都不介意的,“这个没问题,呵呵,亲爱的楚,很希望你能在同邪恶交战的过程中得到提高。”

于是楚云飞拿着画框向刚才决斗的空地走去,露丝跟在后面亦步亦趋,楚云飞回头纳闷地看了她一眼,她才猛然发现这个行为似乎有点不太妥当,停下了脚步。

在场所有的人都兴致勃勃地看着楚云飞,看他怎么分析那个画框。

楚云飞盘坐到地毯上,手执画框,灵觉提升,进入“先天境界”。

在明悟中,细细品味,那木框里似乎并没有什么意识存在,只是在近乎本能地吸收空间中游离的生命能量,而且,生命能量一进入那个木框,好象就失去了踪迹,不好观察了,楚云飞尝试着从木框中夺取那生命能量,却是起不了什么作用。

楚云飞想了想,站起身来,把成树国喊了过来,“我一会儿可能会感受不到外界的时间,一小时以后你叫我,如果我没什么反应,你就拿钢针打烂我手里这个框子。”

成树国皱皱眉头,“何必这样,直接烧了它不就完了?”三人休戚与共,成树国实在不支持楚云飞的钻研精神。

楚云飞拍拍战友的肩膀,“呵呵,没什么,实在是机会难得,不好好琢磨下,咱们三个怎么提高?”说完又坐了下去。

这次楚云飞不再留手,拼命地从那木框里夺取生命能量,过了大概有十来分钟,那木框里的能量终于从无到有地向楚云飞流了过来。

不过那木框中所蕴涵的生命能量实在是有限得很,最多也就是四、五个壮年人那么多,很快就被楚云飞搜刮一空。

楚云飞敏锐地注意到,就在自己夺取能量的同时,那木框还在吸取着空间中游离的生命能量,丝毫不因为自己的能量被剥夺而导致混乱。

\section{第一百三十二章 植物?动物?}

看来,一切的原因都出在这木框身上了,并没有其他未知因素操纵的痕迹。

楚云飞睁大眼睛,仔细地琢磨这个木框,怎么看也看不出它有任何与众不同的地方。

那就试试它能承载多少生命能量好了,它怎么也不可能承载索度墓地里那么庞大的光团吧?除非它本身会产生什么变异!

想到就做,楚云飞酝酿起一个个小小的能量团,使劲向木框塞去,连让木框自主吸收的时间都不给它。

当木框承载了大约有十个壮汉能量那么多时,异变产生了!

那木框本来是四根小木头组成,但现在,那四根木棍成为了浑然的一体,生命能量在木框中欢快地流转着,再也不掩饰那明黄的光芒,就在那一瞬间,整个木框仿佛有了生命一般。不过,那能量似乎还带有些褐色的边缘。

植物变成了动物?楚云飞越发地感到了不可思议,不过,在这同时,他确实感受到了木框的勃勃生机。

楚云飞早存了破坏性研究的心理准备,木框虽好,但这世界却注定没有它生存的空间,还是试试它的承载力吧。

楚云飞又酝酿起几个能量团,向那木框塞去,那木框明显地已经承载不了那么多的能量,吸收得十分勉强,整个木框都变得有些透明了,当然,这只是楚云飞的感受,其他人眼中是看不出什么的。

要透明了,接下来该更明显地变异了吧?楚云飞越发地期待了起来。

一枚钢针飞来,打断了楚先生的研究,时间到了,一个小时了!

那木框本来已经堪堪承受不住巨大的能量冲击了,在这枚小小钢针的作用下,“砰”地炸了开来,化作了漫天的尘屑。

木框里面蕴涵的生命能量也爆发开来,楚云飞顾不上埋怨成树国,美美地享受起大餐来。

这情形,在其他人的眼中,显得越发地诡异了,这个中国人,他扔出去了什么东西,为什么那个画框能瞬间消失,炸成粉碎,而不影响其他东西?

楚云飞吸收了大部分的生命能量,才站了起来,他瞟了一眼成树国,弯腰捡起了那根钢针,这事还真怨不得成树国,不过……实在是有点可惜。

走到大家身边,楚云飞并没有再解释什么,别说是有泄底之虑,怕是他真说出来,也不会有人能听得懂的,他只是平静地跟宾塞斯打了个招呼,“好了,那东西我已经处理了,不过,为了稳妥起见,把那里的地毯包着木屑全烧了吧。”

宾塞斯先生一声令下,那块足有三十平米的地毯在十分钟内就被卷了起来,拉出去烧了。

好了,事情结束了,楚云飞坐那里有一搭没一搭地闲聊着,等对方拿钱出来。

宾塞斯自然知道对方想的是什么,而且人家肯先出手处理问题,那肯定是不怕拿不到钱,这时候自己也不能再做小人了,痛快地开出了一张六十三万英镑的支票,“好了,这是我的酬金,连同达克的违约金,没有问题吧?”

问题肯定是有的,楚云飞自然要声明下,“维伦斯先生,您多给了五千英镑。”

宾塞斯笑笑,“呵呵,零头就算了,对了,楚,能不能帮我把最后一点手续完善了?”

楚云飞点点头,“好吧,没问题,你是说那个做木框的人吧?”

以楚云飞的感受,这个木框的威力实在是有限得紧,如果说这是个阴谋,那策划者一定对维伦斯家不是普通的熟悉,起码要确定这画一直在宾塞斯的卧室挂着,所以他断定这事以巧合的可能性为大,再说,他也实在想去看看那个制作者那里还有没有类似的东西了。

于情于理,他都该去拜访下那个制作画框的工匠。

事实上楚云飞的猜测是正确的,当他和气势汹汹的达克出现在那个古老的手工作坊时,矮小的中年工匠满脸的惶恐,“达克,达克,我真的不是故意的,真不是故意的呀!”

据说这个手工作坊存在已经有三百年了,至于到底是不是,没有人说得清楚,不过,在注重效率的现代化都市中,这种作坊能保持下来,是很能说明问题的。

这么古老的作坊,那中年工匠自然是知道伦敦城内很多辛秘的,维伦斯家族这样的强大势力,他不可能不知道,上次达克来找他换画框,他只有唯唯诺诺的份。

达克本来就是相当地强势,为自己的爷爷办事,那更是分外地小心,叮嘱那工匠一定要做得和原来的框子一模一样,工匠怎么敢拒绝?

“达克,那个画框的木质,根本找不到相似的材料,当时都快愁死我了,后来才偶然得了块木头,看那样子也是古董,可再古董也要先为维伦斯先生考虑啊,我才做出了那个画框。至于耍手段,玩花样,那我怎么敢?再说,总共就那么几片木头,玩得出什么花样啊?”

达克是真有心把对方弄回家,盘问一番,不过,既然他跟楚云飞一起来的,那是轮不到他做主的。

楚云飞点点头,作坊主人的话可信度是相当高的,他也知道画框做怪,主要是因为那材质的原因,要是归咎到做工之类的什么上,实在是没事找事。

“什么样的古董?你又是从哪里得到的?”楚云飞直指问题的中心。

那工匠抬头扫达克一眼,发现达克并没有什么不悦的神情,立刻就明白了,眼前这个黄种人,是需要自己认真配合的。

“呃,先生,这件古董……我实在是不方便详细解释,当然,您要是能保证守口如瓶的话,我倒是能把自己知道的东西奉告一点。”

达克不悦地哼了一声,“戴维,你似乎忘记了我是谁,是不是需要我给你背背维伦斯家族的家规?还是需要我帮你把眼睛弄得大点?”

戴维听了这话,却是如释重负的样子,“那太好了,既然达克少爷肯担保,我自然是有什么说什么。”

达克又因为对方再三地卖关子,烦躁了起来,“好了,戴维,我在考虑需要把你的嘴弄得大点了,你什么时候变得这么罗嗦了?要知道,现在,只有你才能救得了你自己!”

\section{第一百三十三章 戴维过关}

戴维嗫嚅半天,才把事情说个大概出来。

前些日子,作坊门口倒下个奄奄一息的流浪汉,戴维自然是不允许这种人赖在门口的,喊了伙计来把他挪到街口,那时的戴维正在为达克的画框头疼,却猛然发现那流浪汉手里抱着的粗木头棍子似乎很合适。

像戴维这种传统的手工工匠,实际上收入是不菲的,虽然地位不高,但还是秉承了英国人虚伪与傲慢的性格,他惦记着那木棍,却又不好伸手去拿,就派个伙计三番五次地去看那流浪汉的生死,等到那流浪汉前脚一死,伙计后脚就把木棍拿了回来。

那木棍木质很好,通体没有裂纹,长有一米多,直径六、七厘米,略粗的那头有个类似女人乳房一样的造型,还有模糊的乳头。

虽然像是件古董,戴维却没想那么多,把达克交代的事情尽快办好才是真的,由于已经拖得时间不短了,他连夜开工,终于在第二天傍晚完成了任务,把画框交给了取货的达克。

但事情并没有就这样结束,第三天早上,警察局就来人调查,是谁最后见到那个流浪汉的,伦敦市里的另一大黑帮家族也放出风来,要他们这条街的人把流浪汉身上的东西交出来。

戴维当时的想法就是:闯祸了!不过事关维伦斯家族,他怎么敢出头?虽然眼前的维伦斯家族在伦敦的势力并不算强,但,那是笑傲美国华尔街的势力啊,整个英国黑道的旗帜,有无与伦比的号召力。

戴维把小伙计喊来,嘱托他不许乱说,那伙计却哆里哆嗦地承认,自己从那流浪汉身上顺了块金表,并把赃物交了出来。戴维是做艺术品的,眼力不差,发现是六十年代初的瑞士雷达表,那时的雷达表刚推出全球第一款硬金属制成的永不磨损椭圆形钻星表,正是如日中天的时节,那这表主人的身份绝对差不到哪里去。

不过,戴维也明白,仅凭这块手表,是惹不起称霸金融城的黑帮的注意的,何况还有苏格兰场?怕是那根木头才是主角吧?于是越发地不敢声张。还好,他这里时不时有些达官贵族降临,也有些文化名人出没,那些势力倒也没有为难他。

事实证明侥幸心理是要不得的,是祸躲不过,外面那些势力逐渐地风平浪静了,维伦斯家的达克却找上门来算帐了,起因……居然还是那根木棍!!!

说到这里,戴维免不了申辩几句,“达克,这实在是怪不得我啊,你交代了要照原样做,我怎么敢不尽心?要是早知道能惹这么大的麻烦,我宁愿你打我一顿也不接这活。”

达克哼了一声,“戴维,这件事我不会就这么算了的,苏格兰场我有的是关系,格瑞尔家族我也有不错的朋友,如果他们说没有这回事,那么,你还是有多远跑多远好了。”

达克内心已经比较相信对方的说法了,毕竟是这么大的动静,自己想要弄点消息太容易了,而且还是自己渠道很通畅的行业,戴维并不是笨蛋。

楚云飞想的则是另外一回事,“戴维,那个,那个木棍的剩余部分呢?”

达克也是一振,是啊,自己居然忘记要其它的证据了,“戴维,把那些剩下的边角料拿出来。”

戴维的表情变得越发地苦涩,笑容也显得分外地牵强,“达克,你想,我还敢留下那些东西做证据么?我亲自点了把火把它们烧掉了。”

烧了?楚云飞皱皱眉头,那里面的生命能量会怎么样?跑出来么,“烧它们的时候你有什么感觉?”

“感觉?”戴维脸上又换了种笑容,哭笑不得的样子,“我的感觉太多了,最多的就是后悔,没准这东西比宾塞斯先生的《黄昏的花园》还值钱呢,就这么被我烧了,大概是伦敦今年最贵的取暖费了吧?”

楚云飞越来越感觉自己像个外星人了,怎么自己想的东西,别人就都理解不了呢?“我在问你正经话呢,烧那木头的时候,你有没有觉得,呃,觉得自己忽然不瞌睡了,或者说忽然想吃饭什么的?”

戴维翻翻眼睛,认真想想,“没有,我只是烧完以后忽然觉得精力充沛,出去叫了个妓女而已。”

达克以为戴维在讽刺楚云飞,张口哈哈大笑起来,却意外地发现中国人若有所思地点点头,还嘟哝了句什么。

楚云飞说的是:“照这样说,果然是这么回事。”——那些生命能量还是跑出来了,被戴维吸收了点。

班克斯随便打了几个电话,就弄明白了一切,果然是戴维说的那样,那个流浪汉从某个神秘人物那里偷了一根木棍,据说那木棍蕴藏着神秘的力量,不过大家知道的是,那木棍是埃及某个法老的遗物,是非常罕见的古董。

回宾馆的路上,楚云飞一直没有说话,他一直在琢磨这件事的来龙去脉,终于做出了这么一个大概的设定:这木棍取材的树木,目前世界上应该已经绝种,否则乱长那还了得?

至于那个乳房造型,那很简单,无非是种图腾或者代表着什么而已,关键是当那木棍的结构被破坏的时候,里面蕴藏的生命能量能够爆发出来,而且,很有可能任何的改动都造成那样的结果。

所以,当木头被做成画框时,里面的能量被释放了出来,而当作为画框,结构一旦稳定下来,它又开始储备生命能量了。

事实真是这样的么?楚云飞也不敢断定,虽然主导人们认知行为选择的是哲学,但楚云飞自认自己到现在为止,似乎还没有形成属于自己的哲学,只能通过蛛丝马迹的判断和想象力来理解。

不过,这样说来的话,那木棍虽然是有细胞壁的,但它真的属于植物么?

李南鸿的话传进了楚云飞的耳朵里,打断了他的思路,“总算到宾馆了,中午实在吃得不爽,我现在想吃碗方便面。”

楚云飞实在被他弄得哭笑不得,拜托,你能不能说点别的什么?

\section{第一百三十四章 笨拙的狐狸}

李南鸿这么一说,其他三个中国人还真是被勾起了同样的心思。伦敦这么大的都市,买点方便面应该是很容易,这个简单任务就交给了李南鸿。

不过,这家伙行事总是那么匪夷所思,不知道他用了什么手段,居然拉上了伊琳娜同去采购,对此,刘宁的评价是:有志果然不在年高。

露丝一个人闲得无聊,又跑来楚云飞他们的套房聊天,却发现刘宁也在房间里,“刘先生也在这里啊?”

刘宁笑笑,其实露丝对楚云飞的意思,连瞎子都看得出来,不过,这事说穿就没意思了,“是,李先生说他很晚才会回来,我也是无聊,来找他们聊聊。”

很晚么?露丝一愣,“可伊琳娜说她们会尽快回来的。”

三个中国人对视一眼,哈哈大笑,还是楚云飞做了解释,“露丝,伊琳娜那么老实,怎么会是那个家伙的对手?你还是明天早点带她回德比吧。”

听到最后一句,露丝的笑容当时就僵在了脸上。

还是成树国厚道些,“你怎么说话呢?宾塞斯那老家伙不是说让露丝在伦敦好好玩玩么?费用都算他的呢,好了,露丝,等李先生买来食物的话,我们怕是都要饿死了,不如你请我们吃晚饭吧,反正是有人报销的。”

楚云飞知道成树国怎么想的,在那个铁血的“民族主义者”的眼中,玩外国女人绝对是为民族争光的事情,“以前光说外国人强奸中国女人了,现在咱有条件,那是要多弄几个外国女人的,也算是为了被欺负的同胞报仇。”

不过楚云飞实在是不赞同这观点的,这事姑且不用说谁更享受些,首先,自己家乡,有个柔情似水的美人在等着呢,虽然有时候,那水像洪水更多一些。

所以楚云飞站了起来,“我要出去走走,难得遇到个雨天,我出去淋淋雨,顺便考虑点事情,你们去吃吧。”

相聚时日无多,露丝也顾不得羞涩了,“我最喜欢下雨了,尤其是在雨中漫步,楚,我陪你去。呃,这是我的卡,成先生拿去吧,希望你们三个多吃点,多喝点哦。”

说罢,露丝拉着楚云飞扬长而去。

成树国望着桌上金光闪闪的卡发呆,“这丫头净卖嘴了,不给密码怎么用?”

刘宁没理他,而是跑到窗户处向外望。

多尼一直站在窗户边,见状摇摇头,“奇怪,刚才他俩还拉着手,怎么你一过来,俩人就各走各的了呢?”

这话实在是冤枉楚云飞了,其实他只是想出去随便走走,消化一下今天的收获,再为下个行动考虑考虑,关于多尼的事情,是不是该从维伦斯家弄点情报呢?

露丝追了上来,楚云飞也不好拒绝,他的心里实在是很复杂的。

这么漂亮的女孩,要说他纯粹不动心,那是假的,不过传统道德的约束力在他身上还是很有效的,这时,他只能学着高建军,在心里说:抱歉,你迟到了,而且,咱俩毕竟不是很熟,是吧?

但他也没操蛋到把这话直接说出来的地步,作为个男人,这种时候心肠能硬起来的很少,甚至他心里有点点报复的快感:琳琳,你看到了没有?这就是你这俩月不给我写信的后果。

伦敦的纬度毕竟靠北了点,初秋的阴雨,落在身上还是有那么点凉意的,就在这凉意中,楚云飞的心情蓦然地振作了起来,“我想去泰晤士河边看看夜景,你呢,露丝?”

伦敦的夜晚,实在是说不上有多么安全,不过,和楚云飞在一起,露丝是毫无畏惧之心的,“好的,事实上,我有将近四年没有看过泰晤士河的夜景了,上次看,还是在大学的时候。”

两人伸手,等了半天才拦住了一辆的士,那出租车司机是个阿拉伯人,一脸的不耐烦,“两位要去哪里?还有半个小时我就要下班了。”

马路这边,李南鸿和伊琳娜出现了,恰好看到了这一幕,李南鸿惊讶得忘记了再继续纠缠伊琳娜的手臂,“哦,上帝,我看到了什么?他俩……这是要单独出去开房间么?”

伊琳娜狠掐了他手臂一下,李南鸿疼得差点跳起来,“啊~~~轻点,你掐到我伤口上了。”

“活该,谁要你乱说?跟你说,露丝,那是嘴上厉害,开房间?我可没听说过,她不是那种人。”

看来这对里面,谁更能做主还真不好说,李南鸿显然是比较陶醉于这种折磨,“她是不是这种人我不关心,嘿嘿,我只想知道,你是不是这种人呢?”

“啊~~~~~~”,一声惨叫回荡在空旷的街道上空,远处,一辆出租车向着泰晤士河方向疾驰而去。

傍晚的泰晤士河,灯光已经渐次亮起,霓虹闪烁处,把这条古老的河流激发出了青春的活力。

烟雨迷蒙中,露丝紧靠着楚云飞,望向轻纱笼罩的对岸。

楚云飞知道自己这时应该跟露丝保持距离,可是,这样的天气,露丝又只穿了件薄薄的丝质衬衣,他实在是做不出来这样的举动,再说,这样做也太伤人了吧?

不过,这样下去,显然不是办法,楚云飞沉默半晌,终于想出了句蹩脚的暗语,“这样的天气,这么漂亮的景色,要是我的未婚妻在就好了。”

说到玩这个,楚云飞显然差得太远了,露丝沉吟半晌,回了一句,“我没有这种遗憾,因为,我是跟我喜欢的人在一起的,如果可能,我愿意时光永远停留在眼前。”

你说你的,我说我的!

对上这种情景,饶是楚云飞狡诈如狐,也不得不直面这样的表示而无法回避。

“露丝,我真的有女朋友了,就在中国,我想,其实你是非常可爱的。”楚云飞的意思是,看看,你只是迟到了,并不是不招人喜欢。

露丝根本不吃这套,这种幼儿园级别的托词,怎么能瞒过玩弄男生于股掌之上的女士?事实上,楚云飞越是如此地坚守,就越激发她的好感,“那只是你的女朋友,不是么?你才二十二岁,不该这么早匆忙决定的,这世界上,好女孩真的很多的。”

楚云飞最不喜欢别人拿他的年龄说事,不过眼下他实在是没有伤害对方的欲望,“是啊,你也年轻,好男孩也很多的。”

露丝望着远方,眼中满是迷茫,缓缓地摇摇头,“不,你不明白的,我见过很多男孩,你,是让我头一个心动的。”

不知道是因为天气的缘故,还是露丝受到了过份的打击,寂静的河岸上,她的声音显得有一丝颤抖,或者说几分哽咽。

楚云飞只能随手找个东西抵挡,这实在不是他的强项,“呃,我是说,露丝,我是中国人里面很传统……”

话还没说完,露丝扭身抱住了楚云飞,她的嘴吻上了他的唇,火热的唇,冰冷的舌。

\section{第一百三十五章 泰晤士河边}

在这一瞬间,楚云飞的大脑基本上停止了转动:怎么会这个样子?

下一刻,他惊讶地发现,露丝冰冷的细舌在他口腔中不住地伸缩着、搅动着,带着丝丝略微发甜的液体,而自己的舌头也在笨拙地迎合着对方,而那双手,已经下意识地拢住了对方纤细的腰肢。

不能这样!不能对不起琳琳!楚云飞马上做出了判断,急忙松开自己的手,像只受了惊的兔子一样跳了开来。

鼻中还停留着少女的芳香,楚云飞尴尬地解释,“不好意思,我真的不是故意的。”他实在没办法说清楚自己刚才的行为。

露丝的脸上飞起一抹嫣红,虽然天色快黑了,但在点点灯火中还是清晰可见,她很开心地笑了笑,“飞哥,我真的很高兴,原来,你真的也是喜欢我的。”这个称呼一定是跟李南鸿学的。

听到这话,楚云飞老脸一红,这事,实在是不能再说下去了,否则只能是越描越黑,不过,似乎也达到了自己的目的,露丝也相信了自己,那么接下来,该是她不得不承认的“迟到”了吧?

老话中说的“一厢情愿”说得就是楚云飞目前这种心态,露丝的身子又靠了过来,“飞哥,我有点冷,能抱抱我么?”

楚云飞二话不说,脱下了自己的衬衣披到了对方身上,虽然衬衣早已经被雨水打湿,但身上多那么一层衣服肯定还是会暖和些的,同时也算是正告对方:我们的关系,只能是这个样子。

要是露丝是个循规蹈矩的中国女孩,楚云飞这样一系列的言辞和举动,应该可以抹杀对方的任何幻想了,不过,很遗憾,露丝是英国女孩,而且是那种灵动跳脱、不太守规矩的女孩。

那衬衣虽然湿了,却依旧带着楚云飞的体温,露丝披到身上,感受着那丝丝温暖,看着细雨中只穿着背心的楚云飞。那平时不怎么显露的健壮身体,虽然不是那种肌肉坟起的类型,却是无处不散放着强大和力量,活脱脱是一只在城市中觅食的猎豹。看这眼前这一幕,露丝感觉到自己居然变得如同伦敦的天气一般,潮湿了起来。

露丝不说话了,楚云飞自然会以为是自己的暗示奏效了,事实上,他也有过拒绝女孩子的经验,只要他随便做点什么暗示,那些骄傲而脆弱的女孩子就会如风中的蒲公英,飘然而去。

“露丝,你知道么,中国古诗里有很多关于雨景的美妙描写,所以,我是很喜欢下雨的。”楚云飞的英语虽然十分棒,但他实在不知道该用什么恰当的词汇来表示“空灵”和“诗情画意”

露丝的回应吓了楚云飞一跳,“飞哥,我发现自己越来越喜欢你了,有时候我真的怀疑,你就是上帝派来折磨我的魔鬼。”

那男人笨拙地摇摇头,这种场面,他实在是不知道该怎么控制了,还是假装没听见吧。异国的秋雨,虽然缺少了亭台楼阁,但也是难得的美景呢。

丝丝秋雨,落入宽阔的泰晤士河,居然激不起丝毫的涟漪,那种安详和静谥,令远方的游子心怀大释。

两人就这么呆呆地又站了有二十分钟,天色渐黑,楚云飞长出口气,“不错,没有白来,好了,咱们该回去了。”

“不,我不想就这么回去,”露丝望向他,虽然冷得有点发抖,语气却是异常地坚定,“你不能打扰我的梦,我要再呆一阵。”语气中,无尽的遗憾从唇间涌出。

楚云飞长叹一声,不再言语。

声音入耳,露丝眨眼间似乎又换了个人,“如果,让你不开心了,那,我们回去吧,不过,你要答应我个条件。”

“条件?”楚云飞下意识地咀嚼了下这个单词。

“是的,条件,”露丝的语气中充满了萧瑟,不过,实在是天太黑了,否则,楚云飞一定能从她的眼中看出一丝的狡黠,“我希望你能像刚才一样,抱抱我,吻吻我,那样,我就再没有什么遗憾了。”

现在用“其蠢如猪”来形容楚云飞都不算过份,他想的是,虽然这是在背叛的边缘行走,但如果能一了百了,彼此没有任何地伤害,也可以算是处理得比较完美了吧?

男人点点头,于是,下一刻的泰晤士河边,两个孤单的身影紧紧地抱在了一起,唇舌交加。

天地间,迷茫的雨丝徒劳地挥舞着,可它又怎能浇灭那熊熊燃烧的青春火焰?

这一吻,足足持续了有五分钟,露丝彻底地迷失在了浓烈的男性气味中,而且,他的身体又是那么地火热,真想融化在里面。

楚云飞的感觉又不一样了,清心寡欲一年多,而且,连日的疲于奔命,那原始的生理欲望始终被牢牢地压抑在内心深处,而现在,少女的柔情把心灵深处的魔鬼慢慢地释放了出来。

鼻中,淡淡的香水味,压制不住少女的体香;手中,盈盈满握的,是那充满活力和弹性的肌肤,猛然间,楚云飞竟然有了那种男性特有的生理反应。

露丝实在是个玩弄男人的高手,接吻中,用强烈的鼻息暗示:我发现了,你不老实!

楚云飞反应过来这种暗示的时候,羞愧得无地自容,终于结束了这漫长的一吻。

嘴才得了自由,露丝就放肆地“咯咯”大笑起来。

楚云飞越发地愤懑,“露丝,不要笑了,我们该回去了。”

露丝回过来的话,让楚云飞越发地不了解女人这一特殊的情感动物,“楚,你不要再骗自己了,你的身体已经出卖了你,你需要我!”

不是说好最后一吻的么?尽量天气阴凉,楚云飞的头脑还是有点发蒙,“露丝,这个,现在已经不是很重要了吧?”

“怎么会不重要?”露丝显示出了她罕见的刁蛮,“你没听说过么?女人的承诺是最不能相信的,而我,绝对是女人中的女人,你迟早会知道的。”

懒得理你,楚云飞自知没办法辩解,掉头走去拦出租车。

身后,露丝清脆的声音传来,充满了诱惑的威胁,“楚,我发誓,不会让你从我手中逃脱的!”

\section{第一百三十六章 弄个挡箭牌}

依楚云飞的意思,治好了宾塞斯先生,几个人就可以轻装东移,直奔法国,帮多尼处理问题了。

但他的意见遭到了所有人的反对。

李南鸿存了私心,是反对得最激烈的,虽然他根本没有选择的权力,“飞哥,宾塞斯老头不是说要好好地招待你几天么?咱们这样说走就走,是不是太没礼貌了点?怎么说人家也还欠着你三件事呢。”

成树国和刘宁也反对,“好了,云飞,咱们好不容易从那个破地方出来了,怎么也要先歇上两天不是?欧洲咱们三个都是头一次来,操,不好好玩玩怎么对得起自己?以前咱们只有用不完的时间,现在,咱也有钱了,人生得意须尽欢,谁知道这样的日子还会不会再有?”说到这里,成树国的脸色有点黯然。

连当事人多尼都反对,“楚,我觉得你真有必要在这里多呆几天,跟维伦斯先生搞好关系的话,对我的事,还有你的事,会带来很多方便的,你不会告诉我你没看出来他们家族是做什么的吧?”

李南鸿的小算盘不可能瞒过楚云飞,战友的感受他也深深同意,不过多尼的话让他有点微微的不爽,就在这上面随便做了做文章,“多尼,这个,不用你说我也想得到的,事实上,我完全可以要求宾塞斯把我弄到沙特的,想来这事不会很难,可这么一做,我实在就没有帮助你的理由了,你不夸我够朋友也就算了,居然还怀疑我的智商?”

楚云飞实在没有办法解释自己真实的想法:那个露丝,实在是太缠人了,这么下去,很难保自己不犯什么错误。

多尼狐疑地看看楚云飞,“我怎么会怀疑你的智商?我只是奇怪你为什么那么着急地离开英国,我都不着急你却那么急,这次回去,事情绝对不会那么容易办,多准备准备,自然是好的,而且,办事前尽量放松放松,享受下生活吧。”

“我操,”说话的是刘宁,他听出了多尼的意思,现在不享受,谁知道以后还有没有机会?刀头舐血的人,最不喜欢这种不吉利的话,“以前我们这里只有一个乌鸦嘴,现在有俩了,多尼你说点别的什么吧。”

多尼也知道他说话不太小心,马上转移话题,“我就奇怪楚了,你怎么那么着急走呢?哦,我明白了,你是怕不小心爱上索菲娅,对不起家里的情人吧?”

多尼以前也是花丛里打滚的男人,虽然露丝也足以当得起美女,但在他眼里,只有索菲娅的魅力称得上是“不可抵挡”。

李南鸿怀着点私心,不支持多尼的观点,“索菲娅什么的不好说,不过,现在露丝已经够飞哥喝两壶了,哈哈,是吧,飞哥?”

成树国古怪地笑了起来,样子极其暧昧,听了两人的话,他显然已经隐约猜到楚云飞为什么着急地离开了。

刘宁看看成树国的样子,心里也明白了八九分,不过他的话说出来却是更加直接,“这事不难办,不过,云飞,他似乎对这种感情方面的东西弱智得很啊。”

楚云飞被人戳破心思,难免有几分狼狈,不过刘宁的话又勾起了他的兴趣,虽然以前没听说刘宁在这方面有多么擅长,不过听起来他是有些办法的。

碍着多尼和李南鸿,楚云飞也不好直接请教,毕竟在不太熟的人面前承认弱智那是需要勇气的,何况那李南鸿的嘴巴又是那么的大,难保什么时候漏点口风出去,他只好揣着明白装糊涂,“刘宁,你说的,呃,不难办是什么意思?我听不太明白。”

刘宁和成树国对视一眼,这家伙,果然是为了这个才着急走,刘宁不好再笑话他,清清嗓子,“咳,我也就是那么一说,我是说啊,索菲娅,是比露丝漂亮一些,这个,你承认吧?”

有眼睛的人都知道!楚云飞皱着眉头,点点头。

成树国已经猜出来刘宁要说什么了,站起身赶人,“好了,小李子,你和多尼进里面吧,我们哥三个说几句话。”

见到俩人离开,刘宁说出了下文,“你要真的想躲开露丝的纠缠,那就直接说啊,要是不行,拿索菲娅做个挡箭牌也不错,反正那丫头对你绝对没什么好感,不会弄巧成拙的。反正咱们也要跟老头家处好关系的,你说是不是?”

要是李南鸿还在,自然知道这显然不是什么好点子,三人一直在部队里,男男女女的感情这方面,实在是谁也不比谁强多少的。

这样可以么?楚云飞半信半疑,不过,刘宁毕竟是三人里年纪最长的,而且他轻易是不发表什么意见的,所以这个建议最终还是被楚云飞有保留地采纳了。

成树国隐约也觉得刘宁这主意未必有多好,不过他实在是没资格质疑的,于是敲起了边鼓,“唉,多大点事啊?把她俩全搞定不就也成呀,只要哥几个不说,云飞别把麻烦带回家就行了,你那个琳琳还能检查出来不成?人不风流枉少年啊。”

要不说成长的环境很重要呢?楚云飞长期跟俩战友呆在一起,不受点影响也是不可能的,否则起码是不利于团结的,“我总是觉得,琳琳在家等我,我不能做出对不起她的事。”

“我也没要你做对不起她的事呀,”刘宁又开始犯躁了,“不过是权宜之计而已,跟维伦斯家搞好关系,没准能弄到多尼对手的资料呢,那咱三个成功的机会可就大了不少的,咱们命贱,那也不能随便扔啊。”

楚云飞点点头,这点倒是早在他意料之中的,要不他吃撑着了,要什么承诺?

成树国的话可就多了几分沧桑,“夫妻本是同林鸟,大难来临还各自飞呢,我爸爸拼死从印度人手里抢出来的战友,‘文革’时候还不是照样写他黑材料?我可也对我家中勤没那么大的信心,该怎么着就怎么着吧。”

楚云飞不说话了,他又想起了张玉珊,那个女作家冷酷无情的劝告,顿时心里乱做了一团。

沉默半晌,楚云飞终于做出了决定,是不是心口如一那就不得而知了,“唉,算了,我还是不想对不起琳琳,权宜之计就权宜吧。奇怪,就听你俩说话了,怎么不见你俩做出来点什么?”

成树国“嘿嘿”一笑,“刚才你俩出去的时候,小李子就说了,要约我们去红灯区逛逛呢。”

啊?楚云飞张大了嘴巴,“不能这么不像话吧?”

\section{第一百三十七章 图个什么呢}

话一出口,反倒是刘宁和成树国不好意思了,一个圈子里,影响是相互的,楚云飞把话说成这样,这俩没做过类似勾当的人还真不好意思硬撑着。

刘宁道貌岸然地反驳,“这个,呃,我们就是去那里玩玩,蹦蹦的什么的,或者去酒吧喝酒也行,谁说一定要,一定要找女人了?”

楚云飞摇摇头,他们三个实在是太熟悉了,刘宁这种语气,那绝对是抱着什么想法的。

看楚云飞这不以为然的样子,刘宁有点恼羞成怒,“玩玩就玩玩呗,好象就你清高一样,扯淡,都是扯淡,家都没有了,还唧歪个屁,一天不回国,咱们就一天这么憋着?要是一辈子回不去呢?”

成树国倒不担心他俩杠起来,这么点小事根本不可能,他拍拍楚云飞,“云飞,现实点吧,你抱着什么崇高的理想,兄弟们那是管不着的,不过,我们俩家的情况你也是了解的,至于自己糟蹋自己么?”

“咱们哪,可都是死过不止一回的人了。人活这辈子,图个什么呢?”

楚云飞真没想到,环境变得好了,这俩战友反而更消沉了,一边是友情,一边是爱情,自己该偏向那边呢?

摇摇头,算了,还是不用想了,该怎么办就怎么办,走一步说一步吧,都被人当鼻涕一样甩了,还坚持个什么原则呢?再说了,这俩战友还要帮自己完成心愿,那可也得是别着脑袋在玩呢。

想到这里,楚云飞也再不能泼冷水了,“呵呵,那你们玩去吧,不过别碰那些妓女啊,我觉得脏,你们找些什么一夜情之类的,我绝对不反对。”

刘宁很高兴楚云飞的转变,毕竟眼下三人是以这家伙为首的,能说通他是最好的,“行了,别以为就你干净,这要在国内,我俩可比你吃香多了,我们至于眼界那么低么?”

成树国也在那里怪笑,“哈哈,你小子,感情是怕脏啊?要不脏你是不是就要没命的上了?果然是咬人的狗不叫。我们真是去玩玩的,当然,遇到对眼法的,咱也不能留下什么遗憾不是?”

楚云飞笑着摇摇头,“你俩啊,不行,我得离远点,这么纯洁的男生,不能让你们带坏。”

两人异口同声地来了句,“我呸!”成树国还加了句,“那就让露丝和索菲娅把你带坏好了。”

第二天一大早,维伦斯家的车又来了,天遂人愿,车上坐的就是索菲娅,不过,她可是一副不情愿的样子。

老头既然允诺了招待众人,不但是要派车马,还要派人的,几个年长的各有各的事情,达克不但和中国人不太对劲,而且马上要去纽约了,只有索菲娅比较空闲,年龄又相当,而且她还是露丝的好朋友,这差使自然是非她莫属了。

露丝昨夜淋了雨,身体有些微微地不适,看看今天又要下雨的样子,就穿上了厚厚的牛仔夹克,本来的三分野性变成了七分。

她昨天已经把话点明了,自然再也不肯委屈自己,一屁股就坐到了楚云飞的身边,手向楚云飞的胳膊缠去。

楚云飞看着大家都在盯着他,脸上分外地挂不住,早知道,昨天就不去泰晤士河了,弄得这丫头现在这么肆无忌惮。

“咳咳,露丝,你看起来精神不太好,要不回去休息一下吧,天气也不是很好,有索菲娅陪着我们就行了。”楚云飞一边伪做关心地说着,一边不露痕迹地抵挡着露丝的手臂。

露丝还没说话,索菲娅就发言了,“露丝要不去,我自然是要陪她的,还好还有伊琳娜,要不让她带着你们四处走走吧。”

“那也行,”楚云飞马上表示同意,“要不,你们三个都留下,我们有你家的司机和多尼就行了,实在不行,我们还可以找个导游的,肯定能听到不少好听的故事呢。”

索菲娅虽然愿意留下陪露丝,不过听到这话还是非常地不高兴,自己早说不来的,爷爷非要自己来,来了还受到别人轻视,似乎自己还不如个街边随便能找到的导游,“那好吧,露丝,咱们去你房间吧。”

相聚时日无多,露丝自然不肯放弃任何能和楚云飞在一起的机会,“不行,你在伦敦一天,我就要陪你一天。”

这话入耳,索菲娅吃惊地张大了嘴巴,露丝是什么样的人,她是很清楚的,那从来是给别的男人脸色的主,这话居然都能说出来,那就说明露丝真的喜欢上这个中国人了,他有那么好么?

看着索菲娅吃惊的样子,楚云飞才想起来刘宁昨天的建议,似乎刚才又忘了,居然想把这俩都留到宾馆,实在是该打。他正后悔呢,转念一想,有了!

楚云飞把嘴凑到露丝的耳边,“露丝,别这样,索菲娅会看见的,我可不想给大家留下什么太花心的印象。”

露丝听得就是一震,心有所系,这么拙劣的表演她都没看出破绽来。怎么,他什么时候开始介意索菲娅的印象了?还说什么大家,明明就是担心索菲娅误会,他不是在中国有女朋友么?

人有了私心,就总爱往敏感处想。要是同别人比,露丝还没那么担心,她对自己的魅力是相当有信心的,当然,这信心半出天生,一半也是被她戏弄的男人惯出来的。可要是同索菲娅打对台,她可是处处矮着对方一头的,家世、容貌、身材等等,没有一样超过对方。

所以她破天荒地紧张了起来,凑近楚云飞的耳朵悄悄地说,“你想也别想,索菲娅小姐,那是将来要嫁给贵族的,要是你打她的主意,那个老头会把你撕成碎片。”

李南鸿,那是惟恐天下不乱的主,见此情景,怪叫一声,“露丝,小心把飞哥的耳朵咬下来,你说什么呢?讲出来大家听听嘛,我们难道不是朋友么?”

楚云飞是什么人?他最听不得别人的威胁,本来他只是想演出戏而已,这话入耳,再想想昨天和战友们的争吵,一时间,竟然有了假戏真做的念头,索菲娅是吧?她的主意不能打么?我都不知道明天能不能活着,还有不敢做的事么?

聪明的露丝这次弄巧成拙了,在她最擅长的方面。

\section{第一百三十八章 拎女士包的男人}

想归想,楚云飞还是第一时间解答了李南鸿的问题,“没什么,露丝问我,要不要和你换下位子,她怕你晕车,起码你是晕机的。”

李南鸿坐在靠后点的位置,旁边是伊琳娜,在不涉及自身感情问题的时候,楚云飞的反应绝对是一等一的。

李南鸿学着楚云飞的样子,刮刮鼻子,别说要顾及伊琳娜的面子,就算他真想换,也得考虑会不会激怒露丝呢,只能尴尬地笑笑。

索菲娅很纳闷露丝和楚云飞这种“妾意如绵,郎心似铁”的样子,不过再想想,这又关自己什么事呢?只是露丝,这次似乎是遇到克星了呢,回头一定要问问她怎么回事。

一干人等,就在这各怀心思的局面中,畅游了伦敦几个著名景点。

愉快的时间总是过得飞快的,大家觉得还没怎么玩呢,就到了吃午饭的时间了,索菲娅把他们领进了一家装饰典雅的酒店,“这里可以尝到正宗的英国菜。”

楚云飞他们终于知道英国菜为什么那么声名狼籍了,这哪里是人吃的东西?实在是要什么没什么。其制作方式只有两种:放入烤箱烤,或者放进锅里煮。做菜时什么调味品都不放,吃的时候再依个人爱好放些盐、胡椒或芥茉、辣酱油之类。

看到索菲娅和露丝等吃得津津有味,几个中国人不住地摇头:生活在英国,似乎也不是什么快乐的事,算了,还是要点Bitter苦啤酒来喝吧。

看得出来,多尼对英国菜也非常地不感冒,礼节性地点了点,却是根本没有动,和楚云飞他们一样,拿了杯啤酒慢慢喝。

索菲娅正要问问客人们是不是对饮食有什么不习惯,却发现身边的露丝脸色苍白,摇摇欲坠,“露丝,你怎么了?”

楚云飞皱皱眉头,还能怎么了?肯定是病了,“我想,昨天她淋了点雨,应该是感冒了吧。”

“感冒?”索菲娅皱皱眉头,“那怎么还要出来玩?好了,下午你们继续玩吧,我要把露丝带回家照顾。”

露丝笑笑,“没什么,只是头有点晕,我还坚持得住。”

“那怎么可以?”索菲娅的口气不容分辨,“你一定要跟我回去,我可不想你出什么差错。”

露丝见无法反驳,只好同意,不过,她是要拉上楚云飞的,“飞哥,你也要陪我。”

飞哥?索菲娅一个冷战从头到脚,露丝什么时候变得这么肉麻了?

楚云飞心里哀号一声,老天,我这又是招谁惹谁了?

但是,他怎么可能拒绝一个病人的要求?

于是,餐厅里其他人还在不紧不慢地喝着啤酒,索菲娅搀着露丝走了,楚云飞在后面拿着女士手包,一脸地无奈。才走出包厢,背后低沉的笑声就响了起来,听起来十分地刺耳。

到了索菲娅家,车把楚云飞三人放下就掉头走了,没办法,索菲娅打死都不肯坐出租车,只能辛苦她家的司机了。

索菲娅把露丝安排进了客房,又喊来了保健医生,这时候的露丝已经开始发烧了,医生连针都懒得打,直接给她挂上了输液瓶子。

看着露丝难受的样子,楚云飞又插不上手,实在闲得无聊,观察起了她的生命能量。

果然,露丝的生命能量在不断地流逝着,不过同时,她的肌体吸收生命能量的速度也远大于索菲亚,而不象刘宁那次没什么反应,不过,目前的情况是所失大于所得,这加速的能量吸收,就是人们说的“抵抗力”么?

那上次刘宁的抵抗力怎么会接近零呢?楚云飞琢磨一下马上反应了过来,上次自己连平常状态下的游离生命能量都观察不到,凭什么说刘宁没抵抗力啊?那是自己能力不到家!

索菲娅坐在露丝旁边,两人开始喁喁而语,只把楚云飞晾在了一边,楚云飞听得实在没意思,站起身来,“你们俩聊着,我去院子里走走。”

露丝马上反对,“不行,你在这里……”楚云飞没等她说完就装做听不到一溜烟地走了,身后还隐约传来索菲娅异常清脆的嗓音,“叫他做什么?咱俩……”

维伦斯家的院子不算小,几乎可以说是一个微型的园林了,奇怪的是,这里并不是中国人默认的那种规整严谨的西方式(勒诺拓)园林,而是带了几分自然和浪漫的味道,曲折的小径,精巧的棚架和藤蔓,居然还有座小小的中国式凉亭。

这会不会是那些英国传教士从东方带来的风格呢?楚云飞正在思索,却发现索菲娅的父亲班克斯正顺着扭曲的小路向他走来。

两人一度已经离得很近了,但班克斯又被小路远远带走,楚云飞只来得及向对方点头示意而已,他心里不住地嘀咕:都说德国人刻板,这英国人也不遑多让啊。

他这感慨还没来得及发出,班克斯又接近了,这次就是正式地“接触”了。

班克斯微笑着想楚云飞打招呼,“你好,楚先生,很高兴又见到你。”

楚云飞对班克斯很有些好感,在他的印象中,这位先生比较雍容,但不做作;说话时虽然爱认死理,但是非常讲究事实和语气,正合楚云飞心目中的“英国绅士”形象,虽然刚才绕路的举动似乎多余了点。

“哪里啊,我是被索菲娅和露丝两位小姐硬拉来的,本来我是想畅游伦敦的,现在,”楚云飞遗憾地摊开双手,“我只能从同伴嘴里听伦敦的风景了。”

班克斯自然是知道自家发生的事情的,不过,他以为楚云飞应该在客房陪着病人才对,所以当他从窗户上看到中国人独自无聊散步,马上就走了出来。

“哦,你是说露丝小姐吧?呵呵,很遗憾啊,我本来想晚些时候去看看她的,也许她们现在还在忙碌呢,你是不是被撵出来了?因为性别的原因?”

因为性别原因被撵出来?露丝得的又不是妇科病!是英国式的玩笑么?楚云飞被这说不清性质的回答难住了,只好另起话题,“哦,我比较喜欢您这里的风景,看她俩谈得比较投机,就出来看看,这个院子,似乎带了中国园林的格局?”

班克斯先生点点头,很欣慰的样子“是啊,你一眼就看出来了,该让那些刻薄的评论家听听,这怎么还不算中国式园林?”

看着庭院里圆形的花坛和喷泉,楚云飞心里叹口气,这也能算中国式园林么?最多也就是带点中国味道,还是说点别的吧。

\section{第一百三十九章 班克斯先生}

两人随便谈了几句,楚云飞终于问出了他想问的话,“班克斯先生,听说,呃,听说维伦斯家族在美国很有势力,不知道这话是不是真的?”

是不是真的?班克斯心里暗笑,昨天晚上考林斯早把他们白天在车上的对话说出来了,眼前这个年轻人实在是明知故问。

不过,班克斯倒是很欣赏楚云飞的谨慎,点点头,笑着反问,“是这样的,你不会连这都不知道吧?”

这话入耳,楚云飞就明白了,对方不想跟他绕弯子,这正好也是他的希望,虽然他不怵那些说话技巧,但能够简单地表达,确实是最方便和省事的。

“我想问问,维伦斯家在欧洲是不是也有类似的能力?或者说,只在英国有?”

班克斯点点头,“在欧洲,维伦斯家族做不到的事情的确不多,不过,我个人认为,能给楚先生带来困惑的问题应该包括在内。”

楚云飞知道对方是在说客气话,当然也隐约有点袖手旁观的意思,谁让自己一直那么嚣张呢?不过他也没在意,“呃,是这样的,我在美国目前是没什么事,不过在欧洲确实是有点小事,班克斯先生,我不知道能不能获得你们的帮助。”

班克斯笑着点点头,“这样吧,你先说说看你遇到了什么困惑,如果方便的话。”

楚云飞先把多尼的事情说了,反正多尼也是同意自己这么做的。

班克斯认真地听着,从头到尾一言不发,只在最后表了下态,“嗯,波兰脱特斯基家族,这个事情,我会帮你了解清楚的,不过,你确定,只是了解么?”

楚云飞想了想,“先只是了解吧,如果需要你们帮忙解决,我会考虑的,这个,应该算同一件事吧?”

班克斯笑着摇摇头,“我就不知道你为什么那么爱斤斤计较,大家关系真的处好了,多为朋友做几件事又算得了什么呢?”

“你说得也是,”楚云飞点点头,不过他可没有什么悔过的意思,人活在世界上,能不欠别人的,还是不要欠的好。

班克斯很意外楚云飞居然不做任何解释,在他的印象中,眼前这个年轻人是很有自己的主见的,“能跟我说说你们为什么会在非洲么?”难得对方这么容易沟通。

楚云飞有点郁闷,无奈地笑笑,“也许,哦,也许我该这么解释,那是因为一个很糟糕的军事任务。”

班克斯比较倾向于认可这个答案,因为楚云飞他们确实像极了军人,“哦,这么说来,似乎这个糟糕的任务,你们已经完成了?”

楚云飞点点头,“可以这么说吧,我们本来就要去法国的,只是露丝小姐恰好遇到了我们,所以就一起先来了英国。”

班克斯有点奇怪,“请恕我冒昧,你们去法国肯定是因为多尼的事,难道你们不需要归队?还是说中国政府已经参与到了打击法国黑社会团伙的国际行动中?”

这个回答就需要认真斟酌了,既不能漏了自己的底,也不能给同是黑社会的维伦斯家带去什么压力,不过,把事情弄得似是而非,一向是楚云飞的强项。

“事实上,我们离归队的时间还很长,而多尼既是我们的朋友,又肯出钱请我们帮忙,那我们自然是要答应的。”

哦,这样啊,雇佣军性质,班克斯点点头,又问了一个和多尼一样的问题,“那么,你们到底有多少人呢?”

楚云飞回答得异常滑溜,“单就这个糟糕任务而言,我觉得有三个人,已经是非常不幸的事了。”

原来是不愿意说,班克斯听明白了楚云飞的意思,点头表示理解,是啊,对方怎么说也是军人,军队里,是有保密制度的。

理解归理解,班克斯还是想多了解些楚云飞他们的动向,“这个,你们在非洲杀那些武装分子,就是你们的任务么?”

楚云飞知道对方是关心他们自己的处境,所以也没计较对方的交浅言深,“那不是我们的任务,那是那个任务糟糕的地方。好了,现在我们是自由的,很长时间内,可以自由地做任何事情而没有约束,现在,我们不管别人的事情,我们只为自己的朋友提供帮助。”

“而且,就我所知,没有中国公民受到威胁的话,似乎我们国家对任何国家的黑社会都没有兴趣,所以,班克斯先生,我们可以做很好的朋友的。”晓之以礼,动之以情这招在通常情况下,总是很管用的。

班克斯其实很高兴对方对自己有所求,这无疑能拉近彼此的感情,不过,他实在是个很细心的人,“那么,唐人街呢?那里的事情你们管不管?”

“唐人街?你是随口问问吧?”楚云飞觉得班克斯问得很随意,班克斯点点头。

“说实话,那里的情况似乎有点复杂,我也说不好,”说到这里,楚云飞又想起来李南鸿被同胞出卖的事,“总之,如果是华人做了什么不好的事情,我个人是懒得去管那结果的,自然有该管的人去管。”

班克斯点点头,“华人,似乎在我们这里不太怎么惹事,不过,在美国,听说他们也很能折腾的。”

楚云飞正乐得把话题转开,问问维伦斯家族的事,“在美国,你们家族比这里厉害多了,能跟我详细说说么?”

班克斯点点头,“事实上,你的朋友说得很对,我们是黑手党,而黑手党里很重要的规矩就是‘沉默法则’,就是说任何时候都不该把自己了解的情况说出去。”

楚云飞点点头,这个法则他从书上看到过,那是描写意大利黑手党的,意思是说对于泄露秘密的人,黑手党总是不惜代价地加以追杀。

班克斯显然比考林斯能做主,不过,没准是得到了老头的特许,“严格说,我们也不算黑手党,只是它们衍生出来的。”

“黑手党家族来英国的时间并不是很长,只是因为我爷爷的父亲,弗雷德。维伦斯诺,他已经死了将近五十年了,他是在美国长大的意大利西西里人,你不觉得维伦斯这个姓在英国并不多见么?”

\section{第一百四十章 英国的黑手党}

原来,弗雷德。维伦斯诺本人出生于米兰,后来因为是西西里人受到歧视,一家人去了美国投靠他的叔叔,等他长大后已经融入了美国当地的黑手党中。

弗雷德不但凶狠残忍,而且头脑非常聪明,1920年开始禁酒以后靠走私烈酒开始崛起,后来在1927年,西西里本土的黑手党登陆美国,遭到了他们当地同胞的猛烈攻击,那场漫长而残忍的战斗中则更加奠定了他的“教父”位置,成为黑手党有名的五大家族之一。

再后来,黑手党因为二战后期接应盟军登陆有功,在美国被默许存在,但弗雷德却认为这并不是什么好事,因为在他看来,属于黑暗的就是黑暗,被拿到阳光下面显露实在并不值得庆幸。于是他果断地休妻,同出身英国德比郡的一著名黑帮家族联姻,把家族成员成功地转移到了伦敦,接下来美国政府对黑手党的打压证明了他的明智。

不过,虽然家族转移到了伦敦,弗雷德在美国的势力一点都没有削弱,后来他虽然被人杀死在了浴室中,但宾塞斯已经成长起来了。他更为狡猾,把打打杀杀的那套基本摈弃不用,而是混进了金融中心华尔街,成功地洗白了自己。等到其他黑手党家族反应过来有样学样的时候,维伦斯诺家族早已在华尔街站稳了脚跟,当然,知道真相的人不会相信他只是个投资或者说投机商。

实事求是地说,黑手党最猖獗的时期还是在弗雷德那个年代,到了宾塞斯那时,大家都已经开始注意形象了,残忍和冷酷虽然始终伴随着黑手党的历史,但现在基本已经被淡化得接近于没有了。

不客气地说,楚云飞现在想借黑手党的力量对付恐怖组织的话,先别说二者会不会有什么勾结,现在的黑手党根本没有以前那么穷凶极恶了,放到弗雷德那个年代或许他们能跟“基天”之类的组织拼一下,不过,也只是拼一下而已。

班克斯先生相当于是给楚云飞上了堂“黑手党”历史课,年轻的中国人再结合下看过的有关书籍,自然知道眼前面对的是什么样的人了。

“那这么说来,你们家跟其他国家的黑社会并不是很熟悉了?”楚云飞问道。

说到这个,班克斯终于显示出了自己傲慢的一面,不过也确实值得骄傲,“我们能在世界的金融中心说话,所以,我们不需要太熟悉他们,但是,真正的黑社会,肯定会知道我们的,呃,自然,那些爆发户不能算。”

“那么,恐怖分子呢?”楚云飞顺着中年人的口气提出了自己的问题,“恐怖分子算不算爆发户?”

班克斯的头马上涨得有两个大了,难道说,这才是中国士兵的真正任务么?不过,他还是很诚恳地说出了该说的话,“恐怖分子,呃,我觉得那种有政治目的的势力,不能算黑社会,而且,我敢保证,我们维伦斯家,绝对跟‘爱尔兰共和军’没什么联系。”

那就是说跟别的恐怖势力还有联系,楚云飞自然能听出里面的意思,但他也明白,像维伦斯家这种黑势力,接近洗白的黑社会,不可能跟那些恐怖分子有什么真正的交情,无非也就是相互利用的利益关系。

“那你们熟悉‘基天’么?”楚云飞终于提出了自己最想问的问题。

“‘基天’?我们自然很熟悉。”班克斯还保持着一贯的雍容,“这几年,他们给我们家族造成的损失将近一千万英镑,还有公司的四条人命,我们怎么可能不熟悉它呢?哦,下雨了。”

语气虽然平淡,但那骨子里的恨意可是非常明显的。

班克斯不想再说下去了,可楚云飞怎么可能放弃这种机会?

他摇摇头,“啧啧,听你的意思是很不想报仇,我怎么会有你们这样的朋友呢?维伦斯家族……唉,老了。”

班克斯白了他一眼,这是个很不绅士的动作,“呵呵,我们家族的事,我们自然有处理的办法,换了你,你会愿意放弃这么大的家业,去打一场打不赢而且不死不休的战斗么?”

楚云飞点点头,“这确实是个问题,事实上,你们英国和美国的人种太杂乱了,什么人都有,防备那些恐怖分子,难度实在太大了。而我们中国就不同了,我要是杀了‘基天’的人,只要老实地回到中国,他们想报复都不可能,外来人种,在中国实在是很显眼的。”

班克斯的身后来了个仆人,那仆人手里拿着两把伞,递给楚云飞一把,又撑起一把伞为班克斯遮雨。

班克斯点点头,“确实是这么回事,其实,如果英国只有白人,动动‘基天’那些恐怖分子也不是什么难事,咦,我怎么感觉你对‘基天’那些人也很痛恨呢?”

看着身后那个仆人,虽然是个白人,楚云飞却很不想就这么把底交出去,不过想想,维伦斯家族,应该没恐怖分子的卧底吧?再说班克斯说话时也没什么忌惮。

楚云飞点点头,“是的,事实上,我的父亲死在了他们手里,所以,在这段我休假的日子里,我会想办法给他们制造一点麻烦的。不知道,我能不能从你们这里得到什么帮助?”

楚云飞把班克斯猜得很透,英国绅士真的不想惹什么麻烦,那实在不是什么愉快的事。

但楚云飞前面几句的铺垫,使得班克斯不好意思随便地拒绝这个要求,承认怯懦,确实不是容易做到的事情,尤其又是一个黑手党的世家。

“这件事情,实在是关系重大,楚,我必须向父亲汇报一下才能确定,我个人,对你的事情非常地同情,我想……”

班克斯沉默半晌,他也想做出什么承诺,但显然那是不现实的,最后还是说了几句客套话,“我想,你的父亲一定,一定是个很不错的人,所以才能有你这么优秀的儿子,很遗憾,我现在见不到他。”

班克斯的意思是,要等他自己也死了才能见到楚振中。

但是,非常不幸,楚云飞看到索菲娅手持一把小伞,正在缓缓地走过来,按常理估计,她该是能听到这几句话。

\section{第一百四十一章 露丝怀孕了?}

索菲娅确实是听到了这几句话,更要命的是,她绝对没听到前面的话。瞬间,就像一个霹雳在她的头顶炸响:父亲在赞扬楚云飞的父亲,而且很想马上见到他!

这意味这什么?那再清楚不过了,自己……恐怕是要嫁人了!

索菲娅知道,自己固然有自由恋爱的权力,但是,最后恐怕是服从家族的安排可能更大一些,这实在是生在大家族里不幸的事。

她原本想过,仗着爷爷对自己的宠爱,跟爷爷提提自己想寻找自己的幸福,不过,很不幸,现在的她在爷爷眼里还是洋娃娃,还没有合适的机会提。眼下看来是不能再等了,要抓紧!

看到楚云飞瞥了她一眼,索菲娅的心不由得加快了跳动。

不过还好,楚云飞没有提什么敏感的事,而是很有礼貌地问候了一句,“你好,索菲娅小姐,不知道露丝现在怎么样了?”

索菲娅见他这么着紧露丝,而且没有带“小姐”的称谓,瞟了自己父亲一眼,难免心中有点困惑,不过嘴上却是回答得彬彬有礼,“哦,楚先生,她刚输完液,现在睡着了,临睡前,还嘱咐我,一定要挽留你,说她想在一睁眼的时候就见到你。”

这话说完,索菲娅狠狠地瞪了父亲一眼:看看,这就是你打算给我找的女婿!这么花心的人!

随口的客套话谁会记得?班克斯被这一眼瞪得摸不着头脑,愣了愣才反应过来,是不是自己该给年轻人留点空间?“哦,我忽然想起来了,我需要马上给美国打个电话,索菲娅,你替爸爸招呼好客人啊。”

看到父亲泰然远去,索菲娅心里实在是有点七上八下,禁不住紧张地问了句,“楚,你们似乎谈论得很愉快?”

楚云飞可不知道眼前的美人在想什么,他根本没往那方便考虑,不过追美女的基本技巧他还是知道的,于是他很平淡地回答,“是的,我和你父亲聊得非常开心,不过,目前有点事,他需要请教一下你的爷爷。”

完蛋!索菲娅的心登时凉了一半,原来这是父亲的主意,那可真有点不太好办了。父亲很少做什么决定,但一旦做出决定,通常都是最终的方案,连爷爷都很少反对。

不过,绝望的少女还是试探地继续问了一句,“你不知道露丝很喜欢你么?你这么做,对她不太公平吧?”

我报仇又关露丝什么事啦?楚云飞很奇怪索菲娅的思维方式,难道她是说我报仇的时候难免会受到伤害,露丝会痛苦么?

你们家肯帮我的话,危险自然会低些的,“哦,如果维伦斯家族能理解我的苦衷的话,我想不会有什么问题的。”

什么?你还打了一箭双雕的主意?

索菲娅简直要出离愤怒了,真没有见过如此无耻的人!刚才自己陪着露丝,病人还一直在念叨眼前这个中国人,“我一定要去告诉爷爷,你不知道露丝怀了你的孩子么?”

怀了我的孩子?楚云飞当下就懵了,把“俘获芳心”计划都丢在了一边,“怀孕?是她说的?”怪不得班克斯说是妇科病呢,露丝,你玩得太狠了吧?

索菲娅作为年轻女孩,比较传统的那种,自然不好在这话题上多做文章,她想起了别的事,“咦?奇怪啊,父亲知道你是这样,怎么还会答应你的提婚呢?”不知不觉中,她居然把自己的假设说了出来。

我提婚?楚云飞的脑袋瓜更晕了,“我向你父亲提婚?跟你结婚么?哈哈,开什么玩笑啊?”

话一说出口,楚云飞就知道坏了,这显然和计划有冲突,“呃,我没有别的意思,其实,其实你真的很漂亮的。”

晚了,一切解释都晚了,索菲娅已经知道了,自己担心的事并没有发生,该是自己猜错了。

可这个中国人未免可恶得过分了吧?和我结婚很丢人么?索菲娅更生气了,她知道眼前这个中国人很狡猾,狡猾到家里所有的人除了骂他就是夸他的。那刚才后一句徒劳的补充,无疑是坐实前一句的失言。

不过这话显然不能说出口,索菲娅确实是要考虑自己的形象的,“其实露丝也很漂亮的,不是么?我从没见过她这么喜欢一个人。”

楚云飞实在不知道这种情况下再怎么捕获对方芳心,要不就算了,其实索菲娅也是个很可爱的女孩子,自己别玩到最后,再玩出什么火来吧?“我不想说露丝什么话,我现在想回去找我的朋友了,至于怀孕,希望她能尽快实现这个愿望,不过,我保证那个人不是我。”

露丝没有怀孕?索菲娅很奇怪回过来这么个信息,再想想刚才露丝话里的点点滴滴的含意,思考半天,她终于恍然大悟了:露丝是怕自己抢了她的心上人!

要是露丝直接说出来,作为好朋友,又帮了自己这么大个忙,索菲娅自然是不方便再和露丝抢的,反正这个中国人是如此地可恶。可露丝这么遮遮掩掩不够朋友,那说明眼前这个人,实在是有常人不及的地方的,没准真的……

想到这里,索菲娅实在不好意思想下去了,心里却暗暗地思索,是不是,自己也该多了解了解这个人呢?谁叫露丝那么不够朋友?

楚云飞看着眼前的美女脸上刷地红晕一闪,以为是那关于怀孕的话题羞着了索菲娅,“那个,索菲娅,很抱歉,我真地很惦记我的同伴,能帮我把你家的司机喊来么?”

索菲娅犹豫半天,还是果断地摇了摇头,“楚,真的不好意思,我想露丝她是很在意你的,你可以从我家打电话去宾馆,留下来好么?”

楚云飞站在那里一言不发,愣了半天,长叹一声,把手里始终未撑开的雨伞交给对方,转身向院外走去,“那算了,我还是自己打车回去吧,替我向你父亲道个别。”

索菲娅终于明白了,自己和露丝,在对方眼里,都赶不上那几个同伴的重要。

细雨中,看着对方逐渐远去的身影,索菲娅第一次感觉到了:这个男人看起来冷漠,其实,他的感情,是藏在心里的。

\section{第一百四十二章 遭遇足球流氓}

楚云飞在索菲娅家呆的时间着实地长了点,而郊区的的士又是极其地难打,最后还是一个过路的黑人捎了他一程才回到伦敦市区。

等他回到宾馆,天已经快黑了,但是那帮家伙一个都没有回来,看来真是玩得忘乎所以了,楚云飞也懒得多想,洗了个澡打开了电视。

可他在中国就不怎么爱看电视,这里的电视就更吸引不了他了,电视里不是些无聊的八卦就是小丑们的插科打诨,BBC里面的新闻虽然多一点,不过那里的“国际新闻”实在可以用“美国新闻”来代替,其他国家的消息基本上是没有的。

连个三级片都没有!楚云飞恨恨地关掉了电视,坐到床上开始炼气。

他不知道,三级片还是有的,不过要等到九点以后,或者付费频道。

等他再次睁眼的时候,已经接近晚上九点了,不过,同伴们还是没有回来。

楚云飞再也坐不住了,换了身衣服走出了宾馆,刘宁他们,别是出了什么事吧?想想刚才在电视里看到的广告,他开始考虑,是不是,呃,是不是该买一人买一部手机呢?

不过,他没有等多久,终于看到了索菲娅家的白色商务车,车上稀里哗啦下来一堆醉汉。

楚云飞摇摇头,走上去问李南鸿,“小李子,你们这是怎么了?”

李南鸿斜着眼睛看看楚云飞,“呃”地打个酒嗝,“飞、飞哥,我就不明白了,你咋又跑回来了呢?索菲娅家,呃,索菲娅家今天的菜真的不错。”

操,感情,我惦记着你们,巴巴地跑回来,你们倒好,喝得晕天黑地的回来?“刘宁,你们这……太不江湖了吧?”

刘宁指着楚云飞笑了起来,很高兴的那种大笑,“哈哈,哈哈哈哈,你呀……傻帽。”

楚云飞刮刮鼻子,这个……自己平时喝多了也是这种样子么?

成树国晃晃悠悠地走上来,“别,别理他们,他们……呕,他们都喝多了,就我没多。”

多尼醉得最厉害,拽住司机不让走,“朋友,呃,我很少……很少求人,现在我求你了,求你再陪我喝会儿,你知道么,在英国,找个愿意讲法语的,真不容易啊。”说着眼泪都快下来了。

那司机无奈地冲着楚云飞喊着,“楚先生,请你,请你劝劝你的朋友好么?”

几个人正在这里纠缠不清,街边又过来一群醉汉,有十几个吧。

“汉斯,闭上你的狗嘴,你要那么喜欢赢,滚去支持阿森纳了。”

“比克斯,再吵吵我撕了你们两个的狗嘴,你们要实在闲得无聊,去开辆推土机撞马赛队的大巴士好了。”

“呜呜,呜呜,我……我真不明白,西汉姆怎么会堕落到这个地步?我真的好……好伤心啊,五比零,这个比分,呜呜……”

楚云飞皱皱眉头,这就是传说中的英国球迷了吧?似乎他们喜欢的球队刚遭遇到惨败?

司机还在孜孜不倦地传达着索菲娅小姐的邀请,多尼却是听到了自己熟悉的单词,“哦,有人在说马赛?是吗?我喜欢那里,我喜欢马赛!!!”

于是,楚云飞清清楚楚地看到了英国球迷向“足球流氓”的转变过程。

那十几个醉汉有老有小,年纪大的有四五十岁,小的估计也就十二三岁,黑人白人都有,无一例外地拎着酒瓶子,大多数人还是拎着俩。

多尼的声音是那么的大,所有醉汉的眼睛全瞪了过来,目光中,是赤裸裸的仇恨!

开推土机撞巴士不太容易,但是灭一个为马赛叫好的人实在太简单了,一个身材削瘦的黑人率先冲了过来,隔着老远就抡起了酒瓶。

其他醉汉也蜂拥而上。

这时候的成树国和刘宁都醉了,虽然力气难免小了点,手脚也不免慢了点,但遗憾的是,下手轻重掌握得也不太好了。

幸亏楚云飞没有喝酒,所以当刘宁和成树国放倒三个人后,才发现其他人都已经被楚云飞解决了。

醉汉们横七竖八地躺了一地,初秋的雨夜,够他们受的,其中一个似乎还被刘宁一肘打断了肋骨,在地上痛得蜷成了一团。

李南鸿和多尼经这么一吓,酒都醒了大半,那司机也是看得目瞪口呆,没想到三个中国人居然是如此地厉害。

成树国还在那里打晃呢,“呃,人呢?人呢?刚才好象有一百多,怎么现在地上就这么几个?”

楚云飞走上前去,问那司机,“现在我们是不是该等警察来?还是跑路?”幸亏下雨,天也晚了,四周只有宾馆的门童看到了这一幕。

那司机心思灵巧得很,知道有自家的招牌,这事铁定会不了了之,于是借机行诈,“这个,还是去维伦斯家躲躲吧,你不知道,伦敦的警察,很糟糕,抓进去先打人。”

是这样的么?楚云飞怎么看都是司机在趁火打劫,不过,同伴一个个都是这样,似乎也只能听从对方安排了。“那好吧,会不会晚了点,不太方便?”

那司机受到了叮嘱,一力应承,“没问题,班克斯先生交代了,如果楚先生不愿意离开同伴,欢迎一起到维伦斯家做客,反正伊琳娜小姐已经留在那里了。”

“您先招呼您的同伴上车,我去跟那个门童交代一下。”估计那门童要受到些威胁了。

楚云飞摇摇头,早知道是这样,自己还来来去去的折腾个什么劲啊?

耽搁了这么一阵,雨下得有点大了,又打了一架,其他人似乎也清醒了不少,楚云飞看看大家,“看什么?上车吧,这里今天住不成了。”

等到那车又回到维伦斯家的时候,班克斯先生还没休息呢,他从窗户里看到了回来的人,马上喊了管家来,“你去安排他们的住宿,楚先生一定要安排到贵宾房,我不管你用什么方法。”

管家离开,班克斯先生拿起电话,拨了几个号码,“索菲娅,人我可是给你喊回来了,不过现在太晚了,你不许出去,有什么话明天再说。”

几个人一下车,又恢复了那种彬彬有礼的样子,楚云飞终于松了口气,看来,几个同胞还知道轻重,起码在别人家里知道表现得庄重点。

下一刻,他听到成树国在低声地问李南鸿,“小李子,你不想打听伊琳娜住哪个房间么?机会难得啊。”

\section{第一百四十三章 中国酒楼}

第二天一大早楚云飞就醒了,看看屋里的老式落地座钟,才六点十分,贵宾房里的床实在是软了点,一晚上睡得他腰酸背痛的。

洗漱完毕后,才六点二十,楚云飞拉开了房门,去院子里锻炼锻炼吧。

一出楼门,他就发现院子里有人,是索菲娅。

索菲娅穿着宽松的运动衣,正在院子里伸手伸脚,也不知道是在做操还是练舞,脸上连妆都没化,素面朝天。

楚云飞暗暗点点头,李南鸿说过,女人漂不漂亮,要看她卸妆以后,索菲娅绝对是真正的美女,似乎不化妆更好看些。

“早上好,索菲娅,你起得真早啊,你这是做操么?”

索菲娅点点头,她自然不能说,我比平时早起了半个小时,“是的,这是韵律操,中国人不做么?”

楚云飞拉开架势开始打拳,一边打一边回答对方,“哦,不好意思,我还真不知道他们做不做这种操,我一直在部队里的,我们只打拳。”

两人正聊着天,宾塞斯出来,他也穿着一身的运动装,“哦,天哪,我还以为只有我怕死,原来你们也这么热衷锻炼?楚你在练中国功夫么?”

楚云飞知道老头在没话找话,考伦斯都能看出来是军体拳,宾塞斯怎么可能看不出来,微微一笑,“我练的是军队的拳法,中国功夫我其实懂得不多。”

索菲娅换了种操,在地上不停地蹦来蹦去,“爷爷,他一直在军队里呆着,所以练军队的拳也很正常呀。”

宾塞斯差点要开口调笑下孙女,那个“他”是谁,不过,他猛然间反应了过来,孙女今天,似乎起得太早了,这里面有问题!

于是老头不再罗嗦,“看着你们年轻人,我实在是嫉妒,不行,我不能陪着你们练了,我要去散步了。”这事回头再细细盘问索菲娅吧。

索菲娅巴不得爷爷早点离开,要是爷爷随口问问为什么自己今天起这么早,再被眼前这个狡猾的家伙听到,那可就羞死人了。看着爷爷走远,她又问楚云飞,“楚,锻炼完了你等等我,我们一起去看露丝。”

住到这里,楚云飞已经认命了,随便别人怎么看吧,自己记得把握好自己就好了,“好的,不过,锻炼完直接去好了,为什么要等等?”

楚云飞看到索菲娅脸色一红,却不回答,马上想起了自己看过的几本书里的叙述,感情人家是要去洗晨澡啊,这笑话弄大了!一时间却也不方便解释,只好低头闷声打拳。

索菲娅已经明白面前这个人一直在军队生活,这肯定不是有意的调笑,又看到楚云飞那副窘样,心里竟不由得起了份关爱:他好象个孩子啊。

人和人之间,随便调整下视角,居然眼中的世界会如此不同,这是索菲娅也没想到的。要搁在以前,这怕又是一桩某人“好色”的铁证。

当露丝睁开眼,一眼就看到了坐在旁边的楚云飞,“楚,你一直在这里么?我好感动。”说着就想挣扎着起来。

楚云飞刮刮鼻子,“好了,你先安静地躺躺,医生说,注意休息的话,你的病会好得很快的。至于我么,我是听说你怀孕了,所以来向你贺喜。”

露丝本来已经躺下了,又挣扎着支起身子,向四周看看,没人!“楚,你听我说,我……我只是同索菲娅开个玩笑,再说我也没说我怀孕,我只是说了句‘听说怀孕的女人抵抗力要差点’,谁知道她会那样理解啊?”

千万不要同女人讲道理,这句话在很多书上都有,楚云飞自然是知道的,“好了,这也没什么大不了的,该过去的总是要过去的,你还是安心养病吧,今天我要跟朋友出去玩了。”

人真的是一种奇怪的感情动物,露丝在没表白自己心意的时候还能理智地处理问题,现在居然是再也压抑不住一腔的热情。

她掀起被子,光脚就跳下了床,一把抱住了楚云飞“楚,我求你,在这里陪我好么?”话没说完,却感到一阵天旋地转。

楚云飞无奈地摇摇头,把这摇摇欲坠的女士扶上她的床,又为她盖好被子,“好了,不要调皮了,老实养病,有什么事,等你病好了再说吧。”

等你病好,我怕是都到法国了,楚云飞这么想着抬起头,却意外地发现索菲娅站在面前,她什么时候来的?自己现在的警惕心实在是太差了,看来贪图安逸果然是要不得的。

露丝喘两口气,恢复了过来,“咦,索菲娅,你也来了?”

索菲娅来了已经很有一段时间了,该听的不该听的都听到了,不过她并没有太计较,只是把白皙的纤手放到露丝的额头,“嗯,不错,退烧了,你知道么?你差点吓死我……”

边说着,索菲娅的另一只手悄悄地向楚云飞摆摆,示意他溜走。

楚云飞看到索菲娅的动作,趁着露丝的视线被遮挡的功夫,脚下像踩着棉花般地无声开溜,走出门外,他才来得及纳闷:咦,索菲娅今天怎么会这么好心?

楚云飞并没有时间仔细考虑这些,他被成树国和刘宁拉着出去玩了,李南鸿倒是大方地答应留下来照顾露丝和伊琳娜,不过,以楚云飞看来,那家伙的心思,怕还是放在跟维伦斯家套近乎上面。

多尼被班克斯先生挽留了下来,因为班克斯要帮他打听消息,自然希望他本人在一旁提供必要的线索。

在楚云飞的催促下,三个人连早饭都没吃就跑了出来,只是辛苦了那司机了。

天还是阴的,三人本想先回宾馆的,可司机认为现在回去不太好,西汉姆联队的球迷们的疯狂是出了名的。

于是,三人开始开始在司机的引导下逛伦敦,昨天下午去过的地方自然是不去了,著名的景点逛了几个,后来索性就逛街了。

楚云飞先前总听到人说英国人不好看,这么一逛才知道,其实这边帅哥、美女到处都是。不管他们是真正的伦敦人还是印巴等英联邦国家的人,总之满眼都是高高瘦瘦的美眉,金发,褐发,红发;蓝眼睛,灰眼睛,咖啡眼睛,黑眼睛;黑皮肤,白皮肤,古铜肤色等什么样的都有,身材好的绝对是丰胸肥臀,再配上性感的服饰,真是养眼。

中午的时候,四个人找了个中国餐馆,“粤海酒楼”,就在科文特花园旁边。

这条街并不宽,,街边全是东西方餐厅食肆和各种不同风格的酒吧,有点像楚云飞他们先阳市的“食品一条街”,华人也多了不少,很有点三教九流龙蛇混杂的味道。

\section{第一百四十四章 伦敦购物}

“粤海酒楼”一如国内饭店的夸张,其实总共也就能摆放二十余张餐桌,不过,那菜的味道是比较地道的,用服务员的话说就是,“这里也就粤菜正宗点,其中又数我们家最正宗,不信啊,你回头吃吃那些川菜什么的,绝对跑味了。”

很显然,服务员是听到三人用中国话聊天了,才来补上这么几句,算是为他们饭店做做广告。

服务员的话音刚落,旁边一桌的中国人就说话了。

那桌一共四个人,三男一女,年龄都是十七、八岁,说话的是一个戴眼镜,瘦高身材,一嘴的龅牙,“喂,你这话天天讲,腻歪不腻歪啊?大家好歹也都是中国人,没必要背后坏人吧?”

三个士兵相互看看,算了,还是老实吃饭吧,不过,大家心里不免有几分的不爽。

吃完饭,几个人出去继续逛,司机说皇家歌剧院就在附近,于是就去那里转了转,却发现晚上19:30有皇家爱乐乐队的演出。

楚云飞对那个东西没兴趣,刘宁却不愿意放弃,“既然来了,为什么不看看?我也不爱听这个,大不了咱忍受不了走人呗。”

毕竟是可有可无的东西,大家也不再争论,看看时间已经一点多了,就在附近转悠起来。

穿过伦敦气象中心和苏格兰银行,居然无心中走到了唐人街,旁边就是闻名世界的“SOHO红灯区”了。

司机介绍到这里,脸上就多了点暧昧神色,成树国也刘宁对视一眼,“转了半天了,我们找个酒吧喝酒好了。”

楚云飞用脚趾头都能想出来这二位的心思,想着酒吧艳遇?看咱三个的打扮也不象有钱的啊。

“算了,咱们还是去购物吧,买几身差不多的衣服,对了,咱们要不要买三部手机?”

这样的暗示俩战友第一时间就反应了过来,“对,对,对,咱们还是先买些衣服吧。”

在司机的指点下,三人开始采购衣服,不管贵贱,买了一大堆,花了将近两万英镑。因为:他们从里到外,实在是没几件替换衣服。

幸亏司机开的是商务车,看着车后部高高的衣服堆,司机在心里摇摇头:这哥几个,买衣服比索菲娅小姐还厉害!

接下来,司机就拉着他们直奔通讯广场,在司机的极力劝说下,三人选择了三部美国产的“MOTOROLA”的GSM手机,上户自然是英国最大的移动电话运营商“沃达丰”。

英国人对法国和德国人的芥蒂不是一天养成的,也不是一天能消除的,这司机又是个道地的英国人,自然不会推荐他们买什么“西门子”“阿尔卡特”之类的品牌。

由于三人是索度户籍,享受不了那后付费的待遇,只能是预存话费,该存多少?算了,一人五千吧,谁知道下次什么时候再来英国。

买完手机,正好旁边还有卖对讲机的,三人对视一眼,楚云飞直接说话了,“买五个对讲机,要功率大,体积小的。”

司机马上上来制止,“三位先生,这个算了吧,这些都是民用品。你们知道……”

虽然司机没把后话说出来,但大家都明白了,维伦斯家里该是有好货色的!

于是三人向门外走,路过卫星电话的时候,刘宁还是停了一下,“要不咱们买个卫星电话?”

卫星电话不同于普通的移动电话,无需考虑基站的问题,基本上在地球任何一个地方都可以使用,只要那里卫星覆盖得到,支起那小小的圆形天线就行了。不过通讯费也是贵得离谱。

楚云飞苦笑一下,“谁值得咱们用卫星电话联系他?或者说,谁会用这么昂贵的系统来联系咱们?”

虽然说买了很多东西,但以三个人干脆利落的性格,并没有花了多长时间,这时候大家看看时间,还没到六点。

刘宁和成树国一人穿套牛仔的“口袋装”,手里招摇地拿着新买的手机,“呃,云飞,见到索菲娅,说我们俩会晚点回去哦。”

楚云飞没吭声,把身上的迷彩夹克一脱,边脱裤子边说,“司机,见到索菲娅,说我们三个会晚点回去哦。”他实在是怕了露丝了。

那司机四下看看,“呃,克莱斯勒,我能不能……能不能要求你自己回去?”

哄笑声中,一行四人向着红灯区……的隔壁驶去。

时间还早,自然不便去歌剧院门口等着,四个人随便捡了个酒吧钻了进去。

酒吧不是他们想象中的酒吧,其实红灯区附近的酒吧,全是那种重金属或者说超重低音组成的喧嚣空间,和国内的迪厅类似。

英国果然和传说中的其他地方不太一样,很快地过来了一个性感的少妇,不过,也许说臃肿更为合适些。

她没有理会三个手持手机的中国人,尽管他们三个像是个有钱的群体,而是把目光对准了司机,“嗨,帅哥,想请我喝杯酒么?”

司机尴尬地摇摇头,“抱歉,我等人。”肚子里,他怕是已经在后悔没一个人来了吧?

刘宁心血来潮,开起了司机的玩笑,“总裁,你去玩吧,人来了我们帮你招呼。”

楚云飞斜眼瞟着司机的制服,心里暗暗好笑,有穿低级制服的总裁么?

不过,有时候很多事情实在难说清楚,就像欧美A片男主人公总是家政服务工人一样,身份越低微的人好象越有艳遇的福气?

司机可是知道,这种艳遇那是百年难得一遇,但是,对着维伦斯家三位尊贵的客人,再给他个胆子也不敢就这么拍屁股走人。

“女士,不好意思,我真的很遗憾。”

那女人不再纠缠,瞟了几人一眼,悻悻地走了。

臃肿女人走了还没有两分钟,又一个身材苗条的女孩走了过来,不过这次招呼的是中国人,“先生们好,日本人么?”

楚云飞借着昏暗的灯光,看清楚了那浓妆艳抹下的低矮鼻梁,是黄种人!

这次自然是成树国答话了,“你不要跟我说那个肮脏的种族,再说我揍你。”

那女子微微地愣了一下,不过马上反应了过来,“哦,原来三位是韩国人,实在不好意思。”

刘宁可是开始郁闷了,这伦敦市的中国人怎么也比韩国人多得多吧,“奇怪,为什么我们是韩国人?”

\section{第一百四十五章 玩枪的小毛孩子}

刘宁一搭话,那女孩也不客气,一屁股坐到了刘宁旁边,边说话边扫视四人,“三位是才到伦敦的吧?我能不能喝瓶啤酒?”

看着桌上四人要的一打啤酒,虽然心情不太爽,刘宁总不能很小气地说“不准喝”,于是点点头。

女孩刚伸出手,想喊服务员来开酒瓶,刘宁摇摇头,“不用。”拿起一瓶啤酒,拇指和食指微微用力,“碰”地一声,瓶盖就被拿掉了。

女孩“哗”地叫了一声,表示惊讶,“帅哥你好酷啊,晚上要我陪你么?”

刘宁摇摇头,“你先回答我的问题。”

那女孩撇撇嘴,“这还不好猜,你们那么痛恨日本人,又这么有钱,自然是韩国人了。”说完仰头就是一大口啤酒。

楚云飞皱皱眉头,“我操,我们就不能是中国人?”

酒吧里声音比较喧嚣,那女孩并没有听清楚云飞在说什么,不过,看那口型也猜得出些来,“中国人?中国人里有几个有钱的?而且那些越有钱的越不敢招惹日本人,哪里像你们韩国人这么有骨气?”

其实那女孩也知道高丽棒子的骨气强不到哪里去,不过,她已经认准眼前三个是韩国人了,自然要一力巴结。

“至于台湾人,有钱的倒是多些,不过,他们见了日本人,那就是老鼠见了猫,哪里敢胡乱骂人?”

楚云飞真的很想一脚踹到对方脸上,不过再想想,这女人说的未尝就不是事实,如果连客观陈述事实的人都打,那未免有点掩耳盗铃的感觉。于是他反而伸手拉住了想要站起来的成树国。

刘宁也是老大的不舒服,听了这话再舒服那真不能算中国人,“小姐,你能告诉我你是哪国人么?”

那女孩也渐渐地觉出有点不太对头,从语气上就可以知道,这三个人很明显地对她有点小芥蒂,话头也不由得硬了起来,“你们废话怎么那么多?我生在中国,有人愿意出钱买我么?”

成树国终于按捺不住了,用汉语骂了起来,“我操你妈的,你也知道自己是中国人?真你妈的贱。”

话一出口,那女孩当时就愣在了那里,隔了半晌才发泼地尖叫起来,这次用的倒是母语,“你他妈的又算什么东西?敢骂我,信不信我废了你个孙子?”

刘宁懒得再说什么,一脚把那女孩连椅子蹬出五米开外,“滚你妈的,老子不打女人。”

这时候,已经有人注意到了这里的骚乱,一个鼻子上穿环的年轻人走了过来,也是个黄种人,身后还有两个跟班,“露丝,发生了什么事?”

露丝?这女人也叫露丝?

成树国不待那女人说话,直接用汉语训斥,“滚你妈的远点,再废话老子杀了你。”

那鼻子穿环的年轻人皱皱眉头,跟着就是勃然大怒,“大陆仔?阿林、阿伟,帮我把这个杭嘎岑(死全家)的舌头割了。”

这种粗口一出来,楚云飞他们都知道面对的是什么人了,大家都有粤明省的战友的。这家伙不是粤明人就是虹空市人。

那俩跟班随手抄起凳子就冲成树国冲了过来,“丢你老母!”

楚云飞和刘宁根本没有帮忙的意思,坐下来继续喝酒。

成树国双腿连踢,把凳子踢飞,然后双拳左右开弓,就是一顿狠揍,一分钟内就把那俩跟班打倒在地,“我操你妈的,说普通话!再说鸟语,老子把你们的鸡巴塞到嘴里。”

一把椅子居然飞出去了有七、八米,差点砸到其他的客人。整个酒吧的音乐也因此停了下来。

他正在这里发飙,那个穿了鼻环的年轻人已经从怀里掏出了一把手枪,抬手就要开枪。

成树国手一扬,就是一枚钢针,正正打在那年轻人的手腕上,手枪登时落地,手腕上血流如注。

年轻人随后就软倒在地上,捧着手腕声嘶力竭地喊着,“妈的,你们带种的留下名字,这仇不报,我就是婊子养的。”

刘宁走上前去,轻轻一脚,踹在那人脸上,“报你妈的逼仇,不是看在你是中国人的份上,老子今天就杀了你。毛都没长齐,也学人玩枪?”说到这里弯腰把那枪捡起来,“操,这种女人玩的枪你也好意思拿出来?老子扔的垃圾都比这强。”

那年轻人岁数不过二十出头,嚣张气焰已经被这俩人的腾腾杀气镇住不少,脸上已经有了点惶恐的样子,不过嘴皮还是很硬,“好,算你们狠,有本事你们在这里等着。”

楚云飞也走了过来,从刘宁手里拿过枪看了一眼,不动声色地把弹匣卸掉,枪和子弹全丢到地上,“哥哥今天心情好,懒得杀人,你知道你那个露丝,为什么惹了我们么?”

那年轻人听得就是一愣,是啊,感情自己都不知道为什么招惹了眼前这几位,抬头找那露丝,却是早就趁着混乱溜了。

虽然搞不清状况,这位却是不肯服软,年轻,自然脾气大点,“我没必要知道,这里是我们飞龙帮罩的场子,我大哥就是狂龙。”

听到这么垃圾的名字,成树国和刘宁倒是没什么反应,掉头走回座位,楚云飞实在有点无奈,拜托,汉字那么多,你们不能起些别致点的名字么?

“随便你,我们一会儿还要去看音乐会,叫你那大哥到歌剧院等我们好了。你们都不怕丢人,我还说什么?”

事态逐渐平息了下来,音乐再度响起,看来这种事情,在场的也是见得多了,更有那好事的连舞都不跳了,坐在那里等着下一步的好戏。

那年轻人把手枪捡了起来,看看三人没什么反应,想了想,居然不敢再次使用,重新装进了怀里。

跟班已经不知道从哪里弄来了绷带,熟练地给年轻人包起了手腕。那年轻人也有股狠劲,居然不出去,咬着牙在那里坐着,生怕三人离开他的视线。

楚云飞他们的兴致也被这场突如其来的变故打扰了不少,几个人也懒得再四处看美女了,坐在那里大口地灌着啤酒。

那司机也知道客人们心情不好,开导着他们,“算了,都是些小流氓,事实上,唐人街里没什么有名的帮派,也有人说,他们都忙着内部斗了。”

这话实在是起不到协调气氛的作用,三个中国人更烦躁了。

\section{第一百四十六章 狂龙不狂了}

出来玩是要找开心的,既然大家开始郁闷了,那就换个地方好了,楚云飞把手里的啤酒一口气喝光,“好了,剩下四瓶,一人一瓶,清光,咱们走路。”

喝掉剩下的啤酒,四人站起身来走路,那鼻子上长环的年轻人又坐在那里叫了起来,“怎么,怕了?我大哥很快就会来了。”

有司机在场,楚云飞强压着再打此人一顿的欲望,省得别人说中国人只会内斗,“告诉你大哥,让他在皇家歌剧院门口等我们好了。”

那年轻人哪里肯相信这话,疯狂地大笑起来,“哈哈哈哈,你们这话哄鬼去吧,你们也只敢欺负我,一听我大哥要来,跑得比兔子还快。”

成树国听得又是心头火起,“你他妈还真的是给脸不要了?”就要上去出手打人。

一声暴喝响起,说的却是汉语,“谁在这儿闹事?给我滚出来!”那是赣通口音。

随着这一声暴喝,一个五短身材的汉子走进酒吧,身后稀里哗啦地跟进来接近二十条汉子。

楚云飞细细地看了看此人,这时,DJ师早就停止了音乐,灯光也亮了起来。

那汉子约莫四十岁左右,满脸是横竖纵横的伤疤,还少了半只耳朵,右耳上半边不见了。

楚云飞摇摇头,很刻薄地说了一句,“你就是狂龙?你知道不知道你这个样子很影响中国人的形象?”他用的自然是汉语,同胞嘛。

这话出口,狂龙足足愣了有半分钟,才有了反应,他不怒反笑,“哈哈,有意思,实在有意思,很久没见过这么有性格的朋友了,小朋友,你出门的时候,你家大人没跟你说过,有些人你是惹不起的么?”

成树国正待出头,刘宁拉了他一把,“我来。”

“老朋友,你小时侯,你家大人没跟你说过有些人是千万不能招惹的么?”毕竟是乡亲,刘宁的话里带上了赣通的口音。

家乡口音入耳,那汉子自然是愣了一下,不过他马上又暴躁起来,“扯淡,老乡我见多了,不过今天我可没功夫认老乡。”

楚云飞点点头,“那正好,我也没兴趣认识你们这些垃圾。”说毕,人影向前迅疾地飘去。

成树国早掏了一把钢针出来,他倒不担心楚云飞拳脚不行,但谁知道对方还有多少支枪,他必须压阵。

狂龙的强悍出乎了大家的意料,他见势不对,身形暴退,堪堪地避过了楚云飞扇来的耳光,那指尖从他脸前三寸处掠过。

“等等,”狂龙本来是一脸的轻蔑,一瞬间就变成了极度的惊讶,“这位小兄弟有功夫在身,不知道是什么门派的?”

楚云飞也愣了一下,他是没使出全力,可他也清楚,能避开这记耳光的人还是真不多。对方的动作虽然不能说十分自如,但也表明了是可堪一战的对手。

不等对方说话,楚云飞就给对方打上了标签,武林人士,而且身手很棒,起码是自己“团长师傅”那个级别的高手。

不过,楚云飞还是有相当把握能轻松制服对方的,所以根本没理会对方的问话,而是很鄙夷地哼了一声,“我倒是想知道你是哪个门派的,作为个武林中人,你居然把功夫用在欺行霸市、收保护费这上面,你不觉得愧对你的师门么?”

狂龙被楚云飞斥责得脸色白一阵,青一阵,终于还是决定撇开那些枝节,“我的事用不到你管,既然你不肯说,我的礼节也算尽到了,江湖上也不能再说我什么了。小子,你出手吧。”

说毕,狂龙一个“懒扎衣”的起手,等待楚云飞的攻击。

说到这些江湖规矩,楚云飞就差得太多了,因为他既不是什么门派中人,也没有混迹江湖的经历,不知道对方这招的讲究。

“懒扎衣”是很通俗的一种起手,顾名思义,那就是作势懒洋洋地扎扎衣服,收拾停当开始比试。通常比武双方身份大体相当,也没什么化解不了的仇恨,很中性的起手式,大多用在武林中人狭路相逢的时候。

这时的楚云飞该还以别的起手式来表明态度,像表明切磋性质的“喜相逢”,或者是请教姿势的“细柳随风”,态度恶劣点也要有个“博浪一击”这样的起手来回应。

不过,这样的讲究楚云飞是全然不懂的,所以他的回应是狂龙最接受不了的:楚云飞直接冲了过来,起脚便踢。

狂龙大怒之下,也不再留手,气运十足,一板一眼地和楚云飞对打了起来。

几个回合下来,狂龙更愤怒耍苑剿坪醺揪筒欢械墓婢兀坏惺勇遥夷切┮鹾莸氖侄我彩遣愠霾磺睿槐匾饷雌疵桑空飧鲂」镒幽训啦恢溃舅腔牒竦钠Γ罂梢怨饷髡蟮厝∈さ模?

刘宁和成树国也很想看楚云飞动手的过程,不过,三人的配合实在太默契了,现在,显然不是观摩的好时机。

一个明显打手模样的人把手伸进了怀中,还没等成树国反应,刘宁早一个箭步冲了过去,一拧对方那只手臂,“小子,想干啥?”

那人手臂被反拧,怀中东西掉了出来,原来是个哨子。

围观的打手看到刘宁不讲规矩地偷袭,一时群情激愤,马上冲了过来。

正在这时,狂龙发出一声大吼,那是痛苦的吼声,“啊~~~”

大家扭头看去,却见狂龙站在那里,豆大的汗珠从额头滚落,右臂软绵绵地耷拉着,一看就是脱臼了,而出手伤人的楚云飞则在一旁神情肃穆地站着,不知道在想些什么。

楚云飞也不得不佩服狂龙的气魄,完全不顾右臂被制而强行脱身,纵然是肩膀脱臼,自己却也失去了锁拿对方喉咙的机会。

狂龙看着楚云飞没有再借机出手,马上就意识到了对方没有“赶尽杀绝”的心思。虽然刚才对方出手不太讲究,但起码眼前是个好的兆头。

待到看着一干手下围住了刘宁,狂龙再也顾不得许多,先是一声断喝,“你们这帮家伙搞什么?散开!”

这样的高手,肯定是能不招惹就别招惹,这帮家伙,怕事情还闹得不够大么?

\section{第一百四十七章 慑服狂龙}

狂龙的威严是显而易见的,这种级别的高手,走到哪里都不可能被人漠视的,何况他又是这帮人的头头,围着的人马上散开。

狂龙赶开众人,左手托住右臂,手一用力,身子一抖,“咯啦”一声,豆大的汗珠掉了下来,脱臼的胳膊上了回去,高手果然就是高手。

楚云飞看得也是暗暗佩服,接卸关节虽然他也能轻易做到,但是接自己的关节,想做得如同对方般地举重若轻,那是不可能的。就这简单的一晃一抖中,实在是藏着太多的汗水和泪水的。

狂龙微微地摆动下右臂,试试接骨的效果。那结果显然令他比较满意,所以不再试验,而是冲着楚云飞一拱手,“阁下武功高强,在下佩服之至,于某技不如人,输得心服口服。今天的事情如何了结,还请阁下指点。”

言语间,竟然是依足了江湖规矩,连措辞都是如此。

不过这种规矩是难不住楚云飞的,书上多的是范例,废人关和耿风也提过两句,他也拱拱手,“承让,楚某这里先谢过了。”

接下来话锋一转,“至于该怎么了结,你还是先问问你的人,他们做了些什么吧。”

狂龙这时想不配合都难了,他手臂脱臼,虽然已经上好了,但短期内是不能再战了。他用手一指那鼻环青年,“阿基,你说,怎么回事?”

那阿基再也不敢嚣张了,他畏畏缩缩地把事情交代了一番,“……就是这样,我想找露丝问问到底是怎么回事,结果那个婊子不知道跑哪里了,然后他们就打了我一顿。”

狂龙听了,当下就是个大耳光抽了过去,“我操,人都跑了你替谁出这个头?你他妈脖子上长的是脑袋么?”

说完,狂龙看看四周,“弟兄们听好了,以后见着这个贱人,先轮了她大米,每人最少三次,要是你们的鸡巴不顶用,老子就替你们割了,听见没有?”

“听到了”,“知道了,龙哥。”一群人乱哄哄地回答。

狂龙叹口气,他有点后悔,后悔自己平时没好好地调教这帮小子,眼下表现出来的素质竟然是如此低下,实在起不到什么好的展示效果。

楚云飞皱皱眉头,他虽然见不得那个中国露丝,不过,飞龙帮这帮家伙的素质也太低了点吧?还有那个狂龙,他还算得上武林中人么?做事实在太不讲究了。

“好了,也没什么大不了的事,那女人非要在我跟前讲日本人和韩国人的好话,贬低中国人,我实在是听着不爽。”

狂龙听了这话,可是真正的愣住了,他斜眼瞟了一下阿基,脸上惊诧万分,“不是吧,就是为了这么个缘故?至于么?”

阿基还没说话,又轮到成树国暴走了,“操,我怎么听你的意思是,为这点事不值得?那女人骂中国人,你能受得了?”

狂龙翻翻白眼,老天,阿基怎么招惹了这么几个生瓜蛋子?“这个,怎么说呢?女人嘛,又是做皮肉生意的,讨好客人,那也是正常的不是?她肯定以为你们不是中国人呢。”

“说实话,兄弟也给吓了一跳,人手一部手机,谁能想到你们是中国人?”

刘宁不干了,“我说,你怎么说话呢?卖逼卖到外国已经够他妈丢人了,还没命地糟蹋中国人,这还算人么?”

三人里就刘宁没怎么出手,又是狂龙的老乡,所以狂龙回答得也很直接,“你是老板吧?她要不卖逼,连活都活不了啦,那种苦处,你们这些有钱人哪里知道?”

话很实在,不过刘宁绝对不会买帐,“少扯鸡巴蛋,别人逼着她来英国的?还是说少卖几次逼就能饿死?犯贱就是犯贱,亏你还有脸替她说话。”

这么难听的话,狂龙脸上可真有点挂不住了,“三位,这里不方便,咱们换个地方聊聊?”

楚云飞不动声色地抬抬眼皮,“随便,走哪里我哥三个都接着,不过,看在都是中国人的面子上,我先告诉你,狂龙,你手底下再有人玩火器,我让你们飞龙帮在伦敦消失。”

火器?狂龙正摸不着头脑,楚云飞的话又到了,“我可真的是好心,我们兄弟三个,亲手干掉的就一百多条人命,不差多你们几十条。”

这话入耳,狂龙不由得倒吸一口冷气,任是他武功再高,听了这话也不可能无动于衷。一个冷战打过,他一把把正在旁边哆嗦的鼻环青年拽了过来,“阿基,谁刚才用火器了,在伦敦用火器,你们不想活了?”

阿基的威风早飞得不知去向了,吓得鼻涕都出来了,“龙、龙哥,那是我刚得的一把手枪,还没来得及送你呢。”

狂龙又问两句,心里就明白了,那三个年轻人绝对没说什么大话,敢把火器丢还给对方,不是白痴就是强得离谱,显然,对方绝对不是白痴。

狂龙又是一拱手,“手下兄弟不懂事,让三位笑话了,我是说换个地方聊聊,只是聊聊,真没别的意思。”现在,他就算有别的意思也坚决不能承认了。

下一刻,楚云飞他们就出现在了酒吧的一间小包间里,还有人殷勤地送来了一些啤酒和冰块,居然还有两包瓜子。

对方这么笑脸相应,楚云飞倒也再不好说什么,拿起瓜子磕了起来,这种中国小零食,一年多没吃到了。

刘宁可不想这么善罢甘休,狂龙怎么说也是他老乡,绝对不能让战友看笑话,“狂龙,你也觉得中国人自己埋汰自己是正常的?”

同样,狂龙面对刘宁也少了很多顾忌,“操,活都活不了还计较那些?我他妈还是练武的呢,火气不比你小。”

“刚来英国的时候,我自然也是像你们这样想的,可他妈咱们那些同胞真他妈的不争气呀,见了别人跟个孙子似的,就会糟蹋自己人,看到我脸上这些伤疤没有?都是同胞挑唆警察干的。”

楚云飞他们三个互相看看,不约而同地想到了李南鸿的被抓。

狂龙的感慨实在是太深了,也许是压抑太久的缘故吧,“和我一样,好多中国人都是偷渡来的,而且,就算那些正常途径来的人,也很少有人会流利的英语,他们只能混迹在中国人里面,出不了这个圈子。”

“大家都是在社会最底层混的,想过得好点自然要踩着别人的肩膀往上走,踩不了外人,那就只好踩同胞了,唉~”

楚云飞皱皱眉头,“你说的这些,我在书上看到过一些,不过,那是几十年前的事了吧?”

\section{第一百四十八章 老龙不飞啦}

狂龙苦笑一声,“切,有人在伦敦住了三十年都不会英语呢,当然,也有人混起来的,不过,没人傻得再回到这个圈子,那是要惹麻烦的。”

“现在来的中国人普遍素质是高了点,不过,还不是一样?想混好,踩着同胞上,踩别人,那是要出麻烦的。”

说到这里,狂龙摸摸自己的半只耳朵,“这就是教训,一个破落的日本烂人,嫖了女人不给钱,那时我也是不知道,不管不顾地狠揍了那家伙一顿,结果日本大使馆出头把我告了,那小逼本来还欠他们政府好多的钱呢,结果倒好,日本政府出黑钱要警察收拾我,要不是我还有点功夫,这条小命就交代在监狱里面了。”

“更让人气的,那婊子怕惹麻烦,根本不给我做证,直接跑到曼彻斯特去了,后来怕我找她,又跑到荷兰去了。”

怪不得狂龙对那个露丝那么大的意见,原来早有先例的!

说到这里,狂龙再也按捺不住自己的怒火,腾地站了起来,“你们说说,我他妈为什么还要对中国人好?我用同胞之情对你了,这遭的是什么报应?”

“让我学雷锋?我操,雷锋死了好多年了!”

楚云飞刮刮鼻子,发话了,“老龙,别那么激动,坐、坐。”

楚云飞小时候是看着《铁臂阿童木》、《花仙子》之类的日本动画长大的,那时国产的动画片实在少得可怜,他还作为学校舞蹈团的队员迎接过日本客人,所以对日本人本来是没那么大成见的,毕竟那场战争已经距离得比较远了。

可在后来同刘宁和成树国的接触中,楚云飞逐渐知道了很多当时还不方便公开的内幕,对日本人的印象也跟着坏了起来,虽然还赶不上这俩战友那么深恶痛绝,但已经是很讨厌了。

所以狂龙这番诉苦,听在楚云飞耳中,同情之感大起,连带着对这人的看法也转好不少,“老龙,你也知道,这人呐,啥时候也有好人有坏人的,没必要一杆子打落一船人的。”

成树国听着也开始为狂龙抱不平,连带着还支招,“老龙,你就这么放过日本人了?我要是你,怎么也要回头再收拾几个这样的垃圾,要不实在太憋气了。”

狂龙眼睛一瞪,虽然不算大,可眼神还是挺凶,“那可不是废话?我自然要报复,不过,试了一次就没敢再试了,操,事后警察满世界找敲闷棍的了,我又被弄进去问了一天。”

“这事上,不服小日本不行,那鸟国家,保护自己人还不是一般的用心,伦敦的警察,不花几个钱谁会给你这么用力办事?”

刘宁本来很欣慰老乡给自己长脸的,不过眼看着话又要不着调了,赶紧转移话题,“狂龙,你说你也是偷渡来的,有你这功夫,为什么不呆在国内呢?不会比这里活得差吧?”

楚云飞又不干了,他从小受的爱国主义教育显然起到了该有的作用,纵然政府舍弃了他们,但对国家的忠诚还促使着他不由得多说两句。

“老龙,你这话也未必就对,日本政府在英国可以这么混,但你让它去俄国、韩国试试去?就别说去中国了。”

狂龙没理会楚云飞,刘宁的话让他又想起了从前的事,疤痕累累的脸不自主地抽搐了起来,隔了好半天才长叹一口气。

“唉,那都是些陈芝麻烂谷子的事了,我不想提了。总之就是年轻气盛下手没轻重,惹人了,国内呆不下去了,就跑出来了。”

“我他妈就想不明白了,政府政府出问题,同胞同胞没命地阴你,做个中国人,咋就这么难呢?”

话糙理不糙,或者说以狂龙的际遇,他是有资格这么抱怨的,不过楚云飞第一时间就反驳了他,“老龙,你这么说就钻了牛角了,那按你的说法,我们兄弟三个算不算你的同胞?”

看着狂龙点点头,楚云飞继续说了下去,“那你应该能觉出来,我们绝对是不会阴你的,所以说,你的话,就是太偏激了。”

狂龙这下可不服气了,“你们三个那么有钱,和我根本就不是一个圈子里的,我挡不了你们饭碗,那还阴什么阴?不过,说实话,我也能觉出来,你们哥子三个,都算血性汉子,你别说,这种人还真的不多。”

刘宁点点头,“所以说啊,老俵,你就别在那里埋怨天地公道不公道了,自己有实力,比啥都强,没实力,可也就是只能做点下三滥的营生。”

楚云飞也在一旁帮腔附和,“对呀,原始积累阶段,血腥点没啥关系的,就象你现在坐抽保护费这样,我们哥仨的心思,也就是想让你在条件允许的情况下,对同胞,该高高手的时候顺便就放人一马。”

成树国可是还想着别的心思呢,“老龙,不瞒你说,我们哥几个趁俩钱,那也是拿命换来的,你要是有心,大家裱到一起干好了,我们手里可正有活呢,比你这点小买卖强多了。”

在伦敦,狂龙虽然混得不算好,不过,那些阴暗的东西还是知道不少的,“各位兄弟,于某今年42岁了,托个大,当回老哥,你们的心思老哥我明白,不就是以后多给同胞留份情面么?我这里应承下了。”

“至于你们做的买卖,我也多少能想到点,不过就不多说了。老哥年纪大了点,在这里好歹也算有个小局面了,关键是家里孩子还小,不能陪你们闯了,自古英雄出少年,现在的天下,该是你们年轻人的了。当然,你们要是觉得我不识抬举,那就开出价码来好了,老哥绝不含糊。”

说到最后,狂龙脸上已经隐隐有决绝的神情了。

成树国脸色一变,就要开骂,不过楚云飞一把拉住了他,“刘宁,你跟他说吧。”

刘宁自然知道该怎么说,“老俵,你这话就见外了,我这兄弟也是好心,你没兴趣就算了,我们见个普通中国人遭难,都要铁定伸手帮忙,莫不成还强邀你做买卖?你这把年纪,实在是越活越回去了。”

听到这冒犯的话,狂龙不怒反喜。

他的江湖本来就不是白走的,二十多年,足够把个愣头青磨成老人精了,刚才的话自然是有几分造作的意思在里面。

\section{第一百四十九章 一波又起}

江湖越老,胆子越小,这话说得一点都没错狂龙的直觉告诉他:这几个人绝对是别着脑袋玩的主!正在做而且还要做很多惊天动地的事。

要是自己的回应含糊点,逼得对方进一步解释或者劝说,可就想装听不到都晚了。那时候,该圆该扁就由不得自己了。

“兄弟你都这么说了,我还能说什么?老哥在这里给这位兄弟陪个不是了。还有,诸位几个的意思我也都记下了,老哥别的本事没有,说话还是要算数的,以后,同胞的忙,我能帮的一定帮,行不?”

还是那句话,不涉及自身感情的时候,楚云飞并不是个好骗的主,这么说来,情商似乎也应该分为好几个部分的。

狂龙的做作显然没有骗过楚云飞的眼睛,不过人家既然摆明态度要置身事外,他倒还是比较欣赏这种“道不同不相为谋”的态度,“好了,老龙,收起你那套吧,哄谁呢?我们都知道你算上道的啦,我们也不是那种绑架人的主。”

被看穿了心思,狂龙尴尬地笑笑,“呵呵,人在江湖,有些时候,是不得不小心点的,咱们毕竟还不算太熟,不是么?你们做的那些买卖,说实话,我要再年轻十岁铁定跟你们一起走了,现在嘛,实在是拖家带口了,不方便啊!”

楚云飞跟心里明镜似的,狂龙这里,拖家带口是一方面,另一方面才是根本和重点,就是双方的合作根本就没有任何可以依托的平台,可信度是个大问题。

有大钱谁不想挣?狂龙现在经营的这点东西也不能说就太平到什么程度,关键是自己这方的强势让对方有了不踏实的感觉,吃亏吃海了的狂龙还能总是记吃不记打么?

还好,楚云飞本身也并不是很想邀对方参与,那只是成树国的想法而已,在楚云飞看来,一个配合默契的三人团队,要比十几个人的散沙强太多了。

“好了,既然这样我们就不多呆了,我们还要去听音乐会呢……靠,快八点了?”

狂龙看到对方要走,自然是要挽留一下的,“哥几个,留个名字吧,山高水长的,咱们日后也好相见。老哥我先说了,于化龙,干钩于,鲤鱼跃龙门的化龙。”

怪不得要叫飞龙帮呢,感情狂龙的名字是这么个由来,不过“鱼化龙”这名字还算不错。这么想着,楚云飞他们也把名字留了下来,刘宁还把自己的新手机号告诉了对方,“都是老乡,有什么事说话,只要你管好了你这一摊,我们能师出有名就好办。”

到末了,狂龙也没弄清楚眼前这三人是做什么的,而且还有个货真价实的英国司机服侍着,不过,人家表现了善意出来,又肯定是很强势的人物,他也一定要盛情挽留了,“音乐会听不成了,天也不早了,让老哥尽尽地主之谊吧,省得江湖朋友说我不会做人,呵呵。”

话说到这种程度,三人按理说是不能再矫情了,可是,很不幸,刚才狂龙的一番话严重地影响了楚云飞他们的心情,于是三个人还是婉言拒绝了。

出了酒吧,三人都没有说话的欲望,楚云飞喊了司机来,“把我们送到宾馆。”

那司机还想说些什么,看到三人的面色都不善,嘴皮动了两下,最终还是没有说出来。

宾馆里的东西早被维伦斯家的人取走了,连房间都退了,在这样一个国际大都市里,敢这样肆无忌惮地行事,果然是有大家风范。

楚云飞他们本可以就这事发发火的,不过人家毕竟是为自己好,再说,他们现在也都没心情计较。

刚才在同狂龙的争辩中,三人表现得还是戮力同心、一致对外的,但三人内心的真实想法狂龙又怎么可能知道?

三个士兵被判定为“犯罪”,说冤枉倒也算不上,他们实在是没听从部队安排,作为士兵那绝对是犯了大错的,严格地说,上军事法庭也不是不可能。

但在这事里面,三个士兵已经做了他们该做的了,他们也只能做那么多。汇报写了,事情也分析了并得到了首长的认可,谁能想居然是那么个结局呢?

虽然最后三人跑了,但他们长期接受的“爱国主义教育”,还是使他们认为,这是个偶然的事件,是政策在这里拐了个弯,导致三人成为不幸者而已。

刘宁和成树国还好点,俩人毕竟是出身军人世家,老一辈革命家的言传身教,还是在他俩身上留下了鲜明的烙印。出格的事是做了,二人也不再以军人自诩了,但对国家的忠诚度并没有降低多少,狂龙的话带给他俩的,还是以郁闷居多。

楚云飞就不一样了,他看书多,受到的传统教育也相对多些,虽然他做事的准则同俩战友差不多,但在看待问题上,他是非常讲求逻辑和情理的,和战友相比,少了些狂热,多了些客观。所以狂龙的话对他的影响是最大的。

心里有事,回来得也晚,楚云飞心不在焉地同维伦斯家里的几个人打个招呼,就想溜回房间。

李南鸿听到人声,跑了出来,在走道上神神秘秘地拉住了楚云飞,“飞哥,来,我跟你说点事。”

楚云飞心里正烦着呢,没好气地瞟了这家伙一眼,“有什么事你就说吧,你小子还能有正经事不成?”

李南鸿却是很慌张的样子,四下看看,“嘘,小声点,别让人听到了。”

见他这个样子,楚云飞一个激灵,自己这是怎么了?怎么能让情绪带着走呢?于是压低声音,“我知道了,维伦斯家有什么异常么?”

看着李南鸿欲言又止的样子,楚云飞嘴角略微弯了弯,“哈,奇怪,你也知道小心了?说吧,到底什么事,现在附近没人。”

李南鸿是越来越相信楚云飞了,既然飞哥说没人那肯定就是没人,“飞哥,照我的判断,似乎,似乎索菲娅她家,对多尼起了点心思。”

\section{第一百五十章 迷失的云飞}

“哦”?这话听得让楚云飞有点紧张,也顾不上心情不好了,皱皱眉头,“你这话,怎么说?”

李南鸿也皱起了眉头,“是这样的,下午,今天下午那个胖子来看露丝,考林斯,然后找到我聊天,除了问你的事,就是问多尼的事,最关键的是,他好象很想知道多尼和咱们到底是什么关系。”

看到楚云飞点点头,李南鸿接着讲了下去,“多尼和你们怎么回事,我哪里知道啊?所以我就说不清楚,可那胖子后来总是有意无意地把话题往这个上面引,我觉得,他们似乎对多尼有点图谋。”

楚云飞抿着嘴,琢磨了起来。这么说来,显然那个考林斯或者说维伦斯家族的人都小看了李南鸿,以为他不过是个愣头愣脑的小孩子,所以才会肆无忌惮地打探虚实。不过,谁要真把这家伙当愣头青,那肯定是打错了算盘。

那也就是说,多尼的事情肯定是有了点眉目,而且估计让维伦斯家感到有点棘手,所以才这么小心试探吧。

“唉”,楚云飞轻轻叹口气,无奈地摇摇头,自己做人怎么就做得这么累呢?还说能放心地休养一下了,谁能知道维伦斯家的态度,居然又变得这么暧昧起来?

还是哥几个辛苦辛苦吧,没了警惕性的士兵,那绝对就是没牙的老虎,大风大浪闯了过来,可别小河沟里翻了船。

“好了,小李,我知道了,你玩你的去吧,看能不能给你老子弄回去个外国媳妇。”说完,楚云飞还促狭地挤了下眼睛,掉头去找成树国他们去了。

李南鸿却僵在了原地,愣愣地喃喃自语,“找个外国媳妇?嗯……不错,为什么不试试呢?”他心里想的那个媳妇,可不是伊琳娜。

通知了成树国和刘宁警醒点,楚云飞回到房间,洗个澡躺在床上,却是翻来覆去地睡不着。多尼的事,应该严重不到哪里去,他想的还是狂龙的话。

一直以来,楚云飞都坚信自己是妥协政策的牺牲品,而且是那种临时的政策。至于政府的长期政策,他一向以为还是不错的,起码十九世纪和二十世纪初的那个羸弱的中国不见了,能强国的自然是好政策。

对于政府,他虽然不能说颇有好感,但也是很中性地看待着的,并不因为父亲没得到救助,自己被宣布为叛国就有什么仇视的想法,毕竟很多事情,是不能简单地用对错来确定的。

政策和政府的因素对楚云飞的影响其实并不很大,很多时候,他都明白自己是深深地热爱着那个古老的祖国,他愿意为祖国做任何有益的事情。

国家,那是个什么样的概念?那是统治工具,是暴力机器,这点楚云飞是明白的。他心目中的祖国,指的并不是那个,而是那片水土,那土地上生活着的父老乡亲,那才是他深深眷恋着的。

包括那泱泱五千年的文明。

丑陋会在任何地方存在,只要有合适它生存的土壤。看那么多书,楚云飞也明白,林子大了,自然有好鸟和坏鸟的差别,或者说,一群为了争食的鸟儿,难免会有些卑鄙的手段出现。不过,书上写的,远不如现实中接触的令人震撼。

狂龙的话,极其严重地打击了他,真的是这样的么?难道说,大多数的同胞,都是那么那么残忍和无情的么?

诚然,理想和现实会有很大的差距的,但至于冷酷如斯么?

要是成刘二人在,自然会笑话他:那些为了活得好点、或者说向往西方国家繁荣的人们,不惜用种种手段,合理或者不合理地离开自己的祖国,对于他们悲惨的下场,大可用咎由自取来笑看的。

不过,对楚云飞而言,他认为,人人都有追求享受的权力的,事情只要不做得那么太过分,那动机实在是情有可原的。何况还有狂龙那种在国内混不下去了的主,不让人家偷渡,那不活生生要人家命么?

狂龙的性格,楚云飞其实很喜欢,因为这人也曾试图维护过民族的尊严,虽然不是件什么大事,但他确实尽了自己的力了,为此还丢了半只耳朵。

可这样的人都做了吸食同胞鲜血的蠹虫,这不能简单地用幸运和不幸来界定吧?

“我操,雷锋都死了好多年了!”这话又在楚云飞脑中响起,同时出现的,还有狂龙那张发青的脸,愤恨中又带些自嘲的冷笑。

下一刻,“中国露丝”的形象也出现了,年轻的脸上,是赤裸裸的谄媚和巴结,还有惊闻自己是中国人时的意外和被戳穿时的恼羞成怒。

我愿意为我的祖国牺牲,但是,天哪,这样人渣,就是我守护的目标么?

让我这一腔热血,洒在狗身上么?

对年轻的楚云飞来说,这确实是个残酷而又现实的问题。

想着那一张张颓废的脸孔,空洞的笑容,楚云飞不由得想起了书上写的那种“迷失的一代”,文革时有迷失的一代,现在怕又是新的迷失的一代吧?

不同的年代,各自迷失了不同的东西。

这实在是个信仰缺失的年代!楚云飞恨恨地想着。

信仰缺失?那我现在的信仰是什么?想到这里,楚云飞莫名地慌乱了起来,他忽然间发现,自己也实在是说不上有什么信仰,似乎没有咒骂别人的资格。

算了,还是不要想了吧?狂龙说得对,“活下来才是最重要的。”现在,还是想想自己和战友的生存问题吧。

不过还好,我还有良心,比那些人渣强多了,不是么?楚云飞这样安慰自己。

借口找到了,可楚云飞还是睡不着,狂龙的话真的给他造成了无尽的困惑。

“我他妈就想不明白了,政府政府出问题,同胞同胞没命地阴你,做个中国人,咋就这么难呢?”

直到凌晨一点左右,楚云飞才有了点睡意,迷糊中,他升起了一个新的念头:平衡,似乎把握好平衡才是最重要的。

这天晚上,楚云飞又梦到了好久不见的老道,老道是他的师傅,又叨叨了很多很多,其中有一句被他牢牢地记住了。

“境界,境界决定一切。”

\section{第一百五十一章 法国的江湖}

第二天起来,居然还是阴天,露丝身体已经好转,可以出来活动了,不过,为了避免她的纠缠,楚云飞一大早开溜去找考林斯。

考林斯的态度还是很热情,同楚云飞随便聊了两句,不过当楚云飞问起多尼的事的时候,胖子建议他去找班克斯,“事实上,多尼的事我还是听他说的呢,你找他问问吧,我还要马上去趟沙特。”

去趟沙特?楚云飞的脑瓜又转了起来,不过想想多尼那惊弓之鸟的神情,唉,算了,还是先帮他把事摆平吧。

当楚云飞找到班克斯的书房的时候,身后传来了索菲娅的喊声,“楚先生……”

楚云飞就当没听到一样,直接就推门进去了。

一身牛仔服的索菲娅在他身后摇摇头,美目中全是笑意,可怜的中国人,被吓坏了,其实他知道……应该先敲门的。

多尼也在班克斯的书房里!一脸沮丧的样子,而班克斯则衣冠楚楚地在向公文袋里放着什么东西,一副要出门的样子。

看到楚云飞进来,班克斯愣了一下,随后笑脸相迎,“哦,楚先生,早上好。”

楚云飞也笑笑,“早上好,班克斯先生,你这是……要出去么?”

班克斯点点头,“是啊,我想去趟公司,不过,你来了,我去不去也无所谓,你没有同你的同伴一起出去玩么?”

说起那俩,楚云飞就恨得咬牙,脸上还得挂着笑容,“哦,他俩呀,他俩昨天看好点东西,已经出去采购去了,剩下我一个人……来这里问问班克斯先生,不知道多尼的事,您打听到什么消息了没有?”

事实上,成树国和刘宁二人以“保存威慑力量的”的幌子,厚颜无耻地溜出去玩了,谁让楚云飞非要大家提高警惕呢?

班克斯不以为然地笑笑,“你那俩同伴似乎很够朋友啊,呵呵,要你留下来陪美女,不过……”

说着说着,班克斯的神色慢慢严肃了起来,“多尼的事,你还是让他自己跟你说吧,我刚刚才确定了大致情况。”

多尼的嘴唇左撇撇,右挪挪,半天才悻悻地说了起来,“真想不到,这事居然是我们波兰黑帮和法国黑社会一起搞的,法国那个黑社会还是很有名‘克鲁梭工人党’,我认识的朋友里有跟他们很熟悉的人。”

班克斯点点头,“是啊,不光你跟他们熟悉,我们维伦斯家族跟他们都有过联系。”

原来,其实从严格意义上讲,法国并没有什么特别强大的黑势力团伙,那里虽然黑帮众多,社会治安也很糟糕,不过大多是那种小型的帮派或者说流窜犯,并没有一手遮天的超级BOSS.“克鲁梭工人党”已经算其中顶级的黑势力了,它的前身是法国南部重工业城市克鲁梭(Creusot)的流氓无产者组成的一个小小帮派,不过现在已经隐隐有成为黑帮传统势力的趋势了。

“克鲁梭工人党”起家主要是靠了天时和地利,天时就是法国政府长期以来对大型黑势力的整顿,使得整个法国的黑势力成为了一盘散沙,也阻止了国外的地下势力,比如说黑手党、纳粹分子、恐怖分子等的渗透。

地利,那自然是因为法国最大的军火工厂——施奈德钢铁总厂就在克鲁梭,而南部繁华的港口城市马赛也离那里不远。

“克鲁梭工人党”靠着向国内其他黑势力、甚至欧洲其他国家的黑势力贩卖军火起家,当他们的势力发展到马赛,控制了这个工商业城市一定地盘后,想不壮大都不可能了,同时,他们也拥有了众多的国际盟友。

还好法国政府一向秉承“小黑不管,专杀大黑”的优良传统,“克鲁梭工人党”也明白政府的底线在哪里,明智地把势力控制到一定范围内,不敢再强行发展了。

多尼的仇家——托尼那支,在波兰国内找的是类似恐怖组织的“波兰复兴运动”,那个组织在波兰也不敢出现在明面上,因为痛恨这个组织的周边国家实在是有点多。

脱特斯基家族同这个组织一直有着千丝万缕的联系,不过,当他们家族开始向西欧发展后,就和对方明显地拉开了距离,一直牵扯着这个组织的话,实在是有碍于家族在经济上的发展的。

大家都是波兰族,脱特斯基家族这样做肯定是非常令“波兰复兴运动”心寒的,不过鉴于脱特斯基家族在波兰国内强大的影响力,“波兰复兴运动”也无法奈何对方,再说人家也没说索性就断了往来。不过有点芥蒂产生那就难免了。

等到后来托尼这一支找上门来,为夺权寻求帮助,“波兰复兴运动”自然没有坐视的道理,他们很快就派出了杀手潜入法国。

波兰人在法国想做点事,自然要求助当地黑势力,于是“波兰复兴运动”花重金买通了“克鲁梭工人党”。反正脱特斯基家族那么有钱,这钱未来的继承人会报销的。

托尼和多尼也是认识的,托尼虽然是他那一支里最有前途的后辈,但多尼也是颇受看重的族长近支,两人身份大体上是相当的。

这些都是班克斯先生在一天之内打听到的。

班克斯甚至打听到了,托尼并没有杀多尼的心思,只是想吓吓他,让他把家族内的私密帐号和帐本乖乖地交出来。

托尼神通广大,知道多尼跑到了索度而没再转机,不过他刚上任,家族里需要处理的事务太多了,还有几个不太安分的分支也需要安抚,所以他也一直没顾上去找多尼麻烦,虽然他可能连多尼在哪里都未必能够打听到。

按照法国的规矩,多尼手中那几个帐号,可以在三年时效期过后通过相关证明,来变更户主和责任人的,只要多尼不在法国,他手里的钱迟早是要回到脱特斯基家族继承人手中的。至于那两个族长才有权亲自调用的帐户,一来里面的钱未必有多少,二来,多尼肯乖乖交出的话,他至于那么亡命天涯么?

所以托尼暂时不去理会多尼,是有自己的道理的。只要多尼不回来,有的是时间慢慢收拾他。

托尼不知道的是,多尼除了掌管着那七个家族公司的帐号,两个私密帐号外,还有两个绝密帐号。

\section{第一百五十二章 弱肉强食是王道}

听说这事同“克鲁梭工人党”有关系,多尼的朋友们怎么还敢再管闲事,这个黑帮实力虽然不是特别的强,但在法国那绝对是没人敢招惹的。更何况他们背后还有其他跨国的大型黑帮团伙甚至是恐怖组织?

班克斯头疼的也是“克鲁梭工人党”,维伦斯家族当然不会真的怕这么个不太入流的帮会,不过实在是两家以前有过一些买卖,多少是有点江湖情意的,维伦斯家族也不敢保证以后再不会同这个伙伴合作。更何况对方身后的潜势力真的也很惊人。

“波兰复兴运动”有它的政治目的,都有些像恐怖组织了,虽然他们更可怕些,维伦斯家族却也没有多么放在心上。

首先这个组织是标榜“和平”的,跟大多数恐怖分子相比,做事不算特别心狠手辣,起码他们给人是这么个印象;而且他们的影响实在是有限,只有那些波兰人比较多的国度才有他们的生存土壤。对英国人来说,是太遥远的事了。

“千金之子,坐不垂堂。”——以维伦斯家现在的身份和地位,没有足够的理由,不管容易与否,班克斯都不得不承认,随便招惹其中任何一个都不是太划算的举动。

多尼倒是够厉害,这样的主一下惹了俩。

难怪维伦斯家要打听自己和多尼的关系,楚云飞暗想,这种不太划算的买卖,怕是就算为了自己这救命恩人,维伦斯家也未必愿意伸手。

楚云飞微微地摇摇头,笑着问班克斯,“班克斯先生,似乎这件事给你们造成了困惑?”

班克斯也无奈地笑笑,“楚,我真不知道该怎么说你才好,你年纪不大,惹人的本事实在是不小啊,这两家再加上‘基天’,够你忙三十年的啦,呃,前提是,你能活那么久。”

楚云飞知道班克斯在间接建议自己推掉多尼,不过他一向是痛恨“谋事不忠”那种行为的,所以摇摇头,刮刮鼻子,“呵呵,谢谢阁下的建议了,我也很想知道我能不能活那么久。”

班克斯听得也是微微摇头,他也知道眼前这个年轻人很有主见,才待说点什么,楚云飞又张口了。

“呵呵,看得出来,你们家和‘克鲁梭工人党’关系不错,那我也不勉强你了,我只想要求一件事,你们能不能别把我们的事告诉他们?”

班克斯嘿然不语,半天才非常无奈地摇摇头,“这个,实在是抱歉,一开始,我就是向他们打探消息的,你这话说得似乎有点晚了。不过,他们答应我了,只要多尼呆在英国不出去,他们保证不会向多尼动手,当然,他们不能担保波兰的那帮家伙。”

楚云飞皱皱眉头,“啧啧,班克斯先生,恕我冒昧,您看起来,并不是那么冒失的人,怎么会把事情搞成这样呢?”

班克斯又被气到了,很不绅士地翻了翻眼皮,“我向他们打听,那自然有我的理由,再说,谁又能想到事情会这么凑巧?”

楚云飞点点头,是啊,人家只是帮自己打探下消息,自己又没叮嘱,在谨慎上欠缺点实在是再正常不过了,又不是维伦斯家自己的麻烦。

“那你不能动用你的力量,劝说你的那个合作伙伴不趟这趟混水么?”

多尼苦笑一声,“人家还希望班克斯先生能劝说你不趟这趟混水呢,托尼出的价钱比我出的高多了。”反正这事楚云飞很快就会知道,多尼抢在班克斯前面说了出来。

“哦”?楚云飞听得扬了扬眉毛,开心地笑了起来,“哈哈,看来这次我能按最高标准取费了”,说着话锋一转,“多尼你不用害怕,我的主意已经定了,你不用拿话挤兑我,作为个中国人,我把信用看得比什么都重要,我都懒得问托尼打算给我多少钱。”

多尼明显地松了口气,班克斯的眉毛却是微微地皱了皱,“楚,你不再认真地考虑考虑了?或者跟你的同伴商量商量?”

楚云飞斜眼看看班克斯,这个动作其实很不礼貌,“我不需要考虑,因为我做人有我做人的原则,至于我的同伴……我相信,我的决定就是他们的决定,他们的决定也是我的决定。”

班克斯摇头苦笑,“看来这次我又挣不到什么钱了。”

事情是明摆着的,托尼肯开出那么大的价码,自然不会是因为几个没见过面的中国人,哪怕班克斯把他们吹嘘得再厉害,这世界上,没真刀实枪地干过,谁会怕谁?

那钱主要是针对多尼的,托尼肯定有附加条款,就是扣住多尼,移交给他,财神的价钱,自然要高点,这个大家心里跟明镜一样。

当多尼得到这个消息的时候,他马上就反应了过来,维伦斯家族肯把这个信息透漏给他,无非就是两重意思:一是要他识相点,逃跑这种事是想都不要想了;二就是,如果他能把手里的财物舍弃点的话,维伦斯家族可以在伦敦为他提供不错的保护。

不过,这一切的一切,都要看眼前这个年轻的中国人的态度,如果楚云飞真的表现得像往常一般地贪财的话,那多尼只有被压榨得一干二净的份儿,然后,多半就是可以找路瑞或者丽迪娜聊天去了。

事实上班克斯打的就是这个主意,他甚至计划好了,如果楚云飞愿意向自己同伴的下手的话,该怎么吞掉那笔数目不详的财富,如果中国人的胃口超过什么界限之后又该怎么应对,甚至他连联合楚云飞他们恶斗“克鲁梭工人党”的心思都有,只要那钱确实够多。

从哪个角度讲,托尼的如意算盘都是不会成功的,注定是竹篮打水一场空。

不过,像眼前这种可能他也考虑到了,虽然是不太容易发生的情况,但有备无患总是应该的。

就维伦斯家族而言,托尼开出的一百万美圆的数目虽然实在是不小,但是相对于用六十多万英镑才交到的朋友,做个交换也不是什么很划算的买卖,而且那朋友的实力也实在有点深不可测。

现在的班克斯只是有点微微地后悔,为什么这话会让多尼抢先说了呢?中国话实在是说得不错,“先杀人的有理”(先下手为强)!

其实,他要是先说出来,不过是多个“自讨没趣”的感觉就是了。

\section{第一百五十三章 没实力就闭嘴}

多尼长出一口气,终于把班克斯先前禁止他说的话说了出来,“楚,非常非常地感谢,既然你选择了继续支持我,我也可以承诺了,我会把交易金额提高一倍的。”

这话入耳,楚云飞不知道其中内幕,忽然间有点讨厌多尼了。

不过下一刻他就反应了过来,狐疑地看了看班克斯,“这话回头再说吧,你该庆幸的是,你比托尼先遇到了我。”

现在的班克斯只有摇头的份了:这个年轻人,果然是很有主见!

楚云飞拿出一张纸片递给班克斯,“班克斯,这是我的移动电话,以后有事可以通过这个联系我,有效期……呃,半年吧。”

三个人都很有默契地闭口不谈班克斯对多尼的压制,因为这事实在不算他的错误,非要说错的话,那就是多尼不该拥有那么多帐号而穿梭于强者之间。

至于多尼,现在已经是最好的结果了,他还敢再罗嗦什么?可怜,一个曾经在脱特斯基家族呼风唤雨的人物,居然沦落到这步。

班克斯看着楚云飞呆了一阵,接过了纸条,随手从公文袋里拿出两张纸,“楚,既然你已经决定了,那我就不再说什么了,这里是他们的资料,哦,对了,‘克鲁梭工人党’里的瑟利尼你不要去动,那是我朋友的儿子。”

楚云飞接过了那两张纸,两张纸上各有十几个人名和住址,后面还有细小的备注。瑟利尼也赫然在上面。

看来,这个瑟利尼是维伦斯家族同工人党联系的重要枢纽,不过,放他一马确实也不是什么难事,“好了,那就谢谢你了,有了这个,事情就好办多了,还有,班克斯,我的事你问过宾塞斯先生了么?”

班克斯笑笑,那笑容里隐藏了太多的内容,“事实上,瑟利尼他们同‘基天’在利益上也是有冲突的,我本来……”

楚云飞手一挥,打断了班克斯的话,“好了,班克斯,现在说这话一点意义都没有。我可以很坦白地告诉你,我们和多尼关系并不深,但现在我们已经是朋友了,背叛朋友的事情我是不会做的。”

“冒犯我朋友的人,有必要考虑一下我的愤怒。”

这么嚣张的话,班克斯并不是头一次从面前这个年轻人口中听到,他开心地笑了起来,“我没有忘记这个,事实上,我也是你的朋友不是么?”

“是的”,楚云飞点点头,他这时非常希望自己是索度人,用点头来表示否定,恶搞一下这只狡猾的老狐狸!

“所以说,楚,我不会不帮朋友忙的,虽然我的父亲还没有最后决定,但我想那只是时间问题,你还是先帮多尼把这件事情办了好了。”

楚云飞又点点头,这次他是真正明白对方的用意了,维伦斯家族要再次看看自己的实力,如果事情办好了,那自然可以获得对方的帮助。

事实上,这样的处理证明了维伦斯家族的明智。兹事体大,维伦斯家族虽然恨“基天”恨得牙痒,但他们是绝对不会陷入一场“不死不休”的战争中去的。事败的话,虽然自己这几个人未必会出卖了他们家族,但万事小心点总是不错的。

你们需要先证明自己有对抗“基天”的实力,这是班克斯话里的意思。

“哦,好吧,班克斯,那我先帮多尼把事办了好啦,不过,我相信,你们很快就会做出决定的。”

本来想少杀两个人的,看来不能那么仁慈了,楚云飞暗自叹口气。

等到楚云飞走出班克斯的房间的时候,多尼跟了出来,“云飞,你听我解释,那钱真的是家族的钱,早以前我还想着能回家族呢,自然要留下大部分,要不实在不好交代,不过,现在怕是只能把其中几个帐号的钱拿了走人了。”

这话,像那么回事,楚云飞也愿意相信他确实是那么想的,“现在你不想回家了么?有我帮你呢,怕什么。”

按理说多尼现在应该是非常高兴才对,楚云飞已经决定帮他到底了,又得到了维伦斯家族提供的情报,可他的脸色依旧是那么沉重。

多尼缓缓地摇摇头,神色越发肃穆,眼睛也眯了起来,半天才斟词酌句地慢慢开口,“当我听到‘波兰复兴运动’的时候,就知道我回不去了。”

“我并不认同这种极端的民族主义情绪,跟他们也没什么来往,他们的势力是不小,但在法国我并不怎么怕他们。问题的关键是,虽然我并不认同他们,但是我毕竟也是波兰族的啊。”

多尼考虑的并不仅仅是这个,这自然是瞒不住楚云飞的,他点点头,“不错,你的这个观点我是同意的,有时候对着同胞,再恨他不争气也不愿意被外人笑话。”

“不过,以你们斯拉夫人有仇必报的性格,你能说出来这话,恐怕还是怕被别人知道你找外国人来杀同胞吧?他们只要一嚷嚷,民族情绪一旦被激发起来,亡命天涯……那都是很奢侈的想法了,就别说还呆在脱特斯基家族里啦。”

多尼讪笑一下,不过神情并没有轻松多少,“我不得不承认,楚,你实在是太聪明了。”

“我已经决定了,‘波兰复兴运动’我是不会去理他们的,但托尼和他的弟弟沃尔特我是绝对要干掉的。还有法国那个组织,路瑞的死他们绝对逃不了干系。当然,如果让我知道是谁害了丽迪雅,就算是波兰人我也要杀他全家,就这些,二百万美圆,怎么样?”

事实上,楚云飞并不是个很贪财的人。比如说多尼这件事,他并不是很介意到底能拿到多少钱,他更看重的是,自己有没有被欺骗。总不能自己出了力,还被别人在背后耻笑为“冤大头”,这是对自己智商的侮辱!

钱我不计较,但是,这个人情必须由我来做!这就是楚云飞的想法。

“呵呵,算了,这事回头再说吧,毕竟我还有同伴,总是要听听他们的意见的,呃,对了,我似乎忘记了一些事,还要去找班克斯先生问问。”

楚云飞想的是弄点装备,比如说司机说的那种“特种职业”专用的对讲机什么的。至于武器,那只能指望多尼想办法了,飞机上带那东西,实在是太不现实了。

对即将离开的四个人来说,维伦斯家族其实还是满够意思的,他们为这几个人提供了一些装备,还帮他们很快买好了飞机票。当然,再多的事情他们也做不出了,毕竟两边都算是他们的朋友。

李南鸿毕竟年轻,虽然知道和维伦斯家的关系还没靠得很近,但他也想跟着楚云飞他们去法国,看看这几个哥哥怎么发威。当然,如果能保证不死的话,他都想参与一下。不过楚云飞给了他一万英镑,“要么回国,要么想去哪去哪,去法国你是想也不用想的。”

露丝的病已经不碍事了,但是,因为所有的人都没向她泄露信息,当她知道这个消息的时候,楚云飞他们已经坐上了飞往巴黎的飞机,她也只有向索菲娅抱怨的份了。

